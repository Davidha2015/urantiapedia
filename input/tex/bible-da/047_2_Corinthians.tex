\begin{document}

\title{2 Corinthians}


\chapter{1}

\par 1 Paulus, ved Guds Villie Kristi Jesu Apostel, og Broderen Timotheus til Guds Menighed, som er i Korinth, tillige med alle de hellige, som ere i hele Akaja:
\par 2 Nåde være med eder og Fred fra Gud vor Fader og den Herre Jesus Kristus!
\par 3 Lovet være Gud og vor Herres Jesu Kristi Fader, Barmhjertighedens Fader og al Trøsts Gud,
\par 4 som trøster os under al vor Trængsel, for at vi må kunne trøste dem, som ere i alle Hånde Trængsel, med den Trøst, hvormed vi selv trøstes af Gud!
\par 5 Thi ligesom Kristi Lidelser komme rigeligt over os, således bliver også vor Trøst rigelig ved Kristus.
\par 6 Men hvad enten vi lide Trængsel, sker det til eders Trøst og Frelse, eller vi trøstes, sker det til eders Trøst, som viser sin Kraft i, at I udholde de samme Lidelser, som også vi lide; og vort Håb om eder er fast,
\par 7 efterdi vi vide, at ligesom I ere delagtige i Lidelserne, således ere I det også i Trøsten.
\par 8 Thi vi ville ikke, Brødre! at I skulle være uvidende om den Trængsel, som kom over os i Asien, at vi bleve overvættes besværede, over Evne, så at vi endog mistvivlede om Livet.
\par 9 Ja, selv have vi hos os selv fået det Svar: "Døden", for at vi ikke skulde forlade os på os selv, men på Gud, som oprejser de døde,
\par 10 han, som friede os ud af så stor en Dødsfare og vil fri os, til hvem vi have sat vort Håb, at han også fremdeles vil fri os,
\par 11 idet også I komme os til Hjælp med Bøn for os, for at der fra mange Munde må blive rigeligt takket for os, for den Nåde, som er bevist os.
\par 12 Thi dette er vor Ros, vor Samvittigheds Vidnesbyrd, at i Guds Hellighed og Renhed, ikke i kødelig Visdom, men i Guds Nåde have vi færdedes i Verden, men mest hos eder.
\par 13 Thi vi skrive eder ikke andet til end det, som I læse eller også erkende; men jeg håber, at I indtil Enden skulle erkende,
\par 14 ligesom I også til Dels have erkendt om os, at vi ere eders Ros, ligesom I ere vor, på den Herres Jesu Dag.
\par 15 Og i Tillid hertil havde jeg i Sinde at komme først til eder, for at I skulde få Nåde to Gange,
\par 16 og om ad eder at drage til Makedonien og atter fra Makedonien at komme til eder og blive befordret videre af eder til Judæa.
\par 17 Når jeg nu havde dette i Sinde, mon jeg da så handlede i Letsindighed? Eller hvad jeg beslutter, beslutter jeg det efter Kødet, for at der hos mig skal være Ja, Ja og Nej, Nej?
\par 18 Så sandt Gud er trofast, er vor Tale til eder ikke Ja og Nej.
\par 19 Thi Guds Søn, Kristus Jesus, som blev prædiket iblandt eder ved os, ved mig og Silvanus og Timotheus, han blev ikke Ja og Nej, men Ja er vorden i ham.
\par 20 Thi så mange, som Guds Forjættelser ere, i ham have de deres Ja; derfor få de også ved ham deres Amen, Gud til Ære ved os.
\par 21 Men den, som holder os med eder fast til Kristus og salvede os, er Gud,
\par 22 som også beseglede os og gav os Åndens Pant i vore Hjerter.
\par 23 Men jeg kalder Gud til Vidne over min Sjæl på, at det var for at skåne eder, at jeg ikke igen kom til Korinth.
\par 24 Ikke at vi ere Herrer over eders Tro, men vi ere Medarbejdere på eders Glæde; thi i Troen stå I.

\chapter{2}

\par 1 Men jeg beslutte dette hos mig selv, at jeg vilde ikke atter komme til eder med Bedrøvelse.
\par 2 Thi dersom jeg bedrøver eder, hvem er da den, som gør mig glad, uden den, som bedrøves af mig?
\par 3 Og jeg skrev netop derfor, for at jeg ikke, når jeg kom, skulde have Bedrøvelse af dem, som jeg burde have Glæde af, idet jeg havde den Tillid til eder alle, at min Glæde deles af eder alle.
\par 4 Thi ud af stor Hjertets Trængsel og Beklemthed skrev jeg eder til, under mange Tårer, ikke for at I skulde blive bedrøvede, men for at I skulde kende den Kærlighed, som jeg har særlig til eder.
\par 5 Men dersom nogen har voldt Bedrøvelse, har han ikke bedrøvet mig, men til Dels, for ikke at sige det hårdere, eder alle.
\par 6 Det er nok for ham med denne Straf, som han har fået af de fleste,
\par 7 så at I tværtimod snarere skulle tilgive og trøste ham, for at han ikke skal drukne i den alt for store Bedrøvelse.
\par 8 Derfor formaner jeg eder til at vedtage at vise ham Kærlighed.
\par 9 Det var nemlig også derfor, at jeg skrev, for at erfare, hvor vidt I stå Prøve, om I ere lydige i alt.
\par 10 Men hvem I tilgive noget, ham tilgiver også jeg; thi også hvad jeg selv har tilgivet, om jeg har tilgivet noget, det har jeg gjort for eders Skyld, for Kristi Åsyn,
\par 11 for at vi ikke skulle bedrages af Satan; thi hans Anslag ere os ikke ubekendte.
\par 12 Da jeg kom til Troas for at prædike Kristi Evangelium, og der var åbnet mig en Dør i Herren,
\par 13 da havde jeg ingen Ro i min Ånd, fordi jeg ikke fandt Titus, min Broder; men jeg tog Afsked med dem og drog til Makedonien.
\par 14 Men Gud ske Tak, som altid fører os i Sejrstog i Kristus og lader sin Kundskabs Duft blive kendelig ved os på ethvert Sted.
\par 15 Thi en Kristi Vellugt ere vi for Gud, iblandt dem, som frelses, og iblandt dem, som fortabes,
\par 16 for disse en Duft af Død til Død, for hine en Duft af Liv til Liv.
\par 17 Thi vi ere ikke som de mange, at vi gøre en Forretning af Guds Ord, men som af Renhed, som af Gud tale vi for Guds Åsyn i Kristus.

\chapter{3}

\par 1 Begynde vi atter at anbefale os selv? eller behøve vi, som nogle, Anbefalingsbreve til eder eller fra eder?
\par 2 I ere vort Brev, som er indskrevet i vore Hjerter, og som kendes og læses af alle Mennesker,
\par 3 idet det ligger klart som Dagen, at I ere et Kristi Brev, udfærdiget af os, indskrevet ikke med Blæk, men med den levende Guds Ånd, ikke på Stentavler, men på Hjerters Kødtavler.
\par 4 Men en sådan Tillid have vi til Gud ved Kristus,
\par 5 ikke at vi af os selv ere dygtige til at udtænke noget som ud af os selv; men vor Dygtighed er af Gud,
\par 6 som også gjorde os dygtige til at være en ny Pagts Tjenere, ikke Bogstavens, men Åndens; thi Bogstaven ihjelslår, men Ånden levendegør.
\par 7 Men når Dødens Tjeneste, med Bogstaver indristet i Sten, fremtrådte i Herlighed, så at Israels Børn ikke kunde fæste Øjet på Moses's Ansigt på Grund af hans Ansigts Herlighed, som dog forsvandt,
\par 8 hvorledes skal da ikke Åndens Tjeneste end mere være i Herlighed?
\par 9 Thi når Fordømmelsens Tjeneste havde Herlighed, er meget mere Retfærdighedens Tjeneste rig på Herlighed.
\par 10 Ja, det herlige er jo i dette Tilfælde endog uden Herlighed i Sammenligning med den endnu større Herlighed.
\par 11 Thi når det, der forsvandt, fremtrådte med Herlighed, da skal meget mere det, der bliver, være i Herlighed.
\par 12 Efterdi vi altså have et sådant Håb, gå vi frem med stor Frimodighed
\par 13 og gøre ikke som Moses, der lagde et Dække over sit Ansigt, for at Israels Børn ikke skulde fæste Øjet på, at det, der forsvandt, fik Ende.
\par 14 Men deres Tanker bleve forhærdede; thi indtil den Dag i Dag forbliver det samme Dække over Oplæsningen af den gamle Pagt uden at tages bort; thi i Kristus er det, at det svinder.
\par 15 Men der ligger indtil denne Dag et Dække over deres Hjerte, når Moses oplæses;
\par 16 når de derimod omvende sig til Herren, da borttages Dækket.
\par 17 Men Herren er Ånden, og hvor Herrens Ånd er, er der Frihed.
\par 18 Men alle vi, som med ubedækket Ansigt skue Herrens Herlighed som i et Spejl, blive forvandlede til det samme Billede, fra Herlighed til Herlighed, så som det er fra Åndens Herre.

\chapter{4}

\par 1 Derfor, da vi have denne Tjeneste efter den Barmhjertighed, som er bleven os til Del, så tabe vi ikke Modet;
\par 2 men vi have frasagt os de skammelige Smugveje, så vi ikke vandre i Træskhed, ej heller forfalske Guds Ord, men ved Sandhedens Åbenbarelse anbefale os til alle Menneskers Samvittighed for Guds Åsyn.
\par 3 Men om også vort Evangelium er tildækket, da er det tildækket iblandt dem, som fortabes,
\par 4 dem, i hvem denne Verdens Gud har forblindet de vantros Tanker, for at Lyset ikke skulde skinne fra Evangeliet om Kristi Herlighed, han, som er Guds Billede.
\par 5 Thi ikke os selv prædike vi, men Kristus Jesus som Herre, os derimod som eders Tjenere for Jesu Skyld.
\par 6 Thi Gud, som sagde: "Af Mørke skal Lys skinne frem", han har ladet det skinne i vore Hjerter for, at bringe Kundskaben om Guds Herlighed på Kristi Åsyn for Lyset.
\par 7 Men denne Skat have vi i Lerkar, for at den overvættes Kraft må være Guds, og ikke fra os,
\par 8 vi, som trænges på alle Måder, men ikke stænges inde, ere tvivlrådige, men ikke fortvivlede,
\par 9 forfulgte, men ikke forladte, nedslagne, men ikke ihjelslagne,
\par 10 altid bærende Jesu Dødelse om i Legemet, for at også Jesu Liv må åbenbares i vort Legeme.
\par 11 Thi altid overgives vi, som leve, til Død for Jesu Skyld, for at også Jesu Liv må åbenbares i vort dødelige Kød.
\par 12 Således er Døden virksom i os, men Livet i eder!
\par 13 Men efterdi vi have den samme Troens Ånd, som der er skrevet: "Jeg troede, derfor talte jeg," så tro også vi, og derfor tale vi også,
\par 14 idet vi vide, at han, som oprejste den Herre Jesus, skal også oprejse os med Jesus og fremstille os tillige med eder.
\par 15 Thi det sker alt sammen for eders Skyld, for at Nåden må vokse ved at nå til flere, og til Guds Ære forøge Taksigelsen.
\par 16 Derfor tabe vi ikke Modet; men om også vort udvortes Menneske fortæres, fornyes dog vort indvortes Dag for Dag.
\par 17 Thi vor Trængsel, som er stakket og let, virker for os over al Måde og Mål en evig Vægt af Herlighed,
\par 18 idet vi ikke se på de synlige Ting, men på de usynlige; thi de synlige ere timelige, men de usynlige ere evige.

\chapter{5}

\par 1 Thi vi vide, at dersom vor jordiske Teltbolig nedbrydes, have vi en Bygning fra Gud, en Bolig, som ikke er gjort med Hænder, en evig i Himlene.
\par 2 Ja, også i denne sukke vi, længselsfulde efter at overklædes med vor Bolig fra Himmelen,
\par 3 så sandt vi da som iklædte ikke skulle findes nøgne.
\par 4 Ja, vi, som ere i dette Telt, sukke; besværede, efterdi vi ikke ville afklædes, men overklædes, for at det dødelige kan blive opslugt af Livet.
\par 5 Men den, som har sat os i Stand just til dette, er Gud, som gav os Åndens Pant.
\par 6 Derfor ere vi altid frimodige og vide, at medens vi ere hjemme i Legemet, ere vi borte fra Herren
\par 7 thi i Tro vandre vi, ikke i Beskuelse
\par 8 ja, vi ere frimodige og have snarere Lyst til at vandre bort fra Legemet og være hjemme hos Herren.
\par 9 Derfor sætte vi også vor Ære i, hvad enten vi ere hjemme eller borte, at være ham velbehagelige.
\par 10 Thi vi skulle alle åbenbares for Kristi Domstol, for at hver kan få igen, hvad der ved Legemet er gjort, efter det, som han har øvet, enten godt eller ondt.
\par 11 Efterdi vi da kende Frygten for Herren, søge vi at vinde Mennesker; men for Gud ere vi åbenbare; ja, jeg håber, at vi også ere åbenbare for eders Samvittigheder.
\par 12 Ikke anbefale vi atter os selv til eder; men vi give eder Anledning til at rose eder af os, for at I, kunne have noget at svare dem, som rose sig af det udvortes og ikke af Hjertet.
\par 13 Thi når vi "bleve afsindige"; var det for Guds Skyld, og når vi ere besindige, er det for eders Skyld.
\par 14 Thi Kristi Kærlighed tvinger os,
\par 15 idet vi have sluttet således: Een er død for alle, altså ere de alle døde; og han døde for alle, for at de levende ikke mere skulle leve for sig selv, men for ham, som er død og oprejst for dem.
\par 16 Således vide vi fra nu af ikke af nogen efter Kødet; om vi også have kendt Kristus efter Kødet, gøre vi det dog ikke mere nu.
\par 17 Derfor, om nogen er i Kristus, da er han en ny Skabning; det gamle er forbigangent, se, det er blevet nyt!
\par 18 Men alt dette er fra Gud, som forligte os med sig selv ved Kristus og gav os Forligelsens Tjeneste,
\par 19 efterdi det jo var Gud, som i Kristus forligte Verden med sig selv, idet han ikke tilregner dem deres Overtrædelser og har nedlagt Forligelsens Ord i os.
\par 20 Vi ere altså Sendebud i Kristi Sted, som om Gud formaner ved os; vi bede i Kristi Sted: Bliver forligte med Gud!
\par 21 Den, som ikke kendte Synd, har han gjort til Synd for os, for at vi skulle blive Guds Retfærdighed i ham.

\chapter{6}

\par 1 Men som Medarbejdere formane vi også til, at I ikke forgæves må have modtaget Guds Nåde;
\par 2 (han siger jo: "På en behagelig Tid bønhørte jeg dig, og på en Frelsens Dag hjalp jeg dig." Se, nu er det en velbehagelig Tid, se, nu er det en Frelsens Dag;)
\par 3 og vi give ikke i nogen Ting noget Anstød, for at Tjenesten ikke skal blive lastet;
\par 4 men i alting anbefale vi som Guds Tjenere os selv ved stor Udholdenhed i Trængsler, i Nød, i Angster,
\par 5 under Slag, i Fængsler, under Oprør, under Besværligheder, i Nattevågen, i Faste,
\par 6 ved Renhed, ved Kundskab, ved Langmodighed, ved Velvillighed, ved den Helligånd, ved uskrømtet Kærlighed,
\par 7 ved Sandheds Ord, ved Guds Kraft, ved Retfærdighedens Våben både til Angreb og Forsvar;
\par 8 ved Ære og Vanære, ved ondt Rygte og godt Rygte; som Forførere og dog sanddru;
\par 9 som ukendte og dog velkendte; som døende, og se, vi leve; som de, der tugtes, dog ikke til Døde;
\par 10 som bedrøvede, dog altid glade; som fattige, der dog gøre mange rige; som de, der intet have, og dog eje alt.
\par 11 Vor Mund er opladt over for eder, Korinthiere! vort Hjerte er udvidet.
\par 12 I have ikke snæver Plads i os, men der er snæver Plads i eders Hjerter.
\par 13 Men ligeså til Gengæld (jeg taler som til mine Børn), må også I udvide eders Hjerter!
\par 14 Drager ikke i ulige Åg med vantro; thi hvad Fællesskab har Retfærdighed og Lovløshed? eller hvad Samfund har Lys med Mørke?
\par 15 Hvad Samklang er der mellem Kristus og Belial? eller hvad Delagtighed har en troende med en vantro?
\par 16 Hvad Samstemning har Guds Tempel med Afguder? Thi vi ere den levende Guds Tempel, ligesom Gud har sagt: "Jeg vil bo og vandre iblandt dem, og jeg vil være deres Gud, og de skulle være mit Folk."
\par 17 "Derfor går ud fra dem og udskiller eder fra dem, siger Herren, og rører ikke noget urent; og jeg vil antage mig eder,"
\par 18 "og jeg vil være eders Fader, og I skulle være mine Sønner og Døtre, siger Herren, den Almægtige."

\chapter{7}

\par 1 Derfor, efterdi vi have disse Forjættelser, I elskede! så lader os rense os selv fra al Kødets og Åndens Besmittelse, så vi gennemføre Hellighed i Guds Frygt!
\par 2 Giver os Rum; ingen have vi gjort Uret, ingen ødelagt, ingen bedraget.
\par 3 Jeg siger det ikke for at fælde Dom; jeg har jo sagt tilforn, at I ere i vore Hjerter, så at vi dø sammen og leve sammen.
\par 4 Jeg har stor Frimodighed over for eder; jeg roser mig meget af eder, jeg er fuld af Trøst, jeg strømmer over af Glæde under al vor Trængsel.
\par 5 Thi også da vi kom til Makedonien, havde vort Kød ingen Ro, men vi trængtes på alle Måder: udadtil Kampe, indadtil Angster.
\par 6 Men han, som trøster de nedbøjede, Gud, han trøstede os ved Titus's Komme;
\par 7 dog ikke alene ved hans Komme, men også ved den Trøst, hvormed han var bleven trøstet over eder, idet han fortalte os om eders Længsel, eders Gråd, eders Nidkærhed for mig, så at jeg glædede mig end mere.
\par 8 Thi om jeg end har bedrøvet eder ved Brevet, fortryder jeg det ikke.
\par 9 så glæder jeg mig nu, ikke over, at I bleve bedrøvede, men over, at I bleve bedrøvede til Omvendelse; thi I bleve bedrøvede efter Guds Sind, for at I ikke i nogen Måde skulde lide Skade af os.
\par 10 Thi den Bedrøvelse, der er efter Guds Sind, virker Omvendelse til Frelse, som ikke fortrydes; men Verdens Bedrøvelse virker Død.
\par 11 Thi se, just dette, at I bleve bedrøvede efter Guds Sind, hvor stor en Iver virkede det ikke hos eder, ja Forsvar, ja Harme, ja Frygt, ja Længsel, ja Nidkærhed, ja Straf! På enhver Måde beviste I, at I selv vare rene i den Sag.
\par 12 Altså, når jeg skrev til eder, var det ikke for hans Skyld, som gjorde Uret, ikke heller for hans Skyld, som led Uret, men for at eders Iver for os skulde blive åbenbar hos eder for Guds Åsyn.
\par 13 Derfor ere vi blevne trøstede. Men til vor Trøst kom end yderligere Glæden over Titus's Glæde, fordi hans Ånd har fået Vederkvægelse fra eder alle.
\par 14 Thi i hvad jeg end har rost mig af eder for ham, er jeg ikke bleven til Skamme; men ligesom vi i alle Ting have talt Sandhed til eder, således er også vor Ros for Titus bleven Sandhed.
\par 15 Og hans Hjerte drages inderligere til eder, når han mindes Lydigheden hos eder alle,hvorledes I modtoge ham med Frygt og Bæven.
\par 16 Jeg glæder mig over, at jeg i alt kan lide på eder.

\chapter{8}

\par 1 Men vi kundgøre eder, Brødre! den Guds Nåde, som er given i Makedoniens Menigheder,
\par 2 at under megen Trængsels Prøvelse har deres overstrømmende Glæde og deres dybe Fattigdom strømmet over i deres Gavmildheds Rigdom.
\par 3 Thi efter Evne (det vidner jeg) gave de, ja, over Evne af egen Drift,
\par 4 idet de med megen Overtalelse bade os om den Nåde at måtte tage Del i Hjælpen til de hellige,
\par 5 og ikke alene som vi havde håbet, men sig selv gav de først og fremmest til Herren og så til os, ved Guds Villie,
\par 6 så at vi opfordrede Titus til, ligesom han forhen havde begyndt, således også til at tilendebringe hos eder også denne Gave.
\par 7 Men ligesom I ere rige i alt, i Tro og Tale og Erkendelse og al Iver og i eders Kærlighed til os: måtte I da være rige også i denne Gave!
\par 8 Jeg siger det ikke som en Befaling, men for ved andres Iver at prøve også eders Kærligheds Ægthed.
\par 9 I kende jo vor Herres Jesu Kristi Nåde, at han for eders Skyld blev fattig, da han var rig, for at I ved hans Fattigdom skulde blive rige.
\par 10 Og jeg giver min Mening herom til Kende; thi dette er eder gavnligt, I, som jo i Fjor vare de første til at begynde, ikke alene med Gerningen, men endogså med Villien dertil.
\par 11 Men fuldbringer da nu også Gerningen, for at, ligesom I vare redebonne til at ville, I også må fuldbringe det efter eders Evne.
\par 12 Thi når Redebonheden er til Stede, da er den velbehagelig efter, hvad den evner, ikke efter, hvad den ikke evner.
\par 13 Det er nemlig ikke Meningen, at andre skulle have Lettelse og I Trængsel; nej, det skal være ligeligt. Nu for Tiden må eders Overflod komme hines Trang til Hjælp,
\par 14 for at også hines Overflod kan komme eders Trang til Hjælp, for at der kan blive Ligelighed,
\par 15 som der er skrevet: "Den, som sankede meget, fik ikke for meget, og den, som sankede lidet, fik ikke for lidt."
\par 16 Men Gud ske Tak, som giver den samme Iver for eder i Titus's Hjerte!
\par 17 Thi vel har han modtaget min Opfordring; men da han er så ivrig, så er det af egen Drift, at han rejser til eder.
\par 18 Og sammen med ham sende vi den Broder, hvis Ros i Evangeliet går igennem alle Menighederne,
\par 19 og ikke det alene, men han er også udvalgt af Menighederne til at rejse med os med denne Gave, som besørges af os, for at fremme selve Herrens Ære og vor Redebonhed,
\par 20 idet vi undgå dette, at nogen skulde kunne laste os i Anledning af denne rige Hjælp, som besørges af os;
\par 21 thi vi lægge Vind på, hvad der er godt ikke alene i Herrens, men også i Menneskers Øjne.
\par 22 Men sammen med dem sende vi vor Broder, hvis Iver vi ofte i mange Måder have prøvet, men som nu er langt ivrigere på Grund af sin støre Tillid til eder.
\par 23 Hvad Titus angår, da er han min Fælle og Medarbejder hos eder, og hvad vore Brødre angår, da ere de Menighedsudsendinge, Kristi Ære.
\par 24 Så giver dem da for Menighedernes Åsyn Beviset på eders Kærlighed og for det, vi have rost eder for.

\chapter{9}

\par 1 Thi om Hjælpen til de hellige er det overflødigt at skrive til eder;
\par 2 jeg kend eders Redebonhed, for hvilken jeg roser mig af eder hos Makedonierne, at nemlig Akaja alt fra i Fjor har været beredt; og eders Nidkærhed æggede de fleste.
\par 3 Men jeg sender Brødrene, for at vor Ros over eder i dette Stykke ikke skal vise sig tom, og for at I.som jeg sagde, må være beredte.
\par 4 for at ikke, når der kommer Makedoniere med mig, og de finder eder uforberedte, vi (for ej at sige I) da skulle blive til Skamme med denne Tillidsfuldhed.
\par 5 Derfor har jeg anset det for nødvendigt at opfordre Brødrene til at gå i Forvejen til eder og forud bringe eders tidligere lovede Velsignelse i Stand, for at den kan være rede som Velsignelse og ikke som Karrighed.
\par 6 Men dette siger jeg: Den, som sår sparsomt, skal også høste sparsomt, og den, som sår med Velsignelser, skal også høste med Velsignelser.
\par 7 Enhver give, efter som han har sat sig for i sit Hjerte, ikke fortrædeligt eller af Tvang; thi Gud elsker en glad Giver.
\par 8 Men Gud er mægtig til at lade al Nåde rigeligt tilflyde eder, for at I i alting altid kunne have til fuld Tilfredshed og have rigeligt til al god Gerning,
\par 9 som der er skrevet: "Han spredte ud, han gav de fattige, hans Retfærdighed bliver til evig Tid."
\par 10 Men han, som giver "Sædemanden Sæd og Brød til at spise," han vil også skænke og mangfoldiggøre eders Udsæd og give eders Retfærdigheds Frugter Vækst,
\par 11 så I blive rige i alle Måder til al Gavmildhed, hvilken igennem os virker Taksigelse til Gud.
\par 12 Thi denne Offertjenestes Ydelse ikke alene afhjælper de helliges Trang, men giver også et Overskud ved manges Taksigelser til Gud,
\par 13 når de ved det prøvede Sind, som denne Ydelse viser, bringes til at prise Gud for Lydigheden i eders Bekendelse til Kristi Evangelium og for Oprigtigheden i eders Samfund med dem og med alle,
\par 14 også ved deres Bøn for eder, idet de længes efter eder på Grund af Guds overvættes Nåde imod eder.
\par 15 Gud ske Tak for hans uudsigelige Gave!

\chapter{10}

\par 1 Men jeg selv, Paulus, formaner eder ved Kristi Sagtmodighed og Mildhed, jeg, som, "når I se derpå, er ydmyg iblandt eder, men fraværende er modig over for eder",
\par 2 ja, jeg beder eder om ikke nærværende at skulle være modig med den Tillidsfuldhed, hvormed jeg agter at træde dristigt op imod nogle, som anse os for at vandre efter Kødet.
\par 3 Thi om vi end vandre i Kødet, så stride vi dog ikke efter Kødet;
\par 4 thi vore Stridsvåben er ikke kødelige, men mægtige for Gud til Fæstningers Nedbrydelse,
\par 5 idet vi nedbryde Tankebygninger og al Højhed, som rejser sig imod Erkendelsen af Gud, og tage enhver Tanke til Fange til Lydighed imod Kristus
\par 6 og ere rede til at straffe al Ulydighed, når eders Lydighed er bleven fuldkommen.
\par 7 Se I på det udvortes? Dersom nogen trøster sig til selv at høre Kristus til, da slutte han igen fra sig selv, at ligesom han hører Kristus til, således gøre vi det også.
\par 8 Ja, dersom jeg endog vilde rose mig noget mere af vor Magt, som Herren gav os til eders Opbyggelse og ikke til eders Nedbrydelse, skal jeg dog ikke blive til Skamme,
\par 9 for at jeg ikke skal synes at ville skræmme eder ved mine Breve;
\par 10 thi Brevene, siger man, ere vægtige og stærke, men hans legemlige Nærværelse er svag, og hans Tale intet værd.
\par 11 En sådan betænke, at således som vi fraværende ere med Ord ved Breve, således ville vi, også nærværende være i Gerning.
\par 12 Thi vi driste os ikke til at regne os iblandt eller sammenligne os med somme af dem, der anbefale sig selv; men selv indse de ikke, at de måle sig med sig selv og sammenligne sig med sig selv.
\par 13 Vi derimod ville ikke rose os ud i det umålelige, men efter Målet af den Grænselinie, som Gud har tildelt os som Mål, at nå også til eder.
\par 14 Thi vi strække os ikke for vidt, som om vi ikke nåede til eder; vi ere jo komne også indtil eder i Kristi Evangelium,
\par 15 så vi ikke rose os ud i det umålelige af andres Arbejder, men have det Håb, at, når eders Tro vokser, ville vi hos eder blive store, efter vor Grænselinie, så vi kunne komme langt videre
\par 16 og forkynde Evangeliet i Landene hinsides eder, men ikke rose os inden for en andens Grænselinie af det allerede fuldførte.
\par 17 Men den, som roser sig, rose sig af Herren!
\par 18 Thi ikke den, der anbefaler sig selv, står Prøve, men den, hvem Herren anbefaler.

\chapter{11}

\par 1 Gid I vilde finde eder i en Smule Dårskab af mig! Dog, I gør det Jo nok.
\par 2 Thi jeg er nidkær for eder med Guds Nidkærhed; jeg har jo trolovet eder med een Mand for at fremstille en ren Jomfru for Kristus.
\par 3 Men jeg frygter for, at ligesom Slangen bedrog Eva ved sin Træskhed, således skulle eders Tanker fordærves og miste det oprigtige Sindelag over for Kristus.
\par 4 Thi dersom nogen kommer og prædiker en anden Jesus, som vi ikke prædikede, eller I få en anderledes Ånd, som I ikke fik, eller et anderledes Evangelium, som I ikke modtoge, da vilde I kønt finde eder deri.
\par 5 Thi jeg mener ikke at stå tilbage i noget for de såre store Apostle.
\par 6 Er jeg end ulærd i Tale, så er jeg det dog ikke i Kundskab; tværtimod på enhver Måde have vi lagt den for Dagen for eder i alle Stykker.
\par 7 Eller gjorde jeg Synd i at fornedre mig selv, for at I skulde ophøjes, idet jeg forkyndte eder Guds Evangelium for intet?
\par 8 Andre Menigheder plyndrede jeg, idet jeg tog Sold af dem for at tjene eder, og medens jeg var nærværende hos eder og kom i Trang, faldt jeg ingen til Byrde;
\par 9 thi min Trang afhjalp Brødrene, da de kom fra Makedonien, og i alt har jeg holdt og vil jeg holde mig uden Tynge for eder.
\par 10 Så vist som Kristi Sandhed er i mig, skal denne Ros ikke fratages mig i Akajas Egne.
\par 11 Hvorfor? mon fordi jeg ikke elsker eder? Gud ved det.
\par 12 Men hvad jeg gør, det vil jeg fremdeles gøre, for at jeg kan afskære dem Lejligheden, som søge en Lejlighed, til at findes os lige i det, hvoraf de rose sig.
\par 13 Thi sådanne ere falske Apostle, svigefulde Arbejdere, som påtage sig Skikkelse af Kristi Apostle.
\par 14 Og det er intet Under; thi Satan selv påtager sig Skikkelse af en Lysets Engel.
\par 15 Derfor er det ikke noget stort, om også hans Tjenere påtage sig Skikkelse som Retfærdigheds Tjenere; men deres Ende skal være efter deres Gerninger.
\par 16 Atter siger jeg: Ingen må agte mig for en Dåre; men hvis så skal være, så tåler mig endog som en Dåre, for at også jeg kan rose mig en Smule.
\par 17 Hvad jeg nu taler, taler jeg ikke efter Herrens Sind, men som i Dårskab, idet jeg så tillidsfuldt roser mig.
\par 18 Efterdi mange rose sig med Hensyn til Kødet, vil også jeg rose mig.
\par 19 Gerne finde I eder jo i Dårerne, efterdi I ere kloge.
\par 20 I finde eder jo i, om nogen gør eder til Trælle, om nogen æder eder op, om nogen tager til sig, om nogen ophøjer sig, om nogen slår eder i Ansigtet.
\par 21 Med Skamfuldhed siger jeg det, efterdi vi have været svage; men hvad end nogen trodser på (jeg taler i Dårskab), derpå trodser også jeg.
\par 22 Ere de Hebræere? Jeg også. Ere de Israeliter? Jeg også. Ere de Abrahams Sæd? Jeg også.
\par 23 Ere de Kristi Tjenere? Jeg taler i Vanvid: jeg er det mere. Jeg har lidt langt flere Besværligheder, fået langt flere Slag, været hyppigt i Fængsel, ofte i Dødsfare.
\par 24 Af Jøder har jeg fem Gange fået fyrretyve Slag mindre end eet.
\par 25 Tre Gange er jeg bleven pisket, een Gang stenet, tre Gange har jeg lidt Skibbrud, et Døgn har jeg tilbragt på Dybet;
\par 26 ofte på Rejser, i Farer fra Floder, i Farer iblandt Røvere, i Farer fra mit Folk, i Farer fra Hedninger, i Farer i By, i Farer i Ørken, i Farer på Havet, i Farer iblandt falske Brødre;
\par 27 i Møje og Anstrengelse, ofte i Nattevågen, i Hunger og Tørst, ofte i Faste, i Kulde og Nøgenhed;
\par 28 foruden hvad der kommer til, mit daglige Overløb, Bekymringen for alle Menighederne.
\par 29 Hvem er skrøbelig, uden at også jeg er det? hvem bliver forarget, uden at det brænder i mig?
\par 30 Dersom jeg skal rose mig, da vil jeg rose mig af min Magtesløshed.
\par 31 Gud og den Herres Jesu Fader, som er højlovet i Evighed, ved, at jeg ikke lyver.
\par 32 I Damaskus holdt Kong Aretas's Statholder Damaskenernes Stad bevogtet for at gribe mig;
\par 33 men jeg blev igennem en Luge firet ned over Muren i en Kurv og undflyede af hans Hænder.

\chapter{12}

\par 1 Rose mig må jeg Gavnligt er det vel ikke; men jeg vil komme til Syner og Åbenbarelser fra Herren.
\par 2 Jeg kender et Menneske i Kristus, som for fjorten År siden om han var i Legemet, det ved jeg ikke, eller uden for Legemet, det ved jeg ikke, Gud ved det blev bortrykket indtil den tredje Himmel.
\par 3 Og jeg ved, at dette Menneske (om han var i Legemet, eller uden Legemet, det ved jeg ikke, Gud ved det),
\par 4 at han blev bortrykket ind i Paradiset, og hørte uudsigelige Ord, som det ikke er et Menneske tilladt at udtale.
\par 5 Af en sådan vil jeg rose mig; men af mig selv vil jeg ikke rose mig, uden af min Magtesløshed.
\par 6 Thi vel bliver jeg ikke en Dåre, om jeg vilde rose mig; thi det vil være Sandhed, jeg siger; men jeg afholder mig derfra, for at ingen skal tænke højere om mig, end hvad han ser mig være, eller hvad han hører af mig.
\par 7 Og for at jeg ikke skal hovmode mig af de høje Åbenbarelser, blev der givet mig en Torn i Kødet, en Satans Engel, for at han skulde slå mig i Ansigtet, for at jeg ikke skulde hovmode mig.
\par 8 Om denne bad jeg Herren tre Gange, at han måtte vige fra mig;
\par 9 og han har sagt mig: "Min Nåde er dig nok; thi Kraften fuldkommes i Magtesløshed." Allerhelst vil jeg derfor rose mig af min Magtesløshed, for at Kristi Kraft kan tage Bolig i mig.
\par 10 Derfor er jeg veltilfreds under Magtesløshed, under Overlast, under Nød, under Forfølgelser, under Angster for Kristi Skyld; thi når jeg er magtesløs, da er jeg stærk.
\par 11 Jeg er bleven en Dåre. I tvang mig dertil. Jeg burde jo anbefales af eder; thi jeg har ikke stået tilbage i noget for de såre store Apostle, om jeg end, intet er.
\par 12 En Apostels Tegn bleve jo udførte, iblandt eder under Udholdenhed, ved Tegn og Undere og kraftige Gerninger.
\par 13 Thi hvad er det vel, hvori I bleve stillede ringere end de andre Menigheder; uden at jeg ikke selv faldt eder til Byrde? Tilgiver mig denne Uret!
\par 14 Se, dette er nu tredje Gang, jeg står rede til at komme til eder, og jeg vil ikke falde til Byrde; thi jeg søger ikke eders Gods, men eder selv, thi Børnene skulle ikke samle sammen til Forældrene, men Forældrene til Børnene.
\par 15 Men jeg vil med Glæde gøre Opofrelser ja, opofres for eders Sjæle.
\par 16 Men lad så være, at jeg ikke har været eder til Byrde, men jeg var træsk og fangede eder med List!
\par 17 Har jeg da gjort mig Fordel af eder ved nogen af dem, jeg har sendt til eder?
\par 18 Jeg opfordrede Titus og sendte Broderen med; har Titus da gjort sig nogen Fordel af eder? Vandrede vi ikke i den samme Ånd, i de samme Fodspor?
\par 19 Alt længe have I ment, at vi forsvare os for eder. Nej, for Guds Åsyn tale vi i Kristus. Men det sker alt sammen, I elskede, for eders Opbyggelses Skyld.
\par 20 Thi jeg frygter for, at, når jeg kommer, jeg da måske ikke skal finde eder sådanne, som jeg ønsker, og at jeg skal findes af eder sådan, som I ikke ønske; at der skal være Kiv, Nid, Hidsighed, Rænker, Bagtalelser, Øretuderier, Opblæsthed, Klammerier,
\par 21 at min Gud, når jeg kommer igen, skal ydmyge mig i Anledning af eder, og jeg skal sørge over mange af dem, som forhen have syndet og ikke have omvendt sig fra den Urenhed og Utugt og Uterlighed, som de bedreve.

\chapter{13}

\par 1 Det er nu tredje Gang, jeg kommer til eder. På to og tre Vidners Mund skal enhver Sag stå fast.
\par 2 Jeg har sagt det forud og siger det forud, ligesom da jeg anden Gang var nærværende, således også nu fraværende til dem, som forhen have syndet, og til alle de øvrige, at, om jeg kommer igen, vil jeg ikke skåne,
\par 3 efterdi I fordre Bevis på, at Kristus taler i mig, han, som ikke er magtesløs over for eder, men er stærk iblandt eder.
\par 4 Thi vel blev han korsfæstet i Magtesløshed, men han lever ved Guds Kraft; også vi ere svage i ham, men vi skulle leve med ham ved Guds Kraft over for eder.
\par 5 Ransager eder selv, om I ere i Troen; prøver eder selv! Eller erkende I ikke om eder selv, at Jesus Kristus er i eder? ellers ere I udygtige.
\par 6 Men jeg håber, at I skulle kende, at vi ere ikke udygtige.
\par 7 Men vi bede til Gud om, at I intet ondt må gøre; ikke for at vi må vise os dygtige, men for at I må gøre det gode, vi derimod stå som udygtige.
\par 8 Thi vi formå ikke noget imod Sandheden, men for Sandheden.
\par 9 Thi vi glæde os, når vi ere magtesløse, og I ere stærke; dette ønske vi også, at I må blive fuldkommengjorte.
\par 10 Derfor skriver jeg dette fraværende, for at jeg ikke nærværende skal bruge Strenghed, efter den Magt, som Herren har givet mig til Opbyggelse, og ikke til Nedbrydelse.
\par 11 I øvrigt, Brødre! glæder eder, bliver fuldkommengjorte, lader eder formane, værer enige, værer fredsommelige; og Kærlighedens og Fredens Gud skal være med eder.
\par 12 Hilser hverandre med et helligt Kys! Alle de hellige hilse eder.
\par 13 Den Herres Jesu Kristi Nåde og Guds Kærlighed og den Helligånds Samfund være med eder alle!



\end{document}