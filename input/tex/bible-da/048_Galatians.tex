\begin{document}

\title{Galaterbrevet}


\chapter{1}

\par 1 Paulus, Apostel, ikke af Mennesker, ikke heller ved noget Menneske, men ved Jesus Kristus og Gud Fader, som oprejste ham fra de døde,
\par 2 og alle Brødrene, som ere med mig, til Menighederne i Galatien:
\par 3 Nåde være med eder og Fred fra Gud Fader og vor Herre Jesus Kristus,
\par 4 som gav sig selv for vore Synder, for at han kunde udfri os af den nærværende onde Verden, efter vor Guds og Faders Villie,
\par 5 ham være Æren i Evigheders Evighed! Amen.
\par 6 Jeg undrer mig over, at I så snart lade eder føre bort fra ham, som kaldte eder til Kristi Nåde, hen til et anderledes Evangelium;
\par 7 hvilket dog ikke er et andet, men det er kun nogle, som forvirre eder. og ville vende op og ned på Kristi Evangelium.
\par 8 Men selv om vi eller en Engel fra Himmelen forkynder eder Evangeliet anderledes; end vi have forkyndt eder det, han være en Forbandelse!
\par 9 Som vi før have sagt, så siger jeg nu igen: Dersom nogen forkynder eder Evangeliet anderledes, end I have modtaget det, han være en Forbandelse!
\par 10 Taler jeg da nu Mennesker til Villie, eller Gud? eller søger jeg at behage Mennesker? Dersom jeg endnu vilde behage Mennesker, da var jeg ikke en Kristi Tjener.
\par 11 Men jeg kundgør eder, Brødre! at det Evangelium, som er forkyndt af mig, er ikke Menneskeværk;
\par 12 thi heller ikke jeg har modtaget det eller er bleven undervist derom af noget Menneske, men ved Åbenbarelse at Jesus Kristus.
\par 13 I have jo hørt om min Vandel forhen i Jødedommen, at jeg over al Måde forfulgte Guds Menighed og søgte at udrydde den.
\par 14 Og jeg gik videre i Jødedommen end mange jævnaldrende i mit Folk, idet jeg var langt mere nidkær for mine fædrene Overleveringer.
\par 15 Men da det behagede Gud, som fra min Moders Liv havde udtaget mig og havde kaldet mig ved sin Nåde,
\par 16 at åbenbare sin Søn i mig, for at jeg skulde forkynde Evangeliet om ham iblandt Hedningerne: da spurgte jeg straks ikke Kød og Blod til Råds,
\par 17 drog heller ikke op til Jerusalem, til dem, som før mig vare Apostle, men jeg drog bort til Arabien og vendte atter tilbage til Damaskus.
\par 18 Senere, tre År efter, drog jeg op til Jerusalem for at blive kendt med Kefas og blev hos ham i femten Dage.
\par 19 Men nogen anden af Apostlene så jeg ikke, men kun Jakob, Herrens Broder.
\par 20 Men hvad jeg skriver til eder - se, for Guds Åsyn vidner jeg, at jeg ikke lyver.
\par 21 Derefter kom jeg til Syriens og Kilikiens Egne.
\par 22 Men personlig var jeg ukendt for Judæas Menigheder i Kristus;
\par 23 de hørte kun sige: Han, som forhen forfulgte os, forkynder nu Evangeliet om den Tro, som han forhen vilde udrydde;
\par 24 og de priste Gud for mig.

\chapter{2}

\par 1 Senere, efter fjorten Års Forløb, drog jeg atter op til Jerusalem med Brnabas og tog også Titus med.
\par 2 Men jeg drog op ifølge en Åbenbaring og forelagde dem, men særskilt de ansete, det Evangelium, som jeg prædiker iblandt Hedningerne, - om jeg vel løber eller har løbet forgæves.
\par 3 Men end ikke min Ledsager, Titus, som var en Græker, blev tvungen til at omskæres,
\par 4 nemlig for de indsnegne falske Brødres Skyld, som jo havde listet sig ind for at lure på vor Frihed, som vi have i Kristus Jesus, for at de kunde gøre os til Trælle.
\par 5 For dem vege vi end ikke et Øjeblik i Eftergivenhed, for at Evangeliets Sandhed måtte blive varig hos eder.
\par 6 Men fra deres Side, som ansås for at være noget, (hvordan de fordum vare, er mig uden Forskel; Gud ser ikke på et Menneskes Person;) - over for mig nemlig havde de ansete intet at tilføje.
\par 7 Men tværtimod, da de så, at jeg har fået Evangeliet til de uomskårne betroet, ligesom Peter til de omskårne,
\par 8 (thi han, som gav Peter Kraft til Apostelgerning for de omskårne, gav også mig Kaft dertil for Hedningerne;)
\par 9 og da de lærte den mig givne Nåde at kende, gave Jakob og Kefas og Johannes, som ansås for at være Søjler, mig og Barnabas Samfundshånd for at vi skulde gå til Hedningerne og de til de omskårne;
\par 10 kun at vi skulde komme de fattige i Hu, hvad jeg også just har bestræbt mig for at gøre.
\par 11 Men da Kefas kom til Antiokia, trådte jeg op imod ham for hans åbne Øjne, thi domfældt var han.
\par 12 Thi førend der kom nogle fra Jakob, spiste han sammen med Hedningerne; men da de kom, trak han sig tilbage og skilte sig fra dem af Frygt for dem af Omskærelsen.
\par 13 Og med ham hyklede også de øvrige Jøder, så at endog Barnabas blev dragen med af deres Hykleri.
\par 14 Men da jeg så, at de ikke vandrede rettelig efter Evangeliets Sandhed, sagde jeg til Kefas i alles Påhør: Når du, som er en Jøde, lever på hedensk og ikke på jødisk Vis, hvor kan du da tvinge Hedningerne til at opføre sig som Jøder?
\par 15 Vi ere af Natur Jøder og ikke Syndere af hedensk Byrd;
\par 16 men da vi vide, at et Menneske ikke bliver retfærdiggjort af Lovens Gerninger, men kun ved Tro på Jesus Kristus, så have også vi troet på Kristus Jesus, for at vi måtte blive retfærdiggjorte al Tro på Kristus og ikke af Lovens Gerninger; thi af Lovens Geringer skal intet Kød blive retfærdiggjort.
\par 17 Men når vi, idet vi søgte at blive retfærdiggjorte i Kristus, også selv fandtes at være Syndere, så er jo Kristus en Tjener for Synd? Det være langtfra!
\par 18 når jeg nemlig igen bygger det op, som jeg nedbrød, da viser jeg mig selv som Overtræder.
\par 19 Thi jeg er ved Loven død fra Loven, for at jeg skal leve for Gud.
\par 20 Med Kristus er jeg korsfæstet, og det er ikke mere mig, der lever, men Kristus lever i mig; men hvad jeg nu lever, i Kødet, det lever jeg i Troen, på Guds Søn, som elskede mig og gav sig selv hen for mig.
\par 21 Jeg ophæver ikke Guds Nåde; thi er der Retfærdighed ved Loven, da er jo Kristus død forgæves.

\chapter{3}

\par 1 O, I uforstandige Galatere! hvem har fortryllet eder, I, hvem Jesus Kristus blev malet for Øje som korsfæstet?
\par 2 Kun dette vil jeg vide af eder: Var det ved Lovens Gerninger, I modtoge Ånden, eller ved i Tro at høre?
\par 3 Ere I så uforstandige? ville I, som begyndte i Ånd, nu ende i Kød?
\par 4 Have I da prøvet så meget forgæves? hvis det da virkelig er forgæves!
\par 5 Mon da han, som meddeler eder Ånden og virker kraftige Gerninger iblandt eder, gør dette ved Lovens Gerninger eller ved, at I høre i Tro?
\par 6 ligesom jo "Abraham troede Gud, og det blev regnet ham til Retfærdighed".
\par 7 Erkender altså, at de, som ere af Tro, disse ere Abrahams Børn.
\par 8 Men da Skriften forudså, at det er af Tro, at Gud retfærdiggør Hedningerne, forkyndte den forud Abraham det Evangelium: "I dig skulle alle Folkeslagene velsignes",
\par 9 så at de, som ere af Tro, velsignes sammen med den troende Abraham.
\par 10 Thi så mange, som holde sig til Lovens Gerninger, ere under Forbandelse; thi der er skrevet: "Forbandet hver den, som ikke bliver i alle de Ting, som ere skrevne i Lovens Bog, så han gør dem."
\par 11 Men at ingen bliver retfærdiggjort for Gud ved Lov, er åbenbart, thi "deri retfærdige skal leve af Tro."
\par 12 Men Loven beror ikke på Tro; men: "Den, som gør disse Ting, skal leve ved dem."
\par 13 Kristus har løskøbt os fra Lovens Forbandelse, idet han blev en Forbandelse for os (thi der er skrevet: "Forbandet er hver den, som hænger på et Træ"),
\par 14 for at Abrahams Velsignelse måtte komme til Hedningerne i Kristus Jesus, for at vi kunde få Åndens Forjættelse ved Troen.
\par 15 Brødre! jeg taler på Menneskevis: Ingen ophæver dog et Menneskes stadfæstede Arvepagt eller føjer noget dertil.
\par 16 Men Abraham og hans Sæd bleve Forjættelserne tilsagte; der siges ikke: "og Sædene", som om mange, men som om eet: "og din Sæd", hvilken er Kristus.
\par 17 Jeg mener dermed dette: En Pagt, som forud er stadfæstet af Gud, kan Loven, som blev til fire Hundrede og tredive År senere, ikke gøre ugyldig, så at den skulde gøre Forjættelsen til intet.
\par 18 Thi fås Arven ved Lov, da fås den ikke mere ved Forjættelse; men til Abraham har Gud skænket den ved Forjættelse.
\par 19 Hvad skulde da Loven? Den blev føjet til for Overtrædelsernes Skyld (indtil den Sæd kom, hvem Forjættelsen gjaldt), besørget ved Engle, ved en Mellemmands Hånd.
\par 20 Men en Mellemmand er ikke kun for een Part; Gud derimod er een.
\par 21 Er da Loven imod Guds Forjættelser? Det være langtfra! Ja, hvis der var givet en Lov, som kunde levendegøre, da var Retfærdigheden virkelig af Lov.
\par 22 Men Skriften har indesluttet alt under Synd, for at Forjættelsen skulde af Tro på Jesus Kristus gives dem, som tro.
\par 23 Men førend Troen kom, holdtes vi indelukkede under Lovens Bevogtning til den Tro, som skulde åbenbares,
\par 24 så at Loven er bleven os en Tugtemester til Kristus, for at vi skulde blive retfærdiggjorte af Tro.
\par 25 Men efter at Troen er kommen, ere vi ikke mere under Tugtemester.
\par 26 Thi alle ere I Guds Børn ved Troen på Kristus Jesus.
\par 27 Thi I, så mange som bleve døbte til Kristus, have iført eder Kristus.
\par 28 Her er ikke Jøde eller Græker; her er ikke Træl eller fri; her er ikke Mand og Kvinde; thi alle ere I een i Kristus Jesus.
\par 29 Men når I høre Kristus til, da ere I jo Abrahams Sæd, Arvinger ifølge Forjættelse.

\chapter{4}

\par 1 Men jeg siger: Så længe Arvingen er umyndig, er der ingen Forskel imellem ham og en Træl, skønt han er Herre over alt Godset;
\par 2 men han står under Formyndere og Husholdere indtil den af Faderen bestemte Tid.
\par 3 Således stode også vi, dengang vi vare umyndige, som Trælle under Verdens Børnelærdom.
\par 4 Men da Tidens Fylde kom, udsendte Gud sin Søn, født af en Kvinde, født under Loven,
\par 5 for at han skulde løskøbe dem, som vare under Loven, for at vi skulde få Sønneudkårelsen.
\par 6 Men fordi I ere Sønner, har Gud udsendt i vore Hjerter sin Søns Ånd, som råber: Abba, Fader!
\par 7 Altså er du ikke længer Træl, men Søn; men er du Søn, da er du også Arving ved Gud.
\par 8 Dengang derimod, da I ikke kendte Gud, trællede I for de Guder, som af Natur ikke ere det.
\par 9 Men nu, da I have lært Gud at kende, ja, meget mere ere blevne kendte af Gud, hvor kunne I da atter vende tilbage til den svage og fattige Børnelærdom, som I atter forfra ville trælle under?
\par 10 I tage Vare på Dage og Måneder og Tider og År.
\par 11 Jeg frygter for, at jeg måske har arbejdet forgæves på eder.
\par 12 Vorder ligesom jeg, thi også jeg er bleven som I, Brødre! jeg beder eder. I have ikke gjort mig nogen Uret.
\par 13 Men I vide, at det var på Grund af en Kødets Svaghed, at jeg første Gang forkyndte Evangeliet for eder;
\par 14 og det, som i mit Kød var eder til Fristelse, ringeagtede I ikke og afskyede I ikke, men I modtoge mig som en Guds Engel, som Kristus Jesus.
\par 15 Hvor er da nu eders Saligprisning? Thi jeg giver eder det Vidnesbyrd, at, om det havde været muligt, havde I udrevet eders Øjne og givet mig dem.
\par 16 Så er jeg vel bleven eders Fjende ved at tale Sandhed til eder?
\par 17 De ere nidkære for eder, dog ikke for det gode; men de ville udelukke eder, for at I skulle være nidkære for dem.
\par 18 Men det er godt at vise sig nidkær i det gode til enhver Tid, og ikke alene, når jeg er nærværende hos eder.
\par 19 Mine Børn, som jeg atter føder med Smerte, indtil Kristus har vundet Skikkelse i eder!
\par 20 - ja, jeg vilde ønske, at jeg nu var til Stede hos eder og kunde omskifte min Røst; thi jeg er rådvild over for eder.
\par 21 Siger mig, I, som ville være under Loven, høre I ikke Loven?
\par 22 Der er jo skrevet, at Abraham havde to Sønner, en med Tjenestekvinden og en med den frie Kvinde.
\par 23 Men Tjenestekvindens Søn er avlet efter Kødet, den frie Kvindes ved Forjættelsen.
\par 24 Dette har en billedlig Betydning. Thi disse Kvinder ere tvende Pagter, den ene fra Sinai Bjerg, som føder til Trældom: denne er Hagar.
\par 25 Thi "Hagar" er Sinai Bjerg i Arabien, men svarer til det nuværende Jerusalem; thi det er i Trældom med sine Børn.
\par 26 Men Jerusalem heroventil er frit, og hun er vor Moder.
\par 27 Thi der er skrevet: "Fryd dig, du ufrugtbare, du, som ikke føder! bryd ud og råb, du, som ikke har Fødselsveer! thi mange ere den enliges Børn fremfor hendes, som har Manden."
\par 28 Men vi, Brødre! ere Forjættelsens Børn i Lighed med Isak.
\par 29 Men ligesom dengang han, som var avlet efter Kødet, forfulgte ham, som var avlet efter Ånden, således også nu.
\par 30 Men hvad siger Skriften?"Uddriv Tjenestekvinden og hendes Søn; thi Tjenestekvindens Søn skal ingenlunde arve med den frie Kvindes Søn."
\par 31 Derfor, Brødre! ere vi ikke Tjenestekvindens Børn, men den frie Kvindes.

\chapter{5}

\par 1 Til Friheden har Kristus frigjort os. Så står nu fast, og lader eder ikke atter holde under Trældoms Åg!
\par 2 Se, jeg, Paulus, siger eder, at dersom I lade eder omskære, vil Kristus intet gavne eder.
\par 3 Men jeg vidner atter for hvert Menneske, som lader sig omskære, at han er skyldig at opfylde hele Loven.
\par 4 I ere tabte for Kristus, I, som retfærdiggøres ved Loven; I ere faldne ud af Nåden.
\par 5 Vi vente jo ved Ånden af Tro Retfærdigheds Håb.
\par 6 Thi i Kristus Jesus gælder hverken Omskærelse eller Forhud noget, men Tro, som er virksom ved Kærlighed.
\par 7 I vare godt på Vej; hvem har hindret eder i at adlyde Sandhed?
\par 8 Den Overtalelse kom ikke fra ham, som kaldte eder.
\par 9 En liden Surdejg syrer hele Dejgen.
\par 10 Jeg har den Tillid til eder i Herren, at I ikke ville mene noget andet; men den, som forvirrer eder, skal bære sin Dom, hvem han end er.
\par 11 Men jeg, Brødre! dersom jeg endnu prædiker Omskærelse, hvor for forfølges jeg da endnu? Så er jo Korsets Forargelse gjort til intet.
\par 12 Gid de endog måtte lemlæste sig selv, de, som forstyrre eder!
\par 13 I bleve jo kaldede til Frihed, Brødre! kun at I ikke bruge Friheden til en Anledning for Kødet, men værer ved Kærligheden hverandres Tjenere!
\par 14 Thi hele Loven er opfyldt i eet Ord, i det: "Du skal elske din Næste som dig selv."
\par 15 Men når I bide og æde hverandre, da ser til, at I ikke fortæres af hverandre!
\par 16 Men jeg siger: Vandrer efter Ånden, så fuldbyrde I ingenlunde Kødets Begæring.
\par 17 Thi Kødet begærer imod Ånden, og Ånden imod Kødet; disse stå nemlig hinanden imod,for at I ikke skulle gøre, hvad I have Lyst til.
\par 18 Men når I drives af Ånden, ere I ikke under Loven.
\par 19 Men Kødets Gerninger ere åbenbare, såsom: Utugt, Urenhed, Uterlighed,
\par 20 Afgudsdyrkelse,Trolddom,Fjendskaber, Kiv, Nid, Hidsighed, Rænker, Tvedragt, Partier,
\par 21 Avind, Drukkenskab, Svir og deslige; hvorom jeg forud siger eder, ligesom jeg også før har sagt, at de, som øve sådanne Ting, skulle ikke arve Guds Rige.
\par 22 Men Åndens Frugt er Kærlighed, Glæde, Fred, Langmodighed, Mildhed, Godhed, Trofasthed,
\par 23 Sagtmodighed, Afholdenhed Imod sådanne er Loven ikke,
\par 24 men de, som høre Kristus Jesus til, have korsfæstet Kødet med dets Lidenskaber og Begæringer.
\par 25 Når vi leve ved Ånden, da lader os også vandre efter Ånden!
\par 26 Lader os ikke have Lyst til tom Ære, så at vi udæske hverandre og bære Avind imod hverandre.

\chapter{6}

\par 1 Brødre! om også et Menneske bliver overrasket af nogen Forsyndelse, da hjælper en sådan til Rette, I åndelige! med Sagtmodigheds Ånd, og se til dig selv, at ikke også du bliver fristet!
\par 2 Bærer hverandres Byrder og opfylder således Kristi Lov!
\par 3 Thi når nogen mener, at han er noget, skønt han intet er, da bedrager han sig selv.
\par 4 Men hver prøve sin egen Gerning, og da skal han have sin Ros i Forhold til sig selv alene, og ikke til Næsten;
\par 5 thi hver skal bære sin egen Byrde.
\par 6 Men den, som undervises i Ordet skal dele alt godt med den, som underviser ham.
\par 7 Farer ikke vild; Gud lader sig ikke spotte; thi hvad et Menneske sår, det skal han også høste.
\par 8 Thi den, som sår i sit Kød, skal høste Fordærvelse af Kødet; men den, som sår i Ånden, skal høste evigt Liv af Ånden.
\par 9 Men når vi gøre det gode, da lader os ikke blive trætte; thi i sin Tid skulle vi høste, såfremt vi ikke give tabt.
\par 10 Så lader os altså, efter som vi have Lejlighed, gøre det gode imod alle, men mest imod Troens egne!
\par 11 Ser nu, med hvor store Bogstaver jeg skriver til eder med min egen Hånd!
\par 12 Alle de, som ville tage sig godt ud i Kødet, de tvinge eder til at lade eder omskære, alene for at de ikke skulle forfølges for Kristi Kors's Skyld.
\par 13 Thi ikke engang de, som lade sig omskære, holde selv Loven; men de ville, at I skulle lade eder omskære, for at de kunne rose sig af eders Kød.
\par 14 Men det være langt fra mig at rose mig uden af vor Herres Jesu Kristi Kors,ved hvem Verden er korsfæstet for mig, og jeg for Verden.
\par 15 Thi hverken Omskærelse eller Forhud er noget, men en ny Skabning.
\par 16 Og så mange, som vandre efter denne Rettesnor, over dem være Fred og Barmhjertighed, og over Guds Israel!
\par 17 Herefter volde ingen mig Besvær; thi jeg bærer Jesu Mærketegn på mit Legeme:
\par 18 Vor Herres Jesu Kristi Nåde være med eders Ånd, Brødre! Amen.



\end{document}