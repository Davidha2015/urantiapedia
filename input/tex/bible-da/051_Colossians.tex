\begin{document}

\title{Kolossenserbrevet}


\chapter{1}

\par 1 Paulus, Kristi Jesu Apostel ved Guds Villie, og Broderen Timotheus.
\par 2 til de hellige og troende Brødre i Kristus i Kolossæ: Nåde være med eder og Fred fra Gud vor Fader!
\par 3 Vi takke Gud og vor Herres Jesu Kristi Fader altid, når vi bede for eder,
\par 4 da vi have hørt om eders Tro på Kristus Jesus og den Kærlighed, som I have til alle de hellige
\par 5 på Grund af det Håb, som er henlagt til eder i Himlene, om hvilket I forud have hørt i Evangeliets Sandheds Ord,
\par 6 der er kommet til eder, ligesom det også er i den hele Verden, idet det bærer Frugt og vokser, ligesom det også gør iblandt eder fra den Dag, I hørte og erkendte Guds Nåde i Sandhed,
\par 7 således som I have lært af Epafras, vor elskede Medtjener som er en tro Kristi Tjener for eder,
\par 8 han, som også gav os eders Kærlighed i Ånden til Kende.
\par 9 Derfor have også vi fra den Dag, vi hørte det, ikke ophørt at bede for eder og begære, at I måtte fyldes med Erkendelsen af hans Villie i al Visdom og åndelige Indsigt
\par 10 til at vandre Herren værdigt, til alt Velbehag, idet I bære Frugt og vokse i al god Gerning ved Erkendelsen af Gud,
\par 11 idet I styrkes med al Styrke efter hans Herligheds Kraft til al Udholdenhed og Tålmodighed med Glæde
\par 12 og takke Faderen, som gjorde os dygtige til at have Del i de helliges Arvelod i Lyset,
\par 13 han, som friede os ud af Mørkets Magt og satte os over i sin elskede Søns Rige,
\par 14 i hvem vi have Forløsningen, Syndernes Forladelse,
\par 15 han, som er den usynlige Guds Billede, al Skabnings førstefødte;
\par 16 thi i ham bleve alle Ting skabte i Himlene og på Jorden, de synlige og de usynlige, være sig Troner eller Herredømmer eller Magter eller Myndigheder. Alle Ting ere skabte ved ham og til ham;
\par 17 og han er forud for alle Ting, og alle Ting bestå ved ham.
\par 18 Og han er Legemets Hoved, nemlig Menighedens, han, som er Begyndelsen, førstefødt ud af de døde, for at han skulde blive den ypperste i alle Ting;
\par 19 thi det behagede Gud, at i ham skulde hele Fylden bo,
\par 20 og ved ham at forlige alle Ting med sig, være sig dem på Jorden eller dem i Himlene, idet han stiftede Fred ved hans Kors's Blod.
\par 21 Også eder, som fordum vare fremmedgjorte og fjendske af Sindelag i eders onde Gerninger,
\par 22 har han dog nu forligt i sit Køds Legeme ved Døden for at fremstille eder hellige og ulastelige og ustraffelige for sit Åsyn,
\par 23 så sandt I blive i Troen, grundfæstede og faste, uden at lade eder rokke fra Håbet i det Evangelium, som I have hørt, hvilket er blevet prædiket i al Skabningen under Himmelen, og hvis Tjener jeg Paulus er bleven.
\par 24 Nu glæder jeg mig over mine Lidelser for eder, og hvad der fattes i Kristi Trængsler, udfylder jeg i mit Kød for hans Legeme, som er Menigheden,
\par 25 hvis Tjener jeg er bleven efter den Guds Husholdning, som blev given mig over for eder, nemlig fuldelig at forkynde Guds Ord,
\par 26 den Hemmelighed, der var skjult igennem alle Tider og Slægter, men nu er bleven åbenbaret for hans hellige,
\par 27 hvem Gud vilde tilkendegive, hvilken Rigdom på Herlighed iblandt Hedningerne der ligger i denne Hemmelighed, som er Kristus i eder, Herlighedens Håb,
\par 28 hvem vi forkynde, idet vi påminde hvert Menneske og lære hvert Menneske med al Visdom, for at vi kunne fremstille hvert Menneske som fuldkomment i Kristus;
\par 29 hvorpå jeg også arbejder, idet jeg kæmper ifølge hans Kraft, som virker mægtigt i mig.

\chapter{2}

\par 1 Thi jeg vil, at I skulle vide, hvor stor en Kamp jeg har for eder og for dem i Laodikea og for alle, som ikke have set mit Åsyn i Kødet,
\par 2 for at deres Hjerter må opmuntres, idet de sammenknyttes i Kærlighed og til den fuldvisse Indsigts hele Rigdom, til Erkendelse af Guds Hemmelighed, Kristus.
\par 3 i hvem alle Visdommens og Kundskabens Skatte findes skjulte.
\par 4 Dette siger jeg, for at ingen skal bedrage eder med lokkende Tale.
\par 5 Thi om jeg også i Kødet er fraværende, så er jeg dog i Ånden hos eder og glæder mig ved at se eders Orden og Fastheden i eders Tro på Kristus.
\par 6 Derfor, ligesom I have modtaget Kristus Jesus, Herren, så vandrer i ham,
\par 7 idet I ere rodfæstede og opbygges i ham og, stadfæstes ved Troen, således som I bleve oplærte, så I vokse i den med Taksigelse.
\par 8 Ser til, at der ikke skal være nogen, som gør eder til Bytte ved den verdslige Visdom og tomt Bedrag efter Menneskers Overlevering, efter Verdens Børnelærdom og ikke efter Kristus;
\par 9 thi i ham bor Guddommens hele Fylde legemlig,
\par 10 og i ham have I eders Fylde, ham, som er Hovedet for al Magt og Myndighed;
\par 11 i hvem I også ere blevne omskårne med en Omskærelse, som ikke er gjort med Hænder, ved Afførelsen af Kødets Legeme, ved Kristi Omskærelse,
\par 12 idet I bleve begravne med ham i Dåben, i hvilken I også bleve medoprejste ved Troen på Guds Virkekraft, som oprejste ham fra de døde.
\par 13 Også eder, som vare døde i eders Overtrædelser og eders Køds Forhud, eder gjorde han levende tillige med ham, idet han tilgav os alle vore Overtrædelser
\par 14 og udslettede det imod os rettede Gældsbrev med dets Befalinger, hvilket gik os imod, og han har taget det bort ved at nagle det til Korset;
\par 15 efter at have afvæbnet Magterne og Myndighederne, stillede han dem åbenlyst til Skue, da han i ham førte dem i Sejrstog.
\par 16 Lad derfor ingen dømme eder for Mad eller for Drikke eller i Henseende til Højtid eller Nymåne eller Sabbat,
\par 17 hvilket er en Skygge af det, som skulde komme, men Legemet er Kristi.
\par 18 Lad ingen frarøve eder Sejrsprisen, idet han finder Behag i Ydmyghed og Dyrkelse af Englene, idet han indlader sig på, hvad han har set i Syner, forfængeligt opblæst af sit kødelige Sind,
\par 19 og ikke holder fast ved Hovedet, ud fra hvem hele Legemet, idet det hjælpes og sammenknyttes ved sine Bindeled og Bånd, vokser Guds Vækst.
\par 20 Når I med Kristus ere døde fra Verdens Børnelærdom, hvorfor lade I eder da pålægge Befalinger, som om I levede i Verden:
\par 21 "Tag ikke, smag ikke, rør ikke derved!"
\par 22 (hvilket alt er bestemt til at forgå ved at forbruges) efter Menneskenes Bud og Lærdomme?
\par 23 thi alt dette har Ord for Visdom ved selvgjort Dyrkelse og Ydmyghed og Skånselsløshed imod Legemet, ikke ved noget, som er Ære værd, kun til Tilfredsstillelse af Kødet.

\chapter{3}

\par 1 Når I altså, ere blevne oprejste med Kristus, da søger det, som er oventil, hvor Kristus sidder ved Guds højre Hånd.
\par 2 Tragter efter det, som er oventil, ikke efter det, som er på Jorden.
\par 3 Thi I ere døde, og eders Liv er skjult med Kristus i Gud.
\par 4 Når Kristus, vort Liv, åbenbares, da skulle også I åbenbares med ham i Herlighed.
\par 5 Så døder da de jordiske Lemmer, Utugt Urenhed, Brynde, ondt Begær og Havesygen, som jo er Afgudsdyrkelse;
\par 6 for disse Tings Skyld kommer Guds Vrede.
\par 7 I dem vandrede også I fordum, da I levede deri.
\par 8 Men nu skulle også I aflægge det alt sammen, Vrede, Hidsighed, Ondskab, Forhånelse, slem Snak af eders Mund.
\par 9 Lyver ikke for hverandre, da I have afført eder det gamle Menneske med dets Gerninger
\par 10 og iført eder det nye, som fornyes til Erkendelse efter hans Billede, der skabte det;
\par 11 hvor der ikke er Græker og Jøde, Omskærelse og Forhud, Barbar, Skyther, Træl, fri, men Kristus er alt og i alle.
\par 12 Så ifører eder da som Guds udvalgte, hellige og elskede inderlig Barmhjertighed, Godhed, Ydmyghed, Sagtmodighed, Langmodighed,
\par 13 så I bære over med hverandre og tilgive hverandre,dersom nogen har Klagemål imod nogen; ligesom Kristus tilgav eder, således også I!
\par 14 Men over alt dette skulle I iføre eder Kærligheden, hvilket er Fuldkommenhedens Bånd.
\par 15 Og Kristi Fred råde i eders Hjerter, til hvilken I også bleve kaldede i eet Legeme; og vorder taknemmelige!
\par 16 Lad Kristi Ord bo rigeligt iblandt eder, så I med al Visdom lære og påminde hverandre med Salmer, Lovsange og åndelige Viser, idet I synge med Ynde i eders Hjerter for Gud.
\par 17 Og alt, hvad I gøre i Ord eller i Handling, det gører alt i den Herres Jesu Navn, takkende Gud Fader ved ham.
\par 18 I Hustruer! underordner eder under eders Mænd, som det sømmer sig i Herren.
\par 19 I Mænd! elsker eders Hustruer, og værer ikke bitre imod dem!
\par 20 I Børn! adlyder i alle Ting eders Forældre, thi dette er velbehageligt i Herren.
\par 21 I Fædre! opirrer ikke eders Børn, for at de ikke skulle tabe Modet.
\par 22 I Trælle! adlyder i alle Ting eders Herrer efter Kødet, ikke med Øjentjeneste som de, der ville tækkes Mennesker, men i Hjertets Enfold, frygtende Herren.
\par 23 Hvad I end foretage eder, så gører det af Hjertet, som for Herren og ikke for Mennesker,
\par 24 da I vide, at I af Herren skulle få Arven til Vederlag; det er den Herre Kristus, I tjene.
\par 25 Thi den, som gør Uret, skal få igen, hvad Uret han gjorde, og der er ikke Persons Anseelse.

\chapter{4}

\par 1 I Herrer! yder eders Trælle, hvad ret og billigt er, da I vide, at også I have en Herre i Himmelen.
\par 2 Værer vedholdendene i; Bønnen, idet I ere årvågne i den med Taksigelse.
\par 3 idet I tillige bede også for os, at Gud vil oplade os en Ordets Dør til at tale Kristi Hemmelighed, for hvis Skyld jeg også er bunden,
\par 4 for at jeg kan åbenbare den således, som jeg bør tale.
\par 5 Vandrer i Visdom overfor dem, som ere udenfor, så I købe den belejlige Tid.
\par 6 Eders Tale være altid med Ynde, krydret med Salt, så I vide, hvorledes I bør svare enhver især.
\par 7 Hvorledes det går mig, skal Tykikus, den elskede Broder og tro Tjener og Medtjener i Herren, kundgøre eder alt sammen;
\par 8 ham sender jeg til eder, netop for at I skulle lære at kende, hvorledes det står til med os, og for at han skal opmuntre eders Hjerter,
\par 9 tillige med Onesimus, den tro og elskede Broder, som er fra eders By; de skulle fortælle eder, hvorledes alt står til her.
\par 10 Aristarkus, min Medfange, hilser eder, og Markus, Barnabas's Søskendebarn, om hvem I have fået Befalinger - dersom han kommer til eder, da tager imod ham -
\par 11 og Jesus, som kaldes Justus, hvilke af de omskårne ere de eneste Medarbejdere for Guds Rige, som ere blevne mig en Trøst.
\par 12 Epafras hilser eder, han, som er fra eders By, en Kristi Jesu Tjener, som altid strider for eder i sine Bønner, før at I må stå fuldkomne og fuldvisse i al Guds Villie.
\par 13 Thi jeg giver ham det Vidnesbyrd, at han har megen Møje for eder og dem i Laodikea og dem i Hierapolis,
\par 14 Lægen Lukas, den elskede, hilser eder, og Demas.
\par 15 Hilser Brødrene i Laodikea og Nymfas og Menigheden i deres Hus.
\par 16 Og når dette Brev er oplæst hos eder, da sørger for, at det også bliver oplæst i Laodikensernes Menighed, og at I også læse Brevet fra Laodikea.
\par 17 Og siger til Arkippus: Giv Agt på den Tjeneste, som du har modtaget i Herren, at du fuldbyrder den.
\par 18 Hilsenen med min, Paulus's, egen Hånd. Kommer mine Lænker i Hu Nåde være med eder!


\end{document}