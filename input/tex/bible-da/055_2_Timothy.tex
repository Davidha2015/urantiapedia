\begin{document}

\title{Andet Timotheusbrev}


\chapter{1}

\par 1 Paulus; Kristi Jesu Apostel ved Guds Villie, for at bringe Forjættelse om Livet i Kristus Jesus
\par 2 - til Timotheus, sit elskede Barn: Nåde, Barmhjertighed og Fred fra Gud Fader og Kristus Jesus, vor Herre!
\par 3 Jeg takker Gud, hvem jeg fra mine Forfædre af har tjent i en ren Samvittighed, ligesom jeg uafladelig har dig i Erindring i mine Bønner Nat og Dag,
\par 4 da jeg i Mindet om dine Tårer længes efter at se dig, for at jeg må fyldes med Glæde,
\par 5 idet jeg er bleven mindet om den uskrømtede Tro, som er i dig, den, som boede først i din Mormoder Lois og din Moder Eunike, og jeg er vis på, at den også bor i dig.
\par 6 Derfor påminder jeg dig, at du opflammer den Guds Nådegave, som er i dig ved mine Hænders Pålæggelse.
\par 7 Thi Gud har ikke givet os Fejgheds Ånd, men Krafts og Kærligheds og Sindigheds Ånd.
\par 8 Derfor, skam dig ikke ved Vidnesbyrdet om vor Herre eller ved mig, hans Fange, men lid ondt med Evangeliet ved Guds Kraft,
\par 9 han, som frelste os og kaldte os med en hellig Kaldelse, ikke efter vore Gerninger, men efter sit eget Forsæt og Nåden, som blev given os i Kristus Jesus fra evige Tider,
\par 10 men nu er kommen for Dagen ved vor Frelsers Jesu Kristi Åbenbarelse, han, som tilintetgjorde Døden, men bragte Liv og Uforkrænkelighed for Lyset ved Evangeliet,
\par 11 for hvilket jeg er bleven sat til Prædiker og Apostel og Hedningers Lærer,
\par 12 hvorfor jeg også lider dette, men jeg skammer mig ikke derved; thi jeg ved, til hvem jeg har sat min Tro, og jeg er vis på, at han er mægtig til at vogte på den mig betroede Skat til hin Dag.
\par 13 Hav et Forbillede i de sunde Ord, som du har hørt af mig, i Tro og Kærlighed i Kristus Jesus.
\par 14 Vogt på den skønne betroede Skat ved den Helligånd, som bor i os.
\par 15 Du ved dette, at alle de i Asien have vendt sig fra mig, iblandt hvilke ere Fygelus og Hermogenes.
\par 16 Herren vise Onesiforus's Hus Barmhjertighed; thi han har ofte vederkvæget mig og skammede sig ikke ved min Lænke,
\par 17 men da han kom til Rom, søgte han ivrigt efter mig og fandt mig.
\par 18 Herren give ham at finde Barmhjertighed fra Herren på hin Dag! Og hvor megen Tjeneste han har gjort i Efesus, ved du bedst.

\chapter{2}

\par 1 Du derfor, mit Barn! bliv stærk ved Nåden i Kristus Jesus;
\par 2 og hvad du har hørt af mig for mange Vidner, betro det til trofaste Mennesker, som kunne være dygtige også til at lære andre.
\par 3 Vær med til at lide ondt som en god Kristi Jesu Stridsmand.
\par 4 Ingen, som gør Krigstjeneste, indvikler sig i Livets Handeler for at han kan behage den, som tog ham i Sold.
\par 5 Og ligeså, når nogen møder i Væddekamp, bliver han dog ikke bekranset, dersom han ikke kæmper lovmæssigt.
\par 6 Den Bonde, som arbejder, bør først have Del i Frugterne.
\par 7 Mærk, hvad jeg siger; Herren vil jo give dig Indsigt i alle Ting.
\par 8 Kom Jesus Kristus i Hu, oprejst fra de døde, af Davids Sæd, efter mit Evangelium,
\par 9 for hvilket jeg lider ondt lige indtil at være bunden som en Misdæder; men Guds Ord er ikke bundet.
\par 10 Derfor udholder jeg alt for de udvalgtes Skyld, for at også de skulle få Frelsen i Kristus Jesus med evig Herlighed.
\par 11 Den Tale er troværdig; thi dersom vi ere døde med ham, skulle vi også leve med ham;
\par 12 dersom vi holde ud, skulle vi også være Konger med ham; dersom vi fornægte, skal også han fornægte os;
\par 13 dersom vi ere utro, forbliver han dog tro; thi fornægte sig selv kan han ikke.
\par 14 påmind om disse Ting, idet du besværger dem for Herrens Åsyn, at de ikke kives om Ord, hvilket er til ingen Nytte, men til Ødelæggelse for dem, som høre derpå.
\par 15 Gør dig Flid for at fremstille dig selv som prøvet for Gud, som en, Arbejder, der ikke behøver at skamme sig, som rettelig lærer Sandhedens Ord.
\par 16 Men hold dig fra den vanhellige, tomme Snak; thi sådanne ville stedse gå videre i Ugudelighed,
\par 17 og deres Ord vil æde om sig som Kræft. Iblandt dem ere Hymenæus og Filetus,
\par 18 som ere afvegne fra Sandheden, idet de sige, at Opstandelsen er allerede sket, og de forvende Troen hos nogle.
\par 19 Dog, Guds faste Grundvold står og har dette Segl: "Herren kender sine" og: "Hver den, som nævner Herrens Navn, afstå fra Uretfærdighed."
\par 20 Men i et stort Hus er der ikke alene Kar af Guld og Sølv, men også af Træ og Ler, og nogle til Ære, andre til Vanære.
\par 21 Dersom da nogen holder sig ren fra disse, han skal være et Kar til Ære, helliget, Husbonden nyttigt, tilberedt til al god Gerning.
\par 22 Men fly de ungdommelige Begæringer; jag derimod efter Retfærdighed, Troskab, Kærlighed og Fred sammen med dem, som påkalde Herren af et rent Hjerte;
\par 23 og afvis de tåbelige og uforstandige Stridigheder, efterdi du ved, at de avle Kampe,
\par 24 men en Herrens Tjener bør ikke strides, men være mild imod alle, dygtig til at lære, i Stand til at tåle ondt,
\par 25 med Sagtmodighed irettesættende dem, som modsætte sig, om Gud dog engang vilde give dem Omvendelse til Sandheds Erkendelse,
\par 26 og de kunde blive ædru igen fra Djævelens Snare, af hvem de ere fangne til at gøre hans Villie.

\chapter{3}

\par 1 Men vid dette,at i de sidste Dage skulle vanskelige Tider indtræde.
\par 2 Thi Menneskene skulle være egenkærlige, pengegridske, praleriske, hovmodige, spottelystne, ulydige imod Forældre, utaknemmelige, ryggesløse,
\par 3 ukærlige, uforligelige, bagtaleriske, uafholdne, rå, uden Kærlighed til det gode,
\par 4 forræderske, fremfusende, opblæste, Mennesker, som mere elske Vellyst, end de elske Gud,
\par 5 som have Gudfrygtigheds Skin, men have fornægtet dens Kraft. Og fra disse skal du vende dig bort!
\par 6 Thi til dem høre de, som snige sig ind i Husene og fange Kvindfolk, der ere belæssede med Synder og drives af mange Hånde Begæringer
\par 7 og altid lære og aldrig kunne komme til Sandheds Erkendelse.
\par 8 Men ligesom Jannes og Jambres stode Moses imod, således modstå også disse Sandheden: Mennesker, fordærvede i Sindet, forkastelige i Troen.
\par 9 Dog, de skulle ikke få Fremgang ydermere; thi deres Afsind skal blive åbenbart for alle, ligesom også hines blev.
\par 10 Du derimod har efterfulgt mig i Lære, i Vandel, i Forsæt, Tro, Langmodighed, Kærlighed, Udholdenhed,
\par 11 i Forfølgelser, i Lidelser, sådanne, som ere komne over mig i Antiokia, i Ikonium, i Lystra, sådanne Forfølgelser, som jeg har udstået, og Herren har friet mig ud af dem alle.
\par 12 Ja, også alle de, som ville leve gudfrygtigt i Kristus Jesus, skulle forfølges.
\par 13 Men onde Mennesker og Bedragere ville gå frem til det værre; de forføre og forføres.
\par 14 Du derimod, bliv i det, som du har lært, og som du er bleven forvisset om, efterdi du ved, af hvem du har lært det,
\par 15 og efterdi du fra Barn af kender de hellige Skrifter, som kunne gøre dig viis til Frelse ved Troen på Kristus Jesus.
\par 16 Hvert Skrift er indåndet af Gud og nyttig til Belæring, til Irettesættelse, til Forbedring, til Optugtelse i Retfærdighed,
\par 17 for at Guds-Mennesket må vorde fuldkomment, dygtiggjort til al god Gerning.

\chapter{4}

\par 1 Jeg besværger dig for Guds og Kristi Jesu Åsyn, som skal dømme levende og døde, og ved hans Åbenbarelse og hans Rige:
\par 2 Prædike Ordet, vær rede i Tide og i Utide, irettesæt, straf, forman med al Langmodighed og Belæring!
\par 3 Thi den Tid skal komme, da de ikke skulle fordrage den sunde Lære, men efter deres egne Begæringer tage sig selv Lærere i Hobetal, efter hvad der kildrer deres Øren,
\par 4 og de skulle vende Ørene fra Sandheden og vende sig hen til Fablerne.
\par 5 Du derimod, vær ædru i alle Ting, lid ondt, gør en Evangelists Gerning, fuldbyrd din Tjeneste!
\par 6 Thi jeg ofres allerede, og Tiden til mit Opbrud er for Hånden.
\par 7 Jeg har stridt den gode Strid, fuldkommet Løbet og bevaret Troen.
\par 8 I øvrigt henligger Retfærdighedens Krans til mig, hvilken Herren, den retfærdige Dommer, skal give mig på hin Dag, og ikke alene mig, men også alle dem, som have elsket hans Åbenbarelse.
\par 9 Gør dig Flid for at komme snart til mig;
\par 10 thi Demas forlod mig, fordi han fik Kærlighed til den nærværende Verden, og drog til Thessalonika; Kreskens drog til Galatien, Titus til Dalmatien.
\par 11 Lukas er alene hos mig. Tag Markus og bring ham med dig; thi han er mig nyttig til Tjenesten.
\par 12 Men Tykikus har jeg sendt til Efesus.
\par 13 Når du kommer, da bring min Rejsekjortel med dig, som jeg lod blive i Troas hos Karpus, og Bøgerne, især dem på Pergament.
\par 14 Smeden Aleksander har gjort mig meget ondt; Herren vil betale ham efter hans Gerninger.
\par 15 For ham skal også du vogte dig;thi han stod vore Ord hårdt imod.
\par 16 Ved mit første Forsvar kom ingen mig til Hjælp, men alle lode mig i Stikken; (gid det ikke må tilregnes dem! )
\par 17 Men Herren stod hos mig og styrkede mig, for at Ordets Prædiken skulde fuldbyrdes ved mig, og alle Hedningerne høre det; og jeg blev friet fra Løvens Gab.
\par 18 Herren vil fri mig fra al ond Gerning og frelse mig til sit himmelske Rige; ham være Æren i Evighedernes Evigheder! Amen.
\par 19 Hils Priska og Akvila og Onesiforus's Hus!
\par 20 Erastus blev i Korinth, men Trofimmus efterlod jeg syg i Milet.
\par 21 Gør dig Flid for at komme før Vinteren! Eubulus og Pudens og Linus og Klaudia og alle Brødrene hilse dig.
\par 22 Den Herre Jesus være med din Ånd! Nåden være med eder!



\end{document}