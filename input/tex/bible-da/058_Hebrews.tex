\begin{document}

\title{Hebræerbrevet}


\chapter{1}

\par 1 Efter at Gud fordum havde talt mange Gange og på mange Måder, til Fædrene ved Profeterne, så har han ved Slutningen af disse Dage talt til os ved sin Søn,
\par 2 hvem han har sat til Arving af alle Ting, ved hvem han også har skabt Verden;
\par 3 han, som - efterdi han er hans Herligheds Glans og hans Væsens udtrykte Billede og bærer alle Ting med sin Krafts Ord - efter at have gjort Renselse fra Synderne har sat sig ved Majestætens højre Hånd i det høje,
\par 4 idet han er bleven så meget ypperligere end Englene, som han har arvet et herligere Navn fremfor dem.
\par 5 Thi til hvilken af Englene sagde han nogen Sinde: "Du er min Søn, jeg har født dig i Dag"? og fremdeles: "Jeg skal være ham en Fader, og han skal være mig en Søn"?
\par 6 Og når han atter indfører den førstefødte i Verden, hedder det: "Og alle Guds Engle skulle tilbede ham".
\par 7 Og om Englene hedder det: "Han gør sine Engle til Vinde og sine Tjenere til Ildslue";
\par 8 men om Sønnen,: "Din Trone, o Gud! står i al Evighed, og Rettens Kongestav er dit Riges Kongestav.
\par 9 Du elskede Retfærdighed og hadede Lovløshed, derfor har Gud, din Gud, salvet dig med Glædens Olie fremfor dine Medbrødre".
\par 10 Og: "Du, Herre! har i Begyndelsen grundfæstet Jorden, og Himlene ere dine Hænders Gerninger.
\par 11 De skulle forgå, men du bliver; og de skulle til Hobe ældes som et Klædebon,
\par 12 ja, som et Klæde skal du sammenrulle dem, og de skulle omskiftes; men du er den samme, og dine År skulle ikke få Ende".
\par 13 Men til hvilken af Englene sagde han nogen Sinde: "Sæt dig ved min højre Hånd, indtil jeg får lagt dine Fjender som en Skammel for dine Fødder"?
\par 14 Ere de ikke alle tjenende Ånder, som udsendes til Hjælp for deres Skyld, der skulle arve Frelse?

\chapter{2}

\par 1 Derfor bør vi des mere agte på det, vi have hørt, for at vi ikke skulle rives bort.
\par 2 Thi når det Ord, som taltes ved Engle, blev urokket, og hver Overtrædelse og Ulydighed f1k velforskyldt Løn,
\par 3 hvorledes skulle da vi undfly, når vi ikke bryde os om så stor en Frelse, som jo efter først at være bleven forkyndt ved Herren, er bleven stadfæstet for os af dem, som havde hørt ham,
\par 4 idet Gud vidnede med både ved Tegn og Undere og mange Hånde kraftige Gerninger og ved Meddelelse af den Helligånd efter sin Villie.
\par 5 Thi det var ikke Engle, han underlagde den kommende Verden; om hvilken vi tale.
\par 6 Men en har vidnet et Sted og sagt: "Hvad er et Menneske, at du kommer ham i Hu? eller en Menneskesøn, at du ser til ham?
\par 7 Du gjorde ham en kort Tid ringere end Engle; med Herlighed og Ære kronede du ham;
\par 8 alle Ting lagde du under hans Fødder". - Idet han nemlig underlagde ham alle Ting, undtog han intet fra at være ham underlagt. Nu se vi imidlertid endnu ikke alle Ting underlagte ham;
\par 9 men ham, som en kort Tid var bleven gjort ringere end Engle, Jesus, se vi på Grund af Dødens Lidelse kronet med Herlighed og Ære, for at han ved Guds Nåde må have smagt Døden for alle.
\par 10 Thi det sømmede sig ham, for hvis Skyld alle Ting ere, og ved hvem alle Ting ere, når han førte mange Sønner til Herlighed, da at fuldkomme deres Frelses Ophavsmand igennem Lidelser.
\par 11 Thi både den, som helliger, og de, som helliges, ere alle af een; hvorfor han ikke skammer sig ved at kalde dem Brødre,
\par 12 når han siger: "Jeg vil forkynde dit Navn for mine Brødre, midt i en Menighed vil jeg lovsynge dig."
\par 13 Og fremdeles: "Jeg vil forlade mig på ham." Og fremdeles: "Se, her er jeg og de Børn, som Gud har givet mig."
\par 14 Efterdi da Børnene ere delagtige i Blod og Kød, blev også han i lige Måde delagtig deri, for at han ved Døden skulde gøre den magtesløs, som har Dødens Vælde, det er Djævelen,
\par 15 og befri alle dem, som på Grund af Dødsfrygt vare under Trældom al deres Livs Tid.
\par 16 Thi det er jo dog ikke Engle, han tager sig af, men Abrahams Sæd tager han sig af.
\par 17 Derfor måtte han blive sine Brødre lig i alle. Ting, for at han kunde blive en barmhjertig og trofast Ypperstepræst over for Gud til at sone Folkets Synder.
\par 18 Thi idet han har lidt, kan han som den, der selv er fristet, komme dem til Hjælp, som fristes.

\chapter{3}

\par 1 Derfor, hellige Brødre, delagtige i en himmelsk Kaldelse! ser hen til vor Bekendelses Udsending og Ypperstepræst, Jesus,
\par 2 der var tro imod den, som beskikkede ham, ligesom også Moses var det i hele hans Hus.
\par 3 Thi han er kendt værdig til større Herlighed end Moses, i samme Mål som den,der har indrettet et Hus, har større Ære end Huset selv.
\par 4 Thi hvert Hus indrettes af nogen; men den, som har indrettet alt, er Gud.
\par 5 Og Moses var vel tro i hele hans Hus, som en Tjener, til Vidnesbyrd om, hvad der skulde tales;
\par 6 men Kristus er det som en Søn over hans Hus; og hans Hus ere vi, såfremt vi fastholde Håbets Frimodighed og Ros urokket indtil Enden.
\par 7 Derfor, som den Helligånd siger: "I Dag, når I høre hans Røst,
\par 8 da forhærder ikke eders Hjerter, som det skete i Forbitrelsen, på Fristelsens Dag i Ørkenen,
\par 9 hvor eders Fædre fristede mig ved at sætte mig på Prøve, og de så dog mine Gerninger i fyrretyve År.
\par 10 Derfor harmedes jeg på denne Slægt og sagde: De fare altid vild i Hjertet; men de kendte ikke mine Veje,
\par 11 så jeg svor i min Vrede: Sandelig, de skulle ikke gå ind til min Hvile" -
\par 12 så ser til, Brødre! at der ikke nogen Sinde i nogen af eder skal findes et ondt, vantro Hjerte, så at han falder fra den levende Gud.
\par 13 Men formaner hverandre hver Dag, så længe det hedder "i Dag", for at ikke nogen af eder skal forhærdes ved Syndens Bedrag.
\par 14 Thi vi ere blevne delagtige i Kristus, såfremt vi fastholde vor første Fortrøstning urokket indtil Enden.
\par 15 Når der sigs: "I Dag, - når I høre hans Røst, da forhærder ikke eders Hjerter som i Forbitrelsen":
\par 16 hvem vare da vel de, som hørte og dog voldte Forbitrelse? Mon ikke alle, som gik ud af Ægypten ved Moses?
\par 17 Men på hvem harmedes han i fyrretyve År? Mon ikke på dem, som syndede, hvis døde Kroppe faldt i Ørkenen?
\par 18 Og over for hvem tilsvor han, at de ikke skulde gå ind til hans Hvile, uden dem, som vare blevne genstridige?
\par 19 Og vi se, at de ikke kunde gå ind på Grund af Vantro.

\chapter{4}

\par 1 Lader os derfor, da der endnu står en Forjættelse tilbage om at indgå til hans Hvile, vogte os for, at nogen af eder skal mene, at han er kommen for silde.
\par 2 Thi også os er der forkyndt godt Budskab ligesom hine; men Ordet, som de hørte, hjalp ikke dem, fordi det ikke var forenet med Troen hos dem, som hørte det.
\par 3 Thi vi gå ind til Hvilen, vi, som ere komne til Troen, efter hvad han har sagt: "Så svor jeg i min Vrede: Sandelig, de skulle ikke gå ind til min Hvile", omendskønt Gerningerne vare fuldbragte fra Verdens Grundlæggelse.
\par 4 Thi han har et Sted sagt om den syvende Dag således: "Og Gud hvilede på den syvende Dag fra alle sine Gerninger."
\par 5 Og fremdeles på dette Sted: "Sandelig, de skulle ikke gå ind til min Hvile."
\par 6 Efterdi der altså står tilbage, at nogle skulle gå ind til den, og de, hvem der først blev forkyndt godt Budskab, ikke gik ind for deres Genstridigheds Skyld:
\par 7 så bestemmer han atter en Dag: "I Dag", siger han ved David så lang Tid efter, (som ovenfor sagt): "I Dag, når I høre hans Røst, da forhærder ikke eders Hjerter!"
\par 8 Thi dersom Josva havde skaffet dem Hvile, da vilde han ikke tale om en anden Dag siden efter.
\par 9 Altså er der en Sabbatshvile tilbage for Guds Folk.
\par 10 Thi den, som er gået ind til hans Hvile, også han har fået Hvile fra sine Gerninger, ligesom Gud fra sine.
\par 11 Lader os derfor gøre os Flid for at gå ind til hin Hvile, for at ikke nogen skal falde ved den samme Genstridighed, som hine gave Eksempel på.
\par 12 Thi Guds Ord er levende og kraftigt og skarpere end noget tveægget Sværd og trænger igennem, indtil det deler Sjæl og Ånd, Ledemod såvel som Marv, og dømmer over Hjertets Tanker og Råd.
\par 13 Og ingen Skabning er usynlig for hans Åsyn; men alle Ting ere nøgne og udspændte for hans Øjne, hvem vi stå til Regnskab.
\par 14 Efterdi vi altså have en stor Ypperstepræst, som er gået igennem Himlene, Jesus, Guds Søn, da lader os holde fast ved Bekendelsen!
\par 15 Thi vi have ikke en Ypperstepræst, som ej kan have Medlidenhed med vore Skrøbeligheder, men en sådan, som er fristet i alle Ting i Lighed med os, dog uden Synd.
\par 16 Derfor lader os træde frem med Frimodighed for Nådens Trone, for at vi kunne få Barmhjertighed og finde Nåde til betimelig Hjælp.

\chapter{5}

\par 1 Thi hver Ypperstepræst tages iblandt Mennesker og indsættes for Mennesker til Tjenesten for Gud, for at han skal frembære både Gaver og Slagtofre for Synder,
\par 2 som en, der kan bære over med de vankundige og vildfarende, eftersom han også selv er stedt i Skrøbelighed
\par 3 og for dens Skyld må frembære Syndoffer, som for Folket således også for sig selv
\par 4 Og ingen tager sig selv den Ære, men han kaldes af Gud, ligesom jo også Aron.
\par 5 Således har ej heller Kristus tillagt sig selv den Ære at blive Ypperstepræst, men den, som sagde til ham: "Du er min Søn, jeg har født dig i Dag,"
\par 6 som han jo også siger et andet Sted: "Du er Præst til evig Tid, efter Melkisedeks Vis,"
\par 7 han, som i sit Køds Dage med stærkt Råb og Tårer frembar Bønner og ydmyge Begæringer til den, der kunde frelse ham fra Døden, og blev bønhørt i sin Angst,
\par 8 og således, endskønt han var Søn, lærte Lydighed af det, han led,
\par 9 og efter at være fuldkommet blev Årsag til evig Frelse for alle dem, som lyde ham,
\par 10 idet han af Gud blev kaldt Ypperstepræst efter Melkisedeks Vis.
\par 11 Herom have vi meget at sige, og det er vanskeligt at forklare, efterdi I ere blevne sløve til at høre.
\par 12 Thi skønt I efter Tiden endog burde være Lærere, trænge I atter til, at man skal lære eder Begyndelsesgrundene i Guds Ord, og I ere blevne sådanne, som trænge til Mælk og ikke til fast Føde.
\par 13 Thi hver, som får Mælk, er ukyndig i den rette Tale, thi han er spæd;
\par 14 men for de fuldkomne er den faste Føde, for dem, som på Grund af deres Erfaring have Sanserne øvede til at skelne mellem godt og ondt.

\chapter{6}

\par 1 Lader os derfor forbigå Begyndelsesordet om Kristus og skride frem til Fuldkommenhed uden atter at lægge Grundvold med Omvendelse fra døde Gerninger og med Tro på Gud,
\par 2 med Lære om Døbelser og Håndspålæggelse og dødes Opstandelse og evig Dom.
\par 3 Ja, dette ville vi gøre, såfremt Gud tilsteder det.
\par 4 Thi dem, som een Gang ere blevne oplyste og have smagt den himmelske Gave og ere blevne delagtige i den Helligånd
\par 5 og have smagt Guds gode Ord og den kommende Verdens Kræfter, og som ere faldne fra, - dem er det umuligt atter at forny til Omvendelse
\par 6 da de igen korsfæste sig Guds Søn og stille ham til Spot.
\par 7 Thi Jorden, som drikker den ofte derpå faldende Regn og frembringer Vækster, tjenlige for dem, for hvis Skyld den også dyrkes, får Velsignelse fra Gud;
\par 8 men når den bærer Torne og Tidsler, er den ubrugbar og Forbandelse nær; Enden med den er at brændes.
\par 9 Dog, i Henseende til eder, I elskede! ere vi overbeviste om det bedre og det, som bringer Frelse, selv om vi tale således.
\par 10 Thi Gud er ikke uretfærdig, så at han skulde glemme eders Gerning og den Kærlighed, som I have udvist imod hans Navn, idet I have tjent og tjene de hellige.
\par 11 Men vi ønske, at enhver af eder må udvise den samme Iver efter den fulde Vished i Håbet indtil Enden,
\par 12 for at I ikke skulle blive sløve, men efterfølge dem, som ved Tro og Tålmodighed arve Forjættelserne.
\par 13 Thi da Gud gav Abraham Forjættelsen, svor han ved sig selv, fordi han ingen større havde at sværge ved, og sagde:
\par 14 "Sandelig, jeg vil rigeligt velsigne dig og rigeligt mangfoldiggøre dig."
\par 15 Og således opnåede han Forjættelsen ved at vente tålmodigt.
\par 16 Mennesker sværge jo ved en større, og Eden er dem en Ende på al Modsigelse til Stadfæstelse.
\par 17 Derfor, da Gud ydermere vilde vise Forjættelsens Arvinger sit Råds Uforanderlighed, føjede han en Ed dertil,
\par 18 for at vi ved to uforanderlige Ting, i hvilke det var umuligt, at Gud kunde lyve, skulde have en kraftig Opmuntring, vi, som ere flyede hen for at holde fast ved det Håb, som ligger foran os,
\par 19 hvilket vi have som et Sjælens Anker, der er sikkert og fast og går ind inden for Forhænget,
\par 20 hvor Jesus som Forløber gik ind for os, idet han efter Melkisedeks Vis blev Ypperstepræst til evig Tid.

\chapter{7}

\par 1 Thi denne Melkisedek, Konge i Salem, den højeste Guds Præst, som gik Abraham i Møde, da han vendte tilbage fra Kongernes Nederlag, og velsignede ham,
\par 2 hvem også Abraham gav Tiende af alt, og som, når hans Navn udlægges, først er Retfærdigheds Konge, dernæst også Salems Konge, det er: Freds Konge,
\par 3 uden Fader, uden Moder, uden Slægtregister, uden Dages Begyndelse og uden Livs Ende, men gjort lig med Guds Søn, - han forbliver Præst for bestandig.
\par 4 Ser dog, hvor stor denne er, hvem endog Patriarken Abraham gav Tiende af Byttet.
\par 5 Og hine, som, idet de høre til Levi Sønner, få Præstedømmet, have et Bud om at tage Tiende efter Loven af Folket, det er af deres Brødre, endskønt disse ere udgåede af Abrahams Lænd;
\par 6 men han, som ikke regner sin Slægt fra dem, har taget Tiende af Abraham og har velsignet den, som havde Forjættelserne.
\par 7 Men uden al Modsigelse er det den ringere, som velsignes af den ypperligere.
\par 8 Og her er det dødelige Mennesker, som tager Tiende; men der er det en, om hvem der vidnes, at han lever.
\par 9 Ja, så at sige, har endog Levi, som tager Tiende, igennem Abraham givet Tiende;
\par 10 thi han var endnu i Faderens Lænd, da Melkisedek gik denne i Møde.
\par 11 Hvis der altså var Fuldkommelse at få ved det levitiske Præstedømme (thi på Grundlag af dette har jo Folket fået Loven), hvilken Trang var der da yderligere til, at en anden Slags Præst skulde opstå efter Melkisedeks Vis og ikke nævnes efter Arons Vis?
\par 12 Når nemlig Præstedømmet omskiftes, sker der med Nødvendighed også en Omskiftelse af Loven.
\par 13 Thi han, om hvem dette siges, har hørt til en anden Stamme, af hvilken ingen har taget Vare på Alteret.
\par 14 thi det er vitterligt, at af Juda er vor Herre oprunden, og for den Stammes Vedkommende har Moses intet talt om Præster.
\par 15 Og det bliver end ydermere klart, når der i Lighed med Melkisedek opstår en anden Slags Præst,
\par 16 som ikke er bleven det efter et kødeligt Buds Lov, men efter et uopløseligt Livs Kraft.
\par 17 Thi han får det Vidnesbyrd: "Du er Præst til evig Tid efter Melkisedeks Vis."
\par 18 Thi vel sker der Ophævelse af et forudgående Bud, fordi det var svagt og unyttigt
\par 19 (thi Loven har ikke fuldkommet noget); men der sker Indførelse af et bedre Håb, ved hvilket vi nærme os til Gud.
\par 20 Og så vist som det ikke er sket uden Ed,
\par 21 (thi hine ere blevne Præster uden Ed, men denne med Ed, ved den, som siger til ham: "Herren svor, og han skal ikke angre det: Du er Præst til evig Tid"):
\par 22 så vist er Jesus bleven Borgen for en bedre Pagt.
\par 23 Og hine ere blevne Præster, flere efter hinanden, fordi de ved Døden hindredes i at vedblive;
\par 24 men denne har et uforgængeligt Præstedømme, fordi han bliver til evig Tid,
\par 25 hvorfor han også kan fuldkomment frelse dem, som komme til Gud ved ham, efterdi han lever altid til at gå i Forbøn for dem.
\par 26 Thi en sådan Ypperstepræst var det også, som sømmede sig for os, en from, uskyldig, ubesmittet, adskilt fra Syndere og ophøjet over Himlene;
\par 27 en, som ikke hver Dag har nødig, som Ypperstepræsterne, at frembære Ofre først for sine egne Synder, derefter for Folkets; thi dette gjorde han een Gang for alle, da han ofrede sig selv.
\par 28 Thi Loven indsætter til Ypperstepræster Mennesker, som have Skrøbelighed; men Edens Ord, som kom senere end Loven, indsætter en Søn, som er fuldkommet til evig Tid:

\chapter{8}

\par 1 Men Hovedpunktet ved det, hvorom her tales, er dette: vi have en sådan Ypperstepræst, der har taget Sæde på højre Side af Majestætens Trone i Himlene
\par 2 som Tjener ved Helligdommen og det sande Tabernakel, hvilket Herren har oprejst, og ikke et Menneske.
\par 3 Thi hver Ypperstepræst indsættes til at frembære Gaver og Slagtofre; derfor er det nødvendigt, at også denne må have noget at frembære.
\par 4 Dersom han nu var på Jorden, da var han ikke engang Præst, efterdi der her er dem, som frembære Gaverne efter Loven;
\par 5 hvilke jo tjene ved en Afbildning og Skygge af det himmelske, således som det blev Moses betydet af Gud, da han skulde indrette Tabernaklet: "Se til, sagde han, at du gør alting efter det Forbillede, der blev vist dig på Bjerget."
\par 6 Men nu har han fået en så meget ypperligere Tjeneste, som han også er Mellemmand for en bedre Pagt, der jo er grundet på bedre Forjættelser.
\par 7 Thi dersom hin første var udadlelig, da vilde der ikke blive søgt Sted for en anden.
\par 8 Thi dadlende siger han til dem: "Se, der kommer Dage, siger Herren, da jeg vil slutte en ny Pagt med Israels Hus og med Judas Hus;
\par 9 ikke som den Pagt, jeg gjorde med deres Fædre på den Dag, da jeg tog dem ved Hånden for at føre dem ud af Ægyptens Land; thi de bleve ikke i min Pagt, og jeg brød mig ikke om dem, siger Herren.
\par 10 Thi dette er den Pagt, som jeg vil oprette med Israels Hus efter de Dage, siger Herren: Jeg vil give mine Love i deres Sind, og jeg vil indskrive dem i deres Hjerte, og jeg vil være deres Gud, og de skulle være mit Folk.
\par 11 Og de skulle ikke lære hver sin Medborger og hver sin Broder og sige: Kend Herren; thi de skulle alle kende mig, fra den mindste indtil den største iblandt dem.
\par 12 Thi jeg vil være nådig imod deres Uretfærdigheder og ikke mere ihukomme deres Synder."
\par 13 Når han siger: "en ny", har han erklæret den første for gammel; men det, som bliver gammelt og ældes, er nu ved at forsvinde.

\chapter{9}

\par 1 Vel havde også den første Pagt Forskrifter for Gudstjenesten og en jordisk Helligdom.
\par 2 Thi der var indrettet et Telt, det forreste, hvori Lysestagen var og Bordet og Skuebrødene, det, som jo kaldes det Hellige.
\par 3 Men bag det andet Forhæng var et Telt, det, som kaldes det Allerhelligste,
\par 4 som havde et gyldent Røgelsealter og Pagtens Ark, overalt beklædt med Guld, i hvilken der var en Guldkrukke med Mannaen, og Arons Stav, som havde blomstret, og Pagtens Tavler,
\par 5 men oven over den var Herlighedens Keruber, som overskyggede Nådestolen, hvorom der nu ikke skal tales enkeltvis.
\par 6 Idet nu dette er således indrettet, gå Præsterne til Stadighed ind i det forreste Telt, når de forrette Tjenesten;
\par 7 men i det andet går alene Ypperstepræsten ind een Gang om Året, ikke uden Blod,hvilket han ofrer for sig selv og Folkets Forseelser,
\par 8 hvorved den Helligånd giver til Kende, at Vejen til Helligdommen endnu ikke er bleven åbenbar, så længe det førreste Telt endnu står,
\par 9 hvilket jo er et Sindbillede indtil den nærværende Tid, og stemmende hermed frembæres der både Gaver og Ofre, som ikke i Henseende til Samvittigheden kunne fuldkomme den, der forretter sin Gudsdyrkelse,
\par 10 men som kun, ved Siden af Mad og Drikke og forskellige Tvættelser, ere kødelige Forskrifter, pålagte indtil den rette Ordnings Tid.
\par 11 Men da Kristus kom som Ypperstepræst for de kommende Goder, gik han igennem det større og fuldkomnere Telt, som ikke er gjort med Hænder, det er: som ikke er af denne Skabning,
\par 12 og gik ikke heller med Blod af Bukke eller Kalve, men med sit eget Blod een Gang for alle ind i Helligdommen og vandt en evig Forløsning.
\par 13 Thi dersom Blodet af Bukke og Tyre og Aske af en Kvie ved at stænkes på de besmittede helliger til Kødets Renhed:
\par 14 hvor meget mere skal da Kristi Blod, hans, som ved en evig Ånd frembar sig selv lydeløs for Gud, rense eders Samvittighed fra døde Gerninger til at tjene den levende Gud?
\par 15 Og derfor er han Mellemmand for en ny Pagt, for at de kaldede, da der har fundet Død Sted til Genløsning fra Overtrædelserne under den første Pagt, må få den evige Arvs Forjættelse.
\par 16 Thi hvor der er en Arvepagt, der er det nødvendigt, at hans Død, som har oprettet Pagten, skal godtgøres.
\par 17 Thi en Arvepagt er urokkelig efter døde, da den ingen Sinde træder i Kraft, medens den, som har oprettet den, lever.
\par 18 Derfor er heller ikke den første bleven indviet uden Blod
\par 19 Thi da hvert Bud efter Loven var forkyndt af Moses for hele Folket, tog han Kalve- og Bukkeblod med Vand og skarlagenrød Uld og Isop og bestænkede både Bogen selv og hele Folket, idet han sagde:
\par 20 "Dette er den Pagts Blod, hvilken Gud har pålagt eder."
\par 21 Og Tabernaklet og alle Tjenestens Redskaber bestænkede han ligeledes med Blodet.
\par 22 Og næsten alt bliver efter Loven renset med Blod, og uden Blods Udgydelse sker der ikke Forladelse.
\par 23 Altså var det en Nødvendighed, at Afbildningerne af de himmelske Ting skulde renses herved, men selve de himmelske Ting ved bedre Ofre end disse.
\par 24 Thi Kristus gik ikke ind i en Helligdom, som var gjort med Hænder og kun var et Billede af den sande, men ind i selve Himmelen for nu at træde frem for Guds Ansigt til Bedste for os;
\par 25 ikke heller for at han skulde ofre sig selv mange Gange, ligesom Ypperstepræsten hvert År går ind i Helligdommen med fremmed Blod;
\par 26 ellers havde han måttet lide mange Gange fra Verdens Grundlæggelse; men nu er han een Gang for alle ved Tidernes Fuldendelse åbenbaret for at bortskaffe Synden ved sit Offer.
\par 27 Og ligesom det er Menneskene beskikket at dø een Gang og derefter Dom,
\par 28 således skal også Kristus, efter at være bleven een Gang ofret for at bære manges Synder, anden Gang, uden Synd, vise sig for dem, som foruente ham til Frelse.

\chapter{10}

\par 1 Thi da Loven kun har en Skygge af de kommende Goder og ikke Tingenes Skikkelse selv, kan den aldrig ved de samme årlige Ofre, som de bestandig frembære, fuldkomme dem, som træde frem dermed.
\par 2 Vilde man ikke ellers have ophørt at frembære dem, fordi de ofrende ikke mere havde nogen Bevidsthed om Synder, når de een Gang vare rensede?
\par 3 Men ved Ofrene sker År for År Ihukommelse af Synder.
\par 4 Thi det er umuligt, at Blod af Tyre og Bukke kan borttage Synder.
\par 5 Derfor siger han, idet han indtræder i Verden: "Slagtoffer og Madoffer havde du ikke Lyst til; men et Legeme beredte du mig;
\par 6 Brændofre og Syndofre havde du ikke Behag i.
\par 7 Da sagde jeg: Se, jeg er kommen (i Bogrullen er der skrevet om mig) for at gøre, Gud! din Villie."
\par 8 Medens han først siger: "Slagtofre og Madofre og Brændofre og Syndofre havde du ikke Lyst til og ej heller Behag i" (og disse frembæres dog efter Loven),
\par 9 så har han derefter sagt: "Se, jeg er kommen for at gøre din Villie." Han ophæver det første for at fastsætte det andet.
\par 10 Og ved denne Villie ere vi helligede ved Ofringen af Jesu Kristi Legeme een Gang for alle.
\par 11 Og hver Præst står daglig og tjener og ofrer mange Gange de samme Ofre, som dog aldrig kunne borttage Synder.
\par 12 Men denne har efter at have ofret eet Offer for Synderne sat sig for bestandig ved Guds højre Hånd,
\par 13 idet han for øvrigt venter på, at hans Fjender skulle lægges som en Skammel for hans Fødder.
\par 14 Thi med et eneste Offer har han for bestandig fuldkommet dem, som helliges.
\par 15 Men også den Helligånd giver os Vidnesbyrd; thi efter at have sagt:
\par 16 "Dette er den Pagt, som jeg vil oprette med dem efter de Dage," siger Herren: "Jeg vil give mine Love i deres Hjerter, og jeg vil indskrive dem i deres Sind,
\par 17 og deres Synder og deres Overtrædelser vil jeg ikke mere ihukomme."
\par 18 Men hvor der er Forladelse for disse, er der ikke mere Offer for Synd.
\par 19 Efterdi vi da, Brødre! have Frimodighed til den Indgang i Helligdommen ved Jesu Blod,
\par 20 som han indviede os som en ny og levende Vej igennem Forhænget, det er hans Kød,
\par 21 og efterdi vi have en stor Præst over Guds Hus:
\par 22 så lader os træde frem med et sandt Hjerte, i Troens fulde Forvisning, med Hjerterne ved Bestænkelsen rensede fra en ond Samvittighed, og Legemet tvættet med rent Vand;
\par 23 lader os fastholde Håbets Bekendelse urokket; thi trofast er han, som gav Forjættelsen;
\par 24 og lader os give Agt på hverandre, så vi opflamme hverandre til Kærlighed og gode Gerninger
\par 25 og ikke forlade vor egen Forsamling, som nogle have for Skik, men formane hverandre, og det så meget mere, som I se, at Dagen nærmer sig.
\par 26 Thi Synde vi med Villie, efter at have modtaget Sandhedens Erkendelse, er der intet Offer mere tilbage for Synder,
\par 27 men en frygtelig Forventelse at Dom og en brændende Nidkærhed, som skal fortære de genstridige.
\par 28 Når en har brudt med Mose Lov, dør han uden Barmhjertighed på to eller tre Vidners Udsagn;
\par 29 hvor meget værre Straf mene I da, at den skal agtes værd, som træder Guds Søn under Fod og agter Pagtens Blod, hvormed han blev helliget, for urent og forhåner Nådens Ånd?
\par 30 Thi vi kende den, som har sagt: "Mig hører Hævnen til, jeg vil betale, siger Herren;" og fremdeles: "Herren skal dømme sit Folk."
\par 31 Det er frygteligt at falde i den levende Guds Hænder.
\par 32 Men kommer de forrige Dage i Hu, i hvilke I, efter at I vare blevne oplyste, udholdt megen Kamp i Lidelser,
\par 33 idet I dels selv ved Forhånelser og Trængsler bleve et Skuespil, dels gjorde fælles Sag med dem, som fristede sådanne Kår.
\par 34 Thi både havde I Medlidenhed med de fangne, og I fandt eder med Glæde i, at man røvede, hvad I ejede, vidende, at I selv have en bedre og blivende Ejendom.
\par 35 Kaster altså ikke eders Frimodighed bort, hvilken jo har stor Belønning;
\par 36 thi I have Udholdenhed nødig, for at I, når I have gjort Guds Villie, kunne opnå Forjættelsen.
\par 37 Thi "der er endnu kun en såre liden Stund, så kommer han, der skal komme, og han vil ikke tøve.
\par 38 Men min retfærdige skal leve af Tro; og dersom han unddrager sig, har min Sjæl ikke Behag i ham."
\par 39 Men vi ere ikke af dem, som unddrage sig, til Fortabelse, men af dem, som tro, til Sjælens Frelse,

\chapter{11}

\par 1 Men Tro er en Fortrøstning til det, som håbes, en Overbevisning om Ting, som ikke ses.
\par 2 Ved den fik jo de gamle godt Vidnesbyrd.
\par 3 Ved Tro fatte vi, at Verden er bleven skabt ved Guds Ord, så det ikke er af synlige Ting, at det, som ses, er blevet til.
\par 4 Ved Tro ofrede Abel Gud et bedre Offer end Kain, og ved den fik han det Vidnesbyrd, at han var retfærdig, idet Gud bevidnede sit Velbehag i hans Gaver; og ved den taler han endnu efter sin Død.
\par 5 Ved Tro blev Enok borttagen, for at han ikke skulde se Døden, og han blev ikke funden, efterdi Gud havde taget ham bort; thi før Borttagelsen har han fået det Vidnesbyrd, at han har behaget Gud.
\par 6 Men uden Tro er det umuligt at behage ham; thi den, som kommer frem for Gud, bør tro, at han er til, og at han bliver deres Belønner, som søge ham.
\par 7 Ved Tro var det, at Noa, advaret af Gud om det, som endnu ikke sås, i Gudsfrygt indrettede en Ark til Frelse for sit Hus; ved den domfældte han Verden og blev Arving til Retfærdigheden ifølge Tro.
\par 8 Ved Tro adlød Abraham, da han blev kaldet, så han gik ud til et Sted, som han skulde tage til Arv; og han gik ud, skønt han ikke vidste, hvor han kom hen.
\par 9 Ved Tro blev han Udlænding i Forjættelsens Land som i et fremmed og boede i Telte med Isak og Jakob, som vare Medarvinger til samme Forjættelse;
\par 10 thi han forventede den Stad, som har fast Grundvold, hvis Bygmester og Grundlægger er Gud.
\par 11 Ved Tro fik endog Sara selv Kraft til at undfange endog ud over sin Alders Tid; thi hun holdt ham for trofast, som havde forjættet det.
\par 12 Derfor avledes der også af en, og det en udlevet, som Himmelens Stjerner i Mangfoldighed og som Sandet ved Havets Bred, det, som ikke kan tælles.
\par 13 I Tro døde alle disse uden at have opnået Forjættelserne; men de så dem langt borte og hilsede dem og bekendte, at de vare fremmede og Udlændinge på Jorden.
\par 14 De, som sige sådant, give jo klarlig til Kende, at de søge et Fædreland.
\par 15 Og dersom de havde haft det, hvorfra de vare udgåede, i Tanker, havde de vel haft Tid til at vende tilbage;
\par 16 men nu hige de efter et bedre, det er et himmelsk; derfor skammer Gud sig ikke ved dem, ved at kaldes deres Gud; thi han har betedt dem en, Stad.
\par 17 Ved Tro har Abraham ofret Isak, da han blev prøvet, ja, den.
\par 18 til hvem der var sagt: "I Isak skal en Sæd få Navn efter dig; "
\par 19 thi han betænkte, at Gud var mægtig endog til at oprejse fra de døde, hvorfra han jo også lignelsesvis fik ham tilbage.
\par 20 Ved Tro udtalte Isak Velsignelse over Jakob og Esau angående kommende Ting.
\par 21 Ved Tro velsignede Jakob døende hver af Josefs Sønner og tilbad, lænende sig over sin Stav.
\par 22 Ved Tro talte Josef på sit yderste om Israels Børns Udgang og gav Befaling om sine Ben.
\par 23 Ved Tro blev Moses, da han var født, skjult i tre Måneder af sine Forældre, fordi de så, at Barnet var dejligt, og de frygtede ikke for Kongens Befaling.
\par 24 Ved Tro nægtede Moses, da han var bleven stor, at kaldes Søn af Faraos Datter
\par 25 og valgte hellere at lide ondt med Guds Folk end at have en kortvarig Nydelse af Synd,
\par 26 idet han agtede Kristi Forsmædelse for større Rigdom end Ægyptens Skatte; thi han så hen til Belønningen.
\par 27 Ved Tro forlod han Ægypten uden at frygte for Kongens Vrede; thi som om han så den usynlige, holdt han ud.
\par 28 Ved Tro har han indstiftet Påsken og Påstrygelsen af Blodet, for at den, som ødelagde de førstefødte, ikke skulde røre dem.
\par 29 Ved Tro gik de igennem det røde Hav som over tørt Land, medens Ægypterne druknede under Forsøget derpå.
\par 30 Ved Tro faldt Jerikos Mure, efter at de vare omgåede i syv Dage.
\par 31 Ved Tro undgik Skøgen Rahab at omkomme med de genstridige; thi hun modtog Spejderne med Fred.
\par 32 Dog, hvorfor skal jeg tale mere? Tiden vil jo fattes mig, hvis jeg skal fortælle om Gideon, Barak, Samson, Jefta, David og Samuel og Profeterne,
\par 33 som ved Tro overvandt Riger, øvede Retfærdighed, opnåede Forjættelser, stoppede Løvers Mund,
\par 34 slukkede Ilds Kraft, undslap Sværds Od, bleve stærke efter Svaghed, bleve vældige i Krig, bragte fremmedes Hære til at vige.
\par 35 Kvinder fik deres døde igen ved Opstandelse. Andre bleve lagte på Pinebænk og toge ikke imod Befrielse, for at de måtte opnå en bedre Opstandelse.
\par 36 Andre måtte friste Forhånelser og Hudstrygelser, tilmed Lænker og Fængsel;
\par 37 de bleve stenede, gennemsavede, fristede, dræbte med Sværd, gik omkring i Fåre- og Gedeskind, lidende Mangel, betrængte, mishandlede
\par 38 (dem var Verden ikke værd), omvankende i Ørkener og på Bjerge og i Huler og Jordens Kløfter.
\par 39 Og alle disse, skønt de havde Vidnesbyrd for deres Tro, opnåede ikke Forjættelsen,
\par 40 efterdi Gud forud havde udset noget bedre for os, for at de ikke skulde fuldkommes uden os.

\chapter{12}

\par 1 Derfor lader også os, efterdi vi have så stor en Sky af Vidner omkring os, aflægge enhver Byrde og Synden, som lettelig hilder os, og med Udholdenhed gennemløbe den foran os liggende Bane,
\par 2 idet vi se hen til Troens Begynder og Fuldender, Jesus, som for den foran ham liggende Glædes Skyld udholdt et Kors, idet han ringeagtede Skændselen, og som har taget Sæde på højre Side af Guds Trone.
\par 3 Ja, tænker på ham, som har udholdt en sådan Modsigelse imod sig af Syndere, for at I ikke skulle blive trætte og forsagte i eders Sjæle,
\par 4 Endnu have I ikke stået imod indtil Blodet i eders Kamp imod Synden,
\par 5 og I have glemt Formaningen, der jo dog taler til eder som til Sønner: "Min Søn! agt ikke Herrens Tugtelse ringe, vær heller ikke forsagt, når du revses af ham;
\par 6 thi hvem Herren elsker, den tugter han, og han slår hårdelig hver Søn, som han tager sig af."
\par 7 Holder ud og lader eder tugte; Gud handler med eder som med Sønner; thi hvem er den Søn, som Faderen ikke tugter?
\par 8 Men dersom I ere uden Tugtelse, hvori alle have fået Del, da ere I jo uægte og ikke Sønner.
\par 9 Fremdeles, vore kødelige Fædre havde vi til Optugtere, og vi følte Ærefrygt; skulde vi da ikke meget mere underordne os under Åndernes Fader og leve?
\par 10 thi hine tugtede os for nogle få Dage efter deres Tykke, men han gør det til vort Gavn, for at vi skulle få Del i hans Hellighed.
\par 11 Al Tugtelse synes vel, imedens den er nærværende, ikke at være til Glæde, men til Bedrøvelse; men siden giver den til Gengæld dem, som derved ere øvede, en Fredens Frugt i Retfærdighed.
\par 12 Derfor, retter de slappede Hænder og de lammede Knæ,
\par 13 og træder lige Spor med eders Fødder, for at ikke det lamme skal vrides af Led, men snarere helbredes.
\par 14 Stræber efter Fred med alle og efter Helliggørelsen, uden hvilken ingen skal se Herren;
\par 15 og ser til, at ikke nogen går Glip af Guds Nåde, at ikke nogen bitter Rod skyder op og gør Skade, og de mange smittes ved den;
\par 16 at ikke nogen er en utugtig eller en vanhellig som Esau, der for een Ret Mad solgte sin Førstefødselsret.
\par 17 Thi I vide, at han også siden, da han ønskede at arve Velsignelsen, blev forkastet (thi han fandt ikke Rum for Omvendelse), omendskønt han begærede den med Tårer.
\par 18 I ere jo ikke komne til en håndgribelig og brændende Ild og til Mulm og Mørke og Uvejr,
\par 19 og ikke til Basunens Klang og til en talende Røst, hvorom de, der hørte den, bade, at der ikke mere måtte tales til dem.
\par 20 Thi de kunde ikke bære det, som blev påbudt: "Endog om et Dyr rører ved Bjerget, skal det stenes".
\par 21 Og - så frygteligt var Synet - Moses sagde: "Jeg er forfærdet og bæver."
\par 22 Men I ere komne til Zions Bjerg og til den levende Guds Stad, til det himmelske Jerusalem og til Englenes Titusinder i Højtidsskare
\par 23 og til de førstefødtes Menighed, som ere indskrevne i Himlene, og til en Dommer, som er alles Gud, og til de fuldkommede retfærdiges Ånder
\par 24 og til den nye Pagts Mellemmand, Jesus, og til Bestænkelsens Blod, som taler bedre end Abel.
\par 25 Ser til, at I ikke bede eder fri for den, som taler. Thi når de, som bade sig fri for ham, der talte sit Guddomsord på Jorden, ikke undslap, da skulle vi det meget mindre, når vi vende os bort fra ham, der taler fra Himlene,
\par 26 han, hvis Røst dengang rystede Jorden, men som nu har forjættet og sagt: "Endnu een Gang vil jeg ryste, ikke alene Jorden, men også Himmelen."
\par 27 Men dette "endnu een Gang" giver til Kende, at de Ting, der rystes, skulle omskiftes, efterdi de ere skabte, for at de Ting, der ikke rystes, skulle blive.
\par 28 Derfor, efterdi vi modtage et Rige, som ikke kan rystes, så lader os være taknemmelige og derved tjene Gud til hans Velbehag, med Ængstelse og Frygt.
\par 29 Thi vor Gud er en fortærende Ild.

\chapter{13}

\par 1 Broderkærligheden blive ved!
\par 2 Glemmer ikke Gæstfriheden; thi ved den have nogle, uden at vide det, haft Engle til Gæster.
\par 3 Kommer de fangne i Hu, som vare I selv medfangne; dem, der lide ilde, som de, der også selv ere i et Legeme.
\par 4 Ægteskabet være æret hos alle, og Ægtesengen ubesmittet; thi utugtige og Horkarle skal Gud dømme.
\par 5 Eders Vandel være uden Pengegridskhed, nøjes med det, I have; thi han har selv sagt: "Jeg vil ingenlunde slippe dig og ingenlunde forlade dig,"
\par 6 så at vi kunne sige med frit Mod: "Herren er min Hjælper, jeg vil ikke frygte; hvad kan et Menneske gøre mig?"
\par 7 Kommer eders Vejledere i Hu, som have forkyndt eder Guds Ord, og idet I betragte deres Vandrings Udgang, så efterligner deres Tro!
\par 8 Jesus Kristus er i Går og i Dag den samme, ja, til evig Tid.
\par 9 Lader eder ikke lede vild af mange Hånde og fremmede Lærdomme; thi det er godt, at Hjertet styrkes ved Nåden, ikke ved Spiser; thi deraf have de, som holdt sig dertil, ingen Nytte haft.
\par 10 Vi have et Alter, hvorfra de, som tjene ved Tabernaklet, ikke have Ret til at spise.
\par 11 Thi de Dyr, hvis Blod for Syndens Skyld bæres ind i Helligdommen af Ypperstepræsten, deres Kroppe opbrændes uden for Lejren.
\par 12 Derfor led også Jesus uden for Porten, for at han kunde hellige Folket ved sit eget Blod.
\par 13 Så lader os da gå ud til ham uden for Lejren, idet vi bære hans Forsmædelse;
\par 14 thi her have vi ikke en blivende Stad, men vi søge den kommende.
\par 15 Lader os da ved ham altid frembære Gud Lovprisnings Offer, det er: en Frugt af Læber, som bekende hans Navn.
\par 16 Men glemmer ikke at gøre vel og at meddele; thi i sådanne Ofre har Gud Velbehag.
\par 17 Lyder eders Vejledere og retter eder efter dem; thi de våge over, eders Sjæle som de, der skulle gøre Regnskab - for at de må gøre dette med Glæde og ikke sukkende: thi dette er eder ikke gavnligt.
\par 18 Beder for os; thi vi ere forvissede om, at vi have en god Samvittighed, idet vi ønske at vandre rettelig i alle Ting.
\par 19 Og jeg formaner eder des mere til at gøre dette, for at jeg desto snarere kan gives eder igen.
\par 20 Men Fredens Gud, som førte den store Fårenes Hyrde, vor Herre Jesus, op fra de døde med en evig Pagts Blod,
\par 21 han bringe eder til Fuldkommenhed i alt godt, til at gøre hans Villie, og han virke i eder det, som er velbehageligt for hans Åsyn, ved Jesus Kristus: ham være Æren i Evighedernes Evigheder: Amen.
\par 22 Jeg beder eder, Brødre! at I finde eder i dette Formaningsord; thi jeg har jo skrevet til eder i Korthed.
\par 23 Vid, at vor Broder Timotheus er løsladt; sammen med ham vil jeg se eder, dersom han snart kommer.
\par 24 Hilser alle eders Vejledere og alle de hellige! De fra Italien hilse eder.
\par 25 Nåden være med eder alle!



\end{document}