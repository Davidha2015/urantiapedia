\begin{document}

\title{James}


\chapter{1}

\par 1 Jakob, Guds og den Herres Jesu Kristi Tjener, hilser de tolv Stammer i Adspredelsen.
\par 2 Mine Brødre! agter det for idel Glæde, når I stedes i mange Hånde Prøvelser,
\par 3 vidende, at eders Tros Prøve virker Udholdenhed;
\par 4 men Udholdenheden bør medføre fuldkommen Gerning, for at I kunne være fuldkomne og uden Brøst, så I ikke stå tilbage i noget.
\par 5 Men dersom nogen af eder fattes Visdom, han bede derom til Gud, som giver alle gerne og uden Bebrejdelse, så skal den gives ham.
\par 6 Men han bede i Tro, uden at tvivle; thi den, som tvivler, ligner en Havets Bølge, der drives og kastes af Vinden.
\par 7 Ikke må nemlig det Menneske mene, at han skal få noget af Her ren,
\par 8 en tvesindet Mand, som han er, ustadig på alle sine Veje.
\par 9 Men den Broder, som er ringe, rose sig af sin Højhed,
\par 10 den rige derimod af sin Ringhed; thi han skal forgå som Græssets Blomst.
\par 11 Thi Solen står op med sin Hede og hentørrer Græsset, og dets Blomst falder af, og dens Skikkelses Ynde forsvinder; således skal også den rige visne på sine Veje.
\par 12 Salig den Mand, som holder Prøvelse ud; thi når han har stået Prøve, skal han få Livets Krans, som Herren har forjættet dem, der elske ham.
\par 13 Ingen sige, når han fristes: "Jeg fristes af Gud;" thi Gud kan ikke fristes af det onde, og selv frister han ingen;
\par 14 men enhver fristes,når han drages og lokkes af sin egen Begæring;
\par 15 derefter, når Begæringen har undfanget, føder den Synd, men når Synden er fuldvoksen, føder den Død.
\par 16 Farer ikke vild, mine elskede Brødre!
\par 17 Al god Gave og al fuldkommen Gave er ovenfra og kommer ned fra Lysenes Fader, hos hvem der ikke er Forandring eller skiftende Skygge.
\par 18 Efter sin Villie fødte han os ved Sandheds Ord, for at vi skulde være en Førstegrøde af hans Skabninger.
\par 19 I vide det, mine elskede Brødre. Men hvert Menneske være snar til at høre, langsom til at tale, langsom til Vrede;
\par 20 thi en Mands Vrede udretter ikke det, som er ret for Gud.
\par 21 Derfor, aflægger alt Smuds og Levning af Slethed, og modtager med Sagtmodighed Ordet, som er indplantet i eder, og som formår at frelse eders Sjæle.
\par 22 Men vorder Ordets Gørere og ikke alene dets Hørere, hvormed I bedrage eder selv.
\par 23 Thi dersom nogen er Ordets Hører og ikke dets Gører, han ligner en Mand, der betragter sit legemlige Ansigt i et Spejl;
\par 24 thi han betragter sig selv og går bort og glemmer straks, hvor dan han var.
\par 25 Men den, som skuer ind i Frihedens fuldkomne Lov og holder ved dermed, så han ikke bliver en glemsom Tilhører, men en Gerningens Gører, han skal være salig i sin Gerning.
\par 26 Dersom nogen synes, at han dyrker Gud, og ikke holder sin Tunge i Tømme, men bedrager sit Hjerte, hans Gudsdyrkelse er forgæves.
\par 27 En ren og ubesmittet Gudsdyrkelse for Gud og Faderen er dette, at besøge faderløse og Enker i deres Trængsel, at holde sig selv uplettet af Verden.

\chapter{2}

\par 1 Mine Brødre! Eders Tro på vor Herre Jesus Kristus, den herliggjorte, være ikke forbunden med Persons Anseelse!
\par 2 Når der nemlig kommer en Mand ind i eders Forsamling med Guldring på Fingeren, i prægtig Klædning, men der også kommer en fattig ind i smudsig Klædning,
\par 3 og I fæste Øjet på den, som bærer den prægtige Klædning og sige: Sæt du dig her på den gode Plads, og I sige til den fattige: Stå du der eller sæt dig nede ved min Fodskammel:
\par 4 ere I så ikke komne i Strid med eder selv og blevne Dommere med slette Tanker?
\par 5 Hører, mine elskede Brødre! Har Gud ikke udvalgt de for Verden fattige til at være rige i Tro og Arvinger til det Rige, som han har forjættet dem, der elske ham?
\par 6 Men I have vanæret den fattige! Er det ikke de rige, som underkue eder, og er det ikke dem, som slæbe eder for Domstolene?
\par 7 Er det ikke dem, som bespotte det skønne Navn, som er nævnet over eder?
\par 8 Ganske vist, dersom I opfylde den kongelige Lov efter Skriften: "Du skal elske din Næste som dig selv", gøre I ret;
\par 9 men dersom I anse Personer, gøre I Synd og revses af Loven som Overtrædere.
\par 10 Thi den, som holder hele Loven, men støder an i eet Stykke, er bleven skyldig i alle.
\par 11 Thi han, som sagde: "Du må ikke bedrive Hor," sagde også: "Du må ikke slå ihjel." Dersom du da ikke bedriver Hor, men slår ihjel, da er du bleven en Lovens Overtræder.
\par 12 Taler således og gører således, som de, der skulle dømmes efter Frihedens Lov.
\par 13 Thi Dommen er ubarmhjertig imod den, som ikke har øvet Barmhjertighed; Barmhjertighed træder frimodigt op imod Dommen.
\par 14 Hvad gavner det, mine Brødre! om nogen siger, han har Tro, men ikke har Gerninger? mon Troen kan frelse ham?
\par 15 Dersom en Broder eller Søster er nøgen og fattes den daglige Føde,
\par 16 og en af eder siger til dem: Går bort i Fred, varmer eder og mætter eder, men I ikke give dem det, som hører til Legemets Nødtørst, hvad gavner det?
\par 17 Ligeså er også Troen, dersom den ikke har Gerninger, død i sig selv.
\par 18 Men man vil sige: Du har Tro, og jeg har Gerninger. Vis mig din Tro uden Gerningerne, og jeg vil af mine Gerninger vise dig Troen.
\par 19 Du tror, at Gud er een; deri gør du ret; også de onde Ånder tro det og skælve.
\par 20 Men vil du vide, du tomme Menneske! at Troen uden Gerninger er unyttig?
\par 21 Blev ikke vor Fader Abraham retfærdiggjort af Gerninger, da han ofrede sin Søn Isak på Alteret?
\par 22 Du ser, at Troen virkede sammen med hans Gerninger, og ved Gerningerne blev Troen fuldkommet,
\par 23 og Skriften blev opfyldt, som siger: "Abraham troede Gud, og det blev regnet ham til Retfærdighed", og han blev kaldet Guds Ven.
\par 24 I se, at et Menneske retfærdiggøres af Gerninger, og ikke af Tro alene.
\par 25 Ligeså Skøgen Rahab, blev ikke også hun retfærdiggjort af Gerninger, da hun tog imod Sendebudene og lod dem slippe bort ad en anden Vej?
\par 26 Thi ligesom Legemet er dødt uden Ånd, således er også Troen død uden Gerninger.

\chapter{3}

\par 1 Mine Brødre! ikke mange af eder bør blive Lærere, såsom I vide, at vi skulle få en desto tungere Dom.
\par 2 Thi vi støde alle an i mange Ting; dersom nogen ikke støder an i Tale, da er han en fuldkommen Mand, i Stand til også at holde hele Legemet i Tomme.
\par 3 Men når vi lægge Bidsler i Hestenes Munde, for at de skulle adlyde os, så dreje vi også hele deres Legeme.
\par 4 Se, også Skibene, endskønt de ere så store og drives af stærke Vinde, drejes med et såre lidet Ror, hvorhen Styrmandens Hu står.
\par 5 Således er også Tungen et lille Lem og fører store Ord. Se, hvor lille en Ild der stikker så stor en Skov i Brand!
\par 6 Og Tungen er en Ild. Som en Verden af Uretfærdighed sidder Tungen iblandt vore Lemmer; den besmitter hele Legemet og sætter Livets Hjul i Brand, selv sat i Brand af Helvede.
\par 7 Thi enhver Natur, både Dyrs og Fugles, både Krybdyrs og Havdyrs, tæmmes og er tæmmet af den menneskelige Natur;
\par 8 men Tungen kan intet Menneske tæmme, det ustyrlige Onde, fuld af dødbringende Gift.
\par 9 Med den velsigne vi Herren og Faderen, og med den forbande vi Menneskene, som ere blevne til efter Guds Lighed.
\par 10 Af den samme Mund udgår Velsignelse og Forbandelse. Mine Brødre! dette bør ikke være så.
\par 11 Mon en Kilde udgyder sødt Vand og besk Vand af det samme Væld?
\par 12 Mon et Figentræ, mine Brødre! kan give Oliven, eller et Vintræ Figener? Heller ikke kan en Salt Kilde give fersk Vand.
\par 13 Er nogen viis og forstandig iblandt eder, da vise han ved god Omgængelse sine Gerninger i viis Sagtmodighed!
\par 14 Men have I bitter Avind og Rænkesyge i eders Hjerter, da roser eder ikke og lyver ikke imod Sandheden!
\par 15 Dette er ikke den Visdom, som kommer ovenfra, men en jordisk, sjælelig, djævelsk;
\par 16 thi hvor der er Avind og Rænkesyge, der er Forvirring og al ond Handel.
\par 17 Men Visdommen herovenfra er først ren, dernæst fredsommelig, mild, føjelig, fuld at Barmhjertighed og gode Frugter, upartisk, uden Skrømt.
\par 18 Men Retfærdigheds Frugt såes i Fred for dem, som stifte Fred.

\chapter{4}

\par 1 Hvoraf kommer det, at den er Krige og Stridigheder iblandt eder? mon ikke deraf, af eders Lyster, som stride i eders Lemmer?
\par 2 I begære og have ikke; I myrde og misunde og kunne ikke få; I føre Strid og Krig. Og I have ikke, fordi I ikke bede;
\par 3 I bede og få ikke, fordi I bede ilde, for at øde det i eders Lyster.
\par 4 I, utro! vide I ikke, at Venskab med Verden er Fjendskab imod Gud? Derfor, den, som vil være Verdens Ven, bliver Guds Fjende.
\par 5 Eller mene I, at Skriftens Ord ere tomme Ord? Med Nidkærhed længes han efter den Ånd, han har givet Bolig i os, men han skænker desto større Nåde.
\par 6 Derfor siger Skriften: "Gud står de hoffærdige imod, men de ydmyge giver han Nåde."
\par 7 Underordner eder derfor under Gud; men står Djævelen imod, så skal han fly fra eder;
\par 8 holder eder nær til Gud, så skal han holde sig nær til eder! Renser Hænderne, I Syndere! og lutrer Hjerterne, I tvesindede!
\par 9 Jamrer og sørger og græder; eders Latter vende sig til Sorg og Glæden til Bedrøvelse!
\par 10 Ydmyger eder for Herren, så skal han ophøje eder.
\par 11 Taler ikke ilde om hverandre, Brødre! Den, som taler ilde om sin Broder eller dømmer sin Broder, taler ilde om Loven og dømmer Loven; men dømmer du Loven, da er du ikke Lovens Gører, men dens Dommer.
\par 12 Een er Lovgiveren og Dommeren, han, som kan frelse og fordærve; men hvem er du, som dømmer din Næste?
\par 13 Og nu I, som sige: I Dag eller i Morgen ville vi gå til den eller den By og blive der et År og købslå og vinde,
\par 14 I, som ikke vide, hvad der skal ske i Morgen; thi hvordan er eders Liv? I ere jo en Damp, som er til Syne en liden Tid, men derefter forsvinder;
\par 15 i Stedet for at I skulle sige: Dersom Herren vil, og vi leve, da ville vi gøre dette eller hint.
\par 16 Men nu rose I eder i eders Overmod; al sådan Ros er ond.
\par 17 Derfor, den som ved at handle ret og ikke gør det, for ham er det Synd.

\chapter{5}

\par 1 Og nu, I rige! græder og jamrer over de Ulykker, som komme over eder.
\par 2 Eders Rigdom er rådnet, og eders Klæder er mølædte;
\par 3 eders Guld og Sølv er rustet op, og deres Rust skal være til Vidnesbyrd imod eder og æde eders Kød som en Ild; I have samlet Skatte i de sidste Dage.
\par 4 Se, den Løn skriger, som I have forholdt Arbejderne, der høstede eders Marker, og Høstfolkenes Råb ere komne ind for den Herre Zebaoths Øren.
\par 5 I levede i Vellevned på Jorden og efter eders Lyster; I gjorde eders Hjerter til gode som på en Slagtedag.
\par 6 I domfældte, I dræbte den retfærdige; han står eder ikke imod.
\par 7 Derfor, værer tålmodige, Brødre! indtil Herrens Tilkommelse. Se, Bonden venter på Jordens dyrebare Frugt og bier tålmodigt efter den, indtil den får tidlig Regn og sildig Regn.
\par 8 Værer også I tålmodige, styrker eders Hjerter; thi Herrens Tilkommelse er nær.
\par 9 Sukker ikke imod hverandre, Brødre! for at I ikke skulle dømmes; se, Dommeren står for Døren.
\par 10 Brødre! tager Profeterne, som have talt i Herrens Navn, til Forbillede på at lide ondt og være tålmodige.
\par 11 Se, vi prise dem salige, som have holdt ud. I have hørt om Jobs Udholdenhed og vide Udfaldet fra Herren; thi Herren er såre medlidende og barmhjertig.
\par 12 Men for alting, mine Brødre! sværger ikke, hverken ved Himmelen eller ved Jorden eller nogen anden Ed; men eders Ja være Ja, og Nej være Nej, for at I ikke skulle falde under Dom.
\par 13 Lider nogen iblandt eder ondt, han bede; er nogen vel til Mode, han synge Lovsang!
\par 14 Er nogen iblandt eder syg, han kalde Menighedens Ældste til sig, og de skulle bede over ham og salve ham med Olie i Herrens Navn.
\par 15 Og Troens Bøn skal frelse den syge, og Herren skal oprejse ham, og har han gjort Synder, skulle de forlades ham.
\par 16 Bekender derfor Synderne for hverandre og beder for hverandre, for at I må blive helbredte; en retfærdigs Bøn formår meget, når den er alvorlig.
\par 17 Elias var et Menneske, lige Vilkår undergivet med os, og han bad en Bøn, at det ikke måtte regne; og det regnede ikke på Jorden i tre År og seks Måneder.
\par 18 Og han bad atter, og Himmelen gav Regn, og Jorden bar sin Frugt.
\par 19 Mine Brødre! dersom nogen iblandt eder farer vild fra Sandheden, og nogen omvender ham,
\par 20 han vide, at den, som omvender en Synder fra hans Vejs Vildfarelse, han Frelser en Sjæl fra Døden og skjuler en Mangfoldighed af Synder.


\end{document}