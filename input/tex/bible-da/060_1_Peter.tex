\begin{document}

\title{Første Petersbrev}


\chapter{1}

\par 1 Peter, Jesu Kristi Apostel, til Udlændingene i Adspredelse i Pontus, Galatien, Kappadokien, Asien og Bithynien,
\par 2 udvalgte efter Gud Faders Forudviden, ved Åndens Helligelse, til Lydighed og Bestænkelse med Jesu Kristi Blod: Nåde og Fred vorde eder mangfoldig til Del!
\par 3 Lovet være Gud og vor Herres Jesu Kristi Fader, som efter sin store Barmhjertighed har genfødt os til et levende Håb ved Jesu Kristi Opstandelse fra de døde,
\par 4 til en uforkrænkelig og ubesmittelig og uvisnelig Arv, som er bevaret i Himlene til eder,
\par 5 I, som ved Guds Kraft bevogtes ved Tro til en Frelse, som er rede til at åbenbares i den sidste Tid,
\par 6 i hvilken I skulle fryde eder, om I end nu en liden Stund, hvis så skal være, bedrøves i mange Hånde Prøvelser,
\par 7 for at eders prøvede Tro, som er meget dyrebarere end det forgængelige Guld, der dog prøves ved Ild, må findes til Ros og Herlighed og Ære i Jesu Kristi Åbenbarelse,
\par 8 ham, som I ikke have set og dog elske, ham, som I, skønt I nu ikke se, men tro, skulle fryde eder over med en uudsigelig og forherliget Glæde,
\par 9 når I nå Målet for eders Tro, Sjælenes Frelse.
\par 10 Om denne Frelse have Profeter gransket og ransaget, de, som profeterede om den Nåde, der skulde blive eder til Del,
\par 11 idet de granskede over, hvilken eller hvordan en Tid Kristi Ånd, som var i dem, henviste til, når den forud vidnede om Kristi Lidelser og den derpå følgende Herlighed.
\par 12 Og det blev dem åbenbaret, at det ikke var dem selv, men eder, de tjente med disse Ting, som nu ere blevne eder kundgjorte ved dem, der have forkyndt eder Evangeliet i den Helligånd, som blev sendt fra Himmelen, hvilke Ting Engle begære at skue ind i.
\par 13 Derfor, binder op om eders Sinds Lænder, værer ædrue, og sætter fuldt ud eders Håb til den Nåde, som bliver eder til Del i Jesu Kristi Åbenbarelse.
\par 14 Som lydige Børn skulle I ikke skikke eder efter de forrige Lyster i eders Vankundighed;
\par 15 men efter den hellige, som kaldte eder, skulle også I vorde hellige i al eders Vandel;
\par 16 thi der er skrevet: "I skulle være hellige, thi jeg er hellig."
\par 17 Og dersom I påkalde ham som Fader, der dømmer uden Persons Anseelse efter enhvers Gerning, da bør I vandre i Frygt eders Udlændigheds Tid,
\par 18 vel vidende, at det ikke var med forkrænkelige Ting,, Sølv eller Guld, at I bleve løskøbte fra eders tomme Vandel, som var overleveret eder fra Fædrene,
\par 19 men med Kristi dyrebare Blod som et lydeløst og uplettet Lams,
\par 20 han, som var forud kendt for Verdens Grundlæggelse, men blev åbenbaret ved Tidernes Ende for eders Skyld,
\par 21 der ved ham tro på Gud, som oprejste ham fra de døde og gav ham Herlighed, så at eders Tro også er Håb til Gud.
\par 22 Lutrer eders Sjæle i Lydighed imod Sandheden til uskrømtet Broderkærlighed, og elsker hverandre inderligt af Hjertet,
\par 23 genfødte, som I ere, ikke af forkrænkelig, men af uforkrænkelig Sæd, ved Guds levende og blivende Ord.
\par 24 Thi "alt Kød er som Græs, og al dets Herlighed som Græssets Blomst; Græsset visner, og Blomsten falder af;
\par 25 men Herrens Ord bliver evindelig." Og dette er det Ord, som er forkyndt eder ved Evangeliet.

\chapter{2}

\par 1 Derfor aflægger al Ondskab og al Svig og Hykleri og Avind og al Bagtalelse,
\par 2 og higer som nyfødte Børn efter Ordets uforfalskede Mælk, for at I kunne vokse ved den til Frelse,
\par 3 om I da have smagt, at Herren er god.
\par 4 Kommer til ham, den levende Sten, der vel er forkastet af Menneskene, men er udvalgt og dyrebar for Gud,
\par 5 og lader eder selv som levende Sten opbygge som et åndeligt Hus, til et helligt Præsteskab, til at frembære åndelige Ofre, velbehagelige for Gud ved Jesus Kristus.
\par 6 Thi det hedder i et Skriftsted: "Se, jeg lægger i Zion en Hovedhjørnesten, som er udvalgt og dyrebar; og den, som tror på ham, skal ingenlunde blive til Skamme."
\par 7 Eder altså, som tro, hører Æren til; men for de vantro er denne Sten, som Bygningsmændene forkastede, bleven til en Hovedhjørnesten og en Anstødssten og en Forargelses Klippe;
\par 8 og de støde an, idet de ere genstridige imod Ordet, hvortil de også vare bestemte.
\par 9 Men I ere en udvalgt Slægt, et kongeligt Præsteskab, et helligt Folk, et Folk til Ejendom, for at I skulle forkynde hans Dyder, som kaldte eder fra Mørke til sit underfulde Lys,
\par 10 I, som fordum ikke vare et Folk, men nu ere Guds Folk, I, som ikke fandt Barmhjertighed, men nu have fundet Barmhjertighed.
\par 11 I elskede! jeg formaner eder som fremmede og Udlændinge til at afolde eder fra kødelige Lyster, som jo føre Krig imod Sjælen,
\par 12 så I føre en god Vandel iblandt Hedningerne, for at de på Grund af de gode Gerninger, som de få at se, kunne prise Gud på Besøgelsens Dag for det, som de bagtale eder for som Ugerningsmænd.
\par 13 Underordner eder under al menneskelig Ordning for Herrens Skyld, være sig en Konge som den højeste,
\par 14 eller Landshøvdinger som dem, der sendes af ham til Straf for Ugerningsmænd, men til Ros for dem, som gøre det gode.
\par 15 Thi således er det Guds Villie, at I ved at gøre det gode skulle bringe de uforstandige Menneskers Vankundighed til at tie;
\par 16 som frie, og ikke som de, der have Friheden til Ondskabs Skjul, men som Guds Tjenere.
\par 17 Ærer alle, elsker Broderskabet, frygter Gud, ærer Kongen!
\par 18 I Trælle! underordner eder under eders Herrer i al Frygt, ikke alene de gode og milde, men også de urimelige.
\par 19 Thi dette finder Yndest, dersom nogen, bunden til Gud i sin Samvittighed, udholder Genvordigheder, skønt han lider uretfærdigt.
\par 20 Thi hvad Ros er det, om I holde ud, når I Synde og derfor få Næveslag? Men dersom I holde ud, når I gøre det gode og lide derfor, dette finder Yndest hos Gud.
\par 21 Thi dertil bleve I kaldede, efterdi også Kristus har lidt for eder, efterladende eder et Forbillede, for at I skulle følge i hans Fodspor,
\par 22 han, som ikke gjorde Synd, ikke heller blev der fundet Svig i hans Mund,
\par 23 han, som ikke skældte igen, da han blev udskældt, ikke truede, da han led, men overgav det til ham, som dømmer retfærdigt,
\par 24 han, som selv bar vore Synder på sit Legeme op på Træet, for at vi, afdøde fra vore Synder, skulle leve for Retfærdigheden, han, ved hvis Sår I ere blevne lægte.
\par 25 Thi I vare vildfarende som Får, men ere nu vendte om til eders Sjæles Hyrde og Tilsynsmand.

\chapter{3}

\par 1 Ligeså, I Hustruer! underordner eder under eders egne Mænd, for at, selv om nogle ere genstridige imod Ordet, de kunne vindes uden Ord ved Hustruernes Vandel,
\par 2 når de iagttage eders kyske Vandel i Frygt.
\par 3 Eders Prydelse skal ikke være den udvortes med Hårfletning og påhængte Guldsmykker eller Klædedragt,
\par 4 men Hjertets skjulte Menneske med den sagtmodige og stille Ånds uforkrænkelige Prydelse, hvilket er meget kosteligt for Gud.
\par 5 Thi således var det også, at fordum de hellige Kvinder, som håbede på Gud, prydede sig, idet de underordnede sig under deres egne Mænd,
\par 6 som Sara var Abraham lydig, så hun kaldte ham Herre, hun, hvis Børn I ere blevne, når I gøre det gode og ikke frygte nogen Rædsel.
\par 7 Ligeså I Mænd! lever med Forstand sammen med eders Hustruer som med et svagere Kar, og beviser dem Ære som dem, der også ere Medarvinger til Livets Nådegave, for at eders Bønner ikke skulle hindres.
\par 8 Og til Slutning værer alle enssindede, medlidende, kærlige imod Brødrene, barmhjertige, ydmyge;
\par 9 betaler ikke ondt med ondt, eller Skældsord med Skældsord, men tværtimod velsigner, thi dertil bleve I kaldede, at I skulle arve Velsignelse.
\par 10 Thi "den, som vil elske Livet og se gode Dage, skal holde sin Tunge fra ondt og sine Læber fra at tale Svig;
\par 11 han vende sig fra ondt og gøre godt; han søge Fred og jage efter den!
\par 12 Thi Herrens Øjne ere over de retfærdige, og hans Øren til deres Bøn; men Herrens Ansigt er over dem, som gøre ondt."
\par 13 Og hvem er der, som kan volde eder ondt, dersom I ere nidkære for det gode?
\par 14 Men om I også måtte lide for Retfærdigheds Skyld, er I salige.
\par 15 men helliger den Herre Kristus i eders Hjerter, altid rede til at forsvare eder over for enhver, som kræver eder til Regnskab for det Håb, der er i eder, men med Sagtmodighed og Frygt,
\par 16 idet I have en god Samvittighed, for at de, der laste eders gode Vandel i Kristus, må blive til Skamme, når de bagtale eder som Ugerningsmænd.
\par 17 Thi det er bedre, om det så er Guds Villie, at lide, når man gør godt, end når man gør ondt.
\par 18 Thi også Kristus led een Gang for Synder, en retfærdig for uretfærdige, for at han kunde føre os hen til Gud, han, som led Døden i Kødet, men blev levendegjort i Ånden,
\par 19 i hvilken han også gik hen og prædikede for Ånderne, som vare i Forvaring,
\par 20 som fordum vare genstridige, dengang Guds Langmodighed ventede i Noas Dage, medens Arken byggedes, i hvilken få, nemlig otte, Sjæle bleve frelste igennem Vand,
\par 21 hvilket nu også frelser eder i sit Modbillede som Dåb, der ikke er Fjernelse af Kødets Urenhed, men en god Samvittigheds Pagt med Gud ved Jesu Kristi Opstandelse,
\par 22 han, som er faren til Himmelen og er ved Guds højre Hånd, efter at Engle og Myndigheder og Kræfter ere ham underlagte.

\chapter{4}

\par 1 Efterdi da Kristus har lidt i Kødet, så skulle også I væbne eder med det samme Sind (thi den, som har lidt i Kødet, er hørt op med Synd),
\par 2 så at I ikke fremdeles leve den øvrige Tid i Kødet efter Menneskers Lyster, men efter Guds Villie.
\par 3 Thi det er nok i den forbigangne Tid at have gjort Hedningernes Villie, idet I have vandret i Uterlighed, Lyster, Fylderi, Svir, Drik og skammelig Afgudsdyrkelse;
\par 4 hvorfor de forundre sig og spotte, når I ikke løbe med til den samme Ryggesløshedens Pøl;
\par 5 men de skulle gøre ham Regnskab, som er rede til at dømme levende og døde.
\par 6 Thi derfor blev Evangeliet forkyndt også for døde, for at de vel skulde være dømte på Menneskers Vis i Kødet, men leve på Guds Vis i Ånden.
\par 7 Men alle Tings Ende er kommen nær; værer derfor årvågne og ædru til Bønner!
\par 8 Hav fremfor alt en inderlig Kærlighed til hverandre; thi "Kærlighed skjuler en Mangfoldighed af Synder".
\par 9 Vær gæstfri imod hverandre uden Knurren.
\par 10 Eftersom enhver har fået en Nådegave, skulle I tjene hverandre dermed som gode Husholdere over Guds mangfoldige Nåde.
\par 11 Taler nogen, han tale som Guds Ord; har nogen en Tjeneste, han tjene, efter som Gud forlener ham Styrke dertil, for at Gud må æres i alle Ting ved Jesus Kristus, hvem Herligheden og Magten tilhører i Evighedernes Evigheder! Amen.
\par 12 I elskede! undrer eder ikke over den Ild, som brænder iblandt eder til eders Prøvelse, som om der hændtes eder noget underligt;
\par 13 men glæder eder i samme Mål, som I have Del i Kristi Lidelser, for at I også kunne glæde og fryde eder ved hans Herligheds Åbenbarelse.
\par 14 Dersom I hånes for Kristi Navns Skyld, ere I salige; thi Herlighedens og Guds Ånd hviler over eder.
\par 15 Thi ingen af eder bør lide som Morder eller Tyv eller Ugerningsmand eller som en, der blander sig i anden Mands Sager;
\par 16 men lider han som en Kristen, da skamme han sig ikke, men prise Gud for dette Navn!
\par 17 Thi det er Tiden til, at Dommen skal begynde med Guds Hus: men begynder den først med os, hvad Ende vil det da få med dem, som ere genstridige imod Guds Evangelium?
\par 18 Og dersom den retfærdige med Nød og neppe bliver frelst, hvor skal da den ugudelige og Synderen blive af?
\par 19 Derfor skulle også de, som lide efter Guds Villie, befale den trofaste Skaber deres Sjæle, idet de gøre det gode.

\chapter{5}

\par 1 De Ældste iblandt eder formaner jeg som Medældste og Vidne til Kristi Lidelser, som den, der også har Del i Herligheden, der skal åbenbares:
\par 2 Vogter Guds Hjord hos eder, og fører Tilsyn med den, ikke tvungne, men frivilligt, ikke for slet Vindings Skyld, men med Redebonhed;
\par 3 ikke heller som de, der ville herske over Menighederne, men som Mønstre for Hjorden;
\par 4 og når da Overhyrden åbenbares, skulle I få Herlighedens uvisnelige Krans.
\par 5 Ligeså, I unge! underordner eder under de ældre; og ifører eder alle Ydmyghed imod hverandre; thi "Gud står de hoffærdige imod, men de ydmyge giver han Nåde."
\par 6 Derfor ydmyger eder under Guds vældige Hånd, for at han i sin Tid må ophøje eder.
\par 7 Kaster al eders Sørg på ham, thi han har Omsorg for eder.
\par 8 Vær ædru, våger; eders Modstander, Djævelen, går omkring som en brølende Løve, søgende, hvem han kan opsluge.
\par 9 Står ham imod, faste i Troen, vidende, at de samme Lidelser fuldbyrdes på eders Brødre i Verden.
\par 10 Men al Nådes Gud, som kaldte eder til sin evige Herlighed i Kristus Jesus efter en kort Tids Lidelse, han vil selv fuldelig berede eder, styrke, bekræfte, grundfæste eder!
\par 11 Ham tilhører Magten i Evighedernes Evigheder! Amen.
\par 12 Med Silvanus, den trofaste Broder (thi det holder jeg ham for), har jeg i Korthed skrevet eder til for at formane og bevidne, at dette er Guds sande Nåde, hvori I stå.
\par 13 Den medudvalgte) i Babylon og min Søn, Markus, hilser eder.
\par 14 Hilser hverandre med Kærligheds Kys! Fred være med eder alle, som ere i Kristus!



\end{document}