\begin{document}

\title{2 Peter}


\chapter{1}

\par 1 Simon Peter, Jesu Kristi Tjener og Apostel, til dem, der have fået samme dyrebare Tro som vi ved vor Guds og Frelsers Jesu Kristi Retfærdighed:
\par 2 Nåde og Fred vorde eder mangfoldig til Del i Erkendelse af Gud og vor Herre Jesus.
\par 3 Såsom hans guddommelige Magt har skænket os alt, hvad der hører til Liv og Gudfrygtighed ved Erkendelsen af ham, som kaldte os ved sin Herlighed og Kraft,
\par 4 hvorved han har skænket os de største og dyrebare Forjættelser, for at I ved disse skulle få Del i guddommelig Natur, når I undfly Fordærvelsen i Verden, som har sin Grund i Begær,
\par 5 så anvender just derfor al Flid på i eders Tro at udvise Dyd og i Dyden Kundskab
\par 6 og i Kundskaben Afholdenhed og i Afholdenheden Udholdenhed og i Udholdenheden Gudsfrygt
\par 7 og i Gudsfrygten Broderkærlighed og i Broderkærligheden Kærlighed.
\par 8 Thi når dette findes hos eder og er i Tiltagen, lader det eder ikke stå ørkesløse eller ufrugtbare i Erkendelsen af vor Herre Jesus Kristus;
\par 9 den nemlig, som ikke har dette, er blind, svagsynet, idet han har glemt Renselsen fra sine fordums Synder.
\par 10 Derfor, Brødre! gører eder des mere Flid for at befæste eders Kaldelse og Udvælgelse; thi når I gøre dette, skulle I ingen Sinde støde an.
\par 11 Thi så skal der rigelig gives eder Indgang i vor Herres og Frelsers Jesu Kristi evige Rige.
\par 12 Derfor vil jeg ikke forsømme altid at påminde eder om delte, ihvorvel I vide det og ere befæstede i den Sandhed, som er til Stede hos os.
\par 13 Men jeg anser det for ret at vække eder ved Påmindelse, så længe jeg er i dette Telt,
\par 14 da jeg ved, at Aflæggelsen af mit Telt kommer brat, således som jo vor Herre Jesus Kristus har givet mig til Kende.
\par 15 Og jeg vil også gøre mig Flid for, at t til enhver Tid efter min Bortgang kunne drage eder dette i Minde.
\par 16 Thi vi have ikke fulgt klogtigt opdigtede Fabler, da vi kundgjorde eder vor Herres Jesu Kristi Kraft og Tilkommelse, men vi have været Øjenvidner til hans Majestæt,
\par 17 nemlig da han fik Ære og Herlighed af Gud Fader, idet en sådan Røst lød til ham fra den majestætiske Herlighed: "Denne er min Søn, den elskede, i hvem jeg har Velbehag."
\par 18 Og vi hørte denne Røst lyde fra Himmelen, da vi vare med ham på det hellige Bjerg.
\par 19 Og des mere stadfæstet have vi det profetiske Ord, hvilket I gøre vel i at agte på som på et Lys, der skinner på et mørkt Sted, indtil Dagen bryder frem, og Morgenstjernen oprinder i eders Hjerter,
\par 20 idet I fornemmelig mærke eder dette, at ingen Profeti i Skriften beror på egen Tydning.
\par 21 Thi aldrig er nogen Profeti bleven fremført ved et Menneskes Villie; men drevne af den Helligånd talte hellige Guds Mænd.

\chapter{2}

\par 1 Men der opstod også falske Profeter iblandt Folket, ligesom der også iblandt eder vil komme falske Lærere, som ville liste fordærvelige Vranglærdomme ind, idet de endog fornægte den Herre, som købte dem, og bringe en brat Undergang over sig selv,
\par 2 og mange ville efterfølge deres Uterligheder, så Sandhedens Vej for deres Skyld vil blive bespottet,
\par 3 og i Havesyge ville de med falske Ord skaffe sig Vinding af eder; men Dommen over dem har alt fra gamle Dage været i Virksomhed, og deres Undergang slumrer ikke.
\par 4 Thi når Gud ikke sparede Engle, da de syndede, men nedstyrtede dem i Afgrunden og overgav dem til Mørkets Huler for at bevogtes til Dom,
\par 5 og ikke sparede den gamle Verden, men bevarede Retfærdighedens Prædiker Noa selv ottende, da han førte Oversvømmelse over en Verden af ugudelige
\par 6 og lagde Sodomas og Gomorras Stæder i Aske og domfældte dem til Ødelæggelse, så han har sat dem til Forbillede for dem, som i Fremtiden ville leve ugudeligt,
\par 7 og udfriede den retfærdige Lot, som plagedes af de ryggesløses uterlige Vandel,
\par 8 (thi medens den retfærdige boede iblandt dem, pintes han Dag for Dag i sin retfærdige Sjæl ved de lovløse Gerninger, som han så og hørte):
\par 9 - da ved Herren at udfri gudfrygtige af Fristelse, men at straffe og bevogte uretfærdige til Dommens Dag,
\par 10 og mest dem, som vandre efter Kød, i Begær efter Besmittelse, og foragte Herskab. Frække, selvbehagelige, bæve de ikke ved at bespotte Herligheder,
\par 11 hvor dog Engle, som ere større i Styrke og Magt, ikke fremføre bespottende Dom imod dem for Herren.
\par 12 Men disse ligesom ufornuftige Dyr, der af Natur ere fødte til at fanges og ødelægges, skulle de, fordi de bespotte, hvad de ikke kende, også ødelægges med hines Ødelæggelse,
\par 13 idet de få Uretfærdigheds Løn. De sætte deres Lyst i Vellevned om Dagen, disse Skampletter og Skændselsmennesker! De svælge i deres Bedragerier, medens de holde Gilde med eder;
\par 14 deres Øjne ere fulde af Horeri og kunne ikke få nok af Synd; de lokke ubefæstede Sjæle; de have et Hjerte, øvet i Havesyge, Forbandelsens Børn;
\par 15 de have forladt den lige Vej og ere farne vild, følgende Bileams, Beors Søns, Vej, han, som elskede Uretfærdigheds Løn,
\par 16 men fik Revselse for sin Overtrædelse: et umælende Trældyr talte med menneskelig Røst og hindrede Profetens Afsind.
\par 17 Disse ere vandløse Kilder og Tågeskyer, som drives af Storvind; for dem er Mørke og Mulm bevaret.
\par 18 Thi dem, som ere lige ved at undslippe fra dem, der vandre i Vildfarelse, løkke de i Kødets Begæringer ved Uterligheder, idet de tale Tomheds overmodige Ord
\par 19 og love dem Frihed, skønt de selv ere Fordærvelsens Trælle; thi man er Træl af det, som man er overvunden af.
\par 20 Thi dersom de, efter at have undflyet Verdens Besmittelser ved Erkendelse af vor Herre og Frelser Jesus Kristus, igen lade sig indvikle deri og overvindes, da er det sidste blevet værre med dem end det første.
\par 21 Thi bedre havde det været dem ikke at have erkendt Retfærdighedens Vej end efter at have erkendt den at vende sig bort fra det hellige Bud, som var blevet dem overgivet.
\par 22 Det er gået dem efter det sande Ordsprog: "Hunden vender sig om til sit eget Spy, og den toede So til at vælte sig i Sølen."

\chapter{3}

\par 1 Dette er allerede, I elskede! det andet Brev, som jeg skriver til eder, hvori jeg ved Påmindelse vækker eders rene Sind
\par 2 til at komme de Ord i Hu, som forud ere sagte af de hellige Profeter, og eders Apostles Befaling fra Herren og Frelseren,
\par 3 idet I først og fremmest mærke eder dette, at i de sidste Dage skal der komme Spottere med Spot, som vandre efter deres egne Begæringer
\par 4 og sige: "Hvad bliver der af Forjættelsen om hans Tilkommelse? Fra den Dag, Fædrene sov hen, forblive jo alle Ting, som de vare, lige fra Skabningens Begyndelse."
\par 5 Thi med Villie ere de blinde for dette, at fra fordums Tid var der Himle og en Jord, som var fremstået af Vand og ved Vand i Kraft af Guds Ord,
\par 6 hvorved den daværende Verden gik til Grunde i en Vandflod.
\par 7 Men de nuværende Himle og Jorden holdes ved det samme Ord i Forvaring til Ild, idet de bevares til de ugudelige Menneskers Doms og Undergangs Dag.
\par 8 Men dette ene bør ikke undgå eder, I elskede! at een Dag er for Herren som tusinde År, og tusinde År som een Dag.
\par 9 Herren forhaler ikke Forjættelsen (som nogle agte det for en Forhaling), men han er langmodig for eders Skyld, idet han ikke vil, at nogen skal fortabes, men at alle skulle komme til Omvendelse.
\par 10 Men Herrens Dag skal komme som en Tyv; på den skulle Himlene forgå med stort Bulder, og Elementerne skulle komme i Brand og opløses, og Jorden og alt, hvad der er på den, skal opbrændes.
\par 11 Efterdi da alt dette opløses, hvor bør I da ikke færdes i hellig Vandel og Gudsfrygt,
\par 12 idet I forvente og fremskynde Guds Dags Tilkommelse, for hvis Skyld Himle skulle antændes og opløses, og Elementer komme i Brand og smelte.
\par 13 Men vi forvente efter hans Forjættelse nye Himle og en ny Jord, i hvilke Retfærdighed bor.
\par 14 Derfor, I elskede! efterdi I forvente dette, så gører eder Flid for at findes uplettede og ulastelige for ham i Fred,
\par 15 og agter vor Herres Langmodighed for Frelse; ligesom også vor elskede Broder Paulus efter den ham givne Visdom har skrevet til eder,
\par 16 som han også gør i alle sine Breve, når han i dem taler om disse Ting; i dem findes der Ting, vanskelige at forstå, som de ukyndige og ubefæstede fordreje, ligesom også de øvrige Skrifter, til deres egen Undergang.
\par 17 Da I altså, I elskede! vide det forud, såvogter eder, for at I ikke skulle rives med af de ryggesløses Vildfarelse og affalde fra eders egen Fasthed;
\par 18 men vokser i vor Herres og Frelsers Jesu Kristi Nåde og Kundskab! Ham tilhører Herligheden både nu og indtil Evighedens Dag!



\end{document}