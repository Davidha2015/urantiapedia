\begin{document}

\title{Leviticus}



\chapter{1}

\par 1 En de HEERE riep Mozes, en sprak tot hem uit de tent der samenkomst, zeggende:
\par 2 Spreek tot de kinderen Israels, en zeg tot hen: Als een mens uit u den HEERE een offerande zal offeren, gij zult uw offeranden offeren van het vee, van runderen en van schapen.
\par 3 Indien zijn offerande een brandoffer van runderen is, zo zal hij een volkomen mannetje offeren; aan de deur van de tent der samenkomst zal hij dat offeren, naar zijn welgevallen, voor het aangezicht des HEEREN.
\par 4 En hij zal zijn hand op het hoofd des brandoffers leggen, opdat het voor hem aangenaam zij, om hem te verzoenen.
\par 5 Daarna zal hij het jonge rund slachten voor het aangezicht des HEEREN; en de zonen van Aaron, de priesters, zullen het bloed offeren, en het bloed sprengen rondom dat altaar, hetwelk voor de deur van de tent der samenkomst is.
\par 6 Dan zal hij het brandoffer de huid aftrekken, en het in zijn stukken delen.
\par 7 En de zonen van Aaron, den priester, zullen vuur maken op het altaar, en zullen het hout op het vuur schikken.
\par 8 Ook zullen de zonen van Aaron, de priesters, de stukken, het hoofd en het smeer, schikken op het hout, dat op het vuur is, hetwelk op het altaar is.
\par 9 Doch zijn ingewand, en zijn schenkelen zal men met water wassen; en de priester zal dat alles aansteken op het altaar; het is een brandoffer, een vuuroffer, tot een liefelijken reuk den HEERE.
\par 10 En indien zijn offerande is van klein vee, van schapen of van geiten, ten brandoffer, zal hij een volkomen mannetje offeren.
\par 11 En hij zal dat slachten aan de zijde van het altaar noordwaarts, voor het aangezicht des HEEREN; en de zonen van Aaron, de priesters, zullen zijn bloed rondom op het altaar sprengen.
\par 12 Daarna zal hij het in zijn stukken delen, mitsgaders zijn hoofd en zijn smeer; en de priester zal die schikken op het hout, dat op het vuur is, hetwelk op het altaar is.
\par 13 Doch het ingewand en de schenkelen zal men met water wassen; en de priester zal dat alles offeren en aansteken op het altaar; het is een brandoffer, een vuuroffer, tot een liefelijken reuk den HEERE.
\par 14 En indien zijn offerande voor den HEERE een brandoffer van gevogelte is, zo zal hij zijn offerande van tortelduiven, of van jonge duiven, offeren.
\par 15 En de priester zal die tot het altaar brengen, en deszelfs hoofd met zijn nagel splijten, en op het altaar aansteken; en zijn bloed zal aan den wand des altaars uitgeduwd worden.
\par 16 En zijn krop met zijn vederen zal hij wegdoen, en zal het werpen bij het altaar, oostwaarts, aan de plaats der as.
\par 17 Verder zal hij die met zijn vleugelen klieven, niet afscheiden; en de priester zal die aansteken op het altaar, op het hout, dat op het vuur is; het is een brandoffer, een vuuroffer, tot een liefelijken reuk den HEERE.

\chapter{2}

\par 1 Als nu een ziel een offerande van spijsoffer den HEERE zal offeren, zijn offerande zal van meelbloem zijn; en hij zal olie daarop gieten, en wierook daarop leggen.
\par 2 En hij zal het brengen tot de zonen van Aaron, de priesters, een van welke daarvan zijn hand vol grijpen zal uit deszelfs meelbloem, en uit deszelfs olie, met al deszelfs wierook; en de priester zal deszelfs gedenkoffer aansteken op het altaar; het is een vuuroffer, tot een liefelijken reuk den HEERE.
\par 3 Wat nu overblijft van het spijsoffer, zal voor Aaron en zijn zonen zijn; het is een heiligheid der heiligheden van de vuurofferen des HEEREN.
\par 4 En als gij offeren zult een offerande van spijsoffer, een gebak des ovens; het zullen zijn ongezuurde koeken van meelbloem, met olie gemengd, en ongezuurde vladen, met olie bestreken.
\par 5 En indien uw offerande spijsoffer is, in de pan gekookt, zij zal zijn van ongezuurde meelbloem, met olie gemengd.
\par 6 Breekt ze in stukken, en giet olie daarop; het is een spijsoffer.
\par 7 En zo uw offerande een spijsoffer des ketels is, het zal van meelbloem met olie gemaakt worden.
\par 8 Dan zult gij dat spijsoffer, hetwelk daarvan zal gemaakt worden, den HEERE toebrengen; en men zal het tot den priester doen naderen, die het tot het altaar dragen zal.
\par 9 En de priester zal van dat spijsoffer deszelfs gedenkoffer opnemen, en op het altaar aansteken, het is een vuuroffer, tot een liefelijken reuk den HEERE.
\par 10 En wat overblijft van het spijsoffer, zal voor Aaron en zijn zonen zijn; het is een heiligheid der heiligheden van de vuurofferen des HEEREN.
\par 11 Geen spijsoffer, dat gij den HEERE zult offeren, zal met desem gemaakt worden; want van geen zuurdesem, en van geen honig zult gijlieden den HEERE vuuroffer aansteken.
\par 12 De offeranden der eerstelingen zult gij den HEERE offeren; maar op het altaar zullen zij niet komen tot een liefelijken reuk.
\par 13 En alle offerande uws spijsoffers zult gij met zout zouten, en het zout des verbonds van uw God van uw spijsoffer niet laten afblijven; met al uw offerande zult gij zout offeren.
\par 14 En zo gij den HEERE een spijsoffer der eerste vruchten offert, zult gij het spijsoffer uwer eerste vruchten van groene aren, bij het vuur gedord, dat is, het klein gebroken graan van volle groene aren, offeren.
\par 15 En gij zult olie daarop doen, en wierook daarop leggen; het is een spijsoffer.
\par 16 Zo zal de priester deszelfs gedenkoffer aansteken van zijn klein gebroken graan en van zijn olie, met al den wierook; het is een vuuroffer den HEERE.

\chapter{3}

\par 1 En indien zijn offerande een dankoffer is; zo hij ze van de runderen offert, hetzij mannetje of wijfje, volkomen zal hij die offeren, voor het aangezicht des HEEREN.
\par 2 En hij zal zijn hand op het hoofd zijner offerande leggen, en zal ze slachten voor de deur van de tent der samenkomst; en de zonen van Aaron, de priesters, zullen het bloed rondom op het altaar sprengen.
\par 3 Daarna zal hij van dat dankoffer een vuuroffer den HEERE offeren; het vet, dat het ingewand bedekt, en al het vet, hetwelk aan het ingewand is.
\par 4 Dan zal hij beide de nieren, en het vet, hetwelk daaraan is, dat aan de weekdarmen is; en het net over de lever, met de nieren, zal hij afnemen.
\par 5 En de zonen van Aaron zullen dat aansteken op het altaar, op het brandoffer, hetwelk op het hout zal zijn, dat op het vuur is; het is een vuuroffer, tot een liefelijken reuk den HEERE.
\par 6 En indien zijn offerande van klein vee is, den HEERE tot een dankoffer, hetzij mannetje of wijfje, volkomen zal hij die offeren.
\par 7 Indien hij een lam tot zijn offerande offert, zo zal hij het offeren voor het aangezicht des HEEREN.
\par 8 En hij zal zijn hand op het hoofd zijner offerande leggen, en hij zal die slachten voor de tent der samenkomst; en de zonen van Aaron zullen het bloed daarvan sprengen op het altaar rondom.
\par 9 Daarna zal hij van dat dankoffer een vuuroffer den HEERE offeren; zijn vet, den gehelen staart, dien hij dicht aan de ruggegraat zal afnemen, en het vet bedekkende het ingewand, en al het vet, dat aan het ingewand is;
\par 10 Ook beide de nieren, en het vet, dat daaraan is, dat aan de weekdarmen is; en het net over de lever met de nieren, zal hij afnemen.
\par 11 En de priester zal dat aansteken op het altaar; het is een spijs des vuuroffers den HEERE.
\par 12 Indien nu zijn offerande een geit is, zo zal hij die offeren voor het aangezicht des HEEREN.
\par 13 En hij zal zijn hand op haar hoofd leggen, en hij zal haar slachten voor de tent der samenkomst; en de zonen van Aaron zullen haar bloed op het altaar sprengen rondom.
\par 14 Dan zal hij daarvan zijn offerande offeren, een vuuroffer den HEERE; het vet bedekkende het ingewand, en al het vet, dat aan het ingewand is;
\par 15 Mitsgaders de beide nieren, en het vet, dat daaraan is, dat aan de weekdarmen is; en het net over de lever, met de nieren, zal hij afnemen.
\par 16 En de priester zal die aansteken op het altaar; het is een spijs des vuuroffers, tot een liefelijken reuk; alle vet zal des HEEREN zijn.
\par 17 Dit zij een eeuwige inzetting voor uw geslachten, in al uw woningen: geen vet noch bloed zult gij eten.

\chapter{4}

\par 1 Verder sprak de HEERE tot Mozes, zeggende:
\par 2 Spreek tot de kinderen Israels, zeggende: Als een ziel zal gezondigd hebben, door afdwaling van enige geboden des HEEREN, dat niet zou gedaan worden, en tegen een van die zal gedaan hebben;
\par 3 Indien de priester, die gezalfd is, zal gezondigd hebben, tot schuld des volks, zo zal hij voor zijn zonde, die hij gezondigd heeft, offeren een var, een volkomen jong rund, den HEERE ten zondoffer.
\par 4 En hij zal dien var brengen tot de deur van de tent der samenkomst, voor het aangezicht des HEEREN; en hij zal zijn hand op het hoofd van dien var leggen, en hij zal dien var slachten voor het aangezicht des HEEREN.
\par 5 Daarna zal die gezalfde priester van het bloed van den var nemen, en hij zal dat tot de tent der samenkomst brengen.
\par 6 En de priester zal zijn vinger in dat bloed dopen; en van dat bloed zal hij zevenmaal sprengen voor het aangezicht des HEEREN, voor den voorhang van het heilige.
\par 7 Ook zal de priester van dat bloed doen op de hoornen des reukaltaars der welriekende specerijen, voor het aangezicht des HEEREN, dat in de tent der samenkomst is; dan zal hij al het bloed van den var uitgieten aan den bodem van het altaar des brandoffers, hetwelk is aan de deur van de tent der samenkomst.
\par 8 Verder, al het vet van den var des zondoffers zal hij daarvan opnemen; het vet bedekkende het ingewand, en al het vet, dat aan het ingewand is;
\par 9 Daartoe de twee nieren, en het vet, dat daaraan is, dat aan de weekdarmen is, en het net over de lever, met de nieren, zal hij afnemen;
\par 10 Gelijk als het van den os des dankoffers opgenomen wordt; en de priester zal die aansteken op het altaar des brandoffers.
\par 11 Maar de huid van dien var, en al zijn vlees, met zijn hoofd en met zijn schenkelen, en zijn ingewand, en zijn mest;
\par 12 En dien gehelen var zal hij tot buiten het leger uitvoeren, aan een reine plaats, waar men de as uitstort, en zal hem met vuur op het hout verbranden; bij de uitgegoten as zal hij verbrand worden.
\par 13 Indien nu de gehele vergadering van Israel afgedwaald zal zijn, en de zaak voor de ogen der gemeente verborgen is, en zij iets gedaan zullen hebben tegen enige van alle geboden des HEEREN, dat niet zoude gedaan worden, en zijn schuldig geworden;
\par 14 En die zonde, die zij daartegen gezondigd zullen hebben, bekend is geworden; zo zal de gemeente een var, een jong rund, ten zondoffer offeren, en dien voor de tent der samenkomst brengen;
\par 15 En de oudsten der vergadering zullen hun handen op het hoofd van den var leggen, voor het aangezicht des HEEREN; en hij zal den var slachten voor het aangezicht des HEEREN.
\par 16 Daarna zal die gezalfde priester van het bloed van den var tot de tent der samenkomst brengen.
\par 17 En de priester zal zijn vinger indopen, nemende van dat bloed; en hij zal zevenmaal sprengen voor het aangezicht des HEEREN, voor den voorhang.
\par 18 En van dat bloed zal hij doen op de hoornen van het altaar, dat voor het aangezicht des HEEREN is, dat in de tent der samenkomst is; dan zal hij al het bloed uitgieten, aan den bodem van het altaar des brandoffers, hetwelk is voor de deur van de tent der samenkomst.
\par 19 Daartoe zal hij al zijn vet van hem opnemen, en op het altaar aansteken.
\par 20 En hij zal dezen var doen, gelijk als hij den var des zondoffers gedaan heeft, alzo zal hij hem doen; en de priester zal voor hen verzoening doen, en het zal hun vergeven worden.
\par 21 Daarna zal hij dien var tot buiten het leger uitvoeren, en zal hem verbranden, gelijk als hij den eersten var verbrand heeft; het is een zondoffer der gemeente.
\par 22 Als een overste zal gezondigd hebben, en tegen een van de geboden des HEEREN zijns Gods, door afdwaling, gedaan zal hebben, hetwelk niet zou gedaan worden, zodat hij schuldig is;
\par 23 Of men zijn zonde, die hij daartegen gezondigd heeft, aan hem zal bekend gemaakt hebben; zo zal hij tot zijn offer brengen een geitenbok, een volkomen mannetje.
\par 24 En hij zal zijn hand op het hoofd van den bok leggen, en zal hem slachten in de plaats, waar men het brandoffer slacht voor het aangezicht des HEEREN; het is een zondoffer.
\par 25 Daarna zal de priester van het bloed des zondoffers met zijn vinger nemen, en dat op de hoornen van het altaar des brandoffers doen; dan zal hij zijn bloed aan den bodem van het altaar des brandoffers uitgieten.
\par 26 Hij zal ook al zijn vet op het altaar aansteken, gelijk het vet des dankoffers; zo zal de priester voor hem verzoening doen van zijn zonden, en het zal hem vergeven worden.
\par 27 En zo enig mens van het volk des lands door afdwaling zal gezondigd hebben, dewijl hij iets doet tegen een van de geboden des HEEREN, dat niet gedaan zou worden, zodat hij schuldig is;
\par 28 Of men zijn zonde, die hij gezondigd heeft, aan hem zal bekend gemaakt hebben; zo zal hij tot zijn offerande brengen een jonge geit, een volkomen wijfje, voor zijn zonde, die hij gezondigd heeft.
\par 29 En hij zal zijn hand op het hoofd des zondoffers leggen; en men zal dat zondoffer slachten in de plaats des brandoffers.
\par 30 Daarna zal de priester van haar bloed met zijn vinger nemen, en doen het op de hoornen van het altaar des brandoffers; dan zal hij al het bloed daarvan aan den bodem van dat altaar uitgieten.
\par 31 En al haar vet zal hij afnemen, gelijk als het vet van het dankoffer afgenomen wordt, en de priester zal het aansteken op het altaar, tot een liefelijken reuk den HEERE; en de priester zal voor hem verzoening doen, en het zal hem vergeven worden.
\par 32 Maar zo hij een lam voor zijn offerande ten zondoffer brengt, het zal een volkomen wijfje zijn, dat hij brengt.
\par 33 En hij zal zijn hand op het hoofd des zondoffers leggen, en hij zal dat slachten tot een zondoffer, in de plaats, waar men het brandoffer slacht.
\par 34 Daarna zal de priester van het bloed des zondoffers met zijn vinger nemen, en zal het doen op de hoornen van het altaar des brandoffers; dan zal hij al het bloed daarvan aan den bodem van dat altaar uitgieten.
\par 35 En al het vet daarvan zal hij afnemen, gelijk als het vet van het lam des dankoffers afgenomen wordt, en de priester zal die aansteken op het altaar, op de vuurofferen des HEEREN; en de priester zal voor hem verzoening doen over zijn zonde, die hij gezondigd heeft, en het zal hem vergeven worden.

\chapter{5}

\par 1 Als nu een mens zal gezondigd hebben, dat hij gehoord heeft een stem des vloeks, waarvan hij getuige is, hetzij dat hij het gezien of geweten heeft; indien hij het niet te kennen geeft, zo zal hij zijn ongerechtigheid dragen.
\par 2 Of wanneer een mens enig onrein ding zal aangeroerd hebben, hetzij het dode aas van een wild onrein gedierte, of het dode aas van onrein vee, of het dode aas van onrein kruipend gedierte; al is het voor hem verborgen geweest, nochtans is hij onrein en schuldig.
\par 3 Of als hij zal aangeroerd hebben de onreinigheid van een mens, naar al zijn onreinigheid, waarmede hij onrein wordt; en het is voor hem verborgen geweest, en hij is het gewaar geworden, zo is hij schuldig.
\par 4 Of als een mens zal gezworen hebben, onbedacht met zijn lippen uitsprekende, om kwaad te doen, of om goed te doen; naar al wat de mens in den eed onbedacht uitspreekt, en het is voor hem verborgen geweest, en hij is het gewaar geworden, zo is hij aan een van die schuldig.
\par 5 Het zal dan geschieden, als hij aan een van die schuldig is, dat hij belijden zal, waarin hij gezondigd heeft;
\par 6 En tot zijn schuldoffer den HEERE voor zijn zonde, die hij gezondigd heeft, brengen zal een wijfje van klein vee, een lam of een jonge geit, voor de zonde; zo zal de priester voor hem vanwege zijn zonde verzoening doen.
\par 7 Maar indien zijn hand zoveel niet bereiken kan, als genoeg is tot een stuk klein vee, zo zal hij tot zijn offer voor de schuld, die hij gezondigd heeft, den HEERE brengen twee tortelduiven, of twee jonge duiven, een ten zondoffer, en een ten brandoffer.
\par 8 En hij zal die tot den priester brengen, welke eerst die zal offeren, die tot het zondoffer is; en zal haar hoofd met zijn nagel nevens haar nek splijten, maar niet afscheiden.
\par 9 En van het bloed des zondoffers zal hij aan den wand van het altaar sprengen; maar het overgeblevene van dat bloed zal uitgeduwd worden aan den bodem van het altaar; het is een zondoffer.
\par 10 En de andere zal hij ten brandoffer maken, naar de wijze; zo zal de priester voor hem, vanwege zijn zonde, die hij gezondigd heeft, verzoening doen, en het zal hem vergeven worden.
\par 11 Maar indien zijn hand niet bereiken kan aan twee tortelduiven of twee jonge duiven, zo zal hij, die gezondigd heeft, tot zijn offerande brengen het tiende deel van een efa meelbloem ten zondoffer; hij zal geen olie daarover doen, noch wierook daarop leggen; want het is een zondoffer.
\par 12 En hij zal dat tot den priester brengen, en de priester zal daarvan zijn hand vol, der gedachtenis deszelven, grijpen, en dat aansteken op het altaar, op de vuurofferen des HEEREN; het is een zondoffer.
\par 13 Zo zal de priester voor hem verzoening doen over zijn zonde, die hij gezondigd heeft in enige van die stukken, en het zal hem vergeven worden; en het zal des priesters zijn, gelijk het spijsoffer.
\par 14 Wijders sprak de HEERE tot Mozes, zeggende:
\par 15 Als een mens door overtreding overtreden, en door afdwaling gezondigd zal hebben, wat onwetende van de heilige dingen des HEEREN, zo zal hij tot zijn schuldoffer den HEERE brengen een volkomen ram uit de kudde, met uw schatting aan zilveren sikkelen, naar den sikkel des heiligdoms, ten schuldoffer.
\par 16 Zo zal hij, dat hij zondigende heeft onwetend van de heilige dingen, wedergeven, en zal deszelfs vijfde deel daarenboven toedoen, dat hij den priester geven zal; alzo zal de priester met den ram des schuldoffers voor hem verzoening doen, en het zal hem vergeven worden.
\par 17 En indien een mens zal gezondigd hebben, en gedaan tegen een van alle geboden des HEEREN, hetwelk niet zou gedaan worden, al is het dat hij het niet geweten heeft, nochtans is hij schuldig, en zal zijn ongerechtigheid dragen.
\par 18 En hij zal een volkomen ram uit de kudde tot den priester brengen, met uw schatting, ten schuldoffer; en de priester zal voor hem verzoening doen over zijn afdwaling, door welke hij afgedwaald is, die hij niet geweten had; zo zal het hem vergeven worden.
\par 19 Het is een schuldoffer; hij heeft zich voorzeker schuldig gemaakt aan den HEERE.

\chapter{6}

\par 1 Verder sprak de HEERE tot Mozes, zeggende:
\par 2 Als een mens gezondigd, en tegen den HEERE door overtreding overtreden zal hebben, dat hij aan zijn naaste zal gelogen hebben van hetgeen hem in bewaring gegeven, of ter hand gesteld was, of van roof, of dat hij met geweld zijn naaste onthoudt;
\par 3 Of dat hij het verlorene gevonden, en daarover gelogen, en met valsheid gezworen zal hebben; over iets van alles, dat de mens doet, daarin zondigende.
\par 4 Het zal dan geschieden, dewijl hij gezondigd heeft, en schuldig geworden is, dat hij wederuitkeren zal den roof, dien hij geroofd, of het onthoudene, dat hij met geweld onthoudt, of het bewaarde, dat bij hem te bewaren gegeven was, of het verlorene, dat hij gevonden heeft;
\par 5 Of van al, waarover hij valselijk gezworen heeft, dat hij hetzelve in zijn hoofdsom wedergeve, en nog het vijfde deel daarenboven toedoen zal; wiens dat is, dien zal hij dat geven op den dag zijner schuld.
\par 6 En hij zal den HEERE zijn schuldoffer brengen tot den priester, een volkomen ram uit de kudde, met uw schatting, ten schuldoffer.
\par 7 Dan zal de priester voor hem verzoening doen voor het aangezicht des HEEREN, en het zal hem vergeven worden; over iets van al, wat hij doet, waar hij schuld aan heeft.
\par 8 Verder sprak de HEERE tot Mozes, zeggende:
\par 9 Gebied Aaron en zijn zonen, zeggende: Dit is de wet des brandoffers; het is hetgeen, wat door de branding op het altaar den gansen nacht tot aan den morgen opvaart; alwaar het vuur des altaars zal brandende gehouden worden.
\par 10 En de priester zal zijn linnen kleed aantrekken, en de linnen onderbroek over zijn vlees aantrekken, en zal de as opnemen, als het vuur het brandoffer op het altaar zal verteerd hebben, en zal die bij het altaar leggen.
\par 11 Daarna zal hij zijn klederen uittrekken, en zal andere klederen aandoen, en zal de as tot buiten het leger uitdragen aan een reine plaats.
\par 12 Het vuur nu op het altaar zal daarop brandende gehouden worden, het zal niet uitgeblust worden; maar de priester zal daar elken morgen hout aansteken, en zal daarop het brandoffer schikken, en het vet der dankofferen daarop aansteken.
\par 13 Het vuur zal geduriglijk op het altaar brandende gehouden worden; het zal niet uitgeblust worden.
\par 14 Dit is nu de wet des spijsoffers; een der zonen van Aaron zal dat voor het aangezicht des HEEREN offeren, voor het altaar.
\par 15 En hij zal daarvan opnemen zijn hand vol, uit de meelbloem des spijsoffers, en van deszelfs olie, en al den wierook, die op het spijsoffer is; dan zal hij het aansteken op het altaar; het is een liefelijke reuk tot deszelfs gedachtenis voor den HEERE.
\par 16 En het overblijvende daarvan zullen Aaron en zijn zonen eten; ongezuurd zal het gegeten worden in de heilige plaats; in den voorhof van de tent der samenkomst zullen zij dat eten.
\par 17 Het zal niet gedesemd gebakken worden; het is hun deel, dat Ik gegeven heb van Mijn vuurofferen; het is een heiligheid der heiligheden, gelijk het zondoffer en gelijk het schuldoffer.
\par 18 Al wat mannelijk is onder de zonen van Aaron zal het eten; het zij een eeuwige inzetting voor uw geslachten van de vuurofferen des HEEREN; al wat die zal aanroeren, zal heilig zijn.
\par 19 Wijders sprak de HEERE tot Mozes, zeggende:
\par 20 Dit is de offerande van Aaron en van zijn zonen, die zij den HEERE offeren zullen, ten dage als hij zal gezalfd worden: het tiende deel ener efa meelbloem, een spijsoffer gedurig; de helft daarvan op den morgen, en de helft daarvan op den avond.
\par 21 Het zal in een pan met olie gemaakt worden; geroost zult gij het brengen; en de gebakken stukken des spijsoffers zult gij offeren, tot een liefelijken reuk den HEERE.
\par 22 Ook zal de priester, die uit zijn zonen in zijn plaats de gezalfde zal worden, hetzelfde doen; het zij een eeuwige inzetting; het zal voor den HEERE geheel aangestoken worden.
\par 23 Alzo zal alle spijsoffer des priesters ganselijk zijn; het zal niet gegeten worden.
\par 24 Verder sprak de HEERE tot Mozes, zeggende:
\par 25 Spreek tot Aaron en tot zijn zonen, zeggende: Dit is de wet des zondoffers: in de plaats, waar het brandoffer geslacht wordt, zal het zondoffer voor het aangezicht des HEEREN geslacht worden; het is een heiligheid der heiligheden.
\par 26 De priester, die het voor de zonde offert, zal het eten; in de heilige plaats zal het gegeten worden, in den voorhof van de tent der samenkomst.
\par 27 Al wat deszelfs vlees zal aanroeren, zal heilig zijn; zo wie van zijn bloed op een kleed zal gesprengd hebben, dat, waarop hij gesprengd zal hebben, zult gij in de heilige plaats wassen.
\par 28 En het aarden vat, waarin het gezoden is, zal gebroken worden; maar zo het in een koperen vat gezoden is, zo zal het geschuurd en in water gespoeld worden.
\par 29 Al wat mannelijk is onder de priesteren, zal dat eten; het is een heiligheid der heiligheden.
\par 30 Maar geen zondoffer, van welks bloed in de tent der samenkomst zal gebracht worden, om in het heiligdom te verzoenen, zal gegeten worden; het zal in het vuur verbrand worden.

\chapter{7}

\par 1 Dit is nu de wet des schuldoffers; het is een heiligheid der heiligheden.
\par 2 In de plaats, waar zij het brandoffer slachten, zullen zij het schuldoffer slachten; en men zal deszelfs bloed rondom op het altaar sprengen.
\par 3 En daarvan zal men al zijn vet offeren, den staart, en het vet, dat het ingewand bedekt;
\par 4 Ook de beide nieren, en het vet, dat daaraan is, dat op de weekdarmen is; en het net over de lever, met de nieren, zal men afnemen.
\par 5 En de priester zal die aansteken op het altaar, ten vuuroffer den HEERE; het is een schuldoffer.
\par 6 Al wat mannelijk is onder de priesteren zal dat eten; in de heilige plaats zal het gegeten worden; het is een heiligheid der heiligheden.
\par 7 Gelijk het zondoffer, alzo zal ook het schuldoffer zijn; enerlei wet zal voor dezelve zijn; het zal des priesters zijn, die daarmede verzoening gedaan zal hebben.
\par 8 Ook de priester, die iemands brandoffer offert, die priester zal de huid des brandoffers hebben, dat hij geofferd heeft.
\par 9 Daartoe al het spijsoffer, dat in den oven gebakken wordt, met al wat in den ketel en in den pan bereid wordt, zal des priesters zijn, die dat offert.
\par 10 Ook alle spijsoffer met olie gemengd, of droog, zal voor alle zonen van Aaron zijn, voor den enen als voor den anderen.
\par 11 Dit is nu de wet des dankoffers, dat men den HEERE offeren zal.
\par 12 Indien hij dat tot een lof offer offert, zo zal hij, nevens het lofoffer, ongezuurde koeken met olie gemengd, en ongezuurde vladen met olie bestreken, offeren; en zullen die koeken met olie gemengd van geroost meelbloem zijn.
\par 13 Benevens de koeken zal hij tot zijn offerande gedesemd brood offeren, met het lofoffer zijns dankoffers.
\par 14 En een daarvan uit de ganse offerande zal hij den HEERE ten hefoffer offeren; het zal voor den priester zijn, die het bloed des dankoffers sprengt.
\par 15 Maar het vlees van het lofoffer zijns dankoffers zal op den dag van deszelfs offerande gegeten worden; daarvan zal men niet tot den morgen overlaten.
\par 16 En zo het slachtoffer zijner offerande een gelofte, of vrijwillig offer is, dat zal ten dage als hij zijn offer offeren zal, gegeten worden, en het overgeblevene daarvan zal ook des anderen daags gegeten worden.
\par 17 Wat nog van het vlees des slachtoffers overgebleven is, zal op den derden dag met vuur verbrand worden;
\par 18 Want zo enigszins van dat vlees zijns dankoffers op den derden dag gegeten wordt, die dat geofferd heeft, zal niet aangenaam zijn; het zal hem niet toegerekend worden, het zal een afgrijselijk ding zijn; en de ziel, die daarvan eet, zal haar ongerechtigheid dragen.
\par 19 En het vlees, dat iets onreins aangeroerd zal hebben, zal niet gegeten worden; met vuur zal het verbrand worden; maar aangaande het andere vlees, dat vlees zal een ieder, die rein is, mogen eten.
\par 20 Doch als een ziel het vlees van het dankoffer, hetwelk des HEEREN is, gegeten zal hebben, en haar onreinigheid aan haar is, zo zal die ziel uit haar volken uitgeroeid worden.
\par 21 En wanneer een ziel iets onreins zal aangeroerd hebben, als de onreinigheid des mensen, of het onreine vee, of enig onrein verfoeisel, en zal van het vlees des dankoffers, hetwelk des HEEREN is, gegeten hebben, zo zal die ziel uit haar volken uitgeroeid worden.
\par 22 Daarna sprak de HEERE tot Mozes, zeggende:
\par 23 Spreek tot de kinderen Israels, zeggende: Geen vet van een os, of schaap, of geit, zult gij eten.
\par 24 Maar het vet van een dood aas, en het vet van het verscheurde, mag tot alle werk gebezigd worden; doch gij zult het ganselijk niet eten.
\par 25 Want al wie het vet van vee eten zal, van hetwelk men den HEERE een vuuroffer zal geofferd hebben, die ziel, die het gegeten zal hebben, zal uit haar volken uitgeroeid worden.
\par 26 Ook zult gij in uw woningen geen bloed eten, hetzij van het gevogelte, of van het vee.
\par 27 Alle ziel, die enig bloed eten zal, die ziel zal uit haar volken uitgeroeid worden.
\par 28 Voorts sprak de HEERE tot Mozes, zeggende:
\par 29 Spreek tot de kinderen Israels, zeggende: Wie zijn dankoffer den HEERE offert, zal zijn offerande van zijn dankoffer den HEERE toebrengen.
\par 30 Zijn handen zullen de vuurofferen des HEEREN brengen; het vet aan de borst zal hij met die borst brengen, om die tot een beweegoffer voor het aangezicht des HEEREN te bewegen.
\par 31 En de priester zal dat vet op het altaar aansteken; doch de borst zal voor Aaron en zijn zonen zijn.
\par 32 Gij zult ook den rechterschouder tot een hefoffer den priester geven, uit uw dankofferen.
\par 33 Wie uit de zonen van Aaron het bloed des dankoffers en het vet offert, dien zal de rechterschouder ten dele zijn.
\par 34 Want de beweegborst en den hefschouder heb Ik van de kinderen Israels uit hun dankofferen genomen, en heb dezelve aan Aaron, den priester, en aan zijn zonen, tot een eeuwige inzetting gegeven, van de kinderen Israels.
\par 35 Dit is de zalving van Aaron en de zalving van zijn zonen, van de vuurofferen des HEEREN; ten dage als Hij hen deed naderen, om het priesterdom den HEERE te bedienen;
\par 36 Hetwelk de HEERE hun van de kinderen Israels te geven geboden heeft, ten dage als Hij hen zalfde; het zij een eeuwige inzetting voor hun geslachten.
\par 37 Dit is de wet des brandoffers, des spijsoffers, des zondoffers, des schuldoffers, des vuloffers en des dankoffers;
\par 38 Die de HEERE Mozes op den berg Sinai geboden heeft, ten dage als Hij den kinderen Israels gebood, dat zij hun offeranden den HEERE, in de woestijn van Sinai, zouden offeren.

\chapter{8}

\par 1 Verder sprak de HEERE tot Mozes, zeggende:
\par 2 Neem Aaron en zijn zonen met hem, en de klederen, en de zalfolie, daartoe den var des zondoffers, en de twee rammen, en den korf van de ongezuurde broden;
\par 3 En verzamel de ganse vergadering aan de deur van de tent der samenkomst.
\par 4 Mozes nu deed, gelijk als de HEERE hem geboden had; en de vergadering werd verzameld aan de deur van de tent der samenkomst.
\par 5 Toen zeide Mozes tot de vergadering: Dit is de zaak, die de HEERE geboden heeft te doen.
\par 6 En Mozes deed Aaron en zijn zonen naderen, en wies hen met dat water.
\par 7 Daar deed hij hem den rok aan, en gordde hem met den gordel, en trok hem den mantel aan; en deed hij hem den efod aan, en gordde dien met den kunstelijken riem des efods, en ombond hem daarmede.
\par 8 Voorts deed hij hem den borstlap aan, en voegde aan den borstlap de Urim en de Thummim.
\par 9 En hij zette den hoed op zijn hoofd; en aan den hoed boven zijn aangezicht zette hij de gouden plaat, de kroon der heiligheid, gelijk als de HEERE Mozes geboden had.
\par 10 Toen nam Mozes de zalfolie, en zalfde den tabernakel, en al wat daarin was, en heiligde ze.
\par 11 En hij sprengde daarvan op het altaar zevenmaal; en hij zalfde het altaar, en al zijn gereedschap, mitsgaders het wasvat en zijn voet, om die te heiligen.
\par 12 Daarna goot hij van de zalfolie op het hoofd van Aaron, en hij zalfde hem, om hem te heiligen.
\par 13 Ook deed Mozes de zonen van Aaron naderen, en trok hun rokken aan, en gordde hen met een gordel, en bond hun mutsen op, gelijk als de HEERE Mozes geboden had.
\par 14 Toen deed hij den var des zondoffers bijeenkomen; en Aaron en zijn zonen leiden hun handen op het hoofd van den var des zondoffers;
\par 15 En men slachtte hem; en Mozes nam het bloed, en deed het met zijn vinger rondom op de hoornen des altaars, en ontzondigde het altaar; daarna goot hij het bloed uit aan den bodem des altaars, en heiligde het, om voor hetzelve verzoening te doen.
\par 16 Voorts nam hij al het vet, dat aan het ingewand is, en het net der lever, en de twee nieren en haar vet; en Mozes stak het aan op het altaar.
\par 17 Maar den var met zijn huid, en zijn vlees, en zijn mest, heeft hij buiten het leger met vuur verbrand, gelijk als de HEERE Mozes geboden had.
\par 18 Daarna deed hij den ram des brandoffers bijbrengen; en Aaron en zijn zonen leiden hun handen op het hoofd van den ram.
\par 19 En men slachtte hem; en Mozes sprengde het bloed op het altaar rondom.
\par 20 Hij deelde ook den ram in zijn delen; en Mozes stak het hoofd aan, en die delen, en het smeer;
\par 21 Doch het ingewand en de schenkelen wies hij met water; en Mozes stak dien gehelen ram aan op het altaar; het was een brandoffer tot een liefelijken reuk, een vuuroffer was het den HEERE, gelijk als de HEERE Mozes geboden had.
\par 22 Daarna deed hij den anderen ram, den ram des vuloffers, bijbrengen; en Aaron met zijn zonen leiden hun handen op het hoofd van den ram.
\par 23 En men slachtte hem; en Mozes nam van zijn bloed, en deed het op het lapje van Aarons rechteroor, en op den duim zijner rechterhand, en op den groten teen van zijn rechtervoet.
\par 24 Hij deed ook de zonen van Aaron naderen; en Mozes deed van dat bloed op het lapje van hun rechteroor, en op den duim van hun rechterhand, en op den groten teen van hun rechtervoet; daarna sprengde Mozes dat bloed rondom op het altaar.
\par 25 En hij nam het vet, en den staart, en al het vet, dat aan het ingewand is, en het net der lever, en de beide nieren, en haar vet, daartoe den rechterschouder.
\par 26 Ook nam hij uit den korf van de ongezuurde broden, die voor het aangezicht des HEEREN was, een ongezuurde koek, en een geolieden broodkoek, en een vlade; en hij leide ze op dat vet, en op den rechterschouder.
\par 27 En hij gaf dat alles in de handen van Aaron, en in de handen zijner zonen; en bewoog die ten beweegoffer, voor het aangezicht des HEEREN.
\par 28 Daarna nam Mozes ze uit hun handen, en stak ze aan op het altaar, op het brandoffer; zij waren vulofferen tot een liefelijken reuk; het was een vuuroffer den HEERE.
\par 29 Voorts nam Mozes de borst, en bewoog ze ten beweegoffer voor het aangezicht des HEEREN; zij werd Mozes ten dele van den ram des vuloffers, gelijk als de HEERE Mozes geboden had.
\par 30 Mozes nam ook van de zalfolie, en van het bloed, hetwelk op het altaar was, en sprengde het op Aaron, op zijn klederen, en op zijn zonen, en op de klederen zijner zonen met hem; en hij heiligde Aaron, zijn klederen, en zijn zonen, en de klederen zijner zonen met hem.
\par 31 En Mozes zeide tot Aaron en tot zijn zonen: Ziedt dat vlees voor de deur van de tent der samenkomst, en eet hetzelve daar, mitsgaders het brood, dat in den korf des vuloffers is; gelijk als ik geboden heb, zeggende: Aaron en zijn zonen zullen dat eten.
\par 32 Maar het overige van het vlees en van het brood zult gij met vuur verbranden.
\par 33 Ook zult gij uit de deur van de tent der samenkomst, zeven dagen, niet uitgaan, tot aan den dag, dat vervuld worden de dagen uws vuloffers; want zeven dagen zal men uw handen vullen.
\par 34 Gelijk men gedaan heeft op dezen dag, heeft de HEERE te doen geboden, om voor u verzoening te doen.
\par 35 Gij zult dan aan de deur van de tent der samenkomst blijven, dag en nacht, zeven dagen, en zult de wacht des HEEREN waarnemen, opdat gij niet sterft; want alzo is het mij geboden.
\par 36 Aaron nu en zijn zonen deden al de dingen, die de HEERE door den dienst van Mozes geboden had.

\chapter{9}

\par 1 En het geschiedde op den achtsten dag, dat Mozes riep Aaron en zijn zonen, en de oudsten van Israel;
\par 2 En hij zeide tot Aaron: Neem u een kalf, een jong rund, ten zondoffer, en een ram ten brandoffer, die volkomen zijn; en breng ze voor het aangezicht des HEEREN.
\par 3 Daarna spreek tot de kinderen Israels, zeggende: Neemt een geitenbok ten zondoffer, en een kalf, en een lam, eenjarig, volkomen, ten brandoffer;
\par 4 Ook een os en ram ten dankoffer, om voor het aangezicht des HEEREN te offeren; en spijsoffer met olie gemengd; want heden zal de HEERE u verschijnen.
\par 5 Toen namen zij hetgeen Mozes geboden had, brengende dat tot voor aan de tent der samenkomst; en de gehele vergadering naderde, en stond voor het aangezicht des HEEREN.
\par 6 En Mozes zeide: Deze zaak, die de HEERE geboden heeft, zult gij doen; en de heerlijkheid des HEEREN zal u verschijnen.
\par 7 En Mozes zeide tot Aaron: Nader tot het altaar, en maak uw zondoffer, en uw brandoffer toe; en doe verzoening voor u en voor het volk; maak daarna de offerande des volks toe, en doe de verzoening voor hen, gelijk als de HEERE geboden heeft.
\par 8 Toen naderde Aaron tot het altaar, en slachtte het kalf des zondoffers, dat voor hem was.
\par 9 En de zonen van Aaron brachten het bloed tot hem, en hij doopte zijn vinger in dat bloed, en deed het op de hoornen des altaars; daarna goot hij het bloed uit aan den bodem des altaars.
\par 10 Maar het vet, en de nieren, en het net van de lever van het zondoffer heeft hij op het altaar aangestoken, gelijk als de HEERE Mozes geboden had.
\par 11 Doch het vlees, en de huid verbrandde hij met vuur buiten het leger.
\par 12 Daarna slachtte hij het brandoffer; en de zonen van Aaron leverden aan hem het bloed; en hij sprengde dat rondom op het altaar.
\par 13 Ook leverden zij aan hem het brandoffer in zijn stukken, met het hoofd; en hij stak het aan op het altaar.
\par 14 En hij wies het ingewand en de schenkelen; en hij stak ze aan op het brandoffer, op het altaar.
\par 15 Daarna deed hij de offerande des volks toebrengen; en nam den bok des zondoffers, die voor het volk was, en slachtte hem, en bereidde hem ten zondoffer, gelijk het eerste.
\par 16 Verder deed hij het brandoffer toebrengen, en maakte dat toe naar het recht.
\par 17 En hij deed het spijsoffer toebrengen, en vulde daarvan zijn hand, en stak het aan op het altaar, behalve het morgenbrandoffer.
\par 18 Daarna slachtte hij den os, en den ram ten dankoffer, dat voor het volk was; en de zonen van Aaron leverden het bloed aan hem, hetwelk hij rondom op het altaar sprengde;
\par 19 En het vet van den os, en van den ram, den staart, en wat het ingewand bedekt, en de nieren, en het net der lever;
\par 20 En zij leiden het vet op de borsten; en hij stak dat vet aan op het altaar.
\par 21 Maar de borsten en den rechterschouder bewoog Aaron ten beweegoffer voor het aangezicht des HEEREN, gelijk als Mozes geboden had.
\par 22 Daarna hief Aaron zijn handen op tot het volk, en zegende hen; en hij kwam af, nadat hij het zondoffer, en brandoffer, en dankoffer gedaan had.
\par 23 Toen ging Mozes met Aaron in de tent der samenkomst; daarna kwamen zij uit, en zegenden het volk; en de heerlijkheid des HEEREN verscheen al het volk.
\par 24 Want een vuur ging uit van het aangezicht des HEEREN, en verteerde op het altaar het brandoffer, en het vet. Als het ganse volk dit zag, zo juichten zij, en vielen op hun aangezichten.

\chapter{10}

\par 1 En de zonen van Aaron, Nadab en Abihu, namen een ieder zijn wierookvat, en deden vuur daarin, en leiden reukwerk daarop, en brachten vreemd vuur voor het aangezicht des HEEREN, hetwelk hij hen niet geboden had.
\par 2 Toen ging een vuur uit van het aangezicht des HEEREN, en verteerde hen; en zij stierven voor het aangezicht des HEEREN.
\par 3 En Mozes zeide tot Aaron: Dat is het, wat de HEERE gesproken heeft, zeggende: In degenen, die tot Mij naderen, zal Ik geheiligd worden, en voor het aangezicht van al het volk zal Ik verheerlijkt worden. Doch Aaron zweeg stil.
\par 4 En Mozes riep Misael en Elzafan, de zonen van Uzziel, den oom van Aaron, en zeide tot hen: Treedt toe, draagt uw broederen weg, van voor het heiligdom tot buiten het leger.
\par 5 Toen traden zij toe, en droegen hen, in hun rokken, tot buiten het leger, gelijk als Mozes gesproken had.
\par 6 En Mozes zeide tot Aaron, en tot Eleazar, en tot Ithamar, zijn zonen: Gij zult uw hoofden niet ontbloten, noch uw klederen verscheuren, opdat gij niet sterft, en grote toorn over de ganse vergadering kome; maar uw broederen, het ganse huis van Israel, zullen dezen brand, dien de HEERE aan gestoken heeft, bewenen.
\par 7 Gij zult ook uit de deur van de tent der samenkomst niet uitgaan, opdat gij niet sterft; want de zalfolie des HEEREN is op u. En zij deden naar het woord van Mozes.
\par 8 En de HEERE sprak tot Aaron, zeggende:
\par 9 Wijn en sterken drank zult gij niet drinken, gij, noch uw zonen met u, als gij gaan zult in de tent der samenkomst, opdat gij niet sterft; het zij een eeuwige inzetting onder uw geslachten;
\par 10 En om onderscheid te maken tussen het heilige en tussen het onheilige, en tussen het onreine en tussen het reine;
\par 11 En om den kinderen Israels te leren al de inzettingen, die de HEERE door den dienst van Mozes tot hen gesproken heeft.
\par 12 En Mozes sprak tot Aaron, en tot Eleazar, en tot Ithamar, zijn overgebleven zonen: Neemt het spijsoffer, dat van de vuurofferen des HEEREN overgebleven is, en eet hetzelve ongezuurd bij het altaar; want het is een heiligheid der heiligheden.
\par 13 Daarom zult gij dat eten in de heilige plaats, dewijl het uw bescheiden deel en het bescheiden deel uwer zonen uit des HEEREN vuurofferen is; want alzo is mij geboden.
\par 14 Ook de beweegborst en den hefschouder zult gij in een reine plaats eten, gij, en uw zonen, en uw dochteren met u; want tot uw bescheiden deel, en uwer zonen bescheiden deel, zijn zij uit de dankofferen der kinderen Israels gegeven.
\par 15 Den hefschouder en de beweegborst zullen zij nevens de vuurofferen des vets toebrengen, om ten beweegoffer voor het aangezicht des HEEREN te bewegen; hetwelk, voor u en uw zonen met u, tot een eeuwige inzetting zijn zal, gelijk als de HEERE geboden heeft.
\par 16 En Mozes zocht zeer naarstiglijk den bok des zondoffers; en ziet, hij was verbrand. Dies was hij op Eleazar en op Ithamar, de overgebleven zonen van Aaron, zeer toornig, zeggende:
\par 17 Waarom hebt gij dat zondoffer niet gegeten in de heilige plaats? Want het is een heiligheid der heiligheden, en Hij heeft u dat gegeven, opdat gij de ongerechtigheid der vergadering zoudt dragen, om over die verzoening te doen voor het aangezicht des HEEREN.
\par 18 Ziet, deszelfs bloed is niet binnen in het heiligdom gedragen; gij moest dat ganselijk gegeten hebben in het heiligdom, gelijk als ik geboden heb.
\par 19 Toen sprak Aaron tot Mozes: Zie, heden hebben zij hun zondoffer en hun brandoffer voor het aangezicht des HEEREN geofferd, en zulke dingen zijn mij wedervaren; en had ik heden het zondoffer gegeten, zou dat goed geweest zijn in de ogen des HEEREN?
\par 20 Als Mozes dit hoorde, zo was het goed in zijn ogen.

\chapter{11}

\par 1 En de HEERE sprak tot Mozes en tot Aaron, zeggende tot hen:
\par 2 Spreekt tot de kinderen Israels, zeggende: Dit is het gedierte, dat gij eten zult uit alle beesten, die op de aarde zijn.
\par 3 Al wat onder de beesten den klauw verdeelt, en de kloof der klauwen in tweeen klieft, en herkauwt, dat zult gij eten.
\par 4 Deze nochtans zult gij niet eten, van degenen, die alleen herkauwen, of de klauwen alleen verdelen: den kemel, want hij herkauwt wel, maar verdeelt den klauw niet; die zal u onrein zijn;
\par 5 En het konijntje, want het herkauwt wel, maar verdeelt den klauw niet; dat zal u onrein zijn;
\par 6 En den haas, want hij herkauwt wel, maar verdeelt den klauw niet; die zal u onrein zijn.
\par 7 Ook het zwijn, want dat verdeelt wel den klauw, en klieft de klove der klauwen in tweeen, maar herkauwt het gekauwde niet; dat zal u onrein zijn.
\par 8 Van hun vlees zult gij niet eten, en hun dood aas niet aanroeren, zij zullen u onrein zijn.
\par 9 Dit zult gij eten van al wat in de wateren is: al wat in de wateren, in de zeeen en in de rivieren, vinnen en schubben heeft, dat zult gij eten;
\par 10 Maar al wat in de zeeen en in de rivieren, van alle gewemel der wateren, en van alle levende ziel, die in de wateren is, geen vinnen of schubben heeft, dat zal u een verfoeisel zijn.
\par 11 Ja, een verfoeisel zullen zij u zijn; van hun vlees zult gij niet eten, en hun dood aas zult gij verfoeien.
\par 12 Al wat in de wateren geen vinnen en schubben heeft, dat zal u een verfoeisel zijn.
\par 13 En van het gevogelte zult gij deze verfoeien, zij zullen niet gegeten worden, zij zullen een verfoeisel zijn: de arend, en de havik, en de zeearend,
\par 14 En de gier, en de kraai, naar haar aard;
\par 15 Alle rave naar haar aard;
\par 16 En de struis, en de nachtuil, en de koekoek, en de sperwer naar zijn aard;
\par 17 En de steenuil, en het duikertje, en de schuifuit,
\par 18 En de kauw, en de roerdomp, en de pelikaan,
\par 19 En de ooievaar, de reiger naar zijn aard, en de hop, en de vledermuis.
\par 20 Alle kruipend gevogelte, dat op vier voeten gaat, zal u een verfoeisel zijn.
\par 21 Dit nochtans zult gij eten van al het kruipend gevogelte, dat op vier voeten gaat, hetwelk boven aan zijn voeten schenkelen heeft, om daarmede op de aarde te springen;
\par 22 Van die zult gij deze eten: den sprinkhaan naar zijn aard, en den solham naar zijn aard, en den hargol naar zijn aard, en den hagab naar zijn aard.
\par 23 En alle kruipend gevogelte, dat vier voeten heeft, zal u een verfoeisel zijn.
\par 24 En aan deze zult gij verontreinigd worden; zo wie hun dood aas zal aangeroerd hebben, zal onrein zijn tot aan den avond.
\par 25 Zo wie van hun dood aas gedragen zal hebben, zal zijn klederen wassen, en onrein zijn tot aan den avond.
\par 26 Alle beest, dat den klauw verdeelt, doch de klove niet in tweeen klieft, en niet herkauwt, zal u onrein zijn; zo wie hetzelve aangeroerd zal hebben, zal onrein zijn.
\par 27 En al wat op zijn poten gaat onder alle gedierte, op vier voeten gaande, die zullen u onrein zijn; al wie hun dood aas aangeroerd zal hebben, zal onrein zijn tot aan den avond.
\par 28 Ook die hun dood aas zal gedragen hebben, zal zijn klederen wassen, en onrein zijn tot aan den avond; zij zullen u onrein zijn.
\par 29 Verder zal u dit onder het kruipend gedierte, dat op de aarde kruipt, onrein zijn: het wezeltje, en de muis, en de schildpad, naar haar aard;
\par 30 En de zwijnegel, en de krokodil, en de hagedis, en de slak, en de mol;
\par 31 Die zullen u onrein zijn onder alle kruipend gedierte; zo wie die zal aangeroerd hebben, als zij dood zijn, zal onrein zijn tot aan den avond.
\par 32 Daartoe al hetgeen, waarop iets van dezelve vallen zal, als zij dood zijn, zal onrein zijn, hetzij van alle houten vat, of kleed, of vel, of zak, of alle vat, waarmede enig werk gedaan wordt; het zal in het water gestoken worden, en onrein zijn tot aan den avond; daarna zal het rein zijn.
\par 33 En alle aarden vat, waarin iets van dezelve zal gevallen zijn, al wat daarin is, zal onrein zijn, en gij zult dat breken.
\par 34 Van alle spijze, die men eet, waarop het water zal gekomen zijn, die zal onrein zijn; en alle drank, die men drinkt, zal in alle vat onrein zijn.
\par 35 En waarop iets van hun dood aas zal vallen, zal onrein zijn; de oven en de aarden pan zal verbroken worden; zij zijn onrein, daarom zullen zij u onrein zijn.
\par 36 Doch een fontein, of put van vergadering der wateren, zal rein zijn; maar wie hun dood aas zal aangeroerd hebben, zal onrein zijn.
\par 37 En wanneer van hun dood aas zal gevallen zijn op enig zaaibaar zaad, dat gezaaid wordt, dat zal rein zijn.
\par 38 Maar als water op het zaad gedaan zal worden, en van hun dood aas daarop zal gevallen zijn, dat zal u onrein zijn.
\par 39 En wanneer van de dieren, die u tot spijze zijn, iets zal gestorven zijn, wie deszelfs dood aas zal aangeroerd hebben, zal onrein zijn tot aan den avond.
\par 40 Ook die van hun dood aas gegeten zal hebben, zal zijn klederen wassen, en onrein zijn tot aan den avond; en die hun dood aas zal gedragen hebben, zal zijn klederen wassen, en onrein zijn tot aan den avond.
\par 41 Voorts alle kruipend gedierte, dat op de aarde kruipt, zal een verfoeisel zijn; het zal niet gegeten worden.
\par 42 Al wat op zijn buik gaat, en al wat gaat op zijn vier voeten, of al wat vele voeten heeft, onder alle kruipend gedierte, dat op de aarde kruipt, die zult gij niet eten, want zij zijn een verfoeisel.
\par 43 Maakt uw zielen niet verfoeilijk aan enig kruipend gedierte, dat kruipt; en verontreinigt u niet daaraan, dat gij daaraan verontreinigd zoudt worden.
\par 44 Want Ik ben de HEERE, uw God; daarom zult gij u heiligen, en heilig zijn, dewijl Ik heilig ben; en gij zult uw ziel niet verontreinigen aan enig kruipend gedierte, dat zich op de aarde roert.
\par 45 Want Ik ben de HEERE, die u uit Egypteland doe optrekken, opdat Ik u tot een God zij, en opdat gij heilig zijt, dewijl Ik heilig ben.
\par 46 Dit is de wet van de beesten, en van het gevogelte, en van alle levende ziel, die zich roert in de wateren, en van alle ziel, die kruipt op de aarde;
\par 47 Om te onderscheiden tussen het onreine en tussen het reine, en tussen het gedierte, dat men eten, en tussen het gedierte, dat men niet eten zal.

\chapter{12}

\par 1 Verder sprak de HEERE tot Mozes, zeggende:
\par 2 Spreek tot de kinderen Israels, zeggende: Wanneer een vrouw zaad gegeven, en een knechtje gebaard zal hebben, zo zal zij zeven dagen onrein zijn; volgens de dagen der afzondering harer krankheid zal zij onrein zijn.
\par 3 En op den achtsten dag zal het vlees zijner voorhuid besneden worden.
\par 4 Daarna zal zij drie en dertig dagen blijven in het bloed harer reiniging; niets heiligs zal zij aanroeren, en tot het heiligdom zal zij niet komen, totdat de dagen harer reiniging vervuld zijn.
\par 5 Maar indien zij een meisje gebaard zal hebben, zo zal zij twee weken onrein zijn, volgens haar afzondering; daarna zal zij zes en zestig dagen blijven in het bloed harer reiniging.
\par 6 En als de dagen harer reiniging voor den zoon, of voor de dochter, vervuld zullen zijn, zo zal zij een eenjarig lam ten brandoffer, en een jonge duif, of tortelduif, ten zondoffer brengen, voor de deur van de tent der samenkomst, tot den priester.
\par 7 Die zal dat offeren voor het aangezicht des HEEREN, en zal voor haar verzoening doen, zo zal zij rein zijn van den vloed haars bloeds. Dit is de wet dergene, die een knechtje of meisje gebaard heeft.
\par 8 Maar indien haar hand niet genoeg voor een lam vindt, zo zal zij twee tortelduiven, of twee jonge duiven nemen, een ten brandoffer, en een ten zondoffer; en de priester zal voor haar verzoening doen; zo zal zij rein zijn.

\chapter{13}

\par 1 Verder sprak de HEERE tot Mozes en tot Aaron, zeggende:
\par 2 Een mens, als in het vel zijns vleses een gezwel, of gezweer, of witte blaar zal zijn, welke in het vel zijns vleses tot een plaag der melaatsheid zou worden, hij zal dan tot den priester Aaron, of tot een uit zijn zonen, de priesteren, gebracht worden.
\par 3 En de priester zal de plaag in het vel des vleses bezien; zo het haar in die plaag in wit veranderd is, en het aanzien der plaag dieper is dan het vel zijns vleses, het is de plaag der melaatsheid; als de priester hem bezien zal hebben, dan zal hij hem onrein verklaren.
\par 4 Maar zo de blaar in het vel zijn vleses wit is, en haar aanzien niet dieper is dan het vel, en het haar niet in wit veranderd is, zo zal de priester hem, die de plaag heeft, zeven dagen opsluiten.
\par 5 Daarna zal de priester op den zevenden dag hem bezien; indien, ziet, de plaag, naar dat hij zien kan, is staande gebleven, en de plaag in het vel niet uitgespreid is, zo zal de priester hem zeven andere dagen opsluiten.
\par 6 En de priester zal hem andermaal op den zevenden dag bezien; indien, ziet, de plaag ingetrokken, en de plaag in het vel niet uitgespreid is, zo zal de priester hem rein verklaren; het was een verzwering; en hij zal zijn klederen wassen, zo is hij rein.
\par 7 Maar zo de verzwering in het vel ganselijk uitgespreid is, nadat hij aan den priester tot zijn reiniging zal vertoond zijn, zo zal hij andermaal aan den priester vertoond worden.
\par 8 Indien de priester merken zal, dat, ziet, de verzwering in het vel uitgespreid is, zo zal de priester hem onrein verklaren; het is melaatsheid.
\par 9 Wanneer de plaag der melaatsheid in een mens zal zijn, zo zal hij tot den priester gebracht worden.
\par 10 Indien de priester merken zal, dat, ziet, een wit gezwel in het vel is, hetwelk het haar in wit veranderd heeft, en gezondheid van levend vlees in dat gezwel is;
\par 11 Dat is een verouderde melaatsheid in het vel zijns vleses; daarom zal hem de priester onrein verklaren; hij zal hem niet doen opsluiten, want hij is onrein.
\par 12 En zo de melaatsheid in het vel ganselijk uitbot, en de melaatsheid het gehele vel desgenen, die de plaag heeft, van zijn hoofd tot zijn voeten, bedekt heeft, naar al het gezicht van de ogen des priesters;
\par 13 En de priester merken zal, dat, ziet, de melaatsheid zijn gehele vlees bedekt heeft, zo zal hij hem, die de plaag heeft, rein verklaren; zij is geheel in wit veranderd; hij is rein.
\par 14 Maar ten welken dage levend vlees daarin gezien zal worden, zal hij onrein zijn.
\par 15 Als dan de priester dat levende vlees gezien zal hebben, zal hij hem onrein verklaren; dat levende vlees is onrein; het is melaatsheid.
\par 16 Of als dat levende vlees verkeert, en in wit veranderd zal worden, zo zal hij tot den priester komen.
\par 17 Als de priester hem bezien zal hebben, dat, ziet, de plaag in wit veranderd is, zo zal de priester hem, die de plaag heeft, rein verklaren; hij is rein.
\par 18 Het vlees ook, als in deszelfs vel een zweer zal geweest zijn, zo het genezen is;
\par 19 En in de plaats van die zweer een wit gezwel, of een witte roodachtige blaar worden zal, zo zal het aan den priester vertoond worden.
\par 20 Indien de priester merken zal, dat, ziet, haar aanzien lager is dan het vel, en derzelver haar in wit veranderd is, zo zal de priester hem onrein verklaren; het is de plaag der melaatsheid, zij is door de zweer uitgebot.
\par 21 Wanneer nu de priester die bezien zal hebben, dat, ziet, geen wit haar daaraan is, en die niet lager dan het vel, maar ingetrokken is, zo zal de priester hem zeven dagen opsluiten.
\par 22 Zo zij daarna gans in het vel uitgespreid zal zijn, zo zal de priester hem onrein verklaren; het is de plaag.
\par 23 Maar indien de blaar in haar plaats zal staande blijven, niet uitgespreid zijnde, het is de roof van die zweer, zo zal de priester hem rein verklaren;
\par 24 Of wanneer in het vel des vleses een vurige brand zal geweest zijn, en het gezonde van dien brand een witte roodachtige of witte blaar is;
\par 25 En de priester die gezien zal hebben, dat, ziet, het haar op de blaar in wit veranderd is, en haar aanzien dieper is dan het vel; het is melaatsheid, door den brand is zij uitgebot; daarom zal hem de priester onrein verklaren; het is de plaag der melaatsheid.
\par 26 Maar indien de priester die merken zal, dat, ziet, op de blaar geen wit haar is, en zij niet lager dan het vel, maar ingetrokken is, zo zal de priester hem zeven dagen opsluiten.
\par 27 Daarna zal de priester hem op den zevenden dag bezien; indien zij gans uitgespreid is in het vel, zo zal de priester hem onrein verklaren; het is de plaag der melaatsheid.
\par 28 Maar indien de blaar in haar plaats staande zal blijven, noch in het vel uitgespreid, maar ingetrokken zal zijn, het is een gezwel van den brand; daarom zal de priester hem rein verklaren, want het is de roof van den brand.
\par 29 Verder, als in een man of vrouw een plaag zal zijn in het hoofd, of in den baard;
\par 30 En de priester de plaag zal bezien hebben, dat, ziet, haar aanzien dieper is dan het vel, en geelachtig dun haar daarop is, zo zal de priester hem onrein verklaren; het is schurftheid, het is melaatsheid van het hoofd of van den baard.
\par 31 Maar als de priester de plaag der schurftheid zal bezien hebben, dat, ziet, haar aanzien niet dieper is dan het vel, en geen zwart haar daarop is, zo zal de priester hem, die de plaag der schurftheid heeft, zeven dagen doen opsluiten.
\par 32 Daarna zal de priester die plaag op den zevenden dag bezien; indien, ziet, de schurftheid niet uitgespreid, en daarop geen geelachtig haar is, noch het aanzien der schurftheid dieper dan het vel is;
\par 33 Zo zal hij zich scheren laten; maar de schurftheid zal hij niet scheren; en de priester zal hem, die de schurftheid heeft, andermaal zeven dagen doen opsluiten.
\par 34 Daarna zal de priester die schurftheid op den zevenden dag bezien; indien, ziet, de schurftheid in het vel niet uitgespreid is, en haar aanzien niet dieper is dan het vel, zo zal de priester hem rein verklaren; en hij zal zijn klederen wassen, en rein zijn.
\par 35 Maar indien de schurftheid in het vel gans uitgespreid is, na zijn reiniging;
\par 36 En de priester hem zal bezien hebben, dat, ziet, de schurftheid in het vel uitgespreid is, de priester zal naar het geelachtig haar niet zoeken; hij is onrein.
\par 37 Maar indien die schurftheid, naar dat hij zien kan, is staande gebleven, en zwart haar daarop gewassen is, die schurftheid is genezen, hij is rein; daarom zal de priester hem rein verklaren.
\par 38 Verder als een man, of vrouw, aan het vel van hun vlees blaren zullen hebben, witte blaren;
\par 39 En de priester zal gemerkt hebben, dat, ziet, ingetrokken witte blaren in het vel van hun vlees zijn; het is een witte puist in het vel uitgebot, hij is rein.
\par 40 En als een man zijn hoofdhaar zal uitgevallen zijn, hij is kaal, hij is rein.
\par 41 En zo van de zijde zijns aangezichts het haar van zijn hoofd zal uitgevallen zijn, hij is bles, hij is rein.
\par 42 Maar zo in de kaalheid, of in de blesse, een witte roodachtige plaag is, dat is melaatsheid, uitbottende in zijn kaalheid, of in zijn blesse.
\par 43 Als de priester hem zal bezien hebben, dat, ziet, het gezwel van die plaag in zijn kaalheid, of blesse, wit roodachtig is, gelijk het aanzien der melaatsheid van het vel des vleses;
\par 44 Die man is melaats, hij is onrein; de priester zal hem ganselijk onrein verklaren, zijn plaag is op zijn hoofd.
\par 45 Voorts zullen de klederen des melaatsen, in wien die plaag is, gescheurd zijn, en zijn hoofd zal ontbloot zijn, en hij zal de bovenste lip bewimpelen; daartoe zal hij roepen: Onrein, onrein!
\par 46 Al de dagen, in welke deze plaag aan hem zal zijn, zal hij onrein zijn; onrein is hij, hij zal alleen wonen; buiten het leger zal zijn woning wezen.
\par 47 Verder als aan een kleed de plaag der melaatsheid zal zijn, aan een wollen kleed, of aan een linnen kleed,
\par 48 Of aan den scheerdraad, of aan den inslag van linnen, of van wol, of aan vel, of aan enig vellenwerk;
\par 49 En die plaag aan het kleed, of aan het vel, of aan den scheerdraad, of aan den inslag, of aan enig vellentuig, groenachtig of roodachtig is; het is de plaag der melaatsheid; daarom zal zij den priester vertoond worden.
\par 50 En de priester zal de plaag bezien; en hij zal hetgeen de plaag heeft, zeven dagen doen opsluiten.
\par 51 Daarna zal hij op den zevenden dag de plaag bezien; zo de plaag uitgespreid is aan het kleed, of aan den scheerdraad, of aan den inslag, of aan het vel, tot wat werk dat vel zou mogen gemaakt zijn, die plaag is een knagende melaatsheid, het is onrein.
\par 52 Daarom zal hij dat kleed, of die werpte, of dien inslag van wol, of van linnen, of alle vellentuig, waarin die plaag zal zijn, verbranden; want het is een knagende melaatsheid; het zal met vuur verbrand worden.
\par 53 Doch indien de priester zal zien, dat, ziet, de plaag aan het kleed, of aan den scheerdraad, of aan den inslag, of aan enig vellentuig niet uitgespreid is;
\par 54 Zo zal de priester gebieden, dat men hetgeen, waaraan die plaag is, wasse, en hij zal dat andermaal zeven dagen doen opsluiten.
\par 55 Als de priester, nadat het gewassen is, de plaag zal bezien hebben, dat, ziet, de plaag haar gedaante niet veranderd heeft, en de plaag niet uitgespreid is, het is onrein, gij zult het met vuur verbranden; het is een ingraving aan zijn achterste of aan zijn voorste zijde.
\par 56 Indien nu de priester merken zal, dat, ziet, die plaag, nadat zij zal gewassen zijn, ingetrokken is; dan zal hij ze van het kleed, of van het vel, of van den scheerdraad, of van den inslag afscheuren.
\par 57 Maar zo zij nog aan het kleed, of aan den scheerdraad, of aan den inslag, of aan enig vellentuig, gezien wordt, het is uitbottende melaatsheid; gij zult hetgeen, waaraan de plaag is, met vuur verbranden.
\par 58 Maar het kleed, of de werpte, of de inslag, of alle vellentuig, dat gij gewassen zult hebben, als de plaag daarvan geweken zal zijn, dat zal andermaal gewassen worden, en het zal rein zijn.
\par 59 Dit is de wet van de plaag der melaatsheid, van een wollen of linnen kleed, of een werpte, of een inslag, of alle vellentuig, om dat rein te verklaren, of onrein te verklaren.

\chapter{14}

\par 1 Daarna sprak de HEERE tot Mozes, zeggende:
\par 2 Dit zal de wet des melaatsen zijn, ten dage zijner reiniging: dat hij tot den priester zal gebracht worden.
\par 3 En de priester zal buiten het leger gaan; als de priester merken zal, dat, ziet, die plaag der melaatsheid van den melaatse genezen is;
\par 4 Zo zal de priester gebieden, dat men voor hem, die te reinigen zal zijn, twee levende reine vogelen neme, mitsgaders cederenhout, en scharlaken, en hysop.
\par 5 De priester zal ook gebieden, dat men den ene vogel slachte, in een aarden vat, over levend water.
\par 6 Dien levenden vogel zal hij nemen, en het cederhout, en het scharlaken, en den hysop; en zal die, en den levenden vogel dopen in het bloed des vogels, die boven het levende water geslacht is.
\par 7 En hij zal over hem, die van de melaatsheid te reinigen is, zevenmaal sprengen; daarna zal hij hem rein verklaren, en den levenden vogel in het open veld vliegen laten.
\par 8 Die nu te reinigen is, zal zijn klederen wassen, en al zijn haar afscheren, en zich in het water afwassen, zo zal hij rein zijn; daarna zal hij in het leger komen, maar zal buiten zijn tent zeven dagen blijven.
\par 9 En op den zevenden dag zal het geschieden, dat hij al zijn haar zal afscheren, zijn hoofd, en zijn baard, en de wenkbrauwen zijner ogen; ja, al zijn haar zal hij afscheren, en al zijn klederen wassen, en zijn vlees met water baden, zo zal hij rein zijn.
\par 10 En op den achtsten dag zal hij twee volkomen lammeren, en een eenjarig volkomen schaap nemen, mitsgaders drie tienden meelbloem ten spijsoffer, met olie gemengd, en een log olie.
\par 11 De priester nu, die de reiniging doet, zal den man, die te reinigen is, en die dingen, stellen voor het aangezicht des HEEREN, aan de deur van de tent der samenkomst.
\par 12 En de priester zal dat ene lam nemen, en hetzelve offeren tot een schuldoffer met den log olie; en zal die ten beweegoffer voor het aangezicht des HEEREN bewegen.
\par 13 Daarna zal hij dat lam slachten in de plaats, waar men het zondoffer en het brandoffer slacht, in de heilige plaats; want het schuldoffer, gelijk het zondoffer, is voor den priester; het is een heiligheid der heiligheden.
\par 14 En de priester zal van het bloed des schuldoffers nemen, hetwelk de priester doen zal op het lapje van het rechteroor desgenen, die te reinigen is, en op den duim zijner rechterhand, en op den groten teen zijns rechtervoets.
\par 15 De priester zal ook uit den log der olie nemen, en zal ze op des priesters linkerhand gieten.
\par 16 Dan zal de priester zijn rechtervinger indopen, nemende van die olie, die in zijn linkerhand is, en zal met zijn vinger van die olie zevenmaal sprengen, voor het aangezicht des HEEREN.
\par 17 En van het overige van die olie, die in zijn hand zal zijn, zal de priester doen op het lapje van het rechteroor desgenen, die te reinigen is, en op den duim zijner rechterhand, en op den groten teen zijns rechtervoets, boven op het bloed des schuldoffers.
\par 18 Dat nog overgebleven zal zijn van die olie, die in de hand des priesters geweest is, zal hij doen op het hoofd desgenen, die te reinigen is; zo zal de priester over hem verzoening doen voor het aangezicht des HEEREN.
\par 19 De priester zal ook het zondoffer bereiden, en voor hem, die van zijn onreinigheid te reinigen is, verzoening doen; en daarna zal hij het brandoffer slachten.
\par 20 En de priester zal dat brandoffer en dat spijsoffer op het altaar offeren; zo zal de priester de verzoening voor hem doen, en hij zal rein zijn.
\par 21 Maar indien hij arm is, en zijn hand dat niet bereikt, zo zal hij een lam ten schuldoffer, ter beweging nemen, om voor hem verzoening te doen; daartoe een tiende meelbloem, met olie gemengd, ten spijsoffer, en een log olie;
\par 22 Mitsgaders twee tortelduiven, of twee jonge duiven, die zijn hand bereiken zal, welker ene ten zondoffer, en een ten brandoffer zijn zal.
\par 23 En hij zal die, op den achtsten dag zijner reiniging, tot den priester brengen, aan de deur van de tent der samenkomst, voor het aangezicht des HEEREN.
\par 24 En de priester zal het lam des schuldoffers, en den log der olie nemen; en de priester zal die ten beweegoffer voor het aangezicht des HEEREN bewegen.
\par 25 Daarna zal hij het lam des schuldoffers slachten, en de priester zal van het bloed des schuldoffers nemen, en doen op het rechteroorlapje desgenen, die te reinigen is, en op den duim zijner rechterhand, en op den groten teen zijns rechtervoets.
\par 26 Ook zal de priester van die olie op des priesters linkerhand gieten.
\par 27 Daarna zal de priester met zijn rechtervinger van die olie, die op zijn linkerhand is, sprengen, zevenmaal, voor het aangezicht des HEEREN.
\par 28 En de priester zal van de olie, die op zijn hand is, doen aan het lapje van het rechteroor desgenen, die te reinigen is, en aan den duim zijner rechterhand, en aan den groten teen zijns rechtervoets, op de plaats van het bloed des schuldoffers.
\par 29 En het overgeblevene van de olie, die in de hand des priesters is, zal hij doen op het hoofd desgenen, die te reinigen is, om de verzoening voor hem te doen, voor het aangezicht des HEEREN.
\par 30 Daarna zal hij de ene van de tortelduiven, of van de jonge duiven bereiden, van hetgeen zijn hand bereikt zal hebben.
\par 31 Van hetgeen zijn hand bereikt zal hebben, zal het een ten zondoffer, en het een ten brandoffer zijn, boven het spijsoffer; zo zal de priester voor hem, die te reinigen is, verzoening doen voor het aangezicht des HEEREN.
\par 32 Dit is de wet desgenen, in wien de plaag der melaatsheid zal zijn, wiens hand in zijn reiniging dat niet bereikt zal hebben.
\par 33 Verder sprak de HEERE tot Mozes en tot Aaron, zeggende:
\par 34 Als gij zult gekomen zijn in het land van Kanaan, hetwelk Ik u tot bezitting geven zal, en Ik de plaag der melaatsheid aan een huis van dat land uwer bezitting zal gegeven hebben;
\par 35 Zo zal hij, van wien dat huis is, komen, en den priester te kennen geven, zeggende: Het schijnt mij, alsof er een plaag in het huis ware.
\par 36 En de priester zal gebieden, dat zij dat huis ruimen, aleer de priester komt, om die plaag te bezien, opdat niet al wat in dat huis is, onrein worde; en daarna zal de priester komen, om dat huis te bezien.
\par 37 Als hij die plaag bezien zal, dat, ziet, die plaag aan de wanden van dat huis zijn groenachtige of roodachtige kuiltjes, en hun aanzien lager is dan die wand;
\par 38 De priester zal uit dat huis uitgaan, aan de deur van het huis, en hij zal dat huis zeven dagen doen toesluiten.
\par 39 Daarna zal de priester op den zevenden dag wederkeren; indien hij merken zal, dat, ziet, die plaag aan de wanden van dat huis uitgespreid is;
\par 40 Zo zal de priester gebieden, dat zij de stenen, in welke die plaag is, uitbreken, en dezelve tot buiten de stad werpen, aan een onreine plaats;
\par 41 En dat huis zal hij rondom van binnen doen schrabben, en zij zullen het stof, dat zij afgeschrabd hebben, tot buiten de stad aan een onreine plaats uitstorten.
\par 42 Daarna zullen zij andere stenen nemen, en in de plaats van gene stenen brengen; en men zal ander leem nemen, en dat huis bestrijken.
\par 43 Maar indien die plaag wederkeert, en in dat huis uitbot, nadat men de stenen uitgebroken heeft, en na het afschrabben van het huis, en nadat het zal bestreken zijn;
\par 44 Zo zal de priester komen; als hij nu zal merken, dat, ziet, die plaag aan dat huis uitgespreid is, het is een knagende melaatsheid in dat huis, het is onrein.
\par 45 Daarom zal men dat huis, zijn stenen, en zijn hout even afbreken, mitsgaders al het leem van het huis, en men zal het tot buiten de stad uitvoeren, aan een onreine plaats.
\par 46 En die in dat huis gaat te enigen dage, als men hetzelve zal toegesloten hebben, zal onrein zijn tot aan den avond.
\par 47 Die ook in dat huis te slapen ligt, zal zijn klederen wassen; insgelijks, die in dat huis eet, zal zijn klederen wassen.
\par 48 Maar als de priester zal weder ingegaan zijn, en zal merken, dat, ziet, die plaag aan dat huis niet uitgespreid is, nadat het huis zal bestreken zijn; zo zal de priester dat huis rein verklaren, dewijl die plaag genezen is.
\par 49 Daarna zal hij, om dat huis te ontzondigen, twee vogeltjes nemen, mitsgaders cederenhout, en scharlaken, en hysop.
\par 50 En hij zal den enen vogel slachten in een aarden vat, over levend water.
\par 51 Dan zal hij dat cederenhout, en dien hysop, en het scharlaken, en den levenden vogel nemen, en zal die in het bloed des geslachten vogels en in het levende water dopen; en hij zal dat huis zevenmaal besprengen.
\par 52 Zo zal hij dat huis ontzondigen met het bloed des vogels, en met dat levend water, en met den levenden vogel, en met dat cederenhout, en met den hysop, en met het scharlaken.
\par 53 Den levenden vogel nu zal hij tot buiten de stad, in het open veld, laten vliegen; zo zal hij over het huis verzoening doen, en het zal rein zijn.
\par 54 Dit is de wet voor alle plage der melaatsheid, en voor schurftheid;
\par 55 En voor melaatsheid der klederen, en der huizen;
\par 56 Mitsgaders voor gezwel, en voor gezweer, en voor blaren;
\par 57 Om te leren, op welken dag iets onrein, en op welken dag iets rein is. Dit is de wet der melaatsheid.

\chapter{15}

\par 1 Verder sprak de HEERE tot Mozes en tot Aaron, zeggende:
\par 2 Spreekt tot de kinderen Israels, en zegt tot hen: Een ieder man, als hij vloeiende zal zijn uit zijn vlees, zal om zijn vloed onrein zijn.
\par 3 Dit nu zal zijn onreinigheid om zijn vloed zijn: zo zijn vlees zijn vloed uitzevert, of zijn vlees van zijn vloed zich verstopt, dat is zijn onreinigheid.
\par 4 Alle leger, waarop hij, die den vloed heeft, zal liggen, zal onrein zijn, en alle tuig, waarop hij zal zitten, zal onrein zijn.
\par 5 Een ieder ook, die zijn leger zal aanroeren, zal zijn klederen wassen, en zich met water baden, en zal onrein zijn tot aan den avond.
\par 6 En die op dat tuig zit, waarop hij, die den vloed heeft, gezeten zal hebben, zal zijn klederen wassen, en zich met water baden, en zal onrein zijn tot aan den avond.
\par 7 En die het vlees desgenen, die den vloed heeft, aanroert, zal zijn klederen wassen, en zich met water baden, en onrein zijn tot aan den avond.
\par 8 Als ook hij, die den vloed heeft, op een reine zal gespogen hebben, dan zal hij zijn klederen wassen, en zal zich met water baden, en onrein zijn tot aan den avond.
\par 9 Insgelijks alle zadel, waarop hij, die den vloed heeft, zal gereden hebben, zal onrein zijn.
\par 10 En al wie iets aanroert, dat onder hem zal geweest zijn, zal onrein zijn tot aan den avond; en die hetzelve draagt, zal zijn klederen wassen, en zich met water baden, en onrein zijn tot aan den avond.
\par 11 Daartoe een ieder, wien hij, die den vloed heeft, zal aangeroerd hebben, zonder zijn handen met water gespoeld te hebben, die zal zijn klederen wassen, en zich met water baden, en onrein zijn tot aan den avond.
\par 12 Ook het aarden vat, hetwelk hij, die den vloed heeft, zal aangeroerd hebben, zal gebroken worden; maar alle houten vat zal met water gespoeld worden.
\par 13 Als hij nu, die den vloed heeft, van zijn vloed gereinigd zal zijn, zo zal hij tot zijn reiniging zeven dagen voor zich tellen, en zijn klederen wassen, en hij zal zijn vlees met levend water baden, zo zal hij rein zijn.
\par 14 En op den achtsten dag zal hij voor zich twee tortelduiven of twee jonge duiven nemen; en zal voor het aangezicht des HEEREN, aan de deur van de tent der samenkomst komen, en zal ze den priester geven.
\par 15 En de priester zal die bereiden, een ten zondoffer, en een ten brandoffer; zo zal de priester over hem voor het aangezicht des HEEREN, vanwege zijn vloed, verzoening doen.
\par 16 Verder een man, als van hem het zaad des bijliggens zal uitgegaan zijn, die zal zijn ganse vlees met water baden, en onrein zijn tot aan den avond.
\par 17 Ook alle kleed, en alle vel, aan hetwelk het zaad des bijliggens wezen zal, dat zal met water gewassen worden, en onrein zijn tot aan den avond.
\par 18 Mitsgaders de vrouw, als een man met het zaad des bijliggens bij haar gelegen zal hebben; daarom zullen zij zich met water baden, en onrein zijn tot aan den avond.
\par 19 Maar als een vrouw vloeiende zijn zal, zijnde haar vloed van bloed in haar vlees, zo zal zij zeven dagen in haar afzondering zijn; en al wie haar aanroert, zal onrein zijn tot aan den avond.
\par 20 En al hetgeen, waarop zij in haar afzondering zal gelegen hebben, zal onrein zijn; mitsgaders alles, waarop zij zal gezeten hebben, zal onrein zijn.
\par 21 En al wie haar leger aanroert, zal zijn klederen wassen, en zich met water baden, en onrein zijn tot aan den avond.
\par 22 Ook al wie enig tuig, waarop zij gezeten zal hebben, aanroert, zal zijn klederen wassen, en zich met water baden, en onrein zijn tot aan den avond.
\par 23 Zelfs indien het op het leger geweest zal zijn, of op het tuig, waarop zij zat, als hij dat aanroerde, hij zal onrein zijn tot aan den avond.
\par 24 Insgelijks zo iemand zekerlijk bij haar gelegen heeft, dat haar afzondering op hem zij, zo zal hij zeven dagen onrein zijn; daartoe alle leger, waarop hij zal gelegen hebben, zal onrein zijn.
\par 25 Wanneer ook een vrouw, vele dagen buiten den tijd harer afzondering, van den vloed haars bloeds vloeien zal, of wanneer zij vloeien zal boven hare afzondering, zij zal al den dagen van den vloed harer onreinigheid, als in de dagen harer afzondering onrein zijn.
\par 26 Alle leger, waarop zij al de dagen haars vloeds gelegen zal hebben, zal haar zijn als het leger harer afzondering; en alle tuig, waarop zij zal gezeten hebben, zal onrein zijn, naar de onreinigheid harer afzondering.
\par 27 En zo wie die dingen aanroert, zal onrein zijn; daarom zal hij zijn klederen wassen, en zich met water baden, en onrein zijn tot aan den avond.
\par 28 Maar als zij van haar vloed rein wordt, dan zal zij voor zich zeven dagen tellen, daarna zal zij rein zijn.
\par 29 En op den achtsten dag zal zij voor zich twee tortelduiven, of twee jonge duiven nemen, en zij zal die tot den priester brengen, aan de deur van de tent der samenkomst.
\par 30 Dan zal de priester een ten zondoffer en een ten brandoffer bereiden; en de priester zal voor haar, van den vloed harer onreinigheid, verzoening doen voor het aangezicht des HEEREN.
\par 31 Alzo zult gij de kinderen Israels afzonderen van hun onreinigheid; opdat zij in hun onreinigheid niet sterven, als zij Mijn tabernakel, die in het midden van hen is, verontreinigen zouden.
\par 32 Dit is de wet desgenen, die den vloed heeft, en van wien het zaad der bijligging uitgaat; zodat hij daardoor onrein wordt;
\par 33 Mitsgaders van een zwakke vrouw in haar afzondering, en van degene, die van zijn vloed is vloeiende, voor een man, en voor een vrouw; en voor een man, die bij een onreine zal gelegen hebben.

\chapter{16}

\par 1 En de HEERE sprak tot Mozes, nadat de twee zonen van Aaron gestorven waren, als zij genaderd waren voor het aangezicht des HEEREN, en gestorven waren;
\par 2 De HEERE dan zeide tot Mozes: Spreek tot uw broeder Aaron, dat hij niet te allen tijde ga in het heilige, binnen den voorhang, voor het verzoendeksel, dat op de ark is, opdat hij niet sterve; want Ik verschijn in een wolk op het verzoendeksel.
\par 3 Hiermede zal Aaron in het heilige gaan: met een var, een jong rund ten zondoffer, en een ram ten brandoffer.
\par 4 Hij zal den heiligen linnen rok aandoen, en een linnen onderbroek zal aan zijn vlees zijn, en met een linnen gordel zal hij zich gorden, en met een linnen hoed bedekken; dit zijn heilige klederen; daarom zal hij zijn vlees met water baden, als hij ze zal aandoen.
\par 5 En aan de vergadering der kinderen Israels zal hij nemen twee geitenbokken ten zondoffer, en een ram ten brandoffer.
\par 6 Daarna zal Aaron den var des zondoffers, die voor hem zal zijn, offeren, en zal voor zich en voor zijn huis verzoening doen.
\par 7 Hij zal ook beide bokken nemen, en hij zal die stellen voor het aangezicht des HEEREN, aan de deur van de tent der samenkomst.
\par 8 En Aaron zal de loten over die twee bokken werpen: een lot voor den HEERE, en een lot voor den weggaanden bok.
\par 9 Dan zal Aaron den bok, op denwelken het lot voor den HEERE zal gekomen zijn, toebrengen, en zal hem ten zondoffer maken.
\par 10 Maar de bok, op denwelken het lot zal gekomen zijn, om een weggaande bok te zijn, zal levend voor het aangezicht des HEEREN gesteld worden, om door hem verzoening te doen; opdat men hem als een weggaanden bok naar de woestijn uitlate.
\par 11 Aaron dan zal den var des zondoffers, die voor hemzelven zal zijn, toebrengen, en voor zichzelven en voor zijn huis verzoening doen, en zal den var des zondoffers, die voor hemzelven zal zijn, slachten.
\par 12 Hij zal ook een wierookvat vol vurige kolen nemen van het altaar, van voor het aangezicht des HEEREN, en zijn handen vol reukwerk van welriekende specerijen, klein gestoten; en hij zal het binnen den voorhang dragen.
\par 13 En hij zal dat reukwerk op het vuur leggen, voor het aangezicht des HEEREN, opdat de nevel des reukwerks het verzoendeksel, hetwelk is op de getuigenis, bedekke, en dat hij niet sterve.
\par 14 En hij zal van het bloed van den var nemen, en zal met zijn vinger op het verzoendeksel oostwaarts sprengen; en voor het verzoendeksel zal hij zevenmaal met zijn vinger van dat bloed sprengen.
\par 15 Daarna zal hij den bok des zondoffers, die voor het volk zal zijn, slachten, en zal zijn bloed tot binnen in den voorhang dragen, en zal met zijn bloed doen, gelijk als hij met het bloed van den var gedaan heeft, en zal dat sprengen op het verzoendeksel, en voor het verzoendeksel.
\par 16 Zo zal hij voor het heilige, vanwege de onreinigheden der kinderen Israels, en vanwege hun overtredingen, naar al hun zonden, verzoening doen; en alzo zal hij doen aan de tent der samenkomst, welke met hen woont in het midden hunner onreinigheden.
\par 17 En geen mens zal in de tent der samenkomst zijn, als hij zal ingaan, om in het heilige verzoening te doen, totdat hij zal uitkomen; alzo zal hij verzoening doen, voor zichzelven, en voor zijn huis, en voor de gehele gemeente van Israel.
\par 18 Daarna zal hij tot het altaar, dat voor het aangezicht des HEEREN is, uitkomen, en verzoening voor hetzelve doen; en hij zal van het bloed van den var, en van het bloed van den bok nemen, en doen het rondom op de hoornen des altaars.
\par 19 En hij zal daarop van dat bloed met zijn vinger zevenmaal sprengen, en hij zal dat reinigen en heiligen van de onreinigheden der kinderen Israels.
\par 20 Als hij nu zal geeindigd hebben van het heilige, en de tent der samenkomst, en het altaar te verzoenen, zo zal hij dien levenden bok toebrengen.
\par 21 En Aaron zal beide zijn handen op het hoofd van den levenden bok leggen, en zal daarop al de ongerechtigheden der kinderen Israels, en al hun overtredingen, naar al hun zonden, belijden; en hij zal die op het hoofd des boks leggen, en zal hem door de hand eens mans, die voorhanden is, naar de woestijn uitlaten.
\par 22 Alzo zal die bok op zich al hun ongerechtigheden in een afgezonderd land wegdragen; en hij zal dien bok in de woestijn uitlaten.
\par 23 Daarna zal Aaron komen in de tent der samenkomst, en zal de linnen klederen uitdoen, die hij aangedaan had, als hij in het heilige ging, en hij zal ze daar laten.
\par 24 En hij zal zijn vlees in de heilige plaats met water baden, en zijn klederen aandoen; dan zal hij uitgaan, en zijn brandoffer, en het brandoffer des volks bereiden, en voor zich en voor het volk verzoening doen.
\par 25 Ook zal hij het vet des zondoffers op het altaar aansteken.
\par 26 En die den bok, welke een weggaande bok was, zal uitgelaten hebben, zal zijn klederen wassen, en zijn vlees met water baden; en daarna zal hij in het leger komen.
\par 27 Maar den var des zondoffers, en den bok des zondoffers, welker bloed ingebracht is, om verzoening te doen in het heilige, zal men tot buiten het leger uitvoeren; doch hun vellen, hun vlees en hun mest zullen zij met vuur verbranden.
\par 28 Die nu dezelve verbrandt, zal zijn klederen wassen, en zijn vlees met water baden; en daarna zal hij in het leger komen.
\par 29 En dit zal voor u tot een eeuwige inzetting zijn: gij zult in de zevende maand, op den tienden der maand, uw zielen verootmoedigen, en geen werk doen, inboorling noch vreemdeling, die in het midden van u als vreemdeling verkeert.
\par 30 Want op dien dag zal hij voor u verzoening doen, om u te reinigen; van al uw zonden zult gij voor het aangezicht des HEEREN gereinigd worden.
\par 31 Dat zal u een sabbat der rust zijn, opdat gij uw zielen verootmoedigt; het is een eeuwige inzetting.
\par 32 En de priester, dien men gezalfd, en wiens hand men gevuld zal hebben, om voor zijn vader het priesterambt te bedienen, zal de verzoening doen, als hij de linnen klederen, de heilige klederen, zal aangetrokken hebben.
\par 33 Zo zal hij het heilige heiligdom verzoenen, en de tent der samenkomst, en het altaar zal hij verzoenen; desgelijks voor de priesteren, en voor al het volk der gemeente zal hij verzoening doen.
\par 34 En dit zal u tot een eeuwige inzetting zijn, om voor de kinderen Israels van al hun zonden, eenmaal des jaars, verzoening te doen. En men deed, gelijk als de HEERE Mozes geboden had.

\chapter{17}

\par 1 Verder sprak de HEERE tot Mozes, zeggende:
\par 2 Spreek tot Aaron, en tot zijn zonen, en tot al de kinderen Israels, en zeg tot hen: Dit is het woord, hetwelk de HEERE geboden heeft, zeggende:
\par 3 Een ieder van het huis Israels, die een os, of lam, of geit in het leger slachten zal, of die ze slachten zal buiten het leger;
\par 4 En dezelve aan de deur van de tent der samenkomst niet brengen zal, om een offerande den HEERE voor den tabernakel des HEEREN te offeren; het bloed zal dienzelven man toegerekend worden, hij heeft bloed vergoten; daarom zal dezelve man uit het midden zijns volks uitgeroeid worden;
\par 5 Opdat, wanneer de kinderen Israels hun slachtofferen brengen, welke zij op het veld slachten, dat zij die den HEERE toebrengen, aan de deur van de tent der samenkomst tot den priester, en dezelve tot dankofferen den HEERE slachten.
\par 6 En de priester zal het bloed op het altaar des HEEREN, aan de deur van de tent der samenkomst, sprengen; en hij zal het vet aansteken, tot een liefelijken reuk den HEERE.
\par 7 En zij zullen ook niet meer hun slachtofferen den duivelen, welke zij nahoereren, offeren; dat zal hun een eeuwige inzetting zijn voor hun geslachten.
\par 8 Zeg dan tot hen: Een ieder van het huis Israels, en van de vreemdelingen, die in het midden van hen als vreemdelingen verkeren, die een brandoffer of slachtoffer zal offeren,
\par 9 En dat tot de deur van de tent der samenkomst niet zal brengen, om hetzelve den HEERE te bereiden; diezelve man zal uit zijn volken uitgeroeid worden.
\par 10 En een ieder uit het huis Israels, en uit de vreemdelingen, die in het midden van hen als vreemdelingen verkeren, die enig bloed zal gegeten hebben, tegen diens ziel, die dat bloed zal gegeten hebben, zal Ik Mijn aangezicht zetten, en zal die uit het midden haars volks uitroeien.
\par 11 Want de ziel van het vlees is in het bloed; daarom heb Ik het u op het altaar gegeven, om over uw zielen verzoening te doen; want het is het bloed, dat voor de ziel verzoening zal doen.
\par 12 Daarom heb Ik tot de kinderen Israels gezegd: Geen ziel van u zal bloed eten; noch de vreemdeling, die als vreemdeling in het midden van u verkeert, zal bloed eten.
\par 13 Een ieder ook van de kinderen Israels en van de vreemdelingen, die als vreemdelingen in het midden van hen verkeren, die enig wild gedierte, of gevogelte, dat gegeten wordt, in de jacht gevangen zal hebben; die zal deszelfs bloed vergieten, en zal dat met stof bedekken.
\par 14 Want het is de ziel van alle vlees; zijn bloed is voor zijn ziel; daarom heb Ik tot de kinderen Israels gezegd: Gij zult geens vleses bloed eten; want de ziel van alle vlees, dat is zijn bloed; zo wie dat eet, zal uitgeroeid worden.
\par 15 En alle ziel onder de inboorlingen of onder de vreemdelingen, die een dood aas of het verscheurde zal gegeten hebben, die zal zijn klederen wassen, en zich met water baden, en onrein zijn tot aan den avond; daarna zal hij rein zijn.
\par 16 Maar indien hij die niet wast, en zijn vlees niet baadt, zo zal hij zijn ongerechtigheid dragen.

\chapter{18}

\par 1 Verder sprak de HEERE tot Mozes, zeggende:
\par 2 Spreek tot de kinderen Israels en zeg tot hen: Ik ben de HEERE, uw God!
\par 3 Gij zult niet doen naar de werken des Egyptischen lands, waarin gij gewoond hebt; en naar de werken des lands Kanaan, waarheen Ik u brenge, zult gij niet doen, en zult in hun inzettingen niet wandelen.
\par 4 Mijn rechten zult gij doen, en Mijn inzettingen zult gij houden, om in die te wandelen; Ik ben de HEERE, uw God!
\par 5 Ja, Mijn inzettingen en Mijn rechten zult gij houden; welk mens dezelve zal doen, die zal door dezelve leven; Ik ben de HEERE!
\par 6 Niemand zal tot enige nabestaande zijns vleses naderen, om de schaamte te ontdekken; Ik ben de HEERE!
\par 7 Gij zult de schaamte uws vaders en de schaamte uwer moeder niet ontdekken; zij is uw moeder; gij zult haar schaamte niet ontdekken.
\par 8 Gij zult de schaamte der huisvrouw uws vaders niet ontdekken; het is de schaamte uws vaders.
\par 9 De schaamte uwer zuster, der dochter uws vaders, of der dochter uwer moeder, te huis geboren of buiten geboren, haar schaamte zult gij niet ontdekken.
\par 10 De schaamte der dochter uws zoons, of der dochter uwer dochter, haar schaamte zult gij niet ontdekken; want zij zijn uw schaamte.
\par 11 De schaamte van de dochter der huisvrouw uws vaders, die uw vader geboren is (zij is uw zuster), haar schaamte zult gij niet ontdekken.
\par 12 Gij zult de schaamte van de zuster uws vaders niet ontdekken; zij is uws vaders nabestaande.
\par 13 Gij zult de schaamte van de zuster uwer moeder niet ontdekken; want zij is uwer moeder nabestaande.
\par 14 Gij zult de schaamte van den broeder uws vaders niet ontdekken; tot zijn huisvrouw zult gij niet naderen; zij is uw moei.
\par 15 Gij zult de schaamte uwer schoondochter niet ontdekken; zij is uws zoons huisvrouw; gij zult haar schaamte niet ontdekken.
\par 16 Gij zult de schaamte der huisvrouw uws broeders niet ontdekken; het is de schaamte uws broeders.
\par 17 Gij zult de schaamte ener vrouw en harer dochter niet ontdekken; de dochter haars zoons, noch de dochter van haar dochter zult gij nemen, om haar schaamte te ontdekken; zij zijn nabestaanden; het is een schandelijke daad.
\par 18 Gij zult ook geen vrouw tot haar zuster nemen, om haar te benauwen, mits haar schaamte nevens haar, in haar leven, te ontdekken.
\par 19 Ook zult gij tot de vrouw in de afzondering van haar onreinigheid niet naderen, om haar schaamte te ontdekken.
\par 20 En gij zult niet liggen bij uws naasten huisvrouw ter bezading, om met haar onrein te worden.
\par 21 En van uw zaad zult gij niet geven, om voor den Molech door het vuur te doen gaan; en den Naam uws Gods zult gij niet ontheiligen; Ik ben de HEERE!
\par 22 Bij een manspersoon zult gij niet liggen met vrouwelijke bijligging; dit is een gruwel.
\par 23 Insgelijks zult gij bij geen beest liggen, om daarmede onrein te worden; een vrouw zal ook niet staan voor een beest, om daarmede te doen te hebben; het is een gruwelijke vermenging.
\par 24 Verontreinigt u niet met enige van deze; want de heidenen, die Ik van uw aangezicht uitwerpe, zijn met alle deze verontreinigd;
\par 25 Zodat het land onrein is, en Ik over hetzelve zijn ongerechtigheid bezoeke, en het land zijn inwoners uitspuwt.
\par 26 Maar gij zult Mijn inzettingen en Mijn rechten onderhouden, en van al die gruwelen niets doen, inboorling noch vreemdeling, die in het midden van u als vreemdeling verkeert.
\par 27 Want de lieden dezes lands, die voor u geweest zijn, hebben al deze gruwelen gedaan; en het land is onrein geworden.
\par 28 Dat u dat land niet uitspuwe, als gij hetzelve zult verontreinigd hebben; gelijk als het het volk, dat voor u was, uitgespuwd heeft.
\par 29 Want al wie enige van deze gruwelen doen zal, die zielen, die ze doen, zullen uit het midden van haar volk uitgeroeid worden.
\par 30 Daarom zult gij Mijn bevel onderhouden, dat gij niet doet van die gruwelijke inzettingen, die voor u zijn gedaan geweest, en u daarmede niet verontreinigt; Ik ben de HEERE, uw God!

\chapter{19}

\par 1 Verder sprak de HEERE tot Mozes, zeggende:
\par 2 Spreek tot de ganse vergadering der kinderen Israels, en zeg tot hen: Gij zult heilig zijn, want Ik, de HEERE, uw God, ben heilig!
\par 3 Want ieder zal zijn moeder en zijn vader vrezen, en Mijn sabbatten houden; Ik ben de HEERE, uw God!
\par 4 Gij zult u tot de afgoden niet keren, en u geen gegoten goden maken; Ik ben de HEERE, uw God!
\par 5 En wanneer gij een dankoffer den HEERE offeren zult, naar uw welgevallen zult gij dat offeren.
\par 6 Op den dag van uw offeren, en des anderen daags, zal het gegeten worden; maar wat tot op den derden dag overblijft zal met vuur verbrand worden.
\par 7 En zo het op den derden dag enigszins gegeten wordt, het is een afgrijselijk ding, het zal niet aangenaam zijn.
\par 8 En zo wie dat eet, zal zijn ongerechtigheid dragen, omdat hij het heilige des HEEREN ontheiligd heeft; daarom zal dezelve ziel, uit haar volken uitgeroeid worden.
\par 9 Als gij ook den oogst uws lands inoogsten zult, gij zult den hoek uws velds niet ganselijk afoogsten, en dat van uw oogst op te zamelen is, niet opzamelen.
\par 10 Insgelijks zult gij uw wijngaard niet nalezen, en de afgevallen bezien van uw wijngaard niet opzamelen; den arme en den vreemdeling zult gij die overlaten; Ik ben de HEERE, uw God!
\par 11 Gij zult niet stelen, en gij zult niet liegen, noch valselijk handelen, een iegelijk tegen zijn naaste.
\par 12 Gij zult niet valselijk bij Mijn Naam zweren; want gij zoudt den Naam uws Gods ontheiligen; Ik ben de HEERE.
\par 13 Gij zult uw naaste niet bedriegelijk verdrukken, noch beroven; des dagloners arbeidsloon zal bij u niet vernachten tot aan den morgen.
\par 14 Gij zult den dove niet vloeken, en voor het aangezicht des blinden geen aanstoot zetten; maar gij zult voor uw God vrezen; Ik ben de HEERE!
\par 15 Gij zult geen onrecht doen in het gericht; gij zult het aangezicht des geringen niet aannemen, noch het aangezicht des groten voortrekken; in gerechtigheid zult gij uw naaste richten.
\par 16 Gij zult niet wandelen als een achterklapper onder uw volken; gij zult niet staan tegen het bloed van uw naaste; Ik ben de HEERE!
\par 17 Gij zult uw broeder in uw hart niet haten; gij zult uw naaste naarstiglijk berispen, en zult de zonde in hem niet verdragen.
\par 18 Gij zult niet wreken, noch toorn behouden tegen de kinderen uws volks; maar gij zult uw naaste liefhebben als uzelven; Ik ben de HEERE!
\par 19 Gij zult Mijn inzettingen houden; gij zult geen tweeerlei aard uwer beesten laten samen te doen hebben; uwen akker zult gij niet met tweeerlei zaad bezaaien, en een kleed van tweeerlei stof, dooreen vermengd, zal aan u niet komen.
\par 20 En wanneer een man, door bijligging des zaads, bij een vrouw zal gelegen hebben, die een dienstmaagd is, bij den man versmaad, en geenszins gelost is, en haar geen vrijheid is gegeven; die zullen gegeseld worden; zij zullen niet gedood worden; want zij was niet vrij gemaakt.
\par 21 En hij zal zijn schuldoffer den HEERE aan de deur van de tent der samenkomst brengen, een ram ten schuldoffer.
\par 22 En de priester zal met den ram des schuldoffers, voor hem over zijn zonde, die hij gezondigd heeft, voor het aangezicht des HEEREN verzoening doen; en hem zal vergeving geschieden van zijn zonde, die hij gezondigd heeft.
\par 23 Als gij ook in dat land gekomen zult zijn, en alle geboomte ter spijze geplant zult hebben, zo zult gij de voorhuid daarvan, deszelfs vrucht, besnijden; drie jaren zal het u onbesneden zijn, daarvan zal niet gegeten worden.
\par 24 Maar in het vierde jaar zal al zijn vrucht een heilig ding zijn, ter lofzegging voor den HEERE.
\par 25 En in het vijfde jaar zult gij deszelfs vrucht eten, om het inkomen daarvan voor u te vermeerderen; Ik ben de HEERE, uw God!
\par 26 Gij zult niets met het bloed eten. Gij zult op geen vogelgeschrei acht geven, noch guichelarij plegen.
\par 27 Gij zult de hoeken uws hoofds niet rond afscheren; ook zult gij de hoeken uws baards niet verderven.
\par 28 Gij zult om een dood lichaam geen snijding in uw vlees maken, noch schrift van een ingedrukt teken in u maken; Ik ben de HEERE!
\par 29 Gij zult uw dochter niet ontheiligen, haar ter hoererij houdende; opdat het land niet hoerere, en het land met schandelijke daden vervuld worde.
\par 30 Gij zult Mijn sabbatten houden, en Mijn heiligdom zult gij vrezen; Ik ben de HEERE!
\par 31 Gij zult u niet keren tot de waarzeggers, en tot de duivelskunstenaars; zoekt hen niet, u met hen verontreinigende; Ik ben de HEERE, uw God!
\par 32 Voor het grauwe haar zult gij opstaan, en zult het aangezicht des ouden vereren; en gij zult vrezen voor uw God; Ik ben de HEERE!
\par 33 En wanneer een vreemdeling bij u in uw land als vreemdeling verkeren zal, gij zult hem niet verdrukken.
\par 34 De vreemdeling, die als vreemdeling bij u verkeert, zal onder u zijn als een inboorling van ulieden; gij zult hem liefhebben als uzelven; want gij zijt vreemdeling geweest in Egypteland; Ik ben de HEERE, uw God!
\par 35 Gij zult geen onrecht doen in het gericht, met de el, met het gewicht, of met de maat.
\par 36 Gij zult een rechte wage hebben, rechte weegstenen, een rechte efa, en een rechte hin; Ik ben de HEERE, uw God, die u uit Egypteland uitgevoerd heb!
\par 37 Daarom zult gij al Mijn inzettingen en al Mijn rechten onderhouden, en zult ze doen; Ik ben de HEERE!

\chapter{20}

\par 1 Verder sprak de HEERE tot Mozes, zeggende:
\par 2 Gij zult ook tot de kinderen Israels zeggen: Een ieder uit de kinderen Israels, of uit de vreemdelingen, die in Israel als vreemdelingen verkeren, die van zijn zaad den Molech gegeven zal hebben, zal zekerlijk gedood worden; het volk des lands zal hem met stenen stenigen.
\par 3 En Ik zal Mijn aangezicht tegen dien man zetten, en zal hem uit het midden zijns volks uitroeien; want hij heeft van zijn zaad den Molech gegeven, opdat hij Mijn heiligdom ontreinigen, en Mijn heiligen Naam ontheiligen zou.
\par 4 En indien het volk des lands hun ogen enigszins verbergen zal van dien man, als hij van zijn zaad den Molech zal gegeven hebben, dat het hem niet dode;
\par 5 Zo zal Ik Mijn aangezicht tegen dien man en tegen zijn huisgezin zetten, en Ik zal hem, en al degenen, die hem nahoereren, om den Molech na te hoereren, uit het midden huns volks uitroeien.
\par 6 Wanneer er een ziel is, die zich tot de waarzeggers en tot de duivelskunstenaars zal gekeerd hebben, om die na te hoereren, zo zal Ik Mijn aangezicht tegen die ziel zetten, en zal ze uit het midden haars volks uitroeien.
\par 7 Daarom heiligt u, en weest heilig; want Ik ben de HEERE, uw God!
\par 8 En onderhoudt Mijn inzettingen, en doet dezelve; Ik ben de HEERE, die u heilige.
\par 9 Als er iemand is, die zijn vader of zijn moeder zal gevloekt hebben, die zal zekerlijk gedood worden; hij heeft zijn vader of zijn moeder gevloekt; zijn bloed is op hem!
\par 10 Een man ook, die met iemands huisvrouw overspel zal gedaan hebben, dewijl hij met zijns naasten vrouw overspel gedaan heeft, zal zekerlijk gedood worden, de overspeler en de overspeelster.
\par 11 En een man, die bij zijns vaders huisvrouw zal gelegen hebben, heeft zijns vaders schaamte ontdekt; zij beiden zullen zekerlijk gedood worden; hun bloed is op hen!
\par 12 Insgelijks, als de man bij de vrouw zijns zoons zal gelegen hebben, zij zullen beiden zekerlijk gedood worden; zij hebben een gruwelijke vermenging gedaan; hun bloed is op hen!
\par 13 Wanneer ook een man bij een manspersoon zal gelegen hebben, met vrouwelijke bijligging, zij hebben beiden een gruwel gedaan; zij zullen zekerlijk gedood worden; hun bloed is op hen!
\par 14 En wanneer een man een vrouw en haar moeder zal genomen hebben, het is een schandelijke daad; men zal hem, en diezelve met vuur verbranden, opdat geen schandelijke daad in het midden van u zij.
\par 15 Daartoe als een man bij enig vee zal gelegen hebben, hij zal zekerlijk gedood worden; ook zult gijlieden het beest doden.
\par 16 Alzo wanneer een vrouw tot enig beest genaderd zal zijn, om daarmede te doen te hebben, zo zult gij die vrouw en dat beest doden; zij zullen zekerlijk gedood worden; hun bloed is op hen!
\par 17 En als een man zijn zuster, de dochter zijns vaders, of de dochter zijner moeder, zal genomen hebben, en hij haar schaamte gezien, en zij zijn schaamte zal gezien hebben, het is een schandvlek; daarom zullen zij voor de ogen van de kinderen huns volks uitgeroeid worden; hij heeft de schaamte zijner zuster ontdekt, hij zal zijn ongerechtigheid dragen.
\par 18 En als een man bij een vrouw, die haar krankheid heeft, zal gelegen en haar schaamte ontdekt, haar fontein ontbloot, en zij zelve de fontein haars bloeds ontdekt zal hebben, zo zullen zij beiden uit het midden huns volks uitgeroeid worden.
\par 19 Daartoe zult gij de schaamte van de zuster uwer moeder, en van de zuster uws vaders niet ontdekken; dewijl hij zijn nabestaande ontbloot heeft, zullen zij hun ongerechtigheid dragen.
\par 20 Als ook een man bij zijn moei zal gelegen hebben, hij heeft de schaamte zijns ooms ontdekt; zij zullen hun zonde dragen; zonder kinderen zullen zij sterven.
\par 21 En wanneer een man zijns broeders huisvrouw zal genomen hebben, het is onreinigheid; hij heeft de schaamte zijns broeders ontdekt; zij zullen zonder kinderen zijn.
\par 22 Onderhoudt dan al Mijn inzettingen en al Mijn rechten, en doet dezelve; opdat u dat land, waarheen Ik u brenge, om daarin te wonen, niet uitspuwe.
\par 23 En wandelt niet in de inzettingen des volks, hetwelk Ik voor uw aangezicht uitwerp; want al deze dingen hebben zij gedaan; daarom ben Ik op hen verdrietig geworden.
\par 24 En Ik heb u gezegd: Gij zult hun land erfelijk bezitten, en Ik zal u dat geven, opdat gij hetzelve erfelijk bezit, een land vloeiende van melk en honig; Ik ben de HEERE, uw God, Die u van de volken afgezonderd heb!
\par 25 Daarom zult gij onderscheid maken tussen reine en onreine beesten, en tussen het onreine en reine gevogelte; en gij zult uw zielen niet verfoeilijk maken aan de beesten en aan het gevogelte, en aan al wat op den aardbodem kruipt, hetwelk Ik voor u afgezonderd heb, opdat gij het onrein houdt.
\par 26 En gij zult Mij heilig zijn, want Ik, de HEERE, ben heilig; en Ik heb u van de volken afgezonderd, opdat gij Mijns zoudt zijn.
\par 27 Als nu een man en vrouw in zich een waarzeggenden geest zal hebben, of een duivelskunstenaar zal zijn, zij zullen zekerlijk gedood worden; men zal hen met stenen stenigen; hun bloed is op hen.

\chapter{21}

\par 1 Daarna zeide de HEERE tot Mozes: Spreek tot de priesters, de zonen van Aaron, en zeg tot hen: Over een dode zal een priester zich niet verontreinigen onder zijn volken.
\par 2 Behalve over zijn bloedvriend, die hem ten naaste bestaat, over zijn moeder en over zijn vader, en over zijn zoon, en over zijn dochter, en over zijn broeder.
\par 3 En over zijn zuster, die maagd is, hem nabestaande, die nog geen man toebehoord heeft; over die zal hij zich verontreinigen.
\par 4 Hij zal zich niet verontreinigen over een overste onder zijn volken, om zich te ontheiligen.
\par 5 Zij zullen op hun hoofd geen kaalheid maken, en zullen den hoek van hun baard niet afscheren, en in hun vlees zullen zij geen sneden snijden.
\par 6 Zij zullen hun God heilig zijn, en den Naam huns Gods zullen zij niet ontheiligen; want zij offeren de vuurofferen des HEEREN, de spijze huns Gods; daarom zullen zij heilig zijn.
\par 7 Zij zullen geen vrouw nemen, die een hoer of ontheiligde is, noch een vrouw nemen, die van haar man verstoten is; want hij is zijn God heilig.
\par 8 Daarom zult gij hem heiligen, omdat hij de spijze uws Gods offert; hij zal u heilig zijn, want Ik ben heilig; Ik ben de HEERE, die u heilige!
\par 9 Als nu de dochter van enigen priester zal beginnen te hoereren, zij ontheiligt haar vader; met vuur zal zij verbrand worden.
\par 10 En hij, die de hogepriester onder zijn broederen is, op wiens hoofd de zalfolie gegoten is, en wiens hand men gevuld heeft, om die klederen aan te trekken, zal zijn hoofd niet ontbloten, noch zijn klederen scheuren.
\par 11 Hij zal ook bij geen dode lichamen komen; zelfs over zijn vader en over zijn moeder zal hij zich niet verontreinigen.
\par 12 En uit het heiligdom zal hij niet uitgaan, dat hij het heiligdom zijns Gods niet ontheilige, want de kroon der zalfolie zijns Gods is op hem; Ik ben de HEERE!
\par 13 Hij zal ook een vrouw in haar maagdom nemen.
\par 14 Een weduwe, of verstotene, of ontheiligde hoer, dezulke zal hij niet nemen; maar een maagd uit zijn volken zal hij tot een vrouw nemen.
\par 15 En hij zal zijn zaad onder zijn volken niet ontheiligen; want Ik ben de HEERE, die hem heilige!
\par 16 Wijders sprak de HEERE tot Mozes, zeggende:
\par 17 Spreek tot Aaron, zeggende: Niemand uit uw zaad, naar hun geslachten, in wien een gebrek zal zijn, zal naderen, om de spijze zijns Gods te offeren.
\par 18 Want geen man, in wien een gebrek zal zijn, zal naderen, hij zij een blind man, of kreupel, of te kort, of te lang in leden;
\par 19 Of een man, in wien een breuk des voets, of een breuk der hand zal zijn;
\par 20 Of die bultachtig, of dwergachtig zal zijn, of een vel op zijn oog zal hebben, of droge schurftheid, of etterige schurftheid, of die gebroken zal zijn aan zijn gemacht.
\par 21 Geen man, uit het zaad van Aaron, den priester, in wien een gebrek is, zal toetreden om de vuurofferen des HEEREN te offeren; een gebrek is in hem, hij zal niet toetreden, om de spijs zijns Gods te offeren.
\par 22 De spijs zijns Gods, van de allerheiligste dingen, en van de heilige dingen, zal hij mogen eten;
\par 23 Doch tot den voorhang zal hij niet komen, en tot het altaar niet toetreden, omdat een gebrek in hem is; opdat hij Mijn heiligdommen niet ontheilige; want Ik ben de HEERE, die hen heilige!
\par 24 En Mozes sprak zulks tot Aaron en tot zijn zonen, en tot al de kinderen Israels.

\chapter{22}

\par 1 Daarna sprak de HEERE tot Mozes, zeggende:
\par 2 Spreek tot Aaron en tot zijn zonen, dat zij zich van de heilige dingen der kinderen Israels, die zij Mij heiligen, afzonderen, opdat zij de Naam Mijner heiligheid niet ontheiligen; Ik ben de HEERE!
\par 3 Zeg tot hen: Alle man onder uw geslachten, die uit uw ganse zaad tot de heilige dingen, die de kinderen Israels den HEERE heiligen, naderen zal, als zijn onreinigheid op hem is; diezelve mens zal van voor Mijn aangezicht uitgeroeid worden; Ik ben de HEERE!
\par 4 Niemand van het zaad van Aaron, die melaats is, of een vloed heeft, zal van die heilige dingen eten, totdat hij rein is; mitsgaders die iets aanroert, dat onrein is van een dood lichaam, of iemand, wien het zaad der bijligging ontgaat.
\par 5 Of zo wie aangeroerd zal hebben enig kruipend gedierte, waarvan hij onrein is, of een mens, waarvan hij onrein is, naar al zijn onreinigheid;
\par 6 De mens, die dat aangeroerd zal hebben, zal onrein zijn tot aan den avond, en hij zal van die heilige dingen niet eten, maar zal zijn vlees met water baden.
\par 7 Als de zon zal ondergegaan zijn, dan zal hij rein zijn; en daarna zal hij van die heilige dingen eten; want dat is zijn spijze.
\par 8 Het dode aas, en het verscheurde zal hij niet eten, om daarmede onrein te worden; Ik ben de HEERE!
\par 9 Zij zullen dan Mijn bevel onderhouden, opdat zij geen zonde daarover dragen en daarin sterven, als zij die ontheiligd zouden hebben; Ik ben de HEERE, die hen heilige!
\par 10 Ook zal geen vreemde het heilige eten; een bijwoner des priesters, en een dagloner, zullen het heilige niet eten.
\par 11 Wanneer dan nog de priester een ziel met zijn geld zal gekocht hebben, die zal daarvan eten; en de ingeborene van zijn huis, die zullen van zijn spijze eten.
\par 12 Maar als des priesters dochter een vreemden man zal toebehoren, zij zal van het hefoffer der heilige dingen niet eten.
\par 13 Doch als des priesters dochter een weduwe of een verstotene zal zijn, en geen zaad hebben, en tot haars vaders huis, als in haar jonkheid, zal wedergekeerd zijn, zo zal zij van de spijze haars vaders eten; maar geen vreemde zal daarvan eten.
\par 14 En wanneer iemand het heilige door dwaling zal gegeten hebben, zo zal hij deszelfs vijfde deel daarboven toedoen, en zal het den priester met het heilige wedergeven.
\par 15 Zo zullen zij niet ontheiligen de heilige dingen der kinderen Israels, die zij den HEERE zullen gegeven hebben;
\par 16 En hen doen dragen de ongerechtigheid der schuld, als zij hun heilige dingen zouden eten; want Ik ben de HEERE, Die hen heilige!
\par 17 Verder sprak de HEERE tot Mozes, zeggende:
\par 18 Spreek tot Aaron, en tot zijn zonen, en tot al de kinderen Israels, en zeg tot hen: Zo wie uit het huis van Israel, en uit de vreemdelingen in Israel is, die zijn offerande zal offeren naar al hun geloften, en naar al hun vrijwillige offeren, die zij den HEERE ten brandoffer zullen offeren;
\par 19 Het zal naar uw welgevallen zijn, een volkomen mannetje, van de runderen, van de lammeren, of van de geiten.
\par 20 Gij zult niet offeren iets, waarin een gebrek is; want het zou niet aangenaam zijn voor u.
\par 21 En als iemand een dankoffer den HEERE zal offeren, uitzonderende van de runderen of van de schapen een gelofte, of vrijwillig offer, het zal volkomen zijn, opdat het aangenaam zij; geen gebrek zal daarin zijn.
\par 22 Het blinde, of gebrokene, of verlamde, of wratte, of droge schurftheid, of etterige schurftheid hebbende, deze zult gij den HEERE niet offeren, en daarvan zult gij den HEERE geen vuuroffer op het altaar geven.
\par 23 Doch een os, of klein vee, te lang of te verkrompen in leden, die zult gij tot een vrijwillig offer bereiden; doch tot een gelofte zou het niet aangenaam zijn.
\par 24 Het gedrukte, of gestotene, of gescheurde, of gesnedene, zult gij den HEERE niet offeren; dat zult gij in uw land niet doen.
\par 25 Gij zult ook uit de hand des vreemden van al deze dingen uw God geen spijs offeren; want hun verdorvenheid is in hen, in dezelve is gebrek, zij zouden niet aangenaam zijn voor u.
\par 26 Wijders sprak de HEERE tot Mozes, zeggende:
\par 27 Wanneer een os, of lam, of geit zal geboren zijn, zo zal die zeven dagen onder zijn moeder zijn; daarna, van den achtsten dag en daarover, zal hij aangenaam zijn tot offerande des vuuroffers den HEERE.
\par 28 Gij zult ook een os, of klein vee, hem en zijn jong, op een dag niet slachten.
\par 29 En als gij een lofoffer den HEERE zult slachten, naar uw wil zult gij het slachten.
\par 30 Het zal op denzelfden dag gegeten worden; gij zult daarvan niet overlaten tot op den morgen; Ik ben de HEERE!
\par 31 Daarom zult gij Mijn geboden houden, en dezelve doen; Ik ben de HEERE!
\par 32 En gij zult Mijn heiligen Naam niet ontheiligen, opdat Ik in het midden der kinderen Israels geheiligd worde; Ik ben de HEERE, die u heilige!
\par 33 Die u uit Egypteland uitgevoerd heb, opdat Ik u tot een God zij; Ik ben de HEERE!

\chapter{23}

\par 1 Daarna sprak de HEERE tot Mozes, zeggende:
\par 2 Spreek tot de kinderen Israels, en zeg tot hen: De gezette hoogtijden des HEEREN, welke gijlieden uitroepen zult, zullen heilige samenroepingen zijn; deze zijn Mijn gezette hoogtijden.
\par 3 Zes dagen zal men het werk doen, maar op den zevenden dag is de sabbat der rust, een heilige samenroeping; geen werk zult gij doen; het is des HEEREN sabbat, in al uw woningen.
\par 4 Deze zijn de gezette hoogtijden des HEEREN, de heilige samenroepingen, welke gij uitroepen zult op hun gezetten tijd.
\par 5 In de eerste maand, op den veertienden der maand, tussen twee avonden is des HEEREN pascha.
\par 6 En op den vijftienden dag der derzelver maand is het feest van de ongezuurde broden des HEEREN; zeven dagen zult gij ongezuurde broden eten.
\par 7 Op den eersten dag zult gij een heilige samenroeping hebben; geen dienstwerk zult gij doen.
\par 8 Maar gij zult zeven dagen vuuroffer den HEERE offeren; en op den zevenden dag zal een heilige samenroeping wezen; geen dienstwerk zult gij doen.
\par 9 En de HEERE sprak tot Mozes, zeggende:
\par 10 Spreek tot de kinderen Israels, en zeg tot hen: Als gij in het land zult gekomen zijn, hetwelk Ik u geven zal, en gij zijn oogst zult inoogsten, dan zult gij een garf der eerstelingen van uw oogst tot den priester brengen.
\par 11 En hij zal die garf voor het aangezicht des HEEREN bewegen, opdat het voor u aangenaam zij; des anderen daags na den sabbat zal de priester die bewegen.
\par 12 Gij zult ook op den dag, als gij die garf bewegen zult, bereiden een volkomen lam, dat eenjarig is, ten brandoffer den HEERE;
\par 13 En zijn spijsoffer twee tienden meelbloem, met olie gemengd, ten vuuroffer, den HEERE tot een liefelijken reuk; en zijn drankoffer van wijn, het vierde deel van een hin.
\par 14 En gij zult geen brood, noch geroost koren, noch groene aren eten, tot op dienzelven dag, dat gij de offerande uws Gods zult gebracht hebben; het is een eeuwige inzetting voor uw geslachten, in al uw woningen.
\par 15 Daarna zult gij u tellen van den anderen dag na den sabbat, van den dag, dat gij de garf des beweegoffers zult gebracht hebben; het zullen zeven volkomen sabbatten zijn;
\par 16 Tot den anderen dag, na den zevenden sabbat, zult gij vijftig dagen tellen, dan zult gij een nieuw spijsoffer den HEERE offeren.
\par 17 Gijlieden zult uit uw woningen twee beweegbroden brengen, zij zullen van twee tienden meelbloem zijn, gedesemd zullen zij gebakken worden; het zijn de eerstelingen den HEERE.
\par 18 Gij zult ook met het brood zeven volkomen eenjarige lammeren, en een var, het jong van een rund, en twee rammen offeren; zij zullen den HEERE een brandoffer zijn, met hun spijsoffer en hun drankofferen, een vuuroffer, tot een liefelijken reuk den HEERE.
\par 19 Ook zult gij een geitenbok ten zondoffer, en twee eenjarige lammeren ten dankoffer bereiden.
\par 20 Dan zal de priester dezelve met het brood der eerstelingen ten beweegoffer, voor het aangezicht des HEEREN, met de twee lammeren bewegen; zij zullen den HEERE een heilig ding zijn, voor den priester.
\par 21 En gij zult op dienzelfden dag uitroepen, dat gij een heilige samenroeping zult hebben; geen dienstwerk zult gij doen; het is een eeuwige inzetting in al uw woningen voor uw geslachten.
\par 22 Als gij nu den oogst uws lands zult inoogsten, gij zult, in uw inoogsten, den hoek des velds niet ganselijk afmaaien, en de opzameling van uw oogst niet opzamelen; voor den arme en voor den vreemdeling zult gij ze laten; Ik ben de HEERE, uw God!
\par 23 En de HEERE sprak tot Mozes, zeggende:
\par 24 Spreek tot de kinderen Israels, zeggende: In de zevende maand, op den eersten der maand, zult gij een rust hebben, een gedachtenis des geklanks, een heilige samenroeping.
\par 25 Geen dienstwerk zult gij doen; maar gij zult den HEERE vuuroffer offeren.
\par 26 Verder sprak de HEERE tot Mozes, zeggende:
\par 27 Doch op den tienden dezer zevende maand zal de verzoendag zijn, een heilige samenroeping zult gij hebben; dan zult gij uw zielen verootmoedigen, en zult den HEERE een vuuroffer offeren.
\par 28 En op dienzelven dag zult gij geen werk doen; want het is de verzoendag, om over u verzoening te doen voor het aangezicht des HEEREN uws Gods.
\par 29 Want alle ziel, welken op dienzelven dag niet zal verootmoedigd zijn geweest, die zal uitgeroeid worden uit haar volken.
\par 30 Ook alle ziel, die enig werk op dienzelven dag gedaan zal hebben, die ziel zal Ik uit het midden haars volks verderven.
\par 31 Gij zult geen werk doen; het is een eeuwige inzetting voor uw geslachten, in al uw woningen.
\par 32 Het zal u een sabbat der rust zijn; dan zult gij uw zielen verootmoedigen; op den negenden der maand in den avond, van den avond tot den avond, zult gij uw sabbat rusten.
\par 33 En de HEERE sprak tot Mozes, zeggende:
\par 34 Spreek tot de kinderen Israels, zeggende: Op den vijftienden dag van deze zevende maand zal het feest der loofhutten zeven dagen den HEERE zijn.
\par 35 Op den eersten dag zal een heilige samenroeping zijn; geen dienstwerk zult gij doen.
\par 36 Zeven dagen zult gij den HEERE vuurofferen offeren; op den achtsten dag zult gij een heilige samenroeping hebben, en zult den HEERE vuuroffer offeren; het is een verbodsdag; gij zult geen dienstwerk doen.
\par 37 Dit zijn de gezette hoogtijden des HEEREN, welke gij zult uitroepen tot heilige samenroepingen, om den HEERE vuuroffer, brandoffer en spijsoffer, slachtoffer en drankofferen, elk dagelijks op zijn dag, te offeren;
\par 38 Behalve de sabbatten des HEEREN, en behalve uw gaven, en behalve al uw geloften, en behalve al uw vrijwillige offeren, welke gij den HEERE geven zult.
\par 39 Doch op den vijftienden dag der zevende maand, als gij het inkomen des lands zult ingegaderd hebben, zult gij des HEEREN feest zeven dagen vieren; op den eersten dag zal er rust zijn, en op den achtsten dag zal er rust zijn.
\par 40 En op den eersten dag zult gij u nemen takken van schoon geboomte, palmtakken, en meien van dichte bomen, met beekwilgen; en gij zult voor het aangezicht des HEEREN, uws Gods, zeven dagen vrolijk zijn.
\par 41 En gij zult dat feest den HEERE zeven dagen in het jaar vieren; het is een eeuwige inzetting voor uw geslachten; in de zevende maand zult gij het vieren.
\par 42 Zeven dagen zult gij in de loofhutten wonen; alle inboorlingen in Israel zullen in loofhutten wonen;
\par 43 Opdat uw geslachten weten, dat Ik de kinderen Israels in loofhutten heb doen wonen, als Ik hen uit Egypteland uitgevoerd heb; Ik ben de HEERE, uw God!
\par 44 Alzo heeft Mozes de gezette hoogtijden des HEEREN tot de kinderen Israels uitgesproken.

\chapter{24}

\par 1 En de HEERE sprak tot Mozes, zeggende:
\par 2 Gebied den kinderen Israels, dat zij tot u brengen zuivere gestoten olijfolie, voor den luchter, om de lampen gedurig aan te steken.
\par 3 Aaron zal die voor het aangezicht des HEEREN gedurig toerichten, van den avond tot den morgen, buiten den voorhang van de getuigenis, in de tent der samenkomst; het is een eeuwige inzetting voor uw geslachten.
\par 4 Hij zal op den louteren kandelaar die lampen voor het aangezicht des HEEREN gedurig toerichten.
\par 5 Gij zult ook meelbloem nemen, en twaalf koeken daarvan bakken; van twee tienden zal een koek zijn.
\par 6 En gij zult ze in twee rijen leggen, zes in een rij, op de reine tafel, voor het aangezicht des HEEREN.
\par 7 En op elke rij zult gij zuiveren wierook leggen, welke het brood ten gedenkoffer zal zijn; het is een vuuroffer den HEERE.
\par 8 Op elken sabbatdag gedurig zal men dat voor het aangezicht des HEEREN toerichten, vanwege de kinderen Israels, tot een eeuwig verbond.
\par 9 En het zal voor Aaron en zijn zonen zijn, die dat in de heilige plaats zullen eten; want het is voor hem een heiligheid der heiligheden uit de vuurofferen des HEEREN, een eeuwige inzetting.
\par 10 En er ging de zoon ener Israelietische vrouw uit, die, in het midden der kinderen Israels, de zoon van een Egyptische man was; en de zoon van deze Israelietische en een Israelietisch man twistten in het leger.
\par 11 Toen lasterde de zoon der Israelietische vrouw uitdrukkelijk den NAAM, en vloekte; daarom brachten zij hem tot Mozes; de naam nu zijner moeder was Selomith, de dochter van Dibri, van den stam Dan.
\par 12 En zij leidden hem in de gevangenis, opdat hem, naar den mond des HEEREN, verklaring geschieden zou.
\par 13 En de HEERE sprak tot Mozes, zeggende:
\par 14 Breng den vloeker uit tot buiten het leger, en allen, die het gehoord hebben, zullen hun handen op zijn hoofd leggen; daarna zal hem de gehele vergadering stenigen.
\par 15 En tot de kinderen Israels zult gij spreken, zeggende: Een ieder, als hij zijn God gevloekt zal hebben, zo zal hij zijn zonde dragen.
\par 16 En wie den Naam des HEEREN gelasterd zal hebben, zal zekerlijk gedood worden; de ganse vergadering zal hem zekerlijk stenigen; alzo zal de vreemdeling zijn, gelijk de inboorling, als hij den NAAM zal gelasterd hebben, hij zal gedood worden.
\par 17 En als iemand enige ziel des mensen zal verslagen hebben, hij zal zekerlijk gedood worden.
\par 18 Maar wie de ziel van enig vee zal verslagen hebben, hij zal het wedergeven, ziel voor ziel.
\par 19 Als ook iemand aan zijn naaste een gebrek zal aangebracht hebben; gelijk als hij gedaan heeft, zo zal ook aan hem gedaan worden:
\par 20 Breuk voor breuk, oog voor oog, tand voor tand; gelijk als hij een gebrek een mens zal aangebracht hebben, zo zal ook hem aangebracht worden.
\par 21 Wie dan enig vee verslaat, die zal het wedergeven; maar wie een mens verslaat, die zal gedood worden.
\par 22 Enerlei recht zult gij hebben; zo zal de vreemdeling zijn, als de inboorling; want Ik ben de HEERE, uw God!
\par 23 En Mozes zeide tot de kinderen Israels, dat zij den vloeker tot buiten het leger uitbrengen, en hem met stenen stenigen zouden. En de kinderen Israels deden, gelijk als de HEERE Mozes geboden had.

\chapter{25}

\par 1 Verder sprak de HEERE tot Mozes, aan den berg Sinai, zeggende:
\par 2 Spreek tot de kinderen Israels, en zeg tot hen: Wanneer gij zult gekomen zijn in dat land, dat Ik u geve, dan zal dat land rusten, een sabbat den HEERE.
\par 3 Zes jaren zult gij uw akker bezaaien, en zes jaren uw wijngaard besnijden, en de inkomst daarvan inzamelen.
\par 4 Doch in het zevende jaar zal voor het land een sabbat der rust zijn, een sabbat den HEERE; uw akker zult gij niet bezaaien en uw wijngaard niet besnijden.
\par 5 Wat van zelf van uw oogst zal gewassen zijn, zult gij niet inoogsten, en de druiven uwer afzondering zult gij niet afsnijden; het zal een jaar der ruste voor het land zijn.
\par 6 En de inkomst van den sabbat des lands zal voor u tot spijze zijn, voor u, en voor uw knecht, en voor uw dienstmaagd, en voor uw dagloner, en voor uw bijwoner, die bij u als vreemdelingen verkeren;
\par 7 Mitsgaders voor het vee, en voor het gedierte, dat in uw land is, zal al de inkomst daarvan tot spijze zijn.
\par 8 Gij zult u ook tellen zeven jaarweken, zevenmaal zeven jaren; zodat de dagen der zeven jaarweken u negen en veertig jaren zullen zijn.
\par 9 Daarna zult gij in de zevende maand, op den tienden der maand, de bazuin des geklanks doen doorgaan; op den verzoendag zult gij de bazuin doen doorgaan in uw ganse land.
\par 10 En gij zult dat vijftigste jaar heiligen, en vrijheid uitroepen in het land, voor al zijn inwoners; het zal u een jubeljaar zijn; en gij zult wederkeren een ieder tot zijn bezittingen, en zult wederkeren een ieder tot zijn geslacht.
\par 11 Dit jubeljaar zal u het vijftigste jaar zijn; gij zult niet zaaien, noch inoogsten wat van zelf daarin zal gewassen zijn, noch ook de druiven der afzonderingen in hetzelve afsnijden.
\par 12 Want dat is het jubeljaar; het zal u heilig zijn; gij zult uit het veld de inkomst daarvan eten.
\par 13 Op dat jubeljaar zult gij ieder wederkeren tot zijn bezitting.
\par 14 Daarom, wanneer gij aan uw naaste wat veilbaars verkopen, of uit de hand uws naasten kopen zult, dat niemand de een den ander verdrukke.
\par 15 Naar het getal der jaren, van het jubeljaar af, zult gij van uw naaste kopen, en naar het getal van de jaren der inkomsten zal hij het aan u verkopen.
\par 16 Naar de veelheid der jaren zult gij zijn koop vermeerderen, en naar de weinigheid der jaren zult gij zijn koop verminderen; want hij verkoopt aan u het getal der inkomsten.
\par 17 Dat dan niemand zijn naaste verdrukke; maar vreest voor uw God; want Ik ben de HEERE, uw God!
\par 18 En doet Mijn inzettingen, en houdt Mijn rechten, en doet dezelve; zo zult gij zeker wonen in het land.
\par 19 En het land zal zijn vrucht geven, en gij zult eten tot verzadiging toe; en gij zult zeker daarin wonen.
\par 20 En als gij zoudt zeggen: Wat zullen wij eten in het zevende jaar! Ziet, wij zullen niet zaaien, en onze inkomst niet inzamelen;
\par 21 Zo zal Ik Mijn zegen gebieden over u in het zesde jaar, dat het de inkomst voor drie jaren zal voortbrengen.
\par 22 Het achtste jaar nu zult gij zaaien, en zult van de oude inkomst eten, tot het negende jaar toe; totdat zijn inkomst ingekomen is, zult gij het oude eten.
\par 23 Het land ook zal niet voor altoos verkocht worden; want het land is het Mijne, dewijl gij vreemdelingen en bijwoners bij Mij zijt.
\par 24 Daarom zult gij, in het ganse land uwer bezitting, lossing voor het land toelaten.
\par 25 Wanneer uw broeder zal verarmd zijn, en iets van zijn bezitting verkocht zal hebben, zo zal zijn losser, die hem nabestaande is, komen, en zal het verkochte zijns broeders lossen.
\par 26 En wanneer iemand geen losser zal hebben, maar zijn hand bekomen en hij gevonden zal hebben, zoveel genoeg is tot zijn lossing;
\par 27 Dan zal hij de jaren zijner verkoping rekenen, en het overschot zal hij den man, wien hij het verkocht had, weder uitkeren; en hij zal weder tot zijn bezitting komen.
\par 28 Maar indien zijn hand niet gevonden heeft, wat genoeg is, om aan hem weder uit te keren, zo zal zijn verkochte goed zijn in de hand van deszelfs koper tot het jubeljaar toe; maar in het jubeljaar zal het uitgaan, en hij zal tot zijn bezitting wederkeren.
\par 29 Insgelijks, wanneer iemand een woonhuis in een bemuurde stad zal verkocht hebben, zo zal zijn lossing zijn, totdat het jaar zijner verkoping volkomen zal zijn; in een vol jaar zal zijn lossing wezen.
\par 30 Maar is het, dat het niet gelost wordt, tegen dat hem het gehele jaar zal vervuld zijn, zo zal dat huis, hetwelk in die stad is, die een muur heeft, voor altoos blijven aan hem, die dat gekocht heeft, onder zijn geslachten; het zal in het jubeljaar niet uitgaan.
\par 31 Doch de huizen der dorpen, die rondom geen muur hebben, zullen als het veld des lands gerekend worden; daarvoor zal lossing zijn, en zij zullen in het jubeljaar uitgaan.
\par 32 Aangaande de steden der Levieten, en de huizen der steden hunner bezitting; de Levieten zullen een eeuwige lossing hebben.
\par 33 En als men onder de Levieten lossing zal gedaan hebben, zo zal de koop van het huis en van de stad zijner bezitting in het jubeljaar uitgaan; want de huizen van de steden der Levieten zijn hun bezitting in het midden van de kinderen Israels.
\par 34 Doch het veld van de voorstad hunner steden zal niet verkocht worden; want het is een eeuwige bezitting voor hen.
\par 35 En als uw broeder zal verarmd zijn, en zijn hand bij u wankelen zal, zo zult gij hem vasthouden, zelfs een vreemdeling en bijwoner, opdat hij bij u leve.
\par 36 Gij zult geen woeker noch overwinst van hem nemen; maar gij zult vrezen voor uw God, opdat uw broeder bij u leve.
\par 37 Uw geld zult gij hem niet op woeker geven, en gij zult uw spijze niet op overwinst geven.
\par 38 Ik ben de HEERE, uw God, Die u uit Egypteland gevoerd heb, om u het land Kanaan te geven, opdat Ik u tot een God zij.
\par 39 Desgelijks, wanneer uw broeder bij u zal verarmd zijn, en zich aan u verkocht zal hebben, gij zult hem niet doen dienen den dienst van een slaaf;
\par 40 Als een dagloner, als een bijwoner zal hij bij u zijn; tot het jubeljaar zal hij bij u dienen.
\par 41 Dan zal hij van u uitgaan, hij en zijn kinderen met hem, en hij zal tot zijn geslacht wederkeren, en tot de bezitting zijner vaderen wederkeren.
\par 42 Want zij zijn Mijn dienstknechten, die Ik uit Egypteland uitgevoerd heb; zij zullen niet verkocht worden, gelijk men een slaaf verkoopt.
\par 43 Gij zult geen heerschappij over hem hebben met wreedheid; maar gij zult vrezen voor uw God.
\par 44 Aangaande uw slaaf of uw slavin, die gij zult hebben, die zullen van de volken zijn, die rondom u zijn; van die zult gij een slaaf of een slavin kopen.
\par 45 Gij zult ze ook kopen van de kinderen der bijwoners, die bij u als vreemdelingen verkeren, uit hen en uit hun geslachten, die bij u zullen zijn, die zij in uw land zullen gewonnen hebben; en zij zullen u tot een bezitting zijn.
\par 46 En gij zult u tot bezitters over hen stellen voor uw kinderen na u, opdat zij de bezitting erven; gij zult hen in eeuwigheid doen dienen; maar over uw broeders, de kinderen Israels, een iegelijk over zijn broeder, gij zult over hem geen heerschappij hebben met wreedheid.
\par 47 En wanneer de hand eens vreemdelings en bijwoners, die bij u is, wat bekomen zal hebben, en uw broeder, die bij hem is, verarmd zal zijn, dat hij zich aan den vreemdeling, den bijwoner, die bij u is, of aan den stam van het geslacht des vreemdelings zal verkocht hebben;
\par 48 Nadat hij zich zal verkocht hebben, zal er lossing voor hem zijn; een van zijn broeders zal hem lossen;
\par 49 Of zijn oom, of de zoon zijns ooms, zal hem lossen, of die uit de naasten zijns vleses van zijn geslacht is, zal hem lossen; of heeft zijn hand wat bekomen, dat hij zichzelven losse.
\par 50 En hij zal met zijn koper rekenen van dat jaar af, dat hij zich aan hem verkocht heeft tot het jubeljaar toe; alzo dat het geld zijner verkoping zal zijn naar het getal van de jaren, naar de dagen eens dagloners zal het met hem zijn.
\par 51 Indien nog vele van die jaren zijn, naar die zal hij tot zijn lossing van het geld, waarover hij gekocht is, wedergeven.
\par 52 En indien er nog weinige van die jaren overgebleven zijn, tot aan het jubeljaar, zo zal hij met hem rekenen; naar zijn jaren zal hij zijn lossing wedergeven.
\par 53 Als een dagloner zal hij van jaar tot jaar bij hem zijn; men zal over hem geen heerschappij hebben met wreedheid voor uw ogen.
\par 54 En is het, dat hij hierdoor niet gelost wordt, zo zal hij in het jubeljaar uitgaan, hij en zijn kinderen met hem.
\par 55 Want de kinderen Israels zijn Mij tot dienstknechten; Mijn dienstknechten zijn zij, die Ik uit Egypteland uitgevoerd heb; Ik ben de HEERE, uw God!

\chapter{26}

\par 1 Gij zult ulieden geen afgoden maken; noch gesneden beeld, noch opgericht beeld zult gij u stellen, noch gebeelden steen in uw land zetten, om u daarvoor te buigen; want Ik ben de HEERE, uw God!
\par 2 Mijn sabbatten zult gij houden, en Mijn heiligdom zult gij vrezen; Ik ben de HEERE!
\par 3 Indien gij in Mijn inzettingen wandelen, en Mijn geboden houden, en die doen zult;
\par 4 Zo zal Ik uw regens geven op hun tijd; en het land zal zijn inkomst geven, en het geboomte des velds zal zijn vrucht geven;
\par 5 En de dorstijd zal u reiken tot den wijnoogst, en de wijnoogst zal reiken tot den zaaitijd; en gij zult uw brood eten tot verzadiging toe, en gij zult zeker in uw land wonen.
\par 6 Ook zal Ik vrede geven in het land, dat gij zult te slapen liggen, en niemand zij, die verschrikke; en Ik zal het boos gedierte uit het land doen ophouden, en het zwaard zal door uw land niet doorgaan.
\par 7 En gij zult uw vijanden vervolgen; en zij zullen voor uw aangezicht door het zwaard vallen.
\par 8 Vijf uit u zullen honderd vervolgen, en honderd uit u zullen tien duizend vervolgen; en uw vijanden zullen voor uw aangezicht door het zwaard vallen.
\par 9 En Ik zal Mij tot u wenden, en zal u vruchtbaar maken, en u vermenigvuldigen; en Mijn verbond zal Ik met u bevestigen.
\par 10 En gij zult het oude, dat verouderd is, eten; en het oude zult gij vanwege het nieuwe uitbrengen.
\par 11 En Ik zal Mijn tabernakel in het midden van u zetten; en Mijn ziel zal van u niet walgen.
\par 12 En Ik zal in het midden van u wandelen, en zal u tot een God zijn, en gij zult Mij tot een volk zijn.
\par 13 Ik ben de HEERE, uw God, Die u uit het land der Egyptenaren uitgevoerd heb, opdat gij hun slaven niet zoudt zijn; en Ik heb de disselbomen van uw juk verbroken, en heb u doen rechtop staan.
\par 14 Maar indien gij Mij niet zult horen, en al deze geboden niet zult doen;
\par 15 En zo gij Mijn inzettingen zult smadelijk verwerpen, en zo uw ziel van Mijn rechten zal walgen, dat gij niet doet al Mijn geboden, om Mijn verbond te vernietigen;
\par 16 Dit zal Ik u ook doen, dat Ik over u stellen zal verschrikking, tering en koorts, die de ogen verteren en de ziel pijnigen; gij zult ook uw zaad te vergeefs zaaien, en uw vijanden zullen dat opeten.
\par 17 Daartoe zal Ik Mijn aangezicht tegen ulieden zetten, dat gij geslagen zult worden voor het aangezicht uwer vijanden; en uw haters zullen over u heerschappij hebben, en gij zult vlieden, als u iemand vervolgt.
\par 18 En zo gij Mij tot deze dingen toe nog niet horen zult, Ik zal nog daar toedoen, om u zevenvoudig over uw zonden te tuchtigen.
\par 19 Want Ik zal de hovaardigheid uwer kracht verbreken, en zal uw hemel als ijzer maken, en uw aarde als koper.
\par 20 En uw macht zal ijdellijk verdaan worden; en uw land zal zijn inkomsten niet geven, en het geboomte des lands zal zijn vrucht niet geven.
\par 21 En zo gij met Mij in tegenheid wandelen zult, en Mij niet zult willen horen, zo zal Ik over u, naar uw zonden, zevenvoudig slagen toedoen.
\par 22 Want Ik zal onder u zenden het gedierte des velds, hetwelk u beroven, en uw vee uitroeien, en u verminderen zal; en uw wegen zullen woest worden.
\par 23 Indien gij nog door deze dingen Mij niet getuchtigd zult zijn, maar met Mij in tegenheid wandelen;
\par 24 Zo zal Ik ook met u in tegenheid wandelen, en Ik zal u ook zevenvoudig over uw zonden slaan.
\par 25 Want Ik zal een zwaard over u brengen, dat de wraak des verbonds wreken zal, zodat gij in uw steden vergaderd zult worden; dan zal Ik de pest in het midden van u zenden, en gij zult in de hand des vijands overgegeven worden.
\par 26 Als Ik u den staf des broods zal gebroken hebben, dan zullen tien vrouwen uw brood in een oven bakken, en zullen uw brood bij het gewicht wedergeven; en gij zult eten, maar niet verzadigd worden.
\par 27 Als gij ook hierom Mij niet horen zult, maar met Mij wandelen zult in tegenheid;
\par 28 Zo zal Ik ook met u in heetgrimmige tegenheid wandelen, en Ik zal u ook zevenvoudig over uw zonden tuchtigen.
\par 29 Want gij zult het vlees uwer zonen eten, en het vlees uwer dochteren zult gij eten.
\par 30 En Ik zal uw hoogten verderven, en uw zonnebeelden uitroeien, en zal uw dode lichamen op de dode lichamen uwer drekgoden werpen; en Mijn ziel zal aan u walgen.
\par 31 En Ik zal uw steden een woestijn maken, en uw heiligdommen verwoesten; en Ik zal uw liefelijken reuk niet rieken.
\par 32 Ja, Ik zal dat land verwoesten; dat uw vijanden, die daarin zullen wonen, zich daarover ontzetten zullen.
\par 33 Daartoe zal Ik u onder de heidenen verstrooien; en een zwaard achter u uittrekken; en uw land zal woest, en uw steden zullen een woestijn zijn.
\par 34 Dan zal het land aan zijn sabbatten een welgevallen hebben, al de dagen der verwoesting, en gij zult in het land uwer vijanden zijn; dan zal het land rusten, en aan zijn sabbatten een welgevallen hebben.
\par 35 Al de dagen der verwoesting zal het rusten, overmits het niet rustte in uw sabbatten, als gij daarin woondet.
\par 36 En aangaande de overgeblevenen onder u, Ik zal in hun hart een wekigheid in de landen hunner vijanden laten komen; zodat het geruis van een gedreven blad hen jagen zal, en zij zullen vlieden, gelijk men vliedt voor een zwaard, en zullen vallen, waar niemand is, die jaagt.
\par 37 En zij zullen de een op den ander als voor het zwaard vallen, waar niemand is, die jaagt; en gij zult voor het aangezicht uwer vijanden niet kunnen bestaan.
\par 38 Maar gij zult omkomen onder de heidenen, en het land uwer vijanden zal u verteren.
\par 39 En de overgeblevenen onder u zullen om hun ongerechtigheid in de landen uwer vijanden uitteren; ja, ook om de ongerechtigheden hunner vaderen zullen zij met hen uitteren.
\par 40 Dan zullen zij hun ongerechtigheid belijden, en de ongerechtigheid hunner vaderen met hun overtredingen, waarmede zij tegen Mij overtreden hebben, en ook dat zij met Mij in tegenheid gewandeld hebben.
\par 41 Dat Ik ook met hen in tegenheid gewandeld, en hen in het land hunner vijanden gebracht zal hebben. Zo dan hun onbesneden hart gebogen wordt, en zij dan aan de straf hunner ongerechtigheid een welgevallen hebben;
\par 42 Dan zal Ik gedenken aan Mijn verbond met Jakob, en ook aan Mijn verbond met Izak, en ook aan Mijn verbond met Abraham zal Ik gedenken, en aan het land zal Ik gedenken;
\par 43 Als het land om hunnentwil zal verlaten zijn geweest, en aan zijn sabbatten een welgevallen gehad hebben, wanneer het om hunnentwil verwoest was, en zij aan de straf hunner ongerechtigheid een welgevallen zullen gehad hebben; daarom, en omdat zij Mijn rechten hadden verworpen, en hun ziel van Mijn inzettingen gewalgd had.
\par 44 En hierenboven is dit ook; als zij in het land hunner vijanden zullen zijn, zal Ik hen niet verwerpen, noch van hen walgen, om een einde van hen te maken, vernietigende Mijn verbond met hen; want Ik ben de HEERE, hun God!
\par 45 Maar Ik zal hun ten beste gedenken aan het verbond der voorouderen, die Ik uit Egypteland voor de ogen der heidenen uitgevoerd heb, opdat Ik hun tot een God ware; Ik ben de HEERE!
\par 46 Dit zijn die inzettingen, en die rechten, en die wetten, welke de HEERE gegeven heeft, tussen Zich en tussen de kinderen Israels, op den berg Sinai, door de hand van Mozes.

\chapter{27}

\par 1 Verder sprak de HEERE tot Mozes, zeggende:
\par 2 Spreek tot de kinderen Israels, en zeg tot hen: Wanneer iemand een gelofte zal afgezonderd hebben, naar uw schatting zullen de zielen des HEEREN zijn.
\par 3 Als uw schatting eens mans zal zijn van twintig jaren oud, tot een, die zestig jaren oud is; dan zal uw schatting zijn van vijftig sikkelen zilvers, naar den sikkel des heiligdoms.
\par 4 Maar is het een vrouw, dan zal uw schatting zijn dertig sikkelen.
\par 5 En is het van een, die vijf jaren oud is, tot een, die twintig jaren oud is, zo zal uw schatting van een man twintig sikkelen zijn, en voor een vrouw tien sikkelen.
\par 6 Maar is het van een, die een maand oud is, tot een, die vijf jaren oud is, zo zal uw schatting van een man zijn vijf sikkelen zilvers, en uw schatting over een vrouw zal zijn drie sikkelen zilvers.
\par 7 En is het van een, die zestig jaren oud is en daarboven, is het een man, zo zal uw schatting zijn vijftien sikkelen, en voor een vrouw tien sikkelen.
\par 8 Maar zo hij armer is, dan uw schatting, zo zal hij zich voor het aangezicht des priesters zetten, opdat de priester hem schatte; naar dat de hand desgenen, die de gelofte gedaan heeft, zal kunnen bekomen, zal de priester hem schatten.
\par 9 En indien het een beest is, waarvan men den HEERE offerande offert; al wat hij daarvan den HEERE zal gegeven hebben, zal heilig zijn.
\par 10 Hij zal niet vermangelen, noch hetzelve verwisselen, een goed voor een kwaad, of een kwaad voor een goed; indien hij nochtans een beest voor een beest enigszins verwisselt, zo zal dit, en wat daarvoor verwisseld is, heilig zijn.
\par 11 En indien het enig onrein beest is, van hetwelk men den HEERE geen offerande offert, zo zal hij dat beest voor het aangezicht des priesters zetten.
\par 12 En de priester zal dat schatten, naar dat het goed of kwaad is; naar uw schatting, priester! zo zal het zijn.
\par 13 Maar indien hij het immers lossen zal, zo zal hij deszelfs vijfde deel boven uw schatting toedoen.
\par 14 En wanneer iemand zijn huis zal geheiligd hebben, dat het den HEERE heilig zij, zo zal de priester dat schatten, naar dat het goed of kwaad is; gelijk als de priester dat geschat zal hebben, zo zal het stand hebben.
\par 15 En indien hij, die het geheiligd heeft, zijn huis zal lossen, zo zal hij een vijfde deel des gelds uwer schatting daarboven toedoen, zo zal het zijne zijn.
\par 16 Indien ook iemand van den akker zijner bezitting den HEERE wat geheiligd zal hebben, zo zal uw schatting zijn naar zijn zaad; een homer gerstezaad zal zijn op vijftig sikkelen zilvers.
\par 17 Indien hij zijn akker van het jubeljaar af geheiligd zal hebben, zo zal het naar uw schatting stand hebben.
\par 18 Maar zo hij zijn akker na het jubeljaar geheiligd zal hebben, dan zal hem de priester het geld rekenen, naar de jaren, die nog overig zijn tot het jubeljaar; en het zal van uw schatting afgetrokken worden.
\par 19 En indien hij, die den akker geheiligd heeft, denzelven ganselijk lossen zal, zo zal hij een vijfde deel des gelds uwer schatting daarboven toedoen, en dezelve zal hem gevestigd zijn.
\par 20 En indien hij dien akker niet zal lossen, of indien hij dien akker aan een anderen man verkocht heeft, zo zal hij niet meer gelost worden.
\par 21 Maar die akker, nadat hij in het jubeljaar zal uitgegaan zijn, zal den HEERE heilig zijn, als een verbannen akker; de bezitting daarvan zal des priesters zijn.
\par 22 En indien hij den HEERE een akker heeft geheiligd, dien hij gekocht heeft, en niet is van den akker zijner bezitting;
\par 23 Zo zal de priester hem rekenen de som uwer schatting tot het jubeljaar; en hij zal op denzelven dag uw schatting geven, een heiligheid den HEERE.
\par 24 In het jubeljaar zal die akker wederkomen tot dien, van wien hij hem gekocht had, tot hem, wiens de bezitting van dat land was.
\par 25 Al uw schatting nu zal naar den sikkel des heiligdoms geschieden; de sikkel zal zijn van twintig gera.
\par 26 Maar het eerstgeborene, dat den HEERE van een beest eerstgeboren wordt, dat zal niemand heiligen; hetzij een os, of klein vee, het is des HEEREN.
\par 27 Doch is het van een onrein beest, hij zal dat lossen naar uw schatting, en zal zijn vijfde deel daarboven toedoen; en indien het niet gelost wordt, zo zal het verkocht worden, naar uw schatting.
\par 28 Evenwel niets, dat verbannen is, dat iemand den HEERE zal verbannen hebben, van al hetgeen hij heeft, van een mens, of van een beest, of van den akker zijner bezitting, zal verkocht noch gelost worden; al wat verbannen is, zal den HEERE een heiligheid der heiligheden zijn.
\par 29 Al wat verbannen is, dat van de mensen zal verbannen zijn, zal niet gelost worden; het zal zekerlijk gedood worden.
\par 30 Ook alle tienden des lands, van het zaad des lands, van de vrucht van het geboomte, zijn des HEEREN; zij zijn den HEERE heilig.
\par 31 Maar zo iemand van zijn tienden immer iets lossen zal, hij zal zijn vijfde deel daarboven toedoen.
\par 32 Aangaande al de tienden van runderen en klein vee, alles wat onder de roede zal doorgaan, het tiende zal den HEERE heilig zijn.
\par 33 Hij zal tussen het goede en het kwade niet onderzoeken; hij zal het ook niet verwisselen; maar indien hij het immers verwisselen zal, zo zal dit, en wat daarvoor verwisseld is, heilig zijn; het zal niet gelost worden.
\par 34 Dit zijn de geboden, die de HEERE Mozes geboden heeft, aan de kinderen Israels, op den berg Sinai.



\end{document}