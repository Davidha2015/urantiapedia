\begin{document}

\title{Numbers}


Num 1:1  Voorts sprak de HEERE tot Mozes, in de woestijn van Sinai, in de tent der samenkomst, op den eersten der tweede maand, in het tweede jaar, nadat zij uit Egypteland uitgetogen ware, zeggende:
Num 1:2  Neem op de som van de gehele vergadering der kinderen Israels, naar hun geslachten, naar het huis hunner vaderen, in het getal der namen, van al wat mannelijk is, hoofd voor hoofd.
Num 1:3  Van twintig jaren oud en daarboven, allen, die ten heire in Israel uittrekken; die zult gij tellen naar hun heiren, gij en Aaron.
Num 1:4  En met ulieden zullen zijn van elken stam een man, die een hoofdman is over het huis zijner vaderen.
Num 1:5  Deze zijn nu de namen der mannen, die bij u staan zullen: van Ruben, Elizur, de zoon van Sedeur.
Num 1:6  Van Simeon, Selumiel, de zoon van Zurisaddai.
Num 1:7  Van Juda, Nahesson, de zoon van Amminadab.
Num 1:8  Van Issaschar, Nethaneel, de zoon van Zuar.
Num 1:9  Van Zebulon, Eliab, de zoon van Helon.
Num 1:10  Van de kinderen van Jozef: van Efraim, Elisama, de zoon van Ammihud; van Manasse, Gamaliel, de zoon van Pedazur.
Num 1:11  Van Benjamin, Abidan, de zoon van Gideoni.
Num 1:12  Van Dan, Ahiezer, de zoon van Ammisaddai.
Num 1:13  Van Aser, Pagiel, de zoon van Ochran.
Num 1:14  Van Gad, Eljasaf, de zoon van Dehuel.
Num 1:15  Van Nafthali, Ahira, de zoon van Enan.
Num 1:16  Dezen waren de geroepenen der vergadering, de oversten der stammen hunner vaderen; zij waren de hoofden der duizenden van Israel.
Num 1:17  Toen namen Mozes en Aaron die mannen, welken met namen uitgedrukt zijn.
Num 1:18  En zij verzamelden de gehele vergadering, op den eersten dag der tweede maand; en die verklaarden hun afkomst, naar hun geslachten, naar het huis hunner vaderen, in het getal der namen, van die twintig jaren oud was en daarboven, hoofd voor hoofd.
Num 1:19  Gelijk als de HEERE Mozes geboden had, zo heeft hij hen geteld in de woestijn van Sinai.
Num 1:20  Zo waren de zonen van Ruben, den eerstgeborene van Israel, hun geboorten, naar hun geslachten, naar het huis hunner vaderen, in het getal der namen, hoofd voor hoofd, al wat mannelijk was, van twintig jaren oud en daarboven, allen, die ten heire uittrokken;
Num 1:21  Hun getelden van den stam van Ruben waren zes en veertig duizend en vijfhonderd.
Num 1:22  Van de zonen van Simeon, hun geboorten, naar hun geslachten, naar het huis hunner vaderen, zijn getelden, in het getal der namen, hoofd voor hoofd, al wat mannelijk was, van twintig jaren oud en daarboven, allen, die ten heire uittrokken;
Num 1:23  Hun getelden van den stam van Simeon waren negen en vijftig duizend en driehonderd.
Num 1:24  Van de zonen van Gad, hun geboorten, naar hun geslachten, naar het huis hunner vaderen, in het getal der namen, van twintig jaren oud en daarboven, allen, die ten heire uittrokken.
Num 1:25  Waren hun getelden van den stam van Gad vijf en veertig duizend zeshonderd en vijftig.
Num 1:26  Van de zonen van Juda, hun geboorten, naar hun geslachten, naar het huis hunner vaderen, in het getal der namen, van twintig jaren oud en daarboven, allen, die ten heire uittrokken,
Num 1:27  Waren hun getelden van den stam van Juda vier en zeventig duizend en zeshonderd.
Num 1:28  Van de zonen van Issaschar, hun geboorten, naar hun geslachten, naar het huis hunner vaderen, in het getal der namen van twintig jaren oud en daarboven, allen, die ten heire uittrokken,
Num 1:29  Waren hun getelden van den stam van Issaschar vier en vijftig duizend en vierhonderd.
Num 1:30  Van de zonen van Zebulon, hun geboorten, naar hun geslachten, naar het huis hunner vaderen, in het getal der namen, van twintig jaren oud en daarboven, allen, die ten heire uittrokken,
Num 1:31  Waren hun getelden van den stam van Zebulon zeven en vijftig duizend en vierhonderd.
Num 1:32  Van de zonen van Jozef: van de zonen van Efraim, hun geboorten, naar hun geslachten, naar het huis hunner vaderen, in het getal der namen, van twintig jaren oud en daarboven, allen, die ten heire uittrokken,
Num 1:33  Waren hun getelden van den stam van Efraim veertig duizend en vijfhonderd;
Num 1:34  Van de zonen van Manasse, hun geboorten, naar hun geslachten, naar het huis hunner vaderen, in het getal der namen, van twintig jaren oud en daarboven, allen, die ten heire uittrokken,
Num 1:35  Waren hun getelden van den stam van Manasse twee en dertig duizend en tweehonderd.
Num 1:36  Van de zonen van Benjamin, hun geboorten, naar hun geslachten, naar het huis hunner vaderen, in het getal der namen, van twintig jaren oud en daarboven, allen, die ten heire uittrokken,
Num 1:37  Waren hun getelden van den stam van Benjamin vijf en dertig duizend en vierhonderd.
Num 1:38  Van de zonen van Dan, hun geboorten, naar hun geslachten, naar het huis hunner vaderen, in het getal der namen, van twintig jaren oud en daarboven, allen, die ten heire uittrokken,
Num 1:39  Waren hun getelden van den stam van Dan twee en zestig duizend en zevenhonderd.
Num 1:40  Van de zonen van Aser, hun geboorten, naar hun geslachten, naar het huis hunner vaderen, in het getal der namen, van twintig jaren oud en daarboven, allen, die ten heire uittrokken,
Num 1:41  Waren hun getelden van den stam van Aser een en veertig duizend en vijfhonderd.
Num 1:42  Van de zonen van Nafthali, hun geboorten, naar hun geslachten, naar het huis hunner vaderen, in het getal der namen, van twintig jaren oud en daarboven, allen, die ten heire uittrokken,
Num 1:43  Waren hun getelden van den stam van Nafthali drie en vijftig duizend en vierhonderd.
Num 1:44  Dezen zijn de getelden, welke Mozes geteld heeft, en Aaron, en de oversten van Israel; twaalf mannen waren zij, elk over het huis zijner vaderen.
Num 1:45  Alzo waren al de getelden der zonen van Israel, naar het huis hunner vaderen, van twintig jaren oud en daarboven, allen, die in Israel ten heire uittrokken,
Num 1:46  Al de getelden dan waren zeshonderd drie duizend vijfhonderd en vijftig.
Num 1:47  Maar de Levieten, naar den stam hunner vaderen, werden onder hen niet geteld.
Num 1:48  Want de HEERE had tot Mozes gesproken, zeggende:
Num 1:49  Alleen den stam van Levi zult gij niet tellen, noch hun som opnemen, onder de zonen van Israel.
Num 1:50  Maar gij, stel de Levieten over den tabernakel der getuigenis, en over al zijn gereedschap, en over alles, wat daartoe behoort; zij zullen den tabernakel dragen, en al zijn gereedschap; en zij zullen dien bedienen, en zij zullen zich rondom den tabernakel legeren.
Num 1:51  En als de tabernakel zal optrekken, de Levieten zullen denzelven afnemen; en wanneer de tabernakel zich legeren zal, zullen de Levieten denzelven oprichten; en de vreemde, die daarbij komt, zal gedood worden.
Num 1:52  En de kinderen Israels zullen zich legeren, een iegelijk bij zijn leger, en een iegelijk bij zijn banier, naar hun heiren.
Num 1:53  Maar de Levieten zullen zich legeren rondom den tabernakel der getuigenis, opdat geen verbolgenheid over de vergadering van de kinderen Israels zij; daarom zullen de Levieten de wacht van den tabernakel der getuigenis waarnemen.
Num 1:54  Zo deden de kinderen Israels; naar alles, wat de HEERE Mozes geboden had, zo deden zij.
Num 2:1  En de HEERE sprak tot Mozes en tot Aaron, zeggende:
Num 2:2  De kinderen Israels zullen zich legeren, een ieder onder zijn banier, naar de tekenen van het huis hunner vaderen; rondom tegenover de tent der samenkomst zullen zij zich legeren.
Num 2:3  Die zich nu legeren zullen oostwaarts tegen den opgang, zal zijn de banier des legers van Juda, naar hun heiren; en Nahesson, de zoon van Amminadab, zal de overste der zonen van Juda zijn.
Num 2:4  Zijn heir nu, en zijn getelden waren vier en zeventig duizend en zeshonderd.
Num 2:5  En nevens zal zich legeren de stam van Issaschar; en Nethaneel, de zoon van Zuar, zal de overste der zonen van Issaschar zijn.
Num 2:6  Zijn heir nu, en zijn getelden waren vier en vijftig duizend en vierhonderd.
Num 2:7  Daartoe de stam van Zebulon; en Eliab, de zoon van Helon, zal de overste der zonen van Zebulon zijn.
Num 2:8  Zijn heir nu, en zijn getelden waren zeven en vijftig duizend en vierhonderd.
Num 2:9  Al de getelden des legers van Juda waren honderd zes en tachtig duizend en vierhonderd, naar hun heiren. Zij zullen vooraan optrekken.
Num 2:10  De banier des legers van Ruben, naar hun heiren, zal tegen het zuiden zijn; en Elizur, de zoon van Sedeur, zal de overste der zonen van Ruben zijn.
Num 2:11  Zijn heir nu, en zijn getelden waren zes en veertig duizend en vijfhonderd.
Num 2:12  En nevens hem zal zich legeren de stam van Simeon; en Selumiel, de zoon van Zurisaddai, zal de overste der zonen van Simeon zijn.
Num 2:13  Zijn heir nu, en zijn getelden waren negen en vijftig duizend en driehonderd.
Num 2:14  Daartoe de stam van Gad; en Eljasaf, de zoon van Rehuel, zal de overste der zonen van Gad zijn.
Num 2:15  Zijn heir nu, en zijn getelden waren vijf en veertig duizend zeshonderd en vijftig.
Num 2:16  Al de getelden in het leger van Ruben waren honderd een en vijftig duizend vierhonderd en vijftig; naar hun heiren. En zij zullen de tweede optrekken.
Num 2:17  Daarna zal de tent der samenkomst optrekken, met het leger der Levieten, in het midden van de legers; gelijk als zij zich legeren zullen, alzo zullen zij optrekken, een iegelijk aan zijn plaats, naar hun banieren.
Num 2:18  De banier des legers van Efraim, naar hun heiren, zal tegen het westen zijn; en Elisama, de zoon van Ammihud, zal de overste der zonen van Efraim zijn.
Num 2:19  Zijn heir nu, en zijn getelden waren veertig duizend en vijfhonderd.
Num 2:20  En nevens hem de stam van Manasse; en Gamaliel, de zoon van Pedazur, zal de overste der zonen van Manasse zijn.
Num 2:21  Zijn heir nu, en zijn getelden waren twee en dertig duizend en tweehonderd.
Num 2:22  Daartoe de stam van Benjamin; en Abidan, de zoon van Gideoni, zal de overste der zonen van Benjamin zijn.
Num 2:23  Zijn heir nu, en zijn getelden waren vijf en dertig duizend en vierhonderd.
Num 2:24  Al de getelden in het leger van Efraim waren honderd acht duizend en eenhonderd, naar hun heiren. En zij zullen de derde optrekken.
Num 2:25  De banier des legers van Dan zal tegen het noorden zijn, naar hun heiren; en Ahiezer, de zoon van Ammisaddai, zal de overste der zonen van Dan zijn.
Num 2:26  Zijn heir nu, en zijn getelden waren twee en zestig duizend en zevenhonderd.
Num 2:27  En nevens hem zal zich legeren de stam van Aser; en Pagiel, de zoon van Ochran, zal de overste der zonen van Aser zijn.
Num 2:28  Zijn heir nu, en zijn getelden waren een en veertig duizend en vijfhonderd.
Num 2:29  Daartoe de stam van Nafthali; en Ahira, de zoon van Enan, zal de overste der zonen van Nafthali zijn.
Num 2:30  Zijn heir nu, en zijn getelden waren drie en vijftig duizend en vierhonderd.
Num 2:31  Al de getelden in het leger van Dan waren honderd zeven en vijftig duizend en zeshonderd. In het achterste zullen zij optrekken, naar hun banieren.
Num 2:32  Dezen zijn de getelden van de kinderen Israels, naar het huis hunner vaderen; al de getelden der legers, naar hun heiren, waren zeshonderd drie duizend vijfhonderd en vijftig.
Num 2:33  Maar de Levieten werden niet geteld onder de zonen van Israel, gelijk als de HEERE Mozes geboden had.
Num 2:34  En de kinderen Israels deden naar alles, wat de HEERE Mozes geboden had, zo legerden zij zich naar hun banieren, en zo trokken zij op, een iegelijk naar zijn geslachten, naar het huis zijner vaderen.
Num 3:1  Dit nu zijn de geboorten van Aaron en Mozes; ten dage als de HEERE met Mozes gesproken heeft op den berg Sinai.
Num 3:2  En dit zijn de namen der zonen van Aaron: de eerstgeborene, Nadab, daarna Abihu, Eleazar, en Ithamar.
Num 3:3  Dit zijn de namen der zonen van Aaron, der priesteren, die gezalfd waren, welker hand men gevuld had, om het priesterambt te bedienen.
Num 3:4  Maar Nadab en Abihu stierven voor het aangezicht des HEEREN, als zij vreemd vuur voor het aangezicht des HEEREN in de woestijn van Sinai brachten, en hadden geen kinderen, doch Eleazar en Ithamar bedienden het priesterambt voor het aangezicht van hun vader Aaron.
Num 3:5  En de HEERE sprak tot Mozes, zeggende:
Num 3:6  Doe den stam van Levi naderen, en stel hem voor het aangezicht van den priester Aaron, opdat zij hem dienen;
Num 3:7  En dat zij waarnemen zijn wacht, en de wacht der gehele vergadering, voor de tent der samenkomst, om den dienst des tabernakels te bedienen;
Num 3:8  En dat zij al het gereedschap van de tent der samenkomst, en de wacht der kinderen Israels waarnemen, om den dienst des tabernakels te bedienen.
Num 3:9  Gij zult dan, aan Aaron en aan zijn zonen, de Levieten geven; zij zijn gegeven, zij zijn hem gegeven uit de kinderen Israels.
Num 3:10  Maar Aaron en zijn zonen zult gij stellen, dat zij hun priesterambt waarnemen; en de vreemde, die nadert, zal gedood worden.
Num 3:11  En de HEERE sprak tot Mozes, zeggende:
Num 3:12  En Ik, zie, Ik heb de Levieten uit het midden van de kinderen Israels genomen, in plaats van allen eerstgeborene, die de baarmoeder opent, uit de kinderen Israels; en de Levieten zullen Mijne zijn.
Num 3:13  Want alle eerstgeborene is Mijn; van den dag, dat Ik alle eerstgeborenen in Egypteland sloeg, heb Ik Mij geheiligd alle eerstgeborenen in Israel, van de mensen tot de beesten; zij zullen Mijn zijn; Ik ben de HEERE!
Num 3:14  En de HEERE sprak tot Mozes in de woestijn van Sinai, zeggende:
Num 3:15  Tel de zonen van Levi naar het huis hunner vaderen, naar hun geslachten, al wat mannelijk is, van een maand oud en daarboven, die zult gij tellen.
Num 3:16  En Mozes telde hen naar het bevel des HEEREN, gelijk als hem geboden was.
Num 3:17  Dit nu waren de zonen van Levi met hun namen: Gerson, en Kahath, en Merari.
Num 3:18  En dit zijn de namen der zonen van Gerson, naar hun geslachten: Libni en Simei.
Num 3:19  En de zonen van Kahath, naar hun geslachten; Amram en Izhar, Hebron en Uzziel.
Num 3:20  En de zonen van Merari, naar hun geslachten: Maheli en Musi; dit zijn de geslachten der Levieten, naar het huis hunner vaderen.
Num 3:21  Van Gerson was het geslacht der Libnieten, en het geslacht der Simeieten; dit zijn de geslachten der Gersonieten.
Num 3:22  Hun getelden in getal waren van al wat mannelijk was, van een maand oud en daarboven; hun getelden waren zeven duizend en vijfhonderd.
Num 3:23  De geslachten der Gersonieten zullen zich legeren achter den tabernakel, westwaarts.
Num 3:24  De overste nu van het vaderlijke huis der Gersonieten zal zijn Eljasaf, de zoon van Lael.
Num 3:25  En de wacht der zonen van Gerson in de tent der samenkomst zal zijn de tabernakel en de tent, haar deksel, en het deksel aan de deur van de tent der samenkomst;
Num 3:26  En de behangselen des voorhofs, en het deksel van de deur des voorhofs, welke bij den tabernakel en bij het altaar rondom zijn; mitsgaders de zelen, tot zijn gansen dienst.
Num 3:27  En van Kahath is het geslacht der Amramieten, en het geslacht der Izharieten, en het geslacht der Hebronieten, en het geslacht der Uzzielieten; dit zijn de geslachten der Kohathieten.
Num 3:28  In getal van al wat mannelijk was, van een maand oud en daarboven, waren acht duizend en zeshonderd, waarnemende de wacht des heiligdoms.
Num 3:29  De geslachten der zonen van Kohath zullen zich legeren aan de zijde des tabernakels, zuidwaarts.
Num 3:30  De overste nu van het vaderlijke huis der geslachten van de Kohathieten, zal zijn Elisafan, de zoon van Uzziel.
Num 3:31  Hun wacht nu zal zijn de ark, en de tafel, en de kandelaar, en de altaren en het gereedschap des heiligdoms, met hetwelk zij dienst doen, en het deksel, en al wat tot zijn dienst behoort.
Num 3:32  De overste nu der oversten van Levi zal zijn Eleazar, de zoon van Aaron, den priester; zijn opzicht zal zijn over degenen, die de wacht des heiligdoms waarnemen.
Num 3:33  Van Merari is het geslacht der Mahelieten, en het geslacht der Musieten; dit zijn de geslachten van Merari.
Num 3:34  En hun getelden in getal van al wat mannelijk was, van een maand oud en daarboven, waren zes duizend en tweehonderd.
Num 3:35  De overste nu van het vaderlijke huis der geslachten van Merari zal zijn Zuriel, de zoon van Abihail; zij zullen zich legeren aan de zijde des tabernakels, noordwaarts.
Num 3:36  En het opzicht der wachten van de zonen van Merari zal zijn over de berderen des tabernakels, en zijn richelen, en zijn pilaren, en zijn voeten, en al zijn gereedschap, en al wat tot zijn dienst behoort;
Num 3:37  En de pilaren des voorhofs rondom, en hun voeten, en hun pennen, en hun zelen.
Num 3:38  Die nu zich legeren zullen voor den tabernakel oostwaarts, voor de tent der samenkomst, tegen den opgang, zullen zijn Mozes, en Aaron met zijn zonen, waarnemende de wacht des heiligdoms, voor de wacht der kinderen Israels; en de vreemde, die nadert, zal gedood worden.
Num 3:39  Alle getelden der Levieten, welke Mozes en Aaron, op het bevel des HEEREN, naar hun geslachten, geteld hebben, al wat mannelijk was, van een maand oud en daarboven, waren twee en twintig duizend.
Num 3:40  En de HEERE zeide tot Mozes: Tel alle eerstgeborenen, wat mannelijk is onder de kinderen Israels, van een maand oud en daarboven; en neem het getal hunner namen op.
Num 3:41  En gij zult voor Mij de Levieten nemen (Ik ben de HEERE!), in plaats van alle eerstgeborenen onder de kinderen Israels, en de beesten der Levieten, in plaats van alle eerstgeborenen onder de beesten der kinderen Israels.
Num 3:42  Mozes dan telde, gelijk als de HEERE hem geboden had, alle eerstgeborenen onder de kinderen Israels.
Num 3:43  En alle eerstgeborenen, die mannelijk waren, in het getal der namen, van een maand oud en daarboven, naar hun getelden, waren twee en twintig duizend tweehonderd en drie en zeventig.
Num 3:44  En de HEERE sprak tot Mozes, zeggende:
Num 3:45  Neem de Levieten, in plaats van alle eerstgeboorte onder de kinderen Israels, en de beesten der Levieten, in plaats van hun beesten; want de Levieten zullen Mijn zijn; Ik ben de HEERE!
Num 3:46  Aangaande de tweehonderd drie en zeventig, die gelost zullen worden, die overschieten, boven de Levieten, van de eerstgeborenen van de kinderen Israels;
Num 3:47  Gij zult voor elk hoofd vijf sikkels nemen; naar den sikkel des heiligdoms zult gij ze nemen; die sikkel is twintig gera.
Num 3:48  En gij zult dat geld aan Aaron en zijn zonen geven, het geld der gelosten die onder hen overschieten.
Num 3:49  Toen nam Mozes dat losgeld van degenen, die overschoten boven de gelosten door de Levieten.
Num 3:50  Van de eerstgeborenen van de kinderen Israels nam hij dat geld, duizend driehonderd vijf en zestig sikkelen, naar den sikkel des heiligdoms.
Num 3:51  En Mozes gaf dat geld der gelosten aan Aaron en aan zijn zonen, naar het bevel des HEEREN, gelijk als de HEERE Mozes geboden had.
Num 4:1  En de HEERE sprak tot Mozes en tot Aaron, zeggende:
Num 4:2  Neemt op de som der zonen van Kohath, uit het midden der zonen van Levi, naar hun geslachten, naar het huis hunner vaderen.
Num 4:3  Van dertig jaren oud en daarboven, tot vijftig jaren oud; al wie tot dezen strijd inkomt, om het werk in de tent der samenkomst te doen.
Num 4:4  Dit zal de dienst zijn der zonen van Kohath, in de tent der samenkomst, te weten de heiligheid der heiligheden.
Num 4:5  In het optrekken des legers, zo zullen Aaron en zijn zonen komen, en den voorhang des deksels afnemen, en zullen daarmede de ark der getuigenis bedekken.
Num 4:6  En zij zullen een deksel van dassenvellen daarop leggen, en een geheel kleed van hemelsblauw daar bovenop uitspreiden; en zij zullen derzelver handbomen aanleggen.
Num 4:7  Zij zullen ook op de toontafel een kleed van hemelsblauw uitspreiden, en zullen daarop zetten de schotels, en de reukschalen, en de kroezen, en de dekschotels; ook zal het gedurig brood daarop zijn.
Num 4:8  Daarna zullen zij een scharlaken kleed daarover uitspreiden, en zullen dat met een deksel van dassenvellen bedekken; en zij zullen derzelver handbomen aanleggen.
Num 4:9  Dan zullen zij een kleed van hemelsblauw nemen, en bedekken den kandelaar des luchters, en zijn lampen, en zijn snuiters, en zijn blusvaten, en al zijn olievaten, met welke zij aan denzelven dienen.
Num 4:10  Zij zullen ook denzelven, en al zijn gereedschap, in een deksel van dassenvellen doen, en zullen hem op den draagboom leggen.
Num 4:11  En over het gouden altaar zullen zij een kleed van hemelsblauw uitspreiden, en zullen dat met een deksel van dassenvellen bedekken; en zij zullen deszelfs handbomen aanleggen.
Num 4:12  Zij zullen ook nemen alle gereedschap van den dienst, met hetwelk zij in het heiligdom dienen, en zullen het leggen in een kleed van hemelsblauw, en zullen hetzelve met een deksel van dassenvellen bedekken; en zij zullen het op den draagboom leggen.
Num 4:13  En zij zullen de as van het altaar vegen, en zij zullen daarover een kleed van purper uitspreiden.
Num 4:14  En zij zullen daarop leggen al zijn gereedschap, waarmede zij aan hetzelve dienen, de koolpannen, de krauwelen, en de schoffelen, en de sprengbekkens, al het gereedschap des altaars; en zij zullen daarover een deksel van dassenvellen uitspreiden, en zullen deszelfs handbomen aanleggen.
Num 4:15  Als nu Aaron en zijn zonen, het dekken van het heiligdom, en van alle gereedschap des heiligdoms, in het optrekken des legers, zullen voleind hebben, zo zullen daarna de zonen van Kohath komen om te dragen; maar zij zullen dat heilige niet aanroeren, dat zij niet sterven. Dit is de last der zonen van Kohath, in de tent der samenkomst.
Num 4:16  Het opzicht nu van Eleazar, den zoon van Aaron, den priester, zal zijn over de olie des luchters, en het reukwerk der welriekende specerijen, en het gedurig spijsoffer, en de zalfolie; het opzicht des gansen tabernakels, en alles wat daarin is, aan het heiligdom en aan zijn gereedschap.
Num 4:17  En de HEERE sprak tot Mozes en tot Aaron, zeggende:
Num 4:18  Gij zult den stam van de geslachten der Kohathieten niet laten uitgeroeid worden, uit het midden der Levieten;
Num 4:19  Maar dit zult gij hun doen, opdat zij leven en niet sterven, als zij tot de heiligheid der heiligheden toetreden zullen: Aaron en zijn zonen zullen komen, en stellen hen een ieder over zijn dienst en aan zijn last.
Num 4:20  Doch zij zullen niet inkomen om te zien, als men het heiligdom inwindt, opdat zij niet sterven.
Num 4:21  En de HEERE sprak tot Mozes, zeggende:
Num 4:22  Neem ook op de som der zonen van Gerson, naar het huis hunner vaderen, naar hun geslachten.
Num 4:23  Gij zult hen tellen van dertig jaren oud en daarboven, tot vijftig jaren oud, al wie inkomt om den strijd te strijden, opdat hij den dienst bediene in de tent der samenkomst.
Num 4:24  Dit zal zijn de dienst der geslachten van de Gersonieten, in het dienen en in den last.
Num 4:25  Zij zullen dan dragen de gordijnen des tabernakels, en de tent der samenkomst; te weten haar deksel, en het dassendeksel, dat er bovenop is, en het deksel der deur van de tent der samenkomst,
Num 4:26  En de behangselen des voorhofs, en het deksel der deur van de poort des voorhofs, hetwelk is bij den tabernakel en bij het altaar rondom; en hun zelen, en al het gereedschap van hun dienst, mitsgaders al wat daarvoor bereid wordt, opdat zij dienen.
Num 4:27  De gehele dienst van de zonen der Gersonieten, in al hun last, en in al hun dienst, zal zijn naar het bevel van Aaron en van zijn zonen; en gijlieden zult hun ter bewaring al hun last bevelen.
Num 4:28  Dit is de dienst van de geslachten der zonen van de Gersonieten, in de tent der samenkomst; en hun wacht zal zijn onder de hand van Ithamar, den zoon van Aaron, den priester.
Num 4:29  Aangaande de zonen van Merari, die zult gij naar hun geslachten, en naar het huis hunner vaderen tellen.
Num 4:30  Gij zult hen tellen van dertig jaren oud en daarboven, tot vijftig jaren oud, al wie inkomt tot dezen strijd, om te bedienen den dienst van de tent der samenkomst.
Num 4:31  Dit zal nu zijn de onderhouding van hun last, naar al hun dienst, in de tent der samenkomst: de berderen des tabernakels, en zijn richelen, en zijn pilaren, en zijn voeten;
Num 4:32  Mitsgaders de pilaren des voorhofs rondom, hun voeten, en hun pennen, en hun zelen, met al hun gereedschap, en met al hun dienst; en het gereedschap van de waarneming van hun last zult gij bij namen tellen.
Num 4:33  Dit is de dienst van de geslachten der zonen van Merari, naar hun gansen dienst, in de tent der samenkomst, onder de hand van Ithamar, den zoon van Aaron, den priester.
Num 4:34  Mozes dan en Aaron, en de oversten der vergadering telden de zonen der Kohathieten, naar hun geslachten, en naar het huis hunner vaderen:
Num 4:35  Van dertig jaren oud en daarboven, tot vijftig jaren oud, al wie inkwam tot dezen strijd, tot den dienst in de tent der samenkomst;
Num 4:36  Hun getelden nu waren, naar hun geslachten, twee duizend zevenhonderd en vijftig.
Num 4:37  Dit zijn de getelden van de geslachten der Kohathieten, van al wie in de tent der samenkomst diende, welke Mozes en Aaron geteld hebben, naar het bevel des HEEREN, door de hand van Mozes.
Num 4:38  Insgelijks de getelden der zonen van Gerson, naar hun geslachten, en naar het huis hunner vaderen;
Num 4:39  Van dertig jaren oud en daarboven, tot vijftig jaren oud, al wie inkwam tot dezen strijd, tot den dienst in de tent der samenkomst;
Num 4:40  Hun getelden waren, naar hun geslachten, naar het huis hunner vaderen, twee duizend zeshonderd en dertig.
Num 4:41  Dezen zijn de getelden van de geslachten der zonen van Gerson, van al wie in de tent der samenkomst diende, welke Mozes en Aaron telden, naar het bevel des HEEREN.
Num 4:42  En de getelden van de geslachten der zonen van Merari, naar hun geslachten, naar het huis hunner vaderen,
Num 4:43  Van dertig jaren oud en daarboven, tot vijftig jaren oud, al wie inkwam tot dezen strijd, tot den dienst in de tent der samenkomst;
Num 4:44  Hun getelden nu waren, naar hun geslachten, drie duizend en tweehonderd.
Num 4:45  Dezen zijn de getelden van de geslachten der zonen van Merari, welke Mozes en Aaron geteld hebben, naar het bevel des HEEREN, door de hand van Mozes.
Num 4:46  Al de getelden, welke Mozes en Aaron, en de oversten van Israel geteld hebben van de Levieten, naar hun geslachten, en naar het huis hunner vaderen,
Num 4:47  Van dertig jaren oud en daarboven, tot vijftig jaren oud, al wie inkwam, om den dienst der bediening en den dienst van den last, in de tent der samenkomst, te bedienen;
Num 4:48  Hun getelden waren acht duizend vijfhonderd en tachtig.
Num 4:49  Men telde hen, naar het bevel des HEEREN, door de hand van Mozes, een ieder naar zijn dienst, en naar zijn last; en zijn getelden waren, die de HEERE Mozes geboden had.
Num 5:1  En de HEERE sprak tot Mozes, zeggende:
Num 5:2  Gebied den kinderen Israels, dat zij uit het leger wegzenden alle melaatsen, en alle vloeienden, en allen, die onrein zijn van een dode.
Num 5:3  Van den man tot de vrouw toe zult gij hen wegzenden; tot buiten het leger zult gij hen wegzenden; opdat zij niet verontreinigen hun legers, in welker midden Ik wone.
Num 5:4  En de kinderen Israels deden alzo, en zonden hen tot buiten het leger; gelijk de HEERE tot Mozes gesproken had, alzo deden de kinderen Israels.
Num 5:5  Verder sprak de HEERE tot Mozes, zeggende:
Num 5:6  Spreek tot de kinderen Israels: Wanneer een man of een vrouw iets van enige menselijke zonden gedaan zullen hebben, overtreden hebbende door overtreding tegen den HEERE, zo is diezelve ziel schuldig.
Num 5:7  En zij zullen hun zonde, welke zij gedaan hebben, belijden; daarna zal hij zijn schuld weder uitkeren, naar de hoofdsom daarvan, en derzelfder vijfde deel zal hij daarboven toedoen, en zal het dien geven, aan wien hij zich verschuldigd heeft.
Num 5:8  Maar zo die man geen losser zal hebben, om de schuld aan hem weder uit te keren, zal die schuld, welken den HEERE weder uitgekeerd wordt, des priesters zijn; behalve den ram der verzoening, met welken hij voor hem verzoening doen zal.
Num 5:9  Desgelijks zal alle heffing van alle geheiligde dingen der kinderen Israels, welke zij tot den priester brengen, zijne zijn.
Num 5:10  En een ieders geheiligde dingen zullen zijne zijn; wat iemand den priester zal gegeven hebben, zal zijne zijn.
Num 5:11  Wijders sprak de HEERE tot Mozes, zeggende:
Num 5:12  Spreek tot de kinderen Israels, en zeg tot hen: Wanneer van iemand zijn huisvrouw zal afgeweken zijn, en door overtreding tegen hem overtreden zal hebben;
Num 5:13  Dat een man bij haar door bijligging des zaads zal gelegen hebben, en het voor de ogen haars mans zal verborgen zijn, en zij zich verheeld zal hebben, zijnde nochtans onrein geworden; en geen getuige tegen haar is, en zij niet betrapt is;
Num 5:14  En de ijvergeest over hem gekomen is, dat hij ijvert over zijn huisvrouw, dewijl zij onrein geworden is; of dat over hem de ijvergeest gekomen is, dat hij over zijn huisvrouw ijvert, hoewel zij niet onrein geworden is;
Num 5:15  Dan zal die man zijn huisvrouw tot den priester brengen, en zal haar offerande voor haar medebrengen, een tiende deel van een efa gerstemeel; hij zal geen olie daarop gieten, noch wierook daarop leggen, dewijl het een spijsoffer der ijveringen is, een spijsoffer der gedachtenis, dat de ongerechtigheid in gedachtenis brengt.
Num 5:16  En de priester zal haar doen naderen; hij zal haar stellen voor het aangezicht des HEEREN.
Num 5:17  En de priester zal heilig water in een aarden vat nemen; en van het stof, hetwelk op den vloer des tabernakels is, zal de priester nemen, en in het water doen.
Num 5:18  Daarna zal de priester de vrouw voor het aangezicht des HEEREN stellen, en zal het hoofd van de vrouw ontbloten, en zal het spijsoffer der gedachtenis op haar handen leggen, hetwelk het spijsoffer der ijveringen is; en in de hand des priesters zal dat bitter water zijn, hetwelk den vloek medebrengt.
Num 5:19  En de priester zal haar beedigen, en zal tot die vrouw zeggen: Indien iemand bij u gelegen heeft, en indien gij, onder uw man zijnde, niet afgeweken zijt tot onreinigheid, wees vrij van dit bitter water, hetwelk den vloek medebrengt!
Num 5:20  Maar zo gij, onder uw man zijnde, afgeweken zijt, en zo gij onrein geworden zijt, dat een man bij u gelegen heeft, behalve uw man:
Num 5:21  (Dan zal de priester die vrouw met den eed der vervloeking beedigen, en de priester zal tot die vrouw zeggen:) De HEERE zette u tot een vloek, en tot een eed, in het midden uws volks, mits dat de HEERE uw heup vervallende, en uw buik zwellende make;
Num 5:22  Dat ditzelve water, hetwelk de vervloeking medebrengt, in uw ingewand inga, om den buik te doen zwellen, en de heup te doen vervallen! Dan zal die vrouw zeggen: Amen, amen!
Num 5:23  Daarna zal de priester deze zelfde vloeken op een cedeltje schrijven, en hij zal het met het bitter water uitdoen.
Num 5:24  En hij zal die vrouw dat bitter water, hetwelk de vervloeking medebrengt, te drinken geven, dat het water, hetwelk de vervloeking medebrengt, in haar tot bitterheden inga.
Num 5:25  En de priester zal uit de hand van die vrouw het spijsoffer der ijveringen nemen, en hij zal datzelve spijsoffer voor het aangezicht des HEEREN bewegen, en zal dat op het altaar offeren.
Num 5:26  De priester zal ook van dat spijsoffer, deszelfs gedenkoffer, een handvol grijpen, en zal het op het altaar aansteken; en daarna zal hij dat water die vrouw te drinken geven.
Num 5:27  Als hij haar nu dat water zal te drinken gegeven hebben, het zal geschieden, indien zij onrein geworden is, en tegen haar man door overtreding zal overtreden hebben, dat het water, hetwelk vervloeking medebrengt, tot bitterheid in haar ingaan zal, en haar buik zwellen, en haar heup vervallen zal; en die vrouw zal in het midden van haar volk tot een vloek zijn.
Num 5:28  Doch indien de vrouw niet onrein geworden is, maar rein is, zo zal zij vrij zijn, en zal met zaad bezadigd worden.
Num 5:29  Dit is de wet der ijveringen, als een vrouw, onder haar man zijnde, zal afgeweken en onrein geworden zijn;
Num 5:30  Of als over en man die ijvergeest zal gekomen zijn, en hij over zijn huisvrouw zal geijverd hebben, dat hij de vrouw voor het aangezicht des HEEREN stelle, en de priester aan haar deze ganse wet volbrenge.
Num 5:31  En de man zal van de ongerechtigheid onschuldig zijn; maar diezelve vrouw zal haar ongerechtigheid dragen.
Num 6:1  En de HEERE sprak tot Mozes, zeggende:
Num 6:2  Spreek tot de kinderen Israels, en zeg tot hen: Wanneer een man of een vrouw zich afgescheiden zal hebben, belovende de gelofte eens Nazireers, om zich den HEERE af te zonderen;
Num 6:3  Van wijn en sterken drank zal hij zich afzonderen; wijnedik, en edik van sterken drank zal hij niet drinken, noch enige vochtigheid van druiven zal hij drinken, noch verse of gedroogde druiven eten.
Num 6:4  Al de dagen van zijn Nazireerschap zal hij niet eten van iets, dat van den wijnstok des wijns gemaakt is, van de kernen af tot de basten toe.
Num 6:5  Al de dagen der gelofte van zijn Nazireerschap zal het scheermes over zijn hoofd niet gaan; totdat die dagen vervuld zullen zijn, die hij zich den HEERE zal afgezonderd hebben, zal hij heilig zijn, latende de lokken van het haar zijns hoofds wassen.
Num 6:6  Al de dagen, die hij zich den HEERE zal afgezonderd hebben, zal hij tot het lichaam eens doden niet gaan.
Num 6:7  Om zijn vader of om zijn moeder, om zijn broeder of om zijn zuster, om hen zal hij zich niet verontreinigen, als zij dood zijn; want het Nazireerschap zijns Gods is op zijn hoofd.
Num 6:8  Al de dagen van zijn Nazireerschap is hij den HEERE heilig.
Num 6:9  En zo de gestorvene bij hem onvoorziens haastelijk gestorven ware, dat hij het hoofd van zijn Nazireerschap zou verontreinigd hebben, zo zal hij op den dag zijner reiniging zijn hoofd bescheren; op den zevenden dag zal hij het bescheren.
Num 6:10  En op den achtsten dag zal hij twee tortelduiven, of twee jonge duiven brengen tot den priester, tot de deur van de tent der samenkomst.
Num 6:11  De priester nu zal een bereiden ten zondoffer, en een ten brandoffer, en zal voor hem verzoening doen, van dat hij aan het dode lichaam gezondigd heeft; alzo zal hij zijn hoofd op dienzelfden dag heiligen.
Num 6:12  Daarna zal hij de dagen van zijn Nazireerschap den HEERE afzonderen, en zal een lam, dat eenjarig is, brengen ten schuldoffer; en de vorige dagen zullen vallen, omdat zijn Nazireerschap verontreinigd was.
Num 6:13  En dit is de wet des Nazireers: op den dag, als de dagen van zijn Nazireerschap zullen vervuld zijn, zal hij dit brengen tot de deur van de tent der samenkomst.
Num 6:14  Hij dan zal tot zijn offerande den HEERE offeren een volkomen eenjarig lam ten brandoffer, en een volkomen eenjarig ooilam ten zondoffer, en een volkomen ram ten dankoffer.
Num 6:15  En een korf ongezuurde koeken, koeken van meelbloem, met olie gemengd, en ongezuurde vladen, met olie bestreken, mitsgaders hun spijsoffer, en hun drankofferen;
Num 6:16  En de priester zal het voor het aangezicht des HEEREN brengen, en zal zijn zondoffer en zijn brandoffer bereiden.
Num 6:17  Hij zal ook den ram ten dankoffer den HEERE bereiden, met den korf der ongezuurde koeken; en de priester zal zijn spijsoffer en zijn drankoffer bereiden.
Num 6:18  Alsdan zal de Nazireer, aan de deur van de tent der samenkomst, het hoofd van zijn Nazireerschap bescheren; en hij zal het hoofdhaar van zijn Nazireerschap nemen, en hij zal het leggen op het vuur, dat onder het dankoffer is.
Num 6:19  Daarna zal de priester een gezoden schouder nemen van den ram, en een ongezuurden koek uit den korf, en een ongezuurde vlade; en hij zal ze op de handen des Nazireers leggen, nadat hij zijn Nazireerschap afgeschoren heeft.
Num 6:20  En de priester zal die bewegen ten beweegoffer, voor het aangezicht des HEEREN; het is een heilig ding voor den priester, met de borst des beweegoffers, en met den schouder des hefoffers; en daarna zal die Nazireer wijn drinken.
Num 6:21  Dit is de wet des Nazireers, die zijn offerande den HEERE voor zijn Nazireerschap zal beloofd hebben, behalve wat zijn hand bekomen zal; naar zijn gelofte, welke hij beloofd zal hebben, alzo zal hij doen, naar de wet van zijn Nazireerschap.
Num 6:22  En de HEERE sprak tot Mozes, zeggende:
Num 6:23  Spreek tot Aaron en zijn zonen, zeggende: Alzo zult gijlieden de kinderen Israels zegenen, zeggende tot hen:
Num 6:24  De HEERE zegene u, en behoede u!
Num 6:25  De HEERE doe Zijn aangezicht over u lichten, en zij u genadig!
Num 6:26  De HEERE verheffe Zijn aangezicht over u, en geve u vrede!
Num 6:27  Alzo zullen zij Mijn Naam op de kinderen Israels leggen; en Ik zal hen zegenen.
Num 7:1  En het geschiedde ten dage, als Mozes geeindigd had den tabernakel op te richten, en dat hij dien gezalfd, en dien geheiligd had, en al zijn gereedschap, mitsgaders het altaar en al zijn gereedschap, en hij ze gezalfd, en dezelve geheiligd had;
Num 7:2  Dat de oversten van Israel, de hoofden van het huis hunner vaderen, offerden; deze waren de oversten der stammen, die over de getelden stonden.
Num 7:3  En zij brachten hun offerande voor het aangezicht des HEEREN, zes overdekte wagens, en twaalf runderen; een wagen voor twee oversten, en een os voor elk een; en brachten ze voor den tabernakel.
Num 7:4  En de HEERE sprak tot Mozes, zeggende:
Num 7:5  Neem ze van hen, opdat zij zijn mogen om te bedienen den dienst van de tent der samenkomst; en gij zult dezelve den Levieten geven, een ieder naar zijn dienst.
Num 7:6  Alzo nam Mozes die wagens, en die runderen, en gaf dezelve den Levieten.
Num 7:7  Twee wagens en vier runderen gaf hij den zonen van Gerson, naar hun dienst;
Num 7:8  En vier wagens en acht runderen gaf hij den zonen van Merari, naar hun dienst; onder de hand van Ithamar, den zoon van Aaron, den priester.
Num 7:9  Maar de zonen van Kohath gaf hij niet; want de dienst der heilige dingen was op hen, die zij op de schouderen droegen.
Num 7:10  En de oversten offerden ter inwijding des altaars, op den dag als hetzelve gezalfd werd; de oversten dan offerden hun offeranden voor het altaar.
Num 7:11  En de HEERE zeide tot Mozes: Elke overste zal (een iegelijk op zijn dag) zijn offerande offeren, ter inwijding des altaars.
Num 7:12  Die nu op den eersten dag zijn offerande offerde, was Nahesson, de zoon van Amminadab, voor den stam van Juda.
Num 7:13  En zijn offerande was: een zilveren schotel, welks gewicht was honderd dertig sikkelen; een zilveren sprengbekken van zeventig sikkelen, naar den sikkel des heiligdoms; zij waren beide vol meelbloem met olie gemengd, ten spijsoffer;
Num 7:14  Een reukschaal van tien gouden sikkelen, vol reukwerks;
Num 7:15  Een var, een jong rund, een ram, een lam, dat eenjarig was, ten brandoffer;
Num 7:16  Een geitenbok, ten zondoffer;
Num 7:17  En ten dankoffer: twee runderen, vijf rammen, vijf bokken, vijf eenjarige lammeren. Dit was de offerande van Nahesson, den zoon van Amminadab.
Num 7:18  Op den tweeden dag offerde Nethaneel, de zoon van Zuar, de overste van Issaschar.
Num 7:19  Hij offerde zijn offerande: een zilveren schotel, welks gewicht was honderd dertig sikkelen; een zilveren sprengbekken van zeventig sikkelen, naar den sikkel des heiligdoms; zij waren beide vol meelbloem met olie gemengd, ten spijsoffer;
Num 7:20  En een reukschaal van tien gouden sikkelen, vol reukwerks;
Num 7:21  Een var, een jong rund, een ram, een lam, dat eenjarig was, ten brandoffer;
Num 7:22  Een geitenbok, ten zondoffer;
Num 7:23  En ten dankoffer: twee runderen, vijf rammen, vijf bokken, vijf eenjarige lammeren. Dit was de offerande van Nethaneel, den zoon van Zuar.
Num 7:24  Op den derden dag offerde de overste der zonen van Zebulon, Eliab, de zoon van Helon.
Num 7:25  Zijn offerande was: een zilveren schotel, welks gewicht was honderd dertig sikkelen; een zilveren sprengbekken van zeventig sikkelen, naar den sikkel des heiligdoms; zij waren beide vol meelbloem met olie gemengd, ten spijsoffer;
Num 7:26  Een reukschaal van tien gouden sikkelen, vol reukwerks;
Num 7:27  Een var, een jong rund, een ram, een lam, dat eenjarig was, ten brandoffer;
Num 7:28  Een geitenbok, ten zondoffer;
Num 7:29  En ten dankoffer: twee runderen, vijf rammen, vijf bokken, vijf eenjarige lammeren. Dit was de offerande van Eliab, den zoon van Helon.
Num 7:30  Op den vierden dag offerde de overste der kinderen van Ruben, Elizur, de zoon van Sedeur.
Num 7:31  Zijn offerande was: een zilveren schotel, welks gewicht was honderd dertig sikkelen; een zilveren sprengbekken van zeventig sikkelen, naar den sikkel des heiligdoms; zij waren beide vol meelbloem met olie gemengd, ten spijsoffer;
Num 7:32  Een reukschaal van tien gouden sikkelen, vol reukwerks;
Num 7:33  Een var, een jong rund, een ram, een lam, dat eenjarig was, ten brandoffer;
Num 7:34  Een geitenbok, ten zondoffer;
Num 7:35  En ten dankoffer: twee runderen, vijf rammen, vijf bokken, vijf eenjarige lammeren. Dit was de offerande van Elizur, den zoon van Sedeur.
Num 7:36  Op den vijfden dag offerde de overste der kinderen van Simeon, Selumiel, de zoon van Zurisaddai.
Num 7:37  Zijn offerande was: een zilveren schotel, welks gewicht was honderd dertig sikkelen; een zilveren sprengbekken van zeventig sikkelen, naar den sikkel des heiligdoms; zij waren beide vol meelbloem met olie gemengd, ten spijsoffer;
Num 7:38  Een reukschaal van tien gouden sikkelen, vol reukwerks;
Num 7:39  Een var, een jong rund, een ram, een lam, dat eenjarig was, ten brandoffer;
Num 7:40  Een geitenbok, ten zondoffer;
Num 7:41  En ten dankoffer: twee runderen, vijf rammen, vijf bokken, vijf eenjarige lammeren. Dit was de offerande van Selumiel, den zoon van Zurisaddai.
Num 7:42  Op den zesden dag offerde de overste der kinderen van Gad, Eljasaf, den zoon van Dehuel.
Num 7:43  Zijn offerande was: een zilveren schotel, welks gewicht was honderd dertig sikkelen; een zilveren sprengbekken van zeventig sikkelen, naar den sikkel des heiligdoms; beide vol meelbloem gemengd met olie, ten spijsoffer;
Num 7:44  Een reukschaal van tien gouden sikkelen, vol reukwerks;
Num 7:45  Een var, een jong rund, een ram, een lam, dat eenjarig was, ten brandoffer;
Num 7:46  Een geitenbok, ten zondoffer;
Num 7:47  En ten dankoffer: twee runderen, vijf rammen, vijf bokken, vijf eenjarige lammeren. Dit was de offerande van Eljasaf, den zoon van Dehuel.
Num 7:48  Op den zevenden dag offerde de overste der kinderen van Efraim, Elisama, den zoon van Ammihud.
Num 7:49  Zijn offerande was: een zilveren schotel, welks gewicht was honderd dertig sikkelen; een zilveren sprengbekken van zeventig sikkelen, naar den sikkel des heiligdoms; beide vol meelbloem met olie gemengd, ten spijsoffer;
Num 7:50  Een reukschaal van tien gouden sikkelen, vol reukwerks;
Num 7:51  Een var, een jong rund, een ram, een lam, dat eenjarig was, ten brandoffer;
Num 7:52  Een geitenbok, ten zondoffer;
Num 7:53  En ten dankoffer: twee runderen, vijf rammen, vijf bokken, vijf eenjarige lammeren. Dit was de offerande van Elisama, den zoon van Ammihud.
Num 7:54  Op den achtsten dag offerde de overste der kinderen van Manasse, Gamaliel, de zoon van Pedazur.
Num 7:55  Zijn offerande was: een zilveren schotel, welks gewicht was honderd dertig sikkelen; een zilveren sprengbekken van zeventig sikkelen, naar den sikkel des heiligdoms; beide vol meelbloem met olie gemengd, ten spijsoffer;
Num 7:56  Een reukschaal van tien gouden sikkelen, vol reukwerks;
Num 7:57  Een var, een jong rund, een ram, een lam, dat eenjarig was, ten brandoffer;
Num 7:58  Een geitenbok, ten zondoffer;
Num 7:59  En ten dankoffer: twee runderen, vijf rammen, vijf bokken, vijf eenjarige lammeren. Dit was de offerande van Gamaliel, den zoon van Pedazur.
Num 7:60  Op den negenden dag offerde de overste der kinderen van Benjamin, Abidan, de zoon van Gideoni.
Num 7:61  Zijn offerande was: een zilveren schotel, welks gewicht was honderd dertig sikkelen; een zilveren sprengbekken van zeventig sikkelen, naar den sikkel des heiligdoms; zij waren beide vol meelbloem met olie gemengd, ten spijsoffer;
Num 7:62  Een reukschaal van tien gouden sikkelen, vol reukwerks;
Num 7:63  Een var, een jong rund, een ram, een lam, dat eenjarig was, ten brandoffer;
Num 7:64  Een geitenbok, ten zondoffer;
Num 7:65  En ten dankoffer: twee runderen, vijf rammen, vijf bokken, vijf eenjarige lammeren. Dit was de offerande van Abidan, den zoon van Gideoni.
Num 7:66  Op den tienden dag offerde de overste der kinderen van Dan, Ahiezer, de zoon van Ammisaddai.
Num 7:67  Zijn offerande was: een zilveren schotel, welks gewicht was honderd dertig sikkelen; een zilveren sprengbekken van zeventig sikkelen, naar den sikkel des heiligdoms; zij waren beide vol meelbloem met olie gemengd, ten spijsoffer;
Num 7:68  Een reukschaal van tien gouden sikkelen, vol reukwerks;
Num 7:69  Een var, een jong rund, een ram, een lam, dat eenjarig was, ten brandoffer;
Num 7:70  Een geitenbok, ten zondoffer;
Num 7:71  En ten dankoffer: twee runderen, vijf rammen, vijf bokken, vijf eenjarige lammeren. Dit was de offerande van Ahiezer, den zoon van Ammisaddai.
Num 7:72  Op den elfden dag offerde de overste der kinderen van Aser, Pagiel, de zoon van Ochran.
Num 7:73  Zijn offerande was: een zilveren schotel, welks gewicht was honderd dertig sikkelen; een zilveren sprengbekken van zeventig sikkelen, naar den sikkel des heiligdoms; zij waren beide vol meelbloem met olie gemengd, ten spijsoffer;
Num 7:74  Een reukschaal van tien gouden sikkelen, vol reukwerks;
Num 7:75  Een var, een jong rund, een ram, een lam, dat eenjarig was, ten brandoffer;
Num 7:76  Een geitenbok, ten zondoffer;
Num 7:77  En ten dankoffer: twee runderen, vijf rammen, vijf bokken, vijf eenjarige lammeren. Dit was de offerande van Pagiel, den zoon van Ochran.
Num 7:78  Op den twaalfden dag offerde de overste der kinderen van Nafthali, Ahira, de zoon van Enan.
Num 7:79  Zijn offerande was: een zilveren schotel, welks gewicht was honderd dertig sikkelen; een zilveren sprengbekken van zeventig sikkelen, naar den sikkel des heiligdoms; zij waren beide vol meelbloem met olie gemengd, ten spijsoffer;
Num 7:80  Een reukschaal van tien gouden sikkelen, vol reukwerks;
Num 7:81  Een var, een jong rund, een ram, een lam, dat eenjarig was, ten brandoffer;
Num 7:82  Een geitenbok, ten zondoffer;
Num 7:83  En ten dankoffer: twee runderen, vijf rammen, vijf bokken, vijf eenjarige lammeren. Dit was de offerande van Ahira, den zoon van Enan.
Num 7:84  Dit was de inwijding des altaars van de oversten van Israel, op den dag als hetzelve gezalfd werd: twaalf zilveren schotels, twaalf zilveren sprengbekkens, twaalf gouden reukschalen.
Num 7:85  Een zilveren schotel was van honderd dertig sikkelen, en een sprengbekken van zeventig; al het zilver van de vaten was twee duizend en vierhonderd sikkelen, naar den sikkel des heiligdoms.
Num 7:86  Twaalf gouden reukschalen van reukwerks; elke reukschaal was van tien sikkelen, naar den sikkel des heiligdoms; al het goud der reukschalen was honderd en twintig sikkelen.
Num 7:87  Al de runderen ten brandoffer waren twaalf varren, twaalf rammen, twaalf eenjarige lammeren, met hun spijsoffer; en twaalf geitenbokken ten zondoffer.
Num 7:88  En al de runderen ten dankoffer waren vier en twintig varren, de rammen zestig, de bokken zestig, de eenjarige lammeren zestig. Dit is de inwijding des altaars, nadat hetzelve gezalfd was.
Num 7:89  En als Mozes in de tent der samenkomst ging, om met Hem te spreken, zo hoorde hij een stem tot hem sprekende, van boven het verzoendeksel, hetwelk is op de ark der getuigenis, van tussen de twee cherubim. Alzo sprak Hij tot hem.
Num 8:1  En de HEERE sprak tot Mozes, zeggende:
Num 8:2  Spreek tot Aaron, en zeg tot hem: Als gij de lampen aansteken zult, recht tegenover den kandelaar zullen de zeven lampen lichten.
Num 8:3  En Aaron deed alzo: tegenover vooraan den kandelaar stak hij deszelfs lampen aan; gelijk als de HEERE Mozes geboden had.
Num 8:4  Dit werk nu des kandelaars was van dicht goud, tot zijn schacht, tot zijn bloemen was het dicht; naar de gedaante, die de HEERE Mozes vertoond had, alzo had hij den kandelaar gemaakt.
Num 8:5  En de HEERE sprak tot Mozes, zeggende:
Num 8:6  Neem de Levieten uit het midden van de kinderen Israels, en reinig hen.
Num 8:7  En aldus zult gij hun doen, om hen te reinigen: spreng op hen water der ontzondiging; en zij zullen het scheermes over hun ganse vlees doen gaan, en zij zullen hun klederen wassen, en zich reinigen.
Num 8:8  Daarna zullen zij nemen een var, een jong rund, met zijn spijsoffer van meelbloem, met olie gemengd; en een anderen var, een jong rund, zult gij nemen ten zondoffer.
Num 8:9  En gij zult de Levieten voor de tent der samenkomst doen naderen; en gij zult de gehele vergadering der kinderen Israels doen verzamelen.
Num 8:10  Ja, gij zult de Levieten voor het aangezicht des HEEREN doen naderen; en de kinderen Israels zullen hun handen op de Levieten leggen.
Num 8:11  En Aaron zal de Levieten bewegen ten beweegoffer voor het aangezicht des HEEREN, vanwege de kinderen Israels; opdat zij zijn, om den dienst des HEEREN te bedienen.
Num 8:12  En de Levieten zullen hun handen op het hoofd der varren leggen; daarna bereidt gij een ten zondoffer, en een ten brandoffer den HEERE, om over de Levieten verzoening te doen.
Num 8:13  En gij zult de Levieten stellen voor het aangezicht van Aaron, en voor het aangezicht van zijn zonen, en gij zult hen bewegen ten beweegoffer den HEERE.
Num 8:14  En gij zult de Levieten uit het midden van de kinderen Israels uitscheiden, opdat de Levieten Mijn zijn.
Num 8:15  En daarna zullen de Levieten inkomen, om de tent der samenkomst te bedienen; en gij zult hen reinigen, en zult hen ten beweegoffer bewegen.
Num 8:16  Want zij zijn gegeven, zij zijn Mij gegeven uit het midden van de kinderen Israels; voor de opening van alle baarmoeder, voor de eerstgeborenen van een ieder uit de kinderen Israels, heb Ik ze Mij genomen.
Num 8:17  Want alle eerstgeborene onder de kinderen Israels is Mijn, onder de mensen en onder de beesten; ten dage dat Ik alle eerstgeboorte in Egypteland sloeg, heb Ik dezelve Mij geheiligd.
Num 8:18  En Ik heb de Levieten genomen voor alle eerstgeborenen onder de kinderen Israels.
Num 8:19  En Ik heb de Levieten aan Aaron en aan zijn zonen tot een gift gegeven, uit het midden van de kinderen Israels, om den dienst van de kinderen Israels in de tent der samenkomst te bedienen, en om voor de kinderen Israels verzoening te doen, dat er geen plage zij onder de kinderen Israels, als de kinderen Israels tot het heiligdom naderen zouden.
Num 8:20  En Mozes deed, en Aaron, en de ganse vergadering der kinderen Israels, aan de Levieten, naar alles, wat de HEERE Mozes geboden had van de Levieten, zo deden de kinderen Israels aan hen.
Num 8:21  En de Levieten ontzondigden zich, en wiesen hun klederen, en Aaron bewoog hen ten beweegoffer voor het aangezicht des HEEREN; en Aaron deed verzoening over hen, om hen te reinigen.
Num 8:22  En daarna kwamen de Levieten, om hun dienst te bedienen in de tent der samenkomst, voor het aangezicht van Aaron, en voor het aangezicht zijner zonen; gelijk als de HEERE Mozes van de Levieten geboden had, alzo deden zij aan hen.
Num 8:23  En de HEERE sprak tot Mozes, zeggende:
Num 8:24  Dit is het, wat de Levieten aangaat: van vijf en twintig jaren oud en daarboven, zullen zij inkomen, om den strijd te strijden, in den dienst van de tent der samenkomst.
Num 8:25  Maar van dat hij vijftig jaren oud is, zal hij van den strijd van dezen dienst afgaan, en hij zal niet meer dienen.
Num 8:26  Doch hij zal met zijn broederen dienen in de tent der samenkomst, om de wacht waar te nemen; maar den dienst zal hij niet bedienen. Alzo zult gij aan de Levieten doen in hun wachten.
Num 9:1  En de HEERE sprak tot Mozes in de woestijn van Sinai, in het tweede jaar, nadat zij uit Egypteland uitgetogen waren, in de eerste maand, zeggende:
Num 9:2  Dat de kinderen Israels het pascha houden zouden, op zijn gezetten tijd.
Num 9:3  Op den veertienden dag in deze maand, tussen twee avonden zult gij dat houden, op zijn gezetten tijd; naar al zijn inzettingen, en naar al zijn rechten zult gij dat houden.
Num 9:4  Mozes dan sprak tot de kinderen Israels, dat zij het pascha zouden houden.
Num 9:5  En zij hielden het pascha op den veertienden dag der eerste maand, tussen de twee avonden, in de woestijn van Sinai; naar alles, wat de HEERE Mozes geboden had, alzo deden de kinderen Israels.
Num 9:6  Toen waren er lieden geweest, die over het dode lichaam eens mensen onrein waren, en op denzelven dag het pascha niet hadden kunnen houden; daarom naderden zij voor het aangezicht van Mozes, en voor het aangezicht van Aaron op dienzelven dag.
Num 9:7  En diezelve lieden zeiden tot hem: Wij zijn onrein over het dode lichaam eens mensen; waarom zouden wij verkort worden, dat wij de offerande des HEEREN op zijn gezetten tijd niet zouden offeren, in het midden van de kinderen Israels?
Num 9:8  En Mozes zeide tot hen: Blijft staande, dat ik hoor, wat de HEERE u gebieden zal.
Num 9:9  Toen sprak de HEERE tot Mozes, zeggende:
Num 9:10  Spreek tot de kinderen Israels, zeggende: Wanneer iemand onder u, of onder uw geslachten, over een dood lichaam onrein, of op een verren weg zal zijn, hij zal dan nog den HEERE het pascha houden.
Num 9:11  In de tweede maand, op den veertienden dag, tussen de twee avonden, zullen zij dat houden; met ongezuurde broden en bittere saus zullen zij dat eten.
Num 9:12  Zij zullen daarvan niet overlaten tot den morgen, en zullen daaraan geen been breken; naar alle inzetting van het pascha zullen zij dat houden.
Num 9:13  Als een man, die rein is, en op den weg niet is, en nalaten zal het pascha te houden, zo zal diezelve ziel uit haar volken uitgeroeid worden; want hij heeft de offerande des HEEREN op zijn gezetten tijd niet geofferd, diezelve man zal zijn zonde dragen.
Num 9:14  En wanneer een vreemdeling bij u als vreemdeling verkeert, en hij het pascha den HEERE ook houden zal, naar de inzetting van het pascha, en naar zijn wijze, alzo zal hij het houden; het zal enerlei inzetting voor ulieden zijn, beiden den vreemdeling en den inboorling des lands.
Num 9:15  En op den dag van het oprichten des tabernakels bedekte de wolk den tabernakel, op de tent der getuigenis; en in den avond was over den tabernakel als een gedaante des vuurs, tot aan den morgen.
Num 9:16  Alzo geschiedde het geduriglijk; de wolk bedekte denzelven, en des nachts was er een gedaante des vuurs.
Num 9:17  Maar nadat de wolk opgeheven werd van boven de tent, zo verreisden ook daarna de kinderen Israels; en in de plaats, waar de wolk bleef, daar legerden zich de kinderen Israels.
Num 9:18  Naar den mond des HEEREN, verreisden de kinderen Israels, en naar des HEEREN mond legerden zij zich; al de dagen, in dewelke de wolk over den tabernakel bleef, legerden zij zich.
Num 9:19  En als de wolk vele dagen over den tabernakel verbleef, zo namen de kinderen Israels de wacht des HEEREN waar, en verreisden niet.
Num 9:20  Als het nu was, dat de wolk weinige dagen op den tabernakel was, naar den mond des HEEREN legerden zij zich, en naar den mond des HEEREN verreisden zij.
Num 9:21  Maar was het, dat de wolk van den avond tot den morgen daar was, en de wolk in den morgen opgeheven werd, zo verreisden zij; of des daags, of des nachts, als de wolk opgeheven werd, zo verreisden zij.
Num 9:22  Of als de wolk twee dagen, of een maand, of vele dagen vertoog op den tabernakel, blijvende daarop, zo legerden zich de kinderen Israels, en verreisden niet; en als zij verheven werd, verreisden zij.
Num 9:23  Naar den mond des HEEREN legerden zij zich, en naar den mond des HEEREN verreisden zij; zij namen de wacht des HEEREN waar, naar den mond des HEEREN, door de hand van Mozes.
Num 10:1  Verder sprak de HEERE tot Mozes, zeggende:
Num 10:2  Maak u twee zilveren trompetten; van dicht werk zult gij ze maken; en zij zullen u zijn tot de samenroeping der vergadering, en tot den optocht der legers.
Num 10:3  Als zij met dezelve blazen zullen, dan zal de gehele vergadering tot u vergaderd worden, aan de deur van de tent der samenkomst.
Num 10:4  Maar als zij met de ene zullen blazen, dan zullen tot u vergaderd worden de oversten, de hoofden der duizenden van Israel.
Num 10:5  Als gij met een gebroken geklank blazen zult, dan zullen de legers, die tegen het oosten gelegerd zijn, optrekken.
Num 10:6  Maar als gij ten tweeden male met een gebroken klank blazen zult, zullen de legers, die tegen het zuiden legeren, optrekken; met een gebroken klank zullen zij blazen tot hun optochten.
Num 10:7  Maar in het verzamelen van de gemeente, zult gij blazen, doch geen gebroken geklank maken.
Num 10:8  En de zonen van Aaron, de priesters, zullen met die trompetten blazen; en zij zullen ulieden zijn tot een eeuwige inzetting bij uw geslachten.
Num 10:9  En wanneer gijlieden in uw land ten strijde zult trekken tegen den vijand, die u benauwt, zult gij ook met die trompetten een gebroken klank maken; zo zal uwer gedacht worden voor het aangezicht des HEEREN, uws Gods, en gij zult van uw vijanden verlost worden.
Num 10:10  Desgelijks ten dage uwer vrolijkheid, en in uw gezette hoogtijden, en in de beginselen uwer maanden, zult gij ook met de trompetten blazen over uw brandofferen, en over uw dankofferen; en zij zullen u ter gedachtenis zijn voor het aangezicht uws Gods; Ik ben de HEERE, uw God!
Num 10:11  En het geschiedde in het tweede jaar, in de tweede maand, op den twintigsten van de maand, dat de wolk verheven werd van boven den tabernakel der getuigenis.
Num 10:12  En de kinderen Israels togen op, naar hun tochten, uit de woestijn Sinai; en de wolk bleef in de woestijn Paran.
Num 10:13  Alzo togen zij vooreerst op, naar den mond des HEEREN, door de hand van Mozes.
Num 10:14  Want vooreerst toog op de banier van het leger der kinderen van Juda, naar hun heiren; en over zijn heir was Nahesson, de zoon van Amminadab.
Num 10:15  En over het heir van den stam der kinderen van Issaschar was Nethaneel, den zoon van Zuar.
Num 10:16  En over het heir van den stam der kinderen van Zebulon was Eliab, de zoon van Helon.
Num 10:17  Toen werd de tabernakel afgenomen, en de zonen van Gerson, en de zonen van Merari togen op, dragende den tabernakel.
Num 10:18  Daarna toog de banier van het leger van Ruben, naar hun heiren; en over zijn heir was Elizur, de zoon van Sedeur.
Num 10:19  En over het heir van den stam der kinderen van Simeon was Selumiel, de zoon van Zurisaddai.
Num 10:20  En over het heir van den stam der kinderen van Gad was Eljasaf, de zoon van Dehuel.
Num 10:21  Toen togen op de Kohathieten, dragende het heiligdom; en de anderen richtten den tabernakel op, tegen dat dezen kwamen.
Num 10:22  Daarna toog op de banier van het leger der kinderen van Efraim, naar hun heiren; en over het heir was Elisama, de zoon van Ammihud.
Num 10:23  En over het heir van den stam der kinderen van Manasse was Gamaliel, de zoon van Pedazur.
Num 10:24  En over het heir van den stam der kinderen van Benjamin was Abidan, de zoon van Gideoni.
Num 10:25  Toen toog op de banier van het leger der kinderen van Dan, samensluitende al de legers, naar hun heiren; en over zijn heir was Ahiezer de zoon van Ammisaddai.
Num 10:26  En over het heir van den stam der kinderen van Aser was Pagiel, de zoon van Ochran.
Num 10:27  En over het heir van den stam der kinderen van Nafthali was Ahira, de zoon van Enan.
Num 10:28  Dit waren de tochten der kinderen Israels, naar hun heiren, als zij reisden.
Num 10:29  Mozes nu zeide tot Hobab, den zoon van Rehuel, den Midianiet, den schoonvader van Mozes: Wij reizen naar die plaats, van welke de HEERE gezegd heeft: Ik zal u die geven; ga met ons, en wij zullen u weldoen, want de HEERE heeft over Israel het goede gesproken.
Num 10:30  Doch hij zeide tot hem: Ik zal niet gaan; maar ik zal naar mijn land en naar mijn maagschap gaan.
Num 10:31  En hij zeide: Verlaat ons toch niet; want dewijl gij weet, dat wij ons legeren in de woestijn, zo zult gij ons tot ogen zijn.
Num 10:32  En het zal geschieden, als gij met ons zult gaan, en het goede geschieden zal, waarmede de HEERE bij ons weldoen zal, dat wij u ook weldoen zullen.
Num 10:33  Zo togen zij drie dagreizen van den berg des HEEREN; en de ark des verbonds des HEEREN reisde voor hun aangezicht drie dagreizen, om voor hen een rustplaats uit te speuren.
Num 10:34  En de wolk des HEEREN was des daags over hen, als zij uit het leger verreisden.
Num 10:35  Het geschiedde nu in het optrekken van de ark, dat Mozes zeide: Sta op, HEERE! en laat Uw vijanden verstrooid worden, en Uw haters van Uw aangezicht vlieden!
Num 10:36  En als zij rustte, zeide hij: Kom weder, HEERE! tot de tien duizenden der duizenden van Israel!
Num 11:1  En het geschiedde, als het volk zich was beklagende, dat het kwaad was in de oren des HEEREN; want de HEERE hoorde het, zodat Zijn toorn ontstak, en het vuur des HEEREN onder hen ontbrandde, en verteerde, in het uiterste des legers.
Num 11:2  Toen riep het volk tot Mozes; en Mozes bad tot den HEERE; en het vuur werd gedempt.
Num 11:3  Daarom noemde hij den naam dier plaats Thab-era, omdat het vuur des HEEREN onder hen gebrand had.
Num 11:4  En het gemene volk, dat in het midden van hen was, werd met lust bevangen; daarom zo weenden ook de kinderen Israels wederom, en zeiden: Wie zal ons vlees te eten geven?
Num 11:5  Wij gedenken aan de vissen, die wij in Egypte om niet aten; aan de komkommers, en aan de pompoenen, en aan het look, en aan de ajuinen, en aan het knoflook.
Num 11:6  Maar nu is onze ziel dor, er is niet met al, behalve dit Man voor onze ogen!
Num 11:7  Het Man nu was als korianderzaad, en zijn verf was als de verf van den bedolah.
Num 11:8  Het volk liep hier en daar, en verzamelde het, en maalde het met molens, of stiet het in mortieren, en zood het in potten, en maakte daarvan koeken; en zijn smaak was als de smaak van de beste vochtigheid der olie.
Num 11:9  En wanneer de dauw des nachts op het leger nederviel, viel het Man op hetzelve neder.
Num 11:10  Toen hoorde Mozes het volk wenen door hun huisgezinnen, een ieder aan de deur zijner hut; en de toorn des HEEREN ontstak zeer; ook was het kwaad in de ogen van Mozes.
Num 11:11  En Mozes zeide tot de HEERE: Waarom hebt Gij aan Uw knecht kwalijk gedaan, en waarom heb ik geen genade in Uw ogen gevonden, dat Gij den last van dit ganse volk op mij legt?
Num 11:12  Heb ik dan al dit volk ontvangen? heb ik het gebaard? dat Gij tot mij zoudt zeggen: Draag het in uw schoot, gelijk als een voedstervader den zuigeling draagt, tot dat land, hetwelk Gij hun vaderen gezworen hebt?
Num 11:13  Van waar zou ik het vlees hebben, om al dit volk te geven? Want zij wenen tegen mij, zeggende: Geef ons vlees, dat wij eten!
Num 11:14  Ik alleen kan al dit volk niet dragen; want het is mij te zwaar!
Num 11:15  En indien Gij alzo aan mij doet, dood mij toch slechts, indien ik genade in Uw ogen gevonden heb; en laat mij mijn ongeluk niet aanzien!
Num 11:16  En de HEERE zeide tot Mozes: Verzamel Mij zeventig mannen uit de oudsten van Israel, dewelke gij weet, dat zij de oudsten des volks en deszelfs ambtlieden zijn; en gij zult hen brengen voor de tent der samenkomst, en zij zullen zich daar bij u stellen.
Num 11:17  Zo zal Ik afkomen en met u aldaar spreken; en van den Geest, die op u is, zal Ik afzonderen, en op hen leggen; en zij zullen met u den last van dit volk dragen, opdat gij dien alleen niet draagt.
Num 11:18  En tot het volk zult gij zeggen: Heiligt u tegen morgen, en gij zult vlees eten; want gij hebt voor de oren des HEEREN geweend, zeggende: Wie zal ons vlees te eten geven? want het ging ons wel in Egypte! Daarom zal de HEERE u vlees geven, en gij zult eten.
Num 11:19  Gij zult niet een dag, noch twee dagen eten, noch vijf dagen, noch tien dagen, noch twintig dagen;
Num 11:20  Tot een gehele maand toe, totdat het uit uw neus uitga, en u tot walging zij; overmits gij den HEERE, Die in het midden van u is, verworpen hebt, en hebt voor Zijn aangezicht geweend, zeggende: Waarom nu zijn wij uit Egypte getogen?
Num 11:21  En Mozes zeide: Zeshonderd duizend te voet is dit volk, in welks midden ik ben; en Gij hebt gezegd: Ik zal hun vlees geven, en zij zullen een gehele maand eten!
Num 11:22  Zullen dan voor hen schapen en runderen geslacht worden, dat voor hen genoeg zij? zullen al de vissen der zee voor hen verzameld worden, dat voor hen genoeg zij?
Num 11:23  Doch de HEERE zeide tot Mozes: Zou dan des HEEREN hand verkort zijn? Gij zult nu zien, of Mijn woord u wedervaren zal, of niet.
Num 11:24  En Mozes ging uit, en sprak de woorden des HEEREN tot het volk; en hij verzamelde zeventig mannen uit de oudsten des volks, en stelde hen rondom de tent.
Num 11:25  Toen kwam de HEERE af in de wolk, en sprak tot hem, en afzonderende van den Geest, die op hem was, legde Hem op de zeventig mannen, die oudsten; en het geschiedde, als de Geest op hen rustte, dat zij profeteerden, maar daarna niet meer.
Num 11:26  Maar twee mannen waren in het leger overgebleven; des enen naam was Eldad, en des anderen naam Medad; en die Geest rustte op hen (want zij waren onder de aangeschrevenen, hoewel zij tot de tent niet uitgegaan waren), en zij profeteerden in het leger.
Num 11:27  Toen liep een jongen heen, en boodschapte aan Mozes, en zeide: Eldad en Medad profeteren in het leger.
Num 11:28  En Jozua, de zoon van Nun, de dienaar van Mozes, een van zijn uitgelezen jongelingen, antwoordde en zeide: Mijn heer Mozes, verbied hun!
Num 11:29  Doch Mozes zeide tot hem: Zijt gij voor mij ijverende? Och, of al het volk des HEEREN profeten waren, dat de HEERE Zijn Geest over hen gave!
Num 11:30  Daarna verzamelde zich Mozes tot het leger, hij en de oudsten van Israel.
Num 11:31  Toen voer een wind uit van den HEERE, en raapte kwakkelen van de zee, en strooide ze bij het leger, omtrent een dagreize herwaarts, en omtrent een dagreize derwaarts, rondom het leger; en zij waren omtrent twee ellen boven de aarde.
Num 11:32  Toen maakte zich het volk op, dien gehelen dag, en dien gansen nacht, en den gansen anderen dag, en verzamelden de kwakkelen; die het minst had, had tien homers verzameld; en zij spreidden ze voor zich van elkander rondom het leger.
Num 11:33  Dat vlees was nog tussen hun tanden, eer het gekauwd was, zo ontstak de toorn des HEEREN tegen het volk, en de HEERE sloeg het volk met een zeer grote plaag.
Num 11:34  Daarom heet men den naam derzelver plaats Kibroth Thaava; want daar begroeven zij het volk, dat belust was geweest.
Num 11:35  Van Kibroth Thaava verreisde het volk naar Hazeroth; en zij bleven in Hazeroth.
Num 12:1  Mirjam nu sprak, en Aaron, tegen Mozes, ter oorzake der vrouw, der Cuschietische, die hij genomen had; want hij had een Cuschietische ter vrouw genomen.
Num 12:2  En zij zeiden: Heeft dan de HEERE maar alleen door Mozes gesproken? Heeft Hij ook niet door ons gesproken? En de HEERE hoorde het!
Num 12:3  Doch de man Mozes was zeer zachtmoedig, meer dan alle mensen, die op den aardbodem waren.
Num 12:4  Toen sprak de HEERE haastelijk tot Mozes, en tot Aaron, en tot Mirjam: Gij drie, komt uit tot de tent der samenkomst! En zij drie kwamen uit.
Num 12:5  Toen kwam de HEERE af in de wolkkolom, en stond aan de deur der tent; daarna riep Hij Aaron en Mirjam; en zij beiden kwamen uit.
Num 12:6  En Hij zeide: Hoort nu Mijn woorden! Zo er een profeet onder u is, Ik, de HEERE, zal door een gezicht Mij aan hem bekend maken, door een droom zal Ik met hem spreken.
Num 12:7  Alzo is Mijn knecht Mozes niet, die in Mijn ganse huis getrouw is.
Num 12:8  Van mond tot mond spreek Ik met hem, en door aanzien, en niet door duistere woorden; en de gelijkenis des HEEREN aanschouwt hij; waarom dan hebt gijlieden niet gevreesd tegen Mijn knecht, tegen Mozes, te spreken?
Num 12:9  Zo ontstak des HEEREN toorn tegen hen, en Hij ging weg.
Num 12:10  En de wolk week van boven de tent; en ziet, Mirjam was melaats, wit als de sneeuw. En Aaron zag Mirjam aan, en ziet, zij was melaats.
Num 12:11  Daarom zeide Aaron tot Mozes: Och, mijn heer! leg toch niet op ons de zonde, waarmede wij zottelijk gedaan, en waarmede wij gezondigd hebben!
Num 12:12  Laat zij toch niet zijn als een dode, van wiens vlees, als hij uit zijns moeders lijf uitgaat, de helft wel verteerd is!
Num 12:13  Mozes dan riep tot den HEERE, zeggende: O God! heel haar toch!
Num 12:14  En de HEERE zeide tot Mozes: Zo haar vader smadelijk in haar aangezicht gespogen had, zou zij niet zeven dagen beschaamd zijn? Laat haar zeven dagen buiten het leger gesloten, en daarna aangenomen worden!
Num 12:15  Zo werd Mirjam buiten het leger zeven dagen gesloten; en het volk verreisde niet, totdat Mirjam aangenomen werd.
Num 12:16  Maar daarna verreisde het volk van Hazeroth, en zij legerden zich in de woestijn van Paran.
Num 13:1  En de HEERE sprak tot Mozes, zeggende:
Num 13:2  Zend u mannen uit: die het land Kanaan verspieden, hetwelk Ik den kinderen Israels geven zal; van elken stam zijner vaderen zult gijlieden een man zenden, zijnde ieder een overste onder hen.
Num 13:3  Mozes dan zond hen uit de woestijn van Paran, naar den mond des HEEREN; al die mannen waren hoofden der kinderen Israels.
Num 13:4  En dit zijn hun namen: van den stam van Ruben, Sammua, de zoon van Zaccur.
Num 13:5  Van den stam van Simeon, Safat, de zoon van Hori.
Num 13:6  Van den stam van Juda, Kaleb, de zoon van Jefunne.
Num 13:7  Van den stam van Issaschar, Jigeal, de zoon van Jozef.
Num 13:8  Van den stam van Efraim, Hosea, de zoon van Nun.
Num 13:9  Van den stam van Benjamin, Palti, de zoon van Rafu.
Num 13:10  Van den stam van Zebulon, Gaddiel, de zoon van Sodi.
Num 13:11  Van den stam van Jozef, voor den stam van Manasse, Gaddi, de zoon van Susi.
Num 13:12  Van den stam van Dan, Ammiel, de zoon van Gemalli.
Num 13:13  Van den stam van Aser, Sethur, de zoon van Michael.
Num 13:14  Van den stam van Nafthali, Nachbi, de zoon van Wofsi.
Num 13:15  Van den stam van Gad, Guel, de zoon van Machi.
Num 13:16  Dit zijn de namen der mannen, die Mozes zond, om dat land te verspieden; en Mozes noemde Hosea, den zoon van Nun, Jozua.
Num 13:17  Mozes dan zond hen, om het land Kanaan te verspieden; en hij zeide tot hen: Trekt dit henen op tegen het zuiden, en klimt op het gebergte;
Num 13:18  En beziet het land, hoedanig het zij, en het volk, dat daarin woont, of het sterk zij of zwak, of het weinig zij of veel;
Num 13:19  En hoedanig het land zij, waarin hetzelve woont, of het goed zij of kwaad; en hoedanig de steden zijn, in dewelke hetzelve woont, of in legers, of in sterkten;
Num 13:20  Ook hoedanig het land zij, of het vet zij of mager, of er bomen in zijn of niet; en versterkt u, en neemt van de vrucht des lands. Die dagen nu waren de dagen der eerste vruchten van de wijndruiven.
Num 13:21  Alzo trokken zij op, en verspiedden het land, van de woestijn Zin af tot Rechob toe, waar men gaat naar Hamath.
Num 13:22  En zij trokken op in het zuiden, en kwamen tot Hebron toe, en daar waren Ahiman, Sesai en Talmai, kinderen van Enak; Hebron nu was zeven jaren gebouwd voor Zoan in Egypte.
Num 13:23  Daarna kwamen zij tot het dal Eskol, en sneden van daar een rank af met een tros wijndruiven, dien zij droegen met tweeen, op een draagstok; ook van de granaatappelen en van de vijgen.
Num 13:24  Diezelve plaats noemde men het dal Eskol, ter oorzake van den tros, dien de kinderen Israels van daar afgesneden hadden.
Num 13:25  Daarna keerden zij weder van het verspieden des lands, ten einde van veertig dagen.
Num 13:26  En zij gingen heen, en kwamen tot Mozes en tot Aaron, en tot de gehele vergadering der kinderen Israels, in de woestijn van Paran, naar Kades; en brachten bescheid weder aan hen, en aan de gehele vergadering, en lieten hun de vrucht des lands zien.
Num 13:27  En zij vertelden hem, en zeiden: Wij zijn gekomen tot dat land, waarheen gij ons gezonden hebt; en voorwaar, het is van melk en honig vloeiende, en dit is zijn vrucht.
Num 13:28  Behalve dat het een sterk volk is, hetwelk in dat land woont, en de steden zijn vast, en zeer groot; en ook hebben wij daar kinderen van Enak gezien.
Num 13:29  De Amalekieten wonen in het land van het zuiden; maar de Hethieten, en de Jebusieten, en de Amorieten wonen op het gebergte; en de Kanaanieten wonen aan de zee, en aan den oever van de Jordaan.
Num 13:30  Toen stilde Kaleb het volk voor Mozes, en zeide: Laat ons vrijmoedig optrekken, en dat erfelijk bezitten; want wij zullen dat voorzeker overweldigen!
Num 13:31  Maar de mannen, die met hem opgetrokken waren, zeiden: Wij zullen tot dat volk niet kunnen optrekken, want het is sterker dan wij.
Num 13:32  Alzo brachten zij een kwaad gerucht voort van het land, dat zij verspied hadden, aan de kinderen Israels, zeggende: Dat land, door hetwelk wij doorgegaan zijn, om het te verspieden, is een land, dat zijn inwoners verteert; en al het volk, hetwelk wij in het midden van hetzelve gezien hebben, zijn mannen van grote lengte.
Num 13:33  Wij hebben ook daar de reuzen gezien, en de kinderen van Enak, van de reuzen; en wij waren als sprinkhanen in onze ogen, alzo waren wij ook in hun ogen.
Num 14:1  Toen verhief zich de gehele vergadering, en zij hieven hun stem op, en het volk weende in dienzelven nacht.
Num 14:2  En al de kinderen Israels murmureerden tegen Mozes en tegen Aaron; en de gehele vergadering zeide tot hen: Och, of wij in Egypteland gestorven waren! of, och, of wij in deze woestijn gestorven waren!
Num 14:3  En waarom brengt ons de HEERE naar dat land, dat wij door het zwaard vallen, en onze vrouwen, en onze kinderkens ten roof worden? Zou het ons niet goed zijn naar Egypte weder te keren?
Num 14:4  En zij zeiden de een tot den ander: Laat ons een hoofd opwerpen, en wederkeren naar Egypte!
Num 14:5  Toen vielen Mozes en Aaron op hun aangezichten, voor het aangezicht van de ganse gemeente der vergadering van de kinderen Israels.
Num 14:6  En Jozua, de zoon van Nun, en Kaleb, de zoon van Jefunne, zijnde van degenen, die dat land verspied hadden, scheurden hun klederen.
Num 14:7  En zij spraken tot de ganse vergadering der kinderen Israels, zeggende: Het land, door hetwelk wij getrokken zijn, om hetzelve te verspieden, is een uitermate goed land.
Num 14:8  Indien de HEERE een welgevallen aan ons heeft, zo zal Hij ons in dat land brengen, en zal ons dat geven; een land, hetwelk van melk en honig is vloeiende.
Num 14:9  Alleen zijt tegen den HEERE niet wederspannig! en vreest gij niet het volk dezes lands; want zij zijn ons brood! hun schaduw is van hen geweken, en de HEERE is met ons; vreest hen niet!
Num 14:10  Toen zeide de ganse vergadering, dat men hen met stenen stenigen zoude. Maar de heerlijkheid des HEEREN verscheen in de tent der samenkomst, voor al de kinderen Israels.
Num 14:11  En de HEERE zeide tot Mozes: Hoe lang zal mij dit volk tergen? En hoe lang zullen zij aan Mij niet geloven, door alle tekenen, die Ik in het midden van hen gedaan heb?
Num 14:12  Ik zal het met pestilentie slaan, en Ik zal het verstoten; en Ik zal u tot een groter en sterker volk maken, dan dit is.
Num 14:13  En Mozes zeide tot den HEERE: Zo zullen het de Egyptenaars horen; want Gij hebt door Uw kracht dit volk uit het midden van hen doen optrekken;
Num 14:14  En zij zullen zeggen tot de inwoners van dit land, die gehoord hebben, dat Gij, HEERE! in het midden van dit volk zijt; dat Gij, HEERE! oog aan oog gezien wordt, dat Uw wolk over hen staat, en Gij in een wolkkolom voor hun aangezicht gaat des daags, en in een vuurkolom des nachts.
Num 14:15  En zoudt Gij dit volk als een enigen man doden, zo zouden de heidenen, die Uw gerucht gehoord hebben, spreken, zeggende:
Num 14:16  Omdat de HEERE dit volk niet kon brengen in dat land, hetwelk Hij hun gezworen had, zo heeft Hij hen geslacht in de woestijn!
Num 14:17  Nu dan, laat toch de kracht des HEEREN groot worden, gelijk als Gij gesproken hebt, zeggende:
Num 14:18  De HEERE is lankmoedig en groot van weldadigheid, vergevende de ongerechtigheid en overtreding, die den schuldige geenszins onschuldig houdt, bezoekende de ongerechtigheid der vaderen aan de kinderen, in het derde en in het vierde lid.
Num 14:19  Vergeef toch de ongerechtigheid dezes volks, naar de grootte Uwer goedertierenheid, en gelijk Gij ze aan dit volk, van Egypteland af tot hiertoe, vergeven hebt!
Num 14:20  En de HEERE zeide: Ik heb hun vergeven naar uw woord.
Num 14:21  Doch zekerlijk, zo waarachtig als Ik leef, zo zal de ganse aarde met de heerlijkheid des HEEREN vervuld worden!
Num 14:22  Want al de mannen, die gezien hebben Mijn heerlijkheid, en Mijn tekenen, die Ik in Egypte en in de woestijn gedaan heb, en Mij nu tienmaal verzocht hebben, en Mijner stem niet zijn gehoorzaam geweest;
Num 14:23  Zo zij het land, hetwelk Ik aan hun vaderen gezworen heb, zien zullen. Ja, geen van die Mij getergd hebben, zullen dat zien!
Num 14:24  Doch Mijn knecht Kaleb, omdat een andere geest met hem geweest is, en hij volhard heeft Mij na te volgen, zo zal Ik hem brengen tot het land, in hetwelk hij gekomen was, en zijn zaad zal het erfelijk bezitten.
Num 14:25  De Amalekieten nu en de Kanaanieten wonen in het dal; wendt u morgen, en maakt uw reize naar de woestijn, op den weg naar de Schelfzee.
Num 14:26  Daarna sprak de HEERE tot Mozes en tot Aaron, zeggende:
Num 14:27  Hoe lang zal Ik bij deze boze vergadering zijn, die tegen Mij zijn murmurerende? Ik heb gehoord de murmureringen van de kinderen Israels, waarmede zij tegen Mij zijn murmurerende.
Num 14:28  Zeg tot hen: Zo waarachtig als Ik leef, spreekt de HEERE, indien Ik ulieden zo niet doe, gelijk als gij in Mijn oren gesproken hebt!
Num 14:29  Uw dode lichamen zullen in deze woestijn vallen; en al uw getelden, naar uw gehele getal, van twintig jaren oud en daarboven, gij, die tegen Mij gemurmureerd hebt.
Num 14:30  Zo gij in dat land komt, over hetwelk Ik Mijn hand opgeheven heb, dat Ik u daarin zou doen wonen, behalve Kaleb, de zoon van Jefunne, en Jozua, de zoon van Nun.
Num 14:31  En uw kinderkens, waarvan gij zeidet: Zij zullen ten roof worden! die zal Ik daarin brengen, en die zullen bekennen dat land, hetwelk gij smadelijk verworpen hebt.
Num 14:32  Maar u aangaande, uw dode lichamen zullen in deze woestijn vallen!
Num 14:33  En uw kinderen zullen gaan weiden in deze woestijn, veertig jaren, en zullen uw hoererijen dragen, totdat uw dode lichamen verteerd zijn in deze woestijn.
Num 14:34  Naar het getal der dagen, in welke gij dat land verspied hebt, veertig dagen, elken dag voor elk jaar, zult gij uw ongerechtigheden dragen, veertig jaren, en gij zult gewaar worden Mijn afbreking.
Num 14:35  Ik, de HEERE, heb gesproken: zo Ik dit aan deze ganse boze vergadering dergenen, die zich tegen Mij verzameld hebben, niet doe, zij zullen in deze woestijn te niet worden, en zullen daar sterven!
Num 14:36  En die mannen, die Mozes gezonden had, om het land te verspieden, en wedergekomen zijnde, de ganse vergadering tegen hem hadden doen murmureren, een kwaad gerucht over dat land voortbrengende;
Num 14:37  Diezelfde mannen, die een kwaad gerucht van dat land voortgebracht hadden, stierven door een plaag, voor het aangezicht des HEEREN.
Num 14:38  Maar Jozua, de zoon van Nun, en Kaleb, de zoon van Jefunne, bleven levende van de mannen, die heengegaan waren, om het land te verspieden.
Num 14:39  En Mozes sprak deze woorden tot al de kinderen Israels. Toen treurde het volk zeer.
Num 14:40  En zij stonden des morgens vroeg op, en klommen op de hoogte des bergs, zeggende: Ziet, hier zijn wij, en wij zullen optrekken tot de plaats, die de HEERE gezegd heeft; want wij hebben gezondigd!
Num 14:41  Maar Mozes zeide: Waarom overtreedt gij alzo het bevel des HEEREN? Want dat zal geen voorspoed hebben.
Num 14:42  Trekt niet op, want de HEERE zal in het midden van u niet zijn; opdat gij niet geslagen wordt, voor het aangezicht uwer vijanden.
Num 14:43  Want de Amalekieten, en de Kanaanieten zijn daar voor uw aangezicht, en gij zult door het zwaard vallen; want, omdat gij u afgekeerd hebt van den HEERE, zo zal de HEERE met u niet zijn.
Num 14:44  Nochtans poogden zij vermetel, om op de hoogte des bergs te klimmen; maar de ark des verbonds des HEEREN en Mozes scheidden niet uit het midden des legers.
Num 14:45  Toen kwamen af de Amalekieten en de Kanaanieten, die in dat gebergte woonden, en sloegen hen, en versmeten hen, tot Horma toe.
Num 15:1  Daarna sprak de HEERE tot Mozes, zeggende:
Num 15:2  Spreek tot de kinderen Israels, en zeg tot hen: Wanneer gij gekomen zult zijn in het land uwer woningen, dat Ik u geven zal;
Num 15:3  En gij een vuuroffer den HEERE zult doen, een brandoffer, of slachtoffer, om af te zonderen een gelofte, of in een vrijwillig offer, of in uw gezette hoogtijden, om den HEERE een liefelijken reuk te maken, van runderen of van klein vee;
Num 15:4  Zo zal hij, die zijn offerande den HEERE offert, een spijsoffer offeren van een tiende meelbloem, gemengd met een vierendeel van een hin olie.
Num 15:5  En wijn ten drankoffer, een vierendeel van een hin, zult gij bereiden tot een brandoffer of tot een slachtoffer, voor een lam.
Num 15:6  Of voor een ram zult gij een spijsoffer bereiden, van twee tienden meelbloem, gemengd met olie, een derde deel van een hin.
Num 15:7  En wijn ten drankoffer, een derde deel van een hin, zult gij offeren tot een liefelijken reuk den HEERE.
Num 15:8  En wanneer gij een jong rund zult bereiden tot een brandoffer of een slachtoffer, om een gelofte af te zonderen, of ten dankoffer den HEERE;
Num 15:9  Zo zal hij tot een jong rund offeren een spijsoffer van drie tienden meelbloem, gemengd met olie, de helft van een hin.
Num 15:10  En wijn zult gij offeren ten drankoffer, de helft van een hin, tot een vuuroffer van liefelijken reuk den HEERE.
Num 15:11  Alzo zal gedaan worden met den enen os, of met den enen ram, of met het klein vee, van de lammeren, of van de geiten.
Num 15:12  Naar het getal, dat gij bereiden zult, zult gij alzo doen met elkeen, naar hun getal.
Num 15:13  Alle inboorling zal deze dingen alzo doen, offerende een vuuroffer tot een liefelijken reuk den HEERE.
Num 15:14  Wanneer ook een vreemdeling bij u als vreemdeling verkeert, of die in het midden van u is, in uw geslachten, en hij een vuuroffer zal bereiden tot een liefelijken reuk den HEERE; gelijk als gij zult doen, alzo zal hij doen.
Num 15:15  Gij, gemeente, het zij ulieden en den vreemdeling, die als vreemdeling bij u verkeert, enerlei inzetting: ter eeuwige inzetting bij uw geslachten, gelijk gijlieden, alzo zal de vreemdeling voor des HEEREN aangezicht zijn.
Num 15:16  Enerlei wet en enerlei recht zal ulieden zijn, en den vreemdeling, die bij ulieden als vreemdeling verkeert.
Num 15:17  Voorts sprak de HEERE tot Mozes, zeggende:
Num 15:18  Spreek tot de kinderen Israels, en zeg tot hen: Als gij zult gekomen zijn in het land, waarheen Ik u inbrengen zal,
Num 15:19  Zo zal het geschieden, als gij van het brood des lands zult eten, dan zult gij den HEERE een hefoffer offeren.
Num 15:20  De eerstelingen uws deegs, een koek zult gij tot een hefoffer offeren; gelijk het hefoffer des dorsvloers zult gij dat offeren.
Num 15:21  Van de eerstelingen uws deegs zult gij den HEERE een hefoffer geven, bij uw geslachten.
Num 15:22  Voorts wanneer gijlieden afgedwaald zult zijn, en niet gedaan hebben al deze geboden, die de HEERE tot Mozes gesproken heeft;
Num 15:23  Alles, wat u de HEERE door de hand van Mozes geboden heeft; van dien dag af, dat het de HEERE geboden heeft, en voortaan bij uw geslachten;
Num 15:24  Zo zal het geschieden, indien iets bij dwaling gedaan, en voor de ogen der vergadering verborgen is, dat de ganse vergadering een var, een jong rund, zal bereiden ten brandoffer, tot een liefelijken reuk den HEERE, met zijn spijsoffer en zijn drankoffer, naar de wijze; en een geitenbok ten zondoffer.
Num 15:25  En de priester zal de verzoening doen voor de ganse vergadering van de kinderen Israels, en het zal hun vergeven worden; want het was een afdwaling, en zij hebben hun offerande gebracht, een vuuroffer den HEERE, en hun zondoffer, voor het aangezicht des HEEREN, over hun afdwaling.
Num 15:26  Het zal dan aan de ganse vergadering der kinderen Israels vergeven worden, ook den vreemdeling, die in het midden van henlieden als vreemdeling verkeert; want het is het ganse volk door dwaling overkomen.
Num 15:27  En indien een ziel door afdwaling gezondigd zal hebben, die zal een eenjarige geit ten zondoffer offeren.
Num 15:28  En de priester zal de verzoening doen over de dwalende ziel, als zij gezondigd heeft door afdwaling, voor het aangezicht des HEEREN, doende de verzoening over haar; en het zal haar vergeven worden.
Num 15:29  Den inboorling der kinderen Israels, en den vreemdeling, die in hunlieder midden als vreemdeling verkeert, enerlei wet zal ulieden zijn, dengene, die het door afdwaling doet.
Num 15:30  Maar de ziel, die iets gedaan zal hebben met opgeheven hand, hetzij van inboorlingen of van vreemdelingen, die smaadt den HEERE; en diezelve ziel zal uitgeroeid worden uit het midden van haar volk;
Num 15:31  Want zij heeft het woord des HEEREN veracht en Zijn gebod vernietigd; diezelve ziel zal ganselijk uitgeroeid worden; haar ongerechtigheid is op haar.
Num 15:32  Als nu de kinderen Israels in de woestijn waren, zo vonden zij een man, hout lezende op den sabbatdag.
Num 15:33  En die hem vonden, hout lezende, brachten hem tot Mozes, en tot Aaron, en tot de ganse vergadering.
Num 15:34  En zij stelden hem in bewaring; want het was niet verklaard, wat hem gedaan zou worden.
Num 15:35  Zo zeide de HEERE tot Mozes: Die man zal zekerlijk gedood worden; de ganse vergadering zal hem met stenen stenigen buiten het leger.
Num 15:36  Toen bracht hem de ganse vergadering uit tot buiten het leger, en zij stenigden hem met stenen, dat hij stierf, gelijk als de HEERE Mozes geboden had.
Num 15:37  En de HEERE sprak tot Mozes, zeggende:
Num 15:38  Spreek tot de kinderen Israels, en zeg tot hen: Dat zij zich snoertjes maken aan de hoeken hunner klederen, bij hun geslachten; en op de snoertjes des hoeks zullen zij een hemelsblauwen draad zetten.
Num 15:39  En hij zal ulieden aan de snoertjes zijn, opdat gij het aanziet, en aan al de geboden des HEEREN gedenkt, en die doet; en gij zult naar uw hart, en naar uw ogen niet sporen, die gij zijt nahoererende;
Num 15:40  Opdat gij gedenkt en doet al Mijn geboden, en uw God heilig zijt.
Num 15:41  Ik ben de HEERE, uw God, Die u uit Egypteland uitgevoerd heb, om u tot een God te zijn; Ik ben de HEERE, uw God!
Num 16:1  Korach nu, de zoon van Jizhar, zoon van Kohath, zoon van Levi, nam tot zich zo Dathan als Abiram, zonen van Eliab, en On, den zoon van Peleth, zonen van Ruben.
Num 16:2  En zij stonden op voor het aangezicht van Mozes, mitsgaders tweehonderd en vijftig mannen uit de kinderen Israels, oversten der vergadering, de geroepenen der samenkomst, mannen van naam.
Num 16:3  En zij vergaderden zich tegen Mozes, en tegen Aaron, en zeiden tot hen: Het is te veel voor u, want deze ganse vergadering, zij allen, zijn heilig, en de HEERE is in het midden van hen; waarom dan verheft gijlieden u over de gemeente des HEEREN?
Num 16:4  Als Mozes dit hoorde, zo viel hij op zijn aangezicht.
Num 16:5  En hij sprak tot Korach, en tot zijn ganse vergadering, zeggende: Morgen vroeg dan zal de HEERE bekend maken, wie de Zijne, en de heilige is, dien Hij tot Zich zal doen naderen; en wien Hij verkoren zal hebben, dien zal Hij tot Zich doen naderen.
Num 16:6  Doet dit: neemt u wierookvaten, Korach en zijn ganse vergadering;
Num 16:7  En doet morgen vuur daarin, legt reukwerk daarop voor het aangezicht des HEEREN; en het zal geschieden, dat de man, dien de HEERE verkiezen zal, die zal heilig zijn. Het is te veel voor u, gij, kinderen van Levi!
Num 16:8  Voorts zeide Mozes tot Korach: Hoort toch, gij, kinderen van Levi!
Num 16:9  Is het u te weinig, dat de God van Israel u van de vergadering van Israel heeft afgescheiden, om ulieden tot Zich te doen naderen; om den dienst van des HEEREN tabernakel te bedienen, en te staan voor het aangezicht der vergadering, om hen te dienen?
Num 16:10  Daar Hij u, en al uw broederen, de kinderen van Levi, met u, heeft doen naderen; zoekt gij nu ook het priesterambt?
Num 16:11  Daarom gij, en uw ganse vergadering, gij zijt vergaderd tegen den HEERE, want Aaron, wat is hij, dat gij tegen hem murmureert?
Num 16:12  En Mozes schikte heen, om Dathan en Abiram, de zonen van Eliab, te roepen; maar zij zeiden: Wij zullen niet opkomen!
Num 16:13  Is het te weinig, dat gij ons uit een land, van melk en honig vloeiende, hebt opgevoerd, om ons te doden in de woestijn, dat gij ook uzelven ten enenmaal over ons tot een overheer maakt?
Num 16:14  Ook hebt gij ons niet gebracht in een land, dat van melk en honig vloeit, noch ons akkers en wijngaarden ten erfdeel gegeven. Zult gij de ogen dezer mannen uitgraven? Wij zullen niet opkomen!
Num 16:15  Toen ontstak Mozes zeer, en hij zeide tot den HEERE: Zie hun offer niet aan! Ik heb niet een ezel van hen genomen, en niet een van hen kwaad gedaan.
Num 16:16  Voorts zeide Mozes tot Korach: Gij, en uw ganse vergadering, weest voor het aangezicht des HEEREN; gij, en zij, ook Aaron, op morgen.
Num 16:17  En neemt een ieder zijn wierookvat, en legt reukwerk daarin, en brengt voor het aangezicht des HEEREN, een ieder zijn wierookvat, tweehonderd en vijftig wierookvaten; ook gij, en Aaron, een ieder zijn wierookvat.
Num 16:18  Zo namen zij een ieder zijn wierookvat, en deden vuur daarin, en leiden reukwerk daarin; en zij stonden voor de deur van de tent der samenkomst, ook Mozes en Aaron.
Num 16:19  En Korach deed de ganse vergadering tegen hen verzamelen, aan de deur van de tent der samenkomst. Toen verscheen de heerlijkheid des HEEREN aan deze ganse vergadering.
Num 16:20  En de HEERE sprak tot Mozes en tot Aaron, zeggende:
Num 16:21  Scheidt u af uit het midden van deze vergadering, en Ik zal hen als in een ogenblik verteren!
Num 16:22  Maar zij vielen op hun aangezichten, en zeiden: O God! God der geesten van alle vlees! een enig man zal gezondigd hebben, en zult Gij U over deze ganse vergadering grotelijks vertoornen?
Num 16:23  En de HEERE sprak tot Mozes, zeggende:
Num 16:24  Spreek tot deze vergadering, zeggende: Gaat op van rondom de woning van Korach, Dathan en Abiram.
Num 16:25  Toen stond Mozes op, en ging tot Dathan en Abiram; en achter hem gingen de oudsten van Israel.
Num 16:26  En hij sprak tot de vergadering, zeggende: Wijkt toch af van de tenten dezer goddeloze mannen, en roert niets aan van hetgeen hunner is, opdat gij niet misschien verdaan wordt in al hun zonden.
Num 16:27  Zo gingen zij op van de woning van Korach, Dathan en Abiram, van rondom; maar Dathan en Abiram gingen uit, staande in de deur hunner tenten, met hun vrouwen, en hun zonen, en hun kinderkens.
Num 16:28  Toen zeide Mozes: Hieraan zult gij bekennen, dat de HEERE mij gezonden heeft, om al deze daden te doen, dat zij niet uit mijn eigen hart zijn.
Num 16:29  Indien deze zullen sterven, gelijk alle mensen sterven, en over hen een bezoeking zal gedaan worden, naar aller mensen bezoeking, zo heeft mij de HEERE niet gezonden.
Num 16:30  Maar indien de HEERE wat nieuws zal scheppen, en het aardrijk zijn mond zal opendoen, en verslinden hen met alles wat hunner is, en zij levend ter helle zullen nedervaren; alsdan zult gij bekennen, dat deze mannen de HEERE getergd hebben.
Num 16:31  En het geschiedde, als hij geeindigd had al deze woorden te spreken, zo werd het aardrijk, dat onder hen was, gekloofd;
Num 16:32  En de aarde opende haar mond, en verslond hen met hun huizen, en allen mensen, die Korach toebehoorden, en al de have.
Num 16:33  En zij voeren neder, zij en alles wat hunner was, levend ter helle; en de aarde overdekte hen, en zij kwamen om uit het midden der gemeente.
Num 16:34  En het ganse Israel, dat rondom hen was, vlood voor hun geschrei; want zij zeiden: Dat ons de aarde misschien niet verslinde!
Num 16:35  Daartoe ging een vuur uit van den HEERE, en verteerde die tweehonderd en vijftig mannen, die reukwerk offerden.
Num 16:36  En de HEERE sprak tot Mozes, zeggende:
Num 16:37  Zeg tot Eleazar, den zoon van Aaron, den priester, dat hij de wierookvaten uit den brand opneme; en strooi het vuur verre weg; want zij zijn heilig;
Num 16:38  Te weten de wierookvaten van dezen, die tegen hun zielen gezondigd hebben; dat men uitgerekte platen daarvan make, tot een overdeksel voor het altaar; want zij hebben ze gebracht voor het aangezicht des HEEREN, daarom zijn zij heilig; en zij zullen den kinderen Israels tot een teken zijn.
Num 16:39  En Eleazar, de priester, nam de koperen wierookvaten, die de verbranden gebracht hadden, en zij rekten ze uit tot een overtreksel voor het altaar;
Num 16:40  Ter nagedachtenis voor de kinderen Israels, opdat niemand vreemds, die niet uit het zaad van Aaron is, nadere om reukwerk aan te steken voor het aangezicht des HEEREN; opdat hij niet worde als Korach, en zijn vergadering, gelijk als hem de HEERE door den dienst van Mozes gesproken had.
Num 16:41  Maar des anderen daags murmureerde de ganse vergadering der kinderen Israels tegen Mozes en tegen Aaron, zeggende: Gijlieden hebt des HEEREN volk gedood!
Num 16:42  En het geschiedde, als de vergadering zich verzamelde tegen Mozes en Aaron, en zich wendde naar de tent der samenkomst, ziet, zo bedekte haar die wolk; en de heerlijkheid des HEEREN verscheen.
Num 16:43  Mozes nu en Aaron kwamen tot voor de tent der samenkomst.
Num 16:44  Toen sprak de HEERE tot Mozes, zeggende:
Num 16:45  Maak u op uit het midden van deze vergadering, en Ik zal hen verteren, als in een ogenblik! Toen vielen zij op hun aangezichten.
Num 16:46  En Mozes zeide tot Aaron: Neem het wierookvat, en doe vuur daarin van het altaar, en leg reukwerk daarop, haastelijk gaande tot de vergadering, doe over hen verzoening; want een grote toorn is van voor het aangezicht des HEEREN uitgegaan, de plaag heeft aangevangen.
Num 16:47  En Aaron nam het, gelijk als Mozes gesproken had, en liep in het midden der gemeente, en ziet, de plaag had aangevangen onder het volk; en hij leide reukwerk daarin, en deed verzoening over het volk.
Num 16:48  En hij stond tussen de doden en tussen de levenden; alzo werd de plaag opgehouden.
Num 16:49  Die nu aan die plaag gestorven zijn, waren veertien duizend en zevenhonderd, behalve die gestorven waren om de zaak van Korach.
Num 16:50  En Aaron keerde weder tot Mozes aan de deur van de tent der samenkomst; en de plaag was opgehouden.
Num 17:1  Toen sprak de HEERE tot Mozes, zeggende:
Num 17:2  Spreek tot de kinderen Israels, en neem van hen voor elk vaderlijk huis een staf, van al hun oversten, naar het huis hunner vaderen, twaalf staven; eens iegelijken naam zult gij schrijven op zijn staf.
Num 17:3  Doch Aarons naam zult gij schrijven op den staf van Levi; want een staf zal er zijn voor het hoofd van het huis hunner vaderen.
Num 17:4  En gij zult ze wegleggen in de tent der samenkomst, voor de getuigenis, waarheen Ik met ulieden samenkomen zal.
Num 17:5  En het zal geschieden, dat de staf des mans, welke Ik zal verkoren hebben, zal bloeien; en Ik zal stillen de murmureringen van de kinderen Israels tegen Mij, welke zij tegen ulieden murmureerden.
Num 17:6  Mozes dan sprak tot de kinderen Israels, en al hun oversten gaven aan hem een staf, voor elken overste een staf, naar het huis hunner vaderen, twaalf staven; Aarons staf was ook onder hun staven.
Num 17:7  En Mozes leide deze staven weg, voor het aangezicht des HEEREN, in de tent der getuigenis.
Num 17:8  Het geschiedde nu des anderen daags, dat Mozes in de tent der getuigenis inging; en ziet, Aarons staf, voor het huis van Levi, bloeide; want hij bracht bloeisel voort, en bloesemde bloesem, en droeg amandelen.
Num 17:9  Toen bracht Mozes al deze staven uit, van voor het aangezicht des HEEREN, tot al de kinderen Israels; en zij zagen het, en namen elk zijn staf.
Num 17:10  Toen zeide de HEERE tot Mozes: Breng de staf van Aaron weder voor de getuigenis, in bewaring, tot een teken voor de wederspannige kinderen; alzo zult gij een einde maken van hun murmureringen tegen Mij, dat zij niet sterven.
Num 17:11  En Mozes deed het; gelijk als de HEERE hem geboden had, alzo deed hij.
Num 17:12  Toen spraken de kinderen Israels tot Mozes, zeggende: Zie, wij geven den geest, wij vergaan, wij allen vergaan!
Num 17:13  Al wie enigszins nadert tot den tabernakel des HEEREN, zal sterven; zullen wij dan den geest gevende verdaan worden?
Num 18:1  Zo zeide de HEERE tot Aaron: Gij, en uw zonen, en het huis uws vaders met u, zult dragen de ongerechtigheid des heiligdoms; en gij, en uw zonen met u, zult dragen de ongerechtigheid van uw priesterambt.
Num 18:2  En ook zult gij uw broederen, den stam van Levi, den stam uws vaders, met u doen naderen, dat zij u bijgevoegd worden, en u dienen; maar gij, en uw zonen met u, zult zijn voor de tent der getuigenis.
Num 18:3  En zij zullen uw wacht waarnemen, en de wacht der ganse tent; doch tot het gereedschap des heiligdoms en het altaar zullen zij niet naderen, opdat zij niet sterven, zo zij als gijlieden.
Num 18:4  Maar zij zullen u bijgevoegd worden, en de wacht van de tent der samenkomst waarnemen, in allen dienst der tent; en een vreemde zal tot u niet naderen.
Num 18:5  Gijlieden nu zult waarnemen de wacht des heiligdoms, en de wacht des altaars; opdat er geen verbolgenheid meer zij over de kinderen Israels.
Num 18:6  Want Ik, zie, Ik heb uw broederen, de Levieten, uit het midden der kinderen Israels genomen; zij zijn ulieden een gave, gegeven den HEERE, om den dienst van de tent der samenkomst te bedienen.
Num 18:7  Maar gij, en uw zonen met u, zult ulieder priesterambt waarnemen in alle zaken des altaars, en in hetgeen van binnen den voorhang is, dat zult gijlieden bedienen; uw priesterambt geve Ik u tot een dienst van een geschenk; en de vreemde, die nadert, zal gedood worden.
Num 18:8  Voorts sprak de HEERE tot Aaron: En Ik, zie, Ik heb u gegeven de wacht Mijner hefofferen, met alle heilige dingen van de kinderen Israels heb Ik ze u gegeven, om der zalving wil, en aan uw zonen, tot een eeuwige inzetting.
Num 18:9  Dit zult gij hebben van de heiligheid der heiligheden, uit het vuur: al hun offeranden, met al hun spijsoffer, en met al hun zondoffer, en met al hun schuldoffer, dat zij Mij zullen wedergeven; het zal u en uw zonen een heiligheid der heiligheden zijn.
Num 18:10  Aan het allerheiligste zult gij dat eten; al wat mannelijk is zal dat eten; het zal u een heiligheid zijn.
Num 18:11  Ook zal dit het uwe zijn: het hefoffer hunner gave, met alle beweegofferen der kinderen Israels; Ik heb ze aan u gegeven, en aan uw zonen, en aan uw dochteren met u, tot een eeuwige inzetting; al wie in uw huis rein is, zal dat eten.
Num 18:12  Al het beste van de olie, en al het beste van de most, en van koren, hun eerstelingen, die zij den HEERE zullen geven, u heb Ik ze gegeven.
Num 18:13  De eerste vruchten van alles, wat in hun land is, die zij den HEERE zullen brengen, zullen uwe zijn; al wie in uw huis rein is, zal dat eten.
Num 18:14  Al het verbannene in Israel zal het uwe zijn.
Num 18:15  Al wat de baarmoeder opent, van alle vlees, dat zij den HEERE zullen brengen, onder de mensen, en onder de beesten, zal het uwe zijn; doch de eerstgeborenen der mensen zult gij ganselijk lossen; ook zult gij lossen de eerstgeborenen der onreine beesten.
Num 18:16  Die nu onder dezelve gelost zullen worden, zult gij van een maand oud lossen, naar uw schatting, voor het geld van vijf sikkelen, naar den sikkel des heiligdoms, die is twintig gera.
Num 18:17  Maar het eerstgeborene van een koe, of het eerstgeborene van een schaap, of het eerstgeborene van een geit zult gij niet lossen, zij zijn heilig; hun bloed zult gij sprengen op het altaar, en hun ver zult gij aansteken, tot een vuuroffer van liefelijken reuk den HEERE.
Num 18:18  En hun vlees zal het uwe zijn; gelijk de beweegborst, en gelijk de rechterschouder, zal het uwe zijn.
Num 18:19  Alle hefofferen der heilige dingen, die de kinderen Israels den HEERE zullen offeren, heb Ik aan u gegeven, en aan uw zonen, en aan uw dochteren met u, tot een eeuwige inzetting; het zal een eeuwig zoutverbond zijn, voor het aangezicht des HEEREN, voor u en voor uw zaad met u.
Num 18:20  Ook zeide de HEERE tot Aaron: Gij zult in hun land niet erven, en gij zult geen deel in het midden van henlieden hebben; Ik ben uw deel en uw erfenis, in het midden van de kinderen Israels.
Num 18:21  En zie, aan de kinderen van Levi heb Ik alle tienden in Israel ter erfenis gegeven, voor hun dienst, dien zij bedienen, den dienst van de tent der samenkomst.
Num 18:22  En de kinderen Israels zullen niet meer naderen tot de tent der samenkomst, om zonde te dragen en te sterven.
Num 18:23  Maar de Levieten, die zullen bedienen den dienst van de tent der samenkomst, en die zullen hun ongerechtigheid dragen; het zal een eeuwige inzetting zijn voor uw geslachten; en in het midden van de kinderen Israels zullen zij geen erfenis erven.
Num 18:24  Want de tienden der kinderen Israels, die zij den HEERE tot een hefoffer zullen offeren, heb Ik aan de Levieten tot een erfenis gegeven; daarom heb Ik tot hen gezegd: Zij zullen in het midden van de kinderen Israels geen erfenis erven.
Num 18:25  En de HEERE sprak tot Mozes, zeggende:
Num 18:26  Gij zult ook tot de Levieten spreken, en tot hen zeggen: Wanneer gij van de kinderen Israels de tienden zult ontvangen hebben, die Ik u voor uw erfenis van henlieden gegeven heb, zo zult gij daarvan een hefoffer des HEEREN offeren, de tienden van die tienden;
Num 18:27  En het zal u gerekend worden tot uw hefoffer, als koren van den dorsvloer, en als de volheid van de perskuip.
Num 18:28  Alzo zult gij ook een hefoffer des HEEREN offeren van al uw tienden, die gij van de kinderen Israels zult hebben ontvangen; en gij zult daarvan des HEEREN hefoffer geven aan den priester Aaron.
Num 18:29  Van al uw gaven zult gij alle hefoffer des HEEREN offeren; van al het beste van die, van zijn heiliging daarvan.
Num 18:30  Gij zult dan tot hen zeggen: Als gij deszelfs beste daarvan offert, zo zal het den Levieten toegerekend worden als een inkomen des dorsvloers, en als een inkomen der perskuip.
Num 18:31  En gij zult dat eten in alle plaatsen, gij en uw huis; want het is ulieden een loon voor uw dienst in de tent der samenkomst.
Num 18:32  Zo zult gij daarover geen zonde dragen, als gij deszelfs beste daarvan offert; en gij zult de heilige dingen van de kinderen Israels niet ontheiligen, opdat gij niet sterft.
Num 19:1  Wijders sprak de HEERE tot Mozes en tot Aaron, zeggende:
Num 19:2  Dit is de inzetting van de wet, die de HEERE geboden heeft, zeggende: Spreek tot de kinderen Israels, dat zij tot u brengen een rode volkomen vaars, in welke geen gebrek is, op welke geen juk gekomen is.
Num 19:3  En gij zult die geven aan Eleazar, den priester; en hij zal ze uitbrengen tot buiten het leger, en men zal haar voor zijn aangezicht slachten.
Num 19:4  En Eleazar, den priester, zal van haar bloed met zijn vinger nemen, en hij zal van haar bloed recht tegenover de tent der samenkomst zevenmaal sprengen.
Num 19:5  Voorts zal men deze vaars voor zijn ogen verbranden; haar vel, en haar vlees, en haar bloed, met haar mest, zal men verbranden.
Num 19:6  En de priester zal nemen cederhout, en hysop, en scharlaken, en werpen ze in het midden van den brand dezer vaars.
Num 19:7  Dan zal de priester zijn klederen wassen, en zijn vlees met water baden, en daarna in het leger gaan; en de priester zal onrein zijn tot aan den avond.
Num 19:8  Ook die haar verbrand heeft, zal zijn klederen met water wassen, en zijn vlees met water baden, en onrein zijn tot aan den avond.
Num 19:9  En een rein man zal de as dezer vaars verzamelen, en buiten het leger in een reine plaats wegleggen; en het zal zijn ter bewaring voor de vergadering van de kinderen Israels, tot het water der afzondering; het is ontzondiging.
Num 19:10  En die de as dezer vaars verzameld heeft, zal zijn klederen wassen, en onrein zijn tot aan den avond. Dit zal den kinderen Israels, en den vreemdeling, die in het midden van henlieden als vreemdeling verkeert, tot een eeuwige inzetting zijn.
Num 19:11  Wie een dode, enig dood lichaam van een mens, aanroert, die zal zeven dagen onrein zijn.
Num 19:12  Op den derden dag zal hij zich daarmede ontzondigen, zo zal hij op den zevenden dag rein zijn; maar indien hij zich op den derden dag niet ontzondigt, zo zal hij op den zevenden dag niet rein zijn.
Num 19:13  Al wie een dode, het dode lichaam eens mensen, die gestorven zal zijn, aanroert, en zich niet ontzondigd zal hebben, die verontreinigt den tabernakel des HEEREN; daarom zal die ziel uitgeroeid worden uit Israel; omdat het water der afzondering op hem niet gesprengd is, zal hij onrein zijn; zijn onreinigheid is nog in hem.
Num 19:14  Dit is de wet, wanneer een mens zal gestorven zijn in een tent: al wie in die tent ingaat, en al wie in die tent is, zal zeven dagen onrein zijn.
Num 19:15  Ook alle open gereedschap, waarop geen deksel gebonden is, dat is onrein.
Num 19:16  En al wie in het open veld een, die met het zwaard verslagen is, of een dode, of het gebeente eens mensen, of een graf zal aangeroerd hebben, zal zeven dagen onrein zijn.
Num 19:17  Voor een onreine nu zullen zij nemen van het stof des brands der ontzondiging, en daarop levend water doen in een vat.
Num 19:18  En een rein man zal hysop nemen, en in dat water dopen, en sprengen het aan die tent, en op al het gereedschap, en aan de zielen, die daar geweest zijn; insgelijks aan dengene, die een gebeente, of een verslagene, of een dode, of een graf aangeroerd heeft.
Num 19:19  En de reine zal den onreine op den derden dag, en op den zevenden dag besprengen; en op den zevenden dag zal hij hem ontzondigen; en hij zal zijn klederen wassen, en zich met water baden, en op den avond rein zijn.
Num 19:20  Wie daarentegen onrein zal zijn, en zich niet zal ontzondigen, die ziel zal uit het midden der gemeente uitgeroeid worden; want hij heeft het heiligdom des HEEREN verontreinigd, het water der afzondering is op hem niet gesprengd, hij is onrein.
Num 19:21  Dit zal hunlieden zijn tot een eeuwige inzetting. En die het water der afzondering sprengt, zal zijn klederen wassen; ook wie het water der afzondering aanroert, die zal onrein zijn tot aan den avond.
Num 19:22  Ja, al wat die onreine aangeroerd zal hebben, zal onrein zijn; en de ziel, die dat aangeroerd zal hebben, zal onrein zijn tot aan den avond.
Num 20:1  Als de kinderen Israels, de ganse vergadering, in de woestijn Zin gekomen waren, in de eerste maand, zo bleef het volk te Kades. En Mirjam stierf aldaar, en zij werd aldaar begraven.
Num 20:2  En er was geen water voor de vergadering; toen vergaderden zij zich tegen Mozes en tegen Aaron.
Num 20:3  En het volk twistte met Mozes, en zij spraken, zeggende: Och, of wij den geest gegeven hadden, toen onze broeders voor het aangezicht des HEEREN den geest gaven!
Num 20:4  Waarom toch hebt gijlieden de gemeente des HEEREN in deze woestijn gebracht, dat wij daar sterven zouden, wij en onze beesten?
Num 20:5  En waarom hebt gijlieden ons doen optrekken uit Egypte, om ons te brengen in deze kwade plaats? Het is geen plaats van zaad, noch van vijgen, noch van wijnstokken, noch van granaatappelen; ook is er geen water om te drinken.
Num 20:6  Toen gingen Mozes en Aaron van het aangezicht der gemeente tot de deur van de tent der samenkomst, en zij vielen op hun aangezichten; en de heerlijkheid des HEEREN verscheen hun.
Num 20:7  En de HEERE sprak tot Mozes, zeggende:
Num 20:8  Neem dien staf, en verzamel de vergadering, gij en Aaron, uw broeder, en spreekt gijlieden tot de steenrots voor hun ogen, zo zal zij hun water geven; alzo zult gij hun water voortbrengen uit de steenrots, en gij zult de vergadering en haar beesten drenken.
Num 20:9  Toen nam Mozes den staf van voor het aangezicht des HEEREN, gelijk als Hij hem geboden had.
Num 20:10  En Mozes en Aaron vergaderden de gemeente voor de steenrots, en hij zeide tot hen: Hoort toch, gij wederspannigen, zullen wij water voor ulieden uit deze steenrots hervoorbrengen?
Num 20:11  Toen hief Mozes zijn hand op, en hij sloeg de steenrots tweemaal met zijn staf; en er kwam veel waters uit, zodat de vergadering dronk, en haar beesten.
Num 20:12  Derhalve zeide de HEERE tot Mozes en tot Aaron: Omdat gijlieden Mij niet geloofd hebt, dat gij Mij heiligdet voor de ogen der kinderen van Israel, daarom zult gijlieden deze gemeente niet inbrengen in het land, hetwelk Ik hun gegeven heb.
Num 20:13  Dit zijn de wateren van Meriba, daar de kinderen Israels met den HEERE om getwist hebben; en Hij werd aan hen geheiligd.
Num 20:14  Daarna zond Mozes boden uit Kades tot den koning van Edom, welke zeiden: Alzo zegt uw broeder Israel: Gij weet al de moeite, die ons ontmoet is;
Num 20:15  Dat onze vaders naar Egypte afgetogen zijn, en wij in Egypte vele dagen gewoond hebben; en dat de Egyptenaars aan ons en onze vaderen kwaad gedaan hebben.
Num 20:16  Toen riepen wij tot den HEERE, en Hij hoorde onze stem, en Hij zond een Engel, en Hij leidde ons uit Egypte; en ziet, wij zijn te Kades, en stad aan het uiterste uwer landpale.
Num 20:17  Laat ons toch door uw land trekken; wij zullen niet trekken door den akker, noch door de wijngaarden, noch zullen het water der putten drinken; wij zullen den koninklijken weg gaan, wij zullen niet afwijken ter rechter hand noch ter linkerhand, totdat wij door uw landpalen zullen getrokken zijn.
Num 20:18  Doch Edom zeide tot hem: Gij zult door mij niet trekken, opdat ik niet misschien met het zwaard uitga u tegemoet!
Num 20:19  Toen zeiden de kinderen Israels tot hem: Wij zullen door den gebaanden weg optrekken, en indien wij van uw water drinken, ik en mijn vee, zo zal ik deszelfs prijs daarvoor geven; ik zal alleenlijk, zonder iets anders, te voet doortrekken.
Num 20:20  Doch hij zeide: Gij zult niet doortrekken! En Edom is hem tegemoet uitgetrokken, met een zwaar volk, en met een sterke hand.
Num 20:21  Alzo weigerde Edom Israel toe te laten door zijn landpale te trekken; daarom week Israel van hem af.
Num 20:22  Toen reisden zij van Kades; en de kinderen Israels kwamen, de ganse vergadering, aan den berg Hor.
Num 20:23  De HEERE nu sprak tot Mozes, en tot Aaron, aan den berg Hor, aan de pale van het land van Edom, zeggende:
Num 20:24  Aaron zal tot zijn volken verzameld worden; want hij zal niet komen in het land, hetwelk Ik aan de kinderen Israels gegeven heb, omdat gijlieden Mijn mond wederspannig geweest zijt bij de wateren van Meriba.
Num 20:25  Neem Aaron, en Eleazar, zijn zoon, en doe hen opklimmen tot den berg Hor.
Num 20:26  En trek Aaron zijn klederen uit, en trek ze Eleazar, zijn zoon, aan; want Aaron zal verzameld worden, en daar sterven.
Num 20:27  Mozes nu deed, gelijk als de HEERE geboden had; want zij klommen op tot den berg Hor, voor de ogen der ganse vergadering.
Num 20:28  En Mozes trok Aaron zijn klederen uit, en hij trok ze zijn zoon Eleazar aan; en Aaron stierf aldaar, op de hoogte diens bergs. Toen kwam Mozes en Eleazar van dien berg af.
Num 20:29  Toen de ganse vergadering zag, dat Aaron overleden was, zo beweenden zij Aaron dertig dagen, het ganse huis van Israel.
Num 21:1  Als de Kanaaniet, de koning van Harad, wonende tegen het zuiden, hoorde, dat Israel door den weg der verspieders kwam, zo streed hij tegen Israel, en hij voerde enige gevangenen uit denzelven gevankelijk weg.
Num 21:2  Toen beloofde Israel den HEERE een gelofte, en zeide: Indien Gij dit volk geheel in mijn hand geeft, zo zal ik hun steden verbannen.
Num 21:3  De HEERE dan verhoorde de stem van Israel, en gaf de Kanaanieten over; en hij verbande hen en hun steden; en hij noemde den naam dier plaats Horma.
Num 21:4  Toen reisden zij van den berg Hor, op den weg der Schelfzee, dat zij om het land der Edomieten heentogen; doch de ziel des volks werd verdrietig op dezen weg.
Num 21:5  En het volk sprak tegen God en tegen Mozes: Waarom hebt gijlieden ons doen optrekken uit Egypte, opdat wij sterven zouden in de woestijn? Want hier is geen brood, ook geen water, en onze ziel walgt over dit zeer lichte brood.
Num 21:6  Toen zond de HEERE vurige slangen onder het volk, die beten het volk; en er stierf veel volks van Israel.
Num 21:7  Daarom kwam het volk tot Mozes, en zij zeiden: Wij hebben gezondigd, omdat wij tegen den HEERE en tegen u gesproken hebben; bid den HEERE, dat Hij deze slangen van ons wegneme. Toen bad Mozes voor het volk.
Num 21:8  En de HEERE zeide tot Mozes: Maak u een vurige slang, en stel ze op een stang; en het zal geschieden, dat al wie gebeten is, als hij haar aanziet, zo zal hij leven.
Num 21:9  En Mozes maakte een koperen slang, en stelde ze op een stang; en het geschiedde, als een slang iemand beet, zo zag hij de koperen slang aan, en hij bleef levend.
Num 21:10  Toen verreisden de kinderen Israels, en zij legerden zich te Oboth.
Num 21:11  Daarna reisden zij van Oboth, en legerden zich aan de heuvelen van Abarim in de woestijn, die tegenover Moab is, tegen den opgang der zon.
Num 21:12  Van daar reisden zij, en legerden zich bij de beek Zered.
Num 21:13  Van daar reisden zij, en legerden zich aan deze zijde van de Arnon, welke in de woestijn is, uitgaande uit de landpalen der Amorieten; want de Arnon is de landpale van Moab, tussen Moab en tussen de Amorieten.
Num 21:14  (Daarom wordt gezegd in het boek van de oorlogen des HEEREN: Tegen Waheb, in een wervelwind, en tegen de beken Arnon,
Num 21:15  En den afloop der beken, die zich naar de gelegenheid van Ar wendt, en leent aan de landpale van Moab.)
Num 21:16  En van daar reisden zij naar Beer. Dit is de put, van welken de HEERE tot Mozes zeide: Verzamel het volk, zo zal Ik hun water geven.
Num 21:17  (Toen zong Israel dit lied: Spring op, gij put, zingt daarvan bij beurte!
Num 21:18  Gij put, dien de vorsten gegraven hebben, dien de edelen des volks gedolven hebben, door den wetgever, met hun staven.) En van de woestijn reisden zij naar Mattana;
Num 21:19  En van Mattana tot Nahaliel; en van Nahaliel tot Bamoth;
Num 21:20  En van Bamoth tot het dal, dat in het veld van Moab is, aan de hoogte van Pisga, en dat tegen de wildernis ziet.
Num 21:21  Toen zond Israel boden tot Sihon, den koning der Amorieten, zeggende:
Num 21:22  Laat mij door uw land trekken. Wij zullen niet afwijken in de akkers, noch in de wijngaarden; wij zullen het water der putten niet drinken; wij zullen op den koninklijken weg gaan, totdat wij uw landpale doorgetogen zijn.
Num 21:23  Doch Sihon liet Israel niet toe, door zijn landpale door te trekken; maar Sihon vergaderde al zijn volk, en hij ging uit, Israel tegemoet, naar de woestijn, en hij kwam te Jahza, en streed tegen Israel;
Num 21:24  Maar Israel sloeg hem met de scherpte des zwaards, en nam zijn land in erfelijke bezitting, van de Arnon af tot de Jabbok toe, tot aan de kinderen Ammons; want de landpale der kinderen Ammons was vast.
Num 21:25  Alzo nam Israel al deze steden in; en Israel woonde in al de steden der Amorieten, te Hesbon, en in al haar onderhorige plaatsen.
Num 21:26  Want Hesbon was de stad van Sihon, den koning der Amorieten; en hij had gestreden tegen den vorigen koning der Moabieten, en hij had al zijn land uit zijn hand genomen, tot aan de Arnon.
Num 21:27  Daarom zeggen zij, die spreekwoorden gebruiken: Komt tot Hesbon; men bouwe en bevestige de stad van Sihon!
Num 21:28  Want er is een vuur uitgegaan uit Hesbon; een vlam uit de stad van Sihon; zij heeft verteerd Ar der Moabieten, en de heren der hoogten van de Arnon.
Num 21:29  Wee u, Moab! Gij, volk Kamoz zijt verloren! Hij heeft zijn zonen, die ontliepen, en zijn dochters in de gevangenis geleverd aan Sihon, den koning der Amorieten.
Num 21:30  En wij hebben hen nedergeveld! Hesbon is verloren tot Dibon toe; en wij hebben hen verwoest tot Nofat toe, welke tot Medeba toe reikt.
Num 21:31  Alzo woonde Israel in het land van den Amoriet.
Num 21:32  Daarna zond Mozes om Jaezer te verspieden; en zij namen haar onderhorige plaatsen in; en hij dreef de Amorieten, die er waren, uit de bezitting.
Num 21:33  Toen wendden zij zich en trokken op den weg van Basan; en Og, de koning van Basan, ging uit hun tegemoet, hij en al zijn volk, tot den strijd, en Edrei.
Num 21:34  De HEERE nu zeide tot Mozes: Vrees hem niet; want Ik heb hem in uw hand gegeven, en al zijn volk, ook zijn land; en gij zult hem doen, gelijk als gij Sihon, den koning der Amorieten, die te Hesbon woonde, gedaan hebt.
Num 21:35  En zij sloegen hem, en zijn zonen, en al zijn volk, alzo dat hem niemand overbleef; en zij namen zijn land in erfelijke bezitting.
Num 22:1  Daarna reisden de kinderen van Israel, en legerden zich in de vlakke velden van Moab, aan deze zijde van de Jordaan van Jericho.
Num 22:2  Toen Balak, de zoon van Zippor, zag al wat Israel aan de Amorieten gedaan had;
Num 22:3  Zo vreesde Moab zeer voor het aangezicht dezes volks, want het was veel; en Moab was beangstigd voor het aangezicht van de kinderen Israels.
Num 22:4  Derhalve zeide Moab tot de oudsten der Midianieten: Nu zal deze gemeente oplikken al wat rondom ons is, gelijk de os de groente des velds oplikt. Te dier tijd nu was Balak, de zoon van Zippor, koning der Moabieten.
Num 22:5  Die zond boden aan Bileam, den zoon van Beor, te Pethor, hetwelk aan de rivier is, in het land der kinderen zijns volks, om hem te roepen, zeggende: Zie, er is een volk uit Egypte getogen; zie, het heeft het gezicht des lands bedekt, en het blijft liggen recht tegenover mij.
Num 22:6  En nu, kom toch, vervloek mij dit volk, want het is machtiger dan ik; misschien zal ik het kunnen slaan, of het uit het land verdrijven; want ik weet, dat, wien gij zegent, die zal gezegend zijn, en wien gij vervloekt, die zal vervloekt zijn.
Num 22:7  Toen gingen de oudsten der Moabieten, en de oudsten der Midianieten, en hadden het loon der waarzeggingen in hun hand; alzo kwamen zij tot Bileam, en spraken tot hem de woorden van Balak.
Num 22:8  Hij dan zeide tot hen: Vernacht hier dezen nacht, zo zal ik ulieden een antwoord wederbrengen, gelijk als de HEERE tot mij zal gesproken hebben. Toen bleven de vorsten der Moabieten bij Bileam.
Num 22:9  En God kwam tot Bileam en zeide: Wie zijn die mannen, die bij u zijn?
Num 22:10  Toen zeide Bileam tot God: Balak, de zoon van Zippor, de koning der Moabieten, heeft hen tot mij gezonden, zeggende:
Num 22:11  Zie, er is een volk uit Egypte getogen, en het heeft het gezicht des lands bedekt; kom nu, vervloek het mij; misschien zal ik tegen hetzelve kunnen strijden, of het uitdrijven.
Num 22:12  Toen zeide God tot Bileam: Gij zult met hen niet trekken; gij zult dat volk niet vloeken, want het is gezegend.
Num 22:13  Toen stond Bileam des morgens op, en zeide tot de vorsten van Balak: Gaat naar uw land; want de HEERE weigert mij toe te laten met ulieden te gaan.
Num 22:14  Zo stonden dan de vorsten der Moabieten op, en kwamen tot Balak, en zij zeiden: Bileam heeft geweigerd met ons te gaan.
Num 22:15  Doch Balak voer nog voort vorsten te zenden, meer en eerlijker, dan die waren;
Num 22:16  Die tot Bileam kwamen, en hem zeiden: Alzo zegt Balak, de zoon van Zippor: Laat u toch niet beletten tot mij te komen!
Num 22:17  Want ik zal u zeer hoog vereren, en al wat gij tot mij zeggen zult, dat zal ik doen; zo kom toch, vervloek mij dit volk!
Num 22:18  Toen antwoordde Bileam, en zeide tot de dienaren van Balak: Wanneer Balak mij zijn huis vol zilver en goud gave, zo vermocht ik niet het bevel des HEEREN mijns Gods te overtreden, om te doen klein of groot.
Num 22:19  En nu, blijft gijlieden toch ook hier dezen nacht, opdat ik wete, wat de HEERE tot mij verder spreken zal.
Num 22:20  God nu kwam tot Bileam des nachts, en zeide tot hem: Dewijl die mannen gekomen zijn, om u te roepen, sta op, ga met hen; en nochtans zult gij dat doen, hetwelk Ik tot u spreken zal.
Num 22:21  Toen stond Bileam des morgens op, en zadelde zijn ezelin, en hij trok heen met de vorsten van Moab.
Num 22:22  Doch de toorn van God werd ontstoken, omdat hij heentoog; en de Engel des HEEREN stelde Zich in den weg, hem tot een tegenpartij; hij reed nu op zijn ezelin, en twee zijner jongeren waren bij hem.
Num 22:23  De ezelin nu zag den Engel des HEEREN staande in den weg, met Zijn uitgetrokken zwaard in Zijn hand; daarom week de ezelin uit den weg, en ging in het veld. Toen sloeg Bileam de ezelin, om dezelve naar den weg te doen wenden.
Num 22:24  Maar de Engel des HEEREN stond in een pad der wijngaarden, zijnde een muur aan deze, en een muur aan gene zijde.
Num 22:25  Toen de ezelin den Engel des HEEREN zag, zo klemde zij zichzelve aan den wand, en klemde Bileams voet aan den wand; daarom voer hij voort haar te slaan.
Num 22:26  Toen ging de Engel des HEEREN noch verder, en Hij stond in een enge plaats, waar geen weg was om te wijken ter rechter hand noch ter linkerhand.
Num 22:27  Als de ezelin den Engel des HEEREN zag, zo leide zij zich neder onder Bileam; en de toorn van Bileam ontstak, en hij sloeg de ezelin met een stok.
Num 22:28  De HEERE nu opende den mond der ezelin, die tot Bileam zeide: Wat heb ik u gedaan, dat gij mij nu driemaal geslagen hebt?
Num 22:29  Toen zeide Bileam tot de ezelin: Omdat gij mij bespot hebt; och, of ik een zwaard in mijn hand had! want ik zoude u nu doden.
Num 22:30  De ezelin nu zeide tot Bileam: Ben ik niet uw ezelin, op welke gij gereden hebt van toen af, dat gij mijn heer geweest zijt, tot op dezen dag? Ben ik ooit gewend geweest u alzo te doen? Hij dan zeide: Neen!
Num 22:31  Toen ontdekte de HEERE de ogen van Bileam, zodat hij den Engel des HEEREN zag, staande in den weg, en Zijn uitgetrokken zwaard in Zijn hand; daarom neigde hij het hoofd en boog zich op zijn aangezicht.
Num 22:32  Toen zeide de Engel des HEEREN tot hem: Waarom hebt gij uw ezelin nu driemaal geslagen? Zie, Ik ben uitgegaan u tot een tegenpartij, dewijl deze weg van Mij afwijkt.
Num 22:33  Maar de ezelin heeft Mij gezien, en zij is nu driemaal voor Mijn aangezicht geweken; indien zij voor Mijn aangezicht niet geweken ware, zekerlijk Ik zoude u nu ook gedood, en haar bij het leven behouden hebben.
Num 22:34  Toen zeide Bileam tot den Engel des HEEREN: Ik heb gezondigd, want ik heb niet geweten, dat Gij mij tegemoet op dezen weg stond; en nu, is het kwaad in Uw ogen, ik zal wederkeren.
Num 22:35  De Engel des HEEREN nu zeide tot Bileam: Ga heen met deze mannen; maar alleenlijk dat woord, wat Ik tot u spreken zal, dat zult gij spreken. Alzo toog Bileam met de vorsten van Balak.
Num 22:36  Als Balak hoorde, dat Bileam kwam, zo ging hij uit, hem tegemoet, tot de stad der Moabieten, welke aan de landpale van de Arnon ligt, die aan het uiterste der landpale is.
Num 22:37  En Balak zeide tot Bileam: Heb ik niet ernstiglijk tot u gezonden, om u te roepen? Waarom zijt gij niet tot mij gekomen? Kan ik u niet te recht vereren?
Num 22:38  Toen zeide Bileam tot Balak: Zie, ik ben tot u gekomen; zal ik nu enigszins iets kunnen spreken? Het woord, hetwelk God in mijn mond leggen zal, dat zal ik spreken.
Num 22:39  En Bileam ging met Balak; en zij kwamen te Kirjath-huzzoth.
Num 22:40  Toen slachtte Balak runderen en schapen; en hij zond aan Bileam, en aan de vorsten, die bij hem waren.
Num 22:41  En het geschiedde des morgens, dat Balak Bileam nam, en voerde hem op de hoogten van Baal, dat hij van daar zag het uiterste des volks.
Num 23:1  Toen zeide Bileam tot Balak: Bouw mij hier zeven altaren, en bereid mij hier zeven varren en zeven rammen.
Num 23:2  Balak nu deed, gelijk als Bileam gesproken had; en Balak en Bileam offerden een var en een ram, op elk altaar.
Num 23:3  Toen zeide Bileam tot Balak: Blijf staan bij uw brandoffer, en ik zal heengaan; misschien zal de HEERE mij tegemoet komen; en hetgeen Hij wijzen zal, dat zal ik u bekend maken. Toen ging hij op de hoogte.
Num 23:4  Als God Bileam ontmoet was, zo zeide hij tot Hem: Zeven altaren heb ik toegericht, en heb een var en een ram op elk altaar geofferd.
Num 23:5  Toen leide de HEERE het woord in den mond van Bileam, en zeide: Keer weder tot Balak, en spreek aldus.
Num 23:6  Als hij nu tot hem wederkeerde, ziet, zo stond hij bij zijn brandoffer, hij en al de vorsten der Moabieten.
Num 23:7  Toen hief hij zijn spreuk op, en zeide: Uit Syrie heeft mij Balak, de koning der Moabieten, laten halen, van het gebergte tegen het oosten, zeggende: Kom, vervloek mij Jakob, en kom, scheld Israel!
Num 23:8  Wat zal ik vloeken, dien God niet vloekt; en wat zal ik schelden, waar de HEERE niet scheldt?
Num 23:9  Want van de hoogte der steenrotsen zie ik hem, en van de heuvelen aanschouw ik hem; ziet, dat volk zal alleen wonen, en het zal onder de heidenen niet gerekend worden.
Num 23:10  Wie zal het stof van Jakob tellen, en het getal, ja, het vierde deel van Israel? Mijn ziel sterve den dood der oprechten, en mijn uiterste zij gelijk het zijne!
Num 23:11  Toen zeide Balak tot Bileam: Wat hebt gij mij gedaan? Ik heb u genomen, om mijn vijanden te vloeken; maar zie, gij hebt hen doorgaans gezegend!
Num 23:12  Hij nu antwoordde en zeide: Zal ik dat niet waarnemen te spreken, wat de HEERE in mijn mond gelegd heeft?
Num 23:13  Toen zeide Balak tot hem: Kom toch met mij aan een andere plaats, van waar gij hem zult zien; gij zult niet dan zijn einde zien, maar hem niet ganselijk zien; en vervloek hem mij van daar!
Num 23:14  Alzo nam hij hem mede tot het veld Zofim, op de hoogte van Pisga; en hij bouwde zeven altaren, en hij offerde een var en een ram op elk altaar.
Num 23:15  Toen zeide hij tot Balak: Blijf hier staan bij uw brandoffer, en ik zal Hem aldaar ontmoeten.
Num 23:16  Als de HEERE Bileam ontmoet was, zo leide Hij het woord in zijn mond, en Hij zeide: Keer weder tot Balak, en spreek alzo.
Num 23:17  Toen hij tot hem kwam, ziet, zo stond hij bij zijn brandoffer, en de vorsten der Moabieten bij hem. Balak nu zeide tot hem: Wat heeft de HEERE gesproken?
Num 23:18  Toen hief hij zijn spreuk op, en zeide: Sta op, Balak, en hoor! Neig uw oren tot mij, gij, zoon van Zippor!
Num 23:19  God is geen man, dat Hij liegen zou, noch eens mensen kind, dat het Hem berouwen zou; zou Hij het zeggen, en niet doen, of spreken, en niet bestendig maken?
Num 23:20  Zie, ik heb ontvangen te zegenen; dewijl Hij zegent, zo zal ik het niet keren.
Num 23:21  Hij schouwt niet aan de ongerechtigheid in Jakob; ook ziet Hij niet aan de boosheid in Israel. De HEERE, zijn God, is met hem, en het geklank des Konings is bij hem.
Num 23:22  God heeft hen uit Egypte uitgevoerd; zijn krachten zijn als van een eenhoorn.
Num 23:23  Want er is geen toverij tegen Jakob noch waarzeggerij tegen Israel. Te dezer tijd zal van Jakob gezegd worden, en van Israel, wat God gewrocht heeft.
Num 23:24  Zie, het volk zal opstaan als een oude leeuw, en het zal zich verheffen als een leeuw; het zal zich niet neerleggen, totdat het den roof gegeten, en het bloed der verslagenen gedronken zal hebben!
Num 23:25  Toen zeide Balak tot Bileam: Gij zult het ganselijk noch vloeken, noch geenszins zegenen.
Num 23:26  Doch Bileam antwoordde en zeide tot Balak: Heb ik niet tot u gesproken, zeggende: Al wat de HEERE spreken zal, dat zal ik doen?
Num 23:27  Verder zeide Balak tot Bileam: Kom toch, ik zal u aan een ander plaats medenemen; misschien zal het recht zijn in de ogen van dien God, dat gij het mij van daar vervloekt.
Num 23:28  Toen nam Balak Bileam mede tot de hoogte van Peor, die tegen de woestijn ziet.
Num 23:29  En Bileam zeide tot Balak: Bouw mij hier zeven altaren, en bereid mij hier zeven varren en zeven rammen.
Num 23:30  Balak nu deed, gelijk als Bileam gezegd had; en hij offerde een var en een ram op elk altaar.
Num 24:1  Toen Bileam zag, dat het goed was in de ogen des HEEREN, dat hij Israel zegende, zo ging hij ditmaal niet heen, gelijk meermalen, tot de toverijen; maar hij stelde zijn aangezicht naar de woestijn.
Num 24:2  Als Bileam zijn ogen ophief, en Israel zag, wonende naar zijn stammen, zo was de Geest van God op hem.
Num 24:3  En hij hief zijn spreuk op, en zeide: Bileam, de zoon van Beor, spreekt, en de man, wien de ogen geopend zijn, spreekt!
Num 24:4  De hoorder der redenen Gods spreekt, die het gezicht des Almachtigen ziet; die verrukt wordt, en wien de ogen ontdekt worden!
Num 24:5  Hoe goed zijn uw tenten, Jakob! uw woningen, Israel!
Num 24:6  Gelijk de beken breiden zij zich uit, als de hoven aan de rivieren; de HEERE heeft ze geplant, als de sandelbomen, als de cederbomen aan het water.
Num 24:7  Er zal water uit zijn emmeren vloeien, en zijn zaad zal in vele wateren zijn; en zijn koning zal boven Agag verheven worden, en zijn koninkrijk zal verhoogd worden.
Num 24:8  God heeft hem uit Egypte uitgevoerd; zijn krachten zijn als van een eenhoorn; hij zal de heidenen, zijn vijanden, verteren, en hun gebeente breken, en met zijn pijlen doorschieten.
Num 24:9  Hij heeft zich gekromd, hij heeft zich nedergelegd, gelijk een leeuw, en als een oude leeuw; wie zal hem doen opstaan? Zo wie u zegent, die zij gezegend, en vervloekt zij, wie u vervloekt!
Num 24:10  Toen ontstak de toorn van Balak tegen Bileam, en hij sloeg zijn handen samen; en Balak zeide tot Bileam: Ik heb u geroepen, om mijn vijanden te vloeken; maar zie, gij hebt hen nu driemaal gedurig gezegend!
Num 24:11  En nu, pak u weg naar uw plaats! Ik had gezegd, dat ik u hoog vereren zou; maar zie, de HEERE heeft u die eer van u geweerd!
Num 24:12  Toen zeide Bileam tot Balak: Heb ik ook niet tot uw boden, die gij tot mij gezonden hebt, gesproken, zeggende:
Num 24:13  Wanneer mij Balak zijn huis vol zilver en goud gave, zo kan ik het bevel des HEEREN niet overtreden, doende goed of kwaad uit mijn eigen hart; wat de HEERE spreken zal, dat zal ik spreken.
Num 24:14  En nu, zie, ik ga tot mijn volk; kom, ik zal u raad geven, en zeggen wat dit volk uw volk doen zal in de laatste dagen.
Num 24:15  Toen hief hij zijn spreuk op, en zeide: Bileam, de zoon van Beor, spreekt, en die man, wien de ogen geopend zijn, spreekt!
Num 24:16  De hoorder der redenen Gods spreekt, en die de wetenschap des Allerhoogsten weet; die het gezicht des Almachtigen ziet, die verrukt wordt, en wien de ogen ontdekt worden.
Num 24:17  Ik zal hem zien, maar nu niet; ik zal hem aanschouwen, maar niet nabij. Er zal een ster voortkomen uit Jakob, en er zal een scepter uit Israel opkomen; die zal de palen der Moabieten verslaan, en zal al de kinderen van Seth verstoren.
Num 24:18  En Edom zal een erfelijke bezitting zijn; en Seir zal zijn vijanden een erfelijke bezitting zijn; doch Israel zal kracht doen.
Num 24:19  En er zal een uit Jakob heersen, en hij zal de overigen uit de steden ombrengen.
Num 24:20  Toen hij de Amalekieten zag, zo hief hij zijn spreuk op, en zeide: Amalek is de eersteling der heidenen; maar zijn uiterste is ten verderve!
Num 24:21  Toen hij de Kenieten zag, zo hief hij zijn spreuk op, en zeide: Uw woning is vast, en gij hebt uw nest in een steenrots gelegd.
Num 24:22  Evenwel zal Kain verteerd worden, totdat u Assur gevankelijk wegvoeren zal!
Num 24:23  Voorts hief hij zijn spreuk op, en zeide: Och, wie zal leven, als God dit doen zal!
Num 24:24  En de schepen van den oever der Chitteers, die zullen Assur plagen, zij zullen ook Heber plagen; en hij zal ook ten verderve zijn.
Num 24:25  Toen stond Bileam op, en ging heen, en keerde weder tot zijn plaats. Balak ging ook zijn weg.
Num 25:1  En Israel verbleef te Sittim, en het volk begon te hoereren met de dochteren der Moabieten.
Num 25:2  En zij nodigden het volk tot de slachtofferen harer goden; en het volk at, en boog zich voor haar goden.
Num 25:3  Als nu Israel zich koppelde aan Baal-peor, ontstak de toorn des HEEREN tegen Israel.
Num 25:4  En de HEERE zeide tot Mozes: Neem al de hoofden des volks, en hang ze den HEERE tegen de zon, zo zal de hittigheid van des HEEREN toorn gekeerd worden van Israel.
Num 25:5  Toen zeide Mozes tot de rechters van Israel: Een iedere dode zijn mannen, die zich aan Baal-peor gekoppeld hebben!
Num 25:6  En ziet, een man uit de kinderen Israels kwam, en bracht een Midianietin tot zijn broederen voor de ogen van Mozes, en voor de ogen van de ganse vergadering der kinderen Israels, toen zij weenden voor de deur van de tent der samenkomst.
Num 25:7  Toen Pinehas, de zoon van Eleazar, den zoon van Aaron, den priester, dat zag, zo stond hij op uit het midden der vergadering, en nam een spies in zijn hand;
Num 25:8  En hij ging den Israelietischen man na in de hoerenwinkel, en doorstak hen beiden, den Israelietischen man en de vrouw, door hun buik. Toen werd de plaag van over de kinderen Israels opgehouden.
Num 25:9  Degenen nu, die aan de plaag stierven, waren vier en twintig duizend.
Num 25:10  Toen sprak de HEERE tot Mozes, zeggende:
Num 25:11  Pinehas, de zoon van Eleazar, den zoon van Aaron, den priester, heeft Mijn grimmigheid van over de kinderen Israels afgewend, dewijl hij Mijn ijver geijverd heeft in het midden derzelve, zodat Ik de kinderen Israels in Mijn ijver niet vernield heb.
Num 25:12  Daarom spreek: Zie, Ik geef hem Mijn verbond des vredes.
Num 25:13  En hij zal hebben, en zijn zaad na hem, het verbond des eeuwigen priesterdoms, daarom dat hij voor zijn God geijverd, en verzoening gedaan heeft voor de kinderen Israels.
Num 25:14  De naam nu des verslagenen Israelietischen mans, die verslagen was met de Midianietin, was Zimri, de zoon van Salu, een overste van een vaderlijk huis der Simeonieten.
Num 25:15  En de naam der verslagene Midianietische vrouw was Kozbi, een dochter van Zur, die een hoofd was der volken van een vaderlijk huis onder de Midianieten.
Num 25:16  Verder sprak de HEERE tot Mozes, zeggende:
Num 25:17  Handel vijandelijk met de Midianieten, en versla hen;
Num 25:18  Want zij hebben vijandelijk tegen ulieden gehandeld door hun listen, die zij listig tegen u bedacht hebben in de zaak van Peor, en in de zaak van Kozbi, de dochter van den overste der Midianieten, hun zuster, die verslagen is, ten dage der plaag, om de zaak van Peor.
Num 26:1  Het geschiedde nu na die plaag, dat de HEERE sprak tot Mozes, en tot Eleazar, den zoon van Aaron, den priester, zeggende:
Num 26:2  Neem de som van de gehele vergadering der kinderen Israels op, van twintig jaren oud en daarboven, naar het huis hunner vaderen, al wie ten heire in Israel uittrekt.
Num 26:3  Mozes dan en Eleazar, de priester, spraken hen aan, in de vlakke velden van Moab, aan de Jordaan van Jericho, zeggende:
Num 26:4  Dat men opneme van twintig jaren oud en daarboven; gelijk als de HEERE Mozes geboden had, en den kinderen Israels, die uit Egypteland uitgetogen waren.
Num 26:5  Ruben was de eerstgeborene van Israel. De zonen van Ruben waren: Hanoch, van welken was het geslacht der Hanochieten; van Pallu het geslacht der Palluieten;
Num 26:6  Van Hezron het geslacht der Hezronieten; van Karmi het geslacht der Karmieten.
Num 26:7  Dit zijn de geslachten der Rubenieten; en hun getelden waren drie en veertig duizend zevenhonderd en dertig.
Num 26:8  En de zonen van Pallu waren Eliab.
Num 26:9  En de zonen van Eliab waren Nemuel, en Dathan, en Abiram; deze Dathan en Abiram waren de geroepenen der vergadering, die gekijf maakten tegen Mozes en tegen Aaron, in de vergadering van Korach, als zij gekijf tegen den HEERE maakten.
Num 26:10  En de aarde haar mond opendeed, en verslond hen met Korach, als die vergadering stierf, toen het vuur tweehonderd en vijftig mannen verteerde, en werden tot een teken.
Num 26:11  Maar de kinderen van Korach stierven niet.
Num 26:12  De zonen van Simeon, naar hun geslachten: van Nemuel, het geslacht der Nemuelieten; van Jamin het geslacht der Jaminieten; van Jachin het geslacht der Jachinieten;
Num 26:13  Van Zerah het geslacht der Zerahieten; van Saul het geslacht der Saulieten.
Num 26:14  Dat zijn de geslachten der Simeonieten: twee en twintig duizend en tweehonderd.
Num 26:15  De zonen van Gad, naar hun geslachten: van Zefon het geslacht der Zefonieten; van Haggi het geslacht der Haggieten; van Suni het geslacht der Sunieten.
Num 26:16  Van Ozni het geslacht der Oznieten; van Heri het geslacht der Herieten;
Num 26:17  Van Arod het geslacht der Arodieten; van Areli het geslacht der Arelieten.
Num 26:18  Dat zijn de geslachten der zonen van Gad, naar hun getelden: veertig duizend en vijfhonderd.
Num 26:19  De zonen van Juda waren Er en Onan; maar Er en Onan stierven in het land Kanaan.
Num 26:20  Alzo waren de zonen van Juda naar hun geslachten: van Sela het geslacht der Selanieten; van Perez het geslacht der Perezieten; van Zerah het geslacht der Zerahieten.
Num 26:21  En de zonen van Perez waren: van Hezron het geslacht der Hezronieten; van Hamul het geslacht der Hamulieten.
Num 26:22  Dat zijn de geslachten van Juda, naar hun getelden: zes en zeventig duizend en vijfhonderd.
Num 26:23  De zonen van Issaschar, naar hun geslachten, waren: van Tola het geslacht der Tolaieten; van Puva het geslacht der Punieten;
Num 26:24  Van Jasub het geslacht der Jasubieten; van Simron het geslacht der Simronieten.
Num 26:25  Dat zijn de geslachten van Issaschar, naar hun getelden: vier en zestig duizend en driehonderd.
Num 26:26  De zonen van Zebulon, naar hun geslachten, waren: van Sered het geslacht der Seredieten; van Elon het geslacht der Elonieten; van Jahleel het geslacht der Jahleelieten.
Num 26:27  Dat zijn de geslachten der Zebulonieten, naar hun getelden: zestig duizend en vijfhonderd.
Num 26:28  De zonen van Jozef, naar hun geslachten, waren Manasse en Efraim.
Num 26:29  De zonen van Manasse waren: van Machir het geslacht der Machirieten; Machir nu gewon Gilead; van Gilead was het geslacht der Gileadieten.
Num 26:30  Dit zijn de zonen van Gilead: van Jezer het geslacht der Jezerieten; van Helek het geslacht der Helekieten.
Num 26:31  En van Asriel het geslacht der Asrielieten; en van Sechem het geslacht der Sechemieten;
Num 26:32  En van Semida het geslacht der Semidaieten; en van Hefer het geslacht der Heferieten.
Num 26:33  Doch Zelafead, de zoon van Hefer, had geen zonen, maar dochters; en de namen der dochteren van Zelafead waren: Machla en Noa, Hogla, Milka en Tirza.
Num 26:34  Dat zijn de geslachten van Manasse: en hun getelden waren twee en vijftig duizend en zevenhonderd.
Num 26:35  Dit zijn de zonen van Efraim, naar hun geslachten: van Sutelah het geslacht der Sutelahieten; van Becher het geslacht der Becherieten; van Tahan het geslacht der Tahanieten.
Num 26:36  En dit zijn de zonen van Sutelah; van Eran het geslacht der Eranieten.
Num 26:37  Dat zijn de geslachten der zonen van Efraim, naar hun getelden: twee en dertig duizend en vijfhonderd. Dat zijn de zonen van Jozef, naar hun geslachten.
Num 26:38  De zonen van Benjamin, naar hun geslachten: van Bela het geslacht der Belaieten; van Asbel het geslacht der Asbelieten; van Ahiram het geslacht der Ahiramieten;
Num 26:39  Van Sefufam het geslacht der Sufamieten; van Hufam het geslacht der Hufamieten.
Num 26:40  En de zonen van Bela waren Ard en Naaman; van Ard het geslacht der Ardieten; van Naaman het geslacht der Naamieten.
Num 26:41  Dat zijn de zonen van Benjamin, naar hun geslachten; en hun getelden waren vijf en veertig duizend en zeshonderd.
Num 26:42  Dit zijn de zonen van Dan, naar hun geslachten: van Suham het geslacht der Suhamieten; dat zijn de geslachten van Dan, naar hun geslachten.
Num 26:43  Al de geslachten der Suhamieten, naar hun getelden, waren vier en zestig duizend en vierhonderd.
Num 26:44  De zonen van Aser, naar hun geslachten, waren: van Imna het geslacht der Imnaieten; van Isvi het geslacht der Isvieten; van Beria het geslacht der Beriieten.
Num 26:45  Van de zonen van Beria waren: van Heber het geslacht der Heberieten; van Malchiel het geslacht der Malchielieten.
Num 26:46  En de naam der dochter van Aser was Serah.
Num 26:47  Dat zijn de geslachten der zonen van Aser, naar hun getelden: drie en vijftig duizend en vierhonderd.
Num 26:48  De zonen van Nafthali, naar hun geslachten: van Jahzeel het geslacht der Jahzeelieten; van Guni het geslacht der Gunieten;
Num 26:49  Van Jezer het geslacht der Jezerieten; van Sillem het geslacht der Sillemieten.
Num 26:50  Dat zijn de geslachten van Nafthali, naar hun geslachten; en hun getelden waren vijf en veertig duizend en vierhonderd.
Num 26:51  Dat zijn de getelden van de zonen Israels: zeshonderd een duizend zevenhonderd en dertig.
Num 26:52  En de HEERE sprak tot Mozes, zeggende:
Num 26:53  Aan dezen zal het land uitgedeeld worden ter erfenis, naar het getal der namen.
Num 26:54  Aan degenen, die veel zijn, zult gij hun erfenis meerder maken, en aan hen, die weinig zijn, zult gij hun erfenis minder maken; aan een iegelijk zal, naar zijn getelden, zijn erfenis gegeven worden.
Num 26:55  Het land nochtans zal door het lot gedeeld worden; naar de namen der stammen hunner vaderen zullen zij erven.
Num 26:56  Naar het lot zal elks erfenis gedeeld worden tussen de velen en de weinigen.
Num 26:57  Dit zijn nu de getelden van Levi, naar hun geslachten: van Gerson het geslacht der Gersonieten; van Kohath het geslacht der Kohathieten; van Merari het geslacht der Merarieten.
Num 26:58  Dit zijn de geslachten van Levi: het geslacht der Libnieten, het geslacht der Hebronieten, het geslacht der Machlieten, het geslacht der Muzieten, het geslacht der Korachieten. En Kohath gewon Amram.
Num 26:59  En de naam der huisvrouw van Amram was Jochebed, de dochter van Levi, welke de huisvrouw van Levi baarde in Egypte; en deze baarde aan Amram, Aaron, en Mozes, en Mirjam, hun zuster.
Num 26:60  En aan Aaron werden geboren Nadab en Abihu, Eleazar en Ithamar.
Num 26:61  Nadab nu en Abihu waren gestorven, toen zij vreemd vuur brachten voor het aangezicht des HEEREN.
Num 26:62  En hun getelden waren drie en twintig duizend, al wat mannelijk is, van een maand oud en daarboven; want dezen werden niet geteld onder de kinderen Israels, omdat hun geen erfenis gegeven werd onder de kinderen Israels.
Num 26:63  Dat zijn de getelden van Mozes en Eleazar, den priester, die de kinderen Israels telden in de vlakke velden van Moab, aan de Jordaan van Jericho.
Num 26:64  En onder dezen was niemand uit de getelden van Mozes en Aaron, den priester, als zij de kinderen Israels telden in de woestijn van Sinai.
Num 26:65  Want de HEERE had van die gezegd, dat zij in de woestijn gewisselijk zouden sterven; en er was niemand van hen overgebleven, dan Kaleb, de zoon van Jefunne, en Jozua, de zoon van Nun.
Num 27:1  Toen naderden de dochteren van Zelafead, den zoon van Hefer, den zoon van Gilead, den zoon van Machir, den zoon van Manasse, onder de geslachten van Manasse, den zoon van Jozef (en dit zijn de namen zijner dochteren: Machla, Noa, en Hogla, en Milka, en Tirza);
Num 27:2  En zij stonden voor het aangezicht van Mozes, en voor het aangezicht van Eleazar, den priester, en voor het aangezicht van de oversten, en van de ganse vergadering, aan de deur van de tent der samenkomst, zeggende:
Num 27:3  Onze vader is gestorven in de woestijn, en hij is niet geweest in het midden der vergadering dergenen, die zich tegen den HEERE vergaderd hebben in de vergadering van Korach; maar hij is in zijn zonde gestorven, en had geen zonen.
Num 27:4  Waarom zou de naam onzes vaders uit het midden van zijn geslacht weggenomen worden, omdat hij geen zoon heeft? Geef ons een bezitting in het midden der broederen van onzen vader.
Num 27:5  En Mozes bracht haar rechtzaak voor het aangezicht des HEEREN.
Num 27:6  En de HEERE sprak tot Mozes, zeggende:
Num 27:7  De dochteren van Zelafead spreken recht; gij zult haar ganselijk geven de bezitting ener erfenis, in het midden van de broederen haars vaders; en gij zult de erfenis haars vaders op haar doen komen.
Num 27:8  En tot de kinderen Israels zult gij spreken, zeggende: Wanneer iemand sterft, en geen zoon heeft, zo zult gij zijn erfenis op zijn dochter doen komen.
Num 27:9  En indien hij geen dochter heeft, zo zult gij zijn erfenis aan zijn broederen geven.
Num 27:10  Indien hij nu geen broederen heeft, zo zult gij zijn erfenis aan de broederen zijns vaders geven.
Num 27:11  Indien ook zijn vader geen broeders heeft, zo zult gij zijn erfenis geven aan zijn naastbestaande, die hem de naaste van zijn geslacht is, dat hij het erfelijk bezitte. Dit zal den kinderen Israels tot een inzetting des rechts zijn, gelijk als de HEERE Mozes geboden heeft.
Num 27:12  Daarna zeide de HEERE tot Mozes: Klim op dezen berg Abarim, en zie dat land, hetwelk Ik den kinderen Israels gegeven heb.
Num 27:13  Wanneer gij dat gezien zult hebben, dan zult gij tot uw volken verzameld worden, gij ook, gelijk als uw broeder Aaron verzameld geworden is;
Num 27:14  Naardien gijlieden Mijn mond wederspannig zijt geweest in de woestijn Zin, in de twisting der vergadering, om Mij aan de wateren voor hun ogen te heiligen. Dat zijn de wateren van Meriba, van Kades, in de woestijn Zin.
Num 27:15  Toen sprak Mozes tot den HEERE, zeggende:
Num 27:16  Dat de HEERE, de God der geesten van alle vlees, een man stelle over deze vergadering.
Num 27:17  Die voor hun aangezicht uitga, en die voor hun aangezicht inga, en die hen uitleide, en die hen inleide; opdat de vergadering des HEEREN niet zij als schapen, die geen herder hebben.
Num 27:18  Toen zeide de HEERE tot Mozes: Neem tot u Jozua, den zoon van Nun, een man, in wien de Geest is; en leg uw hand op hem;
Num 27:19  En stel hem voor het aangezicht van Eleazar, den priester, en voor het aangezicht der ganse vergadering; en geef hem bevel voor hun ogen;
Num 27:20  En leg op hem van uw heerlijkheid, opdat zij horen, te weten de ganse vergadering der kinderen Israels.
Num 27:21  En hij zal voor het aangezicht van Eleazar, den priester, staan, die voor hem raad vragen zal, naar de wijze van Urim, voor het aangezicht des HEEREN; naar zijn mond zullen zij uitgaan, en naar zijn mond zullen zij ingaan, hij, en al de kinderen Israels met hem, en de ganse vergadering.
Num 27:22  En Mozes deed, gelijk als de HEERE hem geboden had; want hij nam Jozua, en stelde hem voor het aangezicht van Eleazar, den priester, en voor het aangezicht der ganse vergadering.
Num 27:23  En hij leide zijn handen op hem, en gaf hem bevel; gelijk als de HEERE door den dienst van Mozes gesproken had.
Num 28:1  Verder sprak de HEERE tot Mozes, zeggende:
Num 28:2  Gebied den kinderen Israels, en zeg tot hen: Mijn offerande, Mijn spijze voor Mijn vuurofferen, Mijn liefelijken reuk, zult gij waarnemen, om Mij te offeren op zijn gezetten tijd.
Num 28:3  En gij zult tot hen zeggen: Dit is het vuuroffer, hetwelk gij den HEERE offeren zult: twee volkomen eenjarige lammeren des daags, tot een gedurig brandoffer.
Num 28:4  Het ene lam zult gij bereiden des morgens; en het andere lam zult gij bereiden tussen de twee avonden.
Num 28:5  En een tiende deel ener efa meelbloem ten spijsoffer, gemengd met het vierendeel van een hin van gestoten olie.
Num 28:6  Het is het gedurig brandoffer, hetwelk op den berg Sinai ingesteld was tot een liefelijken reuk, een vuuroffer den HEERE.
Num 28:7  En zijn drankoffer zal zijn het vierendeel van een hin, voor het ene lam; in het heiligdom zult gij het drankoffer des sterken dranks den HEERE offeren.
Num 28:8  En het andere lam zult gij bereiden tussen de twee avonden; gelijk het spijsoffer des morgens, en gelijk zijn drankoffer zult gij het bereiden, ten vuuroffer des liefelijken reuks den HEERE.
Num 28:9  Maar op den sabbatdag twee volkomen eenjarige lammeren, en twee tienden meelbloem, ten spijsoffer, met olie gemengd, mitsgaders zijn drankoffer.
Num 28:10  Het is het brandoffer des sabbats op elken sabbat, boven het gedurig brandoffer, en zijn drankoffer.
Num 28:11  En in de beginselen uwer maanden zult gij een brandoffer den HEERE offeren: twee jonge varren, en een ram, zeven volkomen eenjarige lammeren;
Num 28:12  En drie tienden meelbloem ten spijsoffer, met olie gemengd tot den enen var; en twee tienden meelbloem ten spijsoffer, met olie gemengd, tot den enen ram;
Num 28:13  En tot elk tiende deel meelbloem ten spijsoffer, met olie gemengd, tot het ene lam; het is een brandoffer tot een liefelijken reuk, een vuuroffer, den HEERE.
Num 28:14  En hun drankofferen zullen zijn de helft van een hin tot een var, en een derde deel van een hin tot een ram, en een vierendeel van een hin van wijn tot een lam; dat is het brandoffer der nieuwe maan in elke maand, naar de maanden des jaars.
Num 28:15  Daartoe zal een geitenbok ten zondoffer den HEERE, boven het gedurige brandoffer, bereid worden, met zijn drankoffer.
Num 28:16  En in de eerste maand, op den veertienden dag der maand, is het pascha den HEERE.
Num 28:17  En op den vijftienden dag derzelve maand is het feest; zeven dagen zullen ongezuurde broden gegeten worden.
Num 28:18  Op den eersten dag zal een heilige samenroeping zijn; geen dienstwerk zult gijlieden doen;
Num 28:19  Maar gij zult een vuuroffer ten brandoffer den HEERE offeren: twee jonge varren, en een ram, daartoe zeven eenjarige lammeren; volkomen zullen zij u zijn.
Num 28:20  En hun spijsoffer zal zijn meelbloem, met olie gemengd; drie tienden tot een var, en twee tienden tot een ram zult gij bereiden.
Num 28:21  Tot elk zult gij een tiende deel bereiden tot een lam, tot die zeven lammeren toe.
Num 28:22  Daarna een bok ten zondoffer, om over ulieden verzoening te doen.
Num 28:23  Behalve het morgenbrandoffer, hetwelk tot een gedurig brandoffer is, zult gij deze dingen bereiden.
Num 28:24  Achtervolgens deze dingen zult gij des daags, zeven dagen lang, de spijze des vuuroffers bereiden tot een liefelijken reuk den HEERE; boven dat gedurig brandoffer zal het bereid worden, met zijn drankoffer.
Num 28:25  En op den zevenden dag zult gij een heilige samenroeping hebben; geen dienstwerk zult gij doen.
Num 28:26  Insgelijks op den dag der eerstelingen, als gij een nieuw spijsoffer den HEERE zult offeren naar uw werken, zult gij een heilige samenroeping hebben; geen dienstwerk zult gij doen.
Num 28:27  Dan zult gij den HEERE een brandoffer ten liefelijken reuk offeren: twee jonge varren, een ram, zeven eenjarige lammeren;
Num 28:28  En hun spijsoffer van meelbloem, met olie gemengd: drie tienden tot een var, twee tienden tot een ram;
Num 28:29  Tot elk een tiende tot een lam, tot die zeven lammeren toe;
Num 28:30  Een geitenbok, om voor u verzoening te doen.
Num 28:31  Behalve het gedurig brandoffer, en zijn spijsoffer, zult gij ze bereiden; zij zullen u volkomen zijn met hun drankofferen.
Num 29:1  Desgelijks in de zevende maand, op den eersten der maand, zult gij een heilige samenroeping hebben; geen dienstwerk zult gij doen; het zal u een dag des geklanks zijn.
Num 29:2  Dan zult gij een brandoffer, ten liefelijken reuk, den HEERE bereiden: een jongen var, een ram, zeven volkomen eenjarige lammeren;
Num 29:3  En hun spijsoffer van meelbloem, met olie gemengd; drie tienden tot den var, twee tienden tot den ram.
Num 29:4  En een tiende tot een lam, tot die zeven lammeren toe;
Num 29:5  En een geitenbok ten zondoffer, om over ulieden verzoening te doen;
Num 29:6  Behalve het brandoffer der maand, en zijn spijsoffer, en het gedurig brandoffer, en zijn spijsoffer, met hun drankofferen, naar hun wijze, ten liefelijken reuk, ten vuuroffer den HEERE.
Num 29:7  En op den tienden dezer zevende maand zult gij een heilige samenroeping hebben, en gij zult uw zielen verootmoedigen; geen werk zult gij doen;
Num 29:8  Maar gij zult brandoffer, ten liefelijken reuk, den HEERE offeren: een jongen var, een ram, zeven eenjarige lammeren; volkomen zullen zij u zijn.
Num 29:9  En hun spijsoffer van meelbloem, met olie gemengd: drie tienden tot den var, twee tienden tot den enen ram;
Num 29:10  Tot elk een tiende tot een lam, tot die zeven lammeren toe;
Num 29:11  Een geitenbok ten zondoffer, behalve het zondoffer der verzoeningen, en het gedurig brandoffer; en zijn spijsoffer, met hun drankofferen.
Num 29:12  Insgelijks op den vijftienden dag dezer zevende maand, zult gij een heilige samenroeping hebben; geen dienstwerk zult gij doen; maar zeven dagen zult gij den HEERE een feest vieren.
Num 29:13  En gij zult een brandoffer ten vuuroffer offeren, ten liefelijken reuk den HEERE: dertien jonge varren, twee rammen, veertien eenjarige lammeren; zij zullen volkomen zijn;
Num 29:14  En hun spijsoffer van meelbloem, met olie gemengd: drie tienden tot een var, tot die dertien varren toe; twee tienden tot een ram, onder die twee rammen;
Num 29:15  En tot elke een tiende tot een lam, tot die veertien lammeren toe;
Num 29:16  En een geitenbok ten zondoffer; behalve het gedurig brandoffer, zijn spijsoffer, en zijn drankoffer.
Num 29:17  Daarna op den tweeden dag: twaalf jonge varren, twee rammen, veertien volkomen eenjarige lammeren;
Num 29:18  En hun spijsoffer, en hun drankofferen tot de varren, tot de rammen, en tot de lammeren, in hun getal, naar de wijze;
Num 29:19  En een geitenbok ten zondoffer; behalve het gedurig brandoffer, en zijn spijsoffer, met hun drankofferen.
Num 29:20  En op den derden dag: elf varren, twee rammen, veertien volkomen eenjarige lammeren;
Num 29:21  En hun spijsofferen, en hun drankofferen tot de varren, tot de rammen, en tot de lammeren, in hun getal, naar de wijze;
Num 29:22  En een bok ten zondoffer; behalve het gedurig brandoffer, en zijn spijsoffer, en zijn drankoffer.
Num 29:23  Verder op den vierden dag: tien varren, twee rammen, veertien volkomen eenjarige lammeren;
Num 29:24  Hun spijsoffer, en hun drankofferen tot de varren, tot de rammen, en tot de lammeren, in hun getal, naar de wijze;
Num 29:25  En een geitenbok ten zondoffer; behalve het gedurig brandoffer, zijn spijsoffer, en zijn drankoffer.
Num 29:26  En op den vijfden dag: negen varren, twee rammen, en veertien volkomen eenjarige lammeren;
Num 29:27  En hun spijsoffer, en hun drankofferen tot de varren, tot de rammen, en tot de lammeren, in hun getal, naar de wijze;
Num 29:28  En een bok ten zondoffer; behalve het gedurig brandoffer, en zijn spijsoffer, en zijn drankoffer.
Num 29:29  Daarna op den zesden dag: acht varren, twee rammen, veertien volkomen eenjarige lammeren;
Num 29:30  En hun spijsoffer, en hun drankofferen tot de varren, tot de rammen, en tot de lammeren, in hun getal, naar de wijze;
Num 29:31  En een bok ten zondoffer; behalve het gedurig brandoffer, zijn spijsoffer, en zijn drankofferen.
Num 29:32  En op den zevenden dag: zeven varren, twee rammen, veertien volkomen eenjarige lammeren;
Num 29:33  En hun spijsoffer, en hun drankofferen tot de varren, tot de rammen, en tot de lammeren, in hun getal, naar hun wijze;
Num 29:34  En een bok ten zondoffer; behalve het gedurig brandoffer, zijn spijsoffer, en zijn drankoffer.
Num 29:35  Op den achtsten dag zult gij een verbodsdag hebben; geen dienstwerk zult gij doen.
Num 29:36  En gij zult een brandoffer ten vuuroffer offeren, ten liefelijken reuk den HEERE; een var, een ram, zeven volkomen eenjarige lammeren;
Num 29:37  Hun spijsoffer, en hun drankofferen tot den var, tot den ram, en tot de lammeren, in hun getal, naar de wijze;
Num 29:38  En een bok ten zondoffer; behalve het gedurig brandoffer, en zijn spijsoffer, en zijn drankoffer.
Num 29:39  Deze dingen zult gij den HEERE doen op uw gezette hoogtijden; behalve uw geloften, en uw vrijwillige offeren, met uw brandofferen, en met uw spijsofferen en met uw drankofferen, en met uw dankofferen.
Num 29:40  En Mozes sprak tot de kinderen Israels naar al wat de HEERE Mozes geboden had.
Num 30:1  En Mozes sprak tot de hoofden der stammen van de kinderen Israels, zeggende: Dit is de zaak, die de HEERE geboden heeft:
Num 30:2  Wanneer een man den HEERE een gelofte zal beloofd, of een eed zal gezworen hebben, zijn ziel met een verbintenis verbindende, zijn woord zal hij niet ontheiligen; naar alles, wat uit zijn mond gegaan is, zal hij doen.
Num 30:3  Maar als een vrouw den HEERE een gelofte zal beloofd hebben, en zich met een verbintenis in het huis haars vaders in haar jonkheid zal verbonden hebben;
Num 30:4  En haar vader haar gelofte, en haar verbintenis, waarmede zij haar ziel verbonden heeft, zal horen, en haar vader tegen haar zal stilzwijgen, zo zullen al haar geloften bestaan, en alle verbintenis, waarmede zij haar ziel verbonden heeft, zal bestaan.
Num 30:5  Maar indien haar vader dat zal breken, ten dage als hij het hoort, al haar geloften, en haar verbintenissen, waarmede zij haar ziel verbonden heeft, zullen niet bestaan; maar de HEERE zal het haar vergeven; want haar vader heeft ze haar doen breken.
Num 30:6  Doch indien zij immers een man heeft, en haar geloften op haar zijn, of de uitspraak harer lippen, waarmede zij haar ziel verbonden heeft;
Num 30:7  En haar man dat zal horen, en ten dage als hij het hoort, tegen haar zal stilzwijgen, zo zullen haar geloften bestaan, en haar verbintenissen, waarmede zij haar ziel verbonden heeft, zullen bestaan.
Num 30:8  Maar indien haar man ten dage, als hij het hoorde, dat zal breken, en haar gelofte, die op haar was, zal te niet maken, mitsgaders de uitspraak harer lippen, waarmede zij haar ziel verbonden heeft, zo zal het de HEERE haar vergeven.
Num 30:9  Aangaande de gelofte ener weduwe, of ener verstotene: alles, waarmede zij haar ziel verbonden heeft, zal over haar bestaan.
Num 30:10  Maar indien zij ten huize haars mans gelofte gedaan heeft, of met een eed door verbintenis haar ziel verbonden heeft;
Num 30:11  En haar man dat gehoord, en tegen haar stil zal gezwegen hebben, dat niet brekende; zo zullen al haar geloften bestaan, mitsgaders alle verbintenis, waarmede zij haar ziel verbonden heeft, zal bestaan.
Num 30:12  Maar indien haar man die dingen ganselijk te niet maakt, ten dage als hij het hoort, niets van al wat uit haar lippen gegaan is, van haar gelofte, en van de verbintenis harer ziel, zal bestaan; haar man heeft ze te niet gemaakt, en de HEERE zal het haar vergeven.
Num 30:13  Alle gelofte, en allen eed der verbintenis, om de ziel te verootmoedigen, die zal haar man bevestigen, of die zal haar man te niet maken.
Num 30:14  Maar zo haar man tegen haar van dag tot dag ganselijk stilzwijgt, zo bevestigt hij al haar geloften, of al haar verbintenissen, dewelke op haar zijn; hij heeft ze bevestigd, omdat hij tegen haar stilgezwegen heeft, ten dage als hij het hoorde.
Num 30:15  Doch zo hij ze ganselijk te niet maken zal, nadat hij het gehoord zal hebben, zo zal hij haar ongerechtigheid dragen.
Num 30:16  Dat zijn de inzettingen, die de HEERE Mozes geboden heeft, tussen een man en zijn huisvrouw, tussen een vader en zijn dochter, zijnde in haar jonkheid, ten huize haars vaders.
Num 31:1  En de HEERE sprak tot Mozes, zeggende:
Num 31:2  Neem de wraak der kinderen Israels van de Midianieten; daarna zult gij verzameld worden tot uw volken.
Num 31:3  Mozes dan sprak tot het volk, zeggende: Dat zich mannen uit u ten strijde toerusten, en dat zij tegen de Midianieten zijn, om de wraak des HEEREN te doen aan de Midianieten.
Num 31:4  Van elken stam onder alle stammen Israels zult gij een duizend ten strijde zenden.
Num 31:5  Alzo werden geleverd uit de duizenden van Israel, duizend van elken stam, twaalf duizend toegerusten ten strijde.
Num 31:6  En Mozes zond hen ten strijde, duizend van elken stam, hen en Pinehas, den zoon van Eleazar, den priester, ten strijde, met de heilige vaten, en de trompetten des geklanks in zijn hand.
Num 31:7  En zij streden tegen de Midianieten, gelijk als de HEERE Mozes geboden had, en zij doodden al wat mannelijk was.
Num 31:8  Daartoe doodden zij boven hun verslagenen, de koningen der Midianieten, Evi, en Rekem, en Zur, en Hur, en Reba, vijf koningen der Midianieten; ook doodden zij met het zwaard Bileam, den zoon van Beor.
Num 31:9  Maar de kinderen Israels namen de vrouwen der Midianieten, en hun kinderkens gevangen; zij roofden ook al hun beesten, en al hun vee, en al hun vermogen.
Num 31:10  Voorts al hun steden met hun woonplaatsen, en al hun burchten verbrandden zij met vuur.
Num 31:11  En zij namen al den roof, en al den buit, van mensen en van beesten.
Num 31:12  Daarna brachten zij de gevangenen, en den buit, en den roof, tot Mozes en tot Eleazar, den priester, en tot de vergadering der kinderen Israels, in het leger, in de vlakke velden van Moab, dewelke zijn aan de Jordaan van Jericho.
Num 31:13  Maar Mozes en Eleazar, de priester, en alle oversten der vergadering, gingen uit hen tegemoet, tot buiten voor het leger.
Num 31:14  En Mozes werd grotelijks vertoornd tegen de bevelhebbers des heirs, de hoofdlieden der duizenden, en de hoofdlieden der honderden, die uit den strijd van dien oorlog kwamen.
Num 31:15  En Mozes zeide tot hen: Hebt gij dan alle vrouwen laten leven?
Num 31:16  Ziet, deze waren, door den raad van Bileam, den kinderen Israels, om oorzake der overtreding tegen den HEERE te geven, in de zaak van Peor; waardoor die plaag werd onder de vergadering des HEEREN.
Num 31:17  Nu dan, doodt al wat mannelijk is onder de kinderkens; en doodt alle vrouw, die door bijligging des mans een man bekend heeft.
Num 31:18  Doch al de kinderen van vrouwelijk geslacht, die de bijligging des mans niet bekend hebben, laat voor ulieden leven.
Num 31:19  En gijlieden, legert u buiten het leger zeven dagen; een ieder, die een mens gedood, en een ieder, die een verslagene zult aangeroerd hebben, zult u op den derden dag en op den zevenden dag ontzondigen, gij en uw gevangenen.
Num 31:20  Ook zult gij alle kleding, en alle gereedschap van vellen, en alle geiten haren werk, en gereedschap van hout, ontzondigen.
Num 31:21  En Eleazar, de priester, zeide tot de krijgslieden, die tot dien strijd getogen waren: Dit is de inzetting der wet, die de HEERE Mozes geboden heeft.
Num 31:22  Alleen het goud en het zilver, en het koper, het ijzer, het tin en het lood;
Num 31:23  Alle ding, dat het vuur lijdt, zult gij door het vuur laten doorgaan, dat het rein worde; evenwel zal het door het water der afzondering ontzondigd worden; maar al wat het vuur niet lijdt, zult gij door het water laten doorgaan.
Num 31:24  Gij zult ook uw klederen op den zevenden dag wassen, dat gij rein wordt; en daarna zult gij in het leger komen.
Num 31:25  Verder sprak de HEERE tot Mozes, zeggende:
Num 31:26  Neem op de som van den buit der gevangenen van mensen en van beesten; gij en Eleazar, de priester, en de hoofden van de vaderen der vergadering.
Num 31:27  En deel den buit in twee helften tussen degenen, die den strijd aangegrepen hebben, die tot den strijd uitgegaan zijn, en tussen de ganse vergadering.
Num 31:28  Daarna zult gij een schatting voor den HEERE heffen, van de oorlogsmannen, die tot dezen krijg uitgetogen zijn, van vijfhonderd een ziel, uit de mensen en uit de runderen, en uit de ezelen, en uit de schapen.
Num 31:29  Van hun helft zult gij het nemen, en den priester Eleazar geven tot een heffing des HEEREN.
Num 31:30  Maar van de helft der kinderen Israels zult gij een gevangene van vijftig nemen, uit de mensen, uit de runderen, uit de ezelen, en uit de schapen, uit al de beesten; en gij zult ze aan de Levieten geven, die de wacht van de tabernakel des HEEREN waarnemen.
Num 31:31  En Mozes en Eleazar, de priester, deden, gelijk als de HEERE Mozes geboden had.
Num 31:32  De buit nu, het overschot van den roof, dat het krijgsvolk geroofd had, was zeshonderd vijf en zeventig duizend schapen;
Num 31:33  En twee en zeventig duizend runderen;
Num 31:34  En een en zestig duizend ezelen;
Num 31:35  En der mensen zielen, uit de vrouwen, die geen bijligging des mans bekend hadden, alle zielen waren twee en dertig duizend.
Num 31:36  En de helft, te weten het deel dergenen, die tot dezen krijg uitgetogen waren, was in getal driehonderd zeven en dertig duizend en vijfhonderd schapen.
Num 31:37  En de schatting voor den HEERE van schapen was zeshonderd vijf en zeventig.
Num 31:38  En de runderen waren zes en dertig duizend, en hun schatting voor den HEERE twee en zeventig.
Num 31:39  En de ezelen waren dertig duizend en vijfhonderd, en hun schatting voor den HEERE was een en zestig.
Num 31:40  En der mensen zielen waren zestien duizend, en hun schatting voor den HEERE twee en dertig zielen.
Num 31:41  En Mozes gaf Eleazar, den priester, de schatting van de heffing des HEEREN, gelijk als de HEERE Mozes geboden had.
Num 31:42  En van de helft der kinderen Israels, welke Mozes afgedeeld had, van de mannen, die gestreden hadden;
Num 31:43  (Het halve deel nu der vergadering was, uit de schapen, driehonderd zeven en dertig duizend en vijfhonderd;
Num 31:44  En de runderen waren zes en dertig duizend;
Num 31:45  En de ezelen dertig duizend en vijfhonderd;
Num 31:46  En der mensen zielen zestien duizend;)
Num 31:47  Van die helft der kinderen Israels nam Mozes een gevangene uit vijftig, van mensen en van beesten; en hij gaf ze aan de Levieten, die de wacht van den tabernakel des HEEREN waarnamen, gelijk als de HEERE Mozes geboden had.
Num 31:48  Toen traden tot Mozes de bevelhebbers, die over de duizenden des heirs waren, de hoofdlieden der duizenden, en de hoofdlieden der honderden;
Num 31:49  En zij zeiden tot Mozes: Uw knechten hebben opgenomen de som der krijgslieden, die onder onze hand geweest zijn; en uit ons ontbreekt niet een man.
Num 31:50  Daarom hebben wij een offerande des HEEREN gebracht, een ieder wat hij gekregen heeft, een gouden vat, een keten, of een armring, een vingerring, een oorring, of een afhangenden gordel, om voor onze zielen verzoening te doen voor het aangezicht des HEEREN.
Num 31:51  Zo nam Mozes en Eleazar, de priester, van hen het goud, alle welgewrochte vaten.
Num 31:52  En al het goud der heffing, dat zij den HEERE offerden, was zestien duizend zevenhonderd en vijftig sikkelen, van de hoofdlieden der duizenden, en van de hoofdlieden der honderden.
Num 31:53  Aangaande de krijgslieden, een iegelijk had geroofd voor zichzelven.
Num 31:54  Zo nam Mozes en Eleazar, de priester, dat goud van de hoofdlieden der duizenden en der honderden, en zij brachten het in de tent der samenkomst, ter gedachtenis voor de kinderen Israels, voor het aangezicht des HEEREN.
Num 32:1  De kinderen van Ruben nu hadden veel vee, en de kinderen van Gad hadden machtig veel; en zij bezagen het land Jaezer, en het land van Gilead, en ziet, deze plaats was een plaats voor vee.
Num 32:2  Zo kwamen de kinderen van Gad en de kinderen van Ruben, en spraken tot Mozes, en tot Eleazar, den priester, en tot de oversten der vergadering, zeggende:
Num 32:3  Ataroth, en Dibon, en Jaezer, en Nimra, en Hesbon, en Eleale, en Schebam, en Nebo, en Behon;
Num 32:4  Dit land, hetwelk de HEERE voor het aangezicht der vergadering van Israel geslagen heeft, is een land voor vee; en uw knechten hebben vee.
Num 32:5  Voorts zeiden zij: Indien wij genade in uw ogen gevonden hebben, dat ditzelve land aan uw knechten gegeven worde tot een bezitting; en doe ons niet trekken over de Jordaan.
Num 32:6  Maar Mozes zeide tot de kinderen van Gad en tot de kinderen van Ruben: Zullen uw broeders ten strijde gaan, en zult gijlieden hier blijven?
Num 32:7  Waarom toch zult gij het hart der kinderen Israels breken, dat zij niet overtrekken naar het land, dat de HEERE hun gegeven heeft?
Num 32:8  Zo deden uw vaders, als ik hen van Kades-barnea zond, om dit land te bezien.
Num 32:9  Als zij opgekomen waren tot aan het dal Eskol, en dit land bezagen, zo braken zij het hart der kinderen Israels, dat zij niet gingen naar het land, dat de HEERE hun gegeven had.
Num 32:10  Toen ontstak de toorn des HEEREN te dien dage, en Hij zwoer, zeggende:
Num 32:11  Indien deze mannen, die uit Egypte opgetogen zijn, van twintig jaren oud en daarboven, het land zullen zien, dat Ik Abraham, Izak en Jakob gezworen heb! Want zij hebben niet volhard Mij na te volgen;
Num 32:12  Behalve Kaleb, de zoon van Jefunne, den Keniziet, en Jozua, de zoon van Nun; want zij hebben volhard den HEERE na te volgen.
Num 32:13  Alzo ontstak des HEEREN toorn tegen Israel, en Hij deed hen omzwerven in de woestijn, veertig jaren, totdat verteerd was het ganse geslacht, hetwelk gedaan had, wat kwaad was in de ogen des HEEREN.
Num 32:14  En ziet, gijlieden zijt opgestaan in stede van uw vaderen, een menigte van zondige mensen, om de hittigheid van des HEEREN toorn tegen Israel te vermeerderen.
Num 32:15  Wanneer gij van achter Hem u zult afkeren, zo zal Hij wijders voortvaren het te laten in de woestijn; en gij zult al dit volk verderven.
Num 32:16  Toen traden zij toe tot hem, en zeiden: Wij zullen hier schaapskooien bouwen voor ons vee, en steden voor onze kinderen.
Num 32:17  Maar wij zelven zullen ons toerusten, haastende voor het aangezicht der kinderen Israels, totdat wij hen aan hun plaats zullen gebracht hebben; en onze kinderen zullen blijven in de vaste steden, vanwege de inwoners des lands.
Num 32:18  Wij zullen niet wederkeren tot onze huizen, totdat zich de kinderen Israels tot erfelijke bezitters zullen gesteld hebben, een ieder van zijn erfenis.
Num 32:19  Want wij zullen met hen niet erven aan gene zijde van de Jordaan, en verder heen, als onze erfenis ons toegekomen zal zijn aan deze zijde van de Jordaan, tegen den opgang.
Num 32:20  Toen zeide Mozes tot hen: Indien gij deze zaak doen zult, indien gij u voor het aangezicht des HEEREN zult toerusten ten strijde.
Num 32:21  En een ieder van u, die toegerust is, over de Jordaan zal trekken voor het aangezicht des HEEREN, totdat Hij Zijn vijanden voor Zijn aangezicht uit de bezitting zal verdreven hebben.
Num 32:22  En het land voor het aangezicht des HEEREN ten ondergebracht zij; zo zult gij daarna wederkeren, en onschuldig zijn voor den HEERE en voor Israel, en dit land zal u ter bezitting zijn voor het aangezicht des HEEREN.
Num 32:23  Indien gij daarentegen alzo niet zult doen, ziet, zo hebt gij tegen den HEERE gezondigd; doch gij zult uw zonde gewaar worden, als zij u vinden zal!
Num 32:24  Bouwt u steden voor uw kinderen, en kooien voor uw schapen; en doet, wat uit uw mond uitgegaan is.
Num 32:25  Toen spraken de kinderen van Gad en de kinderen van Ruben tot Mozes, zeggende: Uw knechten zullen doen, gelijk als mijn heer gebiedt.
Num 32:26  Onze kinderen, onze vrouwen, onze have en al onze beesten zullen aldaar zijn in de steden van Gilead;
Num 32:27  Maar uw knechten zullen overtrekken, al wie ten heire toegerust is, voor het aangezicht des HEEREN tot den strijd, gelijk als mijn heer gesproken heeft.
Num 32:28  Toen gebood Mozes, hunnenthalve, den priester Eleazar, en Jozua, den zoon van Nun, en den hoofden der vaderen van de stammen der kinderen Israels;
Num 32:29  En Mozes zeide tot hen: Indien de kinderen van Gad, en de kinderen van Ruben, met ulieden over de Jordaan zullen trekken, een ieder, die toegerust is ten oorlog, voor het aangezicht des HEEREN, als het land voor uw aangezicht zal ten ondergebracht zijn; zo zult gij hun het land Gilead ter bezitting geven.
Num 32:30  Maar indien zij niet toegerust met u zullen overtrekken, zo zullen zij tot bezitters gesteld worden in het midden van ulieden in het land Kanaan.
Num 32:31  En de kinderen van Gad en de kinderen van Ruben antwoordden, zeggende: Wat de HEERE tot uw knechten gesproken heeft, zullen wij alzo doen.
Num 32:32  Wij zullen toegerust overtrekken voor het aangezicht des HEEREN naar het land Kanaan; en de bezitting onzer erfenis zullen wij hebben aan deze zijde van de Jordaan.
Num 32:33  Alzo gaf Mozes hunlieden, den kinderen van Gad, en de kinderen van Ruben, en den halven stam van Manasse, den zoon van Jozef, het koninkrijk van Sihon, koning der Amorieten, en het koninkrijk van Og, koning van Bazan; het land met de steden van hetzelve in de landpalen, de steden des lands rondom.
Num 32:34  En de kinderen van Gad bouwden Dibon, en Ataroth, en Aroer,
Num 32:35  En Atroth-sofan, en Jaezer, en Jogbeha,
Num 32:36  En Beth-nimra, en Beth-haran, vaste steden en schaapskooien.
Num 32:37  En de kinderen van Ruben bouwden Hezbon, en Eleale, en Kirjathaim,
Num 32:38  En Nebo, en Baal-meon, veranderd zijnde van naam, en Sibma; en zij noemden de namen der steden, die zij bouwden, met andere namen.
Num 32:39  En de kinderen van Machir, den zoon van Manasse, gingen naar Gilead, en namen dat in, en zij verdreven de Amorieten, die daarin waren, uit de bezitting.
Num 32:40  Zo gaf Mozes Gilead aan Machir, den zoon van Manasse; en hij woonde daarin.
Num 32:41  Jair nu, de zoon van Manasse, ging heen en nam hunlieder dorpen in, en hij noemde die Havvoth-jair.
Num 32:42  En Nobah ging heen, en nam Kenath in, met haar onderhorige plaatsen, en noemde ze Nobah naar zijn naam.
Num 33:1  Dit zijn de reizen der kinderen Israels, die uit Egypteland uitgetogen zijn, naar hun heiren, door de hand van Mozes en Aaron.
Num 33:2  En Mozes schreef hun uittochten, naar hun reizen, naar den mond des HEEREN; en dit zijn hun reizen, naar hun uittochten.
Num 33:3  Zij reisden dan van Rameses; in de eerste maand, op den vijftienden dag der eerste maand, des anderen daags van het pascha, togen de kinderen Israels uit door een hoge hand, voor de ogen van alle Egyptenaren;
Num 33:4  Als de Egyptenaars begroeven degenen, welke de HEERE onder hen geslagen had, alle eerstgeborenen; ook had de HEERE gerichten geoefend aan hun goden.
Num 33:5  Als de kinderen Israels van Rameses verreisd waren, zo legerden zij zich te Sukkoth.
Num 33:6  En zij verreisden van Sukkoth, en legerden zich in Etham, hetwelk aan het einde der woestijn is.
Num 33:7  En zij verreisden van Etham, en keerden weder naar Pi-hachiroth, dat tegenover Baal-sefon is, en zij legerden zich voor Migdol.
Num 33:8  En zij verreisden van Hachiroth, en gingen over, door het midden van de zee, naar de woestijn, en zij gingen drie dagreizen in de woestijn Etham, en legerden zich in Mara.
Num 33:9  En zij verreisden van Mara, en kwamen te Elim; in Elim nu waren twaalf waterfonteinen en zeventig palmbomen, en zij legerden zich aldaar.
Num 33:10  En zij verreisden van Elim, en legerden zich aan de Schelfzee.
Num 33:11  En zij verreisden van de Schelfzee, en legerden zich in de woestijn Sin.
Num 33:12  En zij verreisden uit de woestijn Sin, en zij legerden zich in Dofka.
Num 33:13  En zij verreisden van Dofka, en legerden zich in Aluz.
Num 33:14  En zij verreisden van Aluz, en legerden zich in Rafidim; doch daar was geen water voor het volk, om te drinken.
Num 33:15  En zij verreisden van Rafidim, en legerden zich in de woestijn van Sinai.
Num 33:16  En zij verreisden uit de woestijn van Sinai, en legerden zich in Kibroth-thaava.
Num 33:17  En zij verreisden van Kibroth-thaava, en legerden zich in Hazeroth.
Num 33:18  En zij verreisden van Hazeroth, en legerden zich in Rithma.
Num 33:19  En zij verreisden van Rithma, en legerden zich in Rimmon-perez.
Num 33:20  En zij verreisden van Rimmon-perez, en legerden zich in Libna.
Num 33:21  En zij verreisden van Libna, en legerden zich in Rissa.
Num 33:22  En zij verreisden van Rissa, en legerden zich in Kehelatha.
Num 33:23  En zij verreisden van Kehelatha, en legerden zich in het gebergte van Safer.
Num 33:24  En zij verreisden van het gebergte Safer, en legerden zich in Harada.
Num 33:25  En zij verreisden van Harada, en legerden zich in Makheloth.
Num 33:26  En zij verreisden van Makheloth, en legerden zich in Tachath.
Num 33:27  En zij verreisden van Tachath, en legerden zich in Tharah.
Num 33:28  En zij verreisden van Tharah, en legerden zich in Mithka.
Num 33:29  En zij verreisden van Mithka, en legerden zich in Hasmona.
Num 33:30  En zij verreisden van Hasmona, en legerden zich in Moseroth.
Num 33:31  En zij verreisden van Moseroth, en legerden zich in Bene-jaakan.
Num 33:32  En zij verreisden van Bene-jaakan, en legerden zich in Hor-gidgad.
Num 33:33  En zij verreisden van Hor-gidgad, en legerden zich in Jotbatha.
Num 33:34  En zij verreisden van Jotbatha, en legerden zich in Abrona.
Num 33:35  En zij verreisden van Abrona, en legerden zich in Ezeon-geber.
Num 33:36  En zij verreisden van Ezeon-geber, en legerden zich in de woestijn Zin, dat is Kades.
Num 33:37  En zij verreisden van Kades, en legerden zich aan den berg Hor, aan het einde des lands van Edom.
Num 33:38  Toen ging de priester Aaron op den berg Hor, naar den mond des HEEREN, en stierf aldaar, in het veertigste jaar na den uittocht van de kinderen Israels uit Egypteland, in de vijfde maand, op den eersten der maand.
Num 33:39  Aaron nu was honderd drie en twintig jaren oud, als hij stierf op den berg Hor.
Num 33:40  En de Kanaaniet, de koning van Harad, die in het zuiden woonde in het land Kanaan, hoorde, dat de kinderen Israels aankwamen.
Num 33:41  En zij verreisden van den berg Hor, en legerden zich in Zalmona.
Num 33:42  En zij verreisden van Zalmona, en legerden zich in Funon.
Num 33:43  En zij verreisden van Funon, en legerden zich in Oboth.
Num 33:44  En zij verreisden van Oboth, en legerden zich aan de heuvelen van Abarim, in de landpale van Moab.
Num 33:45  En zij verreisden van de heuvelen van Abarim, en legerden zich in Dibon-gad.
Num 33:46  En zij verreisden van Dibon-gad, en legerden zich in Almon-diblathaim.
Num 33:47  En zij verreisden van Almon-diblathaim, en legerden zich in de bergen Abarim, tegen Nebo.
Num 33:48  En zij verreisden van de bergen Abarim, en legerden zich in de vlakke velden der Moabieten, aan de Jordaan van Jericho.
Num 33:49  En zij legerden zich aan de Jordaan van Beth-jesimoth, tot aan Abel-sittim, in de vlakke velden der Moabieten.
Num 33:50  En de HEERE sprak tot Mozes, in de vlakke velden der Moabieten, aan de Jordaan van Jericho, zeggende:
Num 33:51  Spreek tot de kinderen Israels, en zeg tot hen: Wanneer gijlieden over de Jordaan zult gegaan zijn in het land Kanaan;
Num 33:52  Zo zult gij alle inwoners des lands voor uw aangezicht uit de bezitting verdrijven, en al hun beeltenissen verderven; ook zult gij al hun gegotene beelden verderven, en al hun hoogten verdelgen.
Num 33:53  En gij zult het land in erfelijke bezitting nemen, en daarin wonen; want Ik heb u dat land gegeven, om hetzelve erfelijk te bezitten.
Num 33:54  En gij zult het land in erfelijke bezitting nemen door het lot, naar uw geslachten; dengenen, die veel zijn, zult gij hun erfenis meerder maken, en dien, die weinig zijn, zult gij hun erfenis minder maken; waarheen voor iemand het lot zal uitgaan, dat zal hij hebben; naar de stammen uwer vaderen zult gij de erfenis nemen.
Num 33:55  Maar indien gij de inwoners des lands niet voor uw aangezicht uit de bezitting zult verdrijven, zo zal het geschieden, dat, die gij van hen zult laten overblijven, tot doornen zullen zijn in uw ogen, en tot prikkelen in uw zijden, en u zullen benauwen op het land, waarin gij woont.
Num 33:56  En het zal geschieden, dat Ik u zal doen, gelijk als Ik hun dacht te doen.
Num 34:1  Voorts sprak de HEERE tot Mozes, zeggende:
Num 34:2  Gebied den kinderen Israels, en zeg tot hen: Wanneer gij in het land Kanaan ingaat, zo zal dit land zijn, dat u ter erfenis vallen zal, het land Kanaan, naar zijn landpalen.
Num 34:3  De zuiderhoek nu zal u zijn van de woestijn Zin, aan de zijden van Edom; en de zuider landpale zal u zijn van het einde der Zoutzee tegen het oosten;
Num 34:4  En deze landpale zal u omgaan van het zuiden naar den opgang van Akrabbim, en doorgaan naar Zin; en haar uitgangen zullen zijn, van het zuiden naar Kades-barnea; en zij zal uitgaan naar Hazar-addar, en doorgaan naar Azmon.
Num 34:5  Voorts zal deze landpale omgaan van Azmon naar de rivier van Egypte, en haar uitgangen zullen zijn naar de zee.
Num 34:6  Aangaande de landpale van het westen, daar zal u de grote zee de landpale zijn; dit zal uw landpale van het westen zijn.
Num 34:7  Voorts zal u de landpale van het noorden deze zijn: van de grote zee af zult gij u den berg Hor aftekenen.
Num 34:8  Van den berg Hor zult gij aftekenen tot daar men komt te Hamath; en de uitgangen dezer landpale zullen zijn naar Zedad.
Num 34:9  En deze landpale zal uitgaan naar Zifron, en haar uitgangen zullen zijn te Hazar-enan; dit zal u de noorder landpale zijn.
Num 34:10  Voorts zult gij u tot een landpale tegen het oosten aftekenen van Hazar-enan naar Sefam.
Num 34:11  En deze landpale zal afgaan van Sefam naar Ribla, tegen het oosten van Ain; daarna zal deze landpale afgaan en strekken langs den oever van de zee Cinnereth oostwaarts.
Num 34:12  Voorts zal deze landpale afgaan langs de Jordaan, en haar uitgangen zullen zijn aan de Zoutzee. Dit zal u zijn het land naar zijn landpale rondom.
Num 34:13  En Mozes gebood den kinderen Israels, zeggende: Dit is het land, dat gij door het lot ten erve innemen zult, hetwelk de HEERE aan de negen stammen en den halven stam van Manasse te geven geboden heeft.
Num 34:14  Want de stam van de kinderen der Rubenieten, naar het huis hunner vaderen, en de stam van de kinderen der Gadieten, naar het huis hunner vaderen, hebben ontvangen; mitsgaders de halve stam van Manasse heeft zijn erfenis ontvangen.
Num 34:15  Twee stammen en een halve stam hebben hun erfenis ontvangen aan deze zijde van de Jordaan, van Jericho oostwaarts tegen den opgang.
Num 34:16  Voorts sprak de HEERE tot Mozes, zeggende:
Num 34:17  Dit zijn de namen der mannen, die ulieden het land ten erve zullen uitdelen: Eleazar, de priester, en Jozua, de zoon van Nun.
Num 34:18  Daartoe zult gij uit elken stam een overste nemen, om het land ten erve uit te delen.
Num 34:19  En dit zijn de namen dezer mannen: van den stam van Juda, Kaleb, zoon van Jefunne;
Num 34:20  En van den stam der kinderen van Simeon, Semuel, zoon van Ammihud;
Num 34:21  Van den stam van Benjamin, Elidad, zoon van Chislon;
Num 34:22  En van den stam der kinderen van Dan, de overste Bukki, zoon van Jogli;
Num 34:23  Van de kinderen van Jozef: van den stam der kinderen van Manasse, de overste Hanniel, zoon van Efod;
Num 34:24  En van den stam der kinderen van Efraim, de overste Kemuel, zoon van Siftan;
Num 34:25  En van den stam der kinderen van Zebulon, de overste Elizafan, zoon van Parnach;
Num 34:26  En van den stam der kinderen van Issaschar, de overste Paltiel, zoon van Azzan;
Num 34:27  En van den stam der kinderen van Aser, de overste Achihud, zoon van Selomi;
Num 34:28  En van den stam der kinderen van Nafthali, de overste Pedael, zoon van Ammihud.
Num 34:29  Dit zijn ze, dien de HEERE geboden heeft, den kinderen Israels de erfenissen uit te delen, in het land Kanaan.
Num 35:1  En de HEERE sprak tot Mozes, in de vlakke velden der Moabieten, aan de Jordaan van Jericho, zeggende:
Num 35:2  Gebied den kinderen Israels, dat zij van de erfenis hunner bezitting aan de Levieten steden zullen geven om te bewonen; daartoe zult gijlieden aan de Levieten voorsteden geven, aan de steden rondom dezelve.
Num 35:3  En die steden zullen zij hebben om te bewonen; maar hun voorsteden zullen zijn voor hun beesten, en voor hun have, en voor al hun gedierte,
Num 35:4  En de voorsteden der steden, die gij aan de Levieten zult geven, zullen van den stadsmuur af, en naar buiten, van duizend ellen zijn rondom.
Num 35:5  En gij zult meten van buiten de stad, aan den hoek tegen het oosten, twee duizend ellen, en aan den hoek van het zuiden, twee duizend ellen, en aan den hoek van het westen, twee duizend ellen, en aan den hoek van het noorden, twee duizend ellen; dat de stad in het midden zij. Dit zullen zij hebben tot voorsteden van de steden.
Num 35:6  De steden nu, die gij aan de Levieten zult geven, zullen zijn zes vrijsteden, die gij geven zult, opdat de doodslager daarheen vliede; en boven dezelve zult gij hun twee en veertig steden geven.
Num 35:7  Al de steden, die gij aan de Levieten geven zult, zullen zijn acht en veertig steden, deze met haar voorsteden.
Num 35:8  De steden, die gij van de bezitting der kinderen Israels geven zult, zult gij van dien, die vele heeft, vele nemen, en van dien, die weinig heeft, weinige nemen; een ieder zal naar zijn erfenis, die zij zullen erven, van zijn steden aan de Levieten geven.
Num 35:9  Voorts sprak de HEERE tot Mozes, zeggende:
Num 35:10  Spreek tot de kinderen Israels, en zeg tot hen: Wanneer gij over de Jordaan gaat naar het land Kanaan.
Num 35:11  Zo zult gij maken, dat u steden tegemoet liggen, die u tot vrijsteden zullen zijn; opdat de doodslager daarheen vliede, die een ziel onwetend geslagen heeft.
Num 35:12  En deze steden zullen u tot een toevlucht zijn voor den bloed wreker; opdat de doodslager niet sterve, totdat hij voor de vergadering aan het gericht gestaan hebbe.
Num 35:13  En deze steden, die gij geven zult, zullen zes vrijsteden voor u zijn.
Num 35:14  Drie dezer vrijsteden zult gij geven op deze zijde van de Jordaan, en drie dezer steden zult gij geven in het land Kanaan; vrijsteden zullen het zijn.
Num 35:15  Die zes steden zullen voor de kinderen Israels, en voor den vreemdeling, en den bijwoner in het midden van hen, tot een toevlucht zijn; opdat daarheen vliede, wie een ziel onvoorziens slaat.
Num 35:16  Maar indien hij hem met een ijzeren instrument geslagen heeft, dat hij gestorven zij, een doodslager is hij; deze doodslager zal zekerlijk gedood worden.
Num 35:17  Of indien hij hem met een handsteen, waarvan met zoude kunnen sterven, geslagen heeft, dat hij gestorven zij, een doodslager is hij; deze doodslager zal zekerlijk gedood worden.
Num 35:18  Of indien hij hem met een houten handinstrument, waarvan men zoude kunnen sterven, geslagen heeft, dat hij gestorven zij, een doodslager is hij; deze doodslager zal zekerlijk gedood worden.
Num 35:19  De wreker des bloeds, die zal den doodslager doden; als hij hem ontmoet, zal hij hem doden.
Num 35:20  Indien hij hem ook door haat zal gestoten hebben, of met opzet op hem geworpen heeft, dat hij gestorven zij;
Num 35:21  Of hem door vijandschap met zijn hand geslagen heeft, dat hij gestorven zij; de slager zal zekerlijk gedood worden, een doodslager is hij; de bloedwreker zal dezen doodslager doden, als hij hem ontmoet.
Num 35:22  Maar indien hij hem met der haast zonder vijandschap gestoten heeft, of enig instrument zonder opzet op hem geworpen heeft;
Num 35:23  Of onvoorziens met enigen steen, waarvan men zoude kunnen sterven, en hij dien op hem heeft doen vallen, dat hij gestorven zij, zo hij hem toch geen vijand was, noch zijn kwaad zoekende;
Num 35:24  Zo zal de vergadering richten tussen den slager, en tussen den bloedwreker, naar deze zelve rechten.
Num 35:25  En de vergadering zal den doodslager redden uit den hand des bloedwrekers, en de vergadering zal hem doen wederkeren tot zijn vrijstad, waarheen hij gevloden was; en hij zal daarin blijven tot den dood des hogepriesters, dien men met de heilige olie gezalfd heeft.
Num 35:26  Doch indien de doodslager enigszins zal gaan uit de palen zijner vrijstad, waarheen hij gevloden was,
Num 35:27  En de bloedwreker hem zal vinden buiten de palen zijner vrijstad; zo de bloedwreker den doodslager zal doden, het zal hem geen bloedschuld zijn.
Num 35:28  Want hij zou in zijn vrijstad gebleven zijn tot den dood des hogepriesters; maar na de dood des hogepriesters zal de doodslager wederkeren tot het land zijner bezitting.
Num 35:29  En deze dingen zullen ulieden zijn tot een inzetting van recht, bij uw geslachten, in al uw woningen.
Num 35:30  Al wie de ziel slaat, naar den mond der getuigen zal men den doodslager doden, maar een enig getuige zal niet getuigen tegen een ziel, dat zij sterve.
Num 35:31  En gij zult geen verzoening nemen voor de ziel des doodslagers, die schuldig is te sterven; want hij zal zekerlijk gedood worden.
Num 35:32  Ook zult gij geen verzoening nemen voor dien, die gevlucht is naar zijn vrijstad, dat hij zou wederkeren, om te wonen in het land, tot den dood des hoge priesters.
Num 35:33  Zo zult gij niet ontheiligen het land, waarin gij zijt; want het bloed ontheiligt het land; en voor het land zal geen verzoening gedaan worden over het bloed, dat daarin vergoten is, dan door het bloed desgenen, die dat vergoten heeft.
Num 35:34  Verontreinigt dan het land niet, waarin gij gaat wonen, in welks midden Ik wonen zal; want Ik ben de HEERE, wonende in het midden der kinderen Israels.
Num 36:1  En de hoofden der vaderen van het geslacht de kinderen van Gilead, den zoon van Machir, den zoon van Manasse, uit de geslachten der kinderen van Jozef, traden toe, en spraken voor het aangezicht van Mozes, en voor het aangezicht der oversten, hoofden van de vaderen der kinderen Israels.
Num 36:2  En zeiden: De HEERE heeft mijn heer geboden, dat land door het lot aan de kinderen Israels in erfenis te geven; en mijn heer is door den HEERE geboden, de erfenis van onzen broeder Zelafead te geven aan zijn dochteren.
Num 36:3  Wanneer zij een van de zonen der andere stammen van de kinderen Israels tot vrouwen zouden worden, zo zou haar erfenis van de erfenis onzer vaderen afgetrokken worden, en toegedaan tot de erfenis van dien stam, aan welken zij geworden zouden; alzo zou van het lot onzer erfenis worden afgetrokken.
Num 36:4  Als ook de kinderen Israels een jubeljaar zullen hebben, zo zou haar erfenis toegedaan zijn tot de erfenis van dien stam, aan welken zij zouden geworden zijn; alzo zou haar erfenis van de erfenis van den stam onzer vaderen afgetrokken worden.
Num 36:5  Toen gebood Mozes den kinderen Israels, naar des HEEREN mond, zeggende: De stam der kinderen van Jozef spreekt recht.
Num 36:6  Dit is het woord, dat de HEERE van de dochteren van Zelafead geboden heeft, zeggende: Laat zij dien tot vrouwen worden, die in haar ogen goed zal zijn; alleenlijk, dat zij aan het geslacht van haars vaders stam tot vrouwen worden.
Num 36:7  Zo zal de erfenis van de kinderen Israels niet omgewend worden van stam tot stam; want de kinderen Israels zullen aanhangen, een ieder aan de erfenis van den stam zijner vaderen.
Num 36:8  Voorts zal elke dochter, die een erfenis erft, van de stammen der kinderen Israels, ter vrouw worden aan een van het geslacht van den stam haars vaders; opdat de kinderen Israels erfelijk bezitten, een ieder de erfenis zijner vaderen.
Num 36:9  Zo zal de erfenis niet omgewend worden van den enen stam tot den anderen; want de stammen der kinderen Israels zullen aanhangen, een ieder aan zijn erfenis.
Num 36:10  Gelijk als de HEERE Mozes geboden had, alzo deden de dochteren van Zelafead;
Num 36:11  Want Machla, Thirza, en Hogla, en Milka, en Noha, dochteren van Zelafead, zijn den zonen harer ooms tot vrouwen geworden.
Num 36:12  Onder de geslachten van de kinderen van Manasse, den zoon van Jozef, zijn zij tot vrouwen geworden; alzo bleef haar erfenis aan den stam van het geslacht haars vaders.
Num 36:13  Dat zijn de geboden en de rechten, die de HEERE door de dienst van Mozes aan de kinderen Israels geboden heeft, in de vlakke velden der Moabieten, aan de Jordaan van Jericho.



\end{document}