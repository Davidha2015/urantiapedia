\begin{document}

\title{2 Samuel}



\chapter{1}

\par 1 Voorts geschiedde het na Sauls dood, als David van den slag der Amalekieten was wedergekomen, en David twee dagen te Ziklag gebleven was;
\par 2 Zo geschiedde het op den derden dag, dat, ziet, uit het heirleger van Saul, een man kwam, wiens klederen gescheurd waren, en aarde was op zijn hoofd; en het geschiedde, als hij tot David kwam, zo viel hij ter aarde en boog zich neder.
\par 3 En David zeide tot hem: Van waar komt gij? En hij zeide tot hem: Ik ben ontkomen uit het heirleger van Israel.
\par 4 Voorts zeide David tot hem: Wat is de zaak? Verhaal het mij toch. En hij zeide, dat het volk uit den strijd gevloden was, en dat er ook velen van het volk gevallen en gestorven waren, dat ook Saul en zijn zoon Jonathan dood waren.
\par 5 En David zeide tot den jongen, die hem de boodschap bracht: Hoe weet gij, dat Saul dood is, en zijn zoon Jonathan?
\par 6 Toen zeide de jongen, die hem de boodschap bracht: Ik kwam bij geval op het gebergte van Gilboa; en ziet, Saul leunde op zijn spies; en ziet, de wagens en ritmeesters hielden dicht op hem.
\par 7 Zo zag hij achter zich om, en zag mij, en hij riep mij, en ik zeide: Zie, hier ben ik.
\par 8 En hij zeide tot mij: Wie zijt gij? En ik zeide tot hem: Ik ben een Amalekiet.
\par 9 Toen zeide hij tot mij: Sta toch bij mij, en dood mij; want deze malienkolder heeft mij opgehouden; want mijn leven is nog gans in mij.
\par 10 Zo stond ik bij hem, en doodde hem; want ik wist, dat hij na zijn val niet leven zou; en ik nam de kroon, die op zijn hoofd was, en het armgesmijde, dat aan zijn arm was, en heb ze hier tot mijn heer gebracht.
\par 11 Toen vatte David zijn klederen en scheurde ze; desgelijks ook al de mannen, die met hem waren.
\par 12 En zij weeklaagden, en weenden, en vastten tot op den avond, over Saul en over Jonathan, zijn zoon, en over het volk des HEEREN, en over het huis Israels, omdat zij door het zwaard gevallen waren.
\par 13 Voorts zeide David tot den jongen, die hem de boodschap gebracht had: Van waar zijt gij? En hij zeide: Ik ben de zoon van een vreemden man, van een Amalekiet.
\par 14 En David zeide tot hem: Hoe, hebt gij niet gevreesd uw hand uit te strekken, om den gezalfde des HEEREN te verderven?
\par 15 En David riep een van de jongens, en zeide: Treed toe, val op hem aan. En hij sloeg hem, dat hij stierf.
\par 16 En David zeide tot hem: Uw bloed zij op uw hoofd; want uw mond heeft tegen u getuigd, zeggende: ik heb den gezalfde des HEEREN gedood.
\par 17 David nu klaagde deze klage over Saul en over Jonathan, zijn zoon;
\par 18 Als hij gezegd had, dat men den kinderen van Juda den boog zou leren; ziet, het is geschreven in het boek des Oprechten.
\par 19 O Sieraad van Israel, op uw hoogten is hij verslagen; hoe zijn de helden gevallen!
\par 20 Verkondigt het niet te Gath, boodschapt het niet op de straten van Askelon; opdat de dochters der Filistijnen zich niet verblijden, opdat de dochters der onbesnedenen niet opspringen van vreugde.
\par 21 Gij, bergen van Gilboa, noch dauw noch regen moet zijn op u, noch velden der hefofferen; want aldaar is der helden schild smadelijk weggeworpen, het schild van Saul, alsof hij niet gezalfd ware geweest met olie.
\par 22 Van het bloed der verslagenen, van het vette der helden, werd Jonathans boog niet achterwaarts gedreven; en Sauls zwaard keerde niet ledig weder.
\par 23 Saul en Jonathan, die beminden, en die liefelijken in hun leven, zijn ook in hun dood niet gescheiden; zij waren lichter dan arenden, zij waren sterker dan leeuwen.
\par 24 Gij, dochteren Israels, weent over Saul; die u kleedde met scharlaken, met weelde; die u sieraad van goud deed dragen over uw kleding.
\par 25 Hoe zijn de helden gevallen in het midden van den strijd! Jonathan is verslagen op uw hoogten!
\par 26 Ik ben benauwd om uwentwil, mijn broeder Jonathan! Gij waart mij zeer liefelijk; uw liefde was mij wonderlijker dan liefde der vrouwen.
\par 27 Hoe zijn de helden gevallen, en de krijgswapenen verloren!

\chapter{2}

\par 1 En het geschiedde daarna, dat David den HEERE vraagde, zeggende: Zal ik optrekken in een der steden van Juda? En de HEERE zeide tot hem: Trek op. En David zeide: Waarheen zal ik optrekken? En Hij zeide: Naar Hebron.
\par 2 Alzo toog David derwaarts op, als ook zijn twee vrouwen, Ahinoam, de Jizreelietische, en Abigail, de huisvrouw van Nabal, de Karmeliet.
\par 3 Ook deed David zijn mannen optrekken, die bij hem waren, een iegelijk met zijn huisgezin; en zij woonden in de steden van Hebron.
\par 4 Daarna kwamen de mannen van Juda, en zalfden aldaar David tot een koning over het huis van Juda. Toen boodschapten zij David, zeggende: Het zijn de mannen van Jabes in Gilead, die Saul begraven hebben.
\par 5 Toen zond David boden tot de mannen van Jabes in Gilead, en hij zeide tot hen: Gezegend zijt gij den HEERE, dat gij deze weldadigheid gedaan hebt aan uw heer, aan Saul, en hebt hem begraven.
\par 6 Zo doe nu de HEERE aan u weldadigheid en trouw! En ik ook, ik zal aan u dit goede doen, dewijl gij deze zaak gedaan hebt.
\par 7 En nu, laat uw handen sterk zijn, en zijt dapper, dewijl uw heer Saul gestorven is; en ook hebben mij die van het huis van Juda tot koning over zich gezalfd.
\par 8 Abner nu, de zoon van Ner, de krijgsoverste, dien Saul gehad had, nam Isboseth, Sauls zoon, en voerde hem over naar Mahanaim,
\par 9 En maakte hem ten koning over Gilead, en over de Aschurieten, en over Jizreel, en over Efraim, en over Benjamin, en over gans Israel.
\par 10 Veertig jaren was Isboseth, Sauls zoon, oud, als hij koning werd over Israel; en hij regeerde het tweede jaar; alleenlijk die van het huis van Juda volgden David na.
\par 11 Het getal nu der dagen, die David koning geweest is te Hebron, over het huis van Juda, is zeven jaren en zes maanden.
\par 12 Toen toog Abner, de zoon van Ner, uit, met de knechten van Isboseth, den zoon van Saul, van Mahanaim naar Gibeon.
\par 13 Joab, de zoon van Zeruja, en de knechten van David, togen ook uit; en zij ontmoetten elkander bij den vijver van Gibeon; en zij bleven, deze aan deze zijde des vijvers, en die aan gene zijde des vijvers.
\par 14 En Abner zeide tot Joab: Laat zich nu de jongens opmaken, en voor ons aangezicht spelen. En Joab zeide: Laat hen zich opmaken.
\par 15 Toen maakten zich op, en gingen over in getal, twaalf van Benjamin, te weten voor Isboseth, Sauls zoon, en twaalf van Davids knechten.
\par 16 En de een greep den ander bij het hoofd, en stiet zijn zwaard in de zijde des anderen, en zij vielen te zamen; daarvan noemde men dezelve plaats Chelkath-hazurim, die bij Gibeon is.
\par 17 En er was op dienzelfden dag een gans zeer harde strijd. Doch Abner en de mannen van Israel werden voor het aangezicht der knechten van David geslagen.
\par 18 Nu waren aldaar drie zonen van Zeruja, Joab, en Abisai en Asahel; en Asahel was licht op zijn voeten, als een der reeen, die in het veld zijn.
\par 19 En Asahel jaagde Abner achterna; en hij week niet, om van achter Abner ter rechter hand of ter linkerhand af te gaan.
\par 20 Toen zag Abner achter zich om, en zeide: Zijt gij dit, Asahel? En hij zeide: Ik ben het.
\par 21 En Abner zeide tot hem: Wijk tot uw rechterhand of tot uw linkerhand, en grijp u een van die jongens, en neem voor u hun gewaad; maar Asahel wilde niet afwijken van achter hem.
\par 22 Toen voer Abner wijders voort, zeggende tot Asahel: Wijkt af van achter mij; waarom zal ik u ter aarde slaan? Hoe zou ik dan mijn aangezicht opheffen voor uw broeder Joab?
\par 23 Maar hij weigerde af te wijken. Zo sloeg hem Abner met het achterste van de spies aan de vijfde rib, dat de spies van achter hem uitging; en hij viel aldaar, en stierf op zijn plaats. En het geschiedde, dat allen, die tot de plaats kwamen, alwaar Asahel gevallen en gestorven was, staan bleven.
\par 24 Maar Joab en Abisai jaagden Abner achterna; en de zon ging onder, als zij gekomen waren tot den heuvel van Amma, dewelke is voor Giach, op den weg der woestijn van Gibeon.
\par 25 En de kinderen van Benjamin verzamelden zich achter Abner, en werden tot een hoop; en zij stonden op de spits van een heuvel.
\par 26 Toen riep Abner tot Joab, en zeide: Zal dan het zwaard eeuwiglijk verteren? Weet gij niet, dat het in het laatste bitterheid zal zijn? En hoe lang zult gij het volk niet zeggen, dat zij wederkeren van hun broederen te vervolgen?
\par 27 En Joab zeide: Zo waarachtig als God leeft, ten ware dat gij gesproken hadt, zekerlijk het volk zou al toen van den morgen af weggevoerd zijn geweest, een iegelijk van zijn broeder te vervolgen!
\par 28 Toen blies Joab met de bazuin; en al het volk stond stil, en zij jaagden Israel niet meer achterna, en voeren niet wijders voort te strijden.
\par 29 Abner dan en zijn mannen gingen dienzelfden gansen nacht over het vlakke veld; en zij gingen over de Jordaan en wandelden het ganse Bithron door, en kwamen tot Mahanaim.
\par 30 Joab keerde ook weder van achter Abner, en verzamelde het ganse volk. En er werden van Davids knechten gemist negentien mannen, en Asahel.
\par 31 Maar Davids knechten hadden van Benjamin en onder Abners mannen geslagen: driehonderd en zestig mannen waren er dood gebleven.
\par 32 En zij namen Asahel op, en begroeven hem in zijns vaders graf, dat te Bethlehem was. Joab nu en zijn mannen gingen den gansen nacht, dat hun het licht aanbrak te Hebron.

\chapter{3}

\par 1 En er was een lange krijg tussen het huis van Saul, en tussen het huis van David. Doch David ging en werd sterker; maar die van het huis van Saul gingen en werden zwakker.
\par 2 En David werden zonen geboren te Hebron. Zijn eerstgeborene nu was Amnon, van Ahinoam, de Jizreelietische;
\par 3 En zijn tweede was Chileab, van Abigail, de huisvrouw van Nabal, den Karmeliet; en de derde, Absalom, de zoon van Maacha, de dochter van Thalmai, koning van Gesur;
\par 4 En de vierde, Adonia, de zoon van Haggith; en de vijfde Sefatja, de zoon van Abital;
\par 5 En de zesde, Jithream, van Egla, Davids huisvrouw. Dezen zijn David geboren te Hebron.
\par 6 Terwijl die krijg was tussen het huis van Saul, en tussen het huis van David, zo geschiedde het, dat Abner zich sterkte in het huis van Saul.
\par 7 Saul nu had een bijwijf gehad, welker naam was Rizpa, dochter van Aja; en Isboseth zeide tot Abner: Waarom zijt gij ingegaan tot mijns vaders bijwijf?
\par 8 Toen ontstak Abner zeer over Isboseths woorden, en zeide: Ben ik dan een hondskop, ik, die tegen Juda, aan het huis van Saul, uw vader, aan zijn broederen en aan zijn vrienden, heden weldadigheid doe, en u niet overgeleverd heb in Davids hand, dat gij heden aan mij onderzoekt de ongerechtigheid ener vrouw?
\par 9 God doe Abner zo, en doe hem zo daartoe! Voorzeker, gelijk als de HEERE aan David gezworen heeft, dat ik even alzo aan hem zal doen.
\par 10 Overbrengende het koninkrijk van het huis van Saul, en oprichtende den stoel van David over Israel en over Juda, van Dan tot Ber-seba toe.
\par 11 En hij kon Abner verder niet een woord antwoorden, omdat hij hem vreesde.
\par 12 Toen zond Abner boden voor zich tot David, zeggende: Wiens is het land? zeggende wijders: Maak uw verbond met mij, en zie, mijn hand zal met u zijn, om gans Israel tot u om te keren.
\par 13 En hij zeide: Wel, ik zal een verbond met u maken; doch een ding begeer ik van u, zeggende: Gij zult mijn aangezicht niet zien, tenzij dat gij Michal, Sauls dochter, te voren inbrengt, als gij komt om mijn aangezicht te zien.
\par 14 Ook zond David boden tot Isboseth, den zoon van Saul, zeggende: Geef mij mijn huisvrouw Michal, die ik mij met honderd voorhuiden der Filistijnen ondertrouwd heb.
\par 15 Isboseth dan zond heen, en nam haar van den man, van Paltiel, den zoon van Lais.
\par 16 En haar man ging met haar, al gaande en wenende achter haar, tot Bahurim toe. Toen zeide Abner tot hem: Ga weg, keer weder. En hij keerde weder.
\par 17 Abner nu had woorden met de oudsten van Israel, zeggende: Gij hebt David te voren lang tot een koning over u begeerd.
\par 18 Zo doet het nu; want de HEERE heeft tot David gesproken, zeggende: Door de hand van David, Mijn knecht, zal Ik Mijn volk Israel verlossen van de hand der Filistijnen, en van de hand van al hun vijanden.
\par 19 En Abner sprak ook voor de oren van Benjamin. Voorts ging Abner ook heen, om te Hebron voor Davids oren te spreken alles, wat goed was in de ogen van Israel, en in de ogen van het ganse huis van Benjamin.
\par 20 En Abner kwam tot David te Hebron, en twintig mannen met hem. En David maakte Abner, en den mannen, die met hem waren, een maaltijd.
\par 21 Toen zeide Abner tot David: Ik zal mij opmaken, en heengaan, en vergaderen gans Israel tot mijn heer, den koning, dat zij een verbond met u maken, en gij regeert over alles, wat uw ziel begeert. Alzo liet David Abner gaan, en hij ging in vrede.
\par 22 En ziet, Davids knechten en Joab kwamen van een bende, en brachten met zich een groten roof. Abner nu was niet bij David te Hebron; want hij had hem laten gaan, en hij was gegaan in vrede.
\par 23 Als nu Joab en het ganse heir, dat met hem was, aankwamen, zo gaven zij Joab te kennen, zeggende: Abner, de zoon van Ner, is gekomen tot den koning, en hij heeft hem laten gaan, en hij is gegaan in vrede.
\par 24 Toen ging Joab tot den koning in, en zeide: Wat hebt gij gedaan? Zie, Abner is tot u gekomen; waarom nu hebt gij hem laten gaan, dat hij zo vrij is weggegaan?
\par 25 Gij kent Abner, den zoon van Ner; dat hij gekomen is om u te overreden, en om te weten uw uitgang en uw ingang, ja, om te weten alles, wat gij doet.
\par 26 En Joab ging uit van David, en zond Abner boden na, die hem wederom haalden van den bornput van Sira; maar David wist het niet.
\par 27 Als nu Abner weder te Hebron kwam, zo leidde Joab hem ter zijde af in het midden der poort, om in de stilte met hem te spreken; en hij sloeg hem aldaar aan de vijfde rib, dat hij stierf, om des bloeds wil van zijn broeder Asahel.
\par 28 Als David dat daarna hoorde, zo zeide hij: Ik ben onschuldig, en mijn koninkrijk, bij den HEERE, tot in eeuwigheid, van het bloed van Abner, den zoon van Ner.
\par 29 Het blijve op het hoofd van Joab, en op het ganse huis zijns vaders; en er worde van het huis van Joab niet afgesneden, die een vloed hebbe, en melaats zij, en zich aan den stok houde, en door het zwaard valle, en broodsgebrek hebbe!
\par 30 Alzo hebben Joab en zijn broeder Abisai Abner doodgeslagen, omdat hij hun broeder Asahel te Gibeon in den strijd gedood had.
\par 31 David dan zeide tot Joab en tot al het volk, dat bij hem was: Scheurt uw klederen, en gordt zakken aan, en weeklaagt voor Abner henen; en de koning David ging achter de baar.
\par 32 Als zij nu Abner te Hebron begroeven, zo hief de koning zijn stem op, en weende bij Abners graf; ook weende al het volk.
\par 33 En de koning maakte een klage over Abner, en zeide: Is dan Abner gestorven, als een dwaas sterft?
\par 34 Uw handen waren niet gebonden, noch uw voeten in koperen boeien gedaan, maar gij zijt gevallen, gelijk men valt voor het aangezicht van kinderen der verkeerdheid. Toen weende het ganse volk nog meer over hem.
\par 35 Daarna kwam al het volk, om David brood te doen eten, als het nog dag was; maar David zwoer, zeggende: God doe mij zo, en doe er zo toe, indien ik voor het ondergaan der zon brood of iets smake!
\par 36 Als al het volk dit vernam, zo was het goed in hun ogen, alles, zoals de koning gedaan had, was goed in de ogen van het ganse volk.
\par 37 En al het volk en gans Israel merkten te dienzelven dage, dat het van den koning niet was, dat men Abner, den zoon van Ner, gedood had.
\par 38 Voorts zeide de koning tot zijn knechten: Weet gij niet, dat te dezen dage een vorst, ja, een grote in Israel gevallen is?
\par 39 Maar ik ben heden teder, en gezalfd ten koning, en deze mannen, de zonen van Zeruja, zijn harder dan ik; de HEERE zal den boosdoener vergelden naar zijn boosheid.

\chapter{4}

\par 1 Als nu Sauls zoon hoorde, dat Abner te Hebron gestorven was, werden zijn handen slap, en gans Israel werd verschrikt.
\par 2 En Sauls zoon had twee mannen, oversten van benden; de naam des enen was Baena, en de naam des anderen Rechab, zonen van Rimmon, den Beerothiet, van de kinderen van Benjamin; want ook Beeroth werd aan Benjamin gerekend.
\par 3 En de Beerothieten waren gevloden naar Gitthaim, en waren aldaar vreemdelingen tot op dezen dag.
\par 4 En Jonathan, Sauls zoon, had een zoon, die geslagen was aan beide voeten; vijf jaren was hij oud als het gerucht van Saul en Jonathan uit Jizreel kwam; en zijn voedster hem opnam, en vluchtte; en het geschiedde, als zij haastte, om te vluchten, dat hij viel en kreupel werd; en zijn naam was Mefiboseth.
\par 5 En de zonen van Rimmon: den Beerothiet, Rechab en Baena, gingen heen, en kwamen ten huize van Isboseth, als de dag heet geworden was; en hij lag op de slaapstede, in den middag.
\par 6 En zij kwamen daarin tot het midden des huizes, als zullende tarwe halen; en zij sloegen hem aan de vijfde rib; en Rechab en zijn broeder Baena ontkwamen.
\par 7 Want zij kwamen in huis, als hij op zijn bed lag, in zijn slaapkamer, en sloegen hem, en doodden hem, en hieuwen zijn hoofd af; en zij namen zijn hoofd, en gingen henen, den weg op het vlakke veld, den gansen nacht.
\par 8 En zij brachten het hoofd van Isboseth tot David te Hebron, en zeiden tot den koning: Zie, daar is het hoofd van Isboseth, den zoon van Saul, uw vijand, die uw ziel zocht, alzo heeft de HEERE mijn heer den koning te dezen dage wrake gegeven van Saul en van zijn zaad.
\par 9 Maar David antwoordde Rechab en zijn broeder Baena, den zonen van Rimmon, den Beerothiet, en zeide tot hen: Zo waarachtig als de HEERE leeft, Die mijn ziel uit alle benauwdheid verlost heeft!
\par 10 Dewijl ik hem, die mij boodschapte, zeggende: Zie, Saul is dood; daar hij in zijn ogen was als een, die goede boodschap bracht, nochtans gegrepen en te Ziklag gedood heb, hoewel hij meende, dat ik hem bodenloon zou geven;
\par 11 Hoeveel te meer, wanneer goddeloze mannen een rechtvaardigen man in zijn huis op zijn slaapstede hebben gedood? Nu dan, zou ik zijn bloed van uw handen niet eisen, en u van de aarde wegdoen?
\par 12 En David gebood zijn jongens, en zij doodden hen, en hieuwen hun handen en hun voeten af, en hingen ze op bij den vijver te Hebron, maar het hoofd van Isboseth namen zij, en begroeven het in Abners graf te Hebron.

\chapter{5}

\par 1 Toen kwamen alle stammen van Israel tot David te Hebron; en zij spraken, zeggende: Zie, wij, uw gebeente en uw vlees zijn wij.
\par 2 Daartoe ook te voren, toen Saul koning over ons was, waart gij Israel uitvoerende en inbrengende; ook heeft de HEERE tot u gezegd: Gij zult Mijn volk Israel weiden, en gij zult tot een voorganger zijn over Israel.
\par 3 Alzo kwamen alle oudsten van Israel tot den koning te Hebron; en de koning David maakte een verbond met hen te Hebron, voor het aangezicht des HEEREN; en zij zalfden David tot koning over Israel.
\par 4 Dertig jaar was David oud, als hij koning werd; veertig jaren heeft hij geregeerd.
\par 5 Te Hebron regeerde hij over Juda zeven jaren en zes maanden; en te Jeruzalem regeerde hij drie en dertig jaren over gans Israel en Juda.
\par 6 En de koning toog met zijn mannen naar Jeruzalem, tegen de Jebusieten, die in dat land woonden. En zij spraken tot David, zeggende: Gij zult hier niet inkomen, maar de blinden en kreupelen zullen u afdrijven; dat is te zeggen: David zal hier niet inkomen.
\par 7 Maar David nam den burg Sion in; dezelve is de stad Davids.
\par 8 Want David zeide ten zelfden dage: Al wie de Jebusieten slaat, en geraakt aan die watergoot, en die kreupelen, en die blinden, die van Davids ziel gehaat zijn, die zal tot een hoofd en tot een overste zijn; daarom zegt men: Een blinde en kreupele zal in het huis niet komen.
\par 9 Alzo woonde David in den burg en noemde dien Davids stad. En David bouwde rondom van Millo af en binnenwaarts.
\par 10 David nu ging geduriglijk voort, en werd groot; want de HEERE, de God der heirscharen, was met hem.
\par 11 En Hiram, de koning van Tyrus, zond boden tot David, en cederenhout, en timmerlieden, en metselaars; en zij bouwden David een huis.
\par 12 En David merkte, dat de HEERE hem tot een koning over Israel bevestigd had, en dat Hij zijn koninkrijk verheven had, om Zijns volks Israels wil.
\par 13 En David nam meer bijwijven, en vrouwen van Jeruzalem, nadat hij van Hebron gekomen was; en David werden meer zonen en dochteren geboren.
\par 14 En dit zijn de namen dergenen, die hem te Jeruzalem geboren zijn: Schammua, en Schobab, en Nathan, en Salomo.
\par 15 En Ibchar, en Elischua en Nefeg, en Jafia,
\par 16 En Elischama, en Eljada, en Elifeleth.
\par 17 Als nu de Filistijnen hoorden, dat zij David ten koning over Israel gezalfd hadden, zo togen alle Filistijnen op om David te zoeken; en David, dat horende, toog af, naar den burg.
\par 18 En de Filistijnen kwamen en verspreidden zich in het dal Refaim.
\par 19 Zo vraagde David den HEERE, zeggende: Zal ik optrekken tegen de Filistijnen? Zult Gij ze in mijn hand geven? En de HEERE zeide tot David: Trek op, want Ik zal de Filistijnen zekerlijk in uw hand geven.
\par 20 Toen kwam David te Baal-perazim; en David sloeg hen aldaar, en zeide: De HEERE heeft mijn vijanden voor mijn aangezicht gescheurd, als een scheur der wateren; daarom noemde hij den naam derzelve plaats, Baal-perazim.
\par 21 En zij lieten hun afgoden aldaar; en David en zijn mannen namen ze op.
\par 22 Daarna togen de Filistijnen weder op; en zij verspreidden zich in het dal Refaim.
\par 23 En David vraagde den HEERE, Dewelke zeide: Gij zult niet optrekken; maar trek om tot achter hen, dat gij aan hen komt van tegenover de moerbezienbomen;
\par 24 En het geschiede, als gij hoort het geruis van een gang in de toppen der moerbezienbomen, dan rep u; want alsdan is de HEERE voor uw aangezicht uitgegaan, om het heirleger der Filistijnen te slaan.
\par 25 En David deed alzo, gelijk als de HEERE hem geboden had; en hij sloeg de Filistijnen van Geba af, totdat gij komt te Gezer.

\chapter{6}

\par 1 Daarna verzamelde David wederom alle uitgelezenen in Israel, dertig duizend.
\par 2 En David maakte zich op, en ging heen met al het volk, dat bij hem was, van Baalim-juda, om van daar op te brengen de ark Gods, bij dewelke de Naam wordt aangeroepen, de Naam van den HEERE der heirscharen, Die daarop woont tussen de cherubim.
\par 3 En zij voerden de ark Gods op een nieuwen wagen, en haalden ze uit het huis van Abinadab, dat op een heuvel is; en Uza en Ahio, zonen van Abinadab, leidden den nieuwen wagen.
\par 4 Toen zij hem nu uit het huis van Abinadab, dat op den heuvel is, met de ark Gods, wegvoerden, zo ging Ahio voor de ark henen.
\par 5 En David en het ganse huis Israels speelden voor het aangezicht des HEEREN, met allerlei snarenspel van dennenhout, als met harpen, en met luiten, en met trommelen, ook met schellen, en met cimbalen.
\par 6 Als zij nu kwamen tot aan Nachons dorsvloer, zo strekte Uza zijn hand uit aan de ark Gods, en hield ze, want de runderen struikelden.
\par 7 Toen ontstak de toorn des HEEREN tegen Uza, en God sloeg hem aldaar, om deze onbedachtzaamheid; en hij stierf aldaar bij de ark Gods.
\par 8 En David ontstak, omdat de HEERE een scheur gescheurd had aan Uza; en hij noemde dezelve plaats Perez-uza, tot op dezen dag.
\par 9 En David vreesde den HEERE ten zelven dage; en hij zeide: Hoe zal de ark des HEEREN tot mij komen?
\par 10 David dan wilde de ark des HEEREN niet tot zich laten overbrengen in de stad Davids; maar David deed ze afwijken in het huis van Obed-edom, den Gethiet.
\par 11 En de ark des HEEREN bleef in het huis van Obed-edom, den Gethiet, drie maanden; en de HEERE zegende Obed-edom en zijn ganse huis.
\par 12 Toen boodschapte men den koning David, zeggende: De HEERE heeft het huis van Obed-edom, en al wat hij heeft, gezegend om der ark Gods wil; zo ging David heen en haalde de ark Gods uit het huis van Obed-edom opwaarts in de stad Davids, met vreugde.
\par 13 En het geschiedde, als zij, die de ark des HEEREN droegen, zes treden voortgetreden waren, dat hij ossen en gemest vee offerde.
\par 14 En David huppelde met alle macht voor het aangezicht des HEEREN; en David was omgord met een linnen lijfrok.
\par 15 Alzo brachten David en het ganse huis Israels de ark des HEEREN op, met gejuich en met geluid der bazuinen.
\par 16 En het geschiedde, als de ark des HEEREN in de stad Davids kwam, dat Michal, Sauls dochter, door het venster uitzag. Als zij nu den koning David zag, springende en huppelende voor het aangezicht des HEEREN, verachtte zij hem in haar hart.
\par 17 Toen zij nu de ark des HEEREN inbrachten, stelden zij die in haar plaats, in het midden der tent, die David voor haar gespannen had; en David offerde brandofferen voor des HEEREN aangezicht, en dankofferen.
\par 18 Als David geeindigd had het brandoffer en de dankofferen te offeren, zo zegende hij het volk in den Naam des HEEREN der heirscharen.
\par 19 En hij deelde uit aan het ganse volk, aan de ganse menigte van Israel, van de mannen tot de vrouwen toe, aan een iegelijk een broodkoek, en een schoon stuk vlees, en een fles wijn. Toen ging al dat volk heen, een iegelijk naar zijn huis.
\par 20 Als nu David wederkwam, om zijn huis te zegenen, ging Michal, Sauls dochter, uit, David tegemoet, en zeide: Hoe is heden de koning van Israel verheerlijkt, die zich heden voor de ogen van de dienstmaagden zijner dienstknechten heeft ontbloot, gelijk een van de ijdele lieden zich onbeschaamdelijk ontbloot?
\par 21 Maar David zeide tot Michal: Voor het aangezicht des HEEREN, Die mij verkoren heeft voor uw vader en voor zijn ganse huis, mij instellende tot een voorganger over het volk des HEEREN, over Israel; ja, ik zal spelen voor het aangezicht des HEEREN.
\par 22 Ook zal ik mij nog geringer houden dan alzo, en zal nederig zijn in mijn ogen, en met de dienstmaagden, waarvan gij gezegd hebt, met dezelve zal ik verheerlijkt worden.
\par 23 Michal nu, Sauls dochter, had geen kind, tot den dag van haar dood toe.

\chapter{7}

\par 1 En het geschiedde, als de koning in zijn huis zat, en de HEERE hem rust gegeven had van al zijn vijanden rondom,
\par 2 Zo zeide de koning tot den profeet Nathan: Zie toch, ik woon in een cederen huis, en de ark Gods woont in het midden der gordijnen.
\par 3 En Nathan zeide tot den koning: Ga heen, doe al wat in uw hart is, want de HEERE is met u.
\par 4 Maar het gebeurde in denzelfden nacht, dat het woord des HEEREN tot Nathan geschiedde, zeggende:
\par 5 Ga, en zeg tot Mijn knecht, tot David: Zo zegt de HEERE: Zoudt gij Mij een huis bouwen tot Mijn woning?
\par 6 Want Ik heb in geen huis gewoond, van dien dag af, dat Ik de kinderen Israels uit Egypte opvoerde, tot op dezen dag; maar Ik heb gewandeld in een tent en in een tabernakel.
\par 7 Overal, waar Ik met al de kinderen Israels heb gewandeld, heb Ik wel een woord gesproken met een der stammen Israels, dien Ik bevolen heb Mijn volk Israel te weiden, zeggende: Waarom bouwt gij Mij niet een cederen huis?
\par 8 Nu dan, alzo zult gij tot Mijn knecht, tot David, zeggen: Zo zegt de HEERE der heirscharen: Ik heb u genomen van de schaapskooi, van achter de schapen, dat gij een voorganger zoudt zijn over Mijn volk, over Israel.
\par 9 En Ik ben met u geweest, overal, waar gij gegaan zijt, en heb al uw vijanden voor uw aangezicht uitgeroeid; en Ik heb u een groten naam gemaakt, als den naam der groten, die op de aarde zijn.
\par 10 En Ik heb voor Mijn volk, voor Israel, een plaats besteld, en hem geplant, dat hij aan zijn plaats wone, en niet meer heen en weder gedreven worde; en de kinderen der verkeerdheid zullen hem niet meer verdrukken, gelijk als in het eerst.
\par 11 En van dien dag af, dat Ik geboden heb richters te wezen over Mijn volk Israel. Doch u heb Ik rust gegeven van al uw vijanden. Ook geeft u de HEERE te kennen, dat de HEERE u een huis maken zal.
\par 12 Wanneer uw dagen zullen vervuld zijn, en gij met uw vaderen zult ontslapen zijn, zo zal Ik uw zaad na u doen opstaan, dat uit uw lijf voortkomen zal, en Ik zal zijn koninkrijk bevestigen.
\par 13 Die zal Mijn Naam een huis bouwen; en Ik zal den stoel zijns koninkrijks bevestigen tot in eeuwigheid.
\par 14 Ik zal hem zijn tot een Vader, en hij zal Mij zijn tot een zoon; dewelke als hij misdoet, zo zal Ik hem met een mensenroede en met plagen der mensenkinderen straffen.
\par 15 Maar Mijn goedertierenheid zal van hem niet wijken, gelijk als Ik die weggenomen heb van Saul, dien Ik van voor uw aangezicht heb weggenomen.
\par 16 Doch uw huis zal bestendig zijn, en uw koninkrijk tot in eeuwigheid, voor uw aangezicht; uw stoel zal vast zijn tot in eeuwigheid.
\par 17 Naar al deze woorden, en naar dit ganse gezicht, alzo sprak Nathan tot David.
\par 18 Toen ging de koning David in, en bleef voor het aangezicht des HEEREN, en hij zeide: Wie ben ik, Heere HEERE, en wat is mijn huis, dat Gij mij tot hiertoe gebracht hebt?
\par 19 Daartoe is dit in Uw ogen nog klein geweest, Heere HEERE, maar Gij hebt ook over het huis Uws knechts gesproken tot van verre heen; en dit naar de wet der mensen, Heere HEERE!
\par 20 En wat zal David nog meer tot U spreken? Want Gij kent Uw knecht, Heere HEERE!
\par 21 Om Uws woords wil, en naar Uw hart hebt Gij al deze grote dingen gedaan, om aan Uw knecht bekend te maken.
\par 22 Daarom zijt Gij groot, HEERE God! Want er is niemand gelijk Gij, en er is geen God dan alleen Gij, naar alles, wat wij met onze oren gehoord hebben.
\par 23 En wie is, gelijk Uw volk, gelijk Israel, een enig volk op aarde, hetwelk God is heengegaan Zich tot een volk te verlossen, en om Zich een Naam te zetten, en om voor ulieden deze grote en verschrikkelijke dingen te doen aan Uw land, voor het aangezicht Uws volks, dat Gij U uit Egypte verlost hebt, de heidenen en hun goden verdrijvende.
\par 24 En Gij hebt Uw volk Israel U bevestigd, U tot een volk, tot in eeuwigheid; en Gij, HEERE, zijt hun tot een God geworden.
\par 25 Nu dan, HEERE God, doe dit woord, dat Gij over Uw knecht en over zijn huis gesproken hebt, bestaan tot in eeuwigheid, en doe, gelijk als Gij gesproken hebt.
\par 26 En Uw Naam worde groot gemaakt tot in eeuwigheid, dat men zegge: De HEERE der heirscharen is God over Israel; en het huis van Uw knecht David zal bestendig zijn voor Uw aangezicht.
\par 27 Want Gij, HEERE der heirscharen, Gij, God Israels! Gij hebt voor het oor Uws knechts geopenbaard, zeggende: Ik zal u een huis bouwen; daarom heeft Uw knecht in zijn hart gevonden, dit gebed tot U te bidden.
\par 28 Nu dan, Heere HEERE! Gij zijt die God, en Uw woorden zullen waarheid zijn, en Gij hebt dit goede tot Uw knecht gesproken.
\par 29 Zo believe het U nu, en zegen het huis van Uw knecht, dat het in eeuwigheid voor uw aangezicht zij; want Gij, Heere HEERE, hebt het gesproken, en met Uw zegen zal het huis van Uw knecht gezegend worden in eeuwigheid.

\chapter{8}

\par 1 En het geschiedde daarna, dat David de Filistijnen sloeg, en bracht hen ten onder; en David nam Meteg-amma uit der Filistijnen hand.
\par 2 Ook sloeg hij de Moabieten, en mat hen met een snoer, doende hen ter aarde nederliggen; en hij mat met twee snoeren om te doden, en met een vol snoer om in het leven te laten. Alzo werden de Moabieten David tot knechten, brengende geschenken.
\par 3 David sloeg ook Hadad-ezer, den zoon van Rechob, den koning van Zoba, toen hij heen toog, om zijn hand te wenden naar de rivier Frath.
\par 4 En David nam hem duizend wagens af, en zevenhonderd ruiteren, en twintig duizend man te voet; en David ontzenuwde alle wagenpaarden, en hield daarvan honderd wagenen over.
\par 5 En de Syriers van Damaskus kwamen om Hadad-ezer, den koning van Zoba, te helpen; maar David sloeg van de Syriers twee en twintig duizend man.
\par 6 En David leide bezettingen in Syrie van Damaskus, en de Syriers werden David tot knechten, brengende geschenken; en de HEERE behoedde David overal, waar hij heentoog.
\par 7 En David nam de gouden schilden die bij Hadad-ezers knechten geweest waren, en bracht ze te Jeruzalem.
\par 8 Daartoe nam de koning David zeer veel kopers uit Betach, en uit Berothai, steden van Hadad-ezer.
\par 9 Als nu Thoi, de koning van Hamath, hoorde, dat David het ganse heir van Hadad-ezer geslagen had;
\par 10 Zo zond Thoi zijn zoon Joram tot den koning David, om hem te vragen naar zijn welstand, en om hem te zegenen, vanwege dat hij tegen Hadad-ezer gekrijgd en hem geslagen had, (want Hadad-ezer voerde steeds krijg tegen Thoi); en in zijn hand waren zilveren vaten, en gouden vaten, en koperen vaten;
\par 11 Welke de koning David ook den HEERE heiligde, met het zilver en het goud, dat hij geheiligd had van alle heidenen, die hij zich onderworpen had;
\par 12 Van Syrie, en van Moab, en van de kinderen Ammons, en van de Filistijnen, en van Amalek, en van den roof van Hadad-ezer, den zoon van Rechob, den koning van Zoba.
\par 13 Ook maakte zich David een naam, als hij wederkwam, nadat hij de Syriers geslagen had, in het Zoutdal, achttien duizend.
\par 14 En hij leide bezettingen in Edom; in gans Edom leide hij bezettingen; en alle Edomieten werden David tot knechten; en de HEERE behoedde David overal, waar hij heentoog.
\par 15 Alzo regeerde David over gans Israel, en David deed aan zijn ganse volk recht en gerechtigheid.
\par 16 Joab nu, de zoon van Zeruja, was over het heir; en Josafat, zoon van Achilud, was kanselier.
\par 17 En Zadok, zoon van Ahitub, en Achimelech, zoon van Abjathar, waren priesters; en Seraja was schrijver.
\par 18 Er was ook Benaja, zoon van Jojada, met de Krethi en de Plethi; maar Davids zonen waren prinsen.

\chapter{9}

\par 1 En David zeide: Is er nog iemand die overgebleven is van het huis van Saul, dat ik weldadigheid aan hem doe, om Jonathans wil?
\par 2 Het huis van Saul nu had een knecht, wiens naam was Ziba; en zij riepen hem tot David. En de koning zeide tot hem: Zijt gij Ziba? En hij zeide: Uw knecht.
\par 3 En de koning zeide: Is er nog iemand van het huis van Saul, dat ik Gods weldadigheid bij hem doe? Toen zeide Ziba tot den koning: Er is nog een zoon van Jonathan, die geslagen is aan beide voeten.
\par 4 En de koning zeide tot hem: Waar is hij? En Ziba zeide tot den koning: Zie, hij is in het huis van Machir, den zoon van Ammiel, te Lodebar.
\par 5 Toen zond de koning David heen, en hij nam hem uit het huis van Machir, den zoon van Ammiel, van Lodebar.
\par 6 Als nu Mefiboseth, de zoon van Jonathan, den zoon van Saul, tot David inkwam, zo viel hij op zijn aangezicht, en boog zich neder. En David zeide: Mefiboseth! En hij zeide: Zie, hier is uw knecht.
\par 7 En David zeide tot hem: Vrees niet, want ik zal zekerlijk weldadigheid bij u doen, om uws vaders Jonathans wil; en ik zal u alle akkers van uw vader Saul wedergeven; en gij zult geduriglijk brood eten aan mijn tafel.
\par 8 Toen boog hij zich, en zeide: Wat is uw knecht, dat gij omgezien hebt naar een doden hond, als ik ben?
\par 9 Toen riep de koning Ziba, Sauls jongen, en zeide tot hem: Al wat Saul gehad heeft, en zijn ganse huis, heb ik den zoon uws heren gegeven.
\par 10 Daarom zult gij voor hem het land bearbeiden, gij, en uw zonen, en uw knechten, en zult de vruchten inbrengen, opdat de zoon uws heren brood hebbe, dat hij ete; en Mefiboseth, de zoon uws heren, zal geduriglijk brood eten aan mijn tafel. Ziba nu had vijftien zonen en twintig knechten.
\par 11 En Ziba zeide tot den koning: Naar alles, wat mijn heer de koning zijn knecht gebiedt, alzo zal uw knecht doen. Ook zou Mefiboseth, etende aan mijn tafel, als een van des konings zonen zijn.
\par 12 Mefiboseth nu had een kleinen zoon, wiens naam was Micha; en allen, die in het huis van Ziba woonden, waren knechten van Mefiboseth.
\par 13 Alzo woonde Mefiboseth te Jeruzalem, omdat hij geduriglijk at aan des konings tafel; en hij was kreupel aan beide zijn voeten.

\chapter{10}

\par 1 En het geschiedde daarna, dat de koning der kinderen Ammons stierf, en zijn zoon Hanun werd koning in zijn plaats.
\par 2 Toen zeide David: Ik zal weldadigheid doen aan Hanun, den zoon van Nahas, gelijk als zijn vader weldadigheid aan mij gedaan heeft. Zo zond David heen, om hem door den dienst zijner knechten te troosten over zijn vader. En de knechten van David kwamen in het land van de kinderen Ammons.
\par 3 Toen zeiden de vorsten der kinderen Ammons tot hun heer Hanun: Eert David uw vader in uw ogen, omdat hij troosters tot u gezonden heeft? Heeft David zijn knechten niet daarom tot u gezonden, dat hij deze stad doorzoeke, en die verspiede, en die omkere?
\par 4 Toen nam Hanun Davids knechten, en schoor hun baard half af, en sneed hun klederen half af, tot aan hun billen; en hij liet hen gaan.
\par 5 Als zij dit David lieten weten, zo zond hij hun tegemoet; want deze mannen waren zeer beschaamd. En de koning zeide: Blijft te Jericho, totdat uw baard weder gewassen zal zijn, komt dan weder.
\par 6 Toen nu de kinderen Ammons zagen, dat zij zich bij David stinkende gemaakt hadden, zonden de kinderen Ammons heen, en huurden van de Syriers van Beth-rechob, en van de Syriers van Zoba, twintig duizend voetvolks, en van den koning van Maacha duizend man, en van de mannen van Tob twaalf duizend man.
\par 7 Als David dit hoorde, zond hij Joab heen, en het ganse heir met de helden.
\par 8 En de kinderen Ammons togen uit, en stelden de slagorde voor de deur der poort; maar de Syriers van Zoba, en Rechob, en de mannen van Tob en Maacha waren bijzonder in het veld.
\par 9 Als nu Joab zag, dat de spits der slagorde tegen hem was, van voren en van achteren, zo verkoos hij uit alle uitgelezenen van Israel, en stelde hen in orde tegen de Syriers aan;
\par 10 En het overige des volks gaf hij onder de hand van zijn broeder Abisai, die het in orde stelde tegen de kinderen Ammons aan.
\par 11 En hij zeide: Zo de Syriers mij te sterk zullen zijn, zo zult gij mij komen verlossen; en zo de kinderen Ammons u te sterk zullen zijn, zo zal ik komen om u te verlossen.
\par 12 Wees sterk, en laat ons sterk zijn voor ons volk, en voor de steden onzes Gods; de HEERE nu doe, wat goed is in Zijn ogen.
\par 13 Toen naderde Joab, en het volk, dat bij hem was, tot den strijd tegen de Syriers; en zij vloden voor zijn aangezicht.
\par 14 Als de kinderen Ammons zagen, dat de Syriers vloden, vloden zij ook voor het aangezicht van Abisai, en kwamen in de stad. En Joab keerde weder van de kinderen Ammons, en kwam te Jeruzalem.
\par 15 Toen nu de Syriers zagen, dat zij voor Israels aangezicht geslagen waren, zo vergaderden zij zich weder te zamen.
\par 16 En Hadad-ezer zond heen, en deed de Syriers uitkomen, die op gene zijde der rivier zijn, en zij kwamen te Helam; en Sobach, Hadad-ezers krijgsoverste, toog voor hun aangezicht heen.
\par 17 Als dat David werd aangezegd, verzamelde hij gans Israel, en toog over de Jordaan, en kwam te Helam, en de Syriers stelden de slagorde tegen David aan, en streden met hem.
\par 18 Maar de Syriers vloden voor Israels aangezicht, en David versloeg van de Syriers zevenhonderd wagenen, en veertig duizend ruiteren; daartoe sloeg hij Sobach, hun krijgsoverste, dat hij aldaar stierf.
\par 19 Toen nu al de koningen, die Hadad-ezers knechten waren, zagen, dat zij voor Israels aangezicht geslagen waren, maakten zij vrede met Israel, en dienden hen; en de Syriers vreesden de kinderen Ammons meer te verlossen.

\chapter{11}

\par 1 En het geschiedde met de wederkomst van het jaar, ter tijde als de koningen uittrekken, dat David Joab, en zijn knechten met hem, en gans Israel henenzond, dat zij de kinderen Ammons verderven, en Rabba belegeren zouden. Doch David bleef te Jeruzalem.
\par 2 Zo geschiedde het tegen den avondtijd, dat David van zijn leger opstond, en wandelde op het dak van het koningshuis, en zag van het dak een vrouw, zich wassende; deze vrouw nu was zeer schoon van aanzien.
\par 3 En David zond henen, en ondervraagde naar deze vrouw; en men zeide: Is dat niet Bathseba, de dochter van Eliam, de huisvrouw van Uria, den Hethiet?
\par 4 Toen zond David boden henen, en liet haar halen. En als zij tot hem ingekomen was, lag hij bij haar, (zij nu had zich van haar onreinigheid gezuiverd), daarna keerde zij weder naar haar huis.
\par 5 En die vrouw werd zwanger; zo zond zij henen, en liet David weten, en zeide: Ik ben zwanger geworden.
\par 6 Toen zond David tot Joab, zeggende: Zend Uria, den Hethiet, tot mij. En Joab zond Uria tot David.
\par 7 Als nu Uria tot hem kwam, zo vraagde David naar den welstand van Joab, en naar den welstand des volks, en naar den welstand des krijgs.
\par 8 Daarna zeide David tot Uria: Ga af naar uw huis, en was uw voeten. En toen Uria uit des konings huis uitging, volgde hem een gerecht des konings achterna.
\par 9 Maar Uria leide zich neder voor de deur van des konings huis, met al de knechten zijns heren; en hij ging niet af in zijn huis.
\par 10 En zij gaven het David te kennen, zeggende: Uria is niet afgegaan in zijn huis. Toen zeide David tot Uria: Komt gij niet van de reis? Waarom zijt gij niet afgegaan in uw huis?
\par 11 En Uria zeide tot David: De ark, en Israel, en Juda blijven in de tenten; en mijn heer Joab, en de knechten mijns heren zijn gelegerd op het open veld, en zou ik in mijn huis gaan, om te eten en te drinken, en bij mijn huisvrouw te liggen? Zo waarachtig als gij leeft en uw ziel leeft, indien ik deze zaak doen zal!
\par 12 Toen zeide David tot Uria: Blijf ook heden hier, zo zal ik u morgen afzenden. Alzo bleef Uria te Jeruzalem, dien dag en den anderen dag.
\par 13 En David nodigde hem, zodat hij voor zijn aangezicht at en dronk, en hij maakte hem dronken. Daarna ging hij in den avond uit, om zich neder te leggen op zijn leger, met zijns heren knechten, maar ging niet af in zijn huis.
\par 14 Des morgens nu geschiedde het, dat David een brief schreef aan Joab; en hij zond dien door de hand van Uria.
\par 15 En hij schreef in dien brief, zeggende: Stel Uria vooraan tegenover den sterksten strijd, en keer van achter hem af, opdat hij geslagen worde en sterve.
\par 16 Zo geschiedde het, als Joab op de stad gelet had, dat hij Uria stelde aan de plaats, waarvan hij wist, dat aldaar strijdbare mannen waren.
\par 17 Als nu de mannen der stad uittogen en met Joab streden, vielen er van het volk, van Davids knechten, en Uria, de Hethiet, stierf ook.
\par 18 Toen zond Joab heen, en liet David den gansen handel van dezen strijd weten.
\par 19 En hij beval den bode, zeggende: Als gij zult geeindigd hebben den gansen handel van dezen strijd tot den koning uit te spreken;
\par 20 En het zal geschieden, indien de grimmigheid des konings opkomt, en hij tot u zegt: Waarom zijt gij zo na aan de stad gekomen om te strijden? Wist gij niet, dat zij van den muur zouden schieten?
\par 21 Wie sloeg Abimelech, den zoon van Jerubbeseth? Wierp niet een vrouw een stuk van een molensteen op hem van den muur, dat hij te Thebez stierf? Waarom zijt gij tot den muur genaderd? Dan zult gij zeggen: Uw knecht, Uria, de Hethiet, is ook dood.
\par 22 En de bode ging heen, en kwam in, en gaf David te kennen alles, waar hem Joab om uitgezonden had.
\par 23 En de bode zeide tot David: Die mannen zijn ons zeker te machtig geweest, en zijn tot ons uitgetogen in het veld; maar wij zijn tegen hen aan geweest tot aan de deur der poort.
\par 24 Toen schoten de schutters van den muur af op uw knechten, dat er van des konings knechten dood gebleven zijn; en uw knecht, Uria, de Hethiet, is ook dood.
\par 25 Toen zeide David tot den bode: Zo zult gij tot Joab zeggen: Laat deze zaak niet kwaad zijn in uw ogen, want het zwaard verteert zowel dezen als genen; versterk uw strijd tegen de stad, en verstoor ze; versterk hem alzo.
\par 26 Als nu de huisvrouw van Uria hoorde, dat haar man Uria dood was, zo droeg zij leed over haar heer.
\par 27 En als de rouw was overgegaan, zond David heen, en nam haar in zijn huis; en zij werd hem ter vrouwe, en baarde hem een zoon. Doch deze zaak, die David gedaan had, was kwaad in de ogen des HEEREN.

\chapter{12}

\par 1 En de HEERE zond Nathan tot David. Als die tot hem inkwam, zeide hij tot hem: Er waren twee mannen in een stad, de een rijk en de ander arm.
\par 2 De rijke had zeer veel schapen en runderen.
\par 3 Maar de arme had gans niet dan een enig klein ooilam, dat hij gekocht had, en had het gevoed, dat het groot geworden was bij hem, en bij zijn kinderen tegelijk; het at van zijn bete, en dronk van zijn beker, en sliep in zijn schoot, en het was hem als een dochter.
\par 4 Toen nu den rijken man een wandelaar overkwam, verschoonde hij te nemen van zijn schapen en van zijn runderen, om voor den reizenden man, die tot hem gekomen was, wat te bereiden; en hij nam des armen mans ooilam, en bereidde dat voor den man, die tot hem gekomen was.
\par 5 Toen ontstak Davids toorn zeer tegen dien man; en hij zeide tot Nathan: Zo waarachtig als de HEERE leeft, de man, die dat gedaan heeft, is een kind des doods!
\par 6 En dat ooilam zal hij viervoudig wedergeven, daarom dat hij deze zaak gedaan, en omdat hij niet verschoond heeft.
\par 7 Toen zeide Nathan tot David: Gij zijt die man! Zo zegt de HEERE, de God Israels: Ik heb u ten koning gezalfd over Israel, en Ik heb u uit Sauls hand gered;
\par 8 En Ik heb u uws heren huis gegeven, daartoe uws heren vrouwen in uw schoot, ja, Ik heb u het huis van Israel en Juda gegeven; en indien het weinig is, Ik zou u alzulks en alzulks daartoe doen.
\par 9 Waarom hebt gij dan het woord des HEEREN veracht, doende wat kwaad is in Zijn ogen? Gij hebt Uria, den Hethiet, met het zwaard verslagen, en zijn huisvrouw hebt gij u ter vrouwe genomen; en hem hebt gij met het zwaard van de kinderen Ammons doodgeslagen.
\par 10 Nu dan, het zwaard zal van uw huis niet afwijken tot in eeuwigheid; daarom dat gij Mij veracht hebt, en de huisvrouw van Uria, den Hethiet, genomen hebt, dat zij u ter vrouwe zij.
\par 11 Zo zegt de HEERE: Zie, Ik zal kwaad over u verwekken uit uw huis, en zal uw vrouwen nemen voor uw ogen, en zal haar aan uw naaste geven; die zal bij uw vrouwen liggen, voor de ogen dezer zon.
\par 12 Want gij hebt het in het verborgen gedaan; maar Ik zal deze zaak doen voor gans Israel, en voor de zon.
\par 13 Toen zeide David tot Nathan: Ik heb gezondigd tegen den HEERE! En Nathan zeide tot David: De HEERE heeft ook uw zonde weggenomen, gij zult niet sterven.
\par 14 Nochtans, dewijl gij door deze zaak de vijanden des HEEREN grotelijks hebt doen lasteren, zal ook de zoon, die u geboren is, den dood sterven.
\par 15 Toen ging Nathan naar zijn huis. En de HEERE sloeg het kind, dat de huisvrouw van Uria David gebaard had, dat het zeer krank werd.
\par 16 En David zocht God voor dat jongsken; en David vastte een vasten, en ging in, en lag den nacht over op de aarde.
\par 17 Toen maakten zich de oudsten van zijn huis op tot hem, om hem te doen opstaan van de aarde; maar hij wilde niet, en at geen brood met hen.
\par 18 En het geschiedde op den zevenden dag, dat het kind stierf; en Davids knechten vreesden hem aan te zeggen, dat het kind dood was, want zij zeiden: Ziet, als het kind nog levend was, spraken wij tot hem, maar hij hoorde naar onze stem niet, hoe zullen wij dan tot hem zeggen: Het kind is dood? Want het mocht kwaad doen.
\par 19 Maar David zag, dat zijn knechten mompelden; zo merkte David, dat het kind dood was. Dies zeide David tot zijn knechten: Is het kind dood? En zij zeiden: Het is dood.
\par 20 Toen stond David op van de aarde, en wies en zalfde zich, en veranderde zijn kleding, en ging in het huis des HEEREN, en bad aan; daarna kwam hij in zijn huis, en eiste brood; en zij zetten hem brood voor, en hij at.
\par 21 Zo zeiden zijn knechten tot hem: Wat is dit voor een ding, dat gij gedaan hebt? Om des levenden kinds wil hebt gij gevast en geweend; maar nadat het kind gestorven is, zijt gij opgestaan en hebt brood gegeten.
\par 22 En hij zeide: Als het kind nog leefde, heb ik gevast en geweend; want ik zeide: Wie weet, de HEERE zou mij mogen genadig zijn, dat het kind levend bleve.
\par 23 Maar nu is het dood, waarom zou ik nu vasten? Zal ik hem nog kunnen wederhalen? Ik zal wel tot hem gaan, maar hij zal tot mij niet wederkomen.
\par 24 Daarna troostte David zijn huisvrouw Bathseba, en ging tot haar in, en lag bij haar; en zij baarde een zoon, wiens naam zij noemde Salomo; en de HEERE had hem lief.
\par 25 En zond heen door de hand van den profeet Nathan, en noemde zijn naam Jedid-jah, om des HEEREN wil.
\par 26 Joab nu krijgde tegen Rabba der kinderen Ammons; en hij nam de koninklijke stad in.
\par 27 Toen zond Joab boden tot David, en zeide: Ik heb gekrijgd tegen Rabba, ook heb ik de waterstad ingenomen.
\par 28 Zo verzamel gij nu het overige des volks, en beleger de stad, en neem ze in; opdat niet, zo ik de stad zou innemen, mijn naam over haar uitgeroepen worde.
\par 29 Toen verzamelde David al dat volk, en toog naar Rabba; en hij krijgde tegen haar, en nam ze in.
\par 30 En hij nam de kroon haars konings van zijn hoofd af, welker gewicht was een talent gouds, met edelgesteente, en zij werd op Davids hoofd gezet; ook voerde hij uit een zeer groten roof der stad.
\par 31 Het volk nu, dat daarin was, voerde hij uit, en leide het onder zagen, en onder ijzeren dorswagens, en onder ijzeren bijlen, en deed hen door den ticheloven doorgaan; en alzo deed hij aan alle steden der kinderen Ammons. Daarna keerde David, en al het volk, weder naar Jeruzalem.

\chapter{13}

\par 1 En het geschiedde daarna, alzo Absalom, Davids zoon, een schone zuster had, welker naam was Thamar, dat Amnon, Davids zoon, haar lief kreeg.
\par 2 En Amnon was benauwd tot krank wordens toe, om zijner zuster Thamars wil; want zij was een maagd, zodat het in Amnons ogen zwaar was, haar iets te doen.
\par 3 Doch Amnon had een vriend, wiens naam was Jonadab, een zoon van Simea, Davids broeder; en Jonadab was een zeer wijs man.
\par 4 Die zeide tot hem: Waarom zijt gij van morgen tot morgen zo mager, gij koningszoon, zult gij het mij niet te kennen geven? Toen zeide Amnon tot hem: Ik heb Thamar, de zuster van mijn broeder Absalom, lief.
\par 5 En Jonadab zeide tot hem: Leg u op uw leger, en maak u krank; als dan uw vader zal komen om u te zien, zo zult gij tot hem zeggen: Dat toch mijn zuster Thamar kome, dat zij mij met brood spijzige, en de spijze voor mijn ogen toemake, opdat ik het aanzie, en van haar hand ete.
\par 6 Amnon dan leide zich, en maakte zich krank. Toen nu de koning kwam om hem te zien, zeide Amnon tot den koning: Dat toch mijn zuster Thamar kome, dat zij twee koekjes voor mijn ogen toemake, en ik van haar hand ete.
\par 7 Toen zond David heen tot Thamar in het huis, zeggende: Ga toch heen in het huis van uw broeder Amnon, en maak hem een spijze.
\par 8 En Thamar ging heen in het huis van haar broeder Amnon, (hij nu was nederliggende), en zij nam deeg, en kneedde het, en maakte koekjes toe voor zijn ogen, en bakte de koekjes.
\par 9 En zij nam een pan, en goot ze uit voor zijn aangezicht; maar hij weigerde te eten. En Amnon zeide: Doet alle man van mij uitgaan. En alle man ging van hem uit.
\par 10 Toen zeide Amnon tot Thamar: Breng de spijze in de kamer, dat ik van uw hand ete; zo nam Thamar de koekjes, die zij gemaakt had, en bracht ze haar broeder Amnon in de kamer.
\par 11 Als zij ze nu tot hem nabij bracht, dat hij ate, zo greep hij haar, en zeide tot haar: Kom, lig bij mij, mijn zuster!
\par 12 Maar zij zeide tot hem: Niet, mijn broeder, verkracht mij niet, want alzo doet men niet in Israel; doe deze dwaasheid niet.
\par 13 Want ik, waarhenen zou ik mijn schande brengen? En gij, gij zoudt zijn als een der dwazen in Israel; zo spreek toch nu tot den koning, want hij zal mij van u niet onthouden.
\par 14 Doch hij wilde naar haar stem niet horen; maar sterker zijnde dan zij, zo verkrachtte hij haar, en lag bij haar.
\par 15 Daarna haatte haar Amnon met een zeer groten haat; want de haat, waarmede hij haar haatte, was groter dan de liefde, waarmede hij haar had liefgehad; en Amnon zeide tot haar: Maak u op, ga weg.
\par 16 Toen zeide zij tot hem: Er zijn geen oorzaken om mij uit te drijven; dit kwaad zou groter zijn dan het andere, dat gij bij mij gedaan hebt; maar hij wilde naar haar niet horen.
\par 17 En hij riep zijn jongen, die hem diende, en zeide: Drijf nu deze van mij uit naar buiten, en grendel de deur achter haar toe.
\par 18 Zij nu had een veelvervigen rok aan; want alzo werden des konings dochteren, die maagden waren, met mantels gekleed; en zijn dienaar bracht haar uit tot buiten, en grendelde de deur achter haar toe.
\par 19 Toen nam Thamar as op haar hoofd, en scheurde den veelvervigen rok, dien zij aanhad; en zij leide haar hand op haar hoofd, en ging vast henen en kreet.
\par 20 En haar broeder Absalom zeide tot haar: Is uw broeder Amnon bij u geweest? Nu dan, mijn zuster, zwijg stil, hij is uw broeder; zet uw hart niet op deze zaak. Alzo bleef Thamar en was eenzaam in het huis van haar broeder Absalom.
\par 21 Als de koning David al deze dingen hoorde, zo ontstak hij zeer.
\par 22 Doch Absalom sprak niet met Amnon, noch kwaad noch goed; maar Absalom haatte Amnon, ter oorzake dat hij zijn zuster Thamar verkracht had.
\par 23 En het geschiedde, na twee volle jaren, dat Absalom, schaaps scheerders had te Baal-hazor, dat bij Efraim is; zo nodigde Absalom al des konings zonen.
\par 24 En Absalom kwam tot den koning, en zeide: Zie, nu heeft uw knecht schaaps scheerders; dat toch de koning en zijn knechten met uw knecht gaan.
\par 25 Maar de koning zeide tot Absalom: Niet, mijn zoon, laat ons toch niet al te zamen gaan, opdat wij u niet bezwaarlijk zijn; en hij hield bij hem aan, doch hij wilde niet gaan, maar zegende hem.
\par 26 Toen zeide Absalom: Zo niet, laat toch mijn broeder Amnon met ons gaan. Maar de koning zeide tot hem: Waarom zou hij met u gaan?
\par 27 Als Absalom bij hem aanhield, zo liet hij Amnon en al des konings zonen met hem gaan.
\par 28 Absalom nu gebood zijn jongens, zeggende: Let er nu op, als Amnons hart vrolijk is van den wijn, en ik tot ulieden zal zeggen: Slaat Amnon, dan zult gij hem doden; vreest niet; is het niet, omdat ik het u geboden heb? Zijt sterk en weest dapper.
\par 29 En Absaloms jongens deden aan Amnon, gelijk als Absalom geboden had. Toen stonden alle zonen des konings op, en reden een iegelijk op zijn muildier, en vloden.
\par 30 En het geschiedde, als zij op den weg waren, dat het gerucht tot David kwam, dat men zeide: Absalom heeft al de zonen des konings geslagen, en er is niet een van hen overgelaten.
\par 31 Toen stond de koning op, en scheurde zijn klederen, en leide zich neder ter aarde; desgelijks stonden al zijn knechten met gescheurde klederen.
\par 32 Maar Jonadab, de zoon van Simea, Davids broeder, antwoordde en zeide: Mijn heer zegge niet, dat zij al de jongelingen, des konings zonen, gedood hebben; maar Amnon alleen is dood; want bij Absalom is er op toegeleid, van den dag af, dat hij zijn zuster Thamar verkracht heeft.
\par 33 Zo neme nu mijn heer de koning de zaak niet in zijn hart, denkende: al des konings zonen zijn dood; want Amnon alleen is dood.
\par 34 Absalom nu vluchtte; en de jongen, die de wacht hield, hief zijn ogen op, en zag toe, en ziet, er kwam veel volks van den weg achter hem, aan de zijde van het gebergte.
\par 35 Toen zeide Jonadab tot den koning: Zie, de zonen des konings komen; naar het woord uws knechts, alzo is het geschied.
\par 36 En het geschiedde, als hij geeindigd had te spreken, ziet, zo kwamen de zonen des konings, en hieven hun stemmen op en weenden; en de koning ook en al zijn knechten weenden met een zeer groot geween.
\par 37 (Absalom dan vluchtte, en toog tot Thalmai, den zoon van Ammihur, koning van Gesur.) En hij droeg rouw over zijn zoon, al die dagen.
\par 38 Alzo vluchtte Absalom, en toog naar Gesur; en hij was aldaar drie jaren.
\par 39 Toen verlangde de ziel van den koning David zeer om naar Absalom uit te trekken; want hij had zich getroost over Amnon, dat hij dood was.

\chapter{14}

\par 1 Als nu Joab, de zoon van Zeruja, merkte, dat des konings hart over Absalom was;
\par 2 Zo zond Joab heen naar Thekoa, en nam van daar een wijze vrouw; en hij zeide tot haar: Stel u toch, alsof gij rouw droegt, en trek nu rouwklederen aan, en zalf u niet met olie, en wees als een vrouw, die nu vele dagen rouw gedragen heeft over een dode;
\par 3 En ga in tot den koning, en spreek tot hem naar dit woord. En Joab leide de woorden in haar mond.
\par 4 En de Thekoietische vrouw zeide tot den koning, als zij op haar aangezicht ter aarde was gevallen, en zich nedergebogen had, zo zeide zij: Behoud, o koning!
\par 5 En de koning zeide tot haar: Wat is u? En zij zeide: Zekerlijk, ik ben een weduwvrouw, en mijn man is gestorven.
\par 6 Nu had uw dienstmaagd twee zonen, en deze beiden twistten in het veld, en er was geen scheider tussen hen; zo sloeg de een den ander, en doodde hem.
\par 7 En zie, het ganse geslacht is opgestaan tegen uw dienstmaagd, en hebben gezegd: Geef dien hier, die zijn broeder geslagen heeft, dat wij hem voor de ziel zijns broeders, dien hij doodgeslagen heeft, doden, en ook den erfgenaam verdelgen; alzo zullen zij mijn kool, die overgebleven is, uitblussen, opdat zij mijn man geen naam noch overblijfsel laten op den aardbodem.
\par 8 Toen zeide de koning tot deze vrouw: Ga naar uw huis, en ik zal voor u gebieden.
\par 9 En de Thekoietische vrouw zeide tot den koning: Mijn heer koning, de ongerechtigheid zij op mij en op mijns vaders huis; de koning daarentegen, en zijn stoel, zij onschuldig.
\par 10 En de koning zeide: Spreekt iemand tegen u, zo breng hem tot mij; en hij zal u voortaan niet meer aantasten.
\par 11 En zij zeide: De koning gedenke toch aan den HEERE, uw God, dat de bloedwrekers niet te vele worden om te verderven, dat zij mijn zoon niet verdelgen. Toen zeide hij: Zo waarachtig als de HEERE leeft, indien er een van de haren uws zoons op de aarde zal vallen!
\par 12 Toen zeide deze vrouw: Laat toch uw dienstmaagd een woord tot mijn heer den koning spreken. En hij zeide: Spreek.
\par 13 En de vrouw zeide: Waarom hebt gij dan alzulks tegen Gods volk gedaan? Want daaruit, dat de koning dit woord gesproken heeft, is hij als een schuldige, dewijl de koning zijn verstotene niet wederhaalt.
\par 14 Want wij zullen den dood sterven, en wezen als water, dat, ter aarde uitgestort zijnde, niet verzameld wordt. God dan zal de ziel niet wegnemen, maar Hij zal gedachten denken, dat Hij den verstotene niet van Zich verstote.
\par 15 Nu dan, dat ik gekomen ben, om ditzelve woord tot den koning, mijn heer, te spreken, is omdat het volk mij vreesachtig gemaakt heeft; zo zeide uw dienstmaagd: Ik zal nu tot den koning spreken; misschien zal de koning het woord zijner dienstmaagd doen.
\par 16 Want de koning zal horen, om zijn dienstmaagd te redden van de hand des mans, die voorheeft mij en mijn zoon te zamen van Gods erve te verdelgen.
\par 17 Wijders zeide uw dienstmaagd: Het woord mijns heren, des konings, zij toch tot rust; want gelijk een Engel Gods, alzo is mijn heer de koning, om te horen het goede en het kwade; en de HEERE, uw God, zal met u zijn.
\par 18 Toen antwoordde de koning, en zeide tot de vrouw: Verberg nu niet voor mij de zaak, die ik u vragen zal. En de vrouw zeide: Mijn heer de koning spreke toch.
\par 19 En de koning zeide: Is Joabs hand met u in dit alles? En de vrouw antwoordde en zeide: Zo waarachtig als uw ziel leeft, mijn heer koning, indien iemand ter rechter-of ter linkerhand zou kunnen afwijken van alles, wat mijn heer de koning gesproken heeft; want uw knecht Joab heeft het mij geboden, en die heeft al deze woorden in den mond uwer dienstmaagd gelegd;
\par 20 Dat ik de gestalte dezer zaak alzo omwenden zou, zulks heeft uw knecht Joab gedaan; doch mijn heer is wijs, naar de wijsheid van een Engel Gods, om te merken alles, wat op de aarde is.
\par 21 Toen zeide de koning tot Joab: Zie nu, ik heb deze zaak gedaan; zo ga henen, haal den jongeling Absalom weder.
\par 22 Toen viel Joab op zijn aangezicht ter aarde, en boog zich, en dankte den koning; en Joab zeide: Heden heeft uw knecht gemerkt, dat ik genade gevonden heb in uw ogen, mijn heer koning! Omdat de koning het woord van zijn knecht gedaan heeft.
\par 23 Alzo maakte zich Joab op, en toog naar Gesur; en hij bracht Absalom te Jeruzalem.
\par 24 En de koning zeide: Dat hij in zijn huis kere, en mijn aangezicht niet zie. Alzo keerde Absalom in zijn huis, en zag des konings aangezicht niet.
\par 25 Nu was er in gans Israel geen man zo schoon als Absalom, zeer te prijzen; van zijn voetzool af tot zijn hoofdschedel toe was er geen gebrek in hem.
\par 26 En als hij zijn hoofd beschoor, (nu geschiedde het ten einde van elk jaar, dat hij het beschoor, omdat het hem te zwaar was, zo beschoor hij het), zo woog het haar zijns hoofds tweehonderd sikkelen, naar des konings gewicht.
\par 27 Ook werden Absalom drie zonen geboren, en een dochter, welker naam was Thamar; deze was een vrouw, schoon van aanzien.
\par 28 Alzo bleef Absalom twee volle jaren te Jeruzalem, dat hij des konings aangezicht niet zag.
\par 29 Daarom zond Absalom tot Joab, dat hij hem tot den koning zond; maar hij wilde niet tot hem komen. Zo zond hij nog ten anderen male; evenwel wilde hij niet komen.
\par 30 Zo zeide hij tot zijn knechten: Ziet, het stuk akkers van Joab is aan de zijde van het mijne, en hij heeft gerst daarop; gaat heen, en steekt het aan met vuur, en Absaloms knechten staken dat stuk akkers aan met vuur.
\par 31 Toen maakte zich Joab op en kwam tot Absalom in het huis, en zeide tot hem: Waarom hebben uw knechten het stuk akkers, dat mijn is, met vuur aangestoken?
\par 32 En Absalom zeide tot Joab: Zie, ik heb tot u gezonden, zeggende: Kom herwaarts, dat ik u tot den koning zende, om te zeggen: Waarom ben ik van Gesur gekomen? Het ware mij goed, dat ik nog daar ware; nu dan, laat mij het aangezicht des konings zien; is er dan nog een misdaad in mij, zo dode hij mij.
\par 33 Toen ging Joab in tot den koning, en zeide het hem aan. Toen riep hij Absalom, en hij kwam tot den koning in, en boog zich voor hem op zijn aangezicht ter aarde, voor des konings aangezicht; en de koning kuste Absalom.

\chapter{15}

\par 1 En het geschiedde daarna, dat Absalom zich liet bereiden wagenen en paarden, en vijftig mannen, lopende voor zijn aangezicht henen.
\par 2 Ook maakte zich Absalom des morgens vroeg op, en stond aan de zijde van den weg der poort. En het geschiedde, dat Absalom allen man, die een geschil had, om tot den koning ten gerichte te komen, tot zich riep, en zeide: Uit welke stad zijt gij? Als hij dan zeide: Uw knecht is uit een der stammen Israels;
\par 3 Zo zeide Absalom tot hem: Zie, uw zaken zijn goed en recht; maar gij hebt geen verhoorder van des konings wege.
\par 4 Voorts zeide Absalom: Och, dat men mij ten rechter stelde in het land! Dat alle man tot mij kwame, die een geschil of rechtzaak heeft, dat ik hem recht sprake.
\par 5 Het geschiedde ook, als iemand naderde, om zich voor hem te buigen, zo reikte hij zijn hand uit, en greep hem, en kuste hem.
\par 6 En naar die wijze deed Absalom aan gans Israel, die tot den koning ten gerichte kwamen. Alzo stal Absalom het hart der mannen van Israel.
\par 7 Ten einde nu van veertig jaren is het geschied, dat Absalom tot den koning zeide: Laat mij toch heengaan, en mijn gelofte, die ik den HEERE beloofd heb, te Hebron betalen.
\par 8 Want uw knecht heeft een gelofte beloofd, als ik te Gesur in Syrie woonde, zeggende: Indien de HEERE mij zekerlijk weder te Jeruzalem zal brengen, zo zal ik den HEERE dienen.
\par 9 Toen zeide de koning tot hem: Ga in vrede. Alzo maakte hij zich op, en ging naar Hebron.
\par 10 Absalom nu had verspieders uitgezonden in alle stammen van Israel, om te zeggen: Als gij het geluid der bazuin zult horen, zo zult gij zeggen: Absalom is koning te Hebron.
\par 11 En er gingen met Absalom van Jeruzalem tweehonderd mannen, genodigd zijnde, doch gaande in hun eenvoudigheid, want zij wisten van geen zaak.
\par 12 Absalom zond ook om Achitofel, den Giloniet, Davids raad, uit zijn stad, uit Gilo te halen, als hij offeranden offerde. En de verbintenis werd sterk, en het volk kwam toe en vermeerderde bij Absalom.
\par 13 Toen kwam er een boodschapper tot David, zeggende: Het hart van een iegelijk in Israel volgt Absalom na.
\par 14 Zo zeide David tot al zijn knechten, die met hem te Jeruzalem waren: Maakt u op, en laat ons vlieden, want er zou voor ons geen ontkomen zijn voor Absaloms aangezicht; haast u, om weg te gaan, opdat hij niet misschien haaste, en ons achterhale, en een kwaad over ons drijve, en deze stad sla met de scherpte des zwaards.
\par 15 Toen zeiden de knechten des konings tot den koning: Naar alles, wat mijn heer de koning verkiezen zal, ziet, hier zijn uw knechten.
\par 16 En de koning ging uit met zijn ganse huis te voet; doch de koning liet tien bijwijven, om het huis te bewaren.
\par 17 Als nu de koning met al het volk te voet was uitgegaan, zo bleven zij staan in een verre plaats.
\par 18 En al zijn knechten gingen aan zijn zijde heen, ook al de Krethi en al de Plethi, en al de Gethieten, zeshonderd man, die van Gath te voet gekomen waren, gingen voor des konings aangezicht heen.
\par 19 Zo zeide de koning tot Ithai, den Gethiet: Waarom zoudt gij ook met ons gaan? Keer weder, en blijf bij den koning; want gij zijt vreemd, en ook zult gij weder vertrekken naar uw plaats.
\par 20 Gisteren zijt gij gekomen, en heden zou ik u met ons omvoeren om te gaan? Zo ik toch gaan moet, waarheen ik gaan kan, keer weder; en breng uw broederen wederom; weldadigheid en trouw zij met u.
\par 21 Maar Ithai antwoordde den koning, en zeide: Zo waarachtig als de HEERE leeft, en mijn heer de koning leeft, in de plaats, waar mijn heer de koning zal zijn, hetzij ten dode, hetzij ten leven, daar zal uw knecht voorzeker ook zijn!
\par 22 Toen zeide David tot Ithai: Zo kom, en ga over. Alzo ging Ithai, de Gethiet, over, en al zijn mannen, en al de kinderen die met hem waren.
\par 23 En het ganse land weende met luider stem, als al het volk overging; ook ging de koning over de beek Kidron, en al het volk ging over, recht naar den weg der woestijn.
\par 24 En ziet, Zadok was ook daar, en al de Levieten met hem, dragende de ark des verbonds van God, en zij zetten de ark Gods neder; en Abjathar klom op, totdat al het volk uit de stad geeindigd had over te gaan.
\par 25 Toen zeide de koning tot Zadok: Breng de ark Gods weder in de stad; indien ik genade zal vinden in des HEEREN ogen, zo zal Hij mij wederhalen, en zal ze mij laten zien, mitsgaders Zijn woning.
\par 26 Maar indien Hij alzo zal zeggen: Ik heb geen lust tot u; zie, hier ben ik, Hij doe mij, zo als het in Zijn ogen goed is.
\par 27 Voorts zeide de koning tot den priester Zadok: Zijt gij niet een ziener? Keer weder in de stad met vrede; ook ulieder beide zonen, Ahimaaz, uw zoon, en Jonathan, Abjathars zoon, met u.
\par 28 Zie, ik zal vertoeven in de vlakke velden der woestijn, totdat er een woord van ulieden kome, dat men mij aanzegge.
\par 29 Alzo bracht Zadok, en Abjathar, de ark Gods weder te Jeruzalem, en zij bleven aldaar.
\par 30 En David ging op door den opgang der olijven, opgaande en wenende, en het hoofd was hem bewonden; en hij zelf ging barrevoets; ook had al het volk, dat met hem was, een iegelijk zijn hoofd bedekt, en zij gingen op, opgaande en wenende.
\par 31 Toen gaf men David te kennen, zeggende: Achitofel is onder degenen, die zich met Absalom hebben verbonden. Dies zeide David: O, HEERE! maak toch Achitofels raad tot zotheid.
\par 32 En het geschiedde, als David tot op de hoogte kwam, dat hij aldaar God aanbad; ziet, toen ontmoette hem Husai, de Archiet, hebbende zijn rok gescheurd, en aarde op zijn hoofd.
\par 33 En David zeide tot hem: Zo gij met mij voortgaat, zo zult gij mij tot een last zijn;
\par 34 Maar zo gij weder in de stad gaat, en tot Absalom zegt: Uw knecht, ik zal des konings zijn; ik ben wel uws vaders knecht van te voren geweest, maar nu zal ik uw knecht zijn; zo zoudt gij mij den raad van Achitofel te niet maken.
\par 35 En zijn niet Zadok en Abjathar, de priesters, aldaar met u? Zo zal het geschieden, dat gij alle ding, dat gij uit des konings huis zult horen, den priesteren, Zadok en Abjathar, zult te kennen geven.
\par 36 Ziet, hun beide zonen zijn aldaar bij hen, Ahimaaz, Zadoks, en Jonathan, Abjathars zoon; zo zult gijlieden door hun hand tot mij zenden alle ding, dat gij zult horen.
\par 37 Alzo kwam Husai, Davids vriend, in de stad; en Absalom kwam te Jeruzalem.

\chapter{16}

\par 1 Als nu David een weinig van de hoogte was voortgegaan, ziet, toen ontmoette hem Ziba, Mefiboseths jongen, met een paar gezadelde ezelen, en daarop tweehonderd broden, met honderd stukken rozijnen, en honderd stukken zomervruchten, en een lederen zak wijns.
\par 2 En de koning zeide tot Ziba: Wat zult gij daarmede? En Ziba zeide: De ezels zijn voor het huis des konings, om op te rijden en het brood en de zomervruchten, om te eten voor de jongens; en de wijn, opdat de moeden in de woestijn drinken.
\par 3 Toen zeide de koning: Waar is dan de zoon uws heren? En Ziba zeide tot den koning: Zie, hij blijft te Jeruzalem, want hij zeide: Heden zal mij het huis Israels mijns vaders koninkrijk wedergeven.
\par 4 Zo zeide de koning tot Ziba: Zie, het zal het uwe zijn alles wat Mefiboseth heeft. En Ziba zeide: Ik buig mij neder, laat mij genade vinden in uw ogen, mijn heer koning!
\par 5 Als nu de koning David tot aan Bahurim kwam, ziet, toen kwam van daar een man uit, van het geslacht van het huis van Saul, wiens naam was Simei, de zoon van Gera; hij ging steeds voort, en vloekte.
\par 6 En hij wierp David met stenen, mitsgaders alle knechten van den koning David, hoewel al het volk en al de helden aan zijn rechter-en aan zijn linkerhand waren.
\par 7 Aldus nu zeide Simei in zijn vloeken: Ga uit, ga uit, gij, man des bloeds, en gij, Belials man!
\par 8 De HEERE heeft op u doen wederkomen al het bloed van Sauls huis, in wiens plaats gij geregeerd hebt; nu heeft de HEERE het koninkrijk gegeven in de hand van Absalom, uw zoon; zie nu, gij zijt in uw ongeluk, omdat gij een man des bloeds zijt.
\par 9 Toen zeide Abisai, de zoon van Zeruja, tot den koning: Waarom zou deze dode hond mijn heer den koning vloeken? Laat mij toch overgaan en zijn kop wegnemen.
\par 10 Maar de koning zeide: Wat heb ik met u te doen, gij zonen van Zeruja? Ja, laat hem vloeken; want de HEERE toch heeft tot hem gezegd: Vloek David; wie zou dan zeggen: Waarom hebt gij alzo gedaan?
\par 11 Voorts zeide David tot Abisai en tot al zijn knechten: Ziet, mijn zoon, die van mijn lijf is voortgekomen, zoekt mijn ziel; hoeveel te meer dan nu deze zoon van Jemini? Laat hem geworden, dat hij vloeke, want de HEERE heeft het hem gezegd.
\par 12 Misschien zal de HEERE mijn ellende aanzien; en de HEERE zal mij goed vergelden voor zijn vloek, te dezen dage.
\par 13 Alzo ging David met zijn lieden op den weg; en Simei ging al voort langs de zijde des bergs tegen hem over, en vloekte, en wierp met stenen van tegenover hem, en stoof met stof.
\par 14 En de koning kwam in, en al het volk, dat met hem was, moede zijnde; en hij verkwikte zich aldaar.
\par 15 Absalom nu en al het volk, de mannen van Israel, kwamen te Jeruzalem, en Achitofel met hem.
\par 16 En het geschiedde, als Husai, de Archiet, Davids vriend, tot Absalom kwam, dat Husai tot Absalom zeide: De koning leve, de koning leve!
\par 17 Maar Absalom zeide tot Husai: Is dit uw weldadigheid aan uw vriend? Waarom zijt gij niet met uw vriend getogen?
\par 18 En Husai zeide tot Absalom: Neen, maar welken de HEERE verkiest, en al dit volk, en alle mannen van Israel, diens zal ik zijn, en bij hem zal ik blijven.
\par 19 En ten andere, wien zou ik dienen? Zou het niet zijn voor het aangezicht zijns zoons? Gelijk als ik voor het aangezicht uws vaders gediend heb, alzo zal ik voor uw aangezicht zijn.
\par 20 Toen zeide Absalom tot Achitofel: Geeft onder ulieden raad, wat zullen wij doen?
\par 21 En Achitofel zeide tot Absalom: Ga in tot de bijwijven uws vaders, die hij gelaten heeft om het huis te bewaren; zo zal gans Israel horen, dat gij bij uw vader stinkende zijt geworden, en de handen van allen, die met u zijn, zullen gesterkt worden.
\par 22 Zo spanden zij Absalom een tent op het dak; en Absalom ging in tot de bijwijven zijns vaders, voor de ogen van het ganse Israel.
\par 23 En in die dagen was Achitofels raad, dien hij raadde, als of men naar Gods woord gevraagd had; alzo was alle raad van Achitofel, zo bij David als bij Absalom.

\chapter{17}

\par 1 Voorts zeide Achitofel tot Absalom: Laat mij nu twaalf duizend mannen uitlezen, dat ik mij opmake en David dezen nacht achterna jage.
\par 2 Zo zal ik over hem komen, daar hij moede en slap van handen is, en zal hem verschrikken, en al het volk, dat met hem is, zal vluchten; dan zal ik den koning alleen slaan.
\par 3 En ik zal al het volk tot u doen wederkeren; de man, dien gij zoekt, is gelijk het wederkeren van allen; zo zal al het volk in vrede zijn.
\par 4 Dit woord nu was recht in Absaloms ogen, en in de ogen van alle oudsten Israels.
\par 5 Doch Absalom zeide: Roep toch ook Husai, den Archiet, en laat ons horen, wat hij ook zegt.
\par 6 En als Husai tot Absalom inkwam, zo sprak Absalom tot hem, zeggende: Aldus heeft Achitofel gesproken; zullen wij zijn woord doen? Zo niet, spreek gij.
\par 7 Toen zeide Husai tot Absalom: De raad, dien Achitofel op ditmaal geraden heeft, is niet goed.
\par 8 Wijders zeide Husai: Gij kent uw vader en zijn mannen, dat zij helden zijn, dat zij bitter van gemoed zijn, als een beer, die van de jongen beroofd is in het veld; daartoe is uw vader een krijgsman, en zal niet vernachten met het volk.
\par 9 Zie, nu heeft hij zich verstoken in een der holen, of in een der plaatsen. En het zal geschieden, als er in het eerst sommigen onder hen vallen, dat een ieder, die het zal horen, alsdan zal zeggen: Er is een slag geschied onder het volk, dat Absalom navolgt.
\par 10 Zo zou hij, die ook een dapper man is, wiens hart is als een leeuwenhart, te enen male smelten; want gans Israel weet, dat uw vader een held is, en het dappere mannen zijn, die met hem zijn.
\par 11 Maar ik rade, dat in alle haast tot u verzameld worde gans Israel, van Dan tot Ber-seba toe, als zand, dat aan de zee is, in menigte; en dat uw persoon medega in den strijd.
\par 12 Dan zullen wij tot hem komen, in een der plaatsen, waar hij gevonden wordt, en hem gemakkelijk overvallen, gelijk als de dauw op den aardbodem valt; en er zal van hem, en van al de mannen, die met hem zijn, ook niet een worden overgelaten.
\par 13 En indien hij zich in een stad zal begeven, zo zal gans Israel koorden tot dezelve stad aandragen, en wij zullen ze tot in de beek nedertrekken, totdat ook niet een steentje aldaar gevonden worde.
\par 14 Toen zeide Absalom, en alle man van Israel: De raad van Husai, den Archiet, is beter dan Achitofels raad. Doch de HEERE had het geboden, om den goeden raad van Achitofel te vernietigen, opdat de HEERE het kwaad over Absalom bracht.
\par 15 En Husai zeide tot Zadok en tot Abjathar, de priesters: Alzo en alzo heeft Achitofel Absalom en den oudsten van Israel geraden, maar alzo en alzo heb ik geraden.
\par 16 Nu dan, zendt haastelijk henen, en boodschapt David, zeggende: Vernacht dezen nacht niet in de vlakke velden der woestijn, en ook ga spoedig over; opdat de koning niet verslonden worde, en al het volk, dat met hem is.
\par 17 Jonathan nu en Ahimaaz stonden bij de fontein Rogel; en een dienstmaagd ging henen en zeide het hun aan; en zij gingen henen en zeiden het den koning David aan; want zij mochten zich niet zien laten, dat zij in de stad kwamen.
\par 18 Een jongen dan nog zag hen, en zeide het Absalom aan; doch die beiden gingen haastelijk, en kwamen in eens mans huis te Bahurim, dewelke een put had in zijn voorhof, en zij daalden daarin.
\par 19 En de vrouw nam en spreidde een deksel over het opene van den put, en strooide gort daarop. Alzo werd de zaak niet bekend.
\par 20 Toen nu Absaloms knechten tot de vrouw in het huis kwamen, zeiden zij: Waar zijn Ahimaaz en Jonathan? En de vrouw zeide tot hen: Zij zijn over dat waterriviertje gegaan. En toen zij hen gezocht en niet gevonden hadden, keerden zij weder naar Jeruzalem.
\par 21 En het geschiedde, nadat zij weggegaan waren, zo klommen zij uit den put, en gingen henen en boodschapten het den koning David; en zij zeiden tot David: Maakt ulieden op, en gaat haastelijk over het water, want alzo heeft Achitofel tegen ulieden geraden.
\par 22 Toen maakte zich David op, en al het volk, dat met hem was; en zij gingen over de Jordaan. Aan het morgenlicht ontbrak er niet tot een toe, die niet over de Jordaan gegaan was.
\par 23 Als nu Achitofel zag, dat zijn raad niet gedaan was, zadelde hij den ezel, en maakte zich op, en toog naar zijn huis in zijn stad, en gaf bevel aan zijn huis, en verhing zich. Alzo stierf hij, en werd begraven in zijns vaders graf.
\par 24 David nu kwam te Mahanaim, en Absalom toog over de Jordaan, hij en alle mannen van Israel met hem.
\par 25 En Absalom had Amasa in Joabs plaats gesteld over het heir. Amasa nu was eens mans zoon, wiens naam was Jethra, de Israeliet, die ingegaan was tot Abigail, dochter van Nahas, zuster van Zeruja, Joabs moeder.
\par 26 Israel nu en Absalom legerden zich in het land van Gilead.
\par 27 En het geschiedde, als David te Mahanaim gekomen was, dat Sobi, de zoon van Nahas, van Rabba der kinderen Ammons, en Machir, de zoon van Ammiel, van Lodebar, en Barzillai, de Gileadiet, van Rogelim,
\par 28 Beddewerk, en schalen, en aarden vaten, en tarwe, en gerst, en meel, en geroost koren, en bonen, en linzen, ook geroost,
\par 29 En honig, en boter, en schapen, en koeienkazen, brachten tot David, en tot het volk, dat met hem was, om te eten, want zij zeiden: Dit volk is hongerig, en moede, en dorstig in de woestijn.

\chapter{18}

\par 1 En David monsterde het volk, dat met hem was; en hij stelde over hen oversten van duizenden, en oversten van honderden.
\par 2 Voorts zond David het volk uit, een derde deel onder de hand van Joab, en een derde deel onder de hand van Abisai, den zoon van Zeruja, Joabs broeder, en een derde deel onder de hand van Ithai, den Gethiet. En de koning zeide tot het volk: Ik zal ook zelf zekerlijk met ulieden uittrekken.
\par 3 Maar het volk zeide: Gij zult niet uittrekken; want of wij te enen male vloden, zij zullen het hart op ons niet stellen; ja, of de helft van ons stierf, zij zullen het hart op ons niet stellen; maar gij zijt nu als tien duizend onzer. Zo zal het nu beter zijn, dat gij ons uit de stad ter hulpe zijt.
\par 4 Toen zeide de koning tot hen: Ik zal doen, wat goed is in uw ogen. De koning nu stond aan de zijde van de poort, en al het volk trok uit bij honderden en bij duizenden.
\par 5 En de koning gebood Joab, en Abisai, en Ithai, zeggende: Handelt mij zachtkens met den jongeling, met Absalom. En al het volk hoorde het, als de koning aan al de oversten van Absaloms zaak gebood.
\par 6 Alzo toog het volk uit in het veld, Israel tegemoet, en de strijd geschiedde bij Efraims woud.
\par 7 En het volk van Israel werd aldaar voor het aangezicht van Davids knechten geslagen; en aldaar geschiedde te dienzelven dage een grote slag, van twintig duizend.
\par 8 Want de strijd werd aldaar verspreid over al dat land. En het woud verteerde meer van het volk, dan die het zwaard verteerde, te dienzelven dage.
\par 9 Absalom nu ontmoette voor het aangezicht der knechten Davids; en Absalom reed op een muildier; en als het muildier kwam onder de dichte takken van een groten eik, zo werd zijn hoofd vast aan den eik, dat hij hangen bleef tussen den hemel en tussen de aarde, en het muildier, dat onder hem was, ging door.
\par 10 Als dat een man zag, zo gaf hij het Joab te kennen, en zeide: Zie, ik heb Absalom zien hangen aan een eik.
\par 11 Toen zeide Joab tot den man, die het hem te kennen gaf: Zie toch, gij hebt het gezien, waarom dan hebt gij hem niet aldaar ter aarde geslagen, alzo het aan mij stond om u tien zilverlingen en een gordel te geven?
\par 12 Maar die man zeide tot Joab: En of ik al duizend zilverlingen op mijn handen mocht wegen, zo zou ik mijn hand aan des konings zoon niet slaan; want de koning heeft u, en Abisai, en Ithai, voor onze oren geboden, zeggende: Hoedt u, wie gij zijt, van den jongeling, van Absalom.
\par 13 Of ik al valselijk tegen mijn ziel handelde, zo zou toch geen ding voor den koning verborgen worden; ook gij zelf zoudt er u van tegenover stellen.
\par 14 Toen zeide Joab: Ik zal hier bij u alzo niet vertoeven; en hij nam drie pijlen, en stak ze in Absaloms hart, daar hij nog levend was in het midden van den eik.
\par 15 En tien jongens, wapendragers van Joab, omringden hem, en zij sloegen Absalom, en doodden hem.
\par 16 Toen blies Joab met de bazuin, en al het volk keerde af van Israel achterna te jagen, want Joab hield het volk terug.
\par 17 En zij namen Absalom, en wierpen hem in het woud, in een groten kuil, en stelden op hem een zeer groten steenhoop; en gans Israel vluchtte, een iegelijk naar zijn tent.
\par 18 Absalom nu had genomen, en in zijn leven voor zich opgericht een pilaar, die in het koningsdal is; want hij zeide: Ik heb geen zoon, om aan mijn naam te doen gedenken; en hij had dien pilaar genoemd naar zijn naam; daarom wordt hij tot op dezen dag genoemd: Absaloms hand.
\par 19 Toen zeide Ahimaaz, Zadoks zoon: Laat mij toch heenlopen, en den koning boodschappen, dat de HEERE hem recht gedaan heeft van de hand zijner vijanden.
\par 20 Maar Joab zeide tot hem: Gij zult dezen dag geen boodschapper zijn, maar op een anderen dag zult gij boodschappen; dezen dag nu zult gij niet boodschappen, daarom dat des konings zoon dood is.
\par 21 En Joab zeide tot Cuschi: Ga heen, en zeg den koning aan, wat gij gezien hebt; en Cuschi boog zich voor Joab, en liep heen.
\par 22 Doch Ahimaaz, Zadoks zoon, voer nog voort en zeide tot Joab: Wat het ook zij, laat mij toch ook Cuschi achterna lopen. En Joab zeide: Waarom zoudt gij nu heenlopen, mijn zoon! Zo gij toch geen bekwame boodschap hebt?
\par 23 Wat het ook zij, zeide hij, laat mij heenlopen; zo zeide hij tot hem: Loop heen. En Ahimaaz liep den weg van het effen veld, en kwam Cuschi voorbij.
\par 24 David nu zat tussen de twee poorten; en de wachter ging op het dak der poort aan den muur, en hief zijn ogen op, en zag, en ziet, er liep een man alleen.
\par 25 Zo riep de wachter, en zeide het den koning aan; en de koning zeide: Indien hij alleen is, zo is er een boodschap in zijn mond; en hij ging al voort en naderde.
\par 26 Toen zag de wachter een anderen man lopende, en de wachter riep tot den poortier en zeide: Zie, er loopt nog een man alleen. Toen zeide de koning: Die is ook een boodschapper.
\par 27 Voorts zeide de wachter: Ik zie den loop des eersten aan, als den loop van Ahimaaz, Zadoks zoon. Toen zeide de koning: Dat is een goed man, en hij zal met een goede boodschap komen.
\par 28 Ahimaaz dan riep en zeide tot den koning Vrede! En hij boog zich voor den koning met het aangezicht ter aarde, en hij zeide: Geloofd zij de HEERE, uw God, Die de mannen, dewelke hun hand tegen mijn heer den koning ophieven, heeft overgegeven.
\par 29 Toen zeide de koning: Is het wel met den jongeling, met Absalom? En Ahimaaz zeide: Ik zag een groot rumoer, als Joab, den knecht des konings, en mij uw knecht afzond, maar ik weet niet wat.
\par 30 En de koning zeide: Ga om, stel u hier; zo ging hij om, en bleef staan.
\par 31 En ziet, Cuschi kwam aan; en Cuschi zeide: Mijn heer den koning wordt geboodschapt, dat u de HEERE heden heeft recht gedaan van de hand van al degenen, die tegen u opstonden.
\par 32 Toen zeide de koning tot Cuschi: Is het wel met den jongeling, met Absalom? En Cuschi zeide: De vijanden van mijn heer den koning, en allen, die tegen u ten kwade opstaan, moeten worden als die jongeling.
\par 33 Toen werd de koning zeer beroerd, en ging op naar de opperzaal der poort, en weende; en in zijn gaan zeide hij alzo: Mijn zoon Absalom, mijn zoon, mijn zoon Absalom! Och, dat ik, ik voor u gestorven ware, Absalom, mijn zoon, mijn zoon!

\chapter{19}

\par 1 En Joab werd aangezegd: Zie, de koning weent, en bedrijft rouw over Absalom.
\par 2 Toen werd de verlossing te dienzelven dage het ganse volk tot rouw; want het volk had te dienzelven dage horen zeggen: Het smart den koning over zijn zoon.
\par 3 En het volk kwam te dienzelven dage steelsgewijze in de stad, gelijk als het volk zich wegsteelt, dat beschaamd is, wanneer zij in den strijd gevloden zijn.
\par 4 De koning nu had zijn aangezicht toegewonden, en de koning riep met luider stem: Mijn zoon Absalom, Absalom, mijn zoon, mijn zoon!
\par 5 Toen kwam Joab tot den koning in het huis, en zeide: Gij hebt heden beschaamd het aangezicht van al uw knechten, die uw ziel, en de ziel uwer zonen en uwer dochteren, en de ziel uwer vrouwen, en de ziel uwer bijwijven heden hebben bevrijd;
\par 6 Liefhebbende die u haten, en hatende die u liefhebben; want gij geeft heden te kennen, dat oversten en knechten bij u niets zijn; want ik merk heden, dat zo Absalom leefde, en wij heden allen dood waren, dat het alsdan recht zou zijn in uw ogen.
\par 7 Zo sta nu op, ga uit, en spreek naar het hart uwer knechten; want ik zweer bij den HEERE, als gij niet uitgaat, zo er een man dezen nacht bij u zal vernachten! En dit zal u kwader zijn, dan al het kwaad, dat over u gekomen is van uw jeugd af tot nu toe.
\par 8 Toen stond de koning op, en zette zich in de poort. En zij lieten al het volk weten, zeggende: Ziet, de koning zit in de poort. Toen kwam al het volk voor des konings aangezicht, maar Israel was gevloden, een iegelijk naar zijn tenten.
\par 9 En al het volk, in alle stammen van Israel, was onder zich twistende, zeggende: De koning heeft ons gered van de hand onzer vijanden en hij heeft ons bevrijd van de hand der Filistijnen, en nu is hij uit het land gevlucht voor Absalom;
\par 10 En Absalom, die wij over ons gezalfd hadden, is in den strijd gestorven; nu dan, waarom zwijgt gijlieden van den koning weder te halen?
\par 11 Toen zond de koning David tot Zadok en tot Abjathar, de priesteren, zeggende: Spreekt tot de oudsten van Juda, zeggende: Waarom zoudt gijlieden de laatsten zijn, om den koning weder te halen in zijn huis? (Want de rede van het ganse Israel was tot den koning gekomen in zijn huis.)
\par 12 Gij zijt mijn broederen; mijn been en mijn vlees zijt gij; waarom zoudt gij dan de laatsten zijn, om den koning weder te halen?
\par 13 En tot Amasa zult gijlieden zeggen: Zijt gij niet mijn been en mijn vlees? God doe mij zo, en doe er zo toe, zo gij niet krijgsoverste zult zijn voor mijn aangezicht, te allen dage, in Joabs plaats.
\par 14 Alzo neigde hij het hart aller mannen van Juda, als van een enigen man; en zij zonden henen tot den koning, zeggende: Keer weder, gij en al uw knechten.
\par 15 Toen keerde de koning weder, en kwam tot aan de Jordaan; en Juda kwam te Gilgal, om den koning tegemoet te gaan, dat zij den koning over de Jordaan voerden.
\par 16 En Simei, de zoon van Gera, een zoon van Jemini, die van Bahurim was, haastte zich, en kwam af met de mannen van Juda, den koning David tegemoet;
\par 17 En duizend man van Benjamin met hem; ook Ziba, de knecht van Sauls huis, en zijn vijftien zonen en zijn twintig knechten met hem; en zij togen vaardiglijk over de Jordaan, voor den koning.
\par 18 Als nu de pont overvoer, om het huis des konings over te halen, en te doen, wat goed was in zijn ogen, zo viel Simei, de zoon van Gera, neder voor het aangezicht des konings, als hij over de Jordaan voer;
\par 19 En hij zeide tot den koning: Mijn heer rekene mij niet toe de misdaad, en gedenke niet, wat uw knecht verkeerdelijk gedaan heeft, te dien dage, als mijn heer de koning uit Jeruzalem uitging, dat het de koning zich ter harte zoude nemen.
\par 20 Want uw knecht weet het zekerlijk, ik heb gezondigd; doch zie, ik ben heden gekomen, de eerste van het ganse huis van Jozef, om mijn heer den koning tegemoet af te komen.
\par 21 Toen antwoordde Abisai, de zoon van Zeruja, en zeide: Zou dan Simei hiervoor niet gedood worden? Zo hij toch den gezalfde des HEEREN gevloekt heeft.
\par 22 Maar David zeide: Wat heb ik met ulieden te doen, gij zonen van Zeruja! Dat gij mij heden ten satan zoudt zijn? Zou heden iemand gedood worden in Israel? Want weet ik niet, dat ik heden koning geworden ben over Israel?
\par 23 En de koning zeide tot Simei: Gij zult niet sterven. En de koning zwoer hem.
\par 24 Mefiboseth, Sauls zoon, kwam ook af den koning tegemoet; en hij had zijn voeten niet schoon gemaakt, noch zijn knevelbaard beschoren, noch zijn klederen gewassen, van dien dag af, dat de koning was weggegaan, tot dien dag toe, dat hij met vrede wederkwam.
\par 25 En het geschiedde, als hij te Jeruzalem den koning tegemoet kwam, dat de koning tot hem zeide: Waarom zijt gij niet met mij getogen, Mefiboseth?
\par 26 En hij zeide: Mijn heer koning, mijn knecht heeft mij bedrogen; want uw knecht zeide: Ik zal mij een ezel zadelen, en daarop rijden, en tot den koning trekken, want uw knecht is kreupel.
\par 27 Daartoe heeft hij uw knecht bij mijn heer den koning valselijk aangedragen; doch mijn heer de koning is als een engel Gods; doe dan, wat goed is in uw ogen.
\par 28 Want al mijns vaders huis is niet geweest, dan maar lieden des doods voor mijn heer den koning; nochtans hebt gij uw knecht gezet onder degenen, die aan uw tafel eten; wat heb ik dan meer voor gerechtigheid, en meer te roepen aan den koning?
\par 29 Toen zeide de koning tot hem: Waarom spreekt gij meer van uw zaken? Ik heb gezegd: Gij en Ziba, deelt het land.
\par 30 En Mefiboseth zeide tot den koning: Hij neme het ook gans weg, naardien mijn heer de koning met vrede in zijn huis is gekomen.
\par 31 Barzillai, de Gileadiet, kwam ook af van Rogelim; en hij toog met den koning over de Jordaan, om hem over de Jordaan te geleiden.
\par 32 Barzillai nu was zeer oud, een man van tachtig jaren; en hij had den koning onderhouden, toen hij te Mahanaim zijn verblijf had; want hij was een zeer groot man.
\par 33 En de koning zeide tot Barzillai: Trekt gij met mij over, en ik zal u bij mij te Jeruzalem onderhouden.
\par 34 Maar Barzillai zeide tot den koning: Hoe veel zullen de dagen der jaren mijns levens zijn, dat ik met den koning zou optrekken naar Jeruzalem?
\par 35 Ik ben heden tachtig jaren oud; zou ik kunnen onderscheiden tussen goed en kwaad? Zou uw knecht kunnen smaken, wat ik eet en wat ik drink? Zoude ik meer kunnen horen naar de stem der zangers en zangeressen? En waarom zou uw knecht mijn heer den koning verder tot een last zijn?
\par 36 Uw knecht zal maar een weinig met den koning over de Jordaan gaan; waarom toch zou mij de koning zulk een vergelding doen?
\par 37 Laat toch uw knecht wederkeren, dat ik sterve in mijn stad, bij het graf mijns vaders en mijner moeder; maar zie, daar is uw knecht Chimham, laat dien met mijn heer den koning overtrekken, en doe hem, wat goed is in uw ogen.
\par 38 Toen zeide de koning: Chimham zal met mij overtrekken, en ik zal hem doen, wat goed is in uw ogen; ja, alles, wat gij op mij begeren zult, zal ik u doen.
\par 39 Toen nu al het volk over de Jordaan gegaan was, en de koning ook was overgegaan, kuste de koning Barzillai, en zegende hem; alzo keerde hij weder naar zijn plaats.
\par 40 En de koning toog voort naar Gilgal, en Chimham toog met hem voort; en al het volk van Juda had den koning overgevoerd, als ook een gedeelte van het volk Israels.
\par 41 En ziet, alle mannen van Israel kwamen tot den koning; en zij zeiden tot den koning: Waarom hebben u onze broeders, de mannen van Juda, gestolen, en hebben den koning en zijn huis over de Jordaan gevoerd, en alle mannen Davids met hem?
\par 42 Toen antwoordden alle mannen van Juda tegen de mannen van Israel: Omdat de koning ons na verwant is; en waarom zijt gij nu toornig over deze zaak? Hebben wij dan enigszins gegeten van des konings kost, of heeft hij ons een geschenk geschonken?
\par 43 En de mannen van Israel antwoordden den mannen van Juda, en zeiden: Wij hebben tien delen aan den koning, en ook aan David, wij, meer dan gij; waarom hebt gij ons dan gering geacht, dat ons woord niet het eerste geweest is, om onzen koning weder te halen? Maar het woord der mannen van Juda was harder dan het woord der mannen van Israel.

\chapter{20}

\par 1 Toen was daar bij geval een Belials man, wiens naam was Seba, een zoon van Bichri, een man van Jemini; die blies met de bazuin, en zeide: Wij hebben geen deel aan David, en wij hebben geen erfenis aan den zoon van Isai, een iegelijk naar zijn tenten, o Israel!
\par 2 Toen toog alle man van Israel op van achter David, Seba, den zoon van Bichri, achterna; maar de mannen van Juda kleefden hun koning aan, van de Jordaan af tot aan Jeruzalem.
\par 3 Toen nu David in zijn huis te Jeruzalem kwam, nam de koning de tien vrouwen, zijn bijwijven, die hij gelaten had, om het huis te bewaren, en deed ze in een huis van bewaring, en onderhield ze, maar ging tot haar niet in. En zij waren opgesloten tot op den dag van haarlieder dood, levende als weduwen.
\par 4 Voorts zeide de koning tot Amasa: Roep mij de mannen van Juda te zamen, tegen den derden dag; en gij, stel u dan hier.
\par 5 En Amasa ging heen, om Juda bijeen te roepen; maar hij bleef achter, boven den gezetten tijd, dien hij hem gezet had.
\par 6 Toen zeide David tot Abisai: Nu zal ons Seba, de zoon van Bichri, meer kwaads doen, dan Absalom; neem gij de knechten uws heren, en jaag hem achterna, opdat hij niet misschien vaste steden voor zich vinde, en zich aan onze ogen onttrekke.
\par 7 Toen togen uit, hem achterna, de mannen van Joab, en de Krethi, en de Plethi, en al de helden. Dezen togen uit van Jeruzalem, om Seba, den zoon van Bichri, achterna te jagen.
\par 8 Als zij nu waren bij den groten steen, die bij Gibeon is, zo kwam Amasa voor hun aangezicht. En Joab was omgord over zijn kleed, dat hij aan had, en daarop was een gordel, daar het zwaard aan vastgemaakt was op zijn lenden in zijn schede; en als hij voortging, zo viel het uit.
\par 9 En Joab zeide tot Amasa: Is het wel met u, mijn broeder? En Joab vatte met de rechterhand den baard van Amasa, om hem te kussen.
\par 10 En Amasa hoedde zich niet voor het zwaard, dat in Joabs hand was; zo sloeg hij hem daarmede aan de vijfde rib, en hij stortte zijn ingewand ter aarde uit, en hij sloeg hem niet ten tweeden male, en hij stierf. Toen jaagden Joab en zijn broeder Abisai, Seba, den zoon van Bichri, achterna.
\par 11 Maar een man, van Joabs jongens, bleef bij hem staan, en hij zeide: Wie is er, die lust heeft aan Joab, en wie is er, die voor David is, die volge Joab na!
\par 12 Amasa nu lag in het bloed gewenteld, midden op de straat. Als die man zag, dat al het volk staan bleef, zo deed hij Amasa weg van de straat in het veld, en wierp een kleed op hem, dewijl hij zag, dat al wie bij hem kwam, bleef staan.
\par 13 Toen hij nu van de straat weggenomen was, toog alle man voort, Joab na, om Seba, den zoon van Bichri, achterna te jagen.
\par 14 En hij toog heen door alle stammen van Israel, naar Abel, te weten, Beth-maacha, en het ganse Berim; en zij verzamelden zich, en kwamen hem ook na.
\par 15 En zij kwamen en belegerden hem in Abel Beth-maacha, en zij wierpen een wal op tegen de stad, dat hij aan den buitenmuur stond; en al het volk, dat met Joab was, verdorven den muur, om dien neder te vellen.
\par 16 Toen riep een wijze vrouw uit de stad: Hoort, hoort, zegt toch tot Joab: Nader tot hiertoe, dat ik tot u spreke.
\par 17 Toen hij nu tot haar naderde, zeide de vrouw: Zijt gij Joab? En hij zeide: Ik ben het; en zij zeide tot hem: Hoor de woorden uwer dienstmaagd; en hij zeide: Ik hoor.
\par 18 Toen sprak zij, zeggende: In voortijden spraken zij gemeenlijk, zeggende: Zij zullen zonder twijfel te Abel vragen; en alzo volbrachten zij het.
\par 19 Ik ben een van de vreedzamen, van de getrouwen in Israel, en gij zoekt te doden een stad, die een moeder is in Israel; waarom zoudt gij het erfdeel des HEEREN verslinden?
\par 20 Toen antwoordde Joab, en zeide: Het zij verre, het zij verre van mij, dat ik zou verslinden, en dat ik zou verderven!
\par 21 De zaak is niet alzo; maar een man van het gebergte van Efraim, wiens naam is Seba, de zoon van Bichri, heeft zijn hand opgeheven tegen den koning, tegen David; lever hem alleen, zo zal ik van deze stad aftrekken. Toen zeide de vrouw tot Joab: Zie, zijn hoofd zal tot u over den muur geworpen worden.
\par 22 En de vrouw kwam in tot al het volk, met haar wijsheid; en zij hieuwen Seba, den zoon van Bichri, het hoofd af, en wierpen het tot Joab. Toen blies hij met de bazuin, en zij verstrooiden zich van de stad, een iegelijk naar zijn tenten; en Joab keerde weder naar Jeruzalem tot den koning.
\par 23 Joab nu was over het ganse heir van Israel; en Benaja, de zoon van Jojada, over de Krethi en over de Plethi;
\par 24 En Adoram was over de schatting; en Josafat, de zoon van Ahilud, was kanselier;
\par 25 En Seja was schrijver; en Zadok en Abjathar waren priesters.
\par 26 En ook was Ira, de Jairiet, Davids opperofficier.

\chapter{21}

\par 1 En er was in Davids dagen een honger, drie jaren, jaar achter jaar; en David zocht het aangezicht des HEEREN. En de HEERE zeide: Het is om Saul en om des bloedhuizes wil, omdat hij de Gibeonieten gedood heeft.
\par 2 Toen riep de koning de Gibeonieten, en zeide tot hen: (De Gibeonieten nu waren niet van de kinderen Israels, maar van het overblijfsel der Amorieten; en de kinderen Israels hadden hun gezworen, maar Saul zocht hen te slaan in zijn ijver voor de kinderen van Israel en Juda.)
\par 3 David dan zeide tot de Gibeonieten: Wat zal ik ulieden doen, en waarmede zal ik verzoenen, dat gij het erfdeel des HEEREN zegent?
\par 4 Toen zeiden de Gibeonieten tot hem: Het is ons niet te doen om zilver en goud met Saul en met zijn huis; ook is het ons niet om iemand te doden in Israel. En hij zeide: Wat zegt gij dan, dat ik u doen zal?
\par 5 En zij zeiden tot den koning: De man die ons te niet gemaakt, en tegen ons gedacht heeft, dat wij zouden verdelgd worden, zonder te kunnen bestaan in enige landpale van Israel;
\par 6 Laat ons zeven mannen van zijn zonen gegeven worden, dat wij hen den HEERE ophangen te Gibea Sauls, o, gij verkorene des HEEREN! En de koning zeide: Ik zal hen geven.
\par 7 Doch de koning verschoonde Mefiboseth, den zoon van Jonathan, den zoon van Saul, om den eed des HEEREN, die tussen hen was, tussen David en tussen Jonathan, Sauls zoon.
\par 8 Maar de koning nam de twee zonen van Rizpa, dochter van Aja, die zij Saul gebaard had, Armoni en Mefiboseth; daartoe de vijf zonen van Michals zuster, Sauls dochter, die zij Adriel, den zoon van Barzillai, den Meholathiet, gebaard had;
\par 9 En hij gaf hen in de hand der Gibeonieten, die ze ophingen op den berg voor het aangezicht des HEEREN; en die zeven vielen tegelijk; en zij werden gedood in de dagen van den oogst, in de eerste dagen, in het begin van den gersteoogst.
\par 10 Toen nam Rizpa, de dochter van Aja, een zak, en spande dien voor zich uit op een rotssteen, van het begin van den oogst, totdat er water op hen drupte van den hemel; en zij liet het gevogelte des hemels op hen niet rusten des daags, noch het gedierte van het veld des nachts.
\par 11 En het werd David aangezegd, wat Rizpa, de dochter van Aja, Sauls bijwijf, gedaan had.
\par 12 Zo ging David henen, en nam de beenderen van Saul, en de beenderen van Jonathan, zijn zoon, van de burgeren van Jabes in Gilead, die dezelve gestolen hadden van de straat Beth-san, alwaar de Filistijnen ze hadden opgehangen, ten dage als de Filistijnen Saul sloegen op Gilboa.
\par 13 En hij bracht van daar op de beenderen van Saul, en de beenderen van Jonathan, zijn zoon; ook verzamelden zij de beenderen der gehangenen.
\par 14 En zij begroeven de beenderen van Saul en zijn zoon Jonathan in het land van Benjamin te Zela, in het graf van zijn vader Kis, en deden alles, wat de koning geboden had. Alzo werd God na dezen den lande verbeden.
\par 15 Voorts hadden de Filistijnen nog een krijg tegen Israel. En David toog af, en zijn knechten met hem, en streden tegen de Filistijnen, dat David moede werd.
\par 16 En Isbi Benob, die van de kinderen van Rafa was, en het gewicht zijner spies driehonderd gewicht kopers, en hij was aangegord met een nieuw zwaard; deze dacht David te slaan.
\par 17 Maar Abisai, de zoon van Zeruja, hielp hem, en sloeg den Filistijn, en doodde hem. Toen zwoeren hem de mannen van David, zeggende: Gij zult niet meer met ons uittrekken ten strijde, opdat gij de lamp van Israel niet uitblust.
\par 18 En het geschiedde daarna, dat er wederom een krijg was te Gob tegen de Filistijnen. Toen sloeg Sibbechai, de Husathiet, Saf, die van de kinderen van Rafa was.
\par 19 Voorts was er nog een krijg te Gob tegen de Filistijnen; en Elhanan, de zoon van Jaare-oregim, sloeg Beth-halachmi, dewelke was met Goliath, den Gethiet, wiens spiesenhout was als een weversboom.
\par 20 Nog was er ook een krijg te Gath; en er was een zeer lang man, die zes vingeren had aan zijn handen, en zes tenen aan zijn voeten, vier en twintig in getal, en deze was ook aan Rafa geboren.
\par 21 En hij hoonde Israel; maar Jonathan, de zoon van Simea, Davids broeder, sloeg hem.
\par 22 Deze vier waren aan Rafa geboren te Gath; en zij vielen door de hand van David, en door de hand zijner knechten.

\chapter{22}

\par 1 En David sprak de woorden dezes lieds tot den HEERE, ten dage als de HEERE hem verlost had uit de hand van al zijn vijanden, en uit de hand van Saul.
\par 2 Hij zeide dan: De HEERE is mij mijn Steenrots, en mijn Burg, en mijn Uithelper.
\par 3 God is mijn Rots, ik zal op Hem betrouwen; mijn Schild en de Hoorn mijns heils, mijn Hoog Vertrek en mijn Toevlucht, mijn Verlosser! Van geweld hebt Gij mij verlost!
\par 4 Ik riep den HEERE aan, Die te prijzen is, en ik werd verlost van mijn vijanden.
\par 5 Want baren des doods hadden mij omvangen; beken Belials verschrikten mij.
\par 6 Banden der hel omringden mij; strikken des doods bejegenden mij.
\par 7 Als mij bange was, riep ik den HEERE aan, en riep tot mijn God; en Hij hoorde mijn stem uit Zijn paleis, en mijn geroep kwam in Zijn oren.
\par 8 Toen daverde en beefde de aarde; de fondamenten des hemels beroerden zich, en daverden, omdat Hij ontstoken was.
\par 9 Rook ging op van Zijn neus, en een vuur uit Zijn mond verteerde; kolen werden daarvan aangestoken.
\par 10 En Hij boog den hemel, en daalde neder; en donkerheid was onder Zijn voeten.
\par 11 En Hij voer op een cherub, en vloog, en werd gezien op de vleugelen des winds.
\par 12 En Hij zette duisternis rondom Zich tot tenten, een samenbinding der wateren, wolken des hemels.
\par 13 Van den glans voor Hem henen werden kolen des vuurs aangestoken.
\par 14 De HEERE donderde van den hemel, en de Allerhoogste gaf Zijn stem.
\par 15 En Hij zond pijlen uit en verstrooide ze; bliksemen en verschrikte ze.
\par 16 En de diepe kolken der zee werden gezien, de gronden der wereld werden ontdekt, door het schelden des HEEREN, van het geblaas des winds van Zijn neus.
\par 17 Hij zond van de hoogte, Hij nam mij, Hij trok mij op uit grote wateren.
\par 18 Hij verloste mij van mijn sterken vijand, van mijn haters, omdat zij machtiger waren dan ik.
\par 19 Zij hadden mij bejegend ten dage mijns ongevals; maar de HEERE was mij een Steunsel.
\par 20 En Hij voerde mij uit in de ruimte, en rukte mij uit, want Hij had lust aan mij.
\par 21 De HEERE vergold mij naar mijn gerechtigheid; Hij gaf mij weder naar de reinigheid mijner handen.
\par 22 Want ik heb des HEEREN wegen gehouden, en ben van mijn God niet goddelooslijk afgegaan.
\par 23 Want al Zijn rechten waren voor mij, en Zijn inzettingen, daarvan week ik niet af.
\par 24 Maar ik was oprecht voor Hem; en ik wachtte mij voor mijn ongerechtigheid.
\par 25 Zo gaf mij de HEERE weder naar mijn gerechtigheid, naar mijn reinigheid, voor Zijn ogen.
\par 26 Bij den goedertierene houdt Gij U goedertieren; bij den oprechten held houdt Gij U oprecht.
\par 27 Bij den reine houdt Gij U rein; maar bij den verkeerde houdt Gij U verdraaid.
\par 28 En Gij verlost het bedrukte volk; maar Uw ogen zijn tegen de hogen, Gij zult hen vernederen.
\par 29 Want Gij zijt mijn Lamp, o HEERE, en de HEERE doet mijn duisternis opklaren.
\par 30 Want met U loop ik door een bende; met mijn God spring ik over een muur.
\par 31 Gods weg is volmaakt; de rede des HEEREN is doorlouterd; Hij is een Schild allen, die op Hem betrouwen.
\par 32 Want wie is God, behalve de HEERE, en wie is een rotssteen, behalve onze God?
\par 33 God is mijn Sterkte en Kracht; en Hij heeft mijn weg volkomen geopend.
\par 34 Hij maakt mijn voeten gelijk als der hinden, en stelt mij op mijn hoogten.
\par 35 Hij leert mijn handen ten strijde, zodat een stalen boog met mijn armen verbroken is.
\par 36 Ook hebt Gij mij gegeven het schild Uws heils, en door Uw verootmoedigen hebt Gij mij groot gemaakt.
\par 37 Gij hebt mijn voetstap ruim gemaakt onder mij; en mijn enkelen hebben niet gewankeld.
\par 38 Ik vervolgde mijn vijanden, en verdelgde hen, en keerde niet weder, totdat ik ze verdaan had.
\par 39 En ik verteerde hen, en doorstak ze, dat zij niet weder opstonden; maar zij vielen onder mijn voeten.
\par 40 Want Gij omgorddet mij met kracht ten strijde; Gij deedt onder mij nederbukken, die tegen mij opstonden.
\par 41 En Gij gaaft mij den nek mijner vijanden, mijner haters, en ik vernielde hen.
\par 42 Zij zagen uit, maar er was geen verlosser; naar den HEERE, maar Hij antwoordde hun niet.
\par 43 Toen vergruisde ik hen als stof der aarde; ik stampte ze, ik breidde hen uit als slijk der straten.
\par 44 Ook hebt Gij mij uitgeholpen van de twisten mijns volks, Gij hebt mij bewaard tot een hoofd der heidenen; het volk, dat ik niet kende, heeft mij gediend.
\par 45 Vreemden hebben zich mij geveinsdelijk onderworpen; zo haast als hun oor van mij hoorde, hebben zij mij gehoorzaamd.
\par 46 Vreemden zijn vervallen, en hebben zich aangegord uit hun sloten.
\par 47 De HEERE leeft, en geloofd zij mijn Rotssteen; en verhoogd zij God, de Rotssteen mijns heils!
\par 48 De God, Die mij volkomene wraak geeft, en de volken onder mij nederwerpt;
\par 49 En Die mij uitvoert van mijn vijanden; en Gij verhoogt mij boven degenen, die tegen mij opstaan; Gij redt mij van den man alles gewelds.
\par 50 Daarom zal ik U, o HEERE, loven onder de heidenen, en Uw Naam zal ik psalmzingen.
\par 51 Hij is een Toren der verlossingen Zijns konings, en Hij doet goedertierenheid aan Zijn gezalfde, aan David en aan zijn zaad, tot in eeuwigheid.

\chapter{23}

\par 1 Voorts zijn dit de laatste woorden van David. David, de zoon van Isai zegt, en de man, die hoog is opgericht, de gezalfde van Jakobs God, en liefelijk in psalmen van Israel, zegt:
\par 2 De Geest des HEEREN heeft door mij gesproken, en Zijn rede is op mijn tong geweest.
\par 3 De God Israels heeft gezegd, de Rotssteen Israels heeft tot mij gesproken: Er zal zijn een Heerser over de mensen, een Rechtvaardige, een Heerser in de vreze Gods.
\par 4 En Hij zal zijn gelijk het licht des morgens, wanneer de zon opgaat, des morgens zonder wolken, wanneer van den glans na den regen de grasscheutjes uit de aarde voortkomen.
\par 5 Hoewel mijn huis alzo niet is bij God, nochtans heeft Hij mij een eeuwig verbond gesteld, dat in alles wel geordineerd en bewaard is; voorzeker is daarin al mijn heil, en alle lust, hoewel Hij het nog niet doet uitspruiten.
\par 6 Maar de mannen Belials zullen altemaal zijn als doornen, die weggeworpen worden, omdat men ze met de hand niet kan vatten;
\par 7 Maar een iegelijk, die ze zal aantasten, voorziet zich met ijzer en het hout ener spies; en zij zullen ganselijk met vuur verbrand worden ter zelver plaats.
\par 8 Dit zijn de namen der helden, die David gehad heeft: Joscheb Baschebeth, de zoon van Tachkemoni, de voornaamste der hoofdlieden. Deze was Adino, de Ezniet, die zich stelde tegen achthonderd, die van hem verslagen werden op eenmaal.
\par 9 En na hem was Eleazar, de zoon van Dodo, zoon van Ahohi, deze was onder de drie helden met David, toen zij de Filistijnen beschimpten, die aldaar ten strijde verzameld waren, en de mannen van Israel waren opgetogen.
\par 10 Deze stond op, en sloeg onder de Filistijnen, totdat zijn hand moede werd, ja, zijn hand aan het zwaard kleefde; en de HEERE wrocht een groot heil ten zelven dage; en het volk keerde wederom hem na, alleenlijk om te plunderen.
\par 11 Na hem nu was Samma, de zoon van Age, de Harariet. Toen de Filistijnen verzameld waren in een dorp, en aldaar een stuk akkers was vol linzen, en het volk voor het aangezicht der Filistijnen vluchtte;
\par 12 Zo stelde hij zich in het midden van dat stuk, en verloste dat, en sloeg de Filistijnen; en de HEERE wrocht een groot heil.
\par 13 Ook gingen af drie van de dertig hoofden, en kwamen in den oogst tot David, in de spelonk van Adullam; en de hoop der Filistijnen had zich gelegerd in het dal Rafaim.
\par 14 En David was toen in een vesting; en de bezetting der Filistijnen was toen te Bethlehem.
\par 15 En David kreeg lust, en zeide: Wie zal mij water te drinken geven uit Bethlehems bornput, die in de poort is?
\par 16 Toen braken die drie helden door het leger der Filistijnen, en putten water uit Bethlehems bornput, die in de poort is, en droegen het, en kwamen tot David; doch hij wilde dat niet drinken, maar goot het uit voor den HEERE.
\par 17 En zeide: Het zij verre van mij, o HEERE, dat ik dit zou doen; zou ik drinken het bloed der mannen, die heengegaan zijn met gevaar van hun leven? En hij wilde het niet drinken. Dit deden die drie helden.
\par 18 Abisai, Joabs broeder, de zoon van Zeruja, die was ook een hoofd van drieen; en die hief zijn spies op tegen driehonderd, die van hem verslagen werden; en hij had een naam onder die drie.
\par 19 Was hij niet de heerlijkste van die drie? Daarom was hij hun tot een overste. Maar hij kwam niet tot aan die eerste drie.
\par 20 Voorts Benaja, de zoon van Jojada, de zoon van een dapperen man, groot van daden, van Kabzeel; die sloeg twee sterke leeuwen van Moab; ook ging hij af, en sloeg een leeuw in het midden van een kuil in den sneeuwtijd.
\par 21 Daartoe sloeg hij een Egyptischen man, een man van aanzien; en in de hand des Egyptenaars was een spies, maar hij ging tot hem af met een staf; en hij rukte de spies uit de hand des Egyptenaars, en doodde hem met zijn eigen spies.
\par 22 Die dingen deed Benaja, de zoon van Jojada; dies had hij een naam onder de drie helden.
\par 23 Hij was de heerlijkste van de dertig, maar tot die drie eersten kwam hij niet; en David stelde hem over zijn trawanten.
\par 24 Asahel, Joabs broeder, was onder de dertig; Elhanan, de zoon van Dodo, van Bethlehem;
\par 25 Samma, de Harodiet; Elika, de Harodiet;
\par 26 Helez, de Paltiet; Ira, de zoon van Ikes, de Thekoiet;
\par 27 Abi-ezer, de Anetothiet; Mebunnai, de Husathiet;
\par 28 Zalmon, de Ahohiet; Maharai, de Netofathiet;
\par 29 Heleb, de zoon van Baena, de Netofathiet; Ithai, de zoon van Ribai, van Gibea der kinderen Benjamins;
\par 30 Benaja, de Pirhathoniet; Hiddai, van de beken van Gaas;
\par 31 Abi-albon, de Arbathiet; Azmaveth, de Barhumiet;
\par 32 Eljachba, de Saalboniet; van de zonen van Jazen, Jonathan;
\par 33 Samma, de Harariet; Ahiam, de zoon van Sarar, de Harariet;
\par 34 Elifelet, de zoon van Ahasbai, de zoon van een Maachathiet; Eliam, de zoon van Achitofel, de Giloniet;
\par 35 Hezrai, de Karmeliet; Paerai, de Arbiet;
\par 36 Jig-al, de zoon van Nathan, van Zoba; Bani, de Gadiet;
\par 37 Zelek, de Ammoniet; Naharai, de Beerothiet, de wapendrager van Joab, den zoon van Zeruja;
\par 38 Ira, de Jethriet; Gareb, de Jethriet;
\par 39 Uria, de Hethiet, zeven en dertig in alles.

\chapter{24}

\par 1 En de toorn des HEEREN voer voort te ontsteken tegen Israel; en Hij porde David aan tegen henlieden, zeggende: Ga, tel Israel en Juda.
\par 2 De koning dan zeide tot Joab, den krijgsoverste, die bij hem was: Trek nu om, door alle stammen van Israel, van Dan tot Ber-seba toe, en tel het volk, opdat ik het getal des volks wete.
\par 3 Toen zeide Joab tot den koning: Nu doe de HEERE, uw God, tot dit volk, zoals deze en die nu zijn, honderdmaal meer, dat de ogen van mijn heer den koning het aanzien; maar waarom heeft mijn heer de koning lust tot deze zaak?
\par 4 Doch des konings woord nam de overhand tegen Joab, en tegen de oversten des heirs. Alzo toog Joab uit, met de oversten des heirs, van des konings aangezicht, om het volk Israel te tellen.
\par 5 En zij gingen over de Jordaan, en legerden zich bij Aroer, ter rechterhand der stad, die in het midden is van de beek van Gad, en aan Jaezer.
\par 6 Voorts kwamen zij in Gilead, en in het lage land Hodsi; ook kwamen zij tot Dan-jaan, en rondom bij Sidon.
\par 7 En zij kwamen tot de vesting van Tyrus, en alle steden der Hevieten en der Kanaanieten; en zij kwamen uit aan het zuiden van Juda te Ber-seba.
\par 8 Alzo togen zij om door het ganse land; en ten einde van negen maanden en twintig dagen kwamen zij te Jeruzalem.
\par 9 En Joab gaf de som van het getelde volk aan den koning; en in Israel waren achthonderd duizend strijdbare mannen, die het zwaard uittrokken, en de mannen van Juda waren vijfhonderd duizend man.
\par 10 En Davids hart sloeg hem, nadat hij het volk geteld had; en David zeide tot den HEERE: Ik heb zeer gezondigd in hetgeen ik gedaan heb; maar nu, o HEERE, neem toch de misdaad Uws knechts weg, want ik heb zeer zottelijk gedaan.
\par 11 Als nu David des morgens opstond, zo geschiedde het woord des HEEREN tot den profeet Gad, Davids ziener, zeggende:
\par 12 Ga heen, en spreek tot David: Alzo zegt de HEERE: Drie dingen draag Ik u voor; verkies u een uit die, dat Ik u doe.
\par 13 Zo kwam Gad tot David, en maakte het hem bekend, en zeide tot hem: Zal u een honger van zeven jaren in uw land komen? Of wilt gij drie maanden vlieden voor het aangezicht uwer vijanden, dat die u vervolgen? Of dat er drie dagen pestilentie in uw land zij? Merk nu, en zie toe, wat antwoord ik Dien zal wederbrengen, Die mij gezonden heeft.
\par 14 Toen zeide David tot Gad: Mij is zeer bange; laat ons toch in de hand des HEEREN vallen, want Zijn barmhartigheden zijn vele, maar laat mij in de hand van mensen niet vallen.
\par 15 Toen gaf de HEERE een pestilentie in Israel, van den morgen af tot den gezetten tijd toe; en er stierven van het volk, van Dan tot Ber-seba toe, zeventig duizend mannen.
\par 16 Toen nu de engel zijn hand uitstrekte over Jeruzalem, om haar te verderven, berouwde het den HEERE over dat kwaad, en Hij zeide tot den engel, die het verderf onder het volk maakte: Het is genoeg, trek uw hand nu af. De engel des HEEREN nu was bij den dorsvloer van Arauna, den Jebusiet.
\par 17 En David, als hij den engel zag, die het volk sloeg, sprak tot den HEERE, en zeide: Zie ik, ik heb gezondigd, en ik, ik heb onrecht gehandeld, maar wat hebben deze schapen gedaan? Uw hand zij toch tegen mij en tegen mijns vaders huis.
\par 18 En Gad kwam tot David op dienzelfden dag, en zeide tot hem: Ga op, richt den HEERE een altaar op, op den dorsvloer van Arauna, den Jebusiet.
\par 19 Alzo ging David op naar het woord van Gad, gelijk als de HEERE geboden had.
\par 20 En Arauna zag toe, en zag den koning en zijn knechten tot zich overkomen; zo ging Arauna uit, en boog zich voor den koning met zijn aangezicht ter aarde.
\par 21 En Arauna zeide: Waarom komt mijn heer de koning tot zijn knecht? En David zeide: Om dezen dorsvloer van u te kopen, om den HEERE een altaar te bouwen, opdat deze plage opgehouden worde van over het volk.
\par 22 Toen zeide Arauna tot David: Mijn heer de koning neme en offere, wat goed is in zijn ogen; zie, daar de runderen ten brandoffer, en de sleden en het rundertuig tot hout.
\par 23 Dit alles gaf Arauna, de koning, aan den koning. Voorts zeide Arauna tot den koning: De HEERE uw God neme een welgevallen in u!
\par 24 Doch de koning zeide tot Arauna: Neen, maar ik zal het zekerlijk van u kopen voor den prijs; want ik zal den HEERE, mijn God, niet offeren brandofferen om niet. Alzo kocht David den dorsvloer en de runderen voor vijftig zilveren sikkelen.
\par 25 En David bouwde aldaar den HEERE een altaar, en offerde brandofferen en dankofferen. Alzo werd de HEERE den lande verbeden, en deze plage van over Israel opgehouden.




\end{document}