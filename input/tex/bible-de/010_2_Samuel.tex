\begin{document}

\title{Das zweite Buch Samuel}


\chapter{1}

\par 1 Nach dem Tode Sauls, da David von der Amalekiter Schlacht wiedergekommen und zwei Tage in Ziklag geblieben war,
\par 2 siehe, da kam am dritten Tage ein Mann aus dem Heer von Saul mit zerrissenen Kleidern und Erde auf seinem Haupt. Und da er zu David kam, fiel er zur Erde und beugte sich nieder.
\par 3 David aber sprach zu ihm: Wo kommst du her? Er sprach zu ihm: Aus dem Heer Israels bin ich entronnen.
\par 4 David sprach zu ihm: Sage mir, wie geht es zu? Er sprach: Das Volk ist geflohen vom Streit, und ist viel Volks gefallen; dazu ist Saul tot und sein Sohn Jonathan.
\par 5 David sprach zu dem Jüngling, der ihm solches sagte: Woher weißt du, daß Saul und Jonathan tot sind?
\par 6 Der Jüngling, der ihm solches sagte, sprach: Ich kam von ungefähr aufs Gebirge Gilboa, und siehe Saul lehnte sich auf seinen Spieß, und die Wagen und Reiter jagten hinter ihm her.
\par 7 Und er wandte sich um und sah mich und rief mich. Und ich sprach: Hier bin ich.
\par 8 Und er sprach zu mir: Wer bist du? Ich sprach zu ihm: Ich bin ein Amalekiter.
\par 9 Und er sprach zu mir: Tritt zu mir und töte mich; denn ich bin bedrängt umher, und mein Leben ist noch ganz in mir.
\par 10 Da trat ich zu ihm und tötete ihn; denn ich wußte wohl, daß er nicht leben konnte nach seinem Fall; und nahm die Krone von seinem Haupt und das Armgeschmeide von seinem Arm und habe es hergebracht zu dir, meinem Herrn.
\par 11 Da faßte David seine Kleider und zerriß sie, und alle Männer, die bei ihm waren,
\par 12 und trugen Leid und weinten und fasteten bis an den Abend über Saul und Jonathan, seinen Sohn, und über das Volk des HERRN und über das Haus Israel, daß sie durchs Schwert gefallen waren.
\par 13 Und David sprach zu dem Jüngling, der es ihm ansagte: Wo bist du her? Er sprach: Ich bin eines Fremdlings, eines Amalekiters, Sohn.
\par 14 David sprach zu ihm: Wie, daß du dich nicht gefürchtet hast, deine Hand zu legen an den Gesalbten des HERRN, ihn zu verderben!
\par 15 Und David sprach zu seiner Jünglinge einem: Herzu, und schlag ihn! Und er schlug, ihn daß er starb.
\par 16 Da sprach David zu ihm: Dein Blut sei über deinem Kopf; denn dein Mund hat wider dich selbst geredet und gesprochen: Ich habe den Gesalbten des HERRN getötet.
\par 17 Und David klagte diese Klage über Saul und Jonathan, seinen Sohn,
\par 18 und befahl, man sollte die Kinder Juda das Bogenlied lehren. Siehe, es steht geschrieben im Buch der Redlichen:
\par 19 "Die Edelsten in Israel sind auf deiner Höhe erschlagen. Wie sind die Helden gefallen!
\par 20 Sagt's nicht an zu Gath, verkündet's nicht auf den Gassen zu Askalon, daß sich nicht freuen die Töchter der Philister, daß nicht frohlocken die Töchter der Unbeschnittenen.
\par 21 Ihr Berge zu Gilboa, es müsse weder tauen noch regnen auf euch noch Äcker sein, davon Hebopfer kommen; denn daselbst ist den Helden ihr Schild abgeschlagen, der Schild Sauls, als wäre er nicht gesalbt mit Öl.
\par 22 Der Bogen Jonathans hat nie gefehlt, und das Schwert Sauls ist nie leer wiedergekommen von dem Blut der Erschlagenen und vom Fett der Helden.
\par 23 Saul und Jonathan, holdselig und lieblich in ihrem Leben, sind auch im Tode nicht geschieden; schneller waren sie denn die Adler und stärker denn die Löwen.
\par 24 Ihr Töchter Israels, weint über Saul, der euch kleidete mit Scharlach säuberlich und schmückte euch mit goldenen Kleinoden an euren Kleidern.
\par 25 Wie sind die Helden so gefallen im Streit! Jonathan ist auf deinen Höhen erschlagen.
\par 26 Es ist mir Leid um dich, mein Bruder Jonathan: ich habe große Freude und Wonne an dir gehabt; deine Liebe ist mir sonderlicher gewesen, denn Frauenliebe ist.
\par 27 Wie sind die Helden gefallen und die Streitbaren umgekommen!"

\chapter{2}

\par 1 Nach dieser Geschichte fragte David den HERRN und sprach: Soll ich hinauf in der Städte Juda's eine ziehen? Und der HERR sprach zu ihm: Zieh hinauf! David sprach: Wohin? Er sprach: Gen Hebron.
\par 2 Da zog David dahin mit seinen zwei Weibern Ahinoam, der Jesreelitin, und Abigail, Nabals, des Karmeliten, Weib.
\par 3 Dazu die Männer, die bei ihm waren, führte David hinauf, einen jeglichen mit seinem Hause, und sie wohnten in den Städten Hebrons.
\par 4 Und die Männer Juda's kamen und salbten daselbst David zum König über das Haus Juda. Und da es David ward angesagt, daß die von Jabes in Gilead Saul begraben hatten,
\par 5 sandte er Boten zu ihnen und ließ ihnen sagen: Gesegnet seid ihr dem HERRN, daß ihr solche Barmherzigkeit an eurem Herrn Saul, getan und ihn begraben habt.
\par 6 So tue nun an euch der HERR Barmherzigkeit und Treue; und ich will euch auch Gutes tun, darum daß ihr solches getan habt.
\par 7 So seien nun eure Hände getrost, und seit freudig; denn euer Herr, Saul ist tot; so hat mich das Haus Juda zum König gesalbt über sich.
\par 8 Abner aber, der Sohn Ners, der Sauls Feldhauptmann war, nahm Is-Boseth, Sauls Sohn, und führte ihn gen Mahanaim
\par 9 und machte ihn zum König über Gilead, über die Asuriter, über Jesreel, Ephraim, Benjamin und über ganz Israel.
\par 10 Und Is-Boseth, Sauls Sohn, war vierzig Jahre alt, da er König ward über Israel, und regierte zwei Jahre. Aber das Haus Juda hielt es mit David.
\par 11 Die Zeit aber, da David König war zu Hebron über das Haus Juda, war sieben Jahre und sechs Monate.
\par 12 Und Abner, der Sohn Ners, zog aus samt den Knechten Is-Boseths, des Sohnes Sauls, von Mahanaim, gen Gibeon;
\par 13 und Joab, der Zeruja Sohn, zog aus samt den Knechten Davids; und sie stießen aufeinander am Teich zu Gibeon, und lagerten sich diese auf dieser Seite des Teichs, jene auf jener Seite.
\par 14 Und Abner sprach zu Joab: Laß sich die Leute aufmachen und vor uns spielen. Joab sprach: Es gilt wohl.
\par 15 Da machten sich auf und gingen hin an der Zahl zwölf aus Benjamin auf Is-Boseths Teil, des Sohnes Sauls, und zwölf von den Knechten Davids.
\par 16 Und ein jeglicher ergriff den andern bei dem Kopf und stieß ihm sein Schwert in seine Seite, und fielen miteinander; daher der Ort genannt wird: Helkath-Hazzurim, der zu Gibeon ist.
\par 17 Und es erhob sich ein sehr harter Streit des Tages. Abner aber und die Männer Israels wurden geschlagen vor den Knechten Davids.
\par 18 Es waren aber drei Söhne der Zeruja daselbst: Joab, Abisai und Asahel. Asahel aber war von leichten Füßen wie ein Reh auf den Felde
\par 19 und jagte Abner nach und wich nicht weder zur Rechten noch zur Linken von Abner.
\par 20 Da wandte sich Abner um und sprach: Bist du Asahel? Er sprach: Ja.
\par 21 Abner sprach zu ihm: Hebe dich entweder zur Rechten oder zur Linken und nimm für dich der Leute einen und nimm ihm sein Waffen. Aber Asahel wollte nicht von ihm ablassen.
\par 22 Da sprach Abner weiter zu Asahel: Hebe dich von mir! Warum willst du, daß ich dich zu Boden schlage? Und wie dürfte ich mein Antlitz aufheben vor deinem Bruder Joab?
\par 23 Aber er weigerte sich zu weichen. Da stach ihn Abner mit dem Schaft des Spießes in seinen Bauch, daß der Spieß hinten ausging; und er fiel daselbst und starb vor ihm. Und wer an den Ort kam, da Asahel tot lag, der stand still.
\par 24 Aber Joab und Abisai jagten Abner nach, bis die Sonne unterging. Und da sie kamen auf den Hügel Amma, der vor Giah liegt auf dem Wege zur Wüste Gibeon,
\par 25 versammelten sich die Kinder Benjamin hinter Abner her und wurden ein Haufe und traten auf eines Hügels Spitze.
\par 26 Und Abner rief zu Joab und sprach: Soll denn das Schwert ohne Ende fressen? Weißt du nicht, daß hernach möchte mehr Jammer werden? Wie lange willst du dem Volk nicht sagen, daß es ablasse von seinen Brüdern?
\par 27 Joab sprach: So wahr Gott lebt, hättest du heute morgen so gesagt, das Volk hätte ein jeglicher von seinem Bruder abgelassen.
\par 28 Und Joab blies die Posaune, und alles Volk stand still und jagten nicht mehr Israel nach und stritten auch nicht mehr.
\par 29 Abner aber und seine Männer gingen die ganze Nacht über das Blachfeld und gingen über den Jordan und wandelten durchs ganze Bithron und kamen gen Mahanaim.
\par 30 Joab aber wandte sich von Abner und versammelte das ganze Volk; und es fehlten an den Knechten Davids neunzehn Mann und Asahel.
\par 31 Aber die Knechte Davids hatten geschlagen unter Benjamin und den Männern Abner, daß dreihundertundsechzig Mann waren tot geblieben.
\par 32 Und sie hoben Asahel auf und begruben ihn in seines Vaters Grab zu Bethlehem. Und Joab mit seinen Männern gingen die ganze Nacht, daß ihnen das Licht anbrach zu Hebron.

\chapter{3}

\par 1 Und es war ein langer Streit zwischen dem Hause Sauls und dem Hause Davids. David aber nahm immer mehr zu, und das Haus Saul nahm immer mehr ab.
\par 2 Und es wurden David Kinder geboren zu Hebron: Sein erstgeborener Sohn: Amnon, von Ahinoam, der Jesreelitin;
\par 3 der zweite Chileab, von Abigail, Nabals Weib, des Karmeliten; der dritte: Absalom, der Sohn Maachas, der Tochter Thalmais, des Königs zu Gessur;
\par 4 der vierte: Adonia, der Sohn der Haggith; der fünfte: Sephatja, der Sohn der Abital;
\par 5 der sechste: Jethream, von Egla, dem Weib Davids. Diese sind David geboren zu Hebron.
\par 6 Als nun der Streit war zwischen dem Hause Sauls und dem Hause Davids, stärkte Abner das Haus Sauls.
\par 7 Und Saul hatte ein Kebsweib, die hieß Rizpa, eine Tochter Ajas. Und Is-Boseth sprach zu Abner: Warum hast du dich getan zu meines Vaters Kebsweib?
\par 8 Da ward Abner sehr zornig über die Worte Is-Boseths und sprach: Bin ich denn ein Hundskopf, der ich wider Juda an dem Hause Sauls, deines Vaters, und an seinen Brüdern und Freunden Barmherzigkeit tue und habe dich nicht in Davids Hände gegeben? Und du rechnest mir heute eine Missetat zu um ein Weib?
\par 9 Gott tue Abner dies und das, wenn ich nicht tue, wie der HERR dem David geschworen hat,
\par 10 daß das Königreich vom Hause Saul genommen werde und der Stuhl Davids aufgerichtet werde über Israel und Juda von Dan bis gen Beer-Seba.
\par 11 Da konnte er fürder ihm kein Wort mehr antworten, so fürchtete er sich vor ihm.
\par 12 Und Abner sandte Boten zu David für sich und ließ ihm sagen: Wes ist das Land? Und sprach: Mache einen Bund mit mir; siehe, meine Hand soll mit dir sein, daß ich zu dir kehre das ganze Israel.
\par 13 Er sprach: Wohl, ich will einen Bund mit dir machen. Aber eins bitte ich von dir, daß du mein Angesicht nicht sehest, du bringst denn zuvor Michal, Sauls Tochter, wenn du kommst, mein Angesicht zu sehen.
\par 14 Auch sandte David Boten zu Is-Boseth, dem Sohn Sauls, und ließ ihm sagen: Gib mir mein Weib Michal, die ich mir verlobt habe mit hundert Vorhäuten der Philister.
\par 15 Is-Boseth sandte hin und ließ sie nehmen von dem Mann Paltiel, dem Sohn des Lais.
\par 16 Und ihr Mann ging mit ihr und weinte hinter ihr bis gen Bahurim. Da sprach Abner zu ihm: Kehre um und gehe hin! Und er kehrte um.
\par 17 Und Abner hatte eine Rede mit den Ältesten in Israel und sprach: Ihr habt schon längst nach David getrachtet, daß er König wäre über euch.
\par 18 So tut's nun; denn der HERR hat von David gesagt: Ich will mein Volk Israel erretten durch die Hand Davids, meines Knechtes, von der Philister Hand und aller seiner Feinde Hand.
\par 19 Auch redete Abner vor den Ohren Benjamins und ging auch hin, zu reden vor den Ohren Davids zu Hebron alles, was Israel und dem ganzen Hause Benjamin wohl gefiel.
\par 20 Da nun Abner gen Hebron zu David kam und mit ihm zwanzig Mann, machte ihnen David ein Mahl.
\par 21 Und Abner sprach zu David: Ich will mich aufmachen und hingehen, daß ich das ganze Israel zu meinem Herrn, dem König, sammle und daß sie einen Bund mit dir machen, auf daß du König seist, wie es deine Seele begehrt. Also ließ David Abner von sich, daß er hinginge mit Frieden.
\par 22 Und siehe, die Knechte Davids und Joab kamen von einem Streifzuge und brachten mit sich große Beute. Abner aber war nicht mehr bei David zu Hebron, sondern er hatte ihn von sich gelassen, daß er mit Frieden weggegangen war.
\par 23 Da aber Joab und das ganze Heer mit ihm war gekommen, ward ihm angesagt, daß Abner, der Sohn Ners, zum König gekommen war und hatte er ihn von sich gelassen, daß er mit Frieden war weggegangen.
\par 24 Da ging Joab zum König hinein und sprach: Was hast du getan? Siehe, Abner ist zu dir gekommen; warum hast du ihn von dir gelassen, daß er ist weggegangen?
\par 25 Kennst du Abner, den Sohn Ners, nicht? Denn er ist gekommen, dich zu überreden, daß er erkennt deinen Ausgang und Eingang und erführe alles, was du tust.
\par 26 Und da Joab von David ausging, sandte er Boten Abner nach, daß sie ihn wiederum holten von Bor-Hassira; und David wußte nichts darum.
\par 27 Als nun Abner wieder gen Hebron kam, führte ihn Joab mitten unter das Tor, daß er heimlich mit ihm redete, und stach ihn daselbst in den Bauch, daß er starb, um seines Bruders Asahel Bluts willen.
\par 28 Da das David hernach erfuhr, sprach er: Ich bin unschuldig und mein Königreich vor dem HERRN ewiglich an dem Blut Abners, des Sohnes Ners;
\par 29 es falle aber auf das Haupt Joabs und auf seines Vaters ganzes Haus, und müsse nicht aufhören im Hause Joabs, der einen Eiterfluß und Aussatz habe und am Stabe gehe und durchs Schwert falle und an Brot Mangel habe.
\par 30 Also erwürgten Joab und Abisai Abner, darum daß er ihren Bruder Asahel getötet hatte im Streit zu Gibeon.
\par 31 David aber sprach zu Joab und allem Volk, das mit ihm war: Zerreißt eure Kleider und gürtet Säcke um euch und tragt Leid um Abner! Und der König ging dem Sarge nach.
\par 32 Und da sie Abner begruben zu Hebron, hob der König seine Stimme auf und weinte bei dem Grabe Abners, und weinte auch alles Volk.
\par 33 Und der König klagte um Abner und sprach: Mußte Abner sterben, wie ein Ruchloser stirbt?
\par 34 Deine Hände waren nicht gebunden, deine Füße waren nicht in Fesseln gesetzt; du bist gefallen, wie man vor bösen Buben fällt. Da beweinte ihn alles Volk noch mehr.
\par 35 Da nun alles Volk hineinkam, mit David zu essen, da es noch hoch am Tage war, schwur David und sprach: Gott tue mir dies und das, wo ich Brot esse oder etwas koste, ehe die Sonne untergeht.
\par 36 Und alles Volk erkannte es, und gefiel ihnen auch wohl, wie alles, was der König tat, dem ganzen Volke wohl gefiel;
\par 37 und alles Volk und ganz Israel merkten des Tages, daß es nicht vom König war, daß Abner, der Sohn Ners, getötet ward.
\par 38 Und der König sprach zu seinen Knechten: Wisset ihr nicht, daß auf diesen Tag ein Fürst und Großer gefallen ist in Israel?
\par 39 Ich aber bin noch zart und erst gesalbt zum König. Aber die Männer, die Kinder der Zeruja, sind mir verdrießlich. Der HERR vergelte dem, der Böses tut, nach seiner Bosheit.

\chapter{4}

\par 1 Da aber der Sohn Sauls hörte, daß Abner zu Hebron tot wäre, wurden seine Hände laß, und ganz Israel erschrak.
\par 2 Es waren aber zwei Männer, Hauptleute der streifenden Rotten unter dem Sohn Sauls; einer hieß Baana, der andere Rechab, Söhne Rimmons, des Beerothiters, aus den Kindern Benjamin. (Denn Beeroth ward auch unter Benjamin gerechnet;
\par 3 und die Beerothiter waren geflohen gen Gitthaim und wohnten daselbst gastweise bis auf den heutigen Tag.)
\par 4 Auch hatte Jonathan, der Sohn Sauls, einen Sohn, der war lahm an den Füßen, und war fünf Jahre alt, da das Geschrei von Saul und Jonathan aus Jesreel kam und seine Amme ihn aufhob und floh; und indem sie eilte und floh, fiel er und ward hinkend; und er hieß Mephiboseth.
\par 5 So gingen nun hin die Söhne Rimmons, des Beerothiters, Rechab und Baana, und kamen zum Hause Is-Boseths, da der Tag am heißesten war; und er lag auf seinem Lager am Mittag.
\par 6 Und sie kamen ins Haus, Weizen zu holen, und stachen ihn in den Bauch und entrannen.
\par 7 Denn da sie ins Haus kamen, lag er auf seinem Bette in seiner Schlafkammer; und sie stachen ihn tot und hieben ihm den Kopf ab und nahmen seinen Kopf und gingen hin des Weges auf dem Blachfelde die ganze Nacht
\par 8 und brachten das Haupt Is-Boseths zu David gen Hebron und sprachen zum König: Siehe, da ist das Haupt Is-Boseths, Sauls Sohnes, deines Feindes, der nach deiner Seele stand; der HERR hat heute meinen Herrn, den König, gerächt an Saul und an seinem Samen.
\par 9 Da antwortete ihnen David: So wahr der HERR lebt, der meine Seele aus aller Trübsal erlöst hat,
\par 10 ich griff den, der mir verkündigte und sprach: Saul ist tot! und meinte, er wäre ein guter Bote, und erwürgte ihn zu Ziklag, dem ich sollte Botenlohn geben.
\par 11 Und diese gottlosen Leute haben einen gerechten Mann in seinem Hause auf seinem Lager erwürgt. Ja, sollte ich das Blut nicht fordern von euren Händen und euch von der Erde tun?
\par 12 Und David gebot seinen Jünglingen; die erwürgten sie und hieben ihre Hände und Füße ab und hingen sie auf am Teich zu Hebron. Aber das Haupt Is-Boseths nahmen sie und begruben's in Abners Grab zu Hebron.

\chapter{5}

\par 1 Und es kamen alle Stämme Israels zu David gen Hebron und sprachen: Siehe, wir sind deines Gebeins und deines Fleisches.
\par 2 Dazu auch vormals, da Saul über uns König war, führtest du Israel aus und ein. So hat der HERR dir gesagt: Du sollst mein Volk Israel hüten und sollst ein Herzog sein über Israel.
\par 3 Und es kamen alle Ältesten in Israel zum König gen Hebron. Und der König David machte mit ihnen einen Bund zu Hebron vor dem HERRN, und sie salbten David zum König über Israel.
\par 4 Dreißig Jahre war David alt, da er König ward, und regierte vierzig Jahre.
\par 5 Zu Hebron regierte er sieben Jahre und sechs Monate über Juda; aber zu Jerusalem regierte er dreiunddreißig Jahre über ganz Israel und Juda.
\par 6 Und der König zog hin mit seinen Männern gen Jerusalem wider die Jebusiter, die im Lande wohnten. Sie aber sprachen zu David: Du wirst nicht hier hereinkommen, sondern Blinde und Lahme werden dich abtreiben. Damit meinten sie aber, daß David nicht würde dahinein kommen.
\par 7 David aber gewann die Burg Zion, das ist Davids Stadt.
\par 8 Da sprach David desselben Tages: Wer die Jebusiter schlägt und erlangt die Dachrinnen, die Lahmen und die Blinden, denen die Seele Davids feind ist...! Daher spricht man: Laß keinen Blinden und Lahmen ins Haus kommen.
\par 9 Also wohnte David auf der Burg und hieß sie Davids Stadt. Und David baute ringsumher von Millo an einwärts.
\par 10 Und David nahm immer mehr zu, und der HERR, der Gott Zebaoth, war mit ihm.
\par 11 Und Hiram, der König zu Tyrus sandte Boten zu David und Zedernbäume und Zimmerleute und Steinmetzen, daß sie David ein Haus bauten.
\par 12 Und David merkte, daß ihn der HERR zum König über Israel bestätigt hatte und sein Königreich erhöht um seines Volks Israel willen.
\par 13 Und David nahm noch mehr Weiber und Kebsweiber zu Jerusalem, nachdem er von Hebron gekommmen war; und wurden ihm noch mehr Söhne und Töchter geboren.
\par 14 Und das sind die Namen derer, die ihm zu Jerusalem geboren sind: Sammua, Sobab, Nathan, Salomo,
\par 15 Jibhar, Elisua, Nepheg, Japhia,
\par 16 Elisama, Eljada, Eliphelet.
\par 17 Und da die Philister hörten, daß man David zum König über Israel gesalbt hatte, zogen sie alle herauf, David zu suchen. Da das David erfuhr, zog er hinab in eine Burg.
\par 18 Aber die Philister kamen und ließen sich nieder im Grunde Rephaim.
\par 19 Und David fragte den HERRN und sprach: Soll ich hinaufziehen wider die Philister? und willst du sie in meine Hand geben? Der HERR sprach zu David: Zieh hinauf! ich will die Philister in deine Hände geben.
\par 20 Und David kam gen Baal-Perazim und schlug sie daselbst und sprach: Der HERR hat meine Feinde vor mir voneinander gerissen, wie die Wasser reißen. Daher hieß man den Ort Baal-Perazim.
\par 21 Und sie ließen ihre Götzen daselbst; David aber und seine Männer hoben sie auf.
\par 22 Die Philister aber zogen abermals herauf und ließen sich nieder im Grunde Rephaim.
\par 23 Und David fragte den HERRN; der sprach: Du sollst nicht hinaufziehen, sondern komm von hinten zu ihnen, daß du an sie kommst gegenüber den Maulbeerbäumen.
\par 24 Und wenn du hörst das Rauschen auf den Wipfeln der Maulbeerbäume einhergehen, so eile; denn der HERR ist dann ausgegangen vor dir her, zu schlagen das Heer der Philister.
\par 25 David tat, wie ihm der HERR geboten hatte, und schlug die Philister von Geba an, bis man kommt gen Geser.

\chapter{6}

\par 1 Und David sammelte abermals alle junge Mannschaft in Israel, dreißigtausend,
\par 2 und machte sich auf und ging hin mit allem Volk, das bei ihm war, gen Baal in Juda, daß er die Lade Gottes von da heraufholte, deren Name heißt: Der Name des HERRN Zebaoth wohnt darauf über den Cherubim.
\par 3 Und sie ließen die Lade Gottes führen auf einen neuen Wagen und holten sie aus dem Hause Abinadabs, der auf dem Hügel wohnte. Usa aber und Ahjo, die Söhne Abinadabs, trieben den neuen Wagen.
\par 4 Und da sie ihn mit der Lade Gottes aus dem Hause Abinadabs führten, der auf einem Hügel wohnte, und Ahjo vor der Lade her ging,
\par 5 spielte David und das ganze Haus Israel vor dem HERRN her mit allerlei Saitenspiel von Tannenholz, mit Harfen und Psaltern und Pauken und Schellen und Zimbeln.
\par 6 Und da sie kamen zu Tenne Nachons, griff Usa zu und hielt die Lade Gottes; denn die Rinder traten beiseit aus.
\par 7 Da ergrimmte des HERRN Zorn über Usa, und Gott schlug ihn daselbst um seines Frevels willen, daß er daselbst starb bei der Lade Gottes.
\par 8 Da ward David betrübt, daß der HERR den Usa so wegriß, und man hieß die Stätte Perez-Usa bis auf diesen Tag.
\par 9 Und David fürchtete sich vor dem HERRN des Tages und sprach: Wie soll die Lade des HERRN zu mir kommen?
\par 10 Und wollte sie nicht lassen zu sich bringen in die Stadt Davids, sondern ließ sie bringen ins Haus Obed-Edoms, des Gathiters.
\par 11 Und da die Lade des HERRN drei Monate blieb im Hause Obed-Edoms, des Gathiters, segnete ihn der HERR und sein ganzes Haus.
\par 12 Und es ward dem König David angesagt, daß der HERR das Haus Obed-Edoms segnete und alles, was er hatte, um der Lade Gottes willen. Da ging er hin und holte die Lade Gottes aus dem Hause Obed-Edoms herauf in die Stadt Davids mit Freuden.
\par 13 Und da sie einhergegangen waren mit der Lade des HERRN sechs Gänge, opferte man einen Ochsen und ein fettes Schaf.
\par 14 Und David tanzte mit aller Macht vor dem HERR her und war begürtet mit einem leinenen Leibrock.
\par 15 Und David samt dem ganzen Israel führten die Lade des HERRN herauf mit Jauchzen und Posaunen.
\par 16 Und da die Lade des HERRN in die Stadt Davids kam, sah Michal, die Tochter Sauls, durchs Fenster und sah den König David springen und tanzen vor dem HERRN und verachtete ihn in ihrem Herzen.
\par 17 Da sie aber die Lade des HERRN hereinbrachten, stellten sie die an ihren Ort mitten in der Hütte, die David für sie hatte aufgeschlagen. Und David opferte Brandopfer und Dankopfer vor dem HERRN.
\par 18 Und da David hatte ausgeopfert die Brandopfer und Dankopfer, segnete er das Volk in dem Namen des HERRN Zebaoth
\par 19 und teilte aus allem Volk, der ganzen Menge Israels, sowohl Mann als Weib, einem jeglichen einen Brotkuchen und ein Stück Fleisch und ein halbes Maß Wein. Da kehrte alles Volk heim, ein jeglicher in sein Haus.
\par 20 Da aber David wiederkam sein Haus zu grüßen, ging Michal, die Tochter Sauls, heraus ihm entgegen und sprach: Wie herrlich ist heute der König von Israel gewesen, der sich vor den Mägden seiner Knechte entblößt hat, wie sich die losen Leute entblößen!
\par 21 David aber sprach zu Michal: Ich will vor dem HERRN spielen, der mich erwählt hat vor deinem Vater und vor allem seinem Hause, daß er mir befohlen hat, ein Fürst zu sein über das Volk des HERRN, über Israel,
\par 22 Und ich will noch geringer werden denn also und will niedrig sein in meinen Augen, und mit den Mägden, von denen du geredet hast, zu Ehren kommen.
\par 23 Aber Michal, Sauls Tochter, hatte kein Kind bis an den Tag ihres Todes.

\chapter{7}

\par 1 Da nun der König in seinem Hause saß und der HERR ihm Ruhe gegeben hatte von allen seinen Feinden umher,
\par 2 sprach er zu dem Propheten Nathan: Siehe, ich wohne in einem Zedernhause, und die Lade Gottes wohnt unter den Teppichen.
\par 3 Nathan sprach zu dem König: Gehe hin; alles, was du in deinem Herzen hast, das tue, denn der HERR ist mit dir.
\par 4 Des Nachts aber kam das Wort des HERRN zu Nathan und sprach:
\par 5 Gehe hin und sage meinem Knechte David: So spricht der HERR: Solltest du mir ein Haus bauen, daß ich darin wohne?
\par 6 Habe ich doch in keinem Hause gewohnt seit dem Tage, da ich die Kinder Israel aus Ägypten führte, bis auf diesen Tag, sondern ich habe gewandelt in der Hütte und Wohnung.
\par 7 Wo ich mit allen Kinder Israel hin wandelte, habe ich auch je geredet mit irgend der Stämme Israels einem, dem ich befohlen habe, mein Volk Israel zu weiden, und gesagt: Warum baut ihr mir nicht ein Zedernhaus?
\par 8 So sollst du nun so sagen meinem Knechte David: So spricht der HERR Zebaoth: Ich habe dich genommen von den Schafhürden, daß du sein solltest ein Fürst über mein Volk Israel,
\par 9 und bin mit dir gewesen, wo du hin gegangen bist, und habe alle deine Feinde vor dir ausgerottet und habe dir einen großen Namen gemacht wie der Name der Großen auf Erden.
\par 10 Und ich will meinem Volk Israel einen Ort setzen und will es pflanzen, daß es daselbst wohne und nicht mehr in der Irre gehe, und es Kinder der Bosheit nicht mehr drängen wie vormals und seit der Zeit, daß ich Richter über mein Volk Israel verordnet habe;
\par 11 und will Ruhe geben von allen deinen Feinden. Und der HERR verkündigt dir, daß der HERR dir ein Haus machen will.
\par 12 Wenn nun deine Zeit hin ist, daß du mit deinen Vätern schlafen liegst, will ich deinen Samen nach dir erwecken, der von deinem Leibe kommen soll; dem will ich sein Reich bestätigen.
\par 13 Der soll meinem Namen ein Haus bauen, und ich will den Stuhl seines Königreichs bestätigen ewiglich.
\par 14 Ich will sein Vater sein, und er soll mein Sohn sein. Wenn er eine Missetat tut, will ich ihn mit Menschenruten und mit der Menschenkinder Schlägen strafen;
\par 15 aber meine Barmherzigkeit soll nicht von ihm entwandt werden, wie ich sie entwandt habe von Saul, den ich vor dir habe weggenommen.
\par 16 Aber dein Haus und dein Königreich soll beständig sein ewiglich vor dir, und dein Stuhl soll ewiglich bestehen.
\par 17 Da Nathan alle diese Worte und all dies Gesicht David gesagt hatte,
\par 18 kam David, der König, und blieb vor dem Herrn und sprach: Wer bin ich, HERR HERR, und was ist mein Haus, daß du mich bis hierher gebracht hast?
\par 19 Dazu hast du das zu wenig geachtet HERR HERR, sondern hast dem Hause deines Knechts noch von fernem Zukünftigem geredet, und das nach Menschenweise, HERR HERR!
\par 20 Und was soll David mehr reden mit dir? Du erkennst deinen Knecht, HERR HERR!
\par 21 Um deines Wortes willen und nach deinem Herzen hast du solche großen Dinge alle getan, daß du sie deinem Knecht kundtätest.
\par 22 Darum bist du auch groß geachtet, HERR, Gott; denn es ist keiner wie du und ist kein Gott als du, nach allem, was wir mit unsern Ohren gehört haben.
\par 23 Denn wo ist ein Volk auf Erden wie dein Volk Israel, um welches willen Gott ist hingegangen, sich ein Volk zu erlösen und sich einen Namen zu machen und solch große und schreckliche Dinge zu tun in deinem Lande vor deinem Volk, welches du dir erlöst hast von Ägypten, von den Heiden und ihren Göttern?
\par 24 Und du hast dir dein Volk Israel zubereitet, dir zum Volk in Ewigkeit; und du, HERR, bist ihr Gott geworden.
\par 25 So bekräftige nun, HERR, Gott, das Wort in Ewigkeit, das du über deinen Knecht und über sein Haus geredet hast, und tue, wie du geredet hast!
\par 26 So wird dein Name groß werden in Ewigkeit, daß man wird sagen: Der HERR Zebaoth ist der Gott über Israel, und das Haus deines Knechtes David wird bestehen vor dir.
\par 27 Denn du, HERR Zebaoth, du Gott Israels, hast das Ohr deines Knechts geöffnet und gesagt: Ich will dir ein Haus bauen. Darum hat dein Knecht sein Herz gefunden, daß er dieses Gebet zu dir betet.
\par 28 Nun, HERR HERR, du bist Gott, und deine Worte werden Wahrheit sein. Du hast solches Gute über deinen Knecht geredet.
\par 29 So hebe nun an und segne das Haus deines Knechtes, daß es ewiglich vor dir sei; denn du, HERR HERR, hast's geredet, und mit deinem Segen wird deines Knechtes Haus gesegnet ewiglich.

\chapter{8}

\par 1 Und es begab sich darnach, daß David die Philister schlug und schwächte sie und nahm den Dienstzaum von der Philister Hand.
\par 2 Er schlug auch die Moabiter also zu Boden, daß er zwei Teile zum Tode brachte und einen Teil am Leben ließ. Also wurden die Moabiter David untertänig, daß sie ihm Geschenke zutrugen.
\par 3 David schlug auch Hadadeser, den Sohn Rehobs, König zu Zoba, da er hinzog, seine Macht wieder zu holen an dem Wasser Euphrat.
\par 4 Und David fing aus ihnen tausendundsiebenhundert Reiter und zwanzigtausend Mann Fußvolk und verlähmte alle Rosse der Wagen und behielt übrig hundert Wagen.
\par 5 Es kamen aber die Syrer von Damaskus, zu helfen Hadadeser, dem König zu Zoba; und David schlug der Syrer zweiundzwanzigtausend Mann
\par 6 und legte Volk in das Syrien von Damaskus. Also ward Syrien David untertänig, daß sie ihm Geschenke zutrugen. Denn der HERR half David, wo er hin zog.
\par 7 Und David nahm die goldenen Schilde, die Hadadesers Knechte gehabt hatten, und brachte sie gen Jerusalem.
\par 8 Aber von Betah und Berothai, den Städten Hadadesers, nahm der König David sehr viel Erz.
\par 9 Da aber Thoi, der König zu Hamath, hörte, daß David hatte alle Macht des Hadadesers geschlagen,
\par 10 sandte er Joram, seinen Sohn, zu David, ihn freundlich zu grüßen und ihn zu segnen, daß er wider Hadadeser gestritten und ihn geschlagen hatte (denn Thoi hatte einen Streit mit Hadadeser): und er hatte mit sich silberne, goldene und eherne Kleinode,
\par 11 welche der König David auch dem HERR heiligte samt dem Silber und Gold, das er heiligte von allen Heiden, die er unter sich gebracht;
\par 12 von Syrien, von Moab, von den Kindern Ammon, von den Philistern, von Amalek, von der Beute Hadadesers, des Sohnes Rehobs, König zu Zoba.
\par 13 Auch machte sich David einen Namen da er wiederkam von der Syrer Schlacht und schlug im Salztal achtzehntausend Mann,
\par 14 und legte Volk in ganz Edom, und ganz Edom war David unterworfen; denn der HERR half David, wo er hin zog.
\par 15 Also war David König über ganz Israel, und schaffte Recht und Gerechtigkeit allem Volk.
\par 16 Joab, der Zeruja Sohn, war über das Heer; Josaphat aber, der Sohn Ahiluds war Kanzler;
\par 17 Zadok, der Sohn Ahitobs, und Ahimelech, der Sohn Abjathars, waren Priester; Seraja war Schreiber;
\par 18 Benaja, der Sohn Jojadas, war über die Krether und Plether, und die Söhne Davids waren Priester.

\chapter{9}

\par 1 Und David sprach: Ist auch noch jemand übriggeblieben von dem Hause Sauls, daß ich Barmherzigkeit an ihm tue um Jonathans willen?
\par 2 Es war aber ein Knecht vom Hause Sauls, der hieß Ziba; den riefen sie zu David. Und der König sprach zu ihm: Bist du Ziba? Er sprach: Ja, dein Knecht.
\par 3 Der König sprach: Ist noch jemand vom Hause Sauls, daß ich Gottes Barmherzigkeit an ihm tue? Ziba sprach: Es ist noch da ein Sohn Jonathans, lahm an den Füßen.
\par 4 Der König sprach zu ihm: Wo ist er? Ziba sprach zum König: Siehe, er ist zu Lo-Dabar im Hause Machirs, des Sohnes Ammiels.
\par 5 Da sandte der König David hin und ließ ihn holen von Lo-Dabar aus dem Hause Machirs, des Sohnes Ammiels.
\par 6 Da nun Mephiboseth, der Sohn Jonathans, des Sohnes Sauls, zu David kam, fiel er auf sein Angesicht und beugte sich nieder. David aber sprach: Mephiboseth! Er sprach: Hier bin ich, dein Knecht.
\par 7 David sprach zu ihm: Fürchte dich nicht; denn ich will Barmherzigkeit an dir tun um Jonathans, deines Vaters, willen und will dir allen Acker deines Vaters Saul wiedergeben; du sollst aber täglich an meinem Tisch das Brot essen.
\par 8 Er aber fiel nieder und sprach: Wer bin ich, dein Knecht, daß du dich wendest zu einem toten Hunde, wie ich bin?
\par 9 Da rief der König Ziba, den Diener Sauls, und sprach zu ihm: Alles, was Saul gehört hat und seinem ganzen Hause, habe ich dem Sohn deines Herrn gegeben.
\par 10 So arbeite ihm nun seinen Acker, du und deine Kinder und Knechte, und bringe es ein, daß es das Brot sei des Sohnes deines Herrn, daß er sich nähre; aber Mephiboseth, deines Herrn Sohn, soll täglich das Brot essen an meinem Tisch. Ziba aber hatte fünfzehn Söhne und zwanzig Knechte.
\par 11 Und Ziba sprach zum König: Alles, wie mein Herr, der König, seinem Knecht geboten hat, so soll dein Knecht tun. Und Mephiboseth (sprach David) esse an meinem Tisch wie der Königskinder eins.
\par 12 Und Mephiboseth hatte einen kleinen Sohn, der hieß Micha. Aber alles, was im Hause Zibas wohnte, das diente Mephiboseth.
\par 13 Mephiboseth aber wohnte zu Jerusalem; denn er aß täglich an des Königs Tisch, und er hinkte mit seinen beiden Füßen.

\chapter{10}

\par 1 Und es begab sich darnach, daß der König der Kinder Ammon starb, und sein Sohn Hanun ward König an seiner Statt.
\par 2 Da sprach David: Ich will Barmherzigkeit tun an Hanun, dem Sohn Nahas, wie sein Vater an mir Barmherzigkeit getan hat. Und sandte hin und ließ ihn trösten durch seine Knechte über seinen Vater. Da nun die Knechte Davids ins Land der Kinder Ammon kamen,
\par 3 sprachen die Gewaltigen der Kinder Ammon zu ihrem Herrn, Hanun: Meinst du, daß David deinen Vater ehren wolle, daß er Tröster zu dir gesandt hat? Meinst du nicht, daß er darum hat seine Knechte zu dir gesandt, daß er die Stadt erforsche und erkunde und umkehre?
\par 4 Da nahm Hanun die Knechte David und schor ihnen den Bart halb und schnitt ihnen die Kleider halb ab bis an den Gürtel und ließ sie gehen.
\par 5 Da das David ward angesagt, sandte er ihnen entgegen; denn die Männer waren sehr geschändet. Und der König ließ ihnen sagen: Bleibt zu Jericho, bis euer Bart gewachsen; so kommt dann wieder.
\par 6 Da aber die Kinder Ammon sahen, daß sie vor David stinkend geworden waren, sandten sie hin und dingten die Syrer des Hauses Rehob und die Syrer zu Zoba, zwanzigtausend Mann Fußvolk, und von dem König Maachas tausend Mann und von Is-Tob zwölftausend Mann.
\par 7 Da das David hörte, sandte er Joab mit dem ganzen Heer der Kriegsleute.
\par 8 Und die Kinder Ammon zogen aus und rüsteten sich zum Streit vor dem Eingang des Tors. Die Syrer aber von Zoba, von Rehob, von Is-Tob und von Maacha waren allein im Felde.
\par 9 Da Joab nun sah, daß der Streit auf ihn gestellt war vorn und hinten, erwählte er aus aller jungen Mannschaft in Israel und stellte sich wider die Syrer.
\par 10 Und das übrige Volk tat er unter die Hand seines Bruders Abisai, daß er sich rüstete wider dir Kinder Ammon,
\par 11 und sprach: Werden mir die Syrer überlegen sein, so komm mir zu Hilfe; werden aber die Kinder Ammon dir überlegen sein, so will ich dir zu Hilfe kommen.
\par 12 Sei getrost und laß uns stark sein für unser Volk und für die Städte unsers Gottes; der HERR aber tue, was ihm gefällt.
\par 13 Und Joab machte sich herzu mit dem Volk, das bei ihm war, zu streiten wider die Syrer; und sie flohen vor ihm.
\par 14 Und da die Kinder Ammon sahen, daß die Syrer flohen, flohen sie auch vor Abisai und zogen in die Stadt. Also kehrte Joab um von den Kindern Ammon und kam gen Jerusalem.
\par 15 Und da die Syrer sahen, daß sie geschlagen waren vor Israel, kamen sie zuhauf.
\par 16 Und Hadadeser sandte hin und brachte heraus die Syrer jenseit des Stromes und führte herein ihre Macht; und Sobach, der Feldhauptmann Hadadesers, zog vor ihnen her.
\par 17 Da das David ward angesagt, sammelte er zuhauf das ganze Israel und zog über den Jordan und kam gen Helam. Und die Syrer stellten sich wider David, mit ihm zu streiten.
\par 18 Aber die Syrer flohen vor Israel. Und David verderbte der Syrer siebenhundert Wagen und vierzigtausend Reiter; dazu Sobach, den Feldhauptmann, schlug er, daß er daselbst starb.
\par 19 Da aber die Könige, die unter Hadadeser waren, sahen, daß sie geschlagen waren vor Israel, machten sie Frieden mit Israel und wurden ihnen untertan. Und die Syrer fürchteten sich, den Kindern Ammon mehr zu helfen.

\chapter{11}

\par 1 Und da das Jahr um kam, zur Zeit, wann die Könige pflegen auszuziehen, sandte David Joab und seine Knechte mit ihm das ganze Israel, daß sie die Kinder Ammon verderbten und Rabba belagerten. David aber blieb zu Jerusalem.
\par 2 Und es begab sich, daß David um den Abend aufstand von seinem Lager und ging auf dem Dach des Königshauses und sah vom Dach ein Weib sich waschen; und das Weib war sehr schöner Gestalt.
\par 3 Und David sandte hin und ließ nach dem Weibe fragen, und man sagte: Ist das nicht Bath-Seba, die Tochter Eliams, das Weib des Urias, des Hethiters?
\par 4 Und David sandte Boten hin und ließ sie holen. Und da sie zu ihm hineinkam, schlief er bei ihr. Sie aber reinigte sich von ihrer Unreinigkeit und kehrte wieder zu ihrem Hause.
\par 5 Und das Weib ward schwanger und sandte hin und ließ David verkündigen und sagen: Ich bin schwanger geworden.
\par 6 David aber sandte zu Joab: Sende zu mir Uria, den Hethiter. Und Joab sandte Uria zu David.
\par 7 Und da Uria zu ihm kam, fragte David, ob es mit Joab und mit dem Volk und mit dem Streit wohl stünde?
\par 8 Und David sprach zu Uria: Gehe hinab in dein Haus und wasche deine Füße. Und da Uria zu des Königs Haus hinausging, folgte ihm nach des Königs Geschenk.
\par 9 Aber Uria legte sich schlafen vor der Tür des Königshauses, da alle Knechte seines Herrn lagen, und ging nicht hinab in sein Haus.
\par 10 Da man aber David ansagte: Uria ist nicht hinab in sein Haus gegangen, sprach David zu ihm: Bist du nicht über Feld hergekommen? Warum bist du nicht hinab in dein Haus gegangen?
\par 11 Uria aber sprach zu David: Die Lade und Israel und Juda bleiben in Zelten, und Joab, mein Herr, und meines Herrn Knechte liegen im Felde, und ich sollte in mein Haus gehen, daß ich äße und tränke und bei meinem Weibe läge? So wahr du lebst und deine Seele lebt, ich tue solches nicht.
\par 12 David sprach zu Uria: So bleibe auch heute hier; morgen will ich dich lassen gehen. So blieb Uria zu Jerusalem des Tages und des andern dazu.
\par 13 Und David lud ihn, daß er vor ihm aß und trank, und machte ihn trunken. Aber des Abends ging er aus, daß er sich schlafen legte auf sein Lager mit seines Herrn Knechten, und ging nicht hinab in sein Haus.
\par 14 Des Morgens schrieb David einen Brief an Joab und sandte ihn durch Uria.
\par 15 Er schrieb aber also in den Brief: Stellt Uria an den Streit, da er am härtesten ist, und wendet euch hinter ihm ab, daß er erschlagen werde und sterbe.
\par 16 Als nun Joab um die Stadt lag, stellte er Uria an den Ort, wo er wußte, daß streitbare Männer waren.
\par 17 Und da die Männer der Stadt herausfielen und stritten wider Joab, fielen etliche des Volks von den Knechten Davids, und Uria, der Hethiter, starb auch.
\par 18 Da sandte Joab hin und ließ David ansagen allen Handel des Streits
\par 19 und gebot dem Boten und sprach: Wenn du allen Handel des Streits hast ausgeredet mit dem König
\par 20 und siehst, daß der König sich erzürnt und zu dir spricht: Warum habt ihr euch so nahe zur Stadt gemacht mit dem Streit? Wißt ihr nicht, wie man pflegt von der Mauer zu schießen?
\par 21 Wer schlug Abimelech, den Sohn Jerubbeseths? Warf nicht ein Weib einen Mühlstein auf ihn von der Mauer, daß er starb zu Thebez? Warum habt ihr euch so nahe zur Mauer gemacht? so sollst du sagen: Dein Knecht Uria, der Hethiter, ist auch tot.
\par 22 Der Bote ging hin und kam und sagte an David alles, darum ihn Joab gesandt hatte.
\par 23 Und der Bote sprach zu David: Die Männer nahmen überhand wider uns und fielen zu uns heraus aufs Feld; wir aber waren an ihnen bis vor die Tür des Tors;
\par 24 und die Schützen schossen von der Mauer auf deine Knechte und töteten etliche von des Königs Knechten; dazu ist Uria, der Hethiter, auch tot.
\par 25 David sprach zum Boten: So sollst du zu Joab sagen: Laß dir das nicht übel gefallen; denn das Schwert frißt jetzt diesen, jetzt jenen. Fahre fort mit dem Streit wider die Stadt, daß du sie zerbrechest, und seid getrost.
\par 26 Und da Urias Weib hörte, daß ihr Mann, Uria, tot war, trug sie Leid um ihren Eheherrn.
\par 27 Da sie aber ausgetrauert hatte, sandte David hin und ließ sie in sein Haus holen, und sie ward sein Weib und gebar ihm einen Sohn. Aber die Tat gefiel dem HERRN übel, die David tat.

\chapter{12}

\par 1 Und der HERR sandte Nathan zu David. Da der zu ihm kam, sprach er zu ihm: Es waren zwei Männer in einer Stadt, einer reich, der andere arm.
\par 2 Der Reiche hatte sehr viele Schafe und Rinder;
\par 3 aber der Arme hatte nichts denn ein einziges kleines Schäflein, das er gekauft hatte. Und er nährte es, daß es groß ward bei ihm und bei seinen Kindern zugleich: es aß von seinem Bissen und trank von seinem Becher und schlief in seinem Schoß, und er hielt es wie eine Tochter.
\par 4 Da aber zu dem reichen Mann ein Gast kam, schonte er zu nehmen von seinen Schafen und Rindern, daß er dem Gast etwas zurichtete, der zu ihm gekommen war, und nahm das Schaf des armen Mannes und richtete es zu dem Mann, der zu ihm gekommen war.
\par 5 Da ergrimmte David mit großem Zorn wider den Mann und sprach zu Nathan: So wahr der HERR lebt, der Mann ist ein Kind des Todes, der das getan hat!
\par 6 Dazu soll er vierfältig bezahlen, darum daß er solches getan hat und nicht geschont hat.
\par 7 Da sprach Nathan zu David: Du bist der Mann! So spricht der HERR, der Gott Israels: Ich habe dich zum König gesalbt über Israel und habe dich errettet aus der Hand Sauls,
\par 8 und habe dir deines Herrn Haus gegeben, dazu seine Weiber in deinen Schoß, und habe dir das Haus Israel und Juda gegeben; und ist das zu wenig, will ich noch dies und das dazutun.
\par 9 Warum hast du denn das Wort des HERRN verachtet, daß du solches Übel vor seinen Augen tatest? Uria, den Hethiter, hast du erschlagen mit dem Schwert; sein Weib hast du dir zum Weib genommen; ihn aber hast du erwürgt mit dem Schwert der Kinder Ammon.
\par 10 Nun so soll von deinem Hause das Schwert nicht lassen ewiglich, darum daß du mich verachtet hast und das Weib Urias, des Hethiters, genommen hast, daß sie dein Weib sei.
\par 11 So spricht der HERR: Siehe, ich will Unglück über dich erwecken aus deinem eigenen Hause und will deine Weiber nehmen vor deinen Augen und will sie deinem Nächsten geben, daß er bei deinen Weibern schlafen soll an der lichten Sonne.
\par 12 Denn du hast es heimlich getan; ich aber will dies tun vor dem ganzen Israel und an der Sonne.
\par 13 Da sprach David zu Nathan: Ich habe gesündigt wider den HERRN. Nathan sprach zu David: So hat auch der HERR deine Sünde weggenommen; du wirst nicht sterben.
\par 14 Aber weil du die Feinde des HERRN hast durch diese Geschichte lästern gemacht, wird der Sohn, der dir geboren ist, des Todes sterben.
\par 15 Und Nathan ging heim. Und der HERR schlug das Kind, das Urias Weib David geboren hatte, daß es todkrank ward.
\par 16 Und David suchte Gott um des Knäbleins willen und fastete und ging hinein und lag über Nacht auf der Erde.
\par 17 Da standen auf die Ältesten seines Hauses und wollten ihn aufrichten von der Erde; er wollte aber nicht und aß auch nicht mit ihnen.
\par 18 Am siebenten Tage aber starb das Kind. Und die Knechte Davids fürchteten sich ihm anzusagen, daß das Kind tot wäre; denn sie gedachten: Siehe, da das Kind noch lebendig war, redeten wir mit ihm, und er gehorchte unsrer Stimme nicht; wie viel mehr wird er sich wehe tun, so wir sagen: Das Kind ist tot.
\par 19 Da aber David sah, daß seine Knechte leise redeten, und merkte, daß das Kind tot wäre, sprach er zu seinen Knechten: Ist das Kind tot? Sie sprachen: Ja.
\par 20 Da stand David auf von der Erde und wusch sich und salbte sich und tat andere Kleider an und ging in das Haus des HERRN und betete an. Und da er wieder heimkam, hieß er ihm Brot auftragen und aß.
\par 21 Da sprachen seine Knechte zu ihm: Was ist das für ein Ding, das du tust? Da das Kind lebte, fastetest du und weintest; aber nun es gestorben ist, stehst du auf und ißt?
\par 22 Er sprach: Um das Kind fastete ich und weinte, da es lebte; denn ich gedachte: Wer weiß, ob mir der HERR nicht gnädig wird, daß das Kind lebendig bleibe.
\par 23 Nun es aber tot ist, was soll ich fasten? Kann ich es auch wiederum holen? Ich werde wohl zu ihm fahren; es kommt aber nicht zu mir.
\par 24 Und da David sein Weib Bath-Seba getröstet hatte, ging er zu ihr hinein und schlief bei ihr. Und sie gebar ihm einen Sohn, den hieß er Salomo. Und der HERR liebte ihn.
\par 25 Und er tat ihn unter die Hand Nathans, des Propheten; der hieß ihn Jedidja, um des HERRN willen.
\par 26 So stritt nun Joab wider Rabba der Kinder Ammon königliche Stadt
\par 27 und sandte Boten zu David und ließ ihm sagen: Ich habe gestritten wider Rabba und habe auch gewonnen die Wasserstadt.
\par 28 So nimm nun zuhauf das übrige Volk und belagere die Stadt und gewinne sie, auf daß ich sie nicht gewinne und ich den Namen davon habe.
\par 29 Also nahm David alles Volk zuhauf und zog hin und stritt wider Rabba und gewann es
\par 30 und nahm die Krone seines Königs von seinem Haupt, die am Gewicht einen Zentner Gold hatte und Edelgesteine, und sie ward David auf sein Haupt gesetzt; und er führte aus der Stadt sehr viel Beute.
\par 31 Aber das Volk drinnen führte er heraus und legte sie unter eiserne Sägen und Zacken und eiserne Keile und verbrannte sie in Ziegelöfen. So tat er allen Städten der Kinder Ammon. Da kehrte David und alles Volk wieder gen Jerusalem.

\chapter{13}

\par 1 Und es begab sich darnach, daß Absalom, der Sohn Davids, hatte eine schöne Schwester, die hieß Thamar; und Amnon, der Sohn Davids, gewann sie lieb.
\par 2 Und dem Amnon ward wehe, als wollte er krank werden um Thamars, seiner Schwester, willen. Denn sie war eine Jungfrau, und es deuchte Amnon schwer sein, daß er ihr etwas sollte tun.
\par 3 Amnon aber hatte einen Freund, der hieß Jonadab, ein Sohn Simeas, Davids Bruders; und derselbe Jonadab war ein sehr weiser Mann.
\par 4 Der sprach zu ihm: Warum wirst du so mager, du Königssohn, von Tag zu Tag? Magst du mir's nicht ansagen? Da sprach Amnon zu ihm: Ich habe Thamar, meines Bruders Absalom Schwester, liebgewonnen.
\par 5 Jonadab sprach zu ihm: Lege dich auf dein Bett und stelle dich krank. Wenn dann dein Vater kommt, dich zu besuchen, so sprich zu ihm: Laß doch meine Schwester Thamar kommen, daß sie mir zu essen gebe und mache vor mir das Essen, daß ich zusehe und von ihrer Hand esse.
\par 6 Also legte sich Amnon und stellte sich krank. Da nun der König kam, ihn zu besuchen, sprach Amnon zum König: Laß doch meine Schwester Thamar kommen, daß sie vor mir einen Kuchen oder zwei mache und ich von ihrer Hand esse.
\par 7 Da sandte David nach Thamar ins Haus und ließ ihr sagen: Gehe hin ins Haus deines Bruders Amnon und mache ihm eine Speise.
\par 8 Thamar ging hin ins Haus ihres Bruders Amnon; er aber lag im Bett. Und sie nahm einen Teig und knetete und bereitete es vor seinen Augen und buk die Kuchen.
\par 9 Und sie nahm eine Pfanne und schüttete es vor ihm aus; aber er weigerte sich zu essen. Und Amnon sprach: Laßt jedermann von mir hinausgehen. Und es ging jedermann von ihm hinaus.
\par 10 Da sprach Amnon zu Thamar: Bringe das Essen in die Kammer, daß ich von deiner Hand esse. Da nahm Thamar die Kuchen, die sie gemacht hatte, und brachte sie zu Amnon, ihrem Bruder, in die Kammer.
\par 11 Und da sie es zu ihm brachte, daß er äße, ergriff er sie und sprach zu ihr: Komm her, meine Schwester, schlaf bei mir!
\par 12 Sie aber sprach zu ihm: Nicht, mein Bruder, schwäche mich nicht, denn so tut man nicht in Israel; tue nicht eine solche Torheit!
\par 13 Wo will ich mit meiner Schande hin? Und du wirst sein wie die Toren in Israel. Rede aber mit dem König; der wird mich dir nicht versagen.
\par 14 Aber er wollte nicht gehorchen und überwältigte sie und schwächte sie und schlief bei ihr.
\par 15 Und Amnon ward ihr überaus gram, daß der Haß größer war, denn vorhin die Liebe war. Und Amnon sprach zu ihr: Mache dich auf und hebe dich!
\par 16 Sie aber sprach zu ihm: Das Übel ist größer denn das andere, das du an mir getan hast, daß du mich ausstößest. Aber er gehorchte ihrer Stimme nicht,
\par 17 sondern rief seinen Knaben, der sein Diener war, und sprach: Treibe diese von mir hinaus und schließe die Tür hinter ihr zu!
\par 18 Und sie hatte einen bunten Rock an; denn solche Röcke trugen des Königs Töchter, welche Jungfrauen waren. Und da sie sein Diener hinausgetrieben und die Tür hinter ihr zugeschlossen hatte,
\par 19 warf Thamar Asche auf ihr Haupt und zerriß den bunten Rock, den sie anhatte, und legte ihre Hand auf das Haupt und ging daher und schrie.
\par 20 Und ihr Bruder Absalom sprach zu Ihr: Ist denn dein Bruder Amnon bei dir gewesen? Nun, meine Schwester, schweig still; es ist dein Bruder, und nimm die Sache nicht so zu Herzen. Also blieb Thamar einsam in Absaloms, ihres Bruders, Haus.
\par 21 Und da der König David solches alles hörte, ward er sehr zornig. Aber Absalom redete nicht mit Amnon, weder Böses noch Gutes;
\par 22 denn Absalom war Amnon gram, darum daß er seine Schwester Thamar geschwächt hatte.
\par 23 Über zwei Jahre aber hatte Absalom Schafscherer zu Baal-Hazor, das bei Ephraim liegt; und Absalom lud alle Kinder des Königs
\par 24 und kam zum König und sprach: Siehe, dein Knecht hat Schafscherer; der König wolle samt seinen Knechten mit seinem Knecht gehen.
\par 25 Der König aber sprach zu Absalom: Nicht, mein Sohn, laß uns nicht alle gehen, daß wir dich nicht beschweren. Und da er ihn nötigte, wollte er doch nicht gehen, sondern segnete ihn.
\par 26 Absalom sprach: Soll denn nicht mein Bruder Amnon mit uns gehen? Der König sprach zu ihm: Warum soll er mit dir gehen?
\par 27 Da nötigte ihn Absalom, daß er mit ihm ließ Amnon und alle Kinder des Königs.
\par 28 Absalom aber gebot seinen Leuten und sprach: Sehet darauf, wenn Amnon guter Dinge wird von dem Wein und ich zu euch spreche: Schlagt Amnon und tötet ihn, daß ihr euch nicht fürchtet; denn ich hab's euch geheißen. Seid getrost und frisch daran!
\par 29 Also taten die Leute Absaloms dem Amnon, wie ihnen Absalom geboten hatte. Da standen alle Kinder des Königs auf, und ein jeglicher setzte sich auf sein Maultier und flohen.
\par 30 Und da sie noch auf dem Wege waren, kam das Gerücht vor David, daß Absalom hätte alle Kinder des Königs erschlagen, daß nicht einer von ihnen übrig wäre.
\par 31 Da stand der König auf und zerriß seine Kleider und legte sich auf die Erde; und alle seine Knechte, die um ihn her standen, zerrissen ihre Kleider.
\par 32 Da hob Jonadab an, der Sohn Simeas, des Bruders Davids, und sprach: Mein Herr denke nicht, daß alle jungen Männer, die Kinder des Königs tot sind, sondern Amnon ist allein tot. Denn Absalom hat's bei sich behalten von dem Tage an, da er seine Schwester Thamar schwächte.
\par 33 So nehme nun mein Herr, der König, solches nicht zu Herzen, daß alle Kinder des Königs tot seien, sondern Amnon ist allein tot.
\par 34 Absalom aber floh. Und der Diener auf der Warte hob seine Augen auf und sah; und siehe, ein großes Volk kam auf dem Wege nacheinander an der Seite des Berges.
\par 35 Da sprach Jonadab zum König: Siehe, die Kinder des Königs kommen; wie dein Knecht gesagt hat, so ist's ergangen.
\par 36 Und da er hatte ausgeredet, siehe, da kamen die Kinder des Königs und hoben ihre Stimme auf und weinten. Der König und alle seine Knechte weinten auch gar sehr.
\par 37 Absalom aber floh und zog zu Thalmai, dem Sohn Ammihuds, dem König zu Gessur. Er aber trug Leid über seinen Sohn alle Tage.
\par 38 Da aber Absalom geflohen war und gen Gessur gezogen, blieb er daselbst drei Jahre.
\par 39 Und der König David hörte auf, auszuziehen wider Absalom; denn er hatte sich getröstet über Amnon, daß er tot war.

\chapter{14}

\par 1 Joab aber, der Zeruja Sohn, merkte, daß des Königs Herz war wider Absalom,
\par 2 und sandte hin gen Thekoa und ließ holen von dort ein kluges Weib und sprach zu ihr: Trage Leid und zieh Trauerkleider an und salbe dich nicht mit Öl, sondern stelle dich wie ein Weib, das eine lange Zeit Leid getragen hat über einen Toten;
\par 3 und sollst zum König hineingehen und mit ihm reden so und so. Und Joab gab ihr ein, was sie reden sollte.
\par 4 Und da das Weib von Thekoa mit dem König reden wollte, fiel sie auf ihr Antlitz zur Erde und beugte sich nieder und sprach: Hilf mir, König!
\par 5 Der König sprach zu ihr: Was ist dir? Sie sprach: Ach, ich bin eine Witwe, und mein Mann ist gestorben.
\par 6 Und deine Magd hatte zwei Söhne, die zankten miteinander auf dem Felde, und da kein Retter war, schlug einer den andern und tötete ihn.
\par 7 Und siehe, nun steht auf die ganze Freundschaft wider deine Magd und sagen: Gib her den, der seinen Bruder erschlagen hat, daß wir ihn töten für die Seele seines Bruders, den er erwürgt hat, und auch den Erben vertilgen; und wollen meinen Funken auslöschen, der noch übrig ist, daß meinem Mann kein Name und nichts Übriges bleibe auf Erden.
\par 8 Der König sprach zum Weibe: Gehe heim, ich will für dich gebieten.
\par 9 Und das Weib von Thekoa sprach zum König: Mein Herr König, die Missetat sei auf mir und meines Vaters Hause; der König aber und sein Stuhl sei unschuldig.
\par 10 Der König sprach: Wer wider dich redet, den bringe zu mir, so soll er nicht mehr dich antasten.
\par 11 Sie sprach: Der König gedenke an den HERRN, deinen Gott, daß der Bluträcher nicht noch mehr Verderben anrichte und sie meinen Sohn nicht vertilgen. Er sprach: So wahr der HERR lebt, es soll kein Haar von deinem Sohn auf die Erde fallen.
\par 12 Und das Weib sprach: Laß deine Magd meinem Herrn König etwas sagen. Er sprach: Sage an!
\par 13 Das Weib sprach: Warum bist du also gesinnt wider Gottes Volk? Denn da der König solches geredet hat, ist er wie ein Schuldiger, dieweil er seinen Verstoßenen nicht wieder holen läßt.
\par 14 Denn wir sterben eines Todes und sind wie Wasser, so in die Erde verläuft, das man nicht aufhält; und Gott will nicht das Leben wegnehmen, sondern bedenkt sich, daß nicht das Verstoßene auch von ihm verstoßen werde.
\par 15 So bin ich nun gekommen, mit meinem Herrn König solches zu reden; denn das Volk macht mir bang. Denn deine Magd gedachte: Ich will mit dem König reden; vielleicht wird er tun, was seine Magd sagt.
\par 16 Denn er wird seine Magd erhören, daß er mich errette von der Hand aller, die mich samt meinem Sohn vertilgen wollen vom Erbe Gottes.
\par 17 Und deine Magd gedachte: Meines Herrn, des Königs, Wort soll mir ein Trost sein; denn mein Herr, der König, ist wie ein Engel Gottes, daß er Gutes und Böses hören kann. Darum wird der HERR, dein Gott, mit dir sein.
\par 18 Der König antwortete und sprach zu dem Weibe: Leugne mir nicht, was ich dich frage. Das Weib sprach: Mein Herr, der König, rede.
\par 19 Der König sprach: Ist nicht die Hand Joabs mit dir in diesem allem? Das Weib antwortete und sprach: So wahr deine Seele lebt, mein Herr König, es ist nicht anders, weder zur Rechten noch zur Linken, denn wie mein Herr, der König, geredet hat. Denn dein Knecht Joab hat mir's geboten, und er hat solches alles seiner Magd eingegeben;
\par 20 daß ich diese Sache also wenden sollte, das hat dein Knecht Joab gemacht. Aber mein Herr ist weise wie die Weisheit eines Engels Gottes, daß er merkt alles auf Erden.
\par 21 Da sprach der König zu Joab: Siehe, ich habe solches getan; so gehe hin und bringe den Knaben Absalom wieder.
\par 22 Da fiel Joab auf sein Antlitz zur Erde und beugte sich nieder und dankte dem König und sprach: Heute merkt dein Knecht, daß ich Gnade gefunden habe vor deinen Augen, mein Herr König, da der König tut, was sein Knecht sagt.
\par 23 Also macht sich Joab auf und zog gen Gessur und brachte Absalom gen Jerusalem.
\par 24 Aber der König sprach: Laß ihn wider in sein Haus gehen und mein Angesicht nicht sehen. Also kam Absalom wieder in sein Haus und sah des Königs Angesicht nicht.
\par 25 Es war aber in ganz Israel kein Mann so schön wie Absalom, und er hatte dieses Lob vor allen; von seiner Fußsohle an bis auf seinen Scheitel war nicht ein Fehl an ihm.
\par 26 Und wenn man sein Haupt schor (das geschah gemeiniglich alle Jahre; denn es war ihm zu schwer, daß man's abscheren mußte), so wog sein Haupthaar zweihundert Lot nach dem königlichen Gewicht.
\par 27 Und Absalom wurden drei Söhne geboren und eine Tochter, die hieß Thamar und war ein Weib schön von Gestalt.
\par 28 Also blieb Absalom zwei Jahre zu Jerusalem, daß er des Königs Angesicht nicht sah.
\par 29 Und Absalom sandte nach Joab, daß er ihn zum König sendete; und er wollte nicht zu ihm kommen. Er aber sandte zum andernmal; immer noch wollte er nicht kommen.
\par 30 Da sprach er zu seinen Knechten: Seht das Stück Acker Joabs neben meinem, und er hat Gerste darauf; so geht hin und steckt es mit Feuer an. Da steckten die Knechte Absaloms das Stück mit Feuer an.
\par 31 Da machte sich Joab auf und kam zu Absalom ins Haus und sprach zu ihm: Warum haben deine Knechte mein Stück mit Feuer angesteckt?
\par 32 Absalom sprach zu Joab: Siehe, ich sandte nach dir und ließ dir sagen: Komm her, daß ich dich zum König sende und sagen lasse: Warum bin ich von Gessur gekommen? Es wäre mir besser, daß ich noch da wäre. So laß mich nun das Angesicht des Königs sehen; ist aber eine Missetat an mir, so töte mich.
\par 33 Und Joab ging hinein zum König und sagte es ihm an. Und er rief Absalom, daß er hinein zum König kam; und er fiel nieder vor dem König auf sein Antlitz zur Erde, und der König küßte Absalom.

\chapter{15}

\par 1 Und es begab sich darnach, daß Absalom ließ sich machen einen Wagen und Rosse und fünfzig Mann, die seine Trabanten waren.
\par 2 Auch machte sich Absalom des Morgens früh auf und trat an den Weg bei dem Tor. Und wenn jemand einen Handel hatte, daß er zum König vor Gericht kommen sollte, rief ihn Absalom zu sich und sprach: Aus welcher Stadt bist du? Wenn dann der sprach: Dein Knecht ist aus der Stämme Israels einem,
\par 3 so sprach Absalom zu ihm: Siehe, deine Sache ist recht und schlecht; aber du hast keinen, der dich hört, beim König.
\par 4 Und Absalom sprach: O, wer setzt mich zum Richter im Lande, daß jedermann zu mir käme, der eine Sache und Gerichtshandel hat, daß ich ihm hülfe!
\par 5 Und wenn jemand sich zu ihm tat, daß er wollte vor ihm niederfallen, so reckte er seine Hand aus und ergriff ihn und küßte ihn.
\par 6 Auf diese Weise tat Absalom dem ganzen Israel, wenn sie kamen vor Gericht zum König, und stahl also das Herz der Männer Israels.
\par 7 Nach vierzig Jahren sprach Absalom zum König: Ich will hingehen und mein Gelübde zu Hebron ausrichten, das ich dem HERRN gelobt habe.
\par 8 Denn dein Knecht tat ein Gelübde, da ich zu Gessur in Syrien wohnte, und sprach: Wenn mich der HERR wieder gen Jerusalem bringt, so will ich dem HERRN einen Gottesdienst tun.
\par 9 Der König sprach: Gehe hin mit Frieden. Und er machte sich auf und ging gen Hebron.
\par 10 Absalom aber hatte Kundschafter ausgesandt in alle Stämme Israels und lassen sagen: Wenn ihr der Posaune Schall hören werdet, so sprecht: Absalom ist König geworden zu Hebron.
\par 11 Es gingen aber mit Absalom zweihundert Mann von Jerusalem, die geladen waren; aber sie gingen in ihrer Einfalt und wußten nichts um die Sache.
\par 12 Absalom aber sandte auch nach Ahithophel, dem Giloniten, Davids Rat, aus seiner Stadt Gilo. Da er nun die Opfer tat, ward der Bund stark, und das Volk lief zu und mehrte sich mit Absalom.
\par 13 Da kam einer, der sagte es David an und sprach: Das Herz jedermanns in Israel folgt Absalom nach.
\par 14 David sprach aber zu allen seinen Knechten, die bei ihm waren zu Jerusalem: Auf, laßt uns fliehen! denn hier wird kein entrinnen sein vor Absalom; eilet, daß wir gehen, daß er uns nicht übereile und ergreife uns und treibe ein Unglück auf uns und schlage die Stadt mit der Schärfe des Schwerts.
\par 15 Da sprachen die Knechte des Königs zu ihm: Was mein Herr, der König, erwählt, siehe, hier sind deine Knechte.
\par 16 Und der König zog hinaus und sein ganzes Haus ihm nach. Er ließ aber zehn Kebsweiber zurück, das Haus zu bewahren.
\par 17 Und da der König und alles Volk, das ihm nachfolgte, hinauskamen, blieben sie stehen am äußersten Hause.
\par 18 Und alle seine Knechte gingen an ihm vorüber; dazu alle Krether und Plether und alle Gathiter, sechshundert Mann, die von Gath ihm nachgefolgt waren, gingen an dem König vorüber.
\par 19 Und der König sprach zu Itthai, dem Gathiter: Warum gehst du auch mit uns? Kehre um und bleibe bei dem König; denn du bist hier fremd und von deinem Ort gezogen hierher.
\par 20 Gestern bist du gekommen, und heute sollte ich dich mit uns hin und her ziehen lassen? Denn ich will gehen, wohin ich gehen kann. Kehre um und deine Brüder mit dir; dir widerfahre Barmherzigkeit und Treue.
\par 21 Itthai antwortete und sprach: So wahr der HERR lebt, und so wahr mein König lebt, an welchem Ort mein Herr, der König, sein wird, es gerate zum Tod oder zum Leben, da wird dein Knecht auch sein.
\par 22 David sprach zu Itthai: So komm und gehe mit. Also ging Itthai, der Gathiter, und alle seine Männer und der ganze Haufe Kinder, die mit ihm waren.
\par 23 Und das ganze Land weinte mit lauter Stimme, und alles Volk ging mit. Und der König ging über den Bach Kidron, und alles Volk ging vor auf dem Wege, der zur Wüste geht.
\par 24 Und siehe, Zadok war auch da und alle Leviten, die bei ihm waren, und trugen die Lade des Bundes und stellten sie dahin. Und Abjathar trat empor, bis daß alles Volk zur Stadt hinauskam.
\par 25 Aber der König sprach zu Zadok: Bringe die Lade Gottes wieder in die Stadt. Werde ich Gnade finden vor dem HERRN, so wird er mich wieder holen und wird mich sie sehen lassen und sein Haus.
\par 26 Spricht er aber also: Ich habe nicht Lust zu dir, siehe, hier bin ich. Er mache es mit mir, wie es ihm wohl gefällt.
\par 27 Und der König sprach zu dem Priester Zadok: O du Seher, kehre um wieder in die Stadt mit Frieden und mit euch eure beiden Söhne, Ahimaaz, dein Sohn, und Jonathan, der Sohn Abjathars!
\par 28 Siehe ich will verziehen auf dem blachen Felde in der Wüste, bis daß Botschaft von euch komme, und sage mir an.
\par 29 Also brachten Zadok und Abjathar die Lade Gottes wieder gen Jerusalem und blieben daselbst.
\par 30 David aber ging den Ölberg hinan und weinte, und sein Haupt war verhüllt, und er ging barfuß. Dazu alles Volk, das bei ihm war, hatte ein jeglicher sein Haupt verhüllt und gingen hinan und weinten.
\par 31 Und da es David angesagt ward, daß Ahithophel im Bund mit Absalom war, sprach er: HERR, mache den Ratschlag Ahithophels zur Narrheit!
\par 32 Und da David auf die Höhe kam, da man Gott pflegt anzubeten, siehe, da begegnete ihm Husai, der Arachiter, mit zerrissenem Rock und Erde auf seinem Haupt.
\par 33 Und David sprach zu ihm: Wenn du mit mir gehst, wirst du mir eine Last sein.
\par 34 Wenn du aber wieder in die Stadt gingest und sprächest zu Absalom: Ich bin dein Knecht, ich will des Königs sein; der ich deines Vaters Knecht war zu der Zeit, will nun dein Knecht sein: So würdest du mir zugut den Ratschlag Ahithophels zunichte machen.
\par 35 Auch sind Zadok und Abjathar, die Priester, mit dir. Alles, was du hörst aus des Königs Haus, würdest du ansagen den Priestern Zadok und Abjathar.
\par 36 Siehe, es sind bei ihnen ihre zwei Söhne: Ahimaaz, Zadoks, und Jonathan, Abjathars Sohn. Durch die kannst du mir entbieten, was du hören wirst.
\par 37 Also kam Husai, der Freund Davids, in die Stadt; und Absalom kam gen Jerusalem.

\chapter{16}

\par 1 Und da David ein wenig von der Höhe gegangen war, siehe, da begegnete ihm Ziba, der Diener Mephiboseths, mit einem Paar Esel, gesattelt, darauf waren zweihundert Brote und hundert Rosinenkuchen und hundert Feigenkuchen und ein Krug Wein.
\par 2 Da sprach der König zu Ziba: Was willst du damit machen? Ziba sprach: Die Esel sollen für das Haus des Königs sein, darauf zu reiten, und die Brote und Feigenkuchen für die Diener, zu essen, und der Wein, zu trinken, wenn sie müde werden in der Wüste.
\par 3 Der König sprach: Wo ist der Sohn deines Herrn? Ziba sprach zum König: Siehe, er blieb zu Jerusalem; denn er sprach: Heute wird mir das Haus Israel meines Vaters Reich wiedergeben.
\par 4 Der König sprach zu Ziba: Siehe, es soll dein sein alles, was Mephiboseth hat. Ziba sprach: Ich neige mich; laß mich Gnade finden vor dir, mein Herr König.
\par 5 Da aber der König bis gen Bahurim kam, siehe, da ging ein Mann daselbst heraus, vom Geschlecht des Hauses Sauls, der hieß Simei, der Sohn Geras; der ging heraus und fluchte
\par 6 und warf David mit Steinen und alle Knechte des Königs David. Denn alles Volk und alle Gewaltigen waren zu seiner Rechten und zur Linken.
\par 7 So sprach aber Simei, da er fluchte: Heraus, heraus, du Bluthund, du heilloser Mann!
\par 8 Der HERR hat dir vergolten alles Blut des Hauses Sauls, daß du an seiner Statt bist König geworden. Nun hat der HERR das Reich gegeben in die Hand deines Sohnes Absalom; und siehe, nun steckst du in deinem Unglück; denn du bist ein Bluthund.
\par 9 Aber Abisai, der Zeruja Sohn, sprach zu dem König: Sollte dieser tote Hund meinem Herrn, dem König, fluchen? Ich will hingehen und ihm den Kopf abreißen.
\par 10 Der König sprach: Ihr Kinder der Zeruja, was habe ich mit euch zu schaffen? Laßt ihn fluchen; denn der HERR hat's ihn geheißen: Fluche David! Wer kann nun sagen: Warum tust du also?
\par 11 Und David sprach zu Abisai und zu allen seinen Knechten: Siehe, mein Sohn, der von meinem Leibe gekommen ist, steht mir nach meinem Leben; warum nicht auch jetzt der Benjaminiter? Laßt ihn, daß er fluche; denn der HERR hat's ihn geheißen.
\par 12 Vielleicht wird der HERR mein Elend ansehen und mir mit Gutem vergelten sein heutiges Fluchen.
\par 13 Also ging David mit seinen Leuten des Weges; Aber Simei ging an des Berges Seite her ihm gegenüber und fluchte und warf mit Steinen nach ihm und besprengte ihn mit Erdklößen.
\par 14 Und der König kam hinein mit allem Volk, das bei ihm war, müde und erquickte sich daselbst.
\par 15 Aber Absalom und alles Volk der Männer Israels kamen gen Jerusalem und Ahithophel mit ihm.
\par 16 Da aber Husai, der Arachiter, Davids Freund, zu Absalom hineinkam, sprach er zu Absalom: Glück zu, Herr König! Glück zu, Herr König!
\par 17 Absalom aber sprach zu Husai: Ist das deine Barmherzigkeit an deinem Freunde? Warum bist du nicht mit deinem Freunde gezogen?
\par 18 Husai aber sprach zu Absalom: Nicht also, sondern welchen der HERR erwählt und dies Volk und alle Männer in Israel, des will ich sein und bei ihm bleiben.
\par 19 Zum andern, wem sollte ich dienen? Sollte ich nicht vor seinem Sohn dienen? Wie ich vor deinem Vater gedient habe, so will ich auch vor dir sein.
\par 20 Und Absalom sprach zu Ahithophel: Ratet zu, was sollen wir tun?
\par 21 Ahithophel sprach zu Absalom: Gehe hinein zu den Kebsweibern deines Vaters, die er zurückgelassen hat, das Haus zu bewahren, so wird das ganze Israel hören, daß du dich bei deinem Vater hast stinkend gemacht, und wird aller Hand, die bei dir sind, desto kühner werden.
\par 22 Da machten sie Absalom eine Hütte auf dem Dache, und Absalom ging hinein zu den Kebsweibern seines Vaters vor den Augen des ganzen Israel.
\par 23 Zu der Zeit, wenn Ahithophel einen Rat gab, das war, als wenn man Gott um etwas hätte gefragt; also waren alle Ratschläge Ahithophels bei David und bei Absalom.

\chapter{17}

\par 1 Und Ahithophel sprach zu Absalom: Ich will zwölftausend Mann auslesen und mich aufmachen und David nachjagen bei der Nacht
\par 2 und will ihn überfallen, weil er matt und laß ist. Wenn ich ihn dann erschrecke, daß alles Volk, das bei ihm ist, flieht, will ich den König allein schlagen
\par 3 und alles Volk wieder zu dir bringen. Wenn dann jedermann zu dir gebracht ist, wie du begehrst, so bleibt alles Volk mit Frieden.
\par 4 Das deuchte Absalom gut und alle Ältesten in Israel.
\par 5 Aber Absalom sprach: Laßt doch Husai, den Arachiten, auch rufen und hören, was er dazu sagt.
\par 6 Und da Husai hinein zu Absalom kam, sprach Absalom zu ihm: Solches hat Ahithophel geredet; sage du, sollen wir's tun oder nicht?
\par 7 Da sprach Husai zu Absalom: Es ist nicht ein guter Rat, den Ahithophel auf diesmal gegeben hat.
\par 8 Und Husai sprach weiter: Du kennst deinen Vater wohl und seine Leute, daß sie stark sind und zornigen Gemüts wie ein Bär auf dem Felde, dem die Jungen geraubt sind; dazu ist dein Vater ein Kriegsmann und wird sich nicht säumen mit dem Volk.
\par 9 Siehe, er hat sich jetzt vielleicht verkrochen irgend in einer Grube oder sonst an einen Ort. Wenn's dann geschähe, daß es das erstemal übel geriete und käme ein Geschrei und spräche: Es ist das Volk, welches Absalom nachfolgt, geschlagen worden,
\par 10 so würde jedermann verzagt werden, der auch sonst ein Krieger ist und ein Herz hat wie ein Löwe. Denn es weiß ganz Israel, daß dein Vater stark ist und Krieger, die bei ihm sind.
\par 11 Aber das rate ich, daß du zu dir versammlest ganz Israel von Dan an bis gen Beer-Seba, so viel als der Sand am Meer, und deine Person ziehe unter ihnen.
\par 12 So wollen wir ihn überfallen, an welchem Ort wir ihn finden, und wollen über ihn kommen, wie der Tau auf die Erde fällt, daß wir von ihm und allen seinen Männern nicht einen übriglassen.
\par 13 Wird er sich aber in eine Stadt versammeln, so soll das ganze Israel Stricke an die Stadt werfen und sie in den Bach reißen, daß man nicht ein Kieselein da finde.
\par 14 Da sprach Absalom und jedermann in Israel: Der Rat Husais, des Arachiten, ist besser denn Ahithophels Rat. Aber der HERR schickte es also, daß der gute Rat Ahithophels verhindert wurde, auf daß der HERR Unheil über Absalom brächte.
\par 15 Und Husai sprach zu Zadok und Abjathar, den Priestern: So und so hat Ahithophel Absalom und den Ältesten in Israel geraten; ich aber habe so und so geraten.
\par 16 So sendet nun eilend hin und lasset David ansagen und sprecht: Bleibe nicht auf dem blachen Felde der Wüste, sondern mache dich hinüber, daß der König nicht verschlungen werde und alles Volk, das bei ihm ist.
\par 17 Jonathan aber und Ahimaaz standen bei dem Brunnen Rogel, und eine Magd ging hin und sagte es ihnen an. Sie aber gingen hin und sagten es dem König David an; denn sie durften sich nicht sehen lassen, daß sie in die Stadt kämen.
\par 18 Es sah sie aber ein Knabe und sagte es Absalom an. Aber die beiden gingen eilend hin und kamen in eines Mannes Haus zu Bahurim; der hatte einen Brunnen in seinem Hofe. Dahinein stiegen sie,
\par 19 und das Weib nahm und breitete eine Decke über des Brunnens Loch und breitete Grütze darüber, daß man es nicht merkte.
\par 20 Da nun die Knechte Absaloms zum Weibe ins Haus kamen, sprachen sie: Wo ist Ahimaaz und Jonathan? Das Weib sprach zu ihnen: Sie gingen über das Wässerlein. Und da sie suchten, und nicht fanden, gingen sie wieder gen Jerusalem.
\par 21 Und da sie weg waren, stiegen jene aus dem Brunnen und gingen hin und sagten's David, dem König, an und sprachen zu David: Macht euch auf und geht eilend über das Wasser; denn so und so hat Ahithophel wider euch Rat gegeben.
\par 22 Da machte sich David auf und alles Volk, das bei ihm war, und gingen über den Jordan, bis es lichter Morgen ward, und fehlte nicht an einem, der nicht über den Jordan gegangen wäre.
\par 23 Als aber Ahithophel sah, daß sein Rat nicht ausgeführt ward, sattelte er seinen Esel, machte sich auf und zog heim in seine Stadt und beschickte sein Haus und erhängte sich und starb und ward begraben in seines Vaters Grab.
\par 24 Und David kam gen Mahanaim. Und Absalom zog über den Jordan und alle Männer Israels mit ihm.
\par 25 Und Absalom hatte Amasa an Joabs Statt gesetzt über das Heer. Es war aber Amasa eines Mannes Sohn, der hieß Jethra, ein Israeliter, welcher einging zu Abigail, der Tochter des Nahas, der Schwester der Zeruja, Joabs Mutter.
\par 26 Israel aber und Absalom lagerten sich in Gilead.
\par 27 Da David gen Mahanaim gekommen war, da brachten Sobi, der Sohn Nahas von Rabba der Kinder Ammon, und Machir, der Sohn Ammiels von Lo-Dabar, und Barsillai, ein Gileaditer von Roglim,
\par 28 Bettwerk, Becken, irdene Gefäße, Weizen, Gerste, Mehl, geröstete Körner, Bohnen, Linsen, Grütze,
\par 29 Honig, Butter, Schafe und Rinderkäse zu David und zu dem Volk, das bei ihm war, zu essen. Denn sie gedachten: Das Volk wird hungrig, müde und durstig sein in der Wüste.

\chapter{18}

\par 1 Und David ordnete das Volk, das bei ihm war, und setzte über sie Hauptleute, über tausend und über hundert,
\par 2 und stellte des Volkes einen dritten Teil unter Joab und einen dritten Teil unter Abisai, den Sohn der Zeruja, Joabs Bruder, und einen dritten Teil unter Itthai, den Gathiter. Und der König sprach zum Volk: Ich will auch mit euch ausziehen.
\par 3 Aber das Volk sprach: Du sollst nicht ausziehen; denn ob wir gleich fliehen oder die Hälfte sterben, so werden sie unser nicht achten; denn du bist wie unser zehntausend; so ist's nun besser, daß du uns von der Stadt aus helfen mögst.
\par 4 Der König sprach zu ihnen: Was euch gefällt, das will ich tun. Und der König trat ans Tor, und alles Volk zog aus bei Hunderten und bei Tausenden.
\par 5 Und der König gebot Joab und Abisai und Itthai und sprach: Fahrt mir säuberlich mit dem Knaben Absalom! Und alles Volk hörte es, da der König gebot allen Hauptleuten um Absalom.
\par 6 Und da das Volk hinauskam aufs Feld, Israel entgegen, erhob sich der Streit im Walde Ephraim.
\par 7 Und das Volk Israel ward daselbst geschlagen vor den Knechten Davids, daß desselben Tages eine große Schlacht geschah, zwanzigtausend Mann.
\par 8 Und war daselbst der Streit zerstreut auf allem Lande; und der Wald fraß viel mehr Volk des Tages, denn das Schwert fraß.
\par 9 Und Absalom begegnete den Knechten Davids und ritt auf einem Maultier. Und da das Maultier unter eine große Eiche mit dichten Zweigen kam, blieb sein Haupt an der Eiche hangen, und er schwebte zwischen Himmel und Erde; aber sein Maultier lief unter ihm weg.
\par 10 Da das ein Mann sah, sagte er's Joab an und sprach: Siehe, ich sah Absalom an einer Eiche hangen.
\par 11 Und Joab sprach zu dem Mann, der's ihm hatte angesagt: Siehe, sahst du das, warum schlugst du ihn nicht daselbst zur Erde? so wollte ich dir von meinetwegen zehn Silberlinge und einen Gürtel gegeben haben.
\par 12 Der Mann sprach zu Joab: Wenn du mir tausend Silberlinge in meine Hand gewogen hättest, so wollte ich dennoch meine Hand nicht an des Königs Sohn gelegt haben; denn der König gebot dir und Abisai und Itthai vor unsern Ohren und sprach: Hütet euch, daß nicht jemand dem Knaben Absalom...!
\par 13 Oder wenn ich etwas Falsches getan hätte auf meiner Seele Gefahr, weil dem König nichts verhohlen wird, würdest du selbst wider mich gestanden sein.
\par 14 Joab sprach: Ich kann nicht so lange bei dir verziehen. Da nahm Joab drei Spieße in sein Hand und stieß sie Absalom ins Herz, da er noch lebte an der Eiche.
\par 15 Und zehn Knappen, Joabs Waffenträger, machten sich umher und schlugen ihn zu Tod.
\par 16 Da blies Joab die Posaune und brachte das Volk wieder, daß es nicht weiter Israel nachjagte; denn Joab wollte das Volk schonen.
\par 17 Und sie nahmen Absalom und warfen ihn in den Wald in eine große Grube und legten einen sehr großen Haufen Steine auf ihn. Und das ganze Israel floh, ein jeglicher in seine Hütte.
\par 18 Absalom aber hatte sich eine Säule aufgerichtet, da er noch lebte; die steht im Königsgrunde. Denn er sprach: Ich habe keinen Sohn, darum soll dies meines Namens Gedächtnis sein; er hieß die Säule nach seinem Namen, und sie heißt auch bis auf diesen Tag Absaloms Mal.
\par 19 Ahimaaz, der Sohn Zadoks, sprach: Laß mich doch laufen und dem König verkündigen, daß der HERR ihm Recht verschafft hat von seiner Feinde Händen.
\par 20 Joab aber sprach zu ihm: Du bringst heute keine gute Botschaft. Einen andern Tag sollst du Botschaft bringen, und heute nicht; denn des Königs Sohn ist tot.
\par 21 Aber zu Chusi sprach Joab: Gehe hin und sage dem König an, was du gesehen hast. Und Chusi neigte sich vor Joab und lief.
\par 22 Ahimaaz aber, der Sohn Zadoks, sprach abermals zu Joab: Wie, wenn ich auch liefe dem Chusi nach? Joab sprach: Was willst du laufen, Mein Sohn? Komm her, die Botschaft wird dir nichts einbringen.
\par 23 Wie wenn ich liefe? Er sprach zu ihm: So laufe doch! Also lief Ahimaaz geradewegs und kam Chusi vor.
\par 24 David aber saß zwischen beiden Toren. Und der Wächter ging aufs Dach des Tors an der Mauer und hob seine Augen auf und sah einen Mann laufen allein
\par 25 und rief und sagte es dem König an. Der König aber sprach: Ist er allein, so ist eine gute Botschaft in seinem Munde. Und da derselbe immer näher kam,
\par 26 sah der Wächter einen andern Mann laufen, und rief in das Tor und sprach: Siehe, ein Mann läuft allein. Der König aber sprach: Der ist auch ein guter Bote.
\par 27 Der Wächter sprach: Ich sehe des ersten Lauf wie den Lauf des Ahimaaz, des Sohnes Zadoks. Und der König sprach: Es ist ein guter Mann und bringt eine gute Botschaft.
\par 28 Ahimaaz aber rief und sprach zum König: Friede! Und fiel nieder vor dem König auf sein Antlitz zur Erde und sprach: Gelobt sei der HERR, dein Gott, der die Leute, die ihre Hand wider meinen Herrn, den König, aufhoben, übergeben hat.
\par 29 Der König aber sprach: Geht es auch wohl dem Knaben Absalom? Ahimaaz sprach: Ich sah ein großes Getümmel, da des Königs Knecht Joab mich, deinen Knecht, sandte, und weiß nicht, was es war.
\par 30 Der König sprach: Gehe herum und tritt daher. Und er ging herum und stand allda.
\par 31 Siehe, da kam Chusi und sprach: Hier gute Botschaft, mein Herr König! Der HERR hat dir heute Recht verschafft von der Hand aller, die sich wider dich auflehnten.
\par 32 Der König aber sprach zu Chusi: Geht es dem Knaben Absalom auch wohl? Chusi sprach: Es müsse allen Feinden meines Herrn Königs gehen, wie es dem Knaben geht, und allen, die sich wider ihn auflehnen, übel zu tun.
\par 33 Da ward der König traurig und ging hinauf auf den Söller im Tor und weinte, und im Gehen sprach er also: Mein Sohn Absalom! mein Sohn, mein Sohn Absalom! Wollte Gott, ich wäre für dich gestorben! O Absalom, mein Sohn, mein Sohn!

\chapter{19}

\par 1 Und es ward Joab angesagt: Siehe, der König weint und trägt Leid um Absalom.
\par 2 Und es ward aus dem Sieg des Tages ein Leid unter dem ganzen Volk; denn das Volk hatte gehört des Tages, daß sich der König um seinen Sohn bekümmerte.
\par 3 Und das Volk stahl sich weg an dem Tage in die Stadt, wie sich ein Volk wegstiehlt, das zu Schanden geworden ist, wenn's im Streit geflohen ist.
\par 4 Der König aber hatte sein Angesicht verhüllt und schrie laut: Ach, mein Sohn Absalom! Absalom, mein Sohn, mein Sohn!
\par 5 Joab aber kam zum König ins Haus und sprach: Du hast heute schamrot gemacht alle deine Knechte, die heute deine, deiner Söhne, deiner Töchter, deiner Weiber und deiner Kebsweiber Seele errettet haben,
\par 6 daß du liebhast, die dich hassen, und haßt, die dich liebhaben. Denn du läßt heute merken, daß dir's nicht gelegen ist an den Hauptleuten und Knechten. Denn ich merke heute wohl: wenn dir nur Absalom lebte und wir heute alle tot wären, das wäre dir recht.
\par 7 So mache dich nun auf und gehe heraus und rede mit deinen Knechten freundlich. Denn ich schwöre dir bei dem HERRN: Wirst du nicht herausgehen, es wird kein Mann bei dir bleiben diese Nacht über. Das wird dir ärger sein denn alles Übel, das über dich gekommen ist von deiner Jugend auf bis hierher.
\par 8 Da machte sich der König auf und setzte sich ins Tor. Und man sagte es allem Volk: Siehe, der König sitzt im Tor. Da kam alles Volk vor den König. Aber Israel war geflohen, ein jeglicher in seine Hütte.
\par 9 Und es zankte sich alles Volk in allen Stämmen Israels und sprachen: Der König hat uns errettet von der Hand unsrer Feinde und erlöste uns von der Philister Hand und hat müssen aus dem Lande fliehen vor Absalom.
\par 10 So ist Absalom, den wir über uns gesalbt hatten, gestorben im Streit. Warum seid ihr so nun still, daß ihr den König nicht wieder holet?
\par 11 Der König aber sandte zu Zadok und Abjathar, den Priestern, und ließ ihnen sagen: Redet mit den Ältesten in Juda und sprecht: Warum wollt ihr die letzten sein, den König wieder zu holen in sein Haus? (Denn die Rede des ganzen Israel war vor den König gekommen in sein Haus.)
\par 12 Ihr seid meine Brüder, mein Bein und mein Fleisch; warum wollt ihr denn die letzten sein, den König wieder zu holen?
\par 13 Und zu Amasa sprecht: Bist du nicht mein Bein und mein Fleisch? Gott tue mir dies und das, wo du nicht sollst sein Feldhauptmann vor mir dein Leben lang an Joabs Statt.
\par 14 Und er neigte das Herz aller Männer Juda's wie eines Mannes; und sie sandten hin zum König: Komm wieder, du und alle deine Knechte!
\par 15 Also kam der König wieder. Und da er an den Jordan kam, waren die Männer Juda's gen Gilgal gekommen, hinabzuziehen dem König entgegen, daß sie den König über den Jordan führten.
\par 16 Und Simei, der Sohn Geras, der Benjaminiter, der zu Bahurim wohnte, eilte und zog mit den Männern Juda's hinab, dem König David entgegen;
\par 17 und waren tausend Mann mit ihm von Benjamin, dazu auch Ziba, der Diener des Hauses Sauls, mit seinen fünfzehn Söhnen und zwanzig Knechten; und sie gingen durch den Jordan vor dem König hin;
\par 18 und die Fähre war hinübergegangen, daß sie das Gesinde des Königs hinüberführten und täten, was ihm gefiel. Simei aber, der Sohn Geras, fiel vor dem König nieder, da er über den Jordan fuhr,
\par 19 und sprach zum König: Mein Herr, rechne mir nicht zu die Missetat und gedenke nicht, daß dein Knecht dich beleidigte des Tages, da mein Herr, der König, aus Jerusalem ging, und der König nehme es nicht zu Herzen.
\par 20 Denn dein Knecht erkennt, daß ich gesündigt habe. Und siehe, ich bin heute zuerst gekommen unter dem ganzen Hause Joseph, daß ich meinen Herrn, dem König, entgegen herabzöge.
\par 21 Aber Abisai, der Zeruja Sohn, antwortete und sprach: Und Simei sollte darum nicht sterben, so er doch dem Gesalbten des HERRN geflucht hat?
\par 22 David aber sprach: Was habe ich mit euch zu schaffen, ihr Kinder der Zeruja, daß ihr mir heute wollt zum Satan werden? Sollte heute jemand sterben in Israel? Meinst du, ich wisse nicht, daß ich heute bin König geworden über Israel?
\par 23 Und der König sprach zu Simei: Du sollst nicht sterben. Und der König schwur ihm.
\par 24 Mephiboseth, der Sohn Sauls, kam auch herab, dem König entgegen. Und er hatte seine Füße und seinen Bart nicht gereinigt und seine Kleider nicht gewaschen von dem Tage an, da der König weggegangen war, bis an den Tag, da er mit Frieden kam.
\par 25 Da er nun von Jerusalem kam, dem König zu begegnen, sprach der König zu ihm: Warum bist du nicht mit mir gezogen, Mephiboseth?
\par 26 Und er sprach: Mein Herr König, mein Knecht hat mich betrogen. Denn dein Knecht gedachte, ich will einen Esel satteln und darauf reiten und zum König ziehen, denn dein Knecht ist lahm.
\par 27 Dazu hat er deinen Knecht angegeben vor meinem Herrn, dem König. Aber mein Herr, der König, ist wie ein Engel Gottes; tue, was dir wohl gefällt.
\par 28 Denn all meines Vaters Haus ist nichts gewesen als Leute des Todes vor meinem Herrn, dem König; so hast du deinen Knecht gesetzt unter die, so an deinem Tisch essen. Was habe ich weiter Gerechtigkeit oder weiter zu schreien zu dem König?
\par 29 Der König sprach zu ihm: Was redest du noch weiter von deinem Dinge? Ich habe es gesagt: Du und Ziba teilt den Acker miteinander.
\par 30 Mephiboseth sprach zum König: Er nehme ihn auch ganz dahin, nachdem mein Herr König mit Frieden heimgekommen ist.
\par 31 Und Barsillai, der Gileaditer, kam herab von Roglim und führte den König über den Jordan, daß er ihn über den Jordan geleitete.
\par 32 Und Barsillai war sehr alt, wohl achtzig Jahre, der hatte den König versorgt, als er zu Mahanaim war; denn er war ein Mann von großem Vermögen.
\par 33 Und der König sprach zu Barsillai: Du sollst mit mir hinüberziehen; ich will dich versorgen bei mir zu Jerusalem.
\par 34 Aber Barsillai sprach zum König: was ist's noch, das ich zu leben habe, daß ich mit dem König sollte hinauf gen Jerusalem ziehen?
\par 35 Ich bin heute achtzig Jahre alt. Wie sollte ich kennen, was gut oder böse ist, oder schmecken, was ich esse oder trinke, oder hören, was die Sänger oder Sängerinnen singen? Warum sollte dein Knecht meinem Herrn König weiter beschweren?
\par 36 Dein Knecht soll ein wenig gehen mit dem König über den Jordan. Warum will mir der König eine solche Vergeltung tun?
\par 37 Laß deinen Knecht umkehren, daß ich sterbe in meiner Stadt bei meines Vaters und meiner Mutter Grab. Siehe, da ist dein Knecht Chimham; den laß mit meinem Herrn König hinüberziehen, und tue ihm, was dir wohl gefällt.
\par 38 Der König sprach: Chimham soll mit mir hinüberziehen, und ich will ihm tun, was dir wohl gefällt; auch alles, was du von mir begehrst, will ich dir tun.
\par 39 Und da alles Volk über den Jordan war gegangen und der König auch, küßte der König den Barsillai und segnete ihn; und er kehrte wieder an seinen Ort.
\par 40 Und der König zog hinüber gen Gilgal, und Chimham zog mit ihm. Und alles Volk Juda hatte den König hinübergeführt; aber des Volkes Israel war nur die Hälfte da.
\par 41 Und siehe, da kamen alle Männer Israels zum König und sprachen zu ihm: Warum haben dich unsre Brüder, die Männer Juda's, gestohlen und haben den König und sein Haus über den Jordan geführt und alle Männer Davids mit ihm?
\par 42 Da antworteten die von Juda denen von Israel: Der König gehört uns nahe zu; was zürnt ihr darum? Meint ihr, daß wir von dem König Nahrung und Geschenke empfangen haben?
\par 43 So antworteten dann die von Israel denen von Juda und sprachen: Wir haben zehnmal mehr beim König, dazu auch bei David, denn ihr. Warum hast du mich denn so gering geachtet? Und haben wir nicht zuerst davon geredet, uns unsern König zu holen? Aber die von Juda redeten härter denn die von Israel.

\chapter{20}

\par 1 Es traf sich aber, daß daselbst ein heilloser Mann war, der hieß Seba, ein Sohn Bichris, ein Benjaminiter; der blies die Posaune und sprach: Wir haben keinen Teil an David noch Erbe am Sohn Isais. Ein jeglicher hebe sich zu seiner Hütte, o Israel!
\par 2 Da fiel von David jedermann in Israel, und sie folgten Seba, dem Sohn Bichris. Aber die Männer Juda's hingen an ihrem König vom Jordan an bis gen Jerusalem.
\par 3 Da aber der König David heimkam gen Jerusalem, nahm er die zehn Kebsweiber, die er hatte zurückgelassen, das Haus zu bewahren, und tat sie in eine Verwahrung und versorgte sie; aber er ging nicht zu ihnen ein. Und sie waren also verschlossen bis an ihren Tod und lebten als Witwen.
\par 4 Und der König sprach zu Amasa: Berufe mir alle Männer in Juda auf den dritten Tag, und du sollst auch hier stehen!
\par 5 Und Amasa ging hin, Juda zu berufen; aber er verzog die Zeit, die er ihm bestimmt hatte.
\par 6 Da sprach David zu Abisai: Nun wird uns Seba, der Sohn Bichris, mehr Leides tun denn Absalom. Nimm du die Knechte deines Herrn und jage ihm nach, daß er nicht etwa für sich feste Städte finde und entrinne aus unsern Augen.
\par 7 Da zogen aus, ihm nach, die Männer Joabs, dazu die Krether und Plether und alle Starken. Sie zogen aber aus von Jerusalem, nachzujagen Seba, dem Sohn Bichris.
\par 8 Da sie aber bei dem großen Stein waren zu Gibeon, kam Amasa vor ihnen her. Joab aber war gegürtet über seinem Kleide, das er anhatte, und hatte darüber ein Schwert gegürtet, das hing in seiner Hüfte in der Scheide; das ging gerne aus und ein.
\par 9 Und Joab sprach zu Amasa: Friede sei mit dir, mein Bruder! Und Joab faßte mit seiner rechten Hand Amasa bei dem Bart, daß er ihn küßte.
\par 10 Und Amasa hatte nicht acht auf das Schwert in der Hand Joabs; und er stach ihn damit in den Bauch, daß sein Eingeweide sich auf die Erde schüttete, und gab ihm keinen Stich mehr und er starb. Joab aber und sein Bruder Abisai jagten nach Seba, dem Sohn Bichris.
\par 11 Und es trat ein Mann von den Leuten Joabs neben ihn und sprach: Wer's mit Joab hält und für David ist, der folge Joab nach!
\par 12 Amasa aber lag im Blut gewälzt mitten auf der Straße. Da aber der Mann sah, daß alles Volk da stehenblieb, wandte er Amasa von der Straße auf den Acker und warf Kleider auf ihn, weil er sah, daß, wer an ihn kam, stehenblieb.
\par 13 Da er nun aus der Straße getan war, folgte jedermann Joab nach, Seba, dem Sohn Bichris, nachzujagen.
\par 14 Und er zog durch alle Stämme Israels gen Abel und Beth-Maacha und ganz Haberim; und versammelten sich und folgten ihm nach
\par 15 und kamen und belagerten ihn zu Abel-Beth-Maacha und schütteten einen Wall gegen die Stadt hin, daß er bis an die Vormauer langte; und alles Volk, das mit Joab war, stürmte und wollte die Mauer niederwerfen.
\par 16 Da rief eine weise Frau aus der Stadt: Hört! hört! Sprecht zu Joab, daß er hierher komme; ich will mit ihm reden.
\par 17 Und da er zu ihr kam, sprach die Frau: Bist du Joab? Er sprach: Ja. Sie sprach zu ihm: Höre die Rede deiner Magd. Er sprach: Ich höre.
\par 18 Sie sprach: Vorzeiten sprach man: Wer fragen will, der frage zu Abel; und so ging's wohl aus.
\par 19 Ich bin eine von den friedsamen und treuen Städten in Israel; und du willst die Stadt und Mutter in Israel töten? Warum willst du das Erbteil des HERRN verschlingen?
\par 20 Joab antwortete und sprach: Das sei ferne, das sei ferne von mir, daß ich verschlingen und verderben sollte! Es steht nicht also;
\par 21 sondern ein Mann vom Gebirge Ephraim mit Namen Seba, der Sohn Bichris, hat sich empört wider den König David. Gebt ihn allein her, so will ich von der Stadt ziehen. Die Frau aber sprach zu Joab: Siehe, sein Haupt soll zu dir über die Mauer geworfen werden.
\par 22 Und die Frau kam hinein zu allem Volk mit ihrer Weisheit. Und sie hieben Seba, dem Sohn Bichris, den Kopf ab und warfen ihn zu Joab. Da blies er die Posaune, und sie zerstreuten sich von der Stadt, ein jeglicher in seine Hütte. Joab aber kam wieder gen Jerusalem zum König.
\par 23 Joab aber war über das ganze Heer Israels. Benaja, der Sohn Jojadas, war über die Krether und Plether.
\par 24 Adoram war Rentmeister. Josaphat, der Sohn Ahiluds, war Kanzler.
\par 25 Seja war Schreiber. Zadok und Abjathar waren Priester;
\par 26 dazu war Ira, der Jairiter, Davids Priester.

\chapter{21}

\par 1 Es war auch eine Teuerung zu Davids Zeiten drei Jahre aneinander. Und David suchte das Angesicht des HERRN; und der HERR sprach: Um Sauls willen und um des Bluthauses willen, daß er die Gibeoniter getötet hat.
\par 2 Da ließ der König die Gibeoniter rufen und sprach zu ihnen. (Die Gibeoniter aber waren nicht von den Kindern Israel, sondern übrig von den Amoritern; aber die Kinder Israel hatten ihnen geschworen, und Saul suchte sie zu schlagen in seinem Eifer für die Kinder Israel und Juda.)
\par 3 So sprach nun David zu den Gibeonitern: Was soll ich euch tun? und womit soll ich sühnen, daß ihr das Erbteil des HERRN segnet?
\par 4 Die Gibeoniter sprachen zu ihm: Es ist uns nicht um Gold noch Silber zu tun an Saul und seinem Hause und steht uns nicht zu, jemand zu töten in Israel. Er sprach: Was sprecht ihr denn, daß ich euch tun soll?
\par 5 Sie sprachen zum König: Den Mann, der uns verderbt und zunichte gemacht hat, sollen wir vertilgen, daß ihm nichts bleibe in allen Grenzen Israels.
\par 6 Gebt uns sieben Männer aus seinem Hause, daß wir sie aufhängen dem HERRN zu Gibea Sauls, des Erwählten des HERRN. Der König sprach: Ich will sie geben.
\par 7 Aber der König verschonte Mephiboseth, den Sohn Jonathans, des Sohnes Sauls, um des Eides willen des HERRN, der zwischen ihnen war, zwischen David und Jonathan, dem Sohn Sauls.
\par 8 Aber die zwei Söhne Rizpas, der Tochter Ajas, die sie Saul geboren hatte, Armoni und Mephiboseth, dazu die fünf Söhne Merabs, der Tochter Sauls, die sie dem Adriel geboren hatte, dem Sohn Barsillais, des Meholathiters, nahm der König
\par 9 und gab sie in die Hand der Gibeoniter; die hingen sie auf dem Berge vor dem HERRN. Also fielen diese sieben auf einmal und starben zur Zeit der ersten Ernte, wann die Gerstenernte angeht.
\par 10 Da nahm Rizpa, die Tochter Ajas, einen Sack und breitete ihn auf den Fels am Anfang der Ernte, bis daß Wasser von Himmel über sie troff, und ließ des Tages die Vögel des Himmels nicht auf ihnen ruhen noch des Nachts die Tiere des Feldes.
\par 11 Und es ward David angesagt, was Rizpa, die Tochter Ajas, Sauls Kebsweib, getan hatte.
\par 12 Und David ging hin und nahm die Gebeine Sauls und die Gebeine Jonathans, seines Sohnes, von den Bürgern zu Jabes in Gilead (die sie vom Platz am Tor Beth-Seans gestohlen hatten, dahin sie die Philister gehängt hatten zu der Zeit, da die Philister Saul schlugen auf dem Berge Gilboa),
\par 13 und brachte sie von da herauf; und sie sammelten sie zuhauf mit den Gebeinen der Gehängten
\par 14 und begruben die Gebeine Sauls und seines Sohnes Jonathan im Lande Benjamin zu Zela im Grabe seines Vaters Kis und taten alles, wie der König geboten hatte. Also ward Gott nach diesem dem Lande wieder versöhnt.
\par 15 Es erhob sich aber wieder ein Krieg von den Philistern wider Israel; und David zog hinab und seine Knechte mit ihm und stritten wider die Philister. Und David ward müde.
\par 16 Und Jesbi zu Nob (welcher war der Kinder Raphas einer, und das Gewicht seines Speers war dreihundert Gewicht Erzes, und er hatte neue Waffen), der gedachte David zu schlagen.
\par 17 Aber Abisai, der Zeruja Sohn, half ihm und schlug den Philister tot. Da schwuren ihm die Männer Davids und sprachen: Du sollst nicht mehr mit uns ausziehen in den Streit, daß nicht die Leuchte in Israel verlösche.
\par 18 Darnach erhob sich noch ein Krieg zu Gob mit den Philistern. Da schlug Sibbechai, der Husathiter, den Saph, welcher auch der Kinder Raphas einer war.
\par 19 Und es erhob sich noch ein Krieg zu Gob mit den Philistern. Da schlug El-Hanan, der Sohn Jaere-Orgims, ein Bethlehemiter, den Goliath, den Gathiter, welcher hatte einen Spieß, des Stange war wie ein Weberbaum.
\par 20 Und es erhob sich noch ein Krieg zu Gath. Da war ein langer Mann, der hatte sechs Finger an seinen Händen und sechs Zehen an seinen Füßen, das ist vierundzwanzig an der Zahl; und er war auch geboren dem Rapha.
\par 21 Und da er Israel Hohn sprach, schlug ihn Jonathan, der Sohn Simeas, des Bruders Davids.
\par 22 Diese vier waren geboren dem Rapha zu Gath und fielen durch die Hand Davids und seiner Knechte.

\chapter{22}

\par 1 Und David redete vor dem HERRN die Worte dieses Liedes zur Zeit, da ihn der HERR errettet hatte von der Hand aller seiner Feinde und von der Hand Sauls, und sprach:
\par 2 Der HERR ist mein Fels und meine Burg und mein Erretter.
\par 3 Gott ist mein Hort, auf den ich traue, mein Schild und Horn meines Heils, mein Schutz und meine Zuflucht, mein Heiland, der du mir hilfst vor dem Frevel.
\par 4 Ich rufe an den HERRN, den Hochgelobten, so werde ich von meinen Feinden erlöst.
\par 5 Es hatten mich umfangen die Schmerzen des Todes, und die Bäche des Verderbens erschreckten mich.
\par 6 Der Hölle Bande umfingen mich, und des Todes Stricke überwältigten mich.
\par 7 Da mir angst war, rief ich den HERRN an und schrie zu meinem Gott; da erhörte er meine Stimme von seinem Tempel, und mein Schreien kam vor ihn zu seinen Ohren.
\par 8 Die Erde bebte und ward bewegt; die Grundfesten des Himmels regten sich und bebten, da er zornig war.
\par 9 Dampf ging auf von seiner Nase und verzehrend Feuer von seinem Munde, daß es davon blitzte.
\par 10 Er neigte den Himmel und fuhr herab, und Dunkel war unter seinen Füßen.
\par 11 Und er fuhr auf dem Cherub und flog daher, und er schwebte auf den Fittichen des Windes.
\par 12 Sein Gezelt um ihn her war finster und schwarze, dicke Wolken.
\par 13 Von dem Glanz vor ihm brannte es mit Blitzen.
\par 14 Der HERR donnerte vom Himmel, und der Höchste ließ seinen Donner aus.
\par 15 Er schoß seine Strahlen und zerstreute sie; er ließ blitzen und erschreckte sie.
\par 16 Da sah man das Bett der Wasser, und des Erdbodens Grund ward aufgedeckt von dem Schelten des HERRN, von dem Odem und Schnauben seiner Nase.
\par 17 Er streckte seine Hand aus von der Höhe und holte mich und zog mich aus den großen Wassern.
\par 18 Er errettete mich von meinen starken Feinden, von meinen Hassern, die zu mir mächtig waren,
\par 19 die mich überwältigten zur Zeit meines Unglücks; und der HERR ward meine Zuversicht.
\par 20 Und er führte mich aus in das Weite, er riß mich heraus; denn er hatte Lust zu mir.
\par 21 Der HERR tut wohl an mir nach meiner Gerechtigkeit; er vergilt mir nach der Reinigkeit meiner Hände.
\par 22 Denn ich halte die Wege des HERRN und bin nicht gottlos wider meinen Gott.
\par 23 Denn alle seine Rechte habe ich vor Augen, und seine Gebote werfe ich nicht von mir;
\par 24 sondern ich bin ohne Tadel vor ihm und hüte mich vor Sünden.
\par 25 Darum vergilt mir der HERR nach meiner Gerechtigkeit, nach meiner Reinigkeit vor seinen Augen.
\par 26 Bei den Heiligen bist du heilig, bei den Frommen bist du fromm,
\par 27 bei den Reinen bist du rein, und bei den Verkehrten bist du verkehrt.
\par 28 Denn du hilfst dem elenden Volk, und mit deinen Augen erniedrigst du die Hohen.
\par 29 Denn du, HERR, bist meine Leuchte; der HERR macht meine Finsternis licht.
\par 30 Denn mit dir kann ich Kriegsvolk zerschlagen und mit meinem Gott über die Mauer springen.
\par 31 Gottes Wege sind vollkommen; des HERRN Reden sind durchläutert. Er ist ein Schild allen, die ihm vertrauen.
\par 32 Denn wo ist ein Gott außer dem HERRN, und wo ist ein Hort außer unserm Gott?
\par 33 Gott stärkt mich mit Kraft und weist mir einen Weg ohne Tadel.
\par 34 Er macht meine Füße gleich den Hirschen und stellt mich auf meine Höhen.
\par 35 Er lehrt meine Hände streiten und lehrt meinen Arm den ehernen Bogen spannen.
\par 36 Du gibst mir den Schild deines Heils; und wenn du mich demütigst, machst du mich groß.
\par 37 Du machst unter mir Raum zu gehen, daß meine Knöchel nicht wanken.
\par 38 Ich will meinen Feinden nachjagen und sie vertilgen und will nicht umkehren, bis ich sie umgebracht habe.
\par 39 Ich will sie umbringen und zerschmettern; sie sollen mir nicht widerstehen und müssen unter meine Füße fallen.
\par 40 Du kannst mich rüsten mit Stärke zum Streit; du kannst unter mich werfen, die sich wider mich setzen.
\par 41 Du gibst mir meine Feinde in die Flucht, daß ich verstöre, die mich hassen.
\par 42 Sie sahen sich um, aber da ist kein Helfer, nach dem HERRN; aber er antwortet ihnen nicht.
\par 43 Ich will sie zerstoßen wie Staub auf der Erde; wie Kot auf der Gasse will ich sie verstäuben und zerstreuen.
\par 44 Du hilfst mir von dem zänkischen Volk und behütest mich, daß ich ein Haupt sei unter den Heiden; ein Volk, das ich nicht kannte, dient mir.
\par 45 Den Kindern der Fremde hat's wider mich gefehlt; sie gehorchen mir mit gehorsamen Ohren.
\par 46 Die Kinder der Fremde sind verschmachtet und kommen mit Zittern aus ihren Burgen.
\par 47 Der HERR lebt, und gelobt sei mein Hort; und Gott, der Hort meines Heils, werde erhoben,
\par 48 der Gott, der mir Rache gibt und wirft die Völker unter mich.
\par 49 Er hilft mir aus von meinen Feinden. Du erhöhst mich aus denen, die sich wider mich setzen; du hilfst mir von den Frevlern.
\par 50 Darum will ich dir danken, HERR, unter den Heiden und deinem Namen lobsingen,
\par 51 der seinem Könige großes Heil beweist und wohltut seinem Gesalbten, David und seinem Samen ewiglich.

\chapter{23}

\par 1 Dies sind die letzten Worte Davids: Es sprach David der Sohn Isais, es sprach der Mann, der hoch erhoben ist, der Gesalbte des Gottes Jakobs, lieblich mit Psalmen Israels.
\par 2 Der Geist des HERRN hat durch mich geredet, und seine Rede ist auf meiner Zunge.
\par 3 Es hat der Gott Israels zu mir gesprochen, der Hort Israels hat geredet: Ein Gerechter herrscht unter den Menschen, er herrscht mit der Furcht Gottes
\par 4 und ist wie das Licht des Morgens, wenn die Sonne aufgeht, am Morgen ohne Wolken, da vom Glanz nach dem Regen das Gras aus der Erde wächst.
\par 5 Denn ist mein Haus nicht also bei Gott? Denn er hat mir einen ewigen Bund gesetzt, der in allem wohl geordnet und gehalten wird. All mein Heil und all mein Begehren, das wird er wachsen lassen.
\par 6 Aber die heillosen Leute sind allesamt wie die ausgeworfenen Disteln, die man nicht mit Händen fassen kann;
\par 7 sondern wer sie angreifen soll, muß Eisen und Spießstange in der Hand haben; sie werden mit Feuer verbrannt an ihrem Ort.
\par 8 Dies sind die Namen der Helden Davids: Jasobeam, der Sohn Hachmonis, ein Vornehmster unter den Rittern; er hob seinen Spieß auf und schlug achthundert auf einmal.
\par 9 Nach ihm war Eleasar, der Sohn Dodos, des Sohnes Ahohis, unter den drei Helden mit David. Da sie Hohn sprachen den Philistern und daselbst versammelt waren zum Streit und die Männer Israels hinaufzogen,
\par 10 da stand er und schlug die Philister, bis seine Hand müde am Schwert erstarrte. Und der HERR gab ein großes Heil zu der Zeit, daß das Volk umwandte ihm nach, zu rauben.
\par 11 Nach ihm war Samma, der Sohn Ages, des Harariters. Da die Philister sich versammelten in eine Rotte, und war daselbst ein Stück Acker voll Linsen, und das Volk floh vor den Philistern,
\par 12 da trat er mitten auf das Stück und errettete es und schlug die Philister; und Gott gab ein großes Heil.
\par 13 Und diese drei Vornehmsten unter dreißigen kamen hinab in der Ernte zu David in die Höhle Adullam, und die Rotte der Philister lag im Grunde Rephaim.
\par 14 David aber war dazumal an sicherem Ort; aber der Philister Volk lag zu Bethlehem.
\par 15 Und David ward lüstern und sprach: Wer will mir Wasser zu trinken holen aus dem Brunnen zu Bethlehem unter dem Tor?
\par 16 Da brachen die drei Helden ins Lager der Philister und schöpften Wasser aus dem Brunnen zu Bethlehem unter dem Tor und trugen's und brachten's zu David. Aber er wollte nicht trinken sondern goß es aus dem HERRN
\par 17 und sprach: Das lasse der HERR fern von mir sein, daß ich das tue! Ist's nicht das Blut der Männer, die ihr Leben gewagt haben und dahin gegangen sind? Und wollte es nicht trinken. Das taten die drei Helden.
\par 18 Abisai, Joabs Bruder, der Zeruja Sohn, war auch ein Vornehmster unter den Rittern: er hob seinen Spieß auf und schlug dreihundert, und war auch berühmt unter dreien
\par 19 und der Herrlichste unter dreien und war ihr Oberster; aber er kam nicht bis an jene drei.
\par 20 Und Benaja, der Sohn Jojadas, des Sohnes Is-Hails, von großen Taten, von Kabzeel, der schlug zwei Helden der Moabiter und ging hinab und schlug einen Löwen im Brunnen zur Schneezeit.
\par 21 Und schlug auch einen ägyptischen ansehnlichen Mann, der hatte einen Spieß in seiner Hand. Er aber ging zu ihm hinab mit einem Stecken und riß dem Ägypter den Spieß aus der Hand und erwürgte ihn mit seinem eigenen Spieß.
\par 22 Das tat Benaja, der Sohn Jojadas, und war berühmt unter den drei Helden
\par 23 und herrlicher denn die dreißig; aber er kam nicht bis an jene drei. Und David machte ihn zum heimlichen Rat.
\par 24 Asahel, der Bruder Joabs, war unter den dreißig; Elhanan, der Sohn Dodos, zu Bethlehem;
\par 25 Samma, der Haroditer; Elika, der Haroditer;
\par 26 Helez, der Paltiter; Ira, der Sohn Ikkes, des Thekoiters;
\par 27 Abieser, der Anathothiter; Mebunnai, der Husathiter;
\par 28 Zalmon, der Ahohiter; Maherai, der Netophathiter;
\par 29 Heleb, der Sohn Baanas, der Netophathiter; Itthai, der Sohn Ribais, von Gibea der Kinder Benjamin;
\par 30 Benaja, der Pirathoniter; Hiddai, von Nahale-Gaas;
\par 31 Abi-Albon, der Arbathiter; Asmaveth, der Barhumiter;
\par 32 Eljahba, der Saalboniter; die Kinder Jasen und Jonathan;
\par 33 Samma, der Harariter; Ahiam, der Sohn Sarars, der Harariter;
\par 34 Eliphelet, der Sohn Ahasbais, des Maachathiters; Eliam, der Sohn Ahithophels, des Gileoniters;
\par 35 Hezrai, der Karmeliter; Paerai, der Arbiter;
\par 36 Jigeal, der Sohn Nathans, von Zoba; Bani, der Gaditer;
\par 37 Zelek, der Ammoniter; Naharai, der Beerothiter, der Waffenträger Joabs, des Sohnes der Zeruja;
\par 38 Ira, der Jethriter; Gareb, der Jethriter;
\par 39 Uria, der Hethiter. Das sind allesamt siebenunddreißig.

\chapter{24}

\par 1 Und der Zorn des HERRN ergrimmte abermals wider Israel und er reizte David wider sie, daß er sprach: Gehe hin, zähle Israel und Juda!
\par 2 Und der König sprach zu Joab, seinem Feldhauptmann: Gehe umher in allen Stämmen Israels von Dan an bis gen Beer-Seba und zähle das Volk, daß ich wisse, wieviel sein ist!
\par 3 Joab sprach zu dem König: Der HERR, dein Gott, tue diesem Volk, wie es jetzt ist, noch hundertmal soviel, daß mein Herr, der König, seiner Augen Lust daran sehe; aber was hat mein Herr König zu dieser Sache Lust?
\par 4 Aber des Königs Wort stand fest wider Joab und die Hauptleute des Heeres. Also zog Joab aus und die Hauptleute des Heeres von dem König, daß sie das Volk Israel zählten.
\par 5 Und sie gingen über den Jordan und lagerten sich zu Aroer, zur Rechten der Stadt, die am Bach Gad liegt, und gen Jaser hin,
\par 6 und kamen gen Gilead und ins Niederland Hodsi, und kamen gen Dan-Jaan und um Sidon her,
\par 7 und kamen zu der festen Stadt Tyrus und allen Städten der Heviter und Kanaaniter, und kamen hinaus an den Mittag Juda's gen Beer-Seba,
\par 8 und durchzogen das ganze land und kamen nach neuen Monaten und zwanzig Tagen gen Jerusalem.
\par 9 Und Joab gab dem König die Summe des Volks, das gezählt war. Und es waren in Israel achthundertmal tausend starke Männer, die das Schwert auszogen, und in Juda fünfhundertmal tausend Mann.
\par 10 Und das Herz schlug David, nachdem das Volk gezählt war. Und David sprach zum HERRN: Ich habe schwer gesündigt, daß ich das getan habe; und nun, HERR, nimm weg die Missetat deines Knechtes; denn ich habe sehr töricht getan.
\par 11 Und da David des Morgens aufstand, kam des HERRN Wort zu Gad, dem Propheten, Davids Seher, und sprach:
\par 12 Gehe hin und rede mit David: So spricht der HERR: Dreierlei bringe ich zu dir; erwähle dir deren eins, daß ich es dir tue.
\par 13 Gad kam zu David und sagte es ihm an und sprach zu ihm: Willst du, daß sieben Jahre Teuerung in dein Land komme? oder daß du drei Monate vor deinen Widersachern fliehen müssest und sie dich verfolgen? oder drei Tage Pestilenz in deinem Lande sei? So merke nun und siehe, was ich wieder sagen soll dem, der mich gesandt hat.
\par 14 David sprach zu Gad: Es ist mir sehr angst; aber laß uns in die Hand des HERRN fallen, denn seine Barmherzigkeit ist groß; ich will nicht in der Menschen Hand fallen.
\par 15 Also ließ der HERR Pestilenz in Israel kommen von Morgen an bis zur bestimmten Zeit, daß des Volks starb von Dan an bis gen Beer-Seba siebzigtausend Mann.
\par 16 Und da der Engel seine Hand ausstreckte über Jerusalem, daß er es verderbte, reute den HERRN das Übel, und er sprach zum Engel, zu dem Verderber im Volk: Es ist genug; laß deine Hand ab! Der Engel aber des HERRN war bei der Tenne Aravnas, des Jebusiters.
\par 17 Da aber David den Engel sah, der das Volk schlug, sprach er zum HERRN: Siehe, ich habe gesündigt, ich habe die Missetat getan; was habe diese Schafe getan? Laß deine Hand wider mich und meines Vaters Haus sein!
\par 18 Und Gad kam zu David zur selben Zeit und sprach zu ihm: Gehe hinauf und richte dem HERRN einen Altar auf in der Tenne Aravnas, des Jebusiters!
\par 19 Also ging David hinauf, wie Gad ihm gesagt und der HERR ihm geboten hatte.
\par 20 Und da Aravna sich wandte, sah er den König mit seinen Knechten zu ihm herüberkommen und fiel nieder auf sein Angesicht zur Erde
\par 21 und sprach: Warum kommt mein Herr, der König, zu seinem Knecht? David sprach: Zu kaufen von dir die Tenne und zu bauen dem HERRN einen Altar, daß die Plage vom Volk aufhöre.
\par 22 Aber Aravna sprach zu David: Mein Herr, der König, nehme und opfere, wie es ihm gefällt: siehe, da ist ein Rind zum Brandopfer und Schleifen und Geschirr vom Ochsen zu Holz.
\par 23 Das alles gab Aravna, der König, dem König. Und Aravna sprach zum König: Der HERR, dein Gott, lasse dich ihm angenehm sein.
\par 24 Aber der König sprach zu Aravna: Nicht also, sondern ich will dir's abkaufen um seinen Preis; denn ich will dem HERRN, meinem Gott, nicht Brandopfer tun, das ich umsonst habe. Also kaufte David die Tenne und das Rind um fünfzig Silberlinge
\par 25 und baute daselbst dem HERRN einen Altar und opferte Brandopfer und Dankopfer. Und der HERR ward dem Land versöhnt, und die Plage hörte auf von dem Volk Israel.


\end{document}