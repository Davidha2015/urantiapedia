\begin{document}

\title{2 Chronicles}


2Ch 1:1  En Salomo, de zoon van David, werd versterkt in zijn koninkrijk, want de HEERE, zijn God, was met hem, en maakte hem ten hoogste groot.
2Ch 1:2  En Salomo sprak tot het ganse Israel, tot de oversten der duizenden en der honderden, en tot de richteren, en tot alle oversten in gans Israel, de hoofden der vaderen;
2Ch 1:3  En zij gingen henen, Salomo en de ganse gemeente met hem, naar de hoogte, die te Gibeon was; want daar was de tent der samenkomst Gods, die Mozes, de knecht des HEEREN, in de woestijn gemaakt had.
2Ch 1:4  (Maar de ark Gods had David van Kirjath-jearim opgebracht, ter plaatse, die David voor haar bereid had; want hij had voor haar een tent te Jeruzalem gespannen.)
2Ch 1:5  Ook was het koperen altaar, dat Bezaleel, de zoon van Uri, den zoon van Hur, gemaakt had, aldaar voor den tabernakel des HEEREN; Salomo nu en de gemeente bezochten hetzelve.
2Ch 1:6  En Salomo offerde daar, voor het aangezicht des HEEREN, op het koperen altaar, dat aan de tent der samenkomst was; en hij offerde daarop duizend brandofferen.
2Ch 1:7  In dienzelfden nacht verscheen God aan Salomo; en Hij zeide tot hem: Begeer, wat Ik u geven zal.
2Ch 1:8  En Salomo zeide tot God: Gij hebt aan mijn vader David grote weldadigheid gedaan; en Gij hebt mij koning gemaakt in zijn plaats;
2Ch 1:9  Nu, HEERE God, laat Uw woord waar worden, gedaan aan mijn vader David; want Gij hebt mij koning gemaakt over een volk, menigvuldig als het stof der aarde;
2Ch 1:10  Geef mij nu wijsheid en wetenschap, dat ik voor het aangezicht van dit volk uitga en inga; want wie zou dit Uw groot volk kunnen richten?
2Ch 1:11  Toen zeide God tot Salomo: Daarom, dat dit in uw hart geweest is, en gij niet begeerd hebt rijkdom, goederen, noch eer, noch de ziel uwer haters, noch ook vele dagen begeerd hebt; maar wijsheid en wetenschap voor u begeerd hebt, opdat gij Mijn volk mocht richten, waarover Ik u koning gemaakt heb;
2Ch 1:12  De wijsheid, en de wetenschap is u gegeven; daartoe zal Ik u rijkdom, en goederen, en eer geven, dergelijke geen koningen, die voor u geweest zijn, gehad hebben, en na u zal dergelijke niet zijn.
2Ch 1:13  Alzo kwam Salomo te Jeruzalem, van de hoogte, die te Gibeon is, van voor de tent der samenkomst; en hij regeerde over Israel.
2Ch 1:14  En Salomo vergaderde wagenen en ruiteren, zodat hij duizend en vierhonderd wagenen, en twaalf duizend ruiteren had; en hij leide ze in de wagensteden, en bij den koning te Jeruzalem.
2Ch 1:15  En de koning maakte het zilver en het goud in Jeruzalem te zijn als stenen, en de cederen maakte hij te zijn als wilde vijgebomen, die in de laagten zijn, in menigte.
2Ch 1:16  En het uitbrengen der paarden was hetgeen Salomo uit Egypte had; en aangaande het linnengaren, de kooplieden des konings namen het linnengaren voor den prijs.
2Ch 1:17  En zij brachten op, en voerden een wagen uit van Egypte voor zeshonderd sikkelen zilvers, en een paard voor eenhonderd en vijftig; en alzo voerden zij die door hun hand uit, voor alle koningen der Hethieten, en voor de koningen van Syrie.
2Ch 2:1  Salomo nu dacht voor den Naam des HEEREN een huis te bouwen, en een huis voor zijn koninkrijk.
2Ch 2:2  En Salomo telde zeventig duizend lastdragende mannen, en tachtig duizend mannen, die houwen zouden in het gebergte; mitsgaders drie duizend en zeshonderd opzieners over dezelve.
2Ch 2:3  En Salomo zond tot Huram, den koning van Tyrus, zeggende: Gelijk als gij met mijn vader David gedaan hebt, en hebt hem cederen gezonden, om voor hem een huis te bouwen, om daarin te wonen, zo doe ook met mij.
2Ch 2:4  Zie, ik zal een huis voor den Naam des HEEREN, mijns Gods, bouwen, om Hem te heiligen, om reukwerk der welriekende specerijen voor Zijn aangezicht aan te steken, en voor de toerichting des gedurigen broods, en voor de brandofferen des morgens en des avonds, op de sabbatten, en op de nieuwe maanden, en op de gezette hoogtijden des HEEREN, onzes Gods; hetwelk voor eeuwig is in Israel.
2Ch 2:5  En het huis, dat ik zal bouwen, zal groot zijn; want onze God is groter dan alle goden.
2Ch 2:6  Doch wie zou de kracht hebben, om voor Hem een huis te bouwen, dewijl de hemelen, ja, de hemel der hemelen, Hem niet bevatten zouden? En wie ben ik, dat ik voor Hem een huis zou bouwen, ten ware om reukwerk voor Zijn aangezicht aan te steken?
2Ch 2:7  Zo zend mij nu een wijzen man, om te werken in goud, en in zilver, en in koper, en in ijzer, en in purper, en karmozijn, en hemelsblauw, en die weet graveerselen te graveren, met de wijzen, die bij mij zijn in Juda en in Jeruzalem, die mijn vader David beschikt heeft.
2Ch 2:8  Zend mij ook cederen, dennen, en algummimhout uit Libanon; want ik weet, dat uw knechten het hout van Libanon weten te houwen; en zie, mijn knechten zullen met uw knechten zijn.
2Ch 2:9  En dat om mij hout in menigte te bereiden; want het huis, dat ik zal bouwen, zal groot en wonderlijk zijn.
2Ch 2:10  En zie, ik zal uw knechten, den houwers, die het hout houwen, twintig duizend kor uitgeslagen tarwe, en twintig duizend kor gerst geven; daartoe twintig duizend bath wijn, en twintig duizend bath olie.
2Ch 2:11  Huram nu, de koning van Tyrus, antwoordde door schrift, en zond tot Salomo: Daarom dat de HEERE Zijn volk lief heeft, heeft Hij u over hen tot koning gesteld.
2Ch 2:12  Verder zeide Huram: Geloofd zij de HEERE, de God Israels, Die den hemel en de aarde gemaakt heeft, dat Hij den koning David een wijzen zoon, kloek in voorzichtigheid en verstand, gegeven heeft, die een huis voor den HEERE, en een huis voor zijn koninkrijk bouwe!
2Ch 2:13  Zo zend ik nu een wijzen man, kloek van verstand, Huram Abi;
2Ch 2:14  Den zoon ener vrouw uit de dochteren van Dan, en wiens vader een man geweest is van Tyrus, die weet te werken in goud, en in zilver, in koper, in ijzer, in stenen, en in hout, in purper, in hemelsblauw, en in fijn linnen, en in karmozijn, en om alle graveersels te graveren, en om te bedenken allen vernuftigen vond, die hem zal voorgesteld worden, met uw wijzen, en de wijzen van mijn heer, uw vader David.
2Ch 2:15  Zo zende nu mijn heer zijn knechten de tarwe en de gerst, de olie en den wijn, die hij gezegd heeft.
2Ch 2:16  En wij zullen hout houwen uit den Libanon, naar al uw nooddruft, en zullen het tot u met vlotten, over de zee, naar Jafo brengen; en gij zult het laten ophalen naar Jeruzalem.
2Ch 2:17  En Salomo telde al de vreemde mannen, die in het land van Israel waren, achtervolgens de telling, met dewelke zijn vader David die geteld had; en er werden gevonden honderd drie en vijftig duizend en zeshonderd.
2Ch 2:18  En hij maakte uit dezelve zeventig duizend lastdragers, en tachtig duizend houwers in het gebergte, mitsgaders drie duizend en zeshonderd opzieners, om het volk te doen arbeiden.
2Ch 3:1  En Salomo begon het huis des HEEREN te bouwen te Jeruzalem, op den berg Moria, die zijn vader David gewezen was, in de plaats, die David toebereid had, op den dorsvloer van Ornan, den Jebusiet.
2Ch 3:2  Hij begon nu te bouwen in de tweede maand, op den tweeden dag, in het vierde jaar van zijn koninkrijk.
2Ch 3:3  En deze zijn de grondleggingen van Salomo, om het huis Gods te bouwen: de lengte in ellen, naar de eerste mate, was zestig ellen, en de breedte twintig ellen.
2Ch 3:4  En het voorhuis, hetwelk vooraan was, was in de lengte, naar de breedte van het huis, twintig ellen, en de hoogte honderd en twintig; hetwelk hij van binnen overtrok met louter goud.
2Ch 3:5  Het grote huis nu overdekte hij met dennenhout; daarna overtoog hij dat met goed goud; en hij maakte daarop palmen en ketenwerk.
2Ch 3:6  Hij overtoog ook het huis met kostelijke stenen tot versiering; het goud nu was goud van Parvaim.
2Ch 3:7  Daartoe overdekte hij aan het huis de balken, de posten en de wanden daarvan, en de deuren daarvan met goud; en hij graveerde cherubs aan de wanden.
2Ch 3:8  Verder maakte hij het huis van het heilige der heiligen, welks lengte, naar de breedte van het huis, was twintig ellen, en de breedte daarvan twintig ellen; en hij overtoog dat met goed goud, tot zeshonderd talenten.
2Ch 3:9  En het gewicht der nagelen was tot vijftig sikkelen gouds; en hij overtoog de opperzalen met goud.
2Ch 3:10  Ook maakte hij, in het huis van het heilige der heiligen, twee cherubim van uittrekkend werk, en hij overtoog die met goud.
2Ch 3:11  Aangaande de vleugelen der cherubim, hun lengte was twintig ellen; des enen vleugel was van vijf ellen, rakende aan den wand van het huis, en de andere vleugel van vijf ellen, rakende aan den vleugel des anderen cherubs.
2Ch 3:12  Insgelijks was de vleugel des anderen cherubs van vijf ellen, rakende aan den wand van het huis; en de andere vleugel was van vijf ellen, klevende aan den vleugel des anderen cherubs.
2Ch 3:13  De vleugelen dezer cherubim spreidden zich uit twintig ellen; en zij stonden op hun voeten, en hun aangezichten waren huiswaarts.
2Ch 3:14  Hij maakte ook den voorhang van hemelsblauw, en purper, en karmozijn, en fijn linnen; en hij maakte cherubs daarop.
2Ch 3:15  Nog maakte hij voor het huis twee pilaren, van vijf en dertig ellen in lengte; en het kapiteel, dat op derzelver hoofd was, was van vijf ellen.
2Ch 3:16  Ook maakte hij ketenen, als in de aanspraakplaats, en hij zette ze op de hoofden der pilaren; daartoe maakte hij honderd granaatappelen, en zette ze tussen de ketenen.
2Ch 3:17  En hij richtte de pilaren op voor aan den tempel, een ter rechterhand, en een ter linkerhand; en hij noemde den naam van den rechter Jachin, en den naam van den linker Boaz.
2Ch 4:1  Hij maakte ook een koperen altaar, van twintig ellen in zijn lengte, en twintig ellen in zijn breedte, en tien ellen in zijn hoogte.
2Ch 4:2  Daartoe maakte hij de gegoten zee; van tien ellen was zij, van haar enen rand tot haar anderen rand, rondom rond, en van vijf ellen in haar hoogte, en een meetsnoer van dertig ellen omving ze rondom.
2Ch 4:3  Onder dezelve nu was de gelijkenis van runderen, rondom henen, die omsingelende, tien in een el, omringende de zee rondom; twee rijen dezer runderen waren in haar gieting gegoten.
2Ch 4:4  Zij stond op twaalf runderen, drie ziende naar het noorden, en drie ziende naar het westen, en drie ziende naar het zuiden, en drie ziende naar het oosten; en de zee was boven op dezelve; en al hun achterdelen waren inwaarts.
2Ch 4:5  Haar dikte nu was een hand breed, en haar rand als het werk van den rand eens bekers of ener leliebloem, bevattende vele bathen; zij hield drie duizend.
2Ch 4:6  En hij maakte tien wasvaten, en stelde vijf ter rechter hand en vijf ter linkerhand, om daarin te wassen; wat ten brandoffer behoort, staken zij daarin; maar de zee was, opdat de priesters zich daarin zouden wassen.
2Ch 4:7  Hij maakte ook tien gouden kandelaren, naar hun wijze, en hij stelde ze in den tempel, vijf aan de rechterhand, en vijf aan de linkerhand.
2Ch 4:8  Ook maakte hij tien tafelen, en hij zette ze in den tempel, vijf aan de rechterhand, en vijf aan de linkerhand; en hij maakte honderd gouden sprengbekkens.
2Ch 4:9  Verder maakte hij het voorhof der priesteren, en het grote voorhof, mitsgaders de deuren voor het voorhof, en overtoog hun deuren met koper.
2Ch 4:10  De zee nu zette hij aan de rechterzijde, naar het oosten, tegenover het zuiden.
2Ch 4:11  Daartoe maakte Huram de potten, en de schoffelen, en de sprengbekkens; alzo voleindde Huram het werk te maken, dat hij voor den koning Salomo aan het huis Gods maakte.
2Ch 4:12  De twee pilaren, en de bollen, en de twee kapitelen, op het hoofd der pilaren; en de twee netten, om de twee bollen der kapitelen te bedekken, die op der pilaren hoofd waren;
2Ch 4:13  En de vierhonderd granaatappelen tot de twee netten: twee rijen van granaatappelen tot elk net, om de twee bollen der kapitelen te bedekken, die boven op de pilaren waren.
2Ch 4:14  Hij maakte ook de stellingen; en wasvaten maakte hij op de stellingen;
2Ch 4:15  Een zee, en de twaalf runderen daaronder.
2Ch 4:16  Insgelijks de potten, en de schoffelen, en de krauwelen, en al hun vaten maakte Huram Abi voor den koning Salomo, voor het huis des HEEREN, van gepolijst koper.
2Ch 4:17  In de vlakte van de Jordaan goot ze de koning, in dichte aarde, tussen Sukkoth, en tussen Zeredatha.
2Ch 4:18  En Salomo maakte al deze vaten, in grote menigte; want het gewicht des kopers werd niet onderzocht.
2Ch 4:19  Ook maakte Salomo alle vaten, die voor het huis Gods waren, en het gouden altaar, en de tafelen, waarop de toonbroden zijn;
2Ch 4:20  En de kandelaren met hun lampen, van gesloten goud, om die naar de wijze aan te steken, voor de aanspraakplaats;
2Ch 4:21  En de bloemen, en de lampen, en de snuiters, van goud; het was het volmaaktste goud;
2Ch 4:22  Mitsgaders de gaffelen, en de sprengbekkens, en de rookschalen, en de wierookvaten, van gesloten goud; aangaande den ingang van het huis, zijn binnenste deuren, van het heilige der heiligen, en de deuren van het huis des tempels waren van goud.
2Ch 5:1  Alzo werd al het werk volbracht, dat Salomo aan het huis des HEEREN maakte. Daarna bracht Salomo de geheiligde dingen van zijn vader David; en het zilver, en het goud, en al de vaten leide hij onder de schatten van het huis Gods.
2Ch 5:2  Toen vergaderde Salomo de oudsten van Israel, en al de hoofden der stammen, de oversten der vaderen onder de kinderen Israels, te Jeruzalem, om de ark des verbonds des HEEREN op te brengen uit de stad Davids, dewelke is Sion.
2Ch 5:3  En alle mannen van Israel verzamelden zich tot den koning op het feest, hetwelk was in de zevende maand.
2Ch 5:4  En al de oudsten van Israel kwamen, en de Levieten namen de ark op.
2Ch 5:5  En zij brachten de ark, en de tent der samenkomst opwaarts, mitsgaders al de heilige vaten, die in de tent waren; deze brachten de priesters en Levieten opwaarts.
2Ch 5:6  De koning Salomo nu, en de ganse vergadering van Israel, die bij hem vergaderd waren voor de ark, offerden schapen en runderen, die vanwege de menigte niet konden geteld noch gerekend worden.
2Ch 5:7  Alzo brachten de priesters de ark des verbonds des HEEREN tot haar plaats, tot de aanspraakplaats van het huis, tot het heilige der heiligen, tot onder de vleugelen der cherubim.
2Ch 5:8  Want de cherubim spreidden de beide vleugelen over de plaats der ark; en de cherubim overdekten de ark en haar handbomen van boven.
2Ch 5:9  Daarna schoven zij de handbomen verder uit, dat de hoofden der handbomen gezien werden uit de ark, voor aan de aanspraakplaats, maar buiten niet gezien werden; en zij was daar tot op dezen dag.
2Ch 5:10  Er was niets in de ark, dan alleen de twee tafelen, die Mozes bij Horeb daarin gedaan had als de HEERE een verbond maakte met de kinderen Israels, toen zij uit Egypte uitgetogen waren.
2Ch 5:11  En het geschiedde, als de priesters uit het heilige uitgingen; (want al de priesters, die gevonden werden, hadden zich geheiligd, zonder de verdelingen te houden;
2Ch 5:12  En de Levieten, die zangers waren van hen allen, van Asaf, van Heman, van Jeduthun, en van hun zonen, en van hun broederen, in fijn linnen gekleed, met cimbalen, en met luiten, en harpen, stonden tegen het oosten des altaars, en met hen tot honderd en twintig priesteren toe, trompettende met trompetten.)
2Ch 5:13  Het geschiedde dan, als zij eenpariglijk trompetten en zongen, om een eenparige stem te laten horen, prijzende en lovende den HEERE; en als zij de stem verhieven met trompetten, en met cimbalen, en andere muzikale instrumenten, en als zij den HEERE prezen, dat Hij goed is, dat Zijn weldadigheid is tot in eeuwigheid; dat het huis met een wolk vervuld werd, namelijk het huis des HEEREN.
2Ch 5:14  En de priesters konden, vanwege die wolk, niet staan, om te dienen; want de heerlijkheid des HEEREN had het huis Gods vervuld.
2Ch 6:1  Toen zeide Salomo: De HEERE heeft gezegd, dat Hij in de donkerheid zou wonen.
2Ch 6:2  En ik heb U een huis ter woonstede gebouwd, en een vaste plaats tot Uw eeuwige woning.
2Ch 6:3  Daarna wendde de koning zijn aangezicht om, en zegende de ganse gemeente van Israel; en de ganse gemeente van Israel stond.
2Ch 6:4  En hij zeide: Geloofd zij de HEERE, de God van Israel, Die met Zijn mond tot mijn vader David gesproken heeft, en heeft het met Zijn handen vervuld, zeggende:
2Ch 6:5  Van dien dag af, dat Ik Mijn volk uit Egypteland uitgevoerd heb, heb Ik geen stad verkoren uit alle stammen van Israel, om een huis te bouwen, dat Mijn Naam daar zou wezen; en geen man verkoren om een voorganger te zijn over Mijn volk Israel.
2Ch 6:6  Maar Ik heb Jeruzalem verkoren, dat Mijn Naam daar zou wezen; en Ik heb David verkoren, dat hij over Mijn volk Israel wezen zou.
2Ch 6:7  Het was ook in het hart van mijn vader David, een huis te bouwen den Naam des HEEREN, des Gods van Israel.
2Ch 6:8  Maar de HEERE zeide tot mijn vader David: Dewijl dat in uw hart geweest is, Mijn Naam een huis te bouwen, gij hebt welgedaan, dat het in uw hart geweest is.
2Ch 6:9  Evenwel, gij zult dat huis niet bouwen, maar uw zoon, die uit uw lenden voortkomen zal, die zal Mijn Naam dat huis bouwen.
2Ch 6:10  Zo heeft de HEERE Zijn woord bevestigd, dat Hij gesproken had; want ik ben opgestaan in de plaats van mijn vader David, en ik zit op den troon van Israel, gelijk als de HEERE gesproken heeft; en ik heb een huis gebouwd den Naam des HEEREN, des Gods van Israel.
2Ch 6:11  En ik heb daar de ark gesteld, waarin het verbond des HEEREN is, hetwelk Hij maakte met de kinderen Israels.
2Ch 6:12  En hij stond voor het altaar des HEEREN, tegenover de ganse gemeente van Israel; en hij breidde zijn handen uit;
2Ch 6:13  (Want Salomo had een koperen gestoelte gemaakt, en had het gesteld in het midden des voorhofs; zijnde vijf ellen in zijn lengte en vijf ellen in zijn breedte, en drie ellen in zijn hoogte; en hij stond daarop, en knielde op zijn knieen voor de ganse gemeente van Israel, en breidde zijn handen uit naar den hemel).
2Ch 6:14  En hij zeide: HEERE, God van Israel, er is geen God gelijk Gij, in den hemel noch op de aarde, houdende het verbond en de weldadigheid aan Uw knechten, die voor Uw aangezicht met hun ganse hart wandelen;
2Ch 6:15  Die Uw knecht, mijn vader David, gehouden hebt, wat Gij tot hem gesproken hadt; want met Uw mond hebt Gij gesproken, en met Uw hand vervuld, gelijk het te dezen dage is.
2Ch 6:16  En nu, HEERE, God van Israel, houd Uw knecht, mijn vader David, wat Gij tot hem gesproken hebt, zeggende: Geen man zal u van voor Mijn aangezicht afgesneden worden, die zitte op den troon van Israel; alleenlijk zo uw zonen hun weg bewaren, om te wandelen in Mijn wet, gelijk als gij gewandeld hebt voor Mijn aangezicht.
2Ch 6:17  Nu dan, o HEERE, God van Israel! Laat Uw woord waar worden, hetwelk Gij gesproken hebt tot Uw knecht, tot David.
2Ch 6:18  Maar waarlijk, zou God bij de mensen op de aarde wonen? Ziet de hemelen, ja, de hemel der hemelen, zouden U niet begrijpen, hoeveel te min dit huis, dat ik gebouwd heb?
2Ch 6:19  Wend U dan nog tot het gebed Uws knechts, en tot zijn smeking, o HEERE, mijn God, om te horen naar het geroep en naar het gebed, dat Uw knecht voor Uw aangezicht bidt.
2Ch 6:20  Dat Uw ogen open zijn, dag en nacht, over dit huis, over de plaats, van dewelke Gij gezegd hebt, Uw Naam daar te zullen zetten; om te horen naar het gebed, hetwelk Uw knecht bidden zal in deze plaats.
2Ch 6:21  Hoor dan naar de smekingen van Uw knecht, en van Uw volk Israel, die in deze plaats zullen bidden; en hoor Gij uit de plaats Uwer woning, uit den hemel, ja, hoor, en vergeef.
2Ch 6:22  Wanneer iemand tegen zijn naaste zal gezondigd hebben, en die hem een eed des vloeks opgelegd zal hebben, om zichzelven te vervloeken, en de eed des vloeks voor Uw altaar in dit huis komen zal;
2Ch 6:23  Hoor Gij dan uit den hemel, en doe, en richt Uw knechten, vergeldende den goddeloze, gevende zijn weg op zijn hoofd, en rechtvaardigende den rechtvaardige, gevende hem naar zijn gerechtigheid.
2Ch 6:24  Wanneer ook Uw volk Israel voor het aangezicht des vijands zal geslagen worden, omdat zij tegen U gezondigd zullen hebben, en zich bekeren, en Uw Naam belijden, en voor Uw aangezicht in dit huis bidden en smeken zullen,
2Ch 6:25  Hoor Gij dan uit den hemel, en vergeef de zonden van Uw volk Israel, en breng hen weder in het land, dat Gij hun en hun vaderen gegeven hebt.
2Ch 6:26  Als de hemel zal gesloten zijn, dat er geen regen is, omdat zij tegen U gezondigd zullen hebben; en zij in deze plaats bidden, en Uw Naam belijden, en van hun zonden zich bekeren zullen, als Gij hen geplaagd zult hebben;
2Ch 6:27  Hoor Gij dan in den hemel, en vergeef de zonden Uwer knechten en van Uw volk Israel, als Gij hun zult geleerd hebben den goeden weg, in denwelken zij wandelen zullen; en geef regen op Uw land, dat Gij Uw volk tot een erfenis gegeven hebt.
2Ch 6:28  Als er honger in het land wezen zal, als er pest wezen zal, als er brandkoren of honigdauw, sprinkhanen en kevers wezen zullen, als iemand van zijn vijanden in het land zijner poorten hem belegeren zal, of enige plage, of enige krankheid wezen zal;
2Ch 6:29  Alle gebed, alle smeking, die van enig mens, of van al Uw volk Israel geschieden zal, als zij erkennen, een ieder zijn plage en zijn smarte, en een ieder zijn handen in dit huis uitbreiden zal;
2Ch 6:30  Hoor Gij dan uit den hemel, de vaste plaats Uwer woning, en vergeef, en geef een iegelijk naar al zijn wegen, gelijk Gij zijn hart kent; want Gij alleen kent het hart van de kinderen der mensen.
2Ch 6:31  Opdat zij U vrezen, om te wandelen in Uw wegen, al de dagen, die zij leven zullen op het land, dat Gij onzen vaderen gegeven hebt.
2Ch 6:32  Zelfs ook aangaande den vreemde, die van Uw volk Israel niet zijn zal, maar uit verren lande, om Uws groten Naams, en Uwer sterke hand, en Uws uitgestrekten arms wil, komen zal; als zij komen, en bidden zullen in dit huis;
2Ch 6:33  Hoor Gij dan uit den hemel, uit de vaste plaats Uwer woning, en doe naar alles, waarom die vreemde tot U roepen zal; opdat alle volken der aarde Uw Naam kennen, zo om U te vrezen, gelijk Uw volk Israel, als om te weten, dat Uw Naam genoemd wordt over dit huis, hetwelk ik gebouwd heb.
2Ch 6:34  Wanneer Uw volk in den krijg tegen zijn vijanden uittrekken zal door den weg, dien Gij hen heenzenden zult, en zullen tot U bidden naar den weg dezer stad, die Gij verkoren hebt, en naar dit huis, hetwelk ik Uw Naam gebouwd heb;
2Ch 6:35  Hoor dan uit den hemel hun gebed en hun smeking, en voer hun recht uit.
2Ch 6:36  Wanneer zij gezondigd zullen hebben tegen U (want geen mens is er, die niet zondigt), en Gij tegen hen vertoornd zult zijn, en hen leveren zult voor het aangezicht des vijands, dat degenen, die hen gevangen hebben, hen gevankelijk wegvoeren in een land, dat verre of nabij is;
2Ch 6:37  En zij in het land, waar zij gevankelijk weggevoerd zijn, weder aan hun hart brengen zullen, dat zij zich bekeren, en tot U smeken in het land hunner gevangenis, zeggende: Wij hebben gezondigd, verkeerdelijk gedaan, en goddelooslijk gehandeld;
2Ch 6:38  En zij zich tot U bekeren, met hun ganse hart en met hun ganse ziel, in het land hunner gevangenis, waar zij hen gevankelijk weggevoerd hebben, en bidden zullen naar den weg huns lands, dat Gij hun vaderen gegeven hebt, en naar deze stad, die Gij verkoren hebt, en naar dit huis, dat ik Uw Naam gebouwd heb;
2Ch 6:39  Hoor dan uit den hemel, uit de vaste plaats Uwer woning, hun gebed en hun smekingen, en voer hun recht uit, en vergeef Uw volk, wat zij tegen U gezondigd zullen hebben.
2Ch 6:40  Nu, mijn God, laat toch Uw ogen open en Uw oren opmerkende zijn tot het gebed dezer plaats.
2Ch 6:41  En nu, HEERE God, maak U op tot Uw rust, Gij en de ark Uwer kracht; laat Uw priesters, HEERE God, met heil bekleed worden, en laat Uw gunstgenoten over het goede blijde zijn.
2Ch 6:42  O HEERE God! wend het aangezicht Uws gezalfden niet af; gedenk der weldadigheden van David, Uw knecht.
2Ch 7:1  Als nu Salomo voleind had te bidden, zo daalde het vuur van den hemel, en verteerde het brandoffer en de slachtofferen; en de heerlijkheid des HEEREN vervulde het huis.
2Ch 7:2  En de priesters konden niet ingaan in het huis des HEEREN; want de heerlijkheid des HEEREN had het huis des HEEREN vervuld.
2Ch 7:3  En als al de kinderen Israels dat vuur zagen afdalen, en de heerlijkheid des HEEREN over het huis, zo bukten zij met hun aangezichten ter aarde op den vloer, en aanbaden en loofden den HEERE, dat Hij goedig is, dat Zijn weldadigheid is tot in eeuwigheid.
2Ch 7:4  De koning nu en al het volk offerden slachtofferen voor het aangezicht des HEEREN.
2Ch 7:5  En de koning Salomo offerde slachtofferen van runderen, twee en twintig duizend, en van schapen, honderd en twintig duizend. Alzo hebben de koning en het ganse volk het huis Gods ingewijd.
2Ch 7:6  Ook stonden de priesters in hun wachten, en de Levieten met de muzikale instrumenten des HEEREN, die de koning David gemaakt had, om den HEERE te loven, dat Zijn weldadigheid is in eeuwigheid, als David door hun dienst Hem prees; en de priesters trompetten tegen hen over, en gans Israel stond.
2Ch 7:7  En Salomo heiligde het middelste des voorhofs, hetwelk voor het huis des HEEREN was, dewijl hij daar de brandofferen en het vette der dankofferen bereid had; want het koperen altaar, dat Salomo gemaakt had, kon het brandoffer, en het spijsoffer, en het vette niet vatten.
2Ch 7:8  Salomo hield ook ter zelfder tijd het feest zeven dagen, en gans Israel met hem, een zeer grote gemeente, van den ingang af van Hamath, tot de rivier van Egypte.
2Ch 7:9  En ten achtsten dage hielden zij een verbodsdag; want zij hielden de inwijding des altaars zeven dagen, en het feest zeven dagen.
2Ch 7:10  Doch op den drie en twintigsten dag der zevende maand liet hij het volk gaan tot hun hutten, blijde en goedsmoeds over het goede, dat de HEERE aan David en Salomo, en Zijn volk Israel gedaan had.
2Ch 7:11  Alzo volbracht Salomo het huis des HEEREN, en het huis des konings; en al wat in Salomo's hart gekomen was, om in het huis des HEEREN en in zijn huis te maken, richtte hij voorspoedig uit.
2Ch 7:12  En de HEERE verscheen Salomo des nachts, en Hij zeide tot hem: Ik heb uw gebed verhoord, en heb Mij deze plaats verkoren tot een offerhuis.
2Ch 7:13  Zo Ik den hemel toesluite, dat er geen regen zij, of zo Ik den sprinkhaan gebiede, het land te verteren, of zo Ik pest onder Mijn volk zende;
2Ch 7:14  En Mijn volk, over dewelken Mijn Naam genoemd wordt, zich verootmoedigt en bidt, en zij Mijn aangezicht zoeken, en zich bekeren van hun boze wegen; zo zal Ik uit den hemel horen, en hun zonden vergeven, en hun land genezen.
2Ch 7:15  Nu zullen Mijn ogen open zijn, en Mijn oren opmerkende op het gebed dezer plaats.
2Ch 7:16  Want Ik heb nu dit huis verkoren en geheiligd, opdat Mijn Naam daar zij tot in eeuwigheid en Mijn ogen en Mijn hart zullen daar te allen dage zijn.
2Ch 7:17  En u aangaande, zo gij voor Mijn aangezicht wandelen zult, gelijk als uw vader David gewandeld heeft, en doen naar alles, wat Ik u geboden heb, en Mijn inzettingen en Mijn rechten houden zult;
2Ch 7:18  Zo zal Ik den troon uws koninkrijks bevestigen, gelijk als Ik een verbond met uw vader David gemaakt heb, zeggende: Geen man zal u afgesneden worden, die in Israel heerse.
2Ch 7:19  Maar zo gijlieden u afkeren zult, en Mijn inzettingen en Mijn geboden, die Ik voor uw aangezicht gegeven heb, verlaten, en henengaan, en andere goden dienen, en u voor die nederbuigen zult;
2Ch 7:20  Zo zal Ik hen uitrukken uit Mijn land, dat Ik hun gegeven heb, en dit huis, dat Ik Mijn Naam geheiligd heb, zal Ik van Mijn aangezicht wegwerpen, en zal het tot een spreekwoord en spotrede onder alle volken maken.
2Ch 7:21  En dit huis, dat verheven zal geweest zijn, daarover zal zich een ieder, die voorbijgaat, ontzetten, dat hij zal zeggen: Waarom heeft de HEERE aan dit land en aan dit huis alzo gedaan?
2Ch 7:22  En men zal zeggen: Omdat zij den HEERE, hunner vaderen God, verlaten hebben, Die hen uit Egypteland uitgevoerd had, en hebben zich aan andere goden gehouden, en zich voor dezelve nedergebogen, en hen gediend; daarom heeft Hij al dat kwaad over hen gebracht.
2Ch 8:1  Het geschiedde nu ten einde van twintig jaren, in dewelke Salomo het huis des HEEREN en zijn huis gebouwd had,
2Ch 8:2  Dat Salomo de steden, welke Huram hem gegeven had, bouwde, en de kinderen Israels aldaar deed wonen.
2Ch 8:3  Daarna toog Salomo naar Hamath-zoba, en hij overweldigde het.
2Ch 8:4  Hij bouwde ook Thadmor in de woestijn, en al de schatsteden, die hij bouwde in Hamath.
2Ch 8:5  Ook bouwde hij het hoge Beth-horon en het neder Beth-horon, vaste steden met muren, deuren en grendelen;
2Ch 8:6  Mitsgaders Baalath, en al de schatsteden, die Salomo had, en alle wagensteden, en de steden der ruiteren, en wat de begeerte van Salomo begeerd had te bouwen, in Jeruzalem, en in den Libanon, en in het ganse land zijner heerschappij.
2Ch 8:7  Aangaande al het volk, dat overgebleven was van de Hethieten, en de Amorieten, en de Ferezieten, en de Hevieten, en de Jebusieten, die niet uit Israel waren;
2Ch 8:8  Uit hun kinderen, die na hen in het land overgebleven waren, welke de kinderen Israels niet verdaan hadden, die bracht Salomo op uitschot tot op dezen dag.
2Ch 8:9  Doch uit de kinderen Israels, die Salomo niet maakte tot slaven in zijn werk; (want zij waren krijgslieden, en oversten zijner hoofdlieden, en oversten zijner wagenen en zijner ruiteren;)
2Ch 8:10  Uit dezen dan waren oversten der bestelden, die de koning Salomo had, tweehonderd en vijftig, die over het volk heerschappij hadden.
2Ch 8:11  Salomo nu deed de dochter van Farao opkomen uit de stad Davids, tot het huis, dat hij voor haar gebouwd had; want hij zeide: Mijn vrouw zal in het huis van David, den koning van Israel, niet wonen, omdat de plaatsen heilig zijn, tot dewelke de ark des HEEREN gekomen is.
2Ch 8:12  Toen offerde Salomo den HEERE brandofferen op het altaar des HEEREN, hetwelk hij voor het voorhuis gebouwd had;
2Ch 8:13  Zelfs naar den eis van elken dag, offerende, naar het gebod van Mozes, op de sabbatten, en op de nieuwe maanden, en op de gezette hoogtijden, drie malen in het jaar; op het feest van de ongezuurde broden, en op het feest der weken, en op het feest der loofhutten.
2Ch 8:14  Hij stelde ook, naar de wijze zijns vaders Davids, de verdelingen der priesteren over hun dienst, en der Levieten over hun wachten, om God te prijzen, en voor de priesteren te dienen, naar den eis van elken dag; en de poortiers in hun verdelingen, aan elke poort; want alzo was het gebod van David, den man Gods.
2Ch 8:15  En men week niet van des konings gebod aan de priesteren en de Levieten, aangaande alle zaken, en aangaande de schatten.
2Ch 8:16  Alzo werd al het werk van Salomo bereid tot den dag der grondlegging van het huis des HEEREN, en tot het volbrengen van hetzelve, dat het huis des HEEREN volmaakt werd.
2Ch 8:17  Toen toog Salomo naar Ezeon-geber, en naar Eloth, aan den oever der zee, in het land Edom.
2Ch 8:18  En Huram zond hem, door de hand zijner knechten, schepen, mitsgaders knechten, kenners van de zee; en zij gingen met Salomo's knechten naar Ofir, en zij haalden van daar vierhonderd en vijftig talenten gouds, dewelke zij brachten tot den koning Salomo.
2Ch 9:1  En toen de koningin van Scheba het gerucht van Salomo hoorde, kwam zij, om Salomo met raadselen te verzoeken, te Jeruzalem, met een zeer zwaar heir, en kemelen, dragende specerijen en goud in menigte, en kostelijk gesteente; en zij kwam tot Salomo, en sprak met hem al wat in haar hart was.
2Ch 9:2  En Salomo verklaarde haar al haar woorden; en geen ding was er verborgen voor Salomo, dat hij haar niet verklaarde.
2Ch 9:3  Als nu de koningin van Scheba zag de wijsheid van Salomo, en het huis, dat hij gebouwd had,
2Ch 9:4  En de spijze zijner tafel, en het zitten zijner knechten, en het staan zijner dienaren, en hun kledingen, en zijn schenkers, en hun kledingen, en zijn opgang, waardoor hij opging in het huis des HEEREN, zo was in haar geen geest meer.
2Ch 9:5  En zij zeide tot den koning: Het is een waarachtig woord geweest, dat ik in mijn land gehoord heb, van uw zaken en van uw wijsheid.
2Ch 9:6  En ik heb hun woorden niet geloofd, totdat ik gekomen ben, en mijn ogen dat gezien hebben; en zie, de helft van de grootheid uwer wijsheid is mij niet aangezegd; gij hebt overtroffen het gerucht, dat ik gehoord heb.
2Ch 9:7  Welgelukzalig zijn uw mannen, en welgelukzalig deze uw knechten, die geduriglijk voor uw aangezicht staan, en uw wijsheid horen.
2Ch 9:8  Geloofd zij de HEERE, uw God, Die behagen in u gehad heeft, om u op Zijn troon, den HEERE, uw God, tot een koning te zetten; overmits uw God Israel bemint, om hetzelve tot in eeuwigheid op te richten, zo heeft Hij u tot een koning over hen gesteld, om recht en gerechtigheid te doen.
2Ch 9:9  En zij gaf den koning honderd en twintig talenten gouds, en specerijen in grote menigte, en kostelijk gesteente; en er was gelijk deze specerij, die de koningin van Scheba den koning Salomo gaf, geen geweest.
2Ch 9:10  Verder ook Hurams knechten, en Salomo's knechten, die goud brachten uit Ofir, brachten algummimhout en edelgesteente.
2Ch 9:11  En de koning maakte van dat algummimhout hoge gangen tot het huis des HEEREN en tot het huis des konings, mitsgaders harpen en luiten voor de zangers; desgelijks ook was te voren in het land van Juda niet geweest.
2Ch 9:12  En de koning Salomo gaf de koningin van Scheba al haar behagen, wat zij begeerde, behalve hetgeen zij tot den koning gebracht had; zo keerde zij, en toog naar haar land, zij en haar knechten.
2Ch 9:13  Het gewicht nu van het goud, dat voor Salomo op een jaar inkwam, was zeshonderd zes en zestig talenten gouds;
2Ch 9:14  Behalve dat zij van de kramers en de kooplieden inbrachten; ook brachten alle koningen van Arabie, en de vorsten deszelven lands, goud en zilver aan Salomo.
2Ch 9:15  Daartoe maakte de koning Salomo tweehonderd rondassen van geslagen goud; zeshonderd sikkelen van geslagen goud liet hij opwegen tot elke rondas.
2Ch 9:16  Insgelijks driehonderd schilden van geslagen goud; driehonderd sikkelen gouds liet hij opwegen tot elk schild; en de koning leide ze in het huis des wouds van den Libanon.
2Ch 9:17  Nog maakte de koning een groten elpenbenen troon, en hij overtoog denzelven met louter goud.
2Ch 9:18  En de troon had zes trappen en een voetbank van goud, aan den troon vast zijnde, en leuningen aan beide zijden, tot de zitplaats toe; en twee leeuwen stonden bij de leuningen.
2Ch 9:19  En twaalf leeuwen stonden daar aan beide zijden, op de zes trappen; desgelijks is in geen koninkrijk gemaakt geweest.
2Ch 9:20  Ook waren alle drinkvaten van den koning Salomo van goud, en alle vaten van het huis des wouds van den Libanon waren van gesloten goud; het zilver was in de dagen van Salomo niet voor iets geacht.
2Ch 9:21  Want des konings schepen voeren naar Tharsis, met de knechten van Huram; eens in drie jaren kwamen de schepen van Tharsis in, brengende goud, en zilver, elpenbeen, en apen, en pauwen.
2Ch 9:22  Alzo werd de koning Salomo groter dan alle koningen der aarde in rijkdom en wijsheid.
2Ch 9:23  En alle koningen der aarde zochten Salomo's aangezicht, om zijn wijsheid te horen, die God in zijn hart gegeven had.
2Ch 9:24  En zij brachten een ieder zijn geschenk, zilveren vaten, en gouden vaten, en klederen, harnas, en specerijen, paarden, en muilezelen, van elk van jaar tot jaar.
2Ch 9:25  Ook had Salomo vier duizend paardenstallen, en wagenen, en twaalf duizend ruiteren; en hij leide ze in de wagensteden, en bij den koning te Jeruzalem.
2Ch 9:26  En hij heerste over alle koningen, van de rivier tot aan het land der Filistijnen, en tot aan de landpale van Egypte.
2Ch 9:27  Ook maakte de koning het zilver in Jeruzalem te zijn als stenen, en de cederen maakte hij te zijn als de wilde vijgebomen, die in de laagte zijn, in menigte.
2Ch 9:28  En zij brachten voor Salomo paarden uit Egypte, en uit al die landen.
2Ch 9:29  Het overige nu der geschiedenissen van Salomo, der eerste en der laatste, zijn die niet geschreven in de woorden van Nathan, den profeet, en in de profetie van Ahia, den Siloniet, en in de gezichten van Jedi, den ziener, aangaande Jerobeam, den zoon van Nebat?
2Ch 9:30  En Salomo regeerde te Jeruzalem over gans Israel, veertig jaren.
2Ch 9:31  En Salomo ontsliep met zijn vaderen, en zij begroeven hem in de stad zijns vaders Davids; en zijn zoon Rehabeam werd koning in zijn plaats.
2Ch 10:1  En Rehabeam toog naar Sichem; want het ganse Israel was te Sichem gekomen, om hem koning te maken.
2Ch 10:2  Het geschiedde nu, als Jerobeam, de zoon van Nebat, dat hoorde (dezelve nu was in Egypte, alwaar hij van het aangezicht van den koning Salomo gevloden was), dat Jerobeam uit Egypte wederkeerde;
2Ch 10:3  Want zij zonden henen, en lieten hem roepen; zo kwam Jerobeam met het ganse Israel, en zij spraken tot Rehabeam, zeggende:
2Ch 10:4  Uw vader heeft ons juk hard gemaakt, nu dan, maak gij uws vaders harden dienst, en zijn zwaar juk, dat hij ons opgelegd heeft, lichter, en wij zullen u dienen.
2Ch 10:5  En hij zeide tot hen: Komt over drie dagen weder tot mij. En het volk ging heen.
2Ch 10:6  En de koning Rehabeam hield raad met de oudsten, die gestaan hadden voor het aangezicht van zijn vader Salomo, als hij leefde, zeggende: Hoe raadt gijlieden, dat men dit volk antwoorden zal?
2Ch 10:7  En zij spraken tot hem, zeggende: Indien gij dit volk goedertieren en jegens hen goedwillig wezen zult, en tot hen goede woorden spreken, zo zullen zij te allen dage uw knechten zijn.
2Ch 10:8  Maar hij verliet den raad der oudsten, dien zij hem geraden hadden; en hij hield raad met de jongelingen, die met hem opgewassen waren, die voor zijn aangezicht stonden.
2Ch 10:9  En hij zeide tot hen: Wat raadt gijlieden, dat wij dit volk antwoorden zullen, die tot mij gesproken hebben, zeggende: Maak het juk, dat uw vader ons opgelegd heeft, lichter?
2Ch 10:10  En de jongelingen die met hem opgewassen waren, spraken tot hem, zeggende: Alzo zult gij zeggen tot dat volk, dat tot u gesproken heeft, zeggende: Uw vader heeft ons juk zwaar gemaakt, maar maak gij het over ons lichter; alzo zult gij tot hen spreken: Mijn kleinste vinger zal dikker zijn dan mijns vaders lenden.
2Ch 10:11  Indien nu mijn vader een zwaar juk op u heeft doen laden, zo zal ik boven uw juk nog daartoe doen; mijn vader heeft u met geselen gekastijd, maar ik zal u met schorpioenen kastijden.
2Ch 10:12  Zo kwam Jerobeam en al het volk tot Rehabeam, op den derden dag, gelijk als de koning gesproken had, zeggende: Komt weder tot mij op den derden dag.
2Ch 10:13  En de koning antwoordde hun hardelijk; want de koning Rehabeam verliet den raad der oudsten.
2Ch 10:14  En hij sprak tot hen naar den raad der jongelingen, zeggende: Mijn vader heeft uw juk zwaar gemaakt, maar ik zal nog daarboven toedoen; mijn vader heeft u met geselen gekastijd, maar ik zal u met schorpioenen kastijden.
2Ch 10:15  Alzo hoorde de koning naar het volk niet; want deze omwending was van God, opdat de HEERE Zijn woord bevestigde, hetwelk Hij door den dienst van Ahia, den Siloniet, gesproken had tot Jerobeam, den zoon van Nebat.
2Ch 10:16  Toen het ganse volk Israel zag, dat de koning naar hen niet hoorde, zo antwoordde het volk den koning, zeggende: Wat deel hebben wij aan David? Ja, geen erve hebben wij aan den zoon van Isai; een ieder naar uw tenten, o Israel! Voorzie nu uw huis, o David! Zo ging het ganse Israel naar zijn tenten.
2Ch 10:17  Doch aangaande de kinderen van Israel, die in de steden van Juda woonden, over die regeerde Rehabeam ook.
2Ch 10:18  Toen zond de koning Rehabeam Hadoram, die over de schatting was; en de kinderen Israels stenigden hem met stenen, dat hij stierf; maar de koning Rehabeam verkloekte zich, om op een wagen te klimmen, dat hij naar Jeruzalem vluchtte.
2Ch 10:19  Alzo vielen de Israelieten van het huis van David af, tot op dezen dag.
2Ch 11:1  Toen nu Rehabeam te Jeruzalem gekomen was, vergaderde hij het huis van Juda en Benjamin, eenhonderd en tachtig duizend uitgelezenen, geoefend ten oorlog, om tegen Israel te strijden, opdat hij het koninkrijk weder aan Rehabeam bracht.
2Ch 11:2  Doch het woord des HEEREN geschiedde tot Semaja, den man Gods, zeggende:
2Ch 11:3  Zeg tot Rehabeam, den zoon van Salomo, den koning van Juda, en tot het ganse Israel in Juda en Benjamin, zeggende:
2Ch 11:4  Zo zegt de HEERE: Gij zult niet optrekken, noch strijden tegen uw broederen; een ieder kere weder tot zijn huis, want deze zaak is van Mij geschied. En zij hoorden de woorden des HEEREN, en zij keerden weder van tegen Jerobeam te trekken.
2Ch 11:5  Rehabeam nu woonde te Jeruzalem; en hij bouwde steden tot vastigheden in Juda.
2Ch 11:6  Hij bouwde nu Bethlehem, en Etham, en Thekoa,
2Ch 11:7  En Beth-zur, en Socho, en Adullam,
2Ch 11:8  En Gath, en Maresa, en Zif,
2Ch 11:9  En Adoraim, en Lachis, en Azeka,
2Ch 11:10  En Zora, en Ajalon, en Hebron; dewelke in Juda en in Benjamin de vaste steden waren.
2Ch 11:11  En hij sterkte deze vastigheden, en leide oversten daarin, en schatten van spijs, en olie, en wijn;
2Ch 11:12  En in elke stad rondassen en spiesen, en sterkte ze gans zeer; zo was Juda, en Benjamin zijne.
2Ch 11:13  Daartoe de priesteren en de Levieten, die in het ganse Israel waren, stelden zich bij hem uit al hun landpalen.
2Ch 11:14  Want de Levieten verlieten hun voorsteden en hun bezitting, en kwamen in Juda en in Jeruzalem; want Jerobeam en zijn zonen hadden hen verstoten, van het priesterdom des HEEREN te mogen bedienen.
2Ch 11:15  En hij had zich priesteren gesteld voor de hoogte, en voor de duivelen, en voor de kalveren, die hij gemaakt had.
2Ch 11:16  Na die kwamen ook uit alle stammen van Israel te Jeruzalem, die hun hart begaven, om den HEERE, den God Israels, te zoeken, dat zij den HEERE, den God hunner vaderen, offerande deden.
2Ch 11:17  Alzo sterkten zij het koninkrijk van Juda, en bekrachtigden Rehabeam, den zoon van Salomo, drie jaren; want drie jaren wandelden zij in den weg van David, en Salomo.
2Ch 11:18  En Rehabeam nam zich, benevens Mahalath, de dochter van Jerimoth, den zoon van David, ter vrouwe Abihail, de dochter van Eliab, den zoon van Isai,
2Ch 11:19  Dewelke hem zonen baarde, Jeus, en Semaria, en Zaham.
2Ch 11:20  En na haar nam hij Maacha, de dochter van Absalom; deze baarde hem Abia, en Attai, en Ziza, en Selomith.
2Ch 11:21  En Rehabeam had Maacha, Absaloms dochter, liever dan al zijn vrouwen en zijn bijwijven; want hij had achttien vrouwen genomen, en zestig bijwijven; en hij gewon acht en twintig zonen en zestig dochteren.
2Ch 11:22  En Rehabeam stelde Abia, den zoon van Maacha, tot een hoofd, om een overste te zijn onder zijn broederen; want het was om hem koning te maken.
2Ch 11:23  En hij handelde verstandelijk, dat hij van al zijn zonen, door alle landen van Juda en Benjamin, in alle vaste steden verspreidde, denwelken hij spijze gaf in overvloed; en hij begeerde de veelheid van vrouwen.
2Ch 12:1  Het geschiedde nu, als Rehabeam het koninkrijk bevestigd had, en hij sterk geworden was, dat hij de wet des HEEREN verliet, en gans Israel met hem.
2Ch 12:2  Daarom geschiedde het, in het vijfde jaar van den koning Rehabeam, dat Sisak, de koning van Egypte, tegen Jeruzalem optoog (want zij hadden overtreden tegen den HEERE),
2Ch 12:3  Met duizend en tweehonderd wagenen, en met zestig duizend ruiteren; en des volks was geen getal, dat met hem kwam uit Egypte, Libyers, Suchieten en Moren;
2Ch 12:4  En hij nam de vaste steden in, die Juda had, en hij kwam tot Jeruzalem toe.
2Ch 12:5  Toen kwam Semaja, de profeet, tot Rehabeam en de oversten van Juda, die te Jeruzalem verzameld waren, uit oorzaak van Sisak, en hij zeide tot hen: Alzo zegt de HEERE: Gij hebt Mij verlaten, daarom heb Ik u ook verlaten in de hand van Sisak.
2Ch 12:6  Toen verootmoedigden zich de oversten van Israel en de koning, en zij zeiden: De HEERE is rechtvaardig.
2Ch 12:7  Als nu de HEERE zag, dat zij zich verootmoedigden, geschiedde het woord des HEEREN tot Semaja, zeggende: Zij hebben zich verootmoedigd, Ik zal hen niet verderven; maar Ik zal hun in kort ontkoming geven, dat Mijn grimmigheid over Jeruzalem door de hand van Sisak niet zal uitgegoten worden.
2Ch 12:8  Doch zij zullen hem tot knechten zijn, opdat zij onderkennen Mijn dienst, en den dienst van de koninkrijken der landen.
2Ch 12:9  Zo toog Sisak, de koning van Egypte, op tegen Jeruzalem; en hij nam de schatten van het huis des HEEREN en de schatten van het huis des konings weg; hij nam alles weg; hij nam ook al de gouden schilden weg, die Salomo gemaakt had.
2Ch 12:10  En de koning Rehabeam maakte, in plaats van die, koperen schilden; en hij beval die onder de hand van de oversten der trawanten, die de deur van het huis des konings bewaarden.
2Ch 12:11  En het geschiedde, zo wanneer de koning in het huis des HEEREN ging, dat de trawanten kwamen, en die droegen, en die wederbrachten in der trawanten wachtkamer.
2Ch 12:12  En als hij zich verootmoedigde, keerde de toorn des HEEREN van hem af, opdat Hij hem niet ten uiterste toe verdierf; ook waren in Juda nog goede dingen.
2Ch 12:13  Zo versterkte zich de koning Rehabeam in Jeruzalem, en regeerde; want Rehabeam was een en veertig jaren oud, als hij koning werd, en hij regeerde zeventien jaren in Jeruzalem, de stad, die de HEERE uit alle stammen van Israel verkoren had, om Zijn Naam daar te zetten; en de naam zijner moeder was Naama, een Ammonietische.
2Ch 12:14  En hij deed dat kwaad was, dewijl hij zijn hart niet richtte, om den HEERE te zoeken.
2Ch 12:15  De geschiedenissen nu van Rehabeam, de eerste en de laatste, zijn die niet geschreven in de woorden van Semaja, den profeet, en Iddo, den ziener, verhalende de geslachtsregisteren; daartoe de krijgen van Rehabeam en Jerobeam in al hun dagen?
2Ch 12:16  En Rehabeam ontsliep met zijn vaderen, en werd begraven in de stad Davids; en zijn zoon Abia werd koning in zijn plaats.
2Ch 13:1  In het achttiende jaar van den koning Jerobeam, zo werd Abia koning over Juda.
2Ch 13:2  Hij regeerde drie jaren te Jeruzalem; en de naam zijner moeder was Michaja, de dochter van Uriel, van Gibea; en er was krijg tussen Abia en tussen Jerobeam.
2Ch 13:3  En Abia bond den strijd aan met een heir van strijdbare helden, vierhonderd duizend uitgelezen mannen; en Jerobeam stelde tegen hem de slagorde, met achthonderd duizend uitgelezen mannen, kloeke helden.
2Ch 13:4  En Abia maakte zich op van boven den berg Zemaraim, dewelke is in het gebergte van Efraim; en hij zeide: Hoort mij toe, Jerobeam, en gans Israel!
2Ch 13:5  Staat het u niet toe te weten, dat de HEERE, de God Israels, het koninkrijk over Israel aan David gegeven heeft, tot in eeuwigheid, hem en zijn zonen, met een zoutverbond?
2Ch 13:6  Evenwel is Jerobeam, de zoon van Nebat, de knecht van Salomo, den zoon van David, opgestaan, en heeft gerebelleerd tegen zijn heer.
2Ch 13:7  Daartoe hebben zich ijdele mannen, kinderen Belials, tot hem vergaderd, en hebben zich sterk gemaakt tegen Rehabeam, den zoon van Salomo, als Rehabeam jong was en teder van hart, dat hij zich tegen hen niet kon versterken.
2Ch 13:8  En nu, gij denkt u te versterken tegen het koninkrijk des HEEREN, hetwelk in de hand is der zonen van David; gij zijt wel een grote menigte, maar gij hebt gouden kalveren bij u, die u Jerobeam tot goden gemaakt heeft.
2Ch 13:9  Hebt gij niet de priesteren des HEEREN, de zonen van Aaron, en de Levieten uitgedreven, en hebt u priesteren gemaakt, gelijk de volken der landen? Een iegelijk, die komt om zijn hand te vullen met een jong rund en zeven rammen, die wordt priester dergenen, die geen goden zijn.
2Ch 13:10  Maar ons aangaande, de HEERE is onze God, en wij hebben Hem niet verlaten; en de priesters, die den HEERE dienen, zijn de zonen van Aaron, en de Levieten zijn in het werk.
2Ch 13:11  En zij steken aan voor den HEERE brandofferen, op elken morgen en op elken avond, ook reukwerk van welriekende specerijen, nevens de toerichting des broods op de reine tafel, en den gouden kandelaar en zijn lampen, om die op elken avond te doen branden; want wij nemen waar de wacht des HEEREN, onzes Gods; maar gij hebt Hem verlaten.
2Ch 13:12  Daarom ziet, God is met ons aan de spitse, en Zijn priesteren met de trompetten des geklanks, om tegen u alarmgeklank te maken; o kinderen Israels, strijdt niet tegen den HEERE, den God uwer vaderen, want gij zult geen voorspoed hebben.
2Ch 13:13  Maar Jerobeam deed een achterlage omwenden, om van achter hen te komen; zo waren zij voor het aangezicht van Juda, en de achterlage was achter hen.
2Ch 13:14  Toen nu Juda omzag, ziet, zo hadden zij den strijd voor en achter; en zij riepen tot den HEERE, en de priesters trompetten met de trompetten.
2Ch 13:15  En de mannen van Juda maakten een alarmgeschrei; en het geschiedde, als de mannen van Juda een alarmgeschrei maakten, dat God Jerobeam en het ganse Israel sloeg voor Abia en Juda.
2Ch 13:16  En de kinderen Israels vloden voor het aangezicht van Juda; en God gaf hen in hun hand.
2Ch 13:17  Abia dan, en zijn volk, sloeg hen met een groten slag; want uit Israel vielen verslagen vijfhonderd duizend uitgelezen mannen.
2Ch 13:18  Alzo werden de kinderen Israels vernederd te dier tijd; maar de kinderen van Juda werden machtig, dewijl zij op den HEERE, hunner vaderen God, gesteund hadden.
2Ch 13:19  En Abia jaagde Jerobeam achterna, en nam van hem de steden, Beth-el met haar onderhorige plaatsen, en Jesana met haar onderhorige plaatsen, en Efron met haar onderhorige plaatsen.
2Ch 13:20  En Jerobeam behield geen kracht meer in de dagen van Abia; maar de HEERE sloeg hem, dat hij stierf.
2Ch 13:21  Zo versterkte zich Abia; en hij nam zich veertien vrouwen, en gewon twee en twintig zonen en zestien dochteren.
2Ch 13:22  Het overige nu der geschiedenissen van Abia, zo zijn wegen als zijn woorden, zijn beschreven in de historie van den profeet Iddo.
2Ch 14:1  Zo ontsliep Abia met zijn vaderen, en zij begroeven hem in de stad Davids, en zijn zoon Asa werd koning in zijn plaats. In zijn dagen was het land tien jaren stil.
2Ch 14:2  En Asa deed dat goed en dat recht was in de ogen des HEEREN, zijns Gods.
2Ch 14:3  Want hij nam de altaren der vreemden, en de hoogten weg, en brak de opgerichte beelden, en hieuw de bossen af.
2Ch 14:4  En hij zeide tot Juda, dat zij den HEERE, den God hunner vaderen, zoeken, en dat zij de wet en het gebod doen zouden.
2Ch 14:5  Hij nam ook weg uit alle steden van Juda de hoogten en de zonnebeelden; en het koninkrijk was voor hem stil.
2Ch 14:6  Daartoe bouwde hij vaste steden in Juda; want het land was stil, en er was geen oorlog in die jaren tegen hem, dewijl de HEERE hem rust gaf.
2Ch 14:7  Want hij zeide tot Juda: Laat ons deze steden bouwen, en muren daarom trekken, en torens, deuren en grendelen, terwijl het land nog is voor ons aangezicht; want wij hebben den HEERE, onzen God, gezocht, wij hebben Hem gezocht, en Hij heeft ons rondom henen rust gegeven. Zo bouwden zij en hadden voorspoed.
2Ch 14:8  Asa nu had een heir van driehonderd duizend uit Juda, rondas en spies dragende, en tweehonderd en tachtig duizend uit Benjamin, het schild dragende en den boog spannende; al dezen waren kloeke helden.
2Ch 14:9  En Zerah, de Moor, kwam tegen hen uit, met een heir van duizend maal duizend, en driehonderd wagenen; en hij kwam tot Maresa toe.
2Ch 14:10  Toen toog Asa tegen hem uit; en zij stelden de slagorde in het dal Zefatha bij Maresa.
2Ch 14:11  En Asa riep tot den HEERE, zijn God, en zeide: HEERE, het is niets bij U, te helpen hetzij den machtige, hetzij den krachteloze; help ons, o HEERE, onze God! Want wij steunen op U, en in Uw Naam zijn wij gekomen tegen deze menigte; o HEERE! Gij zijt onze God; laat den sterfelijken mens tegen U niets vermogen.
2Ch 14:12  En de HEERE plaagde de Moren voor Asa en voor Juda; en de Moren vloden.
2Ch 14:13  Asa nu en het volk, dat met hem was, jaagden hen na tot Gerar toe; en zo velen vielen er van de Moren, dat er voor hen geen hervatting was; want zij waren verbroken voor den HEERE en voor Zijn leger; en zij droegen zeer veel roofs daarvan.
2Ch 14:14  En zij sloegen alle steden rondom Gerar; want de verschrikking des HEEREN was over hen; en zij beroofden al de steden, omdat veel roofs in dezelve was.
2Ch 14:15  En zij sloegen ook de tenten van het vee, en voerden weg schapen in menigte, en kemelen; en kwamen weder te Jeruzalem.
2Ch 15:1  Toen kwam de Geest Gods op Azaria, den zoon van Oded.
2Ch 15:2  En hij ging uit, Asa tegen, en hij zeide tot hem: Hoort mij, Asa, en gans Juda, en Benjamin! De HEERE is met ulieden, terwijl gij met Hem zijt; en zo gij Hem zoekt, Hij zal van u gevonden worden; maar zo gij Hem verlaat, Hij zal u verlaten.
2Ch 15:3  Israel nu is vele dagen geweest zonder den waren God, en zonder een lerenden priester, en zonder de wet.
2Ch 15:4  Maar als zij zich in hun nood bekeerden tot den HEERE, den God Israels, en Hem zochten, zo werd Hij van hen gevonden.
2Ch 15:5  En in die tijden was er geen vrede voor dengene, die uitging, en dengene, die inkwam; maar vele beroerten waren over al de inwoners van die landen;
2Ch 15:6  Dat volk tegen volk, en stad tegen stad in stukken gestoten werden; want God had hen met allen angst verschrikt.
2Ch 15:7  Daarom weest gij sterk, en laat uw handen niet verslappen; want er is loon naar uw werk.
2Ch 15:8  Als nu Asa deze woorden hoorde, en de profetie van den profeet Oded, sterkte hij zich, en hij deed weg de verfoeiselen uit het ganse land van Juda en Benjamin, en uit de steden, die hij van het gebergte van Efraim genomen had, en vernieuwde het altaar des HEEREN, dat voor het voorhuis des HEEREN was.
2Ch 15:9  En hij vergaderde het ganse Juda en Benjamin, en de vreemdelingen met hen uit Efraim, en Manasse, en uit Simeon; want uit Israel vielen zij tot hem in menigte, als zij zagen, dat de HEERE, zijn God, met hem was.
2Ch 15:10  En zij vergaderden zich te Jeruzalem, in de derde maand, in het vijftiende jaar van het koninkrijk van Asa.
2Ch 15:11  En zij offerden den HEERE ten zelfden dage van den roof, dien zij gebracht hadden, zevenhonderd runderen en zeven duizend schapen.
2Ch 15:12  En zij traden in een verbond, dat zij den HEERE, den God hunner vaderen, zoeken zouden met hun ganse hart en met hun ganse ziel.
2Ch 15:13  En al wie den HEERE, den God Israels, niet zou zoeken, zou gedood worden, van den kleine tot den grote, en van den man tot de vrouw toe.
2Ch 15:14  En zij zwoeren den HEERE met luider stem en met gejuich, desgelijks met trompetten en met bazuinen.
2Ch 15:15  En gans Juda was verblijd over dezen eed; want zij hadden met hun ganse hart gezworen, en met hun gansen wil Hem gezocht; en Hij werd van hen gevonden, en de HEERE gaf hun rust rondom henen.
2Ch 15:16  Aangaande ook Maacha, de moeder van den koning Asa, hij zette haar af, dat zij geen koningin ware, omdat zij een afgrijselijken afgod in een bos gemaakt had; ook roeide Asa haar afgrijselijken afgod uit, en verbrijzelde en verbrandde hem aan de beek Kidron.
2Ch 15:17  De hoogten werden wel niet weggenomen uit Israel, het hart van Asa nochtans was volkomen al zijn dagen.
2Ch 15:18  En hij bracht in het huis Gods de geheiligde dingen zijns vaders, en zijn geheiligde dingen, zilver en goud, en vaten.
2Ch 15:19  En er was geen oorlog tot in het vijf en dertigste jaar van het koninkrijk van Asa.
2Ch 16:1  In het zes en dertigste jaar van het koninkrijk van Asa, toog Baesa, de koning van Israel, op tegen Juda, en bouwde Rama, opdat hij niemand toeliet uit te gaan en in te komen tot Asa, den koning van Juda.
2Ch 16:2  Toen bracht Asa het zilver en het goud voort, uit de schatten van het huis des HEEREN en van het huis des konings, en zond tot Benhadad, den koning van Syrie, die te Damaskus woonde, zeggende:
2Ch 16:3  Er is een verbond tussen mij en tussen u, en tussen mijn vader en tussen uw vader; zie, ik zend u zilver en goud, ga heen, maak uw verbond te niet met Baesa, den koning van Israel, dat hij van tegen mij aftrekke.
2Ch 16:4  En Benhadad hoorde naar den koning Asa, en zond de oversten der heiren, die hij had, tegen de steden van Israel, en zij sloegen Ijon, en Dan, en Abel-maim, en alle schatsteden van Nafthali.
2Ch 16:5  En het geschiedde, als Baesa zulks hoorde, dat hij afliet van Rama te bouwen, en zijn werk staakte.
2Ch 16:6  Toen nam de koning Asa gans Juda, en zij droegen weg de stenen van Rama, en het hout daarvan, waarmede Baesa gebouwd had; en hij bouwde daarmede Geba en Mizpa.
2Ch 16:7  En in denzelfden tijd kwam de ziener Hanani tot Asa, den koning van Juda, en hij zeide tot hem: Omdat gij gesteund hebt op den koning van Syrie, en niet gesteund hebt op den HEERE, uw God, daarom is het heir des konings van Syrie uit uw hand ontkomen.
2Ch 16:8  Waren niet de Moren en de Libyers een groot heir met zeer veel wagenen en ruiteren? Toen gij nochtans op den HEERE steundet, heeft Hij hen in uw hand gegeven.
2Ch 16:9  Want den HEERE aangaande, Zijn ogen doorlopen de ganse aarde, om Zich sterk te bewijzen aan degenen, welker hart volkomen is tot Hem; gij hebt hierin zottelijk gedaan; want van nu af zullen oorlogen tegen u zijn.
2Ch 16:10  Doch Asa werd toornig tegen den ziener, en leidde hem in het gevangenhuis; want hij was hierover tegen hem ontsteld; daartoe onderdrukte Asa enigen uit het volk ter zelfder tijd.
2Ch 16:11  En ziet, de geschiedenissen van Asa, de eerste met de laatste, ziet, zij zijn beschreven in het boek der koningen van Juda en Israel.
2Ch 16:12  Asa nu werd, in het negen en dertigste jaar van zijn koninkrijk, krank aan zijn voeten; tot op het hoogste toe was zijn krankheid; daartoe ook zocht hij den HEERE niet in zijn krankheid, maar de medicijnmeesters.
2Ch 16:13  Alzo ontsliep Asa met zijn vaderen; en hij stierf in het een en veertigste jaar zijner regering.
2Ch 16:14  En zij begroeven hem in zijn graf, dat hij voor zich gegraven had in de stad Davids, en leiden hem op het bed, hetwelk hij gevuld had met specerijen, en dat van verscheidene soorten, naar apothekerskunst toebereid; en zij brandden over hem een gans grote branding.
2Ch 17:1  En zijn zoon Josafat werd koning in zijn plaats, en hij sterkte zich tegen Israel.
2Ch 17:2  En hij leide krijgsvolk in alle vaste steden van Juda, en leide bezettingen in het land van Juda, en in de steden van Efraim, die zijn vader Asa ingenomen had.
2Ch 17:3  En de HEERE was met Josafat; want hij wandelde in de vorige wegen zijns vaders Davids, en zocht de Baals niet.
2Ch 17:4  Maar hij zocht den God zijns vaders, en wandelde in Zijn geboden, en niet naar het doen van Israel.
2Ch 17:5  En de HEERE bevestigde het koninkrijk in zijn hand, en gans Juda gaf Josafat geschenken; en hij had rijkdom en eer in menigte.
2Ch 17:6  En zijn hart verhief zich in de wegen des HEEREN; en hij nam verder de hoogten en de bossen uit Juda weg.
2Ch 17:7  In het derde jaar nu zijner regering zond hij tot zijn vorsten, tot Ben-chail, en tot Obadja, en tot Zecharja, en tot Nathaneel, en tot Michaja, opdat men zou leren in de steden van Juda.
2Ch 17:8  En met hen de Levieten, Semaja en Nethanja, en Zebadja, en Asael, en Semiramoth, en Jonathan, en Adonia, en Tobia, en Tob-adonia, de Levieten, en met hen de priesters Elisama en Joram.
2Ch 17:9  En zij leerden in Juda, en het wetboek des HEEREN was bij hen; en zij gingen rondom in alle steden van Juda, en leerden onder het volk.
2Ch 17:10  En een verschrikking des HEEREN werd over alle koninkrijken der landen, die rondom Juda waren, dat zij niet krijgden tegen Josafat.
2Ch 17:11  En van de Filistijnen brachten zij Josafat geschenken met het opgelegde geld; ook brachten hem de Arabieren klein vee, zeven duizend en zevenhonderd rammen, en zeven duizend en zevenhonderd bokken.
2Ch 17:12  Alzo nam Josafat toe, en werd ten hoogste groot; daartoe bouwde hij in Juda burchten en schatsteden.
2Ch 17:13  En hij had veel werks in de steden van Juda, en krijgslieden, kloeke helden in Jeruzalem.
2Ch 17:14  Dit nu is hun telling, naar de huizen hunner vaderen. In Juda waren oversten der duizenden: Adna de overste, en met hem waren driehonderd duizend kloeke helden.
2Ch 17:15  Naast hem nu was de overste Johanan; en met hem waren tweehonderd en tachtig duizend;
2Ch 17:16  Naast hem was Amasia, de zoon van Zichri, die zich vrijwillig den HEERE overgegeven had; en met hem waren tweehonderd duizend kloeke helden.
2Ch 17:17  En uit Benjamin was Eljada, een kloek held; en met hem tweehonderd duizend, die met boog en schild gewapend waren.
2Ch 17:18  En naast hem was Jozabad; en met hem waren honderd en tachtig duizend, ten krijge toegerust.
2Ch 17:19  Dezen waren in den dienst des konings; behalve degenen, die de koning in de vaste steden door gans Juda gezet had.
2Ch 18:1  Josafat nu had rijkdom en eer in overvloed; en hij verzwagerde zich aan Achab.
2Ch 18:2  En ten einde van enige jaren toog hij af tot Achab naar Samaria; en Achab slachtte schapen en runderen voor hem in menigte, en voor het volk, dat met hem was; en hij porde hem aan, om op te trekken naar Ramoth in Gilead.
2Ch 18:3  Want Achab, de koning van Israel, zeide tot Josafat, den koning van Juda: Zult gij met mij gaan naar Ramoth in Gilead? En hij zeide tot hem: Zo zal ik zijn, gelijk gij zijt, en gelijk uw volk is, zal mijn volk zijn, en wij zullen met u zijn in dezen krijg.
2Ch 18:4  Verder zeide Josafat tot den koning van Israel: Vraag toch als heden naar het woord des HEEREN.
2Ch 18:5  Toen vergaderde de koning van Israel de profeten, vierhonderd mannen, en hij zeide tot hen: Zullen wij tegen Ramoth in Gilead ten strijde trekken, of zal ik het nalaten? En zij zeiden: Trek op, want God zal hen in de hand des konings geven.
2Ch 18:6  Maar Josafat zeide: Is hier niet nog een profeet des HEEREN, dat wij van hem vragen mochten?
2Ch 18:7  Toen zeide de koning van Israel tot Josafat: Er is nog een man, om door hem den HEERE te vragen; maar ik haat hem, want hij profeteert over mij niets goeds, maar altijd kwaad; deze is Micha, de zoon van Jimla. En Josafat zeide: de koning zegge niet alzo.
2Ch 18:8  Toen riep de koning van Israel een kamerling, en hij zeide: Haal haastelijk Micha, den zoon van Jimla.
2Ch 18:9  De koning van Israel nu en Josafat, de koning van Juda, zaten elk op zijn troon, bekleed met hun klederen, en zij zaten op het plein, aan de deur der poort van Samaria; en al de profeten profeteerden in hun tegenwoordigheid.
2Ch 18:10  En Zedekia, de zoon van Kenaana, had zich ijzeren hoornen gemaakt, en hij zeide: Zo zegt de HEERE: Met deze zult gij de Syriers stoten, totdat gij hen gans verdaan zult hebben.
2Ch 18:11  En al de profeten profeteerden alzo, zeggende: Trek op naar Ramoth in Gilead, en gij zult voorspoedig zijn, want de HEERE zal hen in de hand des konings geven.
2Ch 18:12  De bode nu, die heengegaan was, om Micha te roepen, sprak tot hem, zeggende: Zie, de woorden der profeten zijn, uit een mond, goed tot den koning; dat nu toch uw woord zij, gelijk als van een uit hen, en spreek het goede.
2Ch 18:13  Doch Micha zeide: Zo waarachtig als de HEERE leeft, hetgeen mijn God zeggen zal, dat zal ik spreken!
2Ch 18:14  Als hij tot den koning gekomen was, zo zeide de koning tot hem: Micha, zullen wij naar Ramoth in Gilead ten strijde trekken, of zal ik het nalaten? En hij zeide: Trekt op, en gijlieden zult voorspoedig zijn, want zij zullen in uw hand gegeven worden.
2Ch 18:15  En de koning zeide tot hem: Tot hoevele reizen zal ik u bezweren, opdat gij tot mij niet spreekt, dan de waarheid, in den Naam des HEEREN?
2Ch 18:16  En hij zeide: Ik zag het ganse Israel verstrooid op de bergen, gelijk schapen, die geen herder hebben; en de HEERE zeide: Dezen hebben geen heer; een iegelijk kere weder naar zijn huis in vrede.
2Ch 18:17  Toen zeide de koning van Israel tot Josafat: Heb ik tot u niet gezegd: Hij zal over mij niets goeds, maar kwaad profeteren?
2Ch 18:18  Verder zeide hij: Daarom hoort het woord des HEEREN: Ik zag den HEERE, zittende op Zijn troon, en al het hemelse heir, staande aan Zijn rechter hand en Zijn linkerhand.
2Ch 18:19  En de HEERE zeide: Wie zal Achab, den koning van Israel, overreden, dat hij optrekke, en valle te Ramoth in Gilead? Daarna zeide Hij: Deze zegt aldus, en die zegt alzo.
2Ch 18:20  Toen kwam een geest voort, en stond voor het aangezicht des HEEREN, en zeide: Ik zal hem overreden. En de HEERE zeide tot hem: Waarmede?
2Ch 18:21  En Hij zeide: Ik zal uitgaan, en een leugengeest zijn in den mond van al zijn profeten. En Hij zeide: Gij zult overreden, en zult ook vermogen; ga uit, en doe alzo.
2Ch 18:22  Nu dan, zie, de HEERE heeft een leugengeest in den mond van deze uw profeten gegeven, en de HEERE heeft kwaad over u gesproken.
2Ch 18:23  Toen trad Zedekia, de zoon van Kenaana, toe, en sloeg Micha op het kinnebakken, en hij zeide: Door wat weg is de Geest des HEEREN van mij doorgegaan, om u aan te spreken?
2Ch 18:24  En Micha zeide: Zie, gij zult het zien aan dienzelfden dag, als gij zult gaan van kamer in kamer, om u te versteken.
2Ch 18:25  De koning van Israel nu zeide: Neemt Micha, en brengt hem weder tot Amon, den overste der stad, en tot Joas, den zoon des konings;
2Ch 18:26  En gijlieden zult zeggen: Zo zegt de koning: Zet dezen in het gevangenhuis, en spijst hem met brood der bedruktheid, en met water der bedruktheid, totdat ik met vrede wederkom.
2Ch 18:27  En Micha zeide: Indien gij enigszins met vrede wederkomt, zo heeft de HEERE door mij niet gesproken. Verder zeide hij: Hoort, gij volken altegaar!
2Ch 18:28  Alzo toog de koning van Israel, en Josafat, de koning van Juda, op naar Ramoth in Gilead.
2Ch 18:29  En de koning van Israel zeide tot Josafat: Als ik mij versteld heb, zal ik in den strijd komen; maar gij, trek uw klederen aan. Alzo verstelde zich de koning van Israel, en zij kwamen in den strijd.
2Ch 18:30  De koning nu van Syrie had geboden aan de oversten der wagenen, die hij had, zeggende: Gijlieden zult niet strijden tegen kleinen noch groten, maar tegen den koning van Israel alleen.
2Ch 18:31  Het geschiedde dan, als de oversten der wagenen Josafat zagen, dat zij zeiden: Die is de koning van Israel; en zij togen rondom hem, om te strijden; maar Josafat riep, en de HEERE hielp hem, en God wendde hen van hem af.
2Ch 18:32  Want het geschiedde, als de oversten der wagenen zagen, dat het de koning van Israel niet was, dat zij van achter hem afkeerden.
2Ch 18:33  Toen spande een man den boog in zijn eenvoudigheid, en schoot den koning van Israel tussen de gespen en tussen het pantsier. Toen zeide hij tot den voerman: Keer uw hand en voer mij uit het leger, want ik ben verwond.
2Ch 18:34  En de strijd nam op dien dag toe, en de koning van Israel deed zich met den wagen staande houden tegenover de Syriers, tot den avond toe; en hij stierf ter tijd, als de zon onderging.
2Ch 19:1  En Josafat, de koning van Juda, keerde met vrede weder naar zijn huis te Jeruzalem.
2Ch 19:2  En Jehu, de zoon van Hanani, de ziener, ging uit, hem tegen, en zeide tot den koning Josafat: Zoudt gij den goddeloze helpen, en die den HEERE haten, liefhebben? Nu is daarom over u van het aangezicht des HEEREN grote toornigheid.
2Ch 19:3  Evenwel goede dingen zijn bij u gevonden; want gij hebt de bossen uit het land weggedaan, en uw hart gericht om God te zoeken.
2Ch 19:4  Josafat nu woonde in Jeruzalem; en hij toog wederom uit door het volk, van Ber-seba af tot het gebergte van Efraim toe, en deed hen wederkeren tot den HEERE, hunner vaderen God.
2Ch 19:5  En hij stelde richters in het land, in alle vaste steden van Juda, van stad tot stad.
2Ch 19:6  En hij zeide tot de richters: Ziet wat gij doet, want gij houdt het gericht niet den mens, maar den HEERE; en Hij is bij u in de zaak van het gericht.
2Ch 19:7  Nu dan, de verschrikking des HEEREN zij op ulieden; neemt waar, en doet het; want bij den HEERE, onzen God, is geen onrecht, noch aanneming van personen, noch ontvanging van geschenken.
2Ch 19:8  Daartoe stelde Josafat ook te Jeruzalem enige van de Levieten, en van de priesteren, en van de hoofden der vaderen van Israel, over het gericht des HEEREN, en over rechtsgeschillen, als zij weder te Jeruzalem gekomen waren.
2Ch 19:9  En hij gebood hun, zeggende: Doet alzo in de vreze des HEEREN, met getrouwheid en met een volkomen hart.
2Ch 19:10  En in alle geschil, hetwelk van uw broederen, die in hun steden wonen, tot u zal komen, tussen bloed en bloed, tussen wet en gebod, en inzettingen en rechten, zo vermaant hen, dat zij niet schuldig worden aan den HEERE, en een grote toornigheid over u en over uw broederen zij; doet alzo, en gij zult niet schuldig worden.
2Ch 19:11  En ziet, Amarja, de hoofdpriester, is over u in alle zaak des HEEREN; en Zebadja, de zoon van Ismael, de vorst van het huis van Juda, in alle zaak des konings; ook zijn de ambtlieden, de Levieten, voor uw aangezicht; weest sterk en doet het, en de HEERE zal met den goede zijn.
2Ch 20:1  Het geschiedde nu na dezen, dat de kinderen Moabs, en de kinderen Ammons, en het hen anderen benevens de Ammonieten, kwamen tegen Josafat ten strijde.
2Ch 20:2  Toen kwamen er, die Josafat boodschapten, zeggende: Daar komt een grote menigte tegen u van gene zijde der zee, uit Syrie; en zie, zij zijn te Hazezon-thamar, hetwelk is Engedi.
2Ch 20:3  Josafat nu vreesde, en stelde zijn aangezicht, om den HEERE te zoeken; en hij riep een vasten uit in gans Juda.
2Ch 20:4  En Juda werd vergaderd, om van den HEERE hulp te zoeken; ook kwamen zij uit alle steden van Juda, om den HEERE te zoeken.
2Ch 20:5  En Josafat stond in de gemeente van Juda en Jeruzalem, in het huis des HEEREN, voor het nieuwe voorhof.
2Ch 20:6  En hij zeide: O, HEERE, God onzer vaderen, zijt Gij niet de God in den hemel? Ja, Gij zijt de Heerser over alle koninkrijken der heidenen; en in Uw hand is kracht en sterkte, zodat niemand zich tegen U stellen kan.
2Ch 20:7  Hebt Gij niet, onze God, de inwoners dezes lands van voor het aangezicht van Uw volk Israel verdreven, en dat aan het zaad van Abraham, Uw liefhebber, tot in eeuwigheid gegeven?
2Ch 20:8  Zij nu hebben daarin gewoond, en zij hebben U daarin een heiligdom gebouwd voor Uw Naam, zeggende:
2Ch 20:9  Indien over ons enig kwaad komt, het zwaard des oordeels, of pestilentie, of honger, wij zullen voor dit huis, en voor Uw aangezicht staan, dewijl Uw Naam in dit huis is; en wij zullen uit onze benauwdheid tot U roepen, en Gij zult verhoren en verlossen.
2Ch 20:10  En nu, zie de kinderen Ammons, en Moab, en die van het gebergte Seir, door dewelken Gij Israel niet toeliet te trekken, als zij uit Egypteland togen, maar zij weken van hen, en verdelgden hen niet;
2Ch 20:11  Zie dan, zij vergelden het ons, komende om ons uit Uw erve, die Gij ons te erven gegeven hebt, te verdrijven.
2Ch 20:12  O, onze God, zult Gij geen recht tegen hen oefenen? want in ons is geen kracht tegen deze grote menigte, die tegen ons komt, en wij weten niet, wat wij doen zullen; maar onze ogen zijn op U.
2Ch 20:13  En gans Juda stond voor het aangezicht des HEEREN, ook hun kinderkens, hun vrouwen en hun zonen.
2Ch 20:14  Toen kwam de Geest des HEEREN in het midden der gemeente, op Jahaziel, den zoon van Zecharja, den zoon van Benaja, den zoon van Jehiel, den zoon van Matthanja, den Leviet, uit de zonen van Asaf;
2Ch 20:15  En hij zeide: Merkt op, geheel Juda, en gij, inwoners van Jeruzalem, en gij, koning Josafat! Alzo zegt de HEERE tot ulieden: Vreest gijlieden niet, en wordt niet ontzet vanwege deze grote menigte; want de strijd is niet uwe, maar Gods.
2Ch 20:16  Trekt morgen tot hen af; ziet, zij komen op bij den opgang van Ziz; en gij zult hen vinden in het einde des dals, voor aan de woestijn van Jeruel.
2Ch 20:17  Gij zult in dezen strijd niet te strijden hebben; stelt uzelven, staat en ziet het heil des HEEREN met u, o Juda en Jeruzalem! Vreest niet, en ontzet u niet, gaat morgen uit, hun tegen, want de HEERE zal met u wezen.
2Ch 20:18  Toen neigde zich Josafat met het aangezicht ter aarde; en gans Juda en de inwoners van Jeruzalem vielen neder voor het aangezicht des HEEREN, aanbiddende den HEERE.
2Ch 20:19  En de Levieten uit de kinderen der Kahathieten, en uit de kinderen der Korahieten, stonden op, om den HEERE, den God Israels, met luider stem ten hoogste te prijzen.
2Ch 20:20  En zij maakten zich des morgens vroeg op, en togen uit naar de woestijn van Thekoa; en als zij uittogen, stond Josafat en zeide: Hoort mij, o Juda, en gij, inwoners van Jeruzalem! Gelooft in den HEERE, uw God, zo zult gij bevestigd worden; gelooft aan Zijn profeten, en gij zult voorspoedig zijn.
2Ch 20:21  Hij nu beraadslaagde zich met het volk, en hij stelde den HEERE zangers, die de heilige Majesteit prijzen zouden, voor de toegerusten uitgaande en zeggende: Looft den HEERE, want Zijn goedertierenheid is tot in eeuwigheid!
2Ch 20:22  Ter tijd nu, als zij aanhieven met een vreugdegeroep en lofzang, stelde de HEERE achterlagen tegen de kinderen Ammons, Moab, en die van het gebergte Seir, die tegen Juda gekomen waren; en zij werden geslagen.
2Ch 20:23  Want de kinderen Ammons en Moab stonden op tegen de inwoners van het gebergte Seir, om te verbannen en te verdelgen; en als zij met de inwoners van Seir een einde gemaakt hadden, hielpen zij de een den ander ten verderve.
2Ch 20:24  Als nu Juda tot den wachttoren in de woestijn gekomen was, wendden zij zich naar de menigte; en ziet, het waren dode lichamen, liggende op de aarde, en niemand was ontkomen.
2Ch 20:25  Josafat nu en zijn volk kwamen, om hun buit te roven, en zij vonden bij hen in menigte, zowel have en dode lichamen, als kostelijk gereedschap, en namen voor zich weg, totdat zij niet meer dragen konden; en zij roofden den buit drie dagen, want dies was veel.
2Ch 20:26  En op den vierden dag vergaderden zij zich in het dal van Beracha, want daar loofden zij den HEERE; daarom noemden zij den naam dierzelver plaats het dal van Beracha, tot op dezen dag.
2Ch 20:27  Daarna keerden alle mannen van Juda en Jeruzalem weder, en Josafat in de voorspitse van hen, om wederom met blijdschap tot Jeruzalem te komen; want de HEERE had hen verblijd over hun vijanden.
2Ch 20:28  En zij kwamen te Jeruzalem, met luiten, en met harpen, en met trompetten, tot het huis des HEEREN.
2Ch 20:29  En er werd een verschrikking Gods over alle koninkrijken dier landen, als zij hoorden, dat de HEERE tegen de vijanden van Israel gestreden had.
2Ch 20:30  Alzo was het koninkrijk van Josafat stil; en zijn God gaf hem rust rondom henen.
2Ch 20:31  Zo regeerde Josafat over Juda; hij was vijf en dertig jaren oud, als hij koning werd, en hij regeerde vijf en twintig jaren te Jeruzalem; en de naam zijner moeder was Azuba, een dochter van Silhi.
2Ch 20:32  En hij wandelde in den weg van zijn vader Asa, en hij week daarvan niet af, doende dat recht was in de ogen des HEEREN.
2Ch 20:33  Evenwel werden de hoogten niet weggenomen; want het volk had nog zijn hart niet geschikt tot den God zijner vaderen.
2Ch 20:34  Het overige nu der geschiedenissen van Josafat, de eerste en de laatste, ziet, die zijn geschreven in de geschiedenissen van Jehu, den zoon van Hanani, die men hem optekenen deed in het boek der koningen van Israel.
2Ch 20:35  Doch na dezen vergezelschapte zich Josafat, de koning van Juda, met Ahazia, den koning van Israel; die handelde goddelooslijk in zijn doen.
2Ch 20:36  En hij vergezelschapte zich met hem, om schepen te maken, om naar Tharsis te gaan; en zij maakten de schepen te Ezeon-geber.
2Ch 20:37  Maar Eliezer, de zoon van Dodava, van Maresa, profeteerde tegen Josafat, zeggende: Omdat gij u met Ahazia vergezelschapt hebt, heeft de HEERE uw werken verscheurd. Alzo werden de schepen verbroken, dat zij niet konden naar Tharsis gaan.
2Ch 21:1  Daarna ontsliep Josafat met zijn vaderen, en werd begraven bij zijn vaderen in de stad Davids; en zijn zoon Joram werd koning in zijn plaats.
2Ch 21:2  En hij had broederen, Josafats zonen, Azarja, en Jehiel, en Zecharja, en Azarjahu, en Michael, en Sefatja; deze allen waren zonen van Josafat, den koning van Israel.
2Ch 21:3  En hun vader had hun vele gaven gegeven van zilver, en van goud, en van kostelijkheden, met vaste steden in Juda; maar het koninkrijk gaf hij Joram, omdat hij de eerstgeborene was.
2Ch 21:4  Als Joram tot het koninkrijk zijns vaders opgekomen was, en zich versterkt had, zo doodde hij al zijn broederen met het zwaard, mitsgaders ook enige van de vorsten van Israel.
2Ch 21:5  Twee en dertig jaar was Joram oud, toen hij koning werd, en hij regeerde acht jaren te Jeruzalem.
2Ch 21:6  En hij wandelde in den weg der koningen van Israel, gelijk als het huis van Achab deed; want hij had de dochter van Achab tot een vrouw; en hij deed dat kwaad was in de ogen des HEEREN.
2Ch 21:7  Doch de HEERE wilde het huis Davids niet verderven, om des verbonds wil, dat Hij met David gemaakt had; en gelijk als Hij gezegd had, hem en zijn zonen te allen dage een lamp te zullen geven.
2Ch 21:8  In zijn dagen vielen de Edomieten af van onder het gebied van Juda, en zij maakten over zich een koning.
2Ch 21:9  Daarom toog Joram voort met zijn oversten, en al de wagenen met hem; en hij maakte zich des nachts op, en sloeg de Edomieten, die rondom hem waren, en de oversten der wagenen.
2Ch 21:10  Evenwel vielen de Edomieten af van onder het gebied van Juda, tot op dezen dag; toen ter zelfder tijd viel Libna af, van onder zijn gebied, want hij had den HEERE, den God zijner vaderen, verlaten.
2Ch 21:11  Ook maakte hij hoogten op de bergen van Juda; en hij deed de inwoners van Jeruzalem hoereren, ja, hij dreef Juda daartoe.
2Ch 21:12  Zo kwam een schrift tot hem van den profeet Elia, zeggende: Alzo zegt de HEERE, de God van uw vader David: Omdat gij in de wegen van uw vader Josafat, en in de wegen van Asa, den koning van Juda, niet gewandeld hebt;
2Ch 21:13  Maar hebt gewandeld in den weg der koningen van Israel, en hebt Juda en de inwoners van Jeruzalem doen hoereren, achtervolgens het hoereren van het huis van Achab; en ook uw broederen, van uws vaders huis, gedood hebt, die beter waren dan gij;
2Ch 21:14  Zie, de HEERE zal u plagen met een grote plage aan uw volk, en aan uw kinderen, en aan uw vrouwen, en aan al uw have.
2Ch 21:15  Gij zult ook in grote krankheden zijn, door de krankheid uwer ingewanden, totdat uw ingewanden uitgaan vanwege de krankheid, jaar op jaar.
2Ch 21:16  Zo verwekte de HEERE tegen Joram den geest der Filistijnen en der Arabieren, die aan de zijde der Moren zijn.
2Ch 21:17  Die togen op in Juda, en braken daarin, en voerden alle have weg, die in het huis des konings gevonden werd, zelfs ook zijn kinderen, en zijn vrouwen; zodat hem geen zoon overgelaten werd, dan Joahaz, de kleinste zijner zonen.
2Ch 21:18  En na dit alles plaagde hem de HEERE in zijn ingewand met een krankheid, daar geen genezen aan was.
2Ch 21:19  Dit geschiedde van jaar tot jaar, zodat, wanneer de tijd van het einde der twee jaren uitging, zijn ingewanden met de krankheid uitgingen, dat hij stierf van boze krankheden; en zijn volk maakte hem gene branding, als de branding zijner vaderen.
2Ch 21:20  Hij was twee en dertig jaren oud, als hij koning werd, en regeerde acht jaren te Jeruzalem; en hij ging henen zonder begeerd te zijn; en zij begroeven hem in de stad Davids, maar niet in de graven der koningen.
2Ch 22:1  En de inwoners van Jeruzalem maakten Ahazia, zijn kleinsten zoon, koning in zijn plaats; want een bende, die met de Arabieren in het leger gekomen was, had al de eersten gedood. Ahazia dan, de zoon van Joram, de koning van Juda, regeerde.
2Ch 22:2  Twee en veertig jaar was Ahazia oud, toen hij koning werd, en hij regeerde een jaar te Jeruzalem; en de naam zijner moeder was Athalia, een dochter van Omri.
2Ch 22:3  Hij wandelde ook in de wegen van het huis van Achab; want zijn moeder was zijn raadgeefster, om goddelooslijk te handelen.
2Ch 22:4  En hij deed dat kwaad was in de ogen des HEEREN, gelijk het huis van Achab; want zij waren zijn raadgevers, na den dood zijns vaders, hem ten verderve.
2Ch 22:5  Hij wandelde ook in hun raad, en toog henen met Joram, den zoon van Achab, den koning van Israel, tot den strijd tegen Hazael, den koning van Syrie, bij Ramoth in Gilead; en de Syriers sloegen Joram.
2Ch 22:6  En hij keerde weder om zich te laten genezen te Jizreel; want hij had wonden, die men hem bij Rama geslagen had, als hij streed tegen Hazael, den koning van Syrie; en Azarja, de zoon van Joram, den koning van Juda, kwam af, om Joram, den zoon van Achab, te Jizreel te bezien, want hij was krank.
2Ch 22:7  De vertreding nu van Ahazia was van God, dat hij tot Joram kwam; want als hij gekomen was, toog hij met Joram uit tot Jehu, den zoon van Nimsi, denwelken de HEERE gezalfd had, om het huis van Achab uit te roeien.
2Ch 22:8  Zo geschiedde het, als Jehu het oordeel uitvoerde tegen het huis van Achab, dat hij de vorsten van Juda en de zonen der broederen van Ahazia, die Ahazia dienden, vond, en die doodde.
2Ch 22:9  Daarna zocht hij Ahazia, en zij kregen hem (want hij was verstoken in Samaria), en zij brachten hem tot Jehu, en zij doodden hem, en begroeven hem; want zij zeiden: Hij is de zoon van Josafat, die den HEERE met zijn ganse hart gezocht heeft. Zo had het huis van Ahazia niemand, die kracht behield tot het koninkrijk.
2Ch 22:10  Toen Athalia, de moeder van Ahazia, zag, dat haar zoon dood was, zo maakte zij zich op, en bracht al het koninklijke zaad van het huis van Juda om.
2Ch 22:11  Maar Jozabath, de dochter des konings, nam Joas, den zoon van Ahazia, en stal hem uit het midden van des konings zonen, die gedood werden, en zette hem en zijn voedster in een slaapkamer; zo verborg hem Jozabath, de dochter van den koning Joram, de huisvrouw van den priester Jojada (want zij was de zuster van Ahazia), voor Athalia, dat zij hem niet doodde.
2Ch 22:12  En hij was bij hen verstoken in het huis Gods zes jaren; en Athalia regeerde over het land.
2Ch 23:1  Doch in het zevende jaar versterkte zich Jojada, en nam de oversten der honderden, Azarja, den zoon van Jeroham en Ismael, den zoon van Johanan, en Azarja, den zoon van Obed, en Maaseja, den zoon van Adaja, en Elisafat, den zoon van Zichri, met zich in een verbond.
2Ch 23:2  Die togen om in Juda, en vergaderden de Levieten uit alle steden van Juda, en de hoofden der vaderen van Israel, en zij kwamen naar Jeruzalem.
2Ch 23:3  En die ganse gemeente maakte een verbond in het huis Gods, met den koning; en hij zeide tot hen: Ziet, de zoon des konings zal koning zijn, gelijk als de HEERE van de zonen van David gesproken heeft.
2Ch 23:4  Dit is de zaak, die gij doen zult: een derde deel van u, die op den sabbat ingaan, van de priesteren en van de Levieten, zullen tot poortiers der dorpelen zijn;
2Ch 23:5  En een derde deel zal zijn aan het huis des konings; en een derde deel aan de Fondamentpoort; en al het volk zal in de voorhoven zijn van het huis des HEEREN.
2Ch 23:6  Maar dat niemand kome in het huis des HEEREN, dan de priesteren en de Levieten, die dienen; die zullen ingaan, want zij zijn heilig; maar al het volk zal de wacht des HEEREN waarnemen.
2Ch 23:7  De Levieten nu zullen de koning rondom omsingelen, een ieder met zijn wapenen in zijn hand; en die tot het huis inkomt, zal gedood worden; doch weest gijlieden bij den koning, als hij inkomt en uitgaat.
2Ch 23:8  En de Levieten en gans Juda deden naar alles, wat de priester Jojada geboden had; en zij namen een ieder zijn mannen, die op den sabbat inkwamen, met degenen, die op den sabbat uitgingen; want de priester Jojada had aan de verdelingen geen verlof gegeven.
2Ch 23:9  Verder gaf de priester Jojada aan de oversten der honderden de spiesen, en de rondassen, en de schilden, die van den koning David geweest waren, die in het huis Gods waren.
2Ch 23:10  En hij stelde al het volk, en een ieder met zijn geweer in zijn hand, van de rechterzijde van het huis tot de linkerzijde van het huis, naar het altaar, en naar het huis, bij den koning rondom.
2Ch 23:11  Toen brachten zij des konings zoon voor, en zetten hem de kroon op, en gaven hem de getuigenis, en zij maakten hem koning; en Jojada en zijn zonen zalfden hem, en zeiden: De koning leve!
2Ch 23:12  Toen nu Athalia hoorde de stem des volks, dat toeliep en den koning roemde, kwam zij tot het volk in het huis des HEEREN.
2Ch 23:13  En zij zag toe; en ziet, de koning stond bij zijn pilaar, aan den ingang; en de oversten en de trompetten waren bij den koning; en al het volk des lands was blijde, en blies met de trompetten; en de zangers waren er met muzikale instrumenten, en gaven te kennen, dat men lofzingen zou; toen verscheurde Athalia haar klederen, en zij riep: Verraad, verraad!
2Ch 23:14  Maar de priester Jojada bracht de oversten der honderden, die over het heir gesteld waren, uit, en zeide tot hen: Brengt ze uit tot buiten de ordeningen, en die haar volgt, zal met het zwaard gedood worden; want de priester had gezegd: Gij zult ze in het huis des HEEREN niet doden.
2Ch 23:15  En zij leiden de handen aan haar, en zij ging naar den ingang van de Paardenpoort, naar het huis des konings; en zij doodden ze daar.
2Ch 23:16  En Jojada maakte een verbond tussen zich, en tussen al het volk, en tussen den koning, dat zij den HEERE tot een volk zouden zijn.
2Ch 23:17  Daarna ging al het volk in het huis van Baal, en braken dat af; en zijn altaren en zijn beelden verbraken zij, en Matthan, den priester van Baal, sloegen zij dood voor de altaren.
2Ch 23:18  Jojada nu bestelde de ambten in het huis des HEEREN, onder de hand der Levietische priesteren, die David in het huis des HEEREN afgedeeld had, om de brandofferen des HEEREN te offeren, gelijk in de wet van Mozes geschreven is, met blijdschap en met gezang, naar de instelling van David.
2Ch 23:19  En hij stelde de poortiers aan de poorten van het huis des HEEREN, opdat niemand, in enig ding onrein zijnde, inkwame.
2Ch 23:20  En hij nam de oversten der honderden, en de machtigen, en die heerschappij hadden onder het volk, en al het volk des lands, en bracht den koning van het huis des HEEREN af, en zij kwamen door het midden der hoge poort in het huis des konings; en zij zetten den koning op den troon des koninkrijks.
2Ch 23:21  En al het volk des lands was blijde, en de stad werd stil, nadat zij Athalia met het zwaard gedood hadden.
2Ch 24:1  Joas was zeven jaren oud, toen hij koning werd, en hij regeerde veertig jaren te Jeruzalem; en de naam zijner moeder was Zibja, van Ber-seba.
2Ch 24:2  En Joas deed dat recht was in de ogen des HEEREN, al de dagen van den priester Jojada.
2Ch 24:3  En Jojada nam voor hem twee vrouwen; en hij gewon zonen en dochteren.
2Ch 24:4  Het geschiedde nu na dezen, dat het in het hart van Joas was, het huis des HEEREN te vernieuwen.
2Ch 24:5  Zo vergaderde hij de priesteren en de Levieten, en zeide tot hen: Trekt uit tot de steden van Juda, en vergadert geld van het ganse Israel, om het huis uws Gods te beteren van jaar tot jaar; en gijlieden, haast tot deze zaak; maar de Levieten haastten niet.
2Ch 24:6  En de koning riep Jojada, het hoofd, en zeide tot hem: Waarom hebt gij geen onderzoek gedaan bij de Levieten, dat zij uit Juda en uit Jeruzalem inbrengen zouden de schatting van Mozes, den knecht des HEEREN, en van de gemeente van Israel, voor de tent der getuigenis?
2Ch 24:7  Want als Athalia goddelooslijk handelde, hadden haar zonen het huis Gods opengebroken, ja, zelfs alle geheiligde dingen van het huis des HEEREN besteed aan de Baals.
2Ch 24:8  En de koning gebood, en zij maakten een kist, en stelden die buiten aan de poort van het huis des HEEREN.
2Ch 24:9  En men deed uitroeping in Juda en in Jeruzalem, dat men den HEERE inbrengen zou de schatting van Mozes, den knecht Gods, over Israel in de woestijn.
2Ch 24:10  Toen verblijdden zich alle oversten en al het volk, en zij brachten in, en wierpen in de kist, totdat men voleind had.
2Ch 24:11  Het geschiedde nu ter tijd, als hij de kist, naar des konings bevel, door de hand der Levieten, inbracht, en als zij zagen, dat er veel gelds was, dat de schrijver des konings kwam, en de bestelde van den hoofdpriester, en de kist ledig maakten, en die opnamen, en die wederbrachten aan haar plaats; alzo deden zij van dag tot dag, en verzamelden geld in menigte;
2Ch 24:12  Hetwelk de koning en Jojada gaven aan degenen, die het werk van den dienst van het huis des HEEREN verzorgden; en zij huurden houwers en timmerlieden, om het huis des HEEREN te vernieuwen, mitsgaders ook werkmeesters in ijzer en koper, om het huis des HEEREN te beteren.
2Ch 24:13  Zo deden de verzorgers van het werk, dat de betering des werks door hun hand toenam; en zij herstelden het huis Gods in zijn gestaltenis, en maakten het vast.
2Ch 24:14  Als zij nu voleind hadden, brachten zij voor den koning en Jojada het overige des gelds, waarvan hij vaten maakte voor het huis des HEEREN, vaten om te dienen en te offeren, en rookschalen, en gouden en zilveren vaten; en zij offerden geduriglijk brandofferen in het huis des HEEREN al de dagen van Jojada.
2Ch 24:15  En Jojada werd oud en zat van dagen, en stierf; hij was honderd en dertig jaren oud, toen hij stierf.
2Ch 24:16  En zij begroeven hem in de stad Davids, bij de koningen; want hij had goed gedaan in Israel, beide aan God en zijn huize.
2Ch 24:17  Maar na den dood van Jojada kwamen de vorsten van Juda, en bogen zich neder voor den koning; toen hoorde de koning naar hen.
2Ch 24:18  Zo verlieten zij het huis des HEEREN, des Gods hunner vaderen, en dienden de bossen en de afgoden; toen was een grote toornigheid over Juda en Jeruzalem, om deze hun schuld.
2Ch 24:19  Doch Hij zond profeten onder hen, om hen tot den HEERE te doen wederkeren; die betuigden tegen hen, maar zij neigden de oren niet.
2Ch 24:20  En de Geest Gods toog Zacharia aan, den zoon van Jojada, den priester, die boven het volk stond, en hij zeide tot hen: Zo zegt God: Waarom overtreedt gij de geboden des HEEREN? Daarom zult gij niet voorspoedig zijn; dewijl gij den HEERE verlaten hebt, zo zal Hij u verlaten.
2Ch 24:21  En zij maakten een verbintenis tegen hem, en stenigden hem met stenen door het gebod des konings, in het voorhof van het huis des HEEREN.
2Ch 24:22  Zo gedacht de koning Joas niet der weldadigheid, die zijn vader Jojada aan hem gedaan had, maar doodde zijn zoon; dewelke, als hij stierf, zeide: De HEERE zal het zien en zoeken!
2Ch 24:23  Daarom geschiedde het met den omgang des jaars, dat de heirkracht van Syrie tegen hem optoog, en zij kwamen tot Juda en Jeruzalem, en verdierven uit het volk al de vorsten des volks; en zij zonden al hun roof tot den koning van Damaskus.
2Ch 24:24  Hoewel de heirkracht van Syrie met weinig mannen kwam, evenwel gaf de HEERE in hun hand een heirkracht van grote menigte, dewijl zij den HEERE, den God hunner vaderen, verlaten hadden; alzo voerden zij de oordelen uit tegen Joas.
2Ch 24:25  En toen zij van hem getogen waren (want zij lieten hem in grote krankheden), maakten zijn knechten, om het bloed der zonen van den priester Jojada, een verbintenis tegen hem, en zij sloegen hem dood op zijn bed, dat hij stierf; en zij begroeven hem in de stad Davids, maar zij begroeven hem niet in de graven der koningen.
2Ch 24:26  Dezen nu zijn, die een verbintenis tegen hem maakten: Zabad, de zoon van Simeath, de Ammonietische, en Jozabad, de zoon van Simrith, de Moabietische.
2Ch 24:27  Aangaande nu zijn zonen, en de grootheid van den last, hem opgelegd, en het gebouw van het huis Gods, ziet, zij zijn geschreven in de historie van het boek der koningen; en zijn zoon Amazia werd koning in zijn plaats.
2Ch 25:1  Amazia, vijf en twintig jaren oud zijnde, werd koning, en regeerde negen en twintig jaren te Jeruzalem; en de naam zijner moeder was Joaddan, van Jeruzalem.
2Ch 25:2  En hij deed dat recht was in de ogen des HEEREN, doch niet met een volkomen hart.
2Ch 25:3  Het geschiedde nu, als het koninkrijk aan hem gesterkt was, dat hij zijn knechten, die den koning, zijn vader, geslagen hadden, doodde.
2Ch 25:4  Doch hun kinderen doodde hij niet, maar hij deed, gelijk in de wet, in het boek van Mozes, geschreven is, waar de HEERE geboden heeft, zeggende: De vaders zullen niet sterven om de kinderen, en de kinderen zullen niet sterven om de vaders; maar een ieder zal om zijn zonde sterven.
2Ch 25:5  En Amazia vergaderde Juda, en stelde hen, naar de huizen der vaderen, tot oversten van duizenden en tot oversten van honderden, door gans Juda en Benjamin; en hij monsterde hen, van twintig jaren oud en daarboven, en vond hen driehonderd duizend uitgelezenen, uittrekkende ten heire, handelende spies en rondas.
2Ch 25:6  Daartoe huurde hij uit Israel honderd duizend kloeke helden, voor honderd talenten zilvers.
2Ch 25:7  Maar er kwam een man Gods tot hem, zeggende: O, koning! laat het heir van Israel met u niet gaan; want de HEERE is niet met Israel, met alle kinderen van Efraim.
2Ch 25:8  Maar zo gij gaat, doe het, wees sterk ten strijde; God zal u doen vallen voor den vijand; want in God is kracht, om te helpen en om te doen vallen.
2Ch 25:9  En Amazia zeide tot den man Gods: Maar wat zal men doen met de honderd talenten, die ik aan de benden van Israel gegeven heb? En de man Gods zeide: De HEERE heeft meer dan dit, om u te geven.
2Ch 25:10  Toen scheidde Amazia die af, te weten de benden, die uit Efraim tot hem gekomen waren, dat zij naar hun plaats gingen; daarom ontstak hun toorn zeer tegen Juda, en zij keerden weder tot hun plaats in hittigheid des toorns.
2Ch 25:11  Amazia nu sterkte zich, en leidde zijn volk uit, en toog in het Zoutdal, en sloeg van de kinderen van Seir tien duizend.
2Ch 25:12  Daartoe vingen de kinderen van Juda tien duizend levend, en brachten ze op de hoogte der steenrots, en stieten hen van de spits der steenrots af, dat zij allen barstten.
2Ch 25:13  Maar de mannen der benden, die Amazia had doen wederkeren, dat zij met hem in den strijd niet zouden trekken, die deden een inval in de steden van Juda, van Samaria af tot Beth-horon toe, en sloegen van hen drie duizend, en roofden veel roofs.
2Ch 25:14  Het geschiedde nu, nadat Amazia van het slaan der Edomieten gekomen was, en dat hij de goden der kinderen van Seir medegebracht had, dat hij die zich tot goden stelde, en zich voor dezelve neder boog en dien rookte.
2Ch 25:15  Toen ontstak de toorn des HEEREN tegen Amazia; en Hij zond tot hem een profeet, die zeide tot hem: Waarom hebt gij de goden van dat volk gezocht, die hun volk niet gered hebben uit uw hand?
2Ch 25:16  En het geschiedde, als hij tot hem sprak, dat hij hem zeide: Heeft men u tot des konings raadgever gesteld? Houd gij op; waarom zouden zij u slaan? Toen hield de profeet op, en zeide: Ik merk, dat God besloten heeft u te verderven, dewijl gij dit gedaan, en naar mijn raad niet gehoord hebt.
2Ch 25:17  En Amazia, de koning van Juda, werd te rade, dat hij zond tot Joas, den zoon van Joahaz, den zoon van Jehu, den koning van Israel, om te zeggen: Kom, laat ons elkanders aangezicht zien.
2Ch 25:18  Maar Joas, de koning van Israel, zond tot Amazia, den koning van Juda, om te zeggen: De distel, die op den Libanon is, zond tot den ceder, die op den Libanon is, om te zeggen: Geef uw dochter mijn zoon ter vrouw; maar het gedierte des velds, dat op den Libanon is, ging voorbij, en vertrad de distel.
2Ch 25:19  Gij zegt: Zie, gij hebt de Edomieten geslagen; daarom heeft uw hart u verheven, om te roemen; nu, blijf in uw huis; waarom zoudt gij u in het kwaad mengen, dat gij vallen zoudt; gij en Juda met u?
2Ch 25:20  Doch Amazia hoorde niet, want het was van God, opdat Hij hen in hun hand gave, overmits zij de goden der Edomieten gezocht hadden.
2Ch 25:21  Zo toog Joas, de koning van Israel, op, en hij en Amazia, de koning van Juda, zagen elkanders aangezichten te Beth-semes, dat in Juda is.
2Ch 25:22  En Juda werd geslagen voor het aangezicht van Israel; en zij vloden een iegelijk in zijn tenten.
2Ch 25:23  En Joas, de koning van Israel, greep Amazia, den koning van Juda, den zoon van Joas, den zoon van Joahaz, te Beth-semes; en hij bracht hem te Jeruzalem, en hij brak aan den muur van Jeruzalem, van de poort van Efraim tot aan de Hoekpoort, vierhonderd ellen.
2Ch 25:24  Daartoe nam hij al het goud, en het zilver, en al de vaten, die in het huis Gods gevonden werden, bij Obed-edom, en de schatten van het huis des konings, mitsgaders gijzelaars, en hij keerde weder naar Samaria.
2Ch 25:25  Amazia nu, de zoon van Joas, de koning van Juda, leefde na den dood van Joas, den zoon van Joahaz, den koning van Israel, vijftien jaren.
2Ch 25:26  Het overige nu der geschiedenissen van Amazia, de eerste en de laatste, ziet, zijn die niet geschreven in het boek der koningen van Juda en Israel?
2Ch 25:27  Van den tijd nu af, dat Amazia afgeweken was van achter den HEERE, zo maakten zij in Jeruzalem een verbintenis tegen hem; doch hij vluchtte naar Lachis. Toen zonden zij hem na tot Lachis, en doodden hem aldaar.
2Ch 25:28  En zij brachten hem op paarden, en begroeven hem bij zijn vaderen in de stad van Juda.
2Ch 26:1  Toen nam het ganse volk van Juda Uzzia (die nu zestien jaren oud was), en maakte hem koning in de plaats van zijn vader Amazia.
2Ch 26:2  Dezelve bouwde Eloth, en bracht ze weder aan Juda, nadat de koning met zijn vaderen ontslapen was.
2Ch 26:3  Zestien jaren was Uzzia oud, toen hij koning werd, en hij regeerde twee en vijftig jaren te Jeruzalem; en de naam zijner moeder was Jecholia, van Jeruzalem.
2Ch 26:4  En hij deed dat recht was in de ogen des HEEREN, naar alles, wat zijn vader Amazia gedaan had.
2Ch 26:5  Want hij begaf zich om God te zoeken, in de dagen van Zacharia, die verstandig was in de gezichten Gods; in de dagen nu, dat hij den HEERE zocht, maakte hem God voorspoedig.
2Ch 26:6  Want hij toog uit, en krijgde tegen de Filistijnen, en brak den muur van Gath, en den muur van Jabne, en den muur van Asdod; daartoe bouwde hij steden in Asdod, en onder de Filistijnen.
2Ch 26:7  En God hielp hem tegen de Filistijnen, en tegen de Arabieren, die te Gur-baal woonden, en tegen de Meunieten.
2Ch 26:8  En de Ammonieten gaven Uzzia geschenken; en zijn naam ging tot den ingang van Egypte, want hij sterkte zich ten hoogste.
2Ch 26:9  Daartoe bouwde Uzzia torens te Jeruzalem, aan de Hoekpoort en aan de Dalpoort, en aan de hoeken; en hij sterkte ze.
2Ch 26:10  Hij bouwde ook torens in de woestijn, en hieuw vele putten uit, overmits hij veel vee had, beide in de laagten en in de effene velden; akkerlieden en wijngaardeniers op de bergen en op de vruchtbare velden; want hij was een liefhebber van den land bouw.
2Ch 26:11  Verder had Uzzia een heirkracht van geoefenden ten oorlog, uittrekkende ten heire bij benden, naar het getal hunner monstering, daar de hand van Jeiel, den schrijver, en Mahaseja, den ambtman; onder de hand van Hananja, een van de vorsten des konings.
2Ch 26:12  Het gehele getal van de hoofden der vaderen, der strijdbare helden, was twee duizend en zeshonderd.
2Ch 26:13  En onder hun hand was een krijgsheir van driehonderd zeven duizend en vijfhonderd, die met strijdbare kracht zich ten oorlog oefenden, om den koning tegen den vijand te helpen.
2Ch 26:14  En Uzzia bereidde voor hen, voor het ganse heir, schilden, en spiesen, en helmen, en pantsieren, en bogen, zelfs tot de slingerstenen toe.
2Ch 26:15  Hij maakte ook te Jeruzalem kunstige werken, bedenking van kunstige werkmeesters, dat zij op de torens en op de hoeken zijn zouden, om met pijlen en met grote stenen, te schieten; zo ging zijn naam tot verre toe uit, want hij werd wonderlijk geholpen, totdat hij sterk was.
2Ch 26:16  Maar als hij sterk geworden was, verhief zich zijn hart tot verdervens toe, en hij overtrad tegen den HEERE, zijn God; want hij ging in den tempel des HEEREN, om te roken op het reukaltaar.
2Ch 26:17  Doch Azaria, de priester, ging hem na, en met hem des HEEREN priesters, tachtig kloeke mannen.
2Ch 26:18  En zij wederstonden den koning Uzzia, en zeiden tot hem: Het komt u niet toe, Uzzia, den HEERE te roken, maar den priesteren, Aarons zonen, die geheiligd zijn, om te roken; ga uit het heiligdom, want gij hebt overtreden, en het zal u niet tot eer zijn van den HEERE God.
2Ch 26:19  Toen werd Uzzia toornig, en het reukwerk was in zijn hand, om te roken; als hij nu toornig werd tegen de priesteren, rees de melaatsheid op aan zijn voorhoofd, voor het aangezicht der priesteren in het huis des HEEREN, van boven het reukaltaar.
2Ch 26:20  Alstoen zag de hoofdpriester Azaria op hem, en al de priesteren en ziet, hij was melaats aan zijn voorhoofd, en zij stieten hem met der haast van daar, ja hij zelf werd ook gedreven uit te gaan, omdat de HEERE hem geplaagd had.
2Ch 26:21  Alzo was de koning Uzzia melaats tot aan den dag zijns doods; en melaats zijnde, woonde hij in een afgezonderd huis, want hij was van het huis des HEEREN afgesneden; Jotham nu, zijn zoon, was over het huis des konings, richtende het volk des lands.
2Ch 26:22  Het overige nu der geschiedenissen van Uzzia, de eerste en de laatste, heeft de profeet Jesaja, de zoon van Amos, beschreven.
2Ch 26:23  En Uzzia ontsliep met zijn vaderen, en zij begroeven hem bij zijn vaderen, in het veld van de begrafenis, die van de koningen was; want zij zeiden: hij is melaats; en zijn zoon Jotham werd koning in zijn plaats.
2Ch 27:1  Jotham was vijf en twintig jaren oud, toen hij koning werd, en hij regeerde zestien jaren te Jeruzalem; en de naam zijner moeder was Jerusa, een dochter van Zadok.
2Ch 27:2  En hij deed dat recht was in de ogen des HEEREN, naar alles, wat zijn vader Uzzia gedaan had, behalve dat hij in den tempel des HEEREN niet ging; en het volk verdierf zich nog.
2Ch 27:3  Dezelve bouwde de hoge poorten aan het huis des HEEREN; hij bouwde ook veel aan den muur van Ofel.
2Ch 27:4  Daartoe bouwde hij steden op het gebergte van Juda; en in de wouden bouwde hij burchten en torens.
2Ch 27:5  Hij krijgde ook tegen den koning der kinderen Ammons, en had de overhand over hen, zodat de kinderen Ammons in datzelfde jaar hem gaven honderd talenten zilvers, en tien duizend kor tarwe, en tien duizend gerst; dit brachten hem de kinderen Ammons wederom, ook in het tweede en in het derde jaar.
2Ch 27:6  Alzo versterkte zich Jotham; want hij richtte zijn wegen voor het aangezicht des HEEREN, zijns Gods.
2Ch 27:7  Het overige nu der geschiedenissen van Jotham, en al zijn krijgen, en zijn wegen, ziet, zij zijn geschreven in het boek der koningen van Israel en Juda.
2Ch 27:8  Hij was vijf en twintig jaren oud, toen hij koning werd; en hij regeerde zestien jaren te Jeruzalem.
2Ch 27:9  En Jotham ontsliep met zijn vaderen, en zij begroeven hem in de stad Davids; en zijn zoon Achaz werd koning in zijn plaats.
2Ch 28:1  Achaz was twintig jaren oud, toen hij koning werd, en regeerde zestien jaren te Jeruzalem; en hij deed niet dat recht was in de ogen des HEEREN, gelijk zijn vader David;
2Ch 28:2  Maar hij wandelde in de wegen der koningen van Israel; daartoe maakte hij ook gegotene beelden voor de Baals.
2Ch 28:3  Dezelve rookte ook in het dal des zoons van Hinnom; en hij brandde zijn zonen in het vuur, naar de gruwelen der heidenen, die de HEERE voor het aangezicht der kinderen Israels uit de bezitting verdreven had.
2Ch 28:4  Ook offerde hij en rookte op de hoogten en op de heuvelen, mitsgaders onder alle groen geboomte.
2Ch 28:5  Daarom gaf hem de HEERE, zijn God, in de hand des konings van Syrie, dat zij hem sloegen, en van hem gevankelijk wegvoerden een grote menigte van gevangenen, die zij te Damaskus brachten. En hij werd ook gegeven in de hand des konings van Israel, die hem sloeg met een groten slag.
2Ch 28:6  Want Pekah, de zoon van Remalia, sloeg in Juda honderd en twintig duizend dood op een dag, allen strijdbare mannen, omdat zij den HEERE, den God hunner vaderen, verlaten hadden.
2Ch 28:7  En Zichri, een geweldig man van Efraim, sloeg Maaseja, den zoon des konings, dood, en Azrikam, den huisoverste, mitsgaders Elkana, den tweede na den koning.
2Ch 28:8  En de kinderen Israels voerden van hun broederen gevankelijk weg tweehonderd duizend, vrouwen, zonen en dochteren, en plunderden ook veel roofs van hen; en zij brachten den roof te Samaria.
2Ch 28:9  Aldaar nu was een profeet des HEEREN, wiens naam was Oded; die ging uit, het heir tegen, dat naar Samaria kwam, en zeide tot hen: Ziet, door de grimmigheid des HEEREN, des Gods uwer vaderen, over Juda, heeft Hij hen in uw hand gegeven, en gij hebt hen doodgeslagen in toornigheid, die tot aan den hemel raakt.
2Ch 28:10  Daartoe denkt gij nu de kinderen van Juda en Jeruzalem u tot slaven en slavinnen te onderwerpen; zijt gij het niet alleenlijk? Bij ulieden zijn schulden tegen den HEERE, uw God.
2Ch 28:11  Nu dan, hoort mij, en brengt de gevangenen weder, die gij van uw broederen gevankelijk weggevoerd hebt; want de hitte van des HEEREN toorn is over u.
2Ch 28:12  Toen maakten zich mannen op van de hoofden der kinderen van Efraim, Azaria, de zoon van Johanan, Berechja, de zoon van Mesillemoth en Jehizkia, de zoon van Sallum, en Amasa, de zoon van Hadlai, tegen degenen, die uit het heir kwamen.
2Ch 28:13  En zij zeiden tot hen: Gij zult deze gevangenen hier niet inbrengen, tot een schuld over ons tegen den HEERE; denkt gijlieden toe te doen tot onze zonden en tot onze schulden, hoewel wij vele schulden hebben, en de hitte des toorns over Israel is?
2Ch 28:14  Toen lieten de toegerusten de gevangenen en den roof voor het aangezicht der oversten en der ganse gemeente.
2Ch 28:15  De mannen nu, die met namen uitgedrukt zijn, maakten zich op, en grepen de gevangenen, en kleedden van den roof al hun naakten; en zij kleedden hen, en schoeiden hen, en spijsden hen, en drenkten hen, en zalfden hen, en voerden ze op ezelen, allen die zwak waren, en brachten hen te Jericho, de Palmstad, bij hun broederen; daarna keerden zij weder naar Samaria.
2Ch 28:16  Ter zelfder tijd zond de koning Achaz tot de koningen van Assyrie, dat zij hem helpen zouden.
2Ch 28:17  Daarenboven waren ook de Edomieten gekomen, en hadden Juda geslagen en gevangenen gevankelijk weggevoerd.
2Ch 28:18  Daartoe waren de Filistijnen in de steden der laagte en het zuiden van Juda ingevallen, en hadden ingenomen Beth-semes, en Ajalon, en Gederoth, en Socho en haar onderhorige plaatsen, en Timna en haar onderhorige plaatsen, en Gimzo en haar onderhorige plaatsen; en zij woonden aldaar.
2Ch 28:19  Want de HEERE vernederde Juda, om der wille van Achaz, den koning Israels; want hij had Juda afgetrokken, dat het gans zeer overtrad tegen den HEERE.
2Ch 28:20  En Tiglath-pilneser, de koning van Assyrie, kwam tot hem; doch hij benauwde hem, en sterkte hem niet.
2Ch 28:21  Want Achaz nam een deel van het huis des HEEREN, en van het huis des konings en der vorsten, hetwelk hij den koning van Assyrie gaf; maar hij hielp hem niet.
2Ch 28:22  Ja, ter tijd, als men hem benauwde, zo maakte hij des overtredens tegen den HEERE nog meer; dit was de koning Achaz.
2Ch 28:23  Want hij offerde den goden van Damaskus, die hem geslagen hadden, en zeide: Omdat de goden der koningen van Syrie hen helpen, zal ik hun offeren, opdat zij mij ook helpen; maar zij waren hem tot zijn val, mitsgaders aan gans Israel.
2Ch 28:24  En Achaz verzamelde de vaten van het huis Gods, en hieuw de vaten van het huis Gods in stukken, en sloot de deuren van het huis des HEEREN toe; daartoe maakte hij zich altaren in alle hoeken te Jeruzalem.
2Ch 28:25  Ook maakte hij in elke stad van Juda hoogten, om anderen goden te roken; alzo verwekte hij den HEERE, zijner vaderen God, tot toorn.
2Ch 28:26  Het overige nu der geschiedenissen, en al zijn wegen, de eerste en de laatste, ziet, zij zijn geschreven in het boek der koningen van Juda en Israel.
2Ch 28:27  En Achaz ontsliep met zijn vaderen, en zij begroeven hem in de stad te Jeruzalem; maar zij brachten hem niet in de graven der koningen van Israel; en zijn zoon Jehizkia werd koning in zijn plaats.
2Ch 29:1  Jehizkia werd koning, vijf en twintig jaren oud zijnde, en regeerde negen en twintig jaren te Jeruzalem; en de naam zijner moeder was Abia, een dochter van Zacharia.
2Ch 29:2  En hij deed dat recht was in de ogen des HEEREN, naar alles, wat zijn vader David gedaan had.
2Ch 29:3  Dezelve deed in het eerste jaar zijner regering, in de eerste maand, de deuren van het huis des HEEREN open, en beterde ze.
2Ch 29:4  En hij bracht de priesteren en de Levieten in, en hij verzamelde ze in de Ooststraat.
2Ch 29:5  En hij zeide tot hen: Hoort mij, o Levieten; heiligt nu uzelven, en heiligt het huis des HEEREN, des Gods uwer vaderen, en brengt de onreinigheid uit van het heiligdom.
2Ch 29:6  Want onze vaders hebben overtreden, en gedaan dat kwaad was in de ogen des HEEREN, onzes Gods, en hebben Hem verlaten, en zij hebben hun aangezichten van den tabernakel des HEEREN omgewend, en hebben den nek toegekeerd.
2Ch 29:7  Ook hebben zij de deuren van het voorhuis toegesloten, en de lampen uitgeblust en het reukwerk niet gerookt; en het brandoffer hebben zij in het heiligdom aan den God Israels niet geofferd.
2Ch 29:8  Daarom is een grote toorn des HEEREN over Juda en Jeruzalem geweest; en Hij heeft hen overgegeven ter beroering, ter verwoesting en ter aanfluiting, gelijk als gij ziet met uw ogen.
2Ch 29:9  Want ziet, onze vaders zijn door het zwaard gevallen; daartoe onze zonen, en onze dochteren, en onze vrouwen zijn daarom in gevangenis geweest.
2Ch 29:10  Nu is het in mijn hart een verbond te maken met den HEERE, den God Israels, opdat de hitte Zijns toorns van ons afkere.
2Ch 29:11  Mijn zonen, weest nu niet traag; want de HEERE heeft u verkoren, dat gij voor Zijn aangezicht staan zoudt, om Hem te dienen; en opdat gij Hem dienaars en wierokers zoudt wezen.
2Ch 29:12  Toen maakten zich de Levieten op, Mahath, de zoon van Amasai, en Joel, de zoon van Azarja, van de kinderen der Kahathieten; en van de kinderen van Merari, Kis, de zoon van Abdi, en Azarja, de zoon van Jehaleel; en van de Gersonieten, Joah, de zoon van Zimma, en Eden, de zoon van Joah;
2Ch 29:13  En van de kinderen van Elizafan, Simri en Jeiel; en van de kinderen van Asaf, Zecharja en Mattanja;
2Ch 29:14  En van de kinderen van Heman, Jehiel en Simei; en van de kinderen van Jeduthun, Semaja en Uzziel.
2Ch 29:15  En zij verzamelden hun broederen, en heiligden zich, en kwamen, naar het gebod des konings, door de woorden des HEEREN, om het huis des HEEREN te reinigen.
2Ch 29:16  Maar de priesteren gingen binnen in het huis des HEEREN, om dat te reinigen, en zij brachten uit in het voorhof van het huis des HEEREN al de onreinigheid, die zij in den tempel des HEEREN vonden; en de Levieten namen ze op, om naar buiten uit te brengen, in de beek Kidron.
2Ch 29:17  Zij begonnen nu te heiligen op den eersten der eerste maand, en op den achtsten dag der maand kwamen zij in het voorhuis des HEEREN, en heiligden het huis des HEEREN in acht dagen; en op den zestienden dag der eerste maand maakten zij een einde.
2Ch 29:18  Daarna kwamen zij binnen tot den koning Hizkia, en zeiden: Wij hebben het gehele huis des HEEREN gereinigd, mitsgaders het brandofferaltaar met al zijn gereedschap, en de tafel der toerichting met al haar gereedschap.
2Ch 29:19  Alle gereedschap ook, dat de koning Achaz, onder zijn koninkrijk, door zijn overtreding weggeworpen had, hebben wij bereid en geheiligd; en zie, zij zijn voor het altaar des HEEREN.
2Ch 29:20  Toen maakte zich de koning Jehizkia vroeg op, en verzamelde de oversten der stad, en hij ging op in het huis des HEEREN.
2Ch 29:21  En zij brachten zeven varren, en zeven rammen, en zeven lammeren, en zeven geitenbokken ten zondoffer voor het koninkrijk, en voor het heiligdom, en voor Juda; en hij zeide tot de zonen van Aaron, de priesteren, dat zij die op het altaar des HEEREN zouden offeren.
2Ch 29:22  Zo slachtten zij de runderen, en de priesters ontvingen het bloed, en sprengden het op het altaar; zij slachtten ook de rammen, en sprengden het bloed op het altaar; insgelijks slachtten zij de lammeren, en sprengden het bloed op het altaar.
2Ch 29:23  Daarna brachten zij de bokken bij, ten zondoffer, voor het aangezicht des konings en der gemeente, en zij leiden hun handen op dezelve.
2Ch 29:24  En de priesteren slachtten ze, en ontzondigden met derzelver bloed op het altaar, om verzoening te doen voor het ganse Israel; want de koning had dat brandoffer en dat zondoffer voor gans Israel bevolen.
2Ch 29:25  En hij stelde de Levieten in het huis des HEEREN, met cimbalen, met luiten en harpen, naar het gebod van David, en van Gad, den ziener des konings, en van Nathan, den profeet; want dit gebod was van de hand des HEEREN, door de hand Zijner profeten.
2Ch 29:26  De Levieten nu stonden met de instrumenten van David, en de priesters met de trompetten.
2Ch 29:27  En Hizkia beval, dat men het brandoffer op het altaar zou offeren; ten tijde nu, als dat brandoffer begon, begon het gezang des HEEREN met de trompetten en met de instrumenten van David, den koning van Israel.
2Ch 29:28  De ganse gemeente nu boog zich neder, als men het gezang zong, en met trompetten trompette; dit alles totdat het brandoffer voleind was.
2Ch 29:29  Als men nu geeindigd had te offeren, bukten de koning en allen, die bij hem gevonden waren, en bogen zich neder.
2Ch 29:30  Daarna zeide de koning Jehizkia, en de oversten, tot de Levieten, dat zij den HEERE loven zouden, met de woorden van David en van Asaf, den ziener; en zij loofden tot blijdschap toe; en neigden hun hoofden, en bogen zich neder.
2Ch 29:31  En Jehizkia antwoordde en zeide: Nu hebt gij uw handen den HEERE gevuld, treedt toe, en brengt slachtofferen en lofofferen tot het huis des HEEREN; en de gemeente bracht slachtofferen en lofofferen en alle vrijwilligen van harte brandofferen.
2Ch 29:32  En het getal der brandofferen, die de gemeente bracht, was zeventig runderen, honderd rammen, tweehonderd lammeren; deze alle den HEERE ten brandoffer.
2Ch 29:33  Nog waren der geheiligde dingen zeshonderd runderen en drie duizend schapen.
2Ch 29:34  Doch van de priesteren waren er te weinig, en zij konden al den brandofferen de huid niet aftrekken; daarom hielpen hen hun broederen, de Levieten, totdat het werk geeindigd was, en totdat de andere priesters zich geheiligd hadden; want de Levieten waren rechter van hart, om zich te heiligen, dan de priesteren.
2Ch 29:35  En ook waren de brandofferen in menigte, met het vet der dankofferen, en met de drankofferen, voor de brandofferen; alzo werd de dienst van het huis des HEEREN besteld.
2Ch 29:36  Jehizkia nu en al het volk verblijdden zich over hetgeen God het volk voorbereid had; want deze zaak geschiedde haastelijk.
2Ch 30:1  Daarna zond Jehizkia tot het ganse Israel en Juda, en schreef ook brieven tot Efraim en Manasse, dat zij zouden komen tot het huis des HEEREN te Jeruzalem, om den HEERE, den God Israels, pascha te houden.
2Ch 30:2  Want de koning had raad gehouden met zijn oversten en de ganse gemeente te Jeruzalem, om het pascha te houden, in de tweede maand.
2Ch 30:3  Want zij hadden het niet kunnen houden te dierzelfder tijd, omdat de priesteren zich niet genoeg geheiligd hadden, en het volk zich niet verzameld had te Jeruzalem.
2Ch 30:4  En deze zaak was recht in de ogen des konings, en in de ogen der ganse gemeente.
2Ch 30:5  Zo stelden zij zulks, dat men een stem door gans Israel, van Ber-seba tot Dan, zou laten doorgaan, opdat zij zouden komen om het pascha den HEERE, den God Israels, te houden in Jeruzalem; want zij hadden het in lang niet gehouden, gelijk het geschreven was.
2Ch 30:6  De lopers dan gingen henen met de brieven van de hand des konings en zijner vorsten, door gans Israel en Juda, en naar het gebod des konings, zeggende: Gij, kinderen Israels, bekeert u tot den HEERE, den God van Abraham, Izak en Israel, zo zal Hij Zich keren tot de ontkomenen, die ulieden overgebleven zijn uit de hand der koningen van Assyrie.
2Ch 30:7  En zijt niet als uw vaders en als uw broeders, die tegen den HEERE, den God hunner vaderen, overtreden hebben; waarom Hij hen tot verwoesting overgegeven heeft, gelijk als gij ziet.
2Ch 30:8  Verhardt nu ulieder nek niet, gelijk uw vaderen; geeft den HEERE de hand, en komt tot Zijn heiligdom, hetwelk Hij geheiligd heeft tot in eeuwigheid, en dient den HEERE, uw God; zo zal de hitte Zijns toorns van u afkeren.
2Ch 30:9  Want als gij u bekeert tot den HEERE, zullen uw broederen en uw kinderen barmhartigheid vinden voor het aangezicht dergenen, die hen gevangen hebben, zodat zij in dit land zullen wederkomen; want de HEERE, uw God, is genadig en barmhartig, en zal het aangezicht van u niet afwenden, zo gij u tot Hem bekeert.
2Ch 30:10  Zo gingen de lopers door, van stad tot stad, door het land van Efraim en Manasse, tot Zebulon toe; doch zij belachten hen, en bespotten hen.
2Ch 30:11  Evenwel verootmoedigden zich sommigen van Aser, en Manasse, en van Zebulon, en kwamen te Jeruzalem.
2Ch 30:12  Ook was de hand Gods in Juda, hun enerlei hart gevende, dat zij het gebod des konings en der vorsten deden, naar het woord des HEEREN.
2Ch 30:13  En te Jeruzalem verzamelde zich veel volks, om het feest der ongezuurde broden te houden, in de tweede maand, een zeer grote gemeente.
2Ch 30:14  En zij maakten zich op, en namen de altaren weg, die te Jeruzalem waren; daartoe namen zij alle rooktuig weg, hetwelk zij in de beek Kidron wierpen.
2Ch 30:15  Toen slachtten zij het pascha, op den veertienden der tweede maand; en de priesters en de Levieten waren beschaamd geworden, en hadden zich geheiligd, en hadden brandofferen gebracht in het huis des HEEREN.
2Ch 30:16  En zij stonden in hun stand, naar hun wijze, naar de wet van Mozes, den man Gods; de priesters sprengden het bloed, dat nemende uit de hand der Levieten.
2Ch 30:17  Want een menigte was in die gemeente, die zich niet geheiligd hadden; daarom waren de Levieten over de slachting der paaslammeren, voor iedereen, die niet rein was, om die den HEERE te heiligen.
2Ch 30:18  Want een menigte des volks, velen van Efraim en Manasse, Issaschar en Zebulon, hadden zich niet gereinigd, maar aten het pascha, niet gelijk geschreven is. Doch Jehizkia bad voor hen, zeggende: De HEERE, die goed is, make verzoening voor dien.
2Ch 30:19  Die zijn ganse hart gericht heeft, om God den HEERE, den God zijner vaderen, te zoeken, hoewel niet naar de reinigheid des heiligdoms.
2Ch 30:20  En de HEERE verhoorde Jehizkia, en heelde het volk.
2Ch 30:21  Zo hielden de kinderen Israels, die te Jeruzalem gevonden werden, het feest der ongezuurde broden, zeven dagen, met grote blijdschap. De Levieten nu en de priesteren prezen den HEERE, dag op dag, met sterk luidende instrumenten des HEEREN.
2Ch 30:22  En Jehizkia sprak naar het hart van alle Levieten, die verstand hadden in de goede kennis des HEEREN; en zij aten de offeranden des gezetten hoogtijds zeven dagen, offerende dankofferen, en lovende den HEERE, den God hunner vaderen.
2Ch 30:23  Als nu de ganse gemeente raad gehouden had, om andere zeven dagen te houden, hielden zij nog zeven dagen met blijdschap.
2Ch 30:24  Want Jehizkia, de koning van Juda, gaf de gemeente duizend varren en zeven duizend schapen; en de vorsten gaven de gemeente duizend varren en tien duizend schapen; de priesteren nu hadden zich in menigte geheiligd.
2Ch 30:25  En de ganse gemeente van Juda verblijdde zich, mitsgaders de priesteren en de Levieten, en de gehele gemeente dergenen, die uit Israel gekomen waren; ook de vreemdelingen, die uit het land van Israel gekomen waren, en die in Juda woonden.
2Ch 30:26  Zo was er grote blijdschap te Jeruzalem; want van de dagen van Salomo, den zoon van David, den koning van Israel, was desgelijks in Jeruzalem niet geweest.
2Ch 30:27  Toen stonden de Levietische priesteren op, en zegenden het volk; en hun stem werd gehoord; want hun gebed kwam tot Zijn heilige woning in den hemel.
2Ch 31:1  Als zij nu dit alles voleind hadden, togen alle Israelieten, die er gevonden werden, uit, tot de steden van Juda, en braken de opgerichte beelden, en hieuwen de bossen af, en wierpen de hoogten en de altaren af, uit gans Juda en Benjamin, ook in Efraim en Manasse, totdat zij alles te niet gemaakt hadden; daarna keerden al de kinderen Israels weder, een ieder tot zijn bezitting in hun steden.
2Ch 31:2  En Hizkia bestelde de verdelingen der priesteren en der Levieten, naar hun verdelingen, een ieder naar zijn dienst, de priesteren en de Levieten tot het brandoffer en tot de dankofferen, om te dienen, en om te loven, en om te prijzen in de poort van de legers des HEEREN;
2Ch 31:3  Ook het deel des konings van zijn have tot de brandofferen, tot de brandofferen des morgens en des avonds, en de brandofferen der sabbatten, en der nieuwe maanden, en der gezette hoogtijden; gelijk geschreven is in de wet des HEEREN.
2Ch 31:4  En hij zeide tot het volk, tot de inwoners van Jeruzalem, dat zij het deel der priesteren en Levieten geven zouden, opdat zij versterkt mochten worden in de wet des HEEREN.
2Ch 31:5  Toen nu dat woord uitbrak, brachten de kinderen Israels vele eerstelingen van koren, most, en olie, en honig, en van al de inkomsten des velds; ook brachten zij de tienden van alles in met menigte.
2Ch 31:6  En de kinderen van Israel en Juda, die in de steden van Juda woonden, brachten ook tienden der runderen en der schapen, en tienden der heilige dingen, die den HEERE, hun God, geheiligd waren, en maakten vele hopen.
2Ch 31:7  In de derde maand begonnen zij den grond van die hopen te leggen, en in de zevende maand voleindden zij.
2Ch 31:8  Toen nu Jehizkia en de vorsten kwamen en die hopen zagen, zegenden zij den HEERE en Zijn volk Israel.
2Ch 31:9  En Jehizkia ondervraagde de priesteren en de Levieten aangaande die hopen.
2Ch 31:10  En Azaria, de hoofdpriester, van het huis van Zadok, sprak tot hem en zeide: Van dat men deze heffing begonnen heeft tot het huis des HEEREN te brengen, is er te eten geweest en verzadigd te worden, ja, over te houden tot overvloed toe; want de HEERE heeft Zijn volk gezegend, zodat deze veelheid overgebleven is.
2Ch 31:11  Toen zeide Jehizkia, dat men kameren aan het huis des HEEREN bereiden zou; en zij bereidden ze.
2Ch 31:12  Daarin brachten zij die heffing, en de tienden, en de geheiligde dingen, in getrouwigheid; en daarover was Chonanja, de Leviet, overste, en Simei, zijn broeder, de tweede.
2Ch 31:13  Maar Jehiel, en Azazja, en Nahath, en Asahel, en Jerimoth, en Jozabad, en Eliel, en Jismachja, en Mahath, en Benaja, waren opzieners, onder de hand van Chonanja en Simei, zijn broeder; door het bevel van den koning Jehizkia en van Azaria, den overste van het huis Gods.
2Ch 31:14  En Kore, de zoon van Jimna, de Leviet, de poortier tegen het oosten, was over de vrijwillige gaven Gods, om het hefoffer des HEEREN en het allerheiligste uit te delen.
2Ch 31:15  En aan zijn hand waren Eden, en Minjamin, en Jesua, en Semaja, Amarja en Sechanja, in de steden der priesteren, met getrouwigheid, om aan hun broederen in de verdelingen, zowel aan de kleinen als de groten, uit te delen:
2Ch 31:16  (Benevens die gesteld waren in het geslachtsregister der manspersonen, drie jaren oud en daarboven) allen, die in het huis des HEEREN gingen, tot het dagelijkse werk op elken dag, voor hun dienst, in hun wachten, naar hun verdelingen.
2Ch 31:17  En met die gesteld waren in het geslachtsregister der priesteren naar het huis hunner vaderen, ook de Levieten van twintig jaren oud en daarboven, in hun wachten, naar hun verdelingen;
2Ch 31:18  Ook tot de geslachtsrekening met al hun kinderkens, hun vrouwen, en hun zonen, en hun dochteren, door de ganse gemeente; want zij hadden zich in hun ambt in heiligheid geheiligd.
2Ch 31:19  Ook waren onder de kinderen van Aaron, de priesteren, op de velden der voorsteden hunner steden, in elke stad, mannen, die met namen uitgedrukt waren, om aan alle manspersonen onder de priesteren en aan allen, die in het geslachtsregister onder de Levieten gesteld waren, delen te geven.
2Ch 31:20  En alzo deed Jehizkia in geheel Juda; en hij deed dat goed, en recht, en waarachtig was, voor het aangezicht des HEEREN, zijns Gods.
2Ch 31:21  En in alle werk, dat hij begon in den dienst van het huis Gods, en in de wet en in het gebod, om zijn God te zoeken, deed hij met zijn ganse hart, en had voorspoed.
2Ch 32:1  Na deze geschiedenissen en derzelver bevestiging, kwam Sanherib, de koning van Assyrie, en toog in Juda, en legerde zich tegen de vaste steden, en dacht ze tot zich af te scheuren.
2Ch 32:2  Jehizkia nu ziende, dat Sanherib kwam, en zijn aangezicht was tot den krijg tegen Jeruzalem;
2Ch 32:3  Zo hield hij raad met zijn vorsten en zijn helden, om de fonteinwateren te stoppen, die buiten de stad waren; en zij hielpen hem.
2Ch 32:4  Want veel volks werd vergaderd, dat al de fonteinen stopte, mitsgaders de beek, die door het midden des lands henenvloeide, zeggende: Waarom zouden de koningen van Assyrie komen, en veel waters vinden?
2Ch 32:5  Zo versterkte hij zich, en bouwde den gehelen muur op, die gebroken was, dien hij optrok tot aan de torens, met een anderen muur daarbuiten, en hij versterkte Millo in de stad Davids; en hij maakte geweer en schilden in menigte.
2Ch 32:6  En hij stelde krijgsoversten over het volk, en hij vergaderde hen tot zich in de straat der stadspoort, en sprak naar hun hart, zeggende:
2Ch 32:7  Zijt sterk, en hebt een goeden moed, vreest niet, en ontzet u niet, voor het aangezicht des konings van Assyrie, noch voor het aangezicht der ganse menigte, die met hem is; want met ons is er meer, dan met hem.
2Ch 32:8  Met hem is een vleselijke arm, maar met ons is de HEERE, onze God, om ons te helpen, en om onze krijgen te krijgen. En het volk steunde op de woorden van Jehizkia, den koning van Juda.
2Ch 32:9  Na dezen zond Sanherib, de koning van Assyrie, zijn knechten naar Jeruzalem, (doch hij zelf was voor Lachis, en al zijn heerschappij met hem) tot Jehizkia, den koning van Juda, en tot het ganse Juda, dat te Jeruzalem was, zeggende:
2Ch 32:10  Zo zegt Sanherib, de koning van Assyrie: Waarom vertrouwt gij, dat gij te Jeruzalem blijft in de vesting?
2Ch 32:11  Ruit u Jehizkia niet op, dat hij u overgeve, om door honger en door dorst te sterven, zeggende: De HEERE, onze God, zal ons uit de hand des konings van Assyrie redden?
2Ch 32:12  Heeft niet dezelfde Jehizkia Zijn hoogten en Zijn altaren weggenomen, en tot Juda en tot Jeruzalem gesproken, zeggende: Voor het enige altaar zult gij u nederbuigen, en daarop roken?
2Ch 32:13  Weet gij niet, wat ik gedaan heb, en mijn vaderen aan alle volken der landen? Hebben de goden van de natien dier landen hun land enigszins kunnen redden uit mijn hand?
2Ch 32:14  Wie is er onder alle goden derzelver natien, dewelke mijn vaders verbannen hebben, die zijn volk heeft kunnen redden uit mijn hand, dat uw God u uit mijn hand zou kunnen redden?
2Ch 32:15  Nu dan, dat Jehizkia ulieden niet bedriege, en dat hij u op zulk een wijze niet opruie, en gelooft hem niet; want geen god van enige natie en koninkrijk heeft zijn volk uit mijn hand en mijner vaderen hand kunnen redden; hoeveel te min zal uw God u uit mijn hand kunnen redden?
2Ch 32:16  Daartoe spraken zijn knechten nog meer tegen God, den HEERE, en tegen Zijn knecht Jehizkia.
2Ch 32:17  Ook schreef hij brieven, om den HEERE den God Israels, te honen en om tegen Hem te spreken, zeggende: Gelijk de goden van de natien der landen, die hun volk uit mijn hand niet gered hebben, alzo zal de God van Jehizkia Zijn volk uit mijn hand niet redden.
2Ch 32:18  En zij riepen met luider stem, in het Joods, tegen het volk van Jeruzalem, dat op den muur was, om die bevreesd te maken en die te beroeren, opdat zij de stad mochten innemen.
2Ch 32:19  En zij spraken van den God van Jeruzalem, als van de goden der volkeren der aarde, een werk van mensenhanden.
2Ch 32:20  Maar de koning Jehizkia en de profeet Jesaja, de zoon van Amoz, baden daartegen, en zij riepen naar den hemel.
2Ch 32:21  En de HEERE zond een engel, die alle strijdbare helden, en vorsten, en oversten in het leger des konings van Assyrie verdelgde. Zo is hij met schaamte des aangezichts in zijn land wedergekeerd; en als hij in het huis zijns gods ingegaan was, zo velden hem daar met het zwaard, die uit zijn lijf voortgekomen waren.
2Ch 32:22  Alzo verloste de HEERE Jehizkia en de inwoners van Jeruzalem, uit de hand van Sanherib, den koning van Assyrie, en uit de hand van allen; en Hij geleidde hen rondom heen.
2Ch 32:23  En velen brachten geschenken tot den HEERE te Jeruzalem, en kostelijkheden tot Jehizkia, den koning van Juda, zodat hij daarna voor de ogen van alle heidenen verheven werd.
2Ch 32:24  In die dagen werd Jehizkia krank tot stervens toe, en hij bad tot den HEERE, Die sprak tot hem, en Hij gaf hem een wonderteken.
2Ch 32:25  Maar Jehizkia deed gene vergelding, naar de weldaad aan hem geschied, dewijl zijn hart verheven werd; daarom werd over hem, en over Juda en Jeruzalem, een grote toornigheid.
2Ch 32:26  Doch Jehizkia verootmoedigde zich om de verheffing zijns harten, hij en de inwoners van Jeruzalem, zodat de grote toornigheid des HEEREN over hen niet kwam in de dagen van Jehizkia.
2Ch 32:27  Jehizkia nu had zeer veel rijkdom en eer; en hij maakte zich schatkameren voor zilver en voor goud, en voor kostelijk gesteente, en voor specerijen, en voor schilden, en voor alle begeerlijk gereedschap;
2Ch 32:28  Ook schathuizen voor de inkomsten van koren, en most, en olie; en stallen voor allerlei beesten, en kooien voor de kudden.
2Ch 32:29  Daartoe had hij zich steden gemaakt, mitsgaders bezitting van schapen en runderen in menigte; want God gaf hem zeer grote have.
2Ch 32:30  Doch Jehizkia stopte ook den opperuitgang der wateren van Gihon, en leidde ze recht af beneden naar het westen der stad Davids; want Jehizkia had voorspoed in al zijn werk.
2Ch 32:31  Maar het is alzo, als de gezanten der vorsten van Babel, die tot hem gezonden hadden, om te vragen naar dat wonderteken, dat in het land geschied was, bij hem waren, verliet hem God, om hem te verzoeken, om te weten al wat in zijn hart was.
2Ch 32:32  Het overige nu der geschiedenissen van Jehizkia, en zijn goeddadigheden, ziet, die zijn geschreven in het gezicht van den profeet Jesaja, den zoon van Amoz, en in het boek der koningen van Juda en Israel.
2Ch 32:33  En Jehizkia ontsliep met zijn vaderen, en zij begroeven hem in het hoogste van de graven der zonen van David; daartoe deden gans Juda en de inwoners van Jeruzalem hem eer aan in zijn dood; en zijn zoon Manasse werd koning in zijn plaats.
2Ch 33:1  Manasse was twaalf jaren oud, als hij koning werd, en regeerde vijf en vijftig jaren te Jeruzalem.
2Ch 33:2  En hij deed dat kwaad was in de ogen des HEEREN, naar de gruwelen der heidenen, die de HEERE voor het aangezicht der kinderen Israels uit de bezitting verdreven had.
2Ch 33:3  Want hij bouwde de hoogten weder op, die zijn vader Jehizkia afgebroken had, en richtte den Baals altaren op, en maakte bossen, en boog zich neder voor al het heir des hemels, en diende ze;
2Ch 33:4  En bouwde altaren in het huis des HEEREN, van hetwelk de HEERE gezegd had: Te Jeruzalem zal Mijn Naam zijn tot in eeuwigheid.
2Ch 33:5  Daartoe bouwde hij altaren voor al het heir des hemels, in beide de voorhoven van het huis des HEEREN.
2Ch 33:6  En hij deed zijn zonen door het vuur gaan, in het dal des zoons van Hinnom, en pleegde guichelarij, en gaf op vogelgeschrei acht, en toverde, en hij stelde waarzeggers en duivelskunstenaren; en hij deed zeer veel kwaads in de ogen des HEEREN, om Hem tot toorn te verwekken.
2Ch 33:7  Hij stelde ook de gelijkenis van een gesneden beeld, die hij gemaakt had, in het huis Gods, van hetwelk God gezegd had tot David en tot zijn zoon Salomo: In dit huis, en te Jeruzalem, dat Ik uit alle stammen van Israel verkoren heb, zal Ik Mijn Naam zetten tot in eeuwigheid.
2Ch 33:8  En Ik zal den voet van Israel niet meer doen wijken van het land, dat Ik uw vaderen besteld heb; alleenlijk zo zij waarnemen te doen, al hetgeen Ik hun geboden heb, naar de ganse wet, en inzettingen, en rechten, door de hand van Mozes.
2Ch 33:9  Zo deed Manasse Juda en de inwoners te Jeruzalem dwalen, dat zij erger deden dan de heidenen, die de HEERE voor het aangezicht der kinderen Israels verdelgd had.
2Ch 33:10  De HEERE sprak wel tot Manasse en tot zijn volk; maar zij merkten daar niet op.
2Ch 33:11  Daarom bracht de HEERE over hen de krijgsoversten, die de koning van Assyrie had, dewelke Manasse gevangen namen onder de doornen; en zij bonden hem met twee koperen ketenen, en voerden hem naar Babel.
2Ch 33:12  En als hij hem benauwde, bad hij het aangezicht des HEEREN, zijns Gods, ernstelijk aan, en vernederde zich zeer voor het aangezicht van den God zijner vaderen,
2Ch 33:13  En bad Hem; en Hij liet Zich van hem verbidden, en hoorde zijn smeking, en Hij bracht hem weder te Jeruzalem, in zijn koninkrijk. Toen erkende Manasse, dat de HEERE God is.
2Ch 33:14  En na dezen bouwde hij den buitenmuur aan de stad Davids, aan de westzijde van Gihon in het dal, en tot den ingang van de Vispoort, en omsingelde Ofel, en verhief dien zeer; hij leide ook krijgsoversten in alle vaste steden in Juda.
2Ch 33:15  En hij nam de vreemde goden en die gelijkenis uit het huis des HEEREN weg, mitsgaders al de altaren, die hij gebouwd had op den berg van het huis des HEEREN, en te Jeruzalem; en hij wierp ze buiten de stad.
2Ch 33:16  En hij richtte het altaar des HEEREN toe, en offerde daarop dankofferen en lofofferen, en zeide tot Juda, dat zij den HEERE, den God Israels, dienen zouden.
2Ch 33:17  Maar het volk offerde nog op de hoogten, hoewel aan den HEERE, hun God.
2Ch 33:18  Het overige nu der geschiedenissen van Manasse, en zijn gebed tot zijn God, ook de woorden der zieners, die tot hem gesproken hebben in den Naam van den HEERE, den God Israels, ziet, die zijn in de geschiedenissen der koningen van Israel;
2Ch 33:19  En zijn gebed, en hoe God Zich van hem heeft laten verbidden, ook al zijn zonde, en zijn overtreding, en de plaatsen, waarop hij hoogten gebouwd, en bossen en gesneden beelden gesteld heeft, eer hij vernederd werd, ziet, dat is beschreven in de woorden der zieners.
2Ch 33:20  En Manasse ontsliep met zijn vaderen, en zij begroeven hem in zijn huis; en zijn zoon Amon werd koning in zijn plaats.
2Ch 33:21  Amon was twee en twintig jaren oud, als hij koning werd, en regeerde twee jaren te Jeruzalem.
2Ch 33:22  En hij deed dat kwaad was in de ogen des HEEREN, gelijk als zijn vader Manasse gedaan had; want Amon offerde al den gesneden beelden, die zijn vader Manasse gemaakt had, en diende ze.
2Ch 33:23  Maar hij vernederde zich niet voor het aangezicht des HEEREN, gelijk Manasse, zijn vader, zich vernederd had; maar deze Amon vermenigvuldigde de schuld.
2Ch 33:24  En zijn knechten maakten een verbintenis tegen hem, en doodden hem in zijn huis.
2Ch 33:25  Maar het volk des lands sloeg hen allen, die de verbintenis tegen den koning Amon gemaakt hadden; en het volk des lands maakte zijn zoon Josia koning in zijn plaats.
2Ch 34:1  Josia was acht jaren oud, toen hij koning werd, en regeerde een en dertig jaren te Jeruzalem.
2Ch 34:2  En hij deed dat recht was in de ogen des HEEREN, en wandelde in de wegen van zijn vader David, en week niet af ter rechter hand, noch ter linkerhand.
2Ch 34:3  Want in het achtste jaar zijner regering, toen hij nog een jongeling was, begon hij den God zijns vaders Davids te zoeken; en in het twaalfde jaar begon hij Juda en Jeruzalem van de hoogten en de bossen, en de gesneden en de gegoten beelden te reinigen.
2Ch 34:4  En men brak voor zijn aangezicht af de altaren der Baals; en de zonnebeelden, die omhoog boven dezelve waren, hieuw hij af; de bossen ook, en de gesneden en gegoten beelden verbrak, en vergruisde, en strooide hij op de graven dergenen, die hun geofferd hadden.
2Ch 34:5  En de beenderen der priesteren verbrandde hij op hun altaren; en hij reinigde Juda en Jeruzalem.
2Ch 34:6  Daartoe in de steden van Manasse, en Efraim, en Simeon, ja, tot Nafthali toe, in haar woeste plaatsen rondom,
2Ch 34:7  Brak hij ook de altaren af en de bossen, en de gesneden beelden stampte hij, die vergruizende, en al de zonnebeelden hieuw hij af in het ganse land van Israel; daarna keerde hij weder naar Jeruzalem.
2Ch 34:8  In het achttiende jaar nu zijner regering, als hij het land en het huis gereinigd had, zond hij Safan, den zoon van Azalia, en Maaseja, den overste der stad, en Joha, den zoon van Joahaz, den kanselier, om het huis des HEEREN, zijns Gods, te verbeteren.
2Ch 34:9  En zij kwamen tot Hilkia, den hogepriester, en zij gaven het geld, dat ten huize Gods gebracht was, hetwelk de Levieten, die den dorpel bewaarden, vergaderd hadden uit de hand van Manasse en Efraim, en uit het ganse overblijfsel van Israel, en uit gans Juda en Benjamin, en te Jeruzalem wedergekomen waren;
2Ch 34:10  Zij nu gaven het in de hand der verzorgers van het werk, die besteld waren over het huis des HEEREN, en deze gaven dat dengenen, die het werk deden, die arbeidden aan het huis des HEEREN, om het huis te vermaken en te verbeteren.
2Ch 34:11  Want zij gaven het den werkmeesters en den bouwlieden, om gehouwen stenen te kopen, en hout tot de samenvoegingen, en om de huizen te zolderen, die de koningen van Juda verdorven hadden.
2Ch 34:12  En die mannen handelden trouwelijk in dit werk; en de bestelden over dezelve waren Jahath en Obadja, Levieten van de kinderen van Merari, mitsgaders Zacharia en Mesullam, van de kinderen der Kahathieten, om het werk voort te drijven; en die Levieten waren allen verstandig op instrumenten van muziek.
2Ch 34:13  Zij waren ook over de lastdragers, en de voortdrijvers van allen, die in enig werk arbeidden; want uit de Levieten waren schrijvers, en ambtlieden, en poortiers.
2Ch 34:14  En als zij het geld uitnamen, dat in het huis des HEEREN gebracht was, vond de priester Hilkia het wetboek des HEEREN, gegeven door de hand van Mozes.
2Ch 34:15  En Hilkia antwoordde en zeide tot Safan, den schrijver: Ik heb het wetboek gevonden in het huis des HEEREN. En Hilkia gaf Safan dat boek.
2Ch 34:16  En Safan droeg dat boek tot den koning; daarbenevens bracht hij nog den koning bescheid weder, zeggende: Al wat in de hand uwer knechten gegeven is, dat doen zij;
2Ch 34:17  En zij hebben het geld samengestort, dat in het huis des HEEREN gevonden is, en hebben het gegeven in de hand der bestelden, en in de hand dergenen, die het werk maakten.
2Ch 34:18  Voorts gaf Safan, de schrijver, den koning te kennen, zeggende: Hilkia, de priester, heeft mij een boek gegeven. En Safan las daarin voor het aangezicht des konings.
2Ch 34:19  Het geschiedde nu, als de koning de woorden der wet hoorde, dat hij zijn klederen scheurde.
2Ch 34:20  En de koning gebood Hilkia, en Ahikam, den zoon van Safan, en Abdon, den zoon van Micha, en Safan, den schrijver, en Asaja, den knecht des konings, zeggende:
2Ch 34:21  Gaat heen, vraagt den HEERE voor mij, en voor het overgeblevene in Israel en in Juda, over de woorden dezes boeks, dat gevonden is; want de grimmigheid des HEEREN is groot, die over ons uitgegoten is, omdat onze vaders niet hebben gehouden het woord des HEEREN, om te doen naar al hetgeen in dat boek geschreven is.
2Ch 34:22  Toen ging Hilkia henen, en die des konings waren, tot de profetes Hulda, de huisvrouw van Sallum, den zoon van Tokhath, den zoon van Hasra, den klederbewaarder. Zij nu woonde te Jeruzalem in het tweede deel; en zij spraken zulks tot haar.
2Ch 34:23  En zij zeide tot hen: Zo zegt de HEERE, de God Israels: Zegt den man, die ulieden tot mij gezonden heeft:
2Ch 34:24  Zo zegt de HEERE: Zie, Ik zal kwaad over deze plaats en over haar inwoners brengen; al de vloeken, die geschreven zijn in het boek, dat men voor het aangezicht des konings van Juda gelezen heeft.
2Ch 34:25  Daarom dat zij Mij verlaten, en anderen goden gerookt hebben, opdat zij Mij tot toorn verwekten met alle werken hunner handen; zo zal Mijn grimmigheid uitgegoten worden tegen deze plaats, en niet uitgeblust worden.
2Ch 34:26  Maar tot den koning van Juda, die ulieden gezonden heeft, om den HEERE te vragen, tot hem zult gij alzo zeggen: Zo zegt de HEERE, de God Israels: Aangaande de woorden, die gij hebt gehoord;
2Ch 34:27  Omdat uw hart week geworden is, en gij u voor het aangezicht Gods vernederd hebt, als gij Zijn woorden hoordet tegen deze plaats en tegen haar inwoners, en hebt u vernederd voor Mijn aangezicht, en uw klederen gescheurd, en geweend voor Mijn aangezicht, zo heb Ik u ook verhoord, spreekt de HEERE.
2Ch 34:28  Zie, Ik zal u verzamelen tot uw vaderen, en gij zult met vrede in uw graf verzameld worden, en uw ogen zullen al dat kwaad niet zien, dat Ik over deze plaats en over haar inwoners brengen zal. En zij brachten den koning dit antwoord weder.
2Ch 34:29  Toen zond de koning henen, en verzamelde alle oudsten van Juda en Jeruzalem.
2Ch 34:30  En de koning ging op in het huis des HEEREN, en al de mannen van Juda en de inwoners van Jeruzalem, mitsgaders de priesters en de Levieten, en al het volk, van den grote tot den kleine toe; en men las voor hun oren al de woorden van het boek des verbonds, dat in het huis des HEEREN gevonden was.
2Ch 34:31  En de koning stond in zijn standplaats, en maakte een verbond voor des HEEREN aangezicht, om den HEERE na te wandelen, en om Zijn geboden, en Zijn getuigenissen, en Zijn inzettingen, met zijn ganse hart en met zijn ganse ziel, te onderhouden, doende de woorden des verbonds, die in datzelve boek geschreven zijn.
2Ch 34:32  En hij deed allen, die te Jeruzalem en in Benjamin gevonden werden, staan; en de inwoners van Jeruzalem deden naar het verbond van God, den God hunner vaderen.
2Ch 34:33  Josia dan deed alle gruwelen weg uit alle landen, die der kinderen Israels waren, en maakte allen, die in Israel gevonden werden, te dienen; te dienen den HEERE, hun God; al zijn dagen weken zij niet af van den HEERE, den God hunner vaderen, na te volgen.
2Ch 35:1  Daarna hield Josia het pascha den HEERE te Jeruzalem; en zij slachtten het pascha op den veertienden der eerste maand.
2Ch 35:2  En hij stelde de priesters op hun wachten; en hij sterkte hen tot den dienst van het huis des HEEREN.
2Ch 35:3  En hij zeide tot de Levieten, die gans Israel onderwezen, die den HEERE heilig waren: Zet de heilige ark in het huis, hetwelk Salomo, de zoon van David, de koning van Israel, gebouwd heeft; gij hebt geen last op de schouderen; dient nu den HEERE, uw God, en Zijn volk Israel;
2Ch 35:4  En bereidt u naar de huizen uwer vaderen, naar uw verdelingen, naar het voorschrift van David, den koning van Israel, en naar de beschrijving van zijn zoon Salomo;
2Ch 35:5  En staat in het heiligdom, naar de onderscheiding der vaderlijke huizen, voor uw broederen, het volk, en naar de afdeling van de vaderlijke huizen der Levieten;
2Ch 35:6  En slacht het pascha, en heiligt u, en bereidt dat voor uw broederen, doende naar het woord des HEEREN, door de hand van Mozes.
2Ch 35:7  En Josia gaf voor het volk, van klein vee, lammeren en jonge geitenbokken, die alle tot paasofferen, naar al hetgeen er gevonden werd, in getal dertig duizend; maar van runderen drie duizend; dit was van des konings have.
2Ch 35:8  Ook gaven zijn vorsten tot een vrijwillig offer voor het volk, voor de priesteren, en voor de Levieten; Hilkia, en Zacharia, en Jehiel, de oversten van het huis Gods, gaven den priesteren tot paasofferen, twee duizend en zeshonderd klein vee, en driehonderd runderen.
2Ch 35:9  Daartoe Chonanja, en Semaja, en Nethaneel, zijn broeders, mitsgaders Hasabja, en Jeiel, en Jozabad, de oversten der Levieten, gaven den Levieten tot paasofferen, vijf duizend klein vee en vijfhonderd runderen.
2Ch 35:10  Alzo werd de dienst toebereid; en de priesteren stonden in hun standplaats, en de Levieten in hun verdelingen, naar het gebod des konings.
2Ch 35:11  Daarna slachtte men het pascha, en de priesters sprengden het bloed uit hun handen, en de Levieten trokken de huiden af.
2Ch 35:12  En zij namen het brandoffer daar af, opdat zij die naar de verdelingen der vaderlijke huizen, aan het volk geven mochten, om den HEERE te offeren, gelijk geschreven is in het boek van Mozes; en alzo met de runderen.
2Ch 35:13  En zij kookten het pascha bij het vuur, naar het recht; maar de andere heilige dingen kookten zij in potten, en in ketels, en in pannen; en zij deelden het haastelijk onder al het volk.
2Ch 35:14  Daarna bereidden zij ook voor zichzelven en voor de priesteren; want de priesters, de zonen van Aaron, waren tot aan den nacht in het offeren der brandofferen en des vets; daarom bereidden de Levieten voor zichzelven, en voor de priesteren, de zonen van Aaron.
2Ch 35:15  En de zangers, de zonen van Asaf, waren in hun standplaats, naar het gebod van David, en Asaf, en Heman, en Jeduthun, den ziener des konings, mitsgaders de poortiers aan elke poort; zij behoefden niet te wijken van hun dienst, overmits hun broeders, de Levieten, voor hen bereidden.
2Ch 35:16  Alzo werd de ganse dienst des HEEREN op denzelfden dag beschikt, om pascha te houden, en brandofferen op het altaar des HEEREN te offeren, naar het gebod van den koning Josia.
2Ch 35:17  En de kinderen Israels, die er gevonden werden, hielden het pascha ter zelfder tijd, en het feest der ongezuurde broden, zeven dagen.
2Ch 35:18  Daar was ook geen pascha als dat in Israel gehouden, van de dagen van Samuel, den profeet, af; en geen koningen van Israel hadden zulk een pascha gehouden, gelijk dat Josia hield met de priesters en de Levieten, en gans Juda en Israel, dat er gevonden werd, en de inwoners van Jeruzalem.
2Ch 35:19  In het achttiende jaar van het koninkrijk van Josia, werd dit pascha gehouden.
2Ch 35:20  Na dit alles, toen Josia het huis toebereid had, toog Necho, de koning van Egypte, op, om te krijgen tegen Karchemis, aan den Frath; en Josia toog uit hem tegemoet.
2Ch 35:21  Toen zond hij boden tot hem, zeggende: Wat heb ik met u te doen, gij, koning van Juda? Wat u aangaat, ik ben heden tegen u niet, maar tegen een huis, dat oorlog voert tegen mij; en God heeft gezegd, dat ik mij haasten zou; houd u af van God, Die met mij is, opdat Hij u niet verderve.
2Ch 35:22  Doch Josia keerde zijn aangezicht niet van hem; maar hij verstelde zich, om tegen hem te strijden, en hoorde niet naar de woorden van Necho uit den mond van God; maar hij kwam om te strijden in het dal Megiddo.
2Ch 35:23  En de schutters schoten den koning Josia. Toen zeide de koning tot zijn knechten: Voert mij weg, want ik ben zeer gewond.
2Ch 35:24  En zijn knechten namen hem weg van den wagen, en voerden hem op den tweeden wagen, dien hij had, en brachten hem te Jeruzalem; en hij stierf, en werd begraven in de graven zijner vaderen; en gans Juda en Jeruzalem bedreven rouw over Josia.
2Ch 35:25  En Jeremia maakte een klaaglied over Josia; desgelijks alle zangers en zangeressen spraken in hun klaagliederen van Josia, tot op dezen dag; want zij gaven ze tot een inzetting in Israel; en ziet, zij zijn geschreven in de klaagliederen.
2Ch 35:26  Het overige nu der geschiedenissen van Josia, en zijn goeddadigheden, naar dat geschreven is in de wet des HEEREN;
2Ch 35:27  Zijn geschiedenissen dan, de eerste en de laatste, ziet, die zijn geschreven in het boek der koningen van Israel en van Juda.
2Ch 36:1  Toen nam het volk des lands Joahaz, den zoon van Josia, en zij maakten hem koning, in zijns vaders plaats, te Jeruzalem.
2Ch 36:2  Drie en twintig jaren was Joahaz oud, als hij koning werd, en hij regeerde drie maanden te Jeruzalem.
2Ch 36:3  Want de koning van Egypte zette hem af te Jeruzalem; en hij leide het land een boete op van honderd talenten zilvers en een talent gouds.
2Ch 36:4  En de koning van Egypte maakte zijn broeder Eljakim koning over Juda en Jeruzalem, en veranderde zijn naam in Jojakim; maar zijn broeder Joahaz nam Necho, en bracht hem in Egypte.
2Ch 36:5  Vijf en twintig jaren was Jojakim oud, als hij koning werd, en regeerde elf jaren te Jeruzalem; en hij deed dat kwaad was in de ogen des HEEREN, zijns Gods.
2Ch 36:6  Nebukadnezar, de koning van Babel, toog tegen hem op, en bond hem met twee koperen ketenen, om hem te voeren naar Babel.
2Ch 36:7  Nebukadnezar bracht ook van de vaten van het huis des HEEREN naar Babel, en stelde ze in zijn tempel te Babel.
2Ch 36:8  Het overige nu van de geschiedenissen van Jojakim, en zijn gruwelen, die hij deed, en wat aan hem gevonden werd, ziet, dat is geschreven in het boek der koningen van Israel en Juda; en Jojachin, zijn zoon, werd koning in zijn plaats.
2Ch 36:9  Acht jaren was Jojachin oud, als hij koning werd, en regeerde drie maanden en tien dagen te Jeruzalem, en deed dat kwaad was in de ogen des HEEREN.
2Ch 36:10  En met de wederkomst des jaars zond de koning Nebukadnezar henen, en liet hem naar Babel halen, met de kostelijke vaten van het huis des HEEREN; en hij maakte zijn broeder Zedekia koning over Juda en Jeruzalem.
2Ch 36:11  Een en twintig jaren was Zedekia oud, als hij koning werd, en regeerde elf jaren te Jeruzalem.
2Ch 36:12  En hij deed dat kwaad was in de ogen des HEEREN, zijns Gods; hij verootmoedigde zich niet voor het aangezicht van den profeet Jeremia, sprekende uit den mond des HEEREN.
2Ch 36:13  Daartoe werd hij ook afvallig tegen den koning Nebukadnezar, die hem beedigd had bij God; en verhardde zijn nek, en verstokte zijn hart, dat hij zich niet bekeerde tot den HEERE, den God Israels.
2Ch 36:14  Ook maakten alle oversten der priesteren, en het volk, der overtredingen zeer veel, naar alle gruwelen der heidenen; en zij verontreinigden het huis des HEEREN, dat Hij geheiligd had te Jeruzalem.
2Ch 36:15  En de HEERE, de God hunner vaderen, zond tot hen, door de hand Zijner boden, vroeg op zijnde, om die te zenden; want Hij verschoonde Zijn volk en Zijn woning.
2Ch 36:16  Maar zij spotten met de boden Gods, en verachtten Zijn woorden; zij verleidden zichzelven tegen Zijn profeten; totdat de grimmigheid des HEEREN tegen Zijn volk opging, dat er geen helen aan was.
2Ch 36:17  Want Hij deed tegen hen opkomen den koning der Chaldeen, die hun jongelingen met het zwaard in het huis huns heiligdoms doodde, en hij verschoonde de jongelingen niet, noch de maagden, de ouden noch de stokouden; Hij gaf hen allen in zijn hand.
2Ch 36:18  En alle vaten van het huis Gods, de grote en de kleine, en de schatten van het huis des HEEREN, en de schatten des konings en zijner vorsten, dit alles voerde hij naar Babel.
2Ch 36:19  En zij verbrandden het huis Gods, en zij braken den muur van Jeruzalem af, en al de paleizen daarvan verbrandden zij met vuur, verdervende ook alle kostelijke vaten derzelve.
2Ch 36:20  En wie overgebleven was van het zwaard, voerde hij weg naar Babel, en zij werden hem en zijn zonen tot knechten, tot het regeren des koninkrijks van Perzie;
2Ch 36:21  Opdat het woord des HEEREN vervuld wierd, door den mond van Jeremia, totdat het land aan zijn sabbatten een welgevallen had; het rustte al de dagen der verwoesting, totdat de zeventig jaren vervuld waren.
2Ch 36:22  Maar in het eerste jaar van Kores, koning van Perzie, opdat volbracht wierd het woord des HEEREN, door den mond van Jeremia, verwekte de HEERE den geest van Kores, koning van Perzie, dat hij een stem liet doorgaan door zijn ganse koninkrijk, zelfs ook in geschrift, zeggende:
2Ch 36:23  Zo zegt Kores, koning van Perzie: De HEERE, de God des hemels, heeft mij alle koninkrijken der aarde gegeven; en Hij heeft mij bevolen Hem een huis te bouwen te Jeruzalem, hetwelk in Juda is; wie is onder ulieden van al Zijn volk? De HEERE, zijn God, zij met hem, en hij trekke op.



\end{document}