\begin{document}

\title{Ezra}



\chapter{1}

\par 1 In het eerste jaar nu van Kores, koning van Perzie, opdat volbracht wierd het woord des HEEREN, uit den mond van Jeremia, verwekte de HEERE den geest van Kores, koning van Perzie, dat hij een stem liet doorgaan door zijn ganse koninkrijk, zelfs ook in geschrift, zeggende:
\par 2 Zo zegt Kores, koning van Perzie: De HEERE, de God des hemels, heeft mij alle koninkrijken der aarde gegeven; en Hij heeft mij bevolen Hem een huis te bouwen te Jeruzalem, hetwelk in Juda is.
\par 3 Wie is onder ulieden van al Zijn volk? Zijn God zij met hem, en hij trekke op naar Jeruzalem, dat in Juda is, en hij bouwe het huis des HEEREN, des Gods van Israel; Hij is de God, Die te Jeruzalem woont.
\par 4 En al wie achterblijven zou in enige plaatsen, waar hij als vreemdeling verkeert, dien zullen de lieden zijner plaats bevorderlijk zijn met zilver, en met goud, en met have, en met beesten; benevens een vrijwillige gave, voor het huis Gods, Die te Jeruzalem woont.
\par 5 Toen maakten zich op de hoofden der vaderen van Juda en Benjamin, en de priesteren en de Levieten, benevens een iegelijk, wiens geest God verwekte, dat zij optrokken om te bouwen het huis des HEEREN, die te Jeruzalem woont.
\par 6 Allen nu, die rondom hen waren, sterkten hunlieder handen met zilveren vaten, met goud, met have, en met beesten, en met kostelijkheden; behalve alles, wat vrijwillig gegeven werd.
\par 7 Ook bracht de koning Kores uit, de vaten van het huis des HEEREN, die Nebukadnezar uit Jeruzalem had uitgevoerd, en had gesteld in het huis zijns gods.
\par 8 En Kores, de koning van Perzie, bracht ze uit door de hand van Mithredath, den schatmeester, die ze aan Sesbazar, den vorst van Juda, toetelde.
\par 9 En dit is hun getal: dertig gouden bekkens, duizend zilveren bekkens, negen en twintig messen;
\par 10 Dertig gouden bekers, vierhonderd en tien andere zilveren bekers; andere vaten, duizend.
\par 11 Alle vaten van goud en van zilver waren vijf duizend en vierhonderd; deze alle voerde Sesbazar op, met degenen, die van de gevangenis opgevoerd werden, van Babel naar Jeruzalem.

\chapter{2}

\par 1 Dit zijn de kinderen van dat landschap, die optogen uit de gevangenis, van de weggevoerden, die Nebukadnezar, koning van Babel, weggevoerd had naar Babel, die naar Jeruzalem en Juda zijn wedergekeerd, een iegelijk naar zijn stad;
\par 2 Dewelken kwamen met Zerubbabel, Jesua, Nehemia, Seraja, Reelaja, Mordechai, Bilsan, Mizpar, Bigvai, Rehum en Baena. Dit is het getal der mannen des volks van Israel.
\par 3 De kinderen van Paros, twee duizend honderd twee en zeventig.
\par 4 De kinderen van Sefatja, driehonderd twee en zeventig.
\par 5 De kinderen van Arach, zevenhonderd vijf en zeventig.
\par 6 De kinderen van Pahath-moab, van de kinderen van Jesua-joab, twee duizend achthonderd en twaalf.
\par 7 De kinderen van Elam, duizend tweehonderd vier en vijftig.
\par 8 De kinderen van Zatthu, negenhonderd vijf en zestig.
\par 9 De kinderen van Zakkai, zevenhonderd zestig.
\par 10 De kinderen van Bani, zeshonderd twee en veertig.
\par 11 De kinderen van Bebai, zeshonderd drie en twintig.
\par 12 De kinderen van Azgad, duizend tweehonderd twee en twintig.
\par 13 De kinderen van Adonikam, zeshonderd zes en zestig.
\par 14 De kinderen van Bigvai, twee duizend zes en vijftig.
\par 15 De kinderen van Adin, vierhonderd vier en vijftig.
\par 16 De kinderen van Ater, van Hizkia, acht en negentig.
\par 17 De kinderen van Bezai, driehonderd drie en twintig.
\par 18 De kinderen van Jora, honderd en twaalf.
\par 19 De kinderen van Hasum, tweehonderd drie en twintig.
\par 20 De kinderen van Gibbar, vijf en negentig.
\par 21 De kinderen van Bethlehem, honderd drie en twintig.
\par 22 De mannen van Netofa, zes en vijftig.
\par 23 De mannen van Anathoth, honderd acht en twintig.
\par 24 De kinderen van Azmaveth, twee en veertig.
\par 25 De kinderen van Kirjath-arim, Cefira en Beeroth, zevenhonderd drie en veertig.
\par 26 De kinderen van Rama en Gaba, zeshonderd een en twintig.
\par 27 De mannen van Michmas, honderd twee en twintig.
\par 28 De mannen van Beth-el en Ai, tweehonderd drie en twintig.
\par 29 De kinderen van Nebo, twee en vijftig.
\par 30 De kinderen van Magbis, honderd zes en vijftig.
\par 31 De kinderen van den anderen Elam, duizend tweehonderd vier en vijftig.
\par 32 De kinderen van Harim, driehonderd en twintig.
\par 33 De kinderen van Lod, Hadid en Ono, zevenhonderd vijf en twintig.
\par 34 De kinderen van Jericho, driehonderd vijf en veertig.
Ezr 2:35  De kinderen van Senaa, drie duizend zeshonderd en dertig.
Ezr 2:36  De priesters. De kinderen van Jedaja, van het huis van Jesua, negenhonderd drie en zeventig.
Ezr 2:37  De kinderen van Immer, duizend twee en vijftig.
Ezr 2:38  De kinderen van Pashur, duizend tweehonderd zeven en veertig.
Ezr 2:39  De kinderen van Harim, duizend en zeventien.
Ezr 2:40  De Levieten. De kinderen van Jesua en Kadmiel, van de kinderen van Hodavja, vier en zeventig.
Ezr 2:41  De zangers. De kinderen van Asaf honderd acht en twintig.
Ezr 2:42  De kinderen der poortiers. De kinderen van Sallum, de kinderen van Ater, de kinderen van Talmon, de kinderen van Akkub, de kinderen van Hatita, de kinderen van Sobai; deze allen waren honderd negen en dertig.
Ezr 2:43  De Nethinim. De kinderen van Ziha, de kinderen van Hasufa, de kinderen van Tabbaoth;
Ezr 2:44  De kinderen van Keros, de kinderen van Siaha, de kinderen van Padon;
Ezr 2:45  De kinderen van Lebana, de kinderen van Hagaba, de kinderen van Akkub;
Ezr 2:46  De kinderen van Hagab, de kinderen van Samlai, de kinderen van Hanan;
Ezr 2:47  De kinderen van Giddel, de kinderen van Gahar, de kinderen van Reaja;
Ezr 2:48  De kinderen van Rezin, de kinderen van Nekoda, de kinderen van Gazzam;
Ezr 2:49  De kinderen van Uza, de zonen van Paseah, de kinderen van Bezai;
Ezr 2:50  De kinderen van Asna, de kinderen der Mehunim, de kinderen der Nefusim;
Ezr 2:51  De kinderen van Bakbuk, de kinderen van Hakufa, de kinderen van Harhur;
Ezr 2:52  De kinderen van Bazluth, de kinderen van Mehida, de kinderen van Harsa;
Ezr 2:53  De kinderen van Barkos, de kinderen van Sisera, de kinderen van Thamah;
Ezr 2:54  De kinderen van Neziah, de kinderen van Hatifa.
Ezr 2:55  De kinderen der knechten van Salomo. De kinderen van Sotai, de kinderen van Sofereth, de kinderen van Peruda;
Ezr 2:56  De kinderen van Jaala, de kinderen van Darkon, de kinderen van Giddel;
Ezr 2:57  De kinderen van Sefatja, de kinderen van Hattil, de kinderen van Pocheret-hazebaim, de kinderen van Ami.
Ezr 2:58  Al de Nethinim, en de kinderen der knechten van Salomo, waren driehonderd twee en negentig.
Ezr 2:59  Dezen togen ook op van Tel-melah, Tel-harsa, Cherub, Addan en Immer; doch zij konden hunner vaderen huis en hun zaad niet bewijzen, of zij uit Israel waren.
Ezr 2:60  De kinderen van Delaja, de kinderen van Tobia, de kinderen van Nekoda, zeshonderd twee en vijftig.
Ezr 2:61  En van de kinderen der priesteren, de kinderen van Habaja, de kinderen van Koz, de kinderen van Barzillai, die van de dochteren van Barzillai, den Gileadiet, een vrouw genomen had, en naar hun naam genoemd was.
Ezr 2:62  Dezen zochten hun register, onder degenen, die in het geslachtsregister gesteld waren, maar zij werden niet gevonden; daarom werden zij als onreinen van het priesterdom geweerd.
Ezr 2:63  En Hattirsatha zeide tot hen, dat zij van de heiligste dingen niet zouden eten, totdat er een priester stond met urim en met thummim.
Ezr 2:64  Deze ganse gemeente te zamen was twee en veertig duizend driehonderd en zestig.
Ezr 2:65  Behalve hun knechten en hun maagden, die waren zeven duizend driehonderd zeven en dertig; en zij hadden tweehonderd zangers en zangeressen.
Ezr 2:66  Hun paarden waren zevenhonderd zes en dertig; hun muildieren, tweehonderd vijf en veertig;
Ezr 2:67  Hun kemelen, vierhonderd vijf en dertig; de ezelen, zes duizend zevenhonderd en twintig.
Ezr 2:68  En sommigen van de hoofden der vaderen, als zij kwamen ten huize des HEEREN, die te Jeruzalem woont, gaven vrijwilliglijk ten huize Gods, om dat te zetten op zijn vaste plaats.
Ezr 2:69  Zij gaven naar hun vermogen tot den schat des werks, aan goud, een en zestig duizend drachmen, en aan zilver, vijf duizend ponden, en honderd priesterrokken.
Ezr 2:70  En de priesters en de Levieten, en sommigen uit het volk, zo de zangers als de poortiers, en de Nethinim woonden in hun steden, en gans Israel in zijn steden.
Ezr 3:1  Toen nu de zevende maand aankwam, en de kinderen Israels in de steden waren, verzamelde zich het volk, als een enig man, te Jeruzalem.
Ezr 3:2  En Jesua, de zoon van Jozadak, maakte zich op, en zijn broederen, de priesters en Zerubbabel, de zoon van Sealthiel, en zijn broederen, en zij bouwden het altaar des Gods van Israel, om daarop brandofferen te offeren, gelijk geschreven is in de wet van Mozes, den man Gods.
Ezr 3:3  En zij vestigden het altaar op zijn stelling, maar met verschrikking, die over hen was, vanwege de volken der landen; en zij offerden daarop brandofferen den HEERE, brandofferen des morgens en des avonds.
Ezr 3:4  En zij hielden het feest der loofhutten, gelijk geschreven is; en zij offerden brandofferen dag bij dag in getal, naar het recht, van elk dagelijks op zijn dag.
Ezr 3:5  Daarna ook het gedurig brandoffer, en van de nieuwe maanden, en van alle gezette hoogtijden des HEEREN, die geheiligd waren; ook van een ieder, die een vrijwillige offerande den HEERE vrijwilliglijk offerde.
Ezr 3:6  Van den eersten dag af der zevende maand begonnen zij den HEERE brandofferen te offeren; doch de grond van den tempel des HEEREN was niet gelegd.
Ezr 3:7  Zo gaven zij geld aan de houwers en werkmeesters, ook spijs en drank, en olie aan de Sidoniers en aan de Tyriers, om cederenhout van den Libanon te brengen aan de zee naar Jafo, naar de vergunning van Kores, koning van Perzie, aan hen.
Ezr 3:8  In het tweede jaar nu hunner aankomst ten huize Gods te Jeruzalem, in de tweede maand, begonnen Zerubbabel, de zoon van Sealthiel, en Jesua, de zoon van Jozadak, en de overige hunner broederen, de priesters en de Levieten, en allen, die uit de gevangenis te Jeruzalem gekomen waren; en zij stelden de Levieten, van twintig jaren oud en daarboven, om opzicht te nemen over het werk van des HEEREN huis.
Ezr 3:9  Toen stond Jesua, zijn zonen en zijn broederen, en Kadmiel met zijn zonen, kinderen van Juda, als een man, om opzicht te hebben over degenen, die het werk deden aan het huis Gods, met de zonen van Henadad, hun zonen en hun broederen, de Levieten.
Ezr 3:10  Als nu de bouwlieden den grond van des HEEREN tempel leiden, zo stelden zij de priesteren, aangekleed zijnde, met trompetten, en de Levieten, Asafs zonen, met cimbalen, om den HEERE te loven, naar de instelling van David, den koning van Israel.
Ezr 3:11  En zij zongen bij beurten, met den HEERE te loven en te danken, dat Hij goed is, dat Zijn weldadigheid tot in eeuwigheid is over Israel. En al het volk juichte met groot gejuich, als men den HEERE loofde over de grondlegging van het huis des HEEREN.
Ezr 3:12  Maar velen van de priesteren, en de Levieten, en hoofden der vaderen, die oud waren, die het eerste huis gezien hadden, dit huis in zijn grondlegging voor hun ogen zijnde, weenden met luider stem; maar velen verhieven de stem met gejuich en met vreugde.
Ezr 3:13  Zodat het volk niet onderkende de stem van het gejuich der vreugde, van de stem des geweens van het volk; want het volk juichte met groot gejuich, dat de stem tot van verre gehoord werd.
Ezr 4:1  Toen nu de wederpartijders van Juda en Benjamin hoorden, dat de kinderen der gevangenis den HEERE, den God Israels, den tempel bouwden;
Ezr 4:2  Zo kwamen zij aan tot Zerubbabel, en tot de hoofden der vaderen, en zeiden tot hen: Laat ons met ulieden bouwen, want wij zullen uw God zoeken, gelijk gijlieden; ook hebben wij Hem geofferd sinds de dagen van Esar-haddon, den koning van Assur, die ons herwaarts heeft doen optrekken.
Ezr 4:3  Maar Zerubbabel, en Jesua, en de overige hoofden der vaderen van Israel zeiden tot hen: Het betaamt niet, dat gijlieden en wij onzen God een huis bouwen; maar wij alleen zullen het den HEERE, den God Israels, bouwen, gelijk als de koning Kores, koning van Perzie, ons geboden heeft.
Ezr 4:4  Evenwel maakte het volk des lands de handen des volks van Juda slap, en verstoorde hen in het bouwen;
Ezr 4:5  En zij huurden tegen hen raadslieden, om hun raad te vernietigen, al de dagen van Kores, koning van Perzie, tot aan het koninkrijk van Darius, den koning van Perzie.
Ezr 4:6  En onder het koninkrijk van Ahasveros, in het begin zijns koninkrijks, schreven zij een aanklacht tegen de inwoners van Juda en Jeruzalem.
Ezr 4:7  En in de dagen van Arthahsasta schreef Bislam, Mithredath, Tabeel, en de overigen van zijn gezelschap, aan Arthahsasta, koning van Perzie; en de schrift des briefs was in het Syrisch geschreven, en in het Syrisch uitgelegd.
Ezr 4:8  Rehum, de kanselier, en Simsai, de schrijver, schreven een brief tegen Jeruzalem, aan den koning Arthahsasta, op deze manier:
Ezr 4:9  Toen Rehum, de kanselier, en Simsai, de schrijver, en de overigen van hun gezelschap, de Dinaieten, de Afarsathchieten, de Tarpelieten, de Afarsieten, de Archevieten, de Babyloniers, de Susanchieten, de Dehavieten, de Elamieten,
Ezr 4:10  En de overige volkeren, die de grote en vermaarde Asnappar heeft vervoerd, en doen wonen in de stad van Samaria, ook de overigen, aan deze zijde der rivier, en op zulken tijd.
Ezr 4:11  Dit is een afschrift des briefs, dien zij aan hem, aan den koning Arthahsasta, zonden: Uw knechten, de mannen aan deze zijde der rivier, en op zulken tijd.
Ezr 4:12  Den koning zij bekend, dat de Joden, die van u zijn opgetogen, tot ons gekomen zijn te Jeruzalem, bouwende die rebelle en die boze stad, waarvan zij de muren voltrekken, en de fondamenten samenvoegen.
Ezr 4:13  Zo zij nu den koning bekend, indien dezelve stad zal worden opgebouwd, en de muren voltrokken, dat zij den cijns, ouden impost, en tol niet zullen geven, en gij zult aan de inkomsten der koningen schade aanbrengen.
Ezr 4:14  Nu, omdat wij salaris uit het paleis trekken, en het ons niet betaamt des konings oneer te zien, daarom hebben wij gezonden, en dit den koning bekend gemaakt;
Ezr 4:15  Opdat men zoeke in het boek der kronieken uwer vaderen, zo zult gij vinden in het boek der kronieken, en weten, dat dezelve stad een rebelle stad geweest is, en den koningen en landschappen schade aanbrengende, en dat zij daarbinnen afval gesticht hebben, van oude tijden af; daarom is dezelve stad verwoest.
Ezr 4:16  Wij maken dan de koning bekend, dat, zo dezelve stad zal worden opgebouwd, en haar muren voltrokken, gij daardoor geen deel zult hebben aan deze zijde der rivier.
Ezr 4:17  De koning zond antwoord aan Rehum, den kanselier, en Simsai, den schrijver, en de overigen van hun gezelschappen, die te Samaria woonden; mitsgaders aan de overigen van deze zijde der rivier aldus: Vrede, en op zulken tijd.
Ezr 4:18  De brief, dien gij aan ons geschikt hebt, is duidelijk voor mij gelezen.
Ezr 4:19  En als van mij bevel gegeven was, hebben zij gezocht en gevonden, dat dezelve stad zich van oude tijden af tegen de koningen heeft verheven, en rebellie en afval daarin gesticht is.
Ezr 4:20  Ook zijn er machtige koningen geweest over Jeruzalem, die geheerst hebben overal aan gene zijde der rivier; en hun is cijns, oude impost en tol gegeven.
Ezr 4:21  Geeft dan nu bevel, om diezelve mannen te beletten, dat diezelve stad niet opgebouwd worde, totdat van mij bevel zal worden gegeven.
Ezr 4:22  Weest gewaarschuwd, van feil in dezen te begaan; waarom zou het verderf tot schade der koningen aanwassen?
Ezr 4:23  Toen, van dat het afschrift des briefs van den koning Arthahsasta voor Rehum, en Simsai, den schrijver, en hun gezelschappen gelezen was, togen zij in haast naar Jeruzalem tot de Joden, en beletten hen met arm en geweld.
Ezr 4:24  Toen hield het werk op van het huis Gods, Die te Jeruzalem woont, ja, het hield op tot in het tweede jaar van het koninkrijk van Darius, den koning van Perzie.
Ezr 5:1  Haggai nu, de profeet, en Zacharia, de zoon van Iddo, profeteerden tot de Joden, die in Juda en te Jeruzalem waren; in den naam Gods van Israel profeteerden zij tot hen.
Ezr 5:2  Toen maakten zich op Zerubbabel, de zoon van Sealthiel, en Jesua, de zoon van Jozadak, en begonnen te bouwen het huis Gods, Die te Jeruzalem woont; en met hen de profeten Gods, die hen ondersteunden.
Ezr 5:3  Te dier tijd kwam tot hen Thathnai, de landvoogd aan deze zijde der rivier, en Sthar-boznai, en hun gezelschap, en zeiden aldus tot hen: Wie heeft ulieden bevel gegeven dit huis te bouwen, en dezen muur te voltrekken?
Ezr 5:4  Toen zeiden wij aldus tot hen, en welke de namen waren der mannen, die dit gebouw bouwden.
Ezr 5:5  Doch het oog huns Gods was over de oudsten der Joden, dat zij hun niet beletten, totdat de zaak aan Darius kwam, en zij alsdan daarover een brief wederbrachten.
Ezr 5:6  Afschrift des briefs, dien Thathnai, de landvoogd aan deze zijde der rivier, met Sthar-boznai, en zijn gezelschap, de Afarsechaieten, die aan deze zijde der rivier waren, aan den koning Darius zond.
Ezr 5:7  Zij zonden een verhaal aan hem; en daarin was aldus geschreven: Den koning Darius zij alle vrede.
Ezr 5:8  Den koning zij bekend, dat wij getogen zijn naar het landschap Juda, ten huize des groten Gods, hetwelk gebouwd wordt met grote stenen, en het hout wordt geleid in de wanden; en datzelve werk wordt ras gedaan, en gaat voorspoediglijk door hun handen voort.
Ezr 5:9  Toen hebben wij denzelven oudsten gevraagd, en aldus tot hen gezegd: Wie heeft ulieden bevel gegeven dit huis te bouwen, en dezen muur te voltrekken?
Ezr 5:10  Wijders hebben wij hun ook hun namen afgevraagd, dat wij ze u bekend maakten; dat wij mochten overschrijven de namen der mannen, die hoofden onder hen zijn.
Ezr 5:11  En zij hebben ons dusdanig antwoord wedergegeven, zeggende: Wij zijn knechten van den God des hemels en der aarde, en bouwen het huis, dat vele jaren voor dezen is gebouwd geweest; want een groot koning van Israel had het gebouwd en voltrokken.
Ezr 5:12  Maar nadat onze vaders den God des hemels hadden vertoornd, heeft Hij hen gegeven in de hand van Nebukadnezar, den koning van Babel, den Chaldeer; dewelke dat huis heeft vernield, en het volk naar Babel weggevoerd.
Ezr 5:13  Doch in het eerste jaar van Kores, koning van Babel, heeft de koning Kores bevel gegeven dit huis Gods te bouwen.
Ezr 5:14  Ja, de vaten van Gods huis, welke van goud en zilver waren, die Nebukadnezar uit den tempel, die te Jeruzalem was, had weggenomen en dezelve gebracht in den tempel van Babel, die heeft de koning Kores uitgehaald uit den tempel van Babel, en zij zijn gegeven aan een, wiens naam was Sesbazar, dien hij tot een landvoogd had gesteld.
Ezr 5:15  En hij zeide tot hem: Neem deze vaten, ga ze afvoeren in den tempel, die te Jeruzalem is, en laat het huis Gods gebouwd worden op zijn plaats.
Ezr 5:16  Toen kwam dezelve Sesbazar; hij leide de fondamenten van het huis Gods, Die te Jeruzalem woont; en er is van toen af tot nu toe gebouwd, doch niet volbracht.
Ezr 5:17  Zo het dan nu den koning goeddunkt, laat er gezocht worden in het schathuis des konings aldaar, dat te Babel is, of het zij, dat een bevel van den koning Kores gegeven zij, om dit huis Gods te Jeruzalem te bouwen; en dat men des konings believen hiervan tot ons zende.
Ezr 6:1  Toen gaf de koning Darius bevel; en zij zochten in de kanselarij, waar de schatten waren weggelegd, in Babel.
Ezr 6:2  En te Achmetha, in de burcht, die in het landschap Medie is, werd een rol gevonden; en daarin was aldus geschreven: GEDACHTENIS;
Ezr 6:3  In het eerste jaar van den koning Kores, gaf de koning Kores dit bevel: Het huis Gods te Jeruzalem, dat huis zal gebouwd worden, ter plaatse, waar zij offeranden offeren, en de fondamenten daarvan zullen zwaar zijn; zijn hoogte van zestig ellen, en zijn breedte van zestig ellen;
Ezr 6:4  Met drie rijen van groten steen, en een rij van nieuw hout; en de onkosten zullen uit des konings huis gegeven worden.
Ezr 6:5  Daartoe zal men ook de gouden en zilveren vaten van het huis Gods, die Nebukadnezar uit den tempel, die te Jeruzalem was, heeft weggevoerd, en naar Babel gebracht, wedergeven, dat zij gaan naar den tempel, die te Jeruzalem is, aan zijn plaats, en men zal ze afvoeren ten huize Gods.
Ezr 6:6  Nu, gij Thathnai, landvoogd aan gene zijde der rivier, gij Sthar-boznai, met ulieder gezelschap, gij Afarsechaieten, die aan gene zijde der rivier zijt, weest verre van daar!
Ezr 6:7  Laat hen aan den arbeid van dit huis Gods; dat de landvoogd der Joden en de oudsten der Joden dit huis Gods bouwen aan zijn plaats.
Ezr 6:8  Ook wordt van mij bevel gegeven, wat gijlieden doen zult aan de oudsten dezer Joden, om dit huis Gods te bouwen; te weten, dat uit des konings goederen, van den cijns aan gene zijde der rivier, de onkosten dezen mannen spoediglijk gegeven worden, opdat men hen niet belette.
Ezr 6:9  En wat nodig is, als jonge runderen, en rammen, en lammeren, tot brandofferen aan den God des hemels, tarwe, zout, wijn en olie, naar het zeggen der priesteren, die te Jeruzalem zijn, dat het hun dag bij dag gegeven worde, dat er geen feil zij;
Ezr 6:10  Opdat zij offeranden van liefelijken reuk aan den God des hemels offeren, en bidden voor het leven des konings en zijner kinderen.
Ezr 6:11  Voorts wordt bevel van mij gegeven, dat al dengene, die dit woord zal veranderen, een hout uit zijn huis zal gerukt en opgericht worden, waaraan hij zal worden opgehangen; en zijn huis zal om diens wille tot een drekhoop gemaakt worden.
Ezr 6:12  De God nu, die Zijn Naam aldaar heeft doen wonen, werpe ter neder alle koningen en volken, die hun hand zullen uitstrekken, om te veranderen en te verderven dit huis Gods, dat te Jeruzalem is. Ik, Darius, heb het bevel gegeven, dat het spoediglijk gedaan worde.
Ezr 6:13  Toen deden Thathnai, de landvoogd aan gene zijde der rivier, Sthar-boznai, en hun gezelschap, spoediglijk alzo, naar hetgeen de koning Darius gezonden had.
Ezr 6:14  En de oudsten der Joden bouwden en gingen voorspoediglijk voort, door de profetie van den profeet Haggai en Zacharia, den zoon van Iddo; en zij bouwden en voltrokken het, naar het bevel van den God Israels, en naar het bevel van Kores, en Darius, en Arthahsasta, koning van Perzie.
Ezr 6:15  En dit huis werd volbracht op den derden dag der maand Adar; datzelve was het zesde jaar van het koninkrijk van den koning Darius.
Ezr 6:16  En de kinderen Israels, de priesteren en Levieten, en de overige kinderen der gevangenis deden de inwijding van dit huis Gods met vreugde.
Ezr 6:17  En zij offerden, ter inwijding van dit huis Gods, honderd runderen, tweehonderd rammen, vierhonderd lammeren en twaalf geitenbokken, ten zondoffer voor gans Israel, naar het getal der stammen Israels.
Ezr 6:18  En zij stelden de priesteren in hun onderscheidingen, en de Levieten in hun verdelingen, tot den dienst Gods, Die te Jeruzalem is, naar het voorschrift des boeks van Mozes.
Ezr 6:19  Ook hielden de kinderen der gevangenis het pascha, op den veertienden der eerste maand.
Ezr 6:20  Want de priesters en de Levieten hadden zich gereinigd als een enig man; zij waren allen rein; en zij slachtten het pascha voor alle kinderen der gevangenis, en voor hun broederen, de priesteren, en voor zichzelven.
Ezr 6:21  Alzo aten de kinderen Israels, die uit de gevangenis wedergekomen waren, mitsgaders al wie zich van de onreinigheid der heidenen des lands tot hen afgezonderd had, om den HEERE, den God Israels, te zoeken.
Ezr 6:22  En zij hielden het feest der ongezuurde broden zeven dagen, met blijdschap; want de HEERE had hen verblijd, en het hart des konings van Assur tot hen gewend, om hun handen te sterken in het werk van het huis Gods, des Gods van Israel.
Ezr 7:1  Na deze geschiedenissen nu, in het koninkrijk van Arthahsasta, koning van Perzie: Ezra, de zoon van Seraja, den zoon van Azarja, den zoon van Hilkia,
Ezr 7:2  Den zoon van Sallum, den zoon van Zadok, den zoon van Ahitub,
Ezr 7:3  Den zoon van Amarja, den zoon van Azarja, den zoon van Merajoth,
Ezr 7:4  Den zoon van Zerahja, den zoon van Uzzi, den zoon van Bukki,
Ezr 7:5  Den zoon van Abisua, den zoon van Pinehas, den zoon van Eleazar, den zoon van Aaron, den hoofdpriester.
Ezr 7:6  Deze Ezra toog op uit Babel; en hij was een vaardig schriftgeleerde in de wet van Mozes, die de HEERE, de God Israels, gegeven heeft; en de koning gaf hem, naar de hand des HEEREN, zijns Gods, over hem, al zijn verzoek.
Ezr 7:7  Ook sommigen van de kinderen Israels, en van de priesteren en de Levieten, en de zangers, en de poortiers, en de Nethinim, togen op naar Jeruzalem, in het zevende jaar van den koning Arthahsasta.
Ezr 7:8  En hij kwam te Jeruzalem in de vijfde maand; dat was het zevende jaar dezes konings.
Ezr 7:9  Want op den eersten der eerste maand was het begin des optochts uit Babel, en op den eersten der vijfde maand kwam hij te Jeruzalem, naar de goede hand zijns Gods over hem.
Ezr 7:10  Want Ezra had zijn hart gericht, om de wet des HEEREN te zoeken en te doen, en om in Israel te leren de inzettingen en de rechten.
Ezr 7:11  Dit is nu het afschrift des briefs, dien de koning Arthahsasta gaf aan Ezra, den priester, den schriftgeleerde; den schriftgeleerde van de woorden der geboden des HEEREN, en Zijn inzettingen over Israel:
Ezr 7:12  Arthahsasta koning der koningen, aan Ezra, den priester, den schriftgeleerde der wet van den God des hemels, volkomen vrede en op zulken tijd.
Ezr 7:13  Van mij wordt bevel gegeven, dat al wie vrijwillig is in mijn koninkrijk, van het volk van Israel, en van deszelfs priesteren en Levieten, om te gaan naar Jeruzalem, dat hij met u ga.
Ezr 7:14  Dewijl gij van voor den koning en zijn zeven raadsheren gezonden zijt, om onderzoek te doen in Judea, en te Jeruzalem, naar de wet uws Gods, die in uw hand is;
Ezr 7:15  En om henen te brengen het zilver en goud, dat de koning en zijn raadsheren vrijwilliglijk gegeven hebben aan den God Israels, Wiens woning te Jeruzalem is;
Ezr 7:16  Mitsgaders al het zilver en goud, dat gij vinden zult in het ganse landschap van Babel, met de vrijwillige gave des volks en der priesteren, die vrijwilliglijk geven, ten huize huns Gods, dat te Jeruzalem is;
Ezr 7:17  Opdat gij spoediglijk voor dat geld koopt runderen, rammen, lammeren, met hun spijsofferen, en hun drankofferen, en die offert op het altaar van het huis van ulieder God, dat te Jeruzalem is.
Ezr 7:18  Daartoe, wat u en uw broederen goed dunken zal, met het overige zilver en goud te doen, zult gijlieden doen naar het welgevallen uws Gods.
Ezr 7:19  En geef de vaten, die u gegeven zijn tot den dienst van het huis uws Gods, weder voor den God van Jeruzalem.
Ezr 7:20  Het overige nu, dat van node zal zijn voor het huis uws Gods, dat u voorvallen zal uit te geven, zult gij geven uit het schathuis des konings.
Ezr 7:21  En van mij, mij, koning Arthahsasta, wordt bevel gegeven aan alle schatmeesters, die aan gene zijde der rivier zijt, dat alles, wat Ezra, de priester, de schriftgeleerde der wet van den God des hemels, van u zal begeren, spoediglijk gedaan worde;
Ezr 7:22  Tot honderd talenten zilvers toe, en tot honderd kor tarwe, en tot honderd bath wijn, en tot honderd bath olie, en zout zonder voorschrift.
Ezr 7:23  Al wat naar het bevel van den God des hemels is, dat het vlijtiglijk gedaan worde, voor het huis van den God des hemels; want waartoe zou er grote toorn zijn over het koninkrijk des konings en zijner kinderen?
Ezr 7:24  Ook laten wij ulieden weten, aangaande alle priesteren en Levieten, zangers, poortiers, Nethinim en dienaars van het huis dezes Gods, dat men den cijns, ouden impost en tol hun niet zal vermogen op te leggen.
Ezr 7:25  En gij, Ezra, naar de wijsheid uws Gods, die in uw hand is, stel regeerders en richters, die al het volk richten, dat aan gene zijde der rivier is, allen, die de wetten Gods weten, en die ze niet weet, zult gijlieden die bekend maken.
Ezr 7:26  En al wie de wet uws Gods en de wet des konings niet zal doen, over dien laat spoediglijk recht worden gedaan, hetzij ter dood, of tot uitbanning, of tot boete van goederen, of tot de banden.
Ezr 7:27  Geloofd zij de HEERE, de God onzer vaderen, Die alzulks in het hart des konings gegeven heeft, om te versieren het huis des HEEREN, dat te Jeruzalem is.
Ezr 7:28  En heeft tot mij weldadigheid geneigd, voor het aangezicht des konings en zijner raadsheren, en aller geweldige vorsten des konings! Zo heb ik mij gesterkt, naar de hand des HEEREN, mijns Gods, over mij, en de hoofden uit Israel vergaderd, om met mij op te trekken.
Ezr 8:1  Dit nu zijn de hoofden hunner vaderen, met hun geslachtsrekening, die met mij uit Babel optogen, onder het koninkrijk van den koning Arthahsasta.
Ezr 8:2  Van de kinderen van Pinehas, Gersom; van de kinderen van Ithamar, Daniel; van de kinderen van David, Hattus.
Ezr 8:3  Van de kinderen van Sechanja, van de kinderen van Paros, Zacharja; en met hem werden bij geslachtsregisters gerekend, aan manspersonen, honderd en vijftig.
Ezr 8:4  Van de kinderen van Pahath-moab, Eljehoenai, de zoon van Zerahja; en met hem tweehonderd manspersonen.
Ezr 8:5  Van de kinderen van Sechanja, de zoon van Jahaziel; en met hem driehonderd manspersonen.
Ezr 8:6  En van de kinderen van Adin, Ebed, de zoon van Jonathan; en met hem vijftig manspersonen.
Ezr 8:7  En van de kinderen van Elam, Jesaja, de zoon van Athalja; en met hem zeventig manspersonen.
Ezr 8:8  En van de kinderen van Sefatja, Zebadja, de zoon van Michael; en met hem tachtig manspersonen.
Ezr 8:9  En van de kinderen van Joab, Obadja, de zoon van Jehiel; en met hem tweehonderd en achttien manspersonen.
Ezr 8:10  En van de kinderen van Selomith, de zoon van Josifja; en met hem honderd en zestig manspersonen.
Ezr 8:11  En van de kinderen van Babai, Zacharja, de zoon van Bebai; en met hem acht en twintig manspersonen.
Ezr 8:12  En van de kinderen van Azgad, Johanan, de zoon van Katan; en met hem honderd en tien manspersonen.
Ezr 8:13  En van de laatste kinderen van Adonikam, welker namen deze waren: Elifelet, Jehiel, en Semaja; en met hen zestig manspersonen.
Ezr 8:14  En van de kinderen van Bigvai, Uthai en Zabbud; en met hen zeventig manspersonen.
Ezr 8:15  En ik vergaderde hen aan de rivier, gaande naar Ahava, en wij legerden ons aldaar drie dagen; toen lette ik op het volk en de priesteren, en vond aldaar geen van de kinderen van Levi.
Ezr 8:16  Zo zond ik tot Eliezer, tot Ariel, tot Semaja, en tot Elnathan, en tot Jarib, en tot Elnathan, en tot Nathan, en tot Zacharja, en tot Mesullam, de hoofden; en tot Jojarib en tot Elnathan, de leraars;
Ezr 8:17  En ik gaf hun bevel aan Iddo, het hoofd in de plaats Chasifja; en ik leide de woorden in hun mond, om te zeggen tot Iddo, zijn broeder, en de Nethinim, in de plaats Chasifja, dat zij ons brachten dienaars voor het huis onzes Gods.
Ezr 8:18  En zij brachten ons, naar de goede hand onzes Gods over ons, een man van verstand, van de kinderen van Mahli, den zoon van Levi, den zoon van Israel; namelijk Serebja, met zijn zonen en broederen, achttien;
Ezr 8:19  En Hasabja, en met hem Jesaja, van de kinderen van Merari, met zijn broederen, en hun zonen, twintig;
Ezr 8:20  En van Nethinim, die David en de vorsten ten dienste der Levieten gegeven hadden, tweehonderd en twintig Nethinim, die allen bij namen genoemd werden.
Ezr 8:21  Toen riep ik aldaar een vasten uit aan de rivier Ahava, opdat wij ons verootmoedigden voor het aangezicht onzes Gods, om van Hem te verzoeken een rechten weg, voor ons, en voor onze kinderkens, en voor al onze have.
Ezr 8:22  Want ik schaamde mij van den koning een heir en ruiters te begeren, om ons te helpen van den vijand, op den weg; omdat wij tot den koning hadden gesproken, zeggende: De hand onzes Gods is ten goede over allen, die Hem zoeken, maar Zijn sterkte en Zijn toorn over allen, die Hem verlaten.
Ezr 8:23  Alzo vastten wij; en verzochten zulks van onzen God; en Hij liet zich van ons verbidden.
Ezr 8:24  Toen scheidde ik twaalf uit van de oversten der priesteren: Serebja, Hasabja, en tien van hun broederen met hen.
Ezr 8:25  En ik woog hun toe het zilver, en het goud, en de vaten, zijnde de offering van het huis onzes Gods die de koning en zijn raadsheren, en zijn vorsten, en gans Israel, die er gevonden werden, geofferd hadden;
Ezr 8:26  Ik woog dan aan hun hand zeshonderd en vijftig talenten zilvers, en honderd zilveren vaten in talenten; aan goud, honderd talenten;
Ezr 8:27  En twintig gouden bekers, tot duizend drachmen; en twee vaten van blinkend goed koper, begeerlijk als goud.
Ezr 8:28  En ik zeide tot hen: Gij zijt heilig den HEERE, en deze vaten zijn heilig; ook dit zilver en dit goud, de vrijwillige gave, den HEERE, den God uwer vaderen.
Ezr 8:29  Waakt en bewaart het, totdat gij het opweegt, in tegenwoordigheid van de oversten der priesteren en Levieten, en der vorsten der vaderen van Israel, te Jeruzalem, in de kameren van des HEEREN huis.
Ezr 8:30  Toen ontvingen de priesters en de Levieten het gewicht des zilvers en des gouds, en der vaten, om te brengen te Jeruzalem, ten huize onzes Gods.
Ezr 8:31  Alzo verreisden wij van de rivier Ahava, op den twaalfden der eerste maand, om te gaan naar Jeruzalem; en de hand onzes Gods was over ons, en redde ons van de hand des vijands, en desgenen, die ons lagen leide op den weg.
Ezr 8:32  En wij kwamen te Jeruzalem; en wij bleven aldaar drie dagen.
Ezr 8:33  Op den vierden dag nu werd gewogen het zilver, en het goud, en de vaten, in het huis onzes Gods, aan de hand van Meremoth, den zoon van Uria, den priester, en met hem Eleazar, de zoon van Pinehas; en met hem Jozabad, de zoon van Jesua, en Noadja, de zoon van Binnui, de Levieten.
Ezr 8:34  Naar het getal en naar het gewicht van dat alles; en het ganse gewicht werd ter zelfder tijd opgeschreven.
Ezr 8:35  En de weggevoerden, die uit de gevangenis gekomen waren, offerden den God Israels brandofferen; twaalf varren voor gans Israel, zes en negentig rammen, zeven en zeventig lammeren, twaalf bokken ten zondoffer; alles ten brandoffer den HEERE.
Ezr 8:36  Daarna gaven zij de wetten des konings aan des konings stadhouders en landvoogden aan deze zijde der rivier; en zij bevorderden het volk en het huis Gods.
Ezr 9:1  Als nu deze dingen voleind waren, traden de vorsten tot mij toe, zeggende: Het volk Israels, en de priesters, en de Levieten, zijn niet afgezonderd van de volken dezer landen, naar hun gruwelen, namelijk van de Kanaanieten, de Hethieten, de Ferezieten, de Jebusieten, de Ammonieten, de Moabieten, de Egyptenaren en de Amorieten.
Ezr 9:2  Want zij hebben van hun dochteren genomen voor zichzelven en voor hun zonen, zodat zich vermengd hebben het heilig zaad met de volken dezer landen; ja, de hand der vorsten en overheden is de eerste geweest in deze overtreding.
Ezr 9:3  Als ik nu deze zaak hoorde, scheurde ik mijn kleed en mijn mantel; en ik trok van het haar mijns hoofds en mijns baards uit, en zat verbaasd neder.
Ezr 9:4  Toen verzamelden zich tot mij allen, die voor de woorden van den God Israels beefden, om de overtreding der weggevoerden; doch ik bleef verbaasd zitten tot aan het avondoffer.
Ezr 9:5  En omtrent het avondoffer stond ik op uit mijn bedruktheid, als ik nu mijn kleed en mijn mantel gescheurd had; en ik boog mij op mijn knieen, en breidde mijn handen uit tot den HEERE, mijn God;
Ezr 9:6  En ik zeide: Mijn God, ik ben beschaamd en schaamrood, om mijn aangezicht tot U op te heffen, mijn God; want onze ongerechtigheden zijn vermenigvuldigd tot boven ons hoofd, en onze schuld is groot geworden tot aan den hemel.
Ezr 9:7  Van de dagen onzer vaderen af zijn wij in grote schuld tot op dezen dag; en wij zijn om onze ongerechtigheden overgegeven, wij, onze koningen en onze priesters, in de hand van de koningen der landen, in zwaard, in gevangenis, en in roof, en in schaamte des aangezichts, gelijk het is te dezen dage.
Ezr 9:8  En nu is er, als een klein ogenblik, een genade geschied van den HEERE, onzen God, om ons een ontkoming over te laten, en ons een nagel te geven in Zijn heilige plaats, om onze ogen te verlichten, o onze God, en om ons een weinig levens te geven in onze dienstbaarheid.
Ezr 9:9  Want wij zijn knechten; doch in onze dienstbaarheid heeft ons onze God niet verlaten; maar Hij heeft weldadigheid tot ons geneigd voor het aangezicht der koningen van Perzie, dat Hij ons een weinig levens gave, om het huis onzes Gods te verhogen, en de woestigheden van hetzelve op te richten, en om ons een tuin te geven in Juda en te Jeruzalem.
Ezr 9:10  En nu, wat zullen wij zeggen, o onze God! na dezen? Want wij hebben Uw geboden verlaten,
Ezr 9:11  Die Gij geboden hadt door den dienst Uwer knechten, de profeten, zeggende: Het land, waar gijlieden inkomt, om dat te erven, is een vuil land, door de vuiligheid van de volken der landen, om hun gruwelen, waarmede zij dat vervuld hebben, van het ene einde tot het andere einde, met hun onreinigheid.
Ezr 9:12  Zo zult gij nu uw dochteren niet geven aan hun zonen, en hun dochteren niet nemen voor uw zonen, en zult hun vrede en hun best niet zoeken, tot in eeuwigheid; opdat gij sterk wordt, en het goede des lands eet, en uw kinderen doet erven tot in eeuwigheid.
Ezr 9:13  En na alles, wat over ons gekomen is, om onze boze werken, en om onze grote schuld, omdat Gij, o onze God! belet hebt, dat wij niet te onder zijn vanwege onze ongerechtigheid, en hebt ons een ontkoming gegeven, als deze is;
Ezr 9:14  Zullen wij nu wederkeren, om Uw geboden te vernietigen, en ons te verzwageren met de volken dezer gruwelen? Zoudt Gij niet tegen ons toornen tot verterens toe, dat er geen overblijfsel noch ontkoming zij?
Ezr 9:15  O HEERE, God van Israel! Gij zijt rechtvaardig; want wij zijn overgelaten ter ontkoming, als het is te dezen dage. Zie, wij zijn voor Uw aangezicht in onze schuld; want er is niemand, die voor Uw aangezicht zou kunnen bestaan, om zulks.
Ezr 10:1  Als Ezra alzo bad, en als hij deze belijdenis deed, wenende en zich voor Gods huis nederwerpende, verzamelde zich tot hem uit Israel een zeer grote gemeente van mannen, en vrouwen, en kinderen; want het volk weende met groot geween.
Ezr 10:2  Toen antwoordde Sechanja, de zoon van Jehiel, een van de zonen van Elam, en zeide tot Ezra: Wij hebben overtreden tegen onzen God, en wij hebben vreemde vrouwen van de volken des lands bij ons doen wonen; maar nu, er is hope voor Israel, dezen aangaande.
Ezr 10:3  Laat ons dan nu een verbond maken met onze God, dat wij al die vrouwen, en wat van haar geboren is, zullen doen uitgaan, naar den raad des HEEREN, en dergenen, die beven voor het gebod onzes Gods; en laat er gedaan worden naar de wet.
Ezr 10:4  Sta op, want deze zaak komt u toe; en wij zullen met u zijn; wees sterk en doe het.
Ezr 10:5  Toen stond Ezra op, en deed de oversten der priesteren, de Levieten en gans Israel zweren, te zullen doen naar dit woord; en zij zwoeren.
Ezr 10:6  En Ezra stond op van voor Gods huis, en ging in de kamer van Johanan, den zoon van Eljasib; als hij daar kwam, at hij geen brood, en dronk geen water, want hij bedreef rouw over de overtreding der weggevoerden.
Ezr 10:7  En zij lieten een stem doorgaan door Juda en Jeruzalem, aan al de kinderen der gevangenis, dat zij zich te Jeruzalem zouden verzamelen.
Ezr 10:8  En al wie niet kwam in drie dagen, naar den raad der vorsten en der oudsten, al zijn have zou verbannen zijn; en hij zelf zou afgezonderd wezen van de gemeente der weggevoerden.
Ezr 10:9  Toen verzamelden zich alle mannen van Juda en Benjamin te Jeruzalem in drie dagen; het was de negende maand op den twintigsten in de maand; en al het volk zat op de straat van Gods huis, sidderende om deze zaak, en vanwege de plasregenen.
Ezr 10:10  Toen stond Ezra, de priester, op en zeide tot hen: Gijlieden hebt overtreden, en vreemde vrouwen bij u doen wonen, om Israels schuld te vermeerderen.
Ezr 10:11  Nu dan, doet den HEERE, uwer vaderen God, belijdenis en doet Zijn welgevallen, en scheidt u af van de volken des lands, en van de vreemde vrouwen.
Ezr 10:12  En de ganse gemeente antwoordde en zeide met luider stem: Naar uw woorden, alzo komt het ons toe te doen.
Ezr 10:13  Maar des volks is veel, en het is een tijd van plasregen, dat men hier buiten niet staan kan; en het is geen werk van een dag noch van twee; want velen onzer hebben overtreden in deze zaak.
Ezr 10:14  Laat toch onze vorsten der ganse gemeente hierover staan, en allen, die in onze steden zijn, die vreemde vrouwen bij zich hebben doen wonen, op gezette tijden komen, en met hen de oudsten van elke stad en derzelver rechters; totdat wij van ons afwenden de hittigheid des toorns onzes Gods, om dezer zaken wil.
Ezr 10:15  Alleenlijk Jonathan, de zoon van Asahel, en Jehazia, de zoon van Tikva, stonden hierover; en Mesullam, en Sabbethai, de Leviet, hielpen hen.
Ezr 10:16  En de kinderen der gevangenis deden alzo; en Ezra, de priester, met de mannen, de hoofden der vaderen, naar het huis hunner vaderen, en zij allen, bij namen genoemd, scheidden zich af, en zij zaten op den eersten dag der tiende maand, om deze zaak te onderzoeken.
Ezr 10:17  En zij voleindden het met alle mannen, die vreemde vrouwen bij zich hadden doen wonen, tot op den eersten dag der eerste maand.
Ezr 10:18  En er werden gevonden van de zonen der priesteren, die vreemde vrouwen bij zich hadden doen wonen; van de zonen van Jesua, den zoon van Jozadak, en zijn broederen, Maaseja, en Eliezer, en Jarib, en Gedalja.
Ezr 10:19  En zij gaven hun hand, dat zij hun vrouwen zouden doen uitgaan; en schuldig zijnde, offerden zij een ram van de kudde voor hun schuld.
Ezr 10:20  En van de kinderen van Immer: Hanani en Zebadja.
Ezr 10:21  En van de kinderen van Harim: Maaseja, en Elia, en Semaja, en Jehiel, en Uzia,
Ezr 10:22  En van de kinderen van Pashur: Eljoenai, Maaseja, Ismael, Nethaneel, Jozabad en Elasa.
Ezr 10:23  En van de Levieten: Jozabad, en Simei, en Kelaja (deze is Kelita), Pethahja, Juda en Eliezer.
Ezr 10:24  En van de zangers: Eljasib; en van de poortiers: Sallum, en Telem, en Uri.
Ezr 10:25  En van Israel: van de kinderen van Paros: Ramja, en Jezia, en Malchia, en Mijamin, en Eleazar, en Malchia, en Benaja.
Ezr 10:26  En van de kinderen van Elam: Mattanja, Zacharja, en Jehiel, en Abdi, en Jeremoth, en Elia.
Ezr 10:27  En van de kinderen van Zatthu: Eljoenai, Eljasib, Mattanja, en Jeremoth, en Zabad, Aziza.
Ezr 10:28  En van de kinderen van Bebai: Johanan, Hananja, Sabbai, en Athlai.
Ezr 10:29  En van de kinderen van Bani: Mesullam, Malluch en Adaja, Jasub en Seal, Jeramoth.
Ezr 10:30  En van de kinderen van Pahath-moab: Adna, en Chelal, Benaja, Maaseja, Mattanja, Bezaleel, en Binnui, en Manasse.
Ezr 10:31  En van de kinderen van Harim: Eliezer, Jissia, Malchia, Semaja, Simeon.
Ezr 10:32  Benjamin, Malluch, Semarja.
Ezr 10:33  Van de kinderen van Hasum: Mathnai, Mattata, Zabad, Elifelet, Jeremai, Manasse, Simei.
Ezr 10:34  Van de kinderen van Bani: Maadai, Amram, en Uel,
Ezr 10:35  Benaja, Bedeja, Cheluhu,
Ezr 10:36  Vanja, Meremoth, Eljasib,
Ezr 10:37  Mattanja, Mathnai, en Jaasai,
Ezr 10:38  En Bani, en Binnui, Simei,
Ezr 10:39  En Selemja, en Nathan, en Adaja,
Ezr 10:40  Machnadbai, Sasai, Sarai,
Ezr 10:41  Azareel, Selemja, Semarja,
Ezr 10:42  Sallum, Amarja, Jozef.
Ezr 10:43  Van de kinderen van Nebo: Jeiel, Mattithja, Zabad, Zebina, Jaddai, en Joel, Benaja.
Ezr 10:44  Alle dezen hadden vreemde vrouwen genomen; en sommigen van hen hadden vrouwen, waarbij zij kinderen gekregen hadden.



\end{document}