\begin{document}

\title{Esther}



\chapter{1}

\par 1 Het geschiedde nu in de dagen van Ahasveros, (hij is die Ahasveros, dewelke regeerde van Indie af tot aan Morenland toe, honderd zeven en twintig landschappen).
\par 2 In die dagen, als de koning Ahasveros op den troon zijns koninkrijks zat, die op den burg Susan was;
\par 3 In het derde jaar zijner regering maakte hij een maaltijd al zijn vorsten en zijn knechten; de macht van Perzie en Medie, de grootste heren en de oversten der landschappen waren voor zijn aangezicht;
\par 4 Als hij vertoonde den rijkdom der heerlijkheid zijns rijks, en de kostelijkheid des sieraads zijner grootheid, vele dagen lang, honderd en tachtig dagen.
\par 5 Toen nu die dagen vervuld waren, maakte de koning een maaltijd al den volke, dat gevonden werd op den burg Susan, van den grootste tot den kleinste, zeven dagen lang, in het voorhof van den hof van het koninklijk paleis.
\par 6 Er waren witte, groene en hemelsblauwe behangselen, gevat aan fijn linnen en purperen banden, in zilveren ringen, en aan marmeren pilaren; de bedsteden waren van goud en zilver, op een vloer van porfier steen, en van marmer, en albast, en kostelijke stenen.
\par 7 En men gaf te drinken in vaten van goud, en het ene vat was anders dan het andere vat; en er was veel koninklijke wijn, naar des konings vermogen.
\par 8 En het drinken geschiedde naar de wet, dat niemand dwong; want alzo had de koning vastelijk bevolen aan alle groten zijns huizes, dat zij doen zouden naar den wil van een iegelijk.
\par 9 De koningin Vasthi maakte ook een maaltijd voor de vrouwen in het koninklijk huis, hetwelk de koning Ahasveros had.
\par 10 Op den zevenden dag, toen des konings hart vrolijk was van den wijn, zeide hij tot Mehuman, Biztha, Charbona, Bigtha en Abagtha, Zethar en Charchas, de zeven kamerlingen, dienende voor het aangezicht van den koning Ahasveros,
\par 11 Dat zij Vasthi, de koningin, zouden brengen voor het aangezicht des konings, met de koninklijke kroon, om den volken en den vorsten haar schoonheid te tonen; want zij was schoon van aangezicht.
\par 12 Doch de koningin Vasthi weigerde te komen op het woord des konings, hetwelk door den dienst der kamerlingen haar aangezegd was. Toen werd de koning zeer verbolgen, en zijn grimmigheid ontstak in hem.
\par 13 Toen zeide de koning tot de wijzen, die de tijden verstonden (want alzo moest des konings zaak geschieden, in de tegenwoordigheid van al degenen, die de wet en het recht wisten;
\par 14 De naasten nu bij hem waren Carsena, Sethar, Admatha, Tharsis, Meres, Marsena, Memuchan, zeven vorsten der Perzen en der Meden, die het aangezicht des konings zagen, die vooraan zaten in het koninkrijk),
\par 15 Wat men naar de wet met de koningin Vasthi doen zou, omdat zij niet gedaan had het woord van den koning Ahasveros, door den dienst der kamerlingen?
\par 16 Toen zeide Memuchan voor het aangezicht des konings en der vorsten: De koningin Vasthi heeft niet alleen tegen den koning misdaan, maar ook tegen al de vorsten, en tegen al de volken, die in al de landschappen van den koning Ahasveros zijn.
\par 17 Want deze daad der koningin zal uitkomen tot alle vrouwen, zodat zij haar mannen verachten zullen in haar ogen, als men zeggen zal: De koning Ahasveros zeide, dat men de koningin Vasthi voor zijn aangezicht brengen zou; maar zij kwam niet.
\par 18 Te dezen zelfden dage zullen de vorstinnen van Perzie en Medie ook alzo zeggen tot al de vorsten des konings, als zij deze daad der koningin zullen horen, en er zal verachtens en toorns genoeg wezen.
\par 19 Indien het den koning goeddunkt, dat een koninklijk gebod van hem uitga, hetwelk geschreven worde in de wetten der Perzen en Meden, en dat men het niet overtrede: dat Vasthi niet inga voor het aangezicht van den koning Ahasveros, en de koning geve haar koninkrijk aan haar naaste, die beter is dan zij.
\par 20 Als het bevel des konings, hetwelk hij doen zal in zijn ganse koninkrijk, (want het is groot) gehoord zal worden, zo zullen alle vrouwen aan haar mannen eer geven, van de grootste tot de kleinste toe.
\par 21 Dit woord nu was goed in de ogen des konings en der vorsten; en de koning deed naar het woord van Memuchan.
\par 22 En hij zond brieven aan al de landschappen des konings, aan een iegelijk landschap naar zijn schrift, en aan elk volk naar zijn spraak, dat elk man overheer in zijn huis wezen zou, en spreken naar de spraak zijns volks.

\chapter{2}

\par 1 Na deze geschiedenissen, toen de grimmigheid van den koning Ahasveros gestild was, gedacht hij aan Vasthi, en wat zij gedaan had, en wat over haar besloten was.
\par 2 Toen zeiden de jongelingen des konings, die hem dienden: Men zoeke voor den koning jonge dochters, maagden, schoon van aangezicht.
\par 3 En de koning bestelle toezieners in al de landschappen zijns koninkrijks, dat zij vergaderen alle jonge dochters, maagden, schoon van aangezicht, tot den burg Susan, tot het huis der vrouwen, onder de hand van Hegai, des konings kamerling, bewaarder der vrouwen; en men geve haar haar versierselen.
\par 4 En de jonge dochter, die in des konings oog schoon wezen zal, worde koningin in stede van Vasthi. Deze zaak nu was goed in de ogen des konings, en hij deed alzo.
\par 5 Er was een Joods man op den burg Susan, wiens naam was Mordechai, een zoon van Jair, den zoon van Simei, den zoon van Kis, een man van Jemini;
\par 6 Die weggevoerd was van Jeruzalem met de weggevoerden, die weggevoerd waren met Jechonia, den koning van Juda, denwelken Nebukadnezar, de koning van Babel, had weggevoerd.
\par 7 En hij was het, die opvoedde Hadassa (deze is Esther, de dochter zijns ooms); want zij had geen vader noch moeder; en zij was een jonge dochter, schoon van gedaante, en schoon van aangezicht; en als haar vader en haar moeder stierven, had Mordechai ze zich tot een dochter aangenomen.
\par 8 Het geschiedde nu, toen het woord des konings en zijn wet ruchtbaar was, en toen vele jonge dochters samenvergaderd werden op den burg Susan, onder de hand van Hegai, werd Esther ook genomen in des konings huis, onder de hand van Hegai, den bewaarder der vrouwen.
\par 9 En die jonge dochter was schoon in zijn ogen, en zij verkreeg gunst voor zijn aangezicht; daarom haastte hij met haar versierselen en met haar delen haar te geven, en zeven aanzienlijke jonge dochters haar te geven uit het huis des konings; en hij verplaatste haar en haar jonge dochters naar het beste van het huis der vrouwen.
\par 10 Esther had haar volk en haar maagschap niet te kennen gegeven; want Mordechai had haar geboden, dat zij het niet zou te kennen geven.
\par 11 Mordechai nu wandelde allen dag voor het voorhof van het huis der vrouwen, om te vernemen naar den welstand van Esther, en wat met haar geschieden zou.
\par 12 Als nu de beurt van elke jonge dochter naakte, om tot den koning Ahasveros te komen, nadat haar twaalf maanden lang naar de wet der vrouwen geschied was; want alzo werden vervuld de dagen harer versieringen, zes maanden met mirreolie, en zes maanden met specerijen, en met andere versierselen der vrouwen;
\par 13 Daarmede kwam dan de jonge dochter tot den koning; al wat zij zeide, werd haar gegeven, dat zij daarmede ging uit het huis der vrouwen tot het huis des konings.
\par 14 Des avonds ging zij daarin, en des morgens ging zij weder naar het tweede huis der vrouwen, onder de hand van Saasgaz, den kamerling des konings, bewaarder der bijwijven, zij kwam niet weder tot den koning, ten ware de koning lust tot haar had, en zij bij name geroepen werd.
\par 15 Als de beurt van Esther, de dochter van Abichail, den oom van Mordechai, (die hij zich ter dochter genomen had) naakte, dat zij tot den koning komen zou, begeerde zij niet met al, dan wat Hegai, des konings kamerling, de bewaarder der vrouwen, zeide; en Esther verkreeg genade in de ogen van allen, die haar zagen.
\par 16 Alzo werd Esther genomen tot den koning Ahasveros, tot zijn koninklijk huis, in de tiende maand, welke is de maand Tebeth, in het zevende jaar zijns rijks.
\par 17 En de koning beminde Esther boven alle vrouwen, en zij verkreeg genade en gunst voor zijn aangezicht, boven alle maagden; en hij zette de koninklijke kroon op haar hoofd, en hij maakte haar koningin in de plaats van Vasthi.
\par 18 Toen maakte de koning een groten maaltijd al zijn vorsten en zijn knechten, den maaltijd van Esther; en hij gaf den landschappen rust, en hij gaf geschenken naar des konings vermogen.
\par 19 Toen ten anderen male maagden vergaderd werden, zo zat Mordechai in de poort des konings.
\par 20 Esther nu had haar maagschap en haar volk niet te kennen gegeven, gelijk als Mordechai haar geboden had; want Esther deed het bevel van Mordechai, gelijk als toen zij bij hem opgevoed werd.
\par 21 In die dagen, als Mordechai in de poort des konings zat, werden Bigthan en Theres, twee kamerlingen des konings van de dorpelwachters, zeer toornig, en zij zochten de hand te slaan aan den koning Ahasveros.
\par 22 En deze zaak werd Mordechai bekend gemaakt, en hij gaf ze de koningin Esther te kennen; en Esther zeide het den koning in Mordechai's naam.
\par 23 Als men de zaak onderzocht, is het zo bevonden, en zij beiden werden aan een galg gehangen; en het werd in de kronieken geschreven voor het aangezicht des konings.

\chapter{3}

\par 1 Na deze geschiedenissen maakte de koning Ahasveros Haman groot, den zoon van Hammedatha, den Agagiet, en hij verhoogde hem, en hij zette zijn stoel boven al de vorsten, die bij hem waren.
\par 2 En al de knechten des konings, die in de poort des konings waren, neigden en bogen zich neder voor Haman; want de koning had alzo van hem bevolen; maar Mordechai neigde zich niet, en boog zich niet neder.
\par 3 Toen zeiden de knechten des konings, die in de poort des konings waren, tot Mordechai: Waarom overtreedt gij des konings gebod?
\par 4 Het geschiedde nu, toen zij dit van dag tot dag tot hem zeiden, en hij naar hen niet hoorde, zo gaven zij het Haman te kennen, opdat zij zagen, of de woorden van Mordechai bestaan zouden; want hij had hun te kennen gegeven, dat hij een Jood was.
\par 5 Toen Haman zag, dat Mordechai zich niet neigde, noch zich voor hem nederboog, zo werd Haman vervuld met grimmigheid.
\par 6 Doch hij verachtte in zijn ogen, dat hij aan Mordechai alleen de hand zou slaan (want men had hem het volk van Mordechai aangewezen); maar Haman zocht al de Joden, die in het ganse koninkrijk van Ahasveros waren, namelijk het volk van Mordechai, te verdelgen.
\par 7 In de eerste maand (deze is de maand Nisan) in het twaalfde jaar van den koning Ahasveros, wierp men het Pur, dat is, het lot, voor Hamans aangezicht, van dag tot dag, en van maand tot maand, tot de twaalfde maand toe; deze is de maand Adar.
\par 8 Want Haman had tot den koning Ahasveros gezegd: Er is een volk, verstrooid en verdeeld onder de volken in al de landschappen uws koninkrijks; en hun wetten zijn verscheiden van de wetten aller volken; ook doen zij des konings wetten niet; daarom is het den koning niet oorbaar hen te laten blijven.
\par 9 Indien het den koning goeddunkt, laat er geschreven worden, dat men hen verdoe; zo zal ik tien duizend talenten zilvers opwegen in de handen dergenen, die het werk doen, om in des konings schatten te brengen.
\par 10 Toen trok de koning zijn ring van zijn hand, en hij gaf hem aan Haman, den zoon van Hammedatha, den Agagiet, der Joden tegenpartijder.
\par 11 En de koning zeide tot Haman: Dat zilver zij u geschonken, ook dat volk, om daarmede te doen, naar dat het goed is in uw ogen.
\par 12 Toen werden de schrijvers des konings geroepen, in de eerste maand, op den dertienden dag derzelve, en er werd geschreven naar alles, wat Haman beval, aan de stadhouders des konings, en aan de landvoogden, die over elk landschap waren, en aan de vorsten van elk volk, elk landschap naar zijn schrift, en elk volk naar zijn spraak; er werd geschreven in den naam van den koning Ahasveros, en het werd met des konings ring verzegeld.
\par 13 De brieven nu werden gezonden door de hand der lopers tot al de landschappen des konings, dat men zou verdelgen, doden en verdoen al de Joden, van den jonge tot den oude toe, de kleine kinderen en de vrouwen, op een dag, op den dertienden der twaalfde maand (deze is de maand Adar), en dat men hun buit zou roven.
\par 14 De inhoud van het schrift was, dat er een wet zou gegeven worden in alle landschappen, openbaar aan alle volken, dat zij tegen denzelfden dag zouden gereed zijn.
\par 15 De lopers gingen uit, voortgedrongen zijnde door het woord des konings, en de wet werd uitgegeven in den burg Susan. En de koning en Haman zaten en dronken, doch de stad Susan was verward.

\chapter{4}

\par 1 Als Mordechai wist al wat er geschied was, zo verscheurde Mordechai zijn klederen, en hij trok een zak aan met as; en hij ging uit door het midden der stad, en hij riep met een groot en bitter geroep.
\par 2 En hij kwam tot voor de poort des konings; want niemand mocht in des konings poort inkomen, bekleed met een zak.
\par 3 En in alle en een ieder landschap en plaats, waar het woord des konings en zijn wet aankwam, was een grote rouw onder de Joden, met vasten, en geween, en misbaar; vele lagen in zakken en as.
\par 4 Toen kwamen Esthers jonge dochters en haar kamerlingen, en zij gaven het haar te kennen; en het deed de koningin zeer wee; en zij zond klederen om Mordechai aan te doen, en zijn zak van hem af te doen; maar hij nam ze niet aan.
\par 5 Toen riep Esther Hatach, een van de kamerlingen des konings, welke hij voor haar gesteld had, en zij gaf hem bevel aan Mordechai, om te weten wat dit, en waarom dit ware.
\par 6 Als Hatach uitging tot Mordechai, op de straat der stad, die voor de poort des konings was,
\par 7 Zo gaf Mordechai hem te kennen al wat hem wedervaren was, en de verklaring van het zilver, hetwelk Haman gezegd had te zullen wegen in de schatten des konings, voor de Joden, om dezelve om te brengen.
\par 8 En hij gaf hem het afschrift der geschrevene wet, die te Susan gegeven was, om hen te verdelgen, dat hij het Esther liet zien, en haar te kennen gaf, en haar gebood, dat zij tot den koning ging, om hem te smeken, en van hem te verzoeken voor haar volk.
\par 9 Hatach nu kwam, en gaf Esther de woorden van Mordechai te kennen.
\par 10 Toen zeide Esther tot Hatach, en gaf hem bevel aan Mordechai:
\par 11 Alle knechten des konings, en het volk, der landschappen des konings, weten wel dat al wie tot den koning ingaat in het binnenste voorhof, die niet geroepen is, hij zij man of vrouw, zijn enig vonnis zij, dat men hem dode, tenzij dat de koning den gouden scepter hem toereike, opdat hij levend blijve; ik nu ben deze dertig dagen niet geroepen om tot den koning in te komen.
\par 12 En zij gaven de woorden van Esther aan Mordechai te kennen.
\par 13 Zo zeide Mordechai, dat men Esther wederom zeggen zou: Beeld u niet in, in uw ziel, dat gij zult ontkomen in het huis des konings, meer dan al de andere Joden.
\par 14 Want indien gij enigszins zwijgen zult te dezer tijd, zo zal den Joden verkwikking en verlossing uit een andere plaats ontstaan; maar gij en uws vaders huis zult omkomen; en wie weet, of gij niet om zulken tijd als deze is, tot dit koninkrijk geraakt zijt.
\par 15 Toen zeide Esther, dat men Mordechai weder aanzeggen zou:
\par 16 Ga, vergader al de Joden, die te Susan gevonden worden, en vast voor mij, en eet of drinkt niet, in drie dagen, nacht noch dag; ik en mijn jonge dochters zullen ook alzo vasten, en alzo zal ik tot den koning ingaan, hetwelk niet naar de wet is. Wanneer ik dan omkome, zo kom ik om.
\par 17 Toen ging Mordechai henen, en hij deed naar alles, wat Esther aan hem geboden had.

\chapter{5}

\par 1 Het geschiedde nu aan den derden dag, dat Esther een koninklijk kleed aantrok, en stond in het binnenste voorhof van des konings huis, tegenover het huis des konings; de koning nu zat op zijn koninklijken troon, in het koninklijke huis, tegenover de deur van het huis.
\par 2 En het geschiedde, toen de koning de koningin Esther zag, staande in het voorhof, verkreeg zij genade in zijn ogen, zodat de koning den gouden scepter, die in zijn hand was, Esther toereikte; en Esther naderde, en roerde de spits des scepters aan.
\par 3 Toen zeide de koning tot haar: Wat is u, koningin Esther! of wat is uw verzoek? Het zal u gegeven worden, ook tot de helft des koninkrijks.
\par 4 Esther nu zeide: Indien het den koning goeddunkt, zo kome de koning met Haman heden tot den maaltijd, dien ik hem bereid heb.
\par 5 Toen zeide de koning: Doet Haman spoeden, dat hij het bevel van Esther doe. Als nu de koning met Haman tot den maaltijd, dien Esther bereid had, gekomen was,
\par 6 Zo zeide de koning tot Esther op den maaltijd des wijns: Wat is uw bede? en zij zal u gegeven worden; en wat is uw verzoek? Het zal geschieden, ook tot de helft des koninkrijks.
\par 7 Toen antwoordde Esther, en zeide: Mijn bede en verzoek is:
\par 8 Indien ik genade gevonden heb in de ogen des konings, en indien het den koning goeddunkt, mij te geven mijn bede, en mijn verzoek te doen, zo kome de koning met Haman tot den maaltijd, dien ik hem bereiden zal; zo zal ik morgen doen naar het bevel des konings.
\par 9 Toen ging Haman ten zelfden dage uit, vrolijk en goedsmoeds; maar toen Haman Mordechai zag in de poort des konings, en dat hij niet opstond, noch zich voor hem bewoog, zo werd Haman vervuld met grimmigheid op Mordechai.
\par 10 Doch Haman bedwong zich, en hij kwam tot zijn huis; en hij zond henen, en liet zijn vrienden komen, en Zeres, zijn huisvrouw.
\par 11 En Haman vertelde hun de heerlijkheid zijns rijkdoms, en de veelheid zijner zonen, en alles, waarin de koning hem groot gemaakt had, en waarin hij hem verheven had boven de vorsten en knechten des konings.
\par 12 Verder zeide Haman: Ook heeft de koningin Esther niemand met den koning doen komen tot den maaltijd, dien zij bereid heeft, dan mij; en ik ben ook tegen morgen van haar met den koning genodigd.
\par 13 Doch dit alles baat mij niet, zo langen tijd als ik den Jood Mordechai zie zitten in de poort des konings.
\par 14 Toen zeide zijn huisvrouw Zeres tot hem, mitsgaders al zijn vrienden: Men make een galg, vijftig ellen hoog, en zeg morgen aan den koning, dat men Mordechai daaraan hange; ga dan vrolijk met den koning tot dien maaltijd. Deze raad nu dacht Haman goed, en hij deed de galg maken.

\chapter{6}

\par 1 In denzelfden nacht was de slaap van den koning geweken, en hij zeide, dat men het boek der gedachtenissen, de kronieken, brengen zou; en zij werden in de tegenwoordigheid des konings gelezen.
\par 2 En men vond geschreven, dat Mordechai had te kennen gegeven van Bigthana en Theres, twee kamerlingen des konings, uit de dorpelwachters, die de hand zochten te leggen aan den koning Ahasveros.
\par 3 Toen zeide de koning: Wat eer en verhoging is Mordechai hierover gedaan? En de jongelingen des konings, zijn dienaars, zeiden: Aan hem is niets gedaan.
\par 4 Toen zeide de koning: Wie is in het voorhof? (Haman nu was gekomen in het buitenvoorhof van het huis des konings, om den koning te zeggen, dat men Mordechai zou hangen aan de galg, die hij hem had doen bereiden.)
\par 5 En des konings jongelingen zeiden tot hem: Zie, Haman staat in het voorhof. Toen zeide de koning: Dat hij inkome.
\par 6 Als Haman ingekomen was, zo zeide de koning tot hem: Wat zal men met dien man doen, tot wiens eer de koning een welbehagen heeft? Toen zeide Haman in zijn hart: Tot wien heeft de koning een welbehagen, om hem eer te doen, meer dan tot mij?
\par 7 Daarom zeide Haman tot den koning: Den man, tot wiens eer de koning een welbehagen heeft,
\par 8 Zal men het koninklijke kleed brengen, dat de koning pleegt aan te trekken, en het paard, waarop de koning pleegt te rijden; en dat de koninklijke kroon op zijn hoofd gezet worde.
\par 9 En men zal dat kleed en dat paard geven in de hand van een uit de vorsten des konings, van de grootste heren, en men zal het dien man aantrekken, tot wiens eer de koning een welbehagen heeft; en men zal hem op dat paard doen rijden door de straten der stad, en men zal voor hem roepen: Alzo zal men dien man doen, tot wiens eer de koning een welbehagen heeft!
\par 10 Toen zeide de koning tot Haman: Haast u, neem dat kleed, en dat paard, gelijk als gij gesproken hebt, en doe alzo aan Mordechai, den Jood, dien aan de poort des konings zit; en laat niet een woord vallen van alles, wat gij gesproken hebt.
\par 11 En Haman nam dat kleed en dat paard, en trok het kleed Mordechai aan, en deed hem rijden door de straten der stad, en hij riep voor hem: Alzo zal men dien man doen, tot wiens eer de koning een welbehagen heeft!
\par 12 Daarna keerde Mordechai wederom tot de poort des konings; maar Haman werd voortgedreven naar zijn huis, treurig en met bedekten hoofde.
\par 13 En Haman vertelde aan zijn huisvrouw Zeres en al zijn vrienden al wat hem wedervaren was. Toen zeiden hem zijn wijzen, en Zeres, zijn huisvrouw: Indien Mordechai, voor wiens aangezicht gij hebt begonnen te vallen, van het zaad der Joden is, zo zult gij tegen hem niet vermogen; maar gij zult gewisselijk voor zijn aangezicht vallen.
\par 14 Toen zij nog met hem spraken, zo kwamen des konings kamerlingen nabij, en zij haastten Haman tot den maaltijd te brengen, dien Esther bereid had.

\chapter{7}

\par 1 Toen de koning met Haman gekomen was, om te drinken met de koningin Esther;
\par 2 Zo zeide de koning tot Esther, ook op den tweeden dag, op den maaltijd des wijns: Wat is uw bede, koningin Esther! en zij zal u gegeven worden; en wat is uw verzoek? Het zal geschieden, ook tot de helft des koninkrijks.
\par 3 Toen antwoordde de koningin Esther, en zeide: Indien ik, o koning, genade in uw ogen gevonden heb, en indien het den koning goeddunkt, men geve mij mijn leven, om mijner bede wil, en mijn volk, om mijns verzoeks wil.
\par 4 Want wij zijn verkocht, ik en mijn volk, dat men ons verdelge, dode en ombrenge. Indien wij nog tot knechten en tot dienstmaagden waren verkocht geweest, ik zou gezwegen hebben, ofschoon de onderdrukker de schade des konings geenszins zou kunnen vergoeden.
\par 5 Toen sprak de koning Ahasveros, en zeide tot de koningin Esther: Wie is die, en waar is diezelve, die zijn hart vervuld heeft, om alzo te doen?
\par 6 En Esther zeide: De man, de onderdrukker en vijand, is deze boze Haman! Toen verschrikte Haman voor het aangezicht des konings en der koningin.
\par 7 En de koning stond op in zijn grimmigheid van den maaltijd des wijns, en ging naar den hof van het paleis. En Haman bleef staan, om van de koningin Esther, aangaande zijn leven verzoek te doen; want hij zag, dat het kwaad van de koning over hem ten volle besloten was.
\par 8 Toen de koning wederkwam uit den hof van het paleis in het huis van den maaltijd des wijns, zo was Haman gevallen op het bed, waarop Esther was. Toen zeide de koning: Zou hij ook wel de koningin verkrachten bij mij in het huis? Het woord ging uit des konings mond, en zij bedekten Hamans aangezicht.
\par 9 En Charbona, een van de kamerlingen, voor het aanschijn des konings staande, zeide: Ook zie, de galg, welke Haman gemaakt heeft voor Mordechai, die goed voor den koning gesproken heeft, staat bij Hamans huis, vijftig ellen hoog. Toen zeide de koning: Hang hem daaraan.
\par 10 Alzo hingen zij Haman aan de galg, die hij voor Mordechai had doen bereiden; en de grimmigheid des konings werd gestild.

\chapter{8}

\par 1 Te dienzelfden dage gaf de koning Ahasveros aan de koningin Esther het huis van Haman, den vijand der Joden; en Mordechai kwam voor het aangezicht des konings, want Esther had te kennen gegeven, wat hij voor haar was.
\par 2 En de koning toog zijn ring af, dien hij van Haman genomen had, en gaf hem aan Mordechai; en Esther stelde Mordechai over het huis van Haman.
\par 3 En Esther sprak verder voor het aangezicht des konings, en zij viel voor zijn voeten, en zij weende, en zij smeekte hem, dat hij de boosheid van Haman, den Agagiet, en zijn gedachte, die hij tegen de Joden gedacht had, zou wegnemen.
\par 4 De koning nu reikte den gouden scepter Esther toe. Toen rees Esther op, en zij stond voor het aangezicht des konings.
\par 5 En zij zeide: Indien het den koning goeddunkt, en indien ik genade voor zijn aangezicht gevonden heb, en deze zaak voor den koning recht is, en ik in zijn ogen aangenaam ben, dat er geschreven worde, dat de brieven en de gedachte van Haman, den zoon van Hammedatha, den Agagiet, wederroepen worden, welke hij geschreven heeft, om de Joden om te brengen, die in al de landschappen des konings zijn.
\par 6 Want hoe zal ik vermogen, dat ik aanzie het kwaad, dat mijn volk treffen zal? En hoe zal ik vermogen, dat ik aanzie het verderf van mijn geslacht?
\par 7 Toen zeide de koning Ahasveros tot de koningin Esther en tot Mordechai, den Jood: Ziet, het huis van Haman heb ik Esther gegeven, en hem heeft men aan de galg gehangen, omdat hij zijn hand aan de Joden geslagen had.
\par 8 Schrijft dan gijlieden voor de Joden, zoals het goed is in uw ogen, in des konings naam, en verzegelt het met des konings ring; want het schrift, dat in des konings naam geschreven, en met des konings ring verzegeld is, is niet te wederroepen.
\par 9 Toen werden des konings schrijvers geroepen, ter zelfder tijd, in de derde maand (zij is de maand Sivan), op den drie en twintigsten derzelve, en er werd geschreven naar alles, wat Mordechai gebood, aan de Joden, en aan de stadhouders, en landvoogden, en oversten der landschappen, die van Indie af tot aan Morenland strekken, honderd zeven en twintig landschappen, een ieder landschap naar zijn schrift, een ieder volk naar zijn spraak; ook aan de Joden naar hun schrift en naar hun spraak.
\par 10 En men schreef in den naam van den koning Ahasveros, en men verzegelde het met des konings ring; en men zond de brieven door de hand der lopers te paard, rijdende op snelle kemelen, op muildieren, van merrien geteeld;
\par 11 Dat de koning den Joden toeliet, die in elke stad waren, zich te vergaderen, en voor hun leven te staan, om te verdelgen, om te doden en om om te brengen alle macht des volks en des landschaps, die hen benauwen zou, de kleine kinderen en de vrouwen, en hun buit te roven;
\par 12 Op een dag in al de landschappen van den koning Ahasveros, op den dertienden der twaalfde maand; deze is de maand Adar.
\par 13 De inhoud van dit geschrift was: dat een wet zou gegeven worden in alle landschappen, openbaar aan alle volken; en dat de Joden gereed zouden zijn tegen dien dag, om zich te wreken aan hun vijanden.
\par 14 De lopers, die op snelle kemelen reden en op muildieren, togen snellijk uit, aangedreven zijnde door het woord des konings. Deze wet nu werd gegeven op den burg Susan.
\par 15 En Mordechai ging uit van voor het aangezicht des konings in een hemelsblauw en wit koninklijk kleed, en met een grote gouden kroon, en met een opperkleed van fijn linnen en purper; en de stad Susan juichte en was vrolijk.
\par 16 Bij de Joden was licht, en blijdschap, en vreugde, en eer;
\par 17 Ook in alle en een ieder landschap, en in alle en een iedere stad, ter plaatse, waar des konings woord en zijn wet aankwam, daar was bij de Joden blijdschap en vreugde, maaltijden en vrolijke dagen; en velen uit de volken des lands werden Joden, want de vreze der Joden was op hen gevallen.

\chapter{9}

\par 1 In de twaalfde maand nu (dezelve is de maand Adar), op den dertienden dag derzelve, toen des konings woord en zijn wet nabij gekomen was, dat men het doen zou, ten dage, als de vijanden der Joden hoopten over hen te heersen, zo is het omgekeerd, want de Joden heersten zelven over hun haters.
\par 2 Want de Joden vergaderden zich in hun steden, in al de landschappen van den koning Ahasveros, om de hand te slaan aan degenen, die hun verderf zochten; en niemand bestond voor hen, want hunlieder schrik was op al die volken gevallen.
\par 3 En al de oversten der landschappen, en de stadhouders, en landvoogden, en die het werk des konings deden, verhieven de Joden; want de vreze van Mordechai was op hen gevallen.
\par 4 Want Mordechai was groot in het huis des konings, en zijn gerucht ging uit door alle landschappen; want die man, Mordechai, werd doorgaans groter.
\par 5 De Joden nu sloegen op al hun vijanden, met den slag des zwaards, en der doding, en der verderving; en zij deden met hun haters naar hun welbehagen.
\par 6 En in den burg Susan hebben de Joden gedood en omgebracht vijfhonderd mannen.
\par 7 En Parsandatha, en Dalfon, en Asfata,
\par 8 En Poratha, en Adalia, en Aridatha,
\par 9 En Parmastha, en Arisai, en Aridai, en Vaizatha,
\par 10 De tien zonen van Haman, den zoon van Hammedatha, den vijand der Joden, doodden zij; maar zij sloegen hun handen niet aan den roof.
\par 11 Ten zelfden dage kwam voor den koning het getal der gedoden op den burg Susan.
\par 12 En de koning zeide tot de koningin Esther: Te Susan op den burg hebben de Joden gedood en omgebracht vijfhonderd mannen en de tien zonen van Haman; wat hebben zij in al de andere landschappen des konings gedaan? Wat is nu uw bede? en het zal u gegeven worden; of wat is verder uw verzoek? het zal geschieden.
\par 13 Toen zeide Esther: Dunkt het den koning goed, men late ook morgen den Joden, die te Susan zijn, toe, te doen naar het gebod van heden; en men hange de tien zonen van Haman aan de galg.
\par 14 Toen zeide de koning, dat men alzo doen zou; en er werd een gebod gegeven te Susan, en men hing de tien zonen van Haman op.
\par 15 En de Joden, die te Susan waren, vergaderden ook op den veertienden dag der maand Adar, en zij doodden te Susan driehonderd mannen; maar zij sloegen hun hand niet aan den roof.
\par 16 De overige Joden nu, die in de landschappen des konings waren, vergaderden, opdat zij stonden voor hun leven, en rust hadden van hun vijanden, en zij doodden onder hun haters vijf en zeventig duizend; maar zij sloegen hun hand niet aan den roof.
\par 17 Dit geschiedde op den dertienden dag der maand Adar; en op den veertienden derzelve rustten zij, en zij maakten denzelven een dag der maaltijden en der vreugde.
\par 18 En de Joden, die te Susan waren, vergaderden op den dertienden derzelve, en op den veertienden derzelve; en zij rustten op den vijftienden derzelve, en zij maakten denzelven een dag der maaltijden en der vreugde.
\par 19 Daarom maakten de Joden van de dorpen, die in de dorpsteden woonden, den veertienden dag der maand Adar ter vreugde en maaltijden, en een vrolijken dag, en der zending van delen aan elkander.
\par 20 En Mordechai beschreef deze geschiedenissen; en hij zond brieven aan al de Joden, die in al de landschappen van den koning Ahasveros waren, dien, die nabij, en dien, die verre waren,
\par 21 Om over hen te bevestigen, dat zij zouden onderhouden den veertienden dag der maand Adar, en den vijftienden dag derzelve, in alle en in ieder jaar;
\par 22 Naar de dagen, in dewelke de Joden tot rust gekomen waren van hun vijanden, en de maand, die hun veranderd was van droefenis in blijdschap, en van rouw in een vrolijken dag; dat zij dezelve dagen maken zouden tot dagen der maaltijden, en der vreugde, en der zending van delen aan elkander, en der gaven aan de armen.
\par 23 En de Joden namen aan te doen, wat zij begonnen hadden, en dat Mordechai aan hen geschreven had.
\par 24 Omdat Haman, de zoon van Hammedatha, den Agagiet, aller Joden vijand, tegen de Joden gedacht had hen om te brengen; en dat hij het Pur, dat is, het lot had geworpen, om hen te verslaan, en om hen om te brengen.
\par 25 Maar als zij voor den koning gekomen was, heeft hij door brieven bevolen, dat zijn boze gedachte, die hij gedacht had over de Joden, op zijn hoofd zou wederkeren; en men heeft hem en zijn zonen aan de galg gehangen.
\par 26 Daarom noemt men die dagen Purim, van den naam van dat Pur. Hierom, vanwege al de woorden van dien brief, en hetgeen zij zelven daarvan gezien hadden, en wat tot hen overgekomen was,
\par 27 Bevestigden de Joden, en namen op zich en op hun zaad, en op allen, die zich tot hen vervoegen zouden, dat men het niet overtrade, dat zij deze twee dagen zouden houden, naar het voorschrift derzelve, en naar den bestemden tijd derzelve, in alle en ieder jaar;
\par 28 Dat deze dagen gedacht zouden worden en onderhouden, in alle en elk geslacht, elk huisgezin, elk landschap en elke stad; en dat deze dagen van Purim niet zouden overtreden worden onder de Joden, en dat de gedachtenis derzelve geen einde nemen zou bij hun zaad.
\par 29 Daarna schreef de koningin Esther, de dochter van Abichail, en Mordechai, de Jood, met alle macht, om dezen brief van Purim ten tweeden male te bevestigen.
\par 30 En hij zond de brieven aan al de Joden, in de honderd zeven en twintig landschappen van het koninkrijk van Ahasveros, met woorden van vrede en trouw;
\par 31 Dat zij deze dagen van Purim bevestigen zouden op hun bestemde tijden, gelijk als Mordechai, de Jood, over hen bevestigd had, en Esther, de koningin, en gelijk als zij het bevestigd hadden voor zichzelven en voor hun zaad; de zaken van het vasten en hunlieder geroep.
\par 32 En het bevel van Esther bevestigde de geschiedenissen van deze Purim, en het werd in een boek geschreven.

\chapter{10}

\par 1 Daarna leide de koning Ahasveros schatting op het land, en de eilanden der zee.
\par 2 Al de werken nu zijner macht en zijns gewelds, en de verklaring der grootheid van Mordechai, denwelken de koning groot gemaakt heeft, zijn die niet geschreven in het boek der kronieken der koningen van Medie en Perzie?
\par 3 Want de Jood Mordechai was de tweede bij den koning Ahasveros, en groot bij de Joden, en aangenaam bij de menigte zijner broederen, zoekende het beste voor zijn volk, en sprekende voor den welstand van zijn ganse zaad.



\end{document}