\begin{document}

\title{Der Prediger Salomo}


\chapter{1}

\par 1 Dies sind die Reden des Predigers, des Sohnes Davids, des Königs zu Jerusalem.
\par 2 Es ist alles ganz eitel, sprach der Prediger, es ist alles ganz eitel.
\par 3 Was hat der Mensch für Gewinn von aller seiner Mühe, die er hat unter der Sonne?
\par 4 Ein Geschlecht vergeht, das andere kommt; die Erde aber bleibt ewiglich.
\par 5 Die Sonne geht auf und geht unter und läuft an ihren Ort, daß sie wieder daselbst aufgehe.
\par 6 Der Wind geht gen Mittag und kommt herum zur Mitternacht und wieder herum an den Ort, da er anfing.
\par 7 Alle Wasser laufen ins Meer, doch wird das Meer nicht voller; an den Ort, da sie her fließen, fließen sie wieder hin.
\par 8 Es sind alle Dinge so voll Mühe, daß es niemand ausreden kann. Das Auge sieht sich nimmer satt, und das Ohr hört sich nimmer satt.
\par 9 Was ist's, das geschehen ist? Eben das hernach geschehen wird. Was ist's, das man getan hat? Eben das man hernach tun wird; und geschieht nichts Neues unter der Sonne.
\par 10 Geschieht auch etwas, davon man sagen möchte: Siehe, das ist neu? Es ist zuvor auch geschehen in den langen Zeiten, die vor uns gewesen sind.
\par 11 Man gedenkt nicht derer, die zuvor gewesen sind; also auch derer, so hernach kommen, wird man nicht gedenken bei denen, die darnach sein werden.
\par 12 Ich, der Prediger, war König zu Jerusalem
\par 13 und richtete mein Herz zu suchen und zu forschen weislich alles, was man unter dem Himmel tut. Solche unselige Mühe hat Gott den Menschenkindern gegeben, daß sie sich darin müssen quälen.
\par 14 Ich sah an alles Tun, das unter der Sonne geschieht; und siehe, es war alles eitel und Haschen nach dem Wind.
\par 15 Krumm kann nicht schlicht werden noch, was fehlt, gezählt werden.
\par 16 Ich sprach in meinem Herzen: Siehe, ich bin herrlich geworden und habe mehr Weisheit denn alle, die vor mir gewesen sind zu Jerusalem, und mein Herz hat viel gelernt und erfahren.
\par 17 Und richtete auch mein Herz darauf, daß ich erkennte Weisheit und erkennte Tollheit und Torheit. Ich ward aber gewahr, daß solches auch Mühe um Wind ist.
\par 18 Denn wo viel Weisheit ist, da ist viel Grämens; und wer viel lernt, der muß viel leiden.

\chapter{2}

\par 1 Ich sprach in meinem Herzen: Wohlan, ich will wohl leben und gute Tage haben! Aber siehe, das war auch eitel.
\par 2 Ich sprach zum Lachen: Du bist toll! und zur Freude: Was machst du?
\par 3 Da dachte ich in meinem Herzen, meinen Leib mit Wein zu pflegen, doch also, daß mein Herz mich mit Weisheit leitete, und zu ergreifen, was Torheit ist, bis ich lernte, was dem Menschen gut wäre, daß sie tun sollten, solange sie unter dem Himmel leben.
\par 4 Ich tat große Dinge: ich baute Häuser, pflanzte Weinberge;
\par 5 ich machte mir Gärten und Lustgärten und pflanzte allerlei fruchtbare Bäume darein;
\par 6 ich machte mir Teiche, daraus zu wässern den Wald der grünenden Bäume;
\par 7 ich hatte Knechte und Mägde und auch Gesinde, im Hause geboren; ich hatte eine größere Habe an Rindern und Schafen denn alle, die vor mir zu Jerusalem gewesen waren;
\par 8 ich sammelte mir auch Silber und Gold und von den Königen und Ländern einen Schatz; ich schaffte mir Sänger und Sängerinnen und die Wonne der Menschen, allerlei Saitenspiel;
\par 9 und nahm zu über alle, die vor mir zu Jerusalem gewesen waren; auch blieb meine Weisheit bei mir;
\par 10 und alles, was meine Augen wünschten, das ließ ich ihnen und wehrte meinem Herzen keine Freude, daß es fröhlich war von aller meiner Arbeit; und das hielt ich für mein Teil von aller meiner Arbeit.
\par 11 Da ich aber ansah alle meine Werke, die meine Hand gemacht hatte, und die Mühe, die ich gehabt hatte, siehe, da war es alles eitel und Haschen nach dem Wind und kein Gewinn unter der Sonne.
\par 12 Da wandte ich mich, zu sehen die Weisheit und die Tollheit und Torheit. Denn wer weiß, was der für ein Mensch werden wird nach dem König, den sie schon bereit gemacht haben?
\par 13 Da ich aber sah, daß die Weisheit die Torheit übertraf wie das Licht die Finsternis;
\par 14 daß dem Weisen seine Augen im Haupt stehen, aber die Narren in der Finsternis gehen; und merkte doch, daß es einem geht wie dem andern.
\par 15 Da dachte ich in meinem Herzen: Weil es denn mir geht wie dem Narren, warum habe ich denn nach Weisheit getrachtet? Da dachte ich in meinem Herzen, daß solches auch eitel sei.
\par 16 Denn man gedenkt des Weisen nicht immerdar, ebenso wenig wie des Narren, und die künftigen Tage vergessen alles; und wie der Narr stirbt, also auch der Weise.
\par 17 Darum verdroß mich zu leben; denn es gefiel mir übel, was unter der Sonne geschieht, daß alles eitel ist und Haschen nach dem Wind.
\par 18 Und mich verdroß alle meine Arbeit, die ich unter der Sonne hatte, daß ich dieselbe einem Menschen lassen müßte, der nach mir sein sollte.
\par 19 Denn wer weiß, ob er weise oder toll sein wird? und soll doch herrschen in aller meiner Arbeit, die ich weislich getan habe unter der Sonne. Das ist auch eitel.
\par 20 Darum wandte ich mich, daß mein Herz abließe von aller Arbeit, die ich tat unter der Sonne.
\par 21 Denn es muß ein Mensch, der seine Arbeit mit Weisheit, Vernunft und Geschicklichkeit getan hat, sie einem andern zum Erbteil lassen, der nicht daran gearbeitet hat. Das ist auch eitel und ein großes Unglück.
\par 22 Denn was kriegt der Mensch von aller seiner Arbeit und Mühe seines Herzens, die er hat unter der Sonne?
\par 23 Denn alle seine Lebtage hat er Schmerzen mit Grämen und Leid, daß auch sein Herz des Nachts nicht ruht. Das ist auch eitel.
\par 24 Ist's nun nicht besser dem Menschen, daß er esse und trinke und seine Seele guter Dinge sei in seiner Arbeit? Aber solches sah ich auch, daß es von Gottes Hand kommt.
\par 25 Denn wer kann fröhlich essen und sich ergötzen ohne ihn?
\par 26 Denn dem Menschen, der ihm gefällt, gibt er Weisheit, Vernunft und Freude; aber dem Sünder gibt er Mühe, daß er sammle und häufe, und es doch dem gegeben werde, der Gott gefällt. Darum ist das auch eitel und Haschen nach dem Wind.

\chapter{3}

\par 1 Ein jegliches hat seine Zeit, und alles Vornehmen unter dem Himmel hat seine Stunde.
\par 2 Geboren werden und sterben, pflanzen und ausrotten, was gepflanzt ist,
\par 3 würgen und heilen, brechen und bauen,
\par 4 weinen und lachen, klagen und tanzen,
\par 5 Stein zerstreuen und Steine sammeln, herzen und ferne sein von Herzen,
\par 6 suchen und verlieren, behalten und wegwerfen,
\par 7 zerreißen und zunähen, schweigen und reden,
\par 8 lieben und hassen, Streit und Friede hat seine Zeit.
\par 9 Man arbeite, wie man will, so hat man doch keinen Gewinn davon.
\par 10 Ich sah die Mühe, die Gott den Menschen gegeben hat, daß sie darin geplagt werden.
\par 11 Er aber tut alles fein zu seiner Zeit und läßt ihr Herz sich ängsten, wie es gehen solle in der Welt; denn der Mensch kann doch nicht treffen das Werk, das Gott tut, weder Anfang noch Ende.
\par 12 Darum merkte ich, daß nichts Besseres darin ist denn fröhlich sein und sich gütlich tun in seinem Leben.
\par 13 Denn ein jeglicher Mensch, der da ißt und trinkt und hat guten Mut in aller seiner Arbeit, das ist eine Gabe Gottes.
\par 14 Ich merkte, daß alles, was Gott tut, das besteht immer: man kann nichts dazutun noch abtun; und solches tut Gott, daß man sich vor ihm fürchten soll.
\par 15 Was geschieht, das ist zuvor geschehen, und was geschehen wird, ist auch zuvor geschehen; und Gott sucht wieder auf, was vergangen ist.
\par 16 Weiter sah ich unter der Sonne Stätten des Gerichts, da war ein gottlos Wesen, und Stätten der Gerechtigkeit, da waren Gottlose.
\par 17 Da dachte ich in meinem Herzen: Gott muß richten den Gerechten und den Gottlosen; denn es hat alles Vornehmen seine Zeit und alle Werke.
\par 18 Ich sprach in meinem Herzen: Es geschieht wegen der Menschenkinder, auf daß Gott sie prüfe und sie sehen, daß sie an sich selbst sind wie das Vieh.
\par 19 Denn es geht dem Menschen wie dem Vieh: wie dies stirbt, so stirbt er auch, und haben alle einerlei Odem, und der Mensch hat nichts mehr als das Vieh; denn es ist alles eitel.
\par 20 Es fährt alles an einen Ort; es ist alles von Staub gemacht und wird wieder zu Staub.
\par 21 Wer weiß, ob der Odem der Menschen aufwärts fahre und der Odem des Viehes abwärts unter die Erde fahre?
\par 22 So sah ich denn, daß nichts Besseres ist, als daß ein Mensch fröhlich sei in seiner Arbeit; denn das ist sein Teil. Denn wer will ihn dahin bringen, daß er sehe, was nach ihm geschehen wird?

\chapter{4}

\par 1 Ich wandte mich um und sah an alles Unrecht, das geschah unter der Sonne; und siehe, da waren die Tränen derer, so Unrecht litten und hatten keinen Tröster; und die ihnen Unrecht taten, waren zu mächtig, daß sie keinen Tröster haben konnten.
\par 2 Da lobte ich die Toten, die schon gestorben waren, mehr denn die Lebendigen, die noch das Leben hatten;
\par 3 und besser als alle beide ist, der noch nicht ist und des Bösen nicht innewird, das unter der Sonne geschieht.
\par 4 Ich sah an Arbeit und Geschicklichkeit in allen Sachen; da neidet einer den andern. Das ist auch eitel und Haschen nach dem Wind.
\par 5 Ein Narr schlägt die Finger ineinander und verzehrt sich selbst.
\par 6 Es ist besser eine Handvoll mit Ruhe denn beide Fäuste voll mit Mühe und Haschen nach Wind.
\par 7 Ich wandte mich um und sah die Eitelkeit unter der Sonne.
\par 8 Es ist ein einzelner, und nicht selbander, und hat weder Kind noch Bruder; doch ist seines Arbeitens kein Ende, und seine Augen werden Reichtums nicht satt. Wem arbeite ich doch und breche meiner Seele ab? Das ist auch eitel und eine böse Mühe.
\par 9 So ist's ja besser zwei als eins; denn sie genießen doch ihrer Arbeit wohl.
\par 10 Fällt ihrer einer so hilft ihm sein Gesell auf. Weh dem, der allein ist! Wenn er fällt, so ist keiner da, der ihm aufhelfe.
\par 11 Auch wenn zwei beieinander liegen, wärmen sie sich; wie kann ein einzelner warm werden?
\par 12 Einer mag überwältigt werden, aber zwei mögen widerstehen; und eine dreifältige Schnur reißt nicht leicht entzwei.
\par 13 Ein armes Kind, das weise ist, ist besser denn ein alter König, der ein Narr ist und weiß nicht sich zu hüten.
\par 14 Es kommt einer aus dem Gefängnis zum Königreich; und einer, der in seinem Königreich geboren ist, verarmt.
\par 15 Und ich sah, daß alle Lebendigen unter der Sonne wandelten bei dem andern, dem Kinde, das an jenes Statt sollte aufkommen.
\par 16 Und des Volks, das vor ihm ging, war kein Ende und des, das ihm nachging; und wurden sein doch nicht froh. Das ist auch eitel und Mühe um Wind.

\chapter{5}

\par 1 Bewahre deinen Fuß, wenn du zum Hause Gottes gehst, und komme, daß du hörst. Das ist besser als der Narren Opfer; denn sie wissen nicht, was sie Böses tun.
\par 2 Sei nicht schnell mit deinem Munde und laß dein Herz nicht eilen, was zu reden vor Gott; denn Gott ist im Himmel, und du auf Erden; darum laß deiner Worte wenig sein.
\par 3 Denn wo viel Sorgen ist, da kommen Träume; und wo viel Worte sind, da hört man den Narren.
\par 4 Wenn du Gott ein Gelübde tust, so verzieh nicht, es zu halten; denn er hat kein Gefallen an den Narren. Was du gelobst, das halte.
\par 5 Es ist besser, du gelobst nichts, denn daß du nicht hältst, was du gelobst.
\par 6 Laß deinem Mund nicht zu, daß er dein Fleisch verführe; und sprich vor dem Engel nicht: Es ist ein Versehen. Gott möchte erzürnen über deine Stimme und verderben alle Werke deiner Hände.
\par 7 Wo viel Träume sind, da ist Eitelkeit und viel Worte; aber fürchte du Gott.
\par 8 Siehst du dem Armen Unrecht tun und Recht und Gerechtigkeit im Lande wegreißen, wundere dich des Vornehmens nicht; denn es ist ein hoher Hüter über den Hohen und sind noch Höhere über die beiden.
\par 9 Und immer ist's Gewinn für ein Land, wenn ein König da ist für das Feld, das man baut.
\par 10 Wer Geld liebt, wird Geldes nimmer satt; und wer Reichtum liebt, wird keinen Nutzen davon haben. Das ist auch eitel.
\par 11 Denn wo viel Guts ist, da sind viele, die es essen; und was genießt davon, der es hat, außer daß er's mit Augen ansieht?
\par 12 Wer arbeitet, dem ist der Schaf süß, er habe wenig oder viel gegessen; aber die Fülle des Reichen läßt ihn nicht schlafen.
\par 13 Es ist ein böses Übel, das ich sah unter der Sonne: Reichtum, behalten zum Schaden dem, der ihn hat.
\par 14 Denn der Reiche kommt um mit großem Jammer; und so er einen Sohn gezeugt hat, dem bleibt nichts in der Hand.
\par 15 Wie er nackt ist von seine Mutter Leibe gekommen, so fährt er wieder hin, wie er gekommen ist, und nimmt nichts mit sich von seiner Arbeit in seiner Hand, wenn er hinfährt.
\par 16 Das ist ein böses Übel, daß er hinfährt, wie er gekommen ist. Was hilft's ihm denn, daß er in den Wind gearbeitet hat?
\par 17 Sein Leben lang hat er im Finstern gegessen und in großem Grämen und Krankheit und Verdruß.
\par 18 So sehe ich nun das für gut an, daß es fein sei, wenn man ißt und trinkt und gutes Muts ist in aller Arbeit, die einer tut unter der Sonne sein Leben lang, das Gott ihm gibt; denn das ist sein Teil.
\par 19 Denn welchem Menschen Gott Reichtum und Güter gibt und die Gewalt, daß er davon ißt und trinkt für sein Teil und fröhlich ist in seiner Arbeit, das ist eine Gottesgabe.
\par 20 Denn er denkt nicht viel an die Tage seines Lebens, weil Gott sein Herz erfreut.

\chapter{6}

\par 1 Es ist ein Unglück, das ich sah unter der Sonne, und ist gemein bei den Menschen:
\par 2 einer, dem Gott Reichtum, Güter und Ehre gegeben hat und mangelt ihm keins, das sein Herz begehrt; und Gott gibt doch ihm nicht Macht, es zu genießen, sondern ein anderer verzehrt es; das ist eitel und ein böses Übel.
\par 3 Wenn einer gleich hundert Kinder zeugte und hätte langes Leben, daß er viele Jahre überlebte, und seine Seele sättigte sich des Guten nicht und bliebe ohne Grab, von dem spreche ich, daß eine unzeitige Geburt besser sei denn er.
\par 4 Denn in Nichtigkeit kommt sie, und in Finsternis fährt sie dahin, und ihr Name bleibt in Finsternis bedeckt,
\par 5 auch hat sie die Sonne nicht gesehen noch gekannt; so hat sie mehr Ruhe denn jener.
\par 6 Ob er auch zweitausend Jahre lebte, und genösse keines Guten: kommt's nicht alles an einen Ort?
\par 7 Alle Arbeit des Menschen ist für seinen Mund; aber doch wird die Seele nicht davon satt.
\par 8 Denn was hat ein Weiser mehr als ein Narr? Was hilft's den Armen, daß er weiß zu wandeln vor den Lebendigen?
\par 9 Es ist besser, das gegenwärtige Gut gebrauchen, denn nach anderm gedenken. Das ist auch Eitelkeit und Haschen nach Wind.
\par 10 Was da ist, des Name ist zuvor genannt, und es ist bestimmt, was ein Mensch sein wird; und er kann nicht hadern mit dem, der ihm zu mächtig ist.
\par 11 Denn es ist des eitlen Dinges zuviel; was hat ein Mensch davon?
\par 12 Denn wer weiß, was dem Menschen nütze ist im Leben, solange er lebt in seiner Eitelkeit, welches dahinfährt wie ein Schatten? Oder wer will dem Menschen sagen, was nach ihm kommen wird unter der Sonne?

\chapter{7}

\par 1 Ein guter Ruf ist besser denn gute Salbe, und der Tag des Todes denn der Tag der Geburt.
\par 2 Es ist besser in das Klagehaus gehen, denn in ein Trinkhaus; in jenem ist das Ende aller Menschen, und der Lebendige nimmt's zu Herzen.
\par 3 Es ist Trauern besser als Lachen; denn durch Trauern wird das Herz gebessert.
\par 4 Das Herz der Weisen ist im Klagehause, und das Herz der Narren im Hause der Freude.
\par 5 Es ist besser hören das Schelten der Weisen, denn hören den Gesang der Narren.
\par 6 Denn das Lachen der Narren ist wie das Krachen der Dornen unter den Töpfen; und das ist auch eitel.
\par 7 Ein Widerspenstiger macht einen Weisen unwillig und verderbt ein mildtätiges Herz.
\par 8 Das Ende eines Dinges ist besser denn sein Anfang. Ein geduldiger Geist ist besser denn ein hoher Geist.
\par 9 Sei nicht schnellen Gemütes zu zürnen; denn Zorn ruht im Herzen eines Narren.
\par 10 Sprich nicht: Was ist's, daß die vorigen Tage besser waren als diese? denn du fragst solches nicht weislich.
\par 11 Weisheit ist gut mit einem Erbgut und hilft, daß sich einer der Sonne freuen kann.
\par 12 Denn die Weisheit beschirmt, so beschirmt Geld auch; aber die Weisheit gibt das Leben dem, der sie hat.
\par 13 Siehe an die Werke Gottes; denn wer kann das schlicht machen, was er krümmt?
\par 14 Am guten Tage sei guter Dinge, und den bösen Tag nimm auch für gut; denn diesen schafft Gott neben jenem, daß der Mensch nicht wissen soll, was künftig ist.
\par 15 Allerlei habe ich gesehen in den Tagen meiner Eitelkeit. Da ist ein Gerechter, und geht unter mit seiner Gerechtigkeit; und ein Gottloser, der lange lebt in seiner Bosheit.
\par 16 Sei nicht allzu gerecht und nicht allzu weise, daß du dich nicht verderbest.
\par 17 Sei nicht allzu gottlos und narre nicht, daß du nicht sterbest zur Unzeit.
\par 18 Es ist gut, daß du dies fassest und jenes auch nicht aus deiner Hand lässest; denn wer Gott fürchtet, der entgeht dem allem.
\par 19 Die Weisheit stärkt den Weisen mehr denn zehn Gewaltige, die in der Stadt sind.
\par 20 Denn es ist kein Mensch so gerecht auf Erden, daß er Gutes tue und nicht sündige.
\par 21 Gib auch nicht acht auf alles, was man sagt, daß du nicht hören müssest deinen Knecht dir fluchen.
\par 22 Denn dein Herz weiß, daß du andern oftmals geflucht hast.
\par 23 Solches alles habe ich versucht mit Weisheit. Ich gedachte, ich will weise sein; sie blieb aber ferne von mir.
\par 24 Alles, was da ist, das ist ferne und sehr tief; wer will's finden?
\par 25 Ich kehrte mein Herz, zu erfahren und erforschen und zu suchen Weisheit und Kunst, zu erfahren der Gottlosen Torheit und Irrtum der Tollen,
\par 26 und fand, daß bitterer sei denn der Tod ein solches Weib, dessen Herz Netz und Strick ist und deren Hände Bande sind. Wer Gott gefällt, der wird ihr entrinnen; aber der Sünder wird durch sie gefangen.
\par 27 Schau, das habe ich gefunden, spricht der Prediger, eins nach dem andern, daß ich Erkenntnis fände.
\par 28 Und meine Seele sucht noch und hat's nicht gefunden: unter tausend habe ich einen Mann gefunden; aber ein Weib habe ich unter den allen nicht gefunden.
\par 29 Allein schaue das: ich habe gefunden, daß Gott den Menschen hat aufrichtig gemacht; aber sie suchen viele Künste.

\chapter{8}

\par 1 Wer ist wie der Weise, und wer kann die Dinge auslegen? Die Weisheit des Menschen erleuchtet sein Angesicht; aber ein freches Angesicht wird gehaßt.
\par 2 Halte das Wort des Königs und den Eid Gottes.
\par 3 Eile nicht zu gehen von seinem Angesicht, und bleibe nicht in böser Sache; denn er tut, was er will.
\par 4 In des Königs Wort ist Gewalt; und wer mag zu ihm sagen: Was machst du?
\par 5 Wer das Gebot hält, der wird nichts Böses erfahren; aber eines Weisen Herz weiß Zeit und Weise.
\par 6 Denn ein jeglich Vornehmen hat seine Zeit und Weise; denn des Unglücks des Menschen ist viel bei ihm.
\par 7 Denn er weiß nicht, was geschehen wird; und wer soll ihm sagen, wie es werden soll?
\par 8 Ein Mensch hat nicht Macht über den Geist, den Geist zurückzuhalten, und hat nicht Macht über den Tag des Todes, und keiner wird losgelassen im Streit; und das gottlose Wesen errettet den Gottlosen nicht.
\par 9 Das habe ich alles gesehen, und richtete mein Herz auf alle Werke, die unter der Sonne geschehen. Ein Mensch herrscht zuzeiten über den andern zu seinem Unglück.
\par 10 Und da sah ich Gottlose, die begraben wurden und zur Ruhe kamen; aber es wandelten hinweg von heiliger Stätte und wurden vergessen in der Stadt die, so recht getan hatten. Das ist auch eitel.
\par 11 Weil nicht alsbald geschieht ein Urteil über die bösen Werke, dadurch wird das Herz der Menschen voll, Böses zu tun.
\par 12 Ob ein Sünder hundertmal Böses tut und lange lebt, so weiß ich doch, daß es wohl gehen wird denen, die Gott fürchten, die sein Angesicht scheuen.
\par 13 Aber dem Gottlosen wird es nicht wohl gehen; und wie ein Schatten werden nicht lange leben, die sich vor Gott nicht fürchten.
\par 14 Es ist eine Eitelkeit, die auf Erden geschieht: es sind Gerechte, denen geht es als hätten sie Werke der Gottlosen, und sind Gottlose, denen geht es als hätten sie Werke der Gerechten. Ich sprach: Das ist auch eitel.
\par 15 Darum lobte ich die Freude, daß der Mensch nichts Besseres hat unter der Sonne denn essen und trinken und fröhlich sein; und solches werde ihm von der Arbeit sein Leben lang, das ihm Gott gibt unter der Sonne.
\par 16 Ich gab mein Herz, zu wissen die Weisheit und zu schauen die Mühe, die auf Erden geschieht, daß auch einer weder Tag noch Nacht den Schlaf sieht mit seinen Augen.
\par 17 Und ich sah alle Werke Gottes, daß ein Mensch das Werk nicht finden kann, das unter der Sonne geschieht; und je mehr der Mensch arbeitet, zu suchen, je weniger er findet. Wenn er gleich spricht: "Ich bin weise und weiß es", so kann er's doch nicht finden.

\chapter{9}

\par 1 Denn ich habe solches alles zu Herzen genommen, zu forschen das alles, daß Gerechte und Weise und ihre Werke sind in Gottes Hand; kein Mensch kennt weder die Liebe noch den Haß irgend eines, den er vor sich hat.
\par 2 Es begegnet dasselbe einem wie dem andern: dem Gerechten wie dem Gottlosen, dem Guten und Reinen wie dem Unreinen, dem, der opfert, wie dem, der nicht opfert; wie es dem Guten geht, so geht's auch dem Sünder; wie es dem, der schwört, geht, so geht's auch dem, der den Eid fürchtet.
\par 3 Das ist ein böses Ding unter allem, was unter der Sonne geschieht, daß es einem geht wie dem andern; daher auch das Herz der Menschen voll Arges wird, und Torheit ist in ihrem Herzen, dieweil sie leben; darnach müssen sie sterben.
\par 4 Denn bei allen Lebendigen ist, was man wünscht: Hoffnung; denn ein lebendiger Hund ist besser denn ein toter Löwe.
\par 5 Denn die Lebendigen wissen, daß sie sterben werden; die Toten aber wissen nichts, sie haben auch keinen Lohn mehr, denn ihr Gedächtnis ist vergessen,
\par 6 daß man sie nicht mehr liebt noch haßt noch neidet, und haben kein Teil mehr auf dieser Welt an allem, was unter der Sonne geschieht.
\par 7 So gehe hin und iß dein Brot mit Freuden, trink deinen Wein mit gutem Mut; denn dein Werk gefällt Gott.
\par 8 Laß deine Kleider immer weiß sein und laß deinem Haupt Salbe nicht mangeln.
\par 9 Brauche das Leben mit deinem Weibe, das du liebhast, solange du das eitle Leben hast, das dir Gott unter der Sonne gegeben hat, solange dein eitel Leben währt; denn das ist dein Teil im Leben und in deiner Arbeit, die du tust unter der Sonne.
\par 10 Alles, was dir vor Handen kommt, zu tun, das tue frisch; denn bei den Toten, dahin du fährst, ist weder Werk, Kunst, Vernunft noch Weisheit.
\par 11 Ich wandte mich und sah, wie es unter der Sonne zugeht, daß zum Laufen nicht hilft schnell zu sein, zum Streit hilft nicht stark sein, zur Nahrung hilft nicht geschickt sein, zum Reichtum hilft nicht klug sein; daß einer angenehm sei, dazu hilft nicht, daß er ein Ding wohl kann; sondern alles liegt an Zeit und Glück.
\par 12 Auch weiß der Mensch seine Zeit nicht; sondern, wie die Fische gefangen werden mit einem verderblichen Haken, und wie die Vögel mit einem Strick gefangen werden, so werden auch die Menschen berückt zur bösen Zeit, wenn sie plötzlich über sie fällt.
\par 13 Ich habe auch diese Weisheit gesehen unter der Sonne, die mich groß deuchte:
\par 14 daß eine kleine Stadt war und wenig Leute darin, und kam ein großer König und belagerte sie und baute große Bollwerke darum,
\par 15 und ward darin gefunden ein armer, weiser Mann, der errettete dieselbe Stadt durch seine Weisheit; und kein Mensch gedachte desselben armen Mannes.
\par 16 Da sprach ich: "Weisheit ist ja besser den Stärke; doch wird des Armen Weisheit verachtet und seinen Worten nicht gehorcht."
\par 17 Der Weisen Worte, in Stille vernommen, sind besser denn der Herren Schreien unter den Narren.
\par 18 Weisheit ist besser denn Harnisch; aber eine einziger Bube verderbt viel Gutes.

\chapter{10}

\par 1 Schädliche Fliegen verderben gute Salben; also wiegt ein wenig Torheit schwerer denn Weisheit und Ehre.
\par 2 Des Weisen Herz ist zu seiner Rechten; aber des Narren Herz ist zu seiner Linken.
\par 3 Auch ob der Narr selbst närrisch ist in seinem Tun, doch hält er jedermann für einen Narren.
\par 4 Wenn eines Gewaltigen Zorn wider dich ergeht, so laß dich nicht entrüsten; denn Nachlassen stillt großes Unglück.
\par 5 Es ist ein Unglück, das ich sah unter der Sonne, gleich einem Versehen, das vom Gewaltigen ausgeht:
\par 6 daß ein Narr sitzt in großer Würde, und die Reichen in Niedrigkeit sitzen.
\par 7 Ich sah Knechte auf Rossen, und Fürsten zu Fuß gehen wie Knechte.
\par 8 Aber wer eine Grube macht, der wird selbst hineinfallen; und wer den Zaun zerreißt, den wird eine Schlange stechen.
\par 9 Wer Steine wegwälzt, der wird Mühe damit haben; und wer Holz spaltet, der wird davon verletzt werden.
\par 10 Wenn ein Eisen stumpf wird und an der Schneide ungeschliffen bleibt, muß man's mit Macht wieder schärfen; also folgt auch Weisheit dem Fleiß.
\par 11 Ein Schwätzer ist nichts Besseres als eine Schlange, die ohne Beschwörung sticht.
\par 12 Die Worte aus dem Mund eines Weisen sind holdselig; aber des Narren Lippen verschlingen ihn selbst.
\par 13 Der Anfang seiner Worte ist Narrheit, und das Ende ist schädliche Torheit.
\par 14 Ein Narr macht viele Worte; aber der Mensch weiß nicht, was gewesen ist, und wer will ihm sagen, was nach ihm werden wird?
\par 15 Die Arbeit der Narren wird ihnen sauer, weil sie nicht wissen in die Stadt zu gehen.
\par 16 Weh dir, Land, dessen König ein Kind ist, und dessen Fürsten in der Frühe speisen!
\par 17 Wohl dir, Land, dessen König edel ist, und dessen Fürsten zu rechter Zeit speisen, zur Stärke und nicht zur Lust!
\par 18 Denn durch Faulheit sinken die Balken, und durch lässige Hände wird das Haus triefend.
\par 19 Das macht, sie halten Mahlzeiten, um zu lachen, und der Wein muß die Lebendigen erfreuen, und das Geld muß ihnen alles zuwege bringen.
\par 20 Fluche dem König nicht in deinem Herzen und fluche dem Reichen nicht in deiner Schlafkammer; denn die Vögel des Himmels führen die Stimme fort, und die Fittiche haben, sagen's weiter.

\chapter{11}

\par 1 Laß dein Brot über das Wasser fahren, so wirst du es finden nach langer Zeit.
\par 2 Teile aus unter sieben und unter acht; denn du weißt nicht, was für Unglück auf Erden kommen wird.
\par 3 Wenn die Wolken voll sind, so geben sie Regen auf die Erde; und wenn der Baum fällt, er falle gegen Mittag oder Mitternacht, auf welchen Ort er fällt, da wird er liegen.
\par 4 Wer auf den Wind achtet, der sät nicht; und wer auf die Wolken sieht, der erntet nicht.
\par 5 Gleichwie du nicht weißt den Weg des Windes und wie die Gebeine in Mutterleibe bereitet werden, also kannst du auch Gottes Werk nicht wissen, das er tut überall.
\par 6 Frühe säe deinen Samen und laß deine Hand des Abends nicht ab; denn du weißt nicht, ob dies oder das geraten wird; und ob beides geriete, so wäre es desto besser.
\par 7 Es ist das Licht süß, und den Augen lieblich, die Sonne zu sehen.
\par 8 Wenn ein Mensch viele Jahre lebt, so sei er fröhlich in ihnen allen und gedenke der finstern Tage, daß ihrer viel sein werden; denn alles, was kommt, ist eitel.
\par 9 So freue dich, Jüngling, in deiner Jugend und laß dein Herz guter Dinge sein in deiner Jugend. Tue, was dein Herz gelüstet und deinen Augen gefällt, und wisse, daß dich Gott um dies alles wird vor Gericht führen.
\par 10 Laß die Traurigkeit in deinem Herzen und tue das Übel von deinem Leibe; denn Kindheit und Jugend ist eitel.

\chapter{12}

\par 1 Gedenke an deinen Schöpfer in deiner Jugend, ehe denn die bösen Tage kommen und die Jahre herzutreten, da du wirst sagen: Sie gefallen mir nicht;
\par 2 ehe denn die Sonne und das Licht, Mond und Sterne finster werden und Wolken wieder kommen nach dem Regen;
\par 3 zur Zeit, wenn die Hüter im Hause zittern, und sich krümmen die Starken, und müßig stehen die Müller, weil ihrer so wenig geworden sind, und finster werden, die durch die Fenster sehen,
\par 4 und die Türen an der Gasse geschlossen werden, daß die Stimme der Mühle leise wird, und man erwacht, wenn der Vogel singt, und gedämpft sind alle Töchter des Gesangs;
\par 5 wenn man auch vor Höhen sich fürchtet und sich scheut auf dem Wege; wenn der Mandelbaum blüht, und die Heuschrecke beladen wird, und alle Lust vergeht (denn der Mensch fährt hin, da er ewig bleibt, und die Klageleute gehen umher auf der Gasse);
\par 6 ehe denn der silberne Strick wegkomme, und die goldene Schale zerbreche, und der Eimer zerfalle an der Quelle, und das Rad zerbrochen werde am Born.
\par 7 Denn der Staub muß wieder zu der Erde kommen, wie er gewesen ist, und der Geist wieder zu Gott, der ihn gegeben hat.
\par 8 Es ist alles ganz eitel, sprach der Prediger, ganz eitel.
\par 9 Derselbe Prediger war nicht allein weise, sondern lehrte auch das Volk gute Lehre und merkte und forschte und stellte viel Sprüche.
\par 10 Er suchte, daß er fände angenehme Worte, und schrieb recht die Worte der Wahrheit.
\par 11 Die Worte der Weisen sind Stacheln und Nägel; sie sind geschrieben durch die Meister der Versammlungen und von einem Hirten gegeben.
\par 12 Hüte dich, mein Sohn, vor andern mehr; denn viel Büchermachens ist kein Ende, und viel studieren macht den Leib müde.
\par 13 Laßt uns die Hauptsumme alle Lehre hören: Fürchte Gott und halte seine Gebote; denn das gehört allen Menschen zu.
\par 14 Denn Gott wird alle Werke vor Gericht bringen, alles, was verborgen ist, es sei gut oder böse.


\end{document}