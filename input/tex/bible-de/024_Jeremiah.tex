\begin{document}

\title{Jeremiah}


\chapter{1}

\par 1 Dies sind die Reden Jeremia's, des Sohnes Hilkias, aus den Priestern zu Anathoth im Lande Benjamin,
\par 2 zu welchem geschah das Wort des HERRN zur Zeit Josias, des Sohnes Amons, des Königs in Juda, im dreizehnten Jahr seines Königreichs,
\par 3 und hernach zur Zeit des Königs in Juda, Jojakims, des Sohnes Josias, bis ans Ende des elften Jahres Zedekias, des Sohnes Josias, des Königs in Juda, bis auf die Gefangenschaft Jerusalems im fünften Monat.
\par 4 Und des HERRN Wort geschah zu mir und sprach:
\par 5 Ich kannte dich, ehe denn ich dich im Mutterleibe bereitete, und sonderte dich aus, ehe denn du von der Mutter geboren wurdest, und stellte dich zum Propheten unter die Völker.
\par 6 Ich aber sprach: Ach HERR HERR, ich tauge nicht, zu predigen; denn ich bin zu jung.
\par 7 Der HERR sprach aber zu mir: Sage nicht: "Ich bin zu jung"; sondern du sollst gehen, wohin ich dich sende, und predigen, was ich dich heiße.
\par 8 Fürchte dich nicht vor ihnen; denn ich bin bei dir und will dich erretten, spricht der HERR.
\par 9 Und der HERR reckte seine Hand aus und rührte meinen Mund an und sprach zu mir: Siehe, ich lege meine Worte in deinen Mund.
\par 10 Siehe, ich setze dich heute dieses Tages über Völker und Königreiche, daß du ausreißen, zerbrechen, verstören und verderben sollst und bauen und pflanzen.
\par 11 Und es geschah des HERRN Wort zu mir und sprach: Jeremia, was siehst du? Ich sprach: Ich sehe einen erwachenden Zweig.
\par 12 Und der HERR sprach zu mir: Du hast recht gesehen; denn ich will wachen über mein Wort, daß ich's tue.
\par 13 Und es geschah des HERRN Wort zum andernmal zu mir und sprach: Was siehst du? Ich sprach: Ich sehe einen heißsiedenden Topf von Mitternacht her.
\par 14 Und der HERR sprach zu mir: Von Mitternacht wird das Unglück ausbrechen über alle, die im Lande wohnen.
\par 15 Denn siehe, ich will rufen alle Fürsten in den Königreichen gegen Mitternacht, spricht der HERR, daß sie kommen sollen und ihre Stühle setzen vor die Tore zu Jerusalem und rings um die Mauern her und vor alle Städte Juda's.
\par 16 Und ich will das Recht lassen über sie gehen um all ihrer Bosheit willen, daß sie mich verlassen und räuchern andern Göttern und beten an ihrer Hände Werk.
\par 17 So begürte nun deine Lenden und mache dich auf und predige ihnen alles, was ich dich heiße. Erschrick nicht vor ihnen, auf daß ich dich nicht erschrecke vor ihnen;
\par 18 denn ich will dich heute zur festen Stadt, zur eisernen Säule, zur ehernen Mauer machen im ganzen Lande, wider die Könige Juda's, wider ihre Fürsten, wider ihre Priester, wider das Volk im Lande,
\par 19 daß, wenn sie gleich wider dich streiten, sie dennoch nicht sollen wider dich siegen; denn ich bin bei dir, spricht der HERR, daß ich dich errette.

\chapter{2}

\par 1 Und des HERRN Wort geschah zu mir und sprach:
\par 2 Gehe hin und predige öffentlich zu Jerusalem und sprich: So spricht der HERR: Ich gedenke, da du eine freundliche, junge Dirne und eine liebe Braut warst, da du mir folgtest in der Wüste, in dem Lande, da man nichts sät,
\par 3 da Israel des HERRN eigen war und seine erste Frucht. Wer sie fressen wollte, mußte Schuld haben, und Unglück mußte über ihn kommen, spricht der HERR.
\par 4 Hört des HERRN Wort, ihr vom Hause Jakob und alle Geschlechter vom Hause Israel.
\par 5 So spricht der HERR: Was haben doch eure Väter Unrechtes an mir gefunden, daß sie von mir wichen und hingen an den unnützen Götzen, da sie doch nichts erlangten?
\par 6 und dachten nie einmal: Wo ist der HERR, der uns aus Ägyptenland führte und leitete uns in der Wüste, im wilden, ungebahnten Lande, im dürren und finstern Lande, in dem Lande, da niemand wandelte noch ein Mensch wohnte?
\par 7 Und ich brachte euch in ein gutes Land, daß ihr äßet seine Früchte und Güter. Und da ihr hineinkamt, verunreinigtet ihr mein Land und machtet mir mein Erbe zum Greuel.
\par 8 Die Priester gedachten nicht: Wo ist der HERR? und die das Gesetz treiben, achteten mein nicht, und die Hirten führten die Leute von mir, und die Propheten weissagten durch Baal und hingen an den unnützen Götzen.
\par 9 Darum muß ich noch immer mit euch und mit euren Kindeskindern hadern, spricht der HERR.
\par 10 Gehet hin in die Inseln Chittim und schauet, und sendet nach Kedar und merket mit Fleiß und schauet, ob's daselbst so zugeht!
\par 11 Ob die Heiden ihre Götter ändern, wiewohl sie doch nicht Götter sind! Und mein Volk hat doch seine Herrlichkeit verändert um einen unnützen Götzen.
\par 12 Sollte sich doch der Himmel davor entsetzen, erschrecken und sehr erbeben, spricht der HERR.
\par 13 Denn mein Volk tut eine zwiefache Sünde: mich, die lebendige Quelle, verlassen sie und machen sich hier und da ausgehauenen Brunnen, die doch löcherig sind und kein Wasser geben.
\par 14 Ist denn Israel ein Knecht oder Leibeigen, daß er jedermanns Raub sein muß?
\par 15 Denn Löwen brüllen über ihn und schreien und verwüsten sein Land, und seine Städte werden verbrannt, daß niemand darin wohnt.
\par 16 Dazu zerschlagen die von Noph und Thachpanhes dir den Kopf.
\par 17 Solches machst du dir selbst, weil du den HERRN, deinen Gott, verläßt, so oft er dich den rechten Weg leiten will.
\par 18 Was hilft's dir, daß du nach Ägypten ziehst und willst vom Wasser Sihor trinken? Und was hilft's dir, daß du nach Assyrien ziehst und willst vom Wasser des Euphrat trinken?
\par 19 Es ist deiner Bosheit Schuld, daß du so gestäupt wirst, und deines Ungehorsams, daß du so gestraft wirst. Also mußt du innewerden und erfahren, was es für Jammer und Herzeleid bringt, den HERRN, deinen Gott, verlassen und ihn nicht fürchten, spricht der HERR HERR Zebaoth.
\par 20 Denn du hast immerdar dein Joch zerbrochen und deine Bande zerrissen und gesagt: Ich will nicht unterworfen sein! sondern auf allen hohen Hügeln und unter allen grünen Bäumen liefst du den Götzen nach.
\par 21 Ich aber hatte dich gepflanzt zu einem süßen Weinstock, einen ganz rechtschaffenen Samen. Wie bist du mir denn geraten zu einem bitteren, wilden Weinstock?
\par 22 Und wenn du dich gleich mit Lauge wüschest und nähmest viel Seife dazu, so gleißt doch deine Untugend desto mehr vor mir, spricht der HERR HERR.
\par 23 Wie darfst du denn sagen: Ich bin nicht unrein, ich hänge nicht an den Baalim? Siehe an, wie du es treibst im Tal, und bedenke, wie du es ausgerichtet hast.
\par 24 Du läufst umher wie eine Kamelstute in der Brunst, und wie ein Wild in der Wüste pflegt, wenn es vor großer Brunst lechzt und läuft, daß es niemand aufhalten kann. Wer's wissen will, darf nicht weit laufen; am Feiertage sieht man es wohl.
\par 25 Schone doch deiner Füße, daß sie nicht bloß, und deines Halses das er nicht durstig werde. Aber du sprichst: Da wird nichts draus; ich muß mit den Fremden buhlen und ihnen nachlaufen.
\par 26 Wie ein Dieb zu Schanden wird, wenn er ergriffen wird, also wird das Haus Israel zu Schanden werden samt ihren Königen, Fürsten, Priestern und Propheten,
\par 27 die zum Holz sagen: Du bist mein Vater, und zum Stein: Du hast mich gezeugt. Denn sie kehren mir den Rücken zu und nicht das Angesicht. Aber wenn die Not hergeht, sprechen sie: Auf, und hilf uns!
\par 28 Wo sind aber dann deine Götter, die du dir gemacht hast? Heiße sie aufstehen; laß sehen, ob sie dir helfen können in deiner Not! Denn so manche Stadt, so manchen Gott hast du, Juda.
\par 29 Was wollt ihr noch recht haben wider mich? Ihr seid alle von mir abgefallen, spricht der HERR.
\par 30 Alle Schläge sind verloren an euren Kindern; sie lassen sich doch nicht ziehen. Denn euer Schwert frißt eure Propheten wie ein wütiger Löwe.
\par 31 Du böse Art, merke auf des HERRN Wort! Bin ich denn für Israel eine Wüste oder ödes Land? Warum spricht denn mein Volk: Wir sind die Herren und müssen dir nicht nachlaufen?
\par 32 Vergißt doch eine Jungfrau ihres Schmuckes nicht, noch eine Braut ihres Schleiers; aber mein Volk vergißt mein ewiglich.
\par 33 Was beschönst du viel dein Tun, daß ich dir gnädig sein soll? Unter solchem Schein treibst du je mehr und mehr Bosheit.
\par 34 Überdas findet man Blut der armen und unschuldigen Seelen bei dir an allen Orten, und das ist nicht heimlich, sondern offenbar an diesen Orten.
\par 35 Doch sprichst du: Ich bin unschuldig; er wende seinen Zorn von mir. Siehe, ich will mit dir rechten, daß du sprichst: Ich habe nicht gesündigt.
\par 36 Wie weichst du doch so gern und läufst jetzt dahin, jetzt hierher! Aber du wirst an Ägypten zu Schanden werden, wie du an Assyrien zu Schanden geworden bist.
\par 37 Denn du mußt von dort auch wegziehen und deine Hände über dem Haupt zusammenschlagen; denn der Herr wird deine Hoffnung trügen lassen, und nichts wird dir bei ihnen gelingen.

\chapter{3}

\par 1 Und er spricht: Wenn sich ein Mann von seinem Weibe scheidet, und sie zieht von ihm und nimmt einen andern Mann, darf er sie auch wieder annehmen? Ist's nicht also, daß das Land verunreinigt würde? Du aber hast mit vielen Buhlen gehurt; doch komm wieder zu mir; spricht der HERR.
\par 2 Hebe deine Augen auf zu den Höhen und siehe, wie du allenthalben Hurerei treibst. An den Straßen sitzest du und lauerst auf sie wie ein Araber in der Wüste und verunreinigst das Land mit deiner Hurerei und Bosheit.
\par 3 Darum muß auch der Frühregen ausbleiben und kein Spätregen kommen. Du hast eine Hurenstirn, du willst dich nicht mehr schämen
\par 4 und schreist gleichwohl zu mir: "Lieber Vater, du Meister meiner Jugend!
\par 5 willst du denn ewiglich zürnen und nicht vom Grimm lassen?" Siehe, so redest du, und tust Böses und lässest dir nicht steuern.
\par 6 Und der HERR sprach zu mir zu der Zeit des Königs Josia: Hast du auch gesehen, was Israel, die Abtrünnige tat? Sie ging hin auf alle hohen Berge und unter alle grünen Bäume und trieb daselbst Hurerei.
\par 7 Und ich sprach, da sie solches alles getan hatte: Bekehre dich zu mir. Aber sie bekehrte sich nicht. Und obwohl ihre Schwester Juda, die Verstockte, gesehen hat,
\par 8 wie ich der Abtrünnigen Israel Ehebruch gestraft und sie verlassen und ihr einen Scheidebrief gegeben habe: dennoch fürchtet sich ihre Schwester, die verstockte Juda, nicht, sondern geht hin und treibt auch Hurerei.
\par 9 Und von dem Geschrei ihrer Hurerei ist das Land verunreinigt; denn sie treibt Ehebruch mit Stein und Holz.
\par 10 Und in diesem allem bekehrt sich die verstockte Juda, ihre Schwester, nicht zu mir von ganzem Herzen, sondern heuchelt also, spricht der HERR.
\par 11 Und der HERR sprach zu mir: Die abtrünnige Israel ist fromm gegen die verstockte Juda.
\par 12 Gehe hin und rufe diese Worte gegen die Mitternacht und sprich: Kehre wieder, du abtrünnige Israel, spricht der HERR, so will ich mein Antlitz nicht gegen euch verstellen. Denn ich bin barmherzig, spricht der HERR, und ich will nicht ewiglich zürnen.
\par 13 Allein erkenne deine Missetat, daß du wider den HERRN, deinen Gott, gesündigt hast und bist hin und wieder gelaufen zu den fremden Göttern unter allen grünen Bäumen und habt meiner Stimme nicht gehorcht, spricht der HERR.
\par 14 Bekehret euch nun ihr abtrünnigen Kinder, spricht der HERR; denn ich will euch mir vertrauen und will euch holen, einen aus einer Stadt und zwei aus einem Geschlecht, und will euch bringen gen Zion
\par 15 und will euch Hirten geben nach meinem Herzen, die euch weiden sollen mit Lehre und Weisheit.
\par 16 Und es soll geschehen, wenn ihr gewachsen seid und euer viel geworden sind im Lande, so soll man, spricht der HERR, zur selben Zeit nicht mehr sagen von der Bundeslade des HERRN, auch ihrer nicht mehr gedenken noch davon predigen noch nach ihr fragen, und sie wird nicht wieder gemacht werden;
\par 17 sondern zur selben Zeit wird man Jerusalem heißen "Des HERRN Thron", und es werden sich dahin sammeln alle Heiden um des Namens des HERRN willen zu Jerusalem und werden nicht mehr wandeln nach den Gedanken ihres bösen Herzens.
\par 18 Zu der Zeit wird das Haus Juda gehen zum Hause Israel, und sie werden miteinander kommen von Mitternacht in das Land, das ich euren Vätern zum Erbe gegeben habe.
\par 19 Und ich sagte dir zu: Wie will ich dir so viel Kinder geben und das liebe Land, das allerschönste Erbe unter den Völkern! Und ich sagte dir zu: Du wirst alsdann mich nennen "Lieber Vater!" und nicht von mir weichen.
\par 20 Aber das Haus Israel achtete mich nicht, gleichwie ein Weib ihren Buhlen nicht mehr achtet, spricht der HERR.
\par 21 Darum hört man ein klägliches Heulen und Weinen der Kinder Israel auf den Höhen, dafür daß sie übel getan und des HERRN, ihres Gottes, vergessen haben.
\par 22 So kehret nun wieder, ihr abtrünnigen Kinder, so will ich euch heilen von eurem Ungehorsam. Siehe wir kommen zu dir; denn du bist der HERR, unser Gott.
\par 23 Wahrlich, es ist eitel Betrug mit Hügeln und mit allen Bergen. Wahrlich, es hat Israel keine Hilfe denn am HERRN, unserm Gott.
\par 24 Und die Schande hat gefressen unsrer Väter Arbeit von unsrer Jugend auf samt ihren Schafen, Rindern, Söhnen und Töchtern.
\par 25 Denn worauf wir uns verließen, das ist uns jetzt eitel Schande, und wessen wir uns trösteten, des müssen wir uns jetzt schämen. Denn wir sündigten damit wider den HERRN, unsern Gott, beide, wir und unsre Väter, von unsrer Jugend auf, auch bis auf diesen heutigen Tag, und gehorchten nicht der Stimme des HERRN, unsers Gottes.

\chapter{4}

\par 1 Willst du dich, Israel, bekehren, spricht der HERR, so bekehre dich zu mir. Und so du deine Greuel wegtust von meinem Angesicht, so sollst du nicht vertrieben werden.
\par 2 Alsdann wirst du ohne Heuchelei recht und heilig schwören: So wahr der HERR lebt! und die Heiden werden in ihm gesegnet werden und sich sein rühmen.
\par 3 Denn so spricht der HERR zu denen in Juda und zu Jerusalem: Pflügt ein Neues und säet nicht unter die Hecken.
\par 4 Beschneidet euch dem HERRN und tut weg die Vorhaut eures Herzens, ihr Männer in Juda und ihr Leute zu Jerusalem, auf daß nicht mein Grimm ausfahre wie Feuer und brenne, daß niemand löschen könne, um eurer Bosheit willen.
\par 5 Verkündiget in Juda und schreiet laut zu Jerusalem und sprecht: "Blaset die Drommete im Lande!" Ruft mit voller Stimme und sprecht: "Sammelt euch und laßt uns in die festen Städte ziehen!"
\par 6 Werft zu Zion ein Panier auf; flieht und säumt nicht! Denn ich bringe ein Unglück herzu von Mitternacht und einen großen Jammer.
\par 7 Es fährt daher der Löwe aus seiner Hecke, und der Verstörer der Heiden zieht einher aus seinem Ort, daß er dein Land verwüste und deine Städte ausbrenne, daß niemand darin wohne.
\par 8 Darum ziehet Säcke an, klaget und heulet; denn der grimmige Zorn des HERRN will sich nicht wenden von uns.
\par 9 Zu der Zeit, spricht der HERR, wird dem König und den Fürsten das Herz entfallen; die Priester werden bestürzt und die Propheten erschrocken sein.
\par 10 Ich aber sprach: Ach HERR HERR! du hast's diesem Volk und Jerusalem weit fehlgehen lassen, da sie sagten: "Es wird Friede mit euch sein", so doch das Schwert bis an die Seele reicht.
\par 11 Zu derselben Zeit wird man diesem Volk und Jerusalem sagen: "Es kommt ein dürrer Wind über das Gebirge her aus der Wüste, des Weges zu der Tochter meines Volks, nicht zum Worfeln noch zu Schwingen."
\par 12 Ja, ein Wind kommt, der ihnen zu stark sein wird; da will ich denn auch mit ihnen rechten.
\par 13 Siehe, er fährt daher wie Wolken, und seine Wagen sind wie Sturmwind, seine Rosse sind schneller denn Adler. Weh uns! wir müssen verstört werden."
\par 14 So wasche nun, Jerusalem, dein Herz von der Bosheit, auf daß dir geholfen werde. Wie lange wollen bei dir bleiben deine leidigen Gedanken?
\par 15 Denn es kommt ein Geschrei von Dan her und eine böse Botschaft vom Gebirge Ephraim.
\par 16 Saget an den Heiden, verkündiget in Jerusalem, daß Hüter kommen aus fernen Landen und werden schreien wider die Städte Juda's.
\par 17 Sie werden sich um sie her lagern wie die Hüter auf dem Felde; denn sie haben mich erzürnt, spricht der HERR.
\par 18 Das hast du zum Lohn für dein Wesen und dein Tun. Dann wird dein Herz fühlen, wie deine Bosheit so groß ist.
\par 19 Wie ist mir so herzlich weh! Mein Herz pocht mir im Leibe, und habe keine Ruhe; denn meine Seele hört der Posaunen Hall und eine Feldschlacht
\par 20 und einen Mordschrei über den andern; denn das ganze Land wird verheert, plötzlich werden meine Hütten und meine Gezelte verstört.
\par 21 Wie lange soll ich doch das Panier sehen und der Posaune Hall hören?
\par 22 Aber mein Volk ist toll, und sie glauben mir nicht; töricht sind sie und achten's nicht. Weise sind sie genug, Übles zu tun; aber wohltun wollen sie nicht lernen.
\par 23 Ich schaute das Land an, siehe, das war wüst und öde, und den Himmel, und er war finster.
\par 24 Ich sah die Berge an, und siehe, die bebten, und alle Hügel zitterten.
\par 25 Ich sah, und siehe, da war kein Mensch, und alle Vögel unter dem Himmel waren weggeflogen.
\par 26 Ich sah, und siehe, das Gefilde war eine Wüste; und alle Städte darin waren zerbrochen vor dem HERRN und vor seinem grimmigen Zorn.
\par 27 Denn so spricht der HERR: Das ganze Land soll wüst werden, und ich will's doch nicht gar aus machen.
\par 28 Darum wird das Land betrübt und der Himmel droben traurig sein; denn ich habe es geredet, ich habe es beschlossen, und es soll mich nicht reuen, will auch nicht davon ablassen.
\par 29 Aus allen Städten werden sie vor dem Geschrei der Reiter und Schützen fliehen und in die dicken Wälder laufen und in die Felsen kriechen; alle Städte werden verlassen stehen, daß niemand darin wohnt.
\par 30 Was willst du alsdann tun, du Verstörte? Wenn du dich schon mit Purpur kleiden und mit goldenen Kleinoden schmücken und dein Angesicht schminken würdest, so schmückst du dich doch vergeblich; die Buhlen werden dich verachten, sie werden dir nach dem Leben trachten.
\par 31 Denn ich höre ein Geschrei als einer Gebärerin, eine Angst als einer, die in den ersten Kindesnöten ist, ein Geschrei der Tochter Zion, die da klagt und die Hände auswirft: "Ach, wehe mir! Ich muß fast vergehen vor den Würgern."

\chapter{5}

\par 1 Gehet durch die Gassen zu Jerusalem und schauet und erfahret und suchet auf ihrer Straße, ob ihr jemand findet, der recht tue und nach dem Glauben frage, so will ich dir gnädig sein.
\par 2 Und wenn sie schon sprechen: "Bei dem lebendigen Gott!", so schwören sie doch falsch.
\par 3 HERR, deine Augen sehen nach dem Glauben. Du schlägst sie, aber sie fühlen's nicht; du machst es schier aus mit ihnen, aber sie bessern sich nicht. Sie haben ein härter Angesicht denn ein Fels und wollen sich nicht bekehren.
\par 4 Ich dachte aber: Wohlan, der arme Haufe ist unverständig, weiß nichts um des HERRN Weg und um ihres Gottes Recht.
\par 5 Ich will zu den Gewaltigen gehen und mit ihnen reden; die werden um des HERRN Weg und ihres Gottes Recht wissen. Aber sie allesamt hatten das Joch zerbrochen und die Seile zerrissen.
\par 6 Darum wird sie auch der Löwe, der aus dem Walde kommt, zerreißen, und der Wolf aus der Wüste wird sie verderben, und der Parder wird um ihre Städte lauern; alle, die daselbst herausgehen, wird er fressen. Denn ihrer Sünden sind zuviel, und sie bleiben verstockt in ihrem Ungehorsam.
\par 7 Wie soll ich dir denn gnädig sein, weil mich meine Kinder verlassen und schwören bei dem, der nicht Gott ist? und nun ich ihnen vollauf gegeben habe, treiben sie Ehebruch und laufen ins Hurenhaus.
\par 8 Ein jeglicher wiehert nach seines Nächsten Weib wie die vollen, müßigen Hengste.
\par 9 Und ich sollte sie um solches nicht heimsuchen? spricht der HERR, und meine Seele sollte sich nicht rächen an solchem Volk, wie dies ist?
\par 10 Stürmet ihre Mauern und werfet sie um, und macht's nicht gar aus! Führet ihre Reben weg, denn sie sind nicht des HERRN;
\par 11 sondern sie verachten mich, beide, das Haus Israel und das Haus Juda, spricht der HERR.
\par 12 Sie verleugnen den HERRN und sprechen: "Das ist er nicht, und so übel wird es uns nicht gehen; Schwert und Hunger werden wir nicht sehen.
\par 13 Ja, die Propheten sind Schwätzer und haben auch Gottes Wort nicht; es gehe über sie selbst also!"
\par 14 Darum spricht der HERR, der Gott Zebaoth: Weil ihr solche Rede treibt, siehe, so will ich meine Worte in deinem Munde zu Feuer machen, und dies Volk zu Holz, und es soll sie verzehren.
\par 15 Siehe, ich will über euch vom Hause Israel, spricht der HERR, ein Volk von ferne bringen, ein mächtiges Volk, dessen Sprache du nicht verstehst, und kannst nicht vernehmen, was sie reden.
\par 16 Seine Köcher sind offene Gräber; es sind eitel Helden.
\par 17 Sie werden deine Ernte und dein Brot verzehren; sie werde deine Söhne und Töchter fressen; sie werden deine Schafe und Rinder verschlingen; sie werden deine Weinstöcke und Feigenbäume verzehren; deine festen Städte, darauf du dich verläßt, werden sie mit dem Schwert verderben.
\par 18 Doch will ich's, spricht der HERR, zur selben Zeit mit euch nicht gar aus machen.
\par 19 Und ob sie würden sagen: "Warum tut uns der HERR, unser Gott, solches alles?", sollst du ihnen antworten: Wie ihr mich verlaßt und den fremden Göttern dient in eurem eigenen Lande, also sollt ihr auch Fremden dienen in einem Lande, das nicht euer ist.
\par 20 Solches sollt ihr verkündigen im Hause Jakob und predigen in Juda und sprechen:
\par 21 Höret zu, ihr tolles Volk, das keinen Verstand hat, die da Augen haben, und sehen nicht, Ohren haben, und hören nicht!
\par 22 Wollt ihr mich nicht fürchten? spricht der HERR, und vor mir nicht erschrecken, der ich dem Meer den Sand zum Ufer setzte, darin es allezeit bleiben muß, darüber es nicht gehen darf? Und ob's schon wallet, so vermag's doch nichts; und ob seine Wellen schon toben, so dürfen sie doch nicht darüberfahren.
\par 23 Aber dies Volk hat ein abtrünniges, ungehorsames Herz; sie bleiben abtrünnig und gehen immerfort weg
\par 24 und sprechen nicht einmal in ihrem Herzen: Laßt uns doch den HERRN, unsern Gott, fürchten, der uns Frühregen und Spätregen zu rechter Zeit gibt und uns die Ernte treulich und jährlich behütet.
\par 25 Aber eure Missetaten hindern solches, und eure Sünden wenden das Gute von euch.
\par 26 Denn man findet unter meinem Volk Gottlose, die den Leuten nachstellen und Fallen zurichten, sie zu fangen, wie die Vogler tun.
\par 27 Und ihre Häuser sind voller Tücke, wie ein Vogelbauer voller Lockvögel ist. Daher werden sie gewaltig und reich, fett und glatt.
\par 28 Sie gehen mit bösen Stücken um; sie halten kein Recht, der Waisen Sache fördern sie nicht, daß auch sie Glück hätten, und helfen den Armen nicht zum Recht.
\par 29 Sollte ich denn solches nicht heimsuchen, spricht der HERR, und meine Seele sollte sich nicht rächen an solchem Volk, wie dies ist?
\par 30 Es steht greulich und schrecklich im Lande.
\par 31 Die Propheten weissagen falsch, und die Priester herrschen in ihrem Amt, und mein Volk hat's gern also. Wie will es euch zuletzt darob gehen?

\chapter{6}

\par 1 Fliehet, ihr Kinder Benjamin, aus Jerusalem und blaset die Drommete auf der Warte Thekoa und werft auf ein Panier über der Warte Beth-Cherem! denn es geht daher ein Unglück von Mitternacht und ein großer Jammer.
\par 2 Die Tochter Zion ist wie eine schöne und lustige Aue.
\par 3 Aber es werden die Hirten über sie kommen mit ihren Herden, die werden Gezelte rings um sie her aufschlagen und weiden ein jeglicher an seinem Ort und sprechen:
\par 4 "Rüstet euch zum Krieg wider sie! Wohlauf, laßt uns hinaufziehen, weil es noch hoch Tag ist! Ei, es will Abend werden, und die Schatten werden groß!
\par 5 Wohlan, so laßt uns auf sein, und sollten wir bei Nacht hinaufziehen und ihre Paläste verderben!"
\par 6 Denn also spricht der HERR Zebaoth: Fällt die Bäume und werft einen Wall auf wider Jerusalem; denn sie ist eine Stadt, die heimgesucht werden soll. Ist doch eitel Unrecht darin.
\par 7 Denn gleichwie ein Born sein Wasser quillt, also quillt auch ihre Bosheit. Ihr Frevel und Gewalt schreit über sie, und ihr Morden und Schlagen treiben sie täglich vor mir.
\par 8 Bessere dich Jerusalem, ehe sich mein Herz von dir wendet und ich dich zum wüsten Lande mache, darin niemand wohne!
\par 9 So spricht der HERR Zebaoth: Was übriggeblieben ist von Israel, das muß nachgelesen werden wie am Weinstock. Der Weinleser wird eins nach dem andern in die Butten werfen.
\par 10 Ach, mit wem soll ich doch reden und zeugen? Daß doch jemand hören wollte! Aber ihre Ohren sind unbeschnitten; sie können's nicht hören. Siehe, sie halten des HERRN Wort für einen Spott und wollen es nicht.
\par 11 Darum bin ich von des HERRN Drohen so voll, daß ich's nicht lassen kann. Schütte es aus über die Kinder auf der Gasse und über die Mannschaft im Rat miteinander; denn es sollen beide, Mann und Weib, Alte und der Wohlbetagte, gefangen werden.
\par 12 Ihre Häuser sollen den Fremden zuteil werden samt den Äckern und Weibern; denn ich will meine Hand ausstrecken, spricht der HERR, über des Landes Einwohner.
\par 13 Denn sie geizen allesamt, klein und groß; und beide, Propheten und Priester, gehen allesamt mit Lügen um
\par 14 und trösten mein Volk in seinem Unglück, daß sie es gering achten sollen, und sagen: "Friede! Friede!", und ist doch nicht Friede.
\par 15 Darum werden sie mit Schanden bestehen, daß sie solche Greuel treiben; wiewohl sie wollen ungeschändet sein und wollen sich nicht schämen. Darum müssen sie fallen auf einen Haufen; und wenn ich sie heimsuchen werde, sollen sie stürzen, spricht der HERR.
\par 16 So spricht der HERR: Tretet auf die Wege und schauet und fraget nach den vorigen Wegen, welches der gute Weg sei, und wandelt darin, so werdet ihr Ruhe finden für eure Seele! Aber sie sprechen: Wir wollen's nicht tun!
\par 17 Ich habe Wächter über dich gesetzt: Merket auf die Stimme der Drommete! Aber sie sprechen: Wir wollen's nicht tun!
\par 18 Darum so höret, ihr Heiden, und merket samt euren Leuten!
\par 19 Du, Erde, höre zu! Siehe, ich will ein Unglück über dies Volk bringen, darum daß sie auf meine Worte nicht achten und mein Gesetz verwerfen.
\par 20 Was frage ich nach Weihrauch aus Reicharabien und nach den guten Zimtrinden, die aus fernen Landen kommen? Eure Brandopfer sind mir nicht angenehm, und eure Opfer gefallen mir nicht.
\par 21 Darum spricht der HERR also: Siehe, ich will diesem Volk einen Anstoß in den Weg stellen, daran sich die Väter und Kinder miteinander stoßen und ein Nachbar mit dem andern umkommen sollen.
\par 22 So spricht der HERR: Siehe, es wird ein Volk kommen von Mitternacht, und ein großes Volk wird sich erregen vom Ende der Erde,
\par 23 die Bogen und Lanze führen. Es ist grausam und ohne Barmherzigkeit; sie brausen daher wie ein ungestümes Meer und reiten auf Rossen, gerüstet wie Kriegsleute, wider dich, du Tochter Zion.
\par 24 Wenn wir von ihnen hören werden, so werden uns die Fäuste entsinken; es wird uns angst und weh werden wie einer Gebärerin.
\par 25 Es gehe ja niemand hinaus auf den Acker, niemand gehe über Feld; denn es ist allenthalben unsicher vor dem Schwert des Feindes.
\par 26 O Tochter meines Volks, zieh Säcke an und lege dich in Asche; trage Leid wie um einen einzigen Sohn und klage wie die, so hoch betrübt sind! denn der Verderber kommt über uns plötzlich.
\par 27 Ich habe dich zum Schmelzer gesetzt unter mein Volk, das so hart ist, daß du ihr Wesen erfahren und prüfen sollst.
\par 28 Sie sind allzumal Abtrünnige und wandeln verräterisch, sind Erz und Eisen; alle sind sie verderbt.
\par 29 Der Blasebalg ist verbrannt, das Blei verschwindet; das Schmelzen ist umsonst, denn das Böse ist nicht davon geschieden.
\par 30 Darum heißen sie auch ein verworfenes Silber; denn der HERR hat sie verworfen.

\chapter{7}

\par 1 Dies ist das Wort, welches geschah zu Jeremia vom HERRN, und sprach:
\par 2 Tritt ins Tor im Hause des HERR und predige daselbst dies Wort und sprich: Höret des HERRN Wort, ihr alle von Juda, die ihr zu diesen Toren eingehet, den HERRN anzubeten!
\par 3 So spricht der HERR Zebaoth, der Gott Israels: Bessert euer Leben und Wesen, so will ich bei euch wohnen an diesem Ort.
\par 4 Verlaßt euch nicht auf die Lügen, wenn sie sagen: Hier ist des HERRN Tempel, hier ist des HERRN Tempel, hier ist des HERRN Tempel!
\par 5 sondern bessert euer Leben und Wesen, daß ihr recht tut einer gegen den andern
\par 6 und den Fremdlingen, Waisen und Witwen keine Gewalt tut und nicht unschuldiges Blut vergießt an diesem Ort, und folgt nicht nach andern Göttern zu eurem eigenen Schaden:
\par 7 so will ich immer und ewiglich bei euch wohnen an diesem Ort, in dem Lande, das ich euren Vätern gegeben habe.
\par 8 Aber nun verlasset ihr euch auf Lügen, die nichts nütze sind.
\par 9 Daneben seid ihr Diebe, Mörder, Ehebrecher und Meineidige und räuchert dem Baal und folgt fremden Göttern nach, die ihr nicht kennt.
\par 10 Darnach kommt ihr dann und tretet vor mich in diesem Hause, das nach meinem Namen genannt ist, und sprecht: Es hat keine Not mit uns, weil wir solche Greuel tun.
\par 11 Haltet ihr denn dies Haus, das nach meinem Namen genannt ist, für eine Mördergrube? Siehe, ich sehe es wohl, spricht der HERR.
\par 12 Gehet hin an meinen Ort zu Silo, da vormals mein Name gewohnt hat, und schauet, was ich daselbst getan habe um der Bosheit willen meines Volkes Israel.
\par 13 Weil ihr denn alle solche Stücke treibt, spricht der HERR, und ich stets euch predigen lasse, und ihr wollt nicht hören, ich rufe euch, und ihr wollt nicht antworten:
\par 14 so will ich dem Hause, das nach meinem Namen genannt ist, darauf ihr euch verlaßt, und den Ort, den ich euren Vätern gegeben habe, eben tun, wie ich zu Silo getan habe,
\par 15 und will euch von meinem Angesicht wegwerfen, wie ich weggeworfen habe alle eure Brüder, den ganzen Samen Ephraims.
\par 16 Und du sollst für dies Volk nicht bitten und sollst für sie keine Klage noch Gebet vorbringen, auch nicht sie vertreten vor mir; denn ich will dich nicht hören.
\par 17 Denn siehst du nicht, was sie tun in den Städten Juda's und auf den Gassen zu Jerusalem?
\par 18 Die Kinder lesen Holz, so zünden die Väter das Feuer an, und die Weiber kneten den Teig, daß sie der Himmelskönigin Kuchen backen, und geben Trankopfer den fremden Göttern, daß sie mir Verdruß tun.
\par 19 Aber sie sollen nicht mir damit, spricht der HERR, sondern sich selbst Verdruß tun und müssen zu Schanden werden.
\par 20 Darum spricht der HERR HERR: Siehe, mein Zorn und mein Grimm ist ausgeschüttet über diesen Ort, über Menschen und Vieh, über Bäume auf dem Felde und über die Früchte des Landes; und der soll brennen, daß niemand löschen kann.
\par 21 So spricht der HERR Zebaoth, der Gott Israels: Tut eure Brandopfer und anderen Opfer zuhauf und esset Fleisch.
\par 22 Denn ich habe euren Vätern des Tages, da ich sie aus Ägyptenland führte, weder gesagt noch geboten von Brandopfer und anderen Opfern;
\par 23 sondern dies gebot ich ihnen und sprach: Gehorchet meinem Wort, so will ich euer Gott sein, und ihr sollt mein Volk sein; und wandelt auf allen Wegen, die ich euch gebiete, auf daß es euch wohl gehe.
\par 24 Aber sie wollen nicht hören noch ihre Ohren zuneigen, sondern wandelten nach ihrem eigenen Rat und nach ihres bösen Herzens Gedünken und gingen hinter sich und nicht vor sich.
\par 25 Ja, von dem Tage an, da ich eure Väter aus Ägyptenland geführt habe, bis auf diesen Tag habe ich stets zu euch gesandt alle meine Knechte, die Propheten.
\par 26 Aber sie wollen mich nicht hören noch ihre Ohren neigen, sondern waren halsstarrig und machten's ärger denn ihre Väter.
\par 27 Und wenn du ihnen dies alles schon sagst, so werden sie dich doch nicht hören; rufst du ihnen, so werden sie dir nicht antworten.
\par 28 Darum sprich zu ihnen: Dies ist das Volk, das den HERRN, seinen Gott, nicht hören noch sich bessern will. Der Glaube ist untergegangen und ausgerottet von ihrem Munde.
\par 29 Schneide deine Haare ab und wirf sie von dir und wehklage auf den Höhen; denn der HERR hat dies Geschlecht, über das er zornig ist, verworfen und verstoßen.
\par 30 Denn die Kinder Juda tun übel vor meinen Augen, spricht der HERR. Sie setzen ihre Greuel in das Haus, das nach meinem Namen genannt ist, daß sie es verunreinigen,
\par 31 und bauen die Altäre des Thopheth im Tal Ben-Hinnom, daß sie ihre Söhne und Töchter verbrennen, was ich nie geboten noch in den Sinn genommen habe.
\par 32 Darum siehe, es kommt die Zeit, spricht der HERR, daß man's nicht mehr heißen soll Thopheth und das Tal Ben-Hinnom, sondern Würgetal; und man wird im Thopheth müssen begraben, weil sonst kein Raum mehr sein wird.
\par 33 Und die Leichname dieses Volkes sollen den Vögeln des Himmels und den Tieren auf Erden zur Speise werden, davon sie niemand scheuchen wird.
\par 34 Und ich will in den Städten Juda's und auf den Gassen zu Jerusalem wegnehmen das Geschrei der Freude und Wonne und die Stimme des Bräutigams und der Braut; denn das Land soll wüst sein.

\chapter{8}

\par 1 Zu derselben Zeit, spricht der HERR, wird man die Gebeine der Könige Juda's, die Gebeine ihrer Fürsten, die Gebeine der Priester, die Gebeine der Propheten, die Gebeine der Bürger zu Jerusalem aus ihren Gräbern werfen;
\par 2 und wird sie hinstreuen unter Sonne, Mond und alles Heer des Himmels, welche sie geliebt und denen sie gedient haben, denen sie nachgefolgt sind und die sie gesucht und angebetet haben. Sie sollen nicht wieder aufgelesen und begraben werden, sondern Kot auf der Erde sein.
\par 3 Und alle übrigen von diesem bösen Volk, an welchen Ort sie sein werden, dahin ich sie verstoßen habe, werden lieber tot als lebendig sein wollen, spricht der HERR Zebaoth.
\par 4 Darum sprich zu ihnen: So spricht der HERR: Wo ist jemand, so er fällt, der nicht gerne wieder aufstünde? Wo ist jemand, so er irregeht, der nicht gerne wieder zurechtkäme?
\par 5 Dennoch will dies Volk zu Jerusalem irregehen für und für. Sie halten so hart an dem falschen Gottesdienst, daß sie sich nicht wollen abwenden lassen.
\par 6 Ich sehe und höre, daß sie nichts Rechtes reden. Keiner ist, dem seine Bosheit Leid wäre und der spräche: Was mache ich doch! Sie laufen alle ihren Lauf wie ein grimmiger Hengst im Streit.
\par 7 Ein Storch unter dem Himmel weiß seine Zeit, eine Turteltaube, Kranich und Schwalbe merken ihre Zeit, wann sie wiederkommen sollen, aber mein Volk will das Recht des HERRN nicht wissen.
\par 8 Wie mögt ihr doch sagen: "Wir wissen, was recht ist, und haben die heilige Schrift vor uns"? Ist's doch eitel Lüge, was die Schriftgelehrten setzen.
\par 9 Darum müssen solche Lehrer zu Schanden, erschreckt und gefangen werden; denn was können sie Gutes lehren, weil sie des HERRN Wort verwerfen?
\par 10 Darum will ich ihre Weiber den Fremden geben und ihre Äcker denen, die sie verjagen werden. Denn sie geizen allesamt, beide, klein und groß; und beide, Priester und Propheten, gehen mit Lügen um
\par 11 und trösten mein Volk in ihrem Unglück, daß sie es gering achten sollen, und sagen: "Friede! Friede!", und ist doch nicht Friede.
\par 12 Darum werden sie mit Schanden bestehen, daß sie solche Greuel treiben; wiewohl sie wollen ungeschändet sein und wollen sich nicht schämen. Darum müssen sie fallen auf einen Haufen; und wenn ich sie heimsuchen werde, sollen sie stürzen, spricht der HERR.
\par 13 Ich will sie also ablesen, spricht der HERR, daß keine Trauben am Weinstock und keine Feigen am Feigenbaum bleiben, ja auch die Blätter wegfallen sollen; und was ich ihnen gegeben habe, das soll ihnen genommen werden.
\par 14 Wo werden wir dann wohnen? Ja, sammelt euch dann und laßt uns in die festen Städte ziehen, daß wir daselbst umkommen. Denn der HERR, unser Gott, wird uns umkommen lassen und tränken mit einem bitteren Trunk, daß wir so gesündigt haben wider den HERRN.
\par 15 Wir hofften, es sollte Friede werden, so kommt nichts Gutes; wir hofften, wir sollten heil werden, aber siehe, so ist mehr Schaden da.
\par 16 Man hört ihre Rosse schnauben von Dan her; vom Wiehern ihrer Gäule erbebt das ganze Land. Und sie fahren daher und werden das Land auffressen mit allem, was darin ist, die Städte samt allen, die darin wohnen.
\par 17 Denn siehe, ich will Schlangen und Basilisken unter euch senden, die nicht zu beschwören sind; die sollen euch stechen, spricht der HERR.
\par 18 Was mag mich in meinem Jammer erquicken? Mein Herz ist krank.
\par 19 Siehe, die Tochter meines Volks wird schreien aus fernem Lande her: "Will denn er HERR nicht mehr Gott sein zu Zion, oder soll sie keinen König mehr haben?" Ja, warum haben sie mich so erzürnt durch ihre Bilder und fremde, unnütze Gottesdienste?
\par 20 "Die Ernte ist vergangen, der Sommer ist dahin, und uns ist keine Hilfe gekommen."
\par 21 Mich jammert herzlich, daß mein Volk so verderbt ist; ich gräme mich und gehabe mich übel.
\par 22 Ist denn keine Salbe in Gilead, oder ist kein Arzt da? Warum ist denn die Tochter meines Volks nicht geheilt?

\chapter{9}

\par 1 Ach, daß ich Wasser genug hätte in meinem Haupte und meine Augen Tränenquellen wären, daß ich Tag und Nacht beweinen möchte die Erschlagenen in meinem Volk!
\par 2 Ach, daß ich eine Herberge hätte in der Wüste, so wollte ich mein Volk verlassen und von ihnen ziehen! Denn es sind eitel Ehebrecher und ein frecher Haufe.
\par 3 Sie schießen mit ihren Zungen eitel Lüge und keine Wahrheit und treiben's mit Gewalt im Lande und gehen von einer Bosheit zur andern und achten mich nicht, spricht der HERR.
\par 4 Ein jeglicher hüte sich vor seinem Freunde und traue auch seinem Bruder nicht; denn ein Bruder unterdrückt den andern, und ein Freund verrät den andern.
\par 5 Ein Freund täuscht den andern und reden kein wahres Wort; sie fleißigen sich darauf, wie einer den andern betrüge, und ist ihnen nicht leid, daß sie es ärger machen können.
\par 6 Es ist allenthalben eitel Trügerei unter ihnen, und vor Trügerei wollen sie mich nicht kennen, spricht der HERR.
\par 7 Darum spricht der HERR Zebaoth also: Siehe, ich will sie schmelzen und prüfen. Denn was soll ich sonst tun, wenn ich ansehe die Tochter meines Volks?
\par 8 Ihre falschen Zungen sind mörderische Pfeile; mit ihrem Munde reden sie freundlich gegen den Nächsten, aber im Herzen lauern sie auf ihn.
\par 9 Sollte ich nun solches nicht heimsuchen an ihnen, spricht der HERR, und meine Seele sollte sich nicht rächen an solchem Volk, wie dies ist?
\par 10 Ich muß auf den Bergen weinen und heulen und bei den Hürden in der Wüste klagen; denn sie sind so gar verheert, daß niemand da wandelt und man auch nicht ein Vieh schreien hört. Es ist beides, Vögel des Himmels und das Vieh, alles weg.
\par 11 Und ich will Jerusalem zum Steinhaufen und zur Wohnung der Schakale machen und will die Städte Juda's wüst machen, daß niemand darin wohnen soll.
\par 12 Wer nun weise wäre und ließe es sich zu Herzen gehen und verkündigte, was des HERRN Mund zu ihm sagt, warum das Land verderbt und verheert wird wie eine Wüste, da niemand wandelt!
\par 13 Und der HERR sprach: Darum daß sie mein Gesetz verlassen, daß ich ihnen vorgelegt habe, und gehorchen meiner Rede nicht, leben auch nicht darnach,
\par 14 sondern folgen ihres Herzens Gedünken und den Baalim, wie sie ihre Väter gelehrt haben:
\par 15 darum spricht der HERR Zebaoth, der Gott Israels, also: Siehe ich will dies Volk mit Wermut speisen und mit Galle tränken;
\par 16 ich will sie unter die Heiden zerstreuen, welche weder sie noch ihre Väter gekannt haben, und will das Schwert hinter sie schicken, bis daß es aus mit ihnen sei.
\par 17 So spricht der HERR Zebaoth: Schaffet und bestellt Klageweiber, daß sie kommen, und schickt nach denen, die es wohl können,
\par 18 daß sie eilend um uns klagen, daß unsre Augen von Tränen rinnen und unsre Augenlider von Wasser fließen,
\par 19 daß man ein klägliches Geschrei höre zu Zion: Ach, wie sind wir so gar verstört und zu Schanden geworden! Wir müssen das Land räumen; denn sie haben unsere Wohnungen geschleift.
\par 20 So höret nun, ihr Weiber, des HERRN Wort und nehmet zu Ohren seines Mundes Rede; lehret eure Töchter weinen, und eine lehre die andere klagen:
\par 21 Der Tod ist zu unseren Fenstern eingefallen und in unsere Paläste gekommen, die Kinder zu würgen auf der Gasse und die Jünglinge auf der Straße.
\par 22 So spricht der HERR: Sage: Der Menschen Leichname sollen liegen wie der Mist auf dem Felde und wie Garben hinter dem Schnitter, die niemand sammelt.
\par 23 So spricht der HERR: Ein Weiser rühme sich nicht seiner Weisheit, ein Starker rühme sich nicht seiner Stärke, ein Reicher rühme sich nicht seines Reichtums;
\par 24 sondern wer sich rühmen will, der rühme sich des, daß er mich wisse und kenne, daß ich der HERR bin, der Barmherzigkeit, Recht und Gerechtigkeit übt auf Erden; denn solches gefällt mir, spricht der HERR.
\par 25 Siehe, es kommt die Zeit, spricht der HERR, daß ich heimsuchen werde alle, die Beschnittenen mit den Unbeschnittenen:
\par 26 Ägypten, Juda, Edom, die Kinder Ammon, Moab und alle, die das Haar rundumher abschneiden, die in der Wüste wohnen. Denn alle Heiden haben unbeschnittenen Vorhaut; aber das ganze Israel hat ein unbeschnittenes Herz.

\chapter{10}

\par 1 Höret, was der HERR zu euch vom Hause Israel redet.
\par 2 So spricht der HERR: Ihr sollt nicht nach der Heiden Weise lernen und sollt euch nicht fürchten vor den Zeichen des Himmels, wie die Heiden sich fürchten.
\par 3 Denn der Heiden Satzungen sind lauter Nichts. Denn sie hauen im Walde einen Baum, und der Werkmeister macht Götter mit dem Beil
\par 4 und schmückt sie mit Silber und Gold und heftet sie mit Nägeln und Hämmern, daß sie nicht umfallen.
\par 5 Es sind ja nichts als überzogene Säulen. Sie können nicht reden; so muß man sie auch tragen, denn sie können nicht gehen. Darum sollt ihr euch nicht vor ihnen fürchten: denn sie können weder helfen noch Schaden tun.
\par 6 Aber dir, HERR, ist niemand gleich; du bist groß, und dein Name ist groß, und kannst es mit der Tat beweisen.
\par 7 Wer sollte dich nicht fürchten, du König der Heiden? Dir sollte man gehorchen; denn es ist unter allen Weisen der Heiden und in allen Königreichen deinesgleichen nicht.
\par 8 Sie sind allzumal Narren und Toren; denn ein Holz muß ja ein nichtiger Gottesdienst sein.
\par 9 Silbernes Blech bringt man aus Tharsis, Gold aus Uphas, durch den Meister und Goldschmied zugerichtet; blauen und roten Purpur zieht man ihm an, und ist alles der Weisen Werk.
\par 10 Aber der HERR ist ein rechter Gott, ein lebendiger Gott, ein ewiger König. Vor seinem Zorn bebt die Erde, und die Heiden können sein Drohen nicht ertragen.
\par 11 So sprecht nun zu ihnen also: Die Götter, die Himmel und Erde nicht gemacht haben, müssen vertilgt werden von der Erde und unter dem Himmel.
\par 12 Er hat aber die Erde durch seine Kraft gemacht und den Weltkreis bereitet durch seine Weisheit und den Himmel ausgebreitet durch seinen Verstand.
\par 13 Wenn er donnert, so ist des Wassers die Menge unter dem Himmel, und er zieht die Nebel auf vom Ende der Erde; er macht die Blitze im Regen und läßt den Wind kommen aus seinen Vorratskammern.
\par 14 Alle Menschen sind Narren mit ihrer Kunst, und alle Goldschmiede bestehen mit Schanden mit ihren Bildern; denn ihre Götzen sind Trügerei und haben kein Leben.
\par 15 Es ist eitel Nichts und ein verführerisches Werk; sie müssen umkommen, wenn sie heimgesucht werden.
\par 16 Aber also ist der nicht, der Jakobs Schatz ist; sondern er ist's, der alles geschaffen hat, und Israel ist sein Erbteil. Er heißt HERR Zebaoth.
\par 17 Tue deinen Kram weg aus dem Lande, die du wohnest in der Feste.
\par 18 Denn so spricht der HERR: Siehe, ich will die Einwohner des Landes auf diesmal wegschleudern und will sie ängsten, daß sie es fühlen sollen.
\par 19 Ach mein Jammer und mein Herzeleid! Ich denke aber: Es ist meine Plage; ich muß sie leiden.
\par 20 Meine Hütte ist zerstört, und alle meine Seile sind zerrissen. Meine Kinder sind von mir gegangen und nicht mehr da. Niemand ist, der meine Hütte wieder aufrichte und mein Gezelt aufschlage.
\par 21 Denn die Hirten sind zu Narren geworden und fragen nach dem HERRN nicht; darum können sie auch nichts Rechtes lehren, und ihre ganze Herde ist zerstreut.
\par 22 Siehe, es kommt ein Geschrei daher und ein großes Beben aus dem Lande von Mitternacht, daß die Städte Juda's verwüstet und zur Wohnung der Schakale werden sollen.
\par 23 Ich weiß, HERR, daß des Menschen Tun steht nicht in seiner Gewalt, und steht in niemands Macht, wie er wandle oder seinen Gang richte.
\par 24 Züchtige mich, HERR, doch mit Maßen und nicht in deinem Grimm, auf daß du mich nicht aufreibest.
\par 25 Schütte aber deinen Zorn über die Heiden, so dich nicht kennen, und über die Geschlechter, so deinen Namen nicht anrufen. Denn sie haben Jakob aufgefressen und verschlungen; sie haben ihn weggeräumt und seine Wohnung verwüstet.

\chapter{11}

\par 1 Dies ist das Wort, das zu Jeremia geschah vom HERRN, und sprach:
\par 2 Höret die Worte dieses Bundes, daß ihr sie denen in Juda und den Bürgern zu Jerusalem saget.
\par 3 Und sprich zu ihnen: So spricht der HERR, der Gott Israels: Verflucht sei, wer nicht gehorcht den Worten dieses Bundes,
\par 4 den ich euren Vätern gebot des Tages, da ich sie aus Ägyptenland führte, aus einem eisernen Ofen, und sprach: Gehorchet meiner Stimme und tut, wie ich euch geboten habe, so sollt ihr mein Volk sein, und ich will euer Gott sein,
\par 5 auf daß ich den Eid halten möge, den ich euren Vätern geschworen habe, ihnen zu geben ein Land, darin Milch und Honig fließt, wie es denn heutigestages steht. Ich antwortete und sprach: HERR, ja, es sei also!
\par 6 Und der HERR sprach zu mir: Predige alle diese Worte in den Städten Juda's und auf allen Gassen zu Jerusalem und sprich: Höret die Worte dieses Bundes und tut darnach!
\par 7 Denn ich habe euren Vätern gezeugt von dem Tage an, da ich sie aus Ägyptenland führte, bis auf den heutigen Tag und zeugte stets und sprach: Gehorchet meiner Stimme!
\par 8 Aber sie gehorchten nicht, neigten auch ihre Ohren nicht; sondern ein jeglicher ging nach seines bösen Herzens Gedünken. Darum habe ich auch über sie kommen lassen alle Worte dieses Bundes, den ich geboten habe zu tun, und nach dem sie doch nicht getan haben.
\par 9 Und der HERR sprach zu mir: Ich weiß wohl, wie sie in Juda und zu Jerusalem sich rotten.
\par 10 Sie kehren sich eben zu den Sünden ihrer Väter, die vormals waren, welche auch nicht gehorchen wollten meinen Worten und folgten auch andern Göttern nach und dienten ihnen. Also hat das Haus Israel und das Haus Juda meinen Bund gebrochen, den ich mit ihren Vätern gemacht habe.
\par 11 Darum siehe, spricht der HERR, ich will ein Unglück über sie gehen lassen, dem sie nicht sollen entgehen können; und wenn sie zu mir Schreien, will ich sie nicht hören.
\par 12 So laß denn die Städte Juda's und die Bürger zu Jerusalem hingehen und zu ihren Göttern schreien, denen sie geräuchert haben; aber sie werden ihnen nicht helfen in ihrer Not.
\par 13 Denn so manche Stadt, so manche Götter hast du, Juda; und so manche Gassen zu Jerusalem sind, so manchen Schandaltar habt ihr aufgerichtet, dem Baal zu räuchern.
\par 14 So bitte du nun nicht für dieses Volk und tue kein Flehen noch Gebet für sie; denn ich will sie nicht hören, wenn sie zu mir schreien in ihrer Not.
\par 15 Was haben meine Freunde in meinem Haus zu schaffen? Sie treiben alle Schalkheit und meinen, das heilige Fleisch soll es von ihnen nehmen; und wenn sie übeltun, sind sie guter Dinge darüber.
\par 16 Der HERR nannte dich einen grünen, schönen, fruchtbaren Ölbaum; aber nun hat er mit einem Mordgeschrei ein Feuer um ihn lassen anzünden, daß seine Äste verderben müssen.
\par 17 Denn der HERR Zebaoth, der dich gepflanzt hat, hat dir ein Unglück gedroht um der Bosheit willen des Hauses Israel und des Hauses Juda, welche sie treiben, daß sie mich erzürnen mit ihrem Räuchern, das sie dem Baal tun.
\par 18 Der HERR hat mir's offenbart, daß ich's weiß, und zeigte mir ihr Vornehmen,
\par 19 nämlich, daß sie mich wie ein armes Schaf zur Schlachtbank führen wollen. Denn ich wußte nicht, daß sie wider mich beratschlagt hatten und gesagt: Laßt uns den Baum mit seinen Früchten verderben und ihn aus dem Lande der Lebendigen ausrotten, daß seines Namen nimmermehr gedacht werde.
\par 20 Aber du, HERR Zebaoth, du gerechter Richter, der du Nieren und Herzen prüfst, laß mich deine Rache über sie sehen; denn ich habe dir meine Sache befohlen.
\par 21 Darum spricht der HERR also wider die Männer zu Anathoth, die dir nach deinem Leben stehen und sprechen: Weissage uns nicht im Namen des HERRN, willst du anders nicht von unsern Händen sterben!
\par 22 darum spricht der HERR Zebaoth also: Siehe, ich will sie heimsuchen; ihre junge Mannschaft soll mit dem Schwert getötet werden, und ihre Söhne und Töchter sollen Hungers sterben, daß nichts von ihnen übrigbleibe;
\par 23 denn ich will über die Männer zu Anathtoth Unglück kommen lassen des Jahres, wann sie heimgesucht werden sollen.

\chapter{12}

\par 1 HERR, wenn ich gleich mit dir rechten wollte, so behältst du doch recht; dennoch muß ich vom Recht mit dir reden. Warum geht's doch den Gottlosen so wohl und die Verächter haben alles die Fülle?
\par 2 Du pflanzt sie, daß sie wurzeln und wachsen und Frucht bringen. Nahe bist du in ihrem Munde, aber ferne von ihrem Herzen;
\par 3 mich aber, HERR, kennst du und siehst mich und prüfst mein Herz vor dir. Reiße sie weg wie Schafe, daß sie geschlachtet werden; sondere sie aus, daß sie gewürgt werden.
\par 4 Wie lange soll doch das Land so jämmerlich stehen und das Gras auf dem Felde allenthalben verdorren um der Einwohner Bosheit willen, daß beide, Vieh und Vögel, nimmer da sind? denn sie sprechen: Ja, er weiß viel, wie es uns gehen wird.
\par 5 Wenn dich die müde machen, die zu Fuße gehen, wie will dir's gehen wenn du mit den Reitern laufen sollst? Und so du in dem Lande, da es Friede ist, Sicherheit suchst, was will mit dir werden bei dem stolzen Jordan?
\par 6 Denn es verachten dich auch deine Brüder und deines Vaters Haus und schreien zeter! über dich. Darum vertraue du ihnen nicht, wenn sie gleich freundlich mit dir reden.
\par 7 Ich habe mein Haus verlassen müssen und mein Erbe meiden, und was meine Seele liebt, in der Feinde Hand geben.
\par 8 Mein Erbe ist mir geworden wie ein Löwe im Walde und brüllt wider mich; darum bin ich ihm gram geworden.
\par 9 Mein Erbe ist wie der sprenklige Vogel, um welchen sich die Vögel sammeln. Wohlauf, sammelt euch, alle Feldtiere, kommt und fresset.
\par 10 Es haben Hirten, und deren viel, meinen Weinberg verderbt und meinen Acker zertreten; sie haben meinen schönen Acker zur Wüste gemacht, sie haben's öde gemacht.
\par 11 Ich sehe bereits wie es so jämmerlich verwüstet ist; ja das ganze Land ist wüst. Aber es will's niemand zu Herzen nehmen.
\par 12 Denn die Verstörer fahren daher über alle Hügel der Wüste, und das fressende Schwert des HERRN von einem Ende des Landes bis zum andern; und kein Fleisch wird Frieden haben.
\par 13 Sie säen Weizen, aber Disteln werden sie ernten; sie lassen's sich sauer werden, aber sie werden's nicht genießen; sie werden ihres Einkommens nicht froh werden vor dem grimmigen Zorn des HERRN.
\par 14 So spricht der HERR wider alle meine bösen Nachbarn, so das Erbteil antasten, das ich meinem Volk Israel ausgeteilt habe: Siehe, ich will sie aus ihrem Lande ausreißen und das Haus Juda aus ihrer Mitte reißen.
\par 15 Und wenn ich sie nun ausgerissen habe, will ich mich wiederum über sie erbarmen und will einen jeglichen zu seinem Erbteil und in sein Land wiederbringen.
\par 16 Und soll geschehen, wo sie von meinem Volk lernen werden, daß sie schwören bei meinem Namen: "So wahr der HERR lebt!", wie sie zuvor mein Volk gelehrt haben schwören bei Baal, so sollen sie unter meinem Volk erbaut werden.
\par 17 Wo sie aber nicht hören wollen, so will ich solches Volk ausreißen und umbringen, spricht der HERR.

\chapter{13}

\par 1 So spricht der HERR zu mir: Gehe hin und kaufe dir einen leinenen Gürtel und gürte damit deine Lenden und mache ihn nicht naß.
\par 2 Und ich kaufte einen Gürtel nach dem Befehl des HERRN und gürtete ihn um meine Lenden.
\par 3 Da geschah des HERRN Wort zum andernmal zu mir und sprach:
\par 4 Nimm den Gürtel, den du gekauft und um deine Lenden gegürtet hast, und mache dich auf und gehe hin an den Euphrat und verstecke ihn daselbst in einem Steinritz.
\par 5 Ich ging hin und versteckte ihn am Euphrat, wie mir der HERR geboten hatte.
\par 6 Nach langer Zeit aber sprach der HERR zu mir: Mache dich auf und hole den Gürtel wieder, den ich dich hieß daselbst verstecken.
\par 7 Ich ging hin an den Euphrat und grub auf und nahm den Gürtel von dem Ort, dahin ich ihn versteckt hatte; und siehe, der Gürtel war verdorben, daß er nichts mehr taugte.
\par 8 Da geschah des HERRN Wort zu mir und sprach:
\par 9 So spricht der HERR: Eben also will ich auch verderben die große Hoffart Juda's und Jerusalems.
\par 10 Das böse Volk, das meine Worte nicht hören will, sondern gehen hin nach Gedünken ihres Herzens und folgen andern Göttern, daß sie ihnen dienen und sie anbeten: sie sollen werden wie der Gürtel, der nichts mehr taugt.
\par 11 Denn gleichwie ein Mann den Gürtel um seine Lenden bindet, also habe ich, spricht der HERR, das ganze Haus Israel und das ganze Haus Juda um mich gegürtet, daß sie mein Volk sein sollten, mir zu einem Namen, zu Lob und Ehren; aber sie wollen nicht hören.
\par 12 So sage ihnen nun dies Wort: So spricht der HERR, der Gott Israels: Es sollen alle Krüge mit Wein gefüllt werden. So werden sie zu dir sagen: Wer weiß das nicht, daß man alle Krüge mit Wein füllen soll?
\par 13 So sprich zu ihnen: So spricht der HERR: Siehe, ich will alle, die in diesem Lande wohnen, die Könige, die auf dem Stuhl Davids sitzen, die Priester und Propheten und alle Einwohner zu Jerusalem füllen, daß sie trunken werden sollen;
\par 14 und will einen mit dem andern, die Väter samt den Kindern, verstreuen, spricht der HERR; und will weder schonen noch übersehen noch barmherzig sein über ihrem Verderben.
\par 15 So höret nun und merket auf und trotzt nicht; denn der HERR hat's geredet.
\par 16 Gebet dem HERRN, eurem Gott, die Ehre, ehe denn es finster werde, und ehe eure Füße sich an den dunklen Bergen stoßen, daß ihr des Lichts wartet, so er's doch gar finster und dunkel machen wird.
\par 17 Wollt ihr aber solches nicht hören, so muß meine Seele heimlich weinen über solche Hoffart; meine Augen müssen von Tränen fließen, daß des HERRN Herde gefangen wird.
\par 18 Sage dem König und der Königin: Setzt euch herunter; denn die Krone der Herrlichkeit ist euch von eurem Haupt gefallen.
\par 19 Die Städte gegen Mittag sind verschlossen, und ist niemand, der sie auftue; das ganze Juda ist rein weggeführt.
\par 20 Hebt eure Augen auf und sehet, wie sie von Mitternacht daherkommen. Wo ist nun die Herde, so dir befohlen war, deine herrliche Herde?
\par 21 Was willst du sagen, wenn er dich so heimsuchen wird? Denn du hast sie so gewöhnt wider dich, daß sie Fürsten und Häupter sein wollen. Was gilt's? es wird dich Angst ankommen wie ein Weib in Kindsnöten.
\par 22 Und wenn du in deinem Herzen sagen willst: "Warum begegnet doch mir solches?" Um der Menge willen deiner Missetaten sind dir deine Säume aufgedeckt und ist deinen Fersen Gewalt geschehen.
\par 23 Kann auch ein Mohr seine Haut wandeln oder ein Parder seine Flecken? So könnt ihr auch Gutes tun, die ihr des Bösen gewohnt seid.
\par 24 Darum will ich sie zerstreuen wie Stoppeln, die vor dem Winde aus der Wüste verweht werden.
\par 25 Das soll dein Lohn sein und dein Teil, den ich dir zugemessen habe, spricht der HERR. Darum daß du mein vergessen hast und verlässest dich auf Lügen,
\par 26 so will ich auch deine Säume hoch aufdecken, daß man deine Schande sehen muß.
\par 27 Denn ich habe gesehen deine Ehebrecherei, dein Geilheit, deine freche Hurerei, ja, deine Greuel auf Hügeln und auf Äckern. Weh dir, Jerusalem! Wann wirst du doch endlich rein werden?

\chapter{14}

\par 1 Dies ist das Wort, das der HERR zu Jeremia sagte von der teuren Zeit:
\par 2 Juda liegt jämmerlich, ihre Tore stehen elend; es steht kläglich auf dem Lande, und ist zu Jerusalem ein großes Geschrei.
\par 3 Die Großen schicken die Kleinen nach Wasser; aber wenn sie zum Brunnen kommen, finden sie kein Wasser und bringen ihre Gefäße leer wieder; sie gehen traurig und betrübt und verhüllen ihre Häupter.
\par 4 Darum daß die Erde lechzt, weil es nicht regnet auf die Erde, gehen die Ackerleute traurig und verhüllen ihre Häupter.
\par 5 Denn auch die Hinden, die auf dem Felde werfen, verlassen ihre Jungen, weil kein Gras wächst.
\par 6 Das Wild steht auf den Hügeln und schnappt nach der Luft wie die Drachen und verschmachtet, weil kein Kraut wächst.
\par 7 Ach HERR, unsre Missetaten haben's ja verdient; aber hilf doch um deines Namens willen! denn unser Ungehorsam ist groß; damit wir wider dich gesündigt haben.
\par 8 Du bist der Trost Israels und sein Nothelfer; warum stellst du dich, als wärest du ein Gast im Lande und ein Fremder, der nur über Nacht darin bleibt?
\par 9 Warum stellst du dich wie ein Held, der verzagt ist, und wie ein Riese, der nicht helfen kann? Du bist ja doch unter uns, HERR, und wir heißen nach deinem Namen; verlaß uns nicht!
\par 10 So spricht der HERR von diesem Volk: Sie laufen gern hin und wieder und bleiben nicht gern daheim; darum will sie der HERR nicht, sondern er denkt nun an ihre Missetat und will ihre Sünden heimsuchen.
\par 11 Und der HERR sprach zu mir: Du sollst nicht für dies Volk um Gnade bitten.
\par 12 Denn ob sie gleich fasten, so will ich doch ihr Flehen nicht hören; und ob sie Brandopfer und Speisopfer bringen, so gefallen sie mir doch nicht, sondern ich will sie mit Schwert, Hunger und Pestilenz aufreiben.
\par 13 Da sprach ich: Ach HERR HERR, siehe, die Propheten sagen ihnen: Ihr werdet kein Schwert sehen und keine Teuerung bei euch haben; sondern ich will euch guten Frieden geben an diesem Ort.
\par 14 Und der HERR sprach zu mir: Die Propheten weissagen falsch in meinem Namen; ich habe sie nicht gesandt und ihnen nichts befohlen und nichts mit ihnen geredet. Sie predigen euch falsche Gesichte, Deutungen, Abgötterei und ihres Herzens Trügerei.
\par 15 Darum so spricht der HERR von den Propheten, die in meinem Namen weissagen, so ich sie doch nicht gesandt habe, und die dennoch predigen, es werde kein Schwert und keine Teuerung in dies Land kommen: Solche Propheten sollen sterben durch Schwert und Hunger.
\par 16 Und die Leute, denen sie weissagen, sollen vom Schwert und Hunger auf den Gassen zu Jerusalem hin und her liegen, daß sie niemand begraben wird, also auch ihre Weiber, Söhne und Töchter; und ich will ihre Bosheit über sie schütten.
\par 17 Und du sollst zu ihnen sagen dies Wort: Meine Augen fließen von Tränen Tag und Nacht und hören nicht auf; denn die Jungfrau, die Tochter meines Volks, ist greulich zerplagt und jämmerlich geschlagen.
\par 18 Gehe ich hinaus aufs Feld, siehe, so liegen da Erschlagene mit dem Schwert; komme ich in die Stadt, so liegen da vor Hunger Verschmachtete. Denn es müssen auch die Propheten, dazu auch die Priester in ein Land ziehen, das sie nicht kennen.
\par 19 Hast du denn Juda verworfen, oder hat deine Seele einen Ekel an Zion? Warum hast du uns denn so geschlagen, daß es niemand heilen kann? Wir hofften, es sollte Friede werden; so kommt nichts Gutes. Wir hofften, wir sollten heil werden; aber siehe, so ist mehr Schaden da.
\par 20 HERR, wir erkennen unser gottlos Wesen und unsrer Väter Missetat; denn wir haben wider dich gesündigt.
\par 21 Aber um deines Namens willen laß uns nicht geschändet werden; laß den Thron deiner Herrlichkeit nicht verspottet werden; gedenke doch und laß deinen Bund mit uns nicht aufhören.
\par 22 Es ist doch ja unter der Heiden Götzen keiner, der Regen könnte geben; auch der Himmel kann nicht regnen. Du bist doch ja der HERR, unser Gott, auf den wir hoffen; denn du kannst solches alles tun.

\chapter{15}

\par 1 Und der HERR sprach zu mir: Und wenngleich Mose und Samuel vor mir stünden, so habe ich doch kein Herz zu diesem Volk; treibe sie weg von mir und laß sie hinfahren!
\par 2 Und wenn sie zu dir sagen: Wo sollen wir hin? so sprich zu ihnen: So spricht der HERR: Wen der Tod trifft, den treffe er; wen das Schwert trifft, den treffe es; wen der Hunger trifft, den treffe er; wen das Gefängnis trifft, den treffe es.
\par 3 Denn ich will sie heimsuchen mit vielerlei Plagen, spricht der HERR: mit dem Schwert, daß sie erwürgt werden; mit Hunden, die sie schleifen sollen; mit den Vögeln des Himmels und mit Tieren auf Erden, daß sie gefressen und vertilgt werden sollen.
\par 4 Und ich will sie in allen Königreichen auf Erden hin und her treiben lassen um Manasses willen, des Sohnes Hiskias, des Königs in Juda, um deswillen, was er zu Jerusalem begangen hat.
\par 5 Wer will denn sich dein erbarmen, Jerusalem? Wer wird denn Mitleiden mit dir haben? Wer wird denn hingehen und dir Frieden wünschen?
\par 6 Du hast mich verlassen, spricht der HERR, und bist von mir abgefallen; darum habe ich meine Hand ausgestreckt wider dich, daß ich dich verderben will; ich bin des Erbarmens müde.
\par 7 Ich will sie mit der Wurfschaufel zum Lande hinausworfeln und will mein Volk, so von seinem Wesen sich nicht bekehren will, zu eitel Waisen machen und umbringen.
\par 8 Es sollen mir mehr Witwen unter ihnen werden, denn Sand am Meer ist. Ich will über die Mutter der jungen Mannschaft kommen lassen einen offenbaren Verderber und die Stadt damit plötzlich und unversehens überfallen lassen,
\par 9 daß die, die sieben Kinder hat, soll elend sein und von Herzen seufzen. Denn ihre Sonne soll bei hohem Tage untergehen, daß ihr Ruhm und ihre Freude ein Ende haben soll. Und die übrigen will ich ins Schwert geben vor ihren Feinden, spricht der HERR.
\par 10 Ach, meine Mutter, daß du mich geboren hast, wider den jedermann hadert und zankt im ganzen Lande! Habe ich doch weder auf Wucher geliehen noch genommen; doch flucht mir jedermann.
\par 11 Der HERR sprach: Wohlan, ich will euer etliche übrigbehalten, denen es soll wieder wohl gehen, und will euch zu Hilfe kommen in der Not und Angst unter den Feinden.
\par 12 Meinst du nicht, daß etwa ein Eisen sei, welches könnte das Eisen und Erz von Mitternacht zerschlagen?
\par 13 Ich will aber zuvor euer Gut und eure Schätze zum Raub geben, daß ihr nichts dafür kriegen sollt, und das um aller eurer Sünden willen, die ihr in allen euren Grenzen begangen habt.
\par 14 Und ich will euch zu euren Feinden bringen in ein Land, das ihr nicht kennt; denn es ist das Feuer in meinem Zorn über euch angegangen.
\par 15 Ach HERR, du weißt es; gedenke an mich und nimm dich meiner an und räche mich an meinen Verfolgern. Nimm mich auf und verzieh nicht deinem Zorn über sie; denn du weißt, daß ich um deinetwillen geschmäht werde.
\par 16 Dein Wort ward mir Speise, da ich's empfing; und dein Wort ist meines Herzens Freude und Trost; denn ich bin ja nach deinem Namen genannt; HERR, Gott Zebaoth.
\par 17 Ich habe mich nicht zu den Spöttern gesellt noch mich mit ihnen gefreut, sondern bin allein geblieben vor deiner Hand; denn du hattest mich gefüllt mit deinem Grimm.
\par 18 Warum währt doch mein Leiden so lange, und meine Wunden sind so gar böse, daß sie niemand heilen kann? Du bist mir geworden wie ein Born, der nicht mehr quellen will.
\par 19 Darum spricht der HERR also: Wo du dich zu mir hältst, so will ich mich zu dir halten, und du sollst mein Prediger bleiben. Und wo du die Frommen lehrst sich sondern von den bösen Leuten, so sollst du mein Mund sein. Und ehe du solltest zu ihnen fallen, so müssen sie eher zu dir fallen.
\par 20 Denn ich habe dich wider dies Volk zur festen, ehernen Mauer gemacht; ob sie wider dich streiten, sollen sie dir doch nichts anhaben; denn ich bin bei dir, daß ich dir helfe und dich errette, spricht der HERR,
\par 21 und will dich erretten aus der Hand der Bösen und erlösen aus der Hand der Tyrannen.

\chapter{16}

\par 1 Und des HERRN Wort geschah zu mir und sprach:
\par 2 Du sollst kein Weib nehmen und weder Söhne noch Töchter zeugen an diesem Ort.
\par 3 Denn so spricht der HERR von den Söhnen und Töchtern, die an diesem Ort geboren werden, dazu von ihren Müttern die sie gebären, und von ihren Vätern, die sie zeugen in diesem Lande:
\par 4 Sie sollen an Krankheiten sterben und weder beklagt noch begraben werden, sondern sollen Dung werden auf dem Lande, dazu durch Schwert und Hunger umkommen, und ihre Leichname sollen der Vögel des Himmels und der Tiere auf Erden Speise sein.
\par 5 Denn so spricht der HERR: Du sollst nicht zum Trauerhaus gehen und sollst auch nirgend hin zu Klagen gehen noch Mitleiden über sie haben; denn ich habe meinen Frieden von diesem Volk weggenommen, spricht der HERR, samt meiner Gnade und Barmherzigkeit,
\par 6 daß beide, groß und klein, sollen in diesem Lande sterben und nicht begraben noch beklagt werden, und niemand wird sich über sie zerritzen noch kahl scheren.
\par 7 Und man wird auch nicht unter sie Brot austeilen bei der Klage, sie zu trösten über die Leiche, und ihnen auch nicht aus dem Trostbecher zu trinken geben über Vater und Mutter.
\par 8 Du sollst auch in kein Trinkhaus gehen, bei ihnen zu sitzen, weder zu essen noch zu trinken.
\par 9 Denn so spricht der HERR Zebaoth, der Gott Israels: Siehe, ich will an diesem Ort wegnehmen vor euren Augen und bei eurem Leben die Stimme der Freude und Wonne, die Stimme des Bräutigams und der Braut.
\par 10 Und wenn du solches alles diesem Volk gesagt hast und sie zu dir sprechen werden: Warum redet der HERR über uns all dies Unglück? welches ist die Missetat und Sünde, damit wir wider den HERRN, unsern Gott, gesündigt haben?
\par 11 sollst du ihnen sagen: Darum daß eure Väter mich verlassen haben, spricht der HERR, und andern Göttern gefolgt sind, ihnen gedient und sie angebetet, mich aber verlassen und mein Gesetz nicht gehalten haben
\par 12 und ihr noch ärger tut als eure Väter. Denn siehe, ein jeglicher lebt nach seines bösen Herzens Gedünken, daß er mir nicht gehorche.
\par 13 Darum will ich euch aus diesem Lande stoßen in ein Land, davon weder ihr noch eure Väter gewußt haben; daselbst sollt ihr andern Göttern dienen Tag und Nacht, dieweil ich euch keine Gnade erzeigen will.
\par 14 Darum siehe, es kommt die Zeit, spricht der HERR, daß man nicht mehr sagen wird: So wahr der HERR lebt, der die Kinder Israel aus Ägyptenland geführt hat!
\par 15 sondern: So wahr der HERR lebt, der die Kinder Israel geführt hat aus dem Lande der Mitternacht und aus allen Ländern, dahin er sie verstoßen hatte! Denn ich will sie wiederbringen in das Land, das ich ihren Vätern gegeben habe.
\par 16 Siehe, ich will viel Fischer aussenden, spricht der HERR, die sollen sie fischen; und darnach will ich viel Jäger aussenden, die sollen sie fangen auf allen Bergen und auf allen Hügeln und in allen Steinritzen.
\par 17 Denn meine Augen sehen auf ihre Wege, daß sie vor mir sich nicht verhehlen können; und ihre Missetat ist vor meinen Augen unverborgen.
\par 18 Aber zuvor will ich ihre Missetat und Sünde zwiefach bezahlen, darum daß sie mein Land mit den Leichen ihrer Abgötterei verunreinigt und mein Erbe mit Greueln angefüllt haben.
\par 19 HERR, du bist meine Stärke und Kraft und meine Zuflucht in der Not. Die Heiden werden zu mir kommen von der Welt Enden und sagen: Unsre Väter haben falsche und nichtige Götter gehabt, die nichts nützen können.
\par 20 Wie kann ein Mensch Götter machen, die doch keine Götter sind?
\par 21 Darum siehe, nun will ich sie lehren und meine Hand und Gewalt ihnen kundtun, daß sie erfahren sollen, ich heiße der HERR.

\chapter{17}

\par 1 Die Sünde Juda's ist geschrieben mit eisernen Griffeln, und spitzigen Demanten geschrieben, und auf die Tafel ihres Herzens gegraben und auf die Hörner an ihren Altären,
\par 2 daß die Kinder gedenken sollen derselben Altäre und Ascherabilder bei den grünen Bäumen, auf den hohen Bergen.
\par 3 Aber ich will deine Höhen, beide, auf den Bergen und Feldern, samt deiner Habe und allen deinen Schätzen zum Raube geben um der Sünde willen, in allen deinen Grenzen begangen.
\par 4 Und du sollst aus deinem Erbe verstoßen werden, das ich dir gegeben habe, und ich will dich zu Knechten deiner Feinde machen in einem Lande, das du nicht kennst; denn ihr habt ein Feuer meines Zorns angezündet, das ewiglich brennen wird.
\par 5 So spricht der HERR: Verflucht ist der Mann, der sich auf Menschen verläßt und hält Fleisch für seinen Arm, und mit seinem Herzen vom HERRN weicht.
\par 6 Der wird sein wie die Heide in der Wüste und wird nicht sehen den zukünftigen Trost, sondern bleiben in der Dürre, in der Wüste, in einem unfruchtbaren Lande, da niemand wohnt.
\par 7 Gesegnet aber ist der Mann, der sich auf den HERRN verläßt und des Zuversicht der HERR ist.
\par 8 Der ist wie ein Baum, am Wasser gepflanzt und am Bach gewurzelt. Denn obgleich eine Hitze kommt, fürchtet er sich doch nicht, sondern seine Blätter bleiben grün, und sorgt nicht, wenn ein dürres Jahr kommt sondern er bringt ohne Aufhören Früchte.
\par 9 Es ist das Herz ein trotzig und verzagtes Ding; wer kann es ergründen?
\par 10 Ich, der HERR, kann das Herz ergründen und die Nieren prüfen und gebe einem jeglichen nach seinem Tun, nach den Früchten seiner Werke.
\par 11 Denn gleichwie ein Vogel, der sich über Eier setzt und brütet sie nicht aus, also ist der, so unrecht Gut sammelt; denn er muß davon, wenn er's am wenigsten achtet, und muß zuletzt Spott dazu haben.
\par 12 Aber die Stätte unsers Heiligtums, der Thron göttlicher Ehre, ist allezeit fest geblieben.
\par 13 Denn, Herr, du bist die Hoffnung Israels. Alle, die dich verlassen, müssen zu Schanden werden, und die Abtrünnigen müssen in die Erde geschrieben werden; denn sie verlassen den HERRN, die Quelle des lebendigen Wassers.
\par 14 Heile du mich, HERR, so werde ich heil; hilf du mir, so ist mir geholfen; denn du bist mein Ruhm.
\par 15 Siehe, sie sprechen zu mir: Wo ist denn des HERRN Wort? Laß es doch kommen!
\par 16 Aber ich bin nicht von dir geflohen, daß ich nicht dein Hirte wäre; so habe ich den bösen Tag nicht begehrt, das weißt du; was ich gepredigt habe, das ist recht vor dir.
\par 17 Sei du nur nicht schrecklich, meine Zuversicht in der Not!
\par 18 Laß sie zu Schanden werden, die mich verfolgen, und mich nicht; laß sie erschrecken, und mich nicht; laß den Tag des Unglücks über sie kommen und zerschlage sie zwiefach!
\par 19 So spricht der HERR zu mir: Gehe hin und tritt unter das Tor des Volks, dadurch die Könige Juda's aus und ein gehen, und unter alle Tore zu Jerusalem
\par 20 und sprich zu ihnen: Höret des HERRN Wort, ihr Könige Juda's und ganz Juda und alle Einwohner zu Jerusalem, so zu diesem Tor eingehen.
\par 21 So spricht der HERR: Hütet euch und tragt keine Last am Sabbattage durch die Tore hinein zu Jerusalem
\par 22 und führt keine Last am Sabbattage aus euren Häusern und tut keine Arbeit, sondern heiliget den Sabbattag, wie ich euren Vätern geboten habe.
\par 23 Aber sie hören nicht und neigen ihre Ohren nicht, sondern bleiben halsstarrig, daß sie mich ja nicht hören noch sich ziehen lassen.
\par 24 So ihr mich hören werdet, spricht der HERR, daß ihr keine Last traget des Sabbattages durch dieser Stadt Tore ein, sondern ihn heiliget, daß ihr keine Arbeit an demselben Tage tut:
\par 25 so sollen auch durch dieser Stadt Tore aus und ein gehen Könige und Fürsten, die auf dem Stuhl Davids sitzen, und reiten und fahren, auf Wagen und Rossen, sie und ihre Fürsten samt allen, die in Juda und Jerusalem wohnen; und soll diese Stadt ewiglich bewohnt werden;
\par 26 und sollen kommen aus den Städten Juda's, und die um Jerusalem her liegen, und aus dem Lande Benjamin, aus den Gründen und von den Gebirgen und vom Mittag, die da bringen Brandopfer, Schlachtopfer, Speisopfer und Weihrauch zum Hause des HERRN.
\par 27 Werdet ihr mich aber nicht hören, daß ihr den Sabbattag heiliget und keine Last tragt durch die Tore zu Jerusalem ein am Sabbattage, so will ich ein Feuer unter ihren Toren anzünden, das die Häuser zu Jerusalem verzehren und nicht gelöscht werden soll.

\chapter{18}

\par 1 Dies ist das Wort, das geschah vom HERRN zu Jeremia, und sprach:
\par 2 Mache dich auf und gehe hinab in des Töpfers Haus; daselbst will ich dich meine Worte hören lassen.
\par 3 Und ich ging hinab in des Töpfers Haus, und siehe, er arbeitete eben auf der Scheibe.
\par 4 Und der Topf, den er aus dem Ton machte, mißriet ihm unter den Händen. Da machte er einen andern Topf daraus, wie es ihm gefiel.
\par 5 Da geschah des HERRN Wort zu mir und sprach:
\par 6 Kann ich nicht also mit euch umgehen, ihr vom Hause Israel, wie dieser Töpfer? spricht der HERR. Siehe, wie der Ton ist in des Töpfers Hand, also seid auch ihr vom Hause Israel in meiner Hand.
\par 7 Plötzlich rede ich wider ein Volk und Königreich, daß ich es ausrotten, zerbrechen und verderben wolle.
\par 8 Wo sich's aber bekehrt von seiner Bosheit, dawider ich rede, so soll mich auch reuen das Unglück, das ich ihm gedachte zu tun.
\par 9 Und plötzlich rede ich von einem Volk und Königreich, daß ich's bauen und pflanzen wolle.
\par 10 So es aber Böses tut vor meinen Augen, daß es meiner Stimme nicht gehorcht, so soll mich auch reuen das Gute, das ich ihm verheißen hatte zu tun.
\par 11 So sprich nun zu denen in Juda und zu den Bürgern zu Jerusalem: So spricht der HERR: Siehe, ich bereite euch ein Unglück zu und habe Gedanken wider euch: darum kehre sich ein jeglicher von seinem bösen Wesen und bessert euer Wesen und Tun.
\par 12 Aber sie sprachen: Daraus wird nichts; wir wollen nach unsern Gedanken wandeln und ein jeglicher tun nach Gedünken seines bösen Herzens.
\par 13 Darum spricht der HERR: Fragt doch unter den Heiden. Wer hat je desgleichen gehört? Daß die Jungfrau Israel so gar greuliche Dinge tut!
\par 14 Bleibt doch der Schnee länger auf den Steinen im Felde, wenn's vom Libanon herab schneit, und das Regenwasser verschießt nicht so bald, wie mein Volk vergißt.
\par 15 Sie räuchern den Göttern und richten Ärgernis an auf ihren Wegen für und für und gehen auf ungebahnten Straßen,
\par 16 auf daß ihr Land zur Wüste werde, ihnen zur ewigen Schande, daß, wer vorübergeht, sich verwundere und den Kopf schüttle.
\par 17 Denn ich will sie wie durch einen Ostwind zerstreuen vor ihren Feinden; ich will ihnen den Rücken, und nicht das Antlitz zeigen, wenn sie verderben.
\par 18 Aber sie sprechen: Kommt und laßt uns wider Jeremia ratschlagen; denn die Priester können nicht irre gehen im Gesetz, und die Weisen können nicht fehlen mit Raten, und die Propheten können nicht unrecht lehren! Kommt her, laßt uns ihn mit der Zunge totschlagen und nichts geben auf alle seine Rede!
\par 19 HERR, habe acht auf mich und höre die Stimme meiner Widersacher!
\par 20 Ist's recht, daß man Gutes mit Bösem vergilt? Denn sie haben meiner Seele eine Grube gegraben. Gedenke doch, wie ich vor dir gestanden bin, daß ich ihr Bestes redete und deinen Grimm von ihnen wendete.
\par 21 So strafe nun ihre Kinder mit Hunger und laß sie ins Schwert fallen, daß ihre Weiber ohne Kinder und Witwen seien und ihre Männer zu Tode geschlagen und ihre junge Mannschaft im Streit durchs Schwert erwürgt werde;
\par 22 daß ein Geschrei aus ihren Häusern gehört werde, wie du plötzlich habest Kriegsvolk über sie kommen lassen. Denn sie haben eine Grube gegraben, mich zu fangen, und meinen Füßen Stricke gelegt.
\par 23 Und weil du, HERR, weißt alle ihre Anschläge wider mich, daß sie mich töten wollen, so vergib ihnen ihre Missetat nicht und laß ihre Sünde vor dir nicht ausgetilgt werden. Laß sie vor dir gestürzt werden und handle mit ihnen nach deinem Zorn.

\chapter{19}

\par 1 So spricht nun der HERR: Gehe hin und kaufe dir einen irdenen Krug vom Töpfer, samt etlichen von den Ältesten des Volks und von den Ältesten der Priester,
\par 2 und gehe hinaus ins Tal Ben-Hinnom, das vor dem Ziegeltor liegt, und predige daselbst die Worte, die ich dir sage,
\par 3 und sprich: Höret des HERRN Wort, ihr Könige Juda's und Bürger zu Jerusalem! So spricht der HERR Zebaoth, der Gott Israels: Siehe, ich will ein solch Unglück über diese Stätte gehen lassen, daß, wer es hören wird, dem die Ohren klingen sollen,
\par 4 darum daß sie mich verlassen und diese Stätte einem fremden Gott gegeben haben und andern Göttern darin geräuchert haben, die weder sie noch ihre Väter noch die Könige Juda's gekannt haben, und haben die Stätte voll unschuldigen Bluts gemacht
\par 5 und haben dem Baal Höhen gebaut, ihre Kinder zu verbrennen, dem Baal zu Brandopfern, was ich ihnen weder geboten noch davon geredet habe, was auch in mein Herz nie gekommen ist.
\par 6 Darum siehe, es wird die Zeit kommen, spricht der HERR, daß man diese Stätte nicht mehr Thopheth noch das Tal Ben-Hinnom, sondern Würgetal heißen wird.
\par 7 Und ich will den Gottesdienst Juda's und Jerusalems an diesem Ort zerstören und will sie durchs Schwert fallen lassen vor ihren Feinden, unter der Hand derer, die nach ihrem Leben stehen, und will ihre Leichname den Vögeln des Himmels und den Tieren auf Erden zu fressen geben
\par 8 und will diese Stadt wüst machen und zum Spott, daß alle, die vorübergehen, werden sich verwundern über alle ihre Plage und ihrer spotten.
\par 9 Ich will sie lassen ihrer Söhne und Töchter Fleisch fressen, und einer soll des andern Fleisch fressen in der Not und Angst, damit sie ihre Feinde und die, so nach ihrem Leben stehen, bedrängen werden.
\par 10 Und du sollst den Krug zerbrechen vor den Männern, die mit dir gegangen sind,
\par 11 und sprich zu ihnen: So spricht der HERR Zebaoth: Eben wie man eines Töpfers Gefäß zerbricht, das nicht kann wieder ganz werden, so will ich dies Volk und diese Stadt auch zerbrechen; und sie sollen im Thopheth begraben werden, weil sonst kein Raum sein wird, zu begraben.
\par 12 So will ich mit dieser Stätte, spricht der HERR, und ihren Einwohnern umgehen, daß diese Stadt werden soll gleich wie das Thopheth.
\par 13 Dazu sollen ihre Häuser zu Jerusalem und die Häuser der Könige Juda's ebenso unrein werden wie die Stätte Thopheth, ja, alle Häuser, wo sie auf den Dächern geräuchert haben allem Heer des Himmels und andern Göttern Trankopfer geopfert haben.
\par 14 Und da Jeremia wieder vom Thopheth kam, dahin ihn der HERR gesandt hatte, zu weissagen, trat er in den Vorhof am Hause des HERRN und sprach zu allem Volk:
\par 15 So spricht der HERR Zebaoth, der Gott Israels: Siehe, ich will über diese Stadt und über alle ihre Städte all das Unglück kommen lassen, das ich wider sie geredet habe, darum daß sie halsstarrig sind und meine Worte nicht hören wollen.

\chapter{20}

\par 1 Da aber Pashur, ein Sohn Immers, der Priester, der zum Obersten im Hause des HERRN gesetzt war, Jeremia hörte solche Worte weissagen,
\par 2 schlug er den Propheten Jeremia und legte ihn in den Stock unter dem Obertor Benjamin, welches am Hause des HERRN ist.
\par 3 Und da es Morgen ward, zog Pashur Jeremia aus dem Stock. Da sprach Jeremia zu ihm: Der HERR heißt dich nicht Pashur, sondern Schrecken um und um.
\par 4 Denn so spricht der HERR: Siehe, ich will dich zum Schrecken machen dir selbst und allen deinen Freunden, und sie sollen fallen durchs Schwert ihrer Feinde; das sollst du mit deinen Augen sehen. Und will das ganze Juda in die Hand des Königs zu Babel übergeben; der soll euch wegführen gen Babel und mit dem Schwert töten.
\par 5 Auch will ich alle Güter dieser Stadt samt allem, was sie gearbeitet und alle Kleinode und alle Schätze der Könige Juda's in ihrer Feinde Hand geben, daß sie dieselben rauben, nehmen und gen Babel bringen.
\par 6 Und du, Pashur, sollst mit allen deinen Hausgenossen gefangen gehen und gen Babel kommen; daselbst sollst du sterben und begraben werden samt allen deinen Freunden, welchen du Lügen predigst.
\par 7 HERR, du hast mich überredet, und ich habe mich überreden lassen; du bist mir zu stark gewesen und hast gewonnen; aber ich bin darüber zum Spott geworden täglich, und jedermann verlacht mich.
\par 8 Denn seit ich geredet, gerufen und gepredigt habe von der Plage und Verstörung, ist mir des HERRN Wort zum Hohn und Spott geworden täglich.
\par 9 Da dachte ich: Wohlan, ich will sein nicht mehr gedenken und nicht mehr in seinem Namen predigen. Aber es ward in meinem Herzen wie ein brennendes Feuer, in meinen Gebeinen verschlossen, daß ich's nicht leiden konnte und wäre fast vergangen.
\par 10 Denn ich höre, wie mich viele schelten und schrecken um und um. "Hui, verklagt ihn! Wir wollen ihn verklagen!" sprechen alle meine Freunde und Gesellen, "ob wir ihn übervorteilen und ihm beikommen mögen und uns an ihm rächen."
\par 11 Aber der HERR ist bei mir wie ein starker Held; darum werden meine Verfolger fallen und nicht obliegen, sondern sollen zu Schanden werden, darum daß sie so töricht handeln; ewig wird die Schande sein, deren man nicht vergessen wird.
\par 12 Und nun, HERR Zebaoth, der du die Gerechten prüfst, Nieren und Herz siehst, laß mich deine Rache an ihnen sehen; denn ich habe dir meine Sache befohlen.
\par 13 Singet dem HERRN, rühmt den HERRN, der des Armen Leben aus der Boshaften Händen errettet!
\par 14 Verflucht sei der Tag, darin ich geboren bin; der Tag müsse ungesegnet sein, darin mich meine Mutter geboren hat!
\par 15 Verflucht sei der, so meinem Vater gute Botschaft brachte und sprach: "Du hast einen jungen Sohn", daß er ihn fröhlich machen wollte!
\par 16 Der Mann müsse sein wie die Städte, so der HERR hat umgekehrt, und ihn nicht gereut hat; und müsse des Morgens hören ein Geschrei und des Mittags ein Heulen!
\par 17 Daß du mich doch nicht getötet hast im Mutterleibe, daß meine Mutter mein Grab gewesen und ihr Leib ewig schwanger geblieben wäre!
\par 18 Warum bin ich doch aus Mutterleibe hervorgekommen, daß ich solchen Jammer und Herzeleid sehen muß und meine Tage mit Schanden zubringen!

\chapter{21}

\par 1 Dies ist das Wort, so vom HERRN geschah zu Jeremia, da der König Zedekia zu ihm sandte Pashur, den Sohn Malchias, und Zephanja, den Sohn Maasejas, den Priester, und ließ ihm sagen:
\par 2 Frage doch den HERRN für uns. Denn Nebukadnezar, der König zu Babel, streitet wider uns; daß der HERR doch mit uns tun wolle nach allen seinen Wundern, damit er von uns abzöge.
\par 3 Jeremia sprach zu ihnen: So saget Zedekia:
\par 4 Das spricht der Herr, der Gott Israels: Siehe, ich will die Waffen zurückwenden, die ihr in euren Händen habt, womit ihr streitet wider den König zu Babel und wider die Chaldäer, welche euch draußen an der Mauer belagert haben; und will sie zuhauf sammeln mitten in dieser Stadt.
\par 5 Und ich will wider euch streiten mit ausgereckter Hand, mit starkem Arm, mit Zorn, Grimm und großer Ungnade.
\par 6 Und ich will die Bürger dieser Stadt schlagen, die Menschen und das Vieh, daß sie sterben sollen durch eine große Pestilenz.
\par 7 Und darnach, spricht der HERR, will ich Zedekia, den König Juda's, samt seinen Knechten und dem Volk, das in dieser Stadt vor der Pestilenz, vor Schwert und Hunger übrigbleiben wird, geben in die Hände Nebukadnezars, des Königs zu Babel, und in die Hände ihrer Feinde, und in die Hände derer, die ihnen nach dem Leben stehen, daß er sie mit der Schärfe des Schwerts also schlage, daß kein Schonen noch Gnade noch Barmherzigkeit da sei.
\par 8 Und sage diesem Volk: So spricht der HERR: Siehe, ich lege euch vor den Weg zum Leben und den Weg zum Tode.
\par 9 Wer in dieser Stadt bleibt, der wird sterben müssen durch Schwert, Hunger und Pestilenz; wer aber sich hinausbegibt zu den Chaldäern, die euch belagern, der soll lebendig bleiben und soll sein Leben als eine Ausbeute behalten.
\par 10 Denn ich habe mein Angesicht über diese Stadt gerichtet zum Unglück und zu keinem Guten, spricht der HERR. Sie soll dem König zu Babel übergeben werden, daß er sie mit Feuer verbrenne.
\par 11 Und höret des HERRN Wort, ihr vom Hause des Königs in Juda!
\par 12 Du Haus David, so spricht der Herr: Haltet des Morgens Gericht und errettet die Beraubten aus des Frevlers Hand, auf daß mein Grimm nicht ausfahre wie ein Feuer und brenne also, das niemand löschen könne, um eures bösen Wesens willen.
\par 13 Siehe, spricht der HERR, ich will an dich, die du wohnst im Grunde, auf dem Felsen der Ebene und sprichst: Wer will uns überfallen oder in unsre Feste kommen?
\par 14 Ich will euch heimsuchen, spricht der HERR, nach der Frucht eures Tuns; ich will ein Feuer anzünden in ihrem Walde, das soll alles umher verzehren.

\chapter{22}

\par 1 So spricht der HERR: Gehe hinab in das Haus des Königs in Juda und rede daselbst dies Wort
\par 2 und sprich: Höre des HERRN Wort, du König Juda's, der du auf dem Stuhl Davids sitzest, du und deine Knechte und dein Volk, die zu diesen Toren eingehen.
\par 3 So spricht der HERR: Haltet Recht und Gerechtigkeit, und errettet den Beraubten von des Frevlers Hand, und schindet nicht die Fremdlinge, Waisen und Witwen, und tut niemand Gewalt, und vergießt nicht unschuldig Blut an dieser Stätte.
\par 4 Werdet ihr solches tun, so sollen durch die Tore dieses Hauses einziehen Könige die auf Davids Stuhl sitzen, zu Wagen und zu Rosse, samt ihren Knechten und ihrem Volk.
\par 5 Werdet ihr aber solchem nicht gehorchen, so habe ich bei mir selbst geschworen, spricht der HERR, dies Haus soll zerstört werden.
\par 6 Denn so spricht der HERR von dem Hause des Königs in Juda: Ein Gilead bist du mir, ein Haupt im Libanon. Was gilt's? ich will dich zur Wüste und die Einwohner ohne Städte machen.
\par 7 Denn ich habe den Verderber über dich bestellt, einen jeglichen mit seinen Waffen; die sollen deine auserwählten Zedern umhauen und ins Feuer werfen.
\par 8 So werden viele Heiden vor dieser Stadt vorübergehen und untereinander sagen: Warum hat der HERR mit dieser großen Stadt also gehandelt?
\par 9 Und man wird antworten: Darum daß sie den Bund des HERRN, ihres Gottes, verlassen und andere Götter angebetet und ihnen gedient haben.
\par 10 Weinet nicht über die Toten und grämet euch nicht darum; weinet aber über den, der dahinzieht; denn er wird nimmer wiederkommen, daß er sein Vaterland sehen möchte.
\par 11 Denn so spricht der HERR von Sallum, dem Sohn Josias, des Königs in Juda, welcher König ist anstatt seines Vaters Josia, der von dieser Stätte hinausgezogen ist: Er wird nicht wieder herkommen,
\par 12 sondern muß sterben an dem Ort, dahin er gefangen geführt ist, und wird dies Land nicht mehr sehen.
\par 13 Weh dem, der sein Haus mit Sünden baut und seine Gemächer mit Unrecht, der seinen Nächsten umsonst arbeiten läßt und gibt ihm seinen Lohn nicht
\par 14 und denkt: "Wohlan, ich will mir ein großes Haus bauen und weite Gemächer!" und läßt sich Fenster drein hauen und es mit Zedern täfeln und rot malen!
\par 15 Meinst du, du wollest König sein, weil du mit Zedern prangst? Hat dein Vater nicht auch gegessen und getrunken und hielt dennoch über Recht und Gerechtigkeit, und es ging ihm wohl?
\par 16 Er half dem Elenden und Armen zum Recht, und es ging ihm wohl. Ist's nicht also, daß solches heißt, mich recht erkennen? spricht der HERR.
\par 17 Aber deine Augen und dein Herz stehen nicht also, sondern auf deinen Geiz, auf unschuldig Blut zu vergießen, zu freveln und unterzustoßen.
\par 18 Darum spricht der HERR von Jojakim, dem Sohn Josias, dem König Juda's: Man wird ihn nicht beklagen: "Ach Bruder! ach Schwester!", man wird ihn auch nicht beklagen: "Ach Herr! ach Edler!"
\par 19 Er soll wie ein Esel begraben werden, zerschleift und hinausgeworfen vor die Tore Jerusalems.
\par 20 Gehe hinauf auf den Libanon und schreie und laß dich hören zu Basan und schreie von Abarim; denn alle deine Liebhaber sind zunichte gemacht.
\par 21 Ich habe dir's vorhergesagt, da es noch wohl um dich stand; aber du sprachst: "Ich will nicht hören." Also hast du dein Lebtage getan, daß du meiner Stimme nicht gehorchtest.
\par 22 Alle deine Hirten wird der Wind weiden, und deine Liebhaber ziehen gefangen dahin; da mußt du zum Spott und zu Schanden werden um aller deiner Bosheit willen.
\par 23 Die du jetzt auf dem Libanon wohnest und in Zedern nistest, wie schön wirst du sehen, wenn dir Schmerzen und Wehen kommen werden wie einer in Kindsnöten!
\par 24 So wahr ich lebe, spricht der HERR, wenn Chonja, der Sohn Jojakims, der König Juda's, ein Siegelring wäre an meiner rechten Hand, so wollte ich dich doch abreißen
\par 25 und in die Hände geben derer, die nach deinem Leben stehen und vor welchen du dich fürchtest, in die Hände Nebukadnezars, des Königs zu Babel, und der Chaldäer.
\par 26 Und ich will dich und deine Mutter, die dich geboren hat, in ein anderes Land treiben, das nicht euer Vaterland ist, und sollt daselbst sterben.
\par 27 Und in das Land, da sie von Herzen gern wieder hin wären, sollen sie nicht wiederkommen.
\par 28 Wie ein elender, verachteter, verstoßener Mann ist doch Chonja! ein unwertes Gefäß! Ach wie ist er doch samt seinem Samen so vertrieben und in ein unbekanntes Land geworfen!
\par 29 O Land, Land, Land, höre des HERRN Wort!
\par 30 So spricht der HERR: Schreibet an diesen Mann als einen, der ohne Kinder ist, einen Mann, dem es sein Lebtage nicht gelingt. Denn er wird das Glück nicht haben, daß jemand seines Samens auf dem Stuhl Davids sitze und fürder in Juda herrsche.

\chapter{23}

\par 1 Weh euch Hirten, die ihr die Herde meiner Weide umbringet und zerstreuet! spricht der HERR.
\par 2 Darum spricht der HERR, der Gott Israels, von den Hirten, die mein Volk weiden: Ihr habt meine Herde zerstreut und verstoßen und nicht besucht. Siehe, ich will euch heimsuchen um eures bösen Wesens willen, spricht der HERR.
\par 3 Und ich will die übrigen meiner Herde sammeln aus allen Ländern, dahin ich sie verstoßen habe, und will sie wiederbringen zu ihren Hürden, daß sie sollen wachsen und ihrer viel werden.
\par 4 Und ich will Hirten über sie setzen, die sie weiden sollen, daß sie sich nicht mehr sollen fürchten noch erschrecken noch heimgesucht werden, spricht der HERR.
\par 5 Siehe, es kommt die Zeit, spricht der HERR, daß ich dem David ein gerechtes Gewächs erwecken will, und soll ein König sein, der wohl regieren wird und Recht und Gerechtigkeit auf Erden anrichten.
\par 6 Zu seiner Zeit soll Juda geholfen werden und Israel sicher wohnen. Und dies wird sein Name sein, daß man ihn nennen wird: Der HERR unsre Gerechtigkeit.
\par 7 Darum siehe, es wird die Zeit kommen, spricht der HERR, daß man nicht mehr sagen wird: So wahr der HERR lebt, der die Kinder Israel aus Ägyptenland geführt hat!
\par 8 sondern: So wahr der HERR lebt, der den Samen des Hauses Israel hat herausgeführt aus dem Lande der Mitternacht und aus allen Landen, dahin ich sie verstoßen hatte, daß sie in ihrem Lande wohnen sollen!
\par 9 Wider die Propheten. Mein Herz will mir im Leibe brechen, alle meine Gebeine zittern; mir ist wie einem trunkenen Mann und wie einem, der vom Wein taumelt, vor dem HERRN und vor seinen heiligen Worten;
\par 10 daß das Land so voll Ehebrecher ist, daß das Land so jämmerlich steht, daß es so verflucht ist und die Auen in der Wüste verdorren; und ihr Leben ist böse, und ihr Regiment taugt nicht.
\par 11 Denn beide, Propheten und Priester, sind Schälke; und auch in meinem Hause finde ich ihre Bosheit, spricht der HERR.
\par 12 Darum ist ihr Weg wie ein glatter Weg im Finstern, darauf sie gleiten und fallen; denn ich will Unglück über sie kommen lassen, das Jahr ihrer Heimsuchung, spricht der HERR.
\par 13 Zwar bei den Propheten zu Samaria sah ich Torheit, daß sie weissagten durch Baal und verführten mein Volk Israel;
\par 14 aber bei den Propheten zu Jerusalem sehe ich Greuel, wie sie ehebrechen und gehen mit Lügen um und stärken die Boshaften, auf daß sich ja niemand bekehre von seiner Bosheit. Sie sind alle vor mir gleichwie Sodom, und die Bürger zu Jerusalem wie Gomorra.
\par 15 Darum spricht der HERR Zebaoth von den Propheten also: Siehe, ich will sie mit Wermut speisen und mit Galle tränken; denn von den Propheten zu Jerusalem kommt Heuchelei aus ins ganze Land.
\par 16 So spricht der HERR Zebaoth: Gehorcht nicht den Worten der Propheten, so euch weissagen. Sie betrügen euch; denn sie predigen ihres Herzens Gesicht und nicht aus des HERRN Munde.
\par 17 Sie sagen denen, die mich lästern: "Der HERR hat's gesagt, es wird euch wohl gehen"; und allen, die nach ihres Herzens Dünkel wandeln, sagen sie: "Es wird kein Unglück über euch kommen."
\par 18 Aber wer ist im Rat des HERRN gestanden, der sein Wort gesehen und gehört habe? Wer hat sein Wort vernommen und gehört?
\par 19 Siehe, es wird ein Wetter des HERRN mit Grimm kommen und ein schreckliches Ungewitter den Gottlosen auf den Kopf fallen.
\par 20 Und des HERRN Zorn wird nicht nachlassen, bis er tue und ausrichte, was er im Sinn hat; zur letzten Zeit werdet ihr's wohl erfahren.
\par 21 Ich sandte die Propheten nicht, doch liefen sie; ich redete nicht zu ihnen, doch weissagten sie.
\par 22 Denn wo sie bei meinem Rat geblieben wären und hätten meine Worte meinem Volk gepredigt, so hätten sie dasselbe von seinem bösen Wesen und von seinem bösen Leben bekehrt.
\par 23 Bin ich nur ein Gott, der nahe ist, spricht der HERR, und nicht auch ein Gott von ferneher?
\par 24 Meinst du, daß sich jemand so heimlich verbergen könne, daß ich ihn nicht sehe? spricht der HERR. Bin ich es nicht, der Himmel und Erde füllt? spricht der HERR.
\par 25 Ich höre es wohl, was die Propheten predigen und falsch weissagen in meinem Namen und sprechen: Mir hat geträumt, mir hat geträumt.
\par 26 Wann wollen doch die Propheten aufhören, die falsch weissagen und ihres Herzens Trügerei weissagen
\par 27 und wollen, daß mein Volk meines Namens vergesse über ihren Träumen, die einer dem andern erzählt? gleichwie ihre Väter meines Namens vergaßen über dem Baal.
\par 28 Ein Prophet, der Träume hat, der erzähle Träume; wer aber mein Wort hat, der Predige mein Wort recht. Wie reimen sich Stroh und Weizen zusammen? spricht der HERR.
\par 29 Ist mein Wort nicht wie Feuer, spricht der HERR, und wie ein Hammer, der Felsen zerschmeißt?
\par 30 Darum siehe, ich will an die Propheten, spricht der HERR, die mein Wort stehlen einer dem andern.
\par 31 Siehe, ich will an die Propheten, spricht der HERR, die ihr eigenes Wort führen und sprechen: Er hat's gesagt.
\par 32 Siehe, ich will an die, so falsche Träume weissagen, spricht der HERR, und erzählen dieselben und verführen mein Volk mit ihren Lügen und losen Reden, so ich sie doch nicht gesandt und ihnen nichts befohlen habe und sie auch diesem Volk nichts nütze sind, spricht der HERR.
\par 33 Wenn dich dies Volk oder ein Prophet oder ein Priester fragen wird und sagen: Welches ist die Last des HERRN? sollst du zu ihnen sagen, was die Last sei: Ich will euch hinwerfen, spricht der HERR.
\par 34 Und wo ein Prophet oder Priester oder das Volk wird sagen: "Das ist die Last des HERRN", den will ich heimsuchen und sein Haus dazu.
\par 35 Also sollt ihr aber einer mit dem andern reden und untereinander sagen: "Was antwortet der HERR, und was sagt der HERR?"
\par 36 Und nennt's nicht mehr "Last des HERRN"; denn einem jeglichem wird sein eigenes Wort eine "Last" sein, weil ihr also die Worte des lebendigen Gottes, des HERRN Zebaoth, unsers Gottes, verkehrt.
\par 37 Darum sollt ihr zum Propheten also sagen: Was antwortet dir der HERR, und was sagt der HERR?
\par 38 Weil ihr aber sprecht: "Last des HERRN", darum spricht der HERR also: Nun ihr dieses Wort eine "Last des HERRN" nennt und ich zu euch gesandt habe und sagen lassen, ihr sollt's nicht nennen "Last des HERRN":
\par 39 siehe, so will ich euch hinwegnehmen und euch samt der Stadt, die ich euch und euren Vätern gegeben habe, von meinem Angesicht wegwerfen
\par 40 und will euch ewige Schande und ewige Schmach zufügen, der nimmer vergessen soll werden.

\chapter{24}

\par 1 Siehe, der HERR zeigte mir zwei Feigenkörbe, gestellt vor den Tempel des HERRN, nachdem der König zu Babel, Nebukadnezar, hatte weggeführt Jechonja, den Sohn Jojakims, den König Juda's, samt den Fürsten Juda's und den Zimmerleuten und Schmieden von Jerusalem und gen Babel gebracht.
\par 2 In dem einen Korbe waren sehr gute Feigen, wie die ersten reifen Feigen sind; im andern Korb waren sehr schlechte Feigen, daß man sie nicht essen konnte, so schlecht waren sie.
\par 3 Und der HERR sprach zu mir: Jeremia, was siehst du? Ich sprach: Feigen; die guten Feigen sind sehr gut, und die schlechten sind sehr schlecht, daß man sie nicht essen kann, so schlecht sind sie.
\par 4 Da geschah des HERRN Wort zu mir und sprach:
\par 5 So spricht der HERR, der Gott Israels: Gleichwie diese Feigen gut sind, also will ich mich gnädig annehmen der Gefangenen aus Juda, welche ich habe aus dieser Stätte lassen ziehen in der Chaldäer Land,
\par 6 und will sie gnädig ansehen, und will sie wieder in dies Land bringen, und will sie bauen und nicht abbrechen; ich will sie pflanzen und nicht ausraufen,
\par 7 und will ihnen ein Herz geben, daß sie mich kennen sollen, daß ich der HERR sei. Und sie sollen mein Volk sein, so will ich ihr Gott sein; denn sie werden sich von ganzem Herzen zu mir bekehren.
\par 8 Aber wie die schlechten Feigen so schlecht sind, daß man sie nicht essen kann, spricht der HERR, also will ich dahingeben Zedekia, den König Juda's samt seinen Fürsten, und was übrig ist zu Jerusalem und übrig in diesem Lande und die in Ägyptenland wohnen.
\par 9 Und will ihnen Unglück zufügen und sie in keinem Königreich auf Erden bleiben lassen, daß sie sollen zu Schanden werden, zum Sprichwort, zur Fabel und zum Fluch an allen Orten, dahin ich sie verstoßen werde;
\par 10 und will Schwert, Hunger und Pestilenz unter sie schicken, bis sie umkommen von dem Lande, das ich ihnen und ihren Vätern gegeben habe.

\chapter{25}

\par 1 Dies ist das Wort, welches zu Jeremia geschah über das ganze Volk Juda im vierten Jahr Jojakims, des Sohnes Josias, des Königs in Juda (welches ist das erste Jahr Nebukadnezars, des Königs zu Babel),
\par 2 welches auch der Prophet Jeremia redete zu dem ganzen Volk Juda und zu allen Bürgern zu Jerusalem und sprach:
\par 3 Es ist vom dreizehnten Jahr an Josias, des Sohnes Amons, des Königs Juda's, des HERRN Wort zu mir geschehen bis auf diesen Tag, und ich habe euch nun dreiundzwanzig Jahre mit Fleiß gepredigt; aber ihr habt nie hören wollen.
\par 4 So hat der HERR auch zu euch gesandt alle seine Knechte, die Propheten, fleißig; aber ihr habt nie hören wollen noch eure Ohren neigen, daß ihr gehorchtet,
\par 5 da er sprach: Bekehrt euch, ein jeglicher von seinem bösen Wesen, so sollt ihr in dem Lande, das der HERR euch und euren Vätern gegeben hat, immer und ewiglich bleiben.
\par 6 Folget nicht andern Göttern, daß ihr ihnen dienet und sie anbetet, auf daß ihr mich nicht erzürnt durch eurer Hände Werk und ich euch Unglück zufügen müsse.
\par 7 Aber ihr wolltet mir nicht gehorchen, spricht der HERR, auf daß ihr mich ja wohl erzürntet durch eurer Hände Werk zu eurem eigenen Unglück.
\par 8 Darum so spricht der HERR Zebaoth: Weil ihr denn meine Worte nicht hören wollt,
\par 9 siehe, so will ich ausschicken und kommen lassen alle Völker gegen Mitternacht, spricht der HERR, auch meinen Knecht Nebukadnezar, den König zu Babel, und will sie bringen über dies Land und über diese Völker, so umherliegen, und will sie verbannen und verstören und zum Spott und zur ewigen Wüste machen,
\par 10 und will herausnehmen allen fröhlichen Gesang, die Stimme des Bräutigams und der Braut, die Stimme der Mühle und das Licht der Lampe,
\par 11 daß dies ganze Land wüst und zerstört liegen soll. Und sollen diese Völker dem König zu Babel dienen siebzig Jahre.
\par 12 Wenn aber die siebzig Jahre um sind, will ich den König zu Babel heimsuchen und dies Volk, spricht der HERR, um ihre Missetat, dazu das Land der Chaldäer, und will es zur ewigen Wüste machen.
\par 13 Also will ich über dies Land bringen alle meine Worte, die ich geredet habe wider sie (nämlich alles, was in diesem Buch geschrieben steht, das Jeremia geweissagt hat über alle Völker).
\par 14 Und sie sollen auch großen Völkern und großen Königen dienen. Also will ich ihnen vergelten nach ihrem Verdienst und nach den Werken ihrer Hände.
\par 15 Denn also spricht zu mir der HERR, der Gott Israels: Nimm diesen Becher Wein voll Zorns von meiner Hand und schenke daraus allen Völkern, zu denen ich dich sende,
\par 16 daß sie trinken, taumeln und toll werden vor dem Schwert, das ich unter sie schicken will.
\par 17 Und ich nahm den Becher von der Hand des HERRN und schenkte allen Völkern, zu denen mich der HERR sandte,
\par 18 nämlich Jerusalem, den Städten Juda's, ihren Königen und Fürsten, daß sie wüst und zerstört liegen und ein Spott und Fluch sein sollen, wie es denn heutigestages steht;
\par 19 auch Pharao, dem König in Ägypten, samt seinen Knechten, seinen Fürsten und seinem ganzen Volk;
\par 20 allen Ländern gegen Abend, allen Königen im Lande Uz, allen Königen in der Philister Lande, samt Askalon, Gaza, Ekron und den übrigen zu Asdod;
\par 21 denen zu Edom, denen zu Moab, den Kindern Ammon;
\par 22 allen Königen zu Tyrus, allen Königen zu Sidon, den Königen auf den Inseln jenseit des Meeres;
\par 23 denen von Dedan, denen von Thema, denen von Bus und allen, die das Haar rundherum abschneiden;
\par 24 allen Königen in Arabien, allen Königen gegen Abend, die in der Wüste wohnen;
\par 25 allen Königen in Simri, allen Königen in Elam, allen Königen in Medien;
\par 26 allen Königen gegen Mitternacht, in der Nähe und Ferne, einem mit dem andern, und allen Königen auf Erden, die auf dem Erdboden sind; und der König zu Sesach soll nach diesen trinken.
\par 27 Und sprich zu ihnen: So spricht der HERR Zebaoth, der Gott Israels: Trinket, daß ihr trunken werdet, speiet und niederfallt und nicht aufstehen könnt vor dem Schwert, das ich unter euch schicken will.
\par 28 Und wo sie den Becher nicht wollen von deiner Hand nehmen und trinken, so sprich zu ihnen: Also spricht der HERR Zebaoth: Nun sollt ihr trinken!
\par 29 Denn siehe, in der Stadt, die nach meinem Namen genannt ist, fange ich an zu Plagen; und ihr solltet ungestraft bleiben? Ihr sollt nicht ungestraft bleiben; denn ich rufe das Schwert herbei über alle, die auf Erden wohnen, spricht der HERR Zebaoth.
\par 30 Und du sollst alle diese Wort ihnen weissagen und sprich zu ihnen: Der HERR wird brüllen aus der Höhe und seinen Donner hören lassen aus seiner heiligen Wohnung; er wird brüllen über seine Hürden; er wird singen ein Lied wie die Weintreter über alle Einwohner des Landes, des Hall erschallen wird bis an der Welt Ende.
\par 31 Der HERR hat zu rechten mit den Heiden und will mit allem Fleisch Gericht halten; die Gottlosen wird er dem Schwert übergeben, spricht der HERR.
\par 32 So spricht der HERR Zebaoth: Siehe, es wird eine Plage kommen von einem Volk zum andern, und ein großes Wetter wird erweckt werden aus einem fernen Lande.
\par 33 Da werden die Erschlagenen des HERRN zu derselben Zeit liegen von einem Ende der Erde bis an das andere Ende; die werden nicht beklagt noch aufgehoben noch begraben werden, sondern müssen auf dem Felde liegen und zu Dung werden.
\par 34 Heulet nun, ihr Hirten, und schreiet, wälzet euch in der Asche, ihr Gewaltigen über die Herde; denn die Zeit ist hier, daß ihr geschlachtet und zerstreut werdet und zerfallen müßt wie ein köstliches Gefäß.
\par 35 Und die Hirten werden nicht fliehen können, und die Gewaltigen über die Herde werden nicht entrinnen können.
\par 36 Da werden die Hirten schreien, und die Gewaltigen über die Herde werden heulen, daß der HERR ihre Weide so verwüstet hat
\par 37 und ihre Auen, die so wohl standen, verderbt sind vor dem grimmigen Zorn des HERRN.
\par 38 Er hat seine Hütte verlassen wie ein junger Löwe, und ist also ihr Land zerstört vor dem Zorn des Tyrannen und vor seinem grimmigen Zorn.

\chapter{26}

\par 1 Im Anfang des Königreichs Jojakims, des Sohnes Josias, des Königs in Juda, geschah dies Wort vom HERRN und sprach:
\par 2 So spricht der HERR: Tritt in den Vorhof am Hause des HERRN und predige allen Städten Juda's, die da hereingehen, anzubeten im Hause des HERRN, alle Worte, die ich dir befohlen habe ihnen zu sagen, und tue nichts davon;
\par 3 ob sie vielleicht hören wollen und sich bekehren, ein jeglicher von seinem bösen Wesen, damit mich auch reuen möchte das Übel, das ich gedenke ihnen zu tun um ihres bösen Wandels willen.
\par 4 Und sprich zu ihnen: So spricht der HERR: Werdet ihr mir nicht gehorchen, daß ihr nach meinem Gesetz wandelt, das ich euch vorgelegt habe,
\par 5 daß ihr hört auf die Worte meiner Knechte, der Propheten, welche ich stets zu euch gesandt habe, und ihr doch nicht hören wolltet:
\par 6 so will ich's mit diesem Hause machen wie mit Silo und diese Stadt zum Fluch allen Heiden auf Erden machen.
\par 7 Da nun die Priester, Propheten und alles Volk hörten Jeremia, daß er solche Worte redete im Hause des HERRN,
\par 8 und Jeremia nun ausgeredet hatte alles, was ihm der HERR befohlen hatte, allem Volk zu sagen, griffen ihn die Priester, Propheten und das ganze Volk und sprachen: Du mußt sterben!
\par 9 Warum weissagst du im Namen des HERRN und sagst: Es wird diesem Hause gehen wie Silo, daß niemand mehr darin wohne? Und das ganze Volk sammelte sich im Hause des HERRN wider Jeremia.
\par 10 Da solches hörten die Fürsten Juda's gingen sie aus des Königs Hause hinauf ins Haus des HERRN und setzten sich vor das neue Tor des HERRN.
\par 11 Und die Priester und Propheten sprachen vor den Fürsten und allem Volk: Dieser ist des Todes schuldig; denn er hat geweissagt wider diese Stadt, wie ihr mit euren Ohren gehört habt.
\par 12 Aber Jeremia sprach zu allen Fürsten und zu allem Volk: Der HERR hat mich gesandt, daß ich solches alles, was ihr gehört habt, sollte weissagen wider dies Haus und wider diese Stadt.
\par 13 So bessert nun euer Wesen und Wandel und gehorcht der Stimme des HERRN, eures Gottes, so wird den HERRN auch gereuen das Übel, das er wider euch geredet hat.
\par 14 Siehe, ich bin in euren Händen; ihr mögt es machen mit mir, wie es euch recht und gut dünkt.
\par 15 Doch sollt ihr wissen: wo ihr mich tötet, so werdet ihr unschuldig Blut laden auf euch selbst, auf diese Stadt und ihre Einwohner. Denn wahrlich, der HERR hat mich zu euch gesandt, daß ich solches alles vor euren Ohren reden soll.
\par 16 Da sprachen die Fürsten und das ganze Volk zu den Priestern und Propheten: Dieser ist des Todes nicht schuldig; denn er hat zu uns geredet im Namen des HERRN, unsers Gottes.
\par 17 Und es standen etliche der Ältesten im Lande und sprachen zum ganzen Haufen des Volks:
\par 18 Zur Zeit Hiskias, des Königs in Juda, war ein Prophet, Micha von Moreseth, und sprach zum ganzen Volk Juda: So spricht der HERR Zebaoth: Zion wird wie ein Acker gepflügt werden, und Jerusalem wird zum Steinhaufen werden und der Berg des Tempels zu einer wilden Höhe.
\par 19 Doch ließ ihn Hiskia, der König Juda's und das ganze Juda darum nicht töten; ja sie fürchteten vielmehr den HERRN und beteten vor dem HERRN. Da reute auch den HERRN das Übel, das er wider sie geredet hatte. Darum täten wir sehr übel wider unsre Seelen.
\par 20 So war auch einer, der im Namen des HERRN weissagte, Uria, der Sohn Semajas, von Kirjath-Jearim. Derselbe weissagte wider diese Stadt und wider das Land gleichwie Jeremia.
\par 21 Da aber der König Jojakim und alle seine Gewaltigen und die Fürsten seine Worte hörten, wollte ihn der König töten lassen. Und Uria erfuhr das, fürchtete sich und floh und zog nach Ägypten.
\par 22 Aber der König Jojakim schickte Leute nach Ägypten, Elnathan, den Sohn Achbors, und andere mit ihm;
\par 23 die führten ihn aus Ägypten und brachten ihn zum König Jojakim; der ließ ihn mit dem Schwert töten und ließ seinen Leichnam unter dem gemeinen Pöbel begraben.
\par 24 Aber mit Jeremia war die Hand Ahikams, des Sohnes Saphans, daß er nicht dem Volk in die Hände kam, daß sie ihn töteten.

\chapter{27}

\par 1 Im Anfang des Königreichs Zedekia, des Sohnes Josias, des Königs in Juda, geschah dies Wort vom HERRN zu Jeremia und sprach:
\par 2 So spricht der HERR zu mir: Mache dir ein Joch und hänge es an deinen Hals
\par 3 und schicke es zum König in Edom, zum König in Moab, zum König der Kinder Ammon, zum König von Tyrus und zum König zu Sidon durch die Boten, so zu Zedekia, dem König Juda's, gen Jerusalem gekommen sind,
\par 4 und befiehl ihnen, daß sie ihren Herren sagen: So spricht der HERR Zebaoth, der Gott Israels: So sollt ihr euren Herren sagen:
\par 5 Ich habe die Erde gemacht und Menschen und Vieh, so auf Erden sind, durch meine große Kraft und meinen ausgestreckten Arm und gebe sie, wem ich will.
\par 6 Nun aber habe ich alle diese Lande gegeben in die Hand meines Knechtes Nebukadnezar, des Königs zu Babel, und habe ihm auch die wilden Tiere auf dem Felde gegeben, daß sie ihm dienen sollen.
\par 7 Und sollen alle Völker dienen ihm und seinem Sohn und seines Sohnes Sohn, bis daß die Zeit seines Landes auch komme und er vielen Völkern und großen Königen diene.
\par 8 Welches Volk aber und Königreich dem König zu Babel, Nebukadnezar, nicht dienen will, und wer seinen Hals nicht wird unter das Joch des Königs zu Babel geben, solch Volk will ich heimsuchen mit Schwert, Hunger und Pestilenz, spricht der HERR, bis daß ich sie durch seine Hand umbringe.
\par 9 Darum so gehorcht nicht euren Propheten, Weissagern, Traumdeutern, Tagewählern und Zauberern, die euch sagen: Ihr werdet nicht dienen müssen dem König zu Babel.
\par 10 Denn sie weissagen euch falsch, auf daß sie euch fern aus eurem Lande bringen und ich euch ausstoße und ihr umkommt.
\par 11 Denn welches Volk seinen Hals ergibt unter das Joch des Königs zu Babel und dient ihm, das will ich in seinem Lande lassen, daß es dasselbe baue und bewohne, spricht der HERR.
\par 12 Und ich redete solches alles zu Zedekia, dem König Juda's, und sprach: Ergebt euren Hals unter das Joch des Königs zu Babel und dient ihm und seinem Volk, so sollt ihr lebendig bleiben.
\par 13 Warum wollt ihr sterben, du und dein Volk, durch Schwert, Hunger und Pestilenz, wie denn der HERR geredet hat über das Volk, so dem König zu Babel nicht dienen will?
\par 14 Darum gehorcht nicht den Worten der Propheten, die euch sagen: "Ihr werdet nicht dienen müssen dem König zu Babel!" Denn sie weissagen euch falsch,
\par 15 und ich habe sie nicht gesandt, spricht der HERR; sondern sie weissagen falsch in meinem Namen, auf daß ich euch ausstoße und ihr umkommt samt den Propheten, die euch weissagen.
\par 16 Und zu den Priestern und zu allem diesem Volk redete ich und sprach: So spricht der HERR: Gehorcht nicht den Worten eurer Propheten, die euch weissagen und sprechen: "Siehe, die Gefäße aus dem Hause des HERRN werden nun bald von Babel wieder herkommen!" Denn sie weissagen euch falsch.
\par 17 Gehorchet ihnen nicht, sondern dienet dem König zu Babel, so werdet ihr lebendig bleiben. Warum soll doch diese Stadt zur Wüste werden?
\par 18 Sind sie aber Propheten und haben des HERRN Wort, so laßt sie vom HERRN Zebaoth erbitten, daß die übrigen Gefäße im Hause des HERRN und im Hause des Königs in Juda und zu Jerusalem nicht auch gen Babel geführt werden.
\par 19 Denn also spricht der HERR Zebaoth von den Säulen und vom Meer und von dem Gestühl und von den Gefäßen, die noch übrig sind in dieser Stadt,
\par 20 welche Nebukadnezar, der König zu Babel, nicht wegnahm, da er Jechonja, den Sohn Jojakims, den König Juda's, von Jerusalem wegführte gen Babel samt allen Fürsten in Juda und Jerusalem,
\par 21 denn so spricht der HERR Zebaoth, der Gott Israels, von den Gefäßen, die noch übrig sind im Hause des HERRN und im Hause des Königs in Juda und zu Jerusalem:
\par 22 Sie sollen gen Babel geführt werden und daselbst bleiben bis auf den Tag, da ich sie heimsuche, spricht der HERR, und ich sie wiederum herauf an diesen Ort bringen lasse.

\chapter{28}

\par 1 Und in demselben Jahr, im Anfang des Königreiches Zedekias, des Königs in Juda, im fünften Monat des vierten Jahres, sprach Hananja, der Sohn Assurs, ein Prophet von Gibeon, zu mir im Hause des HERRN, in Gegenwart der Priester und alles Volks, und sagte:
\par 2 So spricht der HERR Zebaoth der Gott Israels: Ich habe das Joch des Königs zu Babel zerbrochen;
\par 3 und ehe zwei Jahre um sind, will ich alle Gefäße des Hauses des HERRN, welche Nebukadnezar, der Könige zu Babel, hat von diesem Ort weggenommen und gen Babel geführt, wiederum an diesen Ort bringen;
\par 4 Dazu Jechonja, den Sohn Jojakims, den König Juda's samt allen Gefangenen aus Juda, die gen Babel geführt sind, will ich auch wieder an diesen Ort bringen, spricht der HERR; denn ich will das Joch des Königs zu Babel zerbrechen.
\par 5 Da sprach der Prophet Jeremia zu dem Propheten Hananja in der Gegenwart der Priester und des ganzen Volks, die im Hause des HERRN standen,
\par 6 und sagte: Amen! Der HERR tue also; der HERR bestätige dein Wort, das du geweissagt hast, daß er die Gefäße aus dem Hause des HERRN von Babel wieder bringe an diesen Ort samt allen Gefangenen.
\par 7 Aber doch höre auch dies Wort, das ich vor deinen Ohren rede und vor den Ohren des ganzen Volks:
\par 8 Die Propheten, die vor mir und vor dir gewesen sind von alters her, die haben wider viel Länder und Königreiche geweissagt von Krieg, von Unglück und von Pestilenz;
\par 9 wenn aber ein Prophet von Frieden weissagt, den wird man kennen, ob ihn der HERR wahrhaftig gesandt hat, wenn sein Wort erfüllt wird.
\par 10 Da nahm Hananja das Joch vom Halse des Propheten Jeremia und zerbrach es.
\par 11 Und Hananja sprach in Gegenwart des ganzen Volks: So spricht der HERR: Ebenso will ich zerbrechen das Joch Nebukadnezars, des Königs zu Babel, ehe zwei Jahre um kommen, vom Halse aller Völker. Und der Prophet Jeremia ging seines Weges.
\par 12 Aber des HERRN Wort geschah zu Jeremia, nachdem der Prophet Hananja das Joch zerbrochen hatte vom Halse des Propheten Jeremia und sprach:
\par 13 Geh hin und sage Hananja: So spricht der HERR: Du hast das hölzerne Joch zerbrochen und hast nun ein eisernes Joch an jenes Statt gemacht.
\par 14 Denn so spricht der HERR Zebaoth, der Gott Israels: Ein eisernes Joch habe ich allen diesen Völkern an den Hals gehängt, damit sie dienen sollen Nebukadnezar, dem König zu Babel, und müssen ihm dienen; denn ich habe ihm auch die wilden Tiere gegeben.
\par 15 Und der Prophet Jeremia sprach zum Propheten Hananja: Höre doch, Hananja! Der HERR hat dich nicht gesandt, und du hast gemacht, daß dies Volk auf Lügen sich verläßt.
\par 16 Darum spricht der HERR also: Siehe, ich will dich vom Erdboden nehmen; dies Jahr sollst du sterben; denn du hast sie mit deiner Rede vom HERRN abgewendet.
\par 17 Also starb der Prophet Hananja desselben Jahres im siebenten Monat.

\chapter{29}

\par 1 Dies sind die Worte in dem Brief, den der Prophet Jeremia sandte von Jerusalem an die übrigen Ältesten, die weggeführt waren, und an die Priester und Propheten und an das ganze Volk, das Nebukadnezar von Jerusalem hatte weggeführt gen Babel
\par 2 (nachdem der König Jechonja und die Königin mit den Kämmerern und Fürsten in Juda und Jerusalem samt den Zimmerleuten und Schmieden zu Jerusalem weg waren),
\par 3 durch Eleasa, den Sohn Saphans, und Gemarja, den Sohn Hilkias, welche Zedekia, der König Juda's, sandte gen Babel zu Nebukadnezar, dem König zu Babel:
\par 4 So spricht der HERR Zebaoth, der Gott Israels, zu allen Gefangenen, die ich habe von Jerusalem wegführen lassen gen Babel:
\par 5 Bauet Häuser, darin ihr wohnen möget, pflanzet Gärten, daraus ihr Früchte essen möget;
\par 6 nehmet Weiber und zeuget Söhne und Töchter; nehmet euren Söhnen Weiber und gebet euren Töchtern Männern, daß sie Söhne und Töchter zeugen; mehret euch daselbst, daß euer nicht wenig sei.
\par 7 Suchet der Stadt Bestes, dahin ich euch habe lassen wegführen, und betet für sie zum HERRN; denn wenn's ihr wohl geht, so geht's auch euch wohl.
\par 8 Denn so spricht der HERR Zebaoth, der Gott Israels: Laßt euch die Propheten, die bei euch sind, und die Wahrsager nicht betrügen und gehorcht euren Träumen nicht, die euch träumen.
\par 9 Denn sie weissagen euch falsch in meinem Namen; ich habe sie nicht gesandt, spricht der HERR.
\par 10 Denn so spricht der HERR: Wenn zu Babel siebzig Jahre aus sind, so will ich euch besuchen und will mein gnädiges Wort über euch erwecken, daß ich euch wieder an diesen Ort bringe.
\par 11 Denn ich weiß wohl, was ich für Gedanken über euch habe, spricht der HERR: Gedanken des Friedens und nicht des Leidens, daß ich euch gebe das Ende, des ihr wartet.
\par 12 Und ihr werdet mich anrufen und hingehen und mich bitten, und ich will euch erhören.
\par 13 Ihr werdet mich suchen und finden. Denn so ihr mich von ganzem Herzen suchen werdet,
\par 14 so will ich mich von euch finden lassen, spricht der HERR, und will euer Gefängnis wenden und euch sammeln aus allen Völkern und von allen Orten, dahin ich euch verstoßen habe, spricht der HERR, und will euch wiederum an diesen Ort bringen, von dem ich euch habe lassen wegführen.
\par 15 Zwar ihr meint, der HERR habe euch zu Babel Propheten erweckt.
\par 16 Aber also spricht der HERR vom König, der auf Davids Stuhl sitzt, und von euren Brüdern, die nicht mit euch gefangen hinausgezogen sind,
\par 17 ja, also spricht der HERR Zebaoth: Siehe, ich will Schwert, Hunger und Pestilenz unter sie schicken und will mit ihnen umgehen wie mit den schlechten Feigen, davor einen ekelt zu essen,
\par 18 und will hinter ihnen her sein mit Schwert, Hunger und Pestilenz und will sie in keinem Königreich auf Erden bleiben lassen, daß sie sollen zum Fluch, zum Wunder, zum Hohn und zum Spott unter allen Völkern werden, dahin ich sie verstoßen werde,
\par 19 darum daß sie meinen Worten nicht gehorchen, spricht der HERR, der ich meine Knechte, die Propheten, zu euch stets gesandt habe; aber ihr wolltet nicht hören, spricht der HERR.
\par 20 Ihr aber alle, die ihr gefangen seid weggeführt, die ich von Jerusalem habe gen Babel ziehen lassen, hört des HERRN Wort!
\par 21 So spricht der HERR Zebaoth, der Gott Israels, wider Ahab, den Sohn Kolajas, und wider Zedekia, den Sohn Maasejas, die euch falsch weissagen in meinem Namen; Siehe, ich will sie geben in die Hände Nebukadnezars, des Königs zu Babel; der soll sie totschlagen lassen vor euren Augen,
\par 22 daß man wird aus ihnen einen Fluch machen unter allen Gefangenen aus Juda, die zu Babel sind, und sagen: Der HERR tue dir wie Zedekia und Ahab, welche der König zu Babel auf Feuer braten ließ,
\par 23 darum daß sie eine Torheit in Israel begingen und trieben Ehebruch mit ihrer Nächsten Weibern und predigten falsch in meinem Namen, was ich ihnen nicht befohlen hatte. Solches weiß ich und bezeuge es, spricht der HERR.
\par 24 Und wider Semaja von Nehalam sollst du sagen:
\par 25 So spricht der HERR Zebaoth, der Gott Israels: Darum daß du unter deinem Namen hast Briefe gesandt an alles Volk, das zu Jerusalem ist, und an den Priester Zephanja, den Sohn Maasejas, und an alle Priester und gesagt:
\par 26 Der HERR hat dich zum Priester gesetzt anstatt des Priesters Jojada, daß ihr sollt Aufseher sein im Hause des HERRN über alle Wahnsinnigen und Weissager, daß du sie in den Kerker und Stock legst.
\par 27 Nun, warum strafst du denn nicht Jeremia von Anathoth, der euch weissagt?
\par 28 darum daß er uns gen Babel geschickt hat und lassen sagen: Es wird noch lange währen; baut Häuser, darin ihr wohnt, und pflanzt Gärten, daß ihr die Früchte davon eßt.
\par 29 Denn Zephanja, der Priester hatte denselben Brief gelesen und den Propheten Jeremia lassen zuhören.
\par 30 Darum geschah des HERRN Wort zu Jeremia und sprach:
\par 31 Sende hin zu allen Gefangenen und laß ihnen sagen: So spricht der HERR wider Semaja von Nehalam: Darum daß euch Semaja weissagt, und ich habe ihn doch nicht gesandt, und macht, daß ihr auf Lügen vertraut,
\par 32 darum spricht der HERR also: Siehe, ich will Semaja von Nehalam heimsuchen samt seinem Samen, daß der Seinen keiner soll unter diesem Volk bleiben, und soll das Gute nicht sehen, das ich meinem Volk tun will, spricht der HERR; denn er hat sie mit seiner Rede vom HERRN abgewendet.

\chapter{30}

\par 1 Dies ist das Wort, das vom HERRN geschah zu Jeremia:
\par 2 So spricht der HERR, der Gott Israels: Schreibe dir alle Worte in ein Buch, die ich zu dir rede.
\par 3 Denn siehe, es kommt die Zeit, spricht der HERR, daß ich das Gefängnis meines Volkes Israel und Juda wenden will, spricht der HERR, und will sie wiederbringen in das Land, das ich ihren Vätern gegeben habe, daß sie es besitzen sollen.
\par 4 Dies sind aber die Worte, welche der HERR redet von Israel und Juda:
\par 5 So spricht der HERR: Wir hören ein Geschrei des Schreckens; es ist eitel Furcht da und kein Friede.
\par 6 Forschet doch und sehet, ob ein Mann gebären könne? Wie geht es denn zu, daß ich alle Männer sehe ihre Hände auf ihren Hüften haben wie Weiber in Kindsnöten und alle Angesichter sind bleich?
\par 7 Es ist ja ein großer Tag, und seinesgleichen ist nicht gewesen, und ist eine Zeit der Angst in Jakob; doch soll ihm daraus geholfen werden.
\par 8 Es soll aber geschehen zu derselben Zeit, spricht der HERR Zebaoth, daß ich sein Joch von deinem Halse zerbrechen will und deine Bande zerreißen, daß er nicht mehr den Fremden dienen muß,
\par 9 sondern sie werden dem HERRN, ihrem Gott, dienen und ihrem König David, welchen ich ihnen erwecken will.
\par 10 Darum fürchte du dich nicht, mein Knecht Jakob, spricht der HERR, und entsetze dich nicht Israel. Denn siehe, ich will dir helfen aus fernen Landen und deinen Samen aus dem Lande des Gefängnisses, daß Jakob soll wiederkommen, in Frieden leben und Genüge haben, und niemand soll ihn schrecken.
\par 11 Denn ich bin bei dir, spricht der HERR, daß ich dir helfe. Denn ich will mit allen Heiden ein Ende machen, dahin ich dich zerstreut habe; aber mit dir will ich nicht ein Ende machen; züchtigen aber will ich dich mit Maßen, daß du dich nicht für unschuldig haltest.
\par 12 Denn also spricht der HERR: Dein Schade ist verzweifelt böse, und deine Wunden sind unheilbar.
\par 13 Deine Sache behandelt niemand, daß er dich verbände; es kann dich niemand heilen.
\par 14 Alle deine Liebhaber vergessen dein, und fragen nichts darnach. Ich habe dich geschlagen, wie ich einen Feind schlüge, mit unbarmherziger Staupe um deiner großen Missetat und deiner starken Sünden willen.
\par 15 Was schreist du über deinen Schaden und über dein verzweifelt böses Leiden? Habe ich dir doch solches getan um deiner großen Missetat und um deiner starken Sünden willen.
\par 16 Darum alle, die dich gefressen haben, sollen gefressen werden, und alle, die dich geängstet haben, sollen alle gefangen werden; die dich beraubt haben sollen beraubt werden, und alle, die dich geplündert haben, sollen geplündert werden.
\par 17 Aber dich will ich wieder gesund machen und deine Wunden heilen, spricht der HERR, darum daß man dich nennt die Verstoßene und Zion, nach der niemand frage.
\par 18 So spricht der HERR: Siehe, ich will das Gefängnis der Hütten Jakobs wenden und mich über seine Wohnungen erbarmen, und die Stadt soll wieder auf ihre Hügel gebaut werden, und der Tempel soll stehen nach seiner Weise.
\par 19 Und soll von dannen herausgehen Lob-und Freudengesang; denn ich will sie mehren und nicht mindern, ich will sie herrlich machen und nicht geringer.
\par 20 Ihre Söhne sollen sein gleichwie vormals und ihre Gemeinde vor mir gedeihen; denn ich will heimsuchen alle, die sie plagen.
\par 21 Und ihr Fürst soll aus ihnen herkommen und ihr Herrscher von ihnen ausgehen, und er soll zu mir nahen; denn wer ist der, so mit willigem Herzen zu mir naht? spricht der HERR.
\par 22 Und ihr sollt mein Volk sein, und ich will euer Gott sein.
\par 23 Siehe, es wird ein Wetter des HERRN mit Grimm kommen; ein schreckliches Ungewitter wird den Gottlosen auf den Kopf fallen.
\par 24 Des HERRN grimmiger Zorn wird nicht nachlassen, bis er tue und ausrichte, was er im Sinn hat; zur letzten Zeit werdet ihr solches erfahren.

\chapter{31}

\par 1 Zu derselben Zeit, spricht der HERR, will ich aller Geschlechter Israels Gott sein, und sie sollen mein Volk sein.
\par 2 So spricht der HERR: Das Volk, so übriggeblieben ist vom Schwert, hat Gnade gefunden in der Wüste; Israel zieht hin zu seiner Ruhe.
\par 3 Der HERR ist mir erschienen von ferne: Ich habe dich je und je geliebt; darum habe ich dich zu mir gezogen aus lauter Güte.
\par 4 Wohlan, ich will dich wiederum bauen, daß du sollst gebaut heißen, du Jungfrau Israel; du sollst noch fröhlich pauken und herausgehen an den Tanz.
\par 5 Du sollst wiederum Weinberge pflanzen an den Bergen Samarias; pflanzen wird man sie und ihre Früchte genießen.
\par 6 Denn es wird die Zeit noch kommen, daß die Hüter an dem Gebirge Ephraim werden rufen: Wohlauf, und laßt uns hinaufgehen gen Zion zu dem HERRN, unserm Gott!
\par 7 Denn also spricht der HERR: Rufet über Jakob mit Freuden und jauchzet über das Haupt unter den Heiden; rufet laut, rühmet und sprecht: HERR, hilf deinem Volk, den übrigen in Israel!
\par 8 Siehe, ich will sie aus dem Lande der Mitternacht bringen und will sie sammeln aus den Enden der Erde, Blinde und Lahme, Schwangere und Kindbetterinnen, daß sie in großen Haufen wieder hierher kommen sollen.
\par 9 Sie werden weinend kommen und betend, so will ich sie leiten; ich will sie leiten an den Wasserbächen auf schlichtem Wege, daß sie sich nicht stoßen; denn ich bin Israels Vater, so ist Ephraim mein erstgeborenen Sohn.
\par 10 Höret ihr Heiden, des HERRN Wort und verkündigt es fern in die Inseln und sprecht: Der Israel zerstreut hat, der wird's auch wieder sammeln und wird sie hüten wie ein Hirte sein Herde.
\par 11 Denn der HERR wird Jakob erlösen und von der Hand des Mächtigen erretten.
\par 12 Und sie werden kommen und auf der Höhe Zion jauchzen und werden zu den Gaben des HERRN laufen, zum Getreide, Most, Öl, und jungen Schafen und Ochsen, daß ihre Seele wird sein wie ein wasserreicher Garten und sie nicht mehr bekümmert sein sollen.
\par 13 Alsdann werden auch die Jungfrauen fröhlich am Reigen sein, dazu die junge Mannschaft und die Alten miteinander. Denn ich will ihr Trauern in Freude verkehren und sie trösten und sie erfreuen nach ihrer Betrübnis.
\par 14 Und ich will der Priester Herz voller Freude machen, und mein Volk soll meiner Gaben die Fülle haben, spricht der HERR.
\par 15 So spricht der HERR: Man hört eine klägliche Stimme und bitteres Weinen auf der Höhe; Rahel weint über ihre Kinder und will sich nicht trösten lassen über ihre Kinder, denn es ist aus mit ihnen.
\par 16 Aber der HERR spricht also: Laß dein Schreien und Weinen und die Tränen deiner Augen; denn deine Arbeit wird wohl belohnt werden, spricht der HERR. Sie sollen wiederkommen aus dem Lande des Feindes;
\par 17 und deine Nachkommen haben viel Gutes zu erwarten, spricht der HERR; denn deine Kinder sollen wieder in ihre Grenze kommen.
\par 18 Ich habe wohl gehört, wie Ephraim klagt: "Du hast mich gezüchtigt, und ich bin auch gezüchtigt wie ein ungebändigtes Kalb; bekehre mich du, so werde ich bekehrt; denn du, HERR, bist mein Gott.
\par 19 Da ich bekehrt ward, tat ich Buße; denn nachdem ich gewitzigt bin, schlage ich mich auf die Hüfte. Ich bin zu Schanden geworden und stehe schamrot; denn ich muß leiden den Hohn meiner Jugend."
\par 20 Ist nicht Ephraim mein teurer Sohn und mein trautes Kind? Denn ich gedenke noch wohl daran, was ich ihm geredet habe; darum bricht mir mein Herz gegen ihn, daß ich mich sein erbarmen muß, spricht der HERR.
\par 21 Richte dir Denkmale auf, setze dir Zeichen und richte dein Herz auf die gebahnte Straße, darauf du gewandelt hast; kehre wieder, Jungfrau Israel, kehre dich wieder zu diesen deinen Städten!
\par 22 Wie lange willst du in der Irre gehen, du abtrünnige Tochter? Denn der HERR wird ein Neues im Lande erschaffen: das Weib wird den Mann umgeben.
\par 23 So spricht der HERR Zebaoth, der Gott Israels: Man wird noch dies Wort wieder reden im Lande Juda und in seinen Städten, wenn ich ihr Gefängnis wenden werde: Der HERR segne dich, du Wohnung der Gerechtigkeit, du heiliger Berg!
\par 24 Und Juda samt allen seinen Städten sollen darin wohnen, dazu Ackerleute und die mit Herden umherziehen;
\par 25 denn ich will die müden Seelen erquicken und die bekümmerten Seelen sättigen.
\par 26 Darüber bin ich aufgewacht und sah auf und hatte so sanft geschlafen.
\par 27 Siehe, es kommt die Zeit, spricht der HERR, daß ich das Haus Israel und das Haus Juda besäen will mit Menschen und mit Vieh.
\par 28 Und gleichwie ich über sie gewacht habe, auszureuten, zu zerreißen, abzubrechen, zu verderben und zu plagen: also will ich über sie wachen, zu bauen und zu pflanzen, spricht der HERR.
\par 29 Zu derselben Zeit wird man nicht mehr sagen: "Die Väter haben Herlinge gegessen, und der Kinder Zähne sind stumpf geworden":
\par 30 sondern ein jeglicher soll um seiner Missetat willen sterben, und welcher Mensch Herlinge ißt, dem sollen seine Zähne stumpf werden.
\par 31 Siehe, es kommt die Zeit, spricht der HERR, da will ich mit dem Hause Israel und mit dem Hause Juda einen neuen Bund machen;
\par 32 nicht wie der Bund gewesen ist, den ich mit ihren Vätern machte, da ich sie bei der Hand nahm, daß ich sie aus Ägyptenland führte, welchen Bund sie nicht gehalten haben, und ich sie zwingen mußte, spricht der HERR;
\par 33 sondern das soll der Bund sein, den ich mit dem Hause Israel machen will nach dieser Zeit, spricht der HERR: Ich will mein Gesetz in ihr Herz geben und in ihren Sinn schreiben; und sie sollen mein Volk sein, so will ich ihr Gott sein;
\par 34 und wird keiner den andern noch ein Bruder den andern lehren und sagen: "Erkenne den HERRN", sondern sie sollen mich alle kennen, beide, klein und groß, spricht der Herr. Denn ich will ihnen ihre Missetat vergeben und ihrer Sünden nimmermehr gedenken.
\par 35 So spricht der HERR, der die Sonne dem Tage zum Licht gibt und den Mond und die Sterne nach ihrem Lauf der Nacht zum Licht; der das Meer bewegt, daß seine Wellen brausen, HERR Zebaoth ist sein Name:
\par 36 Wenn solche Ordnungen vergehen vor mir, spricht der HERR, so soll auch aufhören der Same Israels, daß er nicht mehr ein Volk vor mir sei ewiglich.
\par 37 So spricht der HERR: Wenn man den Himmel oben kann messen und den Grund der Erde erforschen, so will ich auch verwerfen den ganzen Samen Israels um alles, was sie tun, spricht der HERR.
\par 38 Siehe, es kommt die Zeit, spricht der HERR, daß die Stadt des HERRN soll gebaut werden vom Turm Hananeel an bis ans Ecktor;
\par 39 und die Richtschnur wird neben demselben weiter herausgehen bis an den Hügel Gareb und sich gen Goath wenden;
\par 40 und das Tal der Leichen und Asche samt dem ganzen Acker bis an den Bach Kidron, bis zur Ecke am Roßtor gegen Morgen, wird dem Herrn heilig sein, daß es nimmermehr zerrissen noch abgebrochen soll werden.

\chapter{32}

\par 1 Dies ist das Wort, das vom HERRN geschah zu Jeremia im zehnten Jahr Zedekias, des Königs in Juda, welches ist das achtzehnte Jahr Nebukadnezars.
\par 2 Dazumal belagerte das Heer des Königs zu Babel Jerusalem. Aber der Prophet Jeremia lag gefangen im Vorhof des Gefängnisses am Hause des Königs in Juda,
\par 3 dahin Zedekia, der König Juda's, ihn hatte lassen verschließen und gesagt: Warum weissagst du und sprichst: So spricht der HERR: Siehe, ich gebe diese Stadt in die Hände des Königs zu Babel, und er soll sie gewinnen;
\par 4 und Zedekia, der König Juda's, soll den Chaldäern nicht entrinnen, sondern ich will ihn dem König zu Babel in die Hände geben, daß er mündlich mit ihm reden und mit seinen Augen ihn sehen soll.
\par 5 Und er wird Zedekia gen Babel führen; da soll er auch bleiben, bis daß ich ihn heimsuche, spricht der HERR; denn ob ihr schon wider die Chaldäer streitet, soll euch doch nichts gelingen.
\par 6 Und Jeremia sprach: Es ist des HERRN Wort geschehen zu mir und spricht:
\par 7 Siehe, Hanameel, der Sohn Sallums, deines Oheims, kommt zu dir und wird sagen: Kaufe du meinen Acker zu Anathoth; denn du hast das nächste Freundrecht dazu, daß du ihn kaufen sollst.
\par 8 Also kam Hanameel, meines Oheims Sohn, wie der HERR gesagt hatte, zu mir in den Hof des Gefängnisses und sprach zu mir: Kaufe doch meinen Acker zu Anathoth, der im Lande Benjamin liegt; denn du hast Erbrecht dazu, und du bist der nächste; kaufe du ihn! Da merkte ich, daß es des Herrn Wort wäre,
\par 9 und kaufte den Acker von Hanameel, meines Oheims Sohn, zu Anathoth, und wog ihm das Geld dar, siebzehn Silberlinge.
\par 10 Und ich schrieb einen Brief und versiegelte ihn und nahm Zeugen dazu und wog das Geld dar auf einer Waage
\par 11 und nahm zu mir den versiegelten Kaufbrief nach Recht und Gewohnheit und eine offene Abschrift
\par 12 und gab den Kaufbrief Baruch, dem Sohn Nerias, des Sohnes Maasejas, in Gegenwart Hanameels, meines Vetters, und der Zeugen, die im Kaufbrief geschrieben standen, und aller Juden, die im Hofe des Gefängnisses saßen,
\par 13 und befahl Baruch vor ihren Augen und sprach:
\par 14 So spricht der HERR Zebaoth, der Gott Israels: Nimm diese Briefe, den versiegelten Kaufbrief samt dieser offenen Abschrift, und lege sie in ein irdenes Gefäß, daß sie lange bleiben mögen.
\par 15 Denn so spricht der HERR Zebaoth, der Gott Israels: Noch soll man Häuser, Äcker und Weinberge kaufen in diesem Lande.
\par 16 Und da ich den Kaufbrief hatte Baruch, dem Sohn Nerias, gegeben, betete ich zum HERRN und sprach:
\par 17 Ach HERR HERR, siehe, du hast Himmel und Erde gemacht durch deine große Kraft und durch deinen ausgestreckten Arm, und ist kein Ding vor dir unmöglich;
\par 18 der du wohltust vielen Tausenden und vergiltst die Missetat der Väter in den Busen ihrer Kinder nach ihnen, du großer und starker Gott; HERR Zebaoth ist dein Name;
\par 19 groß von Rat und mächtig von Tat, und deine Augen stehen offen über alle Wege der Menschenkinder, daß du einem jeglichen gibst nach seinem Wandel und nach der Frucht seines Wesens;
\par 20 der du in Ägyptenland hast Zeichen und Wunder getan bis auf diesen Tag, an Israel und den Menschen, und hast dir einen Namen gemacht, wie er heutigestages ist;
\par 21 und hast dein Volk Israel aus Ägyptenland geführt durch Zeichen und Wunder, durch deine mächtige Hand, durch ausgestrecktem Arm und durch großen Schrecken;
\par 22 und hast ihnen dies Land gegeben, welches du ihren Vätern geschworen hattest, daß du es ihnen geben wolltest, ein Land, darin Milch und Honig fließt:
\par 23 und da sie hineinkamen und es besaßen, gehorchten sie deiner Stimme nicht, wandelten auch nicht nach deinem Gesetz; und alles, was du ihnen gebotest, daß sie es tun sollten, das ließen sie; darum du auch ihnen all dies Unglück ließest widerfahren;
\par 24 siehe, diese Stadt ist belagert, daß sie gewonnen und vor Schwert, Hunger und Pestilenz in der Chaldäer Hände, welche wider sie streiten, gegeben werden muß; und wie du geredet hast, so geht es, das siehest du,
\par 25 und du sprichst zu mir, HERR HERR: "Kaufe du einen Acker um Geld und nimm Zeugen dazu", so doch die Stadt in der Chaldäer Hände gegeben wird.
\par 26 Und des HERRN Wort geschah zu Jeremia und sprach:
\par 27 Siehe, ich, der HERR, bin ein Gott alles Fleisches; sollte mir etwas unmöglich sein?
\par 28 Darum spricht der HERR also: Siehe, ich gebe diese Stadt in der Chaldäer Hände und in die Hand Nebukadnezars, des Königs zu Babel; und er soll sie gewinnen.
\par 29 Und die Chaldäer, so wider diese Stadt streiten, werden hereinkommen und sie mit Feuer verbrennen samt den Häusern, wo sie auf den Dächern Baal geräuchert und andern Göttern Trankopfer geopfert haben, auf daß sie mich erzürnten.
\par 30 Denn die Kinder Israel und die Kinder Juda haben von ihrer Jugend auf getan, was mir übel gefällt; und die Kinder Israel haben mich erzürnt durch ihrer Hände Werk, spricht der HERR.
\par 31 Denn seitdem diese Stadt gebaut ist, bis auf diesen Tag, hat sie mich zornig gemacht, daß ich sie muß von meinem Angesicht wegtun
\par 32 um aller Bosheit willen der Kinder Israel und der Kinder Juda, die sie getan haben, daß sie mich erzürnten. Sie, ihre Könige, Fürsten, Priester und Propheten und die in Juda und Jerusalem wohnen,
\par 33 haben mir den Rücken und nicht das Angesicht zugekehrt, wiewohl ich sie stets lehren ließ; aber sie wollten nicht hören noch sich bessern.
\par 34 Dazu haben sie ihre Greuel in das Haus gesetzt, das von mir den Namen hat, daß sie es verunreinigten,
\par 35 und haben die Höhen des Baal gebaut im Tal Ben-Hinnom, daß sie ihre Söhne und Töchter dem Moloch verbrennten, davon ich ihnen nichts befohlen habe und ist mir nie in den Sinn gekommen, daß sie solche Greuel tun sollten, damit sie Juda also zu Sünden brächten.
\par 36 Und nun um deswillen spricht der HERR, der Gott Israels, also von dieser Stadt, davon ihr sagt, daß sie werde vor Schwert, Hunger und Pestilenz in die Hände des Königs zu Babel gegeben:
\par 37 Siehe, ich will sie sammeln aus allen Landen, dahin ich sie verstoße durch meinen Zorn, Grimm und große Ungnade, und will sie wiederum an diesen Ort bringen, daß sie sollen sicher wohnen.
\par 38 Und sie sollen mein Volk sein, so will ich ihr Gott sein;
\par 39 und ich will ihnen einerlei Herz und Wesen geben, daß sie mich fürchten sollen ihr Leben lang, auf daß es ihnen und ihren Kindern nach ihnen wohl gehe;
\par 40 und will einen ewigen Bund mit ihnen machen, daß ich nicht will ablassen, ihnen Gutes zu tun; und will ihnen meine Furcht ins Herz geben, daß sie nicht von mir weichen;
\par 41 und soll meine Lust sein, daß ich ihnen Gutes tue; und ich will sie in diesem Lande pflanzen treulich, von ganzem Herzen und von ganzer Seele.
\par 42 Denn so spricht der HERR: Gleichwie ich über dies Volk habe kommen lassen all dies große Unglück, also will ich auch alles Gute über sie kommen lassen, das ich ihnen verheißen habe.
\par 43 Und sollen noch Äcker gekauft werden in diesem Lande, davon ihr sagt, es werde wüst liegen, daß weder Leute noch Vieh darin bleiben, und es werde in der Chaldäer Hände gegeben.
\par 44 Dennoch wird man Äcker um Geld kaufen und verbriefen, versiegeln und bezeugen im Lande Benjamin und um Jerusalem her und in den Städten Juda's, in Städten auf den Gebirgen, in Städten in den Gründen und in Städten gegen Mittag; denn ich will ihr Gefängnis wenden, spricht der HERR.

\chapter{33}

\par 1 Und des HERRN Wort geschah zu Jeremia zum andernmal, da er noch im Vorhof des Gefängnisses verschlossen war, und sprach:
\par 2 So spricht der HERR, der solches macht, tut und ausrichtet, HERR ist sein Name:
\par 3 Rufe mich an, so will ich dir antworten und will dir anzeigen große und gewaltige Dinge, die du nicht weißt.
\par 4 Denn so spricht der HERR, der Gott Israels, von den Häusern dieser Stadt und von den Häusern der Könige Juda's, welche abgebrochen sind, Bollwerke zu machen zur Wehr.
\par 5 Und von denen, so hereingekommen sind, wider die Chaldäer zu streiten, daß sie diese füllen müssen mit Leichnamen der Menschen, welche ich in meinem Zorn und Grimm erschlagen will; denn ich habe mein Angesicht vor dieser Stadt verborgen um all ihrer Bosheit willen:
\par 6 Siehe, ich will sie heilen und gesund machen und will ihnen Frieden und Treue die Fülle gewähren.
\par 7 Denn ich will das Gefängnis Juda's und das Gefängnis Israels wenden und will sie bauen wie von Anfang
\par 8 und will sie reinigen von aller Missetat, damit sie wider mich gesündigt haben, und will ihnen vergeben alle Missetaten, damit sie wider mich gesündigt und übertreten haben.
\par 9 Und das soll mir ein fröhlicher Name, Ruhm und Preis sein unter allen Heiden auf Erden, wenn sie hören werden all das Gute, das ich ihnen tue. Und sie werden sich verwundern und entsetzen über all dem Guten und über all dem Frieden, den ich ihnen geben will.
\par 10 So spricht der HERR: An diesem Ort, davon ihr sagt: Er ist wüst, weil weder Leute noch Vieh in den Städten Juda's und auf den Gassen zu Jerusalem bleiben, die so verwüstet sind, daß weder Leute noch Vieh darin sind,
\par 11 wird man dennoch wiederum hören Geschrei von Freude und Wonne, die Stimme des Bräutigams und der Braut und die Stimme derer, so da sagen: "Danket dem HERRN Zebaoth, denn er ist freundlich, und seine Güte währet ewiglich", wenn sie Dankopfer bringen zum Hause des HERRN. Denn ich will des Landes Gefängnis wenden wie von Anfang, spricht der HERR.
\par 12 So spricht der HERR Zebaoth: An diesem Ort, der so wüst ist, daß weder Leute noch Vieh darin sind, und in allen seinen Städten werden dennoch wiederum Wohnungen sein der Hirten, die da Herden weiden.
\par 13 In Städten auf den Gebirgen und in Städten in Gründen und in Städten gegen Mittag, im Lande Benjamin und um Jerusalem her und in Städten Juda's sollen dennoch wiederum die Herden gezählt aus und ein gehen, spricht der HERR.
\par 14 Siehe, es kommt die Zeit, spricht der HERR, daß ich das gnädige Wort erwecken will, welches ich dem Hause Israel und dem Hause Juda geredet habe.
\par 15 In denselben Tagen und zu derselben Zeit will ich dem David ein gerechtes Gewächs aufgehen lassen, und er soll Recht und Gerechtigkeit anrichten auf Erden.
\par 16 Zu derselben Zeit soll Juda geholfen werden und Jerusalem sicher wohnen, und man wird sie nennen: Der HERR unsre Gerechtigkeit.
\par 17 Denn so spricht der HERR: Es soll nimmermehr fehlen, es soll einer von David sitzen auf dem Stuhl des Hauses Israel.
\par 18 Desgleichen soll's nimmermehr fehlen, es sollen Priester und Leviten sein vor mir, die da Brandopfer tun und Speisopfer anzünden und Opfer schlachten ewiglich.
\par 19 Und des HERRN Wort geschah zu Jeremia und sprach:
\par 20 So spricht der HERR: Wenn mein Bund aufhören wird mit Tag und Nacht, daß nicht Tag und Nacht sei zu seiner Zeit,
\par 21 so wird auch mein Bund aufhören mit meinem Knechte David, daß er nicht einen Sohn habe zum König auf seinem Stuhl, und mit den Leviten und Priestern, meinen Dienern.
\par 22 Wie man des Himmels Heer nicht zählen noch den Sand am Meer messen kann, also will ich mehren den Samen Davids, meines Knechtes, und die Leviten, die mir dienen.
\par 23 Und des HERRN Wort geschah zu Jeremia und sprach:
\par 24 Hast du nicht gesehen, was dies Volk redet und spricht: "Hat doch der HERR auch die zwei Geschlechter verworfen, welche er auserwählt hatte"; und lästern mein Volk, als sollten sie nicht mehr mein Volk sein.
\par 25 So spricht der HERR: Halte ich meinen Bund nicht Tag und Nacht noch die Ordnungen des Himmels und der Erde,
\par 26 so will ich auch verwerfen den Samen Jakobs und Davids, meines Knechtes, daß ich nicht aus ihrem Samen nehme, die da herrschen über den Samen Abrahams, Isaaks und Jakobs. Denn ich will ihr Gefängnis wenden und mich über sie erbarmen.

\chapter{34}

\par 1 Dies ist das Wort, das vom HERRN geschah zu Jeremia, da Nebukadnezar, der König zu Babel, samt seinem Heer und allen Königreichen auf Erden, so unter seiner Gewalt waren, und allen Völkern stritt wider Jerusalem und alle ihre Städte, und sprach:
\par 2 So spricht der HERR, der Gott Israels: Gehe hin und sage Zedekia, dem König Juda's, und sprich zu ihm: So spricht der HERR: Siehe, ich will diese Stadt in die Hände des Königs zu Babel geben, und er soll sie mit Feuer verbrennen.
\par 3 Und du sollst seiner Hand nicht entrinnen, sondern gegriffen und in seine Hand gegeben werden, daß du ihn mit Augen sehen und mündlich mit ihm reden wirst, und gen Babel kommen.
\par 4 Doch aber höre, Zedekia, du König Juda's, des HERRN Wort: So spricht der HERR von dir: Du sollst nicht durchs Schwert sterben,
\par 5 sondern du sollst im Frieden sterben. Und wie deinen Vätern, den vorigen Königen, die vor dir gewesen sind, so wird man auch dir einen Brand anzünden und dich beklagen: "Ach Herr!" denn ich habe es geredet, spricht der HERR.
\par 6 Und der Prophet Jeremia redete alle diese Worte zu Zedekia, dem König Juda's, zu Jerusalem,
\par 7 da das Heer des Königs zu Babel schon stritt wider Jerusalem und wider alle übrigen Städte Juda's, nämlich wider Lachis und Aseka; denn diese waren noch übriggeblieben von den festen Städten Juda's.
\par 8 Dies ist das Wort, so vom HERRN geschah zu Jeremia, nachdem der König Zedekia einen Bund gemacht hatte mit dem ganzen Volk zu Jerusalem, ein Freijahr auszurufen,
\par 9 daß ein jeglicher seinen Knecht und seine Magd, so Hebräer und Hebräerin wären, sollte freigeben, daß kein Jude den andern leibeigen hielte.
\par 10 Da gehorchten alle Fürsten und alles Volk, die solchen Bund eingegangen waren, daß ein jeglicher sollte seinen Knecht und seine Magd freigeben und sie nicht mehr leibeigen halten, und gaben sie los.
\par 11 Aber darnach kehrten sie sich um und forderten die Knechte und Mägde wieder zu sich, die sie freigegeben hatten, und zwangen sie, daß sie Knechte und Mägde sein mußten.
\par 12 Da geschah des HERRN Wort zu Jeremia vom HERRN und sprach:
\par 13 So spricht der HERR, der Gott Israels: Ich habe einen Bund gemacht mit euren Vätern, da ich sie aus Ägyptenland, aus dem Diensthause, führte und sprach:
\par 14 Im siebenten Jahr soll ein jeglicher seinen Bruder, der ein Hebräer ist und sich ihm verkauft und sechs Jahre gedient hat, frei von sich lassen. Aber eure Väter gehorchten mir nicht und neigten ihre Ohren nicht.
\par 15 So habt ihr euch heute bekehrt und getan, was mir wohl gefiel, daß ihr ein Freijahr ließet ausrufen, ein jeglicher seinem Nächsten; und habt darüber einen Bund gemacht vor mir im Hause, das nach meinem Namen genannt ist.
\par 16 Aber ihr seid umgeschlagen und entheiligt meinen Namen; und ein jeglicher fordert seinen Knecht und seine Magd wieder, die ihr hattet freigegeben, daß sie selbst eigen wären, und zwingt sie nun, daß sie eure Knechte und Mägde sein müssen.
\par 17 Darum spricht der HERR also: Ihr gehorchtet mir nicht, daß ihr ein Freijahr ausriefet ein jeglicher seinem Bruder und seinem Nächsten; siehe, so rufe ich, spricht der HERR, euch ein Freijahr aus zum Schwert, zur Pestilenz, zum Hunger, und will euch in keinem Königreich auf Erden bleiben lassen.
\par 18 Und will die Leute, die meinen Bund übertreten und die Worte des Bundes, den sie vor mir gemacht haben, nicht halten, so machen wie das Kalb, das sie in zwei Stücke geteilt haben und sind zwischen den Teilen hingegangen,
\par 19 nämlich die Fürsten Juda's, die Fürsten Jerusalems, die Kämmerer, die Priester und das ganze Volk im Lande, so zwischen des Kalbes Stücken hingegangen sind.
\par 20 Und will sie geben in ihrer Feinde Hand und derer, die ihnen nach dem Leben stehen, daß ihre Leichname sollen den Vögeln unter dem Himmel und den Tieren auf Erden zur Speise werden.
\par 21 Und Zedekia, den König Juda's, und seine Fürsten will ich geben in die Hände ihrer Feinde und derer, die ihnen nach dem Leben stehen, und dem Heer des Königs zu Babel, die jetzt von euch abgezogen sind.
\par 22 Denn siehe, ich will ihnen befehlen, spricht der HERR, und will sie wieder vor diese Stadt bringen, und sollen wider sie streiten und sie gewinnen und mit Feuer verbrennen; und ich will die Städte Juda's verwüsten, daß niemand mehr da wohnen soll.

\chapter{35}

\par 1 Dies ist das Wort, das vom HERRN geschah zu Jeremia zur Zeit Jojakims, des Sohnes Josias, des Königs in Juda, und sprach:
\par 2 Gehe hin zum Hause der Rechabiter und rede mit ihnen und führe sie in des HERRN Haus, in der Kapellen eine, und schenke ihnen Wein.
\par 3 Da nahm ich Jaasanja, den Sohn Jeremia's, des Sohnes Habazinjas, samt seinen Brüdern und allen seinen Söhnen und das ganze Haus der Rechabiter
\par 4 und führte sie in des HERRN Haus, in die Kapelle der Kinder Hanans, des Sohnes Jigdaljas, des Mannes Gottes, welche neben der Fürstenkapelle ist, über der Kapelle Maasejas, des Sohnes Sallums, des Torhüters.
\par 5 Und ich setzte den Kindern von der Rechabiter Hause Becher voll Wein und Schalen vor und sprach zu ihnen: Trinkt Wein!
\par 6 Sie aber antworteten: Wir trinken nicht Wein; denn unser Vater Jonadab, der Sohn Rechabs, hat uns geboten und gesagt: Ihr und eure Kinder sollt nimmermehr Wein trinken
\par 7 und kein Haus bauen, keinen Samen säen, keinen Weinberg pflanzen noch haben, sondern sollt in Hütten wohnen euer Leben lang, auf daß ihr lange lebt in dem Lande, darin ihr wallt.
\par 8 Also gehorchen wir der Stimme unsers Vater Jonadab, des Sohnes Rechabs, in allem, was er uns geboten hat, daß wir keinen Wein trinken unser Leben lang, weder wir noch unsre Weiber noch Söhne noch Töchter,
\par 9 und bauen auch keine Häuser, darin wir wohnten, und haben weder Weinberge noch Äcker noch Samen,
\par 10 sondern wohnen in Hütten und gehorchen und tun alles, wie unser Vater Jonadab geboten hat.
\par 11 Als aber Nebukadnezar, der König zu Babel, herauf ins Land zog, sprachen wir: "Kommt, laßt uns gen Jerusalem ziehen vor dem Heer der Chaldäer und der Syrer!" und sind also zu Jerusalem geblieben.
\par 12 Da geschah des HERRN Wort zu Jeremia und sprach:
\par 13 So spricht der HERR Zebaoth, der Gott Israels; gehe hin und sprich zu denen in Juda und zu den Bürgern zu Jerusalem: Wollt ihr euch denn nicht bessern, daß ihr meinem Wort gehorcht? spricht der HERR.
\par 14 Die Worte Jonadabs, des Sohnes Rechabs, die er den Kindern geboten hat, daß sie nicht sollen Wein trinken, werden gehalten, und sie trinken keinen Wein bis auf diesen Tag, darum daß sie ihres Vaters Gebot gehorchen. Ich aber habe stets euch predigen lassen; doch gehorchtet ihr mir nicht.
\par 15 So habe ich auch stets zu euch gesandt alle meine Knechte, die Propheten, und lasse sagen: Bekehrt euch ein jeglicher von seinem bösen Wesen, und bessert euren Wandel und folgt nicht andern Göttern nach, ihnen zu dienen, so sollt ihr in dem Lande bleiben, welches ich euch und euren Vätern gegeben habe. Aber ihr wolltet eure Ohren nicht neigen noch mir gehorchen,
\par 16 so doch die Kinder Jonadabs, des Sohnes Rechabs, haben ihres Vaters Gebot, das er ihnen geboten hat, gehalten. Aber dies Volk gehorchte mir nicht.
\par 17 Darum so spricht der HERR, der Gott Zebaoth und der Gott Israels: Siehe, ich will über Juda und über alle Bürger zu Jerusalem kommen lassen all das Unglück, das ich wider sie geredet habe, darum daß ich zu ihnen geredet habe und sie nicht wollen hören, daß ich gerufen habe und sie mir nicht wollen antworten.
\par 18 Und zum Hause der Rechabiter sprach Jeremia: So spricht der HERR Zebaoth, der Gott Israels: Darum daß ihr dem Gebot eures Vaters Jonadab habt gehorcht und alle seine Gebote gehalten und alles getan, was er euch geboten hat,
\par 19 darum spricht der HERR Zebaoth, der Gott Israels, also: Es soll dem Jonadab, dem Sohne Rechabs, nimmer fehlen, es soll jemand von den Seinen allezeit vor mir stehen.

\chapter{36}

\par 1 Im vierten Jahr Jojakims, des Sohnes Josias, des Königs in Juda, geschah dies Wort zu Jeremia vom HERRN und sprach:
\par 2 Nimm ein Buch und schreibe darein alle Reden, die ich zu dir geredet habe über Israel, über Juda und alle Völker von der Zeit an, da ich zu dir geredet habe, nämlich von der Zeit Josias an bis auf diesen Tag;
\par 3 ob vielleicht die vom Hause Juda, wo sie hören all das Unglück, das ich ihnen gedenke zu tun, sich bekehren wollten, ein jeglicher von seinem bösen Wesen, damit ich ihnen ihre Missetat und Sünde vergeben könnte.
\par 4 Da rief Jeremia Baruch, den Sohn Nerias. Derselbe Baruch schrieb in ein Buch aus dem Munde Jeremia's alle Reden des HERRN, die er zu ihm geredet hatte.
\par 5 Und Jeremia gebot Baruch und sprach: Ich bin gefangen, daß ich nicht kann in des HERRN Haus gehen.
\par 6 Du aber gehe hinein und lies das Buch, darein du des HERRN Reden aus meinem Munde geschrieben hast, vor dem Volk im Hause des HERRN am Fasttage, und sollst sie auch lesen vor den Ohren des ganzen Juda, die aus ihren Städten hereinkommen;
\par 7 ob sie vielleicht sich mit Beten vor dem HERRN demütigen wollen und sich bekehren, ein jeglicher von seinem bösen Wesen; denn der Zorn und Grimm ist groß, davon der HERR wider dies Volk geredet hat.
\par 8 Und Baruch, der Sohn Nerias, tat alles, wie ihm der Prophet Jeremia befohlen hatte, daß er die Reden des HERRN aus dem Buche läse im Hause des HERRN.
\par 9 Es begab sich aber im fünften Jahr Jojakims, des Sohnes Josias, des Königs Juda's, im neunten Monat, daß man ein Fasten verkündigte vor dem HERRN allem Volk zu Jerusalem und allem Volk, das aus den Städten Juda's gen Jerusalem kommt.
\par 10 Und Baruch las aus dem Buche die Reden Jeremia's im Hause des HERRN, in der Kapelle Gemarjas, des Sohnes Saphans, des Kanzlers, im obern Vorhof, vor dem neuen Tor am Hause des HERRN, vor dem ganzen Volk.
\par 11 Da nun Michaja, der Sohn Gemarjas, des Sohnes Saphans, alle Reden des HERRN gehört hatte aus dem Buche,
\par 12 ging er hinab in des Königs Haus, in die Kanzlei. Und siehe, daselbst saßen alle Fürsten: Elisama, der Kanzler, Delaja, der Sohn Semajas, Elnathan, der Sohn Achbors, Gemarja, der Sohn Saphans, und Zedekia, der Sohn Hananjas, samt allen Fürsten.
\par 13 Und Michaja zeigte ihnen an alle Reden, die er gehört hatte, da Baruch las aus dem Buche vor den Ohren des Volks.
\par 14 Da sandten alle Fürsten Judi, den Sohn Nethanjas, des Sohnes Selemjas, des Sohnes Chusis, nach Baruch und ließen ihm sagen: Nimm das Buch daraus du vor dem Volk gelesen hast, mit dir und komme! Und Baruch, der Sohn Nerias, nahm das Buch mit sich und kam zu ihnen.
\par 15 Und sie sprachen zu ihm: Setze dich und lies, daß wir es hören! Und Baruch las ihnen vor ihren Ohren.
\par 16 Und da sie alle die Reden hörten, entsetzten sie sich einer gegen den andern und sprachen zu Baruch: Wir wollen alle diese Reden dem König anzeigen.
\par 17 Und sie fragten Baruch: Sage uns, wie hast du alle diese Reden aus seinem Munde geschrieben?
\par 18 Baruch sprach zu ihnen: Er sagte vor mir alle diese Reden aus seinem Munde, und ich schrieb sie mit Tinte ins Buch.
\par 19 Da sprachen die Fürsten zu Baruch: Gehe hin und verbirg dich mit Jeremia, daß niemand wisse, wo ihr seid.
\par 20 Sie aber gingen hin zum König in den Vorhof und ließen das Buch behalten in der Kammer Elisamas, des Kanzlers, und sagten vor dem König an alle diese Reden.
\par 21 Da sandte der König den Judi, das Buch zu holen. Der nahm es aus der Kammer Elisamas, des Kanzlers. Und Judi las vor dem König und allen Fürsten, die bei dem König standen.
\par 22 Der König aber saß im Winterhause, im neunten Monat, vor dem Kamin.
\par 23 Wenn aber Judi drei oder vier Blatt gelesen hatte, zerschnitt er es mit einem Schreibmesser und warf es ins Feuer, das im Kaminherde war, bis das Buch ganz verbrannte im Feuer,
\par 24 und niemand entsetzte sich noch zerriß seine Kleider, weder der König noch seine Knechte, so doch alle diese Reden gehört hatten,
\par 25 und wiewohl Elnathan, Delaja und Gemarja den König baten, er wolle das Buch nicht verbrennen, gehorchte er ihnen doch nicht.
\par 26 Dazu gebot noch der König Jerahmeel, dem Königssohn, und Seraja, dem Sohn Asriels, und Selemja, dem Sohn Abdeels, sie sollten Baruch, den Schreiber, und Jeremia, den Propheten, greifen. Aber der HERR hatte sie verborgen.
\par 27 Da geschah des HERRN Wort zu Jeremia, nachdem der König das Buch und die Reden, so Baruch geschrieben aus dem Munde Jeremia's, verbrannt hatte, und sprach:
\par 28 Nimm dir wiederum ein anderes Buch und schreib alle vorigen Reden darein, die im ersten Buche standen, welches Jojakim, der König Juda's, verbrannt hat,
\par 29 und sage von Jojakim, dem König Juda's: So spricht der HERR: Du hast dies Buch verbrannt und gesagt: Warum hast du darein geschrieben, daß der König von Babel werde kommen und dies Land verderben und machen, daß weder Leute noch Vieh darin mehr sein werden?
\par 30 Darum spricht der HERR von Jojakim, dem König Juda's: Es soll keiner von den Seinen auf dem Stuhl Davids sitzen, und sein Leichnam soll hingeworfen des Tages in der Hitze und des Nachts im Frost liegen;
\par 31 und ich will ihn und seinen Samen und seine Knechte heimsuchen um ihrer Missetat willen; und ich will über sie und über die Bürger zu Jerusalem und über die in Juda kommen lassen all das Unglück, davon ich ihnen geredet habe, und sie gehorchten doch nicht.
\par 32 Da nahm Jeremia ein anderes Buch und gab's Baruch, dem Sohn Nerias, dem Schreiber. Der schrieb darein aus dem Munde Jeremia's alle die Reden, so in dem Buch standen, das Jojakim, der König Juda's, hatte mit Feuer verbrennen lassen; und zu denselben wurden dergleichen Reden noch viele hinzugetan.

\chapter{37}

\par 1 Und da Zedekia, der Sohn Josias, ward König anstatt Jechonjas, des Sohnes Jojakims; denn Nebukadnezar, der König zu Babel machte ihn zum König im Lande Juda.
\par 2 Aber er und seine Knechte und das Volk im Lande gehorchten nicht des HERRN Worten, die er durch den Propheten Jeremia redete.
\par 3 Es sandte gleichwohl der König Zedekia Juchal, den Sohn Selemjas, und Zephanja, den Sohn Maasejas, den Priester, zum Propheten Jeremia und ließ ihm sagen: Bitte den HERRN, unsern Gott, für uns!
\par 4 Denn Jeremia ging unter dem Volk aus und ein, und niemand legte ihn ins Gefängnis.
\par 5 Es war aber das Heer Pharaos aus Ägypten gezogen: und die Chaldäer, so vor Jerusalem lagen, da sie solch Gerücht gehört hatten, waren von Jerusalem abgezogen.
\par 6 Und des HERRN Wort geschah zum Propheten Jeremia und sprach:
\par 7 So spricht der HERR, der Gott Israels: So sagt dem König Juda's, der euch zu mir gesandt hat, mich zu fragen: Siehe, das Heer Pharaos, das euch zu Hilfe ist ausgezogen, wird wiederum heim nach Ägypten ziehen;
\par 8 und die Chaldäer werden wiederkommen und wider diese Stadt streiten und sie gewinnen und mit Feuer verbrennen.
\par 9 Darum spricht der HERR also: Betrügt eure Seelen nicht, daß ihr denkt, die Chaldäer werden von uns abziehen; sie werden nicht abziehen.
\par 10 Und wenn ihr schon schlüget das ganze Heer der Chaldäer, so wider euch streiten, und blieben ihrer etliche verwundet übrig, so würden sie doch, ein jeglicher in seinem Gezelt, sich aufmachen und diese Stadt mit Feuer verbrennen.
\par 11 Als nun der Chaldäer Heer von Jerusalem war abgezogen um des Heeres willen Pharaos,
\par 12 ging Jeremia aus Jerusalem und wollte ins Land Benjamin gehen, seinen Acker in Besitz zu nehmen unter dem Volk.
\par 13 Und da er unter das Tor Benjamin kam, da war einer bestellt zum Torhüter, mit Namen Jeria, der Sohn Selemjas, des Sohnes Hananjas; der griff den Propheten Jeremia und sprach: Du willst zu den Chaldäern fallen.
\par 14 Jeremia sprach: Das ist nicht wahr; ich will nicht zu den Chaldäern fallen. Aber Jeria wollte ihn nicht hören, sondern griff Jeremia und brachte ihn zu den Fürsten.
\par 15 Und die Fürsten wurden zornig über Jeremia und ließen ihn schlagen und warfen ihn ins Gefängnis im Hause Jonathans, des Schreibers; den setzten sie zum Kerkermeister.
\par 16 Also ging Jeremia in die Grube und den Kerker und lag lange Zeit daselbst.
\par 17 Und Zedekia, der König, sandte hin und ließ ihn holen und fragte ihn heimlich in seinem Hause und sprach: Ist auch ein Wort vom HERRN vorhanden? Jeremia sprach: Ja; denn du wirst dem König zu Babel in die Hände gegeben werden.
\par 18 Und Jeremia sprach zum König Zedekia: Was habe ich wider dich, wider deine Knechte und wider dein Volk gesündigt, daß sie mich in den Kerker geworfen haben?
\par 19 Wo sind nun eure Propheten, die euch weissagten und sprachen: Der König zu Babel wird nicht über euch noch über dies Land kommen?
\par 20 Und nun, mein Herr König, höre mich und laß meine Bitte vor dir gelten und laß mich nicht wieder in Jonathans, des Schreibers, Haus bringen, daß ich nicht sterbe daselbst.
\par 21 Da befahl der König Zedekia, daß man Jeremia im Vorhof des Gefängnisses behalten sollte, und ließ ihm des Tages ein Laiblein Brot geben aus der Bäckergasse, bis daß alles Brot in der Stadt aufgezehrt war. Also blieb Jeremia im Vorhof des Gefängnisses.

\chapter{38}

\par 1 Es hörten aber Sephatja, der Sohn Matthans, und Gedalja, der Sohn Pashurs, und Juchal, der Sohn Selemjas, und Pashur, der Sohn Malchias, die Reden, so Jeremia zu allem Volk redete und sprach:
\par 2 So spricht der HERR: Wer in dieser Stadt bleibt, der wird durch Schwert, Hunger und Pestilenz sterben müssen; wer aber hinausgeht zu den Chaldäern, der soll lebend bleiben und wird sein Leben wie eine Beute davonbringen.
\par 3 Denn also spricht der HERR: Diese Stadt soll übergeben werden dem Heer des Königs zu Babel, und sie sollen sie gewinnen.
\par 4 Da sprachen die Fürsten zum König: Laß doch diesen Mann töten; denn mit der Weise wendet er die Kriegsleute ab, so noch übrig sind in der Stadt, desgleichen das ganze Volk auch, weil er solche Worte zu ihnen sagt. Denn der Mann sucht nicht, was diesem Volk zum Frieden, sondern zum Unglück dient.
\par 5 Der König Zedekia sprach: Siehe, er ist in euren Händen; denn der König kann nichts wider euch.
\par 6 Da nahmen sie Jeremia und warfen ihn in die Grube Malchias, des Königssohnes, die am Vorhof des Gefängnisses war, da nicht Wasser, sondern Schlamm war, und Jeremia sank in den Schlamm.
\par 7 Als aber Ebed-Melech, der Mohr, ein Kämmerer in des Königs Hause, hörte, daß man Jeremia hatte in die Grube geworfen, und der König eben saß im Tor Benjamin,
\par 8 da ging Ebed-Melech aus des Königs Hause und redete mit dem König und sprach:
\par 9 Mein Herr König, die Männer handeln übel an dem Propheten Jeremia, daß sie ihn haben in die Grube geworfen, da er muß Hungers sterben; denn es ist kein Brot mehr in der Stadt.
\par 10 Da befahl der König Ebed-Melech, dem Mohren, und sprach: Nimm dreißig Männer mit dir von diesen und zieh den Propheten Jeremia aus der Grube, ehe denn er sterbe.
\par 11 Und Ebed-Melech nahm die Männer mit sich und ging in des Königs Haus unter die Schatzkammer und nahm daselbst zerrissene und vertragene alte Lumpen und ließ sie an einem Seil hinab zu Jeremia in die Grube.
\par 12 Und Ebed-Melech, der Mohr, sprach zu Jeremia: Lege diese zerrissenen und vertragenen alten Lumpen unter deine Achseln um das Seil. Und Jeremia tat also.
\par 13 Und sie zogen Jeremia herauf aus der Grube an den Stricken; und blieb also Jeremia im Vorhof des Gefängnisses.
\par 14 Und der König Zedekia sandte hin und ließ den Propheten Jeremia zu sich holen unter den dritten Eingang am Hause des HERRN. Und der König sprach zu Jeremia: Ich will dich etwas fragen; verhalte mir nichts.
\par 15 Jeremia sprach zu Zedekia: Sage ich dir etwas, so tötest du mich doch; gebe ich dir aber einen Rat, so gehorchst du mir nicht.
\par 16 Da schwur der König Zedekia dem Jeremia heimlich und sprach: So wahr der HERR lebt, der uns dieses Leben gegeben hat, so will ich dich nicht töten noch den Männern in die Hände geben, die dir nach dem Leben stehen.
\par 17 Und Jeremia sprach zu Zedekia: So spricht der HERR, der Gott Zebaoth, der Gott Israels: Wirst du hinausgehen zu den Fürsten des Königs zu Babel, so sollst du leben bleiben, und diese Stadt soll nicht verbrannt werden, sondern du und dein Haus sollen am Leben bleiben;
\par 18 wirst du aber nicht hinausgehen zu den Fürsten des Königs zu Babel, so wird diese Stadt den Chaldäern in die Hände gegeben, und sie werden sie mit Feuer verbrennen, und du wirst auch nicht ihren Händen entrinnen.
\par 19 Der König Zedekia sprach zu Jeremia: Ich sorge mich aber, daß ich den Juden, so zu den Chaldäern gefallen sind, möchte übergeben werden, daß sie mein spotten.
\par 20 Jeremia sprach: Man wird dich nicht übergeben. Gehorche doch der Stimme des HERRN, die ich dir sage, so wird dir's wohl gehen, und du wirst lebend bleiben.
\par 21 Wirst du aber nicht hinausgehen, so ist dies das Wort, das mir der HERR gezeigt hat:
\par 22 Siehe, alle Weiber, die noch vorhanden sind in dem Hause des Königs in Juda, werden hinaus müssen zu den Fürsten des Königs zu Babel; diese werden dann sagen: Ach deine Tröster haben dich überredet und verführt und in Schlamm geführt und lassen dich nun stecken.
\par 23 Also werden dann alle deine Weiber und Kinder hinaus müssen zu den Chaldäern, und du selbst wirst ihren Händen nicht entgehen; sondern du wirst vom König von Babel gegriffen, und diese Stadt wird mit Feuer verbrannt werden.
\par 24 Und Zedekia sprach zu Jeremia: Siehe zu, daß niemand diese Rede erfahre, so wirst du nicht sterben.
\par 25 Und wenn's die Fürsten erführen, daß ich mit dir geredet habe, und kämen zu dir und sprächen: Sage an, was hast du mit dem König geredet, leugne es uns nicht, so wollen wir dich nicht töten, und was hat der König mit dir geredet?
\par 26 so sprich: Ich habe den König gebeten, daß er mich nicht wiederum ließe in des Jonathan Haus führen; ich möchte daselbst sterben.
\par 27 Da kamen alle Fürsten zu Jeremia und fragten ihn; und er sagte ihnen, wie ihm der König befohlen hatte. Da ließen sie von ihm, weil sie nichts erfahren konnten.
\par 28 Und Jeremia blieb im Vorhof des Gefängnisses bis auf den Tag, da Jerusalem gewonnen ward.

\chapter{39}

\par 1 Und es geschah, daß Jerusalem gewonnen ward. Denn im neunten Jahr Zedekias, des Königs in Juda, im zehnten Monat, kam Nebukadnezar, der König zu Babel, und all sein Heer vor Jerusalem und belagerten es.
\par 2 Und im elften Jahr Zedekias, am neunten Tage des vierten Monats, brach man in die Stadt;
\par 3 und zogen hinein alle Fürsten des Königs zu Babel und hielten unter dem Mitteltor, nämlich Nergal-Sarezer, Samgar-Nebo, Sarsechim, der oberste Kämmerer, Nergal-Sarezer, der Oberste der Weisen, und alle andern Fürsten des Königs zu Babel.
\par 4 Als sie nun Zedekia, der König Juda's, sah samt seinen Kriegsleuten, flohen sie bei Nacht zur Stadt hinaus bei des Königs Garten durchs Tor zwischen den zwei Mauern und zogen des Weges zum blachen Feld.
\par 5 Aber der Chaldäer Kriegsleute jagten ihnen nach und ergriffen Zedekia im Felde bei Jericho und fingen ihn und brachten ihn zu Nebukadnezar, dem König zu Babel, gen Ribla, das im Lande Hamath liegt; der sprach ein Urteil über ihn.
\par 6 Und der König zu Babel ließ die Söhne Zedekias vor seinen Augen töten zu Ribla und tötete alle Fürsten Juda's.
\par 7 Aber Zedekia ließ er die Augen ausstechen und ihn in Ketten binden, daß er ihn gen Babel führte.
\par 8 Und die Chaldäer verbrannten beide, des Königs Haus und der Bürger Häuser, und zerbrachen die Mauern zu Jerusalem.
\par 9 Was aber noch von Volk in der Stadt war, und was sonst zu ihnen gefallen war, die führte Nebusaradan, der Hauptmann der Trabanten, alle miteinander gen Babel gefangen.
\par 10 Aber von dem geringen Volk, das nichts hatte, ließ zu derselben Zeit Nebusaradan, der Hauptmann, etliche im Lande Juda und gab ihnen Weinberge und Felder.
\par 11 Aber Nebukadnezar, der König zu Babel, hatte Nebusaradan, dem Hauptmann, befohlen von Jeremia und gesagt:
\par 12 Nimm ihn und laß ihn dir befohlen sein und tu ihm kein Leid; sondern wie er's von dir begehrt, so mache es mit ihm.
\par 13 Da sandten hin Nebusaradan, der Hauptmann, und Nebusasban, der oberste Kämmerer, Nergal-Sarezer, der Oberste der Weisen, und alle Fürsten des Königs zu Babel
\par 14 und ließen Jeremia holen aus dem Vorhof des Gefängnisses und befahlen ihn Gedalja, dem Sohn Ahikams, des Sohnes Saphans, daß er ihn hinaus in sein Haus führte. Und er blieb bei dem Volk.
\par 15 Es war auch des HERRN Wort geschehen zu Jeremia, als er noch im Vorhof des Gefängnisses gefangen lag, und hatte gesprochen:
\par 16 Gehe hin und sage Ebed-Melech, dem Mohren: So spricht der HERR Zebaoth, der Gott Israels: siehe, ich will meine Worte kommen lassen über diese Stadt zum Unglück und zu keinem Guten, und du sollst es sehen zur selben Zeit.
\par 17 Aber dich will ich erretten zur selben Zeit, spricht der HERR, und sollst den Leuten nicht zuteil werden, vor welchen du dich fürchtest.
\par 18 Denn ich will dir davonhelfen, daß du nicht durchs Schwert fällst, sondern sollst dein Leben wie eine Beute davonbringen, darum daß du mir vertraut hast, spricht der HERR.

\chapter{40}

\par 1 Dies ist das Wort, so vom HERRN geschah zu Jeremia, da ihn Nebusaradan, der Hauptmann, losließ zu Rama; denn er war mit Ketten gebunden unter allen denen, die zu Jerusalem und in Juda gefangen waren, daß man sie gen Babel wegführen sollte.
\par 2 Da nun der Hauptmann Jeremia zu sich hatte lassen holen, sprach er zu ihm: Der HERR, dein Gott, hat dies Unglück über diese Stätte geredet.
\par 3 und hat's auch kommen lassen und getan, wie er geredet hat; denn ihr habt gesündigt wider den HERRN und seiner Stimme nicht gehorcht; darum ist euch solches widerfahren.
\par 4 Und nun siehe, ich habe dich heute losgemacht von den Ketten, womit deine Hände gebunden waren. Gefällt dir's, mit mir gen Babel zu ziehen, so komm du sollst mir befohlen sein; gefällt dir's aber nicht, mit mir gen Babel zu ziehen, so laß es anstehen. Siehe, da hast du das ganze Land vor dir; wo dich's gut dünkt und dir gefällt, da zieh hin.
\par 5 Denn weiter hinaus wird kein Wiederkehren sein. Darum magst du umkehren zu Gedalja, dem Sohn Ahikams, des Sohnes Saphans, welchen der König zu Babel gesetzt hat über die Städte in Juda, und bei ihm unter dem Volk bleiben; oder gehe, wohin dir's wohl gefällt. Und der Hauptmann gab ihm Zehrung und Geschenke und ließ ihn gehen.
\par 6 Also kam Jeremia zu Gedalja, dem Sohn Ahikams, gen Mizpa und blieb bei ihm unter dem Volk, das im Lande noch übrig war.
\par 7 Da nun die Hauptleute, so auf dem Felde sich hielten, samt ihren Leuten erfuhren, daß der König zu Babel hatte Gedalja, den Sohn Ahikams, über das Land gesetzt und über die Männer und Weiber, Kinder und die Geringen im Lande, welche nicht gen Babel geführt waren,
\par 8 kamen sie zu Gedalja gen Mizpa, nämlich Ismael, der Sohn Nethanjas, Johanan und Jonathan, die Söhne Kareahs, und Seraja, der Sohn Thanhumeths, und die Söhne Ephais von Netopha und Jesanja, der Sohn eines Maachathiters, samt ihren Männern.
\par 9 Und Gedalja, der Sohn Ahikams, des Sohnes Saphans, tat ihnen und ihren Männern einen Eid und sprach: Fürchtet euch nicht, daß ihr den Chaldäern untertan sein sollt; bleibt im Lande und seid dem König zu Babel untertan, so wird's euch wohl gehen.
\par 10 Siehe, ich wohne hier zu Mizpa, daß ich den Chaldäern diene, die zu uns kommen; darum sammelt ein Wein und Feigen und Öl und legt's in eure Gefäße und wohnt in euren Städten, die ihr bekommen habt.
\par 11 Auch allen Juden, so im Lande Moab und der Kinder Ammon und in Edom und in allen Ländern waren, da sie hörten, daß der König zu Babel hätte lassen etliche in Juda übrigbleiben und über sie gesetzt Gedalja, den Sohn Ahikams, des Sohnes Saphans,
\par 12 kamen sie alle wieder von allen Orten dahin sie verstoßen waren, in das Land Juda zu Gedalja gen Mizpa und sammelten ein sehr viel Wein und Sommerfrüchte.
\par 13 Aber Johanan, der Sohn Kareahs, samt allen Hauptleuten, so auf dem Felde sich gehalten hatten, kamen zu Gedalja gen Mizpa
\par 14 und sprachen zu ihm: Weißt du auch, daß Baalis, der König der Kinder Ammon, gesandt hat Ismael, den Sohn Nethanjas, daß er dich soll erschlagen? Das wollte ihnen aber Gedalja, der Sohn Ahikams, nicht glauben.
\par 15 Da sprach Johanan, der Sohn Kareahs, zu Gedalja heimlich zu Mizpa: Ich will hingehen und Ismael, den Sohn Nethanjas, erschlagen, daß es niemand erfahren soll. Warum soll er dich erschlagen, daß alle Juden, so zu dir versammelt sind, zerstreut werden und die noch aus Juda übriggeblieben sind, umkommen?
\par 16 Aber Gedalja, der Sohn Ahikams, sprach zu Johanan, dem Sohn Kareahs: Du sollst das nicht tun; es ist nicht wahr, was du von Ismael sagst.

\chapter{41}

\par 1 Aber im siebenten Monat kam Ismael, der Sohn Nethanjas, des Sohnes Elisamas, aus königlichem Stamm, einer von den Obersten des Königs, und zehn Männer mit ihm zu Gedalja, dem Sohn Ahikams, gen Mizpa und sie aßen daselbst zu Mizpa miteinander.
\par 2 Und Ismael, der Sohn Nethanjas, macht sich auf samt den zehn Männern, die bei ihm waren und schlugen Gedalja, den Sohn Ahikams, des Sohnes Saphans, mit dem Schwert zu Tode, darum daß ihn der König zu Babel über das Land gesetzt hatte;
\par 3 dazu alle Juden, die bei Gedalja waren zu Mizpa, und die Chaldäer, die sie daselbst fanden, alle Kriegsleute, schlug Ismael.
\par 4 Des andern Tages, nachdem Gedalja erschlagen war und es noch niemand wußte,
\par 5 kamen achtzig Männer von Sichem, von Silo und von Samaria und hatten die Bärte abgeschoren und ihre Kleider zerrissen und sich zerritzt und trugen Speisopfer und Weihrauch mit sich, daß sie es brächten zum Hause des HERRN.
\par 6 Und Ismael, der Sohn Nethanjas, ging heraus von Mizpa ihnen entgegen, ging daher und weinte. Als er nun an sie kam, sprach er zu ihnen: Ihr sollt zu Gedalja, dem Sohn Ahikams, kommen.
\par 7 Da sie aber mitten in die Stadt kamen, ermordete sie Ismael, der Sohn Nethanjas, und die Männer, so bei ihm waren, und warf sie in den Brunnen.
\par 8 Aber es waren zehn Männer darunter, die sprachen zu Ismael: Töte uns nicht; wir haben Vorrat im Acker liegen von Weizen, Gerste, Öl und Honig. Also ließ er ab und tötete sie nicht mit den andern.
\par 9 Der Brunnen aber, darein Ismael die Leichname der Männer warf, welche er hatte erschlagen samt dem Gedalja, ist der, den der König Asa machen ließ wider Baesa, den König Israels; den füllte Ismael, der Sohn Nethanjas, mit den Erschlagenen.
\par 10 Und was übriges Volk war zu Mizpa, auch die Königstöchter, führte Ismael, der Sohn Nethanjas, gefangen weg samt allem übrigen Volk zu Mizpa, über welche Nebusaradan, der Hauptmann, hatte gesetzt Gedalja, den Sohn Ahikams, und zog hin und wollte hinüber zu den Kindern Ammon.
\par 11 Da aber Johanan, der Sohn Kareahs, erfuhr und alle Hauptleute des Heeres, die bei ihm waren, all das Übel, das Ismael, der Sohn Nethanjas, begangen hatte,
\par 12 nahmen sie zu sich alle Männer und zogen hin, wider Ismael, den Sohn Nethanjas, zu streiten; und trafen ihn an dem großen Wasser bei Gibeon.
\par 13 Da nun alles Volk, so bei Ismael war, sah den Johanan, den Sohn Kareahs, und alle die Hauptleute des Heeres, die bei ihm waren, wurden sie froh.
\par 14 Und das ganze Volk, das Ismael hatte von Mizpa weggeführt, wandte sich um und kehrte wiederum zu Johanan, dem Sohne Kareahs.
\par 15 Aber Ismael, der Sohn Nethanjas, entrann dem Johanan mit acht Männern, und zog zu den Kindern Ammon.
\par 16 Und Johanan, der Sohn Kareahs, samt allen Hauptleuten des Heeres, so bei ihm waren, nahmen all das übrige Volk, so sie wiedergebracht hatten von Ismael, dem Sohn Nethanjas, aus Mizpa zu sich (weil Gedalja, der Sohn Ahikams, erschlagen war), nämlich die Kriegsmänner, Weiber und die Kinder und Kämmerer, so sie von Gibeon hatten wiedergebracht;
\par 17 und zogen hin und kehrten ein zur Herberge Chimhams, die bei Bethlehem war, und wollten nach Ägypten ziehen vor den Chaldäern.
\par 18 Denn sie fürchteten sich vor ihnen, weil Ismael, der Sohn Nethanjas, Gedalja, den Sohn Ahikams, erschlagen hatte, den der König zu Babel über das Land gesetzt hatte.

\chapter{42}

\par 1 Da traten herzu alle Hauptleute des Heeres, Johanan, der Sohn Kareahs, Jesanja, der Sohn Hosajas, samt dem ganzen Volk, klein und groß,
\par 2 und sprachen zum Propheten Jeremia: Laß doch unser Gebet vor dir gelten und bitte für uns den HERRN, deinen Gott, für alle diese Übrigen (denn unser ist leider wenig geblieben von vielen, wie du uns selbst siehst mit deinen Augen),
\par 3 daß uns der HERR, dein Gott, wolle anzeigen, wohin wir ziehen und was wir tun sollen.
\par 4 Und der Prophet Jeremia sprach zu ihnen: Wohlan, ich will gehorchen; und siehe, ich will den HERRN, euren Gott, bitten, wie ihr gesagt habt; und alles, was euch der HERR antworten wird, das will ich euch anzeigen und will euch nichts verhalten.
\par 5 Und sie sprachen zu Jeremia: Der HERR sei ein gewisser und wahrhaftiger Zeuge zwischen uns, wo wir nicht tun werden alles, was dir der HERR, dein Gott, an uns befehlen wird.
\par 6 Es sei Gutes oder Böses, so wollen wir gehorchen der Stimme des HERRN, unsers Gottes, zu dem wir dich senden; auf daß es uns wohl gehe, so wir der Stimme des HERRN, unsers Gottes, gehorchen.
\par 7 Und nach zehn Tagen geschah des HERRN Wort zu Jeremia.
\par 8 Da rief er Johanan, den Sohn Kareahs, und alle Hauptleute des Heeres, die bei ihm waren, und alles Volk, klein und groß,
\par 9 und sprach zu ihnen: So spricht der HERR, der Gott Israels, zu dem ihr mich gesandt habt, daß ich euer Gebet vor ihn sollte bringen:
\par 10 Werdet ihr in diesem Lande bleiben, so will ich euch bauen und nicht zerbrechen; ich will euch pflanzen und nicht ausreuten; denn es hat mich schon gereut das Übel, das ich euch getan habe.
\par 11 Ihr sollt euch nicht fürchten vor dem König zu Babel, vor dem ihr euch fürchtet, spricht der HERR; ihr sollt euch vor ihm nicht fürchten, denn ich will bei euch sein, daß ich euch helfe und von seiner Hand errette.
\par 12 Ich will euch Barmherzigkeit erzeigen und mich über euch erbarmen und euch wieder in euer Land bringen.
\par 13 Werdet ihr aber sagen: Wir wollen nicht in diesem Lande bleiben, damit ihr ja nicht gehorcht der Stimme des HERRN, eures Gottes,
\par 14 sondern sagen: Nein, wir wollen nach Ägyptenland ziehen, daß wir keinen Krieg sehen noch der Posaune Schall hören und nicht Hunger Brots halben leiden müssen; daselbst wollen wir bleiben:
\par 15 nun so hört des HERRN Wort, ihr übrigen aus Juda! So spricht der HERR Zebaoth, der Gott Israels: Werdet ihr euer Angesicht richten, nach Ägyptenland zu ziehen, daß ihr daselbst bleiben wollt,
\par 16 so soll euch das Schwert, vor dem ihr euch fürchtet, in Ägyptenland treffen, und der Hunger, des ihr euch besorgt, soll stets hinter euch her sein in Ägypten, und sollt daselbst sterben.
\par 17 Denn sie seien, wer sie wollen, die ihr Angesicht richten, daß sie nach Ägypten ziehen, daselbst zu bleiben, die sollen sterben durchs Schwert, Hunger und Pestilenz, und soll keiner übrigbleiben noch entrinnen dem Übel, das ich über sie will kommen lassen.
\par 18 Denn so spricht der HERR Zebaoth, der Gott Israels: Gleichwie mein Zorn und Grimm über die Einwohner zu Jerusalem ausgeschüttet ist, so soll er auch über euch ausgeschüttet werden, wo ihr nach Ägypten zieht, daß ihr zum Fluch, zum Wunder, Schwur und Schande werdet und diese Stätte nicht mehr sehen sollt.
\par 19 Das Wort des HERRN gilt euch, ihr übrigen aus Juda, daß ihr nicht nach Ägypten zieht. Darum so wisset, daß ich euch heute bezeuge;
\par 20 ihr werdet sonst euer Leben verwahrlosen. Denn ihr habt mich gesandt zum HERRN, eurem Gott, und gesagt: Bitte den HERRN, unsern Gott, für uns; und alles, was der HERR, unser Gott, sagen wird, das zeige uns an, so wollen wir darnach tun.
\par 21 Das habe ich euch heute zu wissen getan; aber ihr wollt der Stimme des HERRN, eures Gottes, nicht gehorchen noch alle dem, das er mir befohlen hat.
\par 22 So sollt ihr nun wissen, daß ihr durch Schwert, Hunger und Pestilenz sterben müßt an dem Ort, dahin ihr gedenkt zu ziehen, daß ihr daselbst wohnen wollt.

\chapter{43}

\par 1 Da Jeremia alle Worte des HERRN, ihres Gottes, hatte ausgeredet zu allem Volk, wie ihm denn der HERR, ihr Gott, alle diese Worte an sie befohlen hatte,
\par 2 sprachen Asarja, der Sohn Hosajas, und Johanan, der Sohn Kareahs und alle frechen Männer zu Jeremia: Du lügst; der HERR, unser Gott, hat dich nicht zu uns gesandt noch gesagt: Ihr sollt nicht nach Ägypten ziehen, daselbst zu wohnen;
\par 3 sondern Baruch, der Sohn Nerias, beredet dich, uns zuwider, auf daß wir den Chaldäern übergeben werden, daß sie uns töten und gen Babel wegführen.
\par 4 Also gehorchten Johanan, der Sohn Kareahs, und alle Hauptleute des Heeres samt dem ganzen Volk der Stimme des HERRN nicht, daß sie im Lande Juda wären geblieben;
\par 5 sondern Johanan, der Sohn Kareahs, und alle Hauptleute des Heeres nahmen zu sich alle Übrigen aus Juda, so von allen Völkern, dahin sie geflohen, wiedergekommen waren, daß sie im Lande Juda wohnten,
\par 6 nämlich Männer, Weiber und Kinder, dazu die Königstöchter und alle Seelen, die Nebusaradan, der Hauptmann, bei Gedalja, dem Sohn Ahikams, des Sohnes Saphans, hatte gelassen, auch den Propheten Jeremia und Baruch, den Sohn Nerias,
\par 7 und zogen nach Ägyptenland, denn sie wollten der Stimme des HERRN nicht gehorchen, und kamen nach Thachpanhes.
\par 8 Und des HERRN Wort geschah zu Jeremia zu Thachpanhes und sprach:
\par 9 Nimm große Steine und verscharre sie im Ziegelofen, der vor der Tür am Hause Pharaos ist zu Thachpanhes, daß die Männer aus Juda zusehen;
\par 10 und sprich zu ihnen: So spricht der HERR Zebaoth, der Gott Israels: Siehe, ich will hinsenden und meinen Knecht Nebukadnezar, den König zu Babel, holen lassen und will seinen Stuhl oben auf diese Steine setzen, die ich verscharrt habe; und er soll sein Gezelt darüberschlagen.
\par 11 Und er soll kommen und Ägyptenland schlagen, und töten, wen es trifft, gefangen führen, wen es trifft, mit dem Schwert schlagen, wen es trifft.
\par 12 Und ich will die Häuser der Götter in Ägypten mit Feuer anstecken, daß er sie verbrenne und wegführe. Und er soll sich Ägyptenland anziehen, wie ein Hirt sein Kleid anzieht, und mit Frieden von dannen ziehen.
\par 13 Er soll die Bildsäulen zu Beth-Semes in Ägyptenland zerbrechen und die Götzentempel in Ägypten mit Feuer verbrennen.

\chapter{44}

\par 1 Dies ist das Wort, das zu Jeremia geschah an alle Juden, so in Ägyptenland wohnten, nämlich so zu Migdol, zu Thachpanhes, zu Noph und im Lande Pathros wohnten, und sprach:
\par 2 So spricht der HERR Zebaoth, der Gott Israels: Ihr habt gesehen all das Übel, das ich habe kommen lassen über Jerusalem und über alle Städte in Juda; und siehe, heutigestages sind sie wüst, und wohnt niemand darin;
\par 3 und das um ihrer Bosheit willen, die sie taten, daß sie mich erzürnten und hingingen und räucherten und dienten andern Göttern, welche weder sie noch ihr noch eure Väter kannten.
\par 4 Und ich sandte stets zu euch alle meine Knechte, die Propheten, und ließ euch sagen: Tut doch nicht solche Greuel, die ich hasse.
\par 5 Aber sie gehorchten nicht, neigten auch ihre Ohren nicht, daß sie von ihrer Bosheit sich bekehrt und andern Göttern nicht geräuchert hätten.
\par 6 Darum ging auch mein Zorn und Grimm an und entbrannte über die Städte Juda's und über die Gassen zu Jerusalem, daß sie zur Wüste und Öde geworden sind, wie es heutigestages steht.
\par 7 Nun, so spricht der HERR, der Gott Zebaoth, der Gott Israels: Warum tut ihr doch so großes Übel wider euer eigen Leben, damit unter euch ausgerottet werden Mann und Weib, Kind und Säugling aus Juda und nichts von euch übrigbleibe,
\par 8 und erzürnt mich so durch eurer Hände Werke und räuchert andern Göttern in Ägyptenland, dahin ihr gezogen seid, daselbst zu herbergen, auf daß ihr ausgerottet und zum Fluch und zur Schmach werdet unter allen Heiden auf Erden?
\par 9 Habt ihr vergessen das Unglück eurer Väter, das Unglück der Könige Juda's, das Unglück ihrer Weiber, dazu euer eigenes Unglück und eurer Weiber Unglück, das euch begegnet ist im Lande Juda und auf den Gassen zu Jerusalem?
\par 10 Noch sind sie bis auf diesen Tag nicht gedemütigt, fürchten sich auch nicht und wandeln nicht in meinem Gesetz und den Rechten, die ich euch und euren Vätern vorgestellt habe.
\par 11 Darum spricht der HERR Zebaoth, der Gott Israels, also: Siehe, ich will mein Angesicht wider euch richten zum Unglück, und ganz Juda soll ausgerottet werden.
\par 12 Und ich will die übrigen aus Juda nehmen, so ihr Angesicht gerichtet haben, nach Ägyptenland zu ziehen, daß sie daselbst herbergen; es soll ein Ende mit ihnen allen werden in Ägyptenland. Durchs Schwert sollen sie fallen, und durch Hunger umkommen, beide, klein und groß; sie sollen durch Schwert und Hunger sterben und sollen ein Schwur, Wunder, Fluch und Schmach werden.
\par 13 Ich will auch die Einwohner in Ägyptenland mit Schwert, Hunger und Pestilenz heimsuchen, gleichwie ich zu Jerusalem getan habe,
\par 14 daß aus den übrigen Juda's keiner soll entrinnen noch übrigbleiben, die doch darum hierher gekommen sind nach Ägyptenland zur Herberge, daß sie wiederum ins Land Juda möchten, dahin sie gerne wiederkommen wollten und wohnen; aber es soll keiner wieder dahin kommen, außer, welche von hinnen fliehen.
\par 15 Da antworteten dem Jeremia alle Männer, die da wohl wußten, daß ihre Weiber andern Göttern räucherten, und alle Weiber, so in großem Haufen dastanden, samt allem Volk, die in Ägyptenland wohnten und in Pathros, und sprachen:
\par 16 Nach dem Wort, das du im Namen des HERRN uns sagst, wollen wir dir nicht gehorchen;
\par 17 sondern wir wollen tun nach allem dem Wort, das aus unserem Munde geht, und wollen der Himmelskönigin räuchern und ihr Trankopfer opfern, wie wir und unsre Väter, unsre Könige und Fürsten getan haben in den Städten Juda's und auf den Gassen zu Jerusalem. Da hatten wir auch Brot genug und ging uns wohl und sahen kein Unglück.
\par 18 Seit der Zeit aber, daß wir haben abgelassen, der Himmelskönigin zu räuchern und Trankopfer zu opfern, haben wir allen Mangel gelitten und sind durch Schwert und Hunger umgekommen.
\par 19 Auch wenn wir der Himmelskönigin räuchern und opfern, das tun wir ja nicht ohne unserer Männer Willen, daß wir ihr Kuchen backen und Trankopfer opfern, auf daß sie sich um uns bekümmere.
\par 20 Da sprach Jeremia zum ganzen Volk, Männern und Weibern und allem Volk, die ihm so geantwortet hatten:
\par 21 Ich meine ja, der HERR habe gedacht an das Räuchern, so ihr in den Städten Juda's und auf den Gassen zu Jerusalem getrieben habt samt euren Vätern, Königen, Fürsten und allem Volk im Lande, und hat's zu Herzen genommen,
\par 22 daß er nicht mehr leiden konnte euren bösen Wandel und die Greuel, die ihr tatet; daher auch euer Land zur Wüste, zum Wunder und zum Fluch geworden ist, daß niemand darin wohnt, wie es heutigestages steht.
\par 23 Darum, daß ihr geräuchert habt und wider den HERRN gesündigt und der Stimme des HERRN nicht gehorchtet und in seinem Gesetz, seinen Rechten und Zeugnissen nicht gewandelt habt, darum ist auch euch solches Unglück widerfahren, wie es heutigestages steht.
\par 24 Und Jeremia sprach zu allem Volk und zu allen Weibern: Höret des HERRN Wort, alle ihr aus Juda, so in Ägyptenland sind.
\par 25 So spricht der HERR Zebaoth, der Gott Israels: Ihr und eure Weiber habt mit einem Munde geredet und mit euren Händen vollbracht, was ihr sagt: Wir wollen unser Gelübde halten, die wir gelobt haben der Himmelskönigin, daß wir ihr räuchern und Trankopfer opfern. Wohlan, ihr habt eure Gelübde erfüllt und eure Gelübde gehalten.
\par 26 So hört nun des HERRN Wort, ihr alle aus Juda, die ihr in Ägyptenland wohnt: Siehe, ich schwöre bei meinem großen Namen, spricht der HERR, daß mein Name nicht mehr soll durch irgend eines Menschen Mund aus Juda genannt werden in ganz Ägyptenland, der da sagt: "So wahr der HERR HERR lebt!"
\par 27 Siehe, ich will über sie wachen zum Unglück und zu keinem Guten, daß, wer aus Juda in Ägyptenland ist, soll durch Schwert und Hunger umkommen, bis es ein Ende mit ihnen habe.
\par 28 Welche aber dem Schwert entrinnen, die werden aus Ägyptenland ins Land Juda wiederkommen müssen als ein geringer Haufe. Und also werden dann alle die übrigen aus Juda, so nach Ägyptenland gezogen waren, daß sie sich daselbst herbergten, erfahren, wessen Wort wahr sei, meines oder ihres.
\par 29 Und zum Zeichen, spricht der HERR, daß ich euch an diesem Ort heimsuchen will, damit ihr wißt, daß mein Wort soll wahr werden über euch zum Unglück,
\par 30 so spricht der HERR also: Siehe, ich will Pharao Hophra, den König in Ägypten, übergeben in die Hände seiner Feinde und derer, die ihm nach dem Leben stehen, gleichwie ich Zedekia, den König Juda's, übergeben habe in die Hand Nebukadnezars, des Königs zu Babel, seines Feindes, und der ihm nach seinem Leben stand.

\chapter{45}

\par 1 Dies ist das Wort, so der Prophet Jeremia redete zu Baruch, dem Sohn Nerias, da er diese Reden in ein Buch schrieb aus dem Munde Jeremia's im vierten Jahr Jojakims, des Sohnes Josias, des Königs in Juda, und sprach:
\par 2 So spricht der HERR Zebaoth, der Gott Israels, von dir Baruch:
\par 3 Du sprichst: Weh mir, wie hat mir der HERR Jammer zu meinem Schmerz hinzugefügt! Ich seufze mich müde und finde keine Ruhe.
\par 4 Sage ihm also: So spricht der HERR: Siehe, was ich gebaut habe, das breche ich ab; und was ich gepflanzt habe, das reute ich aus nämlich dies mein ganzes Land.
\par 5 Und du begehrst dir große Dinge? Begehre es nicht! Denn siehe, ich will Unglück kommen lassen über alles Fleisch, spricht der HERR; aber deine Seele will ich dir zur Beute geben, an welchen Ort du ziehst.

\chapter{46}

\par 1 Dies ist das Wort des HERRN, das zu dem Propheten Jeremia geschehen ist wider alle Heiden.
\par 2 Wider Ägypten. Wider das Heer Pharao Nechos, des Königs in Ägypten, welches lag am Wasser Euphrat zu Karchemis, das der König zu Babel, Nebukadnezar, schlug im vierten Jahr Jojakims, des Sohnes Josias, des Königs in Juda:
\par 3 Rüstet Schild und Tartsche und ziehet in den Streit!
\par 4 Spannet Rosse an und lasset Reiter aufsitzen, setzt Helme auf und schärft die Spieße und ziehet den Panzer an!
\par 5 Wie kommt's aber, daß ich sehe, daß sie verzagt sind und die Flucht geben und ihre Helden erschlagen sind? Sie fliehen, daß sie sich auch nicht umsehen. Schrecken ist um und um, spricht der HERR.
\par 6 Der Schnelle kann nicht entfliehen noch der Starke entrinnen. Gegen Mitternacht am Wasser Euphrat sind sie gefallen und darniedergelegt.
\par 7 Wer ist der, so heraufzieht wie der Nil, und seine Wellen erheben sich wie Wasserwellen?
\par 8 Ägypten zieht herauf wie der Nil, und seine Wellen erheben sich wie Wasserwellen, und es spricht: Ich will hinaufziehen, das Land bedecken und die Stadt verderben samt denen, die darin wohnen.
\par 9 Wohlan, sitzt auf die Rosse, rennt mit den Wagen, laßt die Helden ausziehen, die Mohren, und aus Put, die den Schild führen, und die Schützen aus Lud!
\par 10 Denn dies ist der Tag des HERRN HERRN Zebaoth, ein Tag der Rache, daß er sich an seinen Feinden räche, da das Schwert fressen und von ihrem Blut voll und trunken werden wird. Denn sie müssen dem HERRN HERRN Zebaoth ein Schlachtopfer werden im Lande gegen Mitternacht am Wasser Euphrat.
\par 11 Gehe hinauf gen Gilead und hole Salbe, Jungfrau, Tochter Ägyptens! Aber es ist umsonst, daß du viel arzneiest; du wirst doch nicht heil!
\par 12 Deine Schande ist unter die Heiden erschollen, deines Heulens ist das Land voll; denn ein Held fällt über den andern und liegen beide miteinander darnieder.
\par 13 Dies ist das Wort des HERRN, das er zu dem Propheten Jeremia redete, da Nebukadnezar, der König zu Babel, daherzog, Ägyptenland zu schlagen;
\par 14 Verkündiget in Ägypten und saget's an zu Migdol, saget's an zu Noph und Thachpanhes und sprecht: Stelle dich zur Wehr! denn das Schwert wird fressen, was um dich her ist.
\par 15 Wie geht's zu, daß deine Gewaltigen zu Boden fallen und können nicht bestehen? Der HERR hat sie so gestürzt.
\par 16 Er macht, daß ihrer viel fallen, daß einer mit dem andern darniederliegt. Da sprachen sie: Wohlauf, laßt uns wieder zu unserm Volk ziehen, in unser Vaterland vor dem Schwert des Tyrannen!
\par 17 Daselbst schrie man ihnen nach: Pharao, der König Ägyptens, liegt: er hat sein Gezelt gelassen!
\par 18 So wahr als ich lebe, spricht der König, der HERR Zebaoth heißt: Jener wird daherziehen so hoch, wie der Berg Thabor unter den Bergen ist und wie der Karmel am Meer ist.
\par 19 Nimm dein Wandergerät, du Einwohnerin, Tochter Ägyptens; denn Noph wird wüst und verbrannt werden, daß niemand darin wohnen wird.
\par 20 Ägypten ist ein sehr schönes Kalb; aber es kommt von Mitternacht der Schlächter.
\par 21 Auch die, so darin um Sold dienen, sind wie gemästete Kälber; aber sie müssen sich dennoch wenden, flüchtig werden miteinander und werden nicht bestehen; denn der Tag ihres Unfalls wird über sie kommen, die Zeit ihrer Heimsuchung.
\par 22 Man hört sie davonschleichen wie eine Schlange; denn jene kommen mit Heereskraft und bringen Äxte über sie wie die Holzhauer.
\par 23 Die werden hauen also in ihrem Wald, spricht der HERR, daß es nicht zu zählen ist; denn ihrer sind mehr als Heuschrecken, die niemand zählen kann.
\par 24 Die Tochter Ägyptens steht mit Schanden; denn sie ist dem Volk von Mitternacht in die Hände gegeben.
\par 25 Der HERR Zebaoth, der Gott Israels, spricht: Siehe, ich will heimsuchen den Amon zu No und den Pharao und Ägypten samt seinen Göttern und Königen, ja, Pharao mit allen, die sich auf ihn verlassen,
\par 26 daß ich sie gebe in die Hände denen, die ihnen nach ihrem Leben stehen, und in die Hände Nebukadnezars, des Königs zu Babel, und seiner Knechte. Und darnach sollst du bewohnt werden wie vor alters, spricht der HERR.
\par 27 Aber du, mein Knecht Jakob, fürchte dich nicht, und du, Israel, verzage nicht! Denn siehe, ich will dir aus fernen Landen und deinem Samen aus dem Lande seines Gefängnisses helfen, daß Jakob soll wiederkommen und in Frieden sein und die Fülle haben, und niemand soll ihn schrecken.
\par 28 Darum fürchte dich nicht, du, Jakob, mein Knecht, spricht der HERR; denn ich bin bei dir. Mit allen Heiden, dahin ich dich verstoßen habe, will ich ein Ende machen; aber mit dir will ich nicht ein Ende machen, sondern ich will dich züchtigen mit Maßen, auf daß ich dich nicht ungestraft lasse.

\chapter{47}

\par 1 Dies ist das Wort des HERRN, das zum Propheten Jeremia geschah wider die Philister, ehe denn Pharao Gaza schlug.
\par 2 So spricht der HERR: Siehe, es kommen Wasser herauf von Mitternacht, die eine Flut machen werden und das Land und was darin ist, die Städte und die, so darin wohnen, wegreißen werden, daß die leute werden schreien und alle Einwohner im Lande heulen
\par 3 vor dem Getümmel ihrer starken Rosse, so dahertraben, und vor dem Rasseln ihrer Wagen und Poltern ihrer Räder; daß sich die Väter nicht werden umsehen nach den Kindern, so verzagt werden sie sein
\par 4 vor dem Tage, so da kommt, zu verstören alle Philister und auszureuten Tyrus und Sidon samt ihren andern Gehilfen. Denn der HERR wird die Philister, die das übrige sind aus der Insel Kaphthor, verstören.
\par 5 Gaza wird kahl werden, und Askalon samt den übrigen in ihren Gründen wird verderbt. Wie lange ritzest du dich?
\par 6 O du Schwert des HERRN, wann willst du doch aufhören? Fahre doch in deine Scheide und ruhe und sei still!
\par 7 Aber wie kannst du aufhören, weil der HERR dir Befehl getan hat wider die Anfurt am Meer bestellt?

\chapter{48}

\par 1 Wider Moab. So spricht der HERR Zebaoth, der Gott Israels: Weh der Stadt Nebo! denn sie ist zerstört und liegt elend; Kirjathaim ist gewonnen; die hohe Feste steht elend und ist zerrissen.
\par 2 Der Trotz Moabs ist aus, den sie an Hesbon hatten; denn man gedenkt Böses wider sie: "Kommt, wir wollen sie ausrotten, daß sie kein Volk mehr seien." Und du, Madmen, mußt auch verderbt werden; das Schwert wird hinter dich kommen.
\par 3 Man hört ein Geschrei zu Horonaim von Verstören und großem Jammer.
\par 4 Moab ist zerschlagen! man hört ihre Kleinen schreien;
\par 5 denn sie gehen mit Weinen den Weg hinauf gen Luhith, und die Feinde hören ein Jammergeschrei den Weg von Horonaim herab:
\par 6 "Hebt euch weg und errettet euer Leben!" Aber du wirst sein wie die Heide in der Wüste.
\par 7 Darum daß du dich auf deine Gebäude verläßt und auf deine Schätze, sollst du auch gewonnen werden; und Kamos muß hinaus gefangen wegziehen samt seinen Priestern und Fürsten.
\par 8 Denn der Verstörer wird über alle Städte kommen, daß nicht eine Stadt entrinnen wird. Es sollen beide, die Gründe verderbt und die Ebenen verstört werden; denn der HERR hat's gesagt.
\par 9 Gebt Moab Federn: er wird ausgehen, als flöge er; und seine Städte werden wüst liegen, daß niemand darin wohnen wird.
\par 10 Verflucht sei, der des HERRN Werk lässig tut; verflucht sei, der sein Schwert aufhält, daß es nicht Blut vergieße!
\par 11 Moab ist von seiner Jugend auf sicher gewesen und auf seinen Hefen stillgelegen und ist nie aus einem Faß ins andere gegossen und nie ins Gefängnis gezogen; darum ist sein Geschmack ihm geblieben und sein Geruch nicht verändert worden.
\par 12 Darum siehe, spricht der HERR, es kommt die Zeit, daß ich ihnen will Schröter schicken, die sie ausschroten sollen und ihre Fässer ausleeren und ihre Krüge zerschmettern.
\par 13 Und Moab soll über dem Kamos zu Schanden werden, gleichwie das Haus Israel über Beth-El zu Schanden geworden ist, darauf sie sich doch verließen.
\par 14 Wie dürft ihr sagen: Wir sind die Helden und die rechten Kriegsleute?
\par 15 so doch Moab muß verstört und ihre Städte erstiegen werden und ihre beste Mannschaft zur Schlachtbank herabgehen muß, spricht der König, welcher heißt der HERR Zebaoth.
\par 16 Denn der Unfall Moabs wird bald kommen, und ihr Unglück eilt sehr.
\par 17 Habt doch Mitleid mit ihnen alle, die ihr um sie her wohnt und ihren Namen kennt, und sprecht: "Wie ist die starke Rute und der herrliche Stab so zerbrochen!"
\par 18 Herab von der Herrlichkeit, du Einwohnerin, Tochter Dibon, und sitze in der Dürre! Denn der Verstörer Moabs wird zu dir hinaufkommen und deine Festen zerreißen.
\par 19 Tritt auf die Straße und schaue, du Einwohnerin Aroers; frage die, so da fliehen und entrinnen, und sprich: "Wie geht's?"
\par 20 Ach, Moab ist verwüstet und verderbt! Heult und schreit; sagt's am Arnon, daß Moab verstört sei!
\par 21 Die Strafe ist über das ebene Land gegangen, nämlich über Holon, Jahza, Mephaath,
\par 22 Dibon, Nebo, Beth-Diblathaim,
\par 23 Kirjathaim, Beth-Gamul, Beth-Meon,
\par 24 Karioth, Bozra und über alle Städte im Lande Moab, sie liegen fern oder nahe.
\par 25 Das Horn Moabs ist abgehauen, und sein Arm ist zerbrochen, spricht der HERR.
\par 26 Macht es trunken (denn es hat sich wider den HERRN erhoben), daß es speien und die Hände ringen müsse, auf daß es auch zum Gespött werde.
\par 27 Denn Israel hat dein Gespött sein müssen, als wäre es unter den Dieben gefunden; und weil du solches wider dasselbe redest, sollst du auch weg müssen.
\par 28 O ihr Einwohner in Moab, verlaßt die Städte und wohnt in den Felsen und tut wie die Tauben, so da nisten in den hohlen Löchern!
\par 29 Man hat immer gesagt von dem stolzen Moab, daß es sehr stolz sei, hoffärtig, hochmütig, trotzig und übermütig.
\par 30 Aber der HERR spricht: Ich kenne seinen Zorn wohl, daß er nicht soviel vermag und untersteht sich, mehr zu tun, denn sein Vermögen ist.
\par 31 Darum muß ich über Moab heulen und über das ganze Moab schreien und über die Leute zu Kir-Heres klagen.
\par 32 Mehr als über Jaser muß ich über dich, du Weinstock zu Sibma, weinen, dessen Reben über das Meer reichten und bis an das Meer Jaser kamen. Der Verstörer ist in deine Ernte und Weinlese gefallen;
\par 33 Freude und Wonne ist aus dem Felde weg und aus dem Lande Moab, und man wird keinen Wein mehr keltern; der Weintreter wird nicht mehr sein Lied singen
\par 34 von des Geschreies wegen zu Hesbon bis gen Eleale, welches bis gen Jahza erschallt, von Zoar an bis gen Horonaim, bis zum dritten Eglath; denn auch die Wasser Nimrims sollen versiegen.
\par 35 Und ich will, spricht der HERR, in Moab damit ein Ende machen, daß sie nicht mehr auf den Höhen opfern und ihren Göttern räuchern sollen.
\par 36 Darum seufzt mein Herz über Moab wie Flöten, und über die Leute zu Kir-Heres seufzt mein Herz wie Flöten; denn das Gut, das sie gesammelt, ist zu Grunde gegangen.
\par 37 Alle Köpfe werden kahl sein und alle Bärte abgeschoren, aller Hände zerritzt, und jedermann wird Säcke anziehen.
\par 38 Auf allen Dächern und Gassen, allenthalben in Moab, wird man Klagen; denn ich habe Moab zerbrochen wie ein unwertes Gefäß, spricht der HERR.
\par 39 O wie ist es verderbt, wie heulen sie! Wie schändlich hängen sie die Köpfe! Und Moab ist zum Spott und zum Schrecken geworden allen, so ringsumher wohnen.
\par 40 Denn so spricht der HERR: Siehe, er fliegt daher wie ein Adler und breitet seine Flügel aus über Moab.
\par 41 Karioth ist gewonnen, und die festen Städte sind eingenommen; und das Herz der Helden in Moab wird zu derselben Zeit sein wie einer Frau Herz in Kindesnöten.
\par 42 Denn Moab muß vertilgt werden, daß sie kein Volk mehr seien, darum daß es sich wider den HERR erhoben hat.
\par 43 Schrecken, Grube und Strick kommt über dich, du Einwohner in Moab, spricht der HERR.
\par 44 Wer dem Schrecken entflieht, der wird in die Grube fallen, und wer aus der Grube kommt, der wird im Strick gefangen werden; denn ich will über Moab kommen lassen ein Jahr ihrer Heimsuchung, spricht der HERR.
\par 45 Die aus der Schlacht entrinnen, werden Zuflucht suchen zu Hesbon; aber es wird ein Feuer aus Hesbon und eine Flamme aus Sihon gehen, welche die Örter in Moab und die kriegerischen Leute verzehren wird.
\par 46 Weh dir, Moab! verloren ist das Volk des Kamos; denn man hat deine Söhne und Töchter genommen und gefangen weggeführt.
\par 47 Aber in der letzten Zeit will ich das Gefängnis Moabs wenden, spricht der HERR. Das sei gesagt von der Strafe über Moab.

\chapter{49}

\par 1 Wider die Kinder Ammon spricht der HERR also: Hat denn Israel nicht Kinder, oder hat es keinen Erben? Warum besitzt denn Milkom das Land Gad, und sein Volk wohnt in jener Städten?
\par 2 Darum siehe, es kommt die Zeit, spricht der HERR, daß ich will ein Kriegsgeschrei erschallen lassen über Rabba der Kinder Ammon, daß sie soll auf einem Haufen wüst liegen und ihre Töchter mit Feuer angesteckt werden; aber Israel soll besitzen die, von denen sie besessen waren, spricht der HERR.
\par 3 Heule, o Hesbon! denn Ai ist verstört. Schreiet, ihr Töchter Rabbas, und ziehet Säcke an, klaget und lauft auf den Mauern herum! denn Milkom wird gefangen weggeführt samt seinen Priestern und Fürsten.
\par 4 Was trotzest du auf deine Auen? Deine Auen sind ersäuft, du ungehorsame Tochter, die du dich auf deine Schätze verlässest und sprichst in deinem Herzen: Wer darf sich an mich machen?
\par 5 Siehe, spricht der HERR HERR Zebaoth: Ich will Furcht über dich kommen lassen von allen, die um dich her wohnen, daß ein jeglicher seines Weges vor sich hinaus verstoßen werde und niemand sei, der die Flüchtigen sammle.
\par 6 Aber darnach will ich wieder wenden das Gefängnis der Kinder Ammon, spricht der HERR.
\par 7 Wider Edom. So spricht der HERR Zebaoth: Ist denn keine Weisheit mehr zu Theman? ist denn kein Rat mehr bei den Klugen? ist ihre Weisheit so leer geworden?
\par 8 Fliehet, wendet euch und verkriecht euch tief, ihr Bürger zu Dedan! denn ich lasse einen Unfall über Esau kommen, die Zeit seiner Heimsuchung.
\par 9 Es sollen Weinleser über dich kommen, die dir kein Nachlesen lassen; und die Diebe des Nachts sollen über dich kommen, die sollen ihnen genug verderben.
\par 10 Denn ich habe Esau entblößt und seine verborgenen Orte geöffnet, daß er sich nicht verstecken kann; sein Same, seine Brüder und seine Nachbarn sind verstört, daß ihrer keiner mehr da ist.
\par 11 Doch was Übrigbleibt von deinen Waisen, denen will ich das Leben gönnen, und deine Witwen werden auf mich hoffen.
\par 12 Denn so spricht der HERR: Siehe, die, so es nicht verschuldet hatten, den Kelch zu trinken, müssen trinken; und du solltest ungestraft bleiben? Du sollst nicht ungestraft bleiben, sondern du mußt auch trinken.
\par 13 Denn ich habe bei mir selbst geschworen, spricht der HERR, daß Bozra soll ein Wunder, Schmach, Wüste und Fluch werden und alle ihre Städte eine ewige Wüste.
\par 14 Ich habe gehört vom HERRN, daß eine Botschaft unter die Heiden gesandt sei: Sammelt euch und kommt her wider sie, macht euch auf zum Streit!
\par 15 Denn siehe, ich habe dich gering gemacht unter den Heiden und verachtet unter den Menschen.
\par 16 Dein Trotz und dein Hochmut hat dich betrogen, weil du in Felsenklüften wohnst und hohe Gebirge innehast. Wenn du denn gleich dein Nest so hoch machtest wie der Adler, dennoch will ich dich von dort herunterstürzen, spricht der HERR.
\par 17 Also soll Edom wüst werden, daß alle die, so vorübergehen, sich wundern und pfeifen werden über alle ihre Plage;
\par 18 gleichwie Sodom und Gomorra samt ihren Nachbarn umgekehrt ist, spricht der HERR, daß niemand daselbst wohnen noch kein Mensch darin hausen soll.
\par 19 Denn siehe, er kommt herauf wie ein Löwe vom stolzen Jordan her wider die festen Hürden; denn ich will sie daraus eilends wegtreiben, und den, der erwählt ist, darübersetzen. Denn wer ist mir gleich, wer will mich meistern, und wer ist der Hirte, der mir widerstehen kann?
\par 20 So hört nun den Ratschlag des HERRN, den er über Edom hat, und seine Gedanken, die er über die Einwohner in Theman hat. Was gilt's? ob nicht die Hirtenknaben sie fortschleifen werden und ihre Wohnung zerstören,
\par 21 daß die Erde beben wird, wenn's ineinander fällt, und ihr Geschrei wird man am Schilfmeer hören.
\par 22 Siehe, er fliegt herauf wie ein Adler und wird seine Flügel ausbreiten über Bozra. Zur selben Zeit wird das Herz der Helden in Edom sein wie das Herz einer Frau in Kindsnöten.
\par 23 Wider Damaskus. Hamath und Arpad stehen jämmerlich; sie sind verzagt, denn sie hören ein böses Geschrei; die am Meer wohnen, sind so erschrocken, daß sie nicht Ruhe haben können.
\par 24 Damaskus ist verzagt und gibt die Flucht; sie zappelt und ist in Ängsten und Schmerzen wie eine Frau in Kindsnöten.
\par 25 Wie? ist sie nun nicht verlassen, die berühmte und fröhliche Stadt?
\par 26 Darum werden ihre junge Mannschaft auf ihren Gassen darniederliegen und alle ihre Kriegsleute untergehen zur selben Zeit, spricht der HERR Zebaoth.
\par 27 Und ich will in den Mauern von Damaskus ein Feuer anzünden, daß es die Paläste Benhadads verzehren soll.
\par 28 Wider Kedar und die Königreiche Hazors, welche Nebukadnezar, der König zu Babel, schlug. So spricht der HERR: Wohlauf, zieht herauf gegen Kedar und verstört die gegen Morgen wohnen!
\par 29 Man wird ihnen ihre Hütten und Herden nehmen; ihr Gezelt, alle Geräte und Kamele werden sie wegführen; und man wird über sie rufen: Schrecken um und um!
\par 30 Fliehet, hebet euch eilends davon, verkriechet euch tief, ihr Einwohner in Hazor! spricht der HERR; denn Nebukadnezar, der König zu Babel, hat etwas im Sinn wider euch und meint euch.
\par 31 Wohlauf, ziehet herauf wider ein Volk, das genug hat und sicher wohnt, spricht der HERR; sie haben weder Tür noch Riegel und wohnen allein.
\par 32 Ihre Kamele sollen geraubt und die Menge ihres Viehs genommen werden; und ich will sie zerstreuen in alle Winde, alle, die das Haar rundherum abschneiden; und von allen Orten her will ich ihr Unglück über sie kommen lassen, spricht der HERR,
\par 33 daß Hazor soll eine Wohnung der Schakale und eine ewige Wüste werden, daß niemand daselbst wohne und kein Mensch darin hause.
\par 34 Dies ist das Wort des HERRN, welches geschah zu Jeremia, dem Propheten, wider Elam im Anfang des Königreichs Zedekias, des Königs in Juda, und sprach:
\par 35 So spricht der HERR Zebaoth: Siehe, ich will den Bogen Elams zerbrechen, ihre vornehmste Gewalt,
\par 36 und will die vier Winde aus den vier Enden des Himmels über sie kommen lassen und will sie in alle diese Winde zerstreuen, daß kein Volk sein soll, dahin nicht Vertriebene aus Elam kommen werden.
\par 37 Und ich will Elam verzagt machen vor ihren Feinden und denen, die ihnen nach ihrem Leben stehen, und Unglück über sie kommen lassen mit meinem grimmigen Zorn, spricht der HERR; und will das Schwert hinter ihnen her schicken, bis es sie aufreibe.
\par 38 Meinen Stuhl will ich in Elam aufrichten und will beide, den König und die Fürsten, daselbst umbringen, spricht der HERR.
\par 39 Aber in der letzten Zeit will ich das Gefängnis Elams wieder wenden, spricht der HERR.

\chapter{50}

\par 1 Dies ist das Wort, welches der HERR durch den Propheten Jeremia geredet hat wider Babel und das Land der Chaldäer:
\par 2 Verkündiget unter den Heiden und laßt erschallen, werfet ein Panier auf; laßt erschallen, und verberget's nicht und sprecht: Babel ist gewonnen, Bel steht mit Schanden, Merodach ist zerschmettert; ihre Götzen stehen mit Schanden, und ihre Götter sind zerschmettert!
\par 3 Denn es zieht von Mitternacht ein Volk herauf wider sie, welches wird ihr Land zur Wüste machen, daß niemand darin wohnen wird, sondern beide, Leute und Vieh, davonfliehen werden.
\par 4 In denselben Tagen und zur selben Zeit, spricht der HERR, werden kommen die Kinder Israel samt den Kindern Juda und weinend daherziehen und den HERRN, ihren Gott, suchen.
\par 5 Sie werden forschen nach dem Wege gen Zion, dahin sich kehren: Kommt, wir wollen uns zum HERRN fügen mit einem ewigen Bunde, des nimmermehr vergessen werden soll!
\par 6 Denn mein Volk ist wie eine verlorene Herde; ihre Hirten haben sie verführt und auf den Bergen in der Irre gehen lassen, daß sie von den Bergen auf die Hügel gegangen sind und ihre Hürden vergessen haben.
\par 7 Es fraßen sie alle, die sie antrafen; und ihre Feinde sprachen: Wir tun nicht unrecht! darum daß sie sich haben versündigt an dem HERRN in der Wohnung der Gerechtigkeit und an dem HERRN, der ihrer Väter Hoffnung ist.
\par 8 Fliehet aus Babel und ziehet aus der Chaldäer Lande und stellt euch als Böcke vor der Herde her.
\par 9 Denn siehe, ich will große Völker in Haufen aus dem Lande gegen Mitternacht erwecken und wider Babel heraufbringen, die sich wider sie sollen rüsten, welche sie sollen auch gewinnen; ihre Pfeile sind wie die eines guten Kriegers, der nicht fehlt.
\par 10 Und das Chaldäerland soll ein Raub werden, daß alle, die sie berauben, sollen genug davon haben, spricht der HERR;
\par 11 darum daß ihr euch des freut und rühmt, daß ihr mein Erbteil geplündert habt, und hüpft wie die jungen Kälber und wiehert wie die starken Gäule.
\par 12 Eure Mutter besteht mit großer Schande, und die euch geboren hat, ist zum Spott geworden; siehe, unter den Heiden ist sie die geringste, wüst, dürr und öde.
\par 13 Denn vor dem Zorn des HERRN muß sie unbewohnt und ganz wüst bleiben, daß alle, so bei Babel vorübergehen, werden sich verwundern und pfeifen über all ihr Plage.
\par 14 Rüstet euch wider Babel umher, alle Schützen, schießt in sie, spart die Pfeile nicht; denn sie hat wider den HERRN gesündigt.
\par 15 Jauchzt über sie um und um! Sie muß sich ergeben, ihr Grundfesten sind zerfallen, ihre Mauern sind abgebrochen; denn das ist des HERRN Rache. Rächt euch an ihr, tut ihr, wie sie getan hat.
\par 16 Rottet aus von Babel beide, den Säemann und den Schnitter in der Ernte, daß ein jeglicher vor dem Schwert des Tyrannen sich kehre zu seinem Volk und ein jeglicher fliehe in sein Land.
\par 17 Israel hat müssen sein eine zerstreute Herde, die die Löwen verscheucht haben. Am ersten fraß sie der König von Assyrien; darnach überwältigte sie Nebukadnezar, der König zu Babel.
\par 18 Darum spricht der HERR Zebaoth, der Gott Israels, also: Siehe, ich will den König zu Babel heimsuchen und sein Land, gleichwie ich den König von Assyrien heimgesucht habe.
\par 19 Israel aber will ich wieder Heim zu seiner Wohnung bringen, daß sie auf Karmel und Basan weiden und ihre Seele auf dem Gebirge Ephraim und Gilead gesättigt werden soll.
\par 20 Zur selben Zeit und in denselben Tagen wird man die Missetat Israels suchen, spricht der HERR, aber es wird keine da sein, und die Sünden Juda's, aber es wird keine gefunden werden; denn ich will sie vergeben denen, so ich übrigbleiben lasse.
\par 21 Zieh hinauf wider das Land, das alles verbittert hat; zieh hinauf wider die Einwohner der Heimsuchung; verheere und verbanne ihre Nachkommen, spricht der HERR, und tue alles, was ich dir befohlen habe!
\par 22 Es ist ein Kriegsgeschrei im Lande und großer Jammer.
\par 23 Wie geht's zu, daß der Hammer der ganzen Welt zerbrochen und zerschlagen ist? Wie geht's zu, daß Babel eine Wüste geworden ist unter allen Heiden?
\par 24 Ich habe dir nachgestellt, Babel; darum bist du auch gefangen, ehe du dich's versahst; du bist getroffen und ergriffen, denn du hast dem HERRN getrotzt.
\par 25 Der HERR hat seinen Schatz aufgetan und die Waffen seines Zorns hervorgebracht; denn der HERR HERR Zebaoth hat etwas auszurichten in der Chaldäer Lande.
\par 26 Kommt her wider sie, ihr vom Ende, öffnet ihre Kornhäuser, werft sie in einen Haufen und verbannt sie, daß ihr nichts übrigbleibe!
\par 27 Erwürgt alle ihre Rinder, führt sie hinab zu Schlachtbank! Weh ihnen! denn der Tag ist gekommen, die Zeit ihrer Heimsuchung.
\par 28 Man hört ein Geschrei der Flüchtigen und derer, so entronnen sind aus dem Lande Babel, auf daß sie verkündigen zu Zion die Rache des HERRN, unsers Gottes, die Rache seines Tempels.
\par 29 Ruft viel wider Babel, belagert sie um und um, alle Bogenschützen, und laßt keinen davonkommen! Vergeltet ihr, wie sie verdient hat; wie sie getan hat, so tut ihr wieder! denn sie hat stolz gehandelt wider den HERR, den Heiligen in Israel.
\par 30 Darum soll ihre junge Mannschaft fallen auf ihren Gassen, und alle Kriegsleute sollen untergehen zur selben Zeit, spricht der HERR.
\par 31 Siehe, du Stolzer, ich will an dich, spricht der HERR HERR Zebaoth; denn dein Tag ist gekommen, die Zeit deiner Heimsuchung.
\par 32 Da soll der Stolze stürzen und fallen, daß ihn niemand aufrichte; ich will seine Städte mit Feuer anstecken, das soll alles, was um ihn her ist, verzehren.
\par 33 So spricht der HERR Zebaoth: Siehe, die Kinder Israel samt den Kindern Juda müssen Gewalt und Unrecht leiden; alle, die sie gefangen weggeführt haben, halten sie und wollen sie nicht loslassen.
\par 34 Aber ihr Erlöser ist stark, der heißt HERR Zebaoth; der wird ihre Sache so ausführen, daß er das Land bebend und die Einwohner zu Babel zitternd mache.
\par 35 Schwert soll kommen, spricht der HERR, über die Chaldäer und über ihr Einwohner zu Babel und über ihre Fürsten und über ihre Weisen!
\par 36 Schwert soll kommen über ihre Weissager, daß sie zu Narren werden; Schwert soll kommen über ihre Starken, daß sie verzagen!
\par 37 Schwert soll kommen über ihre Rosse und Wagen und alles fremde Volk, so darin sind, daß sie zu Weibern werden! Schwert soll kommen über ihre Schätze, daß sie geplündert werden!
\par 38 Trockenheit soll kommen über ihre Wasser, daß sie versiegen! denn es ist ein Götzenland, und sie trotzen auf ihre schrecklichen Götzen.
\par 39 Darum sollen Wüstentiere und wilde Hunde darin wohnen und die jungen Strauße; und es soll nimmermehr bewohnt werden und niemand darin hausen für und für,
\par 40 gleichwie Gott Sodom und Gomorra samt ihren Nachbarn umgekehrt hat, spricht der HERR, daß niemand darin wohne noch ein Mensch darin hause.
\par 41 Siehe, es kommt ein Volk von Mitternacht her; viele Heiden und viele Könige werden vom Ende der Erde sich aufmachen.
\par 42 Die haben Bogen und Lanze; sie sind grausam und unbarmherzig; ihr Geschrei ist wie das Brausen des Meeres; sie reiten auf Rossen, gerüstet wie Kriegsmänner wider dich, du Tochter Babel.
\par 43 Wenn der König zu Babel ihr Gerücht hören wird, so werden ihm die Fäuste entsinken; ihm wird so angst und bange werden wie einer Frau in Kindsnöten.
\par 44 Siehe, er kommt herauf wie ein Löwe vom stolzen Jordan wider die festen Hürden; denn ich will sie daraus eilends wegtreiben, und den, der erwählt ist, darübersetzen. Denn wer ist mir gleich, wer will micht meistern, und wer ist der Hirte, der mir widerstehen kann?
\par 45 So hört nun den Ratschlag des HERRN, den er über Babel hat, und seine Gedanken, die er hat über die Einwohner im Land der Chaldäer! Was gilt's? ob nicht die Hirtenknaben sie fortschleifen werden und ihre Wohnung zerstören.
\par 46 Und die Erde wird beben von dem Geschrei, und es wird unter den Heiden erschallen, wenn Babel gewonnen wird.

\chapter{51}

\par 1 So spricht der HERR: Siehe, ich will einen scharfen Wind erwecken wider Babel und wider ihre Einwohner, die sich wider mich gesetzt haben.
\par 2 Ich will auch Worfler gen Babel schicken, die sie worfeln sollen und ihr Land ausfegen, die allenthalben um sie sein werden am Tage ihres Unglücks;
\par 3 denn ihre Schützen werden nicht schießen, und ihre Geharnischten werden sich nicht wehren können. So verschont nun ihre junge Mannschaft nicht, verbannet all ihr Heer,
\par 4 daß die Erschlagenen daliegen im Lande der Chaldäer und die Erstochenen auf ihren Gassen!
\par 5 Denn Israel und Juda sollen nicht Witwen von ihrem Gott, dem HERRN Zebaoth, gelassen werden. Denn jener Land hat sich hoch verschuldet am Heiligen in Israel.
\par 6 Fliehet aus Babel, damit ein jeglicher seine Seele errette, daß ihr nicht untergeht in ihrer Missetat! Denn dies ist die Zeit der Rache des HERRN, der ein Vergelter ist und will ihnen bezahlen.
\par 7 Ein goldener Kelch, der alle Welt trunken gemacht hat, war Babel in der Hand des HERRN; alle Heiden haben von ihrem Wein getrunken, darum sind die Heiden so toll geworden.
\par 8 Wie plötzlich ist Babel gefallen und zerschmettert! Heulet über sie, nehmet auch Salbe zu ihren Wunden, ob sie vielleicht möchte heil werden!
\par 9 Wir heilen Babel; aber sie will nicht heil werden. So laßt sie fahren und laßt uns ein jeglicher in sein Land ziehen! Denn ihre Strafe reicht bis an den Himmel und langt hinauf bis an die Wolken.
\par 10 Der HERR hat unsre Gerechtigkeit hervorgebracht; kommt, laßt uns zu Zion erzählen die Werke des HERRN, unsers Gottes!
\par 11 Ja, schärft nun die Pfeile wohl und rüstet die Schilde! Der HERR hat den Mut der Könige in Medien erweckt; denn seine Gedanken stehen wider Babel, daß er sie verderbe. Denn dies ist die Rache des HERRN, die Rache seines Tempels.
\par 12 Ja, steckt nun Panier auf die Mauern zu Babel, nehmt die Wache ein, setzt Wächter, bestellt die Hut! denn der HERR gedenkt etwas und wird auch tun, was er wider die Einwohner zu Babel geredet hat.
\par 13 Die du an großen Wassern wohnst und große Schätze hast, dein Ende ist gekommen, und dein Geiz ist aus!
\par 14 Der HERR Zebaoth hat bei seiner Seele geschworen: Ich will dich mit Menschen füllen, als wären's Käfer; die sollen dir ein Liedlein singen!
\par 15 Er hat die Erde durch seine Kraft gemacht und den Weltkreis durch seine Weisheit bereitet und den Himmel ausgebreitet durch seinen Verstand.
\par 16 Wenn er donnert, so ist da Wasser die Menge unter dem Himmel; er zieht die Nebel auf vom Ende der Erde; er macht die Blitze im Regen und läßt den Wind kommen aus seinen Vorratskammern.
\par 17 Alle Menschen sind Narren mit ihrer Kunst, und die Goldschmiede bestehen mit Schanden mit ihren Bildern; denn ihre Götzen sind Trügerei und haben kein Leben.
\par 18 Es ist eitel Nichts und verführerisches Werk; sie müssen umkommen, wenn sie heimgesucht werden.
\par 19 Aber also ist der nicht, der Jakobs Schatz ist; sondern der alle Dinge schafft, der ist's, und Israel ist sein Erbteil. Er heißt HERR Zebaoth.
\par 20 Du bist mein Hammer, meine Kriegswaffe; durch dich zerschmettere ich die Heiden und zerstöre die Königreiche;
\par 21 durch dich zerschmettere ich Rosse und Reiter und zerschmettere Wagen und Fuhrmänner;
\par 22 durch dich zerschmettere ich Männer und Weiber und zerschmettere Alte und Junge und zerschmettere Jünglinge und Jungfrauen;
\par 23 durch dich zerschmettere ich Hirten und Herden und zerschmettere Bauern und Joche und zerschmettere Fürsten und Herren.
\par 24 Und ich will Babel und allen Einwohnern in Chaldäa vergelten alle ihre Bosheit, die sie an Zion begangen haben, vor euren Augen, spricht der HERR.
\par 25 Siehe, ich will an dich, du schädlicher Berg, der du alle Welt verderbest, spricht der HERR; ich will meine Hand über dich strecken und dich von den Felsen herabwälzen und will einen verbrannten Berg aus dir machen,
\par 26 daß man weder Eckstein noch Grundstein aus dir nehmen könne, sondern eine ewige Wüste sollst du sein, spricht der HERR.
\par 27 Werfet Panier auf im Lande, blaset die Posaune unter den Heiden, heiliget die Heiden wider sie; rufet wider sie die Königreiche Ararat, Minni und Askenas; bestellt Hauptleute wider sie; bringt Rosse herauf wie flatternde Käfer!
\par 28 Heiligt die Heiden wider sie, die Könige aus Medien samt allen ihren Fürsten und Herren und das ganze Land ihrer Herrschaft,
\par 29 daß das Land erbebe und erschrecke; denn die Gedanken des HERRN wollen erfüllt werden wider Babel, daß er das Land Babel zur Wüste mache, darin niemand wohne.
\par 30 Die Helden zu Babel werden nicht zu Felde ziehen, sondern müssen in der Festung bleiben, ihre Stärke ist aus, sie sind Weiber geworden; ihre Wohnungen sind angesteckt und ihre Riegel zerbrochen.
\par 31 Es läuft hier einer und da einer dem andern entgegen, und eine Botschaft begegnet der andern, dem König zu Babel anzusagen, daß seine Stadt gewonnen sei bis ans Ende
\par 32 und die Furten eingenommen und die Seen ausgebrannt sind und die Kriegsleute seien blöde geworden.
\par 33 Denn also spricht der HERR Zebaoth, der Gott Israels: "Die Tochter Babel ist wie eine Tenne, wenn man darauf drischt; es wird ihre Ernte gar bald kommen."
\par 34 Nebukadnezar, der König zu Babel, hat mich gefressen und umgebracht; er hat aus mir ein leeres Gefäß gemacht; er hat mich verschlungen wie ein Drache; er hat seinen Bauch gefüllt mit meinem Köstlichsten; er hat mich verstoßen.
\par 35 Nun aber komme über Babel der Frevel, an mir begangen und an meinem Fleische, spricht die Einwohnerin zu Zion, und mein Blut über die Einwohner in Chaldäa, spricht Jerusalem.
\par 36 Darum spricht der HERR also: Siehe, ich will dir deine Sache ausführen und dich rächen; ich will ihr Meer austrocknen und ihre Brunnen versiegen lassen.
\par 37 Und Babel soll zum Steinhaufen und zur Wohnung der Schakale werden, zum Wunder und zum Anpfeifen, daß niemand darin wohne.
\par 38 Sie sollen miteinander brüllen wie die Löwen und schreien wie die jungen Löwen.
\par 39 Ich will sie mit ihrem Trinken in die Hitze setzen und will sie trunken machen, daß sie fröhlich werden und einen ewigen Schlaf schlafen, von dem sie nimmermehr aufwachen sollen, spricht der HERR.
\par 40 Ich will sie herunterführen wie Lämmer zur Schlachtbank, wie die Widder mit den Böcken.
\par 41 Wie ist Sesach so gewonnen und die Berühmte in aller Welt so eingenommen! Wie ist Babel so zum Wunder geworden unter den Heiden!
\par 42 Es ist ein Meer über Babel gegangen, und es ist mit seiner Wellen Menge bedeckt.
\par 43 Ihre Städte sind zur Wüste und zu einem dürren, öden Lande geworden, zu einem Lande, darin niemand wohnt und darin kein Mensch wandelt.
\par 44 Denn ich habe den Bel zu Babel heimgesucht und habe aus seinem Rachen gerissen, was er verschlungen hatte; und die Heiden sollen nicht mehr zu ihm laufen; denn es sind auch die Mauern zu Babel zerfallen.
\par 45 Ziehet heraus, mein Volk, und errette ein jeglicher seine Seele vor dem grimmigen Zorn des HERRN!
\par 46 Euer Herz möchte sonst weich werden und verzagen vor dem Geschrei, das man im Lande hören wird; denn es wird ein Geschrei übers Jahr gehen und darnach im andern Jahr auch ein Geschrei über Gewalt im Lande und wird ein Fürst wider den andern sein.
\par 47 Darum siehe, es kommt die Zeit, daß ich die Götzen zu Babel heimsuchen will und ihr ganzes Land zu Schanden werden soll und ihre Erschlagenen darin liegen werden.
\par 48 Himmel und Erde und alles was darinnen ist, werden jauchzen über Babel, daß ihre Verstörer von Mitternacht gekommen sind, spricht der HERR.
\par 49 Und wie Babel in Israel die Erschlagenen gefällt hat, also sollen zu Babel die Erschlagenen fallen im ganzen Lande.
\par 50 So ziehet nun hin, die ihr dem Schwert entronnen seid, und säumet euch nicht! Gedenket des HERRN im fernen Lande und lasset euch Jerusalem im Herzen sein!
\par 51 Wir waren zu Schanden geworden, da wir die Schmach hören mußten, und die Scham unser Angesicht bedeckte, da die Fremden über das Heiligtum des Hauses des HERRN kamen.
\par 52 Darum siehe, die Zeit kommt, spricht der HERR, daß ich ihre Götzen heimsuchen will, und im ganzen Lande sollen die tödlich Verwundeten seufzen.
\par 53 Und wenn Babel gen Himmel stiege und ihre Macht in der Höhe festmachte, so sollen doch Verstörer von mir über sie kommen, spricht der HERR.
\par 54 Man hört ein Geschrei zu Babel und einen großen Jammer in der Chaldäer Lande;
\par 55 denn der HERR verstört Babel und verderbt sie mit großem Getümmel; ihre Wellen brausen wie die großen Wasser, es erschallt ihr lautes Toben.
\par 56 Denn es ist über Babel der Verstörer gekommen, ihre Helden werden gefangen, ihre Bogen zerbrochen; denn der Gott der Rache, der HERR, bezahlt ihr.
\par 57 Ich will ihre Fürsten, Weisen, Herren und Hauptleute und Krieger trunken machen, daß sie einen ewigen Schlaf sollen schlafen, davon sie nimmermehr aufwachen, spricht der König, der da heißt HERR Zebaoth.
\par 58 So spricht der HERR Zebaoth: Die Mauern der großen Babel sollen untergraben und ihre hohen Tore mit Feuer angesteckt werden, daß der Heiden Arbeit verloren sei, und daß verbrannt werde, was die Völker mit Mühe erbaut haben.
\par 59 Dies ist das Wort, das der Prophet Jeremia befahl Seraja dem Sohn Nerias, des Sohnes Maasejas, da er zog mit Zedekia, dem König in Juda, gen Babel im vierten Jahr seines Königreichs. Und Seraja war der Marschall für die Reise.
\par 60 Und Jeremia schrieb all das Unglück, so über Babel kommen sollte, in ein Buch, nämlich alle diese Worte, die wider Babel geschrieben sind.
\par 61 Und Jeremia sprach zu Seraja: Wenn du gen Babel kommst, so schaue zu und lies alle diese Worte
\par 62 und sprich: HERR, du hast geredet wider diese Stätte, daß du sie willst ausrotten, daß niemand darin wohne, weder Mensch noch Vieh, sondern daß sie ewiglich wüst sei.
\par 63 Und wenn du das Buch hast ausgelesen, so binde einen Stein daran und wirf es in den Euphrat
\par 64 und sprich: also soll Babel versenkt werden und nicht wieder aufkommen von dem Unglück, das ich über sie bringen will, sondern vergehen. So weit hat Jeremia geredet.

\chapter{52}

\par 1 Zedekia war einundzwanzig Jahre alt, da er König ward und regierte elf Jahre zu Jerusalem. Seine Mutter hieß Hamutal, eine Tochter Jeremia's zu Libna.
\par 2 Und er tat was dem HERRN übel gefiel, gleich wie Jojakim getan hatte.
\par 3 Denn es ging des HERRN Zorn über Jerusalem und Juda, bis er sie von seinem Angesicht verwarf. Und Zedekia fiel ab vom König zu Babel.
\par 4 Aber im neunten Jahr seines Königreichs, am zehnten Tage des zehnten Monats, kam Nebukadnezar, der König zu Babel, samt all seinem Heer wider Jerusalem, und sie belagerten es und machten Bollwerke ringsumher.
\par 5 Und blieb also die Stadt belagert bis ins elfte Jahr des Königs Zedekia.
\par 6 Aber am neunten Tage des vierten Monats nahm der Hunger überhand in der Stadt, und hatte das Volk vom Lande nichts mehr zu essen.
\par 7 Da brach man in die Stadt; und alle Kriegsleute gaben die Flucht und zogen zur Stadt hinaus bei der Nacht auf dem Wege durch das Tor zwischen den zwei Mauern, der zum Garten des Königs geht. Aber die Chaldäer lagen um die Stadt her.
\par 8 Und da diese zogen des Weges zum blachen Feld, jagte der Chaldäer Heer dem König nach und ergriffen Zedekia in dem Felde bei Jericho; da zerstreute sich all sein Heer von ihm.
\par 9 Und sie fingen den König und brachten ihn hinauf zum König zu Babel gen Ribla, das im Lande Hamath liegt, der sprach ein Urteil über ihn.
\par 10 Allda ließ der König zu Babel die Söhne Zedekias vor seinen Augen erwürgen und erwürgte alle Fürsten Juda's zu Ribla.
\par 11 Aber Zedekia ließ er die Augen ausstechen und ließ ihn mit zwei Ketten binden, und führte ihn also der König zu Babel gen Babel und legte ihn ins Gefängnis, bis daß er starb.
\par 12 Am zehnten Tage des fünften Monats, welches ist das neunzehnte Jahr Nebukadnezars, des Königs zu Babel, kam Nebusaradan, der Hauptmann der Trabanten, der stets um den König zu Babel war gen Jerusalem
\par 13 und verbrannte des HERRN Haus und des Königs Haus und alle Häuser zu Jerusalem; alle großen Häuser verbrannte er mit Feuer.
\par 14 Und das ganze Heer der Chaldäer, so bei dem Hauptmann war, riß um alle Mauern zu Jerusalem ringsumher.
\par 15 Aber das arme Volk und andere Volk so noch übrig war in der Stadt, und die zum König zu Babel fielen und das übrige Handwerksvolk führte Nebusaradan, der Hauptmann, gefangen weg.
\par 16 Und vom armen Volk auf dem Lande ließ Nebusaradan, der Hauptmann, bleiben Weingärtner und Ackerleute.
\par 17 Aber die ehernen Säulen am Hause des HERR und das Gestühl und das eherne Meer am Hause des HERRN zerbrachen die Chaldäer und führten all das Erz davon gen Babel.
\par 18 Und die Kessel, Schaufeln, Messer, Becken, Kellen und alle ehernen Gefäße, die man im Gottesdienst pflegte zu brauchen, nahmen sie weg.
\par 19 Dazu nahm der Hauptmann, was golden und silbern war an Bechern, Räuchtöpfen, Becken, Kesseln, Leuchtern, Löffeln und Schalen.
\par 20 Die zwei Säulen, das Meer, die Zwölf ehernen Rinder darunter und die Gestühle, welche der König Salomo hatte lassen machen zum Hause des HERRN, alles dieses Gerätes aus Erz war unermeßlich viel.
\par 21 Der zwei Säulen aber war eine jegliche achtzehn Ellen hoch, und eine Schnur, zwölf Ellen lang, reichte um sie her, und war eine jegliche vier Finger dick und inwendig hohl;
\par 22 und stand auf jeglicher ein eherner Knauf, fünf Ellen hoch, und ein Gitterwerk und Granatäpfel waren an jeglichem Knauf ringsumher, alles ehern; und war eine Säule wie die andere, die Granatäpfel auch.
\par 23 Es waren der Granatäpfel sechsundneunzig daran, und aller Granatäpfel waren hundert an einem Gitterwerk rings umher.
\par 24 Und der Hauptmann nahm den obersten Priester Seraja und den Priester Zephanja, den nächsten nach ihm, und die drei Torhüter
\par 25 und einen Kämmerer aus der Stadt, welcher über die Kriegsleute gesetzt war, und sieben Männer, welche um den König sein mußten, die in der Stadt gefunden wurden, dazu den Schreiber des Feldhauptmanns, der das Volk im Lande zum Heer aufbot, dazu sechzig Mann Landvolks, so in der Stadt gefunden wurden:
\par 26 diese nahm Nebusaradan, der Hauptmann, und brachte sie dem König zu Babel gen Ribla.
\par 27 Und der König zu Babel schlug sie tot zu Ribla, das im Lande Hamath liegt. Also ward Juda aus seinem Lande weggeführt.
\par 28 Dies ist das Volk, welches Nebukadnezar weggeführt hat: im siebenten Jahr dreitausend und dreiundzwanzig Juden;
\par 29 Im achtzehnten Jahr aber des Nebukadnezars achthundert und zweiunddreißig Seelen aus Jerusalem;
\par 30 und im dreiundzwanzigsten Jahr des Nebukadnezars führte Nebusaradan, der Hauptmann, siebenhundert und fünfundvierzig Seelen weg aus Juda. Alle Seelen sind viertausend und sechshundert.
\par 31 Aber im siebenunddreißigsten Jahr, nachdem Jojachin, der König zu Juda, weggeführt war, am fünfundzwanzigsten Tage des zwölften Monats, erhob Evil-Merodach, der König zu Babel, im Jahr, da er König ward, das Haupt Jojachins, des Königs in Juda, und ließ ihn aus dem Gefängnis
\par 32 und redete freundlich mit ihm und setzte seinen Stuhl über der Könige Stühle, die bei ihm zu Babel waren,
\par 33 und wandelte ihm seines Gefängnisses Kleider, daß er vor ihm aß stets sein Leben lang.
\par 34 Und ihm ward stets sein Unterhalt vom König zu Babel gegeben, wie es ihm verordnet war, sein ganzes Leben lang bis an sein Ende.

\end{document}