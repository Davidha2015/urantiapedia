\begin{document}

\title{Jeremiah}



\chapter{1}

\par 1 De woorden van Jeremia, den zoon van Hilkia, uit de priesteren, die te Anathoth waren, in het land van Benjamin;
\par 2 Tot welken het woord des HEEREN geschiedde, in de dagen van Josia, zoon van Amon, koning van Juda, in het dertiende jaar zijner regering.
\par 3 Ook geschiedde het tot hem in de dagen van Jojakim, zoon van Josia, koning van Juda, totdat voleind werd het elfde jaar van Zedekia, zoon van Josia, koning van Juda; totdat Jeruzalem gevankelijk werd weggevoerd in de vijfde maand.
\par 4 Het woord des HEEREN dan geschiedde tot mij, zeggende:
\par 5 Eer Ik u in moeders buik formeerde, heb Ik u gekend, en eer gij uit de baarmoeder voortkwaamt, heb Ik u geheiligd; Ik heb u den volken tot een profeet gesteld.
\par 6 Toen zeide ik: Ach, Heere HEERE! zie, ik kan niet spreken, want ik ben jong.
\par 7 Maar de HEERE zeide tot mij: Zeg niet: Ik ben jong; want overal, waarhenen Ik u zenden zal, zult gij gaan, en alles, wat Ik u gebieden zal, zult gij spreken.
\par 8 Vrees niet voor hun aangezicht, want Ik ben met u, om u te redden, spreekt de HEERE.
\par 9 En de HEERE stak Zijn hand uit, en roerde mijn mond aan; en de HEERE zeide tot mij: Zie, Ik geef Mijn woorden in uw mond.
\par 10 Zie, Ik stel u te dezen dage over de volken en over de koninkrijken, om uit te rukken, en af te breken, en te verderven, en te verstoren; ook om te bouwen en te planten.
\par 11 Wijders geschiedde des HEEREN woord tot mij, zeggende: Wat ziet gij, Jeremia? En ik zeide: Ik zie een amandelroede.
\par 12 En de HEERE zeide tot mij: Gij hebt wel gezien; want Ik zal wakker zijn over Mijn woord, om dat te doen.
\par 13 En des HEEREN woord geschiedde ten tweeden male tot mij, zeggende: Wat ziet gij? En ik zeide: Ik zie een ziedenden pot, welks voorste deel tegen het noorden is.
\par 14 En de HEERE zeide tot mij: Van het noorden zal zich dit kwaad opdoen over alle inwoners des lands.
\par 15 Want zie, Ik roep alle geslachten der koninkrijken van het noorden, spreekt de HEERE; en zij zullen komen, en zetten een iegelijk zijn troon voor de deur der poorten van Jeruzalem, en tegen al haar muren rondom, en tegen alle steden van Juda.
\par 16 En Ik zal Mijn oordelen tegen hen uitspreken over al hun boosheid; dat zij Mij verlaten hebben, en anderen goden gerookt, en zich gebogen hebben voor de werken hunner handen.
\par 17 Gij dan, gord uw lendenen, en maakt u op, en spreek tot hen alles, wat Ik u gebieden zal; wees niet verslagen voor hun aangezicht, opdat Ik u voor hun aangezicht niet versla.
\par 18 Want zie, Ik stel u heden tot een vaste stad, en tot een ijzeren pilaar, en tot koperen muren tegen het ganse land; tegen de koningen van Juda, tegen haar vorsten, tegen haar priesteren, en tegen het volk van het land.
\par 19 En zij zullen tegen u strijden, maar tegen u niet vermogen; want Ik ben met u, spreekt de HEERE, om u uit te helpen.

\chapter{2}

\par 1 En des HEEREN woord geschiedde tot mij, zeggende:
\par 2 Ga en roep voor de oren van Jeruzalem, zeggende: Zo zegt de HEERE: Ik gedenk der weldadigheid uwer jeugd, der liefde uwer ondertrouw, toen gij Mij nawandeldet in de woestijn, in onbezaaid land.
\par 3 Israel was den HEERE een heiligheid, de eerstelingen Zijner inkomste; allen, die hem opaten, werden voor schuldig gehouden; kwaad kwam hun over, spreekt de HEERE.
\par 4 Hoort des HEEREN woord, gij huis van Jakob, en alle geslachten van het huis Israels!
\par 5 Zo zegt de HEERE: Wat voor onrecht hebben uw vaders aan Mij gevonden, dat zij verre van Mij geweken zijn, en hebben de ijdelheid nagewandeld, en zij zijn ijdel geworden?
\par 6 En zeiden niet: Waar is de HEERE, Die ons opvoerde uit Egypteland, Die ons leidde in de woestijn, in een land van wildernissen en kuilen, in een land van dorheid en schaduw des doods, in een land, waar niemand doorging, en waar geen mens woonde?
\par 7 En Ik bracht u in een vruchtbaar land, om de vrucht van hetzelve en het goede er van te eten; maar toen gij daarin kwaamt, verontreinigdet gij Mijn land, en steldet Mijn erfenis tot een gruwel.
\par 8 De priesters zeiden niet: Waar is de HEERE? en die de wet handelden, kenden Mij niet; en de herders overtraden tegen Mij; en de profeten profeteerden door Baal, en wandelden naar dingen, die geen nut doen.
\par 9 Daarom zal Ik nog met ulieden twisten, spreekt de HEERE; ja, met uw kindskinderen zal Ik twisten.
\par 10 Want, gaat over in de eilanden der Chitteers, en ziet toe, en zendt naar Kedar, en merkt er wel op; en ziet, of diesgelijks geschied zij?
\par 11 Heeft ook een volk de goden veranderd, hoewel dezelve geen goden zijn? Nochtans heeft Mijn volk zijn Eer veranderd in hetgeen geen nut doet.
\par 12 Ontzet u hierover, gij hemelen, en zijt verschrikt, wordt zeer woest, spreekt de HEERE.
\par 13 Want Mijn volk heeft twee boosheden gedaan; Mij, den Springader des levenden waters, hebben zij verlaten, om zichzelven bakken uit te houwen, gebroken bakken, die geen water houden.
\par 14 Is dan Israel een knecht, of is hij een ingeborene des huizes? Waarom is hij dan ten roof geworden?
\par 15 De jonge leeuwen hebben over hem gebruld, zij hebben hun stem verheven; en zij hebben zijn land gezet in verwoesting; zijn steden zijn verbrand, dat er niemand in woont.
\par 16 Ook hebben u de kinderen van Nof en Tachpanhes den schedel afgeweid.
\par 17 Doet gij dit niet zelven, doordien gij den HEERE, uw God, verlaat, ten tijde als Hij u op den weg leidt?
\par 18 En nu, wat hebt gij te doen met den weg van Egypte, om de wateren van Sihor te drinken? En wat hebt gij te doen met den weg van Assur, om de wateren der rivier te drinken?
\par 19 Uw boosheid zal u kastijden, en uw afkeringen zullen u straffen; weet dan en ziet, dat het kwaad en bitter is, dat gij den HEERE, uw God, verlaat, en Mijn vreze niet bij u is, spreekt de Heere, de HEERE der heirscharen.
\par 20 Als Ik van ouds uw juk verbroken, en uw banden verscheurd had, zo zeidet gij: Ik zal niet dienen; maar op allen hogen heuvel en onder allen groenen boom loopt gij om, hoererende.
\par 21 Ik had u toch geplant, een edelen wijnstok, een geheel getrouw zaad; hoe zijt gij Mij dan veranderd in verbasterde ranken van een vreemden wijnstok?
\par 22 Want, al wiest gij u met salpeter, en naamt u veel zeep, zo is toch uw ongerechtigheid voor Mijn aangezicht getekend, spreekt de Heere HEERE.
\par 23 Hoe zegt gij: Ik ben niet verontreinigd, ik heb de Baals niet nagewandeld? Zie uw weg in het dal, ken, wat gij gedaan hebt, gij lichte, snelle kemelin, die haar wegen verdraait!
\par 24 Zij is een woudezelin, gewend in de woestijn, naar den lust harer ziel schept zij den wind, wie zou haar ontmoeting afkeren? Allen, die haar zoeken, zullen niet moede worden, in haar maand zullen zij haar vinden.
\par 25 Bedwing uw voet van ontschoeiing, en uw keel van dorst; maar gij zegt: Het is buiten hoop; neen, want ik heb de vreemden lief, en die zal ik nawandelen!
\par 26 Gelijk een dief beschaamd wordt, wanneer hij gevonden wordt, alzo zijn die van het huis Israels beschaamd; zij, hun koningen, hun vorsten, en hun priesters, en hun profeten;
\par 27 Die tot een hout zeggen: Gij zijt mijn vader; en tot een steen: Gij hebt mij gegenereerd; want zij keren Mij den nek toe, en niet het aangezicht; maar ten tijde huns kwaads zeggen zij: Sta op en verlos ons.
\par 28 Waar zijn dan uw goden, die gij u gemaakt hebt? Laat ze opstaan, of zij u ten tijde uws kwaads zullen verlossen; want naar het getal uwer steden zijn uw goden, o Juda!
\par 29 Waarom twist gij tegen Mij? Gij hebt allen tegen Mij overtreden, spreekt de HEERE.
\par 30 Tevergeefs heb Ik uw kinderen geslagen; zij hebben de tucht niet aangenomen; ulieder zwaard heeft uw profeten verteerd, als een verdervende leeuw.
\par 31 O geslacht, aanmerkt toch gijlieden des HEEREN woord! Ben Ik Israel een woestijn geweest, of een land der uiterste donkerheid? Waarom zegt dan Mijn volk: Wij zijn heren, wij zullen niet meer tot U komen?
\par 32 Vergeet ook een jonkvrouw haar versiersel, of een bruid haar bindselen? Nochtans heeft Mijn volk Mij vergeten, dagen zonder getal.
\par 33 Wat maakt gij uw weg goed, daar gij boelering zoekt? Waarom gij ook de booste hoeren uw wegen geleerd hebt.
\par 34 Ja, het bloed van de zielen der onschuldige nooddruftigen is in uw zomen gevonden; Ik heb dat niet met opgraven gevonden, maar aan alle die.
\par 35 Nog zegt gij: Zeker, ik ben onschuldig; Zijn toorn is immers van mij afgekeerd. Ziet, Ik zal met u rechten, omdat gij zegt: Ik heb niet gezondigd.
\par 36 Wat reist gij veel uit, veranderende uw weg? Gij zult ook van Egypte beschaamd worden, gelijk als gij van Assur beschaamd zijt.
\par 37 Gij zult ook van hier uitgaan met uw handen op uw hoofd; want de HEERE heeft al uw vertrouwen verworpen, zodat gij daarmede niet zult gedijen.

\chapter{3}

\par 1 Men zegt: Zo een man zijn huisvrouw verlaat, en zij gaat van hem, en wordt eens anderen mans, zal hij ook tot haar nog wederkeren? Zou datzelve land niet grotelijks ontheiligd worden? Gij nu hebt met veel boeleerders gehoereerd, keer nochtans weder tot Mij, spreekt de HEERE.
\par 2 Hef uw ogen op naar de hoge plaatsen, en zie toe, waar zijt gij niet beslapen? Gij hebt voor hen gezeten aan de wegen, als een Arabier in de woestijn; alzo hebt gij het land ontheiligd met uw hoererijen en met uw boosheid.
\par 3 Daarom zijn de regendruppelen ingehouden, en er is geen spade regen geweest. Maar gij hebt een hoerenvoorhoofd, gij weigert schaamrood te worden.
\par 4 Zult gij niet van nu af tot Mij roepen: Mijn Vader! Gij zijt de leidsman mijner jeugd!
\par 5 Zal Hij in eeuwigheid den toorn behouden? Zal Hij dien gestadig bewaren? Zie, gij spreekt en doet die boosheden, en neemt de overhand.
\par 6 Voorts zeide de HEERE tot mij, in de dagen van den koning Josia: Hebt gij gezien, wat de afgekeerde Israel gedaan heeft? Zij ging henen op allen hogen berg, en tot onder allen groenen boom, en hoereerde aldaar.
\par 7 En Ik zeide, nadat zij zulks alles gedaan had: Bekeer u tot Mij; maar zij bekeerde zich niet. Dit zag de trouweloze, haar zuster Juda.
\par 8 En Ik zag, als Ik ter oorzake van alles, waarin de afgekeerde Israel overspel bedreven had, haar verlaten, en haar haar scheidbrief gegeven had, dat de trouweloze, haar zuster Juda, niet vreesde, maar ging henen, en hoereerde zelve ook.
\par 9 Ja, het geschiedde, vanwege het gerucht harer hoererij, dat zij het land ontheiligde; want zij bedreef overspel met steen en met hout.
\par 10 En zelfs in dit alles heeft zich haar trouweloze zuster Juda tot Mij niet bekeerd met haar ganse hart, maar valselijk, spreekt de HEERE.
\par 11 Dies de HEERE tot mij zeide: De afgekeerde Israel heeft haar ziel gerechtvaardigd, meer dan de trouweloze Juda.
\par 12 Ga henen, en roep deze woorden uit tegen het noorden, en zeg: Bekeer u, gij afgekeerde Israel! spreekt de HEERE, zo zal Ik Mijn toorn op ulieden niet doen vallen; want Ik ben goedertieren, spreekt de HEERE. Ik zal den toorn niet in eeuwigheid behouden.
\par 13 Alleen ken uw ongerechtigheid, dat gij tegen den HEERE, uw God, hebt overtreden, en uw wegen verstrooid hebt tot de vreemden, onder allen groenen boom, maar gij zijt Mijner stem niet gehoorzaam geweest, spreekt de HEERE.
\par 14 Bekeert u, gij afkerige kinderen! spreekt de HEERE, want Ik heb u getrouwd, en Ik zal u aannemen, een uit een stad, en twee uit een geslacht, en zal u brengen te Sion.
\par 15 En Ik zal ulieden herders geven naar Mijn hart; die zullen u weiden met wetenschap en verstand.
\par 16 En het zal geschieden, wanneer gij vermenigvuldigd en vruchtbaar zult geworden zijn in het land, in die dagen, spreekt de HEERE, zullen zij niet meer zeggen: De ark des verbonds des HEEREN, ook zal zij in het hart niet opkomen; en zij zullen aan haar niet gedenken, en haar niet bezoeken, en zij zal niet weder gemaakt worden.
\par 17 Te dier tijd zullen zij Jeruzalem noemen, des HEEREN troon; en al de heidenen zullen tot haar vergaderd worden, om des HEEREN Naams wil, te Jeruzalem; en zij zullen niet meer wandelen naar het goeddunken van hun boos hart.
\par 18 In die dagen zal het huis van Juda gaan tot het huis van Israel; en zij zullen te zamen komen uit het land van het noorden, in het land, dat Ik uw vaderen ten erve gegeven heb.
\par 19 Ik zeide wel: Hoe zal Ik u onder de kinderen zetten, en u geven het gewenste land, de sierlijke erfenis van de heirscharen der heidenen? Maar Ik zeide: Gij zult tot Mij roepen: Mijn Vader! en gij zult van achter Mij niet afkeren.
\par 20 Waarlijk, gelijk een vrouw trouwelooslijk scheidt van haar vriend, alzo hebt gijlieden trouwelooslijk tegen Mij gehandeld, gij huis Israels! spreekt de HEERE.
\par 21 Er is een stem gehoord op de hoge plaatsen, een geween en smekingen der kinderen Israels, omdat zij hun weg verkeerd, en den HEERE, hun God, vergeten hebben.
\par 22 Keert weder, gij afkerige kinderen! Ik zal uw afkeringen genezen. Zie, hier zijn wij, wij komen tot U, want Gij zijt de HEERE, onze God!
\par 23 Waarlijk, tevergeefs verwacht men het van de heuvelen en de menigte der bergen; waarlijk, in den HEERE, onzen God, is Israels heil!
\par 24 Want de schaamte heeft den arbeid onzer vaderen opgegeten, van onze jeugd aan; hun schapen en hun runderen, hun zonen en hun dochteren.
\par 25 Wij liggen in onze schaamte, en onze schande overdekt ons, want wij hebben tegen den HEERE, onzen God, gezondigd, wij en onze vaderen, van onze jeugd aan tot op dezen dag; en wij zijn der stem des HEEREN, onzes Gods, niet gehoorzaam geweest.

\chapter{4}

\par 1 Zo gij u bekeren zult, Israel! spreekt de HEERE, bekeer u tot Mij; en zo gij uw verfoeiselen van Mijn aangezicht zult wegdoen, zo zwerft niet om.
\par 2 Maar zweer: Zo waarachtig als de HEERE leeft! in waarheid, in recht en in gerechtigheid; zo zullen zich de heidenen in Hem zegenen, en zich in Hem beroemen.
\par 3 Want zo zegt de HEERE tot de mannen van Juda, en tot Jeruzalem: Braakt ulieden een braakland, en zaait niet onder de doornen.
\par 4 Besnijdt u den HEERE en doet weg de voorhuiden uwer harten, gij mannen van Juda en inwoners van Jeruzalem! opdat Mijner grimmigheid niet uitvare als een vuur, en brande, dat niemand blussen kunne, vanwege de boosheid uwer handelingen.
\par 5 Verkondigt in Juda, en laat het horen te Jeruzalem, en zegt het; ja, blaast de bazuin in het land; roept met volle stem en zegt: Verzamelt ulieden, en laat ons ingaan in de vaste steden!
\par 6 Werpt de banier op naar Sion, vlucht met hopen, blijft niet staan! want Ik breng een kwaad aan van het noorden, en een grote breuk.
\par 7 De leeuw is opgekomen uit zijn haag, en de verderver der heidenen is opgetrokken, hij is uitgegaan uit zijn plaats, om uw land te stellen in verwoesting; uw steden zullen verstoord worden, dat er niemand in wone.
\par 8 Hierom, gordt zakken aan, bedrijft misbaar en huilt; want de hittigheid van des HEEREN toorn is niet van ons afgekeerd.
\par 9 En het zal te dier tijd geschieden, spreekt de HEERE, dat het hart des konings en het hart der vorsten vergaan zal; en de priesters zullen zich ontzetten, en de profeten zich verwonderen.
\par 10 Toen zeide ik: Ach, Heere HEERE! waarlijk, Gij hebt dit volk en Jeruzalem grotelijks bedrogen, zeggende: Gijlieden zult vrede hebben; daar het zwaard tot aan de ziel raakt.
\par 11 Te dier tijd zal tot dit volk en tot Jeruzalem gezegd worden: Een dorre wind van de hoge plaatsen in de woestijn, van den weg der dochter Mijns volks; niet om te wannen, noch om te zuiveren.
\par 12 Er zal Mij een wind komen, die hun te sterk zal zijn. Nu zal Ik ook oordelen tegen hen uitspreken.
\par 13 Ziet, hij komt op als wolken, en zijn wagenen zijn als een wervelwind, zijn paarden zijn sneller dan arenden; wee ons, want wij zijn verwoest!
\par 14 Was uw hart van boosheid, o Jeruzalem! opdat gij behouden wordt; hoe lang zult gij de gedachten uwer ijdelheid in het binnenste van u laten vernachten?
\par 15 Want een stem verkondigt van Dan af, en doet ellende horen van het gebergte van Efraim.
\par 16 Vermeldt den volke, ziet, doet het horen tegen Jeruzalem; daar komen hoeders uit verren lande; en zij verheffen hun stem tegen de steden van Juda.
\par 17 Als de wachters der velden zijn zij rondom tegen haar; omdat zij tegen Mij wederspannig geweest is, spreekt de HEERE.
\par 18 Uw weg en uw handelingen hebben u deze dingen gedaan; dit is uw boosheid, dat het zo bitter is, dat het tot aan uw hart raakt.
\par 19 O mijn ingewand, mijn ingewand! ik heb barenswee, o wanden mijns harten! mijn hart maakt getier in mij, ik kan niet zwijgen; want gij, mijn ziel! hoort het geluid der bazuin en het krijgsgeschrei.
\par 20 Breuk op breuk wordt er uitgeroepen; want het ganse land is verstoord; haastelijk zijn mijn tenten verstoord, mijn gordijnen in een ogenblik!
\par 21 Hoe lang zal ik de banier zien, het geluid der bazuin horen?
\par 22 Zekerlijk, Mijn volk is dwaas, Mij kennen zij niet; het zijn zotte kinderen, en zij zijn niet verstandig; wijs zijn zij om kwaad te doen, maar goed te doen weten zij niet.
\par 23 Ik zag het land aan, en ziet, het was woest en ledig; ook naar den hemel, en zijn licht was er niet.
\par 24 Ik zag de bergen aan, en ziet, zij beefden; en al de heuvelen schudden.
\par 25 Ik zag, en ziet, er was geen mens; en alle vogelen des hemels waren weggevlogen.
\par 26 Ik zag, en ziet, het vruchtbare land was een woestijn, en al zijn steden waren afgebroken, vanwege den HEERE, vanwege de hittigheid Zijns toorns.
\par 27 Want zo zegt de HEERE: Dit ganse land zal een woestijn zijn (doch Ik zal geen voleinding maken);
\par 28 Hierom zal de aarde treuren, en de hemel daarboven zwart zijn; omdat Ik het heb gesproken, Ik heb het voorgenomen en het zal Mij niet rouwen, en Ik zal Mij daarvan niet afkeren.
\par 29 Van het geroep der ruiteren en boogschutters vluchten al de steden; zij gaan in de wolken, en klimmen op de rotsen; al de steden zijn verlaten, zodat niemand in dezelve woont.
\par 30 Wat zult gij dan doen, gij verwoeste? Al kleeddet gij u met scharlaken, al versierdet gij u met gouden sieraad, al schuurdet gij uw ogen met blanketsel, zo zoudt gij u toch tevergeefs oppronken; de boelen versmaden u, zij zullen uw ziel zoeken.
\par 31 Want ik hoor een stem als van een vrouw, die in arbeid is, een benauwdheid als van een, die in des eersten kinds nood is, de stem van de dochter Sions; zij hijgt, zij breidt haar handen uit, zeggende: O, wee mij nu, want mijn ziel is moede vanwege de doodslagers!

\chapter{5}

\par 1 Gaat om door de wijken van Jeruzalem, en ziet nu toe, en verneemt, en zoekt op haar straten, of gij iemand vindt, of er een is, die recht doet, die waarheid zoekt, zo zal Ik haar genadig zijn.
\par 2 En of zij al zeggen: Zo waarachtig als de HEERE leeft! zo zweren zij toch valselijk.
\par 3 O HEERE! zien Uw ogen niet naar waarheid? Gij hebt hen geslagen, maar zij hebben geen pijn gevoeld; Gij hebt hen verteerd, maar zij hebben geweigerd de tucht aan te nemen; zij hebben hun aangezichten harder gemaakt dan een steenrots, zij hebben geweigerd zich te bekeren.
\par 4 Doch ik zeide: Zekerlijk, deze zijn arm; zij handelen zottelijk, omdat zij den weg des HEEREN, het recht hun Gods niet weten.
\par 5 Ik zal gaan tot de groten, en met hen spreken, want die weten den weg des HEEREN, het recht huns Gods; maar zij hadden te zamen het juk verbroken, en de banden verscheurd.
\par 6 Daarom heeft hen een leeuw uit het woud verslagen, een wolf der wildernissen zal hen verwoesten; een luipaard waakt tegen hun steden; al wie uit dezelve uitgaat, zal verscheurd worden; want hun overtredingen zijn vermenigvuldigd, hun afkeringen zijn machtig veel geworden.
\par 7 Hoe zou Ik over zulks u vergeven? Uw kinderen verlaten Mij, en zweren bij hen, die geen God zijn; als Ik hen verzadigd heb, zo bedrijven zij overspel, en verzamelen bij hopen in het hoerenhuis.
\par 8 Als welgevoederde hengsten zijn zij vroeg op; zij hunkeren een iegelijk naar zijns naasten huisvrouw.
\par 9 Zou Ik over die dingen geen bezoeking doen? spreekt de HEERE. Of zou Mijn ziel zich niet wreken aan zulk een volk, als dit is?
\par 10 Beklimt haar muren, en verderft ze (doch maakt geen voleinding); doet haar spitsen weg, want zij zijn des HEEREN niet.
\par 11 Want het huis van Israel en het huis van Juda hebben gans trouwelooslijk tegen Mij gehandeld, spreekt de HEERE.
\par 12 Zij verloochenen den HEERE, en zeggen: Hij is het niet, en ons zal geen kwaad overkomen, wij zullen noch zwaard noch honger zien.
\par 13 Ja, die profeten zullen tot wind worden, want het woord is niet bij hen; hun zelven zal zo geschieden.
\par 14 Daarom zegt de HEERE, de God der heirscharen, alzo, omdat gijlieden dit woord spreekt: Ziet, Ik zal Mijn woorden in uw mond tot vuur maken, en dit volk tot hout, en het zal hen verteren.
\par 15 Ziet, Ik zal over ulieden een volk van verre brengen, o huis Israels! spreekt de HEERE; het is een sterk volk, het is een zeer oud volk, een volk, welks spraak gij niet zult kennen, en niet horen, wat het spreken zal.
\par 16 Zijn pijlkoker is als een open graf; zij zijn altemaal helden.
\par 17 En het zal uw oogst en uw brood opeten, dat uw zonen en uw dochteren zouden eten; het zal uw schapen en uw runderen opeten; het zal uw wijnstok en uw vijgeboom opeten; uw vaste steden, op dewelke gij vertrouwt, zal het arm maken, door het zwaard.
\par 18 Nochtans zal Ik ook in die dagen, spreekt de HEERE, geen voleinding met ulieden maken.
\par 19 En het zal geschieden, wanneer gij zult zeggen: Waarom heeft ons de HEERE, onze God, al deze dingen gedaan? dat gij tot hen zeggen zult: Gelijk als gijlieden Mij hebt verlaten, en vreemde goden in uw land gediend, alzo zult gij de uitlandse dienen, in een land, dat het uwe niet is.
\par 20 Verkondigt dit in het huis van Jakob, en laat het horen in Juda, zeggende:
\par 21 Hoort nu dit, gij dwaas en harteloos volk! die ogen hebben, maar zien niet, die oren hebben, maar horen niet.
\par 22 Zult gijlieden Mij niet vrezen? spreekt de HEERE; zult gij voor Mijn aangezicht niet beven? Die der zee het zand tot een paal gesteld heb, met een eeuwige inzetting, dat zij daarover niet zal gaan; ofschoon haar golven zich bewegen, zo zullen zij toch niet vermogen, ofschoon zij bruisen, zo zullen zij toch daarover niet gaan.
\par 23 Maar dit volk heeft een afvallig en wederspannig hart; zij zijn afgevallen en heengegaan;
\par 24 En zij zeggen niet in hun hart: Laat ons nu den HEERE, onzen God, vrezen, Die den regen geeft, zo vroegen regen als spaden regen, op Zijn tijd; Die ons de weken, de gezette tijden van den oogst, bewaart.
\par 25 Uw ongerechtigheden wenden die dingen af, en uw zonden weren dat goede van ulieden.
\par 26 Want onder Mijn volk worden goddelozen gevonden; een ieder van hen loert, gelijk zich de vogelvangers schikken; zij zetten een verderfelijken strik, zij vangen de mensen.
\par 27 Gelijk een kouw vol is van gevogelte, alzo zijn hun huizen vol van bedrog; daarom zijn zij groot en rijk geworden.
\par 28 Zij zijn vet, zij zijn glad, zelfs de daden der bozen gaan zij te boven; de rechtzaak richten zij niet, zelfs de rechtzaak des wezen, nochtans zijn zij voorspoedig; ook oordelen zij het recht der nooddruftigen niet.
\par 29 Zou Ik over die dingen geen bezoeking doen? spreekt de HEERE; zou Mijn ziel zich niet wreken aan zulk een volk als dit is?
\par 30 Een schrikkelijke en afschuwelijke zaak geschiedt er in het land.
\par 31 De profeten profeteren valselijk, en de priesters heersen door hun handen; en Mijn volk heeft het gaarne alzo; maar wat zult gij ten einde van dien maken?

\chapter{6}

\par 1 Vlucht met hopen, gij kinderen van Benjamin! uit het midden van Jeruzalem, en blaast de bazuin te Thekoa, en heft een vuurteken op te Beth-cherem; want er kijkt een kwaad uit van het noorden, en een grote breuk.
\par 2 Ik heb wel de dochter Sions bij een schone en wellustige vrouw vergeleken;
\par 3 Maar er zullen herders tot haar komen met hun kudden; zij zullen tenten rondom tegen haar opslaan; zij zullen een iegelijk zijn ruimte afweiden.
\par 4 Heiligt den krijg tegen haar, maakt u op, en laat ons optrekken op den middag; o, wee ons! want de dag heeft zich gewend, want de avondschaduwen neigen zich.
\par 5 Maakt u op, en laat ons optrekken in den nacht, en haar paleizen verderven!
\par 6 Want zo zegt de HEERE der heirscharen: Houwt bomen af, en werpt een wal op tegen Jeruzalem; zij is de stad, die bezocht zal worden; in het midden van haar is enkel verdrukking.
\par 7 Gelijk een bornput zijn water opgeeft, alzo geeft zij haar boosheid op; geweld en verstoring wordt in haar gehoord, weedom en plaging is steeds voor Mijn aangezicht.
\par 8 Laat u tuchtigen, Jeruzalem! opdat Mijn ziel niet van u afgetrokken worde, opdat Ik u niet stelle tot een woestheid, tot een onbewoond land.
\par 9 Zo zegt de HEERE der heirscharen: Zij zullen Israels overblijfsel vlijtiglijk nalezen, gelijk een wijnstok; breng uw hand weder, gelijk een wijnlezer, aan de korven.
\par 10 Tot wie zal ik spreken en betuigen, dat zij het horen? Ziet, hun oor is onbesneden, dat zij niet kunnen toeluisteren; ziet, het woord des HEEREN is hun tot een smaad, zij hebben geen lust daartoe.
\par 11 Daarom ben ik vol van des HEEREN grimmigheid, ik ben moede geworden van inhouden; ik zal ze uitstorten over de kinderkens op de straat, en over de vergadering der jongelingen te zamen; want zelfs de man met de vrouw zullen gevangen worden, de oude met dien, die vol is van dagen.
\par 12 En hun huizen zullen omgewend worden tot anderen, met te zamen de akkers en vrouwen; want Ik zal Mijn hand uitstrekken tegen de inwoners dezes lands, spreekt de HEERE.
\par 13 Want van hun kleinste aan tot hun grootste toe pleegt een ieder van hen gierigheid, en van den profeet aan tot den priester toe bedrijft een ieder van hen valsheid.
\par 14 En zij genezen de breuk van de dochter Mijns volks op het lichtste, zeggende: Vrede, vrede! doch daar is geen vrede.
\par 15 Zijn zij beschaamd, omdat zij gruwel bedreven hebben? Ja, zij schamen zich in het minste niet, weten ook niet van schaamrood te maken; daarom zullen zij vallen onder de vallenden, ten tijde als Ik hen bezoeken zal, zullen zij struikelen, zegt de HEERE.
\par 16 Zo zegt de HEERE: Staat op de wegen, en ziet toe, en vraagt naar de oude paden, waar toch de goede weg zij, en wandelt daarin; zo zult gij rust vinden voor uw ziel; maar zij zeggen: Wij zullen daarin niet wandelen.
\par 17 Ik heb ook wachters over ulieden gesteld, zeggende: Luistert naar het geluid der bazuin; maar zij zeggen: Wij zullen niet luisteren.
\par 18 Daarom hoort, gij heidenen! en verneem, o gij vergadering! wat onder hen is.
\par 19 Hoor toe, gij aarde! Zie, Ik zal een kwaad brengen over dit volk, de vrucht hunner gedachten; want zij merken niet op Mijn woorden, en Mijn wet verwerpen zij.
\par 20 Waartoe zal dan de wierook voor Mij uit Scheba komen, en de beste kalmus uit verren lande? Uw brandofferen zijn Mij niet behagelijk, en uw slachtofferen zijn Mij niet zoet.
\par 21 Daarom zegt de HEERE alzo: Ziet, Ik zal dit volk allerlei aanstoot stellen; en daaraan zullen zich stoten te zamen vaders en kinderen, de nabuur en zijn metgezel, en zullen omkomen.
\par 22 Zo zegt de HEERE: Ziet, er komt een volk uit het land van het noorden, en een grote natie zal opgewekt worden uit de zijden der aarde.
\par 23 Boog en spies zullen zij voeren, het is een wreed volk, en zij zullen niet barmhartig zijn; hun stem zal bruisen als de zee, en op paarden zullen zij rijden; het is toegerust, als een man ten oorlog tegen u, o dochter van Sion!
\par 24 Wij hebben zijn gerucht gehoord, onze handen zijn slap geworden; benauwdheid heeft ons aangegrepen, weedom als van een barende vrouw.
\par 25 Gaat niet uit in het veld, noch wandelt op den weg; want des vijands zwaard is er, schrik van rondom!
\par 26 O dochter Mijns volks! gord een zak aan, en wentel u in de as, maak u rouw eens enigen zoons, een zeer bitter misbaar; want de verstoorder zal ons snellijk overkomen.
\par 27 Ik heb u onder Mijn volk gesteld, tot een wachttoren, tot een vesting; opdat gij hun weg zoudt weten en proeven.
\par 28 Zij zijn allen de afvalligsten der afvalligen, wandelende in achterklap; zij zijn koper en ijzer; zij zijn altemaal verdervers.
\par 29 De blaasbalg is verbrand, het lood is van het vuur verteerd; te vergeefs heeft de smelter zo vlijtiglijk gesmolten, dewijl de bozen niet afgetrokken zijn.
\par 30 Men noemt ze een verworpen zilver; want de HEERE heeft hen verworpen.

\chapter{7}

\par 1 Het woord, dat tot Jeremia geschied is, van den HEERE, zeggende:
\par 2 Sta in de poort van des HEEREN huis, en roep aldaar dit woord uit, en zeg: Hoort des HEEREN woord, o gans Juda! gij, die door deze poorten ingaat, om den HEERE aan te bidden.
\par 3 Zo zegt de HEERE der heirscharen, de God Israels: Maakt uw wegen en uw handelingen goed, zo zal Ik ulieden doen wonen in deze plaats.
\par 4 Vertrouwt niet op valse woorden, zeggende: Des HEEREN tempel, des HEEREN tempel, des HEEREN tempel, zijn deze!
\par 5 Maar indien gij uw wegen en uw handelingen waarlijk zult goed maken; indien gij waarlijk zult recht doen tussen den man en tussen zijn naaste;
\par 6 De vreemdeling, wees en weduwe niet zult verdrukken, en geen onschuldig bloed in deze plaats vergieten; en andere goden niet zult nawandelen, ulieden ten kwade;
\par 7 Zo zal Ik u in deze plaats, in het land, dat Ik uw vaderen gegeven heb, doen wonen van eeuw tot eeuw.
\par 8 Ziet, gij vertrouwt u op valse woorden, die geen nut doen.
\par 9 Zult gij stelen, doodslaan en overspel bedrijven, en valselijk zweren, en Baal roken, en andere goden nawandelen, die gij niet kent?
\par 10 En dan komen en staan voor Mijn aangezicht in dit huis, dat naar Mijn Naam genoemd is, en zeggen: Wij zijn verlost, om al deze gruwelen te doen?
\par 11 Is dan dit huis, dat naar Mijn Naam genoemd is, in uw ogen een spelonk der moordenaren? Ziet, Ik heb het ook gezien, spreekt de HEERE.
\par 12 Want gaat nu henen naar Mijn plaats, die te Silo was, alwaar Ik Mijn Naam in het eerst had doen wonen; en ziet, wat Ik daaraan gedaan heb vanwege de boosheid van Mijn volk Israel.
\par 13 En nu, omdat gijlieden al deze werken doet, spreekt de HEERE, en Ik tot u gesproken heb, vroeg op zijnde en sprekende, maar gij niet gehoord hebt, en Ik u geroepen, maar gij niet geantwoord hebt;
\par 14 Zo zal Ik aan dit huis, dat naar Mijn Naam genoemd is, waarop gij vertrouwt, en aan deze plaats, die Ik u en uw vaderen gegeven heb, doen, gelijk als Ik aan Silo gedaan heb.
\par 15 En Ik zal ulieden van Mijn aangezicht wegwerpen, gelijk als Ik al uw broederen, het ganse zaad van Efraim, weggeworpen heb.
\par 16 Gij dan, bid niet voor dit volk, en hef geen geschrei noch gebed voor hen op, en loop Mij niet aan; want Ik zal u niet horen.
\par 17 Ziet gij niet, wat zij doen in de steden van Juda, en op de straten van Jeruzalem?
\par 18 De kinderen lezen hout op, en de vaders steken het vuur aan, en de vrouwen kneden het deeg, om gebeelde koeken te maken voor de Melecheth des hemels, en anderen goden drankofferen te offeren, om Mij verdriet aan te doen.
\par 19 Doen zij Mij verdriet aan? spreekt de HEERE. Doen zij het zichzelven niet aan, tot beschaming huns aangezichts?
\par 20 Daarom zegt de Heere HEERE alzo: Ziet, Mijn toorn en Mijn grimmigheid zal uitgestort worden over deze plaats, over de mensen en over de beesten, en over het geboomte des velds, en over de vrucht des aardrijks; en zal branden, en niet uitgeblust worden.
\par 21 Zo zegt de HEERE der heirscharen, de God Israels: Doet uw brandofferen tot uw slachtofferen, en eet vlees.
\par 22 Want Ik heb met uw vaderen, ten dage als Ik hen uit Egypteland uitvoerde, niet gesproken, noch hun geboden van zaken des brandoffers of slachtoffers.
\par 23 Maar deze zaak heb Ik hun geboden, zeggende: Hoort naar Mijn stem, zo zal Ik u tot een God zijn, en gij zult Mij tot een volk zijn; en wandelt in al den weg, dien Ik u gebieden zal, opdat het u welga.
\par 24 Doch zij hebben niet gehoord, noch hun oor geneigd, maar gewandeld in de raadslagen, in het goeddunken van hun boos hart; en zij zijn achterwaarts gekeerd, en niet voorwaarts.
\par 25 Van dien dag af, dat uw vaders uit Egypteland zijn uitgegaan, tot op dezen dag, zo heb Ik tot u gezonden al Mijn knechten, de profeten, dagelijks vroeg op zijnde en zendende.
\par 26 Doch zij hebben naar Mij niet gehoord, noch hun oor geneigd; maar zij hebben hun nek verhard, zij hebben het erger gemaakt dan hun vaders.
\par 27 Ook zult gij al deze woorden tot hen spreken, maar zij zullen naar u niet horen; gij zult wel tot hen roepen, maar zij zullen u niet antwoorden.
\par 28 Daarom zeg tot hen: Dit is het volk, dat naar de stem des HEEREN, zijns Gods, niet hoort, en de tucht niet aanneemt; de waarheid is ondergegaan, en uitgeroeid van hun mond.
\par 29 Scheer uw hoofdhaar af, o Jeruzalem! en werp het weg, en verhef een weeklacht op de hoge plaatsen; want de HEERE heeft het geslacht Zijner verbolgenheid verworpen en verlaten.
\par 30 Want de kinderen van Juda hebben gedaan, dat kwaad is in Mijn ogen, spreekt de HEERE; zij hebben hun verfoeiselen gesteld in het huis, dat naar Mijn Naam genoemd is, om dat te verontreinigen.
\par 31 En zij hebben gebouwd de hoogten van Tofeth, dat in het dal des zoons van Hinnom is, om hun zonen en hun dochteren met vuur te verbranden; hetwelk Ik niet heb geboden, noch in Mijn hart is opgekomen.
\par 32 Daarom ziet, de dagen komen, spreekt de HEERE, dat het niet meer zal geheten worden Tofeth, noch dal des zoons van Hinnom, maar moorddal; en zij zullen ze in Tofeth begraven, omdat er geen plaats zal zijn.
\par 33 En de dode lichamen dezes volks zullen het gevogelte des hemels, en het gedierte der aarde tot spijze zijn, en niemand zal ze afschrikken.
\par 34 En Ik zal uit de steden van Juda en uit de straten van Jeruzalem doen ophouden de stem der vrolijkheid en de stem der vreugde, de stem des bruidegoms en de stem der bruid; want het land zal tot een verwoesting worden.

\chapter{8}

\par 1 Ter zelfder tijd, spreekt de HEERE, zullen zij de beenderen der koningen van Juda, en de beenderen hunner vorsten, en de beenderen der priesteren, en de beenderen der profeten, en de beenderen der inwoners van Jeruzalem, uit hun graven uithalen.
\par 2 En zij zullen ze uitspreiden voor de zon, en voor de maan, en voor het ganse heir des hemels, die zij liefgehad, en die zij gediend, en die zij nagewandeld, en die zij gezocht hebben, en voor dewelke zij zich nedergebogen hebben; zij zullen niet verzameld noch begraven worden; tot mest op den aardbodem zullen zij zijn.
\par 3 En de dood zal voor het leven verkoren worden, bij het ganse overblijfsel der overgeblevenen uit dit boze geslacht, in al de plaatsen der overgeblevenen, waar Ik hen henengedreven zal hebben, spreekt de HEERE der heirscharen.
\par 4 Zeg wijders tot hen: Zo zegt de HEERE: Zal men vallen, en niet weder opstaan? Zal men afkeren, en niet wederkeren?
\par 5 Waarom keert dan dit volk te Jeruzalem af met een altoosdurende afkering? Zij houden vast aan bedrog, zij weigeren weder te keren.
\par 6 Ik heb geluisterd en toegehoord, zij spreken dat niet recht is, er is niemand, die berouw heeft over zijn boosheid, zeggende: Wat heb ik gedaan? Een ieder keert zich om in zijn loop, gelijk een onbesuisd paard in den strijd.
\par 7 Zelfs een ooievaar aan den hemel weet zijn gezette tijden, en een tortelduif, en kraan, en zwaluw, nemen den tijd hunner aankomst waar; maar Mijn volk weet het recht des HEEREN niet.
\par 8 Hoe zegt gij dan: Wij zijn wijs en de wet des HEEREN is bij ons! Ziet, waarlijk tevergeefs werkt de valse pen der schriftgeleerden.
\par 9 De wijzen zijn beschaamd, verschrikt en gevangen; ziet, zij hebben des HEEREN woord verworpen, wat wijsheid zouden zij dan hebben?
\par 10 Daarom zal Ik hun vrouwen aan anderen geven, hun akkers aan andere bezitters; want van den kleinste aan tot den grootste toe pleegt een ieder van hen gierigheid; van den profeet aan tot den priester toe bedrijft een ieder van hen valsheid.
\par 11 En zij genezen de breuk van de dochter Mijns volks op het lichtste, zeggende: Vrede, vrede! doch daar is geen vrede.
\par 12 Zijn zij beschaamd, omdat zij gruwel bedreven hebben? Ja, zij schamen zich in het minste niet, en weten niet schaamrood te worden; daarom zullen zij vallen onder de vallenden; ten tijde hunner bezoeking zullen zij struikelen, zegt de HEERE.
\par 13 Ik zal hen voorzeker wegrapen, spreekt de HEERE; er zijn geen druiven aan den wijnstok, en geen vijgen aan den vijgeboom, ja, het blad is afgevallen; en de geboden, die Ik hun gegeven heb, die overtreden zij.
\par 14 Waarom blijven wij zitten? Verzamelt u, en laat ons ingaan in de vaste steden, en aldaar stilzwijgen; immers heeft ons de HEERE, onze God, doen stilzwijgen, en ons met gallewater gedrenkt, omdat wij tegen den HEERE gezondigd hebben.
\par 15 Men wacht naar vrede, maar er is niets goeds, naar tijd van genezing, maar ziet, er is verschrikking.
\par 16 Van Dan af wordt het gesnuif zijner paarden gehoord; het ganse land beeft van het geluid der briesingen zijner sterken; en zij komen daarhenen, dat zij het land opeten en diens volheid, de stad en die daarin wonen.
\par 17 Want ziet, Ik zend slangen, basilisken onder ulieden, tegen dewelke geen bezwering is; die zullen u bijten, spreekt de HEERE.
\par 18 Mijn verkwikking is in droefenis; mijn hart is flauw in mij.
\par 19 Ziet, de stem van het geschrei der dochteren mijns volks is uit zeer verren lande: Is dan de HEERE niet te Sion, is haar koning niet bij haar? Waarom hebben zij Mij vertoornd met hun gesneden beelden, met ijdelheden der vreemden?
\par 20 De oogst is voorbijgaande, de zomer is ten einde; nog zijn wij niet verlost.
\par 21 Ik ben gebroken vanwege de breuk der dochter mijns volks; ik ga in het zwart, ontzetting heeft mij aangegrepen.
\par 22 Is er geen balsem in Gilead? Is er geen heelmeester aldaar? Want waarom is de gezondheid der dochter mijns volks niet gerezen?

\chapter{9}

\par 1 Och, dat mijn hoofd water ware, en mijn oog een springader van tranen! zo zou ik dag en nacht bewenen de verslagenen van de dochter mijns volks.
\par 2 Och, dat ik in de woestijn een herberg der wandelaars had, zo zou ik mijn volk verlaten, en van hen trekken; want zij zijn allen overspelers, een trouweloze hoop.
\par 3 En zij spannen hun tong als hun boog tot leugen; zij worden geweldig in het land, doch niet tot waarheid; want zij gaan voort van boosheid tot boosheid, maar Mij kennen zij niet, spreekt de HEERE.
\par 4 Wacht u, een iegelijk van zijn vriend, en vertrouwt niet op enigen broeder; want elk broeder doet niet dan bedriegen, en elk vriend wandelt in achterklap.
\par 5 En zij handelen bedriegelijk, een ieder met zijn vriend, en spreken de waarheid niet; zij leren hun tong leugen spreken, zij maken zich moede met verkeerdelijk te handelen.
\par 6 Uw woning is in het midden van bedrog; door bedrog weigeren zij Mij te kennen, spreekt de HEERE.
\par 7 Daarom zegt de HEERE der heirscharen alzo: Ziet, Ik zal hen smelten en zal hen beproeven; want hoe zou Ik anders doen ten aanzien der dochter Mijns volks?
\par 8 Hun tong is een moordpijl, zij spreekt bedrog; een ieder spreekt met zijn naaste van vrede met zijn mond, maar in zijn binnenste legt hij lagen.
\par 9 Zou Ik hen om deze dingen niet bezoeken? spreekt de HEERE; zou Mijn ziel zich niet wreken aan zulk een volk, als dit is?
\par 10 Ik zal een geween en een weeklage opheffen over de bergen, en een klaaglied over de herdershutten der woestijn; want zij zijn afgebrand, dat er niemand doorgaat, en men hoort er geen stem van vee; van de vogelen des hemels aan tot de beesten toe zijn zij weggezworven, doorgegaan!
\par 11 En Ik zal Jeruzalem stellen tot steen hopen, tot een woning der draken; en de steden van Juda zal Ik stellen tot een verwoesting, zonder inwoner.
\par 12 Wie is de wijze man, die dit versta? En tot wien heeft de mond des HEEREN gesproken, dat hij het verkondige, waarom het land vergaan en afgebrand zij als een woestijn, dat er niemand doorgaat?
\par 13 En de HEERE zeide: Omdat zij Mijn wet, die Ik voor hun aangezicht gegeven had, verlaten hebben, en naar Mijn stem niet gehoord, noch daarnaar gewandeld hebben;
\par 14 Maar hebben gewandeld naar het goeddunken huns harten, en naar de Baals, hetwelk hun vaders hun geleerd hadden.
\par 15 Daarom zegt de HEERE der heirscharen, de God Israels, alzo: Ziet, Ik zal dit volk spijzen met alsem, en Ik zal hen drenken met gallewater;
\par 16 En Ik zal hen verstrooien onder de heidenen, die zij niet gekend hebben, zij noch hun vaders; en Ik zal het zwaard achter hen zenden, totdat Ik hen verteerd zal hebben.
\par 17 Zo zegt de HEERE der heirscharen: Merkt daarop, en roept klaagvrouwen, dat zij komen; en zendt henen naar de wijze vrouwen, dat zij komen.
\par 18 En haasten, en een weeklage over ons opheffen, dat onze ogen van tranen nederdalen, en onze oogleden van water vlieten.
\par 19 Want er is een stem van weeklage gehoord uit Sion: Hoe zijn wij verstoord! wij zijn zeer beschaamd, omdat wij het land hebben verlaten, omdat zij onze woningen hebben omgeworpen.
\par 20 Hoort dan des HEEREN woord, gij vrouwen! en uw oor ontvange het woord Zijns monds, en leert uw dochters weeklagen, en elke een haar metgezellin klaagliederen.
\par 21 Want de dood is geklommen in onze vensteren, hij is in onze paleizen gekomen, om de kinderkens uit te roeien van de wijken, de jongelingen van de straten.
\par 22 Spreek: Zo spreekt de HEERE: Ja, een dood lichaam des mensen zal liggen, als mest op het open veld, en als een garve achter den maaier, die niemand opzamelt.
\par 23 Zo zegt de HEERE: Een wijze beroeme zich niet in zijn wijsheid, en de sterke beroeme zich niet in zijn sterkheid; een rijke beroeme zich niet in zijn rijkdom;
\par 24 Maar die zich beroemt, beroeme zich hierin, dat hij verstaat, en Mij kent, dat Ik de HEERE ben, doende weldadigheid, recht en gerechtigheid op de aarde, want in die dingen heb Ik lust, spreekt de HEERE.
\par 25 Ziet, de dagen komen, spreekt de HEERE, dat Ik bezoeking zal doen over alle besnedenen, met degenen, die de voorhuid hebben;
\par 26 Over Egypte, en over Juda, en over Edom, en over de kinderen Ammons, en over Moab, en over allen, die aan de hoeken afgekort zijn, die in de woestijn wonen; want al de heidenen hebben de voorhuid, maar het ganse huis Israels heeft de voorhuid des harten.

\chapter{10}

\par 1 Hoort het woord, dat de HEERE tot ulieden spreekt, o huis Israels!
\par 2 Zo zegt de HEERE: Leert den weg der heidenen niet, en ontzet u niet voor de tekenen des hemels, dewijl zich de heidenen voor dezelve ontzetten.
\par 3 Want de inzettingen der volken zijn ijdelheid; want het is hout, dat men uit het woud gehouwen heeft, een werk van des werkmeesters handen met de bijl.
\par 4 Men pronkt het op met zilver en met goud; zij hechten ze met nagelen en met hameren, opdat het niet waggele.
\par 5 Zij zijn gelijk een palmboom van dicht werk, maar kunnen niet spreken; zij moeten gedragen worden, want zij kunnen niet gaan; vreest niet voor hen, want zij kunnen geen kwaad doen, ook is er geen goeddoen bij hen.
\par 6 Omdat niemand U gelijk is, o HEERE! zo zijt Gij groot, en groot is Uw Naam in mogendheid.
\par 7 Wie zou U niet vrezen, Gij Koning der heidenen? Want het komt U toe; omdat toch onder alle wijzen der heidenen, en in hun ganse koninkrijk, niemand U gelijk is.
\par 8 In een ding zijn zij toch onvernuftig en zot: een hout is een onderwijs der ijdelheden.
\par 9 Uitgerekt zilver wordt van Tarsis gebracht, en goud van Ufaz, tot een werk des werkmeesters en van de handen des goudsmids; hemelsblauw en purper is hun kleding, een werk der wijzen zijn zij al te zamen.
\par 10 Maar de HEERE God is de Waarheid, Hij is de levende God, en een eeuwig Koning; van Zijn verbolgenheid beeft de aarde, en de heidenen kunnen Zijn gramschap niet verdragen.
\par 11 (Aldus zult gijlieden tot hen zeggen: De goden, die den hemel en de aarde niet gemaakt hebben, zullen vergaan van de aarde, en van onder dezen hemel.)
\par 12 Die de aarde gemaakt heeft door Zijn kracht, Die de wereld bereid heeft door Zijn wijsheid, en den hemel uitgebreid door Zijn verstand.
\par 13 Als Hij Zijn stem geeft, zo is er een gedruis van wateren in den hemel, en Hij doet de dampen opklimmen van het einde der aarde; Hij maakt de bliksemen met den regen, en doet den wind voortkomen uit Zijn schatkameren.
\par 14 Een ieder mens is onvernuftig geworden, zodat hij geen wetenschap heeft, een ieder goudsmid is beschaamd van het gesneden beeld; want zijn gegoten beeld is leugen; en er is geen geest in hen.
\par 15 Ijdelheid zijn zij, een werk van verleidingen; ten tijde hunner bezoeking zullen zij vergaan.
\par 16 Jakobs deel is niet gelijk die, want Hij is de Formeerder van alles, en Israel is de roede Zijner erfenis; HEERE der heirscharen is Zijn Naam.
\par 17 Raap uw kramerij weg uit het land, gij inwoneres der vesting!
\par 18 Want zo zegt de HEERE: Ziet, Ik zal de inwoners des lands op ditmaal wegslingeren, en zal ze benauwen, opdat zij het vinden.
\par 19 O, wee mij over mijn breuk! mijn plage is smartelijk; en ik had gezegd: Dit is immers een krankheid, die ik wel dragen zal!
\par 20 Mijn tent is verstoord, en al mijn zelen zijn verscheurd; mijn kinderen zijn van mij uitgegaan, en zij zijn er niet; er is niemand meer, die mijn tent uitspant, en mijn gordijnen opricht.
\par 21 Want de herders zijn onvernuftig geworden, en hebben den HEERE niet gezocht; daarom hebben zij niet verstandiglijk gehandeld, en hun ganse weide is verstrooid.
\par 22 Ziet, er komt een stem des geruchts, en een groot beven uit het land van het noorden; dat men de steden van Juda zal stellen tot een verwoesting, een woning der draken.
\par 23 Ik weet, o HEERE! dat bij den mens zijn weg niet is; het is niet bij een man, die wandelt, dat hij zijn gang richte.
\par 24 Kastijd mij, HEERE! doch met mate; niet in Uw toorn, opdat Gij mij niet te niet maakt.
\par 25 Stort Uw grimmigheid uit over de heidenen, die U niet kennen, en over de geslachten, die Uw Naam niet aanroepen; want zij hebben Jakob opgegeten, ja, zij hebben hem opgegeten, en hem verteerd, en zijn woning verwoest.

\chapter{11}

\par 1 Het woord, dat tot Jeremia geschied is, van den HEERE, zeggende:
\par 2 Hoort gijlieden de woorden dezes verbonds, en spreekt tot de mannen van Juda, en tot de inwoners van Jeruzalem;
\par 3 Zeg dan tot hen: Zo zegt de HEERE, de God Israels: Vervloekt zij de man, die niet hoort de woorden dezes verbonds.
\par 4 Dat Ik uw vaderen geboden heb, ten dage als Ik hen uit Egypteland, uit den ijzeroven, uitvoerde, zeggende: Zijt Mijner stem gehoorzaam, en doet dezelve, naar alles wat Ik ulieden gebiede; zo zult gij Mij tot een volk zijn, en Ik zal u tot een God zijn;
\par 5 Opdat Ik den eed bevestige, dien Ik uw vaderen gezworen heb, hun te geven een land, vloeiende van melk en honig, als het is te dezen dage. Toen antwoordde ik en zeide: Amen, o HEERE!
\par 6 En de HEERE zeide tot mij: Roep al deze woorden uit in de steden van Juda, en in de straten van Jeruzalem, zeggende: Hoort de woorden dezes verbonds, en doet dezelve.
\par 7 Want Ik heb uw vaderen ernstiglijk betuigd, ten dage als Ik hen uit Egypteland opvoerde, tot op dezen dag, vroeg op zijnde en betuigende, zeggende: Hoort naar Mijn stem!
\par 8 Maar zij hebben niet gehoord, noch hun oor geneigd, maar hebben gewandeld, een iegelijk naar het goeddunken van hunlieder boos hart; daarom heb Ik over hen gebracht al de woorden dezes verbonds, dat Ik geboden heb te doen, maar zij niet gedaan hebben.
\par 9 Voorts zeide de HEERE tot mij: Er is een verbintenis bevonden onder de mannen van Juda, en onder de inwoners van Jeruzalem.
\par 10 Zij zijn wedergekeerd tot de ongerechtigheden hunner voorvaderen, die Mijn woorden geweigerd hebben te horen; en zij hebben andere goden nagewandeld, om die te dienen; het huis Israels en het huis van Juda hebben Mijn verbond gebroken, dat Ik met hun vaderen gemaakt heb.
\par 11 Daarom zegt de HEERE alzo: Ziet, Ik zal een kwaad over hen brengen, uit hetwelk zij niet zullen kunnen uitkomen; als zij dan tot Mij zullen roepen, zal Ik naar hen niet horen.
\par 12 Dan zullen de steden van Juda en de inwoners van Jeruzalem henengaan, en roepen tot de goden, dien zij gerookt hebben; maar zij zullen hen gans niet kunnen verlossen ten tijde huns kwaads.
\par 13 Want naar het getal uwer steden zijn uw goden geweest, o Juda! en naar het getal der straten van Jeruzalem hebt gijlieden altaren gesteld voor die schaamte, altaren om den Baal te roken.
\par 14 Gij dan, bid niet voor dit volk, en hef geen geschrei noch gebed voor hen op; want Ik zal niet horen, ten tijde als zij over hun kwaad tot Mij zullen roepen.
\par 15 Wat heeft Mijn beminde in Mijn huis te doen, dewijl zij die schandelijke daad met velen doet, en het heilige vlees van u geweken is? Wanneer gij kwaad doet, dan springt gij op van vreugde.
\par 16 De HEERE had uw naam genoemd een groenen olijfboom, schoon van liefelijke vruchten; maar nu heeft Hij met een geluid van een groot geroep een vuur om denzelven aangestoken, en zijn takken zullen verbroken worden.
\par 17 Want de HEERE der heirscharen, Die u heeft geplant, heeft een kwaad over u uitgesproken; om der boosheid wil van het huis Israels en van het huis van Juda, die zij onder zich bedrijven, om Mij te vertoornen, rokende den Baal.
\par 18 De HEERE nu heeft het mij te kennen gegeven, dat ik het wete; toen hebt Gij mij hun handelingen doen zien.
\par 19 En ik was als een lam, als een os, die geleid wordt om te slachten; want ik wist niet, dat zij gedachten tegen mij dachten, zeggende: Laat ons den boom met zijn vrucht verderven, en laat ons hem uit het land der levenden uitroeien, dat zijn naam niet meer gedacht worde.
\par 20 Maar, o HEERE der heirscharen, Gij rechtvaardige Rechter, Die de nieren en het hart proeft! laat mij Uw wraak van hen zien; want aan U heb ik mijn twistzaak ontdekt.
\par 21 Daarom, zo zegt de HEERE van de mannen van Anathoth, die uw ziel zoeken, zeggende: Profeteer niet in den Naam des HEEREN, opdat gij van onze handen niet sterft.
\par 22 Daarom, zo zegt de HEERE der heirscharen: Ziet, Ik zal bezoeking over hen doen: de jongelingen zullen door het zwaard sterven, hun zonen en hun dochteren zullen van honger sterven.
\par 23 En zij zullen geen overblijfsel hebben; want Ik zal een kwaad brengen over de mannen van Anathoth, in het jaar hunner bezoeking.

\chapter{12}

\par 1 Gij zoudt rechtvaardig zijn, o HEERE! wanneer ik tegen U zou twisten; ik zal nochtans van Uw oordelen met U spreken; waarom is der goddelozen weg voorspoedig, waarom hebben zij rust, allen, die trouwelooslijk trouweloosheid bedrijven?
\par 2 Gij hebt ze geplant, zij zijn ook ingeworteld, zij gaan voort, ook dragen zij vrucht; Gij zijt wel nabij in hun mond, maar verre van hun nieren.
\par 3 Maar Gij, o HEERE! kent mij, Gij ziet mij, en proeft mijn hart, dat het met U is. Ruk ze uit als schapen ter slachting, en heilig ze tot den dag der doding.
\par 4 Hoe lang zal het land treuren, en het kruid des gansen velds verdorren? Vanwege de boosheid dergenen, die daarin wonen, vergaan de beesten en het gevogelte; dewijl zij zeggen: Hij ziet ons einde niet.
\par 5 Als gij loopt met de voetgangers, zo maken zij u moede; hoe zult gij u dan mengen met de paarden? Zo gij alleenlijk vertrouwt in een land van vrede, hoe zult gij het dan maken in de verheffing van de Jordaan?
\par 6 Want ook uw broeders en uws vaders huis, ook diezelve handelen trouwelooslijk tegen u; ook diezelve roepen u met volle stem achterna; geloof hen niet, wanneer zij vriendelijk tot u spreken.
\par 7 Ik heb Mijn huis verlaten, Ik heb Mijn erfenis laten varen; Ik heb de beminde Mijner ziel in de hand harer vijanden gegeven.
\par 8 Mijn erfenis is Mij geworden als een leeuw in het woud; zij heeft haar stem tegen Mij verheven, daarom heb Ik haar gehaat.
\par 9 Mijn erfenis is Mij een gesprenkelde vogel; de vogelen zijn rondom tegen haar; komt aan, verzamelt, al gij gedierte des velds, komt om te eten!
\par 10 Veel herders hebben Mijn wijngaard verdorven, zij hebben Mijn akker vertreden; zij hebben Mijn gewensten akker gesteld tot een woeste wildernis.
\par 11 Men heeft hem gesteld tot een woestheid, verwoest zijnde treurt hij tot Mij; het ganse land is verwoest, omdat er niemand is, die het ter harte neemt.
\par 12 Op alle hoge plaatsen in de woestijn zijn verstoorders gekomen; want het zwaard des HEEREN verteert van het ene einde des lands tot aan het andere einde des lands; er is geen vrede voor enig vlees.
\par 13 Zij hebben tarwe gezaaid, maar doornen gemaaid; zij hebben zich gepijnigd, maar niet gevorderd; wordt alzo beschaamd vanwege ulieder inkomsten, vanwege de hittigheid van den toorn des HEEREN.
\par 14 Alzo zegt de HEERE: Aangaande al Mijn boze naburen, die Mijn erfenis aanroeren, dewelke Ik Mijn volke Israel erfelijk gegeven heb; ziet, Ik zal hen uit hun land uitrukken, maar het huis van Juda zal Ik uit hunlieder midden uitrukken.
\par 15 En het zal geschieden, nadat Ik hen zal uitgerukt hebben, zo zal Ik wederkeren, en Mij hunner ontfermen; en Ik zal hen wederbrengen, een iegelijk tot zijn erfenis, en een iegelijk tot zijn land.
\par 16 En het zal geschieden, indien zij de wegen Mijns volks vlijtiglijk zullen leren, zwerende bij Mijn Naam: Zo waarachtig als de HEERE leeft! gelijk als zij Mijn volk geleerd hebben te zweren bij Baal, zo zullen zij in het midden Mijns volks gebouwd worden.
\par 17 Maar indien zij niet zullen horen, zo zal Ik diezelve natie ten enenmale uitrukken en verdoen, spreekt de HEERE.

\chapter{13}

\par 1 Alzo heeft de HEERE tot mij gezegd: Ga henen, en koop u een linnen gordel, en doe dien aan uw lenden, maar breng hem niet in het water.
\par 2 En ik kocht een gordel naar het woord des HEEREN, en ik deed dien aan mijn lenden.
\par 3 Toen geschiedde des HEEREN woord ten tweeden male tot mij, zeggende:
\par 4 Neem den gordel, dien gij gekocht hebt, die aan uw lenden is, en maak u op, en ga henen naar den Frath, en versteek dien aldaar in de klove ener steenrots.
\par 5 Zo ging ik henen, en verstak dien bij den Frath, gelijk als de HEERE mij geboden had.
\par 6 Het geschiedde nu ten einde van vele dagen, dat de HEERE tot mij zeide: Maak u op, ga henen naar den Frath, en neem den gordel van daar, dien Ik u geboden heb aldaar te versteken.
\par 7 Zo ging ik naar den Frath, en groef, en nam den gordel van de plaats, alwaar ik dien verstoken had; en ziet, de gordel was verdorven en deugde nergens toe.
\par 8 Toen geschiedde des HEEREN woord tot mij, zeggende:
\par 9 Zo zegt de HEERE: Alzo zal Ik verderven de hovaardij van Juda, en die grote hovaardij van Jeruzalem.
\par 10 Ditzelve boze volk, dat Mijn woorden weigert te horen, dat in het goeddunken zijns harten wandelt, en andere goden navolgt, om die te dienen, en voor die zich neder te buigen; dat zal worden gelijk deze gordel, die nergens toe deugt.
\par 11 Want gelijk als een gordel kleeft aan de lenden eens mans, alzo heb Ik het ganse huis Israels en het ganse huis van Juda aan Mij doen kleven, spreekt de HEERE, om Mij te zijn tot een volk, en tot een naam, en tot lof, en tot heerlijkheid; maar zij hebben niet gehoord.
\par 12 Daarom zeg dit woord tot hen: Zo zegt de HEERE, de God Israels: Alle flessen zullen met wijn gevuld worden. Dan zullen zij tot u zeggen: Weten wij niet zeer wel, dat alle flessen met wijn gevuld zullen worden?
\par 13 Maar gij zult tot hen zeggen: Zo zegt de HEERE: Ziet, Ik zal alle inwoners deze lands, zelfs de koningen, die op Davids troon zitten, en de priesters, en de profeten, en alle inwoners van Jeruzalem, opvullen met dronkenschap.
\par 14 En Ik zal hen in stukken slaan, den een tegen den ander, zo de vaders als de kinderen te zamen, spreekt de HEERE; Ik zal niet verschonen noch sparen, noch Mij ontfermen, dat Ik hen niet zou verderven.
\par 15 Hoort en neemt ter ore, verheft u niet; want de HEERE heeft het gesproken.
\par 16 Geeft eer den HEERE, uw God, eer dat Hij het duister maakt, en eer uw voeten zich stoten aan de schemerende bergen; dat gij naar licht wacht, en Hij datzelve tot een schaduw des doods stelle, en tot een donkerheid zette.
\par 17 Zult gijlieden dat dan nog niet horen, zo zal mijn ziel in verborgene plaatsen wenen vanwege den hoogmoed, en mijn oog zal bitterlijk tranen, ja, van tranen nederdalen, omdat des HEEREN kudde gevankelijk is weggevoerd.
\par 18 Zeg tot den koning en tot de koningin: Vernedert u, zet u neder; want uw ganse hoofdsieraad, de kroon uwer heerlijkheid, is nedergedaald.
\par 19 De steden van het zuiden zijn toegesloten, en er is niemand, die ze opent; het ganse Juda is weggevoerd, het is geheel en al weggevoerd.
\par 20 Hef uw ogen op, en zie, die daar van het noorden komen! waar is de kudde, die u gegeven was, de schapen uwer heerlijkheid?
\par 21 Wat zult gij zeggen, wanneer Hij bezoeking over u doen zal, daar gij hem geleerd hebt tot vorsten, tot een hoofd over u te zijn; zullen u de smarten niet aangrijpen, als een barende vrouw?
\par 22 Wanneer gij dan in uw hart zult zeggen: Waarom zijn mij deze dingen bejegend? Om de veelheid uwer ongerechtigheid, zijn uw zomen ontdekt, en uw hielen hebben geweld geleden.
\par 23 Zal ook een Moorman zijn huid veranderen? of een luipaard zijn vlekken? Zo zult gijlieden ook kunnen goed doen, die geleerd zijt kwaad te doen.
\par 24 Daarom zal Ik hen verstrooien als een stoppel, die doorgaat, door een wind der woestijn.
\par 25 Dit zal uw lot, het deel uwer maten zijn van Mij, spreekt de HEERE; gij, die Mij hebt vergeten, en op leugen vertrouwt.
\par 26 Zo zal Ik ook uw zomen ontbloten boven uw aangezicht, en uw schande zal gezien worden.
\par 27 Uw overspelen en uw hunkeringen, de schandelijkheid uws hoerdoms, op heuvelen, in het veld; Ik heb uw verfoeiselen gezien; wee u, Jeruzalem! zult gij niet rein worden? Hoe lang nog na dezen?

\chapter{14}

\par 1 Het woord des HEEREN, dat tot Jeremia geschied is, over de zaken der grote droogte.
\par 2 Juda treurt en haar poorten zijn verzwakt; zij zijn in het zwart gekleed ter aarde toe, en Jeruzalems geschrei klimt op.
\par 3 En hun voortreffelijken zenden hun kleinen naar water; zij komen tot de grachten, zij vinden geen water, zij komen met hun vaten ledig weder; zij zijn beschaamd, ja, worden schaamrood, en bedekken hun hoofd.
\par 4 Omdat het aardrijk gescheurd is, dewijl er geen regen op de aarde is; de akkerlieden zijn beschaamd, zij bedekken hun hoofd.
\par 5 Want ook de hinden in het veld werpen jongen, en verlaten die, omdat er geen jong gras is.
\par 6 En de woudezels staan op de hoge plaatsen, zij scheppen den wind gelijk de draken; hun ogen versmachten, omdat er geen kruid is.
\par 7 Hoewel onze ongerechtigheden tegen ons getuigen, o HEERE! doe het om Uws Naams wil; want onze afkeringen zijn menigvuldig, wij hebben tegen U gezondigd.
\par 8 O Israels Verwachting, Zijn Verlosser in tijd van benauwdheid! waarom zoudt Gij zijn als een vreemdeling in het land, en als een reiziger, die slechts inkeert om te vernachten?
\par 9 Waarom zoudt Gij zijn als een versaagd man, als een held, die niet kan verlossen? Gij zijt toch in het midden van ons, o HEERE! en wij zijn naar Uw Naam genoemd, verlaat ons niet.
\par 10 Alzo zegt de HEERE van dit volk: Zij hebben zo liefgehad te zwerven, zij hebben hun voeten niet bedwongen; daarom heeft de HEERE geen welgevallen aan hen, nu zal Hij hunner ongerechtigheden gedenken, en hun zonden bezoeken.
\par 11 Wijders zeide de HEERE tot mij: Bid niet voor dit volk ten goede.
\par 12 Ofschoon zij vasten, Ik zal naar hun geschrei niet horen, en ofschoon zij brandoffer en spijsoffer offeren, Ik zal aan hen geen welgevallen hebben; maar door het zwaard, en door den honger, en door de pestilentie zal Ik hen verteren.
\par 13 Toen zeide ik: Ach, Heere HEERE! zie, die profeten zeggen hun: Gij zult geen zwaard zien, en gij zult geen honger hebben; maar Ik zal u een gewissen vrede geven in deze plaats.
\par 14 En de HEERE zeide tot mij: Die profeten profeteren vals in Mijn Naam; Ik heb hen niet gezonden, noch hun bevel gegeven, noch tot hen gesproken; zij profeteren ulieden een vals gezicht, en waarzegging, en nietigheid, en bedriegerij huns harten.
\par 15 Daarom zegt de HEERE alzo: Aangaande de profeten, die in Mijn Naam profeteren, daar Ik hen niet gezonden heb, en zij dan nog zeggen: Er zal geen zwaard noch honger in dit land zijn; diezelve profeten zullen door het zwaard en door den honger verteerd worden.
\par 16 En het volk, tot hetwelk zij profeteren, zullen op de straten van Jeruzalem weggeworpen zijn vanwege den honger en het zwaard; en er zal niemand zijn, die hen begrave, hen, hun vrouwen, en hun zonen, en hun dochteren; alzo zal Ik hun boosheid over hen uitstorten.
\par 17 Daarom zult gij dit woord tot hen zeggen: Mijn ogen zullen van tranen nederdalen nacht en dag, en niet ophouden; want de jonkvrouw der dochter Mijns volks is gebroken met een grote breuk, een plage, die zeer smartelijk is.
\par 18 Zo ik uitga in het veld, ziet daar de verslagenen van het zwaard, en zo ik in de stad komen, ziet daar de kranken van honger! Ja, zowel de profeten als de priesters lopen om in het land, en weten niet.
\par 19 Hebt Gij dan Juda ganselijk verworpen? Heeft Uw ziel een walging aan Sion? Waarom hebt Gij ons geslagen, dat er geen genezing voor ons is? Men wacht naar vrede, maar daar is niets goeds, en naar tijd van genezing, maar ziet, daar is verschrikking.
\par 20 HEERE! wij kennen onze goddeloosheid, en onzer vaderen ongerechtigheid, want wij hebben tegen U gezondigd.
\par 21 Versmaad ons niet, om Uws Naams wil; werp den troon Uwer heerlijkheid niet neder; gedenk, vernietig niet Uw verbond met ons.
\par 22 Zijn er onder de ijdelheden der heidenen, die doen regenen, of kan de hemel druppelen geven? Zijt Gij die niet, o HEERE, onze God? Daarom zullen wij op U wachten, want Gij doet al die dingen.

\chapter{15}

\par 1 Maar de HEERE zeide tot mij: Al stond Mozes en Samuel voor Mijn aangezicht, zo zou toch Mijn ziel tot dit volk niet wezen; drijf ze weg van Mijn aangezicht, en laat ze uitgaan.
\par 2 En het zal geschieden, wanneer zij tot u zullen zeggen: Waarhenen zullen wij uitgaan? dat gij tot hen zult zeggen: Zo zegt de HEERE: Wie ten dood, ten dode; en wie tot het zwaard, ten zwaarde, en wie tot den honger, ten honger; en wie ter gevangenis, ter gevangenis!
\par 3 Want Ik zal bezoeking over hen doen met vier geslachten, spreekt de HEERE: met het zwaard, om te doden; en met de honden, om te slepen; en met het gevogelte des hemels, en met het gedierte der aarde, om op te eten en te verderven.
\par 4 En Ik zal hen overgeven tot een beroering aan alle koninkrijken der aarde, vanwege Manasse, zoon van Jehizkia, koning van Juda, om hetgeen hij te Jeruzalem gedaan heeft.
\par 5 Want wie zou u verschonen, o Jeruzalem? of wie zou medelijden met u hebben, of wie zou aftreden, om u naar vrede te vragen?
\par 6 Gij hebt Mij verlaten, spreekt de HEERE; gij zijt achterwaarts gegaan; daarom zal Ik Mijn hand tegen u uitstrekken en u verderven; Ik ben des berouwens moede geworden.
\par 7 En Ik zal hen wannen met een wan, in de poorten des lands; Ik heb Mijn volk van kinderen beroofd en verdaan; zij zijn van hun wegen niet wedergekeerd.
\par 8 Hun weduwen zijn Mij meerder geworden dan zand der zeeen; Ik heb hun over de moeder doen komen een jongeling, een verwoester op den middag; Ik heb hem haastelijk hen doen overvallen, de stad met verschrikkingen.
\par 9 Zij, die zeven baarde, is zwak geworden; zij heeft haar ziel uitgeblazen, haar zon is ondergegaan, als het nog dag was; zij is beschaamd en schaamrood geworden; en hunlieder overblijfsel zal Ik aan het zwaard overgeven, voor het aangezicht hunner vijanden, spreekt de HEERE.
\par 10 Wee mij, mijn moeder, dat gij mij gebaard hebt, een man van twist, en een man van krakeel den gansen lande! Ik heb hun niet op woeker gegeven, ook hebben zij mij niet op woeker gegeven, nog vloekt mij een ieder van hen.
\par 11 De HEERE zeide: Zo niet uw overblijfsel ten goede zal zijn! zo Ik niet, in de tijd des kwaads en in tijd der benauwdheid, bij den vijand voor u tussenkome!
\par 12 Zal ook enig ijzer het ijzer van het noorden of koper verbreken?
\par 13 Ik zal uw vermogen en uw schatten tot een roof geven, zonder prijs; en dat om al uw zonden, en in al uw landpalen.
\par 14 En Ik zal u overvoeren met uw vijanden, in een land, dat gij niet kent; want een vuur is aangestoken in Mijn toorn, het zal over u branden.
\par 15 O HEERE! Gij weet het, gedenk mijner, en bezoek mij, en wreek mij van mijn vervolgers; neem mij niet weg in Uw lankmoedigheid over hen; weet, dat ik om Uwentwil versmaadheid drage.
\par 16 Als Uw woorden gevonden zijn, zo heb ik ze opgegeten, en Uw woord is mij geweest tot vreugde en tot blijdschap mijns harten; want ik ben naar Uw Naam genoemd, o HEERE, God der heirscharen!
\par 17 Ik heb in den raad der bespotters niet gezeten, noch ben van vreugde opgesprongen; vanwege Uw hand heb ik alleen gezeten, want Gij hebt mij met gramschap vervuld.
\par 18 Waarom is mijn pijn steeds durende, en mijn plage smartelijk? Zij weigert geheeld te worden; zoudt Gij mij ganselijk zijn als een leugenachtige, als wateren, die niet bestendig zijn?
\par 19 Daarom zegt de HEERE alzo: Zo gij zult wederkeren, zo zal Ik u doen wederkeren; gij zult voor Mijn aangezicht staan; en zo gij het kostelijke van het snode uittrekt, zult gij als Mijn mond zijn; laat hen tot u wederkeren, maar gij zult tot hen niet wederkeren.
\par 20 Want Ik heb u tegen dit volk gesteld tot een koperen vasten muur; zij zullen wel tegen u strijden, maar u niet overmogen; want Ik ben met u, om u te behouden en om u uit te rukken, spreekt de HEERE.
\par 21 Ja, Ik zal u rukken uit de hand der bozen, en Ik zal u verlossen uit de handpalm der tirannen.

\chapter{16}

\par 1 En des HEEREN woord geschiedde tot mij, zeggende:
\par 2 Gij zult u geen vrouw nemen, en gij zult geen zonen noch dochteren hebben in deze plaats.
\par 3 Want zo zegt de HEERE van de zonen en van de dochteren, die in deze plaats geboren worden; daartoe van hun moeders, die ze baren, en van hun vaders, die ze gewinnen in dit land:
\par 4 Zij zullen pijnlijke doden sterven, zij zullen niet beklaagd noch begraven worden, zij zullen tot mest op den aardbodem zijn, en zij zullen door het zwaard en door den honger verteerd worden, en hun dode lichamen zullen het gevogelte des hemels en het gedierte der aarde tot spijze zijn.
\par 5 Want zo zegt de HEERE: Ga niet in het huis desgenen, die een rouwmaaltijd houdt, en ga niet henen om te rouwklagen, en heb geen medelijden met hen; want Ik heb van dit volk (spreekt de HEERE) weggenomen Mijn vrede, goedertierenheid en barmhartigheden;
\par 6 Zodat groten en kleinen in dit land zullen sterven, zij zullen niet begraven worden; en men zal hen niet beklagen, noch zichzelven insnijden, noch kaal maken om hunnentwil.
\par 7 Ook zal men hun niets uitdelen over den rouw, om iemand te troosten over een dode; noch hun te drinken geven uit den troostbeker, over iemands vader of over iemands moeder.
\par 8 Ga ook niet in een huis des maaltijds, om bij hen te zitten, om te eten en te drinken.
\par 9 Want zo zegt de HEERE der heirscharen, de God Israels: Ziet, Ik zal van deze plaats, voor ulieder ogen en in ulieder dagen, doen ophouden de stem der vreugde en de stem der blijdschap, de stem des bruidegoms en de stem der bruid.
\par 10 En het zal geschieden, als gij dit volk al deze woorden zult aanzeggen, en zij tot u zeggen: Waarom spreekt de HEERE al dit grote kwaad over ons, en welke is onze misdaad, en welke is onze zonde, die wij tegen den HEERE, onzen God, gezondigd hebben?
\par 11 Dat gij tot hen zult zeggen: Omdat uw vaders Mij verlaten hebben, spreekt de HEERE, en hebben andere goden nagewandeld, en die gediend, en zich voor die nedergebogen; maar Mij verlaten, en Mijn wet niet gehouden hebben;
\par 12 En gijlieden erger gedaan hebt dan uw vaderen; want ziet, gijlieden wandelt, een iegelijk naar het goeddunken van zijn boos hart, om naar Mij niet te horen.
\par 13 Daarom zal Ik ulieden uit dit land werpen, in een land, dat gij niet gekend hebt, gij noch uw vaders; en aldaar zult gij andere goden dienen, dag en nacht, omdat Ik u geen genade zal geven.
\par 14 Daarom, ziet, de dagen komen, spreekt de HEERE, dat er niet meer zal gezegd worden: Zo waarachtig als de HEERE leeft, Die de kinderen Israels uit Egypteland heeft opgevoerd!
\par 15 Maar: Zo waarachtig als de HEERE leeft, Die de kinderen Israels heeft opgevoerd uit het land van het noorden, en uit al de landen waarhenen Hij hen gedreven had! want Ik zal hen wederbrengen in hun land, dat Ik hun vaderen gegeven heb.
\par 16 Ziet, Ik zal zenden tot veel vissers, spreekt de HEERE, die zullen hen vissen; en daarna zal Ik zenden tot veel jagers, die zullen hen jagen, van op allen berg, en van op allen heuvel, ja, uit de kloven der steenrotsen.
\par 17 Want Mijn ogen zijn op al hun wegen; zij zijn voor Mijn aangezicht niet verborgen, noch hun ongerechtigheid verholen van voor Mijn ogen.
\par 18 Dies zal Ik eerst hun ongerechtigheid en hun zonde dubbel vergelden, omdat zij Mijn land ontheiligd hebben; zij hebben Mijn erfenis met de dode lichamen hunner verfoeiselen en hunner gruwelen vervuld.
\par 19 O HEERE! Gij zijt mijn Sterkte, en mijn Sterkheid, en mijn Toevlucht ten dage der benauwdheid; tot U zullen de heidenen komen van de einden der aarde, en zeggen: Immers hebben onze vaders leugen erfelijk bezeten, en ijdelheid, waarin toch niets was, dat nut deed.
\par 20 Zal een mens zich goden maken? Zij zijn toch geen goden.
\par 21 Daarom, ziet, Ik zal hun bekend maken op ditmaal; Ik zal hun bekend maken Mijn hand en Mijn macht; en zij zullen weten, dat Mijn Naam is HEERE.

\chapter{17}

\par 1 De zonde van Juda is geschreven met een ijzeren griffie, met de punt eens diamants; gegraven in de tafel van hunlieder hart, en aan de hoornen uwer altaren;
\par 2 Gelijk hun kinderen hunner altaren gedenken, en hunner bossen, bij het groen geboomte, op de hoge heuvelen.
\par 3 Ik zal Mijn berg met het veld, uw vermogen en al uw schatten ten roof geven, mitsgaders uw hoogten, om de zonde in al uw landpalen.
\par 4 Alzo zult gij aflaten (en dat om u zelven) van uw erfenis, die Ik u gegeven heb, en Ik zal u uw vijanden doen dienen in een land, dat gij niet kent; want gijlieden hebt een vuur aangestoken in Mijn toorn, tot in eeuwigheid zal het branden.
\par 5 Zo zegt de HEERE: Vervloekt is de man, die op een mens vertrouwt, en vlees tot zijn arm stelt, en wiens hart van den HEERE afwijkt!
\par 6 Want hij zal zijn als de heide in de wildernis, die het niet gevoelt, wanneer het goede komt; maar blijft in dorre plaatsen in de woestijn, in zout en onbewoond land.
\par 7 Gezegend daarentegen is de man, die op den HEERE vertrouwt, en wiens vertrouwen de HEERE is!
\par 8 Want hij zal zijn als een boom, die aan het water geplant is, en zijn wortelen uitschiet aan een rivier, en gevoelt het niet, wanneer er een hitte komt, maar zijn loof blijft groen; en in een jaar van droogte zorgt hij niet, en houdt niet op van vrucht te dragen.
\par 9 Arglistig is het hart, meer dan enig ding, ja, dodelijk is het, wie zal het kennen?
\par 10 Ik, de HEERE, doorgrond het hart, en proef de nieren; en dat, om een iegelijk te geven naar zijn wegen, naar de vrucht zijner handelingen.
\par 11 Gelijk een veldhoen eieren vergadert, maar broedt ze niet uit, alzo is hij, die rijkdom vergadert, doch niet met recht; in de helft zijner dagen zal hij dien moeten verlaten, en in zijn laatste een dwaas zijn.
\par 12 Een troon der heerlijkheid, een hoogheid van het eerste aan, is de plaats onzes heiligdoms.
\par 13 O HEERE, Israels Verwachting! allen, die U verlaten, zullen beschaamd worden; en die van mij afwijken, zullen in de aarde geschreven worden; want zij verlaten den HEERE, den Springader des levenden waters.
\par 14 Genees mij, HEERE! zo zal ik genezen worden, behoud mij, zo zal ik behouden worden; want Gij zijt mijn Lof.
\par 15 Ziet, zij zeggen tot mij: Waar is het woord des HEEREN? Laat het nu komen!
\par 16 Ik heb toch niet aangedrongen, meer dan een herder achter U betaamde; ook heb ik den dodelijken dag niet begeerd, Gij weet het; wat uit mijn lippen is gegaan, is voor Uw aangezicht geweest.
\par 17 Wees Gij mij niet tot een verschrikking; Gij zijt mijn Toevlucht ten dage des kwaads.
\par 18 Laat mijn vervolgers beschaamd worden, maar laat mij niet beschaamd worden; laat hen verschrikt worden, maar laat mij niet verschrikt worden; breng over hen den dag des kwaads, en verbreek hen met een dubbele verbreking.
\par 19 Alzo heeft de HEERE tot mij gezegd: Ga henen en sta in de poort van de kinderen des volks, door dewelke de koningen van Juda ingaan, en door dewelke zij uitgaan, ja, in alle poorten van Jeruzalem;
\par 20 En zeg tot hen: Hoort des HEEREN woord, gij koningen van Juda, en gans Juda, en alle inwoners van Jeruzalem, die door deze poorten ingaat!
\par 21 Zo zegt de HEERE: Wacht u op uw zielen, en draagt geen last op den sabbatdag, noch brengt in door de poorten van Jeruzalem.
\par 22 Ook zult gijlieden geen last uitvoeren uit uw huizen op den sabbatdag, noch enig werk doen; maar gij zult den sabbatdag heiligen, gelijk als Ik uw vaderen geboden heb.
\par 23 Maar zij hebben niet gehoord, noch hun oor geneigd; maar zij hebben hun nek verhard, om niet te horen, en om de tucht niet aan te nemen.
\par 24 Het zal dan geschieden, indien gij vlijtiglijk naar Mij zult horen, spreekt de HEERE, dat gij geen last door de poorten dezer stad op den sabbatdag inbrengt, en gij den sabbatdag heiligt, dat gij geen werk daarop doet;
\par 25 Zo zullen door de poorten dezer stad ingaan koningen en vorsten, zittende op den troon van David, rijdende op wagenen en op paarden, zij en hun vorsten, de mannen van Juda en de inwoners van Jeruzalem; en deze stad zal bewoond worden in eeuwigheid.
\par 26 En zij zullen komen uit de steden van Juda, en uit de plaatsen rondom Jeruzalem, en uit het land van Benjamin, en uit de laagte, en van het gebergte, en van het zuiden, aanbrengende brandoffer, en slachtoffer, en spijsoffer, en wierook, en aanbrengende lofoffer, ten huize des HEEREN.
\par 27 Maar indien gij naar Mij niet zult horen, om den sabbatdag te heiligen, en om geen last te dragen als gij op den sabbatdag door de poorten van Jeruzalem ingaat; zo zal Ik een vuur in haar poorten aansteken, dat de paleizen van Jeruzalem zal verteren, en niet worden uitgeblust.

\chapter{18}

\par 1 Het woord, dat tot Jeremia geschied is van den HEERE, zeggende:
\par 2 Maak u op, en ga af in het huis des pottenbakkers, en aldaar zal Ik u Mijn woorden doen horen.
\par 3 Zo ging ik af in het huis des pottenbakkers; en ziet, hij maakte een werk op de schijven.
\par 4 En het vat, dat hij maakte, werd verdorven, als leem, in de hand des pottenbakkers; toen maakte hij daarvan weder een ander vat, gelijk als het recht was in de ogen des pottenbakkers te maken.
\par 5 Toen geschiedde des HEEREN woord tot mij, zeggende:
\par 6 Zal Ik ulieden niet kunnen doen, gelijk deze pottenbakker, o huis Israels? spreekt de HEERE; ziet, gelijk leem in de hand des pottenbakkers, alzo zijt gijlieden in Mijn hand, o huis Israels!
\par 7 In een ogenblik zal Ik spreken over een volk en over een koninkrijk, dat Ik het zal uitrukken, en afbreken, en verdoen;
\par 8 Maar indien datzelve volk, over hetwelk Ik zulks gesproken heb, zich van zijn boosheid bekeert, zo zal Ik berouw hebben over het kwaad, dat Ik hetzelve gedacht te doen.
\par 9 Ook zal Ik in een ogenblik spreken over een volk en over een koninkrijk, dat Ik het zal bouwen en planten;
\par 10 Maar indien het doet, dat kwaad is in Mijn ogen, dat het naar Mijn stem niet hoort, zo zal Ik berouw hebben over het goede, met hetwelk Ik gezegd had hetzelve te zullen weldoen.
\par 11 Nu dan, spreek nu tot de mannen van Juda en tot de inwoners van Jeruzalem, zeggende: Zo zegt de HEERE: Ziet, Ik formeer een kwaad tegen ulieden, en denk tegen ulieden een gedachte; zo bekeert u nu, een iegelijk van zijn bozen weg, en maakt uw wegen en uw handelingen goed.
\par 12 Doch zij zeggen: Het is buiten hoop; maar wij zullen naar onze gedachten wandelen, en wij zullen doen, een iegelijk het goeddunken van zijn boos hart.
\par 13 Daarom, zo zegt de HEERE: Vraagt nu onder de heidenen; wie heeft alzulks gehoord? De jonkvrouw Israels doet een zeer afschuwelijke zaak.
\par 14 Zal men ook om een rotssteen des velds verlaten de sneeuw van Libanon? Zullen ook de vreemde, koude, vlietende wateren verlaten worden?
\par 15 Nochtans heeft Mijn volk Mij vergeten, zij roken der ijdelheid; want zij hebben hen doen aanstoten op hun wegen, op de oude paden, opdat zij mochten wandelen in stegen van een weg, die niet opgehoogd is;
\par 16 Om hun land te stellen tot een ontzetting, tot eeuwige aanfluitingen; al wie daar voorbijgaat, zal zich ontzetten, en met zijn hoofd schudden.
\par 17 Als een oostenwind zal Ik hen verstrooien voor het aangezicht des vijands; Ik zal hun den nek en niet het aangezicht laten zien, ten dage huns verderfs.
\par 18 Toen zeiden zij: Komt aan, laat ons gedachten tegen Jeremia denken; want de wet zal niet vergaan van den priester, noch de raad van den wijze, noch het woord van den profeet; komt aan, en laat ons hem slaan met de tong, en laat ons niet luisteren naar enige zijner woorden!
\par 19 HEERE! luister naar mij, en hoor naar de stem mijner twisters.
\par 20 Zal dan kwaad voor goed vergolden worden? want zij hebben mijn ziel een kuil gegraven; gedenk, dat ik voor Uw aangezicht gestaan heb, om goed voor hen te spreken, om Uw grimmigheid van hen af te wenden.
\par 21 Daarom, geef hun zonen den honger over, en doe ze wegvloeien door het geweld des zwaards, en laat hun vrouwen van kinderen beroofd en weduwen worden, en laat hun mannen door den dood omgebracht, en hun jongelingen met het zwaard geslagen worden in den strijd.
\par 22 Laat er een geschrei uit hun huizen gehoord worden, wanneer Gij haastelijk een bende over hen zult brengen; dewijl zij een kuil gegraven hebben om mij te vangen, en strikken verborgen voor mijn voeten.
\par 23 Doch Gij, HEERE! weet al hun raad tegen mij ten dode; maak geen verzoening over hun ongerechtigheid, en delg hun zonde niet uit van voor Uw aangezicht; maar laat hen nedergeveld worden voor Uw aangezicht; handel alzo met hen, ten tijde Uws toorns.

\chapter{19}

\par 1 Zo zegt de HEERE: Ga henen en koop een pottenbakkerskruik, en neem tot u van de oudsten des volks, en van de oudsten der priesteren.
\par 2 En ga uit naar het dal des zoons van Hinnom, dat voor de deur der Zonnepoort is, en roep aldaar uit de woorden, die Ik tot u spreken zal;
\par 3 En zeg: Hoort des HEEREN woord, gij koningen van Juda en inwoners van Jeruzalem! Alzo zegt de HEERE der heirscharen, de God Israels: Ziet, Ik zal een kwaad brengen over deze plaats, van hetwelk een ieder, die het hoort, zijn oren klinken zullen;
\par 4 Omdat zij Mij verlaten, en deze plaats vervreemd, en anderen goden daarin gerookt hebben die zij niet gekend hebben, zij, noch hun vaders, noch de koningen van Juda; en hebben deze plaats vervuld met bloed der onschuldigen.
\par 5 Want zij hebben de hoogten van Baal gebouwd, om hun zonen met vuur te verbranden, aan Baal tot brandofferen; hetwelk Ik niet geboden, noch gesproken heb, noch in Mijn hart is opgekomen?
\par 6 Daarom, ziet, de dagen komen, spreekt de HEERE, dat deze plaats niet meer zal genoemd worden het Tofeth, of dal des zoons van Hinnom, maar Moorddal.
\par 7 Want Ik zal den raad van Juda en Jeruzalem in deze plaats verijdelen, en zal hen voor het aangezicht hunner vijanden doen vallen door het zwaard, en door de hand dergenen, die hun ziel zoeken; en Ik zal hun dode lichamen het gevogelte des hemels en het gedierte der aarde tot spijze geven.
\par 8 En Ik zal deze stad zetten tot een ontzetting en tot een aanfluiting; al wie voorbij haar gaat, zal zich ontzetten en fluiten over al haar plagen.
\par 9 En Ik zal hunlieden het vlees hunner zonen en het vlees hunner dochteren doen eten, en zij zullen eten, een iegelijk het vlees zijns naasten, in de belegering en in de benauwing, waarmede hen hun vijanden, en die hun ziel zoeken, benauwen zullen.
\par 10 Dan zult gij de kruik verbreken voor de ogen der mannen, die met u gegaan zijn;
\par 11 En gij zult tot hen zeggen: Zo zegt de HEERE der heirscharen: Alzo zal Ik dit volk en deze stad verbreken, gelijk als men een pottenbakkersvat verbreekt, dat niet weder geheeld kan worden; en zij zullen hen in Tofeth begraven, omdat er geen andere plaats zal zijn om te begraven.
\par 12 Zo zal Ik deze plaats doen, spreekt de HEERE, en haar inwoners; en dat om deze stad te stellen als een Tofeth.
\par 13 En de huizen van Jeruzalem en de huizen der koningen van Juda zullen, gelijk alle plaatsen van Tofeth, onrein worden, met al de huizen, op welker daken zij aan al het heir des hemels gerookt en aan vreemde goden drankofferen geofferd hebben.
\par 14 Toen nu Jeremia van Tofeth kwam, waarhenen hem de HEERE gezonden had, om te profeteren, stond hij in het voorhof van des HEEREN huis, en zeide tot al het volk:
\par 15 Zo zegt de HEERE der heirscharen, de God Israels: Ziet, Ik zal over deze stad, en over al haar steden, al het kwaad brengen, dat Ik over haar gesproken heb; omdat zij hun nek verhard hebben, om Mijn woorden niet te horen.

\chapter{20}

\par 1 Als Pashur, de zoon van Immer, de priester (deze nu was bestelde voorganger in het huis des HEEREN), Jeremia hoorde, diezelve woorden profeterende,
\par 2 Zo sloeg Pashur den profeet Jeremia, en hij stelde hem in de gevangenis, dewelke is in de bovenste poort van Benjamin, die aan het huis des HEEREN is.
\par 3 Maar het geschiedde des anderen daags, dat Pashur Jeremia uit de gevangenis voortbracht; toen zeide Jeremia tot hem: De HEERE noemt uw naam niet Pashur, maar Magor-missabib.
\par 4 Want zo zegt de HEERE: Zie, Ik stel u tot een schrik voor uzelven en voor al uw liefhebbers; die zullen vallen door het zwaard hunner vijanden, dat het uw ogen aanzien; en Ik zal gans Juda geven in de hand des konings van Babel, die hen naar Babel gevankelijk zal wegvoeren, en slaan hen met het zwaard.
\par 5 Ook zal Ik geven al het vermogen dezer stad, en al haar arbeid, en al haar kostelijkheid, en alle schatten der koningen van Juda, Ik zal ze geven in de hand hunner vijanden, die zullen ze roven, zullen ze nemen, en zullen ze brengen naar Babel.
\par 6 En gij, Pashur, en alle inwoners van uw huis! gijlieden zult gaan in de gevangenis; en gij zult te Babel komen, en aldaar sterven, en aldaar begraven worden, gij en al uw vrienden, denwelken gij valselijk geprofeteerd hebt.
\par 7 HEERE! Gij hebt mij overreed, en ik ben overreed geworden; Gij zijt mij te sterk geweest, en hebt overmocht; ik ben den gansen dag tot een belachen, een ieder van hen bespot mij.
\par 8 Want sinds ik spreke, roep ik uit, ik roep geweld en verstoring; omdat mij des HEEREN woord den gansen dag tot smaad en tot schimp is.
\par 9 Dies zeide ik: Ik zal Zijner niet gedenken, en niet meer in Zijn Naam spreken; maar het werd in mijn hart als een brandend vuur, besloten in mijn beenderen; en ik bemoeide mij om te verdragen, maar konde niet.
\par 10 Want ik heb gehoord de naspraak van velen, van Magor-missabib, zeggende: Geef ons te kennen, en wij zullen het te kennen geven; al mijn vredegenoten nemen acht op mijn hinking; zij zeggen: Misschien zal hij overreed worden, dan zullen wij hem overmogen, en onze wraak van hem nemen.
\par 11 Maar de HEERE is met mij als een verschrikkelijk Held; daarom zullen mijn vervolgers struikelen, en niets vermogen; zij zijn zeer beschaamd geworden, omdat zij niet verstandiglijk gehandeld hebben; het zal een eeuwige schande zijn, zij zal niet vergeten worden.
\par 12 Gij dan, o HEERE der heirscharen, Die den rechtvaardige proeft, Die de nieren en het hart ziet, laat mij Uw wraak van hen zien, want ik heb U mijn twistzaak ontdekt.
\par 13 Zingt den HEERE, prijst den HEERE; want Hij heeft de ziel des nooddruftigen uit de hand der boosdoeners verlost.
\par 14 Vervloekt zij de dag, op welken ik geboren ben; de dag, op welken mijn moeder mij gebaard heeft, zij niet gezegend!
\par 15 Vervloekt zij de man, die mijn vader geboodschapt heeft, zeggende: U is een jonge zoon geboren, verblijdende hem grotelijks!
\par 16 Ja, dezelve man zij, als de steden, die de HEERE heeft omgekeerd, en het heeft Hem niet berouwd; en hij hore in den morgenstond een geroep, en op den middagtijd een geschrei.
\par 17 Dat Hij mij niet gedood heeft van de baarmoeder af! Of mijn moeder mijn graf geweest is, of haar baarmoeder als van een, die eeuwiglijk zwanger is!
\par 18 Waarom ben ik toch uit de baarmoeder voortgekomen, om moeite en droefenis te zien, en dat mijn dagen in beschaamdheid vergaan?

\chapter{21}

\par 1 Het woord, dat van den HEERE geschied is tot Jeremia, als koning Zekekia tot hem zond Pashur, den zoon van Malchia, en Zefanja, den zoon van Maaseja, den priester, zeggende:
\par 2 Vraag toch den HEERE voor ons, want Nebukadnezar, de koning van Babel, strijdt tegen ons; misschien zal de HEERE met ons doen naar al Zijn wonderen, dat hij van ons optrekke.
\par 3 Toen zeide Jeremia tot hen: Zo zult gijlieden tot Zedekia zeggen:
\par 4 Zo zegt de HEERE, de God Israels: Ziet, Ik zal de krijgswapenen omwenden, die in ulieder hand zijn, met dewelke gij strijdt tegen den koning van Babel en tegen de Chaldeen, die u belegeren, van buiten aan den muur; en Ik zal ze verzamelen in het midden van deze stad.
\par 5 En Ik Zelf zal tegen ulieden strijden, met een uitgestrekte hand en met een sterken arm, ja, met toorn, en met grimmigheid, en met grote verbolgenheid.
\par 6 En Ik zal de inwoners dezer stad slaan, zowel de mensen als de beesten; door een grote pestilentie zullen zij sterven.
\par 7 En daarna spreekt de HEERE, zal Ik Zedekia, den koning van Juda, en zijn knechten, en het volk, en die in deze stad overgebleven zijn van de pestilentie, van het zwaard en van den honger, geven in de hand van Nebukadnezar, den koning van Babel, en in de hand hunner vijanden, en in de hand dergenen, die hun ziel zoeken; en hij zal ze slaan met de scherpte des zwaards; hij zal ze niet sparen, noch verschonen, noch zich ontfermen.
\par 8 En tot dit volk zult gij zeggen: Zo zegt de HEERE: Ziet, Ik stel voor ulieder aangezicht den weg des levens en den weg des doods.
\par 9 Die in deze stad blijft, zal sterven door het zwaard, of door den honger, of door de pestilentie; maar die er uitgaat en valt tot de Chaldeen, die ulieden belegeren, die zal leven, en zijn ziel zal hem tot een buit zijn.
\par 10 Want Ik heb Mijn aangezicht tegen deze stad gesteld ten kwade en niet ten goede, spreekt de HEERE; zij zal gegeven worden in de hand des konings van Babel, en hij zal ze met vuur verbranden.
\par 11 En aangaande het huis des konings van Juda, hoort des HEEREN woord.
\par 12 O huis Davids! zo zegt de HEERE: Richt des morgens recht, en verlost den beroofde uit den hand des verdrukkers; opdat Mijn gramschap niet uitvare als een vuur, en brande, dat niemand blussen kunne, vanwege de boosheid uwer handelingen.
\par 13 Ziet, Ik wil aan u, gij inwoneres des dals, gij rots van het plein! spreekt de HEERE; gijlieden, die zegt: Wie zou tegen ons afkomen, of wie zou komen in onze woningen?
\par 14 En Ik zal over ulieden bezoeking doen naar de vrucht uwer handelingen, spreekt de HEERE; en Ik zal een vuur aansteken in haar woud, dat zal verteren al wat rondom haar is.

\chapter{22}

\par 1 Alzo zegt de HEERE: Ga af in het huis des konings van Juda, en spreek aldaar dit woord.
\par 2 En zeg: Hoor het woord des HEEREN, gij koning van Juda, gij, die zit op Davids troon, gij, en uw knechten, en uw volk, die door deze poorten ingaan!
\par 3 Zo zegt de HEERE: Doet recht en gerechtigheid, en redt den beroofde uit de hand des verdrukkers; en onderdrukt den vreemdeling niet, den wees noch de weduwe; doet geen geweld en vergiet geen onschuldig bloed in deze plaats.
\par 4 Want indien gijlieden deze zaak ernstiglijk zult doen, zo zullen door de poorten van dit huis koningen ingaan, zittende den David op zijn troon, rijdende op wagens en op paarden, hij, en zijn knechten, en zijn volk.
\par 5 Indien gij daarentegen deze woorden niet zult horen, zo heb Ik bij Mij gezworen, spreekt de HEERE, dat dit huis tot een woestheid worden zal.
\par 6 Want zo zegt de HEERE van het huis des konings van Juda: Gij zijt Mij een Gilead, een hoogte van Libanon; maar zo Ik u niet zette als een woestijn en onbewoonde steden!
\par 7 Want Ik zal verdervers tegen u heiligen, elk met zijn gereedschap; die zullen uw uitgelezen cederen omhouwen, en in het vuur werpen.
\par 8 Dan zullen veel heidenen voorbij deze stad gaan, en zullen zeggen, een ieder tot zijn naaste: Waarom heeft de HEERE alzo gedaan aan deze grote stad?
\par 9 En zij zullen zeggen: Omdat zij het verbond des HEEREN, huns Gods, hebben verlaten, en hebben zich voor andere goden nedergebogen, en die gediend.
\par 10 Weent niet over den dode, en beklaagt hem niet; weent vrij over dien, die weggegaan is, want hij zal nimmermeer wederkomen, dat hij het land zijner geboorte zie.
\par 11 Want zo zegt de HEERE van Sallum, den zoon van Josia, koning van Juda, die in de plaats van zijn vader Josia regeerde, die uit deze plaats is uitgegaan: Hij zal daar nimmermeer wederkomen.
\par 12 Maar in de plaats, waarhenen zij hem gevankelijk hebben weggevoerd, zal hij sterven, en dit land zal hij niet meer zien.
\par 13 Wee dien, die zijn huis bouwt met ongerechtigheid, en zijn opperzalen met onrecht; die zijns naasten dienst om niet gebruikt, en geeft hen zijn arbeidsloon niet!
\par 14 Die daar zegt: Ik zal mij een zeer hoog huis bouwen, en doorluchtige opperzalen; en hij houwt zich vensteren uit, en het is bedekt met ceder, en aangestreken met menie.
\par 15 Zoudt gij regeren, omdat gij u mengt met den ceder? Heeft niet uw vader gegeten en gedronken, en recht en gerechtigheid gedaan, en het ging hem toen wel?
\par 16 Hij heeft de rechtzaak des ellendigen en nooddruftigen gericht, toen ging het hem wel; is dat niet Mij te kennen? spreekt de HEERE.
\par 17 Maar uw ogen en uw hart zijn niet dan op uw gierigheid, en op onschuldig bloed, om dat te vergieten, en op verdrukking en overlast, om die te doen.
\par 18 Daarom zegt de HEERE alzo van Jojakim, zoon van Josia, koning van Juda: Zij zullen hem niet beklagen: Och mijn broeder! of, och zuster! Zij zullen hem niet beklagen: Och, heer! of, och zijn majesteit!
\par 19 Met een ezelsbegrafenis zal hij begraven worden; men zal hem slepen en daarhenen werpen, verre weg van de poorten van Jeruzalem.
\par 20 Klim op den Libanon en roep, en verhef uw stem op den Basan; roep ook van de veren; maar al uw liefhebbers zijn verbroken.
\par 21 Ik sprak u aan in uw groten voorspoed, maar gij zeidet: Ik zal niet horen. Dit is uw weg van uw jeugd af, dat gij Mijner stem niet hebt gehoorzaamd.
\par 22 De wind zal al uw herders weiden, en uw liefhebbers zullen in de gevangenis gaan; dan zult gij zekerlijk beschaamd en te schande worden, vanwege al uw boosheid.
\par 23 O gij, die nu op den Libanon woont, en in de cederen nestelt! hoe begenadigd zult gij zijn, als u de smarten zullen aankomen, het wee als ener barende vrouw!
\par 24 Zo waarachtig als Ik leef, spreekt de HEERE, ofschoon Chonia, de zoon van Jojakim, den koning van Juda, een zegelring ware aan Mijn rechterhand, zo zal Ik u toch van daar wegrukken.
\par 25 En Ik zal u geven in de hand dergenen, die uw ziel zoeken, en in de hand dergenen, voor welker aangezicht gij schrikt, namelijk in de hand van Nebukadnezar, den koning van Babel, en in de hand der Chaldeen.
\par 26 En Ik zal u, en uw moeder, die u gebaard heeft, uitwerpen in een ander land, waarin gijlieden niet geboren zijt, en daar zult gij sterven.
\par 27 En in het land, naar hetwelk hun ziel verlangt om daar weder te komen, daarhenen zullen zij niet wederkomen.
\par 28 Is dan deze man Chonia een veracht, verstrooid, afgodisch beeld? Of is hij een vat, waaraan men geen lust heeft? Waarom zijn hij en zijn zaad uitgeworpen, ja, weggeworpen in een land, dat zij niet kennen?
\par 29 O land, land, land! hoor des HEEREN woord!
\par 30 Zo zegt de HEERE: Schrijft dezen zelfden man kinderloos, een man, die niet voorspoedig zal zijn in zijn dagen; want er zal niemand van zijn zaad voorspoedig zijn, zittende op den troon Davids, en heersende meer in Juda.

\chapter{23}

\par 1 Wee den herderen, die de schapen Mijner weide ombrengen en verstrooien! spreekt de HEERE.
\par 2 Daarom zegt de HEERE, de God Israels, alzo van de herderen, die Mijn volk weiden: Gijlieden hebt Mijn schapen verstrooid, en hebt ze verdreven, en hebt ze niet bezocht; ziet, Ik zal over u bezoeken de boosheid uwer handelingen, spreekt de HEERE.
\par 3 En Ik zal het overblijfsel Mijner schapen Zelf vergaderen uit al de landen, waarhenen Ik ze verdreven heb; en Ik zal ze wederbrengen tot hun kooien, en zij zullen vruchtbaar zijn, en vermenigvuldigen.
\par 4 En Ik zal herderen over hen verwekken, die ze weiden zullen; en zij zullen niet meer vrezen, noch verschrikt worden, noch gemist worden, spreekt de HEERE.
\par 5 Ziet, de dagen komen, spreekt de HEERE, dat Ik aan David een rechtvaardige Spruit zal verwekken; Die zal Koning zijnde regeren, en voorspoedig zijn, en recht en gerechtigheid doen op de aarde.
\par 6 In Zijn dagen zal Juda verlost worden, en Israel zeker wonen; en dit zal Zijn naam zijn, waarmede men Hem zal noemen: De HEERE: ONZE GERECHTIGHEID.
\par 7 Daarom, ziet, de dagen komen, spreekt de HEERE, dat zij niet meer zullen zeggen: Zo waarachtig als de HEERE leeft, Die de kinderen Israels uit Egypteland heeft opgevoerd.
\par 8 Maar: Zo waarachtig als de HEERE leeft, Die het zaad van het huis Israels heeft opgevoerd, en Die het aangebracht heeft uit het land van het noorden, en uit al de landen, waarheen Ik ze gedreven had! want zij zullen wonen in hun land.
\par 9 Aangaande de profeten. Mijn hart wordt in mijn binnenste gebroken, al mijn beenderen bewegen zich; ik ben als een dronken man, en als een man, dien de wijn te boven gaat; vanwege den HEERE, en vanwege de woorden Zijner heiligheid.
\par 10 Want het land is vol overspelers, want het land treurt vanwege den vloek, de weiden der woestijn verdorren, omdat hun loop boos is, en hun macht niet recht.
\par 11 Want beiden profeten en priesters zijn huichelaars; zelfs in Mijn huis vind Ik hun boosheid, spreekt de HEERE.
\par 12 Daarom zal hun weg hun zijn als zeer gladde plaatsen in de donkerheid; zij zullen aangedreven worden en daarin vallen; want Ik zal een kwaad over hen brengen in het jaar hunner bezoeking, spreekt de HEERE.
\par 13 Ik heb wel ongerijmdheid gezien in de profeten van Samaria, die door Baal, profeteerden, en Mijn volk Israel verleidden.
\par 14 Maar in de profeten van Jeruzalem zie Ik afschuwelijkheid; zij bedrijven overspel, en gaan om met valsheid, en sterken de handen der boosdoeners, opdat zij zich niet bekeren, een iegelijk van zijn boosheid; zij allen zijn Mij als Sodom, en haar inwoners als Gomorra.
\par 15 Daarom zegt de HEERE der heirscharen van deze profeten alzo: Ziet, Ik zal hen met alsem spijzigen, en met gallewater drenken; want van Jeruzalems profeten is de huichelarij uitgegaan in het ganse land.
\par 16 Zo zegt de HEERE der heirscharen: Hoort niet naar de woorden der profeten, die u profeteren; zij maken u ijdel; zij spreken het gezicht huns harten, niet uit des HEEREN mond.
\par 17 Zij zeggen steeds tot degenen, die Mij lasteren: De HEERE heeft het gesproken, gijlieden zult vrede hebben; en tot al wie naar zijns harten goeddunken wandelt, zeggen zij: Ulieden zal geen kwaad overkomen.
\par 18 Want wie heeft in des HEEREN raad gestaan, en Zijn woord gezien of gehoord? Wie heeft Zijn woord aangemerkt en gehoord?
\par 19 Ziet, een onweder des HEEREN, een grimmigheid is uitgegaan, ja, een pijnlijk onweder, het zal blijven op der goddelozen hoofd.
\par 20 Des HEEREN toorn zal zich niet afwenden, totdat Hij zal hebben gedaan, en totdat Hij zal hebben daargesteld de gedachten Zijns harten; in het laatste der dagen zult gij met verstand daarop letten.
\par 21 Ik heb die profeten niet gezonden, nochtans hebben zij gelopen; Ik heb tot hen niet gesproken, nochtans hebben zij geprofeteerd.
\par 22 Maar zo zij in Mijn raad hadden gestaan, zo zouden zij Mijn volk Mijn woorden hebben doen horen, en zouden hen afgekeerd hebben van hun bozen weg, en van de boosheid hunner handelingen.
\par 23 Ben Ik een God van nabij, spreekt de HEERE, en niet een God van verre?
\par 24 Zou zich iemand in verborgene plaatsen kunnen verbergen, dat Ik hem niet zou zien? spreekt de HEERE; vervul Ik niet den hemel en de aarde? spreekt de HEERE.
\par 25 Ik heb gehoord, wat de profeten zeggen, die in Mijn Naam leugen profeteren, zeggende: Ik heb gedroomd, ik heb gedroomd.
\par 26 Hoe lang? Is er dan een droom in het hart der profeten, die de leugen profeteren? Ja, het zijn profeten van huns harten bedriegerij.
\par 27 Die daar denken om Mijn volk Mijn Naam te doen vergeten, door hun dromen, die zij, een ieder zijn naaste, vertellen; gelijk als hun vaders Mijn Naam vergeten hebben door Baal.
\par 28 De profeet, bij welken een droom is, die vertelle den droom; en bij welken Mijn woord is, die spreke Mijn woord waarachtiglijk; wat heeft het stro met het koren te doen? spreekt de HEERE.
\par 29 Is Mijn woord niet alzo, als een vuur? spreekt de HEERE, en als een hamer, die een steenrots te morzel slaat?
\par 30 Daarom, ziet, Ik wil aan de profeten, spreekt de HEERE, die Mijn woorden stelen, een ieder van zijn naaste;
\par 31 Ziet, Ik wil aan de profeten, spreekt de HEERE, die hun tong nemen, en spreken: Hij heeft het gesproken;
\par 32 Ziet, Ik wil aan degenen, die valse dromen profeteren, spreekt de HEERE, en vertellen die, en verleiden Mijn volk met hun leugenen en met hun lichtvaardigheid; daar Ik hen niet gezonden, en hun niets bevolen heb, en zij dit volk gans geen nut doen, spreekt de HEERE.
\par 33 Wanneer dan dit volk, of een profeet, of priester u vragen zal, zeggende: Wat is des HEEREN last? Zo zult gij tot hen zeggen: Wat last? Dat Ik ulieden verlaten zal, spreekt de HEERE.
\par 34 En aangaande den profeet, of den priester, of het volk, dat zeggen zal: Des HEEREN last; dat Ik bezoeking zal doen over dien man en over zijn huis.
\par 35 Aldus zult gijlieden zeggen, een iegelijk tot zijn naaste, en een iegelijk tot zijn broeder: Wat heeft de HEERE geantwoord, en wat heeft de HEERE gesproken?
\par 36 Maar des HEEREN last zult gij niet meer gedenken; want een iegelijk zal zijn eigen woord een last zijn, dewijl gij verkeert de woorden van den levenden God, den HEERE der heirscharen, onzen God.
\par 37 Aldus zult gij zeggen tot den profeet: Wat heeft u de HEERE geantwoord en wat heeft de HEERE gesproken?
\par 38 Maar dewijl gij zegt: Des HEEREN last; daarom, zo zegt de HEERE: Omdat gij dit woord zegt: Des HEEREN last, daar Ik tot u gezonden heb, zeggende: Gij zult niet zeggen: Des HEEREN last;
\par 39 Daarom, ziet, Ik zal u ook ganselijk vergeten, en u, mitsgaders de stad, die Ik u en uw vaderen gegeven heb, van Mijn aangezicht laten varen.
\par 40 En Ik zal u eeuwige smaadheid aandoen, en eeuwige schande, die niet zal worden vergeten.

\chapter{24}

\par 1 De HEERE deed mij zien, en ziet, er waren twee vijgenkorven, gezet voor den tempel des HEEREN; nadat Nebukadnezar, koning van Babel, gevankelijk had weggevoerd Jechonia, den zoon van Jojakim, den koning van Juda, mitsgaders de vorsten van Juda, en de timmerlieden, en de smeden van Jeruzalem, en hen te Babel gebracht had.
\par 2 In den enen korf waren zeer goede vijgen, als de eerste rijpe vijgen zijn; maar in den anderen korf waren zeer boze vijgen, die vanwege de boosheid niet konden gegeten worden.
\par 3 En de HEERE zeide tot mij: Wat ziet gij, Jeremia? En ik zeide: Vijgen; de goede vijgen zijn zeer goed, en de boze zeer boos, die vanwege de boosheid niet kunnen gegeten worden.
\par 4 Toen geschiedde des HEEREN woord tot mij, zeggende:
\par 5 Zo zegt de HEERE, de God Israels: Gelijk die goede vijgen, alzo zal Ik kennen de gevankelijk weggevoerden van Juda, die Ik uit deze plaats naar het land der Chaldeen heb weggeschikt, ten goede.
\par 6 En Ik zal Mijn oog op hen stellen ten goede, en zal hen wederbrengen in dit land; en Ik zal hen bouwen, en niet afbreken; en zal hen planten, en niet uitrukken.
\par 7 En Ik zal hun een hart geven om Mij te kennen, dat Ik de HEERE ben; en zij zullen Mij tot een volk zijn, en Ik zal hun tot een God zijn; want zij zullen zich tot Mij met hun ganse hart bekeren.
\par 8 En gelijk de boze vijgen, die vanwege de boosheid niet kunnen gegeten worden (want aldus zegt de HEERE), alzo zal Ik maken Zedekia, den koning van Juda, mitsgaders zijn vorsten, en het overblijfsel van Jeruzalem, die in dit land zijn overgebleven, en die in Egypteland wonen;
\par 9 En Ik zal hen overgeven tot een beroering ten kwade, allen koninkrijken der aarde; tot smaadheid, en tot een spreekwoord, tot een spotrede, en tot een vloek, in al de plaatsen, waarhenen Ik hen gedreven zal hebben;
\par 10 En Ik zal onder hen zenden het zwaard, den honger en de pestilentie, totdat zij verteerd zullen zijn uit het land, dat Ik hun en hun vaderen gegeven had.

\chapter{25}

\par 1 Het woord, dat tot Jeremia geschied is over het ganse volk van Juda, in het vierde jaar van Jojakim, zoon van Josia, koning van Juda (dit was het eerste jaar van Nebukadnezar, koning van Babel);
\par 2 Hetwelk de profeet Jeremia gesproken heeft tot het ganse volk van Juda, en tot al de inwoners van Jeruzalem, zeggende:
\par 3 Van het dertiende jaar van Josia, den zoon van Amon, den koning van Juda, tot op dezen dag toe (dit is het drie en twintigste jaar) is het woord des HEEREN tot mij geschied; en ik heb tot ulieden gesproken, vroeg op zijnde en sprekende, maar gij hebt niet gehoord.
\par 4 Ook heeft de HEERE tot u gezonden al Zijn knechten, de profeten, vroeg op zijnde en zendende (maar gij hebt niet gehoord, noch uw oor geneigd om te horen);
\par 5 Zeggende: Bekeert u toch, een iegelijk van zijn bozen weg, en van de boosheid uwer handelingen, en woont in het land, dat de HEERE u en uw vaderen gegeven heeft, van eeuw tot in eeuw;
\par 6 En wandelt andere goden niet na, om die te dienen, en u voor die neder te buigen; en vertoornt Mij niet door uwer handen werk, opdat Ik u geen kwaad doe.
\par 7 Maar gij hebt naar Mij niet gehoord, spreekt de HEERE; opdat gij Mij vertoorndet door het werk uwer handen, u zelven ten kwade.
\par 8 Daarom, zo zegt de HEERE der heirscharen: Omdat gij Mijn woorden niet hebt gehoord;
\par 9 Ziet, Ik zal zenden, en nemen alle geslachten van het noorden, spreekt de HEERE; en tot Nebukadnezar, den koning van Babel, Mijn knecht; en zal ze brengen over dit land, en over de inwoners van hetzelve, en over al deze volken rondom; en Ik zal ze verbannen, en zal ze stellen tot een ontzetting, en tot een aanfluiting, en tot eeuwige woestheden.
\par 10 En Ik zal van hen doen vergaan de stem der vrolijkheid en de stem de vreugde, de stem des bruidegoms en de stem der bruid, het geluid der molens en het licht der lamp.
\par 11 En dit ganse land zal worden tot een woestheid, tot een ontzetting; en deze volken zullen den koning van Babel dienen zeventig jaren.
\par 12 Maar het zal geschieden, als de zeventig jaren vervuld zijn, dan zal Ik over den koning van Babel, en over dat volk, spreekt de HEERE, hun ongerechtigheid bezoeken, mitsgaders over het land der Chaldeen, en zal dat stellen tot eeuwige verwoestingen.
\par 13 En Ik zal over dat land brengen al Mijn woorden, die Ik daarover gesproken heb; al wat in dit boek geschreven is, wat Jeremia geprofeteerd heeft over al deze volken.
\par 14 Want van hen zullen zich doen dienen, die ook machtige volken en grote koningen zijn; alzo zal Ik hun vergelden naar hun doen, en naar het werk hunner handen.
\par 15 Want alzo heeft de HEERE, de God Israels, tot mij gezegd: Neem dezen beker des wijns der grimmigheid van Mijn hand, en geef dien te drinken al den volken, tot welke Ik u zende;
\par 16 Dat zij drinken, en beven, en dol worden, vanwege het zwaard, dat Ik onder hen zal zenden.
\par 17 En ik nam den beker van des HEEREN hand, en ik gaf te drinken al den volken, tot welke de HEERE mij gezonden had;
\par 18 Namelijk Jeruzalem en de steden van Juda, en haar koningen, en haar vorsten; om die te stellen tot een woestheid, tot een ontzetting, tot een aanfluiting en tot een vloek, gelijk het is te dezen dage;
\par 19 Farao, den koning van Egypte, en zijn knechten, en zijn vorsten, en al zijn volk;
\par 20 En den gansen gemengden hoop, en allen koningen des lands van Uz; en allen koningen van der Filistijnen land, en Askelon, en Gaza, en Ekron, en het overblijfsel van Asdod;
\par 21 Edom, en Moab, en den kinderen Ammons;
\par 22 En allen koningen van Tyrus, en allen koningen van Sidon; en den koningen der eilanden, die aan gene zijde der zee zijn.
\par 23 Dedan, en Thema, en Buz, en allen, die aan de hoeken afgekort zijn;
\par 24 En allen koningen van Arabie; en allen koningen des gemengden hoops, die in de woestijn wonen;
\par 25 En allen koningen van Zimri, en allen koningen van Elam, en allen koningen van Medie;
\par 26 En allen koningen van het noorden, die nabij en die verre zijn, den een met den anderen; ja, allen koninkrijken der aarde, die op den aardbodem zijn. En de koning van Sesach zal na hen drinken.
\par 27 Gij zult dan tot hen zeggen: Zo zegt de HEERE der heirscharen, de God Israels: Drinkt, en wordt dronken, en spuwt, en valt neder, dat gij niet weder opstaat, vanwege het zwaard, dat Ik onder u zal zenden.
\par 28 En het zal geschieden, wanneer zij weigeren zullen den beker van uw hand te nemen om te drinken, dat gij tot hen zeggen zult: Zo zegt de HEERE der heirscharen: Gij zult zekerlijk drinken!
\par 29 Want ziet, in de stad, die naar Mijn Naam genoemd is, begin Ik te plagen, en zoudt gij enigszins onschuldig gehouden worden? Gij zult niet onschuldig worden gehouden; want Ik roep het zwaard over alle inwoners der aarde, spreekt de HEERE der heirscharen.
\par 30 Gij zult dan al deze woorden tot hen profeteren, en gij zult tot hen zeggen: De HEERE zal brullen uit de hoogte, en Zijn stem verheffen uit de woning Zijner heiligheid; Hij zal schrikkelijk brullen over Zijn woonstede; Hij zal een vreugdegeschrei, als de druiven treders, uitroepen tegen alle inwoners der aarde.
\par 31 Het geschal zal komen tot aan het einde der aarde; want de HEERE heeft een twist met de volken, Hij zal gericht houden met alle vlees; de goddelozen heeft Hij aan het zwaard overgegeven, spreekt de HEERE.
\par 32 Zo zegt de HEERE der heirscharen: Ziet, een kwaad gaat er uit van volk tot volk. en een groot onweder zal er verwekt worden van de zijden der aarde.
\par 33 En de verslagenen des HEEREN zullen te dien dage liggen van het ene einde der aarde tot aan het andere einde der aarde; zij zullen niet beklaagd, noch opgenomen, noch begraven worden; tot mest op den aardbodem zullen zij zijn.
\par 34 Huilt, gij herders! en schreeuwt, en wentelt u in de as, gij heerlijken van de kudde! want uw dagen zijn vervuld, dat men slachten zal, en van uw verstrooiingen, dan zult gij vervallen als een kostelijk vat.
\par 35 En de vlucht zal vergaan van de herders, en de ontkoming van de heerlijken der kudde.
\par 36 Er zal zijn een stem des geroeps der herderen, en een gehuil der heerlijken van de kudde, omdat de HEERE hun weide verstoort.
\par 37 Want de landouwen des vredes zullen uitgeroeid worden, vanwege de hittigheid des toorns des HEEREN.
\par 38 Hij heeft, als een jonge leeuw, Zijn hutte verlaten; want hunlieder land is geworden tot een verwoesting, vanwege de hittigheid des verdrukkers, ja, vanwege de hittigheid Zijns toorns.

\chapter{26}

\par 1 In het begin des koninkrijks van Jojakim, den zoon van Josia, koning van Juda, geschiedde dit woord van den HEERE, zeggende:
\par 2 Zo zegt de HEERE: Sta in het voorhof van het huis des HEEREN, en spreek tot alle steden van Juda, die komen om aan te bidden in het huis des HEEREN, al de woorden, die Ik u geboden heb tot hen te spreken, doe er niet een woord af.
\par 3 Misschien zullen zij horen, en zich bekeren, een iegelijk van zijn bozen weg; zo zou Ik berouw hebben over het kwaad, dat Ik hun denk te doen vanwege de boosheid hunner handelingen.
\par 4 Zeg dan tot hen: Zo zegt de HEERE: Zo gijlieden naar Mij niet zult horen, dat gij wandelt in Mijn wet, die Ik voor uw aangezicht gegeven heb;
\par 5 Horende naar de woorden Mijner knechten, de profeten, die Ik tot u zende, zelfs vroeg op zijnde en zendende; doch gij niet gehoord hebt;
\par 6 Zo zal Ik dit huis stellen als Silo, en deze stad zal Ik stellen tot een vloek allen volken der aarde.
\par 7 En de priesters, en de profeten, en al het volk, hoorden Jeremia deze woorden spreken in het huis des HEEREN.
\par 8 Zo geschiedde het, als Jeremia geeindigd had te spreken alles, wat de HEERE geboden had tot al het volk te spreken, dat de priesters en de profeten en al het volk hem grepen, zeggende: Gij zult den dood sterven!
\par 9 Waarom hebt gij in den Naam des HEEREN geprofeteerd, zeggende: Dit huis zal worden als Silo, en deze stad zal woest worden, dat er niemand wone? En het ganse volk werd vergaderd tegen Jeremia, in het huis des HEEREN.
\par 10 Als nu de vorsten van Juda deze woorden hoorden, gingen zij op uit het huis des konings naar het huis des HEEREN; en zij zetten zich bij de deur der nieuwe poort des HEEREN.
\par 11 Toen spraken de priesters en de profeten tot de vorsten en tot al het volk, zeggende: Aan dezen man is een oordeel des doods, want hij heeft geprofeteerd tegen deze stad, gelijk als gij met uw oren gehoord hebt.
\par 12 Maar Jeremia sprak tot al de vorsten en tot al het volk, zeggende: De HEERE heeft mij gezonden, om tegen dit huis en tegen deze stad te profeteren al de woorden, die gij gehoord hebt;
\par 13 Nu dan, maakt uw wegen en uw handelingen goed, en gehoorzaamt de stem des HEEREN, uws Gods; zo zal het den HEERE berouwen over het kwaad, dat Hij tegen u gesproken heeft.
\par 14 Doch ik, ziet, ik ben in uw handen; doet mij, als het goed, en als het recht is in uw ogen;
\par 15 Maar weet voorzeker, dat gij, zo gij mij doodt, gewisselijk onschuldig bloed zult brengen op u, en op deze stad, en op haar inwoners; want in der waarheid, de HEERE heeft mij tot u gezonden, om al deze woorden voor uw oren te spreken.
\par 16 Toen zeiden de vorsten en al het volk tot de priesteren en tot de profeten: Aan dezen man is geen oordeel des doods, want hij heeft tot ons gesproken in den Naam des HEEREN, onzes Gods.
\par 17 Ook stonden er mannen op, van de oudsten des lands, en spraken tot de ganse gemeente des volks, zeggende:
\par 18 Micha, de Morastiet, heeft in de dagen van Hizkia, koning van Juda, geprofeteerd, en tot al het volk van Juda gesproken, zeggende: Zo zegt de HEERE des heirscharen: Sion zal als een akker geploegd, en Jeruzalem tot steen hopen worden, en de berg dezes huizes tot hoogten des wouds.
\par 19 Hebben ook Hizkia, de koning van Juda, en gans Juda hem ooit gedood? Vreesde hij niet den HEERE, en smeekte des HEEREN aangezicht, zodat het den HEERE berouwde over het kwaad, dat Hij tegen hen gesproken had? Wij dan doen een groot kwaad tegen onze zielen.
\par 20 Er was ook een man, die in den Naam des HEEREN profeteerde, Uria, de zoon van Semaja, van Kirjath-jearim; die profeteerde tegen deze stad en tegen dit land, naar al de woorden van Jeremia.
\par 21 En als de koning Jojakim, mitsgaders al zijn geweldigen, en al de vorsten zijn woorden hoorden, zocht de koning hem te doden; als Uria dat hoorde, zo vreesde hij, en vluchtte, en kwam in Egypte;
\par 22 Maar de koning Jojakim zond mannen naar Egypte, Elnathan, den zoon van Achbor, en andere mannen met hem, in Egypte;
\par 23 Die voerden Uria uit Egypte, en brachten hem tot den koning Jojakim, en hij sloeg hem met het zwaard, en hij wierp zijn dood lichaam in de graven van de kinderen des volks.
\par 24 Maar de hand van Ahikam, den zoon van Safan, was met Jeremia, dat men hem niet overgaf in de hand des volk, om hem te doden.

\chapter{27}

\par 1 In het begin des koninkrijks van Jojakim, zoon van Josia, koning van Juda, geschiedde dit woord tot Jeremia, van den HEERE, zeggende:
\par 2 Alzo zeide de HEERE tot mij: Maak u banden en jukken, en doe die aan uw hals.
\par 3 En zend ze tot den koning van Edom, en tot den koning van Moab, en tot den koning der kinderen Ammons, en tot den koning van Tyrus, en tot den koning van Sidon; door de hand der boden, die te Jeruzalem tot Zedekia, den koning van Juda, komen.
\par 4 En beveel hun aan hun heren te zeggen: Zo zegt de HEERE der heirscharen, de God Israels: Zo zult gij tot uw heren zeggen:
\par 5 Ik heb gemaakt de aarde, den mens en het vee, die op den aardbodem zijn, door Mijn grote kracht, en door Mijn uitgestrekten arm, en Ik geef ze aan welken het recht is in Mijn ogen.
\par 6 En nu, Ik heb al deze landen gegeven in de hand van Nebukadnezar, den koning van Babel, Mijn knecht; zelfs ook het gedierte des velds heb Ik hem gegeven, om hem te dienen.
\par 7 En alle volken zullen hem, en zijn zoon, en zijns zoons zoon dienen, totdat ook de tijd zijns eigenen lands kome; dan zullen zich machtige volken en grote koningen van hem doen dienen.
\par 8 En het zal geschieden, het volk en het koninkrijk, dat hem, Nebukadnezar, den koning van Babel, niet zal dienen, en dat zijn hals niet zal geven onder het juk des konings van Babel; over datzelve volk zal Ik, spreekt de HEERE, bezoeking doen door het zwaard, en door den honger, en door de pestilentie, totdat Ik ze zal verteerd hebben door zijn hand.
\par 9 Gijlieden dan, hoort niet naar uw profeten, en naar uw waarzeggers, en naar uw dromers, en naar uw guichelaars, en naar uw tovenaars, dewelke tot u spreken, zeggende: Gij zult den koning van Babel niet dienen.
\par 10 Want zij profeteren u valsheid, om u verre uit uw land te brengen, en dat Ik u uitstote, en gij omkomt.
\par 11 Maar het volk, dat zijn hals zal brengen onder het juk des konings van Babel, en hem dienen, datzelve zal Ik in zijn land laten, spreekt de HEERE, en het zal dat bouwen en daarin wonen.
\par 12 Daarna sprak ik tot Zedekia, den koning van Juda, naar al deze woorden, zeggende: Brengt uw halzen onder het juk des konings van Babel, en dient hem en zijn volk, zo zult gij leven.
\par 13 Waarom zoudt gij sterven, gij en uw volk door het zwaard, door den honger en door de pestilentie, gelijk als de HEERE gesproken heeft van het volk, dat den koning van Babel niet zal dienen.
\par 14 Hoort dan niet naar de woorden der profeten, die tot u spreken, zeggende: Gij zult den koning van Babel niet dienen; want zij profeteren u valsheid.
\par 15 Want Ik heb ze niet gezonden, spreekt de HEERE, en zij profeteren valselijk in Mijn Naam; opdat Ik u uitstote, en gij omkomt, gij en de profeten, die u profeteren.
\par 16 Ook sprak ik tot de priesteren, en tot dit ganse volk, zeggende: Zo zegt de HEERE: Hoort niet naar de woorden uwer profeten, die u profeteren, zeggende: Ziet, de vaten van des HEEREN huis zullen nu haast uit Babel wedergebracht worden; want zij profeteren u valsheid.
\par 17 Hoort niet naar hen, maar dient den koning van Babel, zo zult gijlieden leven; waarom zou deze stad tot een woestheid worden?
\par 18 Maar zo zij profeten zijn, en zo des HEEREN woord bij hen is, laat hen nu bij den HEERE der heirscharen voorbidden, opdat de vaten, die in het huis des HEEREN, en in het huis des konings van Juda, en te Jeruzalem zijn overgebleven, niet naar Babel komen.
\par 19 Want zo zegt de HEERE der heirscharen, van de pilaren, en van de zee, en van de stellingen, en van het overige der vaten, die in deze stad zijn overgebleven.
\par 20 Die Nebukadnezar, de koning van Babel, niet heeft weggenomen, als hij Jechonia, den zoon van Jojakim, koning van Juda, van Jeruzalem, naar Babel gevankelijk wegvoerde, mitsgaders al de edelen van Juda en Jeruzalem;
\par 21 Ja, zo zegt de HEERE der heirscharen, de God Israels, van de vaten, die in het huis des HEEREN, en in het huis des konings van Juda, en te Jeruzalem zijn overgebleven:
\par 22 Naar Babel zullen zij gebracht worden, en aldaar zullen zij zijn, tot den dag toe, dat Ik ze bezoeken zal, spreekt de HEERE; dan zal Ik ze opvoeren, en zal ze wederbrengen tot deze plaats.

\chapter{28}

\par 1 Voorts geschiedde het in hetzelfde jaar, in het begin des koninkrijks van Zedekia, koning van Juda, in het vierde jaar, in de vijfde maand, dat Hananja, zoon van Azur, de profeet, die van Gibeon was, tot mij sprak, in het huis des HEEREN, voor de ogen der priesteren en des gansen volks, zeggende:
\par 2 Zo spreekt de HEERE der heirscharen, de God Israels, zeggende: Ik heb het juk des konings van Babel verbroken.
\par 3 In nog twee volle jaren zal Ik tot deze plaats wederbrengen al de vaten van het huis des HEEREN, die Nebukadnezar, de koning van Babel, uit deze plaats heeft weggenomen, en dezelve naar Babel gebracht.
\par 4 Ook zal Ik Jechonia, den zoon van Jojakim, koning van Juda, en allen, die gevankelijk weggevoerd zijn van Juda, die te Babel gekomen zijn, tot deze plaats wederbrengen, spreekt de HEERE; want Ik zal het juk des konings van Babel verbreken.
\par 5 Toen sprak de profeet Jeremia tot den profeet Hananja, voor de ogen der priesteren, en voor de ogen des gansen volks, die in het huis des HEEREN stonden;
\par 6 En de profeet Jeremia zeide: Amen, de HEERE doe alzo! de HEERE bevestige uw woorden, die gij geprofeteerd hebt, dat Hij de vaten van des HEEREN huis, en allen, die gevankelijk zijn weggevoerd, van Babel wederbrenge tot deze plaats!
\par 7 Maar hoor nu dit woord, dat ik spreek voor uw oren, en voor de oren des gansen volks:
\par 8 De profeten, die voor mij en voor u van ouds geweest zijn, die hebben tegen veel landen en tegen grote koninkrijken geprofeteerd, van krijg, en van kwaad, en van pestilentie.
\par 9 De profeet, die geprofeteerd zal hebben van vrede, als het woord van dien profeet komt, dan zal die profeet bekend worden, dat hem de HEERE in der waarheid gezonden heeft.
\par 10 Toen nam de profeet Hananja het juk van den hals van den profeet Jeremia, en verbrak het.
\par 11 En Hananja sprak voor de ogen des gansen volks, zeggende: Zo zegt de HEERE: Alzo zal Ik verbreken het juk van Nebukadnezar, den koning van Babel, in nog twee volle jaren, van den hals al der volken. En de profeet Jeremia ging zijns weegs.
\par 12 Doch des HEEREN woord geschiedde tot Jeremia (nadat de profeet Hananja het juk van den hals van den profeet Jeremia verbroken had), zeggende:
\par 13 Ga henen en spreek tot Hananja, zeggende: Zo zegt de HEERE: Houten jukken hebt gij verbroken, nu zult gij in plaats van die, ijzeren jukken maken.
\par 14 Want zo zegt de HEERE der heirscharen, de God Israels: Ik heb een ijzeren juk gedaan aan den hals van al deze volken, om Nebukadnezar, den koning van Babel, te dienen, en zij zullen hem dienen; ja, Ik heb hem ook het gedierte des velds gegeven.
\par 15 En de profeet Jeremia zeide tot den profeet Hananja: Hoor nu, Hananja! de HEERE heeft u niet gezonden, maar gij hebt gemaakt, dat dit volk op leugen vertrouwt.
\par 16 Daarom, zo zegt de HEERE: Zie, Ik zal u wegwerpen van den aardbodem; dit jaar zult gij sterven, omdat gij een afval gesproken hebt tegen den HEERE.
\par 17 Alzo stierf de profeet Hananja in datzelfde jaar, in de zevende maand.

\chapter{29}

\par 1 Voorts zijn dit de woorden des briefs, dien de profeet Jeremia zond van Jeruzalem tot de overige oudsten, die gevankelijk waren weggevoerd, mitsgaders tot de priesteren, en tot de profeten, en tot het ganse volk, dat Nebukadnezar van Jeruzalem gevankelijk had weggevoerd naar Babel.
\par 2 (Nadat de koning Jechonia, en de koningin, en de kamerlingen, de vorsten van Juda en Jeruzalem, mitsgaders de timmerlieden en smeden van Jeruzalem waren uitgegaan);
\par 3 Door de hand van Elasa, den zoon van Safan, en Gemarja, den zoon van Hilkia, die Zedekia, de koning van Juda, naar Babel zond, tot Nebukadnezar, den koning van Babel, zeggende:
\par 4 Zo zegt de HEERE der heirscharen, de God Israels, tot allen, die gevankelijk zijn weggevoerd, die Ik gevankelijk heb doen wegvoeren van Jeruzalem naar Babel:
\par 5 Bouwt huizen en woont daarin, en plant hoven en eet de vrucht daarvan;
\par 6 Neemt vrouwen, en gewint zonen en dochteren, en neemt vrouwen voor uw zonen, en geeft uw dochteren aan mannen, dat zij zonen en dochteren baren; en wordt aldaar vermenigvuldigd, en wordt niet verminderd.
\par 7 En zoekt den vrede der stad, waarhenen Ik u gevankelijk heb doen wegvoeren, en bidt voor haar tot den HEERE; want in haar vrede zult gij vrede hebben.
\par 8 Want zo zegt de HEERE der heirscharen, de God Israels: Laat uw profeten en uw waarzeggers, die in het midden van u zijn, u niet bedriegen, en hoort niet naar uw dromers, die gij doet dromen.
\par 9 Want zij profeteren u valselijk in Mijn Naam; Ik heb hen niet gezonden, spreekt de HEERE.
\par 10 Want zo zegt de HEERE: Zekerlijk, als zeventig jaren te Babel zullen vervuld zijn, zal Ik ulieden bezoeken, en Ik zal Mijn goed woord over u verwekken, u wederbrengende tot deze plaats.
\par 11 Want Ik weet de gedachten, die Ik over u denk, spreekt de HEERE, gedachten des vredes, en niet des kwaads, dat Ik u geve het einde en de verwachting.
\par 12 Dan zult gij Mij aanroepen, en henengaan, en tot Mij bidden; en Ik zal naar u horen.
\par 13 En gij zult Mij zoeken en vinden, wanneer gij naar Mij zult vragen met uw ganse hart.
\par 14 En Ik zal van ulieden gevonden worden, spreekt de HEERE, en Ik zal uw gevangenis wenden, en u vergaderen uit al de volken, en uit al de plaatsen, waarhenen Ik u gedreven heb, spreekt de HEERE; en Ik zal u wederbrengen tot de plaats, van waar Ik u gevankelijk heb doen wegvoeren.
\par 15 Omdat gij zegt: de HEERE heeft ons profeten naar Babel verwekt;
\par 16 Daarom zegt de HEERE alzo van den koning, die op Davids troon zit, en van al het volk, dat in deze stad woont, te weten, uw broederen, die met u niet zijn uitgegaan in de gevangenis;
\par 17 Alzo zegt de HEERE der heirscharen: Ziet, Ik zal het zwaard, den honger en de pestilentie onder hen zenden; en Ik zal ze maken als de afschuwelijke vijgen, die vanwege de boosheid niet kunnen gegeten worden.
\par 18 En Ik zal ze achterna jagen met het zwaard, met den honger en met de pestilentie; en Ik zal ze overgeven tot een beroering, allen koninkrijken der aarde, tot een vloek, en tot een schrik, en tot een aanfluiting, en tot een smaadheid, onder al de volken, waar Ik ze henengedreven zal hebben;
\par 19 Omdat zij naar Mijn woorden niet gehoord hebben, spreekt de HEERE, als Ik Mijn knechten, de profeten, tot hen zond, vroeg op zijnde en zendende; maar gijlieden hebt niet gehoord, spreekt de HEERE.
\par 20 Gij dan, hoort des HEEREN woord, gij allen, die gevankelijk zijt weggevoerd, die Ik van Jeruzalem naar Babel heb weggezonden!
\par 21 Zo zegt de HEERE der heirscharen, de God Israels, van Achab, zoon van Kolaja, en van Zedekia, zoon van Maaseja, die ulieden in Mijn Naam valselijk profeteren: Ziet, Ik zal hen geven in de hand van Nebukadnezar, den koning van Babel, en hij zal ze voor uw ogen slaan.
\par 22 En van hen zal een vloek genomen worden bij al de gevankelijk weggevoerden van Juda, die in Babel zijn, dat men zegge: De HEERE stelle u als Zedekia, en als Achab, die de koning van Babel aan het vuur braadde;
\par 23 Omdat zij een dwaasheid deden in Israel, en overspel bedreven met de vrouwen hunner naasten, en spraken het woord valselijk in Mijn Naam, dat Ik hun niet geboden had; en Ik ben Degene, Die het weet, en een getuige daarvan, spreekt de HEERE.
\par 24 Tot Semaja nu, den Nechlamiet, zult gij spreken, zeggende:
\par 25 Zo spreekt de HEERE der heirscharen, de God Israels, zeggende: Omdat gij brieven in uw naam gezonden hebt tot al het volk, dat te Jeruzalem is, en tot Zefanja, den zoon van Maaseja, den priester, en tot al de priesteren, zeggende:
\par 26 De HEERE heeft u tot priester gesteld, in plaats van den priester Jojada, dat gij opzieners zoudt zijn in des HEEREN huis over allen man, die onzinnig is, en zich voor een profeet uitgeeft, dat gij dien stelt in de gevangenis en in den stok.
\par 27 Nu dan, waarom hebt gij Jeremia, den Anathothiet, niet gescholden, die zich bij ulieden voor een profeet uitgeeft?
\par 28 Want daarom heeft hij tot ons naar Babel gezonden, zeggende: Het zal lang duren; bouwt huizen, en woont daarin, en plant hoven, en eet de vrucht daarvan.
\par 29 Zefanja nu, de priester, had dezen brief gelezen voor de oren van den profeet Jeremia.
\par 30 Daarom geschiedde des HEEREN woord tot Jeremia, zeggende:
\par 31 Zend henen tot allen, die gevankelijk weggevoerd zijn, zeggende: Zo zegt de HEERE van Semaja, den Nechlamiet: Omdat Semaja ulieden geprofeteerd heeft, daar Ik hem niet gezonden heb, en heeft gemaakt, dat gij op leugen vertrouwt;
\par 32 Daarom zegt de HEERE alzo: Ziet, Ik zal bezoeking doen over Semaja, den Nechlamiet, en over zijn zaad; hij zal niemand hebben, die in het midden dezes volks wone, en zal het goede niet zien, dat Ik Mijn volke doen zal, spreekt de HEERE; want hij heeft een afval gesproken tegen den HEERE.

\chapter{30}

\par 1 Het woord, dat tot Jeremia geschied is van den HEERE, zeggende:
\par 2 Zo spreekt de HEERE, de God Israels, zeggende: Schrijf u al de woorden, die Ik tot u gesproken heb, in een boek.
\par 3 Want zie, de dagen komen, spreekt de HEERE, dat Ik de gevangenis van Mijn volk, Israel en Juda, wenden zal, zegt de HEERE; en Ik zal hen wederbrengen in het land, dat Ik hun vaderen gegeven heb, en zij zullen het erfelijk bezitten.
\par 4 En dit zijn de woorden, die de HEERE gesproken heeft van Israel en van Juda.
\par 5 Want zo zegt de HEERE: Wij horen een stem der verschrikking; er is vrees en geen vrede.
\par 6 Vraagt toch en ziet, of een manspersoon baart? Waarom zie Ik dan eens iegelijken mans handen op zijn lenden, als van een barende vrouw, en alle aangezichten veranderd in bleekheid?
\par 7 O wee! want die dag is zo groot, dat zijns gelijke niet geweest is; en het is een tijd van benauwdheid voor Jakob; nog zal hij daaruit verlost worden.
\par 8 Want het zal te dien dage geschieden, spreekt de HEERE der heirscharen, dat Ik zijn juk van uw hals verbreken, en uw banden verscheuren zal; en vreemden zullen zich niet meer van hem doen dienen.
\par 9 Maar zij zullen dienen den HEERE, hun God, en hun koning David, dien Ik hun verwekken zal.
\par 10 Gij dan, vrees niet, o Mijn knecht Jakob! spreekt de HEERE, ontzet u niet, Israel! want zie, Ik zal u uit verre landen verlossen, en uw zaad uit het land hunner gevangenis; en Jakob zal wederkomen, en stil en gerust zijn, en er zal niemand zijn, die hem verschrikke.
\par 11 Want Ik ben met u, spreekt de HEERE, om u te verlossen; want Ik zal een voleinding maken met al de heidenen, waarhenen Ik u verstrooid heb; maar met u zal Ik geen voleinding maken; maar Ik zal u kastijden met mate, en u niet gans onschuldig houden.
\par 12 Want zo zegt de HEERE: Uw breuk is dodelijk, uw plage is smartelijk.
\par 13 Er is niemand, die uw zaak oordeelt, aangaande het gezwel; gij hebt geen heelpleisters.
\par 14 Al uw liefhebbers hebben u vergeten, zij vragen niet naar u; want Ik heb u geslagen met eens vijands plage, met de kastijding eens wreden; om de grootheid uwer ongerechtigheid, omdat uw zonden machtig veel zijn.
\par 15 Wat krijt gij over uw breuk, dat uw smart dodelijk is? Om de grootheid uwer ongerechtigheid, omdat uw zonden machtig veel zijn, heb Ik u deze dingen gedaan.
\par 16 Daarom, allen, die u opeten, zullen opgegeten worden, en al uw wederpartijders, zij allen zullen gaan in gevangenis; en die u beroven, zullen ter beroving zijn, en allen, die u plunderen, zal Ik ter plundering overgeven.
\par 17 Want Ik zal u de gezondheid doen rijzen, en u van uw plagen genezen, spreekt de HEERE; omdat zij u noemen: De verdrevene. Het is Sion, zeggen zij; niemand vraagt naar haar.
\par 18 Zo zegt de HEERE: Ziet, Ik zal de gevangenis der tenten Jakobs wenden, en Mij over hun woningen ontfermen; en de stad zal herbouwd worden op haar hoop, en het paleis zal liggen naar zijn wijze.
\par 19 En van hen zal dankzegging uitgaan, en een stem der spelenden; en Ik zal hen vermeerderen, en zij zullen niet verminderd worden, en Ik zal hen verheerlijken, en zij zullen niet gering worden.
\par 20 En zijn zonen zullen zijn als eertijds, en zijn gemeente zal voor Mijn aangezicht bevestigd worden; en Ik zal bezoeking doen over al zijn onderdrukkers.
\par 21 En zijn Heerlijke zal uit hem zijn, en zijn Heerser uit het midden van hem voortkomen; en Ik zal hem doen naderen, en hij zal tot Mij genaken; want wie is hij, die met zijn hart borg worde, om tot Mij te genaken? spreekt de HEERE.
\par 22 En gij zult Mij tot een volk zijn, en Ik zal u tot een God zijn.
\par 23 Ziet, een onweder des HEEREN, een grimmigheid is uitgegaan, een aanhoudend onweder; het zal blijven op het hoofd der goddelozen.
\par 24 De hittigheid van des HEEREN toorn zal zich niet afwenden, totdat Hij gedaan, en totdat Hij daargesteld zal hebben de gedachten Zijns harten; in het laatste der dagen zult gij daarop letten.

\chapter{31}

\par 1 Ter zelfder tijd, spreekt de HEERE, zal Ik allen geslachten Israels tot een God zijn; en zij zullen Mij tot een volk zijn.
\par 2 Zo zegt de HEERE: Het volk der overgeblevenen van het zwaard heeft genade gevonden in de woestijn, namelijk Israel, als Ik henenging om hem tot rust te brengen.
\par 3 De HEERE is mij verschenen van verre tijden! Ja, Ik heb u liefgehad met een eeuwige liefde; daarom heb Ik u getrokken met goedertierenheid.
\par 4 Ik zal u weder bouwen, en gij zult gebouwd worden, o jonkvrouw Israels! gij zult weder versierd zijn met uw trommelen, en uitgaan met den rei der spelenden.
\par 5 Gij zult weder wijngaarden planten op de bergen van Samaria; de planters zullen planten, en de vrucht genieten.
\par 6 Want er zal een dag zijn, waarin de hoeders op Efraims gebergte zullen roepen: Maakt ulieden op, en laat ons opgaan naar Sion, tot den HEERE, onzen God!
\par 7 Want zo zegt de HEERE: Roept luide over Jakob met vreugde, en juicht vanwege het hoofd der heidenen; doet het horen, lofzingt, en zegt: O HEERE! behoud Uw volk, het overblijfsel van Israel.
\par 8 Ziet, Ik zal ze aanbrengen uit het land van het noorden, en zal hen vergaderen van de zijden der aarde; onder hen zullen zijn blinden en lammen, zwangeren en barenden te zamen; met een grote gemeente zullen zij herwaarts wederkomen.
\par 9 Zij zullen komen met geween, en met smekingen zal Ik hen voeren; Ik zal hen leiden aan de waterbeken, in een rechte weg, waarin zij zich niet zullen stoten; want Ik ben Israel tot een Vader, en Efraim is Mijn eerstgeborene.
\par 10 Hoort des HEEREN woord, gij heidenen! en verkondigt in de eilanden, die verre zijn, en zegt: Hij, Die Israel verstrooid heeft, zal hem weder vergaderen, en hem bewaren als een herder zijn kudde.
\par 11 Want de HEERE heeft Jakob vrijgekocht, en Hij heeft hem verlost uit de hand desgenen, die sterker was dan hij.
\par 12 Dies zullen zij komen, en op de hoogte van Sion juichen, en toevloeien tot des HEEREN goed, tot het koren, en tot den most, en tot de olie, en tot de jonge schapen en runderen; en hun ziel zal zijn als een gewaterde hof, en zij zullen voortaan niet meer treurig zijn.
\par 13 Dan zal zich de jonkvrouw verblijden in den rei, daartoe de jongelingen en ouden te zamen; want Ik zal hunlieder rouw in vrolijkheid veranderen, en zal hen troosten, en zal hen verblijden naar hun droefenis.
\par 14 En Ik zal de ziel der priesteren met vettigheid dronken maken; en Mijn volk zal met Mijn goed verzadigd worden, spreekt de HEERE.
\par 15 Zo zegt de HEERE: Er is een stem gehoord in Rama, een klage, een zeer bitter geween; Rachel weent over haar kinderen; zij weigert zich te laten troosten over haar kinderen, omdat zij niet zijn.
\par 16 Zo zegt de HEERE: Bedwing uw stem van geween, en uw ogen van tranen; want er is loon voor uw arbeid, spreekt de HEERE; want zij zullen uit des vijands land wederkomen.
\par 17 En er is verwachting voor uw nakomelingen, spreekt de HEERE; want uw kinderen zullen wederkomen tot hun landpale.
\par 18 Ik heb wel gehoord, dat zich Efraim beklaagt, zeggende: Gij hebt mij getuchtigd, en ik ben getuchtigd geworden als een ongewend kalf. Bekeer mij, zo zal ik bekeerd zijn, want Gij zijt de HEERE, mijn God!
\par 19 Zekerlijk, nadat ik bekeerd ben, heb ik berouw gehad, en nadat ik mijzelven ben bekend gemaakt, heb ik op de heup geklopt, ik ben beschaamd, ja, ook schaamrood geworden, omdat ik de smaadheid mijner jeugd gedragen heb.
\par 20 Is niet Efraim Mij een dierbare zoon, is hij Mij niet een troetelkind? Want sinds Ik tegen hem gesproken heb, denk Ik nog ernstelijk aan hem; daarom rommelt Mijn ingewand over hem; Ik zal Mij zijner zekerlijk ontfermen, spreekt de HEERE.
\par 21 Richt u merktekenen op, stel u spitse pilaren, zet uw hart op de baan, op den weg, dien gij gewandeld hebt; keer weder, o jonkvrouw Israels, keer weder tot deze uw steden!
\par 22 Hoe lang zult gij u onttrekken, gij afkerige dochter? Want de HEERE heeft wat nieuws op de aarde geschapen: de vrouw zal den man omvangen.
\par 23 Zo zegt de HEERE der heirscharen, de God Israels: Dit woord zullen zij nog zeggen in het land van Juda, en in zijn steden, als Ik hun gevangenis wenden zal: De HEERE zegene u, gij woning der gerechtigheid, gij berg der heiligheid!
\par 24 En Juda, mitsgaders al zijn steden, zullen te zamen daarin wonen; de akkerlieden, en die met de kudde reizen.
\par 25 Want Ik heb de vermoeide ziel dronken gemaakt, en Ik heb alle treurige ziel vervuld.
\par 26 (Hierop ontwaakte ik, en zag toe, en mijn slaap was mij zoet.)
\par 27 Ziet, de dagen komen, spreekt de HEERE, dat Ik het huis van Israel en het huis van Juda bezaaien zal met zaad van mensen en zaad van beesten.
\par 28 En het zal geschieden, gelijk als Ik over hen gewaakt heb, om uit te rukken, en af te breken, en te verstoren, en te verderven, en kwaad aan te doen; alzo zal Ik over hen waken, om te bouwen en te planten, spreekt de HEERE.
\par 29 In die dagen zullen zij niet meer zeggen: De vaders hebben onrijpe druiven gegeten, en der kinderen tanden zijn stomp geworden.
\par 30 Maar een iegelijk zal om zijn ongerechtigheid sterven; een ieder mens, die de onrijpe druiven eet, zijn tanden zullen stomp worden.
\par 31 Ziet, de dagen komen, spreekt de HEERE, dat Ik met het huis van Israel en met het huis van Juda een nieuw verbond zal maken;
\par 32 Niet naar het verbond, dat Ik met hun vaderen gemaakt heb, ten dage als Ik hun hand aangreep, om hen uit Egypteland uit te voeren, welk Mijn verbond zij vernietigd hebben, hoewel Ik hen getrouwd had, spreekt de HEERE;
\par 33 Maar dit is het verbond, dat Ik na die dagen met het huis van Israel maken zal, spreekt de HEERE: Ik zal Mijn wet in hun binnenste geven, en zal die in hun hart schrijven; en Ik zal hun tot een God zijn, en zij zullen Mij tot een volk zijn.
\par 34 En zij zullen niet meer, een iegelijk zijn naaste, en een iegelijk zijn broeder, leren, zeggende: Kent den HEERE! want zij zullen Mij allen kennen, van hun kleinste af tot hun grootste toe, spreekt de HEERE; want Ik zal hun ongerechtigheid vergeven, en hunner zonden niet meer gedenken.
\par 35 Zo zegt de HEERE, Die de zon ten lichte geeft des daags, de ordeningen der maan en der sterren ten lichte des nachts, Die de zee klieft, dat haar golven bruisen, HEERE der heirscharen is Zijn Naam:
\par 36 Indien deze ordeningen van voor Mijn aangezicht zullen wijken, spreekt de HEERE, zo zal ook het zaad Israels ophouden, dat het geen volk zij voor Mijn aangezicht, al de dagen.
\par 37 Zo zegt de HEERE: Indien de hemelen daarboven gemeten, en de fondamenten der aarde beneden doorgrond kunnen worden, zo zal Ik ook het ganse zaad Israels verwerpen, om alles, wat zij gedaan hebben, spreekt de HEERE.
\par 38 Ziet, de dagen komen, spreekt de HEERE, dat deze stad den HEERE zal herbouwd worden, van den toren Hananeel af tot aan de Hoekpoort.
\par 39 En het meetsnoer zal wijders nevens dezelve uitgaan tot aan den heuvel Gareb, en zich naar Goath omwenden.
\par 40 En het ganse dal der dode lichamen en der as, en al de velden tot aan de beek Kidron, tot aan den hoek van de Paardenpoort tegen het oosten, zal den HEERE een heiligheid zijn; er zal niets weder uitgerukt, noch afgebroken worden in eeuwigheid.

\chapter{32}

\par 1 Het woord, dat tot Jeremia geschied is van den HEERE, in het tiende jaar van Zedekia, koning van Juda; dit jaar was het achttiende jaar van Nebukadnezar.
\par 2 (Het heir nu des konings van Babel belegerde toen Jeruzalem, en de profeet Jeremia was besloten in het voorhof der bewaring, dat in het huis des konings van Juda is.
\par 3 Want Zedekia, de koning van Juda, had hem besloten, zeggende: Waarom profeteert gij, zeggende: Zo zegt de HEERE: Ziet, Ik geef deze stad in de hand des konings van Babel, en hij zal ze innemen;
\par 4 En Zedekia, de koning van Juda, zal van de hand der Chaldeen niet ontkomen; maar hij zal zekerlijk gegeven worden in de hand des konings van Babel, en zijn mond zal tot deszelfs mond spreken, en zijn ogen zullen deszelfs ogen zien;
\par 5 En hij zal Zedekia naar Babel voeren, en aldaar zal hij zijn, totdat Ik hem bezoek, spreekt de HEERE; ofschoon gijlieden tegen de Chaldeen strijdt, gij zult toch geen geluk hebben.)
\par 6 Jeremia dan zeide: Des HEEREN woord is tot mij geschied, zeggende:
\par 7 Zie, Hanameel, de zoon van Sallum, uw oom, zal tot u komen, zeggende: Koop u mijn veld, dat bij Anathoth is, want gij hebt het recht van lossing, om te kopen.
\par 8 Alzo kwam Hanameel, mijns ooms zoon, naar des HEEREN woord, tot mij, in het voorhof der bewaring, en zeide tot mij: Koop toch mijn veld, hetwelk is bij Anathoth, dat in het land van Benjamin is; want gij hebt het erfrecht, en gij hebt de lossing, koop het voor u. Toen merkte ik, dat het des HEEREN woord was.
\par 9 Dies kocht ik van Hanameel, mijns ooms zoon, het veld, dat bij Anathoth is; en ik woog hem het geld toe, zeventien zilveren sikkelen.
\par 10 En ik onderschreef den brief en verzegelde dien, en deed het getuigen betuigen, als ik het geld op de weegschaal gewogen had.
\par 11 En ik nam den koopbrief, die verzegeld was naar het gebod en de inzettingen, en den open brief;
\par 12 En ik gaf den koopbrief aan Baruch, den zoon van Nerija, den zoon van Machseja, voor de ogen van Hanameel, mijns ooms zoon, en voor de ogen der getuigen die den koopbrief hadden onderschreven; voor de ogen van al de Joden, die in het voorhof der bewaring zaten.
\par 13 En ik beval Baruch voor hun ogen, zeggende:
\par 14 Zo zegt de HEERE der heirscharen, de God Israels: Neem deze brieven, dezen koopbrief, zo den verzegelden als dezen open brief, en doe ze in een aarden vat, opdat zij vele dagen mogen bestaan.
\par 15 Want zo zegt de HEERE der heirscharen, de God Israels: Er zullen nog huizen, en velden, en wijngaarden in dit land gekocht worden.
\par 16 Voorts, nadat ik den koopbrief aan Baruch, den zoon van Nerija, gegeven had, bad ik tot den HEERE, zeggende:
\par 17 Ach, Heere HEERE! Zie, Gij hebt de hemelen en de aarde gemaakt, door Uw grote kracht en door Uw uitgestrekten arm; geen ding is U te wonderlijk.
\par 18 Gij, Die goedertierenheid doet aan duizenden, en de ongerechtigheid der vaderen vergeldt in den schoot hunner kinderen na hen; Gij grote, Gij geweldige God, Wiens Naam is HEERE der heirscharen!
\par 19 Groot van raad en machtig van daad; want Uw ogen zijn open over alle wegen der mensenkinderen, om een iegelijk te geven naar zijn wegen, en naar de vrucht zijner handelingen.
\par 20 Gij, Die tekenen en wonderen gesteld hebt in Egypteland, tot op dezen dag, zo in Israel, als onder andere mensen, en hebt U een Naam gemaakt, als Hij is te dezen dage!
\par 21 En hebt Uw volk Israel uit Egypteland uitgevoerd, door tekenen en door wonderen, en door een sterke hand, en door een uitgestrekten arm, en door grote verschrikking.
\par 22 En hebt hun dit land gegeven, dat Gij hun vaderen gezworen hadt hun te zullen geven, een land vloeiende van melk en honig;
\par 23 Zij zijn er ook ingekomen en hebben het erfelijk bezeten, maar hebben Uwer stem niet gehoorzaamd, en in Uw wet niet gewandeld; zij hebben niets gedaan van alles, wat Gij hun geboden hadt te doen; dies hebt Gij hun al dit kwaad doen bejegenen.
\par 24 Zie, de wallen! zij zijn gekomen aan de stad, om die in te nemen, en de stad is gegeven in de hand der Chaldeen, die tegen haar strijden; vanwege het zwaard en den honger en de pestilentie; en wat Gij gesproken hebt, is geschied, en zie, Gij ziet het.
\par 25 Evenwel hebt Gij tot mij gezegd, Heere HEERE! koop u dat veld voor geld, en doe het getuigen betuigen; daar de stad in der Chaldeen hand gegeven is.
\par 26 Toen geschiedde des HEEREN woord tot Jeremia, zeggende:
\par 27 Zie, Ik ben de HEERE, de God van alle vlees; zou Mij enig ding te wonderlijk zijn?
\par 28 Daarom zegt de HEERE alzo: Zie, Ik geef deze stad in de hand der Chaldeen, en in de hand van Nebukadnezar, den koning van Babel, en hij zal ze innemen.
\par 29 En de Chaldeen, die tegen deze stad strijden, zullen er inkomen, en deze stad met vuur aansteken, en zullen ze verbranden, met de huizen, op welker daken zij aan Baal gerookt, en anderen goden drankofferen geofferd hebben, om Mij te vertoornen.
\par 30 Want de kinderen Israels en de kinderen van Juda hebben van hun jeugd aan alleenlijk gedaan, dat kwaad was in Mijn ogen; want de kinderen Israels hebben Mij door het werk hunner handen alleenlijk vertoornd, spreekt de HEERE.
\par 31 Want tot Mijn toorn en tot Mijn grimmigheid is Mij deze stad geweest, van den dag af, dat zij haar gebouwd hebben, tot op dezen dag toe; opdat Ik haar van Mijn aangezicht wegdeed;
\par 32 Om al de boosheid der kinderen Israels en der kinderen van Juda, die zij gedaan hebben om Mij te vertoornen, zij, hun koningen, hun vorsten, hun priesteren, en hun profeten, en de mannen van Juda, en de inwoners van Jeruzalem;
\par 33 Die Mij den nek hebben toegekeerd en niet het aangezicht; hoewel Ik hen leerde, vroeg op zijnde en lerende, evenwel hoorden zij niet, om tucht aan te nemen;
\par 34 Maar zij hebben hun verfoeiselen gesteld in het huis, dat naar Mijn Naam genoemd is, om dat te verontreinigen.
\par 35 En zij hebben de hoogten van Baal gebouwd, die in het dal des zoons van Hinnom zijn, om hun zonen en hun dochteren den Molech door het vuur te laten gaan; hetwelk Ik hun niet heb geboden, noch in Mijn hart is opgekomen, dat zij dezen gruwel zouden doen; opdat zij Juda mochten doen zondigen.
\par 36 En nu, daarom zegt de HEERE, de God Israels, alzo van deze stad, waar gij van zegt: Zij is gegeven in de hand des konings van Babel, door het zwaard, en door den honger, en door de pestilentie;
\par 37 Ziet, Ik zal hen vergaderen uit al de landen, waarhenen Ik hen zal verdreven hebben in Mijn toorn, en in Mijn grimmigheid, en in grote verbolgenheid; en Ik zal hen tot deze plaats wederbrengen, en zal hen zeker doen wonen.
\par 38 Ja, zij zullen Mij tot een volk zijn, en Ik zal hun tot een God zijn.
\par 39 En Ik zal hun enerlei hart en enerlei weg geven, om Mij te vrezen al de dagen, hun ten goede, mitsgaders hun kinderen na hen.
\par 40 En Ik zal een eeuwig verbond met hen maken, dat Ik van achter hen niet zal afkeren, opdat Ik hun weldoe; en Ik zal Mijn vreze in hun hart geven, dat zij niet van Mij afwijken.
\par 41 En Ik zal Mij over hen verblijden, dat Ik hun weldoe; en Ik zal hen getrouwelijk in dat land planten, met Mijn ganse hart en met Mijn ganse ziel.
\par 42 Want zo zegt de HEERE: Gelijk als Ik over dit volk gebracht heb al dit grote kwaad, alzo zal Ik over hen brengen al het goede, dat Ik over hen spreke.
\par 43 En er zullen velden gekocht worden in dit land, waarvan gij zegt: Het is woest, dat er geen mens noch beest in is; het is in der Chaldeen hand gegeven.
\par 44 Velden zal men voor geld kopen, en de brieven onderschrijven, en verzegelen, en getuigen doen betuigen, in het land van Benjamin, en in de plaatsen rondom Jeruzalem, en in de steden van Juda, en in de steden van het gebergte, en in de steden der laagte, en in de steden van het zuiden; want Ik zal hun gevangenis wenden, spreekt de HEERE.

\chapter{33}

\par 1 Voorts geschiedde des HEEREN woord ten tweeden male tot Jeremia, als hij nog in het voorhof der bewaring was opgesloten, zeggende:
\par 2 Zo zegt de HEERE, Die het doet, de HEERE, Die dat formeert, opdat Hij het bevestige, HEERE is Zijn Naam;
\par 3 Roep tot Mij, en Ik zal u antwoorden, en Ik zal u bekend maken grote en vaste dingen, die gij niet weet.
\par 4 Want zo zegt de HEERE, de God Israels, van de huizen dezer stad, en van de huizen der koningen van Juda, die door de wallen en door het zwaard zijn afgebroken:
\par 5 Er zijn er wel ingekomen, om te strijden tegen de Chaldeen, maar het is om die te vullen met dode lichamen van mensen, die Ik verslagen heb in Mijn toorn en in Mijn grimmigheid; en omdat Ik Mijn aangezicht van deze stad verborgen heb, om al hunlieder boosheid.
\par 6 Zie, Ik zal haar de gezondheid en de genezing doen rijzen, en zal henlieden genezen, en zal hun openbaren overvloed van vrede en waarheid.
\par 7 En Ik zal de gevangenis van Juda en de gevangenis van Israel wenden, en zal ze bouwen als in het eerste.
\par 8 En Ik zal hen reinigen van al hun ongerechtigheid, met dewelke zij tegen Mij gezondigd hebben; en Ik zal vergeven al hun ongerechtigheden, met dewelke zij tegen Mij gezondigd en met dewelke zij tegen Mij overtreden hebben.
\par 9 En het zal Mij zijn tot een vrolijken naam, tot een roem, en tot een sieraad bij alle heidenen der aarde; die al het goede zullen horen, dat Ik hun doe; en zij zullen vrezen en beroerd zijn over al het goede, en over al den vrede, dien Ik hun beschikke.
\par 10 Alzo zegt de HEERE: In deze plaats (waarvan gij zegt: Zij is woest, dat er geen mens en geen beest in is), in de steden van Juda, en op de straten van Jeruzalem, die zo verwoest zijn, dat er geen mens, en geen inwoner, en geen beest in is, zal wederom gehoord worden,
\par 11 De stem der vrolijkheid en de stem der blijdschap, de stem des bruidegoms en de stem der bruid, de stem dergenen, die zeggen: Looft den HEERE der heirscharen, want de HEERE is goed, want Zijn goedertierenheid is in eeuwigheid! de stem dergenen, die lof aanbrengen ten huize des HEEREN; want Ik zal de gevangenis des lands wenden, als in het eerste, zegt de HEERE.
\par 12 Zo zegt de HEERE der heirscharen: In deze plaats, die zo woest is, dat er geen mens, zelfs tot het vee toe, in is, mitsgaders in al derzelver steden, zullen wederom woningen zijn van herderen, die de kudden doen legeren.
\par 13 In de steden van het gebergte, in de steden der laagte, en in de steden van het zuiden, en in het land van Benjamin, en in de plaatsen rondom Jeruzalem, en in de steden van Juda, zullen de kudden wederom onder de handen des tellers doorgaan, zegt de HEERE.
\par 14 Ziet, de dagen komen, spreekt de HEERE, dat Ik het goede woord verwekken zal, dat Ik tot het huis van Israel en over het huis van Juda gesproken heb.
\par 15 In die dagen, en te dier tijd zal Ik David een SPRUIT der gerechtigheid doen uitspruiten; en Hij zal recht en gerechtigheid doen op aarde.
\par 16 In die dagen zal Juda verlost worden, en Jeruzalem zeker wonen; en deze is, die haar roepen zal: De HEERE, onze GERECHTIGHEID.
\par 17 Want zo zegt de HEERE: Aan David zal niet worden afgesneden een Man, Die op den troon van het huis Israels zitte.
\par 18 Ook zal den Levietischen priesteren, van voor Mijn aangezicht, niet worden afgesneden een Man, Die brandoffer offere, en spijsoffer aansteke, en slachtoffer bereide al de dagen.
\par 19 En des HEEREN woord geschiedde tot Jeremia, zeggende:
\par 20 Alzo zegt de HEERE: Indien gijlieden Mijn verbond van den dag; en Mijn verbond van den nacht kondt vernietigen, zodat dag en nacht niet zijn op hun tijd;
\par 21 Zo zal ook vernietigd kunnen worden Mijn verbond met Mijn knecht David, dat hij geen zoon hebbe, die op zijn troon regere, en met de Levieten, de priesteren, Mijn dienaren.
\par 22 Gelijk het heir des hemels niet geteld, en het zand der zee niet gemeten kan worden, alzo zal Ik vermenigvuldigen het zaad van Mijn knecht David, en de Levieten, die Mij dienen.
\par 23 Voorts geschiedde des HEEREN woord tot Jeremia, zeggende:
\par 24 Hebt gij niet gezien, wat dit volk spreekt, zeggende: De twee geslachten, die de HEERE verkoren had, die heeft Hij nu verworpen? Ja, zij versmaden Mijn volk, zodat het geen volk meer is voor hun aangezicht.
\par 25 Zo zegt de HEERE: Indien Mijn verbond niet is van dag en nacht; indien Ik de ordeningen des hemels en der aarde niet gesteld heb;
\par 26 Zo zal Ik ook het zaad van Jakob en van Mijn knecht David verwerpen, dat Ik van zijn zaad niet neme, die daar heerse over het zaad van Abraham, Izak en Jakob; want Ik zal hun gevangenis wenden en Mij hunner ontfermen.

\chapter{34}

\par 1 Het woord, dat tot Jeremia geschied is van den HEERE (als Nebukadnezar, koning van Babel, en zijn ganse heir, en alle koninkrijken der aarde, die onder de heerschappij zijner hand waren, en al de volken tegen Jeruzalem streden, en tegen al haar steden), zeggende:
\par 2 Zo zegt de HEERE, de God Israels: Ga henen en spreek tot Zedekia, den koning van Juda, en zeg tot hem: Zo zegt de HEERE: Zie, Ik geef deze stad in de hand des konings van Babel, en hij zal ze met vuur verbranden.
\par 3 En gij zult van zijn hand niet ontkomen, maar zekerlijk gegrepen, en in zijn hand gegeven worden; en uw ogen zullen de ogen des konings van Babel zien, en zijn mond zal tot uw mond spreken, en gij zult te Babel komen.
\par 4 Maar hoor des HEEREN woord, o Zedekia, koning van Juda! zo zegt de HEERE van u: Gij zult door het zwaard niet sterven.
\par 5 Gij zult sterven in vrede, en naar de brandingen van uw vaderen, de vorige koningen, die voor u geweest zijn, alzo zullen zij over u branden, en u beklagen, zeggende: Och heer! want Ik heb het woord gesproken, spreekt de HEERE.
\par 6 En de profeet Jeremia sprak al deze woorden tot Zedekia, den koning van Juda, te Jeruzalem.
\par 7 Als het heir des konings van Babel streed tegen Jeruzalem, en tegen al de overgeblevene steden van Juda, tegen Lachis en tegen Azeka; want deze, zijnde vaste steden, waren overgebleven onder de steden van Juda.
\par 8 Het woord, dat tot Jeremia geschied is van den HEERE, nadat de koning Zedekia een verbond gemaakt had met het ganse volk, dat te Jeruzalem was, om vrijheid voor hen uit te roepen.
\par 9 Dat een iegelijk zijn knecht, en een iegelijk zijn maagd, zijnde een Hebreer of een Hebreinne, zou laten vrijgaan; zodat niemand zich van hen, van een Jood, zijn broeder, zou doen dienen.
\par 10 Nu hoorden al de vorsten en al het volk, die het verbond hadden ingegaan, dat zij, een iegelijk zijn knecht, en een iegelijk zijn maagd zouden laten vrijgaan, zodat zij zich niet meer van hen zouden doen dienen; zij hoorden dan, en lieten hen gaan;
\par 11 Maar zij keerden daarna wederom, en deden de knechten en maagden wederkomen, die zij hadden laten vrijgaan, en zij brachten hen ten onder tot knechten en tot maagden.
\par 12 Daarom geschiedde des HEEREN woord tot Jeremia, van den HEERE, zeggende:
\par 13 Zo zegt de HEERE, de God Israels: Ik heb een verbond gemaakt met uw vaderen, ten dage, als Ik hen uit Egypteland, uit het diensthuis uitvoerde, zeggende:
\par 14 Ten einde van zeven jaren zult gij laten gaan, een iegelijk zijn broeder, een Hebreer, die u zal verkocht zijn, en u zes jaren gediend heeft; gij zult hem dan van u laten vrijgaan; maar uw vaders hoorden niet naar Mij, en neigden hun oor niet.
\par 15 Gijlieden nu waart heden wedergekeerd, en hadt gedaan, dat recht is in Mijn ogen, vrijheid uitroepende, een iegelijk voor zijn naaste; en gij hadt een verbond gemaakt voor Mijn aangezicht, in het huis, dat naar Mijn Naam genoemd is.
\par 16 Maar gij zijt weder omgekeerd, en hebt Mijn Naam ontheiligd, en doen wederkomen, een iegelijk zijn knecht, en een iegelijk zijn maagd, die gij hadt laten vrijgaan naar hun lust; en gij hebt hen ten ondergebracht, om ulieden te wezen tot knechten en tot maagden.
\par 17 Daarom zegt de HEERE alzo: Gijlieden hebt naar Mij niet gehoord, om vrijheid uit te roepen, een iegelijk voor zijn broeder, en een iegelijk voor zijn naaste; ziet, zo roep Ik uit tegen ulieden, spreekt de HEERE, een vrijheid ten zwaarde, ter pestilentie, en ten honger, en zal u overgeven ter beroering allen koninkrijken der aarde.
\par 18 En Ik zal de mannen overgeven, die Mijn verbond hebben overtreden, die niet bevestigd hebben de woorden des verbonds, dat zij voor Mijn aangezicht gemaakt hadden, met het kalf, dat zij in tweeen hadden gehouwen, en waren tussen zijn stukken doorgegaan:
\par 19 De vorsten van Juda, en de vorsten van Jeruzalem, de kamerlingen, en de priesteren, en al het volk des lands, die door de stukken des kalfs zijn doorgegaan.
\par 20 Ja, Ik zal hen overgeven in de hand hunner vijanden, en in de hand dergenen, die hun ziel zoeken; en hun dode lichamen zullen het gevogelte des hemels en het gedierte der aarde tot spijze zijn.
\par 21 Zelfs Zedekia, den koning van Juda, en zijn vorsten, zal Ik overgeven in de hand hunner vijanden, en in de hand dergenen, die hun ziel zoeken, te weten, in de hand van het heir des konings van Babel, die van ulieden nu zijn opgetogen.
\par 22 Ziet, Ik zal bevel geven, spreekt de HEERE, en zal hen weder tot deze stad brengen, en zij zullen tegen haar strijden, en zullen ze innemen, en zullen ze met vuur verbranden; en Ik zal de steden van Juda stellen tot een verwoesting, dat er niemand in wone.

\chapter{35}

\par 1 Het woord, dat tot Jeremia geschied is van den HEERE, in de dagen van Jojakim, den zoon van Josia, den koning van Juda, zeggende:
\par 2 Ga henen tot der Rechabieten huis, en spreek met hen, en breng hen in des HEEREN huis, in een der kameren, en geef hun wijn te drinken.
\par 3 Toen nam ik Jaazanja, den zoon van Jeremia, den zoon van Habazzinja, mitsgaders zijn broederen, en al zijn zonen, en het ganse huis der Rechabieten;
\par 4 En bracht hen in des HEEREN huis, in de kamer der zonen van Hanan, den zoon van Jigdalia, den man Gods; welke is bij de kamer der oversten, die daar is boven de kamer van Maaseja, den zoon van Sallum, den dorpelbewaarder.
\par 5 En ik zette den kinderen van het huis der Rechabieten koppen vol wijn en bekers voor; en ik zeide tot hen: Drinkt wijn.
\par 6 Maar zij zeiden: Wij zullen geen wijn drinken; want Jonadab, de zoon van Rechab, onze vader, heeft ons geboden, zeggende: Gijlieden zult geen wijn drinken, gij, noch uw kinderen, tot in eeuwigheid.
\par 7 Ook zult gijlieden geen huis bouwen, noch zaad zaaien, noch wijngaard planten, noch hebben; maar gij zult in tenten wonen al uw dagen; opdat gij veel dagen leeft in het land, alwaar gij als vreemdeling verkeert.
\par 8 Zo hebben wij der stemme van Jonadab, den zoon van Rechab, onzen vader, gehoorzaamd in alles, wat hij ons geboden heeft; zodat wij geen wijn drinken al onze dagen, wij, onze vrouwen, onze zonen, en onze dochteren;
\par 9 En dat wij geen huizen bouwen tot onze woning; ook hebben wij geen wijngaard, noch veld, noch zaad;
\par 10 En wij hebben in tenten gewoond; alzo hebben wij gehoord en gedaan naar alles, wat ons onze vader Jonadab geboden heeft.
\par 11 Maar het is geschied, als Nebukadrezar, de koning van Babel, naar dit land optoog, dat wij zeiden: Komt, en laat ons naar Jeruzalem trekken vanwege het heir der Chaldeen, en vanwege het heir der Syriers; alzo zijn wij te Jeruzalem gebleven.
\par 12 Toen geschiedde des HEEREN woord tot Jeremia, zeggende:
\par 13 Zo zegt de HEERE der heirscharen, de God Israels: Ga henen en zeg tot de mannen van Juda en tot de inwoners van Jeruzalem: Zult gijlieden geen tucht aannemen, dat gij hoort naar Mijn woorden? spreekt de HEERE.
\par 14 De woorden van Jonadab, den zoon van Rechab, die hij zijn kinderen geboden heeft, dat zij geen wijn zouden drinken, zijn bevestigd; want zij hebben geen gedronken tot op dezen dag, maar het gebod huns vaders gehoord; en Ik heb tot ulieden gesproken, vroeg op zijnde en sprekende, maar gij hebt naar Mij niet gehoord.
\par 15 En Ik heb tot u gezonden al Mijn knechten, de profeten, vroeg op zijnde en zendende, om te zeggen: Bekeert u toch, een iegelijk van zijn bozen weg, en maakt uw handelingen goed, en wandelt andere goden niet na, om hen te dienen, zo zult gij in het land blijven, dat Ik u en uw vaderen gegeven heb; maar gij hebt uw oor niet geneigd, en naar Mij niet gehoord.
\par 16 Dewijl dan de kinderen van Jonadab, den zoon van Rechab, het gebod huns vaders, dat hij hun geboden heeft, bevestigd hebben, maar dit volk naar Mij niet hoort;
\par 17 Daarom alzo zegt de HEERE, de God der heirscharen, de God Israels: Ziet, Ik zal over Juda en over alle inwoners van Jeruzalem brengen al het kwaad, dat Ik tegen hen gesproken heb; omdat Ik tot hen gesproken heb, maar zij niet gehoord hebben, en Ik tot hen geroepen heb, maar zij niet hebben geantwoord.
\par 18 Tot het huis nu der Rechabieten zeide Jeremia: Zo zegt de HEERE der heirscharen, de God Israels: Omdat gijlieden het gebod van uw vader Jonadab zijt gehoorzaam geweest, en hebt al zijn geboden bewaard, en gedaan naar alles, wat hij ulieden geboden heeft;
\par 19 Daarom alzo zegt de HEERE der heirscharen, de God Israels: Er zal Jonadab, den zoon van Rechab, niet worden afgesneden een man, die voor Mijn aangezicht sta, al de dagen.

\chapter{36}

\par 1 Het gebeurde ook in het vierde jaar van Jojakim, den zoon van Josia, den koning van Juda, dat dit woord tot Jeremia geschiedde van den HEERE, zeggende:
\par 2 Neem u een rol des boeks, en schrijf daarop al de woorden, die Ik tot u gesproken heb, over Israel, en over Juda, en over al de volken, van den dag aan, dat Ik tot u gesproken heb, van de dagen van Josia aan, tot op dezen dag.
\par 3 Misschien zullen die van het huis van Juda horen al het kwaad, dat Ik hun gedenk te doen; opdat zij zich bekeren, een iegelijk van zijn bozen weg, en Ik hun ongerechtigheid en hun zonde vergeve.
\par 4 Toen riep Jeremia Baruch, den zoon van Nerija; en Baruch schreef uit den mond van Jeremia alle woorden des HEEREN, die Hij tot hem gesproken had, op een rol des boeks.
\par 5 En Jeremia gebood Baruch, zeggende: Ik ben opgehouden, ik zal in des HEEREN huis niet kunnen gaan.
\par 6 Zo ga gij henen, en lees in de rol, in dewelke gij uit mijn mond geschreven hebt, de woorden des HEEREN, voor de oren des volks, in des HEEREN huis, op den vastendag; en gij zult ze ook lezen voor de oren van gans Juda, die uit hun steden komen.
\par 7 Misschien zal hunlieder smeking voor des HEEREN aangezicht nedervallen, en zij zullen zich bekeren, een iegelijk van zijn bozen weg; want groot is de toorn en de grimmigheid, die de HEERE tegen dit volk heeft uitgesproken.
\par 8 En Baruch, de zoon van Nerija, deed naar alles, wat hem de profeet Jeremia geboden had, lezende in dat boek de woorden des HEEREN, in het huis des HEEREN.
\par 9 Want het geschiedde in het vijfde jaar van Jojakim, den zoon van Josia, den koning van Juda, in de negende maand, dat zij een vasten voor des HEEREN aangezicht uitriepen, allen volke te Jeruzalem, mitsgaders allen volke, die uit de steden van Juda te Jeruzalem kwamen.
\par 10 Zo las Baruch in dat boek de woorden van Jeremia in des HEEREN huis, in de kamer van Gemarja, den zoon van Safan, den schrijver, in het bovenste voorhof, aan de deur der nieuwe poort van het huis des HEEREN, voor de oren des gansen volks.
\par 11 Als nu Michaja, de zoon van Gemarja, den zoon van Safan, al de woorden des HEEREN uit dat boek gehoord had;
\par 12 Zo ging hij af ten huize des konings in de kamer des schrijvers; en ziet, aldaar zaten al de vorsten: Elisama, de schrijver, en Delaja, de zoon van Semaja, en Elnathan, de zoon van Achbor, en Gemarja, de zoon van Safan, en Zedekia, de zoon van Hananja, en al de vorsten.
\par 13 En Michaja maakte hun bekend al de woorden, die hij gehoord had, als Baruch uit dat boek las voor de oren des volks.
\par 14 Toen zonden al de vorsten Jehudi, den zoon van Nethanja, den zoon van Selemja, den zoon van Cuschi, tot Baruch, om te zeggen: De rol, waarin gij voor de oren des volks gelezen hebt, neem die in uw hand, en kom. Alzo nam Baruch, de zoon van Nerija, de rol in zijn hand, en kwam tot hen.
\par 15 En zij zeiden tot hem: Zit toch neder, en lees ze voor onze oren; en Baruch las voor hun oren.
\par 16 En het geschiedde, als zij al de woorden hoorden, dat zij verschrikten, de een tegen den ander; en zij zeiden tot Baruch: Voorzeker zullen wij al deze woorden den koning bekend maken.
\par 17 En zij vraagden Baruch, zeggende: Verklaar ons toch, hoe hebt gij al deze woorden uit zijn mond geschreven?
\par 18 En Baruch zeide tot hen: Uit zijn mond las hij tot mij al deze woorden, en ik schreef ze met inkt in dit boek.
\par 19 Toen zeiden de vorsten tot Baruch: Ga henen, verberg u, gij en Jeremia; en niemand wete, waar gijlieden zijt.
\par 20 Zij dan gingen in tot den koning in het voorhof; maar de rol leiden zij weg in de kamer van Elisama, den schrijver; en zij verklaarden al die woorden voor de oren des konings.
\par 21 Toen zond de koning Jehudi, om de rol te halen; en hij haalde ze uit de kamer van Elisama, den schrijver; en Jehudi las ze voor de oren des konings, en voor de oren van al de vorsten, die omtrent den koning stonden.
\par 22 (De koning nu zat in het winterhuis in de negende maand; en er was een vuur voor zijn aangezicht op den haard aangestoken.)
\par 23 En het geschiedde, als Jehudi drie stukken, of vier gelezen had, versneed hij ze met een schrijfmes, en wierp ze in het vuur, dat op den haard was, totdat de ganse rol verteerd was in het vuur, dat op den haard was.
\par 24 En zij verschrikten niet, en scheurden hun klederen niet, de koning noch al zijn knechten, die al deze woorden gehoord hadden.
\par 25 Hoewel ook Elnathan, en Delaja, en Gemarja bij den koning daarvoor spraken, dat hij de rol niet zou verbranden; doch hij hoorde niet naar hen.
\par 26 Daartoe gebood de koning aan Jerahmeel, den zoon van Hammelech, en Zeraja, den zoon van Azriel, en Selemja, den zoon van Abdeel, om den schrijver Baruch en den profeet Jeremia te vangen. Maar de HEERE had hen verborgen.
\par 27 Toen geschiedde des HEEREN woord tot Jeremia, nadat de koning de rol en de woorden, die Baruch geschreven had uit den mond van Jeremia, verbrand had, zeggende:
\par 28 Neem u weder een andere rol, en schrijf daarop al de eerste woorden, die geweest zijn op de eerste rol, die Jojakim, de koning van Juda, verbrand heeft.
\par 29 En tot Jojakim, den koning van Juda, zult gij zeggen: Zo zegt de HEERE: Gij hebt deze rol verbrand, zeggende: Waarom hebt gij daarop geschreven, zeggende: De koning van Babel zal zekerlijk komen, en dit land verderven, en maken, dat mens en beest daarin ophouden?
\par 30 Daarom zegt de HEERE alzo van Jojakim, den koning van Juda: Hij zal geen hebben, die op Davids troon zitte; en zijn dood lichaam zal weggeworpen zijn, des daags in de hitte, en des nachts in de vorst.
\par 31 En Ik zal over hem, en over zijn zaad, en over zijn knechten hunlieder ongerechtigheid bezoeken; en Ik zal over hen, en over de inwoners van Jeruzalem, en over de mannen van Juda, al het kwaad brengen, dat Ik tot hen gesproken heb; maar zij hebben niet gehoord.
\par 32 Jeremia dan nam een andere rol, en gaf ze aan den schrijver Baruch, den zoon van Nerija; die schreef daarop, uit den mond van Jeremia, al de woorden des boeks, dat Jojakim, de koning van Juda, met vuur verbrand had; en tot dezelve werden nog veel dergelijke woorden toegedaan.

\chapter{37}

\par 1 En Zedekia, zoon van Josia, regeerde, koning zijnde, in plaats van Chonja, Jojakims zoon, welken Zedekia Nebukadrezar, de koning van Babel, koning gemaakt had in het land van Juda.
\par 2 Maar hij hoorde niet, hij, noch zijn knechten, noch het volk des lands, naar de woorden des HEEREN, die Hij sprak door den dienst van den profeet Jeremia.
\par 3 Nochtans zond de koning Zedekia Juchal, den zoon van Selemja, en Sefanja, den zoon van Maaseja, den priester, tot den profeet Jeremia, om te zeggen: Bid toch voor ons tot den HEERE, onzen God!
\par 4 (Want Jeremia was nog ingaande en uitgaande in het midden des volks, en zij hadden hem nog in het gevangenhuis niet gesteld.
\par 5 En Farao's heir was uit Egypte uitgetogen; en de Chaldeen, die Jeruzalem belegerden, als zij het gerucht van hen gehoord hadden, zo waren zij van Jeruzalem opgetogen).
\par 6 Toen geschiedde des HEEREN woord tot den profeet Jeremia, zeggende:
\par 7 Zo zegt de HEERE, de God Israels: Zo zult gijlieden zeggen tot den koning van Juda, die u tot Mij gezonden heeft, om Mij te vragen: Ziet, Farao's heir, dat u ter hulpe uitgetogen is, zal wederkeren in zijn land, in Egypte;
\par 8 En de Chaldeen zullen wederkeren, en tegen deze stad strijden; en zij zullen ze innemen, en zullen ze met vuur verbranden.
\par 9 Zo zegt de HEERE: Bedriegt uw zielen niet, zeggende: De Chaldeen zullen zekerlijk van ons wegtrekken; want zij zullen niet wegtrekken.
\par 10 Want al sloegt gijlieden het ganse heir der Chaldeen, die tegen u strijden, en er bleven van hen enige verwonde mannen over, zo zouden zich die, een iegelijk in zijn tent, opmaken, en deze stad met vuur verbranden.
\par 11 Voorts geschiedde het, als het heir der Chaldeen van Jeruzalem was opgetogen, vanwege Farao's heir;
\par 12 Dat Jeremia uit Jeruzalem uitging, om te gaan in het land van Benjamin, om van daar te scheiden door het midden des volks.
\par 13 Als hij in de poort van Benjamin was, zo was daar de wachtmeester, wiens naam was Jerija, de zoon van Selemja, den zoon van Hananja; die greep den profeet Jeremia, zeggende: Gij wilt tot de Chaldeen vallen!
\par 14 En Jeremia zeide: Het is vals, ik wil niet tot de Chaldeen vallen. Doch hij hoorde niet naar hem; maar Jerija greep Jeremia aan, en bracht hem tot de vorsten.
\par 15 En de vorsten werden zeer toornig op Jeremia en sloegen hem; en zij stelden hem in het gevangenhuis, ten huize van Jonathan, den schrijver; want zij hadden dat tot een gevangenhuis gemaakt.
\par 16 Als Jeremia in de plaats des kuils, en in de kotjes gekomen was, en Jeremia aldaar veel dagen gezeten had;
\par 17 Zo zond de koning Zedekia henen, en liet hem halen; en de koning vraagde hem in zijn huis, in het verborgene, en zeide: Is er ook een woord van den HEERE? En Jeremia zeide: Er is; en hij zeide: Gij zult in de hand des konings van Babel gegeven worden.
\par 18 Voorts zeide Jeremia tot den koning Zedekia: Wat heb ik tegen u, of tegen uw knechten, of tegen dit volk gezondigd, dat gijlieden mij in het gevangenhuis gesteld hebt?
\par 19 Waar zijn nu ulieder profeten, die u geprofeteerd hebben, zeggende: De koning van Babel zal niet tegen ulieden, noch tegen dit land komen.
\par 20 Nu dan, hoor toch, o mijn heer koning! laat toch mijn smeking voor uw aangezicht nedervallen, en breng mij niet weder in het huis van Jonathan, den schrijver, opdat ik aldaar niet sterve.
\par 21 Toen gaf de koning Zedekia bevel; en zij bestelden Jeremia in het voorhof der bewaring, en men gaf hem des daags een bol broods uit de Bakkerstraat, totdat al het brood van de stad op was. Alzo bleef Jeremia in het voorhof der bewaring.

\chapter{38}

\par 1 Als Sefatja, de zoon van Matthan, en Gedalia, de zoon van Pashur, en Juchal, de zoon van Selemja, en Pashur, de zoon van Malchia, de woorden hoorden, die Jeremia tot al het volk sprak, zeggende:
\par 2 Zo zegt de HEERE: Wie in deze stad blijft, zal door het zwaard, door den honger of door de pestilentie sterven; maar wie tot de Chaldeen uitgaat, die zal leven, want hij zal zijn ziel tot een buit hebben, en zal leven.
\par 3 Zo zegt de HEERE: Deze stad zal zekerlijk gegeven worden in de hand van het heir des konings van Babel, datzelve zal ze innemen;
\par 4 Zo zeiden de vorsten tot den koning: Laat toch dezen man gedood worden; want aldus maakt hij de handen der krijgslieden, die in deze stad zijn overgebleven, en de handen des gansen volks slap, alzulke woorden tot hen sprekende; want deze man zoekt den vrede dezes volks niet, maar het kwaad.
\par 5 En de koning Zedekia zeide: Ziet, hij is in uw hand; want de koning zou geen ding tegen u vermogen.
\par 6 Toen namen zij Jeremia en wierpen hem in den kuil van Malchia, den zoon van Hammelech, die in het voorhof der bewaring was, en zij lieten Jeremia af met zelen; in den kuil nu was geen water, maar slijk; en Jeremia zonk in het slijk.
\par 7 Als nu Ebed-melech, de Moorman, een der kamerlingen, die toen in des konings huis was, hoorde, dat zij Jeremia in den kuil gedaan hadden (de koning nu zat in de poort van Benjamin);
\par 8 Zo ging Ebed-melech uit het huis des konings uit, en hij sprak tot den koning, zeggende:
\par 9 Mijn heer koning! deze mannen hebben kwalijk gehandeld in alles, wat zij gedaan hebben aan den profeet Jeremia, dien zij in den kuil geworpen hebben; daar hij toch in zijn plaats zou gestorven zijn vanwege den honger, dewijl geen brood meer in de stad is.
\par 10 Toen gebood de koning den Moorman Ebed-melech, zeggende: Neem van hier dertig mannen onder uw hand, en haal den profeet Jeremia op uit den kuil, eer dat hij sterft.
\par 11 Alzo nam Ebed-melech de mannen onder zijn hand, en ging in des konings huis tot onder de schatkamer, en nam van daar enige oude verscheurde en oude versleten lompen; en hij liet ze met zelen af tot Jeremia in den kuil.
\par 12 En Ebed-melech, de Moorman, zeide tot Jeremia: Leg nu deze oude verscheurde en versleten lompen onder de oksels uwer armen, van onder aan de zelen. En Jeremia deed alzo.
\par 13 En zij trokken Jeremia bij de zelen, en haalden hem op uit de kuil; en Jeremia bleef in het voorhof der bewaring.
\par 14 Toen zond de koning Zedekia henen, en liet den profeet Jeremia tot zich halen, in den derden ingang, die aan des HEEREN huis was; en de koning zeide tot Jeremia: Ik zal u een ding vragen, verheel geen ding voor mij.
\par 15 En Jeremia zeide tot Zedekia: Als ik het u verklaren zal, zult gij mij niet zekerlijk doden? En als ik u raad zal geven, gij zult toch naar mij niet horen.
\par 16 Toen zwoer de koning Zedekia aan Jeremia in het verborgene, zeggende: Zo waarachtig als de HEERE leeft, Die ons deze ziel gemaakt heeft: Indien ik u zal doden, of indien ik u zal overgeven in de hand dezer mannen, die uw ziel zoeken!
\par 17 Jeremia dan zeide tot Zedekia: Zo zegt de HEERE, de God der heirscharen, de God Israels: Indien gij gewilliglijk tot de vorsten des koning van Babel zult uitgaan, zo zal uw ziel leven, en deze stad zal niet verbrand worden met vuur; en gij zult leven, gij en uw huis.
\par 18 Maar indien gij tot de vorsten des konings van Babel niet zult uitgaan, zo zal deze stad gegeven worden in de hand der Chaldeen, en zij zullen ze met vuur verbranden; ook zult gij van hunlieder hand niet ontkomen.
\par 19 En de koning Zedekia zeide tot Jeremia: Ik ben bevreesd voor de Joden, die tot de Chaldeen gevallen zijn, dat zij mij misschien in derzelver hand overgeven, en zij den spot met mij drijven.
\par 20 En Jeremia zeide: Zij zullen u niet overgeven; wees toch gehoorzaam aan de stem des HEEREN, naar dewelke ik tot u spreek; zo zal het u welgaan, en uw ziel zal leven.
\par 21 Maar indien gij weigert uit te gaan, zo is dit het woord, dat de HEERE mij heeft doen zien;
\par 22 Ziedaar, al de vrouwen, die in het huis des konings van Juda zijn overgebleven, zullen uitgevoerd worden tot de vorsten des konings van Babel; en dezelve zullen zeggen: Uw vredegenoten hebben u aangehitst, en hebben u overmocht; uw voeten zijn in den modder gezonken; zij zijn achterwaarts gekeerd!
\par 23 Zij zullen dan al uw vrouwen en al uw zonen tot de Chaldeen uitvoeren; ook zult gij zelf van hun hand niet ontkomen; maar gij zult door de hand des konings van Babel gegrepen worden, en gij zult deze stad met vuur verbranden.
\par 24 Toen zeide Zedekia tot Jeremia: Dat niemand wete van deze woorden, zo zult gij niet sterven.
\par 25 En als de vorsten zullen horen, dat ik met u gesproken heb, en tot u komen, en tot u zeggen: Verklaar ons nu, wat hebt gij tot den koning gesproken? verheel het niet voor ons, zo zullen wij u niet doden; en wat heeft de koning tot u gesproken?
\par 26 Zo zult gij tot hen zeggen: Ik wierp mijn smeking voor des konings aangezicht neder, dat hij mij niet zou weder laten brengen in Jonathans huis, om aldaar te sterven.
\par 27 Als dan al de vorsten tot Jeremia kwamen, en hem vraagden, verklaarde hij hun, naar al deze woorden, die de koning geboden had; en zij lieten van hem af, omdat de zaak niet was gehoord.
\par 28 En Jeremia bleef in het voorhof der bewaring tot op den dag, dat Jeruzalem werd ingenomen; en hij was er nog, als Jeruzalem was ingenomen.

\chapter{39}

\par 1 In het negende jaar van Zedekia, koning van Juda, in de tiende maand, kwam Nebukadrezar, de koning van Babel, en al zijn heir, tegen Jeruzalem, en zij belegerden haar.
\par 2 In het elfde jaar van Zedekia, in de vierde maand, op den negenden der maand, werd de stad doorgebroken.
\par 3 En alle vorsten des konings van Babel togen henen in, en hielden bij de middelste poort; namelijk Nergal-sarezer Samgar-nebu, Sarsechim Rab-saris, Nergal-sarezer Rab-mag, en al de overige vorsten des konings van Babel.
\par 4 En het geschiedde, als Zedekia, de koning van Juda, en al de krijgslieden hen zagen, zo vloden zij, en togen bij nacht uit de stad, door den weg van des konings hof, door de poort tussen de twee muren; en hij toog uit door den weg des vlakken velds.
\par 5 Doch het heir der Chaldeen jaagde hen achterna; en zij achterhaalden Zedekia in de vlakke velden van Jericho, en vingen hem, en brachten hem opwaarts tot Nebukadrezar, den koning van Babel, naar Ribla, in het land van Hamath; die sprak oordelen tegen hem uit.
\par 6 En de koning van Babel slachtte de zonen van Zedekia te Ribla voor zijn ogen; ook slachtte de koning van Babel alle edelen van Juda.
\par 7 En hij verblindde de ogen van Zedekia, en bond hem met twee koperen ketenen, om hem naar Babel te voeren.
\par 8 En de Chaldeen verbrandden het huis des konings en de huizen des volks met vuur; en zij braken de muren van Jeruzalem af.
\par 9 Het overige nu des volks, die in de stad waren overgebleven, en de afvalligen, die tot hem gevallen waren, met het overige des volks, die overgebleven waren, voerde Nebuzaradan, de overste der trawanten, gevankelijk naar Babel.
\par 10 Maar van het volk, die arm waren, die niet met al hadden, liet Nebuzaradan, de overste der trawanten, enigen overig in het land van Juda; en hij gaf hun te dien dage wijngaarden en akkers.
\par 11 Maar van Jeremia had Nebukadrezar, de koning van Babel, bevel gegeven in de hand van Nebuzaradan, den overste der trawanten, zeggende:
\par 12 Neem hem, en stel uw ogen op hem, en doe hem niets kwaads; maar gelijk als hij tot u spreken zal, doe alzo met hem.
\par 13 Zo zond Nebuzaradan, de overste der trawanten, mitsgaders Nebuschazban Rab-saris en Nergal-sarezer Rab-mag, en al de oversten des konings van Babel;
\par 14 Zij zonden dan henen en namen Jeremia uit het voorhof der bewaring, en gaven hem over aan Gedalia, den zoon van Ahikam, den zoon van Safan, dat hij hem henen uitbracht naar huis; alzo bleef hij in het midden des volks.
\par 15 Het woord des HEEREN was ook tot Jeremia geschied, als hij in het voorhof der bewaring besloten was, zeggende:
\par 16 Ga henen, en spreek tot Ebed-melech, den Moorman, zeggende: Zo zegt de HEERE der heirscharen, de God Israels: Zie, Ik zal Mijn woorden brengen over deze stad, ten kwade en niet ten goede; en zij zullen te dien dage voor uw aangezicht zijn.
\par 17 Maar Ik zal u te dien dage redden, spreekt de HEERE; en gij zult niet overgegeven worden in de hand der mannen, voor welker aangezicht gij vreest.
\par 18 Want Ik zal u zekerlijk bevrijden, en gij zult door het zwaard niet vallen; maar gij zult uw ziel tot een buit hebben, omdat gij op Mij vertrouwd hebt, spreekt de HEERE.

\chapter{40}

\par 1 Het woord, dat van den HEERE geschied is tot Jeremia, nadat Nebuzaradan, de overste der trawanten, hem had laten gaan van Rama; als hij hem had laten halen, daar hij met ketenen gebonden was in het midden aller gevangenen van Jeruzalem en Juda, die naar Babel gevankelijk werden weggevoerd.
\par 2 Want de overste der trawanten liet Jeremia halen, en zeide tot hem: De HEERE, uw God, heeft dit kwaad over deze plaats gesproken.
\par 3 En de HEERE heeft het doen komen, en gedaan, gelijk als Hij gesproken had; want gijlieden hebt gezondigd tegen den HEERE, en Zijner stem niet gehoorzaamd; daarom is ulieden deze zaak geschied.
\par 4 Nu dan, zie, ik heb u heden losgemaakt van de ketenen, die aan uw hand waren; indien het goed is in uw ogen met mij naar Babel te komen, zo kom, en ik zal mijn oog op u stellen; maar indien het kwaad is in uw ogen met mij naar Babel te komen, zo laat het; zie, het ganse land is voor uw aangezicht, waarhenen het goed en recht in uw ogen is te gaan, ga daar.
\par 5 En dewijl hij nog niet zal wederkeren, zo keer gij tot Gedalia, den zoon van Ahikam, den zoon van Safan, dien de koning van Babel over de steden van Juda gesteld heeft; en woon bij hem in het midden des volks; of overal, waar het in uw ogen recht is te gaan, ga er henen. En de overste der trawanten gaf hem reiskost en een geschenk, en liet hem gaan.
\par 6 Alzo kwam Jeremia tot Gedalia, den zoon van Ahikam, te Mizpa; en hij woonde bij hem in het midden des volks, die in het land waren overgelaten.
\par 7 Toen nu alle oversten der heiren, die in het veld waren, zij en hun mannen, hoorden, dat de koning van Babel Gedalia, den zoon van Ahikam, over het land gesteld had, en dat hij aan hem bevolen had de mannen, en de vrouwen, en de kinderkens, en van de armsten des lands, van degenen, die niet naar Babel gevankelijk waren weggevoerd;
\par 8 Zo kwamen zij tot Gedalia te Mizpa, namelijk, Ismael, de zoon van Nethanja, en Johanan en Jonathan, de zonen van Kareah, en Seraja, de zoon van Tanhumeth, en de zonen van Efai, den Netofathiet, en Jezanja, de zoon eens Maachathiets, zij en hun mannen.
\par 9 En Gedalia, de zoon van Ahikam, den zoon van Safan, zwoer hun en hun mannen, zeggende: Vreest niet van de Chaldeen te dienen; blijft in het land, en dient den koning van Babel, zo zal het u welgaan.
\par 10 En ziet, ik woon te Mizpa, om te staan voor het aangezicht der Chaldeen, die tot ons zullen komen; gijlieden dan verzamelt wijn, en zomervruchten, en olie, en doet ze in uw vaten, en woont in uw steden, die gij hebt ingenomen.
\par 11 Als ook al de Joden, die in Moab, en onder de kinderen Ammons, en in Edom, en die in al die landen waren, hoorden, dat de koning van Babel in Juda een overblijfsel gelaten had; en dat hij Gedalia, den zoon van Ahikam, den zoon van Safan, over hen gesteld had;
\par 12 Zo keerden al de Joden weder uit al de plaatsen, waarhenen zij gedreven waren, en kwamen in het land van Juda tot Gedalia te Mizpa; en zij verzamelden zeer veel wijns en zomervruchten.
\par 13 Doch Johanan, de zoon van Kareah, en alle oversten der heiren, die in het veld waren, kwamen tot Gedalia te Mizpa;
\par 14 En zeiden tot hem: Weet gij wel, dat Baalis, de koning der kinderen Ammons, Ismael, den zoon van Nethanja, uitgezonden heeft, om u aan het leven te slaan? Maar Gedalia, de zoon van Ahikam, geloofde hen niet.
\par 15 Johanan nochtans, de zoon van Kareah, sprak tot Gedalia, in het verborgene, te Mizpa, zeggende: Laat mij toch henengaan, en Ismael, den zoon van Nethanja, slaan, en niemand zal het weten; waarom zou hij u aan het leven slaan, en gans Juda, die tot u vergaderd zijn, verstrooid worden, en het overblijfsel van Juda verloren gaan?
\par 16 Maar Gedalia, de zoon van Ahikam, zeide tot Johanan, den zoon van Kareah: Doe deze zaak niet, want gij spreekt vals van Ismael.

\chapter{41}

\par 1 Maar het geschiedde in de zevende maand, dat Ismael, de zoon van Nethanja, den zoon van Elisama, van koninklijken zade, en de oversten des konings, te weten tien mannen, met hem kwamen tot Gedalia, den zoon van Ahikam, te Mizpa; en zij aten aldaar brood te zamen, te Mizpa.
\par 2 En Ismael, de zoon van Nethanja, maakte zich op, mitsgaders de tien mannen, die met hem waren, en zij sloegen Gedalia, den zoon van Ahikam, den zoon van Safan, met het zwaard; alzo doodde hij hem, dien de koning van Babel over het land gesteld had.
\par 3 Ook sloeg Ismael al de Joden, die met hem, namelijk met Gedalia, te Mizpa waren, en de Chaldeen, de krijgslieden, die aldaar gevonden werden.
\par 4 Het geschiedde nu op den tweeden dag, nadat hij Gedalia gedood had, en niemand het wist;
\par 5 Zo kwamen er lieden van Sichem, van Silo, en van Samaria, tachtig man, hebbende den baard afgeschoren, en de klederen gescheurd, en zichzelven gesneden; en spijsoffer en wierook waren in hun hand, om ten huize des HEEREN te brengen.
\par 6 En Ismael, de zoon van Nethanja, ging uit van Mizpa hun tegemoet, al gaande en wenende; en het geschiedde, als hij hen aantrof dat hij zeide: Komt tot Gedalia, den zoon van Ahikam!
\par 7 Maar het geschiedde, als zij in het midden der stad gekomen waren, dat Ismael, de zoon van Nethanja, hen keelde, en wierp hen in het midden des kuils, hij en de mannen, die met hem waren.
\par 8 Doch onder hen werden tien mannen gevonden, die tot Ismael zeiden: Dood ons niet, want wij hebben verborgen schatten in het veld, van tarwe, en gerst, en olie, en honig. Zo liet hij af, en doodde ze niet in het midden hunner broederen.
\par 9 De kuil nu, waarin Ismael al de dode lichamen der mannen, die hij aan de zijde van Gedalia geslagen had, henenwierp, is dezelfde, dien de koning Asa maakte vanwege Baesa, den koning Israels; dezen vulde Ismael, de zoon van Nethanja, met de verslagenen.
\par 10 En Ismael voerde het ganse overblijfsel des volks, dat te Mizpa was, gevankelijk, te weten des konings dochteren, en al het volk, die te Mizpa waren overgelaten, die Nebuzaradan, de overste der trawanten, aan Gedalia, den zoon van Ahikam, bevolen had; Ismael dan, den zoon van Nethanja, voerde ze gevankelijk weg, en toog henen, om over te gaan tot de kinderen Ammons.
\par 11 Toen nu Johanan, de zoon van Kareah, en al de oversten der heiren, die met hem waren, al het kwaad hoorden, dat Ismael, de zoon van Nethanja, gedaan had;
\par 12 Zo namen zij al de mannen, en togen henen, om met Ismael, den zoon van Nethanja, te strijden; en zij vonden hem aan het grote water, dat bij Gibeon is.
\par 13 En het geschiedde, als al het volk, dat met Ismael was, Johanan zag, den zoon van Kareah, en al de oversten der heiren, die met hem waren, zo werden zij verblijd.
\par 14 En al het volk, dat Ismael van Mizpa gevankelijk had weggevoerd, wendde zich om; en zij keerden zich en gingen over tot Johanan, den zoon van Kareah.
\par 15 Doch Ismael, de zoon van Nethanja, ontkwam van Johanans aangezicht, met acht mannen, en hij toog tot de kinderen Ammons.
\par 16 Toen nam Johanan, de zoon van Kareah, mitsgaders al de oversten der heiren, die met hem waren, het ganse overblijfsel des volks, dat hij wedergebracht had van Ismael, den zoon van Nethanja, van Mizpa, (nadat hij Gedalia, den zoon van Ahikam, geslagen had) te weten de mannen, die krijgslieden waren, en de vrouwen, en kinderkens, en kamerlingen, die hij van Gibeon had wedergebracht;
\par 17 En zij togen henen, en sloegen zich neder te Geruth-chimham, dat bij Bethlehem is, om voort te trekken, dat zij in Egypte kwamen.
\par 18 Voor het aangezicht der Chaldeen; want zij vreesden voor hunlieder aangezicht, omdat Ismael, de zoon van Nethanja, Gedalia, den zoon van Ahikam, geslagen had, dien de koning van Babel over het land gesteld had.

\chapter{42}

\par 1 Toen traden toe alle oversten der heiren, Johanan, de zoon van Kareah, en Jezanja, de zoon van Hosaja, en al het volk, van den kleinste tot den grootste toe;
\par 2 En zij zeiden tot den profeet Jeremia: Laat toch onze smeking voor uw aangezicht nedervallen, en bid voor ons tot den HEERE, uw God, voor dit ganse overblijfsel; want wij zijn weinigen van velen overgelaten, gelijk als uw ogen ons zien;
\par 3 Dat ons de HEERE, uw God, bekend make den weg, dien wij zullen ingaan, en de zaak, die wij zullen doen.
\par 4 En de profeet Jeremia zeide tot hen: Ik heb het gehoord; ziet, ik zal tot den HEERE, uw God, bidden naar uw woorden; en het zal geschieden, het ganse woord, dat de HEERE u zal antwoorden, zal ik u bekend maken, ik zal u niet een woord onthouden.
\par 5 Toen zeiden zij tot Jeremia: De HEERE zij tussen ons tot een waarachtig en gewis Getuige: indien wij niet naar alle woord, met hetwelk u de HEERE, uw God, tot ons zal zenden, alzo zullen doen!
\par 6 Hetzij dan goed of kwaad, wij zullen der stem des HEEREN, onzes Gods, tot Welken wij u zenden, gehoorzaam zijn; opdat het ons welga, wanneer wij der stem des HEEREN, onzes Gods, zullen gehoorzaam zijn.
\par 7 En het gebeurde ten einde van tien dagen, dat des HEEREN woord tot Jeremia geschiedde.
\par 8 Toen riep hij Johanan, den zoon van Kareah, en alle oversten der heiren, die met hem waren, en al het volk, van den kleinste af tot den grootste toe;
\par 9 En hij zeide tot hen: Zo zegt de HEERE, de God Israels, tot Welken gij mij gezonden hebt, om uw smeking voor Zijn aangezicht neder te werpen:
\par 10 Indien gijlieden in dit land zult blijven wonen, zo zal Ik u bouwen en niet afbreken, en u planten en niet uitrukken; want Ik heb berouw over het kwaad, dat Ik u aangedaan heb.
\par 11 Vreest niet voor het aangezicht des konings van Babel, voor wiens aangezicht gij vreest; vreest niet voor hem, spreekt de HEERE; want Ik zal met u zijn, om u te behouden en u van zijn hand te redden.
\par 12 En Ik zal ulieden barmhartigheid geven, dat hij zich uwer erbarme, en u weder in uw land brenge.
\par 13 Maar zo gijlieden zult zeggen: Wij zullen in dit land niet blijven; opdat gij der stem des HEEREN, uws Gods, niet gehoorzaam zijt,
\par 14 Zeggende: Neen, maar wij zullen gaan in Egypteland, alwaar wij geen krijg zullen zien, noch het geluid der bazuin horen, noch naar brood hongeren, en daar zullen wij blijven;
\par 15 Nu dan, daarom hoort des HEEREN woord, gij overblijfsel van Juda! Zo zegt de HEERE der heirscharen, de God Israels: Indien gij ganselijk uw aangezichten zult stellen om in Egypte te gaan, en zult henen ingaan, om aldaar als vreemdelingen te verkeren;
\par 16 Zo zal het geschieden, dat het zwaard, waar gij voor vreest, u aldaar in Egypteland zal achterhalen; en de honger, waar gij voor zorgt, zal u aldaar in Egypte achter aankleven, en gij zult aldaar sterven.
\par 17 Zo zullen al de mannen zijn, die hun aangezichten stellen, om in Egypte te gaan, om aldaar als vreemdelingen te verkeren; zij zullen sterven door het zwaard, door den honger en door de pestilentie; en zij zullen niemand hebben, die overblijve of ontkome van het kwaad, dat Ik over hen zal brengen.
\par 18 Want zo zegt de HEERE der heirscharen, de God Israels: Gelijk als Mijn toorn, en Mijn grimmigheid is uitgestort over de inwoners van Jeruzalem, alzo zal Mijn grimmigheid over ulieden uitgestort worden, als gij in Egypte zult gekomen zijn; en gij zult wezen tot een vervloeking, en tot een ontzetting, en tot een vloek, en tot smaadheid, en zult deze plaats niet meer zien.
\par 19 De HEERE heeft tegen ulieden gesproken, gij overblijfsel van Juda! Gaat niet in Egypte; weet zekerlijk, dat ik heden tegen u betuigd heb.
\par 20 Gewisselijk, gij hebt uw zielen verleid; want gij hebt mij tot den HEERE, uw God, gezonden, zeggende: Bid voor ons tot den HEERE, onzen God, en naar alles, wat de HEERE, onze God, zal zeggen, alzo maak het ons bekend, en wij zullen het doen.
\par 21 Nu heb ik het u heden bekend gemaakt; maar gij hebt niet gehoord naar de stem des HEEREN, uws Gods, noch naar al hetgeen, met hetwelk Hij mij tot u gezonden heeft.
\par 22 Zo weet nu zekerlijk, dat gij door het zwaard, door den honger en door de pestilentie sterven zult, ter plaatse, waar het u gelust heeft henen te gaan, om aldaar als vreemdelingen te verkeren.

\chapter{43}

\par 1 En het geschiedde, als Jeremia geeindigd had tot het ganse volk te spreken al de woorden des HEEREN, huns Gods, met dewelke hem de HEERE, hun God, tot hen gezonden had, te weten al die woorden,
\par 2 Zo sprak Azaria, de zoon van Hosaja, en Johanan, de zoon van Kareah, en al de trotse mannen, zeggende tot Jeremia: Gij spreekt leugen; de HEERE, onze God, heeft u niet gezonden, om te zeggen: Gijlieden zult niet gaan in Egypte, om aldaar als vreemdelingen te verkeren.
\par 3 Maar Baruch, de zoon van Nerija, hitst u tegen ons op, opdat hij ons overgeve in de hand der Chaldeen, dat zij ons doden en ons gevankelijk naar Babel wegvoeren.
\par 4 Alzo gehoorzaamde Johanan, de zoon van Kareah, en al de oversten der heiren, en al het volk, der stem des HEEREN niet, om in het land van Juda te blijven.
\par 5 Maar Johanan, de zoon van Kareah, en al de oversten der heiren namen het ganse overblijfsel van Juda, die van al de heidenen, waar zij waren henengedreven, wedergekeerd waren, om in het land van Juda te wonen;
\par 6 De mannen, en de vrouwen, en de kinderkens, en des konings dochteren, en alle ziel, die Nebuzaradan, de overste der trawanten, bij Gedalia, den zoon van Ahikam, den zoon van Safan, gelaten had, ook den profeet Jeremia, en Baruch, den zoon van Nerija;
\par 7 En zij togen in Egypteland, want zij waren der stem des HEEREN niet gehoorzaam; en zij kwamen tot Tachpanhes.
\par 8 Toen geschiedde des HEEREN woord tot Jeremia te Tachpanhes, zeggende:
\par 9 Neem grote stenen in uw hand, en verberg ze in de klei in den ticheloven, die bij de deur van Farao's huis te Tachpanhes is, voor de ogen der Joodse mannen;
\par 10 En zeg tot hen: Zo zegt de HEERE der heirscharen, de God Israels: Ziet, Ik zal henenzenden, en Nebukadrezar, den koning van Babel, Mijn knecht, halen, en Ik zal zijn troon zetten boven op deze stenen, die Ik verborgen heb; en hij zal zijn schone tent daarover spannen.
\par 11 En hij zal komen en Egypteland slaan: wie ten dood, ten dode; en wie ter gevangenis, ter gevangenis; en wie ten zwaard, ten zwaarde.
\par 12 En Ik zal een vuur aansteken in de huizen der goden van Egypte, en hij zal ze verbranden, en gevankelijk wegvoeren; en hij zal Egypteland aantrekken, gelijk als een herder zijn kleed aantrekt, en hij zal van daar uittrekken in vrede.
\par 13 En hij zal de opgerichte beelden van Beth-semes, hetwelk in Egypteland is, verbreken; en hij zal de huizen der goden van Egypte met vuur verbranden.

\chapter{44}

\par 1 Het woord, dat tot Jeremia geschiedde aan al de Joden, die in Egypteland woonden, die te Migdol woonden, en te Tachpanhes, en te Nof, en in het land Pathros, zeggende:
\par 2 Alzo zegt de HEERE der heirscharen, de God Israels: Gij hebt gezien al het kwaad, dat Ik gebracht heb over Jeruzalem en over alle steden van Juda; en ziet, zij zijn een woestheid te deze dage, en niemand woont daarin;
\par 3 Vanwege hun boosheid, die zij gedaan hebben, om Mij te tergen, gaande om te roken en andere goden te dienen, die zij niet kenden, zij, gij, noch uw vaders.
\par 4 En Ik heb tot u gezonden al Mijn knechten, de profeten, vroeg op zijnde en zendende, om te zeggen: Doet toch deze gruwelijke zaak niet, die Ik haat.
\par 5 Maar zij hebben niet gehoord, noch hun oor geneigd, om zich van hun boosheid te bekeren, dat zij anderen goden niet roken.
\par 6 Daarom is Mijn grimmigheid en Mijn toorn uitgestort, en heeft gebrand in de steden van Juda en in de straten van Jeruzalem; zodat zij tot eenzaamheid en tot verwoesting geworden zijn, gelijk het is te dezen dage.
\par 7 En nu, zo zegt de HEERE, de God der heirscharen, de God Israels: Waarom doet gij zulk een groot kwaad tegen uw zielen, opdat gij u den man en de vrouw, het kind en den zuigeling uit het midden van Juda uitroeit, opdat gij u geen overblijfsel overlaat?
\par 8 Tergende Mij door de werken uwer handen, rokende anderen goden in het land van Egypte, alwaar gij gekomen zijt, om daar als vreemdeling te verkeren; opdat gij uzelven uitroeit, en opdat gij wordt tot een vloek, en tot een smaadheid onder alle volken der aarde?
\par 9 Hebt gij vergeten de boosheden uwer vaderen, en de boosheden der koningen van Juda, en de boosheden hunner vrouwen, en uw boosheden, en de boosheden uwer vrouwen, die zij gedaan hebben in het land van Juda en in de straten van Jeruzalem?
\par 10 Zij zijn tot op dezen dag nog niet verbrijzeld van hart, en zij hebben niet gevreesd, noch gewandeld in Mijn wet en in Mijn inzettingen, die Ik voor ulieder aangezicht en voor het aangezicht uwer vaderen gegeven heb.
\par 11 Daarom, zo zegt de HEERE der heirscharen, de God Israels: Ziet, Ik zal Mijn aangezicht tegen ulieden stellen ten kwade, en om gans Juda uit te roeien.
\par 12 En Ik zal het overblijfsel van Juda wegnemen, die hun aangezichten gesteld hebben, om in Egypteland te gaan, om aldaar als vreemdelingen te verkeren; en zij zullen allen in Egypteland verteerd worden; door het zwaard zullen zij vallen, door den honger zullen zij verteerd worden, van den kleinste tot den grootste toe; door het zwaard en door den honger zullen zij sterven; en zij zullen worden tot een vervloeking, tot een ontzetting en tot een vloek, en tot een smaadheid.
\par 13 Want Ik zal bezoeking doen over degenen, die in Egypteland wonen, gelijk als Ik bezoeking gedaan heb over Jeruzalem, door het zwaard, door den honger en door de pestilentie;
\par 14 Zodat het overblijfsel van Juda, die in Egypteland gekomen zijn, om aldaar als vreemdelingen te verkeren, geen zal hebben, die ontkome, of overblijve; te weten om weder te keren in het land van Juda, waarnaar hun ziel verlangt weder te keren, om aldaar te wonen; maar zij zullen er niet wederkeren, behalve die ontkomen zullen.
\par 15 Toen antwoordden aan Jeremia al de mannen, die wisten, dat hun vrouwen anderen goden rookten, en al de vrouwen, die daar stonden, zijnde een grote hoop, mitsgaders al het volk, die in Egypteland, in Pathros, woonde, zeggende:
\par 16 Aangaande het woord, dat gij tot ons in des HEEREN Naam gesproken hebt, wij zullen naar u niet horen.
\par 17 Maar wij zullen ganselijk doen al hetgeen uit onzen mond is uitgegaan, rokende aan Melecheth des hemels, en haar drankofferen offerende, gelijk als wij gedaan hebben, wij en onze vaders, onze koningen en onze vorsten, in de steden van Juda en in de straten van Jeruzalem; toen werden wij met brood verzadigd, en waren vrolijk, en zagen geen kwaad.
\par 18 Maar van toen af, dat wij opgehouden hebben aan Melecheth des hemels te roken, en haar drankofferen te offeren, hebben wij van alles gebrek gehad, en zijn door het zwaard en door den honger verteerd.
\par 19 Ook wanneer wij aan Melecheth des hemels roken en haar drankofferen offeren, maken wij haar gebeelde koeken, om haar af te beelden, en offeren wij haar drankofferen, zonder onze mannen?
\par 20 Toen sprak Jeremia tot al het volk, tot de mannen en tot de vrouwen, en tot al het volk, die hem zulks geantwoord hadden, zeggende:
\par 21 Het roken, dat gijlieden in de steden van Juda en in de straten van Jeruzalem gerookt hebt, gij en uw vaderen, uw koningen en uw vorsten, en het volk des lands, heeft de HEERE daaraan niet gedacht, en is het niet in Zijn hart opgekomen?
\par 22 Zodat het de HEERE niet meer kon verdragen, vanwege de boosheid uwer handelingen, vanwege de gruwelen, die gij deedt; daarom is uw land geworden tot een woestheid, en tot ontzetting, en tot een vloek, dat er niemand in woont, gelijk het is te dezen dage;
\par 23 Vanwege dat gij gerookt hebt, en dat gij tegen den HEERE gezondigd hebt, en des HEEREN stem niet gehoorzaam zijt geweest, en in Zijn wet en in Zijn inzettingen, en in Zijn getuigenissen niet hebt gewandeld; daarom is u dit kwaad wedervaren, gelijk het is te dezen dage.
\par 24 Voorts zeide Jeremia tot al het volk, en tot al de vrouwen: Hoort des HEEREN woord, gij gans Juda, die in Egypteland zijt!
\par 25 Zo spreekt de HEERE der heirscharen, de God Israels, zeggende: Aangaande u en uw vrouwen, zij hebben toch met uw mond gesproken, en gij hebt het met uw handen vervuld, zeggende: Wij zullen onze geloften, die wij beloofd hebben, ganselijk houden, rokende aan Melecheth des hemels, en haar drankofferen offerende; nu, zij hebben uw geloften volkomenlijk bevestigd en uw geloften volkomenlijk gehouden.
\par 26 Daarom hoort des HEEREN woord, gij gans Juda, die in Egypteland woont! Ziet, Ik zweer bij Mijn groten Naam, zegt de HEERE, zo Mijn Naam met den mond van enig man van Juda in gans Egypteland meer zal genoemd worden, die zegge: Zo waarachtig als de Heere HEERE leeft!
\par 27 Ziet, Ik zal over hen waken ten kwade en niet ten goede; en alle mannen van Juda, die in Egypteland zijn, zullen door het zwaard en door den honger verteerd worden, totdat zij ten einde zijn.
\par 28 Maar die van het zwaard ontkomen, zullen uit Egypteland wederkeren in het land van Juda, weinig in getal; en het ganse overblijfsel van Juda, die in Egypteland gekomen zijn, om aldaar als vreemdelingen te verkeren, zullen weten, wiens woord bestaan zal, het Mijn of het hunne.
\par 29 En dit zal ulieden het teken zijn, spreekt de HEERE, dat Ik in deze plaats over u bezoeking zal doen; opdat gij weet, dat Mijn woorden zekerlijk over u bestaan zullen ten kwade;
\par 30 Alzo zegt de HEERE: Ziet, Ik zal Farao Hofra, den koning van Egypte, geven in de hand zijner vijanden, en in de hand dergenen, die zijn ziel zoeken, gelijk als Ik Zedekia, den koning van Juda, gegeven heb in de hand van Nebukadrezar, den koning van Babel, zijn vijand, en die zijn ziel zocht.

\chapter{45}

\par 1 Het woord, dat de profeet Jeremia gesproken heeft tot Baruch, den zoon van Nerija, als hij die woorden uit den mond van Jeremia in een boek schreef, in het vierde jaar van Jojakim, den zoon van Josia, den koning van Juda, zeggende:
\par 2 Alzo zegt de HEERE, de God Israels, van u, o Baruch!
\par 3 Gij zegt: Wee nu mij, want de HEERE heeft droefenis tot mijn smart gedaan; ik ben moede van mijn zuchten, en vind geen rust!
\par 4 Zo zult gij tot hem zeggen: Zo zegt de HEERE: Zie, dat Ik gebouwd heb, breek Ik af, en dat Ik geplant heb, ruk Ik uit, zelfs dit ganse land.
\par 5 En zoudt gij u grote dingen zoeken? Zoek ze niet; want zie, Ik breng een kwaad over alle vlees, spreekt de HEERE; maar Ik zal u uw ziel tot een buit geven, in alle plaatsen, waar gij zult henentrekken.

\chapter{46}

\par 1 Het woord des HEEREN, dat tot den profeet Jeremia geschied is tegen de heidenen.
\par 2 Tegen Egypte; tegen het heir van Farao Necho, koning van Egypte, dat aan de rivier Frath, bij Karchemis was, dat Nebukadrezar, de koning van Babel, sloeg, in het vierde jaar van Jojakim, den zoon van Josia, den koning van Juda.
\par 3 Rust het schild en de rondas toe, en nadert tot den strijd!
\par 4 Spant de paarden aan, en klimt op, gij ruiters! en stelt u met helmen; veegt de spiesen, trekt de pantsiers aan!
\par 5 Waarom zie Ik, dat zij versaagd en achterwaarts gedreven zijn? Zelfs hun helden zijn verslagen, en nemen de vlucht, en zien niet om; er is schrik van rondom, spreekt de HEERE.
\par 6 De snelle ontvliede niet, en de held ontkome niet; tegen het noorden, aan den oever der rivier Frath zijn zij gestruikeld en gevallen.
\par 7 Wie is deze, die optrekt als een stroom, wiens wateren zich bewegen als de rivieren?
\par 8 Egypte trekt op als een stroom, en zijn wateren bewegen zich als de rivieren; en hij zegt: Ik zal optrekken, ik zal de aarde bedekken, ik zal de stad, en die daarin wonen, verderven.
\par 9 Trekt op, gij paarden! en raast, gij wagens! en laat de helden uittrekken: de Moren, en de Puteers, die het schild handelen, en de Lydiers, die den boog handelen en spannen.
\par 10 Maar deze dag is des HEEREN, des HEEREN der heirscharen, een dag der wrake, dat Hij zich wreke van Zijn wederpartijders, en het zwaard zal vreten, en verzadigd, en dronken worden van hun bloed; want de Heere, HEERE der heirscharen, heeft een slachtoffer in het land van het noorden, aan de rivier Frath.
\par 11 Ga henen op naar Gilead, en haal balsem, gij jonkvrouw, dochter van Egypte! Tevergeefs vermenigvuldigt gij de medicijnen, er is geen heling voor u.
\par 12 De volken hebben uw schande gehoord, en het land is vol van uw gekrijt; want zij hebben zich gestoten, held tegen held, zij zijn beiden te zamen gevallen.
\par 13 Het woord, dat de HEERE tot den profeet Jeremia sprak, van de aankomst van Nebukadrezar, den koning van Babel, om Egypteland te slaan.
\par 14 Verkondigt in Egypte, en doet het horen te Migdol; doet het ook horen te Nof en Tachpanhes; zegt: Stelt er u naar, en maakt u gereed, want het zwaard heeft verteerd, wat rondom u is.
\par 15 Waarom zijn uw sterken weggeveegd? Zij stonden niet, omdat hen de HEERE voortdreef.
\par 16 Hij maakte der struikelenden veel; ja, de een viel op den ander; zodat zij zeiden: Staat op en laat ons wederkeren tot ons volk, en tot het land onzer geboorte, vanwege het verdrukkende zwaard.
\par 17 Daar riepen zij: Farao, de koning van Egypte, is maar een gedruis; hij heeft den gezetten tijd laten voorbijgaan.
\par 18 Zo waarachtig als Ik leef, spreekt de Koning, Wiens Naam is HEERE der heirscharen; hij zal voorzeker, als Thabor onder de bergen, en als Karmel bij de zee, aankomen!
\par 19 Maak voor u gereedschap der gevankelijke wegvoering, gij inwoneres, gij dochter van Egypte! want Nof zal ter verwoesting worden, en zal verbrand worden, dat er niemand in wone.
\par 20 Egypte is een zeer schone vaarze; de slachter komt, hij komt van het noorden.
\par 21 Zelfs haar gehuurden in haar midden zijn als gemeste kalveren; maar die hebben zich ook gewend, zij zijn te zamen gevlucht, zij hebben niet gestaan; want de dag huns verderfs is over hen gekomen, de tijd hunner bezoeking.
\par 22 Haar stem zal gaan als van een slang; want zij zullen met krijgsmacht daarhenen trekken, en tot haar met bijlen komen, gelijk houthouwers.
\par 23 Zij hebben haar woud afgehouwen, spreekt de HEERE, hoewel het niet is te onderzoeken; want zij zijn meerder dan de sprinkhanen, zodat men hen niet tellen kan.
\par 24 De dochter van Egypte is beschaamd; zij is gegeven in de hand des volks van het noorden.
\par 25 De HEERE der heirscharen, de God Israels, zegt: Ziet, Ik zal bezoeking doen over de menigte van No, en over Farao, en over Egypte, en over haar goden, en over haar koningen, ja, over Farao, en over degenen, die op hem vertrouwen.
\par 26 En Ik zal hen geven in de hand dergenen, die hunlieder ziel zoeken, en in de hand van Nebukadrezar, den koning van Babel, en in de hand zijner knechten. Maar daarna zal zij bewoond worden als in de dagen van ouds, spreekt de HEERE.
\par 27 Maar gij, Mijn knecht Jakob! vrees niet, en ontzet u niet, o Israel! want zie, Ik zal u verlossen uit verre landen, en uw zaad uit het land hunner gevangenis; en Jakob zal wederkomen, en stil en gerust zijn, en niemand zal hem verschrikken.
\par 28 Gij dan Mijn knecht Jakob! vrees niet, spreekt de HEERE; want Ik ben met u; want Ik zal een voleinding maken met al de heidenen, waarhenen Ik u gedreven zal hebben, doch met u zal Ik geen voleinding maken, maar u kastijden met mate, en u niet gans onschuldig houden.

\chapter{47}

\par 1 Het woord des HEEREN, dat tot den profeet Jeremia geschiedde, tegen de Filistijnen; eer dat Farao Gaza sloeg.
\par 2 Zo zegt de HEERE: Ziet, wateren komen op van het noorden, en zullen worden tot een overlopende beek, en overlopen het land en de volheid van hetzelve, de stad en die daarin wonen; en de mensen zullen schreeuwen, en al de inwoners des lands zullen huilen;
\par 3 Vanwege het geluid van het geklater der hoeven zijner sterke paarden, vanwege het geraas zijner wagenen, en het bulderen zijner raderen; de vaders zien niet om naar de kinderen, vanwege de slappigheid der handen;
\par 4 Vanwege den dag, die er komt om alle Filistijnen te verstoren, om Tyrus en Sidon allen overgeblevenen helper af te snijden; want de HEERE zal de Filistijnen, het overblijfsel des eilands van Kafthor, verstoren.
\par 5 Kaalheid is op Gaza gekomen; Askelon is uitgeroeid, met het overblijfsel huns dals; hoe lang zult gij uzelven insnijdingen maken?
\par 6 O wee, gij zwaard des HEEREN! Hoe lang zult gij niet stil houden? Vaar in uw schede, rust en wees stil!
\par 7 Hoe zoudt gij stil houden? De HEERE heeft toch aan het zwaard bevel gegeven; tegen Askelon en tegen de zeehaven, aldaar heeft Hij het besteld.

\chapter{48}

\par 1 Tegen Moab zegt de HEERE der heirscharen, de God Israels, alzo: Wee over Nebo, want zij is verstoord; Kirjathaim is beschaamd, zij is ingenomen; de stad des hogen vertreks is beschaamd en verschrikt.
\par 2 Moabs roem van Hesbon is er niet meer; zij hebben kwaad tegen haar gedacht, zeggende: Komt, en laat ons haar uitroeien, dat zij geen volk meer zij; ook gij, o Madmen! zult nedergehouwen worden, het zwaard zal achter u heengaan.
\par 3 Er is een stem des gekrijts van Horonaim; verstoring en een grote breuk!
\par 4 Moab is verbroken; haar kleine kinderen hebben een gekrijt laten horen.
\par 5 Want in den opgang van Luhith zal geween bij geween opgaan, want in den afgang van Horonaim hebben Moabs wederpartijders een jammergeschrei gehoord.
\par 6 Vlucht, redt ulieder ziel! en wordt als de heide in de woestijn;
\par 7 Want om uw vertrouwen op uw werken, en op uw schatten, zult gij ook ingenomen worden; en Kamos zal henen uitgaan in gevangenis, zijn priesteren en zijn vorsten te zamen.
\par 8 Want de verstoorder zal komen over elke stad, dat niet een stad ontkomen zal; en het dal zal verderven, en het effen veld verdelgd worden; want de HEERE heeft het gezegd.
\par 9 Geeft Moab vederen, want al vliegende zal zij uitgaan; en haar steden zullen ter verwoesting worden, dat niemand in dezelve wone.
\par 10 Vervloekt zij, die des HEEREN werk bedriegelijk doet; ja, vervloekt zij, die zijn zwaard van het bloed onthoudt!
\par 11 Moab is van zijn jeugd aan gerust geweest, en hij heeft op zijn heffe stil gelegen, en is van vat in vat niet geledigd, en heeft niet gewandeld in gevangenis; daarom is zijn smaak in hem gebleven, en zijn reuk niet veranderd.
\par 12 Daarom, ziet, de dagen komen, spreekt de HEERE, dat Ik hem vreemde gasten zal toeschikken, die hem in vreemde plaatsen zullen voeren, en zijn vaten ledigen, en hunlieder flessen in stukken slaan.
\par 13 En Moab zal beschaamd worden vanwege Kamos, gelijk als het huis Israels beschaamd is geworden vanwege Beth-el, hunlieder vertrouwen.
\par 14 Hoe zult gij zeggen: Wij zijn helden en dappere mannen ten strijde?
\par 15 Moab is verstoord, en uit zijn steden opgegaan, en de keur zijner jongelingen is ter slachting afgegaan, spreekt de Koning, Wiens Naam is HEERE der heirscharen.
\par 16 Moabs verderf is nabij om te komen, en zijn kwaad haast zeer.
\par 17 Beklaagt hem, gij allen, die rondom hem zijt, en allen, die zijn naam kent; zegt: Hoe is de sterke staf, de sierlijke stok verbroken?
\par 18 Daal neder uit uw heerlijkheid, en woon in dorst, gij inwoneres, gij dochter van Dibon! want Moabs verstoorder is tegen u opgetogen, hij heeft uw vestingen verdorven.
\par 19 Sta aan den weg, en zie toe, gij inwoneres van Aroer! Vraag den vluchtenden man en de ontkomene vrouw; zeg: Wat is er geschied?
\par 20 Moab is beschaamd, want hij is verslagen; huilt en krijt! verkondigt te Arnon, dat Moab verstoord is.
\par 21 En het oordeel is gekomen over het vlakke land; over Holon, en over Jahza, en over Mefaath,
\par 22 En over Dibon, en over Nebo, en over Beth-diblathaim,
\par 23 En over Kirjathaim, en over Beth-gamul, en over Beth-meon,
\par 24 En over Kerioth, en over Bozra; ja, over alle steden van Moabs land, die verre en die nabij zijn.
\par 25 Moabs hoorn is afgesneden, en zijn arm verbroken, spreekt de HEERE.
\par 26 Maak hem dronken, omdat hij zich groot gemaakt heeft tegen den HEERE; zo zal Moab met de handen klappen in zijn uitspuwsel, en hij zelf zal ook ter belaching zijn.
\par 27 Want is u niet Israel ter belaching geweest? Was hij onder de dieven gevonden, dat gij u zo bewoogt, van den tijd af, dat uw woorden van hem waren?
\par 28 Verlaat de steden, en woont in de steenrots, gij inwoners van Moab! en wordt gelijk een duif, die in de doorgangen van den mond eens hols nestelt.
\par 29 Wij hebben Moabs hovaardij gehoord (hij is zeer hovaardig), zijn trotsheid, en zijn hovaardij, en zijn hoogmoed, en zijns harten hoogheid.
\par 30 Ik ken zijn verbolgenheid, spreekt de HEERE, maar niet alzo; zijn grendelen doen het zo niet.
\par 31 Daarom zal Ik over Moab huilen, ja, om gans Moab zal Ik krijten; over de lieden van Kir-heres zal men zuchten.
\par 32 Boven het geween van Jaezer zal Ik u bewenen, gij wijnstok van Sibma! uw wijnranken zijn over zee gegaan, zij hebben gereikt tot aan Jaezers zee; maar de verstoorder is gevallen op uw zomervruchten en op uw wijnoogst;
\par 33 Zodat de blijdschap en verheuging uit het vruchtbare veld, namelijk uit Moabs land, weggenomen is; want Ik heb den wijn doen ophouden uit de kuipen; men zal geen druiven treden met vreugdegeschrei; het vreugdegeschrei zal geen vreugdegeschrei zijn.
\par 34 Vanwege Hesbons gekrijt tot Eleale toe, tot Jahaz toe, hebben zij hun stem verheven, van Zoar tot aan Horonaim, die driejarige vaarze; want ook de wateren van Nimrim zullen tot verwoestingen worden.
\par 35 En Ik zal in Moab doen ophouden, spreekt de HEERE, dien, die op de hoogte offert, en die zijn goden rookt.
\par 36 Daarom zal Mijn hart over Moab getier maken als de fluiten; ook zal Mijn hart over de lieden van Kir-heres getier maken als de fluiten, omdat het overschot, dat hij gemaakt had, verloren is.
\par 37 Want alle hoofden zijn kaal, en alle baarden afgekort; op alle handen zijn insnijdingen, en op de lenden is een zak.
\par 38 Op alle daken van Moab, en op al haar straten is overal misbaar; want Ik heb Moab verbroken als een vat, waar men geen lust aan heeft, spreekt de HEERE.
\par 39 Hoe is hij verslagen! zij huilen; hoe heeft Moab den nek met schaamte gewend! Alzo zal Moab allen, die rondom hem zijn, tot belaching en tot een ontzetting worden.
\par 40 Want zo zegt de HEERE: Ziet, hij zal snel vliegen als een arend, en hij zal zijn vleugelen over Moab uitbreiden.
\par 41 Elk een der steden is gewonnen, en elk een der vastigheden is ingenomen; en het hart van Moabs helden zal te dien dage wezen, als het hart ener vrouw, die in nood is.
\par 42 Want Moab zal verdelgd worden, dat hij geen volk zij, omdat hij zich groot gemaakt heeft tegen den HEERE.
\par 43 De vreze, en de kuil, en de strik, over u, gij inwoner van Moab! spreekt de HEERE.
\par 44 Die van de vreze ontvliedt, zal in den kuil vallen, en die uit den kuil opkomt, zal in den strik gevangen worden; want Ik zal over haar, over Moab, het jaar van hunlieder bezoeking brengen, spreekt de HEERE.
\par 45 Die voor des vijands macht vluchtten, bleven staan in de schaduw van Hesbon; maar een vuur is uitgegaan van Hesbon, en een vlam van tussen Sihon, en heeft de hoeken van Moab en den schedel der kinderen van het gedruis verteerd.
\par 46 Wee u, Moab! het volk van Kamos is verloren; want uw zonen zijn weggenomen in gevangenis; ook zijn uw dochters in gevangenis.
\par 47 Maar in het laatste der dagen, zal Ik Moabs gevangenis wenden, spreekt de HEERE. Tot hiertoe is Moabs oordeel.

\chapter{49}

\par 1 Tegen de kinderen Ammons zegt de HEERE alzo: Heeft dan Israel geen kinderen? Heeft hij geen erfgenaam? Waarom is dan Malcham erfgenaam van Gad, en waarom woont zijn volk in deszelfs steden?
\par 2 Daarom ziet, de dagen komen, spreekt de HEERE, dat Ik over Rabba der kinderen Ammons een krijgsgeschrei zal doen horen, en zij zal tot een woesten hoop worden, en haar onderhorige plaatsen zullen met vuur aangestoken worden; en Israel zal erven degenen, die hem geerfd hadden, zegt de HEERE.
\par 3 Huil, o Hesbon! want Ai is verstoord; krijt, gij dochteren van Rabba, gordt zakken aan, drijft misbaar, en loopt om bij de tuinen; want Malcham zal wandelen in gevangenis, zijn priesteren en zijn vorsten te zamen.
\par 4 Wat roemt gij op uw dalen? Uw dal is weggevloten, gij afkerige dochter! die op haar schatten vertrouwt, zeggende: Wie zou tegen mij komen?
\par 5 Ziet, Ik zal vreze over u brengen, spreekt de Heere, de HEERE der heirscharen, van allen, die rondom u zijn, en gijlieden zult, een iegelijk voor zich henen, uitgedreven worden, en niemand zal den omdolende vergaderen.
\par 6 Maar daarna zal Ik de gevangenis der kinderen Ammons wenden, spreekt de HEERE.
\par 7 Tegen Edom zegt de HEERE der heirscharen alzo: Is er dan geen wijsheid meer te Theman? Is de raad vergaan van de verstandigen? Is hunlieder wijsheid onnut geworden?
\par 8 Vliedt, wendt u, woont in diepe plaatsen, gij inwoners van Dedan! want Ik heb Ezau's verderf over hem gebracht, den tijd, dat Ik hem bezocht heb.
\par 9 Zo er wijnlezers tot u gekomen waren, zouden zij niet een nalezing hebben overgelaten? Zo er dieven bij nacht gekomen waren, zouden zij niet verdorven hebben zoveel hun genoeg ware?
\par 10 Maar Ik heb Ezau ontbloot, Ik heb zijn verborgene plaatsen ontdekt, dat hij zich niet zal kunnen versteken; zijn zaad is verstoord, ook zijn broeders, en zijn naburen, en hij is er niet meer.
\par 11 Laat uw wezen achter, en Ik zal hen in het leven behouden, en laat uw weduwen op Mij vertrouwen.
\par 12 Want zo zegt de HEERE: Ziet, degenen, welker oordeel het niet is den beker te drinken, zullen ganselijk drinken; en zoudt gij enigszins onschuldig gehouden worden? Gij zult niet onschuldig worden gehouden, maar gij zult ganselijk drinken.
\par 13 Want Ik heb bij Mijzelven gezworen, spreekt de HEERE, dat Bozra worden zal tot een ontzetting, tot een smaadheid, tot een woestheid, en tot een vloek; en al haar steden zullen worden tot eeuwige woestheden.
\par 14 Ik heb een gerucht gehoord van den HEERE, en er is een gezant geschikt onder de heidenen, om te zeggen: Vergadert u, en komt aan tegen haar, en maakt u op ten strijde.
\par 15 Want zie, Ik heb u klein gemaakt onder de heidenen, veracht onder de mensen.
\par 16 Uw schrikkelijkheid heeft u bedrogen, en de trotsheid uws harten, gij, die woont in de kloven der steenrotsen, die u houdt op de hoogte der heuvelen! Al zoudt gij uw nest zo hoog maken als de arend, zo zal Ik u van daar nederstoten, spreekt de HEERE.
\par 17 Alzo zal Edom worden tot een ontzetting; al wie voorbij haar gaat, zal zich ontzetten, en fluiten over al haar plagen.
\par 18 Gelijk de omkering van Sodom en Gomorra en haar naburen, zal het zijn, zegt de HEERE; niemand zal daar wonen, en geen mensenkind daarin verkeren.
\par 19 Ziet, gelijk een leeuw van de verheffing der Jordaan, zal hij opkomen tegen de sterke woning; want Ik zal hem in een ogenblik daaruit doen lopen; en wie daartoe verkoren is, dien zal Ik tegen haar bestellen; want wie is Mij gelijk, en wie zou Mij dagvaarden, en wie is die herder, die voor Mijn aangezicht bestaan zou?
\par 20 Daarom hoort des HEEREN raadslag, dien Hij over Edom heeft beraadslaagd, en Zijn gedachten, die Hij gedacht heeft over de inwoners van Theman: Zo de geringsten van de kudde hen niet zullen nedertrekken! Indien hij hunlieder woning niet boven hen zal verwoesten!
\par 21 De aarde heeft gebeefd van het geluid huns vals, van het gekrijt, welks geluid gehoord is bij de Schelfzee.
\par 22 Ziet, hij zal opkomen en snel vliegen, als een arend, en zijn vleugelen over Bozra uitbreiden; en het hart van Edoms helden zal te dien dage wezen, als het hart ener vrouw, die in nood is.
\par 23 Tegen Damaskus. Beschaamd is Hamath en Arpad; omdat zij een boos gerucht gehoord hebben, zijn zij gesmolten; bij de zee is bekommernis, men kan er niet rusten.
\par 24 Damaskus is slap geworden, zij heeft zich gewend, om te vluchten, en siddering heeft haar aangegrepen; benauwdheid en smarten als van een barende vrouw hebben haar bevangen;
\par 25 Hoe is de beroemde stad niet gelaten, de stad Mijner vrolijkheid!
\par 26 Daarom zullen haar jongelingen vallen op haar straten; en al haar krijgslieden zullen te dien dage nedergehouwen worden, spreekt de HEERE der heirscharen.
\par 27 En Ik zal een vuur aansteken in den muur van Damaskus, en het zal Benhadads paleizen verteren.
\par 28 Tegen Kedar, en tegen de koninkrijken van Hazor, die Nebukadrezar, de koning van Babel, sloeg, zegt de HEERE alzo: Maakt u op, trekt op tegen Kedar, en verstoort de kinderen van het oosten.
\par 29 Zij zullen hun tenten en hun kudden nemen, hun gordijnen en al hun gereedschap, en hun kemelen voor zich wegnemen; en zij zullen tegen hen uitroepen: Schrik van rondom!
\par 30 Vliedt, zwerft fluks henen weg, woont in diepe plaatsen, gij inwoners van Hazor! spreekt de HEERE; want Nebukadrezar, de koning van Babel, heeft een raadslag tegen ulieden beraadslaagd, en een gedachte tegen hen gedacht.
\par 31 Maakt u op, trekt op tegen het volk, dat rust heeft, dat in zekerheid woont, spreekt de HEERE; dat geen deuren noch grendel heeft, die alleen wonen.
\par 32 En hun kemelen zullen ten roof zijn, en de menigte van hun vee zal ten buit zijn; en Ik zal hen verstrooien in alle winden, te weten degenen, die aan de hoeken afgekort zijn; en Ik zal hunlieder verderf van al zijn zijden aanbrengen, spreekt de HEERE.
\par 33 En Hazor zal worden tot een drakenwoning, een verwoesting tot in eeuwigheid; niemand zal daar wonen, en geen mensenkind daarin verkeren.
\par 34 Het woord des HEEREN, dat tot den profeet Jeremia geschied is tegen Elam, in het begin des koninkrijks van Zedekia, den koning van Juda, zeggende:
\par 35 Zo zegt de HEERE der heirscharen: Ziet, Ik zal verbreken Elams boog, het voornaamste van hunlieder geweld.
\par 36 En Ik zal de vier winden uit de vier hoeken des hemels over Elam aanbrengen, en zal hen in al diezelve winden verstrooien; en er zal geen volk zijn, waarhenen Elams verdrevenen niet zullen komen.
\par 37 En Ik zal Elam versaagd maken voor het aangezicht hunner vijanden, en voor het aangezicht dergenen, die hun ziel zoeken, en zal een kwaad over hen brengen, de hittigheid mijns toorns, spreekt de HEERE; en Ik zal het zwaard achter hen zenden, totdat Ik hen verteerd zal hebben.
\par 38 En Ik zal Mijn troon in Elam stellen; en zal den koning en de vorsten van daar vernielen, spreekt de HEERE;
\par 39 Maar het zal geschieden in het laatste der dagen, dat Ik Elams gevangenis wenden zal, spreekt de HEERE.

\chapter{50}

\par 1 Het woord, dat de HEERE gesproken heeft tegen Babel, tegen het land der Chaldeen, door den dienst van den profeet Jeremia.
\par 2 Verkondigt onder de heidenen, en doet horen, en werpt een banier op, laat horen, verbergt het niet; zegt: Babel is ingenomen, Bel is beschaamd, Merodach is verpletterd, haar afgoden zijn beschaamd, haar drekgoden zijn verpletterd!
\par 3 Want een volk komt tegen haar op van het noorden; dat zal haar land zetten in verwoesting, dat er geen inwoner in zal zijn; van de mensen aan tot de beesten toe zijn zij weggezworven, doorgegaan!
\par 4 In dezelve dagen en ter zelver tijd, spreekt de HEERE, zullen de kinderen Israels komen, zij en de kinderen van Juda te zamen; wandelende en wenende zullen zij henengaan, en den HEERE, hun God, zoeken.
\par 5 Zij zullen naar Sion vragen; op den weg herwaarts zullen hun aangezichten zijn; zij zullen komen en den HEERE toegevoegd worden, met een eeuwig verbond, dat niet zal worden vergeten.
\par 6 Mijn volk waren verloren schapen, hun herders hadden hen verleid, zij hadden hen gevoerd naar de bergen, zij gingen van berg tot heuvel, zij vergaten hun legering.
\par 7 Allen, die hen vonden, aten hen op, en hun wederpartijders zeiden: Wij zullen geen schuld hebben; daarom dat zij gezondigd hebben tegen den HEERE, in de woning der gerechtigheid, ja, tegen den HEERE, de Verwachting hunner vaderen.
\par 8 Vliedt weg uit het midden van Babel, en gaat uit der Chaldeen land; en weest als de bokken voor de kudde henen.
\par 9 Want ziet, Ik zal een verzameling van grote volken uit het land van het noorden verwekken, en tegen Babel opbrengen; die zullen zich tegen haar rusten; van daar zal zij ingenomen worden; hun pijlen zullen zijn als eens kloeken helds, geen zal ledig wederkeren.
\par 10 En Chaldea zal ten roof zijn; allen, die het beroven, zullen verzadigd worden, spreekt de HEERE.
\par 11 Omdat gij u verblijd hebt, omdat gij van vreugde hebt opgesprongen, gij plunderaars Mijner erfenis! omdat gij geil geworden zijt als een grazige vaars, en hebt gebriest als de sterke paarden;
\par 12 Zo is uw moeder zeer beschaamd; die u gebaard heeft, is schaamrood geworden; ziet, zij is geworden de achterste der heidenen, een woestijn, dorheid en wildernis.
\par 13 Vanwege de verbolgenheid des HEEREN zal zij niet bewoond worden, maar zij zal geheel een verwoesting worden; al wie aan Babel voorbijgaat, zal zich ontzetten, en fluiten over al haar plagen.
\par 14 Rust u tegen Babel rondom, gij allen, die den boog spant! schiet in haar, en spaart de pijlen niet; want zij heeft tegen den HEERE gezondigd.
\par 15 Juicht over haar rondom, zij heeft haar hand gegeven; haar fondamenten zijn gevallen, haar muren zijn afgebroken; want dat is des HEEREN wraak, wreekt u aan haar, doet haar, gelijk als zij gedaan heeft!
\par 16 Roeit uit van Babel den zaaier, en dien, die de sikkel handelt in den oogsttijd; laat hen vanwege het verdrukkende zwaard, zich keren, een iegelijk tot zijn volk, en vlieden, een iegelijk naar zijn land.
\par 17 Israel is een verbijsterd lam, dat de leeuwen verjaagd hebben; de eerste, die hem heeft opgegeten, was de koning van Assur, en deze de laatste, Nebukadrezar, de koning van Babel, heeft hem de beenderen verbrijzeld.
\par 18 Daarom, zo zegt de HEERE der heirscharen, de God Israels: Ziet, Ik zal bezoeking doen over den koning van Babel en over zijn land, gelijk als Ik bezoeking gedaan heb over den koning van Assur.
\par 19 En Ik zal Israel weder tot zijn woning brengen, en hij zal weiden op den Karmel en op den Basan; en zijn ziel zal op het gebergte van Efraim en Gilead verzadigd worden.
\par 20 In die dagen en te dier tijd, spreekt de HEERE, zal Israels ongerechtigheid gezocht worden, maar zij zal er niet zijn, en de zonden van Juda, maar zullen niet gevonden worden; want Ik zal ze dengenen vergeven, die Ik zal doen overblijven.
\par 21 Tegen het land Merathaim, trek tegen hetzelve op, en tegen de inwoners van Pekod; verwoest en verban achter hen, spreekt de HEERE, en doe naar alles, wat Ik u geboden heb.
\par 22 Er is een krijgsgeschrei in het land, en een grote breuk.
\par 23 Hoe is de hamer der ganse aarde zo afgehouwen en verbroken! Hoe is Babel geworden tot een ontzetting onder de heidenen.
\par 24 Ik heb u een strik gesteld, dies zijt gij ook gevangen, o Babel! dat gij het niet wist; gij zijt gevonden, en ook gegrepen, omdat gij u tegen den HEERE in strijd gemengd hebt.
\par 25 De HEERE heeft Zijn schatkamer opengedaan, en de instrumenten Zijner gramschap voortgebracht; want dat is een werk van den Heere, den HEERE der heirscharen, in het land der Chaldeen.
\par 26 Komt aan tegen haar van het uiterste, opent haar schuren, vertreedt haar als korenhopen, en verbant ze; laat ze geen overblijfsel hebben.
\par 27 Doodt met het zwaard al haar varren, laat ze afgaan ter slachting; wee over hen, want hun dag is gekomen, de tijd hunner bezoeking!
\par 28 Er is een stem der gevluchten en ontkomenen uit het land van Babel, om in Sion te verkondigen de wraak des HEEREN, onzes Gods, de wraak Zijns tempels.
\par 29 Laat u horen tegen Babel, gij schutters! gij allen, die den boog spant! legert u tegen haar rondom, laat niemand van hen ontkomen; vergeldt haar naar haar werk, doet haar naar alles, wat zij gedaan heeft; want zij heeft trotselijk gehandeld tegen den HEERE, tegen den Heilige Israels.
\par 30 Daarom zullen haar jongelingen vallen op haar straten, en al haar krijgslieden te dien dage uitgeroeid worden, spreekt de HEERE.
\par 31 Ziet, Ik wil aan u, gij trotse! spreekt de Heere, de HEERE der heirscharen; want uw dag is gekomen, de tijd, dat Ik u bezoeken zal.
\par 32 Dan zal de trotse aanstoten en vallen, en er zal niemand zijn, die hem opricht; ja, Ik zal een vuur aansteken in zijn steden, dat zal alle plaatsen rondom hem verteren.
\par 33 Zo zegt de HEERE der heirscharen: De kinderen Israels en de kinderen van Juda zijn te zamen verdrukt geweest; en allen, die hen gevangen hadden, hebben hen vast gehouden; zij hebben hen geweigerd los te laten.
\par 34 Maar hun Verlosser is sterk, HEERE der heirscharen is Zijn Naam; Hij zal hun twist zekerlijk twisten, opdat Hij het land in rust brenge, maar de inwoners van Babel beroere.
\par 35 Het zwaard zal zijn over de Chaldeen, spreekt de HEERE; en over de inwoners van Babel, en over haar vorsten, en over haar wijzen.
\par 36 Het zwaard zal zijn over de leugenaars, dat zij zot worden; het zwaard zal zijn over haar helden, dat zij versagen;
\par 37 Het zwaard zal zijn over zijn paarden en over zijn wagenen, en over den gansen gemengden hoop, die in het midden van hen is, dat zij tot wijven worden; het zwaard zal zijn over haar schatten, dat zij geplunderd worden.
\par 38 Droogte zal zijn over haar wateren, dat zij uitdrogen; want het is een land van gesneden beelden, en zij razen naar de schrikkelijke afgoden.
\par 39 Daarom zo zullen de wilde dieren der woestijnen met de wilde dieren der eilanden daarin wonen; ook zullen de jonge struisen daarin wonen; en men zal er geen verblijf meer hebben in eeuwigheid, en zij zal niet bewoond worden van geslacht tot geslacht.
\par 40 Gelijk God Sodom en Gomorra en haar naburen heeft omgekeerd, spreekt de HEERE, alzo zal niemand aldaar wonen, en geen mensenkind in haar verkeren.
\par 41 Ziet, daar komt een volk uit het noorden; en een grote natie, en geweldige koningen zullen van de zijden der aarde opgewekt worden.
\par 42 Boog en spies zullen zij voeren; wreed zijn zij, en zullen niet barmhartig zijn; hun stem zal bruisen als de zee, en op paarden zullen zij rijden; het is toegerust als een man ten oorlog, tegen u, o dochter van Babel!
\par 43 De koning van Babel heeft hunlieder gerucht gehoord, en zijn handen zijn slap geworden; benauwdheid heeft hem aangegrepen, weedom als van een barende vrouw.
\par 44 Ziet, gelijk een leeuw van de verheffing der Jordaan, zal hij opkomen tegen de sterke woning; want Ik zal hen in een ogenblik daaruit doen lopen; en wie daartoe verkoren is, dien zal Ik tegen haar bestellen; want wie is Mij gelijk, en wie zou Mij dagvaarden? En wie is de herder, die voor Mijn aangezicht bestaan zou?
\par 45 Daarom hoort den raadslag des HEEREN, dien Hij over Babel heeft beraadslaagd, en Zijn gedachten, die Hij gedacht heeft over het land der Chaldeen: Zo de geringsten van de kudde hen niet zullen nedertrekken! Zo hij de woning boven hen niet zal verwoesten!
\par 46 De aarde is bevende geworden van het geluid der inneming van Babel, en het gekrijt is gehoord onder de volken.

\chapter{51}

\par 1 Zo zegt de HEERE: Ziet, Ik zal een verdervenden wind opwekken tegen Babel, en tegen degenen, die daar wonen in het hart van degenen, die tegen Mij opstaan.
\par 2 En Ik zal Babel wanners toeschikken, die haar wannen, en haar land uitledigen zullen; want zij zullen ten dage des kwaads van rondom tegen haar zijn.
\par 3 De schutter spanne zijn boog tegen dien, die spant, en tegen dien, die zich verheft in zijn pantsier; en verschoont haar jongelingen niet, verbant al haar heir;
\par 4 Dat de verslagenen liggen in het land der Chaldeen, en de doorstokenen op haar straten.
\par 5 Want Israel of Juda zal niet in weduwschap gelaten worden van zijn God, van den HEERE der heirscharen (hoewel hunlieder land vol van schuld is), van den Heilige Israels.
\par 6 Vliedt uit het midden van Babel, en redt, een iegelijk zijn ziel; wordt niet uitgeroeid in haar ongerechtigheid; want dit is de tijd der wraak des HEEREN, Die haar de verdienste betaalt.
\par 7 Babel was een gouden beker in de hand des HEEREN, die de ganse aarde dronken maakte; de volken hebben van haar wijn gedronken, daarom zijn de volken dol geworden.
\par 8 Schielijk is Babel gevallen en verbroken; huilt over haar, neemt balsem tot haar pijn, misschien zal zij genezen worden.
\par 9 Wij hebben Babel gemeesterd, maar zij is niet genezen; verlaat haar dan, en laat ons een iegelijk in zijn land trekken; want haar oordeel reikt tot aan den hemel, en is verheven tot aan de bovenste wolken.
\par 10 De HEERE heeft onze gerechtigheden hervoor gebracht; komt en laat ons te Sion het werk des HEEREN, onzes Gods, vertellen!
\par 11 Zuivert de pijlen, rust de schilden volkomenlijk toe; de HEERE heeft den geest der koningen van Medie opgewekt; want Zijn voornemen is tegen Babel, dat Hij haar verderve; want dit is de wraak des HEEREN, de wraak Zijns tempels.
\par 12 Verheft de banier op de muren van Babel, versterkt de wacht, stelt wachters, bereidt de lagen; want gelijk de HEERE heeft voorgenomen, alzo heeft Hij gedaan, wat Hij over de inwoners van Babel gesproken heeft.
\par 13 Gij, die aan vele wateren woont, die machtig zijt van schatten! uw einde is gekomen, de maat uwer gierigheid.
\par 14 De HEERE der heirscharen heeft gezworen bij Zijn ziel: Ofschoon Ik u met mensen als met kevers vervuld heb, nochtans zullen zij elkander een vreugdegeschrei over u toeroepen!
\par 15 Die de aarde gemaakt heeft door Zijn kracht, Die de wereld bereid heeft door Zijn wijsheid, en den hemel uitgebreid door Zijn verstand;
\par 16 Als Hij Zijn stem geeft, zo is er een gedruis van wateren in den hemel, en Hij doet de dampen opklimmen van het einde der aarde; Hij maakt de bliksemen met den regen, en doet den wind voortkomen uit Zijn schatkameren.
\par 17 Een ieder mens is onvernuftig geworden, zodat hij geen wetenschap heeft; een ieder goudsmid is beschaamd van het gesneden beeld; want zijn gegoten beeld is leugen, en er is geen geest in hen.
\par 18 Ijdelheid zijn zij, een werk van verleidingen; ten tijde hunner bezoeking zullen zij vergaan.
\par 19 Jakobs deel is niet gelijk die; want Hij is de Formeerder van alles, en Israel is de roede Zijner erfenis; HEERE der heirscharen is Zijn Naam.
\par 20 Gij zijt Mij een voorhamer, en krijgswapenen; en door u zal Ik volken in stukken slaan, en door u zal Ik koninkrijken verderven.
\par 21 En door u zal Ik in stukken slaan het paard en zijn ruiter; en door u zal Ik in stukken slaan den wagen en zijn ruiter.
\par 22 En door u zal Ik in stukken slaan den man en de vrouw; en door u zal Ik in stukken slaan den oude en den jonge; en door u zal Ik in stukken slaan den jongeling en de jonkvrouw.
\par 23 En door u zal Ik in stukken slaan den herder en zijn kudde; en door u zal Ik in stukken slaan den akkerman en zijn juk ossen; en door u zal Ik in stukken slaan landvoogden en overheden.
\par 24 Maar Ik zal Babel en allen inwoneren van Chaldea vergelden al hun boosheid, die zij gedaan hebben aan Sion, voor ulieder ogen, spreekt de HEERE.
\par 25 Ziet, Ik wil aan u, gij verdervende berg! spreekt de HEERE, gij, die de ganse aarde verderft, en Ik zal Mijn hand tegen u uitstrekken, en u van de steenrotsen afwentelen, en zal u stellen tot een berg des brands.
\par 26 En zij zullen uit u geen steen nemen tot een hoek, ook geen steen tot fondamenten; want gij zult tot eeuwige woestheden zijn, spreekt de HEERE.
\par 27 Verheft de banier in het land, blaast de bazuin onder de heidenen, heiligt de heidenen tegen haar, roept tegen haar bijeen de koninkrijken van Ararat, Minni en Askenaz; bestelt een krijgsoverste tegen haar, brengt paarden opwaarts, als ruige kevers!
\par 28 Heiligt tegen haar de heidenen, de koningen van Medie, haar landvoogden en al haar overheden, ja, het ganse land harer heerschappij.
\par 29 Dan zal het land beven en pijn lijden; want elk een van des HEEREN gedachten staat vast tegen Babel, om Babels land te stellen tot een verwoesting, dat er geen inwoner zij.
\par 30 Babels helden hebben opgehouden te strijden, zij zijn gebleven in de vestingen, hun macht is bezweken, zij zijn tot wijven geworden; zij hebben hun woningen aangestoken, hun grendels zijn verbroken.
\par 31 De loper zal den loper tegemoet lopen, en de kondschapper den kondschapper tegemoet, om den koning van Babel bekend te maken, dat zijn stad van het einde is ingenomen;
\par 32 En dat de veren ingenomen, en de rietpoelen met vuur verbrand zijn; en dat de krijgslieden verbaasd zijn.
\par 33 Want zo zegt de HEERE der heirscharen, de God Israels: De dochter van Babel is als een dorsvloer, het is tijd, dat men ze trede; nog een weinig, dan zal haar de tijd des oogstes overkomen.
\par 34 Nebukadrezar, de koning van Babel, heeft mij opgegeten, hij heeft mij verpletterd, hij heeft mij gesteld als een ledig vat, hij heeft mij verslonden als een draak, hij heeft zijn balg gevuld van mijn lekkernijen; hij heeft mij verdreven.
\par 35 Het geweld, dat mij en mijn vlees is aangedaan, zij op Babel! zegge de inwoneres van Sion; en mijn bloed zij op de inwoners van Chaldea! zegge Jeruzalem.
\par 36 Daarom, zo zegt de HEERE: Ziet, Ik zal uw twist twisten, en uw wraak wreken; en Ik zal haar zee droog maken, en haar springader opdrogen.
\par 37 En Babel zal worden tot steen hopen, een woning der draken, een ontzetting en aanfluiting, dat er geen inwoner zij.
\par 38 Zij zullen te zamen brullen als jonge leeuwen, briesen als leeuwenwelpen.
\par 39 Als zij verhit zijn, zal Ik hun drank opzetten, en zal hen dronken maken, opdat zij opspringen; maar zij zullen een eeuwigen slaap slapen, en niet opwaken, spreekt de HEERE.
\par 40 Ik zal hen afvoeren als lammeren om te slachten, als rammen met bokken.
\par 41 Hoe is Sesach zo veroverd, en de roem der ganse aarde ingenomen! Hoe is Babel geworden tot een ontzetting onder de heidenen!
\par 42 Een zee is over Babel gerezen, door de veelheid harer golven is zij bedekt.
\par 43 Haar steden zijn geworden tot verwoesting, een dor land en wildernis; een land, waarin niemand woont, en waar geen mensenkind doorgaat.
\par 44 En Ik zal bezoeking doen over Bel te Babel, en Ik zal uit zijn muil uithalen, wat hij verslonden heeft; en de heidenen zullen niet meer tot hem toevloeien, want ook Babels muur is gevallen.
\par 45 Gaat uit, Mijn volk, uit het midden van haar, en redt een iegelijk zijn ziel, vanwege de hittigheid van den toorn des HEEREN.
\par 46 En opdat ulieder hart misschien niet week worde, en gij vreest van het gerucht, dat gehoord zal worden in het land; want er zal een gerucht komen in het ene jaar, en daarna een gerucht in het andere jaar; en er zal geweld zijn in het land, heer over heer.
\par 47 Daarom ziet, de dagen komen, dat Ik bezoeking zal doen over de gesneden beelden van Babel; en haar ganse land zal beschaamd worden, en al haar verslagenen zullen in het midden van haar liggen.
\par 48 En de hemel en de aarde, mitsgaders al wat daarin is, zullen juichen over Babel; want van het noorden zullen haar de verstoorders aankomen, spreekt de HEERE.
\par 49 Gelijk Babel geweest is tot een val der verslagenen van Israel, alzo zullen te Babel de verslagenen des gansen lands vallen.
\par 50 Gij ontkomenen van het zwaard, gaat weg, en blijft niet staan; gedenkt des HEEREN van verre, en laat Jeruzalem in ulieder hart opkomen.
\par 51 Gij moogt zeggen: Wij zijn beschaamd geworden, want wij hebben versmaadheid gehoord, schaamroodheid heeft ons aangezicht bedekt; omdat uitlandsen over de heiligdommen van des HEEREN huis gekomen zijn;
\par 52 Daarom ziet, de dagen komen, spreekt de HEERE, dat Ik bezoeking doen zal over haar gesneden beelden; en de dodelijk verwonde zal kermen in haar ganse land.
\par 53 Al klom Babel ten hemel op, en al maakte zij vast de hoogte harer sterkte, zo zullen haar toch verstoorders van Mij overkomen, spreekt de HEERE.
\par 54 Er is een stem des gekrijts uit Babel, en een grote breuk uit het land der Chaldeen.
\par 55 Want de HEERE verstoort Babel, en zal de grootse stem uit haar doen vergaan; want hunlieder golven zullen bruisen als grote wateren; het geruis van hunlieder geluid zal zich verheffen.
\par 56 Want de verstoorder komt over haar, over Babel, en haar helden zullen gevangen worden; hunlieder bogen zijn verbroken; want de HEERE, de God der vergelding, zal hun zekerlijk betalen.
\par 57 En Ik zal haar vorsten, en haar wijzen, haar landvoogden, en haar overheden, en haar helden dronken maken; en zij zullen een eeuwigen slaap slapen, en niet opwaken, spreekt de Koning, Wiens Naam is HEERE der heirscharen.
\par 58 Zo zegt de HEERE der heirscharen: Die brede muur van Babel zal ten enemale ontbloot worden, en haar hoge poorten zullen met vuur aangestoken worden; zodat de volken tevergeefs, en de natien ten vure zullen gearbeid hebben, dat zij mat worden.
\par 59 Het woord, dat de profeet Jeremia beval aan Seraja, den zoon van Nerija, den zoon van Machseja, als hij van Zedekia, den koning van Juda, naar Babel toog, in het vierde jaar zijner regering; en Seraja was een vreemdzaam vorst.
\par 60 Jeremia nu schreef al het kwaad, dat over Babel komen zou, in een boek, te weten al deze woorden, die tegen Babel geschreven zijn.
\par 61 En Jeremia zeide tot Seraja: Als gij te Babel komt, zo zult gij zien en lezen al deze woorden;
\par 62 En gij zult zeggen: O HEERE, Gij hebt over deze plaats gesproken, dat Gij ze zult uitroeien, zodat er geen inwoner in zij, van den mens tot op het beest, maar dat zij worden zal tot eeuwige woestheden.
\par 63 En het zal geschieden, als gij geeindigd zult hebben dit boek te lezen, dan zult gij een steen daaraan binden, en werpen het in het midden van den Frath;
\par 64 En zult zeggen: Alzo zal Babel zinken, en niet weder opkomen, vanwege het kwaad, dat Ik over haar zal brengen, en zij zullen mat worden. Tot hiertoe zijn de woorden van Jeremia.

\chapter{52}

\par 1 Zedekia was een en twintig jaren oud, als hij koning werd, en hij regeerde elf jaren te Jeruzalem; en de naam zijner moeder was Hamutal, een dochter van Jeremia, van Libna.
\par 2 En hij deed, dat kwaad was in de ogen des HEEREN, naar alles, wat Jojakim gedaan had.
\par 3 Want het geschiedde, om den toorn des HEEREN tegen Jeruzalem en Juda, totdat Hij hen van Zijn aangezicht weggeworpen had; en Zedekia rebelleerde tegen den koning van Babel.
\par 4 En het geschiedde in het negende jaar zijner regering, in de tiende maand, op den tienden der maand, dat Nebukadrezar, de koning van Babel, kwam tegen Jeruzalem, hij en zijn ganse heir, en zij legerden zich tegen haar, en zij bouwden tegen haar sterkten rondom.
\par 5 Alzo kwam de stad in belegering, tot in het elfde jaar van den koning Zedekia.
\par 6 In de vierde maand, op den negenden der maand, als de honger in de stad sterk werd, en het volk des lands geen brood had;
\par 7 Toen werd de stad doorgebroken, en al de krijgslieden vloden, en trokken uit des nachts, uit de stad, door den weg der poort tussen de twee muren, die aan des konings hof waren (de Chaldeen nu waren tegen de stad rondom), en zij togen door den weg des vlakken velds.
\par 8 Doch het heir der Chaldeen jaagde den koning na, en zij achterhaalden Zedekia in de vlakke velden van Jericho; en al zijn heir werd van bij hem verstrooid.
\par 9 Zij dan grepen den koning, en voerden hem opwaarts tot den koning van Babel naar Ribla, in het land van Hamath; die sprak oordelen tegen hem.
\par 10 En de koning van Babel slachtte de zonen van Zedekia voor zijn ogen; en hij slachtte ook al de vorsten van Juda te Ribla.
\par 11 En hij verblindde de ogen van Zedekia, en hij bond hem met twee koperen ketenen; alzo bracht hem de koning van Babel naar Babel, en stelde hem in het gevangenhuis, tot den dag zijns doods toe.
\par 12 Daarna, in de vijfde maand, op den tienden der maand (dit jaar was het negentiende jaar van den koning Nebukadrezar, den koning van Babel), als Nebuzaradan, de overste der trawanten, die voor het aangezicht des konings van Babel stond, te Jeruzalem gekomen was;
\par 13 Zo verbrandde hij het huis des HEEREN en het huis des konings; mitsgaders alle huizen van Jeruzalem en alle huizen der groten verbrandde hij met vuur.
\par 14 En het ganse heir der Chaldeen, dat met den overste der trawanten was, brak alle muren van Jeruzalem rondom af.
\par 15 Van de armsten nu des volks en het overige des volks, die in de stad overgelaten waren, en de afvalligen, die tot den koning van Babel gevallen waren, en het overige der menigte, voerde Nebuzaradan, de overste der trawanten, gevankelijk weg.
\par 16 Maar van de armsten des lands liet Nebuzaradan, de overste der trawanten, enigen over tot wijngaardeniers en tot akkerlieden.
\par 17 Verder braken de Chaldeen de koperen pilaren, die in het huis des HEEREN waren, en de stellingen, en de koperen zee, die in het huis des HEEREN was; en zij voerden al het koper daarvan naar Babel.
\par 18 Ook namen zij de potten en de schoffelen, en de gaffelen, en de sprengbekkens, en de rookschalen, en al de koperen vaten, waar men den dienst mede deed.
\par 19 En de overste der trawanten nam weg de schalen, en de wierookvaten, en de sprengbekkens, en de potten, en de kandelaars, en de rookschalen, en de kroezen; wat geheel goud, en wat geheel zilver was.
\par 20 De twee pilaren, de ene zee, en de twaalf koperen runderen, die in de plaats der stellingen waren, die de koning Salomo voor het huis des HEEREN gemaakt had; het koper daarvan, te weten van al deze vaten, was zonder gewicht.
\par 21 Aangaande de pilaren, achttien ellen was de hoogte eens pilaars, en een draad van twaalf ellen omving hem; en zijn dikte was vier vingeren, en hij was hol.
\par 22 En het kapiteel daarop was koper, en de hoogte des enen kapiteels was vijf ellen, en een net, en granaatappelen op het kapiteel rondom, alles koper; en dezen gelijk had de andere pilaar, met granaatappelen.
\par 23 En de granaatappelen waren zes en negentig, gezet naar den wind; alle granaatappelen waren honderd, over het net rondom.
\par 24 Ook nam de overste der trawanten Seraja, den hoofdpriester, en Zefanja, den tweeden priester, en de drie dorpelbewaarders.
\par 25 En uit de stad nam hij een hoveling, die over de krijgslieden gesteld was, en zeven mannen uit degenen, die des konings aangezicht zagen, die in de stad gevonden werden, mitsgaders den oversten schrijver des heirs, die het volk des lands ten oorlog opschreef, en zestig mannen van het volk des lands, die in het midden der stad gevonden werden.
\par 26 Als Nebuzaradan, de overste der trawanten, dezen genomen had, zo bracht hij hen tot den koning van Babel naar Ribla.
\par 27 En de koning van Babel sloeg hen en doodde hen te Ribla, in het land van Hamath. Alzo werd Juda uit zijn land gevankelijk weggevoerd.
\par 28 Dit is het volk, dat Nebukadrezar gevankelijk heeft weggevoerd; in het zevende jaar, drie duizend drie en twintig Joden;
\par 29 In het achttiende jaar van Nebukadrezar, voerde hij gevankelijk weg achthonderd twee en dertig zielen uit Jeruzalem;
\par 30 In het drie en twintigste jaar van Nebukadrezar voerde Nebuzaradan, de overste der trawanten, gevankelijk weg van de Joden zevenhonderd vijf en veertig zielen. Alle zielen zijn vier duizend en zeshonderd.
\par 31 Het geschiedde daarna, in het zeven en dertigste jaar der gevankelijke wegvoering van Jojachin, den koning van Juda, in de twaalfde maand, op den vijf en twintigsten der maand, dat Evilmerodach, de koning van Babel, in het eerste jaar zijns koninkrijks, het hoofd van Jojachin, den koning van Juda, verhief, en hem uit het gevangenhuis uitbracht.
\par 32 En hij sprak vriendelijk met hem, en stelde zijn stoel boven den stoel der koningen, die bij hem te Babel waren.
\par 33 En hij veranderde de klederen zijner gevangenis; en hij at geduriglijk brood voor zijn aangezicht, al de dagen zijns levens.
\par 34 En aangaande zijn tering, een gedurige tering werd hem van den koning van Babel gegeven, elk dagelijks bestemde deel op zijn dag, tot op den dag zijns doods, al de dagen zijns levens.


\end{document}