\begin{document}

\title{Ezekiel}


\chapter{1}

\par 1 Im dreißigsten Jahr, am fünften Tage des vierten Monats, da ich war unter den Gefangenen am Wasser Chebar, tat sich der Himmel auf, und Gott zeigte mir Gesichte.
\par 2 Derselbe fünfte Tag des Monats war eben im fünften Jahr, nachdem Jojachin, der König Juda's, war gefangen weggeführt.
\par 3 Da geschah das Wort des HERRN zu Hesekiel, dem Sohn Busis, dem Priester, im Lande der Chaldäer, am Wasser Chebar; daselbst kam die Hand des HERRN über ihn.
\par 4 Und ich sah, und siehe, es kam ein ungestümer Wind von Mitternacht her mit einer großen Wolke voll Feuer, das allenthalben umher glänzte; und mitten in dem Feuer war es lichthell.
\par 5 Und darin war es gestaltet wie vier Tiere, und dieselben waren anzusehen wie Menschen.
\par 6 Und ein jegliches hatte vier Angesichter und vier Flügel.
\par 7 Und ihre Beine standen gerade, und ihre Füße waren gleich wie Rinderfüße und glänzten wie helles glattes Erz.
\par 8 Und sie hatten Menschenhände unter ihren Flügeln an ihren vier Seiten; denn sie hatten alle vier ihre Angesichter und ihre Flügel.
\par 9 Und je einer der Flügel rührte an den andern; und wenn sie gingen, mußten sie nicht herumlenken, sondern wo sie hin gingen, gingen sie stracks vor sich.
\par 10 Ihre Angesichter waren vorn gleich einem Menschen, und zur rechten Seite gleich einem Löwen bei allen vieren, und zur linken Seite gleich einem Ochsen bei allen vieren, und hinten gleich einem Adler bei allen vieren.
\par 11 Und ihre Angesichter und Flügel waren obenher zerteilt, daß je zwei Flügel zusammenschlugen, und mit zwei Flügeln bedeckten sie ihren Leib.
\par 12 Wo sie hin gingen, da gingen sie stracks vor sich, sie gingen aber, wo der sie hin trieb, und mußten nicht herumlenken, wenn sie gingen.
\par 13 Und die Tiere waren anzusehen wie feurige Kohlen, die da brennen, und wie Fackeln; und das Feuer fuhr hin zwischen den Tieren und gab einen Glanz von sich, und aus dem Feuer gingen Blitze.
\par 14 Die Tiere aber liefen hin und her wie der Blitz.
\par 15 Als ich die Tiere so sah, siehe, da stand ein Rad auf der Erde bei den vier Tieren und war anzusehen wie vier Räder.
\par 16 Und die Räder waren wie Türkis und waren alle vier eins wie das andere, und sie waren anzusehen, als wäre ein Rad im andern.
\par 17 Wenn sie gehen wollten, konnten sie nach allen ihren vier Seiten gehen und sie mußten nicht herumlenken, wenn sie gingen.
\par 18 Ihre Felgen und Höhe waren schrecklich; und ihre Felgen waren voller Augen um und um an allen vier Rädern.
\par 19 Auch wenn die vier Tiere gingen, so gingen die Räder auch neben ihnen; und wenn die Tiere sich von der Erde emporhoben, so hoben sich die Räder auch empor.
\par 20 Wo der Geist sie hin trieb, da gingen sie hin, und die Räder hoben sich neben ihnen empor; denn es war der Geist der Tiere in den Rädern.
\par 21 Wenn sie gingen, so gingen diese auch; wenn sie standen, so standen diese auch; und wenn sie sich emporhoben von der Erde, so hoben sich auch die Räder neben ihnen empor; denn es war der Geist der Tiere in den Rädern.
\par 22 Oben aber über den Tieren war es gestaltet wie ein Himmel, wie ein Kristall, schrecklich, gerade oben über ihnen ausgebreitet,
\par 23 daß unter dem Himmel ihre Flügel einer stracks gegen den andern standen, und eines jeglichen Leib bedeckten zwei Flügel.
\par 24 Und ich hörte die Flügel rauschen wie große Wasser und wie ein Getön des Allmächtigen, wenn sie gingen, und wie ein Getümmel in einem Heer. Wenn sie aber still standen, so ließen sie die Flügel nieder.
\par 25 Und wenn sie stillstanden und die Flügel niederließen, so donnerte es in dem Himmel oben über ihnen.
\par 26 Und über dem Himmel, so oben über ihnen war, war es gestaltet wie ein Saphir, gleichwie ein Stuhl; und auf dem Stuhl saß einer gleichwie ein Mensch gestaltet.
\par 27 Und ich sah, und es war lichthell, und inwendig war es gestaltet wie ein Feuer um und um. Von seinen Lenden überwärts und unterwärts sah ich's wie Feuer glänzen um und um.
\par 28 Gleichwie der Regenbogen sieht in den Wolken, wenn es geregnet hat, also glänzte es um und um. Dies war das Ansehen der Herrlichkeit des HERRN. Und da ich's gesehen hatte, fiel ich auf mein Angesicht und hörte einen reden.

\chapter{2}

\par 1 Und er sprach zu mir: Du Menschenkind, tritt auf deine Füße, so will ich mit dir reden.
\par 2 Und da er so mit mir redete, ward ich erquickt und trat auf meine Füße und hörte dem zu, der mit mir redete.
\par 3 Und er sprach zu mir: Du Menschenkind, ich sende dich zu den Kindern Israel, zu dem abtrünnigen Volk, so von mir abtrünnig geworden sind. Sie samt ihren Vätern haben bis auf diesen heutigen Tag wider mich getan.
\par 4 Aber die Kinder, zu welchen ich dich sende, haben harte Köpfe und verstockte Herzen. Zu denen sollst du sagen: So spricht der HERR HERR!
\par 5 Sie gehorchen oder lassen's. Es ist wohl ein ungehorsames Haus; dennoch sollen sie wissen, daß ein Prophet unter ihnen ist.
\par 6 Und du Menschenkind, sollst dich vor ihnen nicht fürchten noch vor ihren Worten fürchten. Es sind wohl widerspenstige und stachlige Dornen bei dir, und du wohnst unter Skorpionen; aber du sollst dich nicht fürchten vor ihren Worten noch vor ihrem Angesicht dich entsetzen, ob sie wohl ein ungehorsames Haus sind,
\par 7 sondern du sollst ihnen meine Worte sagen, sie gehorchen oder lassen's; denn es ist ein ungehorsames Volk.
\par 8 Aber du, Menschenkind, höre du, was ich dir sage, und sei nicht ungehorsam, wie das ungehorsame Haus ist. Tue deinen Mund auf und iß, was ich dir geben werde.
\par 9 Und ich sah, und siehe, da war eine Hand gegen mich ausgestreckt, die hatte einen zusammengelegten Brief;
\par 10 den breitete sie aus vor mir, und er war beschrieben auswendig und inwendig, und stand darin geschrieben Klage, Ach und Wehe.

\chapter{3}

\par 1 Und er sprach zu mir: Du Menschenkind, iß, was vor dir ist, iß diesen Brief, und gehe hin und predige dem Hause Israel!
\par 2 Da tat ich meinen Mund auf, und er gab mir den Brief zu essen
\par 3 und sprach zu mir: Du Menschenkind, du mußt diesen Brief, den ich dir gebe, in deinen Leib essen und deinen Bauch damit füllen. Da aß ich ihn, und er war in meinem Munde so süß wie Honig.
\par 4 Und er sprach zu mir: Du Menschenkind, gehe hin zum Hause Israel und predige ihnen meine Worte.
\par 5 Denn ich sende dich ja nicht zu einem Volk, das eine fremde Rede und unbekannte Sprache hat, sondern zum Hause Israel;
\par 6 ja, freilich nicht zu großen Völkern, die fremde Rede und unbekannte Sprache haben, welcher Worte du nicht verstehen könntest. Und wenn ich dich gleich zu denselben sendete, würden sie dich doch gern hören.
\par 7 Aber das Haus Israel will dich nicht hören, denn sie wollen mich selbst nicht hören; denn das ganze Haus Israel hat harte Stirnen und verstockte Herzen.
\par 8 Siehe, ich habe dein Angesicht hart gemacht gegen ihr Angesicht und deine Stirn gegen ihre Stirn.
\par 9 Ja, ich habe deine Stirn so hart wie ein Demant, der härter ist denn ein Fels, gemacht. Darum fürchte dich nicht, entsetze dich auch nicht vor ihnen, daß sie so ein ungehorsames Haus sind.
\par 10 Und er sprach zu mir: Du Menschenkind, alle meine Worte, die ich dir sage, die fasse zu Herzen und nimm sie zu Ohren!
\par 11 Und gehe hin zu den Gefangenen deines Volks und predige ihnen und sprich zu ihnen: So spricht der HERR HERR! sie hören's oder lassen's.
\par 12 Und ein Wind hob mich auf, und ich hörte hinter mir ein Getön wie eines großen Erdbebens: Gelobt sei die Herrlichkeit des HERRN an ihrem Ort!
\par 13 Und war ein Rauschen von den Flügeln der Tiere, die aneinander schlugen, und auch das Rasseln der Räder, so hart bei ihnen waren, und das Getön eines großen Erdbebens.
\par 14 Da hob mich der Wind auf und führte mich weg. Und ich fuhr dahin in bitterem Grimm, und des HERRN Hand hielt mich fest.
\par 15 Und ich kam zu den Gefangenen, die am Wasser Chebar wohnten, gen Thel-Abib, und setzte mich zu ihnen, die da saßen, und blieb daselbst unter ihnen sieben Tage ganz traurig.
\par 16 Und da die sieben Tage um waren, geschah des HERRN Wort zu mir und sprach:
\par 17 Du Menschenkind, ich habe dich zum Wächter gesetzt über das Haus Israel; du sollst aus meinem Munde das Wort hören und sie von meinetwegen warnen.
\par 18 Wenn ich dem Gottlosen sage: Du mußt des Todes sterben, und du warnst ihn nicht und sagst es ihm nicht, damit sich der Gottlose vor seinem gottlosen Wesen hüte, auf daß er lebendig bleibe: so wird der Gottlose um seiner Sünde willen sterben; aber sein Blut will ich von deiner Hand fordern.
\par 19 Wo du aber den Gottlosen warnst und er sich nicht bekehrt von seinem gottlosen Wesen und Wege, so wird er um seiner Sünde willen sterben; aber du hast deine Seele errettet.
\par 20 Und wenn sich ein Gerechter von seiner Gerechtigkeit wendet und tut Böses, so werde ich ihn lassen anlaufen, daß er muß sterben. Denn weil du ihn nicht gewarnt hast, wird er um seiner Sünde willen sterben müssen, und seine Gerechtigkeit, die er getan, wird nicht angesehen werden; aber sein Blut will ich von deiner Hand fordern.
\par 21 Wo du aber den Gerechten warnst, daß er nicht sündigen soll, und er sündigt auch nicht, so soll er leben, denn er hat sich warnen lassen; und du hast deine Seele errettet.
\par 22 Und daselbst kam des HERRN Hand über mich, und er sprach zu mir: Mache dich auf und gehe hinaus ins Feld; da will ich mit dir reden.
\par 23 Und ich machte mich auf und ging hinaus ins Feld; und siehe, da stand die Herrlichkeit des HERR daselbst, gleichwie ich sie am Wasser Chebar gesehen hatte; und ich fiel nieder auf mein Angesicht.
\par 24 Und ich ward erquickt und trat auf meine Füße. Und er redete mit mir und sprach zu mir: Gehe hin und verschließ dich in deinem Hause!
\par 25 Und du, Menschenkind, siehe, man wird dir Stricke anlegen und dich damit binden, daß du nicht ausgehen sollst unter sie.
\par 26 Und ich will dir die Zunge an deinem Gaumen kleben lassen, daß du verstummen sollst und nicht mehr sie Strafen könnest; denn es ist ein ungehorsames Haus.
\par 27 Wenn ich aber mit dir reden werde, will ich dir den Mund auftun, daß du zu ihnen sagen sollst: So spricht der HERR HERR! Wer's hört, der höre es; wer's läßt, der lasse es; denn es ist ein ungehorsames Haus.

\chapter{4}

\par 1 Und du, Menschenkind, nimm einen Ziegel; den lege vor dich und entwirf darauf die Stadt Jerusalem
\par 2 und mache eine Belagerung darum und baue ein Bollwerk darum und schütte einen Wall darum und mache ein Heerlager darum und stelle Sturmböcke rings um sie her.
\par 3 Vor dich aber nimm eine eiserne Pfanne; die laß eine eiserne Mauer sein zwischen dir und der Stadt, und richte dein Angesicht gegen sie und belagere sie. Das sei ein Zeichen dem Hause Israel.
\par 4 Du sollst dich auch auf deine linke Seite legen und die Missetat des Hauses Israel auf dieselbe legen; soviel Tage du darauf liegst, so lange sollst du auch ihre Missetat tragen.
\par 5 Ich will dir aber die Jahre ihrer Missetat zur Anzahl der Tage machen, nämlich dreihundertundneunzig Tage; so lange sollst du die Missetat des Hauses Israel tragen.
\par 6 Und wenn du solches ausgerichtet hast, sollst du darnach dich auf deine rechte Seite legen und sollst tragen die Missetat des Hauses Juda vierzig Tage lang; denn ich gebe dir hier auch je einen Tag für ein Jahr.
\par 7 Und richte dein Angesicht und deinen bloßen Arm wider das belagerte Jerusalem und weissage wider dasselbe.
\par 8 Und sieh, ich will dir Stricke anlegen, daß du dich nicht wenden könnest von einer Seite zur andern, bis du die Tage deiner Belagerung vollendet hast.
\par 9 So nimm nun zu dir Weizen, Gerste, Bohnen, Linsen, Hirse und Spelt und tue alles in ein Faß und mache dir Brot daraus, soviel Tage du auf deiner Seite liegst, daß du dreihundertundneunzig Tage daran zu essen hast,
\par 10 also daß deine Speise, die du täglich essen sollst, sei zwanzig Lot nach dem Gewicht. Solches sollst du von einer Zeit zur andern essen.
\par 11 Das Wasser sollst du auch nach dem Maß trinken, nämlich das sechste Teil vom Hin, und sollst solches auch von einer Zeit zur andern trinken.
\par 12 Gerstenkuchen sollst du essen, die du vor ihren Augen auf Menschenmist backen sollst.
\par 13 Und der HERR sprach: Also müssen die Kinder Israel ihr unreines Brot essen unter den Heiden, dahin ich sie verstoßen werde.
\par 14 Ich aber sprach: Ach HERR HERR! siehe, meine Seele ist noch nie unrein geworden; denn ich habe von meiner Jugend auf bis auf diese Zeit kein Aas oder Zerrissenes gegessen, und ist nie unreines Fleisch in meinen Mund gekommen.
\par 15 Er aber sprach zu mir: Siehe, ich will dir Kuhmist für Menschenmist zulassen, darauf du dein Brot machen sollst.
\par 16 Und sprach zu mir: Du Menschenkind, siehe, ich will den Vorrat des Brots zu Jerusalem wegnehmen, daß sie das Brot essen müssen nach dem Gewicht und mit Kummer, und das Wasser nach dem Maß mit Kummer trinken,
\par 17 darum daß es an Brot und Wasser mangeln und einer mit dem andern trauern wird und sie in ihrer Missetat verschmachten sollen.

\chapter{5}

\par 1 Und du, Menschenkind, nimm ein Schwert, scharf wie ein Schermesser, und fahr damit über dein Haupt und deinen Bart und nimm eine Waage und teile das Haar damit.
\par 2 Das eine dritte Teil sollst du mit Feuer verbrennen mitten in der Stadt, wenn die Tage der Belagerung um sind; das andere dritte Teil nimm und schlag's mit dem Schwert ringsumher; das letzte dritte Teil streue in den Wind, daß ich das Schwert hinter ihnen her ausziehe.
\par 3 Nimm aber ein klein wenig davon und binde es in deinen Mantelzipfel.
\par 4 Und nimm wiederum etliches davon und wirf's in ein Feuer und verbrenne es mit Feuer; von dem soll ein Feuer auskommen über das ganze Haus Israel.
\par 5 So spricht der HERR HERR: Das ist Jerusalem, das ich mitten unter die Heiden gesetzt habe und ringsherum Länder.
\par 6 Aber es hat mein Gesetz verwandelt in gottlose Lehre mehr denn die Länder, so ringsherum liegen. Denn sie verwerfen mein Gesetz und wollen nicht nach meinen Rechten leben.
\par 7 Darum spricht der HERR also: Weil ihr's mehr macht denn die Heiden, so um euch her sind, und nach meinen Geboten nicht lebt und nach meinen Rechten nicht tut, sondern nach der Heiden Weise tut, die um euch her sind,
\par 8 so spricht der HERR HERR also: Siehe, ich will auch an dich und will Recht über dich gehen lassen, daß die Heiden zusehen sollen;
\par 9 und will also mit dir umgehen, wie ich nie getan habe und hinfort nicht tun werde, um aller deiner Greuel willen:
\par 10 daß in dir die Väter ihre Kinder und die Kinder ihre Väter fressen sollen; und will solch Recht über dich gehen lassen, daß alle deine übrigen sollen in alle Winde zerstreut werden.
\par 11 Darum, so wahr als ich lebe, spricht der HERR HERR, weil du mein Heiligtum mit allen deinen Greueln und Götzen verunreinigt hast, will ich dich auch zerschlagen, und mein Auge soll dein nicht schonen, und ich will nicht gnädig sein.
\par 12 Es soll ein drittes Teil an der Pestilenz sterben und durch Hunger alle werden in dir, und das andere dritte Teil durchs Schwert fallen rings um dich her; und das letzte dritte Teil will ich in alle Winde zerstreuen und das Schwert hinter ihnen her ausziehen.
\par 13 Also soll mein Zorn vollendet und mein Grimm an ihnen ausgerichtet werden, daß ich meinen Mut kühle; und sie sollen erfahren, daß ich, der HERR, in meinem Eifer geredet habe, wenn ich meine Grimm an ihnen ausgerichtet habe.
\par 14 Ich will dich zur Wüste und zur Schmach setzen vor den Heiden, so um dich her sind, vor den Augen aller, die vorübergehen.
\par 15 Und sollst eine Schmach, Hohn, Beispiel und Wunder sein allen Heiden, die um dich her sind, wenn ich über dich das Recht gehen lasse mit Zorn, Grimm und zornigem Schelten (das sage ich, der HERR)
\par 16 und wenn ich böse Pfeile des Hungers unter sie schießen werde, die da schädlich sein sollen, und ich sie ausschießen werde, euch zu verderben, und den Hunger über euch immer größer werden lasse und den Vorrat des Brots wegnehme.
\par 17 Ja, Hunger und böse, wilde Tiere will ich unter euch schicken, die sollen euch kinderlos machen; und soll Pestilenz und Blut unter dir umgehen, und ich will das Schwert über dich bringen. Ich, der HERR, habe es gesagt.

\chapter{6}

\par 1 Und des HERRN Wort geschah zu mir und sprach:
\par 2 Du Menschenkind, kehre dein Angesicht wider die Berge Israels und weissage wider sie
\par 3 und sprich: Ihr Berge Israels, hört das Wort des HERRN HERRN! So spricht der HERR HERR zu den Bergen und Hügeln, zu den Bächen und Tälern: Siehe, ich will das Schwert über euch bringen und eure Höhen zerstören,
\par 4 daß eure Altäre verwüstet und euer Sonnensäulen zerbrochen werden, und will eure Erschlagenen vor eure Bilder werfen;
\par 5 ja, ich will die Leichname der Kinder Israel vor ihre Bilder hinwerfen und will ihre Gebeine um eure Altäre her zerstreuen.
\par 6 Wo ihr wohnet, da sollen die Städte wüst und die Höhen zur Einöde werden; denn man wird eure Altäre wüst und zur Einöde machen und eure Götzen zerbrechen und zunichte machen und eure Sonnensäulen zerschlagen und eure Machwerke vertilgen.
\par 7 Und sollen Erschlagene unter euch daliegen, daß ihr erfahrt, ich sei der HERR.
\par 8 Ich will aber etliche von euch übrigbleiben lassen, die dem Schwert entgehen unter den Heiden, wenn ich euch in die Länder zerstreut habe.
\par 9 Diese eure Entronnenen werden dann an mich gedenken unter den Heiden, da sie gefangen sein müssen, wenn ich ihr abgöttisches Herz, so von mir gewichen, und ihre abgöttischen Augen, so nach ihren Götzen gesehen, zerschlagen habe; und es wird sie gereuen die Bosheit, die sie durch alle ihre Greuel begangen haben;
\par 10 und sie sollen erfahren, daß ich der HERR sei und nicht umsonst geredet habe, solches Unglück ihnen zu tun.
\par 11 So spricht der HERR HERR: Schlage deine Hände zusammen und stampfe mit deinem Fuß und sprich: Weh über alle Greuel der Bosheit im Hause Israel, darum sie durch Schwert, Hunger und Pestilenz fallen müssen!
\par 12 Wer fern ist, wird an der Pestilenz sterben, und wer nahe ist, wird durchs Schwert fallen; wer aber übrigbleibt und davor behütet ist, wird Hungers sterben. Also will ich meinen Grimm unter ihnen vollenden,
\par 13 daß ihr erfahren sollt, ich sei der HERR, wenn ihre Erschlagenen unter ihren Götzen liegen werden um ihre Altäre her, oben auf allen Hügeln und oben auf allen Bergen und unter allen grünen Bäumen und unter allen dichten Eichen, an welchen Orten sie allerlei Götzen süßes Räuchopfer taten.
\par 14 Ich will meine Hand wider sie ausstrecken und das Land wüst und öde machen von der Wüste an bis gen Dibla, überall, wo sie wohnen; und sie sollen erfahren, daß ich der HERR sei.

\chapter{7}

\par 1 Und des HERRN Wort geschah zu mir und sprach:
\par 2 Du Menschenkind, so spricht der HERR HERR vom Lande Israel: Das Ende kommt, das Ende über alle vier Örter des Landes.
\par 3 Nun kommt das Ende über dich; denn ich will meinen Grimm über dich senden und will dich richten, wie du es verdient hast, und will dir geben, was allen deinen Greueln gebührt.
\par 4 Mein Auge soll dein nicht schonen noch übersehen; sondern ich will dir geben, wie du verdient hast, und deine Greuel sollen unter dich kommen, daß ihr erfahren sollt, ich sei der HERR.
\par 5 So spricht der HERR HERR: Siehe, es kommt ein Unglück über das andere!
\par 6 Das Ende kommt, es kommt das Ende, es ist erwacht über dich; siehe, es kommt!
\par 7 Es geht schon auf und bricht daher über dich, du Einwohner des Landes; die Zeit kommt, der Tag des Jammers ist nahe, da kein Singen auf den Bergen sein wird.
\par 8 Nun will ich bald meinen Grimm über dich schütten und meinen Zorn an dir vollenden und will dich richten, wie du verdient hast, und dir geben, was deinen Greueln allen gebührt.
\par 9 Mein Auge soll dein nicht schonen, und ich will nicht gnädig sein; sondern will dir geben, wie du verdient hast, und deine Greuel sollen unter dich kommen, daß ihr erfahren sollt, ich sei der HERR, der euch schlägt.
\par 10 Siehe, der Tag, siehe, er kommt daher, er bricht an; die Rute blüht, und der Stolze grünt.
\par 11 Der Tyrann hat sich aufgemacht zur Rute über die Gottlosen, daß nichts von ihnen noch von ihrem Volk noch von ihrem Haufen Trost haben wird.
\par 12 Es kommt die Zeit, der Tag naht herzu! Der Käufer freue sich nicht, und der Verkäufer trauere nicht; denn es kommt der Zorn über all ihren Haufen.
\par 13 Darum soll der Verkäufer nach seinem verkauften Gut nicht wieder trachten; denn wer da lebt, der wird's haben. Denn die Weissagung über all ihren Haufen wird nicht zurückkehren; keiner wird sein Leben erhalten, um seiner Missetat willen.
\par 14 Laßt sie die Posaune nur blasen und alles zurüsten, es wird doch niemand in den Krieg ziehen; denn mein Grimm geht über all ihren Haufen.
\par 15 Draußen geht das Schwert; drinnen geht Pestilenz und Hunger. Wer auf dem Felde ist, der wird vom Schwert sterben; wer aber in der Stadt ist, den wird Pestilenz und Hunger fressen.
\par 16 Und welche unter ihnen entrinnen, die müssen auf dem Gebirge sein, und wie die Tauben in den Gründen, die alle untereinander girren, ein jeglicher um seiner Missetat willen.
\par 17 Aller Hände werden dahinsinken, und aller Kniee werden so ungewiß stehen wie Wasser;
\par 18 und werden Säcke um sich gürten und mit Furcht überschüttet sein, und aller Angesichter werden jämmerlich sehen und aller Häupter kahl sein.
\par 19 Sie werden ihr Silber hinaus auf die Gassen werfen und ihr Gold wie Unflat achten; denn ihr Silber und Gold wird sie nicht erretten am Tage des Zorns des HERRN. Und sie werden ihre Seele davon nicht sättigen noch ihren Bauch davon füllen; denn es ist ihnen gewesen ein Anstoß zu ihrer Missetat.
\par 20 Sie haben aus ihren edlen Kleinoden, damit sie Hoffart trieben, Bilder ihrer Greuel und Scheuel gemacht; darum will ich's ihnen zum Unflat machen
\par 21 und will's Fremden in die Hände geben, daß sie es rauben, und den Gottlosen auf Erden zur Ausbeute, daß sie es entheiligen sollen.
\par 22 Ich will mein Angesicht davon kehren, daß sie meinen Schatz entheiligen; ja, Räuber sollen darüber kommen und es entheiligen.
\par 23 Mache Ketten; denn das Land ist voll Blutschulden und die Stadt voll Frevels.
\par 24 So will ich die Ärgsten unter den Heiden kommen lassen, daß sie sollen ihre Häuser einnehmen, und will der Hoffart der Gewaltigen ein Ende machen und ihre Heiligtümer entheiligen.
\par 25 Der Ausrotter kommt; da werden sie Frieden suchen, und wird keiner dasein.
\par 26 Ein Unfall wird über den andern kommen, ein Gerücht über das andere. So werden sie dann ein Gesicht bei den Propheten suchen; auch wird weder Gesetz bei den Priestern noch Rat bei den Alten mehr sein.
\par 27 Der König wird betrübt sein, und die Fürsten werden in Entsetzen gekleidet sein, und die Hände des Volkes im Lande werden verzagt sein. Ich will mit ihnen umgehen, wie sie gelebt haben, und will sie richten, wie sie verdient haben, daß sie erfahren sollen, ich sei der HERR.

\chapter{8}

\par 1 Und es begab sich im sechsten Jahr, am fünften Tage des Sechsten Monats, daß ich in meinem Hause und die Alten aus Juda saßen vor mir; daselbst fiel die Hand des HERRN HERRN auf mich.
\par 2 Und siehe, ich sah, daß es von seinen Lenden herunterwärts war gleichwie Feuer; aber oben über seinen Lenden war es lichthell;
\par 3 und er reckte aus gleichwie eine Hand und ergriff mich bei dem Haar meines Hauptes. Da führte mich ein Wind zwischen Himmel und Erde und brachte mich gen Jerusalem in einem göttlichen Gesichte zu dem Tor am inneren Vorhof, das gegen Mitternacht sieht, da stand ein Bild zu Verdruß dem HAUSHERRN.
\par 4 Und siehe, da war die Herrlichkeit des Gottes Israels, wie ich sie zuvor gesehen hatte im Felde.
\par 5 Und er sprach zu mir: du Menschenkind, hebe deine Augen auf gegen Mitternacht, siehe, da stand gegen Mitternacht das verdrießliche Bild am Tor des Altars, eben da man hineingeht.
\par 6 Und er sprach zu mir: Du Menschenkind, siehst du auch, was diese tun? Große Greuel, die das Haus Israel hier tut, daß sie mich ja fern von meinem Heiligtum treiben. Aber du wirst noch mehr große Greuel sehen.
\par 7 Und er führte mich zur Tür des Vorhofs; da sah ich, und siehe war ein Loch in der Wand.
\par 8 Und er sprach zu mir: Du Menschenkind, grabe durch die Wand. Und da ich durch die Wand grub, siehe, da war eine Tür.
\par 9 Und er sprach zu mir: Gehe hinein und schaue die bösen Greuel, die sie allhier tun.
\par 10 Und da ich hineinkam und sah, siehe, da waren allerlei Bildnisse der Würmer und Tiere, eitel Scheuel, und allerlei Götzen des Hauses Israel, allenthalben umher an der Wand gemacht;
\par 11 vor welchen standen siebzig Männer aus den Ältesten des Hauses Israel, und Jaasanja, der Sohn Saphans, stand auch unter ihnen; und ein jeglicher hatte sein Räuchfaß in der Hand, und ging ein dicker Nebel auf vom Räuchwerk.
\par 12 Und er sprach zu mir: Du Menschenkind, siehst du, was die Ältesten des Hauses Israel tun in der Finsternis, ein jeglicher in seiner Bilderkammer? Denn sie sagen: Der HERR sieht uns nicht, sondern der HERR hat das Land verlassen.
\par 13 Und er sprach zu mir: Du sollst noch mehr Greuel sehen, die sie tun.
\par 14 Und er führte mich hinein zum Tor an des HERRN Hause, das gegen Mitternacht steht; und siehe, daselbst saßen Weiber, die weinten über den Thammus.
\par 15 Und er sprach zu: Menschenkind, siehst du das? Aber du sollst noch größere Greuel sehen, denn diese sind.
\par 16 Und er führte mich in den inneren Hof am Hause des HERRN; und siehe, vor der Tür am Tempel des HERRN, zwischen der Halle und dem Altar, da waren bei fünfundzwanzig Männer, die ihren Rücken gegen den Tempel des HERRN und ihr Angesicht gegen Morgen gekehrt hatten und beteten gegen der Sonne Aufgang.
\par 17 Und er sprach zu mir: Menschenkind, siehst du das? Ist's dem Hause Juda zu wenig, alle solche Greuel hier zu tun, daß sie auch sonst im ganzen Lande eitel Gewalt und Unrecht treiben und reizen mich immer wieder? Und siehe, sie halten die Weinrebe an die Nase.
\par 18 Darum will ich auch wider sie mit Grimm handeln, und mein Auge soll ihrer nicht verschonen, und ich will nicht gnädig sein; und wenn sie gleich mit lauter Stimme vor meinen Ohren schreien, will ich sie doch nicht hören.

\chapter{9}

\par 1 Und er rief mit lauter Stimme vor meinen Ohren und sprach: Laßt herzukommen die Heimsuchung der Stadt, und ein jeglicher habe eine Mordwaffe in seiner Hand.
\par 2 Und siehe, es kamen sechs Männer auf dem Wege vom Obertor her, das gegen Mitternacht steht; und ein jeglicher hatte eine schädliche Waffe in seiner Hand. Aber es war einer unter ihnen der hatte Leinwand an und ein Schreibzeug an seiner Seite. Und sie gingen hinein und traten neben den ehernen Altar.
\par 3 Und die Herrlichkeit des Gottes Israels erhob sich von dem Cherub, über dem sie war, zu der Schwelle am Hause und rief dem, der die Leinwand anhatte und das Schreibzeug an seiner Seite.
\par 4 Und der HERR sprach zu ihm: Gehe durch die Stadt Jerusalem und zeichne mit einem Zeichen an die Stirn die Leute, so da seufzen und jammern über die Greuel, so darin geschehen.
\par 5 Zu jenen aber sprach er, daß ich's hörte: Gehet diesem nach durch die Stadt und schlaget drein; eure Augen sollen nicht schonen noch übersehen.
\par 6 Erwürget Alte, Jünglinge, Jungfrauen, Kinder und Weiber, alles tot; aber die das Zeichen an sich haben, derer sollt ihr keinen anrühren. Fanget aber an an meinem Heiligtum! Und sie fingen an an den alten Leuten, so vor dem Hause waren.
\par 7 Und er sprach zu ihnen: Verunreinigt das Haus und macht die Vorhöfe voll Erschlagener; gehet heraus! Und sie gingen heraus und schlugen in der Stadt.
\par 8 Und da sie ausgeschlagen hatten, war ich noch übrig. Und ich fiel auf mein Angesicht, schrie und sprach: Ach HERR HERR, willst du denn alle übrigen in Israel verderben, daß du deinen Zorn so ausschüttest über Jerusalem?
\par 9 Und er sprach zu mir: Es ist die Missetat des Hauses Israel und Juda allzusehr groß; es ist eitel Blutschuld im Lande und Unrecht in der Stadt. Denn sie sprechen: Der HERR hat das Land verlassen, und der HERR sieht uns nicht.
\par 10 Darum soll mein Auge auch nicht schonen, ich will auch nicht gnädig sein, sondern ihr Tun auf ihren Kopf werfen.
\par 11 Und siehe, der Mann, der die Leinwand anhatte und das Schreibzeug an seiner Seite, antwortete und sprach: Ich habe getan, wie du mir geboten hast.

\chapter{10}

\par 1 Und ich sah, und siehe, an dem Himmel über dem Haupt der Cherubim war es gestaltet wie ein Saphir, und über ihnen war es gleich anzusehen wie ein Thron.
\par 2 Und er sprach zu dem Mann in der Leinwand: Gehe hin zwischen die Räder unter den Cherub und fasse die Hände voll glühender Kohlen, so zwischen den Cherubim sind, und streue sie über die Stadt. Und er ging hinein, daß ich's sah, da er hineinging.
\par 3 Die Cherubim aber standen zur Rechten am Hause, und die Wolke erfüllte den innern Vorhof.
\par 4 Und die Herrlichkeit des HERRN erhob sich von dem Cherub zur Schwelle am Hause; und das Haus ward erfüllt mit der Wolke und der Vorhof voll Glanzes von der Herrlichkeit des HERRN.
\par 5 Und man hörte die Flügel der Cherubim rauschen bis in den äußeren Vorhof wie eine mächtige Stimme des allmächtigen Gottes, wenn er redet.
\par 6 Und da er dem Mann in der Leinwand geboten hatte und gesagt: Nimm Feuer zwischen den Rädern unter den Cherubim! ging er hinein und trat neben das Rad.
\par 7 Und der Cherub streckte seine Hand heraus zwischen den Cherubim zum Feuer, das zwischen den Cherubim war, nahm davon und gab's dem Mann in der Leinwand in die Hände; der empfing's und ging hinaus.
\par 8 Und es erschien an den Cherubim gleichwie eines Menschen Hand unter ihren Flügeln.
\par 9 Und ich sah, und siehe, vier Räder standen bei den Cherubim, bei einem jeglichen Cherub ein Rad; und die Räder waren anzusehen gleichwie ein Türkis
\par 10 und waren alle vier eines wie das andere, als wäre ein Rad im andern.
\par 11 Wenn sie gehen sollten, so konnten sie nach allen vier Seiten gehen und mußten sich nicht herumlenken, wenn sie gingen; sondern wohin das erste ging, da gingen sie nach und mußten sich nicht herumlenken.
\par 12 Und ihr ganzer Leib, Rücken, Hände und Flügel und die Räder waren voll Augen um und um; alle vier hatten ihre Räder.
\par 13 Und die Räder wurden genannt "der Wirbel", daß ich's hörte.
\par 14 Ein jeglicher hatte vier Angesichter; das erste Angesicht war eines Cherubs, das andere eines Menschen, das dritte eines Löwen, das vierte eines Adlers.
\par 15 Und die Cherubim schwebten empor. Es ist eben das Tier, das ich sah am Wasser Chebar.
\par 16 Wenn die Cherubim gingen, so gingen die Räder auch neben ihnen; und wenn die Cherubim ihre Flügel schwangen, daß sie sich von der Erde erhoben, so lenkten sich die Räder auch nicht von Ihnen.
\par 17 Wenn jene standen, so standen diese auch; erhoben sie sich, so erhoben sich diese auch; denn es war der Geist der Tiere in ihnen.
\par 18 Und die Herrlichkeit des HERRN ging wieder aus von der Schwelle am Hause des HERRN und stellt sich über die Cherubim.
\par 19 Da schwangen die Cherubim ihre Flügel und erhoben sich von der Erde vor meinen Augen; und da sie ausgingen, gingen die Räder neben ihnen. Und sie traten zum Tor am Hause des HERRN, gegen Morgen, und die Herrlichkeit des Gottes Israels war oben über ihnen.
\par 20 Das ist das Tier, das ich unter dem Gott Israels sah am Wasser Chebar; und ich merkte, das es Cherubim wären,
\par 21 da ein jegliches vier Angesichter hatte und vier Flügel und unter den Flügeln gleichwie Menschenhände.
\par 22 Es waren ihre Angesichter gestaltet, wie ich sie am Wasser Chebar sah, und sie gingen stracks vor sich.

\chapter{11}

\par 1 Und mich hob ein Wind auf und brachte mich zum Tor am Hause des HERRN, das gegen Morgen sieht; und siehe, unter dem Tor waren fünfundzwanzig Männer; und ich sah unter ihnen Jaasanja, den Sohn Assurs, und Pelatja, den Sohn Benajas, die Fürsten im Volk.
\par 2 Und er sprach zu mir: Menschenkind, diese Leute haben unselige Gedanken und schädliche Ratschläge in dieser Stadt;
\par 3 denn sie sprechen: "Es ist nicht so nahe; laßt uns nur Häuser bauen! Sie ist der Topf, so sind wir das Fleisch."
\par 4 Darum sollst du, Menschenkind, wider sie weissagen.
\par 5 Und der Geist des HERRN fiel auf mich, und er sprach zu mir: Sprich: So sagt der HERR: Ich habe also geredet, ihr vom Hause Israel; und eures Geistes Gedanken kenne ich wohl.
\par 6 Ihr habt viele erschlagen in dieser Stadt, und ihre Gassen liegen voll Toter.
\par 7 Darum spricht der HERR HERR also: Die ihr darin getötet habt, die sind das Fleisch, und sie ist der Topf; aber ihr müßt hinaus.
\par 8 Das Schwert, das ihr fürchtet, das will ich über euch kommen lassen, spricht der HERR HERR.
\par 9 Ich will euch von dort herausstoßen und den Fremden in die Hand geben und will euch euer Recht tun.
\par 10 Ihr sollt durchs Schwert fallen; an der Grenze Israels will ich euch richten, und sollt erfahren, daß ich der HERR bin.
\par 11 Die Stadt aber soll nicht euer Topf sein noch ihr das Fleisch darin; sondern an der Grenze Israels will ich euch richten.
\par 12 Und ihr sollt erfahren, daß ich der HERR bin; denn ihr habt nach meinen Geboten nicht gewandelt und habt meine Rechte nicht gehalten, sondern getan nach der Heiden Weise, die um euch her sind.
\par 13 Und da ich so weissagte, starb Pelatja, der Sohn Benajas. Da fiel ich auf mein Angesicht und schrie mit lauter Stimme und sprach: Ach HERR HERR, du wirst's mit den übrigen Israels gar aus machen!
\par 14 Da geschah des HERRN Wort zu mir und sprach:
\par 15 Du Menschenkind, zu deinen Brüdern und nahen Freunden und dem ganzen Haus Israel sprechen wohl die, so noch zu Jerusalem wohnen: Ihr müsset fern vom HERRN sein, aber wir haben das Land inne.
\par 16 Darum sprich du: So spricht der HERR HERR: Ja, ich habe sie fern weg unter die Heiden lassen treiben und in die Länder zerstreut; doch will ich bald ihr Heiland sein in den Ländern, dahin sie gekommen sind.
\par 17 Darum sprich: So sagt der HERR HERR: Ich will euch sammeln aus den Völkern und will euch sammeln aus den Ländern, dahin ihr zerstreut seid, und will euch das Land Israel geben.
\par 18 Da sollen sie kommen und alle Scheuel und Greuel daraus wegtun.
\par 19 Und ich will euch ein einträchtiges Herz geben und einen neuen Geist in euch geben und will das steinerne Herz wegnehmen aus eurem Leibe und ein fleischernes Herz geben,
\par 20 auf daß sie nach meinen Sitten wandeln und meine Rechte halten und darnach tun. Und sie sollen mein Volk sein, so will ich ihr Gott sein.
\par 21 Denen aber, so nach ihres Herzens Scheueln und Greueln wandeln, will ich ihr Tun auf ihren Kopf werfen, spricht der HERR HERR.
\par 22 Da schwangen die Cherubim ihre Flügel, und die Räder gingen neben ihnen, und die Herrlichkeit des Gottes Israels war oben über ihnen.
\par 23 Und die Herrlichkeit des HERRN erhob sich aus der Stadt und stellte sich auf den Berg, der gegen Morgen vor der Stadt liegt.
\par 24 Und ein Wind hob mich auf und brachte mich im Gesicht und im Geist Gottes nach Chaldäa zu den Gefangenen. Und das Gesicht, so ich gesehen hatte, verschwand vor mir.
\par 25 Und ich sagte den Gefangenen alle Worte des HERRN, die er mir gezeigt hatte.

\chapter{12}

\par 1 Und des HERRN Wort geschah zu mir und sprach:
\par 2 Du Menschenkind, du wohnst unter einem ungehorsamen Haus, welches hat wohl Augen, daß sie sehen könnten, und wollen nicht sehen, Ohren, daß sie hören könnten, und wollen nicht hören, sondern es ist ein ungehorsames Haus.
\par 3 Darum, du Menschenkind, nimm dein Wandergerät und zieh am lichten Tage davon vor ihren Augen. Von deinem Ort sollst du ziehen an einen andern Ort vor ihren Augen, ob sie vielleicht merken wollten, daß sie ein ungehorsames Haus sind.
\par 4 Und sollst dein Gerät heraustun wie Wandergerät bei lichtem Tage vor ihren Augen; und du sollst ausziehen des Abends vor ihren Augen, gleichwie man auszieht, wenn man wandern will;
\par 5 und du sollst durch die Wand ausbrechen vor ihren Augen und durch dieselbe ziehen;
\par 6 und du sollst es auf deine Schulter nehmen vor ihren Augen und, wenn es dunkel geworden ist, hinaustragen; dein Angesicht sollst du verhüllen, daß du das Land nicht siehst. Denn ich habe dich dem Hause Israel zum Wunderzeichen gesetzt.
\par 7 Und ich tat wie mir befohlen war, und trug mein Gerät heraus wie Wandergerät bei lichtem Tage; und am Abend brach ich mit der Hand durch die Wand; und das es dunkel geworden war, nahm ich's auf die Schulter und trug's hinaus vor ihren Augen.
\par 8 Und frühmorgens geschah des HERRN Wort zu mir und sprach:
\par 9 Menschenkind, hat das Haus Israel, das ungehorsame Haus, nicht zu dir gesagt: Was machst du?
\par 10 So sprich zu ihnen: So spricht der HERR HERR: Diese Last betrifft den Fürsten zu Jerusalem und das ganze Haus Israel, das darin ist.
\par 11 Sprich: Ich bin euer Wunderzeichen; wie ich getan habe, also soll ihnen geschehen, daß sie wandern müssen und gefangen geführt werden.
\par 12 Ihr Fürst wird seine Habe auf der Schulter tragen im Dunkel und muß ausziehen durch die Wand, die sie zerbrechen werden, daß sie dadurch ausziehen; sein Angesicht wird verhüllt werden, daß er mit keinem Auge das Land sehe.
\par 13 Ich will auch mein Netz über ihn werfen, daß er in meinem Garn gefangen werde, und will ihn gen Babel bringen in der Chaldäer Land, das er doch nicht sehen wird, und er soll daselbst sterben.
\par 14 Und alle, die um ihn her sind, seine Gehilfen und all sein Anhang, will ich unter alle Winde zerstreuen und das Schwert hinter ihnen her ausziehen.
\par 15 Also sollen sie erfahren, daß ich der HERR sei, wenn ich sie unter die Heiden verstoße und in die Länder zerstreue.
\par 16 Aber ich will ihrer etliche wenige übrigbleiben lassen vor dem Schwert, dem Hunger und der Pestilenz; die sollen jener Greuel erzählen unter den Heiden, dahin sie kommen werden, und sie sollen erfahren, daß ich der HERR sei.
\par 17 Und des HERRN Wort geschah zu mir und sprach:
\par 18 Du Menschenkind, du sollst dein Brot essen mit Beben und dein Wasser trinken mit Zittern und Sorgen.
\par 19 Und sprich zum Volk im Lande: So spricht der HERR HERR von den Einwohnern zu Jerusalem im Lande Israel: Sie müssen ihr Brot essen in Sorgen und ihr Wasser trinken in Elend; denn das Land soll wüst werden von allem, was darin ist, um des Frevels willen aller Einwohner.
\par 20 Und die Städte, so wohl bewohnt sind, sollen verwüstet und das Land öde werden; also sollt ihr erfahren, daß ich der HERR sei.
\par 21 Und des HERRN Wort geschah zu mir und sprach:
\par 22 Du Menschenkind, was habt ihr für ein Sprichwort im Lande Israel und sprecht: Weil sich's so lange verzieht, so wird nun hinfort nichts aus der Weissagung?
\par 23 Darum sprich zu ihnen: So spricht der HERR HERR: Ich will das Sprichwort aufheben, daß man es nicht mehr führen soll in Israel. Und rede zu ihnen: Die Zeit ist nahe und alles, was geweissagt ist.
\par 24 Denn es soll hinfort kein falsches Gesicht und keine Weissagung mit Schmeichelworten mehr sein im Hause Israel.
\par 25 Denn ich bin der HERR; was ich rede, das soll geschehen und nicht länger verzogen werden; sondern bei eurer Zeit, ihr ungehorsames Haus, will ich tun, was ich rede, spricht der HERR HERR.
\par 26 Und des HERRN Wort geschah zu mir und sprach:
\par 27 Du Menschenkind, siehe, das Haus Israel spricht: Das Gesicht, das dieser sieht, da ist noch lange hin; und er weissagt auf die Zeit, die noch ferne ist.
\par 28 Darum sprich zu ihnen: So spricht der HERR HERR: Was ich rede, soll nicht länger verzogen werden, sondern soll geschehen, spricht der HERR HERR.

\chapter{13}

\par 1 Und des HERRN Wort geschah zu mir und sprach:
\par 2 Du Menschenkind, weissage wider die Propheten Israels und sprich zu denen, so aus ihrem eigenen Herzen weissagen: Höret des HERRN Wort!
\par 3 So spricht der HERR HERR: Weh den tollen Propheten, die ihrem eigenen Geist folgen und haben keine Gesichte!
\par 4 O Israel, deine Propheten sind wie die Füchse in den Wüsten!
\par 5 Sie treten nicht vor die Lücken und machen sich nicht zur Hürde um das Haus Israel und stehen nicht im Streit am Tage des HERRN.
\par 6 Ihr Gesicht ist nichts, und ihr Weissagen ist eitel Lügen. Sie sprechen: "Der HERR hat's gesagt", so sie doch der HERR nicht gesandt hat, und warten, daß ihr Wort bestehe.
\par 7 Ist's nicht also, daß euer Gesicht ist nichts und euer Weissagen ist eitel Lügen? und ihr sprecht doch: "Der HERR hat's geredet", so ich's doch nicht geredet habe.
\par 8 Darum spricht der HERR HERR also: Weil ihr das predigt, woraus nichts wird, und Lügen weissagt, so will ich an euch, spricht der HERR HERR.
\par 9 Und meine Hand soll kommen über die Propheten, so das predigen, woraus nichts wird, und Lügen weissagen. Sie sollen in der Versammlung meines Volkes nicht sein und in der Zahl des Hauses Israel nicht geschrieben werden noch ins Land Israels kommen; und ihr sollt erfahren, daß ich der HERR HERR bin.
\par 10 Darum daß sie mein Volk verführen und sagen: "Friede!", so doch kein Friede ist. Das Volk baut die Wand, so tünchen sie dieselbe mit losem Kalk.
\par 11 Sprich zu den Tünchern, die mit losem Kalk tünchen, daß es abfallen wird; denn es wird ein Platzregen kommen und werden große Hagel fallen und ein Windwirbel wird es zerreißen.
\par 12 Siehe, so wird die Wand einfallen. Was gilt's? dann wird man zu euch sagen: Wo ist nun das getünchte, das ihr getüncht habt?
\par 13 So spricht der HERR HERR: Ich will einen Windwirbel reißen lassen in meinem Grimm und einen Platzregen in meinem Zorn und große Hagelsteine im Grimm, die sollen alles umstoßen.
\par 14 Also will ich die Wand umwerfen; die ihr mit losem Kalk getüncht habt, und will sie zu Boden stoßen, daß man ihren Grund sehen soll; so fällt sie, und ihr sollt darin auch umkommen und erfahren, daß ich der HERR sei.
\par 15 Also will ich meinen Grimm vollenden an der Wand und an denen, die sie mit losem Kalk tünchen, und will zu euch sagen: Hier ist weder Wand noch Tüncher.
\par 16 Das sind die Propheten Israels, die Jerusalem weissagen und predigen von Frieden, so doch kein Friede ist, spricht der HERR HERR.
\par 17 Und du, Menschenkind, richte dein Angesicht wider die Töchter in deinem Volk, welche weissagen aus ihrem Herzen, und weissage wider sie
\par 18 und sprich: So spricht der HERR HERR: Wehe euch, die ihr Kissen macht den Leuten unter die Arme und Pfühle zu den Häuptern, beide, Jungen und Alten, die Seelen zu fangen. Wenn ihr nun die Seelen gefangen habt unter meinem Volk, verheißt ihr ihnen das Leben
\par 19 und entheiligt mich in meinem Volk um eine Handvoll Gerste und einen Bissen Brot, damit daß ihr die Seelen zum Tode verurteilt, die doch nicht sollten sterben, und verurteilt zum Leben, die doch nicht leben sollten, durch eure Lügen unter meinem Volk, welches gerne Lügen hört.
\par 20 Darum spricht der HERR HERR: Siehe, ich will an eure Kissen, womit ihr Seelen fangt und vertröstet, und will sie von euren Armen wegreißen und die Seelen, so ihr fangt und vertröstet, losmachen.
\par 21 Und ich will eure Pfühle zerreißen und mein Volk aus eurer Hand erretten, daß ihr sie nicht mehr fangen sollt; und ihr sollt erfahren, daß ich der HERR sei.
\par 22 Darum daß ihr das Herz der Gerechten fälschlich betrübet, die ich nicht betrübt habe, und habt gestärkt die Hände der Gottlosen, daß sie sich von ihrem bösen Wesen nicht bekehren, damit sie lebendig möchten bleiben:
\par 23 darum sollt ihr nicht mehr unnütze Lehre predigen noch weissagen; sondern ich will mein Volk aus ihren Händen erretten, und ihr sollt erfahren, daß ich der HERR bin.

\chapter{14}

\par 1 Und es kamen etliche von den Ältesten Israels zu mir und setzten sich vor mir.
\par 2 Da geschah des HERRN Wort zu mir und sprach:
\par 3 Menschenkind, diese Leute hangen mit ihrem Herzen an ihren Götzen und halten an dem Anstoß zu ihrer Missetat; sollte ich denn ihnen antworten, wenn sie mich fragen?
\par 4 Darum rede mit ihnen und sage zu ihnen: So spricht der HERR HERR: Welcher Mensch vom Hause Israel mit dem Herzen an seinen Götzen hängt und hält an dem Anstoß zu seiner Missetat und kommt zum Propheten, dem will ich, der HERR, antworten, wie er verdient hat mit seiner großen Abgötterei,
\par 5 auf daß ich das Haus Israel fasse an ihrem Herzen, darum daß sie alle von mir gewichen sind durch ihre Abgötterei.
\par 6 Darum sollst du zum Hause Israel sagen: So spricht der HERR HERR: Kehret und wendet euch von eurer Abgötterei und wendet euer Angesicht von allen euren Greueln.
\par 7 Denn welcher Mensch vom Hause Israel oder welcher Fremdling, so in Israel wohnt, von mir weicht und mit seinem Herzen an seinen Götzen hängt und an dem Ärgernis seiner Abgötterei hält und zum Propheten kommt, daß er durch ihn mich frage: dem will ich, der HERR, selbst antworten;
\par 8 und will mein Angesicht wider ihn setzen, daß er soll wüst und zum Zeichen und Sprichwort werden, und ich will ihn aus meinem Volk ausrotten, daß ihr erfahren sollt, ich sei der HERR.
\par 9 Wo aber ein Prophet sich betören läßt, etwas zu reden, den habe ich, der HERR, betört, und will meine Hand über ihn ausstrecken und ihn aus meinem Volk Israel ausrotten.
\par 10 Also sollen sie beide ihre Missetat tragen; wie die Missetat des Fragers, also soll auch sein die Missetat des Propheten,
\par 11 auf daß das Haus Israel nicht mehr irregehe von mir und sich nicht mehr verunreinige in aller seiner Übertretung; sondern sie sollen mein Volk sein, und ich will ihr Gott sein, spricht der HERR HERR.
\par 12 Und des HERRN Wort geschah zu mir und sprach:
\par 13 Du Menschenkind, wenn ein Land an mir sündigt und dazu mich verschmäht, so will ich meine Hand über dasselbe ausstrecken und den Vorrat des Brotes wegnehmen und will Teuerung hineinschicken, daß ich Menschen und Vieh darin ausrotte.
\par 14 Und wenn dann gleich die drei Männer Noah, Daniel und Hiob darin wären, so würden sie allein ihre eigene Seele erretten durch ihre Gerechtigkeit, spricht der HERR HERR.
\par 15 Und wenn ich böse Tiere in das Land bringen würde, die die Leute aufräumten und es verwüsteten, daß niemand darin wandeln könnte vor den Tieren,
\par 16 und diese drei Männer wären auch darin: so wahr ich lebe, spricht der HERR HERR, sie würden weder Söhne noch Töchter erretten, sondern allein sich selbst, und das Land müßte öde werden.
\par 17 Oder ob ich das Schwert kommen ließe über das Land und spräche: Schwert, fahre durch das Land! und würde also Menschen und Vieh ausrotten,
\par 18 und die drei Männer wären darin: so wahr ich lebe, spricht der HERR HERR, sie würden weder Söhne noch Töchter erretten, sondern sie allein würden errettet sein.
\par 19 Oder so ich Pestilenz in das Land schicken und meinen Grimm über dasselbe ausschütten würde mit Blutvergießen, also daß ich Menschen und Vieh ausrottete,
\par 20 und Noah, Daniel und Hiob wären darin: so wahr ich lebe, spricht der HERR HERR, würden sie weder Söhne noch Töchter, sondern allein ihre eigene Seele durch ihre Gerechtigkeit erretten.
\par 21 Denn so spricht der HERR HERR: So ich meine vier bösen Strafen, als Schwert, Hunger, böse Tiere und Pestilenz, über Jerusalem schicken werde, daß ich darin ausrotte Menschen und Vieh,
\par 22 siehe, so sollen etliche übrige darin davonkommen, die herausgebracht werden, Söhne und Töchter, und zu euch herkommen, daß ihr sehen werdet ihr Wesen und Tun und euch trösten über dem Unglück, das ich über Jerusalem habe kommen lassen samt allem andern, was ich über sie habe kommen lassen.
\par 23 Sie werden euer Trost sein, wenn ihr sehen werdet ihr Wesen und Tun; und ihr werdet erfahren, daß ich nicht ohne Ursache getan habe, was ich darin getan habe, spricht der HERR HERR.

\chapter{15}

\par 1 Und des HERRN Wort geschah zu mir und sprach:
\par 2 Du Menschenkind, was ist das Holz vom Weinstock vor anderm Holz oder eine Rebe vor anderm Holz im Walde?
\par 3 Nimmt man es auch und macht etwas daraus? Macht man auch nur einen Nagel daraus, daran man etwas hängen kann?
\par 4 Siehe, man wirft sie ins Feuer, daß es verzehrt wird, daß das Feuer seine beiden Enden verzehrt und sein Mittles versengt; wozu sollte es nun taugen?
\par 5 Siehe, da es noch ganz war, konnte man nichts daraus machen; wie viel weniger kann nun hinfort etwas daraus gemacht werden, so es das Feuer verzehrt und versengt hat!
\par 6 Darum spricht der HERR HERR: Gleichwie ich das Holz vom Weinstock vor anderm Holz im Walde dem Feuer zu verzehren gebe, also will ich mit den Einwohnern zu Jerusalem auch umgehen
\par 7 und will mein Angesicht wider sie setzen, daß das Feuer sie fressen soll, ob sie schon aus dem Feuer herausgekommen sind. Und ihr sollt's erfahren, daß ich der HERR bin, wenn ich mein Angesicht wider sie setze
\par 8 und das Land wüst mache, darum daß sie mich verschmähen, spricht der HERR HERR.

\chapter{16}

\par 1 Und des HERRN Wort geschah zu mir und sprach:
\par 2 Du Menschenkind offenbare der Stadt Jerusalem ihre Greuel und sprich:
\par 3 So spricht der HERR HERR zu Jerusalem: Dein Geschlecht und deine Geburt ist aus der Kanaaniter Lande, dein Vater aus den Amoritern und deine Mutter aus den Hethitern.
\par 4 Deine Geburt ist also gewesen: Dein Nabel, da du geboren wurdest, ist nicht verschnitten; so hat man dich auch nicht in Wasser gebadet, daß du sauber würdest, noch mit Salz abgerieben noch in Windeln gewickelt.
\par 5 Denn niemand jammerte dein, daß er sich über dich hätte erbarmt und der Stücke eins dir erzeigt, sondern du wurdest aufs Feld geworfen. Also verachtet war deine Seele, da du geboren warst.
\par 6 Ich aber ging vor dir vorüber und sah dich in deinem Blut liegen und sprach zu dir, da du so in deinem Blut lagst: Du sollst leben!
\par 7 Und habe dich erzogen und lassen groß werden wie ein Gewächs auf dem Felde; und warst nun gewachsen und groß und schön geworden. Deine Brüste waren gewachsen und hattest schon lange Haare; aber du warst noch nackt und bloß.
\par 8 Und ich ging vor dir vorüber und sah dich an; und siehe, es war die Zeit, um dich zu werben. Da breitete ich meinen Mantel über dich und bedeckte deine Blöße. Und ich gelobte dir's und begab mich mit dir in einen Bund, spricht der HERR HERR, daß du solltest mein sein.
\par 9 Und ich badete dich im Wasser und wusch dich von all deinem Blut und salbte dich mit Balsam
\par 10 und kleidete dich mit gestickten Kleidern und zog dir Schuhe von feinem Leder an; ich gab dir köstliche leinene Kleider und seidene Schleier
\par 11 und zierte dich mit Kleinoden und legte dir Geschmeide an deine Arme und Kettlein an deinen Hals
\par 12 und gab dir ein Haarband an deine Stirn und Ohrenringe an deine Ohren und eine schöne Krone auf dein Haupt.
\par 13 So warst du geziert mit eitel Gold und Silber und gekleidet mit eitel Leinwand, Seide und Gesticktem. Du aßest auch eitel Semmel, Honig und Öl und warst überaus schön und bekamst das Königreich.
\par 14 Und dein Ruhm erscholl unter die Heiden deiner Schöne halben, welche ganz vollkommen war durch den Schmuck, so ich an dich gehängt hatte, spricht der HERR HERR.
\par 15 Aber du verließest dich auf deine Schöne; und weil du so gerühmt warst, triebst du Hurerei, also daß du dich einem jeglichen, wer vorüberging, gemein machtest und tatest seinen Willen.
\par 16 Und nahmst von deinen Kleidern und machtest dir bunte Altäre daraus und triebst deine Hurerei darauf, wie nie geschehen ist noch geschehen wird.
\par 17 Du nahmst auch dein schönes Gerät, das ich dir von meinem Gold und Silber gegeben hatte, und machtest dir Mannsbilder daraus und triebst deine Hurerei mit ihnen.
\par 18 Und nahmst deine bestickten Kleider und bedecktest sie damit und mein Öl und Räuchwerk legtest du ihnen vor.
\par 19 Meine Speise, die ich dir zu essen gab, Semmel, Öl, Honig, legtest du ihnen vor zum süßen Geruch. Ja es kam dahin, spricht der HERR HERR,
\par 20 daß du nahmst deine Söhne und Töchter, die du mir geboren hattest, und opfertest sie denselben zu fressen. Meinst du denn, daß es eine Geringes sei um deine Hurerei,
\par 21 daß du meine Kinder schlachtest und läßt sie denselben verbrennen?
\par 22 Und in allen deinen Greueln und Hurerei hast du nie gedacht an die Zeit deiner Jugend, wie bloß und nackt du warst und in deinem Blut lagst.
\par 23 Über alle diese deine Bosheit (ach weh dir, weh dir! spricht der HERR HERR)
\par 24 bautest du dir Götzenkapellen und machtest dir Altäre auf allen Gassen;
\par 25 und vornan auf allen Straßen bautest du deine Altäre und machtest deine Schöne zu eitel Greuel; du spreiztest deine Beine gegen alle, so vorübergingen, und triebst große Hurerei.
\par 26 Erstlich triebst du Hurerei mit den Kindern Ägyptens, deinen Nachbarn, die großes Fleisch hatten, und triebst große Hurerei, mich zu reizen.
\par 27 Ich aber streckte meine Hand aus wider dich und brach dir an deiner Nahrung ab und übergab dich in den Willen deiner Feinde, der Töchter der Philister, welche sich schämten vor deinem verruchten Wesen.
\par 28 Darnach triebst du Hurerei mit den Kindern Assur und konntest des nicht satt werden; ja, da du mit ihnen Hurerei getrieben hattest und des nicht satt werden konntest,
\par 29 machtest du der Hurerei noch mehr bis ins Krämerland Chaldäa; doch konntest du damit auch nicht satt werden.
\par 30 Wie soll ich dir doch dein Herz beschneiden, spricht der HERR HERR, weil du solche Werke tust einer großen Erzhure,
\par 31 damit daß du deine Götzenkapellen bautest vornan auf allen Straßen und deine Altäre machtest auf allen Gassen? Dazu warst du nicht wie eine andere Hure, die man muß mit Geld kaufen.
\par 32 Du Ehebrecherin, die anstatt ihres Mannes andere zuläßt!
\par 33 Denn allen andern Huren gibt man Geld; du aber gibst allen deinen Buhlern Geld zu und schenkst ihnen, daß sie zu dir kommen allenthalben und mit dir Hurerei treiben.
\par 34 Und findet sich an dir das Widerspiel vor andern Weibern mit deiner Hurerei, weil man dir nicht nachläuft, sondern du Geld zugibst, und man dir nicht Geld zugibt; also treibst du das Widerspiel.
\par 35 Darum, du Hure, höre des HERRN Wort!
\par 36 So spricht der HERR HERR: Weil du denn so milde Geld zugibst und deine Blöße durch deine Hurerei gegen deine Buhlen aufdeckst und gegen alle Götzen deiner Greuel und vergießt das Blut deiner Kinder, welche du ihnen opferst:
\par 37 darum, siehe, will ich sammeln alle deine Buhlen, welchen du wohl gefielst, samt allen, die du für deine Freunde hältst, zu deinen Feinden und will sie beide wider dich sammeln allenthalben und will ihnen deine Blöße aufdecken, daß sie deine Blöße ganz sehen sollen.
\par 38 Und will das Recht der Ehebrecherinnen und Blutvergießerinnen über dich gehen und dein Blut vergießen lassen mit Grimm und Eifer.
\par 39 Und will dich in ihre Hände geben, daß sie deine Kapellen abbrechen und deine Altäre umreißen und dir deine Kleider ausziehen und dein schönes Gerät dir nehmen und dich nackt und bloß sitzen lassen.
\par 40 Und sie sollen Haufen Leute über dich bringen, die dich steinigen und mit ihren Schwertern zerhauen
\par 41 und deine Häuser mit Feuer verbrennen und dir dein Recht tun vor den Augen vieler Weiber. Also will ich deiner Hurerei ein Ende machen, daß du nicht mehr sollst Geld noch zugeben,
\par 42 und will meinen Mut an dir kühlen und meinen Eifer an dir sättigen, daß ich ruhe und nicht mehr zürnen müsse.
\par 43 Darum daß du nicht gedacht hast an die Zeit deiner Jugend, sondern mich mit diesem allem gereizt, darum will ich auch dir all dein Tun auf den Kopf legen, spricht der HERR HERR, wiewohl ich damit nicht getan habe nach dem Laster in deinen Greueln.
\par 44 Siehe, alle die, so Sprichwort pflegen zu üben, werden von dir dies Sprichwort sagen: "Die Tochter ist wie die Mutter."
\par 45 Du bist deiner Mutter Tochter, welche Mann und Kinder von sich stößt, und bist eine Schwester deiner Schwestern, die ihre Männer und Kinder von sich stoßen. Eure Mutter ist eine von den Hethitern und euer Vater ein Amoriter.
\par 46 Samaria ist dein große Schwester mit ihren Töchtern, die dir zur Linken wohnt; und Sodom ist deine kleine Schwester mit ihren Töchtern, die dir zur Rechten wohnt;
\par 47 wiewohl du dennoch nicht gelebt hast nach ihrem Wesen noch getan nach ihren Greueln. Es fehlt nicht viel, daß du es ärger gemacht hast denn sie in allem deinem Wesen.
\par 48 So wahr ich lebe, spricht der HERR HERR, Sodom, deine Schwester, samt ihren Töchtern hat nicht so getan wie du und deine Töchter.
\par 49 Siehe, das war deiner Schwester Sodom Missetat: Hoffart und alles vollauf und guter Friede, den sie und ihre Töchter hatten; aber den Armen und Dürftigen halfen sie nicht,
\par 50 sondern waren stolz und taten Greuel vor mir; darum ich sie auch weggetan habe, da ich begann dareinzusehen.
\par 51 So hat auch Samaria nicht die Hälfte deiner Sünden getan; sondern du hast deiner Greuel so viel mehr als sie getan, daß du deine Schwester fromm gemacht hast gegen alle deine Greuel die du getan hast.
\par 52 So trage auch nun deine Schande, die du deiner Schwester zuerkannt hast. Durch deine Sünden, in welchen du größere Greuel denn sie getan hast, machst du sie frömmer, denn du bist. So sei nun auch du schamrot und trage deine Schande, daß du deine Schwester fromm gemacht hast.
\par 53 Ich will aber ihr Gefängnis wenden, nämlich das Gefängnis dieser Sodom und ihrer Töchter und das Gefängnis dieser Samaria und ihrer Töchter und das Gefängnis deiner Gefangenen samt ihnen,
\par 54 daß du tragen mußt deine Schande und dich schämst alles dessen, was du getan hast ihnen zum Troste.
\par 55 Und deine Schwestern, diese Sodom und ihre Töchter, sollen wieder werden, wie sie zuvor gewesen sind, und Samaria und ihre Töchter sollen wieder werden, wie sie zuvor gewesen sind; dazu auch du und deine Töchter sollt wieder werden, wie ihr zuvor gewesen seid.
\par 56 Und wirst nicht mehr die Sodom, deine Schwester rühmen wie zur Zeit deines Hochmuts,
\par 57 da deine Bosheit noch nicht aufgedeckt war wie zur Zeit, da dich die Töchter Syriens und die Töchter der Philister allenthalben schändeten und verachteten dich um und um,
\par 58 da ihr mußtet eure Laster tragen, spricht der HERR.
\par 59 Denn also spricht der HERR HERR: Ich will dir tun wie du getan hast, daß du den Eid verachtest und brichst den Bund.
\par 60 Ich will aber gedenken an meinen Bund, den ich mit dir gemacht habe zur Zeit deiner Jugend, und will mit dir einen ewigen Bund aufrichten.
\par 61 Da wirst du an deine Wege gedenken und dich schämen, wenn du deine großen und kleinen Schwestern zu dir nehmen wirst, die ich dir zu Töchtern geben werde, aber nicht aus deinem Bund.
\par 62 Sondern ich will meinen Bund mit dir aufrichten, daß du erfahren sollst, daß ich der HERR sei,
\par 63 auf daß du daran gedenkst und dich schämst und vor Schande nicht mehr deinen Mund auftun dürfest, wenn ich dir alles vergeben werde, was du getan hast, spricht der HERR HERR.

\chapter{17}

\par 1 Und des HERRN Wort geschah zu mir und sprach:
\par 2 Du Menschenkind, lege doch dem Hause Israel ein Rätsel vor und ein Gleichnis
\par 3 und sprich: So spricht der HERR HERR: Ein großer Adler mit großen Flügeln und langen Fittichen und voll Federn, die bunt waren, kam auf den Libanon und nahm den Wipfel von der Zeder
\par 4 und brach das oberste Reis ab und führte es ins Krämerland und setzte es in die Kaufmannstadt.
\par 5 Er nahm auch vom Samen des Landes und pflanzte es in gutes Land, da viel Wasser war, und setzte es lose hin.
\par 6 Und es wuchs und ward ein ausgebreiteter Weinstock und niedrigen Stammes; denn seine Reben bogen sich zu ihm, und seine Wurzeln waren unter ihm; und er war also ein Weinstock, der Reben kriegte und Zweige.
\par 7 Und da war ein anderer großer Adler mit großen Flügeln und vielen Federn; und siehe, der Weinstock hatte verlangen an seinen Wurzeln zu diesem Adler und streckte seine Reben aus gegen ihn, daß er gewässert würde, vom Platz, da er gepflanzt war.
\par 8 Und war doch auf einen guten Boden an viel Wasser gepflanzt, da er wohl hätte können Zweige bringen, Früchte tragen und ein herrlicher Weinstock werden.
\par 9 So sprich nun: Also sagt der HERR HERR: Sollte der geraten? Ja, man wird seine Wurzeln ausrotten und seine Früchte abreißen, und er wird verdorren, daß alle Blätter seines Gewächses verdorren werden; und es wird nicht geschehen durch großen Arm noch viel Volks, daß man ihn von seinen Wurzeln wegführe.
\par 10 Siehe, er ist zwar gepflanzt; aber sollte er geraten? Ja, sobald der Ostwind an ihn rühren wird, wird er verdorren auf dem Platz, da er gewachsen ist.
\par 11 Und des HERR Wort geschah zu mir und sprach:
\par 12 Sprich doch zu diesem ungehorsamen Haus: Wißt ihr nicht, was das ist? Und sprich: Siehe, es kam ein König zu Babel gen Jerusalem und nahm ihren König und ihre Fürsten und führte sie weg zu sich gen Babel.
\par 13 Und nahm einen vom königlichen Geschlecht und machte einen Bund mit ihm und nahm einen Eid von ihm; aber die Gewaltigen im Lande nahm er weg,
\par 14 damit das Königreich demütig bliebe und sich nicht erhöbe, auf daß sein Bund gehalten würde und bestünde.
\par 15 Aber derselbe fiel von ihm ab und sandte seine Botschaft nach Ägypten, daß man ihm Rosse und viel Volks schicken sollte. Sollte es dem geraten? Sollte er davonkommen, der solches tut? und sollte der, so den Bund bricht davonkommen?
\par 16 So wahr ich lebe spricht der HERR HERR, an dem Ort des Königs, der ihn zum König gesetzt hat, dessen Eid er verachtet und dessen Bund er gebrochen hat, da soll er sterben, nämlich zu Babel.
\par 17 Auch wird ihm Pharao nicht beistehen im Kriege mit großem Heer und vielem Volk, wenn man den Wall aufwerfen wird und die Bollwerke bauen, daß viel Leute umgebracht werden.
\par 18 Denn weil er den Eid verachtet und den Bund gebrochen hat, darauf er seine Hand gegeben hat, und solches alles tut, wird er nicht davonkommen.
\par 19 Darum spricht der HERR HERR also; So wahr ich lebe, so will ich meinen Eid, den er verachtet hat, und meinen Bund, den er gebrochen hat, auf seinen Kopf bringen.
\par 20 Denn ich will mein Netz über ihn werfen, und er muß in meinem Garn gefangen werden; und ich will ihn gen Babel bringen und will daselbst mit ihm rechten über dem, daß er sich also an mir vergriffen hat.
\par 21 Und alle seine Flüchtigen, die ihm anhingen, sollen durchs Schwert fallen, und ihre übrigen sollen in alle Winde zerstreut werden; und ihr sollt's erfahren, daß ich, der HERR, es geredet habe.
\par 22 So spricht der HERR HERR: Ich will auch von dem Wipfel des hohen Zedernbaumes nehmen und oben auf seinen Zweigen ein zartes Reis brechen und will's auf einen hohen, erhabenen Berg pflanzen;
\par 23 auf den hohen Berg Israels will ich's pflanzen, daß es Zweige gewinne und Früchte bringe und ein herrlicher Zedernbaum werde, also daß allerlei Vögel unter ihm wohnen und allerlei Fliegendes unter dem Schatten seiner Zweige bleiben möge.
\par 24 Und sollen alle Feldbäume erfahren, daß ich, der HERR, den hohen Baum erniedrigt habe und den niedrigen Baum erhöht habe und den grünen Baum ausgedörrt und den dürren Baum grünend gemacht habe. Ich, der HERR, rede es und tue es auch.

\chapter{18}

\par 1 Und des HERRN Wort geschah zu mir und sprach:
\par 2 Was treibt ihr unter euch im Lande Israel dies Sprichwort und sprecht: "Die Väter haben Herlinge gegessen, aber den Kindern sind die Zähne davon stumpf geworden"?
\par 3 So wahr als ich lebe, spricht der HERR HERR, solches Sprichwort soll nicht mehr unter euch gehen in Israel.
\par 4 Denn siehe, alle Seelen sind mein; des Vaters Seele ist sowohl mein als des Sohnes Seele. Welche Seele sündigt, die soll sterben.
\par 5 Wenn nun einer fromm ist, der recht und wohl tut,
\par 6 der auf den Bergen nicht isset, der seine Augen nicht aufhebt zu den Götzen des Hauses Israel und seines Nächsten Weib nicht befleckt und liegt nicht bei der Frau in ihrer Krankheit,
\par 7 der niemand beschädigt, der dem Schuldner sein Pfand wiedergibt, der niemand etwas mit Gewalt nimmt, der dem Hungrigen sein Brot mitteilt und den Nackten kleidet,
\par 8 der nicht wuchert, der nicht Zins nimmt, der seine Hand vom Unrechten kehrt, der zwischen den Leuten recht urteilt,
\par 9 der nach meinen Rechten wandelt und meine Gebote hält, daß er ernstlich darnach tue: das ist ein frommer Mann, der soll das leben haben, spricht der HERR HERR.
\par 10 Wenn er aber einen Sohn zeugt, und derselbe wird ein Mörder, der Blut vergießt oder dieser Stücke eins tut,
\par 11 und der andern Stücke keins tut, sondern auf den Bergen isset und seines Nächsten Weib befleckt,
\par 12 die Armen und Elenden beschädigt, mit Gewalt etwas nimmt, das Pfand nicht wiedergibt, seine Augen zu den Götzen aufhebt und einen Greuel begeht,
\par 13 auf Wucher gibt, Zins nimmt: sollte der Leben? Er soll nicht leben, sondern weil er solche Greuel alle getan hat, soll er des Todes sterben; sein Blut soll auf ihm sein.
\par 14 Wo er aber einen Sohn zeugt, der solche Sünden sieht, so sein Vater tut, und sich fürchtet und nicht also tut,
\par 15 ißt nicht auf den Bergen, hebt seine Augen nicht auf zu den Götzen des Hauses Israel, befleckt nicht seines Nächsten Weib,
\par 16 beschädigt niemand, behält das Pfand nicht, nimmt nicht mit Gewalt etwas, teilt sein Brot mit dem Hungrigen und kleidet den Nackten,
\par 17 der seine Hand vom Unrechten kehrt, keinen Wucher noch Zins nimmt, sondern meine Gebote hält und nach meinen Rechten lebt: der soll nicht sterben um seines Vaters Missetat willen, sondern leben.
\par 18 Aber sein Vater, der Gewalt und Unrecht geübt hat und unter seinem Volk getan hat, was nicht taugt, siehe, der soll sterben um seiner Missetat willen.
\par 19 So sprecht ihr: Warum soll denn ein Sohn nicht tragen seines Vaters Missetat? Darum daß er recht und wohl getan und alle meine Rechte gehalten und getan hat, soll er leben.
\par 20 Denn welche Seele sündigt, die soll sterben. Der Sohn soll nicht tragen die Missetat des Vaters, und der Vater soll nicht tragen die Missetat des Sohnes; sondern des Gerechten Gerechtigkeit soll über ihm sein.
\par 21 Wo sich aber der Gottlose bekehrt von allen seine Sünden, die er getan hat, und hält alle meine Rechte und tut recht und wohl, so soll er leben und nicht sterben.
\par 22 Es soll aller seiner Übertretung, so er begangen hat, nicht gedacht werden; sondern er soll leben um der Gerechtigkeit willen, die er tut.
\par 23 Meinest du, daß ich Gefallen habe am Tode des Gottlosen, spricht der HERR, und nicht vielmehr, daß er sich bekehre von seinem Wesen und lebe?
\par 24 Und wo sich der Gerechte kehrt von seiner Gerechtigkeit und tut Böses und lebt nach all den Greueln, die ein Gottloser tut, sollte der leben? Ja, aller seiner Gerechtigkeit, die er getan hat, soll nicht gedacht werden; sondern in seiner Übertretung und Sünde, die er getan hat, soll er sterben.
\par 25 Doch sprecht ihr: Der HERR handelt nicht recht. So hört nun, ihr vom Hause Israel: Ist's nicht also, daß ich recht habe und ihr unrecht habt?
\par 26 Denn wenn der Gerechte sich kehrt von seiner Gerechtigkeit und tut Böses, so muß er sterben; er muß aber um seiner Bosheit willen, die er getan hat, sterben.
\par 27 Wiederum, wenn der Gottlose kehrt von seiner Ungerechtigkeit, die er getan hat, und tut nun recht und wohl, der wird seine Seele lebendig erhalten.
\par 28 Denn weil er sieht und bekehrt sich von aller Bosheit, die er getan hat, so soll er leben und nicht sterben.
\par 29 Doch sprechen die vom Hause Israel: Der HERR handelt nicht recht. Sollte ich Unrecht haben? Ihr vom Hause Israel habt unrecht.
\par 30 Darum will ich euch richten, ihr vom Hause Israel einen jeglichen nach seinem Wesen, spricht der HERR HERR. Darum so bekehrt euch von aller Übertretung, auf daß ihr nicht fallen müsset um der Missetat willen.
\par 31 Werfet von euch alle eure Übertretung, damit ihr übertreten habt, und machet euch ein neues Herz und einen neuen Geist. Denn warum willst du sterben, du Haus Israel?
\par 32 Denn ich habe keinen Gefallen am Tode des Sterbenden, spricht der HERR HERR. Darum bekehrt euch, so werdet ihr leben.

\chapter{19}

\par 1 Du aber mache eine Wehklage über die Fürsten Israels
\par 2 und sprich: Warum liegt deine Mutter, die Löwin, unter den Löwen und erzieht ihre Jungen unter den Löwen?
\par 3 Deren eines zog sie auf, und ward ein junger Löwe daraus, der gewöhnte sich, die Leute zu zerreißen und zu fressen.
\par 4 Da das die Heiden von ihm hörten, fingen sie ihn in ihren Gruben und führten ihn an Ketten nach Ägyptenland.
\par 5 Da nun die Mutter sah, daß ihre Hoffnung verloren war, da sie lange gehofft hatte, nahm sie ein anderes aus ihren Jungen heraus und machte einen jungen Löwen daraus.
\par 6 Da er unter den Löwen wandelte ward er ein junger Löwe; der gewöhnte sich auch, die Leute zu zerreißen und zu fressen.
\par 7 Er verderbte ihre Paläste und verwüstete ihre Städte, daß das Land und was darin ist, vor der Stimme seines Brüllens sich entsetzte.
\par 8 Da legten sich die Heiden aus allen Ländern ringsumher und warfen ein Netz über ihn und fingen ihn in ihren Gruben
\par 9 und stießen ihn gebunden in einen Käfig und führten ihn zum König zu Babel; und man ließ ihn verwahren, daß seine Stimme nicht mehr gehört würde auf den Bergen Israels.
\par 10 Deine Mutter war wie ein Weinstock, gleich wie du am Wasser gepflanzt; und seine Frucht und Reben wuchsen von dem großen Wasser,
\par 11 daß seine Reben so stark wurden, daß sie zu Herrenzeptern gut waren, und er ward hoch unter den Reben. Und da man sah, daß er so hoch war und viel Reben hatte,
\par 12 ward er mit Grimm ausgerissen und zu Boden geworfen; der Ostwind verdorrte seine Frucht, und seine starken Reben wurden zerbrochen, daß sie verdorrten und verbrannt wurden.
\par 13 Nun aber ist er gepflanzt in der Wüste, in einem dürren, durstigen Lande,
\par 14 und ist ein Feuer ausgegangen von seinen starken Reben, das verzehrte seine Frucht, daß in ihm keine starke Rebe mehr ist zu einem Herrenzepter, das ist ein kläglich und jämmerlich Ding.

\chapter{20}

\par 1 Und es begab sich im siebenten Jahr, am zehnten Tage des fünften Monats, kamen etliche aus den Ältesten Israels, den HERRN zu fragen, und setzten sich vor mir nieder.
\par 2 Da geschah des HERRN Wort zu mir und sprach:
\par 3 Du Menschenkind, sage den Ältesten Israels und sprich zu ihnen: So spricht der HERR HERR: Seid ihr gekommen, mich zu fragen? So wahr ich lebe, ich will von euch ungefragt sein, spricht der HERR HERR.
\par 4 Aber willst du sie strafen, du Menschenkind, so magst du sie also strafen: zeige ihnen an die Greuel ihrer Väter
\par 5 und sprich zu ihnen: So spricht der HERR HERR: Zu der Zeit, da ich Israel erwählte, erhob ich meine Hand zu dem Samen des Hauses Jakob und gab mich ihnen zu erkennen in Ägyptenland. Ja, ich erhob meine Hand zu ihnen und sprach: Ich bin der HERR, euer Gott.
\par 6 Ich erhob aber zur selben Zeit meine Hand, daß ich sie führte aus Ägyptenland in ein Land, das ich ihnen ersehen hatte, das mit Milch und Honig fließt, ein edles Land vor allen Ländern,
\par 7 und sprach zu ihnen: Ein jeglicher werfe weg die Greuel vor seinen Augen, und verunreinigt euch nicht an den Götzen Ägyptens! denn ich bin der HERR, euer Gott.
\par 8 Sie aber waren mir ungehorsam und wollten nicht gehorchen und warf ihrer keiner weg die Greuel vor seinen Augen und verließen die Götzen Ägyptens nicht. Da dachte ich meinem Grimm über sie auszuschütten und all mein Zorn über sie gehen zu lassen noch in Ägyptenland.
\par 9 Aber ich ließ es um meines Namens willen, daß er nicht entheiligt würde vor den Heiden, unter denen sie waren und vor denen ich mich ihnen hatte zu erkennen gegeben, daß ich sie aus Ägyptenland führen wollte.
\par 10 Und da ich sie aus Ägyptenland geführt hatte und in die Wüste gebracht,
\par 11 gab ich ihnen meine Gebote und lehrte sie meine Rechte, durch welche lebt der Mensch, der sie hält.
\par 12 Ich gab ihnen auch meine Sabbate zum Zeichen zwischen mir und ihnen, damit sie lernten, daß ich der HERR sei, der sie heiligt.
\par 13 Aber das Haus Israel war mir ungehorsam auch in der Wüste und lebten nicht nach meinen Geboten und verachteten meine Rechte, durch welche der Mensch lebt, der sie hält, und entheiligten meine Sabbate sehr. Da gedachte ich meinem Grimm über sie auszuschütten in der Wüste und sie ganz umzubringen.
\par 14 Aber ich ließ es um meines Namens willen, auf daß er nicht entheiligt würde vor den Heiden, vor welchen ich sie hatte ausgeführt.
\par 15 Und ich hob auch meine Hand auf wider sie in der Wüste, daß ich sie nicht wollte bringen in das Land, so ich ihnen gegeben hatte, das mit Milch und Honig fließt, ein edles Land vor allen Ländern,
\par 16 darum daß sie meine Rechte verachtet und nach meinen Geboten nicht gelebt und meine Sabbate entheiligt hatten; denn sie wandelten nach den Götzen ihres Herzens.
\par 17 Aber mein Auge verschonte sie, daß ich sie nicht verderbte noch ganz umbrächte in der Wüste.
\par 18 Und ich sprach zu ihren Kindern in der Wüste: Ihr sollt nach eurer Väter Geboten nicht leben und ihre Rechte nicht halten und an ihren Götzen euch nicht verunreinigen.
\par 19 Denn ich bin der HERR, euer Gott; nach meinen Geboten sollt ihr leben, und meine Rechte sollt ihr halten und darnach tun;
\par 20 und meine Sabbate sollt ihr heiligen, daß sie seien ein Zeichen zwischen mir und euch, damit ihr wisset, das ich der HERR, euer Gott bin.
\par 21 Aber die Kinder waren mir auch ungehorsam, lebten nach meinen Geboten nicht, hielten auch meine Rechte nicht, daß sie darnach täten, durch welche der Mensch lebt, der sie hält, und entheiligten meine Sabbate. Da gedachte ich, meinen Grimm über sie auszuschütten und allen meinen Zorn über sie gehen lassen in der Wüste.
\par 22 Ich wandte aber meine Hand und ließ es um meines Namens willen, auf daß er nicht entheiligt würde vor den Heiden, vor welchen ich sie hatte ausgeführt.
\par 23 Ich hob auch meine Hand auf wider sie in der Wüste, daß ich sie zerstreute unter die Heiden und zerstäubte in die Länder,
\par 24 darum daß sie meine Geboten nicht gehalten und meine Rechte verachtet und meine Sabbate entheiligt hatten und nach den Götzen ihrer Väter sahen.
\par 25 Darum übergab ich sie in die Lehre, die nicht gut ist, und in Rechte, darin sie kein Leben konnten haben,
\par 26 und ließ sie unrein werden durch ihre Opfer, da sie alle Erstgeburt durchs Feuer gehen ließen, damit ich sie verstörte und sie lernen mußten, daß ich der HERR sei.
\par 27 Darum rede, du Menschenkind, mit dem Hause Israel und sprich zu ihnen: So spricht der HERR HERR: Eure Väter haben mich noch weiter gelästert und mir getrotzt.
\par 28 Denn da ich sie in das Land gebracht hatte, über welches ich meine Hand aufgehoben hatte, daß ich's ihnen gäbe: wo sie einen hohen Hügel oder dichten Baum ersahen, daselbst opferten sie ihre Opfer und brachten dahin ihre verdrießlichen Gaben und räucherten daselbst ihren süßen Geruch und gossen daselbst ihre Trankopfer.
\par 29 Ich aber sprach zu ihnen: Was soll doch die Höhe, dahin ihr geht? Und also heißt sie bis auf diesen Tag "die Höhe".
\par 30 Darum sprich zum Hause Israel: So spricht der HERR HERR: Ihr verunreinigt euch in dem Wesen eurer Väter und treibt Abgötterei mit ihren Greueln
\par 31 und verunreinigt euch an euren Götzen, welchen ihr eure Gaben opfert und eure Söhne und Töchter durchs Feuer gehen laßt, bis auf den heutigen Tag; und ich sollte mich von euch, Haus Israel, fragen lassen? So wahr ich lebe, spricht der HERR HERR, ich will von euch ungefragt sein.
\par 32 Dazu, was ihr gedenkt: "Wir wollen tun wie die Heiden und wie andere Leute in den Ländern: Holz und Stein anbeten", das soll euch fehlschlagen.
\par 33 So wahr ich lebe, spricht der HERR HERR, ich will über euch herrschen mit starker Hand und ausgestrecktem Arm und mit ausgeschüttetem Grimm
\par 34 und will euch aus den Völkern führen und aus den Ländern, dahin ihr verstreut seid, sammeln mit starker Hand und mit ausgestrecktem Arm und mit ausgeschütteten Grimm,
\par 35 und will euch bringen in die Wüste der Völker und daselbst mit euch rechten von Angesicht zu Angesicht.
\par 36 Wie ich mit euren Vätern in der Wüste bei Ägypten gerechtet habe, ebenso will ich auch mit euch rechten, spricht der HERR HERR.
\par 37 Ich will euch wohl unter die Rute bringen und euch in die Bande des Bundes zwingen
\par 38 und will die Abtrünnigen und so wider mich übertreten, unter euch ausfegen; ja, aus dem Lande, da ihr jetzt wohnt, will ich sie führen und ins Land Israel nicht kommen lassen, daß ihr lernen sollt, ich sei der HERR.
\par 39 Darum, ihr vom Hause Israel, so spricht der HERR HERR: Weil ihr denn mir ja nicht wollt gehorchen, so fahrt hin und diene ein jeglicher seinen Götzen; aber meinen heiligen Namen laßt hinfort ungeschändet mit euren Opfern und Götzen.
\par 40 Denn so spricht der HERR HERR: Auf meinem heiligen Berge, auf dem hohen Berge Israel, daselbst wird mir das ganze Haus Israel, alle die im Lande sind, dienen; daselbst werden sie mir angenehm sein, und daselbst will ich eure Hebopfer und Erstlinge eurer Opfer fordern mit allem, was ihr mir heiligt.
\par 41 Ihr werdet mir angenehm sein mit dem süßen Geruch, wenn ich euch aus den Völkern bringen und aus den Ländern sammeln werde, dahin ihr verstreut seid, und werde in euch geheiligt werden vor den Heiden.
\par 42 Und ihr werdet erfahren, daß ich der HERR bin, wenn ich euch ins Land Israel gebracht habe, in das Land, darüber ich meine Hand aufhob, daß ich's euren Vätern gäbe.
\par 43 Daselbst werdet ihr gedenken an euer Wesen und an all euer Tun, darin ihr verunreinigt seid, und werdet Mißfallen haben über eure eigene Bosheit, die ihr getan habt.
\par 44 Und werdet erfahren, daß ich der HERR bin, wenn ich mit euch tue um meines Namens willen und nicht nach eurem bösen Wesen und schädlichen Tun, du Haus Israel, spricht der HERR HERR.
\par 45 Und des HERRN Wort geschah zu mir und sprach:
\par 46 Du Menschenkind, richte dein Angesicht gegen den Südwind zu und predige gegen den Mittag und weissage wider den Wald im Felde gegen Mittag.
\par 47 Und sprich zum Walde gegen Mittag: Höre des HERRN Wort! So spricht der HERR HERR: Siehe, ich will in dir ein Feuer anzünden, das soll beide, grüne und dürre Bäume, verzehren, daß man seine Flamme nicht wird löschen können; sondern es soll verbrannt werden alles, was vom Mittag gegen Mitternacht steht.
\par 48 Und alles Fleisch soll sehen, daß ich, der HERR, es angezündet habe und niemand löschen kann.
\par 49 Und ich sprach: Ach HERR HERR, sie sagen von mir: Dieser redet eitel Rätselworte.

\chapter{21}

\par 1 Und des HERRN Wort geschah zu mir und sprach:
\par 2 Du Menschenkind, richte dein Angesicht wider Jerusalem und predige wider die Heiligtümer und weissage wider das Land Israel
\par 3 und sprich zum Lande Israel: So spricht der HERR HERR: Siehe, ich will an dich; ich will mein Schwert aus der Scheide ziehen und will in dir ausrotten beide, Gerechte und Ungerechte.
\par 4 Weil ich denn in dir Gerechte und Ungerechte ausrotte, so wird mein Schwert aus der Scheide fahren über alles Fleisch, von Mittag her bis gen Mitternacht.
\par 5 Und soll alles Fleisch erfahren, daß ich, der HERR, mein Schwert habe aus der Scheide gezogen; und es soll nicht wieder eingesteckt werden.
\par 6 Und du, Menschenkind, sollst seufzen, bis dir die Lenden weh tun, ja, bitterlich sollst du seufzen, daß sie es sehen.
\par 7 Und wenn sie zu dir sagen werden: Warum seufzest du? sollst du sagen: Um des Geschreis willen, das da kommt, vor welchem alle Herzen verzagen, und alle Hände sinken, aller Mut fallen und alle Kniee so ungewiß stehen werden wie Wasser. Siehe, es kommt und wird geschehen, spricht der HERR HERR.
\par 8 Und des HERRN Wort geschah zu mir und sprach:
\par 9 Du Menschenkind, weissage und sprich: So spricht der HERR: Sprich: Das Schwert, ja, das Schwert ist geschärft und gefegt.
\par 10 Es ist geschärft, daß es schlachten soll; es ist gefegt, daß es blinken soll. O wie froh wollten wir sein, wenn er gleich alle Bäume zu Ruten machte über die bösen Kinder!
\par 11 Aber er hat ein Schwert zu fegen gegeben, daß man es fassen soll; es ist geschärft und gefegt, daß man's dem Totschläger in die Hand gebe.
\par 12 Schreie und heule, du Menschenkind; denn es geht über mein Volk und über alle Regenten in Israel, die dem Schwert samt meinem Volk verfallen sind. Darum schlage auf deine Lenden.
\par 13 Denn er hat sie oft gezüchtigt; was hat's geholfen? Es will der bösen Kinder Rute nicht helfen, spricht der HERR HERR.
\par 14 Und du, Menschenkind, weissage und schlage deine Hände zusammen. Denn das Schwert wird zweifach, ja dreifach kommen, ein Würgeschwert, ein Schwert großer Schlacht, das sie auch treffen wird in den Kammern, dahin sie fliehen.
\par 15 Ich will das Schwert lassen klingen, daß die Herzen verzagen und viele fallen sollen an allen ihren Toren. Ach, wie glänzt es und haut daher zur Schlacht!
\par 16 Haue drein, zur Rechten und Linken, was vor dir ist!
\par 17 Da will ich dann mit meinen Händen darob frohlocken und meinen Zorn gehen lassen. Ich, der HERR, habe es gesagt.
\par 18 Und des HERRN Wort geschah zu mir und sprach:
\par 19 Du Menschenkind, mache zwei Wege, durch welche kommen soll das Schwert des Königs zu Babel; sie sollen aber alle beide aus einem Lande gehen.
\par 20 Und stelle ein Zeichen vorn an den Weg zur Stadt, dahin es weisen soll; und mache den Weg, daß das Schwert komme gen Rabba der Kinder Ammon und nach Juda, zu der festen Stadt Jerusalem.
\par 21 Denn der König zu Babel wird sich an die Wegscheide stellen, vorn an den zwei Wegen, daß er sich wahrsagen lasse, mit den Pfeilen das Los werfe, seinen Abgott frage und schaue die Leber an.
\par 22 Und die Wahrsagung wird auf die rechte Seite gen Jerusalem deuten, daß er solle Sturmböcke hinanführen lassen und Löcher machen und mit großem Geschrei sie überfalle und morde, und daß er Böcke führen soll wider die Tore und da Wall aufschütte und Bollwerk baue.
\par 23 Aber es wird sie solches Wahrsagen falsch dünken, er schwöre, wie teuer er will. Er aber wird denken an die Missetat, daß er sie gewinne.
\par 24 Darum spricht der HERR HERR also: Darum daß euer gedacht wird um eure Missetat und euer Ungehorsam offenbart ist, daß man eure Sünden sieht in allem eurem Tun, ja, darum daß euer gedacht wird, werdet ihr mit Gewalt gefangen werden.
\par 25 Und du, Fürst in Israel, der du verdammt und verurteilt bist, dessen Tag daherkommen wird, wenn die Missetat zum Ende gekommen ist,
\par 26 so spricht der HERR HERR: Tue weg den Hut und hebe ab die Krone! Denn es wird weder Hut noch die Krone bleiben; sondern der sich erhöht hat, der soll erniedrigt werden, und der sich erniedrigt, soll erhöht werden.
\par 27 Ich will die Krone zunichte, zunichte, zunichte machen, bis der komme, der sie haben soll; dem will ich sie geben.
\par 28 Und du, Menschenkind, weissage und sprich: So spricht der HERR HERR von den Kindern Ammon und von ihrem Schmähen; und sprich: Das Schwert, das Schwert ist gezückt, daß es schlachten soll; es ist gefegt, daß es würgen soll und soll blinken,
\par 29 darum daß du falsche Gesichte dir sagen läßt und Lügen weissagen, damit du auch hingegeben wirst unter die erschlagenen Gottlosen, welchen ihr Tag kam, da die Missetat zum Ende gekommen war.
\par 30 Und ob's schon wieder in die Scheide gesteckt würde, so will ich dich doch richten an dem Ort, da du geschaffen, und in dem Lande, da du geboren bist,
\par 31 und will meinen Zorn über dich schütten; ich will das Feuer meines Grimmes über dich aufblasen und will dich Leuten, die brennen und verderben können, überantworten.
\par 32 Du mußt dem Feuer zur Speise werden, und dein Blut muß im Lande vergossen werden, und man wird dein nicht mehr gedenken; denn ich, der HERR, habe es geredet.

\chapter{22}

\par 1 Und des HERRN Wort geschah zu mir und sprach:
\par 2 Du Menschenkind, willst du nicht strafen die mörderische Stadt und ihr anzeigen alle ihre Greuel?
\par 3 Sprich: So spricht der HERR HERR: O Stadt, die du der Deinen Blut vergießest, auf daß deine Zeit komme, und die du Götzen bei dir machst, dadurch du dich verunreinigst!
\par 4 Du verschuldest dich an dem Blut, das du vergießt, und verunreinigst dich an den Götzen, die du machst; damit bringst du deine Tage herzu und machst, daß deine Jahre kommen müssen. Darum will ich dich zum Spott unter den Heiden und zum Hohn in allen Ländern machen.
\par 5 In der Nähe und in der Ferne sollen sie dein spotten, daß du ein schändlich Gerücht haben und großen Jammer leiden müssest.
\par 6 Siehe, die Fürsten in Israel, ein jeglicher ist mächtig bei dir, Blut zu vergießen.
\par 7 Vater und Mutter verachten sie, den Fremdlingen tun sie Gewalt und Unrecht, die Witwen und die Waisen schinden sie.
\par 8 Du verachtest meine Heiligtümer und entheiligst meine Sabbate.
\par 9 Verräter sind in dir, auf daß sie Blut vergießen. Sie essen auf den Bergen und handeln mutwillig in dir;
\par 10 sie decken auf die Blöße der Väter und nötigen die Weiber in ihrer Krankheit
\par 11 und treiben untereinander, Freund mit Freundes Weibe, Greuel; sie schänden ihre eigene Schwiegertochter mit allem Mutwillen; sie notzüchtigen ihre eigenen Schwestern, ihres Vaters Töchter;
\par 12 sie nehmen Geschenke, auf daß sie Blut vergießen; sie wuchern und nehmen Zins voneinander und treiben ihren Geiz wider ihren Nächsten und tun einander Gewalt und vergessen mein also, spricht der HERR HERR.
\par 13 Siehe, ich schlage meine Hände zusammen über den Geiz, den du treibst, und über das Blut, so in dir vergossen ist.
\par 14 Meinst du aber, dein Herz möge es erleiden, oder werden es deine Hände ertragen zu der Zeit, wann ich mit dir handeln werde? Ich, der HERR, habe es geredet und will's auch tun
\par 15 und will dich zerstreuen unter die Heiden und dich verstoßen in die Länder und will deinem Unflat ein Ende machen,
\par 16 daß du bei den Heiden mußt verflucht geachtet werden und erfahren, daß ich der HERR sei.
\par 17 Und des HERRN Wort geschah zu mir und sprach:
\par 18 Du Menschenkind, das Haus Israel ist mir zu Schlacken geworden und sind alle Erz, Zinn, Eisen und Blei im Ofen; ja, zu Silberschlacken sind sie geworden.
\par 19 Darum spricht der HERR HERR also: Weil ihr denn alle Schlacken geworden seid, siehe, so will ich euch alle gen Jerusalem zusammentun.
\par 20 Wie man Silber, Erz, Eisen, Blei und Zinn zusammentut im Ofen, daß man ein Feuer darunter aufblase und zerschmelze es, also will ich euch auch in meinem Zorn und Grimm zusammentun, einlegen und schmelzen.
\par 21 Ja ich will euch sammeln und das Feuer meines Zorns unter euch aufblasen, daß ihr darin zerschmelzen müsset.
\par 22 Wie das Silber zerschmilzt im Ofen, so sollt ihr auch darin zerschmelzen und erfahren, daß ich, der HERR, meinen Grimm über euch ausgeschüttet habe.
\par 23 Und des HERRN Wort geschah zu mir und sprach:
\par 24 Du Menschenkind, sprich zu ihnen: Du bist ein Land, das nicht zu reinigen ist, wie eines, das nicht beregnet wird zur Zeit des Zorns.
\par 25 Die Propheten, so darin sind, haben sich gerottet, die Seelen zu fressen wie ein brüllender Löwe, wenn er raubt; sie reißen Gut und Geld an sich und machen der Witwen viel darin.
\par 26 Ihre Priester verkehren mein Gesetz freventlich und entheiligen mein Heiligtum; sie halten unter dem Heiligen und Unheiligen keinen Unterschied und lehren nicht, was rein oder unrein sei, und warten meiner Sabbate nicht, und ich werde unter ihnen entheiligt.
\par 27 Ihre Fürsten sind darin wie die reißenden Wölfe, Blut zu vergießen und Seelen umzubringen um ihres Geizes willen.
\par 28 Und ihre Propheten tünchen ihnen mit losem Kalk, predigen loses Gerede und weissagen ihnen Lügen und sagen: "So spricht der HERR HERR", so es doch der HERR nicht geredet hat.
\par 29 Das Volk im Lande übt Gewalt; sie rauben getrost und schinden die Armen und Elenden und tun den Fremdlingen Gewalt und Unrecht.
\par 30 Ich suchte unter ihnen, ob jemand sich zur Mauer machte und wider den Riß stünde vor mir für das Land, daß ich's nicht verderbte; aber ich fand keinen.
\par 31 Darum schüttete ich meinen Zorn über sie, und mit dem Feuer meines Grimmes machte ich ihnen ein Ende und gab ihnen also ihren Verdienst auf ihren Kopf, spricht der HERR HERR.

\chapter{23}

\par 1 Und des HERRN Wort geschah zu mir und sprach:
\par 2 Du Menschenkind, es waren zwei Weiber, einer Mutter Töchter.
\par 3 Die trieben Hurerei in Ägypten in ihrer Jugend; daselbst ließen sie ihre Brüste begreifen und den Busen ihrer Jungfrauschaft betasten.
\par 4 Die große heißt Ohola und ihre Schwester Oholiba. Und ich nahm sie zur Ehe, und sie gebaren mir Söhne und Töchter. Und Ohola heißt Samaria und Oholiba Jerusalem.
\par 5 Ohola trieb Hurerei, da ich sie genommen hatte, und brannte gegen ihre Buhlen, nämlich gegen die Assyrer, die zu ihr kamen,
\par 6 gegen die Fürsten und Herren, die mit Purpur gekleidet waren, und alle junge, liebliche Gesellen, Reisige, so auf Rossen ritten.
\par 7 Und sie buhlte mit allen schönen Gesellen in Assyrien und verunreinigte sich mit allen ihren Götzen, wo sie auf einen entbrannte.
\par 8 Dazu ließ sie auch nicht die Hurerei mit Ägypten, die bei ihr gelegen hatten von ihrer Jugend auf und die Brüste ihrer Jungfrauschaft betastet und große Hurerei mit ihr getrieben hatten.
\par 9 Da übergab ich sie in die Hand ihrer Buhlen, den Kindern Assur, gegen welche sie brannte vor Lust.
\par 10 Die deckten ihre Blöße auf und nahmen ihre Söhne und Töchter weg; sie aber töteten sie mit dem Schwert. Und es kam aus unter den Weibern, wie sie gestraft wäre.
\par 11 Da es aber ihre Schwester Oholiba sah, entbrannte sie noch viel ärger denn jene und trieb die Hurerei mehr denn ihre Schwester;
\par 12 und entbrannte gegen die Kinder Assur, nämlich die Fürsten und Herren, die zu ihr kamen wohl gekleidet, Reisige, so auf Rossen ritten, und alle junge, liebliche Gesellen.
\par 13 Da sah ich, daß sie alle beide gleichermaßen verunreinigt waren.
\par 14 Aber diese treib ihre Hurerei mehr. Denn da sie sah gemalte Männer an der Wand in roter Farbe, die Bilder der Chaldäer,
\par 15 um ihre Lenden gegürtet und bunte Mützen auf ihren Köpfen, und alle gleich anzusehen wie gewaltige Leute, wie denn die Kinder Babels, die Chaldäer, tragen in ihrem Vaterlande:
\par 16 entbrannte sie gegen sie, sobald sie ihrer gewahr ward, und schickte Botschaft zu ihnen nach Chaldäa.
\par 17 Als nun die Kinder Babels zu ihr kamen, bei ihr zu schlafen nach der Liebe, verunreinigten sie dieselbe mit ihrer Hurerei, und sie verunreinigte sich mit ihnen, bis sie ihrer müde ward.
\par 18 Und da ihre Hurerei und Schande so gar offenbar war, ward ich ihrer überdrüssig, wie ich ihrer Schwester auch war müde geworden.
\par 19 Sie aber trieb ihre Hurerei immer mehr und gedachte an die Zeit ihrer Jugend, da sie in Ägyptenland Hurerei getrieben hatte,
\par 20 und entbrannte gegen ihre Buhlen, welcher Brunst war wie der Esel und der Hengste Brunst.
\par 21 Und du bestelltest deine Unzucht wie in deiner Jugend, da die in Ägypten deine Brüste begriffen und deinen Busen betasteten.
\par 22 Darum, Oholiba, so spricht der HERR HERR: Siehe, ich will deine Buhlen, deren du müde bist geworden, wider dich erwecken und will sie ringsumher wider dich bringen,
\par 23 nämlich die Kinder Babels und alle Chaldäer mit Hauptleuten, Fürsten und Herren und alle Assyrer mit ihnen, die schöne junge Mannschaft, alle Fürsten und Herren, Ritter und Edle, die alle auf Rossen reiten.
\par 24 Und sie werden über dich kommen, gerüstet mit Wagen und Rädern und mit großem Haufen Volks, und werden dich belagern mit Tartschen, Schilden und Helmen um und um. Denen will ich das Recht befehlen, daß sie dich richten sollen nach ihrem Recht.
\par 25 Ich will meinen Eifer über dich gehen lassen, daß sie unbarmherzig mit dir handeln sollen. Sie sollen dir Nase und Ohren abschneiden; und was übrigbleibt, soll durchs Schwert fallen. Sie sollen deine Söhne und Töchter wegnehmen und das übrige mit Feuer verbrennen.
\par 26 Sie sollen dir deine Kleider ausziehen und deinen Schmuck wegnehmen.
\par 27 Also will ich deiner Unzucht und deiner Hurerei mit Ägyptenland ein Ende machen, daß du deine Augen nicht mehr nach ihnen aufheben und Ägyptens nicht mehr gedenken sollst.
\par 28 Denn so spricht der HERR HERR: Siehe, ich will dich überantworten, denen du feind geworden und deren du müde bist.
\par 29 Die sollen wie Feinde mit dir umgehen und alles nehmen, was du erworben hast, und dich nackt und bloß lassen, daß die Schande deiner Unzucht und Hurerei offenbar werde.
\par 30 Solches wird dir geschehen um deiner Hurerei willen, so du mit den Heiden getrieben, an deren Götzen du dich verunreinigt hast.
\par 31 Du bist auf dem Wege deiner Schwester gegangen; darum gebe ich dir auch deren Kelch in deine Hand.
\par 32 So spricht der HERR HERR: Du mußt den Kelch deiner Schwester trinken, so tief und weit er ist: du sollst zu so großem Spott und Hohn werden, daß es unerträglich sein wird.
\par 33 Du mußt dich des starken Tranks und Jammers vollsaufen; denn der Kelch deiner Schwester Samaria ist ein Kelch des Jammers und Trauerns.
\par 34 Denselben mußt du rein austrinken, darnach die Scherben zerwerfen und deine Brüste zerreißen; denn ich habe es geredet, spricht der HERR HERR.
\par 35 Darum so spricht der HERR HERR: Darum, daß du mein vergessen und mich hinter deinen Rücken geworfen hast, so trage auch nun deine Unzucht und deine Hurerei.
\par 36 Und der HERR sprach zu mir; Du Menschenkind, willst du nicht Ohola und Oholiba strafen und ihnen zeigen ihre Greuel?
\par 37 Wie sie Ehebrecherei getrieben und Blut vergossen und die Ehe gebrochen haben mit den Götzen; dazu ihre Kinder, die sie mir geboren hatten, verbrannten sie denselben zum Opfer.
\par 38 Überdas haben sie mir das getan: sie haben meine Heiligtümer verunreinigt dazumal und meine Sabbate entheiligt.
\par 39 Denn da sie ihre Kinder den Götzen geschlachtet hatten, gingen sie desselben Tages in mein Heiligtum, es zu entheiligen. Siehe, solches haben sie in meinem Hause begangen.
\par 40 Sie haben auch Boten geschickt nach Leuten, die aus fernen Landen kommen sollten; und siehe, da sie kamen, badetest du dich und schminktest dich und schmücktest dich mit Geschmeide zu ihren Ehren
\par 41 und saßest auf einem herrlichen Polster, vor welchem stand ein Tisch zugerichtet; darauf legtest du mein Räuchwerk und mein Öl.
\par 42 Daselbst erhob sich ein großes Freudengeschrei; und es gaben ihnen die Leute, so allenthalben aus großem Volk und aus der Wüste gekommen waren, Geschmeide an ihre Arme und schöne Kronen auf ihre Häupter.
\par 43 Ich aber gedachte: Sie ist der Ehebrecherei gewohnt von alters her; sie kann von der Hurerei nicht lassen.
\par 44 Denn man geht zu ihr ein, wie man zu einer Hure eingeht; ebenso geht man zu Ohola und Oholiba, den unzüchtigen Weibern.
\par 45 Darum werden sie die Männer strafen, die das Recht vollbringen, wie man die Ehebrecherinnen und Blutvergießerinnen strafen soll. Denn sie sind Ehebrecherinnen, und ihre Hände sind voll Blut.
\par 46 Also spricht der HERR HERR: Führe einen großen Haufen über sie herauf und gib sie zu Raub und Beute,
\par 47 daß die Leute sie steinigen und mit ihren Schwertern erstechen und ihre Söhne und Töchter erwürgen und ihre Häuser mit Feuer verbrennen.
\par 48 Also will ich der Unzucht im Lande ein Ende machen, daß alle Weiber sich warnen lassen und nicht nach solcher Unzucht tun.
\par 49 Und man soll eure Unzucht auf euch legen, und ihr sollt eurer Götzen Sünden tragen, auf daß ihr erfahret, daß ich der HERR HERR bin.

\chapter{24}

\par 1 Und es geschah das Wort des HERRN zu mir im neunten Jahr, am zehnten Tage des zehnten Monats, und sprach:
\par 2 Du Menschenkind, schreib diesen Tag an, ja, eben diesen Tag; denn der König zu Babel hat sich eben an diesem Tage wider Jerusalem gelagert.
\par 3 Und gib dem ungehorsamen Volk ein Gleichnis und sprich zu ihnen: So spricht der HERR HERR: Setze einen Topf zu, setze zu und gieß Wasser hinein;
\par 4 tue die Stücke zusammen darein, die hinein sollen, alle besten Stücke, die Lenden und Schultern, und fülle ihn mit den besten Knochenstücken;
\par 5 nimm das Beste von der Herde und mache ein Feuer darunter, Knochenstücke zu kochen, und laß es getrost sieden und die Knochenstücke darin wohl kochen.
\par 6 Darum spricht der HERR HERR: O der mörderischen Stadt, die ein solcher Topf ist, da der Rost daran klebt und nicht abgehen will! Tue ein Stück nach dem andern heraus; und du darfst nicht darum losen, welches zuerst heraus soll.
\par 7 Denn ihr Blut ist darin, das sie auf einen bloßen Felsen und nicht auf die Erde verschüttet hat, da man's doch hätte mit Erde können zuscharren.
\par 8 Und ich habe auch darum sie lassen das Blut auf einen bloßen Felsen schütten, daß es nicht zugescharrt würde, auf daß der Grimm über sie käme und es gerächt würde.
\par 9 Darum spricht der HERR HERR also: O du mörderische Stadt, welche ich will zu einem großen Feuer machen!
\par 10 Trage nur viel Holz her, zünde das Feuer an, daß das Fleisch gar werde, und würze es wohl, und die Knochenstücke sollen anbrennen.
\par 11 Lege auch den Topf leer auf die Glut, auf das er heiß werde und sein Erz entbrenne, ob seine Unreinigkeit zerschmelzen und sein Rost abgehen wolle.
\par 12 Aber wie sehr er brennt, will sein Rost doch nicht abgehen, denn es ist zuviel des Rosts; er muß im Feuer zerschmelzen.
\par 13 Deine Unreinigkeit ist so verhärtet, daß, ob ich dich gleich reinigen wollte, dennoch du nicht willst dich reinigen lassen von deiner Unreinigkeit. Darum kannst du hinfort nicht wieder rein werden, bis mein Grimm sich an dir gekühlt habe.
\par 14 Ich, der HERR, habe es geredet! Es soll kommen, ich will's tun und nicht säumen; ich will nicht schonen noch mich's reuen lassen; sondern sie sollen dich richten, wie du gelebt und getan hast, spricht der HERR HERR.
\par 15 Und des HERRN Wort geschah zu mir und sprach:
\par 16 Du Menschenkind, siehe, ich will dir deiner Augen Lust nehmen durch eine Plage, aber du sollst nicht klagen noch weinen noch eine Träne lassen.
\par 17 Heimlich magst du seufzen, aber keine Totenklage führen; sondern du sollst deinen Schmuck anlegen und deine Schuhe anziehen. Du sollst deinen Mund nicht verhüllen und nicht das Trauerbrot essen.
\par 18 Und da ich des Morgens früh zum Volk geredet hatte, starb mir am Abend mein Weib. Und ich tat des andern Morgens, wie mir befohlen war.
\par 19 Und das Volk sprach zu mir: Willst du uns nicht anzeigen, was uns das bedeutet, was du tust?
\par 20 Und ich sprach zu ihnen: Der HERR hat mir geredet und gesagt:
\par 21 Sage dem Hause Israel, daß der HERR HERR spricht also: Siehe, ich will mein Heiligtum, euren höchsten Trost, die Lust eurer Augen und eures Herzens Wunsch entheiligen; und euer Söhne und Töchter, die ihr verlassen mußtet, werden durchs Schwert fallen.
\par 22 Und müßt tun, wie ich getan habe: euren Mund sollt ihr nicht verhüllen und das Trauerbrot nicht essen,
\par 23 sondern sollt euren Schmuck auf euer Haupt setzen und eure Schuhe anziehen. Ihr werdet nicht klagen noch weinen, sondern über eure Sünden verschmachten und untereinander seufzen.
\par 24 Und soll also Hesekiel euch ein Wunderzeichen sein, daß ihr tun müßt, wie er getan hat, wenn es nun kommen wird, damit ihr erfahrt, daß ich der HERR HERR bin.
\par 25 Und du, Menschenkind, zu der Zeit, wann ich wegnehmen werde von ihnen ihre Macht und ihren Trost, die Lust ihrer Augen und des Herzens Wunsch, ihre Söhne und Töchter,
\par 26 ja, zur selben Zeit wird einer, so entronnen ist, zu dir kommen und dir's kundtun.
\par 27 Zur selben Zeit wird dein Mund aufgetan werden samt dem, der entronnen ist, daß du reden sollst und nicht mehr schweigen; denn du mußt ihr Wunderzeichen sein, daß sie erfahren, ich sei der HERR.

\chapter{25}

\par 1 Und des HERRN Wort geschah zu mir und sprach:
\par 2 Du Menschenkind, richte dein Angesicht gegen die Kinder Ammon und weissage wider sie
\par 3 und sprich zu den Kindern Ammon: Höret des HERRN HERRN Wort! So spricht der HERR HERR: Darum daß ihr über mein Heiligtum sprecht: "Ha! es ist entheiligt!" und über das Land Israel: "Es ist verwüstet!" und über das Haus Juda: "Es ist gefangen weggeführt!",
\par 4 darum siehe, ich will dich den Kindern des Morgenlandes übergeben, daß sie ihre Zeltdörfer in dir bauen und ihre Wohnungen in dir machen sollen; sie sollen deine Früchte essen und deine Milch trinken.
\par 5 Und will Rabba zum Kamelstall machen und das Land der Kinder Ammon zu Schafhürden machen; und ihr sollt erfahren, daß ich der HERR bin.
\par 6 Denn so spricht der HERR HERR: Darum daß du mit deinen Händen geklatscht und mit den Füßen gescharrt und über das Land Israel von ganzem Herzen so höhnisch dich gefreut hast,
\par 7 darum siehe, ich will meine Hand über dich ausstrecken und dich den Heiden zur Beute geben und dich aus den Völkern ausrotten und aus den Ländern umbringen und dich vertilgen; und sollst erfahren, daß ich der HERR bin.
\par 8 So spricht der HERR HERR: Darum daß Moab und Seir sprechen: Siehe das Haus Juda ist eben wie alle Heiden!
\par 9 siehe, so will ich Moab zur Seite öffnen in seinen Städten und in seinen Grenzen, das edle Land von Beth-Jesimoth, Baal-Meon und Kirjathaim,
\par 10 und will es den Kindern des Morgenlandes zum Erbe geben samt dem Lande der Kinder Ammon, daß man der Kinder Ammon nicht mehr gedenken soll unter den Heiden.
\par 11 Und will das Recht gehen lassen über Moab; und sie sollen erfahren, daß ich der HERR bin.
\par 12 So spricht der HERR HERR: Darum daß sich Edom am Hause Juda gerächt hat und sich verschuldet mit seinem Rächen,
\par 13 darum spricht der HERR HERR also: Ich will meine Hand ausstrecken über Edom und will ausrotten von ihm Menschen und Vieh und will es wüst machen von Theman bis gen Dedan und durchs Schwert fällen;
\par 14 und will mich an Edom rächen durch mein Volk Israel, und sie sollen mit Edom umgehen nach meinem Zorn und Grimm, daß sie meine Rache erfahren sollen, spricht der HERR HERR.
\par 15 So spricht der HERR HERR: Darum daß die Philister sich gerächt haben und den alten Haß gebüßt nach allem ihrem Willen am Schaden meines Volkes,
\par 16 darum spricht der HERR HERR also: Siehe, ich will meine Hand ausstrecken über die Philister und die Krether ausrotten und will die übrigen am Ufer des Meeres umbringen;
\par 17 und will große Rache an ihnen üben und mit Grimm sie strafen, daß sie erfahren sollen, ich sei der HERR, wenn ich meine Rache an ihnen geübt habe.

\chapter{26}

\par 1 Und es begab sich im elften Jahr, am ersten Tage des ersten Monats, geschah des HERRN Wort zu mir und sprach:
\par 2 Du Menschenkind, darum daß Tyrus spricht über Jerusalem: "Ha! die Pforte der Völker ist zerbrochen; es ist zu mir gewandt; ich werde nun voll werden, weil sie wüst ist!",
\par 3 darum spricht der HERR HERR also: Siehe, ich will an dich, Tyrus, und will viele Heiden über dich heraufbringen, gleich wie sich ein Meer erhebt mit seinen Wellen.
\par 4 Die sollen die Mauern zu Tyrus verderben und ihre Türme abbrechen; ja ich will auch ihren Staub von ihr wegfegen und will einen bloßen Fels aus ihr machen
\par 5 und einen Ort am Meer, darauf man die Fischgarne aufspannt; denn ich habe es geredet, spricht der HERR HERR, und sie soll den Heiden zum Raub werden.
\par 6 Und ihre Töchter, so auf dem Felde liegen, sollen durchs Schwert erwürgt werden und sollen erfahren, daß ich der HERR bin.
\par 7 Denn so spricht der HERR HERR: Siehe, ich will über Tyrus kommen lassen Nebukadnezar, den König zu Babel, von Mitternacht her, der ein König aller Könige ist, mit Rossen, Wagen, Reitern und mit großem Haufen Volks.
\par 8 Der soll deine Töchter, so auf dem Felde liegen, mit dem Schwert erwürgen; aber wider dich wird er Bollwerke aufschlagen und einen Wall aufschütten und Schilde wider dich rüsten.
\par 9 Er wird mit Sturmböcken deine Mauern zerstoßen und deine Türme mit seinen Werkzeugen umreißen.
\par 10 Der Staub von der Menge seiner Pferde wird dich bedecken; so werden auch deine Mauern erbeben vor dem Getümmel seiner Rosse, Räder und Reiter, wenn er zu deinen Toren einziehen wird, wie man pflegt in eine zerrissene Stadt einzuziehen.
\par 11 Er wird mit den Füßen seiner Rosse alle deine Gassen zertreten. Dein Volk wird er mit dem Schwert erwürgen und deine starken Säulen zu Boden reißen.
\par 12 Sie werden dein Gut rauben und deinen Handel plündern. Deine Mauern werden sie abbrechen und deine feinen Häuser umreißen und werden deine Steine, Holz und Staub ins Wasser werfen.
\par 13 Also will ich mit Getön deines Gesanges ein Ende machen, daß man den Klang deiner Harfen nicht mehr hören soll.
\par 14 Und ich will einen bloßen Fels aus dir machen und einen Ort, darauf man Fischgarne aufspannt, daß du nicht mehr gebaut wirst; denn ich bin der HERR, der solches redet, spricht der HERR HERR.
\par 15 So spricht der HERR HERR wider Tyrus: was gilt's? die Inseln werden erbeben, wenn du so greulich zerfallen wirst und deine Verwundeten seufzen werden, so in dir sollen ermordet werden.
\par 16 Alle Fürsten am Meer werden herab von ihren Stühlen steigen und ihre Röcke von sich tun und ihre gestickten Kleider ausziehen und werden in Trauerkleidern gehen und auf der Erde sitzen und werden erschrecken und sich entsetzen über deinen plötzlichen Fall.
\par 17 Sie werden über dich wehklagen und von dir sagen: Ach, wie bist du so gar wüst geworden, du berühmte Stadt, die du am Meer lagst und so mächtig warst auf dem Meer samt deinen Einwohnern, daß sich das ganze Land vor dir fürchten mußte!
\par 18 Ach, wie entsetzen sich die Inseln über deinen Fall! ja die Inseln im Meer erschrecken über deinen Untergang.
\par 19 Denn so spricht der HERR HERR: Ich will dich zu einer wüsten Stadt machen wie andere Städte, darin niemand wohnt, und eine große Flut über dich kommen lassen, daß dich große Wasser bedecken,
\par 20 und will dich hinunterstoßen zu denen, die in die Grube gefahren sind, zu dem Volk der Toten. Ich will dich unter die Erde hinabstoßen in die ewigen Wüsten zu denen, die in die Grube gefahren sind, auf daß niemand in dir wohne. Ich will dich, du Prächtige im Lande der Lebendigen,
\par 21 ja, zum Schrecken will ich dich machen, daß du nichts mehr seist; und wenn man nach dir fragt, daß man dich ewiglich nimmer finden könne, spricht der HERR HERR.

\chapter{27}

\par 1 Und des HERRN Wort geschah zu mir und sprach:
\par 2 Du Menschenkind, mach eine Wehklage über Tyrus
\par 3 und sprich zu Tyrus, die da liegt vorn am Meer und mit vielen Inseln der Völker handelt: So spricht der HERR HERR: O Tyrus, du sprichst: Ich bin die Allerschönste.
\par 4 Deine Grenzen sind mitten im Meer und deine Bauleute haben dich aufs allerschönste zugerichtet.
\par 5 Sie haben all dein Tafelwerk aus Zypressenholz vom Senir gemacht und die Zedern vom Libanon führen lassen und deine Mastbäume daraus gemacht
\par 6 und deine Ruder von Eichen aus Basan und deine Bänke von Elfenbein, gefaßt in Buchsbaumholz aus den Inseln der Chittiter.
\par 7 Dein Segel war von gestickter, köstlicher Leinwand aus Ägypten, daß es dein Panier wäre, und deine Decken von blauem und rotem Purpur aus den Inseln Elisa.
\par 8 Die von Sidon und Arvad waren deine Ruderknechte, und hattest geschickte Leute zu Tyrus, zu schiffen.
\par 9 Die Ältesten und Klugen von Gebal mußten deine Risse bessern. Alle Schiffe im Meer und ihre Schiffsleute fand man bei dir; die hatten ihren Handel in dir.
\par 10 Die aus Persien, Lud und Lybien waren dein Kriegsvolk, die ihre Schilde und Helme in dir aufhingen und haben dich so schön geschmückt.
\par 11 Die von Arvad waren unter deinem Heer rings um die Mauern und Wächter auf deinen Türmen; die haben ihre Schilde allenthalben von deinen Mauern herabgehängt und dich so schön geschmückt.
\par 12 Tharsis hat dir mit seinem Handel gehabt und allerlei Waren, Silber, Eisen, Zinn und Blei auf die Märkte gebracht.
\par 13 Javan, Thubal und Mesech haben mit dir gehandelt und haben dir leibeigene Leute und Geräte von Erz auf deine Märkte gebracht.
\par 14 Die von Thogarma haben dir Rosse und Wagenpferde und Maulesel auf deine Märkte gebracht.
\par 15 Die von Dedan sind deine Händler gewesen, und hast allenthalben in den Inseln gehandelt; die haben dir Elfenbein und Ebenholz verkauft.
\par 16 Die Syrer haben bei dir geholt deine Arbeit, was du gemacht hast, und Rubine, Purpur, Teppiche, feine Leinwand und Korallen und Kristalle auf deine Märkte gebracht.
\par 17 Juda und das Land Israel haben auch mit dir gehandelt und haben Weizen von Minnith und Balsam und Honig und Öl und Mastix auf deine Märkte gebracht.
\par 18 Dazu hat auch Damaskus bei dir geholt deine Arbeit und allerlei Ware um Wein von Helbon und köstliche Wolle.
\par 19 Dan und Javan und Mehusal haben auch auf deine Märkte gebracht Eisenwerk, Kassia und Kalmus, daß du damit handeltest.
\par 20 Dedan hat mit dir gehandelt mit Decken zum Reiten.
\par 21 Arabien und alle Fürsten von Kedar haben mit dir gehandelt mit Schafen, Widdern und Böcken.
\par 22 Die Kaufleute aus Saba und Ragma haben mit dir gehandelt und allerlei köstliche Spezerei und Edelsteine und Gold auf deine Märkte gebracht.
\par 23 Haran und Kanne und Eden samt den Kaufleuten aus Seba, Assur und Kilmad sind auch deine Händler gewesen.
\par 24 Die haben alle mit dir gehandelt mit köstlichem Gewand, mit purpurnen und gestickten Tüchern, welche sie in köstlichen Kasten, von Zedern gemacht und wohl verwahrt, auf deine Märkte geführt haben.
\par 25 Aber die Tharsisschiffe sind die vornehmsten auf deinen Märkten gewesen. Also bist du sehr reich und prächtig geworden mitten im Meer.
\par 26 Deine Ruderer haben dich oft auf große Wasser geführt; ein Ostwind wird dich mitten auf dem Meer zerbrechen,
\par 27 also daß dein Reichtum, dein Kaufgut, deine Ware, deine Schiffsleute, deine Schiffsherren und die, so deine Risse bessern und die deinen Handel treiben und alle deine Kriegsleute und alles Volk in dir mitten auf dem Meer umkommen werden zur Zeit, wann du untergehst;
\par 28 daß auch die Anfurten erbeben werden vor dem Geschrei deiner Schiffsherren.
\par 29 Und alle, die an den Rudern ziehen, samt den Schiffsknechten und Meistern werden aus ihren Schiffen ans Land treten
\par 30 und laut über dich schreien, bitterlich klagen und werden Staub auf ihre Häupter werfen und sich in der Asche wälzen.
\par 31 Sie werden sich kahl scheren über dir und Säcke um sich gürten und von Herzen bitterlich um dich weinen und trauern.
\par 32 Es werden auch ihre Kinder über dich wehklagen: Ach! wer ist jemals auf dem Meer so still geworden wie du, Tyrus?
\par 33 Da du deinen Handel auf dem Meer triebst, da machtest du viele Länder reich, ja, mit der Menge deiner Ware und deiner Kaufmannschaft machtest du reich die Könige auf Erden.
\par 34 Nun aber bist du vom Meer in die rechten, tiefen Wasser gestürzt, daß dein Handel und all dein Volk in dir umgekommen ist.
\par 35 Alle die auf den Inseln wohnen, erschrecken über dich, und ihre Könige entsetzen sich und sehen jämmerlich.
\par 36 Die Kaufleute in den Ländern pfeifen dich an, daß du so plötzlich untergegangen bist und nicht mehr aufkommen kannst.

\chapter{28}

\par 1 Und des HERRN Wort geschah zu mir und sprach:
\par 2 Du Menschenkind, sage dem Fürsten zu Tyrus: So spricht der HERR HERR: Darum daß sich dein Herz erhebt und spricht: "Ich bin Gott, ich sitze auf dem Thron Gottes mitten im Meer", so du doch ein Mensch und nicht Gott bist, doch erhebt sich dein Herz, als wäre es eines Gottes Herz:
\par 3 siehe, du hältst dich für klüger denn Daniel, daß dir nichts verborgen sei
\par 4 und habest durch deine Klugheit und deinen Verstand solche Macht zuwege gebracht und Schätze von Gold und Silber gesammelt
\par 5 und habest durch deine große Weisheit und Hantierung so große Macht überkommen; davon bist du so stolz geworden, daß du so mächtig bist;
\par 6 darum spricht der HERR HERR also: Weil sich denn dein Herz erhebt, als wäre es eines Gottes Herz,
\par 7 darum, siehe, ich will Fremde über dich schicken, nämlich die Tyrannen der Heiden; die sollen ihr Schwert zücken über deine schöne Weisheit und deine große Ehre zu Schanden machen.
\par 8 Sie sollen dich hinunter in die Grube stoßen, daß du mitten auf dem Meer stirbst wie die Erschlagenen.
\par 9 Was gilt's, ob du dann vor deinem Totschläger wirst sagen: "Ich bin Gott", so du doch nicht Gott, sondern ein Mensch und in deiner Totschläger Hand bist?
\par 10 Du sollst sterben wie die Unbeschnittenen von der Hand der Fremden; denn ich habe es geredet, spricht der HERR HERR.
\par 11 Und des HERRN Wort geschah zu mir und sprach:
\par 12 Du Menschenkind, mache eine Wehklage über den König zu Tyrus und sprich von Ihm: So spricht der HERR HERR: Du bist ein reinliches Siegel, voller Weisheit und aus der Maßen schön.
\par 13 Du bist im Lustgarten Gottes und mit allerlei Edelsteinen geschmückt: mit Sarder, Topas, Demant, Türkis, Onyx, Jaspis, Saphir, Amethyst, Smaragd und Gold. Am Tage, da du geschaffen wurdest, mußten da bereitet sein bei dir deine Pauken und Pfeifen.
\par 14 Du bist wie ein Cherub, der sich weit ausbreitet und decket; und ich habe dich auf den heiligen Berg Gottes gesetzt, daß du unter den feurigen Steinen wandelst.
\par 15 Du warst ohne Tadel in deinem Tun von dem Tage an, da du geschaffen wurdest, bis sich deine Missetat gefunden hat.
\par 16 Denn du bist inwendig voll Frevels geworden vor deiner großen Hantierung und hast dich versündigt. Darum will ich dich entheiligen von dem Berge Gottes und will dich ausgebreiteten Cherub aus den feurigen Steinen verstoßen.
\par 17 Und weil sich dein Herz erhebt, daß du so schön bist, und hast dich deine Klugheit lassen betrügen in deiner Pracht, darum will ich dich zu Boden stürzen und ein Schauspiel aus dir machen vor den Königen.
\par 18 Denn du hast dein Heiligtum verderbt mit deiner großen Missetat und unrechtem Handel. Darum will ich ein Feuer aus dir angehen lassen, das dich soll verzehren, und will dich zu Asche machen auf der Erde, daß alle Welt zusehen soll.
\par 19 Alle, die dich kennen unter den Heiden, werden sich über dich entsetzen, daß du so plötzlich bist untergegangen und nimmermehr aufkommen kannst.
\par 20 Und des HERRN Wort geschah zu mir und sprach:
\par 21 Du Menschenkind, richte dein Angesicht wider Sidon und weissage wider sie
\par 22 und sprich: So spricht der HERR HERR: Siehe, ich will an dich, Sidon, und will an dir Ehre einlegen, daß man erfahren soll, daß ich der HERR bin, wenn ich das Recht über sie gehen lasse und an ihr erzeige, daß ich heilig sei.
\par 23 Und ich will Pestilenz und Blutvergießen unter sie schicken auf ihren Gassen, und sie sollen tödlich verwundet drinnen fallen durchs Schwert, welches allenthalben über sie gehen wird; und sollen erfahren, daß ich der HERR bin.
\par 24 Und forthin sollen allenthalben um das Haus Israel, da ihre Feinde sind, keine Dornen, die da stechen, noch Stacheln, die da wehe tun, bleiben, daß sie erfahren, daß ich der HERR HERR bin.
\par 25 So spricht der HERR HERR: Wenn ich das Haus Israel wieder versammeln werde von den Völkern, dahin sie zerstreut sind, so will ich vor den Heiden an ihnen erzeigen, daß ich heilig bin. Und sie sollen wohnen in ihrem Lande, das ich meinem Knecht Jakob gegeben habe;
\par 26 und sollen sicher darin wohnen und Häuser bauen und Weinberge pflanzen; ja, sicher sollen sie wohnen, wenn ich das Recht gehen lasse über alle ihre Feinde um und um; und sollen erfahren, daß ich, der HERR, ihr Gott bin.

\chapter{29}

\par 1 Im zehnten Jahr, am zwölften Tage des zehnten Monats, geschah des HERRN Wort zu mir und sprach:
\par 2 Du Menschenkind, richte dein Angesicht wider Pharao, den König in Ägypten, und weissage wider ihn und wider ganz Ägyptenland.
\par 3 Predige und sprich: So spricht der HERR HERR: Siehe, ich will an dich, Pharao, du König in Ägypten, du großer Drache, der du in deinem Wasser liegst und sprichst: Der Strom ist mein, und ich habe ihn mir gemacht.
\par 4 Aber ich will dir ein Gebiß ins Maul legen, und die Fische in deinen Wassern an deine Schuppen hängen und will dich aus deinem Strom herausziehen samt allen Fischen in deinen Wassern, die an deinen Schuppen hangen.
\par 5 Ich will dich mit den Fischen aus deinen Wassern in die Wüste wegwerfen; du wirst aufs Land fallen und nicht wieder aufgelesen noch gesammelt werden, sondern den Tieren auf dem Lande und den Vögeln des Himmels zur Speise werden.
\par 6 Und alle, die in Ägypten wohnen, sollen erfahren, daß ich der HERR bin; darum daß sie dem Hause Israel ein Rohrstab gewesen sind.
\par 7 Wenn sie ihn in die Hand faßten, so brach er und stach sie in die Seite; wenn sie sich darauf lehnten, so zerbrach er und stach sie in die Lenden.
\par 8 Darum spricht der HERR HERR also: Siehe, ich will das Schwert über dich kommen lassen und Leute und Vieh in dir ausrotten.
\par 9 Und Ägyptenland soll zur Wüste und Öde werden, und sie sollen erfahren, daß ich der HERR sei, darum daß du sprichst: Der Wasserstrom ist mein, und ich bin's, der's tut.
\par 10 Darum, siehe, ich will an dich und an deine Wasserströme und will Ägyptenland wüst und öde machen von Migdol bis gen Syene und bis an die Grenze des Mohrenlandes,
\par 11 daß weder Vieh noch Leute darin gehen oder da wohnen sollen vierzig Jahre lang.
\par 12 Denn ich will Ägyptenland wüst machen wie andere wüste Länder und ihre Städte wüst liegen lassen wie andere wüste Städte vierzig Jahre lang; und will die Ägypter zerstreuen unter die Heiden, und in die Länder will ich sie verjagen.
\par 13 Doch so spricht der HERR HERR: Wenn die vierzig Jahre aus sein werden, will ich die Ägypter wieder sammeln aus den Völkern, darunter sie zerstreut sollen werden,
\par 14 und will das Gefängnis Ägyptens wenden und sie wiederum ins Land Pathros bringen, welches ihr Vaterland ist; und sie sollen daselbst ein kleines Königreich sein.
\par 15 Denn sie sollen klein sein gegen andere Königreiche und nicht mehr sich erheben über die Heiden; und ich will sie gering machen, damit sie nicht über die Heiden herrschen sollen,
\par 16 daß sich das Haus Israel nicht mehr auf sie verlasse und sich damit versündige, wenn sie sich an sie hängen; und sie sollen erfahren, daß ich der HERR HERR bin.
\par 17 Und es begab sich im siebenundzwanzigsten Jahr, am ersten Tage des ersten Monats, geschah des HERRN Wort zu mir und sprach:
\par 18 Du Menschenkind, Nebukadnezar, der König zu Babel, hat sein Heer mit großer Mühe vor Tyrus arbeiten lassen, daß alle Häupter kahl und alle Schultern wund gerieben waren; und ist doch weder ihm noch seinem Heer seine Arbeit vor Tyrus belohnt worden.
\par 19 Darum spricht der HERR HERR also: Siehe, ich will Nebukadnezar, dem König zu Babel, Ägyptenland geben, daß er all ihr Gut wegnehmen und sie berauben und plündern soll, daß er seinem Heer den Sold gebe.
\par 20 Zum Lohn für seine Arbeit, die er getan hat, will ich ihm das Land Ägypten geben; denn sie haben mir gedient, spricht der HERR HERR.
\par 21 Zur selben Zeit will ich das Horn des Hauses Israel wachsen lassen und will deinen Mund unter ihnen auftun, daß sie erfahren, daß ich der HERR bin.

\chapter{30}

\par 1 Und des HERRN Wort geschah zu mir und sprach:
\par 2 Du Menschenkind, weissage und sprich: So spricht der HERR HERR: Heult: "O weh des Tages!"
\par 3 Denn der Tag ist nahe, ja, des HERRN Tag ist nahe, ein finsterer Tag; die Zeit der Heiden kommt.
\par 4 Und das Schwert soll über Ägypten kommen; und Mohrenland muß erschrecken, wenn die Erschlagenen in Ägypten fallen werden und sein Volk weggeführt und seine Grundfesten umgerissen werden.
\par 5 Mohrenland und Libyen und Lud mit allerlei Volk und Chub und die aus dem Lande des Bundes sind, sollen samt ihnen durchs Schwert fallen.
\par 6 So spricht der HERR: Die Schutzherren Ägyptens müssen fallen, und die Hoffart seiner Macht muß herunter; von Migdol bis gen Syene sollen sie durchs Schwert fallen, spricht der HERR HERR.
\par 7 Und sie sollen wie andere wüste Länder wüst werden, und ihre Städte unter andren wüsten Städten wüst liegen,
\par 8 daß sie erfahren, daß ich der HERR sei, wenn ich ein Feuer in Ägypten mache, daß alle, die ihnen helfen, verstört werden.
\par 9 Zur selben Zeit werden Boten von mir ausziehen in Schiffen, Mohrenland zu schrecken, das jetzt so sicher ist; und wird ein Schrecken unter ihnen sein, gleich wie es Ägypten ging, da seine Zeit kam; denn siehe, es kommt gewiß.
\par 10 So spricht der HERR HERR: Ich will die Menge in Ägypten wegräumen durch Nebukadnezar, den König zu Babel.
\par 11 Denn er und sein Volk mit ihm, die Tyrannen der Heiden, sind herzugebracht, das Land zu verderben, und werden ihre Schwerter ausziehen wider Ägypten, daß das Land allenthalben voll Erschlagener liege.
\par 12 Und ich will die Wasserströme trocken machen und das Land bösen Leuten verkaufen, und will das Land und was darin ist, durch Fremde verwüsten. Ich, der HERR, habe es geredet.
\par 13 So spricht der HERR HERR: Ich will die Götzen zu Noph ausrotten und die Abgötter vertilgen, und Ägypten soll keinen Fürsten mehr haben, und ich will einen Schrecken in Ägyptenland schicken.
\par 14 Ich will Pathros wüst machen und ein Feuer zu Zoan anzünden und das Recht über No gehen lassen
\par 15 und will meinen Grimm ausschütten über Sin, die Festung Ägyptens, und will die Menge zu No ausrotten.
\par 16 Ich will ein Feuer in Ägypten anzünden, und Sin soll angst und bange werden, und No soll zerrissen und Noph täglich geängstet werden.
\par 17 Die junge Mannschaft zu On und Bubastus sollen durchs Schwert fallen und die Weiber gefangen weggeführt werden.
\par 18 Thachphanhes wird einen finstern Tag haben, wenn ich das Joch Ägyptens daselbst zerbrechen werde, daß die Hoffart seiner Macht darin ein Ende habe; sie wird mit Wolken bedeckt werden, und ihre Töchter werden gefangen weggeführt werden.
\par 19 Und ich will das Recht über Ägypten gehen lassen, daß sie erfahren, daß ich der HERR sei.
\par 20 Und es begab sich im elften Jahr, am siebenten Tage des elften Monats, geschah des HERRN Wort zu mir und sprach:
\par 21 Du Menschenkind, ich habe den Arm Pharaos, des Königs von Ägypten, zerbrochen; und siehe, er soll nicht verbunden werden, daß er heilen möge, noch mit Binden zugebunden werden, daß er stark werde und ein Schwert fassen könne.
\par 22 Darum spricht der HERR HERR also: Siehe, ich will an Pharao, den König von Ägypten, und will seine Arme zerbrechen, beide, den starken und den zerbrochenen, daß ihm das Schwert aus seiner Hand entfallen muß;
\par 23 und ich will die Ägypter unter die Heiden zerstreuen und in die Länder verjagen.
\par 24 Aber die Arme des Königs zu Babel will ich stärken und ihm mein Schwert in seine Hand geben, und will die Arme Pharaos zerbrechen, daß er vor ihm winseln soll wie ein tödlich Verwundeter.
\par 25 Ja, ich will die Arme des Königs zu Babel stärken, daß die Arme Pharaos dahinfallen, auf daß sie erfahren, daß ich der HERR sei, wenn ich mein Schwert dem König zu Babel in die Hand gebe, daß er's über Ägyptenland zücke,
\par 26 und ich die Ägypter unter die Heiden zerstreue und in die Länder verjage, daß sie erfahren, daß ich der HERR bin.

\chapter{31}

\par 1 Und es begab sich im elften jahr, am ersten Tage des dritten Monats, geschah des HERRN Wort zu mir und sprach:
\par 2 Du Menschenkind, sage zu Pharao, dem König von Ägypten, und zu allem seinem Volk: Wem meinst du denn, daß du gleich seist in deiner Herrlichkeit?
\par 3 Siehe, Assur war wie ein Zedernbaum auf dem Libanon, von schönen Ästen und dick von Laub und sehr hoch, daß sein Wipfel hoch stand unter großen, dichten Zweigen.
\par 4 Die Wasser machten, daß er groß ward, und die Tiefe, daß er hoch wuchs. Ihre Ströme gingen rings um seinen Stamm her und ihre Bäche zu allen Bäumen im Felde.
\par 5 Darum ist er höher geworden als alle Bäume im Felde und kriegte viel Äste und lange Zweige; denn er hatte Wasser genug, sich auszubreiten.
\par 6 Alle Vögel des Himmels nisteten auf seinen Ästen, und alle Tiere im Felde hatten Junge unter seinen Zweigen; und unter seinem Schatten wohnten alle großen Völker.
\par 7 Er hatte schöne, große und lange Äste; denn seine Wurzeln hatten viel Wasser.
\par 8 Und war ihm kein Zedernbaum gleich in Gottes Garten, und die Tannenbäume waren seinen Ästen nicht zu vergleichen, und die Kastanienbäume waren nichts gegen seine Zweige. Ja, er war so schön wie kein Baum im Garten Gottes.
\par 9 Ich hatte ihn so schön gemacht, daß er so viel Äste kriegte, daß ihn alle lustigen Bäume im Garten Gottes neideten.
\par 10 Darum spricht der HERR HERR also: Weil er so hoch geworden ist, daß sein Wipfel stand unter großen, hohen, dichten Zweigen, und sein Herz sich erhob, daß er so hoch geworden war,
\par 11 darum gab ich ihn dem Mächtigen unter den Heiden in die Hände, daß der mit ihm umginge und ihn vertriebe, wie er verdient hat mit seinem gottlosen Wesen,
\par 12 daß Fremde ihn ausrotten sollten, nämlich die Tyrannen der Heiden, und ihn zerstreuen, und seine Äste auf den Bergen und in allen Tälern liegen mußten und seine Zweige zerbrachen an allen Bächen im Lande; daß alle Völker auf Erden von seinem Schatten wegziehen mußten und ihn verlassen;
\par 13 und alle Vögel des Himmels auf seinem umgefallenen Stamm saßen und alle Tiere im Felde sich legten auf seine Äste;
\par 14 auf daß sich forthin kein Baum am Wasser seiner Höhe überhebe, daß sein Wipfel unter großen, dichten Zweigen stehe, und kein Baum am Wasser sich erhebe über die andern; denn sie müssen alle unter die Erde und dem Tod übergeben werden wie andere Menschen, die in die Grube fahren.
\par 15 So spricht der HERR HERR: Zu der Zeit, da er hinunter in die Hölle fuhr, da machte ich ein Trauern, daß ihn die Tiefe bedeckte und seine Ströme stillstehen mußten und die großen Wasser nicht laufen konnten; und machte, daß der Libanon um ihn trauerte und alle Feldbäume verdorrten über ihm.
\par 16 Ich erschreckte die Heiden, da sie ihn hörten fallen, da ich ihn hinunterließ zur Hölle, zu denen, so in die Grube gefahren sind. Und alle lustigen Bäume unter der Erde, die edelsten und besten auf dem Libanon, und alle, die am Wasser gestanden hatten, gönnten's ihm wohl.
\par 17 Denn sie mußten auch mit ihm hinunter zur Hölle, zu den Erschlagenen mit dem Schwert, weil sie unter dem Schatten seines Arms gewohnt hatten unter den Heiden.
\par 18 Wie groß meinst du denn, Pharao, daß du seist mit deiner Pracht und Herrlichkeit unter den lustigen Bäumen? Denn du mußt mit den lustigen Bäumen unter die Erde hinabfahren und unter den Unbeschnittenen liegen, so mit dem Schwert erschlagen sind. Also soll es Pharao gehen samt allem seinem Volk, spricht der HERR HERR.

\chapter{32}

\par 1 Und es begab sich im zwölften Jahr, am ersten Tage des zwölften Monats, geschah des HERRN Wort zu mir und sprach:
\par 2 Du Menschenkind, mache eine Wehklage über Pharao, den König von Ägypten, und sprich zu ihm: Du bist gleich wie ein Löwe unter den Heiden und wie ein Meerdrache und springst in deinen Strömen und rührst das Wasser auf mit deinen Füßen und machst seine Ströme trüb.
\par 3 So spricht der HERR HERR: Ich will mein Netz über dich auswerfen durch einen großen Haufen Volks, die dich sollen in mein Garn jagen;
\par 4 und will dich aufs Land ziehen und aufs Feld werfen, daß alle Vögel des Himmels auf dir sitzen sollen und alle Tiere auf Erden von dir satt werden.
\par 5 Und will dein Aas auf die Berge werfen und mit deiner Höhe die Täler ausfüllen.
\par 6 Das Land, darin du schwimmst, will ich von deinem Blut rot machen bis an die Berge hinan, daß die Bäche von dir voll werden.
\par 7 Und wenn du nun ganz dahin bist, so will ich den Himmel verhüllen und seine Sterne verfinstern und die Sonne mit Wolken überziehen, und der Mond soll nicht scheinen.
\par 8 Alle Lichter am Himmel will ich über dir lassen dunkel werden, und will eine Finsternis in deinem Lande machen, spricht der HERR HERR.
\par 9 Dazu will ich vieler Völker Herz erschreckt machen, wenn ich die Heiden deine Plage erfahren lasse und viele Länder, die du nicht kennst.
\par 10 Viele Völker sollen sich über dich entsetzen, und ihren Königen soll vor dir grauen, wenn ich mein Schwert vor ihnen blinken lasse, und sollen plötzlich erschrecken, daß ihnen das Herz entfallen wird über deinen Fall.
\par 11 Denn so spricht der HERR HERR: Das Schwert des Königs zu Babel soll dich treffen.
\par 12 Und ich will dein Volk fällen durch das Schwert der Helden, durch allerlei Tyrannen der Heiden; die sollen die Herrlichkeit Ägyptens verheeren, daß all ihr Volk vertilgt werde.
\par 13 Und ich will alle Tiere umbringen an den großen Wassern, daß sie keines Menschen Fuß und keines Tieres Klaue mehr trüb machen soll.
\par 14 Alsdann will ich ihre Wasser lauter machen, daß ihre Ströme fließen wie Öl, spricht der HERR HERR,
\par 15 wenn ich das Land Ägypten verwüstet und alles, was im Lande ist, öde gemacht und alle, so darin wohnen, erschlagen habe, daß sie erfahren, daß ich der HERR sei.
\par 16 Das wird der Jammer sein, den man wohl mag klagen; ja, die Töchter der Heiden werden solche Klage führen; über Ägypten und all ihr Volk wird man klagen, spricht der HERR HERR.
\par 17 Und im zwölften Jahr, am fünfzehnten Tage desselben Monats, geschah des HERRN Wort zu mir und sprach:
\par 18 Du Menschenkind, beweine das Volk in Ägypten und stoße es mit den Töchtern der starken Heiden hinab unter die Erde zu denen, die in die Grube gefahren sind.
\par 19 Wo ist nun deine Wollust? Hinunter, und lege dich zu den Unbeschnittenen!
\par 20 Sie werden fallen unter denen, die mit dem Schwert erschlagen sind. Das Schwert ist schon gefaßt und gezückt über ihr ganzes Volk.
\par 21 Von ihm werden sagen in der Hölle die starken Helden mit ihren Gehilfen, die alle hinuntergefahren sind und liegen da unter den Unbeschnittenen und mit dem Schwert Erschlagenen.
\par 22 Daselbst liegt Assur mit allem seinem Volk umher begraben, die alle erschlagen und durchs Schwert gefallen sind;
\par 23 ihre Gräber sind tief in der Grube, und sein Volk liegt allenthalben umher begraben, die alle erschlagen und durchs Schwert gefallen sind, vor denen sich die ganze Welt fürchtete.
\par 24 Da liegt auch Elam mit allem seinem Haufen umher begraben, die alle erschlagen und durchs Schwert gefallen sind und hinuntergefahren als die Unbeschnittenen unter die Erde, vor denen sich auch alle Welt fürchtete; und müssen ihre Schande tragen mit denen, die in die Grube gefahren sind.
\par 25 Man hat sie unter die Erschlagenen gelegt samt allem ihrem Haufen, und liegen umher begraben; und sind alle, wie die Unbeschnittenen und mit dem Schwert Erschlagenen, vor denen sich auch alle Welt fürchten mußte; und müssen ihre Schande tragen mit denen, die in die Grube gefahren sind, und unter den Erschlagenen bleiben.
\par 26 Da liegt Mesech und Thubal mit allem ihrem Haufen umher begraben, die alle unbeschnitten sind, vor denen sich auch die ganze Welt fürchten mußte;
\par 27 und alle andern Helden, die unter den Unbeschnittenen gefallen und mit ihrer Kriegswehr zur Hölle gefahren sind und ihre Schwerter unter ihre Häupter haben müssen legen und deren Missetat über ihre Gebeine gekommen ist, die doch auch gefürchtete Helden waren in der ganzen Welt; also müssen sie liegen.
\par 28 So mußt du freilich auch unter den Unbeschnittenen zerschmettert werden und unter denen, die mit dem Schwert erschlagen sind, liegen.
\par 29 Da liegt Edom mit seinen Königen und alle seine Fürsten unter den Unbeschnittenen und mit dem Schwert Erschlagenen samt andern, so in die Grube gefahren sind, die doch mächtig waren.
\par 30 Da sind alle Fürsten von Mitternacht und alle Sidonier, die mit den Erschlagenen hinabgefahren sind; und ihre schreckliche Gewalt ist zu Schanden geworden, und müssen liegen unter den Unbeschnittenen und denen, so mit dem Schwert erschlagen sind, und ihre Schande tragen samt denen, die in die Grube gefahren sind.
\par 31 Diese wird Pharao sehen und sich trösten über all sein Volk, die unter ihm mit dem Schwert erschlagen sind, und über sein ganzes Heer, spricht der HERR HERR.
\par 32 Denn es soll sich auch einmal alle Welt vor mir fürchten, daß Pharao und alle seine Menge liegen unter den Unbeschnittenen und mit dem Schwert Erschlagenen, spricht der HERR HERR.

\chapter{33}

\par 1 Und des HERRN Wort geschah zu mir und sprach:
\par 2 Du Menschenkind, predige den Kindern deines Volkes und sprich zu ihnen: Wenn ich ein Schwert über das Land führen würde, und das Volk im Lande nähme einen Mann unter ihnen und machten ihn zu ihrem Wächter,
\par 3 und er sähe das Schwert kommen über das Land und bliese die Drommete und warnte das Volk,
\par 4 wer nun der Drommete Hall hörte und wollte sich nicht warnen lassen, und das Schwert käme und nähme ihn weg: desselben Blut sei auf seinem Kopf;
\par 5 denn er hat der Drommete Hall gehört und hat sich dennoch nicht warnen lassen; darum sei sein Blut auf ihm. Wer sich aber warnen läßt, der wird sein Leben davonbringen.
\par 6 Wo aber der Wächter sähe das Schwert kommen und die Drommete nicht bliese noch sein Volk warnte, und das Schwert käme und nähme etliche weg: dieselben würden wohl um ihrer Sünden willen weggenommen; aber ihr Blut will ich von des Wächters Hand fordern.
\par 7 Und nun, du Menschenkind, ich habe dich zum Wächter gesetzt über das Haus Israel, wenn du etwas aus meinem Munde hörst, daß du sie von meinetwegen warnen sollst.
\par 8 Wenn ich nun zu dem Gottlosen sage: Du Gottloser mußt des Todes sterben! und du sagst ihm solches nicht, daß sich der Gottlose warnen lasse vor seinem Wesen, so wird wohl der Gottlose um seines gottlosen Wesens willen sterben; aber sein Blut will ich von deiner Hand fordern.
\par 9 Warnst du aber den Gottlosen vor seinem Wesen, daß er sich davon bekehre, und er will sich nicht von seinem Wesen bekehren, so wird er um seiner Sünde sterben, und du hast deine Seele errettet.
\par 10 Darum, du Menschenkind, sage dem Hause Israel: Ihr sprecht also: Unsre Sünden und Missetaten liegen auf uns, daß wir darunter vergehen; wie können wir denn leben?
\par 11 So sprich zu ihnen: So wahr als ich lebe, spricht der HERR HERR, ich habe keinen Gefallen am Tode des Gottlosen, sondern daß sich der Gottlose bekehre von seinem Wesen und lebe. So bekehret euch doch nun von eurem bösen Wesen. Warum wollt ihr sterben, ihr vom Hause Israel?
\par 12 Und du, Menschenkind, sprich zu deinem Volk: Wenn ein Gerechter Böses tut, so wird's ihm nicht helfen, daß er fromm gewesen ist; und wenn ein Gottloser fromm wird, so soll's ihm nicht schaden, daß er gottlos gewesen ist. So kann auch der Gerechte nicht leben, wenn er sündigt.
\par 13 Denn wo ich zu dem Gerechten spreche, er soll leben, und er verläßt sich auf seine Gerechtigkeit und tut Böses, so soll aller seiner Frömmigkeit nicht gedacht werden; sondern er soll sterben in seiner Bosheit, die er tut.
\par 14 Und wenn ich zum Gottlosen spreche, er soll sterben und er bekehrt sich von seiner Sünde und tut, was recht und gut ist,
\par 15 also daß der Gottlose das Pfand wiedergibt und bezahlt, was er geraubt hat, und nach dem Wort des Lebens wandelt, daß er kein Böses tut: so soll er leben und nicht sterben,
\par 16 und aller seiner Sünden, die er getan hat, soll nicht gedacht werden; denn er tut nun, was recht und gut ist; darum soll er leben.
\par 17 Aber dein Volk spricht: Der HERR urteilt nicht recht, so sie doch unrecht haben.
\par 18 Denn wo der Gerechte sich kehrt von seiner Gerechtigkeit und tut Böses, so stirbt er ja billig darum.
\par 19 Und wo sich der Gottlose bekehrt von seinem gottlosen Wesen und tut, was recht und gut ist, so soll er ja billig leben.
\par 20 Doch sprecht ihr: Der HERR urteilt nicht recht, so ich doch euch vom Hause Israel einen jeglichen nach seinem Wesen richte.
\par 21 Und es begab sich im zwölften Jahr unserer Gefangenschaft, am fünften Tage des zehnten Monats, kam zu mir ein Entronnener von Jerusalem und sprach: Die Stadt ist geschlagen.
\par 22 Und die Hand des HERRN war über mir des Abends, ehe der Entronnene kam, und tat mir meinen Mund auf, bis er zu mir kam des Morgens; und tat mir meinen Mund auf, also daß ich nicht mehr schweigen mußte.
\par 23 Und des HERRN Wort geschah zu mir und sprach:
\par 24 Du Menschenkind, die Einwohner dieser Wüsten im Lande Israel sprechen also: Abraham war ein einziger Mann und erbte dies Land; unser aber sind viele, desto billiger gehört das Land uns zu.
\par 25 Darum sprich zu ihnen: So spricht der HERR HERR: Ihr habt Blutiges gegessen und eure Augen zu den Götzen aufgehoben und Blut vergossen: und ihr meint, ihr wollt das Land besitzen?
\par 26 Ja, ihr fahret immer fort mit Morden und übet Greuel, und einer schändet dem andern sein Weib; und ihr meint, ihr wollt das Land besitzen?
\par 27 So sprich zu ihnen: So spricht der HERR HERR: So wahr ich lebe, sollen alle, so in den Wüsten wohnen, durchs Schwert fallen; und die auf dem Felde sind, will ich den Tieren zu fressen geben; und die in den Festungen und Höhlen sind, sollen an der Pestilenz sterben.
\par 28 Denn ich will das Land ganz verwüsten und seiner Hoffart und Macht ein Ende machen, daß das Gebirge Israel so wüst werde, daß niemand dadurchgehe.
\par 29 Und sie sollen erfahren, daß ich der HERR bin, wenn ich das Land ganz verwüstet habe um aller ihrer Greuel willen, die sie üben.
\par 30 Und du, Menschenkind, dein Volk redet über dich an den Wänden und unter den Haustüren, und einer spricht zum andern: Kommt doch und laßt uns hören, was der HERR sage!
\par 31 Und sie werden zu dir kommen in die Versammlung und vor dir sitzen als mein Volk und werden deine Worte hören, aber nicht darnach tun; sondern sie werden sie gern in ihrem Munde haben, und gleichwohl fortleben in ihrem Geiz.
\par 32 Und siehe, du mußt ihnen sein wie ein liebliches Liedlein, wie einer, der eine schöne Stimme hat und wohl spielen kann. Also werden sie deine Worte hören und nicht darnach tun.
\par 33 Wenn es aber kommt, was kommen soll, siehe, so werden sie erfahren, daß ein Prophet unter ihnen gewesen ist.

\chapter{34}

\par 1 Und des HERRN Wort geschah zu mir und sprach:
\par 2 Du Menschenkind, weissage wider die Hirten Israels, weissage und sprich zu ihnen: So spricht der HERR HERR: Weh den Hirten Israels, die sich selbst weiden! Sollen nicht die Hirten die Herde weiden?
\par 3 Aber ihr fresset das Fette und kleidet euch mit der Wolle und schlachtet das Gemästete; aber die Schafe wollt ihr nicht weiden.
\par 4 Der Schwachen wartet ihr nicht, und die Kranken heilt ihr nicht, das Verwundete verbindet ihr nicht, das Verirrte holt ihr nicht und das Verlorene sucht ihr nicht; sondern streng und hart herrschet ihr über sie.
\par 5 Und meine Schafe sind zerstreut, als sie keinen Hirten haben, und allen wilden Tieren zur Speise geworden und gar zerstreut.
\par 6 Und gehen irre hin und wieder auf den Bergen und auf den hohen Hügeln und sind auf dem ganzen Lande zerstreut; und ist niemand, der nach ihnen frage oder ihrer achte.
\par 7 Darum höret, ihr Hirten, des HERRN Wort!
\par 8 So wahr ich lebe, spricht der HERR HERR, weil ihr meine Schafe lasset zum Raub und meine Herde allen wilden Tieren zur Speise werden, weil sie keinen Hirten haben und meine Hirten nach meiner Herde nicht fragen, sondern sind solche Hirten, die sich selbst weiden, aber meine Schafe wollen sie nicht weiden:
\par 9 darum, ihr Hirten, höret des HERRN Wort!
\par 10 So spricht der HERR HERR: Siehe, ich will an die Hirten und will meine Herde von ihren Händen fordern und will mit ihnen ein Ende machen, daß sie nicht mehr sollen Hirten sein und sollen sich nicht mehr selbst weiden. Ich will meine Schafe erretten aus ihrem Maul, daß sie sie forthin nicht mehr fressen sollen.
\par 11 Denn so spricht der HERR HERR: Siehe, ich will mich meiner Herde selbst annehmen und sie suchen.
\par 12 Wie ein Hirte seine Schafe sucht, wenn sie von seiner Herde verirrt sind, also will ich meine Schafe suchen und will sie erretten von allen Örtern, dahin sie zerstreut waren zur Zeit, da es trüb und finster war.
\par 13 Ich will sie von allen Völkern ausführen und aus allen Ländern versammeln und will sie in ihr Land führen und will sie weiden auf den Berge Israels und in allen Auen und auf allen Angern des Landes.
\par 14 Ich will sie auf die beste Weide führen, und ihre Hürden werden auf den hohen Bergen in Israel stehen; daselbst werden sie in sanften Hürden liegen und fette Weide haben auf den Bergen Israels.
\par 15 Ich will selbst meine Schafe weiden, und ich will sie lagern, spricht der HERR HERR.
\par 16 Ich will das Verlorene wieder suchen und das Verirrte wiederbringen und das Verwundete verbinden und des Schwachen warten; aber was fett und stark ist, will ich vertilgen und will es weiden mit Gericht.
\par 17 Aber zu euch, meine Herde, spricht der HERR HERR also: Siehe, ich will richten zwischen Schaf und Schaf und zwischen Widdern und Böcken.
\par 18 Ist's euch nicht genug, so gute Weide zu haben, daß ihr das übrige mit Füßen tretet, und so schöne Borne zu trinken, daß ihr auch noch dareintretet und sie trüb macht,
\par 19 daß meine Schafe essen müssen, was ihr mit euren Füßen zertreten habt, und trinken, was ihr mit euren Füßen trüb gemacht habt?
\par 20 Darum so spricht der HERR HERR zu ihnen: Siehe, ich will richten zwischen den fetten und mageren Schafen,
\par 21 darum daß ihr mit der Seite und Schulter drängt und die Schwachen von euch stoßt mit euren Hörnern, bis ihr sie alle von euch zerstreut.
\par 22 Und ich will meiner Herde helfen, daß sie nicht mehr sollen zum Raub werden, und will richten zwischen Schaf und Schaf.
\par 23 Und ich will ihnen einen einigen Hirten erwecken, der sie weiden soll, nämlich meinen Knecht David. Der wird sie weiden und soll ihr Hirte sein,
\par 24 und ich, der HERR, will ihr Gott sein; aber mein Knecht David soll der Fürst unter ihnen sein, das sage ich, der HERR.
\par 25 Und ich will einen Bund des Friedens mit ihnen machen und alle bösen Tiere aus dem Land ausrotten, daß sie in der Wüste sicher wohnen und in den Wäldern schlafen sollen.
\par 26 Ich will sie und alles, was um meinen Hügel her ist, segnen und auf sie regnen lassen zu rechter Zeit; das sollen gnädige Regen sein,
\par 27 daß die Bäume auf dem Felde ihre Früchte bringen und das Land sein Gewächs geben wird; und sie sollen sicher auf dem Lande wohnen und sollen erfahren, daß ich der HERR bin, wenn ich ihr Joch zerbrochen und sie errettet habe von der Hand derer, denen sie dienen mußten.
\par 28 Und sie sollen nicht mehr den Heiden zum Raub werden, und kein Tier auf Erden soll sie mehr fressen, sondern sollen sicher wohnen ohne alle Furcht.
\par 29 Und ich will ihnen eine herrliche Pflanzung aufgehen lassen, daß sie nicht mehr Hunger leiden im Lande und ihre Schmach unter den Heiden nicht mehr tragen sollen.
\par 30 Und sie sollen erfahren, daß ich, der HERR, ihr Gott, bei ihnen bin und daß sie vom Haus Israel mein Volk seien, spricht der HERR HERR.
\par 31 Ja, ihr Menschen sollt die Herde meiner Weide sein, und ich will euer Gott sein, spricht der HERR HERR.

\chapter{35}

\par 1 Und des HERRN Wort geschah zu mir und sprach:
\par 2 Du Menschenkind, richte dein Angesicht wider das Gebirge Seir und weissage dawider,
\par 3 und sprich zu ihm: So spricht der HERR HERR: Siehe, ich will an dich, du Berg Seir, und meine Hand wider dich ausstrecken und will dich gar wüst machen.
\par 4 Ich will deine Städte öde machen, daß du sollst zur Wüste werden und erfahren, daß ich der HERR bin.
\par 5 Darum daß ihr ewige Feindschaft tragt wider die Kinder Israel und triebet sie ins Schwert zur Zeit, da es ihnen übel ging und ihre Missetat zum Ende gekommen war,
\par 6 darum, so wahr ich lebe, spricht der HERR HERR, will ich dich auch blutend machen, und du sollst dem Bluten nicht entrinnen; weil du Lust zum Blut hast, sollst du dem Bluten nicht entrinnen.
\par 7 Und ich will den Berg Seir wüst und öde machen, daß niemand darauf wandeln noch gehen soll.
\par 8 Und will sein Gebirge und alle Hügel, Täler und alle Gründe voll Toter machen, die durchs Schwert sollen erschlagen daliegen.
\par 9 Ja, zu einer ewigen Wüste will ich dich machen, daß niemand in deinen Städten wohnen soll; und ihr sollt erfahren, daß ich der HERR bin.
\par 10 Und darum daß du sprichst: Diese beiden Völker mit beiden Ländern müssen mein werden, und wir wollen sie einnehmen, obgleich der HERR da wohnt,
\par 11 darum, so wahr ich lebe, spricht der HERR HERR, will ich nach deinem Zorn und Haß mit dir umgehen, wie du mit ihnen umgegangen bist aus lauter Haß, und ich will bei ihnen bekannt werden, wenn ich dich gestraft habe.
\par 12 Und du sollst erfahren, daß ich, der HERR, all dein Lästern gehört habe, so du geredet hast wider die Berge Israels und gesagt: "Sie sind verwüstet und uns zu verderben gegeben."
\par 13 Und ihr habt euch wider mich gerühmt und heftig wider mich geredet; das habe ich gehört.
\par 14 So spricht nun der HERR HERR: Ich will dich zur Wüste machen, daß sich alles Land freuen soll.
\par 15 Und wie du dich gefreut hast über das Erbe des Hauses Israel, darum daß es wüst geworden, ebenso will ich mit dir tun, daß der Berg Seir wüst sein muß samt dem ganzen Edom; und sie sollen erfahren, daß ich der HERR bin.

\chapter{36}

\par 1 Und du, Menschenkind, weissage den Bergen Israels und sprich: Höret des HERRN Wort ihr Berge Israels!
\par 2 So spricht der HERR HERR: Darum daß der Feind über euch rühmt: Ha! die ewigen Höhen sind nun unser Erbe geworden!
\par 3 darum weissage und sprich: So spricht der HERR HERR: Weil man euch allenthalben verwüstet und vertilgt, und ihr seid den übrigen Heiden zuteil geworden und seid den Leuten ins Maul gekommen und ein böses Geschrei geworden,
\par 4 darum hört, ihr Berge Israels, das Wort des HERRN HERRN! So spricht der HERR HERR zu den Bergen und Hügeln, zu den Bächen und Tälern, zu den öden Wüsten und verlassenen Städten, welche den übrigen Heiden ringsumher zum Raub und Spott geworden sind:
\par 5 ja, so spricht der HERR HERR: Ich habe in meinem feurigen Eifer geredet wider die Heiden und wider das ganze Edom, welche mein Land eingenommen haben mit Freuden von ganzem Herzen und mit Hohnlachen, es zu verheeren und zu plündern.
\par 6 Darum weissage von dem Lande Israel und sprich zu den Bergen und Hügeln, zu den Bächen und Tälern: So spricht der HERR HERR: Siehe, ich habe in meinem Eifer und Grimm geredet, weil ihr solche Schmach von den Heiden tragen müsset.
\par 7 Darum spricht der HERR HERR also: Ich hebe meine Hand auf, daß eure Nachbarn, die Heiden umher, ihre Schande tragen sollen.
\par 8 Aber ihr Berge Israels sollt wieder grünen und eure Frucht bringen meinem Volk Israel; und es soll in kurzem geschehen.
\par 9 Denn siehe, ich will mich wieder zu euch wenden und euch ansehen, daß ihr gebaut und besät werdet;
\par 10 und will bei euch der Leute viel machen, das ganze Israel allzumal; und die Städte sollen wieder bewohnt und die Wüsten erbaut werden.
\par 11 Ja, ich will bei euch der Leute und des Viehs viel machen, daß sie sich mehren und wachsen sollen. Und ich will euch wieder bewohnt machen wie vorher und will euch mehr Gutes tun denn zuvor je; und ihr sollt erfahren, daß ich der HERR sei.
\par 12 Ich will euch Leute herzubringen, mein Volk Israel, die werden dich besitzen; und sollst ihr Erbteil sein und sollst sie nicht mehr ohne Erben machen.
\par 13 So spricht der HERR HERR: Weil man das von euch sagt: Du hast Leute gefressen und hast dein Volk ohne Erben gemacht,
\par 14 darum sollst du nun nicht mehr Leute fressen noch dein Volk ohne Erben machen, spricht der HERR HERR.
\par 15 Und ich will dich nicht mehr lassen hören die Schmähung der Heiden, und sollst den Spott der Heiden nicht mehr tragen und sollst dein Volk nicht mehr ohne Erben machen, spricht der HERR HERR.
\par 16 Und des HERRN Wort geschah weiter zu mir:
\par 17 Du Menschenkind, da das Haus Israel in seinem Lande wohnte und es verunreinigte mit seinem Wesen und Tun, daß ihr Wesen vor mir war wie die Unreinigkeit eines Weibes in ihrer Krankheit,
\par 18 da schüttete ich meinen Grimm über sie aus um des Blutes willen, das sie im Lande vergossen, und weil sie es verunreinigt hatten durch ihre Götzen.
\par 19 Und ich zerstreute sie unter die Heiden und zerstäubte sie in die Länder und richtete sie nach ihrem Wesen und Tun.
\par 20 Und sie hielten sich wie die Heiden, zu denen sie kamen, und entheiligten meinen heiligen Namen, daß man von ihnen sagte: Ist das des HERRN Volk, das aus seinem Lande hat müssen ziehen?
\par 21 Aber ich schonte meines heiligen Namens, welchen das Haus Israel entheiligte unter den Heiden, dahin sie kamen.
\par 22 Darum sollst du zum Hause Israel sagen: So spricht der HERR HERR: Ich tue es nicht um euretwillen, ihr vom Hause Israel, sondern um meines heiligen Namens willen, welchen ihr entheiligt habt unter den Heiden, zu welchen ihr gekommen seid.
\par 23 Denn ich will meinen großen Namen, der vor den Heiden entheiligt ist, den ihr unter ihnen entheiligt habt, heilig machen. Und die Heiden sollen erfahren, daß ich der HERR sei, spricht der HERR HERR, wenn ich mich vor ihnen an euch erzeige, daß ich heilig sei.
\par 24 Denn ich will euch aus den Heiden holen und euch aus allen Landen versammeln und wieder in euer Land führen.
\par 25 Und will reines Wasser über euch sprengen, daß ihr rein werdet; von all eurer Unreinigkeit und von allen euren Götzen will ich euch reinigen.
\par 26 Und ich will euch ein neues Herz und einen neuen Geist in euch geben und will das steinerne Herz aus eurem Fleische wegnehmen und euch ein fleischernes Herz geben;
\par 27 ich will meinen Geist in euch geben und will solche Leute aus euch machen, die in meinen Geboten wandeln und meine Rechte halten und darnach tun.
\par 28 Und ihr sollt wohnen im Lande, das ich euren Vätern gegeben habe, und sollt mein Volk sein, und ich will euer Gott sein.
\par 29 Ich will euch von aller Unreinigkeit losmachen und will dem Korn rufen und will es mehren und will euch keine Teuerung kommen lassen.
\par 30 Ich will die Früchte auf den Bäumen und das Gewächs auf dem Felde mehren, daß euch die Heiden nicht mehr verspotten mit der Teuerung.
\par 31 Alsdann werdet ihr an euer böses Wesen gedenken und an euer Tun, das nicht gut war, und wird euch eure Sünde und Abgötterei gereuen.
\par 32 Solches will ich tun, nicht um euretwillen, spricht der HERR HERR, daß ihr's wißt; sondern ihr werdet schamrot werden, ihr vom Hause Israel, über eurem Wesen.
\par 33 So spricht der HERR HERR: Zu der Zeit, wann ich euch reinigen werde von allen euren Sünden, so will ich die Städte wieder besetzen, und die Wüsten sollen wieder gebaut werden.
\par 34 Das verwüstete Land soll wieder gepflügt werden, dafür es verheert war; daß es sehen sollen alle, die dadurchgehen,
\par 35 und sagen: Dies Land war verheert, und jetzt ist's wie der Garten Eden; und diese Städte waren zerstört, öde und zerrissen, und stehen nun fest gebaut.
\par 36 Und die Heiden, so um euch her übrigbleiben werden, sollen erfahren, daß ich der HERR bin, der da baut, was zerrissen ist, und pflanzt, was verheert war. Ich, der HERR, sage es und tue es auch.
\par 37 So spricht der HERR HERR: Auch darin will ich mich vom Hause Israel finden lassen, daß ich es ihnen erzeige: ich will die Menschen bei ihnen mehren wie eine Herde.
\par 38 Wie eine heilige Herde, wie eine Herde zu Jerusalem auf ihren Festen, so sollen die verheerten Städte voll Menschenherden werden und sollen erfahren, daß ich der HERR bin.

\chapter{37}

\par 1 Und des HERRN Wort kam über mich, und er führte mich hinaus im Geist des HERRN und stellte mich auf ein weites Feld, das voller Totengebeine lag.
\par 2 Und er führte mich allenthalben dadurch. Und siehe, des Gebeins lag sehr viel auf dem Feld; und siehe, sie waren sehr verdorrt.
\par 3 Und er sprach zu mir: Du Menschenkind, meinst du auch, daß diese Gebeine wieder lebendig werden? Und ich sprach: HERR HERR, das weißt du wohl.
\par 4 Und er sprach zu mir: Weissage von diesen Gebeinen und sprich zu ihnen: Ihr verdorrten Gebeine, höret des HERRN Wort!
\par 5 So spricht der HERR HERR von diesen Gebeinen: Siehe, ich will einen Odem in euch bringen, daß ihr sollt lebendig werden.
\par 6 Ich will euch Adern geben und Fleisch lassen über euch wachsen und euch mit Haut überziehen und will euch Odem geben, daß ihr wieder lebendig werdet, und ihr sollt erfahren, daß ich der HERR bin.
\par 7 Und ich weissagte, wie mir befohlen war; und siehe, da rauschte es, als ich weissagte, und siehe, es regte sich, und die Gebeine kamen wieder zusammen, ein jegliches zu seinem Gebein.
\par 8 Und ich sah, und siehe, es wuchsen Adern und Fleisch darauf, und sie wurden mit Haut überzogen; es war aber noch kein Odem in ihnen.
\par 9 Und er sprach zu mir: Weissage zum Winde; weissage, du Menschenkind, und sprich zum Wind: So spricht der HERR HERR: Wind komm herzu aus den vier Winden und blase diese Getöteten an, daß sie wieder lebendig werden!
\par 10 Und ich weissagte, wie er mir befohlen hatte. Da kam Odem in sie, und sie wurden wieder lebendig und richteten sich auf ihre Füße. Und ihrer war ein großes Heer.
\par 11 Und er sprach zu mir: Du Menschenkind, diese Gebeine sind das ganze Haus Israel. Siehe, jetzt sprechen sie: Unsere Gebeine sind verdorrt, und unsere Hoffnung ist verloren, und es ist aus mit uns.
\par 12 Darum weissage und sprich zu ihnen: So spricht der HERR HERR: Siehe, ich will eure Gräber auftun und will euch, mein Volk, aus denselben herausholen und euch ins Land Israel bringen;
\par 13 und ihr sollt erfahren, daß ich der HERR bin, wenn ich eure Gräber geöffnet und euch, mein Volk, aus denselben gebracht habe.
\par 14 Und ich will meinen Geist in euch geben, daß ihr wieder leben sollt, und will euch in euer Land setzen, und sollt erfahren, daß ich der HERR bin. Ich rede es und tue es auch, spricht der HERR.
\par 15 Und des HERRN Wort geschah zu mir und sprach:
\par 16 Du Menschenkind, nimm dir ein Holz und schreibe darauf: Des Juda und der Kinder Israel, seiner Zugetanen. Und nimm noch ein Holz und schreibe darauf: Des Joseph, nämlich das Holz Ephraims, und des ganzen Hauses Israel, seiner Zugetanen.
\par 17 Und tue eins zum andern zusammen, daß es ein Holz werde in deiner Hand.
\par 18 So nun dein Volk zu dir wird sagen und sprechen: Willst du uns nicht zeigen, was du damit meinst?
\par 19 So sprich zu ihnen: So spricht der HERR HERR: Siehe, ich will das Holz Josephs, welches ist in Ephraims Hand, nehmen mit samt seinen Zugetanen, den Stämmen Israels, und will sie zu dem Holz Juda's tun und ein Holz daraus machen, und sollen eins in meiner Hand sein.
\par 20 Und sollst also die Hölzer, darauf du geschrieben hast, in deiner Hand halten, daß sie zusehen,
\par 21 und sollst zu ihnen sagen: So spricht der HERR HERR: Siehe, ich will die Kinder Israel holen aus den Heiden, dahin sie gezogen sind, und will sie allenthalben sammeln und will sie wieder in ihr Land bringen
\par 22 und will ein Volk aus ihnen machen im Lande auf den Bergen Israels, und sie sollen allesamt einen König haben und sollen nicht mehr zwei Völker noch in zwei Königreiche zerteilt sein;
\par 23 sollen sich auch nicht mehr verunreinigen mit ihren Götzen und Greueln und allerlei Sünden. Ich will ihnen heraushelfen aus allen Örtern, da sie gesündigt haben, und will sie reinigen; und sie sollen mein Volk sein, und ich will ihr Gott sein.
\par 24 Und mein Knecht David soll ihr König und ihrer aller einiger Hirte sein. Und sie sollen wandeln in meinen Rechten und meine Gebote halten und darnach tun.
\par 25 Und sie sollen wieder in dem Lande wohnen, das ich meinem Knecht Jakob gegeben habe, darin ihre Väter gewohnt haben. Sie sollen darin wohnen ewiglich, und mein Knecht David soll ewiglich ihr Fürst sein.
\par 26 Und ich will mit ihnen einen Bund des Friedens machen, das soll ein ewiger Bund sein mit ihnen; und will sie erhalten und mehren, und mein Heiligtum soll unter ihnen sein ewiglich.
\par 27 Und ich will unter ihnen wohnen und will ihr Gott sein, und sie sollen mein Volk sein,
\par 28 daß auch die Heiden sollen erfahren, daß ich der HERR bin, der Israel heilig macht, wenn mein Heiligtum ewiglich unter ihnen sein wird.

\chapter{38}

\par 1 Und des HERRN Wort geschah zu mir und sprach:
\par 2 Du Menschenkind, wende dich gegen Gog, der im Lande Magog ist und der oberste Fürst in Mesech und Thubal, und weissage von ihm
\par 3 und sprich: So spricht der HERR HERR: Siehe, ich will an dich Gog! der du der oberste Fürst bist in Mesech und Thubal.
\par 4 Siehe, ich will dich herumlenken und will dir einen Zaum ins Maul legen und will dich herausführen mit allem deinem Heer, Roß und Mann, die alle wohl gekleidet sind; und ihrer ist ein großer Haufe, die alle Tartsche und Schild und Schwert führen.
\par 5 Du führst mit dir Perser, Mohren und Libyer, die alle Schild und Helm führen,
\par 6 dazu Gomer und all sein Heer samt dem Hause Thogarma, so gegen Mitternacht liegt, mit all seinem Heer; ja, du führst ein großes Volk mit dir.
\par 7 Wohlan, rüste dich wohl, du und alle deine Haufen, so bei dir sind, und sei du ihr Hauptmann!
\par 8 Nach langer Zeit sollst du heimgesucht werden. Zur letzten Zeit wirst du kommen in das Land, das vom Schwert wiedergebracht und aus vielen Völkern zusammengekommen ist, nämlich auf die Berge Israels, welche lange Zeit wüst gewesen sind; und nun ist es ausgeführt aus den Völkern, und wohnen alle sicher.
\par 9 Du wirst heraufziehen und daherkommen mit großem Ungestüm; und wirst sein wie eine Wolke, das Land zu bedecken, du und all dein Heer und das große Volk mit dir.
\par 10 So spricht der HERR HERR: Zu der Zeit wirst du solches vornehmen und wirst Böses im Sinn haben
\par 11 und gedenken: "Ich will das Land ohne Mauern überfallen und über sie kommen, so still und sicher wohnen, als die alle ohne Mauern dasitzen und haben weder Riegel noch Tore",
\par 12 auf daß du rauben und plündern mögest und dein Hand lassen gehen über die verstörten Örter, so wieder bewohnt sind, und über das Volk, so aus den Heiden zusammengerafft ist und sich in die Nahrung und Güter geschickt hat und mitten auf der Erde wohnt.
\par 13 Das reiche Arabien, Dedan und die Kaufleute von Tharsis und alle Gewaltigen, die daselbst sind, werden dir sagen: Ich meine ja, du seist recht gekommen, zu rauben, und hast deine Haufen versammelt, zu plündern, auf daß du wegnimmst Silber und Gold und sammelst Vieh und Güter, und großen Raub treibest.
\par 14 Darum so weissage, du Menschenkind, und sprich zu Gog: So spricht der HERR HERR: Ist's nicht also, daß du wirst merken, wenn mein Volk Israel sicher wohnen wird?
\par 15 So wirst du kommen aus deinem Ort, von den Enden gegen Mitternacht, du und großes Volk mit dir, alle zu Rosse, ein großer Haufe und ein mächtiges Heer,
\par 16 und wirst heraufziehen über mein Volk Israel wie eine Wolke, das Land zu bedecken. Solches wird zur letzten Zeit geschehen. Ich will dich aber darum in mein Land kommen lassen, auf daß die Heiden mich erkennen, wie ich an dir, o Gog, geheiligt werde vor ihren Augen.
\par 17 So spricht der HERR HERR: Du bist's, von dem ich vorzeiten gesagt habe durch meine Diener, die Propheten in Israel, die zur selben Zeit weissagten, daß ich dich über sie kommen lassen wollte.
\par 18 Und es wird geschehen zu der Zeit, wann Gog kommen wird über das Land Israel, spricht der HERR HERR, wird heraufziehen mein Zorn in meinem Grimm.
\par 19 Und ich rede solches in meinem Eifer und im Feuer meines Zorns. Denn zur selben Zeit wird großes Zittern sein im Lande Israel,
\par 20 daß vor meinem Angesicht zittern sollen die Fische im Meer, die Vögel unter dem Himmel, die Tiere auf dem Felde und alles, was sich regt und bewegt auf dem Lande, und alle Menschen, so auf der Erde sind; und sollen die Berge umgekehrt werden und die Felswände und alle Mauern zu Boden fallen.
\par 21 Ich will aber wider ihn herbeirufen das Schwert auf allen meinen Bergen, spricht der HERR HERR, daß eines jeglichen Schwert soll wider den andern sein.
\par 22 Und ich will ihn richten mit Pestilenz und Blut und will regnen lassen Platzregen mit Schloßen, Feuer und Schwefel über ihn und sein Heer und über das große Volk, das mit ihm ist.
\par 23 Also will ich denn herrlich, heilig und bekannt werden vor vielen Heiden, daß sie erfahren sollen, daß ich der HERR bin.

\chapter{39}

\par 1 Und du, Menschenkind, weissage wider Gog und sprich: Also spricht der HERR HERR: Siehe, ich will an dich, Gog, der du der oberste Fürst bist in Mesech und Thubal.
\par 2 Siehe, ich will dich herumlenken und locken und aus den Enden von Mitternacht bringen und auf die Berge Israels kommen lassen.
\par 3 Und ich will dir den Bogen aus deiner linken Hand schlagen und deine Pfeile aus deiner rechten Hand werfen.
\par 4 Auf den Bergen Israels sollst du niedergelegt werden, du mit allem deinem Heer und mit dem Volk, das bei dir ist. Ich will dich den Vögeln, woher sie fliegen, und den Tieren auf dem Felde zu fressen geben.
\par 5 Du sollst auf dem Felde darniederliegen; denn ich, der HERR HERR, habe es gesagt.
\par 6 Und ich will Feuer werfen über Magog und über die, so in den Inseln sicher wohnen; und sollen's erfahren, daß ich der HERR bin.
\par 7 Denn ich will meinen heiligen Namen kundmachen unter meinem Volk Israel und will meinen heiligen Namen nicht länger schänden lassen; sondern die Heiden sollen erfahren, daß ich der HERR bin, der Heilige in Israel.
\par 8 Siehe, es ist gekommen und ist geschehen, spricht der HERR HERR; das ist der Tag, davon ich geredet habe.
\par 9 Und die Bürger in den Städten Israels werden herausgehen und Feuer machen und verbrennen die Waffen, Schilde, Tartschen, Bogen, Pfeile, Keulen und langen Spieße; und sie werden sieben Jahre lang Feuer damit machen,
\par 10 daß sie nicht müssen Holz auf dem Felde holen noch im Walde hauen, sondern von den Waffen werden sie Feuer machen; und sollen die berauben, von denen sie beraubt sind, und plündern, von denen sie geplündert sind, spricht der HERR HERR.
\par 11 Und soll zu der Zeit geschehen, da will ich Gog einen Ort geben zum Begräbnis in Israel, nämlich das Tal, da man geht am Meer gegen Morgen, also daß die, so vorübergehen, sich davor scheuen werden, weil man daselbst Gog mit seiner Menge begraben hat; und soll heißen "Gogs Haufental".
\par 12 Es wird sie aber das Haus Israel begraben sieben Monden lang, damit das Land gereinigt werde.
\par 13 Ja, alles Volk im Lande wird an ihnen zu begraben haben, und sie werden Ruhm davon haben des Tages, da ich meine Herrlichkeit erzeige, spricht der HERR HERR.
\par 14 Und sie werden Leute aussondern, die stets im Lande umhergehen und mit ihnen die Totengräber, zu begraben die übrigen auf dem Lande, damit es gereinigt werde; nach sieben Monden werden sie forschen.
\par 15 Und die, so im Lande umhergehen und eines Menschen Gebein sehen, werden dabei ein Mal aufrichten, bis es die Totengräber auch in Gogs Haufental begraben.
\par 16 So soll auch die Stadt heißen Hamona. Also werden sie das Land reinigen.
\par 17 Nun, du Menschenkind, so spricht der HERR HERR: Sage allen Vögeln, woher sie fliegen, und allen Tieren auf dem Felde: Sammelt euch und kommt her, findet euch allenthalben zuhauf zu meinem Schlachtopfer, das ich euch schlachte, ein großes Schlachtopfer auf den Bergen Israels, fresset Fleisch und saufet Blut!
\par 18 Fleisch der Starken sollt ihr fressen, und Blut der Fürsten auf Erden sollt ihr saufen, der Widder, der Hammel, der Böcke, der Ochsen, die allzumal feist und gemästet sind.
\par 19 Und sollt das Fett fressen, daß ihr voll werdet, und das Blut saufen, daß ihr trunken werdet, von dem Schlachtopfer, das ich euch schlachte.
\par 20 Sättigt euch nun an meinem Tisch von Rossen und Reitern, von Starken und allerlei Kriegsleuten, spricht der HERR HERR.
\par 21 Und ich will meine Herrlichkeit unter die Heiden bringen, daß alle Heiden sehen sollen mein Urteil, das ich habe ergehen lassen, und meine Hand, die ich an sie gelegt habe,
\par 22 und also das ganze Haus Israel erfahre, daß ich, der HERR, ihr Gott bin von dem Tage an und hinfürder,
\par 23 und die Heiden erfahren, wie das Haus Israel um seiner Missetat willen sei weggeführt. Weil sie sich an mir versündigt hatten, darum habe ich mein Angesicht vor ihnen verborgen und habe sie übergeben in die Hände ihrer Widersacher, daß sie allzumal durchs Schwert fallen mußten.
\par 24 Ich habe ihnen getan, wie ihre Sünde und Übertretung verdient haben, und also mein Angesicht vor ihnen verborgen.
\par 25 Darum so spricht der HERR HERR: Nun will ich das Gefängnis Jakobs wenden und mich des ganzen Hauses Israel erbarmen und um meinen heiligen Namen eifern.
\par 26 Sie aber werden ihre Schmach und alle ihre Sünde, damit sie sich an mir versündigt haben, tragen, wenn sie nun sicher in ihrem Lande wohnen, daß sie niemand schrecke,
\par 27 und ich sie wieder aus den Völkern gebracht und aus den Landen ihrer Feinde versammelt habe und ich an ihnen geheiligt worden bin vor den Augen vieler Heiden.
\par 28 Also werden sie erfahren, daß ich, der HERR, ihr Gott bin, der ich sie habe lassen unter die Heiden wegführen und wiederum in ihr Land versammeln und nicht einen von ihnen dort gelassen habe.
\par 29 Und ich will mein Angesicht nicht mehr vor ihnen verbergen; denn ich habe meinen Geist über das Haus Israel ausgegossen, spricht der HERR HERR.

\chapter{40}

\par 1 Im fünfundzwanzigsten Jahr unserer Gefangenschaft, im Anfang des Jahres, am zehnten Tage des Monats, im vierzehnten Jahr, nachdem die Stadt geschlagen war, eben an diesem Tage kam des HERRN Hand über mich und führte mich dahin.
\par 2 Durch göttliche Gesichte führte er mich ins Land Israel und stellte mich auf einen hohen Berg, darauf war's wie eine gebaute Stadt gegen Mittag.
\par 3 Und da er mich hingebracht hatte, siehe, da war ein Mann, des Ansehen war wie Erz; der hatte eine lange leinene Schnur und eine Meßrute in seiner Hand und stand unter dem Tor.
\par 4 Und er sprach zu mir: Du Menschenkind, siehe und höre fleißig zu und merke auf alles, was ich dir zeigen will. Denn darum bist du hergebracht, daß ich dir solches zeige, auf daß du solches alles, was du hier siehst, verkündigst dem Hause Israel.
\par 5 Und siehe, es ging eine Mauer auswendig um das Haus ringsumher. Und der Mann hatte die Meßrute in der Hand, die war sechs Ellen lang; eine jegliche Elle war eine Handbreit länger denn eine gemeine Elle. Und er maß das Gebäude in die Breite eine Rute und in die Höhe auch eine Rute.
\par 6 Und er ging ein zum Tor, das gegen Morgen lag, und ging hinauf auf seinen Stufen und maß die Schwelle, eine Rute breit.
\par 7 Und die Gemächer, so beiderseits neben dem Tor waren, maß er auch nach der Länge eine Rute und nach der Breite eine Rute; und der Raum zwischen den Gemächern war fünf Ellen weit. Und er maß auch die Schwelle am Tor neben der Halle, die nach dem Hause zu war, eine Rute.
\par 8 Und er maß die Halle am Tor, die nach dem Hause zu war, eine Rute.
\par 9 Und maß die Halle am Tor acht Ellen und ihre Pfeiler zwei Ellen, und die Halle am Tor war nach dem Hause zu.
\par 10 Und der Gemächer waren auf jeglicher Seite drei am Tor gegen Morgen, je eins so weit wie das andere, und die Pfeiler auf beiden Seiten waren gleich groß.
\par 11 Darnach maß er die Weite der Tür im Tor zehn Ellen und die Länge des Tors dreizehn Ellen.
\par 12 Und vorn an den Gemächern war Raum abgegrenzt auf beiden Seiten, je eine Elle; aber die Gemächer waren je sechs Ellen auf beiden Seiten.
\par 13 Dazu maß er das Tor vom Dach der Gemächer auf der einen Seite bis zum Dach der Gemächer auf der andern Seite fünfundzwanzig Ellen breit; und eine Tür stand gegenüber der andern.
\par 14 Und er machte die Pfeiler sechzig Ellen, und an den Pfeilern war der Vorhof, am Tor ringsherum.
\par 15 Und vom Tor, da man hineingeht, bis außen an die Halle an der innern Seite des Tors waren fünfzig Ellen.
\par 16 Und es waren enge Fensterlein an den Gemächern und an den Pfeilern hineinwärts am Tor ringsumher. Also waren auch Fenster inwendig an der Halle herum, und an den Pfeilern war Palmlaubwerk.
\par 17 Und er führte mich weiter zum äußern Vorhof, und siehe, da waren Kammern und ein Pflaster gemacht am Vorhofe herum; dreißig Kammern waren auf dem Pflaster.
\par 18 Und es war das Pflaster zur Seite der Tore, solang die Tore waren, nämlich das untere Pflaster.
\par 19 Und er maß die Breite von dem untern Tor an bis vor den innern Hof auswendig hundert Ellen, gegen Morgen und gegen Mitternacht.
\par 20 Er maß auch das Tor, so gegen Mitternacht lag, am äußern Vorhof, nach der Länge und Breite.
\par 21 Das hatte auch auf jeder Seite drei Gemächer und hatte auch seine Pfeiler und Halle, gleich so groß wie am vorigen Tor, fünfzig Ellen die Länge und fünfundzwanzig Ellen die Breite.
\par 22 Und hatte auch seine Fenster und seine Halle und sein Palmlaubwerk, gleich wie das Tor gegen Morgen; und hatte seine Halle davor.
\par 23 Und es waren Tore am innern Vorhof gegenüber den Toren, so gegen Mitternacht und Morgen standen; und er maß hundert Ellen von einem Tor zum andern.
\par 24 Darnach führte er mich gegen Mittag, und siehe, da war auch ein Tor gegen Mittag; und er maß seine Pfeiler und Halle gleich wie die andern.
\par 25 Und es waren auch Fenster an ihm und an seiner Halle umher, gleich wie jene Fenster; und es war fünfzig Ellen lang und fünfundzwanzig Ellen breit.
\par 26 Und waren auch sieben Stufen hinauf und eine Halle davor und Palmlaubwerk an ihren Pfeilern auf jeglicher Seite.
\par 27 Und es war auch ein Tor am innern Vorhof gegen Mittag, und er maß hundert Ellen von dem einen Mittagstor zum andern.
\par 28 Und er führte mich weiter durchs Mittagstor in den innern Vorhof und maß dasselbe Tor gleich so groß wie die andern,
\par 29 mit seinen Gemächern, Pfeilern und Halle und mit Fenstern an ihm und an seiner Halle, ebenso groß wie jene, ringsumher; und es waren fünfzig Ellen lang und fünfundzwanzig Ellen breit.
\par 30 Und es ging eine Halle herum, fünfundzwanzig Ellen lang und fünf Ellen breit.
\par 31 Und die Halle, so gegen den äußeren Vorhof stand, hatte auch Palmlaubwerk an den Pfeilern; es waren aber acht Stufen hinaufzugehen.
\par 32 Darnach führte er mich zum innern Vorhof gegen Morgen und maß das Tor gleich so groß wie die andern,
\par 33 mit seinen Gemächern, Pfeilern und Halle, gleich so groß wie die andern, und mit Fenstern an ihm und an seiner Halle ringsumher; und es war fünfzig Ellen lang und fünfundzwanzig Ellen breit.
\par 34 Und seine Halle stand auch gegen den äußern Vorhof und Palmlaubwerk an ihren Pfeilern zu beiden Seiten und acht Stufen hinauf.
\par 35 Darnach führte er mich zum Tor gegen Mitternacht; das maß er gleich so groß wie die andern,
\par 36 mit seinen Gemächern, Pfeilern und Halle und ihren Fenstern ringsumher, fünfzig Ellen lang und fünfundzwanzig Ellen breit.
\par 37 Und seine Halle stand auch gegen den äußern Vorhof und Palmlaubwerk an den Pfeilern zu beiden Seiten und acht Stufen hinauf.
\par 38 Und unten an den Pfeilern an jedem Tor war eine Kammer mit einer Tür, darin man das Brandopfer wusch.
\par 39 Aber in der Halle des Tors standen auf jeglicher Seite zwei Tische, darauf man die Brandopfer, Sündopfer und Schuldopfer schlachten sollte.
\par 40 Und herauswärts zur Seite, da man hinaufgeht zum Tor gegen Mitternacht, standen auch zwei Tische und an der andern Seite unter der Halle des Tors auch zwei Tische.
\par 41 Also standen auf jeder Seite des Tors vier Tische; das sind zusammen acht Tische, darauf man schlachtete.
\par 42 Und noch vier Tische, zum Brandopfer gemacht, die waren aus gehauenen Steinen, je anderthalb Ellen lang und breit und eine Elle hoch, darauf man legte allerlei Geräte, womit man Brandopfer und andere Opfer schlachtete.
\par 43 Und es gingen Leisten herum, hineinwärts gebogen, eine quere Hand hoch. Und auf die Tische sollte man das Opferfleisch legen.
\par 44 Und außen vor dem innern Tor waren zwei Kammern im innern Vorhofe: eine an der Seite neben dem Tor zur Mitternacht, die sah gegen Mittag; die andere zur Seite des Tors gegen Mittag, die sah gegen Mitternacht.
\par 45 Und er sprach zu mir: Die Kammer gegen Mittag gehört den Priestern, die im Hause dienen sollen;
\par 46 aber die Kammer gegen Mitternacht gehört den Priestern, die auf dem Altar dienen. Dies sind die Kinder Zadok, welche allein unter den Kindern Levi vor den HERRN treten sollen, ihm zu dienen.
\par 47 Und er maß den Vorhof, nämlich hundert Ellen lang und hundert Ellen breit ins Gevierte; und der Altar stand vorn vor dem Tempel.
\par 48 Und er führte mich hinein zur Halle des Tempels und maß die Pfeiler der Halle fünf Ellen auf jeder Seite und das Tor vierzehn Ellen, und die Wände zu beiden Seiten an der Tür drei Ellen auf jeder Seite.
\par 49 Aber die Halle war zwanzig Ellen lang und elf Ellen weit und hatte Stufen, da man hinaufging; und Säulen standen an den Pfeilern, auf jeder Seite eine.

\chapter{41}

\par 1 Und er führte mich hinein in den Tempel und maß die Pfeiler an den Wänden; die waren zu jeder Seite sechs Ellen breit, soweit das Haus war.
\par 2 Und die Tür war zehn Ellen weit; aber die Wände zu beiden Seiten an der Tür waren jede fünf Ellen breit. Und er maß den Raum im Tempel; der hatte vierzig Ellen in die Länge und zwanzig Ellen in die Breite.
\par 3 Und er ging inwendig hinein und maß die Pfeiler der Tür zwei Ellen; und die Tür hatte sechs Ellen, und die Breite zu beiden Seiten an der Tür je sieben Ellen.
\par 4 Und er maß zwanzig Ellen in die Breite am Tempel. Und er sprach zu mir: Dies ist das Allerheiligste.
\par 5 Und er maß die Wand des Hauses sechs Ellen dick. Daran waren Gänge allenthalben herum, geteilt in Gemächer, die waren allenthalben vier Ellen weit.
\par 6 Und derselben Gemächer waren je dreißig, dreimal übereinander, und reichten bis auf die Wand des Hauses, an der die Gänge waren allenthalben herum, und wurden also festgehalten, daß sie in des Hauses Wand nicht eingriffen.
\par 7 Und die Gänge rings um das Haus her mit ihren Gemächern waren umso weiter, je höher sie lagen; und aus dem untern ging man in den mittleren und aus dem mittleren in den obersten.
\par 8 Und ich sah am Hause eine Erhöhung ringsumher als Grundlage der Gänge, die hatte eine volle Rute von sechs Ellen bis an den Rand.
\par 9 Und die Breite der Wand außen an den Gängen war fünf Ellen; und es war ein freigelassener Raum an den Gemächern am Hause.
\par 10 Und die Breite bis zu den Kammern war zwanzig Ellen um das Haus herum.
\par 11 Und es waren zwei Türen an den Gängen nach dem freigelassenen Raum, eine gegen Mitternacht, die andere gegen Mittag; und der freigelassene Raum war fünf Ellen weit ringsumher.
\par 12 Und das Gebäude am Hofraum gegen Abend war siebzig Ellen weit und die Mauer des Gebäudes war fünf Ellen breit allenthalben umher, und es war neunzig Ellen lang.
\par 13 Und er maß die Länge des Hauses, die hatte hundert Ellen; und der Hofraum samt dem Gebäude und seinen Mauern war auch hundert Ellen lang.
\par 14 Und die Weite der vordern Seite des Hauses samt dem Hofraum gegen Morgen war auch hundert Ellen.
\par 15 Und er maß die Länge des Gebäudes am Hofraum, welches hinter ihm liegt, mit seinen Umgängen von der Seite bis zur andern hundert Ellen, und den innern Tempel und die Hallen im Vorhofe
\par 16 samt den Schwellen, den engen Fenstern und den drei Umgängen ringsumher; und es war Tafelwerk allenthalben herum.
\par 17 Er maß auch, wie hoch von der Erde bis zu den Fenstern war und wie breit die Fenster sein sollten; und maß vom Tor bis zum Allerheiligsten auswendig und inwendig herum.
\par 18 Und am ganzen Hause herum waren Cherubim und Palmlaubwerk zwischen die Cherubim gemacht.
\par 19 Und ein jeder Cherub hatte zwei Angesichter: auf einer Seite wie ein Menschenkopf, auf der andern Seite wie ein Löwenkopf.
\par 20 Vom Boden an bis hinauf über die Tür waren die Cherubim und die Palmen geschnitzt, desgleichen an der Wand des Tempels.
\par 21 Und die Türpfosten im Tempel waren viereckig, und war alles artig aneinander gefügt.
\par 22 Und der hölzerne Altar war drei Ellen hoch und zwei Ellen lang und breit, und seine Ecken und alle seine Seiten waren hölzern. und er sprach zu mir: Das ist der Tisch, der vor dem HERRN stehen soll.
\par 23 Und die Türen am Tempel und am Allerheiligsten
\par 24 hatten zwei Türflügel, und ein jeder derselben hatte zwei Blätter, die man auf und zu tat.
\par 25 Und waren auch Cherubim und Palmlaubwerk daran wie an den Wänden. Und ein hölzerner Aufgang war außen vor der Halle.
\par 26 Und es waren enge Fenster und viel Palmlaubwerk herum an der Halle und an den Wänden.

\chapter{42}

\par 1 Und er führte mich hinaus zum äußeren Vorhof gegen Mitternacht und brachte mich zu den Kammern, so gegenüber dem Hofraum und gegenüber dem Gebäude nach Mitternacht zu lagen,
\par 2 entlang den hundert Ellen an der Tür gegen Mitternacht; und ihre Breite war fünfzig Ellen.
\par 3 Gegenüber den zwanzig Ellen des innern Vorhofs und gegenüber dem Pflaster im äußern Vorhof war Umgang an Umgang dreifach.
\par 4 Und inwendig vor den Kammern war ein Weg zehn Ellen breit vor den Türen der Kammern; die lagen alle gegen Mitternacht.
\par 5 Und die obern Kammern waren enger als die untern Kammern; denn die die Umgänge nahmen Raum von ihnen weg.
\par 6 Denn es waren drei Gemächer hoch, und sie hatten keine Säulen, wie die Vorhöfe Säulen hatten. Darum war von den untern und mittleren Kammern Raum weggenommen von untenan.
\par 7 Und die Mauer außen vor den Kammern nach dem äußeren Vorhof war fünfzig Ellen lang.
\par 8 Denn die Länge der Kammern nach dem äußeren Vorhof zu war fünfzig Ellen; aber gegen den Tempel waren es hundert Ellen.
\par 9 Und unten an diesen Kammern war ein Eingang gegen Morgen, da man aus dem äußeren Vorhof zu ihnen hineinging.
\par 10 Und an der Mauer gegen Mittag waren auch Kammern gegenüber dem Hofraum und gegenüber dem Gebäude.
\par 11 Und war auch ein Weg davor wie vor jenen Kammern, so gegen Mitternacht lagen; und war alles gleich mit der Länge, Breite und allem, was daran war, wie droben an jenen.
\par 12 Und wie die Türen jener, also waren auch die Türen der Kammern gegen Mittag; und am Anfang des Weges war eine Tür, dazu man kommt von der Mauer, die gegen Morgen liegt.
\par 13 Und er sprach zu mir: Die Kammern gegen Mittag gegenüber dem Hofraum, das sind die heiligen Kammern, darin die Priester, welche dem HERRN nahen, die hochheiligen Opfer, nämlich Speisopfer, Sündopfer und Schuldopfer, da hineinlegen; denn es ist eine heilige Stätte.
\par 14 Und wenn die Priester hineingehen, sollen sie nicht wieder aus dem Heiligtum gehen in den äußeren Vorhof, sonder sollen zuvor ihre Kleider, darin sie gedient haben, in den Kammern weglegen, denn sie sind heilig; und sollen ihre anderen Kleider anlegen und alsdann heraus unters Volk gehen.
\par 15 Und da er das Haus inwendig ganz gemessen hatte, führte er mich heraus zum Tor gegen Morgen und maß von demselben allenthalben herum.
\par 16 Gegen Morgen maß er fünfhundert Ruten lang;
\par 17 und gegen Mitternacht maß er auch fünfhundert Ruten lang;
\par 18 desgleichen gegen Mittag auch fünfhundert Ruten;
\par 19 und da er kam gegen Abend, maß er auch fünfhundert Ruten lang.
\par 20 Also hatte die Mauer, die er gemessen, ins Gevierte auf jeder Seite herum fünfhundert Ruten, damit das Heilige von dem Unheiligen unterschieden wäre.

\chapter{43}

\par 1 Und er führte mich wieder zum Tor gegen Morgen.
\par 2 Und siehe, die Herrlichkeit des Gottes Israels kam von Morgen und brauste, wie ein großes Wasser braust; und es ward sehr licht auf der Erde von seiner Herrlichkeit.
\par 3 Und es war eben wie das Gesicht, das ich sah, da ich kam, daß die Stadt sollte zerstört werden, und wie das Gesicht, das ich gesehen hatte am Wasser Chebar. Da fiel ich nieder auf mein Angesicht.
\par 4 Und die Herrlichkeit des HERRN kam hinein zum Hause durchs Tor gegen Morgen.
\par 5 Da hob mich ein Wind auf und brachte mich in den innern Vorhof; und siehe, die Herrlichkeit des HERRN erfüllte das Haus.
\par 6 Und ich hörte einen mit mir reden vom Hause heraus, und ein Mann stand neben mir.
\par 7 Der sprach zu mir: Du Menschenkind, das ist der Ort meines Throns und die Stätte meiner Fußsohlen, darin ich ewiglich will wohnen unter den Kindern Israel. Und das Haus Israel soll nicht mehr meinen heiligen Namen verunreinigen, weder sie noch ihre Könige, durch ihre Abgötterei und durch die Leichen ihrer Könige in ihren Höhen,
\par 8 welche ihre Schwelle an meine Schwelle und ihre Pfoste an meine Pfoste gesetzt haben, daß nur eine Wand zwischen mir und ihnen war; und haben also meinen heiligen Namen verunreinigt durch ihre Greuel, die sie taten, darum ich sie auch in meinem Zorn verzehrt habe.
\par 9 Nun aber sollen sie ihre Abgötterei und die Leichen ihrer Könige fern von mir wegtun; und ich will ewiglich unter ihnen wohnen.
\par 10 Und du, Menschenkind, zeige dem Haus Israel den Tempel an, daß sie sich schämen ihrer Missetaten, und laß sie ein reinliches Muster davon nehmen.
\par 11 Und wenn sie sich nun alles ihres Tuns schämen, so zeige ihnen die Gestalt und das Muster des Hauses und seine Ausgänge und Eingänge und alle seine Weise und alle seine Sitten und alle seine Weise und alle seine Gesetze; und schreibe es ihnen vor, daß sie alle seine Weise und alle seine Sitten halten und darnach tun.
\par 12 Das soll aber das Gesetz des Hauses sein: Auf der Höhe des Berges, soweit ihr Umfang ist, soll das Allerheiligste sein; das ist das Gesetz des Hauses.
\par 13 Das aber ist das Maß des Altars nach der Elle, welche eine handbreit länger ist den die gemeine Elle: sein Fuß ist eine Elle hoch und eine Elle breit; und die Leiste an seinem Rand ist eine Spanne breit umher.
\par 14 Und das ist die Höhe: Von dem Fuße auf der Erde bis an den untern Absatz sind zwei Ellen hoch und eine Elle breit; aber von demselben kleineren Absatz sind's vier Ellen hoch und eine Elle breit.
\par 15 Und der Harel (der Gottesberg) vier Ellen hoch, und vom Ariel (dem Gottesherd) überwärts die vier Hörner.
\par 16 Der Ariel aber war zwölf Ellen lang und zwölf Ellen breit ins Geviert.
\par 17 Und der oberste Absatz war vierzehn Ellen lang und vierzehn Ellen breit ins Geviert; und eine Leiste ging allenthalben umher, eine halbe Elle breit; und sein Fuß war eine Elle hoch, und seine Stufen waren gegen Morgen.
\par 18 Und er sprach zu mir: Du Menschenkind, so spricht der HERR HERR: Dies sollen die Sitten des Altars sein des Tages, da er gemacht ist, daß man Brandopfer darauf lege und Blut darauf sprenge.
\par 19 Und den Priestern von Levi aus dem Samen Zadoks, die da vor mich treten, daß sie mir dienen, spricht der HERR HERR, sollst du geben einen jungen Farren zum Sündopfer.
\par 20 Und von desselben Blut sollst du nehmen und seine vier Hörner damit besprengen und die vier Ecken an dem obersten Absatz und um die Leiste herum; damit sollst du ihn entsündigen und versöhnen.
\par 21 Und sollst den Farren des Sündopfers nehmen und ihn verbrennen an einem Ort am Hause, der dazu verordnet ist außerhalb des Heiligtums.
\par 22 Aber am andern Tage sollst du einen Ziegenbock opfern, der ohne Fehl sei, zu einem Sündopfer und den Altar damit entsündigen, wie er mit dem Farren entsündigt ist.
\par 23 Und wenn das Entsündigen vollendet ist, sollst du einen jungen Farren opfern, der ohne Fehl sei, und einen Widder von der Herde ohne Fehl.
\par 24 Und sollst sie beide vor dem HERRN opfern; und die Priester sollen Salz darauf streuen und sollen sie also opfern dem HERRN zum Brandopfer.
\par 25 Also sollst du sieben Tage nacheinander täglich einen Bock zum Sündopfer opfern; und sie sollen einen jungen Farren und einen Widder von der Herde, die beide ohne Fehl sind, opfern.
\par 26 Und sollen also sieben Tage lang den Altar versöhnen und ihn reinigen und ihre Hände füllen.
\par 27 Und nach denselben Tagen sollen die Priester am achten Tag und hernach für und für auf dem Altar opfern eure Brandopfer und eure Dankopfer, so will ich euch gnädig sein, spricht der HERR HERR.

\chapter{44}

\par 1 Und er führte mich wiederum zu dem äußern Tor des Heiligtums gegen Morgen; es war aber verschlossen.
\par 2 Und der HERR sprach zu mir: Dies Tor soll zugeschlossen bleiben und nicht aufgetan werden, und soll niemand dadurchgehen; denn der HERR, der Gott Israels, ist dadurch eingegangen, darum soll es zugeschlossen bleiben.
\par 3 Doch den Fürsten ausgenommen; denn der Fürst soll daruntersitzen, das Brot zu essen vor dem HERRN. Durch die Halle des Tors soll er hineingehen und durch dieselbe wieder herausgehen.
\par 4 Darnach führte er mich zum Tor gegen Mitternacht vor das Haus. Und ich sah, und siehe, des HERRN Haus war voll der Herrlichkeit des HERRN; und ich fiel auf mein Angesicht.
\par 5 Und der HERR sprach zu mir: Du Menschenkind, merke darauf und siehe und höre fleißig auf alles, was ich dir sagen will von den Sitten und Gesetzen im Haus des HERRN; und merke, wie man hineingehen soll, und auf alle Ausgänge des Heiligtums.
\par 6 Und sage dem ungehorsamen Hause Israel: So spricht der HERR HERR: Ihr macht es zuviel, ihr vom Hause Israel, mit allen euren Greueln,
\par 7 denn ihr führt fremde Leute eines unbeschnittenen Herzens und unbeschnittenen Fleisches in mein Heiligtum, dadurch ihr mein Haus entheiligt, wenn ihr mein Brot, Fettes und Blut opfert, und brecht also meinen Bund mit allen euren Greueln;
\par 8 und haltet die Sitten meines Heiligtums nicht, sondern macht euch selbst neue Sitten in meinem Heiligtum.
\par 9 Darum spricht der HERR HERR also: Es soll kein Fremder eines unbeschnittenen Herzens und unbeschnittenen Fleisches in mein Heiligtum kommen aus allen Fremdlingen, so unter den Kindern Israel sind;
\par 10 sondern die Leviten, die von mir gewichen sind und samt Israel von mir irregegangen nach ihren Götzen, die sollen ihre Sünde tragen,
\par 11 und sollen in meinem Heiligtum dienen als Hüter an den Türen des Hauses und als Diener des Hauses; und sollen nur das Brandopfer und andere Opfer, so das Volk herzubringt, schlachten und vor den Leuten stehen, daß sie ihnen dienen.
\par 12 Darum daß sie ihnen gedient vor ihren Götzen und dem Haus Israel einen Anstoß zur Sünde gegeben haben, darum habe ich meine Hand über sie ausgestreckt, spricht der HERR HERR, daß sie müssen ihre Sünde tragen.
\par 13 Und sie sollen nicht zu mir nahen, Priesteramt zu führen, noch kommen zu allen meinen Heiligtümern, zu den hochheiligen Opfern, sondern sie sollen ihre Schande tragen und ihre Greuel, die sie geübt haben.
\par 14 Darum habe ich sie zu Hütern gemacht an allem Dienst des Hauses und zu allem, was man darin tun soll.
\par 15 Aber die Priester aus den Leviten, die Kinder Zadok, so die Sitten meines Heiligtums gehalten haben, da die Kinder Israel von mir abfielen, die sollen vor mich treten und mir dienen und vor mir stehen, daß sie mir das Fett und Blut opfern, spricht der HERR HERR.
\par 16 Und sie sollen hineingehen in mein Heiligtum und vor meinen Tisch treten, mir zu dienen und meine Sitten zu halten.
\par 17 Und wenn sie durch die Tore des innern Vorhofs gehen wollen, sollen sie leinene Kleider anziehen und nichts Wollenes anhaben, wenn sie in den Toren im innern Vorhofe und im Hause dienen.
\par 18 Und sollen leinenen Schmuck auf ihrem Haupt haben und leinene Beinkleider um ihre Lenden, und sollen sich nicht im Schweiß gürten.
\par 19 Und wenn sie in den äußern Vorhof zum Volk herausgehen, sollen sie die Kleider, darin sie gedient haben, ausziehen und dieselben in die Kammern des Heiligtums legen und andere Kleider anziehen und das Volk nicht heiligen in ihren eigenen Kleidern.
\par 20 Ihr Haupt sollen sie nicht kahl scheren, und sollen auch nicht die Haare frei wachsen lassen, sondern sollen die Haare umher verschneiden.
\par 21 Und soll auch kein Priester Wein trinken, wenn sie in den innern Vorhof gehen sollen.
\par 22 Und sollen keine Witwe noch Verstoßene zur Ehe nehmen, sondern Jungfrauen vom Samen des Hauses Israel oder eines Priesters nachgelassene Witwe.
\par 23 Und sie sollen mein Volk lehren, daß sie wissen Unterschied zu halten zwischen Heiligem und Unheiligem und zwischen Reinem und Unreinem.
\par 24 Und wo eine Sache vor sie kommt, sollen sie stehen und richten und nach meinen Rechten sprechen und sollen meine Gebote und Sitten halten und alle meine Feste halten und meine Sabbate heiligen.
\par 25 Und sollen zu keinem Toten gehen und sich verunreinigen, nur allein zu Vater und Mutter, Sohn oder Tochter, Bruder oder Schwester, die noch keinen Mann gehabt hat; über denen mögen sie sich verunreinigen.
\par 26 Und nach seiner Reinigung soll man zählen sieben Tage.
\par 27 Und wenn er wieder hinein zum Heiligtum geht in den innern Vorhof, daß er im Heiligtum diene, so soll er sein Sündopfer opfern, spricht der HERR HERR.
\par 28 Aber das Erbteil, das sie haben sollen, das will ich selbst sein. Darum sollt ihr ihnen kein eigen Land geben in Israel; denn ich bin ihr Erbteil.
\par 29 Sie sollen ihre Nahrung haben vom Speisopfer, Sündopfer und Schuldopfer, und alles Verbannte in Israel soll ihnen gehören.
\par 30 Und alle ersten Früchte und alle Hebopfer von allem, davon ihr Hebopfer bringt, sollen den Priestern gehören. Ihr sollt auch den Priestern die Erstlinge eures Teiges geben, damit der Segen in deinem Hause bleibe.
\par 31 Was aber ein Aas oder zerrissen ist, es sei von Vögeln oder Tieren, das sollen die Priester nicht essen.

\chapter{45}

\par 1 Wenn ihr nun das Land durchs Los austeilt, so sollt ihr ein Hebopfer vom Lande absondern, das dem HERRN heilig sein soll, fünfundzwanzigtausend Ruten lang und zehntausend breit; der Platz soll heilig sein, soweit er reicht.
\par 2 Und von diesem sollen zum Heiligtum kommen je fünfhundert Ruten ins Gevierte und dazu ein freier Raum umher fünfzig Ellen.
\par 3 Und auf dem Platz, der fünfundzwanzigtausend Ruten lang und zehntausend breit ist, soll das Heiligtum stehen, das Allerheiligste.
\par 4 Das übrige aber vom geheiligten Lande soll den Priestern gehören, die im Heiligtum dienen und vor den HERRN treten, ihm zu dienen, daß sie Raum zu Häusern haben, und soll auch heilig sein.
\par 5 Aber die Leviten, so vor dem Hause dienen, sollen auch fünfundzwanzigtausend Ruten lang und zehntausend breit haben zu ihrem Teil, daß sie da wohnen.
\par 6 Und der Stadt sollt ihr auch einen Platz lassen für das ganze Haus Israel, fünftausend Ruten breit und fünfundzwanzigtausend lang, neben dem geheiligten Lande.
\par 7 Dem Fürsten aber sollt ihr auch einen Platz geben zu beiden Seiten, neben dem geheiligten Lande und neben dem Platz der Stadt, und soll der Platz gegen Abend und gegen Morgen so weit reichen als die Teile der Stämme.
\par 8 Das soll sein eigen Teil sein in Israel, damit meine Fürsten nicht mehr meinem Volk das Ihre nehmen, sondern sollen das Land dem Haus Israel lassen für ihre Stämme.
\par 9 Denn so spricht der HERR HERR: Ihr habt's lange genug gemacht, ihr Fürsten Israels; lasset ab von Frevel und Gewalt und tut, was recht und gut ist, und tut ab von meinem Volk euer Austreiben, spricht der HERR HERR.
\par 10 Ihr sollt rechtes Gewicht und rechte Scheffel und rechtes Maß haben.
\par 11 Epha und Bath sollen gleich sein, daß ein Bath den zehnten Teil vom Homer habe und das Epha den zehnten Teil vom Homer; denn nach dem Homer soll man sie beide messen.
\par 12 Aber ein Lot soll zwanzig Gera haben; und eine Mina macht zwanzig Lot, fünfundzwanzig Lot und fünfzehn Lot.
\par 13 Das soll nun das Hebopfer sein, das ihr heben sollt, nämlich den sechsten Teil eines Epha von einem Homer Weizen und den sechsten Teil eines Epha von einem Homer Gerste.
\par 14 Und vom Öl sollt ihr geben je den zehnten Teil eines Bath vom Kor, welches zehn Bath oder ein Homer ist; denn zehn Bath machen einen Homer.
\par 15 Und je ein Lamm von zweihundert Schafen aus der Herde auf der Weide Israels zum Speisopfer und Brandopfer und Dankopfer, zur Versöhnung für sie, spricht der HERR HERR.
\par 16 Alles Volk im Lande soll solches Hebopfer zum Fürsten in Israel bringen.
\par 17 Und der Fürst soll die Brandopfer, Speisopfer und Trankopfer ausrichten auf die Feste, Neumonde und Sabbate, auf alle Feiertage des Hauses Israel; er soll die Sündopfer und Speisopfer, Brandopfer und Dankopfer tun zur Versöhnung für das Haus Israel.
\par 18 So spricht der HERR HERR: Am ersten Tage des ersten Monats sollst du nehmen einen jungen Farren, der ohne Fehl sei, und das Heiligtum entsündigen.
\par 19 Und der Priester soll von dem Blut des Sündopfers nehmen und die Pfosten am Hause damit besprengen und die vier Ecken des Absatzes am Altar samt den Pfosten am Tor des Innern Vorhofs.
\par 20 Also sollst du auch tun am siebenten Tage des Monats wegen derer, die geirrt haben oder weggeführt worden sind, daß ihr das Haus entsündigt.
\par 21 Am vierzehnten Tage des ersten Monats sollt ihr das Passah halten und sieben Tage feiern und ungesäuertes Brot essen.
\par 22 Und am selben Tage soll der Fürst für sich und für alles Volk im Lande einen Farren zum Sündopfer opfern.
\par 23 Aber die sieben Tage des Festes soll er dem HERRN täglich ein Brandopfer tun: je sieben Farren und sieben Widder, die ohne Fehl seien; und je einen Ziegenbock zum Sündopfer.
\par 24 Zum Speisopfer aber soll er je ein Epha zu einem Farren und ein Epha zu einem Widder opfern und je ein Hin Öl zu einem Epha.
\par 25 Am fünfzehnten Tage des siebenten Monats soll er sieben Tage nacheinander feiern, gleichwie jene sieben Tage, und es ebenso halten mit Sündopfer, Brandopfer, Speisopfer samt dem Öl.

\chapter{46}

\par 1 So spricht der HERR HERR: Das Tor am innern Vorhof morgenwärts soll die sechs Werktage zugeschlossen sein; aber am Sabbat und am Neumonde soll man's auftun.
\par 2 Und der Fürst soll von draußen unter die Halle des Tors treten und bei dem Pfosten am Tor stehenbleiben. Und die Priester sollen sein Brandopfer und Dankopfer opfern; er aber soll auf der Schwelle des Tors anbeten und darnach wieder hinausgehen; das Tor aber soll offen bleiben bis an den Abend.
\par 3 Desgleichen das Volk im Lande soll an der Tür desselben Tors anbeten vor dem HERRN an den Sabbaten und Neumonden.
\par 4 Das Brandopfer aber, so der Fürst vor dem HERRN opfern soll am Sabbattage, soll sein sechs Lämmer, die ohne Fehl seien, und ein Widder ohne Fehl;
\par 5 Und je ein Epha zu einem Widder zum Speisopfer, zu den Lämmern aber, soviel seine Hand gibt, zum Speisopfer, und je ein Hin Öl zu einem Epha.
\par 6 Am Neumonde aber soll er einen jungen Farren opfern, der ohne Fehl sei, und sechs Lämmer und einen Widder auch ohne Fehl;
\par 7 und je ein Epha zum Farren und je ein Epha zum Widder zum Speisopfer, aber zu den Lämmern soviel, als er geben mag, und je ein Hin Öl zu einem Epha.
\par 8 Und wenn der Fürst hineingeht, soll er durch die Halle des Tors hineingehen und desselben Weges wieder herausgehen.
\par 9 Aber das Volk im Lande, so vor den HERRN kommt auf die hohen Feste und zum Tor gegen Mitternacht hineingeht, anzubeten, das soll durch das Tor gegen Mittag wieder herausgehen; und welche zum Tor gegen Mittag hineingehen, die sollen zum Tor gegen Mitternacht wieder herausgehen; und sollen nicht wieder zu dem Tor hinausgehen, dadurch sie hinein sind gegangen, sondern stracks vor sich hinausgehen.
\par 10 Der Fürst aber soll mit ihnen hinein und heraus gehen.
\par 11 Aber an den Feiertagen und hohen Festen soll man zum Speisopfer je zu einem Farren ein Epha und je zu einem Widder ein Epha opfern und zu den Lämmern, soviel seine Hand gibt, und je ein Hin Öl zu einem Epha.
\par 12 Wenn aber der Fürst ein freiwilliges Brandopfer oder Dankopfer dem HERRN tun wollte, so soll man ihm das Tor gegen Morgen auftun, daß er sein Brandopfer und Dankopfer opfere, wie er's sonst am Sabbat pflegt zu opfern; und wenn er wieder herausgeht, soll man das Tor nach ihm zuschließen.
\par 13 Und er soll dem HERRN täglich ein Brandopfer tun, nämlich ein jähriges Lamm ohne Fehl; dasselbe soll er alle Morgen opfern.
\par 14 Und soll alle Morgen den sechsten Teil von einem Epha zum Speisopfer darauftun und den dritten Teil von einem Hin Öl auf das Semmelmehl zu träufen, dem HERRN zum Speisopfer; das soll ein ewiges Recht sein vom täglichem Opfer.
\par 15 Und also sollen sie das Lamm samt dem Speisopfer und Öl alle Morgen opfern zum täglichen Brandopfer.
\par 16 So spricht der HERR HERR: Wenn der Fürst seiner Söhne einem ein Geschenk gibt von seinem Erbe, dasselbe soll seinen Söhnen bleiben, und sie sollen es erblich besitzen.
\par 17 Wo er aber seiner Knechte einem von seinem Erbteil etwas schenkt, das sollen sie besitzen bis aufs Freijahr und soll alsdann dem Fürsten wieder heimfallen; denn sein Teil soll allein auf seine Söhne erben.
\par 18 Es soll auch der Fürst dem Volk nichts nehmen von seinem Erbteil noch sie aus ihren Gütern stoßen, sondern soll sein eigenes Gut auf seine Kinder vererben, auf daß meines Volks nicht jemand von seinem Eigentum zerstreut werde.
\par 19 Und er führte mich durch den Eingang an der Seite des Tors gegen Mitternacht zu den Kammern des Heiligtums, so den Priestern gehörten; und siehe, daselbst war ein Raum in der Ecke gegen Abend.
\par 20 Und er sprach zu mir: Dies ist der Ort, da die Priester kochen sollen das Schuldopfer und Sündopfer und das Speisopfer backen, daß sie es nicht hinaus in den äußeren Vorhof tragen müssen, das Volk zu heiligen.
\par 21 Darnach führte er mich hinaus in den äußeren Vorhof und hieß mich gehen in die vier Ecken des Vorhofs.
\par 22 Und siehe, da war in jeglicher der vier Ecken ein anderes Vorhöflein, vierzig Ellen lang und dreißig Ellen breit, alle vier einerlei Maßes.
\par 23 Und es ging ein Mäuerlein um ein jegliches der vier; da waren Herde herum gemacht unten an den Mauern.
\par 24 Und er sprach zu mir: Dies sind die Küchen, darin die Diener des Hauses kochen sollen, was das Volk opfert.

\chapter{47}

\par 1 Und er führte mich wieder zu der Tür des Tempels. Und siehe, da floß ein Wasser heraus unter der Schwelle des Tempels gegen Morgen; denn die vordere Seite des Tempels war gegen Morgen. Und des Tempels Wasser lief an der rechten Seite des Tempels neben dem Altar hin gegen Mittag.
\par 2 Und er führte mich hinaus zum Tor gegen Mitternacht und brachte mich auswendig herum zum äußern Tor gegen Morgen; und siehe, das Wasser sprang heraus von der rechten Seite.
\par 3 Und der Mann ging heraus gegen Morgen und hatte die Meßschnur in der Hand; und er maß tausend Ellen und führte mich durchs Wasser, das mir's an die Knöchel ging.
\par 4 Und maß abermals tausend Ellen und führte mich durchs Wasser, daß mir's an die Kniee ging. Und maß noch tausend Ellen und ließ mich dadurchgehen, daß es mir an die Lenden ging.
\par 5 Da maß er noch tausend Ellen, und es ward so tief, daß ich nicht mehr Grund hatte; denn das Wasser war zu hoch, daß man darüber schwimmen mußte und keinen Grund hatte.
\par 6 Und er sprach zu mir: Du Menschenkind, das hast du ja gesehen. Und er führte mich wieder zurück am Ufer des Bachs.
\par 7 Und siehe, da standen sehr viel Bäume am Ufer auf beiden Seiten.
\par 8 Und er sprach zu mir: Dies Wasser, das da gegen Morgen herausfließt, wird durchs Blachfeld fließen ins Meer, da sollen desselben Wasser gesund werden.
\par 9 Ja, alles, was darin lebt und webt, dahin diese Ströme kommen, das soll leben; und es soll sehr viel Fische haben; und soll alles gesund werden und leben, wo dieser Strom hin kommt.
\par 10 Und es werden die Fischer an demselben stehen; von Engedi bis zu En-Eglaim wird man die Fischgarne aufspannen; denn es werden daselbst sehr viel Fische von allerlei Art sein, gleichwie im großen Meer.
\par 11 Aber die Teiche und Lachen daneben werden nicht gesund werden, sondern gesalzen bleiben.
\par 12 Und an demselben Strom, am Ufer auf beiden Seiten, werden allerlei fruchtbare Bäume wachsen, und ihre Blätter werden nicht verwelken noch ihre Früchte ausgehen; und sie werden alle Monate neue Früchte bringen, denn ihr Wasser fließt aus dem Heiligtum. Ihre Frucht wird zur Speise dienen und ihre Blätter zur Arznei.
\par 13 So spricht der HERR HERR: Dies sind die Grenzen, nach denen ihr das Land sollt austeilen den zwölf Stämmen Israels; denn zwei Teile gehören dem Stamm Joseph.
\par 14 Und ihr sollt's gleich austeilen, einem wie dem andern; denn ich habe meine Hand aufgehoben, das Land euren Vätern und euch zum Erbteil zu geben.
\par 15 Dies ist nun die Grenze des Landes gegen Mitternacht: von dem großen Meer an des Weges nach Hethlon gen Zedad,
\par 16 Hamath, Berotha, Sibraim, das an Damaskus und Hamath grenzt, und Hazar-Thichon, das an Hauran grenzt.
\par 17 Das soll die Grenze sein vom Meer an bis gen Hazar-Enon, und Damaskus und Hamath sollen das Ende sein. Das sei die Grenze gegen Mitternacht.
\par 18 Aber die Grenze gegen Morgen sollt ihr messen zwischen Hauran und Damaskus und zwischen Gilead und dem Lande Israel, am Jordan hinab bis an das Meer gegen Morgen. Das soll die Grenze gegen Morgen sein.
\par 19 Aber die Grenze gegen Mittag ist von Thamar bis ans Haderwasser zu Kades und den Bach hinab bis an das große Meer. Das soll die Grenze gegen Mittag sein.
\par 20 Und an der Seite gegen Abend ist das große Meer von der Grenze an bis gegenüber Hamath. Das sei die Grenze gegen Abend.
\par 21 Also sollt ihr das Land austeilen unter die Stämme Israels.
\par 22 Und wenn ihr das Los werft, das Land unter euch zu teilen, so sollt ihr die Fremdlinge, die bei euch wohnen und Kinder unter euch zeugen, halten gleich wie die Einheimischen unter den Kindern Israel;
\par 23 und sie sollen auch ihr Teil im Lande haben, ein jeglicher unter seinem Stamm, dabei er wohnt, spricht der HERR HERR.

\chapter{48}

\par 1 Dies sind die Namen der Stämme: von Mitternacht, an dem Wege nach Hethlon, gen Hamath und Hazar-Enon und von Damaskus gegen Hamath, das soll Dan für seinen Teil haben von Morgen bis gen Abend.
\par 2 Neben Dan soll Asser seinen Teil haben, von Morgen bis gen Abend.
\par 3 Neben Asser soll Naphthali seinen Teil haben, von Morgen bis gen Abend.
\par 4 Neben Naphthali soll Manasse seinen Teil haben, von Morgen bis gen Abend.
\par 5 Neben Manasse soll Ephraim seinen Teil haben, von Morgen bis gen Abend.
\par 6 Neben Ephraim soll Ruben seinen Teil haben, von Morgen bis gen Abend.
\par 7 Neben Ruben soll Juda seinen Teil haben, von Morgen bis gen Abend.
\par 8 Neben Juda aber sollt ihr einen Teil absondern, von Morgen bis gen Abend, der fünfundzwanzigtausend Ruten breit und so lang sei, wie sonst ein Teil ist von Morgen bis gen Abend; darin soll das Heiligtum stehen.
\par 9 Und davon sollt ihr dem HERRN einen Teil absondern, fünfundzwanzigtausend Ruten lang und zehntausend Ruten breit.
\par 10 Und dieser heilige Teil soll den Priestern gehören, nämlich fünfundzwanzigtausend Ruten lang gegen Mitternacht und gegen Mittag und zehntausend breit gegen Morgen und gegen Abend. Und das Heiligtum des HERRN soll mittendarin stehen.
\par 11 Das soll geheiligt sein den Priestern, den Kindern Zadok, welche meine Sitten gehalten haben und sind nicht abgefallen mit den Kindern Israel, wie die Leviten abgefallen sind.
\par 12 Und soll also dieser abgesonderte Teil des geheiligten Landes ihr eigen sein als Hochheiliges neben der Leviten Grenze.
\par 13 Die Leviten aber sollen neben der Priester Grenze auch fünfundzwanzigtausend Ruten in die Länge und zehntausend Ruten in die Breite haben; denn alle Länge soll fünfunzwanzigtausend und die Breite zehntausend Ruten haben.
\par 14 Und sollen nichts davon verkaufen noch verändern, damit des Landes Erstling nicht wegkomme; denn es ist dem HERRN geheiligt.
\par 15 Aber die übrigen fünftausend Ruten in die Breite gegen fünfunzwanzigtausend Ruten in die Länge, das soll gemeines Land sein zur Stadt, darin zu wohnen, und zu Vorstädten; und die Stadt soll mittendarin stehen.
\par 16 Und das soll ihr Maß sein: viertausend und fünfhundert Ruten gegen Mitternacht und gegen Mittag, desgleichen gegen Morgen und gegen Abend auch viertausend und fünfhundert.
\par 17 Die Vorstadt aber soll haben zweihundertundfünfzig Ruten gegen Mitternacht und gegen Mittag, desgleichen auch gegen Morgen und gegen Abend zweihundertundfünfzig Ruten.
\par 18 Aber das übrige an der Länge neben dem Abgesonderten und Geheiligten, nämlich zehntausend Ruten gegen Morgen und zehntausend Ruten gegen Abend, das gehört zum Unterhalt derer, die in der Stadt arbeiten.
\par 19 Und die Arbeiter aus allen Stämmen Israels sollen in der Stadt arbeiten.
\par 20 Also soll die ganze Absonderung fünfundzwanzigtausend Ruten ins Gevierte sein; ein Vierteil der geheiligten Absonderung sei zu eigen der Stadt.
\par 21 Was aber noch übrig ist auf beiden Seiten neben dem abgesonderten heiligen Teil und neben der Stadt Teil, nämlich fünfundzwanzigtausend Ruten gegen Morgen und gegen Abend neben den Teilen der Stämme, das soll alles dem Fürsten gehören. Aber der abgesonderte Teil und das Haus des Heiligtums soll mitteninnen sein.
\par 22 Was aber neben der Leviten Teil und neben der Stadt Teil zwischen der Grenze Juda's und der Grenze Benjamins liegt, das soll dem Fürsten gehören.
\par 23 Darnach sollen die andern Stämme sein: Benjamin soll seinen Teil haben, von Morgen bis gen Abend.
\par 24 Aber neben der Grenze Benjamin soll Simeon seinen Teil haben, von Morgen bis gen Abend.
\par 25 Neben der Grenze Simeons soll Isaschar seinen Teil haben, von Morgen bis gen Abend.
\par 26 Neben der Grenze Isaschars soll Sebulon seinen Teil haben, von Morgen bis gen Abend.
\par 27 Neben der Grenze Sebulons soll Gad seinen Teil haben, von Morgen bis gen Abend.
\par 28 Aber neben Gad ist die Grenze gegen Mittag von Thamar bis ans Haderwasser zu Kades und an den Bach hinab bis an das große Meer.
\par 29 Das ist das Land, das ihr austeilen sollt zum Erbteil unter die Stämme Israels; und das sollen ihre Erbteile sein, spricht der HERR HERR.
\par 30 Und so weit soll die Stadt sein: viertausend und fünfhundert Ruten gegen Mitternacht.
\par 31 Und die Tore der Stadt sollen nach den Namen der Stämme Israels genannt werden, drei Toren gegen Mitternacht: das erste Tor Ruben, das zweite Juda, das dritte Levi.
\par 32 Also auch gegen Morgen viertausend und fünfhundert Ruten und auch drei Tore: nämlich das erste Tor Joseph, das zweite Benjamin, das dritte Dan.
\par 33 Gegen Mittag auch also viertausend und fünfhundert Ruten und auch drei Tore: das erste Tor Simeon, das zweite Isaschar, das dritte Sebulon.
\par 34 Also auch gegen Abend viertausend und fünfhundert Ruten und drei Tore: ein Tor Gad, das zweite Asser, das dritte Naphthali.
\par 35 Also sollen es um und um achtzehntausend Ruten sein. Und alsdann soll die Stadt genannt werden: "Hier ist der HERR".

\end{document}