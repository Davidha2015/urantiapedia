\begin{document}

\title{Habakkuk}



\chapter{1}

\par 1 De last, welken Habakuk, de profeet, gezien heeft.
\par 2 HEERE! hoe lang schreeuw ik, en Gij hoort niet, hoe lang roep ik geweld, tot U, en Gij verlost niet!
\par 3 Waarom laat Gij mij ongerechtigheid zien, en aanschouwt de kwelling? Want verwoesting en geweld is tegen mij over, en er is twist, en men neemt gekijf op.
\par 4 Daarom wordt de wet onderlaten, en het recht komt nimmermeer voort; want de goddeloze omringt den rechtvaardige; daarom komt het recht verdraaid voor.
\par 5 Ziet onder de heidenen, en aanschouwt, en verwondert u, verwondert u, want Ik werk een werk in ulieder dagen, hetwelk gij niet geloven zult, als het verteld zal worden.
\par 6 Want ziet, Ik verwek de Chaldeen, een bitter en snel volk, trekkende door de breedten der aarde, om erfelijk te bezitten woningen, die de zijne niet zijn.
\par 7 Schrikkelijk en vreselijk is hetzelve; zijn recht en zijn hoogheid gaat van hemzelven uit.
\par 8 Want zijn paarden zijn lichter dan de luipaarden, en zij zijn scherper dan de avondwolven, en zijn ruiters verspreiden zich; ja, zijn ruiters zullen van verre komen, zij zullen vliegen als een arend, zich spoedende om te eten.
\par 9 Het zal geheellijk tot geweld komen, wat zij inslorpen zullen met hun aangezichten, zullen zij brengen naar het oosten; en het zal de gevangenen verzamelen als zand.
\par 10 En hij zal de koningen beschimpen, en de prinsen zullen hem een belaching zijn; hij zal alle vesting belachen; want hij zal stof vergaderen, en hij zal ze innemen.
\par 11 Dan zal hij den geest veranderen, en hij zal doortrekken, en zich schuldig maken, houdende deze zijn kracht voor zijn God.
\par 12 Zijt Gij niet van ouds af de HEERE, mijn God, mijn Heilige? Wij zullen niet sterven; o HEERE! tot een oordeel hebt Gij hem gesteld, en o Rots! om te straffen, hebt Gij hem gegrondvest.
\par 13 Gij zijt te rein van ogen, dan dat Gij het kwade zoudt zien, en de kwelling kunt Gij niet aanschouwen; waarom zoudt Gij aanschouwen die trouwelooslijk handelen? Waarom zoudt Gij zwijgen, als de goddeloze dien verslindt, die rechtvaardiger is dan hij?
\par 14 En waarom zoudt Gij de mensen maken, als de vissen der zee, als het kruipend gedierte, dat geen heerser heeft?
\par 15 Hij trekt ze allen met den angel op, hij vergadert ze in zijn garen, en hij verzamelt ze in zijn net; daarom verblijdt en verheugt hij zich.
\par 16 Daarom offert hij aan zijn garen, en rookt aan zijn net; want door dezelve is zijn deel vet geworden, en zijn spijze smoutig.
\par 17 Zal hij dan daarom altoos zijn garen ledig maken, en zal hij niet verschonen, met altoos de volken te doden?

\chapter{2}

\par 1 Ik stond op mijn wacht, en ik stelde mij op de sterkte, en ik hield wacht om te zien, wat Hij in mij spreken zou, en wat ik antwoorden zou op mijn bestraffing.
\par 2 Toen antwoordde mij de HEERE, en zeide: Schrijf het gezicht, en stel het duidelijk op tafelen, opdat daarin leze die voorbijloopt.
\par 3 Want het gezicht zal nog tot een bestemden tijd zijn, dan zal Hij het op het einde voortbrengen, en niet liegen; zo Hij vertoeft, verbeid Hem, want Hij zal gewisselijk komen, Hij zal niet achterblijven.
\par 4 Ziet, zijn ziel verheft zich, zij is niet recht in hem; maar de rechtvaardige zal door zijn geloof leven.
\par 5 En ook dewijl hij trouwelooslijk handelt bij den wijn, een trots man is, en in zijn woning niet blijft; die zijn ziel wijd opendoet als het graf, en gelijk de dood is, die niet zat wordt, en tot zich verzamelt al de heidenen, en vergadert tot zich alle volken.
\par 6 Zouden dan niet al dezelve van hem een spreekwoord opnemen, en een uitlegging der raadselen van hem? En men zal zeggen: Wee dien, die vermeerdert hetgeen het zijne niet is (hoe lange!), en dien, die op zich laadt dik slijk.
\par 7 Zullen niet onvoorziens opstaan, die u bijten zullen, en ontwaken, die u zullen bewegen, en zult gij hun niet tot plundering worden?
\par 8 Omdat gij vele heidenen beroofd hebt, zo zullen alle overgeblevene volken u beroven; om het bloed der mensen, en het geweld aan het land, de stad, en alle inwoners derzelve.
\par 9 Wee dien, die met kwade gierigheid giert voor zijn huis, opdat hij in de hoogte zijn nest stelle, om bevrijd te zijn uit de hand des kwaads.
\par 10 Gij hebt schaamte beraadslaagd voor uw huis; uitroeiende vele volken, zo hebt gij gezondigd tegen uw ziel.
\par 11 Want de steen uit den muur roept, en de balk uit het hout antwoordt dien.
\par 12 Wee dien, die de stad met bloed bouwt, en die de stad met onrecht bevestigt!
\par 13 Ziet, is het niet van den HEERE der heirscharen, dat de volken arbeiden ten vure, en de lieden zich vermoeien tevergeefs?
\par 14 Want de aarde zal vervuld worden, dat zij de heerlijkheid des HEEREN bekennen, gelijk de wateren den bodem der zee bedekken.
\par 15 Wee dien, die zijn naaste te drinken geeft, gij, die uw wijnfles daarbij voegt, en ook dronken maakt, opdat gij hun naaktheden aanschouwt.
\par 16 Gij zult ook verzadigd worden met schande, voor eer; drinkt gij ook, en ontbloot de voorhuid; de beker der rechterhand des HEEREN zal zich tot u wenden, en er zal een schandelijk uitbraaksel over uw heerlijkheid zijn.
\par 17 Want het geweld, dat tegen Libanon begaan is, zal u bedekken, en de verwoesting der beesten zal ze verschrikken, om des bloeds wil der mensen, en des gewelds in het land, de stad en aan alle inwoners derzelve.
\par 18 Wat zal het gesneden beeld baten, dat zijn formeerder het gesneden heeft? of het gegoten beeld, hetwelk een leugenleraar is, dat de formeerder op zijn formeersel vertrouwt, als hij stomme afgoden gemaakt heeft?
\par 19 Wee dien, die tot het hout zegt: Word wakker! en: Ontwaak! tot den zwijgenden steen. Zou het leren? Ziet, het is met goud en zilver overtrokken, en er is gans geen geest in het midden van hetzelve.
\par 20 Maar de HEERE is in Zijn heiligen tempel. Zwijg voor Zijn aangezicht, gij ganse aarde!

\chapter{3}

\par 1 Een gebed van Habakuk, den profeet, op Sjigjonoth.
\par 2 HEERE! als ik Uw rede gehoord heb, heb ik gevreesd; Uw werk, o HEERE! behoud dat in het leven in het midden der jaren, maak het bekend in het midden der jaren; in den toorn gedenk des ontfermens.
\par 3 God kwam van Theman, en de Heilige van den berg Paran. Sela. Zijn heerlijkheid bedekte de hemelen, en het aardrijk was vol van Zijn lof.
\par 4 En er was een glans als des lichts, Hij had hoornen aan Zijn hand, en aldaar was Zijn sterkte verborgen.
\par 5 Voor Zijn aangezicht ging de pestilentie, en de vurige kool ging voor Zijn voeten henen.
\par 6 Hij stond, en mat het land, Hij zag toe, en maakte de heidenen los, en de gedurige bergen zijn verstrooid geworden; de heuvelen der eeuwigheid hebben zich gebogen; de gangen der eeuw zijn Zijne.
\par 7 Ik zag de tenten van Kusan onder de ijdelheid; de gordijnen des lands van Midian schudden.
\par 8 Was de HEERE ontstoken tegen de rivieren? Was Uw toorn tegen de rivieren, was Uw verbolgenheid tegen de zee, toen Gij op Uw paarden reedt? Uw wagens waren heil.
\par 9 De naakte grond werd ontbloot door Uw boog, om de eden, aan de stammen gedaan door het woord. Sela. Gij hebt de rivieren der aarde gekloofd.
\par 10 De bergen zagen U, en leden smart; de waterstroom ging door, de afgrond gaf zijn stem, hij hief zijn zijden op in de hoogte.
\par 11 De zon en de maan stonden stil in haar woning; met het licht gingen Uw pijlen daarhenen, met glans Uw bliksemende spies.
\par 12 Met gramschap tradt Gij door het land, met toorn dorstet Gij de heidenen.
\par 13 Gij toogt uit tot verlossing Uws volks, tot verlossing met Uw Gezalfde; Gij doorwonddet het hoofd van het huis des goddelozen, ontblotende den grond tot den hals toe. Sela.
\par 14 Gij doorboordet met zijn staven het hoofd zijner dorplieden; zij hebben gestormd, om mij te verstrooien; die zich verheugden, alsof zij de ellendigen in het verborgen zouden opeten.
\par 15 Gij betradt met Uw paarden de zee; de geweldige wateren werden een hoop.
\par 16 Als ik het hoorde, zo werd mijn buik beroerd; voor de stem hebben mijn lippen gebeefd; verrotting kwam in mijn gebeente, en ik werd beroerd in mijn plaats. Zekerlijk, ik zal rusten ten dage der benauwdheid, als hij optrekken zal tegen het volk, dat hij het met benden aanvalle.
\par 17 Alhoewel de vijgeboom niet bloeien zal, en geen vrucht aan den wijnstok zijn zal, dat het werk des olijfbooms liegen zal, en de velden geen spijze voortbrengen; dat men de kudde uit de kooi afscheuren zal, en dat er geen rund in de stallingen wezen zal;
\par 18 Zo zal ik nochtans in den HEERE van vreugde opspringen, ik zal mij verheugen in den God mijns heils.
\par 19 De Heere HEERE is mijn Sterkte; en Hij zal mijn voeten maken als der hinden, en Hij zal mij doen treden op mijn hoogten. Voor den opperzangmeester op mijn Neginoth.


\end{document}