\begin{document}

\title{Gospel of Matthew}


\chapter{1}

\par 1 Dies ist das Buch von der Geburt Jesu Christi, der da ist ein Sohn Davids, des Sohnes Abrahams.
\par 2 Abraham zeugte Isaak. Isaak zeugte Jakob. Jakob zeugte Juda und seine Brüder.
\par 3 Juda zeugte Perez und Serah von Thamar. Perez zeugte Hezron. Hezron zeugte Ram.
\par 4 Ram zeugte Amminadab. Amminadab zeugte Nahesson. Nahesson zeugte Salma.
\par 5 Salma zeugte Boas von der Rahab. Boas zeugte Obed von der Ruth. Obed zeugte Jesse.
\par 6 Jesse zeugte den König David. Der König David zeugte Salomo von dem Weib des Uria.
\par 7 Salomo zeugte Rehabeam. Rehabeam zeugte Abia. Abia zeugte Asa.
\par 8 Asa zeugte Josaphat. Josaphat zeugte Joram. Joram zeugte Usia.
\par 9 Usia zeugte Jotham. Jotham zeugte Ahas. Ahas zeugte Hiskia.
\par 10 Hiskia zeugte Manasse. Manasse zeugte Amon. Amon zeugte Josia.
\par 11 Josia zeugte Jechonja und seine Brüder um die Zeit der babylonischen Gefangenschaft.
\par 12 Nach der babylonischen Gefangenschaft zeugte Jechonja Sealthiel. Sealthiel zeugte Serubabel.
\par 13 Serubabel zeugte Abiud. Abiud zeugte Eliakim. Eliakim zeugte Asor.
\par 14 Asor zeugte Zadok. Zadok zeugte Achim. Achim zeugte Eliud.
\par 15 Eliud zeugte Eleasar. Eleasar zeugte Matthan. Matthan zeugte Jakob.
\par 16 Jakob zeugte Joseph, den Mann Marias, von welcher ist geboren Jesus, der da heißt Christus.
\par 17 Alle Glieder von Abraham bis auf David sind vierzehn Glieder. Von David bis auf die Gefangenschaft sind vierzehn Glieder. Von der babylonischen Gefangenschaft bis auf Christus sind vierzehn Glieder.
\par 18 Die Geburt Christi war aber also getan. Als Maria, seine Mutter, dem Joseph vertraut war, fand sich's ehe er sie heimholte, daß sie schwanger war von dem heiligen Geist.
\par 19 Joseph aber, ihr Mann, war fromm und wollte sie nicht in Schande bringen, gedachte aber, sie heimlich zu verlassen.
\par 20 Indem er aber also gedachte, siehe, da erschien ihm ein Engel des HERRN im Traum und sprach: Joseph, du Sohn Davids, fürchte dich nicht, Maria, dein Gemahl, zu dir zu nehmen; denn das in ihr geboren ist, das ist von dem heiligen Geist.
\par 21 Und sie wird einen Sohn gebären, des Namen sollst du Jesus heißen; denn er wird sein Volk selig machen von ihren Sünden.
\par 22 Das ist aber alles geschehen, auf daß erfüllt würde, was der HERR durch den Propheten gesagt hat, der da spricht:
\par 23 "Siehe, eine Jungfrau wird schwanger sein und einen Sohn gebären, und sie werden seinen Namen Immanuel heißen", das ist verdolmetscht: Gott mit uns.
\par 24 Da nun Joseph vom Schlaf erwachte, tat er, wie ihm des HERRN Engel befohlen hatte, und nahm sein Gemahl zu sich.
\par 25 Und er erkannte sie nicht, bis sie ihren ersten Sohn gebar; und hieß seinen Namen Jesus.

\chapter{2}

\par 1 Da Jesus geboren war zu Bethlehem im jüdischen Lande, zur Zeit des Königs Herodes, siehe, da kamen die Weisen vom Morgenland nach Jerusalem und sprachen:
\par 2 Wo ist der neugeborene König der Juden? Wir haben seinen Stern gesehen im Morgenland und sind gekommen, ihn anzubeten.
\par 3 Da das der König Herodes hörte, erschrak er und mit ihm das ganze Jerusalem.
\par 4 Und ließ versammeln alle Hohenpriester und Schriftgelehrten unter dem Volk und erforschte von ihnen, wo Christus sollte geboren werden.
\par 5 Und sie sagten ihm: Zu Bethlehem im jüdischen Lande; denn also steht geschrieben durch den Propheten:
\par 6 "Und du Bethlehem im jüdischen Lande bist mitnichten die kleinste unter den Fürsten Juda's; denn aus dir soll mir kommen der Herzog, der über mein Volk Israel ein HERR sei."
\par 7 Da berief Herodes die Weisen heimlich und erlernte mit Fleiß von ihnen, wann der Stern erschienen wäre,
\par 8 und wies sie gen Bethlehem und sprach: Ziehet hin und forschet fleißig nach dem Kindlein; wenn ihr's findet, so sagt mir's wieder, daß ich auch komme und es anbete.
\par 9 Als sie nun den König gehört hatten, zogen sie hin. Und siehe, der Stern, den sie im Morgenland gesehen hatten, ging vor ihnen hin, bis daß er kam und stand oben über, da das Kindlein war.
\par 10 Da sie den Stern sahen, wurden sie hoch erfreut
\par 11 und gingen in das Haus und fanden das Kindlein mit Maria, seiner Mutter, und fielen nieder und beteten es an und taten ihre Schätze auf und schenkten ihm Gold, Weihrauch und Myrrhe.
\par 12 Und Gott befahl ihnen im Traum, daß sie sich nicht sollten wieder zu Herodes lenken; und sie zogen durch einen anderen Weg wieder in ihr Land.
\par 13 Da sie aber hinweggezogen waren, siehe, da erschien der Engel des HERRN dem Joseph im Traum und sprach: Stehe auf und nimm das Kindlein und seine Mutter zu dir und flieh nach Ägyptenland und bleib allda, bis ich dir sage; denn es ist vorhanden, daß Herodes das Kindlein suche, dasselbe umzubringen.
\par 14 Und er stand auf und nahm das Kindlein und seine Mutter zu sich bei der Nacht und entwich nach Ägyptenland.
\par 15 Und blieb allda bis nach dem Tod des Herodes, auf daß erfüllet würde, was der HERR durch den Propheten gesagt hat, der da spricht: "Aus Ägypten habe ich meinen Sohn gerufen."
\par 16 Da Herodes nun sah, daß er von den Weisen betrogen war, ward er sehr zornig und schickte aus und ließ alle Kinder zu Bethlehem töten und an seinen ganzen Grenzen, die da zweijährig und darunter waren, nach der Zeit, die er mit Fleiß von den Weisen erlernt hatte.
\par 17 Da ist erfüllt, was gesagt ist von dem Propheten Jeremia, der da spricht:
\par 18 "Auf dem Gebirge hat man ein Geschrei gehört, viel Klagens, Weinens und Heulens; Rahel beweinte ihre Kinder und wollte sich nicht trösten lassen, denn es war aus mit ihnen."
\par 19 Da aber Herodes gestorben war, siehe, da erschien der Engel des HERRN dem Joseph im Traum in Ägyptenland
\par 20 und sprach: Stehe auf und nimm das Kindlein und seine Mutter zu dir und zieh hin in das Land Israel; sie sind gestorben, die dem Kinde nach dem Leben standen.
\par 21 Und er stand auf und nahm das Kindlein und sein Mutter zu sich und kam in das Land Israel.
\par 22 Da er aber hörte, daß Archelaus im jüdischen Lande König war anstatt seines Vaters Herodes, fürchtete er sich, dahin zu kommen. Und im Traum empfing er Befehl von Gott und zog in die Örter des galiläischen Landes.
\par 23 und kam und wohnte in der Stadt die da heißt Nazareth; auf das erfüllet würde, was da gesagt ist durch die Propheten: Er soll Nazarenus heißen.

\chapter{3}

\par 1 Zu der Zeit kam Johannes der Täufer und predigte in der Wüste des jüdischen Landes
\par 2 und sprach: Tut Buße, das Himmelreich ist nahe herbeigekommen!
\par 3 Und er ist der, von dem der Prophet Jesaja gesagt hat und gesprochen: "Es ist eine Stimme eines Predigers in der Wüste: Bereitet dem HERRN den Weg und macht richtig seine Steige!"
\par 4 Er aber, Johannes, hatte ein Kleid von Kamelhaaren und einen ledernen Gürtel um seine Lenden; seine Speise aber war Heuschrecken und wilder Honig.
\par 5 Da ging zu ihm hinaus die Stadt Jerusalem und das ganze jüdische Land und alle Länder an dem Jordan
\par 6 und ließen sich taufen von ihm im Jordan und bekannten ihre Sünden.
\par 7 Als er nun viele Pharisäer und Sadduzäer sah zu seiner Taufe kommen, sprach er zu ihnen: Ihr Otterngezüchte, wer hat denn euch gewiesen, daß ihr dem künftigen Zorn entrinnen werdet?
\par 8 Sehet zu, tut rechtschaffene Frucht der Buße!
\par 9 Denket nur nicht, daß ihr bei euch wollt sagen: Wir haben Abraham zum Vater. Ich sage euch: Gott vermag dem Abraham aus diesen Steinen Kinder zu erwecken.
\par 10 Es ist schon die Axt den Bäumen an die Wurzel gelegt. Darum, welcher Baum nicht gute Frucht bringt, wird abgehauen und ins Feuer geworfen.
\par 11 Ich taufe euch mit Wasser zur Buße; der aber nach mir kommt, ist stärker denn ich, dem ich nicht genugsam bin, seine Schuhe zu tragen; der wird euch mit dem Heiligen Geist und mit Feuer taufen.
\par 12 Und er hat seine Wurfschaufel in der Hand: er wird seine Tenne fegen und den Weizen in seine Scheune sammeln; aber die Spreu wird er verbrennen mit ewigem Feuer.
\par 13 Zu der Zeit kam Jesus aus Galiläa an den Jordan zu Johannes, daß er sich von ihm taufen ließe.
\par 14 Aber Johannes wehrte ihm und sprach: Ich bedarf wohl, daß ich von dir getauft werde, und du kommst zu mir?
\par 15 Jesus aber antwortete und sprach zu ihm: Laß es jetzt also sein! also gebührt es uns, alle Gerechtigkeit zu erfüllen. Da ließ er's ihm zu.
\par 16 Und da Jesus getauft war, stieg er alsbald herauf aus dem Wasser; und siehe, da tat sich der Himmel auf Über ihm. Und er sah den Geist Gottes gleich als eine Taube herabfahren und über ihn kommen.
\par 17 Und siehe, eine Stimme vom Himmel herab sprach: Dies ist mein lieber Sohn, an welchem ich Wohlgefallen habe.

\chapter{4}

\par 1 Da ward Jesus vom Geist in die Wüste geführt, auf daß er von dem Teufel versucht würde.
\par 2 Und da er vierzig Tage und vierzig Nächte gefastet hatte, hungerte ihn.
\par 3 Und der Versucher trat zu ihm und sprach: Bist du Gottes Sohn, so sprich, daß diese Steine Brot werden.
\par 4 Und er antwortete und sprach: Es steht geschrieben: "Der Mensch lebt nicht vom Brot allein, sondern von einem jeglichen Wort, das durch den Mund Gottes geht."
\par 5 Da führte ihn der Teufel mit sich in die Heilige Stadt und stellte ihn auf die Zinne des Tempels
\par 6 und sprach zu ihm: Bist du Gottes Sohn, so laß dich hinab; denn es steht geschrieben: Er wird seinen Engeln über dir Befehl tun, und sie werden dich auf Händen tragen, auf daß du deinen Fuß nicht an einen Stein stoßest.
\par 7 Da sprach Jesus zu ihm: Wiederum steht auch geschrieben: "Du sollst Gott, deinen HERRN, nicht versuchen."
\par 8 Wiederum führte ihn der Teufel mit sich auf einen sehr hohen Berg und zeigte ihm alle Reiche der Welt und ihre Herrlichkeit
\par 9 und sprach zu ihm: Das alles will ich dir geben, so du niederfällst und mich anbetest.
\par 10 Da sprach Jesus zu ihm: Hebe dich weg von mir Satan! denn es steht geschrieben: "Du sollst anbeten Gott, deinen HERRN, und ihm allein dienen."
\par 11 Da verließ ihn der Teufel; und siehe, da traten die Engel zu ihm und dienten ihm.
\par 12 Da nun Jesus hörte, daß Johannes überantwortet war, zog er in das galiläische Land.
\par 13 Und verließ die Stadt Nazareth, kam und wohnte zu Kapernaum, das da liegt am Meer, im Lande Sebulon und Naphthali,
\par 14 auf das erfüllet würde, was da gesagt ist durch den Propheten Jesaja, der da spricht:
\par 15 "Das Land Sebulon und das Land Naphthali, am Wege des Meeres, jenseit des Jordans, und das heidnische Galiläa,
\par 16 das Volk, das in Finsternis saß, hat ein großes Licht gesehen; und die da saßen am Ort und Schatten des Todes, denen ist ein Licht aufgegangen."
\par 17 Von der Zeit an fing Jesus an, zu predigen und zu sagen: Tut Buße, das Himmelreich ist nahe herbeigekommen!
\par 18 Als nun Jesus an dem Galiläischen Meer ging, sah er zwei Brüder, Simon, der da heißt Petrus, und Andreas, seinen Bruder, die warfen ihre Netze ins Meer; denn sie waren Fischer.
\par 19 Und er sprach zu ihnen: Folget mir nach; ich will euch zu Menschenfischern machen!
\par 20 Alsbald verließen sie ihre Netze und folgten ihm nach.
\par 21 Und da er von da weiterging, sah er zwei andere Brüder, Jakobus, den Sohn des Zebedäus, und Johannes, seinen Bruder, im Schiff mit ihrem Vater Zebedäus, daß sie ihre Netze flickten; und er rief sie.
\par 22 Alsbald verließen sie das Schiff und ihren Vater und folgten ihm nach.
\par 23 Und Jesus ging umher im ganzen galiläischen Lande, lehrte sie in ihren Schulen und predigte das Evangelium von dem Reich und heilte allerlei Seuche und Krankheit im Volk.
\par 24 Und sein Gerücht erscholl in das ganze Syrienland. Und sie brachten zu ihm allerlei Kranke, mit mancherlei Seuchen und Qual behaftet, die Besessenen, die Mondsüchtigen und Gichtbrüchigen; und er machte sie alle gesund.
\par 25 Und es folgte ihm nach viel Volks aus Galiläa, aus den Zehn-Städten, von Jerusalem, aus dem jüdischen Lande und von jenseits des Jordans.

\chapter{5}

\par 1 Da er aber das Volk sah, ging er auf einen Berg und setzte sich; und seine Jünger traten zu ihm,
\par 2 Und er tat seinen Mund auf, lehrte sie und sprach:
\par 3 Selig sind, die da geistlich arm sind; denn das Himmelreich ist ihr.
\par 4 Selig sind, die da Leid tragen; denn sie sollen getröstet werden.
\par 5 Selig sind die Sanftmütigen; denn sie werden das Erdreich besitzen.
\par 6 Selig sind, die da hungert und dürstet nach der Gerechtigkeit; denn sie sollen satt werden.
\par 7 Selig sind die Barmherzigen; denn sie werden Barmherzigkeit erlangen.
\par 8 Selig sind, die reines Herzens sind; denn sie werden Gott schauen.
\par 9 Selig sind die Friedfertigen; denn sie werden Gottes Kinder heißen.
\par 10 Selig sind, die um Gerechtigkeit willen verfolgt werden; denn das Himmelreich ist ihr.
\par 11 Selig seid ihr, wenn euch die Menschen um meinetwillen schmähen und verfolgen und reden allerlei Übles gegen euch, so sie daran lügen.
\par 12 Seid fröhlich und getrost; es wird euch im Himmel wohl belohnt werden. Denn also haben sie verfolgt die Propheten, die vor euch gewesen sind.
\par 13 Ihr seid das Salz der Erde. Wo nun das Salz dumm wird, womit soll man's salzen? Es ist hinfort zu nichts nütze, denn das man es hinausschütte und lasse es die Leute zertreten.
\par 14 Ihr seid das Licht der Welt. Es kann die Stadt, die auf einem Berge liegt, nicht verborgen sein.
\par 15 Man zündet auch nicht ein Licht an und setzt es unter einen Scheffel, sondern auf einen Leuchter; so leuchtet es denn allen, die im Hause sind.
\par 16 Also laßt euer Licht leuchten vor den Leuten, daß sie eure guten Werke sehen und euren Vater im Himmel preisen.
\par 17 Ihr sollt nicht wähnen, daß ich gekommen bin, das Gesetz oder die Propheten aufzulösen; ich bin nicht gekommen, aufzulösen, sondern zu erfüllen.
\par 18 Denn ich sage euch wahrlich: Bis daß Himmel und Erde zergehe, wird nicht zergehen der kleinste Buchstabe noch ein Tüttel vom Gesetz, bis daß es alles geschehe.
\par 19 Wer nun eines von diesen kleinsten Geboten auflöst und lehrt die Leute also, der wird der Kleinste heißen im Himmelreich; wer es aber tut und lehrt, der wird groß heißen im Himmelreich.
\par 20 Denn ich sage euch: Es sei denn eure Gerechtigkeit besser als der Schriftgelehrten und Pharisäer, so werdet ihr nicht in das Himmelreich kommen.
\par 21 Ihr habt gehört, daß zu den Alten gesagt ist: "Du sollst nicht töten; wer aber tötet, der soll des Gerichts schuldig sein."
\par 22 Ich aber sage euch: Wer mit seinem Bruder zürnet, der ist des Gerichts schuldig; wer aber zu seinem Bruder sagt: Racha! der ist des Rats schuldig; wer aber sagt: Du Narr! der ist des höllischen Feuers schuldig.
\par 23 Darum, wenn du deine Gabe auf dem Altar opferst und wirst allda eingedenk, daß dein Bruder etwas wider dich habe,
\par 24 so laß allda vor dem Altar deine Gabe und gehe zuvor hin und versöhne dich mit deinem Bruder, und alsdann komm und opfere deine Gabe.
\par 25 Sei willfährig deinem Widersacher bald, dieweil du noch bei ihm auf dem Wege bist, auf daß dich der Widersacher nicht dermaleinst überantworte dem Richter, und der Richter überantworte dich dem Diener, und wirst in den Kerker geworfen.
\par 26 Ich sage dir wahrlich: Du wirst nicht von dannen herauskommen, bis du auch den letzten Heller bezahlest.
\par 27 Ihr habt gehört, daß zu den Alten gesagt ist: "Du sollst nicht ehebrechen."
\par 28 Ich aber sage euch: Wer ein Weib ansieht, ihrer zu begehren, der hat schon mit ihr die Ehe gebrochen in seinem Herzen.
\par 29 Ärgert dich aber dein rechtes Auge, so reiß es aus und wirf's von dir. Es ist dir besser, daß eins deiner Glieder verderbe, und nicht der ganze Leib in die Hölle geworfen werde.
\par 30 Ärgert dich deine rechte Hand, so haue sie ab und wirf sie von dir. Es ist dir besser, daß eins deiner Glieder verderbe, und nicht der ganze Leib in die Hölle geworfen werde.
\par 31 Es ist auch gesagt: "Wer sich von seinem Weibe scheidet, der soll ihr geben einen Scheidebrief."
\par 32 Ich aber sage euch: Wer sich von seinem Weibe scheidet (es sei denn um Ehebruch), der macht, daß sie die Ehe bricht; und wer eine Abgeschiedene freit, der bricht die Ehe.
\par 33 Ihr habt weiter gehört, daß zu den Alten gesagt ist: "Du sollst keinen falschen Eid tun und sollst Gott deinen Eid halten."
\par 34 Ich aber sage euch, daß ihr überhaupt nicht schwören sollt, weder bei dem Himmel, denn er ist Gottes Stuhl,
\par 35 noch bei der Erde, denn sie ist seiner Füße Schemel, noch bei Jerusalem, denn sie ist des großen Königs Stadt.
\par 36 Auch sollst du nicht bei deinem Haupt schwören, denn du vermagst nicht ein einziges Haar schwarz oder weiß zu machen.
\par 37 Eure Rede aber sei: Ja, ja; nein, nein. Was darüber ist, das ist vom Übel.
\par 38 Ihr habt gehört, daß da gesagt ist: "Auge um Auge, Zahn um Zahn."
\par 39 Ich aber sage euch, daß ihr nicht widerstreben sollt dem Übel; sondern, so dir jemand einen Streich gibt auf deinen rechten Backen, dem biete den andern auch dar.
\par 40 Und so jemand mit dir rechten will und deinen Rock nehmen, dem laß auch den Mantel.
\par 41 Und so dich jemand nötigt eine Meile, so gehe mit ihm zwei.
\par 42 Gib dem, der dich bittet, und wende dich nicht von dem, der dir abborgen will.
\par 43 Ihr habt gehört, daß gesagt ist: "Du sollst deinen Nächsten lieben und deinen Feind hassen."
\par 44 Ich aber sage euch: Liebet eure Feinde; segnet, die euch fluchen; tut wohl denen, die euch hassen; bittet für die, so euch beleidigen und verfolgen,
\par 45 auf daß ihr Kinder seid eures Vater im Himmel; denn er läßt seine Sonne aufgehen über die Bösen und über die Guten und läßt regnen über Gerechte und Ungerechte.
\par 46 Denn so ihr liebet, die euch lieben, was werdet ihr für Lohn haben? Tun nicht dasselbe auch die Zöllner?
\par 47 Und so ihr euch nur zu euren Brüdern freundlich tut, was tut ihr Sonderliches? Tun nicht die Zöllner auch also?
\par 48 Darum sollt ihr vollkommen sein, gleichwie euer Vater im Himmel vollkommen ist.

\chapter{6}

\par 1 Habt acht auf eure Almosen, daß ihr die nicht gebet vor den Leuten, daß ihr von ihnen gesehen werdet; ihr habt anders keinen Lohn bei eurem Vater im Himmel.
\par 2 Wenn du Almosen gibst, sollst du nicht lassen vor dir posaunen, wie die Heuchler tun in den Schulen und auf den Gassen, auf daß sie von den Leuten gepriesen werden. Wahrlich ich sage euch: Sie haben ihren Lohn dahin.
\par 3 Wenn du aber Almosen gibst, so laß deine linke Hand nicht wissen, was die rechte tut,
\par 4 auf daß dein Almosen verborgen sei; und dein Vater, der in das Verborgene sieht, wird dir's vergelten öffentlich.
\par 5 Und wenn du betest, sollst du nicht sein wie die Heuchler, die da gerne stehen und beten in den Schulen und an den Ecken auf den Gassen, auf daß sie von den Leuten gesehen werden. Wahrlich ich sage euch: Sie haben ihren Lohn dahin.
\par 6 Wenn aber du betest, so gehe in dein Kämmerlein und schließ die Tür zu und bete zu deinem Vater im Verborgenen; und dein Vater, der in das Verborgene sieht, wird dir's vergelten öffentlich.
\par 7 Und wenn ihr betet, sollt ihr nicht viel plappern wie die Heiden; denn sie meinen, sie werden erhört, wenn sie viel Worte machen.
\par 8 Darum sollt ihr euch ihnen nicht gleichstellen. Euer Vater weiß, was ihr bedürfet, ehe ihr ihn bittet.
\par 9 Darum sollt ihr also beten: Unser Vater in dem Himmel! Dein Name werde geheiligt.
\par 10 Dein Reich komme. Dein Wille geschehe auf Erden wie im Himmel.
\par 11 Unser täglich Brot gib uns heute.
\par 12 Und vergib uns unsere Schuld, wie wir unseren Schuldigern vergeben.
\par 13 Und führe uns nicht in Versuchung, sondern erlöse uns von dem Übel. Denn dein ist das Reich und die Kraft und die Herrlichkeit in Ewigkeit. Amen.
\par 14 Denn so ihr den Menschen ihre Fehler vergebet, so wird euch euer himmlischer Vater auch vergeben,
\par 15 Wo ihr aber den Menschen ihre Fehler nicht vergebet, so wird euch euer Vater eure Fehler auch nicht vergeben.
\par 16 Wenn ihr fastet, sollt ihr nicht sauer sehen wie die Heuchler; denn sie verstellen ihr Angesicht, auf daß sie vor den Leuten scheinen mit ihrem Fasten. Wahrlich ich sage euch: Sie haben ihren Lohn dahin.
\par 17 Wenn du aber fastest, so salbe dein Haupt und wasche dein Angesicht,
\par 18 auf daß du nicht scheinest vor den Leuten mit deinem Fasten, sondern vor deinem Vater, welcher verborgen ist; und dein Vater, der in das Verborgene sieht, wird dir's vergelten öffentlich.
\par 19 Ihr sollt euch nicht Schätze sammeln auf Erden, da sie die Motten und der Rost fressen und da die Diebe nachgraben und stehlen.
\par 20 Sammelt euch aber Schätze im Himmel, da sie weder Motten noch Rost fressen und da die Diebe nicht nachgraben noch stehlen.
\par 21 Denn wo euer Schatz ist, da ist auch euer Herz.
\par 22 Das Auge ist des Leibes Licht. Wenn dein Auge einfältig ist, so wird dein ganzer Leib licht sein;
\par 23 ist aber dein Auge ein Schalk, so wird dein ganzer Leib finster sein. Wenn nun das Licht, das in dir ist, Finsternis ist, wie groß wird dann die Finsternis sein!
\par 24 Niemand kann zwei Herren dienen: entweder er wird den einen hassen und den andern lieben, oder er wird dem einen anhangen und den andern verachten. Ihr könnt nicht Gott dienen und dem Mammon.
\par 25 Darum sage ich euch: Sorget nicht für euer Leben, was ihr essen und trinken werdet, auch nicht für euren Leib, was ihr anziehen werdet. Ist nicht das Leben mehr denn Speise? und der Leib mehr denn die Kleidung?
\par 26 Sehet die Vögel unter dem Himmel an: sie säen nicht, sie ernten nicht, sie sammeln nicht in die Scheunen; und euer himmlischer Vater nährt sie doch. Seid ihr denn nicht viel mehr denn sie?
\par 27 Wer ist aber unter euch, der seiner Länge eine Elle zusetzen möge, ob er gleich darum sorget?
\par 28 Und warum sorget ihr für die Kleidung? Schaut die Lilien auf dem Felde, wie sie wachsen: sie arbeiten nicht, auch spinnen sie nicht.
\par 29 Ich sage euch, daß auch Salomo in aller seiner Herrlichkeit nicht bekleidet gewesen ist wie derselben eins.
\par 30 So denn Gott das Gras auf dem Felde also kleidet, das doch heute steht und morgen in den Ofen geworfen wird: sollte er das nicht viel mehr euch tun, o ihr Kleingläubigen?
\par 31 Darum sollt ihr nicht sorgen und sagen: Was werden wir essen, was werden wir trinken, womit werden wir uns kleiden?
\par 32 Nach solchem allem trachten die Heiden. Denn euer himmlischer Vater weiß, daß ihr des alles bedürfet.
\par 33 Trachtet am ersten nach dem Reich Gottes und nach seiner Gerechtigkeit, so wird euch solches alles zufallen.
\par 34 Darum sorgt nicht für den andern Morgen; denn der morgende Tag wird für das Seine sorgen. Es ist genug, daß ein jeglicher Tag seine eigene Plage habe.

\chapter{7}

\par 1 Richtet nicht, auf daß ihr nicht gerichtet werdet.
\par 2 Denn mit welcherlei Gericht ihr richtet, werdet ihr gerichtet werden; und mit welcherlei Maß ihr messet, wird euch gemessen werden.
\par 3 Was siehst du aber den Splitter in deines Bruders Auge, und wirst nicht gewahr des Balkens in deinem Auge?
\par 4 Oder wie darfst du sagen zu deinem Bruder: Halt, ich will dir den Splitter aus deinem Auge ziehen, und siehe, ein Balken ist in deinem Auge?
\par 5 Du Heuchler, zieh am ersten den Balken aus deinem Auge; darnach siehe zu, wie du den Splitter aus deines Bruders Auge ziehst!
\par 6 Ihr sollt das Heiligtum nicht den Hunden geben, und eure Perlen nicht vor die Säue werfen, auf daß sie dieselben nicht zertreten mit ihren Füßen und sich wenden und euch zerreißen.
\par 7 Bittet, so wird euch gegeben; suchet, so werdet ihr finden; klopfet an, so wird euch aufgetan.
\par 8 Denn wer da bittet, der empfängt; und wer da sucht, der findet; und wer da anklopft, dem wird aufgetan.
\par 9 Welcher ist unter euch Menschen, so ihn sein Sohn bittet ums Brot, der ihm einen Stein biete?
\par 10 oder, so er ihn bittet um einen Fisch, der ihm eine Schlange biete?
\par 11 So denn ihr, die ihr doch arg seid, könnt dennoch euren Kindern gute Gaben geben, wie viel mehr wird euer Vater im Himmel Gutes geben denen, die ihn bitten!
\par 12 Alles nun, was ihr wollt, daß euch die Leute tun sollen, das tut ihr ihnen auch. Das ist das Gesetz und die Propheten.
\par 13 Gehet ein durch die enge Pforte. Denn die Pforte ist weit, und der Weg ist breit, der zur Verdammnis abführt; und ihrer sind viele, die darauf wandeln.
\par 14 Und die Pforte ist eng, und der Weg ist schmal, der zum Leben führt; und wenige sind ihrer, die ihn finden.
\par 15 Seht euch vor vor den falschen Propheten, die in Schafskleidern zu euch kommen, inwendig aber sind sie reißende Wölfe.
\par 16 An ihren Früchten sollt ihr sie erkennen. Kann man auch Trauben lesen von den Dornen oder Feigen von den Disteln?
\par 17 Also ein jeglicher guter Baum bringt gute Früchte; aber ein fauler Baum bringt arge Früchte.
\par 18 Ein guter Baum kann nicht arge Früchte bringen, und ein fauler Baum kann nicht gute Früchte bringen.
\par 19 Ein jeglicher Baum, der nicht gute Früchte bringt, wird abgehauen und ins Feuer geworfen.
\par 20 Darum an ihren Früchten sollt ihr sie erkennen.
\par 21 Es werden nicht alle, die zu mir sagen: HERR, HERR! ins Himmelreich kommen, sondern die den Willen tun meines Vaters im Himmel.
\par 22 Es werden viele zu mir sagen an jenem Tage: HERR, HERR! haben wir nicht in deinem Namen geweissagt, haben wir nicht in deinem Namen Teufel ausgetrieben, und haben wir nicht in deinem Namen viele Taten getan?
\par 23 Dann werde ich ihnen bekennen: Ich habe euch noch nie erkannt; weichet alle von mir, ihr Übeltäter!
\par 24 Darum, wer diese meine Rede hört und tut sie, den vergleiche ich einem klugen Mann, der sein Haus auf einen Felsen baute.
\par 25 Da nun ein Platzregen fiel und ein Gewässer kam und wehten die Winde und stießen an das Haus, fiel es doch nicht; denn es war auf einen Felsen gegründet.
\par 26 Und wer diese meine Rede hört und tut sie nicht, der ist einem törichten Manne gleich, der sein Haus auf den Sand baute.
\par 27 Da nun ein Platzregen fiel und kam ein Gewässer und wehten die Winde und stießen an das Haus, da fiel es und tat einen großen Fall.
\par 28 Und es begab sich, da Jesus diese Rede vollendet hatte, entsetzte sich das Volk über seine Lehre.
\par 29 Denn er predigte gewaltig und nicht wie die Schriftgelehrten.

\chapter{8}

\par 1 Da er aber vom Berg herabging, folgte ihm viel Volks nach.
\par 2 Und siehe, ein Aussätziger kam und betete ihn an und sprach: HERR, so du willst, kannst du mich wohl reinigen.
\par 3 Und Jesus streckte seine Hand aus, rührte ihn an und sprach: Ich will's tun; sei gereinigt! Und alsbald ward er vom Aussatz rein.
\par 4 Und Jesus sprach zu ihm: Siehe zu, sage es niemand; sondern gehe hin und zeige dich dem Priester und opfere die Gabe, die Mose befohlen hat, zu einem Zeugnis über sie.
\par 5 Da aber Jesus einging zu Kapernaum, trat ein Hauptmann zu ihm, der bat ihn
\par 6 und sprach: HERR, mein Knecht liegt zu Hause und ist gichtbrüchig und hat große Qual.
\par 7 Jesus sprach zu ihm: Ich will kommen und ihn gesund machen.
\par 8 Der Hauptmann antwortete und sprach: HERR, ich bin nicht wert, daß du unter mein Dach gehest; sondern sprich nur ein Wort, so wird mein Knecht gesund.
\par 9 Denn ich bin ein Mensch, der Obrigkeit untertan, und habe unter mir Kriegsknechte; und wenn ich sage zu einem: Gehe hin! so geht er; und zum andern: Komm her! so kommt er; und zu meinem Knecht: Tu das! so tut er's.
\par 10 Da das Jesus hörte, verwunderte er sich und sprach zu denen, die ihm nachfolgten: Wahrlich ich sage euch: Solchen Glauben habe ich in Israel nicht gefunden!
\par 11 Aber ich sage euch viele werden kommen vom Morgen und vom Abend und mit Abraham und Isaak und Jakob im Himmelreich sitzen;
\par 12 aber die Kinder des Reiches werden ausgestoßen in die Finsternis hinaus; da wird sein Heulen und Zähneklappen.
\par 13 Und Jesus sprach zu dem Hauptmann: Gehe hin; dir geschehe, wie du geglaubt hast. Und sein Knecht ward gesund zu derselben Stunde.
\par 14 Und Jesus kam in des Petrus Haus und sah, daß seine Schwiegermutter lag und hatte das Fieber.
\par 15 Da griff er ihre Hand an, und das Fieber verließ sie. Und sie stand auf und diente ihnen.
\par 16 Am Abend aber brachten sie viele Besessene zu ihm; und er trieb die Geister aus mit Worten und machte allerlei Kranke gesund,
\par 17 auf das erfüllt würde, was gesagt ist durch den Propheten Jesaja, der da spricht: "Er hat unsere Schwachheiten auf sich genommen, und unsere Seuchen hat er getragen."
\par 18 Und da Jesus viel Volks um sich sah, hieß er hinüber jenseit des Meeres fahren.
\par 19 Und es trat zu ihm ein Schriftgelehrter, der sprach zu ihm: Meister, ich will dir folgen, wo du hin gehst.
\par 20 Jesus sagt zu ihm: Die Füchse haben Gruben, und die Vögel unter dem Himmel haben Nester; aber des Menschen Sohn hat nicht, da er sein Haupt hin lege.
\par 21 Und ein anderer unter seinen Jüngern sprach zu ihm: HERR, erlaube mir, daß hingehe und zuvor meinen Vater begrabe.
\par 22 Aber Jesus sprach zu ihm: Folge du mir und laß die Toten ihre Toten begraben!
\par 23 Und er trat in das Schiff, und seine Jünger folgten ihm.
\par 24 Und siehe, da erhob sich ein großes Ungestüm im Meer, also daß auch das Schifflein mit Wellen bedeckt ward; und er schlief.
\par 25 Und die Jünger traten zu ihm und weckten ihn auf und sprachen: HERR, hilf uns, wir verderben!
\par 26 Da sagte er zu ihnen: Ihr Kleingläubigen, warum seid ihr so furchtsam? Und stand auf und bedrohte den Wind und das Meer; da ward es ganz stille.
\par 27 Die Menschen aber verwunderten sich und sprachen: Was ist das für ein Mann, daß ihm Wind und Meer gehorsam ist?
\par 28 Und er kam jenseit des Meeres in die Gegend der Gergesener. Da liefen ihm entgegen zwei Besessene, die kamen aus den Totengräbern und waren sehr grimmig, also daß niemand diese Straße wandeln konnte.
\par 29 Und siehe, sie schrieen und sprachen: Ach Jesu, du Sohn Gottes, was haben wir mit dir zu tun? Bist du hergekommen, uns zu quälen, ehe denn es Zeit ist?
\par 30 Es war aber ferne von ihnen ein große Herde Säue auf der Weide.
\par 31 Da baten ihn die Teufel und sprachen: Willst du uns austreiben, so erlaube uns, in die Herde Säue zu fahren.
\par 32 Und er sprach: Fahret hin! Da fuhren sie aus und in die Herde Säue. Und siehe, die ganze Herde Säue stürzte sich von dem Abhang ins Meer und ersoffen im Wasser.
\par 33 Und die Hirten flohen und gingen hin in die Stadt und sagten das alles und wie es mit den Besessenen ergangen war.
\par 34 Und siehe, da ging die ganze Stadt heraus Jesu entgegen. Und da sie ihn sahen, baten sie ihn, daß er aus ihrer Gegend weichen wollte.

\chapter{9}

\par 1 Da trat er in das Schiff und fuhr wieder herüber und kam in seine Stadt.
\par 2 Und siehe, da brachten sie zu ihm einen Gichtbrüchigen, der lag auf einem Bett. Da nun Jesus ihren Glauben sah, sprach er zu dem Gichtbrüchigen: Sei getrost, mein Sohn; deine Sünden sind dir vergeben.
\par 3 Und siehe, etliche unter den Schriftgelehrten sprachen bei sich selbst: Dieser lästert Gott.
\par 4 Da aber Jesus ihre Gedanken sah, sprach er: Warum denkt ihr so arges in euren Herzen?
\par 5 Welches ist leichter: zu sagen: Dir sind deine Sünden vergeben, oder zu sagen: Stehe auf und wandle?
\par 6 Auf das ihr aber wisset, daß des Menschen Sohn Macht habe, auf Erden die Sünden zu vergeben (sprach er zu dem Gichtbrüchigen): Stehe auf, hebe dein Bett auf und gehe heim!
\par 7 Und er stand auf und ging heim.
\par 8 Da das Volk das sah, verwunderte es sich und pries Gott, der solche Macht den Menschen gegeben hat.
\par 9 Und da Jesus von dannen ging, sah er einen Menschen am Zoll sitzen, der hieß Matthäus; und er sprach zu ihm: Folge mir! Und er stand auf und folgte ihm.
\par 10 Und es begab sich, da er zu Tische saß im Hause, siehe, da kamen viele Zöllner und Sünder und saßen zu Tische mit Jesu und seinen Jüngern.
\par 11 Da das die Pharisäer sahen, sprachen sie zu seinen Jüngern: Warum isset euer Meister mit den Zöllnern und Sündern?
\par 12 Da das Jesus hörte, sprach er zu ihnen: Die Starken bedürfen des Arztes nicht, sondern die Kranken.
\par 13 Gehet aber hin und lernet, was das sei: "Ich habe Wohlgefallen an Barmherzigkeit und nicht am Opfer." Ich bin gekommen die Sünder zur Buße zu rufen, und nicht die Gerechten.
\par 14 Indes kamen die Jünger des Johannes zu ihm und sprachen: Warum fasten wir und die Pharisäer so viel, und deine Jünger fasten nicht?
\par 15 Jesus sprach zu ihnen: Wie können die Hochzeitleute Leid tragen, solange der Bräutigam bei ihnen ist? Es wird aber die Zeit kommen, daß der Bräutigam von ihnen genommen wird; alsdann werden sie fasten.
\par 16 Niemand flickt ein altes Kleid mit einem Lappen von neuem Tuch; denn der Lappen reißt doch wieder vom Kleid, und der Riß wird ärger.
\par 17 Man faßt auch nicht Most in alte Schläuche; sonst zerreißen die Schläuche und der Most wird verschüttet, und die Schläuche kommen um. Sondern man faßt Most in neue Schläuche, so werden sie beide miteinander erhalten.
\par 18 Da er solches mit ihnen redete, siehe, da kam der Obersten einer und fiel vor ihm nieder und sprach: HERR, meine Tochter ist jetzt gestorben; aber komm und lege deine Hand auf sie, so wird sie lebendig.
\par 19 Und Jesus stand auf und folgte ihm nach und seine Jünger.
\par 20 Und siehe, ein Weib, das zwölf Jahre den Blutgang gehabt, trat von hinten zu ihm und rührte seines Kleides Saum an.
\par 21 Denn sie sprach bei sich selbst: Möchte ich nur sein Kleid anrühren, so würde ich gesund.
\par 22 Da wandte sich Jesus um und sah sie und sprach: Sei getrost, meine Tochter; dein Glaube hat dir geholfen. Und das Weib ward gesund zu derselben Stunde.
\par 23 Und als er in des Obersten Haus kam und sah die Pfeifer und das Getümmel des Volks,
\par 24 sprach er zu ihnen: Weichet! denn das Mägdlein ist nicht tot, sondern es schläft. Und sie verlachten ihn.
\par 25 Als aber das Volk hinausgetrieben war, ging er hinein und ergriff es bei der Hand; da stand das Mädglein auf.
\par 26 Und dies Gerücht erscholl in dasselbe ganze Land.
\par 27 Und da Jesus von da weiterging, folgten ihm zwei Blinde nach, die schrieen und sprachen: Ach, du Sohn Davids, erbarme dich unser!
\par 28 Und da er heimkam, traten die Blinden zu ihm. Und Jesus sprach zu ihnen: Glaubt ihr, daß ich euch solches tun kann? Da sprachen sie zu ihm: HERR, ja.
\par 29 Da rührte er ihre Augen an und sprach: Euch geschehe nach eurem Glauben.
\par 30 Und ihre Augen wurden geöffnet. Und Jesus bedrohte sie und sprach: Seht zu, daß es niemand erfahre!
\par 31 Aber sie gingen aus und machten ihn ruchbar im selben ganzen Lande.
\par 32 Da nun diese waren hinausgekommen, siehe, da brachten sie zu ihm einen Menschen, der war stumm und besessen.
\par 33 Und da der Teufel war ausgetrieben, redete der Stumme. Und das Volk verwunderte sich und sprach: Solches ist noch nie in Israel gesehen worden.
\par 34 Aber die Pharisäer sprachen: Er treibt die Teufel aus durch der Teufel Obersten.
\par 35 Und Jesus ging umher in alle Städte und Märkte, lehrte in ihren Schulen und predigte das Evangelium von dem Reich und heilte allerlei Seuche und allerlei Krankheit im Volke.
\par 36 Und da er das Volk sah, jammerte ihn desselben; denn sie waren verschmachtet und zerstreut wie die Schafe, die keinen Hirten haben.
\par 37 Da sprach er zu seinen Jüngern: Die Ernte ist groß, aber wenige sind der Arbeiter.
\par 38 Darum bittet den HERRN der Ernte, daß er Arbeiter in seine Ernte sende.

\chapter{10}

\par 1 Und er rief seine zwölf Jünger zu sich und gab ihnen Macht über die unsauberen Geister, daß sie sie austrieben und heilten allerlei Seuche und allerlei Krankheit.
\par 2 Die Namen aber der zwölf Apostel sind diese: der erste Simon, genannt Petrus, und Andreas, sein Bruder; Jakobus, des Zebedäus Sohn, und Johannes, sein Bruder;
\par 3 Philippus und Bartholomäus; Thomas und Matthäus, der Zöllner; Jakobus, des Alphäus Sohn, Lebbäus, mit dem Zunamen Thaddäus;
\par 4 Simon von Kana und Judas Ischariot, welcher ihn verriet.
\par 5 Diese zwölf sandte Jesus, gebot ihnen und sprach: Gehet nicht auf der Heiden Straße und ziehet nicht in der Samariter Städte,
\par 6 sondern gehet hin zu den verlorenen Schafen aus dem Hause Israel.
\par 7 Geht aber und predigt und sprecht: Das Himmelreich ist nahe herbeigekommen.
\par 8 Macht die Kranken gesund, reinigt die Aussätzigen, weckt die Toten auf, treibt die Teufel aus. Umsonst habt ihr's empfangen, umsonst gebt es auch.
\par 9 Ihr sollt nicht Gold noch Silber noch Erz in euren Gürteln haben,
\par 10 auch keine Tasche zur Wegfahrt, auch nicht zwei Röcke, keine Schuhe, auch keinen Stecken. Denn ein Arbeiter ist seiner Speise wert.
\par 11 Wo ihr aber in eine Stadt oder einen Markt geht, da erkundigt euch, ob jemand darin sei, der es wert ist; und bei demselben bleibet, bis ihr von dannen zieht.
\par 12 Wo ihr aber in ein Haus geht, so grüßt es;
\par 13 und so es das Haus wert ist, wird euer Friede auf sie kommen. Ist es aber nicht wert, so wird sich euer Friede wieder zu euch wenden.
\par 14 Und wo euch jemand nicht annehmen wird noch eure Rede hören, so geht heraus von demselben Haus oder der Stadt und schüttelt den Staub von euren Füßen.
\par 15 Wahrlich ich sage euch: Dem Lande der Sodomer und Gomorrer wird es erträglicher gehen am Jüngsten Gericht denn solcher Stadt.
\par 16 Siehe, ich sende euch wie Schafe mitten unter die Wölfe; darum seid klug wie die Schlangen und ohne Falsch wie die Tauben.
\par 17 Hütet euch vor den Menschen; denn sie werden euch überantworten vor ihre Rathäuser und werden euch geißeln in ihren Schulen.
\par 18 Und man wird euch vor Fürsten und Könige führen um meinetwillen, zum Zeugnis über sie und über die Heiden.
\par 19 Wenn sie euch nun überantworten werden, so sorget nicht, wie oder was ihr reden sollt; denn es soll euch zu der Stunde gegeben werden, was ihr reden sollt.
\par 20 Denn ihr seid es nicht, die da reden, sondern eures Vaters Geist ist es, der durch euch redet.
\par 21 Es wird aber ein Bruder den andern zum Tod überantworten und der Vater den Sohn, und die Kinder werden sich empören wider die Eltern und ihnen zum Tode helfen.
\par 22 Und ihr müsset gehaßt werden von jedermann um meines Namens willen. Wer aber bis an das Ende beharrt, der wird selig.
\par 23 Wenn sie euch aber in einer Stadt verfolgen, so flieht in eine andere. Wahrlich ich sage euch: Ihr werdet mit den Städten Israels nicht zu Ende kommen, bis des Menschen Sohn kommt.
\par 24 Der Jünger ist nicht über seinen Meister noch der Knecht über den Herrn.
\par 25 Es ist dem Jünger genug, daß er sei wie sein Meister und der Knecht wie sein Herr. Haben sie den Hausvater Beelzebub geheißen, wie viel mehr werden sie seine Hausgenossen also heißen!
\par 26 So fürchtet euch denn nicht vor ihnen. Es ist nichts verborgen, das es nicht offenbar werde, und ist nichts heimlich, das man nicht wissen werde.
\par 27 Was ich euch sage in der Finsternis, das redet im Licht; und was ihr hört in das Ohr, das predigt auf den Dächern.
\par 28 Und fürchtet euch nicht vor denen, die den Leib töten, und die Seele nicht können töten; fürchtet euch aber vielmehr vor dem, der Leib und Seele verderben kann in der Hölle.
\par 29 Kauft man nicht zwei Sperlinge um einen Pfennig? Dennoch fällt deren keiner auf die Erde ohne euren Vater.
\par 30 Nun aber sind auch eure Haare auf dem Haupte alle gezählt.
\par 31 So fürchtet euch denn nicht; ihr seid besser als viele Sperlinge.
\par 32 Wer nun mich bekennet vor den Menschen, den will ich bekennen vor meinem himmlischen Vater.
\par 33 Wer mich aber verleugnet vor den Menschen, den will ich auch verleugnen vor meinem himmlischen Vater.
\par 34 Ihr sollt nicht wähnen, daß ich gekommen sei, Frieden zu senden auf die Erde. Ich bin nicht gekommen, Frieden zu senden, sondern das Schwert.
\par 35 Denn ich bin gekommen, den Menschen zu erregen gegen seinen Vater und die Tochter gegen ihre Mutter und die Schwiegertochter gegen ihre Schwiegermutter.
\par 36 Und des Menschen Feinde werden seine eigenen Hausgenossen sein.
\par 37 Wer Vater oder Mutter mehr liebt denn mich, der ist mein nicht wert; und wer Sohn oder Tochter mehr liebt denn mich, der ist mein nicht wert.
\par 38 Und wer nicht sein Kreuz auf sich nimmt und folgt mir nach, der ist mein nicht wert.
\par 39 Wer sein Leben findet, der wird's verlieren; und wer sein Leben verliert um meinetwillen, der wird's finden.
\par 40 Wer euch aufnimmt, der nimmt mich auf; und wer mich aufnimmt, der nimmt den auf, der mich gesandt hat.
\par 41 Wer einen Propheten aufnimmt in eines Propheten Namen, der wird eines Propheten Lohn empfangen. Wer einen Gerechten aufnimmt in eines Gerechten Namen, der wird eines Gerechten Lohn empfangen.
\par 42 Und wer dieser Geringsten einen nur mit einem Becher kalten Wassers tränkt in eines Jüngers Namen, wahrlich ich sage euch: es wird ihm nicht unbelohnt bleiben.

\chapter{11}

\par 1 Und es begab sich, da Jesus solch Gebot an seine zwölf Jünger vollendet hatte, ging er von da weiter, zu lehren und zu predigen in ihren Städten.
\par 2 Da aber Johannes im Gefängnis die Werke Christi hörte, sandte er seiner Jünger zwei
\par 3 und ließ ihm sagen: Bist du, der da kommen soll, oder sollen wir eines anderen warten?
\par 4 Jesus antwortete und sprach zu ihnen: Gehet hin und saget Johannes wieder, was ihr sehet und höret:
\par 5 die Blinden sehen und die Lahmen gehen, die Aussätzigen werden rein und die Tauben hören, die Toten stehen auf und den Armen wird das Evangelium gepredigt;
\par 6 und selig ist, der sich nicht an mir ärgert.
\par 7 Da die hingingen, fing Jesus an, zu reden zu dem Volk von Johannes: Was seid ihr hinausgegangen in die Wüste zu sehen? Wolltet ihr ein Rohr sehen, das der Wind hin und her bewegt?
\par 8 Oder was seid ihr hinausgegangen zu sehen? Wolltet ihr einen Menschen in weichen Kleidern sehen? Siehe, die da weiche Kleider tragen, sind in der Könige Häusern.
\par 9 Oder was seid ihr hinausgegangen zu sehen? Wolltet ihr einen Propheten sehen? Ja, ich sage euch, der auch mehr ist denn ein Prophet.
\par 10 Denn dieser ist's, von dem geschrieben steht: "Siehe, ich sende meinen Engel vor dir her, der deinen Weg vor dir bereiten soll."
\par 11 Wahrlich ich sage euch: Unter allen, die von Weibern geboren sind, ist nicht aufgekommen, der größer sei denn Johannes der Täufer; der aber der Kleinste ist im Himmelreich, ist größer denn er.
\par 12 Aber von den Tagen Johannes des Täufers bis hierher leidet das Himmelreich Gewalt, und die Gewalt tun, die reißen es an sich.
\par 13 Denn alle Propheten und das Gesetz haben geweissagt bis auf Johannes.
\par 14 Und (so ihr's wollt annehmen) er ist Elia, der da soll zukünftig sein.
\par 15 Wer Ohren hat, zu hören, der höre!
\par 16 Wem soll ich aber dies Geschlecht vergleichen? Es ist den Kindlein gleich, die an dem Markt sitzen und rufen gegen ihre Gesellen
\par 17 und sprechen: Wir haben euch gepfiffen, und ihr wolltet nicht tanzen; wir haben euch geklagt, und ihr wolltet nicht weinen.
\par 18 Johannes ist gekommen, aß nicht und trank nicht; so sagen sie: Er hat den Teufel.
\par 19 Des Menschen Sohn ist gekommen, ißt und trinkt; so sagen sie: Siehe, wie ist der Mensch ein Fresser und ein Weinsäufer, der Zöllner und der Sünder Geselle! Und die Weisheit muß sich rechtfertigen lassen von ihren Kindern.
\par 20 Da fing er an, die Städte zu schelten, in welchen am meisten seiner Taten geschehen waren, und hatten sich doch nicht gebessert:
\par 21 Wehe dir Chorazin! Weh dir, Bethsaida! Wären solche Taten zu Tyrus und Sidon geschehen, wie bei euch geschehen sind, sie hätten vorzeiten im Sack und in der Asche Buße getan.
\par 22 Doch ich sage euch: Es wird Tyrus und Sidon erträglicher gehen am Jüngsten Gericht als euch.
\par 23 Und du, Kapernaum, die du bist erhoben bis an den Himmel, du wirst bis in die Hölle hinuntergestoßen werden. Denn so zu Sodom die Taten geschehen wären, die bei euch geschehen sind, sie stände noch heutigestages.
\par 24 Doch ich sage euch, es wird dem Sodomer Lande erträglicher gehen am Jüngsten Gericht als dir.
\par 25 Zu der Zeit antwortete Jesus und sprach: Ich preise dich, Vater und HERR Himmels und der Erde, daß du solches den Weisen und Klugen verborgen hast und hast es den Unmündigen offenbart.
\par 26 Ja, Vater; denn es ist also wohlgefällig gewesen vor dir.
\par 27 Alle Dinge sind mir übergeben von meinem Vater. Und niemand kennet den Sohn denn nur der Vater; und niemand kennet den Vater denn nur der Sohn und wem es der Sohn will offenbaren.
\par 28 Kommet her zu mir alle, die ihr mühselig und beladen seid; ich will euch erquicken.
\par 29 Nehmet auf euch mein Joch und lernet von mir; denn ich bin sanftmütig und von Herzen demütig; so werdet ihr Ruhe finden für eure Seelen.
\par 30 Denn mein Joch ist sanft, und meine Last ist leicht.

\chapter{12}

\par 1 Zu der Zeit ging Jesus durch die Saat am Sabbat; und seine Jünger waren hungrig, fingen an, Ähren auszuraufen, und aßen.
\par 2 Da das die Pharisäer sahen, sprachen sie zu ihm: Siehe, deine Jünger tun, was sich nicht ziemt am Sabbat zu tun.
\par 3 Er aber sprach zu ihnen: Habt ihr nicht gelesen, was David tat, da ihn und die mit ihm waren, hungerte?
\par 4 wie er in das Gotteshaus ging und aß die Schaubrote, die ihm doch nicht ziemte zu essen noch denen, die mit ihm waren, sondern allein den Priestern?
\par 5 Oder habt ihr nicht gelesen im Gesetz, wie die Priester am Sabbat im Tempel den Sabbat brechen und sind doch ohne Schuld?
\par 6 Ich sage aber euch, daß hier der ist, der auch größer ist denn der Tempel.
\par 7 Wenn ihr aber wüßtet, was das sei: "Ich habe Wohlgefallen an der Barmherzigkeit und nicht am Opfer", hättet ihr die Unschuldigen nicht verdammt.
\par 8 Des Menschen Sohn ist ein HERR auch über den Sabbat.
\par 9 Und er ging von da weiter und kam in ihre Schule.
\par 10 Und siehe, da war ein Mensch, der hatte eine verdorrte Hand. Und sie fragten ihn und sprachen: Ist's auch recht, am Sabbat heilen? auf daß sie eine Sache gegen ihn hätten.
\par 11 Aber er sprach zu ihnen: Wer ist unter euch, so er ein Schaf hat, das ihm am Sabbat in eine Grube fällt, der es nicht ergreife und aufhebe?
\par 12 Wie viel besser ist nun ein Mensch denn ein Schaf! Darum mag man wohl am Sabbat Gutes tun.
\par 13 Da sprach er zu dem Menschen: Strecke deine Hand aus! Und er streckte sie aus; und sie ward ihm wieder gesund gleichwie die andere.
\par 14 Da gingen die Pharisäer hinaus und hielten einen Rat über ihn, wie sie ihn umbrächten.
\par 15 Aber da Jesus das erfuhr, wich er von dannen. Und ihm folgte viel Volks nach, und er heilte sie alle
\par 16 und bedrohte sie, daß sie ihn nicht meldeten,
\par 17 auf das erfüllet würde, was gesagt ist von dem Propheten Jesaja, der da spricht:
\par 18 "Siehe, das ist mein Knecht, den ich erwählt habe, und mein Liebster, an dem meine Seele Wohlgefallen hat; Ich will meinen Geist auf ihn legen, und er soll den Heiden das Gericht verkünden.
\par 19 Er wird nicht zanken noch schreien, und man wird sein Geschrei nicht hören auf den Gassen;
\par 20 das zerstoßene Rohr wird er nicht zerbrechen, und den glimmenden Docht wird er nicht auslöschen, bis daß er ausführe das Gericht zum Sieg;
\par 21 und die Heiden werden auf seinen Namen hoffen."
\par 22 Da ward ein Besessener zu ihm gebracht, der ward blind und stumm; und er heilte ihn, also daß der Blinde und Stumme redete und sah.
\par 23 Und alles Volk entsetzte sich und sprach: Ist dieser nicht Davids Sohn?
\par 24 Aber die Pharisäer, da sie es hörten, sprachen sie: Er treibt die Teufel nicht anders aus denn durch Beelzebub, der Teufel Obersten.
\par 25 Jesus kannte aber ihre Gedanken und sprach zu ihnen: Ein jegliches Reich, so es mit sich selbst uneins wird, das wird wüst; und eine jegliche Stadt oder Haus, so es mit sich selbst uneins wird, kann's nicht bestehen.
\par 26 So denn ein Satan den andern austreibt, so muß er mit sich selbst uneins sein; wie kann denn sein Reich bestehen?
\par 27 So ich aber die Teufel durch Beelzebub austreibe, durch wen treiben sie eure Kinder aus? Darum werden sie eure Richter sein.
\par 28 So ich aber die Teufel durch den Geist Gottes austreibe, so ist ja das Reich Gottes zu euch gekommen.
\par 29 Oder wie kann jemand in eines Starken Haus gehen und ihm seinen Hausrat rauben, es sei denn, daß er zuvor den Starken binde und alsdann ihm sein Haus beraube?
\par 30 Wer nicht mit mir ist, der ist wider mich; und wer nicht mit mir sammelt, der zerstreut.
\par 31 Darum sage ich euch: Alle Sünde und Lästerung wird den Menschen vergeben; aber die Lästerung wider den Geist wird den Menschen nicht vergeben.
\par 32 Und wer etwas redet wider des Menschen Sohn, dem wird es vergeben; aber wer etwas redet wider den Heiligen Geist, dem wird's nicht vergeben, weder in dieser noch in jener Welt.
\par 33 Setzt entweder einen guten Baum, so wird die Frucht gut; oder setzt einen faulen Baum, so wird die Frucht faul. Denn an der Frucht erkennt man den Baum.
\par 34 Ihr Otterngezüchte, wie könnt ihr Gutes reden, dieweil ihr böse seid? Wes das Herz voll ist, des geht der Mund über.
\par 35 Ein guter Mensch bringt Gutes hervor aus seinem guten Schatz des Herzens; und ein böser Mensch bringt Böses hervor aus seinem bösen Schatz.
\par 36 Ich sage euch aber, daß die Menschen müssen Rechenschaft geben am Jüngsten Gericht von einem jeglichen unnützen Wort, das sie geredet haben.
\par 37 Aus deinen Worten wirst du gerechtfertigt werden, und aus deinen Worten wirst du verdammt werden.
\par 38 Da antworteten etliche unter den Schriftgelehrten und Pharisäern und sprachen: Meister, wir wollten gern ein Zeichen von dir sehen.
\par 39 Und er antwortete und sprach zu ihnen: Die böse und ehebrecherische Art sucht ein Zeichen; und es wird ihr kein Zeichen gegeben werden denn das Zeichen des Propheten Jona.
\par 40 Denn gleichwie Jona war drei Tage und drei Nächte in des Walfisches Bauch, also wird des Menschen Sohn drei Tage und drei Nächte mitten in der Erde sein.
\par 41 Die Leute von Ninive werden auftreten am Jüngsten Gericht mit diesem Geschlecht und werden es verdammen; denn sie taten Buße nach der Predigt des Jona. Und siehe, hier ist mehr denn Jona.
\par 42 Die Königin von Mittag wird auftreten am Jüngsten Gericht mit diesem Geschlecht und wird es verdammen; denn sie kam vom Ende der Erde, Salomons Weisheit zu hören. Und siehe, hier ist mehr denn Salomo.
\par 43 Wenn der unsaubere Geist von dem Menschen ausgefahren ist, so durchwandelt er dürre Stätten, sucht Ruhe, und findet sie nicht.
\par 44 Da spricht er denn: Ich will wieder umkehren in mein Haus, daraus ich gegangen bin. Und wenn er kommt, so findet er's leer, gekehrt und geschmückt.
\par 45 So geht er hin und nimmt zu sich sieben andere Geister, die ärger sind denn er selbst; und wenn sie hineinkommen, wohnen sie allda; und es wird mit demselben Menschen hernach ärger, denn es zuvor war. Also wird's auch diesem argen Geschlecht gehen.
\par 46 Da er noch also zu dem Volk redete, siehe, da standen seine Mutter und seine Brüder draußen, die wollten mit ihm reden.
\par 47 Da sprach einer zu ihm: Siehe, deine Mutter und deine Brüder stehen draußen und wollen mit dir reden.
\par 48 Er antwortete aber und sprach zu dem, der es ihm ansagte: Wer ist meine Mutter, und wer sind meine Brüder?
\par 49 Und er reckte die Hand aus über seine Jünger und sprach: Siehe da, das ist meine Mutter und meine Brüder!
\par 50 Denn wer den Willen tut meines Vaters im Himmel, der ist mein Bruder, Schwester und Mutter.

\chapter{13}

\par 1 An demselben Tage ging Jesus aus dem Hause und setzte sich an das Meer.
\par 2 Und es versammelte sich viel Volks zu ihm, also daß er in das Schiff trat und saß, und alles Volk stand am Ufer.
\par 3 Und er redete zu ihnen mancherlei durch Gleichnisse und sprach: Siehe, es ging ein Säemann aus, zu säen.
\par 4 Und indem er säte, fiel etliches an den Weg; da kamen die Vögel und fraßen's auf.
\par 5 Etliches fiel in das Steinige, wo es nicht viel Erde hatte; und ging bald auf, darum daß es nicht tiefe Erde hatte.
\par 6 Als aber die Sonne aufging, verwelkte es, und dieweil es nicht Wurzel hatte, ward es dürre.
\par 7 Etliches fiel unter die Dornen; und die Dornen wuchsen auf und erstickten's.
\par 8 Etliches fiel auf gutes Land und trug Frucht, etliches hundertfältig, etliches sechzigfältig, etliches dreißigfältig.
\par 9 Wer Ohren hat zu hören, der höre!
\par 10 Und die Jünger traten zu ihm und sprachen: Warum redest du zu ihnen durch Gleichnisse?
\par 11 Er antwortete und sprach: Euch ist es gegeben, daß ihr das Geheimnis des Himmelreichs verstehet; diesen aber ist es nicht gegeben.
\par 12 Denn wer da hat, dem wird gegeben, daß er die Fülle habe; wer aber nicht hat, von dem wird auch das genommen was er hat.
\par 13 Darum rede ich zu ihnen durch Gleichnisse. Denn mit sehenden Augen sehen sie nicht, und mit hörenden Ohren hören sie nicht; denn sie verstehen es nicht.
\par 14 Und über ihnen wird die Weissagung Jesaja's erfüllt, die da sagt: "Mit den Ohren werdet ihr hören, und werdet es nicht verstehen; und mit sehenden Augen werdet ihr sehen, und werdet es nicht verstehen.
\par 15 Denn dieses Volkes Herz ist verstockt, und ihre Ohren hören übel, und ihre Augen schlummern, auf daß sie nicht dermaleinst mit den Augen sehen und mit den Ohren hören und mit dem Herzen verstehen und sich bekehren, daß ich ihnen hülfe."
\par 16 Aber selig sind eure Augen, daß sie sehen, und eure Ohren, daß sie hören.
\par 17 Wahrlich ich sage euch: Viele Propheten und Gerechte haben begehrt zu sehen, was ihr sehet, und haben's nicht gesehen, und zu hören, was ihr höret, und haben's nicht gehört.
\par 18 So hört nun ihr dieses Gleichnis von dem Säemann:
\par 19 Wenn jemand das Wort von dem Reich hört und nicht versteht, so kommt der Arge und reißt hinweg, was da gesät ist in sein Herz; und das ist der, bei welchem an dem Wege gesät ist.
\par 20 Das aber auf das Steinige gesät ist, das ist, wenn jemand das Wort hört und es alsbald aufnimmt mit Freuden;
\par 21 aber er hat nicht Wurzel in sich, sondern ist wetterwendisch; wenn sich Trübsal und Verfolgung erhebt um des Wortes willen, so ärgert er sich alsbald.
\par 22 Das aber unter die Dornen gesät ist, das ist, wenn jemand das Wort hört, und die Sorge dieser Welt und der Betrug des Reichtums erstickt das Wort, und er bringt nicht Frucht.
\par 23 Das aber in das gute Land gesät ist, das ist, wenn jemand das Wort hört und versteht es und dann auch Frucht bringt; und etlicher trägt hundertfältig, etlicher aber sechzigfältig, etlicher dreißigfältig.
\par 24 Er legte ihnen ein anderes Gleichnis vor und sprach: Das Himmelreich ist gleich einem Menschen, der guten Samen auf seinen Acker säte.
\par 25 Da aber die Leute schliefen, kam sein Feind und säte Unkraut zwischen den Weizen und ging davon.
\par 26 Da nun das Kraut wuchs und Frucht brachte, da fand sich auch das Unkraut.
\par 27 Da traten die Knechte zu dem Hausvater und sprachen: Herr, hast du nicht guten Samen auf deinen Acker gesät? Woher hat er denn das Unkraut?
\par 28 Er sprach zu ihnen: Das hat der Feind getan. Da sagten die Knechte: Willst du das wir hingehen und es ausjäten?
\par 29 Er sprach: Nein! auf daß ihr nicht zugleich den Weizen mit ausraufet, so ihr das Unkraut ausjätet.
\par 30 Lasset beides miteinander wachsen bis zur Ernte; und um der Ernte Zeit will ich zu den Schnittern sagen: Sammelt zuvor das Unkraut und bindet es in Bündlein, daß man es verbrenne; aber den Weizen sammelt mir in meine Scheuer.
\par 31 Ein anderes Gleichnis legte er ihnen vor und sprach: Das Himmelreich ist gleich einem Senfkorn, das ein Mensch nahm und säte es auf seinen Acker;
\par 32 welches ist das kleinste unter allem Samen; wenn er erwächst, so ist es das größte unter dem Kohl und wird ein Baum, daß die Vögel unter dem Himmel kommen und wohnen unter seinen Zweigen.
\par 33 Ein anderes Gleichnis redete er zu ihnen: Das Himmelreich ist gleich einem Sauerteig, den ein Weib nahm und unter drei Scheffel Mehl vermengte, bis es ganz durchsäuert ward.
\par 34 Solches alles redete Jesus durch Gleichnisse zu dem Volk, und ohne Gleichnis redete er nicht zu ihnen,
\par 35 auf das erfüllet würde, was gesagt ist durch den Propheten, der da spricht: Ich will meinen Mund auftun in Gleichnissen und will aussprechen die Heimlichkeiten von Anfang der Welt.
\par 36 Da ließ Jesus das Volk von sich und kam heim. Und seine Jünger traten zu ihm und sprachen: Deute uns das Geheimnis vom Unkraut auf dem Acker.
\par 37 Er antwortete und sprach zu ihnen: Des Menschen Sohn ist's, der da Guten Samen sät.
\par 38 Der Acker ist die Welt. Der gute Same sind die Kinder des Reiches. Das Unkraut sind die Kinder der Bosheit.
\par 39 Der Feind, der sie sät, ist der Teufel. Die Ernte ist das Ende der Welt. Die Schnitter sind die Engel.
\par 40 Gleichwie man nun das Unkraut ausjätet und mit Feuer verbrennt, so wird's auch am Ende dieser Welt gehen:
\par 41 des Menschen Sohn wird seine Engel senden; und sie werden sammeln aus seinem Reich alle Ärgernisse und die da unrecht tun,
\par 42 und werden sie in den Feuerofen werfen; da wird sein Heulen und Zähneklappen.
\par 43 Dann werden die Gerechten leuchten wie die Sonne in ihres Vaters Reich. Wer Ohren hat zu hören, der höre!
\par 44 Abermals ist gleich das Himmelreich einem verborgenem Schatz im Acker, welchen ein Mensch fand und verbarg ihn und ging hin vor Freuden über denselben und verkaufte alles, was er hatte, und kaufte den Acker.
\par 45 Abermals ist gleich das Himmelreich einem Kaufmann, der gute Perlen suchte.
\par 46 Und da er eine köstliche Perle fand, ging er hin und verkaufte alles, was er hatte, und kaufte sie.
\par 47 Abermals ist gleich das Himmelreich einem Netze, das ins Meer geworfen ist, womit man allerlei Gattung fängt.
\par 48 Wenn es aber voll ist, so ziehen sie es heraus an das Ufer, sitzen und lesen die guten in ein Gefäß zusammen; aber die faulen werfen sie weg.
\par 49 Also wird es auch am Ende der Welt gehen: die Engel werden ausgehen und die Bösen von den Gerechten scheiden
\par 50 und werden sie in den Feuerofen werfen; da wird Heulen und Zähneklappen sein.
\par 51 Und Jesus sprach zu ihnen: Habt ihr das alles verstanden? Sie sprachen: Ja, HERR.
\par 52 Da sprach er: Darum ein jeglicher Schriftgelehrter, zum Himmelreich gelehrt, ist gleich einem Hausvater, der aus seinem Schatz Neues und Altes hervorträgt.
\par 53 Und es begab sich, da Jesus diese Gleichnisse vollendet hatte, ging er von dannen
\par 54 und kam in seine Vaterstadt und lehrte sie in ihrer Schule, also auch, daß sie sich entsetzten und sprachen: Woher kommt diesem solche Weisheit und Taten?
\par 55 Ist er nicht eines Zimmermann's Sohn? Heißt nicht seine Mutter Maria? und seine Brüder Jakob und Joses und Simon und Judas?
\par 56 Und seine Schwestern, sind sie nicht alle bei uns? Woher kommt ihm denn das alles?
\par 57 Und sie ärgerten sich an ihm. Jesus aber sprach zu ihnen: Ein Prophet gilt nirgend weniger denn in seinem Vaterland und in seinem Hause.
\par 58 Und er tat daselbst nicht viel Zeichen um ihres Unglaubens willen.

\chapter{14}

\par 1 Zu der Zeit kam das Gerücht von Jesu vor den Vierfürsten Herodes.
\par 2 Und er sprach zu seinen Knechten: Dieser ist Johannes der Täufer; er ist von den Toten auferstanden, darum tut er solche Taten.
\par 3 Denn Herodes hatte Johannes gegriffen und in das Gefängnis gelegt wegen der Herodias, seines Bruders Philippus Weib.
\par 4 Denn Johannes hatte zu ihm gesagt: Es ist nicht recht, daß du sie habest.
\par 5 Und er hätte ihn gern getötet, fürchtete sich aber vor dem Volk; denn sie hielten ihn für einen Propheten.
\par 6 Da aber Herodes seinen Jahrestag beging, da tanzte die Tochter der Herodias vor ihnen. Das gefiel Herodes wohl.
\par 7 Darum verhieß er ihr mit einem Eide, er wollte ihr geben, was sie fordern würde.
\par 8 Und wie sie zuvor von ihrer Mutter angestiftet war, sprach sie: Gib mir her auf einer Schüssel das Haupt Johannes des Täufers!
\par 9 Und der König ward traurig; doch um des Eides willen und derer, die mit ihm zu Tische saßen, befahl er's ihr zu geben.
\par 10 Und schickte hin und enthauptete Johannes im Gefängnis.
\par 11 Und sein Haupt ward hergetragen in einer Schüssel und dem Mägdlein gegeben; und sie brachte es ihrer Mutter.
\par 12 Da kamen seine Jünger und nahmen seinen Leib und begruben ihn; und kamen und verkündigten das Jesus.
\par 13 Da das Jesus hörte, wich er von dannen auf einem Schiff in eine Wüste allein. Und da das Volk das hörte, folgte es ihm nach zu Fuß aus den Städten.
\par 14 Und Jesus ging hervor und sah das große Volk; und es jammerte ihn derselben, und er heilte ihre Kranken.
\par 15 Am Abend aber traten seine Jünger zu ihm und sprachen: Dies ist eine Wüste, und die Nacht fällt herein; Laß das Volk von dir, daß sie hin in die Märkte gehen und sich Speise kaufen.
\par 16 Aber Jesus sprach zu ihnen: Es ist nicht not, daß sie hingehen; gebt ihr ihnen zu essen.
\par 17 Sie sprachen: Wir haben hier nichts denn fünf Brote und zwei Fische.
\par 18 Und er sprach: Bringet sie mir her.
\par 19 Und er hieß das Volk sich lagern auf das Gras und nahm die fünf Brote und die zwei Fische, sah auf zum Himmel und dankte und brach's und gab die Brote den Jüngern, und die Jünger gaben sie dem Volk.
\par 20 Und sie aßen alle und wurden satt und hoben auf, was übrigblieb von Brocken, zwölf Körbe voll.
\par 21 Die aber gegessen hatten waren, waren bei fünftausend Mann, ohne Weiber und Kinder.
\par 22 Und alsbald trieb Jesus seine Jünger, daß sie in das Schiff traten und vor ihm herüberfuhren, bis er das Volk von sich ließe.
\par 23 Und da er das Volk von sich gelassen hatte, stieg er auf einen Berg allein, daß er betete. Und am Abend war er allein daselbst.
\par 24 Und das Schiff war schon mitten auf dem Meer und litt Not von den Wellen; denn der Wind war ihnen zuwider.
\par 25 Aber in der vierten Nachtwache kam Jesus zu ihnen und ging auf dem Meer.
\par 26 Und da ihn die Jünger sahen auf dem Meer gehen, erschraken sie und sprachen: Es ist ein Gespenst! und schrieen vor Furcht.
\par 27 Aber alsbald redete Jesus mit ihnen und sprach: Seid getrost, Ich bin's; fürchtet euch nicht!
\par 28 Petrus aber antwortete ihm und sprach: HERR, bist du es, so heiß mich zu dir kommen auf dem Wasser.
\par 29 Und er sprach: Komm her! Und Petrus trat aus dem Schiff und ging auf dem Wasser, daß er zu Jesu käme.
\par 30 Er sah aber einen starken Wind; da erschrak er und hob an zu sinken, schrie und sprach: HERR, hilf mir!
\par 31 Jesus reckte alsbald die Hand aus und ergriff ihn und sprach zu ihm: O du Kleingläubiger, warum zweifeltest du?
\par 32 Und sie traten in das Schiff, und der Wind legte sich.
\par 33 Die aber im Schiff waren, kamen und fielen vor ihm nieder und sprachen: Du bist wahrlich Gottes Sohn!
\par 34 Und sie schifften hinüber und kamen in das Land Genezareth.
\par 35 Und da die Leute am selbigen Ort sein gewahr wurden, schickten sie aus in das ganze Land umher und brachten allerlei Ungesunde zu ihm
\par 36 und baten ihn, daß sie nur seines Kleides Saum anrührten. Und alle, die ihn anrührten, wurden gesund.

\chapter{15}

\par 1 Da kamen zu ihm die Schriftgelehrten und Pharisäer von Jerusalem und sprachen:
\par 2 Warum übertreten deine Jünger der Ältesten Aufsätze? Sie waschen ihre Hände nicht, wenn sie Brot essen.
\par 3 Er antwortete und sprach zu ihnen: Warum übertretet denn ihr Gottes Gebot um eurer Aufsätze willen?
\par 4 Gott hat geboten: "Du sollst Vater und Mutter ehren; wer Vater und Mutter flucht, der soll des Todes sterben."
\par 5 Ihr aber lehret: Wer zum Vater oder Mutter spricht: "Es ist Gott gegeben, was dir sollte von mir zu Nutz kommen", der tut wohl.
\par 6 Damit geschieht es, daß niemand hinfort seinen Vater oder seine Mutter ehrt, und also habt ihr Gottes Gebot aufgehoben um eurer Aufsätze willen.
\par 7 Ihr Heuchler, wohl fein hat Jesaja von euch geweissagt und gesprochen:
\par 8 "Dies Volk naht sich zu mir mit seinem Munde und ehrt mich mit seinen Lippen, aber ihr Herz ist fern von mir;
\par 9 aber vergeblich dienen sie mir, dieweil sie lehren solche Lehren, die nichts denn Menschengebote sind."
\par 10 Und er rief das Volk zu sich und sprach zu ihm: Höret zu und fasset es!
\par 11 Was zum Munde eingeht, das verunreinigt den Menschen nicht; sondern was zum Munde ausgeht, das verunreinigt den Menschen.
\par 12 Da traten seine Jünger zu ihm und sprachen: Weißt du auch, daß sich die Pharisäer ärgerten, da sie das Wort hörten?
\par 13 Aber er antwortete und sprach: Alle Pflanzen, die mein himmlischer Vater nicht pflanzte, die werden ausgereutet.
\par 14 Lasset sie fahren! Sie sind blinde Blindenleiter. Wenn aber ein Blinder den andern leitet, so fallen sie beide in die Grube.
\par 15 Da antwortete Petrus und sprach zu ihm: Deute uns dieses Gleichnis.
\par 16 Und Jesus sprach zu ihnen: Seid ihr denn auch noch unverständig?
\par 17 Merket ihr noch nicht, daß alles, was zum Munde eingeht, das geht in den Bauch und wird durch den natürlichen Gang ausgeworfen?
\par 18 Was aber zum Munde herausgeht, das kommt aus dem Herzen, und das verunreinigt den Menschen.
\par 19 Denn aus dem Herzen kommen arge Gedanken: Mord, Ehebruch, Hurerei, Dieberei, falsch Zeugnis, Lästerung.
\par 20 Das sind Stücke, die den Menschen verunreinigen. Aber mit ungewaschenen Händen essen verunreinigt den Menschen nicht.
\par 21 Und Jesus ging aus von dannen und entwich in die Gegend von Tyrus und Sidon.
\par 22 Und siehe, ein kanaanäisches Weib kam aus derselben Gegend und schrie ihm nach und sprach: Ach HERR, du Sohn Davids, erbarme dich mein! Meine Tochter wird vom Teufel übel geplagt.
\par 23 Und er antwortete ihr kein Wort. Da traten zu ihm seine Jünger, baten ihn und sprachen: Laß sie doch von dir, denn sie schreit uns nach.
\par 24 Er antwortete aber und sprach: Ich bin nicht gesandt denn nur zu den verlorenen Schafen von dem Hause Israel.
\par 25 Sie kam aber und fiel vor ihm nieder und sprach: HERR, hilf mir!
\par 26 Aber er antwortete und sprach: Es ist nicht fein, daß man den Kindern ihr Brot nehme und werfe es vor die Hunde.
\par 27 Sie sprach: Ja, HERR; aber doch essen die Hündlein von den Brosamlein, die von ihrer Herren Tisch fallen.
\par 28 Da antwortete Jesus und sprach zu ihr: O Weib, dein Glaube ist groß! Dir geschehe, wie du willst. Und ihre Tochter ward gesund zu derselben Stunde.
\par 29 Und Jesus ging von da weiter und kam an das Galiläische Meer und ging auf einen Berg und setzte sich allda.
\par 30 Und es kam zu ihm viel Volks, die hatten mit sich Lahme, Blinde, Stumme, Krüppel und viele andere und warfen sie Jesu vor die Füße, und er heilte sie,
\par 31 daß sich das Volk verwunderte, da sie sahen, daß die Stummen redeten, die Krüppel gesund waren, die Lahmen gingen, die Blinden sahen; und sie priesen den Gott Israels.
\par 32 Und Jesus rief seine Jünger zu sich und sprach: Es jammert mich des Volks; denn sie beharren nun wohl drei Tage bei mir und haben nichts zu essen; und ich will sie nicht ungegessen von mir lassen, auf daß sie nicht verschmachten auf dem Wege.
\par 33 Da sprachen seine Jünger zu ihm: Woher mögen wir so viel Brot nehmen in der Wüste, daß wir so viel Volks sättigen?
\par 34 Und Jesus sprach zu ihnen: Wie viel Brote habt ihr? Sie sprachen: Sieben und ein wenig Fischlein.
\par 35 Und er hieß das Volk sich lagern auf die Erde
\par 36 und nahm die sieben Brote und die Fische, dankte, brach sie und gab sie seinen Jüngern; und die Jünger gaben sie dem Volk.
\par 37 Und sie aßen alle und wurden satt; und hoben auf, was übrig blieb von Brocken, sieben Körbe voll.
\par 38 Und die da gegessen hatten, derer waren viertausend Mann, ausgenommen Weiber und Kinder.
\par 39 Und da er das Volk hatte von sich gelassen, trat er in ein Schiff und kam in das Gebiet Magdalas.

\chapter{16}

\par 1 Da traten die Pharisäer und Sadduzäer zu ihm; die versuchten ihn und forderten, daß er sie ein Zeichen vom Himmel sehen ließe.
\par 2 Aber er antwortete und sprach: Des Abends sprecht ihr: Es wird ein schöner Tag werden, denn der Himmel ist rot;
\par 3 und des Morgens sprecht ihr: Es wird heute Ungewitter sein, denn der Himmel ist rot und trübe. Ihr Heuchler! über des Himmels Gestalt könnt ihr urteilen; könnt ihr denn nicht auch über die Zeichen dieser Zeit urteilen?
\par 4 Diese böse und ehebrecherische Art sucht ein Zeichen; und soll ihr kein Zeichen gegeben werden denn das Zeichen des Propheten Jona. Und er ließ sie und ging davon.
\par 5 Und da seine Jünger waren hinübergefahren, hatten sie vergessen, Brot mit sich zu nehmen.
\par 6 Jesus aber sprach zu ihnen: Sehet zu und hütet euch vor dem Sauerteig der Pharisäer und Sadduzäer!
\par 7 Da dachten sie bei sich selbst und sprachen: Das wird's sein, daß wir nicht haben Brot mit uns genommen.
\par 8 Da das Jesus merkte, sprach er zu ihnen: Ihr Kleingläubigen, was bekümmert ihr euch doch, daß ihr nicht habt Brot mit euch genommen?
\par 9 Vernehmet ihr noch nichts? Gedenket ihr nicht an die fünf Brote unter die fünftausend und wie viel Körbe ihr da aufhobt?
\par 10 auch nicht an die sieben Brote unter die viertausend und wie viel Körbe ihr da aufhobt?
\par 11 Wie, versteht ihr denn nicht, daß ich euch nicht sage vom Brot, wenn ich sage: Hütet euch vor dem Sauerteig der Pharisäer und Sadduzäer!
\par 12 Da verstanden sie, daß er nicht gesagt hatte, daß sie sich hüten sollten vor dem Sauerteig des Brots, sondern vor der Lehre der Pharisäer und Sadduzäer.
\par 13 Da kam Jesus in die Gegend der Stadt Cäsarea Philippi und fragte seine Jünger und sprach: Wer sagen die Leute, daß des Menschen Sohn sei?
\par 14 Sie sprachen: Etliche sagen, du seist Johannes der Täufer; die andern, du seist Elia; etliche du seist Jeremia oder der Propheten einer.
\par 15 Er sprach zu ihnen: Wer sagt denn ihr, daß ich sei?
\par 16 Da antwortete Simon Petrus und sprach: Du bist Christus, des lebendigen Gottes Sohn!
\par 17 Und Jesus antwortete und sprach zu ihm: Selig bist du, Simon, Jona's Sohn; denn Fleisch und Blut hat dir das nicht offenbart, sondern mein Vater im Himmel.
\par 18 Und ich sage dir auch: Du bist Petrus, und auf diesen Felsen will ich bauen meine Gemeinde, und die Pforten der Hölle sollen sie nicht überwältigen.
\par 19 Und ich will dir des Himmelsreichs Schlüssel geben: alles, was du auf Erden binden wirst, soll auch im Himmel gebunden sein, und alles, was du auf Erden lösen wirst, soll auch im Himmel los sein.
\par 20 Da verbot er seinen Jüngern, daß sie niemand sagen sollten, daß er, Jesus, der Christus wäre.
\par 21 Von der Zeit an fing Jesus an und zeigte seinen Jüngern, wie er müßte hin gen Jerusalem gehen und viel leiden von den Ältesten und Hohenpriestern und Schriftgelehrten und getötet werden und am dritten Tage auferstehen.
\par 22 Und Petrus nahm ihn zu sich, fuhr ihn an und sprach: HERR, schone dein selbst; das widerfahre dir nur nicht!
\par 23 Aber er wandte sich um und sprach zu Petrus: Hebe dich, Satan, von mir! du bist mir ärgerlich; denn du meinst nicht was göttlich, sondern was menschlich ist.
\par 24 Da sprach Jesus zu seinen Jüngern: Will mir jemand nachfolgen, der verleugne sich selbst und nehme sein Kreuz auf sich und folge mir.
\par 25 Denn wer sein Leben erhalten will, der wird's verlieren; wer aber sein Leben verliert um meinetwillen, der wird's finden.
\par 26 Was hülfe es dem Menschen, so er die ganze Welt gewönne und nähme Schaden an seiner Seele? Oder was kann der Mensch geben, damit er seine Seele wieder löse?
\par 27 Denn es wird geschehen, daß des Menschen Sohn komme in der Herrlichkeit seines Vaters mit seinen Engeln; und alsdann wird er einem jeglichen vergelten nach seinen Werken.
\par 28 Wahrlich ich sage euch: Es stehen etliche hier, die nicht schmecken werden den Tod, bis daß sie des Menschen Sohn kommen sehen in seinem Reich.

\chapter{17}

\par 1 Und nach sechs Tagen nahm Jesus zu sich Petrus und Jakobus und Johannes, seinen Bruder, und führte sie beiseits auf einen hohen Berg.
\par 2 Und er ward verklärt vor ihnen, und sein Angesicht leuchtete wie die Sonne, und seine Kleider wurden weiß wie ein Licht.
\par 3 Und siehe, da erschienen ihnen Mose und Elia; die redeten mit ihm.
\par 4 Petrus aber antwortete und sprach zu Jesu: HERR, hier ist gut sein! Willst du, so wollen wir hier drei Hütten machen: dir eine, Mose eine und Elia eine.
\par 5 Da er noch also redete, siehe, da überschattete sie eine lichte Wolke. Und siehe, eine Stimme aus der Wolke sprach: Dies ist mein lieber Sohn, an welchem ich Wohlgefallen habe, den sollt ihr hören!
\par 6 Da das die Jünger hörten, fielen sie auf ihr Angesicht und erschraken sehr.
\par 7 Jesus aber trat zu ihnen, rührte sie an und sprach: Stehet auf und fürchtet euch nicht!
\par 8 Da sie aber ihre Augen aufhoben, sahen sie niemand denn Jesum allein.
\par 9 Und da sie vom Berge herabgingen, gebot ihnen Jesus und sprach: Ihr sollt dies Gesicht niemand sagen, bis das des Menschen Sohn von den Toten auferstanden ist.
\par 10 Und seine Jünger fragten ihn und sprachen: Was sagen denn die Schriftgelehrten, Elia müsse zuvor kommen?
\par 11 Jesus antwortete und sprach zu ihnen: Elia soll ja zuvor kommen und alles zurechtbringen.
\par 12 Doch ich sage euch: Es ist Elia schon gekommen, und sie haben ihn nicht erkannt, sondern haben an ihm getan, was sie wollten. Also wird auch des Menschen Sohn leiden müssen von ihnen.
\par 13 Da verstanden die Jünger, daß er von Johannes dem Täufer zu ihnen geredet hatte.
\par 14 Und da sie zu dem Volk kamen, trat zu ihm ein Mensch und fiel ihm zu Füßen
\par 15 und sprach: HERR, erbarme dich über meinen Sohn! denn er ist mondsüchtig und hat ein schweres Leiden: er fällt oft ins Feuer und oft ins Wasser;
\par 16 und ich habe ihn zu deinen Jüngern gebracht, und sie konnten ihm nicht helfen.
\par 17 Jesus aber antwortete und sprach: O du ungläubige und verkehrte Art, wie lange soll ich bei euch sein? wie lange soll ich euch dulden? Bringt ihn hierher!
\par 18 Und Jesus bedrohte ihn; und der Teufel fuhr aus von ihm, und der Knabe ward gesund zu derselben Stunde.
\par 19 Da traten zu ihm seine Jünger besonders und sprachen: Warum konnten wir ihn nicht austreiben?
\par 20 Jesus aber antwortete und sprach zu ihnen: Um eures Unglaubens willen. Denn wahrlich ich sage euch: So ihr Glauben habt wie ein Senfkorn, so mögt ihr sagen zu diesem Berge: Hebe dich von hinnen dorthin! so wird er sich heben; und euch wird nichts unmöglich sein.
\par 21 Aber diese Art fährt nicht aus denn durch Beten und Fasten.
\par 22 Da sie aber ihr Wesen hatten in Galiläa, sprach Jesus zu ihnen: Es wird geschehen, daß des Menschen Sohn überantwortet wird in der Menschen Hände;
\par 23 und sie werden ihn töten, und am dritten Tage wird er auferstehen. Und sie wurden sehr betrübt.
\par 24 Da sie nun gen Kapernaum kamen, gingen zu Petrus, die den Zinsgroschen einnahmen, und sprachen: Pflegt euer Meister nicht den Zinsgroschen zu geben?
\par 25 Er sprach: Ja. Und als er heimkam, kam ihm Jesus zuvor und sprach: Was dünkt dich, Simon? Von wem nehmen die Könige auf Erden den Zoll oder Zins? Von Ihren Kindern oder von den Fremden?
\par 26 Da sprach zu ihm Petrus: Von den Fremden. Jesus sprach zu ihm: So sind die Kinder frei.
\par 27 Auf daß aber wir sie nicht ärgern, so gehe hin an das Meer und wirf die Angel, und den ersten Fisch, der herauffährt, den nimm; und wenn du seinen Mund auftust, wirst du einen Stater finden; den nimm und gib ihnen für mich und dich.

\chapter{18}

\par 1 Zu derselben Stunde traten die Jünger zu Jesu und sprachen: Wer ist doch der Größte im Himmelreich?
\par 2 Jesus rief ein Kind zu sich und stellte das mitten unter sie
\par 3 und sprach: Wahrlich ich sage euch: Es sei denn, daß ihr umkehret und werdet wie die Kinder, so werdet ihr nicht ins Himmelreich kommen.
\par 4 Wer nun sich selbst erniedrigt wie dies Kind, der ist der Größte im Himmelreich.
\par 5 Und wer ein solches Kind aufnimmt in meinem Namen, der nimmt mich auf.
\par 6 Wer aber ärgert dieser Geringsten einen, die an mich glauben, dem wäre es besser, daß ein Mühlstein an seinen Hals gehängt und er ersäuft werde im Meer, da es am tiefsten ist.
\par 7 Weh der Welt der Ärgernisse halben! Es muß ja Ärgernis kommen; doch weh dem Menschen, durch welchen Ärgernis kommt!
\par 8 So aber deine Hand oder dein Fuß dich ärgert, so haue ihn ab und wirf ihn von dir. Es ist besser, daß du zum Leben lahm oder als Krüppel eingehst, denn daß du zwei Hände oder zwei Füße hast und wirst in das höllische Feuer geworfen.
\par 9 Und so dich dein Auge ärgert, reiß es aus und wirf's von dir. Es ist dir besser, daß du einäugig zum Leben eingehest, denn daß du zwei Augen habest und wirst in das höllische Feuer geworfen.
\par 10 Sehet zu, daß ihr nicht jemand von diesen Kleinen verachtet. Denn ich sage euch: Ihre Engel im Himmel sehen allezeit in das Angesicht meines Vaters im Himmel.
\par 11 Denn des Menschen Sohn ist gekommen, selig zu machen, das verloren ist.
\par 12 Was dünkt euch? Wenn irgend ein Mensch hundert Schafe hätte und eins unter ihnen sich verirrte: läßt er nicht die neunundneunzig auf den Bergen, geht hin und sucht das verirrte?
\par 13 Und so sich's begibt, daß er's findet, wahrlich ich sage euch, er freut sich darüber mehr denn über die neunundneunzig, die nicht verirrt sind.
\par 14 Also auch ist's vor eurem Vater im Himmel nicht der Wille, daß jemand von diesen Kleinen verloren werde.
\par 15 Sündigt aber dein Bruder an dir, so gehe hin und strafe ihn zwischen dir und ihm allein. Hört er dich, so hast du deinen Bruder gewonnen.
\par 16 Hört er dich nicht, so nimm noch einen oder zwei zu dir, auf daß alle Sache bestehe auf zweier oder dreier Zeugen Mund.
\par 17 Hört er die nicht, so sage es der Gemeinde. Hört er die Gemeinde nicht, so halt ihn als einen Zöllner oder Heiden.
\par 18 Wahrlich ich sage euch: Was ihr auf Erden binden werdet, soll auch im Himmel gebunden sein, und was ihr auf Erden lösen werdet, soll auch im Himmel los sein.
\par 19 Weiter sage ich euch: wo zwei unter euch eins werden, warum es ist, daß sie bitten wollen, das soll ihnen widerfahren von meinem Vater im Himmel.
\par 20 Denn wo zwei oder drei versammelt sind in meinem Namen, da bin ich mitten unter ihnen.
\par 21 Da trat Petrus zu ihm und sprach: HERR, wie oft muß ich denn meinem Bruder, der an mir sündigt, vergeben? Ist's genug siebenmal?
\par 22 Jesus sprach zu ihm: Ich sage dir: Nicht siebenmal, sondern siebzigmal siebenmal.
\par 23 Darum ist das Himmelreich gleich einem König, der mit seinen Knechten rechnen wollte.
\par 24 Und als er anfing zu rechnen, kam ihm einer vor, der war ihm zehntausend Pfund schuldig.
\par 25 Da er's nun nicht hatte, zu bezahlen, hieß der Herr verkaufen ihn und sein Weib und seine Kinder und alles, was er hatte, und bezahlen.
\par 26 Da fiel der Knecht nieder und betete ihn an und sprach: Herr, habe Geduld mit mir, ich will dir's alles bezahlen.
\par 27 Da jammerte den Herrn des Knechtes, und er ließ ihn los, und die Schuld erließ er ihm auch.
\par 28 Da ging derselbe Knecht hinaus und fand einen seiner Mitknechte, der war ihm hundert Groschen schuldig; und er griff ihn an und würgte ihn und sprach: Bezahle mir, was du mir schuldig bist!
\par 29 Da fiel sein Mitknecht nieder und bat ihn und sprach: Habe Geduld mit mir; ich will dir's alles bezahlen.
\par 30 Er wollte aber nicht, sondern ging hin und warf ihn ins Gefängnis, bis daß er bezahlte, was er schuldig war.
\par 31 Da aber seine Mitknechte solches sahen, wurden sie sehr betrübt und kamen und brachten vor ihren Herrn alles, was sich begeben hatte.
\par 32 Da forderte ihn sein Herr vor sich und sprach zu ihm: Du Schalksknecht, alle diese Schuld habe ich dir erlassen, dieweil du mich batest;
\par 33 solltest du denn dich nicht auch erbarmen über deinen Mitknecht, wie ich mich über dich erbarmt habe?
\par 34 Und sein Herr ward sehr zornig und überantwortete ihn den Peinigern, bis daß er bezahlte alles, was er ihm schuldig war.
\par 35 Also wird euch mein himmlischer Vater auch tun, so ihr nicht vergebt von eurem Herzen, ein jeglicher seinem Bruder seine Fehler.

\chapter{19}

\par 1 Und es begab sich, da Jesus diese Reden vollendet hatte, erhob er sich aus Galiläa und kam in das Gebiet des jüdischen Landes jenseit des Jordans;
\par 2 und es folgte ihm viel Volks nach, und er heilte sie daselbst.
\par 3 Da traten zu ihm die Pharisäer, versuchten ihn und sprachen zu ihm: Ist's auch recht, daß sich ein Mann scheide von seinem Weibe um irgendeine Ursache?
\par 4 Er antwortete aber und sprach zu ihnen: Habt ihr nicht gelesen, daß, der im Anfang den Menschen gemacht hat, der machte, daß ein Mann und ein Weib sein sollte,
\par 5 und sprach: "Darum wird ein Mensch Vater und Mutter verlassen und an seinem Weibe hangen, und werden die zwei ein Fleisch sein"?
\par 6 So sind sie nun nicht zwei, sondern ein Fleisch. Was nun Gott zusammengefügt hat, das soll der Mensch nicht scheiden.
\par 7 Da sprachen sie: Warum hat denn Mose geboten, einen Scheidebrief zu geben und sich von ihr zu scheiden?
\par 8 Er sprach zu ihnen: Mose hat euch erlaubt zu scheiden von euren Weibern wegen eures Herzens Härtigkeit; von Anbeginn aber ist's nicht also gewesen.
\par 9 Ich sage aber euch: Wer sich von seinem Weibe scheidet (es sei denn um der Hurerei willen) und freit eine andere, der bricht die Ehe; und wer die Abgeschiedene freit, der bricht auch die Ehe.
\par 10 Da sprachen die Jünger zu ihm: Steht die Sache eines Mannes mit seinem Weibe also, so ist's nicht gut, ehelich werden.
\par 11 Er sprach zu ihnen: Das Wort faßt nicht jedermann, sondern denen es gegeben ist.
\par 12 Denn es sind etliche verschnitten, die sind aus Mutterleibe also geboren; und sind etliche verschnitten, die von Menschen verschnitten sind; und sind etliche verschnitten, die sich selbst verschnitten haben um des Himmelreiches willen. Wer es fassen kann, der fasse es!
\par 13 Da wurden Kindlein zu ihm gebracht, daß er die Hände auf sie legte und betete. Die Jünger aber fuhren sie an.
\par 14 Aber Jesus sprach: Lasset die Kindlein zu mir kommen und wehret ihnen nicht, denn solcher ist das Reich Gottes.
\par 15 Und legte die Hände auf sie und zog von dannen.
\par 16 Und siehe, einer trat zu ihm und sprach: Guter Meister, was soll ich Gutes tun, daß ich das ewige Leben möge haben?
\par 17 Er aber sprach zu ihm: Was heißest du mich gut? Niemand ist gut denn der einige Gott. Willst du aber zum Leben eingehen, so halte die Gebote.
\par 18 Da sprach er zu ihm: Welche? Jesus aber sprach: "Du sollst nicht töten; du sollst nicht ehebrechen; du sollst nicht stehlen; du sollst nicht falsch Zeugnis geben;
\par 19 ehre Vater und Mutter;" und: "Du sollst deinen Nächsten lieben wie dich selbst."
\par 20 Da sprach der Jüngling zu ihm: Das habe ich alles gehalten von meiner Jugend auf; was fehlt mir noch?
\par 21 Jesus sprach zu ihm: Willst du vollkommen sein, so gehe hin, verkaufe, was du hast, und gib's den Armen, so wirst du einen Schatz im Himmel haben; und komm und folge mir nach!
\par 22 Da der Jüngling das Wort hörte, ging er betrübt von ihm, denn er hatte viele Güter.
\par 23 Jesus aber sprach zu seinen Jüngern: Wahrlich, ich sage euch: Ein Reicher wird schwer ins Himmelreich kommen.
\par 24 Und weiter sage ich euch: Es ist leichter, daß ein Kamel durch ein Nadelöhr gehe, denn daß ein Reicher ins Reich Gottes komme.
\par 25 Da das seine Jünger hörten, entsetzten sie sich sehr und sprachen: Ja, wer kann denn selig werden?
\par 26 Jesus aber sah sie an und sprach zu ihnen: Bei den Menschen ist es unmöglich; aber bei Gott sind alle Dinge möglich.
\par 27 Da antwortete Petrus und sprach zu ihm: Siehe, wir haben alles verlassen und sind dir nachgefolgt; was wird uns dafür?
\par 28 Jesus aber sprach zu ihnen: Wahrlich ich sage euch: Ihr, die ihr mir seid nachgefolgt, werdet in der Wiedergeburt, da des Menschen Sohn wird sitzen auf dem Stuhl seiner Herrlichkeit, auch sitzen auf zwölf Stühlen und richten die zwölf Geschlechter Israels.
\par 29 Und wer verläßt Häuser oder Brüder oder Schwestern oder Vater oder Mutter oder Weib oder Kinder oder Äcker um meines Namens willen, der wird's hundertfältig nehmen und das ewige Leben ererben.
\par 30 Aber viele, die da sind die Ersten, werden die Letzten, und die Letzten werden die Ersten sein.

\chapter{20}

\par 1 Das Himmelreich ist gleich einem Hausvater, der am Morgen ausging, Arbeiter zu mieten in seinen Weinberg.
\par 2 Und da er mit den Arbeitern eins ward um einen Groschen zum Tagelohn, sandte er sie in seinen Weinberg.
\par 3 Und ging aus um die dritte Stunde und sah andere an dem Markte müßig stehen
\par 4 und sprach zu ihnen: Gehet ihr auch hin in den Weinberg; ich will euch geben, was recht ist.
\par 5 Und sie gingen hin. Abermals ging er aus um die sechste und die neunte Stunde und tat gleichalso.
\par 6 Um die elfte Stunde aber ging er aus und fand andere müßig stehen und sprach zu ihnen: Was steht ihr hier den ganzen Tag müßig?
\par 7 Sie sprachen zu ihm: Es hat uns niemand gedingt. Er sprach zu ihnen: Gehet ihr auch hin in den Weinberg, und was recht sein wird, soll euch werden.
\par 8 Da es nun Abend ward, sprach der Herr des Weinberges zu seinem Schaffner: Rufe die Arbeiter und gib ihnen den Lohn und heb an an den Letzten bis zu den Ersten.
\par 9 Da kamen, die um die elfte Stunde gedingt waren, und empfing ein jeglicher seinen Groschen.
\par 10 Da aber die ersten kamen, meinten sie, sie würden mehr empfangen; und sie empfingen auch ein jeglicher seinen Groschen.
\par 11 Und da sie den empfingen, murrten sie wider den Hausvater
\par 12 und sprachen: Diese haben nur eine Stunde gearbeitet, und du hast sie uns gleich gemacht, die wir des Tages Last und die Hitze getragen haben.
\par 13 Er antwortete aber und sagte zu einem unter ihnen: Mein Freund, ich tue dir nicht Unrecht. Bist du nicht mit mir eins geworden für einen Groschen?
\par 14 Nimm, was dein ist, und gehe hin! Ich will aber diesem letzten geben gleich wie dir.
\par 15 Oder habe ich nicht Macht, zu tun, was ich will, mit dem Meinen? Siehst du darum so scheel, daß ich so gütig bin?
\par 16 Also werden die Letzten die Ersten und die Ersten die Letzten sein. Denn viele sind berufen, aber wenige auserwählt.
\par 17 Und er zog hinauf gen Jerusalem und nahm zu sich die zwölf Jünger besonders auf dem Wege und sprach zu ihnen:
\par 18 Siehe, wir ziehen hinauf gen Jerusalem, und des Menschen Sohn wird den Hohenpriestern und Schriftgelehrten überantwortet werden; sie werden ihn verdammen zum Tode
\par 19 und werden ihn überantworten den Heiden, zu verspotten und zu geißeln und zu kreuzigen; und am dritten Tage wird er wieder auferstehen.
\par 20 Da trat zu ihm die Mutter der Kinder des Zebedäus mit ihren Söhnen, fiel vor ihm nieder und bat etwas von ihm.
\par 21 Und er sprach zu ihr: Was willst du? Sie sprach zu ihm: Laß diese meine zwei Söhne sitzen in deinem Reich, einen zu deiner Rechten und den andern zu deiner Linken.
\par 22 Aber Jesus antwortete und sprach: Ihr wisset nicht, was ihr bittet. Könnt ihr den Kelch trinken, den ich trinken werde, und euch taufen lassen mit der Taufe, mit der ich getauft werde? Sie sprachen zu ihm: Jawohl.
\par 23 Und er sprach zu ihnen: Meinen Kelch sollt ihr zwar trinken, und mit der Taufe, mit der ich getauft werde, sollt ihr getauft werden; aber das sitzen zu meiner Rechten und Linken zu geben steht mir nicht zu, sondern denen es bereitet ist von meinem Vater.
\par 24 Da das die zehn hörten, wurden sie unwillig über die zwei Brüder.
\par 25 Aber Jesus rief sie zu sich und sprach: Ihr wisset, daß die weltlichen Fürsten herrschen und die Obersten haben Gewalt.
\par 26 So soll es nicht sein unter euch. Sondern, so jemand will unter euch gewaltig sein, der sei euer Diener;
\par 27 und wer da will der Vornehmste sein, der sei euer Knecht,
\par 28 gleichwie des Menschen Sohn ist nicht gekommen, daß er sich dienen lasse, sondern daß er diene und gebe sein Leben zu einer Erlösung für viele.
\par 29 Und da sie von Jericho auszogen, folgte ihm viel Volks nach.
\par 30 Und siehe, zwei Blinde saßen am Wege; und da sie hörten, daß Jesus vorüberging, schrieen sie und sprachen: Ach HERR, du Sohn Davids, erbarme dich unser!
\par 31 Aber das Volk bedrohte sie, daß sie schweigen sollten. Aber sie schrieen viel mehr und sprachen: Ach HERR, du Sohn Davids, erbarme dich unser!
\par 32 Jesus aber stand still und rief sie und sprach: Was wollt ihr, daß ich euch tun soll?
\par 33 Sie sprachen zu ihm: HERR, daß unsere Augen aufgetan werden.
\par 34 Und es jammerte Jesum, und er rührte ihre Augen an; und alsbald wurden ihre Augen wieder sehend, und sie folgten ihm nach.

\chapter{21}

\par 1 Da sie nun nahe an Jerusalem kamen, gen Bethphage an den Ölberg, sandte Jesus seiner Jünger zwei
\par 2 und sprach zu ihnen: Gehet hin in den Flecken, der vor euch liegt, und alsbald werdet ihr eine Eselin finden angebunden und ihr Füllen bei ihr; löset sie auf und führet sie zu mir!
\par 3 Und so euch jemand etwas wird sagen, so sprecht: Der HERR bedarf ihrer; sobald wird er sie euch lassen.
\par 4 Das geschah aber alles, auf daß erfüllt würde, was gesagt ist durch den Propheten, der da spricht:
\par 5 "Saget der Tochter Zion: Siehe, dein König kommt zu dir sanftmütig und reitet auf einem Esel und auf einem Füllen der lastbaren Eselin."
\par 6 Die Jünger gingen hin und taten, wie ihnen Jesus befohlen hatte,
\par 7 und brachten die Eselin und das Füllen und legten ihre Kleider darauf und setzten ihn darauf.
\par 8 Aber viel Volks breitete die Kleider auf den Weg; die andern hieben Zweige von den Bäumen und streuten sie auf den Weg.
\par 9 Das Volk aber, das vorging und nachfolgte, schrie und sprach: Hosianna dem Sohn Davids! Gelobt sei, der da kommt in dem Namen des HERRN! Hosianna in der Höhe!
\par 10 Und als er zu Jerusalem einzog, erregte sich die ganze Stadt und sprach: Wer ist der?
\par 11 Das Volk aber sprach: Das ist der Jesus, der Prophet von Nazareth aus Galiläa.
\par 12 Und Jesus ging zum Tempel Gottes hinein und trieb heraus alle Verkäufer und Käufer im Tempel und stieß um der Wechsler Tische und die Stühle der Taubenkrämer
\par 13 und sprach zu ihnen: Es steht geschrieben: "Mein Haus soll ein Bethaus heißen"; ihr aber habt eine Mördergrube daraus gemacht.
\par 14 Und es gingen zu ihm Blinde und Lahme im Tempel, und er heilte sie.
\par 15 Da aber die Hohenpriester und Schriftgelehrten sahen die Wunder, die er tat, und die Kinder, die im Tempel schrieen und sagten: Hosianna dem Sohn Davids! wurden sie entrüstet
\par 16 und sprachen zu ihm: Hörst du auch, was diese sagen? Jesus sprach zu ihnen: Ja! Habt ihr nie gelesen: "Aus dem Munde der Unmündigen und Säuglinge hast du Lob zugerichtet"?
\par 17 Und er ließ sie da und ging zur Stadt hinaus gen Bethanien und blieb daselbst.
\par 18 Als er aber des Morgens wieder in die Stadt ging, hungerte ihn;
\par 19 und er sah einen Feigenbaum am Wege und ging hinzu und fand nichts daran denn allein Blätter und sprach zu ihm: Nun wachse auf dir hinfort nimmermehr eine Frucht! Und der Feigenbaum verdorrte alsbald.
\par 20 Und da das die Jünger sahen, verwunderten sie sich und sprachen: Wie ist der Feigenbaum so bald verdorrt?
\par 21 Jesus aber antwortete und sprach zu ihnen: Wahrlich ich sage euch: So ihr Glauben habt und nicht zweifelt, so werdet ihr nicht allein solches mit dem Feigenbaum tun, sondern, so ihr werdet sagen zu diesem Berge: Hebe dich auf und wirf dich ins Meer! so wird's geschehen.
\par 22 Und alles, was ihr bittet im Gebet, so ihr glaubet, werdet ihr's empfangen.
\par 23 Und als er in den Tempel kam, traten zu ihm, als er lehrte, die Hohenpriester und die Ältesten im Volk und sprachen: Aus was für Macht tust du das? und wer hat dir die Macht gegeben?
\par 24 Jesus aber antwortete und sprach zu ihnen: Ich will euch auch ein Wort fragen; so ihr mir das sagt, will ich euch auch sagen aus was für Macht ich das tue:
\par 25 Woher war die Taufe des Johannes? War sie vom Himmel oder von den Menschen? Da dachten sie bei sich selbst und sprachen: Sagen wir, sie sei vom Himmel gewesen, so wird er zu uns sagen: Warum glaubtet ihr ihm denn nicht?
\par 26 Sagen wir aber, sie sei von Menschen gewesen, so müssen wir uns vor dem Volk fürchten; denn sie halten alle Johannes für einen Propheten.
\par 27 Und sie antworteten Jesu und sprachen: Wir wissen's nicht. Da sprach er zu ihnen: So sage ich euch auch nicht, aus was für Macht ich das tue.
\par 28 Was dünkt euch aber? Es hatte ein Mann zwei Söhne und ging zu dem ersten und sprach: Mein Sohn, gehe hin und arbeite heute in meinem Weinberg.
\par 29 Er antwortete aber und sprach: Ich will's nicht tun. Darnach reute es ihn und er ging hin.
\par 30 Und er ging zum andern und sprach gleichalso. Er antwortete aber und sprach: Herr, ja! -und ging nicht hin.
\par 31 Welcher unter den zweien hat des Vaters Willen getan? Sie sprachen zu ihm: Der erste. Jesus sprach zu ihnen: Wahrlich ich sage euch: Die Zöllner und Huren mögen wohl eher ins Himmelreich kommen denn ihr.
\par 32 Johannes kam zu euch und lehrte euch den rechten Weg, und ihr glaubtet ihm nicht; aber die Zöllner und Huren glaubten ihm. Und ob ihr's wohl sahet, tatet ihr dennoch nicht Buße, daß ihr ihm darnach auch geglaubt hättet.
\par 33 Höret ein anderes Gleichnis: Es war ein Hausvater, der pflanzte einen Weinberg und führte einen Zaun darum und grub eine Kelter darin und baute einen Turm und tat ihn den Weingärtnern aus und zog über Land.
\par 34 Da nun herbeikam die Zeit der Früchte, sandte er seine Knechte zu den Weingärtnern, daß sie seine Früchte empfingen.
\par 35 Da nahmen die Weingärtner seine Knechte; einen stäupten sie, den andern töteten sie, den dritten steinigten sie.
\par 36 Abermals sandte er andere Knechte, mehr denn der ersten waren; und sie taten ihnen gleichalso.
\par 37 Darnach sandte er seinen Sohn zu ihnen und sprach: Sie werden sich vor meinem Sohn scheuen.
\par 38 Da aber die Weingärtner den Sohn sahen, sprachen sie untereinander: Das ist der Erbe; kommt laßt uns ihn töten und sein Erbgut an uns bringen!
\par 39 Und sie nahmen ihn und stießen ihn zum Weinberg hinaus und töteten ihn.
\par 40 Wenn nun der Herr des Weinberges kommen wird, was wird er diesen Weingärtnern tun?
\par 41 Sie sprachen zu ihm: Er wird die Bösewichte übel umbringen und seinen Weinberg anderen Weingärtnern austun, die ihm die Früchte zur rechten Zeit geben.
\par 42 Jesus sprach zu ihnen: Habt ihr nie gelesen in der Schrift: "Der Stein, den die Bauleute verworfen haben, der ist zum Eckstein geworden. Von dem HERRN ist das geschehen, und es ist wunderbar vor unseren Augen"?
\par 43 Darum sage ich euch: Das Reich Gottes wird von euch genommen und einem Volke gegeben werden, das seine Früchte bringt.
\par 44 Und wer auf diesen Stein fällt, der wird zerschellen; auf wen aber er fällt, den wird er zermalmen.
\par 45 Und da die Hohenpriester und Pharisäer seine Gleichnisse hörten, verstanden sie, daß er von ihnen redete.
\par 46 Und sie trachteten darnach, wie sie ihn griffen; aber sie fürchteten sich vor dem Volk, denn es hielt ihn für einen Propheten.

\chapter{22}

\par 1 Und Jesus antwortete und redete abermals durch Gleichnisse zu ihnen und sprach:
\par 2 Das Himmelreich ist gleich einem Könige, der seinem Sohn Hochzeit machte.
\par 3 Und sandte seine Knechte aus, daß sie die Gäste zur Hochzeit riefen; und sie wollten nicht kommen.
\par 4 Abermals sandte er andere Knechte aus und sprach: Sagt den Gästen: Siehe, meine Mahlzeit habe ich bereitet, meine Ochsen und mein Mastvieh ist geschlachtet und alles ist bereit; kommt zur Hochzeit!
\par 5 Aber sie verachteten das und gingen hin, einer auf seinen Acker, der andere zu seiner Hantierung;
\par 6 etliche griffen seine Knechte, höhnten sie und töteten sie.
\par 7 Da das der König hörte, ward er zornig und schickte seine Heere aus und brachte diese Mörder um und zündete ihre Stadt an.
\par 8 Da sprach er zu seinen Knechten: Die Hochzeit ist zwar bereit, aber die Gäste waren's nicht wert.
\par 9 Darum gehet hin auf die Straßen und ladet zur Hochzeit, wen ihr findet.
\par 10 Und die Knechte gingen aus auf die Straßen und brachten zusammen, wen sie fanden, Böse und Gute; und die Tische wurden alle voll.
\par 11 Da ging der König hinein, die Gäste zu besehen, und sah allda einen Menschen, der hatte kein hochzeitlich Kleid an;
\par 12 und er sprach zu ihm: Freund, wie bist du hereingekommen und hast doch kein hochzeitlich Kleid an? Er aber verstummte.
\par 13 Da sprach der König zu seinen Dienern: Bindet ihm Hände und Füße und werfet ihn in die Finsternis hinaus! da wird sein Heulen und Zähneklappen.
\par 14 Denn viele sind berufen, aber wenige sind auserwählt.
\par 15 Da gingen die Pharisäer hin und hielten einen Rat, wie sie ihn fingen in seiner Rede.
\par 16 Und sandten zu ihm ihre Jünger samt des Herodes Dienern. Und sie sprachen: Meister, wir wissen, daß du wahrhaftig bist und lehrst den Weg Gottes recht und du fragst nach niemand; denn du achtest nicht das Ansehen der Menschen.
\par 17 Darum sage uns, was dünkt dich: Ist's recht, daß man dem Kaiser den Zins gebe, oder nicht?
\par 18 Da nun Jesus merkte ihre Schalkheit, sprach er: Ihr Heuchler, was versucht ihr mich?
\par 19 Weiset mir die Zinsmünze! Und sie reichten ihm einen Groschen dar.
\par 20 Und er sprach zu ihnen: Wes ist das Bild und die Überschrift?
\par 21 Sie sprachen zu ihm: Des Kaisers. Da sprach er zu ihnen: So gebet dem Kaiser, was des Kaisers ist, und Gott, was Gottes ist!
\par 22 Da sie das hörten, verwunderten sie sich und ließen ihn und gingen davon.
\par 23 An dem Tage traten zu ihm die Sadduzäer, die da halten, es sei kein Auferstehen, und fragten ihn
\par 24 und sprachen: Meister, Mose hat gesagt: So einer stirbt und hat nicht Kinder, so soll sein Bruder sein Weib freien und seinem Bruder Samen erwecken.
\par 25 Nun sind bei uns gewesen sieben Brüder. Der erste freite und starb; und dieweil er nicht Samen hatte, ließ er sein Weib seinem Bruder;
\par 26 desgleichen der andere und der dritte bis an den siebenten.
\par 27 Zuletzt nach allen starb auch das Weib.
\par 28 Nun in der Auferstehung, wes Weib wird sie sein unter den sieben? Sie haben sie ja alle gehabt.
\par 29 Jesus aber antwortete und sprach zu ihnen: Ihr irrt und wisset die Schrift nicht, noch die Kraft Gottes.
\par 30 In der Auferstehung werden sie weder freien noch sich freien lassen, sondern sie sind gleichwie die Engel Gottes im Himmel.
\par 31 Habt ihr nicht gelesen von der Toten Auferstehung, was euch gesagt ist von Gott, der da spricht:
\par 32 "Ich bin der Gott Abrahams und der Gott Isaaks und der Gott Jakobs"? Gott aber ist nicht ein Gott der Toten, sondern der Lebendigen.
\par 33 Und da solches das Volk hörte, entsetzten sie sich über seine Lehre.
\par 34 Da aber die Pharisäer hörten, wie er den Sadduzäern das Maul gestopft hatte, versammelten sie sich.
\par 35 Und einer unter ihnen, ein Schriftgelehrter, versuchte ihn und sprach:
\par 36 Meister, welches ist das vornehmste Gebot im Gesetz?
\par 37 Jesus aber sprach zu ihm: "Du sollst lieben Gott, deinen HERRN, von ganzem Herzen, von ganzer Seele und von ganzem Gemüte."
\par 38 Dies ist das vornehmste und größte Gebot.
\par 39 Das andere aber ist ihm gleich; Du sollst deinen Nächsten lieben wie dich selbst.
\par 40 In diesen zwei Geboten hängt das ganze Gesetz und die Propheten.
\par 41 Da nun die Pharisäer beieinander waren, fragte sie Jesus
\par 42 und sprach: Wie dünkt euch um Christus? wes Sohn ist er? Sie sprachen: Davids.
\par 43 Er sprach zu ihnen: Wie nennt ihn denn David im Geist einen Herrn, da er sagt:
\par 44 "Der HERR hat gesagt zu meinem Herrn: Setze dich zu meiner Rechten, bis daß ich lege deine Feinde zum Schemel deiner Füße"?
\par 45 So nun David ihn einen Herrn nennt, wie ist er denn sein Sohn?
\par 46 Und niemand konnte ihm ein Wort antworten, und wagte auch niemand von dem Tage an hinfort, ihn zu fragen.

\chapter{23}

\par 1 Da redete Jesus zu dem Volk und zu seinen Jüngern
\par 2 und sprach: Auf Mose's Stuhl sitzen die Schriftgelehrten und Pharisäer.
\par 3 Alles nun, was sie euch sagen, daß ihr halten sollt, das haltet und tut's; aber nach ihren Werken sollt ihr nicht tun: sie sagen's wohl, und tun's nicht.
\par 4 Sie binden aber schwere und unerträgliche Bürden und legen sie den Menschen auf den Hals; aber sie selbst wollen dieselben nicht mit einem Finger regen.
\par 5 Alle ihre Werke aber tun sie, daß sie von den Leuten gesehen werden. Sie machen ihre Denkzettel breit und die Säume an ihren Kleidern groß.
\par 6 Sie sitzen gern obenan über Tisch und in den Schulen
\par 7 und haben's gern, daß sie gegrüßt werden auf dem Markt und von den Menschen Rabbi genannt werden.
\par 8 Aber ihr sollt euch nicht Rabbi nennen lassen; denn einer ist euer Meister, Christus; ihr aber seid alle Brüder.
\par 9 Und sollt niemand Vater heißen auf Erden, denn einer ist euer Vater, der im Himmel ist.
\par 10 Und ihr sollt euch nicht lassen Meister nennen; denn einer ist euer Meister, Christus.
\par 11 Der Größte unter euch soll euer Diener sein.
\par 12 Denn wer sich selbst erhöht, der wird erniedrigt; und wer sich selbst erniedrigt, der wird erhöht.
\par 13 Weh euch, Schriftgelehrte und Pharisäer, ihr Heuchler, die ihr das Himmelreich zuschließet vor den Menschen! Ihr kommt nicht hinein, und die hinein wollen, laßt ihr nicht hineingehen.
\par 14 Weh euch, Schriftgelehrte und Pharisäer, ihr Heuchler, die ihr der Witwen Häuser fresset und wendet lange Gebete vor! Darum werdet ihr desto mehr Verdammnis empfangen.
\par 15 Weh euch, Schriftgelehrte und Pharisäer, ihr Heuchler, die ihr Land und Wasser umziehet, daß ihr einen Judengenossen macht; und wenn er's geworden ist, macht ihr aus ihm ein Kind der Hölle, zwiefältig mehr denn ihr seid!
\par 16 Weh euch, verblendete Leiter, die ihr sagt: "Wer da schwört bei dem Tempel, das ist nichts; wer aber schwört bei dem Gold am Tempel, der ist's schuldig."
\par 17 Ihr Narren und Blinden! Was ist größer: das Gold oder der Tempel, der das Gold heiligt?
\par 18 "Wer da schwört bei dem Altar, das ist nichts; wer aber schwört bei dem Opfer, das darauf ist, der ist's schuldig."
\par 19 Ihr Narren und Blinden! Was ist größer: das Opfer oder der Altar, der das Opfer heiligt?
\par 20 Darum, wer da schwört bei dem Altar, der schwört bei demselben und bei allem, was darauf ist.
\par 21 Und wer da schwört bei dem Tempel, der schwört bei demselben und bei dem, der darin wohnt.
\par 22 Und wer da schwört bei dem Himmel, der schwört bei dem Stuhl Gottes und bei dem, der darauf sitzt.
\par 23 Weh euch, Schriftgelehrte und Pharisäer, ihr Heuchler, die ihr verzehntet die Minze, Dill und Kümmel, und laßt dahinten das Schwerste im Gesetz, nämlich das Gericht, die Barmherzigkeit und den Glauben! Dies soll man tun und jenes nicht lassen.
\par 24 Ihr verblendeten Leiter, die ihr Mücken seihet und Kamele verschluckt!
\par 25 Weh euch, Schriftgelehrte und Pharisäer, ihr Heuchler, die ihr die Becher und Schüsseln auswendig reinlich haltet, inwendig aber ist's voll Raubes und Fraßes!
\par 26 Du blinder Pharisäer, reinige zum ersten das Inwendige an Becher und Schüssel, auf das auch das Auswendige rein werde!
\par 27 Weh euch, Schriftgelehrte und Pharisäer, ihr Heuchler, die ihr gleich seid wie die übertünchten Gräber, welche auswendig hübsch scheinen, aber inwendig sind sie voller Totengebeine und alles Unflats!
\par 28 Also auch ihr: von außen scheint ihr den Menschen fromm, aber in wendig seid ihr voller Heuchelei und Untugend.
\par 29 Weh euch, Schriftgelehrte und Pharisäer, ihr Heuchler, die ihr der Propheten Gräber bauet und schmücket der Gerechten Gräber
\par 30 und sprecht: Wären wir zu unsrer Väter Zeiten gewesen, so wollten wir nicht teilhaftig sein mit ihnen an der Propheten Blut!
\par 31 So gebt ihr über euch selbst Zeugnis, daß ihr Kinder seid derer, die die Propheten getötet haben.
\par 32 Wohlan, erfüllet auch ihr das Maß eurer Väter!
\par 33 Ihr Schlangen und Otterngezücht! wie wollt ihr der höllischen Verdammnis entrinnen?
\par 34 Darum siehe, ich sende zu euch Propheten und Weise und Schriftgelehrte; und deren werdet ihr etliche töten und kreuzigen, und etliche werdet ihr geißeln in ihren Schulen und werdet sie verfolgen von einer Stadt zu der anderen;
\par 35 auf daß über euch komme all das gerechte Blut, das vergossen ist auf Erden, von dem Blut des gerechten Abel an bis auf das Blut des Zacharias, des Sohnes Berechja's, welchen ihr getötet habt zwischen dem Tempel und dem Altar.
\par 36 Wahrlich ich sage euch, daß solches alles wird über dies Geschlecht kommen.
\par 37 Jerusalem, Jerusalem, die du tötest die Propheten und steinigst, die zu dir gesandt sind! wie oft habe ich deine Kinder versammeln wollen, wie eine Henne versammelt ihre Küchlein unter ihre Flügel; und ihr habt nicht gewollt!
\par 38 Siehe, euer Haus soll euch wüst gelassen werden.
\par 39 Denn ich sage euch: Ihr werdet mich von jetzt an nicht sehen, bis ihr sprecht: Gelobt sei, der da kommt im Namen des HERRN!

\chapter{24}

\par 1 Und Jesus ging hinweg von dem Tempel, und seine Jünger traten zu ihm, daß sie ihm zeigten des Tempels Gebäude.
\par 2 Jesus aber sprach zu ihnen: Sehet ihr nicht das alles? Wahrlich, ich sage euch: Es wird hier nicht ein Stein auf dem anderen bleiben, der nicht zerbrochen werde.
\par 3 Und als er auf dem Ölberge saß, traten zu ihm seine Jünger besonders und sprachen: Sage uns, wann wird das alles geschehen? Und welches wird das Zeichen sein deiner Zukunft und des Endes der Welt?
\par 4 Jesus aber antwortete und sprach zu ihnen: Sehet zu, daß euch nicht jemand verführe.
\par 5 Denn es werden viele kommen unter meinem Namen, und sagen: "Ich bin Christus" und werden viele verführen.
\par 6 Ihr werdet hören Kriege und Geschrei von Kriegen; sehet zu und erschreckt euch nicht. Das muß zum ersten alles geschehen; aber es ist noch nicht das Ende da.
\par 7 Denn es wird sich empören ein Volk wider das andere und ein Königreich gegen das andere, und werden sein Pestilenz und teure Zeit und Erdbeben hin und wieder.
\par 8 Da wird sich allererst die Not anheben.
\par 9 Alsdann werden sie euch überantworten in Trübsal und werden euch töten. Und ihr müßt gehaßt werden um meines Namens willen von allen Völkern.
\par 10 Dann werden sich viele ärgern und werden untereinander verraten und werden sich untereinander hassen.
\par 11 Und es werden sich viel falsche Propheten erheben und werden viele verführen.
\par 12 und dieweil die Ungerechtigkeit wird überhandnehmen, wird die Liebe in vielen erkalten.
\par 13 Wer aber beharret bis ans Ende, der wird selig.
\par 14 Und es wird gepredigt werden das Evangelium vom Reich in der ganzen Welt zu einem Zeugnis über alle Völker, und dann wird das Ende kommen.
\par 15 Wenn ihr nun sehen werdet den Greuel der Verwüstung (davon gesagt ist durch den Propheten Daniel), daß er steht an der heiligen Stätte (wer das liest, der merke darauf!),
\par 16 alsdann fliehe auf die Berge, wer im jüdischen Lande ist;
\par 17 und wer auf dem Dach ist, der steige nicht hernieder, etwas aus seinem Hause zu holen;
\par 18 und wer auf dem Felde ist, der kehre nicht um, seine Kleider zu holen.
\par 19 Weh aber den Schwangeren und Säugerinnen zu der Zeit!
\par 20 Bittet aber, daß eure Flucht nicht geschehe im Winter oder am Sabbat.
\par 21 Denn es wird alsbald eine große Trübsal sein, wie nicht gewesen ist von Anfang der Welt bisher und wie auch nicht werden wird.
\par 22 Und wo diese Tage nicht verkürzt würden, so würde kein Mensch selig; aber um der Auserwählten willen werden die Tage verkürzt.
\par 23 So alsdann jemand zu euch wird sagen: Siehe, hier ist Christus! oder: da! so sollt ihr's nicht glauben.
\par 24 Denn es werden falsche Christi und falsche Propheten aufstehen und große Zeichen und Wunder tun, daß verführt werden in dem Irrtum (wo es möglich wäre) auch die Auserwählten.
\par 25 Siehe, ich habe es euch zuvor gesagt.
\par 26 Darum, wenn sie zu euch sagen werden: Siehe, er ist in der Wüste! so gehet nicht hinaus, -siehe, er ist in der Kammer! so glaubt nicht.
\par 27 Denn gleichwie ein Blitz ausgeht vom Aufgang und scheint bis zum Niedergang, also wird auch sein die Zukunft des Menschensohnes.
\par 28 Wo aber ein Aas ist, da sammeln sich die Adler.
\par 29 Bald aber nach der Trübsal derselben Zeit werden Sonne und Mond den Schein verlieren, und Sterne werden vom Himmel fallen, und die Kräfte der Himmel werden sich bewegen.
\par 30 Und alsdann wird erscheinen das Zeichen des Menschensohnes am Himmel. Und alsdann werden heulen alle Geschlechter auf Erden und werden sehen kommen des Menschen Sohn in den Wolken des Himmels mit großer Kraft und Herrlichkeit.
\par 31 Und er wird senden seine Engel mit hellen Posaunen, und sie werden sammeln seine Auserwählten von den vier Winden, von einem Ende des Himmels zu dem anderen.
\par 32 An dem Feigenbaum lernet ein Gleichnis: wenn sein Zweig jetzt saftig wird und Blätter gewinnt, so wißt ihr, daß der Sommer nahe ist.
\par 33 Also auch wenn ihr das alles sehet, so wisset, daß es nahe vor der Tür ist.
\par 34 Wahrlich ich sage euch: Dies Geschlecht wird nicht vergehen, bis daß dieses alles geschehe.
\par 35 Himmel und Erde werden vergehen; aber meine Worte werden nicht vergehen.
\par 36 Von dem Tage aber und von der Stunde weiß niemand, auch die Engel nicht im Himmel, sondern allein mein Vater.
\par 37 Aber gleichwie es zur Zeit Noah's war, also wird auch sein die Zukunft des Menschensohnes.
\par 38 Denn gleichwie sie waren in den Tagen vor der Sintflut, sie aßen, sie tranken, sie freiten und ließen sich freien, bis an den Tag, da Noah zu der Arche einging.
\par 39 und achteten's nicht, bis die Sintflut kam und nahm sie alle dahin, also wird auch sein die Zukunft des Menschensohnes.
\par 40 Dann werden zwei auf dem Felde sein; einer wird angenommen, und der andere wird verlassen werden.
\par 41 Zwei werden mahlen auf der Mühle; eine wird angenommen, und die andere wird verlassen werden.
\par 42 Darum wachet, denn ihr wisset nicht, welche Stunde euer HERR kommen wird.
\par 43 Das sollt ihr aber wissen: Wenn der Hausvater wüßte, welche Stunde der Dieb kommen wollte, so würde er ja wachen und nicht in sein Haus brechen lassen.
\par 44 Darum seid ihr auch bereit; denn des Menschen Sohn wird kommen zu einer Stunde, da ihr's nicht meinet.
\par 45 Welcher ist aber nun ein treuer und kluger Knecht, den der Herr gesetzt hat über sein Gesinde, daß er ihnen zu rechter Zeit Speise gebe?
\par 46 Selig ist der Knecht, wenn sein Herr kommt und findet ihn also tun.
\par 47 Wahrlich ich sage euch: Er wird ihn über alle seine Güter setzen.
\par 48 So aber jener, der böse Knecht, wird in seinem Herzen sagen: Mein Herr kommt noch lange nicht,
\par 49 und fängt an zu schlagen seine Mitknechte, ißt und trinkt mit den Trunkenen:
\par 50 so wird der Herr des Knechtes kommen an dem Tage, des er sich nicht versieht, und zu einer Stunde, die er nicht meint,
\par 51 und wird ihn zerscheitern und wird ihm den Lohn geben mit den Heuchlern: da wird sein Heulen und Zähneklappen.

\chapter{25}

\par 1 Dann wird das Himmelreich gleich sein zehn Jungfrauen, die ihre Lampen nahmen und gingen aus, dem Bräutigam entgegen.
\par 2 Aber fünf unter ihnen waren töricht, und fünf waren klug.
\par 3 Die törichten nahmen Öl in ihren Lampen; aber sie nahmen nicht Öl mit sich.
\par 4 Die klugen aber nahmen Öl in ihren Gefäßen samt ihren Lampen.
\par 5 Da nun der Bräutigam verzog, wurden sie alle schläfrig und schliefen ein.
\par 6 Zur Mitternacht aber ward ein Geschrei: Siehe, der Bräutigam kommt; geht aus ihm entgegen!
\par 7 Da standen diese Jungfrauen alle auf und schmückten ihre Lampen.
\par 8 Die törichten aber sprachen zu den klugen: Gebt uns von eurem Öl, denn unsere Lampen verlöschen.
\par 9 Da antworteten die klugen und sprachen: Nicht also, auf daß nicht uns und euch gebreche; geht aber hin zu den Krämern und kauft für euch selbst.
\par 10 Und da sie hingingen, zu kaufen, kam der Bräutigam; und die bereit waren, gingen mit ihm hinein zur Hochzeit, und die Tür ward verschlossen.
\par 11 Zuletzt kamen auch die anderen Jungfrauen und sprachen: Herr, Herr, tu uns auf!
\par 12 Er antwortete aber und sprach: Wahrlich ich sage euch: Ich kenne euch nicht.
\par 13 Darum wachet; denn ihr wisset weder Tag noch Stunde, in welcher des Menschen Sohn kommen wird.
\par 14 Gleichwie ein Mensch, der über Land zog, rief seine Knechte und tat ihnen seine Güter aus;
\par 15 und einem gab er fünf Zentner, dem andern zwei, dem dritten einen, einem jedem nach seinem Vermögen, und zog bald hinweg.
\par 16 Da ging der hin, der fünf Zentner empfangen hatte, und handelte mit ihnen und gewann andere fünf Zentner.
\par 17 Desgleichen, der zwei Zentner empfangen hatte, gewann auch zwei andere.
\par 18 Der aber einen empfangen hatte, ging hin und machte eine Grube in die Erde und verbarg seines Herrn Geld.
\par 19 Über eine lange Zeit kam der Herr dieser Knechte und hielt Rechenschaft mit ihnen.
\par 20 Da trat herzu, der fünf Zentner empfangen hatte, und legte andere fünf Zentner dar und sprach: Herr, du hast mir fünf Zentner ausgetan; siehe da, ich habe damit andere fünf Zentner gewonnen.
\par 21 Da sprach sein Herr zu ihm: Ei, du frommer und getreuer Knecht, du bist über wenigem getreu gewesen, ich will dich über viel setzen; gehe ein zu deines Herrn Freude!
\par 22 Da trat auch herzu, der zwei Zentner erhalten hatte, und sprach: Herr, du hast mir zwei Zentner gegeben; siehe da, ich habe mit ihnen zwei andere gewonnen.
\par 23 Sein Herr sprach zu ihm: Ei du frommer und getreuer Knecht, du bist über wenigem getreu gewesen, ich will dich über viel setzen; gehe ein zu deines Herrn Freude!
\par 24 Da trat auch herzu, der einen Zentner empfangen hatte, und sprach: Herr, ich wußte, das du ein harter Mann bist: du schneidest, wo du nicht gesät hast, und sammelst, wo du nicht gestreut hast;
\par 25 und fürchtete mich, ging hin und verbarg deinen Zentner in die Erde. Siehe, da hast du das Deine.
\par 26 Sein Herr aber antwortete und sprach zu ihm: Du Schalk und fauler Knecht! wußtest du, daß ich schneide, da ich nicht gesät habe, und sammle, da ich nicht gestreut habe?
\par 27 So solltest du mein Geld zu den Wechslern getan haben, und wenn ich gekommen wäre, hätte ich das Meine zu mir genommen mit Zinsen.
\par 28 Darum nehmt von ihm den Zentner und gebt es dem, der zehn Zentner hat.
\par 29 Denn wer da hat, dem wird gegeben werden, und er wird die Fülle haben; wer aber nicht hat, dem wird auch, was er hat, genommen werden.
\par 30 Und den unnützen Knecht werft hinaus in die Finsternis; da wird sein Heulen und Zähneklappen.
\par 31 Wenn aber des Menschen Sohn kommen wird in seiner Herrlichkeit und alle heiligen Engel mit ihm, dann wird er sitzen auf dem Stuhl seiner Herrlichkeit,
\par 32 und werden vor ihm alle Völker versammelt werden. Und er wird sie voneinander scheiden, gleich als ein Hirte die Schafe von den Böcken scheidet,
\par 33 und wird die Schafe zu seiner Rechten stellen und die Böcke zu seiner Linken.
\par 34 Da wird dann der König sagen zu denen zu seiner Rechten: Kommt her, ihr Gesegneten meines Vaters ererbt das Reich, das euch bereitet ist von Anbeginn der Welt!
\par 35 Denn ich bin hungrig gewesen, und ihr habt mich gespeist. Ich bin durstig gewesen, und ihr habt mich getränkt. Ich bin Gast gewesen, und ihr habt mich beherbergt.
\par 36 Ich bin nackt gewesen und ihr habt mich bekleidet. Ich bin krank gewesen, und ihr habt mich besucht. Ich bin gefangen gewesen, und ihr seid zu mir gekommen.
\par 37 Dann werden ihm die Gerechten antworten und sagen: Wann haben wir dich hungrig gesehen und haben dich gespeist? oder durstig und haben dich getränkt?
\par 38 Wann haben wir dich als einen Gast gesehen und beherbergt? oder nackt und dich bekleidet?
\par 39 Wann haben wir dich krank oder gefangen gesehen und sind zu dir gekommen?
\par 40 Und der König wird antworten und sagen zu ihnen: Wahrlich ich sage euch: Was ihr getan habt einem unter diesen meinen geringsten Brüdern, das habt ihr mir getan.
\par 41 Dann wird er auch sagen zu denen zur Linken: Gehet hin von mir, ihr Verfluchten, in das ewige Feuer, das bereitet ist dem Teufel und seinen Engeln!
\par 42 Ich bin hungrig gewesen, und ihr habt mich nicht gespeist. Ich bin durstig gewesen, und ihr habt mich nicht getränkt.
\par 43 Ich bin ein Gast gewesen, und ihr habt mich nicht beherbergt. Ich bin nackt gewesen, und ihr habt mich nicht bekleidet. Ich bin krank und gefangen gewesen, und ihr habt mich nicht besucht.
\par 44 Da werden sie ihm antworten und sagen: HERR, wann haben wir dich gesehen hungrig oder durstig oder als einen Gast oder nackt oder krank oder gefangen und haben dir nicht gedient?
\par 45 Dann wird er ihnen antworten und sagen: Wahrlich ich sage euch: Was ihr nicht getan habt einem unter diesen Geringsten, das habt ihr mir auch nicht getan.
\par 46 Und sie werden in die ewige Pein gehen, aber die Gerechten in das ewige Leben.

\chapter{26}

\par 1 Und es begab sich, da Jesus alle diese Reden vollendet hatte, sprach er zu seinen Jüngern:
\par 2 Ihr wisset, daß nach zwei Tagen Ostern wird; und des Menschen Sohn wird überantwortet werden, daß er gekreuzigt werde.
\par 3 Da versammelten sich die Hohenpriester und Schriftgelehrten und die Ältesten im Volk in den Palast des Hohenpriesters, der da hieß Kaiphas,
\par 4 und hielten Rat, wie sie Jesus mit List griffen und töteten.
\par 5 Sie sprachen aber: Ja nicht auf das Fest, auf daß nicht ein Aufruhr werde im Volk!
\par 6 Da nun Jesus war zu Bethanien im Hause Simons, des Aussätzigen,
\par 7 da trat zu ihm ein Weib, das hatte ein Glas mit köstlichem Wasser und goß es auf sein Haupt, da er zu Tische saß.
\par 8 Da das seine Jünger sahen, wurden sie unwillig und sprachen: Wozu diese Vergeudung?
\par 9 Dieses Wasser hätte mögen teuer verkauft und den Armen gegeben werden.
\par 10 Da das Jesus merkte, sprach er zu ihnen: Was bekümmert ihr das Weib? Sie hat ein gutes Werk an mir getan.
\par 11 Ihr habt allezeit Arme bei euch; mich aber habt ihr nicht allezeit.
\par 12 Daß sie dies Wasser hat auf meinen Leib gegossen, hat sie getan, daß sie mich zum Grabe bereite.
\par 13 Wahrlich ich sage euch: Wo dies Evangelium gepredigt wird in der ganzen Welt, da wird man auch sagen zu ihrem Gedächtnis, was sie getan hat.
\par 14 Da ging hin der Zwölf einer, mit Namen Judas Ischariot, zu den Hohenpriestern
\par 15 und sprach: Was wollt ihr mir geben? Ich will ihn euch verraten. Und sie boten ihm dreißig Silberlinge.
\par 16 Und von dem an suchte er Gelegenheit, daß er ihn verriete.
\par 17 Aber am ersten Tag der süßen Brote traten die Jünger zu Jesus und sprachen zu ihm: Wo willst du, daß wir dir bereiten das Osterlamm zu essen?
\par 18 Er sprach: Gehet hin in die Stadt zu einem und sprecht der Meister läßt dir sagen: Meine Zeit ist nahe; ich will bei dir Ostern halten mit meinen Jüngern.
\par 19 Und die Jünger taten wie ihnen Jesus befohlen hatte, und bereiteten das Osterlamm.
\par 20 Und am Abend setzte er sich zu Tische mit den Zwölfen.
\par 21 Und da sie aßen, sprach er: Wahrlich ich sage euch: Einer unter euch wird mich verraten.
\par 22 Und sie wurden sehr betrübt und hoben an, ein jeglicher unter ihnen, und sagten zu ihm: HERR, bin ich's?
\par 23 Er antwortete und sprach: Der mit der Hand mit mir in die Schüssel tauchte, der wird mich verraten.
\par 24 Des Menschen Sohn geht zwar dahin, wie von ihm geschrieben steht; doch weh dem Menschen, durch welchen des Menschen Sohn verraten wird! Es wäre ihm besser, daß er nie geboren wäre.
\par 25 Da antwortete Judas, der ihn verriet, und sprach: Bin ich's Rabbi? Er sprach zu ihm: Du sagst es.
\par 26 Da sie aber aßen, nahm Jesus das Brot, dankte und brach's und gab's den Jüngern und sprach: Nehmet, esset; das ist mein Leib.
\par 27 Und er nahm den Kelch und dankte, gab ihnen den und sprach: Trinket alle daraus;
\par 28 das ist mein Blut des neuen Testaments, welches vergossen wird für viele zur Vergebung der Sünden.
\par 29 Ich sage euch: Ich werde von nun an nicht mehr von diesen Gewächs des Weinstocks trinken bis an den Tag, da ich's neu trinken werde mit euch in meines Vaters Reich.
\par 30 Und da sie den Lobgesang gesprochen hatte, gingen sie hinaus an den Ölberg.
\par 31 Da sprach Jesus zu ihnen: In dieser Nacht werdet ihr euch alle ärgern an mir. Denn es steht geschrieben: "Ich werde den Hirten schlagen, und die Schafe der Herde werden sich zerstreuen."
\par 32 Wenn ich aber auferstehe, will ich vor euch hingehen nach Galiläa.
\par 33 Petrus aber antwortete und sprach zu ihm: Wenn sich auch alle an dir ärgerten, so will ich doch mich nimmermehr ärgern.
\par 34 Jesus sprach zu ihm: Wahrlich ich sage dir: In dieser Nacht, ehe der Hahn kräht, wirst du mich dreimal verleugnen.
\par 35 Petrus sprach zu ihm: Und wenn ich mit dir sterben müßte, so will ich dich nicht verleugnen. Desgleichen sagten auch alle Jünger.
\par 36 Da kam Jesus mit ihnen zu einem Hofe, der hieß Gethsemane, und sprach zu seinen Jüngern: Setzet euch hier, bis daß ich dorthin gehe und bete.
\par 37 Und nahm zu sich Petrus und die zwei Söhne des Zebedäus und fing an zu trauern und zu zagen.
\par 38 Da sprach Jesus zu ihnen: Meine Seele ist betrübt bis an den Tod; bleibet hier und wachet mit mir!
\par 39 Und ging hin ein wenig, fiel nieder auf sein Angesicht und betete und sprach: Mein Vater, ist's möglich, so gehe dieser Kelch von mir; doch nicht, wie ich will, sondern wie du willst!
\par 40 Und er kam zu seinen Jüngern und fand sie schlafend und sprach zu Petrus: Könnet ihr denn nicht eine Stunde mit mir wachen?
\par 41 Wachet und betet, daß ihr nicht in Anfechtung fallet! Der Geist ist willig; aber das Fleisch ist schwach.
\par 42 Zum andernmal ging er wieder hin, betete und sprach: Mein Vater, ist's nicht möglich, daß dieser Kelch von mir gehe, ich trinke ihn denn, so geschehe dein Wille!
\par 43 Und er kam und fand sie abermals schlafend, und ihre Augen waren voll Schlafs.
\par 44 Und er ließ sie und ging abermals hin und betete zum drittenmal und redete dieselben Worte.
\par 45 Da kam er zu seinen Jüngern und sprach zu ihnen: Ach wollt ihr nur schlafen und ruhen? Siehe, die Stunde ist hier, daß des Menschen Sohn in der Sünder Hände überantwortet wird.
\par 46 Stehet auf, laßt uns gehen! Siehe, er ist da, der mich verrät!
\par 47 Und als er noch redete, siehe, da kam Judas, der Zwölf einer, und mit ihm eine große Schar, mit Schwertern und mit Stangen, von den Hohenpriestern und Ältesten des Volks.
\par 48 Und der Verräter hatte ihnen ein Zeichen gegeben und gesagt: Welchen ich küssen werde, der ist's; den greifet.
\par 49 Und alsbald trat er zu Jesus und sprach: Gegrüßet seist du, Rabbi! und küßte ihn.
\par 50 Jesus aber sprach zu ihm: Mein Freund, warum bist du gekommen? Da traten sie hinzu und legten die Hände an Jesus und griffen ihn.
\par 51 Und siehe, einer aus denen, die mit Jesus waren, reckte die Hand aus und zog sein Schwert aus und schlug des Hohenpriesters Knecht und hieb ihm ein Ohr ab.
\par 52 Da sprach Jesus zu ihm; Stecke dein Schwert an seinen Ort! denn wer das Schwert nimmt, der soll durchs Schwert umkommen.
\par 53 Oder meinst du, daß ich nicht könnte meinen Vater bitten, daß er mir zuschickte mehr denn zwölf Legionen Engel?
\par 54 Wie würde aber die Schrift erfüllet? Es muß also gehen.
\par 55 Zu der Stunde sprach Jesus zu den Scharen: Ihr seid ausgegangen wie zu einem Mörder, mit Schwertern und Stangen, mich zu fangen. Bin ich doch täglich gesessen bei euch und habe gelehrt im Tempel, und ihr habt mich nicht gegriffen.
\par 56 Aber das ist alles geschehen, daß erfüllet würden die Schriften der Propheten. Da verließen ihn die Jünger und flohen.
\par 57 Die aber Jesus gegriffen hatten, führten ihn zu dem Hohenpriester Kaiphas, dahin die Schriftgelehrten und Ältesten sich versammelt hatten.
\par 58 Petrus aber folgte ihm nach von ferne bis in den Palast des Hohenpriesters und ging hinein und setzte sich zu den Knechten, auf daß er sähe, wo es hinaus wollte.
\par 59 Die Hohenpriester aber und die Ältesten und der ganze Rat suchten falsch Zeugnis gegen Jesus, auf daß sie ihn töteten,
\par 60 und fanden keins. Und wiewohl viel falsche Zeugen herzutraten, fanden sie doch keins. Zuletzt traten herzu zwei falsche Zeugen
\par 61 und sprachen: Er hat gesagt: Ich kann den Tempel Gottes abbrechen und in drei Tagen ihn bauen.
\par 62 Und der Hohepriester stand auf und sprach zu ihm: Antwortest du nichts zu dem, was diese wider dich zeugen?
\par 63 Aber Jesus schwieg still. Und der Hohepriester antwortete und sprach zu ihm: Ich beschwöre dich bei dem lebendigen Gott, daß du uns sagest, ob du seist Christus, der Sohn Gottes.
\par 64 Jesus sprach zu ihm: Du sagst es. Doch ich sage euch: Von nun an wird's geschehen, daß ihr werdet sehen des Menschen Sohn sitzen zur Rechten der Kraft und kommen in den Wolken des Himmels.
\par 65 Da zerriß der Hohepriester seine Kleider und sprach: Er hat Gott gelästert! Was bedürfen wir weiteres Zeugnis? Siehe, jetzt habt ihr seine Gotteslästerung gehört.
\par 66 Was dünkt euch? Sie antworteten und sprachen: Er ist des Todes schuldig!
\par 67 Da spieen sie aus in sein Angesicht und schlugen ihn mit Fäusten. Etliche aber schlugen ihn ins Angesicht
\par 68 und sprachen: Weissage uns, Christe, wer ist's, der dich schlug?
\par 69 Petrus aber saß draußen im Hof; und es trat zu ihm eine Magd und sprach: Und du warst auch mit dem Jesus aus Galiläa.
\par 70 Er leugnete aber vor ihnen allen und sprach: Ich weiß nicht, was du sagst.
\par 71 Als er aber zur Tür hinausging, sah ihn eine andere und sprach zu denen, die da waren: Dieser war auch mit dem Jesus von Nazareth.
\par 72 Und er leugnete abermals und schwur dazu: Ich kenne den Menschen nicht.
\par 73 Und über eine kleine Weile traten die hinzu, die dastanden, und sprachen zu Petrus: Wahrlich du bist auch einer von denen; denn deine Sprache verrät dich.
\par 74 Da hob er an sich zu verfluchen und zu schwören: Ich kenne diesen Menschen nicht. Uns alsbald krähte der Hahn.
\par 75 Da dachte Petrus an die Worte Jesu, da er zu ihm sagte: "Ehe der Hahn krähen wird, wirst du mich dreimal verleugnen", und ging hinaus und weinte bitterlich.

\chapter{27}

\par 1 Des Morgens aber hielten alle Hohenpriester und die Ältesten des Volks einen Rat über Jesus, daß sie ihn töteten.
\par 2 Und banden ihn, führten ihn hin und überantworteten ihn dem Landpfleger Pontius Pilatus.
\par 3 Da das sah Judas, der ihn verraten hatte, daß er verdammt war zum Tode, gereute es ihn, und brachte wieder die dreißig Silberlinge den Hohenpriestern und den Ältesten
\par 4 und sprach: Ich habe übel getan, daß ich unschuldig Blut verraten habe.
\par 5 Sie sprachen: Was geht uns das an? Da siehe du zu! Und er warf die Silberlinge in den Tempel, hob sich davon, ging hin und erhängte sich selbst.
\par 6 Aber die Hohenpriester nahmen die Silberlinge und sprachen: Es taugt nicht, daß wir sie in den Gotteskasten legen, denn es ist Blutgeld.
\par 7 Sie hielten aber einen Rat und kauften den Töpfersacker darum zum Begräbnis der Pilger.
\par 8 Daher ist dieser Acker genannt der Blutacker bis auf den heutigen Tag.
\par 9 Da ist erfüllt, was gesagt ist durch den Propheten Jeremia, da er spricht: "Sie haben genommen dreißig Silberlinge, damit bezahlt war der Verkaufte, welchen sie kauften von den Kindern Israel,
\par 10 und haben sie gegeben um den Töpfersacker, wie mir der HERR befohlen hat."
\par 11 Jesus aber stand vor dem Landpfleger; und der Landpfleger fragte ihn und sprach: Bist du der Juden König? Jesus aber sprach zu ihm: Du sagst es.
\par 12 Und da er verklagt ward von den Hohenpriestern und Ältesten, antwortete er nicht.
\par 13 Da sprach Pilatus zu ihm: Hörst du nicht, wie hart sie dich verklagen?
\par 14 Und er antwortete ihm nicht auf ein Wort, also daß der Landpfleger sich verwunderte.
\par 15 Auf das Fest aber hatte der Landpfleger die Gewohnheit, dem Volk einen Gefangenen loszugeben, welchen sie wollten.
\par 16 Er hatte aber zu der Zeit einen Gefangenen, einen sonderlichen vor anderen, der hieß Barabbas.
\par 17 Und da sie versammelt waren, sprach Pilatus zu ihnen: Welchen wollt ihr, daß ich euch losgebe? Barabbas oder Jesus, von dem gesagt wird, er sei Christus?
\par 18 Denn er wußte wohl, daß sie ihn aus Neid überantwortet hatten.
\par 19 Und da er auf dem Richtstuhl saß, schickte sein Weib zu ihm und ließ ihm sagen: Habe du nichts zu schaffen mit diesem Gerechten; ich habe heute viel erlitten im Traum seinetwegen.
\par 20 Aber die Hohenpriester und die Ältesten überredeten das Volk, daß sie um Barabbas bitten sollten und Jesus umbrächten.
\par 21 Da antwortete nun der Landpfleger und sprach zu ihnen: Welchen wollt ihr unter diesen zweien, den ich euch soll losgeben? Sie sprachen: Barabbas.
\par 22 Pilatus sprach zu ihnen: Was soll ich denn machen mit Jesus, von dem gesagt wird er sei Christus? Sie sprachen alle: Laß ihn kreuzigen!
\par 23 Der Landpfleger sagte: Was hat er denn Übles getan? Sie schrieen aber noch mehr und sprachen: Laß ihn kreuzigen!
\par 24 Da aber Pilatus sah, daß er nichts schaffte, sondern daß ein viel größer Getümmel ward, nahm er Wasser und wusch die Hände vor dem Volk und sprach: Ich bin unschuldig an dem Blut dieses Gerechten, sehet ihr zu!
\par 25 Da antwortete das ganze Volk und sprach: Sein Blut komme über uns und unsere Kinder.
\par 26 Da gab er ihnen Barabbas los; aber Jesus ließ er geißeln und überantwortete ihn, daß er gekreuzigt würde.
\par 27 Da nahmen die Kriegsknechte des Landpflegers Jesus zu sich in das Richthaus und sammelten über ihn die ganze Schar
\par 28 und zogen ihn aus und legten ihm einen Purpurmantel an
\par 29 und flochten eine Dornenkrone und setzten sie auf sein Haupt und ein Rohr in seine rechte Hand und beugten die Kniee vor ihm und verspotteten ihn und sprachen: Gegrüßet seist du, der Juden König!
\par 30 und spieen ihn an und nahmen das Rohr und schlugen damit sein Haupt.
\par 31 Und da sie ihn verspottet hatten, zogen sie ihm seine Kleider an und führten ihn hin, daß sie ihn kreuzigten.
\par 32 Und indem sie hinausgingen, fanden sie einen Menschen von Kyrene mit Namen Simon; den zwangen sie, daß er ihm sein Kreuz trug.
\par 33 Und da sie an die Stätte kamen mit Namen Golgatha, das ist verdeutscht Schädelstätte,
\par 34 gaben sie ihm Essig zu trinken mit Galle vermischt; und da er's schmeckte, wollte er nicht trinken.
\par 35 Da sie ihn aber gekreuzigt hatten, teilten sie seine Kleider und warfen das Los darum, auf daß erfüllet würde, was gesagt ist durch den Propheten: "Sie haben meine Kleider unter sich geteilt, und über mein Gewand haben sie das Los geworfen."
\par 36 Und sie saßen allda und hüteten sein.
\par 37 Und oben zu seinen Häupten setzten sie die Ursache seines Todes, und war geschrieben: Dies ist Jesus, der Juden König.
\par 38 Und da wurden zwei Mörder mit ihm gekreuzigt, einer zur Rechten und einer zur Linken.
\par 39 Die aber vorübergingen, lästerten ihn und schüttelten ihre Köpfe
\par 40 und sprachen: Der du den Tempel Gottes zerbrichst und baust ihn in drei Tagen, hilf dir selber! Bist du Gottes Sohn, so steig herab von Kreuz.
\par 41 Desgleichen auch die Hohenpriester spotteten sein samt den Schriftgelehrten und Ältesten und sprachen:
\par 42 Andern hat er geholfen, und kann sich selber nicht helfen. Ist er der König Israels, so steige er nun vom Kreuz, so wollen wir ihm glauben.
\par 43 Er hat Gott vertraut; der erlöse ihn nun, hat er Lust zu ihm; denn er hat gesagt: Ich bin Gottes Sohn.
\par 44 Desgleichen schmähten ihn auch die Mörder, die mit ihm gekreuzigt waren.
\par 45 Und von der sechsten Stunde an ward eine Finsternis über das ganze Land bis zu der neunten Stunde.
\par 46 Und um die neunte Stunde schrie Jesus laut und sprach: Eli, Eli, lama asabthani? das heißt: Mein Gott, mein Gott, warum hast du mich verlassen?
\par 47 Etliche aber, die dastanden, da sie das hörten, sprachen sie: Der ruft den Elia.
\par 48 Und alsbald lief einer unter ihnen, nahm einen Schwamm und füllte ihn mit Essig und steckte ihn an ein Rohr und tränkte ihn.
\par 49 Die andern aber sprachen: Halt, laß sehen, ob Elia komme und ihm helfe.
\par 50 Aber Jesus schrie abermals laut und verschied.
\par 51 Und siehe da, der Vorhang im Tempel zerriß in zwei Stücke von obenan bis untenaus.
\par 52 Und die Erde erbebte, und die Felsen zerrissen, die Gräber taten sich auf, und standen auf viele Leiber der Heiligen, die da schliefen,
\par 53 und gingen aus den Gräbern nach seiner Auferstehung und kamen in die heilige Stadt und erschienen vielen.
\par 54 Aber der Hauptmann und die bei ihm waren und bewahrten Jesus, da sie sahen das Erdbeben und was da geschah, erschraken sie sehr und sprachen: Wahrlich dieser ist Gottes Sohn gewesen!
\par 55 Und es waren viele Weiber da, die von ferne zusahen, die da Jesus waren nachgefolgt aus Galiläa und hatten ihm gedient;
\par 56 unter welchen war Maria Magdalena und Maria, die Mutter der Kinder des Zebedäus.
\par 57 Am Abend aber kam ein reicher Mann von Arimathia, der hieß Joseph, welcher auch ein Jünger Jesu war.
\par 58 Der ging zu Pilatus und bat ihn um den Leib Jesus. Da befahl Pilatus man sollte ihm ihn geben.
\par 59 Und Joseph nahm den Leib und wickelte ihn in eine reine Leinwand
\par 60 und legte ihn in sein eigenes Grab, welches er hatte lassen in einen Fels hauen, und wälzte einen großen Stein vor die Tür des Grabes und ging davon.
\par 61 Es war aber allda Maria Magdalena und die andere Maria, die setzten sich gegen das Grab.
\par 62 Des andern Tages, der da folgt nach dem Rüsttage, kamen die Hohenpriester und Pharisäer sämtlich zu Pilatus
\par 63 und sprachen: Herr, wir haben gedacht, daß dieser Verführer sprach, da er noch lebte: Ich will nach drei Tagen auferstehen.
\par 64 Darum befiehl, daß man das Grab verwahre bis an den dritten Tag, auf daß nicht seine Jünger kommen und stehlen ihn und sagen dem Volk: Er ist auferstanden von den Toten, und werde der letzte Betrug ärger denn der erste.
\par 65 Pilatus sprach zu ihnen: Da habt ihr die Hüter; gehet hin und verwahret, wie ihr wisset.
\par 66 Sie gingen hin und verwahrten das Grab mit Hütern und versiegelten den Stein.

\chapter{28}

\par 1 Als aber der Sabbat um war und der erste Tag der Woche anbrach, kam Maria Magdalena und die andere Maria, das Grab zu besehen.
\par 2 Und siehe, es geschah ein großes Erdbeben. Denn der Engel des HERRN kam vom Himmel herab, trat hinzu und wälzte den Stein von der Tür und setzte sich darauf.
\par 3 Und seine Gestalt war wie der Blitz und sein Kleid weiß wie Schnee.
\par 4 Die Hüter aber erschraken vor Furcht und wurden, als wären sie tot.
\par 5 Aber der Engel antwortete und sprach zu den Weibern: Fürchtet euch nicht! Ich weiß, daß ihr Jesus, den Gekreuzigten, sucht.
\par 6 Er ist nicht hier; er ist auferstanden, wie er gesagt hat. Kommt her und seht die Stätte, da der HERR gelegen hat.
\par 7 Und gehet eilend hin und sagt es seinen Jüngern, daß er auferstanden sei von den Toten. Und siehe, er wird vor euch hingehen nach Galiläa; da werdet ihr ihn sehen. Siehe, ich habe es euch gesagt.
\par 8 Und sie gingen eilend zum Grabe hinaus mit Furcht und großer Freude und liefen, daß sie es seinen Jüngern verkündigten. Und da sie gingen seinen Jüngern zu verkündigen,
\par 9 siehe, da begegnete ihnen Jesus und sprach: Seid gegrüßet! Und sie traten zu ihm und griffen an seine Füße und fielen vor ihm nieder.
\par 10 Da sprach Jesus zu ihnen: Fürchtet euch nicht! Geht hin und verkündigt es meinen Brüdern, daß sie gehen nach Galiläa; daselbst werden sie mich sehen.
\par 11 Da sie aber hingingen, siehe, da kamen etliche von den Hütern in die Stadt und verkündigten den Hohenpriestern alles, was geschehen war.
\par 12 Und sie kamen zusammen mit den Ältesten und hielten einen Rat und gaben den Kriegsknechten Geld genug
\par 13 und sprachen: Saget: Seine Jünger kamen des Nachts und stahlen ihn, dieweil wir schliefen.
\par 14 Und wo es würde auskommen bei dem Landpfleger, wollen wir ihn stillen und schaffen, daß ihr sicher seid.
\par 15 Und sie nahmen das Geld und taten, wie sie gelehrt waren. Solches ist eine gemeine Rede geworden bei den Juden bis auf den heutigen Tag.
\par 16 Aber die elf Jünger gingen nach Galiläa auf einen Berg, dahin Jesus sie beschieden hatte.
\par 17 Und da sie ihn sahen, fielen sie vor ihm nieder; etliche aber zweifelten.
\par 18 Und Jesus trat zu ihnen, redete mit ihnen und sprach: Mir ist gegeben alle Gewalt im Himmel und auf Erden.
\par 19 Darum gehet hin und lehret alle Völker und taufet sie im Namen des Vaters und des Sohnes und des heiligen Geistes,
\par 20 und lehret sie halten alles, was ich euch befohlen habe. Und siehe, ich bin bei euch alle Tage bis an der Welt Ende.

\end{document}