\begin{document}

\title{Matthew}



\chapter{1}

\par 1 Het boek des geslachts van JEZUS CHRISTUS, den Zoon van David, den zoon van Abraham.
\par 2 Abraham gewon Izak, en Izak gewon Jakob, en Jakob gewon Juda, en zijn broeders;
\par 3 En Juda gewon Fares en Zara bij Thamar; en Fares gewon Esrom, en Esrom gewon Aram;
\par 4 En Aram gewon Aminadab, en Aminadab gewon Nahasson, en Nahasson gewon Salmon;
\par 5 En Salmon gewon Booz bij Rachab, en Booz gewon Obed bij Ruth, en Obed gewon Jessai;
\par 6 En Jessai gewon David, den koning; en David, den koning, gewon Salomon bij degene, die Uria's vrouw was geweest;
\par 7 En Salomon gewon Roboam, en Roboam gewon Abia, en Abia gewon Asa;
\par 8 En Asa gewon Josafat, en Josafat gewon Joram, en Joram gewon Ozias;
\par 9 En Ozias gewon Joatham, en Joatham gewon Achaz, en Achaz gewon Ezekias;
\par 10 En Ezekias gewon Manasse, en Manasse gewon Amon, en Amon gewon Josias;
\par 11 En Josias gewon Jechonias, en zijn broeders, omtrent de Babylonische overvoering.
\par 12 En na de Babylonische overvoering gewon Jechonias Salathiel, en Salathiel gewon Zorobabel;
\par 13 En Zorobabel gewon Abiud, en Abiud gewon Eljakim, en Eljakim gewon Azor;
\par 14 En Azor gewon Sadok, en Sadok gewon Achim, en Achim gewon Elihud;
\par 15 En Elihud gewon Eleazar, en Eleazar gewon Matthan, en Matthan gewon Jakob;
\par 16 En Jakob gewon Jozef, den man van Maria, uit welke geboren is JEZUS, gezegd Christus.
\par 17 Al de geslachten dan, van Abraham tot David, zijn veertien geslachten; en van David tot de Babylonische overvoering, zijn veertien geslachten; en van de Babylonische overvoering tot Christus, zijn veertien geslachten.
\par 18 De geboorte van Jezus Christus was nu aldus; want als Maria, zijn moeder, met Jozef ondertrouwd was, eer zij samengekomen waren, werd zij zwanger bevonden uit den Heiligen Geest.
\par 19 Jozef nu, haar man, alzo hij rechtvaardig was, en haar niet wilde openbaarlijk te schande maken, was van wil haar heimelijk te verlaten.
\par 20 En alzo hij deze dingen in den zin had, ziet, de engel des Heeren verscheen hem in den droom, zeggende: Jozef, gij zone Davids! wees niet bevreesd Maria, uw vrouw, tot u te nemen; want hetgeen in haar ontvangen is, dat is uit den Heiligen Geest;
\par 21 En zij zal een Zoon baren, en gij zult Zijn naam heten JEZUS; want Hij zal Zijn volk zalig maken van hun zonden.
\par 22 En dit alles is geschied, opdat vervuld zou worden, hetgeen van den Heere gesproken is, door den profeet, zeggende:
\par 23 Ziet, de maagd zal zwanger worden, en een Zoon baren, en gij zult Zijn naam heten Emmanuel; hetwelk is, overgezet zijnde, God met ons.
\par 24 Jozef dan, opgewekt zijnde van den slaap, deed, gelijk de engel des Heeren hem bevolen had, en heeft zijn vrouw tot zich genomen;
\par 25 En bekende haar niet, totdat zij dezen haar eerstgeboren Zoon gebaard had; en heette Zijn naam JEZUS.

\chapter{2}

\par 1 Toen nu Jezus geboren was te Bethlehem, gelegen in Judea, in de dagen van den koning Herodes, ziet, enige wijzen van het Oosten zijn te Jeruzalem aangekomen.
\par 2 Zeggende: Waar is de geboren Koning der Joden? want wij hebben gezien Zijn ster in het Oosten, en zijn gekomen om Hem te aanbidden.
\par 3 De koning Herodes nu, dit gehoord hebbende, werd ontroerd, en geheel Jeruzalem, met hem.
\par 4 En bijeenvergaderd hebbende al de overpriesters en Schriftgeleerden des volks, vraagde van hen, waar de Christus zou geboren worden.
\par 5 En zij zeiden tot hem: Te Bethlehem, in Judea gelegen; want alzo is geschreven door den profeet:
\par 6 En gij Bethlehem, gij land Juda! zijt geenszins de minste onder de vorsten van Juda; want uit u zal de Leidsman voortkomen, Die Mijn volk Israel weiden zal.
\par 7 Toen heeft Herodes de wijzen heimelijk geroepen, en vernam naarstiglijk van hen den tijd, wanneer de ster verschenen was;
\par 8 En hen naar Bethlehem zendende, zeide: Reist heen, en onderzoekt naarstiglijk naar dat Kindeken, en als gij Het zult gevonden hebben, boodschapt het mij, opdat ik ook kome en Datzelve aanbidde.
\par 9 En zij, den koning gehoord hebbende, zijn heengereisd; en ziet, de ster, die zij in het oosten gezien hadden, ging hun voor, totdat zij kwam en stond boven de plaats, waar het Kindeken was.
\par 10 Als zij nu de ster zagen, verheugden zij zich met zeer grote vreugde.
\par 11 En in het huis gekomen zijnde, vonden zij het Kindeken met Maria, Zijn moeder, en nedervallende hebben zij Hetzelve aangebeden; en hun schatten opengedaan hebbende, brachten zij Hem geschenken: goud en wierook, en mirre.
\par 12 En door Goddelijke openbaring vermaand zijnde in den droom, dat zij niet zouden wederkeren tot Herodes, vertrokken zij door een anderen weg weder naar hun land.
\par 13 Toen zij nu vertrokken waren, ziet, de engel des Heeren verschijnt Jozef in den droom, zeggende: Sta op, en neem tot u het Kindeken en Zijn moeder, en vlied in Egypte, en wees aldaar, totdat ik het u zeggen zal; want Herodes zal het Kindeken zoeken, om Hetzelve te doden.
\par 14 Hij dan opgestaan zijnde, nam het Kindeken en Zijn moeder tot zich in den nacht, en vertrok naar Egypte;
\par 15 En was aldaar tot den dood van Herodes; opdat vervuld zou worden hetgeen van den Heere gesproken is door den profeet, zeggende: Uit Egypte heb Ik Mijn Zoon geroepen.
\par 16 Als Herodes zag, dat hij van de wijzen bedrogen was, toen werd hij zeer toornig, en enigen afgezonden hebbende, heeft omgebracht al de kinderen, die binnen Bethlehem, en in al deszelfs landpalen waren, van twee jaren oud en daaronder, naar den tijd, dien hij van de wijzen naarstiglijk onderzocht had.
\par 17 Toen is vervuld geworden, hetgeen gesproken is door den profeet Jeremia, zeggende:
\par 18 Een stem is in Rama gehoord, geklag, geween en veel gekerm; Rachel beweende haar kinderen, en wilde niet vertroost wezen, omdat zij niet zijn!
\par 19 Toen Herodes nu gestorven was, ziet, de engel des Heeren verschijnt Jozef in den droom, in Egypte.
\par 20 Zeggende: Sta op, neem het Kindeken en Zijn moeder tot u, en trek in het land Israels; want zij zijn gestorven, die de ziel van het Kindeken zochten.
\par 21 Hij dan, opgestaan zijnde, heeft tot zich genomen het Kindeken en Zijn moeder, en is gekomen in het land Israels.
\par 22 Maar als hij hoorde, dat Archelaus in Judea koning was, in de plaats van zijn vader Herodes, vreesde hij daarheen te gaan; maar door Goddelijke openbaring vermaand in den droom, is hij vertrokken in de delen van Galilea.
\par 23 En daar gekomen zijnde, nam hij zijn woonplaats in de stad, genaamd Nazareth; opdat vervuld zou worden, wat door de profeten gezegd is, dat Hij Nazarener zal geheten worden.

\chapter{3}

\par 1 En in die dagen kwam Johannes de Doper, predikende in de woestijn van Judea,
\par 2 En zeggende: Bekeert u; want het Koninkrijk der hemelen is nabij gekomen.
\par 3 Want deze is het, van denwelken gesproken is door Jesaja, den profeet, zeggende: De stem des roependen in de woestijn: Bereidt den weg des Heeren, maakt Zijn paden recht!
\par 4 En dezelve Johannes had zijn kleding van kemelshaar, en een lederen gordel om zijn lenden; en zijn voedsel was sprinkhanen en wilde honig.
\par 5 Toen is tot hem uitgegaan Jeruzalem en geheel Judea, en het gehele land rondom de Jordaan;
\par 6 En werden van hem gedoopt in de Jordaan, belijdende hun zonden.
\par 7 Hij dan, ziende velen van de Farizeen en Sadduceen tot zijn doop komen, sprak tot hen: Gij adderengebroedsels! wie heeft u aangewezen te vlieden van den toekomenden toorn?
\par 8 Brengt dan vruchten voort, der bekering waardig.
\par 9 En meent niet bij u zelven te zeggen: Wij hebben Abraham tot een vader; want ik zeg u, dat God zelfs uit deze stenen Abraham kinderen kan verwekken.
\par 10 En ook is alrede de bijl aan den wortel der bomen gelegd; alle boom dan, die geen goede vrucht voortbrengt, wordt uitgehouwen en in het vuur geworpen.
\par 11 Ik doop u wel met water tot bekering; maar Die na mij komt, is sterker dan ik, Wiens schoenen ik niet waardig ben Hem na te dragen; Die zal u met den Heiligen Geest en met vuur dopen.
\par 12 Wiens wan in Zijn hand is, en Hij zal Zijn dorsvloer doorzuiveren, en Zijn tarwe in Zijn schuur samenbrengen, en zal het kaf met onuitblusselijk vuur verbranden.
\par 13 Toen kwam Jezus van Galilea naar de Jordaan, tot Johannes, om van hem gedoopt te worden.
\par 14 Doch Johannes weigerde Hem zeer, zeggende: Mij is nodig van U gedoopt te worden, en komt Gij tot mij?
\par 15 Maar Jezus, antwoordende, zeide tot hem: Laat nu af; want aldus betaamt ons alle gerechtigheid te vervullen. Toen liet hij van Hem af.
\par 16 En Jezus, gedoopt zijnde, is terstond opgeklommen uit het water; en ziet, de hemelen werden Hem geopend, en hij zag den Geest Gods nederdalen, gelijk een duive, en op Hem komen.
\par 17 En ziet, een stem uit de hemelen, zeggende: Deze is Mijn Zoon, Mijn Geliefde, in Denwelken Ik Mijn welbehagen heb!

\chapter{4}

\par 1 Toen werd Jezus van den Geest weggeleid in de woestijn, om verzocht te worden van den duivel.
\par 2 En als Hij veertig dagen en veertig nachten gevast had, hongerde Hem ten laatste.
\par 3 En de verzoeker, tot Hem gekomen zijnde, zeide: Indien Gij Gods Zoon zijt, zeg, dat deze stenen broden worden.
\par 4 Doch Hij, antwoordende, zeide: Er is geschreven: De mens zal bij brood alleen niet leven, maar bij alle woord, dat door den mond Gods uitgaat.
\par 5 Toen nam Hem de duivel mede naar de heilige stad, en stelde Hem op de tinne des tempels;
\par 6 En zeide tot Hem: Indien Gij Gods Zoon zijt, werp Uzelven nederwaarts; want er is geschreven, dat Hij Zijn engelen van U bevelen zal, en dat zij U op de handen zullen nemen, opdat Gij niet te eniger tijd Uw voet aan een steen aanstoot.
\par 7 Jezus zeide tot hem: Er is wederom geschreven: Gij zult den Heere, uw God, niet verzoeken.
\par 8 Wederom nam Hem de duivel mede op een zeer hogen berg, en toonde Hem al de koninkrijken der wereld, en hun heerlijkheid;
\par 9 En zeide tot Hem: Al deze dingen zal ik U geven, indien Gij, nedervallende, mij zult aanbidden.
\par 10 Toen zeide Jezus tot hem: Ga weg, satan, want er staat geschreven: Den Heere, uw God, zult gij aanbidden, en Hem alleen dienen.
\par 11 Toen liet de duivel van Hem af; en ziet, de engelen zijn toegekomen, en dienden Hem.
\par 12 Als nu Jezus gehoord had, dat Johannes overgeleverd was, is Hij wedergekeerd naar Galilea;
\par 13 En Nazareth verlaten hebbende, is komen wonen te Kapernaum, gelegen aan de zee, in de landpale van Zebulon en Nafthali;
\par 14 Opdat vervuld zou worden, hetgeen gesproken is door Jesaja, den profeet, zeggende:
\par 15 Het land Zebulon en het land Nafthali aan den weg der zee over de Jordaan, Galilea der volken;
\par 16 Het volk, dat in duisternis zat, heeft een groot licht gezien; en degenen, die zaten in het land en de schaduwe des doods, denzelven is een licht opgegaan.
\par 17 Van toen aan heeft Jezus begonnen te prediken en te zeggen: Bekeert u; want het Koninkrijk der hemelen is nabij gekomen.
\par 18 En Jezus, wandelende aan de zee van Galilea, zag twee broeders, namelijk Simon, gezegd Petrus, en Andreas, zijn broeder, het net in de zee werpende (want zij waren vissers);
\par 19 En Hij zeide tot hen: Volgt Mij na, en Ik zal u vissers der mensen maken.
\par 20 Zij dan, terstond de netten verlatende, zijn Hem nagevolgd.
\par 21 En Hij, van daar voortgegaan zijnde, zag twee andere broeders, namelijk Jakobus, den zoon van Zebedeus, en Johannes, zijn broeder, in het schip met hun vader Zebedeus, hun netten vermakende, en heeft hen geroepen.
\par 22 Zij dan, terstond verlatende het schip en hun vader, zijn Hem nagevolgd.
\par 23 En Jezus omging geheel Galilea, lerende in hun synagogen en predikende het Evangelie des Koninkrijks, en genezende alle ziekte en alle kwale onder het volk.
\par 24 En Zijn gerucht ging van daar uit in geheel Syrie; en zij brachten tot Hem allen, die kwalijk gesteld waren, met verscheidene ziekten en pijnen bevangen zijnde, en van den duivel bezeten, en maanzieken en geraakten; en Hij genas dezelve.
\par 25 En vele scharen volgden Hem na, van Galilea en van Dekapolis, en van Jeruzalem, en van Judea, en van over de Jordaan.

\chapter{5}

\par 1 En Jezus, de schare ziende, is geklommen op een berg, en als Hij nedergezeten was, kwamen Zijn discipelen tot Hem.
\par 2 En Zijn mond geopend hebbende, leerde Hij hen, zeggende:
\par 3 Zalig zijn de armen van geest; want hunner is het Koninkrijk der hemelen.
\par 4 Zalig zijn die treuren; want zij zullen vertroost worden.
\par 5 Zalig zijn de zachtmoedigen; want zij zullen het aardrijk beerven.
\par 6 Zalig zijn die hongeren en dorsten naar de gerechtigheid; want zij zullen verzadigd worden.
\par 7 Zalig zijn de barmhartigen; want hun zal barmhartigheid geschieden.
\par 8 Zalig zijn de reinen van hart; want zij zullen God zien.
\par 9 Zalig zijn de vreedzamen; want zij zullen Gods kinderen genaamd worden.
\par 10 Zalig zijn die vervolgd worden om der gerechtigheid wil; want hunner is het Koninkrijk der hemelen.
\par 11 Zalig zijt gij, als u de mensen smaden, en vervolgen, en liegende alle kwaad tegen u spreken, om Mijnentwil.
\par 12 Verblijdt en verheugt u; want uw loon is groot in de hemelen; want alzo hebben zij vervolgd de profeten, die voor u geweest zijn.
\par 13 Gij zijt het zout der aarde; indien nu het zout smakeloos wordt, waarmede zal het gezouten worden? Het deugt nergens meer toe, dan om buiten geworpen, en van de mensen vertreden te worden.
\par 14 Gij zijt het licht der wereld; een stad boven op een berg liggende, kan niet verborgen zijn.
\par 15 Noch steekt men een kaars aan, en zet die onder een koornmaat, maar op een kandelaar, en zij schijnt allen, die in het huis zijn;
\par 16 Laat uw licht alzo schijnen voor de mensen, dat zij uw goede werken mogen zien, en uw Vader, Die in de hemelen is, verheerlijken.
\par 17 Meent niet, dat Ik gekomen ben, om de wet of de profeten te ontbinden; Ik ben niet gekomen, om die te ontbinden, maar te vervullen.
\par 18 Want voorwaar zeg Ik u: Totdat de hemel en de aarde voorbijgaan, zal er niet een jota noch een tittel van de wet voorbijgaan, totdat het alles zal zijn geschied.
\par 19 Zo wie dan een van deze minste geboden zal ontbonden, en de mensen alzo zal geleerd hebben, die zal de minste genaamd worden in het Koninkrijk der hemelen; maar zo wie dezelve zal gedaan en geleerd hebben, die zal groot genaamd worden in het Koninkrijk der hemelen.
\par 20 Want Ik zeg u: Tenzij uw gerechtigheid overvloediger zij, dan der Schriftgeleerden en der Farizeen, dat gij in het Koninkrijk der hemelen geenszins zult ingaan.
\par 21 Gij hebt gehoord, dat tot de ouden gezegd is: Gij zult niet doden; maar zo wie doodt, die zal strafbaar zijn door het gericht.
\par 22 Doch Ik zeg u: Zo wie te onrecht op zijn broeder toornig is, die zal strafbaar zijn door het gericht; en wie tot zijn broeder zegt: Raka! die zal strafbaar zijn door den groten raad; maar wie zegt: Gij dwaas! die zal strafbaar zijn door het helse vuur.
\par 23 Zo gij dan uw gave zult op het altaar offeren, en aldaar gedachtig wordt, dat uw broeder iets tegen u heeft;
\par 24 Laat daar uw gave voor het altaar, en gaat heen, verzoent u eerst met uw broeder, en komt dan en offert uw gave.
\par 25 Weest haastelijk welgezind jegens uw wederpartij, terwijl gij nog met hem op den weg zijt; opdat de wederpartij niet misschien u den rechter overlevere, en de rechter u den dienaar overlevere, en gij in de gevangenis geworpen wordt.
\par 26 Voorwaar, Ik zeg u: Gij zult daar geenszins uitkomen, totdat gij den laatsten penning zult betaald hebben.
\par 27 Gij hebt gehoord, dat van de ouden gezegd is: Gij zult geen overspel doen.
\par 28 Maar Ik zeg u, dat zo wie een vrouw aan ziet, om dezelve te begeren, die heeft alrede overspel in zijn hart met haar gedaan.
\par 29 Indien dan uw rechteroog u ergert, trekt het uit, en werpt het van u; want het is u nut, dat een uwer leden verga, en niet uw gehele lichaam in de hel geworpen worde.
\par 30 En indien uw rechterhand u ergert, houwt ze af, en werpt ze van u; want het is u nut, dat een uwer leden verga, en niet uw gehele lichaam in de hel geworpen worde.
\par 31 Er is ook gezegd: Zo wie zijn vrouw verlaten zal, die geve haar een scheidbrief.
\par 32 Maar Ik zeg u, dat zo wie zijn vrouw verlaten zal, anders dan uit oorzake van hoererij, die maakt, dat zij overspel doet; en zo wie de verlatene zal trouwen, die doet overspel.
\par 33 Wederom hebt gij gehoord, dat van de ouden gezegd is: Gij zult den eed niet breken, maar gij zult den Heere uw eden houden.
\par 34 Maar Ik zeg u: Zweert ganselijk niet, noch bij den hemel, omdat hij is de troon Gods;
\par 35 Noch bij de aarde, omdat zij is de voetbank Zijner voeten; noch bij Jeruzalem, omdat zij is de stad des groten Konings;
\par 36 Noch bij uw hoofd zult gij zweren, omdat gij niet een haar kunt wit of zwart maken;
\par 37 Maar laat zijn uw woord ja, ja; neen, neen; wat boven deze is, dat is uit den boze.
\par 38 Gij hebt gehoord, dat gezegd is: Oog om oog, en tand om tand.
\par 39 Maar Ik zeg u, dat gij den boze niet wederstaat; maar, zo wie u op de rechterwang slaat, keert hem ook de andere toe;
\par 40 En zo iemand met u rechten wil, en uw rok nemen, laat hem ook den mantel;
\par 41 En zo wie u zal dwingen een mijl te gaan, gaat met hem twee mijlen.
\par 42 Geeft dengene, die iets van u bidt, en keert u niet af van dengene, die van u lenen wil.
\par 43 Gij hebt gehoord, dat er gezegd is: Gij zult uw naaste liefhebben, en uw vijand zult gij haten.
\par 44 Maar Ik zeg u: Hebt uw vijanden lief; zegent ze, die u vervloeken; doet wel dengenen, die u haten; en bidt voor degenen, die u geweld doen, en die u vervolgen;
\par 45 Opdat gij moogt kinderen zijn uws Vaders, Die in de hemelen is; want Hij doet Zijn zon opgaan over bozen en goeden, en regent over rechtvaardigen en onrechtvaardigen.
\par 46 Want indien gij liefhebt, die u liefhebben, wat loon hebt gij? Doen ook de tollenaars niet hetzelfde?
\par 47 En indien gij uw broeders alleen groet, wat doet gij boven anderen? Doen ook niet de tollenaars alzo?
\par 48 Weest dan gijlieden volmaakt, gelijk uw Vader, Die in de hemelen is, volmaakt is.

\chapter{6}

\par 1 Hebt acht, dat gij uw aalmoes niet doet voor de mensen, om van hen gezien te worden; anders zo hebt gij geen loon bij uw Vader, Die in de hemelen is.
\par 2 Wanneer gij dan aalmoes doet, zo laat voor u niet trompetten, gelijk de geveinsden in de synagogen en op de straten doen, opdat zij van de mensen geeerd mogen worden. Voorwaar zeg Ik u: Zij hebben hun loon weg.
\par 3 Maar als gij aalmoes doet, zo laat uw linker hand niet weten, wat uw rechter doet;
\par 4 Opdat uw aalmoes in het verborgen zij; en uw Vader, Die in het verborgen ziet, Die zal het u in het openbaar vergelden.
\par 5 En wanneer gij bidt, zo zult gij niet zijn gelijk de geveinsden; want die plegen gaarne, in de synagogen en op de hoeken der straten staande, te bidden, opdat zij van de mensen mogen gezien worden. Voorwaar, Ik zeg u, dat zij hun loon weg hebben.
\par 6 Maar gij, wanneer gij bidt, gaat in uw binnenkamer, en uw deur gesloten hebbende, bidt uw Vader, Die in het verborgen is; en uw Vader, Die in het verborgen ziet, zal het u in het openbaar vergelden.
\par 7 En als gij bidt, zo gebruikt geen ijdel verhaal van woorden, gelijk de heidenen; want zij menen, dat zij door hun veelheid van woorden zullen verhoord worden.
\par 8 Wordt dan hun niet gelijk; want uw Vader weet, wat gij van node hebt, eer gij Hem bidt.
\par 9 Gij dan bidt aldus: Onze Vader, Die in de hemelen zijt! Uw Naam worde geheiligd.
\par 10 Uw Koninkrijk kome. Uw wil geschiede, gelijk in den hemel alzo ook op de aarde.
\par 11 Geef ons heden ons dagelijks brood.
\par 12 En vergeef ons onze schulden, gelijk ook wij vergeven onzen schuldenaren.
\par 13 En leid ons niet in verzoeking, maar verlos ons van den boze. Want Uw is het Koninkrijk, en de kracht, en de heerlijkheid, in der eeuwigheid, amen.
\par 14 Want indien gij den mensen hun misdaden vergeeft, zo zal uw hemelse Vader ook u vergeven.
\par 15 Maar indien gij den mensen hun misdaden niet vergeeft, zo zal ook uw Vader uw misdaden niet vergeven.
\par 16 En wanneer gij vast, toont geen droevig gezicht, gelijk de geveinsden; want zij mismaken hun aangezichten, opdat zij van de mensen mogen gezien worden, als zij vasten. Voorwaar, Ik zeg u, dat zij hun loon weg hebben.
\par 17 Maar gij, als gij vast, zalft uw hoofd, en wast uw aangezicht;
\par 18 Opdat het van de mensen niet gezien worde, als gij vast, maar van uw Vader, Die in het verborgen is; en uw Vader, Die in het verborgen ziet, zal het u in het openbaar vergelden.
\par 19 Vergadert u geen schatten op de aarde, waar ze de mot en de roest verderft, en waar de dieven doorgraven en stelen;
\par 20 Maar vergadert u schatten in den hemel, waar ze noch mot noch roest verderft, en waar de dieven niet doorgraven noch stelen;
\par 21 Want waar uw schat is, daar zal ook uw hart zijn.
\par 22 De kaars des lichaams is het oog; indien dan uw oog eenvoudig is, zo zal uw gehele lichaam verlicht wezen;
\par 23 Maar indien uw oog boos is, zo zal geheel uw lichaam duister zijn. Indien dan het licht, dat in u is, duisternis is, hoe groot zal de duisternis zelve zijn!
\par 24 Niemand kan twee heren dienen; want of hij zal den enen haten en den anderen liefhebben, of hij zal den enen aanhangen en den anderen verachten; gij kunt niet God dienen en den Mammon.
\par 25 Daarom zeg Ik u: Zijt niet bezorgd voor uw leven, wat gij eten, en wat gij drinken zult; noch voor uw lichaam, waarmede gij u kleden zult; is het leven niet meer dan het voedsel, en het lichaam dan de kleding?
\par 26 Aanziet de vogelen des hemels, dat zij niet zaaien, noch maaien, noch verzamelen in de schuren; en uw hemelse Vader voedt nochtans dezelve; gaat gij dezelve niet zeer veel te boven?
\par 27 Wie toch van u kan, met bezorgd te zijn, een el tot zijn lengte toedoen?
\par 28 En wat zijt gij bezorgd voor de kleding? Aanmerkt de lelien des velds, hoe zij wassen; zij arbeiden niet, en spinnen niet;
\par 29 En Ik zeg u, dat ook Salomo, in al zijn heerlijkheid, niet is bekleed geweest, gelijk een van deze.
\par 30 Indien nu God het gras des velds, dat heden is, en morgen in den oven geworpen wordt, alzo bekleedt, zal Hij u niet veel meer kleden, gij kleingelovigen?
\par 31 Daarom zijt niet bezorgd, zeggende: Wat zullen wij eten, of wat zullen wij drinken, of waarmede zullen wij ons kleden?
\par 32 Want al deze dingen zoeken de heidenen; want uw hemelse Vader weet, dat gij al deze dingen behoeft.
\par 33 Maar zoekt eerst het Koninkrijk Gods en Zijn gerechtigheid, en al deze dingen zullen u toegeworpen worden.
\par 34 Zijt dan niet bezorgd tegen den morgen; want de morgen zal voor het zijne zorgen; elke dag heeft genoeg aan zijn zelfs kwaad.

\chapter{7}

\par 1 Oordeelt niet, opdat gij niet geoordeeld wordt.
\par 2 Want met welk oordeel gij oordeelt, zult gij geoordeeld worden; en met welke mate gij meet, zal u wedergemeten worden.
\par 3 En wat ziet gij den splinter, die in het oog uws broeders is, maar den balk, die in uw oog is, merkt gij niet?
\par 4 Of, hoe zult gij tot uw broeder zeggen: Laat toe, dat ik den splinter uit uw oog uitdoe; en zie, er is een balk in uw oog?
\par 5 Gij geveinsde! werp eerst den balk uit uw oog, en dan zult gij bezien, om den splinter uit uws broeders oog uit te doen.
\par 6 Geeft het heilige den honden niet, noch werpt uw paarlen voor de zwijnen; opdat zij niet te eniger tijd dezelve met hun voeten vertreden, en zich omkerende, u verscheuren.
\par 7 Bidt, en u zal gegeven worden; zoekt, en gij zult vinden; klopt, en u zal opengedaan worden.
\par 8 Want een iegelijk, die bidt, die ontvangt; en die zoekt, die vindt; en die klopt, dien zal opengedaan worden.
\par 9 Of wat mens is er onder u, zo zijn zoon hem zou bidden om brood, die hem een steen zal geven?
\par 10 En zo hij hem om een vis zou bidden, die hem een slang zal geven?
\par 11 Indien dan gij, die boos zijt, weet uw kinderen goede gaven te geven, hoeveel te meer zal uw Vader, Die in de hemelen is, goede gaven geven dengenen, die ze van Hem bidden!
\par 12 Alle dingen dan, die gij wilt, dat u de mensen zouden doen, doet gij hun ook alzo; want dat is de wet en de profeten.
\par 13 Gaat in door de enge poort; want wijd is de poort, en breed is de weg, die tot het verderf leidt, en velen zijn er, die door dezelve ingaan;
\par 14 Want de poort is eng, en de weg is nauw, die tot het leven leidt, en weinigen zijn er, die denzelven vinden.
\par 15 Maar wacht u van de valse profeten, dewelke in schaapsklederen tot u komen, maar van binnen zijn zij grijpende wolven.
\par 16 Aan hun vruchten zult gij hen kennen. Leest men ook een druif van doornen, of vijgen van distelen?
\par 17 Alzo een ieder goede boom brengt voort goede vruchten, en een kwade boom brengt voort kwade vruchten.
\par 18 Een goede boom kan geen kwade vruchten voortbrengen, noch een kwade boom goede vruchten voortbrengen.
\par 19 Een ieder boom, die geen goede vrucht voortbrengt, wordt uitgehouwen en in het vuur geworpen.
\par 20 Zo zult gij dan dezelve aan hun vruchten kennen.
\par 21 Niet een iegelijk, die tot Mij zegt: Heere, Heere! zal ingaan in het Koninkrijk der hemelen, maar die daar doet den wil Mijns Vaders, Die in de hemelen is.
\par 22 Velen zullen te dien dage tot Mij zeggen: Heere, Heere! hebben wij niet in Uw Naam geprofeteerd, en in Uw Naam duivelen uitgeworpen, en in Uw Naam vele krachten gedaan?
\par 23 En dan zal Ik hun openlijk aanzeggen: Ik heb u nooit gekend; gaat weg van Mij, gij, die de ongerechtigheid werkt!
\par 24 Een iegelijk dan, die deze Mijn woorden hoort en dezelve doet, dien zal Ik vergelijken bij een voorzichtig man, die zijn huis op een steenrots gebouwd heeft;
\par 25 En er is slagregen nedergevallen, en de waterstromen zijn gekomen, en de winden hebben gewaaid, en zijn tegen hetzelve huis aangevallen, en het is niet gevallen, want het was op de steenrots gegrond.
\par 26 En een iegelijk, die deze Mijn woorden hoort en dezelve niet doet, die zal bij een dwazen man vergeleken worden, die zijn huis op het zand gebouwd heeft;
\par 27 En de slagregen is nedergevallen, en de waterstromen zijn gekomen, en de winden hebben gewaaid, en zijn tegen hetzelve huis aangeslagen, en het is gevallen, en zijn val was groot.
\par 28 En het is geschied, als Jezus deze woorden geeindigd had, dat de scharen zich ontzetten over Zijn leer;
\par 29 Want Hij leerde hen, als macht hebbende, en niet als de Schriftgeleerden.

\chapter{8}

\par 1 Toen Hij nu van den berg afgeklommen was, zijn Hem vele scharen gevolgd.
\par 2 En ziet, een melaatse kwam, en aanbad Hem, zeggende: Heere! indien Gij wilt, Gij kunt mij reinigen.
\par 3 En Jezus, de hand uitstrekkende, heeft hem aangeraakt, zeggende: Ik wil, word gereinigd! En terstond werd hij van zijn melaatsheid gereinigd.
\par 4 En Jezus zeide tot hem: Zie, dat gij dit niemand zegt; maar ga heen, toon uzelven den priester, en offer de gave, die Mozes geboden heeft, hun tot een getuigenis.
\par 5 Als nu Jezus te Kapernaum ingegaan was, kwam tot Hem een hoofdman over honderd, biddende Hem,
\par 6 En zeggende: Heere! mijn knecht ligt te huis geraakt, en lijdt zware pijnen.
\par 7 En Jezus zeide tot hem: Ik zal komen en hem genezen.
\par 8 En de hoofdman over honderd, antwoordende, zeide: Heere! ik ben niet waardig, dat Gij onder mijn dak zoudt inkomen; maar spreek alleenlijk een woord, en mijn knecht zal genezen worden.
\par 9 Want ik ben ook een mens onder de macht van anderen, hebbende onder mij krijgsknechten; en ik zeg tot dezen: Ga! en hij gaat; en tot den anderen: Kom! en hij komt; en tot mijn dienstknecht: Doe dat! en hij doet het.
\par 10 Jezus nu, dit horende, heeft Zich verwonderd, en zeide tot degenen, die Hem volgden: Voorwaar zeg Ik u, Ik heb zelfs in Israel zo groot een geloof niet gevonden.
\par 11 Doch Ik zeg u, dat velen zullen komen van oosten en westen en zullen met Abraham, en Izak, en Jakob, aanzitten in het Koninkrijk der hemelen;
\par 12 En de kinderen des Koninkrijks zullen uitgeworpen worden in de buitenste duisternis; aldaar zal wening zijn, en knersing der tanden.
\par 13 En Jezus zeide tot den hoofdman over honderd: Ga heen, en u geschiede, gelijk gij geloofd hebt. En zijn knecht is gezond geworden te dierzelver ure.
\par 14 En Jezus gekomen zijnde in het huis van Petrus, zag zijn vrouws moeder te bed liggen, hebbende de koorts.
\par 15 En Hij raakte haar hand aan, en de koorts verliet haar; en zij stond op, en diende henlieden.
\par 16 En als het laat geworden was, hebben zij velen, van den duivel bezeten, tot Hem gebracht, en Hij wierp de boze geesten uit met den woorde, en Hij genas allen, die kwalijk gesteld waren;
\par 17 Opdat vervuld zou worden, dat gesproken was door Jesaja, den profeet, zeggende: Hij heeft onze krankheden op Zich genomen, en onze ziekten gedragen.
\par 18 En Jezus, vele scharen ziende rondom Zich, beval aan de andere zijde over te varen.
\par 19 En er kwam een zeker Schriftgeleerde tot Hem, en zeide tot Hem: Meester! ik zal U volgen, waar Gij ook henengaat.
\par 20 En Jezus zeide tot hem: De vossen hebben holen, en de vogelen des hemels nesten; maar de Zoon des mensen heeft niet, waar Hij het hoofd nederlegge.
\par 21 En een ander uit Zijn discipelen zeide tot Hem: Heere! laat mij toe, dat ik eerst heenga, en mijn vader begrave.
\par 22 Doch Jezus zeide tot hem: Volg Mij, en laat de doden hun doden begraven.
\par 23 En als Hij in het schip gegaan was, zijn Hem Zijn discipelen gevolgd.
\par 24 En ziet, er ontstond een grote onstuimigheid in de zee, alzo dat het schip van de golven bedekt werd; doch Hij sliep.
\par 25 En Zijn discipelen, bij Hem komende, hebben Hem opgewekt, zeggende: Heere, behoed ons, wij vergaan!
\par 26 En Hij zeide tot hen: Wat zijt gij vreesachtig, gij kleingelovigen? Toen stond Hij op, en bestrafte de winden en de zee; en er werd grote stilte.
\par 27 En de mensen verwonderden zich, zeggende: Hoedanig een is Deze, dat ook de winden en de zee Hem gehoorzaam zijn!
\par 28 En als Hij over aan de andere zijde was gekomen in het land der Gergesenen, zijn Hem twee, van den duivel bezeten, ontmoet, komende uit de graven, die zeer wreed waren, alzo dat niemand door dien weg kon voorbij gaan.
\par 29 En ziet, zij riepen, zeggende: Jezus, Gij Zone Gods! wat hebben wij met U te doen? Zijt Gij hier gekomen om ons te pijnigen voor den tijd?
\par 30 En verre van hen was een kudde veler zwijnen, weidende.
\par 31 En de duivelen baden Hem, zeggende: Indien Gij ons uitwerpt, laat ons toe, dat wij in die kudde zwijnen varen.
\par 32 En Hij zeide tot hen: Gaat heen. En zij uitgaande, voeren heen in de kudde zwijnen; en ziet, de gehele kudde zwijnen stortte van de steilte af in de zee, en zij stierven in het water.
\par 33 En die ze weidden, zijn gevlucht; en als zij in de stad gekomen waren, boodschapten zij al deze dingen, en wat den bezetenen geschied was.
\par 34 En ziet, de gehele stad ging uit, Jezus tegemoet; en als zij Hem zagen, baden zij, dat Hij uit hun landpalen wilde vertrekken.

\chapter{9}

\par 1 En in het schip gegaan zijnde, voer Hij over en kwam in Zijn stad. En ziet, zij brachten tot Hem een geraakte, op een bed liggende.
\par 2 En Jezus, hun geloof ziende, zeide tot den geraakte: Zoon! wees welgemoed; uw zonden zijn u vergeven.
\par 3 En ziet, sommigen der Schriftgeleerden zeiden in zichzelven: Deze lastert God.
\par 4 En Jezus, ziende hun gedachten, zeide: Waarom overdenkt gij kwaad in uw harten?
\par 5 Want wat is lichter te zeggen: De zonden zijn u vergeven? of te zeggen: Sta op en wandel?
\par 6 Doch opdat gij moogt weten, dat de Zoon des mensen macht heeft op de aarde, de zonden te vergeven (toen zeide Hij tot den geraakte): Sta op, neem uw bed op, en ga heen naar uw huis.
\par 7 En hij opgestaan zijnde, ging heen naar zijn huis.
\par 8 De scharen nu dat ziende, hebben zich verwonderd, en God verheerlijkt, die zodanige macht den mensen gegeven had.
\par 9 En Jezus, van daar voortgaande, zag een mens in het tolhuis zitten, genaamd Mattheus; en zeide tot hem: Volg Mij. En hij opstaande, volgde Hem.
\par 10 En het geschiedde, als Hij in het huis van Mattheus aanzat, ziet, vele tollenaars en zondaars kwamen en zaten mede aan, met Jezus en Zijn discipelen.
\par 11 En de Farizeen, dat ziende, zeiden tot Zijn discipelen: Waarom eet uw Meester met de tollenaren en de zondaren?
\par 12 Maar Jezus, zulks horende, zeide tot hen: Die gezond zijn hebben den medicijnmeester niet van node, maar die ziek zijn.
\par 13 Doch gaat heen en leert, wat het zij: Ik wil barmhartigheid, en niet offerande; want Ik ben niet gekomen om te roepen rechtvaardigen, maar zondaars tot bekering.
\par 14 Toen kwamen de discipelen van Johannes tot Hem, zeggende: Waarom vasten wij en de Farizeen veel, en Uw discipelen vasten niet?
\par 15 En Jezus zeide tot hen: Kunnen ook de bruiloftskinderen treuren, zolang de Bruidegom bij hen is? Maar de dagen zullen komen, wanneer de Bruidegom van hen zal weggenomen zijn, en dan zullen zij vasten.
\par 16 Ook zet niemand een lap ongevold laken op een oud kleed; want deszelfs aangezette lap scheurt af van het kleed, en er wordt een ergere scheur.
\par 17 Noch doet men nieuwen wijn in oude leder zakken; anders zo bersten de leder zakken, en de wijn wordt uitgestort, en de leder zakken verderven, maar men doet nieuwen wijn in nieuwe leder zakken, en beide te zamen worden behouden.
\par 18 Als Hij deze dingen tot hen sprak, ziet, een overste kwam en aanbad Hem, zeggende: Mijn dochter is nu terstond gestorven, doch kom en leg Uw hand op haar, en zij zal leven.
\par 19 En Jezus opgestaan zijnde, volgde hem, en Zijn discipelen.
\par 20 (En ziet, een vrouw die twaalf jaren het bloedvloeien gehad had, komende tot Hem van achteren, raakte den zoom Zijns kleeds aan;
\par 21 Want zij zeide in zichzelven: Indien ik alleenlijk Zijn kleed aanraak, zo zal ik gezond worden.
\par 22 En Jezus, Zich omkerende, en haar ziende, zeide: Wees welgemoed, dochter! uw geloof heeft u behouden. En de vrouw werd gezond van dezelve ure af.)
\par 23 En als Jezus in het huis des oversten kwam, en zag de pijpers en de woelende schare,
\par 24 Zeide Hij tot hen: Vertrekt; want het dochtertje is niet dood, maar slaapt. En zij belachten Hem.
\par 25 Als nu de schare uitgedreven was, ging Hij in, en greep haar hand; en het dochtertje stond op.
\par 26 En dit gerucht ging uit door dat gehele land.
\par 27 En als Jezus van daar voortging, zijn Hem twee blinden gevolgd, roepende en zeggende: Gij Zone Davids, ontferm U onzer!
\par 28 En als Hij in huis gekomen was, kwamen de blinden tot Hem. En Jezus zeide tot hen: Gelooft gij, dat Ik dat doen kan? Zij zeiden tot Hem: Ja, Heere!
\par 29 Toen raakte Hij hun ogen aan, zeggende: U geschiede naar uw geloof.
\par 30 En hun ogen zijn geopend geworden. En Jezus heeft hun zeer gestrengelijk verboden, zeggende: Ziet, dat niemand het wete.
\par 31 Maar zij, uitgegaan zijnde, hebben Hem ruchtbaar gemaakt door dat gehele land.
\par 32 Als dezen nu uitgingen, ziet, zo brachten zij tot Hem een mens, die stom en van den duivel bezeten was.
\par 33 En als de duivel uitgeworpen was, sprak de stomme. En de scharen verwonderden zich, zeggende: Er is nooit desgelijks in Israel gezien!
\par 34 Maar de Farizeen zeiden: Hij werpt de duivelen uit door den overste der duivelen.
\par 35 En Jezus omging al de steden en vlekken, lerende in hun synagogen, en predikende het Evangelie des Koninkrijks, en genezende alle ziekte en alle kwale onder het volk.
\par 36 En Hij, de scharen ziende, werd innerlijk met ontferming bewogen over hen, omdat zij vermoeid en verstrooid waren, gelijk schapen, die geen herder hebben.
\par 37 Toen zeide Hij tot Zijn discipelen: De oogst is wel groot; maar de arbeiders zijn weinige;
\par 38 Bidt dan den Heere des oogstes, dat Hij arbeiders in Zijn oogst uitstote.

\chapter{10}

\par 1 En Zijn twaalf discipelen tot Zich geroepen hebbende, heeft Hij hun macht gegeven over de onreine geesten, om dezelve uit te werpen, en om alle ziekte en alle kwale te genezen.
\par 2 De namen nu der twaalf apostelen zijn deze: de eerste, Simon, gezegd Petrus, en Andreas, zijn broeder; Jakobus, de zoon van Zebedeus, en Johannes, zijn broeder;
\par 3 Filippus en Bartholomeus; Thomas en Mattheus, de tollenaar; Jakobus, de zoon van Alfeus, en Lebbeus, toegenaamd Thaddeus;
\par 4 Simon Kananites, en Judas Iskariot, die Hem ook verraden heeft.
\par 5 Deze twaalf heeft Jezus uitgezonden, en hun bevel gegeven, zeggende: Gij zult niet heengaan op den weg der heidenen, en gij zult niet ingaan in enige stad der Samaritanen.
\par 6 Maar gaat veel meer heen tot de verloren schapen van het huis Israels.
\par 7 En heengaande predikt, zeggende: Het Koninkrijk der hemelen is nabij gekomen.
\par 8 Geneest de kranken; reinigt de melaatsen; wekt de doden op; werpt de duivelen uit. Gij hebt het om niet ontvangen, geeft het om niet.
\par 9 Verkrijgt u noch goud, noch zilver, noch koper geld in uw gordels;
\par 10 Noch male tot den weg, noch twee rokken, noch schoenen, noch staf; want de arbeider is zijn voedsel waardig.
\par 11 En in wat stad of vlek gij zult inkomen, onderzoekt, wie daarin waardig is; en blijft aldaar, totdat gij daar uitgaat.
\par 12 En als gij in het huis gaat, zo groet hetzelve.
\par 13 En indien dat huis waardig is, zo kome uw vrede over hetzelve, maar indien het niet waardig is, zo kere uw vrede weder tot u.
\par 14 En zo iemand u niet zal ontvangen, noch uw woorden horen, uitgaande uit dat huis of uit dezelve stad, schudt het stof uwer voeten af.
\par 15 Voorwaar zeg Ik u: Het zal den lande van Sodom en Gomorra verdragelijker zijn in den dag des oordeels, dan dezelve stad.
\par 16 Ziet, Ik zende u als schapen in het midden der wolven; zijt dan voorzichtig gelijk de slangen, en oprecht gelijk de duiven.
\par 17 Maar wacht u voor de mensen; want zij zullen u overleveren in de raadsvergaderingen, en in hun synagogen zullen zij u geselen.
\par 18 En gij zult ook voor stadhouders en koningen geleid worden, om Mijnentwil, hun en den heidenen tot getuigenis.
\par 19 Doch wanneer zij u overleveren, zo zult gij niet bezorgd zijn, hoe of wat gij spreken zult; want het zal u in dezelve ure gegeven worden, wat gij spreken zult.
\par 20 Want gij zijt het niet, die spreekt, maar het is de Geest uws Vaders, Die in u spreekt.
\par 21 En de ene broeder zal den anderen broeder overleveren tot den dood, en de vader het kind, en de kinderen zullen opstaan tegen de ouders, en zullen hen doden.
\par 22 En gij zult van allen gehaat worden om Mijn Naam; maar die volstandig zal blijven tot het einde, die zal zalig worden.
\par 23 Wanneer zij u dan in deze stad vervolgen, vliedt in de andere; want voorwaar zeg ik u: Gij zult uw reis door de steden Israels niet geeindigd hebben, of de Zoon des mensen zal gekomen zijn.
\par 24 De discipel is niet boven den meester, noch de dienstknecht boven zijn heer.
\par 25 Het zij den discipel genoeg, dat hij worde gelijk zijn meester, en de dienstknecht gelijk zijn heer. Indien zij den Heere des huizes Beelzebul hebben geheten, hoeveel te meer Zijn huisgenoten!
\par 26 Vreest dan hen niet; want er is niets bedekt, hetwelk niet zal ontdekt worden, en verborgen, hetwelk niet zal geweten worden.
\par 27 Hetgeen Ik u zeg in de duisternis, zegt het in het licht; en hetgeen gij hoort in het oor, predikt dat op de daken.
\par 28 En vreest u niet voor degenen, die het lichaam doden, en de ziel niet kunnen doden; maar vreest veel meer Hem, Die beide ziel en lichaam kan verderven in de hel.
\par 29 Worden niet twee musjes om een penningsken verkocht? En niet een van deze zal op de aarde vallen zonder uw Vader.
\par 30 En ook uw haren des hoofds zijn alle geteld.
\par 31 Vreest dan niet; gij gaat vele musjes te boven.
\par 32 Een iegelijk dan, die Mij belijden zal voor de mensen, dien zal Ik ook belijden voor Mijn Vader, Die in de hemelen is.
\par 33 Maar zo wie Mij verloochend zal hebben voor de mensen, dien zal Ik ook verloochenen voor Mijn Vader, Die in de hemelen is.
\par 34 Meent niet, dat Ik gekomen ben, om vrede te brengen op de aarde; Ik ben niet gekomen om vrede te brengen, maar het zwaard.
\par 35 Want Ik ben gekomen, om den mens tweedrachtig te maken tegen zijn vader, en de dochter tegen haar moeder, en de schoondochter tegen haar schoonmoeder.
\par 36 En zij zullen des mensen vijanden worden, die zijn huisgenoten zijn.
\par 37 Die vader of moeder liefheeft boven Mij, is Mijns niet waardig; en die zoon of dochter liefheeft boven Mij, is Mijns niet waardig.
\par 38 En die zijn kruis niet op zich neemt, en Mij navolgt, is Mijns niet waardig.
\par 39 Die zijn ziel vindt, zal dezelve verliezen; en die zijn ziel zal verloren hebben om Mijnentwil, zal dezelve vinden.
\par 40 Die u ontvangt, ontvangt Mij; en die Mij ontvangt, ontvangt Hem, Die Mij gezonden heeft.
\par 41 Die een profeet ontvangt in den naam eens profeten, zal het loon eens profeten ontvangen; en die een rechtvaardige ontvangt in den naam eens rechtvaardigen, zal het loon eens rechtvaardigen ontvangen.
\par 42 En zo wie een van deze kleinen te drinken geeft alleenlijk een beker koud water, in den naam eens discipels, voorwaar zeg Ik u, hij zal zijn loon geenszins verliezen.

\chapter{11}

\par 1 En het is geschied, toen Jezus geeindigd had Zijn twaalf discipelen bevelen te geven, dat Hij van daar voortging, om te leren en te prediken in hun steden.
\par 2 En Johannes, in de gevangenis gehoord hebbende de werken van Christus, zond twee van zijn discipelen;
\par 3 En zeide tot hem: Zijt Gij Degene, Die komen zou, of verwachten wij een anderen?
\par 4 En Jezus antwoordde en zeide tot hen: Gaat heen en boodschapt Johannes weder, hetgeen gij hoort en ziet:
\par 5 De blinden worden ziende, en de kreupelen wandelen; de melaatsen worden gereinigd, en de doven horen; de doden worden opgewekt, en den armen wordt het Evangelie verkondigd.
\par 6 En zalig is hij, die aan Mij niet zal geergerd worden.
\par 7 Als nu dezen heengingen, heeft Jezus tot de scharen begonnen te zeggen van Johannes: Wat zijt gij uitgegaan in de woestijn te aanschouwen? Een riet, dat van den wind ginds en weder bewogen wordt?
\par 8 Maar wat zijt gij uitgegaan te zien? Een mens, met zachte klederen bekleed? Ziet, die zachte klederen dragen, zijn in der koningen huizen.
\par 9 Maar wat zijt gij uitgegaan te zien? Een profeet? Ja, Ik zeg u, ook veel meer dan een profeet.
\par 10 Want deze is het, van dewelken geschreven staat: Ziet, Ik zende Mijn engel voor Uw aangezicht, die Uw weg bereiden zal voor U heen.
\par 11 Voorwaar zeg Ik u: onder degenen, die van vrouwen geboren zijn, is niemand opgestaan meerder dan Johannes de Doper; doch die de minste is in het Koninkrijk der hemelen, is meerder dan hij.
\par 12 En van de dagen van Johannes den Doper tot nu toe, wordt het Koninkrijk der hemelen geweld aangedaan, en de geweldigers nemen hetzelve met geweld.
\par 13 Want al de profeten en de wet hebben tot Johannes toe geprofeteerd.
\par 14 En zo gij het wilt aannemen, hij is Elias, die komen zou.
\par 15 Wie oren heeft om te horen, die hore.
\par 16 Doch waarbij zal Ik dit geslacht vergelijken? Het is gelijk aan de kinderkens, die op de markten zitten, en hun gezellen toeroepen.
\par 17 En zeggen: Wij hebben u op de fluit gespeeld, en gij hebt niet gedanst; wij hebben u klaagliederen gezongen, en gij hebt niet geweend.
\par 18 Want Johannes is gekomen, noch etende, noch drinkende, en zij zeggen: Hij heeft den duivel.
\par 19 De Zoon des mensen is gekomen, etende en drinkende, en zij zeggen: Ziet daar, een Mens, Die een vraat en wijnzuiper is, een Vriend van tollenaren en zondaren. Doch de Wijsheid is gerechtvaardigd geworden van Haar kinderen.
\par 20 Toen begon Hij de steden, in dewelke Zijn krachten meest geschied waren, te verwijten, omdat zij zich niet bekeerd hadden.
\par 21 Wee u, Chorazin! wee u Bethsaida! want zo in Tyrus en Sidon de krachten waren geschied, die in u geschied zijn, zij zouden zich eertijds in zak en as bekeerd hebben.
\par 22 Doch Ik zeg u: Het zal Tyrus en Sidon verdragelijker zijn in den dag des oordeels, dan ulieden.
\par 23 En gij, Kapernaum! die tot den hemel toe zijt verhoogd, gij zult tot de hel toe nedergestoten worden. Want zo in Sodom die krachten waren geschied, die in u geschied zijn, zij zouden tot op den huidigen dag gebleven zijn.
\par 24 Doch Ik zeg u, dat het den lande van Sodom verdragelijker zal zijn in den dag des oordeels, dan u.
\par 25 In dienzelfden tijd antwoordde Jezus en zeide: Ik dank U, Vader! Heere des hemels en der aarde! dat Gij deze dingen voor de wijzen en verstandigen verborgen hebt, en hebt dezelve den kinderkens geopenbaard.
\par 26 Ja, Vader! Want alzo is geweest het welbehagen voor U.
\par 27 Alle dingen zijn Mij overgegeven van Mijn Vader; en niemand kent den Zoon dan de Vader, noch iemand kent den Vader dan de Zoon, en dien het de Zoon wil openbaren.
\par 28 Komt herwaarts tot Mij, allen die vermoeid en belast zijt, en Ik zal u rust geven.
\par 29 Neemt Mijn juk op u, en leert van Mij, dat Ik zachtmoedig ben en nederig van hart; en gij zult rust vinden voor uw zielen.
\par 30 Want Mijn juk is zacht, en Mijn last is licht.

\chapter{12}

\par 1 In dien tijd ging Jezus, op een sabbatdag, door het gezaaide, en Zijn discipelen hadden honger, en begonnen aren te plukken, en te eten.
\par 2 En de Farizeen, dat ziende, zeiden tot Hem: Zie, Uw discipelen doen, wat niet geoorloofd is te doen op den sabbat.
\par 3 Maar Hij zeide tot hen: Hebt gij niet gelezen, wat David gedaan heeft, toen hem hongerde, en hun, die met hem waren?
\par 4 Hoe hij gegaan is in het huis Gods, en de toonbroden gegeten heeft, die hem niet geoorloofd waren te eten, noch ook hun, die met hem waren, maar den priesteren alleen.
\par 5 Of hebt gij niet gelezen in de wet, dat de priesters den sabbat ontheiligen in den tempel, op de sabbatdagen, en nochtans onschuldig zijn?
\par 6 En Ik zeg u, dat Een, meerder dan de tempel, hier is.
\par 7 Doch zo gij geweten hadt, wat het zij: Ik wil barmhartigheid en niet offerande, gij zoudt de onschuldigen niet veroordeeld hebben.
\par 8 Want de Zoon des mensen is een Heere ook van den sabbat.
\par 9 En van daar voortgaande, kwam Hij in hun synagoge.
\par 10 En ziet, er was een mens, die een dorre hand had, en zij vraagden Hem, zeggende: Is het ook geoorloofd op de sabbatdagen te genezen? (opdat zij Hem mochten beschuldigen).
\par 11 En Hij zeide tot hen: Wat mens zal er zijn onder u, die een schaap heeft, en zo datzelve op een sabbatdag in een gracht valt, die hetzelve niet zal aangrijpen en uitheffen?
\par 12 Hoe veel gaat nu een mens een schaap te boven? Zo is het dan op de sabbatdagen geoorloofd wel te doen.
\par 13 Toen zeide Hij tot dien mens: Strek uw hand uit; en hij strekte ze uit, en zij werd hersteld, gezond gelijk de andere.
\par 14 En de Farizeen, uitgegaan zijnde, hielden te zamen raad tegen Hem, hoe zij Hem doden mochten.
\par 15 Maar Jezus, dat wetende, vertrok van daar, en vele scharen volgden Hem, en Hij genas ze allen.
\par 16 En Hij gebood hun scherpelijk, dat zij Hem niet openbaar maken zouden;
\par 17 Opdat vervuld zou worden, hetgeen gesproken is door Jesaja, den profeet, zeggende:
\par 18 Ziet, Mijn Knecht, Welken Ik verkoren heb, Mijn Beminde, in Welken Mijn ziel een welbehagen heeft; Ik zal Mijn Geest op Hem leggen, en Hij zal het oordeel den heidenen verkondigen.
\par 19 Hij zal niet twisten, noch roepen, noch zal er iemand Zijn stem op de straten horen.
\par 20 Het gekrookte riet zal Hij niet verbreken, en het rokende lemmet zal Hij niet uitblussen, totdat Hij het oordeel zal uitbrengen tot overwinning.
\par 21 En in Zijn Naam zullen de heidenen hopen.
\par 22 Toen werd tot Hem gebracht een van den duivel bezeten, die blind en stom was; en Hij genas hem, alzo dat de blinde en stomme beide sprak en zag.
\par 23 En al de scharen ontzetten zich, en zeiden: Is niet Deze de Zoon van David?
\par 24 Maar de Farizeen, dit gehoord hebbende, zeiden: Deze werpt de duivelen niet uit, dan door Beelzebul, den overste der duivelen.
\par 25 Doch Jezus, kennende hun gedachten, zeide tot hen: Een ieder koninkrijk, dat tegen zichzelf verdeeld is, wordt verwoest; en een iedere stad, of huis, dat tegen zichzelf verdeeld is, zal niet bestaan.
\par 26 En indien de satan den satan uitwerpt, zo is hij tegen zichzelf verdeeld; hoe zal dan zijn rijk bestaan?
\par 27 En indien Ik door Beelzebul de duivelen uitwerp, door wien werpen ze dan uw zonen uit? Daarom zullen die uw rechters zijn.
\par 28 Maar indien Ik door den Geest Gods de duivelen uitwerp, zo is dan het Koninkrijk Gods tot u gekomen.
\par 29 Of hoe kan iemand in het huis eens sterken inkomen, en zijn vaten ontroven, tenzij dat hij eerst den sterke gebonden hebbe? en alsdan zal hij zijn huis beroven.
\par 30 Wie met Mij niet is, die is tegen Mij; en wie met Mij niet vergadert, die verstrooit.
\par 31 Daarom zeg Ik u: Alle zonde en lastering zal den mensen vergeven worden; maar de lastering tegen den Geest zal den mensen niet vergeven worden.
\par 32 En zo wie enig woord gesproken zal hebben tegen den Zoon des mensen, het zal hem vergeven worden; maar zo wie tegen den Heiligen Geest zal gesproken hebben, het zal hem niet vergeven worden, noch in deze eeuw, noch in de toekomende.
\par 33 Of maakt den boom goed en zijn vrucht goed; of maakt den boom kwaad en zijn vrucht kwaad; want uit de vrucht wordt de boom gekend.
\par 34 Gij adderengebroedsels! hoe kunt gij goede dingen spreken, daar gij boos zijt? want uit den overvloed des harten spreekt de mond.
\par 35 De goede mens brengt goede dingen voort uit den goede schat des harten, en de boze mens brengt boze dingen voort uit den boze schat.
\par 36 Maar Ik zeg u, dat van elk ijdel woord, hetwelk de mensen zullen gesproken hebben, zij van hetzelve zullen rekenschap geven in den dag des oordeels.
\par 37 Want uit uw woorden zult gij gerechtvaardigd worden, en uit uw woorden zult gij veroordeeld worden.
\par 38 Toen antwoordden sommigen der Schriftgeleerden en Farizeen, zeggende: Meester! wij willen van U wel een teken zien.
\par 39 Maar Hij antwoordde en zeide tot hen: Het boos en overspelig geslacht verzoekt een teken; en hun zal geen teken gegeven worden, dan het teken van Jonas, den profeet.
\par 40 Want gelijk Jonas drie dagen en drie nachten was in den buik van den walvis, alzo zal de Zoon des mensen drie dagen en drie nachten wezen in het hart der aarde.
\par 41 De mannen van Nineve zullen opstaan in het oordeel met dit geslacht, en zullen hetzelve veroordelen; want zij hebben zich bekeerd op de prediking van Jonas; en ziet, meer dan Jona is hier!
\par 42 De koningin van het zuiden zal opstaan in het oordeel met dit geslacht, en hetzelve veroordelen; want zij is gekomen van het einde der aarde, om te horen de wijsheid van Salomo; en ziet, meer dan Salomo is hier!
\par 43 En wanneer de onreine geest van den mens uitgegaan is, zo gaat hij door dorre plaatsen, zoekende rust, en vindt ze niet.
\par 44 Dan zegt hij: Ik zal wederkeren in mijn huis, van waar ik uitgegaan ben; en komende, vindt hij het ledig, met bezemen gekeerd en versierd.
\par 45 Dan gaat hij heen en neemt met zich zeven andere geesten, bozer dan hij zelf, en ingegaan zijnde, wonen zij aldaar; en het laatste van denzelven mens wordt erger dan het eerste. Alzo zal het ook met dit boos geslacht zijn.
\par 46 En als Hij nog tot de scharen sprak, ziet, Zijn moeder en broeders stonden buiten, zoekende Hem te spreken.
\par 47 En iemand zeide tot Hem: Zie, Uw moeder en Uw broeders staan daar buiten, zoekende U te spreken.
\par 48 Maar Hij, antwoordende, zeide tot dengene die Hem dat zeide: Wie is Mijn moeder, en wie zijn Mijn broeders?
\par 49 En Zijn hand uitstrekkende over Zijn discipelen, zeide Hij: Ziet, Mijn moeder en Mijn broeders.
\par 50 Want zo wie den wil Mijns Vaders doet, Die in de hemelen is, dezelve is Mijn broeder, en zuster, en moeder.

\chapter{13}

\par 1 En te dien dage Jezus, uit het huis gegaan zijnde, zat bij de zee.
\par 2 En tot Hem vergaderden vele scharen, zodat Hij in een schip ging en nederzat, en al de schare stond op den oever.
\par 3 En Hij sprak tot hen vele dingen door gelijkenissen, zeggende: Ziet, een zaaier ging uit om te zaaien.
\par 4 En als hij zaaide, viel een deel van het zaad bij den weg; en de vogelen kwamen en aten datzelve op.
\par 5 En een ander deel viel op steenachtige plaatsen, waar het niet veel aarde had; en het ging terstond op, omdat het geen diepte van aarde had.
\par 6 Maar als de zon opgegaan was, zo is het verbrand geworden; en omdat het geen wortel had, is het verdord.
\par 7 En een ander deel viel in de doornen; en de doornen wiesen op, en verstikten hetzelve.
\par 8 En een ander deel viel in de goede aarde, en gaf vrucht, het een honderd-,het ander zestig-,en het ander dertig voud.
\par 9 Wie oren heeft om te horen, die hore.
\par 10 En de discipelen tot Hem komende, zeiden tot Hem: Waarom spreekt Gij tot hen door gelijkenissen?
\par 11 En Hij, antwoordende, zeide tot hen: Omdat het u gegeven is, de verborgenheden van het Koninkrijk der hemelen te weten, maar dien is het niet gegeven.
\par 12 Want wie heeft, dien zal gegeven worden, en hij zal overvloediglijk hebben; maar wie niet heeft, van dien zal genomen worden, ook dat hij heeft.
\par 13 Daarom spreek Ik tot hen door gelijkenissen, omdat zij ziende niet zien, en horende niet horen, noch ook verstaan.
\par 14 En in hen wordt de profetie van Jesaja vervuld, die zegt: Met het gehoor zult gij horen, en geenszins verstaan; en ziende zult gij zien, en geenszins bemerken.
\par 15 Want het hart dezes volks is dik geworden, en zij hebben met de oren zwaarlijk gehoord, en hun ogen hebben zij toegedaan; opdat zij niet te eniger tijd met de ogen zouden zien, en met de oren horen, en met het hart verstaan, en zich bekeren, en Ik hen geneze.
\par 16 Doch uw ogen zijn zalig, omdat zij zien, en uw oren, omdat zij horen.
\par 17 Want voorwaar zeg Ik u, dat vele profeten en rechtvaardigen hebben begeerd te zien de dingen, die gij ziet, en hebben ze niet gezien; en te horen de dingen, die gij hoort, en hebben ze niet gehoord.
\par 18 Gij dan, hoort de gelijkenis van den zaaier.
\par 19 Als iemand dat Woord des Koninkrijks hoort, en niet verstaat, zo komt de boze, en rukt weg, hetgeen in zijn hart gezaaid was; deze is degene, die bij den weg bezaaid is.
\par 20 Maar die in steenachtige plaatsen bezaaid is, deze is degene, die het Woord hoort, en dat terstond met vreugde ontvangt;
\par 21 Doch hij heeft geen wortel in zichzelven, maar is voor een tijd; en als verdrukking of vervolging komt, om des Woords wil, zo wordt hij terstond geergerd.
\par 22 En die in de doornen bezaaid is, deze is degene, die het Woord hoort; en de zorgvuldigheid dezer wereld, en de verleiding des rijkdoms verstikt het Woord, en het wordt onvruchtbaar.
\par 23 Die nu in de goede aarde bezaaid is, deze is degene, die het Woord hoort en verstaat, die ook vrucht draagt en voortbrengt, de een honderd-,de ander zestig-,en de ander dertig voud.
\par 24 Een andere gelijkenis heeft Hij hun voorgesteld, zeggende: Het Koninkrijk der hemelen is gelijk aan een mens, die goed zaad zaaide in zijn akker.
\par 25 En als de mensen sliepen, kwam zijn vijand, en zaaide onkruid midden in de tarwe, en ging weg.
\par 26 Toen het nu tot kruid opgeschoten was, en vrucht voortbracht, toen openbaarde zich ook het onkruid.
\par 27 En de dienstknechten van den heer des huizes gingen en zeiden tot hem: Heere! hebt gij niet goed zaad in uw akker gezaaid? Van waar heeft hij dan dit onkruid?
\par 28 En hij zeide tot hen: Een vijandig mens heeft dat gedaan. En de dienstknechten zeiden tot hem: Wilt gij dan, dat wij heengaan en datzelve vergaderen?
\par 29 Maar hij zeide: Neen, opdat gij, het onkruid vergaderende, ook mogelijk met hetzelve de tarwe niet uittrekt.
\par 30 Laat ze beiden te zamen opwassen tot den oogst, en in den tijd des oogstes zal ik tot de maaiers zeggen: Vergadert eerst dat onkruid, en bindt het in busselen, om hetzelve te verbranden; maar brengt de tarwe samen in mijn schuur.
\par 31 Een andere gelijkenis heeft Hij hun voorgesteld, zeggende: Het Koninkrijk der hemelen is gelijk aan het mosterdzaad, hetwelk een mens heeft genomen en in zijn akker gezaaid;
\par 32 Hetwelk wel het minste is onder al de zaden, maar wanneer het opgewassen is, dan is 't het meeste van de moeskruiden, en het wordt een boom, alzo dat de vogelen des hemels komen en nestelen in zijn takken.
\par 33 Een andere gelijkenis sprak Hij tot hen, zeggende: Het Koninkrijk der hemelen is gelijk aan een zuurdesem, welken een vrouw nam en verborg in drie maten meels, totdat het geheel gezuurd was.
\par 34 Al deze dingen heeft Jezus tot de scharen gesproken door gelijkenissen, en zonder gelijkenis sprak Hij tot hen niet.
\par 35 Opdat vervuld zou worden, wat gesproken is door den profeet, zeggende: Ik zal Mijn mond opendoen door gelijkenissen; Ik zal voortbrengen dingen, die verborgen waren van de grondlegging der wereld.
\par 36 Toen nu Jezus de scharen van Zich gelaten had, ging Hij naar huis. En Zijn discipelen kwamen tot Hem, zeggende: Verklaar ons de gelijkenis van het onkruid des akkers.
\par 37 En Hij, antwoordende, zeide tot hen: Die het goede zaad zaait, is de Zoon des mensen;
\par 38 En de akker is de wereld; en het goede zaad zijn de kinderen des Koninkrijks; en het onkruid zijn de kinderen des bozen;
\par 39 En de vijand, die hetzelve gezaaid heeft, is de duivel; en de oogst is de voleinding der wereld; en de maaiers zijn de engelen.
\par 40 Gelijkerwijs dan het onkruid vergaderd, en met vuur verbrand wordt, alzo zal het ook zijn in de voleinding dezer wereld.
\par 41 De Zoon des mensen zal Zijn engelen uitzenden, en zij zullen uit Zijn Koninkrijk vergaderen al de ergernissen, en degenen, die de ongerechtigheid doen;
\par 42 En zullen dezelve in den vurigen oven werpen; daar zal wening zijn en knersing der tanden.
\par 43 Dan zullen de rechtvaardigen blinken, gelijk de zon, in het Koninkrijk huns Vaders. Die oren heeft om te horen, die hore.
\par 44 Wederom is het Koninkrijk der hemelen gelijk aan een schat, in den akker verborgen, welken een mens gevonden hebbende, verborg dien, en van blijdschap over denzelven, gaat hij heen, en verkoopt al wat hij heeft, en koopt dienzelven akker.
\par 45 Wederom is het Koninkrijk der hemelen gelijk aan een koopman, die schone paarlen zoekt;
\par 46 Dewelke, hebbende een parel van grote waarde gevonden, ging heen en verkocht al wat hij had, en kocht dezelve.
\par 47 Wederom is het Koninkrijk der hemelen gelijk aan een net, geworpen in de zee, en dat allerlei soorten van vissen samenbrengt;
\par 48 Hetwelk, wanneer het vol geworden is, de vissers aan den oever optrekken, en nederzittende, lezen het goede uit in hun vaten, maar het kwade werpen zij weg.
\par 49 Alzo zal het in de voleinding der eeuwen wezen; de engelen zullen uitgaan, en de bozen uit het midden der rechtvaardigen afscheiden;
\par 50 En zullen dezelve in den vurigen oven werpen; daar zal zijn wening en knersing der tanden.
\par 51 En Jezus zeide tot hen: Hebt gij dit alles verstaan? Zij zeiden tot Hem: Ja, Heere!
\par 52 En Hij zeide tot hen: Daarom, een iegelijk Schriftgeleerde, in het Koninkrijk der hemelen onderwezen, is gelijk aan een heer des huizes, die uit zijn schat nieuwe en oude dingen voortbrengt.
\par 53 En het is geschied, als Jezus deze gelijkenissen geeindigd had, vertrok Hij van daar.
\par 54 En gekomen zijnde in Zijn vaderland, leerde Hij hen in hun synagoge, zodat zij zich ontzetten, en zeiden: Van waar komt Dezen die wijsheid en die krachten?
\par 55 Is Deze niet de Zoon des timmermans? en is Zijn moeder niet genaamd Maria, en Zijn broeders Jakobus en Joses, en Simon en Judas?
\par 56 En Zijn zusters, zijn zij niet allen bij ons? Van waar komt dan Dezen dit alles?
\par 57 En zij werden aan Hem geergerd. Maar Jezus zeide tot hen: Een profeet is niet ongeeerd, dan in zijn vaderland, en in zijn huis.
\par 58 En Hij heeft aldaar niet vele krachten gedaan, vanwege hun ongeloof.

\chapter{14}

\par 1 Te dierzelfder tijd hoorde Herodes, de viervorst, het gerucht van Jezus;
\par 2 En zeide tot zijn knechten: Deze is Johannes de Doper; hij is opgewekt van de doden, en daarom werken die krachten in Hem.
\par 3 Want Herodes had Johannes gevangen genomen, en hem gebonden, en in den kerker gezet, om Herodias' wil, de huisvrouw van Filippus, zijn broeder.
\par 4 Want Johannes zeide tot hem: Het is u niet geoorloofd haar te hebben.
\par 5 En willende hem doden, vreesde hij het volk, omdat zij hem hielden voor een profeet.
\par 6 Maar als de dag der geboorte van Herodes gehouden werd, danste de dochter van Herodias in het midden van hen, en zij behaagde aan Herodes.
\par 7 Waarom hij haar met ede beloofde te geven, wat zij ook zou eisen.
\par 8 En zij, te voren onderricht zijnde van haar moeder, zeide: Geef mij hier in een schotel het hoofd van Johannes den Doper.
\par 9 En de koning werd bedroefd; doch om de eden, en degenen, die met hem aanzaten, gebood hij, dat het haar zou gegeven worden;
\par 10 En zond heen, en onthoofdde Johannes in den kerker.
\par 11 En zijn hoofd werd gebracht in een schotel, en het dochtertje gegeven; en zij droeg het tot haar moeder.
\par 12 En zijn discipelen kwamen, en namen het lichaam weg, en begroeven hetzelve; en gingen en boodschapten het Jezus.
\par 13 En als Jezus dit hoorde, vertrok Hij van daar te scheep, naar een woeste plaats alleen; en de scharen, dat horende, zijn Hem te voet gevolgd uit de steden.
\par 14 En Jezus uitgaande, zag een grote schare, en werd innerlijk met ontferming over hen bewogen, en genas hun kranken.
\par 15 En als het nu avond werd, kwamen Zijn discipelen tot Hem, zeggende: Deze plaats is woest, en de tijd is nu voorbijgegaan; laat de scharen van U, opdat zij heengaan in de vlekken en zichzelven spijze kopen.
\par 16 Maar Jezus zeide tot hen: Het is hun niet van node heen te gaan, geeft gij hun te eten.
\par 17 Doch zij zeiden tot Hem: Wij hebben hier niet, dan vijf broden en twee vissen.
\par 18 En Hij zeide: Brengt Mij dezelve hier.
\par 19 En Hij beval de scharen neder te zitten op het gras, en nam de vijf broden en de twee vissen, en opwaarts ziende naar den hemel, zegende dezelve; en als Hij ze gebroken had, gaf Hij de broden den discipelen, en de discipelen aan de scharen.
\par 20 En zij aten allen en werden verzadigd, en zij namen op, het overschot der brokken, twaalf volle korven.
\par 21 Die nu gegeten hadden, waren omtrent vijf duizend mannen, zonder de vrouwen en kinderen.
\par 22 En terstond dwong Jezus Zijn discipelen in het schip te gaan, en voor Hem af te varen naar de andere zijde, terwijl Hij de scharen van Zich zou laten.
\par 23 En als Hij nu de scharen van Zich gelaten had, klom Hij op den berg alleen, om te bidden. En als het nu avond was geworden, zo was Hij daar alleen.
\par 24 En het schip was nu midden in de zee, zijnde in nood van de baren; want de wind was hun tegen.
\par 25 Maar ter vierde wake des nachts kwam Jezus af tot hen, wandelende op de zee.
\par 26 En de discipelen, ziende Hem op de zee wandelen, werden ontroerd, zeggende: Het is een spooksel! En zij schreeuwden van vreze.
\par 27 Maar terstond sprak hen Jezus aan, zeggende: Zijt goedsmoeds, Ik ben het, vreest niet.
\par 28 En Petrus antwoordde Hem, en zeide: Heere! indien Gij het zijt, zo gebied mij tot U te komen op het water.
\par 29 En Hij zeide: Kom. En Petrus klom neder van het schip, en wandelde op het water, om tot Jezus te komen.
\par 30 Maar ziende den sterken wind, werd hij bevreesd, en als hij begon neder te zinken, riep hij, zeggende: Heere, behoud mij!
\par 31 En Jezus, terstond de hand uitstekende, greep hem aan, en zeide tot hem: Gij kleingelovige! waarom hebt gij gewankeld?
\par 32 En als zij in het schip geklommen waren, stilde de wind.
\par 33 Die nu in het schip waren, kwamen en aanbaden Hem, zeggende: Waarlijk, Gij zijt Gods Zoon!
\par 34 En overgevaren zijnde, kwamen zij in het land Gennesaret.
\par 35 En als de mannen van die plaats Hem werden kennende, zonden zij in dat gehele omliggende land, en brachten tot Hem allen, die kwalijk gesteld waren;
\par 36 En baden Hem, dat zij alleenlijk den zoom Zijns kleeds zouden mogen aanraken; en zovelen als Hem aanraakten, werden gezond.

\chapter{15}

\par 1 Toen kwamen tot Jezus enige Schriftgeleerden en Farizeen, die van Jeruzalem waren, zeggende:
\par 2 Waarom overtreden Uw discipelen de inzetting der ouden? Want zij wassen hun handen niet, wanneer zij brood zullen eten.
\par 3 Maar Hij, antwoordende, zeide tot hen: Waarom overtreedt ook gij het gebod Gods, door uw inzetting?
\par 4 Want God heeft geboden, zeggende: Eert uw vader en moeder, en: Wie vader of moeder vloekt, die zal den dood sterven.
\par 5 Maar gij zegt: Zo wie tot vader of moeder zal zeggen: Het is een gave, zo wat u van mij zou kunnen ten nutte komen; en zijn vader of zijn moeder geenszins zal eren, die voldoet.
\par 6 En gij hebt alzo Gods gebod krachteloos gemaakt door uw inzetting.
\par 7 Gij geveinsden! Wel heeft Jesaja van u geprofeteerd, zeggende:
\par 8 Dit volk genaakt Mij met hun mond, en eert Mij met de lippen, maar hun hart houdt zich verre van Mij;
\par 9 Doch tevergeefs eren zij Mij, lerende leringen, die geboden van mensen zijn.
\par 10 En als Hij de schare tot Zich geroepen had, zeide Hij tot hen: Hoort en verstaat.
\par 11 Hetgeen ten monde ingaat, ontreinigt den mens niet; maar hetgeen ten monde uitgaat, dat ontreinigt den mens.
\par 12 Toen kwamen Zijn discipelen tot Hem, en zeiden tot Hem: Weet Gij wel, dat de Farizeen deze rede horende, geergerd zijn geweest?
\par 13 Maar Hij, antwoordende zeide: Alle plant, die Mijn hemelse Vader niet geplant heeft, zal uitgeroeid worden.
\par 14 Laat hen varen; zij zijn blinde leidslieden der blinden. Indien nu de blinde den blinde leidt, zo zullen zij beiden in den gracht vallen.
\par 15 En Petrus, antwoordende, zeide tot Hem: Verklaar ons deze gelijkenis.
\par 16 Maar Jezus zeide: Zijt ook gijlieden alsnog onwetende?
\par 17 Verstaat gij nog niet, dat al wat ten monde ingaat, in den buik komt, en in de heimelijkheid wordt uitgeworpen?
\par 18 Maar die dingen, die ten monde uitgaan, komen voort uit het hart, en dezelve ontreinigen den mens.
\par 19 Want uit het hart komen voort boze bedenkingen, doodslagen, overspelen, hoererijen, dieverijen, valse getuigenissen, lasteringen.
\par 20 Deze dingen zijn het, die den mens ontreinigen; maar het eten met ongewassen handen ontreinigt den mens niet.
\par 21 En Jezus van daar gaande, vertrok naar de delen van Tyrus en Sidon.
\par 22 En ziet, een Kananese vrouw, uit die landpalen komende, riep tot Hem, zeggende: Heere! Gij Zone Davids, ontferm U mijner! mijn dochter is deerlijk van den duivel bezeten.
\par 23 Doch Hij antwoordde haar niet een woord. En Zijn discipelen, tot Hem komende, baden Hem, zeggende: Laat haar van U; want zij roept ons na.
\par 24 Maar Hij, antwoordende, zeide: Ik ben niet gezonden, dan tot de verloren schapen van het huis Israels.
\par 25 En zij kwam en aanbad Hem, zeggende: Heere, help mij!
\par 26 Doch Hij antwoordde en zeide: Het is niet betamelijk het brood der kinderen te nemen, en den hondekens voor te werpen.
\par 27 En zij zeide: Ja, Heere! doch de hondekens eten ook van de brokjes die er vallen van de tafel van hun heren.
\par 28 Toen antwoordde Jezus, en zeide tot haar: O vrouw! groot is uw geloof; u geschiede, gelijk gij wilt. En haar dochter werd gezond van diezelfde ure.
\par 29 En Jezus, van daar vertrekkende, kwam aan de zee van Galilea, en klom op den berg, en zat daar neder.
\par 30 En vele scharen zijn tot Hem gekomen, hebbende bij zich kreupelen, blinden, stommen, lammen, en vele anderen, en wierpen ze voor de voeten van Jezus; en Hij genas dezelve.
\par 31 Alzo dat de scharen zich verwonderden, ziende de stommen sprekende, de lammen gezond, de kreupelen wandelende, en de blinden ziende; en zij verheerlijkten den God Israels.
\par 32 En Jezus, Zijn discipelen tot Zich geroepen hebbende, zeide: Ik word innerlijk met ontferming bewogen over de schare, omdat zij nu drie dagen bij Mij gebleven zijn, en hebben niet wat zij eten zouden; en Ik wil hen niet nuchteren van Mij laten, opdat zij op den weg niet bezwijken.
\par 33 En Zijn discipelen zeiden tot Hem: Van waar zullen wij zovele broden in de woestijn bekomen, dat wij zulk een grote schare zouden verzadigen?
\par 34 En Jezus zeide tot hen: Hoevele broden hebt gij? Zij zeiden: Zeven, en weinige visjes.
\par 35 En Hij gebood den scharen neder te zitten op de aarde.
\par 36 En Hij nam de zeven broden en de vissen, en als Hij gedankt had, brak Hij ze, en gaf ze Zijn discipelen; en de discipelen gaven ze aan de schare.
\par 37 En zij aten allen en werden verzadigd, en zij namen op, het overschot der brokken, zeven volle manden.
\par 38 En die daar gegeten hadden, waren vier duizend mannen, zonder de vrouwen en kinderen.
\par 39 En de scharen van Zich gelaten hebbende, ging Hij in het schip, en kwam in de landpalen van Magdala.

\chapter{16}

\par 1 En de Farizeen en Sadduceen tot Hem gekomen zijnde, en Hem verzoekende, begeerden van Hem, dat Hij hun een teken uit den hemel zou tonen.
\par 2 Maar Hij antwoordde, en zeide tot hen: Als het avond geworden is, zegt gij: Schoon weder; want de hemel is rood;
\par 3 En des morgens: Heden onweder; want de hemel is droevig rood. Gij geveinsden! het aanschijn des hemels weet gij wel te onderscheiden, en kunt gij de tekenen der tijden niet onderscheiden?
\par 4 Het boos en overspelig geslacht verzoekt een teken; en hun zal geen teken gegeven worden, dan het teken van Jona, den profeet. En hen verlatende, ging Hij weg.
\par 5 En als Zijn discipelen op de andere zijde gekomen waren, hadden zij vergeten broden mede te nemen.
\par 6 En Jezus zeide tot hen: Ziet toe, en wacht u van den zuurdesem der Farizeen en Sadduceen.
\par 7 En zij overlegden bij zichzelven, zeggende: Het is omdat wij geen broden mede genomen hebben.
\par 8 En Jezus, dat wetende, zeide tot hen: Wat overlegt gij bij uzelven, gij kleingelovigen! dat gij geen broden mede genomen hebt?
\par 9 Verstaat gij nog niet? en gedenkt gij niet aan de vijf broden der vijf duizend mannen; en hoevele korven gij opnaamt?
\par 10 Noch aan de zeven broden der vier duizend mannen, en hoevele manden gij opnaamt?
\par 11 Hoe verstaat gij niet, dat Ik u van geen brood gesproken heb, als Ik zeide, dat gij u wachten zoudt van den zuurdesem der Farizeen en Sadduceen.
\par 12 Toen verstonden zij, dat Hij niet gezegd had, dat zij zich wachten zouden van den zuurdesem des broods, maar van de leer der Farizeen en Sadduceen?
\par 13 Als nu Jezus gekomen was in de delen van Cesarea Filippi, vraagde Hij Zijn discipelen, zeggende: Wie zeggen de mensen, dat Ik, de Zoon des mensen, ben?
\par 14 En zij zeiden: Sommigen: Johannes de Doper; en anderen: Elias; en anderen: Jeremia of een van de profeten.
\par 15 Hij zeide tot hen: Maar gij, wie zegt gij, dat Ik ben?
\par 16 En Simon Petrus, antwoordende, zeide: Gij zijt de Christus, de Zoon des levenden Gods.
\par 17 En Jezus, antwoordende, zeide tot hem: Zalig zijt gij, Simon, Bar-jona! want vlees en bloed heeft u dat niet geopenbaard, maar Mijn Vader, Die in de hemelen is.
\par 18 En Ik zeg u ook, dat gij zijt Petrus, en op deze petra zal Ik Mijn gemeente bouwen, en de poorten der hel zullen dezelve niet overweldigen.
\par 19 En Ik zal u geven de sleutelen van het Koninkrijk der hemelen; en zo wat gij zult binden op de aarde, zal in de hemelen gebonden zijn; en zo wat gij ontbinden zult op de aarde, zal in de hemelen ontbonden zijn.
\par 20 Toen verbood Hij Zijn discipelen, dat zij iemand zeggen zouden, dat Hij was Jezus, de Christus.
\par 21 Van toen aan begon Jezus Zijn discipelen te vertonen, dat Hij moest heengaan naar Jeruzalem, en veel lijden van de ouderlingen, en overpriesteren, en Schriftgeleerden, en gedood worden, en ten derden dage opgewekt worden.
\par 22 En Petrus, Hem tot zich genomen hebbende, begon Hem te bestraffen, zeggende: Heere, wees U genadig! dit zal U geenszins geschieden.
\par 23 Maar Hij, Zich omkerende, zeide tot Petrus: Ga weg achter Mij, satanas! gij zijt Mij een aanstoot, want gij verzint niet de dingen, die Gods zijn, maar die der mensen zijn.
\par 24 Toen zeide Jezus tot Zijn discipelen: Zo iemand achter Mij wil komen, die verloochene zichzelven, en neme zijn kruis op, en volge Mij.
\par 25 Want zo wie zijn leven zal willen behouden, die zal hetzelve verliezen; maar zo wie zijn leven verliezen zal, om Mijnentwil, die zal hetzelve vinden.
\par 26 Want wat baat het een mens, zo hij de gehele wereld gewint, en lijdt schade zijner ziel? Of wat zal een mens geven, tot lossing van zijn ziel?
\par 27 Want de Zoon des mensen zal komen in de heerlijkheid Zijns Vaders, met Zijn engelen, en alsdan zal Hij een iegelijk vergelden naar zijn doen.
\par 28 Voorwaar zeg Ik u: Er zijn sommigen van die hier staan, dewelke den dood niet smaken zullen, totdat zij den Zoon des mensen zullen hebben zien komen in Zijn Koninkrijk.

\chapter{17}

\par 1 En na zes dagen nam Jezus met Zich Petrus, en Jakobus, en Johannes, zijn broeder, en bracht hen op een hogen berg alleen.
\par 2 En Hij werd voor hen veranderd van gedaante; en Zijn aangezicht blonk gelijk de zon, en Zijn klederen werden wit gelijk het licht.
\par 3 En ziet, van hen werden gezien Mozes en Elias, met Hem samensprekende.
\par 4 En Petrus, antwoordende, zeide tot Jezus: Heere! het is goed, dat wij hier zijn; zo Gij wilt, laat ons hier drie tabernakelen maken, voor U een, en voor Mozes een, en een voor Elias.
\par 5 Terwijl hij nog sprak, ziet, een luchtige wolk heeft hen overschaduwd; en ziet, een stem uit de wolk, zeggende: Deze is Mijn geliefde Zoon, in Denwelken Ik Mijn welbehagen heb; hoort Hem!
\par 6 En de discipelen, dit horende, vielen op hun aangezicht, en werden zeer bevreesd.
\par 7 En Jezus, bij hen komende, raakte hen aan, en zeide: Staat op en vreest niet.
\par 8 En hun ogen opheffende, zagen zij niemand, dan Jezus alleen.
\par 9 En als zij van den berg afkwamen, gebood hun Jezus, zeggende: Zegt niemand dit gezicht, totdat de Zoon des mensen zal opgestaan zijn uit de doden.
\par 10 En Zijn discipelen vraagden Hem, zeggende: Wat zeggen dan de Schriftgeleerden, dat Elias eerst moet komen?
\par 11 Doch Jezus, antwoordende, zeide tot hen: Elias zal wel eerst komen, en alles weder oprichten.
\par 12 Maar Ik zeg u, dat Elias nu gekomen is, en zij hebben hem niet gekend; doch zij hebben aan hem gedaan, al wat zij hebben gewild; alzo zal ook de Zoon des mensen van hen lijden.
\par 13 Toen verstonden de discipelen dat Hij hun van Johannes de Doper gesproken had.
\par 14 En als zij bij de schare gekomen waren, kwam tot Hem een mens, vallende voor Hem op de knieen, en zeggende:
\par 15 Heere! ontferm U over mijn zoon; want hij is maanziek, en is in zwaar lijden; want menigmaal valt hij in het vuur, en menigmaal in het water.
\par 16 En ik heb hem tot Uw discipelen gebracht, en zij hebben hem niet kunnen genezen.
\par 17 En Jezus, antwoordende, zeide: O, ongelovig en verkeerd geslacht, hoe lang zal Ik nog met ulieden zijn, hoe lang zal Ik u nog verdragen? Brengt hem Mij hier.
\par 18 En Jezus bestrafte hem, en de duivel ging van hem uit, en het kind werd genezen van die ure af.
\par 19 Toen kwamen de discipelen tot Jezus alleen, en zeiden: Waarom hebben wij hem niet kunnen uitwerpen?
\par 20 En Jezus zeide tot hen: Om uws ongeloofs wil; want voorwaar zeg Ik u: Zo gij een geloof hadt als een mosterdzaad, gij zoudt tot dezen berg zeggen: Ga heen van hier derwaarts, en hij zal heengaan; en niets zal u onmogelijk zijn.
\par 21 Maar dit geslacht vaart niet uit, dan door bidden en vasten.
\par 22 En als zij in Galilea verkeerden, zeide Jezus tot hen: De Zoon des mensen zal overgeleverd worden in de handen der mensen;
\par 23 En zij zullen Hem doden, en ten derden dage zal Hij opgewekt worden. En zij werden zeer bedroefd.
\par 24 En als zij te Kapernaum ingekomen waren, gingen tot Petrus die de didrachmen ontvingen, en zeiden: Uw Meester, betaalt Hij de didrachmen niet?
\par 25 Hij zeide: Ja. En toen hij in huis gekomen was, voorkwam hem Jezus, zeggende: Wat dunkt u, Simon! de koningen der aarde, van wie nemen zij tollen of schatting, van hun zonen, of van de vreemden?
\par 26 Petrus zeide tot Hem: Van de vreemden. Jezus zeide tot hem: Zo zijn dan de zonen vrij.
\par 27 Maar opdat wij hun geen aanstoot geven, ga heen naar de zee, werp den angel uit, en den eersten vis, die opkomt, neem, en zijn mond geopend hebbende, zult gij een stater vinden; neem dien, en geef hem aan hen voor Mij en u.

\chapter{18}

\par 1 Te dierzelfder ure kwamen de discipelen tot Jezus, zeggende: Wie is toch de meeste in het Koninkrijk der hemelen?
\par 2 En Jezus een kindeken tot Zich geroepen hebbende, stelde dat in het midden van hen;
\par 3 En zeide: Voorwaar zeg Ik u: Indien gij u niet verandert, en wordt gelijk de kinderkens, zo zult gij in het Koninkrijk der hemelen geenszins ingaan.
\par 4 Zo wie dan zichzelven zal vernederen, gelijk dit kindeken, deze is de meeste in het Koninkrijk der hemelen.
\par 5 En zo wie zodanig een kindeken ontvangt in Mijn Naam, die ontvangt Mij.
\par 6 Maar zo wie een van deze kleinen, die in Mij geloven, ergert, het ware hem nutter, dat een molensteen aan zijn hals gehangen, en dat hij verzonken ware in de diepte der zee.
\par 7 Wee der wereld van de ergernissen, want het is noodzakelijk, dat de ergernissen komen; doch wee dien mens, door welken de ergernis komt!
\par 8 Indien dan uw hand of uw voet u ergert, houwt ze af en werpt ze van u. Het is u beter, tot het leven in te gaan, kreupel of verminkt zijnde, dan twee handen of twee voeten hebbende, in het eeuwige vuur geworpen te worden.
\par 9 En indien uw oog u ergert, trekt het uit, en werpt het van u. Het is u beter, maar een oog hebbende, tot het leven in te gaan, dan twee ogen hebbende, in het helse vuur geworpen te worden.
\par 10 Ziet toe, dat gij niet een van deze kleinen veracht. Want Ik zeg ulieden, dat hun engelen, in de hemelen, altijd zien het aangezicht Mijns Vaders, Die in de hemelen is.
\par 11 Want de Zoon des mensen is gekomen om zalig te maken, dat verloren was.
\par 12 Wat dunkt u, indien enig mens honderd schapen had, en een uit dezelve afgedwaald ware, zal hij niet de negen en negentig laten, en op de bergen heengaande, het afgedwaalde zoeken?
\par 13 En indien het geschiedt, dat hij hetzelve vindt, voorwaar zeg Ik u, dat hij zich meer verblijdt over hetzelve, dan over de negen en negentig, die niet afgedwaald zijn geweest.
\par 14 Alzo is de wil niet uws Vaders, Die in de hemelen is, dat een van deze kleinen verloren ga.
\par 15 Maar indien uw broeder tegen u gezondigd heeft, ga heen en bestraf hem tussen u en hem alleen; indien hij u hoort, zo hebt gij uw broeder gewonnen.
\par 16 Maar indien hij u niet hoort, zo neem nog een of twee met u; opdat in den mond van twee of drie getuigen alle woord besta.
\par 17 En indien hij denzelven geen gehoor geeft; zo zeg het der gemeente; en indien hij ook der gemeente geen gehoor geeft, zo zij hij u als de heiden en de tollenaar.
\par 18 Voorwaar zeg Ik u: Al wat gij op de aarde binden zult, zal in den hemel gebonden wezen; en al wat gij op de aarde ontbinden zult, zal in den hemel ontbonden wezen.
\par 19 Wederom zeg Ik u: Indien er twee van u samenstemmen op de aarde, over enige zaak, die zij zouden mogen begeren, dat die hun zal geschieden van Mijn Vader, Die in de hemelen is.
\par 20 Want waar twee of drie vergaderd zijn in Mijn Naam, daar ben Ik in het midden van hen.
\par 21 Toen kwam Petrus tot Hem, en zeide: Heere! hoe menigmaal zal mijn broeder tegen mij zondigen, en ik hem vergeven! Tot zevenmaal?
\par 22 Jezus zeide tot hem: Ik zeg u, niet tot zevenmaal, maar tot zeventigmaal zeven maal.
\par 23 Daarom wordt het Koninkrijk der hemelen vergeleken bij een zeker koning, die rekening met zijn dienstknechten houden wilde.
\par 24 Als hij nu begon te rekenen, werd tot hem gebracht een, die hem schuldig was tien duizend talenten.
\par 25 En als hij niet had, om te betalen, beval zijn heer, dat men hem zou verkopen, en zijn vrouw en kinderen, en al wat hij had, en dat de schuld zou betaald worden.
\par 26 De dienstknecht dan, nedervallende, aanbad hem, zeggende: Heer! wees lankmoedig over mij, en ik zal u alles betalen.
\par 27 En de heer van dezen dienstknecht, met barmhartigheid innerlijk bewogen zijnde, heeft hem ontslagen, en de schuld hem kwijtgescholden.
\par 28 Maar dezelve dienstknecht, uitgaande, heeft gevonden een zijner mededienstknechten, die hem honderd penningen schuldig was, en hem aanvattende, greep hem bij de keel, zeggende: Betaal mij, wat gij schuldig zijt.
\par 29 Zijn mededienstknecht dan, nedervallende aan zijn voeten, bad hem, zeggende: Wees lankmoedig over mij, en ik zal u alles betalen.
\par 30 Doch hij wilde niet, maar ging heen, en wierp hem in de gevangenis, totdat hij de schuld zou betaald hebben.
\par 31 Als nu zijn mededienstknechten zagen, hetgeen geschied was, zijn zij zeer bedroefd geworden; en komende, verklaarden zij hunnen heer al wat er geschied was.
\par 32 Toen heeft hem zijn heer tot zich geroepen, en zeide tot hem: Gij boze dienstknecht, al die schuld heb ik u kwijtgescholden, dewijl gij mij gebeden hebt;
\par 33 Behoordet gij ook niet u over uw mededienstknecht te ontfermen, gelijk ik ook mij over u ontfermd heb?
\par 34 En zijn heer, vertoornd zijnde, leverde hem den pijnigers over, totdat hij zou betaald hebben al wat hij hem schuldig was.
\par 35 Alzo zal ook Mijn hemelse Vader u doen, indien gij niet van harte vergeeft een iegelijk zijn broeder zijn misdaden.

\chapter{19}

\par 1 En het geschiedde, toen Jezus deze woorden geeindigd had, dat Hij vertrok van Galilea, en kwam over de Jordaan, in de landpalen van Judea.
\par 2 En vele scharen volgden Hem, en Hij genas ze aldaar.
\par 3 En de Farizeen kwamen tot Hem, verzoekende Hem, en zeggende tot Hem: Is het een mens geoorloofd zijn vrouw te verlaten, om allerlei oorzaak?
\par 4 Doch Hij, antwoordende, zeide tot hen: Hebt gij niet gelezen, Die van den beginne den mens gemaakt heeft, dat Hij ze gemaakt heeft man en vrouw?
\par 5 En gezegd heeft: Daarom zal een mens vader en moeder verlaten, en zal zijn vrouw aanhangen, en die twee zullen tot een vlees zijn;
\par 6 Alzo dat zij niet meer twee zijn, maar een vlees. Hetgeen dan God samengevoegd heeft, scheide de mens niet.
\par 7 Zij zeiden tot hem: Waarom heeft dan Mozes geboden een scheidbrief te geven en haar te verlaten?
\par 8 Hij zeide tot hen: Mozes heeft vanwege de hardigheid uwer harten u toegelaten uw vrouwen te verlaten; maar van den beginne is het alzo niet geweest.
\par 9 Maar Ik zeg u, dat zo wie zijn vrouw verlaat, anders dan om hoererij, en een andere trouwt, die doet overspel, en die de verlatene trouwt, doet ook overspel.
\par 10 Zijn discipelen zeiden tot Hem: Indien de zaak des mensen met de vrouw alzo staat, zo is het niet oorbaar te trouwen.
\par 11 Doch Hij zeide tot hen: Allen vatten dit woord niet, maar dien het gegeven is.
\par 12 Want er zijn gesnedenen, die uit moeders lijf alzo geboren zijn; en er zijn gesnedenen, die van de mensen gesneden zijn; en er zijn gesnedenen, die zichzelven gesneden hebben, om het Koninkrijk der hemelen. Die dit vatten kan, vatte het.
\par 13 Toen werden kinderkens tot Hem gebracht, opdat Hij de handen hun zou opleggen en bidden; en de discipelen bestraften dezelve.
\par 14 Maar Jezus zeide: Laat af van de kinderkens, en verhindert hen niet tot Mij te komen; want derzulken is het Koninkrijk der hemelen.
\par 15 En als Hij hun de handen opgelegd had, vertrok Hij van daar.
\par 16 En ziet, er kwam een tot Hem, en zeide tot Hem: Goede Meester! wat zal ik goeds doen, opdat ik het eeuwige leven hebbe?
\par 17 En Hij zeide tot hem: Wat noemt gij Mij goed? Niemand is goed dan Een, namelijk God. Doch wilt gij in het leven ingaan, onderhoud de geboden.
\par 18 Hij zeide tot Hem: Welke? En Jezus zeide: Deze: Gij zult niet doden; gij zult geen overspel doen; gij zult niet stelen; gij zult geen valse getuigenis geven;
\par 19 Eer uw vader en moeder; en: Gij zult uw naaste liefhebben als uzelven.
\par 20 De jongeling zeide tot Hem: Al deze dingen heb ik onderhouden van mijn jonkheid af; wat ontbreekt mij nog?
\par 21 Jezus zeide tot hem: Zo gij wilt volmaakt zijn, ga heen, verkoop wat gij hebt, en geef het den armen, en gij zult een schat hebben in den hemel; en kom herwaarts, volg Mij.
\par 22 Als nu de jongeling dit woord hoorde, ging hij bedroefd weg; want hij had vele goederen.
\par 23 En Jezus zeide tot Zijn discipelen: Voorwaar, Ik zeg u, dat een rijke bezwaarlijk in het Koninkrijk der hemelen zal ingaan.
\par 24 En wederom zeg Ik u: Het is lichter, dat een kemel ga door het oog van een naald, dan dat een rijke inga in het Koninkrijk Gods.
\par 25 Zijn discipelen nu, dit horende, werden zeer verslagen, zeggende: Wie kan dan zalig worden?
\par 26 En Jezus, hen aanziende, zeide tot hen: Bij de mensen is dat onmogelijk, maar bij God zijn alle dingen mogelijk.
\par 27 Toen antwoordde Petrus, en zeide tot Hem: Zie, wij hebben alles verlaten, en zijn U gevolgd, wat zal ons dan geworden?
\par 28 En Jezus zeide tot hen: Voorwaar, Ik zeg u, dat gij, die Mij gevolgd zijt, in de wedergeboorte, wanneer de Zoon des mensen zal gezeten zijn op den troon Zijner heerlijkheid, dat gij ook zult zitten op twaalf tronen, oordelende de twaalf geslachten Israels.
\par 29 En zo wie zal verlaten hebben, huizen, of broeders, of zusters, of vader, of moeder, of vrouw, of kinderen, of akkers, om Mijns Naams wil, die zal honderdvoud ontvangen, en het eeuwige leven beerven.
\par 30 Maar vele eersten zullen de laatsten zijn, en vele laatsten de eersten.

\chapter{20}

\par 1 Want het Koninkrijk der hemelen is gelijk een heer des huizes, die met den morgenstond uitging, om arbeiders te huren in zijn wijngaard.
\par 2 En als hij met de arbeiders eens geworden was, voor een penning des daags, zond hij hen heen in zijn wijngaard.
\par 3 En uitgegaan zijnde omtrent de derde ure, zag hij anderen, ledig staande op de markt.
\par 4 En hij zeide tot dezelve: Gaat ook gij heen in den wijngaard, en zo wat recht is, zal ik u geven. En zij gingen.
\par 5 Wederom uitgegaan zijnde omtrent de zesde en negende ure, deed hij desgelijks.
\par 6 En uitgegaan zijnde omtrent de elfde ure, vond hij anderen ledig staande, en zeide tot hen: Wat staat gij hier den gehelen dag ledig?
\par 7 Zij zeiden tot hem: Omdat ons niemand gehuurd heeft. Hij zeide tot hen: Gaat ook gij heen in den wijngaard, en zo wat recht is, zult gij ontvangen.
\par 8 Als het nu avond geworden was, zeide de heer des wijngaards, tot zijn rentmeester: Roep de arbeiders, en geef hun het loon, beginnende van de laatsten tot de eersten.
\par 9 En als zij kwamen, die ter elfder ure gehuurd waren, ontvingen zij ieder een penning.
\par 10 En de eersten komende, meenden, dat zij meer ontvangen zouden; en zij zelven ontvingen ook elk een penning.
\par 11 En dien ontvangen hebbende, murmureerden zij tegen den heer des huizes,
\par 12 Zeggende: Deze laatsten hebben maar een uur gearbeid, en gij hebt ze ons gelijk gemaakt, die den last des daags en de hitte gedragen hebben.
\par 13 Doch hij, antwoordende, zeide tot een van hen: Vriend! ik doe u geen onrecht; zijt gij niet met mij eens geworden voor een penning?
\par 14 Neem het uwe en ga heen. Ik wil dezen laatsten ook geven, gelijk als u.
\par 15 Of is het mij niet geoorloofd, te doen met het mijne, wat ik wil? Of is uw oog boos, omdat ik goed ben?
\par 16 Alzo zullen de laatsten de eersten zijn, en de eersten de laatsten; want velen zijn geroepen, maar weinigen uitverkoren.
\par 17 En Jezus, opgaande naar Jeruzalem, nam tot Zich de twaalf discipelen alleen op den weg, en zeide tot hen:
\par 18 Ziet, wij gaan op naar Jeruzalem, en de Zoon des mensen zal den overpriesteren en Schriftgeleerden overgeleverd worden, en zij zullen Hem ter dood veroordelen;
\par 19 En zij zullen Hem den heidenen overleveren, om Hem te bespotten en te geselen, en te kruisigen; en ten derden dage zal Hij weder opstaan.
\par 20 Toen kwam de moeder der zonen van Zebedeus tot Hem met haar zonen, Hem aanbiddende, en begerende wat van Hem.
\par 21 En Hij zeide tot haar: Wat wilt gij? Zij zeide tot Hem: Zeg, dat deze mijn twee zonen zitten mogen, de een tot Uw rechter-en de ander tot Uw linker hand in Uw Koninkrijk.
\par 22 Maar Jezus antwoordde en zeide: Gijlieden weet niet wat gij begeert; kunt gij den drinkbeker drinken, dien Ik drinken zal, en met den doop gedoopt worden, waarmede Ik gedoopt worde? Zij zeiden tot Hem: Wij kunnen.
\par 23 En Hij zeide tot hen: Mijn drinkbeker zult gij wel drinken, en met den doop, waarmede Ik gedoopt worde, zult gij gedoopt worden; maar het zitten tot Mijn rechter-,en tot Mijn linker hand, staat bij Mij niet te geven, maar het zal gegeven worden dien het bereid is van Mijn Vader.
\par 24 En als de andere tien dat hoorden, namen zij het zeer kwalijk van de twee broeders.
\par 25 En als Jezus hen tot Zich geroepen had, zeide Hij: Gij weet, dat de oversten der volken heerschappij voeren over hen, en de groten gebruiken macht over hen.
\par 26 Doch alzo zal het onder u niet zijn; maar zo wie onder u zal willen groot worden, die zij uw dienaar;
\par 27 En zo wie onder u zal willen de eerste zijn, die zij uw dienstknecht.
\par 28 Gelijk de Zoon des mensen niet is gekomen om gediend te worden, maar om te dienen, en Zijn ziel te geven tot een rantsoen voor velen.
\par 29 En als zij van Jericho uitgingen, is Hem een grote schare gevolgd.
\par 30 En ziet, twee blinden, zittende aan den weg, als zij hoorden, dat Jezus voorbijging, riepen, zeggende: Heere, Gij Zone Davids! ontferm U onzer.
\par 31 En de schare bestrafte hen, opdat zij zwijgen zouden; maar zij riepen te meer, zeggende: Ontferm U onzer, Heere, Gij Zone Davids!
\par 32 En Jezus, stil staande, riep hen en zeide: Wat wilt gij, dat Ik u doe?
\par 33 Zij zeiden tot Hem: Heere! dat onze ogen geopend worden.
\par 34 En Jezus, innerlijk bewogen zijnde met barmhartigheid, raakte hun ogen aan; en terstond werden hun ogen ziende, en zij volgden Hem.

\chapter{21}

\par 1 En als zij nu Jeruzalem genaakten, en gekomen waren te Beth-fage, aan den Olijfberg, toen zond Jezus twee discipelen, zeggende tot hen:
\par 2 Gaat heen in het vlek, dat tegen u over ligt, en gij zult terstond een ezelin gebonden vinden, en een veulen met haar; ontbindt ze, en brengt ze tot Mij.
\par 3 En indien u iemand iets zegt, zo zult gij zeggen, dat de Heere deze van node heeft, en hij zal ze terstond zenden.
\par 4 Dit alles nu is geschied, opdat vervuld worde, hetgeen gesproken is door den profeet, zeggende:
\par 5 Zegt der dochter Sions: Zie, uw Koning komt tot u, zachtmoedig en gezeten op een ezelin en een veulen, zijnde een jong ener jukdragende ezelin.
\par 6 En de discipelen heengegaan zijnde, en gedaan hebbende, gelijk Jezus hun bevolen had,
\par 7 Brachten de ezelin en het veulen, en leiden hun klederen op dezelve, en zetten Hem daarop.
\par 8 En de meeste schare spreidden hun klederen op den weg, en anderen hieuwen takken van de bomen, en spreidden ze op den weg.
\par 9 En de scharen, die voorgingen en die volgden, riepen, zeggende: Hosanna den Zone Davids! Gezegend is Hij, Die komt in den Naam des Heeren! Hosanna in de hoogste hemelen!
\par 10 En als Hij te Jeruzalem inkwam, werd de gehele stad beroerd, zeggende: Wie is Deze?
\par 11 En de scharen zeiden: Deze is Jezus, de Profeet van Nazareth in Galilea.
\par 12 En Jezus ging in den tempel Gods, en dreef uit allen, die verkochten en kochten in den tempel, en keerde om de tafelen der wisselaars, en de zitstoelen dergenen, die de duiven verkochten.
\par 13 En Hij zeide tot hen: Er is geschreven: Mijn huis zal een huis des gebeds genaamd worden; maar gij hebt dat tot een moordenaarskuil gemaakt.
\par 14 En er kwamen blinden en kreupelen tot Hem in den tempel, en Hij genas dezelve.
\par 15 Als nu de overpriesters en Schriftgeleerden zagen de wonderheden, die Hij deed, en de kinderen, roepende in den tempel, en zeggende: Hosanna den Zone Davids! namen zij dat zeer kwalijk;
\par 16 En zeiden tot Hem: Hoort Gij wel, wat dezen zeggen? En Jezus zeide tot hen: Ja; hebt gij nooit gelezen: Uit den mond der jonge kinderen en der zuigelingen hebt Gij U lof toebereid?
\par 17 En hen verlatende, ging Hij van daar uit de stad, naar Bethanie, en overnachtte aldaar.
\par 18 En des morgens vroeg, als Hij wederkeerde naar de stad, hongerde Hem.
\par 19 En ziende, een vijgeboom aan den weg, ging Hij naar hem toe, en vond niets aan denzelven, dan alleenlijk bladeren; en zeide tot hem: Uit u worde geen vrucht meer in der eeuwigheid! En de vijgeboom verdorde terstond.
\par 20 En de discipelen, dat ziende, verwonderden zich, zeggende: Hoe is de vijgeboom zo terstond verdord?
\par 21 Doch Jezus, antwoordende, zeide tot hen: Voorwaar zeg Ik u: Indien gij geloof hadt, en niet twijfeldet, gij zoudt niet alleenlijk doen, hetgeen den vijgeboom is geschied; maar indien gij ook tot dezen berg zeidet: Word opgeheven en in de zee geworpen! het zou geschieden.
\par 22 En al wat gij zult begeren in het gebed, gelovende, zult gij ontvangen.
\par 23 En als Hij in den tempel gekomen was, kwamen tot Hem, terwijl Hij leerde, de overpriesters en de ouderlingen des volks, zeggende: Door wat macht doet Gij deze dingen? En Wie heeft U deze macht gegeven?
\par 24 En Jezus, antwoordende, zeide tot hen: Ik zal u ook een woord vragen, hetwelk indien gij Mij zult zeggen, zo zal Ik u ook zeggen, door wat macht Ik deze dingen doe.
\par 25 De doop van Johannes, van waar was die, uit den hemel, of uit de mensen? En zij overlegden bij zichzelven en zeiden: Indien wij zeggen: Uit den hemel; zo zal Hij ons zeggen: Waarom hebt gij hem dan niet geloofd?
\par 26 En indien wij zeggen: Uit de mensen: zo vrezen wij de schare; want zij houden allen Johannes voor een profeet.
\par 27 En zij, Jezus antwoordende, zeiden: Wij weten het niet. En Hij zeide tot hen: Zo zeg Ik u ook niet, door wat macht Ik dit doe.
\par 28 Maar wat dunkt u? Een mens had twee zonen, en gaande tot den eersten, zeide: Zoon! ga heen, werk heden in mijn wijngaard.
\par 29 Doch hij antwoordde en zeide: Ik wil niet; en daarna berouw hebbende, ging hij heen.
\par 30 En gaande tot den tweeden, zeide desgelijks, en deze antwoordde en zeide: Ik ga, heer! en hij ging niet.
\par 31 Wie van deze twee heeft den wil des vaders gedaan? Zij zeiden tot Hem: De eerste. Jezus zeide tot hen: Voorwaar, Ik zeg u, dat de tollenaars en de hoeren u voorgaan in het Koninkrijk Gods.
\par 32 Want Johannes is tot u gekomen in den weg der gerechtigheid, en gij hebt hem niet geloofd; maar de tollenaars en de hoeren hebben hem geloofd; doch gij, zulks ziende, hebt daarna geen berouw gehad, om hem te geloven.
\par 33 Hoort een andere gelijkenis. Er was een heer des huizes, die een wijngaard plantte, en zette een tuin daarom, en groef een wijnpersbak daarin, en bouwde een toren, en verhuurde dien den landlieden, en reisde buiten 's lands.
\par 34 Toen nu de tijd der vruchten genaakte, zond hij zijn dienstknechten tot de landlieden, om zijn vruchten te ontvangen.
\par 35 En de landlieden, nemende zijn dienstknechten, hebben den een geslagen, en den anderen gedood, en den derden gestenigd.
\par 36 Wederom zond hij andere dienstknechten, meer in getal dan de eersten, en zij deden hun desgelijks.
\par 37 En ten laatste zond hij tot hen zijn zoon, zeggende: Zij zullen mijn zoon ontzien.
\par 38 Maar de landlieden, den zoon ziende, zeiden onder elkander: Deze is de erfgenaam, komt, laat ons hem doden, en zijn erfenis aan ons behouden.
\par 39 En hem nemende, wierpen zij hem uit, buiten den wijngaard, en doodden hem.
\par 40 Wanneer dan de heer des wijngaards komen zal, wat zal hij dien landlieden doen?
\par 41 Zij zeiden tot hem: Hij zal den kwaden een kwaden dood aandoen, en zal den wijngaard aan andere landlieden verhuren, die hem de vruchten op haar tijden zullen geven.
\par 42 Jezus zeide tot hen: Hebt gij nooit gelezen in de Schriften: De steen, dien de bouwlieden verworpen hebben, deze is geworden tot een hoofd des hoeks; van den Heere is dit geschied, en het is wonderlijk in onze ogen?
\par 43 Daarom zeg Ik ulieden, dat het Koninkrijk Gods van u zal weggenomen worden, en een volk gegeven, dat zijn vruchten voortbrengt.
\par 44 En wie op dezen steen valt, die zal verpletterd worden; en op wien hij valt, dien zal hij vermorzelen.
\par 45 En als de overpriesters en Farizeen deze Zijn gelijkenissen hoorden, verstonden zij, dat Hij van hen sprak.
\par 46 En zoekende Hem te vangen, vreesden zij de scharen, dewijl deze Hem hielden voor een profeet.

\chapter{22}

\par 1 En Jezus, antwoordende, sprak tot hen wederom door gelijkenissen, zeggende:
\par 2 Het Koninkrijk der hemelen is gelijk een zeker koning, die zijn zoon een bruiloft bereid had;
\par 3 En zond zijn dienstknechten uit, om de genoden ter bruiloft te roepen; en zij wilden niet komen.
\par 4 Wederom zond hij andere dienstknechten uit, zeggende: Zegt den genoden: Ziet, ik heb mijn middagmaal bereid; mijn ossen, en de gemeste beesten zijn geslacht, en alle dingen zijn gereed; komt tot de bruiloft.
\par 5 Maar zij, zulks niet achtende, zijn heengegaan, deze tot zijn akker, gene tot zijn koopmanschap.
\par 6 En de anderen grepen zijn dienstknechten, deden hun smaadheid aan, en doodden hen.
\par 7 Als nu de koning dat hoorde, werd hij toornig, en zijn krijgsheiren zendende, heeft die doodslagers vernield, en hun stad in brand gestoken.
\par 8 Toen zeide hij tot zijn dienstknechten: De bruiloft is wel bereid, doch de genoden waren het niet waardig.
\par 9 Daarom gaat op de uitgangen der wegen, en zovelen als gij er zult vinden, roept ze tot de bruiloft.
\par 10 En dezelve dienstknechten, uitgaande op de wegen, vergaderden allen, die zij vonden, beiden kwaden en goeden; en de bruiloft werd vervuld met aanzittende gasten.
\par 11 En als de koning ingegaan was, om de aanzittende gasten te overzien, zag hij aldaar een mens, niet gekleed zijnde met een bruiloftskleed;
\par 12 En zeide tot hem: Vriend! hoe zijt gij hier ingekomen, geen bruiloftskleed aan hebbende? En hij verstomde.
\par 13 Toen zeide de koning tot de dienaars: Bindt zijn handen en voeten, neemt hem weg, en werpt hem uit in de buitenste duisternis; daar zal zijn wening en knersing der tanden.
\par 14 Want velen zijn geroepen, maar weinigen uitverkoren.
\par 15 Toen gingen de Farizeen heen, en hielden te zamen raad, hoe zij Hem verstrikken zouden in Zijn rede.
\par 16 En zij zonden uit tot Hem hun discipelen, met de Herodianen, zeggende: Meester! wij weten, dat Gij waarachtig zijt, en den weg Gods in der waarheid leert, en naar niemand vraagt; want Gij ziet den persoon der mensen niet aan;
\par 17 Zeg ons dan: wat dunkt U? Is het geoorloofd, den keizer schatting te geven of niet?
\par 18 Maar Jezus, bekennende hun boosheid, zeide:
\par 19 Gij geveinsden, wat verzoekt gij Mij? Toont Mij de schattingpenning. En zij brachten Hem een penning.
\par 20 En Hij zeide tot hen: Wiens is dit beeld en het opschrift?
\par 21 Zij zeiden tot Hem: Des keizers. Toen zeide Hij tot hen: Geeft dan den keizer, dat des keizers is, en Gode, dat Gods is.
\par 22 En zij, dit horende, verwonderden zich, en Hem verlatende, zijn zij weggegaan.
\par 23 Te dienzelfden dage kwamen tot Hem de Sadduceen, die zeggen, dat er geen opstanding is, en vraagden Hem.
\par 24 Zeggende: Meester! Mozes heeft gezegd: Indien iemand sterft, geen kinderen hebbende, zo zal zijn broeder deszelfs vrouw trouwen, en zijn broeder zaad verwekken.
\par 25 Nu waren er bij ons zeven broeders; en de eerste, een vrouw getrouwd hebbende, stierf; en dewijl hij geen zaad had, zo liet hij zijn vrouw voor zijn broeder.
\par 26 Desgelijks ook de tweede, en de derde, tot den zevende toe.
\par 27 Ten laatste na allen, is ook de vrouw gestorven.
\par 28 In de opstanding dan, wiens vrouw zal zij wezen van die zeven, want zij hebben ze allen gehad?
\par 29 Maar Jezus antwoordde en zeide tot hen: Gij dwaalt, niet wetende de Schriften, noch de kracht Gods.
\par 30 Want in de opstanding nemen zij niet ten huwelijk, noch worden ten huwelijk uitgegeven; maar zij zijn als engelen Gods in den hemel.
\par 31 En wat aangaat de opstanding der doden, hebt gij niet gelezen, hetgeen van God tot ulieden gesproken is, Die daar zegt:
\par 32 Ik ben de God Abrahams, en de God Izaks, en de God Jakobs! God is niet een God der doden, maar der levenden.
\par 33 En de scharen, dit horende, werden verslagen over Zijn leer.
\par 34 En den Farizeen, gehoord hebbende, dat Hij den Sadduceen den mond gestopt had, zijn te zamen bijeenvergaderd.
\par 35 En een uit hen, zijnde een wetgeleerde, heeft gevraagd, Hem verzoekende, en zeggende:
\par 36 Meester! welk is het grote gebod in de wet?
\par 37 En Jezus zeide tot hem: Gij zult liefhebben den Heere, uw God, met geheel uw hart, en met geheel uw ziel, en met geheel uw verstand.
\par 38 Dit is het eerste en het grote gebod.
\par 39 En het tweede aan dit gelijk, is: Gij zult uw naaste liefhebben als uzelven.
\par 40 Aan deze twee geboden hangt de ganse wet en de profeten.
\par 41 Als nu de Farizeen samenvergaderd waren, vraagde hun Jezus,
\par 42 En zeide: Wat dunkt u van den Christus? Wiens Zoon is Hij? Zij zeiden tot Hem: Davids Zoon.
\par 43 Hij zeide tot hen: Hoe noemt Hem dan David, in den Geest, zijn Heere? zeggende:
\par 44 De Heere heeft gezegd tot Mijn Heere: Zit aan Mijn rechter hand, totdat Ik Uw vijanden zal gezet hebben tot een voetbank Uwer voeten.
\par 45 Indien Hem dan David noemt zijn Heere, hoe is Hij zijn Zoon?
\par 46 En niemand kon Hem een woord antwoorden; noch iemand durfde Hem van dien dag aan iets meer vragen.

\chapter{23}

\par 1 Toen sprak Jezus tot de scharen en tot Zijn discipelen,
\par 2 Zeggende: De Schriftgeleerden en de Farizeen zijn gezeten op den stoel van Mozes;
\par 3 Daarom, al wat zij u zeggen, dat gij houden zult, houdt dat en doet het; maar doet niet naar hun werken; want zij zeggen het, en doen het niet.
\par 4 Want zij binden lasten, die zwaar zijn en kwalijk om te dragen, en leggen ze op de schouderen der mensen; maar zij willen die met hun vinger niet verroeren.
\par 5 En al hun werken doen zij, om van de mensen gezien te worden; want zij maken hun gedenkcedels breed, en maken de zomen van hun klederen groot.
\par 6 En zij beminnen de vooraanzitting in de maaltijden, en de voorgestoelten in de synagogen;
\par 7 Ook de begroetingen op de markten, en van de mensen genaamd te worden: Rabbi, Rabbi!
\par 8 Doch gij zult niet Rabbi genaamd worden; want Een is uw Meester, namelijk Christus; en gij zijt allen broeders.
\par 9 En gij zult niemand uw vader noemen op de aarde; want Een is uw Vader, namelijk Die in de hemelen is.
\par 10 Noch zult gij meesters genoemd worden; want Een is uw Meester, namelijk Christus.
\par 11 Maar de meeste van u zal uw dienaar zijn.
\par 12 En wie zichzelven verhogen zal, die zal vernederd worden; en wie zichzelven zal vernederen, die zal verhoogd worden.
\par 13 Maar wee u, gij Schriftgeleerden en Farizeen, gij geveinsden! want gij sluit het Koninkrijk der hemelen voor de mensen, overmits gij daar niet ingaat, noch degenen, die ingaan zouden, laat ingaan.
\par 14 Wee u, gij Schriftgeleerden en Farizeen, gij geveinsden, want gij eet de huizen der weduwen op, en dat onder den schijn van lang te bidden; daarom zult gij te zwaarder oordeel ontvangen.
\par 15 Wee u, gij Schriftgeleerden en Farizeen, gij geveinsden, want gij omreist zee en land, om een Jodengenoot te maken, en als hij het geworden is, zo maakt gij hem een kind der helle, tweemaal meer dan gij zijt.
\par 16 Wee u, gij blinde leidslieden, die zegt: Zo wie gezworen zal hebben bij den tempel, dat is niets; maar zo wie gezworen zal hebben bij het goud des tempels, die is schuldig.
\par 17 Gij dwazen en blinden, want wat is meerder, het goud, of de tempel, die het goud heiligt?
\par 18 En zo wie gezworen zal hebben bij het altaar, dat is niets; maar zo wie gezworen zal hebben bij de gave, die daarop is, die is schuldig.
\par 19 Gij dwazen en blinden, want wat is meerder, de gave, of het altaar, dat de gave heiligt?
\par 20 Daarom wie zweert bij het altaar, die zweert bij hetzelve, en bij al wat daarop is.
\par 21 En wie zweert bij den tempel, die zweert bij denzelven, en bij Dien, Die daarin woont.
\par 22 En wie zweert bij den hemel, die zweert bij den troon Gods, en bij Dien, Die daarop zit.
\par 23 Wee u, gij Schriftgeleerden en Farizeen, gij geveinsden, want gij vertient de munte, en de dille, en den komijn, en gij laat na het zwaarste der wet, namelijk het oordeel, en de barmhartigheid, en het geloof. Deze dingen moest men doen, en de andere niet nalaten.
\par 24 Gij blinde leidslieden, die de mug uitzijgt, en den kemel doorzwelgt.
\par 25 Wee u, gij Schriftgeleerden en Farizeen, gij geveinsden, want gij reinigt het buitenste des drinkbekers, en des schotels, maar van binnen zijn zij vol van roof en onmatigheid.
\par 26 Gij blinde Farizeer, reinig eerst wat binnen in den drinkbeker en den schotel is, opdat ook het buitenste derzelve rein worde.
\par 27 Wee u, gij Schriftgeleerden en Farizeen, gij geveinsden, want gij zijt den witgepleisterden graven gelijk, die van buiten wel schoon schijnen, maar van binnen zijn zij vol doodsbeenderen en alle onreinigheid.
\par 28 Alzo ook schijnt gij wel den mensen van buiten rechtvaardig, maar van binnen zijt gij vol geveinsdheid en ongerechtigheid.
\par 29 Wee u, gij Schriftgeleerden en Farizeen, gij geveinsden, want gij bouwt de graven der profeten op, en versiert de graftekenen der rechtvaardigen;
\par 30 En zegt: Indien wij in de tijden onzer vaderen waren geweest, wij zouden met hen geen gemeenschap gehad hebben aan het bloed der profeten.
\par 31 Aldus getuigt gij tegen uzelven, dat gij kinderen zijt dergenen, die de profeten gedood hebben.
\par 32 Gij dan ook, vervult de mate uwer vaderen!
\par 33 Gij slangen, gij adderengebroedsels! hoe zoudt gij de helse verdoemenis ontvlieden?
\par 34 Daarom ziet, Ik zende tot u profeten, en wijzen, en schriftgeleerden, en uit dezelve zult gij sommigen doden en kruisigen, en sommigen uit dezelve zult gij geselen in uw synagogen, en zult hen vervolgen van stad tot stad;
\par 35 Opdat op u kome al het rechtvaardige bloed, dat vergoten is op de aarde, van het bloed des rechtvaardigen Abels af, tot op het bloed van Zacharia, den zoon van Barachia, welken gij gedood hebt tussen den tempel en het altaar.
\par 36 Voorwaar zeg Ik u: Al deze dingen zullen komen over dit geslacht.
\par 37 Jeruzalem, Jeruzalem! gij, die de profeten doodt, en stenigt, die tot u gezonden zijn! hoe menigmaal heb Ik uw kinderen willen bijeenvergaderen, gelijkerwijs een hen haar kiekens bijeenvergadert onder de vleugels; en gijlieden hebt niet gewild.
\par 38 Ziet, uw huis wordt u woest gelaten.
\par 39 Want Ik zeg u: Gij zult Mij van nu aan niet zien, totdat gij zeggen zult: Gezegend is Hij, Die komt in den Naam des Heeren!

\chapter{24}

\par 1 En Jezus ging uit en vertrok van den tempel; en Zijn discipelen kwamen bij Hem, om Hem de gebouwen des tempels te tonen.
\par 2 En Jezus zeide tot hen: Ziet gij niet al deze dingen? Voorwaar zeg Ik: Hier zal niet een steen op den anderen steen gelaten worden, die niet afgebroken zal worden.
\par 3 En als Hij op den Olijfberg gezeten was, gingen de discipelen tot Hem alleen, zeggende: Zeg ons, wanneer zullen deze dingen zijn, en welk zal het teken zijn van Uw toekomst, en van de voleinding der wereld?
\par 4 En Jezus, antwoordende, zeide tot hen: Ziet toe, dat u niemand verleide.
\par 5 Want velen zullen komen onder Mijn Naam, zeggende: Ik ben de Christus; en zij zullen velen verleiden.
\par 6 En gij zult horen van oorlogen, en geruchten van oorlogen; ziet toe, wordt niet verschrikt; want al die dingen moeten geschieden, maar nog is het einde niet.
\par 7 Want het ene volk zal tegen het andere volk opstaan, en het ene koninkrijk tegen het andere koninkrijk; en er zullen zijn hongersnoden, en pestilentien, en aardbevingen in verscheidene plaatsen.
\par 8 Doch al die dingen zijn maar een beginsel der smarten.
\par 9 Alsdan zullen zij u overleveren in verdrukking, en zullen u doden, en gij zult gehaat worden van alle volken, om Mijns Naams wil.
\par 10 En dan zullen er velen geergerd worden, en zullen elkander overleveren, en elkander haten.
\par 11 En vele valse profeten zullen opstaan, en zullen er velen verleiden.
\par 12 En omdat de ongerechtigheid vermenigvuldigd zal worden, zo zal de liefde van velen verkouden.
\par 13 Maar wie volharden zal tot het einde, die zal zalig worden.
\par 14 En dit Evangelie des Koninkrijks zal in de gehele wereld gepredikt worden tot een getuigenis allen volken; en dan zal het einde komen.
\par 15 Wanneer gij dan zult zien den gruwel der verwoesting, waarvan gesproken is door Daniel, den profeet, staande in de heilige plaats; (die het leest, die merke daarop!)
\par 16 Dat alsdan, die in Judea zijn, vlieden op de bergen;
\par 17 Die op het dak is, kome niet af, om iets uit zijn huis weg te nemen;
\par 18 En die op den akker is, kere niet weder terug, om zijn klederen weg te nemen.
\par 19 Maar wee den bevruchten, en den zogenden vrouwen in die dagen!
\par 20 Doch bidt, dat uw vlucht niet geschiede des winters, noch op een sabbat.
\par 21 Want alsdan zal grote verdrukking wezen, hoedanige niet is geweest van het begin der wereld tot nu toe, en ook niet zijn zal.
\par 22 En zo die dagen niet verkort werden, geen vlees zou behouden worden; maar om der uitverkorenen wil zullen die dagen verkort worden.
\par 23 Alsdan, zo iemand tot ulieden zal zeggen: Ziet, hier is de Christus, of daar, gelooft het niet.
\par 24 Want er zullen valse christussen en valse profeten opstaan, en zullen grote tekenen en wonderheden doen, alzo dat zij (indien het mogelijk ware) ook de uitverkorenen zouden verleiden.
\par 25 Ziet, Ik heb het u voorzegd!
\par 26 Zo zij dan tot u zullen zeggen: Ziet, hij is in de woestijn; gaat niet uit; Ziet, hij is in de binnenkameren; gelooft het niet.
\par 27 Want gelijk de bliksem uitgaat van het oosten, en schijnt tot het westen, alzo zal ook de toekomst van den Zoon des mensen wezen.
\par 28 Want alwaar het dode lichaam zal zijn, daar zullen de arenden vergaderd worden.
\par 29 En terstond na de verdrukking dier dagen, zal de zon verduisterd worden, en de maan zal haar schijnsel niet geven, en de sterren zullen van den hemel vallen, en de krachten der hemelen zullen bewogen worden.
\par 30 En alsdan zal in den hemel verschijnen het teken van den Zoon des mensen; en dan zullen al de geslachten der aarde wenen, en zullen den Zoon des mensen zien, komende op de wolken des hemels, met grote kracht en heerlijkheid.
\par 31 En Hij zal Zijn engelen uitzenden met een bazuin van groot geluid, en zij zullen Zijn uitverkorenen bijeenvergaderen uit de vier winden, van het ene uiterste der hemelen tot het andere uiterste derzelve.
\par 32 En leert van den vijgeboom deze gelijkenis: wanneer zijn tak nu teder wordt, en de bladeren uitspruiten, zo weet gij, dat de zomer nabij is.
\par 33 Alzo ook gijlieden, wanneer gij al deze dingen zult zien, zo weet, dat het nabij is, voor de deur.
\par 34 Voorwaar, Ik zeg u: Dit geslacht zal geenszins voorbijgaan, totdat al deze dingen zullen geschied zijn.
\par 35 De hemel en de aarde zullen voorbijgaan, maar Mijn woorden zullen geenszins voorbijgaan.
\par 36 Doch van dien dag en die ure weet niemand, ook niet de engelen der hemelen, dan Mijn Vader alleen.
\par 37 En gelijk de dagen van Noach waren, alzo zal ook zijn de toekomst van den Zoon des mensen.
\par 38 Want gelijk zij waren in de dagen voor den zondvloed, etende en drinkende, trouwende en ten huwelijk uitgevende, tot den dag toe, in welken Noach in de ark ging;
\par 39 En bekenden het niet, totdat de zondvloed kwam, en hen allen wegnam; alzo zal ook zijn de toekomst van den Zoon des mensen.
\par 40 Alsdan zullen er twee op den akker zijn, de een zal aangenomen, en de ander zal verlaten worden.
\par 41 Er zullen twee vrouwen malen in den molen, de ene zal aangenomen, en de andere zal verlaten worden.
\par 42 Waakt dan; want gij weet niet, in welke ure uw Heere komen zal.
\par 43 Maar weet dit, dat zo de heer des huizes geweten had, in welke nachtwake de dief komen zou, hij zou gewaakt hebben, en zou zijn huis niet hebben laten doorgraven.
\par 44 Daarom, zijt ook gij bereid; want in welke ure gij het niet meent, zal de Zoon des mensen komen.
\par 45 Wie is dan de getrouwe en voorzichtige dienstknecht, denwelken zijn heer over zijn dienstboden gesteld heeft, om hunlieder hun voedsel te geven ter rechter tijd?
\par 46 Zalig is die dienstknecht, welken zijn heer, komende, zal vinden alzo doende.
\par 47 Voorwaar, Ik zeg u, dat hij hem zal zetten over al zijn goederen.
\par 48 Maar zo die kwade dienstknecht in zijn hart zou zeggen: Mijn heer vertoeft te komen;
\par 49 En zou beginnen zijn mededienstknechten te slaan, en te eten en te drinken met de dronkaards;
\par 50 Zo zal de heer van dezen dienstknecht komen ten dage, in welken hij hem niet verwacht, en ter ure, die hij niet weet;
\par 51 En zal hem afscheiden, en zijn deel zetten met de geveinsden; daar zal wening zijn en knersing der tanden.

\chapter{25}

\par 1 Alsdan zal het Koninkrijk der hemelen gelijk zijn aan tien maagden, welke haar lampen namen, en gingen uit, den bruidegom tegemoet.
\par 2 En vijf van haar waren wijzen, en vijf waren dwazen.
\par 3 Die dwaas waren, haar lampen nemende, namen geen olie met zich.
\par 4 Maar de wijzen namen olie in haar vaten, met haar lampen.
\par 5 Als nu de bruidegom vertoefde, werden zij allen sluimerig, en vielen in slaap.
\par 6 En ter middernacht geschiedde een geroep: Ziet, de bruidegom komt, gaat uit hem tegemoet!
\par 7 Toen stonden al die maagden op, en bereidden haar lampen.
\par 8 En de dwazen zeiden tot de wijzen: Geeft ons van uw olie; want onze lampen gaan uit.
\par 9 Doch de wijzen antwoordden, zeggende: Geenszins, opdat er misschien voor ons en voor u niet genoeg zij; maar gaat liever tot de verkopers, en koopt voor uzelven.
\par 10 Als zij nu heengingen om te kopen, kwam de bruidegom; en die gereed waren, gingen met hem in tot de bruiloft, en de deur werd gesloten.
\par 11 Daarna kwamen ook de andere maagden, zeggende: Heer, heer, doe ons open!
\par 12 En hij, antwoordende, zeide: Voorwaar zeg ik u: Ik ken u niet.
\par 13 Zo waakt dan; want gij weet den dag niet, noch de ure, in dewelke de Zoon des mensen komen zal.
\par 14 Want het is gelijk een mens, die buiten 's lands reizende, zijn dienstknechten riep, en gaf hun zijn goederen over.
\par 15 En den ene gaf hij vijf talenten, en den ander twee, en den derden een, een iegelijk naar zijn vermogen, en verreisde terstond.
\par 16 Die nu de vijf talenten ontvangen had, ging heen, en handelde daarmede, en won andere vijf talenten.
\par 17 Desgelijks ook die de twee ontvangen had, die won ook andere twee.
\par 18 Maar die het ene ontvangen had, ging heen en groef in de aarde, en verborg het geld zijns heren.
\par 19 En na een langen tijd kwam de heer van dezelve dienstknechten, en hield rekening met hen.
\par 20 En die de vijf talenten ontvangen had, kwam, en bracht tot hem andere vijf talenten, zeggende: Heer, vijf talenten hebt gij mij gegeven; zie, andere vijf talenten heb ik boven dezelve gewonnen.
\par 21 En zijn heer zeide tot hem: Wel, gij goede en getrouwe dienstknecht! over weinig zijt gij getrouw geweest; over veel zal ik u zetten; ga in, in de vreugde uws heeren.
\par 22 En die de twee talenten ontvangen had, kwam ook tot hem, en zeide: Heer, twee talenten hebt gij mij gegeven; zie, twee andere talenten heb ik boven dezelve gewonnen.
\par 23 Zijn heer zeide tot hem: Wel, gij goede en getrouwe dienstknecht, over weinig zijt gij getrouw geweest; over veel zal ik u zetten; ga in, in de vreugde uws heeren.
\par 24 Maar die het ene talent ontvangen had, kwam ook en zeide: Heer! ik kende u, dat gij een hard mens zijt, maaiende, waar gij niet gezaaid hebt, en vergaderende van daar, waar gij niet gestrooid hebt;
\par 25 En bevreesd zijnde, ben ik heengegaan, en heb uw talent verborgen in de aarde; zie, gij hebt het uwe.
\par 26 Maar zijn heer, antwoordende, zeide tot hem: Gij boze en luie dienstknecht! gij wist, dat ik maai, waar ik niet gezaaid heb, en van daar vergader, waar ik niet gestrooid heb.
\par 27 Zo moest gij dan mijn geld den wisselaren gedaan hebben, en ik, komende, zou het mijne wedergenomen hebben met woeker.
\par 28 Neemt dan van hem het talent weg, en geeft het dengene, die de tien talenten heeft.
\par 29 Want een iegelijk die heeft, dien zal gegeven worden, en hij zal overvloedig hebben; maar van dengene, die niet heeft, van dien zal genomen worden, ook dat hij heeft.
\par 30 En werpt den onnutten dienstknecht uit in de buitenste duisternis; daar zal wening zijn en knersing der tanden.
\par 31 En wanneer de Zoon des mensen komen zal in Zijn heerlijkheid, en al de heilige engelen met Hem, dan zal Hij zitten op den troon Zijner heerlijkheid.
\par 32 En voor Hem zullen al de volken vergaderd worden, en Hij zal ze van elkander scheiden, gelijk de herder de schapen van de bokken scheidt.
\par 33 En Hij zal de schapen tot Zijn rechter hand zetten, maar de bokken tot Zijn linker hand.
\par 34 Alsdan zal de Koning zeggen tot degenen, die tot Zijn rechter hand zijn: Komt, gij gezegenden Mijns Vaders! beerft dat Koninkrijk, hetwelk u bereid is van de grondlegging der wereld.
\par 35 Want Ik ben hongerig geweest, en gij hebt Mij te eten gegeven; Ik ben dorstig geweest, en gij hebt Mij te drinken gegeven; Ik was een vreemdeling, en gij hebt Mij geherbergd.
\par 36 Ik was naakt, en gij hebt Mij gekleed; Ik ben krank geweest, en gij hebt Mij bezocht; Ik was in de gevangenis, en gij zijt tot Mij gekomen.
\par 37 Dan zullen de rechtvaardigen Hem antwoorden, zeggende: Heere! wanneer hebben wij U hongerig gezien, en gespijzigd, of dorstig, en te drinken gegeven?
\par 38 En wanneer hebben wij U een vreemdeling gezien, en geherbergd, of naakt en gekleed?
\par 39 En wanneer hebben wij U krank gezien, of in de gevangenis, en zijn tot U gekomen?
\par 40 En de Koning zal antwoorden en tot hen zeggen: Voorwaar zeg Ik u: Voor zoveel gij dit een van deze Mijn minste broeders gedaan hebt, zo hebt gij dat Mij gedaan.
\par 41 Dan zal Hij zeggen ook tot degenen, die ter linker hand zijn: Gaat weg van Mij, gij vervloekten, in het eeuwige vuur, hetwelk den duivel en zijn engelen bereid is.
\par 42 Want Ik ben hongerig geweest, en gij hebt Mij niet te eten gegeven; Ik ben dorstig geweest, en gij hebt Mij niet te drinken gegeven;
\par 43 Ik was een vreemdeling; en gij hebt Mij niet geherbergd; naakt, en gij hebt Mij niet gekleed; krank, en in de gevangenis, en gij hebt Mij niet bezocht.
\par 44 Dan zullen ook dezen Hem antwoorden, zeggende: Heere, wanneer hebben wij U hongerig gezien, of dorstig, of een vreemdeling, of naakt, of krank, of in de gevangenis, en hebben U niet gediend?
\par 45 Dan zal Hij hun antwoorden en zeggen: Voorwaar zeg Ik u: Voor zoveel gij dit een van deze minsten niet gedaan hebt, zo hebt gij het Mij ook niet gedaan.
\par 46 En dezen zullen gaan in de eeuwige pijn; maar de rechtvaardigen in het eeuwige leven.

\chapter{26}

\par 1 En het is geschied, als Jezus al deze woorden geeindigd had, dat Hij tot Zijn discipelen zeide:
\par 2 Gij weet, dat na twee dagen het pascha is, en de Zoon des mensen zal overgeleverd worden, om gekruisigd te worden.
\par 3 Toen vergaderden de overpriesters en de Schriftgeleerden, en de ouderlingen des volks, in de zaal des hogepriesters, die genaamd was Kajafas;
\par 4 En zij beraadslaagden te zamen, dat zij Jezus met listigheid vangen en doden zouden.
\par 5 Doch zij zeiden: Niet in het feest, opdat er geen oproer worde onder het volk.
\par 6 Als nu Jezus te Bethanie was, ten huize van Simon, den melaatse,
\par 7 Kwam tot Hem een vrouw, hebbende een albasten fles met zeer kostelijke zalf, en goot ze uit op Zijn hoofd, daar Hij aan tafel zat.
\par 8 En Zijn discipelen, dat ziende, namen het zeer kwalijk, zeggende: Waartoe dit verlies?
\par 9 Want deze zalf had kunnen duur verkocht, en de penningen den armen gegeven worden.
\par 10 Maar Jezus, zulks verstaande, zeide tot hen: Waarom doet gij deze vrouw moeite aan? want zij heeft een goed werk aan Mij gewrocht.
\par 11 Want de armen hebt gij altijd met u, maar Mij hebt gij niet altijd.
\par 12 Want als zij deze zalf op Mijn lichaam gegoten heeft, zo heeft zij het gedaan tot een voorbereiding van Mijn begrafenis.
\par 13 Voorwaar zeg Ik u: Alwaar dit Evangelie gepredikt zal worden in de gehele wereld, daar zal ook tot haar gedachtenis gesproken worden van hetgeen zij gedaan heeft.
\par 14 Toen ging een van de twaalven, genaamd Judas Iskariot, tot de overpriesters,
\par 15 En zeide: Wat wilt gij mij geven, en ik zal Hem u overleveren? En zij hebben hem toegelegd dertig zilveren penningen.
\par 16 En van toen af zocht hij gelegenheid, opdat hij Hem overleveren mocht.
\par 17 En op den eersten dag der ongehevelde broden kwamen de discipelen tot Jezus, zeggende tot Hem: Waar wilt Gij, dat wij U bereiden het pascha te eten?
\par 18 En Hij zeide: Gaat heen in de stad, tot zulk een, en zegt hem: De Meester zegt: Mijn tijd is nabij, Ik zal bij u het pascha houden met Mijn discipelen.
\par 19 En de discipelen deden, gelijk Jezus hun bevolen had, en bereidden het pascha.
\par 20 En als het avond geworden was, zat Hij aan met de twaalven.
\par 21 En toen zij aten, zeide Hij: Voorwaar, Ik zeg u, dat een van u, Mij zal verraden.
\par 22 En zij, zeer bedroefd geworden zijnde, begon een iegelijk van hen tot Hem te zeggen: Ben ik het, Heere?
\par 23 En Hij, antwoordende, zeide: Die de hand met Mij in den schotel indoopt, die zal Mij verraden.
\par 24 De Zoon des mensen gaat wel heen, gelijk van Hem geschreven is; maar wee dien mens, door welken de Zoon des mensen verraden wordt; het ware hem goed, zo die mens niet geboren was geweest.
\par 25 En Judas, die Hem verried, antwoordde en zeide: Ben ik het, Rabbi? Hij zeide tot hem: Gij hebt het gezegd.
\par 26 En als zij aten, nam Jezus het brood, en gezegend hebbende, brak Hij het, en gaf het den discipelen, en zeide: Neemt, eet, dat is Mijn lichaam.
\par 27 En Hij nam den drinkbeker, en gedankt hebbende, gaf hun dien, zeggende: Drinkt allen daaruit;
\par 28 Want dat is Mijn bloed, het bloed des Nieuwen Testaments, hetwelk voor velen vergoten wordt, tot vergeving der zonden.
\par 29 En Ik zeg u, dat Ik van nu aan niet zal drinken van de vrucht des wijnstoks tot op dien dag, wanneer Ik met u dezelve nieuw zal drinken in het Koninkrijk Mijns Vaders.
\par 30 En als zij den lofzang gezongen hadden, gingen zij uit naar den Olijfberg.
\par 31 Toen zeide Jezus tot hen: Gij zult allen aan Mij geergerd worden in dezen nacht; want er is geschreven: Ik zal den Herder slaan, en de schapen der kudde zullen verstrooid worden.
\par 32 Maar nadat Ik zal opgestaan zijn, zal Ik u voorgaan naar Galilea.
\par 33 Doch Petrus, antwoordende, zeide tot Hem: Al werden zij ook allen aan U geergerd, ik zal nimmermeer geergerd worden.
\par 34 Jezus zeide tot hem: Voorwaar, Ik zeg u, dat gij in dezen zelfden nacht, eer de haan gekraaid zal hebben, Mij driemaal zult verloochenen.
\par 35 Petrus zeide tot Hem: Al moest ik ook met U sterven, zo zal ik U geenszins verloochenen! Desgelijks zeiden ook al de discipelen.
\par 36 Toen ging Jezus met hen in een plaats genaamd Gethsemane, en zeide tot de discipelen: Zit hier neder, totdat Ik heenga, en aldaar zal gebeden hebben.
\par 37 En met Zich nemende Petrus, en de twee zonen van Zebedeus, begon Hij droevig en zeer beangst te worden.
\par 38 Toen zeide Hij tot hen: Mijn ziel is geheel bedroefd tot den dood toe; blijft hier en waakt met Mij.
\par 39 En een weinig voortgegaan zijnde, viel Hij op Zijn aangezicht, biddende en zeggende: Mijn Vader, indien het mogelijk is, laat dezen drinkbeker van Mij voorbijgaan! doch niet, gelijk Ik wil, maar gelijk Gij wilt.
\par 40 En Hij kwam tot de discipelen en vond hen slapende, en zeide tot Petrus: Kunt gij dan niet een uur met Mij waken?
\par 41 Waakt en bidt, opdat gij niet in verzoeking komt; de geest is wel gewillig, maar het vlees is zwak.
\par 42 Wederom ten tweeden male heengaande, bad Hij, zeggende: Mijn Vader! Indien deze drinkbeker van Mij niet voorbij kan gaan, tenzij dat Ik hem drinke, Uw wil geschiede!
\par 43 En komende bij hen, vond Hij hen wederom slapende; want hun ogen waren bezwaard.
\par 44 En hen latende, ging Hij wederom heen, en bad ten derden male, zeggende dezelfde woorden.
\par 45 Toen kwam Hij tot Zijn discipelen, en zeide tot hen: Slaapt nu voort, en rust; ziet, de ure is nabij gekomen, en de Zoon des mensen wordt overgeleverd in de handen der zondaren.
\par 46 Staat op, laat ons gaan; ziet, hij is nabij, die Mij verraadt.
\par 47 En als Hij nog sprak, ziet, Judas, een van de twaalven, kwam, en met hem een grote schare, met zwaarden en stokken, gezonden van de overpriesters en ouderlingen des volks.
\par 48 En die Hem verried, had hun een teken gegeven, zeggende: Dien ik zal kussen, Dezelve is het, grijpt Hem.
\par 49 En terstond komende tot Jezus, zeide hij: Wees gegroet, Rabbi! en hij kuste Hem.
\par 50 Maar Jezus zeide tot hem: Vriend! waartoe zijt gij hier! Toen kwamen zij toe, en sloegen de handen aan Jezus en grepen Hem.
\par 51 En ziet, een van degenen, die met Jezus waren, de hand uitstekende, trok zijn zwaard uit, en slaande den dienstknecht des hogepriesters, hieuw zijn oor af.
\par 52 Toen zeide Jezus tot hem: Keer uw zwaard weder in zijn plaats; want allen, die het zwaard nemen, zullen door het zwaard vergaan.
\par 53 Of meent gij, dat Ik Mijn Vader nu niet kan bidden, en Hij zal Mij meer dan twaalf legioenen engelen bijzetten?
\par 54 Hoe zouden dan de Schriften vervuld worden, die zeggen, dat het alzo geschieden moet?
\par 55 Terzelfder ure sprak Jezus tot de scharen: Gij zijt uitgegaan als tegen een moordenaar, met zwaarden en stokken, om Mij te vangen; dagelijks zat Ik bij u, lerende in den tempel, en gij hebt Mij niet gegrepen;
\par 56 Doch dit alles is geschied, opdat de Schriften der profeten zouden vervuld worden. Toen vluchtten al de discipelen, Hem verlatende.
\par 57 Die nu Jezus gevangen hadden, leidden Hem heen tot Kajafas, den hogepriester, alwaar de Schriftgeleerden en ouderlingen vergaderd waren.
\par 58 En Petrus volgde Hem van verre tot aan de zaal des hogepriesters, en binnengegaan zijnde, zat hij bij de dienaren, om het einde te zien.
\par 59 En de overpriesters, en de ouderlingen, en de gehele grote raad zochten valse getuigenis tegen Jezus, opdat zij Hem doden mochten; en vonden niet.
\par 60 En hoewel er vele valse getuigen gekomen waren, zo vonden zij toch niet.
\par 61 Maar ten laatste kwamen twee valse getuigen, en zeiden: Deze heeft gezegd: Ik kan den tempel Gods afbreken, en in drie dagen denzelven opbouwen.
\par 62 En de hogepriester, opstaande, zeide tot Hem: Antwoordt Gij niets? Wat getuigen dezen tegen U?
\par 63 Doch Jezus zweeg stil. En de hogepriester, antwoordende, zeide tot Hem: Ik bezweer U bij den levenden God, dat Gij ons zegt, of Gij zijt de Christus, de Zoon van God?
\par 64 Jezus zeide tot hem: Gij hebt het gezegd. Doch Ik zeg ulieden: Van nu aan zult gij zien den Zoon des mensen, zittende ter rechter hand der kracht Gods, en komende op de wolken des hemels.
\par 65 Toen verscheurde de hogepriester zijn klederen, zeggende: Hij heeft God gelasterd, wat hebben wij nog getuigen van node? Ziet, nu hebt gij Zijn gods lastering gehoord.
\par 66 Wat dunkt ulieden? En zij, antwoordende, zeiden: Hij is des doods schuldig.
\par 67 Toen spogen zij in Zijn aangezicht, en sloegen Hem met vuisten.
\par 68 En anderen gaven Hem kinnebakslagen, zeggende: Profeteer ons, Christus, wie is het, die U geslagen heeft?
\par 69 En Petrus zat buiten in de zaal; en een dienstmaagd kwam tot hem, zeggende: Gij waart ook met Jezus, den Galileer.
\par 70 Maar hij loochende het voor allen, zeggende: Ik weet niet, wat gij zegt.
\par 71 En als hij naar de voorpoort uitging, zag hem een andere dienstmaagd, en zeide tot degenen, die aldaar waren: Deze was ook met Jezus den Nazarener.
\par 72 En hij loochende het wederom met een eed, zeggende: Ik ken den Mens niet.
\par 73 En een weinig daarna, die er stonden, bijkomende, zeiden tot Petrus: Waarlijk, gij zijt ook van die, want ook uw spraak maakt u openbaar.
\par 74 Toen begon hij zich te vervloeken, en te zweren: Ik ken den Mens niet.
\par 75 En terstond kraaide de haan; en Petrus werd indachtig het woord van Jezus, Die tot hem gezegd had: Eer de haan gekraaid zal hebben, zult gij Mij driemaal verloochenen. En naar buiten gaande, weende hij bitterlijk.

\chapter{27}

\par 1 Als het nu morgenstond geworden was, hebben al de overpriesters en de ouderlingen des volks te zamen raad genomen tegen Jezus, dat zij Hem doden zouden.
\par 2 En Hem gebonden hebbende, leidden zij Hem weg, en gaven Hem over aan Pontius Pilatus, den stadhouder.
\par 3 Toen heeft Judas, dien Hem verraden had, ziende, dat Hij veroordeeld was, berouw gehad, en heeft de dertig zilveren penningen den overpriesters en den ouderlingen wedergebracht,
\par 4 Zeggende: Ik heb gezondigd, verradende het onschuldig bloed! Maar zij zeiden: Wat gaat ons dat aan? Gij moogt toezien.
\par 5 En als hij de zilveren penningen in den tempel geworpen had, vertrok hij, en heengaande verworgde zichzelven.
\par 6 En de overpriesters, de zilveren penningen nemende, zeiden: Het is niet geoorloofd, dezelve in de offerkist te leggen, dewijl het een prijs des bloeds is.
\par 7 En te zamen raad gehouden hebbende, kochten zij daarmede den akker des pottenbakkers, tot een begrafenis voor de vreemdelingen.
\par 8 Daarom is die akker genaamd de akker des bloeds, tot op den huidigen dag.
\par 9 Toen is vervuld geworden, hetgeen gesproken is door den profeet Jeremia, zeggende: En zij hebben de dertig zilveren penningen genomen, de waarde des Gewaardeerden van de kinderen Israels, Denwelken zij gewaardeerd hebben;
\par 10 En hebben dezelve gegeven voor den akker des pottenbakkers; volgens hetgeen mij de Heere bevolen heeft.
\par 11 En Jezus stond voor den stadhouder; en de stadhouder vraagde Hem, zeggende: Zijt Gij de Koning der Joden? En Jezus zeide tot hem: Gij zegt het.
\par 12 En als Hij van de overpriesters en de ouderlingen beschuldigd werd, antwoordde Hij niets.
\par 13 Toen zeide Pilatus tot Hem: Hoort Gij niet, hoevele zaken zij tegen U getuigen?
\par 14 Maar Hij antwoordde hem niet op een enig woord, alzo dat de stadhouder zich zeer verwonderde.
\par 15 En op het feest was de stadhouder gewoon den volke een gevangene los te laten, welken zij wilden.
\par 16 En zij hadden toen een welbekenden gevangene, genaamd Bar-abbas.
\par 17 Als zij dan vergaderd waren, zeide Pilatus tot hen: Welken wilt gij, dat ik u zal loslaten, Bar-abbas, of Jezus, Die genaamd wordt Christus?
\par 18 Want hij wist, dat zij Hem door nijdigheid overgeleverd hadden.
\par 19 En als hij op den rechterstoel zat, zo heeft zijn huisvrouw tot hem gezonden, zeggende: Heb toch niet te doen met dien Rechtvaardige; want ik heb heden veel geleden in den droom om Zijnentwil.
\par 20 Maar de overpriesters en de ouderlingen hebben den scharen aangeraden, dat zij zouden Bar-abbas begeren, en Jezus doden.
\par 21 En de stadhouder, antwoordende, zeide tot hen: Welken van deze twee wilt gij, dat ik u zal loslaten? En zij zeiden: Bar-abbas.
\par 22 Pilatus zeide tot hen: Wat zal ik dan doen met Jezus, Die genaamd wordt Christus? Zij zeiden allen tot hem: Laat Hem gekruisigd worden.
\par 23 Doch de stadhouder zeide: Wat heeft Hij dan kwaads gedaan? En zij riepen te meer, zeggende: Laat Hem gekruisigd worden!
\par 24 Als nu Pilatus zag, dat hij niet vorderde, maar veel meer dat er oproer werd, nam hij water en wies de handen voor de schare, zeggende: Ik ben onschuldig aan het bloed dezes Rechtvaardigen; gijlieden moogt toezien.
\par 25 En al het volk, antwoordende, zeide: Zijn bloed kome over ons, en over onze kinderen.
\par 26 Toen liet hij hun Bar-abbas los, maar Jezus gegeseld hebbende, gaf hij Hem over om gekruisigd te worden.
\par 27 Toen namen de krijgsknechten des stadhouders Jezus met zich in het rechthuis, en vergaderden over Hem de ganse bende.
\par 28 En als zij Hem ontkleed hadden, deden zij Hem een purperen mantel om;
\par 29 En een kroon van doornen gevlochten hebbende, zetten die op Zijn hoofd, en een rietstok in Zijn rechter hand; en vallende op hun knieen voor Hem, bespotten zij Hem, zeggende: Wees gegroet, Gij Koning der Joden!
\par 30 En op Hem gespogen hebbende, namen zij den rietstok en sloegen op Zijn hoofd.
\par 31 En toen zij Hem bespot hadden, deden zij Hem den mantel af, en deden Hem Zijn klederen aan, en leidden Hem heen om te kruisigen.
\par 32 En uitgaande, vonden zij een man van Cyrene, met name Simon; dezen dwongen zij, dat hij Zijn kruis droeg.
\par 33 En gekomen zijnde tot de plaats, genaamd Golgotha, welke is gezegd Hoofdschedelplaats,
\par 34 Gaven zij Hem te drinken edik met gal gemengd; en als Hij dien gesmaakt had, wilde Hij niet drinken.
\par 35 Toen zij nu Hem gekruisigd hadden, verdeelden zij Zijn klederen, het lot werpende; opdat vervuld zou worden, hetgeen gezegd is door den profeet: Zij hebben Mijn klederen onder zich verdeeld, en hebben het lot over Mijn kleding geworpen.
\par 36 En zij, nederzittende, bewaarden Hem aldaar.
\par 37 En zij stelden boven Zijn hoofd Zijn beschuldiging geschreven: DEZE IS JEZUS, DE KONING DER JODEN.
\par 38 Toen werden met Hem twee moordenaars gekruisigd, een ter rechter-,en een ter linker zijde.
\par 39 En die voorbijgingen, lasterden Hem, schuddende hun hoofden.
\par 40 En zeggende: Gij, Die den tempel afbreekt, en in drie dagen opbouwt, verlos Uzelven. Indien Gij de Zone Gods zijt, zo kom af van het kruis.
\par 41 En desgelijks ook de overpriesters met de Schriftgeleerden, en ouderlingen, en Farizeen, Hem bespottende, zeiden:
\par 42 Anderen heeft Hij verlost, Hij kan Zichzelven niet verlossen. Indien Hij de Koning Israels is, dat Hij nu afkome van het kruis, en wij zullen Hem geloven.
\par 43 Hij heeft op God betrouwd; dat Hij Hem nu verlosse, indien Hij Hem wel wil; want Hij heeft gezegd: Ik ben Gods Zoon.
\par 44 En hetzelfde verweten Hem ook de moordenaars, die met Hem gekruisigd waren.
\par 45 En van de zesde ure aan werd er duisternis over de gehele aarde, tot de negende ure toe.
\par 46 En omtrent de negende ure riep Jezus met een grote stem zeggende: ELI, ELI, LAMA SABACHTHANI! dat is: Mijn God! Mijn God! Waarom hebt Gij Mij verlaten!
\par 47 En sommigen van die daar stonden, zulks horende, zeiden: Deze roept Elias.
\par 48 En terstond een van hen toe lopende, nam een spons, en die met edik gevuld hebbende, stak ze op een rietstok, en gaf Hem te drinken.
\par 49 Doch de anderen zeiden: Houd op, laat ons zien, of Elias komt, om Hem te verlossen.
\par 50 En Jezus, wederom met een grote stem roepende, gaf den geest.
\par 51 En ziet, het voorhangsel des tempels scheurde in tweeen, van boven tot beneden; en de aarde beefde, en de steenrotsen scheurden.
\par 52 En de graven werden geopend, en vele lichamen der heiligen, die ontslapen waren, werden opgewekt;
\par 53 En uit de graven uitgegaan zijnde, na Zijn opstanding, kwamen zij in de heilige stad, en zijn velen verschenen.
\par 54 En de hoofdman over honderd, en die met hem Jezus bewaarden, ziende de aardbeving, en de dingen, die geschied waren, werden zeer bevreesd, zeggende: Waarlijk, Deze was Gods Zoon!
\par 55 En aldaar waren vele vrouwen, van verre aanschouwende, die Jezus gevolgd waren van Galilea, om Hem te dienen.
\par 56 Onder dewelke was Maria Magdalena, en Maria, de moeder van Jakobus en Joses, en de moeder der zonen van Zebedeus.
\par 57 En als het avond geworden was, kwam een rijk man van Arimathea, met name Jozef, die ook zelf een discipel van Jezus was.
\par 58 Deze kwam tot Pilatus, en begeerde het lichaam van Jezus. Toen beval Pilatus, dat hem het lichaam gegeven zou worden.
\par 59 En Jozef, het lichaam nemende, wond hetzelve in een zuiver fijn lijnwaad.
\par 60 En leide dat in zijn nieuw graf, hetwelk hij in een steenrots uitgehouwen had; en een groten steen tegen de deur des grafs gewenteld hebbende, ging hij weg.
\par 61 En aldaar was Maria Magdalena, en de andere Maria, zittende tegenover het graf.
\par 62 Des anderen daags nu, welke is na de voorbereiding, vergaderden de overpriesters en de Farizeen tot Pilatus,
\par 63 Zeggende: Heer, wij zijn indachtig, dat deze verleider, nog levende, gezegd heeft: Na drie dagen zal Ik opstaan.
\par 64 Beveel dan, dat het graf verzekerd worde tot den derden dag toe, opdat Zijn discipelen misschien niet komen bij nacht, en stelen Hem, en zeggen tot het volk: Hij is opgestaan van de doden; en zo zal de laatste dwaling erger zijn, dan de eerste.
\par 65 En Pilatus zeide tot henlieden: Gij hebt een wacht; gaat heen, verzekert het, gelijk gij het verstaat.
\par 66 En zij heengaande, verzekerden het graf met de wacht, den steen verzegeld hebbende.

\chapter{28}

\par 1 En laat na de sabbat, als het begon te lichten, tegen den eersten dag der week, kwam Maria Magdalena, en de andere Maria, om het graf te bezien.
\par 2 En ziet, er geschiedde een grote aardbeving; want een engel des Heeren, nederdalende uit den hemel, kwam toe, en wentelde den steen af van de deur, en zat op denzelven.
\par 3 En zijn gedaante was gelijk een bliksem, en zijn kleding wit gelijk sneeuw.
\par 4 En uit vrees van hem zijn de wachters zeer verschrikt geworden, en werden als doden.
\par 5 Maar de engel, antwoordende, zeide tot de vrouwen: Vreest gijlieden niet; want ik weet, dat gij zoekt Jezus, Die gekruisigd was.
\par 6 Hij is hier niet; want Hij is opgestaan, gelijk Hij gezegd heeft. Komt herwaarts, ziet de plaats, waar de Heere gelegen heeft.
\par 7 En gaat haastelijk henen, en zegt Zijn discipelen, dat Hij opgestaan is van de doden; en ziet, Hij gaat u voor naar Galilea, daar zult gij Hem zien. Ziet, ik heb het ulieden gezegd.
\par 8 En haastelijk uitgaande van het graf, met vreze en grote blijdschap, liepen zij henen, om hetzelve Zijn discipelen te boodschappen.
\par 9 En als zij heengingen, om Zijn discipelen te boodschappen, ziet, Jezus is haar ontmoet, zeggende: Weest gegroet! En zij, tot Hem komende, grepen Zijn voeten, en aanbaden Hem.
\par 10 Toen zeide Jezus tot haar: Vreest niet; gaat henen, boodschapt Mijn broederen, dat zij heengaan naar Galilea, en aldaar zullen zij Mij zien.
\par 11 En als zij heengingen, ziet, enigen van de wacht kwamen in de stad, en boodschapten den overpriesters al de dingen, die geschied waren.
\par 12 En zij vergaderd zijnde met de ouderlingen, en te zamen raad genomen hebbende, gaven zij den krijgsknechten veel gelds,
\par 13 En zeiden: Zegt: Zijn discipelen zijn des nachts gekomen, en hebben Hem gestolen, als wij sliepen.
\par 14 En indien zulks komt gehoord te worden van den stadhouder, wij zullen hem tevreden stellen, en maken, dat gij zonder zorg zijt.
\par 15 En zij, het geld genomen hebbende, deden, gelijk zij geleerd waren. En dit woord is verbreid geworden bij de Joden tot op den huidigen dag.
\par 16 En de elf discipelen zijn heengegaan naar Galilea, naar den berg, waar Jezus hen bescheiden had.
\par 17 En als zij Hem zagen, baden zij Hem aan; doch sommigen twijfelden.
\par 18 En Jezus, bij hen komende, sprak tot hen, zeggende: Mij is gegeven alle macht in hemel en op aarde.
\par 19 Gaat dan henen, onderwijst al de volken, dezelve dopende in den Naam des Vaders, en des Zoons, en des Heiligen Geestes; lerende hen onderhouden alles, wat Ik u geboden heb.
\par 20 En ziet, Ik ben met ulieden al de dagen tot de voleinding der wereld. Amen.




\end{document}