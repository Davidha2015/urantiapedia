\begin{document}

\title{Der zweite Brief des Paulus an die Korinther}


\chapter{1}

\par 1 Paulus, ein Apostel Jesu Christi durch den Willen Gottes, und Bruder Timotheus der Gemeinde Gottes zu Korinth samt allen Heiligen in ganz Achaja:
\par 2 Gnade sei mit euch und Friede von Gott, unserm Vater, und dem HERRN Jesus Christus!
\par 3 Gelobet sei Gott und der Vater unsers HERRN Jesu Christi, der Vater der Barmherzigkeit und Gott alles Trostes,
\par 4 der uns tröstet in aller unsrer Trübsal, daß auch wir trösten können, die da sind in allerlei Trübsal, mit dem Trost, damit wir getröstet werden von Gott.
\par 5 Denn gleichwie wir des Leidens Christi viel haben, also werden wir auch reichlich getröstet durch Christum.
\par 6 Wir haben aber Trübsal oder Trost, so geschieht es euch zugute. Ist's Trübsal, so geschieht es euch zu Trost und Heil; welches Heil sich beweist, so ihr leidet mit Geduld, dermaßen, wie wir leiden. Ist's Trost, so geschieht auch das euch zu Trost und Heil;
\par 7 und unsre Hoffnung steht fest für euch, dieweil wir wissen, daß, wie ihr des Leidens teilhaftig seid, so werdet ihr auch des Trostes teilhaftig sein.
\par 8 Denn wir wollen euch nicht verhalten, liebe Brüder, unsre Trübsal, die uns in Asien widerfahren ist, da wir über die Maßen beschwert waren und über Macht, also daß wir auch am Leben verzagten
\par 9 und bei uns beschlossen hatten, wir müßten sterben. Das geschah aber darum, damit wir unser Vertrauen nicht auf uns selbst sollen stellen, sondern auf Gott, der die Toten auferweckt,
\par 10 welcher uns von solchem Tode erlöst hat und noch täglich erlöst; und wir hoffen auf ihn, er werde uns auch hinfort erlösen,
\par 11 durch Hilfe auch eurer Fürbitte für uns, auf daß über uns für die Gabe, die uns gegeben ist, durch viel Personen viel Dank geschehe.
\par 12 Denn unser Ruhm ist dieser: das Zeugnis unsers Gewissens, daß wir in Einfalt und göttlicher Lauterkeit, nicht in fleischlicher Weisheit, sondern in der Gnade Gottes auf der Welt gewandelt haben, allermeist aber bei euch.
\par 13 Denn wir schreiben euch nichts anderes, als was ihr leset und auch befindet. Ich hoffe aber, ihr werdet uns auch bis ans Ende also befinden, gleichwie ihr uns zum Teil befunden habt.
\par 14 Denn wir sind euer Ruhm, gleichwie auch ihr unser Ruhm seid auf des HERRN Jesu Tag.
\par 15 Und auf solch Vertrauen gedachte ich jenes Mal zu euch zu kommen, auf daß ihr abermals eine Wohltat empfinget,
\par 16 und ich durch euch nach Mazedonien reiste und wiederum aus Mazedonien zu euch käme und von euch geleitet würde nach Judäa.
\par 17 Bin ich aber leichtfertig gewesen, da ich solches dachte? Oder sind meine Anschläge fleischlich? Nicht also; sondern bei mir ist Ja Ja, und Nein ist Nein.
\par 18 Aber, o ein treuer Gott, daß unser Wort an euch nicht Ja und Nein gewesen ist.
\par 19 Denn der Sohn Gottes, Jesus Christus, der unter euch durch uns gepredigt ist, durch mich und Silvanus und Timotheus, der war nicht Ja und Nein, sondern es war Ja in ihm.
\par 20 Denn alle Gottesverheißungen sind Ja in ihm und sind Amen in ihm, Gott zu Lobe durch uns.
\par 21 Gott ist's aber, der uns befestigt samt euch in Christum und uns gesalbt
\par 22 und versiegelt und in unsre Herzen das Pfand, den Geist, gegeben hat.
\par 23 Ich rufe aber Gott an zum Zeugen auf meine Seele, daß ich euch verschont habe in dem, daß ich nicht wieder gen Korinth gekommen bin.
\par 24 Nicht daß wir Herren seien über euren Glauben, sondern wir sind Gehilfen eurer Freude; denn ihr stehet im Glauben.

\chapter{2}

\par 1 Ich dachte aber solches bei mir, daß ich nicht abermals in Traurigkeit zu euch käme.
\par 2 Denn, so ich euch traurig mache, wer ist, der mich fröhlich mache, wenn nicht, der da von mir betrübt wird?
\par 3 Und dasselbe habe ich euch geschrieben, daß ich nicht, wenn ich käme, über die traurig sein müßte, über welche ich mich billig soll freuen; sintemal ich mich des zu euch allen versehe, daß meine Freude euer aller Freude sei.
\par 4 Denn ich schrieb euch in großer Trübsal und Angst des Herzens mit viel Tränen; nicht, daß ihr solltet betrübt werden, sondern auf daß ihr die Liebe erkennet, welche ich habe sonderlich zu euch.
\par 5 So aber jemand eine Betrübnis hat angerichtet, der hat nicht mich betrübt, sondern zum Teil, auf daß ich nicht zu viel sage, euch alle.
\par 6 Es ist aber genug, daß derselbe von vielen also gestraft ist,
\par 7 daß ihr nun hinfort ihm desto mehr vergebet und ihn tröstet, auf daß er nicht in allzu große Traurigkeit versinke.
\par 8 Darum ermahne ich euch, daß ihr die Liebe an ihm beweiset.
\par 9 Denn darum habe ich euch auch geschrieben, daß ich erkennte, ob ihr rechtschaffen seid, gehorsam zu sein in allen Stücken.
\par 10 Welchem aber ihr etwas vergebet, dem vergebe ich auch. Denn auch ich, so ich etwas vergebe jemand, das vergebe ich um euretwillen an Christi Statt,
\par 11 auf daß wir nicht übervorteilt werden vom Satan; denn uns ist nicht unbewußt, was er im Sinn hat.
\par 12 Da ich aber gen Troas kam, zu predigen das Evangelium Christi, und mir eine Tür aufgetan war in dem HERRN,
\par 13 hatte ich keine Ruhe in meinem Geist, da ich Titus, meinen Bruder, nicht fand; sondern ich machte meinen Abschied mit ihnen und fuhr aus nach Mazedonien.
\par 14 Aber Gott sei gedankt, der uns allezeit Sieg gibt in Christo und offenbart den Geruch seiner Erkenntnis durch uns an allen Orten!
\par 15 Denn wir sind Gott ein guter Geruch Christi unter denen, die selig werden, und unter denen, die verloren werden:
\par 16 diesen ein Geruch des Todes zum Tode, jenen aber ein Geruch des Lebens zum Leben. Und wer ist hierzu tüchtig?
\par 17 Denn wir sind nicht, wie die vielen, die das Wort Gottes verfälschen; sondern als aus Lauterkeit und als aus Gott reden wir vor Gott in Christo.

\chapter{3}

\par 1 Heben wir denn abermals an, uns selbst zu preisen? Oder bedürfen wir, wie etliche, der Lobebriefe an euch oder Lobebriefe von euch?
\par 2 Ihr seid unser Brief, in unser Herz geschrieben, der erkannt und gelesen wird von allen Menschen;
\par 3 die ihr offenbar geworden seid, daß ihr ein Brief Christi seid, durch unsern Dienst zubereitet, und geschrieben nicht mit Tinte, sondern mit dem Geist des lebendigen Gottes, nicht in steinerne Tafeln, sondern in fleischerne Tafeln des Herzens.
\par 4 Ein solch Vertrauen aber haben wir durch Christum zu Gott.
\par 5 Nicht, daß wir tüchtig sind von uns selber, etwas zu denken als von uns selber; sondern daß wir tüchtig sind, ist von Gott,
\par 6 welcher auch uns tüchtig gemacht hat, das Amt zu führen des Neuen Testaments, nicht des Buchstaben, sondern des Geistes. Denn der Buchstabe tötet, aber der Geist macht lebendig.
\par 7 So aber das Amt, das durch die Buchstaben tötet und in die Steine gebildet war, Klarheit hatte, also daß die Kinder Israel nicht konnten ansehen das Angesicht Mose's um der Klarheit willen seines Angesichtes, die doch aufhört,
\par 8 wie sollte nicht viel mehr das Amt, das den Geist gibt, Klarheit haben!
\par 9 Denn so das Amt, das die Verdammnis predigt, Klarheit hat, wie viel mehr hat das Amt, das die Gerechtigkeit predigt, überschwengliche Klarheit.
\par 10 Denn auch jenes Teil, das verklärt war, ist nicht für Klarheit zu achten gegen die überschwengliche Klarheit.
\par 11 Denn so das Klarheit hatte, das da aufhört, wie viel mehr wird das Klarheit haben, das da bleibt.
\par 12 Dieweil wir nun solche Hoffnung haben, sind wir voll großer Freudigkeit
\par 13 und tun nicht wie Mose, der die Decke vor sein Angesicht hing, daß die Kinder Israel nicht ansehen konnten das Ende des, das aufhört;
\par 14 sondern ihre Sinne sind verstockt. Denn bis auf den heutigen Tag bleibt diese Decke unaufgedeckt über dem alten Testament, wenn sie es lesen, welche in Christo aufhört;
\par 15 aber bis auf den heutigen Tag, wenn Mose gelesen wird, hängt die Decke vor ihrem Herzen.
\par 16 Wenn es aber sich bekehrte zu dem HERRN, so würde die Decke abgetan.
\par 17 Denn der HERR ist der Geist; wo aber der Geist des HERRN ist, da ist Freiheit.
\par 18 Nun aber spiegelt sich in uns allen des HERRN Klarheit mit aufgedecktem Angesicht, und wir werden verklärt in dasselbe Bild von einer Klarheit zu der andern, als vom HERRN, der der Geist ist.

\chapter{4}

\par 1 Darum, dieweil wir ein solch Amt haben, wie uns denn Barmherzigkeit widerfahren ist, so werden wir nicht müde,
\par 2 sondern meiden auch heimliche Schande und gehen nicht mit Schalkheit um, fälschen auch nicht Gottes Wort; sondern mit Offenbarung der Wahrheit beweisen wir uns wohl an aller Menschen Gewissen vor Gott.
\par 3 Ist nun unser Evangelium verdeckt, so ist's in denen, die verloren werden, verdeckt;
\par 4 bei welchen der Gott dieser Welt der Ungläubigen Sinn verblendet hat, daß sie nicht sehen das helle Licht des Evangeliums von der Klarheit Christi, welcher ist das Ebenbild Gottes.
\par 5 Denn wir predigen nicht uns selbst, sondern Jesum Christum, daß er sei der HERR, wir aber eure Knechte um Jesu willen.
\par 6 Denn Gott, der da hieß das Licht aus der Finsternis hervorleuchten, der hat einen hellen Schein in unsere Herzen gegeben, daß durch uns entstünde die Erleuchtung von der Erkenntnis der Klarheit Gottes in dem Angesichte Jesu Christi.
\par 7 Wir haben aber solchen Schatz in irdischen Gefäßen, auf daß die überschwengliche Kraft sei Gottes und nicht von uns.
\par 8 Wir haben allenthalben Trübsal, aber wir ängsten uns nicht; uns ist bange, aber wir verzagen nicht;
\par 9 wir leiden Verfolgung, aber wir werden nicht verlassen; wir werden unterdrückt, aber wir kommen nicht um;
\par 10 und tragen allezeit das Sterben des HERRN Jesu an unserm Leibe, auf daß auch das Leben des HERRN Jesu an unserm Leibe offenbar werde.
\par 11 Denn wir, die wir leben, werden immerdar in den Tod gegeben um Jesu willen, auf das auch das Leben Jesu offenbar werde an unserm sterblichen Fleische.
\par 12 Darum ist nun der Tod mächtig in uns, aber das Leben in euch.
\par 13 Dieweil wir aber denselbigen Geist des Glaubens haben, nach dem, das geschrieben steht: "Ich glaube, darum rede ich", so glauben wir auch, darum so reden wir auch
\par 14 und wissen, daß der, so den HERRN Jesus hat auferweckt, wird uns auch auferwecken durch Jesum und wird uns darstellen samt euch.
\par 15 Denn das geschieht alles um euretwillen, auf daß die überschwengliche Gnade durch vieler Danksagen Gott reichlich preise.
\par 16 Darum werden wir nicht müde; sondern, ob unser äußerlicher Mensch verdirbt, so wird doch der innerliche von Tag zu Tag erneuert.
\par 17 Denn unsre Trübsal, die zeitlich und leicht ist, schafft eine ewige und über alle Maßen wichtige Herrlichkeit
\par 18 uns, die wir nicht sehen auf das Sichtbare, sondern auf das Unsichtbare. Denn was sichtbar ist, das ist zeitlich; was aber unsichtbar ist, das ist ewig.

\chapter{5}

\par 1 Wir wissen aber, so unser irdisch Haus dieser Hütte zerbrochen wird, daß wir einen Bau haben, von Gott erbauet, ein Haus, nicht mit Händen gemacht, das ewig ist, im Himmel.
\par 2 Und darüber sehnen wir uns auch nach unsrer Behausung, die vom Himmel ist, und uns verlangt, daß wir damit überkleidet werden;
\par 3 so doch, wo wir bekleidet und nicht bloß erfunden werden.
\par 4 Denn dieweil wir in der Hütte sind, sehnen wir uns und sind beschwert; sintemal wir wollten lieber nicht entkleidet, sondern überkleidet werden, auf daß das Sterbliche würde verschlungen von dem Leben.
\par 5 Der uns aber dazu bereitet, das ist Gott, der uns das Pfand, den Geist, gegeben hat.
\par 6 So sind wir denn getrost allezeit und wissen, daß, dieweil wir im Leibe wohnen, so wallen wir ferne vom HERRN;
\par 7 denn wir wandeln im Glauben, und nicht im Schauen.
\par 8 Wir sind aber getrost und haben vielmehr Lust, außer dem Leibe zu wallen und daheim zu sein bei dem HERRN.
\par 9 Darum fleißigen wir uns auch, wir sind daheim oder wallen, daß wir ihm wohl gefallen.
\par 10 Denn wir müssen alle offenbar werden vor dem Richtstuhl Christi, auf daß ein jeglicher empfange, nach dem er gehandelt hat bei Leibesleben, es sei gut oder böse.
\par 11 Dieweil wir denn wissen, daß der HERR zu fürchten ist, fahren wir schön mit den Leuten; aber Gott sind wir offenbar. Ich hoffe aber, daß wir auch in eurem Gewissen offenbar sind.
\par 12 Wir loben uns nicht abermals bei euch, sondern geben euch eine Ursache, zu rühmen von uns, auf daß ihr habt zu rühmen wider die, so sich nach dem Ansehen rühmen, und nicht nach dem Herzen.
\par 13 Denn tun wir zu viel, so tun wir's Gott; sind wir mäßig, so sind wir euch mäßig.
\par 14 Denn die Liebe Christi dringt in uns also, sintemal wir halten, daß, so einer für alle gestorben ist, so sind sie alle gestorben;
\par 15 und er ist darum für alle gestorben, auf daß die, so da leben, hinfort nicht sich selbst leben, sondern dem, der für sie gestorben und auferstanden ist.
\par 16 Darum kennen wir von nun an niemand nach dem Fleisch; und ob wir auch Christum gekannt haben nach dem Fleisch, so kennen wir ihn doch jetzt nicht mehr.
\par 17 Darum, ist jemand in Christo, so ist er eine neue Kreatur; das Alte ist vergangen, siehe, es ist alles neu geworden!
\par 18 Aber das alles von Gott, der uns mit ihm selber versöhnt hat durch Jesum Christum und das Amt gegeben, das die Versöhnung predigt.
\par 19 Denn Gott war in Christo und versöhnte die Welt mit ihm selber und rechnete ihnen ihre Sünden nicht zu und hat unter uns aufgerichtet das Wort von der Versöhnung.
\par 20 So sind wir nun Botschafter an Christi Statt, denn Gott vermahnt durch uns; so bitten wir nun an Christi Statt: Lasset euch versöhnen mit Gott.
\par 21 Denn er hat den, der von keiner Sünde wußte, für uns zur Sünde gemacht, auf daß wir würden in ihm die Gerechtigkeit, die vor Gott gilt.

\chapter{6}

\par 1 Wir ermahnen aber euch als Mithelfer, daß ihr nicht vergeblich die Gnade Gottes empfanget.
\par 2 Denn er spricht: "Ich habe dich in der angenehmen Zeit erhört und habe dir am Tage des Heils geholfen." Sehet, jetzt ist die angenehme Zeit, jetzt ist der Tag des Heils!
\par 3 Und wir geben niemand irgend ein Ärgernis, auf daß unser Amt nicht verlästert werde;
\par 4 sondern in allen Dingen beweisen wir uns als die Diener Gottes: in großer Geduld, in Trübsalen, in Nöten, in Ängsten,
\par 5 in Schlägen, in Gefängnissen, in Aufruhren, in Arbeit, in Wachen, in Fasten,
\par 6 in Keuschheit, in Erkenntnis, in Langmut, in Freundlichkeit, in dem heiligen Geist, in ungefärbter Liebe,
\par 7 in dem Wort der Wahrheit, in der Kraft Gottes, durch Waffen der Gerechtigkeit zur Rechten und zur Linken,
\par 8 durch Ehre und Schande, durch böse Gerüchte und gute Gerüchte: als die Verführer, und doch wahrhaftig;
\par 9 als die Unbekannten, und doch bekannt; als die Sterbenden, und siehe, wir leben; als die Gezüchtigten, und doch nicht ertötet;
\par 10 als die Traurigen, aber allezeit fröhlich; als die Armen, aber die doch viele reich machen; als die nichts innehaben, und doch alles haben.
\par 11 O ihr Korinther! unser Mund hat sich zu euch aufgetan, unser Herz ist weit.
\par 12 Ihr habt nicht engen Raum in uns; aber eng ist's in euren Herzen.
\par 13 Ich rede mit euch als mit meinen Kindern, daß ihr euch auch also gegen mich stellet und werdet auch weit.
\par 14 Ziehet nicht am fremden Joch mit den Ungläubigen. Denn was hat die Gerechtigkeit zu schaffen mit der Ungerechtigkeit? Was hat das Licht für Gemeinschaft mit der Finsternis?
\par 15 Wie stimmt Christus mit Belial? Oder was für ein Teil hat der Gläubige mit dem Ungläubigen?
\par 16 Was hat der Tempel Gottes für Gleichheit mit den Götzen? Ihr aber seid der Tempel des lebendigen Gottes; wie denn Gott spricht: "Ich will unter ihnen wohnen und unter ihnen wandeln und will ihr Gott sein, und sie sollen mein Volk sein.
\par 17 Darum gehet aus von ihnen und sondert euch ab, spricht der HERR, und rührt kein Unreines an, so will ich euch annehmen
\par 18 und euer Vater sein, und ihr sollt meine Söhne und Töchter sein, spricht der allmächtige HERR."

\chapter{7}

\par 1 Dieweil wir nun solche Verheißungen haben, meine Liebsten, so lasset uns von aller Befleckung des Fleisches und des Geistes uns reinigen und fortfahren mit der Heiligung in der Furcht Gottes.
\par 2 Fasset uns: Wir haben niemand Leid getan, wir haben niemand verletzt, wir haben niemand übervorteilt.
\par 3 Nicht sage ich solches, euch zu verdammen; denn ich habe droben zuvor gesagt, daß ihr in unsern Herzen seid, mitzusterben und mitzuleben.
\par 4 Ich rede mit großer Freudigkeit zu euch; ich rühme viel von euch; ich bin erfüllt mit Trost; ich bin überschwenglich in Freuden und in aller unsrer Trübsal.
\par 5 Denn da wir nach Mazedonien kamen, hatte unser Fleisch keine Ruhe; sondern allenthalben waren wir in Trübsal: auswendig Streit, inwendig Furcht.
\par 6 Aber Gott, der die Geringen tröstet, der tröstete auch uns durch die Ankunft des Titus;
\par 7 nicht allein aber durch seine Ankunft, sondern auch durch den Trost, mit dem er getröstet war an euch, da er uns verkündigte euer Verlangen, euer Weinen, euren Eifer um mich, also daß ich mich noch mehr freute.
\par 8 Denn daß ich euch durch den Brief habe traurig gemacht, reut mich nicht. Und ob's mich reute, dieweil ich sehe, daß der Brief vielleicht eine Weile euch betrübt hat,
\par 9 so freue ich mich doch nun, nicht darüber, daß ihr seid betrübt worden, sondern daß ihr betrübt seid worden zur Reue. Denn ihr seid göttlich betrübt worden, daß ihr von uns ja keinen Schaden irgendworin nehmet.
\par 10 Denn göttliche Traurigkeit wirkt zur Seligkeit einen Reue, die niemand gereut; die Traurigkeit aber der Welt wirkt den Tod.
\par 11 Siehe, daß ihr göttlich seid betrübt worden, welchen Fleiß hat das in euch gewirkt, dazu Verantwortung, Zorn, Furcht, Verlangen, Eifer, Rache! Ihr habt euch bewiesen in allen Stücken, daß ihr rein seid in der Sache.
\par 12 Darum, ob ich euch geschrieben habe, so ist's doch nicht geschehen um des willen, der beleidigt hat, auch nicht um des willen, der beleidigt ist, sondern um deswillen, daß euer Fleiß gegen uns offenbar sein würde bei euch vor Gott.
\par 13 Derhalben sind wir getröstet worden, daß ihr getröstet seid. Überschwenglicher aber haben wir uns noch gefreut über die Freude des Titus; denn sein Geist ist erquickt an euch allen.
\par 14 Denn was ich vor ihm von euch gerühmt habe, darin bin ich nicht zu Schanden geworden; sondern, gleichwie alles wahr ist, was ich von euch geredet habe, also ist auch unser Rühmen vor Titus wahr geworden.
\par 15 Und er ist überaus herzlich wohl gegen euch gesinnt, wenn er gedenkt an euer aller Gehorsam, wie ihr ihn mit Furcht und Zittern habt aufgenommen.
\par 16 Ich freue mich, daß ich mich zu euch alles Guten versehen darf.

\chapter{8}

\par 1 Ich tue euch kund, liebe Brüder, die Gnade Gottes, die in den Gemeinden in Mazedonien gegeben ist.
\par 2 Denn ihre Freude war überschwenglich, da sie durch viel Trübsal bewährt wurden; und wiewohl sie sehr arm sind, haben sie doch reichlich gegeben in aller Einfalt.
\par 3 Denn nach allem Vermögen (das bezeuge ich) und über Vermögen waren sie willig
\par 4 und baten uns mit vielem Zureden, daß wir aufnähmen die Wohltat und Gemeinschaft der Handreichung, die da geschieht den Heiligen;
\par 5 und nicht, wie wir hofften, sondern sie ergaben sich selbst, zuerst dem HERRN und darnach uns, durch den Willen Gottes,
\par 6 daß wir mußten Titus ermahnen, auf daß er, wie er zuvor angefangen hatte, also auch unter euch solche Wohltat ausrichtete.
\par 7 Aber gleichwie ihr in allen Stücken reich seid, im Glauben und im Wort und in der Erkenntnis und in allerlei Fleiß und in eurer Liebe zu uns, also schaffet, daß ihr auch in dieser Wohltat reich seid.
\par 8 Nicht sage ich, daß ich etwas gebiete; sondern, dieweil andere so fleißig sind, versuche ich auch eure Liebe, ob sie rechter Art sei.
\par 9 Denn ihr wisset die Gnade unsers HERRN Jesu Christi, daß, ob er wohl reich ist, ward er doch arm um euretwillen, auf daß ihr durch seine Armut reich würdet.
\par 10 Und meine Meinung hierin gebe ich; denn solches ist euch nützlich, die ihr angefangen habt vom vorigen Jahre her nicht allein das Tun, sondern auch das Wollen;
\par 11 nun aber vollbringet auch das Tun, auf daß, gleichwie da ist ein geneigtes Gemüt, zu wollen, so sei auch da ein geneigtes Gemüt, zu tun von dem, was ihr habt.
\par 12 Denn so einer willig ist, so ist er angenehm, nach dem er hat, nicht nach dem er nicht hat.
\par 13 Nicht geschieht das in der Meinung, daß die andern Ruhe haben, und ihr Trübsal, sondern daß es gleich sei.
\par 14 So diene euer Überfluß ihrem Mangel diese teure Zeit lang, auf daß auch ihr Überfluß hernach diene eurem Mangel und ein Ausgleich geschehe;
\par 15 wie geschrieben steht: "Der viel sammelte, hatte nicht Überfluß, der wenig sammelte, hatte nicht Mangel."
\par 16 Gott aber sei Dank, der solchen Eifer für euch gegeben hat in das Herz des Titus.
\par 17 Denn er nahm zwar die Ermahnung an; aber dieweil er fleißig war, ist er von selber zu euch gereist.
\par 18 Wir haben aber einen Bruder mit ihm gesandt, der das Lob hat am Evangelium durch alle Gemeinden.
\par 19 Nicht allein aber das, sondern er ist auch verordnet von den Gemeinden zum Gefährten unsrer Fahrt in dieser Wohltat, welche durch uns ausgerichtet wird dem HERRN zu Ehren und zum Preis eures guten Willens.
\par 20 Also verhüten wir, daß uns nicht jemand übel nachreden möge solcher reichen Steuer halben, die durch uns ausgerichtet wird;
\par 21 und sehen darauf, daß es redlich zugehe, nicht allein vor dem HERRN sondern auch vor den Menschen.
\par 22 Auch haben wir mit ihnen gesandt unsern Bruder, den wir oft erfunden haben in vielen Stücken, daß er fleißig sei, nun aber viel fleißiger.
\par 23 Und wir sind großer Zuversicht zu euch, es sei des Titus halben, welcher mein Geselle und Gehilfe unter euch ist, oder unsrer Brüder halben, welche Boten sind der Gemeinden und eine Ehre Christi.
\par 24 Erzeiget nun die Beweisung eurer Liebe und unsers Rühmens von euch an diesen auch öffentlich vor den Gemeinden!

\chapter{9}

\par 1 Denn von solcher Steuer, die den Heiligen geschieht, ist mir nicht not, euch zu schreiben.
\par 2 Denn ich weiß euren guten Willen, davon ich rühme bei denen aus Mazedonien und sage: Achaja ist schon voriges Jahr bereit gewesen; und euer Beispiel hat viele gereizt.
\par 3 Ich habe aber diese Brüder darum gesandt, daß nicht unser Rühmen von euch zunichte würde in dem Stücke, und daß ihr bereit seid, gleichwie ich von euch gesagt habe;
\par 4 auf daß nicht, so die aus Mazedonien mit mir kämen und euch unbereit fänden, wir (will nicht sagen: ihr) zu Schanden würden mit solchem Rühmen.
\par 5 So habe ich es nun für nötig angesehen, die Brüder zu ermahnen, daß sie voranzögen zu euch, fertigzumachen diesen zuvor verheißenen Segen, daß er bereit sei, also daß es sei ein Segen und nicht ein Geiz.
\par 6 Ich meine aber das: Wer da kärglich sät, der wird auch kärglich ernten; und wer da sät im Segen, der wird auch ernten im Segen.
\par 7 Ein jeglicher nach seiner Willkür, nicht mit Unwillen oder aus Zwang; denn einen fröhlichen Geber hat Gott lieb.
\par 8 Gott aber kann machen, daß allerlei Gnade unter euch reichlich sei, daß ihr in allen Dingen volle Genüge habt und reich seid zu allerlei guten Werken;
\par 9 wie geschrieben steht: "Er hat ausgestreut und gegeben den Armen; seine Gerechtigkeit bleibt in Ewigkeit."
\par 10 Der aber Samen reicht dem Säemann, der wird auch das Brot reichen zur Speise und wird vermehren euren Samen und wachsen lassen das Gewächs eurer Gerechtigkeit,
\par 11 daß ihr reich seid in allen Dingen mit aller Einfalt, welche wirkt durch uns Danksagung Gott.
\par 12 Denn die Handreichung dieser Steuer erfüllt nicht allein den Mangel der Heiligen, sondern ist auch überschwenglich darin, daß viele Gott danken für diesen unsern treuen Dienst
\par 13 und preisen Gott über euer untertäniges Bekenntnis des Evangeliums Christi und über eure einfältige Steuer an sie und an alle,
\par 14 indem auch sie nach euch verlangt im Gebet für euch um der überschwenglichen Gnade Gottes willen in euch.
\par 15 Gott aber sei Dank für seine unaussprechliche Gabe!

\chapter{10}

\par 1 Ich aber, Paulus, ermahne euch durch die Sanftmütigkeit und Lindigkeit Christi, der ich gegenwärtig unter euch gering bin, abwesend aber dreist gegen euch.
\par 2 Ich bitte aber, daß mir nicht not sei, gegenwärtig dreist zu handeln und der Kühnheit zu brauchen, die man mir zumißt, gegen etliche, die uns schätzen, als wandelten wir fleischlicherweise.
\par 3 Denn ob wir wohl im Fleisch wandeln, so streiten wir doch nicht fleischlicherweise.
\par 4 Denn die Waffen unsrer Ritterschaft sind nicht fleischlich, sondern mächtig vor Gott, zu zerstören Befestigungen;
\par 5 wir zerstören damit die Anschläge und alle Höhe, die sich erhebt wider die Erkenntnis Gottes, und nehmen gefangen alle Vernunft unter den Gehorsam Christi
\par 6 und sind bereit, zu rächen allen Ungehorsam, wenn euer Gehorsam erfüllt ist.
\par 7 Richtet ihr nach dem Ansehen? Verläßt sich jemand darauf, daß er Christo angehöre, der denke solches auch wiederum bei sich, daß, gleichwie er Christo angehöre, also auch wir Christo angehören.
\par 8 Und so ich auch etwas weiter mich rühmte von unsrer Gewalt, welche uns der HERR gegeben hat, euch zu bessern, und nicht zu verderben, wollte ich nicht zu Schanden werden.
\par 9 Das sage ich aber, daß ihr nicht euch dünken lasset, als hätte ich euch wollen schrecken mit Briefen.
\par 10 Denn die Briefe, sprechen sie, sind schwer und stark; aber die Gegenwart des Leibes ist schwach und die Rede verächtlich.
\par 11 Wer ein solcher ist, der denke, daß, wie wir sind mit Worten in den Briefen abwesend, so werden wir auch wohl sein mit der Tat gegenwärtig.
\par 12 Denn wir wagen uns nicht unter die zu rechnen oder zu zählen, so sich selbst loben, aber dieweil sie an sich selbst messen und halten allein von sich selbst, verstehen sie nichts.
\par 13 Wir aber rühmen uns nicht über das Ziel hinaus, sondern nur nach dem Ziel der Regel, mit der uns Gott abgemessen hat das Ziel, zu gelangen auch bis zu euch.
\par 14 Denn wir fahren nicht zu weit, als wären wir nicht gelangt zu euch; denn wir sind ja auch zu euch gekommen mit dem Evangelium Christi;
\par 15 und rühmen uns nicht übers Ziel hinaus in fremder Arbeit und haben Hoffnung, wenn nun euer Glaube in euch wächst, daß wir in unsrer Regel nach wollen weiterkommen
\par 16 und das Evangelium auch predigen denen, die jenseit von euch wohnen, und uns nicht rühmen in dem, was mit fremder Regel bereitet ist.
\par 17 Wer sich aber rühmt, der rühme sich des HERRN.
\par 18 Denn darum ist einer nicht tüchtig, daß er sich selbst lobt, sondern daß ihn der HERR lobt.

\chapter{11}

\par 1 Wollte Gott, ihr hieltet mir ein wenig Torheit zugut! doch ihr haltet mir's wohl zugut.
\par 2 Denn ich eifere um euch mit göttlichem Eifer; denn ich habe euch vertraut einem Manne, daß ich eine reine Jungfrau Christo zubrächte.
\par 3 Ich fürchte aber, daß, wie die Schlange Eva verführte mit ihrer Schalkheit, also auch eure Sinne verrückt werden von der Einfalt in Christo.
\par 4 Denn so, der da zu euch kommt, einen andern Jesus predigte, den wir nicht gepredigt haben, oder ihr einen andern Geist empfinget, den ihr nicht empfangen habt, oder ein ander Evangelium, das ihr nicht angenommen habt, so vertrüget ihr's billig.
\par 5 Denn ich achte, ich sei nicht weniger, als die "hohen" Apostel sind.
\par 6 Und ob ich nicht kundig bin der Rede, so bin ich doch nicht unkundig der Erkenntnis. Doch ich bin bei euch allenthalben wohl bekannt.
\par 7 Oder habe ich gesündigt, daß ich mich erniedrigt habe, auf daß ihr erhöht würdet? Denn ich habe euch das Evangelium Gottes umsonst verkündigt
\par 8 und habe andere Gemeinden beraubt und Sold von ihnen genommen, daß ich euch predigte.
\par 9 Und da ich bei euch war gegenwärtig und Mangel hatte, war ich niemand beschwerlich. Denn mein Mangel erstatteten die Brüder, die aus Mazedonien kamen, so habe ich mich in allen Stücken euch unbeschwerlich gehalten und will auch noch mich also halten.
\par 10 So gewiß die Wahrheit Christi in mir ist, so soll mir dieser Ruhm in den Ländern Achajas nicht verstopft werden.
\par 11 Warum das? Daß ich euch nicht sollte liebhaben? Gott weiß es.
\par 12 Was ich aber tue und tun will, das tue ich darum, daß ich die Ursache abschneide denen, die Ursache suchen, daß sie rühmen möchten, sie seien wie wir.
\par 13 Denn solche falsche Apostel und trügliche Arbeiter verstellen sich zu Christi Aposteln.
\par 14 Und das ist auch kein Wunder; denn er selbst, der Satan, verstellt sich zum Engel des Lichtes.
\par 15 Darum ist es auch nicht ein Großes, wenn sich seine Diener verstellen als Prediger der Gerechtigkeit; welcher Ende sein wird nach ihren Werken.
\par 16 Ich sage abermals, daß nicht jemand wähne, ich sei töricht; wo aber nicht, so nehmet mich als einen Törichten, daß ich mich auch ein wenig rühme.
\par 17 Was ich jetzt rede, das rede ich nicht als im HERRN, sondern als in der Torheit, dieweil wir in das Rühmen gekommen sind.
\par 18 Sintemal viele sich rühmen nach dem Fleisch, will ich mich auch rühmen.
\par 19 Denn ihr vertraget gern die Narren, dieweil ihr klug seid.
\par 20 Ihr vertraget, so euch jemand zu Knechten macht, so euch jemand schindet, so euch jemand gefangennimmt, so jemand euch trotzt, so euch jemand ins Angesicht streicht.
\par 21 Das sage ich nach der Unehre, als wären wir schwach geworden. Worauf aber jemand kühn ist (ich rede in Torheit!), darauf bin ich auch kühn.
\par 22 Sie sind Hebräer? Ich auch! Sie sind Israeliter? Ich auch! Sie sind Abrahams Same? Ich auch!
\par 23 Sie sind Diener Christi? Ich rede töricht: Ich bin's wohl mehr: Ich habe mehr gearbeitet, ich habe mehr Schläge erlitten, bin öfter gefangen, oft in Todesnöten gewesen;
\par 24 von den Juden habe ich fünfmal empfangen vierzig Streiche weniger eins;
\par 25 ich bin dreimal gestäupt, einmal gesteinigt, dreimal Schiffbruch erlitten, Tag und Nacht habe ich zugebracht in der Tiefe des Meers;
\par 26 ich bin oft gereist, ich bin in Gefahr gewesen durch die Flüsse, in Gefahr durch die Mörder, in Gefahr unter den Juden, in Gefahr unter den Heiden, in Gefahr in den Städten, in Gefahr in der Wüste, in Gefahr auf dem Meer, in Gefahr unter den falschen Brüdern;
\par 27 in Mühe und Arbeit, in viel Wachen, in Hunger und Durst, in viel Fasten, in Frost und Blöße;
\par 28 außer was sich sonst zuträgt, nämlich, daß ich täglich werde angelaufen und trage Sorge für alle Gemeinden.
\par 29 Wer ist schwach, und ich werde nicht schwach? Wer wird geärgert, und ich brenne nicht?
\par 30 So ich mich ja rühmen soll, will ich mich meiner Schwachheit rühmen.
\par 31 Gott und der Vater unsers HERRN Jesu Christi, welcher sei gelobt in Ewigkeit, weiß, daß ich nicht lüge.
\par 32 Zu Damaskus verwahrte der Landpfleger des Königs Aretas die Stadt der Damasker und wollte mich greifen,
\par 33 und ich ward in einem Korbe zum Fenster hinaus durch die Mauer niedergelassen und entrann aus seinen Händen.

\chapter{12}

\par 1 Es ist mir ja das Rühmen nichts nütze; doch will ich kommen auf die Gesichte und Offenbarung des HERRN.
\par 2 Ich kenne einen Menschen in Christo; vor vierzehn Jahren (ist er in dem Leibe gewesen, so weiß ich's nicht; oder ist er außer dem Leibe gewesen, so weiß ich's nicht; Gott weiß es) ward derselbe entzückt bis in den dritten Himmel.
\par 3 Und ich kenne denselben Menschen (ob er im Leibe oder außer dem Leibe gewesen ist, weiß ich nicht; Gott weiß es);
\par 4 der ward entzückt in das Paradies und hörte unaussprechliche Worte, welche kein Mensch sagen kann.
\par 5 Für denselben will ich mich rühmen; für mich selbst aber will ich mich nichts rühmen, nur meiner Schwachheit.
\par 6 Und so ich mich rühmen wollte, täte ich daran nicht töricht; denn ich wollte die Wahrheit sagen. Ich enthalte mich aber dessen, auf daß nicht jemand mich höher achte, als er an mir sieht oder von mir hört.
\par 7 Und auf daß ich mich nicht der hohen Offenbarung überhebe, ist mir gegeben ein Pfahl ins Fleisch, nämlich des Satans Engel, der mich mit Fäusten schlage, auf daß ich mich nicht überhebe.
\par 8 Dafür ich dreimal zum HERRN gefleht habe, daß er von mir wiche.
\par 9 Und er hat zu mir gesagt: Laß dir an meiner Gnade genügen; denn meine Kraft ist in den Schwachen mächtig. Darum will ich mich am allerliebsten rühmen meiner Schwachheit, auf daß die Kraft Christi bei mir wohne.
\par 10 Darum bin ich gutes Muts in Schwachheiten, in Mißhandlungen, in Nöten, in Verfolgungen, in Ängsten, um Christi willen; denn, wenn ich schwach bin, so bin ich stark.
\par 11 Ich bin ein Narr geworden über dem Rühmen; dazu habt ihr mich gezwungen. Denn ich sollte von euch gelobt werden, sintemal ich nichts weniger bin, als die "hohen" Apostel sind, wiewohl ich nichts bin.
\par 12 Denn es sind ja eines Apostels Zeichen unter euch geschehen mit aller Geduld, mit Zeichen und mit Wundern und mit Taten.
\par 13 Was ist's, darin ihr geringer seid denn die andern Gemeinden, außer daß ich selbst euch nicht habe beschwert? Vergebet mir diese Sünde!
\par 14 Siehe, ich bin bereit zum drittenmal zu euch zu kommen, und will euch nicht beschweren; denn ich suche nicht das Eure, sondern euch. Denn es sollen nicht die Kinder den Eltern Schätze sammeln, sondern die Eltern den Kindern.
\par 15 Ich aber will sehr gern hingeben und hingegeben werden für eure Seelen; wiewohl ich euch gar sehr liebe, und doch weniger geliebt werde.
\par 16 Aber laß es also sein, daß ich euch nicht habe beschwert; sondern, die weil ich tückisch bin, habe ich euch mit Hinterlist gefangen.
\par 17 Habe ich aber etwa jemand übervorteilt durch derer einen, die ich zu euch gesandt habe?
\par 18 Ich habe Titus ermahnt und mit ihm gesandt einen Bruder. Hat euch etwa Titus übervorteilt? Haben wir nicht in einem Geist gewandelt? Sind wir nicht in einerlei Fußtapfen gegangen?
\par 19 Lasset ihr euch abermals dünken, wir verantworten uns vor euch? Wir reden in Christo vor Gott; aber das alles geschieht, meine Liebsten, euch zur Besserung.
\par 20 Denn ich fürchte, wenn ich komme, daß ich euch nicht finde, wie ich will, und ihr mich auch nicht findet, wie ihr wollt; daß Hader, Neid, Zorn, Zank, Afterreden, Ohrenblasen, Aufblähen, Aufruhr dasei;
\par 21 daß mich, wenn ich abermals komme, mein Gott demütige bei euch und ich müsse Leid tragen über viele, die zuvor gesündigt und nicht Buße getan haben für die Unreinigkeit und Hurerei und Unzucht, die sie getrieben haben.

\chapter{13}

\par 1 Komme ich zum drittenmal zu euch, so soll in zweier oder dreier Zeugen Mund bestehen allerlei Sache.
\par 2 Ich habe es euch zuvor gesagt und sage es euch zuvor, wie, als ich zum andernmal gegenwärtig war, so auch nun abwesend schreibe ich es denen, die zuvor gesündigt haben, und den andern allen: Wenn ich abermals komme, so will ich nicht schonen;
\par 3 sintemal ihr suchet, daß ihr einmal gewahr werdet des, der in mir redet, nämlich Christi, welcher unter euch nicht schwach ist, sondern ist mächtig unter euch.
\par 4 Und ob er wohl gekreuzigt ist in der Schwachheit, so lebt er doch in der Kraft Gottes. Und ob wir auch schwach sind in ihm, so leben wir doch mit ihm in der Kraft Gottes unter euch.
\par 5 Versuchet euch selbst, ob ihr im Glauben seid; prüfet euch selbst! Oder erkennet ihr euch selbst nicht, daß Jesus Christus in euch ist? Es sei denn, daß ihr untüchtig seid.
\par 6 Ich hoffe aber, ihr erkennet, daß wir nicht untüchtig sind.
\par 7 Ich bitte aber Gott, daß ihr nichts Übles tut; nicht, auf daß wir als tüchtig angesehen werden, sondern auf daß ihr das Gute tut und wir wie die Untüchtigen seien.
\par 8 Denn wir können nichts wider die Wahrheit, sondern für die Wahrheit.
\par 9 Wir freuen uns aber, wenn wir schwach sind, und ihr mächtig seid. Und dasselbe wünschen wir auch, nämlich eure Vollkommenheit.
\par 10 Derhalben schreibe ich auch solches abwesend, auf daß ich nicht, wenn ich gegenwärtig bin, Schärfe brauchen müsse nach der Macht, welche mir der HERR, zu bessern und nicht zu verderben, gegeben hat.
\par 11 Zuletzt, liebe Brüder, freuet euch, seid vollkommen, tröstet euch, habt einerlei Sinn, seid friedsam! so wird der Gott der Liebe und des Friedens mit euch sein.
\par 12 Grüßet euch untereinander mit dem heiligen Kuß.
\par 13 Es grüßen euch alle Heiligen.
\par 14 Die Gnade unsers HERRN Jesu Christi und die Liebe Gottes und die Gemeinschaft des heiligen Geistes sei mit euch allen! Amen.

\end{document}