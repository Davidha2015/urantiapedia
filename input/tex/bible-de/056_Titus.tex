\begin{document}

\title{Titus}


\chapter{1}

\par 1 Paulus, ein Knecht Gottes und ein Apostel Jesu Christi, nach dem Glauben der Auserwählten Gottes und der Erkenntnis der Wahrheit zur Gottseligkeit,
\par 2 auf Hoffnung des ewigen Lebens, welches verheißen hat, der nicht lügt, Gott, vor den Zeiten der Welt,
\par 3 aber zu seiner Zeit hat er offenbart sein Wort durch die Predigt, die mir vertrauet ist nach dem Befehl Gottes, unsers Heilandes,
\par 4 dem Titus, meinem rechtschaffenen Sohn nach unser beider Glauben: Gnade, Barmherzigkeit, Friede von Gott, dem Vater, und dem HERRN Jesus Christus, unserm Heiland!
\par 5 Derhalben ließ ich dich in Kreta, daß du solltest vollends ausrichten, was ich gelassen habe, und besetzen die Städte hin und her mit Ältesten, wie ich dir befohlen haben;
\par 6 wo einer ist untadelig, eines Weibes Mann, der gläubige Kinder habe, nicht berüchtigt, daß sie Schwelger und ungehorsam sind.
\par 7 Denn ein Bischof soll untadelig sein als ein Haushalter Gottes, nicht eigensinnig, nicht zornig, nicht ein Weinsäufer, nicht raufen, nicht unehrliche Hantierung treiben;
\par 8 sondern gastfrei, gütig, züchtig, gerecht, heilig, keusch,
\par 9 und haltend ob dem Wort, das gewiß ist, und lehrhaft, auf daß er mächtig sei, zu ermahnen durch die heilsame Lehre und zu strafen die Widersprecher.
\par 10 Denn es sind viel freche und unnütze Schwätzer und Verführer, sonderlich die aus den Juden,
\par 11 welchen man muß das Maul stopfen, die da ganze Häuser verkehren und lehren, was nicht taugt, um schändlichen Gewinns willen.
\par 12 Es hat einer aus ihnen gesagt, ihr eigener Prophet: "Die Kreter sind immer Lügner, böse Tiere und faule Bäuche."
\par 13 Dies Zeugnis ist wahr. Um der Sache willen strafe sie scharf, auf daß sie gesund seien im Glauben
\par 14 und nicht achten auf die jüdischen Fabeln und Gebote von Menschen, welche sich von der Wahrheit abwenden.
\par 15 Den Reinen ist alles rein; den Unreinen aber und Ungläubigen ist nichts rein, sondern unrein ist ihr Sinn sowohl als ihr Gewissen.
\par 16 Sie sagen, sie erkennen Gott; aber mit den Werken verleugnen sie es, sintemal sie es sind, an welchen Gott Greuel hat, und gehorchen nicht und sind zu allem guten Werk untüchtig.

\chapter{2}

\par 1 Du aber rede, wie sich's ziemt nach der heilsamen Lehre:
\par 2 den Alten sage, daß sie nüchtern seien, ehrbar, züchtig, gesund im Glauben, in der Liebe, in der Geduld;
\par 3 den alten Weibern desgleichen, daß sie sich halten wie den Heiligen ziemt, nicht Lästerinnen seien, nicht Weinsäuferinnen, gute Lehrerinnen;
\par 4 daß sie die jungen Weiber lehren züchtig sein, ihre Männer lieben, Kinder lieben,
\par 5 sittig sein, keusch, häuslich, gütig, ihren Männern untertan, auf daß nicht das Wort Gottes verlästert werde.
\par 6 Desgleichen die jungen Männer ermahne, daß sie züchtig seien.
\par 7 Allenthalben aber stelle dich selbst zum Vorbilde guter Werke, mit unverfälschter Lehre, mit Ehrbarkeit,
\par 8 mit heilsamem und untadeligem Wort, auf daß der Widersacher sich schäme und nichts habe, daß er von uns möge Böses sagen.
\par 9 Den Knechten sage, daß sie ihren Herren untertänig seien, in allen Dingen zu Gefallen tun, nicht widerbellen,
\par 10 nicht veruntreuen, sondern alle gute Treue erzeigen, auf daß sie die Lehre Gottes, unsers Heilandes, zieren in allen Stücken.
\par 11 Denn es ist erschienen die heilsame Gnade Gottes allen Menschen
\par 12 und züchtigt uns, daß wir sollen verleugnen das ungöttliche Wesen und die weltlichen Lüste, und züchtig, gerecht und gottselig leben in dieser Welt
\par 13 und warten auf die selige Hoffnung und Erscheinung der Herrlichkeit des großen Gottes und unsers Heilandes, Jesu Christi,
\par 14 der sich selbst für uns gegeben hat, auf daß er uns erlöste von aller Ungerechtigkeit und reinigte sich selbst ein Volk zum Eigentum, das fleißig wäre zu guten Werken.
\par 15 Solches rede und ermahne und strafe mit gutem Ernst. Laß dich niemand verachten.

\chapter{3}

\par 1 Erinnere sie, daß sie den Fürsten und der Obrigkeit untertan und gehorsam seien, zu allem guten Werk bereit seien,
\par 2 niemand lästern, nicht hadern, gelinde seien, alle Sanftmütigkeit beweisen gegen alle Menschen.
\par 3 Denn wir waren weiland auch unweise, ungehorsam, verirrt, dienend den Begierden und mancherlei Wollüsten, und wandelten in Bosheit und Neid, waren verhaßt und haßten uns untereinander.
\par 4 Da aber erschien die Freundlichkeit und Leutseligkeit Gottes, unsers Heilandes,
\par 5 nicht um der Werke willen der Gerechtigkeit, die wir getan hatten, sondern nach seiner Barmherzigkeit machte er uns selig durch das Bad der Wiedergeburt und Erneuerung des heiligen Geistes,
\par 6 welchen er ausgegossen hat über uns reichlich durch Jesum Christum, unsern Heiland,
\par 7 auf daß wir durch desselben Gnade gerecht und Erben seien des ewigen Lebens nach der Hoffnung.
\par 8 Das ist gewißlich wahr; solches will ich, daß du fest lehrest, auf daß die, so an Gott gläubig geworden sind, in einem Stand guter Werke gefunden werden. Solches ist gut und nütze den Menschen.
\par 9 Der törichten Fragen aber, der Geschlechtsregister, des Zankes und Streites über das Gesetz entschlage dich; denn sie sind unnütz und eitel.
\par 10 Einen ketzerischen Menschen meide, wenn er einmal und abermals ermahnt ist,
\par 11 und wisse, daß ein solcher verkehrt ist und sündigt, als der sich selbst verurteilt hat.
\par 12 Wenn ich zu dir senden werde Artemas oder Tychikus, so komm eilend zu mir gen Nikopolis; denn daselbst habe ich beschlossen den Winter zu bleiben.
\par 13 Zenas, den Schriftgelehrten, und Apollos fertige ab mit Fleiß, auf daß ihnen nichts gebreche.
\par 14 Laß aber auch die Unsern lernen, daß sie im Stand guter Werke sich finden lassen, wo man ihrer bedarf, auf daß sie nicht unfruchtbar seien.
\par 15 Es grüßen dich alle, die mit mir sind. Grüße alle, die uns lieben im Glauben. Die Gnade sei mit euch allen! Amen.

\end{document}