\begin{document}

\title{Genesis}


\chapter{1}

\par 1 Εν αρχή εποίησεν ο Θεός τον ουρανόν και την γην.
\par 2 Η δε γη ήτο άμορφος και έρημος· και σκότος επί του προσώπου της αβύσσου. Και πνεύμα Θεού εφέρετο επί της επιφανείας των υδάτων.
\par 3 Και είπεν ο Θεός, Γενηθήτω φώς· και έγεινε φώς·
\par 4 και είδεν ο Θεός το φως ότι ήτο καλόν· και διεχώρισεν ο Θεός το φως από του σκότους·
\par 5 και εκάλεσεν ο Θεός το φως, Ημέραν· το δε σκότος εκάλεσε, Νύκτα. Και έγεινεν εσπέρα και έγεινε πρωΐ, ημέρα πρώτη.
\par 6 Και είπεν ο Θεός, Γενηθήτω στερέωμα αναμέσον των υδάτων, και ας διαχωρίζη ύδατα από υδάτων.
\par 7 Και εποίησεν ο Θεός το στερέωμα, και διεχώρισε τα ύδατα τα υποκάτωθεν του στερεώματος από των υδάτων των επάνωθεν του στερεώματος. Και έγεινεν ούτω.
\par 8 Και εκάλεσεν ο Θεός το στερέωμα, Ουρανόν. Και έγεινεν εσπέρα και έγεινε πρωΐ, ημέρα δευτέρα.
\par 9 Και είπεν ο Θεός, Ας συναχθώσι τα ύδατα τα υποκάτω του ουρανού εις τόπον ένα, και ας φανή η ξηρά. Και έγεινεν ούτω.
\par 10 Και εκάλεσεν ο Θεός την ξηράν, γήν· και το σύναγμα των υδάτων εκάλεσε, Θαλάσσας· και είδεν ο Θεός ότι ήτο καλόν.
\par 11 Και είπεν ο Θεός, Ας βλαστήση η γη χλωρόν χόρτον, χόρτον κάμνοντα σπόρον, και δένδρον κάρπιμον κάμνον καρπόν κατά το είδος αυτού, του οποίου το σπέρμα να ήναι εν αυτώ επί της γης. Και έγεινεν ούτω.
\par 12 Και εβλάστησεν η γη χλωρόν χόρτον, χόρτον κάμνοντα σπόρον κατά το είδος αυτού, και δένδρον κάμνον καρπόν, του οποίου το σπέρμα είναι εν αυτώ κατά το είδος αυτού· και είδεν ο Θεός ότι ήτο καλόν.
\par 13 Και έγεινεν εσπέρα και έγεινε πρωΐ, ημέρα τρίτη.
\par 14 Και είπεν ο Θεός, Ας γείνωσι φωστήρες εν τω στερεώματι του ουρανού, διά να διαχωρίζωσι την ημέραν από της νυκτός· και ας ήναι διά σημεία και καιρούς και ημέρας και ενιαυτούς·
\par 15 και ας ήναι διά φωστήρας εν τω στερεώματι του ουρανού, διά να φέγγωσιν επί της γης. Και έγεινεν ούτω.
\par 16 Και έκαμεν ο Θεός τους δύο φωστήρας τους μεγάλους, τον φωστήρα τον μέγαν διά να εξουσιάζη επί της ημέρας, και τον φωστήρα τον μικρόν διά να εξουσιάζη επί της νυκτός· και τους αστέρας·
\par 17 και έθεσεν αυτούς ο Θεός εν τω στερεώματι του ουρανού, διά να φέγγωσιν επί της γης,
\par 18 και να εξουσιάζωσιν επί της ημέρας και επί της νυκτός και να διαχωρίζωσι το φως από του σκότους. Και είδεν ο Θεός ότι ήτο καλόν.
\par 19 Και έγεινεν εσπέρα και έγεινε πρωΐ, ημέρα τετάρτη.
\par 20 Και είπεν ο Θεός, Ας γεννήσωσι τα ύδατα εν αφθονία νηκτά έμψυχα και πετεινά ας πέτωνται επάνωθεν της γης κατά το στερέωμα του ουρανού.
\par 21 Και εποίησεν ο Θεός τα κήτη τα μεγάλα και παν έμψυχον κινούμενον, τα οποία εγέννησαν εν αφθονία τα ύδατα κατά το είδος αυτών, και παν πετεινόν πτερωτόν κατά το είδος αυτού. Και είδεν ο Θεός ότι ήτο καλόν.
\par 22 Και ευλόγησεν αυτά ο Θεός, λέγων, Αυξάνεσθε και πληθύνεσθε και γεμίσατε τα ύδατα εν ταις θαλάσσαις· και τα πετεινά ας πληθύνωνται επί της γης.
\par 23 Και έγεινεν εσπέρα και έγεινε πρωΐ, ημέρα πέμπτη.
\par 24 Και είπεν ο Θεός, Ας γεννήση η γη ζώα έμψυχα κατά το είδος αυτών, κτήνη και ερπετά και ζώα της γης κατά το είδος αυτών· και έγεινεν ούτω.
\par 25 Και έκαμεν ο Θεός τα ζώα της γης κατά το είδος αυτών, και τα κτήνη κατά το είδος αυτών, και παν ερπετόν της γης κατά το είδος αυτού. Και είδεν ο Θεός ότι ήτο καλόν.
\par 26 Και είπεν ο θεός, Ας κάμωμεν άνθρωπον κατ' εικόνα ημών, καθ' ομοίωσιν ημών· και ας εξουσιάζη επί των ιχθύων της θαλάσσης και επί των πετεινών του ουρανού και επί των κτηνών και επί πάσης της γης και επί παντός ερπετού, έρποντος επί της γης.
\par 27 Και εποίησεν ο Θεός τον άνθρωπον κατ' εικόνα εαυτού· κατ' εικόνα Θεού εποίησεν αυτόν· άρσεν και θήλυ εποίησεν αυτούς·
\par 28 και ευλόγησεν αυτούς ο Θεός· και είπε προς αυτούς ο Θεός, Αυξάνεσθε και πληθύνεσθε και γεμίσατε την γην και κυριεύσατε αυτήν, και εξουσιάζετε επί των ιχθύων της θαλάσσης και επί των πετεινών του ουρανού και επί παντός ζώου κινουμένου επί της γης.
\par 29 Και είπεν ο Θεός, Ιδού, σας έδωκα πάντα χόρτον κάμνοντα σπόρον, όστις είναι επί του προσώπου πάσης της γης, και παν δένδρον, το οποίον έχει εν εαυτώ καρπόν δένδρου κάμνοντος σπόρον· ταύτα θέλουσιν είσθαι εις εσάς προς τροφήν·
\par 30 και εις πάντα τα ζώα της γης και εις πάντα τα πετεινά του ουρανού και εις παν ερπετόν έρπον επί της γης και έχον εν εαυτώ ψυχήν ζώσαν, έδωκα πάντα χλωρόν χόρτον εις τροφήν. Και έγεινεν ούτω.
\par 31 Και είδεν ο Θεός πάντα όσα εποίησε· και ιδού, ήσαν καλά λίαν. Και έγεινεν εσπέρα και έγεινε πρωΐ, ημέρα έκτη.

\chapter{2}

\par 1 Και συνετελέσθησαν ο ουρανός και η γη και πάσα η στρατιά αυτών.
\par 2 Και είχε συντετελεσμένα ο Θεός εν τη ημέρα τη εβδόμη τα έργα αυτού, τα οποία έκαμε· και ανεπαύθη την ημέραν την εβδόμην από πάντων των έργων αυτού, τα οποία έκαμε.
\par 3 Και ευλόγησεν ο Θεός την ημέραν την εβδόμην και ηγίασεν αυτήν· διότι εν αυτή ανεπαύθη από πάντων των έργων αυτού, τα οποία έκτισε και έκαμεν ο Θεός.
\par 4 Αύτη είναι η γένεσις του ουρανού και της γης, ότε εκτίσθησαν αυτά, καθ' ην ημέραν εποίησε Κύριος ο Θεός γην και ουρανόν,
\par 5 και πάντα τα φυτά του αγρού, πριν γείνωσιν επί της γης, και πάντα χόρτον του αγρού, πριν βλαστήση· διότι δεν είχε βρέξει Κύριος ο Θεός επί της γης, και άνθρωπος δεν ήτο διά να εργάζηται την γήν·
\par 6 ο ατμός δε ανέβαινεν από της γης και επότιζε παν το πρόσωπον της γης.
\par 7 Και έπλασε Κύριος ο Θεός τον άνθρωπον από χώματος εκ της γης. και ενεφύσησεν εις τους μυκτήρας αυτού πνοήν ζωής, και έγεινεν ο άνθρωπος εις ψυχήν ζώσαν.
\par 8 Και εφύτευσε Κύριος ο Θεός παράδεισον εν τη Εδέμ κατά ανατολάς και έθεσεν εκεί τον άνθρωπον, τον οποίον έπλασε.
\par 9 Και Κύριος ο Θεός έκαμε να βλαστήση εκ της γης παν δένδρον ώραίον εις την όρασιν και καλόν εις την γεύσιν· και το ξύλον της ζωής εν μέσω του παραδείσου και το ξύλον της γνώσεως του καλού και του κακού.
\par 10 Ποταμός δε εξήρχετο εκ της Εδέμ διά να ποτίζη τον παράδεισον· και εκείθεν εμερίζετο εις τέσσαρας κλάδους.
\par 11 Το όνομα του ενός, Φισών· ούτος είναι ο περικυκλόνων πάσαν την γην Αβιλά· όπου ευρίσκεται το χρυσίον·
\par 12 το δε χρυσίον της γης εκείνης είναι καλόν· εκεί είναι το βδέλλιον και ο λίθος ο ονυχίτης.
\par 13 Και το όνομα του ποταμού του δευτέρου, Γιών· ούτος είναι ο περικυκλόνων πάσαν την γην Χούς.
\par 14 Και το όνομα του ποταμού του τρίτου, Τίγρις· ούτος είναι ο ρέων προς ανατολάς της Ασσυρίας. Ο δε ποταμός ο τέταρτος, ούτος είναι ο Ευφράτης.
\par 15 Και έλαβε Κύριος ο Θεός τον άνθρωπον και έθεσεν αυτόν εν τω παραδείσω της Εδέμ διά να εργάζηται αυτόν και να φυλάττη αυτόν.
\par 16 Προσέταξε δε Κύριος ο Θεός εις τον Αδάμ λέγων, Από παντός δένδρου του παραδείσου ελευθέρως θέλεις τρώγει,
\par 17 από δε του ξύλου της γνώσεως του καλού και του κακού δεν θέλεις φάγει απ' αυτού· διότι καθ' ην ημέραν φάγης απ' αυτού, θέλεις εξάπαντος αποθάνει.
\par 18 Και είπε Κύριος ο Θεός, Δεν είναι καλόν να ήναι ο άνθρωπος μόνος· θέλω κάμει εις αυτόν βοηθόν όμοιον με αυτόν.
\par 19 Έπλασε δε Κύριος ο Θεός εκ της γης πάντα τα ζώα του αγρού και πάντα τα πετεινά του ουρανού, και έφερεν αυτά προς τον Αδάμ, διά να ίδη πως να ονομάση αυτά· και ό,τι όνομα ήθελε δώσει ο Αδάμ εις παν έμψυχον, τούτο να ήναι το όνομα αυτού.
\par 20 Και έδωκεν ο Αδάμ ονόματα εις πάντα τα κτήνη και εις τα πτηνά του ουρανού και εις πάντα τα ζώα του αγρού· εις δε τον Αδάμ δεν ευρίσκετο βοηθός όμοιος με αυτόν.
\par 21 Και επέβαλε Κύριος ο Θεός έκστασιν επί τον Αδάμ, και εκοιμήθη· και έλαβε μίαν εκ των πλευρών αυτού και έκλεισε με σάρκα τον τόπον αυτής.
\par 22 Και κατεσκεύασε Κύριος ο Θεός την πλευράν, την οποίαν έλαβεν από του Αδάμ, εις γυναίκα και έφερεν αυτήν προς τον Αδάμ.
\par 23 Και είπεν ο Αδάμ, Τούτο είναι τώρα οστούν εκ των οστέων μου και σαρξ εκ της σαρκός μου· αύτη θέλει ονομασθή ανδρίς, διότι εκ του ανδρός αύτη ελήφθη.
\par 24 Διά τούτο θέλει αφήσει ο άνθρωπος τον πατέρα αυτού και την μητέρα αυτού, και θέλει προσκολληθή εις την γυναίκα αυτού· και θέλουσιν είσθαι οι δύο εις σάρκα μίαν.
\par 25 Ήσαν δε και οι δύο γυμνοί, ο Αδάμ και η γυνή αυτού, και δεν ησχύνοντο.

\chapter{3}

\par 1 Ο δε όφις ήτο το φρονιμώτερον πάντων των ζώων του αγρού, τα οποία έκαμε Κύριος ο Θεός· και είπεν ο όφις προς την γυναίκα, Τω όντι είπεν ο Θεός, Μη φάγητε από παντός δένδρου του παραδείσου;
\par 2 Και είπεν η γυνή προς τον όφιν, Από του καρπού των δένδρων του παραδείσου δυνάμεθα να φάγωμεν·
\par 3 από δε του καρπού του δένδρου, το οποίον είναι εν μέσω του παραδείσου, είπεν ο Θεός, Μη φάγητε απ' αυτού, μηδέ εγγίσητε αυτόν, διά να μη αποθάνητε.
\par 4 Και είπεν ο όφις προς την γυναίκα, Δεν θέλετε βεβαίως αποθάνει
\par 5 αλλ' εξεύρει ο Θεός, ότι καθ' ην ημέραν φάγητε απ' αυτού, θέλουσιν ανοιχθή οι οφθαλμοί σας, και θέλετε είσθαι ως θεοί, γνωρίζοντες το καλόν και το κακόν.
\par 6 Και είδεν η γυνή, ότι το δένδρον ήτο καλόν εις βρώσιν, και ότι ήτο αρεστόν εις τους οφθαλμούς, και επιθυμητόν το δένδρον ως δίδον γνώσιν· και λαβούσα εκ του καρπού αυτού, έφαγε· και έδωκε και εις τον άνδρα αυτής μεθ' εαυτής, και αυτός έφαγε.
\par 7 Και ηνοίχθησαν οι οφθαλμοί αμφοτέρων, και εγνώρισαν ότι ήσαν γυμνοί· και ράψαντες φύλλα συκής, έκαμον εις εαυτούς περιζώματα.
\par 8 Και ήκουσαν την φωνήν Κυρίου του Θεού, περιπατούντος εν τω παραδείσω προς το δειλινόν· και εκρύφθησαν ο Αδάμ και η γυνή αυτού από προσώπου Κυρίου του Θεού, μεταξύ των δένδρων του παραδείσου.
\par 9 Εκάλεσε δε Κύριος ο Θεός τον Αδάμ, και είπε προς αυτόν, Που είσαι;
\par 10 Ο δε είπε, Την φωνήν σου ήκουσα εν τω παραδείσω, και εφοβήθην, διότι είμαι γυμνός· και εκρύφθην.
\par 11 Και είπε προς αυτόν ο Θεός, Τις εφανέρωσεν εις σε ότι είσαι γυμνός; Μήπως έφαγες από του δένδρου, από του οποίου προσέταξα εις σε να μη φάγης;
\par 12 Και είπεν ο Αδάμ, Η γυνή την οποίαν έδωκας να ήναι μετ' εμού, αυτή μοι έδωκεν από του δένδρου, και έφαγον.
\par 13 Και είπε Κύριος ο Θεός προς την γυναίκα, Τι είναι τούτο το οποίον έκαμες; Και η γυνή είπεν, Ο όφις με ηπάτησε, και έφαγον.
\par 14 Και είπε Κύριος ο Θεός προς τον όφιν, Επειδή έκαμες τούτο, επικατάρατος να ήσαι μεταξύ πάντων των κτηνών, και πάντων των ζώων του αγρού· επί της κοιλίας σου θέλεις περιπατεί, και χώμα θέλεις τρώγει, πάσας τας ημέρας της ζωής σου·
\par 15 και έχθραν θέλω στήσει αναμέσον σου και της γυναικός, και αναμέσον του σπέρματός σου και του σπέρματος αυτής· αυτό θέλει σου συντρίψει την κεφαλήν, και συ θέλεις κεντήσει την πτέρναν αυτού.
\par 16 Προς δε την γυναίκα είπε, Θέλω υπερπληθύνει τας λύπας σου και τους πόνους της κυοφορίας σου· με λύπας θέλεις γεννά τέκνα· και προς τον άνδρα σου θέλει είσθαι η επιθυμία σου, και αυτός θέλει σε εξουσιάζει.
\par 17 Προς δε τον Αδάμ είπεν, Επειδή υπήκουσας εις τον λόγον της γυναικός σου, και έφαγες από του δένδρου, από του οποίου προσέταξα εις σε λέγων, Μη φάγης απ' αυτού, κατηραμένη να ήναι η γη εξ αιτίας σου· με λύπας θέλεις τρώγει τους καρπούς αυτής πάσας τας ημέρας της ζωής σου·
\par 18 και ακάνθας και τριβόλους θέλει βλαστάνει εις σέ· και θέλεις τρώγει τον χόρτον του αγρού·
\par 19 εν τω ιδρώτι του προσώπου σου θέλεις τρώγει τον άρτον σου, εωσού επιστρέψης εις την γην, εκ της οποίας ελήφθης· επειδή γη είσαι, και εις γην θέλεις επιστρέψει.
\par 20 Και εκάλεσεν ο Αδάμ το όνομα της γυναικός αυτού, Εύαν· διότι αυτή ήτο μήτηρ πάντων των ζώντων.
\par 21 Και έκαμε Κύριος ο Θεός εις τον Αδάμ και εις την γυναίκα αυτού χιτώνας δερματίνους, και ενέδυσεν αυτούς.
\par 22 Και είπε Κύριος ο Θεός, Ιδού, έγεινεν ο Αδάμ ως εις εξ ημών, εις το γινώσκειν το καλόν και το κακόν· και τώρα μήπως εκτείνη την χείρα αυτού, και λάβη και από του ξύλου της ζωής, και φάγη, και ζήση αιωνίως·
\par 23 Όθεν Κύριος ο Θεός εξαπέστειλεν αυτόν εκ του παραδείσου της Εδέμ, διά να εργάζηται την γην εκ της οποίας ελήφθη.
\par 24 Και εξεδίωξε τον Αδάμ· και κατά ανατολάς του παραδείσου της Εδέμ έθεσε τα Χερουβείμ, και την ρομφαίαν την φλογίνην, την περιστρεφομένην, διά να φυλάττωσι την οδόν του ξύλου της ζωής.

\chapter{4}

\par 1 Ο δε Αδάμ εγνώρισεν Εύαν την γυναίκα αυτού· και συνέλαβε, και εγέννησε τον Κάϊν· και είπεν, Απέκτησα άνθρωπον διά του Κυρίου.
\par 2 Και προσέτι εγέννησε τον αδελφόν αυτού τον Άβελ. Και ήτο ο Άβελ ποιμήν προβάτων, ο δε Κάϊν ήτο γεωργός.
\par 3 Και μεθ' ημέρας προσέφερεν ο Κάϊν από των καρπών της γης προσφοράν προς τον Κύριον.
\par 4 Και ο Άβελ προσέφερε και αυτός από των πρωτοτόκων των προβάτων αυτού, και από των στεάτων αυτών. Και επέβλεψε με ευμένειαν Κύριος επί τον Άβελ και επί την προσφοράν αυτού·
\par 5 επί δε τον Κάϊν και επί την προσφοράν αυτού δεν επέβλεψε. Και ηγανάκτησεν ο Κάϊν σφόδρα, και εκατηφίασε το πρόσωπον αυτού
\par 6 Και είπε Κύριος προς τον Κάϊν, Διά τι ηγανάκτησας; και διά τι εκατηφίασε το πρόσωπόν σου;
\par 7 αν συ πράττης καλώς, δεν θέλεις είσθαι ευπρόσδεκτος; και εάν δεν πράττης καλώς, εις την θύραν κείται η αμαρτία. Αλλ' εις σε θέλει είσθαι η επιθυμία αυτού, και συ θέλεις εξουσιάζει επ' αυτού.
\par 8 Και είπεν ο Κάϊν προς Άβελ τον αδελφόν αυτού, Ας υπάγωμεν εις την πεδιάδα· και ενώ ήσαν εν τη πεδιάδι, σηκωθείς ο Κάϊν κατά του αδελφού αυτού Άβελ εφόνευσεν αυτόν.
\par 9 Και είπε Κύριος προς τον Κάϊν, Που είναι Άβελ ο αδελφός σου; Ο δε είπε, Δεν εξεύρω· μη φύλαξ του αδελφού μου είμαι εγώ;
\par 10 Και είπεν ο Θεός, Τι έκαμες; η φωνή του αίματος του αδελφού σου βοά προς εμέ εκ της γής·
\par 11 και τώρα επικατάρατος να ήσαι από της γης, ήτις ήνοιξε το στόμα αυτής διά να δεχθή το αίμα του αδελφού σου εκ της χειρός σου·
\par 12 όταν εργάζησαι την γην, δεν θέλει εις το εξής σοι δώσει τον καρπόν αυτής· πλανήτης και φυγάς θέλεις είσθαι επί της γης.
\par 13 Και είπεν ο Κάϊν προς τον Κύριον, Η αμαρτία μου είναι μεγαλητέρα παρ' ώστε να συγχωρηθή·
\par 14 ιδού, με διώκεις σήμερον από προσώπου της γης, και από του προσώπου σου θέλω κρυφθή, και θέλω είσθαι πλανήτης και φυγάς επί της γής· και πας όστις με εύρη, θέλει με φονεύσει.
\par 15 Είπε δε προς αυτόν ο Κύριος, διά τούτο, πας όστις φονεύση τον Κάϊν, επταπλασίως θέλει τιμωρηθή. Και έβαλεν ο Κύριος σημείον εις τον Κάϊν, διά να μη φονεύση αυτόν πας όστις εύρη αυτόν.
\par 16 Και εξήλθεν ο Κάϊν από προσώπου του Κυρίου, και κατώκησεν εν τη γη Νωδ, προς ανατολάς της Εδέμ.
\par 17 Εγνώρισε δε ο Κάϊν την γυναίκα αυτού, και συνέλαβε, και εγέννησε τον Ενώχ· έκτισε δε πόλιν, και εκάλεσε το όνομα της πόλεως κατά το όνομα του υιού αυτού, Ενώχ.
\par 18 Εγεννήθη δε εις τον Ενώχ ο Ιράδ· και Ιράδ εγέννησε τον Μεχουϊαήλ· και Μεχουϊαήλ εγέννησε τον Μεθουσαήλ· και Μεθουσαήλ εγέννησε τον Λάμεχ.
\par 19 Και έλαβεν εις εαυτόν ο Λάμεχ δύο γυναίκας· το όνομα της μιας, Αδά, και το όνομα της άλλης, Σιλλά.
\par 20 Και εγέννησεν η Αδά τον Ιαβάλ· ούτος ήτο πατήρ των κατοικούντων εν σκηναίς και τρεφόντων κτήνη.
\par 21 Και το όνομα του αδελφού αυτού ήτο Ιουβάλ· ούτος ήτο πατήρ πάντων των παιζόντων κιθάραν και αυλόν.
\par 22 Η Σιλλά δε και αυτή εγέννησε τον Θουβάλ-κάϊν, χαλκέα παντός εργαλείου χαλκού και σιδήρου· αδελφή δε του Θουβάλ-κάϊν ήτο η Νααμά.
\par 23 Και είπεν ο Λάμεχ προς τας γυναίκας εαυτού, Αδά και Σιλλά, Ακούσατε την φωνήν μου· γυναίκες του Λάμεχ, ακροασθήτε τους λόγους μου· επειδή άνδρα εφόνευσα εις πληγήν μου· και νέον εις μάστιγά μου·
\par 24 διότι ο μεν Κάϊν επταπλασίως θέλει εκδικηθή· ο δε Λάμεχ εβδομηκοντάκις επτά.
\par 25 Εγνώρισε δε πάλιν ο Αδάμ την γυναίκα αυτού, και εγέννησεν υιόν, και εκάλεσε το όνομα αυτού Σηθ, λέγουσα, Ότι έδωκεν εις εμέ ο Θεός άλλο σπέρμα αντί του Άβελ, τον οποίον εφόνευσεν ο Κάϊν.
\par 26 Και εις τον Σηθ ομοίως εγεννήθη υιός· και εκάλεσε το όνομα αυτού Ενώς. Τότε έγεινεν αρχή να ονομάζωνται με το όνομα του Κυρίου.

\chapter{5}

\par 1 Τούτο είναι το βιβλίον της γενεαλογίας του ανθρώπου. Καθ' ην ημέραν εποίησεν ο Θεός τον Αδάμ, κατ' εικόνα Θεού εποίησεν αυτόν.
\par 2 Άρσεν και θήλυ εποίησεν αυτούς· και ευλόγησεν αυτούς, και εκάλεσε το όνομα αυτών, Αδάμ, καθ' ην ημέραν εποίησεν αυτούς.
\par 3 Έζησε δε ο Αδάμ εκατόν τριάκοντα έτη, και εγέννησεν υιόν κατά την ομοίωσιν αυτού, κατά την εικόνα αυτού, και εκάλεσε το όνομα αυτού Σήθ·
\par 4 και έγειναν αι ημέραι του Αδάμ, αφού εγέννησε τον Σηθ, οκτακόσια έτη· και εγέννησεν υιούς και θυγατέρας·
\par 5 και έγειναν πάσαι αι ημέραι του Αδάμ, τας οποίας έζησεν, εννεακόσια τριάκοντα έτη· και απέθανε.
\par 6 Και έζησεν ο Σηθ εκατόν πέντε έτη, και εγέννησε τον Ενώς·
\par 7 και έζησεν ο Σηθ αφού εγέννησε τον Ενώς, οκτακόσια επτά έτη, και εγέννησεν υιούς και θυγατέρας·
\par 8 έγειναν δε πάσαι αι ημέραι του Σηθ εννεακόσια δώδεκα έτη· και απέθανε.
\par 9 Και έζησεν ο Ενώς ενενήκοντα έτη, και εγέννησε τον Καϊνάν·
\par 10 έζησε δε ο Ενώς, αφού εγέννησε τον Καϊνάν, οκτακόσια δεκαπέντε έτη, και εγέννησεν υιούς και θυγατέρας·
\par 11 και έγειναν πάσαι αι ημέραι του Ενώς εννεακόσια πέντε έτη· και απέθανε.
\par 12 Και έζησεν ο Καϊνάν εβδομήκοντα έτη, και εγέννησε τον Μααλαλεήλ·
\par 13 έζησε δε ο Καϊνάν, αφού εγέννησε τον Μααλαλεήλ, οκτακόσια τεσσαράκοντα έτη, και εγέννησεν υιούς και θυγατέρας·
\par 14 και έγειναν πάσαι αι ημέραι του Καϊνάν εννεακόσια δέκα έτη· και απέθανε.
\par 15 Και έζησεν ο Μααλαλεήλ εξήκοντα πέντε έτη, και εγέννησε τον Ιάρεδ·
\par 16 έζησε δε ο Μααλαλεήλ, αφού εγέννησε τον Ιάρεδ, οκτακόσια τριάκοντα έτη, και εγέννησεν υιούς και θυγατέρας·
\par 17 και έγειναν πάσαι αι ημέραι του Μααλαλεήλ οκτακόσια ενενήκοντα πέντε έτη· και απέθανε.
\par 18 Και έζησεν ο Ιάρεδ εκατόν εξήκοντα δύο έτη, και εγέννησε τον Ενώχ·
\par 19 έζησε δε ο Ιάρεδ, αφού εγέννησε τον Ενώχ, οκτακόσια έτη, και εγέννησεν υιούς και θυγατέρας·
\par 20 και έγειναν πάσαι αι ημέραι του Ιάρεδ εννεακόσια εξήκοντα δύο έτη· και απέθανε.
\par 21 Και έζησεν ο Ενώχ εξήκοντα πέντε έτη, και εγέννησε τον Μαθουσάλα·
\par 22 και περιεπάτησεν ο Ενώχ μετά του Θεού, αφού εγέννησε τον Μαθουσάλα, τριακόσια έτη, και εγέννησεν υιούς και θυγατέρας·
\par 23 και έγειναν πάσαι αι ημέραι του Ενώχ τριακόσια εξήκοντα πέντε έτη.
\par 24 Και περιεπάτησεν ο Ενώχ μετά του Θεού, και δεν ευρίσκετο πλέον· διότι μετέθεσεν αυτόν ο Θεός.
\par 25 Και έζησεν ο Μαθουσάλα εκατόν ογδοήκοντα επτά έτη, και εγέννησε τον Λάμεχ·
\par 26 έζησε δε ο Μαθουσάλα, αφού εγέννησε τον Λάμεχ, επτακόσια ογδοήκοντα δύο έτη, και εγέννησεν υιούς και θυγατέρας·
\par 27 και έγειναν πάσαι αι ημέραι του Μαθουσάλα εννεακόσια εξήκοντα εννέα έτη· και απέθανε.
\par 28 Έζησε δε ο Λάμεχ εκατόν ογδοήκοντα δύο έτη, και εγέννησεν υιόν·
\par 29 και εκάλεσε το όνομα αυτού Νώε, λέγων, Ούτος θέλει ανακουφίσει ημάς από του έργου ημών, και από του μόχθου των χειρών ημών, εξ αιτίας της γης την οποίαν κατηράσθη ο Κύριος.
\par 30 Έζησε δε ο Λάμεχ, αφού εγέννησε τον Νώε, πεντακόσια ενενήκοντα πέντε έτη, και εγέννησεν υιούς και θυγατέρας·
\par 31 και έγειναν πάσαι αι ημέραι του Λάμεχ επτακόσια εβδομήκοντα επτά έτη· και απέθανε.
\par 32 Και ο Νώε ήτο ηλικίας πεντακοσίων ετών· και εγέννησεν ο Νώε τον Σημ, τον Χαμ, και τον Ιάφεθ.

\chapter{6}

\par 1 Και ότε ήρχισαν οι άνθρωποι να πληθύνωνται επί του προσώπου της γης, και θυγατέρες εγεννήθησαν εις αυτούς,
\par 2 ιδόντες οι υιοί του Θεού τας θυγατέρας των ανθρώπων, ότι ήσαν ώραίαι, έλαβον εις εαυτούς γυναίκας εκ πασών όσας έκλεξαν.
\par 3 Και είπε Κύριος, Δεν θέλει καταμείνει πάντοτε το πνεύμά μου μετά του ανθρώπου, διότι είναι σάρξ· αι ημέραι αυτού θέλουσιν είσθαι ακόμη εκατόν είκοσι έτη.
\par 4 Κατ' εκείνας τας ημέρας ήσαν οι γίγαντες επί της γης, και έτι, ύστερον, αφού οι υιοί του Θεού εισήλθον εις τας θυγατέρας των ανθρώπων, και αύται ετεκνοποίησαν εις αυτούς· εκείνοι ήσαν οι δυνατοί, οι έκπαλαι άνδρες ονομαστοί.
\par 5 Και είδεν ο Κύριος ότι επληθύνετο η κακία του ανθρώπου επί της γης, και πάντες οι σκοποί των διαλογισμών της καρδίας αυτού ήσαν μόνον κακία πάσας τας ημέρας.
\par 6 Και μετεμελήθη ο Κύριος ότι εποίησε τον άνθρωπον επί της γης. και ελυπήθη εν τη καρδία αυτού.
\par 7 Και είπεν ο Κύριος, Θέλω εξαλείψει τον άνθρωπον, τον οποίον εποίησα, από προσώπου της γής· από ανθρώπου έως κτήνους, έως ερπετού, και έως πτηνού του ουρανού· επειδή μετεμελήθην ότι εποίησα αυτούς.
\par 8 Ο δε Νώε εύρε χάριν ενώπιον Κυρίου.
\par 9 Αύτη είναι η γενεαλογία του Νώε. Ο Νώε ήτο άνθρωπος δίκαιος, τέλειος μεταξύ των συγχρόνων αυτού· μετά του Θεού περιεπάτησεν ο Νώε.
\par 10 Και εγέννησεν ο Νώε τρεις υιούς, τον Σημ, τον Χαμ και τον Ιάφεθ.
\par 11 Διεφθάρη δε η γη ενώπιον του Θεού, και ενεπλήσθη η γη αδικίας.
\par 12 Και είδεν ο Θεός την γην, και ιδού, ήτο διεφθαρμένη· διότι πάσα σαρξ είχε διαφθείρει την οδόν αυτής επί της γης.
\par 13 Και είπεν ο Θεός προς τον Νώε, Το τέλος πάσης σαρκός ήλθεν ενώπιόν μου, διότι η γη ενεπλήσθη αδικίας απ' αυτών· και ιδού, θέλω εξολοθρεύσει αυτούς και την γην.
\par 14 Κάμε εις σεαυτόν κιβωτόν εκ ξύλων Γόφερ· κατά δωμάτια θέλεις κάμει την κιβωτόν, και θέλεις αλείψει αυτήν έσωθεν και έξωθεν με πίσσαν.
\par 15 Και ούτω θέλεις κάμει αυτήν· το μεν μήκος της κιβωτού θέλει είσθαι τριακοσίων πηχών, το δε πλάτος αυτής πεντήκοντα πηχών, και το ύψος αυτής τριάκοντα πηχών.
\par 16 Στέγην θέλεις κάμει εις την κιβωτόν, και εις πήχην θέλεις τελειώσει αυτήν άνωθεν· και την θύραν της κιβωτού θέλεις βάλει εκ πλαγίων· κατώγαια, διώροφα, και τριώροφα θέλεις κάμει αυτήν.
\par 17 Εγώ δε, ιδού, εγώ επιφέρω τον κατακλυσμόν των υδάτων επί της γης, διά να εξολοθρεύσω πάσαν σάρκα, έχουσαν εν εαυτή πνεύμα ζωής υποκάτω του ουρανού· παν ό,τι είναι επί της γης, θέλει αποθάνει.
\par 18 Και θέλω στήσει την διαθήκην μου προς σέ· και θέλεις εισέλθει εις την κιβωτόν, συ, και οι υιοί σου, και η γυνή σου, και αι γυναίκες των υιών σου μετά σου.
\par 19 Και από παντός ζώου εκ πάσης σαρκός, ανά δύο εκ πάντων θέλεις εισάξει εις την κιβωτόν, διά να φυλάξης την ζωήν αυτών μετά σεαυτού· άρσεν και θήλυ θέλουσιν είσθαι.
\par 20 Από των πτηνών κατά το είδος αυτών, και από των κτηνών κατά το είδος αυτών, από πάντων των ερπετών της γης κατά το είδος αυτών, ανά δύο εκ πάντων θέλουσιν εισέλθει προς σε, διά να φυλάξης την ζωήν αυτών.
\par 21 Και συ λάβε εις σεαυτόν από παντός φαγητού το οποίον τρώγεται, και θέλεις συνάξει αυτό πλησίον σου· και θέλει είσθαι εις σε, και εις αυτά, προς τροφήν.
\par 22 Και έκαμεν ο Νώε κατά πάντα όσα προσέταξεν εις αυτόν ο Θεός· ούτως έκαμε.

\chapter{7}

\par 1 Και είπε Κύριος προς τον Νώε, Είσελθε συ, και πας ο οίκός σου, εις την κιβωτόν· διότι σε είδον δίκαιον ενώπιόν μου εν τη γενεά ταύτη·
\par 2 από πάντων των κτηνών των καθαρών λάβε εις σεαυτόν επτά επτά, άρσεν και το θήλυ αυτού· και από των κτηνών των μη καθαρών ανά δύο, άρσεν και το θήλυ αυτού·
\par 3 και από των πτηνών του ουρανού επτά επτά, άρσεν και θήλυ· διά να διατηρήσης σπέρμα επί προσώπου πάσης της γής·
\par 4 επειδή έτι μετά επτά ημέρας εγώ φέρω βροχήν επί της γης τεσσαράκοντα ημέρας και τεσσαράκοντα νύκτας· και θέλω εξαλείψει από προσώπου της γης παν ό,τι υπάρχει, το οποίον εποίησα.
\par 5 Και έκαμεν ο Νώε κατά πάντα όσα προσέταξεν εις αυτόν ο Κύριος.
\par 6 Ήτο δε ο Νώε εξακοσίων ετών, ότε έγεινεν ο κατακλυσμός των υδάτων επί της γης.
\par 7 Και εισήλθεν ο Νώε, και οι υιοί αυτού, και η γυνή αυτού, και αι γυναίκες των υιών αυτού μετ' αυτού, εις την κιβωτόν, εξ αιτίας των υδάτων του κατακλυσμού.
\par 8 Από των κτηνών των καθαρών, και από των κτηνών των μη καθαρών, και από των πτηνών, και από πάντων των ερπόντων επί της γης,
\par 9 δύο δύο εισήλθον προς τον Νώε εις την κιβωτόν, άρσεν και θήλυ, καθώς προσέταξεν ο Θεός εις τον Νώε.
\par 10 Και μετά τας επτά ημέρας, τα ύδατα του κατακλυσμού επήλθον επί της γης.
\par 11 Το εξακοσιοστόν έτος της ζωής του Νώε, τον δεύτερον μήνα, την δεκάτην εβδόμην ημέραν του μηνός, ταύτην την ημέραν εσχίσθησαν πάσαι αι πηγαί της μεγάλης αβύσσου, και οι καταρράκται των ουρανών ηνοίχθησαν.
\par 12 Και έγεινεν ο υετός επί της γης τεσσαράκοντα ημέρας και τεσσαράκοντα νύκτας.
\par 13 Κατά την αυτήν ταύτην ημέραν εισήλθεν ο Νώε, και οι υιοί του Νώε, Σημ και Χαμ και Ιάφεθ, και η γυνή του Νώε, και αι τρεις γυναίκες των υιών αυτού μετ' αυτών, εις την κιβωτόν·
\par 14 αυτοί, και πάντα τα ζώα κατά το είδος αυτών, και πάντα τα κτήνη κατά το είδος αυτών, και πάντα τα ερπετά τα έρποντα επί της γης κατά το είδος αυτών, και πάντα τα πτηνά κατά το είδος αυτών, και παν πτερωτόν παντός είδους.
\par 15 Και εισήλθον προς τον Νώε εις την κιβωτόν, δύο δύο από πάσης σαρκός ήτις έχει πνεύμα ζωής.
\par 16 Και τα εισερχόμενα, άρσεν και θήλυ από πάσης σαρκός, εισήλθον, καθώς προσέταξεν εις αυτόν ο Θεός· και έκλεισεν ο Κύριος την κιβωτόν επάνω αυτού.
\par 17 Και έγεινεν ο κατακλυσμός τεσσαράκοντα ημέρας επί της γής· και επληθύνθησαν τα ύδατα, και εσήκωσαν την κιβωτόν, και υψώθη υπεράνω της γης.
\par 18 Και εκραταιούντο τα ύδατα, και επληθύνοντο σφόδρα επί της γής· και η κιβωτός εφέρετο επί της επιφανείας των υδάτων.
\par 19 Και τα ύδατα υπερεκραταιούντο σφόδρα επί της γής· και εσκεπάσθησαν πάντα τα όρη τα υψηλά τα υποκάτω παντός του ουρανού.
\par 20 Δεκαπέντε πήχας υπεράνω υψώθησαν τα ύδατα, και εσκεπάσθησαν τα όρη.
\par 21 Και απέθανε πάσα σαρξ κινουμένη επί της γης, των πτηνών και των κτηνών και των ζώων, και πάντων των ερπετών των ερπόντων επί της γης, και πας άνθρωπος.
\par 22 Εκ πάντων των όντων επί της ξηράς, πάντα όσα είχον πνοήν ζωής εις τους μυκτήρας αυτών, απέθανον.
\par 23 Και εξηλείφθη παν το υπάρχον επί του προσώπου της γης, από ανθρώπου έως κτήνους, έως ερπετού και έως πτηνού του ουρανού, και εξηλείφθησαν από της γής· έμενε δε μόνον ο Νώε, και όσα ήσαν μετ' αυτού εν τη κιβωτώ.
\par 24 Και εκραταιούντο τα ύδατα επί της γης εκατόν πεντήκοντα ημέρας.

\chapter{8}

\par 1 Και ενεθυμήθη ο Θεός τον Νώε, και πάντα τα ζώα, και πάντα τα κτήνη, τα μετ' αυτού εν τη κιβωτώ· και διεβίβασεν ο Θεός άνεμον επί την γην, και τα ύδατα εστάθησαν.
\par 2 Και εκλείσθησαν αι πηγαί της αβύσσου, και οι καταρράκται του ουρανού, και εκρατήθη ο υετός από των ουρανών.
\par 3 Και εσύροντο τα ύδατα από της γης κατά συνέχειαν· και ωλιγόστευον τα ύδατα μετά τας εκατόν πεντήκοντα ημέρας.
\par 4 Και εκάθισεν η κιβωτός την δεκάτην εβδόμην του εβδόμου μηνός επί των ορέων Αραράτ.
\par 5 Τα δε ύδατα ωλιγόστευον κατά συνέχειαν έως του δεκάτου μηνός· την πρώτην του δεκάτου μηνός εφάνησαν αι κορυφαί των ορέων.
\par 6 Και μετά τεσσαράκοντα ημέρας ήνοιξεν ο Νώε την θυρίδα της κιβωτού, την οποίαν είχε κάμει·
\par 7 και απέστειλε τον κόρακα, όστις εξελθών υπήγαινε και ήρχετο, εωσού εξηράνθησαν τα ύδατα από της γης.
\par 8 Και απέστειλε την περιστεράν κατόπιν αυτού, διά να ίδη αν έπαυσαν τα ύδατα από προσώπου της γής·
\par 9 και μη ευρίσκουσα η περιστερά ανάπαυσιν των ποδών αυτής, επέστρεψε προς αυτόν εις την κιβωτόν, διότι τα ύδατα ήσαν επί του προσώπου πάσης της γής· και εκτείνας την χείρα αυτού, επίασεν αυτήν και εισήγαγεν αυτήν προς εαυτόν εις την κιβωτόν.
\par 10 Και ανέμεινεν έτι άλλας επτά ημέρας, και πάλιν απέστειλε την περιστεράν εκ της κιβωτού·
\par 11 και επέστρεψε προς αυτόν η περιστερά προς το εσπέρας, και ιδού, ήτο εν τω στόματι αυτής φύλλον ελαίας, απεσπασμένον· και εγνώρισεν ο Νώε ότι έπαυσαν τα ύδατα από της γης.
\par 12 Και ανέμεινεν έτι άλλας επτά ημέρας, και απέστειλε την περιστεράν· και δεν επανέστρεψε πλέον προς αυτόν.
\par 13 Κατά δε το εξακοσιοστόν πρώτον έτος του Νώε, την πρώτην του πρώτου μηνός, εξέλιπον τα ύδατα από της γής· και εσήκωσεν ο Νώε την στέγην της κιβωτού, και είδε, και ιδού, εξέλιπε το ύδωρ από προσώπου της γης.
\par 14 Και την εικοστήν εβδόμην ημέραν του δευτέρου μηνός εξηράνθη η γή·
\par 15 και ελάλησεν ο Θεός προς τον Νώε, λέγων,
\par 16 Έξελθε εκ της κιβωτού, συ, και η γυνή σου, και οι υιοί σου, και αι γυναίκες των υιών σου μετά σού·
\par 17 πάντα τα ζώα τα μετά σου, από πάσης σαρκός, και πτηνά και κτήνη και παν ερπετόν έρπον επί της γης, εξάγαγε μετά σου, και ας πολυπλασιασθώσιν επί της γης, και ας αυξηνθώσι και ας πληθυνθώσιν επί της γης.
\par 18 Και εξήλθεν ο Νώε, και οι υιοί αυτού, και η γυνή αυτού, και αι γυναίκες των υιών αυτού μετ' αυτού·
\par 19 πάντα τα ζώα, πάντα τα ερπετά και πάντα τα πτηνά, παν ό,τι κινείται επί της γης, κατά τα είδη αυτών, εξήλθον εκ της κιβωτού.
\par 20 Και ωκοδόμησεν ο Νώε θυσιαστήριον εις τον Κύριον· και έλαβεν από παντός κτήνους καθαρού, και από παντός πτηνού καθαρού, και προσέφερεν ολοκαυτώματα επί του θυσιαστηρίου.
\par 21 Και ωσφράνθη Κύριος οσμήν ευωδίας· και είπε Κύριος εν τη καρδία αυτού, Δεν θέλω καταρασθή πλέον την γην εξ αιτίας του ανθρώπου· διότι ο λογισμός της καρδίας του ανθρώπου είναι κακός εκ νηπιότητος αυτού· ουδέ θέλω πατάξει πλέον πάντα τα ζώντα, καθώς έκαμον·
\par 22 εν όσω μένει γη, σπορά και θερισμός, και ψύχος και καύμα, και θέρος και χειμών, και ημέρα και νυξ, δεν θέλουσι παύσει.

\chapter{9}

\par 1 Και ευλόγησεν ο Θεός τον Νώε και τους υιούς αυτού· και είπε προς αυτούς, Αυξάνεσθε και πληθύνεσθε, και γεμίσατε την γήν·
\par 2 και ο φόβος σας και ο τρόμος σας θέλει είσθαι επί πάντα τα ζώα της γης, και επί πάντα τα πτηνά του ουρανού, επί παν ό,τι έρπει επί της γης, και επί πάντας τους ιχθύας της θαλάσσης· εις τας χείρας σας εδόθησαν·
\par 3 παν κινούμενον, το οποίον ζη, θέλει είσθαι εις σας προς τροφήν· ως τον χλωρόν χόρτον έδωκα τα πάντα εις εσάς·
\par 4 κρέας όμως με την ζωήν αυτού, με το αίμα αυτού, δεν θέλετε φάγει·
\par 5 και εξάπαντος το αίμα σας, το αίμα της ζωής σας, θέλω εκζητήσει εκ της χειρός παντός ζώου θέλω εκζητήσει αυτό, και εκ της χειρός του ανθρώπου· εκ της χειρός παντός αδελφού αυτού θέλω εκζητήσει την ζωήν του ανθρώπου·
\par 6 όστις χύση αίμα ανθρώπου, υπό ανθρώπου θέλει χυθή το αίμα αυτού· διότι κατ' εικόνα Θεού εποίησεν ο Θεός τον άνθρωπον·
\par 7 σεις δε αυξάνεσθε και πληθύνεσθε, πολλαπλασιάζεσθε επί της γης, και πληθύνεσθε επ' αυτής.
\par 8 Και είπεν ο Θεός προς τον Νώε και προς τους υιούς αυτού μετ' αυτού, λέγων,
\par 9 Και εγώ, ιδού, στήνω την διαθήκην μου προς εσάς, και προς το σπέρμα σας ύστερον από σάς·
\par 10 και προς παν έμψυχον ζώον, το οποίον είναι με σας, εκ των πτηνών, εκ των κτηνών και εκ πάντων των ζώων της γης, τα οποία είναι με σάς· από παντός του εξελθόντος εκ της κιβωτού, έως παντός ζώου της γής·
\par 11 και στήνω την διαθήκην μου προς εσάς· και δεν θέλει πλέον εξολοθρευθή πάσα σαρξ από των υδάτων του κατακλυσμού· ουδέ θέλει είσθαι πλέον κατακλυσμός διά να φθείρη την γην.
\par 12 Και είπεν ο Θεός, Τούτο είναι το σημείον της διαθήκης, την οποίαν εγώ κάμνω μεταξύ εμού και υμών και παντός εμψύχου ζώου το οποίον είναι με σας, εις γενεάς αιωνίους·
\par 13 Θέτω το τόξον μου εν τη νεφέλη, και θέλει είσθαι εις σημείον διαθήκης μεταξύ εμού και της γής·
\par 14 και όταν συννεφώσω νεφέλην επί της γης, θέλει φανή το τόξον εν τη νεφέλη·
\par 15 και θέλω ενθυμηθή την διαθήκην μου, την μεταξύ εμού και υμών, και παντός εμψύχου ζώου εκ πάσης σαρκός· και τα ύδατα δεν θέλουσιν είσθαι πλέον εις κατακλυσμόν διά να εξαλείψωσι πάσαν σάρκα·
\par 16 και το τόξον θέλει είσθαι εν τη νεφέλη· και θέλω βλέπει αυτό, διά να ενθυμώμαι την παντοτεινήν διαθήκην την μεταξύ Θεού και παντός εμψύχου ζώου εκ πάσης σαρκός ήτις είναι επί της γης.
\par 17 Και είπεν ο Θεός προς τον Νώε, Τούτο είναι το σημείον της διαθήκης, την οποίαν έστησα μεταξύ εμού και πάσης σαρκός ήτις είναι επί της γης.
\par 18 Ήσαν δε οι υιοί του Νώε, οι εξελθόντες εκ της κιβωτού, Σημ και Χαμ και Ιάφεθ. Ο δε Χαμ ήτο πατήρ του Χαναάν.
\par 19 Οι τρεις ούτοι είναι οι υιοί του Νώε, και εκ τούτων διεσπάρησαν εις πάσαν την γην.
\par 20 Και ήρχισεν ο Νώε να ήναι γεωργός και εφύτευσεν αμπελώνα·
\par 21 και έπιεν εκ του οίνου και εμεθύσθη, και εγυμνώθη εν τη σκηνή αυτού.
\par 22 Και είδεν ο Χαμ, ο πατήρ του Χαναάν, την γύμνωσιν του πατρός αυτού· και ανήγγειλε τούτο προς τους δύο αδελφούς αυτού έξω.
\par 23 Και λαβόντες ο Σημ και ο Ιάφεθ το ένδυμα, επέθηκαν αυτό επί τα δύο αυτών νώτα· και βαδίσαντες οπισθόνωτα, εσκέπασαν την γύμνωσιν του πατρός αυτών· και τα πρόσωπα αυτών ήσαν προς τα οπίσω, και την γύμνωσιν του πατρός αυτών δεν είδον.
\par 24 Ανανήψας δε ο Νώε από του οίνου αυτού, έμαθεν όσα έκαμεν εις αυτόν ο υιός αυτού ο νεώτερος.
\par 25 Και είπεν, Επικατάρατος ο Χαναάν· δούλος των δούλων θέλει είσθαι εις τους αδελφούς αυτού.
\par 26 Και είπεν, Ευλογητός Κύριος ο Θεός του Σημ. Και ο Χαναάν θέλει είσθαι δούλος εις αυτόν·
\par 27 ο Θεός θέλει πλατύνει τον Ιάφεθ, και θέλει κατοικήσει εν ταις σκηναίς του Σημ, ο δε Χαναάν θέλει είσθαι δούλος εις αυτόν·
\par 28 Και έζησεν ο Νώε μετά τον κατακλυσμόν τριακόσια πεντήκοντα έτη.
\par 29 Και έγειναν πάσαι αι ημέραι του Νώε εννεακόσια πεντήκοντα έτη· και απέθανε.

\chapter{10}

\par 1 Και αύται είναι αι γενεαλογίαι των υιών του Νώε, Σημ, Χαμ και Ιάφεθ· και εγεννήθησαν εις αυτούς υιοί μετά τον κατακλυσμόν.
\par 2 Οι υιοί του Ιάφεθ ήσαν Γομέρ, και Μαγώγ, και Μαδαΐ, και Ιαυάν, και Θουβάλ, και Μεσέχ, και Θειράς.
\par 3 Και οι υιοί του Γομέρ, Ασχενάζ, και Ριφάθ, και Θωγαρμά.
\par 4 Και οι υιοί του Ιαυάν, Ελεισά, και Θαρσείς, Κιττείμ, και Δωδανείμ.
\par 5 Εκ τούτων εμοιράσθησαν αι νήσοι των εθνών εις τους τόπους αυτών· εκάστου κατά την γλώσσαν αυτού, κατά τας φυλάς αυτών, εις τα έθνη αυτών.
\par 6 Και οι υιοί του Χαμ, Χούς, και Μισραΐμ, και Φούθ, και Χαναάν.
\par 7 Και οι υιοί του Χούς, Σεβά, και Αβιλά, και Σαβθά, και Ρααμά, και Σαβθεκά· και οι υιοί του Ρααμά, Σεβά, και Δαιδάν.
\par 8 Και ο Χούς εγέννησε τον Νεβρώδ· ούτος ήρχισε να ήναι ισχυρός επί της γής·
\par 9 αυτός ήτο ισχυρός κυνηγός ενώπιον του Κυρίου· διά τούτο λέγεται, Ως Νεβρώδ, ισχυρός κυνηγός ενώπιον του Κυρίου·
\par 10 και η αρχή της βασιλείας αυτού εστάθη Βαβυλών, και Ερέχ, και Αχάδ, και Χαλνέ, εν τη γη Σενναάρ.
\par 11 Εκ της γης εκείνης εξήλθεν ο Ασσούρ, και ωκοδόμησε την Νινευή, και την πόλιν Ρεχωβώθ, και την Χαλάχ,
\par 12 και την Ρεσέν μεταξύ της Νινευή και της Χαλάχ· αύτη είναι η πόλις η μεγάλη.
\par 13 Και ο Μισραΐμ εγέννησε τους Λουδείμ, και τους Αναμείμ, και τους Λεαβείμ, και τους Ναφθουχείμ,
\par 14 και τους Πατρουσείμ, και τους Χασλουχείμ, εκ των οποίων εξήλθον οι Φιλισταίοι, και τους Χαφθορείμ.
\par 15 Και ο Χαναάν εγέννησε τον Σιδώνα, πρωτότοκον αυτού, και τον Χετταίον,
\par 16 και τον Ιεβουσαίον, και τον Αμορραίον, και τον Γεργεσαίον,
\par 17 και τον Ευαίον, και τον Αρουχαίον, και τον Ασενναίον,
\par 18 και τον Αρβάδιον, και τον Σαμαραίον, και τον Αμαθαίον. Και μετά τούτο διεσπάρησαν αι φυλαί των Χαναναίων.
\par 19 Και ήσαν τα όρια των Χαναναίων από Σιδώνος, καθώς υπάγει τις εις Γέραρα, έως Γάζης, και καθώς υπάγει τις εις Σόδομα και Γόμορρα, και Αδαμά, και Σεβωείμ, έως Λασά.
\par 20 Ούτοι είναι οι υιοί του Χαμ, κατά τας φυλάς αυτών, κατά τας γλώσσας αυτών, εις τους τόπους αυτών, εις τα έθνη αυτών.
\par 21 Και εις τον Σημ, τον πατέρα πάντων των υιών του Έβερ, τον αδελφόν Ιάφεθ του μεγαλητέρου, εγεννήθησαν και εις αυτόν υιοί.
\par 22 Υιοί του Σημ ήσαν Ελάμ, και Ασσούρ, και Αρφαξάδ, και Λούδ, και Αράμ.
\par 23 Και οι υιοί του Αράμ, Ουζ, και Ουλ, και Γεθέρ, και Μας.
\par 24 Και ο Αρφαξάδ εγέννησε τον Σαλά· και ο Σαλά εγέννησε τον Έβερ.
\par 25 Και εις τον Έβερ εγεννήθησαν δύο υιοί· το όνομα του ενός, Φαλέγ· διότι εν ταις ημέραις αυτού διεμερίσθη η γή· και το όνομα του αδελφού αυτού, Ιοκτάν.
\par 26 Και ο Ιοκτάν εγέννησε τον Αλμωδάδ, και τον Σαλέφ, και τον Ασαρμαβέθ, και τον Ιαράχ,
\par 27 και τον Αδωράμ, και τον Ουζάλ, και τον Δικλά,
\par 28 και τον Οβάλ, και τον Αβιμαήλ, και τον Σεβά,
\par 29 και τον Οφείρ, και τον Αβιλά, και τον Ιωβάβ· πάντες ούτοι ήσαν υιοί του Ιοκτάν.
\par 30 Και η κατοικία αυτών ήτο από Μησά, καθώς υπάγει τις εις Σεφαρά, όρος της Ανατολής.
\par 31 Ούτοι είναι οι υιοί του Σημ, κατά τας φυλάς αυτών, κατά τας γλώσσας αυτών, εις τους τόπους αυτών, κατά τα έθνη αυτών.
\par 32 Αύται είναι αι φυλαί των υιών του Νώε, κατά τας γενεάς αυτών, εις τα έθνη αυτών· και εκ τούτων διεσπάρησαν τα έθνη επί της γης μετά τον κατακλυσμόν.

\chapter{11}

\par 1 Και ήτο πάσα η γη μιας γλώσσης και μιας φωνής.
\par 2 Και ότε εκίνησαν από της ανατολής, εύρον πεδιάδα εν τη γη Σενναάρ· και κατώκησαν εκεί.
\par 3 Και είπεν ο εις προς τον άλλον, Έλθετε, ας κάμωμεν πλίνθους, και ας ψήσωμεν αυτάς εν πυρί· και εχρησίμευσεν εις αυτούς η μεν πλίνθος αντί πέτρας, η δε άσφαλτος εχρησίμευσεν εις αυτούς αντί πηλού.
\par 4 Και είπον, Έλθετε, ας οικοδομήσωμεν εις εαυτούς πόλιν και πύργον, του οποίου η κορυφή να φθάνη έως του ουρανού· και ας αποκτήσωμεν εις εαυτούς όνομα, μήπως διασπαρώμεν επί του προσώπου πάσης της γης.
\par 5 Κατέβη δε ο Κύριος διά να ίδη την πόλιν και τον πύργον, τον οποίον ωκοδόμησαν οι υιοί των ανθρώπων.
\par 6 Και είπεν ο Κύριος, Ιδού, εις λαός, και πάντες έχουσι μίαν γλώσσαν, και ήρχισαν να κάμνωσι τούτο· και τώρα δεν θέλει εμποδισθή εις αυτούς παν ό,τι σκοπεύουσι να κάμωσιν·
\par 7 έλθετε, ας καταβώμεν και ας συγχύσωμεν εκεί την γλώσσαν αυτών, διά να μη εννοή ο εις του άλλου την γλώσσαν.
\par 8 Και διεσκόρπισεν αυτούς ο Κύριος εκείθεν επί του προσώπου πάσης της γής· και έπαυσαν να οικοδομώσι την πόλιν.
\par 9 Διά τούτο ωνομάσθη το όνομα αυτής Βαβέλ· διότι εκεί συνέχεεν ο Κύριος την γλώσσαν πάσης της γής· και εκείθεν διεσκόρπισεν αυτούς ο Κύριος επί το πρόσωπον πάσης της γης.
\par 10 Αύτη είναι η γενεαλογία του Σημ. Ο Σημ ήτο ετών εκατόν, ότε εγέννησε τον Αρφαξάδ δύο έτη μετά τον κατακλυσμόν·
\par 11 και έζησεν ο Σημ, αφού εγέννησε τον Αρφαξάδ, πεντακόσια έτη, και εγέννησεν υιούς και θυγατέρας.
\par 12 Και ο Αρφαξάδ έζησε τριάκοντα πέντε έτη, και εγέννησε τον Σαλά·
\par 13 και έζησεν ο Αρφαξάδ, αφού εγέννησε τον Σαλά, τετρακόσια τρία έτη, και εγέννησεν υιούς και θυγατέρας.
\par 14 Και ο Σαλά έζησε τριάκοντα έτη, και εγέννησε τον Εβερ·
\par 15 και έζησεν ο Σαλά, αφού εγέννησε τον Έβερ, τετρακόσια τρία έτη, και εγέννησεν υιούς και θυγατέρας.
\par 16 Και έζησεν ο Έβερ τριάκοντα τέσσαρα έτη, και εγέννησε τον Φαλέγ·
\par 17 και έζησεν ο Έβερ, αφού εγέννησε τον Φαλέγ, τετρακόσια τριάκοντα έτη, και εγέννησεν υιούς και θυγατέρας.
\par 18 Και έζησεν ο Φαλέγ τριάκοντα έτη, και εγέννησε τον Ραγαύ·
\par 19 και έζησεν ο Φαλέγ, αφού εγέννησε τον Ραγαύ, διακόσια εννέα έτη, και εγέννησεν υιούς και θυγατέρας.
\par 20 Και έζησεν ο Ραγαύ τριάκοντα δύο έτη, και εγέννησε τον Σερούχ·
\par 21 και έζησεν ο Ραγαύ, αφού εγέννησε τον Σερούχ, διακόσια επτά έτη, και εγέννησεν υιούς και θυγατέρας.
\par 22 Και έζησεν ο Σερούχ τριάκοντα έτη, και εγέννησε τον Ναχώρ·
\par 23 και έζησεν ο Σερούχ, αφού εγέννησε τον Ναχώρ, διακόσια έτη, και εγέννησεν υιούς και θυγατέρας.
\par 24 Και έζησεν ο Ναχώρ εικοσιεννέα έτη, και εγέννησε τον Θάρα·
\par 25 και έζησεν ο Ναχώρ, αφού εγέννησε τον Θάρα, εκατόν δεκαεννέα έτη, και εγέννησεν υιούς και θυγατέρας.
\par 26 Και έζησεν ο Θάρα εβδομήκοντα έτη, και εγέννησε τον Άβραμ, τον Ναχώρ, και τον Αρράν.
\par 27 Και αύτη είναι η γενεαλογία του Θάρα· ο Θάρα εγέννησε τον Άβραμ, τον Ναχώρ, και τον Αρράν· και ο Αρράν εγέννησε τον Λωτ.
\par 28 Και απέθανεν ο Αρράν ενώπιον Θάρα του πατρός αυτού εν τω τόπω της γεννήσεως αυτού, εν Ουρ των Χαλδαίων.
\par 29 Και έλαβον ο Άβραμ και ο Ναχώρ εις εαυτούς γυναίκας· το όνομα της γυναικός του Άβραμ ήτο Σάρα· και το όνομα της γυναικός του Ναχώρ, Μελχά, θυγάτηρ του Αρράν, πατρός Μελχά, και πατρός Ιεσχά.
\par 30 Η δε Σάρα ήτο στείρα, δεν είχε τέκνον.
\par 31 Και έλαβεν ο Θάρα Άβραμ τον υιόν αυτού και Λωτ τον υιόν του Αρράν εγγονόν εαυτού, και Σάραν την εαυτού νύμφην, την γυναίκα Άβραμ του υιού αυτού· και εξήλθον ομού από της Ουρ των Χαλδαίων, διά να υπάγωσιν εις την γην Χαναάν· και ήλθον έως Χαρράν και κατώκησαν εκεί.
\par 32 Και έγειναν αι ημέραι του Θάρα διακόσια πέντε έτη· και απέθανεν ο Θάρα εν Χαρράν.

\chapter{12}

\par 1 Ο δε Κύριος είπε προς τον Άβραμ, Έξελθε εκ της γης σου, και εκ της συγγενείας σου, και εκ του οίκου του πατρός σου, εις την γην την οποίαν θέλω σοι δείξει·
\par 2 και θέλω σε κάμει εις έθνος μέγα· και θέλω σε ευλογήσει, και θέλω μεγαλύνει το όνομά σου· και θέλεις είσθαι εις ευλογίαν·
\par 3 και θέλω ευλογήσει τους ευλογούντάς σε, και τους καταρωμένους σε θέλω καταρασθή· και θέλουσιν ευλογηθή εν σοι πάσαι αι φυλαί της γης.
\par 4 Και υπήγεν ο Άβραμ, καθώς είπε προς αυτόν ο Κύριος· και μετ' αυτού υπήγε και ο Λώτ· ο δε Άβραμ ήτο ηλικίας εβδομήκοντα πέντε ετών, ότε εξήλθεν από Χαρράν.
\par 5 Και έλαβεν ο Άβραμ Σάραν την γυναίκα αυτού, και Λωτ τον υιόν του αδελφού αυτού, και πάντα τα υπάρχοντα αυτών όσα είχον αποκτήσει, και τους ανθρώπους τους οποίους είχον αποκτήσει εν Χαρράν, και εξήλθον διά να υπάγωσιν εις την γην Χαναάν· και ήλθον εις την γην Χαναάν.
\par 6 Και διεπέρασεν ο Άβραμ την γην εκείνην έως του τόπου Συχέμ, έως της δρυός Μορέχ· οι δε Χαναναίοι τότε κατώκουν εν τη γη ταύτη.
\par 7 Και εφάνη ο Κύριος εις τον Άβραμ και είπεν, Εις το σπέρμα σου θέλω δώσει την γην ταύτην. Και ωκοδόμησεν εκεί θυσιαστήριον εις τον Κύριον, όστις εφάνη εις αυτόν.
\par 8 Και εκείθεν μετέβη προς το όρος, το κατά ανατολάς της Βαιθήλ, και έστησε την σκηνήν αυτού έχων την Βαιθήλ προς δυσμάς και την Γαί προς ανατολάς· και ωκοδόμησεν εκεί θυσιαστήριον εις τον Κύριον, και επεκαλέσθη το όνομα του Κυρίου.
\par 9 Και μετεσκήνωσεν ο Άβραμ, οδοιπορών και προχωρών προς μεσημβρίαν.
\par 10 Έγεινε δε πείνα εν τη γη ταύτη· και κατέβη ο Άβραμ εις την Αίγυπτον διά να παροικήση εκεί· διότι η πείνα ήτο βαρεία εν τη γη.
\par 11 Και ότε επλησίαζε να εισέλθη εις την Αίγυπτον, είπε προς Σάραν την γυναίκα αυτού, Ιδού, γνωρίζω ότι είσαι γυνή ευειδής·
\par 12 θέλει συμβή λοιπόν, ώστε καθώς σε ίδωσιν οι Αιγύπτιοι, θέλουσιν ειπεί, Γυνή αυτού είναι αύτη· και θέλουσι φονεύσει εμέ, σε δε θέλουσι φυλάξει ζώσαν.
\par 13 Ειπέ λοιπόν, ότι είσαι αδελφή μου, διά να γείνη καλόν εις εμέ εξ αιτίας σου, και να φυλαχθή η ζωή μου διά σε.
\par 14 Και ότε εισήλθεν ο Άβραμ εις την Αίγυπτον, είδον οι Αιγύπτιοι την γυναίκα ότι ήτο ώραία σφόδρα.
\par 15 Και οι άρχοντες του Φαραώ είδον αυτήν, και επήνεσαν αυτήν προς τον Φαραώ· και ελήφθη η γυνή εις την οικίαν του Φαραώ.
\par 16 Τον δε Άβραμ μετεχειρίσθησαν καλώς δι' αυτήν· και είχε πρόβατα και βόας και όνους και δούλους και δούλας και όνους θηλυκάς και καμήλους.
\par 17 Και επέφερεν ο Κύριος επί τον Φαραώ και επί τον οίκον αυτού πληγάς μεγάλας εξ αιτίας Σάρας της γυναικός του Άβραμ.
\par 18 Εκάλεσε δε ο Φαραώ τον Άβραμ, και είπε, Τι είναι τούτο, το οποίον έκαμες εις εμέ; διά τι δεν μ' εφανέρωσας ότι αύτη είναι γυνή σου;
\par 19 διά τι είπας, Αδελφή μου είναι αύτη; και έλαβον αυτήν εις εμαυτόν διά γυναίκα· και τώρα, ιδού η γυνή σου· λάβε αυτήν, και ύπαγε.
\par 20 Και διώρισεν ο Φαραώ ανθρώπους εις αυτόν· και συμπροέπεμψαν αυτόν, και την γυναίκα αυτού και πάντα όσα είχε.

\chapter{13}

\par 1 Ανέβη δε ο Άβραμ εξ Αιγύπτου, αυτός και η γυνή αυτού, και πάντα όσα είχε, και ο Λωτ μετ' αυτού, προς μεσημβρίαν.
\par 2 Και ο Άβραμ ήτο πλούσιος σφόδρα εις κτήνη, εις αργύριον και εις χρυσίον.
\par 3 Και υπήγεν οδεύων από μεσημβρίας έως Βαιθήλ έως του τόπου όπου ήτο η σκηνή αυτού το πρότερον μεταξύ Βαιθήλ και Γαί·
\par 4 εις τον τόπον του θυσιαστηρίου, το οποίον είχε κάμει εκεί καταρχάς· και επεκαλέσθη εκεί ο Άβραμ το όνομα του Κυρίου.
\par 5 Και ο Λωτ ακόμη, ο συμπορευόμενος μετά του Άβραμ, είχε πρόβατα και βόας και σκηνάς.
\par 6 Και δεν εχώρει αυτούς η γη διά να κατοικώσιν ομού· διότι ήσαν τα υπάρχοντα αυτών πολλά, και δεν ηδύναντο να κατοικώσιν ομού.
\par 7 Και συνέβη έρις μεταξύ των ποιμένων των κτηνών του Άβραμ και των ποιμένων των κτηνών του Λώτ· οι δε Χαναναίοι και οι Φερεζαίοι κατώκουν τότε την γην.
\par 8 Είπε δε ο Άβραμ προς τον Λωτ, Ας μη ήναι, παρακαλώ, έρις μεταξύ εμού και σου και μεταξύ των ποιμένων μου και των ποιμένων σου· διότι αδελφοί είμεθα ημείς·
\par 9 δεν είναι πάσα η γη έμπροσθέν σου; διαχωρίσθητι λοιπόν απ' εμού· εάν συ υπάγης εις τα αριστερά, εγώ υπάγω εις τα δεξιά· και εάν συ εις τα δεξιά, εγώ εις τα αριστερά.
\par 10 Και υψώσας ο Λωτ τους οφθαλμούς αυτού, είδε πάσαν την περίχωρον του Ιορδάνου, ότι εποτίζετο όλη προ του να καταστρέψη ο Κύριος τα Σόδομα και τα Γόμορρα, ως παράδεισος του Κυρίου, ως η γη της Αιγύπτου, έως να υπάγη τις εις Σηγώρ.
\par 11 Και έκλεξεν εις εαυτόν ο Λωτ πάσαν την περίχωρον του Ιορδάνου· και μετεσκήνωσεν ο Λωτ προς ανατολάς, και διεχωρίσθησαν ο εις από του άλλου.
\par 12 Ο μεν Άβραμ κατώκησεν εν τη γη Χαναάν· ο δε Λωτ κατώκησε μεταξύ των πόλεων της περιχώρου, και έστησε τας σκηνάς αυτού έως Σοδόμων.
\par 13 Οι δε άνθρωποι των Σοδόμων ήσαν κακοί και αμαρτωλοί σφόδρα ενώπιον του Κυρίου.
\par 14 Και είπε Κύριος προς τον Άβραμ, αφού διεχωρίσθη ο Λωτ απ' αυτού, Ύψωσον τώρα τους οφθαλμούς σου, και ιδέ από του τόπου όπου είσαι, προς άρκτον και μεσημβρίαν και ανατολήν και δύσιν·
\par 15 διότι πάσαν την γην την οποίαν βλέπεις, εις σε θέλω δώσει αυτήν και εις το σπέρμα σου έως αιώνος·
\par 16 και θέλω καταστήσει το σπέρμα σου ως την άμμον της γής· ώστε εάν δύναταί τις να εξαριθμήση την άμμον της γης, θέλει αριθμηθή και το σπέρμα σου.
\par 17 Σηκωθείς διόδευσον την γην εις τε το μήκος αυτής και εις το πλάτος αυτής· διότι εις σε θέλω δώσει αυτήν.
\par 18 Και εσήκωσε την σκηνήν αυτού ο Άβραμ, και ελθών κατώκησε πλησίον των δρυών Μαμβρή, αίτινες είναι εν Χεβρών, και ωκοδόμησεν εκεί θυσιαστήριον εις τον Κύριον.

\chapter{14}

\par 1 Επί των ημερών δε του Αμαρφέλ βασιλέως Σενναάρ, του Αριώχ βασιλέως Ελλασάρ, του Χοδολλογομόρ βασιλέως Ελάμ, και του Θαργάλ βασιλέως εθνών,
\par 2 έκαμον αυτοί πόλεμον μετά του Βερά βασιλέως Σοδόμων, και του Βαρσά βασιλέως Γομόρρων, του Σενναάβ βασιλέως Αδαμά, και του Σεμοβόρ βασιλέως Σεβωείμ, και του βασιλέως της Βελά· αύτη είναι η Σηγώρ.
\par 3 Πάντες ούτοι ηνώθησαν ομού εν τη κοιλάδι Σιδδίμ ήτις είναι η αλμυρά θάλασσα.
\par 4 Δώδεκα έτη εδούλευον εις τον Χοδολλογομόρ· εν δε τω δεκάτω τρίτω απεστάτησαν.
\par 5 Και εν τω δεκάτω τετάρτω έτει ήλθεν ο Χοδολλογομόρ και οι βασιλείς οι μετ' αυτού, και επάταξαν τους Ραφαείμ εν Ασταρώθ-καρναΐμ, και τους Ζουζείμ εν Αμ, και τους Εμμαίους εν Σαυή-κιριαθαΐμ,
\par 6 και τους Χορραίους εν τω όρει αυτών Σηείρ έως της πεδιάδος Φαράν, ήτις είναι εν τη ερήμω.
\par 7 Επέστρεψαν δε και ήλθον εις την Εν-μισπάτ ήτις είναι η Κάδης· και επάταξαν πάντα τον τόπον του Αμαλήκ, και τους Αμορραίους τους κατοικούντας εν Ασασών-θαμάρ.
\par 8 Εξήλθε δε ο βασιλεύς των Σοδόμων, και ο βασιλεύς των Γομόρρων, και ο βασιλεύς της Αδαμά, και ο βασιλεύς των Σεβωείμ, και ο βασιλεύς της Βελά, ήτις είναι η Σηγώρ· και συνεκρότησαν μάχην μετ' αυτών εν τη κοιλάδι Σιδδίμ,
\par 9 μετά του Χοδολλογομόρ βασιλέως Ελάμ, και του Θαργάλ βασιλέως εθνών, και του Αμραφέλ βασιλέως Σενναάρ, και του Αριώχ βασιλέως Ελλασάρ· τέσσαρες βασιλείς προς πέντε.
\par 10 Η δε κοιλάς Σιδδίμ ήτο πλήρης φρεάτων ασφάλτου· ετράπησαν δε εις φυγήν οι βασιλείς των Σοδόμων και των Γομόρρων και έπεσον εκεί· οι δε εναπολειφθέντες έφυγον εις το όρος.
\par 11 Και έλαβον πάντα τα υπάρχοντα των Σοδόμων και των Γομόρρων και πάσαν αυτών την ζωοτροφίαν, και ανεχώρησαν.
\par 12 Έλαβον δε και τον Λωτ υιόν του αδελφού του Άβραμ, όστις κατώκει εν Σοδόμοις, και τα υπάρχοντα αυτού, και ανεχώρησαν.
\par 13 Υπήγε δε τις εκ των διασωθέντων και απήγγειλε τούτο προς τον Άβραμ τον Εβραίον, όστις κατώκει πλησίον των δρυών Μαμβρή του Αμορραίου, αδελφού του Εσχώλ, και αδελφού του Ανήρ, οίτινες ήσαν σύμμαχοι του Άβραμ.
\par 14 Ακούσας δε ο Άβραμ ότι ηχμαλωτίσθη ο αδελφός αυτού, εφώπλισε τριακοσίους δεκαοκτώ εκ των δούλων αυτού, των γεννηθέντων εν τη οικία αυτού, και κατεδίωξεν οπίσω αυτών έως Δαν.
\par 15 Και διαιρέσας τους εαυτού ώρμησε κατ' αυτών την νύκτα, αυτός και οι δούλοι αυτού, και επάταξεν αυτούς, και κατεδίωξεν αυτούς έως Χοβά ήτις είναι κατά τα αριστερά της Δαμασκού.
\par 16 Και επανέφερε πάντα τα υπάρχοντα και έτι επανέφερε Λωτ τον αδελφόν αυτού και τα υπάρχοντα αυτού, έτι δε και τας γυναίκας και τον λαόν.
\par 17 Εξήλθε δε ο βασιλεύς των Σοδόμων εις συνάντησιν αυτού, αφού επέστρεψεν από της καταστροφής του Χοδολλογομόρ και των βασιλέων των μετ' αυτού, εν τη κοιλάδι Σαυή ήτις είναι η κοιλάς του βασιλέως.
\par 18 Και ο Μελχισεδέκ βασιλεύς Σαλήμ έφερεν έξω άρτον και οίνον· ήτο δε ιερεύς του Θεού του Υψίστου.
\par 19 Και ευλόγησεν αυτόν και είπεν, Ευλογημένος ο Άβραμ παρά του Θεού του Υψίστου, όστις έκτισε τον ουρανόν και την γήν·
\par 20 και ευλογητός ο Θεός ο Ύψιστος όστις παρέδωκε τους εχθρούς σου εις την χείρα σου. Και Άβραμ έδωκεν εις αυτόν δέκατον από πάντων.
\par 21 Και είπεν ο βασιλεύς των Σοδόμων προς τον Άβραμ, Δος μοι τους ανθρώπους, τα δε υπάρχοντα λάβε εις σεαυτόν.
\par 22 Είπε δε ο Άβραμ προς τον βασιλέα των Σοδόμων, Εγώ ύψωσα την χείρα μου προς Κύριον, τον Θεόν τον Ύψιστον, όστις έκτισε τον ουρανόν και την γην,
\par 23 ότι δεν θέλω λάβει από πάντων των ιδικών σου από κλωστής έως λωρίου υποδήματος, διά να μη είπης, Εγώ επλούτισα τον Αβραμ·
\par 24 εκτός μόνον εκείνου το οποίον έφαγον οι νέοι, και της μερίδος των ανθρώπων των ελθόντων μετ' εμού, του Ανήρ του Εσχώλ και του Μαμβρή, ούτοι ας λάβωσι την μερίδα αυτών.

\chapter{15}

\par 1 Μετά τα πράγματα ταύτα έγεινε λόγος Κυρίου προς τον Άβραμ εν οράματι, λέγων, Μη φοβού, Αβραμ· εγώ είμαι ο υπερασπιστής σου, ο μισθός σου θέλει είσθαι πολύς σφόδρα.
\par 2 Και είπεν ο Άβραμ, Δέσποτα Κύριε, τι θέλεις δώσει εις εμέ, ενώ εγώ απέρχομαι άτεκνος, ο δε κληρονόμος της οικίας μου είναι ούτος ο εκ Δαμασκού Ελιέζερ;
\par 3 είπε προσέτι ο Άβραμ, Ιδού, δεν έδωκας εις εμέ σπέρμα· και ιδού, οικέτης μου θέλει με κληρονομήσει.
\par 4 Και ιδού, λόγος Κυρίου έγεινε προς αυτόν, λέγων, Δεν θέλει σε κληρονομήσει ούτος· αλλ' εκείνος όστις θέλει εξέλθει εκ των σπλάγχνων σου, αυτός θέλει σε κληρονομήσει.
\par 5 Και έφερεν αυτόν έξω και είπεν, Ανάβλεψον τώρα εις τον ουρανόν και αρίθμησον τα άστρα, εάν δύνασαι να εξαριθμήσης αυτά· και είπε προς αυτόν, Ούτω θέλει είσθαι το σπέρμα σου.
\par 6 Και επίστευσεν εις τον Κύριον· και ελογίσθη εις αυτόν εις δικαιοσύνην.
\par 7 Και είπε προς αυτόν, Εγώ είμαι ο Κύριος όστις σε εξήγαγον εκ της Ουρ των Χαλδαίων, διά να σοι δώσω την γην ταύτην εις κληρονομίαν.
\par 8 Ο δε είπε, Δέσποτα Κύριε, Πόθεν να γνωρίσω ότι θέλω κληρονομήσει αυτήν;
\par 9 Και είπε προς αυτόν, Λάβε μοι δάμαλιν τριών ετών, και αίγα τριών ετών, και κριόν τριών ετών, και τρυγόνα, και περιστεράν.
\par 10 Και έλαβεν εις αυτόν πάντα ταύτα, και διέσχισεν αυτά εις το μέσον, και έθεσεν έκαστον τμήμα απέναντι του ομοίου αυτού· τα πτηνά όμως δεν διέσχισε.
\par 11 Κατέβησαν δε όρνεα επί τα πτώματα, και ο Άβραμ εδίωξεν αυτά.
\par 12 Περί δε την δύσιν του ηλίου, επέπεσεν έκστασις επί τον Αβραμ· και ιδού, φόβος σκοτεινός μέγας επιπίπτει επ' αυτόν.
\par 13 Και είπεν ο Κύριος προς τον Άβραμ, Έξευρε βεβαίως ότι το σπέρμα σου θέλει παροικήσει εν γη ουχί εαυτών, και θέλουσι δουλώσει αυτούς, και θέλουσι καταθλίψει αυτούς, τετρακόσια έτη·
\par 14 το έθνος όμως, εις το οποίον θέλουσι δουλωθή, εγώ θέλω κρίνει· μετά δε ταύτα θέλουσιν εξέλθει με μεγάλα υπάρχοντα·
\par 15 συ δε θέλεις απέλθει προς τους πατέρας σου εν ειρήνη· θέλεις ενταφιασθή εν γήρατι καλώ·
\par 16 εν δε τη τετάρτη γενεά θέλουσιν επιστρέψει εδώ· διότι ακόμη δεν ανεπληρώθη η ανομία των Αμορραίων.
\par 17 Ότε δε ο ήλιος έδυσε και έγεινε πυκνόν σκότος, ιδού, κάμινος καπνίζουσα και λαμπάς πυρός ήτις διεπέρασε μεταξύ των διχοτομημάτων τούτων.
\par 18 Την ημέραν εκείνην έκαμε διαθήκην ο Κύριος προς τον Άβραμ, λέγων, εις το σπέρμα σου έδωκα την γην ταύτην, από του ποταμού της Αιγύπτου έως του ποταμού του μεγάλου, του ποταμού Ευφράτου·
\par 19 τους Κεναίους, και τους Κενεζαίους, και τους Κεδμωναίους,
\par 20 και τους Χετταίους, και τους Φερεζαίους, και τους Ραφαείμ,
\par 21 και τους Αμορραίους, και τους Χαναναίους, και τους Γεργεσαίους, και τους Ιεβουσαίους.

\chapter{16}

\par 1 Η δε Σάρα, η γυνή του Άβραμ, δεν ετεκνοποίει εις αυτόν· είχε δε δούλην Αιγυπτίαν, ονομαζομένην Άγαρ.
\par 2 Και είπεν η Σάρα προς τον Άβραμ, Ιδού, ο Κύριος με απέκλεισε της τεκνοποιΐας· είσελθε λοιπόν προς την δούλην μου, ίσως αποκτήσω τέκνον εξ αυτής. Υπήκουσε δε ο Άβραμ εις τον λόγον της Σάρας.
\par 3 Και έλαβεν η Σάρα η γυνή του Άβραμ την Άγαρ την Αιγυπτίαν, την δούλην αυτής, αφού ο Άβραμ είχε κατοικήσει δέκα έτη εν τη γη Χαναάν, και έδωκεν αυτήν εις Άβραμ τον άνδρα αυτής, διά να ήναι γυνή αυτού.
\par 4 Και εισήλθε προς την Άγαρ, και εκείνη συνέλαβε· και ότε είδεν ότι συνέλαβεν, η κυρία αυτής κατεφρονείτο ενώπιον αυτής.
\par 5 Και είπεν η Σάρα προς τον Άβραμ, Εξ αιτίας σου αδικούμαι. Εγώ έδωκα την δούλην μου εις τον κόλπον σου· και αφού είδεν ότι συνέλαβεν, εγώ κατεφρονήθην ενώπιον αυτής· ας κρίνη ο Κύριος μεταξύ εμού και σου.
\par 6 Ο δε Άβραμ είπε προς την Σάραν, Ιδού, η δούλη σου είναι εις την χείρα σου· κάμε εις αυτήν όπως είναι αρεστόν εις τους οφθαλμούς σου. Και μετεχειρίσθη η Σάρα αυτήν κακώς, και εκείνη έφυγεν από προσώπου αυτής.
\par 7 Εύρε δε αυτήν άγγελος Κυρίου πλησίον πηγής ύδατος, εν τη ερήμω, πλησίον της πηγής κατά την οδόν Σούρ·
\par 8 και είπεν, Άγαρ, δούλη της Σάρας, πόθεν έρχεσαι και που υπάγεις; Η δε είπεν, Από προσώπου Σάρας της κυρίας μου φεύγω.
\par 9 Και είπε προς αυτήν ο άγγελος του Κυρίου, Επίστρεψον προς την κυρίαν σου και ταπεινώθητι υπό τας χείρας αυτής.
\par 10 Είπεν έτι ο άγγελος του Κυρίου προς αυτήν, Θέλω πληθύνει σφόδρα το σπέρμα σου, ώστε να μη αριθμήται διά το πλήθος.
\par 11 Και είπε προς αυτήν ο άγγελος του Κυρίου, Ιδού, συ είσαι έγκυος, και θέλεις γεννήσει υιόν, και θέλεις καλέσει το όνομα αυτού Ισμαήλ· διότι ήκουσεν ο Κύριος την θλίψιν σου·
\par 12 και ούτος θέλει είσθαι άνθρωπος άγριος· η χειρ αυτού θέλει είσθαι εναντίον πάντων, και η χειρ πάντων εναντίον αυτού· και κατά πρόσωπον πάντων των αδελφών αυτού θέλει κατοικήσει.
\par 13 Και εκάλεσεν Άγαρ το όνομα του Κυρίου του λαλούντος προς αυτήν, Συ Θεός όστις με είδες· διότι είπεν, Είδον έτι εγώ ενταύθα εκείνον όστις με είδε;
\par 14 Διά τούτο ωνομάσθη το φρέαρ εκείνο, Φρέαρ Λαχαΐ-ροΐ· ιδού, κείται μεταξύ Κάδης και Βαράδ.
\par 15 Και εγέννησεν η Άγαρ υιόν εις τον Αβραμ· και ο Άβραμ εκάλεσε το όνομα του υιού αυτού, τον οποίον εγέννησεν Άγαρ, Ισμαήλ.
\par 16 Ήτο δε ο Άβραμ ογδοήκοντα εξ ετών, ότε η Άγαρ εγέννησε τον Ισμαήλ εις τον Άβραμ.

\chapter{17}

\par 1 Και ότε ήτο ο Άβραμ ενενήκοντα εννέα ετών, εφάνη ο Κύριος εις τον Άβραμ και είπε προς αυτόν, Εγώ είμαι Θεός ο Παντοκράτωρ· περιπάτει ενώπιόν μου, και έσο τέλειος.
\par 2 Και θέλω στήσει την διαθήκην μου αναμέσον εμού και σού· και θέλω σε πληθύνει σφόδρα σφόδρα.
\par 3 Και έπεσεν ο Άβραμ επί πρόσωπον αυτού· και ελάλησε προς αυτόν ο Θεός, λέγων,
\par 4 Εγώ, ιδού, η διαθήκη μου είναι προς σέ· και θέλεις γείνει πατήρ πλήθους εθνών·
\par 5 και δεν θέλει καλείσθαι πλέον το όνομά σου Άβραμ, αλλά το όνομά σου θέλει είσθαι Αβραάμ· διότι πατέρα πλήθους εθνών σε κατέστησα·
\par 6 και θέλω σε αυξήσει σφόδρα σφόδρα, και θέλω σε καταστήσει εις έθνη, και βασιλείς θέλουσιν εξέλθει εκ σού·
\par 7 και θέλω στήσει την διαθήκην μου αναμέσον εμού και σου, και του σπέρματός σου μετά σε εις τας γενεάς αυτών, εις διαθήκην αιώνιον, διά να ήμαι Θεός εις σε και εις το σπέρμα σου μετά σέ·
\par 8 και θέλω δώσει εις σε, και εις το σπέρμα σου μετά σε, την γην της παροικίας σου, πάσαν την γην Χαναάν, εις κατάσχεσιν αιώνιον· και θέλω είσθαι ο Θεός αυτών.
\par 9 Και είπεν ο Θεός προς τον Αβραάμ, συ δε την διαθήκην μου θέλεις φυλάξει, συ και το σπέρμα σου μετά σε εις τας γενεάς αυτών.
\par 10 Αύτη είναι η διαθήκη μου την οποίαν θέλετε φυλάξει, αναμέσον εμού και υμών και του σπέρματός σου μετά σέ· παν άρσεν υμών θέλει περιτέμνεσθαι.
\par 11 Και θέλετε περιτέμνει την σάρκα της ακροβυστίας υμών, και θέλει είσθαι εις σημείον της διαθήκης μεταξύ εμού και υμών·
\par 12 και παιδίον οκτώ ημερών θέλει περιτέμνεσθαι μεταξύ σας, παν άρσεν εις τας γενεάς υμών ο γεγεννημένος εν τη οικία, και ο αργυρώνητος εκ παντός ξένου, όστις δεν είναι εκ του σπέρματός σου·
\par 13 θέλει εξάπαντος περιτέμνεσθαι ο γεγεννημένος εν τη οικία σου, και ο αργυρώνητός σου· και θέλει είσθαι η διαθήκη μου επί της σαρκός υμών εις διαθήκην αιώνιον·
\par 14 και το απερίτμητον άρσεν, του οποίου δεν ήθελε περιτμηθή η σαρξ της ακροβυστίας αυτού, η ψυχή εκείνη θέλει εξολοθρευθή εκ μέσου του λαού αυτής· την διαθήκην μου παρέβη.
\par 15 Και είπεν ο Θεός προς τον Αβραάμ, Σάραν την γυναίκα σου, δεν θέλεις καλέσει πλέον το όνομα αυτής Σάραν, αλλά Σάρρα θέλει είσθαι το όνομα αυτής.
\par 16 Και θέλω ευλογήσει αυτήν, και θέλω προσέτι δώσει εις σε υιόν εξ αυτής· και θέλω ευλογήσει αυτήν, και θέλει γείνει μήτηρ εθνών· βασιλείς λαών θέλουσιν εξέλθει εξ αυτής.
\par 17 Και έπεσεν ο Αβραάμ επί πρόσωπον αυτού και εγέλασε, και είπεν εν τη καρδία αυτού, Εις άνθρωπον εκατονταετή θέλει γεννηθή τέκνον; και η Σάρρα, γυνή ενενήκοντα ετών, θέλει γεννήσει;
\par 18 Και είπεν ο Αβραάμ προς τον Θεόν, Είθε ο Ισμαήλ να ζήση ενώπιόν σου
\par 19 Και είπεν ο Θεός, Ναι η γυνή σου Σάρρα θέλει γεννήσει υιόν εις σε, και θέλεις καλέσει το όνομα αυτού Ισαάκ· και θέλω στήσει την διαθήκην μου προς αυτόν εις διαθήκην αιώνιον, και προς το σπέρμα αυτού μετ' αυτόν·
\par 20 περί δε του Ισμαήλ σου εισήκουσα· Ιδού, ευλόγησα αυτόν, και θέλω αυξήσει αυτόν, και θέλω πληθύνει αυτόν σφόδρα σφόδρα· δώδεκα άρχοντας θέλει γεννήσει, και θέλω κάμει αυτόν έθνος μέγα·
\par 21 αλλά την διαθήκην μου θέλω στήσει προς τον Ισαάκ, τον οποίον θέλει γεννήσει η Σάρρα εις σε το ερχόμενον έτος, εν τω αυτώ τούτω καιρώ.
\par 22 Αφού δε ετελείωσε να λαλή μετ' αυτού, ανέβη ο Θεός από του Αβραάμ.
\par 23 Και έλαβεν ο Αβραάμ Ισμαήλ τον υιόν αυτού, και πάντας τους γεγεννημένους εν τη οικία αυτού, και πάντας τους αργυρωνήτους αυτού, παν άρσεν των ανθρώπων της οικίας του Αβραάμ, και περιέτεμε την σάρκα της ακροβυστίας αυτών την αυτήν εκείνην ημέραν, καθώς είπε προς αυτόν ο Θεός.
\par 24 Ο δε Αβραάμ ήτο ενενήκοντα εννέα ετών, ότε περιετμήθη την σάρκα της ακροβυστίας αυτού.
\par 25 Ισμαήλ δε ο υιός αυτού ήτο δεκατριών ετών, ότε περιετμήθη την σάρκα της ακροβυστίας αυτού.
\par 26 την αυτήν εκείνην ημέραν περιετμήθη ο Αβραάμ, και Ισμαήλ ο υιός αυτού·
\par 27 και πάντες οι άνθρωποι της οικίας αυτού, οι γεγεννημένοι εν τη οικία, και οι εξ αλλογενών αργυρώνητοι περιετμήθησαν μετ' αυτού.

\chapter{18}

\par 1 Και εφάνη εις αυτόν ο Κύριος εις τας δρυς Μαμβρή, ενώ εκάθητο εν τη θύρα της σκηνής εις το καύμα της ημέρας.
\par 2 Και υψώσας τους οφθαλμούς αυτού, είδε· και ιδού, τρεις άνδρες ιστάμενοι έμπροσθεν αυτού· και ως είδεν, έδραμεν εις προϋπάντησιν αυτών από της θύρας της σκηνής, και προσεκύνησεν έως εδάφους·
\par 3 και είπε, Κύριέ μου, εάν εύρηκα χάριν εις τους οφθαλμούς σου, μη παρέλθης, παρακαλώ, τον δούλον σου·
\par 4 ας φερθή, παρακαλώ, ολίγον ύδωρ, και νίψατε τους πόδας σας, και αναπαύθητε υπό το δένδρον·
\par 5 και εγώ θέλω φέρει ολίγον άρτον, και στηρίξατε την καρδίαν σας· έπειτα θέλετε παρέλθει· επειδή διά τούτο επεράσατε προς τον δούλον σας· οι δε είπον, Κάμε ούτω, καθώς είπας.
\par 6 Και έσπευσεν ο Αβραάμ εις την σκηνήν προς την Σάρραν και είπε, Σπεύσον ζύμωσον τρία μέτρα σεμιδάλεως, και κάμε εγκρυφίας.
\par 7 Ο δε Αβραάμ έδραμεν εις τους βόας, και έλαβε μοσχάριον απαλόν και καλόν, και έδωκεν εις τον δούλον· ο δε έσπευσε να ετοιμάση αυτό·
\par 8 έπειτα έλαβε βούτυρον και γάλα και το μοσχάριον, το οποίον ητοίμασε, και έθεσεν έμπροσθεν αυτών· αυτός δε ίστατο πλησίον αυτών υπό το δένδρον, και αυτοί έφαγον.
\par 9 Είπον δε προς αυτόν, Που είναι Σάρρα η γυνή σου; Ο δε είπεν, Ιδού, εν τη σκηνή.
\par 10 Και είπεν, Εξάπαντος θέλω επιστρέψει προς σε κατά τον αυτόν τούτον καιρόν του έτους· και ιδού, Σάρρα η γυνή σου θέλει έχει υιόν. Η δε Σάρρα ήκουσεν εν τη θύρα της σκηνής ήτις ήτο όπισθεν αυτού.
\par 11 Ο δε Αβραάμ και η Σάρρα ήσαν γέροντες, προβεβηκότες εις ηλικίαν· εις την Σάρραν είχον παύσει να γίνωνται τα γυναικεία.
\par 12 Εγέλασε δε η Σάρρα καθ' εαυτήν λέγουσα, Αφού εγήρασα, θέλει γείνει εις εμέ ηδονή και ο κύριός μου γέρων;
\par 13 Και είπε Κύριος προς τον Αβραάμ, Διά τι εγέλασεν η Σάρρα, λέγουσα, Αφού εγώ εγήρασα, θέλω τωόντι γεννήσει;
\par 14 είναι τι αδύνατον εις τον Κύριον; εν τω ωρισμένω καιρώ θέλω επιστρέψει προς σε κατά τον αυτόν τούτον καιρόν του έτους, και η Σάρρα θέλει έχει υιόν.
\par 15 Τότε η Σάρρα ηρνήθη λέγουσα, δεν εγέλασα· διότι εφοβήθη. Ο δε είπεν, Ουχί, αλλ' εγέλασας.
\par 16 Σηκωθέντες δε εκείθεν οι άνδρες διευθύνθησαν προς τα Σόδομα· και ο Αβραάμ επορεύετο μετ' αυτών διά να συμπροπέμψη αυτούς.
\par 17 Και είπε Κύριος, Θέλω κρύψει εγώ από του Αβραάμ ό,τι κάμνω;
\par 18 ο δε Αβραάμ θέλει εξάπαντος γείνει έθνος μέγα και δυνατόν· και θέλουσιν ευλογηθή εις αυτόν πάντα τα έθνη της γής·
\par 19 επειδή γνωρίζω αυτόν ότι θέλει διατάξει προς τους υιούς αυτού και προς τον οίκον αυτού, μεθ' εαυτόν, και θέλουσι φυλάξει την οδόν του Κυρίου, διά να πράττωσι δικαιοσύνην και κρίσιν, ώστε να επιφέρη ο Κύριος επί τον Αβραάμ τα όσα ελάλησε προς αυτόν.
\par 20 Είπε δε Κύριος, Η κραυγή των Σοδόμων και των Γομόρρων επλήθυνε, και η αμαρτία αυτών βαρεία σφόδρα·
\par 21 θέλω λοιπόν καταβή και θέλω ιδεί αν έπραξαν ολοκλήρως κατά την κραυγήν την ερχομένην προς εμέ· και θέλω γνωρίσει, αν ουχί.
\par 22 Και αναχωρήσαντες εκείθεν οι άνδρες υπήγον προς τα Σόδομα· ο δε Αβραάμ ίστατο έτι ενώπιον του Κυρίου.
\par 23 Και πλησιάσας ο Αβραάμ είπε, Μήπως θέλεις απολέσει τον δίκαιον μετά του ασεβούς;
\par 24 εάν ήναι πεντήκοντα δίκαιοι εν τη πόλει, θέλεις άρα γε απολέσει αυτούς; και δεν ήθελες συγχωρήσει εις τον τόπον διά τους πεντήκοντα δικαίους, τους εν αυτώ;
\par 25 μη γένοιτο ποτέ συ να πράξης τοιούτον πράγμα, να θανατώσης δίκαιον μετά ασεβούς, και ο δίκαιος να ήναι ως ο ασεβής μη γένοιτο ποτέ εις σε ο κρίνων πάσαν την γην δεν θέλει κάμει κρίσιν;
\par 26 Είπε δε Κύριος, Εάν εύρω εν Σοδόμοις πεντήκοντα δικαίους εν τη πόλει, θέλω συγχωρήσει εις πάντα τον τόπον δι' αυτούς.
\par 27 Και αποκριθείς ο Αβραάμ είπεν, Ιδού, τώρα ετόλμησα να ομιλήσω προς τον Κύριόν μου, ενώ είμαι γη και σποδός·
\par 28 εάν λείψωσι πέντε εκ των πεντήκοντα δικαίων, θέλεις απολέσει πάσαν την πόλιν εξ αιτίας των πέντε; Και είπε, Δεν θέλω απολέσει αυτήν εάν εύρω εκεί τεσσαράκοντα πέντε.
\par 29 Και προσέθεσεν έτι ο Αβραάμ να λαλήση προς αυτόν, και είπεν, Εάν ευρεθώσιν εκεί τεσσαράκοντα; Και είπε, Δεν θέλω απολέσει αυτήν χάριν των τεσσαράκοντα.
\par 30 Και είπεν ο Αβραάμ, Ας μη παροξυνθή ο Κύριός μου εάν έτι λαλήσω· εάν ευρεθώσιν εκεί τριάκοντα; Και είπε, Δεν θέλω απολέσει αυτήν εάν εύρω εκεί τριάκοντα.
\par 31 Και είπεν ο Αβραάμ, Ιδού, τώρα ετόλμησα να λαλήσω προς τον Κύριόν μου· εάν ευρεθώσιν εκεί είκοσι; και είπε, Δεν θέλω απολέσει αυτήν χάριν των είκοσι.
\par 32 Και είπεν ο Αβραάμ, Ας μη παροξυνθή ο Κύριός μου, εάν λαλήσω έτι άπαξ· εάν ευρεθώσιν εκεί δέκα; και είπε, Δεν θέλω απολέσει αυτήν χάριν των δέκα.
\par 33 Και ανεχώρησεν ο Κύριος, αφού έπαυσε να λαλή προς τον Αβραάμ· και ο Αβραάμ επέστρεψεν εις τον τόπον αυτού.

\chapter{19}

\par 1 Ήλθον δε οι δύο άγγελοι εις τα Σόδομα το εσπέρας· και εκάθητο ο Λωτ παρά την πύλην των Σοδόμων· ιδών δε ο Λωτ, εσηκώθη εις συνάντησιν αυτών και προσεκύνησεν επί πρόσωπον έως εδάφους·
\par 2 και είπεν, Ιδού, κύριοί μου, εκκλίνατε, παρακαλώ, προς την οικίαν του δούλου σας, και διανυκτερεύσατε και πλύνατε τους πόδας σας· και σηκωθέντες πρωΐ, θέλετε υπάγει εις την οδόν σας· οι δε είπον, Ουχί, αλλ' εν τη πλατεία θέλομεν διανυκτερεύσει.
\par 3 Αφού δε εβίασεν αυτούς πολύ, εξέκλιναν προς αυτόν και εισήλθον εις την οικίαν αυτού· και έκαμεν εις αυτούς συμπόσιον, και έψησεν άζυμα και έφαγον.
\par 4 Πριν δε κοιμηθώσιν, οι άνδρες της πόλεως, οι άνδρες των Σοδόμων, περιεκύκλωσαν την οικίαν, νέοι και γέροντες, άπας ο λαός ομού πανταχόθεν·
\par 5 και έκραζον προς τον Λωτ και έλεγον προς αυτόν, Που είναι οι άνδρες οι εισελθόντες προς σε την νύκτα; έκβαλε αυτούς προς ημάς, διά να γνωρίσωμεν αυτούς.
\par 6 Εξήλθε δε ο Λωτ προς αυτούς εις το πρόθυρον, και έκλεισε την θύραν οπίσω αυτού,
\par 7 και είπε, Μη, αδελφοί μου, μη πράξητε τοιούτον κακόν·
\par 8 ιδού, έχω δύο θυγατέρας αίτινες δεν εγνώρισαν άνδρα· να σας φέρω λοιπόν αυτάς έξω· και κάμετε εις αυτάς, όπως φανή εις εσάς αρεστόν· μόνον εις τους άνδρας τούτους μη πράξητε μηδέν, επειδή διά τούτο εισήλθον υπό την σκιάν της στέγης μου.
\par 9 Οι δε είπον, Φύγε απ' εκεί. Και είπον, ούτος ήλθε διά να παροικήση· θέλει να γείνη και κριτής; τώρα θέλομεν καποποιήσει σε μάλλον παρά εκείνους. Και εβίαζον τον άνθρωπον τον Λωτ καθ' υπερβολήν, και επλησίασαν διά να συντρίψωσι την θύραν·
\par 10 Εκτείναντες δε οι άνδρες τας χείρας αυτών έσυραν τον Λωτ προς εαυτούς εις την οικίαν, και έκλεισαν την θύραν·
\par 11 τους δε ανθρώπους, τους όντας εις την θύραν της οικίας, εκτύπησαν με αορασίαν από μικρού έως μεγάλου, ώστε απέκαμον ζητούντες την θύραν.
\par 12 Και είπον οι άνδρες προς τον Λωτ, Έχεις εδώ άλλον τινά; γαμβρόν υιούς ή θυγατέρας ή όντινα άλλον έχεις εν τη πόλει, εξάγαγε αυτούς εκ του τόπου·
\par 13 διότι ημείς καταστρέφομεν τον τόπον τούτον, επειδή η κραυγή αυτών εμεγάλυνεν ενώπιον του Κυρίου· και απέστειλεν ημάς ο Κύριος διά να καταστρέψωμεν αυτόν.
\par 14 Εξήλθε λοιπόν ο Λωτ και ελάλησε προς τους γαμβρούς αυτού, τους μέλλοντας να λάβωσι τας θυγατέρας αυτού, και είπε, Σηκώθητε, εξέλθετε εκ του τόπου τούτου· διότι καταστρέφει ο Κύριος την πόλιν. Αλλ' εφάνη εις τους γαμβρούς αυτού ως αστεϊζόμενος.
\par 15 Και ότε έγεινεν αυγή, εβίαζον οι άγγελοι τον Λωτ, λέγοντες· Σηκώθητι, λάβε την γυναίκα σου και τας δύο σου θυγατέρας, τας ευρισκομένας εδώ, διά να μη συναπολεσθής και συ εν τη ανομία της πόλεως.
\par 16 Επειδή δε εβράδυνεν, οι άνδρες πιάσαντες την χείρα αυτού και την χείρα της γυναικός αυτού και τας χείρας των δύο θυγατέρων αυτού, διότι εσπλαγχνίσθη αυτόν ο Κύριος, εξήγαγον αυτόν και έθεσαν αυτόν έξω της πόλεως.
\par 17 Και ότε εξήγαγον αυτούς έξω, είπεν ο Κύριος, Διάσωσον την ζωήν σου· μη περιβλέψης οπίσω σου, και μη σταθής καθ' όλην την περίχωρον· διασώθητι εις το όρος, διά να μη απολεσθής.
\par 18 Και είπεν ο Λωτ προς αυτούς, Μη, παρακαλώ, Κύριε·
\par 19 ιδού, ο δούλός σου εύρηκε χάριν ενώπιόν σου, και εμεγάλυνας το έλεός σου, το οποίον έκαμες προς εμέ, φυλάττων την ζωήν μου· αλλ' εγώ δεν θέλω δυνηθή να διασωθώ εις το όρος, μήπως με προφθάση το κακόν και αποθάνω·
\par 20 ιδού, παρακαλώ, η πόλις αύτη είναι πλησίον ώστε να καταφύγω εκεί, και είναι μικρά· εκεί, παρακαλώ, να διασωθώ· δεν είναι μικρά; και θέλει ζήσει η ψυχή μου.
\par 21 Και είπε προς αυτόν ο Κύριος, Ιδού, επήκουσά σου και εις το πράγμα τούτο, να μη καταστρέψω την πόλιν, περί της οποίας ελάλησας·
\par 22 τάχυνον, διασώθητι εκεί· διότι δεν θέλω δυνηθή να κάμω ουδέν, εωσού φθάσης εκεί· διά τούτο εκάλεσε το όνομα της πόλεως Σηγώρ.
\par 23 Ο ήλιος ανέτειλεν επί την γην, ότε ο Λωτ εισήλθεν εις Σηγώρ.
\par 24 Και έβρεξεν ο Κύριος επί τα Σόδομα και Γόμορρα θείον και πυρ παρά Κυρίου εκ του ουρανού·
\par 25 και κατέστρεψε τας πόλεις ταύτας, και πάντα τα περίχωρα και πάντας τους κατοίκους των πόλεων και τα φυτά της γης.
\par 26 Αλλ' γυνή αυτού περιβλέψασα όπισθεν αυτού έγεινε στήλη άλατος.
\par 27 Ο δε Αβραάμ σηκωθείς ενωρίς το πρωΐ ήλθεν εις τον τόπον όπου είχε σταθή ενώπιον του Κυρίου·
\par 28 και βλέψας επί τα Σόδομα και Γόμορρα και εφ' όλην την γην της περιχώρου, είδε, και ιδού, ανέβαινε καπνός από της γης, ως καπνός καμίνου.
\par 29 Ούτω λοιπόν, ότε ο Θεός κατέστρεψε τας πόλεις της περιχώρου, ενεθυμήθη ο Θεός τον Αβραάμ, και εξαπέστειλε τον Λωτ εκ μέσου της καταστροφής, ότε κατέστρεψε τας πόλεις, εις τας οποίας κατώκει ο Λωτ.
\par 30 Ανέβη δε ο Λωτ από Σηγώρ και κατώκησεν εν τω όρει, και μετ' αυτού αι δύο θυγατέρες αυτού, διότι εφοβήθη να κατοικήση εν Σηγώρ· και κατώκησεν εν σπηλαίω, αυτός και αι δύο θυγατέρες αυτού.
\par 31 Και είπεν η πρεσβυτέρα προς την νεωτέραν, Ο πατήρ ημών είναι γέρων, και άνθρωπος δεν είναι επί της γης, διά να εισέλθη προς ημάς κατά την συνήθειαν πάσης της γής·
\par 32 ελθέ, ας ποτίσωμεν τον πατέρα, ημών οίνον, και ας κοιμηθώμεν μετ' αυτού, και ας αναστήσωμεν σπέρμα εκ του πατρός ημών.
\par 33 Επότισαν λοιπόν τον πατέρα αυτών οίνον κατ' εκείνην την νύκτα· και εισήλθεν η πρεσβυτέρα και εκοιμήθη μετά του πατρός αυτής· και εκείνος δεν ενόησεν ούτε πότε επλαγίασεν αυτή, και πότε εσηκώθη.
\par 34 Και την επαύριον είπεν η πρεσβυτέρα προς την νεωτέραν, Ιδού, εγώ εκοιμήθην χθές την νύκτα μετά του πατρός ημών· ας ποτίσωμεν αυτόν οίνον και την νύκτα ταύτην, και εισελθούσα συ, κοιμήθητι μετ' αυτού, και ας αναστήσωμεν σπέρμα εκ του πατρός ημών.
\par 35 Επότισαν λοιπόν και την νύκτα εκείνην τον πατέρα αυτών οίνον, και σηκωθείσα η νεωτέρα, εκοιμήθη μετ' αυτού· και εκείνος δεν ενόησεν ούτε πότε επλαγίασεν αυτή, και πότε εσηκώθη.
\par 36 Και συνέλαβον αι δύο θυγατέρες του Λωτ εκ του πατρός αυτών.
\par 37 Και εγέννησεν η πρεσβυτέρα υιόν και εκάλεσε το όνομα αυτού Μωάβ· ούτος είναι ο πατήρ των Μωαβιτών έως της σήμερον.
\par 38 Εγέννησε δε και η νεωτέρα υιόν και εκάλεσε το όνομα αυτού Βεν-αμμί· ούτος είναι ο πατήρ των Αμμωνιτών έως της σήμερον.

\chapter{20}

\par 1 Και εκίνησεν εκείθεν ο Αβραάμ εις την γην την προς μεσημβρίαν, και κατώκησε μεταξύ Κάδης και Σούρ· και παρώκησεν εν Γεράροις.
\par 2 Και είπεν ο Αβραάμ περί Σάρρας της γυναικός αυτού, Αδελφή μου είναι. Έστειλε δε Αβιμέλεχ ο βασιλεύς των Γεράρων, και έλαβε την Σάρραν.
\par 3 Και ήλθεν ο Θεός προς τον Αβιμέλεχ κατ' όναρ την νύκτα, και είπε προς αυτόν, Ιδού, συ αποθνήσκεις εξ αιτίας της γυναικός, την οποίαν έλαβες· διότι είναι νενυμφευμένη με άνδρα.
\par 4 Ο δε Αβιμέλεχ δεν είχε πλησιάσει εις αυτήν· και είπε, Κύριε, ήθελες φονεύσει έθνος έτι και δίκαιον;
\par 5 δεν μοι είπεν αυτός, Αδελφή μου είναι; και αυτή πάλιν, αυτή είπεν, Αδελφός μου είναι. Εν ευθύτητι της καρδίας μου και εν καθαρότητι των χειρών μου έπραξα τούτο.
\par 6 Είπε δε προς αυτόν ο Θεός κατ' όναρ, Και εγώ εγνώρισα ότι εν ευθύτητι της καρδίας σου έπραξας τούτο· όθεν και εγώ σε εμπόδισα από του να αμαρτήσης εις εμέ· διά τούτο δεν σε αφήκα να εγγίσης αυτήν·
\par 7 τώρα λοιπόν απόδος την γυναίκα προς τον άνθρωπον, διότι είναι προφήτης· και θέλει προσευχηθή υπέρ σου και θέλεις ζήσει· αλλ' εάν δεν αποδώσης αυτήν, έξευρε ότι εξάπαντος θέλεις αποθάνει, συ και πάντα όσα έχεις.
\par 8 Σηκωθείς δε ο Αβιμέλεχ ενωρίς το πρωΐ, εκάλεσε πάντας τους δούλους αυτού, και ελάλησε πάντας τους λόγους τούτους εις επήκοον αυτών· και εφοβήθησαν οι άνθρωποι σφόδρα.
\par 9 Και εκάλεσεν ο Αβιμέλεχ τον Αβραάμ και είπε προς αυτόν, Τι έπραξας εις ημάς; και τι αμάρτημα έπραξα εις σε, ώστε επέφερες επ' εμέ και επί το βασίλειόν μου, αμαρτίαν μεγάλην; έπραξας εις εμέ πράγμα, το οποίον δεν έπρεπε να πραχθή.
\par 10 Και είπεν ο Αβιμέλεχ προς τον Αβραάμ, Τι είδες, ώστε να πράξης το πράγμα τούτο;
\par 11 Και είπεν ο Αβραάμ, Επειδή εγώ είπον, Βέβαια δεν είναι φόβος Θεού εν τω τόπω τούτω και θέλουσι με φονεύσει διά την γυναίκα μου·
\par 12 και όμως αληθώς αδελφή μου είναι, θυγάτηρ του πατρός μου, αλλ' ουχί θυγάτηρ της μητρός μου· και έγεινε γυνή μου.
\par 13 και ότε με έκαμεν ο Θεός να εξέλθω εκ του οίκου του πατρός μου, είπον προς αυτήν, Ταύτην την χάριν θέλεις κάμει εις εμέ· εις πάντα τόπον όπου αν υπάγωμεν, λέγε περί εμού, Ούτος είναι αδελφός μου.
\par 14 Και έλαβεν ο Αβιμέλεχ πρόβατα και βόας και δούλους και δούλας, και έδωκεν εις τον Αβραάμ, και απέδωκεν εις αυτόν Σάρραν την γυναίκα αυτού.
\par 15 και είπεν ο Αβιμέλεχ, Ιδού, η γη μου έμπροσθέν σου. κατοίκησον όπου σοι αρέσκει.
\par 16 Και προς την Σάρραν είπεν, Ιδού, έδωκα χίλια αργύρια εις τον αδελφόν σου· ιδού, αυτός είναι εις σε σκέπη των οφθαλμών προς πάντας τους μετά σου και προς πάντας τους άλλους. Ούτως αύτη επεπλήχθη.
\par 17 Προσηυχήθη δε ο Αβραάμ προς τον Θεόν· και εθεράπευσεν ο Θεός τον Αβιμέλεχ και την γυναίκα αυτού και τας θεραπαίνας αυτού, και ετεκνοποίησαν.
\par 18 διότι ο Κύριος είχε κλείσει διόλου πάσαν μήτραν εν τη οικία του Αβιμέλεχ, εξ αιτίας Σάρρας της γυναικός του Αβραάμ.

\chapter{21}

\par 1 Και επεσκέφθη ο Κύριος την Σάρραν, ως είπε· και έκαμεν ο Κύριος εις την Σάρραν, ως ελάλησε.
\par 2 Και συνέλαβεν η Σάρρα, και εγέννησεν εις τον Αβραάμ υιόν εν τω γήρατι αυτού· κατά τον καιρόν, τον οποίον είπε προς αυτόν ο Θεός.
\par 3 Και εκάλεσεν ο Αβραάμ το όνομα του υιού αυτού, του γεννηθέντος εις αυτόν, τον οποίον η Σάρρα εγέννησεν εις αυτόν, Ισαάκ.
\par 4 Περιέτεμε δε ο Αβραάμ τον υιόν αυτού Ισαάκ την ογδόην ημέραν, ως προσέταξεν εις αυτόν ο Θεός.
\par 5 Ήτο δε ο Αβραάμ εκατόν ετών, ότε εγεννήθη εις αυτόν Ισαάκ ο υιός αυτού.
\par 6 Και είπεν η Σάρρα, Ο Θεός με έκαμε να γελώ· όστις ακούση, θέλει γελά μετ' εμού.
\par 7 Και είπε, Τις ήθελεν ειπεί προς τον Αβραάμ, ότι ήθελε θηλάσει τέκνα η Σάρρα; επειδή εγέννησα υιόν εν τω γήρατι αυτού.
\par 8 Το δε παιδίον ηύξησε και απεγαλακτίσθη· και έκαμεν ο Αβραάμ μέγα συμπόσιον, καθ' ην ημέραν απεγαλακτίσθη ο Ισαάκ.
\par 9 Και είδεν η Σάρρα τον υιόν της Άγαρ της Αιγυπτίας, τον οποίον εγέννησεν εις τον Αβραάμ, περιγελώντα τον Ισαάκ.
\par 10 Και είπε προς τον Αβραάμ, Δίωξον την δούλην ταύτην και τον υιόν αυτής· διότι δεν θέλει κληρονομήσει ο υιός της δούλης ταύτης μετά του υιού μου, του Ισαάκ.
\par 11 Εφάνη δε σκληρόν σφόδρα το πράγμα εις τους οφθαλμούς του Αβραάμ περί του υιού αυτού.
\par 12 Και είπεν ο Θεός προς τον Αβραάμ, Ας μη φανή σκληρόν εις τους οφθαλμούς σου περί του παιδίου και περί της δούλης σου· κατά πάντα όσα είπη προς σε η Σάρρα, άκουε τους λόγους αυτής· διότι εν τω Ισαάκ θέλει κληθή εις σε σπέρμα·
\par 13 και τον υιόν δε της δούλης εις έθνος θέλω καταστήσει αυτόν· διότι είναι σπέρμα σου.
\par 14 Σηκωθείς δε ο Αβραάμ ενωρίς το πρωΐ, έλαβεν άρτους και ασκόν ύδατος και έδωκεν εις την Άγαρ, επιθέσας αυτά επί τον ώμον αυτής, και το παιδίον, και απέπεμψεν αυτήν. Η δε αναχωρήσασα περιεπλανάτο εν τη ερήμω Βηρ-σαβεέ.
\par 15 Και αφού ετελείωσε το ύδωρ από του ασκού, έρριψε το παιδίον υποκάτω ενός θάμνου·
\par 16 και ελθούσα εκάθισεν απέναντι, μακράν έως τόξου βολής· διότι είπε, να μη ίδω τον θάνατον του παιδίου. Και εκάθισεν απέναντι και ύψωσε την φωνήν αυτής και έκλαυσεν.
\par 17 Εισήκουσε δε ο Θεός την φωνήν του παιδίου· και εφώνησεν άγγελος Θεού προς την Άγαρ εκ του ουρανού, και είπε προς αυτήν, Τι έχεις, Άγαρ; μη φοβού· διότι ήκουσεν ο Θεός την φωνήν του παιδίου εκ του τόπου ένθα κείται·
\par 18 σηκώθητι, λάβε το παιδίον, και κράτει αυτό με την χείρα σου· διότι θέλω καταστήσει αυτό εις έθνος μέγα.
\par 19 Και ήνοιξεν ο Θεός τους οφθαλμούς αυτής, και ιδούσα φρέαρ ύδατος υπήγε και εγέμισε τον ασκόν ύδωρ και επότισε το παιδίον.
\par 20 Και ήτο ο Θεός μετά του παιδίου, και ηύξησε, και κατώκησεν εν τη ερήμω και έγεινε τοξότης.
\par 21 Και κατώκησεν εν τη ερήμω Φαράν· και η μήτηρ αυτού έλαβεν εις αυτόν γυναίκα εκ γης Αιγύπτου.
\par 22 Κατ' εκείνον δε τον καιρόν ο Αβιμέλεχ, μετά του Φιχόλ αρχιστρατήγου της δυνάμεως αυτού, είπε προς τον Αβραάμ, λέγων, Ο Θεός είναι μετά σου εις πάντα όσα πράττεις·
\par 23 τώρα λοιπόν όμοσον προς εμέ εδώ εις τον Θεόν, ότι δεν θέλεις ψευσθή προς εμέ, ούτε προς τον υιόν μου, ούτε προς τους εγγόνους μου· αλλά κατά το έλεος, το οποίον έκαμα εις σε, θέλεις κάμει εις εμέ, και εις την γην όπου παρώκησας.
\par 24 Και είπεν ο Αβραάμ, Εγώ θέλω ομόσει.
\par 25 Και έλεγξεν ο Αβραάμ τον Αβιμέλεχ διά το φρέαρ του ύδατος, το οποίον αφήρπασαν οι δούλοι του Αβιμέλεχ.
\par 26 Και είπεν ο Αβιμέλεχ, Δεν εξεύρω τις έπραξε το πράγμα τούτο· και ούτε συ με εφανέρωσας και ούτε εγώ ήκουσα, ειμή σήμερον.
\par 27 Και λαβών ο Αβραάμ πρόβατα και βόας, έδωκεν εις τον Αβιμέλεχ· και έκαμον αμφότεροι συνθήκην.
\par 28 Και έβαλεν ο Αβραάμ κατά μέρος επτά θηλυκά αρνία του ποιμνίου.
\par 29 Και είπεν ο Αβιμέλεχ προς τον Αβραάμ, Τι είναι ταύτα τα επτά θηλυκά αρνία, τα οποία έβαλες κατά μέρος;
\par 30 Ο δε είπεν, Ότι ταύτα τα επτά θηλυκά αρνία θέλεις λάβει εκ της χειρός μου, διά να ήναι εις εμέ εις μαρτύριον ότι εγώ έσκαψα το φρέαρ τούτο.
\par 31 διά τούτο ωνόμασε τον τόπον εκείνον, Βηρ-σαβεέ· διότι εκεί ώμοσαν αμφότεροι.
\par 32 Και έκαμον συνθήκην εν Βηρ-σαβεέ. Εσηκώθη δε ο Αβιμέλεχ και Φιχόλ ο αρχιστράτηγος της δυνάμεως αυτού, και επέστρεψαν εις την γην των Φιλισταίων.
\par 33 Και εφύτευσεν ο Αβραάμ δρυμόν εν Βηρ-σαβεέ· και επεκαλέσθη εκεί το όνομα του Κυρίου, του αιωνίου Θεού.
\par 34 Παρώκησε δε ο Αβραάμ εν τη γη των Φιλισταίων ημέρας πολλάς.

\chapter{22}

\par 1 Μετά δε τα πράγματα ταύτα ο Θεός εδοκίμασε τον Αβραάμ, και είπε προς αυτόν, Αβραάμ· ο δε είπεν, Ιδού, εγώ.
\par 2 Και είπε, Λάβε τώρα τον υιόν σου τον μονογενή, τον οποίον ηγάπησας, τον Ισαάκ, και ύπαγε εις τον τόπον Μοριά, και πρόσφερε αυτόν εκεί εις ολοκαύτωμα, επί ενός των ορέων, το οποίον θέλω σοι ειπεί.
\par 3 Σηκωθείς δε Αβραάμ ενωρίς το πρωΐ, εσαμάρωσε την όνον αυτού και έλαβε μεθ' εαυτού δύο εκ των δούλων αυτού και Ισαάκ τον υιόν αυτού· και σχίσας ξύλα διά την ολοκαύτωσιν, εσηκώθη και υπήγεν εις τον τόπον τον οποίον είπε προς αυτόν ο Θεός.
\par 4 Την δε τρίτην ημέραν υψώσας ο Αβραάμ τους οφθαλμούς αυτού, είδε τον τόπον μακρόθεν.
\par 5 Και είπεν ο Αβραάμ προς τους δούλους αυτού, Σεις καθίσατε αυτού μετά της όνου· εγώ δε και το παιδάριον θέλομεν υπάγει έως εκεί· και αφού προσκυνήσωμεν, θέλομεν επιστρέψει προς εσάς.
\par 6 Και λαβών ο Αβραάμ τα ξύλα της ολοκαυτώσεως, επέθεσεν επί τον Ισαάκ τον υιόν αυτού· και έλαβεν εις την χείρα αυτού το πυρ, και την μάχαιραν, και υπήγον οι δύο ομού.
\par 7 Τότε ελάλησεν ο Ισαάκ προς Αβραάμ τον πατέρα αυτού και είπε, Πάτερ μου. Ο δε είπεν, Ιδού, εγώ, τέκνον μου. Και είπεν ο Ισαάκ, Ιδού, το πυρ και τα ξύλα· αλλά που το πρόβατον διά την ολοκαύτωσιν;
\par 8 Και είπεν ο Αβραάμ, Ο Θεός, τέκνον μου, θέλει προβλέψει εις εαυτόν το πρόβατον διά την ολοκαύτωσιν. Και επορεύοντο οι δύο ομού.
\par 9 Αφού δε έφθασαν εις τον τόπον τον οποίον είπε προς αυτόν ο Θεός, ωκοδόμησεν εκεί ο Αβραάμ το θυσιαστήριον και διέθεσε τα ξύλα, και δέσας τον Ισαάκ τον υιόν αυτού έβαλεν αυτόν επί το θυσιαστήριον επάνω των ξύλων·
\par 10 και εκτείνας ο Αβραάμ την χείρα αυτού, έλαβε την μάχαιραν διά να σφάξη τον υιόν αυτού.
\par 11 Άγγελος δε Κυρίου εφώνησε προς αυτόν εκ του ουρανού και είπεν, Αβραάμ, Αβραάμ. Ο δε είπεν, Ιδού, εγώ.
\par 12 Και είπε, Μη επιβάλης την χείρα σου επί το παιδάριον, και μη πράξης εις αυτό μηδέν· διότι τώρα εγνώρισα ότι συ φοβείσαι τον Θεόν, επειδή δεν ελυπήθης τον υιόν σου τον μονογενή δι' εμέ.
\par 13 Και υψώσας ο Αβραάμ τους οφθαλμούς αυτού είδε· και ιδού, κριός όπισθεν αυτού, κρατούμενος από των κεράτων αυτού εις φυτόν πυκνόκλαδον· και ελθών ο Αβραάμ, έλαβε τον κριόν και προσέφερεν αυτόν εις ολοκαύτωμα αντί του υιού αυτού.
\par 14 Και εκάλεσεν ο Αβραάμ το όνομα του τόπου εκείνου Ιεοβά-ιρέ· ως λέγεται και την σήμερον, Εν τω όρει ο Κύριος θέλει εμφανισθή.
\par 15 Και εφώνησε δεύτερον ο άγγελος του Κυρίου προς τον Αβραάμ εκ του ουρανού,
\par 16 και είπεν, Ώμοσα εις εμαυτόν, λέγει Κύριος, ότι, επειδή έπραξας το πράγμα τούτο και δεν ελυπήθης τον υιόν σου, τον μονογενή σου,
\par 17 ότι ευλογών θέλω σε ευλογήσει, και πληθύνων θέλω πληθύνει το σπέρμα σου ως τα άστρα του ουρανού και ως την άμμον την παρά το χείλος της θαλάσσης· και το σπέρμα σου θέλει κυριεύσει τας πύλας των εχθρών αυτού·
\par 18 και εν τω σπέρματί σου θέλουσιν ευλογηθή πάντα τα έθνη της γής· διότι υπήκουσας εις την φωνήν μου.
\par 19 Και επέστρεψεν ο Αβραάμ προς τους δούλους αυτού· και σηκωθέντες, υπήγον ομού εις Βηρ-σαβεέ· και κατώκησεν ο Αβραάμ εν Βηρ-σαβεέ.
\par 20 Μετά δε τα πράγματα ταύτα, ανήγγειλαν προς τον Αβραάμ λέγοντες, Ιδού, η Μελχά εγέννησε και αυτή υιούς εις τον Ναχώρ τον αδελφόν σου·
\par 21 τον Ουζ πρωτότοκον αυτού, και τον Βουζ αδελφόν αυτού, και τον Κεμουήλ τον πατέρα του Αράμ,
\par 22 και τον Κεσέδ, και τον Αζαύ, και τον Φαλδές, και τον Ιελδάφ, και τον Βαθουήλ.
\par 23 Ο δε Βαθουήλ εγέννησε την Ρεβέκκαν· τους οκτώ τούτους εγέννησεν η Μελχά εις τον Ναχώρ τον αδελφόν του Αβραάμ.
\par 24 Και η παλλακή αυτού, η ονομαζομένη Ρευμά, εγέννησε και αυτή τον Ταβέκ και τον Γαάμ και τον Ταχάς και τον Μααχά.

\chapter{23}

\par 1 Και έζησεν η Σάρρα εκατόν εικοσιεπτά έτη· ταύτα είναι τα έτη της ζωής της Σάρρας.
\par 2 Και απέθανεν η Σάρρα εν Κιριάθ-αρβά· αύτη είναι η Χεβρών εν γη Χαναάν· και ήλθεν ο Αβραάμ διά να κλαύση την Σάρραν και να πενθήση αυτήν.
\par 3 Και σηκωθείς ο Αβραάμ απ' έμπροσθεν του νεκρού αυτού, ελάλησε προς τους υιούς του Χετ λέγων,
\par 4 ξένος και πάροικος είμαι εγώ μεταξύ σας· δότε μοι κτήμα τάφου μεταξύ σας, διά να θάψω τον νεκρόν μου απ' έμπροσθέν μου.
\par 5 Απεκρίθησαν δε οι υιοί του Χετ προς τον Αβραάμ λέγοντες προς αυτόν,
\par 6 Άκουσον ημάς, κύριέ μου· συ είσαι μεταξύ ημών ηγεμών εκ Θεού· θάψον τον νεκρόν σου εις το εκλεκτότερον εκ των μνημείων ημών· ουδείς εξ ημών θέλει αρνηθή το μνημείον αυτού προς σε, διά να θάψης τον νεκρόν σου.
\par 7 Τότε σηκωθείς ο Αβραάμ προσεκύνησε προς τον λαόν του τόπου, προς τους υιούς του Χέτ·
\par 8 και ελάλησε προς αυτούς λέγων, Εάν ευαρεστήται η ψυχή σας να θάψω τον νεκρόν μου απ' έμπροσθέν μου, ακούσατέ μου και μεσιτεύσατε υπέρ εμού προς τον Εφρών τον υιόν του Σωάρ,
\par 9 και ας μοι δώση το σπήλαιον αυτού Μαχπελάχ, το εν τη άκρα του αγρού αυτού· εις πλήρη τιμήν ας μοι δώση αυτό, διά κτήμα τάφου μεταξύ σας.
\par 10 Ο δε Εφρών εκάθητο εν τω μέσω των υιών του Χέτ· και απεκρίθη ο Εφρών ο Χετταίος προς τον Αβραάμ εις επήκοον των υιών του Χετ, πάντων των εισερχομένων εις την πύλην της πόλεως αυτού, λέγων,
\par 11 Ουχί, κύριέ μου, άκουσόν μου· σοι δίδω τον αγρόν, σοι δίδω και το σπήλαιον το εν αυτώ· επί παρουσία των υιών του λαού μου δίδω αυτά εις σέ· θάψον τον νεκρόν σου.
\par 12 Και προσεκύνησεν ο Αβραάμ έμπροσθεν του λαού του τόπου·
\par 13 και είπε προς τον Εφρών εις επήκοον του λαού του τόπου λέγων, Εάν συ θέλης, άκουσόν μου, παρακαλώ· θέλω δώσει το αργύριον του αγρού· λάβε αυτό παρ' εμού, και θέλω θάψει τον νεκρόν μου εκεί.
\par 14 Ο δε Εφρών απεκρίθη προς τον Αβραάμ, λέγων προς αυτόν,
\par 15 Ακουσόν μου, κύριέ μου· γη τετρακοσίων σίκλων αργυρίου, τι είναι μεταξύ εμού και σου; θάψον λοιπόν τον νεκρόν σου.
\par 16 Και ήκουσεν ο Αβραάμ τον Εφρών· και εζύγισεν ο Αβραάμ εις τον Εφρών το αργύριον, το οποίον είπεν εις επήκοον των υιών του Χετ τετρακοσίους σίκλους αργυρίου, δεκτού μεταξύ εμπόρων.
\par 17 Και ο αγρός του Εφρών, όστις ήτο εν Μαχπελάχ, έμπροσθεν της Μαμβρή, ο αγρός και το σπήλαιον το εν αυτώ και πάντα τα δένδρα τα εν τω αγρώ και εν πάσι τοις ορίοις κύκλω, ησφαλίσθησαν
\par 18 εις τον Αβραάμ διά κτήμα, ενώπιον των υιών του Χετ, ενώπιον πάντων των εισερχομένων εις την πύλην της πόλεως αυτού.
\par 19 Και μετά ταύτα έθαψεν ο Αβραάμ Σάρραν την γυναίκα αυτού εν τω σπηλαίω του αγρού Μαχπελάχ, έμπροσθεν της Μαμβρή· αύτη είναι Χεβρών εν γη Χαναάν.
\par 20 Και ο αγρός και το σπήλαιον το εν αυτώ, ησφαλίσθησαν εις τον Αβραάμ διά κτήμα τάφον παρά των υιών του Χετ.

\chapter{24}

\par 1 Ήτο δε ο Αβραάμ γέρων προβεβηκώς την ηλικίαν· και ο Κύριος ευλόγησε τον Αβραάμ κατά πάντα.
\par 2 Και είπεν ο Αβραάμ προς τον δούλον αυτού τον πρεσβύτερον της οικίας αυτού, τον επιστάτην πάντων των υπαρχόντων αυτού, Βάλε, παρακαλώ, την χείρα σου υπό τον μηρόν μου·
\par 3 και θέλω σε ορκίσει εις Κύριον τον Θεόν του ουρανού και τον Θεόν της γης, ότι δεν θέλεις λάβει γυναίκα εις τον υιόν μου εκ των θυγατέρων των Χαναναίων, μεταξύ των οποίων εγώ κατοικώ·
\par 4 αλλ' εις τον τόπον μου, και εις την συγγένειάν μου θέλεις υπάγει, και θέλεις λάβει γυναίκα εις τον υιόν μου τον Ισαάκ.
\par 5 Είπε δε προς αυτόν ο δούλος, Ίσως δεν θελήση η γυνή να μοι ακολουθήση εις την γην ταύτην· πρέπει να φέρω τον υιόν σου εις την γην εκ της οποίας εξήλθες;
\par 6 Και είπε προς αυτόν ο Αβραάμ, Πρόσεχε, μη φέρης τον υιόν μου εκεί·
\par 7 Κύριος ο Θεός του ουρανού, όστις με έλαβεν εκ του οίκου του πατρός μου και εκ της γης της γεννήσεώς μου, και όστις ελάλησε προς εμέ και όστις ώμοσεν εις εμέ λέγων, εις το σπέρμα σου θέλω δώσει την γην ταύτην, αυτός θέλει αποστείλει τον άγγελον αυτού έμπροσθέν σου· και θέλεις λάβει γυναίκα εις τον υιόν μου εκείθεν·
\par 8 εάν δε η γυνή δεν θέλη να σε ακολουθήση, τότε θέλεις είσθαι ελεύθερος από του όρκου μου τούτου· μόνον τον υιόν μου να μη φέρης εκεί.
\par 9 Και έβαλεν ο δούλος την χείρα αυτού υπό τον μηρόν του Αβραάμ του κυρίου αυτού, και ώρκίσθη εις αυτόν περί του πράγματος τούτου.
\par 10 Και έλαβεν ο δούλος δέκα καμήλους εκ των καμήλων του κυρίου αυτού και ανεχώρησε, φέρων μεθ' εαυτού από πάντων των αγαθών του κυρίου αυτού· και σηκωθείς, υπήγεν εις την Μεσοποταμίαν, εις την πόλιν του Ναχώρ.
\par 11 Και εγονάτισε τας καμήλους έξω της πόλεως παρά το φρέαρ του ύδατος, προς το εσπέρας, ότε εξέρχονται αι γυναίκες διά να αντλήσωσιν ύδωρ.
\par 12 Και είπε, Κύριε Θεέ του κυρίου μου Αβραάμ, δος μοι, δέομαι, καλόν συνάντημα σήμερον, και κάμε έλεος εις τον κύριόν μου Αβραάμ·
\par 13 ιδού, εγώ ίσταμαι πλησίον της πηγής του ύδατος· αι δε θυγατέρες των κατοίκων της πόλεως εξέρχονται διά να αντλήσωσιν ύδωρ·
\par 14 και η κόρη προς την οποίαν είπω, Επίκλινον, παρακαλώ, την υδρίαν σου διά να πίω, και αυτή είπη, Πίε και θέλω ποτίσει και τας καμήλους σου, αύτη ας ήναι εκείνη, την οποίαν ητοίμασας εις τον δούλον σου τον Ισαάκ· και εκ τούτου θέλω γνωρίσει ότι έκαμες έλεος εις τον κύριόν μου.
\par 15 Και πριν αυτός παύση λαλών, ιδού, εξήρχετο η Ρεβέκκα, ήτις εγεννήθη εις τον Βαθουήλ, υιόν της Μελχάς, γυναικός του Ναχώρ, αδελφού του Αβραάμ, έχουσα την υδρίαν αυτής επί του ώμου αυτής.
\par 16 Η δε κόρη ήτο ώραία την όψιν σφόδρα, παρθένος, και ανήρ δεν είχε γνωρίσει αυτήν· αφού λοιπόν κατέβη εις την πηγήν, εγέμισε την υδρίαν αυτής και ανέβαινε.
\par 17 Δραμών δε ο δούλος εις συνάντησιν αυτής είπε, Πότισόν με, παρακαλώ, ολίγον ύδωρ εκ της υδρίας σου.
\par 18 Η δε είπε, Πίε, κύριέ μου. και έσπευσε και κατεβίβασε την υδρίαν αυτής επί τον βραχίονα αυτής, και επότισεν αυτόν.
\par 19 και αφού έπαυσε ποτίζουσα αυτόν είπε, Και διά τας καμήλους σου θέλω αντλήσει, εωσού πίωσι πάσαι.
\par 20 Και παρευθύς εξεκένωσε την υδρίαν αυτής εις την ποτίστραν, και έδραμεν έτι εις το φρέαρ διά να αντλήση, και ήντλησε διά πάσας τας καμήλους αυτού.
\par 21 Ο δε άνθρωπος, θαυμάζων δι' αυτήν, εσιώπα, διά να γνωρίση αν κατευώδωσεν ο Κύριος την οδόν αυτού ή ουχί.
\par 22 Και αφού έπαυσαν αι κάμηλοι πίνουσαι, έλαβεν ο άνθρωπος ενώτια χρυσά βάρους ημίσεος σίκλου, και δύο βραχιόλια διά τας χείρας αυτής, βάρους δέκα σίκλων χρυσίου·
\par 23 και είπε, Τίνος θυγάτηρ είσαι συ; ειπέ μοι, παρακαλώ· είναι εν τη οικία του πατρός σου τόπος δι' ημάς προς κατάλυμα;
\par 24 Η δε είπε προς αυτόν· είμαι θυγάτηρ Βαθουήλ του υιού της Μελχάς, τον οποίον εγέννησεν εις τον Ναχώρ.
\par 25 είπεν έτι προς αυτόν, Είναι εις ημάς και άχυρα και τροφή πολλή και τόπος προς κατάλυμα.
\par 26 Τότε έκλινεν ο άνθρωπος και προσεκύνησε τον Κύριον·
\par 27 και είπεν, Ευλογητός Κύριος ο Θεός του κυρίου μου Αβραάμ, όστις δεν εγκατέλιπε το έλεος αυτού και την αλήθειαν αυτού από του κυρίου μου· ο Κύριος με κατευώδωσεν εις τον οίκον των αδελφών του κυρίου μου.
\par 28 Δραμούσα δε η κόρη ανήγγειλεν εις τον οίκον της μητρός αυτής τα πράγματα ταύτα.
\par 29 Είχε δε η Ρεβέκκα αδελφόν ονομαζόμενον Λάβαν· και έδραμεν ο Λάβαν προς τον άνθρωπον έξω εις την πηγήν.
\par 30 Και ως είδε τα ενώτια και τα βραχιόλια εις τας χείρας της αδελφής αυτού, και ως ήκουσε τους λόγους Ρεβέκκας της αδελφής αυτού, λεγούσης, Ούτως ελάλησε προς εμέ ο άνθρωπος, ήλθε προς τον άνθρωπον· και ιδού, ίστατο πλησίον των καμήλων επί της πηγής.
\par 31 Και είπεν, Είσελθε, ευλογημένε του Κυρίου· διά τι ίστασαι έξω; επειδή εγώ ητοίμασα την οικίαν και τόπον διά τας καμήλους.
\par 32 Και εισήλθεν ο άνθρωπος εις την οικίαν, και εκείνος εξεφόρτωσε τας καμήλους και έδωκεν άχυρα και τροφήν εις τας καμήλους και ύδωρ διά νίψιμον των ποδών αυτού και των ποδών των ανθρώπων των μετ' αυτού.
\par 33 Και παρετέθη έμπροσθεν αυτού φαγητόν· αυτός όμως είπε, Δεν θέλω φάγει, εωσού λαλήσω τον λόγον μου. Ο δε είπε, Λάλησον.
\par 34 Και είπεν, Εγώ είμαι δούλος του Αβραάμ.
\par 35 Και ο Κύριος ευλόγησε τον κύριόν μου σφόδρα, και έγεινε μέγας· και έδωκεν εις αυτόν πρόβατα και βόας και αργύριον και χρυσίον και δούλους και δούλας και καμήλους και όνους.
\par 36 Και εγέννησε Σάρρα, η γυνή του κυρίου μου, υιόν εις τον κύριόν μου, αφού εγήρασε· και έδωκεν εις αυτόν πάντα όσα έχει.
\par 37 Και με ώρκισεν ο κύριός μου, λέγων, Δεν θέλεις λάβει γυναίκα εις τον υιόν μου εκ των θυγατέρων των Χαναναίων, εις την γην των οποίων εγώ κατοικώ·
\par 38 αλλ' εις τον οίκον του πατρός μου θέλεις υπάγει και εις την συγγένειάν μου, και θέλεις λάβει γυναίκα εις τον υιόν μου.
\par 39 Και είπον προς τον κύριόν μου, Ίσως δεν θελήση η γυνή να με ακολουθήση.
\par 40 Ο δε είπε προς εμέ, Ο Κύριος, έμπροσθεν του οποίου περιεπάτησα, θέλει αποστείλει τον άγγελον αυτού μετά σου και θέλει κατευοδώσει την οδόν σου· και θέλεις λάβει γυναίκα εις τον υιόν μου εκ της συγγενείας μου και εκ του οίκου του πατρός μου·
\par 41 τότε θέλεις είσθαι ελεύθερος από του ορκισμού μου· όταν υπάγης προς την συγγένειάν μου και δεν δώσωσιν εις σε, τότε θέλεις είσθαι ελεύθερος από του ορκισμού μου.
\par 42 Και ελθών σήμερον εις την πηγήν, είπον, Κύριε ο Θεός του κυρίου μου Αβραάμ, κατευόδωσον, δέομαι, την οδόν μου, εις την οποίαν εγώ υπάγω·
\par 43 ιδού, εγώ ίσταμαι πλησίον της πηγής του ύδατος· και η κόρη ήτις εξέρχεται διά να αντλήση και προς την οποίαν είπω, Πότισόν με, παρακαλώ, ολίγον ύδωρ εκ της υδρίας σου,
\par 44 και αυτή με είπη, Και συ πίε, και διά τας καμήλους σου ακόμη θέλω αντλήσει, αύτη ας ήναι η γυνή, την οποίαν ητοίμασεν ο Κύριος διά τον υιόν του κυρίου μου.
\par 45 Και πριν παύσω λαλών εν τη καρδία μου, ιδού, η Ρεβέκκα εξήρχετο έχουσα την υδρίαν αυτής επί του ώμου αυτής· και κατέβη εις την πηγήν και ήντλησεν· είπον δε προς αυτήν, Πότισόν με, παρακαλώ.
\par 46 Η δε έσπευσε και κατεβίβασε την υδρίαν αυτής επάνωθεν αυτής και είπε, Πίε, και θέλω ποτίσει και τας καμήλους σου· έπιον λοιπόν και επότισε και τας καμήλους.
\par 47 Και ηρώτησα αυτήν και είπον, Τίνος θυγάτηρ είσαι; η δε είπε, Θυγάτηρ του Βαθουήλ, υιού του Ναχώρ, τον οποίον εγέννησεν εις αυτόν η Μελχά· και περιέθεσα τα ενώτια εις το πρόσωπον αυτής και τα βραχιόλια επί τας χείρας αυτής.
\par 48 Και κλίνας προσεκύνησα τον Κύριον· και ευλόγησα Κύριον τον Θεόν του κυρίον μου Αβραάμ, όστις με κατευώδωσεν εις την αληθινήν οδόν, διά να λάβω την θυγατέρα του αδελφού του κυρίου μου εις τον υιόν αυτού.
\par 49 Τώρα λοιπόν, εάν θέλητε να κάμητε έλεος και αλήθειαν προς τον κύριόν μου, είπατέ μοι, ει δε μη, είπατέ μοι, διά να στραφώ δεξιά ή αριστερά.
\par 50 Και αποκριθέντες ο Λάβαν και ο Βαθουήλ, είπον, Παρά Κυρίου εξήλθε το πράγμα· ημείς δεν δυνάμεθα να σοι είπωμεν κακόν ή καλόν·
\par 51 ιδού, η Ρεβέκκα έμπροσθέν σου· λάβε αυτήν και ύπαγε· και ας ήναι γυνή του υιού του κυρίου σου, καθώς ελάλησεν ο Κύριος.
\par 52 Και ότε ήκουσεν ο δούλος του Αβραάμ τους λόγους αυτών, προσεκύνησεν έως εδάφους τον Κύριον.
\par 53 Και εκβαλών ο δούλος σκεύη αργυρά και σκεύη χρυσά και ενδύματα, έδωκεν εις την Ρεβέκκαν· έδωκεν έτι δώρα εις τον αδελφόν αυτής και εις την μητέρα αυτής.
\par 54 Και έφαγον και έπιον, αυτός και οι άνθρωποι οι μετ' αυτού, και διενυκτέρευσαν· και αφού εσηκώθησαν το πρωΐ, είπεν, Εξαποστείλατέ με προς τον κύριόν μου.
\par 55 Είπον δε ο αδελφός αυτής και η μήτηρ αυτής, Ας μείνη η κόρη μεθ' ημών ημέρας τινάς, τουλάχιστον δέκα· μετά ταύτα θέλει απέλθει.
\par 56 Και είπε προς αυτούς, Μη με κρατείτε, διότι ο Κύριος κατευώδωσε την οδόν μου· εξαποστείλατέ με να υπάγω προς τον κύριόν μου.
\par 57 Οι δε είπον, Ας καλέσωμεν την κόρην και ας ερωτήσωμεν την γνώμην αυτής.
\par 58 Και εκάλεσαν την Ρεβέκκαν και είπον προς αυτήν, Υπάγεις μετά του ανθρώπου τούτου; Η δε είπεν, Υπάγω.
\par 59 Και εξαπέστειλαν την Ρεβέκκαν την αδελφήν αυτών και την τροφόν αυτής, και τον δούλον του Αβραάμ και τους ανθρώπους αυτού
\par 60 Και ευλόγησαν την Ρεβέκκαν και είπον προς αυτήν, Αδελφή ημών είσαι, είθε να γείνης εις χιλιάδας μυριάδων, και το σπέρμα σου να εξουσιάση τας πύλας των εχθρών αυτού
\par 61 Και εσηκώθη η Ρεβέκκα και αι θεράπαιναι αυτής, και εκάθισαν επί τας καμήλους, και υπήγον κατόπιν του ανθρώπου· και έλαβεν ο δούλος την Ρεβέκκαν και ανεχώρησεν.
\par 62 Ο δε Ισαάκ επέστρεφεν από του φρέατος Λαχαΐ-ροΐ· διότι κατώκει εν τη γη της μεσημβρίας.
\par 63 Και εξήλθεν ο Ισαάκ να προσευχηθή εν τη πεδιάδι περί το εσπέρας· και υψώσας τους οφθαλμούς αυτού, είδε, και ιδού, ήρχοντο κάμηλοι.
\par 64 και υψώσασα η Ρεβέκκα τους οφθαλμούς αυτής είδε τον Ισαάκ και κατεπήδησεν από της καμήλου.
\par 65 Διότι είχεν ειπεί προς τον δούλον, Τις είναι ο άνθρωπος εκείνος, ο ερχόμενος διά της πεδιάδος εις συνάντησιν ημών; Ο δε δούλος είχεν ειπεί, Είναι ο κύριός μου. Και αυτή λαβούσα την καλύπτραν, εσκεπάσθη.
\par 66 Και διηγήθη ο δούλος προς τον Ισαάκ πάντα όσα είχε πράξει.
\par 67 Ο δε Ισαάκ έφερεν αυτήν εις την σκηνήν της μητρός αυτού Σάρρας· και έλαβε την Ρεβέκκαν, και έγεινεν αυτού γυνή, και ηγάπησεν αυτήν· και παρηγορήθη ο Ισαάκ περί της μητρός αυτού.

\chapter{25}

\par 1 Έλαβε δε ο Αβραάμ και άλλην γυναίκα, ονομαζομένην Χεττούραν.
\par 2 Και αύτη εγέννησεν εις αυτόν τον Ζεμβράν και τον Ιοξάν και τον Μαδάν και τον Μαδιάμ και τον Ιεσβώκ και τον Σουά.
\par 3 και ο Ιοξάν εγέννησε τον Σεβά και τον Δαιδάν· οι δε υιοί του Δαιδάν ήσαν Ασσουρείμ και Λετουσιείμ και Λαωμείμ.
\par 4 Οι υιοί δε του Μαδιάμ ήσαν Γεφά και Εφέρ και Ανώχ και Αβειδά και Ελδαγά· πάντες ούτοι υιοί της Χεττούρας.
\par 5 Έδωκε δε ο Αβραάμ πάντα τα υπάρχοντα αυτού εις τον Ισαάκ.
\par 6 Εις δε τους υιούς των παλλακών αυτού έδωκεν ο Αβραάμ χαρίσματα, και εξαπέστειλεν αυτούς, έτι ζων, μακράν από του υιού αυτού Ισαάκ προς ανατολάς, εις την γην της Ανατολής.
\par 7 Και ταύτα είναι τα έτη των ημερών της ζωής του Αβραάμ, όσα έζησεν, έτη εκατόν εβδομήκοντα πέντε.
\par 8 Και εκπνεύσας απέθανεν ο Αβραάμ εν γήρατι καλώ, γέρων και πλήρης ημερών· και προσετέθη εις τον λαόν αυτού.
\par 9 Και έθαψαν αυτόν ο Ισαάκ και ο Ισμαήλ οι υιοί αυτού εν τω σπηλαίω Μαχπελάχ, εν τω αγρώ του Εφρών, υιού του Σωάρ του Χετταίου, τω απέναντι της Μαμβρή·
\par 10 τω αγρώ, τον οποίον ηγόρασεν ο Αβραάμ παρά των υιών του Χέτ· εκεί ετάφη ο Αβραάμ και Σάρρα η γυνή αυτού.
\par 11 Και μετά τον θάνατον του Αβραάμ, ευλόγησεν ο Θεός Ισαάκ τον υιόν αυτού· και κατώκησεν ο Ισαάκ πλησίον του φρέατος Λαχαΐ-ροΐ.
\par 12 Αύτη δε είναι η γενεαλογία του Ισμαήλ, υιού του Αβραάμ, τον οποίον εγέννησεν εις τον Αβραάμ Άγαρ η Αιγυπτία, η δούλη της Σάρρας·
\par 13 και ταύτα είναι τα ονόματα των υιών του Ισμαήλ, κατά τα ονόματα αυτών, εις τας γενεάς αυτών· πρωτότοκος του Ισμαήλ Ναβαϊώθ, έπειτα Κηδάρ και Αβδεήλ και Μιβσάμ,
\par 14 και Μισμά, και Δουμά και Μασσά
\par 15 Χαδδάρ, και Θαιμά, Ιετούρ, Ναφίς, και Κεδμά·
\par 16 ούτοι είναι οι υιοί του Ισμαήλ, και ταύτα τα ονόματα αυτών κατά τας κώμας αυτών και κατά τας κατοικίας αυτών· δώδεκα άρχοντες κατά τα έθνη αυτών.
\par 17 Και ταύτα είναι τα έτη της ζωής του Ισμαήλ, έτη εκατόν τριάκοντα επτά· και εκπνεύσας απέθανε και προσετέθη εις τον λαόν αυτού.
\par 18 Κατώκησαν δε από Αβιλά έως Σούρ, της κατά πρόσωπον Αιγύπτου, καθώς υπάγει τις προς την Ασσυρίαν· ο Ισμαήλ κατώκησεν έμπροσθεν πάντων των αδελφών αυτού.
\par 19 Και αύτη είναι η γενεαλογία του Ισαάκ, υιού του Αβραάμ· ο Αβραάμ εγέννησε τον Ισαάκ·
\par 20 ήτο δε ο Ισαάκ ετών τεσσαράκοντα, ότε έλαβεν εις εαυτόν γυναίκα την Ρεβέκκαν, θυγατέρα Βαθουήλ του Σύρου από Παδάν-αράμ, αδελφήν Λάβαν του Σύρου.
\par 21 Και εδέετο ο Ισαάκ προς τον Κύριον περί της γυναικός αυτού, διότι ήτο στείρα· και επήκουσεν ο Κύριος αυτού, και συνέλαβεν η Ρεβέκκα η γυνή αυτού.
\par 22 Και τα παιδία συνεκρούοντο εντός αυτής· και είπεν, Αν μέλλη ούτω να γείνη, διά τι εγώ να συλλάβω; και υπήγε να ερωτήση τον Κύριον.
\par 23 Και είπεν ο Κύριος προς αυτήν, Δύο έθνη είναι εν τη κοιλία σου· και δύο λαοί θέλουσι διαχωρισθή από των εντοσθίων σου· και ο εις λαός θέλει είσθαι δυνατώτερος του άλλου λαού· και ο μεγαλήτερος θέλει δουλεύσει εις τον μικρότερον.
\par 24 Και ότε επληρώθησαν αι ημέραι αυτής διά να γεννήση, ιδού, ήσαν δίδυμα εν τη κοιλία αυτής.
\par 25 Εξήλθε δε ο πρώτος ερυθρός και όλος ως δέρμα δασύτριχος· και εκάλεσαν το όνομα αυτού, Ησαύ.
\par 26 Και έπειτα εξήλθεν ο αδελφός αυτού· και η χειρ αυτού εκράτει την πτέρναν του Ησαύ· διά τούτο ωνομάσθη Ιακώβ· ο δε Ισαάκ ήτο ετών εξήκοντα, ότε εγέννησεν αυτούς.
\par 27 Ηύξησαν δε τα παιδία· και έγεινεν ο μεν Ησαύ άνθρωπος έμπειρος εις το κυνήγιον, άνθρωπος του αγρού· ο δε Ιακώβ, άνθρωπος απλούς, κατοικών εν σκηναίς.
\par 28 Και ο μεν Ισαάκ ηγάπα τον Ησαύ, διότι το κυνήγιον ήτο τροφή εις αυτόν· η δε Ρεβέκκα ηγάπα τον Ιακώβ.
\par 29 Εμαγείρευε δε ο Ιακώβ μαγείρευμα· και ήλθεν ο Ησαύ εκ του αγρού και ήτο αποκαμωμένος·
\par 30 και είπεν ο Ησαύ προς τον Ιακώβ, Δος μοι, παρακαλώ, να φάγω από το κόκκινον, το κόκκινον τούτο, διότι είμαι αποκαμωμένος· διά τούτο εκλήθη το όνομα αυτού, Εδώμ.
\par 31 Και είπεν ο Ιακώβ, Πώλησόν μοι σήμερον τα πρωτοτόκιά σου.
\par 32 Και ο Ησαύ είπεν, Ιδού, εγώ υπάγω να αποθάνω, και τι με ωφελούσι ταύτα τα πρωτοτόκια;
\par 33 Και είπεν ο Ιακώβ, Ομοσόν μοι σήμερον· και ώμοσεν εις αυτόν· και επώλησε τα πρωτοτόκια αυτού εις τον Ιακώβ.
\par 34 Τότε ο Ιακώβ έδωκεν εις τον Ησαύ άρτον και μαγείρευμα της φακής· και έφαγε και έπιε και σηκωθείς ανεχώρησεν· ούτως ο Ησαύ κατεφρόνησε τα πρωτοτόκια.

\chapter{26}

\par 1 Έγεινε δε πείνα εν τη γη, εκτός της προτέρας πείνης, της γενομένης επί των ημερών του Αβραάμ. Και υπήγεν ο Ισαάκ προς τον Αβιμέλεχ, βασιλέα των Φιλισταίων, εις Γέραρα.
\par 2 Εφάνη δε εις αυτόν ο Κύριος και είπε, Μη καταβής εις Αίγυπτον· κατοίκησον εν τη γη την οποίαν θέλω σοι ειπεί·
\par 3 παροίκει εν τη γη ταύτη, και εγώ θέλω είσθαι μετά σου, και θέλω σε ευλογήσει διότι εις σε και εις το σπέρμα σου θέλω δώσει πάντας τους τόπους τούτους· και θέλω εκπληρώσει τον όρκον, τον οποίον ώμοσα προς Αβραάμ τον πατέρα σου·
\par 4 και θέλω πληθύνει το σπέρμα σου ως τα άστρα του ουρανού, και θέλω δώσει εις το σπέρμα σου πάντας τους τόπους τούτους, και θέλουσιν ευλογηθή εν τω σπέρματί σου πάντα τα έθνη της γής·
\par 5 επειδή ο Αβραάμ υπήκουσεν εις την φωνήν μου, και εφύλαξε τα προστάγματά μου, τας εντολάς μου, τα διατάγματά μου και τους νόμους μου.
\par 6 Και κατώκησεν ο Ισαάκ εν Γεράροις.
\par 7 Ηρώτησαν δε οι άνδρες του τόπου περί της γυναικός αυτού· και είπεν, Αδελφή μου είναι· διότι εφοβήθη να είπη, Γυνή μου είναι· λέγων, Μήπως με φονεύσωσιν οι άνδρες του τόπου διά την Ρεβέκκαν· επειδή ήτο ώραία την όψιν.
\par 8 Και αφού διέτριψεν εκεί πολλάς ημέρας, Αβιμέλεχ ο βασιλεύς των Φιλισταίων, κύψας από της θυρίδος είδε, και ιδού, ο Ισαάκ έπαιζε μετά Ρεβέκκας της γυναικός αυτού.
\par 9 Εκάλεσε δε ο Αβιμέλεχ τον Ισαάκ και είπεν, Ιδού, βεβαίως γυνή σου είναι αύτη· διά τι λοιπόν είπας, Αδελφή μου είναι; Και είπε προς αυτόν ο Ισαάκ, διότι είπον, Μήπως αποθάνω εξ αιτίας αυτής.
\par 10 Και είπεν ο Αβιμέλεχ, Τι είναι τούτο, το οποίον έκαμες εις ημάς; παρ' ολίγον ήθελε κοιμηθή τις εκ του λαού μετά της γυναικός σου, και ήθελες φέρει εφ' ημάς ανομίαν.
\par 11 Και προσέταξεν ο Αβιμέλεχ εις πάντα τον λαόν, λέγων, Όστις εγγίση τον άνθρωπον τούτον ή την γυναίκα αυτού, θέλει εξάπαντος θανατωθή.
\par 12 Έσπειρε δε ο Ισαάκ εν τη γη εκείνη και εσύναξε κατ' εκείνον τον χρόνον εκατονταπλάσια· και ευλόγησεν αυτόν ο Κύριος.
\par 13 Και εμεγαλύνετο ο άνθρωπος και επροχώρει αυξανόμενος, εωσού έγεινε μέγας σφόδρα·
\par 14 και απέκτησε πρόβατα και βόας και δούλους πολλούς· εφθόνησαν δε αυτόν οι Φιλισταίοι.
\par 15 Και πάντα τα φρέατα, τα οποία έσκαψαν οι δούλοι του πατρός αυτού επί των ημερών Αβραάμ του πατρός αυτού, ενέφραξαν ταύτα οι Φιλισταίοι και εγέμισαν αυτά χώμα.
\par 16 Και είπεν ο Αβιμέλεχ προς τον Ισαάκ, Άπελθε αφ' ημών, διότι έγεινες δυνατώτερος ημών σφόδρα.
\par 17 Και απήλθεν εκείθεν ο Ισαάκ και έστησε την σκηνήν αυτού εν τη κοιλάδι των Γεράρων και κατώκησεν εκεί.
\par 18 Και ήνοιξε πάλιν ο Ισαάκ τα φρέατα του ύδατος, τα οποία έσκαψαν επί των ημερών Αβραάμ του πατρός αυτού, οι δε Φιλισταίοι ενέφραξαν αυτά μετά τον θάνατον του Αβραάμ· και ωνόμασεν αυτά κατά τα ονόματα, με τα οποία ο πατήρ αυτού είχεν ονομάσει αυτά.
\par 19 Και έσκαψαν οι δούλοι του Ισαάκ εν τη κοιλάδι και εύρηκαν εκεί φρέαρ ύδατος ζώντος.
\par 20 Ελογομάχησαν δε οι ποιμένες των Γεράρων μετά των ποιμένων του Ισαάκ, λέγοντες, Ιδικόν μας είναι το ύδωρ· και ωνόμασε το φρέαρ Εσέκ· διότι εφιλονείκησαν μετ' αυτού.
\par 21 Και έσκαψαν άλλο φρέαρ και ελογομάχησαν και περί αυτού· διά τούτο ωνόμασεν αυτό Σιτνά.
\par 22 Και μετοικήσας εκείθεν έσκαψεν άλλο φρέαρ, και περί τούτου δεν ελογομάχησαν· και ωνόμασεν αυτό Ρεχωβώθ, λέγων, διότι τώρα επλάτυνεν ημάς ο Κύριος και ηύξησεν ημάς επί της γης.
\par 23 Και εκείθεν ανέβη εις Βηρ-σαβεέ.
\par 24 Και εφάνη εις αυτόν ο Κύριος την νύκτα εκείνην, και είπεν, Εγώ είμαι ο Θεός Αβραάμ του πατρός σου· μη φοβού, διότι εγώ είμαι μετά σου, και θέλω σε ευλογήσει και θέλω πληθύνει το σπέρμα σου, διά Αβραάμ τον δούλον μου.
\par 25 Και ωκοδόμησεν εκεί θυσιαστήριον και επεκαλέσθη το όνομα του Κυρίου· και έστησεν εκεί την σκηνήν αυτού· έσκαψαν δε εκεί οι δούλοι του Ισαάκ φρέαρ.
\par 26 Τότε ο Αβιμέλεχ υπήγε προς αυτόν από Γεράρων, και Οχοζάθ ο οικείος αυτού, και Φιχόλ ο αρχιστράτηγος της δυνάμεως αυτού.
\par 27 Και είπε προς αυτούς ο Ισαάκ, Διά τι ήλθετε προς εμέ, αφού σεις με εμισήσατε και με εδιώξατε από σας;
\par 28 οι δε είπον, Είδομεν φανερά, ότι ο Κύριος είναι μετά σου, και είπομεν, Ας γείνη τώρα όρκος μεταξύ ημών, μεταξύ ημών και σου, και ας κάμωμεν συνθήκην μετά σου,
\par 29 ότι δεν θέλεις κάμει κακόν εις ημάς, καθώς ημείς δεν σε ηγγίσαμεν, και καθώς επράξαμεν εις σε μόνον καλόν, και σε εξαπεστείλαμεν εν ειρήνη· τώρα συ είσαι ευλογημένος του Κυρίου.
\par 30 Και έκαμεν εις αυτούς συμπόσιον· και έφαγον και έπιον.
\par 31 Και εσηκώθησαν ενωρίς το πρωΐ, και ώμοσεν ο εις προς τον άλλον· τότε ο Ισαάκ εξαπέστειλεν αυτούς, και απήλθον απ' αυτού εν ειρήνη.
\par 32 Και την ημέραν εκείνην ήλθον οι δούλοι του Ισαάκ και ανήγγειλαν προς αυτόν περί του φρέατος το οποίον έσκαψαν, και είπαν προς αυτόν, Ευρήκαμεν ύδωρ.
\par 33 Και ωνόμασεν αυτό Σαβεέ· διά τούτο είναι το όνομα της πόλεως Βηρ-σαβεέ έως της σήμερον.
\par 34 Ήτο δε ο Ησαύ ετών τεσσαράκοντα, ότε έλαβεν εις γυναίκα Ιουδίθ, την θυγατέρα Βεηρί του Χετταίου, και Βασεμάθ, την θυγατέρα Αιλών του Χετταίου·
\par 35 και αύται ήσαν πικρία ψυχής εις τον Ισαάκ και την Ρεβέκκαν.

\chapter{27}

\par 1 Και αφού εγήρασεν ο Ισαάκ, και οι οφθαλμοί αυτού ημβλύνθησαν, ώστε δεν έβλεπεν, εκάλεσεν Ησαύ τον υιόν αυτού τον μεγαλήτερον, και είπε προς αυτόν, Υιέ μου. Ο δε είπε προς αυτόν, Ιδού, εγώ.
\par 2 Και εκείνος είπεν, Ιδού, τώρα, εγώ εγήρασα· δεν γνωρίζω την ημέραν του θανάτου μου·
\par 3 λάβε λοιπόν, παρακαλώ, τα όπλα σου, την φαρέτραν σου και το τόξον σου, και έξελθε εις την πεδιάδα και κυνήγησόν μοι κυνήγιον·
\par 4 και κάμε μοι εδέσματα καθώς αγαπώ, και φέρε μοι να φάγω, διά να σε ευλογήση η ψυχή μου πριν αποθάνω.
\par 5 Η δε Ρεβέκκα ήκουσεν ενώ ελάλει ο Ισαάκ προς Ησαύ τον υιόν αυτού. Και υπήγεν ο Ησαύ εις την πεδιάδα διά να κυνηγήση κυνήγιον και να φέρη αυτό.
\par 6 Και η Ρεβέκκα ελάλησε προς Ιακώβ τον υιόν αυτής, λέγουσα, Ιδού, εγώ ήκουσα τον πατέρα σου λαλούντα προς Ησαύ τον αδελφόν και λέγοντα,
\par 7 Φέρε μοι κυνήγιον και κάμε μοι εδέσματα, διά να φάγω, και να σε ευλογήσω ενώπιον του Κυρίου πριν αποθάνω.
\par 8 Τώρα λοιπόν, υιέ μου, άκουσον την φωνήν μου εις όσα εγώ σοι παραγγέλλω·
\par 9 ύπαγε τώρα εις το ποίμνιον, και λάβε μοι εκείθεν δύο καλά ερίφια εξ αιγών· διά να κάμω αυτά εδέσματα διά τον πατέρα σου, καθώς αγαπά·
\par 10 και θέλεις φέρει αυτά προς τον πατέρα σου να φάγη, διά σε ευλογήση πριν αποθάνη.
\par 11 Και είπεν ο Ιακώβ προς Ρεβέκκαν την μητέρα αυτού, Ιδού, ο Ησαύ ο αδελφός μου είναι ανήρ δασύτριχος, εγώ δε ανήρ άτριχος·
\par 12 ίσως με ψηλαφήση ο πατήρ μου, και θέλω φανή εις αυτόν ως απατεών, και θέλω σύρει επ' εμαυτόν κατάραν και ουχί ευλογίαν.
\par 13 Είπε δε προς αυτόν η μήτηρ αυτού, Επ' εμέ η κατάρα σου, τέκνον μου· μόνον υπάκουσον εις την φωνήν μου και ύπαγε, φέρε μοι αυτά.
\par 14 Και υπήγε, και έλαβε, και έφερεν αυτά προς την μητέρα αυτού· και έκαμεν η μήτηρ αυτού εδέσματα καθώς ηγάπα ο πατήρ αυτού.
\par 15 Και λαβούσα η Ρεβέκκα τα καλήτερα φορέματα Ησαύ του μεγαλητέρου υιού αυτής, τα οποία είχεν εν τη οικία, ενέδυσε με αυτά Ιακώβ, τον υιόν αυτής τον νεώτερον·
\par 16 και με τα δέρματα των εριφίων εσκέπασε τας χείρας αυτού, και τα γυμνά του τραχήλου αυτού·
\par 17 και έδωκεν εις τας χείρας Ιακώβ του υιού αυτής τα εδέσματα και τον άρτον, τα οποία ητοίμασε.
\par 18 Και ήλθε προς τον πατέρα αυτού· και είπε, Πάτερ μου. Ο δε είπεν, Ιδού, εγώ· τις είσαι, τέκνον μου;
\par 19 Και είπεν ο Ιακώβ προς τον πατέρα αυτού, Εγώ είμαι Ησαύ ο πρωτότοκός σου· έκαμα καθώς μοι είπας, σηκώθητι λοιπόν, κάθισον και φάγε εκ του κυνηγίου μου, διά να με ευλογήση η ψυχή σου.
\par 20 Και είπεν ο Ισαάκ προς τον υιόν αυτού, Πόθεν τούτο, τέκνον μου, ότι εύρηκας τόσον ταχέως; Ο δε είπε, Διότι Κύριος ο Θεός σου έφερεν αυτό έμπροσθέν μου.
\par 21 Και είπεν ο Ισαάκ προς τον Ιακώβ, Πλησίασον, τέκνον μου, διά να σε ψηλαφήσω, αν συ ήσαι αυτός ο υιός Ησαύ, ή ουχί.
\par 22 Και επλησίασεν ο Ιακώβ εις τον Ισαάκ τον πατέρα αυτού· ο δε εψηλάφησεν αυτόν, και είπεν, Η μεν φωνή είναι φωνή Ιακώβ, αι δε χείρες, χείρες Ησαύ.
\par 23 Και δεν εγνώρισεν αυτόν, διότι αι χείρες αυτού ήσαν ως αι χείρες Ησαύ αδελφού αυτού, δασύτριχοι· και ευλόγησεν αυτόν.
\par 24 Και είπε, Συ είσαι αυτός ο υιός μου Ησαύ; Ο δε είπεν, Εγώ.
\par 25 Και είπε, Φέρε πλησίον μου, και θέλω φάγει εκ του κυνηγίου του υιού μου, διά να σε ευλογήση η ψυχή μου. Και έφερε πλησίον αυτού, και έφαγεν· έφερε δε προς αυτόν οίνον και έπιε.
\par 26 Και είπε προς αυτόν Ισαάκ ο πατήρ αυτού, Πλησίασον τώρα, και φίλησόν με, τέκνον μου.
\par 27 Και επλησίασε, και εφίλησεν αυτόν· και ωσφράνθη την οσμήν των ενδυμάτων αυτού, και ευλόγησεν αυτόν και είπεν, Ιδού, η οσμή του υιού μου είναι ως οσμή πεδιάδος, την οποίαν ευλόγησεν ο Κύριος·
\par 28 Λοιπόν ο Θεός να σοι δώση από της δρόσου του ουρανού και από του πάχους της γης και αφθονίαν σίτου και οίνου·
\par 29 Λαοί να σε δουλεύσωσι και έθνη να σε προσκυνήσωσι· να ήσαι κύριος των αδελφών σου, και οι υιοί της μητρός σου να σε προσκυνήσωσι· κατηραμένος όστις σε καταράται, και ευλογημένος όστις σε ευλογεί
\par 30 Και καθώς έπαυσεν ο Ισαάκ ευλογών τον Ιακώβ, μόλις ο Ιακώβ είχεν εξέλθει απ' έμπροσθεν του πατρός αυτού Ισαάκ· και ήλθεν Ησαύ ο αδελφός αυτού εκ του κυνηγίου αυτού.
\par 31 Και έκαμε και αυτός εδέσματα και έφερε προς τον πατέρα αυτού· και είπε προς τον πατέρα αυτού, Ας σηκωθή ο πατήρ μου, και ας φάγη εκ του κυνηγίου του υιού αυτού, διά να με ευλογήση η ψυχή σου.
\par 32 Και είπε προς αυτόν Ισαάκ ο πατήρ αυτού, Τις είσαι; Ο δε είπεν, Είμαι ο υιός σου, ο πρωτότοκός σου Ησαύ.
\par 33 Και εξεπλάγη ο Ισαάκ έκπληξιν μεγάλην σφόδρα, και είπε, Ποίος είναι λοιπόν εκείνος, όστις εκυνήγησε κυνήγιον, και μοι έφερε και έφαγον από πάντων πριν εισέλθης, και ευλόγησα αυτόν; και ευλογημένος θέλει είσθαι.
\par 34 Ότε ήκουσεν ο Ησαύ τους λόγους του πατρός αυτού, ανέκραξε κραυγήν μεγάλην και πικράν σφόδρα· και είπε προς τον πατέρα αυτού, Ευλόγησόν με, και εμέ, πάτερ μου.
\par 35 Ο δε είπεν, Ήλθεν ο αδελφός σου μετά δόλου, και έλαβε την ευλογίαν σου.
\par 36 Και είπεν ο Ησαύ, Δικαίως εκαλέσθη το όνομα αυτού Ιακώβ, διότι τώρα δευτέραν ταύτην φοράν με υπεσκέλισεν· έλαβε τα πρωτοτόκιά μου, και ιδού, τώρα έλαβε και την ευλογίαν μου. Και είπε, Δεν εφύλαξας δι' εμέ ευλογίαν;
\par 37 Και, απεκρίθη ο Ισαάκ, και είπε προς τον Ησαύ, Ιδού, κύριόν σου έκαμα αυτόν, και πάντας τους αδελφούς αυτού έκαμα δούλους αυτού, και εστήριξα αυτόν με σίτον και οίνον· και τι λοιπόν να κάμω εις σε, τέκνον μου;
\par 38 Και είπεν ο Ησαύ προς τον πατέρα αυτού, Μήπως ταύτην μόνην την ευλογίαν έχεις, πάτερ μου; ευλόγησόν με, και εμέ, πάτερ μου. και ύψωσεν ο Ησαύ την φωνήν αυτού, και έκλαυσε.
\par 39 Και απεκρίθη Ισαάκ ο πατήρ αυτού, και είπε προς αυτόν, Ιδού, η κατοίκησίς σου θέλει είσθαι εις το πάχος της γης, και εις την δρόσον του ουρανού άνωθεν·
\par 40 και με την μάχαιράν σου θέλεις ζη, και εις τον αδελφόν σου θέλεις δουλεύσει, όταν δε υπερισχύσης, θέλεις συντρίψει τον ζυγόν αυτού από του τραχήλου σου.
\par 41 Και εμίσει ο Ησαύ τον Ιακώβ, διά την ευλογίαν με την οποίαν ευλόγησεν αυτόν ο πατήρ αυτού· και είπεν ο Ησαύ εν τη καρδία αυτού, Πλησιάζουσιν αι ημέραι του πένθους του πατρός μου· τότε θέλω φονεύσει Ιακώβ τον αδελφόν μου.
\par 42 Ανηγγέλθησαν, δε προς την Ρεβέκκαν οι λόγοι Ησαύ του υιού αυτής του μεγαλητέρου· και πέμψασα εκάλεσεν Ιακώβ τον υιόν αυτής τον νεώτερον, και είπε προς αυτόν, Ιδού, Ησαύ ο αδελφός σου παρηγορεί εαυτόν κατά σου, ότι θέλει σε φονεύσει.
\par 43 Τώρα λοιπόν, τέκνον μου, άκουσον την φωνήν μου· και σηκωθείς, φύγε προς Λάβαν τον αδελφόν μου εις Χαρράν·
\par 44 και κατοίκησον μετ' αυτού ημέρας τινάς, εωσού παρέλθη ο θυμός του αδελφού σου·
\par 45 εωσού παύση η κατά σου οργή του αδελφού σου, και λησμονήση τα όσα έπραξας εις αυτόν· τότε θέλω στείλει, και θέλω σε φέρει εκείθεν· διά τι να σας στερηθώ και τους δύο εν μιά ημέρα;
\par 46 Και είπεν η Ρεβέκκα προς τον Ισαάκ, Αηδίασα την ζωήν μου εξ αιτίας των θυγατέρων του Χέτ· εάν ο Ιακώβ λάβη γυναίκα εκ των θυγατέρων του Χετ, καθώς είναι αύται εκ των θυγατέρων της γης ταύτης, τι με ωφελεί να ζω;

\chapter{28}

\par 1 Και προσκαλέσας ο Ισαάκ τον Ιακώβ ευλόγησεν αυτόν, και παρήγγειλε προς αυτόν λέγων, Δεν θέλεις λάβει γυναίκα εκ των θυγατέρων Χαναάν·
\par 2 σηκωθείς ύπαγε εις Παδάν-αράμ, εις την οικίαν Βαθουήλ του πατρός της μητρός σου· και εκείθεν λάβε εις σεαυτόν γυναίκα, εκ των θυγατέρων Λάβαν του αδελφού της μητρός σου·
\par 3 και ο Θεός ο Παντοδύναμος να σε ευλογήση και να σε αυξήση και να σε πληθύνη, ώστε να γείνης εις πλήθος λαών·
\par 4 και να σοι δώση την ευλογίαν του Αβραάμ, εις σε και εις το σπέρμα σου μετά σε, διά να κληρονομήσης την γην της παροικήσεώς σου, την οποίαν έδωκεν ο Θεός εις τον Αβραάμ.
\par 5 Και εξαπέστειλεν ο Ισαάκ τον Ιακώβ· και υπήγεν εις Παδάν-αράμ προς Λάβαν, τον υιόν του Βαθουήλ του Σύρου, τον αδελφόν Ρεβέκκας της μητρός του Ιακώβ και του Ησαύ.
\par 6 Ιδών δε ο Ησαύ ότι ευλόγησεν ο Ισαάκ τον Ιακώβ και εξαπέστειλεν αυτόν εις Παδάν-αράμ, διά να λάβη εις εαυτόν γυναίκα εκείθεν, και ότι, ενώ ευλόγει αυτόν, παρήγγειλεν εις αυτόν, λέγων, Δεν θέλεις λάβει γυναίκα εκ των θυγατέρων Χαναάν·
\par 7 και ότι υπήκουσεν ο Ιακώβ εις τον πατέρα αυτού και την μητέρα αυτού· και υπήγεν εις Παδάν-αράμ·
\par 8 και ιδών ο Ησαύ ότι αι θυγατέρες Χαναάν είναι μισηταί εις τους οφθαλμούς του πατρός αυτού Ισαάκ,
\par 9 υπήγεν ο Ησαύ προς τον Ισμαήλ, και εκτός των άλλων γυναικών αυτού έλαβεν εις εαυτόν γυναίκα την Μαελέθ, θυγατέρα Ισμαήλ του υιού του Αβραάμ, την αδελφήν του Ναβαϊώθ.
\par 10 Και εξήλθεν ο Ιακώβ από Βηρ-σαβεέ, και υπήγεν εις Χαρράν.
\par 11 Και έφθασεν εις τινά τόπον και διενυκτέρευσεν εκεί, διότι είχε δύσει ο ήλιος· και έλαβεν εκ των λίθων του τόπου και έθεσε προσκεφάλαιον αυτού, και εκοιμήθη εν τω τόπω εκείνω.
\par 12 Και είδεν ενύπνιον, και ιδού, κλίμαξ εστηριγμένη εις την γην, της οποίας η κορυφή έφθανεν εις τον ουρανόν· και ιδού, οι άγγελοι του Θεού ανέβαινον και κατέβαινον επ' αυτής.
\par 13 Και ιδού, ο Κύριος ίστατο επάνωθεν αυτής και είπεν, Εγώ είμαι Κύριος ο Θεός του Αβραάμ του πατρός σου, και ο Θεός του Ισαάκ· την γην, επί της οποίας κοιμάσαι, εις σε θέλω δώσει αυτήν και εις το σπέρμα σου.
\par 14 και θέλει είσθαι το σπέρμα σου ως η άμμος της γης, και θέλεις εξαπλωθή προς δύσιν και προς ανατολήν και προς βορράν και προς νότον· και θέλουσιν ευλογηθή εν σοι, και εν τω σπέρματί σου πάσαι αι φυλαί της γής·
\par 15 και ιδού, εγώ είμαι μετά σου, και θέλω σε διαφυλάττει πανταχού, όπου αν υπάγης, και θέλω σε επαναφέρει εις την γην ταύτην· διότι δεν θέλω σε εγκαταλείψει, εωσού κάμω όσα ελάλησα προς σε.
\par 16 Και εξεγερθείς ο Ιακώβ εκ του ύπνου αυτού, είπε, Βέβαια ο Κύριος είναι εν τω τόπω τούτω, και εγώ δεν ήξευρον.
\par 17 Και εφοβήθη και είπε, Πόσον φοβερός είναι ο τόπος ούτος· δεν είναι τούτο, ειμή οίκος Θεού, και αύτη η πύλη του ουρανού.
\par 18 Και σηκωθείς ο Ιακώβ ενωρίς το πρωΐ, έλαβε τον λίθον τον οποίον είχε θέσει προσκεφάλαιον αυτού, και έστησεν αυτόν διά στήλην και έχυσεν έλαιον επί την κορυφήν αυτής.
\par 19 Και εκάλεσε το όνομα του τόπου εκείνου, Βαιθήλ· το δε όνομα της πόλεως εκείνης ήτο πρότερον Λούζ.
\par 20 Και ευχήθη ο Ιακώβ ευχήν, λέγων, Αν ο Θεός ήναι μετ' εμού και με διαφυλάξη εν τη οδώ ταύτη εις την οποίαν υπάγω, και μοι δώση άρτον να φάγω και ένδυμα να ενδυθώ,
\par 21 και επιστρέψω εν ειρήνη εις τον οίκον του πατρός μου, τότε ο Κύριος θέλει είσθαι Θεός μου·
\par 22 και ο λίθος ούτος, τον οποίον έστησα διά στήλην, θέλει είσθαι οίκος Θεού· και εκ πάντων όσα μοι δώσης, το δέκατον θέλω προσφέρει εις σε.

\chapter{29}

\par 1 Και εκίνησεν ο Ιακώβ και υπήγεν εις την γην των κατοίκων της ανατολής.
\par 2 Και είδε, και ιδού, φρέαρ εν τη πεδιάδι και ιδού, εκεί τρία ποίμνια προβάτων αναπαυόμενα πλησίον αυτού, διότι εκ του φρέατος εκείνου επότιζον τα ποίμνια· λίθος δε μέγας ήτο επί το στόμιον του φρέατος.
\par 3 Και ότε συνήγοντο εκεί πάντα τα ποίμνια, απεκύλιον τον λίθον από του στομίου του φρέατος, και επότιζον τα ποίμνια· έπειτα έθετον πάλιν τον λίθον επί το στόμιον του φρέατος εις τον τόπον αυτού.
\par 4 Και είπε προς αυτούς ο Ιακώβ, Αδελφοί, πόθεν είσθε; Οι δε είπον, Εκ της Χαρράν είμεθα.
\par 5 Και είπε προς αυτούς, Γνωρίζετε Λάβαν τον υιόν του Ναχώρ; οι δε είπον, Γνωρίζομεν.
\par 6 Και είπε προς αυτούς, Υγιαίνει; Οι δε είπον, Υγιαίνει· και ιδού, Ραχήλ η θυγάτηρ αυτού έρχεται μετά των προβάτων.
\par 7 Και είπεν, Ιδού, μένει ακόμη ημέρα πολλή, δεν είναι ώρα να συρθώσι τα κτήνη· ποτίσατε τα πρόβατα και υπάγετε να βοσκήσητε αυτά.
\par 8 Οι δε είπον, Δεν δυνάμεθα, εωσού συναχθώσι πάντα τα ποίμνια, και να αποκυλίσωσι τον λίθον από του στομίου του φρέατος· τότε ποτίζομεν τα πρόβατα.
\par 9 Και ενώ ακόμη ελάλει προς αυτούς, ήλθεν η Ραχήλ μετά των προβάτων του πατρός αυτής· διότι αυτή έβοσκε.
\par 10 Και ως είδεν ο Ιακώβ την Ραχήλ, θυγατέρα του Λάβαν του αδελφού της μητρός αυτού, και τα πρόβατα του Λάβαν του αδελφού της μητρός αυτού, επλησίασεν ο Ιακώβ και απεκύλισε τον λίθον από του στομίου του φρέατος, και επότισε τα πρόβατα του Λάβαν, του αδελφού της μητρός αυτού.
\par 11 Και εφίλησεν ο Ιακώβ την Ραχήλ και υψώσας την φωνήν αυτού έκλαυσε.
\par 12 Και απήγγειλεν ο Ιακώβ προς την Ραχήλ, ότι είναι αδελφός του πατρός αυτής, και ότι είναι υιός της Ρεβέκκας· και εκείνη δραμούσα απήγγειλε τούτο εις τον πατέρα αυτής.
\par 13 Και ως ήκουσεν ο Λάβαν το όνομα του Ιακώβ του υιού της αδελφής αυτού, έδραμεν εις συνάντησιν αυτού· και εναγκαλισθείς αυτόν, εφίλησεν αυτόν και έφερεν αυτόν εις την οικίαν αυτού· και διηγήθη ο Ιακώβ προς τον Λάβαν πάντα τα γενόμενα.
\par 14 Και είπε προς αυτόν ο Λάβαν, Βέβαια οστούν μου και σαρξ μου είσαι. Και κατώκησε μετ' αυτού ένα μήνα.
\par 15 Και είπεν ο Λάβαν προς τον Ιακώβ, Επειδή είσαι αδελφός μου, διά τούτο θέλεις με δουλεύει δωρεάν; ειπέ μοι, τις θέλει είσθαι ο μισθός σου;
\par 16 Είχε δε Λάβαν δύο θυγατέρας· το όνομα της πρεσβυτέρας, Λεία, και το όνομα της μικροτέρας Ραχήλ.
\par 17 Και της μεν Λείας οι οφθαλμοί ήσαν ασθενείς· η δε Ραχήλ ήτο ευειδής και ώραία την όψιν.
\par 18 Και ηγάπησεν ο Ιακώβ την Ραχήλ· και είπε, Θέλω σε δουλεύει επτά έτη διά την Ραχήλ, την θυγατέρα σου την μικροτέραν.
\par 19 Και είπεν ο Λάβαν, Καλήτερα να δώσω αυτήν εις σε, παρά να δώσω αυτήν εις άλλον άνδρα· κατοίκησον μετ' εμού.
\par 20 Και εδούλευσεν ο Ιακώβ διά την Ραχήλ επτά έτη· και εφαίνοντο εις αυτόν ως ημέραι ολίγαι, διά την προς αυτήν αγάπην αυτού.
\par 21 Και είπεν ο Ιακώβ προς τον Λάβαν, Δος μοι την γυναίκα μου, διότι επληρώθησαν αι ημέραι μου, διά να εισέλθω προς αυτήν.
\par 22 Και συνήγαγεν ο Λάβαν πάντας τους ανθρώπους του τόπου και έκαμε συμπόσιον.
\par 23 Και το εσπέρας, λαβών την Λείαν την θυγατέρα αυτού, έφερεν αυτήν προς αυτόν· και εισήλθε προς αυτήν.
\par 24 Και έδωκεν ο Λάβαν εις Λείαν την θυγατέρα αυτού, διά θεράπαιναν αυτής, Ζελφάν την θεράπαιναν αυτού.
\par 25 Και το πρωΐ, ιδού, αύτη ήτο η Λεία· και είπε προς τον Λάβαν, Τι τούτο το οποίον έπραξας εις εμέ; δεν σε εδούλευσα διά την Ραχήλ; και διά τι με ηπάτησας;
\par 26 Και είπεν ο Λάβαν, Δεν γίνεται ούτως εν τω τόπω ημών, να δίδωται η μικροτέρα προ της πρεσβυτέρας·
\par 27 εκπλήρωσον την εβδομάδα ταύτης, και θέλω σοι δώσει και αυτήν, αντί της εργασίας την οποίαν θέλεις κάμει εις εμέ ακόμη άλλα επτά έτη.
\par 28 Και έκαμεν ο Ιακώβ ούτω και εξεπλήρωσε την εβδομάδα αυτής· και έδωκεν εις αυτόν την Ραχήλ την θυγατέρα αυτού εις γυναίκα.
\par 29 Και έδωκεν ο Λάβαν εις Ραχήλ την θυγατέρα αυτού, διά θεράπαιναν αυτής, Βαλλάν την θεράπαιναν αυτού.
\par 30 Και εισήλθεν ο Ιακώβ και προς την Ραχήλ· και ηγάπησε την Ραχήλ περισσότερον παρά την Λείαν· και εδούλευσεν αυτόν ακόμη άλλα επτά έτη.
\par 31 Και ιδών ο Κύριος ότι εμισείτο η Λεία, ήνοιξε την μήτραν αυτής· η δε Ραχήλ ήτο στείρα.
\par 32 Και συνέλαβεν η Λεία και εγέννησεν υιόν και εκάλεσε το όνομα αυτού Ρουβήν· διότι είπεν, Είδε βέβαια ο Κύριος την ταπείνωσίν μου· τώρα λοιπόν θέλει με αγαπήσει ο ανήρ μου.
\par 33 Και συνέλαβε πάλιν και εγέννησεν υιόν· και είπεν, Επειδή ήκουσεν ο Κύριος ότι μισούμαι, διά τούτο μοι έδωκεν ακόμη και τούτον· και εκάλεσε το όνομα αυτού Συμεών.
\par 34 Και συνέλαβεν ακόμη και εγέννησεν υιόν· και είπε, Τώρα ταύτην την φοράν ο ανήρ μου θέλει ενωθή μετ' εμού, διότι εγέννησα εις αυτόν τρεις υιούς· διά τούτο ωνόμασεν αυτόν Λευΐ.
\par 35 Και συνέλαβε πάλιν και εγέννησεν υιόν· και είπε, Ταύτην την φοράν θέλω δοξολογήσει τον Κύριον· διά τούτο εκάλεσε το όνομα αυτού Ιούδαν· και έπαυσε να γεννά.

\chapter{30}

\par 1 Και ότε είδεν η Ραχήλ ότι δεν ετεκνοποίησεν εις τον Ιακώβ, εφθόνησεν η Ραχήλ την αδελφήν αυτής· και είπε προς τον Ιακώβ, Δος μοι τέκνα· ειδέ μη, εγώ αποθνήσκω.
\par 2 Και εξήφθη ο θυμός του Ιακώβ κατά της Ραχήλ και είπε, Μήπως είμαι εγώ αντί του Θεού όστις σε εστέρησεν από καρπού κοιλίας;
\par 3 Η δε είπεν, Ιδού, η θεράπαινά μου Βαλλά· είσελθε προς αυτήν, και θέλει γεννήσει επί των γονάτων μου, διά να αποκτήσω και εγώ τέκνα εξ αυτής.
\par 4 Και έδωκεν εις αυτόν την Βαλλάν την θεράπαιναν αυτής διά γυναίκα· και εισήλθεν ο Ιακώβ προς αυτήν.
\par 5 Και συνέλαβεν η Βαλλά, και εγέννησεν υιόν εις τον Ιακώβ·
\par 6 και είπεν η Ραχήλ, Ο Θεός με έκρινε και ήκουσε και την φωνήν μου και μοι έδωκεν υιόν· διά τούτο εκάλεσε το όνομα αυτού Δαν.
\par 7 Και συνέλαβε πάλιν η Βαλλά, η θεράπαινα της Ραχήλ, και εγέννησε δεύτερον υιόν εις τον Ιακώβ·
\par 8 και είπεν η Ραχήλ, Δυνατήν πάλην επάλαισα μετά της αδελφής μου, και υπερίσχυσα· και εκάλεσε το όνομα αυτού Νεφθαλί.
\par 9 Και ότε είδεν η Λεία ότι έπαυσε να γεννά, έλαβε την Ζελφάν την θεράπαιναν αυτής, και έδωκεν αυτήν εις τον Ιακώβ διά γυναίκα.
\par 10 Και η Ζελφά, η θεράπαινα της Λείας, εγέννησεν υιόν εις τον Ιακώβ·
\par 11 και είπεν η Λεία, Ευτυχία έρχεται· και εκάλεσε το όνομα αυτού Γαδ.
\par 12 Και εγέννησεν η Ζελφά, η θεράπαινα της Λείας, δεύτερον υιόν εις τον Ιακώβ·
\par 13 και είπεν η Λεία, Μακαρία εγώ, διότι θέλουσι με μακαρίζει αι γυναίκες· και εκάλεσε το όνομα αυτού Ασήρ.
\par 14 Και υπήγεν ο Ρουβήν εν ταις ημέραις του θερισμού του σίτου και εύρηκε μανδραγόρας εν τω αγρώ, και έφερεν αυτούς προς την Λείαν την μητέρα αυτού. Είπε δε η Ραχήλ προς την Λείαν, Δος μοι, παρακαλώ, από τους μανδραγόρας του υιού σου.
\par 15 Η δε είπε προς αυτήν, Μικρόν πράγμα είναι, ότι έλαβες τον άνδρα μου; και θέλεις να λάβης και τους μανδραγόρας του υιού μου; και η Ραχήλ είπε, Λοιπόν ας κοιμηθή μετά σου ταύτην την νύκτα, διά τους μανδραγόρας του υιού σου.
\par 16 Και ήλθεν ο Ιακώβ το εσπέρας εκ του αγρού, και εξελθούσα η Λεία εις συνάντησιν αυτού, είπε, Προς εμέ θέλεις εισέλθει, διότι σε εμίσθωσα τωόντι με τους μανδραγόρας του υιού μου. Και εκοιμήθη μετ' αυτής εκείνην την νύκτα.
\par 17 Και εισήκουσεν ο Θεός της Λείας· και συνέλαβε και εγέννησεν εις τον Ιακώβ πέμπτον υιόν.
\par 18 Και είπεν η Λεία, Εδωκέ μοι ο Θεός τον μισθόν μου, διότι έδωκα την θεράπαινάν μου εις τον άνδρα μου· και εκάλεσε το όνομα αυτού Ισσάχαρ.
\par 19 Και συνέλαβεν ακόμη η Λεία, και εγέννησεν έκτον υιόν εις τον Ιακώβ·
\par 20 και είπεν η Λεία, Με επροίκισεν ο Θεός με καλήν προίκα· τώρα θέλει κατοικήσει μετ' εμού ο ανήρ μου, διότι εγέννησα εις αυτόν εξ υιούς· και εκάλεσε το όνομα αυτού Ζαβουλών.
\par 21 Και μετά ταύτα εγέννησε θυγατέρα, και εκάλεσε το όνομα αυτής Δείναν.
\par 22 Ενεθυμήθη δε ο Θεός την Ραχήλ και εισήκουσεν αυτής ο Θεός, και ήνοιξε την μήτραν αυτής·
\par 23 και συνέλαβε, και εγέννησεν υιόν· και είπεν, Ο Κύριος αφήρεσε το όνειδός μου.
\par 24 Και εκάλεσε το όνομα αυτού Ιωσήφ, λέγουσα, Ο Θεός να προσθέση εις εμέ και άλλον υιόν.
\par 25 Και αφού η Ραχήλ εγέννησε τον Ιωσήφ, είπεν ο Ιακώβ προς τον Λάβαν, Εξαπόστειλόν με, διά να απέλθω εις τον τόπον μου, και εις την πατρίδα μου·
\par 26 δος μοι τας γυναίκάς μου και τα παιδία μου, διά τας οποίας σε εδούλευσα διά να απέλθω· διότι συ γνωρίζεις την δούλευσίν μου, την οποίαν σε εδούλευσα.
\par 27 Είπε δε προς αυτόν ο Λάβαν, Παρακαλώ σε, να εύρω χάριν έμπροσθέν σου· εγνώρισα εκ πείρας, ότι ο Κύριος με ευλόγησεν εξ αιτίας σου.
\par 28 Και είπε, Διόρισόν μοι τον μισθόν σου, και θέλω σοι δώσει αυτόν.
\par 29 Ο δε είπε προς αυτόν, Συ γνωρίζεις τίνι τρόπω σε εδούλευσα, και πόσα έγειναν τα κτήνη σου μετ' εμού·
\par 30 διότι όσα είχες προ εμού ήσαν ολίγα, και τώρα ηύξησαν εις πλήθος· και ο Κύριος σε ευλόγησε με την έλευσίν μου· και τώρα πότε θέλω προβλέψει και εγώ διά τον οίκόν μου;
\par 31 Ο δε είπε, Τι να σοι δώσω; Και ο Ιακώβ είπε, δεν θέλεις μοι δώσει ουδέν· εάν κάμης εις εμέ το πράγμα τούτο, πάλιν θέλω ποιμαίνει το ποίμνιόν σου και φυλάττει αυτό·
\par 32 να περάσω σήμερον διά μέσον όλου του ποιμνίου σου, διαχωρίζων εκείθεν παν πρόβατον έχον ποικίλματα και κηλίδας, και παν το μελανωπόν μεταξύ των αρνίων, και το έχον κηλίδας και ποικίλματα μεταξύ των αιγών· και ταύτα να ήναι ο μισθός μου·
\par 33 και εις το εξής η δικαιοσύνη μου θέλει μαρτυρήσει περί εμού, όταν έλθη έμπροσθέν σου διά τον μισθόν μου· παν ό,τι δεν είναι με ποικίλματα και κηλίδας μεταξύ των αιγών, και μελανωπόν μεταξύ των αρνίων, θέλει λογισθή κλεμμένον υπ' εμού.
\par 34 Και είπεν ο Λάβαν, Ιδού, έστω κατά τον λόγον σου.
\par 35 Και την ημέραν εκείνην διεχώρισε τους τράγους τους παρδαλούς και κηλιδωτούς και πάσας τας αίγας όσαι είχον ποικίλματα και κηλίδας, πάντα όσα ήσαν διάλευκα, και πάντα τα μελανωπά μεταξύ των αρνίων, και έδωκεν αυτά εις τας χείρας των υιών αυτού·
\par 36 και έθεσε τριών ημερών οδόν μεταξύ εαυτού και του Ιακώβ· ο δε Ιακώβ εποίμαινε το υπόλοιπον του ποιμνίου του Λάβαν.
\par 37 Και έλαβεν εις εαυτόν ο Ιακώβ ράβδους χλωράς εκ λεύκης και καρύας και πλατάνου και εξελέπισεν αυτάς κατά λεπίσματα λευκά, ώστε εφαίνετο το λευκόν το εις τας ράβδους·
\par 38 και έθεσε τας ράβδους, τας οποίας εξελέπισεν, εις τα αυλάκια του ύδατος, εις τας ποτίστρας, όπου τα ποίμνια ήρχοντο να πίνωσι, διά να συλλαμβάνωσι τα ποίμνια ενώ ήρχοντο να πίνωσι.
\par 39 Και συνελάμβανον τα ποίμνια βλέποντα τας ράβδους, και εγέννων πρόβατα παρδαλά, ποικίλα και κηλιδωτά.
\par 40 Διεχώρισε δε ο Ιακώβ τα αρνία, και έστρεψε τα πρόσωπα των προβάτων του ποιμνίου του Λάβαν προς τα παρδαλά και προς πάντα τα μελανωπά· τα δε εαυτού ποίμνια έθεσε χωριστά, και δεν έθεσεν αυτά μετά των προβάτων του Λάβαν.
\par 41 Και καθ' ον καιρόν τα πρώϊμα πρόβατα ήρχοντο εις σύλληψιν, ο Ιακώβ έθετε τας ράβδους εις τα αυλάκια έμπροσθεν των οφθαλμών του ποιμνίου, διά να συλλαμβάνωσι βλέποντα προς τας ράβδους·
\par 42 ότε δε τα πρόβατα ήσαν όψιμα, δεν έθετεν αυτάς· και ούτω τα όψιμα ήσαν του Λάβαν, τα δε πρώϊμα του Ιακώβ.
\par 43 Και ηύξησεν ο άνθρωπος σφόδρα σφόδρα, και απέκτησε ποίμνια πολλά και δούλας και δούλους και καμήλους και όνους.

\chapter{31}

\par 1 Και ήκουσεν ο Ιακώβ τους λόγους των υιών του Λάβαν, λεγόντων, Ο Ιακώβ έλαβε πάντα τα υπάρχοντα του πατρός ημών, και εκ των υπαρχόντων του πατρός ημών απέκτησε πάσαν την δόξαν ταύτην.
\par 2 Και είδεν ο Ιακώβ το πρόσωπον του Λάβαν, και ιδού, δεν ήτο προς αυτόν ως χθές και προχθές.
\par 3 Είπε δε ο Κύριος προς τον Ιακώβ, Επίστρεψον εις την γην των πατέρων σου, και εις την συγγένειάν σου, και θέλω είσθαι μετά σου.
\par 4 Τότε έστειλεν ο Ιακώβ και εκάλεσε την Ραχήλ και την Λείαν εις την πεδιάδα προς το ποίμνιον αυτού·
\par 5 και είπε προς αυτάς, Βλέπω το πρόσωπον του πατρός σας, ότι δεν είναι προς εμέ ως χθές και προχθές· ο Θεός όμως του πατρός μου εστάθη μετ' εμού·
\par 6 και σεις εξεύρετε ότι εν όλη τη δυνάμει μου εδούλευσα τον πατέρα σας·
\par 7 αλλ' ο πατήρ σας με ηπάτησε και ήλλαξε τους μισθούς μου δεκάκις· πλην ο Θεός δεν αφήκεν αυτόν να με κακοποιήση·
\par 8 ότε έλεγεν ούτω, τα ποικίλα θέλουσιν είσθαι ο μισθός σου, τότε άπαν το ποίμνιον εγέννα ποικίλα· και ότε έλεγεν ούτω, τα παρδαλά θέλουσιν είσθαι ο μισθός σου, τότε άπαν το ποίμνιον εγέννα παρδαλά.
\par 9 Ούτως αφήρεσεν ο Θεός το ποίμνιον του πατρός σας και έδωκεν εις εμέ.
\par 10 Και καθ' ον καιρόν συνελάμβανε το ποίμνιον, ύψωσα τους οφθαλμούς μου και είδον κατ' όναρ, και ιδού, οι τράγοι και οι κριοί, οι αναβαίνοντες επί τα πρόβατα και τας αίγας, ήσαν παρδαλοί, ποικίλοι και στικτοί.
\par 11 Και μοι είπεν ο άγγελος του Θεού κατ' όναρ, Ιακώβ· και είπα, Ιδού, εγώ.
\par 12 Και είπεν, Ύψωσον τώρα τους οφθαλμούς σου, και ιδέ πάντας τους τράγους και τους κριούς, τους αναβαίνοντας επί τα πρόβατα και τας αίγας, ότι είναι παρδαλοί, ποικίλοι και στικτοί· διότι είδον πάντα όσα κάμνει εις σε ο Λάβαν·
\par 13 εγώ είμαι ο Θεός της Βαιθήλ, όπου έχρισας την στήλην και όπου ευχήθης ευχήν προς εμέ· σηκώθητι τώρα, έξελθε εκ της γης ταύτης και επίστρεψον εις την γην της συγγενείας σου.
\par 14 Και απεκρίθησαν η Ραχήλ και η Λεία και είπον προς αυτόν, Έχομεν ημείς πλέον μερίδιον ή κληρονομίαν εν τω οίκω του πατρός ημών;
\par 15 δεν εθεωρήθημεν υπ' αυτού ως ξέναι; διότι επώλησεν ημάς και ακόμη ολοκλήρως κατέφαγε το αργύριον ημών.
\par 16 Όθεν πάντα τα πλούτη, τα οποία αφήρεσεν ο Θεός από του πατρός ημών, είναι ημών και των τέκνων ημών· τώρα λοιπόν κάμε όσα σοι είπεν ο Θεός.
\par 17 Τότε σηκωθείς ο Ιακώβ, έβαλε τα παιδία αυτού και τας γυναίκας αυτού επί τας καμήλους·
\par 18 και απήγαγε πάντα τα κτήνη αυτού, και πάντα τα αγαθά αυτού τα οποία απέκτησε, το ποίμνιον της αποκτήσεως αυτού, το οποίον απέκτησεν εις Παδάν-αράμ, διά να απέλθη προς Ισαάκ τον πατέρα αυτού εις γην Χαναάν.
\par 19 Ο δε Λάβαν είχεν υπάγει διά να κουρεύση τα πρόβατα αυτού και η Ραχήλ έκλεψε τα είδωλα του πατρός αυτής.
\par 20 Έκρυψε δε ο Ιακώβ την φυγήν αυτού εις τον Λάβαν τον Σύρον, μη αναγγείλας προς αυτόν ότι αναχωρεί·
\par 21 και έφυγεν αυτός μετά πάντων των υπαρχόντων αυτού και εσηκώθη και διέβη τον ποταμόν και διευθύνθη προς το όρος Γαλαάδ.
\par 22 Και την τρίτην ημέραν ανηγγέλθη προς τον Λάβαν, ότι έφυγεν ο Ιακώβ·
\par 23 και παραλαβών τους αδελφούς αυτού μεθ' εαυτού, κατεδίωξεν οπίσω αυτού οδόν επτά ημερών· και επρόφθασεν αυτόν εν τω όρει Γαλαάδ.
\par 24 Ήλθε δε ο Θεός προς Λάβαν τον Σύρον κατ' όναρ την νύκτα, και είπε προς αυτόν, Φυλάχθητι, μη λαλήσης σκληρά προς τον Ιακώβ.
\par 25 Επρόφθασε λοιπόν ο Λάβαν τον Ιακώβ· ο δε Ιακώβ είχε στήσει την σκηνήν αυτού επί του όρους· ο δε Λάβαν μετά των αδελφών αυτού εσκήνωσεν επί του όρους Γαλαάδ.
\par 26 Και είπεν ο Λάβαν προς τον Ιακώβ, Τι έκαμες, και διά τι έκρυψας εις εμέ την φυγήν σου και απήγαγες τας θυγατέρας μου ως αιχμαλώτους μαχαίρας;
\par 27 διά τι έφυγες κρυφίως και έκλεψας σεαυτόν απ' εμού και δεν μοι εφανέρωσας τούτο; διότι εγώ ήθελον σε εξαποστείλει μετ' ευφροσύνης και μετά ασμάτων, μετά τυμπάνων και κιθάρας·
\par 28 και δεν με ηξίωσας μηδέ να φιλήσω τους υιούς μου, και τας θυγατέρας μου; τώρα αφρόνως έπραξας τούτο·
\par 29 δυνατή είναι η χειρ μου να σας κακοποιήση· πλην ο Θεός του πατρός σας χθές την νύκτα είπε προς εμέ, λέγων, Φυλάχθητι, μη λαλήσης σκληρά προς τον Ιακώβ·
\par 30 τώρα λοιπόν έστω, ανεχώρησας, επειδή επεθύμησας πολύ τον οίκον του πατρός σου· αλλά διά τι έκλεψας τους θεούς μου;
\par 31 Και αποκριθείς ο Ιακώβ είπε προς τον Λάβαν, Έφυγον επειδή εφοβήθην· διότι είπον, Μήπως αφαιρέσης τας θυγατέρας σου απ' εμού·
\par 32 εις όντινα όμως εύρης τους θεούς σου, ας μη ζήση· έμπροσθεν των αδελφών ημών γνώρισον τι ευρίσκεται εις εμέ εκ των ιδικών σου, και λάβε. Διότι δεν ήξευρεν ο Ιακώβ ότι η Ραχήλ είχε κλέψει αυτούς.
\par 33 Εισήλθε λοιπόν ο Λάβαν εις την σκηνήν του Ιακώβ, και εις την σκηνήν της Λείας, και εις τας σκηνάς των δύο θεραπαινών· αλλά δεν εύρηκεν αυτούς. Τότε εξήλθεν εκ της σκηνής της Λείας, και εισήλθεν εις την σκηνήν της Ραχήλ.
\par 34 Η δε Ραχήλ είχε λάβει τα είδωλα, και βάλει αυτά εις σαμάριον καμήλου, και εκάθητο επ' αυτά. Και ερευνήσας ο Λάβαν όλην την σκηνήν, δεν εύρηκεν.
\par 35 Η δε είπε προς τον πατέρα αυτής, Ας μη φανή βαρύ εις τον κύριόν μου, διότι δεν δύναμαι να σηκωθώ έμπροσθέν σου, επειδή έχω τα γυναικεία. Και αυτός ηρεύνησεν, αλλά δεν εύρηκε τα είδωλα.
\par 36 Και ωργίσθη ο Ιακώβ και επέπληξε τον Λάβαν· και αποκριθείς ο Ιακώβ είπε προς τον Λάβαν, Τι είναι το ανόμημά μου; τι το αμάρτημά μου, ότι κατεδίωξας οπίσω μου;
\par 37 αφού ηρεύνησας πάντα τα σκεύη μου, τι εύρηκας εκ πάντων των σκευών της οικίας σου; θες αυτό εδώ έμπροσθεν των αδελφών μου και αδελφών σου, διά να κρίνωσι μεταξύ των δύο ημών·
\par 38 είκοσι έτη είναι τώρα, αφ' ότου είμαι μετά σού· τα πρόβατά σου και αι αίγές σου δεν ητεκνώθησαν, και τους κριούς του ποιμνίου σου δεν έφαγον.
\par 39 θηριάλωτον δεν έφερα εις σέ· εγώ επλήρωνον αυτό· από της χειρός μου εζήτεις ό,τι με εκλέπτετο την ημέραν, ή ό,τι με εκλέπτετο την νύκτα·
\par 40 την ημέραν εκαιόμην υπό του καύματος και την νύκτα υπό του παγετού· και έφευγεν ο ύπνος μου από των οφθαλμών μου·
\par 41 είκοσι έτη ήδη ευρίσκομαι εν τη οικία σου· δεκατέσσαρα έτη σε εδούλευσα διά τας δύο σου θυγατέρας, και εξ έτη διά τα πρόβατά σου· και ήλλαξας τον μισθόν μου δεκάκις·
\par 42 εάν ο Θεός του πατρός μου, ο Θεός του Αβραάμ και ο φόβος του Ισαάκ, δεν ήτο μετ' εμού, βέβαια κενόν ήθελες με εξαποστείλει τώρα· είδεν ο Θεός την ταλαιπωρίαν μου και τον κόπον των χειρών μου, και σε ήλεγξεν εχθές την νύκτα.
\par 43 Και αποκριθείς ο Λάβαν, είπε προς τον Ιακώβ, Αι θυγατέρες αύται είναι θυγατέρες μου, και οι υιοί ούτοι υιοί μου, και τα πρόβατα ταύτα πρόβατά μου, και πάντα όσα βλέπεις είναι ιδικά μου· και τι να κάμω σήμερον εις τας θυγατέρας μου ταύτας, ή εις τα τέκνα αυτών, τα οποία εγέννησαν;
\par 44 ελθέ λοιπόν τώρα, ας κάμωμεν συνθήκην, εγώ και σύ· διά να ήναι εις μαρτύριον μεταξύ εμού και σου.
\par 45 Και έλαβεν ο Ιακώβ λίθον και έστησεν αυτόν στήλην.
\par 46 Και είπεν ο Ιακώβ προς τους αδελφούς αυτού, Συνάξατε λίθους· και έλαβον λίθους, και έκαμον σωρόν· και έφαγον εκεί επί του σωρού.
\par 47 Και ο μεν Λάβαν εκάλεσεν αυτόν Ιεγάρ-σαχαδουθά· ο δε Ιακώβ εκάλεσεν αυτόν Γαλεέδ.
\par 48 Και είπεν ο Λάβαν, Ο σωρός ούτος είναι σήμερον μαρτύριον μεταξύ εμού και σού· διά τούτο εκαλέσθη το όνομα αυτού Γαλεέδ,
\par 49 και Μισπά, διότι είπεν, Ας επιβλέψη ο Κύριος αναμέσον εμού και σου, όταν αποχωρισθώμεν ο εις από του άλλου·
\par 50 εάν ταλαιπωρήσης τας θυγατέρας μου, ή εάν λάβης άλλας γυναίκας εκτός των θυγατέρων μου, δεν είναι ουδείς μεθ' ημών· βλέπε, ο Θεός είναι μάρτυς μεταξύ εμού και σου.
\par 51 Και είπεν ο Λάβαν προς τον Ιακώβ, Ιδού, ο σωρός ούτος, και ιδού, η στήλη αύτη, την οποίαν έστησα μεταξύ εμού και σού·
\par 52 ο σωρός ούτος είναι μαρτύριον, και η στήλη μαρτύριον, ότι εγώ δεν θέλω διαβή τον σωρόν τούτον προς σε, ούτε συ θέλεις διαβή τον σωρόν τούτον και την στήλην ταύτην, προς εμέ, διά κακόν·
\par 53 ο Θεός του Αβραάμ και ο Θεός του Ναχώρ, ο Θεός του πατρός αυτών, ας κρίνη αναμέσον ημών. Ο δε Ιακώβ ώμοσεν εις τον φόβον του πατρός αυτού Ισαάκ.
\par 54 Τότε έθυσεν ο Ιακώβ θυσίαν επί του όρους και προσεκάλεσε τους αδελφούς αυτού διά να φάγωσιν άρτον· και έφαγον άρτον και διενυκτέρευσαν επί του όρους.
\par 55 Και σηκωθείς ο Λάβαν ενωρίς το πρωΐ, εφίλησε τους υιούς αυτού και τας θυγατέρας αυτού, και ευλόγησεν αυτούς· και ανεχώρησεν ο Λάβαν και επέστρεψεν εις τον τόπον αυτού.

\chapter{32}

\par 1 Και απήλθεν ο Ιακώβ εις την οδόν αυτού· και συνήντησαν αυτόν οι άγγελοι του Θεού.
\par 2 Και ότε είδεν αυτούς ο Ιακώβ είπε, Στρατόπεδον Θεού είναι τούτο· και εκάλεσε το όνομα του τόπου εκείνου, Μαχαναΐμ.
\par 3 Και απέστειλεν ο Ιακώβ μηνυτάς έμπροσθεν αυτού προς Ησαύ τον αδελφόν αυτού εις την γην Σηείρ, εις τον τόπον του Εδώμ.
\par 4 Και παρήγγειλεν εις αυτούς, λέγων, ούτω θέλετε ειπεί προς τον κύριόν μου τον Ησαύ, Ούτω λέγει ο δούλός σου Ιακώβ, μετά του Λάβαν παρώκησα, και διέμεινα έως του νύν·
\par 5 και απέκτησα βόας και όνους πρόβατα και δούλους και δούλας· και απέστειλα να αναγγείλω προς τον κύριόν μου, διά να εύρω χάριν έμπροσθέν σου.
\par 6 Και επέστρεψαν οι μηνυταί προς τον Ιακώβ, λέγοντες, Υπήγαμεν προς τον αδελφόν σου τον Ησαύ, και μάλιστα έρχεται εις συνάντησίν σου, και τετρακόσιοι άνδρες μετ' αυτού.
\par 7 Εφοβήθη δε ο Ιακώβ σφόδρα και ήτο εν αμηχανία· και διήρεσε τον λαόν, τον μεθ' αυτού, και τα ποίμνια και τους βόας και τας καμήλους, εις δύο τάγματα·
\par 8 λέγων, Εάν έλθη ο Ησαύ εις το εν τάγμα και πατάξη αυτό, το επίλοιπον τάγμα θέλει διασωθή.
\par 9 Και είπεν ο Ιακώβ, Θεέ του πατρός μου Αβραάμ και Θεέ του πατρός μου Ισαάκ, Κύριε, όστις είπας προς εμέ· Επίστρεψον εις την γην σου και εις την συγγένειάν σου και θέλω σε αγαθοποιήσει·
\par 10 πολύ μικρός είμαι ως προς πάντα τα ελέη και πάσαν την αλήθειαν τα οποία έκαμες εις τον δούλον σου· διότι με την ράβδον μου διέβην τον Ιορδάνην τούτον, και τώρα έγεινα δύο τάγματα·
\par 11 σώσον με, δέομαί σου, εκ της χειρός του αδελφού μου, εκ της χειρός του Ησαύ· διότι φοβούμαι αυτόν, μήπως ελθών πατάξη εμέ και την μητέρα επί τα τέκνα·
\par 12 συ δε είπας, Βέβαια θέλω σε αγαθοποιήσει, και θέλω καταστήσει το σπέρμα σου ως την άμμον της θαλάσσης, ήτις εκ του πλήθους δεν δύναται να αριθμηθή.
\par 13 Και εκοιμήθη εκεί την νύκτα εκείνην· και έλαβεν εκ των όσα έτυχον εν τη χειρί αυτού, δώρον προς Ησαύ τον αδελφόν αυτού·
\par 14 αίγας διακοσίας και τράγους είκοσι, πρόβατα διακόσια και κριούς είκοσι,
\par 15 καμήλους θηλαζούσας μετά των τέκνων αυτών τριάκοντα, δαμάλια τεσσαράκοντα και ταύρους δέκα, όνους θηλυκάς είκοσι και πωλάρια δέκα.
\par 16 Και παρέδωκεν εις τας χείρας των δούλων αυτού, έκαστον ποίμνιον χωριστά· και είπε προς τους δούλους αυτού, Περάσατε έμπροσθέν μου και αφήσατε διάστημα μεταξύ ποιμνίου και ποιμνίου.
\par 17 Και εις τον πρώτον παρήγγειλε, λέγων, Όταν σε συναντήση Ησαύ ο αδελφός μου, και σε ερωτήση λέγων, Τίνος είσαι; και που υπάγεις; και τίνος είναι ταύτα, τα οποία έχεις έμπροσθέν σου;
\par 18 τότε θέλεις ειπεί, Ταύτα είναι του δούλου σου του Ιακώβ, δώρα στελλόμενα προς τον κύριόν μου Ησαύ· και ιδού, και αυτός οπίσω ημών.
\par 19 ούτω παρήγγειλε και εις τον δεύτερον, και εις τον τρίτον και εις πάντας τους ακολουθούντας οπίσω των ποιμνίων, λέγων, Κατά τους λόγους τούτους θέλετε λαλήσει προς τον Ησαύ, όταν εύρητε αυτόν·
\par 20 και θέλετε ειπεί, Ιδού, οπίσω ημών και αυτός ο δούλός σου Ιακώβ. Διότι έλεγε, Θέλω εξιλεώσει το πρόσωπον αυτού με το δώρον, το προπορευόμενον έμπροσθέν μου· και μετά ταύτα θέλω ιδεί το πρόσωπον αυτού· ίσως θέλει με δεχθή.
\par 21 Το δώρον λοιπόν επέρασεν έμπροσθεν αυτού· αυτός δε έμεινε την νύκτα εκείνην εν τω στρατοπέδω.
\par 22 Σηκωθείς δε την νύκτα εκείνην, έλαβε τας δύο γυναίκας αυτού και τας δύο θεραπαίνας αυτού και τα ένδεκα παιδία αυτού και διέβη το πέρασμα του Ιαβόκ.
\par 23 Και έλαβεν αυτούς και διεβίβασεν αυτούς τον χείμαρρον· διεβίβασε και τα υπάρχοντα αυτού.
\par 24 Ο δε Ιακώβ έμεινε μόνος· και επάλαιε μετ' αυτού άνθρωπος έως τα χαράγματα της αυγής·
\par 25 ιδών δε ότι δεν υπερίσχυσε κατ' αυτού, ήγγισε την άρθρωσιν του μηρού αυτού· και μετετοπίσθη η άρθρωσις του μηρού του Ιακώβ, ενώ επάλαιε μετ' αυτού.
\par 26 Ο δε είπεν, Άφες με να απέλθω, διότι εχάραξεν η αυγή. Και αυτός είπε, δεν θέλω σε αφήσει να απέλθης, εάν δεν με ευλογήσης.
\par 27 Και είπε προς αυτόν, Τι είναι το όνομά σου; Ο δε είπεν, Ιακώβ.
\par 28 Και εκείνος είπε, Δεν θέλει καλεσθή πλέον το όνομά σου Ιακώβ, αλλά Ισραήλ· διότι ενίσχυσας μετά Θεού, και μετά ανθρώπων θέλεις είσθαι δυνατός.
\par 29 Ηρώτησε δε ο Ιακώβ λέγων, Φανέρωσόν μοι, παρακαλώ, το όνομά σου. Ο δε είπε, Διά τι ερωτάς το όνομά μου; Και ευλόγησεν αυτόν εκεί.
\par 30 Και εκάλεσεν Ιακώβ το όνομα του τόπου εκείνον Φανουήλ, λέγων, Διότι είδον τον Θεόν πρόσωπον προς πρόσωπον, και εφυλάχθη η ζωή μου.
\par 31 Και ανέτειλεν ο ήλιος επ' αυτού, καθώς διέβη το Φανουήλ· εχώλαινε δε κατά τον μηρόν αυτού.
\par 32 Διά τούτο μέχρι της σήμερον δεν τρώγουσιν οι υιοί του Ισραήλ τον ναρκωθέντα μυώνα, όστις είναι επί της αρθρώσεως του μηρού· διότι εκείνος ήγγισε την άρθρωσιν του μηρού του Ιακώβ κατά τον μυώνα τον ναρκωθέντα.

\chapter{33}

\par 1 Αναβλέψας δε ο Ιακώβ είδε· και ιδού, ο Ησαύ ήρχετο, και μετ' αυτού τετρακόσιοι άνδρες· και εμοίρασεν ο Ιακώβ τα παιδία εις την Λείαν και εις την Ραχήλ και εις τας δύο θεραπαίνας.
\par 2 Και τας μεν θεραπαίνας και τα τέκνα αυτών έβαλεν έμπροσθεν, την δε Λείαν και τα τέκνα αυτής, κατόπιν, και την Ραχήλ και τον Ιωσήφ, τελευταίους.
\par 3 Αυτός δε επέρασεν έμπροσθεν αυτών και προσεκύνησεν έως εδάφους επτάκις, έως να πλησιάση εις τον αδελφόν αυτού.
\par 4 Και έδραμεν ο Ησαύ εις συνάντησιν αυτού και ενηγκαλίσθη αυτόν και έπεσεν επί τον τράχηλον αυτού και κατεφίλησεν αυτόν· και έκλαυσαν.
\par 5 Και αναβλέψας είδε τας γυναίκας και τα παιδία· και είπε, Τι σου είναι ούτοι; Ο δε είπε τα παιδία, τα οποία εχάρισεν ο Θεός εις τον δούλον σου.
\par 6 Τότε επλησίασαν αι θεράπαιναι, αυταί και τα τέκνα αυτών, και προσεκύνησαν·
\par 7 παρομοίως επλησίασαν και η Λεία και τα τέκνα αυτής, και προσεκύνησαν· και μετά ταύτα επλησίασαν ο Ιωσήφ και η Ραχήλ και προσεκύνησαν.
\par 8 Και είπε, Προς τι άπαν το στρατόπεδόν σου τούτο, το οποίον απήντησα; Ο δε είπε, διά να εύρω χάριν έμπροσθεν του κυρίου μου.
\par 9 Και είπεν ο Ησαύ, Έχω πολλά, αδελφέ μου· έχε συ τα ιδικά σου.
\par 10 Και είπεν ο Ιακώβ, Ουχί, παρακαλώ· εάν εύρηκα χάριν έμπροσθέν σου, δέξαι το δώρον μου εκ των χειρών μου· διότι διά τούτο είδον το πρόσωπόν σου, ως εάν έβλεπον πρόσωπον Θεού, και συ ευηρεστήθης εις εμέ·
\par 11 δέξαι, παρακαλώ, τας ευλογίας μου, τας προσφερομένας προς σέ· διότι με ηλέησεν ο Θεός και έχω τα πάντα. Και εβίασεν αυτόν και εδέχθη.
\par 12 Και είπεν, Ας σηκωθώμεν και ας υπάγωμεν, και εγώ θέλω προπορεύεσθαι έμπροσθέν σου.
\par 13 Και είπε προς αυτόν ο Ιακώβ, Ο κύριός μου εξεύρει ότι τα παιδία είναι τρυφερά, και έχω μετ' εμού εγκυμονούντα πρόβατα και βόας· και εάν βιάσωσιν αυτά μίαν μόνην ημέραν, άπαν το ποίμνιον θέλει αποθάνει.
\par 14 Ας περάση, παρακαλώ, ο κύριός μου έμπροσθεν του δούλου αυτού· και εγώ θέλω ακολουθεί βραδέως, κατά το βάδισμα των κτηνών των έμπροσθέν μου, και κατά το βάδισμα των παιδαρίων, εωσού φθάσω προς τον κύριόν μου εις Σηείρ.
\par 15 Και είπεν ο Ησαύ, Ας αφήσω λοιπόν μετά σου μέρος εκ του λαού, του μετ' εμού. Ο δε είπε, Διά τι τούτο; αρκεί ότι εύρηκα χάριν έμπροσθεν του κυρίου μου.
\par 16 Επέστρεψε λοιπόν ο Ησαύ την ημέραν εκείνην εις την οδόν αυτού εις Σηείρ.
\par 17 Και απήλθεν ο Ιακώβ εις Σοκχώθ, και ωκοδόμησεν εις εαυτόν οικίαν, και διά τα κτήνη αυτού έκαμε σκηνάς· διά τούτο εκάλεσε το όνομα του τόπου Σοκχώθ.
\par 18 Και αφού επέστρεψεν ο Ιακώβ από Παδάν-αράμ, ήλθεν εις Σαλήμ, πόλιν Συχέμ, την εν τη γη Χαναάν· και κατεσκήνωσεν έμπροσθεν της πόλεως.
\par 19 Και ηγόρασε την μερίδα του αγρού, όπου έστησε την σκηνήν αυτού, παρά των υιών του Εμμώρ, πατρός του Συχέμ, δι' εκατόν αργύρια.
\par 20 Και έστησεν εκεί θυσιαστήριον, και εκάλεσεν αυτό Ελ-ελωέ-Ισραήλ.

\chapter{34}

\par 1 Και εξήλθε Δείνα η θυγάτηρ της Λείας, την οποίαν εγέννησεν εις τον Ιακώβ, διά να ίδη τας θυγατέρας του τόπου.
\par 2 Και ιδών αυτήν Συχέμ, ο υιός του Εμμώρ του Ευαίου, άρχοντος του τόπου, έλαβεν αυτήν, και εκοιμήθη μετ' αυτής και εταπείνωσεν αυτήν.
\par 3 Και η ψυχή αυτού προσεκολλήθη εις την Δείναν, την θυγατέρα του Ιακώβ· και ηγάπησε την κόρην και ελάλησε κατά την καρδίαν της κόρης.
\par 4 Και είπεν ο Συχέμ προς Εμμώρ τον πατέρα αυτού, λέγων, Λάβε μοι την κόρην ταύτην εις γυναίκα.
\par 5 Και ήκουσεν ο Ιακώβ, ότι εμίανε την Δείναν την θυγατέρα αυτού· οι δε υιοί αυτού ήσαν μετά των κτηνών αυτού εν τω αγρώ· και παρεσιώπησεν ο Ιακώβ εωσού έλθωσιν.
\par 6 Εμμώρ δε, ο πατήρ του Συχέμ, εξήλθε προς τον Ιακώβ, διά να ομιλήση μετ' αυτού.
\par 7 Και ήλθον οι υιοί του Ιακώβ εκ του αγρού, καθώς ήκουσαν τούτο· και ηγανάκτησαν οι άνδρες και εθυμώθησαν σφόδρα, ότι έπραξεν αισχρά εις τον Ισραήλ, κοιμηθείς μετά της θυγατρός του Ιακώβ· το οποίον δεν έπρεπε να γείνη.
\par 8 Και ελάλησε προς αυτούς ο Εμμώρ, λέγων, Η ψυχή του Συχέμ του υιού μου προσηλώθη εις την θυγατέρα σας· δότε αυτήν εις αυτόν, παρακαλώ, εις γυναίκα·
\par 9 και συμπενθερεύσατε μεθ' ημών· τας θυγατέρας σας δότε εις ημάς, και τας θυγατέρας ημών λάβετε εις εαυτούς·
\par 10 και κατοικήσατε μεθ' ημών· ιδού, η γη είναι έμπροσθέν σας· κατοικείτε και εμπορεύεσθε επ' αυτής και κάμετε κτήματα εν αυτή.
\par 11 Είπε δε ο Συχέμ προς τον πατέρα αυτής και προς τους αδελφούς αυτής, Ας εύρω χάριν έμπροσθέν σας· και ό,τι είπητε εις εμέ θέλω δώσει·
\par 12 ζητήσατε παρ' εμού όσην προίκα θέλετε, και όσα χαρίσματα, και θέλω δώσει αυτά, καθώς ηθέλετε μοι ειπεί· μόνον δότε μοι την κόρην εις γυναίκα.
\par 13 Απεκρίθησαν δε οι υιοί του Ιακώβ προς τον Συχέμ και προς τον Εμμώρ τον πατέρα αυτού, μετά δόλου· και ελάλησαν επειδή αυτός είχε μιάνει την Δείναν την αδελφήν αυτών
\par 14 και είπον προς αυτούς, Δεν δυνάμεθα να κάμωμεν το πράγμα τούτο, να δώσωμεν την αδελφήν ημών εις άνθρωπον απερίτμητον· διότι τούτο είναι όνειδος εις ημάς·
\par 15 επί τούτω μόνον θέλομεν συμφωνήσει με σάς· Εάν σεις γείνετε ως ημείς, περιτέμνοντες παν αρσενικόν μεταξύ σας,
\par 16 τότε θέλομεν δώσει τας θυγατέρας ημών εις εσάς, και τας θυγατέρας σας θέλομεν λάβει εις ημάς, και θέλομεν κατοικήσει με σας και θέλομεν γείνει εις λαός·
\par 17 εάν όμως δεν μας ακούσητε να περιτμηθήτε, τότε θέλομεν λάβει την θυγατέρα ημών και θέλομεν αναχωρήσει.
\par 18 Και ήρεσαν οι λόγοι αυτών εις τον Εμμώρ και εις τον Συχέμ τον υιόν του Εμμώρ·
\par 19 και δεν εβράδυνεν ο νέος να κάμη το πράγμα, διότι υπερηγάπα την θυγατέρα του Ιακώβ· και ήτο ο ενδοξότερος παντός του οίκου του πατρός αυτού.
\par 20 Και ήλθεν ο Εμμώρ και ο Συχέμ ο υιός αυτού εις την πύλην της πόλεως αυτών, και ελάλησαν προς τους άνδρας της πόλεως αυτών λέγοντες,
\par 21 Οι άνθρωποι ούτοι είναι ειρηνικοί μεθ' ημών· ας κατοικήσωσι λοιπόν εν τη γη και ας εμπορεύωνται εν αυτή· διότι η γη, ιδού, είναι αρκετά ευρύχωρος δι' αυτούς· τας θυγατέρας αυτών ας λάβωμεν εις γυναίκας, και τας θυγατέρας ημών ας δώσωμεν εις αυτούς·
\par 22 επί τούτω μόνον θέλουσι συμφωνήσει με ημάς οι άνθρωποι διά να κατοικήσωσι μεθ' ημών, ώστε να γείνωμεν εις λαός, εάν περιτμηθή παν αρσενικόν μεταξύ ημών, καθώς αυτοί περιτέμνονται·
\par 23 τα ποίμνια αυτών και τα υπάρχοντα αυτών και πάντα τα κτήνη αυτών δεν θέλουσιν είσθαι ιδικά μας; μόνον ας συμφωνήσωμεν με αυτούς, και θέλουσι κατοικήσει μεθ' ημών.
\par 24 Και εισήκουσαν του Εμμώρ και Συχέμ του υιού αυτού πάντες οι εξερχόμενοι εκ της πύλης της πόλεως αυτού· και περιετμήθη παν αρσενικόν, πάντες οι εξερχόμενοι διά της πύλης της πόλεως αυτού.
\par 25 Την δε τρίτην ημέραν, ότε ήσαν εν τω πόνω, δύο εκ των υιών του Ιακώβ, ο Συμεών και ο Λευΐ, αδελφοί της Δείνας, έλαβον έκαστος την μάχαιραν αυτού, και εισήλθον εις την πόλιν ασφαλώς και εφόνευσαν παν αρσενικόν.
\par 26 Και τον Εμμώρ και τον Συχέμ τον υιόν αυτού εφόνευσαν εν στόματι μαχαίρας· και έλαβον την Δείναν εκ του οίκου του Συχέμ και εξήλθον.
\par 27 Οι δε υιοί του Ιακώβ ήλθον επί τους πεφονευμένους και διήρπασαν την πόλιν, επειδή είχον μιάνει την αδελφήν αυτών.
\par 28 Έλαβον τα πρόβατα αυτών και τους βόας αυτών και τους όνους αυτών και ό,τι ήτο εν τη πόλει και ό,τι εν τω αγρώ·
\par 29 και πάσαν την περιουσίαν αυτών και πάντα τα παιδία αυτών και τας γυναίκας αυτών ηχμαλώτισαν· και παν ό,τι ευρίσκετο εν ταις οικίαις διήρπασαν.
\par 30 Είπε δε ο Ιακώβ προς τον Συμεών και προς τον Λευΐ, Εις ταραχήν με εβάλετε, κάμνοντές με μισητόν μεταξύ των κατοίκων της γης, μεταξύ των Χαναναίων και Φερεζαίων· εγώ δε ολίγους ανθρώπους έχω, και εκείνοι θέλουσι συναχθή εναντίον μου και θέλουσι με πατάξει και θέλω απολεσθή εγώ και ο οίκός μου.
\par 31 Οι δε είπον, Έπρεπε λοιπόν την αδελφήν ημών να μεταχειρισθώσιν ως πόρνην;

\chapter{35}

\par 1 Και είπεν ο Θεός προς τον Ιακώβ, Σηκωθείς ανάβηθι εις Βαιθήλ και κατοίκησον εκεί· και κάμε εκεί θυσιαστήριον εις τον Θεόν, όστις εφάνη εις σε ότε έφευγες από προσώπου Ησαύ του αδελφού σου.
\par 2 Και είπεν ο Ιακώβ προς τον οίκον αυτού και προς πάντας τους μεθ' εαυτού, Εκβάλετε τους θεούς τους ξένους τους μεταξύ σας, και καθαρίσθητε και αλλάξατε τα ενδύματά σας·
\par 3 και σηκωθέντες, ας αναβώμεν εις Βαιθήλ· και εκεί θέλω κάμει θυσιαστήριον εις τον Θεόν, όστις μου επήκουσεν εν τη ημέρα της θλίψεώς μου και ήτο μετ' εμού εν τη οδώ, καθ' ην επορευόμην.
\par 4 Και έδωκαν εις τον Ιακώβ πάντας τους ξένους θεούς, όσοι ήσαν εις τας χείρας αυτών, και τα ενώτια τα εις τα ωτία αυτών· και έκρυψεν αυτά ο Ιακώβ υπό την δρυν, την πλησίον της Συχέμ.
\par 5 Μετά ταύτα ανεχώρησαν· και επέπεσε τρόμος του Θεού επί τας πόλεις τας κύκλω αυτών· και δεν κατεδίωξαν οπίσω των υιών του Ιακώβ.
\par 6 Ήλθε δε ο Ιακώβ εις Λούζ, την εν τη γη Χαναάν, ήτις είναι η Βαιθήλ, αυτός και πας ο λαός ο μετ' αυτού.
\par 7 Και ωκοδόμησεν εκεί θυσιαστήριον, και εκάλεσε το όνομα του τόπου Ελ-βαιθήλ· διότι εκεί εφανερώθη εις αυτόν ο Θεός, ότε έφευγεν από προσώπου του αδελφού αυτού.
\par 8 Απέθανε δε η Δεβόρρα, η τροφός της Ρεβέκκας, και ετάφη παρακάτω της Βαιθήλ, υπό την δρύν· και ωνομάσθη η δρυς Αλλόν-βακούθ.
\par 9 Εφάνη δε πάλιν ο Θεός εις τον Ιακώβ, αφού επέστρεψεν από Παδάν-αράμ, και ευλόγησεν αυτόν.
\par 10 Και είπε προς αυτόν ο Θεός, Το όνομά σου είναι Ιακώβ· δεν θέλεις ονομάζεσθαι πλέον Ιακώβ, αλλά Ισραήλ θέλει είσθαι το όνομα σου· και εκάλεσε το όνομα αυτού Ισραήλ.
\par 11 Είπε δε προς αυτόν ο Θεός, Εγώ είμαι ο Θεός ο Παντοκράτωρ· αυξάνου και πληθύνου· έθνος, και πλήθος εθνών θέλουσι γείνει εκ σου, και βασιλείς θέλουσιν εξέλθει εκ της οσφύος σου·
\par 12 και την γην, την οποίαν έδωκα εις τον Αβραάμ και εις τον Ισαάκ, εις σε θέλω δώσει αυτήν· και εις το σπέρμα σου μετά σε θέλω δώσει την γην ταύτην.
\par 13 Και ανέβη ο Θεός απ' αυτού, εκ του τόπου όπου ελάλησε μετ' αυτού.
\par 14 Και έστησεν ο Ιακώβ στήλην εν τω τόπω όπου ελάλησε μετ' αυτού, στήλην λιθίνην· και έκαμεν επ' αυτήν σπονδήν και επέχυσεν επ' αυτήν έλαιον.
\par 15 Και εκάλεσεν ο Ιακώβ το όνομα του τόπου, όπου ελάλησε μετ' αυτού ο Θεός, Βαιθήλ.
\par 16 Μετά ταύτα ανεχώρησαν από Βαιθήλ· και ενώ έμενεν ολίγον διάστημα διά να φθάσωσιν εις Εφραθά, εγέννησεν η Ραχήλ· και υπέφερε μεγάλον αγώνα εις την γένναν αυτής.
\par 17 Ενώ δε ευρίσκετο εις τον σκληρόν αγώνα της γέννας, είπε προς αυτήν η μαία, Μη φοβού, διότι και ούτος σου είναι υιός·
\par 18 και ενώ παρέδιδε την ψυχήν διότι απέθανεν, εκάλεσε το όνομα αυτού Βεν-ονί· ο δε πατήρ αυτού εκάλεσεν αυτόν Βενιαμίν.
\par 19 Και απέθανεν η Ραχήλ και ετάφη εν τη οδώ της Εφραθά, ήτις είναι Βηθλεέμ.
\par 20 Και έστησεν ο Ιακώβ στήλην επί του τάφου αυτής· αύτη είναι η στήλη του τάφου της Ραχήλ μέχρι της σήμερον.
\par 21 Σηκωθείς δε ο Ισραήλ, έστησε την σκηνήν αυτού πέραν του Μιγδώλ-εδέρ.
\par 22 Και ότε κατώκει ο Ισραήλ εν τη γη εκείνη, υπήγεν ο Ρουβήν και εκοιμήθη μετά της Βαλλάς παλλακής του πατρός αυτού· και ήκουσε τούτο ο Ισραήλ. Ήσαν δε οι υιοί του Ιακώβ δώδεκα·
\par 23 οι υιοί της Λείας, Ρουβήν, ο πρωτότοκος του Ιακώβ, και Συμεών και Λευΐ και Ιούδας και Ισσάχαρ και Ζαβουλών·
\par 24 οι υιοί της Ραχήλ, Ιωσήφ και Βενιαμίν·
\par 25 οι δε υιοί της Βαλλάς, θεραπαίνης της Ραχήλ, Δαν και Νεφθαλί·
\par 26 και οι υιοί της Ζελφάς, θεραπαίνης της Λείας, Γαδ και Ασήρ· ούτοι είναι οι υιοί του Ιακώβ, οίτινες εγεννήθησαν εις αυτόν εν Παδάν-αράμ.
\par 27 Ήλθε δε ο Ιακώβ προς Ισαάκ τον πατέρα αυτού εις Μαμβρή, εις Κιριάθ-αρβά, ήτις είναι η Χεβρών, όπου ο Αβραάμ και ο Ισαάκ είχον παροικήσει.
\par 28 Και ήσαν αι ημέραι του Ισαάκ εκατόν ογδοήκοντα έτη.
\par 29 Και εκπνεύσας ο Ισαάκ απέθανε και προσετέθη εις τον λαόν αυτού, γέρων και πλήρης ημερών· και έθαψαν αυτόν Ησαύ και Ιακώβ οι υιοί αυτού.

\chapter{36}

\par 1 Αύτη δε είναι η γενεαλογία του Ησαύ, όστις είναι ο Εδώμ.
\par 2 Ο Ησαύ έλαβε γυναίκας εις εαυτόν εκ των θυγατέρων Χαναάν· την Αδά, θυγατέρα Αιλών του Χετταίου, και την Ολιβαμά, θυγατέρα του Ανά, εγγονήν Σεβεγών του υαίου·
\par 3 και την Βασεμάθ, θυγατέρα του Ισμαήλ, αδελφήν του Νεβαϊώθ.
\par 4 Εγέννησε δε εις τον Ησαύ η Αδά τον Ελιφάς· και η Βασεμάθ εγέννησε τον Ραγουήλ·
\par 5 και η Ολιβαμά εγέννησε τον Ιεούς και τον Ιεγλόμ και τον Κορέ. Ούτοι είναι οι υιοί του Ησαύ, οι γεννηθέντες εις αυτόν εν τη γη Χαναάν.
\par 6 Έλαβε δε ο Ησαύ τας γυναίκας αυτού και τους υιούς αυτού και τας θυγατέρας αυτού και πάντας τους ανθρώπους του οίκου αυτού και τα ποίμνια αυτού και πάντα τα κτήνη αυτού και πάντα τα υπάρχοντα τα οποία απέκτησεν εν γη Χαναάν, και υπήγεν εις άλλην γην μακράν από Ιακώβ του αδελφού αυτού·
\par 7 διότι τα υπάρχοντα αυτών ήσαν τόσον πολλά, ώστε δεν ηδύναντο να κατοικήσωσιν ομού· και δεν ηδύνατο η γη της παροικήσεως αυτών να χωρέση αυτούς, εξ αιτίας των κτηνών αυτών.
\par 8 Κατώκησε δε ο Ησαύ εν τω όρει Σηείρ· ο Ησαύ είναι ο Εδώμ.
\par 9 Και αύτη είναι η γενεαλογία του Ησαύ, πατρός των Εδωμιτών, εν τω όρει Σηείρ·
\par 10 ταύτα είναι τα ονόματα των υιών του Ησαύ· Ελιφάς ο υιός της Αδά γυναικός του Ησαύ, Ραγουήλ ο υιός της Βασεμάθ γυναικός του Ησαύ.
\par 11 Και οι υιοί του Ελιφάς ήσαν Θαιμάν, Ωμάρ, Σωφάρ και Γοθώμ και Κενέζ.
\par 12 Η δε Θαμνά ήτο παλλακή του Ελιφάς υιού του Ησαύ, και εγέννησεν εις τον Ελιφάς τον Αμαλήκ· ούτοι ήσαν οι υιοί της Αδά γυναικός του Ησαύ.
\par 13 Και ούτοι είναι οι υιοί του Ραγουήλ· Ναχάθ και Ζερά, Σομέ και Μοζέ· ούτοι ήσαν οι υιοί της Βασεμάθ γυναικός του Ησαύ.
\par 14 Και ούτοι ήσαν οι υιοί της Ολιβαμά θυγατρός του Ανά, εγγόνης του Σεβεγών, της γυναικός του Ησαύ· και εγέννησεν εις τον Ησαύ τον Ιεούς και τον Ιεγλόμ και τον Κορέ.
\par 15 Ούτοι ήσαν οι ηγεμόνες των υιών Ησαύ· οι υιοί του Ελιφάς πρωτοτόκου του Ησαύ, ηγεμών Θαιμάν, ηγεμών Ωμάρ, ηγεμών Σωφάρ, ηγεμών Κενέζ,
\par 16 ηγεμών Κορέ, ηγεμών Γοθώμ, ηγεμών Αμαλήκ· ούτοι είναι οι ηγεμόνες του Ελιφάς εν τη γη Εδώμ· ούτοι ήσαν οι υιοί της Αδά.
\par 17 Και ούτοι ήσαν οι υιοί του Ραγουήλ υιού του Ησαύ· ηγεμών Ναχάθ, ηγεμών Ζερά, ηγεμών Σομέ, ηγεμών Μοζέ· ούτοι είναι οι ηγεμόνες του Ραγουήλ εν τη γη Εδώμ· ούτοι ήσαν οι υιοί της Βασεμάθ γυναικός του Ησαύ.
\par 18 Και ούτοι ήσαν οι υιοί της Ολιβαμά γυναικός του Ησαύ· ηγεμών Ιεούς, ηγεμών Ιεγλόμ, ηγεμών Κορέ· ούτοι ήσαν οι ηγεμόνες της Ολιβαμά θυγατρός του Ανά, γυναικός του Ησαύ.
\par 19 Ούτοι είναι οι υιοί του Ησαύ, όστις είναι ο Εδώμ· και ούτοι οι ηγεμόνες αυτών.
\par 20 Ούτοι είναι οι υιοί του Σηείρ του Χορραίου, οίτινες κατώκουν την γήν· Λωτάν και Σωβάλ και Σεβεγών και Ανά,
\par 21 και Δησών και Εσέρ και Δισάν· ούτοι είναι οι ηγεμόνες των Χορραίων, των υιών του Σηείρ, εν τη γη Εδώμ.
\par 22 Οι δε υιοί του Λωτάν ήσαν Χορρί και Αιμάμ· αδελφή δε του Λωτάν, η Θαμνά.
\par 23 Ούτοι δε ήσαν οι υιοί του Σωβάλ· Αλβάν και Μαναχάθ και Εβάλ, Σεφώ και Ωνάμ.
\par 24 Ούτοι δε ήσαν οι υιοί του Σεβεγών· και Αϊέ και Ανά· ούτος είναι ο Ανά όστις εύρηκε τα ύδατα εν τη ερήμω, ότε έβοσκε τους όνους Σεβεγών του πατρός αυτού.
\par 25 Ούτοι δε ήσαν οι υιοί του Ανά· Δησών και Ολιβαμά η θυγάτηρ του Ανά.
\par 26 Ούτοι δε ήσαν οι υιοί του Δησών· Αμαδάν και Ασβάν και Ιθράμ και Χαρράν.
\par 27 Ούτοι ήσαν οι υιοί του Εσέρ· Βαλαάν και Ζααβάν και Ακάν.
\par 28 Ούτοι ήσαν οι υιοί του Δισάν· Ουζ και Αράν.
\par 29 Ούτοι είναι οι ηγεμόνες των Χορραίων· ηγεμών Λωτάν, ηγεμών Σωβάλ, ηγεμών Σεβεγών, ηγεμών Ανά,
\par 30 ηγεμών Δησών, ηγεμών Εσέρ, ηγεμών Δισάν· ούτοι είναι οι ηγεμόνες των Χορραίων μεταξύ των ηγεμόνων αυτών εν τη γη Σηείρ.
\par 31 Και ούτοι είναι οι βασιλείς οίτινες εβασίλευσαν εν τη γη Εδώμ, πριν βασιλεύση βασιλεύς επί τους υιούς Ισραήλ.
\par 32 Και εβασίλευσεν εν Εδώμ Βελά, ο υιός του Βεώρ· το δε όνομα της πόλεως αυτού ήτο Δενναβά.
\par 33 Και απέθανεν ο Βελά και εβασίλευσεν αντ' αυτού ο Ιωβάβ, υιός του Ζερά, εκ Βοσόρρας·
\par 34 Και απέθανεν ο Ιωβάβ και εβασίλευσεν αντ' αυτού ο Χουσάμ εκ της γης των Θαιμανιτών.
\par 35 Και απέθανεν ο Χουσάμ και εβασίλευσεν αντ' αυτού ο Αδάδ, υιός του Βεράδ, ο πατάξας τους Μαδιανίτας εν τη πεδιάδι Μωάβ· το δε όνομα της πόλεως αυτού ήτο Αβίθ.
\par 36 Και απέθανεν ο Αδάδ, και εβασίλευσεν αντ' αυτού ο Σαμλά εκ Μασρεκάς.
\par 37 Και απέθανεν ο Σαμλά και εβασίλευσεν αντ' αυτού ο Σαούλ εκ Ρεχωβώθ της παρά τον ποταμόν.
\par 38 Και απέθανεν ο Σαούλ και εβασίλευσεν αντ' αυτού Βάαλ-ανάν, ο υιός του Αχβώρ.
\par 39 Και απέθανεν ο Βάαλ-ανάν, υιός του Αχβώρ, και εβασίλευσεν αντ' αυτού ο Χαδδάρ· το δε όνομα της πόλεως αυτού ήτο Παού· και το όνομα της γυναικός αυτού, Μεεταβεήλ, θυγάτηρ του Ματραίδ, εγγονή του Μαιζαάβ.
\par 40 Και ταύτα είναι τα ονόματα των ηγεμόνων του Ησαύ, κατά τας οικογενείας αυτών, κατά τους τόπους αυτών, κατά τα ονόματα αυτών. ηγεμών Θαμνά, ηγεμών Αλβά, ηγεμών Ιεθέθ,
\par 41 ηγεμών Ολιβαμά, ηγεμών Ηλά, ηγεμών Φινών,
\par 42 ηγεμών Κενέζ, ηγεμών Θαιμάν, ηγεμών Μιβσάρ,
\par 43 ηγεμών Μαγεδιήλ, ηγεμών Ιράμ· ούτοι είναι οι ηγεμόνες του Εδώμ, κατά τας κατοικίας αυτών εν τη γη της κτήσεως αυτών· ούτος είναι ο Ησαύ, ο πατήρ των Εδωμιτών.

\chapter{37}

\par 1 Κατώκησε δε ο Ιακώβ εν τη γη, εν ή παρώκησεν ο πατήρ αυτού, εν τη γη Χαναάν.
\par 2 Αύτη είναι γενεαλογία του Ιακώβ. Ο Ιωσήφ, νέος ων ετών δεκαεπτά, εποίμαινε τα πρόβατα μετά των αδελφών αυτού, των υιών της Βαλλάς και των υιών της Ζελφάς, των γυναικών του πατρός αυτού· και ανέφερεν ο Ιωσήφ προς τον πατέρα αυτών την κακήν αυτών φήμην.
\par 3 Ο δε Ισραήλ ηγάπα τον Ιωσήφ υπέρ πάντας τους υιούς αυτού, διότι ήτο υιός του γήρατος αυτού· και έκαμεν εις αυτόν χιτώνα ποικιλόχρωμον.
\par 4 Βλέποντες δε οι αδελφοί αυτού, ότι αυτόν ηγάπα ο πατήρ αυτών υπέρ πάντας τους αδελφούς αυτού, εμίσησαν αυτόν και δεν ηδύναντο να ομιλώσι προς αυτόν ειρηνικώς.
\par 5 Ενυπνιασθείς δε ο Ιωσήφ ενύπνιον, διηγήθη αυτό εις τους αδελφούς αυτού· και εμίσησαν αυτόν έτι μάλλον.
\par 6 Και είπε προς αυτούς, Ακούσατε, παρακαλώ, το ενύπνιον τούτο το οποίον ενυπνιάσθην.
\par 7 Ιδού, ημείς εδένομεν δεμάτια εν μέσω της πεδιάδος· και ιδού, εσηκώθη το ιδικόν μου δεμάτιον και εστάθη όρθιον· και ιδού, τα ιδικά σας δεμάτια περιστραφέντα προσεκύνησαν το ιδικόν μου δεμάτιον.
\par 8 Είπον δε προς αυτόν οι αδελφοί αυτού, Βασιλεύς θέλεις γείνει εφ' ημάς; ή κύριος θέλεις γείνει εις ημάς; Και εμίσησαν αυτόν έτι μάλλον διά τα ενύπνια αυτού και διά τους λόγους αυτού.
\par 9 Ενυπνιάσθη δε και άλλο ενύπνιον, και διηγήθη αυτό προς τους αδελφούς αυτού· και είπεν, Ιδού, ενυπνιάσθην άλλο ενύπνιον· και ιδού, ο ήλιος και η σελήνη και ένδεκα αστέρες με προσεκύνουν.
\par 10 Και διηγήθη αυτό προς τον πατέρα αυτού και προς τους αδελφούς αυτού και επέπληξεν αυτόν ο πατήρ αυτού και είπε προς αυτόν, Τι είναι το ενύπνιον τούτο, το οποίον ενυπνιάσθης; άραγε θέλομεν ελθεί, εγώ και η μήτηρ σου και οι αδελφοί σου, διά να σε προσκυνήσωμεν έως εδάφους;
\par 11 Και εφθόνησαν αυτόν οι αδελφοί αυτού· ο δε πατήρ αυτού εφύλαττε τον λόγον.
\par 12 Και υπήγαν οι αδελφοί αυτού να βοσκήσωσι τα πρόβατα του πατρός αυτών εις Συχέμ.
\par 13 Και είπεν ο Ισραήλ προς τον Ιωσήφ, Δεν βόσκουσιν οι αδελφοί σου εν Συχέμ; ελθέ να σε στείλω προς αυτούς. Ο δε είπε προς αυτόν, Ιδού, εγώ.
\par 14 Και είπε προς αυτόν, Ύπαγε λοιπόν να ίδης, αν ήναι καλά οι αδελφοί σου και καλά τα πρόβατα, και φέρε μοι είδησιν. Και απέστειλεν αυτόν από της κοιλάδος της Χεβρών· και ήλθεν εις Συχέμ.
\par 15 Και εύρηκεν αυτόν άνθρωπός τις, ενώ περιεπλανάτο εν τη πεδιάδι· και ηρώτησεν αυτόν ο άνθρωπος, λέγων, Τι ζητείς;
\par 16 Ο δε είπε, Τους αδελφούς μου ζητώ· ειπέ μοι, παρακαλώ, που βόσκουσι.
\par 17 Και είπεν ο άνθρωπος, Ανεχώρησαν από εδώ· διότι ήκουσα αυτούς λέγοντας, Ας υπάγωμεν εις Δωθάν. Και υπήγεν ο Ιωσήφ κατόπιν των αδελφών αυτού, και εύρηκεν αυτούς εν Δωθάν.
\par 18 Οι δε ιδόντες αυτόν μακρόθεν, πριν πλησιάση εις αυτούς, συνεβουλεύθησαν κατ' αυτού να φονεύσωσιν αυτόν.
\par 19 Και είπεν ο εις προς τον άλλον, Ιδού, έρχεται εκείνος ο κύριος των ενυπνίων·
\par 20 έλθετε λοιπόν τώρα και ας φονεύσωμεν αυτόν και ας ρίψωμεν αυτόν εις ένα εκ των λάκκων· και θέλομεν ειπεί, Θηρίον κακόν κατέφαγεν αυτόν· και θέλομεν ιδεί τι θέλουσι γείνει τα ενύπνια αυτού.
\par 21 Και ακούσας ο Ρουβήν ηλευθέρωσεν αυτόν εκ των χειρών αυτών, λέγων, Ας μη βλάψωμεν αυτόν εις την ζωήν.
\par 22 Και είπε προς αυτούς ο Ρουβήν, Μη χύσητε αίμα· ρίψατε αυτόν εις τούτον τον λάκκον, τον εν τη ερήμω, και χείρα μη βάλητε επ' αυτόν· διά να ελευθερώση αυτόν εκ των χειρών αυτών, και να αποδώση αυτόν εις τον πατέρα αυτού.
\par 23 Ότε λοιπόν ήλθεν ο Ιωσήφ προς τους αδελφούς αυτού, εξέδυσαν τον Ιωσήφ τον χιτώνα αυτού, τον χιτώνα τον ποικιλόχρωμον, τον επ' αυτόν·
\par 24 και λαβόντες αυτόν, έρριψαν εις τον λάκκον· ο δε λάκκος ήτο κενός· δεν είχεν ύδωρ.
\par 25 Έπειτα εκάθησαν να φάγωσιν άρτον, και αναβλέψαντες είδον· και ιδού, συνοδία Ισμαηλιτών ήρχετο από Γαλαάδ μετά των καμήλων αυτών φορτωμένων αρώματα και βάλσαμον και μύρον, και επορεύοντο να φέρωσιν αυτά κάτω εις την Αίγυπτον.
\par 26 Και είπεν ο Ιούδας προς τους αδελφούς αυτού, Τις η ωφέλεια, εάν φονεύσωμεν τον αδελφόν ημών και κρύψωμεν το αίμα αυτού;
\par 27 έλθετε και ας πωλήσωμεν αυτόν εις τους Ισμαηλίτας· και ας μη βάλωμεν τας χείρας ημών επ' αυτόν· διότι αδελφός ημών, σαρξ ημών είναι. Και υπήκουσαν οι αδελφοί αυτού.
\par 28 Και ενώ διέβαινον οι Μαδιανίται έμποροι, ανέσυραν και ανεβίβασαν τον Ιωσήφ εκ του λάκκου και επώλησαν τον Ιωσήφ διά είκοσι αργύρια εις τους Ισμαηλίτας· οι δε έφεραν τον Ιωσήφ εις Αίγυπτον.
\par 29 Επέστρεψε δε ο Ρουβήν εις τον λάκκον, και ιδού, ο Ιωσήφ δεν ήτο εν τω λάκκω· και διέσχισε τα ιμάτια αυτού.
\par 30 Και επέστρεψε προς τους αδελφούς αυτού, και είπε, Το παιδίον δεν υπάρχει και εγώ, εγώ που να υπάγω;
\par 31 Τότε έλαβον τον χιτώνα του Ιωσήφ και έσφαξαν ερίφιον εκ των αιγών, και έβαψαν τον χιτώνα εν τω αίματι·
\par 32 και απέστειλαν τον χιτώνα τον ποικιλόχρωμον, και έφεραν αυτόν προς τον πατέρα αυτών και είπον, Ευρήκαμεν τούτον· γνώρισον τώρα, αν ήναι ο χιτών του υιού σου ή ουχί.
\par 33 Ο δε εγνώρισεν αυτόν και είπε, Ο χιτών του υιού μου είναι· θηρίον κακόν κατέφαγεν αυτόν· όλος κατεσπαράχθη ο Ιωσήφ.
\par 34 Και διέσχισεν ο Ιακώβ τα ιμάτια αυτού και έβαλε σάκκον εις την οσφύν αυτού και επένθησε τον υιόν αυτού ημέρας πολλάς.
\par 35 Και εσηκώθησαν πάντες οι υιοί αυτού και πάσαι αι θυγατέρες αυτού, διά να παρηγορήσωσιν αυτόν· αλλά δεν ήθελε να παρηγορηθή, λέγων, Ότι πενθών θέλω καταβή προς τον υιόν μου εις τον τάφον. Και έκλαυσεν αυτόν ο πατήρ αυτού.
\par 36 Οι δε Μαδιανίται επώλησαν αυτόν εν τη Αιγύπτω εις τον Πετεφρήν, αυλικόν του Φαραώ, άρχοντα των σωματοφυλάκων.

\chapter{38}

\par 1 Και κατ' εκείνον τον καιρόν κατέβη ο Ιούδας από των αδελφών αυτού και ετράπη προς άνθρωπον τινά Οδολλαμίτην ονομαζόμενον Ειρά.
\par 2 Και είδεν εκεί ο Ιούδας την θυγατέρα τινός Χαναναίου, ονομαζομένου Σουά· και έλαβεν αυτήν και εισήλθε προς αυτήν.
\par 3 Η δε συνέλαβε, και εγέννησεν υιόν· και εκάλεσε το όνομα αυτού Ηρ.
\par 4 Συνέλαβε δε πάλιν και εγέννησεν υιόν· και εκάλεσε το όνομα αυτού Αυνάν.
\par 5 Εγέννησε δε πάλιν και άλλον υιόν· και εκάλεσε το όνομα αυτού Σηλά· ήτο δε ο Ιούδας εν Χασβί, ότε εγέννησε τούτον.
\par 6 Και έλαβεν ο Ιούδας γυναίκα εις τον Ηρ τον πρωτότοκον αυτού, ονομαζομένην Θάμαρ.
\par 7 Ο Ηρ δε ο πρωτότοκος του Ιούδα εστάθη κακός έμπροσθεν του Κυρίου· και εθανάτωσεν αυτόν ο Κύριος.
\par 8 Είπε δε ο Ιούδας προς τον Αυνάν· είσελθε προς την γυναίκα του αδελφού σου, και νυμφεύθητι αυτήν, και ανάστησον σπέρμα εις τον αδελφόν σου.
\par 9 Αλλ' ο Αυνάν ήξευρεν, ότι το σπέρμα δεν ήθελεν είσθαι ιδικόν του· διά τούτο, ότε εισήρχετο προς την γυναίκα του αδελφού αυτού, εξέχυνεν επί την γην, διά να μη δώση σπέρμα εις τον αδελφόν αυτού.
\par 10 Και τούτο το οποίον έπραττεν εφάνη κακόν έμπροσθεν του Κυρίου· όθεν εθανάτωσε και τούτον.
\par 11 Και είπεν ο Ιούδας προς την Θάμαρ την νύμφην αυτού, Κάθου χήρα εν τω οίκω του πατρός σου, εωσού Σηλά ο υιός μου γείνη μεγάλος· διότι έλεγε, Μήπως αποθάνη και ούτος, καθώς οι αδελφοί αυτού. Υπήγε λοιπόν η Θάμαρ και κατώκησεν εν τω οίκω του πατρός αυτής.
\par 12 Και μετά πολλάς ημέρας απέθανεν η θυγάτηρ του Σουά, η γυνή του Ιούδα· και αφού παρηγορήθη ο Ιούδας, ανέβη προς τους κουρευτάς των προβάτων αυτού εις Θαμνά, αυτός και ο φίλος αυτού Ειρά ο Οδολλαμίτης.
\par 13 Και ανήγγειλαν προς την Θάμαρ, λέγοντες, Ιδού, ο πενθερός σου αναβαίνει εις Θαμνά διά να κουρεύση τα πρόβατα αυτού.
\par 14 Η δε απεκδυθείσα τα ενδύματα της χηρείας αυτής, εσκεπάσθη με κάλυμμα και περιετυλίχθη και εκάθισε κατά την δίοδον την εν τη οδώ της Θαμνά· διότι είδεν ότι έγεινε μεγάλος ο Σηλά, και αυτή δεν εδόθη εις αυτόν διά γυναίκα.
\par 15 Και ότε είδεν αυτήν ο Ιούδας, ενόμισεν αυτήν πόρνην· διότι είχε κεκαλυμμένον το πρόσωπον αυτής.
\par 16 Και κατά την οδόν ετράπη προς αυτήν, και είπεν, Άφες με, παρακαλώ, να εισέλθω προς σέ· διότι δεν εγνώρισεν ότι ήτο η νύμφη αυτού. Η δε είπε, Τι θέλεις μοι δώσει, διά να εισέλθης προς εμέ;
\par 17 Ο δε είπεν, Εγώ θέλω σοι στείλει ερίφιον αιγών εκ του ποιμνίου. Και εκείνη είπε, Μοι δίδεις ενέχυρον, εωσού να στείλης αυτό;
\par 18 Ο δε είπε, Τι ενέχυρον να σοι δώσω; Και εκείνη είπε, την σφραγίδά σου και το περιδέρραιόν σου και την ράβδον σου την εν τη χειρί σου. Και έδωκεν αυτά εις αυτήν και εισήλθε προς αυτήν, και συνέλαβεν εξ αυτού.
\par 19 Μετά ταύτα σηκωθείσα, ανεχώρησε και απεκδυθείσα το κάλυμμα αυτής, ενεδύθη τα ενδύματα της χηρείας αυτής.
\par 20 Ο δε Ιούδας έστειλε το ερίφιον των αιγών διά χειρός του φίλου αυτού του Οδολλαμίτου, διά να παραλάβη το ενέχυρον εκ της χειρός της γυναικός· πλην δεν εύρηκεν αυτήν·
\par 21 και ηρώτησε τους ανθρώπους του τόπου αυτής, λέγων, Που είναι η πόρνη, ήτις ήτο κατά την δίοδον επί της οδού; οι δε είπον, Δεν εστάθη εδώ πόρνη.
\par 22 Και επέστρεψε προς τον Ιούδαν και είπε, Δεν εύρηκα αυτήν· μάλιστα οι άνθρωποι του τόπου είπον, Δεν εστάθη εδώ πόρνη.
\par 23 Και είπεν ο Ιούδας, Ας έχη αυτά, διά να μη γείνωμεν όνειδος· ιδού, εγώ έστειλα το ερίφιον τούτο, συ όμως δεν εύρηκας αυτήν.
\par 24 Και μετά τρεις μήνας περίπου, ανήγγειλαν προς τον Ιούδαν, λέγοντες, Θάμαρ η νύμφη σου επορνεύθη, και μάλιστα, ιδού, είναι έγκυος εκ πορνείας. Και είπεν ο Ιούδας, Φέρετε αυτήν έξω και ας κατακαυθή.
\par 25 Και ότε εφέρετο έξω, απέστειλε προς τον πενθερόν αυτής, λέγουσα, Εκ του ανθρώπου, του οποίου είναι ταύτα, είμαι έγγυος· και είπεν έτι, Γνώρισον, παρακαλώ, τίνος είναι η σφραγίς και το περιδέρραιον, και η ράβδος αύτη.
\par 26 Και ο Ιούδας εγνώρισεν αυτά· και είπεν, Αύτη είναι δικαιοτέρα εμού, διότι δεν έδωκα αυτήν εις τον Σηλά τον υιόν μου. Και έτι πλέον δεν εγνώρισεν αυτήν.
\par 27 Και καθ' ον καιρόν έμελλε να γεννήση, ιδού, δίδυμα εν τη κοιλία αυτής.
\par 28 Και ενώ εγέννα, το εν επρόβαλεν έξω την χείρα· και η μαία λαβούσα, έδεσεν επί την χείρα αυτού νήμα κόκκινον, λέγουσα, Ούτος εξήλθε πρώτος.
\par 29 Και καθώς έσυρεν οπίσω την χείρα αυτού, ιδού, εξήλθεν ο αδελφός αυτού· και αυτή είπε, Ποίον χαλασμόν έκαμες; επί σε ας ήναι ο χαλασμός· διά τούτο εκαλέσθη το όνομα αυτού Φαρές.
\par 30 Και έπειτα εξήλθεν ο αδελφός αυτού, όστις είχε το κόκκινον νήμα επί την χείρα αυτού· και εκαλέσθη το όνομα αυτού Ζαρά.

\chapter{39}

\par 1 Ο δε Ιωσήφ κατεβιβάσθη εις την Αίγυπτον· και ο Πετεφρής ο αυλικός του Φαραώ, ο άρχων των σωματοφυλάκων, άνθρωπος Αιγύπτιος, ηγόρασεν αυτόν εκ των χειρών των Ισμαηλιτών, οίτινες κατεβίβασαν αυτόν εκεί.
\par 2 Και ήτο ο Κύριος μετά του Ιωσήφ, και ήτο άνθρωπος ευοδούμενος· και ευρίσκετο εν τω οίκω του κυρίου αυτού Αιγυπτίου.
\par 3 Και είδεν ο κύριος αυτού, ότι ο Κύριος ήτο μετ' αυτού, και ευώδωνεν ο Κύριος εις τας χείρας αυτού πάντα όσα έκαμνε.
\par 4 Και εύρηκεν ο Ιωσήφ χάριν έμπροσθεν αυτού και υπηρέτει αυτόν· και κατέστησεν αυτόν επιστάτην επί του οίκου αυτού· και πάντα όσα είχε, παρέδωκεν εις τας χείρας αυτού.
\par 5 Και εξ εκείνου του καιρού, αφού κατέστησεν αυτόν επιστάτην επί του οίκου αυτού και επί πάντων όσα είχεν, ευλόγησεν ο Κύριος τον οίκον του Αιγυπτίου εξ αιτίας του Ιωσήφ· και ήτο η ευλογία του Κυρίου επί πάντα όσα είχεν, εν τω οίκω και εν τοις αγροίς.
\par 6 Και παρέδωκε πάντα όσα είχεν εις τας χείρας του Ιωσήφ· και δεν ήξευρεν εκ των υπαρχόντων αυτού ουδέν, πλην του άρτου τον οποίον έτρωγεν. Ήτο δε ο Ιωσήφ ευειδής και ώραίος την όψιν.
\par 7 Και μετά τα πράγματα ταύτα, η γυνή του κυρίου αυτού έρριψε τους οφθαλμούς αυτής επί τον Ιωσήφ· και είπε, Κοιμήθητι μετ' εμού.
\par 8 Αλλ' εκείνος δεν ήθελε, και είπε προς την γυναίκα του κυρίου αυτού, Ιδού· ο κύριός μου δεν γνωρίζει ουδέν εκ των όσα είναι μετ' εμού εν τω οίκω· και πάντα όσα έχει, παρέδωκεν εις τας χείρας μου·
\par 9 δεν είναι εν τω οίκω τούτω ουδείς μεγαλήτερός μου, ούτε είναι απηγορευμένον εις εμέ άλλο τι πλην σου, διότι είσαι η γυνή αυτού· και πως να πράξω τούτο το μέγα κακόν, και να αμαρτήσω εναντίον του Θεού;
\par 10 Αν και ελάλει προς τον Ιωσήφ καθ' εκάστην ημέραν, ούτος όμως δεν υπήκουσεν εις αυτήν να κοιμηθή μετ' αυτής, διά να συνευρεθή μετ' αυτής.
\par 11 Και ημέραν τινά εισήλθεν ο Ιωσήφ εις την οικίαν διά να κάμη τα έργα αυτού, και ουδείς εκ των ανθρώπων του οίκου ήτο εκεί εν τω οίκω.
\par 12 Και εκείνη ήρπασεν αυτόν από του ιματίου αυτού, λέγουσα, Κοιμήθητι μετ' εμού· αλλ' εκείνος αφήσας το ιμάτιον αυτού εις τας χείρας αυτής, έφυγε, και εξήλθεν έξω.
\par 13 Και ως είδεν ότι αφήκε το ιμάτιον αυτού εις τας χείρας αυτής και έφυγεν έξω,
\par 14 εβόησε προς τους ανθρώπους της οικίας αυτής και ελάλησε προς αυτούς, λέγουσα, Ίδετε, έφερεν εις ημάς άνθρωπον Εβραίον διά να μας εμπαίξη· εισήλθε προς εμέ διά να κοιμηθή μετ' εμού και εγώ εβόησα μετά φωνής μεγάλης·
\par 15 και ως ήκουσεν ότι ύψωσα την φωνήν μου και εβόησα, αφήσας το ιμάτιον αυτού παρ' εμοί έφυγε και εξήλθεν έξω.
\par 16 Και απέθεσε το ιμάτιον αυτού παρ' αυτή, εωσού ήλθεν ο κύριος αυτού εις τον οίκον αυτού.
\par 17 Και είπε προς αυτόν κατά τους λόγους τούτους, λέγουσα, Ο δούλος ο Εβραίος, τον οποίον έφερες εις ημάς, εισήλθε προς εμέ διά να με εμπαίξη,
\par 18 και ως ύψωσα την φωνήν μου και εβόησα, αφήσας το ιμάτιον αυτού παρ' εμοί, έφυγεν έξω.
\par 19 Και ως ήκουσεν ο κύριος αυτού τους λόγους της γυναικός αυτού, τους οποίους ελάλησε προς αυτόν, λέγουσα, Ούτω μοι έκαμεν ο δούλός σου, εξήφθη η οργή αυτού.
\par 20 Και λαβών ο κύριος του Ιωσήφ αυτόν, έβαλεν αυτόν εις την οχυράν φυλακήν, εις τον τόπον όπου οι δέσμιοι του βασιλέως ήσαν πεφυλακισμένοι και έμενεν εκεί εν τη οχυρά φυλακή.
\par 21 Αλλ' ο Κύριος ήτο μετά του Ιωσήφ και επέχεεν εις αυτόν έλεος, και έδωκε χάριν εις αυτόν έμπροσθεν του αρχιδεσμοφύλακος.
\par 22 Και παρέδωκεν ο αρχιδεσμοφύλαξ εις τας χείρας του Ιωσήφ πάντας τους δεσμίους, τους εν τη οχυρά φυλακή· και πάντα όσα επράττοντο εκεί, αυτός έκαμνεν αυτά.
\par 23 Ο αρχιδεσμοφύλαξ δεν εθεώρει ουδέν εκ των όσα ήσαν εις τας χείρας αυτού· διότι ο Κύριος ήτο μετ' αυτού και ο Κύριος ευώδονεν όσα αυτός έκαμνε.

\chapter{40}

\par 1 Και μετά τα πράγματα ταύτα, ο οινοχόος του βασιλέως της Αιγύπτου και ο αρτοποιός ημάρτησαν εις τον κύριον αυτών τον βασιλέα της Αιγύπτου.
\par 2 Και ωργίσθη ο Φαραώ κατά των δύο αυλικών αυτού, κατά του αρχιοινοχόου, και κατά του αρχισιτοποιού.
\par 3 Και έβαλεν αυτούς υπό φύλαξιν εν τω οίκω του άρχοντος των σωματοφυλάκων, εις την οχυράν φυλακήν, εις τον τόπον όπου ο Ιωσήφ ήτο πεφυλακισμένος.
\par 4 Ο δε άρχων των σωματοφυλάκων ενεπιστεύθη αυτούς εις τον Ιωσήφ και ούτος υπηρέτει αυτούς· ήσαν δε καιρόν τινά εν τη φυλακή·
\par 5 και ο οινοχόος και ο αρτοποιός του βασιλέως της Αιγύπτου, οίτινες ήσαν πεφυλακισμένοι εν τη οχυρά φυλακή, ενυπνιάσθησαν ενύπνιον αμφότεροι, έκαστος το ενύπνιον αυτού κατά την αυτήν νύκτα, έκαστος κατά την εξήγησιν του ενυπνίου αυτού.
\par 6 Ο δε Ιωσήφ εισελθών προς αυτούς το πρωΐ, είδεν αυτούς· και ιδού, ήσαν τεταραγμένοι.
\par 7 Και ηρώτησε τους αυλικούς του Φαραώ, τους όντας μετ' αυτού εν τη φυλακή εν τω οίκω του κυρίου αυτού, λέγων, Διά τι τα πρόσωπά σας είναι σκυθρωπά σήμερον;
\par 8 Οι δε είπον προς αυτόν, Ενυπνιάσθημεν ενύπνιον και δεν είναι ουδείς όστις να εξηγήση αυτό. Και είπε προς αυτούς ο Ιωσήφ, Δεν ανήκουσιν εις τον Θεόν αι εξηγήσεις; διηγήθητέ μοι, παρακαλώ.
\par 9 Και διηγήθη ο αρχιοινοχόος το ενύπνιον αυτού προς τον Ιωσήφ και είπε προς αυτόν, Είδον εις το όνειρόν μου και ιδού, άμπελος έμπροσθέν μου·
\par 10 και εις την άμπελον ήσαν τρεις κλάδοι και εφαίνετο ως βλαστάνουσα και τα άνθη αυτής εξήνθησαν και οι βότρυες της σταφυλής ωρίμασαν·
\par 11 το δε ποτήριον του Φαραώ ήτο εν τη χειρί μου· και έλαβον τα σταφύλια και έθλιψα αυτά εις το ποτήριον του Φαραώ και έδωκα το ποτήριον εις την χείρα του Φαραώ.
\par 12 Και είπεν ο Ιωσήφ προς αυτόν, Αύτη είναι η εξήγησις αυτού· οι τρεις κλάδοι είναι τρεις ημέραι·
\par 13 μετά τρεις ημέρας, ο Φαραώ θέλει υψώσει την κεφαλήν σου και θέλει σε αποκαταστήσει εις το υπούργημά σου· και θέλεις δώσει το ποτήριον του Φαραώ εις την χείρα αυτού κατά την προτέραν συνήθειαν, ότε ήσο οινοχόος αυτού·
\par 14 πλην ενθυμήθητί με, όταν γείνη εις σε το καλόν· και κάμε, παρακαλώ, έλεος προς εμέ και ανάφερε περί εμού προς τον Φαραώ και εξάγαγέ με εκ του οίκου τούτου·
\par 15 επειδή τη αληθεία εκλέφθην εκ της γης των Εβραίων· και εδώ πάλιν δεν έπραξα ουδέν, ώστε να με βάλωσιν εις τον λάκκον τούτον.
\par 16 Και ιδών ο αρχισιτοποιός ότι η εξήγησις ήτο καλή, είπε προς τον Ιωσήφ, Και εγώ είδον εις το όνειρόν μου και ιδού, τρία κάνιστρα λευκά επί της κεφαλής μου·
\par 17 εν δε τω κανίστρω τω ανωτέρω ήσαν εκ πάντων των φαγητών του Φαραώ, της τέχνης του αρτοποιού· και τα πτηνά έτρωγον αυτά εκ του κανίστρου επάνωθεν της κεφαλής μου.
\par 18 Και αποκριθείς ο Ιωσήφ είπεν, Αύτη είναι η εξήγησις τούτου· τα τρία κάνιστρα είναι τρεις ημέραι·
\par 19 μετά τρεις ημέρας ο Φαραώ θέλει υψώσει την κεφαλήν σου επάνωθέν σου και θέλει σε κρεμάσει εις ξύλον και τα πτηνά θέλουσι φάγει την σάρκα σου επάνωθέν σου.
\par 20 Και την τρίτην ημέραν, ημέραν των γενεθλίων του Φαραώ, έκαμε συμπόσιον εις πάντας τους δούλους αυτού· και ύψωσε την κεφαλήν του αρχιοινοχόου και την κεφαλήν του αρχισιτοποιού μεταξύ των δούλων αυτού.
\par 21 Και τον μεν αρχιοινοχόον αποκατέστησεν εις την οινοχοΐαν αυτού, και έδωκε το ποτήριον εις την χείρα του Φαραώ·
\par 22 τον δε αρχισιτοποιόν εκρέμασε, καθώς εξήγησεν ο Ιωσήφ εις αυτούς.
\par 23 Ο αρχιοινοχόος όμως δεν ενεθυμήθη τον Ιωσήφ, αλλά ελησμόνησεν αυτόν.

\chapter{41}

\par 1 Και μετά παρέλευσιν δύο ετών ο Φαραώ είδεν ενύπνιον· και ιδού, ίστατο πλησίον του ποταμού.
\par 2 και ιδού, επτά δαμάλια εύμορφα και παχύσαρκα ανέβαινον εκ του ποταμού και εβόσκοντο εις το λιβάδιον.
\par 3 και ιδού, άλλα επτά δαμάλια ανέβαινον μετ' εκείνα εκ του ποταμού, άσχημα και λεπτόσαρκα, και ίσταντο πλησίον των άλλων δαμαλίων επί το χείλος του ποταμού·
\par 4 και τα δαμάλια τα άσχημα και λεπτόσαρκα κατέφαγον τα επτά δαμάλια τα εύμορφα και παχύσαρκα. Τότε εξύπνησεν ο Φαραώ.
\par 5 Και αποκοιμηθείς ενυπνιάσθη δευτέραν φοράν· και ιδού, επτά αστάχυα παχέα και καλά ανέβαινον εκ του αυτού κορμού·
\par 6 και ιδού, άλλα επτά αστάχυα λεπτά και κεκαυμένα υπό του ανατολικού ανέμου ανεφύοντο μετ' εκείνα·
\par 7 και τα αστάχυα τα λεπτά κατέπιον τα επτά αστάχυα τα παχέα και μεστά. Και εξύπνησεν ο Φαραώ και ιδού, ήτο όνειρον.
\par 8 Και το πρωΐ το πνεύμα αυτού ήτο τεταραγμένον· και αποστείλας εκάλεσε πάντας τους μάγους της Αιγύπτου και πάντας τους σοφούς αυτής· και διηγήθη προς αυτούς ο Φαραώ τα ενύπνια αυτού· αλλά δεν ήτο ουδείς όστις να εξηγήση αυτά προς τον Φαραώ.
\par 9 Τότε ο αρχιοινοχόος ελάλησε προς τον Φαραώ λέγων, την αμαρτίαν μου ενθυμούμαι σήμερον·
\par 10 ο Φαραώ είχεν οργισθή εναντίον των δούλων αυτού και με έβαλεν εις φυλακήν εν τω οίκω του άρχοντος των σωματοφυλάκων, εμέ και τον αρχισιτοποιόν·
\par 11 και είδομεν ενύπνιον κατά την αυτήν νύκτα, εγώ και εκείνος· ενυπνιάσθημεν έκαστος κατά την εξήγησιν του ενυπνίου αυτού·
\par 12 και ήτο εκεί μεθ' ημών νέος τις Εβραίος, δούλος του άρχοντος των σωματοφυλάκων· και διηγήθημεν προς αυτόν και εξήγησεν εις ημάς τα ενύπνια ημών· εις έκαστον κατά το ενύπνιον αυτού έκαμε την εξήγησιν·
\par 13 και καθώς εξήγησεν εις ημάς, ούτω και συνέβη· εμέ μεν αποκατέστησεν εις το υπούργημά μου, εκείνον δε εκρέμασε.
\par 14 Τότε αποστείλας ο Φαραώ, εκάλεσε τον Ιωσήφ, και εξήγαγον αυτόν μετά σπουδής εκ της φυλακής· και εξυρίσθη και ήλλαξε την στολήν αυτού και ήλθε προς τον Φαραώ.
\par 15 Και είπεν ο Φαραώ προς τον Ιωσήφ, Είδον ενύπνιον, και δεν είναι ουδείς όστις να εξηγήση αυτό· και εγώ ήκουσα περί σου να λέγωσιν ότι εννοείς τα όνειρα ώστε να εξηγής αυτά.
\par 16 Και απεκρίθη ο Ιωσήφ προς τον Φαραώ λέγων, Ουχί εγώ· ο Θεός θέλει δώσει εις τον Φαραώ σωτήριον απόκρισιν.
\par 17 Και είπεν ο Φαραώ προς τον Ιωσήφ, Εις το όνειρόν μου, ιδού, ιστάμην επί το χείλος του ποταμού·
\par 18 και ιδού, επτά δαμάλια παχύσαρκα και εύμορφα ανέβαινον εκ του ποταμού και εβόσκοντο εις το λιβάδιον·
\par 19 και ιδού, άλλα επτά δαμάλια ανέβαινον κατόπιν εκείνων αδύνατα και πολύ άσχημα και λεπτόσαρκα, οποία δεν είδον ποτέ ασχημότερα καθ' όλην την γην της Αιγύπτου·
\par 20 και τα δαμάλια τα λεπτά και άσχημα κατέφαγον τα πρώτα επτά δαμάλια τα παχέα·
\par 21 και αφού εισήλθον εις τας κοιλίας αυτών, δεν διεκρίνετο ότι εισήλθον εις τας κοιλίας αυτών, αλλ' η θεωρία αυτών ήτο άσχημος καθώς και πρότερον· τότε εξύπνησα.
\par 22 Έπειτα είδον εις το όνειρόν μου και ιδού, επτά αστάχυα ανέβαινον εκ του αυτού κορμού μεστά και καλά·
\par 23 και ιδού, άλλα επτά αστάχυα ξηρά, λεπτά, κεκαυμένα υπό του ανατολικού ανέμου, ανεφύοντο κατόπιν αυτών·
\par 24 και τα αστάχυα τα λεπτά κατέπιον τα επτά αστάχυα τα καλά· και είπον ταύτα προς τους μάγους, αλλά δεν ήτο ουδείς όστις να μοι εξηγήση αυτά.
\par 25 Και είπεν ο Ιωσήφ προς τον Φαραώ, Το ενύπνιον του Φαραώ εν είναι· ο Θεός εφανέρωσεν εις τον Φαραώ όσα μέλλει να κάμη.
\par 26 Τα επτά δαμάλια τα καλά είναι επτά έτη· και τα επτά αστάχυα τα καλά είναι επτά έτη· το ενύπνιον εν είναι.
\par 27 Και τα επτά δαμάλια τα λεπτά και άσχημα, τα οποία ανέβαινον κατόπιν αυτών, είναι επτά έτη· και τα επτά αστάχυα τα άμεστα, τα κεκαυμένα υπό του ανατολικού ανέμου, θέλουσιν είσθαι επτά έτη πείνης.
\par 28 Τούτο είναι το πράγμα το οποίον είπα προς τον Φαραώ· ο Θεός εφανέρωσεν εις τον Φαραώ όσα μέλλει να κάμη.
\par 29 Ιδού, έρχονται επτά έτη μεγάλης αφθονίας καθ' όλην την γην της Αιγύπτου·
\par 30 και θέλουσιν επέλθει μετά ταύτα επτά έτη πείνης· και όλη η αφθονία θέλει λησμονηθή εν τη γη της Αιγύπτου και η πείνα θέλει καταφθείρει την γήν·
\par 31 και δεν θέλει γνωρισθή η αφθονία επί της γης εξ αιτίας εκείνης της πείνης, ήτις μέλλει να ακολουθήση· διότι θέλει είσθαι βαρεία σφόδρα.
\par 32 Το δε ότι εδευτερώθη το ενύπνιον εις τον Φαραώ δις, φανερόνει ότι το πράγμα είναι αποφασισμένον παρά του Θεού και ότι ο Θεός θέλει ταχύνει να εκτελέση αυτό.
\par 33 Τώρα λοιπόν ας προβλέψη ο Φαραώ άνθρωπον συνετόν και φρόνιμον και ας καταστήση αυτόν επί της γης της Αιγύπτου·
\par 34 ας κάμη ο Φαραώ και ας διορίση επιστάτας επί της γής· και ας λαμβάνη το πέμπτον από της γης Αιγύπτου εις τα επτά έτη της αφθονίας·
\par 35 και ας συνάξωσι πάσας τας τροφάς τούτων των ερχομένων καλών ετών, και ας αποταμιεύσωσι σίτον υπό την χείρα του Φαραώ διά τροφάς εις τας πόλεις, και ας φυλάττωσιν αυτόν·
\par 36 και αι τροφαί θέλουσι μένει πεφυλαγμέναι διά την γην εις τα επτά έτη της πείνης, τα οποία θέλουσιν ακολουθήσει εν τη γη της Αιγύπτου, διά να μη απολεσθή ο τόπος υπό της πείνης.
\par 37 Και ήρεσεν ο λόγος εις τον Φαραώ και εις πάντας τους δούλους αυτού.
\par 38 Και είπεν ο Φαραώ προς τους δούλους αυτού, Δυνάμεθα να εύρωμεν καθώς τούτον, άνθρωπον εις τον οποίον υπάρχει το πνεύμα του Θεού;
\par 39 Και είπεν ο Φαραώ προς τον Ιωσήφ, Επειδή ο Θεός έδειξεν εις σε πάντα ταύτα, δεν είναι ουδείς τόσον συνετός και φρόνιμος όσον συ.
\par 40 Συ θέλεις είσθαι επί του οίκου μου και εις τον λόγον του στόματός σου θέλει υπακούει πας ο λαός μου· μόνον κατά τον θρόνον θέλω είσθαι ανώτερός σου.
\par 41 Και είπεν ο Φαραώ προς τον Ιωσήφ, Ιδού, σε κατέστησα εφ' όλης της γης Αιγύπτου.
\par 42 Και εκβαλών ο Φαραώ το δακτυλίδιον αυτού εκ της χειρός αυτού, έβαλεν αυτό εις την χείρα του Ιωσήφ και ενέδυσεν αυτόν ιμάτια βύσσινα, και περιέβαλε χρυσούν περιδέρραιον περί τον τράχηλον αυτού.
\par 43 Και ανεβίβασεν αυτόν επί την άμαξαν αυτού την δευτέραν· και εκήρυττον έμπροσθεν αυτού, Γονατίσατε· και κατέστησεν αυτόν εφ' όλης της γης Αιγύπτου.
\par 44 Και είπεν ο Φαραώ προς τον Ιωσήφ, Εγώ είμαι ο Φαραώ, και χωρίς σου ουδείς θέλει σηκώσει την χείρα αυτού ή τον πόδα αυτού καθ' όλην την γην της Αιγύπτου.
\par 45 Και ωνόμασεν ο Φαραώ τον Ιωσήφ αφνάθ-πανεάχ· και έδωκεν εις αυτόν διά γυναίκα Ασενέθ, την θυγατέρα του Ποτιφερά ιερέως της Ων. Και εξήλθεν ο Ιωσήφ εις την γην της Αιγύπτου.
\par 46 Ήτο δε ο Ιωσήφ τριάκοντα ετών, ότε παρεστάθη έμπροσθεν του Φαραώ βασιλέως της Αιγύπτου· και εξήλθεν ο Ιωσήφ απ' έμπροσθεν του Φαραώ, και διήλθεν όλην την γην της Αιγύπτου.
\par 47 Και εκαρποφόρησεν η γη πλουσιοπαρόχως εις τα επτά έτη της αφθονίας·
\par 48 και συνήγαγε πάσας τας τροφάς των επτά ετών των γενομένων εν τη γη της Αιγύπτου· και εναπέθεσε τας τροφάς εν ταις πόλεσι· τας τροφάς των αγρών των πέριξ εκάστης πόλεως έθεσεν εν αυτή.
\par 49 Και συνήγαγεν ο Ιωσήφ σίτον ως την άμμον της θαλάσσης πολύν σφόδρα, ώστε έπαυσε να μετρή αυτόν· διότι ήτο αμέτρητος.
\par 50 Εγεννήθησαν δε δύο υιοί εις τον Ιωσήφ, πριν έλθωσι τα έτη της πείνης· τους οποίους εγέννησεν εις αυτόν Ασενέθ, η θυγάτηρ του Ποτιφερά ιερέως της Ων.
\par 51 Και εκάλεσεν ο Ιωσήφ το όνομα του πρωτοτόκου Μανασσή· διότι είπεν, Ο Θεός με έκαμε να λησμονήσω πάντας τους πόνους μου και πάντα τον οίκον του πατρός μου.
\par 52 Το δε όνομα του δευτέρου εκάλεσεν Εφραΐμ· διότι είπεν, Ο Θεός με ηύξησεν εν τη γη της θλίψεώς μου.
\par 53 Και παρήλθον τα επτά έτη της αφθονίας, της γενομένης εν τη γη της Αιγύπτου.
\par 54 Και ήρχισαν να έρχωνται τα επτά έτη της πείνης, καθώς είπεν ο Ιωσήφ· και έγεινεν η πείνα εις πάντας τους τόπους· καθ' όλην όμως την γην της Αιγύπτου ήτο άρτος.
\par 55 Και ότε επείνασε πάσα η γη της Αιγύπτου, εβόησεν ο λαός προς τον Φαραώ διά άρτον. Και είπεν ο Φαραώ προς πάντας τους Αιγυπτίους, Υπάγετε προς τον Ιωσήφ· ό,τι σας είπη, κάμετε.
\par 56 Και ήτο η πείνα επί παν το πρόσωπον της γης. Ήνοιξε δε ο Ιωσήφ πάσας τας αποθήκας και επώλει σίτον εις τους Αιγυπτίους· και η πείνα επεβάρυνεν επί την γην της Αιγύπτου.
\par 57 Και πάντες οι τόποι ήρχοντο εις την Αίγυπτον προς τον Ιωσήφ διά να αγοράζωσι σίτον· διότι η πείνα επεβάρυνεν επί πάσαν την γην.

\chapter{42}

\par 1 Και είδεν ο Ιακώβ ότι ευρίσκετο σίτος εν Αιγύπτω· και είπεν ο Ιακώβ προς τους υιούς αυτού, Τι βλέπετε ο εις τον άλλον;
\par 2 Και είπεν, Ιδού, ήκουσα ότι ευρίσκεται σίτος εν Αιγύπτω· κατάβητε εκεί και αγοράσατε δι' ημάς εκείθεν, διά να ζήσωμεν και να μη αποθάνωμεν.
\par 3 Και κατέβησαν οι δέκα αδελφοί του Ιωσήφ διά να αγοράσωσι σίτον εξ Αιγύπτου.
\par 4 τον Βενιαμίν όμως, τον αδελφόν του Ιωσήφ, δεν απέστειλεν ο Ιακώβ μετά των αδελφών αυτού· διότι είπε, Μήπως συμβή εις αυτόν συμφορά.
\par 5 Και ήλθον οι υιοί του Ισραήλ διά να αγοράσωσι σίτον μεταξύ των εκεί ερχομένων· διότι η πείνα ήτο εν τη γη Χαναάν.
\par 6 Ο Ιωσήφ δε ήτο ο διοικητής του τόπου· αυτός επώλει εις πάντα τον λαόν του τόπου· ήλθον λοιπόν οι αδελφοί του Ιωσήφ και προσεκύνησαν αυτόν επί πρόσωπον έως εδάφους.
\par 7 Ιδών δε ο Ιωσήφ τους αδελφούς αυτού, εγνώρισεν αυτούς· προσεποιήθη όμως τον ξένον προς αυτούς και ελάλει προς αυτούς σκληρά· και είπε προς αυτούς, Πόθεν έρχεσθε; οι δε είπον, Εκ της γης Χαναάν, διά να αγοράσωμεν τροφάς.
\par 8 Και ο μεν Ιωσήφ εγνώρισε τους αδελφούς αυτού· εκείνοι όμως δεν εγνώρισαν αυτόν.
\par 9 Και ενεθυμήθη ο Ιωσήφ τα ενύπνια, τα οποία ενυπνιάσθη περί αυτών· και είπε προς αυτούς, Κατάσκοποι είσθε· ήλθετε να παρατηρήσητε τα γυμνά του τόπου.
\par 10 Οι δε είπον προς αυτόν, Ουχί, κύριέ μου· αλλ' ήλθομεν οι δούλοί σου διά να αγοράσωμεν τροφάς·
\par 11 ημείς πάντες είμεθα υιοί ενός ανθρώπου· καλοί άνθρωποι είμεθα· οι δούλοί σου δεν είναι κατάσκοποι.
\par 12 Και είπε προς αυτούς, Ουχί, αλλά τα γυμνά του τόπου ήλθετε διά να παρατηρήσητε.
\par 13 Οι δε είπον, Οι δούλοί σου είμεθα δώδεκα αδελφοί, υιοί ενός ανθρώπου εν τη γη Χαναάν· και ιδού, ο νεώτερος ευρίσκεται σήμερον μετά του πατρός ημών, ο δε άλλος δεν υπάρχει.
\par 14 Και είπε προς αυτούς ο Ιωσήφ, τούτο είναι το οποίον σας είπα λέγων, Κατάσκοποι είσθε.
\par 15 Με τούτο θέλετε δοκιμασθή· Μα την ζωήν του Φαραώ δεν θέλετε εξέλθει εντεύθεν, εάν δεν έλθη εδώ ο αδελφός σας ο νεώτερος·
\par 16 αποστείλατε ένα από σας και ας φέρη τον αδελφόν σας· σεις δε θέλετε μένει δέσμιοι, εωσού αποδειχθώσιν οι λόγοι σας, αν λέγητε την αλήθειαν· ει δε μη, μα την ζωήν του Φαραώ, κατάσκοποι βεβαίως είσθε.
\par 17 Και έθεσεν αυτούς υπό φύλαξιν τρεις ημέρας.
\par 18 Και την τρίτην ημέραν είπε προς αυτούς ο Ιωσήφ, τούτο κάμετε και θέλετε ζήσει διότι εγώ φοβούμαι τον Θεόν·
\par 19 Εάν ήσθε καλοί, εις εκ των αδελφών σας ας μείνη δέσμιος εν τη φυλακή, όπου είσθε· σεις δε υπάγετε, λάβετε σίτον διά την πείναν των οικιών σας·
\par 20 φέρετε όμως προς εμέ τον αδελφόν σας τον νεώτερον· ούτω θέλουσιν αληθεύσει οι λόγοι σας και δεν θέλετε αποθάνει. Και έκαμον ούτω.
\par 21 Και είπεν ο εις προς τον άλλον, Αληθώς ένοχοι είμεθα διά τον αδελφόν ημών, καθότι είδομεν την θλίψιν της ψυχής αυτού, ότε παρεκάλει ημάς και δεν εισηκούσαμεν αυτού· διά τούτο επήλθεν εφ' ημάς η θλίψις αύτη.
\par 22 Απεκρίθη δε ο Ρουβήν προς αυτούς λέγων, Δεν είπον προς εσάς λέγων, Μη αμαρτήσητε κατά του παιδίου, και δεν εισηκούσατε; διά τούτο, ιδού, και το αίμα αυτού εκζητείται.
\par 23 Και αυτοί δεν ήξευρον ότι ενόει ο Ιωσήφ· διότι συνωμίλουν δι' ερμηνέως.
\par 24 Και συρθείς από πλησίον αυτών έκλαυσε· και πάλιν επέστρεψε προς αυτούς και ελάλει εις αυτούς· και έλαβεν εξ αυτών τον Συμεών και έδεσεν αυτόν ενώπιον αυτών.
\par 25 Τότε προσέταξεν ο Ιωσήφ να γεμίσωσι τα αγγεία αυτών σίτον και να επιστρέψωσι το αργύριον εκάστου εν τω σακκίω αυτού και να δώσωσιν εις αυτούς ζωοτροφίαν διά την οδόν· και έγεινεν εις αυτούς ούτω.
\par 26 Και φορτώσαντες τον σίτον αυτών επί τους όνους αυτών, ανεχώρησαν εκείθεν.
\par 27 Ότε δε εις εξ αυτών έλυσε το σακκίον αυτού, διά να δώση εις τον όνον αυτού τροφήν εν τω καταλύματι, είδε το αργύριον αυτού, και ιδού, ήτο εν τω στόματι του σακκίου αυτού.
\par 28 Και είπε προς τους αδελφούς αυτού, το αργύριόν μου μοι εδόθη οπίσω και μάλιστα ιδού, είναι εν τω σακκίω μου· και εξεπλάγη η καρδία αυτών και συνεταράχθησαν, λέγοντες προς αλλήλους, Τι είναι τούτο, το οποίον μας έκαμεν ο Θεός;
\par 29 Ήλθον δε προς Ιακώβ τον πατέρα αυτών εις την γην Χαναάν και απήγγειλαν προς αυτόν πάντα τα συμβάντα εις αυτούς, λέγοντες,
\par 30 Ο άνθρωπος, ο κύριος του τόπου, ελάλησε προς ημάς σκληρά και εξέλαβεν ημάς ως κατασκόπους του τόπου.
\par 31 Και είπομεν εις αυτόν, Είμεθα καλοί άνθρωποι δεν είμεθα κατάσκοποι·
\par 32 δώδεκα αδελφοί είμεθα, υιοί του πατρός ημών· ο εις δεν υπάρχει ο δε νεώτερος είναι την σήμερον μετά του πατρός ημών εν τη γη Χαναάν.
\par 33 Είπε δε προς ημάς ο άνθρωπος, ο κύριος του τόπου, Με τούτο θέλω γνωρίσει ότι είσθε καλοί άνθρωποι· ένα εκ των αδελφών σας αφήσατε μετ' εμού, και λαβόντες σίτον διά την πείναν των οικιών σας απέλθετε·
\par 34 και φέρετε προς εμέ τον αδελφόν σας τον νεώτερον· τότε θέλω γνωρίσει ότι δεν είσθε κατάσκοποι, άλλ' είσθε καλοί· και θέλω σας αποδώσει τον αδελφόν σας και θέλετε εμπορεύεσθαι εν τω τόπω.
\par 35 Και ότε εκένουν τα σακκία αυτών, ιδού, εκάστου το κομβόδεμα του αργυρίου ήτο εν τω σακκίω αυτού· και ιδόντες αυτοί και ο πατήρ αυτών τα κομβοδέματα του αργυρίου αυτών, εφοβήθησαν.
\par 36 Και είπε προς αυτούς Ιακώβ ο πατήρ αυτών, Σεις με ητεκνώσατε· ο Ιωσήφ δεν υπάρχει και ο Συμεών δεν υπάρχει, και τον Βενιαμίν θέλετε λάβει επ' εμέ ήλθον πάντα ταύτα.
\par 37 Είπε δε ο Ρουβήν προς τον πατέρα αυτού λέγων, τους δύο υιούς μου θανάτωσον, εάν δεν φέρω αυτόν προς σέ· παράδος αυτόν εις την χείρα μου και εγώ θέλω επαναφέρει αυτόν προς σε.
\par 38 Ο δε είπε, δεν θέλει καταβή ο υιός μου μεθ' υμών· διότι ο αδελφός αυτού απέθανε και αυτός μόνος έμεινε· και εάν συμβή εις αυτόν συμφορά εν τη οδώ, όπου υπάγετε, τότε θέλετε καταβιβάσει την πολιάν μου μετά λύπης εις τον τάφον.

\chapter{43}

\par 1 Η δε πείνα επεβάρυνεν επί την γην.
\par 2 Και αφού ετελείωσαν τρώγοντες τον σίτον, τον οποίον έφεραν εξ Αιγύπτου, είπε προς αυτούς ο πατήρ αυτών, Υπάγετε πάλιν, αγοράσατε εις ημάς ολίγας τροφάς.
\par 3 Και είπε προς αυτόν ο Ιούδας λέγων, Εντόνως διεμαρτυρήθη προς ημάς ο άνθρωπος λέγων, Δεν θέλετε ιδεί το πρόσωπόν μου, εάν δεν ήναι μεθ' υμών ο αδελφός υμών.
\par 4 Εάν λοιπόν αποστείλης τον αδελφόν ημών μεθ' ημών, θέλομεν καταβή και θέλομεν σοι αγοράσει τροφάς·
\par 5 αλλ' εάν δεν αποστείλης αυτόν, δεν θέλομεν καταβή· διότι ο άνθρωπος είπε προς ημάς, Δεν θέλετε ιδεί το πρόσωπόν μου, εάν ο αδελφός υμών δεν ήναι μεθ' υμών.
\par 6 Είπε δε ο Ισραήλ, Διά τι με εκακοποιήσατε, φανερόνοντες προς τον άνθρωπον ότι έχετε άλλον αδελφόν;
\par 7 Οι δε είπον, Ο άνθρωπος ηρώτησεν ημάς ακριβώς περί ημών και περί της συγγενείας ημών λέγων, Ο πατήρ σας έτι ζη; έχετε άλλον αδελφόν; Και απεκρίθημεν προς αυτόν κατά την ερώτησιν ταύτην· ηδυνάμεθα να εξεύρωμεν ότι ήθελεν ειπεί, Φέρετε τον αδελφόν σας;
\par 8 Και είπεν ο Ιούδας προς Ισραήλ τον πατέρα αυτού, Απόστειλον το παιδάριον μετ' εμού, και σηκωθέντες ας υπάγωμεν, διά να ζήσωμεν και να μη αποθάνωμεν και ημείς και συ και αι οικογένειαι ημών·
\par 9 εγώ εγγυώμαι περί αυτού· εκ της χειρός μου ζήτησον αυτόν· εάν δεν φέρω αυτόν προς σε και στήσω αυτόν έμπροσθέν σου, τότε ας ήμαι διαπαντός υπεύθυνος προς σέ·
\par 10 επειδή, εάν δεν εβραδύνομεν, βέβαια έως τώρα δευτέραν ταύτην φοράν ηθέλομεν επιστρέψει.
\par 11 Και είπε προς αυτούς Ισραήλ ο πατήρ αυτών, Εάν ούτω πρέπη να γείνη, κάμετε λοιπόν τούτο· λάβετε εις τα αγγείά σας εκ των καλητέρων καρπών της γης και φέρετε δώρα προς τον άνθρωπον, ολίγον βάλσαμον και ολίγον μέλι, αρώματα και μύρον, πιστάκια και αμύγδαλα·
\par 12 και λάβετε διπλάσιον αργύριον εις τας χείρας σας· και το αργύριον το επιστραφέν εν τω στόματι των σακκίων σας φέρετε πάλιν εις τας χείρας σας· ίσως έγεινε κατά λάθος·
\par 13 και τον αδελφόν σας λάβετε και σηκωθέντες επιστρέψατε προς τον άνθρωπον·
\par 14 και ο Θεός ο Παντοδύναμος να σας δώση χάριν έμπροσθεν του ανθρώπου, διά να αποστείλη με σας τον άλλον σας αδελφόν και τον Βενιαμίν· και εγώ, αν ήναι να ατεκνωθώ, ας ατεκνωθώ.
\par 15 Λαβόντες δε οι άνθρωποι τα δώρα ταύτα, έλαβον και αργύριον διπλάσιον εις τας χείρας αυτών και τον Βενιαμίν· και σηκωθέντες κατέβησαν εις Αίγυπτον και παρεστάθησαν έμπροσθεν του Ιωσήφ.
\par 16 Και ότε είδεν ο Ιωσήφ τον Βενιαμίν μετ' αυτών, είπε προς τον επιστάτην της οικίας αυτού, Φέρε τους ανθρώπους εις την οικίαν και σφάξον σφακτόν και ετοίμασον, διότι μετ' εμού θέλουσι φάγει οι άνθρωποι το μεσημέριον.
\par 17 Και έπραξεν ο άνθρωπος καθώς ελάλησεν ο Ιωσήφ· και ο άνθρωπος εισήγαγε τους ανθρώπους εις την οικίαν του Ιωσήφ.
\par 18 Και εφοβήθησαν οι άνθρωποι, διότι εισήχθησαν εις την οικίαν του Ιωσήφ· και είπον, διά το αργύριον το επιστραφέν εις τα σακκία ημών την πρώτην φοράν ημείς εισαγόμεθα, διά να εύρη αφορμήν εναντίον ημών και να επιπέση εφ' ημάς και να λάβη ημάς δούλους και τους όνους ημών.
\par 19 Και προσελθόντες προς τον άνθρωπον τον επιστάτην της οικίας του Ιωσήφ, ελάλησαν προς αυτόν εν τη πύλη της οικίας·
\par 20 και είπον, Δεόμεθα, κύριε· κατέβημεν την πρώτην φοράν διά να αγοράσωμεν τροφάς·
\par 21 και ότε ήλθομεν εις το κατάλυμα, ηνοίξαμεν τα σακκία ημών και ιδού, εκάστου το αργύριον ήτο εν τω στόματι του σακκίου αυτού, το αργύριον ημών σωστόν· όθεν εφέραμεν αυτό οπίσω εις τας χείρας ημών·
\par 22 εφέραμεν και άλλο αργύριον εις τας χείρας ημών, διά να αγοράσωμεν τροφάς· δεν εξεύρομεν τις έβαλε το αργύριον ημών εις τα σακκία ημών.
\par 23 Ο δε είπεν, Ειρήνη εις εσάς· μη φοβείσθε· ο Θεός σας και ο Θεός του πατρός σας, έδωκεν εις εσάς θησαυρόν εις τα σακκία σας· το αργύριόν σας ήλθεν εις εμέ. Και εξήγαγε προς αυτούς τον Συμεών.
\par 24 Και ο άνθρωπος εισήγαγε τους ανθρώπους εις την οικίαν του Ιωσήφ και έδωκεν ύδωρ και ένιψαν τους πόδας αυτών· και έδωκε τροφήν εις τους όνους αυτών.
\par 25 Οι δε ητοίμασαν τα δώρα, εωσού έλθη ο Ιωσήφ το μεσημέριον· διότι ήκουσαν ότι εκεί μέλλουσι να φάγωσιν άρτον.
\par 26 Και ότε ήλθεν ο Ιωσήφ εις την οικίαν, προσέφεραν εις αυτόν τα δώρα τα εις τας χείρας αυτών εν τη οικία και προσεκύνησαν αυτόν έως εδάφους.
\par 27 Και ηρώτησεν αυτούς περί της υγιείας αυτών· και είπεν, Υγιαίνει ο πατήρ σας, ο γέρων περί του οποίου μοι είπετε; έτι ζη;
\par 28 Οι δε είπον, Υγιαίνει ο δούλός σου ο πατήρ ημών· έτι ζη. Και κύψαντες προσεκύνησαν.
\par 29 Υψώσας δε τους οφθαλμούς αυτού είδε τον Βενιαμίν τον αδελφόν αυτού τον ομομήτριον και είπεν, Ούτος είναι ο αδελφός σας ο νεώτερος, περί του οποίου μοι είπετε; Και είπεν, Ο Θεός να σε ελεήση, τέκνον μου.
\par 30 Και έσπευσε να αποσυρθή ο Ιωσήφ· διότι συνεταράττοντο τα σπλάγχνα αυτού διά τον αδελφόν αυτού· και εζήτει τόπον να κλαύση· εισελθών δε εις το ταμείον, έκλαυσεν εκεί.
\par 31 Έπειτα νίψας το πρόσωπον αυτού εξήλθε, και συγκρατών εαυτόν είπε, Βάλετε άρτον.
\par 32 Και έβαλον χωριστά δι' αυτόν και χωριστά δι' εκείνους και διά τους Αιγυπτίους, τους συντρώγοντας μετ' αυτού, χωριστά· διότι οι Αιγύπτιοι δεν ηδύναντο να συμφάγωσιν άρτον μετά των Εβραίων, επειδή τούτο είναι βδέλυγμα εις τους Αιγυπτίους.
\par 33 Εκάθισαν λοιπόν έμπροσθεν αυτού, ο πρωτότοκος κατά την πρωτοτοκίαν αυτού και ο νεώτερος κατά την νεότητα αυτού· και εθαύμαζον οι άνθρωποι προς αλλήλους.
\par 34 Λαβών δε απ' έμπροσθεν αυτού μερίδια έστειλε προς αυτούς· το μερίδιον όμως του Βενιαμίν ήτο πενταπλασίως μεγαλήτερον παρά εκάστου αυτών. Και έπιον και ευφράνθησαν μετ' αυτού.

\chapter{44}

\par 1 Προσέταξε δε τον επιστάτην της οικίας αυτού λέγων, Γέμισον τα σακκία των ανθρώπων τροφάς, όσας δύνανται να φέρωσι, και βάλε το αργύριον εκάστου εν τω στόματι του σακκίου αυτού·
\par 2 και βάλε το ποτήριόν μου, το ποτήριον το αργυρούν, εν τω στόματι του σακκίου του νεωτέρου και το αργύριον του σίτου αυτού. Και έκαμε κατά τον λόγον τον οποίον είπεν ο Ιωσήφ.
\par 3 Το πρωΐ καθώς έφεγξεν, απεστάλησαν οι άνθρωποι, αυτοί και οι όνοι αυτών.
\par 4 Αφού δε εξήλθον εκ της πόλεως, πριν απομακρυνθώσι πολύ, είπεν ο Ιωσήφ προς τον επιστάτην της οικίας αυτού, Σηκωθείς, δράμε κατόπιν των ανθρώπων· και προφθάσας ειπέ προς αυτούς, διά τι ανταπεδώκατε κακόν αντί καλού;
\par 5 δεν είναι τούτο το ποτήριον, εις το οποίον πίνει ο κύριός μου, και διά του οποίου αληθώς μαντεύει; κακώς εκάμετε πράξαντες τούτο.
\par 6 Και καθώς επρόφθασεν αυτούς, είπε προς αυτούς τους λόγους τούτους.
\par 7 Οι δε είπον προς αυτόν, Διά τι ο κύριος ημών λαλεί κατά τους λόγους τούτους; μη γένοιτο οι δούλοί σου να πράξωσι τοιούτον πράγμα·
\par 8 ιδού, το αργύριον, το οποίον ευρήκαμεν εν τω στόματι των σακκίων ημών, επεστρέψαμεν προς σε εκ της γης Χαναάν, και πως ηθέλομεν κλέψει εκ της οικίας του κυρίου σου αργύριον ή χρυσίον;
\par 9 εις όντινα εκ των δούλων σου ευρεθή, ας αποθάνη, και ημείς έτι θέλομεν γείνει δούλοι του κυρίου ημών.
\par 10 Ο δε είπε, Και τώρα ας γείνη καθώς λέγετε· εις όντινα ευρεθή, θέλει γείνει δούλός μου, σεις δε θέλετε είσθαι αθώοι.
\par 11 Και σπεύσαντες, κατεβίβασαν έκαστος το σακκίον αυτού εις την γην και ήνοιξεν έκαστος το σακκίον αυτού.
\par 12 Και ηρεύνησεν, αρχίσας από του πρεσβυτέρου και τελειώσας εις τον νεώτερον· και ευρέθη το ποτήριον εν τω σακκίω του Βενιαμίν.
\par 13 Τότε έσχισαν τα ιμάτια αυτών και φορτώσαντες έκαστος τον όνον αυτού, επέστρεψαν εις την πόλιν.
\par 14 Εισήλθε δε ο Ιούδας και οι αδελφοί αυτού εις την οικίαν του Ιωσήφ, έτι αυτού όντος εκεί· και έπεσαν έμπροσθεν αυτού επί την γην.
\par 15 Και είπε προς αυτούς ο Ιωσήφ, Τι είναι το πράγμα τούτο, το οποίον επράξατε; δεν εξεύρετε ότι άνθρωπος οποίος εγώ αληθώς μαντεύει;
\par 16 Και είπεν ο Ιούδας, Τι να είπωμεν προς τον κύριόν μου; τι να λαλήσωμεν; ή πως να δικαιωθώμεν; ο Θεός εύρηκε την αδικίαν των δούλων σου· ιδού, είμεθα δούλοι του κυρίου μου και εμείς και εκείνος εις τον οποίον ευρέθη το ποτήριον.
\par 17 Ο δε είπε, Μη γένοιτο εις εμέ να πράξω τούτο· ο άνθρωπος εις τον οποίον ευρέθη το ποτήριον, ούτος θέλει είσθαι εις εμέ δούλος· σεις δε ανάβητε εν ειρήνη προς τον πατέρα σας.
\par 18 Τότε επλησίασεν εις αυτόν ο Ιούδας και είπε, Δέομαι, κύριέ μου· ας λαλήση, παρακαλώ, ο δούλός σου λόγον εις τα ώτα του κυρίου μου και ας μη εξαφθή ο θυμός σου κατά του δούλου σου· διότι συ είσαι ως Φαραώ.
\par 19 Ο κύριός μου ηρώτησε τους δούλους αυτού λέγων, Έχετε πατέρα, ή αδελφόν;
\par 20 Και είπομεν προς τον κύριόν μου, Έχομεν πατέρα γέροντα και παιδίον του γήρατος αυτού μικρόν, ο δε αδελφός αυτού απέθανε· και αυτός μόνος έμεινεν εκ της μητρός αυτού και ο πατήρ αυτού αγαπά αυτόν.
\par 21 Και είπας προς τους δούλους σου, Φέρετε αυτόν προς εμέ διά να ίδω αυτόν ιδίοις οφθαλμοίς.
\par 22 Και είπομεν προς τον κύριόν μου, το παιδίον δεν δύναται να αφήση τον πατέρα αυτού διότι εάν αφήση τον πατέρα αυτού, ούτος θέλει αποθάνει.
\par 23 Συ δε είπας προς τους δούλους σου, Εάν δεν καταβή ο αδελφός υμών ο νεώτερος μεθ' υμών, δεν θέλετε ιδεί πλέον το πρόσωπόν μου.
\par 24 Και ότε ανέβημεν προς τον δούλον σου τον πατέρα μου, απηγγείλαμεν προς αυτόν τους λόγους του κυρίου μου.
\par 25 Ο δε πατήρ ημών είπεν, Υπάγετε πάλιν, αγοράσατε εις ημάς ολίγας τροφάς.
\par 26 Και είπομεν, Δεν δυνάμεθα να καταβώμεν· εάν ο αδελφός ημών ο νεώτερος ήναι μεθ' ημών, τότε θέλομεν καταβή· διότι δεν δυνάμεθα να ίδωμεν το πρόσωπον του ανθρώπου, εάν ο αδελφός ημών ο νεώτερος δεν ήναι μεθ' ημών.
\par 27 Και ο δούλός σου ο πατήρ μου είπε προς ημάς, Σεις εξεύρετε ότι δύο υιούς εγέννησεν εις εμέ η γυνή μου·
\par 28 και ο εις εξήλθεν από πλησίον μου και είπα, Βεβαίως κατεσπαράχθη υπό θηρίου· και δεν είδον αυτόν έως του νύν·
\par 29 εάν δε λάβητε και τούτον απ' έμπροσθέν μου και συμβή εις αυτόν συμφορά, θέλετε καταβιβάσει την πολιάν μου μετά λύπης εις τον τάφον.
\par 30 Τώρα λοιπόν όταν υπάγω προς τον δούλον σου τον πατέρα μου, και το παιδίον δεν ήναι μεθ' ημών επειδή η ψυχή αυτού κρέμαται εκ της ψυχής εκείνου,
\par 31 καθώς ίδη ότι το παιδίον δεν είναι, θέλει αποθάνει και οι δούλοί σου θέλουσι καταβιβάσει την πολιάν του δούλου σου του πατρός ημών μετά λύπης εις τον τάφον.
\par 32 Διότι ο δούλός σου εγγυήθη περί του παιδίου προς τον πατέρα μου λέγων, Εάν δεν φέρω αυτόν προς σε, τότε θέλω είσθαι υπεύθυνος προς τον πατέρα μου διαπαντός.
\par 33 Τώρα λοιπόν, δέομαί σου, ας μείνη ο δούλός σου αντί του παιδίου δούλος εις τον κύριόν μου, το δε παιδίον ας αναβή μετά των αδελφών αυτού·
\par 34 διότι πως να αναβώ προς τον πατέρα μου, εάν το παιδίον δεν ήναι μετ' εμού; ουχί, διά να μη ίδω το κακόν, το οποίον θέλει ευρεί τον πατέρα μου.

\chapter{45}

\par 1 Τότε ο Ιωσήφ δεν ηδυνήθη να συγκρατήση εαυτόν ενώπιον πάντων των παρισταμένων έμπροσθεν αυτού· και εφώνησεν, Εκβάλετε πάντας απ' εμού· και δεν έμεινεν ουδείς μετ' αυτού, ενώ ο Ιωσήφ ανεγνωρίζετο εις τους αδελφούς αυτού.
\par 2 και αφήκε φωνήν μετά κλαυθμού· και ήκουσαν οι Αιγύπτιοι· ήκουσε δε και ο οίκος του Φαραώ.
\par 3 Και είπεν ο Ιωσήφ προς τους αδελφούς αυτού, Εγώ είμαι ο Ιωσήφ· ο πατήρ μου έτι ζη; Και δεν ηδύναντο οι αδελφοί αυτού να αποκριθώσι προς αυτόν· διότι εταράχθησαν εκ της παρουσίας αυτού.
\par 4 Και είπεν ο Ιωσήφ προς τους αδελφούς αυτού, Πλησιάσατε προς εμέ, παρακαλώ. Και επλησίασαν. Και είπεν, Εγώ είμαι Ιωσήφ ο αδελφός σας, τον οποίον επωλήσατε εις την Αίγυπτον.
\par 5 Τώρα λοιπόν μη λυπείσθε μηδ' ας φανή εις εσάς σκληρόν ότι με επωλήσατε εδώ· επειδή εις διατήρησιν ζωής με απέστειλεν ο Θεός έμπροσθέν σας.
\par 6 Διότι τούτο είναι το δεύτερον έτος της πείνης επί της γής· και μένουσιν ακόμη πέντε έτη, εις τα οποία δεν θέλει είσθαι ούτε αροτρίασις ούτε θερισμός.
\par 7 Και ο Θεός με απέστειλεν έμπροσθέν σας διά να διατηρήσω εις εσάς διαδοχήν επί της γης και να διαφυλάξω την ζωήν σας μετά μεγάλης λυτρώσεως.
\par 8 Τώρα λοιπόν δεν με απεστείλατε εδώ σεις, αλλ' ο Θεός· και με έκαμε πατέρα εις τον Φαραώ και κύριον παντός του οίκου αυτού και άρχοντα πάσης της γης Αιγύπτου.
\par 9 Σπεύσαντες ανάβητε προς τον πατέρα μου και είπατε προς αυτόν, Ούτω λέγει ο υιός σου Ιωσήφ· Ο Θεός με έκαμε κύριον πάσης Αιγύπτου· κατάβηθι προς εμέ, μη σταθής·
\par 10 και θέλεις κατοικήσει εν γη Γεσέν και θέλεις είσθαι πλησίον μου, συ και οι υιοί σου και οι υιοί των υιών σου και τα ποίμνιά σου και αι αγέλαι σου, και πάντα όσα έχεις·
\par 11 και θέλω σε εκτρέφει εκεί διότι μένουσιν ακόμη πέντε έτη πείνης, διά να μη έλθης εις στέρησιν, συ και ο οίκος σου και πάντα όσα έχεις.
\par 12 Και ιδού, οι οφθαλμοί σας βλέπουσι και οι οφθαλμοί του αδελφού μου Βενιαμίν, ότι το στόμα μου λαλεί προς εσάς·
\par 13 απαγγείλατε λοιπόν προς τον πατέρα μου πάσαν την δόξαν μου εν Αιγύπτω και πάντα όσα είδετε, και σπεύσαντες καταβιβάσατε τον πατέρα μου εδώ.
\par 14 Και πεσών επί τον τράχηλον Βενιαμίν του αδελφού αυτού, έκλαυσε· και ο Βενιαμίν έκλαυσεν επί τον τράχηλον εκείνου.
\par 15 Και καταφιλήσας πάντας τους αδελφούς αυτού, έκλαυσεν επ' αυτούς· και μετά ταύτα ώμίλησαν οι αδελφοί αυτού μετ' αυτού.
\par 16 Και η φήμη ηκούσθη εις τον οίκον του Φαραώ λέγουσα, Οι αδελφοί του Ιωσήφ ήλθον· εχάρη δε ο Φαραώ και οι δούλοι αυτού.
\par 17 Και είπεν ο Φαραώ προς τον Ιωσήφ, Ειπέ προς τους αδελφούς σου, τούτο κάμετε· φορτώσατε τα ζώα σας και υπάγετε, ανάβητε εις γην Χαναάν·
\par 18 και παραλαβόντες τον πατέρα σας, και τας οικογενείας σας, έλθετε προς εμέ· και θέλω σας δώσει τα αγαθά της γης Αιγύπτου και θέλετε φάγει το πάχος της γης.
\par 19 Και συ πρόσταξον· Τούτο κάμετε, λάβετε εις εαυτούς αμάξας εκ της γης Αιγύπτου, διά τα παιδία σας και διά τας γυναίκάς σας· και σηκώσαντες τον πατέρα σας έλθετε·
\par 20 και μη λυπηθήτε την αποσκευήν σας· διότι τα αγαθά πάσης της γης Αιγύπτου θέλουσιν είσθαι ιδικά σας.
\par 21 Και έκαμον ούτως οι υιοί του Ισραήλ· και ο Ιωσήφ έδωκεν εις αυτούς αμάξας κατά την προσταγήν του Φαραώ· έδωκεν εις αυτούς και ζωοτροφίαν διά την οδόν.
\par 22 Εις πάντας αυτούς έδωκεν εις έκαστον αλλαγάς ενδυμάτων· εις δε τον Βενιαμίν έδωκε τριακόσια αργύρια και πέντε αλλαγάς ενδυμάτων.
\par 23 Προς δε τον πατέρα αυτού έστειλε ταύτα· δέκα όνους φορτωμένους εκ των αγαθών της Αιγύπτου και δέκα θηλυκάς όνους φορτωμένας σίτον και άρτους και ζωοτροφίας εις τον πατέρα αυτού διά την οδόν.
\par 24 Και εξαπέστειλε τους αδελφούς αυτού και ανεχώρησαν· και είπε προς αυτούς, Μη συγχύζεσθε καθ' οδόν.
\par 25 Και ανέβησαν εξ Αιγύπτου και ήλθον εις γην Χαναάν προς Ιακώβ τον πατέρα αυτών.
\par 26 Και απήγγειλαν προς αυτόν λέγοντες, Έτι ζη ο Ιωσήφ και είναι άρχων εφ' όλης της γης Αιγύπτου· και ελειποθύμησεν η καρδία αυτού· διότι δεν επίστευεν αυτούς.
\par 27 Είπον δε προς αυτόν πάντας τους λόγους του Ιωσήφ, τους οποίους είχεν ειπεί προς αυτούς· και αφού είδε τας αμάξας τας οποίας έστειλεν ο Ιωσήφ διά να σηκώσωσιν αυτόν, ανεζωπυρήθη το πνεύμα του Ιακώβ του πατρός αυτών.
\par 28 Και είπεν ο Ισραήλ, Αρκεί· Ιωσήφ ο υιός μου έτι ζή· θέλω υπάγει και θέλω ιδεί αυτόν, πριν αποθάνω.

\chapter{46}

\par 1 Αναχωρήσας δε ο Ισραήλ μετά πάντων των υπαρχόντων αυτού, ήλθεν εις Βηρ-σαβεέ και προσέφερε θυσίας εις τον Θεόν του πατρός αυτού Ισαάκ.
\par 2 Και είπεν ο Θεός προς τον Ισραήλ δι' οράματος της νυκτός λέγων, Ιακώβ, Ιακώβ. Ο δε είπεν, Ιδού, εγώ.
\par 3 Και είπεν, Εγώ είμαι ο Θεός, ο Θεός του πατρός σου· μη φοβηθής να καταβής εις Αίγυπτον· διότι έθνος μέγα θέλω σε καταστήσει εκεί·
\par 4 εγώ θέλω καταβή μετά σου εις Αίγυπτον και εγώ βεβαίως θέλω σε αναβιβάσει πάλιν· και ο Ιωσήφ θέλει βάλει τας χείρας αυτού επί τους οφθαλμούς σου.
\par 5 Και εσηκώθη ο Ιακώβ από Βηρ-σαβεέ και έβαλον οι υιοί του Ισραήλ Ιακώβ τον πατέρα αυτών και τα παιδία αυτών και τας γυναίκας αυτών επί τας αμάξας τας οποίας έστειλεν ο Φαραώ διά να σηκώσωσιν αυτόν.
\par 6 Και λαβόντες τα κτήνη αυτών και τα υπάρχοντα αυτών, τα οποία απέκτησαν εν τη γη Χαναάν, ήλθον εις Αίγυπτον, ο Ιακώβ και παν το σπέρμα αυτού μετ' αυτού·
\par 7 τους υιούς αυτού και τους υιούς των υιών αυτού μεθ' εαυτού, τας θυγατέρας αυτού και τας θυγατέρας των υιών αυτού και παν το σπέρμα αυτού έφερε μεθ' εαυτού εις Αίγυπτον.
\par 8 Ταύτα δε είναι τα ονόματα των υιών Ισραήλ, των εισελθόντων εις Αίγυπτον, Ιακώβ και οι υιοί αυτού· Ρουβήν ο πρωτότοκος του Ιακώβ·
\par 9 και οι υιοί του Ρουβήν, Ανώχ και Φαλλού και Εσρών και Χαρμί.
\par 10 Και οι υιοί του Συμεών, Ιεμουήλ και Ιαμείν και Αώδ και Ιαχείν και Σωάρ και Σαούλ, υιός Χανανίτιδος.
\par 11 Και οι υιοί του Λευΐ, Γηρσών, Καάθ και Μεραρί.
\par 12 Και οι υιοί του Ιούδα, Ηρ και Αυνάν και Σηλά και Φαρές και Ζαρά· ο Ηρ όμως και ο Αυνάν απέθανον εν τη γη Χαναάν. Και οι υιοί του Φαρές ήσαν Εσρών και Αμούλ.
\par 13 Και οι υιοί του Ισσάχαρ, Θωλά και Φουά και Ιώβ και Σιμβρών.
\par 14 Και οι υιοί του Ζαβουλών, Σερέδ και Αιλών και Ιαλεήλ.
\par 15 Ούτοι είναι οι υιοί της Λείας, τους οποίους εγέννησεν εις τον Ιακώβ εν Παδάν-αράμ, και Δείναν την θυγατέρα αυτού· πάσαι αι ψυχαί, οι υιοί αυτού και αι θυγατέρες αυτού, ήσαν τριάκοντα τρεις.
\par 16 Και οι υιοί του Γαδ, Σιφών και Αγγί, Σουνί και Εσβών, Ηρί και Αροδί και Αριηλί.
\par 17 Και οι υιοί του Ασήρ, Ιεμνά και Ιεσσουά και Ιεσουεί και Βεριά και Σερά η αδελφή αυτών. Και οι υιοί του Βεριά, Έβερ και Μαλχιήλ.
\par 18 Ούτοι είναι οι υιοί της Ζελφάς, την οποίαν έδωκεν ο Λάβαν εις την Λείαν την θυγατέρα αυτού· και τούτους εγέννησεν εις τον Ιακώβ, δεκαέξ ψυχάς.
\par 19 Οι δε υιοί της Ραχήλ γυναικός του Ιακώβ, Ιωσήφ και Βενιαμίν.
\par 20 Εγεννήθησαν δε εις τον Ιωσήφ εν τη γη της Αιγύπτου Μανασσής και Εφραΐμ, τους οποίους εγέννησεν εις αυτόν Ασενέθ, η θυγάτηρ του Ποτιφερά ιερέως της Ων.
\par 21 Και οι υιοί του Βενιαμίν ήσαν Βελά και Βεχέρ και Ασβήλ, Γηρά και Νααμάν, Ηχί και Ρως, Μουπίμ και Ουπίμ και Αρέδ.
\par 22 Ούτοι είναι οι υιοί της Ραχήλ, οι γεννηθέντες εις τον Ιακώβ· πάσαι αι ψυχαί δεκατέσσαρες.
\par 23 Και οι υιοί του Δαν, Ουσίμ.
\par 24 Και οι υιοί του Νεφθαλί, Ιασιήλ και Γουνί και Ιεσέρ και Σιλλήμ.
\par 25 Ούτοι είναι οι υιοί της Βαλλάς, την οποίαν έδωκεν ο Λάβαν εις Ραχήλ την θυγατέρα αυτού· και τούτους εγέννησεν εις τον Ιακώβ· πάσαι αι ψυχαί επτά.
\par 26 Πάσαι αι ψυχαί αι εισελθούσαι μετά του Ιακώβ εις Αίγυπτον, αίτινες εξήλθον εκ των μηρών αυτού, χωρίς των γυναικών των υιών του Ιακώβ, πάσαι αι ψυχαί ήσαν εξήκοντα εξ.
\par 27 Και οι υιοί του Ιωσήφ, οι γεννηθέντες εις αυτόν εν Αιγύπτω, ήσαν ψυχαί δύο· πάσαι αι ψυχαί του οίκου του Ιακώβ, αι εισελθούσαι εις Αίγυπτον, ήσαν εβδομήκοντα.
\par 28 Απέστειλε δε ο Ιακώβ τον Ιούδαν έμπροσθεν αυτού προς τον Ιωσήφ, διά να καταβή προ αυτού εις Γεσέν· και ήλθον εις την γην Γεσέν.
\par 29 Ζεύξας δε ο Ιωσήφ την άμαξαν αυτού, ανέβη εις συνάντησιν Ισραήλ του πατρός αυτού εις Γεσέν· και ιδών αυτόν, έπεσεν επί τον τράχηλον αυτού και έκλαυσε πολλήν ώραν επί τον τράχηλον αυτού.
\par 30 Και είπεν ο Ισραήλ προς τον Ιωσήφ, Ας αποθάνω τώρα, αφού είδον το πρόσωπόν σου, διότι συ έτι ζης.
\par 31 Είπε δε ο Ιωσήφ προς τους αδελφούς αυτού και προς τον οίκον του πατρός αυτού, Εγώ θέλω αναβή και θέλω απαγγείλει προς τον Φαραώ και ειπεί προς αυτόν, Οι αδελφοί μου και ο οίκος του πατρός μου, οίτινες ήσαν εν τη γη Χαναάν, ήλθον προς εμέ·
\par 32 οι δε άνθρωποι είναι ποιμένες, διότι άνδρες κτηνοτρόφοι είναι και έφεραν τα ποίμνια αυτών και τας αγέλας αυτών και πάντα όσα έχουσι.
\par 33 Εάν λοιπόν σας καλέση ο Φαραώ και είπη, Ποίον το επιτήδευμά σας;
\par 34 θέλετε ειπεί, Άνδρες κτηνοτρόφοι είμεθα οι δούλοί σου εκ νεότητος ημών έως του νυν και ημείς και οι πατέρες ημών· διά να κατοικήσητε εν τη γη Γεσέν· διότι είναι βδέλυγμα εις τους Αιγυπτίους πας ποιμήν προβάτων.

\chapter{47}

\par 1 Ελθών δε ο Ιωσήφ, απήγγειλε προς τον Φαραώ λέγων, Ο πατήρ μου και οι αδελφοί μου, και τα ποίμνια αυτών και αι αγέλαι αυτών και πάντα όσα έχουσιν, ήλθον εκ της γης Χαναάν· και ιδού, είναι εν τη γη Γεσέν.
\par 2 Και παραλαβών εκ των αδελφών αυτού πέντε άνδρας, παρέστησεν αυτούς ενώπιον του Φαραώ.
\par 3 Και είπεν ο Φαραώ προς τους αδελφούς αυτού, Τι είναι το επιτήδευμά σας; οι δε είπον προς τον Φαραώ, Ποιμένες προβάτων είναι οι δούλοί σου και ημείς και οι πατέρες ημών.
\par 4 Είπον έτι προς τον Φαραώ, Ήλθομεν διά να παροικήσωμεν εν τη γή· διότι δεν υπάρχει βοσκή διά τα ποίμνια των δούλων σου, επειδή επεβάρυνεν η πείνα εν τη γη Χαναάν· τώρα λοιπόν ας κατοικήσωσι, παρακαλούμεν, οι δούλοί σου εν τη γη Γεσέν.
\par 5 Και είπεν ο Φαραώ προς τον Ιωσήφ λέγων, Ο πατήρ σου και οι αδελφοί σου ήλθον προς σέ·
\par 6 η γη της Αιγύπτου είναι έμπροσθέν σου· εις το καλήτερον της γης κατοίκισον τον πατέρα σου και τους αδελφούς σου· ας κατοικήσωσιν εν τη γη Γεσέν· και εάν γνωρίζης ότι ευρίσκονται μεταξύ αυτών άνδρες άξιοι, κατάστησον αυτούς επιστάτας επί των ποιμνίων μου.
\par 7 Εισήγαγε δε ο Ιωσήφ Ιακώβ τον πατέρα αυτού και παρέστησεν αυτόν ενώπιον του Φαραώ· και ευλόγησεν ο Ιακώβ τον Φαραώ.
\par 8 Και είπεν ο Φαραώ προς τον Ιακώβ, Ως πόσαι είναι αι ημέραι των ετών της ζωής σου;
\par 9 Και ο Ιακώβ είπε προς τον Φαραώ, Αι ημέραι των ετών της παροικίας μου είναι εκατόν τριάκοντα έτη· ολίγαι και κακαί υπήρξαν αι ημέραι των ετών της ζωής μου και δεν έφθασαν εις τας ημέρας των ετών της ζωής των πατέρων μου εν ταις ημέραις της παροικίας αυτών.
\par 10 Και ευλόγησεν ο Ιακώβ τον Φαραώ και εξήλθεν απ' έμπροσθεν του Φαραώ.
\par 11 Και κατώκισεν ο Ιωσήφ τον πατέρα αυτού και τους αδελφούς αυτού, και έδωκεν εις αυτούς ιδιοκτησίαν εν τη γη της Αιγύπτου, εις το καλήτερον της γης, εν τη γη Ραμεσσή, καθώς προσέταξεν ο Φαραώ.
\par 12 Και έτρεφεν ο Ιωσήφ τον πατέρα αυτού και τους αδελφούς αυτού και πάντα τον οίκον του πατρός αυτού με άρτον, κατά τας οικογενείας αυτών.
\par 13 Και άρτος δεν ήτο καθ' όλην την γήν· διότι η πείνα ήτο βαρεία σφόδρα, ώστε η γη της Αιγύπτου και η γη της Χαναάν απέκαμον υπό της πείνης.
\par 14 Και συνήγαγεν ο Ιωσήφ άπαν το αργύριον, το ευρισκόμενον εν τη γη της Αιγύπτου και εν τη γη Χαναάν, διά τον σίτον τον οποίον ηγόραζον· και έφερεν ο Ιωσήφ το αργύριον εις τον οίκον του Φαραώ.
\par 15 Και αφού εξέλιπε το αργύριον εκ της γης Αιγύπτου και εκ της γης Χαναάν, ήλθον πάντες οι Αιγύπτιοι προς τον Ιωσήφ, λέγοντες, Δος άρτον εις ημάς· επειδή διά τι να αποθάνωμεν έμπροσθέν σου; διότι εξέλιπε το αργύριον.
\par 16 Είπε δε ο Ιωσήφ, Φέρετε τα κτήνη σας και θέλω σας δώσει άρτον αντί των κτηνών σας, εάν εξέλιπε το αργύριον.
\par 17 Και έφεραν τα κτήνη αυτών προς τον Ιωσήφ και έδωκεν εις αυτούς ο Ιωσήφ άρτον αντί των ίππων και αντί των προβάτων και αντί των βοών και αντί των όνων· και έθρεψεν αυτούς με άρτον εν τω ενιαυτώ εκείνω αντί πάντων των κτηνών αυτών.
\par 18 Αφού δε ετελείωσεν ο ενιαυτός εκείνος, ήλθον προς αυτόν το δεύτερον έτος και είπον προς αυτόν, δεν θέλομεν κρύψει από του κυρίου ημών ότι εξέλιπε το αργύριον· και τα κτήνη έγειναν του κυρίου ημών· δεν έμεινεν άλλο έμπροσθεν του κυρίου ημών, ειμή τα σώματα ημών και η γη ημών·
\par 19 διά τι να απολεσθώμεν ενώπιόν σου και ημείς και η γη ημών; αγόρασον ημάς και την γην ημών διά άρτον· και θέλομεν είσθαι ημείς και η γη ημών δούλοι εις τον Φαραώ· και δος εις ημάς σπόρον, διά να ζήσωμεν και να μη αποθάνωμεν και η γη να μη ερημωθή.
\par 20 Και ηγόρασεν ο Ιωσήφ πάσαν την γην Αιγύπτου διά τον Φαραώ· διότι οι Αιγύπτιοι επώλησαν έκαστος τον αγρόν αυτού, επειδή η πείνα υπερεβάρυνεν επ' αυτούς· ούτως η γη έγεινε του Φαραώ·
\par 21 τον δε λαόν μετετόπισεν αυτόν εις πόλεις, απ' άκρου των ορίων της Αιγύπτου έως άκρου αυτής·
\par 22 μόνον την γην των ιερέων δεν ηγόρασε· διότι οι ιερείς είχον μερίδιον προσδιωρισμένον υπό του Φαραώ· και έτρωγον το μερίδιον αυτών, το οποίον έδωκεν εις αυτούς ο Φαραώ· διά τούτο δεν επώλησαν την γην αυτών.
\par 23 Τότε είπεν ο Ιωσήφ προς τον λαόν, Ιδού, ηγόρασα εσάς και την γην σας σήμερον εις τον Φαραώ· ιδού, λάβετε σπόρον και σπείρατε την γήν·
\par 24 και εν τω καιρώ των γεννημάτων, θέλετε δώσει το πέμπτον εις τον Φαραώ· τα δε τέσσαρα μέρη θέλουσιν είσθαι εις εσάς διά σπόρον των αγρών και διά τροφήν σας και διά τους όντας εν τοις οίκοις υμών και διά τροφήν των παιδίων υμών.
\par 25 Οι δε είπον, συ έσωσας την ζωήν ημών· ας εύρωμεν χάριν έμπροσθεν του κυρίου ημών και θέλομεν είσθαι δούλοι του Φαραώ.
\par 26 Και έθεσεν ο Ιωσήφ τούτο νόμον εν τη γη της Αιγύπτου μέχρι της σήμερον, να δίδεται το πέμπτον εις τον Φαραώ· εκτός της γης των ιερέων μόνον, ήτις δεν έγεινε του Φαραώ.
\par 27 Κατώκησε δε ο Ισραήλ εν τη γη της Αιγύπτου, εν τη γη Γεσέν· και απέκτησαν εν αυτή κτήματα, και ηυξήνθησαν και επληθύνθησαν σφόδρα.
\par 28 Επέζησε δε ο Ιακώβ εν τη γη της Αιγύπτου δεκαεπτά έτη· και έγειναν αι ημέραι των ετών της ζωής του Ιακώβ εκατόν τεσσαράκοντα επτά έτη.
\par 29 Και επλησίασαν αι ημέραι του Ισραήλ διά να αποθάνη· και καλέσας τον υιόν αυτού τον Ιωσήφ, είπε προς αυτόν, Εάν εύρηκα τώρα χάριν έμπροσθέν σου, βάλε, παρακαλώ, την χείρα σου υπό τον μηρόν μου, και κάμε εις εμέ έλεος και αλήθειαν· μη με θάψης, παρακαλώ, εν τη Αιγύπτω·
\par 30 αλλά θέλω κοιμηθή μετά των πατέρων μου και θέλεις με μετακομίσει εκ της Αιγύπτου και θέλεις με θάψει εν τω τάφω αυτών. Ο δε είπεν, Εγώ θέλω κάμει κατά τον λόγον σου.
\par 31 Ο δε είπεν, Ομοσόν μοι και ώμοσεν εις αυτόν. Και προσεκύνησεν ο Ισραήλ επί το άκρον της ράβδου αυτού.

\chapter{48}

\par 1 Μετά δε τα πράγματα ταύτα, είπον προς τον Ιωσήφ, Ιδού, ο πατήρ σου ασθενεί. Και έλαβε μεθ' εαυτού τους δύο υιούς αυτού, τον Μανασσή και τον Εφραΐμ.
\par 2 Και απήγγειλαν προς τον Ιακώβ, λέγοντες, Ιδού, ο υιός σου Ιωσήφ έρχεται προς σέ· και αναλαβών δύναμιν, ο Ισραήλ εκάθισεν επί την κλίνην.
\par 3 Και είπεν ο Ιακώβ προς τον Ιωσήφ, Ο Θεός ο Παντοδύναμος εφάνη εις εμέ εν Λούζ εν τη γη Χαναάν και με ευλόγησε·
\par 4 και είπε προς εμέ, Ιδού, εγώ θέλω σε αυξήσει και θέλω σε πληθύνει και θέλω σε καταστήσει εις πλήθος λαών· και θέλω δώσει την γην ταύτην εις το σπέρμα σου μετά σε παντοτεινήν ιδιοκτησίαν.
\par 5 Τώρα λοιπόν οι δύο υιοί σου, οι γεννηθέντες εις σε εν τη γη της Αιγύπτου, πριν εγώ έλθω προς σε εις την Αίγυπτον είναι ιδικοί μου· ο Εφραΐμ και ο Μανασσής θέλουσιν είσθαι εις εμέ ως ο Ρουβήν και ο Συμεών·
\par 6 τα δε τέκνα σου όσα γεννήσης μετά τούτους, θέλουσιν είσθαι ιδικά σου· κατά το όνομα των αδελφών αυτών θέλουσιν ονομασθή εν τη κληρονομία αυτών.
\par 7 Ότε δε εγώ ηρχόμην από Παδάν, απέθανεν εις εμέ η Ραχήλ καθ' οδόν εν τη γη Χαναάν, ενώ δεν έλειπεν ειμή ολίγον διάστημα διά να φθάσωμεν εις Εφραθά· και έθαψα αυτήν εκεί εν τη οδώ της Εφραθά· αύτη είναι η Βηθλεέμ.
\par 8 Ιδών δε ο Ισραήλ τους υιούς του Ιωσήφ, είπε, Τίνες είναι ούτοι;
\par 9 και είπεν ο Ιωσήφ προς τον πατέρα αυτού, Ούτοι είναι οι υιοί μου, τους οποίους μοι έδωκεν ο Θεός ενταύθα. Ο δε είπε, Φέρε αυτούς, παρακαλώ, προς εμέ, διά να ευλογήσω αυτούς.
\par 10 Ήσαν δε οι οφθαλμοί του Ισραήλ βαρυωποί υπό του γήρατος, δεν ηδύνατο να βλέπη. Και επλησίασεν αυτούς προς αυτόν· και εφίλησεν αυτούς και ενηγκαλίσθη αυτούς.
\par 11 Και είπεν ο Ισραήλ προς τον Ιωσήφ, Δεν ήλπιζον να ίδω το πρόσωπόν σου· και ιδού, ο Θεός έδειξεν εις εμέ και το σπέρμα σου.
\par 12 Και εξήγαγεν αυτούς ο Ιωσήφ εκ μέσου των γονάτων αυτού. Και προσεκύνησεν επί πρόσωπον έως εδάφους.
\par 13 Λαβών δε αυτούς ο Ιωσήφ αμφοτέρους, τον Εφραΐμ εν τη δεξιά αυτού προς την αριστεράν του Ισραήλ, και τον Μανασσή εν τη αριστερά αυτού προς την δεξιάν του Ισραήλ, επλησίασεν εις αυτόν.
\par 14 Και εκτείνας ο Ισραήλ την δεξιάν αυτού επέθεσεν επί την κεφαλήν του Εφραΐμ, όστις ήτο ο νεώτερος, την δε αριστεράν αυτού επί την κεφαλήν του Μανασσή, εναλλάξας τας χείρας αυτού· διότι ο Μανασσής ήτο ο πρωτότοκος.
\par 15 Και ευλόγησε τον Ιωσήφ και είπεν, Ο Θεός, έμπροσθεν του οποίου περιεπάτησαν οι πατέρες μου Αβραάμ και Ισαάκ, ο Θεός όστις με εποίμανεν εκ γεννήσεώς μου έως της ημέρας ταύτης,
\par 16 ο άγγελος όστις με ελύτρωσεν εκ πάντων των κακών, να ευλογήση τα παιδία ταύτα· και να ονομασθή επ' αυτά το όνομά μου και το όνομα των πατέρων μου Αβραάμ και Ισαάκ, και να πληθυνθώσιν εις πλήθος μέγα επί της γης.
\par 17 Και ιδών ο Ιωσήφ ότι επέθεσεν ο πατήρ αυτού την χείρα αυτού την δεξιάν επί την κεφαλήν του Εφραΐμ, δυσηρεστήθη· και επίασε την χείρα του πατρός αυτού διά να μεταθέση αυτήν από της κεφαλής του Εφραΐμ επί την κεφαλήν του Μανασσή.
\par 18 Και είπεν ο Ιωσήφ προς τον πατέρα αυτού, Μη ούτω, πάτερ μου, διότι ούτος είναι ο πρωτότοκος· επίθες την δεξιάν σου επί την κεφαλήν αυτού.
\par 19 Αλλ' ο πατήρ αυτού δεν ηθέλησε· και είπεν, Εξεύρω, τέκνον μου, εξεύρω· και ούτος θέλει κατασταθή λαός και ούτος έτι θέλει γείνει μέγας· αλλ' όμως ο αδελφός αυτού ο νεώτερος θέλει είσθαι μεγαλήτερος αυτού και το σπέρμα αυτού θέλει γείνει πλήθος εθνών.
\par 20 Και ευλόγησεν αυτούς την ημέραν εκείνην, λέγων, Εις σε αναφερόμενος θέλει ευλογεί ο Ισραήλ, λέγων, Ο Θεός να σε κάμη ως τον Εφραΐμ και ως τον Μανασσή. Και έστησε τον Εφραΐμ προ του Μανασσή.
\par 21 Και είπεν ο Ισραήλ προς τον Ιωσήφ, Ιδού, εγώ αποθνήσκω· και ο Θεός θέλει είσθαι με σας και θέλει σας επαναφέρει εις την γην των πατέρων σας·
\par 22 και εγώ δίδω εις σε μερίδιον εν υπέρ τους αδελφούς σου, το οποίον έλαβον εκ της χειρός των Αμορραίων διά της μαχαίρας μου και διά του τόξου μου.

\chapter{49}

\par 1 Εκάλεσε δε ο Ιακώβ τους υιούς αυτού και είπε, Συνάχθητε, διά να σας αναγγείλω τι μέλλει να συμβή εις εσάς εν ταις εσχάταις ημέραις·
\par 2 συνάχθητε και ακούσατε, υιοί του Ιακώβ, και ακροάσθητε τον Ισραήλ τον πατέρα σας.
\par 3 Ρουβήν ο πρωτότοκός μου, συ ισχύς μου και αρχή των δυνάμεών μου, έξοχος κατά την αξίαν και έξοχος κατά την δύναμιν.
\par 4 Εξέβρασας ως ύδωρ· δεν θέλεις έχει την υπεροχήν· διότι ανέβης επί την κλίνην του πατρός σου· τότε εμίανας αυτήν· επί την κλίνην μου ανέβη.
\par 5 Συμεών και Λευΐ οι αδελφοί, όργανα αδικίας είναι αι μάχαιραι αυτών·
\par 6 εις την βουλήν αυτών μη εισέλθης, ψυχή μου· εις την συνέλευσιν αυτών μη ενωθής, τιμή μου· διότι εν τω θυμώ αυτών εφόνευσαν ανθρώπους και εν τω πείσματι αυτών κατηδάφισαν τείχος.
\par 7 Επικατάρατος ο θυμός αυτών, διότι ήτο αυθάδης· και η οργή αυτών, διότι ήτο σκληρά· θέλω διαμοιράσει αυτούς εις τον Ιακώβ, και θέλω διασκορπίσει αυτούς εις τον Ισραήλ.
\par 8 Ιούδα, σε θέλουσι επαινέσει οι αδελφοί σου· η χειρ σου θέλει είσθαι επί τον τράχηλον των εχθρών σου· οι υιοί του πατρός σου θέλουσι σε προσκυνήσει·
\par 9 σκύμνος λέοντος είναι ο Ιούδας· εκ του θηρεύματος, υιέ μου, ανέβης· αναπεσών εκοιμήθη ως λέων και ως σκύμνος λέοντος· τις θέλει εγείρει αυτόν;
\par 10 Δεν θέλει εκλείψει το σκήπτρον εκ του Ιούδα ουδέ νομοθέτης εκ μέσου των ποδών αυτού, εωσού έλθη ο Σηλώ· και εις αυτόν θέλει είσθαι η υπακοή των λαών.
\par 11 Εις την άμπελον δένει το πωλάριον αυτού, και εις τον εκλεκτόν βλαστόν το παιδίον της όνου αυτού· θέλει πλύνει εν οίνω το ένδυμα αυτού και εν τω αίματι της σταφυλής το περιβόλαιον αυτού·
\par 12 Οι οφθαλμοί αυτού θέλουσιν είσθαι ερυθροί εκ του οίνου και οι οδόντες αυτού λευκοί εκ του γάλακτος.
\par 13 Ο Ζαβουλών θέλει κατοικήσει εν λιμένι θαλάσσης και θέλει είσθαι εν λιμένι πλοίων· το δε όριον αυτού θέλει εκταθή έως Σιδώνος.
\par 14 Ο Ισσάχαρ είναι όνος δυνατός, κοιτώμενος εν τω μέσω των επαύλεων·
\par 15 Και ιδών ότι η ανάπαυσις ήτο καλή και ο τόπος τερπνός, έκλινε τον ώμον αυτού εις φορτίον και έγεινε δούλος υποτελής.
\par 16 Ο Δαν θέλει κρίνει τον λαόν αυτού, ως μία εκ των φυλών του Ισραήλ·
\par 17 Ο Δαν θέλει είσθαι όφις επί της οδού, ασπίς επί της τρίβου, δάκνων τας πτέρνας του ίππου, ώστε ο ιππεύς αυτού θέλει πίπτει εις τα οπίσω.
\par 18 Την σωτηρίαν σου περιέμεινα, Κύριε.
\par 19 Τον Γαδ θέλουσι πειρατεύσει πειραταί· πλην και αυτός εις το τέλος θέλει πειρατεύσει.
\par 20 Του Ασήρ ο άρτος θέλει είσθαι παχύς· και αυτός θέλει δίδει βασιλικάς τρυφάς.
\par 21 Ο Νεφθαλί είναι έλαφος απολελυμένη, δίδων λόγους αρεστούς.
\par 22 Ο Ιωσήφ, κλάδος καρποφόρος, κλάδος καρποφόρος πλησίον πηγής, του οποίου οι βλαστοί εκτείνονται επί του τοίχου·
\par 23 Οι τοξόται επίκραναν αυτόν και ετόξευσαν κατ' αυτού, και εχθρεύθησαν αυτόν.
\par 24 Αλλά το τόξον αυτού έμεινε δυνατόν και οι βραχίονες των χειρών αυτού ενεδυναμώθησαν διά των χειρών του ισχυρού Θεού του Ιακώβ· εκείθεν ο ποιμήν, η πέτρα του Ισραήλ·
\par 25 και τούτο διά του Θεού του πατρός σου, όστις θέλει σε βοηθεί, και διά του Παντοδυνάμου, όστις θέλει σε ευλογεί, ευλογίας του ουρανού άνωθεν, ευλογίας της αβύσσου κάτωθεν, ευλογίας των μαστών και της μήτρας·
\par 26 Αι ευλογίαι του πατρός σου υπερίσχυσαν υπέρ τας ευλογίας των προγόνων μου έως των υψηλών κορυφών των αιωνίων ορέων· θέλουσιν είσθαι επί της κεφαλής του Ιωσήφ και επί της κορυφής του εκλεκτού μεταξύ των αδελφών αυτού.
\par 27 Ο Βενιαμίν θέλει είσθαι λύκος άρπαξ· το πρωΐ θέλει κατατρώγει θήραμα, και το εσπέρας θέλει διαιρεί λάφυρα.
\par 28 Πάντες ούτοι είναι αι δώδεκα φυλαί του Ισραήλ, και τούτο είναι το οποίον ελάλησε προς αυτούς ο πατήρ αυτών και ευλόγησεν αυτούς· έκαστον κατά την ευλογίαν αυτού ευλόγησεν αυτούς.
\par 29 Και παρήγγειλεν εις αυτούς και είπε προς αυτούς, Εγώ προστίθεμαι εις τον λαόν μου· θάψατέ με μετά των πατέρων μου εν τω σπηλαίω τω εν τω αγρώ Εφρών του Χετταίου·
\par 30 εν τω σπηλαίω τω εν τω αγρώ Μαχπελάχ τω απέναντι της Μαμβρή εν τη γη Χαναάν, το οποίον ο Αβραάμ ηγόρασε μετά του αγρού παρά του Εφρών του Χετταίου διά κτήμα μνημείου·
\par 31 εκεί έθαψαν τον Αβραάμ και την Σάρραν την γυναίκα αυτού· εκεί έθαψαν τον Ισαάκ και την Ρεβέκκαν την γυναίκα αυτού· και εκεί έθαψα εγώ την Λείαν·
\par 32 η αγορά του αγρού και του σπηλαίου του εν αυτώ έγεινε παρά των υιών του Χετ.
\par 33 Και αφού ετελείωσεν ο Ιακώβ παραγγέλλων εις τους υιούς αυτού, έσυρε τους πόδας αυτού επί την κλίνην και εξέπνευσε· και προσετέθη εις τον λαόν αυτού.

\chapter{50}

\par 1 Και έπεσεν ο Ιωσήφ επί το πρόσωπον του πατρός αυτού και έκλαυσεν επ' αυτόν και εφίλησεν αυτόν.
\par 2 Και προσέταξεν ο Ιωσήφ τους δούλους αυτού τους ιατρούς να βαλσαμώσωσι τον πατέρα αυτού· και εβαλσάμωσαν οι ιατροί τον Ισραήλ.
\par 3 Και συνεπληρώθησαν δι' αυτόν τεσσαράκοντα ημέραι· διότι ούτω συμπληρούνται αι ημέραι της βαλσαμώσεως· και επένθησαν αυτόν οι Αιγύπτιοι εβδομήκοντα ημέρας.
\par 4 Αφού δε παρήλθον αι ημέραι του πένθους αυτού, ελάλησεν ο Ιωσήφ προς τον οίκον του Φαραώ, λέγων, Εάν τώρα εύρηκα χάριν ενώπιόν σας, λαλήσατε, παρακαλώ, εις τα ώτα του Φαραώ, λέγοντες,
\par 5 Ο πατήρ μου με ώρκισε, λέγων, Ιδού, εγώ αποθνήσκω· εις το μνημείόν μου, το οποίον έσκαψα εις εμαυτόν εν γη Χαναάν, εκεί θέλεις με θάψει· τώρα λοιπόν ας αναβώ, παρακαλώ, και ας θάψω τον πατέρα μου· και θέλω επιστρέψει.
\par 6 Και είπεν ο Φαραώ, Ανάβηθι και θάψον τον πατέρα σου καθώς σε ώρκισε.
\par 7 Και ανέβη ο Ιωσήφ διά να θάψη τον πατέρα αυτού· και συνανέβησαν μετ' αυτού πάντες οι δούλοι του Φαραώ, οι πρεσβύτεροι του οίκου αυτού, και πάντες οι πρεσβύτεροι της γης Αιγύπτου
\par 8 και πας ο οίκος του Ιωσήφ και οι αδελφοί αυτού και ο οίκος του πατρός αυτού· μόνον τας οικογενείας αυτών και τα ποίμνια αυτών και τας αγέλας αυτών αφήκαν εν τη γη Γεσέν.
\par 9 Και συνανέβησαν μετ' αυτού και άμαξαι και ιππείς· ώστε έγεινε συνοδία μεγάλη σφόδρα·
\par 10 και ήλθον εις το αλώνιον του Ατάδ το πέραν του Ιορδάνου· και εκεί εθρήνησαν θρήνον μέγαν και δυνατόν σφόδρα· και έκαμεν ο Ιωσήφ διά τον πατέρα αυτού πένθος επτά ημέρας.
\par 11 Και ιδόντες οι κάτοικοι του τόπου, οι Χαναναίοι, το πένθος εν τω αλωνίω του Ατάδ, είπον, Πένθος μέγα είναι τούτο εις τους Αιγυπτίους· διά τούτο ωνομάσθη το όνομα αυτού Αβέλ-μισραΐμ, το οποίον είναι πέραν του Ιορδάνου.
\par 12 Και έκαμον εις αυτόν οι υιοί αυτού καθώς παρήγγειλεν εις αυτούς·
\par 13 και μετακομίσαντες αυτόν οι υιοί αυτού εις γην Χαναάν, έθαψαν αυτόν εν τω σπηλαίω του αγρού Μαχπελάχ, το οποίον ο Αβραάμ ηγόρασε μετά του αγρού διά κτήμα μνημείου παρά του Εφρών του Χετταίου κατέναντι της Μαμβρή.
\par 14 Και αφού ο Ιωσήφ έθαψε τον πατέρα αυτού, επέστρεψεν εις Αίγυπτον αυτός και οι αδελφοί αυτού και πάντες οι συναναβάντες μετ' αυτού διά να θάψωσι τον πατέρα αυτού.
\par 15 Και ιδόντες οι αδελφοί του Ιωσήφ ότι απέθανεν ο πατήρ αυτών, είπον, Ίσως ο Ιωσήφ θέλει μνησικακήσει εις ημάς και θέλει μας ανταποδώσει αυστηρώς πάντα τα κακά όσα επράξαμεν εις αυτόν.
\par 16 Και εμήνυσαν προς τον Ιωσήφ, λέγοντες, Ο πατήρ σου προσέταξε, πριν αποθάνη, λέγων,
\par 17 Ούτω θέλετε ειπεί προς τον Ιωσήφ· Συγχώρησον, παρακαλώ, την αδικίαν των αδελφών σου και την αμαρτίαν αυτών· διότι έπραξαν κακόν εις σέ· τώρα λοιπόν συγχώρησον, παρακαλούμεν, την αδικίαν των δούλων του Θεού του πατρός σου. Και έκλαυσεν ο Ιωσήφ ότε ελάλησαν προς αυτόν.
\par 18 Υπήγαν δε και οι αδελφοί αυτού και πεσόντες έμπροσθεν αυτού, είπον, Ιδού, ημείς είμεθα δούλοί σου.
\par 19 Και είπε προς αυτούς ο Ιωσήφ, Μη φοβείσθε· μήπως αντί Θεού είμαι εγώ;
\par 20 σεις μεν εβουλεύθητε κακόν εναντίον μου· ο δε Θεός εβουλεύθη να μεταστρέψη τούτο εις καλόν, διά να γείνη καθώς την σήμερον, ώστε να σώση την ζωήν πολλού λαού·
\par 21 τώρα λοιπόν μη φοβείσθε· εγώ θέλω θρέψει εσάς και τας οικογενείας σας. Και παρηγόρησεν αυτούς και ελάλησε κατά την καρδίαν αυτών.
\par 22 Και κατώκησεν ο Ιωσήφ εν Αιγύπτω, αυτός και ο οίκος του πατρός αυτού· και έζησεν ο Ιωσήφ εκατόν δέκα έτη.
\par 23 Και είδεν ο Ιωσήφ τέκνα του Εφραΐμ, έως τρίτης γενεάς· και τα παιδία του Μαχείρ, υιού του Μανασσή, επί των γονάτων του Ιωσήφ εγεννήθησαν.
\par 24 Και είπεν ο Ιωσήφ προς τους αδελφούς αυτού, Εγώ αποθνήσκω· ο δε Θεός θέλει βεβαίως σας επισκεφθή και θέλει σας αναβιβάσει εκ της γης ταύτης εις την γην, την οποίαν ώμοσε προς τον Αβραάμ, προς τον Ισαάκ και προς τον Ιακώβ.
\par 25 Και ώρκισεν ο Ιωσήφ τους υιούς Ισραήλ, λέγων, Ο Θεός βεβαίως θέλει σας επισκεφθή και θέλετε αναβιβάσει τα οστά μου εντεύθεν.
\par 26 Και ετελεύτησεν ο Ιωσήφ εν ηλικία ετών εκατόν δέκα· και εβαλσάμωσαν αυτόν· και ετέθη εις θήκην εν τη Αιγύπτω.





\end{document}