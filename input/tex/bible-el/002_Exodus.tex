\begin{document}

\title{Exodus}


\chapter{1}

\par Και ταύτα είναι τα ονόματα των υιών Ισραήλ, των εισελθόντων εις Αίγυπτον μετά του Ιακώβ· έκαστος μετά της οικογένειας αυτού εισήλθον.
\par 2 Ρουβήν, Συμεών, Λευΐ και Ιούδας,
\par 3 Ισσάχαρ, Ζαβουλών και Βενιαμίν,
\par 4 Δαν και Νεφθαλί, Γαδ και Ασήρ.
\par 5 Και πάσαι αι ψυχαί αι εξελθούσαι εκ του μηρού του Ιακώβ ήσαν ψυχαί εβδομήκοντα· ο δε Ιωσήφ ήτο ήδη εν Αιγύπτω.
\par 6 Ετελεύτησε δε ο Ιωσήφ και πάντες οι αδελφοί αυτού, και πάσα η γενεά εκείνη.
\par 7 Και ηυξήνθησαν οι υιοί Ισραήλ και επληθύνθησαν, και επολλαπλασιάσθησαν, και ενεδυναμώθησαν σφόδρα, ώστε ο τόπος εγέμισεν απ' αυτών.
\par 8 Εσηκώθη δε νέος βασιλεύς επί την Αίγυπτον, όστις δεν εγνώριζε τον Ιωσήφ.
\par 9 Και είπε προς τον λαόν αυτού, Ιδού, ο λαός των υιών Ισραήλ είναι πολύ πλήθος και ισχυρότερος ημών·
\par 10 έλθετε, ας σοφισθώμεν κατ' αυτών, διά να μη πολλαπλασιασθώσι, και αν συμβή πόλεμος ενωθώσι και ούτοι μετά των εχθρών ημών και πολεμήσωσιν ημάς και αναχωρήσωσιν εκ του τόπου.
\par 11 Κατέστησαν λοιπόν επ' αυτούς επιστάτας των εργασιών, διά να καταθλίβωσιν αυτούς με τα βάρη αυτών· και ωκοδόμησαν εις τον Φαραώ πόλεις αποθηκών, την Πιθώμ και την Ραμεσσή.
\par 12 Όσον όμως κατέθλιβον αυτούς, τόσω μάλλον επληθύνοντο και ηυξάνοντο. Και οι Αιγύπτιοι απεστρέφοντο τους υιούς Ισραήλ.
\par 13 Και κατεδυνάστευον οι Αιγύπτιοι τους υιούς Ισραήλ αυστηρώς·
\par 14 και κατεπίκραινον την ζωήν αυτών διά της σκληράς δουλείας εις τον πηλόν και εις τας πλίνθους, και εις πάσας τας εργασίας των πεδιάδων· πάσαι αι εργασίαι αυτών, με τας οποίας κατεδυνάστευον αυτούς, ήσαν αυστηραί.
\par 15 Και ελάλησεν ο βασιλεύς των Αιγυπτίων προς τας μαίας των Εβραίων, εκ των οποίων η μία ωνομάζετο Σεπφώρα, και η άλλη Φουά,
\par 16 και είπεν, Όταν μαιεύητε τας Εβραίας και ίδητε αυτάς επί της γέννας, εάν μεν ήναι αρσενικόν, θανατόνετε αυτό· εάν δε ήναι θηλυκόν, τότε ας ζήση.
\par 17 Εφοβήθησαν δε αι μαίαι τον Θεόν και δεν έκαμνον ως είπε προς αυτάς ο βασιλεύς της Αιγύπτου, αλλ' άφινον ζώντα τα αρσενικά.
\par 18 Καλέσας δε ο βασιλεύς της Αιγύπτου τας μαίας, είπε προς αυτάς, Διά τι εκάμετε το πράγμα τούτο, και αφίνετε ζώντα τα αρσενικά;
\par 19 Και απεκρίθησαν αι μαίαι προς τον Φαραώ, Ότι αι Εβραίαι δεν είναι ως αι γυναίκες της Αιγύπτου· διότι είναι εύρωστοι και γεννώσι πριν εισέλθωσιν εις αυτάς αι μαίαι.
\par 20 Ο δε Θεός ηγαθοποίει τας μαίας· και επληθύνετο ο λαός και ενεδυναμούτο σφόδρα.
\par 21 Και επειδή αι μαίαι εφοβούντο τον Θεόν, έκαμεν εις αυτάς οίκους.
\par 22 Ο δε Φαραώ προσέταξε πάντα τον λαόν αυτού, λέγων, Παν αρσενικόν το οποίον γεννηθή, εις τον ποταμόν ρίπτετε αυτό· παν δε θηλυκόν, αφίνετε να ζη.

\chapter{2}

\par Υπήγε δε άνθρωπός τις εκ του οίκου Λευΐ, και έλαβεν εις γυναίκα μίαν εκ των θυγατέρων Λευΐ.
\par 2 Και συνέλαβεν η γυνή και εγέννησεν υιόν· ιδούσα δε αυτόν ότι ήτο εύμορφος, έκρυψεν αυτόν τρεις μήνας.
\par 3 Μη δυναμένη δε να κρύπτη αυτόν πλέον, έλαβε δι' αυτόν κιβώτιον σπάρτινον και κατέχρισεν αυτό με άσφαλτον και πίσσαν και ενέβαλε το παιδίον εις αυτό και έθεσεν εις το ελώδες μέρος παρά το χείλος του ποταμού.
\par 4 Η δε αδελφή αυτού παρεμόνευε μακρόθεν, διά να ίδη το αποβησόμενον εις αυτό.
\par 5 Και κατέβη η θυγάτηρ του Φαραώ διά να λουσθή εις τον ποταμόν, αι δε θεράπαιναι αυτής περιεπάτουν επί την όχθην του ποταμού· και ότε είδε το κιβώτιον εις το ελώδες μέρος, έστειλε την παιδίσκην αυτής και έλαβεν αυτό·
\par 6 και ανοίξασα βλέπει το παιδίον και ιδού, το νήπιον έκλαιε· και ελυπήθη αυτό, λέγουσα, Εκ των παιδίων των Εβραίων είναι τούτο.
\par 7 Τότε είπεν η αδελφή αυτού προς την θυγατέρα του Φαραώ, Θέλεις να υπάγω να καλέσω εις σε γυναίκα θηλάζουσαν εκ των Εβραίων, διά να σοι θηλάση το παιδίον;
\par 8 Και είπε προς αυτήν η θυγάτηρ του Φαραώ, Ύπαγε. Και υπήγε το κοράσιον και εκάλεσε την μητέρα του παιδίου.
\par 9 Και είπε προς αυτήν η θυγάτηρ του Φαραώ, Λάβε το παιδίον τούτο και θήλασόν μοι αυτό, και εγώ θέλω σοι δώσει τον μισθόν σου.
\par 10 Έλαβε δε η γυνή το παιδίον και εθήλαζεν αυτό. Και αφού εμεγάλωσε το παιδίον, έφερεν αυτό προς την θυγατέρα του Φαραώ, και έγεινεν υιός αυτής· και εκάλεσε το όνομα αυτού Μωϋσήν, λέγουσα, Ότι εκ του ύδατος ανέσυρα αυτό.
\par 11 Κατά δε τας ημέρας εκείνας, αφού ο Μωϋσής εμεγάλωσεν, εξήλθε προς τους αδελφούς αυτού· και παρατηρών τα βάρη αυτών, βλέπει άνθρωπον Αιγύπτιον τύπτοντα Εβραίον τινά εκ των αδελφών αυτού.
\par 12 Περιβλέψας δε εδώ και εκεί και ιδών ότι δεν ήτο ουδείς, επάταξε τον Αιγύπτιον και έκρυψεν αυτόν εν τη άμμω.
\par 13 Και εξήλθε την ακόλουθον ημέραν και ιδού, δύο άνδρες Εβραίοι διεπληκτίζοντο· και λέγει προς τον αδικούντα, Διά τι τύπτεις τον πλησίον σου;
\par 14 Ο δε είπε, Τις σε κατέστησεν άρχοντα και κριτήν εφ' ημάς; Μήπως θέλεις συ να με φονεύσης, καθώς εφόνευσας τον Αιγύπτιον; Και εφοβήθη ο Μωϋσής και είπε, Βεβαίως το πράγμα τούτο έγεινε γνωστόν.
\par 15 Ακούσας δε ο Φαραώ το πράγμα τούτο, εζήτει να θανατώση τον Μωϋσήν· αλλ' ο Μωϋσής έφυγεν από προσώπου του Φαραώ και κατώκησεν εν τη γη Μαδιάμ· εκάθισε δε πλησίον του φρέατος.
\par 16 Ο δε ιερεύς της Μαδιάμ είχεν επτά θυγατέρας, αίτινες ελθούσαι ήντλησαν ύδωρ και εγέμισαν τας ποτίστρας διά να ποτίσωσι τα πρόβατα του πατρός αυτών.
\par 17 Ελθόντες δε οι ποιμένες εδίωξαν αυτάς· και σηκωθείς ο Μωϋσής εβοήθησεν αυτάς και επότισε τα πρόβατα αυτών.
\par 18 Και ότε ήλθον προς Ραγουήλ τον πατέρα αυτών, είπε προς αυτάς, Διά τι τόσον ταχέως ήλθετε σήμερον;
\par 19 Αι δε είπον, Άνθρωπος Αιγύπτιος ελύτρωσεν ημάς εκ των χειρών των ποιμένων και προσέτι ήντλησεν εις ημάς ύδωρ και επότισε τα πρόβατα.
\par 20 Ο δε είπε προς τας θυγατέρας αυτού, Και που είναι; διά τι αφήκατε τον άνθρωπον; καλέσατε αυτόν διά να φάγη άρτον.
\par 21 Και ευχαριστήθη ο Μωϋσής να κατοική μετά του ανθρώπου· όστις έδωκεν εις τον Μωϋσήν εις γυναίκα Σεπφώραν την θυγατέρα αυτού.
\par 22 Και εγέννησεν υιόν· και εκάλεσε το όνομα αυτού Γηρσώμ, λέγων, Πάροικος είμαι εν ξένη γή·
\par 23 Μετά δε πολύν καιρόν, ετελεύτησεν ο βασιλεύς της Αιγύπτου· και κατεστέναξαν οι υιοί Ισραήλ διά την δουλείαν και ανεβόησαν· και η βοή αυτών ανέβη προς τον Θεόν εξ αιτίας της δουλείας.
\par 24 Και εισήκουσεν ο Θεός των στεναγμών αυτών· και ενεθυμήθη ο Θεός την διαθήκην αυτού την προς τον Αβραάμ, τον Ισαάκ και τον Ιακώβ·
\par 25 και επέβλεψεν ο Θεός επί τους υιούς Ισραήλ και ηλέησεν αυτούς ο Θεός.

\chapter{3}

\par Ο δε Μωϋσής έβοσκε τα πρόβατα του Ιοθόρ, πενθερού αυτού, ιερέως της Μαδιάμ· και έφερε τα πρόβατα εις το όπισθεν μέρος της ερήμου και ήλθεν εις το όρος του Θεού, το Χωρήβ.
\par 2 Εφάνη δε εις αυτόν άγγελος Κυρίου εν φλογί πυρός εκ μέσου της βάτου· και είδε και ιδού, η βάτος εκαίετο υπό του πυρός και η βάτος δεν κατεκαίετο.
\par 3 Και είπεν ο Μωϋσής, Ας στρέψω και ας παρατηρήσω το μέγα τούτο θέαμα, διά τι η βάτος δεν κατακαίεται.
\par 4 Και ως είδεν ο Κύριος τον Μωϋσήν ότι έστρεψε να παρατηρήση, εφώνησε προς αυτόν ο Θεός εκ μέσου της βάτου και είπε, Μωϋσή, Μωϋσή. Ο δε είπεν, Ιδού, εγώ.
\par 5 Και είπε, Μη πλησιάσης εδώ· λύσον τα υποδήματά σου εκ των ποδών σου· διότι ο τόπος επί του οποίου ίστασαι είναι γη αγία.
\par 6 Και είπε προς αυτόν, Εγώ είμαι ο Θεός του πατρός σου, ο Θεός του Αβραάμ, ο Θεός του Ισαάκ και ο Θεός του Ιακώβ. Έκρυψε δε το πρόσωπον αυτού ο Μωϋσής· διότι εφοβείτο να εμβλέψη εις τον Θεόν.
\par 7 Και είπεν ο Κύριος, Είδον, είδον την ταλαιπωρίαν του λαού μου του εν Αιγύπτω και ήκουσα την κραυγήν αυτών εξ αιτίας των εργοδιωκτών αυτών· διότι εγνώρισα την οδύνην αυτών·
\par 8 και κατέβην διά να ελευθερώσω αυτούς εκ της χειρός των Αιγυπτίων και να αναβιβάσω αυτούς εκ της γης εκείνης εις γην καλήν και ευρύχωρον, εις γην ρέουσαν γάλα και μέλι, εις τον τόπον των Χαναναίων και Χετταίων και Αμορραίων και Φερεζαίων και Ευαίων και Ιεβουσαίων·
\par 9 και τώρα ιδού, η κραυγή των υιών Ισραήλ ήλθεν εις εμέ· και είδον έτι την κατάθλιψιν, με την οποίαν οι Αιγύπτιοι καταθλίβουσιν αυτούς·
\par 10 ελθέ λοιπόν τώρα και θέλω σε αποστείλει προς τον Φαραώ, και θέλεις εξαγάγει τον λαόν μου τους υιούς Ισραήλ εξ Αιγύπτου.
\par 11 Και απεκρίθη ο Μωϋσής προς τον Θεόν, Τις είμαι εγώ, διά να υπάγω προς τον Φαραώ και να εξαγάγω τους υιούς Ισραήλ εξ Αιγύπτου;
\par 12 Και είπεν ο Θεός, Βεβαίως εγώ θέλω είσθαι μετά σού· και τούτο θέλει είσθαι εις σε το σημείον, ότι εγώ σε απέστειλα· Αφού εξαγάγης τον λαόν μου εξ Αιγύπτου, θέλετε λατρεύσει τον Θεόν επί του όρους τούτου.
\par 13 Και είπεν ο Μωϋσής προς τον Θεόν, Ιδού, όταν εγώ υπάγω προς τους υιούς Ισραήλ και είπω προς αυτούς, Ο Θεός των πατέρων σας με απέστειλε προς εσάς, και εκείνοι μ' ερωτήσωσι, Τι είναι το όνομα αυτού; τι θέλω ειπεί προς αυτούς;
\par 14 Και είπεν ο Θεός προς τον Μωϋσήν, Εγώ είμαι ο Ων· και είπεν, Ούτω θέλεις ειπεί προς τους υιούς Ισραήλ· Ο Ων με απέστειλε προς εσάς.
\par 15 Και είπεν έτι ο Θεός προς τον Μωϋσήν, Ούτω θέλεις ειπεί προς τους υιούς Ισραήλ· Κύριος ο Θεός των πατέρων σας, ο Θεός του Αβραάμ, ο Θεός του Ισαάκ και ο Θεός του Ιακώβ, με απέστειλε προς εσάς· τούτο θέλει είσθαι το όνομά μου εις τον αιώνα και τούτο το μνημόσυνόν μου εις γενεάς γενεών·
\par 16 ύπαγε και σύναξον τους πρεσβυτέρους του Ισραήλ και ειπέ προς αυτούς, Κύριος ο Θεός των πατέρων σας, ο Θεός του Αβραάμ, του Ισαάκ και του Ιακώβ, εφάνη εις εμέ, λέγων, Επεσκέφθην αληθώς εσάς και τα όσα κάμνουσιν εις εσάς εν Αιγύπτω·
\par 17 και είπα, Θέλω σας αναβιβάσει εκ της ταλαιπωρίας των Αιγυπτίων εις την γην των Χαναναίων και Χετταίων και Αμορραίων και Φερεζαίων και Ευαίων και Ιεβουσαίων, εις γην ρέουσαν γάλα και μέλι·
\par 18 και θέλουσιν υπακούσει εις την φωνήν σου· και θέλεις υπάγει, συ και οι πρεσβύτεροι του Ισραήλ, προς τον βασιλέα της Αιγύπτου και θέλετε ειπεί προς αυτόν, Κύριος ο Θεός των Εβραίων συνήντησεν ημάς· τώρα λοιπόν άφες να υπάγωμεν οδόν τριών ημερών εις την έρημον, διά να προσφέρωμεν θυσίαν εις Κύριον τον Θεόν ημών·
\par 19 εγώ δε εξεύρω, ότι δεν θέλει σας αφήσει ο βασιλεύς της Αιγύπτου να υπάγητε, ειμή διά χειρός κραταιάς·
\par 20 και εκτείνας την χείρα μου, θέλω πατάξει την Αίγυπτον με πάντα τα θαυμάσιά μου τα οποία θέλω κάμει εν μέσω αυτής· και μετά ταύτα θέλει σας εξαποστείλει·
\par 21 και θέλω δώσει χάριν εις τον λαόν τούτον έμπροσθεν των Αιγυπτίων· και όταν αναχωρήτε, δεν θέλετε αναχωρήσει κενοί·
\par 22 αλλά πάσα γυνή θέλει ζητήσει παρά της γείτονος αυτής και παρά της συγκατοίκου αυτής σκεύη αργυρά και σκεύη χρυσά και ενδύματα· και θέλετε επιθέσει αυτά επί τους υιούς σας και επί τας θυγατέρας σας και θέλετε γυμνώσει τους Αιγυπτίους.

\chapter{4}

\par Απεκρίθη δε ο Μωϋσής και είπε, Αλλ' ιδού, δεν θέλουσι πιστεύσει εις εμέ ουδέ θέλουσιν εισακούσει εις την φωνήν μου· διότι θέλουσιν ειπεί, δεν εφάνη εις σε ο Κύριος.
\par 2 Και είπε προς αυτόν ο Κύριος, Τι είναι τούτο, το εν τη χειρί σου; Ο δε είπε, Ράβδος.
\par 3 Και είπε, Ρίψον αυτήν κατά γης. Και έρριψεν αυτήν κατά γης και έγεινεν όφις· και έφυγεν ο Μωϋσής απ' έμπροσθεν αυτού.
\par 4 Και είπε Κύριος προς τον Μωϋσήν, Έκτεινον την χείρα σου και πίασον αυτόν από της ουράς· και εκτείνας την χείρα αυτού επίασεν αυτόν και έγεινε ράβδος εν τη χειρί αυτού·
\par 5 διά να πιστεύσωσιν ότι εφάνη εις σε Κύριος ο Θεός των πατέρων αυτών, ο Θεός του Αβραάμ, ο Θεός του Ισαάκ και ο Θεός του Ιακώβ.
\par 6 Και είπεν έτι προς αυτόν ο Κύριος, Βάλε τώρα την χείρα σου εις τον κόλπον σου. Και έβαλε την χείρα αυτού εις τον κόλπον αυτού· και ότε εξήγαγεν αυτήν, ιδού, η χειρ αυτού λεπρά ως χιών.
\par 7 Και είπε, Βάλε πάλιν την χείρα σου εις τον κόλπον σου. Και έβαλε την χείρα αυτού εις τον κόλπον αυτού· και ότε εξήγαγεν αυτήν εκ του κόλπου αυτού, ιδού, αποκατεστάθη καθώς η σαρξ αυτού.
\par 8 Εάν δε, είπεν ο Κύριος, δεν πιστεύσωσιν εις σε μηδέ εισακούσωσιν εις την φωνήν του σημείου του πρώτου, θέλουσι πιστεύσει εις την φωνήν του σημείου του δευτέρου·
\par 9 εάν δε δεν πιστεύσωσι και εις τα δύο ταύτα σημεία μηδέ εισακούσωσιν εις την φωνήν σου, θέλεις λάβει εκ του ύδατος του ποταμού και θέλεις χύσει αυτό επί της ξηράς· και το ύδωρ, το οποίον ήθελες λάβει εκ του ποταμού, θέλει γείνει αίμα επί της ξηράς.
\par 10 Και είπεν ο Μωϋσής προς τον Κύριον, Δέομαι, Κύριε· εγώ δεν είμαι εύλαλος ούτε από χθές ούτε από προχθές ούτε αφ' ης ώρας ελάλησας προς τον δούλον σου· αλλ' είμαι βραδύστομος και βραδύγλωσσος.
\par 11 Και είπε προς αυτόν ο Κύριος, Τις έδωκε στόμα εις τον άνθρωπον; ή τις έκαμε τον εύλαλον, ή τον κωφόν ή τον βλέποντα ή τον τυφλόν; ουχί εγώ ο Κύριος;
\par 12 ύπαγε λοιπόν τώρα και εγώ θέλω είσθαι μετά του στόματός σου και θέλω σε διδάξει ό,τι μέλλεις να λαλήσης.
\par 13 Ο δε είπε, Δέομαι, Κύριε, απόστειλον όντινα έχεις να αποστείλης.
\par 14 Και εξήφθη ο θυμός του Κυρίου κατά του Μωϋσέως· και είπε, Δεν είναι Ααρών ο αδελφός σου ο Λευΐτης; εξεύρω ότι αυτός δύναται να λαλή καλώς· και μάλιστα, ιδού, εξέρχεται εις συνάντησίν σου και όταν σε ίδη, θέλει χαρή εν τη καρδία αυτού·
\par 15 συ λοιπόν θέλεις λαλεί προς αυτόν και θέλεις βάλλει τους λόγους εις το στόμα αυτού· εγώ δε θέλω είσθαι μετά του στόματός σου και μετά του στόματος εκείνου και θέλω σας διδάξει ό,τι πρέπει να πράξητε·
\par 16 και αυτός θέλει λαλεί αντί σου προς τον λαόν· και αυτός θέλει είσθαι εις σε αντί στόματος, συ δε θέλεις είσθαι εις αυτόν αντί Θεού·
\par 17 λάβε δε εις την χείρα σου την ράβδον ταύτην, με την οποίαν θέλεις κάμνει τα σημεία.
\par 18 Και ανεχώρησεν ο Μωϋσής και επέστρεψε προς τον Ιοθόρ τον πενθερόν αυτού και είπε προς αυτόν, Ας υπάγω, παρακαλώ, και ας επιστρέψω προς τους αδελφούς μου, τους εν Αιγύπτω, και ας ίδω αν ζώσιν έτι. Και είπεν ο Ιοθόρ προς τον Μωϋσήν, Ύπαγε εν ειρήνη.
\par 19 Ο δε Κύριος είπε προς τον Μωϋσήν εν Μαδιάμ, Ύπαγε, επίστρεψον εις Αίγυπτον· διότι απέθανον πάντες οι άνθρωποι οι ζητούντες την ψυχήν σου.
\par 20 Τότε παραλαβών ο Μωϋσής την γυναίκα αυτού και τα τέκνα αυτού και καθίσας αυτά επί όνους επέστρεψεν εις την γην της Αιγύπτου· έλαβε δε ο Μωϋσής την ράβδον του Θεού εν τη χειρί αυτού.
\par 21 Και είπε Κύριος προς τον Μωϋσήν, Όταν υπάγης και επιστρέψης εις Αίγυπτον, ιδέ να κάμης έμπροσθεν του Φαραώ πάντα τα θαυμάσια, τα οποία έδωκα εις την χείρα σου· πλην εγώ θέλω σκληρύνει την καρδίαν αυτού, και δεν θέλει εξαποστείλει τον λαόν·
\par 22 και θέλεις ειπεί προς τον Φαραώ, Ούτω λέγει Κύριος· Υιός μου είναι, πρωτότοκός μου, ο Ισραήλ·
\par 23 και προς σε λέγω, Εξαπόστειλον τον υιόν μου, διά να με λατρεύση· και εάν δεν θέλης να εξαποστείλης αυτόν, ιδού, εγώ θέλω θανατώσει τον υιόν σου, τον πρωτότοκόν σου.
\par 24 Ενώ δε ο Μωϋσής ήτο εν τη οδώ, εν τω καταλύματι, συνήντησεν αυτόν ο Κύριος και εζήτει να θανατώση αυτόν.
\par 25 Και λαβούσα η Σεπφώρα λιθάριον κοπτερόν, περιέτεμε την ακροβυστίαν του υιού αυτής, και έρριψεν εις τους πόδας αυτού, λέγουσα, Βεβαίως νυμφίος αιμάτων είσαι εις εμέ.
\par 26 Και απήλθεν απ' αυτού· η δε είπε, Νυμφίος αιμάτων είσαι ένεκα της περιτομής.
\par 27 Είπε δε Κύριος προς τον Ααρών, Ύπαγε προς συνάντησιν του Μωϋσέως εις την έρημον. Και υπήγε και συνήντησεν αυτόν εν τω όρει του Θεού και ησπάσθη αυτόν.
\par 28 Και απήγγειλεν ο Μωϋσής προς τον Ααρών πάντας τους λόγους του Κυρίου, τους οποίους παρήγγειλεν εις αυτόν, και πάντα τα σημεία, τα οποία προσέταξεν εις αυτόν.
\par 29 Υπήγαν λοιπόν ο Μωϋσής και ο Ααρών και συνήγαγον πάντας τους πρεσβυτέρους των υιών Ισραήλ·
\par 30 και ελάλησεν ο Ααρών πάντας τους λόγους, τους οποίους ο Κύριος ελάλησε προς τον Μωϋσήν, και έκαμε τα σημεία ενώπιον του λαού.
\par 31 Και επίστευσεν ο λαός· και ότε ήκουσεν ότι ο Κύριος επεσκέφθη τους υιούς Ισραήλ και ότι επέβλεψεν επί την ταλαιπωρίαν αυτών, κύψαντες προσεκύνησαν.

\chapter{5}

\par Μετά δε ταύτα, εισελθόντες ο Μωϋσής και ο Ααρών, είπαν προς τον Φαραώ, Ούτω λέγει Κύριος ο Θεός του Ισραήλ· Εξαπόστειλον τον λαόν μου, διά να εορτάσωσιν εις εμέ εν τη ερήμω.
\par 2 Ο δε Φαραώ είπε, Τις είναι ο Κύριος, εις του οποίου την φωνήν θέλω υπακούσει, ώστε να εξαποστείλω τον Ισραήλ; δεν γνωρίζω τον Κύριον και ουδέ τον Ισραήλ θέλω εξαποστείλει.
\par 3 Οι δε είπον, Ο Θεός των Εβραίων συνήντησεν ημάς· άφες λοιπόν να υπάγωμεν οδόν τριών ημερών εις την έρημον, διά να προσφέρωμεν θυσίαν εις Κύριον τον Θεόν ημών, μήποτε έλθη καθ' ημών με θανατικόν ή με μάχαιραν.
\par 4 Και είπε προς αυτούς ο βασιλεύς της Αιγύπτου, Διά τι, Μωϋσή και Ααρών, αποκόπτετε τον λαόν από των εργασιών αυτού; υπάγετε εις τα έργα σας.
\par 5 Και είπεν ο Φαραώ, Ιδού, ο λαός του τόπου είναι τώρα πολυπληθής και σεις κάμνετε αυτούς να παύωσιν από των έργων αυτών.
\par 6 Και την αυτήν ημέραν προσέταξεν ο Φαραώ τους εργοδιώκτας τον λαού και τους επιτρόπους αυτών, λέγων,
\par 7 Δεν θέλετε δώσει πλέον εις τον λαόν τούτον άχυρον καθώς χθές και προχθές, διά να κάμνωσι τας πλίνθους· ας υπάγωσιν αυτοί και ας συνάγωσιν εις εαυτούς άχυρον·
\par 8 θέλετε όμως επιβάλει εις αυτούς το ποσόν των πλίνθων, το οποίον έκαμνον πρότερον· παντελώς δεν θέλετε ελαττώσει αυτό· διότι μένουσιν αργοί και διά τούτο φωνάζουσι, λέγοντες, Άφες να υπάγωμεν, διά να προσφέρωμεν θυσίαν εις τον Θεόν ημών·
\par 9 ας επιβαρυνθώσιν αι εργασίαι των ανθρώπων τούτων, διά να ήναι ενησχολημένοι εις αυτάς και να μη προσέχωσιν εις λόγια μάταια.
\par 10 Εξήλθον λοιπόν οι εργοδιώκται του λαού και οι επίτροποι αυτού και ελάλησαν προς τον λαόν, λέγοντες, Ούτως είπεν ο Φαραώ· Δεν σας δίδω άχυρον·
\par 11 σεις αυτοί υπάγετε, συνάγετε άχυρον, όπου δύνασθε να εύρητε· πλην δεν θέλει ελαττωθή εκ των εργασιών σας ουδέν.
\par 12 Και διεσπάρη ο λαός καθ' όλην την γην της Αιγύπτου, διά να συνάγη καλάμην αντί αχύρου.
\par 13 Οι δε εργοδιώκται εβίαζον αυτούς, λέγοντες, Τελειόνετε τας εργασίας σας, το διωρισμένον καθ' ημέραν, καθώς ότε εδίδετο το άχυρον.
\par 14 Και εμαστιγώθησαν οι επίτροποι των υιών Ισραήλ, οι διωρισμένοι επ' αυτούς υπό των εργοδιωκτών του Φαραώ, λεγόντων, Διά τι δεν ετελειώσατε χθές και σήμερον το διωρισμένον εις εσάς ποσόν των πλίνθων, καθώς πρότερον;
\par 15 Εισελθόντες δε οι επίτροποι των υιών Ισραήλ, κατεβόησαν προς τον Φαραώ, λέγοντες, Διά τι κάμνεις ούτως εις τους δούλους σου;
\par 16 άχυρον δεν δίδεται εις τους δούλους σου και λέγουσιν εις ημάς, Κάμνετε πλίνθους· και ιδού, εμαστιγώθησαν οι δούλοί σου· το δε σφάλμα είναι του λαού σου.
\par 17 Ο δε απεκρίθη, Οκνηροί είσθε, οκνηροί· διά τούτο λέγετε, Άφες να υπάγωμεν να προσφέρωμεν θυσίαν προς τον Κύριον·
\par 18 υπάγετε λοιπόν τώρα, δουλεύετε· διότι άχυρον δεν θέλει σας δοθή· θέλετε όμως αποδίδει το ποσόν των πλίνθων.
\par 19 Και έβλεπον εαυτούς οι επίτροποι των υιών Ισραήλ εν κακή περιστάσει, αφού ερρέθη προς αυτούς, Δεν θέλει ελαττωθή ουδέν από του καθημερινού ποσού των πλίνθων.
\par 20 Εξερχόμενοι δε από του Φαραώ, συνήντησαν τον Μωϋσήν και τον Ααρών, ερχομένους εις συνάντησιν αυτών·
\par 21 και είπον προς αυτούς, Ο Κύριος να σας ίδη και να κρίνη· διότι σεις εκάμετε βδελυκτήν την οσμήν ημών έμπροσθεν του Φαραώ και έμπροσθεν των δούλων αυτού, ώστε να δώσητε εις τας χείρας αυτών μάχαιραν διά να θανατώσωσιν ημάς.
\par 22 Και επέστρεψεν ο Μωϋσής προς τον Κύριον και είπε, Κύριε, διά τι κατέθλιψας τον λαόν τούτον; και διά τι με απέστειλας;
\par 23 διότι, αφού ήλθον προς τον Φαραώ να ομιλήσω εν ονόματί σου, κατέθλιψε τον λαόν τούτον· και συ ποσώς δεν ηλευθέρωσας τον λαόν σου.

\chapter{6}

\par Και είπε Κύριος προς τον Μωϋσήν, Τώρα θέλεις ιδεί τι θέλω κάμει εις τον Φαραώ· διότι διά χειρός κραταιάς θέλει εξαποστείλει αυτούς· και διά χειρός κραταιάς θέλει εκδιώξει αυτούς εκ της γης αυτού.
\par 2 Ο Θεός ελάλησεν έτι προς τον Μωϋσήν και είπε προς αυτόν, Εγώ είμαι ο Κυριος·
\par 3 και εφάνην εις τον Αβραάμ, εις τον Ισαάκ και εις τον Ιακώβ, με το όνομα, Θεός Παντοκράτωρ· δεν εγνωρίσθην όμως εις αυτούς με το όνομά μου Ιεοβά·
\par 4 και έτι έστησα προς αυτούς την διαθήκην μου, να δώσω εις αυτούς την γην Χαναάν την γην της παροικίας αυτών, εν ή παρώκησαν·
\par 5 εγώ προσέτι ήκουσα τους στεναγμούς των υιών Ισραήλ διά την υπό των Αιγυπτίων καταδούλωσιν αυτών· και ενεθυμήθην την διαθήκην μου·
\par 6 διά τούτο ειπέ προς τους υιούς Ισραήλ, Εγώ είμαι ο Κύριος· και θέλω σας εκβάλει υποκάτωθεν των φορτίων των Αιγυπτίων και θέλω σας ελευθερώσει από της δουλείας αυτών και θέλω σας λυτρώσει με βραχίονα εξηπλωμένον και με κρίσεις μεγάλας·
\par 7 και θέλω σας λάβει εις εμαυτόν διά λαόν μου και θέλω είσθαι Θεός υμών· και θέλετε γνωρίσει ότι εγώ είμαι Κύριος ο Θεός υμών, όστις σας εκβάλλω υποκάτωθεν των φορτίων των Αιγυπτίων·
\par 8 και θέλω σας φέρει εις την γην, περί της οποίας ύψωσα την χείρα μου, ότι θέλω δώσει αυτήν εις τον Αβραάμ, εις τον Ισαάκ και εις τον Ιακώβ· και θέλω σας δώσει αυτήν εις κληρονομίαν. Εγώ ο Κύριος.
\par 9 Και ελάλησεν ο Μωϋσής ούτω προς τους υιούς Ισραήλ· αλλά δεν εισήκουσαν εις τον Μωϋσήν, διά την στενοχωρίαν της ψυχής αυτών και διά την σκληράν δουλείαν.
\par 10 Και ελάλησε Κύριος προς τον Μωϋσήν, λέγων,
\par 11 Είσελθε, λάλησον προς Φαραώ, τον βασιλέα της Αιγύπτου, διά να εξαποστείλη τους υιούς Ισραήλ εκ της γης αυτού.
\par 12 Και ελάλησεν ο Μωϋσής ενώπιον του Κυρίου, λέγων, Ιδού, οι υιοί Ισραήλ δεν μου εισήκουσαν· και πως θέλει μου εισακούσει ο Φαραώ, και εγώ είμαι απερίτμητος τα χείλη;
\par 13 και ελάλησε Κύριος προς τον Μωϋσήν και προς τον Ααρών και απέστειλεν αυτούς προς τους υιούς Ισραήλ και προς Φαραώ τον βασιλέα της Αιγύπτου, διά να εξαγάγωσι τους υιούς Ισραήλ εκ γης Αιγύπτου.
\par 14 Ούτοι είναι οι αρχηγοί των οίκων των πατριών αυτών· Οι υιοί του Ρουβήν, του πρωτοτόκου του Ισραήλ, Ανώχ και Φαλλού, Εσρών και Χαρμί· αύται είναι αι συγγένειαι του Ρουβήν.
\par 15 Και οι υιοί του Συμεών, Ιεμουήλ και Ιαμείν και Αώδ και Ιαχείν και Σωάρ και Σαούλ υιός Χανανίτιδος· αύται είναι αι συγγένειαι του Συμεών.
\par 16 Τα δε ονόματα των υιών του Λευΐ κατά τας γενεάς αυτών είναι ταύτα· Γηρσών και Καάθ και Μεραρί· και τα έτη της ζωής του Λευΐ έγειναν εκατόν τριάκοντα επτά έτη.
\par 17 Οι υιοί του Γηρσών, Λιβνί και Σεμεΐ, κατά τας συγγενείας αυτών.
\par 18 Και οι υιοί του Καάθ, Αμράμ και Ισαάρ και Χεβρών και Οζιήλ· και τα έτη της ζωής του Καάθ έγειναν εκατόν τριάκοντα τρία έτη.
\par 19 Και οι υιοί του Μεραρί, Μααλί και Μουσί· αύται είναι αι συγγένειαι του Λευΐ, κατά τας γενεάς αυτών.
\par 20 Έλαβε δε ο Αμράμ εις γυναίκα εαυτού την Ιωχαβέδ, θυγατέρα του αδελφού του πατρός αυτού· και εγέννησεν εις αυτόν τον Ααρών και τον Μωϋσήν· τα δε έτη της ζωής του Αμράμ έγειναν εκατόν τριάκοντα επτά έτη.
\par 21 Και οι υιοί του Ισαάρ, Κορέ και Νεφέγ και Ζιθρί.
\par 22 Και οι υιοί του Οζιήλ, Μισαήλ και Ελισαφάν και Σιθρί.
\par 23 Έλαβε δε ο Ααρών εις γυναίκα εαυτού την Ελισάβετ, θυγατέρα του Αμμιναδάβ, αδελφήν του Ναασσών· και εγέννησεν εις αυτόν τον Ναδάβ και τον Αβιούδ, τον Ελεάζαρ και τον Ιθάμαρ.
\par 24 Και οι υιοί του Κορέ, Ασείρ και Ελκανά και Αβιάσαφ· αύται είναι αι συγγένειαι των Κοριτών.
\par 25 Ο δε Ελεάζαρ, ο υιός του Ααρών, έλαβεν εις γυναίκα εαυτού μίαν εκ των θυγατέρων του Φουτιήλ· και εγέννησεν εις αυτόν τον Φινεές· ούτοι είναι οι αρχηγοί των πατριών των Λευϊτών, κατά τας συγγενείας αυτών.
\par 26 Ούτοι είναι ο Ααρών και ο Μωϋσής, προς τους οποίους είπεν ο Κύριος, Εξαγάγετε τους υιούς Ισραήλ, εκ γης Αιγύπτου κατά τα τάγματα αυτών.
\par 27 Ούτοι είναι οι λαλήσαντες προς Φαραώ τον βασιλέα της Αιγύπτου, διά να εξαγάγωσι τους υιούς Ισραήλ εξ Αιγύπτου· αυτοί, ο Μωϋσής και ο Ααρών.
\par 28 Καθ' ην δε ημέραν ελάλησε Κύριος προς τον Μωϋσήν εν τη γη της Αιγύπτου,
\par 29 είπε Κύριος προς τον Μωϋσήν, λέγων, Εγώ είμαι ο Κύριος· λάλησον προς Φαραώ, τον βασιλέα της Αιγύπτου, πάντα όσα λέγω προς σε.
\par 30 Και είπεν ο Μωϋσής ενώπιον του Κυρίου, Ιδού, εγώ είμαι απερίτμητος τα χείλη· και πως θέλει μου εισακούσει ο Φαραώ;

\chapter{7}

\par Και είπε Κύριος προς τον Μωϋσήν, Ιδέ, εγώ σε κατέστησα Θεόν εις τον Φαραώ· και Ααρών ο αδελφός σου θέλει είσθαι προφήτης σου·
\par 2 συ θέλεις λαλήσει πάντα όσα σε προστάζω· ο δε Ααρών ο αδελφός σου θέλει λαλήσει προς τον Φαραώ, διά να εξαποστείλη τους υιούς Ισραήλ εκ της γης αυτού·
\par 3 εγώ δε θέλω σκληρύνει την καρδίαν του Φαραώ και θέλω πληθύνει τα σημείά μου και τα θαυμάσιά μου εν τη γη της Αιγύπτου·
\par 4 πλην ο Φαραώ δεν θέλει σας υπακούσει και θέλω επιβάλει την χείρα μου επί την Αίγυπτον και θέλω εξαγάγει τα στρατεύματά μου, τον λαόν μου, τους υιούς Ισραήλ, εκ γης Αιγύπτου με κρίσεις μεγάλας·
\par 5 και θέλουσι γνωρίσει οι Αιγύπτιοι ότι εγώ είμαι ο Κύριος, όταν εκτείνω την χείρα μου επί την Αίγυπτον και εξαγάγω τους υιούς Ισραήλ εκ μέσου αυτών.
\par 6 Έκαμον δε ο Μωϋσής και ο Ααρών καθώς προσέταξεν εις αυτούς ο Κύριος· ούτως έκαμον.
\par 7 Ήτο δε ο Μωϋσής ηλικίας ογδοήκοντα ετών, ο δε Ααρών ογδοήκοντα τριών ετών, ότε ελάλησαν προς τον Φαραώ.
\par 8 Και είπε Κύριος προς τον Μωϋσήν, και προς τον Ααρών, λέγων,
\par 9 Όταν σας είπη ο Φαραώ, λέγων, Δείξατε σεις θαύμα· τότε θέλεις ειπεί προς τον Ααρών, Λάβε την ράβδον σου και ρίψον έμπροσθεν του Φαραώ· και θέλει γείνει όφις.
\par 10 Εισήλθον λοιπόν ο Μωϋσής και ο Ααρών προς τον Φαραώ, και έκαμον ούτως ως προσέταξεν ο Κύριος· και έρριψεν ο Ααρών την ράβδον αυτού έμπροσθεν του Φαραώ και έμπροσθεν των δούλων αυτού, και έγεινεν όφις.
\par 11 Εκάλεσε δε και ο Φαραώ τους σοφούς και τους μάγους· και οι μάγοι της Αιγύπτου έκαμον και αυτοί ωσαύτως με τας επωδάς αυτών.
\par 12 Διότι έρριψαν έκαστος την ράβδον αυτού, και έγειναν όφεις· η ράβδος όμως του Ααρών κατέπιε τας ράβδους εκείνων.
\par 13 Και εσκληρύνθη η καρδία του Φαραώ και δεν εισήκουσεν εις αυτούς, καθώς ελάλησεν ο Κύριος.
\par 14 Και είπε Κύριος προς τον Μωϋσήν, Εσκληρύνθη η καρδία του Φαραώ, ώστε να μη εξαποστείλη τον λαόν·
\par 15 ύπαγε προς τον Φαραώ το πρωΐ· ιδού, εξέρχεται εις το ύδωρ· και θέλεις σταθή παρά το χείλος του ποταμού, διά να συναντήσης αυτόν· και την ράβδον, την μεταβληθείσαν εις όφιν, θέλεις κρατεί εις την χείρα σου·
\par 16 και θέλεις ειπεί προς αυτόν· Κύριος ο Θεός των Εβραίων με απέστειλε προς σε, λέγων, Εξαπόστειλον τον λαόν μου, διά να με λατρεύση εν τη ερήμω· αλλ ιδού, δεν εισήκουσας έως του νύν·
\par 17 ούτω λέγει Κύριος· Με τούτο θέλεις γνωρίσει ότι εγώ είμαι ο Κύριος· ιδού, με την ράβδον την εν τη χειρί μου θέλω κτυπήσει επί τα ύδατα του ποταμού και θέλουσι μεταβληθή εις αίμα·
\par 18 και τα οψάρια τα εν τω ποταμώ θέλουσι τελευτήσει, και ο ποταμός θέλει βρωμήσει, και οι Αιγύπτιοι θέλουσιν αηδιάσει να πίωσιν ύδωρ εκ του ποταμού·
\par 19 Και είπε Κύριος προς τον Μωϋσήν, Ειπέ προς τον Ααρών, Λάβε την ράβδον σου και έκτεινον την χείρα σου επί τα ύδατα της Αιγύπτου, επί τους ρύακας αυτών, επί τους ποταμούς αυτών, επί τας λίμνας αυτών, και επί πάσαν συναγωγήν ύδατος αυτών, και θέλουσι γείνει αίμα· και θέλει είσθαι αίμα καθ' όλην την γην της Αιγύπτου και εις τα ξύλινα και πέτρινα αγγεία.
\par 20 Και έκαμον ούτως ο Μωϋσής και ο Ααρών καθώς προσέταξεν ο Κύριος· και υψώσας ο Ααρών την ράβδον, εκτύπησε τα ύδατα του ποταμού ενώπιον του Φαραώ και ενώπιον των θεραπόντων αυτού· και μετεβλήθησαν εις αίμα πάντα τα ύδατα του ποταμού.
\par 21 Και τα οψάρια τα εν τω ποταμώ ετελεύτησαν, και ο ποταμός εβρώμησεν, ώστε οι Αιγύπτιοι δεν ηδύναντο να πίωσιν ύδωρ εκ του ποταμού· και ήτο αίμα καθ' όλην την γην της Αιγύπτου.
\par 22 Έκαμον δε το όμοιον και οι μάγοι της Αιγύπτου με τας επωδάς αυτών· και εσκληρύνθη η καρδία του Φαραώ και δεν εισήκουσεν εις αυτούς, καθώς είπεν ο Κύριος.
\par 23 Και επιστρέψας ο Φαραώ, ήλθεν εις τον οίκον αυτού, και δεν επέστησε την καρδίαν αυτού ουδέ εις τούτο.
\par 24 Πάντες δε οι Αιγύπτιοι έσκαπτον πέριξ του ποταμού διά να πίωσιν ύδωρ, διότι δεν ηδύναντο να πίωσιν εκ του ύδατος του ποταμού.
\par 25 Και συνεπληρώθησαν επτά ημέραι, αφού ο Κύριος εκτύπησε τον ποταμόν.

\chapter{8}

\par Και είπε Κύριος προς τον Μωϋσήν, Ύπαγε προς τον Φαραώ, και ειπέ προς αυτόν, ούτω λέγει Κύριος, Εξαπόστειλον τον λαόν μου διά να με λατρεύση·
\par 2 και αν δεν θέλης να εξαποστείλης αυτόν, ιδού, εγώ θέλω κτυπήσει πάντα τα όριά σου με βατράχους·
\par 3 και ο ποταμός θέλει εξεμέσει βατράχους, οίτινες αναβαίνοντες θέλουσιν εισέλθει εις τον οίκόν σου και εις τον κοιτώνά σου και επί της κλίνης σου και εις τας οικίας των θεραπόντων σου και επί τον λαόν σου και εις τους κλιβάνους σου και εις τας σκάφας σου·
\par 4 και επί σε και επί τον λαόν σου και επί πάντας τους θεράποντάς σου θέλουσιν αναβή οι βάτραχοι.
\par 5 Είπε δε Κύριος προς τον Μωϋσήν, Ειπέ προς τον Ααρών, Έκτεινον την χείρα σου με την ράβδον σου επί τους ρύακας, επί τους ποταμούς και επί τας λίμνας και ανάγαγε τους βατράχους επί την γην της Αιγύπτου.
\par 6 Και εξέτεινεν ο Ααρών την χείρα αυτού επί τα ύδατα της Αιγύπτου· και ανέβησαν οι βάτραχοι και εκάλυψαν την γην της Αιγύπτου.
\par 7 Και έκαμον ομοίως οι μάγοι με τας επωδάς αυτών και ανήγαγον τους βατράχους επί την γην της Αιγύπτου.
\par 8 Τότε εκάλεσεν ο Φαραώ τον Μωϋσήν και τον Ααρών και είπε, Δεήθητε του Κυρίου να σηκώση τους βατράχους απ' εμού και από του λαού μου· και θέλω εξαποστείλει τον λαόν διά να θυσιάσωσιν εις τον Κύριον.
\par 9 Και είπεν ο Μωϋσής προς τον Φαραώ, Διόρισον εις εμέ, πότε να δεηθώ υπέρ σου και υπέρ των θεραπόντων σου και υπέρ του λαού σου· διά να εξαλείψη τους βατράχους από σου, και από των οικιών σου, και μόνον εν τω ποταμώ να μείνωσιν.
\par 10 Ο δε είπεν, Αύριον. Και είπε, Θέλει γείνει κατά τον λόγον σου· διά να γνωρίσης ότι δεν είναι ουδείς ως ο Κύριος ο Θεός ημών·
\par 11 και θέλουσι σηκωθή οι βάτραχοι από σου και από των οικιών σου και από των θεραπόντων σου και από του λαού σου· μόνον εν τω ποταμώ θέλουσι μείνει.
\par 12 Τότε εξήλθον ο Μωϋσής και ο Ααρών από του Φαραώ· και εβόησεν ο Μωϋσής προς τον Κύριον περί των βατράχων, τους οποίους έφερεν επί τον Φαραώ.
\par 13 Και έκαμεν ο Κύριος κατά τον λόγον του Μωϋσέως· και ετελεύτησαν οι βάτραχοι εκ των οικιών, εκ των επαύλεων και εκ των αγρών.
\par 14 Και συνήγαγον αυτούς σωρούς, και εβρώμησεν η γη.
\par 15 Ιδών δε ο Φαραώ ότι έγεινεν αναψυχή, εσκλήρυνε την καρδίαν αυτού, και δεν εισήκουσεν εις αυτούς, καθώς ελάλησεν ο Κύριος.
\par 16 Και είπε Κύριος προς τον Μωϋσήν, Ειπέ προς τον Ααρών, Έκτεινον την ράβδον σου και κτύπησον το χώμα της γης, διά να γείνη σκνίπες καθ' όλην την γην της Αιγύπτου.
\par 17 Και έκαμον ούτω· διότι εξέτεινεν ο Ααρών την χείρα αυτού με την ράβδον αυτού, και εκτύπησε το χώμα της γης, και έγεινε σκνίπες εις τους ανθρώπους και εις τα κτήνη· όλον το χώμα της γης έγεινε σκνίπες καθ' όλην την γην της Αιγύπτου.
\par 18 Και έκαμον ομοίως οι μάγοι με τας επωδάς αυτών διά να εκβάλωσι σκνίπας· πλην δεν ηδυνήθησαν· οι σκνίπες λοιπόν ήσαν επί τους ανθρώπους και επί τα κτήνη.
\par 19 Τότε είπον οι μάγοι προς τον Φαραώ, Δάκτυλος Θεού είναι τούτο. Η καρδία όμως του Φαραώ εσκληρύνθη και δεν εισήκουσεν εις αυτούς, καθώς ελάλησεν ο Κύριος.
\par 20 Και είπε Κύριος προς τον Μωϋσήν, Σηκώθητι ενωρίς το πρωΐ και στάθητι ενώπιον του Φαραώ· ιδού, εξέρχεται εις το ύδωρ· και ειπέ προς αυτόν, Ούτω λέγει Κύριος· Εξαπόστειλον τον λαόν μου διά να με λατρεύση·
\par 21 διότι εάν δεν εξαποστείλης τον λαόν μου, ιδού, θέλω στείλει επί σε και επί τους θεράποντάς σου και επί τον λαόν σου και επί τας οικίας σου κυνόμυιαν, και αι οικίαι των Αιγυπτίων και η γη έτι επί της οποίας κατοικούσι θέλουσι γεμίσει από κυνόμυιαν·
\par 22 θέλω όμως εξαιρέσει εν εκείνη τη ημέρα την γην Γεσέν, εν ή κατοικεί ο λαός μου, ώστε να μη ήναι εκεί παντελώς κυνόμυια· διά να γνωρίσης ότι εγώ είμαι ο Κύριος εν τω μέσω της γής·
\par 23 και θέλω βάλει διαφοράν μεταξύ του λαού μου και του λαού σου· αύριον θέλει γείνει το σημείον τούτο.
\par 24 Και έκαμε Κύριος ούτω· και ήλθε κυνόμυια πλήθος εις την οικίαν του Φαραώ και εις τας οικίας των θεραπόντων αυτού και εις όλην την γην της Αιγύπτου· η γη διεφθάρη εκ του πλήθους της κυνομυίας.
\par 25 Και εκάλεσεν ο Φαραώ τον Μωϋσήν και τον Ααρών και είπεν, Υπάγετε, κάμετε θυσίαν εις τον Θεόν σας εν ταύτη τη γη.
\par 26 Είπε δε ο Μωϋσής, Δεν αρμόζει να γείνη ούτω· διότι ημείς θυσιάζομεν εις Κύριον τον Θεόν ημών θυσίας, τας οποίας οι Αιγύπτιοι βδελύττονται· ιδού, εάν ημείς θυσιάσωμεν θυσίας, τας οποίας οι Αιγύπτιοι βδελύττονται, έμπροσθεν των οφθαλμών αυτών, δεν θέλουσι μας λιθοβολήσει;
\par 27 θέλομεν υπάγει οδόν τριών ημερών εις την έρημον και θέλομεν θυσιάσει εις Κύριον τον Θεόν ημών, καθώς είπε προς ημάς.
\par 28 Τότε είπεν ο Φαραώ, Εγώ θέλω σας εξαποστείλει, διά να θυσιάσητε εις Κύριον τον Θεόν σας εν τη ερήμω· μόνον να μη υπάγητε πολύ μακράν· δεήθητε υπέρ εμού.
\par 29 Και είπεν ο Μωϋσής, Ιδού, εγώ εξέρχομαι από σου και θέλω δεηθή του Κυρίου, ώστε η κυνόμυια να σηκωθή αύριον από του Φαραώ, από των θεραπόντων αυτού και από του λαού αυτού· πλην ας μη εξακολουθή ο Φαραώ να απατά ημάς, μη εξαποστέλλων τον λαόν, διά να θυσιάση εις τον Κύριον.
\par 30 Τότε εξήλθεν ο Μωϋσής από του Φαραώ και εδεήθη του Κυρίου.
\par 31 Και έκαμε Κύριος κατά τον λόγον του Μωϋσέως· και εσήκωσε την κυνόμυιαν από του Φαραώ, από των θεραπόντων αυτού και από του λαού αυτού· δεν έμεινεν ουδέ μία.
\par 32 Πλην ο Φαραώ και ταύτην την φοράν εσκλήρυνε την καρδίαν αυτού και δεν εξαπέστειλε τον λαόν.

\chapter{9}

\par Και είπε Κύριος προς τον Μωϋσήν, Ύπαγε προς τον Φαραώ και ειπέ προς αυτόν, Ούτω λέγει Κύριος ο Θεός των Εβραίων. Εξαπόστειλον τον λαόν μου, διά να με λατρεύση·
\par 2 διότι, εάν δεν θέλης να εξαποστείλης και εάν έτι κρατής αυτούς,
\par 3 ιδού, η χειρ του Κυρίου θέλει είσθαι επί τα κτήνη σου τα εν τω αγρώ, επί τους ίππους, επί τους όνους, επί τας καμήλους, επί τους βόας, και επί τα πρόβατα· θανατικόν βαρύ σφόδρα·
\par 4 και θέλει κάμει ο Κύριος διάκρισιν μεταξύ των κτηνών του Ισραήλ και των κτηνών των Αιγυπτίων· και εκ πάντων των ανηκόντων εις τους υιούς Ισραήλ δεν θέλει αποθάνει ουδέ εν.
\par 5 Και διώρισεν ο Κύριος καιρόν, λέγων, Αύριον θέλει κάμει ο Κύριος το πράγμα τούτο εν τη γη.
\par 6 Και έκαμεν ο Κύριος το πράγμα τούτο την επαύριον, και απέθανον πάντα τα κτήνη των Αιγυπτίων· εκ δε των κτηνών των υιών Ισραήλ δεν απέθανεν ουδέ εν.
\par 7 Και απέστειλεν ο Φαραώ να ίδωσι, και ιδού, εκ των κτηνών του Ισραήλ δεν απέθανεν ουδέ έν· και εσκληρύνθη η καρδία του Φαραώ και δεν εξαπέστειλε τον λαόν.
\par 8 Τότε είπεν ο Κύριος προς τον Μωϋσήν και προς τον Ααρών, Γεμίσατε τας χείρας σας από στάκτην καμίνου και ας σκορπίση αυτήν ο Μωϋσής προς τον ουρανόν έμπροσθεν του Φαραώ·
\par 9 και θέλει γείνει λεπτός κονιορτός εφ' όλην την γην της Αιγύπτου· και θέλει γείνει επί τους ανθρώπους και επί τα κτήνη καύσις αναδιδούσα ελκώδη εξανθήματα καθ' όλην την γην της Αιγύπτου.
\par 10 Έλαβον λοιπόν την στάκτην της καμίνου και εστάθησαν ενώπιον του Φαραώ· και εσκόρπισεν αυτήν ο Μωϋσής προς τον ουρανόν, και έγεινε καύσις αναδιδούσα ελκώδη εξανθήματα επί τους ανθρώπους και επί τα κτήνη·
\par 11 και δεν ηδύναντο οι μάγοι να σταθώσιν έμπροσθεν του Μωϋσέως εξ αιτίας της καύσεως· διότι η καύσις ήτο επί τους μάγους και επί πάντας τους Αιγυπτίους.
\par 12 Εσκλήρυνε δε Κύριος την καρδίαν του Φαραώ, και δεν εισήκουσεν εις αυτούς, καθώς ελάλησε Κύριος προς τον Μωϋσήν.
\par 13 Και είπε Κύριος προς τον Μωϋσήν, Σηκώθητι ενωρίς το πρωΐ και παραστάθητι έμπροσθεν του Φαραώ και ειπέ προς αυτόν, ούτω λέγει Κύριος ο Θεός των Εβραίων· Εξαπόστειλον τον λαόν μου, διά να με λατρεύση·
\par 14 διότι ταύτην την φοράν εγώ εξαποστέλλω πάσας μου τας πληγάς επί την καρδίαν σου και επί τους θεράποντάς σου και επί τον λαόν σου· διά να γνωρίσης ότι δεν είναι ουδείς όμοιός μου εν πάση τη γή·
\par 15 επειδή τώρα θέλω εκτείνει την χείρα μου και θέλω πατάξει σε και τον λαόν σου με θανατικόν, και θέλεις απολεσθή από της γής·
\par 16 και διά τούτο βεβαίως σε διετήρησα, διά να δείξω εν σοι την δύναμίν μου και να κηρυχθή το όνομά μου εν πάση τη γή·
\par 17 ότι επεγείρεσαι κατά του λαού μου, διά να μη εξαποστείλης αυτόν;
\par 18 ιδού, αύριον περί την ώραν ταύτην θέλω βρέξει χάλαζαν βαρείαν σφόδρα, οποία δεν έγεινε ποτέ εν τη Αιγύπτω αφ' ης ημέρας εθεμελιώθη μέχρι του νύν·
\par 19 τώρα λοιπόν απόστειλον να συνάξης τα κτήνη σου και πάντα όσα έχεις εν τοις αγροίς· διότι πας άνθρωπος και ζώον, το οποίον ευρεθή εν τοις αγροίς και δεν φερθή εις οικίαν, και η χάλαζα καταβή επ' αυτά, θέλουσιν αποθάνει.
\par 20 Όστις εκ των θεραπόντων του Φαραώ εφοβήθη τον λόγον του Κυρίου, συνήγαγε ταχέως εις τας οικίας τους δούλους αυτού και τα κτήνη αυτού·
\par 21 όστις όμως δεν επρόσεξεν εις τον λόγον του Κυρίου, αφήκε τους δούλους αυτού και τα κτήνη αυτού εν τοις αγροίς.
\par 22 Και είπε Κύριος προς τον Μωϋσήν, Έκτεινον την χείρα σου προς τον ουρανόν, και θέλει γείνει χάλαζα εφ' όλην την γην της Αιγύπτου, επί ανθρώπους και επί κτήνη και επί πάντα χόρτον του αγρού εν τη γη της Αιγύπτου.
\par 23 Και εξέτεινεν ο Μωϋσής την ράβδον αυτού προς τον ουρανόν, και ο Κύριος έπεμψε βροντάς και χάλαζαν και διέτρεχε το πυρ επί την γήν· και ο Κύριος έβρεξε χάλαζαν επί την γην της Αιγύπτου·
\par 24 ώστε ήτο χάλαζα και πυρ φλογίζον εν τη χαλάζη, χάλαζα βαρεία, οποία δεν έγεινε ποτέ εφ' όλην την γην της Αιγύπτου, αφού κατεστάθη έθνος.
\par 25 Και επάταξεν η χάλαζα εν πάση τη γη της ιγύπτου παν το εν τοις αγροίς, από ανθρώπου έως κτήνους· και πάντα τον χόρτον του αγρού επάταξεν η χάλαζα και πάντα τα δένδρα του αγρού συνέτριψε.
\par 26 Μόνον εν τη γη Γεσέν, όπου ήσαν οι υιοί Ισραήλ, δεν έγεινε χάλαζα.
\par 27 Τότε ο Φαραώ αποστείλας εκάλεσε τον Μωϋσήν και τον Ααρών και είπε προς αυτούς, Ταύτην την φοράν ημάρτησα· ο Κύριος είναι δίκαιος· εγώ δε και ο λαός μου είμεθα ασεβείς·
\par 28 δεήθητε του Κυρίου, ώστε να παύσωσι του να γίνωνται βρονταί Θεού και χάλαζα· και εγώ θέλω σας εξαποστείλει, και δεν θέλετε μείνει πλέον.
\par 29 Και είπεν ο Μωϋσής προς αυτόν, καθώς εξέλθω εκ της πόλεως, θέλω εκτείνει τας χείρας μου προς τον Κύριον· αι βρονταί θέλουσι παύσει και χάλαζα δεν θέλει είσθαι πλέον· διά να γνωρίσης ότι του Κυρίου είναι η γή·
\par 30 πλην συ και οι θεράποντές σου, εξεύρω ότι ακόμη δεν θέλετε φοβηθή από προσώπου Κυρίου του Θεού.
\par 31 Εκτυπήθησαν δε το λινάριον και η κριθή· διότι η κριθή ήτο σταχυωμένη και το λινάριον καλαμωμένον·
\par 32 ο σίτος όμως και η ζέα δεν εκτυπήθησαν, διότι ήσαν όψιμα.
\par 33 Και εξήλθεν ο Μωϋσής έξω της πόλεως από του Φαραώ και εξέτεινε τας χείρας αυτού προς τον Κύριον· και αι βρονταί και η χάλαζα έπαυσαν και βροχή δεν έσταξε πλέον επί της γης.
\par 34 Και ότε είδεν ο Φαραώ ότι έπαυσεν η βροχή και η χάλαζα και αι βρονταί, εξηκολούθησε να αμαρτάνη και εσκλήρυνε την καρδίαν αυτού, αυτός και οι θεράποντες αυτού.
\par 35 Και εσκληρύνθη η καρδία του Φαραώ και δεν εξαπέστειλε τους υιούς Ισραήλ, καθώς ελάλησε Κύριος διά του Μωϋσέως.

\chapter{10}

\par Και είπε Κύριος προς τον Μωϋσήν, Είσελθε προς τον Φαραώ· διότι εγώ εσκλήρυνα την καρδίαν αυτού και την καρδίαν των θεραπόντων αυτού, διά να δείξω τα σημείά μου ταύτα εν μέσω αυτών·
\par 2 και διά να διηγήσαι εις τα ώτα του υιού σου και εις τον υιόν του υιού σου, τα όσα έπραξα εις τους Αιγυπτίους και τα σημείά μου όσα έκαμα εν μέσω αυτών, και να γνωρίσητε ότι εγώ είμαι ο Κύριος.
\par 3 Εισήλθον δε ο Μωϋσής και ο Ααρών προς τον Φαραώ και είπον προς αυτόν, Ούτω λέγει Κύριος ο Θεός των Εβραίων· Έως πότε αρνείσαι να ταπεινωθής έμπροσθέν μου; εξαπόστειλον τον λαόν μου διά να με λατρεύση·
\par 4 διότι εάν δεν θέλης να εξαποστείλης τον λαόν μου, ιδού, αύριον θέλω φέρει ακρίδα επί τα όριά σου·
\par 5 και θέλει σκεπάσει το πρόσωπον της γης, ώστε να μη δύναταί τις να ίδη την γήν· και θέλει καταφάγει το επίλοιπον το διασωθέν, όσον αφήκεν εις εσάς η χάλαζα, και θέλει καταφάγει πάντα τα δένδρα τα φυόμενα εις εσάς εκ των αγρών·
\par 6 και θέλουσι γεμισθή αι οικίαι σου και αι οικίαι πάντων των θεραπόντων σου και αι οικίαι πάντων των Αιγυπτίων· το οποίον δεν είδον οι πατέρες σου ούτε οι πατέρες των πατέρων σου, αφ' ης ημέρας υπήρξαν επί της γης μέχρι της σήμερον. Έπειτα στραφείς εξήλθεν από του Φαραώ.
\par 7 Και είπον οι θεράποντες του Φαραώ προς αυτόν, Έως πότε ούτος θέλει είσθαι πρόσκομμα εις ημάς; εξαπόστειλον τους ανθρώπους, διά να λατρεύσωσι Κύριον τον Θεόν αυτών· ακόμη δεν εξεύρεις ότι ηφανίσθη η Αίγυπτος;
\par 8 Τότε έφεραν πάλιν τον Μωϋσήν και τον Ααρών προς τον Φαραώ· και είπε προς αυτούς, Υπάγετε, λατρεύσατε τον Κύριον τον Θεόν σας· αλλά ποίοι και ποίοι θέλουσιν υπάγει;
\par 9 Και είπεν ο Μωϋσής· μετά των νέων ημών και μετά των γερόντων ημών θέλομεν υπάγει, μετά των υιών ημών και μετά των θυγατέρων ημών, μετά των προβάτων ημών και μετά των βοών ημών θέλομεν υπάγει διότι έχομεν εορτήν εις τον Κύριον.
\par 10 Ο δε είπε προς αυτούς, Ούτως ας ήναι ο Κύριος μεθ' υμών, καθώς εγώ θέλω σας εξαποστείλει μετά των τέκνων σας· ίδετε· διότι κακόν πρόκειται έμπροσθέν σας·
\par 11 ουχί ούτως, οι άνδρες υπάγετε τώρα, και λατρεύσατε τον Κύριον, διότι τούτο ζητείτε. Και εξέβαλεν αυτούς ο Φαραώ απ' έμπροσθεν αυτού.
\par 12 Είπε δε Κύριος προς τον Μωϋσήν, Έκτεινον την χείρα σου επί την γην της Αιγύπτου διά την ακρίδα, διά να αναβή επί την γην της Αιγύπτου και να καταφάγη πάντα τον χόρτον της γης, παν ό,τι η χάλαζα αφήκε.
\par 13 Και εξέτεινεν ο Μωϋσής την ράβδον αυτού επί την γην της Αίγυπτον, και ο Κύριος επέφερεν επί την γην όλην την ημέραν εκείνην και όλην την νύκτα ανατολικόν άνεμον· και το πρωΐ ο άνεμος ο ανατολικός έφερε την ακρίδα.
\par 14 Και ανέβη η ακρίς εφ' όλην την γην της Αιγύπτου και εκάθισεν επί πάντα τα όρια της Αιγύπτου, πολλή σφόδρα· πρότερον αυτής δεν υπήρξε τοιαύτη ακρίς, ουδέ θέλει υπάρξει τοιαύτη μετ' αυτήν·
\par 15 και εκάλυψε το πρόσωπον όλης της γης και εσκοτίσθη η γή· και κατέφαγε πάντα τον χόρτον της γης και πάντας τους καρπούς των δένδρων, όσους η χάλαζα αφήκε, και δεν έμεινεν ουδέν χλωρόν ούτε εις τα δένδρα ούτε εις τα χόρτα του αγρού καθ' όλην την γην της Αιγύπτου.
\par 16 Τότε έσπευσεν ο Φαραώ να καλέση τον Μωϋσήν και τον Ααρών και είπεν, Ημάρτησα εις Κύριον τον Θεόν σας και εις εσάς·
\par 17 πλην τώρα συγχωρήσατέ μοι, παρακαλώ, το αμάρτημά μου, μόνον ταύτην την φοράν, και δεήθητε Κυρίου του Θεού υμών διά να σηκώση απ' εμού τον θάνατον τούτον μόνον.
\par 18 Και εξήλθεν ο Μωϋσής από του Φαραώ και εδεήθη του Κυρίου.
\par 19 Και μετέφερεν ο Κύριος σφοδρότατον δυτικόν άνεμον, όστις εσήκωσε την ακρίδα και έρριψεν αυτήν εις την Ερυθράν θάλασσαν· δεν έμεινεν ουδεμία ακρίς επί πάντα τα όρια της Αιγύπτου.
\par 20 Πλην ο Κύριος εσκλήρυνε την καρδίαν του Φαραώ, και δεν εξαπέστειλε τους υιούς Ισραήλ.
\par 21 Και είπε Κύριος προς τον Μωϋσήν, Έκτεινον την χείρα σου προς τον ουρανόν και θέλει γείνει σκότος επί την γην της Αιγύπτου και σκότος ψηλαφητόν.
\par 22 Και εξέτεινεν ο Μωϋσής την χείρα αυτού προς τον ουρανόν, και έγεινε σκότος πυκνόν εφ' όλην την γην της Αιγύπτου τρεις ημέρας.
\par 23 Δεν έβλεπεν ο εις τον άλλον· ουδέ εσηκώθη τις από του τόπου αυτού τρεις ημέρας· εις πάντας δε τους υιούς Ισραήλ ήτο φως εν ταις κατοικίαις αυτών.
\par 24 Τότε εκάλεσεν ο Φαραώ τον Μωϋσήν και είπεν, Υπάγετε, λατρεύσατε τον Κύριον· μόνον τα πρόβατά σας και οι βόες σας ας μείνωσι και τα τέκνα σας ας έλθωσι μεθ' υμών.
\par 25 Και είπεν ο Μωϋσής, Αλλά και θυσίας και ολοκαυτώματα πρέπει συ να μας δώσης, διά να θυσιάσωμεν εις Κύριον τον Θεόν ημών·
\par 26 τα κτήνη ημών ομοίως θέλουσιν υπάγει μεθ' ημών· δεν θέλει μείνει οπίσω ουδέ ονύχιον· διότι εκ τούτων πρέπει να λάβωμεν, διά να λατρεύσωμεν Κύριον τον Θεόν ημών· και ημείς δεν εξεύρομεν με τι έχομεν να λατρεύσωμεν τον Κύριον, εωσού να φθάσωμεν εκεί.
\par 27 Αλλ' ο Κύριος εσκλήρυνε την καρδίαν του Φαραώ, και δεν ηθέλησε να εξαποστείλη αυτούς.
\par 28 Και είπεν ο Φαραώ προς αυτόν, Φύγε απ' εμού· πρόσεχε εις σεαυτόν, να μη ίδης πλέον το πρόσωπόν μου· διότι εις οποίαν ημέραν ίδης το πρόσωπόν μου, θέλεις αποθάνει.
\par 29 Και είπεν ο Μωϋσής, Καθώς είπας, δεν θέλω ιδεί πλέον το πρόσωπόν σου.

\chapter{11}

\par Είπε δε Κύριος προς τον Μωϋσήν, Έτι μίαν πληγήν θέλω φέρει επί τον Φαραώ και επί την Αίγυπτον· μετά ταύτα θέλει σας εξαποστείλει εντεύθεν· εξαποστέλλων υμάς θέλει βεβαίως και διώξει υμάς ολοκλήρως εντεύθεν·
\par 2 λάλησον τώρα εις τα ώτα του λαού, και ας ζητήση πας ανήρ παρά του γείτονος αυτού, και πάσα γυνή παρά της γείτονος αυτής, σκεύη αργυρά, και σκεύη χρυσά.
\par 3 Και έδωκεν ο Κύριος χάριν εις τον λαόν ενώπιον των Αιγυπτίων· έτι δε ο άνθρωπος ο Μωϋσής ήτο μέγας σφόδρα εν τη γη της Αιγύπτου έμπροσθεν των θεραπόντων του Φαραώ και έμπροσθεν του λαού.
\par 4 Και είπεν ο Μωϋσής, Ούτω λέγει ο Κύριος· Περί το μεσονύκτιον εγώ θέλω εξέλθει εις το μέσον της Αιγύπτου·
\par 5 και παν πρωτότοκον εν τη γη της Αιγύπτου θέλει αποθάνει, από του πρωτοτόκου του Φαραώ, όστις κάθηται επί του θρόνου αυτού, έως του πρωτοτόκου της δούλης, ήτις δουλεύει εν τω μύλω, και παν πρωτότοκον των κτηνών·
\par 6 και θέλει είσθαι καθ' όλην την γην της Αιγύπτου κραυγή μεγάλη, οποία ποτέ δεν έγεινεν, ουδέ μετά ταύτα θέλει γείνει τοιαύτη·
\par 7 επί πάντας όμως τους υιούς Ισραήλ δεν θέλει κινήσει σκύλος την γλώσσαν αυτού, από ανθρώπου έως κτήνους· διά να γνωρίσητε ότι ο Κύριος έκαμε διάκρισιν μεταξύ των Αιγυπτίων και του Ισραήλ·
\par 8 και πάντες ούτοι οι δούλοί σου θέλουσι καταβή προς εμέ και θέλουσι προσπέσει έμπροσθέν μου λέγοντες, Έξελθε συ και πας ο λαός ο ακολουθών σε· και μετά ταύτα θέλω εξέλθει. Και εξήλθεν ο Μωϋσής από του Φαραώ μετά θυμού μεγάλου.
\par 9 Και είπε Κύριος προς τον Μωϋσήν, Δεν θέλει σας εισακούσει ο Φαραώ, διά να πληθυνθώσι τα θαυμάσιά μου εν τη γη της Αιγύπτου.
\par 10 Ο Μωϋσής δε και ο Ααρών έκαμον πάντα τα θαυμάσια ταύτα ενώπιον του Φαραώ· ο δε Κύριος εσκλήρυνε την καρδίαν του Φαραώ, και δεν εξαπέστειλε τους υιούς Ισραήλ εκ της γης αυτού.

\chapter{12}

\par Και είπε Κύριος προς τον Μωϋσήν και προς τον Ααρών εν τη γη της Αιγύπτου, λέγων,
\par 2 Ο μην ούτος θέλει είσθαι εις εσάς αρχή μηνών· θέλει είσθαι εις εσάς πρώτος των μηνών του ενιαυτού.
\par 3 Λαλήσατε προς πάσαν την συναγωγήν του Ισραήλ, λέγοντες, Την δεκάτην τούτου του μηνός ας λάβωσιν εις εαυτούς έκαστος εν αρνίον κατά τους οίκους των πατριών αυτών, εν αρνίον δι' έκαστον οίκον.
\par 4 Εάν όμως ήναι οι εν τω οίκω ολιγοστοί διά το αρνίον, αυτός και ο γείτων αυτού ο πλησιέστερος της οικίας αυτού ας λάβωσιν αυτό κατά τον αριθμόν των ψυχών· έκαστος θέλει συναριθμείσθαι διά το αρνίον αναλόγως με το αρκετόν εις αυτόν να φάγη.
\par 5 Το δε αρνίον σας θέλει είσθαι τέλειον, αρσενικόν ενιαύσιον· εκ των προβάτων ή εκ των αιγών θέλετε λάβει αυτό.
\par 6 Και θέλετε φυλάττει αυτό μέχρι της δεκάτης τετάρτης του αυτού μηνός· και τότε άπαν το πλήθος της συναγωγής του Ισραήλ θέλει σφάξει αυτό προς το εσπέρας.
\par 7 Και θέλουσι λάβει εκ του αίματος και βάλει επί τους δύο παραστάτας και επί το ανώφλιον της θύρας των οικιών, όπου θέλουσι φάγει αυτό.
\par 8 Και θέλουσι φάγει το κρέας την νύκτα εκείνην, οπτόν εν πυρί· με άζυμα, και με χόρτα πικρά θέλουσι φάγει αυτό·
\par 9 μη φάγητε απ' αυτού ωμόν, μηδέ βραστόν εν ύδατι, αλλά οπτόν εν πυρί· την κεφαλήν αυτού μετά των ποδών αυτού και μετά των εντοσθίων αυτού·
\par 10 και μη αφήσητε υπόλοιπον απ' αυτού έως το πρωΐ· ό,τι δε περισσεύση απ' αυτού έως το πρωΐ, καύσατε εν πυρί.
\par 11 Και ούτω θέλετε φάγει αυτό· Εζωσμένοι τας οσφύας σας, έχοντες τα υποδήματά σας εις τους πόδας σας και την ράβδον σας εις την χείρα σας· και θέλετε φάγει αυτό μετά σπουδής· είναι πάσχα του Κυρίου.
\par 12 Διότι την νύκτα ταύτην θέλω περάσει διά μέσου της γης της Αιγύπτου και θέλω πατάξει παν πρωτότοκον εν τη γη της Αιγύπτου, από ανθρώπου έως κτήνους· και θέλω κάμει κρίσεις εναντίον πάντων των θεών της Αιγύπτου. Εγώ ο Κύριος.
\par 13 Και το αίμα θέλει είσθαι εις εσάς διά σημείον επί των οικιών, εις τας οποίας κατοικείτε· και όταν ίδω το αίμα, θέλω σας παρατρέξει, και η πληγή δεν θέλει είσθαι εις εσάς διά να σας εξολοθρεύση, όταν πατάξω την γην της Αιγύπτου.
\par 14 Και η ημέρα αύτη θέλει είσθαι εις εσάς εις μνημόσυνον· και θέλετε εορτάζει αυτήν εορτήν εις τον Κύριον εις τας γενεάς σας· κατά νόμον παντοτεινόν θέλετε εορτάζει αυτήν.
\par 15 Επτά ημέρας θέλετε τρώγει άζυμα· από της πρώτης ημέρας θέλετε σηκώσει το προζύμιον εκ των οικιών σας· διότι όστις φάγη ένζυμα από της πρώτης έως της εβδόμης ημέρας, η ψυχή εκείνη θέλει εξολοθρευθή εκ του Ισραήλ.
\par 16 Και εν τη πρώτη ημέρα θέλει είσθαι σύναξις αγία· και εν τη εβδόμη ημέρα σύναξις αγία θέλει είσθαι εις εσάς· ουδεμία εργασία θέλει γίνεσθαι εν αυταίς, εκτός ό,τι χρειάζεται εις έκαστον άνθρωπον διά να φάγη· τούτο μόνον θέλετε κάμει.
\par 17 Θέλετε φυλάξει λοιπόν την εορτήν των αζύμων· διότι την αυτήν ταύτην ημέραν θέλω εξαγάγει τα τάγματά σας εκ της γης της Αιγύπτου· όθεν κατά νόμον παντοτεινόν θέλετε φυλάττει την ημέραν ταύτην εις τας γενεάς σας·
\par 18 αρχόμενοι από της δεκάτης τετάρτης ημέρας του μηνός αφ' εσπέρας, θέλετε τρώγει άζυμα έως της εικοστής πρώτης ημέρας του μηνός την εσπέραν·
\par 19 επτά ημέρας δεν θέλει ευρίσκεσθαι προζύμιον εν ταις οικίαις υμών· διότι όστις φάγη ένζυμα, η ψυχή εκείνη θέλει εξολοθρευθή εκ της συναγωγής του Ισραήλ, είτε ξένος είναι είτε αυτόχθων·
\par 20 ουδέν ένζυμον θέλετε φάγει· εν πάσαις ταις κατοικίαις υμών άζυμα θέλετε τρώγει.
\par 21 Τότε εκάλεσεν ο Μωϋσής πάντας τους πρεσβυτέρους του Ισραήλ και είπε προς αυτούς, Εκλέξατε και λάβετε εις εαυτούς εν αρνίον, κατά τας οικογενείας σας, και θύσατε το πάσχα·
\par 22 έπειτα θέλετε λάβει δέσμην υσσώπου και θέλετε εμβάψει αυτήν εις το αίμα, το οποίον θέλει είσθαι εις λεκάνην· και από του αίματος του εν τη λεκάνη θέλετε κτυπήσει το ανώφλιον και τους δύο παραστάτας των θυρών· και ουδείς από σας θέλει εξέλθει εκ της θύρας της οικίας αυτού έως το πρωΐ·
\par 23 διότι ο Κύριος θέλει περάσει διά να πατάξη τους Αιγυπτίους· και όταν ίδη το αίμα επί το ανώφλιον και επί τους δύο παραστάτας, ο Κύριος θέλει παρατρέξει την θύραν, και δεν θέλει αφήσει τον εξολοθρευτήν να εισέλθη εις τας οικίας σας, διά να πατάξη.
\par 24 Και θέλετε φυλάξει το πράγμα τούτο ως νόμον, εις σεαυτόν και εις τους υιούς σου, έως αιώνος.
\par 25 Και όταν εισέλθητε εις την γην, την οποίαν ο Κύριος θέλει σας δώσει καθώς ελάλησε, θέλετε φυλάξει την λατρείαν ταύτην.
\par 26 Και όταν σας λέγωσιν οι υιοί σας, Τι σημαίνει εις εσάς η λατρεία αύτη;
\par 27 θέλετε αποκρίνεσθαι, Τούτο είναι θυσία του πάσχα εις τον Κύριον, διότι παρέτρεξε τας οικίας των υιών Ισραήλ εν Αιγύπτω, ότε επάταξε τους Αιγυπτίους και έσωσε τας οικίας ημών. Τότε ο λαός κύψας προσεκύνησε.
\par 28 Και αναχωρήσαντες οι υιοί Ισραήλ, έκαμον καθώς προσέταξεν ο Κύριος εις τον Μωϋσήν και τον Ααρών· ούτως έκαμον.
\par 29 Κατά δε το μεσονύκτιον ο Κύριος επάταξε παν πρωτότοκον εν τη γη της Αιγύπτου· από του πρωτοτόκου του Φαραώ όστις κάθηται επί του θρόνου αυτού, έως του πρωτοτόκου του αιχμαλώτου του εν τω δεσμωτηρίω· και πάντα τα πρωτότοκα των κτηνών.
\par 30 Και εσηκώθη ο Φαραώ την νύκτα, αυτός και πάντες οι θεράποντες αυτού και πάντες οι Αιγύπτιοι· και έγεινε βοή μεγάλη εν τη Αιγύπτω· διότι δεν ήτο οικία εις την οποίαν δεν υπήρχε νεκρός.
\par 31 Και εκάλεσε τον Μωϋσήν και τον Ααρών διά νυκτός και είπε, Σηκώθητε, εξέλθετε εκ μέσου του λαού μου και σεις και οι υιοί του Ισραήλ· και υπάγετε, λατρεύσατε τον Κύριον, καθώς είπετε·
\par 32 και τα ποίμνιά σας και τας αγέλας σας λάβετε, καθώς είπετε, και απέλθετε· ευλογήσατε δε και εμέ.
\par 33 Και εβίαζον οι Αιγύπτιοι τον λαόν διά να εκβάλωσιν αυτόν ταχέως εκ του τόπου· διότι είπον, Ημείς πάντες αποθνήσκομεν.
\par 34 Και εσήκωσεν ο λαός την ζύμην αυτού πριν αναβή, έχων έκαστος την σκάφην αυτού επί τους ώμους αυτού, εντετυλιγμένην εις τα φορέματα αυτού.
\par 35 Και έκαμον οι υιοί του Ισραήλ κατά τον λόγον του Μωϋσέως και εζήτησαν παρά των Αιγυπτίων σκεύη αργυρά και σκεύη χρυσά, και ενδύματα·
\par 36 και ο Κύριος έδωκεν εις τον λαόν χάριν ενώπιον των Αιγυπτίων, και εδάνεισαν εις αυτούς όσα εζήτησαν· και εγύμνωσαν τους Αιγυπτίους.
\par 37 Ανεχώρησαν δε οι υιοί Ισραήλ από Ραμεσσή εις Σοκχώθ, περίπου εξακόσιαι χιλιάδες άνδρες πεζοί χωρίς των παιδίων.
\par 38 Μετ' αυτών συνανέβη και μέγα πλήθος σύμμικτον ανθρώπων και ποίμνια και αγέλαι, κτήνη πολλά σφόδρα.
\par 39 Και εκ της ζύμης, την οποίαν έφεραν εξ Αιγύπτου, έψησαν εγκρυφίας αζύμους· διότι δεν ήτο προζύμιον, επειδή εδιώχθησαν εξ Αιγύπτου και δεν εδυνήθησαν να βραδύνωσιν, ουδέ εφόδιον προητοίμασαν εις εαυτούς.
\par 40 Ο καιρός δε της παροικίας των υιών Ισραήλ, την οποίαν παρώκησαν εν Αιγύπτω, ήτο τετρακόσια και τριάκοντα έτη.
\par 41 Και μετά τα τετρακόσια και τριάκοντα έτη, την αυτήν εκείνην ημέραν εξήλθον πάντα τα τάγματα του Κυρίου εκ γης Αιγύπτου.
\par 42 Αύτη είναι νυξ, ήτις πρέπει να φυλάττηται εις τον Κύριον, διότι εξήγαγεν αυτούς εκ γης Αιγύπτου· αύτη είναι η νυξ εκείνη του Κυρίου, ήτις πρέπει να φυλάττηται παρά πάντων των υιών Ισραήλ εις τας γενεάς αυτών.
\par 43 Είπε δε Κύριος προς τον Μωϋσήν και Ααρών, Ούτος είναι ο νόμος του πάσχα· ουδείς αλλογενής θέλει φάγει απ' αυτού·
\par 44 και έκαστος δούλος αργυρώνητος αφού περιτμηθή, τότε θέλει φάγει απ' αυτού·
\par 45 ο ξένος δε και ο μισθωτός δεν θέλουσι φάγει απ' αυτού.
\par 46 Εν τη αυτή οικία θέλει φαγωθή· από του κρέατος δεν θέλετε φέρει έξω της οικίας, και οστούν δεν θέλετε συντρίψει απ' αυτού.
\par 47 Πάσα η συναγωγή του Ισραήλ θέλει κάμει τούτο.
\par 48 Και εάν τις ξένος, παροικών μετά σου, θέλη να κάμη το πάσχα εις τον Κύριον, ας περιτμηθώσι πάντα τα αρσενικά αυτού, και τότε ας πλησιάση διά να κάμη αυτό· και θέλει είσθαι ως ο αυτόχθων της γής· διότι ουδείς απερίτμητος θέλει φάγει απ' αυτού.
\par 49 Ο αυτός νόμος θέλει είσθαι διά τον αυτόχθονα και διά τον ξένον τον παροικούντα μεταξύ σας.
\par 50 Και έκαμον πάντες οι υιοί του Ισραήλ καθώς προσέταξεν ο Κύριος εις τον Μωϋσήν και τον Ααρών· ούτως έκαμον.
\par 51 Και την αυτήν εκείνην ημέραν εξήγαγεν ο Κύριος τους υιούς Ισραήλ εκ γης Αιγύπτου κατά τα τάγματα αυτών.

\chapter{13}

\par Και ελάλησε Κύριος προς τον Μωϋσήν, λέγων,
\par 2 Καθιέρωσον εις εμέ παν πρωτότοκον διανοίγον πάσαν μήτραν μεταξύ των υιών Ισραήλ, από ανθρώπου έως κτήνους· ιδικόν μου είναι τούτο.
\par 3 Και είπεν ο Μωϋσής προς τον λαόν, Έχετε εις την μνήμην σας την ημέραν ταύτην, καθ' ην εξήλθετε εξ Αιγύπτου εξ οίκου δουλείας· διότι ο Κύριος διά χειρός κραταιάς εξήγαγεν υμάς εκείθεν· ουδείς θέλει φάγει ένζυμα.
\par 4 Σήμερον εξέρχεσθε κατά τον μήνα Αβίβ.
\par 5 Όταν λοιπόν ο Κύριος σε φέρη εις την γην των Χαναναίων και των Χετταίων και των Αμορραίων και των Ευαίων και των Ιεβουσαίων, την οποίαν ώμοσε προς τους πατέρας σου ότι θέλει σοι δώσει, γην ρέουσαν γάλα και μέλι, τότε θέλεις κάμει την λατρείαν ταύτην κατά τούτον τον μήνα.
\par 6 Επτά ημέρας θέλεις τρώγει άζυμα· εις δε την εβδόμην ημέραν θέλει είσθαι εορτή εις τον Κύριον.
\par 7 Άζυμα θέλουσι τρώγεσθαι τας επτά ημέρας· και δεν θέλει φανή παρά σοι ένζυμον ουδέ θέλει φανή παρά σοι προζύμιον καθ' όλα τα όριά σου.
\par 8 Και κατ' εκείνην την ημέραν θέλεις αναγγείλει προς τον υιόν σου, λέγων, Τούτο γίνεται δι' εκείνο, το οποίον ο Κύριος έκαμεν εις εμέ, ότε εξήλθον εξ Αιγύπτου.
\par 9 Και τούτο θέλει είσθαι εις σε διά σημείον επί της χειρός σου και διά ενθύμησιν μεταξύ των οφθαλμών σου, διά να ήναι ο νόμος του Κυρίου εν τω στόματί σου· διότι διά χειρός κραταιάς σε εξήγαγεν ο Κύριος εξ Αιγύπτου.
\par 10 Θέλεις φυλάττει λοιπόν τον νόμον τούτον εν τω καιρώ αυτού κατ' έτος.
\par 11 Και όταν ο Κύριος σε φέρη εις την γην των Χαναναίων, καθώς ώμοσε προς σε και προς τους πατέρας σου, και δώση αυτήν εις σε,
\par 12 τότε θέλεις αποχωρίσει διά τον Κύριον παν το ανοίγον μήτραν και παν πρωτότοκον των ζώων σου όσα έχεις· τα αρσενικά θέλουσιν είσθαι του Κυρίου.
\par 13 Και παν πρωτότοκον όνου θέλεις εξαγοράζει με αρνίον· και αν δεν εξαγοράσης αυτό, τότε θέλεις λαιμοτομήσει αυτό· και παν πρωτότοκον ανθρώπου μεταξύ των υιών σου θέλεις εξαγοράζει.
\par 14 Και όταν εις το μέλλον σε ερωτήση ο υιός σου, λέγων, Τι είναι τούτο; θέλεις ειπεί προς αυτόν, Διά κραταιάς χειρός εξήγαγεν ημάς ο Κύριος εξ Αιγύπτου, εξ οίκου δουλείας·
\par 15 και ότε ο Φαραώ επέμεινεν εις το να μη μας εξαποστείλη, ο Κύριος εθανάτωσε παν πρωτότοκον εν τη γη της Αιγύπτου, από πρωτοτόκου ανθρώπου έως πρωτοτόκου κτήνους· διά τούτο θυσιάζω εις τον Κύριον παν αρσενικόν το οποίον ανοίγει την μήτραν, και παν πρωτότοκον των υιών μου εξαγοράζω.
\par 16 Και τούτο θέλει είσθαι διά σημείον επί της χειρός σου και διά προμετωπίδιον μεταξύ των οφθαλμών σου· επειδή διά κραταιάς χειρός εξήγαγεν ημάς ο Κύριος εξ Αιγύπτου.
\par 17 Ότε δε ο Φαραώ εξαπέστειλε τον λαόν, ο Θεός δεν ώδήγησεν αυτούς διά της οδού της γης των Φιλισταίων, αν και ήτο η συντομωτέρα· διότι ο Θεός είπε, Μήποτε ο λαός ιδών πόλεμον μεταμεληθή, και επιστρέψη εις Αίγυπτον.
\par 18 Αλλ' ο Θεός περιέφερε τον λαόν διά της οδού της ερήμου προς την Ερυθράν θάλασσαν· και ανέβησαν οι υιοί Ισραήλ εκ της γης Αιγύπτου παρατεταγμένοι.
\par 19 Και έλαβε μεθ' εαυτού ο Μωϋσής τα οστά του Ιωσήφ· διότι είχεν ορκίσει μεθ' όρκου τους υιούς Ισραήλ, λέγων· Ο Θεός βεβαίως θέλει σας επισκεφθή· και θέλετε αναβιβάσει τα οστά μου εντεύθεν μεθ' υμών.
\par 20 Και αναχωρήσαντες από Σοκχώθ, εστρατοπέδευσαν εν Εθάμ κατά τα άκρα της ερήμου.
\par 21 Ο δε Κύριος προεπορεύετο αυτών, την ημέραν εν στύλω νεφέλης, διά να οδηγή αυτούς εν τη οδώ, την δε νύκτα εν στύλω πυρός, διά να φέγγη εις αυτούς· ώστε να οδοιπορώσιν ημέραν και νύκτα·
\par 22 δεν απεμάκρυνεν από της όψεως του λαού τον στύλον της νεφέλης την ημέραν, ούτε τον στύλον του πυρός την νύκτα.

\chapter{14}

\par Και ελάλησε Κύριος προς τον Μωϋσήν, λέγων,
\par 2 Ειπέ προς τους υιούς Ισραήλ να στρέψωσι και να στρατοπεδεύσωσιν απέναντι Πι-αϊρώθ μεταξύ Μιγδώλ και της θαλάσσης, κατάντικρυ Βέελ-σεφών· κατάντικρυ τούτου θέλετε στρατοπεδεύσει πλησίον της θαλάσσης·
\par 3 διότι ο Φαραώ θέλει ειπεί περί των υιών Ισραήλ, Αυτοί πλανώνται εν τη γή· συνέκλεισεν αυτούς η έρημος·
\par 4 και εγώ θέλω σκληρύνει την καρδίαν του Φαραώ, ώστε να καταδιώξη οπίσω αυτών· και θέλω δοξασθή επί τον Φαραώ και επί παν το στράτευμα αυτού· και οι Αιγύπτιοι θέλουσι γνωρίσει ότι εγώ είμαι ο Κύριος. Και έκαμον ούτω.
\par 5 Ανηγγέλθη δε προς τον βασιλέα της Αιγύπτου ότι έφυγεν ο λαός· και η καρδία του Φαραώ και των θεραπόντων αυτού μετεβλήθη κατά του λαού και είπον, Διά τι εκάμομεν τούτο, ώστε να εξαποστείλωμεν τον Ισραήλ και να μη μας δουλεύη πλέον;
\par 6 Έζευξε λοιπόν την άμαξαν αυτού και παρέλαβε τον λαόν αυτού μεθ' εαυτού·
\par 7 έλαβε δε εξακοσίας αμάξας εκλεκτάς, και πάσας τας αμάξας της Αιγύπτου, και αρχηγούς επί πάντων.
\par 8 Και εσκλήρυνε Κύριος την καρδίαν Φαραώ του βασιλέως της Αιγύπτου, και κατεδίωξεν οπίσω των υιών Ισραήλ· οι δε υιοί Ισραήλ εξήρχοντο διά χειρός υψηλής.
\par 9 Και κατεδίωξαν οι Αιγύπτιοι οπίσω αυτών, πάντες οι ίπποι, αι άμαξαι του Φαραώ, και οι ιππείς αυτού, και το στράτευμα αυτού· και έφθασαν αυτούς εστρατοπεδευμένους πλησίον της θαλάσσης απέναντι Πι-αϊρώθ, κατάντικρυ Βέελ-σεφών.
\par 10 Και ότε επλησίασεν ο Φαραώ, οι υιοί Ισραήλ ύψωσαν τους οφθαλμούς αυτών, και ιδού, οι Αιγύπτιοι ήρχοντο οπίσω αυτών· και εφοβήθησαν σφόδρα· και ανεβόησαν οι υιοί Ισραήλ προς τον Κύριον.
\par 11 Και είπον προς τον Μωϋσήν, Διότι δεν ήσαν μνήματα εν Αιγύπτω, εξήγαγες ημάς διά να αποθάνωμεν εν τη ερήμω; Διά τι έκαμες εις ημάς τούτο και εξήγαγες ημάς εξ Αιγύπτου;
\par 12 δεν είναι ούτος ο λόγος τον οποίον σοι είπομεν εν Αιγύπτω, λέγοντες, Άφες ημάς και ας δουλεύωμεν τους Αιγυπτίους; διότι καλήτερον ήτο εις ημάς να δουλεύωμεν τους Αιγυπτίους, παρά να αποθάνωμεν εν τη ερήμω.
\par 13 Και είπεν ο Μωϋσής προς τον λαόν, Μη φοβείσθε· σταθήτε και βλέπετε την σωτήριαν του Κυρίου, την οποίαν θέλει κάμει εις εσάς σήμερον· διότι τους Αιγυπτίους, τους οποίους είδετε σήμερον, δεν θέλετε ιδεί αυτούς πλέον εις τον αιώνα·
\par 14 ο Κύριος θέλει πολεμήσει διά σάς· σεις δε θέλετε μένει ήσυχοι.
\par 15 Και είπε Κύριος προς τον Μωϋσήν, Τι βοάς προς εμέ; ειπέ προς τους υιούς Ισραήλ να κινήσωσι·
\par 16 συ δε ύψωσον την ράβδον σου και έκτεινον την χείρα σου επί την θάλασσαν και σχίσον αυτήν, και ας διέλθωσιν οι υιοί Ισραήλ διά ξηράς εν μέσω της θαλάσσης·
\par 17 και εγώ, ιδού, θέλω σκληρύνει την καρδίαν των Αιγυπτίων, και θέλουσιν εμβή κατόπιν αυτών· και θέλω δοξασθή επί τον Φαραώ και επί παν το στράτευμα αυτού, επί τας αμάξας αυτού και επί τους ιππείς αυτού·
\par 18 και θέλουσι γνωρίσει οι Αιγύπτιοι ότι εγώ είμαι ο Κύριος, όταν δοξασθώ επί τον Φαραώ, επί τας αμάξας αυτού και επί τους ιππείς αυτού.
\par 19 Τότε ο άγγελος του Θεού, ο προπορευόμενος του στρατεύματος του Ισραήλ, εσηκώθη και ήλθεν οπίσω αυτών· και ο στύλος της νεφέλης εσηκώθη απ' έμπροσθεν αυτών, και εστάθη όπισθεν αυτών·
\par 20 και ήλθε μεταξύ του στρατεύματος των Αιγυπτίων και του στρατεύματος του Ισραήλ· και εις εκείνους μεν ήτο νέφος σκοτίζον, εις τούτους δε φωτίζον την νύκτα· ώστε το εν δεν επλησίασε το άλλο καθ' όλην την νύκτα.
\par 21 Ο δε Μωϋσής εξέτεινε την χείρα αυτού επί την θάλασσαν· και έκαμεν ο Κύριος την θάλασσαν να συρθή όλην εκείνην την νύκτα υπό σφοδρού ανατολικού ανέμου και κατέστησε την θάλασσαν ξηράν, και τα ύδατα διεχωρίσθησαν.
\par 22 Και εισήλθον οι υιοί του Ισραήλ εις το μέσον της θαλάσσης κατά το ξηρόν, και τα ύδατα ήσαν εις αυτούς τοίχος εκ δεξιών και εξ αριστερών αυτών.
\par 23 Κατεδίωξαν δε οι Αιγύπτιοι και εισήλθον κατόπιν αυτών, πάντες οι ίπποι του Φαραώ, αι άμαξαι αυτού και οι ιππείς αυτού, εν τω μέσω της θαλάσσης.
\par 24 Και εν τη φυλακή τη πρωϊνή επέβλεψεν ο Κύριος εκ του στύλου του πυρός και της νεφέλης επί το στράτευμα των Αιγυπτίων και συνετάραξε το στράτευμα των Αιγυπτίων·
\par 25 και εξέβαλε τους τροχούς των αμαξών αυτών, ώστε εσύροντο δυσκόλως· και είπον οι Αιγύπτιοι, Ας φύγωμεν απ' έμπροσθεν του Ισραήλ, διότι ο Κύριος πολεμεί τους Αιγυπτίους υπέρ αυτών.
\par 26 Ο δε Κύριος είπε προς τον Μωϋσήν, Έκτεινον την χείρα σου επί την θάλασσαν, και ας επαναστρέψωσι τα ύδατα επί τους Αιγυπτίους, επί τας αμάξας αυτών και επί τους ιππείς αυτών.
\par 27 Και εξέτεινεν ο Μωϋσής την χείρα αυτού επί την θάλασσαν· και η θάλασσα επανέλαβε την ορμήν αυτής περί την αυγήν· οι δε Αιγύπτιοι φεύγοντες απήντησαν αυτήν· και κατέστρεψε Κύριος τους Αιγυπτίους εν τω μέσω της θαλάσσης·
\par 28 διότι τα ύδατα επαναστρέψαντα εσκέπασαν τας αμάξας και τους ιππείς, παν το στράτευμα του Φαραώ, το οποίον είχεν εμβή κατόπιν αυτών εις την θάλασσαν· δεν έμεινεν εξ αυτών ουδέ εις.
\par 29 Οι δε υιοί Ισραήλ επέρασαν διά ξηράς εν μέσω της θαλάσσης· και τα ύδατα ήσαν εις αυτούς τοίχος εκ δεξιών αυτών και εξ αριστερών αυτών.
\par 30 Και έσωσε Κύριος εν τη ημέρα εκείνη τον Ισραήλ εκ χειρός των Αιγυπτίων· και είδεν ο Ισραήλ τους Αιγυπτίους νεκρούς επί το χείλος της θαλάσσης.
\par 31 Και είδεν ο Ισραήλ το μέγα εκείνο έργον, το οποίον έκαμεν ο Κύριος επί τους Αιγυπτίους· και εφοβήθη ο λαός τον Κύριον, και επίστευσεν εις τον Κύριον, και εις τον Μωϋσήν τον θεράποντα αυτού.

\chapter{15}

\par Τότε έψαλεν ο Μωϋσής και οι υιοί Ισραήλ την ωδήν ταύτην προς τον Κύριον, και είπον λέγοντες, Ας ψάλλω προς τον Κύριον· διότι εδοξάσθη ενδόξως· τον ίππον και τον αναβάτην αυτού έρριψεν εις την θάλασσαν.
\par 2 Ο Κύριος είναι η δύναμίς μου και το άσμά μου, και εστάθη η σωτηρία μου· αυτός είναι Θεός μου και θέλω δοξάσει αυτόν· Θεός του πατρός μου, και θέλω υψώσει αυτόν.
\par 3 Ο Κύριος είναι δυνατός πολεμιστής· Κύριος το όνομα αυτού.
\par 4 Του Φαραώ τας αμάξας και το στράτευμα αυτού έρριψεν εις την θάλασσαν· και εκλεκτοί πολέμαρχοι αυτού κατεποντίσθησαν εν τη Ερυθρά θαλάσση.
\par 5 Αι άβυσσοι εσκέπασαν αυτούς· ως πέτρα κατεβυθίσθησαν εις τα βάθη.
\par 6 Η δεξιά σου, Κύριε, εδοξάσθη εις δύναμιν· η δεξιά σου, Κύριε, συνέτριψε τον εχθρόν.
\par 7 Και με το μέγεθος της υπεροχής σου εξωλόθρευσας τους υπεναντίους σου· εξαπέστειλας την οργήν σου και κατέφαγεν αυτούς ως καλάμην.
\par 8 Και με την πνοήν του θυμού σου τα ύδατα επεσωρεύθησαν ομού· τα κύματα εστάθησαν ως σωρός, αι άβυσσοι έπηξαν εν τω μέσω της θαλάσσης.
\par 9 Ο εχθρός είπε, Θέλω καταδιώξει, θέλω καταφθάσει, θέλω διαμοιρασθή τα λάφυρα· η ψυχή μου θέλει χορτασθή επ' αυτούς· θέλω σύρει την μάχαιράν μου, η χειρ μου θέλει αφανίσει αυτούς.
\par 10 Εφύσησας με τον άνεμόν σου και η θάλασσα εσκέπασεν αυτούς· κατεβυθίσθησαν ως μόλυβδος εις τα φοβερά ύδατα.
\par 11 Τις όμοιός σου Κύριε, μεταξύ των θεών; Τις όμοιός σου, ένδοξος εις αγιότητα, θαυμαστός εις ύμνους, ενεργών τεράστια;
\par 12 Εξέτεινας την δεξιάν σου, και η γη κατέπιεν αυτούς.
\par 13 Με το έλεός σου ώδήγησας τον λαόν τούτον, τον οποίον ελύτρωσας· ώδήγησας αυτόν με την δύναμίν σου προς την κατοικίαν της αγιότητός σου.
\par 14 Οι λαοί θέλουσιν ακούσει και φρίξει· πόνοι θέλουσι κατακυριεύσει τους κατοίκους της Παλαιστίνης.
\par 15 Τότε θέλουσιν εκπλαγή οι ηγεμόνες Εδώμ· τρόμος θέλει καταλάβει τους άρχοντας του Μωάβ· πάντες οι κάτοικοι της Χαναάν θέλουσιν αναλυθή.
\par 16 Φόβος και τρόμος θέλει επιπέσει επ' αυτούς· από του μεγέθους του βραχίονός σου θέλουσιν απολιθωθή, εωσού περάση ο λαός σου, Κύριε, εωσού περάση ο λαός ούτος, τον οποίον απέκτησας.
\par 17 Θέλεις εισαγάγει αυτούς και φυτεύσει αυτούς εις το όρος της κληρονομίας σου, τον τόπον, Κύριε, τον οποίον ητοίμασας διά κατοικίαν σου, το αγιαστήριον, Κύριε, το οποίον αι χείρες σου έστησαν.
\par 18 Ο Κύριος θέλει βασιλεύει εις τους αιώνας των αιώνων.
\par 19 Διότι εισήλθον οι ίπποι του Φαραώ εις την θάλασσαν μετά των αμαξών αυτού και μετά των ιππέων αυτού, και ο Κύριος έστρεψεν επ' αυτούς τα ύδατα της θαλάσσης· οι δε υιοί Ισραήλ επέρασαν διά ξηράς εν τω μέσω της θαλάσσης.
\par 20 Μαριάμ δε η προφήτις, η αδελφή του Ααρών έλαβε το τύμπανον εν τη χειρί αυτής και πάσαι αι γυναίκες εξήλθον κατόπιν αυτής μετά τυμπάνων και χορών.
\par 21 Και η Μαριάμ ανταπεκρίνετο προς αυτούς, λέγουσα, Ψάλλετε εις τον Κύριον· διότι εδοξάσθη ενδόξως· τον ίππον και τον αναβάτην αυτού έρριψεν εις θάλασσαν.
\par 22 Τότε εσήκωσεν ο Μωϋσής τους Ισραηλίτας από της Ερυθράς θαλάσσης, και εξήλθον εις την έρημον Σούρ· και περιεπάτουν τρεις ημέρας εν τη ερήμω και δεν εύρισκον ύδωρ.
\par 23 Και εκείθεν ήλθον εις Μερράν· δεν ηδύναντο όμως να πίωσιν εκ των υδάτων της Μερράς, διότι ήσαν πικρά· διά τούτο και επωνομάσθη Μερρά.
\par 24 Και εγόγγυζεν ο λαός κατά του Μωϋσέως, λέγων, Τι θέλομεν πίει;
\par 25 Ο δε Μωϋσής εβόησε προς τον Κύριον· και έδειξεν εις αυτόν ο Κύριος ξύλον, το οποίον ότε έρριψεν εις τα ύδατα, τα ύδατα εγλυκάνθησαν. Εκεί έδωκεν εις αυτούς παραγγελίαν και διάταγμα, και εκεί εδοκίμασεν αυτούς·
\par 26 και είπεν, Εάν ακούσης επιμελώς την φωνήν Κυρίου του Θεού σου και πράττης το αρεστόν εις τους οφθαλμούς αυτού και δώσης ακρόασιν εις τας εντολάς αυτού και φυλάξης πάντα τα προστάγματα αυτού, δεν θέλω φέρει επί σε ουδεμίαν εκ των νόσων, τας οποίας έφερα κατά των Αιγυπτίων· διότι εγώ είμαι ο Κύριος ο θεραπεύων σε.
\par 27 Έπειτα ήλθον εις Αιλείμ, όπου ήσαν δώδεκα πηγαί υδάτων και εβδομήκοντα δένδρα φοινίκων· και εκεί εστρατοπέδευσαν πλησίον των υδάτων.

\chapter{16}

\par Εσηκώθησαν δε από Αιλείμ· και ήλθον πάσα η συναγωγή των υιών Ισραήλ εις την έρημον Σιν, την μεταξύ Αιλείμ και Σινά, την δεκάτην πέμπτην ημέραν του δευτέρου μηνός αφού εξήλθον εκ γης Αιγύπτου.
\par 2 Και εγόγγυζε πάσα η συναγωγή των υιών Ισραήλ κατά του Μωϋσέως και κατά του Ααρών εν τη ερήμω.
\par 3 Και είπον προς αυτούς οι υιοί Ισραήλ, Είθε να απεθνήσκομεν υπό της χειρός του Κυρίου εν τη γη της Αιγύπτου, ότε εκαθήμεθα πλησίον των λεβήτων του κρέατος και ότε ετρώγομεν άρτον εις χορτασμόν· διότι εξηγάγετε ημάς εις την έρημον ταύτην, διά να θανατώσητε με την πείναν πάσαν την συναγωγήν ταύτην.
\par 4 Και είπε Κύριος προς τον Μωϋσήν, Ιδού, θέλω βρέξει εις εσάς άρτον εξ ουρανού· και θέλει εξέρχεσθαι ο λαός και συνάγει καθ' ημέραν το αρκούν της ημέρας, διά να δοκιμάσω αυτούς, αν θέλωσι περιπατεί εις τον νόμον μου ή ουχί·
\par 5 την δε έκτην ημέραν ας ετοιμάζωσιν εκείνο το οποίον ήθελον εισαγάγει, και ας ήναι διπλάσιον του όσον συνάγουσι καθ' ημέραν.
\par 6 Και είπον ο Μωϋσής και ο Ααρών προς πάντας τους υιούς Ισραήλ, Το εσπέρας θέλετε γνωρίσει ότι ο Κύριος εξήγαγεν υμάς εκ γης Αιγύπτου·
\par 7 και το πρωΐ θέλετε ιδεί την δόξαν του Κυρίου, διότι ήκουσε τους γογγυσμούς σας εναντίον του Κυρίου· επειδή ημείς τι είμεθα, ώστε να γογγύζητε καθ' ημών;
\par 8 Και είπεν ο Μωϋσής, Τούτο θέλει γείνει, όταν ο Κύριος δώση εις εσάς το εσπέρας κρέας να φάγητε και το πρωΐ άρτον εις χορτασμόν· διότι ήκουσε Κύριος τους γογγυσμούς σας τους οποίους γογγύζετε κατ' αυτού· και τι είμεθα ημείς; οι γογγυσμοί σας δεν είναι καθ' ημών, αλλά κατά του Κυρίου.
\par 9 Και είπεν ο Μωϋσής προς τον Ααρών, Ειπέ προς πάσαν την συναγωγήν των υιών Ισραήλ, Πλησιάσατε έμπροσθεν του Κυρίου· διότι ήκουσε τους γογγυσμούς σας.
\par 10 Και ενώ ελάλει ο Ααρών προς πάσαν την συναγωγήν των υιών Ισραήλ, έστρεψαν το πρόσωπον προς την έρημον, και ιδού, η δόξα του Κυρίου εφάνη εν τη νεφέλη.
\par 11 Και ελάλησε Κύριος προς τον Μωϋσήν, λέγων,
\par 12 Ήκουσα τους γογγυσμούς των υιών Ισραήλ· λάλησον προς αυτούς, λέγων, Το εσπέρας θέλετε φάγει κρέας, και το πρωΐ θέλετε χορτασθή από άρτου, και θέλετε γνωρίσει ότι εγώ είμαι Κύριος ο Θεός σας.
\par 13 Και το εσπέρας ανέβησαν ορτύκια και εσκέπασαν το στρατόπεδον· και το πρωΐ καθ' όλα τα πέριξ του στρατοπέδου ήτο στρώμα δρόσου.
\par 14 Και αφού το στρώμα της δρόσου ανέβη, ιδού, επί το πρόσωπον της ερήμου ήτο λεπτόν τι στρογγύλον, λεπτόν ως πάχνη επί της γης.
\par 15 Και ότε είδον οι υιοί Ισραήλ, είπον προς αλλήλους, Τι είναι τούτο; διότι δεν ήξευρον τι ήτο. Και ο Μωϋσής είπε προς αυτούς, Ούτος είναι ο άρτος, τον οποίον ο Κύριος σας δίδει διά να φάγητε·
\par 16 ούτος είναι ο λόγος τον οποίον προσέταξεν ο Κύριος, Συνάξατε εξ αυτού έκαστος όσον χρειάζεται διά να φάγη, εν γομόρ κατά κεφαλήν, κατά τον αριθμόν των ψυχών σας· λάβετε έκαστος διά τους ομοσκήνους αυτού.
\par 17 Και έκαμον ούτως οι υιοί Ισραήλ, και συνήγαγον άλλος πολύ και άλλος ολίγον.
\par 18 Και ότε εμέτρησαν με το γομόρ, όστις είχε συνάξει πολύ, δεν ελάμβανε πλειότερον· και όστις είχε συνάξει ολίγον, δεν ελάμβανεν ολιγώτερον· έκαστος ελάμβανεν όσον εχρειάζετο εις αυτόν διά τροφήν.
\par 19 είπε δε προς αυτούς ο Μωϋσής, Ας μη αφίνη εξ αυτού μηδείς υπόλοιπον έως πρωΐ.
\par 20 Πλην δεν υπήκουσαν εις τον Μωϋσήν· αλλά αφήκαν τινές υπόλοιπον εξ αυτού έως πρωΐ, και εγέννησε σκώληκας και εβρώμησε· και εθυμώθη εναντίον αυτών ο Μωϋσής.
\par 21 Και συνήγον αυτό καθ' εκάστην πρωΐαν, έκαστος όσον εχρειάζετο διά τροφήν αυτού· και ότε ο ήλιος εθέρμαινε, διελύετο.
\par 22 Την δε έκτην ημέραν συνήγαγον τροφήν διπλασίαν, δύο γομόρ δι' ένα· και ήλθον πάντες οι άρχοντες της συναγωγής και ανήγγειλαν τούτο προς τον Μωϋσήν.
\par 23 Ο δε είπε προς αυτούς, Τούτο είναι το οποίον είπε Κύριος· Αύριον είναι σάββατον, ανάπαυσις αγία εις τον Κύριον· ψήσατε ό,τι έχετε να ψήσητε και βράσατε ό,τι έχετε να βράσητε· και παν το περισσεύον εναποταμιεύσατε εις εαυτούς διά να φυλάττηται έως πρωΐ.
\par 24 Και εναπεταμίευσαν αυτό έως πρωΐ, καθώς προσέταξεν ο Μωϋσής· και δεν εβρώμησεν ουδέ έγεινε σκώληξ εν αυτώ.
\par 25 Και είπεν ο Μωϋσής, Φάγετε αυτό σήμερον· διότι σήμερον είναι σάββατον εις τον Κύριον· σήμερον δεν θέλετε ευρεί αυτό εν τη πεδιάδι·
\par 26 εξ ημέρας θέλετε συνάγει αυτό· εν τη εβδόμη όμως ημέρα, τω σαββάτω, εν ταύτη δεν θέλει ευρίσκεσθαι.
\par 27 Τινές δε εκ του λαού εξήλθον την εβδόμην ημέραν διά να συνάξωσι, πλην δεν εύρον.
\par 28 Και είπε Κύριος προς τον Μωϋσήν, Έως πότε δεν θέλετε να φυλάττητε τας εντολάς μου και τους νόμους μου;
\par 29 ίδετε ότι ο Κύριος έδωκεν εις εσάς το σάββατον, διά τούτο την έκτην ημέραν σας δίδει άρτον δύο ημερών· καθίσατε έκαστος εις τον τόπον αυτού· ας μη εξέρχεται μηδείς εκ του τόπου αυτού την εβδόμην ημέραν.
\par 30 Και έκαμε κατάπαυσιν ο λαός την εβδόμην ημέραν.
\par 31 Και εκάλεσεν ο οίκος του Ισραήλ το όνομα αυτού Μάν· ήτο δε όμοιον με σπόρον κοριάνδρου λευκόν· και η γεύσις αυτού ως πλακούντιον με μέλι.
\par 32 Και είπεν ο Μωϋσής, Ούτος είναι ο λόγος τον οποίον προσέταξεν ο Κύριος· Γεμίσατε εξ αυτού εν γομόρ, διά να φυλάττηται εις τας γενεάς σας, διά να βλέπωσι τον άρτον με τον οποίον έθρεψα υμάς εν τη ερήμω, αφού εξήγαγον υμάς εκ γης Αιγύπτου.
\par 33 Και είπεν ο Μωϋσής προς τον Ααρών, Λάβε μίαν στάμνον, και βάλε εν αυτή εν γομόρ πλήρες από μάννα, και θες αυτήν έμπροσθεν του Κυρίου, διά να φυλάττηται εις τας γενεάς σας.
\par 34 Και έθεσεν αυτήν ο Ααρών έμπροσθεν του Μαρτυρίου, διά να φυλάττηται, καθώς προσέταξεν ο Κύριος εις τον Μωϋσήν.
\par 35 Και έτρωγον οι υιοί Ισραήλ το μάννα τεσσαράκοντα έτη, εωσού ήλθον εις γην κατοικουμένην· έτρωγον το μάννα, εωσού ήλθον εις τα όρια της γης Χαναάν.
\par 36 Το δε γομόρ είναι το δέκατον του εφά.

\chapter{17}

\par Και εσηκώθη πάσα η συναγωγή των υιών Ισραήλ εκ της ερήμου Σιν, ακολουθούντες τας οδοιπορείας αυτών κατά την προσταγήν του Κυρίου, και εστρατοπέδευσαν εν Ραφιδείν· όπου δεν ήτο ύδωρ διά να πίη ο λαός.
\par 2 Και ελοιδόρει ο λαός κατά του Μωϋσέως, λέγοντες, Δος εις ημάς ύδωρ διά να πίωμεν. Και είπε προς αυτούς ο Μωϋσής, Διά τι λοιδορείτε κατ' εμού; διά τι πειράζετε τον Κύριον;
\par 3 Και εδίψησεν ο λαός εκεί διά ύδωρ· και εγόγγυζεν ο λαός κατά του Μωϋσέως, λέγοντες, Διά τι τούτο; ανεβίβασας ημάς εξ Αιγύπτου, διά να θανατώσης ημάς και τα τέκνα ημών και τα κτήνη ημών με την δίψαν;
\par 4 Και εβόησεν ο Μωϋσής προς τον Κύριον, λέγων, Τι να κάμω εις τούτον τον λαόν; ολίγον λείπει να με λιθοβολήσωσι.
\par 5 Και είπε Κύριος προς τον Μωϋσήν, Διάβα έμπροσθεν του λαού, και λάβε μετά σεαυτού εκ των πρεσβυτέρων του Ισραήλ· και την ράβδον, σου, με την οποίαν εκτύπησας τον ποταμόν, λάβε εν τη χειρί σου και ύπαγε·
\par 6 ιδού, εγώ θέλω σταθή εκεί έμπροσθέν σου επί της πέτρας εν Χωρήβ, και θέλεις κτυπήσει την πέτραν και θέλει εξέλθει ύδωρ εξ αυτής διά να πίη ο λαός. Και έκαμεν ούτως ο Μωϋσής ενώπιον των πρεσβυτέρων του Ισραήλ.
\par 7 Και εκάλεσε το όνομα του τόπου Μασσά, και Μεριβά, διά την λοιδορίαν των υιών Ιαραήλ, και διότι επείρασαν τον Κύριον, λέγοντες, Είναι ο Κύριος μεταξύ ημών ή ουχί;
\par 8 Τότε ήλθεν ο Αμαλήκ και επολέμησε με τον Ισραήλ εν Ραφιδείν.
\par 9 Και είπεν ο Μωϋσής προς τον Ιησούν, Έκλεξοω εις ημάς άνδρας και εξελθών πολέμησον με τον Αμαλήκ· αύριον εγώ θέλω σταθή επί της κορυφής του βουνού, κρατών εν τη χειρί μου την ράβδον του Θεού.
\par 10 Και έκαμεν ο Ιησούς καθώς είπε προς αυτόν ο Μωϋσής και επολέμησε με τον Αμαλήκ· ο δε Μωϋσής, ο Ααρών και ο Ωρ ανέβησαν επί την κορυφήν του βουνού.
\par 11 Και οπότε ο Μωϋσής ύψονε την χείρα αυτού, ενίκα ο Ισραήλ· οπότε δε κατεβίβαζε την χείρα αυτού, ενίκα ο Αμαλήκ.
\par 12 Αι χείρες δε του Μωϋσέως ήσαν βεβαρημέναι· όθεν λαβόντες λίθον, έθεσαν υποκάτω αυτού και εκάθισεν επ' αυτού· ο δε Ααρών και ο Ωρ, εις εκ του ενός μέρους και εις εκ του άλλου, υπεστήριζον τας χείρας αυτού· και αι χείρες αυτού έμενον εστηριγμέναι μέχρι δύσεως ηλίου.
\par 13 Και κατέστρεψεν ο Ιησούς τον Αμαλήκ και τον λαόν αυτού, εν στόματι μαχαίρας.
\par 14 Και είπε Κύριος προς τον Μωϋσήν, Γράψον τούτο εν βιβλίω προς μνημόσυνον, και παράδος εις τα ώτα του Ιησού· ότι θέλω εξαλείψει εξάπαντος την μνήμην του Αμαλήκ εκ της υπό τον ουρανόν.
\par 15 Και ωκοδόμησεν εκεί ο Μωϋσής θυσιαστήριον και εκάλεσε το όνομα αυτού Ιεοβά-Νισσί·
\par 16 και είπεν, Επειδή χειρ υψώθη κατά του θρόνου του Κυρίου, θέλει είσθαι πόλεμος του Κυρίου προς τον Αμαλήκ από γενεάς εις γενεάν.

\chapter{18}

\par Ήκουσε δε ο Ιοθόρ, ο ιερεύς της Μαδιάμ, ο πενθερός του Μωϋσέως, πάντα όσα έκαμεν ο Θεός εις τον Μωϋσήν και εις τον Ισραήλ τον λαόν αυτού, ότι εξήγαγεν ο Κύριος τον Ισραήλ εξ Αιγύπτου.
\par 2 Και έλαβεν ο Ιοθόρ, ο πενθερός του Μωϋσέως, Σεπφώραν την γυναίκα του Μωϋσέως, την οποίαν είχε πέμψει οπίσω,
\par 3 και τους δύο αυτής υιούς, εκ των οποίων του ενός το όνομα ήτο Γηρσώμ, Διότι πάροικος, είπεν, εστάθην εν ξένη γή·
\par 4 του δε άλλου το όνομα Ελιέζερ, Διότι ο Θεός, είπε, του πατρός μου εστάθη βοηθός μου και με έσωσεν εκ της μαχαίρας του Φαραώ·
\par 5 και ήλθεν ο Ιοθόρ ο πενθερός του Μωϋσέως προς τον Μωϋσήν μετά των υιών αυτού και μετά της γυναικός αυτού εις την έρημον, όπου ήτο εστρατοπεδευμένος εις το όρος του Θεού·
\par 6 και ανήγγειλε προς τον Μωϋσήν, Εγώ Ιοθόρ ο πενθερός σου έρχομαι προς σε και η γυνή σου και οι δύο υιοί αυτής μετ' αυτής.
\par 7 Και εξήλθεν ο Μωϋσής εις συνάντησιν του πενθερού αυτού και προσεκύνησεν αυτόν και εφίλησεν αυτόν· και ηρώτησαν ο εις τον άλλον περί της υγείας αυτών, και εισήλθον εις την σκηνήν.
\par 8 Και διηγήθη ο Μωϋσής προς τον πενθερόν αυτού πάντα όσα ο Κύριος έκαμεν εις τον Φαραώ και εις τους Αιγυπτίους υπέρ του Ισραήλ, πάντας τους μόχθους οίτινες συνέβησαν εις αυτούς καθ' οδόν, και ηλευθέρωσεν αυτούς ο Κύριος.
\par 9 Υπερεχάρη δε ο Ιοθόρ διά πάντα τα αγαθά όσα ο Κύριος έκαμεν εις τον Ισραήλ, τον οποίον ηλευθέρωσεν εκ χειρός των Αιγυπτίων.
\par 10 Και είπεν ο Ιοθόρ, Ευλογητός Κύριος, όστις σας ηλευθέρωσεν εκ χειρός των Αιγυπτίων και εκ χειρός του Φαραώ· όστις ηλευθέρωσε τον λαόν υποκάτωθεν της χειρός των Αιγυπτίων·
\par 11 τώρα γνωρίζω ότι ο Κύριος είναι μέγας υπέρ πάντας τους θεούς· διότι εις το πράγμα, εις το οποίον υπερηφανεύθησαν, εστάθη ανώτερος αυτών.
\par 12 Έλαβεν έπειτα ο Ιοθόρ, ο πενθερός του Μωϋσέως, ολοκαυτώματα και θυσίας διά να προσφέρη εις τον Θεόν· και ήλθεν ο Ααρών, και πάντες οι πρεσβύτεροι του Ισραήλ, να φάγωσιν άρτον μετά του πενθερού του Μωϋσέως, έμπροσθεν του Θεού.
\par 13 Και την επαύριον εκάθισεν ο Μωϋσής διά να κρίνη τον λαόν. και παρίστατο ο λαός έμπροσθεν του Μωϋσέως από πρωΐας έως εσπέρας.
\par 14 Και ιδών ο πενθερός του Μωϋσέως πάντα όσα έκαμνεν εις τον λαόν, είπε, Τι είναι τούτο το πράγμα, το οποίον κάμνεις εις τον λαόν; διά τι συ κάθησαι μόνος, άπας δε ο λαός παρίσταται έμπροσθέν σου από πρωΐας έως εσπέρας;
\par 15 Και είπεν ο Μωϋσής προς τον πενθερόν αυτού, διότι ο λαός έρχεται προς εμέ διά να ερωτήση τον Θεόν·
\par 16 όταν έχωσιν υπόθεσίν τινά, έρχονται προς εμέ και εγώ κρίνω μεταξύ του ενός και του άλλου· και δεικνύω εις αυτούς τα προστάγματα του Θεού και τους νόμους αυτού.
\par 17 Και είπεν ο πενθερός του Μωϋσέως προς αυτόν, Δεν είναι καλόν το πράγμα, το οποίον κάμνεις·
\par 18 βεβαίως και συ θέλεις αποκάμει και ο λαός ούτος ο μετά σού· διότι το πράγμα είναι πολύ βαρύ διά σέ· δεν δύνασαι μόνος να κάμνης τούτο.
\par 19 Άκουσον λοιπόν την φωνήν μου· θέλω σε συμβουλεύσει και ο Θεός θέλει είσθαι μετά σού· συ μεν έσο ενώπιον του Θεού υπέρ του λαού, διά να αναφέρης τας υποθέσεις προς τον Θεόν·
\par 20 και δίδασκε αυτούς τα προστάγματα και τους νόμους και δείκνυε προς αυτούς την οδόν εις την οποίαν πρέπει να περιπατώσι, και τα έργα τα οποία πρέπει να πράττωσι·
\par 21 πλην έκλεξον εκ παντός του λαού άνδρας αξίους, φοβουμένους τον Θεόν, άνδρας φιλαλήθεις, μισούντας την φιλαργυρίαν· και κατάστησον αυτούς επ' αυτών χιλιάρχους, εκατοντάρχους, πεντηκοντάρχους και δεκάρχους·
\par 22 και ας κρίνωσι τον λαόν πάντοτε· και πάσαν μεν μεγάλην υπόθεσιν ας αναφέρωσι προς σέ· πάσαν δε μικράν υπόθεσιν ας κρίνωσιν αυτοί· ούτω θέλεις ανακουφισθή, και θέλουσι βαστάζει το βάρος μετά σου.
\par 23 εάν κάμης τούτο το πράγμα και ο Θεός σε προστάζη ούτω, τότε θέλεις δυνηθή να ανθέξης, και προσέτι πας ο λαός ούτος θέλει φθάσει εις τον τόπον αυτού εν ειρήνη.
\par 24 Και ήκουσεν ο Μωϋσής την φωνήν του πενθερού αυτού και έκαμε πάντα όσα είπε.
\par 25 Και έκλεξεν ο Μωϋσής εκ παντός του Ισραήλ άνδρας αξίους και κατέστησεν αυτούς αρχηγούς επί του λαού, χιλιάρχους, εκατοντάρχους, πεντηκοντάρχους και δεκάρχους·
\par 26 και έκρινον τον λαόν εν παντί καιρώ· τας μεν υποθέσεις τας δυσκόλους ανέφερον προς τον Μωϋσήν, πάσαν δε μικράν υπόθεσιν έκρινον αυτοί.
\par 27 Έπειτα προέπεμψεν ο Μωϋσής τον πενθερόν αυτού και απήλθεν εις την γην αυτού.

\chapter{19}

\par Εις τον τρίτον μήνα της εξόδου των υιών Ισραήλ εκ της Αιγύπτου, την ημέραν ταύτην ήλθον εις την έρημον Σινά.
\par 2 Εσηκώθησαν δε από Ραφιδείν και ήλθον εις την έρημον Σινά και εστρατοπέδευσαν εν τη ερήμω· και εκεί κατεσκήνωσεν ο Ισραήλ απέναντι του όρους.
\par 3 Ο δε Μωϋσής ανέβη προς τον Θεόν· και εκάλεσεν αυτόν ο Κύριος εκ του όρους, λέγων, Ούτω θέλεις ειπεί προς τον οίκον Ιακώβ, και αναγγείλει προς τους υιούς Ισραήλ.
\par 4 Σεις είδετε όσα έκαμα εις τους Αιγυπτίους, και σας εσήκωσα ως επί πτερύγων αετού και σας έφερα προς εμαυτόν·
\par 5 τώρα λοιπόν εάν τωόντι υπακούσητε εις την φωνήν μου, και φυλάξητε την διαθήκην μου, θέλετε είσθαι εις εμέ ο εκλεκτός από πάντων των λαών· διότι ιδική μου είναι πάσα η γή·
\par 6 και σεις θέλετε είσθαι εις εμέ βασίλειον ιεράτευμα και έθνος άγιον. Ούτοι είναι οι λόγοι, τους οποίους θέλεις ειπεί προς τους υιούς Ισραήλ.
\par 7 Και ήλθεν ο Μωϋσής και εκάλεσε τους πρεσβυτέρους του λαού και έθεσεν έμπροσθεν αυτών πάντας εκείνους τους λόγους, τους οποίους προσέταξεν εις αυτόν ο Κύριος.
\par 8 Και απεκρίθη ομοφώνως πας ο λαός, λέγων, Πάντα όσα είπεν ο Κύριος θέλομεν πράξει. Και ανέφερεν ο Μωϋσής προς τον Κύριον τους λόγους του λαού.
\par 9 Και είπε Κύριος προς τον Μωϋσήν, Ιδού, εγώ έρχομαι προς σε εν νεφέλη πυκνή, διά να ακούση ο λαός όταν λαλήσω προς σε, και έτι να πιστεύη εις σε πάντοτε. Ανήγγειλε δε ο Μωϋσής προς τον Κύριον τους λόγους του λαού.
\par 10 Και είπε Κύριος προς τον Μωϋσήν, Ύπαγε προς τον λαόν και αγίασον αυτούς σήμερον και αύριον, και ας πλύνωσι τα ιμάτια αυτών·
\par 11 και ας ήναι έτοιμοι εις την ημέραν την τρίτην· διότι εν τη ημέρα τη τρίτη θέλει καταβή ο Κύριος επί το όρος Σινά ενώπιον παντός του λαού·
\par 12 και θέλεις βάλει εις τον λαόν όρια κυκλόθεν, λέγων, Προσέχετε εις εαυτούς μη αναβήτε εις το όρος ή εγγίσητε εις τα άκρα αυτού· όστις εγγίση το όρος, θέλει εξάπαντος θανατωθή·
\par 13 δεν θέλει εγγίσει εις αυτόν χειρ, διότι με λίθους θέλει λιθοβοληθή ή με βέλη θέλει κατατοξευθή· είτε ζώον είναι είτε άνθρωπος, δεν θέλει ζήσει. Όταν η σάλπιγξ ηχήση, τότε θέλουσιν αναβή επί το όρος.
\par 14 Και κατέβη ο Μωϋσής από του όρους προς τον λαόν και ηγίασε τον λαόν· και έπλυναν τα ιμάτια αυτών.
\par 15 Και είπε προς τον λαόν, Γίνεσθε έτοιμοι διά την ημέραν την τρίτην· μη πλησιάσητε εις γυναίκα.
\par 16 Και εν τη ημέρα τη τρίτη το πρωΐ έγειναν βρονταί και αστραπαί, και νεφέλη πυκνή ήτο επί του όρους, και φωνή σάλπιγγος δυνατή σφόδρα· και έτρεμε πας ο λαός ο εν τω στρατοπέδω.
\par 17 Τότε εξήγαγεν ο Μωϋσής τον λαόν εκ του στρατοπέδου εις την συνάντησιν του Θεού· και εστάθησαν υπό το όρος.
\par 18 Το δε όρος Σινά ήτο όλον καπνός, διότι κατέβη ο Κύριος εν πυρί επ' αυτό· ανέβαινε δε ο καπνός αυτού ως καπνός καμίνου και όλον το όρος εσείετο σφόδρα.
\par 19 Και ότε η φωνή της σάλπιγγος προέβαινεν αυξανομένη σφόδρα, ο Μωϋσής ελάλει και ο Θεός απεκρίνετο προς αυτόν μετά φωνής.
\par 20 Και κατέβη ο Κύριος επί το όρος Σινά, επί την κορυφήν του όρους· και εκάλεσε Κύριος τον Μωϋσήν επί την κορυφήν του όρους, και ανέβη ο Μωϋσής.
\par 21 Και είπε Κύριος προς τον Μωϋσήν, Καταβάς, διαμαρτυρήθητι προς τον λαόν, μήποτε υπερβώσι τα όρια και αναβώσι προς τον Κύριον διά να περιεργασθώσι και πέσωσι πολλοί εξ αυτών·
\par 22 και οι ιερείς δε οι πλησιάζοντες προς τον Κύριον ας αγιασθώσι, διά να μη εξορμήση ο Κύριος επ' αυτούς.
\par 23 Και είπεν ο Μωϋσής προς τον Κύριον, Ο λαός δεν δύναται να αναβή εις το όρος Σινά· διότι συ προσέταξας εις ημάς, λέγων, Βάλε όρια κυκλόθεν του όρους και αγίασον αυτό.
\par 24 Και είπε Κύριος προς αυτόν, Ύπαγε, κατάβα· έπειτα θέλεις αναβή, συ και ο Ααρών μετά σού· οι ιερείς όμως και ο λαός ας μη υπερβώσι τα όρια διά να αναβώσι προς τον Κύριον, διά να μη εξορμήση επ' αυτούς.
\par 25 Και κατέβη ο Μωϋσής προς τον λαόν και ώμίλησε προς αυτούς.

\chapter{20}

\par Και ελάλησεν ο Θεός πάντας τους λόγους τούτους, λέγων,
\par 2 Εγώ είμαι Κύριος ο Θεός σου, ο εξαγαγών σε εκ γης Αιγύπτου, εξ οίκου δουλείας.
\par 3 Μη έχης άλλους θεούς πλην εμού.
\par 4 Μη κάμης εις σεαυτόν είδωλον, μηδέ ομοίωμά τινός, όσα είναι εν τω ουρανώ άνω, ή όσα εν τη γη κάτω, όσα εν τοις ύδασιν υποκάτω της γής·
\par 5 μη προσκυνήσης αυτά μηδέ λατρεύσης αυτά· διότι εγώ Κύριος ο Θεός σου είμαι Θεός ζηλότυπος, ανταποδίδων τας αμαρτίας των πατέρων επί τα τέκνα, έως τρίτης και τετάρτης γενεάς των μισούντων με·
\par 6 και κάμνων έλεος εις χιλιάδας γενεών των αγαπώντων με και φυλαττόντων τα προστάγματά μου.
\par 7 Μη λάβης το όνομα Κυρίου του Θεού σου επί ματαίω· διότι δεν θέλει αθωώσει ο Κύριος τον λαμβάνοντα επί ματαίω το όνομα αυτού.
\par 8 Ενθυμού την ημέραν του σαββάτου, διά να αγιάζης αυτήν·
\par 9 εξ ημέρας εργάζου και κάμνε πάντα τα έργα σου·
\par 10 η ημέρα όμως η εβδόμη είναι σάββατον Κυρίου του Θεού σου· μη κάμης εν ταύτη ουδέν έργον, μήτε συ, μήτε ο υιός σου, μήτε η θυγάτηρ σου, μήτε ο δούλός σου, μήτε η δούλη σου, μήτε το κτήνός σου, μήτε ο ξένος σου, ο εντός των πυλών σου·
\par 11 διότι εις εξ ημέρας εποίησεν ο Κύριος τον ουρανόν και την γην, την θάλασσαν και πάντα τα εν αυτοίς· εν δε τη ημέρα τη εβδόμη κατέπαυσε· διά τούτο ευλόγησε Κύριος την ημέραν του σαββάτου και ηγίασεν αυτήν.
\par 12 Τίμα τον πατέρα σου και την μητέρα σου, διά να γείνης μακροχρόνιος επί της γης, την οποίαν σοι δίδει Κύριος ο Θεός σου.
\par 13 Μη φονεύσης

\par 14 Μη μοιχεύσης.
\par 15 Μη κλέψης.
\par 16 Μη ψευδομαρτυρήσης κατά του πλησίον σου μαρτυρίαν ψευδή.
\par 17 Μη επιθυμήσης την οικίαν του πλησίον σου· μη επιθυμήσης την γυναίκα του πλησίον σου· μηδέ τον δούλον αυτού· μηδέ την δούλην αυτού, μηδέ τον βουν αυτού, μηδέ τον όνον αυτού, μηδέ παν ό,τι είναι του πλησίον σου.
\par 18 Και πας ο λαός έβλεπε τας βροντάς και τας αστραπάς και την φωνήν της σάλπιγγος και το όρος καπνίζον· και ότε ο λαός είδε ταύτα, απεσύρθησαν και εστάθησαν μακρόθεν.
\par 19 Και είπον προς τον Μωϋσήν, συ λάλησον προς ημάς και θέλομεν ακούσει και ας μη λαλήση προς ημάς ο Θεός, διά να μη αποθάνωμεν.
\par 20 Και είπεν ο Μωϋσής προς τον λαόν, Μη φοβείσθε· διότι ο Θεός ήλθε διά να σας δοκιμάση, και διά να ήναι ο φόβος αυτού έμπροσθέν σας διά να μη αμαρτάνητε.
\par 21 Και εστάθη ο λαός μακρόθεν· ο δε Μωϋσής επλησίασεν εις την ομίχλην όπου ήτο ο Θεός.
\par 22 Είπε δε Κύριος προς τον Μωϋσήν, Ούτως ειπέ προς τους υιούς Ισραήλ· Σεις είδετε ότι εκ του ουρανού ελάλησα με σάς·
\par 23 μη κάμητε θεούς μετ' εμού αργυρούς· μηδέ κάμητε εις εαυτούς θεούς χρυσούς·
\par 24 θυσιαστήριον εκ γης κάμε εις εμέ· και θυσίαζε επ' αυτού τα ολοκαυτώματά σου και τας ειρηνικάς προσφοράς σου, τα πρόβατά σου και τους βόας σου· εν παντί τόπω όπου αναμνήσω το όνομά μου, θέλω έρχεσθαι προς σε και θέλω σε ευλογεί·
\par 25 εάν δε εκ λίθων κάμης θυσιαστήριον εις εμέ, δεν θέλεις οικοδομήσει αυτό εκ πέτρας πελεκητής· διότι εάν περάσης επάνω αυτού το εργαλείόν σου, θέλεις μολύνει αυτό·
\par 26 και μη αναβής δι' αναβαθμίδων επί το θυσιαστήριόν μου, διά να μη αποκαλυφθή επ' αυτού η γυμνωσίς σου.

\chapter{21}

\par Αύται δε είναι αι κρίσεις, τας οποίας θέλεις εκθέσει έμπροσθεν αυτών.
\par 2 Εάν αγοράσης δούλον Εβραίον, εξ έτη θέλει δουλεύσει· εν δε τω εβδόμω θέλει εξέλθει ελεύθερος, δωρεάν.
\par 3 Εάν εισήλθε μόνος, μόνος θέλει εξέλθει· εάν είχε γυναίκα, τότε η γυνή αυτού θέλει εξέλθει μετ' αυτού.
\par 4 Εάν ο κύριος αυτού έδωκεν εις αυτόν γυναίκα, και εγέννησεν εις αυτόν υιούς η θυγατέρας, η γυνή και τα τέκνα αυτής θέλουσιν είσθαι του κυρίου αυτής, αυτός δε θέλει εξέλθει μόνος.
\par 5 Αλλ' εάν ο δούλος είπη φανερά, Αγαπώ τον κύριόν μου, την γυναίκα μου και τα τέκνα μου, δεν θέλω εξέλθει ελεύθερος·
\par 6 τότε ο κύριος αυτού θέλει φέρει αυτόν προς τους κριτάς· και θέλει φέρει αυτόν εις την θύραν ή εις τον παραστάτην της θύρας, και ο κύριος αυτού θέλει τρυπήσει το ωτίον αυτού με τρυπητήριον· και θέλει δουλεύει αυτόν διαπαντός.
\par 7 Και εάν τις πωλήση την θυγατέρα αυτού διά δούλην, δεν θέλει εξέλθει καθώς εξέρχονται οι δούλοι.
\par 8 Εάν δεν αρέση εις τον κύριον αυτής, όστις ηρραβωνίσθη αυτήν εις εαυτόν, τότε θέλει απολυτρώσει αυτήν· εις ξένον έθνος δεν θέλει έχει εξουσίαν να πωλήση αυτήν, επειδή εφέρθη προς αυτήν απίστως.
\par 9 Αν όμως ηρραβώνισεν αυτήν με τον υιόν αυτού, θέλει κάμει προς αυτήν κατά το δικαίωμα των θυγατέρων.
\par 10 Εάν λάβη εις εαυτόν άλλην, δεν θέλει στερήσει την τροφήν αυτής, τα ενδύματα αυτής, και το προς αυτήν χρέος του γάμου.
\par 11 Εάν όμως δεν κάμνη εις αυτήν τα τρία ταύτα, τότε θέλει εξέλθει δωρεάν άνευ αργυρίου.
\par 12 Όστις πατάξη άνθρωπον, και αποθάνη, θέλει εξάπαντος θανατωθή·
\par 13 εάν όμως δεν παρεμόνευσεν, αλλ' ο Θεός παρέδωκεν αυτόν εις την χείρα αυτού, τότε εγώ θέλω σοι διορίσει τόπον, όπου θέλει καταφύγει·
\par 14 εάν δε τις εγερθή κατά του πλησίον αυτού διά να δολοφονήση αυτόν, από του θυσιαστηρίου μου θέλεις αποσπάσει αυτόν διά να θανατωθή.
\par 15 Και όστις πατάξη τον πατέρα αυτού ή την μητέρα αυτού, θέλει εξάπαντος θανατωθή.
\par 16 Και όστις κλέψη άνθρωπον και πωλήση αυτόν, ή εάν ευρεθή εις τας χείρας αυτού, θέλει εξάπαντος θανατωθή.
\par 17 Και όστις κακολογή τον πατέρα αυτού ή την μητέρα αυτού, θέλει εξάπαντος θανατωθή.
\par 18 Και εάν άνθρωποι λογομαχώσι μετ' αλλήλων και ο εις πατάξη τον άλλον με λίθον ή με γρόνθον, και δεν αποθάνη αλλά γείνη κλινήρης,
\par 19 εάν σηκωθή και περιπατήση έξω με την βακτηρίαν αυτού, τότε θέλει είσθαι ελεύθερος ο πατάξας· μόνον θέλει αποζημιώσει αυτόν διά την αργίαν αυτού και θέλει επιμεληθή την τελείαν θεραπείαν αυτού.
\par 20 Και εάν τις πατάξη τον δούλον αυτού ή την δούλην αυτού με ράβδον, και αποθάνη υπό τας χείρας αυτού, θέλει εξάπαντος τιμωρηθή.
\par 21 Αν όμως ζήση μίαν ημέραν ή δύο, δεν θέλει τιμωρηθή· διότι είναι αργύριον αυτού.
\par 22 Εάν μάχωνται άνδρες και πατάξωσι γυναίκα έγκυον και εξέλθη το παιδίον αυτής, δεν συμβή όμως συμφορά· θέλει εξάπαντος κάμει αποζημίωσιν ο πατάξας, οποίαν ο ανήρ της γυναικός επιβάλη εις αυτόν· και θέλει πληρώσει κατά την απόφασιν των κριτών.
\par 23 Αν όμως συμβή συμφορά, τότε θέλεις δώσει ζωήν αντί ζωής,
\par 24 οφθαλμόν αντί οφθαλμού, οδόντα αντί οδόντος, χείρα αντί χειρός, πόδα αντί ποδός,
\par 25 καύσιμον αντί καυσίματος, πληγήν αντί πληγής, κτύπημα αντί κτυπήματος.
\par 26 Εάν τις πατάξη τον οφθαλμόν του δούλου αυτού ή τον οφθαλμόν της δούλης αυτού και τυφλώση αυτόν, θέλει αφήσει αυτόν ελεύθερον εξ αιτίας του οφθαλμού αυτού.
\par 27 Και εάν εκβάλη τον οδόντα του δούλον αυτού ή τον οδόντα της δούλης αυτού, θέλει αφήσει αυτόν ελεύθερον εξ αιτίας του οδόντος αυτού.
\par 28 Εάν βους κερατίση άνδρα ή γυναίκα, και αποθάνη, τότε ο βους θέλει λιθοβοληθή με λίθους και δεν θέλει τρώγεσθαι το κρέας αυτού· ο κύριος δε του βοός θέλει είσθαι αθώος.
\par 29 Εάν όμως ο βους ήτο κερατιστής από πρότερον, και έγεινε διαμαρτυρία εις τον κύριον αυτού και δεν εφύλαξεν αυτόν, εάν θανατώση άνδρα ή γυναίκα, ο βους θέλει λιθοβοληθή και ακόμη ο κύριος αυτού θέλει θανατωθή.
\par 30 Εάν επιβληθή εις αυτόν τιμή εξαγοράσεως, θέλει δώσει διά την εξαγόρασιν της ζωής αυτού όσα ήθελον επιβληθή εις αυτόν.
\par 31 Είτε υιόν κερατίση, είτε θυγατέρα κερατίση, κατά την κρίσιν ταύτην θέλει γείνει εις αυτόν.
\par 32 Εάν ο βους κερατίση δούλον ή δούλην, θέλει δώσει εις τον κύριον αυτών τριάκοντα σίκλους αργυρίου· ο δε βους θέλει λιθοβοληθή.
\par 33 Και εάν τις ανοίξη λάκκον ή εάν τις σκάψη λάκκον και δεν σκεπάση αυτόν, και πέση εις αυτόν βους ή όνος,
\par 34 ο κύριος του λάκκου θέλει κάμει αποζημίωσιν, αργύριον θέλει αποδώσει εις τον κύριον αυτών· το δε θανατωθέν θέλει είσθαι αυτού.
\par 35 Και εάν ο βους τινός κερατίση τον βουν του πλησίον αυτού και θανατωθή, τότε θέλουσι πωλήσει τον ζώντα βουν, και θέλουσι μοιρασθή το αργύριον αυτού και τον θανατωθέντα ομοίως θέλουσι μοιρασθή.
\par 36 Εάν όμως ήναι γνωστόν ότι ο βους ήτο κερατιστής από πρότερον, και ο κύριος αυτού δεν εφύλαξεν αυτόν, θέλει εξάπαντος πληρώσει βουν αντί βοός· ο δε θανατωθείς θέλει είσθαι αυτού.

\chapter{22}

\par Εάν τις κλέψη βουν ή πρόβατον και σφάξη αυτό ή πωλήση αυτό, θέλει πληρώσει πέντε βόας αντί του βοός και τέσσαρα πρόβατα αντί του προβάτου.
\par 2 Εάν ο κλέπτης ευρεθή κάμνων ρήξιν και κτυπηθή και αποθάνη, δεν θέλει χυθή αίμα δι' αυτόν.
\par 3 Εάν όμως ο ήλιος ανατείλη επάνω αυτού, θέλει χυθή αίμα δι' αυτόν· πρέπει να κάμη ανταπόδοσιν· και αν δεν έχη, θέλει πωληθή διά την κλοπήν αυτού.
\par 4 Εάν το κλοπιμαίον ευρεθή εις τας χείρας αυτού ζων, είτε βους είτε όνος είτε πρόβατον, θέλει αποδώσει το διπλούν.
\par 5 Εάν τις καταβοσκήση αγρόν ή αμπελώνα και αφήση το κτήνος αυτού να βοσκηθή εν αγρώ ξένου ανθρώπου, θέλει κάμει ανταπόδοσιν εκ του καλητέρου του αγρού αυτού και εκ του καλητέρου του αμπελώνος αυτού.
\par 6 Εάν εξέλθη πυρ και εύρη ακάνθας, και καώσι θημωνίαι σίτου ή αστάχυα ιστάμενα ή αγρός, ο ανάψας το πυρ θέλει εξάπαντος κάμει ανταπόδοσιν.
\par 7 Εάν τις παραδώση εις τον πλησίον αυτού αργύριον ή σκεύη διά να φυλάττη αυτά, και κλαπώσιν εκ της οικίας του ανθρώπου, αν ευρεθή ο κλέπτης, θέλει αποδώσει το διπλούν·
\par 8 αν ο κλέπτης δεν ευρεθή, τότε ο κύριος της οικίας θέλει φερθή έμπροσθεν των κριτών, διά να εξετασθή αν δεν έβαλε την χείρα αυτού επί τα κτήματα του πλησίον αυτού.
\par 9 Περί παντός είδους αδικήματος, περί βοός, περί όνου, περί προβάτου, περί ενδύματος, περί παντός πράγματος χαμένου, το οποίον άλλος ήθελε διαφιλονεικεί ότι είναι αυτού, η κρίσις αμφοτέρων θέλει ελθεί έμπροσθεν των κριτών· και όντινα καταδικάσωσιν οι κριταί, εκείνος θέλει αποδώσει το διπλούν εις τον πλησίον αυτού.
\par 10 Εάν τις παραδώση εις τον πλησίον αυτού όνον ή βουν ή πρόβατον ή οποιονδήποτε κτήνος, διά να φυλάττη αυτό, και αποθάνη ή συντριφθή ή αρπαχθή χωρίς να ίδη τις,
\par 11 όρκος Θεού θέλει γείνει ανά μέσον αμφοτέρων αυτών, ότι δεν έβαλε την χείρα αυτού επί το κτήμα του πλησίον αυτού· και ο κύριος αυτού θέλει λάβει αυτό, ο δε άλλος δεν θέλει κάμει ανταπόδοσιν.
\par 12 Εάν όμως εκλέφθη παρ' αυτού, θέλει κάμει ανταπόδοσιν εις τον κύριον αυτού.
\par 13 Εάν έγεινε θηριάλωτον, θέλει φέρει αυτό διά μαρτυρίαν και δεν θέλει πληρώσει το θηριάλωτον.
\par 14 Και εάν τις δανεισθή ζώον παρά του πλησίον αυτού, και συντριφθή ή αποθάνη, ο δε κύριος αυτού δεν ήναι μετ' αυτού, θέλει εξάπαντος πληρώσει αυτό.
\par 15 Εάν όμως ο κύριος αυτού ήναι μετ' αυτού, δεν θέλει πληρώσει· αν ήτο μεμισθωμένον, ήλθε διά τον μισθόν αυτού.
\par 16 Και εάν τις απατήση παρθένον μη ηρραβωνισμένην, και κοιμηθή μετ' αυτής, θέλει εξάπαντος προικίσει αυτήν με προίκα διά γυναίκα εις εαυτόν.
\par 17 Εάν όμως ο πατήρ αυτής δεν στέργη να δώση αυτήν εις αυτόν, αργύριον θέλει πληρώσει κατά την προίκα των παρθένων.
\par 18 Μάγισσαν δεν θέλεις αφήσει να ζη.
\par 19 Όστις συνευρεθή με κτήνος, θέλει εξάπαντος θανατωθή.
\par 20 Ο θυσιάζων εις θεούς, εκτός εις μόνον τον Κύριον, θέλει εξολοθρευθή.
\par 21 Και ξένον δεν θέλεις κακοποιήσει ουδέ θέλεις καταδυναστεύσει αυτόν· διότι ξένοι εστάθητε εν τη γη της Αιγύπτου.
\par 22 Ουδεμίαν χήραν ή ορφανόν δεν θέλετε καταθλίψει.
\par 23 Εάν καταθλίψητε αυτούς οπωσδήποτε και βοήσωσι προς εμέ, θέλω εξάπαντος εισακούσει της φωνής αυτών,
\par 24 και ο θυμός μου θέλει εξαφθή και θέλω σας θανατώσει εν μαχαίρα· και αι γυναίκές σας θέλουσιν είσθαι χήραι και τα τέκνα σας ορφανά.
\par 25 Εάν δανείσης αργύριον εις τον πτωχόν γείτονά σου μεταξύ του λαού μου, δεν θέλεις φερθή προς αυτόν ως τοκιστής, δεν θέλεις επιβάλει επ' αυτόν τόκον.
\par 26 Εάν λάβης ενέχυρον το ένδυμα του πλησίον σου, θέλεις επιστρέψει αυτό προς αυτόν πριν δύση ο ήλιος·
\par 27 διότι τούτο μόνον είναι το σκέπασμα αυτού, τούτο το ένδυμα του δέρματος αυτού· με τι θέλει κοιμηθή; και όταν βοήση προς εμέ, θέλω εισακούσει· διότι εγώ είμαι ελεήμων.
\par 28 Δεν θέλεις κακολογήσει κριτάς, ουδέ θέλεις καταρασθή άρχοντα του λαού σου.
\par 29 Τας απαρχάς του αλωνίου σου και του ληνού σου δεν θέλεις καθυστερήσει· τον πρωτότοκόν σου εκ των υιών σου θέλεις δώσει εις εμέ·
\par 30 ομοίως θέλεις κάμει διά τον βουν σου και διά το πρόβατόν σου· επτά ημέρας θέλει είσθαι μετά της μητρός αυτού, την ογδόην ημέραν θέλεις δώσει αυτό εις εμέ.
\par 31 Και άνδρες άγιοι θέλετε είσθαι εις εμέ· και κρέας θηριάλωτον εν τω αγρώ δεν θέλετε φάγει· εις τον σκύλον θέλετε ρίψει αυτό.

\chapter{23}

\par Δεν θέλεις διαδώσει ψευδή φήμην· δεν θέλεις συμφωνήσει μετά του αδίκου διά να γείνης ψευδομάρτυς.
\par 2 Δεν θέλεις ακολουθήσει τους πολλούς επί κακώ· ουδέ θέλεις ομιλήσει εν κρισολογία, ώστε να κλίνης κατόπιν πολλών διά να διαστρέψης κρίσιν·
\par 3 ουδέ θέλεις αποβλέψει εις πρόσωπον πτωχού εν τη κρίσει αυτού.
\par 4 Εάν απαντήσης τον βουν του εχθρού σου ή τον όνον αυτού πλανώμενον, θέλεις εξάπαντος επιστρέψει αυτόν προς αυτόν.
\par 5 Εάν ίδης τον όνον του μισούντός σε πεπτωκότα υπό το φορτίον αυτού και ήθελες αποφύγει να βοηθήσης αυτόν, εξάπαντος θέλεις συμβοηθήσει αυτόν.
\par 6 Δεν θέλεις διαστρέψει το δίκαιον του πένητός σου εν τη κρίσει αυτού.
\par 7 Άπεχε από αδίκου υποθέσεως· και μη γείνης αιτία να θανατωθή ο αθώος και ο δίκαιος· διότι εγώ δεν θέλω δικαιώσει τον ασεβή.
\par 8 Και δώρα δεν θέλεις λάβει· διότι τα δώρα τυφλόνουσι και τους σοφούς, και διαστρέφουσι τους λόγους των δικαίων.
\par 9 Και ξένον δεν θέλεις καταδυναστεύσει διότι σεις γνωρίζετε την ψυχήν του ξένου, επειδή ξένοι εστάθητε εν τη γη της Αιγύπτου.
\par 10 Και εξ έτη θέλεις σπείρει την γην σου και θέλεις συνάγει τα γεννήματα αυτής·
\par 11 το δε έβδομον θέλεις αφήσει αυτήν να αναπαυθή και να μένη αργή, διά να τρώγωσιν οι πτωχοί του λαού σου· και το εναπολειφθέν αυτών ας τρώγωσι τα ζώα του αγρού. Ούτω θέλεις κάμει διά τον αμπελώνά σου και διά τον ελαιώνά σου.
\par 12 Εξ ημέρας θέλεις κάμνει τας εργασίας σου· την δε εβδόμην ημέραν θέλεις αναπαύεσθαι, διά να αναπαυθή ο βους σου και ο οίνος σου και να λάβη αναψυχήν ο υιός της δούλης σου και ο ξένος.
\par 13 Και εις πάντα όσα ελάλησα προς εσάς, θέλετε προσέξει· και όνομα άλλων θεών δεν θέλετε αναφέρει, ουδέ θέλει ακουσθή εκ του στόματός σου.
\par 14 Τρίς του ενιαυτού θέλεις κάμνει εορτήν εις εμέ.
\par 15 Θέλεις φυλάττει την εορτήν των αζύμων· επτά ημέρας θέλεις τρώγει άζυμα, καθώς προσέταξα εις σε, κατά τον διωρισμένον καιρόν του μηνός Αβίβ· διότι εν τούτω εξήλθες εξ Αιγύπτου· και ουδείς θέλει φανή ενώπιόν μου κενός·
\par 16 και την εορτήν του θερισμού, των πρωτογεννημάτων των κόπων σου, τα οποία έσπειρας εις τον αγρόν· και την εορτήν της συγκομιδής των καρπών, εις το τέλος του ενιαυτού, αφού συνάξης τους καρπούς σου εκ του αγρού.
\par 17 Τρίς του ενιαυτού θέλει εμφανίζεσθαι παν αρσενικόν σου ενώπιον Κυρίου του Θεού.
\par 18 Δεν θέλεις προσφέρει το αίμα της θυσίας μου με άρτον ένζυμον· ουδέ θέλει μένει το πάχος της εορτής μου έως πρωΐ.
\par 19 Τας απαρχάς των πρωτογεννημάτων της γης σου θέλεις φέρει εις τον οίκον Κυρίου του Θεού σου. Δεν θέλεις ψήσει ερίφιον εν τω γάλακτι της μητρός αυτού.
\par 20 Ιδού, εγώ αποστέλλω άγγελον έμπροσθέν σου διά να σε φυλάττη εν τη οδώ, και να σε φέρη εις τον τόπον τον οποίον προητοίμασα·
\par 21 φοβού αυτόν, και υπάκουε εις την φωνήν αυτού· μη παροργίσης αυτόν· διότι δεν θέλει συγχωρήσει τας παραβάσεις σας· επειδή το όνομά μου είναι εν αυτώ.
\par 22 Εάν όμως προσέχης να υπακούης εις την φωνήν αυτού και πράττης πάντα όσα λέγω, τότε εγώ θέλω είσθαι εχθρός των εχθρών σου και εναντίος των εναντίων σου.
\par 23 Διότι ο άγγελός μου θέλει προπορεύσθαι έμπροσθέν σου, και θέλει σε εισαγάγει εις τους Αμορραίους και Χετταίους και Φερεζαίους και Χαναναίους, Ευαίους και Ιεβουσαίους· και θέλω εξολοθρεύσει αυτούς.
\par 24 Δεν θέλεις προσκυνήσει τους θεούς αυτών, ουδέ θέλεις λατρεύσει αυτούς, ουδέ θέλεις πράξει κατά τα έργα εκείνων· αλλά θέλεις εξολοθρεύσει αυτούς, και θέλεις κατασυντρίψει τα είδωλα αυτών.
\par 25 Και θέλετε λατρεύει Κύριον τον Θεόν σας, και αυτός θέλει ευλογεί τον άρτον σου, και το ύδωρ σου· και θέλω απομακρύνει πάσαν νόσον εκ μέσου σου·
\par 26 και δεν θέλει είσθαι άγονος και στείρα επί της γης σου· τον αριθμόν των ημερών σου θέλω κάμει πλήρη.
\par 27 τον φόβον μου θέλει στείλει έμπροσθέν σου και θέλω καταστρέψει πάντα λαόν επί τον οποίον έρχεσαι και θέλω κάμει πάντας τους εχθρούς σου να στρέψωσι τα νώτα εις σέ·
\par 28 και θέλω στείλει έμπροσθέν σου σφήκας, και θέλουσιν εκδιώξει τους Ευαίους, τους Χαναναίους και τους Χετταίους απ' έμπροσθέν σου.
\par 29 Δεν θέλω εκδιώξει αυτούς απ' έμπροσθέν σου εις εν έτος, διά να μη γείνη έρημος η γη και πληθυνθώσι τα θηρία του αγρού εναντίον σου·
\par 30 ολίγον κατ' ολίγον θέλω εκδιώξει αυτούς απ' έμπροσθέν σου, εωσού αυξηθής και κυριεύσης την γην.
\par 31 Και θέλω θέσει τα όριά σου από της Ερυθράς θαλάσσης μέχρι της θαλάσσης των Φιλισταίων, και από της ερήμου μέχρι του ποταμού· διότι εις τας χείρας υμών θέλω παραδώσει τους κατοίκους του τόπου, και θέλεις εκδιώξει αυτούς απ' έμπροσθέν σου.
\par 32 Δεν θέλεις κάμει μετ' αυτών, ουδέ μετά των θεών αυτών, συνθήκην·
\par 33 δεν θέλουσι κατοικεί εν τη γη σου, διά να μη σε κάμωσι να αμαρτήσης εις εμέ· διότι αν λατρεύσης τους θεούς αυτών, τούτο θέλει εξάπαντος είσθαι παγίς εις σε.

\chapter{24}

\par Μετά ταύτα είπε προς τον Μωϋσήν, Ανάβα προς τον Κύριον, συ και Ααρών, Ναδάβ και Αβιούδ, και εβδομήκοντα εκ των πρεσβυτέρων του Ισραήλ, και προσκυνήσατε μακρόθεν·
\par 2 και ο Μωϋσής μόνος θέλει πλησιάσει προς τον Κύριον, αυτοί όμως δεν θέλουσι πλησιάσει ουδέ ο λαός θέλει αναβή μετ' αυτού.
\par 3 Και ήλθεν ο Μωϋσής και διηγήθη προς τον λαόν πάντας τους λόγους του Κυρίου και πάντα τα δικαιώματα αυτού· απεκρίθη δε πας ο λαός ομοφώνως και είπε, Πάντας τους λόγους, τους οποίους ελάλησεν ο Κύριος, θέλομεν κάμει.
\par 4 Και έγραψεν ο Μωϋσής πάντας τους λόγους του Κυρίου· και σηκωθείς ενωρίς το πρωΐ, ωκοδόμησε θυσιαστήριον υπό το όρος, και έστησε δώδεκα στήλας κατά τας δώδεκα φυλάς του Ισραήλ.
\par 5 Και απέστειλε τους νεανίσκους των υιών Ισραήλ, και προσέφεραν ολοκαυτώματα και εθυσίασαν θυσίας ειρηνικάς εις τον Κύριον, μοσχάρια.
\par 6 Λαβών δε ο Μωϋσής το ήμισυ του αίματος, έβαλεν εις λεκάνας· και το ήμισυ του αίματος ερράντισεν επί το θυσιαστήριον.
\par 7 Έπειτα λαβών το βιβλίον της διαθήκης, ανέγνωσεν εις τα ώτα του λαού· οι δε είπον, Πάντα όσα ελάλησεν ο Κύριος, θέλομεν κάμνει και θέλομεν υπακούει.
\par 8 Και λαβών ο Μωϋσής το αίμα, ερράντισεν επί τον λαόν, και είπεν, Ιδού, το αίμα της διαθήκης, την οποίαν ο Κύριος έκαμε προς εσάς κατά πάντας τούτους τους λόγους.
\par 9 Τότε ανέβη Μωϋσής και Ααρών, Ναδάβ και Αβιούδ και εβδομήκοντα εκ των πρεσβυτέρων του Ισραήλ·
\par 10 και είδον τον Θεόν του Ισραήλ· και υπό τους πόδας αυτού ως έδαφος εστρωμένον εκ λίθου σαπφείρου και ως το στερέωμα του ουρανού κατά την καθαρότητα·
\par 11 και επί τους εκλεκτούς των υιών Ισραήλ δεν έβαλε την χείρα αυτού· και είδον τον Θεόν, και έφαγον και έπιον.
\par 12 Και είπε Κύριος προς τον Μωϋσήν, Ανάβα προς εμέ εις το όρος και έσο εκεί· και θέλω σοι δώσει τας πλάκας τας λιθίνας, και τον νόμον, και τας εντολάς τας οποίας έγραψα, διά να διδάσκης αυτούς.
\par 13 Και εσηκώθη ο Μωϋσής μετά Ιησού του θεράποντος αυτού, και ανέβη ο Μωϋσής επί το όρος του Θεού.
\par 14 Προς δε τους πρεσβυτέρους είπε, Περιμένετε ημάς εδώ, εωσού επιστρέψωμεν προς εσάς· και ιδού, Ααρών και Ωρ είναι μεθ' υμών· εάν τις έχη υπόθεσιν, ας έρχηται προς αυτούς.
\par 15 Ο Μωϋσής λοιπόν ανέβη επί το όρος, και η νεφέλη εσκέπασε το όρος.
\par 16 Και εκάθησεν η δόξα του Κυρίου επί του όρους Σινά, και η νεφέλη εσκέπασεν αυτό εξ ημέρας· και την εβδόμην ημέραν εκάλεσεν ο Κύριος τον Μωϋσήν εκ μέσου της νεφέλης.
\par 17 Και η θέα της δόξης του Κυρίου ήτο, εις τους οφθαλμούς των υιών Ισραήλ, ως πυρ κατατρώγον επί της κορυφής του όρους.
\par 18 Και εισήλθεν ο Μωϋσής εις το μέσον της νεφέλης και ανέβη επί το όρος· και εστάθη ο Μωϋσής επί του όρους τεσσαράκοντα ημέρας και τεσσαράκοντα νύκτας.

\chapter{25}

\par Και ελάλησε Κύριος προς τον Μωϋσήν, λέγων,
\par 2 Ειπέ προς τους υιούς Ισραήλ να φέρωσι προς εμέ προσφοράν· παρά παντός ανθρώπου προαιρουμένου εν τη καρδία αυτού θέλετε λάβει την προσφοράν μου.
\par 3 Και αύτη είναι η προσφορά, την οποίαν θέλετε λάβει παρ' αυτών· χρυσίον και αργύριον και χαλκός,
\par 4 και κυανούν και πορφυρούν και κόκκινον και βύσσος και τρίχες αιγών,
\par 5 και δέρματα κριών κοκκινοβαφή και δέρματα θώων και ξύλον σιττίμ,
\par 6 έλαιον διά το φως, αρώματα διά το έλαιον του χρίσματος και διά το ευώδες θυμίαμα,
\par 7 λίθοι ονυχίται και λίθοι διά να εντεθώσιν εις το εφόδ και εις το περιστήθιον.
\par 8 Και ας κάμωσιν εις εμέ αγιαστήριον, διά να κατοικώ μεταξύ αυτών.
\par 9 Κατά πάντα όσα εγώ δεικνύω προς σε, κατά το παράδειγμα της σκηνής, και κατά το παράδειγμα πάντων των σκευών αυτής, ούτω θέλετε κάμει.
\par 10 Και θέλουσι κατασκευάσει κιβωτόν εκ ξύλου σιττίμ· δύο πηχών και ημισείας το μήκος αυτής, και μιας πήχης και ημισείας το πλάτος αυτής, και μιας πήχης και ημισείας το ύψος αυτής·
\par 11 και θέλεις περικαλύψει αυτήν με καθαρόν χρυσίον, έσωθεν και έξωθεν θέλεις περικαλύψει αυτήν, και επ' αυτής θέλεις κάμει χρυσήν στεφάνην κύκλω.
\par 12 Και θέλεις χύσει δι' αυτήν τέσσαρας κρίκους χρυσούς και θέλεις βάλει αυτούς εις τας τέσσαρας γωνίας αυτής· δύο μεν κρίκους εις την μίαν πλευράν αυτής, δύο δε κρίκους εις την άλλην πλευράν αυτής.
\par 13 Και θέλεις κάμει μοχλούς εκ ξύλου σιττίμ, και θέλεις περικαλύψει αυτούς με χρυσίον·
\par 14 και θέλεις εισάξει τους μοχλούς εις τους κρίκους των πλευρών της κιβωτού, διά να βαστάζηται η κιβωτός δι' αυτών·
\par 15 εν τοις κρίκοις της κιβωτού θέλουσι μένει οι μοχλοί· δεν θέλουσι μετακινείσθαι απ' αυτής.
\par 16 και θέλεις θέσει εν τη κιβωτώ τα μαρτύρια τα οποία θέλω δώσει εις σε.
\par 17 Και θέλεις κάμει ιλαστήριον εκ χρυσίου καθαρού· δύο πηχών και ημισείας το μήκος αυτού, και μιας πήχης και ημισείας το πλάτος αυτού.
\par 18 Και θέλεις κάμει δύο χερουβείμ εκ χρυσίου· σφυρήλατα θέλεις κάμει αυτά, επί των δύο άκρων του ιλαστηρίου·
\par 19 και κάμε εν χερούβ επί του ενός άκρου, και εν χερούβ επί του άλλου άκρου· εκ του ιλαστηρίου θέλεις κάμει τα χερουβείμ επί των δύο άκρων αυτού·
\par 20 και θέλουσιν εκτείνει τα χερουβείμ επάνωθεν τας πτέρυγας, επικαλύπτοντα με τας πτέρυγας αυτών το ιλαστήριον· και τα πρόσωπα αυτών θέλουσι βλέπει το εν προς το άλλο· προς το ιλαστήριον θέλουσιν είσθαι τα πρόσωπα των χερουβείμ.
\par 21 Και θέλεις επιθέσει το ιλαστήριον επί της κιβωτού άνωθεν· και θέλεις θέσει εν τη κιβωτώ τα μαρτύρια, τα οποία θέλω δώσει εις σέ·
\par 22 και εκεί θέλω γνωρισθή προς σέ· και επάνωθεν του ιλαστηρίου, εκ του μέσου των δύο χερουβείμ, των επί της κιβωτού του μαρτυρίου, θέλω λαλήσει προς σε περί πάντων όσα θέλω προστάξει εις σε να είπης προς τους υιούς Ισραήλ.
\par 23 Και θέλεις κάμει τράπεζαν εκ ξύλου σιττίμ· δύο πηχών το μήκος αυτής, και μιας πήχης το πλάτος αυτής, το δε ύψος αυτής μιας πήχης και ημισείας·
\par 24 και θέλεις περικαλύψει αυτήν με χρυσίον καθαρόν, και θέλεις κάμει εις αυτήν χρυσήν στεφάνην κύκλω.
\par 25 Και θέλεις κάμει εις αυτήν χείλος κύκλω μιας παλάμης το πλάτος και θέλεις κάμει επί το χείλος αυτής στεφάνην χρυσήν κύκλω.
\par 26 Και θέλεις κάμει εις αυτήν τέσσαρας κρίκους χρυσούς και θέλεις βάλει τους κρίκους επί τας τέσσαρας γωνίας, τας επί των τεσσάρων ποδών αυτής·
\par 27 οι κρίκοι θέλουσιν είσθαι υπό το χείλος θήκαι των μοχλών, διά να βαστάζηται η τράπεζα.
\par 28 Και θέλεις κάμει τους μοχλούς εκ ξύλου σιττίμ, και θέλεις περικαλύψει αυτούς με χρυσίον, διά να βαστάζηται η τράπεζα δι' αυτών.
\par 29 Και θέλεις κάμει τους δίσκους αυτής και τους θυμιαματοδόχους αυτής και τα σπονδεία αυτής και τας λεκάνας αυτής, διά να γίνωνται δι' αυτών αι σπονδαί· εκ χρυσίου καθαρού θέλεις κάμει αυτά.
\par 30 Και θέλεις θέσει επί της τραπέζης άρτους προθέσεως ενώπιόν μου διαπαντός.
\par 31 Και θέλεις κάμει λυχνίαν εκ χρυσίου καθαρού· σφυρήλατον θέλεις κάμει την λυχνίαν· ο κορμός αυτής και οι κλάδοι αυτής, αι λεκάναι αυτής, οι κόμβοι αυτής και τα άνθη αυτής, θέλουσιν είσθαι εν σώμα μετ' αυτής.
\par 32 Και θέλουσιν εξέρχεσθαι εξ κλάδοι εκ των πλαγίων αυτής· τρεις κλάδοι της λυχνίας εκ του ενός πλαγίου, και τρεις κλάδοι της λυχνίας εκ του άλλου πλαγίου·
\par 33 εις τον ένα κλάδον θέλουσιν είσθαι τρεις λεκάναι αμυγδαλοειδείς, εις κόμβος και εν άνθος· και εις τον άλλον κλάδον τρεις λεκάναι αμυγδαλοειδείς, εις κόμβος και εν άνθος· ούτω θέλει γείνει εις τους εξ κλάδους, τους εξερχομένους εκ της λυχνίας.
\par 34 Και εις την λυχνίαν θέλουσιν είσθαι τέσσαρες λεκάναι αμυγδαλοειδείς, οι κόμβοι αυτών και τα άνθη αυτών.
\par 35 Και θέλει είσθαι εις κόμβος υπό τους δύο κλάδους εξ αυτής, και εις κόμβος υπό τους δύο κλάδους εξ αυτής, και εις κόμβος υπό τους δύο κλάδους εξ αυτής, εις τους εξ κλάδους τους εξερχομένους εκ της λυχνίας.
\par 36 Οι κόμβοι αυτών και οι κλάδοι αυτών θέλουσιν είσθαι εν σώμα μετ' αυτής· το όλον αυτής εν σφυρήλατον εκ χρυσίου καθαρού.
\par 37 Και θέλεις κάμει τους λύχνους αυτής επτά· και θέλουσιν ανάπτει τους λύχνους αυτής, διά να φέγγωσιν έμπροσθεν αυτής,
\par 38 Και τα λυχνοψάλιδα αυτής και τα υποθέματα αυτής θέλουσιν είσθαι εκ χρυσίου καθαρού.
\par 39 Εξ ενός ταλάντου χρυσίου καθαρού θέλει κατασκευασθή αυτή και πάντα ταύτα τα σκεύη.
\par 40 Και πρόσεχε να κάμης κατά τον τύπον αυτών τον δειχθέντα εις σε επί του όρους.

\chapter{26}

\par Και θέλεις κάμει την σκηνήν, δέκα παραπετάσματα εκ βύσσου κεκλωσμένης και κυανού και πορφυρού και κοκκίνου· με χερουβείμ εντέχνως ενειργασμένα θέλεις κάμει αυτά.
\par 2 Το μήκος του ενός παραπετάσματος εικοσιοκτώ πηχών, και το πλάτος του ενός παραπετάσματος τεσσάρων πηχών· πάντα τα παραπετάσματα του αυτού μέτρου.
\par 3 Τα πέντε παραπετάσματα θέλουσι συνάπτεσθαι το εν μετά του άλλου· και τα άλλα πέντε παραπετάσματα θέλουσι συνάπτεσθαι το εν μετά του άλλου.
\par 4 Και θέλεις κάμει θηλυκωτήρια κυανά επί της άκρας του πρώτου παραπετάσματος, κατά το πλάγιον όπου γίνεται η ένωσις· ομοίως θέλεις κάμει και επί της τελευταίας άκρας του δευτέρου παραπετάσματος, όπου γίνεται η ένωσις του δευτέρου·
\par 5 πεντήκοντα θηλυκωτήρια θέλεις κάμει εις το εν παραπέτασμα, και πεντήκοντα θηλυκωτήρια θέλεις κάμει εις την άκραν του παραπετάσματος την κατά την ένωσιν του δευτέρου, διά να αντικρύζωσι τα θηλυκωτήρια προς άλληλα.
\par 6 Και θέλεις κάμει πεντήκοντα περόνας χρυσάς, και με τας περόνας θέλεις συνάψει τα παραπετάσματα προς άλληλα· ούτως η σκηνή θέλει είσθαι μία.
\par 7 Και θέλεις κάμει παραπετάσματα εκ τριχών αιγών, διά να ήναι κάλυμμα επί της σκηνής· ένδεκα θέλεις κάμει τα παραπετάσματα ταύτα·
\par 8 το μήκος του ενός παραπετάσματος τριάκοντα πηχών, και το πλάτος του ενός παραπετάσματος τεσσάρων πηχών· του αυτού μέτρου θέλουσιν είσθαι τα ένδεκα παραπετάσματα.
\par 9 Και θέλεις συνάψει τα πέντε παραπετάσματα χωριστά, και τα εξ παραπετάσματα χωριστά· το έκτον όμως παραπέτασμα θέλεις επιδιπλώσει κατά το πρόσωπον της σκηνής.
\par 10 Και θέλεις κάμει πεντήκοντα θηλυκωτήρια επί της άκρας του ενός παραπετάσματος του τελευταίου κατά την ένωσιν, και πεντήκοντα θηλυκωτήρια επί της άκρας του παραπετάσματος, το οποίον ενόνεται με το δεύτερον.
\par 11 Θέλεις κάμει και πεντήκοντα περόνας χαλκίνας, και θέλεις εμβάλει τας περόνας εις τα θηλυκωτήρια, και θέλεις συνάψει την σκηνήν, ώστε να ήναι μία.
\par 12 Το δε υπόλοιπον, το περισσεύον εκ των παραπετασμάτων της σκηνής, το ήμισυ του παραπετάσματος του εναπολειπομένου, θέλει κρέμασθαι επί τα όπισθεν της σκηνής.
\par 13 Και μία πήχη εκ του ενός πλαγίου και μία πήχη εκ του άλλου πλαγίου εκ του εναπολειπομένου εις το μήκος των παραπετασμάτων της σκηνής θέλει κρέμασθαι επάνωθεν επί τα πλάγια της σκηνής εντεύθεν και εντεύθεν, διά να καλύπτη αυτήν.
\par 14 Και θέλεις κάμει κατακάλυμμα διά την σκηνήν εκ δερμάτων κριών κοκκινοβαφών και επικάλυμμα υπεράνωθεν εκ δερμάτων θώων.
\par 15 Και θέλεις κάμει διά την σκηνήν σανίδας εκ ξύλου σιττίμ ορθίας·
\par 16 δέκα πηχών το μήκος της μιας σανίδος, και μιας πήχης και ημισείας το πλάτος της μιας σανίδος.
\par 17 Δύο αγκωνίσκοι θέλουσιν είσθαι εις την μίαν σανίδα αντικρύζοντες προς αλλήλους· ούτω θέλεις κάμει εις πάσας τας σανίδας της σκηνής.
\par 18 Και θέλεις κάμει τας σανίδας διά την σκηνήν, είκοσι σανίδας από το νότιον μέρος προς μεσημβρίαν.
\par 19 και υποκάτω των είκοσι σανίδων θέλεις κάμει τεσσαράκοντα υποβάσια αργυρά· δύο υποβάσια υποκάτω της μιας σανίδος διά τους δύο αγκωνίσκους αυτής, και δύο υποβάσια υποκάτω της άλλης σανίδος διά τους δύο αγκωνίσκους αυτής.
\par 20 Και διά το δεύτερον μέρος της σκηνής το προς βορράν, θέλεις κάμει είκοσι σανίδας.
\par 21 και τα τεσσαράκοντα αυτών υποβάσια αργυρά, δύο υποβάσια υποκάτω της μιας σανίδος, και δύο υποβάσια υποκάτω της άλλης σανίδος.
\par 22 Και διά τα όπισθεν μέρη της σκηνής τα προς δυσμάς θέλεις κάμει εξ σανίδας.
\par 23 Θέλεις κάμει και δύο σανίδας διά τας γωνίας της σκηνής εις τα όπισθεν μέρη·
\par 24 και θέλουσιν ενωθή κάτωθεν και θέλουσιν ενωθή ομού άνωθεν δι' ενός κρίκου· ούτω θέλει είσθαι δι' αυτάς αμφοτέρας· διά τας δύο γωνίας θέλουσιν είσθαι.
\par 25 και θέλουσιν είσθαι οκτώ σανίδες και τα αργυρά υποβάσια αυτών, δεκαέξ υποβάσια· δύο υποβάσια υποκάτω της μιας σανίδος και δύο υποβάσια υποκάτω της άλλης σανίδος.
\par 26 Και θέλεις κάμει μοχλούς εκ ξύλου σιττίμ· πέντε διά τας σανίδας του ενός μέρους της σκηνής,
\par 27 και πέντε μοχλούς διά τας σανίδας του άλλου μέρους της σκηνής, και πέντε μοχλούς διά τας σανίδας του μέρους της σκηνής διά το πλάγιον το προς δυσμάς.
\par 28 και ο μέσος μοχλός, ο εν τω μέσω των σανίδων, θέλει διαπερά απ' άκρου έως άκρου.
\par 29 Και τας σανίδας θέλεις περικαλύψει με χρυσίον και τους κρίκους αυτών θέλεις κάμει χρυσούς, διά να ήναι θήκαι των μοχλών. και θέλεις περικαλύψει τους μοχλούς με χρυσίον.
\par 30 Και θέλεις ανεγείρει την σκηνήν κατά το σχέδιον αυτής το δειχθέν εις σε επί του όρους.
\par 31 Και θέλεις κάμει καταπέτασμα εκ κυανού και πορφυρού και κοκκίνου και βύσσου κεκλωσμένης, εντέχνου εργασίας· με χερουβείμ θέλει είσθαι κατεσκευασμένον.
\par 32 Και θέλεις κρεμάσει αυτό επί τεσσάρων στύλων εκ σιττίμ περικεκαλυμμένων με χρυσίον· τα άγκιστρα αυτών θέλουσιν είσθαι χρυσά, επί των τεσσάρων αργυρών υποβασίων.
\par 33 Και θέλεις κρεμάσει το καταπέτασμα υπό τας περόνας, διά να φέρης εκεί, έσωθεν του καταπετάσματος, την κιβωτόν του μαρτυρίου· και το καταπέτασμα θέλει κάμνει εις εσάς χώρισμα μεταξύ του αγίου και του αγίου των αγίων.
\par 34 Και θέλεις επιθέσει το ιλαστήριον επί της κιβωτού του μαρτυρίου εν τω αγίω των αγίων.
\par 35 Και θέλεις θέσει την τράπεζαν έξωθεν του καταπετάσματος και την λυχνίαν αντικρύ της τραπέζης προς το νότιον μέρος της σκηνής· την δε τράπεζαν θέλεις θέσει προς το βόρειον μέρος.
\par 36 Και θέλεις κάμει διά την θύραν της σκηνής τάπητα εκ κυανού και πορφυρού και κοκκίνου και βύσσου κεκλωσμένης, κατεσκευασμένον με εργασίαν κεντητού.
\par 37 Και θέλεις κάμει διά τον τάπητα πέντε στύλους εκ σιττίμ, και θέλεις περικαλύψει αυτούς με χρυσίον· τα άγκιστρα αυτών θέλουσιν είσθαι χρυσά· και θέλεις χύσει δι' αυτούς πέντε υποβάσια χάλκινα.

\chapter{27}

\par Και θέλεις κάμει θυσιαστήριον εκ ξύλου σιττίμ, πέντε πηχών το μήκος και πέντε πηχών το πλάτος· τετράγωνον θέλει είσθαι το θυσιαστήριον· και το ύψος αυτού τριών πηχών·
\par 2 και θέλεις κάμει τα κέρατα αυτού επί των τεσσάρων γωνιών αυτού· τα κέρατα αυτού θέλουσιν είσθαι εκ του αυτού και θέλεις περικαλύψει αυτό με χαλκόν.
\par 3 Και θέλεις κάμει τους στακτοδόχους λέβητας αυτού και τα πτυάρια αυτού και τας λεκάνας αυτού και τας κρεάγρας αυτού και τα πυροδόχα αυτού· χάλκινα θέλεις κάμει πάντα τα σκεύη αυτού.
\par 4 Και θέλεις κάμει δι' αυτό χαλκίνην εσχάραν δικτυωτής εργασίας· και επί του δικτύου θέλεις κάμει τέσσαρας κρίκους χαλκίνους επί των τεσσάρων γωνιών αυτού.
\par 5 Και θέλεις θέσει αυτήν υπό την περιοχήν του θυσιαστηρίου κάτωθεν, ώστε το δίκτυον να ήναι μέχρι του μέσου του θυσιαστηρίου.
\par 6 Και θέλεις κάμει μοχλούς διά το θυσιαστήριον, μοχλούς εκ ξύλου σιττίμ, και θέλεις περικαλύψει αυτούς με χαλκόν·
\par 7 και οι μοχλοί θέλουσι τεθή εντός των κρίκων και θέλουσιν είσθαι οι μοχλοί επί των δύο πλευρών του θυσιαστηρίου, διά να βαστάζωσιν αυτό.
\par 8 Κοίλον σανιδωτόν θέλεις κάμει αυτό, καθώς εδείχθη εις σε επί του όρους· ούτω θέλουσι κάμει.
\par 9 Και θέλεις κάμει την αυλήν της σκηνής· από το νότιον μέρος προς μεσημβρίαν θέλουσιν είσθαι παραπετάσματα διά την αυλήν εκ βύσσου κεκλωσμένης, το μήκος εκατόν πηχών διά το εν πλευρόν.
\par 10 Και οι είκοσι στύλοι αυτής και τα είκοσι υποβάσια τούτων θέλουσιν είσθαι χάλκινα· τα άγκιστρα των στύλων και αι ζώναι αυτών αργυρά.
\par 11 Και ομοίως κατά το βόρειον πλευρόν κατά μήκος θέλουσιν είσθαι παραπετάσματα, μήκος εκατόν πηχών, και οι είκοσι στύλοι αυτών και τα είκοσι αυτών χάλκινα υποβάσια· τα δε άγκιστρα των στύλων και αι ζώναι αυτών αργυρά.
\par 12 Και διά το πλάτος της αυλής κατά το δυτικόν πλευρόν θέλουσιν είσθαι παραπετάσματα πεντήκοντα πηχών· στύλοι αυτών δέκα και υποβάσια αυτών δέκα.
\par 13 Και το πλάτος της αυλής κατά το ανατολικόν πλευρόν το προς ανατολάς θέλει είσθαι πεντήκοντα πηχών.
\par 14 Και τα παραπετάσματα του ενός μέρους της πύλης θέλουσιν είσθαι δεκαπέντε πηχών· στύλοι αυτών τρεις και υποβάσια αυτών τρία.
\par 15 Και εις το άλλο μέρος θέλουσιν είσθαι παραπετάσματα δεκαπέντε πηχών· στύλοι αυτών τρεις και υποβάσια αυτών τρία.
\par 16 Διά δε την πύλην της αυλής θέλει είσθαι καταπέτασμα είκοσι πηχών, εκ κυανού και πορφυρού και κοκκίνου και βύσσου κεκλωσμένης, κατεσκευασμένον με εργασίαν κεντητού· στύλοι αυτών τέσσαρες και υποβάσια τούτων τέσσαρα.
\par 17 Πάντες οι στύλοι κύκλω της αυλής θέλουσιν είσθαι εζωσμένοι με άργυρον, τα άγκιστρα αυτών αργυρά και τα υποβάσια αυτών χάλκινα.
\par 18 Το μήκος της αυλής θέλει είσθαι εκατόν πηχών και το πλάτος εκατέρωθεν πεντήκοντα και το ύψος πέντε πηχών, εκ βύσσου κεκλωσμένης, και τα υποβάσια αυτών χάλκινα.
\par 19 Πάντα τα σκεύη της σκηνής διά πάσαν την υπηρεσίαν αυτής και πάντες οι πάσσαλοι αυτής και πάντες οι πάσσαλοι της αυλής θέλουσιν είσθαι χάλκινοι.
\par 20 Και συ πρόσταξον τους υιούς Ισραήλ να φέρωσι προς σε καθαρόν έλαιον από ελαίας κοπανισμένας διά το φως, διά να καίη πάντοτε ο λύχνος.
\par 21 Εν τη σκηνή του μαρτυρίου έξωθεν του καταπετάσματος, το οποίον είναι έμπροσθεν του μαρτυρίου, ο Ααρών και οι υιοί αυτού θέλουσι διαθέσει αυτόν αφ' εσπέρας έως πρωΐ έμπροσθεν του Κυρίου· τούτο θέλει είσθαι νόμος παντοτεινός εις τους υιούς Ισραήλ κατά τας γενεάς αυτών.

\chapter{28}

\par Και συ προσάγαγε προς σεαυτόν Ααρών τον αδελφόν σου και τους υιούς αυτού μετ' αυτού, εκ μέσου των υιών Ισραήλ, διά να ιερατεύωσιν εις εμέ, Ααρών, Ναδάβ και Αβιούδ, Ελεάζαρ και Ιθάμαρ, τους υιούς του Ααρών.
\par 2 Και θέλεις κάμει στολήν αγίαν εις τον Ααρών τον αδελφόν σου προς δόξαν και τιμήν.
\par 3 Και συ λάλησον προς πάντας τους σοφούς την καρδίαν, τους οποίους εγώ ενέπλησα από πνεύματος σοφίας, να κάμωσι την στολήν του Ααρών, διά να καθιερώσης αυτόν, ώστε να ιερατεύη εις εμέ.
\par 4 Και αύτη είναι η στολή την οποίαν θέλουσι κάμει· περιστήθιον και εφόδ και ποδήρης και χιτών κεντητός, μίτρα και ζώνη· και θέλουσι κάμει στολάς αγίας εις τον Ααρών τον αδελφόν σου, και εις τους υιούς αυτού, διά να ιερατεύωσιν εις εμέ.
\par 5 Και αυτοί θέλουσι λάβει το χρυσίον και το κυανούν και το πορφυρούν και το κόκκινον και την βύσσον.
\par 6 Και θέλουσι κάμει το εφόδ εκ χρυσίου, εκ κυανού και πορφυρού, εκ κοκκίνου και βύσσου κεκλωσμένης, εντέχνου εργασίας·
\par 7 θέλει έχει τας δύο επωμίδας αυτού συναπτάς κατά τα δύο άκρα αυτού, ώστε να συνάπτωνται.
\par 8 Και η κεντητή ζώνη του εφόδ, η επ' αυτό, θέλει είσθαι εκ του αυτού, κατά την εργασίαν αυτού· εκ χρυσίου, εκ κυανού και πορφυρού και κοκκίνου και βύσσου κεκλωσμένης.
\par 9 Και θέλεις λάβει δύο ονυχίτας λίθους, και θέλεις εγχαράξει επ' αυτούς τα ονόματα των υιών Ισραήλ·
\par 10 εξ εκ των ονομάτων αυτών επί του ενός λίθου και τα λοιπά εξ ονόματα επί του άλλου λίθου, κατά τας γενέσεις αυτών·
\par 11 με εργασίαν λιθογλύφου κατά την γλυφήν της σφραγίδος, θέλεις εγχαράξει τους δύο λίθους με τα ονόματα των υιών Ισραήλ· θέλεις εναρμόσει αυτούς εις χρυσούς οικίσκους.
\par 12 Και θέλεις θέσει τους δύο λίθους επί των επωμίδων του εφόδ, λίθους μνημοσύνης εις τους υιούς Ισραήλ· και ο Ααρών θέλει βαστάζει τα ονόματα αυτών ενώπιον του Κυρίου επί των δύο ώμων αυτού εις μνημόσυνον.
\par 13 Και θέλεις κάμει οικίσκους χρυσούς·
\par 14 και δύο αλύσεις εκ καθαρού χρυσίου επί των άκρων· εργασίαν πλεκτήν θέλεις κάμει αυτάς, και θέλεις συνάψει τας πλεκτάς αλύσεις με τους οικίσκους.
\par 15 Και θέλεις κάμει το περιστήθιον της κρίσεως εντέχνου εργασίας· κατά την εργασίαν του εφόδ θέλεις κάμει αυτό· εκ χρυσίου, κυανού, και πορφυρού και κοκκίνου και βύσσου κεκλωσμένης θέλεις κάμει αυτό·
\par 16 τετράγωνον θέλει είσθαι διπλούν· μιας σπιθαμής το μήκος αυτού και μιας σπιθαμής το πλάτος αυτού.
\par 17 Και θέλεις εναρμόσει εις αυτό λίθους, τέσσαρας σειράς λίθων· σειρά σαρδίου, τοπαζίου και σμαράγδου θέλει είσθαι πρώτη σειρά·
\par 18 και η δευτέρα σειρά, άνθραξ, σάπφειρος και αδάμας·
\par 19 και η τρίτη σειρά, λιγύριον, αχάτης και αμέθυστος·
\par 20 και η τετάρτη σειρά, βηρύλλιον και όνυξ και ίασπις· ενηρμοσμένοι θέλουσιν είσθαι εις τους χρυσούς οικίσκους αυτών·
\par 21 και οι λίθοι θέλουσιν είσθαι με τα ονόματα των υιών Ισραήλ, δώδεκα, κατά τα ονόματα αυτών, κατά την γλυφήν της σφραγίδος· έκαστος με το όνομα αυτού θέλουσιν είσθαι κατά τας δώδεκα φυλάς
\par 22 Και θέλεις κάμει επί το περιστήθιον αλύσεις κατά τα άκρα, πλεκτής εργασίας εκ χρυσίου καθαρού.
\par 23 Και θέλεις κάμει επί το περιστήθιον δύο κρίκους χρυσούς, και θέλεις περάσει τους δύο κρίκους εις τα δύο άκρα του περιστηθίου.
\par 24 Και θέλεις περάσει τας δύο πλεκτάς αλύσεις χρυσάς εις τους δύο κρίκους, τους εις τα άκρα του περιστηθίου.
\par 25 Και τα άλλα δύο άκρα των δύο πλεκτών αλύσεων θέλεις συνάψει με τους δύο οικίσκους και θέλεις βάλει αυτούς εις τας επωμίδας του εφόδ έμπροσθεν αυτού.
\par 26 Και θέλεις κάμει δύο κρίκους χρυσούς και θέλεις βάλει αυτούς επί των δύο άκρων του περιστηθίου εις το χείλος αυτού, το οποίον είναι κατά το μέρος του εφόδ έσωθεν·
\par 27 και θέλεις κάμει δύο άλλους κρίκους χρυσούς, και θέλεις βάλει αυτούς εις τα δύο πλάγια του εφόδ κάτωθεν, προς το εμπροσθινόν μέρος αυτού, αντικρύ της άλλης ενώσεως αυτού, άνωθεν της κεντητής ζώνης του εφόδ.
\par 28 Και θέλουσι δένει το περιστήθιον διά των κρίκων αυτού εις τους κρίκους του εφόδ με ταινίαν εκ κυανού, διά να ήναι άνωθεν της κεντητής ζώνης του εφόδ και διά να μη ήναι το περιστήθιον κεχωρισμένον από του εφόδ.
\par 29 Και ο Ααρών θέλει βαστάζει τα ονόματα των υιών Ισραήλ εν τω περιστηθίω της κρίσεως επί της καρδίας αυτού, όταν εισέρχηται εις το άγιον, εις μνημόσυνον ενώπιον του Κυρίου διαπαντός.
\par 30 Και θέλεις βάλει εις το περιστήθιον της κρίσεως το Ουρίμ και το Θουμμίμ, και θέλουσιν είσθαι επί της καρδίας του Ααρών, όταν εισέρχηται ενώπιον του Κυρίου· και ο Ααρών θέλει βαστάζει την κρίσιν των υιών Ισραήλ επί της καρδίας αυτού ενώπιον του Κυρίου διαπαντός.
\par 31 Και θέλεις κάμει τον ποδήρη του εφόδ όλον εκ κυανού.
\par 32 Και θέλει είσθαι εις την κορυφήν αυτού άνοιγμα κατά το μέσον αυτού· θέλει έχει ταινίαν υφαντήν κύκλω του ανοίγματος αυτού, καθώς είναι το άνοιγμα του θώρακος, διά να μη σχίζηται.
\par 33 Και θέλεις κάμει επί των κρασπέδων αυτού ρόδια εκ κυανού και πορφυρού και κοκκίνου επί των κρασπέδων αυτού κύκλω· και κώδωνας χρυσούς μεταξύ αυτών κύκλω·
\par 34 χρυσούν κώδωνα και ρόδιον, χρυσούν κώδωνα και ρόδιον, επί των κρασπέδων του ποδήρους κύκλω.
\par 35 Και θέλει είσθαι επί του Ααρών διά να λειτουργή· και ο ήχος αυτού θέλει είσθαι ακουστός, όταν εισέρχηται εις το άγιον ενώπιον του Κυρίου και όταν εξέρχηται, διά να μη αποθάνη.
\par 36 Και θέλεις κάμει πέταλον εκ χρυσίου καθαρού και θέλεις εγχαράξει επ' αυτό, ως χάραγμα σφραγίδος, ΑΓΙΑΣΜΟΣ ΕΙΣ ΤΟΝ ΚΥΡΙΟΝ.
\par 37 Και θέλεις βάλει αυτό επί κυανής ταινίας, διά να ήναι επί της μίτρας· εις το έμπροσθεν μέρος της μίτρας θέλει είσθαι.
\par 38 Και θέλει είσθαι επί του μετώπου του Ααρών, διά να σηκόνη ο Ααρών την ανομίαν των αγίων πραγμάτων, τα οποία οι υιοί του Ισραήλ θέλουσιν αγιάζει εις πάσας αυτών τας αγίας προσφοράς· και θέλει είσθαι διαπαντός επί του μετώπου αυτού, διά να ήναι δεκταί ενώπιον του Κυρίου.
\par 39 Και θέλεις υφάνει τον χιτώνα εκ βύσσου και θέλεις κάμει μίτραν εκ βύσσου και θέλεις κάμει ζώνην εργασίας κεντητού.
\par 40 Και διά τους υιούς του Ααρών θέλεις κάμει χιτώνας και θέλεις κάμει δι' αυτούς ζώνας και μιτρίδια θέλεις κάμει δι' αυτούς προς δόξαν και τιμήν.
\par 41 Και θέλεις ενδύσει αυτά τον Ααρών τον αδελφόν σου και τους υιούς αυτού μετ' αυτού, και θέλεις χρίσει αυτούς και θέλεις καθιερώσει αυτούς και αγιάσει αυτούς, διά να ιερατεύωσιν εις εμέ.
\par 42 Και θέλεις κάμει εις αυτούς λινά περισκελή, διά να σκεπάζωσι την γύμνωσιν της σαρκός αυτών· από της οσφύος μέχρι των μηρών θέλουσι φθάνει·
\par 43 και θέλουσιν είσθαι επί του Ααρών και επί των υιών αυτού, όταν εισέρχωνται εις την σκηνήν του μαρτυρίου ή όταν πλησιάζωσιν εις το θυσιαστήριον διά να λειτουργήσωσιν εν τω αγίω, διά να μη φέρωσιν εφ' εαυτούς ανομίαν και αποθάνωσι τούτο θέλει είσθαι νόμος παντοτεινός εις αυτόν και εις το σπέρμα αυτού μετ' αυτόν.

\chapter{29}

\par Και τούτο είναι το πράγμα, το οποίον θέλεις κάμει εις αυτούς διά να αγιάσης αυτούς, ώστε να ιερατεύωσιν εις εμέ. Λάβε εν μοσχάριον βοός και δύο κριούς αμώμους,
\par 2 και άζυμον άρτον και πήττας αζύμους εζυμωμένας με έλαιον και λάγανα άζυμα κεχρισμένα με έλαιον· εκ σεμιδάλεως σίτου θέλεις κάμει αυτά.
\par 3 Και θέλεις βάλει αυτά εις εν κάνιστρον και θέλεις φέρει αυτά εν τω κανίστρω μετά του μοσχαρίου και των δύο κριών.
\par 4 Και τον Ααρών και τους υιούς αυτού θέλεις προσαγάγει εις την θύραν της σκηνής του μαρτυρίου και θέλεις λούσει αυτούς εν ύδατι.
\par 5 Και θέλεις λάβει τας στολάς και θέλεις ενδύσει τον Ααρών τον χιτώνα και τον ποδήρη του εφόδ και το εφόδ και το περιστήθιον, και θέλεις ζώσει αυτόν με την κεντητήν ζώνην του εφόδ.
\par 6 Και θέλεις βάλει την μίτραν επί την κεφαλήν αυτού και θέλεις βάλει το άγιον διάδημα επί την μίτραν.
\par 7 Τότε θέλεις λάβει το έλαιον του χρίσματος και θέλεις χύσει εξ αυτού επί την κεφαλήν αυτού και θέλεις χρίσει αυτόν.
\par 8 Και θέλεις προσαγάγει τους υιούς αυτού και ενδύσει αυτούς χιτώνας·
\par 9 και θέλεις ζώσει αυτούς με ζώνας, τον Ααρών και τους υιούς αυτού, και θέλεις περιθέσει εις αυτούς μιτρίδια, και η ιερατεία θέλει είσθαι εις αυτούς κατά νόμον παντοτεινόν· και θέλεις καθιερώσει τον Ααρών και τους υιούς αυτού.
\par 10 Και θέλεις προσαγάγει το μοσχάριον έμπροσθεν της σκηνής του μαρτυρίου, και ο Ααρών και οι υιοί αυτού θέλουσιν επιθέσει τας χείρας αυτών επί την κεφαλήν του μοσχαρίου·
\par 11 και θέλεις σφάξει το μοσχάριον ενώπιον Κυρίου παρά την θύραν της σκηνής του μαρτυρίου.
\par 12 Και θέλεις λάβει εκ του αίματος του μοσχαρίου και θέσει επί των κεράτων του θυσιαστηρίου με τον δάκτυλόν σου· και θέλεις χύσει όλον το αίμα παρά την βάσιν του θυσιαστηρίου.
\par 13 Και θέλεις λάβει όλον το στέαρ το περικαλύπτον τα εντόσθια και τον επάνω λοβόν του ήπατος και τους δύο νεφρούς και το στέαρ το επ' αυτών και θέλεις καύσει αυτά επί του θυσιαστηρίου.
\par 14 Το δε κρέας του μοσχαρίου και το δέρμα αυτού και την κόπρον αυτού θέλεις καύσει εν πυρί έξω του στρατοπέδου· τούτο είναι θυσία περί αμαρτίας.
\par 15 Και τον κριόν τον ένα θέλεις λάβει, και θέλουσιν επιθέσει ο Ααρών και οι υιοί αυτού τας χείρας αυτών επί την κεφαλήν του κριού·
\par 16 και θέλεις σφάξει τον κριόν και θέλεις λάβει το αίμα αυτού και ραντίσει επί το θυσιαστήριον κύκλω·
\par 17 και θέλεις διαμελίσει τον κριόν εις τμήματα και θέλεις πλύνει τα εντόσθια αυτού και τους πόδας αυτού, και βάλει αυτά μετά των τμημάτων αυτού και μετά της κεφαλής αυτού·
\par 18 και θέλεις καύσει όλον τον κριόν επί του θυσιαστηρίου· τούτο είναι ολοκαύτωμα εις τον Κύριον· είναι οσμή ευωδίας, θυσία γινομένη διά πυρός εις τον Κύριον.
\par 19 Και θέλεις λάβει τον δεύτερον κριόν· και θέλουσιν επιθέσει ο Ααρών και οι υιοί αυτού τας χείρας αυτών επί την κεφαλήν του κριού·
\par 20 τότε θέλεις σφάξει τον κριόν και θέλεις λάβει εκ του αίματος αυτού και θέσει επί τον λοβόν του δεξιού ωτίου του Ααρών, και επί τον λοβόν του δεξιού ωτίου των υιών αυτού, και επί τον αντίχειρα της δεξιάς χειρός αυτών, και επί τον μεγάλον δάκτυλον του δεξιού ποδός αυτών, και θέλεις ραντίσει το αίμα επί το θυσιαστήριον κύκλω.
\par 21 Και θέλεις λάβει εκ του αίματος, του επί του θυσιαστηρίου, και εκ του ελαίου του χρίσματος, και θέλεις ραντίσει επί τον Ααρών, και επί τας στολάς αυτού και επί τους υιούς αυτού και επί τας στολάς των υιών αυτού μετ' αυτού· και θέλουσιν αγιασθή, αυτός, και αι στολαί αυτού, και οι υιοί αυτού, και αι στολαί των υιών αυτού μετ' αυτού.
\par 22 Και θέλεις λάβει εκ του κριού το στέαρ και την ουράν και το στέαρ το περικαλύπτον τα εντόσθια και τον επάνω λοβόν του ήπατος και τους δύο νεφρούς, και το στέαρ το επ' αυτών και τον δεξιόν βραχίονα, διότι είναι κριός καθιερώσεως,
\par 23 και ένα ψωμόν, και μίαν πήτταν ελαιωμένην, και εν λάγανον εκ του κανίστρου των αζύμων των προτεθειμένων ενώπιον Κυρίου·
\par 24 και θέλεις επιθέσει τα πάντα εις τας χείρας του Ααρών και εις τας χείρας των υιών αυτού· και θέλεις κινήσει αυτά εις κινητήν προσφοράν ενώπιον Κυρίου.
\par 25 Και θέλεις λάβει αυτά εκ των χειρών αυτών και καύσει επί του θυσιαστηρίου επάνω του ολοκαυτώματος εις οσμήν ευωδίας ενώπιον Κυρίου· τούτο είναι θυσία γινομένη διά πυρός εις τον Κύριον,
\par 26 Και θέλεις λάβει το στήθος εκ του κριού της καθιερώσεως, όστις είναι διά τον Ααρών, και θέλεις κινήσει αυτό εις κινητήν προσφοράν ενώπιον Κυρίου και θέλει είσθαι μερίδιόν σου.
\par 27 Και θέλεις αγιάσει το στήθος της κινητής προσφοράς και τον βραχίονα της προσφοράς της υψώσεως, ήτις εκινήθη και ήτις υψώθη, εκ του κριού της καθιερώσεως, εξ εκείνου όστις είναι διά τον Ααρών, και εξ εκείνου όστις είναι διά τους υιούς αυτού·
\par 28 και θέλει είσθαι του Ααρών και των υιών αυτού κατά νόμον παντοτεινόν παρά των υιών Ισραήλ· διότι είναι προσφορά υψώσεως· και θέλει είσθαι προσφορά υψώσεως παρά των υιών Ισραήλ εκ των ειρηνικών θυσιών αυτών, η υψουμένη προσφορά αυτών προς τον Κύριον.
\par 29 Και η αγία στολή του Ααρών θέλει είσθαι των υιών αυτού μετ' αυτόν, διά να χρισθώσιν εν αυτή και να καθιερωθώσιν εν αυτή.
\par 30 Επτά ημέρας θέλει ενδύεσθαι αυτήν ο ιερεύς, ο αντ' αυτού εκ των υιών αυτού, όστις εισέρχεται εις την σκηνήν του μαρτυρίου διά να λειτουργήση εν τω αγίω.
\par 31 Και θέλεις λάβει τον κριόν της καθιερώσεως και βράσει το κρέας αυτού εν τόπω αγίω.
\par 32 Και θέλουσι φάγει ο Ααρών και οι υιοί αυτού το κρέας του κριού και τον άρτον τον εν τω κανίστρω παρά την θύραν της σκηνής του μαρτυρίου.
\par 33 Και θέλουσι φάγει εκείνα, διά των οποίων έγεινεν η εξιλέωσις, προς καθιέρωσιν και αγιασμόν αυτών· ξένος όμως δεν θέλει φάγει, διότι είναι άγια·
\par 34 και αν μείνη τι εκ του κρέατος των καθιερώσεων ή εκ του άρτου έως πρωΐ, τότε θέλεις καύσει το εναπολειφθέν εν πυρί· δεν θέλει φαγωθή, διότι είναι άγιον.
\par 35 Και ούτω θέλεις κάμει εις τον Ααρών και εις τους υιούς αυτού κατά πάντα όσα προσέταξα εις σέ· επτά ημέρας θέλεις καθιερώσει αυτούς·
\par 36 και θέλεις προσφέρει πάσαν ημέραν εν μοσχάριον εις προσφοράν περί αμαρτίας διά εξιλέωσιν. Και θέλεις καθαρίζει το θυσιαστήριον, κάμνων εξιλέωσιν υπέρ αυτού, και θέλεις χρίσει αυτό διά να αγιάσης αυτό.
\par 37 Επτά ημέρας θέλεις κάμνει εξιλέωσιν υπέρ του θυσιαστηρίου και θέλεις αγιάζει αυτό· και θέλει είσθαι θυσιαστήριον αγιώτατον· παν το εγγίζον το θυσιαστήριον θέλει είσθαι άγιον.
\par 38 Τούτο δε είναι εκείνο, το οποίον θέλεις προσφέρει επί του θυσιαστηρίου· δύο αρνία ενιαύσια την ημέραν διαπαντός.
\par 39 το εν αρνίον θέλεις προσφέρει το πρωΐ, και το άλλο αρνίον θέλεις προσφέρει το δειλινόν·
\par 40 και μετά του ενός αρνίου εν δέκατον σεμιδάλεως εζυμωμένης με το τέταρτον ενός ιν ελαίου κοπανισμένου· και το τέταρτον ενός ιν οίνου διά σπονδήν.
\par 41 και το δεύτερον αρνίον θέλεις προσφέρει το δειλινόν· κατά την προσφοράν της πρωΐας, και κατά την σπονδήν αυτής, θέλεις κάμει εις αυτό, εις οσμήν ευωδίας, θυσίαν γινομένην διά πυρός προς τον Κύριον.
\par 42 τούτο θέλει είσθαι παντοτεινόν ολοκαύτωμα εις τας γενεάς σας παρά την θύραν της σκηνής του μαρτυρίου ενώπιον Κυρίου· όπου θέλω εμφανίζεσθαι εις σας, διά να λαλώ εκεί προς σε.
\par 43 Και εκεί θέλει εμφανίζεσθαι εις τους υιούς Ισραήλ, και η σκηνή θέλει αγιάζεσθαι με την δόξαν μου.
\par 44 Και θέλω αγιάζει την σκηνήν του μαρτυρίου και το θυσιαστήριον· θέλω αγιάζει και τον Ααρών και τους υιούς αυτού, διά να ιερατεύωσιν εις εμέ.
\par 45 Και θέλω κατοικεί εν μέσω των υιών Ισραήλ, και θέλω είσθαι Θεός αυτών.
\par 46 Και αυτοί θέλουσι γνωρίζει ότι εγώ είμαι Κύριος ο Θεός αυτών, ο εξαγαγών αυτούς εκ γης Αιγύπτου διά να κατοικώ εν μέσω αυτών· εγώ Κύριος ο Θεός αυτών.

\chapter{30}

\par Και θέλεις κάμει θυσιαστήριον διά να θυμιάζης θυμίαμα· εκ ξύλου σιττίμ θέλεις κάμει αυτό.
\par 2 μιας πήχης το μήκος αυτού και μιας πήχης το πλάτος αυτού τετράγωνον θέλει είσθαι και δύο πηχών το ύψος αυτού τα κέρατα αυτού εκ του αυτού.
\par 3 Και θέλεις περικαλύψει αυτό με χρυσίον καθαρόν, την κορυφήν αυτού και τα πλάγια αυτού κύκλω και τα κέρατα αυτού και θέλεις κάμει εις αυτό στεφάνην χρυσήν κύκλω.
\par 4 Και δύο χρυσούς κρίκους θέλεις κάμει εις αυτό υπό την στεφάνην αυτού· πλησίον των δύο γωνιών αυτού επί τα δύο πλάγια αυτού θέλεις κάμει αυτούς, και θέλουσιν είσθαι θήκαι των μοχλών, ώστε να βαστάζωσιν αυτό δι' αυτών.
\par 5 Και θέλεις κάμει τους μοχλούς εκ ξύλου σιττίμ, και θέλεις περικαλύψει αυτούς με χρυσίον.
\par 6 Και θέλεις βάλει αυτό απέναντι του καταπετάσματος του ενώπιον της κιβωτού του μαρτυρίου, αντικρύ του ιλαστηρίου του επί του μαρτυρίου, όπου θέλω εμφανίζεσθαι εις σε.
\par 7 Και θέλει θυμιάζει ο Ααρών επ' αυτού θυμίαμα ευώδες καθ' εκάστην πρωΐαν· όταν ετοιμάζη τους λύχνους, θέλει θυμιάζει επ' αυτού.
\par 8 Και όταν ανάπτη ο Ααρών τους λύχνους το εσπέρας, θέλει θυμιάζει επ' αυτού, θυμίαμα παντοτεινόν ενώπιον του Κυρίου εις τας γενεάς σας.
\par 9 δεν θέλετε προσφέρει επ' αυτού ξένον θυμίαμα ουδέ ολοκαύτωμα ουδέ προσφοράν εξ αλφίτων ουδέ θέλετε χύσει επ' αυτού σπονδήν.
\par 10 Και θέλει κάμνει ο Ααρών εξιλέωσιν επί των κεράτων αυτού άπαξ του ενιαυτού με το αίμα της περί αμαρτίας προσφοράς της εξιλεώσεως· άπαξ του ενιαυτού θέλει κάμνει εξιλέωσιν επ' αυτού εις τας γενεάς σας· τούτο είναι αγιώτατον προς τον Κύριον.
\par 11 Και ελάλησε Κύριος προς τον Μωϋσήν, λέγων,
\par 12 Όταν λαμβάνης το κεφάλαιον των υιών Ισραήλ κατά την απαρίθμησιν αυτών, τότε θέλουσι δώσει πας άνθρωπος λύτρον διά την ψυχήν αυτού προς τον Κύριον, όταν απαριθμής αυτούς, διά να μη επέλθη πληγή επ' αυτούς, όταν απαριθμής αυτούς·
\par 13 τούτο θέλουσι δίδει πας όστις περνά εις την απαρίθμησιν, ήμισυ του σίκλου κατά τον σίκλον του αγίου· ο σίκλος είναι είκοσι γερά· ήμισυ του σίκλου θέλει είσθαι η προσφορά του Κυρίου.
\par 14 πας όστις περνά εις την απαρίθμησιν, από είκοσι ετών ηλικίας και επάνω, θέλει δώσει προσφοράν εις τον Κύριον.
\par 15 Ο πλούσιος δεν θέλει δώσει πλειότερον, και ο πτωχός δεν θέλει δώσει ολιγώτερον ημίσεος σίκλου, όταν δίδωσι την προσφοράν εις τον Κύριον διά να κάμωσιν εξιλέωσιν υπέρ των ψυχών υμών.
\par 16 Και θέλεις λάβει το αργύριον της εξιλεώσεως παρά των υιών Ισραήλ, και θέλεις μεταχειρισθή αυτό εις την υπηρεσίαν της σκηνής του μαρτυρίου, και θέλει είσθαι εις τους υιούς Ισραήλ εις μνημόσυνον ενώπιον του Κυρίου, διά να γείνη εξιλέωσις υπέρ των ψυχών υμών.
\par 17 Και ελάλησε Κύριος προς τον Μωϋσήν, λέγων,
\par 18 Και θέλεις κάμει νιπτήρα χάλκινον και την βάσιν αυτού χαλκίνην, διά να νίπτωνται και θέλεις θέσει αυτόν μεταξύ της σκηνής του μαρτυρίου και του θυσιαστηρίου και θέλεις βάλει ύδωρ εις αυτόν·
\par 19 και θέλουσι νίπτει ο Ααρών και οι υιοί αυτού τας χείρας αυτών και τους πόδας αυτών εξ αυτού·
\par 20 Όταν εισέρχωνται εις την σκηνήν του μαρτυρίου, θέλουσι νίπτεσθαι με ύδωρ, διά να μη αποθάνωσιν· ή όταν πλησιάζωσιν εις το θυσιαστήριον διά να λειτουργήσωσι, διά να καύσωσι θυσίαν γινομένην διά πυρός εις τον Κύριον·
\par 21 τότε θέλουσι νίπτει τας χείρας αυτών και τους πόδας αυτών, διά να μη αποθάνωσι και τούτο θέλει είσθαι νόμος παντοτεινός εις αυτούς, εις αυτόν και εις το σπέρμα αυτού εις τας γενεάς αυτών.
\par 22 Και ελάλησε Κύριος προς τον Μωϋσήν, λέγων,
\par 23 Και συ λάβε εις σεαυτόν εκλεκτά αρώματα, καθαράς σμύρνης πεντακοσίους σίκλους και ευώδους κινναμώμου ήμισυ αυτής, διακοσίους πεντήκοντα, και ευώδους καλάμου διακοσίους πεντήκοντα,
\par 24 και κασσίας πεντακοσίους, κατά τον σίκλον του αγίου, και ελαίου ελαίας εν ίν·
\par 25 και θέλεις κάμει αυτό έλαιον αγίου χρίσματος, χρίσμα μυρεψικόν κατά την τέχνην του μυρεψού· άγιον χριστήριον έλαιον θέλει είσθαι.
\par 26 Και θέλεις χρίσει με αυτό την σκηνήν του μαρτυρίου και την κιβωτόν του μαρτυρίου,
\par 27 και την τράπεζαν και πάντα τα σκεύη αυτής και την λυχνίαν και τα σκεύη αυτής και το θυσιαστήριον του θυμιάματος,
\par 28 και το θυσιαστήριον του ολοκαυτώματος μετά πάντων των σκευών αυτού και τον νιπτήρα και την βάσιν αυτού.
\par 29 Και θέλεις αγιάσει αυτά, διά να ήναι αγιώτατα· παν το εγγίζον αυτά θέλει είσθαι άγιον.
\par 30 Και τον Ααρών και τους υιούς αυτού θέλεις χρίσει και θέλεις αγιάσει αυτούς, διά να ιερατεύωσιν εις εμέ.
\par 31 Και θέλεις λαλήσει προς τους υιούς Ισραήλ, λέγων, τούτο θέλει είσθαι εις εμέ άγιον χριστήριον έλαιον εις τας γενεάς σας·
\par 32 επί σάρκα ανθρώπου δεν θέλει χυθή ουδέ θέλετε κάμει όμοιον αυτού κατά την σύνθεσιν αυτού· τούτο είναι άγιον και άγιον θέλει είσθαι εις εσάς·
\par 33 όστις συνθέση όμοιον αυτού ή όστις βάλη εξ αυτού επί αλλογενή, θέλει εξολοθρευθή εκ του λαού αυτού.
\par 34 Και είπε Κύριος προς τον Μωϋσήν, Λάβε εις σεαυτόν ευώδη αρώματα, στακτήν και όνυχα και χαλβάνην, ταύτα τα ευώδη αρώματα μετά καθαρού λιβανίου· ίσου βάρους θέλει είσθαι έκαστον.
\par 35 Και θέλεις κάμει τούτο θυμίαμα, σύνθεσιν κατά την τέχνην του μυρεψού μεμιγμένον, καθαρόν, άγιον·
\par 36 και θέλεις κοπανίσει μέρος εκ τούτου πολλά λεπτόν, και θέλεις βάλει εξ αυτού έμπροσθεν του μαρτυρίου εν τη σκηνή του μαρτυρίου, όπου θέλω εμφανίζεσθαι εις σέ· τούτο θέλει είσθαι εις εσάς αγιώτατον.
\par 37 Κατά δε την σύνθεσιν του θυμιάματος τούτου, το οποίον θέλεις κάμει, σεις δεν θέλετε κάμει εις εαυτούς· άγιον θέλει είσθαι εις σε διά τον Κύριον·
\par 38 όστις κάμη όμοιον αυτού διά να μυρίζηται αυτό, θέλει εξολοθρευθή εκ του λαού αυτού.

\chapter{31}

\par Και ελάλησε Κύριος προς τον Μωϋσήν, λέγων,
\par 2 Ιδέ, εγώ εκάλεσα εξ ονόματος Βεσελεήλ τον υιόν του Ουρί, υιού του Ωρ, εκ της φυλής του Ιούδα·
\par 3 και ενέπλησα αυτόν πνεύματος θείου, σοφίας και συνέσεως και επιστήμης και πάσης καλλιτεχνίας,
\par 4 διά να επινοή έντεχνα έργα, ώστε να εργάζηται εις χρυσόν και εις άργυρον και εις χαλκόν,
\par 5 και να γλύφη λίθους ενθέσεως, και να σκαλίζη ξύλα δι' εργασίαν εις πάσαν καλλιτεχνίαν.
\par 6 Και εγώ, ιδού, έδωκα εις αυτόν Ελιάβ τον υιόν του Αχισαμάχ, εκ της φυλής του Δάν· και εις πάντα συνετόν την καρδίαν έδωκα σοφίαν, διά να κάμωσι πάντα όσα προσέταξα εις σέ·
\par 7 την σκηνήν του μαρτυρίου, και την κιβωτόν του μαρτυρίου και το ιλαστήριον το επάνωθεν αυτής και πάντα τα σκεύη της σκηνής,
\par 8 και την τράπεζαν και τα σκεύη αυτής και την καθαράν λυχνίαν μετά πάντων των σκευών αυτής και το θυσιαστήριον του θυμιάματος,
\par 9 και το θυσιαστήριον του ολοκαυτώματος μετά των σκευών αυτού και τον νιπτήρα και την βάσιν αυτού,
\par 10 και τας στολάς τας λειτουργικάς, και τας αγίας στολάς του Ααρών του ιερέως, και τας στολάς των υιών αυτού, διά να ιερατεύωσι,
\par 11 και το χριστήριον έλαιον, και το ευώδες θυμίαμα διά το άγιον· κατά πάντα όσα προσέταξα εις σε θέλουσι κάμει.
\par 12 Και ελάλησε Κύριος προς τον Μωϋσήν, λέγων,
\par 13 Και συ λάλησον προς τους υιούς Ισραήλ, λέγων, Προσέχετε να φυλάττητε τα σάββατά μου· διότι τούτο είναι σημείον μεταξύ εμού και υμών εις τας γενεάς υμών, διά να γνωρίζητε ότι εγώ είμαι Κύριος, ο αγιάζων υμάς·
\par 14 και θέλετε φυλάττει το σάββατον, διότι είναι άγιον εις εσάς· όστις βεβηλώση αυτό, θέλει εξάπαντος θανατωθή· διότι πας όστις κάμη εργασίαν εν αυτώ, η ψυχή εκείνη θέλει εξολοθρευθή εκ μέσου του λαού αυτής.
\par 15 Εξ ημέρας θέλει γίνεσθαι εργασία· εν δε τη εβδόμη ημέρα σάββατον θέλει είσθαι, ανάπαυσις αγία εις τον Κύριον· και όστις κάμη εργασίαν εν τη ημέρα του σαββάτου θέλει εξάπαντος θανατωθή.
\par 16 Και θέλουσι φυλάττει οι υιοί Ισραήλ το σάββατον, διά να εορτάζωσιν αυτό εις τας γενεάς αυτών εις διαθήκην αιώνιον.
\par 17 Τούτο είναι σημείον μεταξύ εμού και των υιών Ισραήλ διαπαντός· διότι εις εξ ημέρας εποίησεν ο Κύριος τον ουρανόν και την γην, εν δε τη εβδόμη ημέρα κατέπαυσε και ανεπαύθη.
\par 18 Και έδωκεν εις τον Μωϋσήν, αφού ετελείωσε λαλών προς αυτόν επί του όρους Σινά, δύο πλάκας του μαρτυρίου, πλάκας λιθίνας γεγραμμένας με τον δάκτυλον του Θεού.

\chapter{32}

\par Και ιδών ο λαός ότι εβράδυνεν ο Μωϋσής να καταβή εκ του όρους, συνήχθη ο λαός επί τον Ααρών και έλεγον προς αυτόν, Σηκώθητι, κάμε εις ημάς θεούς, οίτινες να προπορεύωνται ημών· διότι ούτος ο Μωϋσής, ο άνθρωπος όστις εξήγαγεν ημάς εκ γης Αιγύπτου, δεν εξεύρομεν τι απέγεινεν αυτός.
\par 2 Και είπε προς αυτούς ο Ααρών, Αφαιρέσατε τα χρυσά ενώτια, τα οποία είναι εις τα ώτα των γυναικών σας, των υιών σας και των θυγατέρων σας, και φέρετε προς εμέ.
\par 3 Και αφήρεσε πας ο λαός τα χρυσά ενώτια, τα οποία ήσαν εις τα ώτα αυτών, και έφεραν προς τον Ααρών.
\par 4 Και λαβών εκ των χειρών αυτών, διεμόρφωσεν αυτό με εργαλείον εγχαρακτικόν, και έκαμεν αυτό μόσχον χωνευτόν· οι δε είπον, Ούτοι είναι οι θεοί σου, Ισραήλ, οίτινες σε ανεβίβασαν εκ γης Αιγύπτου.
\par 5 Και ότε είδε τούτο ο Ααρών, ωκοδόμησε θυσιαστήριον έμπροσθεν αυτού· και εκήρυξεν ο Ααρών, λέγων, Αύριον είναι εορτή εις τον Κύριον.
\par 6 Και σηκωθέντες ενωρίς την επαύριον, προσέφεραν ολοκαυτώματα και έφεραν ειρηνικάς προσφοράς· και εκάθισεν ο λαός να φάγη και να πίη, και εσηκώθησαν να παίζωσι.
\par 7 Και είπε Κύριος προς τον Μωϋσήν, Ύπαγε, κατάβηθι διότι ηνόμησεν ο λαός σου, τον οποίον εξήγαγες εκ γης Αιγύπτου·
\par 8 εξετράπησαν ταχέως εκ της οδού την οποίαν προσέταξα εις αυτούς· έκαμαν εις εαυτούς μόσχον χωνευτόν και προσεκύνησαν αυτόν και εθυσίασαν εις αυτόν και είπον, Ούτοι είναι οι θεοί σου, Ισραήλ, οίτινες σε ανεβίβασαν εκ γης Αιγύπτου.
\par 9 Και είπε Κύριος προς τον Μωϋσήν, είδον τον λαόν τούτον, και ιδού, είναι λαός σκληροτράχηλος·
\par 10 τώρα λοιπόν, άφες με, και θέλει εξαφθή η οργή μου εναντίον αυτών και θέλω εξολοθρεύσει αυτούς· και θέλω σε καταστήσει έθνος μέγα.
\par 11 Και ικέτευσεν ο Μωϋσής Κύριον τον Θεόν αυτού και είπε, Διά τι, Κύριε, εξάπτεται η οργή σου εναντίον του λαού σου, τον οποίον εξήγαγες εκ γης Αιγύπτου μετά μεγάλης δυνάμεως και κραταιάς χειρός;
\par 12 διά τι να είπωσιν οι Αιγύπτιοι, λέγοντες, Με πονηρίαν εξήγαγεν αυτούς, διά να θανατώση αυτούς εις τα όρη και να εξολοθρεύση αυτούς από προσώπου της γης; επίστρεψον από της εξάψεως της οργής σου και μεταμελήθητι περί του κακού του προς τον λαόν σου·
\par 13 ενθυμήθητι τον Αβραάμ, τον Ισαάκ και τον Ισραήλ, τους δούλους σου, προς τους οποίους ώμοσας επί σεαυτόν και είπας προς αυτούς, Θέλω πληθύνει το σπέρμα σας ως τα άστρα του ουρανού· και πάσαν την γην ταύτην περί της οποίας ελάλησα, θέλω δώσει εις το σπέρμα σας, και θέλουσι κληρονομήσει αυτήν διαπαντός.
\par 14 Και μετεμελήθη ο Κύριος περί του κακού, το οποίον είπε να κάμη κατά του λαού αυτού.
\par 15 Και στραφείς ο Μωϋσής κατέβη εκ του όρους, και αι δύο πλάκες του μαρτυρίου ήσαν εν ταις χερσίν αυτού· πλάκες γεγραμμέναι εξ αμφοτέρων των μερών· εκ του ενός μέρους και εκ του άλλου ήσαν γεγραμμέναι.
\par 16 Και αι πλάκες ήσαν έργον Θεού και η γραφή ήτο γραφή Θεού εγκεχαραγμένη επί τας πλάκας.
\par 17 Και ακούσας ο Ιησούς τον θόρυβον του λαού αλαλάζοντος, είπε προς τον Μωϋσήν, Θόρυβος πολέμου είναι εν τω στρατοπέδω.
\par 18 Ο δε είπε, Δεν είναι φωνή αλαλαζόντων διά νίκην ουδέ φωνή βοώντων διά ήτταν· φωνήν αδόντων εγώ ακούω.
\par 19 Καθώς δε επλησίασεν εις το στρατόπεδον, είδε τον μόσχον και χορούς· και εξήφθη ο θυμός του Μωϋσέως, και έρριψε τας πλάκας από των χειρών αυτού και συνέτριψεν αυτάς υπό το όρος·
\par 20 και λαβών τον μόσχον, τον οποίον είχον κάμει, κατέκαυσεν εν πυρί, και συντρίψας εωσού ελεπτύνθη, έσπειρεν επί το ύδωρ και επότισε τους υιούς Ισραήλ.
\par 21 Και είπεν ο Μωϋσής προς τον Ααρών, Τι έκαμεν εις σε ο λαός ούτος, ώστε επέφερες επ' αυτούς αμαρτίαν μεγάλην;
\par 22 Και είπεν ο Ααρών, Ας μη εξάπτηται ο θυμός του κυρίου μου· συ γνωρίζεις τον λαόν, ότι έγκειται εις την κακίαν·
\par 23 διότι είπον προς εμέ, Κάμε εις ημάς θεούς, οίτινες να προπορεύωνται ημών· διότι ούτος ο Μωϋσής, ο άνθρωπος όστις εξήγαγεν ημάς εκ γης Αιγύπτου, δεν εξεύρομεν τι απέγεινεν αυτός·
\par 24 και είπα προς αυτούς, Όστις έχει χρυσίον, ας αφαιρέσωσιν αυτό· και έδωκαν εις εμέ· τότε έρριψα αυτό εις το πυρ, και εξήλθεν ο μόσχος ούτος.
\par 25 Και ιδών ο Μωϋσής τον λαόν ότι ήτο αχαλίνωτος, διότι ο Ααρών είχε αφήσει αυτούς αχαλινώτους προς καταισχύνην, μεταξύ των εχθρών αυτών,
\par 26 εστάθη ο Μωϋσής παρά την πύλην του στρατοπέδου και είπεν, Όστις είναι του Κυρίου, ας έλθη προς εμέ. Και συνήχθησαν προς αυτόν πάντες οι υιοί του Λευΐ.
\par 27 Και είπε προς αυτούς, Ούτω λέγει Κύριος ο Θεός του Ισραήλ· Ας βάλη έκαστος την ρομφαίαν αυτού επί τον μηρόν αυτού· και διέλθετε και εξέλθετε από πύλης εις πύλην διά του στρατοπέδου, και ας θανατώση έκαστος τον αδελφόν αυτού και έκαστος τον φίλον αυτού και έκαστος τον πλησίον αυτού.
\par 28 Και έκαμον οι υιοί του Λευΐ κατά τον λόγον του Μωϋσέως· και έπεσαν εκ του λαού εκείνην την ημέραν περίπου τρεις χιλιάδες άνδρες.
\par 29 διότι είπεν ο Μωϋσής, Καθιερώσατε εαυτούς σήμερον εις τον Κύριον, έκαστος επί τον υιόν αυτού και έκαστος επί τον αδελφόν αυτού, διά να δοθή εις εσάς ευλογία σήμερον.
\par 30 Και την επαύριον είπεν ο Μωϋσής προς τον λαόν, Σεις ημαρτήσατε αμαρτίαν μεγάλην· και τώρα θέλω αναβή προς τον Κύριον· ίσως κάμω εξιλέωσιν διά την αμαρτίαν σας.
\par 31 Και επέστρεψεν ο Μωϋσής προς τον Κύριον και είπε, Δέομαι· ούτος ο λαός ημάρτησεν αμαρτίαν μεγάλην και έκαμον εις εαυτούς θεούς χρυσούς·
\par 32 πλην τώρα, εάν συγχωρήσης την αμαρτίαν αυτών· ει δε μη, εξάλειψόν με, δέομαι, εκ της βίβλου σου, την οποίαν έγραψας.
\par 33 Και είπε Κύριος προς τον Μωϋσήν, Όστις ημάρτησεν εναντίον εμού, τούτον θέλω εξαλείψει εκ της βίβλου μου·
\par 34 όθεν τώρα ύπαγε, οδήγησον τον λαόν εις τον τόπον περί του οποίου σε είπα· ιδού, ο άγγελός μου θέλει προπορεύεσθαι έμπροσθέν σου' αλλ' όμως εν τη ημέρα της ανταποδώσεώς μου θέλω ανταποδώσει την αμαρτίαν αυτών επ' αυτούς.
\par 35 Και επάταξε Κύριος τον λαόν, διά την κατασκευήν του μόσχου τον οποίον κατεσκεύασεν ο Ααρών.

\chapter{33}

\par Και είπε Κύριος προς τον Μωϋσήν, Ύπαγε, ανάβηθι εντεύθεν συ και ο λαός τον οποίον εξήγαγες εκ γης Αιγύπτου, εις την γην την οποίαν ώμοσα προς τον Αβραάμ, προς τον Ισαάκ και προς τον Ιακώβ, λέγων, Εις το σπέρμα σου θέλω δώσει αυτήν·
\par 2 και θέλω αποστείλει άγγελον έμπροσθέν σου και θέλω εκδιώξει τον Χαναναίον, τον Αμορραίον και τον Χετταίον και τον Φερεζαίον τον Ευαίον και τον Ιεβουσαίον·
\par 3 εις γην ρέουσαν γάλα και μέλι διότι εγώ δεν θέλω αναβή εν τω μέσω σου, επειδή είσαι λαός σκληροτράχηλος, διά να μη σε εξολοθρεύσω καθ' οδόν.
\par 4 Και ότε ήκουσεν ο λαός τον κακόν τούτον λόγον, κατεπένθησαν και ουδείς έβαλε τον στολισμόν αυτού εφ' εαυτόν.
\par 5 Διότι ο Κύριος είπε προς τον Μωϋσήν, Ειπέ προς τους υιούς Ισραήλ, Σεις είσθε λαός σκληροτράχηλος· μίαν στιγμήν εάν αναβώ εις το μέσον σου, θέλω σε εξολοθρεύσει· όθεν τώρα εκδύθητι τους στολισμούς σου από σου, διά να γνωρίσω τι θέλω κάμει εις σε.
\par 6 Και εξεδύθησαν οι υιοί του Ισραήλ τους στολισμούς αυτών πλησίον του όρους Χωρήβ.
\par 7 Και λαβών ο Μωϋσής την σκηνήν, έστησεν αυτήν έξω του στρατοπέδου, μακράν του στρατοπέδου, και ωνόμασεν αυτήν Σκηνήν του μαρτυρίου· και πας ο ζητών τον Κύριον εξήρχετο προς την σκηνήν του μαρτυρίου την έξω του στρατοπέδου.
\par 8 Και ότε εξήρχετο ο Μωϋσής προς την σκηνήν, πας ο λαός εσηκόνετο και ίστατο έκαστος παρά την θύραν της σκηνής αυτού και έβλεπον κατόπιν του Μωϋσέως, εωσού εισήρχετο εις την σκηνήν.
\par 9 Και καθώς εισήρχετο ο Μωϋσής εις την σκηνήν, κατέβαινεν ο στύλος της νεφέλης και ίστατο επί των θυρών της σκηνής· και ελάλει ο Κύριος μετά του Μωϋσέως.
\par 10 Και έβλεπε πας ο λαός τον στύλον της νεφέλης ιστάμενον επί των θυρών της σκηνής· και πας ο λαός ανιστάμενος προσεκύνει, έκαστος από της θύρας της σκηνής αυτού.
\par 11 Και ελάλει ο Κύριος προς τον Μωϋσήν πρόσωπον προς πρόσωπον, καθώς λαλεί άνθρωπος προς τον φίλον αυτού. Και επέστρεφεν εις το στρατόπεδον· ο δε θεράπων αυτού νέος, Ιησούς ο υιός του Ναυή, δεν ανεχώρει από της σκηνής.
\par 12 Και είπεν ο Μωϋσής προς τον Κύριον, Ιδέ, συ μοι λέγεις, Ανάγαγε τον λαόν τούτον· και συ δεν με εφανέρωσας ποίον θέλεις αποστείλει μετ' εμού· και συ είπας, σε γνωρίζω κατ' όνομα, και μάλιστα εύρηκας χάριν έμπροσθέν μου·
\par 13 τώρα λοιπόν, εάν εύρηκα χάριν έμπροσθέν σου, δείξόν μοι, δέομαι, την οδόν σου, διά να σε γνωρίσω, διά να εύρω χάριν ενώπιόν σου· και ιδέ ότι τούτο το έθνος είναι ο λαός σου.
\par 14 και είπεν, Η παρουσία μου θέλει ελθεί μετά σου και θέλω σοι δώσει ανάπαυσιν.
\par 15 Ο δε είπε προς αυτόν, Εάν η παρουσία σου δεν έλθη μετ' εμού, μη αναγάγης ημάς εντεύθεν·
\par 16 διότι πως θέλει γνωρισθή τώρα, ότι εύρηκα χάριν ενώπιόν σου εγώ και ο λαός σου; ουχί διά της ελεύσεώς σου μεθ' ημών; ούτω θέλομεν διακριθή, εγώ και ο λαός σου, από παντός λαού, του επί προσώπου της γης.
\par 17 Και είπε Κύριος προς τον Μωϋσήν, Και τούτο το πράγμα το οποίον είπας, θέλω κάμει διότι εύρηκας χάριν ενώπιόν μου και σε γνωρίζω κατ' όνομα.
\par 18 Και είπε, Δείξον μοι, δέομαι, την δόξαν σου.
\par 19 Ο δε είπεν, Εγώ θέλω κάμει να περάση έμπροσθέν σου όλη η αγαθότης μου και θέλω κηρύξει το όνομα του Κυρίου έμπροσθέν σου και θέλω ελεήσει όντινα ελεώ και θέλω οικτειρήσει όντινα οικτείρω.
\par 20 Και είπε, δεν δύνασαι να ίδης το πρόσωπόν μου· διότι άνθρωπος δεν θέλει με ιδεί και ζήσει.
\par 21 Και είπεν ο Κύριος, Ιδού, τόπος πλησίον μου, και θέλεις σταθή επί της πέτρας·
\par 22 και όταν η δόξα μου διαβαίνη, θέλω σε βάλει εις το σχίσμα της πέτρας και θέλω σε σκεπάσει με την χείρα μου, εωσού παρέλθω·
\par 23 και θέλω σηκώσει την χείρα μου και θέλεις ιδεί τα οπίσω μου· το δε πρόσωπόν μου δεν θέλεις ιδεί.

\chapter{34}

\par Και είπε Κύριος προς τον Μωϋσήν, Κόψον εις σεαυτόν δύο πλάκας λιθίνας καθώς τας πρώτας· και θέλω γράψει επί των πλακών τους λόγους, οίτινες ήσαν επί των πρώτων πλακών, τας οποίας συνέτριψας·
\par 2 και γίνου έτοιμος το πρωΐ, και ανάβηθι το πρωΐ επί το όρος Σινά, και παράστηθι εκεί ενώπιόν μου επί της κορυφής του όρους·
\par 3 και ουδείς θέλει αναβή μετά σου ουδέ θέλει φανή τις καθ' όλον το όρος· και τα ποίμνια και αι αγέλαι δεν θέλουσι βοσκηθή έμπροσθεν του όρους εκείνου.
\par 4 Και έκοψε δύο πλάκας λιθίνας καθώς τας πρώτας· και σηκωθείς ο Μωϋσής ενωρίς το πρωΐ, ανέβη επί το όρος Σινά, καθώς προσέταξεν εις αυτόν ο Κύριος, και έλαβεν εις τας χείρας αυτού τας δύο πλάκας τας λιθίνας.
\par 5 Και κατέβη ο Κύριος εν νεφέλη και εστάθη μετ' αυτού εκεί και εκήρυξε το όνομα του Κυρίου.
\par 6 Και παρήλθε Κύριος έμπροσθεν αυτού και εκήρυξε, Κύριος, Κύριος ο Θεός, οικτίρμων και ελεήμων, μακρόθυμος και πολυέλεος, και αληθινός,
\par 7 φυλάττων έλεος εις χιλιάδας, συγχωρών ανομίαν και παράβασιν και αμαρτίαν και ουδόλως αθωόνων τον ένοχον· ανταποδίδων την ανομίαν των πατέρων επί τα τέκνα και επί τα τέκνα των τέκνων, έως τρίτης και τετάρτης γενεάς.
\par 8 Και έσπευσεν ο Μωϋσής και κύψας εις την γην, προσεκύνησε·
\par 9 και είπεν, Εάν τώρα εύρηκα χάριν ενώπιόν σου, Κύριε, ας έλθη, δέομαι, ο Κύριός μου εν τω μέσω ημών· διότι ο λαός ούτος είναι σκληροτράχηλος· και συγχώρησον την ανομίαν ημών και την αμαρτίαν ημών και λάβε ημάς εις κληρονομίαν σου.
\par 10 Και είπεν, Ιδού, εγώ κάμνω διαθήκην· έμπροσθεν παντός του λαού σου θέλω κάμει θαυμάσια, οποία δεν έγειναν καθ' όλην την γην και εις ουδέν έθνος· και πας ο λαός, εν μέσω του οποίου είσαι, θέλει ιδεί το έργον του Κυρίου· διότι φοβερόν είναι εκείνο, το οποίον εγώ θέλω κάμει μετά σου.
\par 11 Φύλαξον εκείνο, το οποίον εγώ σε προστάζω σήμερον· ιδού, εγώ εκβάλλω απ' έμπροσθέν σου τον Αμορραίον και τον Χαναναίον και τον Χετταίον και τον Φερεζαίον και τον Ευαίον και τον Ιεβουσαίον.
\par 12 Προσέχε εις σεαυτόν, μη κάμης συνθήκην μετά των κατοίκων της γης εις την οποίαν υπάγεις, μήποτε γείνη παγίς εν τω μέσω σου·
\par 13 αλλά τους βωμούς αυτών θέλεις καταστρέψει και τα είδωλα αυτών θέλεις συντρίψει και τα άλση αυτών θέλεις κατακόψει.
\par 14 Διότι δεν θέλεις προσκυνήσει άλλον θεόν· επειδή ο Κύριος, του οποίου το όνομα είναι Ζηλότυπος, είναι Θεός ζηλότυπος·
\par 15 μήποτε κάμης συνθήκην μετά των κατοίκων της γης, και όταν πορνεύσωσι κατόπιν των θεών αυτών και θυσιάσωσι προς τους θεούς αυτών, σε προσκαλέση τις και φάγης από της θυσίας αυτού·
\par 16 και μήποτε λάβης εκ των θυγατέρων αυτού εις τους υιούς σου, και όταν αι θυγατέρες αυτού πορνεύσωσι κατόπιν των θεών αυτών, κάμωσι τους υιούς σου να πορνεύσωσι κατόπιν των θεών αυτών.
\par 17 Θεούς χωνευτούς δεν θέλεις κάμει εις σεαυτόν.
\par 18 Την εορτήν των αζύμων θέλεις φυλάττει. Επτά ημέρας θέλεις τρώγει άζυμα, καθώς προσέταξα εις σε, κατά τον καιρόν του μηνός Αβίβ· διότι κατά τον μήνα Αβίβ εξήλθες εξ Αιγύπτου.
\par 19 Παν το διανοίγον μήτραν είναι ιδικόν μου· και παν πρωτότοκον αρσενικόν μεταξύ των κτηνών σου, είτε βους είτε πρόβατον.
\par 20 Το δε πρωτότοκον της όνου θέλεις εξαγοράζει με αρνίον· και εάν δεν εξαγοράσης αυτό, τότε θέλεις λαιμοτομήσει αυτό. Πάντας τους πρωτοτόκους των υιών σου θέλεις εξαγοράζει. Και ουδείς θέλει φανή ενώπιόν μου κενός.
\par 21 Εξ ημέρας θέλεις εργάζεσθαι την δε εβδόμην ημέραν θέλεις αναπαύεσθαι κατά τον σπορητόν και κατά τον θερισμόν θέλεις αναπαύεσθαι.
\par 22 Και θέλεις φυλάττει την εορτήν των εβδομάδων, των απαρχών του θερισμού του σίτου, και την εορτήν της συγκομιδής εις την επιστροφήν του ενιαυτού.
\par 23 Τρίς του ενιαυτού θέλει εμφανίζεσθαι παν αρσενικόν σου ενώπιον Κυρίου, Κυρίου του Θεού του Ισραήλ.
\par 24 Διότι αφού εκδιώξω τα έθνη απ' έμπροσθέν σου και πλατύνω τα όριά σου, δεν θέλει επιθυμήσει ουδείς την γην σου, όταν αναβαίνης διά να εμφανισθής έμπροσθεν Κυρίου του Θεού σου τρίς του ενιαυτού.
\par 25 Δεν θέλεις προσφέρει το αίμα της θυσίας μου με ένζυμα· και η θυσία της εορτής του πάσχα δεν θέλει μείνει έως το πρωΐ.
\par 26 Τα πρωτογεννήματα της γης σου θέλεις φέρει εις τον οίκον Κυρίου του Θεού σου. Δεν θέλεις ψήσει ερίφιον εν τω γάλακτι της μητρός αυτού.
\par 27 Και είπε Κύριος προς τον Μωϋσήν, Γράψον εις σεαυτόν τους λόγους τούτους· διότι κατά τους λόγους τούτους έκαμα διαθήκην προς σε και προς τον Ισραήλ,
\par 28 Και ήτο εκεί μετά του Κυρίου τεσσαράκοντα ημέρας και τεσσαράκοντα νύκτας· άρτον δεν έφαγε και ύδωρ δεν έπιε. Και έγραψεν επί των πλακών τους λόγους της διαθήκης, τας δέκα εντολάς.
\par 29 Και ότε κατέβαινεν ο Μωϋσής από του όρους Σινά, και αι δύο πλάκες του μαρτυρίου ήσαν εις την χείρα του Μωϋσέως, ότε κατέβαινεν από του όρους, ο Μωϋσής δεν ήξευρεν ότι το δέρμα του προσώπου αυτού έγεινε λαμπρόν ενώ ελάλει μετ' αυτού.
\par 30 Και είδεν ο Ααρών και πάντες οι υιοί Ισραήλ τον Μωϋσήν, και ιδού, το δέρμα του προσώπου αυτού έλαμπε· και εφοβήθησαν να πλησιάσωσιν εις αυτόν.
\par 31 Και εκάλεσεν αυτούς ο Μωϋσής· και επεστράφησαν προς αυτόν ο Ααρών και πάντες οι άρχοντες της συναγωγής, και ελάλησε προς αυτούς ο Μωϋσής.
\par 32 Και μετά ταύτα πάντες οι υιοί Ισραήλ προσήλθον· και προσέταξεν εις αυτούς πάντα όσα ελάλησεν ο Κύριος προς αυτόν επί του όρους Σινά.
\par 33 Και ετελείωσεν ο Μωϋσής λαλών προς αυτούς· είχε δε κάλυμμα επί το πρόσωπον αυτού.
\par 34 Και ότε εισήρχετο ο Μωϋσής ενώπιον του Κυρίου διά να λαλήση μετ' αυτού, εσήκονε το κάλυμμα, εωσού εξέλθη. Και εξήρχετο και ελάλει προς τους υιούς Ισραήλ ό,τι ήτο προστεταγμένος.
\par 35 Και είδον οι υιοί Ισραήλ το πρόσωπον του Μωϋσέως, ότι το δέρμα του προσώπου του Μωϋσέως έλαμπε· και έβαλλε πάλιν ο Μωϋσής το κάλυμμα επί το πρόσωπον αυτού, εωσού εισέλθη διά να λαλήση μετ' αυτού.

\chapter{35}

\par Και συνήθροισεν ο Μωϋσής πάσαν την συναγωγήν των υιών Ισραήλ, και είπε προς αυτούς, Ούτοι είναι οι λόγοι, τους οποίους προσέταξεν ο Κύριος, διά να κάμνητε αυτούς.
\par 2 Εξ ημέρας θέλει γίνεσθαι εργασία· η δε εβδόμη ημέρα θέλει είσθαι εις εσάς αγία, σάββατον αναπαύσεως εις τον Κύριον· πας όστις κάμη εν αυτή εργασίαν θέλει θανατωθή·
\par 3 δεν θέλετε ανάπτει πυρ εν πάσαις ταις κατοικίαις υμών την ημέραν του σαββάτου.
\par 4 Και ελάλησεν ο Μωϋσής προς πάσαν την συναγωγήν των υιών Ισραήλ, λέγων, τούτο είναι το πράγμα το οποίον ο Κύριος προσέταξε, λέγων,
\par 5 Λάβετε από ό,τι έχετε προσφοράν εις τον Κύριον· όστις προαιρείται εν τη καρδία αυτού, ας φέρη την προσφοράν του Κυρίου· χρυσίον και αργύριον και χαλκόν,
\par 6 και κυανούν και πορφυρούν και κόκκινον και βύσσον και τρίχας αιγών,
\par 7 και δέρματα κριών κοκκινοβαφή και δέρματα θώων και ξύλον σιττίμ,
\par 8 και έλαιον διά το φως και αρώματα διά το χριστήριον έλαιον και διά το ευώδες θυμίαμα,
\par 9 και λίθους ονυχίτας και λίθους διά να εντεθώσιν εις το εφόδ και εις το περιστήθιον.
\par 10 Και πας συνετός την καρδίαν μεταξύ σας θέλει ελθεί και κάμει πάντα όσα προσέταξεν ο Κύριος·
\par 11 την σκηνήν, το περικάλυμμα αυτής και την σκέπην αυτής, τας περόνας αυτής και τας σανίδας αυτής, τους μοχλούς αυτής, τους στύλους αυτής και τα υποβάσια αυτής,
\par 12 την κιβωτόν και τους μοχλούς αυτής, το ιλαστήριον και το καλυπτήριον καταπέτασμα,
\par 13 την τράπεζαν και τους μοχλούς αυτής και πάντα τα σκεύη αυτής και τον άρτον της προθέσεως,
\par 14 και την λυχνίαν διά το φως και τα σκεύη αυτής και τους λύχνους αυτής και το έλαιον του φωτός,
\par 15 και το θυσιαστήριον του θυμιάματος, και τους μοχλούς αυτού και το χριστήριον έλαιον και το ευώδες θυμίαμα και τον τάπητα της θύρας της εισόδου της σκηνής,
\par 16 το θυσιαστήριον του ολοκαυτώματος και την χαλκίνην εσχάραν αυτού τους μοχλούς αυτού και πάντα τα σκεύη αυτού, τον νιπτήρα και την βάσιν αυτού,
\par 17 τα παραπετάσματα της αυλής, τους στύλους αυτής και τα υποβάσια αυτών και το παραπέτασμα της θύρας της αυλής,
\par 18 τους πασσάλους της σκηνής και τους πασσάλους της αυλής και τα σχοινία αυτών,
\par 19 τας λειτουργικάς στολάς διά να λειτουργώσιν εν τω αγίω, τας αγίας στολάς διά τον Ααρών τον ιερέα και τας στολάς των υιών αυτού, διά να ιερατεύωσι.
\par 20 Και εξήλθε πάσα η συναγωγή των υιών Ισραήλ απ' έμπροσθεν του Μωϋσέως.
\par 21 Και ήλθον, πας άνθρωπος του οποίου η καρδία διήγειρεν αυτόν· και πας τις τον οποίον το πνεύμα αυτού έκαμε πρόθυμον, έφεραν την προσφοράν του Κυρίου διά το έργον της σκηνής του μαρτυρίου και διά πάσαν την υπηρεσίαν αυτής και διά τας αγίας στολάς.
\par 22 Και ήλθον, άνδρες τε και γυναίκες, όσοι ήσαν προθύμου καρδίας, φέροντες βραχιόλια και ενώτια και δακτυλίδια και περιδέραια, παν σκεύος χρυσούν· και πάντες όσοι προσέφεραν προσφοράν χρυσίον εις τον Κύριον.
\par 23 Και πας άνθρωπος εις τον οποίον ευρίσκετο κυανούν και πορφυρούν και κόκκινον και βύσσος και τρίχες αιγών και δέρματα κριών κοκκινοβαφή και δέρματα θώων, έφεραν αυτά.
\par 24 Πας όστις ηδύνατο να κάμη προσφοράν αργυρίου και χαλκού, έφεραν την προσφοράν του Κυρίου· και πας άνθρωπος, εις τον οποίον ευρίσκετο ξύλον σιττίμ διά παν έργον της υπηρεσίας, έφεραν αυτό.
\par 25 Και πάσα γυνή συνετή την καρδίαν έκλωθον με τας χείρας αυτών και έφερον κεκλωσμένα, το κυανούν και το πορφυρούν, το κόκκινον και την βύσσον.
\par 26 Και πάσαι αι γυναίκες, των οποίων η καρδία διήγειρεν αυτάς εις ευμηχανίαν, έκλωσαν τας τρίχας των αιγών.
\par 27 Και οι άρχοντες έφεραν τους λίθους τους ονυχίτας και τους λίθους της ενθέσεως διά το εφόδ και διά το περιστήθιον·
\par 28 και τα αρώματα, και το έλαιον διά το φως και διά το χριστήριον έλαιον και διά το ευώδες θυμίαμα.
\par 29 Οι υιοί Ισραήλ έφεραν προαιρετικήν προσφοράν εις τον Κύριον, πας ανήρ και γυνή, των οποίων η καρδία έκαμεν αυτούς προθύμους εις το να φέρωσι διά πάσαν την εργασίαν, την οποίαν προσέταξεν ο Κύριος να γείνη διά χειρός του Μωϋσέως.
\par 30 Και είπεν ο Μωϋσής προς τους υιούς Ισραήλ, Ιδέτε, ο Κύριος εκάλεσεν εξ ονόματος Βεσελεήλ τον υιόν του Ουρί, υιού του Ωρ, εκ φυλής Ιούδα·
\par 31 και ενέπλησεν αυτόν πνεύματος θείου, σοφίας συνέσεως και επιστήμης και πάσης καλλιτεχνίας·
\par 32 και διά να επινοή έντεχνα έργα, ώστε να εργάζηται εις χρυσίον και εις αργύριον και εις χαλκόν·
\par 33 και να γλύφη λίθους ενθέσεως και να σκαλίζη ξύλα δι' εργασίαν, διά παν έντεχνον έργον.
\par 34 Και έδωκεν εις την καρδίαν αυτού το να διδάσκη, αυτός και Ελιάβ ο υιός του Αχισαμάχ, εκ φυλής Δαν.
\par 35 Τούτους ενέπλησε συνέσεως καρδίας, διά να εργάζωνται παν έργον εγχαράκτου και καλλιτέχνου και κεντητού εις κυανούν και εις πορφυρούν, εις κόκκινον και εις βύσσον, και υφαντού, των εργαζομένων παν έργον και επινοούντων έντεχνα έργα.

\chapter{36}

\par Και έκαμεν ο Βεσελεήλ και ο Ελιάβ και πας σοφός την καρδίαν, εις τον οποίον ο Κύριος έδωκε σοφίαν και σύνεσιν διά να εξεύρη να εργάζηται παν το έργον της υπηρεσίας του αγιαστηρίου, κατά πάντα όσα προσέταξεν ο Κύριος.
\par 2 Και εκάλεσεν ο Μωϋσής τον Βεσελεήλ και τον Ελιάβ και πάντα σοφόν την καρδίαν, εις του οποίου την καρδίαν ο Κύριος έδωκε σοφίαν, πάντα άνθρωπον του οποίου η καρδία διήγειρεν αυτόν εις το να έλθη προς το έργον διά να κάμη αυτό.
\par 3 Και έλαβον απ' έμπροσθεν του Μωϋσέως πάσας τας προσφοράς, τας οποίας έφεραν οι υιοί Ισραήλ διά το έργον της υπηρεσίας του αγιαστηρίου, διά να κάμωσιν αυτό. Και έφερον έτι προς αυτόν αυτοπροαιρέτους προσφοράς καθ' εκάστην πρωΐαν.
\par 4 Και ήλθον πάντες οι σοφοί οι εργαζόμενοι παν το έργον του αγιαστηρίου, έκαστος από του έργου αυτού, το οποίον έκαμνον·
\par 5 και είπον προς τον Μωϋσήν, λέγοντες, Ο λαός φέρει πλειότερον παρά το ικανόν διά την υπηρεσίαν του έργον, το οποίον ο Κύριος προσέταξε να γείνη.
\par 6 Και προσέταξεν ο Μωϋσής και εκήρυξαν εν τω στρατοπέδω, λέγοντες, Μηδείς ανήρ μήτε γυνή, ας μη κάμνη πλέον εργασίαν διά την προσφοράν του αγιαστηρίου. Και ο λαός έπαυσεν από του να φέρη·
\par 7 διότι η ύλη, την οποίαν είχον, ήτο ικανή δι' όλον το έργον, ώστε να κάμωσιν αυτό, και επερίσσευεν.
\par 8 Και πας σοφός την καρδίαν εκ των εργαζομένων το έργον της σκηνής έκαμον δέκα παραπετάσματα εκ βύσσου κεκλωσμένης και κυανού και πορφυρού και κοκκίνου· με χερουβείμ εντέχνου εργασίας έκαμον αυτά·
\par 9 το μήκος του ενός παραπετάσματος εικοσιοκτώ πηχών και το πλάτος του ενός παραπετάσματος τεσσάρων πηχών· πάντα τα παραπετάσματα του αυτού μέτρου·
\par 10 και συνήψε τα πέντε παραπετάσματα το εν μετά του άλλου· και τα άλλα πέντε παραπετάσματα συνήψε το εν μετά του άλλου.
\par 11 Και έκαμε θυλειάς κυανά επί της άκρας του ενός παραπετάσματος κατά το πλάγιον όπου έγεινεν η ένωσις· ομοίως έκαμεν επί της τελευταίας άκρας του δευτέρου παραπετάσματος, όπου έγεινεν η ένωσις του δευτέρου·
\par 12 πεντήκοντα θυλειάς έκαμεν εις το εν παραπέτασμα και πεντήκοντα θυλειάς έκαμεν επί της άκρας του παραπετάσματος, όπου έγεινεν η ένωσις του δευτέρου, διά να αντικρύζωσι αι θυλειαί προς άλληλα.
\par 13 Και έκαμε πεντήκοντα περόνας χρυσάς και συνήψε τα παραπετάσματα προς άλληλα με τας περόνας. και έγεινεν η σκηνή μία.
\par 14 Και έκαμε παραπετάσματα εκ τριχών αιγών διά να ήναι κάλυμμα επί της σκηνής· ένδεκα παραπετάσματα έκαμεν αυτά·
\par 15 το μήκος του ενός παραπετάσματος τριάκοντα πηχών και το πλάτος του ενός παραπετάσματος τεσσάρων πηχών· τα ένδεκα παραπετάσματα του αυτού μέτρου·
\par 16 και συνήψε τα πέντε παραπετάσματα χωριστά, και τα εξ παραπετάσματα χωριστά.
\par 17 Και έκαμε πεντήκοντα θυλειάς επί της τελευταίας άκρας του παραπετάσματος κατά την ένωσιν, και πεντήκοντα θυλειάς έκαμεν επί της άκρας του παραπετάσματος, κατά την ένωσιν του δευτέρου.
\par 18 Έκαμεν έτι πεντήκοντα περόνας χαλκίνας, διά να συνάψη την σκηνήν, ώστε να ήναι μία.
\par 19 Και έκαμε επικάλυμμα διά την σκηνήν εκ δερμάτων κριών κοκκινοβαφών, και επικάλυμμα υπεράνωθεν εκ δερμάτων θώων.
\par 20 Και έκαμε τας σανίδας διά την σκηνήν εκ ξύλου σιττίμ, ορθίας·
\par 21 το μήκος της μιας σανίδος δέκα πηχών, και το πλάτος της μιας σανίδος μιας πήχης και ημισείας·
\par 22 μία σανίς είχε δύο αγκωνίσκους αντικρύζοντας προς αλλήλους· ούτως έκαμε δι' όλας τας σανίδας της σκηνής.
\par 23 Και έκαμε τας σανίδας διά την σκηνήν, είκοσι σανίδας από του νοτίου μέρους προς τα δεξιά.
\par 24 Και τεσσαράκοντα υποβάσια αργυρά έκαμεν υποκάτω των είκοσι σανίδων· δύο υποβάσια υποκάτω της μιας σανίδος διά τους δύο αγκωνίσκους αυτής και δύο υποβάσια υποκάτω της άλλης σανίδος διά τους δύο αγκωνίσκους αυτής.
\par 25 Και διά το δεύτερον μέρος της σκηνής, το προς βορράν, έκαμεν είκοσι σανίδας,
\par 26 και τα τεσσαράκοντα αυτών υποβάσια αργυρά· δύο υποβάσια υποκάτω της μιας σανίδος και δύο υποβάσια υποκάτω της άλλης σανίδος.
\par 27 Και διά τα μέρη της σκηνής τα προς δυσμάς έκαμεν εξ σανίδας.
\par 28 Και δύο σανίδας έκαμε διά τας γωνίας της σκηνής εις τα δύο πλάγια·
\par 29 και ηνώθησαν κάτωθεν και ηνώθησαν ομού άνωθεν διά του ενός κρίκου· ούτως έκαμε δι' αυτάς αμφοτέρας διά τας δύο γωνίας.
\par 30 Και ήσαν οκτώ σανίδες· και τα υποβάσια αυτών δεκαέξ υποβάσια αργυρά, ανά δύο υποβάσια υποκάτω εκάστης σανίδος.
\par 31 Και έκαμε τους μοχλούς εκ ξύλου σιττίμ· πέντε διά τας σανίδας του ενός μέρους της σκηνής,
\par 32 και πέντε μοχλούς διά τας σανίδας του άλλου μέρους της σκηνής και πέντε μοχλούς διά τας σανίδας της σκηνής, διά τα όπισθεν μέρη τα προς δυσμάς·
\par 33 και έκαμε τον μέσον μοχλόν διά να διαπερά διά των σανίδων απ' άκρου έως άκρου.
\par 34 Και περιεκάλυψε τας σανίδας με χρυσίον και έκαμε τους κρίκους αυτών χρυσούς διά να ήναι θήκαι των μοχλών, και περιεκάλυψε τους μοχλούς με χρυσίον.
\par 35 Και έκαμε το καταπέτασμα εκ κυανού και πορφυρού και κοκκίνου και βύσσου κεκλωσμένης· εντέχνου εργασίας έκαμεν αυτό με χερουβείμ.
\par 36 Και έκαμεν εις αυτό τους τέσσαρας στύλους εκ ξύλου σιττίμ και περιεκάλυψεν αυτούς με χρυσίον· τα άγκιστρα αυτών χρυσά· και έχυσε δι' αυτούς τέσσαρα υποβάσια αργυρά.
\par 37 Και έκαμε τον τάπητα διά την θύραν της σκηνής εκ κυανού και πορφυρού και κοκκίνου και βύσσου κεκλωσμένης, εργασίας κεντητού·
\par 38 και τους πέντε στύλους αυτής και τα άγκιστρα αυτών· και περιεκάλυψε τα κιονόκρανα αυτών και τας ταινίας αυτών με χρυσίον· τα πέντε όμως υποβάσια αυτών ήσαν χάλκινα.

\chapter{37}

\par Και έκαμεν ο Βεσελεήλ την κιβωτόν εκ ξύλου σιττίμ· δύο πηχών και ημισείας το μήκος αυτής και μιας πήχης και ημισείας το πλάτος αυτής και μιας πήχης και ημισείας το ύψος αυτής·
\par 2 και περιεκάλυψεν αυτήν με καθαρόν χρυσίον έσωθεν και έξωθεν και έκαμεν εις αυτήν στεφάνην χρυσήν κύκλω.
\par 3 Και έχυσε δι' αυτήν τέσσαρας κρίκους χρυσούς διά τας τέσσαρας γωνίας αυτής· δύο μεν κρίκους εις το εν πλάγιον αυτής δύο δε κρίκους εις το άλλο πλάγιον αυτής.
\par 4 Και έκαμε μοχλούς εκ ξύλου σιττίμ και περιεκάλυψεν αυτούς με χρυσίον·
\par 5 και εισήγαγε τους μοχλούς εις τους κρίκους κατά τα πλάγια της κιβωτού, διά να βαστάζωσι την κιβωτόν.
\par 6 Και έκαμε το ιλαστήριον εκ χρυσίου καθαρού· δύο πηχών και ημισείας το μήκος αυτού και μιας πήχης και ημισείας το πλάτος αυτού.
\par 7 Και έκαμε δύο χερουβείμ εκ χρυσίου· σφυρήλατα έκαμεν αυτά, επί των δύο άκρων του ιλαστηρίου·
\par 8 εν χερούβ επί του ενός άκρου, και εν χερούβ επί του άλλου άκρου· επί του ιλαστηρίου έκαμε τα χερουβείμ επί των δύο άκρων αυτού·
\par 9 και τα χερουβείμ εξέτεινον τας πτέρυγας άνωθεν, επικαλύπτοντα με τας πτέρυγας αυτών το ιλαστήριον και τα πρόσωπα αυτών έβλεπον το εν προς το άλλο· προς το ιλαστήριον ήσαν τα πρόσωπα των χερουβείμ.
\par 10 Και έκαμε την τράπεζαν εκ ξύλου σιττίμ· δύο πηχών το μήκος αυτής και μιας πήχης το πλάτος αυτής, το δε ύψος αυτής μιας πήχης και ημισείας·
\par 11 και περιεκάλυψεν αυτήν με χρυσίον καθαρόν, και έκαμεν εις αυτήν στεφάνην χρυσήν κύκλω.
\par 12 Έκαμεν έτι εις αυτήν χείλος κύκλω, μιας παλάμης το πλάτος· και επί το χείλος αυτής κύκλω έκαμε στεφάνην χρυσήν.
\par 13 Και έχυσε δι' αυτήν τέσσαρας κρίκους χρυσούς, και έβαλε τους κρίκους επί τας τέσσαρας γωνίας, τας επί των τεσσάρων ποδών αυτής.
\par 14 υπό το χείλος ήσαν οι κρίκοι, θήκαι των μοχλών, διά να βαστάζωσι την τράπεζαν.
\par 15 και έκαμε τους μοχλούς εκ ξύλου σιττίμ, και περιεκάλυψεν αυτούς με χρυσίον, διά να βαστάζωσι την τράπεζαν.
\par 16 και έκαμε τα σκεύη τα επί της τραπέζης, τους δίσκους αυτής και τους θυμιαματοδόχους αυτής και τας λεκάνας αυτής και τα σπονδεία, διά να γίνωνται δι' αυτών αι σπονδαί, εκ χρυσίου καθαρού.
\par 17 και έκαμε την λυχνίαν εκ χρυσίου καθαρού· σφυρήλατον έκαμε την λυχνίαν· ο κορμός αυτής και οι κλάδοι αυτής, αι λεκάναι αυτής, οι κόμβοι αυτής και τα άνθη αυτής ήσαν εν σώμα μετ' αυτής.
\par 18 και εξ κλάδοι εξήρχοντο εκ των πλαγίων αυτής· τρεις κλάδοι της λυχνίας εκ του ενός πλαγίου αυτής και τρεις κλάδοι της λυχνίας εκ του άλλου πλαγίου αυτής·
\par 19 τρεις λεκάναι αμυγδαλοειδείς εις τον ένα κλάδον, εις κόμβος και εν άνθος· και τρεις λεκάναι αμυγδαλοειδείς εις τον άλλον κλάδον, εις κόμβος και εν άνθος· ούτως έκαμεν εις τους εξ κλάδους τους εξερχομένους εκ της λυχνίας.
\par 20 Και εις την λυχνίαν ήσαν τέσσαρες λεκάναι αμυγδαλοειδείς, οι κόμβοι αυτών και τα άνθη αυτών.
\par 21 Και εις κόμβος υπό τους δύο κλάδους εξ αυτής και εις κόμβος υπό τους δύο κλάδους εξ αυτής και εις κόμβος υπό τους δύο κλάδους εξ αυτής, εις τους εξ κλάδους τους εξερχομένους εξ αυτής.
\par 22 Οι κόμβοι αυτών και οι κλάδοι αυτών ήσαν εν σώμα μετ' αυτής· το όλον αυτής εν σφυρήλατον εκ χρυσίου καθαρού.
\par 23 Και έκαμε τους επτά λύχνους αυτής, και τα λυχνοψάλιδα αυτής και τα υποθέματα αυτής εκ χρυσίου καθαρού.
\par 24 Εξ ενός ταλάντου χρυσίου καθαρού έκαμεν αυτήν και πάντα τα σκεύη αυτής.
\par 25 Και έκαμε το θυσιαστήριον του θυμιάματος εκ ξύλου σιττίμ· το μήκος αυτού μιας πήχης και το πλάτος αυτού μιας πήχης, τετράγωνον· και δύο πηχών το ύψος αυτού· τα κέρατα αυτού ήσαν εκ του αυτού.
\par 26 Και περιεκάλυψεν αυτό με χρυσίον καθαρόν, την κορυφήν αυτού και τα πλάγια αυτού κύκλω και τα κέρατα αυτού· και έκαμεν εις αυτό στεφάνην χρυσήν κύκλω.
\par 27 Και δύο κρίκους χρυσούς έκαμε δι' αυτό υπό την στεφάνην αυτού πλησίον των δύο γωνιών αυτού επί τα δύο πλάγια αυτού, διά να ήναι θήκαι των μοχλών, ώστε να βαστάζωσιν αυτό δι' αυτών.
\par 28 Και έκαμε τους μοχλούς εκ ξύλου σιττίμ και περιεκάλυψεν αυτούς με χρυσίον.
\par 29 Και έκαμε το άγιον χριστήριον έλαιον και το καθαρόν ευώδες θυμίαμα κατά την τέχνην του μυρεψού.

\chapter{38}

\par Και έκαμε το θυσιαστήριον του ολοκαυτώματος εκ ξύλου σιττίμ· πέντε πηχών το μήκος αυτού και πέντε πηχών το πλάτος αυτού, τετράγωνον· και το ύψος αυτού τριών πηχών·
\par 2 και έκαμε τα κέρατα αυτού επί των τεσσάρων γωνιών αυτού· τα κέρατα αυτού ήσαν εκ του αυτού· και περιεκάλυψεν αυτό χαλκόν.
\par 3 Και έκαμε πάντα τα σκεύη του θυσιαστηρίου, τους λέβητας και τα πτυάρια και τας λεκάνας, τας κρεάγρας και τα πυροδόχα· πάντα τα σκεύη αυτού έκαμε χάλκινα.
\par 4 Και έκαμε διά το θυσιαστήριον χαλκίνην εσχάραν δικτυωτής εργασίας υπό την περιοχήν αυτού κάτωθεν έως του μέσου αυτού.
\par 5 Και έχυσε τέσσαρας κρίκους διά τα τέσσαρα άκρα της χαλκίνης εσχάρας, διά να ήναι θήκαι των μοχλών.
\par 6 Και έκαμε τους μοχλούς εκ ξύλου σιττίμ, και περιεκάλυψεν αυτούς με χαλκόν.
\par 7 Και εισήξε τους μοχλούς εις τους κρίκους κατά τα πλάγια του θυσιαστηρίου, διά να βαστάζωσιν αυτό δι' αυτών· κοίλον σανιδωτόν έκαμεν αυτό.
\par 8 Και έκαμε τον νιπτήρα χάλκινον και την βάσιν αυτού χαλκίνην εκ των κατόπτρων των συναθροιζομένων γυναικών, αίτινες συνηθροίζοντο παρά την θύραν της σκηνής του μαρτυρίου.
\par 9 Και έκαμε την αυλήν· κατά το πλευρόν το προς μεσημβρίαν τα παραπετάσματα της αυλής ήσαν εκ βύσσου κεκλωσμένης, εκατόν πηχών.
\par 10 Οι στύλοι αυτών ήσαν είκοσι και τα χάλκινα αυτών υποβάσια είκοσι τα άγκιστρα των στύλων και αι ζώναι αυτών αργυρά.
\par 11 Και κατά το βόρειον πλευρόν τα παραπετάσματα ήσαν εκατόν πηχών· οι στύλοι αυτών είκοσι και τα χάλκινα υποβάσια αυτών είκοσι τα άγκιστρα των στύλων και αι ζώναι αυτών αργυρά.
\par 12 Και κατά το δυτικόν πλευρόν ήσαν παραπετάσματα πεντήκοντα πηχών· οι στύλοι αυτών δέκα και τα υποβάσια αυτών δέκα· τα άγκιστρα των στύλων και αι ζώναι αυτών αργυρά.
\par 13 Και κατά το ανατολικόν πλευρόν το προς ανατολάς, πεντήκοντα πηχών.
\par 14 Τα παραπετάσματα του ενός μέρους της πύλης ήσαν δεκαπέντε πηχών· οι στύλοι αυτών τρεις και τα υποβάσια αυτών τρία.
\par 15 Και εις το άλλο μέρος της πύλης της αυλής εκατέρωθεν ήσαν παραπετάσματα δεκαπέντε πηχών· οι στύλοι αυτών τρεις και τα υποβάσια αυτών τρία.
\par 16 Πάντα τα παραπετάσματα της αυλής κύκλω ήσαν εκ βύσσου κεκλωσμένης.
\par 17 Και τα υποβάσια διά τους στύλους ήσαν χάλκινα· τα άγκιστρα των στύλων και αι ζώναι αυτών αργυρά· και τα κιονόκρανα αυτών ήσαν περικεκαλυμμένα με αργύριον· και πάντες οι στύλοι της αυλής ήσαν εζωσμένοι με αργύριον.
\par 18 Και το καταπέτασμα διά την πύλην της αυλής ήτο εργασίας κεντητού εκ κυανού και πορφυρού και κοκκίνου και βύσσου κεκλωσμένης· και ήτο είκοσι πηχών το μήκος και το ύψος εις το πλάτος πέντε πηχών, καθώς εις τα παραπετάσματα της αυλής.
\par 19 Και οι στύλοι αυτών τέσσαρες και τα χάλκινα υποβάσια αυτών τέσσαρα· τα άγκιστρα αυτών αργυρά, και τα κιονόκρανα αυτών περικεκαλυμμένα με αργύριον και αι ζώναι αυτών αργυραί.
\par 20 Και πάντες οι πάσσαλοι της σκηνής και της αυλής κύκλω χάλκινοι.
\par 21 Αύτη είναι η απαρίθμησις των πραγμάτων της σκηνής, της σκηνής του μαρτυρίου, καθώς ηριθμήθησαν κατά την προσταγήν του Μωϋσέως, διά την υπηρεσίαν των Λευϊτών διά χειρός του Ιθάμαρ, υιού του Ααρών του ιερέως.
\par 22 Και ο Βεσελεήλ ο υιός του Ουρί, υιού του Ωρ, εκ φυλής Ιούδα, έκαμε πάντα όσα προσέταξεν ο Κύριος εις τον Μωϋσήν.
\par 23 Και ήτο μετ' αυτού Ελιάβ, ο υιός του Αχισαμάχ, εκ φυλής Δαν, εγχαράκτης και ευμήχανος τεχνίτης και κεντητής εις κυανούν και εις πορφυρούν και εις κόκκινον και εις βύσσον.
\par 24 Παν το χρυσίον το δαπανηθέν διά την εργασίαν εις όλον το έργον του αγιαστηρίου, το χρυσίον της προσφοράς, ήτο εικοσιεννέα τάλαντα και επτακόσιοι τριάκοντα σίκλοι, κατά τον σίκλον του αγιαστηρίου.
\par 25 Και το αργύριον των απαριθμηθέντων εκ της συναγωγής εκατόν τάλαντα, και χίλιοι επτακόσιοι και εβδομήκοντα πέντε σίκλοι, κατά τον σίκλον του αγιαστηρίου·
\par 26 εν βεκάχ κατά κεφαλήν, το ήμισυ του σίκλου, κατά τον σίκλον του αγιαστηρίου, διά πάντα περνώντα εις την απαρίθμησιν, από είκοσι ετών ηλικίας και επάνω, διά εξακοσίας και τρεις χιλιάδας και πεντακοσίους και πεντήκοντα ανθρώπους.
\par 27 Και εκ του αργυρίου των εκατόν ταλάντων εχύθησαν τα υποβάσια του αγιαστηρίου και τα υποβάσια του καταπετάσματος· εκατόν υποβάσια από εκατόν ταλάντων, εν τάλαντον δι' εν υποβάσιον.
\par 28 Και από των χιλίων επτακοσίων εβδομήκοντα πέντε σίκλων έκαμεν άγκιστρα διά τους στύλους και περιεκάλυψε τα κιονόκρανα αυτών και έζωσεν αυτούς.
\par 29 Και ο χαλκός της προσφοράς ήτο εβδομήκοντα τάλαντα και δύο χιλιάδες και τετρακόσιοι σίκλοι.
\par 30 Και εκ τούτου έκαμε τα υποβάσια εις την θύραν της σκηνής του μαρτυρίου και το χάλκινον θυσιαστήριον και την χαλκίνην εσχάραν δι' αυτό, και πάντα τα σκεύη του θυσιαστηρίου,
\par 31 και τα υποβάσια της αυλής κύκλω και τα υποβάσια της πύλης της αυλής και πάντας τους πασσάλους της σκηνής και πάντας τους πασσάλους της αυλής κύκλω.

\chapter{39}

\par Και εκ του κυανού και πορφυρού και κοκκίνου έκαμον στολάς λειτουργικάς διά να λειτουργώσιν εν τω αγίω, και έκαμον τας αγίας στολάς διά τον Ααρών, καθώς προσέταξεν ο Κύριος εις τον Μωϋσήν.
\par 2 Και έκαμε το εφόδ εκ χρυσίου, εκ κυανού και πορφυρού και κοκκίνου και βύσσου κεκλωσμένης.
\par 3 Και εσφυρηλάτησαν το χρυσίον εις λεπτάς πλάκας και έκοψαν αυτό εις σύρματα, διά να εργασθώσιν αυτό εις το κυανούν και εις το πορφυρούν και εις το κόκκινον και εις την βύσσον με έντεχνον εργασίαν.
\par 4 Έκαμον επωμίδας συναπτάς δι' αυτό· συναπτομένας επί των δύο άκρων αυτού.
\par 5 Και η κεντητή ζώνη του εφόδ επ' αυτό ήτο εκ του αυτού κατά την εργασίαν αυτού· εκ χρυσίου, εκ κυανού και πορφυρού και κοκκίνου και βύσσου κεκλωσμένης, καθώς προσέταξεν ο Κύριος εις τον Μωϋσήν.
\par 6 Και ειργάσθησαν τους ονυχίτας λίθους ενηρμοσμένους εν οικίσκοις χρυσοίς, εγκεχαραγμένους, καθώς εγχαράττονται αι σφραγίδες, με τα ονόματα των υιών Ισραήλ.
\par 7 Και έθεσεν αυτούς επί των επωμίδων του εφόδ, λίθους μνημοσύνου εις τους υιούς Ισραήλ, καθώς προσέταξεν ο Κύριος εις τον Μωϋσήν.
\par 8 Και έκαμε το περιστήθιον εντέχνου εργασίας, κατά την εργασίαν του εφόδ, εκ χρυσίου, εκ κυανού και πορφυρού και κοκκίνου και βύσσου κεκλωσμένης.
\par 9 Τετράγωνον ήτο· διπλούν έκαμον το περιστήθιον· μιας σπιθαμής το μήκος αυτού και μιας σπιθαμής το πλάτος αυτού, διπλού.
\par 10 Και ενήρμοσαν εις αυτό τέσσαρας σειράς λίθων· σειρά σαρδίου, τοπαζίου και σμαράγδου ήτο η σειρά η πρώτη.
\par 11 Και η δευτέρα σειρά, άνθραξ, σάπφειρος και αδάμας.
\par 12 Και η τρίτη σειρά, λιγύριον, αχάτης και αμέθυστος.
\par 13 Και η τετάρτη σειρά, βηρύλλιον, όνυξ και ίασπις· ούτοι ήσαν ενηρμοσμένοι εν οικίσκοις χρυσοίς εις τα περικλείσματα αυτών.
\par 14 Και οι λίθοι ήσαν κατά τα ονόματα των υιών Ισραήλ, δώδεκα, κατά τα ονόματα αυτών, κατά την γλυφήν της σφραγίδος, έκαστος με το όνομα αυτού κατά τας δώδεκα φυλάς.
\par 15 Και έκαμον επί το περιστήθιον αλύσεις κατά τα άκρα, πλεκτής εργασίας εκ χρυσίου καθαρού.
\par 16 Και έκαμον δύο οικίσκους χρυσούς και δύο κρίκους χρυσούς και επέρασαν τους δύο κρίκους εις τα δύο άκρα του περιστηθίου.
\par 17 Και επέρασαν τας δύο πλεκτάς χρυσάς αλύσεις εις τους δύο κρίκους τους εις τα άκρα του περιστηθίου.
\par 18 Και τα δύο άκρα των δύο πλεκτών αλύσεων συνήψαν με τους δύο οικίσκους και έβαλον αυτούς επί των επωμίδων του εφόδ, εις το έμπροσθεν μέρος αυτού.
\par 19 Και έκαμον δύο κρίκους χρυσούς και έβαλον αυτούς επί των δύο άκρων του περιστηθίου, εις το χείλος αυτού, το οποίον ήτο κατά το μέρος του εφόδ έσωθεν.
\par 20 Και έκαμον δύο άλλους κρίκους χρυσούς και έβαλον αυτούς εις τα δύο πλάγια του εφόδ κάτωθεν, προς το εμπροσθινόν μέρος αυτού αντικρύ της άλλης ενώσεως αυτού άνωθεν της κεντητής ζώνης του εφόδ.
\par 21 Και έδεσαν το περιστήθιον διά των κρίκων αυτού εις τους κρίκους του εφόδ με ταινίαν εκ κυανού, διά να ήναι άνωθεν της κεντητής ζώνης του εφόδ, και διά να μη ήναι το περιστήθιον κεχωρισμένον από του εφόδ· καθώς προσέταξεν ο Κύριος εις τον Μωϋσήν.
\par 22 Και έκαμε τον ποδήρη του εφόδ εργασίας υφαντής, όλον εκ κυανού.
\par 23 Και ήτο άνοιγμα εν τω μέσω του ποδήρους, ως άνοιγμα θώρακος, με ταινίαν κύκλω του ανοίγματος, διά να μη σχίζηται.
\par 24 Και έκαμον επί των κρασπέδων του ποδήρους ρόδια εκ κυανού και πορφυρού και κοκκίνου και βύσσου κεκλωσμένης.
\par 25 Και έκαμον κώδωνας εκ χρυσίου καθαρού και έβαλον τους κώδωνας μεταξύ των ροδίων επί του κρασπέδου του ποδήρους κύκλω μεταξύ των ροδίων·
\par 26 κώδωνα και ρόδιον, κώδωνα και ρόδιον, επί των κρασπέδων του ποδήρους του λειτουργικού κύκλω καθώς προσέταξεν ο Κύριος εις τον Μωϋσήν.
\par 27 Και έκαμον τους χιτώνας εκ βύσσου υφαντής εργασίας, διά τον Ααρών και διά τους υιούς αυτού,
\par 28 και την μίτραν εκ βύσσου και τα μιτρίδια κεκοσμημένα εκ βύσσου και τα λινά περισκελή εκ βύσσου κεκλωσμένης,
\par 29 και την ζώνην εκ βύσσου κεκλωσμένης και κυανού και πορφυρού και κοκκίνου, κεντητής εργασίας· καθώς προσέταξεν ο Κύριος εις τον Μωϋσήν.
\par 30 Και έκαμον το πέταλον του ιερού στέμματος εκ χρυσίου καθαρού και ενεχάραξαν επ' αυτό γράμματα ως χάραγμα σφραγίδος, ΑΓΙΑΣΜΟΣ ΕΙΣ ΤΟΝ ΚΥΡΙΟΝ.
\par 31 Και έδεσαν εις αυτό ταινίαν κυανήν, διά να συνάψωσιν αυτό άνωθεν επί της μίτρας· καθώς προσέταξεν ο Κύριος εις τον Μωϋσήν.
\par 32 Ούτως ετελειώθη άπαν το έργον της σκηνής του μαρτυρίου· και έκαμον οι υιοί Ισραήλ κατά πάντα όσα προσέταξεν ο Κύριος εις τον Μωϋσήν· ούτως έκαμον.
\par 33 Και έφεραν την σκηνήν προς τον Μωϋσήν· την σκηνήν, και πάντα τα σκεύη αυτής, τας περόνας αυτής, τας σανίδας αυτής, τους μοχλούς αυτής και τους στύλους αυτής, και τα υποβάσια αυτής,
\par 34 και το επικάλυμμα το εκ δερμάτων κριών κοκκινοβαφών και το επικάλυμμα το εκ δερμάτων θώων και το καλυπτήριον καταπέτασμα,
\par 35 την κιβωτόν του μαρτυρίου και τους μοχλούς αυτής και το ιλαστήριον,
\par 36 την τράπεζαν, πάντα τα σκεύη αυτής και τους άρτους της προθέσεως,
\par 37 την καθαράν λυχνίαν, τους λύχνους αυτής, τους λύχνους κατά την διάταξιν αυτών και πάντα τα σκεύη αυτής και το έλαιον του φωτός,
\par 38 και το χρυσούν θυσιαστήριον και το χριστήριον έλαιον και το ευώδες θυμίαμα και τον τάπητα διά την θύραν της σκηνής,
\par 39 το χάλκινον θυσιαστήριον και την χαλκίνην εσχάραν αυτού, τους μοχλούς αυτού και πάντα τα σκεύη αυτού, τον νιπτήρα και την βάσιν αυτού,
\par 40 τα παραπετάσματα της αυλής, τους στύλους αυτής και τα υποβάσια αυτής και το καταπέτασμα διά την πύλην της αυλής, τα σχοινία αυτής και τους πασσάλους αυτής και πάντα τα σκεύη της υπηρεσίας της σκηνής διά την σκηνήν του μαρτυρίου,
\par 41 τας λειτουργικάς στολάς, διά να λειτουργώσιν εν τω αγίω, και τας αγίας στολάς διά τον Ααρών τον ιερέα και τας στολάς των υιών αυτού, διά να ιερατεύωσι.
\par 42 Κατά πάντα όσα προσέταξεν ο Κύριος εις τον Μωϋσήν, ούτως έκαμον οι υιοί Ισραήλ άπαν το έργον.
\par 43 Και είδεν ο Μωϋσής άπαν το έργον και ιδού, είχον κάμει αυτό καθώς προσέταξεν ο Κύριος· ούτως έκαμον· και ευλόγησεν αυτούς ο Μωϋσής.

\chapter{40}

\par Και ελάλησε Κύριος προς τον Μωϋσήν, λέγων,
\par 2 Την πρώτην ημέραν του πρώτου μηνός θέλεις στήσει την σκηνήν, την σκηνήν του μαρτυρίου.
\par 3 Και θέλεις θέσει εκεί την κιβωτόν του μαρτυρίου, και σκεπάσει την κιβωτόν με το καταπέτασμα.
\par 4 Και θέλεις εισάξει την τράπεζαν και διατάξει τα διατακτέα επ' αυτής· και θέλεις εισάξει την λυχνίαν και ανάψει τους λύχνους αυτής.
\par 5 Και θέλεις θέσει το χρυσούν θυσιαστήριον του θυμιάματος έμπροσθεν της κιβωτού του μαρτυρίου και επιβάλει τον τάπητα της θύρας εις την σκηνήν.
\par 6 Και θέλεις θέσει το θυσιαστήριον του ολοκαυτώματος έμπροσθεν της θύρας της σκηνής, της σκηνής του μαρτυρίου.
\par 7 Και θέλεις θέσει τον νιπτήρα μεταξύ της σκηνής του μαρτυρίου και του θυσιαστηρίου και βάλει ύδωρ εν αυτώ.
\par 8 Και θέλεις στήσει την αυλήν κύκλω και κρεμάσει το καταπέτασμα της πύλης της αυλής.
\par 9 Και θέλεις λάβει το χριστήριον έλαιον και χρίσει την σκηνήν και πάντα τα εν αυτή, και θέλεις αγιάσει αυτήν και πάντα τα σκεύη αυτής και θέλει είσθαι αγία.
\par 10 Και θέλεις χρίσει το θυσιαστήριον του ολοκαυτώματος και πάντα τα σκεύη αυτού και θέλεις αγιάσει το θυσιαστήριον· και θέλει είσθαι θυσιαστήριον αγιώτατον.
\par 11 Και θέλεις χρίσει τον νιπτήρα και την βάσιν αυτού και αγιάσει αυτόν.
\par 12 Και θέλεις προσαγάγει τον Ααρών και τους υιούς αυτού εις την θύραν της σκηνής του μαρτυρίου και νίψει αυτούς με ύδωρ.
\par 13 Και θέλεις ενδύσει τον Ααρών τας αγίας στολάς και θέλεις χρίσει αυτόν, και αγιάσει αυτόν, και θέλει ιερατεύει εις εμέ.
\par 14 Και θέλεις προσαγάγει τους υιούς αυτού και ενδύσει αυτούς χιτώνας.
\par 15 Και θέλεις χρίσει αυτούς, καθώς έχρισας τον πατέρα αυτών, και θέλουσιν ιερατεύει εις εμέ· και θέλει είσθαι εις αυτούς το χρίσμα αυτών προς παντοτεινήν ιερατείαν εις τας γενεάς αυτών.
\par 16 Και έκαμεν ο Μωϋσής κατά πάντα όσα προσέταξεν ο Κύριος εις αυτόν· ούτως έκαμε.
\par 17 Και τον πρώτον μήνα του δευτέρου έτους, την πρώτην του μηνός, εστήθη η σκηνή.
\par 18 Και έστησεν ο Μωϋσής την σκηνήν και έβαλε τα υποβάσια αυτής και έστησε τας σανίδας αυτής και έβαλε τους μοχλούς αυτής και έστησε τους στύλους αυτής.
\par 19 Και εξήπλωσε τα παραπετάσματα επί την σκηνήν, και έβαλε το επικάλυμμα της σκηνής επ' αυτήν άνωθεν· καθώς προσέταξεν ο Κύριος εις τον Μωϋσήν.
\par 20 Και λαβών το μαρτύριον έθεσεν εν τη κιβωτώ, και έβαλε τους μοχλούς εις την κιβωτόν, και έβαλε το ιλαστήριον επί την κιβωτόν άνωθεν,
\par 21 και έφερε την κιβωτόν εις την σκηνήν, και επέθηκε το καλυπτήριον καταπέτασμα και εσκέπασε την κιβωτόν του μαρτυρίου· καθώς προσέταξεν ο Κύριος εις τον Μωϋσήν.
\par 22 Και έθεσε την τράπεζαν εν τη σκηνή του μαρτυρίου κατά το μέρος της σκηνής το προς βορράν έξωθεν του καταπετάσματος,
\par 23 και διέταξεν επ' αυτής τους άρτους τους διατεταγμένους, ενώπιον Κυρίου· καθώς προσέταξεν ο Κύριος εις τον Μωϋσήν.
\par 24 Και έθεσε την λυχνίαν εν τη σκηνή του μαρτυρίου απέναντι της τραπέζης κατά το μέρος της σκηνής το προς μεσημβρίαν,
\par 25 και ανήψε τους λύχνους ενώπιον Κυρίου· καθώς προσέταξεν ο Κύριος εις τον Μωϋσήν.
\par 26 Και έθεσε το χρυσούν θυσιαστήριον εν τη σκηνή του μαρτυρίου απέναντι του καταπετάσματος,
\par 27 και εθυμίασεν επ' αυτού ευώδες θυμίαμα· καθώς προσέταξεν ο Κύριος εις τον Μωϋσήν.
\par 28 Και επέθηκε τον τάπητα εις την θύραν της σκηνής.
\par 29 Και το θυσιαστήριον του ολοκαυτώματος έθεσε παρά την θύραν της σκηνής, της σκηνής του μαρτυρίου, και προσέφερεν επ' αυτού το ολοκαύτωμα και την εξ αλφίτων προσφοράν· καθώς προσέταξεν ο Κύριος εις τον Μωϋσήν.
\par 30 Και έθεσε τον νιπτήρα μεταξύ της σκηνής του μαρτυρίου και του θυσιαστηρίου και έβαλεν εν αυτώ, ύδωρ, διά να νίπτωνται·
\par 31 και ένιπτον εξ αυτού ο Μωϋσής και ο Ααρών και οι υιοί αυτού τας χείρας αυτών και τους πόδας αυτών.
\par 32 Ότε εισήρχοντο εις την σκηνήν του μαρτυρίου και ότε προσήρχοντο εις το θυσιαστήριον, ενίπτοντο καθώς προσέταξεν ο Κύριος εις τον Μωϋσήν.
\par 33 Και έστησε την αυλήν κύκλω της σκηνής και του θυσιαστηρίου και εκρέμασε τον τάπητα της πύλης της αυλής. Και συνετέλεσεν ο Μωϋσής το έργον.
\par 34 Τότε εκάλυψεν η νεφέλη την σκηνήν του μαρτυρίου και δόξα Κυρίου ενέπλησε την σκηνήν.
\par 35 Και δεν ηδυνήθη ο Μωϋσής να εισέλθη εις την σκηνήν του μαρτυρίου· διότι η νεφέλη εκάθητο επ' αυτήν, και δόξα Κυρίου ενέπλησε την σκηνήν.
\par 36 Και ότε η νεφέλη ανέβαινεν επάνωθεν της σκηνής, οι υιοί Ισραήλ εσηκόνοντο καθ' όλας αυτών τας οδοιπορίας·
\par 37 αν όμως η νεφέλη δεν ανέβαινε, τότε δεν εσηκόνοντο μέχρι της ημέρας της αναβάσεως αυτής.
\par 38 διότι η νεφέλη του Κυρίου ήτο επί της σκηνής την ημέραν, και πυρ ήτο επ' αυτής την νύκτα, ενώπιον παντός του οίκου Ισραήλ· καθ' όλας αυτών τας οδοιπορίας.


\end{document}