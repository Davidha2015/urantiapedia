\begin{document}

\title{Leviticus}


\chapter{1}

\par Και εκάλεσε Κύριος τον Μωϋσήν και ελάλησε προς αυτόν εκ της σκηνής του μαρτυρίου, λέγων,
\par 2 Λάλησον προς τους υιούς Ισραήλ και ειπέ προς αυτούς, Εάν τις από σας προσφέρη δώρον προς τον Κύριον, θέλετε προσφέρει το δώρον σας από των κτηνών, από των βοών ή από των προβάτων.
\par 3 Εάν το δώρον αυτού ήναι ολοκαύτωμα από των βοών, αρσενικόν άμωμον ας προσφέρη αυτό· παρά την θύραν της σκηνής του μαρτυρίου θέλει προσφέρει αυτό, διά να ήναι δεκτόν ενώπιον του Κυρίου.
\par 4 Και θέλει επιθέσει την χείρα αυτού επί την κεφαλήν του ολοκαυτώματος και θέλει είσθαι δεκτόν υπέρ αυτού, διά να γείνη εξιλέωσις περί αυτού.
\par 5 Και θέλουσι σφάξει τον μόσχον ενώπιον Κυρίου· και οι υιοί του Ααρών, οι ιερείς, θέλουσι φέρει το αίμα και θέλουσι ραντίσει το αίμα κύκλω επί το θυσιαστήριον το παρά την θύραν της σκηνής του μαρτυρίου.
\par 6 Και θέλουσιν εκδάρει το ολοκαύτωμα και θέλουσι διαμελίσει αυτό κατά τα μέλη αυτού.
\par 7 Και οι υιοί του Ααρών του ιερέως θέλουσι βάλει πυρ επί το θυσιαστήριον και θέλουσι στοιβάσει ξύλα επί το πυρ.
\par 8 Και οι υιοί του Ααρών, οι ιερείς, θέλουσιν επιστοιβάσει τα μέλη, την κεφαλήν και το στέαρ, επί τα ξύλα τα επί του πυρός, του επί του θυσιαστηρίου·
\par 9 τα δε εντόσθια αυτού και τους πόδας αυτού θέλουσι πλύνει με ύδωρ· και θέλει καύσει ο ιερεύς τα πάντα επί του θυσιαστηρίου· ολοκαύτωμα είναι, θυσία γινομένη διά πυρός εις οσμήν ευωδίας προς τον Κύριον.
\par 10 Εάν δε το δώρον αυτού διά το ολοκαύτωμα ήναι εκ των ποιμνίων, εκ των προβάτων ή εκ των αιγών, αρσενικόν άμωμον θέλει προσφέρει αυτό.
\par 11 Και θέλουσι σφάξει αυτό εις τα πλάγια του θυσιαστηρίου προς βορράν ενώπιον Κυρίου· και θέλουσι ραντίσει οι υιοί του Ααρών, οι ιερείς, το αίμα αυτού επί το θυσιαστήριον κύκλω·
\par 12 και θέλουσι διαμελίσει αυτό κατά τα μέλη αυτού και την κεφαλήν αυτού και το στέαρ αυτού· και θέλει επιστοιβάσει αυτά ο ιερεύς επί τα ξύλα τα επί του πυρός του επί του θυσιαστηρίου·
\par 13 τα δε εντόσθια και τους πόδας θέλουσι πλύνει με ύδωρ· και θέλει φέρει τα πάντα ο ιερεύς και καύσει αυτά επί του θυσιαστηρίου· ολοκαύτωμα είναι, θυσία γινομένη διά πυρός εις οσμήν ευωδίας προς τον Κύριον.
\par 14 Και εάν το δώρον αυτού προς τον Κύριον ήναι ολοκαύτωμα από πτηνών, τότε θέλει προσφέρει το δώρον αυτού από τρυγόνων ή από νεοσσών περιστερών.
\par 15 Και θέλει προσαγάγει αυτό ο ιερεύς προς το θυσιαστήριον και θέλει αποκόψει διά των ονύχων την κεφαλήν αυτού και καύσει αυτό επί του θυσιαστηρίου· και το αίμα αυτού θέλει στραγγίσει προς το πλάγιον του θυσιαστηρίου·
\par 16 και θέλει εκβάλει τον πρόλοβον αυτού μετά της κόπρου αυτού και ρίψει αυτά εις τα πλάγια του θυσιαστηρίου κατά ανατολάς, εις τον τόπον της στάκτης·
\par 17 και θέλει διασχίσει αυτό εκ των πτερύγων αυτού, πλην δεν θέλει διαχωρίσει και θέλει καύσει αυτό ο ιερεύς επί του θυσιαστηρίου, επί των ξύλων των επί του πυρός· ολοκαύτωμα είναι, θυσία γινομένη διά πυρός εις οσμήν ευωδίας προς τον Κύριον.

\chapter{2}

\par Εάν δε προσφέρη τις δώρον προσφοράν εξ αλφίτων προς τον Κύριον, σεμίδαλις θέλει είσθαι το δώρον αυτού· και θέλει επιχύσει επ' αυτό έλαιον και επιθέσει επ' αυτό λιβάνιον.
\par 2 Και θέλει φέρει αυτό προς τους υιούς του Ααρών, τους ιερείς· και ο ιερεύς θέλει δράξει από της σεμιδάλεως αυτού και από του ελαίου αυτού όσον χωρεί η χειρ αυτού και άπαν το λιβάνιον αυτού· και θέλει καύσει ο ιερεύς το μνημόσυνον αυτού επί του θυσιαστηρίου· είναι θυσία γινομένη διά πυρός εις οσμήν ευωδίας προς τον Κύριον.
\par 3 Το δε υπόλοιπον της εξ αλφίτων προσφοράς θέλει είσθαι του Ααρών και των υιών αυτού· είναι αγιώτατον εκ των θυσιών των γινομένων διά πυρός προς τον Κύριον.
\par 4 Και όταν προσφέρης δώρον προσφοράν εξ αλφίτων εψημένην εν κλιβάνω, θέλει είσθαι άρτοι άζυμοι από σεμιδάλεως εζυμωμένης μετά ελαίου και λάγανα άζυμα κεχρισμένα με έλαιον.
\par 5 Εάν δε το δώρον σου ήναι προσφορά εξ αλφίτων εψημένη εις κάψαν, άζυμον θέλει είσθαι εκ σεμιδάλεως εζυμωμένης μετά ελαίου.
\par 6 Θέλεις διαμερίσει αυτήν εις τμήματα και θέλεις επιχύσει επ' αυτής έλαιον· είναι προσφορά εξ αλφίτων.
\par 7 Και εάν το δώρον σου ήναι προσφορά εξ αλφίτων εψημένη εν τηγανίω, θέλει γείνει από σεμιδάλεως μετά ελαίου.
\par 8 Και θέλεις φέρει προς τον Κύριον την εξ αλφίτων προσφοράν, την οποίαν έκαμες εκ τούτων· και όταν φερθή προς τον ιερέα, αυτός θέλει πλησιάσει αυτήν εις το θυσιαστήριον.
\par 9 Και θέλει χωρίσει ο ιερεύς από της εξ αλφίτων προσφοράς το μνημόσυνον αυτής και καύσει αυτό επί του θυσιαστηρίου· είναι θυσία γινομένη διά πυρός εις οσμήν ευωδίας προς τον Κύριον.
\par 10 Το δε υπόλοιπον της εξ αλφίτων προσφοράς θέλει είσθαι του Ααρών και των υιών αυτού· είναι αγιώτατον εκ τών διά πυρός γινομένων θυσιών εις τον Κύριον.
\par 11 Ουδεμία προσφορά εξ αλφίτων, την οποίαν προσφέρετε προς τον Κύριον, θέλει είσθαι ένζυμος· διότι ουδέν προζύμιον ουδέ μέλι θέλετε καύσει εις ουδεμίαν θυσίαν γινομένην διά πυρός προς τον Κύριον.
\par 12 Περί δε του δώρου των απαρχών, θέλετε προσφέρει αυτάς προς τον Κύριον· δεν θέλουσιν όμως καυθή επί του θυσιαστηρίου εις οσμήν ευωδίας.
\par 13 Και παν δώρον της εξ αλφίτων προσφοράς σου θέλεις αλατίζει με άλας· και δεν θέλεις αφήσει να λείψη από της εξ αλφίτων προσφοράς σου το άλας της διαθήκης του Θεού σου· επί παντός δώρου σου θέλεις προσφέρει άλας.
\par 14 Και εάν προσφέρης προσφοράν εξ αλφίτων από των πρωτογεννημάτων σου προς τον Κύριον, θέλεις προσφέρει διά την εξ αλφίτων προσφοράν των πρωτογεννημάτων σου αστάχυα χλωρά εψημένα εν πυρί, σίτον πεφρυγανισμένον εξ ασταχύων μεστών.
\par 15 Και θέλεις χύσει επ' αυτήν έλαιον και θέλεις θέσει επ' αυτήν λιβάνιον· είναι προσφορά εξ αλφίτων.
\par 16 Και ο ιερεύς θέλει καύσει το μνημόσυνον αυτής εκ του πεφρυγανισμένον σίτου αυτής και εκ του ελαίου αυτής, μεθ' όλου λιβανίου αυτής· είναι θυσία γινομένη διά πυρός εις τον Κύριον.

\chapter{3}

\par Εάν δε το δώρον αυτού ήναι θυσία ειρηνική, εάν προσφέρη αυτό εκ των βοών, είτε αρσενικόν είτε θηλυκόν, άμωμον θέλει προσφέρει αυτό ενώπιον του Κυρίου·
\par 2 και θέλει επιθέσει την χείρα αυτού επί την κεφαλήν του δώρου αυτού, και θέλουσι σφάξει αυτό παρά την θύραν της σκηνής του μαρτυρίου· και οι υιοί του Ααρών, οι ιερείς, θέλουσι ραντίσει το αίμα επί το θυσιαστήριον κύκλω.
\par 3 Και θέλει προσφέρει εκ της ειρηνικής προσφοράς, θυσίαν γινομένην διά πυρός εις τον Κύριον· το στέαρ το περικαλύπτον τα εντόσθια και άπαν το στέαρ το επί των εντοσθίων·
\par 4 και τους δύο νεφρούς και το στέαρ το επ' αυτών το προς τα πλευρά, και τον επάνω λοβόν του ήπατος, τον οποίον μετά των νεφρών θέλει αφαιρέσει.
\par 5 Και οι υιοί του Ααρών θέλουσι καύσει αυτά επί του θυσιαστηρίου επί του ολοκαυτώματος του επί των ξύλων των επί του πυρός· είναι θυσία γινομένη διά πυρός εις οσμήν ευωδίας προς τον Κύριον.
\par 6 Εάν δε το δώρον αυτού, το εις θυσίαν ειρηνικήν προσφερόμενον προς τον Κύριον, ήναι εκ του ποιμνίου, αρσενικόν ή θηλυκόν, άμωμον θέλει προσφέρει αυτό.
\par 7 Εάν αρνίον προσφέρη το δώρον αυτού, θέλει προσφέρει αυτό ενώπιον του Κυρίου·
\par 8 και θέλει επιθέσει την χείρα αυτού επί την κεφαλήν του δώρου αυτού, και θέλουσι σφάξει αυτό έμπροσθεν της σκηνής του μαρτυρίου· και οι υιοί του Ααρών θέλουσι ραντίσει το αίμα αυτού επί το θυσιαστήριον κύκλω.
\par 9 Και θέλει προσφέρει εκ της προσφοράς της ειρηνικής θυσίαν γινομένην διά πυρός εις τον Κύριον· το στέαρ αυτού, την ουράν ολόκληρον, την οποίαν θέλει αφαιρέσει από της ράχης και το στέαρ το περικαλύπτον τα εντόσθια και παν το στέαρ το επί των εντοσθίων·
\par 10 και τους δύο νεφρούς και το στέαρ το επ' αυτών το προς τα πλευρά, και τον επάνω λοβόν του ήπατος, τον οποίον μετά των νεφρών θέλει αφαιρέσει.
\par 11 Και θέλει καύσει αυτά ο ιερεύς επί του θυσιαστηρίου· είναι τροφή της διά πυρός γινομένης θυσίας εις τον Κύριον.
\par 12 Εάν δε το δώρον αυτού ήναι εξ αιγών, τότε θέλει προσφέρει αυτό ενώπιον του Κυρίου·
\par 13 και θέλει επιθέσει την χείρα αυτού επί την κεφαλήν αυτού, και θέλουσι σφάξει αυτό έμπροσθεν της σκηνής του μαρτυρίου· και οι υιοί του Ααρών θέλουσι ραντίσει το αίμα αυτού επί το θυσιαστήριον κύκλω.
\par 14 Και θέλει προσφέρει εξ αυτού το δώρον αυτού, θυσίαν γινομένην διά πυρός εις τον Κύριον· το στέαρ το περικαλύπτον τα εντόσθια και παν το στέαρ το επί των εντοσθίων·
\par 15 και τους δύο νεφρούς και το στέαρ το επ' αυτών το προς τα πλευρά, και τον επάνω λοβόν του ήπατος, τον οποίον μετά των νεφρών θέλει αφαιρέσει.
\par 16 Και θέλει καύσει αυτά ο ιερεύς επί του θυσιαστηρίου· είναι τροφή της θυσίας της γινομένης διά πυρός εις οσμήν ευωδίας· παν το στέαρ είναι του Κυρίου.
\par 17 Νόμιμον αιώνιον θέλει είσθαι εις τας γενεάς σας, εις πάσας τας κατοικήσεις σας· δεν θέλετε τρώγει ούτε στέαρ ούτε αίμα.

\chapter{4}

\par Και ελάλησε Κύριος προς τον Μωϋσήν, λέγων,
\par 2 Λάλησον προς τους υιούς Ισραήλ, λέγων, Εάν ψυχή τις αμαρτήση εξ αγνοίας και εκ των όσα είναι προστεταγμένον υπό του Κυρίου να μη πράττωνται πράξη τι εξ αυτών·
\par 3 εάν μεν ο ιερεύς ο κεχρισμένος αμαρτήση ώστε να ενοχοποιήση τον λαόν, τότε θέλει φέρει, διά την αμαρτίαν αυτού την οποίαν ημάρτησε, μόσχον βοός άμωμον προς τον Κύριον διά προσφοράν περί αμαρτίας.
\par 4 Και θέλει φέρει τον μόσχον εις την θύραν της σκηνής του μαρτυρίου ενώπιον του Κυρίου· και θέλει επιθέσει την χείρα αυτού επί την κεφαλήν του μόσχου, και θέλουσι σφάξει τον μόσχον ενώπιον του Κυρίου.
\par 5 Και θέλει λάβει ο ιερεύς ο κεχρισμένος από του αίματος του μόσχου και φέρει αυτό εις την σκηνήν του μαρτυρίου·
\par 6 και θέλει εμβάψει ο ιερεύς τον δάκτυλον αυτού εις το αίμα και θέλει ραντίσει από του αίματος επτάκις ενώπιον του Κυρίου, έμπροσθεν του καταπετάσματος του αγιαστηρίου.
\par 7 Και θέλει βάλει ο ιερεύς από του αίματος επί τα κέρατα του θυσιαστηρίου του ευώδους θυμιάματος, το οποίον είναι ενώπιον του Κυρίου εν τη σκηνή του μαρτυρίου· και θέλει χύσει παν το αίμα του μόσχου εις την βάσιν του θυσιαστηρίου του ολοκαυτώματος, το οποίον είναι εις την θύραν της σκηνής του μαρτυρίου.
\par 8 Και άπαν το στέαρ του μόσχου της περί αμαρτίας προσφοράς θέλει αφαιρέσει απ' αυτού· το στέαρ το περικαλύπτον τα εντόσθια, και άπαν το στέαρ το επί των εντοσθίων·
\par 9 και τους δύο νεφρούς και το στέαρ το επ' αυτών το προς τα πλευρά, και τον επάνω λοβόν του ήπατος, τον οποίον μετά των νεφρών θέλει αφαιρέσει,
\par 10 καθ' ον τρόπον αφαιρείται από του μόσχου της θυσίας της ειρηνικής· και θέλει καύσει αυτά ο ιερεύς επί του θυσιαστηρίου του ολοκαυτώματος·
\par 11 και το δέρμα του μόσχου και παν το κρέας αυτού μετά της κεφαλής αυτού και μετά των ποδών αυτού και τα εντόσθια αυτού και την κόπρον αυτού·
\par 12 και θέλει φέρει όλον τον μόσχον έξω του στρατοπέδου εις τόπον καθαρόν, όπου χύνεται η στάκτη, και θέλει καύσει αυτόν επί ξύλων διά πυρός· όπου χύνεται η στάκτη, θέλει καυθή.
\par 13 Εάν δε πάσα η συναγωγή του Ισραήλ αμαρτήση εξ αγνοίας και το πράγμα κρυφθή από των οφθαλμών της συναγωγής και εκ των όσα είναι προστεταγμένον υπό του Κυρίου να μη πράττωνται, πράξωσι και ήναι ένοχοι·
\par 14 όταν γνωρισθή η αμαρτία την οποίαν ημάρτησαν κατά τούτο, τότε θέλει προσφέρει η συναγωγή μόσχον εκ βοών διά την αμαρτίαν και θέλει φέρει αυτόν έμπροσθεν της σκηνής του μαρτυρίου.
\par 15 Και οι πρεσβύτεροι της συναγωγής θέλουσιν επιθέσει τας χείρας αυτών επί την κεφαλήν του μόσχου ενώπιον του Κυρίου· και θέλουσι σφάξει τον μόσχον ενώπιον του Κυρίου.
\par 16 Και ο ιερεύς ο κεχρισμένος θέλει φέρει από του αίματος του μόσχον εις την σκηνήν του μαρτυρίου·
\par 17 και θέλει εμβάψει ο ιερεύς τον δάκτυλον αυτού εις το αίμα και θέλει ραντίσει επτάκις ενώπιον του Κυρίου έμπροσθεν του καταπετάσματος·
\par 18 και θέλει βάλει από του αίματος επί τα κέρατα του θυσιαστηρίου, του ενώπιον του Κυρίου, το οποίον είναι εν τη σκηνή του μαρτυρίου· και θέλει χύσει παν το αίμα εις την βάσιν του θυσιαστηρίου του ολοκαυτώματος, το οποίον είναι εις την θύραν της σκηνής του μαρτυρίου.
\par 19 Και παν το στέαρ αυτού θέλει αφαιρέσει απ' αυτού και καύσει επί του θυσιαστηρίου.
\par 20 Και θέλει κάμει εις τον μόσχον καθ' ον τρόπον έκαμεν εις τον μόσχον της περί αμαρτίας προσφοράς· ούτω θέλει κάμει εις αυτόν· και θέλει κάμει εξιλέωσιν υπέρ αυτών ο ιερεύς και θέλει συγχωρηθή εις αυτούς.
\par 21 Και θέλει εκβάλει τον μόσχον έξω του στρατοπέδου και καύσει αυτόν, καθώς έκαυσε τον πρώτον μόσχον· είναι προσφορά περί αμαρτίας υπέρ της συναγωγής·
\par 22 Όταν δε άρχων τις αμαρτήση και πράξη εξ αγνοίας τι εκ των όσα είναι προστεταγμένον υπό Κυρίου του Θεού αυτού να μη πράττωνται, και ήναι ένοχος·
\par 23 ή εάν η αμαρτία αυτού, την οποίαν ημάρτησε, γνωστοποιηθή εις αυτόν, τότε θέλει φέρει την προσφοράν αυτού, τράγον εξ αιγών, αρσενικόν άμωμον.
\par 24 και θέλει επιθέσει την χείρα αυτού επί την κεφαλήν του τράγου, και θέλουσι σφάξει αυτό εν τω τόπω όπου σφάζουσι το ολοκαύτωμα ενώπιον του Κυρίου· είναι προσφορά περί αμαρτίας.
\par 25 Και θέλει λάβει ο ιερεύς από του αίματος της περί αμαρτίας προσφοράς διά του δακτύλου αυτού, και βάλει επί τα κέρατα του θυσιαστηρίου του ολοκαυτώματος και θέλει χύσει το αίμα αυτού εις την βάσιν του θυσιαστηρίου του ολοκαυτώματος.
\par 26 Και άπαν το στέαρ αυτού θέλει καύσει επί του θυσιαστηρίου, ως το στέαρ της θυσίας της ειρηνικής προσφοράς· και θέλει κάμει εξιλέωσιν υπέρ αυτού ο ιερεύς περί της αμαρτίας αυτού, και θέλει συγχωρηθή εις αυτόν.
\par 27 Εάν δε ψυχή τις εκ του λαού της γης αμαρτήση εξ αγνοίας, πράττων τι εκ των όσα είναι προστεταγμένον υπό του Κυρίου να μη πράττωνται, και ήναι ένοχος.
\par 28 ή εάν γνωστοποιηθή εις αυτόν η αμαρτία αυτού την οποίαν ημάρτησε· τότε θέλει φέρει την προσφοράν αυτού, τράγον εξ αιγών, θηλυκόν άμωμον, διά την αμαρτίαν αυτού την οποίαν ημάρτησε·
\par 29 και θέλει επιθέσει την χείρα αυτού επί την κεφαλήν της περί αμαρτίας προσφοράς, και θέλουσι σφάξει την περί αμαρτίας προσφοράν εν τω τόπω του ολοκαυτώματος.
\par 30 Και θέλει λάβει ο ιερεύς διά του δακτύλου αυτού από του αίματος αυτού, και βάλει επί τα κέρατα του θυσιαστηρίου του ολοκαυτώματος και παν το αίμα αυτού θέλει χύσει εις την βάσιν του θυσιαστηρίου·
\par 31 και παν το στέαρ αυτού θέλει αφαιρέσει, καθώς αφαιρείται το στέαρ από της θυσίας της ειρηνικής προσφοράς· και θέλει καύσει αυτό ο ιερεύς επί του θυσιαστηρίου εις οσμήν ευωδίας προς τον Κύριον· και θέλει κάμει εξιλέωσιν υπέρ αυτού ο ιερεύς, και θέλει συγχωρηθή εις αυτόν.
\par 32 Εάν δε φέρη πρόβατον διά προσφοράν αυτού περί αμαρτίας, θέλει φέρει αυτό θηλυκόν άμωμον·
\par 33 και θέλει επιθέσει την χείρα αυτού επί την κεφαλήν της περί αμαρτίας προσφοράς, και θέλουσι σφάξει αυτό διά προσφοράν περί αμαρτίας, εν τω τόπω όπου σφάζουσι το ολοκαύτωμα.
\par 34 Και θέλει λάβει ο ιερεύς από του αίματος της περί αμαρτίας προσφοράς διά του δακτύλου αυτού και βάλει επί τα κέρατα του θυσιαστηρίου του ολοκαυτώματος και άπαν το αίμα αυτού θέλει χύσει εις την βάσιν του θυσιαστηρίου·
\par 35 και θέλει αφαιρέσει παν το στέαρ αυτού, καθώς αφαιρείται το στέαρ του προβάτου από της θυσίας της ειρηνικής προσφοράς· και θέλει καύσει αυτά ο ιερεύς επί του θυσιαστηρίου κατά τας προσφοράς τας γινομένας διά πυρός εις τον Κύριον· και θέλει κάμει ο ιερεύς εξιλέωσιν περί της αμαρτίας αυτού την οποίαν ημάρτησε, και θέλει συγχωρηθή εις αυτόν.

\chapter{5}

\par Εάν δε τις αμαρτήση και ακούση φωνήν ορκισμού και ήναι μάρτυς, είτε είδεν είτε εξεύρει εάν δεν φανερώση αυτό, τότε θέλει βαστάσει την ανομίαν αυτού.
\par 2 Η εάν τις εγγίση πράγμά τι ακάθαρτον, είτε θνησιμαίον ακαθάρτον θηρίου είτε θνησιμαίον ακαθάρτου κτήνους είτε θνησιμαίον ερπετών ακαθάρτων, και έλαθεν αυτόν, όμως θέλει είσθαι ακάθαρτος και ένοχος.
\par 3 Η εάν εγγίση ακαθαρσίαν ανθρώπου, εξ οποιασδήποτε ήθελεν είσθαι η ακαθαρσία αυτού, διά της οποίας μιαίνεταί τις, και έλαθεν αυτόν· όταν αυτός γνωρίση τούτο, τότε θέλει είσθαι ένοχος.
\par 4 Η εάν τις ομόση, προφέρων αστοχάστως διά των χειλέων αυτού να κακοποιήση, ή να αγαθοποιήση εις παν ό,τι ήθελε προφέρει αστοχάστως ο άνθρωπος μεθ' όρκου και έλαθεν αυτόν· όταν γνωρίση τούτο, τότε θέλει είσθαι ένοχος εις εν εκ τούτων.
\par 5 Όταν λοιπόν είναι τις ένοχος εις εν εκ τούτων, θέλει εξομολογηθή κατά τι ημάρτησε·
\par 6 και θέλει φέρει προς τον Κύριον προσφοράν περί της παραβάσεως αυτού, διά την αμαρτίαν αυτού την οποίαν ημάρτησε, θηλυκόν αρνίον εκ προβάτων ή τράγον εξ αιγών, εις προσφοράν περί αμαρτίας· και θέλει κάμει εξιλέωσιν ο ιερεύς υπέρ αυτού περί της αμαρτίας αυτού.
\par 7 Και εάν δεν ευπορή να φέρη πρόβατον ή αίγα, θέλει φέρει προς τον Κύριον, διά την αμαρτίαν αυτού την οποίαν ημάρτησε, δύο τρυγόνας ή δύο νεοσσούς περιστερών· μίαν διά προσφοράν περί αμαρτίας και μίαν διά ολοκαύτωμα.
\par 8 Και θέλει φέρει αυτάς προς τον ιερέα, όστις θέλει προσφέρει πρώτον εκείνην την περί αμαρτίας προσφοράν· και θέλει κόψει διά των ονύχων την κεφαλήν αυτής από του αυχένος αυτής, πλην δεν θέλει διαχωρίσει αυτήν.
\par 9 Και από του αίματος της περί αμαρτίας προσφοράς θέλει ραντίσει τον τοίχον του θυσιαστηρίου· το δε εναπολειφθέν του αίματος θέλει στραγγίσει έξω εις την βάσιν του θυσιαστηρίου· είναι προσφορά περί αμαρτίας.
\par 10 Την δε δευτέραν θέλει κάμει ολοκαύτωμα κατά το διατεταγμένον· και θέλει κάμει ο ιερεύς εξιλέωσιν υπέρ αυτού, περί της αμαρτίας αυτού την οποίαν ημάρτησε, και θέλει συγχωρηθή εις αυτόν.
\par 11 Αλλ' εάν δεν ευπορή να φέρη δύο τρυγόνας ή δύο νεοσσούς περιστερών, τότε θέλει φέρει ο αμαρτήσας διά προσφοράν αυτού το δέκατον ενός εφά σεμιδάλεως εις προσφοράν περί αμαρτίας· δεν θέλει βάλει επ' αυτήν έλαιον ουδέ θέλει βάλει επ' αυτήν λιβάνιον· διότι είναι προσφορά περί αμαρτίας.
\par 12 και θέλει φέρει αυτήν προς τον ιερέα· και ο ιερεύς θέλει δράξει απ' αυτής όσον χωρεί η χειρ αυτού, το μνημόσυνον αυτής, και θέλει καύσει αυτό επί του θυσιαστηρίου, κατά τας προσφοράς τας διά πυρός γινομένας εις τον Κύριον· είναι προσφορά περί αμαρτίας.
\par 13 Και θέλει κάμει ο ιερεύς εξιλέωσιν υπέρ αυτού περί της αμαρτίας αυτού την οποίαν ημάρτησεν εις εν εκ τούτων, και θέλει συγχωρηθή εις αυτόν· το δε υπόλοιπον θέλει είσθαι του ιερέως, ως η εξ αλφίτων προσφορά.
\par 14 Και ελάλησε Κύριος προς τον Μωϋσήν, λέγων,
\par 15 Εάν τις πράξη παρανομίαν και αμαρτήση εξ αγνοίας εις τα άγια του Κυρίου, τότε θέλει φέρει προς τον Κύριον διά την ανομίαν αυτού κριόν άμωμον εκ του ποιμνίου, κατά την εκτίμησίν σου εις σίκλους αργυρίου, κατά τον σίκλον του αγιαστηρίου, διά προσφοράν περί ανομίας·
\par 16 και θέλει αποδώσει ό,τι ημάρτησεν εις τα άγια και θέλει προσθέσει επ' αυτό το πέμπτον αυτού και δώσει αυτό εις τον ιερέα· και θέλει κάμει ο ιερεύς εξιλέωσιν υπέρ αυτού διά του κριού της περί ανομίας προσφοράς, και θέλει συγχωρηθή εις αυτόν.
\par 17 Και εάν τις αμαρτήση και πράξη τι εκ των όσα είναι προστεταγμένον υπό του Κυρίου να μη πράττωνται, και δεν εγνώρισεν αυτό, όμως θέλει είσθαι ένοχος και θέλει βαστάσει την ανομίαν αυτού·
\par 18 και θέλει φέρει κριόν άμωμον εκ του ποιμνίου κατά την εκτίμησίν σου, εις προσφοράν περί ανομίας, προς τον ιερέα· και θέλει κάμει ο ιερεύς εξιλέωσιν υπέρ αυτού περί της αγνοίας αυτού, εις την οποίαν ελανθάσθη και δεν εγνώρισε τούτο, και θέλει συγχωρηθή εις αυτόν.
\par 19 Είναι προσφορά περί ανομίας· αυτός ανομίαν έπραξε κατά του Κυρίου.

\chapter{6}

\par Και ελάλησε Κύριος προς τον Μωϋσήν, λέγων,
\par 2 Εάν τις αμαρτήση και πράξη παρανομίαν κατά του Κυρίου και ψευσθή προς τον πλησίον αυτού διά παρακαταθήκην ή διά πράγμά τι εμπεπιστευμένον εις τας χείρας αυτού, ή διά αρπαγήν, ή ηπάτησε τον πλησίον αυτού,
\par 3 ή εύρε πράγμα χαμένον και ψεύδεται περί αυτού, ή ομόση ψευδώς περί τινός εκ πάντων όσα πράττει ο άνθρωπος, ώστε να αμαρτήση εις αυτά·
\par 4 όταν αμαρτήση και ήναι ένοχος, θέλει αποδώσει το άρπαγμα το οποίον ήρπασεν, ή το πράγμα το οποίον έλαβε δι' απάτης, ή την παρακαταθήκην την εμπιστευθείσαν εις αυτόν, ή το χαμένον πράγμα το οποίον εύρεν,
\par 5 ή παν εκείνο περί του οποίου ώμοσε ψευδώς· θέλει αποδώσει το κεφάλαιον αυτού, και θέλει προσθέσει το πέμπτον επ' αυτό· εις όντινα ανήκει, εις τούτον θέλει αποδώσει αυτό την ημέραν καθ' ην φανερωθή ένοχος.
\par 6 Και θέλει φέρει προς τον Κύριον την περί ανομίας προσφοράν αυτού, κριόν άμωμον εκ του ποιμνίου, κατά την εκτίμησίν σου, εις προσφοράν περί ανομίας, προς τον ιερέα·
\par 7 και ο ιερεύς θέλει κάμει εξιλέωσιν υπέρ αυτού ενώπιον του Κυρίου· και θέλει συγχωρηθή εις αυτόν, περί παντός πράγματος εκ των όσα έπραξε, ώστε να ανομήση εις αυτά.
\par 8 Και ελάλησε Κύριος προς τον Μωϋσήν, λέγων,
\par 9 Πρόσταξον τον Ααρών και τους υιούς αυτού, λέγων, Ούτος είναι ο νόμος του ολοκαυτώματος· το ολοκαύτωμα θέλει καίεσθαι επί του θυσιαστηρίου όλην την νύκτα έως το πρωΐ, και το πυρ του θυσιαστηρίου θέλει καίεσθαι επ' αυτού.
\par 10 Και θέλει ενδυθή ο ιερεύς χιτώνα λινούν και περισκελή λινά θέλει φορέσει επί την σάρκα αυτού, και θέλει αφαιρέσει την στάκτην του ολοκαυτώματος το οποίον κατέφαγε το πυρ επί του θυσιαστηρίου· και θέλει βάλει αυτήν εις το πλάγιον του θυσιαστηρίου.
\par 11 Και θέλει εκδυθή την στολήν αυτού και ενδυθή άλλην στολήν· και θέλει φέρει την στάκτην έξω του στρατοπέδου εις τόπον καθαρόν.
\par 12 Και το πυρ το επί του θυσιαστηρίου θέλει καίεσθαι επ' αυτού· δεν θέλει σβεσθή· και θέλει καίει ο ιερεύς επ' αυτού ξύλα καθ' εκάστην πρωΐαν, και θέλει στοιβάσει το ολοκαύτωμα επ' αυτού, και θέλει καίει επ' αυτού το στέαρ της ειρηνικής προσφοράς.
\par 13 Το πυρ θέλει καίεσθαι διαπαντός επί του θυσιαστηρίου· δεν θέλει σβεσθή.
\par 14 Ούτος δε είναι ο νόμος της αλφίτων προσφοράς· οι υιοί του Ααρών θέλουσι προσφέρει αυτήν ενώπιον του Κυρίου έμπροσθεν του θυσιαστηρίου.
\par 15 Και θέλει αφαιρέσει απ' αυτής όσον χωρεί η χειρ αυτού, από της σεμιδάλεως της εξ αλφίτων προσφοράς μετά του ελαίου αυτής και παν το λιβάνιον το επί της εξ αλφίτων προσφοράς· και θέλει καύσει αυτό επί του θυσιαστηρίου εις οσμήν ευωδίας, μνημόσυνον αυτής προς τον Κύριον.
\par 16 Το δε εναπολειφθέν εκ τούτων θέλουσι φάγει ο Ααρών και οι υιοί αυτού· άζυμον θέλει τρώγεσθαι εν τόπω αγίω εν τη αυλή της σκηνής του μαρτυρίου θέλουσι τρώγει αυτό.
\par 17 Δεν θέλει εψηθή μετά προζυμίου· διά μερίδιον αυτών έδωκα αυτό από των διά πυρός γινομένων προσφορών μου· είναι αγιώτατον, καθώς η περί αμαρτίας προσφορά, και καθώς η περί ανομίας.
\par 18 Παν αρσενικόν μεταξύ των τέκνων του Ααρών θέλει τρώγει αυτό· τούτο θέλει είσθαι νόμιμον αιώνιον εις τας γενεάς σας, από των διά πυρός γινομένων προσφορών του Κυρίου· πας όστις εγγίση αυτά, θέλει αγιασθή.
\par 19 Και ελάλησε Κύριος προς τον Μωϋσήν, λέγων,
\par 20 τούτο είναι το δώρον του Ααρών και των υιών αυτού το οποίον θέλουσι προσφέρει προς τον Κύριον την ημέραν καθ' ην χρισθή· το δέκατον ενός εφά σεμιδάλεως εις παντοτεινήν προσφοράν εξ αλφίτων, το ήμισυ αυτής το πρωΐ, και το ήμισυ αυτής το εσπέρας·
\par 21 επί κάψης θέλει ετοιμασθή μετά ελαίου· εψημένον θέλεις φέρει αυτό· και τα εψημένα τμήματα των εξ αλφίτων προσφορών θέλεις προσφέρει εις οσμήν ευωδίας προς τον Κύριον.
\par 22 Και ο ιερεύς ο κεχρισμένος αντ' αυτού μεταξύ των υιών αυτού θέλει προσφέρει αυτό· τούτο είναι νόμιμον αιώνιον διά τον Κύριον. ολοκλήρως θέλει καίεσθαι.
\par 23 Και πάσα προσφορά εξ αλφίτων ιερέως θέλει καίεσθαι ολοκλήρως· δεν θέλει τρώγεσθαι.
\par 24 Και ελάλησε Κύριος προς τον Μωϋσήν, λέγων,
\par 25 Λάλησον προς τον Ααρών και προς τους υιούς αυτού, λέγων, Ούτος είναι ο νόμος της περί αμαρτίας προσφοράς· Εν τω τόπω όπου σφάζεται το ολοκαύτωμα, θέλει σφαγή η περί αμαρτίας προσφορά έμπροσθεν του Κυρίου· είναι αγιώτατον.
\par 26 Ο ιερεύς ο προσφέρων αυτήν περί αμαρτίας θέλει τρώγει αυτήν· εν τόπω αγίω θέλει τρώγεσθαι, εν τη αυλή της σκηνής του μαρτυρίου.
\par 27 παν ό,τι εγγίση το κρέας αυτής θέλει είσθαι άγιον· και εάν ραντισθή από του αίματος αυτής επί τι φόρεμα, εκείνο, επί του οποίου ερραντίσθη, θέλει πλύνεσθαι εν τόπω αγίω.
\par 28 Το δε πήλινον αγγείον εν τω οποίω έβρασε, θέλει συντρίβεσθαι αλλ' εάν βράση εν αγγείω χαλκίνω, τούτο θέλει τρίβεσθαι επιμελώς και θέλει πλύνεσθαι με ύδωρ.
\par 29 Παν αρσενικόν μεταξύ των ιερέων θέλει τρώγει εξ αυτής· είναι αγιώτατον.
\par 30 Και πάσα προσφορά περί αμαρτίας, από του αίματος της οποίας φέρεται εις την σκηνήν του μαρτυρίου διά να γείνη εξιλέωσις εν τω αγιαστηρίω, δεν θέλει τρώγεσθαι με πυρ θέλει καίεσθαι.

\chapter{7}

\par Ούτος δε είναι ο νόμος της περί ανομίας προσφοράς· είναι αγιώτατον.
\par 2 Εν τω τόπω όπου σφάζουσι το ολοκαύτωμα, θέλουσι σφάζει την περί ανομίας προσφοράν· και το αίμα αυτής θέλει ραντίζεσθαι επί το θυσιαστήριον κύκλω.
\par 3 Και θέλει προσφέρεσθαι εξ αυτής παν το στέαρ αυτής, η ουρά και το στέαρ το περικαλύπτον τα εντόσθια,
\par 4 και οι δύο νεφροί και το στέαρ το επ' αυτών, το προς τα πλευρά, και ο επάνω λοβός του ήπατος, όστις μετά των νεφρών θέλει αφαιρείσθαι.
\par 5 Και θέλει καίει αυτά ο ιερεύς επί του θυσιαστηρίου, εις προσφοράν γινομένην διά πυρός προς τον Κύριον· είναι προσφορά περί ανομίας.
\par 6 Παν αρσενικόν μεταξύ των ιερέων θέλει τρώγει αυτήν· εν τόπω αγίω θέλει τρώγεσθαι· είναι αγιώτατον.
\par 7 Καθώς είναι η περί αμαρτίας προσφορά, ούτω και η περί ανομίας προσφορά· εις νόμος είναι περί αυτών· ο ιερεύς, όστις κάμνει εξιλέωσιν δι' αυτής, θέλει λαμβάνει αυτήν.
\par 8 Ο δε ιερεύς όστις προσφέρει ολοκαύτωμα τινός, ο ιερεύς θέλει λαμβάνει δι' εαυτόν το δέρμα του ολοκαυτώματος, το οποίον προσέφερε.
\par 9 Και πάσα προσφορά εξ αλφίτων, ήτις ήθελεν εψηθή εν κλιβάνω, και παν ό,τι ετοιμάζεται εν τηγανίω και επί κάψης, θέλει είσθαι του ιερέως του προσφέροντος αυτήν.
\par 10 Και πάσα προσφορά εξ αλφίτων, εζυμωμένη μετά ελαίου ή ξηρά, θέλει είσθαι πάντων των υιών του Ααρών, ίσον το μερίδιον εκάστου.
\par 11 Και ούτος είναι ο νόμος της θυσίας της ειρηνικής προσφοράς, την οποίαν θέλει προσφέρει τις εις τον Κύριον.
\par 12 Εάν προσφέρη αυτήν περί ευχαριστίας, τότε θέλει προσφέρει μετά της ευχαριστηρίου προσφοράς, πήττας αζύμους εζυμωμένας με έλαιον και λάγανα άζυμα κεχρισμένα μετά ελαίου και σεμίδαλιν κατεσκευασμένην, πήττας εζυμωμένας μετά ελαίου.
\par 13 Με τας πήττας άρτον ένζυμον θέλει προσφέρει διά το δώρον αυτού μετά της προς ευχαριστίαν αυτού ειρηνικής προσφοράς.
\par 14 Και εκ τούτων θέλει προσφέρει εν από πάντων των δώρων αυτού προσφοράν υψουμένην προς τον Κύριον· τούτο θέλει είσθαι του ιερέως του ραντίζοντος το αίμα της ειρηνικής προσφοράς.
\par 15 Και το κρέας της θυσίας της προς ευχαριστίαν ειρηνικής αυτού προσφοράς θέλει τρώγεσθαι την αυτήν ημέραν καθ' ην προσφέρεται δεν θέλουσιν αφήσει απ' αυτού έως το πρωΐ.
\par 16 Και εάν η θυσία της προσφοράς αυτού ήναι ευχή, ή προσφορά προαιρετική, θέλει τρώγεσθαι την αυτήν ημέραν καθ' ην προσφέρει τις την θυσίαν αυτού· και εάν μείνη τι, τούτο θέλει τρώγεσθαι την επαύριον.
\par 17 Το εναπολειφθέν όμως του κρέατος της θυσίας έως της τρίτης ημέρας με πυρ θέλει καίεσθαι.
\par 18 Εάν δε φαγωθή τι από του κρέατος της θυσίας της ειρηνικής προσφοράς αυτού την τρίτην ημέραν, δεν θέλει είσθαι δεκτός ο προσφέρων αυτήν, ουδέ θέλει λογισθή εις αυτόν· βδέλυγμα θέλει είσθαι· η δε ψυχή, ήτις ήθελε φάγει απ' αυτού, θέλει βαστάσει την ανομίαν αυτής.
\par 19 Και το κρέας, το οποίον ήθελεν εγγίσει ακάθαρτόν τι, δεν θέλει τρώγεσθαι· εν πυρί θέλει καίεσθαι περί δε του κρέατος, όστις είναι καθαρός θέλει τρώγει κρέας.
\par 20 Η δε ψυχή ήτις έχουσα την ακαθαρσίαν αυτής εφ' εαυτής ήθελε φάγει από του κρέατος της θυσίας της ειρηνικής προσφοράς, ήτις είναι του Κυρίου, η ψυχή αύτη θέλει απολεσθή εκ του λαού αυτής.
\par 21 Και η ψυχή ήτις ήθελεν εγγίσει ακάθαρτόν τι, ακαθαρσίαν ανθρώπου ή ζώον ακάθαρτον ή βδελυρόν τι ακάθαρτον, και φάγει από του κρέατος της θυσίας της ειρηνικής προσφοράς, ήτις είναι του Κυρίου, και η ψυχή αύτη θέλει απολεσθή εκ του λαού αυτής.
\par 22 Και ελάλησε Κύριος προς τον Μωϋσήν, λέγων,
\par 23 Λάλησον προς τους υιούς Ισραήλ, λέγων, Δεν θέλετε τρώγει παντελώς στέαρ βοός ή προβάτου ή αιγός.
\par 24 Και το στέαρ του θνησιμαίου ζώου και το στέαρ του θηριαλώτου δύναται να χρησιμεύη εις πάσαν άλλην χρείαν· δεν θέλετε όμως τρώγει διόλου απ' αυτού.
\par 25 Διότι όστις φάγη το στέαρ του ζώου, από του οποίου προσφέρεται θυσία γινομένη διά πυρός εις τον Κύριον, και εκείνη ψυχή, ήτις ήθελε φάγει θέλει απολεσθή εκ του λαού αυτής.
\par 26 Παρομοίως δεν θέλετε τρώγει ουδέν αίμα, είτε πτηνού είτε ζώου εν ουδεμιά εκ των κατοικιών σας.
\par 27 Πάσα ψυχή ήτις ήθελε φάγει οποιονδήποτε αίμα, και εκείνη η ψυχή θέλει απολεσθή εκ του λαού αυτής.
\par 28 Και ελάλησε Κύριος προς τον Μωϋσήν, λέγων,
\par 29 Λάλησον προς τους υιούς Ισραήλ, λέγων, Ο προσφέρων την θυσίαν της ειρηνικής προσφοράς αυτού προς τον Κύριον, θέλει φέρει το δώρον αυτού προς τον Κύριον, από της θυσίας της ειρηνικής προσφοράς αυτού.
\par 30 Αι χείρες αυτού θέλουσι φέρει τας διά πυρός γινομένας προσφοράς του Κυρίου· θέλει φέρει το στέαρ μετά του στήθους, διά να κινήται το στήθος ως προσφορά κινητή έμπροσθεν του Κυρίου.
\par 31 Και ο ιερεύς θέλει καίει το στέαρ επί του θυσιαστηρίου· το στήθος όμως θέλει είσθαι του Ααρών και των υιών αυτού.
\par 32 Και θέλετε δίδει προς τον ιερέα προσφοράν υψουμένην, τον δεξιόν ώμον εκ των θυσιών της ειρηνικής προσφοράς σας.
\par 33 Όστις εκ των υιών του Ααρών προσφέρει το αίμα της ειρηνικής προσφοράς και το στέαρ, θέλει λαμβάνει τον δεξιόν ώμον εις μερίδιον αυτού.
\par 34 Διότι το κινητόν στήθος και τον υψούμενον ώμον έλαβον παρά των υιών Ισραήλ εκ των θυσιών της ειρηνικής προσφοράς αυτών, και έδωκα αυτά προς τον Ααρών τον ιερέα και προς τους υιούς αυτού εις νόμιμον αιώνιον μεταξύ των υιών Ισραήλ.
\par 35 Τούτο είναι το χρίσμα του Ααρών, και το χρίσμα των υιών αυτού, από των διά πυρός γινομένων προσφορών του Κυρίου, την ημέραν καθ' ην παρέστησεν αυτούς διά να ιερατεύωσιν εις τον Κύριον·
\par 36 το οποίον προσέταξεν ο Κύριος να δίδωται εις αυτούς παρά των υιών Ισραήλ, καθ' ην ημέραν έχρισεν αυτούς, εις νόμιμον αιώνιον εις τας γενεάς αυτών.
\par 37 Ούτος είναι ο νόμος του ολοκαυτώματος, της εξ αλφίτων προσφοράς και της περί αμαρτίας προσφοράς και της περί ανομίας προσφοράς και των καθιερώσεων και της θυσίας της ειρηνικής προσφοράς·
\par 38 τον οποίον προσέταξεν ο Κύριος εις τον Μωϋσήν εν τω όρει Σινά, καθ' ην ημέραν προσέταξε τους υιούς Ισραήλ να προσφέρωσι προς τον Κύριον τα δώρα αυτών εν τη ερήμω Σινά.

\chapter{8}

\par Και ελάλησε Κύριος προς τον Μωϋσήν, λέγων,
\par 2 Λάβε τον Ααρών και τους υιούς αυτού μετ' αυτού, και τας στολάς και το έλαιον του χρίσματος και τον μόσχον της περί αμαρτίας προσφοράς και τους δύο κριούς και το κάνιστρον των αζύμων.
\par 3 Και σύναξον πάσαν την συναγωγήν εις την θύραν της σκηνής του μαρτυρίου.
\par 4 Και έκαμεν ο Μωϋσής καθώς προσέταξεν εις αυτόν ο Κύριος· και συνήχθη η συναγωγή εις την θύραν της σκηνής του μαρτυρίου.
\par 5 Και είπεν ο Μωϋσής προς την συναγωγήν, Ούτος είναι ο λόγος τον οποίον προσέταξεν ο Κύριος να γείνη.
\par 6 Και έφερεν ο Μωϋσής τον Ααρών και τους υιούς αυτού και έλουσεν αυτούς με ύδωρ.
\par 7 Και έβαλε τον χιτώνα επ' αυτόν, και έζωσεν αυτόν την ζώνην, και ενέδυσεν αυτόν τον ποδήρη, και έβαλεν επ' αυτού το εφόδ, και έζωσεν αυτόν την κεντητήν ζώνην του εφόδ, και περιέζωσεν αυτόν με αυτήν.
\par 8 Και έβαλεν επ' αυτού το περιστήθιον· εις δε το περιστήθιον έβαλε το Ουρίμ και το Θουμμίμ.
\par 9 Και έβαλε την μίτραν επί της κεφαλής αυτού· επί δε της μίτρας, κατά το έμπροσθεν αυτής, έβαλε το πέταλον το χρυσούν, το διάδημα το άγιον, καθώς προσέταξεν ο Κύριος εις τον Μωϋσήν.
\par 10 Και έλαβεν ο Μωϋσής το έλαιον του χρίσματος και έχρισε την σκηνήν και πάντα τα εν αυτή και ηγίασεν αυτά.
\par 11 Και ερράντισεν απ' αυτού επί το θυσιαστήριον επτάκις και έχρισε το θυσιαστήριον· και πάντα τα σκεύη αυτού και τον νιπτήρα και την βάσιν αυτού, διά να αγιάση αυτά.
\par 12 Και έχυσεν από του ελαίου του χρίσματος επί την κεφαλήν του Ααρών και έχρισεν αυτόν, διά να αγιάση αυτόν.
\par 13 Και έφερεν ο Μωϋσής τους υιούς του Ααρών και ενέδυσεν αυτούς χιτώνας και έζωσεν αυτούς ζώνας και έβαλε μιτρίδια επ' αυτών, καθώς προσέταξεν ο Κύριος εις τον Μωϋσήν.
\par 14 Και έφερε τον μόσχον της περί αμαρτίας προσφοράς· ο δε Ααρών και οι υιοί αυτού επέθεσαν τας χείρας αυτών επί την κεφαλήν του μόσχου της περί αμαρτίας προσφοράς.
\par 15 Και έσφαξεν αυτόν και έλαβεν ο Μωϋσής από του αίματος και έβαλεν επί τα κέρατα του θυσιαστηρίου κύκλω διά του δακτύλου αυτού, και εκαθάρισε το θυσιαστήριον· και το αίμα έχυσεν εις την βάσιν του θυσιαστηρίου και ηγίασεν αυτό, διά να κάμη εξιλέωσιν επ' αυτού.
\par 16 Και έλαβε παν το στέαρ το επί των εντοσθίων και τον λοβόν του ήπατος και τους δύο νεφρούς και το στέαρ αυτών, και έκαυσεν αυτά ο Μωϋσής επί του θυσιαστηρίου.
\par 17 Τον μόσχον όμως και το δέρμα αυτού και το κρέας αυτού και την κόπρον αυτού, έκαυσεν εν πυρί έξω του στρατοπέδου, καθώς προσέταξεν ο Κύριος εις τον Μωϋσήν.
\par 18 Και έφερε τον κριόν του ολοκαυτώματος· και ο Ααρών και οι υιοί αυτού επέθεσαν τας χείρας αυτών επί την κεφαλήν του κριού.
\par 19 Και έσφαξεν αυτόν και ερράντισεν ο Μωϋσής το αίμα επί το θυσιαστήριον κύκλω.
\par 20 Και διεμέλισε τον κριόν κατά τα μέλη αυτού· και έκαυσεν ο Μωϋσής την κεφαλήν και τα μέλη και το στέαρ.
\par 21 Τα δε εντόσθια και τους πόδας έπλυνε με ύδωρ· και έκαυσεν ο Μωϋσής όλον τον κριόν επί του θυσιαστηρίου· ήτο ολοκαύτωμα εις οσμήν ευωδίας, προσφορά γινομένη διά πυρός εις τον Κύριον· καθώς προσέταξεν ο Κύριος εις τον Μωϋσήν.
\par 22 Και έφερε τον κριόν τον δεύτερον, τον κριόν της καθιερώσεως· ο δε Ααρών και οι υιοί αυτού επέθεσαν τας χείρας αυτών επί την κεφαλήν του κριού.
\par 23 Και έσφαξεν αυτόν, και έλαβεν ο Μωϋσής από του αίματος αυτού και έβαλεν επί τον λοβόν του δεξιού ωτίου του Ααρών, και επί τον αντίχειρα της δεξιάς αυτού χειρός, και επί τον μεγάλον δάκτυλον του δεξιού αυτού ποδός.
\par 24 Και έφερε τους υιούς του Ααρών και έβαλεν ο Μωϋσής από του αίματος επί τον λοβόν του δεξιού ωτίου αυτών και επί τους αντίχειρας των δεξιών χειρών αυτών και επί τους μεγάλους δακτύλους των δεξιών ποδών αυτών· και ερράντισεν ο Μωϋσής το αίμα επί του θυσιαστηρίου κύκλω.
\par 25 Και έλαβε το στέαρ και την ουράν και παν το στέαρ το επί των εντοσθίων και τον λοβόν του ήπατος και τους δύο νεφρούς και το στέαρ αυτών και τον δεξιόν ώμον·
\par 26 και από του κανίστρου των αζύμων του έμπροσθεν του Κυρίου έλαβε μίαν πήτταν άζυμον, και ένα άρτον ελαιωμένον και εν λάγανον και έβαλεν αυτά επί το στέαρ και επί τον δεξιόν ώμον·
\par 27 και έβαλε τα πάντα εις τας χείρας του Ααρών και εις τας χείρας των υιών αυτού, και εκίνησεν αυτά εις προσφοράν κινητήν έμπροσθεν του Κυρίου.
\par 28 Και έλαβεν αυτά ο Μωϋσής εκ των χειρών αυτών και έκαυσεν επί του θυσιαστηρίου επί το ολοκαύτωμα· αύται ήσαν καθιερώσεις εις οσμήν ευωδίας· ήτο θυσία γινομένη διά πυρός εις τον Κύριον.
\par 29 Και λαβών ο Μωϋσής το στήθος, εκίνησεν αυτό εις προσφοράν κινητήν έμπροσθεν του Κυρίου· εκ του κριού της καθιερώσεως τούτο ήτο το μερίδιον του Μωϋσέως, καθώς προσέταξεν ο Κύριος εις τον Μωϋσήν.
\par 30 Και έλαβεν ο Μωϋσής από του ελαίου του χρίσματος και από του αίματος του επί του θυσιαστηρίου και ερράντισεν επί τον Ααρών, επί τας στολάς αυτού και επί τους υιούς αυτού και επί τας στολάς των υιών αυτού μετ' αυτού· και ηγίασε τον Ααρών, τας στολάς αυτού και τους υιούς αυτού και τας στολάς των υιών αυτού μετ' αυτού·
\par 31 Και είπεν ο Μωϋσής προς τον Ααρών και προς τους υιούς αυτού, Βράσατε το κρέας εις την θύραν της σκηνής του μαρτυρίου· και εκεί φάγετε αυτό και τον άρτον τον εν τω κανίστρω των καθιερώσεων, καθώς με προσέταξεν ο Κύριος, λέγων, Ο Ααρών και οι υιοί αυτού θέλουσι τρώγει αυτά.
\par 32 Το δε υπόλοιπον του κρέατος και του άρτου εν πυρί θέλετε κατακαύσει.
\par 33 Και από της θύρας της σκηνής του μαρτυρίου δεν θέλετε εξέλθει επτά ημέρας, εωσού πληρωθώσιν αι ημέραι της καθιερώσεώς σας· διότι εν επτά ημέραις θέλει τελειωθή η καθιέρωσίς σας.
\par 34 Καθώς έκαμεν εις την ημέραν ταύτην, ούτω προσέταξε Κύριος να εκτελήται, διά να γίνηται εξιλέωσις διά σας.
\par 35 Θέλετε λοιπόν καθίσει επτά ημέρας εις την θύραν της σκηνής του μαρτυρίου ημέραν και νύκτα· και θέλετε φυλάττει τας παραγγελίας του Κυρίου, διά να μη αποθάνητε· διότι ούτω προσετάχθην.
\par 36 Και έκαμεν ο Ααρών και οι υιοί αυτού πάντας τους λόγους, τους οποίους προσέταξεν ο Κύριος διά χειρός του Μωϋσέως.

\chapter{9}

\par Και την ογδόην ημέραν ο Μωϋσής εκάλεσε τον Ααρών και τους υιούς αυτού και τους πρεσβυτέρους του Ισραήλ·
\par 2 και είπε προς τον Ααρών, Λάβε εις σεαυτόν μόσχον εκ βοών διά προσφοράν περί αμαρτίας και κριόν διά ολοκαύτωμα άμωμα και πρόσφερε αυτά έμπροσθεν του Κυρίου.
\par 3 Και εις τους υιούς του Ισραήλ θέλεις λαλήσει, λέγων, Λάβετε τράγον εξ αιγών διά προσφοράν περί αμαρτίας και μόσχον και αρνίον ενιαύσια, άμωμα, διά ολοκαύτωμα,
\par 4 και βουν και κριόν διά ειρηνικήν προσφοράν, εις θυσίαν έμπροσθεν του Κυρίου, και προσφοράν εξ αλφίτων εζυμωμένην μετά ελαίου· διότι σήμερον θέλει εμφανισθή ο Κύριος εις εσάς.
\par 5 Και έφεραν ό,τι προσέταξεν ο Μωϋσής έμπροσθεν της σκηνής του μαρτυρίου· και επλησίασε πάσα η συναγωγή και εστάθη έμπροσθεν του Κυρίου.
\par 6 Και είπεν ο Μωϋσής, Ούτος είναι ο λόγος τον οποίον προσέταξε Κύριος να κάμνητε· και θέλει εμφανισθή εις εσάς η δόξα του Κυρίου.
\par 7 Και είπεν ο Μωϋσής προς τον Ααρών, Πρόσελθε εις το θυσιαστήριον και κάμε την περί αμαρτίας προσφοράν σου, και το ολοκαύτωμά σου και κάμε εξιλέωσιν υπέρ σεαυτού και υπέρ του λαού· και πρόσφερε το δώρον του λαού και κάμε εξιλέωσιν υπέρ αυτών, καθώς προσέταξεν ο Κύριος.
\par 8 Και προσήλθεν ο Ααρών εις το θυσιαστήριον και έσφαξε τον μόσχον της περί αμαρτίας προσφοράς, όστις ήτο δι' αυτόν.
\par 9 Και οι υιοί του Ααρών έφεραν το αίμα προς αυτόν· και ενέβαψε τον δάκτυλον αυτού εις το αίμα και έβαλεν επί τα κέρατα του θυσιαστηρίου και έχυσε το αίμα εις την βάσιν του θυσιαστηρίου.
\par 10 Το στέαρ όμως και τους νεφρούς και τον επάνω λοβόν του ήπατος της περί αμαρτίας προσφοράς έκαυσεν επί του θυσιαστηρίου, καθώς προσέταξεν ο Κύριος εις τον Μωϋσήν.
\par 11 Το δε κρέας και το δέρμα έκαυσεν εν πυρί έξω του στρατοπέδου.
\par 12 Και έσφαξε το ολοκαύτωμα· και οι υιοί του Ααρών παρέστησαν εις αυτόν το αίμα, και ερράντισεν αυτό επί του θυσιαστηρίου κύκλω.
\par 13 Και έφεραν προς αυτόν το ολοκαύτωμα διαμεμελισμένον και την κεφαλήν· και έκαυσεν αυτά επί του θυσιαστηρίου.
\par 14 Και έπλυνε τα εντόσθια και τους πόδας· και έκαυσεν αυτά επί το ολοκαύτωμα επί του θυσιαστηρίου.
\par 15 Και προσέφερε το δώρον του λαού· και έλαβε τον τράγον της περί αμαρτίας προσφοράς του λαού και έσφαξεν αυτόν, και προσέφερεν αυτόν περί αμαρτίας, καθώς και το πρώτον.
\par 16 Και προσέφερε το ολοκαύτωμα και έκαμεν αυτό κατά το διατεταγμένον.
\par 17 Και προσέφερε την εξ αλφίτων προσφοράν· και ενέπλησε την χείρα αυτού απ' αυτής και έκαυσεν αυτήν επί του θυσιαστηρίου, εκτός του πρωϊνού ολοκαυτώματος.
\par 18 Έσφαξεν έτι τον βουν και τον κριόν της ειρηνικής θυσίας της υπέρ του λαού· και οι υιοί του Ααρών παρέστησαν το αίμα προς αυτόν, και ερράντισεν αυτό επί του θυσιαστηρίου κύκλω,
\par 19 και το στέαρ του βοός και του κριού, την ουράν και το στέαρ το καλύπτον τα εντόσθια και τους νεφρούς και τον λοβόν του ήπατος·
\par 20 και έθεσαν τα στέατα επί τα στήθη, και έκαυσε τα στέατα επί του θυσιαστηρίου·
\par 21 τα δε στήθη και τον ώμον τον δεξιόν εκίνησεν ο Ααρών εις προσφοράν κινητήν ενώπιον του Κυρίου, καθώς προσέταξεν ο Μωϋσής.
\par 22 Και υψώσας ο Ααρών τας χείρας αυτού προς τον λαόν, ευλόγησεν αυτούς· και κατέβη, αφού προσέφερε την περί αμαρτίας προσφοράν και το ολοκαύτωμα και τας ειρηνικάς προσφοράς.
\par 23 Και εισήλθεν ο Μωϋσής και ο Ααρών εις την σκηνήν του μαρτυρίου· και εξελθόντες ευλόγησαν τον λαόν· και εφάνη η δόξα του Κυρίου εις πάντα τον λαόν.
\par 24 Και εξήλθε πυρ απ' έμπροσθεν του Κυρίου και κατέφαγεν επί του θυσιαστηρίου το ολοκαύτωμα, και τα στέατα· ιδών δε πας ο λαός ηλάλαξαν και έπεσον κατά πρόσωπον αυτών.

\chapter{10}

\par Και λαβόντες οι υιοί του Ααρών, Ναδάβ και Αβιούδ, έκαστος το θυμιατήριον αυτού, έβαλον πυρ εις αυτό, και επ' αυτό έβαλον θυμίαμα και προσέφεραν ενώπιον του Κυρίου πυρ ξένον, το οποίον δεν προσέταξεν εις αυτούς.
\par 2 Και εξήλθε πυρ παρά του Κυρίου και κατέφαγεν αυτούς· και απέθανον έμπροσθεν του Κυρίου.
\par 3 Τότε είπεν ο Μωϋσής προς τον Ααρών, Τούτο είναι το οποίον είπεν ο Κύριος, λέγων, Εγώ θέλω αγιασθή εις τους πλησιάζοντας εις εμέ, και έμπροσθεν παντός του λαού θέλω δοξασθή. Και ο Ααρών εσιώπησε.
\par 4 Και εκάλεσεν ο Μωϋσής τον Μισαήλ και τον Ελισαφάν, υιούς του Οζιήλ, θείου του Ααρών, και είπε προς αυτούς, Πλησιάσατε, σηκώσατε τους αδελφούς σας απ' έμπροσθεν του αγιαστηρίου έξω του στρατοπέδου.
\par 5 Και επλησίασαν και εσήκωσαν αυτούς με τους χιτώνας αυτών έξω του στρατοπέδου, καθώς είπεν ο Μωϋσής.
\par 6 Και είπεν ο Μωϋσής προς τον Ααρών και προς τον Ελεάζαρ και προς τον Ιθάμαρ, τους υιούς αυτού, Τας κεφαλάς σας μη αποκαλύψητε, και τα ιμάτιά σας μη διασχίσητε, διά να μη αποθάνητε και έλθη οργή εφ' όλην την συναγωγήν· αλλ' οι αδελφοί σας, πας ο οίκος του Ισραήλ, ας κλαύσωσι το καύσιμον το οποίον έκαμεν ο Κύριος·
\par 7 και δεν θέλετε εξέλθει εκ της θύρας της σκηνής του μαρτυρίου, διά να μη αποθάνητε· διότι το έλαιον του χρίσματος του Κυρίου είναι εφ' υμάς. Και έκαμον κατά τον λόγον του Μωϋσέως.
\par 8 Και ελάλησε Κύριος προς τον Ααρών, λέγων,
\par 9 Οίνον και σίκερα δεν θέλετε πίει, συ, και οι υιοί σου μετά σου, όταν εισέρχησθε εις την σκηνήν του μαρτυρίου, διά να μη αποθάνητε· τούτο θέλει είσθαι νόμιμον αιώνιον εις τας γενεάς σας·
\par 10 και διά να διακρίνητε μεταξύ αγίου και βεβήλου και μεταξύ ακαθάρτου και καθαρού·
\par 11 και διά να διδάσκητε τους υιούς Ισραήλ πάντα τα διατάγματα, όσα ελάλησε Κύριος προς αυτούς διά χειρός του Μωϋσέως.
\par 12 Και είπεν ο Μωϋσής προς τον Ααρών και προς τον Ελεάζαρ και προς τον Ιθάμαρ, τους υιούς αυτού τους εναπολειφθέντας, Λάβετε την εξ αλφίτων προσφοράν την εναπολειφθείσαν από των διά πυρός γινομένων θυσιών του Κυρίου και φάγετε αυτήν άζυμον πλησίον του θυσιαστηρίου· διότι είναι αγιώτατον·
\par 13 και θέλετε φάγει αυτήν εν τόπω αγίω· επειδή είναι το δίκαιόν σου και το δίκαιον των υιών σου εκ των διά πυρός γινομένων θυσιών του Κυρίου· διότι ούτω προσετάχθην·
\par 14 και το κινητόν στήθος και τον υψούμενον ώμον θέλετε φάγει εν καθαρώ τόπω, συ και οι υιοί σου και αι θυγατέρες σου μετά σού· διότι είναι το δίκαιόν σου και το δίκαιον των υιών σου, δοθέντα εκ των θυσιών της ειρηνικής προσφοράς των υιών του Ισραήλ·
\par 15 τον υψούμενον ώμον και το κινητόν στήθος θέλουσι φέρει μετά των διά πυρός γινομένων προσφορών του στέατος, διά να κινήσωσιν αυτά εις κινητήν προσφοράν ενώπιον του Κυρίου· και θέλει είσθαι εις σε και εις τους υιούς σου μετά σου εις νόμιμον αιώνιον, καθώς προσέταξεν ο Κύριος.
\par 16 Και εζήτησεν επιμελώς ο Μωϋσής τον τράγον της περί αμαρτίας προσφοράς· και ιδού, ήτο κατακεκαυμένος· και εθυμώθη κατά του Ελεάζαρ και κατά του Ιθάμαρ, των υιών του Ααρών των εναπολειφθέντων, λέγων,
\par 17 Διά τι δεν εφάγετε την περί αμαρτίας προσφοράν εν τόπω αγίω; διότι είναι αγιώτατον· και έδωκεν αυτό εις εσάς Κύριος διά να σηκόνητε την ανομίαν της συναγωγής, ώστε να κάμνητε εξιλέωσιν υπέρ αυτών ενώπιον του Κυρίου·
\par 18 ιδού, το αίμα αυτού δεν εφέρθη εις το αγιαστήριον· πρέπει εξάπαντος να φάγητε αυτό εν τω αγιαστηρίω, καθώς προσέταξα.
\par 19 Και είπεν ο Ααρών προς τον Μωϋσήν, Ιδού, αυτοί προσέφεραν σήμερον την περί αμαρτίας προσφοράν αυτών και το ολοκαύτωμα αυτών ενώπιον του Κυρίου και συνέβησαν εις εμέ τοιαύτα· εάν λοιπόν ήθελον φάγει την περί αμαρτίας προσφοράν σήμερον, τούτο ήθελεν είσθαι αρεστόν εις τους οφθαλμούς του Κυρίου;
\par 20 Και ήκουσεν ο Μωϋσής και ήρεσεν εις αυτόν.

\chapter{11}

\par Και ελάλησε Κύριος προς τον Μωϋσήν και προς τον Ααρών λέγων προς αυτούς,
\par 2 Λαλήσατε προς τους υιούς Ισραήλ, λέγοντες, ταύτα είναι τα ζώα τα οποία θέλετε τρώγει εκ πάντων των κτηνών των επί της γης.
\par 3 Παν δίχηλον μεταξύ των κτηνών έχον τον πόδα εσχισμένον και αναμασσών, τούτο θέλετε τρώγει.
\par 4 Ταύτα όμως δεν θέλετε τρώγει εκ των όσα αναμασσώσιν ή εκ των όσα είναι δίχηλα· την κάμηλον, διότι αναμασσά μεν, πλην δεν είναι δίχηλος· είναι ακάθαρτος εις εσάς·
\par 5 και τον δασύποδα, διότι αναμασσά μεν, πλην δεν είναι δίχηλος· είναι ακάθαρτος εις εσάς·
\par 6 και τον λαγωόν, διότι αναμασσά μεν, πλην δεν είναι δίχηλος· είναι ακάθαρτος εις εσάς·
\par 7 και τον χοίρον, διότι είναι μεν δίχηλος και έχει τον πόδα εσχισμένον, πλην δεν αναμασσά· είναι ακάθαρτος εις εσάς·
\par 8 από του κρέατος αυτών δεν θέλετε τρώγει και το θνησιμαίον αυτών δεν θέλετε εγγίζει είναι ακάθαρτα εις εσάς.
\par 9 Ταύτα θέλετε τρώγει εκ πάντων των εν τοις ύδασι πάντα όσα έχουσι πτερά και λέπη, εν τοις ύδασι, εν ταις θαλάσσαις και εν τοις ποταμοίς, ταύτα θέλετε τρώγει.
\par 10 Και πάντα όσα δεν έχουσι πτερά και λέπη, εν ταις θαλάσσαις και εν τοις ποταμοίς, από πάντων όσα κινούνται εν τοις ύδασι και από παντός εμψύχου ζώου το οποίον είναι εν τοις ύδασι, θέλουσιν είσθαι βδελυκτά εις εσάς·
\par 11 ταύτα εξάπαντος θέλουσιν είσθαι βδελυκτά εις εσάς· από του κρέατος αυτών δεν θέλετε τρώγει και το θνησιμαίον αυτών θέλετε βδελύττεσθαι.
\par 12 Πάντα όσα εν τοις ύδασι δεν έχουσι πτερά ούτε λέπη, θέλουσιν είσθαι βδελυκτά εις εσάς.
\par 13 Ταύτα δε θέλετε βδελύττεσθαι μεταξύ των πτηνών· δεν θέλουσι τρώγεσθαι είναι βδελυκτά· ο αετός, και ο γρυπαετός, και ο μελαναετός,
\par 14 και ο γυψ, και ο ίκτινος κατά το είδος αυτού·
\par 15 πας κόραξ κατά το είδος αυτού·
\par 16 και η στρουθοκάμηλος, και η γλαύξ, και ο ίβις, και ο ιέραξ κατά το είδος αυτού,
\par 17 και ο νυκτικόραξ, και η αίθυια, και η μεγάλη γλαύξ,
\par 18 και ο κύκνος, και ο πελεκάν, και η κίσσα,
\par 19 και ο πελαργός, και ο ερωδιός κατά το είδος αυτού, και ο έποψ, και η νυκτερίς.
\par 20 Πάντα τα πετώμενα ερπετά, τα οποία περιπατούσιν επί τέσσαρας πόδας, θέλουσιν είσθαι βδελυκτά εις εσάς.
\par 21 Ταύτα όμως δύνασθε να τρώγητε από παντός πετωμένου ερπετού, περιπατούντος επί τέσσαρας πόδας, τα οποία έχουσι σκέλη όπισθεν των ποδών αυτών, διά να πηδώσι δι' αυτών επί της γής·
\par 22 ταύτα θέλετε τρώγει εξ αυτών· τον βρούχον κατά το είδος αυτού και τον αττάκην κατά το είδος αυτού και τον οφιομάχον κατά το είδος αυτού και την ακρίδα κατά το είδος αυτής.
\par 23 Πάντα δε τα πετώμενα ερπετά, έχοντα τέσσαρας πόδας, θέλουσιν είσθαι βδελυκτά εις εσάς.
\par 24 Και εις ταύτα θέλετε είσθαι ακάθαρτοι· πας ο εγγίζων το θνησιμαίον αυτών θέλει είσθαι ακάθαρτος έως εσπέρας.
\par 25 Και πας όστις βαστάση από του θνησιμαίον αυτών, θέλει πλύνει τα ιμάτια αυτού και θέλει είσθαι ακάθαρτος έως εσπέρας.
\par 26 Εκ πάντων των κτηνών, όσα είναι δίχηλα, πλην δεν είναι ο πους αυτών εσχισμένος ουδέ αναμασσώσι, θέλουσιν είσθαι ακάθαρτα εις εσάς· πας ο εγγίζων αυτά θέλει είσθαι ακάθαρτος.
\par 27 Και όσα περιπατούσιν επί τας παλάμας αυτών, μεταξύ πάντων των ζώων των περιπατούντων επί τέσσαρας πόδας, θέλουσιν είσθαι ακάθαρτα εις εσάς· πας ο εγγίζων το θνησιμαίον αυτών θέλει είσθαι ακάθαρτος έως εσπέρας.
\par 28 Και όστις βαστάση το θνησιμαίον αυτών, θέλει πλύνει τα ιμάτια αυτού και θέλει είσθαι ακάθαρτος έως εσπέρας· ταύτα θέλουσιν είσθαι ακάθαρτα εις εσάς.
\par 29 Και ταύτα θέλουσιν είσθαι ακάθαρτα εις εσάς μεταξύ των ερπετών των ερπόντων επί της γής· η γαλή και ο ποντικός, και η χελώνη κατά το είδος αυτής·
\par 30 και ο ακανθόχοιρος, και ο χαμαιλέων, και η σαύρα, και ο σαμιάμιθος, και ο ασπάλαξ.
\par 31 Ταύτα είναι ακάθαρτα εις εσάς μεταξύ πάντων των ερπετών· πας ο εγγίζων αυτά τεθνεώτα, θέλει είσθαι ακάθαρτος έως εσπέρας.
\par 32 Και παν πράγμα επί του οποίου ήθελε πέσει τι εκ τούτων τεθνεώτων θέλει είσθαι ακάθαρτον· παν αγγείον ξύλινον ή ιμάτιον ή δέρμα ή σάκκος ή οποιονδήποτε αγγείον, εις το οποίον γίνεται εργασία, θέλει εμβληθή εις ύδωρ και θέλει είσθαι ακάθαρτον έως εσπέρας· τότε θέλει είσθαι καθαρόν·
\par 33 και παν αγγείον πήλινον, εις το οποίον εμπέση τι εκ τούτων, παν ό,τι είναι εντός αυτού θέλει είσθαι ακάθαρτον· αυτό δε θέλετε συντρίψει·
\par 34 από παντός φαγητού εσθιωμένου, εις το οποίον εμβαίνει ύδωρ, θέλει είσθαι ακάθαρτον· και παν ποτόν πινόμενον εν οποιωδήποτε αγγείω, θέλει είσθαι ακάθαρτον.
\par 35 Και παν πράγμα επί το οποίον πέση από του θνησιμαίου αυτών, θέλει είσθαι ακάθαρτον· κλίβανος είτε εστία θέλουσι κρημνισθή· είναι ακάθαρτα και ακάθαρτα θέλουσιν είσθαι εις εσάς.
\par 36 Πηγή όμως ή λάκκος, σύναξις υδάτων, θέλει είσθαι καθαρόν· πλην ό,τι εγγίση το θνησιμαίον αυτών, θέλει είσθαι ακάθαρτον.
\par 37 Και εάν πέση από του θνησιμαίου αυτών επί τι σπέρμα σπόριμον, το οποίον μέλλει να σπαρθή, καθαρόν θέλει είσθαι.
\par 38 Εάν δε επιχυθή ύδωρ επί του σπέρματος και πέση από του θνησιμαίου αυτών επ' αυτό, ακάθαρτον θέλει είσθαι εις εσάς.
\par 39 Και εάν αποθάνη τι εκ των κτηνών τα οποία δύνασθε να τρώγητε, όστις εγγίση το θνησιμαίον αυτού, θέλει είσθαι ακάθαρτος έως εσπέρας.
\par 40 Και όστις φάγη από του θνησιμαίου αυτού, θέλει πλύνει τα ιμάτια αυτού και θέλει είσθαι ακάθαρτος έως εσπέρας· και όστις βαστάση το θνησιμαίον αυτού, θέλει πλύνει τα ιμάτια αυτού και θέλει είσθαι ακάθαρτος έως εσπέρας.
\par 41 Και παν ερπετόν, έρπον επί της γης, θέλει είσθαι βδέλυγμα· δεν θέλει τρώγεσθαι.
\par 42 Παν ό,τι περιπατεί επί της κοιλίας και παν ό,τι περιπατεί επί τέσσαρας πόδας ή παν το έχον πολλούς πόδας, μεταξύ πάντων των ερπετών των ερπόντων επί της γης, ταύτα δεν θέλετε τρώγει, διότι είναι βδέλυγμα.
\par 43 Δεν θέλετε κάμει βδελυκτάς τας ψυχάς σας δι' ουδενός ερπετού έρποντος ουδέ θέλετε μιανθή δι' αυτών, ώστε να γείνητε ακάθαρτοι δι' αυτών.
\par 44 Διότι εγώ είμαι Κύριος ο Θεός σας· θέλετε λοιπόν αγιασθή και θέλετε είσθαι άγιοι, διότι άγιος είμαι εγώ· και δεν θέλετε μιάνει τας ψυχάς σας δι' ουδενός ερπετού έρποντος επί της γης.
\par 45 Διότι εγώ είμαι ο Κύριος, όστις σας ανεβίβασα εκ γης Αιγύπτου, διά να ήμαι Θεός σας· θέλετε λοιπόν είσθαι άγιοι, διότι άγιος είμαι εγώ.
\par 46 Ούτος είναι ο νόμος περί των κτηνών και περί των πτηνών και περί παντός εμψύχου όντος κινουμένου εν τοις ύδασι και περί παντός όντος έρποντος επί της γής·
\par 47 διά να διακρίνητε μεταξύ του ακαθάρτου και του καθαρού και μεταξύ των ζώων τα οποία τρώγονται των ζώων και οποία δεν τρώγονται.

\chapter{12}

\par Και ελάλησε Κύριος προς τον Μωϋσήν, λέγων,
\par 2 Λάλησον προς τους υιούς Ισραήλ, λέγων, Εάν γυνή τις συλλάβη και γεννήση αρσενικόν, τότε θέλει είσθαι ακάθαρτος επτά ημέρας· κατά τας ημέρας του χωρισμού διά τα γυναικεία αυτής θέλει είσθαι ακάθαρτος.
\par 3 Και την ογδόην ημέραν θέλει περιτέμνεσθαι η σαρξ της ακροβυστίας αυτού.
\par 4 Και έτι τριάκοντα τρεις ημέρας θέλει μείνει εις το αίμα του καθαρισμού αυτής· ουδέν πράγμα άγιον θέλει εγγίσει και εις το αγιαστήριον δεν θέλει εισέλθει, εωσού πληρωθώσιν αι ημέραι του καθαρισμού αυτής.
\par 5 Αλλ' εάν γεννήση θηλυκόν, τότε θέλει είσθαι ακάθαρτος δύο εβδομάδας, καθώς εν τω χωρισμώ αυτής· και θέλει μείνει έτι εις το αίμα του καθαρισμού αυτής εξήκοντα εξ ημέρας.
\par 6 Και αφού πληρωθώσιν αι ημέραι του καθαρισμού αυτής, διά υιόν, ή διά θυγατέρα, θέλει φέρει αρνίον ενιαύσιον εις ολοκαύτωμα, και νεοσσόν περιστεράς, ή τρυγόνα, διά προσφοράν περί αμαρτίας, εις την θύραν της σκηνής του μαρτυρίου προς τον ιερέα·
\par 7 ούτος δε θέλει προσφέρει αυτό ενώπιον του Κυρίου και θέλει κάμει εξιλέωσιν υπέρ αυτής και θέλει καθαρισθή από της ροής του αίματος αυτής. Ούτος είναι ο νόμος της γεννώσης αρσενικόν ή θηλυκόν.
\par 8 Εάν όμως δεν ευπορή να φέρη αρνίον, τότε θέλει φέρει δύο τρυγόνας ή δύο νεοσσούς περιστερών, μίαν διά ολοκαύτωμα και μίαν διά προσφοράν περί αμαρτίας· και θέλει κάμει εξιλέωσιν υπέρ αυτής ο ιερεύς, και θέλει καθαρισθή.

\chapter{13}

\par Και ελάλησε Κύριος προς τον Μωϋσήν και προς τον Ααρών, λέγων,
\par 2 Όταν άνθρωπός τις έχη επί του δέρματος της σαρκός αυτού πρήσμα ή ψώραν, ή εξάνθημα, και γείνη εις το δέρμα της σαρκός αυτού πληγή λέπρας, τότε θέλει φερθή προς τον Ααρών τον ιερέα ή προς ένα των υιών αυτού των ιερέων·
\par 3 και ο ιερεύς θέλει θεωρήσει την πληγήν εις το δέρμα της σαρκός. Και εάν η θριξ εις την πληγήν μετεβλήθη εις λευκήν, και η πληγή εις την όψιν ήναι βαθυτέρα του δέρματος της σαρκός αυτού, είναι πληγή λέπρας· ο δε ιερεύς θέλει θεωρήσει αυτόν και θέλει κρίνει αυτόν ακάθαρτον.
\par 4 Αλλ' εάν το εξάνθημα ήναι λευκόν εις το δέρμα της σαρκός αυτού και εις την όψιν δεν είναι βαθύτερον του δέρματος και η θριξ αυτού δεν μετεβλήθη εις λευκήν, τότε θέλει κλείσει ο ιερεύς τον έχοντα την πληγήν επτά ημέρας·
\par 5 και θέλει θεωρήσει αυτόν ο ιερεύς την εβδόμην ημέραν· και ιδού, εάν ίδη ότι η πληγή είναι εις στάσιν και η πληγή δεν εξηπλώθη εις το δέρμα, τότε θέλει κλείσει αυτόν ο ιερεύς άλλας επτά ημέρας·
\par 6 και θέλει θεωρήσει αυτόν ο ιερεύς εκ δευτέρου την εβδόμην ημέραν· και ιδού, εάν η πληγή ημαυρώθη και δεν εξηπλώθη η πληγή εις το δέρμα, θέλει κρίνει αυτόν ο ιερεύς καθαρόν· αύτη είναι ψώρα· και θέλει πλύνει τα ιμάτια αυτού και θέλει είσθαι καθαρός.
\par 7 Εάν όμως εξηπλώθη περισσότερον η ψώρα επί του δέρματος, αφού εθεωρήθη υπό του ιερέως διά τον καθαρισμόν αυτού, θέλει δειχθή πάλιν εις τον ιερέα.
\par 8 Και εάν ίδη ο ιερεύς, ότι εξηπλώθη η ψώρα επί του δέρματος, τότε θέλει κρίνει αυτόν ο ιερεύς ακάθαρτον· είναι λέπρα.
\par 9 Όταν η πληγή της λέπρας ήναι εις άνθρωπον, τότε θέλει φερθή προς τον ιερέα.
\par 10 και θέλει θεωρήσει ο ιερεύς· και ιδού, εάν το πρήσμα ήναι λευκόν εις το δέρμα και μετέβαλε την τρίχα εις λευκήν, και ευρίσκεται κρέας ζων εις το πρήσμα,
\par 11 είναι λέπρα παλαιά εις το δέρμα της σαρκός αυτού και θέλει κρίνει αυτόν ο ιερεύς ακάθαρτον· δεν θέλει κλείσει αυτόν, διότι είναι ακάθαρτος.
\par 12 Αλλ' εάν εξηπλώθη πολύ η λέπρα επί του δέρματος και η λέπρα εσκέπασεν όλον το δέρμα του έχοντος την πληγήν από κεφαλής αυτού και έως ποδών αυτού, όπου και αν θεωρήση ο ιερεύς,
\par 13 τότε θέλει θεωρήσει ο ιερεύς, και ιδού, εάν η λέπρα εσκέπασεν όλην την σάρκαν αυτού, θέλει κρίνει καθαρόν τον έχοντα την πληγήν· είναι καθαρός.
\par 14 Αλλ' εν οποία ημέρα φανή εις αυτόν κρέας ζων, θέλει είσθαι ακάθαρτος.
\par 15 Και θέλει θεωρήσει ο ιερεύς το κρέας το ζων και θέλει κρίνει αυτόν ακάθαρτον· το ζων κρέας είναι ακάθαρτον· είναι λέπρα.
\par 16 Η εάν το κρέας το ζων αλλάξη πάλιν και μεταβληθή εις λευκόν, θέλει ελθεί προς τον ιερέα.
\par 17 Και θέλει θεωρήσει αυτόν ο ιερεύς· και ιδού, εάν η πληγή μετεβλήθη εις λευκήν, τότε θέλει κρίνει ο ιερεύς καθαρόν τον έχοντα την πληγήν· είναι καθαρός.
\par 18 Η δε σαρξ επί του δέρματος της οποίας ήτο έλκος, και ιατρεύθη,
\par 19 και εν τω τόπω του έλκους έγεινε πρήσμα λευκόν, ή εξάνθημα λευκόν κοκκινωπόν, θέλει δειχθή εις τον ιερέα·
\par 20 και θέλει θεωρήσει ο ιερεύς, και ιδού, εάν φαίνηται βαθύτερον του δέρματος και η θριξ αυτού μετεβλήθη εις λευκήν, θέλει κρίνει αυτόν ο ιερεύς ακάθαρτον· είναι πληγή λέπρας, ήτις εξήνθησεν εις το έλκος.
\par 21 Εάν δε θεωρήση αυτό ο ιερεύς, και ιδού, δεν ήναι λευκαί τρίχες εις αυτό και δεν ήναι βαθύτερον του δέρματος και ήναι ημαυρωμένον, τότε ο ιερεύς θέλει κλείσει αυτόν επτά ημέρας·
\par 22 και εάν εξηπλώθη πολύ επί του δέρματος, τότε θέλει κρίνει αυτόν ο ιερεύς ακάθαρτον· είναι πληγή.
\par 23 Αλλ' εάν το εξάνθημα μένη εν τω τόπω αυτού και δεν εξηπλώθη, τούτο είναι ουλή του έλκους· και θέλει κρίνει αυτόν ο ιερεύς καθαρόν.
\par 24 Εάν δε ήναι κρέας, έχον επί του δέρματος αυτού καυστικήν φλόγωσιν, και το ζων κρέας του πεφλογισμένου μέρους έχη εξάνθημα λευκόν, κοκκινωπόν ή κατάλευκον,
\par 25 τότε θέλει θεωρήσει αυτό ο ιερεύς· και ιδού, εάν η θριξ εις το εξάνθημα μετεβλήθη εις λευκήν και εις την όψιν ήναι βαθύτερον του δέρματος, είναι λέπρα εξανθήσασα εις την φλόγωσιν· και θέλει κρίνει αυτόν ο ιερεύς ακάθαρτον· είναι πληγή λέπρας.
\par 26 Αλλ' εάν ο ιερεύς θεωρήση αυτό, και ιδού, δεν ήναι θριξ λευκή εις το εξάνθημα και δεν ήναι βαθύτερον του δέρματος και ήναι ημαυρωμένον, τότε θέλει κλείσει αυτόν ο ιερεύς επτά ημέρας·
\par 27 και θέλει θεωρήσει αυτόν ο ιερεύς την εβδόμην ημέραν· και εάν αυτό εξηπλώθη πολύ εις το δέρμα, τότε θέλει κρίνει αυτόν ο ιερεύς ακάθαρτον· είναι πληγή λέπρας.
\par 28 Εάν δε το εξάνθημα μένη εν τω τόπω αυτού και δεν εξηπλώθη επί του δέρματος και ήναι ημαυρωμένον, είναι πρήσμα φλογώσεως, και θέλει κρίνει αυτόν ο ιερεύς καθαρόν· επειδή είναι ουλή της φλογώσεως.
\par 29 Και εάν ανήρ, ή γυνή, έχη πληγήν εις την κεφαλήν, ή εις το πωγώνιον,
\par 30 τότε θέλει θεωρήσει ο ιερεύς την πληγήν· και ιδού, εάν εις την όψιν ήναι βαθυτέρα του δέρματος και εις αυτήν θριξ ξανθίζουσα, τότε θέλει κρίνει αυτόν ο ιερεύς ακάθαρτον· είναι κασίδα, λέπρα της κεφαλής ή του πωγωνίου.
\par 31 Και εάν θεωρήση ο ιερεύς την πληγήν της κασίδας και ιδού, εις την όψιν δεν ήναι βαθυτέρα του δέρματος και δεν ήναι θριξ μελανή εν αυτή, τότε θέλει κλείσει ο ιερεύς επτά ημέρας τον έχοντα την πληγήν της κασίδας·
\par 32 και θέλει θεωρήσει ο ιερεύς την πληγήν την εβδόμην ημέραν· και ιδού, εάν δεν εξηπλώθη η κασίδα και δεν ήναι εις αυτήν θριξ ξανθίζουσα και εις την όψιν η κασίδα δεν ήναι βαθυτέρα του δέρματος,
\par 33 αυτός θέλει ξυρισθή, αλλ' η κασίδα δεν θέλει ξυρισθή· ο δε ιερεύς θέλει κλείσει τον έχοντα την κασίδαν άλλας επτά ημέρας.
\par 34 Και την εβδόμην ημέραν θέλει θεωρήσει ο ιερεύς την κασίδαν· και ιδού, εάν η κασίδα δεν εξηπλώθη εις το δέρμα και εις την όψιν δεν ήναι βαθυτέρα του δέρματος, τότε θέλει κρίνει αυτόν ο ιερεύς καθαρόν· και αυτός θέλει πλύνει τα ιμάτια αυτού και θέλει είσθαι καθαρός.
\par 35 Αλλ' εάν η κασίδα εξηπλώθη πολύ επί του δέρματος μετά τον καθαρισμόν αυτού,
\par 36 τότε θέλει θεωρήσει αυτόν ο ιερεύς· και ιδού, εάν η κασίδα εξηπλώθη επί του δέρματος, δεν θέλει ερευνήσει ο ιερεύς περί της ξανθιζούσης τριχός· είναι ακάθαρτος.
\par 37 Αλλ' εάν θεωρήση ότι η κασίδα είναι εις στάσιν και εκφύεται θριξ μελανή εν αυτή, η κασίδα είναι τεθεραπευμένη· είναι καθαρός· και θέλει κρίνει αυτόν ο ιερεύς καθαρόν.
\par 38 Και εάν ανήρ, ή γυνή, έχωσιν επί του δέρματος της σαρκός αυτών εξανθήματα, εξανθήματα λευκωπά,
\par 39 τότε θέλει θεωρήσει ο ιερεύς· και ιδού, εάν τα εξανθήματα επί του δέρματος της σαρκός αυτών ήναι υπόλευκα, είναι αλφός εξανθών επί του δέρματος· είναι καθαρός.
\par 40 Εάν δε η κεφαλή τινός μαδήση, αυτός είναι φαλακρός· είναι καθαρός.
\par 41 Και εάν η κεφαλή αυτού μαδήση προς το πρόσωπον, είναι αναφάλαντος· είναι καθαρός.
\par 42 Αλλ' εάν ήναι εις το φαλάκρωμα, ή εις το αναφαλάντωμα, πληγή λευκή κοκκινωπή, είναι λέπρα εξανθήσασα εις το φαλάκρωμα αυτού ή εις το αναφαλάντωμα αυτού.
\par 43 Και θέλει θεωρήσει αυτόν ο ιερεύς· και ιδού, εάν το πρήσμα της πληγής ήναι λευκόν κοκκινωπόν εις το φαλάκρωμα αυτού ή εις το αναφαλάντωμα αυτού, ως το φαινόμενον της λέπρας επί του δέρματος της σαρκός,
\par 44 είναι άνθρωπος λεπρός, είναι ακάθαρτος· θέλει κρίνει αυτόν ο ιερεύς όλως ακάθαρτον· εις την κεφαλήν αυτού είναι η πληγή αυτού.
\par 45 Και του λεπρού, εις τον οποίον είναι η πληγή, τα ιμάτια αυτού θέλουσι σχισθή και η κεφαλή αυτού θέλει είσθαι ασκεπής, και το επάνω χείλος αυτού θέλει καλύψει και θέλει φωνάζει, Ακάθαρτος, ακάθαρτος.
\par 46 Πάσας τας ημέρας καθ' ας η πληγή θέλει είσθαι εν αυτώ, θέλει είσθαι ακάθαρτος· είναι ακάθαρτος· μόνος θέλει κατοικεί· έξω του στρατοπέδου θέλει είσθαι η κατοικία αυτού.
\par 47 Και εάν υπάρχη εις ιμάτιον πληγή λέπρας, εις ιμάτιον μάλλινον ή εις ιμάτιον λινούν,
\par 48 είτε εις στημόνιον, είτε εις υφάδιον, εκ λινού ή εκ μαλλίου, είτε εις δέρμα, είτε εις παν πράγμα κατεσκευασμένον εκ δέρματος,
\par 49 και η πληγή ήναι πρασινωπή, ή κοκκινωπή, εις το ιμάτιον, ή εις το δέρμα, ή εις το στημόνιον, εις το υφάδιον, ή εις παν σκεύος δερμάτινον, είναι πληγή λέπρας και θέλει δειχθή εις τον ιερέα·
\par 50 ο δε ιερεύς θέλει θεωρήσει την πληγήν και θέλει κλείσει το έχον την πληγήν επτά ημέρας.
\par 51 Και θέλει θεωρήσει την πληγήν την εβδόμην ημέραν· εάν η πληγή εξηπλώθη επί του ιματίου, ή επί του στημονίου, ή επί του υφαδίου, ή επί του δέρματος, εκ παντός πράγματος το οποίον είναι κατεσκευασμένον εκ δέρματος, η πληγή είναι λέπρα διαβρωτική· τούτο είναι ακάθαρτον.
\par 52 Και θέλει καύσει το ιμάτιον, ή το στημόνιον, ή το υφάδιον, μάλλινον, ή λινούν, ή παν σκεύος δερμάτινον επί του οποίου είναι η πληγή· διότι είναι λέπρα διαβρωτική· με πυρ θέλει καυθή.
\par 53 Και εάν ίδη ο ιερεύς, και ιδού, η πληγή δεν εξηπλώθη επί του ιματίου, είτε επί του στημονίου, είτε επί του υφαδίου, ή επί παντός σκεύους δερματίνου,
\par 54 τότε θέλει προστάξει ο ιερεύς να πλυθή το έχον την πληγήν και θέλει κλείσει αυτό άλλας επτά ημέρας·
\par 55 και θέλει θεωρήσει ο ιερεύς την πληγήν, αφού επλύθη· και ιδού, εάν η πληγή δεν ήλλαξε το χρώμα αυτής και δεν εξηπλώθη η πληγή, είναι ακάθαρτον· με πυρ θέλεις καύσει αυτό· είναι διαβρωτικόν, το οποίον προχωρεί υποκάτωθεν ή επάνωθεν.
\par 56 Και εάν ίδη ο ιερεύς, και ιδού, η πληγή, αφού επλύθη, είναι ημαυρωμένη, τότε θέλει εκκόψει αυτήν από του ιματίου, ή από του δέρματος, ή από του στημονίου, ή από του υφαδίου.
\par 57 Αλλ' εάν φανή έτι επί του ιματίου, επί του στημονίου, ή επί του υφαδίου, ή επί παντός σκεύους δερματίνου, είναι λέπρα εξανθίζουσα· με πυρ θέλεις καύσει το έχον την πληγήν.
\par 58 Και το ιμάτιον, ή το στημόνιον, ή το υφάδιον, παν σκεύος δερμάτινον, το οποίον ήθελες πλύνει, εάν η πληγή εξηλείφθη απ' αυτών, τότε θέλει πλυθή εκ δευτέρου και θέλει είσθαι καθαρόν.
\par 59 Ούτος είναι ο νόμος της πληγής της λέπρας επί ιματίου μαλλίνου, ή λινού, είτε στημονίου, είτε υφαδίου, ή παντός σκεύους δερματίνου, διά να κρίνηται καθαρόν, ή να κρίνηται ακάθαρτον.

\chapter{14}

\par Και ελάλησε Κύριος προς τον Μωϋσήν, λέγων,
\par 2 Ούτος είναι ο νόμος του λεπρού εν τη ημέρα του καθαρισμού αυτού· θέλει φερθή προς τον ιερέα·
\par 3 και θέλει εξέλθει ο ιερεύς έξω του στρατοπέδου και θέλει θεωρήσει ο ιερεύς, και ιδού, εάν ιατρεύθη η πληγή της λέπρας εις τον λεπρόν,
\par 4 τότε θέλει προστάξει ο ιερεύς να λάβωσι διά τον καθαριζόμενον δύο πτηνά ζώντα καθαρά και ξύλον κέδρινον και κόκκινον και ύσσωπον.
\par 5 Και θέλει προστάξει ο ιερεύς να σφάξωσι το εν πτηνόν εις αγγείον πήλινον επάνω ύδατος ζώντος·
\par 6 το δε πτηνόν το ζων, θέλει λάβει αυτό και το ξύλον το κέδρινον και το κόκκινον και τον ύσσωπον και θέλει εμβάψει αυτά και το πτηνόν το ζων εις το αίμα του πτηνού του εσφαγμένου επάνω του ύδατος του ζώντος·
\par 7 και θέλει ραντίσει επί τον καθαριζόμενον από της λέπρας επτάκις και θέλει κρίνει αυτόν καθαρόν· και θέλει απολύσει το πτηνόν το ζων επί πρόσωπον της πεδιάδος.
\par 8 Και θέλει πλύνει ο καθαριζόμενος τα ιμάτια αυτού και θέλει ξυρίσει πάσας τας τρίχας αυτού και θέλει λουσθή εν ύδατι και θέλει είσθαι καθαρός· και μετά ταύτα θέλει ελθεί εις το στρατόπεδον και θέλει διατρίψει έξω της σκηνής αυτού επτά ημέρας.
\par 9 Και την εβδόμην ημέραν θέλει ξυρίσει πάσας τας τρίχας αυτού, την κεφαλήν αυτού και τον πώγωνα αυτού και τα οφρύδια αυτού και πάσας τας τρίχας αυτού θέλει ξυρίσει και θέλει πλύνει τα ιμάτια αυτού και θέλει λούσει το σώμα αυτού εν ύδατι και θέλει είσθαι καθαρός.
\par 10 Και την ογδόην ημέραν θέλει λάβει δύο αρνία αρσενικά άμωμα και εν αρνίον θηλυκόν ενιαύσιον άμωμον και τρία δέκατα σεμιδάλεως διά προσφοράν εξ αλφίτων, εζυμωμένης μετά ελαίου, και εν λογ ελαίου·
\par 11 και θέλει παραστήσει ο ιερεύς ο καθαρίζων τον άνθρωπον τον καθαριζόμενον και αυτά ενώπιον του Κυρίου, εις την θύραν της σκηνής του μαρτυρίου.
\par 12 Και θέλει λάβει ο ιερεύς το εν αρσενικόν αρνίον και θέλει προσφέρει αυτό εις προσφοράν περί ανομίας και το λογ του ελαίου, και θέλει κινήσει αυτά εις κινητήν προσφοράν ενώπιον του Κυρίου.
\par 13 Και θέλει σφάξει το αρνίον εν τω τόπω όπου σφάζουσι την περί αμαρτίας προσφοράν και το ολοκαύτωμα, εν τω τόπω τω αγίω· διότι καθώς η περί αμαρτίας προσφορά, η περί ανομίας προσφορά είναι του ιερέως· είναι αγιώτατον.
\par 14 Και θέλει λάβει ο ιερεύς από του αίματος της περί ανομίας προσφοράς και θέλει βάλει αυτό ο ιερεύς επί τον λοβόν του δεξιού ωτίου του καθαριζομένου και επί τον αντίχειρα της δεξιάς αυτού χειρός και επί τον μεγάλον δάκτυλον του δεξιού αυτού ποδός·
\par 15 και θέλει λάβει ο ιερεύς από του λογ του ελαίου και θέλει χύσει αυτό εις την παλάμην της αριστεράς αυτού χειρός·
\par 16 και θέλει εμβάψει ο ιερεύς τον δάκτυλον αυτού τον δεξιόν εις το έλαιον το εν τη αριστερά αυτού παλάμη, και θέλει ραντίσει εκ του ελαίου διά του δακτύλου αυτού επτάκις ενώπιον του Κυρίου·
\par 17 και εκ του υπολοίπου ελαίου του εν τη παλάμη αυτού θέλει βάλει ο ιερεύς επί τον λοβόν του δεξιού ωτίου του καθαριζομένου, και επί τον αντίχειρα της δεξιάς αυτού χειρός και επί τον μεγάλον δάκτυλον του δεξιού αυτού ποδός, επί το αίμα της περί ανομίας προσφοράς·
\par 18 το δε εναπολειφθέν έλαιον το εν τη παλάμη του ιερέως θέλει χύσει επί την κεφαλήν του καθαριζομένου· και θέλει κάμει εξιλέωσιν ο ιερεύς υπέρ αυτού ενώπιον του Κυρίου.
\par 19 Και θέλει προσφέρει ο ιερεύς την περί αμαρτίας προσφοράν, και θέλει κάμει εξιλέωσιν υπέρ του καθαριζομένου από της ακαθαρσίας αυτού· και έπειτα θέλει σφάξει το ολοκαύτωμα.
\par 20 Και θέλει προσφέρει ο ιερεύς το ολοκαύτωμα και την εξ αλφίτων προσφοράν επί του θυσιαστηρίου· και θέλει κάμει εξιλέωσιν υπέρ αυτού ο ιερεύς, και θέλει είσθαι καθαρός.
\par 21 Εάν δε ήναι πτωχός και δεν ευπορή να φέρη τόσα, τότε θέλει λάβει εν αρνίον διά προσφοράν κινητήν περί ανομίας, διά να κάμη εξιλέωσιν υπέρ αυτού, και εν δέκατον σεμιδάλεως εζυμωμένης μετά ελαίου διά την εξ αλφίτων προσφοράν και εν λογ ελαίου
\par 22 και δύο τρυγόνας ή δύο νεοσσούς περιστερών, όπως ευπορεί να φέρη· και η μεν μία θέλει είσθαι διά την περί αμαρτίας προσφοράν, η δε άλλη διά ολοκαύτωμα.
\par 23 Και θέλει φέρει αυτά την ογδόην ημέραν διά τον καθαρισμόν αυτού προς τον ιερέα εις την θύραν της σκηνής του μαρτυρίου ενώπιον του Κυρίου.
\par 24 Και θέλει λάβει ο ιερεύς το αρνίον της περί ανομίας προσφοράς και το λογ του ελαίου και θέλει κινήσει αυτά ο ιερεύς εις προσφοράν κινητήν ενώπιον του Κυρίου.
\par 25 Και θέλει σφάξει το αρνίον της περί ανομίας προσφοράς· και θέλει λάβει ο ιερεύς από του αίματος της περί ανομίας προσφοράς και θέλει βάλει αυτό επί τον λοβόν του δεξιού ωτίου του καθαριζομένου και επί τον αντίχειρα της δεξιάς αυτού χειρός και επί τον μεγάλον δάκτυλον του δεξιού αυτού ποδός.
\par 26 Και θέλει χύσει ο ιερεύς από του ελαίου εις την παλάμην της αριστεράς αυτού χειρός·
\par 27 και θέλει ραντίσει ο ιερεύς διά του δακτύλου αυτού του δεξιού από του ελαίου, του εν τη παλάμη αυτού τη αριστερά, επτάκις ενώπιον του Κυρίου·
\par 28 και θέλει βάλει ο ιερεύς από του ελαίου, του εν τη παλάμη αυτού, επί τον λοβόν του δεξιού ωτίου του καθαριζομένου, και επί τον αντίχειρα της δεξιάς αυτού χειρός και επί τον μεγάλον δάκτυλον του δεξιού αυτού ποδός, επί τον τόπον του αίματος της περί ανομίας προσφοράς·
\par 29 το δε εναπολειφθέν εκ του ελαίου, του εν τη παλάμη του ιερέως, θέλει βάλει επί την κεφαλήν του καθαριζομένου, διά να κάμη εξιλέωσιν υπέρ αυτού ενώπιον του Κυρίου.
\par 30 Και θέλει προσφέρει την μίαν εκ των τρυγόνων ή εκ των νεοσσών των περιστερών, όπως ευπορεί να φέρη·
\par 31 όπως ευπορεί να φέρη, την μεν διά προσφοράν περί αμαρτίας, την δε άλλην διά ολοκαύτωμα, μετά της εξ αλφίτων προσφοράς· και θέλει κάμει ο ιερεύς εξιλέωσιν υπέρ του καθαριζομένου ενώπιον του Κυρίου.
\par 32 Ούτος είναι ο νόμος περί του έχοντος πληγήν λέπρας, όστις δεν ευπορεί να φέρη τα προς τον καθαρισμόν αυτού.
\par 33 Και ελάλησε Κύριος προς τον Μωϋσήν και προς τον Ααρών, λέγων,
\par 34 Όταν εισέλθητε εις την γην Χαναάν, την οποίαν εγώ σας δίδω εις ιδιοκτησίαν, και βάλω την πληγήν της λέπρας εις τινά οικίαν της γης της ιδιοκτησίας σας·
\par 35 και εκείνος, του οποίου είναι η οικία, έλθη και αναγγείλη προς τον ιερέα, λέγων, Εφάνη εις εμέ ως πληγή εν τη οικία·
\par 36 τότε θέλει προστάξει ο ιερεύς να εκκενώσωσι την οικίαν, πριν υπάγη ο ιερεύς διά να θεωρήση την πληγήν, διά να μη γείνωσιν ακάθαρτα πάντα τα εν τη οικία και μετά ταύτα θέλει εμβή ο ιερεύς διά να θεωρήση την οικίαν·
\par 37 και θέλει θεωρήσει την πληγήν· και ιδού, εάν η πληγή ήναι εις τους τοίχους της οικίας με κοιλώματα πρασινίζοντα ή κοκκινωπά και η θεωρία αυτών ήναι βαθυτέρα του τοίχου·
\par 38 τότε θέλει εξέλθει ο ιερεύς εκ της οικίας εις την θύραν της οικίας και θέλει κλείσει την οικίαν επτά ημέρας.
\par 39 Και θέλει επιστρέψει ο ιερεύς την εβδόμην ημέραν και θέλει θεωρήσει και ιδού, εάν η πληγή εξηπλώθη εις τους τοίχους της οικίας,
\par 40 τότε ο ιερεύς θέλει προστάξει να εκβάλωσι τους λίθους, εις τους οποίους είναι η πληγή, και θέλουσι ρίψει αυτούς έξω της πόλεως εις τόπον ακάθαρτον.
\par 41 Και θέλει κάμει να αποξύσωσι την οικίαν έσωθεν κύκλω, και θέλουσι ρίψει το χώμα το απεξυσμένον έξω της πόλεως εις τόπον ακάθαρτον·
\par 42 και θέλουσι λάβει άλλους λίθους, και βάλει αυτούς αντί των λίθων εκείνων· και θέλουσι λάβει άλλο χώμα, και θέλουσι χρίσει την οικίαν.
\par 43 Και εάν έλθη πάλιν η πληγή και αναφανή εις την οικίαν, αφού εξέβαλον τους λίθους και αφού απέξυσαν την οικίαν και αφού αυτή εχρίσθη,
\par 44 τότε θέλει εισέλθει ο ιερεύς και θέλει θεωρήσει· και ιδού, εάν η πληγή εξηπλώθη εις την οικίαν, είναι λέπρα διαβρωτική εν τη οικία· είναι ακάθαρτος.
\par 45 Και θέλουσι κρημνίσει την οικίαν, τους λίθους αυτής και τα ξύλα αυτής και παν το χώμα της οικίας· και θέλουσι φέρει αυτά έξω της πόλεως εις τόπον ακάθαρτον.
\par 46 Και όστις εισέλθη εις την οικίαν κατά πάσας τας ημέρας, καθ' ας είναι κεκλεισμένη, θέλει είσθαι ακάθαρτος έως εσπέρας.
\par 47 Και όστις κοιμηθή εν τη οικία, θέλει πλύνει τα ιμάτια αυτού· και όστις φάγη εν τη οικία, θέλει πλύνει τα ιμάτια αυτού.
\par 48 Αλλ' εάν ο ιερεύς εισελθών θεωρήση και ιδού, δεν εξηπλώθη η πληγή εν τη οικία, αφού εχρίσθη η οικία, τότε ο ιερεύς θέλει κρίνει την οικίαν καθαράν, διότι ιατρεύθη η πληγή.
\par 49 Και θέλει λάβει, διά να καθαρίση την οικίαν, δύο πτηνά, και ξύλον κέδρινον και κόκκινον και ύσσωπον.
\par 50 Και θέλει σφάξει το εν πτηνόν εις αγγείον πήλινον επάνω ύδατος ζώντος.
\par 51 Και θέλει λάβει το ξύλον το κέδρινον και τον ύσσωπον και το κόκκινον και το πτηνόν το ζων, και εμβάψει αυτά εις το αίμα του εσφαγμένου πτηνού και εις το ύδωρ το ζων, και θέλει ραντίσει την οικίαν επτάκις.
\par 52 Και θέλει καθαρίσει την οικίαν διά του αίματος του πτηνού και διά του ύδατος του ζώντος και διά του πτηνού του ζώντος και διά του ξύλου του κεδρίνου και διά του υσσώπου και διά του κοκκίνου.
\par 53 Το δε ζων πτηνόν θέλει απολύσει έξω της πόλεως επί πρόσωπον της πεδιάδος, και θέλει κάμει εξιλέωσιν υπέρ της οικίας· και θέλει είσθαι καθαρά.
\par 54 Ούτος είναι ο νόμος περί πάσης πληγής λέπρας και κασίδας,
\par 55 και περί λέπρας ιματίου και οικίας,
\par 56 και περί πρήσματος και περί ψώρας και περί εξανθήματος·
\par 57 διά να γίνηται γνωστόν πότε είναι τι ακάθαρτον και πότε καθαρόν· ούτος είναι ο νόμος περί της λέπρας.

\chapter{15}

\par Και ελάλησε Κύριος προς τον Μωϋσήν και προς τον Ααρών, λέγων,
\par 2 Λαλήσατε προς τους υιούς Ισραήλ, και είπατε προς αυτούς, Εάν τις άνθρωπος έχη ρεύσιν εκ του σώματος αυτού, διά την ρεύσιν αυτού είναι ακάθαρτος.
\par 3 Και αύτη θέλει είσθαι η ακαθαρσία αυτού εν τη ρεύσει αυτού· αν τε το σώμα αυτού παύση από της ρεύσεως αυτού· είναι η ακαθαρσία εν αυτώ.
\par 4 Πάσα κλίνη, επί της οποίας ήθελε κοιμηθή ο έχων την ρεύσιν, θέλει είσθαι ακάθαρτος· και παν σκεύος, επί του οποίου ήθελε καθίσει, θέλει είσθαι ακάθαρτον.
\par 5 Και ο άνθρωπος, όστις εγγίση την κλίνην αυτού, θέλει πλύνει τα ιμάτια αυτού και θέλει λουσθή εν ύδατι και θέλει είσθαι ακάθαρτος έως εσπέρας.
\par 6 Και όστις καθίση επί του σκεύους, επί του οποίου εκάθισεν ο έχων την ρεύσιν, θέλει πλύνει τα ιμάτια αυτού και θέλει λουσθή εν ύδατι και θέλει είσθαι ακάθαρτος έως εσπέρας.
\par 7 Και όστις εγγίση το σώμα του έχοντος την ρεύσιν, θέλει πλύνει τα ιμάτια αυτού και θέλει λουσθή εν ύδατι και θέλει είσθαι ακάθαρτος έως εσπέρας.
\par 8 Και εάν ο έχων την ρεύσιν πτύση επί τον καθαρόν, ούτος θέλει πλύνει τα ιμάτια αυτού και θέλει λουσθή εν ύδατι και θέλει είσθαι ακάθαρτος έως εσπέρας.
\par 9 Και παν σαμάριον επί του οποίου ήθελε καθίσει ο έχων την ρεύσιν, θέλει είσθαι ακάθαρτον.
\par 10 Και όστις εγγίση πάντα, όσα ήθελον είσθαι υποκάτω αυτού, θέλει είσθαι ακάθαρτος έως εσπέρας· και όστις βαστάση αυτά, θέλει πλύνει τα ιμάτια αυτού και θέλει λουσθή εν ύδατι και θέλει είσθαι ακάθαρτος έως εσπέρας.
\par 11 Και όντινα εγγίση ο έχων την ρεύσιν, χωρίς να έχη νιμμένας τας χείρας αυτού εν ύδατι, ούτος θέλει πλύνει τα ιμάτια αυτού και θέλει λουσθή εν ύδατι και θέλει είσθαι ακάθαρτος έως εσπέρας.
\par 12 Και το αγγείον το πήλινον, το οποίον ήθελεν εγγίσει ο έχων την ρεύσιν, θέλει συντριφθή· και παν σκεύος ξύλινον θέλει πλυθή εν ύδατι.
\par 13 Και αφού ο έχων την ρεύσιν καθαρισθή από της ρεύσεως αυτού, τότε θέλει αριθμήσει εις εαυτόν επτά ημέρας διά τον καθαρισμόν αυτού· και θέλει πλύνει τα ιμάτια αυτού και θέλει λούσει το σώμα αυτού εν ύδατι ζώντι και θέλει είσθαι καθαρός.
\par 14 Και την ογδόην ημέραν θέλει λάβει εις εαυτόν δύο τρυγόνας ή δύο νεοσσούς περιστερών και θέλει ελθεί ενώπιον του Κυρίου εις την θύραν της σκηνής του μαρτυρίου και θέλει δώσει αυτάς εις τον ιερέα·
\par 15 και θέλει προσφέρει αυτάς ο ιερεύς, την μεν διά προσφοράν περί αμαρτίας, την δε άλλην διά ολοκαύτωμα· και θέλει κάμει εξιλέωσιν ο ιερεύς υπέρ αυτού ενώπιον του Κυρίου διά την ρεύσιν αυτού.
\par 16 Και ο άνθρωπος, εκ του οποίου ήθελεν εξέλθει σπέρμα συνουσίας, θέλει λούσει όλον αυτού το σώμα εν ύδατι και θέλει είσθαι ακάθαρτος έως εσπέρας.
\par 17 Και παν ιμάτιον και παν δέρμα, επί του οποίου ήθελεν είσθαι σπέρμα συνουσίας, θέλει πλυθή εν ύδατι και θέλει είσθαι ακάθαρτον έως εσπέρας.
\par 18 Η δε γυνή, μετά της οποίας ήθελε συγκοιμηθή άνθρωπος εν σπέρματι συνουσίας, θέλουσι λουσθή εν ύδατι και θέλουσιν είσθαι ακάθαρτοι έως εσπέρας.
\par 19 Και εάν η γυνή έχη ρεύσιν, η δε ρεύσις αυτής εν τω σώματι αυτής ήναι αίμα, θέλει είσθαι αποκεχωρισμένη επτά ημέρας· και πας όστις εγγίση αυτήν, θέλει είσθαι ακάθαρτος έως εσπέρας.
\par 20 Και παν πράγμα, επί του οποίου κοίτεται εις τον αποχωρισμόν αυτής, θέλει είσθαι ακάθαρτον· και παν πράγμα, επί του οποίου κάθηται, θέλει είσθαι ακάθαρτον.
\par 21 Και πας όστις εγγίση την κλίνην αυτής, θέλει πλύνει τα ιμάτια αυτού και θέλει λουσθή εν ύδατι και θέλει είσθαι ακάθαρτος έως εσπέρας.
\par 22 Και πας όστις εγγίση σκεύος τι, επί του οποίου αυτή εκάθισε, θέλει πλύνει τα ιμάτια αυτού και θέλει λουσθή εν ύδατι και θέλει είσθαι ακάθαρτος έως εσπέρας.
\par 23 Και εάν υπάρχη τι επί της κλίνης ή επί τινός σκεύους, επί του οποίου αυτή κάθηται, όταν αυτός εγγίση αυτό, θέλει είσθαι ακάθαρτος έως εσπέρας.
\par 24 Και εάν τις συγκοιμηθή μετ' αυτής και έλθωσι τα γυναικεία αυτής επ' αυτόν, θέλει είσθαι ακάθαρτος επτά ημέρας· και πάσα κλίνη, επί της οποίας ήθελε κοιμηθή, θέλει είσθαι ακάθαρτος.
\par 25 Και εάν τις γυνή έχη ρεύσιν του αίματος αυτής πολλάς ημέρας, εκτός του καιρού του αποχωρισμού αυτής, ή εάν έχη ρεύσιν επέκεινα του αποχωρισμού αυτής, πάσαι αι ημέραι της ρεύσεως της ακαθαρσίας αυτής θέλουσιν είσθαι ως αι ημέραι του αποχωρισμού αυτής· θέλει είσθαι ακάθαρτος.
\par 26 Πάσα κλίνη, επί της οποίας κοίτεται καθ' όλας τας ημέρας της ρεύσεως αυτής, θέλει είσθαι εις αυτήν ως κλίνη του αποχωρισμού αυτής· και παν σκεύος, επί του οποίου κάθηται, θέλει είσθαι ακάθαρτον, ως η ακαθαρσία του αποχωρισμού αυτής.
\par 27 Και πας όστις εγγίση αυτά, θέλει είσθαι ακάθαρτος και θέλει πλύνει τα ιμάτια αυτού και θέλει λουσθή εν ύδατι και θέλει είσθαι ακάθαρτος έως εσπέρας.
\par 28 Αλλ' εάν καθαρισθή από της ρεύσεως αυτής, τότε θέλει αριθμήσει εις εαυτήν επτά ημέρας, και μετά ταύτα θέλει είσθαι καθαρά.
\par 29 Και την ογδόην ημέραν θέλει λάβει μεθ' εαυτής δύο τρυγόνας ή δύο νεοσσούς περιστερών και θέλει φέρει αυτάς προς τον ιερέα εις την θύραν της σκηνής του μαρτυρίου.
\par 30 Και θέλει προσφέρει ο ιερεύς την μεν διά προσφοράν περί αμαρτίας, την δε άλλην διά ολοκαύτωμα· και ο ιερεύς θέλει κάμει εξιλέωσιν περί αυτής ενώπιον του Κυρίου διά την ρεύσιν της ακαθαρσίας αυτής.
\par 31 Ούτω θέλετε χωρίζει τους υιούς Ισραήλ από των ακαθαρσιών αυτών· και δεν θέλουσιν αποθάνει διά την ακαθαρσίαν αυτών, μιαίνοντες την σκηνήν μου την εν τω μέσω αυτών.
\par 32 Ούτος είναι ο νόμος περί του έχοντος ρεύσιν· και περί εκείνου, εκ του οποίου εξέρχεται το σπέρμα συνουσίας, διά να μιαίνηται δι' αυτού·
\par 33 και περί της ασθενούσης διά τα γυναικεία αυτής· και περί του έχοντος την ρεύσιν αυτού, ανδρός ή γυναικός, και περί του συγκοιμηθέντος μετά της ακαθάρτου.

\chapter{16}

\par Και ελάλησε Κύριος προς τον Μωϋσήν μετά τον θάνατον των δύο υιών του Ααρών, ότε έκαμον προσφοράν ενώπιον του Κυρίου και απέθανον·
\par 2 και είπε Κύριος προς τον Μωϋσήν, Λάλησον προς Ααρών τον αδελφόν σου, να μη εισέρχηται πάσαν ώραν εις το αγιαστήριον το ένδοθεν του καταπετάσματος έμπροσθεν του ιλαστηρίου του επί της κιβωτού, διά να μη αποθάνη· διότι εν νεφέλη θέλω εμφανίζεσθαι επί του ιλαστηρίου.
\par 3 Ούτω θέλει εισέρχεσθαι ο Ααρών εις το αγιαστήριον, μετά μόσχου εκ βοών διά προσφοράν περί αμαρτίας και κριού διά ολοκαύτωμα.
\par 4 Χιτώνα λινούν ηγιασμένον θέλει ενδύεσθαι, και περισκελή λινά θέλουσιν είσθαι επί της σαρκός αυτού, και ζώνην λινήν θέλει είσθαι εζωσμένος και μίτραν λινήν θέλει φορεί· ταύτα είναι ενδύματα άγια· και θέλει λούει εν ύδατι το σώμα αυτού και θέλει ενδύεσθαι αυτά.
\par 5 Και παρά της συναγωγής των υιών Ισραήλ θέλει λάβει δύο τράγους εξ αιγών διά προσφοράν περί αμαρτίας και ένα κριόν διά ολοκαύτωμα.
\par 6 Και θέλει προσφέρει ο Ααρών τον μόσχον της περί αμαρτίας προσφοράς, όστις είναι δι' εαυτόν, και θέλει κάμει εξιλέωσιν υπέρ εαυτού και υπέρ του οίκου αυτού.
\par 7 Και θέλει λάβει τους δύο τράγους και στήσει αυτούς ενώπιον του Κυρίου εις την θύραν της σκηνής του μαρτυρίου.
\par 8 Και θέλει ρίψει ο Ααρών κλήρους επί τους δύο τράγους· ένα κλήρον διά τον Κύριον και ένα κλήρον διά τον τράγον τον απολυτέον.
\par 9 Και θέλει φέρει ο Ααρών τον τράγον, επί του οποίου έπεσεν ο κλήρος του Κυρίου, και θέλει προσφέρει αυτόν διά προσφοράν περί αμαρτίας.
\par 10 Τον δε τράγον, επί του οποίου έπεσεν ο κλήρος του να απολυθή, θέλει στήσει ζώντα ενώπιον του Κυρίου, διά να κάμη εξιλέωσιν επ' αυτού, ώστε να αποστείλη αυτόν απόλυτον εις την έρημον.
\par 11 Και θέλει φέρει ο Ααρών τον μόσχον της περί αμαρτίας προσφοράς, όστις είναι δι' εαυτόν, και θέλει κάμει εξιλέωσιν υπέρ εαυτού και υπέρ του οίκου αυτού· και θέλει σφάξει τον μόσχον της περί αμαρτίας προσφοράς τον περί εαυτού.
\par 12 Και θέλει λάβει το θυμιατήριον πλήρες ανθράκων πυρός εκ του θυσιαστηρίου απ' έμπροσθεν του Κυρίου· και θέλει γεμίσει τας χείρας αυτού από ευώδους θυμιάματος λειοτριβημένον και θέλει φέρει αυτό ένδον του καταπετάσματος.
\par 13 Και θέλει βάλει το θυμίαμα επί το πυρ ενώπιον του Κυρίου, και θέλει καλύψει ο καπνός του θυμιάματος το ιλαστήριον το επί του μαρτυρίου, διά να μη αποθάνη.
\par 14 Και θέλει λάβει από του αίματος του μόσχου και ραντίσει διά του δακτύλου αυτού επί το ιλαστήριον κατά ανατολάς· και έμπροσθεν του ιλαστηρίου θέλει ραντίσει επτάκις από του αίματος διά του δακτύλου αυτού.
\par 15 Τότε θέλει σφάξει τον τράγον της περί αμαρτίας προσφοράς τον περί του λαού· και θέλει φέρει το αίμα αυτού ένδον του καταπετάσματος και θέλει κάμει το αίμα αυτού, καθώς έκαμε το αίμα του μόσχου, και θέλει ραντίσει αυτό επί το ιλαστήριον και έμπροσθεν του ιλαστηρίου.
\par 16 Και θέλει κάμει εξιλέωσιν υπέρ του αγιαστηρίου διά τας ακαθαρσίας των υιών Ισραήλ, και διά τας παραβάσεις αυτών καθ' όλας αυτών τας αμαρτίας· και ούτω θέλει κάμει περί της σκηνής του μαρτυρίου, ήτις κατοικεί μεταξύ αυτών εν τω μέσω της ακαθαρσίας αυτών.
\par 17 Ουδείς δε άνθρωπος θέλει είσθαι εν τη σκηνή του μαρτυρίου, όταν αυτός εισέρχηται να κάμη εξιλέωσιν εις το αγιαστήριον, εωσού εξέλθη, αφού κάμη εξιλέωσιν υπέρ εαυτού και υπέρ του οίκου αυτού και υπέρ πάσης της συναγωγής του Ισραήλ.
\par 18 Τότε θέλει εξέλθει προς το θυσιαστήριον το ενώπιον του Κυρίου και θέλει κάμει εξιλέωσιν περί αυτού· και θέλει λάβει από του αίματος του μόσχου και από του αίματος του τράγου και βάλει επί τα κέρατα του θυσιαστηρίου κύκλω.
\par 19 Και θέλει ραντίσει επ' αυτό από του αίματος διά του δακτύλου αυτού επτάκις και θέλει καθαρίσει αυτό, και αγιάσει αυτό από των ακαθαρσιών των υιών Ισραήλ.
\par 20 Αφού δε τελειώση να κάμνη εξιλέωσιν υπέρ του αγιαστηρίου και της σκηνής του μαρτυρίου και του θυσιαστηρίου, θέλει φέρει τον τράγον τον ζώντα·
\par 21 και θέλει επιθέσει ο Ααρών τας δύο χείρας αυτού επί την κεφαλήν του τράγου του ζώντος και θέλει εξομολογηθή επ' αυτού πάσας τας ανομίας των υιών Ισραήλ και πάσας τας παραβάσεις αυτών καθ' όλας αυτών τας αμαρτίας· και θέλει επιθέσει αυτάς εις την κεφαλήν του τράγου· και θέλει αποστείλει αυτόν διά χειρός διωρισμένου ανθρώπου εις την έρημον.
\par 22 Και θέλει βαστάσει ο τράγος εφ' εαυτού πάσας τας ανομίας αυτών εις γην ακατοίκητον· και θέλει απολύσει τον τράγον εις την έρημον.
\par 23 Και θέλει εισέλθει ο Ααρών εις την σκηνήν του μαρτυρίου και θέλει εκδυθή την λινήν στολήν, την οποίαν ενεδύθη εισερχόμενος εις το αγιαστήριον, και θέλει αποθέσει αυτήν εκεί·
\par 24 και θέλει λούσει το σώμα αυτού εν ύδατι εν τόπω αγίω και ενδυθή τα ιμάτια αυτού, και θέλει έλθει και προσφέρει το ολοκαύτωμα αυτού και το ολοκαύτωμα του λαού και θέλει κάμει εξιλέωσιν περί εαυτού και περί του λαού.
\par 25 Το δε στέαρ της περί αμαρτίας προσφοράς θέλει καύσει επί του θυσιαστηρίου.
\par 26 Και ο αποστείλας τον τράγον τον απολυτέον θέλει πλύνει τα ιμάτια αυτού και λούσει το σώμα αυτού εν ύδατι και μετά ταύτα θέλει εισέλθει εις το στρατόπεδον.
\par 27 Τον δε μόσχον της περί αμαρτίας προσφοράς και τον τράγον της περί αμαρτίας προσφοράς, των οποίων το αίμα εισήχθη διά να γείνη εξιλέωσις εις το αγιαστήριον, θέλουσι φέρει έξω του στρατοπέδου· και θέλουσι καύσει εν τω πυρί τα δέρματα αυτών και το κρέας αυτών και την κόπρον αυτών.
\par 28 Και ο καίων αυτά θέλει πλύνει τα ιμάτια αυτού και λούσει το σώμα αυτού εν ύδατι και μετά ταύτα θέλει εισέλθει εις το στρατόπεδον.
\par 29 Και τούτο θέλει είσθαι εις εσάς νόμιμον αιώνιον· εις τον έβδομον μήνα, την δεκάτην του μηνός, θέλετε ταπεινώσει τας ψυχάς σας και δεν θέλετε κάμει ουδέν έργον ούτε ο αυτόχθων ούτε ο ξένος ο παροικών μεταξύ σας·
\par 30 διότι εν τη ημέρα ταύτη ο ιερεύς θέλει κάμει εξιλέωσιν διά σας, διά να σας καθαρίση, ώστε να ήσθε καθαροί από πασών των αμαρτιών υμών ενώπιον του Κυρίου.
\par 31 Σάββατον αναπαύσεως θέλει είσθαι εις εσάς, και θέλετε ταπεινώσει τας ψυχάς σας κατά νόμιμον αιώνιον.
\par 32 Και θέλει κάμει την εξιλέωσιν ο ιερεύς, ο χρισθείς και καθιερωθείς διά να ιερατεύη αντί του πατρός αυτού, και θέλει ενδυθή την λινήν στολήν, την στολήν την αγίαν.
\par 33 Και θέλει κάμει εξιλέωσιν υπέρ του αγίου αγιαστηρίου και θέλει κάμει εξιλέωσιν υπέρ της σκηνής του μαρτυρίου και υπέρ του θυσιαστηρίου· και θέλει κάμει εξιλέωσιν υπέρ των ιερέων και υπέρ παντός του λαού της συναγωγής.
\par 34 Και τούτο θέλει είσθαι εις εσάς νόμιμον αιώνιον, να κάμνητε εξιλέωσιν υπέρ των υιών Ισραήλ περί πασών των αμαρτιών αυτών άπαξ του ενιαυτού. Και έγεινε καθώς προσέταξεν ο Κύριος εις τον Μωϋσήν.

\chapter{17}

\par Και ελάλησε Κύριος προς τον Μωϋσήν, λέγων,
\par 2 Λάλησον προς τον Ααρών και προς τους υιούς αυτού και προς πάντας τους υιούς Ισραήλ και ειπέ προς αυτούς, Ούτος είναι ο λόγος τον οποίον προσέταξεν ο Κύριος, λέγων.
\par 3 Όστις άνθρωπος εκ του οίκου Ισραήλ σφάξη βουν ή αρνίον ή αίγα εν τω στρατοπέδω, ή όστις σφάξη έξω του στρατοπέδου,
\par 4 και εις την θύραν της σκηνής του μαρτυρίου δεν φέρη αυτό, διά να προσφέρη προσφοράν εις τον Κύριον έμπροσθεν της σκηνής του Κυρίου, αίμα θέλει λογισθή εις εκείνον τον άνθρωπον· αίμα έχυσε και θέλει εξολοθρευθή ο άνθρωπος εκείνος εκ μέσου του λαού αυτού·
\par 5 διά να φέρωσιν οι υιοί Ισραήλ τας θυσίας αυτών, τας οποίας θυσιάζουσιν εν τη πεδιάδι, και να προσφέρωσιν αυτάς προς τον Κύριον εις την θύραν της σκηνής του μαρτυρίου προς τον ιερέα και να θυσιάζωσιν αυτάς εις προσφοράς ειρηνικάς προς τον Κύριον.
\par 6 Και θέλει ραντίσει ο ιερεύς το αίμα επί το θυσιαστήριον του Κυρίου εις την θύραν της σκηνής του μαρτυρίου και θέλει καύσει το στέαρ εις οσμήν ευωδίας προς τον Κύριον.
\par 7 Και δεν θέλουσι θυσιάσει πλέον τας θυσίας αυτών εις τους δαίμονας, κατόπιν των οποίων αυτοί πορνεύουσι· τούτο θέλει είσθαι εις αυτούς νόμιμον αιώνιον εις τας γενεάς αυτών.
\par 8 Και θέλεις ειπεί προς αυτούς, Όστις άνθρωπος εκ του οίκου Ισραήλ ή εκ των ξένων των παροικούντων μεταξύ σας προσφέρη ολοκαύτωμα ή θυσίαν,
\par 9 και εις την θύραν της σκηνής του μαρτυρίου δεν φέρη αυτό, διά να προσφέρη αυτό προς τον Κύριον, θέλει εξολοθρευθή ο άνθρωπος εκείνος εκ μέσου του λαού αυτού.
\par 10 Και όστις άνθρωπος εκ του οίκου Ισραήλ ή εκ των ξένων των παροικούντων μεταξύ σας φάγη οιονδήποτε αίμα, θέλω στήσει το πρόσωπόν μου εναντίον εκείνης της ψυχής ήτις τρώγει το αίμα, και θέλω εξολοθρεύσει αυτήν εκ μέσου του λαού αυτής·
\par 11 διότι η ζωή της σαρκός είναι εν τω αίματι και εγώ έδωκα αυτό εις εσάς, διά να κάμνητε εξιλέωσιν υπέρ των ψυχών σας επί του θυσιαστηρίου· διότι το αίμα τούτο κάμνει εξιλασμόν υπέρ της ψυχής.
\par 12 Διά τούτο είπα προς τους υιούς Ισραήλ, Ουδεμία ψυχή από σας θέλει φάγει αίμα· ουδέ ο ξένος, ο παροικών μεταξύ σας, θέλει φάγει αίμα.
\par 13 Και όστις άνθρωπος εκ των υιών Ισραήλ ή εκ των ξένων των παροικούντων μεταξύ σας, κυνηγήση και πιάση ζώον ή πτηνόν, το οποίον τρώγεται, θέλει χύσει το αίμα αυτού και θέλει σκεπάσει αυτό με χώμα.
\par 14 Διότι η ζωή πάσης σαρκός είναι το αίμα αυτής· διά την ζωήν αυτής είναι· όθεν είπα προς τους υιούς Ισραήλ, Δεν θέλετε φάγει αίμα ουδεμιάς σαρκός· διότι η ζωή πάσης σαρκός είναι το αίμα αυτής· πας ο τρώγων αυτό θέλει εξολοθρευθή.
\par 15 Και πάσα ψυχή, ήτις φάγη θνησιμαίον ή διεσπαραγμένον υπό θηρίου, αυτόχθων ή ξένος, θέλει πλύνει τα ιμάτια αυτού και θέλει λουσθή εν ύδατι και θέλει είσθαι ακάθαρτος έως εσπέρας· τότε θέλει είσθαι καθαρός.
\par 16 Αλλ' εάν δεν πλύνη αυτά μηδέ λούση το σώμα αυτού, τότε θέλει βαστάσει την ανομίαν αυτού.

\chapter{18}

\par Και ελάλησε Κύριος προς τον Μωϋσήν, λέγων,
\par 2 Λάλησον προς τους υιούς Ισραήλ και ειπέ προς αυτούς, Εγώ είμαι Κύριος, ο Θεός σας.
\par 3 Κατά τας πράξεις της γης Αιγύπτου, εν ή κατωκήσατε, δεν θέλετε πράξει και κατά τας πράξεις της γης Χαναάν, εις την οποίαν εγώ σας φέρω, δεν θέλετε πράξει και κατά τα νόμιμα αυτών δεν θέλετε περιπατήσει.
\par 4 Τας κρίσεις μου θέλετε κάμει και τα προστάγματά μου θέλετε φυλάττει, διά να περιπατήτε εις αυτά. Εγώ είμαι Κύριος ο Θεός σας.
\par 5 Θέλετε φυλάττει λοιπόν τα προστάγματά μου και τας κρίσεις μου· τα οποία κάμνων ο άνθρωπος, θέλει ζήσει δι' αυτών. Εγώ είμαι ο Κύριος.
\par 6 Ουδείς άνθρωπος θέλει πλησιάσει εις ουδένα συγγενή αυτού κατά σάρκα, διά να αποκαλύψη την ασχημοσύνην αυτού. Εγώ είμαι ο Κύριος.
\par 7 Ασχημοσύνην πατρός σου, ή ασχημοσύνην μητρός σου δεν θέλεις αποκαλύψει· είναι μήτηρ σου· δεν θέλεις αποκαλύψει την ασχημοσύνην αυτής.
\par 8 Ασχημοσύνην γυναικός του πατρός σου δεν θέλεις αποκαλύψει· είναι ασχημοσύνη του πατρός σου.
\par 9 Ασχημοσύνην αδελφής σου θυγατρός του πατρός σου ή θυγατρός της μητρός σου, γεννημένης εν τη οικία ή γεννημένης έξω, τούτων την ασχημοσύνην δεν θέλεις αποκαλύψει.
\par 10 Ασχημοσύνην θυγατρός του υιού σου ή θυγατρός της θυγατρός σου, τούτων την ασχημοσύνην δεν θέλεις αποκαλύψει διότι ιδική σου είναι η ασχημοσύνη αυτών.
\par 11 Ασχημοσύνην θυγατρός της γυναικός του πατρός σου, γεννημένης από του πατρός σου, ήτις είναι αδελφή σου, δεν θέλεις αποκαλύψει την ασχημοσύνην αυτής.
\par 12 Ασχημοσύνην αδελφής του πατρός σου δεν θέλεις αποκαλύψει είναι στενή συγγενής του πατρός σου.
\par 13 Ασχημοσύνην αδελφής της μητρός σου δεν θέλεις αποκαλύψει· διότι είναι στενή συγγενής της μητρός σου.
\par 14 Ασχημοσύνην αδελφού του πατρός σου δεν θέλεις αποκαλύψει· εις την γυναίκα αυτού δεν θέλεις πλησιάσει· είναι θεία σου.
\par 15 Ασχημοσύνην νύμφης σου δεν θέλεις αποκαλύψει· είναι γυνή του υιού σου· δεν θέλεις αποκαλύψει την ασχημοσύνην αυτής.
\par 16 Ασχημοσύνην αδελφού σου δεν θέλεις αποκαλύψει· είναι η ασχημοσύνη του αδελφού σου.
\par 17 Ασχημοσύνην γυναικός και της θυγατρός αυτής δεν θέλεις αποκαλύψει ουδέ θέλεις λάβει την θυγατέρα του υιού αυτής ή την θυγατέρα της θυγατρός αυτής, διά να αποκαλύψης την ασχημοσύνην αυτής· είναι στεναί συγγενείς αυτής· είναι ασέβημα.
\par 18 Και γυναίκα προς τη αδελφή αυτής αντίζηλον δεν θέλεις λάβει, διά να αποκαλύψης την ασχημοσύνην αυτής προς τη άλλη, εν όσω ζη.
\par 19 Και εις γυναίκα, εν καιρώ αποχωρισμού διά την ακαθαρσίαν αυτής δεν θέλεις πλησιάσει διά να αποκαλύψης την ασχημοσύνην αυτής.
\par 20 Και μετά της γυναικός του πλησίον σου δεν θέλεις συνουσιασθή, διά να μιανθής μετ' αυτής.
\par 21 Και δεν θέλεις αφήσει τινά εκ του σπέρματός σου να περάση διά του πυρός εις τον Μολόχ και δεν θέλεις βεβηλώσει το όνομα του Θεού σου. Εγώ είμαι ο Κύριος.
\par 22 Και μετά άρρενος δεν θέλεις συνουσιασθή, ως μετά γυναικός· είναι βδέλυγμα.
\par 23 Ουδέ θέλεις συνουσιασθή μετ' ουδενός κτήνους, διά να μιανθής μετ' αυτού· ουδέ γυνή θέλει σταθή έμπροσθεν κτήνους, διά να βατευθή· είναι μυσαρόν.
\par 24 Μη μιαίνεσθε εις ουδέν εκ τούτων· διότι εις πάντα ταύτα εμιάνθησαν τα έθνη, τα οποία εγώ εκδιώκω απ' έμπροσθέν σας·
\par 25 και εμιάνθη η γή· διά τούτο ανταποδίδω την ανομίαν αυτής επ' αυτήν, και η γη θέλει εξεμέσει τους κατοίκους αυτής.
\par 26 Σεις λοιπόν θέλετε φυλάξει τα προστάγματά μου και τας κρίσεις μου και δεν θέλετε πράττει ουδέν εκ πάντων των βδελυγμάτων τούτων, ο αυτόχθων ή ο ξένος ο παροικών μεταξύ σας·
\par 27 διότι πάντα τα βδελύγματα ταύτα έπραξαν οι άνθρωποι της γης, οι προ υμών, και εμιάνθη η γή·
\par 28 διά να μη σας εξεμέση η γη, όταν μιάνητε αυτήν, καθώς εξήμεσε τα έθνη τα προ υμών.
\par 29 Διότι πας όστις πράξη τι εκ των βδελυγμάτων τούτων, αι ψυχαί αίτινες ήθελον πράξει αυτά θέλουσιν εξολοθρευθή εκ μέσου του λαού αυτών.
\par 30 Όθεν θέλετε φυλάττει τα προστάγματά μου, ώστε να μη πράξητε μηδέν εκ των βδελυρών τούτων νομίμων, τα οποία επράχθησαν προ υμών, και να μη μιανθήτε εις αυτά. Εγώ είμαι Κύριος ο Θεός σας.

\chapter{19}

\par Και ελάλησε Κύριος προς τον Μωϋσήν, λέγων,
\par 2 Λάλησον προς πάσαν την συναγωγήν των υιών Ισραήλ και ειπέ προς αυτούς, Άγιοι θέλετε είσθαι· διότι άγιος είμαι εγώ Κύριος ο Θεός σας.
\par 3 Θέλετε φοβείσθαι έκαστος την μητέρα αυτού και τον πατέρα αυτού· και τα σάββατά μου θέλετε φυλάττει. Εγώ είμαι Κύριος ο Θεός σας.
\par 4 Μη στραφήτε εις είδωλα μηδέ κάμητε εις εαυτούς θεούς χωνευτούς. Εγώ είμαι Κύριος ο Θεός σας.
\par 5 Και όταν προσφέρητε θυσίαν ειρηνικής προσφοράς προς τον Κύριον, αυτοπροαιρέτως θέλετε προσφέρει αυτήν.
\par 6 Θέλει τρώγεσθαι την ημέραν καθ' ην προσφέρετε αυτήν, και την επαύριον· εάν δε μείνη τι έως της τρίτης ημέρας, με πυρ θέλει κατακαυθή.
\par 7 Εάν δε ποτέ φαγωθή την ημέραν την τρίτην, είναι βδελυκτόν· δεν θέλει είσθαι ευπρόσδεκτος.
\par 8 Διά τούτο όστις φάγη αυτήν, θέλει βαστάσει την ανομίαν αυτού, διότι εβεβήλωσε τα άγια του Κυρίου· και η ψυχή αύτη θέλει εξολοθρευθή εκ του λαού αυτής.
\par 9 Και όταν θερίζητε τον θερισμόν της γης σας, δεν θέλεις θερίσει ολοκλήρως τας άκρας του αγρού σου και τα αποπίπτοντα του θερισμού σου δεν θέλεις συλλέξει.
\par 10 Και τον αμπελώνά σου δεν θέλεις επανατρυγήσει ούτε τας ρώγας του αμπελώνός σου θέλεις συλλέξει· εις τον πτωχόν και εις τον ξένον θέλεις αφήσει αυτάς. Εγώ είμαι Κύριος ο Θεός σας.
\par 11 Δεν θέλετε κλέπτει ουδέ θέλετε ψεύδεσθαι ουδέ θέλετε απατήσει έκαστος τον πλησίον αυτού.
\par 12 Και δεν θέλετε ομνύει εις το όνομά μου ψευδώς και δεν θέλεις βεβηλόνει το όνομα του Θεού σου. Εγώ είμαι ο Κύριος.
\par 13 Δεν θέλεις αδικήσει τον πλησίον σου ουδέ θέλεις αρπάσει· δεν θέλει διανυκτερεύσει ο μισθός του μισθωτού μετά σου έως πρωΐ.
\par 14 Δεν θέλεις κακολογήσει τον κωφόν, και έμπροσθεν του τυφλού δεν θέλεις βάλει πρόσκομμα, αλλά θέλεις φοβηθή τον Θεόν σου. Εγώ είμαι ο Κύριος.
\par 15 Δεν θέλετε κάμει αδικίαν εις κρίσιν· δεν θέλεις αποβλέψει εις πρόσωπον πτωχού ουδέ θέλεις σεβασθή πρόσωπον δυνάστου· εν δικαιοσύνη θέλεις κρίνει τον πλησίον σου.
\par 16 Δεν θέλεις περιφέρεσθαι συκοφαντών μεταξύ του λαού σου· ουδέ θέλεις σηκωθή κατά του αίματος του πλησίον σου. Εγώ είμαι ο Κύριος.
\par 17 Δεν θέλεις μισήσει τον αδελφόν σου εν τη καρδία σου· θέλεις ελέγξει παρρησία τον πλησίον σου και δεν θέλεις υποφέρει αμαρτίαν επ' αυτόν.
\par 18 Δεν θέλεις εκδικείσθαι ουδέ θέλεις μνησικακεί κατά των υιών του λαού σου· αλλά θέλεις αγαπά τον πλησίον σου ως σεαυτόν. Εγώ είμαι ο Κύριος.
\par 19 Τα νόμιμά μου θέλετε φυλάττει· δεν θέλεις κάμει τα κτήνη σου να βατεύωνται με ετεροειδή· εις τον αγρόν σου δεν θέλεις σπείρει ετεροειδή σπέρματα· ουδέ θέλεις βάλει επάνω σου ένδυμα σύμμικτον εξ ετεροειδούς κλωστής.
\par 20 Και εάν τις συνουσιασθή μετά γυναικός, ήτις είναι δούλη ηρραβωνισμένη μετά ανδρός και δεν είναι εξηγορασμένη, ουδέ εδόθη εις αυτήν η ελευθερία, θέλουσι μαστιγωθή· δεν θέλουσι φονευθή, διότι αυτή δεν ήτο ελευθέρα.
\par 21 Και αυτός θέλει φέρει την περί ανομίας προσφοράν αυτού προς τον Κύριον εις την θύραν της σκηνής του μαρτυρίου, κριόν διά προσφοράν περί ανομίας.
\par 22 Και θέλει κάμει ο ιερεύς εξιλέωσιν περί αυτού διά του κριού της περί ανομίας προσφοράς ενώπιον του Κυρίου, διά την αμαρτίαν αυτού την οποίαν ημάρτησε· και θέλει συγχωρηθή εις αυτόν η αμαρτία αυτού την οποίαν ημάρτησε.
\par 23 Και όταν εισέλθητε εις την γην και φυτεύσητε παν δένδρον τρόφιμον, τότε θέλετε περικαθαρίζει τον καρπόν αυτού ως ακάθαρτον· τρία έτη θέλει είσθαι εις εσάς ακάθαρτος· δεν θέλει τρώγεσθαι.
\par 24 Και εις το τέταρτον έτος θέλει είσθαι όλος ο καρπός αυτού άγιος εις δόξαν του Κυρίου.
\par 25 Εις δε το πέμπτον έτος θέλετε τρώγει τον καρπόν αυτού, διά να πληθυνθή εις εσάς το εισόδημα αυτού. Εγώ είμαι Κύριος ο Θεός σας.
\par 26 Δεν θέλετε τρώγει ουδέν μετά του αίματος αυτού· ουδέ θέλετε μεταχειρίζεσθαι μαντείας ουδέ θέλετε προμαντεύει καιρούς.
\par 27 Δεν θέλετε κουρεύσει κυκλοειδώς την κόμην της κεφαλής σας ουδέ θέλετε φθείρει τα άκρα των πωγώνων σας.
\par 28 Δεν θέλετε κάμει εντομίδας εις το σώμα σας διά νεκρόν, ουδέ γράμματα στικτά θέλετε εγχαράξει επάνω σας. Εγώ είμαι ο Κύριος.
\par 29 Δεν θέλεις βεβηλώσει την θυγατέρα σου, καθιστών αυτήν πόρνην· μήπως ο τόπος πέση εις πορνείαν και γεμίση ο τόπος από ασεβείας.
\par 30 Τα σάββατά μου θέλετε φυλάττει, και το αγιαστήριόν μου θέλετε σέβεσθαι. Εγώ είμαι ο Κύριος.
\par 31 Δεν θέλετε ακολουθεί τους έχοντας πνεύμα μαντείας ουδέ θέλετε προσκολληθή εις επαοιδούς, ώστε να μιαίνησθε δι' αυτών. Εγώ είμαι Κύριος ο Θεός σας.
\par 32 Ενώπιον της πολιάς θέλεις προσηκόνεσθαι και θέλεις τιμήσει το πρόσωπον του γέροντος και θέλεις φοβηθή τον Θεόν σου. Εγώ είμαι ο Κύριος.
\par 33 Και εάν τις ξένος παροική μετά σου εν τη γη υμών, δεν θέλετε θλίψει αυτόν·
\par 34 ο ξένος, ο παροικών με σας, θέλει είσθαι εις εσάς ως ο αυτόχθων, και θέλεις αγαπά αυτόν ως σεαυτόν· διότι ξένοι εστάθητε εν γη Αιγύπτου. Εγώ είμαι Κύριος ο Θεός σας.
\par 35 Δεν θέλετε πράξει αδικίαν εις κρίσιν, εις μέτρα, εις σταθμά και εις ζύγια·
\par 36 ζύγια δίκαια σταθμά δίκαια, εφά δίκαιον, και ιν δίκαιον, θέλετε έχει. Εγώ είμαι Κύριος ο Θεός σας, όστις σας εξήγαγον εκ γης Αιγύπτου.
\par 37 Θέλετε φυλάττει λοιπόν πάντα τα διατάγματά μου και πάσας τας κρίσεις μου και θέλετε κάμνει αυτά. Εγώ είμαι ο Κύριος.

\chapter{20}

\par Και ελάλησε Κύριος προς τον Μωϋσήν, λέγων,
\par 2 Και προς τους υιούς Ισραήλ θέλεις ειπεί, Όστις εκ των υιών Ισραήλ ή εκ των ξένων των παροικούντων εν τω Ισραήλ δώση από του σπέρματος αυτού εις τον Μολόχ, θέλει εξάπαντος θανατωθή· ο λαός του τόπου θέλει λιθοβολήσει αυτόν με λίθους.
\par 3 Και εγώ θέλω επιστήσει το πρόσωπόν μου κατά του ανθρώπου εκείνου και θέλω εξολοθρεύσει αυτόν εκ μέσου του λαού αυτού· διότι από του σπέρματος αυτού έδωκεν εις τον Μολόχ, διά να μιάνη το αγιαστήριόν μου και να βεβηλώση το όνομά μου το άγιον.
\par 4 Εάν δε ο λαός του τόπου παραβλέψη με τους οφθαλμούς αυτού τον άνθρωπον εκείνον, όταν δίδη από του σπέρματος αυτού εις τον Μολόχ, και δεν φονεύση αυτόν,
\par 5 τότε θέλω επιστήσει εγώ το πρόσωπόν μου κατά του ανθρώπου εκείνου και κατά της συγγενείας αυτού· και θέλω εξολοθρεύσει εκ μέσου του λαού αυτού αυτόν, και πάντας τους ακολουθούντας αυτόν εις την πορνείαν, διά να πορνεύωσι κατόπιν του Μολόχ.
\par 6 Και ψυχή, ήτις ακολουθήση τους έχοντας πνεύμα μαντείας και τους επαοιδούς, διά να πορνεύη κατόπιν αυτών, θέλω επιστήσει το πρόσωπόν μου κατά της ψυχής εκείνης, και θέλω εξολοθρεύσει αυτήν εκ μέσου του λαού αυτής.
\par 7 Αγιάσθητε λοιπόν και γίνεσθε άγιοι διότι εγώ είμαι Κύριος ο Θεός σας·
\par 8 Και θέλετε φυλάττει τα διατάγματά μου και θέλετε εκτελεί αυτά. Εγώ είμαι ο Κύριος, ο αγιάζων υμάς.
\par 9 Πας άνθρωπος, όστις κακολογήση τον πατέρα αυτού ή την μητέρα αυτού, εξάπαντος θέλει θανατωθή· τον πατέρα αυτού ή την μητέρα αυτού, εκακολόγησε· το αίμα αυτού θέλει είσθαι επ' αυτόν.
\par 10 Και άνθρωπος, όστις μοιχεύση την γυναίκα τινός, όστις μοιχεύση την γυναίκα του πλησίον αυτού, εξάπαντος θέλει θανατωθή, ο μοιχεύων και η μοιχευομένη.
\par 11 Και άνθρωπος, όστις κοιμηθή μετά της γυναικός του πατρός αυτού, την ασχημοσύνην του πατρός αυτού απεκάλυψεν· εξάπαντος θέλουσι θανατωθή αμφότεροι· το αίμα αυτών θέλει είσθαι επ' αυτούς.
\par 12 Και εάν τις κοιμηθή μετά της νύμφης αυτού, εξάπαντος θέλουσι θανατωθή αμφότεροι· σύγχυσιν έπραξαν· το αίμα αυτών θέλει είσθαι επ' αυτούς.
\par 13 Εάν δε τις κοιμηθή μετά άρρενος, καθώς κοιμάται μετά γυναικός, βδέλυγμα έπραξαν αμφότεροι· εξάπαντος θέλουσι θανατωθή· το αίμα αυτών θέλει είσθαι επ' αυτούς.
\par 14 Και εάν τις λάβη γυναίκα και την μητέρα αυτής, είναι ανομία· εν πυρί θέλουσι καυθή, αυτός και αυταί, και δεν θέλει είσθαι ανομία μεταξύ σας.
\par 15 Και εάν τις συνουσιασθή μετά κτήνους, εξάπαντος θέλει θανατωθή· και το κτήνος θέλετε φονεύσει.
\par 16 Και η γυνή, ήτις πλησιάση εις οιονδήποτε κτήνος διά να βατευθή, θέλεις φονεύσει την γυναίκα και το κτήνος· εξάπαντος θέλουσι θανατωθή· το αίμα αυτών θέλει είσθαι επ' αυτούς.
\par 17 Και εάν τις λάβη την αδελφήν αυτού, την θυγατέρα του πατρός αυτού ή την θυγατέρα της μητρός αυτού, και ίδη την ασχημοσύνην αυτής και αυτή ίδη την ασχημοσύνην εκείνου, είναι αισχρόν· και θέλουσιν εξολοθρευθή έμπροσθεν του λαού αυτών· την ασχημοσύνην της αδελφής αυτού απεκάλυψε· την ανομίαν αυτού θέλει βαστάσει.
\par 18 Και άνθρωπος, όστις κοιμηθή μετά γυναικός εχούσης τα γυναικεία αυτής και αποκαλύψη την ασχημοσύνην αυτής, ούτος την πηγήν αυτής εξεσκέπασε και αύτη την πηγήν του αίματος αυτής απεκάλυψεν· όθεν αμφότεροι θέλουσιν εξολοθρευθή εκ μέσου του λαού αυτών.
\par 19 Και την ασχημοσύνην της αδελφής της μητρός σου ή της αδελφής του πατρός σου δεν θέλεις αποκαλύψει· διότι την στενήν συγγενή αυτού αποκαλύπτει· την ανομίαν αυτών θέλουσι βαστάσει.
\par 20 Εάν δε τις κοιμηθή μετά της θείας αυτού, την ασχημοσύνην του θείου αυτού απεκάλυψε· την αμαρτίαν αυτών θέλουσι βαστάσει· άτεκνοι θέλουσιν αποθάνει.
\par 21 Και εάν τις λάβη την γυναίκα του αδελφού αυτού, είναι ακαθαρσία· την ασχημοσύνην του αδελφού αυτού απεκάλυψεν· άτεκνοι θέλουσι μείνει.
\par 22 Θέλετε λοιπόν φυλάττει πάντα τα διατάγματά μου και πάσας τας κρίσεις μου, και θέλετε κάμνει αυτά· διά να μη σας εξεμέση η γη, όπου εγώ σας φέρω διά να κατοικήσητε εν αυτή.
\par 23 Και δεν θέλετε περιπατεί κατά τα νόμιμα των εθνών, τα οποία εγώ εκδιώκω απ' έμπροσθέν σας· διότι πάντα ταύτα έπραξαν, όθεν εβδελύχθην αυτούς.
\par 24 Και είπα προς εσάς, Σεις θέλετε κληρονομήσει την γην αυτών, και εγώ θέλω δώσει αυτήν εις εσάς προς ιδιοκτησίαν, γην ρέουσαν γάλα και μέλι. Εγώ είμαι Κύριος ο Θεός σας, όστις σας απεχώρισα από των λαών.
\par 25 Διά τούτο θέλετε αποχωρίσει τα κτήνη τα καθαρά από των ακαθάρτων και τα πτηνά τα ακάθαρτα από των καθαρών· και δεν θέλετε μιάνει τας ψυχάς σας με τα κτήνη ή με τα πτηνά ή με παν ό,τι έρπει επί της γης, τα οποία εγώ απεχώρισα εις σας ως ακάθαρτα.
\par 26 Και θέλετε είσθαι άγιοι εις εμέ· διότι άγιος είμαι εγώ ο Κύριος και σας απεχώρισα από των λαών, διά να ήσθε εμού.
\par 27 Και ανήρ η γυνή, ήτις έχει πνεύμα μαντείας, ή είναι επαοιδός, εξάπαντος θέλει θανατωθή· με λίθους θέλουσι λιθοβολήσει αυτούς· το αίμα αυτών θέλει είσθαι επ' αυτούς.

\chapter{21}

\par Και είπε Κύριος προς τον Μωϋσήν, Λάλησον προς τους ιερείς τους υιούς του Ααρών, και ειπέ προς αυτούς, Ουδείς θέλει μιανθή μεταξύ του λαού αυτού διά νεκρόν·
\par 2 ει μη διά τον συγγενή αυτού τον πλησιέστερον, διά την μητέρα αυτού και διά τον πατέρα αυτού και διά τον υιόν αυτού και διά την θυγατέρα αυτού και διά τον αδελφόν αυτού,
\par 3 και διά την αδελφήν αυτού, παρθένον ούσαν, την πλησιεστάτην εις αυτόν, ήτις δεν έλαβεν άνδρα· διά ταύτην δύναται να μιανθή.
\par 4 Δεν θέλει μιανθή αρχηγός ων του λαού αυτού, ώστε να βεβηλώση εαυτόν.
\par 5 Δεν θέλουσι φαλακρώσει την κεφαλήν αυτών ουδέ θέλουσι ξυρίσει τα πλάγια των πωγώνων αυτών ουδέ θέλουσι κάμει εντομίδας επί τας σάρκας αυτών.
\par 6 Άγιοι θέλουσιν είσθαι εις τον Θεόν αυτών και δεν θέλουσι βεβηλώσει το όνομα του Θεού αυτών· διότι τας διά πυρός γινομένας προσφοράς του Κυρίου, τον άρτον του Θεού αυτών, προσφέρουσι διά τούτο θέλουσιν είσθαι άγιοι.
\par 7 Γυναίκα πόρνην και βεβηλωμένην δεν θέλουσι λάβει ουδέ γυναίκα αποβεβλημένην από του ανδρός αυτής θέλουσι λάβει· διότι ο ιερεύς είναι άγιος εις τον Θεόν αυτού.
\par 8 Θέλεις λοιπόν αγιάσει αυτόν· διότι αυτός τον άρτον του Θεού σου προσφέρει· άγιος θέλει είσθαι εις σέ· διότι άγιος είμαι εγώ ο Κύριος, ο αγιάζων υμάς.
\par 9 Και θυγάτηρ ιερέως τινός, εάν βεβηλωθή διά πορνείας, τον πατέρα αυτής αυτή βεβηλόνει· εν πυρί θέλει κατακαυθή.
\par 10 Και ο ιερεύς ο μέγας μεταξύ των αδελφών αυτού, επί την κεφαλήν του οποίου εχύθη το έλαιον του χρίσματος, και όστις καθιερώθη διά να ενδύηται τας ιεράς στολάς, την κεφαλήν αυτού δεν θέλει αποκαλύψει, ουδέ τα ιμάτια αυτού θέλει διασχίσει·
\par 11 και εις ουδέν σώμα νεκρόν θέλει εισέλθει ουδέ διά τον πατέρα αυτού ή διά την μητέρα αυτού θέλει μιανθή.
\par 12 Και εκ του αγιαστηρίου δεν θέλει εξέλθει ουδέ θέλει βεβηλώσει το αγιαστήριον του Θεού αυτού· διότι το άγιον έλαιον του χρίσματος του Θεού αυτού είναι επ' αυτόν. Εγώ είμαι ο Κύριος.
\par 13 Και ούτος θέλει λάβει γυναίκα παρθένον·
\par 14 χήραν ή αποβεβλημένην ή βέβηλον ή πόρνην, ταύτας δεν θέλει λάβει· αλλά παρθένον εκ του λαού αυτού θέλει λάβει εις γυναίκα.
\par 15 Και δεν θέλει βεβηλώσει το σπέρμα αυτού μεταξύ του λαού αυτού· διότι εγώ είμαι ο Κύριος, ο αγιάζων αυτόν.
\par 16 Και ελάλησε Κύριος προς τον Μωϋσήν, λέγων,
\par 17 Ειπέ προς τον Ααρών, λέγων, Όστις εκ του σπέρματός σου εις τας γενεάς αυτών έχει μώμον, ας μη πλησιάση διά να προσφέρη τον άρτον του Θεού αυτού·
\par 18 διότι πας όστις έχει μώμον δεν θέλει πλησιάσει· άνθρωπος τυφλός, ή χωλός, ή κολοβομύττης, ή έχων τι περιττόν,
\par 19 ή άνθρωπος όστις έχει σύντριμμα ποδός, ή σύντριμμα χειρός,
\par 20 ή είναι κυρτός, ή πολύ ισχνός, ή όστις έχει βεβλαμμένους τους οφθαλμούς, ή έχει ψώραν ξηράν, ή λειχήνα, ή είναι εσπασμένος·
\par 21 ουδείς άνθρωπος εκ του σπέρματος του Ααρών του ιερέως, όστις έχει μώμον, θέλει πλησιάσει διά να προσφέρη τας διά πυρός γινομένας προσφοράς εις τον Κύριον· μώμον έχει· δεν θέλει πλησιάσει διά να προσφέρη τον άρτον του Θεού αυτού.
\par 22 Θέλει τρώγει τον άρτον του Θεού αυτού εκ των αγιωτάτων, και εκ των αγίων.
\par 23 Πλην εις το καταπέτασμα δεν θέλει εισέρχεσθαι ουδέ εις το θυσιαστήριον θέλει πλησιάσει, διότι έχει μώμον· διά να μη βεβηλώση το αγιαστήριόν μου· διότι εγώ είμαι ο Κύριος, ο αγιάζων αυτούς.
\par 24 Και ελάλησεν ο Μωϋσής ταύτα προς τον Ααρών και προς τους υιούς αυτού και προς πάντας τους υιούς Ισραήλ.

\chapter{22}

\par Και ελάλησε Κύριος προς τον Μωϋσήν, λέγων,
\par 2 Ειπέ προς τον Ααρών και προς τους υιούς αυτού να απέχωσιν από των αγίων των υιών Ισραήλ, και να μη βεβηλόνωσι το όνομα το άγιόν μου, εις όσα αγιάζουσιν εις εμέ. Εγώ είμαι ο Κύριος.
\par 3 Ειπέ προς αυτούς, πας άνθρωπος εκ παντός του σπέρματός σας εις τας γενεάς σας, όστις πλησιάση εις τα άγια, τα οποία οι υιοί του Ισραήλ αγιάζουσιν εις τον Κύριον, έχων την ακαθαρσίαν αυτού εφ' εαυτόν, η ψυχή εκείνη θέλει εξολοθρευθή απ' έμπροσθέν μου. Εγώ είμαι ο Κύριος.
\par 4 Όστις εκ του σπέρματος του Ααρών είναι λεπρός ή έχει ρεύσιν, από των αγίων δεν θέλει τρώγει, εωσού καθαρισθή. Και όστις εγγίση παν ακάθαρτον από νεκρόν ή άνθρωπον εκ του οποίου έγεινε ρεύσις σπέρματος,
\par 5 ή όστις εγγίση οιονδήποτε ερπετόν, εκ του οποίου δύναται να μιανθή, ή άνθρωπον, εκ του οποίου δύναται να μιανθή, οποιαδήποτε είναι η ακαθαρσία αυτού·
\par 6 η ψυχή, ήτις εγγίση αυτά, θέλει είσθαι ακάθαρτος έως εσπέρας· και δεν θέλει φάγει από των αγίων· εάν μη λούση το σώμα αυτού εν ύδατι.
\par 7 Και αφού δύση ο ήλιος θέλει είσθαι καθαρός, και έπειτα θέλει φάγει από των αγίων· διότι είναι η τροφή αυτού.
\par 8 Θνησιμαίον ή θηριάλωτον δεν θέλει φάγει, ώστε να μιανθή εν αυτοίς. Εγώ είμαι ο Κύριος.
\par 9 Όθεν θέλουσι φυλάττει τα διατάγματά μου, διά να μη βαστάσωσιν αμαρτίαν εκ τούτου και αποθάνωσι δι' αυτό, εάν βεβηλώσωσιν αυτά. Εγώ είμαι ο Κύριος, ο αγιάζων αυτούς.
\par 10 Και ουδείς αλλογενής θέλει φάγει από των αγίων· συγκάτοικος του ιερέως ή μισθωτός δεν θέλει φάγει από των αγίων.
\par 11 Αλλ' εάν ο ιερεύς αγοράση ψυχήν διά του αργυρίου αυτού, ούτος θέλει τρώγει εξ αυτών, καθώς και ο γεννηθείς εν τη οικία αυτού· ούτοι θέλουσι τρώγει από του άρτου αυτού.
\par 12 Και η θυγάτηρ του ιερέως, αν ήναι νενυμφευμένη μετά ανδρός ξένου, αύτη δεν θέλει τρώγει από των αγίων των προσφορών.
\par 13 Αλλ' εάν η θυγάτηρ του ιερέως χηρεύση ή αποβληθή και δεν έχη τέκνον και επιστρέψη εις τον πατρικόν αυτής οίκον, καθώς ευρίσκετο εν τη νεότητι αυτής, θέλει τρώγει από του άρτου του πατρός αυτής· ουδείς όμως ξένος θέλει φάγει απ' αυτού.
\par 14 Εάν δε άνθρωπός τις φάγη από των αγίων εξ αγνοίας, τότε θέλει προσθέσει το πέμπτον τούτου εις αυτό, και θέλει αποδώσει εις τον ιερέα το άγιον.
\par 15 Και δεν θέλουσι βεβηλώσει τα άγια των υιών Ισραήλ, τα οποία προσφέρουσιν εις τον Κύριον,
\par 16 και δεν θέλουσιν αναλάβει εφ' εαυτούς ανομίαν παραβάσεως, τρώγοντες τα άγια αυτών· διότι εγώ είμαι ο Κύριος, ο αγιάζων αυτούς.
\par 17 Και ελάλησε Κύριος προς τον Μωϋσήν, λέγων,
\par 18 Λάλησον προς τον Ααρών και προς τους υιούς αυτού και προς πάντας τους υιούς Ισραήλ και ειπέ προς αυτούς, Πας άνθρωπος εκ του οίκου Ισραήλ ή εκ των ξένων των εν τω Ισραήλ, όστις προσφέρη το δώρον αυτού, κατά πάσας τας ευχάς αυτών ή κατά πάσας τας αυτοπροαιρέτους προσφοράς αυτών, τας οποίας προσφέρουσιν εις τον Κύριον διά ολοκαύτωμα,
\par 19 θέλετε προσφέρει, διά να ήσθε δεκτοί, αρσενικόν άμωμον εκ των βοών, εκ των προβάτων ή εκ των αιγών.
\par 20 παν ό,τι έχει μώμον, δεν θέλετε προσφέρει διότι δεν θέλει είσθαι δεκτόν διά σας.
\par 21 Και όστις προσφέρει θυσίαν ειρηνικής προσφοράς εις τον Κύριον διά να εκπληρώση ευχήν, ή προσφοράν αυτοπροαίρετον, εκ των βοών ή εκ των προβάτων, θέλει είσθαι άμωμον διά να ήναι δεκτόν· ουδείς μώμος θέλει είσθαι εις αυτό.
\par 22 Τυφλόν, ή συντετριμμένον, ή κολοβόν, ή έχον εξόγκωμα, ή ψώραν ξηράν, ή λειχήνας, ταύτα δεν θέλετε προσφέρει εις τον Κύριον, ουδέ θέλετε κάμει εξ αυτών προσφοράν διά πυρός εις τον Κύριον επί του θυσιαστηρίου.
\par 23 Μόσχον δε ή πρόβατον, το οποίον έχει τι περιττόν ή κολοβόν, δύνασαι να προσφέρης αυτό διά προσφοράν αυτοπροαίρετον· δι' ευχήν όμως δεν θέλει είσθαι δεκτόν.
\par 24 Θλαδίαν, ή εκτεθλιμμένον, ή εκτομίαν, ή ευνουχισμένον, δεν θέλετε προσφέρει εις τον Κύριον· ουδέ θέλετε κάμει τούτο εν τη γη υμών.
\par 25 Ουδέ εκ χειρός αλλογενούς θέλετε προσφέρει τον άρτον του Θεού σας εκ πάντων τούτων· διότι η διαφθορά αυτών είναι εν αυτοίς· μώμος είναι εν αυτοίς· δεν θέλουσιν είσθαι δεκτά διά σας.
\par 26 Και ελάλησε Κύριος προς τον Μωϋσήν, λέγων,
\par 27 Όταν μόσχος ή αρνίον ή ερίφιον γεννηθή, τότε θέλει είσθαι επτά ημέρας υποκάτω της μητρός αυτού· από δε της ογδόης ημέρας και επέκεινα θέλει είσθαι δεκτόν εις θυσίαν διά πυρός γινομένην εις τον Κύριον.
\par 28 Και δάμαλιν ή πρόβατον δεν θέλετε σφάξει αυτό και το παιδίον αυτού εν μιά ημέρα.
\par 29 Και όταν προσφέρητε θυσίαν ευχαριστίας εις τον Κύριον, θέλετε προσφέρει αυτήν αυτοπροαιρέτως.
\par 30 Την αυτήν ημέραν θέλει φαγωθή· δεν θέλετε αφήσει ουδέν εξ αυτής έως το πρωΐ. Εγώ είμαι ο Κύριος.
\par 31 Θέλετε λοιπόν φυλάττει τας εντολάς μου και θέλετε εκτελεί αυτάς. Εγώ είμαι ο Κύριος.
\par 32 Και δεν θέλετε βεβηλόνει το όνομά μου το άγιον· αλλά θέλω αγιάζεσθαι μεταξύ των υιών Ισραήλ. Εγώ είμαι ο Κύριος, ο αγιάζων υμάς·
\par 33 όστις εξήγαγον υμάς εκ γης Αιγύπτου, διά να ήμαι Θεός υμών. Εγώ είμαι ο Κύριος.

\chapter{23}

\par Και ελάλησε Κύριος προς τον Μωϋσήν, λέγων,
\par 2 Λάλησον προς τους υιούς Ισραήλ και ειπέ προς αυτούς, Αι εορταί του Κυρίου, τας οποίας θέλετε διακηρύξει συγκαλέσεις αγίας, αύται είναι αι εορταί μου.
\par 3 Εξ ημέρας θέλεις κάμνει εργασίαν, την δε εβδόμην ημέραν είναι σάββατον αναπαύσεως, συγκάλεσις αγία· ουδεμίαν εργασίαν θέλετε κάμει· είναι σάββατον του Κυρίου εις πάσας τας κατοικίας σας.
\par 4 Αύται είναι αι εορταί του Κυρίου, συγκαλέσεις άγιαι, τας οποίας θέλετε διακηρύξει εν τοις καιροίς αυτών.
\par 5 Τον πρώτον μήνα, την δεκάτην τετάρτην του μηνός, εις το δειλινόν, είναι πάσχα του Κυρίου.
\par 6 Και την δεκάτην πέμπτην ημέραν του αυτού μηνός, εορτή των αζύμων εις τον Κύριον· επτά ημέρας άζυμα θέλετε τρώγει.
\par 7 Εν τη πρώτη ημέρα θέλει είσθαι εις εσάς συγκάλεσις αγία· ουδέν έργον δουλευτικόν θέλετε κάμει.
\par 8 Και θέλετε προσφέρει προσφοράν γινομένην διά πυρός εις τον Κύριον επτά ημέρας· εν τη εβδόμη ημέρα είναι συγκάλεσις αγία· ουδέν έργον δουλευτικόν θέλετε κάμει.
\par 9 Και ελάλησε Κύριος προς τον Μωϋσήν, λέγων,
\par 10 Λάλησον προς τους υιούς Ισραήλ και ειπέ προς αυτούς, Όταν εισέλθητε εις την γην, την οποίαν εγώ δίδω εις εσάς, και θερίσητε τον θερισμόν αυτής, τότε θέλετε φέρει εν δράγμα εκ των απαρχών του θερισμού σας προς τον ιερέα·
\par 11 και θέλει κινήσει το δράγμα ενώπιον του Κυρίου, διά να γείνη δεκτόν διά σάς· την επαύριον του σαββάτου θέλει κινήσει αυτό ο ιερεύς.
\par 12 Και την ημέραν εκείνην, καθ' ην κινήσητε το δράγμα, θέλετε προσφέρει αρνίον άμωμον ενιαύσιον διά ολοκαύτωμα προς τον Κύριον·
\par 13 και την εξ αλφίτων προσφοράν αυτού, δύο δέκατα σεμιδάλεως εζυμωμένης μετά ελαίου, εις προσφοράν γινομένην διά πυρός προς τον Κύριον, εις οσμήν ευωδίας· και την σπονδήν αυτού, το τέταρτον του ιν οίνου.
\par 14 Και άρτον ή σίτον εψημένον ή αστάχυα δεν θέλετε φάγει, μέχρι της αυτής ταύτης ημέρας καθ' ην προσφέρητε το δώρον του Θεού σας· θέλει είσθαι νόμιμον αιώνιον εις τας γενεάς σας κατά πάσας τας κατοικίας σας.
\par 15 Και θέλετε αριθμήσει εις εαυτούς από της επαύριον του σαββάτου, αφ' ης ημέρας προσφέρητε το δράγμα της κινητής προσφοράς, επτά ολοκλήρους εβδομάδας·
\par 16 μέχρι της επαύριον του εβδόμου σαββάτου θέλετε αριθμήσει πεντήκοντα ημέρας και θέλετε προσφέρει νέαν προσφοράν εξ αλφίτων προς τον Κύριον.
\par 17 Από των κατοικιών σας θέλετε φέρει εις προσφοράν κινητήν δύο άρτους· δύο δέκατα σεμιδάλεως θέλουσιν είσθαι· ένζυμα θέλουσιν εψηθή· πρωτογεννήματα είναι εις τον Κύριον.
\par 18 Και θέλετε προσφέρει μετά του άρτου επτά αρνία άμωμα ενιαύσια και ένα μόσχον εκ βοών και δύο κριούς· ολοκαύτωμα θέλουσιν είσθαι εις τον Κύριον μετά της εξ αλφίτων προσφοράς αυτών και μετά των σπονδών αυτών, προσφορά γινομένη διά πυρός εις οσμήν ευωδίας προς τον Κύριον.
\par 19 Και θέλετε προσφέρει ένα τράγον εξ αιγών εις προσφοράν περί αμαρτίας και δύο αρνία, ενιαύσια εις θυσίαν ειρηνικής προσφοράς.
\par 20 Και θέλει κινήσει αυτά ο ιερεύς μετά του άρτου των πρωτογεννημάτων εις προσφοράν κινητήν ενώπιον του Κυρίου, μετά των δύο αρνίων· άγια θέλουσιν είσθαι εις τον Κύριον διά τον ιερέα.
\par 21 Και θέλετε διακηρύξει την αυτήν εκείνην ημέραν, συγκάλεσιν αγίαν διά σάς· ουδέν έργον δουλευτικόν θέλετε κάμει θέλει είσθαι νόμιμον αιώνιον κατά πάσας τας κατοικίας σας εις τας γενεάς σας.
\par 22 Και όταν θερίζητε τον θερισμόν της γης σας, δεν θέλεις θερίσει ολοκλήρως τας άκρας του αγρού σου και τα πίπτοντα του θερισμού σου δεν θέλεις συλλέξει· εις τον πτωχόν και εις τον ξένον θέλεις αφήσει αυτά. Εγώ είμαι Κύριος ο Θεός σας.
\par 23 Και ελάλησε Κύριος προς τον Μωϋσήν, λέγων,
\par 24 Λάλησον προς τους υιούς Ισραήλ, λέγων, Τον έβδομον μήνα, την πρώτην του μηνός, θέλει είσθαι εις εσάς σάββατον, μνημόσυνον μετά αλαλαγμού σαλπίγγων, συγκάλεσις αγία.
\par 25 Ουδέν έργον δουλευτικόν θέλετε κάμει και θέλετε προσφέρει προσφοράν γινομένην διά πυρός προς τον Κύριον.
\par 26 Και ελάλησε Κύριος προς τον Μωϋσήν, λέγων,
\par 27 Και την δεκάτην του εβδόμου τούτου μηνός θέλει είσθαι ημέρα εξιλασμού· συγκάλεσις αγία θέλει είσθαι εις εσάς· και θέλετε ταπεινώσει τας ψυχάς σας και θέλετε προσφέρει προσφοράν γινομένην διά πυρός προς τον Κύριον.
\par 28 Και ουδεμίαν εργασίαν θέλετε κάμει εις αυτήν ταύτην την ημέραν· διότι είναι ημέρα εξιλασμού, διά να γείνη εξιλέωσις διά σας ενώπιον Κυρίου του Θεού σας.
\par 29 Επειδή πάσα ψυχή, ήτις δεν ταπεινωθή εις αυτήν ταύτην την ημέραν, θέλει εξολοθρευθή εκ του λαού αυτής.
\par 30 Και πάσα ψυχή, ήτις κάμη οποιανδήποτε εργασίαν εις αυτήν ταύτην την ημέραν, θέλω εξολοθρεύσει την ψυχήν εκείνην εκ μέσου του λαού αυτής.
\par 31 Ουδεμίαν εργασίαν θέλετε κάμει· θέλει είσθαι νόμιμον αιώνιον εις τας γενεάς σας, κατά πάσας τας κατοικίας σας.
\par 32 Σάββατον αναπαύσεως θέλει είσθαι διά σας, και θέλετε ταπεινώσει τας ψυχάς σας την εννάτην του μηνός το εσπέρας· από εσπέρας έως εσπέρας, θέλετε εορτάσει το σάββατόν σας.
\par 33 Και ελάλησε Κύριος προς τον Μωϋσήν, λέγων,
\par 34 Λάλησον προς τους υιούς Ισραήλ, λέγων, την δεκάτην πέμπτην ημέραν του εβδόμου τούτου μηνός θέλει είσθαι η εορτή των σκηνών επτά ημέρας εις τον Κύριον.
\par 35 Την πρώτην ημέραν θέλει είσθαι συγκάλεσις αγία· ουδέν έργον δουλευτικόν θέλετε κάμει.
\par 36 Επτά ημέρας θέλετε προσφέρει προσφοράν γινομένην διά πυρός προς τον Κύριον· την ογδόην ημέραν θέλει είσθαι εις εσάς συγκάλεσις αγία, και θέλετε προσφέρει προσφοράν γινομένην διά πυρός προς τον Κύριον· είναι σύναξις επίσημος· ουδέν έργον δουλευτικόν θέλετε κάμει.
\par 37 Αύται είναι αι εορταί του Κυρίου, τας οποίας θέλετε διακηρύξει συγκαλέσεις αγίας, διά να προσφέρητε προσφοράν γινομένην διά πυρός προς τον Κύριον, ολοκαύτωμα και προσφοράν εξ αλφίτων, θυσίαν και σπονδάς, το δι' εκάστην διωρισμένον εις την ημέραν αυτού·
\par 38 εκτός των σαββάτων του Κυρίου και εκτός των δώρων σας και εκτός πασών των ευχών σας και εκτός πασών των αυτοπροαιρέτων προσφορών σας, τας οποίας δίδετε εις τον Κύριον.
\par 39 Και την δεκάτην πέμπτην ημέραν του εβδόμου μηνός, αφού συνάξητε τα γεννήματα της γης, θέλετε εορτάσει την εορτήν του Κυρίου επτά ημέρας· την πρώτην ημέραν θέλει είσθαι ανάπαυσις και την ογδόην ημέραν ανάπαυσις.
\par 40 Και την πρώτην ημέραν θέλετε λάβει εις εαυτούς καρπόν δένδρου ώραίου, κλάδους φοινίκων και κλάδους δένδρων δασέων και ιτέας από χειμάρρου· και θέλετε ευφρανθή ενώπιον Κυρίου του Θεού σας επτά ημέρας.
\par 41 Και θέλετε εορτάσει αυτήν εορτήν εις τον Κύριον επτά ημέρας του ενιαυτού· νόμιμον αιώνιον θέλει είσθαι εις τας γενεάς σας· τον έβδομον μήνα θέλετε εορτάζει αυτήν.
\par 42 Εν σκηναίς θέλετε κατοικεί επτά ημέρας· πάντες οι αυτόχθονες Ισραηλίται θέλουσι κατοικεί εν σκηναίς·
\par 43 διά να γνωρίσωσιν αι γενεαί σας ότι εν σκηναίς κατώκισα τους υιούς Ισραήλ, ότε εξήγαγον αυτούς εκ γης Αιγύπτου· εγώ Κύριος ο Θεός σας.
\par 44 Και εφανέρωσεν ο Μωϋσής τας εορτάς του Κυρίου προς τους υιούς Ισραήλ.

\chapter{24}

\par Και ελάλησε Κύριος προς τον Μωϋσήν, λέγων,
\par 2 Πρόσταξον τους υιούς Ισραήλ να φέρωσι προς σε έλαιον καθαρόν από ελαίας κοπανισμένας διά το φως, διά να καίη ο λύχνος διαπαντός.
\par 3 Έξωθεν του καταπετάσματος του μαρτυρίου, εν τη σκηνή του μαρτυρίου, θέλει βάλει αυτόν ο Ααρών από εσπέρας έως το πρωΐ ενώπιον του Κυρίου διαπαντός· νόμιμον αιώνιον θέλει είσθαι εις τας γενεάς σας.
\par 4 Επί την λυχνίαν την καθαράν θέλει διαθέσει τους λύχνους ενώπιον του Κυρίου πάντοτε.
\par 5 Και θέλεις λάβει σεμίδαλιν και θέλεις εψήσει απ' αυτής δώδεκα άρτους· δύο δέκατα θέλει είσθαι έκαστος άρτος.
\par 6 Και θέλεις βάλει αυτούς εις δύο σειράς, εξ κατά την σειράν, επί την τράπεζαν την καθαράν ενώπιον του Κυρίου.
\par 7 Και θέλεις βάλει εφ' εκάστην σειράν λιβάνιον καθαρόν, και θέλει είσθαι επί τον άρτον προς μνημόσυνον, εις προσφοράν γινομένην διά πυρός προς τον Κύριον.
\par 8 Πάσαν ημέραν σαββάτου θέλει διαθέσει ταύτα διαπαντός ενώπιον του Κυρίου, παρά των υιών Ισραήλ εις διαθήκην αιώνιον.
\par 9 Και θέλουσιν είσθαι του Ααρών και των υιών αυτού· και θέλουσι τρώγει αυτά εν τόπω αγίω διότι είναι αγιώτατα εις αυτόν εκ των διά πυρός γινομένων προσφορών του Κυρίου εις νόμιμον αιώνιον.
\par 10 Και εξήλθεν υιός γυναικός τινός Ισραηλίτιδος, όστις ήτο υιός ανδρός Αιγυπτίου, μεταξύ των υιών Ισραήλ· και εμάχοντο εν τω στρατοπέδω ο υιός της Ισραηλίτιδος και άνθρωπός τις Ισραηλίτης.
\par 11 Και εβλασφήμησεν ο υιός της γυναικός της Ισραηλίτιδος το όνομα του Κυρίου και κατηράσθη· και έφεραν αυτόν προς τον Μωϋσήν. Και το όνομα της μητρός αυτού ήτο Σελωμείθ, θυγάτηρ του Διβρεί, εκ της φυλής Δαν.
\par 12 Και έβαλον αυτόν εις φυλακήν, εωσού φανερωθή εις αυτούς η θέλησις του Κυρίου.
\par 13 Και ελάλησε Κύριος προς τον Μωϋσήν, λέγων,
\par 14 Φέρε έξω του στρατοπέδου εκείνον όστις κατηράσθη· και ας θέσωσι πάντες οι ακούσαντες αυτόν τας χείρας αυτών επί την κεφαλήν αυτού, και ας λιθοβολήση αυτόν πάσα η συναγωγή.
\par 15 Και λάλησον προς τους υιούς Ισραήλ, λέγων, Όστις καταρασθή τον Θεόν αυτού, Θέλει βαστάσει την ανομίαν αυτού·
\par 16 και όστις βλασφημήση το όνομα του Κυρίου, εξάπαντος θέλει θανατωθή· με λίθους θέλει λιθοβολήσει αυτόν πάσα η συναγωγή· άντε ξένος, άντε αυτόχθων, όταν βλασφημήση το όνομα του Κυρίου, θέλει θανατωθή.
\par 17 Και όστις φονεύση άνθρωπον, εξάπαντος θέλει θανατωθή.
\par 18 Και όστις θανατώση κτήνος, θέλει ανταποδώσει ζώον αντί ζώου.
\par 19 Και εάν τις κάμη βλάβην εις τον πλησίον αυτού, καθώς έκαμεν, ούτω θέλει γείνει εις αυτόν·
\par 20 σύντριμμα αντί συντρίμματος, οφθαλμόν αντί οφθαλμού, οδόντα αντί οδόντος· καθώς έκαμε βλάβην εις τον άνθρωπον, ούτω θέλει γείνει εις αυτόν.
\par 21 Και όστις θανατώση κτήνος, θέλει ανταποδώσει αυτό· και όστις φονεύσει άνθρωπον, θέλει θανατωθή.
\par 22 Κρίσις μία θέλει είσθαι εις εσάς· ως εις τον ξένον, ούτω θέλει γίνεσθαι και εις τον αυτόχθονα· διότι εγώ είμαι Κύριος ο Θεός σας.
\par 23 Και είπεν ο Μωϋσής προς τους υιούς Ισραήλ, και έφεραν έξω του στρατοπέδου εκείνον όστις κατηράσθη και ελιθοβόλησαν αυτόν με λίθους· και οι υιοί Ισραήλ έκαμον καθώς προσέταξεν ο Κύριος εις τον Μωϋσήν.

\chapter{25}

\par Και ελάλησε Κύριος προς τον Μωϋσήν εν τω όρει Σινά, λέγων,
\par 2 Λάλησον προς τους υιούς Ισραήλ και ειπέ προς αυτούς, Όταν εισέλθητε εις την γην, την οποίαν εγώ δίδω εις εσάς, τότε η γη θέλει φυλάξει σάββατον εις τον Κύριον.
\par 3 Εξ έτη θέλεις σπείρει τον αγρόν σου και εξ έτη θέλεις κλαδεύει την άμπελόν σου και θέλεις συνάγει τον καρπόν αυτής·
\par 4 το δε έβδομον έτος θέλει είσθαι σάββατον αναπαύσεως εις την γην, σάββατον διά τον Κύριον· τον αγρόν σου δεν θέλεις σπείρει και την άμπελόν σου δεν θέλεις κλαδεύσει.
\par 5 Δεν θέλεις θερίσει τον βλαστάνοντα αφ' εαυτού θερισμόν σου και τα σταφύλια της ακλαδεύτου αμπέλου σου δεν θέλεις τρυγήσει ενιαυτός αναπαύσεως θέλει είσθαι εις την γήν·
\par 6 και το σάββατον της γης θέλει είσθαι τροφή εις εσάς· εις σε, και εις τον δούλον σου, και εις την δούλην σου, και εις τον μισθωτόν σου, και εις τον ξένον τον παροικούντα μετά σου.
\par 7 Και εις τα κτήνη σου, και εις τα ζώα τα εν τη γη σου, θέλει είσθαι όλον το προϊόν αυτού εις τροφήν.
\par 8 Και θέλεις αριθμήσει εις σεαυτόν επτά εβδομάδας ετών, επτάκις επτά έτη· και αι ημέραι των επτά εβδομάδων των ετών θέλουσιν είσθαι εις σε τεσσαράκοντα εννέα έτη.
\par 9 Τότε θέλεις κάμει να ηχήση ο αλαλαγμός της σάλπιγγος την δεκάτην του εβδόμου μηνός· την ημέραν του εξιλασμού θέλετε κάμει να ηχήση η σάλπιγξ καθ' όλην την γην σας.
\par 10 Και θέλετε αγιάσει το πεντηκοστόν έτος και θέλετε διακηρύξει άφεσιν εις την γην προς πάντας τους κατοίκους αυτής· ούτος θέλει είσθαι ενιαυτός αφέσεως εις εσάς· και θέλετε επιστρέψει έκαστος εις το κτήμα αυτού και θέλετε επιστρέψει έκαστος εις την οικογένειαν αυτού.
\par 11 Ενιαυτός αφέσεως θέλει είσθαι εις εσάς το πεντηκοστόν έτος· δεν θέλετε σπείρει ουδέ θέλετε θερίσει το βλαστάνον αφ' εαυτού εν αυτώ και δεν θέλετε τρυγήσει την ακλάδευτον άμπελον αυτού·
\par 12 διότι ενιαυτός αφέσεως είναι άγιος θέλει είσθαι εις εσάς· από της πεδιάδος θέλετε τρώγει το προϊόν αυτής.
\par 13 Εις το έτος τούτο της αφέσεως θέλετε επιστρέψει έκαστος εις το κτήμα αυτού.
\par 14 Και εάν πωλήσης τι εις τον πλησίον σου ή αγοράσης παρά του πλησίον σου, ουδείς εξ υμών θέλει δυναστεύσει τον αδελφόν αυτού.
\par 15 Κατά τον αριθμόν των ετών μετά την άφεσιν θέλεις αγοράσει παρά του πλησίον σου, και κατά τον αριθμόν των ετών των γεννημάτων θέλει πωλήσει εις σε.
\par 16 Κατά το πλήθος των ετών θέλεις αυξήσει την τιμήν αυτού και κατά την ολιγότητα των ετών θέλεις ελαττώσει την τιμήν αυτού· διότι κατά τον αριθμόν των ετών των γεννημάτων θέλει πωλήσει εις σε.
\par 17 Και δεν θέλετε δυναστεύσει έκαστος τον πλησίον αυτού αλλά θέλεις φοβηθή τον Θεόν σου· διότι εγώ είμαι Κύριος ο Θεός σας.
\par 18 Και θέλετε κάμνει τα προστάγματά μου και τας κρίσεις μου θέλετε φυλάττει και θέλετε εκτελεί αυτά· και θέλετε κατοικεί ασφαλώς επί της γης.
\par 19 Και η γη θέλει δίδει τους καρπούς αυτής και θέλετε τρώγει εις χορτασμόν, και θέλετε κατοικεί ασφαλώς επ' αυτής.
\par 20 Εάν δε είπητε, Τι θέλομεν φάγει το έβδομον έτος, αν ημείς δεν σπείρωμεν μήτε συνάξωμεν τα γεννήματα ημών;
\par 21 τότε θέλω προστάξει την ευλογίαν μου να έλθη εφ' υμάς το έκτον έτος, και θέλει κάμει τα γεννήματα αυτής διά τρία έτη.
\par 22 Και θέλετε σπείρει το όγδοον έτος, και θέλετε τρώγει από των παλαιών γεννημάτων μέχρι του εννάτου έτους· εωσού έλθωσι τα γεννήματα αυτής θέλετε τρώγει παλαιά.
\par 23 Και η γη δεν θέλει πωλείσθαι εις απαλλοτρίωσιν· διότι ιδική μου είναι η γή· διότι σεις είσθε ξένοι και πάροικοι έμπροσθέν μου.
\par 24 Διά τούτο καθ' όλην την γην της ιδιοκτησίας σας θέλετε συγχωρεί εξαγόρασιν της γης.
\par 25 Εάν ο αδελφός σου πτωχεύση και πωλήση εκ των κτημάτων αυτού και έλθη ο πλησιέστερος αυτού συγγενής διά να εξαγοράση αυτά, τότε θέλει εξαγοράσει ό,τι επώλησεν ο αδελφός αυτού.
\par 26 Εάν δε ο άνθρωπος δεν έχη συγγενή διά να εξαγοράση αυτά, και ευπόρησε και εύρηκεν ικανά διά να εξαγοράση αυτά,
\par 27 τότε ας αριθμήση τα έτη της πωλήσεως αυτού και ας αποδώση το περιπλέον εις τον άνθρωπον, εις τον οποίον επώλησεν αυτά, και ας επιστρέψη εις τα κτήματα αυτού.
\par 28 Αλλ' εάν δεν ήναι ικανός ώστε να δώση την τιμήν εις αυτόν, τότε το πωληθέν θέλει μένει εν τη χειρί του αγοράσαντος αυτό μέχρι του έτους της αφέσεως· και θέλει απελευθερωθή εν τη αφέσει και θέλει επιστρέψει εις τα κτήματα αυτού.
\par 29 Και εάν τις πωλήση οίκον οικήσιμον εν πόλει περιτετειχισμένη, τότε δύναται να εξαγοράση αυτόν εντός ενός έτους από της πωλήσεως αυτού· εντός ενός ολοκλήρου έτους δύναται να εξαγοράση αυτόν.
\par 30 Αλλ' εάν δεν εξαγορασθή εωσού συμπληρωθή εις αυτόν ολόκληρον το έτος, τότε ο οίκος ο εν τη περιτετειχισμένη πόλει θέλει κυρωθή διαπαντός εις τον αγοράσαντα, εις τας γενεάς αυτού· δεν θέλει απελευθερωθή εν τη αφέσει.
\par 31 Αι οικίαι όμως των χωρίων, τα οποία δεν είναι περιτετειχισμένα, θέλουσι λογίζεσθαι ως οι αγροί της γής· δύνανται να εξαγοράζωνται και θέλουσιν απελευθερούσθαι εν τη αφέσει.
\par 32 Περί δε των πόλεων των Λευϊτών, αι οικίαι των πόλεων της ιδιοκτησίας αυτών δύνανται να εξαγορασθώσιν υπό των Λευϊτών εν παντί καιρώ.
\par 33 Και εάν τις αγοράση παρά τινός των Λευϊτών, τότε η εν τη πόλει της ιδιοκτησίας αυτού πωληθείσα οικία θέλει απελευθερωθή εν τη αφέσει· διότι αι οικίαι των πόλεων των Λευϊτών είναι η ιδιοκτησία αυτών μεταξύ των υιών Ισραήλ.
\par 34 Αλλ' ο αγρός των προαστείων των πόλεων αυτών δεν θέλει πωλείσθαι· διότι είναι παντοτεινή ιδιοκτησία αυτών.
\par 35 Και εάν πτωχεύση ο αδελφός σου και δυστυχήση, τότε θέλεις βοηθήσει αυτόν ως ξένον ή πάροικον, διά να ζήση μετά σου.
\par 36 Μη λάβης παρ' αυτού τόκον ή πλεονασμόν· αλλά φοβού τον Θεόν σου· διά να ζη ο αδελφός σου μετά σου.
\par 37 Το αργύριόν σου δεν θέλεις δώσει εις αυτόν επί τόκω, και επί πλεονασμώ, δεν θέλεις δώσει τας τροφάς σου.
\par 38 Εγώ είμαι Κύριος ο Θεός σας, όστις εξήγαγον σας εκ γης Αιγύπτου, διά να δώσω εις εσάς την γην Χαναάν, ώστε να ήμαι Θεός σας.
\par 39 Και εάν πτωχεύση ο αδελφός σου πλησίον σου και πωληθή εις σε, δεν θέλεις επιβάλει εις αυτόν δουλείαν δούλου.
\par 40 Ως μισθωτός ή πάροικος θέλει είσθαι πλησίον σου· μέχρι του έτους της αφέσεως θέλει δουλεύει εις σε.
\par 41 Τότε θέλει εξέλθει από σου αυτός και τα τέκνα αυτού μετ' αυτού και θέλει επιστρέψει εις την συγγένειαν αυτού και εις την ιδιοκτησίαν την πατρικήν αυτού θέλει επιστρέψει.
\par 42 Διότι δούλοί μου είναι ούτοι, τους το οποίους εξήγαγον εκ γης Αιγύπτου· δεν θέλουσι πωλείσθαι, καθώς πωλείται δούλος.
\par 43 Δεν θέλεις δεσπόζει επ' αυτόν μετά αυστηρότητος αλλά θέλεις φοβηθή τον Θεόν σου.
\par 44 Ο δε δούλός σου και η δούλη σου, όσους αν έχης, από των εθνών των πέριξ υμών, εκ τούτων θέλετε αγοράζει δούλον και δούλην.
\par 45 Και εκ των υιών έτι των ξένων των παροικούντων μεταξύ σας, εκ τούτων θέλετε αγοράζει και εκ των συγγενειών αυτών αίτινες είναι μεταξύ σας, όσοι εγεννήθησαν εν τη γη υμών· και θέλουσιν είσθαι εις εσάς εις ιδιοκτησίαν.
\par 46 Και θέλετε έχει αυτούς κληρονομίαν διά τα τέκνα σας ύστερον από σας, διά να κληρονομήσωσιν αυτούς ως ιδιοκτησίαν· δούλοί σας θέλουσιν είσθαι διαπαντός· πλην επί τους αδελφούς σας, τους υιούς Ισραήλ, δεν θέλετε εξουσιάζει ο εις επί τον άλλον μετά αυστηρότητος.
\par 47 Και όταν ο ξένος και ο παροικών μετά σου πλουτήση, ο δε αδελφός σου ο μετ' αυτού πτωχεύση και πωληθή εις τον ξένον, τον παροικούντα μετά σου, ή εις την γενεάν της συγγενείας του ξένου·
\par 48 αφού πωληθή, θέλει εξαγορασθή πάλιν· εις εκ των αδελφών αυτού θέλει εξαγοράσει αυτόν·
\par 49 ή ο θείος αυτού ή ο υιός του θείου αυτού θέλει εξαγοράσει αυτόν, ή εξ αίματος αυτού συγγενής εκ της συγγενείας αυτού θέλει εξαγοράσει αυτόν· ή εάν αυτός ευπόρησε, θέλει εξαγοράσει αυτός εαυτόν.
\par 50 Και θέλει λογαριάσει μετά του αγοραστού αυτού από του έτους, καθ' ο επωλήθη εις αυτόν, μέχρι του έτους της αφέσεως· και η τιμή της πωλήσεως αυτού θέλει είσθαι κατά τον αριθμόν των ετών· αναλόγως του χρόνου ενός μισθωτού θέλει λογαριασθή εις αυτόν.
\par 51 Εάν δε μένωσι πολλά έτη, αναλόγως τούτων θέλει αποδώσει την τιμήν της εξαγοράς αυτού εκ του αργυρίου δι' ου ηγοράσθη.
\par 52 Και εάν υπολείπωνται ολίγα έτη μέχρι του έτους της αφέσεως, θέλει κάμει λογαριασμόν μετ' αυτού και κατά τα έτη αυτού θέλει αποδώσει την τιμήν της εξαγοράς αυτού.
\par 53 Ως ετήσιος μισθωτός θέλει είσθαι μετ' αυτού· δεν θέλει δεσπόζει επ' αυτόν μετά αυστηρότητος ενώπιόν σου.
\par 54 Και εάν δεν εξαγορασθή κατά τα έτη ταύτα, τότε θέλει απελευθερωθή εις το έτος της αφέσεως, αυτός και τα τέκνα αυτού μετ' αυτού.
\par 55 Διότι εις εμέ οι υιοί του Ισραήλ είναι δούλοι· δούλοί μου είναι, τους οποίους εξήγαγον εκ γης Αιγύπτου. Εγώ είμαι Κύριος ο Θεός σας.

\chapter{26}

\par Δεν θέλετε κάμει εις εαυτούς είδωλα ουδέ γλυπτά, ουδέ θέλετε ανεγείρει άγαλμα εις εαυτούς, ουδέ θέλετε στήσει λίθον εικονόγλυπτον εν τη γη υμών, διά να προσκυνήτε αυτόν· διότι εγώ είμαι Κύριος ο Θεός σας.
\par 2 Τα σάββατά μου θέλετε φυλάττει και το αγιαστήριόν μου θέλετε σέβεσθαι. Εγώ είμαι ο Κύριος.
\par 3 Εάν περιπατήτε εις τα προστάγματά μου και φυλάττητε τας εντολάς μου και εκτελήτε αυτάς,
\par 4 τότε θέλω δώσει τας βροχάς σας εις τους καιρούς αυτών, και η γη θέλει δώσει τα γεννήματα αυτής, και τα δένδρα του αγρού θέλουσι δώσει τον καρπόν αυτών.
\par 5 Και το αλώνισμά σας θέλει σας φθάσει μέχρι του τρυγητού, και ο τρυγητός θέλει φθάσει μέχρι του σπορητού· και θέλετε τρώγει τον άρτον σας εις χορτασμόν· και θέλετε κατοικεί ασφαλώς εν τη γη υμών.
\par 6 Και θέλω δώσει ειρήνην εις την γην, και θέλετε πλαγιάζει και ουδείς θέλει σας φοβίζει και θέλω εξολοθρεύσει τα πονηρά θηρία από της γης και μάχαιρα δεν θέλει περάσει διά μέσου της γης σας.
\par 7 Και θέλετε διώξει τους εχθρούς σας και θέλουσι πέσει έμπροσθέν σας εν μαχαίρα·
\par 8 και πέντε από σας θέλουσι διώξει εκατόν, και εκατόν από σας θέλουσι διώξει μυρίους· και οι εχθροί σας θέλουσι πέσει έμπροσθέν σας εν μαχαίρα.
\par 9 Και θέλω επιβλέψει εις εσάς και θέλω σας αυξήσει και θέλω σας πληθύνει και θέλω στερεώσει την διαθήκην μου με σας.
\par 10 Και θέλετε φάγει παλαιά παλαιών, και θέλετε εκβάλει τα παλαιά απ' έμπροσθεν των νέων.
\par 11 Και θέλω στήσει την σκηνήν μου μεταξύ σας· και η ψυχή μου δεν θέλει σας βδελυχθή·
\par 12 και θέλω περιπατεί μεταξύ σας και θέλω είσθαι Θεός σας και σεις θέλετε είσθαι λαός μου.
\par 13 Εγώ είμαι Κύριος ο Θεός σας, όστις σας εξήγαγον εκ της γης των Αιγυπτίων, εκ της δουλείας αυτών· και συνέτριψα τους δεσμούς του ζυγού σας και σας έκαμα να περιπατήτε όρθιοι.
\par 14 Αλλ' εάν δεν μου υπακούσητε και δεν εκτελήτε πάσας ταύτας τας εντολάς μου,
\par 15 και εάν καταφρονήσητε τα προστάγματά μου ή εάν η ψυχή σας αποστραφή τας κρίσεις μου, ώστε να μη εκτελήτε πάσας τας εντολάς μου, ώστε να εξουδενώσητε την διαθήκην μου,
\par 16 και εγώ θέλω κάμει τούτο εις εσάς· θέλω βάλει εφ' υμάς τρόμον, μαρασμόν, και καύσωνα, τα οποία θέλουσι φθείρει τους οφθαλμούς σας και θέλουσι κατατήκει την ψυχήν· και θέλετε σπείρει τον σπόρον σας εις μάτην, διότι οι εχθροί σας θέλουσι τρώγει αυτόν.
\par 17 Και θέλω στήσει το πρόσωπόν μου εναντίον σας, και θέλετε φονευθή έμπροσθεν των εχθρών σας· και εκείνοι, οίτινες σας μισούσι, θέλουσι σας εξουσιάσει και θέλετε φεύγει, ουδενός διώκοντος υμάς.
\par 18 Και εάν μέχρι τούτου δεν μου υπακούσητε, θέλω επιβάλει εις εσάς επταπλάσιον τιμωρίαν διά τας αμαρτίας σας.
\par 19 Και θέλω συντρίψει την υπερηφανίαν της δυνάμεώς σας· και θέλω κάμει τον ουρανόν σας ως σίδηρον και την γην σας ως χαλκόν·
\par 20 και η δύναμίς σας θέλει αναλωθή εις μάτην· διότι η γη σας δεν θέλει δίδει τα γεννήματα αυτής και τα δένδρα της γης δεν θέλουσι δίδει τον καρπόν αυτών.
\par 21 Και εάν πορεύησθε εναντίοι εις εμέ και δεν θέλητε να μου υπακούσητε, θέλω προσθέσει εις εσάς επταπλασίους πληγάς κατά τας αμαρτίας σας.
\par 22 Και θέλω αποστείλει εναντίον σας τα θηρία τα άγρια, τα οποία θέλουσι καταφάγει τα τέκνα σας και εξολοθρεύσει τα κτήνη σας και θέλουσι σας κάμει ολιγοστούς· και θέλουσιν ερημωθή αι οδοί σας.
\par 23 Και εάν εκ τούτων δεν διορθωθήτε επιστρέφοντες εις εμέ, αλλά πορεύησθε εναντίοι εις εμέ,
\par 24 τότε θέλω πορευθή και εγώ εναντίος εις εσάς, και θέλω σας παιδεύσει και εγώ επταπλασίως διά τας αμαρτίας σας.
\par 25 Και θέλω φέρει εφ' υμάς μάχαιραν, ήτις θέλει κάμει την εκδίκησιν της διαθήκης μου· και όταν καταφύγητε εις τας πόλεις σας, θέλω στείλει θανατικόν εν μέσω υμών· και θέλετε παραδοθή εις τας χείρας του εχθρού.
\par 26 Και όταν κατασυντρίψω το στήριγμα του άρτου σας, δέκα γυναίκες θέλουσι ψήνει τους άρτους σας εν ενί κλιβάνω, και οι άρτοι σας θέλουσιν αποδοθή εις εσάς με ζύγιον· και θέλετε τρώγει και δεν θέλετε χορταίνει.
\par 27 Εάν δε και διά τούτων δεν μου υπακούσητε, αλλά πορεύησθε εναντίοι εις εμέ,
\par 28 τότε εγώ θέλω πορευθή εναντίος εις εσάς μετά θυμού και θέλω σας παιδεύσει και εγώ επταπλασίως διά τας αμαρτίας σας.
\par 29 Και θέλετε φάγει τας σάρκας των υιών σας και τας σάρκας των θυγατέρων σας θέλετε φάγει.
\par 30 Και θέλω κατεδαφίσει τους υψηλούς τόπους σας και θέλω καταστρέψει τα είδωλά σας και θέλω ρίψει τα πτώματά σας επί τα πτώματα των βδελυρών ειδώλων σας· και θέλει σας βδελυχθή η ψυχή μου.
\par 31 Και θέλω καταστήσει τας πόλεις σας ερήμους και θέλω εξερημώσει τα αγιαστήριά σας και δεν θέλω οσφρανθή την οσμήν των ευωδιών σας·
\par 32 και θέλω εξερημώσει εγώ την γην σας· και θέλουσι θαυμάσει εις τούτο οι εχθροί σας, οι κατοικούντες εν αυτή.
\par 33 Και θέλω σας διασπείρει μεταξύ των εθνών· και θέλω σύρει οπίσω σας μάχαιραν· και η γη σας θέλει μένει έρημος και αι πόλεις σας θέλουσιν είσθαι έρημοι.
\par 34 Τότε η γη θέλει απολαύσει τα σάββατα αυτής καθ' όλον τον καιρόν όσον αυτή μείνη έρημος και σεις εν τη γη των εχθρών σας· τότε θέλει αναπαυθή η γη και θέλει απολαύσει τα σάββατα αυτής.
\par 35 Καθ' όλον τον καιρόν της ερημώσεως αυτής θέλει αναπαύεσθαι διότι δεν ανεπαύετο εις τα σάββατά σας, ότε κατωκείτε επ' αυτής.
\par 36 Επί δε τους εναπολειφθέντας από σας θέλω επιφέρει δειλίαν εις την καρδίαν αυτών εν τοις τόποις των εχθρών αυτών· και ήχος φύλλου σειομένου θέλει διώκει αυτούς· και θέλουσι φεύγει, ως φεύγοντες από μαχαίρας και θέλουσι πίπτει, ουδενός διώκοντος.
\par 37 Και θέλουσι πίπτει ο εις επί τον άλλον ως έμπροσθεν μαχαίρας, ουδενός διώκοντος· και δεν θέλετε δυνηθή να σταθήτε έμπροσθεν των εχθρών σας.
\par 38 Και θέλετε απολεσθή μεταξύ των εθνών, και η γη των εχθρών σας θέλει σας καταφάγει.
\par 39 Και οι εναπολειφθέντες από σας θέλουσι φθείρεσθαι διά τας ανομίας αυτών εν τοις τόποις των εχθρών σας· και ότι διά τας ανομίας των πατέρων αυτών θέλουσι φθείρεσθαι μετ' αυτών.
\par 40 Εάν δε ομολογήσωσι την ανομίαν αυτών και την ανομίαν των πατέρων αυτών διά την παράβασιν αυτών, την οποίαν παρέβησαν εναντίον μου, και διότι επορεύθησαν έτι εναντίοι εις εμέ,
\par 41 και εγώ επορεύθην εναντίος εις αυτούς, και έφερα αυτούς εις την γην των εχθρών αυτών· εάν τότε ταπεινωθή η καρδία αυτών η απερίτμητος και δεχθώσι τότε την τιμωρίαν της ανομίας αυτών,
\par 42 τότε θέλω ενθυμηθή την διαθήκην μου την προς τον Ιακώβ, και την διαθήκην μου την προς τον Ισαάκ, και την διαθήκην μου την προς τον Αβραάμ θέλω ενθυμηθή· και την γην θέλω ενθυμηθή.
\par 43 Και η γη θέλει μείνει παρητημένη απ' αυτών και θέλει απολαύσει τα σάββατα αυτής, μένουσα έρημος αυτών· και αυτοί θέλουσι δεχθή την τιμωρίαν της ανομίας αυτών· διότι κατεφρόνησαν τας κρίσεις μου και διότι η ψυχή αυτών απεστράφη τα προστάγματά μου.
\par 44 Αλλά και ούτως ενώ ευρίσκονται εν τη γη των εχθρών αυτών, δεν θέλω απορρίψει αυτούς, ουδέ θέλω βδελυχθή αυτούς, ώστε να εξολοθρεύσω αυτούς, και να ματαιώσω την διαθήκην μου την προς αυτούς· διότι εγώ είμαι Κύριος ο Θεός αυτών·
\par 45 αλλά θέλω ενθυμηθή υπέρ αυτών την διαθήκην των πατέρων αυτών, τους οποίους εξήγαγον εκ γης Αιγύπτου, ενώπιον των εθνών, διά να ήμαι Θεός αυτών. Εγώ είμαι ο Κύριος.
\par 46 Ταύτα είναι τα προστάγματα και αι κρίσεις και οι νόμοι, τους οποίους έκαμεν ο Κύριος μεταξύ εαυτού και των υιών Ισραήλ επί του όρους Σινά διά χειρός του Μωϋσέως.

\chapter{27}

\par Και ελάλησε Κύριος προς τον Μωϋσήν, λέγων,
\par 2 Λάλησον προς τους υιούς Ισραήλ και ειπέ προς αυτούς, Όταν τις κάμη επίσημον ευχήν, συ θέλεις κάμνει την εκτίμησιν των ψυχών προς τον Κύριον.
\par 3 Και η εκτίμησίς σου θέλει είσθαι του μεν αρσενικού, από είκοσι ετών μέχρις εξήκοντα ετών, η εκτίμησίς σου βεβαίως θέλει είσθαι πεντήκοντα σίκλοι αργυρίου, κατά τον σίκλον του αγιαστηρίου·
\par 4 εάν δε ήναι θηλυκόν, η εκτίμησίς σου θέλει είσθαι τριάκοντα σίκλοι.
\par 5 Εάν δε ήναι από πέντε ετών μέχρις είκοσι, η εκτίμησίς σου θέλει είσθαι του μεν αρσενικού είκοσι σίκλοι, του δε θηλυκού δέκα σίκλοι.
\par 6 Εάν δε ήναι από ενός μηνός μέχρι πέντε ετών, η εκτίμησίς σου θέλει είσθαι του μεν αρσενικού πέντε σίκλοι αργυρίου· του δε θηλυκού η εκτίμησίς σου τρεις σίκλοι αργυρίου.
\par 7 Εάν δε από εξήκοντα ετών και επάνω, εάν μεν ήναι αρσενικόν, η εκτίμησίς σου θέλει είσθαι δεκαπέντε σίκλοι εάν δε θηλυκόν, δέκα σίκλοι.
\par 8 Και εάν ήναι πτωχότερος της εκτιμήσεώς σου, θέλει παρασταθή έμπροσθεν του ιερέως, και ο ιερεύς θέλει εκτιμήσει αυτόν· κατά την δύναμιν εκείνου όστις έκαμε την ευχήν, ο ιερεύς θέλει εκτιμήσει αυτόν.
\par 9 Και εάν η ευχή ήναι κτήνος, εκ των όσα προσφέρονται δώρον προς τον Κύριον, παν ό,τι δίδει τις εκ τούτων εις τον Κύριον θέλει είσθαι άγιον.
\par 10 Δεν θέλει αλλάξει αυτό, ουδέ θέλει αντικαταστήσει καλόν αντί κακού, ή κακόν αντί καλού· εάν δε ποτέ ανταλλάξη κτήνος αντί κτήνους, τότε και αυτό και το αντάλλαγμα αυτού θέλουσιν είσθαι άγια.
\par 11 Εάν δε ήναι τι κτήνος ακάθαρτον, εκ των όσα δεν προσφέρονται δώρον προς τον Κύριον, τότε θέλει παραστήσει το κτήνος έμπροσθεν του ιερέως·
\par 12 και θέλει εκτιμήσει αυτό ο ιερεύς, είτε καλόν είναι είτε κακόν· κατά την εκτίμησίν σου, ω ιερεύ, ούτω θέλει είσθαι.
\par 13 Και εάν τις θελήση να εξαγοράση αυτό, τότε θέλει προσθέσει το πέμπτον αυτού εις την εκτίμησίν σου.
\par 14 Και όταν τις αφιερώση την οικίαν αυτού αφιέρωμα εις τον Κύριον, τότε ο ιερεύς θέλει εκτιμήσει αυτήν, είτε καλή είναι είτε κακή· καθώς εκτιμήση αυτήν ο ιερεύς, ούτω θέλει είσθαι.
\par 15 Και εάν ο αφιερώσας αυτήν θελήση να εξαγοράση την οικίαν αυτού, θέλει προσθέσει το πέμπτον του αργυρίου της εκτιμήσεώς σου εις αυτήν και θέλει είσθαι αυτού.
\par 16 Και εάν τις αφιερώση εις τον Κύριον μέρος του αγρού της ιδιοκτησίας αυτού, η εκτίμησίς σου θέλει είσθαι κατά τον σπόρον αυτού· εν χομόρ σπόρου κριθής θέλει εκτιμηθή αντί πεντήκοντα σίκλων αργυρίου.
\par 17 Εάν από του έτους της αφέσεως αφιερώση τον αγρόν αυτού, κατά την εκτίμησίν σου θέλει είσθαι.
\par 18 Αλλ' εάν μετά την άφεσιν αφιερώση τον αγρόν αυτού, ο ιερεύς θέλει λογαριάσει εις αυτόν το αργύριον κατά τα επίλοιπα έτη μέχρι του έτους της αφέσεως, και θέλει αφαιρεθή από της εκτιμήσεώς σου.
\par 19 Εάν δε ποτέ ο αφιερώσας τον αγρόν θελήση να εξαγοράση αυτόν, θέλει προσθέσει εις αυτόν το πέμπτον του αργυρίου της εκτιμήσεώς σου, και θέλει είσθαι αυτού.
\par 20 Και εάν δεν εξαγοράση τον αγρόν ή εάν επώλησε τον αγρόν εις άλλον τινά, δεν θέλει εξαγοράζεσθαι πλέον.
\par 21 Αλλ' όταν ο αγρός παρέλθη την άφεσιν, θέλει είσθαι άγιος εις τον Κύριον, ως αγρός καθιερωμένος· η κυριότης αυτού θέλει είσθαι του ιερέως.
\par 22 Εάν δε αφιερώση τις εις τον Κύριον αγρόν τον οποίον ηγόρασεν, όστις δεν είναι εκ των αγρών της ιδιοκτησίας αυτού·
\par 23 ο ιερεύς θέλει λογαριάσει εις αυτόν την αξίαν της εκτιμήσεώς σου μέχρι του έτους της αφέσεως· και θέλει δώσει την εκτίμησίν σου την ημέραν εκείνην· είναι άγιον εις τον Κύριον.
\par 24 Εις το έτος της αφέσεως ο αγρός θέλει αποδοθή εις εκείνον, από του οποίου ηγοράσθη, εις τον έχοντα την κυριότητα της γης.
\par 25 Και πάσαι αι εκτιμήσεις σου θέλουσιν είσθαι κατά τον σίκλον του αγιαστηρίου· είκοσι γερά θέλει είσθαι ο σίκλος.
\par 26 Πλην το πρωτότοκον μεταξύ των κτηνών, το οποίον ανήκει ως πρωτότοκον εις τον Κύριον, ουδείς θέλει αφιερώσει αυτό· είτε μόσχος είτε αρνίον, του Κυρίου είναι.
\par 27 Και εάν ήναι από ακαθάρτων κτηνών, θέλει εξαγοράσει αυτό κατά την εκτίμησίν σου και θέλει προσθέσει το πέμπτον αυτού επ' αυτό· ή εάν δεν εξαγοράζηται, θέλει πωληθή κατά την εκτίμησίν σου.
\par 28 Ουδέν όμως καθιέρωμα, το οποίον καθιερώση τις εις τον Κύριον εκ των όσα έχει, από ανθρώπου έως κτήνους και έως αγρού της ιδιοκτησίας αυτού, θέλει πωληθή ουδέ θέλει εξαγορασθή· παν καθιέρωμα είναι αγιώτατον εις τον Κύριον.
\par 29 Ουδέν καθιέρωμα καθιερωθέν παρά ανθρώπου θέλει εξαγορασθή· εξάπαντος θέλει θανατωθή.
\par 30 Και παν δέκατον της γης, είτε εκ του σπόρου της γης είτε εκ του καρπού των δένδρων, του Κυρίου είναι· είναι άγιον εις τον Κύριον.
\par 31 Και εάν ποτέ θελήση τις να εξαγοράση το δέκατον αυτού, θέλει προσθέσει εις αυτό το πέμπτον αυτού.
\par 32 Και παν δέκατον βοών και προβάτων, παντός ζώου διαβαίνοντος υποκάτωθεν της ράβδου, το δέκατον θέλει είσθαι άγιον εις τον Κύριον.
\par 33 Δεν θέλει διακρίνει είτε καλόν είναι είτε κακόν ουδέ θέλει αλλάξει αυτό· και εάν ποτέ αλλάξη αυτό, και αυτό και το αντάλλαγμα αυτού θέλουσιν είσθαι άγια· δεν θέλει εξαγορασθή.
\par 34 Αύται είναι αι εντολαί, τας οποίας προσέταξε Κύριος εις τον Μωϋσήν διά τους υιούς Ισραήλ εν τω όρει Σινά.


\end{document}