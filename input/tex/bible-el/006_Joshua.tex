\begin{document}

\title{Ιησούς του Ναυή}


\chapter{1}

\par 1 Και μετά την τελευτήν του Μωϋσέως του δούλου του Κυρίου, είπε Κύριος προς Ιησούν τον υιόν του Ναυή, τον υπηρέτην του Μωϋσέως, λέγων,
\par 2 Μωϋσής ο θεράπων μου ετελεύτησε· τώρα λοιπόν σηκωθείς διάβηθι τον Ιορδάνην τούτον, συ και πας ο λαός ούτος, προς την γην την οποίαν εγώ δίδω εις αυτούς, εις τους υιούς Ισραήλ.
\par 3 Πάντα τον τόπον, επί του οποίου πατήση το ίχνος των ποδών σας, εις εσάς έδωκα αυτόν, καθώς είπα προς τον Μωϋσήν·
\par 4 από της ερήμου και του Λιβάνου τούτου και έως του ποταμού του μεγάλου, του ποταμού του Ευφράτου, πάσα η γη των Χετταίων, και έως της θαλάσσης της μεγάλης προς δυσμάς του ηλίου, θέλει είσθαι το όριόν σας.
\par 5 Δεν θέλει δυνηθή άνθρωπος να σταθή εναντίον σου πάσας τας ημέρας της ζωής σου· καθώς ήμην μετά του Μωϋσέως, θέλω είσθαι μετά σού· δεν θέλω σε αφήσει ουδέ σε εγκαταλείψει.
\par 6 Ίσχυε και ανδρίζου· διότι συ θέλεις κληροδοτήσει εις τον λαόν τούτον την γην, την οποίαν ώμοσα προς τους πατέρας αυτών να δώσω εις αυτούς.
\par 7 Μόνον ίσχυε και ανδρίζου σφόδρα, διά να προσέχης να κάμνης κατά πάντα τον νόμον, τον οποίον προσέταξεν εις σε Μωϋσής ο θεράπων μου· μη εκκλίνης απ' αυτού δεξιά ή αριστερά, διά να φέρησαι μετά συνέσεως πανταχού όπου αν υπάγης.
\par 8 Δεν θέλει απομακρυνθή τούτο το βιβλίον του νόμου από του στόματός σου, αλλ' εν αυτώ θέλεις μελετά ημέραν και νύκτα, διά να προσέχης να κάμνης κατά πάντα όσα είναι γεγραμμένα εν αυτώ· διότι τότε θέλεις ευοδούσθαι εις την οδόν σου, και τότε θέλεις φέρεσθαι μετά συνέσεως.
\par 9 Δεν σε προστάζω εγώ; ίσχυε και ανδρίζου· μη φοβηθής μηδέ δειλιάσης· διότι είναι μετά σου Κύριος ο Θεός σου όπου αν υπάγης.
\par 10 Και προσέταξεν ο Ιησούς τους άρχοντας του λαού, λέγων,
\par 11 Περάσατε διά μέσου του στρατοπέδου και προστάξατε τον λαόν, λέγοντες, Ετοιμάσατε εις εαυτούς εφόδια· διότι μετά τρεις ημέρας θέλετε διαβή τον Ιορδάνην τούτον, διά να εισέλθητε να κληρονομήσητε την γην, την οποίαν Κύριος ο Θεός σας δίδει εις εσάς διά να κληρονομήσητε αυτήν.
\par 12 Και προς τους Ρουβηνίτας και προς τους Γαδίτας και προς το ήμισυ της φυλής του Μανασσή είπεν ο Ιησούς, λέγων,
\par 13 Ενθυμήθητε τον λόγον τον οποίον προσέταξεν εις εσάς Μωϋσής ο δούλος του Κυρίου, λέγων, Κύριος ο Θεός σας σας ανέπαυσε και σας έδωκε την γην ταύτην·
\par 14 αι γυναίκές σας, τα τέκνα σας και τα κτήνη σας θέλουσι μείνει εν τη γη, την οποίαν ο Μωϋσής έδωκεν εις εσάς εντεύθεν του Ιορδάνου· σεις δε θέλετε διαβή έμπροσθεν των αδελφών σας ώπλισμένοι, πάντες οι δυνατοί εν ισχύϊ, και θέλετε βοηθήσει αυτούς·
\par 15 εωσού αναπαύση ο Κύριος τους αδελφούς σας καθώς και εσάς, και να κληρονομήσωσι και αυτοί την γην, την οποίαν Κύριος ο Θεός σας δίδει εις αυτούς· τότε θέλετε επιστρέψει εις την γην της κληρονομίας σας, και θέλετε κληρονομήσει αυτήν, την οποίαν Μωϋσής ο δούλος του Κυρίου έδωκεν εις εσάς εντεύθεν του Ιορδάνου, προς ανατολάς ηλίου.
\par 16 Και απεκρίθησαν προς τον Ιησούν, λέγοντες, Πάντα όσα προστάζεις εις ημάς θέλομεν κάμει· και πανταχού όπου αποστείλης ημάς, θέλομεν υπάγει·
\par 17 καθώς υπηκούομεν κατά πάντα εις τον Μωϋσήν, ούτω θέλομεν υπακούει και εις σέ· μόνον Κύριος ο Θεός σου να ήναι μετά σου, καθώς ήτο μετά του Μωϋσέως·
\par 18 πας άνθρωπος, όστις εναντιωθή εις τας προσταγάς σου και δεν υπακούση εις τους λόγους σου κατά πάντα όσα προστάξης αυτόν, ας θανατόνηται· μόνον ίσχυε και ανδρίζου.

\chapter{2}

\par 1 Και απέστειλεν Ιησούς ο υιός του Ναυή εκ Σιττείμ δύο άνδρας να κατασκοπεύσωσι κρυφίως, λέγων, Υπάγετε, ίδετε την γην και την Ιεριχώ. Οι δε υπήγον και εισήλθον εις οικίαν γυναικός πόρνης, ονομαζομένης Ραάβ, και κατέλυσαν εκεί.
\par 2 Απήγγειλαν δε προς τον βασιλέα της Ιεριχώ, λέγοντες, Ιδού, ήλθον ενταύθα την νύκτα άνδρες εκ των υιών Ισραήλ, διά να κατασκοπεύσωσι την γην.
\par 3 Και απέστειλεν ο βασιλεύς της Ιεριχώ προς την Ραάβ, λέγων, Εξάγαγε τους άνδρας τους εισελθόντας προς σε, οίτινες εισήλθον εις την οικίαν σου· διότι ήλθον να κατασκοπεύσωσι πάσαν την γην.
\par 4 Και λαβούσα η γυνή τους δύο άνδρας έκρυψεν αυτούς και είπε, Ναι μεν εισήλθον προς εμέ οι άνδρες και δεν εξεύρω πόθεν ήσαν·
\par 5 ενώ δε έμελλε να κλεισθή η πύλη, ότε εσκότασεν, οι άνδρες εξήλθον· δεν εξεύρω που υπήγον οι άνδρες· τρέξατε ταχέως κατόπιν αυτών, διότι θέλετε προφθάσει αυτούς.
\par 6 Αυτή όμως είχεν αναβιβάσει αυτούς επί το δώμα και σκεπάσει αυτούς με λινοκαλάμην, την οποίαν είχεν εστοιβαγμένην επί του δώματος.
\par 7 Και οι άνδρες έτρεξαν κατόπιν αυτών διά της οδού της προς τον Ιορδάνην, μέχρι των διαβάσεων· και ευθύς αφού ανεχώρησαν οι τρέχοντες κατόπιν αυτών, εκλείσθη η πύλη.
\par 8 Και πριν εκείνοι πλαγιάσωσιν, αυτή ανέβη προς αυτούς επί το δώμα.
\par 9 Και είπε προς τους άνδρας, Γνωρίζω ότι ο Κύριος έδωκεν εις εσάς την γήν· και ότι ο τρόμος σας επέπεσεν εφ' ημάς, και ότι πάντες οι κάτοικοι της γης ενεκρώθησαν εκ του φόβου σας·
\par 10 επειδή ηκούσαμεν πως ο Κύριος εξήρανε τα ύδατα της Ερυθράς θαλάσσης έμπροσθέν σας, ότε εξήλθετε εξ Αιγύπτου· και τι εκάμετε εις τους δύο βασιλείς των Αμορραίων, τους πέραν του Ιορδάνου, εις τον Σηών και εις τον Ωγ, τους οποίους εξωλοθρεύσατε·
\par 11 και καθώς ηκούσαμεν, διελύθη καρδία ημών, και δεν έμεινε πλέον πνοή εις ουδένα εκ του φόβου σας· διότι Κύριος ο Θεός σας, αυτός είναι Θεός εν τω ουρανώ άνω και επί της γης κάτω.
\par 12 Και τώρα, ομόσατέ μοι, παρακαλώ, εις τον Κύριον ότι, καθώς εγώ έκαμα έλεος εις εσάς, θέλετε κάμει και σεις έλεος εις την οικογένειαν του πατρός μου· και δότε εις εμέ σημείον πίστεως,
\par 13 ότι θέλετε φυλάξει την ζωήν εις τον πατέρα μου και εις την μητέρα μου και εις τους αδελφούς μου και εις τας αδελφάς μου και πάντα όσα έχουσι, και θέλετε σώσει την ζωήν ημών εκ του θανάτου.
\par 14 Και απεκρίθησαν προς αυτήν οι άνδρες, Η ζωή ημών εις θάνατον ας παραδοθή αντί της ιδικής σας, αν μόνον δεν φανερώσητε ταύτην την υπόθεσιν ημών, εάν ημείς, όταν ο Κύριος παραδώση εις ημάς την γην, δεν δείξωμεν έλεος και πίστιν εις σε.
\par 15 Τότε κατεβίβασεν αυτούς με σχοινίον διά της θυρίδος· διότι η οικία αυτής ήτο εν τω τείχει της πόλεως και εν τω τείχει κατώκει.
\par 16 Και είπε προς αυτούς, Απέλθετε εις την ορεινήν, διά να μη σας συναντήσωσιν οι καταδιώκοντες· και κρύφθητε εκεί τρεις ημέρας, εωσού επιστρέψωσιν οι καταδιώκοντες· και έπειτα θέλετε υπάγει εις την οδόν σας.
\par 17 Και είπαν προς αυτήν οι άνδρες, Ούτω θέλομεν είσθαι καθαροί από του όρκου σου τούτου, τον οποίον έκαμες ημάς να ομόσωμεν·
\par 18 ιδού, όταν ημείς εισερχώμεθα εις την γην, θέλεις δέσει το σχοινίον τούτου του κοκκίνου νήματος εις την θυρίδα, από της οποίας κατεβίβασας ημάς· και τον πατέρα σου και την μητέρα σου και τους αδελφούς σου και πάσαν την οικογένειαν του πατρός σου, θέλεις συνάξει προς σεαυτήν εις την οικίαν·
\par 19 και πας όστις εξέλθη εκ της θύρας της οικίας σου, το αίμα αυτού θέλει είσθαι επί της κεφαλής αυτού, ημείς δε θέλομεν είσθαι καθαροί· όστις δε μένη μετά σου εν τη οικία, το αίμα αυτού θέλει είσθαι επί της κεφαλής ημών, εάν τις βάλη χείρα επ' αυτόν·
\par 20 αλλ' εάν φανερώσης την υπόθεσιν ημών ταύτην, τότε θέλομεν είσθαι λελυμένοι από του όρκου σου, τον οποίον έκαμες ημάς να ομόσωμεν.
\par 21 Και είπε, Κατά τους λόγους σας, ούτως, ας γείνη. Και εξαπέστειλεν αυτούς, και ανεχώρησαν· αυτή δε έδεσε το κόκκινον σχοινίον εις την θυρίδα.
\par 22 Και ανεχώρησαν και ήλθον εις την ορεινήν και έμειναν εκεί τρεις ημέρας, εωσού επέστρεψαν οι καταδιώκοντες· και εζήτησαν αυτούς οι καταδιώκοντες καθ' όλην την οδόν, πλην δεν εύρηκαν.
\par 23 Και υπέστρεψαν οι δύο άνδρες και κατέβησαν εκ του όρους και διέβησαν και ήλθον προς Ιησούν τον υιόν του Ναυή, και διηγήθησαν προς αυτόν πάντα όσα συνέβησαν εις αυτούς.
\par 24 Και είπον προς τον Ιησούν, Βεβαίως ο Κύριος παρέδωκεν εις τας χείρας ημών πάσαν την γήν· και μάλιστα πάντες οι κάτοικοι του τόπου ενεκρώθησαν εκ του φόβου ημών.

\chapter{3}

\par 1 Και εξηγέρθη ο Ιησούς πρωΐ· και ανεχώρησαν εκ Σιττείμ και ήλθον έως του Ιορδάνου, αυτός και πάντες οι υιοί Ισραήλ, και διενυκτέρευσαν εκεί πριν διαβώσι.
\par 2 μετά δε τρεις ημέρας επέρασαν διά μέσον του στρατοπέδου οι άρχοντες,
\par 3 και προσέταξαν τον λαόν, λέγοντες, Όταν ίδητε την κιβωτόν της διαθήκης Κυρίου του Θεού σας και τους ιερείς τους Λευΐτας βαστάζοντας αυτήν, τότε σεις θέλετε κινηθή από των τόπων σας και υπάγει οπίσω αυτής·
\par 4 πλην ας ήναι διάστημα μεταξύ υμών και εκείνης, έως δύο χιλιάδων πηχών κατά το μέτρον, μη πλησιάσητε εις αυτήν, διά να γνωρίζητε την οδόν την οποίαν πρέπει να βαδίζητε· διότι δεν επεράσατε την οδόν ταύτην χθές και προχθές.
\par 5 Και είπεν ο Ιησούς προς τον λαόν, Καθαρίσθητε, διότι αύριον θέλει κάμει ο Κύριος εν μέσω υμών θαυμάσια.
\par 6 Και είπεν ο Ιησούς προς τους ιερείς λέγων, Σήκωσατε την κιβωτόν της διαθήκης και προπορεύεσθε έμπροσθεν του λαού. Και εσήκωσαν την κιβωτόν της διαθήκης και επορεύοντο έμπροσθεν του λαού.
\par 7 Και είπε Κύριος προς τον Ιησούν, Εν τη ημέρα ταύτη αρχίζω να σε μεγαλύνω ενώπιον παντός του Ισραήλ· διά να γνωρίσωσιν ότι, καθώς ήμην μετά του Μωϋσέως, θέλω είσθαι και μετά σού·
\par 8 συ λοιπόν πρόσταξον τους ιερείς τους βαστάζοντας την κιβωτόν της διαθήκης, λέγων, Όταν φθάσητε εις το χείλος του ύδατος του Ιορδάνου, θέλετε σταθή εν τω Ιορδάνη.
\par 9 Και είπεν ο Ιησούς προς τους υιούς Ισραήλ, Προσέλθετε ενταύθα και ακούσατε τους λόγους Κυρίου του Θεού σας.
\par 10 Και είπεν ο Ιησούς, Εκ τούτου θέλετε γνωρίσει, ότι ο Θεός ο ζων είναι εν τω μέσω υμών, και ότι κατά κράτος θέλει εξολοθρεύσει απ' έμπροσθέν σας τους Χαναναίους και τους Χετταίους και τους Ευαίους και τους Φερεζαίους και τους Γεργεσαίους και τους Αμορραίους και τους Ιεβουσαίους·
\par 11 ιδού, η κιβωτός της διαθήκης του Κυρίου πάσης της γης προβαίνει έμπροσθέν σας εις τον Ιορδάνην·
\par 12 και τώρα εκλέξατε εις εαυτούς δώδεκα άνδρας από των φυλών του Ισραήλ, ανά ένα άνδρα κατά φυλήν·
\par 13 και καθώς τα ίχνη των ποδών των ιερέων, των βασταζόντων την κιβωτόν του Κυρίου, του Κυρίου πάσης της γης, πατήσωσιν εν τοις ύδασι του Ιορδάνου, τα ύδατα του Ιορδάνου θέλουσι διακοπή, τα ύδατα τα καταβαίνοντα άνωθεν, και θέλουσι σταθή εις σωρόν ένα.
\par 14 Και καθώς εσηκώθη ο λαός εκ των σκηνών αυτών, διά να διαβώσι τον Ιορδάνην, και οι ιερείς οι βαστάζοντες την κιβωτόν της διαθήκης έμπροσθεν του λαού,
\par 15 και καθώς ήλθον οι βαστάζοντες την κιβωτόν έως του Ιορδάνου, και οι πόδες των ιερέων των βασταζόντων την κιβωτόν εβράχησαν κατά το χείλος του ύδατος, διότι ο Ιορδάνης πλημμυρεί καθ' όλας τας όχθας αυτού πάσας τας ημέρας του θερισμού,
\par 16 εστάθησαν τα ύδατα τα καταβαίνοντα άνωθεν και υψώθησαν εις ένα σωρόν πολύ μακράν, από της πόλεως Αδάμ, ήτις είναι εις τα πλάγια της Ζαρετάν· τα δε καταβαίνοντα κάτω προς την θάλασσαν της πεδιάδος, την αλμυράν θάλασσαν, αποκοπέντα εξέλιπον· και ο λαός επέρασε κατέναντι της Ιεριχώ.
\par 17 Και οι ιερείς, οι βαστάζοντες την κιβωτόν της διαθήκης του Κυρίου, ίσταντο στερεοί επί ξηράς εν μέσω του Ιορδάνου· και πάντες οι Ισραηλίται διέβαινον διά ξηράς, εωσού ετελείωσε πας ο λαός διαβαίνων τον Ιορδάνην.

\chapter{4}

\par 1 Και αφού πας ο λαός ετελείωσε διαβαίνων τον Ιορδάνην, είπε Κύριος προς τον Ιησούν, λέγων,
\par 2 Λάβετε εις εαυτούς δώδεκα άνδρας εκ του λαού, ανά ένα άνδρα κατά φυλήν,
\par 3 και πρόσταξον αυτούς, λέγων, Λάβετε εντεύθεν εκ μέσου του Ιορδάνου, εκ του τόπου όπου των ιερέων οι πόδες εστάθησαν στερεοί, δώδεκα λίθους· και θέλετε μετακομίσει αυτούς μεθ' εαυτών, και θέλετε θέσει αυτούς εν τω τόπω όπου στρατοπεδεύσητε ταύτην την νύκτα.
\par 4 Τότε ο Ιησούς προσεκάλεσε τους δώδεκα άνδρας, τους οποίους είχε διορίσει εκ των υιών Ισραήλ, ανά ένα άνδρα κατά φυλήν·
\par 5 και είπε προς αυτούς ο Ιησούς, διάβητε έμπροσθεν της κιβωτού Κυρίου του Θεού σας εν τω μέσω του Ιορδάνου, και σηκώσατε έκαστος από σας ένα λίθον επί τους ώμους αυτού, κατά τον αριθμόν των φυλών των υιών Ισραήλ·
\par 6 διά να ήναι τούτο σημείον μεταξύ σας· ώστε όταν ερωτώσιν οι υιοί σας εις το μέλλον λέγοντες, Τι δηλούσιν εις εσάς οι λίθοι ούτοι;
\par 7 τότε θέλετε αποκριθή προς αυτούς, Ότι εκόπησαν τα ύδατα του Ιορδάνου απ' έμπροσθεν της κιβωτού της διαθήκης του Κυρίου· ότε διέβαινε τον Ιορδάνην, τα ύδατα του Ιορδάνου εκόπησαν· και οι λίθοι ούτοι θέλουσιν είσθαι προς τους υιούς Ισραήλ εις μνημόσυνον έως αιώνος.
\par 8 Και έκαμον ούτως οι υιοί Ισραήλ καθώς προσέταξεν ο Ιησούς, και έλαβον δώδεκα λίθους εκ μέσου του Ιορδάνου, καθώς είπε Κύριος προς τον Ιησούν, κατά τον αριθμόν των φυλών των υιών Ισραήλ, και μετεκόμισαν αυτούς μεθ' εαυτών εις τον τόπον όπου κατέλυσαν και έθεσαν αυτούς εκεί.
\par 9 Και άλλους δώδεκα λίθους έστησεν ο Ιησούς εν τω μέσω του Ιορδάνου, εν τω τόπω όπου εστάθησαν οι πόδες των ιερέων, των βασταζόντων την κιβωτόν της διαθήκης· και εκεί είναι μέχρι της σήμερον.
\par 10 Οι δε ιερείς οι βαστάζοντες την κιβωτόν ίσταντο εν τω μέσω του Ιορδάνου, εωσού ετελέσθησαν πάντα όσα ο Κύριος προσέταξεν εις τον Ιησούν να είπη προς τον λαόν, κατά πάντα όσα ο Μωϋσής προσέταξεν εις τον Ιησούν· και έσπευσεν ο λαός και διέβη.
\par 11 Και αφού πας ο λαός ετελείωσε διαβαίνων, διέβη και η κιβωτός του Κυρίου και οι ιερείς έμπροσθεν του λαού.
\par 12 Και οι υιοί Ρουβήν και οι υιοί Γαδ και το ήμισυ της φυλής Μανασσή διέβησαν ώπλισμένοι έμπροσθεν των υιών Ισραήλ, καθώς είπε προς αυτούς ο Μωϋσής.
\par 13 Έως τεσσαράκοντα χιλιάδες ένοπλοι διέβησαν έμπροσθεν του Κυρίου εις πόλεμον, προς τας πεδιάδας της Ιεριχώ.
\par 14 Εν εκείνη τη ημέρα εμεγάλυνεν ο Κύριος τον Ιησούν ενώπιον παντός του Ισραήλ, και εφοβούντο αυτόν, καθώς εφοβούντο τον Μωϋσήν, πάσας τας ημέρας της ζωής αυτού.
\par 15 Και είπε Κύριος προς τον Ιησούν λέγων,
\par 16 Πρόσταξον τους ιερείς τους βαστάζοντας την κιβωτόν του μαρτυρίου, να αναβώσιν εκ του Ιορδάνου.
\par 17 Και προσέταξεν ο Ιησούς τους ιερείς λέγων, Ανάβητε εκ του Ιορδάνου.
\par 18 Και αφού οι ιερείς οι βαστάζοντες την κιβωτόν της διαθήκης του Κυρίου ανέβησαν εκ μέσου του Ιορδάνου, και τα ίχνη των ποδών των ιερέων επάτησαν επί της ξηράς, τα ύδατα του Ιορδάνου επιστρέψαντα εις τον τόπον αυτών επλημμύρησαν πάσας τας όχθας αυτού, καθώς πρότερον.
\par 19 Και ανέβη ο λαός εκ του Ιορδάνου την δεκάτην του πρώτου μηνός και εστρατοπέδευσαν εν Γαλγάλοις προς το ανατολικόν μέρος της Ιεριχώ.
\par 20 Και τους δώδεκα εκείνους λίθους, τους οποίους έλαβον εκ του Ιορδάνου, έστησεν ο Ιησούς εν Γαλγάλοις.
\par 21 Και είπε προς τους υιούς Ισραήλ, λέγων, Όταν εις το μέλλον ερωτώσιν οι υιοί σας τους πατέρας αυτών, λέγοντες, Τι δηλούσιν οι λίθοι ούτοι;
\par 22 τότε θέλετε αναγγείλει προς τους υιούς σας, λέγοντες, Διά ξηράς διέβη ο Ισραήλ τον Ιορδάνην τούτον·
\par 23 διότι απεξήρανε Κύριος ο Θεός σας τα ύδατα του Ιορδάνου έμπροσθέν σας, εωσού διέβητε, καθώς έκαμε Κύριος ο Θεός σας εις την Ερυθράν θάλασσαν, την οποίαν απεξήρανεν έμπροσθεν ημών, εωσού διέβημεν·
\par 24 διά να γνωρίσωσι πάντες οι λαοί της γης την χείρα του Κυρίου, ότι είναι κραταιά· διά να φοβήσθε Κύριον τον Θεόν σας πάντοτε.

\chapter{5}

\par 1 Και ότε ήκουσαν πάντες οι βασιλείς των Αμορραίων, οι πέραν του Ιορδάνου προς δυσμάς, και πάντες οι βασιλείς των Χαναναίων, οι παρά την θάλασσαν, ότι ο Κύριος απεξήρανε τα ύδατα του Ιορδάνου απ' έμπροσθεν των υιών Ισραήλ εωσού διέβησαν, διελύθησαν αι καρδίαι αυτών· και δεν έμεινε πλέον εις αυτούς πνοή, από του φόβου των υιών Ισραήλ.
\par 2 Κατ' εκείνον τον καιρόν είπεν ο Κύριος προς τον Ιησούν, Κάμε εις σεαυτόν λιθίνας μαχαίρας κοπτεράς, και περίτεμε εκ δευτέρου τους υιούς Ισραήλ.
\par 3 Και έκαμεν ο Ιησούς εις εαυτόν λιθίνας μαχαίρας κοπτεράς, και περιέτεμε τους υιούς Ισραήλ επί του βουνού των ακροβυστιών.
\par 4 Και η αιτία, διά την οποίαν ο Ιησούς έκαμε την περιτομήν, είναι ότι πας ο λαός ο εξελθών εξ Αιγύπτου, τα αρσενικά, πάντες οι άνδρες του πολέμου, απέθανον εν τη ερήμω καθ' οδόν, αφού εξήλθον εξ Αιγύπτου.
\par 5 Και πας ο λαός ο εξελθών ήτο περιτετμημένος· πας δε ο λαός όστις εγεννήθη εν τη ερήμω καθ' οδόν, αφού εξήλθον εξ Αιγύπτου, δεν είχε περιτμηθή.
\par 6 Διότι τεσσαράκοντα έτη περιήρχοντο οι υιοί Ισραήλ εν τη ερήμω, εωσού ετελεύτησαν πας ο λαός, οι άνδρες του πολέμου, οι εξελθόντες εξ Αιγύπτου, επειδή δεν υπήκουσαν εις την φωνήν του Κυρίου· προς τους οποίους ο Κύριος ώμοσεν, ότι δεν θέλει αφήσει αυτούς να ίδωσι την γην, την οποίαν ώμοσεν ο Κύριος προς τους πατέρας αυτών ότι θέλει δώσει εις ημάς, γην ρέουσαν γάλα και μέλι.
\par 7 Αντί δε τούτων αντικατέστησε τους υιούς αυτών, τους οποίους ο Ιησούς περιέτεμε· διότι ήσαν απερίτμητοι, επειδή δεν είχον περιτέμει αυτούς καθ' οδόν.
\par 8 Και αφού ετελείωσαν περιτέμνοντες πάντα τον λαόν, εκάθηντο εις τους τόπους αυτών εν τω στρατοπέδω, εωσού ιατρεύθησαν.
\par 9 Και είπε Κύριος προς τον Ιησούν, Ταύτην την ημέραν αφήρεσα αφ' υμών τον ονειδισμόν της Αιγύπτου. Διά τούτο ωνομάσθη ο τόπος εκείνος Γάλγαλα έως της σήμερον.
\par 10 Και οι υιοί Ισραήλ εστρατοπέδευσαν εν Γαλγάλοις και έκαμον το πάσχα τη δεκάτη τετάρτη του μηνός προς το εσπέρας, εις τας πεδιάδας της Ιεριχώ.
\par 11 Και τη επαύριον του πάσχα έφαγον άζυμα από του σίτου της γης, και σίτον πεφρυγμένον την αυτήν εκείνην ημέραν.
\par 12 Και τη επαύριον αφού έφαγον από του σίτου της γης, εξέλιπε το μάννα· και δεν είχον πλέον μάννα οι υιοί Ισραήλ, αλλ' έτρωγον από των γεννημάτων της γης Χαναάν τον ενιαυτόν εκείνον.
\par 13 Και ότε ο Ιησούς ήτο πλησίον της Ιεριχώ, ύψωσε τους οφθαλμούς αυτού και είδε, και ιδού, ίστατο κατέναντι αυτού άνθρωπος και η ρομφαία αυτού ήτο γεγυμνωμένη εν τη χειρί αυτού· και προσελθών ο Ιησούς είπε προς αυτόν, Ημέτερος είσαι ή των υπεναντίων ημών;
\par 14 Ο δε είπεν, Ουχί· αλλ' εγώ Αρχιστράτηγος της δυνάμεως του Κυρίου τώρα ήλθον. Και έπεσεν ο Ιησούς επί την γην κατά πρόσωπον αυτού και προσεκύνησε, και είπε προς αυτόν, Τι προστάζει ο κύριός μου εις τον δούλον αυτού;
\par 15 Και ο Αρχιστράτηγος της δυνάμεως του Κυρίου είπε προς τον Ιησούν, Λύσον το υπόδημά σου εκ των ποδών σου· διότι ο τόπος, επί του οποίου ίστασαι, είναι άγιος. Και ο Ιησούς έκαμεν ούτω.

\chapter{6}

\par 1 Η δε Ιεριχώ ήτο συγκεκλεισμένη και ωχυρωμένη εξ αιτίας των υιών Ισραήλ· ουδείς εξήρχετο και ουδείς εισήρχετο.
\par 2 Και είπε Κύριος προς τον Ιησούν, Ιδού, παρέδωκα εις την χείρα σου την Ιεριχώ και τον βασιλέα αυτής και τους δυνατούς εν ισχύϊ.
\par 3 Και θέλετε περιέλθει την πόλιν, πάντες οι άνδρες του πολέμου, κύκλω της πόλεως άπαξ· ούτω θέλεις κάμνει εξ ημέρας.
\par 4 Και επτά ιερείς θέλουσι βαστάζει έμπροσθεν της κιβωτού επτά σάλπιγγας κερατίνας· και την εβδόμην ημέραν θέλετε περιέλθει την πόλιν επτάκις· και οι ιερείς θέλουσι σαλπίζει με τας σάλπιγγας.
\par 5 Και όταν σαλπίσωσι με την κερατίνην επεκτείνοντες, καθώς ακούσητε τον ήχον της σάλπιγγος, πας ο λαός θέλει αλαλάξει μέγαν αλαλαγμόν, και θέλει καταπέσει το τείχος της πόλεως υφ' εαυτό, και ο λαός θέλει αναβή, έκαστος κατ' ενώπιον αυτού.
\par 6 Και εκάλεσεν Ιησούς ο υιός του Ναυή τους ιερείς και είπε προς αυτούς, Λάβετε την κιβωτόν της διαθήκης, και επτά ιερείς ας βαστάζωσιν επτά σάλπιγγας κερατίνας έμπροσθεν της κιβωτού του Κυρίου.
\par 7 Και είπε προς τον λαόν, Περάσατε και περιέλθετε την πόλιν, και οι ώπλισμένοι ας περάσωσιν έμπροσθεν της κιβωτού του Κυρίου.
\par 8 Και αφού ο Ιησούς ελάλησε προς τον λαόν, οι επτά ιερείς βαστάζοντες τας επτά κερατίνας σάλπιγγας έμπροσθεν του Κυρίου επέρασαν και εσάλπιζον με τας σάλπιγγας, και η κιβωτός της διαθήκης του Κυρίου ηκολούθει αυτούς.
\par 9 Και οι ώπλισμένοι προεπορεύοντο των ιερέων, των σαλπιζόντων με τας σάλπιγγας, και η οπισθοφυλακή ηκολούθει όπισθεν της κιβωτού, ενώ οι ιερείς προχωρούντες εσάλπιζον με τας σάλπιγγας.
\par 10 Και προσέταξεν ο Ιησούς τον λαόν, λέγων, Δεν θέλετε αλαλάξει, ουδέ θέλει ακουσθή η φωνή σας, ουδέ θέλει εξέλθει λόγος εκ του στόματός σας, μέχρι της ημέρας καθ' ην θέλω σας ειπεί να αλαλάξητε· τότε θέλετε αλαλάξει.
\par 11 Και η κιβωτός του Κυρίου περιήλθε την πόλιν κύκλω άπαξ· και ήλθον εις το στρατόπεδον και διενυκτέρευσαν εν τω στρατοπέδω.
\par 12 Και εξηγέρθη ο Ιησούς το πρωΐ, και οι ιερείς εσήκωσαν την κιβωτόν του Κυρίου·
\par 13 και οι επτά ιερείς, βαστάζοντες τας επτά κερατίνας σάλπιγγας, προεπορεύοντο της κιβωτού του Κυρίου, πορευόμενοι και σαλπίζοντες με τας σάλπιγγας· και έμπροσθεν αυτών επορεύοντο οι ώπλισμένοι· η δε οπισθοφυλακή ηκολούθει όπισθεν της κιβωτού του Κυρίου, ενώ οι ιερείς προχωρούντες εσάλπιζον με τας σάλπιγγας.
\par 14 Και την δευτέραν ημέραν περιήλθον την πόλιν άπαξ, και επέστρεψαν εις το στρατόπεδον· ούτως έκαμνον εξ ημέρας.
\par 15 Και την εβδόμην ημέραν εξηγέρθησαν περί τα χαράγματα και περιήλθον την πόλιν επτάκις κατά τον αυτόν τρόπον· μόνον εν ταύτη τη ημέρα περιήλθον την πόλιν επτάκις.
\par 16 Και εις την εβδόμην φοράν, ενώ εσάλπιζον οι ιερείς με τας σάλπιγγας, είπεν ο Ιησούς προς τον λαόν, Αλαλάξατε· διότι ο Κύριος παρέδωκεν εις εσάς την πόλιν·
\par 17 και η πόλις θέλει είσθαι ανάθεμα εις τον Κύριον, αυτή και πάντα τα εν αυτή· εις μόνην την Ραάβ την πόρνην θέλει φυλαχθή η ζωή, εις αυτήν και εις πάντας τους όντας εν τη οικία μετ' αυτής· διότι έκρυψε τους κατασκόπους, τους οποίους απεστείλαμεν·
\par 18 σεις όμως φυλάχθητε από του αναθέματος, διά να μη γείνητε ανάθεμα, λαμβάνοντες από του αναθέματος, και καταστήσητε το στρατόπεδον του Ισραήλ ανάθεμα και ταράξητε αυτό·
\par 19 άπαν δε το αργύριον και το χρυσίον και τα σκεύη τα χάλκινα και τα σιδηρά είναι αφιερωμένα εις τον Κύριον· εις το θησαυροφυλάκιον του Κυρίου θέλουσιν εισαχθή.
\par 20 Και ηλάλαξεν ο λαός, ότε εσάλπισαν με τας σάλπιγγας· και ως ήκουσεν ο λαός την φωνήν των σαλπίγγων, τότε ηλάλαξεν ο λαός αλαλαγμόν μέγαν, και κατέπεσε το τείχος υφ' εαυτό, και ανέβη ο λαός εις την πόλιν, έκαστος κατ' ενώπιον αυτού, και εκυρίευσαν την πόλιν.
\par 21 Και εξωλόθρευσαν εν στόματι μαχαίρας πάντας τους εν τη πόλει, άνδρας και γυναίκας, νέους και γέροντας, και βόας και πρόβατα και όνους.
\par 22 Είπε δε ο Ιησούς προς τους δύο άνδρας, τους κατασκοπεύσαντας την γην, Εισέλθετε εις την οικίαν της πόρνης και εξαγάγετε εκείθεν την γυναίκα, και πάντα όσα έχει, καθώς ώμόσατε προς αυτήν.
\par 23 Και εισήλθον οι νέοι οι κατάσκοποι και εξήγαγον την Ραάβ και τον πατέρα αυτής και την μητέρα αυτής και τους αδελφούς αυτής, και πάντα όσα είχε· και εξήγαγον πάσαν την συγγένειαν αυτής και εφύλαξαν αυτούς έξω του στρατοπέδου του Ισραήλ.
\par 24 Και κατέκαυσαν την πόλιν εν πυρί και πάντα τα εν αυτή· μόνον το αργύριον και το χρυσίον και τα σκεύη τα χάλκινα και τα σιδηρά έδωκαν εις το θησαυροφυλάκιον του οίκου του Κυρίου.
\par 25 Και εις την Ραάβ την πόρνην και εις την οικογένειαν του πατρός αυτής και εις πάντα όσα είχε, ο Ιησούς εφύλαξε την ζωήν· και κατοικεί εν τω μέσω του Ισραήλ έως της σήμερον· διότι έκρυψε τους κατασκόπους, τους οποίους απέστειλεν ο Ιησούς διά να κατασκοπεύσωσι την Ιεριχώ.
\par 26 Και ώμοσεν ο Ιησούς κατ' εκείνον τον καιρόν, λέγων, Κατηραμένος ενώπιον του Κυρίου ο άνθρωπος, όστις αναστήση και κτίση την πόλιν ταύτην την Ιεριχώ· με τον θάνατον του πρωτοτόκου υιού αυτού θέλει βάλει τα θεμέλια αυτής, και με τον θάνατον του νεωτάτου υιού αυτού θέλει στήσει τας πύλας αυτής.
\par 27 Και ο Κύριος ήτο μετά του Ιησού, και το όνομα αυτού διεφημίσθη καθ' όλην την γην.

\chapter{7}

\par 1 Οι δε υιοί Ισραήλ έκαμον παράβασιν εις το ανάθεμα· διότι Αχάν, ο υιός του Χαρμί, υιού του Ζαβδί, υιού του Ζερά, εκ της φυλής Ιούδα, έλαβεν από του αναθέματος· και εξήφθη η οργή του Κυρίου κατά των υιών Ισραήλ.
\par 2 Και απέστειλεν ο Ιησούς ανθρώπους εκ της Ιεριχώ εις Γαί, την πλησίον της Βαιθ-αυέν, προς το ανατολικόν μέρος της Βαιθήλ· και είπε προς αυτούς λέγων, Ανάβητε και κατασκοπεύσατε την γην. Και οι άνθρωποι ανέβησαν και κατεσκόπευσαν την Γαί.
\par 3 Και επιστρέψαντες προς τον Ιησούν είπαν προς αυτόν, Ας μη αναβή πας ο λαός, αλλ' ως δύο ή τρεις χιλιάδες άνδρες ας αναβώσι και ας πατάξωσι την Γαί· μη βάλης πάντα τον λαόν εις κόπον φέρων αυτόν έως εκεί· διότι είναι ολίγοι.
\par 4 Και ανέβησαν εκεί εκ του λαού ως τρεις χιλιάδες άνδρες· και έφυγον από προσώπου των ανδρών της Γαί.
\par 5 Και οι άνδρες της Γαί επάταξαν εξ αυτών έως τριάκοντα εξ άνδρας· και κατεδίωξαν αυτούς απ' έμπροσθεν της πύλης έως Σιβαρείμ, και επάταξαν αυτούς εις το κατωφερές· διά το οποίον αι καρδίαι του λαού διελύθησαν, και έγειναν ως ύδωρ.
\par 6 Και διέρρηξεν ο Ιησούς τα ιμάτια αυτού, και έπεσε κατά γης επί πρόσωπον αυτού, έμπροσθεν της κιβωτού του Κυρίου έως εσπέρας, αυτός και οι πρεσβύτεροι του Ισραήλ, και επέθεσαν χώμα επί τας κεφαλάς αυτών.
\par 7 Και είπεν ο Ιησούς, Α Δέσποτα Κύριε, διά τι διεβίβασας τον λαόν τούτον διά του Ιορδάνου, διά να μας παραδώσης εις τας χείρας των Αμορραίων, ώστε να αφανίσωσιν ημάς; είθε να ευχαριστούμεθα καθήμενοι πέραν του Ιορδάνου
\par 8 Ω Κύριε, τι να είπω, αφού ο Ισραήλ έστρεψε τα νώτα έμπροσθεν των εχθρών αυτού;
\par 9 και ακούσαντες οι Χαναναίοι και πάντες οι κάτοικοι της γης, θέλουσι περικυκλώσει ημάς και εξαλείψει το όνομα ημών από της γής· και τι θέλεις κάμει περί του ονόματός σου του μεγάλου;
\par 10 Και είπε Κύριος προς τον Ιησούν, Σηκώθητι· διά τι έπεσες ούτως επί το πρόσωπόν σου;
\par 11 ημάρτησεν ο Ισραήλ, και μάλιστα παρέβησαν την διαθήκην μου, την οποίαν προσέταξα αυτούς· και έτι έλαβον από του αναθέματος και έτι έκλεψαν και έτι εψεύσθησαν και έτι έβαλον αυτό εις τα σκεύη αυτών·
\par 12 διά τούτο δεν θέλουσι δυνηθή οι υιοί Ισραήλ να σταθώσιν έμπροσθεν των εχθρών αυτών, αλλά θέλουσι στρέψει τα νώτα έμπροσθεν των εχθρών αυτών, διότι έγειναν ανάθεμα· ουδέ θέλω είσθαι πλέον με σας, εάν δεν εξαλείψητε το ανάθεμα εκ μέσου σας·
\par 13 σηκωθείς αγίασον τον λαόν και ειπέ, Αγιάσθητε διά την αύριον· διότι ούτω λέγει Κύριος ο Θεός του Ισραήλ· Ανάθεμα είναι εν τω μέσω σου, Ισραήλ· δεν δύνασαι να σταθής έμπροσθεν των εχθρών σου, εωσού αφαιρέσητε το ανάθεμα εκ μέσου σας·
\par 14 προσέλθετε λοιπόν το πρωΐ κατά τας φυλάς σας· και η φυλή, την οποίαν πιάση ο Κύριος, θέλει προσέλθει κατά συγγενείας· και η συγγένεια, την οποίαν πιάση ο Κύριος, θέλει προσέλθει κατ' οικογενείας· και η οικογένεια, την οποίαν πιάση ο Κύριος, θέλει προσέλθει κατά άνδρας·
\par 15 και όστις πιασθή έχων το ανάθεμα, θέλει κατακαυθή εν πυρί, αυτός και πάντα όσα έχει· διότι παρέβη την διαθήκην του Κυρίου και διότι έπραξεν ανομίαν εν τω Ισραήλ.
\par 16 Και εξεγερθείς ο Ιησούς το πρωΐ, προσήγαγε τον Ισραήλ κατά τας φυλάς αυτών· και επιάσθη η φυλή του Ιούδα·
\par 17 και προσήγαγε τας συγγενείας του Ιούδα, και επιάσθη η συγγένεια των Ζαραϊτών· και προσήγαγε την συγγένειαν των Ζαραϊτών κατά άνδρας, και επιάσθη ο Ζαβδί·
\par 18 και προσήγαγε την οικογένειαν αυτού κατά άνδρας, και επιάσθη ο Αχάν, ο υιός του Χαρμί, υιού του Ζαβδί, υιού του Ζερά, εκ της φυλής Ιούδα.
\par 19 Και είπεν ο Ιησούς προς τον Αχάν, Τέκνον μου, δος τώρα δόξαν εις Κύριον τον Θεόν του Ισραήλ, και εξομολογήθητι εις αυτόν, και ειπέ μοι τώρα τι έπραξας· μη κρύψης αυτό απ' εμού.
\par 20 Και απεκρίθη ο Αχάν προς τον Ιησούν και είπε, Αληθώς εγώ ήμαρτον εις Κύριον τον Θεόν του Ισραήλ και έπραξα ούτω και ούτω·
\par 21 ιδών μεταξύ των λαφύρων μίαν καλήν Βαβυλωνικήν στολήν και διακοσίους σίκλους αργυρίου και έλασμα χρυσού βάρους πεντήκοντα σίκλων, επεθύμησα αυτά και έλαβον αυτά· και ιδού, είναι κεκρυμμένα εν τη γη, κατά το μέσον της σκηνής μου, και το αργύριον υποκάτω αυτών.
\par 22 Και απέστειλεν ο Ιησούς ανθρώπους· και έτρεξαν εις την σκηνήν, και ιδού, ήσαν κεκρυμμένα εν τη σκηνή αυτού, και το αργύριον υποκάτω αυτών.
\par 23 Και έλαβον αυτά εκ μέσου της σκηνής, και έφεραν αυτά προς τον Ιησούν και προς πάντας τους υιούς Ισραήλ, και έθεσαν αυτά ενώπιον του Κυρίου.
\par 24 Τότε ο Ιησούς, και πας ο Ισραήλ μετ' αυτού, επίασαν τον Αχάν τον υιόν του Ζερά, και το αργύριον και την στολήν και το έλασμα του χρυσού και τους υιούς αυτού και τας θυγατέρας αυτού και τους βόας αυτού και τους όνους αυτού και τα πρόβατα αυτού και την σκηνήν αυτού και πάντα όσα είχε, και έφεραν αυτούς εις την κοιλάδα Αχώρ.
\par 25 Και είπεν ο Ιησούς, Διά τι κατετάραξας ημάς; ο Κύριος θέλει σε καταταράξει την ημέραν ταύτην. Και πας ο Ισραήλ ελιθοβόλησαν αυτόν με λίθους και κατέκαυσαν αυτούς εν πυρί και ελιθοβόλησαν αυτούς με λίθους.
\par 26 Και έστησαν επ' αυτόν σωρόν λίθων μέγαν, όστις μένει έως της σήμερον· ούτως έπαυσεν ο Κύριος από της εξάψεως του θυμού αυτού· διά τούτο καλείται το όνομα του τόπου εκείνου Κοιλάς Αχώρ έως της ημέρας ταύτης.

\chapter{8}

\par 1 Και είπε Κύριος προς τον Ιησούν, Μη φοβηθής μηδέ δειλιάσης· λάβε μετά σου πάντας τους πολεμικούς άνδρας, και σηκωθείς ανάβα εις Γαί· ιδού, εγώ παρέδωκα εις την χείρα σου; τον βασιλέα της Γαί και τον λαόν αυτού και την πόλιν αυτού και την γην αυτού·
\par 2 και θέλεις κάμει εις την Γαί και εις τον βασιλέα αυτής, καθώς έκαμες εις την Ιεριχώ και εις τον βασιλέα αυτής· μόνον τα λάφυρα αυτής και τα κτήνη αυτής θέλετε λαφυραγωγήσει εις εαυτούς· στήσον ενέδραν κατά της πόλεως όπισθεν αυτής.
\par 3 Και εσηκώθη ο Ιησούς και πας ο λαός ο πολεμιστής, διά να αναβώσιν εις την Γαί· και εξέλεξεν ο Ιησούς τριάκοντα χιλιάδας άνδρας δυνατούς εν ισχύϊ και εξαπέστειλεν αυτούς διά νυκτός,
\par 4 και προσέταξεν εις αυτούς λέγων, Ιδού, σεις θέλετε ενεδρεύει κατά της πόλεως όπισθεν αυτής· μη απομακρυνθήτε πολύ από της πόλεως, και να ήσθε πάντες έτοιμοι·
\par 5 εγώ δε και πας ο λαός ο μετ' εμού θέλομεν πλησιάσει εις την πόλιν· και όταν εξέλθωσιν εναντίον ημών, καθώς πρότερον, τότε ημείς θέλομεν φύγει απ' έμπροσθεν αυτών·
\par 6 και θέλουσιν εξέλθει κατόπιν ημών, εωσού απομακρύνωμεν αυτούς από της πόλεως, διότι θέλουσιν ειπεί, Αυτοί φεύγουσιν απ' έμπροσθεν ημών, καθώς πρότερον· και ημείς θέλομεν φύγει απ' έμπροσθεν αυτών·
\par 7 τότε σεις σηκωθέντες εκ της ενέδρας, θέλετε κυριεύσει την πόλιν· διότι Κύριος ο Θεός σας θέλει παραδώσει αυτήν εις την χείρα σας·
\par 8 και αφού κυριεύσητε την πόλιν, θέλετε καύσει την πόλιν εν πυρί· κατά την προσταγήν του Κυρίου θέλετε κάμει· ιδού, προσέταξα εις εσάς.
\par 9 Ο Ιησούς λοιπόν εξαπέστειλεν αυτούς, και υπήγον εις ενέδραν και εκάθισαν μεταξύ Βαιθήλ και Γαί, προς το δυτικόν μέρος της Γαί· ο δε Ιησούς έμεινε την νύκτα εκείνην εν τω μέσω του λαού.
\par 10 Και εξεγερθείς ο Ιησούς το πρωΐ, επεσκέφθη τον λαόν, και ανέβη αυτός και οι πρεσβύτεροι του Ισραήλ, έμπροσθεν του λαού προς Γαί.
\par 11 Και πας ο λαός ο πολεμιστής, ο μετ' αυτού, ανέβη και επλησίασε και ήλθε κατέναντι της πόλεως και εστρατοπέδευσε κατά το βόρειον μέρος της Γαί· ήτο δε κοιλάς μεταξύ αυτών και της Γαί.
\par 12 Και λαβών έως πέντε χιλιάδας ανδρών, εκάθισεν αυτούς εις ενέδραν μεταξύ Βαιθήλ και Γαί, προς το δυτικόν μέρος της πόλεως.
\par 13 Και αφού διέταξαν τον λαόν, άπαν το στράτευμα το προς βορράν της πόλεως και την ενέδραν αυτού προς δυσμάς της πόλεως, υπήγεν ο Ιησούς εκείνην την νύκτα εις το μέσον της κοιλάδος.
\par 14 Και ως είδεν ο βασιλεύς της Γαί, αυτός και πας ο λαός αυτού, οι άνδρες της πόλεως, έσπευσαν και εξηγέρθησαν πρωΐ και εξήλθον εις συνάντησιν του Ισραήλ προς μάχην, εις ωρισμένην ώραν, επί την πεδιάδα· πλην αυτός δεν ήξευρεν ότι ήτο ένεδρα κατ' αυτού όπισθεν της πόλεως.
\par 15 Και ο Ιησούς και πας ο Ισραήλ προσεποιήθησαν ότι κατετροπώθησαν έμπροσθεν αυτών, και έφευγον διά της οδού της ερήμου.
\par 16 Και συνεκαλέσθησαν πας ο λαός ο εν Γαί, διά να καταδιώξωσιν αυτούς· και κατεδίωξαν τον Ιησούν και απεμακρύνθησαν από της πόλεως.
\par 17 Και δεν απέμεινεν άνθρωπος εν Γαί και εν Βαιθήλ, όστις δεν εξήλθε κατόπιν του Ισραήλ· και αφήκαν ανοικτήν την πόλιν, και κατεδίωκον τον Ισραήλ.
\par 18 Και είπε Κύριος προς τον Ιησούν, Έκτεινον την λόγχην, την εν τη χειρί σου, προς την Γαί· διότι θέλω παραδώσει αυτήν εις την χείρα σου. Και εξέτεινεν ο Ιησούς την λόγχην, την εν τη χειρί αυτού, προς την πόλιν.
\par 19 Και η ενέδρα εσηκώθη μετά σπουδής από της θέσεως αυτής, και ώρμησαν ευθύς ότε εξέτεινε την χείρα αυτού· και εισήλθον εις την πόλιν και εκυρίευσαν αυτήν, και σπεύσαντες έκαυσαν την πόλιν εκ πυρί.
\par 20 Και ότε περιέβλεψαν εις τα οπίσω αυτών οι άνδρες της Γαί, είδον, και ιδού, ανέβαινεν ο καπνός της πόλεως προς τον ουρανόν, και δεν ηδύναντο να φύγωσιν εδώ και εκεί· επειδή ο λαός ο φεύγων προς την έρημον εστράφησαν οπίσω εναντίον των καταδιωκόντων.
\par 21 Ο δε Ιησούς και πας ο Ισραήλ, ιδόντες ότι η ενέδρα είχε κυριεύσει την πόλιν και ότι ανέβαινεν ο καπνός της πόλεως, εστράφησαν οπίσω και επάταξαν τους άνδρας της Γαί.
\par 22 Και οι άλλοι εξήλθον εκ της πόλεως εναντίον αυτών, ώστε ήσαν εν τω μέσω του Ισραήλ εντεύθεν και εκείθεν· και επάταξαν αυτούς, ώστε δεν αφήκαν ουδένα εξ αυτών μείναντα ή διαφυγόντα.
\par 23 Τον δε βασιλέα της Γαί συνέλαβον ζώντα και έφεραν αυτόν προς τον Ιησούν.
\par 24 Και αφού ο Ισραήλ ετελείωσε φονεύων πάντας τους κατοίκους της Γαί εν τη πεδιάδι εκ τη ερήμω, όπου κατεδίωκον αυτούς, και έπεσον πάντες εν στόματι μαχαίρας, εωσού εξωλοθρεύθησαν, επέστρεψε πας ο Ισραήλ εις την Γαί και επάταξαν αυτήν εν στόματι μαχαίρας.
\par 25 Και πάντες οι πεσόντες εν τη ημέρα εκείνη, άνδρες τε και γυναίκες, ήσαν δώδεκα χιλιάδες, πάντες οι άνθρωποι της Γαί.
\par 26 Και δεν έσυρεν ο Ιησούς οπίσω την χείρα αυτού, την οποίαν εξέτεινε με την λόγχην, εωσού εξωλόθρευσε πάντας τους κατοίκους της Γαί.
\par 27 Μόνον τα κτήνη και τα λάφυρα της πόλεως εκείνης ελαφυραγώγησεν ο Ισραήλ εις εαυτόν, κατά τον λόγον του Κυρίου, τον οποίον προσέταξεν εις τον Ιησούν.
\par 28 Και κατέκαυσεν ο Ιησούς την Γαί, και κατέστησεν αυτήν σωρόν παντοτεινόν αοίκητον έως της ημέρας ταύτης.
\par 29 Τον δε βασιλέα της Γαί εκρέμασεν επί ξύλου έως εσπέρας· και ως έδυσεν ο ήλιος, προσέταξεν ο Ιησούς και κατεβίβασαν το πτώμα αυτού από του ξύλου, και έρριψαν αυτό εις την είσοδον της πύλης της πόλεως, και ύψωσαν επ' αυτού σωρόν λίθων μέγαν, όστις μένει έως της σήμερον.
\par 30 Τότε ωκοδόμησεν ο Ιησούς θυσιαστήριον εις Κύριον τον Θεόν του Ισραήλ επί το όρος Εβάλ,
\par 31 καθώς ο Μωϋσής ο δούλος του Κυρίου προσέταξε τους υιούς Ισραήλ, κατά το γεγραμμένον εν τω βιβλίω του νόμου του Μωϋσέως, θυσιαστήριον εκ λίθων ολοκλήρων, επί των οποίων σίδηρος δεν επεβλήθη· και προσέφεραν επ' αυτό ολοκαύτωματα προς τον Κύριον και εθυσίασαν ειρηνικάς προσφοράς.
\par 32 Και έγραψεν εκεί επί τους λίθους το αντίγραφον του νόμου του Μωϋσέως, τον οποίον είχε γράψει ενώπιον των υιών Ισραήλ.
\par 33 Και πας ο Ισραήλ και οι πρεσβύτεροι αυτών και οι άρχοντες και οι κριταί αυτών εστάθησαν εντεύθεν και εντεύθεν της κιβωτού απέναντι των ιερέων των Λευϊτών, των βασταζόντων την κιβωτόν της διαθήκης του Κυρίου, και ο ξένος και ο αυτόχθων· το ήμισυ αυτών προς το όρος Γαριζίν και το ήμισυ αυτών προς το όρος Εβάλ· καθώς πρότερον προσέταξεν ο Μωϋσής ο δούλος του Κυρίου, διά να ευλογήσωσι τον λαόν του Ισραήλ.
\par 34 Και μετά ταύτα ανέγνωσε πάντας τους λόγους του νόμου, τας ευλογίας και τας κατάρας, κατά πάντα τα γεγραμμένα εν τω βιβλίω του νόμου.
\par 35 Δεν ήτο λόγος εκ πάντων όσα προσέταξεν ο Μωϋσής, τον οποίον ο Ιησούς δεν ανέγνωσεν ενώπιον πάσης της συναγωγής του Ισραήλ, μετά των γυναικών και των παιδίων και των ξένων των παρευρισκομένων μεταξύ αυτών.

\chapter{9}

\par 1 Και ότε ήκουσαν πάντες οι βασιλείς, οι εντεύθεν του Ιορδάνου, οι εν τη ορεινή και οι εν τη πεδινή και οι εν πάσι τοις παραλίοις της θαλάσσης της μεγάλης, έως κατέναντι του Λιβάνου, οι Χετταίοι και οι Αμορραίοι, οι Χαναναίοι, οι Φερεζαίοι, οι Ευαίοι και οι Ιεβουσαίοι,
\par 2 συνήχθησαν πάντες ομού, διά να πολεμήσωσι τον Ιησούν και τον Ισραήλ.
\par 3 Οι δε κάτοικοι της Γαβαών ήκουσαν ό,τι έκαμεν ο Ιησούς εις την Ιεριχώ και εις την Γαί,
\par 4 και έπραξαν και ούτοι μετά πανουργίας, και υπήγον και ητοιμάσθησαν με εφόδια, και έλαβον σάκκους παλαιούς επί των όνων αυτών και ασκούς οίνου παλαιούς και κατεσχισμένους και δεδεμένους,
\par 5 και εις τους πόδας αυτών υποδήματα παλαιά και εμβαλωμένα, και ιμάτια παλαιά εφ' εαυτών· και όλος ο άρτος του εφοδιασμού αυτών ήτο ξηρός και κατατεθρυμμένος.
\par 6 Και ήλθον προς τον Ιησούν εις το στρατόπεδον εις Γάλγαλα, και είπον προς αυτόν και προς τους άνδρας του Ισραήλ, Από γης μακράς ήλθομεν· τώρα λοιπόν κάμετε συνθήκην προς ημάς.
\par 7 Και είπον οι άνδρες του Ισραήλ προς τους Ευαίους τούτους, Σεις κατοικείτε ίσως εν τω μέσω ημών, και πως θέλομεν κάμει συνθήκην προς εσάς;
\par 8 Οι δε είπον προς τον Ιησούν, Δούλοί σου είμεθα. Είπε δε προς αυτούς ο Ιησούς, Ποίοι είσθε; και πόθεν έρχεσθε;
\par 9 Και είπον προς αυτόν, Από πολύ μακράς γης ήλθον οι δούλοί σου διά το όνομα Κυρίου του Θεού σου· διότι ηκούσαμεν την φήμην αυτού και πάντα όσα έκαμεν εν Αιγύπτω,
\par 10 και πάντα όσα έκαμεν εις τους δύο βασιλείς των Αμορραίων, τους πέραν του Ιορδάνου, εις τον Σηών βασιλέα της Εσεβών, και εις τον Ωγ βασιλέα της Βασάν, τον εν Ασταρώθ·
\par 11 διά τούτο είπον προς ημάς οι πρεσβύτεροι ημών και πάντες οι κάτοικοι της γης ημών, λέγοντες, Λάβετε εις εαυτούς εφόδια διά την οδόν, και υπάγετε εις συνάντησιν αυτών και είπατε προς αυτούς, δούλοί σας είμεθα· τώρα λοιπόν κάμετε συνθήκην προς ημάς·
\par 12 τον άρτον ημών τούτον ζεστόν ελάβομεν εκ των οικιών ημών, καθ' ην ημέραν εξήλθομεν διά να έλθωμεν προς εσάς· και τώρα, ιδού, είναι ξηρός και κατατεθρυμμένος·
\par 13 και ούτοι οι ασκοί του οίνου, τους οποίους εγεμίσαμεν νέους, και ιδού, είναι κατεσχισμένοι· και τα ιμάτια ημών ταύτα και τα υποδήματα ημών επαλαιώθησαν διά την πολύ μακράν οδόν.
\par 14 Και εδέχθησαν τους άνδρας εξ αιτίας των εφοδίων αυτών, και δεν ηρώτησαν τον Κύριον.
\par 15 Και έκαμεν ο Ιησούς ειρήνην προς αυτούς και έκαμε συνθήκην προς αυτούς, να φυλάξη την ζωήν αυτών· και οι άρχοντες της συναγωγής ώμοσαν προς αυτούς.
\par 16 Και μετά τρεις ημέρας, αφού έκαμον συνθήκην προς αυτούς, ήκουσαν ότι ήσαν γείτονες αυτών και κατώκουν μεταξύ αυτών.
\par 17 Και σηκωθέντες οι υιοί Ισραήλ υπήγον εις τας πόλεις αυτών την τρίτην ημέραν· αι δε πόλεις αυτών ήσαν Γαβαών και Χεφειρά και Βηρώθ και Κιριάθ-ιαρείμ.
\par 18 Και δεν επάταξαν αυτούς οι υιοί Ισραήλ, διότι οι άρχοντες της συναγωγής είχον ομόσει προς αυτούς τον Κύριον τον Θεόν του Ισραήλ. Και εγόγγυζε πάσα η συναγωγή κατά των αρχόντων.
\par 19 Πάντες όμως οι άρχοντες είπον προς πάσαν την συναγωγήν, Ημείς ώμόσαμεν προς αυτούς τον Κύριον τον Θεόν του Ισραήλ· τώρα λοιπόν δεν δυνάμεθα να εγγίσωμεν αυτούς·
\par 20 τούτο θέλομεν κάμει εις αυτούς· θέλομεν φυλάξει την ζωήν αυτών, διά να μη ήναι οργή Θεού εφ' ημάς, διά τον όρκον τον οποίον ώμόσαμεν προς αυτούς.
\par 21 Και οι άρχοντες είπον προς αυτούς, Ας ζώσι· πλην ας ήναι ξυλοκόποι και υδροφόροι εις πάσαν συναγωγήν· καθώς οι άρχοντες υπεσχέθησαν προς αυτούς.
\par 22 Και συνεκάλεσεν αυτούς ο Ιησούς και είπε προς αυτούς, λέγων, Διά τι ηπατήσατε ημάς λέγοντες, πολύ μακράν είμεθα από σας, ενώ σεις κατοικείτε μεταξύ ημών;
\par 23 τώρα λοιπόν επικατάρατοι είσθε, και δεν θέλει λείψει από σας δούλος και ξυλοκόπος και υδροφόρος εις τον οίκον του Θεού μου.
\par 24 Και απεκρίθησαν προς τον Ιησούν λέγοντες, Επειδή οι δούλοί σου έμαθον μετά πληροφορίας όσα Κύριος ο Θεός σου διέταξεν εις τον δούλον αυτού Μωϋσήν, να δώση εις εσάς πάσαν την γην και να εξολοθρεύση έμπροσθέν σας πάντας τους κατοίκους της γης, διά τούτο εφοβήθημεν από σας σφόδρα διά την ζωήν ημών και εκάμομεν το πράγμα τούτο·
\par 25 και τώρα, ιδού, εις τας χείρας σου είμεθα· ό,τι σοι φανή καλόν και αρεστόν να κάμης εις ημάς, κάμε.
\par 26 Και έκαμεν ούτως εις αυτούς, και ηλευθέρωσεν αυτούς εκ της χειρός των υιών Ισραήλ, και δεν εφόνευσαν αυτούς.
\par 27 Και την ημέραν εκείνην έκαμεν αυτούς ο Ιησούς ξυλοκόπους και υδροφόρους μέχρι τούδε, εις την συναγωγήν και εις το θυσιαστήριον του Κυρίου, εις τον τόπον όντινα εκλέξη.

\chapter{10}

\par 1 Και ως ήκουσεν Αδωνισεδέκ ο βασιλεύς της Ιερουσαλήμ ότι ο Ιησούς εκυρίευσε την Γαί και εξωλόθρευσεν αυτήν, ότι, καθώς έκαμεν εις την Ιεριχώ και εις τον βασιλέα αυτής, ούτως έκαμεν εις την Γαί και εις τον βασιλέα αυτής, και ότι οι κάτοικοι της Γαβαών έκαμον ειρήνην μετά του Ισραήλ και έμενον μεταξύ αυτών,
\par 2 εφοβήθησαν σφόδρα· διότι ήτο πόλις μεγάλη η Γαβαών, ως μία των βασιλικών πόλεων, και διότι ήτο μεγαλητέρα της Γαί, και πάντες οι άνδρες αυτής δυνατοί.
\par 3 Διά τούτο απέστειλεν Αδωνισεδέκ ο βασιλεύς της Ιερουσαλήμ προς τον Ωάμ βασιλέα της Χεβρών, και προς τον Πιράμ βασιλέα της Ιαρμούθ, και προς τον Ιαφιά βασιλέα της Λαχείς, και προς τον Δεβείρ βασιλέα της Εγλών, λέγων,
\par 4 Ανάβητε προς εμέ και βοηθήσατέ μοι, διά να πατάξωμεν την Γαβαών· διότι έκαμεν ειρήνην μετά του Ιησού και μετά των υιών Ισραήλ.
\par 5 Και συναχθέντες οι πέντε βασιλείς των Αμορραίων, ο βασιλεύς της Ιερουσαλήμ, ο βασιλεύς της Χεβρών, ο βασιλεύς της Ιαρμούθ, ο βασιλεύς της Λαχείς, ο βασιλεύς της Εγλών, ανέβησαν, αυτοί και πάντα τα στρατεύματα αυτών, και εστρατοπέδευσαν έμπροσθεν της Γαβαών και επολέμουν εναντίον αυτής.
\par 6 Και οι Γαβαωνίται απέστειλαν προς τον Ιησούν εις το στρατόπεδον εις Γάλγαλα, λέγοντες, Μη αποσύρης την χείρα σου από των δούλων σου· ανάβα προς ημάς ταχέως και σώσον ημάς και βοήθησον ημάς· διότι συνήχθησαν εναντίον ημών πάντες οι βασιλείς των Αμορραίων, οι κατοικούντες την ορεινήν.
\par 7 Και ανέβη ο Ιησούς από Γαλγάλων, αυτός και πας ο λαός ο πολεμιστής μετ' αυτού και πάντες οι δυνατοί εν ισχύϊ.
\par 8 Και είπε Κύριος προς τον Ιησούν, Μη φοβηθής αυτούς· διότι παρέδωκα αυτούς εις την χείρα σου· δεν θέλει σταθή έμπροσθέν σου ουδείς εξ αυτών.
\par 9 Ήλθε λοιπόν επ' αυτούς ο Ιησούς εξαίφνης, αναβάς από Γαλγάλων δι' όλης της νυκτός.
\par 10 Και κατετρόπωσεν αυτούς ο Κύριος έμπροσθεν του Ισραήλ, και επάταξαν αυτούς εν σφαγή μεγάλη εν Γαβαών, και κατεδίωξαν αυτούς εις την οδόν την αναβαίνουσαν προς Βαιθ-ωρών, και κατέκοπτον αυτούς έως Αζηκά και έως Μακκηδά.
\par 11 Και ενώ, φεύγοντες απ' έμπροσθεν του Ισραήλ, ήσαν εν τη καταβάσει της Βαιθ-ωρών, ο Κύριος έρριψεν εκ του ουρανού κατ' αυτών λίθους μεγάλους έως Αζηκά, και απέθανον· περισσότεροι ήσαν οι αποθανόντες εκ των λίθων της χαλάζης, παρ' όσους οι υιοί Ισραήλ κατέκοψαν εν μαχαίρα.
\par 12 Τότε ελάλησεν ο Ιησούς προς τον Κύριον, καθ' ην ημέραν ο Κύριος παρέδωκε τους Αμορραίους έμπροσθεν των υιών Ισραήλ, και είπεν ενώπιον του Ισραήλ, Στήθι, ήλιε, επί την Γαβαών, και συ, σελήνη, επί την φάραγγα Αιαλών.
\par 13 Και ο ήλιος εστάθη και η σελήνη έμεινεν, εωσού ο λαός εξεδικήθη τους εχθρούς αυτού. Δεν είναι τούτο γεγραμμένον εν τω βιβλίω του Ιασήρ; Και εστάθη ο ήλιος εν τω μέσω του ουρανού, και δεν έσπευσε να δύση έως μιας ολοκλήρου ημέρας.
\par 14 Και τοιαύτη ημέρα δεν υπήρξεν ούτε πρότερον ούτε ύστερον, ώστε ο Κύριος να ακούση φωνήν ανθρώπου· διότι ο Κύριος επολέμει υπέρ του Ισραήλ.
\par 15 Και επέστρεψεν ο Ιησούς, και πας ο Ισραήλ μετ' αυτού, εις το στρατόπεδον εις Γάλγαλα.
\par 16 Οι δε πέντε βασιλείς ούτοι έφυγον και εκρύφθησαν εις σπήλαιον εν Μακκηδά.
\par 17 Και ανήγγειλαν προς τον Ιησούν, λέγοντες, οι πέντε βασιλείς ευρέθησαν κεκρυμμένοι εις σπήλαιον εν Μακκηδά.
\par 18 Και είπεν ο Ιησούς, Κυλίσατε λίθους μεγάλους εις το στόμα του σπηλαίου, και καταστήσατε ανθρώπους πλησίον αυτού διά να φυλάττωσιν αυτούς·
\par 19 και σεις μη στέκεσθε· καταδιώκετε τους εχθρούς σας και πατάξατε το όπισθεν αυτών· μη αφήσητε αυτούς να εισέλθωσιν εις τας πόλεις αυτών· διότι Κύριος ο Θεός σας παρέδωκεν αυτούς εις τας χείρας σας.
\par 20 Και αφού ο Ιησούς και οι υιοί Ισραήλ ετελείωσαν φονεύοντες αυτούς εν σφαγή μεγάλη σφόδρα, εωσού εξωλοθρεύθησαν, οι λοιποί εξ αυτών, όσοι διεσώθησαν, εισήλθον εις ωχυρωμένας πόλεις.
\par 21 Και πας ο λαός επέστρεψεν εις το στρατόπεδον προς τον Ιησούν εις Μακκηδά εν ειρήνη· ουδείς εκίνησε την γλώσσαν αυτού κατά τινός εκ των υιών Ισραήλ.
\par 22 Και είπεν ο Ιησούς, Ανοίξατε το στόμα του σπηλαίου και εξαγάγετε προς εμέ τους πέντε βασιλείς εκείνους εκ του σπηλαίου.
\par 23 Και έκαμον ούτω και εξήγαγον προς αυτόν τους πέντε βασιλείς εκείνους εκ του σπηλαίου, τον βασιλέα της Ιερουσαλήμ, τον βασιλέα της Χεβρών, τον βασιλέα της Ιαρμούθ, τον βασιλέα της Λαχείς, τον βασιλέα της Εγλών.
\par 24 Και αφού εξήγαγον προς τον Ιησούν τους βασιλείς εκείνους, εκάλεσεν ο Ιησούς πάντας τους άνδρας του Ισραήλ, και είπε προς τους αρχηγούς των πολεμιστών τους ελθόντας μετ' αυτού, Πλησιάσατε, βάλετε τους πόδας σας επί τους τραχήλους των βασιλέων τούτων. Και αυτοί επλησίασαν και έβαλον τους πόδας αυτών επί τους τραχήλους αυτών.
\par 25 Και είπε προς αυτούς ο Ιησούς, Μη φοβείσθε μηδέ δειλιάτε· ανδρίζεσθε και ενδυναμούσθε· επειδή ούτω θέλει κάμει ο Κύριος εις πάντας τους εχθρούς σας, κατά των οποίων μάχεσθε.
\par 26 Και μετά ταύτα επάταξεν αυτούς ο Ιησούς και εθανάτωσεν αυτούς και εκρέμασεν αυτούς εις πέντε ξύλα· και εκρέμοντο εις τα ξύλα έως εσπέρας.
\par 27 Περί δε την δύσιν του ηλίου, προσέταξεν ο Ιησούς, και κατεβίβασαν αυτούς από των ξύλων και έρριψαν αυτούς εις το σπήλαιον, όπου είχον κρυφθή, και εκύλισαν λίθους μεγάλους εις το στόμα του σπηλαίου, οίτινες μένουσιν έως της σήμερον ημέρας.
\par 28 Και εν εκείνη τη ημέρα εκυρίευσεν ο Ιησούς την Μακκηδά, και επάταξεν εν στόματι μαχαίρας αυτήν και τον βασιλέα αυτής· εξωλόθρευσεν αυτούς και πάσας τας ψυχάς τας εν αυτή· δεν αφήκεν υπόλοιπον· και έκαμεν εις τον βασιλέα της Μακκηδά, καθώς έκαμεν εις τον βασιλέα της Ιεριχώ.
\par 29 Και διέβη ο Ιησούς, και πας ο Ισραήλ μετ' αυτού, εκ Μακκηδά εις Λιβνά και επολέμει την Λιβνά.
\par 30 Και παρέδωκεν ο Κύριος και αυτήν και τον βασιλέα αυτής εις την χείρα του Ισραήλ· και επάταξεν εν στόματι μαχαίρας αυτήν και πάσας τας ψυχάς τας εν αυτή· δεν αφήκεν εν αυτή υπόλοιπον· και έκαμεν εις τον βασιλέα αυτής, καθώς έκαμεν εις τον βασιλέα της Ιεριχώ.
\par 31 Και διέβη ο Ιησούς, και πας ο Ισραήλ μετ' αυτού, εκ Λιβνά εις Λαχείς, και εστρατοπέδευσε κατέναντι αυτής και επολέμει αυτήν.
\par 32 Και παρέδωκεν ο Κύριος την Λαχείς εις την χείρα του Ισραήλ, και εκυρίευσεν αυτήν την δευτέραν ημέραν, και επάταξεν εν στόματι μαχαίρας αυτήν και πάσας τας ψυχάς τας εν αυτή, κατά πάντα όσα έκαμεν εις την Λιβνά.
\par 33 Τότε ανέβη Ωράμ ο βασιλεύς της Γεζέρ διά να βοηθήση την Λαχείς· και ο Ιησούς επάταξεν αυτόν και τον λαόν αυτού, εωσού δεν αφήκεν εις αυτόν υπόλοιπον.
\par 34 Και διέβη ο Ιησούς, και πας ο Ισραήλ μετ' αυτού, εκ Λαχείς εις Εγλών, και εστρατοπέδευσαν κατέναντι αυτής και επολέμουν αυτήν·
\par 35 και εκυρίευσαν αυτήν εν τη ημέρα εκείνη, και επάταξαν αυτήν εν στόματι μαχαίρας· και εξωλόθρευσεν εν εκείνη τη ημέρα πάσας τας ψυχάς τας εν αυτή, κατά πάντα όσα έκαμεν εις την Λαχείς.
\par 36 Και ανέβη ο Ιησούς, και πας ο Ισραήλ μετ' αυτού, εξ Εγλών εις Χεβρών, και επολέμουν αυτήν·
\par 37 και εκυρίευσαν αυτήν και επάταξαν εν στόματι μαχαίρας αυτήν και τον βασιλέα αυτής και πάσας τας πόλεις αυτής και πάσας τας ψυχάς τας εν αυτή· δεν αφήκεν υπόλοιπον· κατά πάντα όσα έκαμεν εις την Εγλών· και εξωλόθρευσεν αυτήν και πάσας τας ψυχάς τας εν αυτή.
\par 38 Και έστρεψεν ο Ιησούς, και πας ο Ισραήλ μετ' αυτού· εις Δεβείρ και επολέμουν αυτήν·
\par 39 και εκυρίευσεν αυτήν και τον βασιλέα αυτής και πάσας τας πόλεις αυτής· και επάταξαν αυτούς εν στόματι μαχαίρας και εξωλόθρευσαν πάσας τας ψυχάς τας εν αυτή, δεν αφήκεν υπόλοιπον· καθώς έκαμεν εις την Χεβρών, ούτως έκαμεν εις την Δεβείρ και εις τον βασιλέα αυτής· και καθώς έκαμεν εις την Λιβνά και εις τον βασιλέα αυτής.
\par 40 Ούτως επάταξεν ο Ιησούς πάσαν την γην την ορεινήν και την μεσημβρινήν και την πεδινήν και την Ασδώθ και πάντας τους βασιλείς αυτών· δεν αφήκεν υπόλοιπον, αλλ' εξωλόθρευσε παν το έχον πνοήν, καθώς προσέταξε Κύριος ο Θεός του Ισραήλ.
\par 41 Και επάταξεν αυτούς ο Ιησούς από Κάδης-βαρνή έως Γάζης, και πάσαν την γην Γεσέν, έως Γαβαών.
\par 42 Και πάντας τούτους τους βασιλείς και την γην αυτών ο Ιησούς εκυρίευσε διά μιας, διότι Κύριος ο Θεός του Ισραήλ επολέμει υπέρ του Ισραήλ.
\par 43 Και επέστρεψεν ο Ιησούς, και πας Ισραήλ μετ' αυτού, εις το στρατόπεδον εις Γάλγαλα.

\chapter{11}

\par 1 Και ως ήκουσεν Ιαβείν ο βασιλεύς της Ασώρ, απέστειλε προς τον Ιωβάβ βασιλέα της Μαδών και προς τον βασιλέα της Σιμβρών και προς τον βασιλέα της Αχσάφ,
\par 2 και προς τους βασιλείς τους προς βορράν, εις την ορεινήν και εις την πεδινήν, κατέναντι της Χιννερώθ, και εις την κοιλάδα και εις Νάφαθ-δωρ προς δυσμάς,
\par 3 και προς τους Χαναναίους τους προς ανατολάς και δυσμάς, και προς τους Αμορραίους και τους Χετταίους και τους Φερεζαίους και τους Ιεβουσαίους τους εν τη ορεινή, και προς τους Ευαίους τους υπό την Αερμών εν τη γη Μισπά.
\par 4 Και εξήλθον, αυτοί και πάντα τα στρατεύματα αυτών μετ' αυτών, λαός πολύς ως η άμμος η παρά το χείλος της θαλάσσης κατά το πλήθος, μεθ' ίππων και αμαξών πολλών σφόδρα.
\par 5 Και συναχθέντες πάντες ούτοι οι βασιλείς, ήλθον και εστρατοπέδευσαν ομού πλησίον των υδάτων Μερώμ, διά να πολεμήσωσι τον Ισραήλ.
\par 6 Και είπε Κύριος προς τον Ιησούν, Μη φοβηθής από προσώπου αυτών· διότι αύριον, περί την ώραν ταύτην, εγώ θέλω παραδώσει αυτούς πάντας πεφονευμένους έμπροσθεν του Ισραήλ· τους ίππους αυτών θέλεις νευροκοπήσει και τας αμάξας αυτών θέλεις κατακαύσει εν πυρί.
\par 7 Και υπήγεν εξαίφνης ο Ιησούς, και πας ο πολεμιστής λαός μετ' αυτού, εναντίον αυτών εις τα ύδατα Μερώμ, και επέπεσον επ' αυτούς.
\par 8 Και παρέδωκεν αυτούς ο Κύριος εις την χείρα του Ισραήλ, και επάταξεν αυτούς και κατεδίωξεν αυτούς έως της μεγάλης Σιδώνος· και έως Μισρεφώθ-μαΐμ και έως της κοιλάδος Μισπά προς ανατολάς· και επάταξαν αυτούς, εωσού δεν αφήκαν εις αυτούς υπόλοιπον.
\par 9 Και έκαμεν ο Ιησούς εις αυτούς καθώς προσέταξεν εις αυτόν ο Κύριος· τους ίππους αυτών ενευροκόπησε και τας αμάξας αυτών κατέκαυσεν εν πυρί.
\par 10 Και έστρεψεν ο Ιησούς κατά τον αυτόν καιρόν και εκυρίευσε την Ασώρ και επάταξε τον βασιλέα αυτής εν μαχαίρα· διότι η Ασώρ ήτο πρότερον η πρωτεύουσα πασών των βασιλειών τούτων.
\par 11 Και πάσας τας ψυχάς τας εν αυτή επάταξαν εν στόματι μαχαίρας και εξωλόθρευσαν αυτούς· δεν έμεινεν ουδέν έχον πνοήν· και την Ασώρ κατέκαυσεν εν πυρί.
\par 12 Και πάσας τας πόλεις των βασιλέων εκείνων και πάντας τους βασιλείς αυτών συνέλαβεν ο Ιησούς και επάταξεν αυτούς εν στόματι μαχαίρας· εξωλόθρευσεν αυτούς, καθώς προσέταξε Μωϋσής ο δούλος του Κυρίου.
\par 13 Πάσας δε τας πόλεις, όσαι έμενον μετά των προχωμάτων αυτών, δεν έκαυσεν αυτάς ο Ισραήλ, εκτός μόνην την Ασώρ κατέκαυσεν ο Ιησούς.
\par 14 Και πάντα τα λάφυρα των πόλεων τούτων και τα κτήνη ελαφυραγώγησαν εις εαυτούς οι υιοί Ισραήλ· τους δε ανθρώπους πάντας επάταξαν εν στόματι μαχαίρας, εωσού εξωλόθρευσαν αυτούς· δεν αφήκαν ουδέν έχον πνοήν.
\par 15 Καθώς προσέταξεν ο Κύριος εις τον Μωϋσήν τον δούλον αυτού, ούτω προσέταξεν ο Μωϋσής τον Ιησούν, και ούτως έκαμεν ο Ιησούς· δεν παρέβη ουδέν εκ πάντων όσα προσέταξεν ο Κύριος εις τον Μωϋσήν.
\par 16 Και εκυρίευσεν ο Ιησούς πάσαν εκείνην την γην, την ορεινήν, και πάσαν την μεσημβρινήν, και πάσαν την γην Γεσέν, και την κοιλάδα και την πεδινήν, και το όρος του Ισραήλ και την κοιλάδα αυτού,
\par 17 από του όρους Αλάκ, του αναβαίνοντος προς την Σηείρ, έως Βάαλ-γαδ εις την κοιλάδα του Λιβάνου, υπό το όρος Αερμών· και συνέλαβε πάντας τους βασιλείς αυτών και επάταξεν αυτούς και εθανάτωσεν αυτούς.
\par 18 Πολύν καιρόν επολέμει ο Ιησούς προς πάντας τούτους τους βασιλείς.
\par 19 Δεν ήτο πόλις ήτις έκαμεν ειρήνην μετά των υιών Ισραήλ, εκτός των Ευαίων των κατοικούντων εν Γαβαών· πάσας εκυρίευσαν εν πολέμω·
\par 20 διότι παρά Κυρίου έγεινε το να σκληρυνθώσιν αι καρδίαι αυτών, να έλθωσιν εις μάχην κατά του Ισραήλ, διά να εξολοθρευθώσι, να μη γείνη εις αυτούς έλεος, αλλά να εξαφανισθώσι, καθώς ο Κύριος προσέταξεν εις τον Μωϋσήν.
\par 21 Και ήλθεν ο Ιησούς κατ' εκείνον τον καιρόν, και ηφάνισε τους Ανακείμ από των ορέων, από Χεβρών, από Δεβείρ, από Ανάβ και από πάντων των ορέων του Ιούδα και από πάντων των ορέων του Ισραήλ· εξωλόθρευσεν αυτούς ο Ιησούς μετά των πόλεων αυτών.
\par 22 Δεν έμεινον Ανακείμ εν τη γη των υιών Ισραήλ· μόνον εν τη Γάζη, εν τη Γαθ και εν τη Αζώτω έμεινον.
\par 23 Και εκυρίευσεν ο Ιησούς πάσαν την γην κατά πάντα όσα είπεν ο Κύριος προς τον Μωϋσήν· και έδωκεν αυτήν ο Ιησούς εις τον Ισραήλ κληρονομίαν, κατά τον διαμερισμόν αυτών εις τας φυλάς αυτών. Και η γη ησύχασεν από του πολέμου.

\chapter{12}

\par 1 Ούτοι δε είναι οι βασιλείς της γης, τους οποίους επάταξαν οι υιοί Ισραήλ και κατεκυρίευσαν την γην αυτών, εις το πέραν του Ιορδάνου, προς ανατολάς ηλίου, από του ποταμού Αρνών έως του όρους Αερμών, και πάσαν την πεδινήν προς ανατολάς·
\par 2 τον Σηών βασιλέα των Αμορραίων, τον κατοικούντα εν Εσεβών, τον δεσπόζοντα από Αροήρ της παρά το χείλος του ποταμού Αρνών, και το μέσον του ποταμού, και το ήμισυ της Γαλαάδ έως του ποταμού Ιαβόκ, του ορίου των υιών Αμμών·
\par 3 και από της πεδινής έως της θαλάσσης Χιννερώθ προς ανατολάς, και έως της θαλάσσης της πεδιάδος, της αλμυράς θαλάσσης προς ανατολάς, κατά την οδόν την προς Βαιθ-ιεσιμώθ, και από του μεσημβρινού μέρους υπό την Ασδώθ-φασγά·
\par 4 και τα όρια του Ωγ, βασιλέως της Βασάν, του εναπολειφθέντος εκ των γιγάντων και κατοικούντος εν Ασταρώθ και εν Εδρεΐ·
\par 5 όστις εξουσίαζεν εν τω όρει Αερμών και εν Σαλχά και εν πάση τη Βασάν, έως των ορίων των Γεσσουριτών και των Μααχαθιτών, και επί του ημίσεως της Γαλαάδ, ορίου του Σηών βασιλέως της Εσεβών.
\par 6 Τούτους επάταξεν ο Μωϋσής ο δούλος του Κυρίου, και οι υιοί Ισραήλ· και έδωκε την γην αυτών ο Μωϋσής ο δούλος του Κυρίου κληρονομίαν εις τους Ρουβηνίτας και εις τους Γαδίτας και εις το ήμισυ της φυλής Μανασσή.
\par 7 Και ούτοι είναι οι βασιλείς της γης, τους οποίους επάταξεν ο Ιησούς και οι υιοί Ισραήλ, εντεύθεν του Ιορδάνου προς δυσμάς, από Βάαλ-γαδ εν τη κοιλάδι του Λιβάνου, και έως του όρους Αλάκ, του αναβαίνοντος εις Σηείρ· και έδωκεν αυτήν ο Ιησούς εις τας φυλάς του Ισραήλ κληρονομίαν, κατά τον διαμερισμόν αυτών·
\par 8 εις τα όρη και εις τας κοιλάδας και εις τας πεδιάδας και εις Ασδώθ και εις την έρημον και εις το μέρος το μεσημβρινόν· τους Χετταίους, τους Αμορραίους και τους Χαναναίους, τους Φερεζαίους, τους Ευαίους και τους Ιεβουσαίους·
\par 9 τον βασιλέα της Ιεριχώ, ένα· τον βασιλέα της Γαί, της παρά την Βαιθήλ, ένα·
\par 10 τον βασιλέα της Ιερουσαλήμ, ένα· τον βασιλέα της Χεβρών, ένα·
\par 11 τον βασιλέα της Ιαρμούθ, ένα· τον βασιλέα της Λαχείς, ένα.
\par 12 τον βασιλέα της Εγλών, ένα· τον βασιλέα της Γεζέρ, ένα·
\par 13 τον βασιλέα της Δεβείρ, ένα· τον βασιλέα της Γεδέρ, ένα·
\par 14 τον βασιλέα της Ορμά, ένα· τον βασιλέα της Αράδ, ένα.
\par 15 τον βασιλέα της Λιβνά, ένα· τον βασιλέα της Οδολλάμ, ένα·
\par 16 τον βασιλέα της Μακκηδά, ένα· τον βασιλέα της Βαιθήλ, ένα·
\par 17 τον βασιλέα της Θαπφουά, ένα· τον βασιλέα της Εφέρ, ένα·
\par 18 τον βασιλέα της Αφέκ, ένα· τον βασιλέα της Λασαρών, ένα.
\par 19 τον βασιλέα της Μαδών, ένα· τον βασιλέα της Ασώρ, ένα.
\par 20 τον βασιλέα της Σιμβρών-μερών, ένα· τον βασιλέα της Αχσάφ, ένα·
\par 21 τον βασιλέα της Θαανάχ, ένα· τον βασιλέα της Μεγιδδώ, ένα·
\par 22 τον βασιλέα της Κέδες, ένα· τον βασιλέα της Ιοκνεάμ εν Καρμέλ, ένα·
\par 23 τον βασιλέα της Δωρ εν Νάφαθ-δωρ, ένα· τον βασιλέα των εθνών εν Γαλγάλοις, ένα·
\par 24 τον βασιλέα της Θερσά, ένα. Πάντες οι βασιλείς, τριάκοντα και εις.

\chapter{13}

\par 1 Ο δε Ιησούς ήτο γέρων, προβεβηκώς εις την ηλικίαν· και είπε προς αυτόν ο Κύριος, Συ είσαι γέρων, προβεβηκώς εις την ηλικίαν, μένει δε έτι πολλή γη να κυριευθή.
\par 2 Αύτη είναι η γη, η μένουσα έτι· πάντα τα όρια των Φιλισταίων και πάσα η Γεσσουρί,
\par 3 από Σιώρ, του κατέναντι της Αιγύπτου, έως των ορίων της Ακκαρών προς βορράν, τα οποία αριθμούνται εις τους Χαναναίους. αι πέντε ηγεμονίαι των Φιλισταίων, των Γαζαίων, των Αζωτίων, των Ασκαλωνιτών, των Γετθαίων και των Ακκαρωνιτών, και η των Αυϊτών·
\par 4 από μεσημβρίας, πάσα η γη των Χαναναίων, και η Μεαρά η των Σιδωνίων έως Αφέκ, έως των ορίων των Αμορραίων·
\par 5 και η γη των Γιβλιτών και πας ο Λίβανος, προς ανατολάς ηλίου, από Βάαλ-γαδ υπό το όρος Αερμών, έως της εισόδου Αιμάθ·
\par 6 πάντες οι κάτοικοι της ορεινής, από του Λιβάνου έως Μισρεφώθ-μαΐμ, πάντες οι Σιδώνιοι· τούτους εγώ θέλω εξολοθρεύσει απ' έμπροσθεν των υιών Ισραήλ· συ δε διαμοίρασον αυτήν διά κλήρων εις τους Ισραηλίτας, καθώς προσέταξα εις σέ·
\par 7 τώρα λοιπόν διαμοίρασον την γην ταύτην εις κληρονομίαν εις τας εννέα φυλάς και εις το ήμισυ της φυλής του Μανασσή.
\par 8 Οι Ρουβηνίται και οι Γαδίται, μετά του άλλου ημίσεως αυτής, έλαβον την κληρονομίαν αυτών, την οποίαν έδωκεν εις αυτούς ο Μωϋσής, πέραν του Ιορδάνου προς ανατολάς, καθώς ο Μωϋσής ο δούλος του Κυρίου έδωκεν εις αυτούς,
\par 9 από της Αροήρ, της παρά το χείλος του ποταμού Αρνών, και την πόλιν την εν τω μέσω του ποταμού· και πάσαν την πεδινήν Μεδεβά έως Δαιβών,
\par 10 και πάσας τας πόλεις του Σηών βασιλέως των Αμορραίων, του βασιλεύοντος εν Εσεβών, έως των ορίων των υιών Αμμών,
\par 11 και την Γαλαάδ, και τα όρια των Γεσσουριτών και των Μααχαθιτών, και παν το όρος Αερμών, και πάσαν την Βασάν έως Σαλχά,
\par 12 άπαν το βσσίλειον του Ωγ εν Βασάν, του βασιλεύοντος εν Ασταρώθ και εν Εδρεΐ, όστις εναπελείφθη από των υπολοίπων των γιγάντων· διότι τούτους ο Μωϋσής επάταξε και εξωλόθρευσεν αυτούς.
\par 13 Τους Γεσσουρίτας όμως και τους Μααχαθίτας οι υιοί Ισραήλ δεν εξωλόθρευσαν, αλλά κατοικούσιν οι Γεσσουρίται και οι Μααχαθίται μεταξύ του Ισραήλ έως της σήμερον.
\par 14 Μόνον εις την φυλήν του Λευΐ δεν έδωκε κληρονομίαν· αι διά πυρός γινόμεναι θυσίαι Κυρίου του Θεού του Ισραήλ είναι η κληρονομία αυτών, καθώς είπε προς αυτούς.
\par 15 Και έδωκεν ο Μωϋσής εις την φυλήν των υιών Ρουβήν κληρονομίαν, κατά τας συγγενείας αυτών.
\par 16 και τα όρια αυτών ήσαν από Αροήρ, της παρά το χείλος του ποταμού Αρνών, και η πόλις εν τω μέσω του ποταμού, και πάσα η πεδινή έως Μεδεβά,
\par 17 η Εσεβών και πάσαι αι πόλεις αυτής, αι εν τη πεδινή· Δαιβών, και η Βαμώθ-βαάλ, και η Βαιθ-βάαλ-μεών,
\par 18 και η Ιασσά, και η Κεδημώθ, και η Μηφαάθ,
\par 19 και η Κιριαθαΐμ, και η Σιβμά, και η Ζαρέθ-σαάρ εν τω όρει της κοιλάδος,
\par 20 και η Βαιθ-φεγώρ, και η Ασδώθ-φασγά, και η Βαιθ-ιεσιμώθ,
\par 21 και πάσαι αι πόλεις της πεδινής, και άπαν το βασίλειον του Σηών βασιλέως των Αμορραίων, του βασιλεύοντος εν Εσεβών, τον οποίον επάταξεν ο Μωϋσής, αυτόν και τους ηγεμόνας της Μαδιάμ, τον Ευΐ και τον Ρεκέμ και τον Σούρ και τον Ουρ και τον Ρεβά άρχοντας του Σηών, κατοικούντας την γην.
\par 22 Και τον Βαλαάμ· τον υιόν του Βεώρ, τον μάντιν, οι υιοί Ισραήλ εθανάτωσαν εν μαχαίρα μεταξύ των φονευθέντων υπ' αυτών.
\par 23 Και των υιών Ρουβήν, ο Ιορδάνης ήτο το όριον αυτών. Αύτη είναι η κληρονομία των υιών Ρουβήν κατά τας συγγενείας αυτών, αι πόλεις και αι κώμαι αυτών.
\par 24 Και έδωκεν ο Μωϋσής κληρονομίαν εις την φυλήν Γαδ, εις τους υιούς Γαδ, κατά τας συγγενείας αυτών·
\par 25 και το όριον αυτών ήτο η Ιαζήρ, και πάσαι αι πόλεις της Γαλαάδ και το ήμισυ της γης των υιών Αμμών, έως της Αροήρ, της κατέναντι Ραββά,
\par 26 και από Εσεβών μέχρι Ραμάθ-μισπά και Βετονίμ, και από Μαχαναΐμ έως των ορίων της Δεβείρ,
\par 27 και εν τη κοιλάδι, Βαιθ-αράμ και Βαιθ-νιμρά και Σοκχώθ και Σαφών, το επίλοιπον του βασιλείου του Σηών βασιλέως της Εσεβών, και ο Ιορδάνης το όριον έως του άκρου της θαλάσσης Χιννερώθ πέραν του Ιορδάνου προς ανατολάς.
\par 28 Αύτη είναι η κληρονομία των υιών Γαδ κατά τας συγγενείας αυτών, αι πόλεις και αι κώμαι αυτών.
\par 29 Και έδωκεν ο Μωϋσής κληρονομίαν εις το ήμισυ της φυλής Μανασσή· και έγεινε κτήμα εις το ήμισυ της φυλής των υιών Μανασσή κατά τας συγγενείας αυτών.
\par 30 Και το όριον αυτών ήτο από Μαχαναΐμ, πάσα η Βασάν, άπαν το βασίλειον του Ωγ βασιλέως της Βασάν και πάσαι αι κωμοπόλεις του Ιαείρ, αι εν Βασάν, εξήκοντα πόλεις·
\par 31 και το ήμισυ της Γαλαάδ και Ασταρώθ και Εδρεΐ, πόλεις του βασιλείου του Ωγ εκ Βασάν, εδόθησαν εις τους υιούς Μαχείρ υιού Μανασσή, εις το ήμισυ των υιών Μαχείρ κατά τας συγγενείας αυτών.
\par 32 Ούτοι είναι οι τόποι, τους οποίους εκληροδότησεν ο Μωϋσής εις τας πεδιάδας Μωάβ, εις το πέραν του Ιορδάνου, πλησίον της Ιεριχώ, προς ανατολάς.
\par 33 Εις δε την φυλήν του Λευΐ δεν έδωκε κληρονομίαν ο Μωϋσής. Κύριος ο Θεός του Ισραήλ, αυτός ήτο η κληρονομία αυτών, καθώς είπε προς αυτούς.

\chapter{14}

\par 1 Και ούτοι είναι οι τόποι, τους οποίους οι υιοί Ισραήλ εκληρονόμησαν εν τη γη Χαναάν, τους οποίους εκληροδότησαν εις αυτούς Ελεάζαρ ο ιερεύς και Ιησούς ο υιός του Ναυή και οι αρχηγοί των πατριών των φυλών των υιών Ισραήλ.
\par 2 Διά κλήρου έγεινεν η κληρονομία των εννέα τούτων φυλών και της ημισείας φυλής, καθώς προσέταξεν ο Κύριος διά του Μωϋσέως.
\par 3 Διότι ο Μωϋσής είχε δώσει την κληρονομίαν των δύο φυλών και της ημισείας φυλής από του πέραν του Ιορδάνου· εις τους Λευΐτας όμως δεν έδωκε κληρονομίαν μεταξύ αυτών.
\par 4 Διότι οι υιοί Ιωσήφ ήσαν δύο φυλαί, του Μανασσή και του Εφραΐμ· και δεν έδωκαν εις τους Λευΐτας μερίδιον εν τη γη ειμή πόλεις διά να κατοικώσι, μετά των προαστείων αυτών, διά τα κτήνη αυτών και διά την περιουσίαν αυτών.
\par 5 Καθώς προσέταξε Κύριος εις τον Μωϋσήν, ούτως έκαμον οι υιοί Ισραήλ, και διεμοίρασαν την γην.
\par 6 Και προσήλθον οι υιοί Ιούδα προς τον Ιησούν εις Γάλγαλα, και είπε προς αυτόν Χάλεβ ο υιός του Ιεφοννή ο Κενεζαίος, Συ εξεύρεις τον λόγον τον οποίον ελάλησεν ο Κύριος προς τον Μωϋσήν, τον άνθρωπον του Θεού, περί εμού και σου εν Κάδης-βαρνή·
\par 7 ήμην τεσσαράκοντα ετών ηλικίας, ότε με απέστειλεν ο Μωϋσής ο δούλος του Κυρίου από Κάδης-βαρνή διά να κατασκοπεύσω την γήν· και απήγγειλα προς αυτόν λόγον, όστις ήτο εν τη καρδία μου·
\par 8 οι αδελφοί μου όμως, οι συναναβάντες μετ' εμού, ενέκρωσαν την καρδίαν του λαού· αλλ' εγώ ηκολούθησα εντελώς Κύριον τον Θεόν μου·
\par 9 και ώμοσεν ο Μωϋσής την ημέραν εκείνην λέγων, Εξάπαντος η γη, την οποίαν επάτησαν οι πόδες σου, θέλει είσθαι κληρονομία ιδική σου και των υιών σου διαπαντός· διότι εντελώς ηκολούθησας Κύριον τον Θεόν μου·
\par 10 και τώρα, ιδού, ο Κύριος με εφύλαξε ζώντα, καθώς είπε, τα τεσσαράκοντα πέντε ταύτα έτη, αφ' ης ημέρας ελάλησεν ο Κύριος τον λόγον τούτον προς τον Μωϋσήν, ότε ο Ισραήλ επορεύετο εν τη ερήμω· και τώρα, ιδού, εγώ είμαι σήμερον ογδοήκοντα πέντε ετών ηλικίας·
\par 11 έτι και την σήμερον είμαι δυνατός, καθώς την ημέραν ότε με απέστειλεν ο Μωϋσής· ως ήτο τότε η δύναμίς μου διά πόλεμον και διά να εξέρχωμαι και διά να εισέρχωμαι·
\par 12 τώρα λοιπόν δος μοι το όρος τούτο, περί του οποίου ελάλησεν ο Κύριος την ημέραν εκείνην· διότι συ ήκουσας την ημέραν εκείνην, ότι είναι εκεί Ανακείμ και πόλεις μεγάλαι ωχυρωμέναι· εάν ο Κύριος ήναι μετ' εμού, εγώ θέλω δυνηθή να εκδιώξω αυτούς, καθώς είπεν ο Κύριος.
\par 13 Και ευλόγησεν αυτόν ο Ιησούς και έδωκεν εις τον Χάλεβ τον υιόν του Ιεφοννή την Χεβρών εις κληρονομίαν.
\par 14 Διά τούτο η Χεβρών αποκατέστη κληρονομία του Χάλεβ υιού του Ιεφοννή του Κενεζαίου έως της σήμερον, διότι εντελώς ηκολούθησε Κύριον τον Θεόν του Ισραήλ.
\par 15 το δε όνομα της Χεβρών πρότερον ήτο Κιριάθ-αρβά· ήτο δε ο Αρβά άνθρωπος μέγας μεταξύ των Ανακείμ. Και η γη ησύχασεν από του πολέμου.

\chapter{15}

\par 1 Ο δε κλήρος της φυλής των υιών Ιούδα κατά τας συγγενείας αυτών ήτο εις τα όρια της Ιδουμαίας· η έρημος Σιν η προς νότον ήτο το άκρον το μεσημβρινόν.
\par 2 Και τα μεσημβρινά αυτών όρια ήσαν από των παραλίων της αλμυράς θαλάσσης, από του κόλπου του βλέποντος προς μεσημβρίαν·
\par 3 και εξετείνοντο προς το μεσημβρινόν μέρος εις την ανάβασιν Ακραββίμ, και διέβαινον εις Σιν και ανέβαινον από μεσημβρίας εις Κάδης-βαρνή, και διέβαινον την Εσρών και ανέβαινον εις Αδδάρ και έστρεφον προς Καρκαά·
\par 4 και διέβαινον εις Ασμών και εξήρχοντο έως του χειμάρρου της Αιγύπτου, και ετελείονον τα όρια εις την θάλασσαν· ταύτα θέλουσιν είσθαι τα μεσημβρινά όριά σας.
\par 5 Το δε ανατολικόν όριον ήτο η θάλασσα η αλμυρά, έως άκρου του Ιορδάνου. Και το όριον κατά το βόρειον μέρος ήρχετο από του κόλπου της θαλάσσης κατά το άκρον του Ιορδάνου·
\par 6 και ανέβαινε το όριον έως Βαιθ-ογλά, και διέβαινεν από βορρά της Βαιθ-αραβά· και ανέβαινε το όριον έως του λίθου του Βοάν υιού του Ρουβήν·
\par 7 και ανέβαινε το όριον προς Δεβείρ από της κοιλάδος Αχώρ, και εξετείνετο προς βορράν βλέπον εις Γάλγαλα, την κατέναντι της αναβάσεως Αδουμμίμ, ήτις είναι προς το μεσημβρινόν του ποταμού· έπειτα διέβαινε το όριον επί τα ύδατα του Εν-σεμές, και εξήρχετο εις Εν-ρωγήλ·
\par 8 και ανέβαινε το όριον διά της φάραγγος του υιού του Εννόμ κατά το μεσημβρινόν πλάγιον της Ιεβούς· αύτη είναι η Ιερουσαλήμ· και ανέβαινε το όριον εις την κορυφήν του όρους, του κατέναντι της φάραγγος Εννόμ προς δυσμάς, ήτις είναι εις το τέλος της κοιλάδος των Ραφαείμ προς βορράν·
\par 9 και διέβαινε το όριον από της κορυφής του όρους έως της πηγής των υδάτων Νεφθωά, και εξήρχετο εις τας κωμοπόλεις του όρους Εφρών· και διευθύνετο το όριον εις Βααλά, ήτις είναι η Κιριάθ-ιαρείμ·
\par 10 και έστρεφε το όριον από Βααλά προς δυσμάς εις το όρος Σηείρ, και διέβαινεν εις το πλάγιον του όρους Ιαρείμ, όπου είναι η Χασαλών, προς βορράν· και κατέβαινεν εις την Βαιθ-σεμές και διέβαινεν εις Θαμνά·
\par 11 έπειτα εξήρχετο το όριον εις το πλάγιον της Ακκαρών προς βορράν· και διευθύνετο το όριον εις Σικρών, και διέβαινεν εις το όρος της Βααλά, και εξήρχετο εις Ιαβνήλ, και έκαμνε το όριον διέξοδον εις την θάλασσαν.
\par 12 Και το όριον το δυτικόν ήτο η θάλασσα η μεγάλη και τα παράλια. Ταύτα είναι τα όρια των υιών Ιούδα κύκλω, κατά τας συγγενείας αυτών.
\par 13 Και εις τον Χάλεβ τον υιόν του Ιεφοννή έδωκε μερίδιον μεταξύ των υιών Ιούδα, κατά την προσταγήν του Κυρίου την προς τον Ιησούν, την πόλιν του Αρβά πατρός του Ανάκ, ήτις είναι η Χεβρών.
\par 14 Και εξεδίωξεν εκείθεν ο Χάλεβ τους τρεις υιούς του Ανάκ, τον Σεσαΐ και τον Αχιμάν και τον Θαλμαΐ, τους υιούς του Ανάκ.
\par 15 Και εκείθεν ανέβη επί τους κατοίκους της Δεβείρ· το δε όνομα της Δεβείρ πρότερον ήτο Κιριάθ-σεφέρ.
\par 16 Και είπεν ο Χάλεβ, Όστις πατάξη την Κιριάθ-σεφέρ και κυριεύση αυτήν, θέλω δώσει εις τούτον Αχσάν την θυγατέρα μου εις γυναίκα.
\par 17 Και εκυρίευσεν αυτήν Γοθονιήλ ο υιός του Κενέζ αδελφός του Χάλεβ· και έδωκεν εις αυτόν Αχσάν την θυγατέρα αυτού εις γυναίκα.
\par 18 Και αυτή, ότε απήρχετο, παρεκίνησεν αυτόν να ζητήση παρά του πατρός αυτής αγρόν· και κατέβη από του όνου, και είπε προς αυτήν ο Χάλεβ, Τι θέλεις;
\par 19 Η δε είπε, Δος μοι ευλογίαν· επειδή έδωκας εις εμέ γην μεσημβρινήν, δος μοι και πηγάς υδάτων. Και έδωκεν εις αυτήν τας άνω πηγάς και τας κάτω πηγάς.
\par 20 Αύτη είναι η κληρονομία της φυλής των υιών Ιούδα κατά τας συγγενείας αυτών.
\par 21 Και ήσαν αι έσχαται πόλεις της φυλής των υιών Ιούδα προς τα όρια της Εδώμ προς μεσημβρίαν, Καβσεήλ, και Εδέρ, και Ιαγούρ,
\par 22 και Κινά, και Διμωνά, και Αδαδά,
\par 23 και Κέδες, και Ασώρ, και Ιθνάν,
\par 24 Ζιφ, και Τελέμ, και Βαλώθ,
\par 25 και Ασώρ, Αδαττά, και Κιριώθ-εσρών, η και Ασώρ,
\par 26 Αμάμ, και Σεμά, και Μωλαδά,
\par 27 και Ασάρ-γαδδά, και Εσεμών, και Βαιθ-φαλέθ,
\par 28 και Ασάρ-σουάλ, και Βηρ-σαβεέ, και Βιζιοθιά,
\par 29 Βααλά, και Ιείμ, και Ασέμ,
\par 30 και Ελθωλάδ, και Χεσίλ, και Ορμά,
\par 31 και Σικλάγ, και Μαδμαννά, και Σανσαννά,
\par 32 και Λεβαώθ, και Σιλεείμ, και Αείν, και Ριμμών· πάσαι αι πόλεις είκοσι εννέα, και αι κώμαι αυτών.
\par 33 Εν τη πεδινή ήσαν Εσθαόλ, και Σαραά, και Ασνά,
\par 34 και Ζανωά, και Εν-γαννίμ, Θαπφουά, και Ηνάμ,
\par 35 Ιαρμούθ, και Οδολλάμ, Σωχώ, και Αζηκά,
\par 36 και Σαγαρείμ, και Αδιθαείμ, και Γεδηρά, και αι επαύλεις αυτών, πόλεις δεκατέσσαρες, και αι κώμαι αυτών.
\par 37 Σενάν, και Αδασά, και Μάγδαλ-γαδ,
\par 38 και Διλαάν, και Μισπά, και Ιοκθεήλ,
\par 39 Λαχείς, και Βασκάθ, και Εγλών,
\par 40 και Χαββών, και Λαμάς, και Χιθλείς,
\par 41 και Γεδηρώθ, Βαιθ-δαγών, και Νααμά, και Μακκηδά, πόλεις δεκαέξ, και αι κώμαι αυτών.
\par 42 Λιβνά, και Εθέρ, και Ασάν,
\par 43 και Ιεφθά, και Ασνά, και Νεσίβ,
\par 44 και Κεειλά, και Αχζίβ, και Μαρησά, πόλεις εννέα, και αι κώμαι αυτών.
\par 45 Ακκαρών, και αι κωμοπόλεις αυτής, και αι κώμαι αυτής·
\par 46 από της Ακκαρών έως της θαλάσσης, πάσαι αι πλησίον της Αζώτου, και αι κώμαι αυτών.
\par 47 Άζωτος, αι κωμοπόλεις αυτής και αι κώμαι αυτής, Γάζα, αι κωμοπόλεις αυτής και αι κώμαι αυτής, έως του χειμάρρου της Αιγύπτου, και η θάλασσα η μεγάλη το όριον.
\par 48 Και εν τη ορεινή, Σαμείρ, και Ιαθείρ, και Σωχώ,
\par 49 και Δαννά, και Κιριάθ-σαννά, ήτις είναι η Δεβείρ,
\par 50 και Ανάβ, και Εσθεμώ, και Ανείμ,
\par 51 και Γεσέν, και Ωλών, και Γιλώ, πόλεις ένδεκα, και αι κώμαι αυτών.
\par 52 Αράβ, και Δουμά, και Εσάν,
\par 53 και Ιανούμ, και Βαιθ-θαπφουά, και Αφεκά,
\par 54 και Χουματά, και Κιριάθ-αρβά, ήτις είναι η Χεβρών, και Σιώρ, πόλεις εννέα, και αι κώμαι αυτών.
\par 55 Μαών, Καρμέλ, και Ζιφ, και Ιουτά,
\par 56 και Ιεζραέλ, και Ιοκδεάμ, και Ζανωά,
\par 57 Ακαΐν, Γαβαά, και Θαμνά, πόλεις δέκα, και αι κώμαι αυτών.
\par 58 Αλούλ, Βαιθ-σούρ, και Γεδώρ,
\par 59 και Μααράθ, και Βαιθ-ανώθ, και Ελτεκών, πόλεις εξ, και αι κώμαι αυτών.
\par 60 Κιριάθ-βαάλ, ήτις είναι η Κιριάθ-ιαρείμ, και Ραββά, πόλεις δύο, και αι κώμαι αυτών.
\par 61 Εν τη ερήμω, Βαιθ-αραβά, Μιδδίν, και Σεχαχά,
\par 62 και Νιβσάν, και η πόλις του άλατος, και Εν-γαδδί, πόλεις εξ, και αι κώμαι αυτών.
\par 63 τους δε Ιεβουσαίους, τους κατοικούντας την Ιερουσαλήμ, οι υιοί Ιούδα δεν ηδυνήθησαν να εκδιώξωσιν αυτούς· αλλά κατοικούσιν οι Ιεβουσαίοι μετά των υιών Ιούδα εν τη Ιερουσαλήμ έως της ημέρας ταύτης.

\chapter{16}

\par 1 Και έπεσεν ο κλήρος των υιών Ιωσήφ από του Ιορδάνου πλησίον της Ιεριχώ, έως των υδάτων της Ιεριχώ κατά ανατολάς, προς την έρημον την αναβαίνουσαν από της Ιεριχώ διά του όρους Βαιθήλ,
\par 2 και εκτείνεται από Βαιθήλ έως Λούζ, και διαβαίνει διά των ορίων του Αρχί-αταρώθ,
\par 3 και καταβαίνει εκ δυσμών εις τα όρια του Ιαφλαιτί, έως των ορίων της κάτω Βαιθ-ωρών, και έως Γεζέρ, και εξέρχεται εις την θάλασσαν.
\par 4 Και έλαβον την κληρονομίαν αυτών οι υιοί Ιωσήφ, ο Μανασσής και ο Εφραΐμ.
\par 5 Και τα όρια των υιών Εφραΐμ κατά τας συγγενείας αυτών ήσαν ταύτα· τα όρια της κληρονομίας αυτών προς το ανατολικόν μέρος ήσαν η Αταρώθ-αδάρ, έως της άνω Βαιθ-ωρών·
\par 6 και εξετείνοντο τα όρια προς την θάλασσαν εις Μιχμεθά κατά το βόρειον· και τα όρια έστρεφον κατά το ανατολικόν έως Ταανάθ-σηλώ, και εκείθεν διέβαινον κατά ανατολάς εις Ιανωχά·
\par 7 και κατέβαινον από Ιανωχά εις Αταρώθ, και εις Νααράθ, και ήρχοντο εις την Ιεριχώ, και εξήρχοντο εις τον Ιορδάνην·
\par 8 τα όρια εξηκολούθησαν από Θαπφουά προς δυσμάς έως του χειμάρρου Κανά, και είχον την διέξοδον αυτών προς την θάλασσαν. Αύτη είναι η κληρονομία της φυλής των υιών Εφραΐμ κατά τας συγγενείας αυτών.
\par 9 Ήσαν και πόλεις κεχωρισμέναι διά τους υιούς Εφραΐμ μεταξύ της κληρονομίας των υιών Μανασσή, πάσαι αι πόλεις και αι κώμαι αυτών.
\par 10 Και δεν εξεδίωξαν τους Χαναναίους τους κατοικούντας εις Γεζέρ· αλλ' οι Χαναναίοι κατοικούσι μεταξύ των Εφραϊμιτών έως της ημέρας ταύτης, και έγειναν δούλοι υποτελείς.

\chapter{17}

\par 1 Ήτο και κλήρος διά την φυλήν του Μανασσή, διότι αυτός ήτο ο πρωτότοκος του Ιωσήφ, διά τον Μαχείρ τον πρωτότοκον του Μανασσή, τον πατέρα του Γαλαάδ· επειδή αυτός ήτο ανήρ πολεμιστής, διά τούτο έλαβε την Γαλαάδ και την Βασάν.
\par 2 Ήτο κλήρος και διά τους λοιπούς υιούς Μανασσή κατά τας συγγενείας αυτών, διά τους υιούς του Αβί-εζέρ, και διά τους υιούς του Χελέκ, και διά τους υιούς του Ασριήλ, και διά τους υιούς του Συχέμ, και διά τους υιούς του Εφέρ, και διά τους υιούς του Σεμιδά. Ταύτα ήσαν τα αρσενικά τέκνα του Μανασσή υιού του Ιωσήφ, κατά τας συγγενείας αυτών.
\par 3 Ο Σαλπαάδ όμως, ο υιός του Εφέρ, υιού του Γαλαάδ, υιού του Μαχείρ, υιού του Μανασσή, δεν είχεν υιούς, αλλά θυγατέρας· και ταύτα είναι τα ονόματα των θυγατέρων αυτού, Μααλά και Νουά, Αγλά, Μελχά και Περσά.
\par 4 Και προσελθούσαι ενώπιον Ελεάζαρ του ιερέως, και ενώπιον Ιησού υιού του Ναυή και ενώπιον των αρχόντων, είπον, Ο Κύριος προσέταξεν εις τον Μωϋσήν να δώση εις ημάς κληρονομίαν μεταξύ των αδελφών ημών. Και εδόθη εις αυτάς κατά την προσταγήν του Κυρίου κληρονομία μεταξύ των αδελφών του πατρός αυτών.
\par 5 Και έπεσον εις τον Μανασσή δέκα μερίδια, εκτός της γης Γαλαάδ και Βασάν, των πέραν του Ιορδάνου·
\par 6 διότι αι θυγατέρες του Μανασσή έλαβον κληρονομίαν μεταξύ των υιών αυτού· και οι επίλοιποι υιοί του Μανασσή έλαβον την γην Γαλαάδ.
\par 7 Και τα όρια του Μανασσή ήσαν από Ασήρ έως της Μιχμεθά κειμένης απέναντι της Συχέμ· και εξετείνοντο τα όρια κατά τα δεξιά, έως των κατοίκων της Εν-θαπφουά.
\par 8 Ο δε Μανασσής είχε την γην Θαπφουά· η δε Θαπφουά επί των ορίων του Μανασσή ανήκεν εις τους υιούς Εφραΐμ.
\par 9 Και κατέβαινε το όριον έως του χειμάρρου Κανά, προς μεσημβρίαν του χειμάρρου· αύται αι πόλεις του Εφραΐμ ήσαν μεταξύ των πόλεων του Μανασσή· και το όριον του Μανασσή ήτο προς βορράν του χειμάρρου, και η διέξοδος αυτού προς την θάλασσαν.
\par 10 προς μεσημβρίαν ήτο του Εφραΐμ, και προς βορράν του Μανασσή· και η θάλασσα ήτο το όριον αυτού· και ηνόνοντο προς βορράν με το του Ασήρ, και προς ανατολάς με το του Ισσάχαρ.
\par 11 Και είχεν ο Μανασσής, εν τη γη Ισσάχαρ και Ασήρ, την Βαιθ-σαν και τας κωμοπόλεις αυτής, και την Ιβλεάμ και τας κωμοπόλεις αυτής, και τους κατοίκους της Δωρ και τας κωμοπόλεις αυτής, και τους κατοίκους της Εν-δωρ και τας κωμοπόλεις αυτής, και τους κατοίκους της Θαανάχ και τας κωμοπόλεις αυτής, και τους κατοίκους της Μεγιδδώ και τας κωμοπόλεις αυτής, τρεις επαρχίας.
\par 12 Οι δε υιοί Μανασσή δεν ηδυνήθησαν να εκδιώξωσι τους κατοίκους των πόλεων τούτων, αλλ' οι Χαναναίοι επέμενον να κατοικώσιν εν τη γη εκείνη.
\par 13 Αφού όμως υπερίσχυσαν οι υιοί Ισραήλ, καθυπέβαλον τους Χαναναίους εις φόρον, πλην δεν εξεδίωξαν αυτούς ολοκλήρως.
\par 14 Και είπον προς τον Ιησούν οι υιοί Ιωσήφ λέγοντες, Διά τι έδωκας εις ημάς ένα μόνον κλήρον και μίαν μερίδα να κληρονομήσωμεν, ενώ είμεθα λαός πολύς, καθώς ο Κύριος ευλόγησεν ημάς έως του νυν;
\par 15 Και είπε προς αυτούς ο Ιησούς, Εάν ήσθε λαός πολύς, ανάβητε εις τον δρυμόν και κόψατε μέρος αυτού δι' εαυτούς εν τη γη των Φερεζαίων και των Ραφαείμ, εάν το όρος Εφραΐμ ήναι παραπολύ στενόχωρον διά σας.
\par 16 Και είπον οι υιοί Ιωσήφ, Δεν αρκεί εις ημάς το όρος· και πάντες οι Χαναναίοι οι κατοικούντες την γην της κοιλάδος έχουσιν αμάξας σιδηράς, και οι της Βαιθ-σαν και των κωμοπόλεων αυτής και οι της κοιλάδος Ιεζραέλ.
\par 17 Και είπεν ο Ιησούς προς τον οίκον Ιωσήφ, προς τον Εφραΐμ και προς τον Μανασσή, λέγων, Συ είσαι λαός πολύς· και εις δύναμιν μεγάλην· συ δεν θέλεις έχει ένα μόνον κλήρον·
\par 18 αλλά το όρος θέλει είσθαι ιδικόν σου· επειδή είναι δρυμός, και θέλεις κατακόψει αυτόν· και έως των άκρων αυτού θέλει είσθαι ιδικόν σου· επειδή θέλεις εκδιώξει τους Χαναναίους, αν και έχωσιν αμάξας σιδηράς και ήναι δυνατοί.

\chapter{18}

\par 1 Συνηθροίσθη δε πάσα η συναγωγή των υιών Ισραήλ εν Σηλώ, και έστησαν εκεί την σκηνήν του μαρτυρίου· και υπετάχθη η γη εις αυτούς.
\par 2 Και έμενον μεταξύ των υιών Ισραήλ επτά φυλαί, αίτινες δεν είχον λάβει την κληρονομίαν αυτών.
\par 3 Και είπεν ο Ιησούς προς τους υιούς Ισραήλ, Έως πότε θέλετε μένει νωθροί εις το να υπάγητε να κυριεύσητε την γην, την οποίαν Κύριος ο Θεός των πατέρων σας έδωκεν εις εσάς;
\par 4 εκλέξατε εις εαυτούς τρεις άνδρας κατά φυλήν· και θέλω αποστείλει αυτούς, και σηκωθέντες θέλουσι περιέλθει την γην και διαγράψει αυτήν κατά τας κληρονομίας αυτών, και θέλουσιν επιστρέψει προς εμέ·
\par 5 και θέλουσι διαιρέσει αυτήν εις επτά μερίδια· ο Ιούδας θέλει κατοικεί εν τοις ορίοις αυτού προς μεσημβρίαν, και ο οίκος Ιωσήφ θέλουσι κατοικεί εν τοις ορίοις αυτών προς βορράν·
\par 6 θέλετε λοιπόν διαγράψει την γην εις επτά μέρη, και θέλετε φέρει εδώ προς εμέ την διαγραφήν, και εγώ θέλω εκβάλει κλήρους διά σας ενταύθα ενώπιον Κυρίου του Θεού ημών·
\par 7 διότι οι Λευΐται δεν έχουσι μερίδιον μεταξύ σας· διότι η ιερατεία του Κυρίου είναι η κληρονομία αυτών· και ο Γαδ και ο Ρουβήν και το ήμισυ της φυλής Μανασσή έλαβον την κληρονομίαν αυτών πέραν του Ιορδάνου προς ανατολάς, την οποίαν Μωϋσής ο δούλος του Κυρίου έδωκεν εις αυτούς.
\par 8 Και σηκωθέντες οι άνδρες απήλθον· και προσέταξεν ο Ιησούς τους απελθόντας διά να διαγράψωσι την γην, λέγων, Υπάγετε και περιέλθετε την γην και διαγράψατε αυτήν και επιστρέψατε προς εμέ, και εγώ θέλω εκβάλει κλήρους διά σας ενταύθα ενώπιον του Κυρίου εν Σηλώ.
\par 9 Και υπήγαν οι άνδρες και περιήλθον την γην και διέγραψαν αυτήν εν βιβλίω κατά πόλεις εις μερίδια επτά, και ήλθον προς τον Ιησούν εις το στρατόπεδον εις Σηλώ.
\par 10 Και έρριψεν ο Ιησούς κλήρους δι' αυτούς εν Σηλώ ενώπιον του Κυρίου· και διεμοίρασεν ο Ιησούς εκεί την γην εις τους υιούς Ισραήλ κατά τους μερισμούς αυτών.
\par 11 Και εξήλθεν ο κλήρος της φυλής των υιών Βενιαμίν κατά τας συγγενείας αυτών, και έπεσε το όριον της κληρονομίας αυτών μεταξύ των υιών Ιούδα και των υιών Ιωσήφ.
\par 12 Και ήτο το όριον αυτών προς βορράν από του Ιορδάνου, και ανέβαινε το όριον προς το πλάγιον της Ιεριχώ κατά βορράν, και ανέβαινε διά των ορέων προς δυσμάς, και ετελείονεν εις την έρημον Βαιθ-αυέν.
\par 13 Και εκείθεν διέβαινε το όριον προς την Λούζ, κατά το μεσημβρινόν πλάγιον της Λούζ, ήτις είναι η Βαιθήλ· και κατέβαινε το όριον εις την Αταρώθ-αδδάρ, εις το όρος το προς μεσημβρίαν της κάτω αιθ-ωρών.
\par 14 Και το όριον εξετείνετο εκείθεν, και περιήρχετο το δυτικόν μέρος προς μεσημβρίαν, από του όρους του απέναντι της Βαιθ-ωρών κατά μεσημβρίαν· και ετελείονεν εις την Κιριάθ-βαάλ, ήτις είναι η Κιριάθ-ιαρείμ, πόλις των υιών Ιούδα· τούτο ήτο το δυτικόν μέρος.
\par 15 Και το μεσημβρινόν μέρος ήτο από του άκρον της Κιριάθ-ιαρείμ, και διέβαινε το όριον προς δυσμάς, και εξήρχετο εις το φρέαρ των υδάτων του Νεφθωά·
\par 16 και κατέβαινε το όριον εις το τέλος του όρους του κατέναντι της φάραγγος του υιού του Εννόμ, ήτις είναι εν τη κοιλάδι των Ραφαείμ προς βορράν, και κατέβαινε διά της φάραγγος του Εννόμ εις το μεσημβρινόν πλάγιον της Ιεβούς, και κατέβαινεν εις Εν-ρωγήλ·
\par 17 και εκτεινόμενον από βορράν διέβαινεν εις Εν-σεμές και εξήρχετο εις Γαλιλώθ, ήτις είναι κατέναντι της αναβάσεως του Αδουμμίμ, και κατέβαινεν εις τον λίθον του Βοάν υιού του Ρουβήν,
\par 18 και διέβαινε προς το βόρειον πλάγιον το κατέναντι της Αραβά και κατέβαινεν εις Αραβά·
\par 19 και διέβαινε το όριον προς το βόρειον πλάγιον της αιθ-ογλά· και ετελείονε το όριον εις τον βόρειον κόλπον της θαλάσσης της αλμυράς, εις την εκβολήν του Ιορδάνου κατά μεσημβρίαν· τούτο ήτο το μεσημβρινόν όριον.
\par 20 Ο δε Ιορδάνης ήτο το προς ανατολάς όριον αυτού. Αύτη ήτο κατά το όριον αυτής κύκλω, η κληρονομία των υιών Βενιαμίν κατά τας συγγενείας αυτών.
\par 21 Αι δε πόλεις της φυλής των υιών Βενιαμίν κατά τας συγγενείας αυτών ήσαν Ιεριχώ, και Βαιθ-ογλά, και Εμέκ-κεσείς,
\par 22 και Βαιθ-αραβά, και Σεμαραΐμ, και Βαιθήλ,
\par 23 και Αυείμ, και Φαρά, και Οφρά,
\par 24 και Χεφάρ-αμμωνά, και Οφνεί, και Γαβαά, πόλεις δώδεκα, και αι κώμαι αυτών·
\par 25 Γαβαών, και Ραμά, και Βηρώθ,
\par 26 και Μισπά, και Χεφειρά, και Μωσά,
\par 27 και Ρεκέμ, και Ιορφαήλ, και Θαραλά,
\par 28 και Σηλά, Ελέφ, και Ιεβούς, ήτις είναι η Ιερουσαλήμ, Γαβαάθ, και Κιριάθ, πόλεις δεκατέσσαρες, και αι κώμαι αυτών. Αύτη είναι η κληρονομία των υιών Βενιαμίν κατά τας συγγενείας αυτών.

\chapter{19}

\par 1 Και εξήλθεν ο δεύτερος κλήρος εις τον Συμεών, εις την φυλήν των υιών Συμεών κατά τας συγγενείας αυτών· και ήτο η κληρονομία αυτών εντός της κληρονομίας των υιών Ιούδα.
\par 2 Και έλαβον εις κληρονομίαν αυτών Βηρ-σαβεέ, και Σαβεέ, και Μωλαδά,
\par 3 και Ασάρ-σουάλ, και Βαλά, και Ασέμ,
\par 4 και Ελθωλάδ, και Βεθούλ, και Ορμά,
\par 5 και Σικλάγ, και Βαιθ-μαρκαβώθ, και Ασάρ-σουσά,
\par 6 και Βαιθ-λεβαώθ, και Σαρουέν, πόλεις δεκατρείς, και τας κώμας αυτών.
\par 7 Αείν, Ρεμμών, και Εθέρ, και Ασάν, πόλεις τέσσαρας, και τας κώμας αυτών·
\par 8 και πάσας τας κώμας τας πέριξ των πόλεων τούτων έως Βαλάθ-βηρ, ήτις είναι η Ραμάθ κατά μεσημβρίαν. Αύτη είναι κληρονομία της φυλής των υιών Συμεών κατά τας συγγενείας αυτών.
\par 9 Εκ του μεριδίου των υιών Ιούδα εδόθη η κληρονομία των υιών Συμεών, διότι το μερίδιον των υιών Ιούδα ήτο παραπολύ μεγάλον δι' αυτούς· όθεν οι υιοί Συμεών έλαβον την κληρονομίαν αυτών εντός της κληρονομίας εκείνων.
\par 10 Και εξήλθεν ο τρίτος κλήρος εις τους υιούς Ζαβουλών κατά τας συγγενείας αυτών· και το όριον της κληρονομίας αυτών ήτο έως Σαρείδ·
\par 11 και ανέβαινε το όριον αυτών προς την θάλασσαν και την Μαραλά, και ήρχετο εις την Δαβασαίθ, και έφθανε προς τον χείμαρρον τον κατέναντι Ιοκνεάμ·
\par 12 και έστρεφεν από της Σαρείδ, κατά ανατολάς ηλίου προς το όριον της Κισλώθ-θαβώρ, και εξήρχετο εις Δαβράθ, και ανέβαινεν εις Ιαφιά·
\par 13 και εκείθεν εξετείνετο προς ανατολάς εις Γιθθά-εφέρ, εις Ιττά-κασίν, και εξήρχετο εις Ρεμμών-μεθωάρ προς Νεά·
\par 14 και περιήρχετο το όριον κατά το βόρειον εις Ανναθών, και ετελείονεν εις την κοιλάδα Ιεφθαήλ·
\par 15 και περιελάμβανε την Καττάθ, και Νααλάλ, και Σιμβρών, και Ιδαλά, και Βηθλεέμ· πόλεις δώδεκα, και τας κώμας αυτών.
\par 16 Αύτη είναι η κληρονομία των υιών Ζαβουλών κατά τας συγγενείας αυτών, αι πόλεις αύται και αι κώμαι αυτών.
\par 17 Εξήλθεν ο τέταρτος κλήρος εις τον Ισσάχαρ, εις τους υιούς Ισσάχαρ κατά τας συγγενείας αυτών.
\par 18 Και ήτο το όριον αυτών Ιεζραέλ, και Κεσουλώθ, και Σουνήμ,
\par 19 και Αφεραΐμ, και Σαιών, και Αναχαράθ,
\par 20 και Ραββίθ, και Κισιών, και Αβές,
\par 21 και Ραιμέθ, και Εν-γαννίμ, και Εν-αδδά και Βαιθ-φασής·
\par 22 και έφθανε το όριον εις Θαβώρ, και Σαχασειμά, και Βαιθ-σεμές, και το όριον αυτών ετελείονεν εις τον Ιορδάνην· πόλεις δεκαέξ, και αι κώμαι αυτών.
\par 23 Αύτη είναι η κληρονομία της φυλής των υιών Ισσάχαρ κατά τας συγγενείας αυτών, αι πόλεις και αι κώμαι αυτών.
\par 24 Και εξήλθεν ο πέμπτος κλήρος εις την φυλήν των υιών Ασήρ κατά τας συγγενείας αυτών.
\par 25 Και ήτο το όριον αυτών Χελκάθ, και Αλεί, και Βετέν, και Αχσάφ,
\par 26 και Αλαμμέλεχ, και Αμάδ, και Μισάλ· και έφθανεν εις Καρμέλ προς δυσμάς και εις Σιχώρ-λιβνάθ·
\par 27 και έστρεφε προς ανατολάς ηλίου εις την Βαιθ-δαγών, και έφθανεν εις Ζαβουλών και εις την κοιλάδα Ιεφθαήλ προς το βόρειον της Βαιθ-εμέκ, και Ναϊήλ, και εξήρχετο εις την Χαβούλ κατά τα αριστερά,
\par 28 και Χεβρών, και Ρεώβ, και Αμμών, και Κανά, έως Σιδώνος της μεγάλης·
\par 29 και έστρεφε το όριον εις την Ραμά, και έως της οχυράς πόλεως Τύρου, και έστρεφε το όριον εις την Οσά, και ετελείονεν εις την θάλασσαν κατά το μέρος του Αχζίβ·
\par 30 και Αμμά, και Αφέκ, και Ρεώβ· πόλεις εικοσιδύο, και αι κώμαι αυτών.
\par 31 Αύτη είναι η κληρονομία της φυλής των υιών Ασήρ κατά τας συγγενείας αυτών, αι πόλεις αύται και αι κώμαι αυτών.
\par 32 Εξήλθεν ο έκτος κλήρος εις τους υιούς Νεφθαλί, εις τους υιούς Νεφθαλί κατά τας συγγενείας αυτών.
\par 33 Και ήτο το όριον αυτών από Ελέφ, από Αλλόν πλησίον Σαανανείμ, και Αδαμί, Νεκέβ, και Ιαβνήλ, έως Λακκούμ, και ετελείονεν εις τον Ιορδάνην.
\par 34 και έστρεφε το όριον από δυσμών εις Αζνώθ-θαβώρ, και εκείθεν εξήρχετο εις Ουκκώκ, και έφθανεν εις Ζαβουλών κατά μεσημβρίαν, και έφθανεν εις Ασήρ κατά δυσμάς και εις Ιούδα κατά ανατολάς ηλίου επί του Ιορδάνου.
\par 35 Και ήσαν αι τετειχισμέναι πόλεις Σηδδίμ, Σερ, και Αμμάθ, Ρακκάθ, και Χιννερώθ,
\par 36 και Αδαμά, και Ραμά, και Ασώρ,
\par 37 και Κέδες, και Εδρεΐ, και Εν-ασώρ,
\par 38 και Ιρών, και Μιγδαλήλ, Ωρέμ, και Βαιθ-ανάθ, και Βαιθ-σεμές· πόλεις δεκαεννέα, και αι κώμαι αυτών.
\par 39 Αύτη είναι η κληρονομία της φυλής των υιών Νεφθαλί κατά τας συγγενείας αυτών, αι πόλεις και αι κώμαι αυτών.
\par 40 Εξήλθεν ο έβδομος κλήρος εις την φυλήν των υιών Δαν κατά τας συγγενείας αυτών.
\par 41 Και ήτο το όριον της κληρονομίας αυτών Σαραά, και Εσθαόλ, και Ιρ-σεμές,
\par 42 και Σαλαβείν, και Αιαλών, και Ιεθλά,
\par 43 και Αιλών, και Θαμναθά, και Ακκαρών,
\par 44 και Ελθεκώ, και Γιββεθών, και Βααλάθ,
\par 45 και Ιούδ, και Βανή-βαράκ, και Γαθ-ριμμών,
\par 46 και Με-ιαρκών, και Ρακκών, μετά του ορίου του κατέναντι της Ιόππης.
\par 47 Το δε όριον των υιών Δαν παρετάθη υπ' αυτών· διά τούτο οι υιοί Δαν ανέβησαν να πολεμήσωσι την Λεσέμ και εκυρίευσαν αυτήν και επάταξαν αυτήν εν στόματι μαχαίρας, και εξουσίασαν αυτήν και κατώκησαν εν αυτή· και ωνόμασαν αυτήν Λεσέμ Δαν, κατά το όνομα Δαν του πατρός αυτών.
\par 48 Αύτη είναι η κληρονομία της φυλής των υιών Δαν κατά τας συγγενείας αυτών, αι πόλεις αύται και αι κώμαι αυτών.
\par 49 Αφού δε ετελείωσαν μεριζόμενοι την γην κατά τα όρια αυτής, οι υιοί Ισραήλ έδωκαν μεταξύ αυτών κληρονομίαν εις τον Ιησούν τον υιόν του Ναυή·
\par 50 κατά τον λόγον του Κυρίου έδωκαν εις αυτόν την πόλιν, την οποίαν εζήτησε, την Θαμνάθ-σαράχ εν τω όρει Εφραΐμ· και έκτισε την πόλιν και κατώκησεν εν αυτή.
\par 51 Αύται είναι αι κληρονομίαι, τας οποίας Ελεάζαρ ο ιερεύς και Ιησούς ο υιός του Ναυή και οι αρχηγοί των πατριών των φυλών των υιών Ισραήλ διεμοίρασαν διά κλήρων εν Σηλώ, ενώπιον του Κυρίου παρά την θύραν της σκηνής του μαρτυρίου. Και ετελείωσαν τον διαμερισμόν της γης.

\chapter{20}

\par 1 Και ελάλησε Κύριος προς τον Ιησούν λέγων,
\par 2 Είπε προς τους υιούς Ισραήλ λέγων, Διορίσατε εις εαυτούς τας πόλεις της καταφυγής, περί των οποίων είπα προς εσάς διά του Μωϋσέως·
\par 3 διά να φεύγη εκεί ο φονεύς, όστις φονεύση άνθρωπον ακουσίως εξ αγνοίας· και αύται θέλουσιν είσθαι εις εσάς διά καταφύγιον από του εκδικητού του αίματος.
\par 4 Και όταν ο φεύγων εις μίαν εκ των πόλεων τούτων σταθή εις την είσοδον της πύλης της πόλεως, και λαλήση την υπόθεσιν αυτού εις επήκοον των πρεσβυτέρων της πόλεως εκείνης, ούτοι θέλουσι δεχθή αυτόν εις την πόλιν προς εαυτούς και δώσει τόπον εις αυτόν, και θέλει κατοικεί μετ' αυτών.
\par 5 Και εάν ο εκδικητής του αίματος καταδιώξη αυτόν, δεν θέλουσι παραδώσει τον φονέα εις τας χείρας αυτού· διότι εξ αγνοίας επάταξε τον πλησίον αυτού και δεν εμίσει αυτόν πρότερον.
\par 6 Και θέλει κατοικεί εν εκείνη τη πόλει, εωσού παρασταθή ενώπιον της συναγωγής εις κρίσιν, έως του θανάτου του ιερέως του μεγάλου, του όντος εν ταις ημέραις εκείναις· τότε ο φονεύς θέλει επιστρέψει και υπάγει εις την πόλιν αυτού και εις την οικίαν αυτού, εις την πόλιν όθεν έφυγε.
\par 7 Και διώρισαν την Κέδες εν τη Γαλιλαία εν τω όρει Νεφθαλί, και την Συχέμ εν τω όρει Εφραΐμ, και την Κιριάθ-αρβά, ήτις είναι η Χεβρών, εν τη ορεινή του Ιούδα.
\par 8 Εις δε το πέραν του Ιορδάνου, πλησίον της Ιεριχώ, προς ανατολάς, διώρισαν την Βοσόρ εν τη ερήμω επί της πεδιάδος εκ της φυλής Ρουβήν, και την Ραμώθ εν τη Γαλαάδ εκ της φυλής Γαδ, και την Γωλάν εν Βασάν εκ της φυλής Μανασσή.
\par 9 Αύται ήσαν αι πόλεις αι διορισθείσαι διά πάντας τους υιούς Ισραήλ και διά τους ξένους τους παροικούντας μεταξύ αυτών, ώστε πας ο φονεύσας τινά εξ αγνοίας να φεύγη εκεί, και να μη θανατωθή εκ της χειρός του εκδικητού του αίματος, εωσού παρασταθή ενώπιον της συναγωγής.

\chapter{21}

\par 1 Και προσήλθον οι αρχηγοί των πατριών των Λευϊτών προς Ελεάζαρ τον ιερέα και προς Ιησούν τον υιόν του Ναυή και προς τους αρχηγούς των πατριών των φυλών των υιών Ισραήλ,
\par 2 και είπον προς αυτούς εν Σηλώ εν τη γη Χαναάν λέγοντες, Ο Κύριος προσέταξε διά του Μωϋσέως να δοθώσιν εις ημάς πόλεις να κατοικώμεν, και τα περίχωρα αυτών διά τα κτήνη ημών.
\par 3 Και έδωκαν οι υιοί Ισραήλ εις τους Λευΐτας εκ της κληρονομίας αυτών, κατά τον λόγον του Κυρίου, τας πόλεις ταύτας και τα περίχωρα αυτών.
\par 4 Και εξήλθεν ο κλήρος εις τας συγγενείας των Κααθιτών· και οι υιοί Ααρών του ιερέως, οι εκ των Λευϊτών, έλαβον διά κλήρον εκ της φυλής Ιούδα και εκ της φυλής Συμεών, και εκ της φυλής Βενιαμίν δεκατρείς πόλεις.
\par 5 Οι δε υιοί Καάθ οι επίλοιποι έλαβον διά κλήρον εκ των συγγενειών της φυλής Εφραΐμ και εκ της φυλής Δαν και εκ του ημίσεως της φυλής Μανασσή, δέκα πόλεις.
\par 6 Και οι υιοί Γηρσών έλαβον διά κλήρου εκ των συγγενειών της φυλής Ισσάχαρ και εκ της φυλής Ασήρ και εκ της φυλής Νεφθαλί και εκ του ημίσεως της φυλής Μανασσή εν Βασάν, δεκατρείς πόλεις.
\par 7 Οι υιοί Μεραρί κατά τας συγγενείας αυτών έλαβον εκ της φυλής Ρουβήν και εκ της φυλής Γαδ και εκ της φυλής Ζαβουλών, δώδεκα πόλεις.
\par 8 Και έδωκαν οι υιοί Ισραήλ διά κλήρου εις τους Λευΐτας τας πόλεις ταύτας και τα περίχωρα αυτών, καθώς ο Κύριος προσέταξε διά του Μωϋσέως.
\par 9 Και έδωκαν εκ της φυλής των υιών Ιούδα και εκ της φυλής των υιών Συμεών τας πόλεις ταύτας, αίτινες αναφέρονται ενταύθα κατ' όνομα·
\par 10 και έλαβον αυτάς οι υιοί του Ααρών, οι εκ των συγγενειών των Κααθιτών, οι εκ των υιών του Λευΐ· διότι τούτων ήτο ο πρώτος κλήρος.
\par 11 Και έδωκαν εις αυτούς την πόλιν του Αρβά πατρός του Ανάκ, ήτις είναι η Χεβρών, εν τη ορεινή του Ιούδα, και τα περίχωρα αυτής κύκλω.
\par 12 Τους δε αγρούς της πόλεως και τας κώμας αυτής έδωκαν εις Χάλεβ τον υιόν του Ιεφοννή, εις ιδιοκτησίαν αυτού.
\par 13 Και έδωκαν εις τους υιούς Ααρών του ιερέως την πόλιν του καταφυγίου διά τον φονέα, την Χεβρών και τα περίχωρα αυτής, και την Λιβνά και τα περίχωρα αυτής,
\par 14 και την Ιαθείρ και τα περίχωρα αυτής, και την Εσθεμωά και τα περίχωρα αυτής,
\par 15 και την Ωλών και τα περίχωρα αυτής, και την Δεβείρ και τα περίχωρα αυτής,
\par 16 και την Αείν και τα περίχωρα αυτής, και την Ιουτά και τα περίχωρα αυτής, την Βαιθ-σεμές και τα περίχωρα αυτής· πόλεις εννέα εκ των δύο τούτων φυλών·
\par 17 και εκ της φυλής Βενιαμίν, την Γαβαών και τα περίχωρα αυτής, την Γαβαά και τα περίχωρα αυτής,
\par 18 την Αναθώθ και τα περίχωρα αυτής, και την Αλμών και τα περίχωρα αυτής· πόλεις τέσσαρας.
\par 19 Πάσαι αι πόλεις των υιών του Ααρών των ιερέων, πόλεις δεκατρείς και τα περίχωρα αυτών.
\par 20 Και αι συγγένειαι των υιών του Καάθ των Λευϊτών, των επιλοίπων εκ των υιών Καάθ, έλαβον τας πόλεις του κλήρου αυτών εκ της φυλής του Εφραΐμ.
\par 21 Και έδωκαν εις αυτούς την πόλιν του καταφυγίου διά τον φονέα, την Συχέμ και τα περίχωρα αυτής εν τω όρει Εφραΐμ, και την Γεζέρ και τα περίχωρα αυτής,
\par 22 και την Κιβσαείμ και τα περίχωρα αυτής, και την Βαιθ-ωρών και τα περίχωρα αυτής· πόλεις τέσσαρας·
\par 23 και εκ της φυλής Δαν, την Ελθεκώ και τα περίχωρα αυτής, την Γιββεθών και τα περίχωρα αυτής,
\par 24 την Αιαλών και τα περίχωρα αυτής, την Γαθ-ριμμών και τα περίχωρα αυτής· πόλεις τέσσαρας·
\par 25 και εκ του ημίσεως της φυλής του Μανασσή, την Θαανάχ και τα περίχωρα αυτής, και την Γαθ-ριμμών και τα περίχωρα αυτής· πόλεις δύο.
\par 26 Πάσαι αι πόλεις ήσαν δέκα, και τα περίχωρα αυτών, διά τας συγγενείας των επιλοίπων υιών του Καάθ.
\par 27 Εις δε τους υιούς Γηρσών, εκ των συγγενειών των Λευϊτών, έδωκαν εκ του άλλου ημίσεως της φυλής Μανασσή, την πόλιν του καταφυγίου διά τον φονέα, την Γωλάν εν Βασάν και τα περίχωρα αυτής, και την Βεεσθερά και τα περίχωρα αυτής· πόλεις δύο·
\par 28 και εκ της φυλής Ισσάχαρ, την Κισιών και τα περίχωρα αυτής, την Δαβράθ και τα περίχωρα αυτής,
\par 29 την Ιαρμούθ και τα περίχωρα αυτής, την Εν-γαννίμ και τα περίχωρα αυτής· πόλεις τέσσαρας·
\par 30 και εκ της φυλής Ασήρ, την Μισαάλ και τα περίχωρα αυτής, την Αβδών και τα περίχωρα αυτής,
\par 31 την Χελκάθ και τα περίχωρα αυτής, και την Ρεώβ και τα περίχωρα αυτής· πόλεις τέσσαρας·
\par 32 και εκ της φυλής Νεφθαλί, την πόλιν του καταφυγίου διά τον φονέα, την Κέδες εν τη Γαλιλαία και τα περίχωρα αυτής, και την Αμμώθ-δωρ και τα περίχωρα αυτής, και την Καρθάν και τα περίχωρα αυτής· πόλεις τρεις.
\par 33 Πάσαι αι πόλεις των Γηρσωνιτών, κατά τας συγγενείας αυτών, ήσαν πόλεις δεκατρείς και τα περίχωρα αυτών.
\par 34 Εις δε τας συγγενείας των υιών Μεραρί, των επιλοίπων εκ των Λευϊτών, έδωκαν εκ της φυλής Ζαβουλών, την Ιοκνεάμ και τα περίχωρα αυτής, την Καρθά και τα περίχωρα αυτής,
\par 35 την Διμνά και τα περίχωρα αυτής, την Νααλώλ και τα περίχωρα αυτής· πόλεις τέσσαρας·
\par 36 και εκ της φυλής Ρουβήν έδωκαν την Βοσόρ και τα περίχωρα αυτής, και την Ιααζά και τα περίχωρα αυτής,
\par 37 την Κεδημώθ και τα περίχωρα αυτής, και την Μηφαάθ και τα περίχωρα αυτής· πόλεις τέσσαρας·
\par 38 και εκ της φυλής Γαδ έδωκαν την πόλιν του καταφυγίου διά τον φονέα, την Ραμώθ εν Γαλαάδ και τα περίχωρα αυτής, και την Μαχαναΐμ και τα περίχωρα αυτής,
\par 39 την Εσεβών και τα περίχωρα αυτής, την Ιαζήρ και τα περίχωρα αυτής· πάσαι αι πόλεις τέσσαρες.
\par 40 πάσαι αι πόλεις αι δοθείσαι διά κλήρων εις τους υιούς Μεραρί, κατά τας συγγενείας αυτών, τους υπολοίπους εκ των συγγενειών των Λευϊτών, ήσαν πόλεις δώδεκα.
\par 41 Πάσαι αι πόλεις των Λευϊτών, αι μεταξύ της ιδιοκτησίας των υιών Ισραήλ, ήσαν τεσσαράκοντα οκτώ πόλεις και τα περίχωρα αυτών.
\par 42 Αι πόλεις αύται ήσαν εκάστη μετά των περιχώρων αυτών κύκλω· ούτως ήσαν πάσαι αι πόλεις αύται.
\par 43 Και έδωκεν ο Κύριος εις τον Ισραήλ πάσαν την γην, την οποίαν ώμοσε να δώση προς τους πατέρας αυτών· και εκυρίευσαν αυτήν και κατώκησαν εν αυτή.
\par 44 Και έδωκεν ο Κύριος εις αυτούς ανάπαυσιν πανταχόθεν, κατά πάντα όσα ώμοσε προς τους πατέρας αυτών· και ουδείς εκ πάντων των εχθρών αυτών ηδυνήθη να σταθή κατά πρόσωπον αυτών· πάντας τους εχθρούς αυτών παρέδωκεν ο Κύριος εις την χείρα αυτών.
\par 45 Δεν διέπεσεν ουδέ εις εκ πάντων των αγαθών λόγων, τους οποίους ο Κύριος ελάλησε προς τον οίκον Ισραήλ· πάντες εξετελέσθησαν.

\chapter{22}

\par 1 Τότε συνεκάλεσεν ο Ιησούς τους Ρουβηνίτας και τους Γαδίτας και το ήμισυ της φυλής του Μανασσή,
\par 2 και είπε προς αυτούς, Σεις εφυλάξατε πάντα όσα προσέταξεν εις εσάς Μωϋσής ο δούλος του Κυρίου, και υπηκούσατε εις την φωνήν μου κατά πάντα όσα εγώ προσέταξα εις εσάς·
\par 3 δεν εγκατελίπετε τους αδελφούς σας εις τας πολλάς ταύτας ημέρας έως της σήμερον, αλλ' εφυλάξατε εντελώς την εντολήν Κυρίου του Θεού σας·
\par 4 και τώρα Κύριος ο Θεός σας έδωκεν ανάπαυσιν εις τους αδελφούς σας, καθώς υπεσχέθη προς αυτούς· τώρα λοιπόν επιστρέψατε και υπάγετε εις τας κατοικίας σας, εις την γην της ιδιοκτησίας σας, την οποίαν Μωϋσής ο δούλος του Κυρίου έδωκεν εις εσάς εις το πέραν του Ιορδάνου·
\par 5 προσέχετε όμως σφόδρα να εκτελήτε τας εντολάς και τον νόμον, τον οποίον Μωϋσής ο δούλος του Κυρίου προσέταξεν εις εσάς, να αγαπάτε Κύριον τον Θεόν σας, και να περιπατήτε εις πάσας τας οδούς αυτού, και να φυλάττητε τας εντολάς αυτού, και να ήσθε προσηλωμένοι εις αυτόν, και να λατρεύητε αυτόν εξ όλης της καρδίας σας και εξ όλης της ψυχής σας.
\par 6 Και ευλόγησεν αυτούς ο Ιησούς και απέλυσεν αυτούς· και απήλθον εις τας κατοικίας αυτών.
\par 7 Και εις μεν το ήμισυ της φυλής του Μανασσή έδωκεν ο Μωϋσής κληρονομίαν εν Βασάν· εις δε το άλλο ήμισυ αυτής έδωκεν ο Ιησούς κληρονομίαν μεταξύ των αδελφών αυτών εντεύθεν του Ιορδάνου προς δυσμάς. Και ότε ο Ιησούς απέστειλεν αυτούς εις τας κατοικίας αυτών, ευλόγησεν αυτούς·
\par 8 και ελάλησε προς αυτούς, λέγων, Επιστρέψατε με πολλά πλούτη εις τας κατοικίας σας, και με κτήνη πολλά σφόδρα, με άργυρον και με χρυσόν και με χαλκόν και με σίδηρον και με ιμάτια πολλά σφόδρα, μοιράσθητε τα λάφυρα των εχθρών σας μετά των αδελφών σας.
\par 9 Και οι υιοί Ρουβήν και οι υιοί Γαδ και το ήμισυ της φυλής Μανασσή έστρεψαν και ανεχώρησαν από των υιών Ισραήλ εκ της Σηλώ, της εν τη γη Χαναάν, διά να υπάγωσιν εις την γην Γαλαάδ, εις την γην της ιδιοκτησίας αυτών, την οποίαν εκληρονόμησαν κατά τον λόγον του Κυρίου διά του Μωϋσέως.
\par 10 Και ελθόντες εις τα πέριξ του Ιορδάνου, τα εντός της γης Χαναάν, οι υιοί Ρουβήν και οι υιοί Γαδ και το ήμισυ της φυλής Μανασσή ωκοδόμησαν εκεί θυσιαστήριον παρά τον Ιορδάνην, θυσιαστήριον μέγα εις την όψιν.
\par 11 Και ήκουσαν οι υιοί Ισραήλ να λέγηται, Ιδού, οι υιοί Ρουβήν και οι υιοί Γαδ και το ήμισυ της φυλής Μανασσή ωκοδόμησαν θυσιαστήριον κατέναντι της γης Χαναάν, εις τα πέριξ του Ιορδάνου, κατά την διάβασιν των Ισραήλ.
\par 12 Και ότε ήκουσαν οι υιοί Ισραήλ, συνήχθη πάσα η συναγωγή των υιών Ισραήλ εις Σηλώ, διά να αναβώσι να πολεμήσωσι κατ' αυτών.
\par 13 Και απέστειλαν οι υιοί Ισραήλ προς τους υιούς Ρουβήν και προς τους υιούς Γαδ και προς το ήμισυ της φυλής Μανασσή εις την γην Γαλαάδ τον Φινεές υιόν Ελεάζαρ τον ιερέα,
\par 14 και μετ' αυτού δέκα άρχοντας, ανά ένα άρχοντα αρχηγόν πατριών κατά φυλήν του Ισραήλ, και έκαστος ήτο ο πρώτος του οίκου των πατέρων αυτών, επί τας χιλιάδας του Ισραήλ.
\par 15 Και υπήγον προς τους υιούς Ρουβήν και προς τους υιούς Γαδ και προς το ήμισυ της φυλής Μανασσή εις την γην Γαλαάδ, και ελάλησαν προς αυτούς λέγοντες,
\par 16 Ταύτα λέγει πάσα η συναγωγή Κυρίου· Τις αύτη η ανομία, την οποίαν επράξατε εναντίον του Θεού του Ισραήλ, να απομακρυνθήτε σήμερον από του Κυρίου, οικοδομήσαντες θυσιαστήριον εις εαυτούς, ώστε να αποστατήσητε σήμερον από Κυρίου;
\par 17 Μικρόν εστάθη το αμάρτημα ημών εις Φεγώρ, από του οποίου έως της σήμερον δεν εκαθαρίσθημεν, και έγεινε πληγή εις την συναγωγήν του Κυρίου,
\par 18 και σεις θέλετε σήμερον αποστατήσει από του Κυρίου; βεβαίως, εάν σεις αποστατήσητε σήμερον από του Κυρίου, αύριον θέλει οργισθή εναντίον πάσης της συναγωγής του Ισραήλ.
\par 19 Εάν η γη της ιδιοκτησίας σας ήναι ακάθαρτος, διάβητε εις την γην της ιδιοκτησίας του Κυρίου, όπου η σκηνή του Κυρίου κατοικεί, και λάβετε ιδιοκτησίαν μεταξύ ημών· και μη αποστατήσητε από του Κυρίου, μηδέ αφ' ημών αποστατήσητε, οικοδομούντες εις εαυτούς θυσιαστήριον εκτός του θυσιαστηρίου Κυρίου του Θεού ημών.
\par 20 Δεν έπραξεν Αχάν ο υιός του Ζερά εν τω αναθέματι, και έπεσεν οργή εφ' όλην την συναγωγήν του Ισραήλ; και ο άνθρωπος εκείνος δεν ηφανίσθη μόνος εν τη ανομία αυτού.
\par 21 Τότε απεκρίθησαν οι υιοί Ρουβήν και οι υιοί Γαδ και το ήμισυ της φυλής Μανασσή και είπον προς τους αρχηγούς των χιλιάδων του Ισραήλ·
\par 22 Ο ισχυρός Θεός ο Κύριος, ο ισχυρός Θεός ο Κύριος, αυτός εξεύρει, και αυτός θέλει γνωρίσει· εάν επράξαμεν τούτο διά αποστασίαν ή εάν διά ανομίαν εναντίον του Κυρίου, μη λυτρώσης ημάς την ημέραν ταύτην.
\par 23 Εάν ωκοδομήσαμεν εις εαυτούς θυσιαστήριον διά να αποχωρισθώμεν από του Κυρίου, ή εάν διά να προσφέρωμεν επ' αυτού ολοκαύτωμα ή προσφοράς, ή εάν διά να προσφέρωμεν επ' αυτού ειρηνικάς θυσίας, αυτός ο Κύριος ας εκζητήση τούτο.
\par 24 Και εάν δεν επράξαμεν αυτό μάλλον εκ φόβου του πράγματος τούτου, λέγοντες, Αύριον δύνανται τα τέκνα σας να είπωσι προς τα τέκνα ημών, λέγοντα, Τι έχετε σεις να κάμητε μετά του Κυρίου του Θεού του Ισραήλ;
\par 25 διότι ο Κύριος έθεσε τον Ιορδάνην όριον μεταξύ ημών και υμών, υιοί Ρουβήν και υιοί Γάδ· δεν έχετε μέρος μετά του Κυρίου· και κάμωσιν οι υιοί σας τους υιούς ημών να παύσωσιν από του να φοβώνται τον Κύριον.
\par 26 Διά τούτο είπομεν, Ας επιχειρισθώμεν να οικοδομήσωμεν εις εαυτούς το θυσιαστήριον· ουχί διά ολοκαύτωμα ουδέ διά θυσίαν,
\par 27 αλλά διά να ήναι μαρτύριον αναμέσον ημών και υμών, και αναμέσον των γενεών ημών μεθ' ημάς, ότι ημείς κάμνομεν την λατρείαν του Κυρίου ενώπιον αυτού με τα ολοκαυτώματα ημών και με τας θυσίας ημών και με τας ειρηνικάς προσφοράς ημών· διά να μη είπωσιν αύριον τα τέκνα σας προς τα τέκνα ημών, Σεις δεν έχετε μέρος μετά του Κυρίου.
\par 28 Διά τούτο είπομεν, Εάν τύχη να λαλήσωσιν ούτω προς ημάς ή προς τας γενεάς ημών αύριον, τότε θέλομεν αποκριθή, Ιδού, το ομοίωμα του θυσιαστηρίου του Κυρίου, το οποίον ωκοδόμησαν οι πατέρες ημών, ουχί διά ολοκαύτωμα ουδέ διά θυσίαν, αλλά διά να ήναι μαρτύριον αναμέσον ημών και υμών.
\par 29 Μη γένοιτο να αποστατήσωμεν από του Κυρίου και να αποχωρισθώμεν σήμερον από του Κυρίου, οικοδομούντες θυσιαστήριον διά ολοκαύτωμα, διά προσφοράς και διά θυσίαν, εκτός του θυσιαστηρίου Κυρίου του Θεού ημών, το οποίον είναι έμπροσθεν της σκηνής αυτού.
\par 30 Και ακούσαντες Φινεές ο ιερεύς και οι άρχοντες της συναγωγής και οι αρχηγοί των χιλιάδων του Ισραήλ, οι όντες μετ' αυτού, τους λόγους τους οποίους ελάλησαν οι υιοί Ρουβήν και οι υιοί Γαδ και οι υιοί Μανασσή, ευχαριστήθησαν.
\par 31 Και είπε Φινεές ο υιός του Ελεάζαρ ο ιερεύς προς τους υιούς Ρουβήν και προς τους υιούς Γαδ και προς τους υιούς Μανασσή, Σήμερον εγνωρίσαμεν ότι ο Κύριος είναι εν μέσω ημών, διότι δεν επράξατε την ανομίαν ταύτην εναντίον του Κυρίου· διά τούτου ελυτρώσατε τους υιούς Ισραήλ από της χειρός του Κυρίου.
\par 32 Και επέστρεψεν ο Φινεές ο υιός του Ελεάζαρ ο ιερεύς και οι άρχοντες από των υιών Ρουβήν και από των υιών Γαδ εκ της γης Γαλαάδ εις την γην Χαναάν προς τους υιούς Ισραήλ, και έφεραν απόκρισιν προς αυτούς.
\par 33 Και το πράγμα ήρεσεν εις τους υιούς Ισραήλ· και ευλόγησαν τον Θεόν οι υιοί Ισραήλ, και δεν είπαν πλέον να αναβώσιν εναντίον αυτών εις μάχην, διά να αφανίσωσι την γην όπου κατώκουν οι υιοί Ρουβήν και οι υιοί Γαδ.
\par 34 Και ωνόμασαν οι υιοί Ρουβήν και οι υιοί Γαδ το θυσιαστήριον Εδ· Διότι, είπαν, τούτο θέλει είσθαι μαρτύριον αναμέσον ημών, ότι ο Κύριος είναι ο Θεός.

\chapter{23}

\par 1 Και μετά πολύν καιρόν αφού ο Κύριος έδωκεν εις τον Ισραήλ ανάπαυσιν από πάντων των εχθρών αυτού κύκλω, και ο Ιησούς ήτο γέρων, προβεβηκώς την ηλικίαν,
\par 2 συνεκάλεσεν ο Ιησούς πάντα τον Ισραήλ, τους πρεσβυτέρους αυτών και τους αρχηγούς αυτών και τους κριτάς αυτών, και τους άρχοντας αυτών και είπε προς αυτούς, Εγώ εγήρασα, είμαι προβεβηκώς την ηλικίαν.
\par 3 Και σεις είδετε πάντα όσα έκαμε Κύριος ο Θεός σας εις πάντα τα έθνη ταύτα διά σάς· διότι Κύριος ο Θεός σας, αυτός είναι ο πολεμήσας υπέρ υμών.
\par 4 Ιδού, εγώ εμοίρασα διά κλήρου εις εσάς ταύτα τα εναπολειφθέντα έθνη, διά κληρονομίαν εις τας φυλάς σας, μετά πάντων των εθνών τα οποία εξωλόθρευσα, από του Ιορδάνου έως της θαλάσσης της μεγάλης, προς δυσμάς ηλίου.
\par 5 Και Κύριος ο Θεός σας, αυτός θέλει εξώσει αυτούς απ' έμπροσθέν σας, και θέλει εκδιώξει αυτούς από προσώπου σας· και θέλετε κυριεύσει την γην αυτών, καθώς Κύριος ο Θεός σας υπεσχέθη προς εσάς.
\par 6 Ανδρίζεσθε λοιπόν σφόδρα εις το να φυλάττητε και να εκτελήτε πάντα τα γεγραμμένα εν τω βιβλίω του νόμου του Μωϋσέως, διά να μη εκκλίνητε απ' αυτού δεξιά ή αριστερά·
\par 7 διά να μη αναμιχθήτε μετά των εθνών τούτων, των εναπολειφθέντων μεταξύ σας, μηδέ να μνημονεύητε τα ονόματα των θεών αυτών, μηδέ να ομόσητε, μηδέ να λατρεύσητε αυτούς, μηδέ να προσκυνήσητε αυτούς·
\par 8 αλλ' εις Κύριον τον Θεόν σας να ήσθε προσκεκολλημένοι, καθώς εκάμετε έως της ημέρας ταύτης.
\par 9 Διότι ο Κύριος εξεδίωξεν απ' έμπροσθέν σας έθνη μεγάλα και δυνατά· και ουδείς ηδυνήθη έως της σήμερον να σταθή έμπροσθέν σας·
\par 10 εις από σας θέλει διώξει χιλίους· διότι Κύριος ο Θεός σας, αυτός είναι ο πολεμήσας υπέρ υμών, καθώς υπεσχέθη προς εσάς.
\par 11 Προσέχετε λοιπόν σφόδρα εις εαυτούς, να αγαπάτε Κύριον τον Θεόν σας.
\par 12 Επειδή εάν ποτέ στραφήτε οπίσω και προσκολληθήτε μετά του υπολοίπου των εθνών τούτων, μετά τούτων των εναπολειφθέντων μεταξύ σας, και συμπενθερεύσητε μετ' αυτών και αναμιχθήτε μετ' αυτών και εκείνα μεθ' υμών,
\par 13 εξεύρετε βεβαίως ότι Κύριος ο Θεός σας δεν θέλει πλέον εκδιώξει απ' έμπροσθέν σας τα έθνη ταύτα· αλλά θέλουσιν είσθαι παγίδες και ενέδραι εις εσάς, και μάστιγες εις τας πλευράς σας και άκανθαι εις τους οφθαλμούς σας, εωσού εξολοθρευθήτε από της γης ταύτης της αγαθής, την οποίαν Κύριος ο Θεός σας έδωκεν εις εσάς.
\par 14 Και ιδού, σήμερον εγώ πορεύομαι την οδόν πάσης της γης, και σεις γνωρίζετε εν όλη τη καρδία υμών και εν όλη τη ψυχή υμών, ότι δεν διέπεσεν ουδέ εις εκ πάντων των αγαθών λόγων, τους οποίους Κύριος ο Θεός σας ελάλησε διά σάς· πάντες ετελέσθησαν εις εσάς, ουδέ εις εξ αυτών διέπεσε.
\par 15 Διά τούτο, καθώς ήλθον εφ' υμάς πάντες οι αγαθοί λόγοι, τους οποίους ελάλησε προς υμάς Κύριος ο Θεός υμών, ούτως ο Κύριος θέλει επιφέρει εφ' υμάς πάντας τους λόγους τους κακούς, εωσού εξολοθρεύση υμάς από της γης της αγαθής ταύτης, την οποίαν έδωκεν εις υμάς Κύριος ο Θεός υμών.
\par 16 Όταν παραβήτε την διαθήκην Κυρίου του Θεού σας, την οποίαν προσέταξεν εις εσάς, και υπάγητε και λατρεύσητε άλλους θεούς και προσκυνήσητε αυτούς, τότε η οργή του Κυρίου θέλει εξαφθή εναντίον σας, και θέλετε αφανισθή ταχέως από της γης της αγαθής, την οποίαν έδωκεν εις εσάς.

\chapter{24}

\par 1 Και συνήθροισεν ο Ιησούς πάσας τας φυλάς του Ισραήλ εν Συχέμ, και συνεκάλεσε τους πρεσβυτέρους του Ισραήλ και τους αρχηγούς αυτών και τους κριτάς αυτών και τους άρχοντας αυτών· και παρεστάθησαν ενώπιον του Θεού.
\par 2 Και είπεν ο Ιησούς προς πάντα τον λαόν, Ούτω λέγει Κύριος ο Θεός του Ισραήλ· πέραν του ποταμού κατώκησαν απ' αρχής οι πατέρες σας, Θάρρα ο πατήρ του Αβραάμ και ο πατήρ του Ναχώρ, και ελάτρευσαν άλλους θεούς.
\par 3 Και έλαβον τον πατέρα σας τον Αβραάμ εκ του πέραν του ποταμού, και ώδήγησα αυτόν διά πάσης της γης Χαναάν, και επλήθυνα το σπέρμα αυτού, και έδωκα τον Ισαάκ εις αυτόν.
\par 4 Και εις τον Ισαάκ έδωκα τον Ιακώβ και τον Ησαύ· και έδωκα εις τον Ησαύ το όρος Σηείρ, διά να κληρονομήση αυτό· ο δε Ιακώβ και οι υιοί αυτού κατέβησαν εις την Αίγυπτον.
\par 5 Και απέστειλα τον Μωϋσήν και τον Ααρών, και επάταξα την Αίγυπτον διά πληγών, τας οποίας έκαμον εν μέσω αυτής, και μετά ταύτα εξήγαγον υμάς.
\par 6 Και αφού εξήγαγον τους πατέρας υμών εξ Αιγύπτου, ήλθετε εις την θάλασσαν· και κατεδίωξαν οι Αιγύπτιοι οπίσω των πατέρων υμών με αμάξας και ίππους εις την θάλασσαν την Ερυθράν·
\par 7 και εβόησαν προς Κύριον και αυτός έθεσε σκότος αναμέσον υμών και των Αιγυπτίων, και επήγαγεν επ' αυτούς την θάλασσαν και εκάλυψεν αυτούς, και οι οφθαλμοί υμών είδον τι έκαμον εν τη Αιγύπτω· και κατωκήσατε εν τη ερήμω ημέρας πολλάς.
\par 8 Και σας έφερα εις την γην των Αμορραίων, των κατοικούντων πέραν του Ιορδάνου, και σας επολέμησαν και παρέδωκα αυτούς εις τας χείρας σας, και κατεκληρονομήσατε την γην αυτών, και εξωλόθρευσα αυτούς απ' έμπροσθέν σας.
\par 9 Και εσηκώθη Βαλάκ ο υιός του Σεπφώρ, βασιλεύς του Μωάβ, και επολέμησε προς τον Ισραήλ· και αποστείλας προσεκάλεσε τον Βαλαάμ υιόν του Βεώρ διά να σας καταρασθή·
\par 10 αλλ' εγώ δεν ηθέλησα να ακούσω τον Βαλαάμ· μάλιστα δε και σας ευλόγησε, και σας ηλευθέρωσα εκ των χειρών αυτού.
\par 11 Και διέβητε τον Ιορδάνην και ήλθετε εις Ιεριχώ· και σας επολέμησαν οι άνδρες της Ιεριχώ, οι Αμορραίοι και οι Φερεζαίοι και οι Χαναναίοι και οι Χετταίοι και οι Γεργεσαίοι, οι Ευαίοι και οι Ιεβουσαίοι· και παρέδωκα αυτούς εις τας χείρας σας.
\par 12 Και εξαπέστειλα έμπροσθέν σας τας σφήκας, και εξεδίωξαν αυτούς απ' έμπροσθέν σας, τους δύο βασιλείς των Αμορραίων· ουχί διά της μαχαίρας σου ουδέ διά του τόξου σου.
\par 13 Και έδωκα εις εσάς γην, εις την οποίαν δεν εκοπιάσατε, και πόλεις τας οποίας δεν εκτίσατε, και κατωκήσατε εν αυταίς· και τρώγετε αμπελώνας και ελαιώνας, τους οποίους δεν εφυτεύσατε.
\par 14 Τώρα λοιπόν φοβήθητε τον Κύριον και λατρεύσατε αυτόν εν ακεραιότητι και αληθεία· και αποβάλετε τους θεούς, τους οποίους ελάτρευσαν οι πατέρες σας πέραν του ποταμού και εν τη Αιγύπτω, και λατρεύσατε τον Κύριον.
\par 15 Αλλ' εάν δεν αρέσκη εις εσάς να λατρεύητε τον Κύριον, εκλέξατε σήμερον ποίον θέλετε να λατρεύητε· ή τους θεούς, τους οποίους ελάτρευσαν οι πατέρες σας πέραν του ποταμού, ή τους θεούς των Αμορραίων, εις των οποίων την γην κατοικείτε· εγώ όμως και ο οίκός μου θέλομεν λατρεύει τον Κύριον.
\par 16 Και απεκρίθη ο λαός λέγων, Μη γένοιτο να αφήσωμεν τον Κύριον, διά να λατρεύσωμεν άλλους θεούς·
\par 17 διότι Κύριος ο Θεός ημών, αυτός ανεβίβασεν ημάς και τους πατέρας ημών εκ γης Αιγύπτου, εξ οίκου δουλείας, και αυτός έκαμεν ενώπιον ημών εκείνα τα σημεία τα μεγάλα, και διεφύλαξεν ημάς καθ' όλην την οδόν την οποίαν διωδεύσαμεν, και μεταξύ πάντων των εθνών διά των οποίων διέβημεν·
\par 18 και εξεδίωξεν ο Κύριος απ' έμπροσθεν ημών πάντας τους λαούς και τους Αμορραίους τους κατοικούντας εν τη γή· και ημείς θέλομεν λατρεύει τον Κύριον· διότι αυτός είναι Θεός ημών.
\par 19 Και είπεν ο Ιησούς προς τον λαόν, Δεν θέλετε δυνηθή να λατρεύητε τον Κύριον· διότι αυτός είναι Θεός άγιος· είναι Θεός ζηλωτής· δεν θέλει συγχωρήσει τας ανομίας σας και τας αμαρτίας σας·
\par 20 διότι θέλετε εγκαταλείψει τον Κύριον και λατρεύσει ξένους Θεούς· τότε στραφείς θέλει σας κακώσει και θέλει σας εξολοθρεύσει, αφού σας αγαθοποίησε.
\par 21 Και είπεν ο λαός εις τον Ιησούν, Ουχί· αλλά τον Κύριον θέλομεν λατρεύει.
\par 22 Και είπεν ο Ιησούς προς τον λαόν, Σεις είσθε μάρτυρες εις εαυτούς ότι σεις εξελέξατε εις εαυτούς τον Κύριον, διά να λατρεύητε αυτόν. Και εκείνοι είπον, Μάρτυρες.
\par 23 Τώρα λοιπόν αποβάλετε τους ξένους θεούς, τους εν τω μέσω υμών, και κλίνατε την καρδίαν υμών προς Κύριον τον Θεόν του Ισραήλ.
\par 24 Και είπεν ο λαός προς τον Ιησούν, Κύριον τον Θεόν ημών θέλομεν λατρεύει και εις την φωνήν αυτού θέλομεν υπακούει.
\par 25 Και έκαμεν ο Ιησούς διαθήκην προς τον λαόν εν τη ημέρα εκείνη, και έθεσεν εις αυτούς νόμον και κρίσιν εν Συχέμ.
\par 26 Και έγραψεν ο Ιησούς τους λόγους τούτους εν τω βιβλίω του νόμου του Θεού· και λαβών λίθον μέγαν, έστησεν αυτόν εκεί υπό την δρυν, την πλησίον του αγιαστηρίου του Κυρίου.
\par 27 Και είπεν ο Ιησούς προς πάντα τον λαόν, Ιδού, ο λίθος ούτος θέλει είσθαι εις υμάς εις μαρτύριον, διότι αυτός ήκουσε πάντας τους λόγους του Κυρίου τους οποίους ελάλησε προς ημάς· θέλει είσθαι λοιπόν εις μαρτύριον εις εσάς, διά να μη αρνηθήτε τον Θεόν σας.
\par 28 Και απέστειλεν ο Ιησούς τον λαόν, έκαστον εις την κληρονομίαν αυτού.
\par 29 Και μετά τα πράγματα ταύτα, ετελεύτησεν Ιησούς ο υιός του Ναυή, ο δούλος του Κυρίου, ηλικίας εκατόν δέκα ετών.
\par 30 Και έθαψαν αυτόν εν τοις ορίοις της κληρονομίας αυτού εν Φαμνάθ-σαράχ, ήτις είναι εν τω όρει Εφραΐμ, προς βορράν του όρους Γαάς.
\par 31 Και ελάτρευσεν ο Ισραήλ τον Κύριον πάσας τας ημέρας του Ιησού και πάσας τας ημέρας των πρεσβυτέρων, οίτινες επέζησαν μετά τον Ιησούν και οίτινες εγνώρισαν πάντα τα έργα του Κυρίου, όσα έκαμεν υπέρ του Ισραήλ.
\par 32 Τα δε οστά του Ιωσήφ, τα οποία ανεβίβασαν οι υιοί Ισραήλ εξ Αιγύπτου, έθαψαν εν Συχέμ, εν τη μερίδι του αγρού την οποίαν ηγόρασεν ο Ιακώβ παρά των υιών του Εμμώρ, πατρός του Συχέμ, δι' εκατόν αργύρια, και έγεινε κληρονομία των υιών Ιωσήφ.
\par 33 Και ετελεύτησεν Ελεάζαρ ο υιός του Ααρών, και έθαψαν αυτόν εν τω λόφω του Φινεές του υιού αυτού, όστις εδόθη εις αυτόν εκ τω όρει Εφραΐμ.


\end{document}