\begin{document}

\title{Ruth}


\chapter{1}

\par 1 Και εν ταις ημέραις καθ' ας οι κριταί έκρινον, έγεινε πείνα εν τη γη. Και υπήγεν άνθρωπός τις από Βηθλεέμ Ιούδα να παροικήση εν γη Μωάβ, αυτός και η γυνή αυτού και οι δύο υιοί αυτού.
\par 2 Το δε όνομα του ανθρώπου ήτο Ελιμέλεχ, και το όνομα της γυναικός αυτού Ναομί, και το όνομα των δύο υιών αυτού Μααλών και Χελαιών, Εφραθαίοι εκ Βηθλεέμ Ιούδα. Και ήλθον εις γην Μωάβ και ήσαν εκεί.
\par 3 Και απέθανεν Ελιμέλεχ ο ανήρ της Ναομί· και έμεινεν αυτή και οι δύο υιοί αυτής.
\par 4 Και ούτοι έλαβον εις εαυτούς γυναίκας Μωαβίτιδας· το όνομα της μιας Ορφά και το όνομα της άλλης Ρούθ· και κατώκησαν εκεί έως δέκα έτη.
\par 5 Απέθανον δε αμφότεροι, ο Μααλών και ο Χελαιών· και εστερήθη η γυνή των δύο υιών αυτής και του ανδρός αυτής.
\par 6 Τότε εσηκώθη αυτή και αι νύμφαι αυτής και επέστρεψαν εκ της γης Μωάβ· διότι ήκουσεν εν γη Μωάβ, ότι επεσκέφθη ο Κύριος τον λαόν αυτού δίδων εις αυτούς άρτον.
\par 7 Και εξήλθεν εκ του τόπου όπου ήτο, και αι δύο νύμφαι αυτής μετ' αυτής και επορεύοντο την οδόν διά να επιστρέψωσιν εις γην Ιούδα.
\par 8 Είπε δε η Ναομί προς τας δύο νύμφας αυτής, Υπάγετε, επιστρέψατε εκάστη εις τον οίκον της μητρός αυτής. Ο Κύριος να κάμη έλεος εις εσάς, καθώς σεις εκάμετε εις τους αποθανόντας και εις εμέ·
\par 9 ο Κύριος να σας δώση να εύρητε ανάπαυσιν, εκάστη εν τω οίκω του ανδρός αυτής. Και εφίλησεν αυτάς· και αυταί ύψωσαν την φωνήν αυτών και έκλαυσαν.
\par 10 Και είπον προς αυτήν, Ουχί· αλλά μετά σου θέλομεν επιστρέψει εις τον λαόν σου.
\par 11 Και είπεν η Ναομί, Επιστρέψατε, θυγατέρες μου· διά τι να έλθητε μετ' εμού; μήπως έχω έτι υιούς εν τη κοιλία μου, διά να γείνωσιν άνδρες σας;
\par 12 επιστρέψατε, θυγατέρες μου, υπάγετε· διότι εγήρασα και δεν είμαι διά άνδρα· εάν έλεγον, Έχω ελπίδα, εάν μάλιστα υπανδρευόμην ταύτην την νύκτα και εγέννων έτι υιούς,
\par 13 σεις ηθέλετε προσμένει αυτούς εωσού μεγαλώσωσιν; ηθέλετε δι' αυτούς αναβάλει το να υπανδρευθήτε; μη, θυγατέρες μου· επειδή επικράνθην πολύ πλέον παρά σεις, ότι η χειρ του Κυρίου εξήλθε κατ' εμού.
\par 14 Εκείναι δε ύψωσαν την φωνήν αυτών και έκλαυσαν πάλιν· και κατεφίλησεν η Ορφά την πενθεράν αυτής· η δε Ρούθ επροσκολλήθη εις αυτήν.
\par 15 Και είπεν η Ναομί, Ιδού, η σύννυμφός σου επέστρεψε προς τον λαόν αυτής και προς τους θεούς αυτής· επίστρεψον και συ κατόπιν της συννύμφου σου.
\par 16 Αλλ' η Ρούθ είπε, Μη με ανάγκαζε να σε αφήσω, διά να αναχωρήσω απ' όπισθέν σου· διότι όπου αν συ υπάγης, και εγώ θέλω υπάγει· και όπου αν συ παραμείνης, και εγώ θέλω παραμείνει· ο λαός σου, λαός μου, και ο Θεός σου, Θεός μου·
\par 17 όπου αν αποθάνης, θέλω αποθάνει και εκεί θέλω ταφή· ούτω να κάμη ο Κύριος εις εμέ και ούτω να προσθέση, εάν άλλο τι παρά τον θάνατον χωρίση εμέ από σου.
\par 18 Ιδούσα δε η Ναομί ότι αυτή διϊσχυρίζετο να υπάγη μετ' αυτής, έπαυσε να λαλή προς αυτήν.
\par 19 Περιεπάτησαν δε αμφότεραι, εωσού έφθασαν εις Βηθλεέμ. Και ότε έφθασαν εις Βηθλεέμ, πάσα η πόλις συνεκινήθη δι' αυτάς, και αι γυναίκες έλεγον, Αύτη είναι η Ναομί;
\par 20 Και αυτή είπε προς αυτάς, Μη με ονομάζετε Ναομί· ονομάζετέ με Μαρά· διότι ο Παντοδύναμος με επίκρανε σφόδρα·
\par 21 εγώ πλήρης ανεχώρησα, και κενήν επανήγαγέ με ο Κύριος· διά τι με ονομάζετε Ναομί, αφού ο Κύριος εμαρτύρησε κατ' εμού και ο Παντοδύναμος με κατέθλιψεν;
\par 22 Επέστρεψε λοιπόν η Ναομί, και μετ' αυτής Ρούθ η Μωαβίτις η νύμφη αυτής ελθούσα εκ γης Μωάβ· και αύται έφθασαν εις Βηθλεέμ εν τη αρχή του θερισμού των κριθών.

\chapter{2}

\par 1 Είχε δε η Ναομί συγγενή τινά του ανδρός αυτής, άνθρωπον δυνατόν εν ισχύϊ, εκ της συγγενείας του Ελιμέλεχ· και το όνομα αυτού Βοόζ.
\par 2 Και είπεν η Ρούθ η Μωαβίτις προς την Ναομί, Ας υπάγω, παρακαλώ, εις τον αγρόν διά να συνάξω αστάχυα κατόπιν ούτινος εύρω χάριν εις τους οφθαλμούς· και είπε προς αυτήν, Ύπαγε, θυγάτηρ μου.
\par 3 Και υπήγε και ελθούσα εσταχυολόγει εν τω αγρώ κατόπιν των θεριστών· και έτυχεν εν μέρει του αγρού του Βοόζ, όστις ήτο εκ της συγγενείας του Ελιμέλεχ.
\par 4 Και ιδού, ο Βοόζ ήλθεν εκ Βηθλεέμ και είπε προς τους θεριστάς, Κύριος μεθ' υμών. Και απεκρίθησαν προς αυτόν, Κύριος να σε ευλογήση.
\par 5 Τότε είπεν ο Βοόζ προς τον υπηρέτην αυτού, τον επιστάτην των θεριστών, Τίνος είναι η νέα αύτη;
\par 6 Και ο υπηρέτης ο επιστάτης των θεριστών απεκρίθη και είπεν, είναι η νέα η Μωαβίτις, η επιστρέψασα μετά της Ναομί εκ γης Μωάβ·
\par 7 και είπεν, Ας σταχυολογήσω, παρακαλώ, και ας συνάξω τι μεταξύ των δεματίων κατόπιν των θεριστών· και ήλθε και εστάθη από πρωΐας έως ταύτης της ώρας· ολίγον μόνον ανεπαύθη εν τη οικία.
\par 8 Και είπεν ο Βοόζ προς την Ρούθ, Δεν ακούεις, θυγάτηρ μου; μη υπάγης να σταχυολογήσης εν άλλω αγρώ, μηδέ να αναχωρήσης εντεύθεν, αλλά μένε ενταύθα μετά των κορασίων μου·
\par 9 ας ήναι οι οφθαλμοί σου επί τον αγρόν όπου θερίζουσι, και ύπαγε κατόπιν αυτών· δεν προσέταξα εγώ εις τους νέους να μη σε εγγίσωσι; και όταν διψήσης ύπαγε εις τα αγγεία και πίνε από ό,τι αντλήσωσιν οι νέοι.
\par 10 Η δε έπεσε κατά πρόσωπον και προσεκύνησεν έως εδάφους και είπε προς αυτόν, Πως εγώ εύρηκα χάριν εις τους οφθαλμούς σου, ώστε να λάβης πρόνοιαν περί εμού, ενώ είμαι ξένη;
\par 11 Και ο Βοόζ απεκρίθη και είπε προς αυτήν, Ανηγγέλθησαν προς εμέ ακριβώς πάντα όσα έκαμες εις την πενθεράν σου μετά τον θάνατον του ανδρός σου· και ότι αφήκας τον πατέρα σου και την μητέρα σου και την γην της γεννήσεώς σου, και ήλθες προς λαόν, τον οποίον δεν εγνώριζες πρότερον·
\par 12 ο Κύριος να ανταμείψη το έργον σου, και ο μισθός σου να ήναι πλήρης παρά Κυρίου του Θεού του Ισραήλ, υπό τας πτέρυγας του οποίου ήλθες να σκεπασθής.
\par 13 Η δε είπεν, Ας εύρω χάριν εις τους οφθαλμούς σου, κύριέ μου· επειδή με παρηγόρησας και επειδή ελάλησας ευμενώς προς την δούλην σου, αν και εγώ δεν είμαι ουδέ ως μία των θεραπαινίδων σου.
\par 14 Και είπε προς αυτήν ο Βοόζ την ώραν του φαγητού, Ελθέ και φάγε εκ του άρτου και βρέξον το ψωμίον σου εις το όξος. Και αυτή εκάθισεν εις τα πλάγια των θεριστών· εκείνος δε έδωκεν εις αυτήν σίτον πεφρυγανισμένον, και έφαγε και εχορτάσθη και επερίσσευσε.
\par 15 Και εσηκώθη να σταχυολογήση, και προσέταξεν ο Βοόζ εις τους νέους αυτού, λέγων, Και μεταξύ των δεματίων ας σταχυολογή, και μη επιπλήττετε αυτήν·
\par 16 και μάλιστα αφίνετε να πίπτη τι από των χειροβόλων διά αυτήν και αφίνετε να συλλέγη και μη ελέγχετε αυτήν.
\par 17 Και εσταχυολόγησεν εν τω αγρώ έως εσπέρας και εκοπάνισεν όσον εσταχυολόγησε· και ήτο έως εν εφά κριθής.
\par 18 Και εσήκωσεν αυτό και εισήλθεν εις την πόλιν· και είδεν η πενθερά αυτής όσον εσταχυολόγησε· και εκβαλούσα η Ρούθ, έδωκεν εις αυτήν ό,τι είχε περισσεύσει αφού εχορτάσθη.
\par 19 Και είπε προς αυτήν η πενθερά αυτής, Που εσταχυολόγησας σήμερον; και που εδούλευσας; ευλογημένος να ήναι εκείνος όστις έλαβε πρόνοιαν περί σου. Και εκείνη εφανέρωσε προς την πενθεράν αυτής εις τίνος αγρόν εδούλευσε και είπε, το όνομα του ανθρώπου, εις τον οποίον εδούλευσα σήμερον, είναι Βοόζ.
\par 20 Και είπεν η Ναομί προς την νύμφην αυτής, Ευλογημένος παρά Κυρίου εκείνος όστις δεν αφήκε το έλεος αυτού προς τους ζώντας και προς τους τεθνεώτας. Και είπε προς αυτήν η Ναομί, Συγγενής ημών είναι ο άνθρωπος ούτος εκ των πλησίον συγγενών ημών.
\par 21 Και είπεν η Ρούθ η Μωαβίτις, Αυτός με είπε προσέτι, Συ θέλεις μένει μετά των ανθρώπων μου, εωσού τελειώσωσιν όλον τον θερισμόν μου.
\par 22 Και είπεν η Ναομί προς την Ρούθ την νύμφην αυτής, Είναι καλόν, θυγάτηρ μου, να εκβαίνης μετά των κορασίων αυτού, και να μη σε απαντήσωσιν εν άλλω αγρώ.
\par 23 Και προσεκολλήθη εις τα κοράσια του Βοόζ διά να σταχυολογή, εωσού τελειώση ο θερισμός των κριθών και ο θερισμός του σίτου· και εκάθητο μετά της πενθεράς αυτής.

\chapter{3}

\par 1 Και είπε προς αυτήν η Ναομί η πενθερά αυτής, Θυγάτηρ μου, να μη ζητήσω ανάπαυσιν εις σε διά να ευημερήσης;
\par 2 και τώρα, μήπως δεν είναι Βοόζ εκ της συγγενείας ημών, μετά των κορασίων του οποίου ήσο; ιδού, αυτός λικμίζει ταύτην την νύκτα το αλώνιον των κριθών·
\par 3 λούσθητι λοιπόν και αλείφθητι και ενδύθητι την στολήν σου και κατάβα εις το αλώνιον· μη γνωρισθής εις τον άνθρωπον, εωσού τελειώση από του να φάγη και να πίη·
\par 4 και ενώ πλαγιάζει, παρατήρησον τον τόπον όπου πλαγιάζει, και ελθούσα σήκωσον το σκέπασμα από των ποδών αυτού, και πλαγίασον· και εκείνος θέλει σοι ειπεί τι να κάμης.
\par 5 Η δε είπε προς αυτήν, Πάντα όσα λέγεις εις εμέ, θέλω κάμει.
\par 6 Και κατέβη εις το αλώνιον και έκαμε πάντα όσα προσέταξεν εις αυτήν η πενθερά αυτής.
\par 7 Και αφού ο Βοόζ έφαγε και έπιε, και ευφράνη η καρδία αυτού, υπήγε να πλαγιάση εις την άκραν του σωρού του σίτου· εκείνη δε ήλθε κρυφίως και εσήκωσε ο σκέπασμα από των ποδών αυτού και επλαγίασε.
\par 8 Και προς το μεσονύκτιον εξέστη ο άνθρωπος και συνεταράχθη· και ιδού, γυνή εκοιμάτο παρά τους πόδας αυτού.
\par 9 Και είπε, Ποία είσαι συ; Εκείνη δε απεκρίθη, Εγώ η Ρούθ η δούλη σου· άπλωσον λοιπόν την πτέρυγά σου επί την δούλην σου· διότι είσαι ο πλησιέστερος συγγενής μου.
\par 10 Ο δε είπεν, Ευλογημένη να ήσαι παρά Κυρίου, θύγατερ· διότι έδειξας περισσοτέραν αγαθωσύνην εσχάτως παρά πρότερον, μη υπάγουσα κατόπιν νέων, είτε πτωχών είτε πλουσίων·
\par 11 και τώρα, θύγατερ, μη φοβού· θέλω κάμει εις σε παν ό,τι είπης· διότι πάσα η πόλις του λαού μου εξεύρει ότι είσαι γυνή ενάρετος·
\par 12 και τώρα είναι αληθές ότι εγώ είμαι στενός συγγενής· είναι όμως άλλος συγγενής πλησιέστερος εμού·
\par 13 μείνον ταύτην την νύκτα· και το πρωΐ εάν αυτός θέλη να εκπληρώση προς σε το χρέος το συγγενικόν, καλόν· ας το εκπληρώση· αλλ' εάν δεν θέλη να εκπληρώση προς σε το χρέος το συγγενικόν, τότε εγώ θέλω εκπληρώσει τούτο προς σε, ζη Κύριος· κοιμήθητι έως πρωΐ.
\par 14 Και εκοιμήθη παρά τους πόδας αυτού έως πρωΐ· και εσηκώθη πριν διακρίνη άνθρωπος άνθρωπον. Και εκείνος είπεν, Ας μη γνωρισθή ότι ήλθεν η γυνή εις το αλώνιον.
\par 15 Είπε προσέτι, Φέρε το περικάλυμμα το επάνω σου και κράτει αυτό. Και εκείνη εκράτει αυτό, και αυτός εμέτρησεν εξ μέτρα κριθής και έβαλεν επ' αυτήν· και υπήγεν εις την πόλιν.
\par 16 Και ότε ήλθε προς την πενθεράν αυτής, εκείνη είπε, Τι έγεινεν εις σε, θυγάτηρ μου; Και αυτή ανήγγειλε προς αυτήν πάντα όσα έκαμεν εις αυτήν ο άνθρωπος·
\par 17 και είπεν, Έδωκεν εις εμέ ταύτα τα εξ μέτρα της κριθής· διότι, Δεν θέλεις υπάγει, μοι είπε, κενή προς την πενθεράν σου.
\par 18 Η δε είπε, Κάθου, θυγάτηρ μου, εωσού ίδης πως θέλει τελειώσει το πράγμα· διότι ο άνθρωπος δεν θέλει ησυχάσει, εωσού τελειώση το πράγμα σήμερον.

\chapter{4}

\par 1 Και ανέβη ο Βοόζ εις την πύλην και εκάθισεν εκεί· και ιδού, διέβαινεν ο συγγενής, περί του οποίου ώμίλησεν ο Βοόζ. Και είπεν, Ω συ, στρέψον, κάθισον ενταύθα. Και εστράφη και εκάθισε.
\par 2 Και έλαβεν ο Βοόζ δέκα άνδρας εκ των πρεσβυτέρων της πόλεως, και είπε, Καθίσατε ενταύθα. Και εκάθισαν.
\par 3 Και είπε προς τον συγγενή, Η Ναομί, η επιστρέψασα εκ γης Μωάβ, πωλεί το μερίδιον του αγρού, το οποίον ήτο του αδελφού ημών Ελιμέλεχ·
\par 4 και εγώ είπα να σε ειδοποιήσω, λέγων, Αγόρασον αυτό έμπροσθεν των κατοίκων και έμπροσθεν των πρεσβυτέρων του λαού μου· εάν θέλης να εξαγοράσης αυτό ως συγγενής, εξαγόρασον· αλλ' εάν δεν θέλης να εξαγοράσης αυτό, ειπέ προς εμέ, διά να εξεύρω· διότι δεν είναι άλλος να εξαγοράση αυτό ως συγγενής παρά σύ· και εγώ είμαι μετά σε. Ο δε είπεν, Εγώ θέλω εξαγοράσει αυτό.
\par 5 Και είπεν ο Βοόζ, Καθ' ην ημέραν αγοράσης τον αγρόν εκ της χειρός της Ναομί, πρέπει να λάβης και την Ρούθ την Μωαβίτιν, γυναίκα του αποθανόντος, διά να αναστήσης το όνομα του αποθανόντος επί της κληρονομίας αυτού.
\par 6 Και είπεν ο συγγενής, Δεν δύναμαι να εκπληρώσω το χρέος το συγγενικόν, μήποτε φθείρω την κληρονομίαν μου· εκπλήρωσον συ το χρέος μου το συγγενικόν, διότι δεν δύναμαι εγώ να εκπληρώσω αυτό.
\par 7 Ούτος δε ήτο ο τρόπος το πάλαι εν τω Ισραήλ περί του δικαιώματος της συγγενείας και περί της απαλλοτριώσεως, διά να βεβαιούται πας λόγος· ο άνθρωπος λύων το υπόδημα αυτού, έδιδεν εις τον πλησίον αυτού· και τούτο ήτο μαρτυρία εν τω Ισραήλ.
\par 8 Διά τούτο είπεν ο συγγενής προς τον Βοόζ, Αγόρασον αυτό εις σεαυτόν. Και έλυσε το υπόδημα αυτού.
\par 9 Τότε είπεν ο Βοόζ προς τους πρεσβυτέρους και πάντα τον λαόν, Μάρτυρες είσθε σήμερον, ότι ηγόρασα πάντα τα του Ελιμέλεχ και πάντα τα του Χελαιών και Μααλών, εκ της χειρός της Ναομί·
\par 10 και προσέτι, την Ρούθ την Μωαβίτιν την γυναίκα του Μααλών, έλαβον εις εμαυτόν διά γυναίκα, διά να αναστήσω το όνομα του αποθανόντος επί της κληρονομίας αυτού, διά να μη εξαλειφθή το όνομα του αποθανόντος εκ των αδελφών αυτού και εκ της πόλεως της κατοικίας αυτού· μάρτυρες είσθε σήμερον.
\par 11 Και πας ο λαός ο εν τη πύλη και οι πρεσβύτεροι είπαν, Μάρτυρες· ο Κύριος να κάμη την γυναίκα, ήτις εισέρχεται εις τον οίκόν σου, ως την Ραχήλ και ως την Λείαν, αίτινες ωκοδόμησαν αμφότεραι τον οίκον Ισραήλ· και ίσχυε εν Εφραθά και έσο περίφημος εν Βηθλεέμ·
\par 12 και ας γείνη ο οίκός σου ως ο οίκος του Φαρές, τον οποίον εγέννησεν η Θάμαρ εις τον Ιούδαν, εκ του σπέρματος το οποίον ο Κύριος θέλει δώσει εις σε εκ της νέας ταύτης.
\par 13 Και έλαβεν ο Βοόζ την Ρούθ, και έγεινε γυνή αυτού· και ότε εισήλθε προς αυτήν, ο Κύριος έδωκεν εις αυτήν σύλληψιν, και εγέννησεν υιόν.
\par 14 Και είπαν αι γυναίκες προς την Ναομί, Ευλογητός ο Κύριος, όστις σήμερον δεν σε απεστέρησε συγγενούς, ώστε το όνομα αυτού να καλήται εν τω Ισραήλ·
\par 15 και ούτος θέλει είσθαι εις σε αναψυχωτής της ζωής και θέλει θρέψει την πολιάν σου· διότι εγέννησεν αυτόν η νύμφη σου, ήτις σε αγαπά, ήτις είναι εις σε καλητέρα παρά επτά υιούς.
\par 16 Τότε έλαβεν η Ναομί το παιδίον και έθεσεν αυτό εις τον κόλπον αυτής και έγεινεν εις αυτό τροφός.
\par 17 Και αι γείτονες έδωκαν εις αυτό όνομα, λέγουσαι, Υιός εγεννήθη εις την Ναομί· και εκάλεσαν το όνομα αυτού Ωβήδ· ούτος είναι ο πατήρ του Ιεσσαί πατρός του Δαβίδ.
\par 18 Και αύτη είναι η γενεαλογία του Φαρές· ο Φαρές εγέννησε τον Εσρών,
\par 19 Εσρών δε εγέννησε τον Αράμ, Αράμ δε εγέννησε τον Αμιναδάβ,
\par 20 Αμιναδάβ δε εγέννησε τον Ναασσών, Ναασσών δε εγέννησε τον Σαλμών,
\par 21 Σαλμών δε εγέννησε τον Βοόζ, Βοόζ δε εγέννησε τον Ωβήδ,
\par 22 Ωβήδ δε εγέννησε τον Ιεσσαί, και ο Ιεσσαί εγέννησε τον Δαβίδ.


\end{document}