\begin{document}

\title{1 Samuel}


\chapter{1}

\par 1 Ήτο δε άνθρωπός τις εκ Ραμαθαΐμ-σοφίμ, εκ του όρους Εφραΐμ, και το όνομα αυτού Ελκανά, υιός του Ιεροάμ, υιού Ελιού, υιού Θοού, υιού Σούφ, Εφραθαίος.
\par 2 Και είχεν ούτος δύο γυναίκας· το όνομα της μιας Άννα, και το όνομα της δευτέρας Φενίννα· η μεν Φενίννα είχε τέκνα, η δε Άννα δεν είχε τέκνα.
\par 3 Ανέβαινε δε ο άνθρωπος ούτος εκ της πόλεως αυτού κατ' έτος, διά να προσκυνήση και να προσφέρη θυσίαν προς τον Κύριον των δυνάμεων εν Σηλώ. Και ήσαν εκεί οι δύο υιοί του Ηλεί, Οφνεί και Φινεές, ιερείς του Κυρίου.
\par 4 Έφθασε δε η ημέρα, καθ' ην εθυσίασεν ο Ελκανά και έδωκε μερίδας εις την Φενίνναν την γυναίκα αυτού και εις πάντας τους υιούς αυτής και τας θυγατέρας αυτής.
\par 5 εις δε την Άνναν έδωκε διπλασίαν μερίδα· διότι ηγάπα την Ανναν· αλλ' ο Κύριος είχε κλείσει την μήτραν αυτής.
\par 6 Και η αντίζηλος αυτής παρώξυνεν αυτήν σφόδρα, ώστε να κάμνη αυτήν να αδημονή, ότι ο Κύριος είχε κλείσει την μήτραν αυτής.
\par 7 Και ούτως έκαμνε κατ' έτος· οσάκις ανέβαινεν εις τον οίκον του Κυρίου, ούτω παρώξυνεν αυτήν· και εκείνη έκλαιε και δεν έτρωγεν.
\par 8 Είπε δε προς αυτήν Ελκανά ο ανήρ αυτής, Άννα, διά τι κλαίεις; και διά τι δεν τρώγεις; και διά τι η καρδία σου είναι τεθλιμμένη; δεν είμαι εγώ εις σε καλήτερος παρά δέκα υιούς;
\par 9 Και εσηκώθη η Άννα, αφού έφαγον εν Σηλώ και αφού έπιον· ο δε Ηλεί ο ιερεύς εκάθητο επί καθέδρας, πλησίον του παραστάτου της πύλης του ναού του Κυρίου.
\par 10 Και αυτή ήτο καταπικραμένη την ψυχήν και προσηύχετο εις τον Κύριον, κλαίουσα καθ' υπερβολήν.
\par 11 Και ηυχήθη ευχήν, λέγουσα, Κύριε των δυνάμεων, εάν επιβλέψης τωόντι εις την ταπείνωσιν της δούλης σου και με ενθυμηθής και δεν λησμονήσης την δούλην σου, αλλά δώσης εις την δούλην σου τέκνον αρσενικόν, τότε θέλω δώσει αυτό εις τον Κύριον πάσας τας ημέρας της ζωής αυτού, και ξυράφιον δεν θέλει αναβή επί την κεφαλήν αυτού.
\par 12 Ενώ δε αυτή εξηκολούθει προσευχομένη ενώπιον του Κυρίου, ο Ηλεί παρετήρει το στόμα αυτής.
\par 13 Πλην η Άννα αυτή ελάλει εν τη καρδία αυτής· μόνον τα χείλη αυτής εκινούντο, αλλ' η φωνή αυτής δεν ηκούετο· όθεν ο Ηλεί ενόμισεν ότι ήτο μεθυσμένη.
\par 14 Και είπε προς αυτήν ο Ηλεί, Έως πότε θέλεις είσθαι μεθύουσα; απόβαλε τον οίνόν σου από σου.
\par 15 Και απεκρίθη η Άννα και είπεν, Ουχί, κύριέ μου, εγώ είμαι γυνή κατατεθλιμμένη την ψυχήν· ούτε οίνον ούτε σίκερα δεν έπιον, αλλ' εξέχεα την ψυχήν μου ενώπιον του Κυρίου·
\par 16 μη υπολάβης την δούλην σου ως αχρείαν γυναίκα· διότι εκ του πλήθους του πόνου μου και της θλίψεώς μου ελάλησα έως τώρα.
\par 17 Τότε απεκρίθη ο Ηλεί και είπεν, Ύπαγε εις ειρήνην· και ο Θεός του Ισραήλ ας σοι δώση την αίτησίν σου, την οποίαν ήτησας παρ' αυτού.
\par 18 Η δε είπεν, Είθε η δούλη σου να εύρη χάριν εις τους οφθαλμούς σου. Τότε απήλθεν η γυνή εις την οδόν αυτής και έφαγε, και το πρόσωπον αυτής δεν ήτο πλέον σκυθρωπόν.
\par 19 Και το πρωΐ εσηκώθησαν ενωρίς, και προσκυνήσαντες ενώπιον του Κυρίου, επέστρεψαν και ήλθον εις την οικίαν αυτών εις Ραμάθ. Και ο Ελκανά εγνώρισεν Άνναν την γυναίκα αυτού· και ο Κύριος ενεθυμήθη αυτήν.
\par 20 Και ότε επληρώθησαν αι ημέραι αφότου η Άννα συνέλαβεν, εγέννησεν υιόν και εκάλεσε το όνομα αυτού Σαμουήλ, Διότι παρά Κυρίου ήτησα αυτόν, είπε.
\par 21 Και ανέβη ο άνθρωπος Ελκανά και πας ο οίκος αυτού, διά να προσφέρη προς τον Κύριον την ετήσιον θυσίαν και την ευχήν αυτού.
\par 22 Αλλ' η Άννα δεν ανέβη· διότι είπε προς τον άνδρα αυτής, Δεν θέλω αναβή εωσού το παιδίον απογαλακτισθή· και τότε θέλω φέρει αυτό, διά να εμφανισθή ενώπιον του Κυρίου και εκεί να κατοική διαπαντός.
\par 23 Και είπε προς αυτήν Ελκανά ο ανήρ αυτής, Κάμε ό,τι σοι φαίνεται καλόν· κάθου εωσού απογαλακτίσης αυτό· μόνον ο Κύριος να εκπληρώση τον λόγον αυτού. Και εκάθισεν η γυνή και εθήλαζε τον υιόν αυτής, εωσού απεγαλάκτισεν αυτόν.
\par 24 Και αφού απεγαλάκτισεν αυτόν, ανεβίβασεν αυτόν μεθ' εαυτής, μετά τριών μόσχων και ενός εφά αλεύρου και ασκού οίνου, και έφερεν αυτόν εις τον οίκον του Κυρίου εν Σηλώ· το δε παιδίον ήτο μικρόν.
\par 25 Και έσφαξαν τον μόσχον και έφεραν το παιδίον προς τον Ηλεί.
\par 26 Και είπεν η Άννα, Ω, κύριέ μου ζη η ψυχή σου, κύριέ μου, εγώ είμαι η γυνή, ήτις εστάθη ενταύθα πλησίον σου, δεομένη του Κυρίου·
\par 27 περί του παιδίου τούτου εδεόμην· και ο Κύριος έδωκεν εις εμέ την αίτησίν μου, την οποίαν ήτησα παρ' αυτού·
\par 28 όθεν και εγώ εδάνεισα αυτό εις τον Κύριον· πάσας τας ημέρας της ζωής αυτού θέλει είσθαι δανεισμένον εις τον Κύριον. Και προσεκύνησεν εκεί τον Κύριον.

\chapter{2}

\par 1 Και προσηυχήθη η Άννα, και είπεν, Ευφράνθη η καρδία μου εις τον Κύριον· υψώθη το κέρας μου διά του Κυρίου· επλατύνθη το στόμα μου εναντίον των εχθρών μου· διότι ευφράνθην εις την σωτηρίαν σου.
\par 2 Δεν υπάρχει άγιος καθώς ο Κύριος· διότι δεν είναι άλλος πλην σου, ουδέ βράχος καθώς ο Θεός ημών.
\par 3 Μη καυχάσθε, μη λαλείτε υπερηφάνως· ας μη εξέλθη μεγαλορρημοσύνη εκ του στόματός σας· διότι ο Κύριος είναι Θεός γνώσεων, και παρ' αυτού σταθμίζονται αι πράξεις.
\par 4 Τα τόξα των δυνατών συνετρίβησαν, και οι αδύνατοι περιεζώσθησαν δύναμιν.
\par 5 Οι κεχορτασμένοι εμίσθωσαν εαυτούς διά άρτον· οι δε πεινώντες έπαυσαν· έως και η στείρα εγέννησεν επτά, η δε πολύτεκνος εξησθένησεν.
\par 6 Ο Κύριος θανατόνει και ζωοποιεί· καταβιβάζει εις τον άδην και αναβιβάζει.
\par 7 Ο Κύριος πτωχίζει και πλουτίζει, ταπεινόνει και υψόνει.
\par 8 Ανεγείρει τον πένητα από του χώματος, και ανυψόνει τον πτωχόν από της κοπρίας, διά να καθίση αυτούς μεταξύ των αρχόντων, και να κάμη αυτούς να κληρονομήσωσι θρόνον δόξης· διότι του Κυρίου είναι οι στύλοι της γης, και έστησε την οικουμένην επ' αυτούς.
\par 9 Θέλει φυλάττει τους πόδας των οσίων αυτού· οι δε ασεβείς θέλουσιν απολεσθή εν τω σκότει· επειδή διά δυνάμεως δεν θέλει υπερισχύσει ο άνθρωπος.
\par 10 Ο Κύριος θέλει συντρίψει τους αντιδίκους αυτού· εξ ουρανού θέλει βροντήσει επ' αυτούς· ο Κύριος θέλει κρίνει τα πέρατα της γής· και θέλει δώσει ισχύν εις τον βασιλέα αυτού, και υψώσει το κέρας του χριστού αυτού.
\par 11 Τότε ανεχώρησεν ο Ελκανά εις Ραμάθ προς τον οίκον αυτού. Το δε παιδίον υπηρέτει τον Κύριον ενώπιον Ηλεί του ιερέως.
\par 12 Του Ηλεί όμως οι υιοί ήσαν αχρείοι άνθρωποι δεν εγνώριζον τον Κύριον.
\par 13 Η συνήθεια δε των ιερέων προς τον λαόν ήτο τοιαύτη· ότε τις προσέφερε θυσίαν, ήρχετο ο υπηρέτης του ιερέως, ενώ εψήνετο το κρέας, έχων εις την χείρα αυτού κρεάγραν τριόδοντον·
\par 14 και εβύθιζεν αυτήν εις το κακκάβιον, ή εις τον λέβητα, ή εις την χύτραν, ή εις το χαλκείον· και ό,τι ανεβίβαζεν η κρεάγρα, ελάμβανεν ο ιερεύς δι' εαυτόν. Ούτως έκαμνον προς πάντας τους Ισραηλίτας τους ερχομένους εκεί εις Σηλώ.
\par 15 Πριν έτι καύσωσι το πάχος, ήρχετο ο υπηρέτης του ιερέως, και έλεγε προς τον άνθρωπον τον προσφέροντα την θυσίαν, Δος κρέας διά ψητόν εις τον ιερέα· διότι δεν θέλει να λάβη παρά σου κρέας βρασμένον, αλλά ωμόν.
\par 16 Και εάν ο άνθρωπος έλεγε προς αυτόν, Ας καύσωσι πρώτον το πάχος, και έπειτα λάβε όσον επιθυμεί η ψυχή σου· τότε απεκρίνετο προς αυτόν, Ουχί, αλλά τώρα θέλεις δώσει ειδέ μη, θέλω λάβει μετά βίας.
\par 17 Διά τούτο η αμαρτία των νέων ήτο μεγάλη σφόδρα ενώπιον του Κυρίου· διότι οι άνθρωποι απεστρέφοντο την θυσίαν του Κυρίου.
\par 18 Ο δε Σαμουήλ υπηρέτει ενώπιον του Κυρίου, παιδάριον περιεζωσμένον λινούν εφόδ.
\par 19 Και η μήτηρ αυτού έκαμνεν εις αυτόν επένδυμα μικρόν, και έφερε προς αυτόν κατ' έτος, ότε ανέβαινε μετά του ανδρός αυτής διά να προσφέρη την ετήσιον θυσίαν.
\par 20 Και ευλόγησεν ο Ηλεί τον Ελκανά και την γυναίκα αυτού, λέγων, Ο Κύριος να αποδώση εις σε σπέρμα εκ της γυναικός ταύτης, αντί του δανείου το οποίον εδάνεισεν εις τον Κύριον. Και ανεχώρησαν εις τον τόπον αυτών.
\par 21 Επεσκέφθη δε ο Κύριος την Ανναν· και συνέλαβε και εγέννησε τρεις υιούς και δύο θυγατέρας· το δε παιδίον ο Σαμουήλ εμεγάλονεν ενώπιον του Κυρίου.
\par 22 Ήτο δε ο Ηλεί πολύ γέρων· και ήκουσε πάντα όσα έπραττον οι υιοί αυτού εις πάντα τον Ισραήλ· και ότι εκοιμώντο μετά των γυναικών, των συνερχομένων εις την θύραν της σκηνής του μαρτυρίου.
\par 23 Και είπε προς αυτούς, Διά τι κάμνετε τοιαύτα πράγματα; διότι εγώ ακούω κακά πράγματα διά σας παρά παντός τούτου του λαού·
\par 24 μη, τέκνα μου· διότι δεν είναι καλή η φήμη, την οποίαν εγώ ακούω· σεις κάμνετε τον λαόν του Κυρίου να γίνηται παραβάτης·
\par 25 εάν αμαρτήση άνθρωπος εις άνθρωπον, θέλει ικεσία γίνεσθαι εις τον Θεόν υπέρ αυτού· αλλ' εάν τις αμαρτήση εις τον Κύριον, τις θέλει ικετεύσει υπέρ αυτού; Εκείνοι όμως δεν υπήκουον εις την φωνήν του πατρός αυτών· διότι ο Κύριος ήθελε να θανατώση αυτούς.
\par 26 Το δε παιδίον ο Σαμουήλ εμεγάλονε και ευηρέστει και εις τον Κύριον και εις τους ανθρώπους.
\par 27 Ήλθε δε άνθρωπός τις του Θεού προς τον Ηλεί και είπε προς αυτόν, Ούτω λέγει Κύριος· Δεν απεκαλύφθην φανερά εις τον οίκον του πατρός σου, ότε αυτοί ήσαν εν τη Αιγύπτω εν τω οίκω του Φαραώ;
\par 28 Και δεν εξέλεξα αυτόν εκ πασών των φυλών του Ισραήλ εις εμαυτόν διά ιερέα, διά να κάμνη προσφοράς επί του θυσιαστηρίου μου, να καίη θυμίαμα, να φορή εφόδ ενώπιόν μου; και δεν έδωκα εις τον οίκον του πατρός σου πάσας τας διά πυρός γινομένας προσφοράς των υιών Ισραήλ;
\par 29 Διά τι λακτίζετε εις την θυσίαν μου και εις την προσφοράν μου, την οποίαν προσέταξα να κάμνωσιν εν τω κατοικητηρίω μου, και δοξάζεις τους υιούς σου υπέρ εμέ, ώστε να παχύνησθε με το καλήτερον πασών των προσφορών του Ισραήλ του λαού μου;
\par 30 Διά τούτο Κύριος ο Θεός του Ισραήλ λέγει, Είπα βεβαίως ότι ο οίκός σου και ο οίκος του πατρός σου ήθελον περιπατεί ενώπιόν μου έως αιώνος· αλλά τώρα ο Κύριος λέγει, Μακράν απ' εμού· διότι τους δοξάζοντάς με θέλω δοξάσει, οι δε καταφρονούντές με θέλουσιν ατιμασθή.
\par 31 Ιδού, έρχονται ημέραι, ότε θέλω κόψει τον βραχίονά σου και τον βραχίονα του οίκου του πατρός σου, ώστε άνθρωπος γέρων δεν θέλει είσθαι εν τω οίκω σου.
\par 32 Και θέλεις ιδεί εν τω κατοικητηρίω μου αντίπαλον, μεταξύ πάντων των διδομένων αγαθών εις τον Ισραήλ· και δεν θέλει υπάρχει γέρων εν τω οίκω σου εις τον αιώνα.
\par 33 Όντινα δε εκ των ιδικών σου δεν αποκόψω από του θυσιαστηρίου μου, θέλει είσθαι διά να καταναλίσκη τους οφθαλμούς σου και να κατατήκη την ψυχήν σου· πάντες δε οι έκγονοι του οίκου σου θέλουσι τελευτά εις ανδρικήν ηλικίαν.
\par 34 Και τούτο θέλει είσθαι σημείον εις σε, το οποίον θέλει ελθεί επί τους δύο υιούς σου, επί Οφνεί και Φινεές· εν μιά ημέρα θέλουσιν αποθάνει αμφότεροι.
\par 35 Και θέλω ανεγείρει εις εμαυτόν ιερέα πιστόν, πράττοντα κατά την καρδίαν μου και κατά την ψυχήν μου· και θέλω οικοδομήσει εις αυτόν οίκον ασφαλή· και θέλει περιπατεί ενώπιον του χριστού μου εις τον αιώνα.
\par 36 Και πας ο εναπολειφθείς εν τω οίκω σου θέλει έρχεσθαι προσπίπτων εις αυτόν διά ολίγον αργύριον και διά κομμάτιον ψωμίου, και θέλει λέγει, Διόρισόν με, παρακαλώ, εις τινά των ιερατικών υπηρεσιών, διά να τρώγω ολίγον άρτον.

\chapter{3}

\par 1 Και το παιδίον ο Σαμουήλ υπηρέτει τον Κύριον έμπροσθεν του Ηλεί. Ο λόγος δε του Κυρίου ήτο σπάνιος κατ' εκείνας τας ημέρας· όρασις δεν εφαίνετο.
\par 2 Κατ' εκείνον δε τον καιρόν, ότε ο Ηλεί εκοίτετο εν τω τόπω αυτού, και οι οφθαλμοί αυτού ήσαν ημαυρωμένοι, ώστε δεν ηδύνατο να βλέπη,
\par 3 ο δε Σαμουήλ εκοίτετο εν τω ναώ του Κυρίου, όπου ήτο η κιβωτός του Θεού, πριν ο λύχνος του Θεού σβεσθή,
\par 4 εκάλεσεν ο Κύριος τον Σαμουήλ· ο δε απεκρίθη, Ιδού, εγώ.
\par 5 Και έτρεξε προς τον Ηλεί και είπεν, Ιδού, εγώ· διότι με εκάλεσας. Ο δε είπε, Δεν σε εκάλεσα· επίστρεψον να κοιμηθής. Και υπήγε να κοιμηθή.
\par 6 Ο δε Κύριος εκάλεσε πάλιν εκ δευτέρου, Σαμουήλ. Και εσηκώθη ο Σαμουήλ και υπήγε προς τον Ηλεί και είπεν, Ιδού, εγώ· διότι με εκάλεσας. Ο δε απεκρίθη, Δεν σε εκάλεσα, τέκνον μου· επίστρεψον να κοιμηθής.
\par 7 Και Σαμουήλ δεν εγνώριζεν έτι τον Κύριον, και ο λόγος του Κυρίου δεν είχεν έτι αποκαλυφθή εις αυτόν.
\par 8 Και εκάλεσεν ο Κύριος τον Σαμουήλ πάλιν εκ τρίτου. Και εσηκώθη και υπήγε προς τον Ηλεί και είπεν, Ιδού, εγώ· διότι με εκάλεσας. Και ενόησεν ο Ηλεί ότι ο Κύριος εκάλεσε το παιδίον.
\par 9 Και είπεν ο Ηλεί προς τον Σαμουήλ, Ύπαγε να κοιμηθής· και εάν σε κράξη, θέλεις ειπεί, Λάλησον, Κύριε· διότι ο δούλός σου ακούει. Και ο Σαμουήλ υπήγε και εκοιμήθη εν τω τόπω αυτού.
\par 10 Και ήλθεν ο Κύριος και σταθείς εκάλεσε καθώς το πρότερον, Σαμουήλ, Σαμουήλ. Τότε ο Σαμουήλ απεκρίθη, Λάλησον, διότι ο δούλός σου ακούει.
\par 11 Και είπεν ο Κύριος προς τον Σαμουήλ, Ιδού, εγώ θέλω κάμει εις τον Ισραήλ πράγμα, ώστε παντός ακούοντος αυτό θέλουσιν ηχήσει αμφότερα τα ώτα·
\par 12 εν εκείνη τη ημέρα θέλω εκτελέσει εναντίον του Ηλεί πάντα όσα ελάλησα περί του οίκου αυτού· θέλω αρχίσει και θέλω επιτελέσει,
\par 13 διότι ανήγγειλα προς αυτόν, ότι εγώ θέλω κρίνει τον οίκον αυτού έως αιώνος διά την ανομίαν· επειδή γνωρίσας ότι οι υιοί αυτού έφερον κατάραν εφ' εαυτούς, δεν συνέστειλεν αυτούς·
\par 14 και διά τούτο ώμοσα εναντίον του οίκου του Ηλεί, ότι η ανομία των υιών του Ηλεί δεν θέλει καθαρισθή εις τον αιώνα διά θυσίας ουδέ διά προσφοράς.
\par 15 Και εκοιμήθη ο Σαμουήλ έως πρωΐας· έπειτα ήνοιξε τας θύρας του οίκου του Κυρίου. Και εφοβείτο ο Σαμουήλ να αναγγείλη την όρασιν προς τον Ηλεί.
\par 16 Εκάλεσε δε ο Ηλεί τον Σαμουήλ και είπε, Σαμουήλ, τέκνον μου. Ο δε απεκρίθη, Ιδού, εγώ.
\par 17 Και είπε, Ποίος είναι ο λόγος, ο λαληθείς προς σε; μη κρύψης αυτόν, παρακαλώ, απ' εμού· ούτω να κάμη εις σε ο Θεός και ούτω να προσθέση, εάν κρύψης απ' εμού τινά εκ πάντων των λόγων των λαληθέντων προς σε.
\par 18 Και ανήγγειλε προς αυτόν ο Σαμουήλ πάντας τους λόγους, και δεν έκρυψεν απ' αυτού ουδένα. Και είπεν ο Ηλεί, Αυτός είναι Κύριος· ας κάμη το αρεστόν εις τους οφθαλμούς αυτού.
\par 19 Και εμεγάλονεν ο Σαμουήλ· και ο Κύριος ήτο μετ' αυτού και δεν άφινε να πίπτη ουδείς εκ των λόγων αυτού εις την γην.
\par 20 Και πας ο Ισραήλ, από Δαν έως Βηρ-σαβεέ, εγνώρισεν ότι ο Σαμουήλ ήτο διωρισμένος εις το να ήναι προφήτης του Κυρίου.
\par 21 Και εξηκολούθησεν ο Κύριος να φανερόνηται εν Σηλώ· διότι απεκαλύπτετο ο Κύριος προς τον Σαμουήλ εν Σηλώ διά του λόγου του Κυρίου.

\chapter{4}

\par 1 Και έγεινε λόγος του Σαμουήλ προς πάντα τον Ισραήλ. Και εξήλθεν ο Ισραήλ εναντίον των Φιλισταίων εις μάχην, και εστρατοπέδευσαν πλησίον του Έβεν-έζερ· οι δε Φιλισταίοι εστρατοπέδευσαν εν Αφέκ.
\par 2 Και παρετάχθησαν οι Φιλισταίοι εναντίον του Ισραήλ· και ότε εξηπλώθη η μάχη, εκτυπήθη ο Ισραήλ έμπροσθεν των Φιλισταίων· και εφονεύθησαν εν τω πεδίω κατά την συμπλοκήν έως τέσσαρες χιλιάδες ανδρών.
\par 3 Ότε δε ήλθεν ο λαός εις το στρατόπεδον, είπον οι πρεσβύτεροι του Ισραήλ, Διά τι ο Κύριος επάταξεν ημάς σήμερον έμπροσθεν των Φιλισταίων; ας λάβωμεν προς εαυτούς από Σηλώ την κιβωτόν της διαθήκης του Κυρίου, και ελθούσα εν μέσω ημών θέλει σώσει ημάς εκ της χειρός των εχθρών ημών.
\par 4 Και απέστειλεν ο λαός εις Σηλώ, και εσήκωσαν εκείθεν την κιβωτόν της διαθήκης του Κυρίου των δυνάμεων, του καθημένου επί των χερουβείμ· και αμφότεροι οι υιοί του Ηλεί, Οφνεί και Φινεές, ήσαν εκεί μετά της κιβωτού της διαθήκης του Θεού.
\par 5 Και ότε ήλθεν η κιβωτός της διαθήκης του Κυρίου εις το στρατόπεδον, πας ο Ισραήλ ηλάλαξε μετά φωνής μεγάλης, ώστε αντήχησεν η γη.
\par 6 Και ακούσαντες οι Φιλισταίοι την φωνήν του αλαλαγμού, είπον, Τι σημαίνει η φωνή του μεγάλου τούτου αλαλαγμού εν τω στρατοπέδω των Εβραίων; Και έμαθον ότι η κιβωτός του Κυρίου ήλθεν εις το στρατόπεδον.
\par 7 Και εφοβήθησαν οι Φιλισταίοι, λέγοντες, Ο Θεός ήλθεν εις το στρατόπεδον. Και είπον, Ουαί εις ημάς. Διότι δεν εστάθη τοιούτον πράγμα χθές και προχθές·
\par 8 ουαί εις ημάς. Τις θέλει σώσει ημάς εκ της χειρός των θεών τούτων των ισχυρών; ούτοι είναι οι θεοί, οι πατάξαντες τους Αιγυπτίους εν πάση πληγή εν τη ερήμω·
\par 9 ενδυναμώθητε, Φιλισταίοι, και στάθητε ως άνδρες, διά να μη γείνητε δούλοι εις τους Εβραίους, καθώς αυτοί εστάθησαν δούλοι εις εσάς· στάθητε ως άνδρες, και πολεμήσατε αυτούς.
\par 10 Τότε οι Φιλισταίοι επολέμησαν· και εκτυπήθη ο Ισραήλ, και έφυγεν έκαστος εις την σκηνήν αυτού· και έγεινε σφαγή μεγάλη σφόδρα· και έπεσον εκ του Ισραήλ τριάκοντα χιλιάδες πεζοί.
\par 11 Και η κιβωτός του Θεού επιάσθη· και αμφότεροι οι υιοί του Ηλεί, Οφνεί και Φινεές, εθανατώθησαν.
\par 12 Και έδραμεν εκ της μάχης άνθρωπός τις εκ του Βενιαμίν, και ήλθεν εις Σηλώ την αυτήν ημέραν, έχων τα ιμάτια αυτού διεσχισμένα και χώμα επί την κεφαλήν αυτού.
\par 13 Και ότε ήλθεν, ιδού, ο Ηλεί εκάθητο επί της καθέδρας, κατά το πλάγιον της οδού, σκοπεύων· διότι η καρδία αυτού έτρεμε περί της κιβωτού του Θεού. Και ότε ο άνθρωπος ελθών εις την πόλιν ανήγγειλε ταύτα, ανεβόησε πάσα η πόλις.
\par 14 Και ακούσας ο Ηλεί την φωνήν της βοής, είπε, Τι σημαίνει η φωνή της βοής ταύτης; Και ο άνθρωπος ήλθε σπεύδων και ανήγγειλε προς τον Ηλεί.
\par 15 Ήτο δε ο Ηλεί ενενήκοντα οκτώ ετών· και οι οφθαλμοί αυτού ήσαν ημαυρωμένοι, ώστε δεν ηδύνατο να βλέπη.
\par 16 Και είπεν ο άνθρωπος προς τον Ηλεί, Εγώ είμαι ο ελθών εκ της μάχης, και έφυγον εγώ εκ της μάχης σήμερον. Και είπε, Τι έγεινε, τέκνον μου;
\par 17 Και απεκρίθη ο μηνυτής και είπεν, Έφυγεν ο Ισραήλ έμπροσθεν των Φιλισταίων, και έτι μεγάλη σφαγή έγεινεν εις τον λαόν· και προσέτι αμφότεροι οι υιοί σου, Οφνεί και Φινεές, απέθανον· και η κιβωτός του Θεού επιάσθη.
\par 18 Και καθώς ανέφερε περί της κιβωτού του Θεού, ο Ηλεί έπεσεν εκ της καθέδρας εις τα οπίσθια προς το πλάγιον της πύλης, και συνετρίβη ο τράχηλος αυτού, και απέθανε· διότι ήτο γέρων ο άνθρωπος και βαρύς. Έκρινε δε αυτός τον Ισραήλ τεσσαράκοντα έτη.
\par 19 Και η νύμφη αυτού, η γυνή του Φινεές, ούσα έγκυος, ετοίμη να γεννήση, ως ήκουσε την αγγελίαν, ότι η κιβωτός του Θεού επιάσθη και ότι ο πενθερός αυτής και ο ανήρ αυτής απέθανον, εκυρτώθη και εγέννησε· διότι ήλθον εις αυτήν οι πόνοι.
\par 20 Και καθ' ον καιρόν απέθνησκεν, αι γυναίκες αι παριστάμεναι είπον προς αυτήν, Μη φοβού· διότι εγέννησας υιόν. Εκείνη όμως δεν απεκρίθη ουδέ έβαλεν αυτό εις την καρδίαν αυτής.
\par 21 Και εκάλεσε το παιδίον Ιχαβώδ, λέγουσα, Η δόξα έφυγεν εκ του Ισραήλ· διότι η κιβωτός του Θεού επιάσθη, και διότι ο πενθερός αυτής και ο ανήρ αυτής απέθανον.
\par 22 Και είπεν, Η δόξα έφυγεν εκ του Ισραήλ· διότι επιάσθη η κιβωτός του Θεού.

\chapter{5}

\par 1 Οι δε Φιλισταίοι έλαβον την κιβωτόν του Θεού και έφεραν αυτήν από Έβεν-έζερ εις Άζωτον.
\par 2 Και έλαβον οι Φιλισταίοι την κιβωτόν του Θεού και έφεραν αυτήν εις τον οίκον του Δαγών, και έθεσαν αυτήν πλησίον του Δαγών.
\par 3 Και ότε οι Αζώτιοι εσηκώθησαν ενωρίς την επαύριον, ιδού, ο Δαγών πεσμένος κατά πρόσωπον αυτού επί της γης ενώπιον της κιβωτού του Κυρίου. Και λαβόντες τον Δαγών, κατέστησαν αυτόν εις τον τόπον αυτού.
\par 4 Και την επαύριον ότε εσηκώθησαν ενωρίς το πρωΐ, ιδού, ο Δαγών πεσμένος κατά πρόσωπον αυτού επί της γης ενώπιον της κιβωτού του Κυρίου· και η κεφαλή του Δαγών και αι δύο παλάμαι των χειρών αυτού αποκεκομμέναι επί του κατωφλίου· μόνον ο κορμός του Δαγών εναπέμεινεν εις αυτόν.
\par 5 Διά τούτο εν τη Αζώτω οι ιερείς του Δαγών, και πας ο εισερχόμενος εις τον οίκον του Δαγών, δεν πατούσιν εις το κατώφλιον του Δαγών έως της ημέρας ταύτης.
\par 6 Και επεβαρύνθη η χειρ του Κυρίου επί τους Αζωτίους, και εξωλόθρευσεν αυτούς και επάταξεν αυτούς με αιμορροΐδας, την Άζωτον και τα όρια αυτής.
\par 7 Και ότε είδον οι άνδρες της Αζώτου ότι έγεινεν ούτως, είπον, Η κιβωτός του Θεού του Ισραήλ δεν θέλει κατοικεί μεθ' ημών· διότι η χειρ αυτού εσκληρύνθη εφ' ημάς και επί τον Δαγών τον θεόν ημών.
\par 8 Όθεν αποστείλαντες εσύναξαν προς εαυτούς πάντας τους σατράπας των Φιλισταίων και είπον, Τι θέλομεν κάμει εις την κιβωτόν του Θεού του Ισραήλ; οι δε είπον, Η κιβωτός του Θεού του Ισραήλ ας μετακομισθή εις Γαθ. Και μετεκόμισαν την κιβωτόν του Θεού του Ισραήλ.
\par 9 Αφού δε μετεκόμισαν αυτήν, η χειρ του Κυρίου ήτο εναντίον της πόλεως με όλεθρον μέγαν σφόδρα· και επάταξε τους άνδρας της πόλεως, από μικρού έως μεγάλου, και εξεφύησαν εις αυτούς αιμορροΐδες.
\par 10 Διά τούτο απέστειλαν την κιβωτόν του Θεού εις Ακκαρών. Και ως ήλθεν η κιβωτός του Θεού εις Ακκαρών, οι Ακκαρωνίται εβόησαν, λέγοντες, Έφεραν την κιβωτόν του Θεού του Ισραήλ εις ημάς, διά να θανατώση ημάς και τον λαόν ημών.
\par 11 Και αποστείλαντες εσύναξαν πάντας τους σατράπας των Φιλισταίων και είπον, Αποπέμψατε την κιβωτόν του Θεού του Ισραήλ, και ας επιστρέψη εις τον τόπον αυτής, διά να μη θανατώση ημάς και τον λαόν ημών· διότι ήτο τρόμος θανάτου εφ' όλην την πόλιν· η χειρ του Θεού ήτο εκεί βαρεία σφόρα.
\par 12 Και οι άνδρες όσοι δεν απέθανον, εκτυπήθησαν από αιμορροΐδας· και η κραυγή της πόλεως ανέβη εις τον ουρανόν.

\chapter{6}

\par 1 Και ήτο η κιβωτός του Κυρίου εν τη γη των Φιλισταίων επτά μήνας.
\par 2 Και έκραξαν οι Φιλισταίοι τους ιερείς και τους μάντεις, λέγοντες, Τι να κάμωμεν εις την κιβωτόν του Κυρίου; φανερώσατε εις ημάς τίνι τρόπω θέλομεν αποστείλει αυτήν εις τον τόπον αυτής.
\par 3 Οι δε είπον, Εάν εξαποστείλητε την κιβωτόν του Θεού του Ισραήλ, μη αποστείλητε αυτήν κενήν· αλλά κατά πάντα τρόπον απόδοτε εις αυτόν προσφοράν περί ανομίας· τότε θέλετε ιαθή και θέλετε γνωρίσει διά τι η χειρ αυτού δεν απεσύρθη από σας.
\par 4 Και είπον, Ποία είναι η περί ανομίας προσφορά, την οποίαν θέλομεν αποδώσει εις αυτόν; Οι δε απεκρίθησαν, Κατά τον αριθμόν των σατραπών των Φιλισταίων, πέντε αιμορροΐδες χρυσαί και πέντε χρυσοί ποντικοί· διότι η αυτή πληγή ήτο επί πάντας υμάς και επί τους σατράπας υμών·
\par 5 διά τούτο θέλετε κάμει ομοιώματα των αιμορροΐδων σας και ομοιώματα των ποντικών σας των φθειρόντων την γήν· και θέλετε δώσει δόξαν εις τον Θεόν του Ισραήλ· ίσως ελαφρύνη την χείρα αυτού αφ' υμών και από των θεών υμών και από της γης υμών·
\par 6 διά τι λοιπόν σκληρύνετε τας καρδίας σας, καθώς οι Αιγύπτιοι και ο Φαραώ εσκλήρυναν τας καρδίας αυτών; ότε έκαμε τεράστια εν τω μέσω αυτών, δεν αφήκαν αυτούς να υπάγωσι, και αυτοί ανεχώρησαν;
\par 7 τώρα λοιπόν λάβετε και ετοιμάσατε μίαν άμαξαν νέαν και δύο βους θηλαζούσας, εις τας οποίας δεν επεβλήθη ζυγός, και ζεύξατε τας βους εις την άμαξαν, τους δε μόσχους αυτών επαναφέρετε απ' όπισθεν αυτών εις τον οίκον·
\par 8 και λάβετε την κιβωτόν του Κυρίου και θέσατε αυτήν επί της αμάξης· και τα σκεύη τα χρυσά, τα οποία αποδίδετε εις αυτόν προσφοράν περί ανομίας, θέσατε εν κιβωτίω εις τα πλάγια αυτής· και εξαποστείλατε αυτήν να υπάγη·
\par 9 και βλέπετε, εάν αναβαίνη διά της οδού των ορίων αυτής εις Βαιθ-σεμές, αυτός έκαμεν εις ημάς το μέγα τούτο κακόν· εάν δε μη, τότε θέλομεν γνωρίσει ότι δεν επάταξεν ημάς η χειρ αυτού, αλλ' ότι τούτο εστάθη τυχαίον εις ημάς.
\par 10 Και έκαμον ούτως οι άνδρες, και λαβόντες δύο βους θηλαζούσας, έζευξαν αυτάς εις την άμαξαν, τους δε μόσχους αυτών απέκλεισαν εν τω οίκω.
\par 11 Και έθεσαν την κιβωτόν του Κυρίου επί της αμάξης και το κιβώτιον μετά των χρυσών ποντικών και των ομοιωμάτων των αιμορροΐδων αυτών.
\par 12 Και διευθύνθησαν αι βους εις την οδόν την εις Βαιθ-σεμές· την αυτήν οδόν εξηκολούθουν, μυκώμεναι ενώ υπήγαινον, και δεν μετεστρέφοντο δεξιά ή αριστερά· οι δε σατράπαι των Φιλισταίων επορεύοντο κατόπιν αυτών έως των ορίων της Βαιθ-σεμές.
\par 13 Και οι Βαιθ-σεμίται εθέριζον τον σίτον αυτών εν τη κοιλάδι και υψώσαντες τους οφθαλμούς αυτών, είδον την κιβωτόν και ιδόντες υπερεχάρησαν.
\par 14 Και εισήλθεν η άμαξα εις τον αγρόν Ιησού του Βαιθ-σεμίτου και εστάθη εκεί, όπου ήτο λίθος μέγας· και έσχισαν τα ξύλα της αμάξης, και προσέφεραν τας βους ολοκαύτωμα εις τον Κύριον.
\par 15 Και οι Λευΐται κατεβίβασαν την κιβωτόν του Κυρίου και το κιβώτιον το μετ' αυτής, το περιέχον τα χρυσά σκεύη, και έθεσαν επί του λίθου του μεγάλου· και οι άνδρες της Βαιθ-σεμές προσέφεραν ολοκαυτώματα και έθυσαν θυσίας εις τον Κύριον την αυτήν ημέραν.
\par 16 Και αφού οι πέντε σατράπαι των Φιλισταίων είδον, επέστρεψαν εις Ακκαρών την αυτήν ημέραν.
\par 17 Αύται δε ήσαν αι αιμορροΐδες αι χρυσαί, τας οποίας οι Φιλισταίοι απέδωκαν προσφοράν περί ανομίας εις τον Κύριον· της Αζώτου μία, της Γάζης μία, της Ασκαλώνος μία, της Γαθ μία, της Ακκαρών μία·
\par 18 και οι ποντικοί οι χρυσοί κατά τον αριθμόν πασών των πόλεων των Φιλισταίων, των πέντε σατραπών, από πόλεων περιτετειχισμένων και κωμών απεριτειχίστων, έως μάλιστα του λίθου του μεγάλου, Αβέλ, επί του οποίου κατέθεσαν την κιβωτόν του Κυρίου· όστις σώζεται έως της ημέρας ταύτης εν τω αγρώ Ιησού του Βαιθ-σεμίτου.
\par 19 Και επάταξεν ο Κύριος τους άνδρας της Βαιθ-σεμές, διότι ενέβλεψαν εις την κιβωτόν του Κυρίου· και επάταξεν εκ του λαού άνδρας πεντήκοντα χιλιάδας και εβδομήκοντα· και επένθησεν ο λαός, διότι επάταξεν αυτόν ο Κύριος εν πληγή μεγάλη.
\par 20 Και είπαν οι άνδρες της Βαιθ-σεμές, Τις δύναται να σταθή ενώπιον του Κυρίου, του αγίου τούτου Θεού; και προς τίνα θέλει αναβή αφ' ημών;
\par 21 Και απέστειλαν μηνυτάς προς τους κατοίκους της Κιριάθ ιαρείμ, λέγοντες, Οι Φιλισταίοι έφεραν οπίσω την κιβωτόν του Κυρίου· κατάβητε, αναβιβάσατε αυτήν προς εαυτούς.

\chapter{7}

\par 1 Και ήλθον οι άνδρες της Κιριάθ-ιαρείμ, και ανεβίβασαν την κιβωτόν του Κυρίου και έφεραν αυτήν εις τον οίκον του Αβιναδάβ επί τον λόφον, και Ελεάζαρ τον υιόν αυτού καθιέρωσαν, διά να φυλάττη την κιβωτόν του Κυρίου.
\par 2 Και αφ' ης ημέρας ετέθη η κιβωτός εν Κιριάθ-ιαρείμ, παρήλθε καιρός πολύς· και έγειναν είκοσι έτη· και πας ο οίκος Ισραήλ εστέναζεν, αναζητών τον Κύριον.
\par 3 Και είπε Σαμουήλ προς πάντα τον οίκον Ισραήλ, λέγων, Εάν σεις επιστρέφητε εξ όλης υμών της καρδίας προς τον Κύριον, αποβάλετε εκ μέσου υμών τους Θεούς τους αλλοτρίους και τας Ασταρώθ, και ετοιμάσατε τας καρδίας υμών προς τον Κύριον και αυτόν μόνον λατρεύετε· και θέλει ελευθερώσει υμάς εκ χειρός των Φιλισταίων.
\par 4 Τότε απέβαλον οι υιοί Ισραήλ τους Βααλείμ και τας Ασταρώθ και ελάτρευσαν τον Κύριον μόνον.
\par 5 Και είπε Σαμουήλ, Συνάξατε πάντα τον Ισραήλ εις Μισπά, και θέλω προσευχηθή υπέρ υμών προς τον Κύριον.
\par 6 Και συνήχθησαν ομού εις Μισπά, και ήντλησαν ύδωρ και εξέχεαν ενώπιον του Κυρίου, και ενήστευσαν την ημέραν εκείνην και είπον εκεί, Ημαρτήσαμεν εις τον Κύριον. Και έκρινεν ο Σαμουήλ τους υιούς Ισραήλ εν Μισπά.
\par 7 Ότε δε ήκουσαν οι Φιλισταίοι ότι συνηθροίσθησαν οι υιοί Ισραήλ εις Μισπά ανέβησαν οι σατράπαι των Φιλισταίων κατά του Ισραήλ. Και ακούσαντες οι υιοί Ισραήλ, εφοβήθησαν από προσώπου των Φιλισταίων.
\par 8 Και είπον οι υιοί Ισραήλ προς τον Σαμουήλ, Μη παύσης βοών υπέρ ημών προς Κύριον τον Θεόν ημών, διά να σώση ημάς εκ χειρός των Φιλισταίων.
\par 9 Και έλαβεν ο Σαμουήλ εν αρνίον γαλαθηνόν, και προσέφερεν ολόκληρον ολοκαύτωμα εις τον Κύριον· και εβόησεν ο Σαμουήλ προς τον Κύριον υπέρ του Ισραήλ· και επήκουσεν αυτού ο Κύριος.
\par 10 Και ενώ προσέφερεν ο Σαμουήλ το ολοκαύτωμα, επλησίασαν οι Φιλισταίοι διά να πολεμήσωσι κατά του Ισραήλ· και εβρόντησεν ο Κύριος εν φωνή μεγάλη την ημέραν εκείνην επί τους Φιλισταίους και κατετρόπωσεν αυτούς· και εκτυπήθησαν έμπροσθεν του Ισραήλ.
\par 11 Και εξήλθον οι άνδρες Ισραήλ εκ Μισπά, και κατεδίωξαν τους Φιλισταίους και επάταξαν αυτούς, έως υποκάτω της Βαιθ-χαρ.
\par 12 Τότε έλαβεν ο Σαμουήλ ένα λίθον και έστησε μεταξύ Μισπά και Σεν και εκάλεσε το όνομα αυτού Έβεν-έζερ, λέγων, Μέχρι τούδε εβοήθησεν ημάς ο Κύριος.
\par 13 Και εταπεινώθησαν οι Φιλισταίοι και δεν ήλθον πλέον εις τα όρια του Ισραήλ· και ήτο η χειρ του Κυρίου κατά των Φιλισταίων πάσας τας ημέρας του Σαμουήλ.
\par 14 Και αι πόλεις, τας οποίας οι Φιλισταίοι είχον λάβει από του Ισραήλ, απεδόθησαν εις τον Ισραήλ, από Ακκαρών έως Γάθ· και ηλευθέρωσεν ο Ισραήλ τα όρια αυτών εκ χειρός των Φιλισταίων. Και ήτο ειρήνη μεταξύ Ισραήλ και Αμορραίων.
\par 15 Και έκρινεν ο Σαμουήλ τον Ισραήλ πάσας τας ημέρας της ζωής αυτού·
\par 16 και επορεύετο κατ' έτος περιερχόμενος εις Βαιθήλ και Γάλγαλα και Μισπά και έκρινε τον Ισραήλ εν πάσι τοις τόποις τούτοις·
\par 17 η δε επιστροφή αυτού ήτο εις Ραμά· διότι εκεί ήτο ο οίκος αυτού και εκεί έκρινε τον Ισραήλ· εκεί προσέτι ωκοδόμησε θυσιαστήριον εις τον Κύριον.

\chapter{8}

\par 1 Και ότε εγήρασεν ο Σαμουήλ, κατέστησε τους υιούς αυτού κριτάς επί τον Ισραήλ.
\par 2 Ήτο δε το όνομα του πρωτοτόκου υιού αυτού Ιωήλ και το όνομα του δευτέρου αυτού Αβιά· ούτοι ήσαν κριταί εν Βηρ-σαβεέ.
\par 3 Πλην δεν περιεπάτησαν οι υιοί αυτού εις τας οδούς αυτού, αλλ' εξέκλιναν οπίσω του κέρδους και εδωροδοκούντο και διέστρεφον την κρίσιν.
\par 4 Όθεν συνηθροίσθησαν πάντες οι πρεσβύτεροι του Ισραήλ και ήλθον προς τον Σαμουήλ εις Ραμά,
\par 5 και είπον προς αυτόν, Ιδού, συ εγήρασας, και οι υιοί σου δεν περιπατούσιν εις τας οδούς σου· κατάστησον λοιπόν εις ημάς βασιλέα διά να κρίνη ημάς, καθώς έχουσι πάντα τα έθνη.
\par 6 Το πράγμα όμως δεν ήρεσεν εις τον Σαμουήλ, ότι είπον, Δος εις ημάς βασιλέα διά να κρίνη ημάς. Και εδεήθη ο Σαμουήλ προς τον Κύριον.
\par 7 Και είπεν ο Κύριος προς τον Σαμουήλ, Άκουσον της φωνής του λαού, κατά πάντα όσα λέγουσι προς σέ· διότι δεν απέβαλον σε, αλλ' εμέ απέβαλον από του να βασιλεύω επ' αυτούς·
\par 8 κατά πάντα τα έργα τα οποία έπραξαν, αφ' ης ημέρας ανεβίβασα αυτούς εξ Αιγύπτου έως της ημέρας ταύτης, εγκαταλίποντές με και λατρεύσαντες άλλους θεούς, ούτω κάμνουσι και προς σέ·
\par 9 τώρα λοιπόν άκουσον της φωνής αυτών· πλην διαμαρτυρήθητι παρρησία προς αυτούς και δείξον εις αυτούς τον τρόπον του βασιλέως, όστις θέλει βασιλεύσει επ' αυτούς.
\par 10 Και ελάλησεν ο Σαμουήλ πάντας τους λόγους του Κυρίου προς τον λαόν, τον ζητούντα παρ' αυτού βασιλέα·
\par 11 και είπεν, Ούτος θέλει είσθαι ο τρόπος του βασιλέως, όστις θέλει βασιλεύσει εφ' υμάς· τους υιούς υμών θέλει λαμβάνει και διορίζει εις εαυτόν, διά τας αμάξας αυτού και διά ιππείς αυτού και διά να προτρέχωσι των αμαξών αυτού.
\par 12 Και θέλει διορίζει εις εαυτόν χιλιάρχους και πεντηκοντάρχους· και εις το να εργάζωνται την γην αυτού και να θερίζωσι τον θερισμόν αυτού, και να κατασκευάζωσι τα πολεμικά αυτού σκεύη και την σκευήν των αμαξών αυτού.
\par 13 Και τας θυγατέρας σας θέλει λαμβάνει διά μυρεψούς και μαγειρίσσας και αρτοποιούς·
\par 14 και τους αγρούς σας και τους αμπελώνάς σας και τους ελαιώνάς σας τους καλητέρους θέλει λάβει και δώσει εις τους δούλους αυτού.
\par 15 Και το δέκατον των σπαρτών σας και των αμπελώνων σας θέλει λαμβάνει και δίδει εις τους ευνούχους αυτού και εις τους δούλους αυτού.
\par 16 Και τους δούλους σας και τας δούλας σας και τους καλητέρους νέους σας και τους όνους σας θέλει λαμβάνει και διορίζει εις τας εργασίας αυτού.
\par 17 Τα ποίμνιά σας θέλει δεκατίζει· και σεις θέλετε είσθαι δούλοι αυτού.
\par 18 Και θέλετε βοά εν εκείνη τη ημέρα ένεκα του βασιλέως σας, τον οποίον σεις εκλέξατε εις εαυτούς· αλλ' ο Κύριος δεν θέλει σας επακούσει εν εκείνη τη ημέρα.
\par 19 Ο λαός όμως δεν ηθέλησε να υπακούση εις την φωνήν του Σαμουήλ· και είπον, Ουχί· αλλά βασιλεύς θέλει είσθαι εφ' ημάς·
\par 20 διά να ήμεθα και ημείς ως πάντα τα έθνη· και να κρίνη ημάς ο βασιλεύς ημών και να εξέρχηται έμπροσθεν ημών και να μάχηται τας μάχας ημών.
\par 21 Και ήκουσεν ο Σαμουήλ πάντας τους λόγους του λαού και ανέφερεν αυτούς εις τα ώτα του Κυρίου.
\par 22 Και είπεν ο Κύριος προς τον Σαμουήλ, Άκουσον της φωνής αυτών και κατάστησον επ' αυτούς βασιλέα. Και είπεν ο Σαμουήλ προς τους άνδρας του Ισραήλ, Υπάγετε έκαστος εις την πόλιν αυτού.

\chapter{9}

\par 1 Ήτο δε ανήρ τις εκ του Βενιαμίν, ονομαζόμενος Κείς, υιός του Αβιήλ, υιού του Σερώρ, υιού του Βεχωράθ, υιού του Αφιά, ανδρός Βενιαμίτου, δυνατός εν ισχύϊ.
\par 2 Είχε δε ούτος υιόν, ονομαζόμενον Σαούλ, εκλεκτόν και ώραίον· και δεν υπήρχε μεταξύ των υιών Ισραήλ άνθρωπος ώραιότερος αυτού· από των ώμων αυτού και επάνω εξείχεν υπέρ παντός του λαού.
\par 3 Και αι όνοι του Κείς πατρός του Σαούλ εχάθησαν· και είπεν ο Κείς προς τον Σαούλ τον υιόν αυτού, Λάβε τώρα μετά σου ένα των υπηρετών, και σηκωθείς ύπαγε να ζητήσης τας όνους.
\par 4 Και επέρασε διά του όρους Εφραΐμ και επέρασε διά της γης Σαλισά, αλλά δεν εύρηκαν αυτάς· και επέρασαν διά της γης Σααλείμ, πλην δεν ήσαν εκεί· και επέρασε διά της γης Ιεμινί, αλλά δεν εύρηκαν αυτάς.
\par 5 Ότε δε ήλθον εις την γην Σούφ, είπεν ο Σαούλ προς τον υπηρέτην αυτού τον μετ' αυτού, Ελθέ, και ας επιστρέψωμεν, μήποτε ο πατήρ μου, αφήσας την φροντίδα των όνων, συλλογίζηται περί ημών.
\par 6 Ο δε είπε προς αυτόν, Ιδού τώρα, εν τη πόλει ταύτη είναι άνθρωπος του Θεού, και ο άνθρωπος είναι ένδοξος· παν ό,τι είπη γίνεται εξάπαντος· ας υπάγωμεν λοιπόν εκεί· ίσως φανερώση εις ημάς την οδόν ημών, την οποίαν πρέπει να υπάγωμεν.
\par 7 Και είπεν ο Σαούλ προς τον υπηρέτην αυτού, Αλλ' ιδού, θέλομεν υπάγει, πλην τι θέλομεν φέρει προς τον άνθρωπον; διότι ο άρτος εξέλιπεν εκ των αγγείων ημών· και δώρον δεν υπάρχει να προσφέρωμεν εις τον άνθρωπον του Θεού· τι έχομεν;
\par 8 Και αποκριθείς πάλιν ο υπηρέτης προς τον Σαούλ, είπεν, Ιδού, ευρίσκεται εν τη χειρί μου εν τέταρτον σίκλου αργυρίου, το οποίον θέλω δώσει εις τον άνθρωπον του Θεού, και θέλει φανερώσει εις ημάς την οδόν ημών.
\par 9 Το πάλαι εν τω Ισραήλ, οπότε τις υπήγαινε να ερωτήση τον Θεόν, έλεγεν ούτως· Έλθετε, και ας υπάγωμεν έως εις τον βλέποντα· διότι ο σήμερον προφήτης εκαλείτο το πάλαι ο βλέπων.
\par 10 Τότε είπεν ο Σαούλ προς τον υπηρέτην αυτού, Καλός ο λόγος σου· ελθέ, ας υπάγωμεν. Υπήγαν λοιπόν εις την πόλιν, όπου ήτο ο άνθρωπος του Θεού.
\par 11 Και ενώ ανέβαινον το ανήφορον της πόλεως, εύρηκαν κοράσια εξερχόμενα διά να αντλήσωσιν ύδωρ· και είπον προς αυτά, Είναι ενταύθα ο βλέπων;
\par 12 Και εκείνα απεκρίθησαν προς αυτούς και είπον, Είναι ιδού, έμπροσθέν σου· τάχυνον λοιπόν· διότι σήμερον ήλθεν εις την πόλιν, επειδή είναι σήμερον θυσία του λαού επί του υψηλού τόπου·
\par 13 ευθύς όταν εισέλθητε εις την πόλιν, θέλετε ευρεί αυτόν, πριν αναβή εις τον υψηλόν τόπον διά να φάγη· διότι ο λαός δεν τρώγει εωσού έλθη αυτός, επειδή ούτος ευλογεί την θυσίαν· μετά ταύτα τρώγουσιν οι κεκλημένοι τώρα λοιπόν ανάβητε· διότι περί την ώραν ταύτην θέλετε ευρεί αυτόν.
\par 14 Και ανέβησαν εις την πόλιν· και ενώ εισήρχοντο εις την πόλιν, ιδού, ο Σαμουήλ εξήρχετο ενώπιον αυτών, διά να αναβή εις τον υψηλόν τόπον.
\par 15 Είχε δε αποκαλύψει ο Κύριος προς τον Σαμουήλ, μίαν ημέραν πριν έλθη ο Σαούλ, λέγων;
\par 16 Αύριον περί την ώραν ταύτην θέλω αποστείλει προς σε άνθρωπον εκ γης Βενιαμίν, και θέλεις χρίσει αυτόν άρχοντα επί τον λαόν μου Ισραήλ, και θέλει σώσει τον λαόν μου εκ χειρός των Φιλισταίων· διότι επέβλεψα επί τον λαόν μου, επειδή η βοή αυτών ήλθεν εις εμέ.
\par 17 Και ότε ο Σαμουήλ είδε τον Σαούλ, ο Κύριος είπε προς αυτόν, Ιδού, ο άνθρωπος, περί του οποίου σοι είπα· ούτος θέλει άρχει επί τον λαόν μου.
\par 18 Τότε επλησίασεν ο Σαούλ προς τον Σαμουήλ εις την πύλην και είπε, Δείξον μοι, παρακαλώ, που είναι η οικία του βλέποντος.
\par 19 Και απεκρίθη ο Σαμουήλ προς τον Σαούλ και είπεν, Εγώ είμαι ο βλέπων· ανάβα έμπροσθέν μου εις τον υψηλόν τόπον· και θέλετε φάγει σήμερον μετ' εμού, και το πρωΐ θέλω σε εξαποστείλει και πάντα όσα είναι εν τη καρδία σου θέλω αναγγείλει προς σέ·
\par 20 περί δε των όνων, τας οποίας έχασας ήδη τρεις ημέρας, μη φρόντιζε περί αυτών, διότι ευρέθησαν· και προς τίνα είναι πάσα η επιθυμία του Ισραήλ; δεν είναι προς σε, και προς πάντα τον οίκον του πατρός σου;
\par 21 Αποκριθείς δε ο Σαούλ είπε, Δεν είμαι εγώ Βενιαμίτης, εκ της μικροτέρας των φυλών Ισραήλ; και η οικογένειά μου η ελαχίστη πασών των οικογενειών της φυλής Βενιαμίν; διά τι λοιπόν λαλείς ούτω προς εμέ;
\par 22 Και έλαβεν ο Σαμουήλ τον Σαούλ και τον υπηρέτην αυτού και έφερεν αυτούς εις το οίκημα, και έδωκεν εις αυτούς την πρώτην θέσιν μεταξύ των κεκλημένων, οίτινες ήσαν περίπου τριάκοντα άνδρες.
\par 23 Και είπεν ο Σαμουήλ προς τον μάγειρον, Φέρε το μερίδιον το οποίον σοι έδωκα, περί του οποίον σοι είπα, Φύλαττε τούτο πλησίον σου.
\par 24 Και ύψωσεν ο μάγειρος την πλάτην και το επ' αυτήν και έθεσεν έμπροσθεν του Σαούλ. Και είπεν ο Σαμουήλ, Ιδού, το εναπολειφθέν· θες αυτό έμπροσθέν σου, φάγε· διότι διά την ώραν ταύτην εφυλάχθη διά σε, ότε είπα, Προσεκάλεσα τον λαόν. Και έφαγεν ο Σαούλ μετά του Σαμουήλ εν τη ημέρα εκείνη.
\par 25 Και αφού κατέβησαν εκ του υψηλού τόπου εις την πόλιν, συνωμίλησεν ο Σαμουήλ μετά του Σαούλ επί του δώματος.
\par 26 Και εσηκώθησαν ενωρίς· και περί τα χαράγματα της ημέρας, εκάλεσεν ο Σαμουήλ τον Σαούλ όντα επί του δώματος, λέγων, Σηκώθητι, διά να σε εξαποστείλω. Και εσηκώθη ο Σαούλ, και εξήλθον αμφότεροι, αυτός και ο Σαμουήλ, έως έξω.
\par 27 Καθώς δε κατέβαινον εις το τέλος της πόλεως, είπεν ο Σαμουήλ προς τον Σαούλ, Πρόσταξον τον υπηρέτην να περάση έμπροσθεν ημών· και εκείνος επέρασε· συ όμως στάθητι ολίγον, και θέλω σοι αναγγείλει τον λόγον του Θεού.

\chapter{10}

\par 1 Τότε έλαβεν ο Σαμουήλ την φιάλην του ελαίου, και έχυσεν επί την κεφαλήν αυτού, και εφίλησεν αυτόν και είπε, Δεν σε έχρισε Κύριος άρχοντα επί της κληρονομίας αυτού;
\par 2 Αφού αναχωρήσης απ' εμού σήμερον, θέλεις ευρεί δύο ανθρώπους πλησίον του τάφου της Ραχήλ, κατά το όριον του Βενιαμίν εν Σελσά· και θέλουσιν ειπεί προς σε, Ευρέθησαν αι όνοι, τας οποίας υπήγες να ζητήσης· και ιδού, ο πατήρ σου, αφήσας την φροντίδα των όνων, υπερλυπείται διά σας, λέγων, Τι να κάμω περί του υιού μου;
\par 3 Και προχωρήσας εκείθεν, θέλεις ελθεί έως της δρυός του Θαβώρ, και εκεί θέλουσι σε ευρεί τρεις άνθρωποι αναβαίνοντες προς τον Θεόν εις Βαιθήλ, ο εις φέρων τρία ερίφια, και ο άλλος φέρων τρεις άρτους, και ο άλλος φέρων ασκόν οίνου·
\par 4 και θέλουσι σε χαιρετήσει και σοι δώσει δύο άρτους, τους οποίους θέλεις δεχθή εκ των χειρών αυτών.
\par 5 Μετά ταύτα θέλεις υπάγει εις το βουνόν του Θεού, όπου είναι η φρουρά των Φιλισταίων· και όταν υπάγης εκεί εις την πόλιν, θέλεις απαντήσει άθροισμα προφητών καταβαινόντων από του υψηλού τόπου εν ψαλτηρίω και τυμπάνω και αυλώ και κιθάρα έμπροσθεν αυτών, και προφητευόντων.
\par 6 Και θέλει επέλθει επί σε πνεύμα Κυρίου, και θέλεις προφητεύσει μετ' αυτών και θέλεις μεταβληθή εις άλλον άνθρωπον.
\par 7 Και όταν τα σημεία ταύτα έλθωσιν επί σε, κάμνε ό,τι δύνασαι διότι ο Θεός είναι μετά σου.
\par 8 Και θέλεις καταβή προ εμού εις Γάλγαλα· και ιδού, εγώ θέλω καταβή προς σε, διά να προσφέρω ολοκαυτώματα, να θυσιάσω θυσίας ειρηνικάς· πρόσμενε επτά ημέρας, εωσού έλθω προς σε και σοι αναγγείλω τι έχεις να κάμης.
\par 9 Και ότε έστρεψε τα νώτα αυτού διά να αναχωρήση από του Σαμουήλ, ο Θεός έδωκεν εις αυτόν άλλην καρδίαν· και ήλθον πάντα εκείνα τα σημεία εν τη ημέρα εκείνη.
\par 10 Και ότε ήλθον εκεί εις το βουνόν, ιδού, άθροισμα προφητών συνήντησεν αυτόν· και επήλθεν επ' αυτόν Πνεύμα Θεού, και επροφήτευσε μεταξύ αυτών.
\par 11 Και ως είδον οι γνωρίζοντες αυτόν πρότερον, και ιδού, προεφήτευε μετά των προφητών, τότε έλεγεν ο λαός, έκαστος προς τον πλησίον αυτού, Τι είναι τούτο, το οποίον έγεινεν εις τον υιόν του Κείς; και Σαούλ εν προφήταις;
\par 12 Εις δε εκ των εκεί απεκρίθη και είπεν, Και τις είναι ο πατήρ αυτών; Διά τούτο έγεινε παροιμία, Και Σαούλ εν προφήταις;
\par 13 Και αφού ετελείωσε προφητεύων, ήλθεν εις τον υψηλόν τόπον.
\par 14 Και είπεν ο θείος του Σαούλ προς αυτόν και προς τον υπηρέτην αυτού, Που υπήγετε; Και είπε, να ζητήσωμεν τας όνους. και ότε είδομεν ότι δεν ήσαν, ήλθομεν προς τον Σαμουήλ.
\par 15 Και είπεν ο θείος του Σαούλ, Ανάγγειλόν μοι, σε παρακαλώ, τι σας είπεν ο Σαμουήλ.
\par 16 Και είπεν ο Σαούλ προς τον θείον αυτού, Μας είπε μετά βεβαιότητος ότι ευρέθησαν αι όνοι· τον λόγον όμως περί της βασιλείας, τον οποίον ο Σαμουήλ είπε, δεν εφανέρωσεν εις αυτόν.
\par 17 Και συνήγαγεν ο Σαμουήλ τον λαόν προς τον Κύριον εις Μισπά·
\par 18 και είπε προς τους υιούς Ισραήλ, Ούτω λέγει Κύριος ο Θεός του Ισραήλ· Εγώ ανεβίβασα τον Ισραήλ εξ Αιγύπτου, και σας ηλευθέρωσα εκ χειρός των Αιγυπτίων και εκ χειρός πασών των βασιλειών, αίτινες σας κατέθλιβον·
\par 19 και σεις την ημέραν ταύτην απεβάλετε τον Θεόν σας, όστις σας έσωσεν από πάντων των κακών σας και των θλίψεών σας, και είπετε προς αυτόν, Ουχί, αλλά κατάστησον βασιλέα εφ' ημάς. Τώρα λοιπόν παρουσιάσθητε ενώπιον του Κυρίου, κατά τας φυλάς σας και κατά τας χιλιάδας σας.
\par 20 Και ότε έκαμεν ο Σαμουήλ πάσας τας φυλάς του Ισραήλ να πλησιάσωσιν, επιάσθη η φυλή του Βενιαμίν.
\par 21 Και αφού έκαμε την φυλήν του Βενιαμίν να πλησιάση κατά τας οικογενείας αυτών, επιάσθη η οικογένεια του Ματρεί, και επιάσθη ο Σαούλ ο υιός του Κείς· εζήτησαν δε αυτόν και δεν ευρέθη.
\par 22 Όθεν εζήτησαν έτι παρά του Κυρίου, αν ο άνθρωπος έρχηται έτι εκεί. Και είπε Κύριος, Ιδού, αυτός είναι κεκρυμμένος μεταξύ της αποσκευής.
\par 23 Τότε έδραμον και έλαβον αυτόν εκείθεν· και ότε εστάθη μεταξύ του λαού, εξείχεν υπέρ πάντα τον λαόν, από τους ώμους αυτού και επάνω.
\par 24 Και είπεν ο Σαμουήλ προς πάντα τον λαόν, Βλέπετε εκείνον, τον οποίον εξέλεξεν ο Κύριος, ότι δεν είναι όμοιος αυτού μεταξύ παντός του λαού; Και πας ο λαός ηλάλαξε και είπε, Ζήτω ο βασιλεύς.
\par 25 Και είπεν ο Σαμουήλ προς τον λαόν τον τρόπον της βασιλείας, και έγραψεν αυτόν εν βιβλίω και έθεσεν έμπροσθεν του Κυρίου. Και απέλυσεν ο Σαμουήλ πάντα τον λαόν, έκαστον εις τον οίκον αυτού.
\par 26 Και ο Σαούλ ομοίως ανεχώρησεν εις τον οίκον αυτού εις Γαβαά· και υπήγε μετ' αυτού εκεί τάγμα πολεμιστών, των οποίων τας καρδίας είχε διαθέσει ο Θεός.
\par 27 Άνθρωποι όμως κακοί είπον, Πως θέλει σώσει ημάς ούτος; Και κατεφρόνησαν αυτόν και δεν προσέφεραν προς αυτόν δώρα· εκείνος όμως έκαμνε τον κωφόν.

\chapter{11}

\par 1 Ανέβη δε Νάας ο Αμμωνίτης και εστρατοπέδευσεν εναντίον της Ιαβείς-γαλαάδ· και είπον πάντες οι άνδρες της Ιαβείς εις τον Νάας, Κάμε συνθήκην προς ημάς, και θέλομεν σε δουλεύει.
\par 2 Και είπε προς αυτούς Νάας ο Αμμωνίτης, Με τούτο θέλω κάμει συνθήκην προς εσάς, να εξορύξω πάντας τους δεξιούς οφθαλμούς σας, και να βάλω τούτο όνειδος επί πάντα τον Ισραήλ.
\par 3 Και είπον προς αυτόν οι πρεσβύτεροι της Ιαβείς, Δος εις ημάς επτά ημερών αναβολήν, διά να αποστείλωμεν μηνυτάς εις πάντα τα όρια του Ισραήλ· και τότε, εάν δεν ήναι τις να μας σώση, θέλομεν εξέλθει προς σε.
\par 4 Ήλθον λοιπόν οι μηνυταί εις Γαβαά του Σαούλ και είπον τους λόγους εις τα ώτα του λαού· και ύψωσαν πας ο λαός την φωνήν αυτών και έκλαυσαν.
\par 5 Και ιδού, ο Σαούλ ήρχετο κατόπιν της αγέλης εκ του αγρού· και είπεν ο Σαούλ, Τι έχει ο λαός και κλαίει; Και διηγήθησαν προς αυτόν τους λόγους των ανδρών της Ιαβείς.
\par 6 Και επήλθεν επί τον Σαούλ πνεύμα Θεού, ότε ήκουσε τους λόγους εκείνους· και εξήφθη η οργή αυτού σφόδρα.
\par 7 Και έλαβε ζεύγος βοών, και κατακόψας αυτούς εις τμήματα, απέστειλεν αυτά κατά πάντα τα όρια του Ισραήλ διά χειρός μηνυτών, λέγων, Όστις δεν εξέλθη κατόπιν του Σαούλ και κατόπιν του Σαμουήλ, ούτω θέλει γείνει εις τους βόας αυτού. Και επέπεσε φόβος Κυρίου επί τον λαόν, και εξήλθον ως εις άνθρωπος.
\par 8 Και ότε απηρίθμησεν αυτούς εν Βεζέκ, οι υιοί Ισραήλ ήσαν τριακόσιαι χιλιάδες και οι άνδρες Ιούδα τριάκοντα χιλιάδες.
\par 9 Και είπον προς τους ελθόντας μηνυτάς, Ούτω θέλετε ειπεί προς τους άνδρας της Ιαβείς-γαλαάδ· Αύριον, καθώς ο ήλιος θερμάνη, θέλει είσθαι εις εσάς σωτηρία. Και ήλθον οι μηνυταί και ανήγγειλαν προς τους άνδρας της Ιαβείς· και υπερεχάρησαν.
\par 10 Και είπον οι άνδρες της Ιαβείς, Αύριον θέλομεν εξέλθει προς εσάς, και θέλετε κάμει εις ημάς παν ό,τι σας φαίνεται καλόν.
\par 11 Και την επαύριον διήρεσεν ο Σαούλ τον λαόν εις τρία τάγματα· και εισήλθον εις το μέσον του στρατοπέδου, εν τη πρωϊνή φυλακή, και επάταξαν τους Αμμωνίτας εωσού θερμάνη η ημέρα· και οι εναπολειφθέντες διεσκορπίσθησαν, ώστε ουδέ δύο εξ αυτών δεν έμειναν ηνωμένοι.
\par 12 Και είπεν ο λαός προς τον Σαμουήλ, Τις είναι εκείνος όστις είπεν, Ο Σαούλ θέλει βασιλεύσει εφ' ημάς; παραδώσατε τους άνδρας, διά να θανατώσωμεν αυτούς.
\par 13 Και είπεν ο Σαούλ, Δεν θέλει θανατωθή ουδείς την ημέραν ταύτην· διότι σήμερον έκαμεν ο Κύριος σωτηρίαν εν τω Ισραήλ.
\par 14 Τότε είπεν ο Σαμουήλ προς τον λαόν, Έλθετε, και ας υπάγωμεν εις Γάλγαλα και ας εγκαινίσωμεν εκεί την βασιλείαν.
\par 15 Και υπήγε πας ο λαός εις Γάλγαλα· και εκεί έκαμον τον Σαούλ βασιλέα ενώπιον του Κυρίου εν Γαλγάλοις· και εκεί εθυσίασαν θυσίας ειρηνικάς ενώπιον του Κυρίου· και εκεί ευφράνθησαν ο Σαούλ και πάντες οι άνδρες Ισραήλ σφόδρα.

\chapter{12}

\par 1 Και είπεν ο Σαμουήλ προς πάντα τον Ισραήλ, Ιδού, υπήκουσα εις την φωνήν σας κατά πάντα όσα είπετε προς εμέ, και κατέστησα βασιλέα εφ' υμάς·
\par 2 και τώρα, ιδού, ο βασιλεύς πορεύεται έμπροσθέν σας· εγώ δε είμαι γέρων και πολιός· και οι υιοί μου, ιδού, είναι μεθ' υμών· και εγώ περιεπάτησα ενώπιόν σας εκ νεότητός μου έως της ημέρας ταύτης·
\par 3 ιδού, εγώ· μαρτυρήσατε κατ' εμού ενώπιον του Κυρίου και ενώπιον του κεχρισμένου αυτού· τίνος τον βουν έλαβον; ή τίνος τον όνον έλαβον; ή τίνα ηδίκησα; τίνα κατεδυνάστευσα; ή εκ χειρός τίνος έλαβον δώρα, διά να τυφλώσω τους οφθαλμούς μου διά τούτων; και θέλω αποδώσει εις εσάς.
\par 4 Οι δε είπον, Δεν ηδίκησας ημάς ουδέ κατεδυνάστευσας ημάς ουδέ έλαβές τι εκ της χειρός τινός.
\par 5 Και είπε προς αυτούς, Μάρτυς ο Κύριος εις εσάς, μάρτυς και ο κεχρισμένος αυτού την ημέραν ταύτην, ότι δεν ευρήκατε εις την χείρα μου ουδέν. Και απεκρίβησαν, Μάρτυς.
\par 6 Και είπεν ο Σαμουήλ προς τον λαόν, Μάρτυς ο Κύριος ο καταστήσας τον Μωϋσήν και τον Ααρών, και αναβιβάσας τους πατέρας σας εκ γης Αιγύπτου.
\par 7 Τώρα λοιπόν στάθητε, διά να διαλεχθώ με σας ενώπιον του Κυρίου, διά πάσας τας δικαιοσύνας του Κυρίου, τας οποίας έκαμεν εις εσάς και εις τους πατέρας σας.
\par 8 Αφού ο Ιακώβ ήλθεν εις την Αίγυπτον, και οι πατέρες σας εβόησαν προς τον Κύριον, τότε απέστειλεν ο Κύριος τον Μωϋσήν και τον Ααρών, και εξήγαγον τους πατέρας σας εξ Αιγύπτου και κατώκισαν αυτούς εν τω τόπω τούτω.
\par 9 Ελησμόνησαν όμως Κύριον τον Θεόν αυτών· όθεν παρέδωκεν αυτούς εις την χείρα του Σισάρα, αρχηγού του στρατεύματος του Ασώρ, και εις την χείρα των Φιλισταίων και εις την χείρα του βασιλέως Μωάβ, και επολέμησαν εναντίον αυτών.
\par 10 Και εβόησαν προς τον Κύριον και είπον, Ημαρτήσαμεν, επειδή εγκατελίπομεν τον Κύριον και ελατρεύσαμεν τους Βααλείμ και τας Ασταρώθ· αλλά τώρα ελευθέρωσον ημάς εκ της χειρός των εχθρών ημών, και θέλομεν λατρεύσει σε.
\par 11 Και απέστειλεν ο Κύριος τον Ιεροβάαλ και τον Βεδάν και τον Ιεφθάε και τον Σαμουήλ, και σας ηλευθέρωσεν εκ της χειρός των εχθρών σας πανταχόθεν, και κατωκήσατε εν ασφαλεία.
\par 12 Αλλ' ότε είδετε ότι Νάας ο βασιλεύς των υιών Αμμών ήλθεν εναντίον σας, είπετε προς εμέ, Ουχί, αλλά βασιλεύς θέλει βασιλεύει εφ' ημάς· ενώ Κύριος ο Θεός σας ήτο ο βασιλεύς σας.
\par 13 Τώρα λοιπόν, ιδού, ο βασιλεύς, τον οποίον εξελέξατε, τον οποίον εζητήσατε· και ιδού, ο Κύριος κατέστησε βασιλέα εφ' υμάς.
\par 14 Εάν φοβήσθε τον Κύριον και λατρεύητε αυτόν και υπακούητε εις την φωνήν αυτού και δεν στασιάζητε εναντίον της προσταγής του Κυρίου, τότε και σεις και ο βασιλεύς ο βασιλεύων εφ' υμάς θέλετε περιπατεί κατόπιν Κυρίου του Θεού σας·
\par 15 εάν όμως δεν υπακούητε εις την φωνήν του Κυρίου, αλλά στασιάζητε εναντίον της προσταγής του Κυρίου, τότε η χειρ του Κυρίου θέλει είσθαι εναντίον σας, καθώς εστάθη εναντίον των πατέρων σας.
\par 16 Τώρα λοιπόν παραστάθητε και ίδετε το μέγα τούτο πράγμα, το οποίον ο Κύριος θέλει κάμει έμπροσθεν των οφθαλμών σας·
\par 17 δεν είναι θερισμός των σίτων σήμερον; θέλω επικαλεσθή τον Κύριον, και θέλει πέμψει βροντάς και βροχήν· διά να γνωρίσητε και να ίδητε ότι το κακόν σας είναι μέγα, το οποίον επράξατε ενώπιον του Κυρίου, ζητήσαντες εις εαυτούς βασιλέα.
\par 18 Τότε επεκαλέσθη ο Σαμουήλ τον Κύριον· και έπεμψεν ο Κύριος βροντάς και βροχήν την ημέραν εκείνην· και πας ο λαός εφοβήθη σφόδρα τον Κύριον και τον Σαμουήλ.
\par 19 Και είπε πας ο λαός προς τον Σαμουήλ, Δεήθητι υπέρ των δούλων σου προς Κύριον τον Θεόν σου, διά να μη αποθάνωμεν· διότι επροσθέσαμεν εις πάσας τας αμαρτίας ημών το κακόν, να ζητήσωμεν εις εαυτούς βασιλέα.
\par 20 Και είπεν ο Σαμουήλ προς τον λαόν, Μη φοβείσθε· σεις επράξατε όλον τούτο το κακόν· πλην μη παραδρομήσητε από όπισθεν του Κυρίου, αλλά λατρεύετε τον Κύριον εξ όλης της καρδίας σας·
\par 21 και μη παραδρομήσητε· διότι τότε ηθέλετε υπάγει κατόπιν των ματαίων, τα οποία δεν δύνανται να ωφελήσωσιν ουδέ να ελευθερώσωσιν, επειδή είναι μάταια·
\par 22 διότι δεν θέλει εγκαταλείψει ο Κύριος τον λαόν αυτού, διά το όνομα αυτού το μέγα, επειδή ηυδόκησεν ο Κύριος να σας κάμη λαόν αυτού·
\par 23 εις εμέ δε μη γένοιτο να αμαρτήσω εις τον Κύριον, ώστε να παύσω από του να δέωμαι υπέρ υμών· αλλά θέλω σας διδάσκει την οδόν την αγαθήν και ευθείαν·
\par 24 μόνον φοβείσθε τον Κύριον και λατρεύετε αυτόν εν αληθεία εξ όλης καρδίας σας· διότι είδετε πόσα μεγαλεία έκαμεν υπέρ υμών·
\par 25 αλλ' εάν εξακολουθήτε να πράττητε το κακόν, και σεις και ο βασιλεύς υμών θέλετε απολεσθή.

\chapter{13}

\par 1 Ο Σαούλ ήτο βασιλεύς ενός έτους· αφού δε εβασίλευσε δύο έτη επί τον Ισραήλ,
\par 2 εξέλεξεν ο Σαούλ εις εαυτόν τρεις χιλιάδας εκ του Ισραήλ· και ήσαν μετά του Σαούλ δύο χιλιάδες εν Μιχμάς και εν τω όρει Βαιθήλ, και χίλιοι ήσαν μετά του Ιωνάθαν εν Γαβαά του Βενιαμίν· το δε υπόλοιπον του λαού εξαπέστειλεν έκαστον εις την σκηνήν αυτού.
\par 3 Και επάταξεν ο Ιωνάθαν την φρουράν των Φιλισταίων την εν τω βουνώ· και ήκουσαν οι Φιλισταίοι. Και εσάλπισεν ο Σαούλ διά της σάλπιγγος εν πάση τη γη, λέγων, Ας ακούσωσιν οι Εβραίοι.
\par 4 Και πας ο Ισραήλ ήκουσε να λέγωσιν, Επάταξεν ο Σαούλ την φρουράν των Φιλισταίων, και μάλιστα ο Ισραήλ μισείται υπό των Φιλισταίων. Και συνήχθη ο λαός κατόπιν του Σαούλ εν Γαλγάλοις.
\par 5 Οι δε Φιλισταίοι συνηθροίσθησαν διά να πολεμήσωσι μετά του Ισραήλ, τριάκοντα χιλιάδες αμαξών και εξ χιλιάδες ιππέων και λαός ως η άμμος η επί του χείλους της θαλάσσης κατά το πλήθος· και ανέβησαν και εστρατοπέδευσαν εν Μιχμάς, προς ανατολάς της Βαιθ-αυέν.
\par 6 Ότε οι άνδρες του Ισραήλ είδον ότι ήσαν εν αμηχανία, διότι ο λαός εμικροψύχει, τότε εκρύπτετο ο λαός εις τα σπήλαια και εις τα πυκνόφυτα και εις τους βράχους και εις τα οχυρά μέρη και εις τους λάκκους.
\par 7 Και τινές εκ των Εβραίων διέβησαν τον Ιορδάνην προς την γην Γαδ και Γαλαάδ. Ο δε Σαούλ αυτός ήτο ακόμη εν Γαλγάλοις· και πας ο λαός τρέμων κατόπιν αυτού.
\par 8 Και περιέμεινεν επτά ημέρας, κατά τον διωρισμένον καιρόν υπό του Σαμουήλ· αλλ' ο Σαμουήλ δεν ήρχετο εις Γάλγαλα· και ο λαός διεσκορπίζετο από πλησίον αυτού.
\par 9 Και είπεν ο Σαούλ, Φέρετε εδώ προς εμέ το ολοκαύτωμα, και τας ειρηνικάς προσφοράς. Και προσέφερε το ολοκαύτωμα.
\par 10 Και ως ετελείωσε προσφέρων το ολοκαύτωμα, ιδού, ήλθεν ο Σαμουήλ· και εξήλθεν ο Σαούλ εις συνάντησιν αυτού, διά να χαιρετήση αυτόν.
\par 11 Και είπεν ο Σαμουήλ, Τι έκαμες; Και απεκρίθη ο Σαούλ, Επειδή είδον ότι ο λαός διεσκορπίζετο απ' εμού, και συ δεν ήλθες την διωρισμένην ημέραν, οι δε Φιλισταίοι συνηθροίζοντο εις Μιχμάς,
\par 12 διά τούτο είπα, Τώρα θέλουσι καταβή οι Φιλισταίοι εναντίον μου εις Γάλγαλα, και εγώ δεν έκαμα δέησιν προς τον Κύριον· ετόλμησα λοιπόν, και προσέφερα το ολοκαύτωμα.
\par 13 Και είπεν ο Σαμουήλ προς τον Σαούλ, Συ έπραξας αφρόνως· δεν εφύλαξας το πρόσταγμα Κυρίου του Θεού σου, το οποίον προσέταξεν εις σέ· διότι τώρα ο Κύριος ήθελε στερεώσει την βασιλείαν σου επί τον Ισραήλ έως του αιώνος·
\par 14 αλλά τώρα η βασιλεία σου δεν θέλει στηριχθή· ο Κύριος εζήτησεν εις εαυτόν άνθρωπον κατά την καρδίαν αυτού, και διώρισεν ο Κύριος αυτόν να ήναι άρχων επί τον λαόν αυτού, επειδή δεν εφύλαξας εκείνο το οποίον προσέταξεν εις σε ο Κύριος.
\par 15 Και εσηκώθη ο Σαμουήλ και ανέβη από Γαλγάλων εις Γαβαά του Βενιαμίν. Ο δε Σαούλ ηρίθμησε τον λαόν τον ευρεθέντα μετ' αυτού, περίπου εξακοσίους άνδρας.
\par 16 Και ο Σαούλ και Ιωνάθαν ο υιός αυτού και ο λαός ο ευρεθείς μετ' αυτών, εκάθηντο εν Γαβαά του Βενιαμίν· οι δε Φιλισταίοι ήσαν εστρατοπεδευμένοι εν Μιχμάς.
\par 17 Και εξήλθον λεηλάται εκ του στρατοπέδου των Φιλισταίων εις τρία σώματα· το εν σώμα εστράφη εις την οδόν Οφρά, προς την γην Σωγάλ·
\par 18 και το άλλο σώμα εστράφη εις την οδόν Βαιθ-ωρών· και το άλλο σώμα εστράφη εις την οδόν του ορίου, το οποίον βλέπει προς την κοιλάδα Σεβωείμ, κατά την έρημον.
\par 19 Και σιδηρουργός δεν ευρίσκετο εν πάση τη γη Ισραήλ· διότι οι Φιλισταίοι είπον, Μήποτε οι Εβραίοι κατασκευάσωσι ρομφαίας ή λόγχας·
\par 20 αλλά κατέβαινον πάντες οι Ισραηλίται προς τους Φιλισταίους, διά να ακονώσιν έκαστος το υνίον αυτού και την δικέλλαν αυτού και την αξίνην αυτού, και την σκαπάνην αυτού,
\par 21 οσάκις ήθελον αμβλυνθή αι σκαπάναι και αι δικέλλαι και τα τρίκρανα και αι αξίναι αυτών· και διά να οξύνωσι τα βούκεντρα αυτών.
\par 22 Διά τούτο εν τη ημέρα της μάχης, δεν ευρίσκετο ούτε μάχαιρα ούτε λόγχη εις την χείρα τινός εκ του λαού του όντος μετά του Σαούλ και Ιωνάθαν· εις τον Σαούλ όμως και εις τον Ιωνάθαν τον υιόν αυτού ευρέθησαν.
\par 23 Η δε φρουρά των Φιλισταίων εξήλθε προς το πέρασμα Μιχμάς.

\chapter{14}

\par 1 Ημέραν δε τινά είπεν Ιωνάθαν, ο υιός του Σαούλ, προς τον νέον τον βαστάζοντα τα όπλα αυτού, Ελθέ, και ας περάσωμεν προς την φρουράν των Φιλισταίων, την εν τω πέραν· προς τον πατέρα αυτού όμως δεν εφανέρωσε τούτο.
\par 2 Ο δε Σαούλ εκάθητο επί του άκρου του Γαβαά, υπό την ροδιάν την εν Μιγρών· και ο λαός ο μετ' αυτού ήτο έως εξακόσιοι άνδρες·
\par 3 και Αχιά, ο υιός του Αχιτώβ, αδελφού του Ιχαβώδ, υιού του Φινεές, υιού του Ηλεί, ιερεύς του Κυρίου εν Σηλώ, φορών εφόδ. Και ο λαός δεν ήξευρεν ότι υπήγεν ο Ιωνάθαν.
\par 4 Μεταξύ δε των διαβάσεων, διά των οποίων ο Ιωνάθαν εζήτει να περάση προς την φρουράν των Φιλισταίων, ήτο απότομος βράχος εξ ενός μέρους και απότομος βράχος εκ του άλλου μέρους· και το όνομα του ενός Βοσές, το δε όνομα του άλλον Σενέ.
\par 5 Το μέτωπον του ενός βράχου ήτο προς βορράν απέναντι Μιχμάς, και το του άλλου προς νότον απέναντι Γαβαά.
\par 6 Και είπεν ο Ιωνάθαν προς τον νέον τον βαστάζοντα τα όπλα αυτού, Ελθέ, και ας περάσωμεν προς την φρουράν των απεριτμήτων τούτων· ίσως ενεργήση ο Κύριος υπέρ ημών· διότι δεν είναι εις τον Κύριον εμπόδιον να σώση διά πολλών ή δι' ολίγων.
\par 7 Και είπε προς αυτόν ο οπλοφόρος αυτού, Κάμε ό,τι είναι εν τη καρδία σου· προχώρει· ιδού, εγώ είμαι μετά σου κατά την καρδίαν σου.
\par 8 Τότε είπεν ο Ιωνάθαν, Ιδού, ημείς θέλομεν περάσει προς τους άνδρας και θέλομεν δειχθή εις αυτούς·
\par 9 εάν είπωσι προς ημάς ούτω, Στάθητε έως να έλθωμεν προς εσάς· τότε θέλομεν σταθή εν τω τόπω ημών και δεν θέλομεν αναβή προς αυτούς·
\par 10 αλλ' εάν είπωσιν ούτως, Ανάβητε προς ημάς· τότε θέλομεν αναβή· διότι ο Κύριος παρέδωκεν αυτούς εις την χείρα ημών· και τούτο θέλει είσθαι εις ημάς το σημείον.
\par 11 Εδείχθησαν λοιπόν αμφότεροι εις την φρουράν των Φιλισταίων· και οι Φιλισταίοι είπον, Ιδού, οι Εβραίοι εξέρχονται εκ των τρυπών, όπου είχον κρυφθή.
\par 12 Και ελάλησαν οι άνδρες της φρουράς προς τον Ιωνάθαν και προς τον βαστάζοντα τα όπλα αυτού, και είπον, Ανάβητε προς ημάς, και θέλομεν σας φανερώσει τι. Και είπεν ο Ιωνάθαν προς τον οπλοφόρον αυτού, Ανάβα κατόπιν μου· διότι παρέδωκεν αυτούς ο Κύριος εις την χείρα του Ισραήλ.
\par 13 Και ανέρπυσεν ο Ιωνάθαν με τας χείρας αυτού και με τους πόδας αυτού, και ο βαστάζων τα όπλα αυτού κατόπιν αυτού· και έπεσον έμπροσθεν του Ιωνάθαν· και ο βαστάζων τα όπλα αυτού εθανάτονεν αυτούς κατόπιν αυτού.
\par 14 Αύτη δε η πρώτη σφαγή, την οποίαν έκαμον ο Ιωνάθαν και ο οπλοφόρος αυτού, ήτο περίπου είκοσι άνδρες, εις διάστημα γης ημίσεως στρέμματος.
\par 15 Και έγεινε τρόμος εν τω στρατοπέδω, εν τοις αγροίς και εν παντί τω λαώ· η φρουρά και οι λεηλατούντες, και αυτοί κατετρόμαξαν, και η γη συνεταράχθη· ώστε ήτο ως τρόμος Θεού.
\par 16 Και είδον οι φρουροί του Σαούλ εν Γαβαά του Βενιαμίν, και ιδού, το πλήθος διελύετο και βαθμηδόν διεσκορπίζετο.
\par 17 Τότε είπεν ο Σαούλ προς τον λαόν τον μετ' αυτού, Απαριθμήσατε τώρα και ιδέτε τις ανεχώρησεν εξ ημών. Και ότε απηρίθμησαν, ιδού, ο Ιωνάθαν και ο οπλοφόρος αυτού δεν ήσαν.
\par 18 Και είπεν ο Σαούλ προς τον Αχιά, Φέρε εδώ την κιβωτόν του Θεού. Διότι η κιβωτός του Θεού ήτο τότε μετά των υιών Ισραήλ.
\par 19 Και ενώ ελάλει ο Σαούλ προς τον ιερέα, ο θόρυβος εν τω στρατοπέδω των Φιλισταίων επροχώρει επί το μάλλον και επληθύνετο· ο δε Σαούλ είπε προς τον ιερέα, Σύρε οπίσω την χείρα σου.
\par 20 Και συνηθροίσθησαν ο Σαούλ και πας ο λαός ο μετ' αυτού και ήλθον έως εις την μάχην· και ιδού, παντός ανδρός η ρομφαία ήτο εναντίον του συντρόφου αυτού, σφαγή μεγάλη σφόδρα.
\par 21 οι δε Εβραίοι οι μετά των Φιλισταίων όντες ως άλλοτε, οίτινες είχον αναβή μετ' αυτών εις το στρατόπεδον εκ των πέριξ, και αυτοί έτι ηνώθησαν μετά των Ισραηλιτών, οίτινες ήσαν μετά του Σαούλ και Ιωνάθαν.
\par 22 Και πάντες οι άνδρες του Ισραήλ οι κρυπτόμενοι εν τω όρει Εφραΐμ, ακούσαντες ότι οι Φιλισταίοι έφευγον, έδραμον και αυτοί κατόπιν αυτών εις πόλεμον.
\par 23 Και έσωσεν ο Κύριος τον Ισραήλ εν τη ημέρα εκείνη· και η μάχη επέρασεν εις Βαιθ-αυέν.
\par 24 Οι δε άνδρες του Ισραήλ απέκαμον την ημέραν εκείνην· διότι ο Σαούλ είχεν ορκίσει τον λαόν, λέγων, Επικατάρατος ο άνθρωπος, όστις φάγη τροφήν έως εσπέρας, και εκδικηθώ από των εχθρών μου. Όθεν δεν εγεύθη τροφήν πας ο λαός.
\par 25 Και παν το πλήθος ήλθεν εις δάσος, όπου ήτο μέλι κατά γης.
\par 26 Και ότε εισήλθεν ο λαός εις το δάσος, ιδού, το μέλι εστάλαξεν· ουδείς όμως επλησίασε την χείρα αυτού εις το στόμα αυτού· διότι εφοβήθη ο λαός τον όρκον.
\par 27 Ο Ιωνάθαν όμως δεν είχεν ακούσει, ότε ο πατήρ αυτού ώρκισε τον λαόν· όθεν ήπλωσε το άκρον της ράβδου της εν τη χειρί αυτού και εβύθισεν αυτό εις κηρήθραν και έβαλε την χείρα αυτού εις το στόμα αυτού, και ανέβλεψαν οι οφθαλμοί αυτού.
\par 28 Απεκρίθη δε εις εκ του λαού και είπεν, Ο πατήρ σου ώρκισε δι' όρκου τον λαόν, λέγων, Επικατάρατος ο άνθρωπος, όστις φάγη τροφήν σήμερον· διά τούτο ο λαός είναι εκλελυμένος.
\par 29 Ο δε Ιωνάθαν είπεν, Ετάραξεν ο πατήρ μου τον κόσμον· ιδέτε, παρακαλώ, πόσον ανέβλεψαν οι οφθαλμοί μου, διότι εγεύθην ολίγον εκ τούτου του μέλιτος·
\par 30 πόσω μάλλον, εάν ο λαός ήθελε φάγει την σήμερον ελευθέρως εκ των λαφύρων των εχθρών αυτού, τα οποία εύρηκε; διότι δεν ήθελε γείνει τώρα πολύ μεγαλητέρα σφαγή μεταξύ των Φιλισταίων;
\par 31 Επάταξαν δε εν εκείνη τη ημέρα τους Φιλισταίους από Μιχμάς έως Αιαλών· και ο λαός ήτο εκλελυμένος σφόδρα.
\par 32 Όθεν ερρίφθη ο λαός εις τα λάφυρα, και έλαβε πρόβατα και βόας και μόσχους και έσφαξαν κατά γής· και έτρωγεν ο λαός μετά του αίματος.
\par 33 Ανήγγειλαν δε προς τον Σαούλ, λέγοντες, Ιδού, ο λαός αμαρτάνει εις τον Κύριον, διότι τρώγουσι μετά του αίματος. Και είπε, Παραβάται εστάθητε· κυλίσατε προς εμέ σήμερον λίθον μέγαν.
\par 34 Και είπεν ο Σαούλ, Διασπάρθητε μεταξύ του λαού και είπατε προς αυτούς, Φέρετέ μοι ενταύθα έκαστος τον βουν αυτού και έκαστος το πρόβατον αυτού, και σφάξατε ενταύθα και φάγετε· και μη αμαρτάνετε εις τον Κύριον, τρώγοντες μετά του αίματος. Και έφεραν πας ο λαός έκαστος τον βουν αυτού μεθ' εαυτού εκείνην την νύκτα και έσφαξαν εκεί.
\par 35 Και ωκοδόμησεν ο Σαούλ θυσιαστήριον εις τον Κύριον· τούτο ήτο το πρώτον θυσιαστήριον, το οποίον ωκοδόμησεν ο Σαούλ εις τον Κύριον.
\par 36 Και είπεν ο Σαούλ, Ας καταβώμεν εξοπίσω των Φιλισταίων διά νυκτός, και ας διαρπάσωμεν αυτούς έως να φέγξη η ημέρα, και ας μη αφήσωμεν μηδέ ένα εξ αυτών. Και είπον, Κάμε παν ό,τι σοι φαίνεται καλόν. Τότε είπεν ο ιερεύς, Ας προσέλθωμεν ενταύθα εις τον Θεόν.
\par 37 Και ηρώτησεν ο Σαούλ τον Θεόν, Να καταβώ εξοπίσω των Φιλισταίων; θέλεις παραδώσει αυτούς εις την χείρα του Ισραήλ; Αλλά δεν απεκρίθη προς αυτόν την ημέραν εκείνην.
\par 38 Και είπεν ο Σαούλ, Πλησιάσατε ενταύθα πάντες οι αρχηγοί του λαού· και μάθετε και ιδέτε, εις ποίον εστάθη η αμαρτία αύτη σήμερον·
\par 39 διότι ζη Κύριος, ο σώσας τον Ισραήλ, ότι και εις τον Ιωνάθαν τον υιόν μου αν εστάθη, θέλει βεβαίως θανατωθή. Και δεν ευρέθη ουδείς μεταξύ παντός του λαού, όστις απεκρίθη προς αυτόν.
\par 40 Και είπε προς πάντα τον Ισραήλ, Σταθήτε σεις εκ του ενός μέρους, εγώ δε και Ιωνάθαν ο υιός μου θέλομεν σταθή εκ του άλλου μέρους. Και είπεν ο λαός προς τον Σαούλ, Κάμε παν ό,τι σοι φαίνεται καλόν.
\par 41 Τότε είπεν ο Σαούλ προς τον Κύριον τον Θεόν του Ισραήλ, Δείξον τον αθώον. Και επιάσθη ο Ιωνάθαν και ο Σαούλ· ο δε λαός απελύθη.
\par 42 Και είπεν ο Σαούλ, Ρίψατε κλήρους μεταξύ εμού και Ιωνάθαν του υιού μου. Και επιάσθη ο Ιωνάθαν.
\par 43 Τότε είπεν ο Σαούλ προς τον Ιωνάθαν, Φανέρωσόν μοι τι έπραξας. Και εφανέρωσε προς αυτόν ο Ιωνάθαν, και είπε, Τωόντι εγεύθην ολίγον μέλι διά του άκρου της ράβδου της εν τη χειρί μου· ιδού, εγώ, αποθνήσκω.
\par 44 Και απεκρίθη ο Σαούλ, Ούτω να κάμη ο Θεός και ούτω να προσθέση· βεβαίως θέλεις αποθάνει, Ιωνάθαν.
\par 45 Ο δε λαός είπε προς τον Σαούλ, Ο Ιωνάθαν θέλει αποθάνει, όστις έκαμε την μεγάλην ταύτην σωτηρίαν εις τον Ισραήλ; Μη γένοιτο· Ζη Κύριος, ουδέ μία θριξ εκ της κεφαλής αυτού θέλει πέσει εις την γήν· διότι ενήργησε μετά του Θεού την ημέραν ταύτην. Και ελύτρωσεν ο λαός τον Ιωνάθαν, και δεν απέθανε.
\par 46 Τότε ανέβη ο Σαούλ εκ της καταδιώξεως των Φιλισταίων· και οι Φιλισταίοι υπήγαν εις τον τόπον αυτών.
\par 47 Και έλαβεν ο Σαούλ την βασιλείαν επί τον Ισραήλ, και επολέμησεν εναντίον πάντων των εχθρών αυτού κύκλω· εναντίον του Μωάβ και εναντίον των υιών του Αμμών και εναντίον του Εδώμ και εναντίον των βασιλέων της Σωβά και εναντίον των Φιλισταίων· και εναντίον πάντων όπου και αν εστρέφετο, κατετρόπονε.
\par 48 Συνεκρότησεν έτι δύναμιν και επάταξε τον Αμαλήκ, και ηλευθέρωσε τον Ισραήλ εκ χειρός των διαρπαζόντων αυτούς.
\par 49 Οι δε υιοί του Σαούλ ήσαν Ιωνάθαν και Ισονεί και Μελχί-σουέ· και τα ονόματα των δύο θυγατέρων αυτού, το όνομα της πρωτοτόκου Μεράβ, και το όνομα της νεωτέρας Μιχάλ·
\par 50 το δε όνομα της γυναικός του Σαούλ ήτο Αχινοάμ, θυγάτηρ του Αχιμάας. Και το όνομα του αρχιστρατήγου αυτού Αβενήρ, υιός του Νηρ, θείου του Σαούλ.
\par 51 Ο δε Κείς ο πατήρ του Σαούλ, και ο Νηρ ο πατήρ του Αβενήρ, ήσαν υιοί του Αβιήλ.
\par 52 Ήτο δε πόλεμος δυνατός εναντίον των Φιλισταίων κατά πάσας τας ημέρας του Σαούλ· και οπότε έβλεπεν ο Σαούλ άνδρα τινά δυνατόν ή άνδρείον, παρελάμβανεν αυτόν πλησίον εαυτού.

\chapter{15}

\par 1 Είπε δε Σαμουήλ προς τον Σαούλ, Εμέ απέστειλεν ο Κύριος να σε χρίσω βασιλέα επί τον λαόν αυτού, επί τον Ισραήλ· τώρα λοιπόν άκουσον της φωνής των λόγων του Κυρίου.
\par 2 Ούτω λέγει ο Κύριος των δυνάμεων· Θέλω εκδικήσει όσα έκαμεν ο Αμαλήκ εις τον Ισραήλ, ότι αντεστάθη εις αυτόν εν τη οδώ, ότε ανέβαινεν εξ Αιγύπτου·
\par 3 ύπαγε τώρα και πάταξον τον Αμαλήκ, και εξολόθρευσον παν ό,τι έχει και μη φεισθής αυτούς· αλλά θανάτωσον και άνδρα και γυναίκα και παιδίον και θηλάζον και βουν και πρόβατον και κάμηλον και όνον.
\par 4 Και ο Σαούλ εκάλεσε τον λαόν και απηρίθμησεν αυτούς εν Τελαΐμ, διακοσίας χιλιάδας πεζών και δέκα χιλιάδας ανδρών Ιούδα.
\par 5 Και ήλθεν ο Σαούλ έως της πόλεως του Αμαλήκ και ενέδρευσεν εν τη φάραγγι.
\par 6 Και είπεν ο Σαούλ προς τους Κεναίους, Υπάγετε, αναχωρήσατε, κατάβητε εκ μέσου των Αμαληκιτών, διά να μη σας συμπεριλάβω μετ' αυτών· διότι σεις εδείξατε έλεος εις πάντας τους υιούς Ισραήλ, ότε ανέβαινον εξ Αιγύπτου. Και ανεχώρησαν οι Κεναίοι εκ μέσου των Αμαληκιτών.
\par 7 Και επάταξεν ο Σαούλ τους Αμαληκίτας από Αβιλά έως της εισόδου Σούρ, της κατά πρόσωπον Αιγύπτου.
\par 8 Και συνέλαβεν Αγάγ τον βασιλέα των Αμαληκιτών ζώντα, πάντα δε τον λαόν εξωλόθρευσεν εν στόματι μαχαίρας.
\par 9 Πλην εφείσθη ο Σαούλ και ο λαός τον Αγάγ και τα καλήτερα των προβάτων και των βοών και των δευτερευόντων και των αρνίων και παντός αγαθού, και δεν ήθελον να εξολοθρεύσωσιν αυτά· αλλά παν το ευτελές και εξουδενωμένον, εκείνο εξωλόθρευσαν.
\par 10 Τότε έγεινε λόγος Κυρίου προς τον Σαμουήλ, λέγων,
\par 11 Μετεμελήθην ότι έκαμα τον Σαούλ βασιλέα· διότι εστράφη από όπισθέν μου και τους λόγους μου δεν εξετέλεσε. Και τούτο ελύπησε τον Σαμουήλ, και εβόησε προς τον Κύριον δι' όλης της νυκτός.
\par 12 Και ότε εξηγέρθη ο Σαμουήλ ενωρίς διά να υπάγη εις συνάντησιν του Σαούλ το πρωΐ, ανήγγειλαν προς τον Σαμουήλ, λέγοντες, Ο Σαούλ ήλθεν εις τον Κάρμηλον, και ιδού, ανήγειρεν εις εαυτόν τρόπαιον· έπειτα εστράφη και διεπέρασε και κατέβη εις Γάλγαλα.
\par 13 Και υπήγεν ο Σαμουήλ προς τον Σαούλ· και είπεν ο Σαούλ προς αυτόν, Ευλογημένος να ήσαι παρά του Κυρίου· εξετέλεσα τον λόγον του Κυρίου.
\par 14 Είπε δε ο Σαμουήλ, Και τις η φωνή αύτη των προβάτων εις τα ώτα μου, και η φωνή των βοών, την οποίαν ακούω;
\par 15 Και είπεν ο Σαούλ, Εκ των Αμαληκιτών έφεραν αυτά· διότι ο λαός εφείσθη τα καλήτερα των προβάτων και των βοών, διά να θυσιάση εις Κύριον τον Θεόν σου· τα δε λοιπά εξωλοθρεύσαμεν.
\par 16 Τότε είπεν ο Σαμουήλ προς τον Σαούλ, Άφες, και θέλω απαγγείλει προς σε τι ελάλησεν ο Κύριος εις εμέ την νύκτα. Ο δε είπε προς αυτόν, Λέγε.
\par 17 Και είπεν ο Σαμουήλ, Ενώ συ ήσο μικρός έμπροσθεν των οφθαλμών σου, δεν έγεινες η κεφαλή των φυλών του Ισραήλ, και σε έχρισεν ο Κύριος βασιλέα επί τον Ισραήλ;
\par 18 και σε έστειλεν ο Κύριος εις την οδόν και είπεν, Ύπαγε και εξολόθρευσον τους αμαρτάνοντας εις εμέ, τους Αμαληκίτας, και πολέμησον εναντίον αυτών εωσού εξαφανίσης αυτούς·
\par 19 διά τι λοιπόν δεν υπήκουσας της φωνής του Κυρίου, αλλ' ώρμησας επί τα λάφυρα και έπραξας το κακόν ενώπιον του Κυρίου;
\par 20 Και είπεν ο Σαούλ προς τον Σαμουήλ, Ναι, υπήκουσα της φωνής του Κυρίου και υπήγα εις την οδόν εις την οποίαν ο Κύριος με απέστειλε και έφερα τον Αγάγ τον βασιλέα του Αμαλήκ, τους δε Αμαληκίτας εξωλόθρευσα·
\par 21 ο λαός όμως έλαβεν εκ των λαφύρων πρόβατα και βόας, τα καλήτερα από των απηγορευμένων, διά να θυσιάση εις Κύριον τον Θεόν σου εν Γαλγάλοις.
\par 22 Και είπεν ο Σαμουήλ, Μήπως ο Κύριος αρέσκεται εις τα ολοκαυτώματα και εις τας θυσίας, καθώς εις το να υπακούωμεν της φωνής του Κυρίου; ιδού, η υποταγή είναι καλητέρα παρά την θυσίαν· η υπακοή, παρά το πάχος των κριών·
\par 23 διότι η απείθεια είναι καθώς το αμάρτημα της μαγείας· και το πείσμα, καθώς η ασέβεια και ειδωλολατρεία· επειδή συ απέρριψας τον λόγον του Κυρίου, διά τούτο και αυτός απέρριψε σε από του να ήσαι βασιλεύς.
\par 24 Και είπεν ο Σαούλ προς τον Σαμουήλ, Ημάρτησα· διότι παρέβην το πρόσταγμα του Κυρίου και τους λόγους σου, φοβηθείς τον λαόν και υπακούσας εις την φωνήν αυτών·
\par 25 τώρα λοιπόν συγχώρησον, παρακαλώ, το αμάρτημά μου και επίστρεψον μετ' εμού, διά να προσκυνήσω τον Κύριον.
\par 26 Ο δε Σαμουήλ είπε προς τον Σαούλ, Δεν θέλω επιστρέψει μετά σού· διότι απέρριψας τον λόγον του Κυρίου, και ο Κύριος απέρριψε σε από του να ήσαι βασιλεύς επί τον Ισραήλ.
\par 27 Και καθώς εστράφη ο Σαμουήλ διά να αναχωρήση, εκείνος επίασεν αυτόν από του κρασπέδου του ιματίου αυτού· και εξεσχίσθη.
\par 28 Και είπε προς αυτόν ο Σαμουήλ, Εξέσχισεν η Κύριος την βασιλείαν του Ισραήλ από σου σήμερον και έδωκεν αυτήν εις τον πλησίον σου, τον καλήτερόν σου·
\par 29 ουδέ θέλει ψευσθή ο Ισχυρός του Ισραήλ ουδέ μεταμεληθή· διότι ούτος δεν είναι άνθρωπος, ώστε να μεταμεληθή.
\par 30 Ο δε είπεν, Ημάρτησα· αλλά τίμησόν με τώρα, παρακαλώ, έμπροσθεν των πρεσβυτέρων του λαού μου και έμπροσθεν του Ισραήλ, και επίστρεψον μετ' εμού, διά να προσκυνήσω Κύριον τον Θεόν σου.
\par 31 Και επέστρεψεν ο Σαμουήλ κατόπιν του Σαούλ και προσεκύνησεν ο Σαούλ τον Κύριον.
\par 32 Τότε είπεν ο Σαμουήλ, Φέρετέ μοι ενταύθα Αγάγ τον βασιλέα των Αμαληκιτών. Και ήλθε προς αυτόν ο Αγάγ χαριέντως· διότι έλεγεν ο Αγάγ, Βεβαίως η πικρία του θανάτου επέρασεν.
\par 33 Ο δε Σαμουήλ είπε, Καθώς ητέκνωσε γυναίκας η ρομφαία σου, ούτω θέλει ατεκνωθή μεταξύ των γυναικών η μήτηρ σου. Και κατέκοψεν ο Σαμουήλ τον Αγάγ ενώπιον του Κυρίου εν Γαλγάλοις.
\par 34 Τότε ανεχώρησεν ο Σαμουήλ εις Ραμά· ο δε Σαούλ ανέβη εις τον οίκον αυτού, εις Γαβαά Σαούλ.
\par 35 Ο δε Σαμουήλ δεν είδε πλέον τον Σαούλ έως της ημέρας του θανάτου αυτού· επένθησεν όμως ο Σαμουήλ διά τον Σαούλ. Και ο Κύριος μετεμελήθη ότι έκαμε τον Σαούλ βασιλέα επί τον Ισραήλ.

\chapter{16}

\par 1 Και είπε Κύριος προς τον Σαμουήλ, Έως πότε συ πενθείς διά τον Σαούλ, επειδή εγώ απεδοκίμασα αυτόν από του να βασιλεύη επί τον Ισραήλ; γέμισον το κέρας σου έλαιον και ύπαγε· εγώ σε αποστέλλω προς τον Ιεσσαί τον Βηθλεεμίτην· διότι προέβλεψα εις εμαυτόν βασιλέα μεταξύ των υιών αυτού.
\par 2 Και είπεν ο Σαμουήλ, Πως να υπάγω; διότι θέλει ακούσει τούτο ο Σαούλ και θέλει με θανατώσει. Και είπεν ο Κύριος, Λάβε μετά σου δάμαλιν και ειπέ, Ήλθον να θυσιάσω προς τον Κύριον.
\par 3 Και κάλεσον τον Ιεσσαί εις την θυσίαν, και εγώ θέλω φανερώσει προς σε τι θέλεις κάμει και θέλεις χρίσει εις εμέ όντινα σοι είπω.
\par 4 Και έκαμεν ο Σαμουήλ εκείνο το οποίον είπεν ο Κύριος, και ήλθεν εις Βηθλεέμ. Ετρόμαξαν δε οι πρεσβύτεροι της πόλεως εις την συνάντησιν αυτού και είπον, Εν ειρήνη έρχεσαι;
\par 5 Ο δε είπεν, Εν ειρήνη· έρχομαι διά να θυσιάσω προς τον Κύριον· αγιάσθητε και έλθετε μετ' εμού εις την θυσίαν. Και ηγίασε τον Ιεσσαί και τους υιούς αυτού και εκάλεσεν αυτούς εις την θυσίαν.
\par 6 Και ενώ εισήρχοντο, ιδών τον Ελιάβ, είπε, Βεβαίως έμπροσθεν του Κυρίου είναι ο κεχρισμένος αυτού.
\par 7 Και είπε Κύριος προς τον Σαμουήλ, Μη επιβλέψης εις την όψιν αυτού ή εις το ύψος του αναστήματος αυτού, επειδή απεδοκίμασα αυτόν· διότι δεν βλέπει ο Κύριος καθώς βλέπει ο άνθρωπος· διότι ο άνθρωπος βλέπει το φαινόμενον, ο δε Κύριος βλέπει την καρδίαν.
\par 8 Τότε εκάλεσεν ο Ιεσσαί τον Αβιναδάβ και διεβίβασεν αυτόν ενώπιον του Σαμουήλ. Και είπεν, ουδέ τούτον δεν εξέλεξεν ο Κύριος.
\par 9 Τότε διεβίβασεν ο Ιεσσαί τον Σαμμά. Ο δε είπεν, Ουδέ τούτον δεν εξέλεξεν ο Κύριος.
\par 10 Και διεβίβασεν ο Ιεσσαί επτά εκ των υιών αυτού ενώπιον του Σαμουήλ. Και είπεν ο Σαμουήλ προς τον Ιεσσαί, Ο Κύριος δεν εξέλεξε τούτους.
\par 11 Και είπεν ο Σαμουήλ προς τον Ιεσσαί, Ετελείωσαν τα παιδία; Και είπε, Μένει έτι ο νεώτερος· και ιδού, ποιμαίνει τα πρόβατα. Και είπεν ο Σαμουήλ προς τον Ιεσσαί, Πέμψον και φέρε αυτόν· διότι δεν θέλομεν καθίσει εις την τράπεζαν, εωσού έλθη ενταύθα.
\par 12 Και έστειλε και έφερεν αυτόν. Ήτο δε ξανθός και ευόφθαλμος και ώραίος την όψιν. Και είπεν ο Κύριος, Σηκώθητι, χρίσον αυτόν· διότι ούτος είναι.
\par 13 Τότε έλαβεν ο Σαμουήλ το κέρας του ελαίου και έχρισεν αυτόν εν μέσω των αδελφών αυτού· και επήλθε πνεύμα Κυρίου επί τον Δαβίδ από της ημέρας εκείνης και εφεξής. Σηκωθείς δε ο Σαμουήλ απήλθεν εις Ραμά.
\par 14 Και το Πνεύμα του Κυρίου απεσύρθη από του Σαούλ, και πνεύμα πονηρόν παρά Κυρίου ετάραττεν αυτόν.
\par 15 Και είπον οι δούλοι του Σαούλ προς αυτόν, Ιδού τώρα, πονηρόν πνεύμα παρά Θεού σε ταράττει·
\par 16 ας προστάξη τώρα ο κύριος ημών τους δούλους σου, τους έμπροσθέν σου, να ζητήσωσιν άνθρωπον ειδήμονα εις το να παίζη κιθάραν· και οπότε το πονηρόν πνεύμα παρά Θεού είναι επί σε, να παίζη με την χείρα αυτού, και καλόν θέλει είσθαι εις σε.
\par 17 Και είπεν ο Σαούλ προς τους δούλους αυτού, Προβλέψατέ μοι λοιπόν άνθρωπον παίζοντα καλώς και φέρετε προς εμέ.
\par 18 Τότε απεκρίθη εις εκ των δούλων και είπεν, Ιδού, είδον υιόν του Ιεσσαί του Βηθλεεμίτου, ειδήμονα εις το παίζειν και ανδρειότατον και άνδρα πολεμικόν και συνετόν εις λόγον και άνθρωπον ώραίον, και ο Κύριος είναι μετ' αυτού.
\par 19 Και απέστειλεν ο Σαούλ μηνυτάς προς τον Ιεσσαί, λέγων, Πέμψον μοι Δαβίδ τον υιόν σου, όστις είναι μετά των προβάτων.
\par 20 Και έλαβεν ο Ιεσσαί όνον φορτωμένον με άρτους και ασκόν οίνου και εν ερίφιον εξ αιγών, και έπεμψεν αυτά διά του Δαβίδ του υιού αυτού προς τον Σαούλ.
\par 21 Και ήλθεν ο Δαβίδ προς τον Σαούλ και εστάθη έμπροσθεν αυτού· και ηγάπησεν αυτόν σφόδρα· και έγεινεν οπλοφόρος αυτού.
\par 22 Και απέστειλεν ο Σαούλ προς τον Ιεσσαί, λέγων, Ο Δαβίδ ας στέκηται, παρακαλώ, έμπροσθέν μου· διότι εύρηκε χάριν εις τους οφθαλμούς μου.
\par 23 Και οπότε το πονηρόν πνεύμα παρά Θεού ήτο επί τον Σαούλ, ο Δαβίδ ελάμβανε την κιθάραν και έπαιζε διά της χειρός αυτού· τότε ανεκουφίζετο ο Σαούλ και ανεπαύετο και απεσύρετο απ' αυτού το πνεύμα το πονηρόν.

\chapter{17}

\par 1 Συνήθροισαν δε οι Φιλισταίοι τα στρατεύματα αυτών διά πόλεμον και ήσαν συνηθροισμένοι εν Σοκχώ, ήτις είναι του Ιούδα, και εστρατοπέδευσαν μεταξύ Σοκχώ και Αζηκά, εν Εφές-δαμμείμ.
\par 2 Ο δε Σαούλ και οι άνδρες Ισραήλ συνηθροίσθησαν, και εστρατοπέδευσαν εν τη κοιλάδι Ηλά, και παρετάχθησαν εις μάχην εναντίον των Φιλισταίων.
\par 3 Και οι μεν Φιλισταίοι ίσταντο επί του όρους εντεύθεν, ο δε Ισραήλ ίστατο επί του όρους εκείθεν· η δε κοιλάς ήτο μεταξύ αυτών.
\par 4 Και εξήλθεν ανήρ προμαχητής εκ του στρατοπέδου των Φιλισταίων ονομαζόμενος Γολιάθ, εκ της Γαθ, ύψους εξ πηχών και σπιθαμής·
\par 5 είχε δε περικεφαλαίαν χαλκίνην επί της κεφαλής αυτού και ήτο ενδεδυμένος θώρακα αλυσιδωτόν· και το βάρος του θώρακος ήτο πέντε χιλιάδες σίκλων χαλκού·
\par 6 και κνημίδας χαλκίνας επί των σκελών αυτού και ασπίδα χαλκίνην μεταξύ των ώμων αυτού.
\par 7 Και το κοντάριον του δόρατος αυτού ήτο ως αντίον υφαντού· και η λόγχη του δόρατος αυτού εζύγιζεν εξακοσίους σίκλους σιδήρου· εις δε κρατών τον θυρεόν προεπορεύετο αυτού.
\par 8 Και σταθείς εβόησε προς τας παρατάξεις του Ισραήλ και είπε προς αυτούς, Διά τι εξέρχεσθε να παραταχθήτε εις μάχην; δεν είμαι εγώ ο Φιλισταίος, και σεις δούλοι του Σαούλ; εκλέξατε εις εαυτούς άνδρα, και ας καταβή προς εμέ·
\par 9 εάν μεν δυνηθή να πολεμήση μετ' εμού και με θανατώση, τότε ημείς θέλομεν είσθαι δούλοί σας· αλλ' εάν εγώ υπερισχύσω κατ' αυτού και θανατώσω αυτόν, τότε σεις θέλετε είσθαι δούλοι ημών και θέλετε δουλεύει ημάς.
\par 10 Και είπεν ο Φιλισταίος, Εγώ εξουθένησα τας παρατάξεις του Ισραήλ την ημέραν ταύτην· δότε εις εμέ άνδρα, διά να μονομαχήσωμεν.
\par 11 Ότε ήκουσεν ο Σαούλ και πας ο Ισραήλ εκείνους τους λόγους του Φιλισταίου, εξέστησαν και εφοβήθησαν σφόδρα.
\par 12 Ήτο δε Δαβίδ ο υιός εκείνου του Εφραθαίου εκ Βηθλεέμ Ιούδα, ονομαζομένου Ιεσσαί· είχε δε οκτώ υιούς· και ο άνθρωπος εις τας ημέρας του Σαούλ είχε τάξιν γέροντος μεταξύ των ανθρώπων.
\par 13 Και υπήγαν οι τρεις υιοί του Ιεσσαί οι μεγαλήτεροι ακολουθούντες τον Σαούλ εις την μάχην· και τα ονόματα των τριών υιών αυτού οίτινες υπήγαν εις την μάχην ήσαν Ελιάβ ο πρωτότοκος, και ο δεύτερος αυτού Αβιναδάβ, και ο τρίτος Σαμμά.
\par 14 Ο δε Δαβίδ ήτο ο νεώτερος· και οι τρεις οι μεγαλήτεροι ηκολούθουν τον Σαούλ.
\par 15 Και ανεχώρει ο Δαβίδ και επέστρεφεν από του Σαούλ, διά να ποιμαίνη τα πρόβατα του πατρός αυτού εν Βηθλεέμ.
\par 16 Ο δε Φιλισταίος επλησίαζε πρωΐ και εσπέρας και εστηλόνετο τεσσαράκοντα ημέρας.
\par 17 Και είπεν Ιεσσαί προς Δαβίδ τον υιόν αυτού, Λάβε τώρα διά τους αδελφούς σου εν εφά εκ τούτου του πεφρυγανισμένου σίτου και τους δέκα τούτους άρτους, και τρέξον εις το στρατόπεδον προς τους αδελφούς σου·
\par 18 και τα δέκα ταύτα νωπά τυρία φέρε προς τον χιλίαρχον, και ιδέ αν υγιαίνωσιν οι αδελφοί σου και λάβε σημείον παρ' αυτών.
\par 19 Ο δε Σαούλ και αυτοί και πάντες οι άνδρες Ισραήλ ήσαν εν τη κοιλάδι Ηλά, μαχόμενοι μετά των Φιλισταίων.
\par 20 Και εξηγέρθη ο Δαβίδ ενωρίς το πρωΐ· και αφήσας τα πρόβατα εις φύλακα, έλαβε και υπήγε, καθώς προσέταξεν αυτόν ο Ιεσσαί· και ήλθεν εις το περιχαράκωμα, ενώ το στράτευμα εξήρχετο εις παράταξιν· και ηλάλαξαν προς την μάχην·
\par 21 διότι παρετάχθησαν ο Ισραήλ και οι Φιλισταίοι, στράτευμα κατά πρόσωπον στρατεύματος.
\par 22 Και ο Δαβίδ, αφήσας επάνωθεν αυτού τα σκεύη εις την χείρα του σκευοφύλακος, έδραμε προς το στράτευμα και ήλθε και ηρώτησε τους αδελφούς αυτού πως έχουσι.
\par 23 Και ενώ ωμίλει μετ' αυτών, ιδού, ανέβαινεν ο προμαχητής, ο Φιλισταίος ο εκ της Γαθ, Γολιάθ το όνομα, εκ των στρατευμάτων των Φιλισταίων, και ελάλησε κατά τους αυτούς λόγους· και ήκουσεν ο Δαβίδ.
\par 24 Πάντες δε οι άνδρες Ισραήλ, ως είδον τον άνδρα, έφυγον από προσώπου αυτού και εφοβήθησαν σφόδρα.
\par 25 Και έλεγον οι άνδρες Ισραήλ, Είδετε τον άνδρα τούτον τον αναβαίνοντα; βεβαίως ανέβη διά να εξουθενήση τον Ισραήλ· και όστις θανατώση αυτόν, τούτον θέλει πλουτίσει ο βασιλεύς με πλούτη μεγάλα, και την θυγατέρα αυτού θέλει δώσει εις αυτόν, και τον οίκον του πατρός αυτού θέλει κάμει ελεύθερον μεταξύ του Ισραήλ.
\par 26 Και είπεν ο Δαβίδ προς τους άνδρας τους ισταμένους πλησίον αυτού, λέγων, Τι θέλει γείνει εις τον άνδρα, όστις πατάξη τον Φιλισταίον τούτον και αφαιρέση το όνειδος από του Ισραήλ; διότι τις είναι ο Φιλισταίος ούτος ο απερίτμητος, ώστε να εξουθενή τα στρατεύματα του Θεού του ζώντος;
\par 27 Και απεκρίθη προς αυτόν ο λαός κατά τον λόγον τούτον, λέγων, ούτω θέλει γείνει εις τον άνδρα, όστις πατάξη αυτόν.
\par 28 Και ήκουσεν Ελιάβ ο αδελφός αυτού ο μεγαλήτερος, ενώ ελάλει προς τους άνδρας· και εξήφθη ο θυμός του Ελιάβ εναντίον του Δαβίδ, και είπε, Διά τι κατέβης ενταύθα; και εις ποίον αφήκες τα ολίγα εκείνα πρόβατα εν τη ερήμω; εγώ εξεύρω την υπερηφανίαν σου και την πονηρίαν της καρδίας σου· βεβαίως διά να ιδής την μάχην κατέβης.
\par 29 Και είπεν ο Δαβίδ, Τι έκαμα τώρα; δεν είναι αιτία;
\par 30 Και εστράφη απ' αυτού προς άλλον και ελάλησε κατά τον αυτόν τρόπον· και ο λαός απεκρίθη πάλιν προς αυτόν κατά τον πρώτον λόγον.
\par 31 Και ότε ηκούσθησαν οι λόγοι, τους οποίους ελάλησεν ο Δαβίδ, ανήγγειλαν προς τον Σαούλ· και παρέλαβεν αυτόν.
\par 32 Και είπεν ο Δαβίδ προς τον Σαούλ, Μηδενός ανθρώπου η καρδία ας μη ταπεινόνηται διά τούτον· ο δούλός σου θέλει υπάγει και πολεμήσει μετά του Φιλισταίου τούτου.
\par 33 Και είπεν ο Σαούλ προς τον Δαβίδ, Δεν δύνασαι να υπάγης εναντίον του Φιλισταίου τούτου διά να πολεμήσης μετ' αυτού· διότι συ είσαι παιδίον, αυτός δε ανήρ πολεμιστής εκ νεότητος αυτού.
\par 34 Και είπεν ο Δαβίδ προς τον Σαούλ, Ο δούλός σου έβοσκε τα πρόβατα του πατρός αυτού, και ήλθε λέων και άρκτος και ήρπασε πρόβατον εκ του ποιμνίου·
\par 35 και εξήλθον κατόπιν αυτού και επάταξα αυτόν και ηλευθέρωσα αυτό εκ του στόματος αυτού· και καθώς εσηκώθη εναντίον μου, ήρπασα αυτόν από της σιαγόνος και επάταξα αυτόν και εθανάτωσα αυτόν·
\par 36 επάταξεν ο δούλός σου και τον λέοντα και την άρκτον· και ο Φιλισταίος ούτος ο απερίτμητος θέλει είσθαι ως εν εκ τούτων, επειδή εξουθένησε τα στρατεύματα του Θεού του ζώντος.
\par 37 Και είπεν ο Δαβίδ, Ο Κύριος ο ελευθερώσας με εκ χειρός του λέοντος και εκ χειρός της άρκτου, ούτος θέλει με ελευθερώσει εκ χειρός του Φιλισταίου τούτου. Και είπεν ο Σαούλ προς τον Δαβίδ, Ύπαγε, και ο Κύριος ας ήναι μετά σου.
\par 38 Και ώπλισεν ο Σαούλ τον Δαβίδ με την πανοπλίαν αυτού και έβαλε χαλκίνην περικεφαλαίαν επί της κεφαλής αυτού· και ενέδυσεν αυτόν θώρακα.
\par 39 Και εζώσθη ο Δαβίδ την ρομφαίαν αυτού επάνωθεν της πανοπλίας αυτού και ηθέλησε να περιπατήση· διότι δεν είχε δοκιμάσει. Και είπεν ο Δαβίδ προς τον Σαούλ, Δεν δύναμαι να περιπατήσω με ταύτα· διότι δεν εδοκίμασα ποτέ. Και εξεδύθη ο Δαβίδ αυτά επάνωθεν αυτού.
\par 40 Και έλαβε την ράβδον αυτού εν τη χειρί αυτού, και εξέλεξεν εις εαυτόν πέντε λίθους ομαλούς εκ του χειμάρρου, και θέσας αυτούς εις το ποιμενικόν αυτού σακκίον και θυλάκιον, την δε σφενδόνην αυτού εις την χείρα αυτού, επλησίαζε προς τον Φιλισταίον.
\par 41 Ο δε Φιλισταίος ήρχετο προχωρών και επλησίαζε προς τον Δαβίδ· και ο ανήρ ο ασπιδοφόρος έμπροσθεν αυτού.
\par 42 Και ότε περιέβλεψεν ο Φιλισταίος και είδε τον Δαβίδ, κατεφρόνησεν αυτόν· διότι ήτο παιδίον και ξανθός και ώραίος την όψιν.
\par 43 Και είπεν ο Φιλισταίος προς τον Δαβίδ, Κύων είμαι εγώ, ώστε έρχεσαι προς εμέ με ράβδους; Και κατηράσθη ο Φιλισταίος τον Δαβίδ εις τους θεούς αυτού.
\par 44 Και είπεν ο Φιλισταίος προς τον Δαβίδ, Ελθέ προς εμέ, και θέλω παραδώσει τας σάρκας σου εις τα πετεινά του ουρανού και εις τα θηρία του αγρού.
\par 45 Και είπεν ο Δαβίδ προς τον Φιλισταίον, Συ έρχεσαι εναντίον μου με ρομφαίαν και δόρυ και ασπίδα· εγώ δε έρχομαι εναντίον σου εν τω ονόματι του Κυρίου των δυνάμεων, του Θεού των στρατευμάτων του Ισραήλ, τα οποία συ εξουθένησας·
\par 46 την ημέραν ταύτην θέλει σε παραδώσει ο Κύριος εις την χείρα μου· και θέλω σε πατάξει και αφαιρέσει από σου την κεφαλήν σου· και θέλω παραδώσει τα πτώματα του στρατοπέδου των Φιλισταίων την ημέραν ταύτην εις τα πετεινά του ουρανού, και εις τα θηρία της γής· διά να γνωρίση πάσα η γη ότι είναι Θεός εις τον Ισραήλ·
\par 47 και θέλει γνωρίσει παν το πλήθος τούτο ότι ο Κύριος δεν σώζει με ρομφαίαν και δόρυ· διότι του Κυρίου είναι η μάχη, και αυτός θέλει σας παραδώσει εις την χείρα ημών.
\par 48 Και ότε εσηκώθη ο Φιλισταίος και ήρχετο και επλησίαζεν εις συνάντησιν του Δαβίδ, ο Δαβίδ έσπευσε και έδραμε προς μάχην εναντίον του Φιλισταίου.
\par 49 Και εκτείνας ο Δαβίδ την χείρα αυτού εις το σακκίον, έλαβεν εκείθεν λίθον και εσφενδόνησε και εκτύπησε τον Φιλισταίον κατά το μέτωπον αυτού, ώστε ο λίθος ενεπήχθη εις το μέτωπον αυτού· και έπεσε κατά πρόσωπον εις την γην.
\par 50 και υπερίσχυσεν ο Δαβίδ κατά του Φιλισταίου διά της σφενδόνης και διά του λίθου, και εκτύπησε τον Φιλισταίον και εθανάτωσεν αυτόν. Αλλά δεν ήτο ρομφαία εν τη χειρί του Δαβίδ·
\par 51 όθεν ο Δαβίδ έδραμε και σταθείς επί τον Φιλισταίον, έλαβε την ρομφαίαν αυτού και έσυρεν αυτήν εκ της θήκης αυτής, και θανατώσας αυτόν, απέκοψε την κεφαλήν αυτού με αυτήν. Ιδόντες δε οι Φιλισταίοι, ότι απέθανεν ο ισχυρός αυτών, έφυγον·
\par 52 Τότε εσηκώθησαν οι άνδρες του Ισραήλ και του Ιούδα και ηλάλαξαν και κατεδίωξαν τους Φιλισταίους, έως της εισόδου της κοιλάδος, και έως των πυλών της Ακκαρών. Και έπεσον οι τραυματισμένοι των Φιλισταίων εν τη οδώ Σααραείμ, έως Γαθ και έως Ακκαρών.
\par 53 Και επέστρεψαν οι υιοί Ισραήλ εκ της καταδιώξεως των Φιλισταίων και διήρπασαν τα στρατόπεδα αυτών.
\par 54 Ο δε Δαβίδ έλαβε την κεφαλήν του Φιλισταίου, και έφερεν αυτήν εις Ιερουσαλήμ· την δε πανοπλίαν αυτού έβαλεν εν τη σκηνή αυτού.
\par 55 Ότε δε είδεν ο Σαούλ τον Δαβίδ εξερχόμενον εναντίον του Φιλισταίου, είπε προς Αβενήρ, τον αρχηγόν του στρατεύματος, Αβενήρ, τίνος υιός είναι ο νέος ούτος; Και ο Αβενήρ είπε, Ζη η ψυχή σου, βασιλεύ, δεν εξεύρω.
\par 56 Και είπεν ο βασιλεύς, Ερώτησον συ, τίνος υιός είναι ο νεανίσκος ούτος.
\par 57 Και καθώς επέστρεψεν ο Δαβίδ, πατάξας τον Φιλισταίον, παρέλαβεν αυτόν ο Αβενήρ και έφερεν αυτόν ενώπιον του Σαούλ· και η κεφαλή του Φιλισταίου ήτο εν τη χειρί αυτού.
\par 58 Και είπε προς αυτόν ο Σαούλ, Τίνος υιός είσαι, νέε; και απεκρίθη ο Δαβίδ, Ο υιός του δούλου σου Ιεσσαί του Βηθλεεμίτου.

\chapter{18}

\par 1 Και ως ετελείωσε λαλών προς τον Σαούλ, η ψυχή του Ιωνάθαν συνεδέθη μετά της ψυχής του Δαβίδ, και ηγάπησεν αυτόν ο Ιωνάθαν ως την ιδίαν αυτού ψυχήν.
\par 2 Και παρέλαβεν αυτόν ο Σαούλ εκείνην την ημέραν και δεν αφήκεν αυτόν να επιστρέψη πλέον εις τον οίκον του πατρός αυτού.
\par 3 Τότε ο Ιωνάθαν έκαμε συνθήκην μετά του Δαβίδ· διότι ηγάπα αυτόν ως την ιδίαν αυτού ψυχήν.
\par 4 και εκδυθείς ο Ιωνάθαν το επένδυμα το εφ' εαυτόν, έδωκεν αυτό εις τον Δαβίδ, και την στολήν αυτού, έως και αυτό το ξίφος αυτού και το τόξον αυτού και την ζώνην αυτού.
\par 5 και εξήρχετο ο Δαβίδ πανταχού όπου έπεμπεν αυτόν ο Σαούλ, και εφέρετο μετά συνέσεως· και κατέστησεν αυτόν ο Σαούλ επί τους άνδρας του πολέμου· και ήτο αρεστός εις τους οφθαλμούς παντός του λαού, έτι δε και εις τους οφθαλμούς των δούλων του Σαούλ.
\par 6 Καθώς δε ήρχοντο, ενώ επέστρεφεν ο Δαβίδ εκ της σφαγής του Φιλισταίου, εξήρχοντο αι γυναίκες εκ πασών των πόλεων του Ισραήλ, ψάλλουσαι και χορεύουσαι, εις συνάντησιν του βασιλέως Σαούλ, μετά τυμπάνων, μετά χαράς και μετά κυμβάλων.
\par 7 Και απεκρίνοντο αι γυναίκες αι παίζουσαι προς αλλήλας, και έλεγον, Ο Σαούλ επάταξε τας χιλιάδας αυτού, και ο Δαβίδ τας μυριάδας αυτού.
\par 8 Παρωξύνθη δε σφόδρα ο Σαούλ, και εφάνη δυσάρεστος εις τους οφθαλμούς αυτού ο λόγος ούτος, και είπεν, Απέδωκαν εις τον Δαβίδ τας μυριάδας, εις εμέ δε απέδωκαν τας χιλιάδας· και τι λείπεται πλέον εις αυτόν παρά η βασιλεία;
\par 9 Και υπέβλεπεν ο Σαούλ τον Δαβίδ απ' εκείνης της ημέρας και εις το εξής.
\par 10 Και την επαύριον επήλθε πνεύμα πονηρόν παρά Θεού επί τον Σαούλ, και επροφήτευεν εν μέσω του οίκου· και ο Δαβίδ έπαιζε διά της χειρός αυτού, ως καθ' εκάστην ημέραν· ήτο δε το δοράτιον εν τη χειρί του Σαούλ·
\par 11 και έρριψεν ο Σαούλ το δοράτιον, λέγων, Θέλω κτυπήσει τον Δαβίδ έως και εις τον τοίχον. Αλλ' ο Δαβίδ εξέκλινεν απ' έμπροσθεν αυτού δις.
\par 12 Εφοβήθη δε ο Σαούλ από προσώπου Δαβίδ, επειδή ο Κύριος ήτο μετ' αυτού, από δε του Σαούλ είχεν απομακρυνθή.
\par 13 Όθεν απεμάκρυνεν αυτόν ο Σαούλ από πλησίον εαυτού και κατέστησεν αυτόν χιλίαρχον· και εξήρχετο και εισήρχετο έμπροσθεν του λαού.
\par 14 Και εφέρετο ο Δαβίδ μετά συνέσεως εν πάσαις ταις οδοίς αυτού· και ο Κύριος ήτο μετ' αυτού.
\par 15 Διά τούτο ο Σαούλ, βλέπων ότι εφέρετο μετά μεγάλης συνέσεως, εφοβείτο από προσώπου αυτού.
\par 16 Πας δε ο Ισραήλ και ο Ιούδας ηγάπα τον Δαβίδ, επειδή εξήρχετο και εισήρχετο έμπροσθεν αυτών.
\par 17 Και είπεν ο Σαούλ προς τον Δαβίδ, Ιδού, η μεγαλητέρα θυγάτηρ μου Μεράβ· ταύτην θέλω σοι δώσει εις γυναίκα· μόνον έσο ανδρείος εις εμέ και μάχου τας μάχας του Κυρίου. Διότι είπεν ο Σαούλ, Ας μη ήναι η χειρ μου επ' αυτόν, αλλ' η χειρ των Φιλισταίων ας ήναι επ' αυτόν.
\par 18 Και είπεν ο Δαβίδ προς τον Σαούλ, Ποίος εγώ; και ποία η ζωή μου και η οικογένεια του πατρός μου μεταξύ του Ισραήλ, ώστε να γείνω γαμβρός του βασιλέως;
\par 19 Αλλά καθ' ον καιρόν η Μεράβ η θυγάτηρ του Σαούλ έπρεπε να δοθή εις τον Δαβίδ, αυτή εδόθη εις τον Αδριήλ τον Μεολαθίτην εις γυναίκα.
\par 20 Ηγάπα δε τον Δαβίδ Μιχάλ η θυγάτηρ του Σαούλ· και ανήγγειλαν τούτο προς τον Σαούλ· και το πράγμα ήρεσεν εις αυτόν.
\par 21 Και είπεν ο Σαούλ, Θέλω δώσει αυτήν εις αυτόν, διά να γείνη παγίς εις αυτόν, και διά να ήναι επ' αυτόν η χειρ των Φιλισταίων. Όθεν είπεν ο Σαούλ προς τον Δαβίδ, Σήμερον θέλεις είσθαι γαμβρός μου με την δευτέραν.
\par 22 Και προσέταξεν ο Σαούλ τους δούλους αυτού, λέγων, Λαλήσατε προς τον Δαβίδ κρυφίως και είπατε, Ιδού, ο βασιλεύς ευαρεστείται εις σε, και πάντες οι δούλοι αυτού σε αγαπώσι· τώρα λοιπόν γενού γαμβρός του βασιλέως.
\par 23 Και ελάλησαν οι δούλοι του Σαούλ τους λόγους τούτους εις τα ώτα του Δαβίδ. Και είπεν ο Δαβίδ, Σας φαίνεται μικρόν να γείνη τις γαμβρός βασιλέως; αλλ' εγώ είμαι άνθρωπος πτωχός και ποταπός.
\par 24 Και ανήγγειλαν οι δούλοι του Σαούλ προς αυτόν, λέγοντες, Κατά τους λόγους τούτους ελάλησεν ο Δαβίδ.
\par 25 Και είπεν ο Σαούλ, Ούτω θέλετε ειπεί προς τον Δαβίδ, Ο βασιλεύς δεν θέλει δώρα νυμφικά, αλλ' εκατόν ακροβυστίας Φιλισταίων, διά να εκδικηθή ο βασιλεύς εναντίον των εχθρών αυτού. Ο Σαούλ όμως εστοχάζετο να κάμη τον Δαβίδ να πέση διά χειρός των Φιλισταίων.
\par 26 Και ότε ανήγγειλαν οι δούλοι αυτού προς τον Δαβίδ τους λόγους τούτους, ήρεσεν εις τον Δαβίδ να γείνη γαμβρός του βασιλέως· όθεν και πριν αι ημέραι πληρωθώσιν,
\par 27 εσηκώθη ο Δαβίδ και υπήγεν, αυτός και οι άνδρες αυτού, και εθανάτωσεν εκ των Φιλισταίων διακοσίους άνδρας· και έφερεν ο Δαβίδ τας ακροβυστίας αυτών, και απέδωκαν αυτάς πλήρεις εις τον βασιλέα, διά να γείνη γαμβρός του βασιλέως. Και έδωκεν εις αυτόν ο Σαούλ Μιχάλ την θυγατέρα αυτού εις γυναίκα.
\par 28 Και είδεν ο Σαούλ και εγνώρισεν ότι ο Κύριος ήτο μετά του Δαβίδ· και Μιχάλ η θυγάτηρ του Σαούλ ηγάπα αυτόν.
\par 29 Και έτι μάλλον εφοβείτο ο Σαούλ από προσώπου του Δαβίδ· και έγεινεν ο Σαούλ παντοτεινός εχθρός του Δαβίδ.
\par 30 Εξήλθον δε οι άρχοντες των Φιλισταίων εις πόλεμον· και αφ' ης ημέρας εξήλθον, ο Δαβίδ εφέρετο μετά συνέσεως μεγαλητέρας παρά πάντας τους δούλους του Σαούλ· όθεν το όνομα αυτού ετιμήθη σφόδρα.

\chapter{19}

\par 1 Και είπεν ο Σαούλ προς Ιωνάθαν τον υιόν αυτού και προς πάντας τους δούλους αυτού, να θανατώσωσι τον Δαβίδ.
\par 2 Ο Ιωνάθαν όμως, ο υιός του Σαούλ, ηγάπα καθ' υπερβολήν τον Δαβίδ· και απήγγειλεν ο Ιωνάθαν προς τον Δαβίδ, λέγων, Σαούλ ο πατήρ μου ζητεί να σε θανατώση· τώρα λοιπόν φυλάχθητι, παρακαλώ, έως πρωΐ, και μένε εν αποκρύφω τόπω και κρύπτου·
\par 3 εγώ δε θέλω εξέλθει και σταθή πλησίον του πατρός μου εν τω αγρώ όπου θέλεις είσθαι, και θέλω ομιλήσει περί σου προς τον πατέρα μου· και θέλω ιδεί τι είναι και θέλω σοι απαγγείλει.
\par 4 Και ελάλησεν ο Ιωνάθαν καλά περί του Δαβίδ προς τον Σαούλ τον πατέρα αυτού και είπε προς αυτόν, Ας μη αμαρτήση ο βασιλεύς εναντίον του δούλου αυτού, εναντίον του Δαβίδ· επειδή δεν ημάρτησεν εναντίον σου και επειδή τα έργα αυτού εστάθησαν εις σε πολύ καλά·
\par 5 διότι ερριψοκινδύνευσε την ζωήν αυτού και εθανάτωσε τον Φιλισταίον, και ο Κύριος έκαμε σωτηρίαν μεγάλην εις πάντα τον Ισραήλ· είδες και εχάρης· διά τι λοιπόν θέλεις να αμαρτήσης εναντίον αθώου αίματος, θανατόνων τον Δαβίδ χωρίς αιτίας;
\par 6 Και υπήκουσεν ο Σαούλ εις την φωνήν του Ιωνάθαν· και ώμοσεν ο Σαούλ, λέγων, Ζη Κύριος, δεν θέλει θανατωθή.
\par 7 Και έκραξεν ο Ιωνάθαν τον Δαβίδ και απήγγειλε προς αυτόν ο Ιωνάθαν πάντας τους λόγους τούτους. Και έφερεν ο Ιωνάθαν τον Δαβίδ προς τον Σαούλ, και ήτο ενώπιον αυτού ως το πρότερον.
\par 8 Έγεινε δε πάλιν πόλεμος· και εξήλθεν ο Δαβίδ και επολέμησε μετά των Φιλισταίων και επάταξεν αυτούς εν σφαγή μεγάλη· και έφυγον από προσώπου αυτού.
\par 9 Και το πονηρόν πνεύμα παρά Κυρίου εστάθη επί τον Σαούλ, ενώ εκάθητο εν τω οίκω αυτού μετά του δορατίου εν τη χειρί αυτού· ο δε Δαβίδ έπαιζε το όργανον διά της χειρός αυτού.
\par 10 Και εζήτησεν ο Σαούλ να κτυπήση με το δοράτιον τον Δαβίδ και έως εις τον τοίχον· εξέκλινεν όμως από προσώπου του Σαούλ και εκτύπησε τον τοίχον με το δοράτιον· ο δε Δαβίδ έφυγε και διεσώθη εκείνην την νύκτα.
\par 11 Και απέστειλεν ο Σαούλ μηνυτάς προς τον οίκον του Δαβίδ, διά να παραφυλάξωσιν αυτόν και να θανατώσωσιν αυτόν το πρωΐ· απήγγειλε δε προς τον Δαβίδ η Μιχάλ, η γυνή αυτού, λέγουσα, Εάν δεν σώσης την ζωήν σου την νύκτα ταύτην, αύριον θέλεις θανατωθή.
\par 12 Και κατεβίβασεν η Μιχάλ τον Δαβίδ διά της θυρίδος· και ανεχώρησε και έφυγε και διεσώθη.
\par 13 Τότε λαβούσα η Μιχάλ ομοίωμα, έθεσεν επί της κλίνης και έβαλεν εις την κεφαλήν αυτού προσκεφάλαιον εκ τριχών αιγών και εσκέπασεν αυτό με φόρεμα.
\par 14 Και ότε απέστειλεν ο Σαούλ μηνυτάς διά να συλλάβωσι τον Δαβίδ, εκείνη είπεν, Άρρωστος είναι.
\par 15 Πάλιν απέστειλεν ο Σαούλ τους μηνυτάς διά να ίδωσι τον Δαβίδ, λέγων, Φέρετέ μοι αυτόν επί της κλίνης, διά να θανατώσω αυτόν.
\par 16 Και ότε εισήλθον οι μηνυταί, ιδού, ήτο το ομοίωμα επί της κλίνης και προσκεφάλαιον εις την κεφαλήν αυτού εκ τριχών αιγών.
\par 17 Και είπεν ο Σαούλ προς την Μιχάλ, Διά τι με ηπάτησας ούτω και απέπεμψας τον εχθρόν μου και διεσώθη; Και απεκρίθη Μιχάλ προς τον Σαούλ, Αυτός είπε προς εμέ, Άφες με να φύγω· διά τι να σε θανατώσω;
\par 18 Και έφυγεν ο Δαβίδ και διεσώθη και ήλθε προς τον Σαμουήλ εις Ραμά, και απήγγειλε προς αυτόν πάντα όσα είχε κάμει εις αυτόν ο Σαούλ· και υπήγαν, αυτός και ο Σαμουήλ, και κατώκησαν εν Ναυϊώθ.
\par 19 Απήγγειλαν δε προς τον Σαούλ και είπον, Ιδού, ο Δαβίδ είναι εν Ναυϊώθ εν Ραμά.
\par 20 Και απέστειλεν ο Σαούλ μηνυτάς να συλλάβωσι τον Δαβίδ· και ότε είδον την σύναξιν των προφητών προφητευόντων και τον Σαμουήλ προϊστάμενον επ' αυτούς, επήλθε Πνεύμα Θεού επί τους μηνυτάς του Σαούλ, και προεφήτευον και αυτοί.
\par 21 Και ότε απηγγέλθη προς τον Σαούλ, απέστειλεν άλλους μηνυτάς· και αυτοί ομοίως προεφήτευον. Και απέστειλε πάλιν ο Σαούλ τρίτην φοράν μηνυτάς, και αυτοί έτι προεφήτευον.
\par 22 Τότε υπήγε και αυτός εις Ραμά και ήλθεν έως του μεγάλου φρέατος του εν Σοκχώ· και ηρώτησε, λέγων, Που είναι ο Σαμουήλ και ο Δαβίδ; Και είπον, Ιδού, εν Ναυϊώθ εν Ραμά.
\par 23 Και υπήγεν εκεί εις Ναυϊώθ την εν Ραμά· και Πνεύμα Θεού επήλθε και επ' αυτόν· και εξηκολούθει την οδόν αυτού προφητεύων, εωσού ήλθεν εις Ναυϊώθ εν Ραμά.
\par 24 Και εκδυθείς τα ιμάτια αυτού και αυτός, προεφήτευεν ενώπιον του Σαμουήλ κατά τον αυτόν τρόπον, και κατέκειτο γυμνός όλην εκείνην την ημέραν και όλην την νύκτα. Διά τούτο λέγουσι, Και Σαούλ εν προφήταις;

\chapter{20}

\par 1 Και έφυγεν ο Δαβίδ εκ Ναυϊώθ της εν Ραμά, και ήλθε και είπεν ενώπιον του Ιωνάθαν, Τι έπραξα; τι το αδίκημά μου και τι το αμάρτημά μου έμπροσθεν του πατρός σου, διά το οποίον ζητεί την ψυχήν μου;
\par 2 Ο δε είπε προς αυτόν, Μη γένοιτο· συ δεν θέλεις αποθάνει ιδού, ο πατήρ μου δεν θέλει κάμει ουδέν, είτε μέγα είτε μικρόν, το οποίον να μη φανερώση εις εμέ· και διά τι ο πατήρ μου ήθελε κρύψει το πράγμα τούτο απ' εμού; δεν είναι ούτω.
\par 3 Και ώμοσεν ο Δαβίδ έτι και είπεν, Ο πατήρ σου εξεύρει βεβαίως ότι εγώ εύρηκα χάριν ενώπιόν σου· όθεν λέγει, Ας μη εξεύρη τούτο ο Ιωνάθαν, μήποτε λυπηθή. Αλλά, ζη Κύριος και ζη η ψυχή σου, δεν είναι παρά εν βήμα μεταξύ εμού και του θανάτου.
\par 4 Τότε είπεν ο Ιωνάθαν προς τον Δαβίδ, ό,τι επιθυμεί η ψυχή σου θέλω κάμει εις σε.
\par 5 Και είπεν ο Δαβίδ προς τον Ιωνάθαν, Ιδού, αύριον είναι νεομηνία, καθ' ην εγώ συνειθίζω να κάθωμαι μετά του βασιλέως να συντρώγω· άφες με λοιπόν να υπάγω, διά να κρυφθώ εν τω αγρώ μέχρι της εσπέρας της τρίτης ημέρας.
\par 6 εάν ο πατήρ σου περιβλέπων με ζητήση, τότε ειπέ, Ο Δαβίδ εζήτησεν ενθέρμως παρ' εμού να τρέξη εις Βηθλεέμ την πόλιν αυτού. διότι γίνεται εκεί ετήσιος θυσία υφ' όλης της συγγενείας αυτού·
\par 7 εάν είπη ούτω, Καλώς· θέλει είσθαι ειρήνη εις τον δούλον σου· εάν όμως οργισθή πολύ, έξευρε ότι το κακόν είναι αποφασισμένον παρ' αυτού·
\par 8 θέλεις λοιπόν κάμει έλεος προς τον δούλον σου· διότι εις συνθήκην Κυρίου εισήγαγες τον δούλον σου μετά σεαυτού· εάν όμως ήναι αδικία εν εμοί, θανάτωσόν με σύ· και διά τι να με φέρης έως του πατρός σου;
\par 9 Και είπεν ο Ιωνάθαν, Μη γένοιτο ποτέ τούτο εις σέ· διότι, εάν τω όντι γνωρίσω ότι το κακόν είναι αποφασισμένον παρά του πατρός μου να έλθη επί σε, βεβαίως θέλω σοι απαγγείλει τούτο.
\par 10 Και είπεν ο Δαβίδ προς τον Ιωνάθαν, Τις θέλει μοι απαγγείλει εάν ο πατήρ σου αποκριθή εις σε σκληρά;
\par 11 Και είπεν ο Ιωνάθαν προς τον Δαβίδ, Ελθέ, και ας εξέλθωμεν εις τον αγρόν. Και εξήλθον αμφότεροι εις τον αγρόν.
\par 12 Και είπεν ο Ιωνάθαν προς τον Δαβίδ, Κύριε Θεέ του Ισραήλ· όταν ποτέ περί την αύριον ή την μετά την αύριον, εξιχνιάσω τον πατέρα μου, και ιδού, είναι τι καλόν περί του Δαβίδ, εάν δεν αποστείλω τότε προς σε να σοι το απαγγείλω,
\par 13 ούτω να κάμη ο Κύριος εις τον Ιωνάθαν και ούτω να προσθέση· εάν δε ο πατήρ μου απεφάσισε το κακόν εναντίον σου, θέλω σοι απαγγείλει τούτο και σε εξαποστείλει, και θέλεις υπάγει εν ειρήνη· και ο Κύριος ας ήναι μετά σου, καθώς εστάθη μετά του πατρός μου·
\par 14 και ουχί μόνον ενόσω ζω, θέλεις δείξει προς εμέ το έλεος του Κυρίου, διά να μη αποθάνω·
\par 15 αλλά και δεν θέλεις αποκόψει το έλεός σου από του οίκου μου εις τον αιώνα· ουχί, ουδέ όταν ο Κύριος αφανίση τους εχθρούς του Δαβίδ έκαστον από προσώπου της γης.
\par 16 Και έκαμεν ο Ιωνάθαν συνθήκην μετά του οίκου του Δαβίδ, επιλέγων, Και ο Κύριος να εκζητήση λόγον παρά των εχθρών του Δαβίδ.
\par 17 Και έκαμεν έτι ο Ιωνάθαν τον Δαβίδ να ομόση εις την αγάπην αυτού την προς αυτόν· διότι ηγάπα αυτόν καθώς ηγάπα την ιδίαν αυτού ψυχήν.
\par 18 Και είπε προς αυτόν ο Ιωνάθαν, Αύριον είναι νεομηνία· και θέλεις ζητηθή, διότι η καθέδρα σου θέλει είσθαι κενή·
\par 19 και αφού σταθής τρεις ημέρας, θέλεις καταβή μετά σπουδής και ελθεί εις τον τόπον, όπου εκρύφθης την ημέραν της πράξεως, και θέλεις καθίσει πλησίον της πέτρας Εζήλ·
\par 20 και εγώ θέλω τοξεύσει τρία βέλη εις το πλάγιον αυτής, ως τοξεύων εις σημείον·
\par 21 και ιδού, θέλω αποστείλει τον υπηρέτην, λέγων, Ύπαγε, ευρέ τα βέλη· εάν ρητώς είπω εις τον υπηρέτην, Ιδού, τα βέλη είναι εδώθεν από σου, λάβε αυτά· τότε ελθέ, διότι είναι ειρήνη εις σε, και ουδεμία βλάβη, ζη Κύριος·
\par 22 εάν όμως είπω ούτω προς τον νέον, Ιδού, τα βέλη είναι επέκεινα από σού· ύπαγε την οδόν σου, διότι σε εξαπέστειλεν ο Κύριος·
\par 23 περί δε του λόγου, τον οποίον ώμιλήσαμεν εγώ και συ, ιδού, ο Κύριος ας ήναι μάρτυς μεταξύ εμού και σου εις τον αιώνα.
\par 24 Εκρύφθη λοιπόν ο Δαβίδ εν τω αγρώ· και ότε ήλθεν η νεομηνία, ο βασιλεύς εκάθισεν εις την τράπεζαν διά να φάγη.
\par 25 Και ο βασιλεύς εκάθισεν επί της καθέδρας αυτού, ως άλλοτε, επί καθέδρας πλησίον του τοίχου· και ο Ιωνάθαν εσηκώθη και εκάθισεν ο Αβενήρ πλησίον του Σαούλ, ο δε τόπος του Δαβίδ ήτο κενός.
\par 26 Ο Σαούλ όμως δεν ελάλησεν ουδέν την ημέραν εκείνην· διότι είπε καθ' εαυτόν, Τίποτε συνέβη εις αυτόν ώστε να μη ήναι καθαρός· βεβαίως δεν είναι καθαρός.
\par 27 Και το πρωΐ, την δευτέραν του μηνός, ο τόπος του Δαβίδ ήτο κενός· και είπεν ο Σαούλ προς Ιωνάθαν τον υιόν αυτού, Διά τι δεν ήλθεν ο υιός του Ιεσσαί εις την τράπεζαν, ούτε χθές ούτε σήμερον;
\par 28 Και απεκρίθη ο Ιωνάθαν προς τον Σαούλ, Ο Δαβίδ εζήτησεν ενθέρμως παρ' εμού να υπάγη έως Βηθλεέμ,
\par 29 και είπεν, Ας υπάγω, παρακαλώ, διότι η συγγένεια ημών κάμνει θυσίαν εν τη πόλει· και ο αδελφός μου αυτός παρήγγειλεν εις εμέ να παρευρεθώ· τώρα λοιπόν, εάν εύρηκα χάριν εις τους οφθαλμούς σου, άφες με, παρακαλώ, να υπάγω και να ίδω τους αδελφούς μου· διά τούτο δεν ήλθεν εις την τράπεζαν του βασιλέως.
\par 30 Τότε εξήφθη η οργή του Σαούλ κατά του Ιωνάθαν, και είπε προς αυτόν, Υιέ διεφθαρμένης και αποστάτιδος, δεν εξεύρω ότι συ εξέλεξας τον υιόν του Ιεσσαί δι' αισχύνην σου και δι' αισχύνην της γυμνώσεως της μητρός σου;
\par 31 διότι ενόσω ο υιός του Ιεσσαί ζη επί της γης, συ δεν θέλεις στερεωθή ουδέ η βασιλεία σου· τώρα λοιπόν πέμψον και φέρε αυτόν προς εμέ· διότι εξάπαντος θέλει αποθάνει.
\par 32 Και απεκρίθη ο Ιωνάθαν προς τον Σαούλ τον πατέρα αυτού και είπε προς αυτόν, Διά τι να θανατωθή; τι έπραξε;
\par 33 Και έρριψεν ο Σαούλ δοράτιον κατ' αυτού, διά να κτυπήση αυτόν· τότε εγνώρισεν ο Ιωνάθαν, ότι ήτο αποφασισμένον παρά του πατρός αυτού να θανατώση τον Δαβίδ.
\par 34 Και εσηκώθη ο Ιωνάθαν από της τραπέζης με έξαψιν θυμού και δεν έφαγεν άρτον την δευτέραν ημέραν του μηνός· διότι ήτο λυπημένος διά τον Δαβίδ, επειδή είχε καταισχύνει αυτόν ο πατήρ αυτού.
\par 35 Και το πρωΐ εξήλθεν ο Ιωνάθαν εις τον αγρόν, κατά τον καιρόν τον προσδιορισθέντα μετά του Δαβίδ, έχων μεθ' εαυτού μικρόν παιδάριον.
\par 36 Και είπε προς το παιδάριον αυτού, Τρέξον, ευρέ τώρα τα βέλη, τα οποία εγώ τοξεύω. Και καθώς έτρεχε το παιδάριον, ετόξευσε το βέλος πέραν αυτού.
\par 37 Και ότε το παιδάριον ήλθεν εις τον τόπον του βέλους, το οποίον ο Ιωνάθαν είχε τοξεύσει, εφώναξεν ο Ιωνάθαν κατόπιν του παιδαρίου και είπε, Δεν είναι το βέλος πέραν από σου;
\par 38 Και εφώναξεν ο Ιωνάθαν κατόπιν του παιδαρίου, Τάχυνον, σπεύσον, μη σταθής. Και εσύναξε το παιδάριον του Ιωνάθαν τα βέλη και ήλθε προς τον κύριον αυτού.
\par 39 Το παιδάριον όμως δεν ήξευρεν ουδέν· μόνος ο Ιωνάθαν και ο Δαβίδ ήξευρον την υπόθεσιν.
\par 40 Και έδωκεν ο Ιωνάθαν τα όπλα αυτού εις το παιδάριον το μεθ' αυτού και είπε προς αυτό, Ύπαγε, φέρε αυτά εις την πόλιν.
\par 41 Καθώς δε ανεχώρησε το παιδάριον, εσηκώθη ο Δαβίδ εκ του μεσημβρινού μέρους και έπεσε κατά πρόσωπον αυτού εις την γην και προσεκύνησε τρίς· και ησπάσθησαν αλλήλους και έκλαυσαν αμφότεροι· ο δε Δαβίδ έκαμε κλαυθμόν μέγαν.
\par 42 Και είπεν ο Ιωνάθαν προς τον Δαβίδ, Ύπαγε εν ειρήνη, καθώς ώμόσαμεν ημείς αμφότεροι εις το όνομα του Κυρίου, λέγοντες, Ο Κύριος ας ήναι μεταξύ εμού και σου, και μεταξύ του σπέρματός μου και του σπέρματός σου εις τον αιώνα. Και εσηκώθη και ανεχώρησεν· ο δε Ιωνάθαν εισήλθεν εις την πόλιν.

\chapter{21}

\par 1 Και ήλθεν ο Δαβίδ εις Νωβ, προς Αχιμέλεχ τον ιερέα· εξεπλάγη δε ο Αχιμέλεχ εις την συνάντησιν του Δαβίδ και είπε προς αυτόν, Διά τι συ μόνος, και δεν είναι ουδείς μετά σου;
\par 2 Και είπεν ο Δαβίδ προς τον Αχιμέλεχ τον ιερέα, Ο βασιλεύς προσέταξεν εις εμέ υπόθεσίν τινά και μοι είπεν, Ας μη εξεύρη μηδείς μηδέν περί της υποθέσεως, διά την οποίαν εγώ σε αποστέλλω, μηδέ τι προσέταξα εις εσέ· και διώρισα εις τους δούλους τον δείνα και δείνα τόπον.
\par 3 Τώρα λοιπόν τι σοι είναι πρόχειρον; δος πέντε άρτους εις την χείρα μου, ή ό,τι ευρίσκεται.
\par 4 Και απεκρίθη ο ιερεύς προς τον Δαβίδ, και είπε, Δεν έχω πρόχειρον ουδένα κοινόν άρτον, αλλ' είναι άρτοι ηγιασμένοι· οι νέοι εφυλάχθησαν καθαροί τουλάχιστον από γυναικών;
\par 5 Και απεκρίθη ο Δαβίδ προς τον ιερέα και είπε προς αυτόν, Μάλιστα αι γυναίκες είναι μακράν αφ' ημών εις τας τρεις ταύτας ημέρας, αφού εξήλθον, και τα σκεύη των νέων είναι καθαρά· και ούτος ο άρτος είναι τρόπον τινά κοινός, μάλιστα επειδή σήμερον είναι άλλος ηγιασμένος εις τα σκεύη.
\par 6 Έδωκε λοιπόν ο ιερεύς εις αυτόν τους άρτους τους αγίους· διότι δεν ήτο εκεί άρτος παρά τους άρτους της προθέσεως, οίτινες είχον σηκωθή απ' έμπροσθεν του Κυρίου, διά να θέσωσιν άρτους ζεστούς καθ' ην ημέραν εσηκώθησαν εκείνοι.
\par 7 Ήτο δε εκεί άνθρωπός τις εκ των δούλων του Σαούλ, την ημέραν εκείνην, κρατούμενος ενώπιον του Κυρίου· και το όνομα αυτού Δωήκ, ο Ιδουμαίος, ο πρώτιστος των ποιμένων του Σαούλ.
\par 8 Και είπεν ο Δαβίδ προς τον Αχιμέλεχ, Και δεν έχεις εδώ πρόχειρον κανέν δόρυ ή ρομφαίαν; διότι ούτε την ρομφαίαν μου ούτε τα όπλα μου έλαβον εν τη χειρί μου, επειδή του βασιλέως η υπόθεσις ήτο κατεπείγουσα.
\par 9 Και είπεν ο ιερεύς, Η ρομφαία Γολιάθ του Φιλισταίου, τον οποίον επάταξας εν τη κοιλάδι Ηλά, ιδού είναι περιτετυλιγμένη εις φόρεμα όπισθεν του εφόδ· εάν θέλης να λάβης αυτήν, λάβε· διότι ενταύθα δεν είναι άλλη παρά εκείνην. Και είπεν ο Δαβίδ. Δεν είναι ουδεμία ως αυτή· δος μοι αυτήν.
\par 10 Και εσηκώθη ο Δαβίδ και έφυγε την ημέραν εκείνην από προσώπου του Σαούλ, και υπήγε προς τον Αγχούς, βασιλέα της Γαθ
\par 11 Και είπον οι δούλοι του Αγχούς προς αυτόν, Δεν είναι ούτος ο Δαβίδ ο βασιλεύς του τόπου; δεν είναι ούτος, εις τον οποίον αμοιβαίως έψαλλον εν τοις χοροίς, λέγουσαι, Ο Σαούλ επάταξε τας χιλιάδας αυτού, και ο Δαβίδ τας μυριάδας αυτού;
\par 12 Και έβαλεν ο Δαβίδ τους λόγους τούτους εν τη καρδία αυτού και εφοβήθη σφόδρα από του Αγχούς βασιλέως της Γαθ.
\par 13 Και ήλλαξε τον τρόπον αυτού έμπροσθεν αυτών, και προσεποιήθη τον τρελλόν μεταξύ των χειρών αυτών, και έξυεν επάνω των θυρών της πύλης, και άφινε τον σίελον αυτού να καταπίπτη εις το γένειον αυτού.
\par 14 Τότε είπεν ο Αγχούς προς τους δούλους αυτού, Ιδού, σεις βλέπετε τον άνθρωπον ότι είναι τρελλός· διά τι εφέρετε αυτόν προς εμέ;
\par 15 μήπως εγώ στερούμαι τρελλών, ώστε να φέρητε τούτον διά να κάμνη τον τρελλόν έμπροσθέν μου; ούτος ήθελεν εισέλθει εις την οικίαν μου;

\chapter{22}

\par 1 Ανεχώρησε δε ο Δαβίδ εκείθεν και διεσώθη εις το σπήλαιον Οδολλάμ· και ότε ήκουσαν οι αδελφοί αυτού και πας ο οίκος του πατρός αυτού, κατέβησαν εκεί προς αυτόν.
\par 2 Και συνηθροίσθησαν προς αυτόν, πας όστις ήτο εν στενοχωρία και πας χρεωφειλέτης και πας δυσηρεστημένος· και έγεινεν αρχηγός επ' αυτών· και ήσαν μετ' αυτού έως τετρακόσιοι άνδρες.
\par 3 Και ανεχώρησεν ο Δαβίδ εκείθεν εις Μισπά της Μωάβ· και είπε προς τον βασιλέα Μωάβ, Ας έλθωσι, παρακαλώ, ο πατήρ μου και η μήτηρ μου προς εσάς, εωσού γνωρίσω τι θέλει κάμει ο Θεός εις εμέ.
\par 4 Και έφερεν αυτούς ενώπιον του βασιλέως Μωάβ και κατώκησαν μετ' αυτού όλον τον καιρόν καθ' ον ο Δαβίδ ήτο εν τω οχυρώματι.
\par 5 Είπε δε Γαδ ο προφήτης προς τον Δαβίδ, Μη μένης εν τω οχυρώματι· αναχώρησον και είσελθε εις την γην Ιούδα. Τότε ανεχώρησεν ο Δαβίδ και εισήλθεν εις το δάσος Αρέθ.
\par 6 Ακούσας δε ο Σαούλ ότι εφανερώθη ο Δαβίδ και οι άνδρες οι μετ' αυτού εκάθητο δε ο Σαούλ εν Γαβαά υπό το δένδρον εν Ραμά, έχων το δόρυ αυτού εν τη χειρί αυτού, και πάντες οι δούλοι αυτού ίσταντο ενώπιον αυτού,
\par 7 τότε είπεν ο Σαούλ προς τους δούλους αυτού τους παρεστώτας ενώπιον αυτού, Ακούσατε τώρα, Βενιαμίται· μήπως εις όλους σας θέλει δώσει ο υιός του Ιεσσαί αγρούς και αμπελώνας, και όλους σας θέλει κάμει χιλιάρχους και εκατοντάρχους,
\par 8 ώστε σεις να συνομόσητε πάντες εναντίον μου και να μη ήναι μηδείς όστις να απαγγείλη εις εμέ ότι ο υιός μου έκαμε συνθήκην μετά του υιού του Ιεσσαί, και μηδείς από σας να μη ήναι όστις να πονή δι' εμέ ή να απαγγείλη εις εμέ ότι ο υιός μου διήγειρε τον δούλον μου εναντίον μου, διά να ενεδρεύη καθώς την σήμερον;
\par 9 Και απεκρίθη Δωήκ ο Ιδουμαίος, όστις ήτο διωρισμένος επί τους δούλους του Σαούλ, και είπεν, Είδον τον υιόν του Ιεσσαί ελθόντα εις Νωβ, προς Αχιμέλεχ τον υιόν του Αχιτώβ·
\par 10 όστις ηρώτησε περί αυτού τον Κύριον, και τροφάς έδωκεν εις αυτόν, και την ρομφαίαν Γολιάθ του Φιλισταίου έδωκεν εις αυτόν.
\par 11 Τότε απέστειλεν ο βασιλεύς να καλέσωσιν Αχιμέλεχ τον υιόν του Αχιτώβ, τον ιερέα, και πάντα τον οίκον του πατρός αυτού, τους ιερείς τους εν Νώβ· και ήλθον πάντες προς τον βασιλέα.
\par 12 Και είπεν ο Σαούλ, Άκουσον τώρα, υιέ του Αχιτώβ. Ο δε απεκρίθη, Ιδού εγώ, κύριέ μου.
\par 13 Και είπε προς αυτόν ο Σαούλ, Διά τι συνωμόσατε εναντίον μου, συ και ο υιός του Ιεσσαί, ώστε να δώσης εις αυτόν άρτον και ρομφαίαν και να ερωτήσης τον Θεόν περί αυτού, ώστε να σηκωθή εναντίον μου, να ενεδρεύη, καθώς την σήμερον;
\par 14 Και απεκρίθη ο Αχιμέλεχ προς τον βασιλέα και είπε, Και τις μεταξύ πάντων των δούλων σου είναι καθώς ο Δαβίδ πιστός, και γαμβρός του βασιλέως και πορευόμενος εις το πρόσταγμά σου και τιμώμενος εν τω οίκω σου;
\par 15 σήμερον ήρχισα να ερωτώ τον Θεόν περί αυτού; μη γένοιτο· ας μη αναθέση ο βασιλεύς μηδέν επί τον δούλον αυτού μηδέ επί πάντα τον οίκον του πατρός μου· διότι ο δούλός σου δεν εξεύρει ουδέν περί πάντων τούτων, ούτε μικρόν ούτε μέγα.
\par 16 Και είπεν ο βασιλεύς, Εξάπαντος θέλεις αποθάνει, Αχιμέλεχ, συ και πας ο οίκος του πατρός σου.
\par 17 Και είπεν ο βασιλεύς προς τους δορυφόρους τους περιεστώτας εις αυτόν, Στρέψατε και θανατώσατε τους ιερείς του Κυρίου· επειδή έχουσι και αυτοί την χείρα αυτών μετά του Δαβίδ, και επειδή εγνώρισαν ότι αυτός έφευγε και δεν μοι απήγγειλαν τούτο. Δεν ηθέλησαν όμως οι δούλοι του βασιλέως να εκτείνωσι τας χείρας αυτών διά να πέσωσιν επί τους ιερείς του Κυρίου.
\par 18 Και είπεν ο βασιλεύς προς τον Δωήκ, Στρέψον συ και πέσον επί τους ιερείς. Και έστρεψε Δωήκ ο Ιδουμαίος και έπεσεν επί τους ιερείς, και εθανάτωσεν εκείνην την ημέραν ογδοήκοντα πέντε άνδρας φορούντας λινούν εφόδ.
\par 19 Και την Νωβ, την πόλιν των ιερέων, επάταξεν εν στόματι μαχαίρας, άνδρας και γυναίκας, παιδία και βρέφη θηλάζοντα, και βόας και όνους και πρόβατα, εν στόματι μαχαίρας.
\par 20 Διεσώθη δε εις εκ των υιών του Αχιμέλεχ υιού του Αχιτώβ, ονόματι Αβιάθαρ, και έφυγε κατόπιν του Δαβίδ.
\par 21 Και απήγγειλεν ο Αβιάθαρ προς τον Δαβίδ, ότι εθανάτωσεν ο Σαούλ τους ιερείς του Κυρίου.
\par 22 Και είπεν ο Δαβίδ προς τον Αβιάθαρ, Ήξευρον εν εκείνη τη ημέρα, καθ' ην Δωήκ ο Ιδουμαίος ήτο εκεί, ότι ήθελε βεβαίως απαγγείλει προς τον Σαούλ· εγώ εστάθην αιτία του θανάτου πάντων των ανθρώπων του οίκου του πατρός σου·
\par 23 κάθου μετ' εμού, μη φοβού· διότι ο ζητών την ζωήν μου ζητεί και την ζωήν σου· πλην συ θέλεις είσθαι μετ' εμού εν ασφαλεία.

\chapter{23}

\par 1 Απήγγειλαν δε προς τον Δαβίδ, λέγοντες, Ιδού, οι Φιλισταίοι πολεμούσιν εν Κεειλά και διαρπάζουσι τα αλώνια.
\par 2 Και ηρώτησεν ο Δαβίδ τον Κύριον, λέγων, Να υπάγω και να πατάξω τους Φιλισταίους τούτους; Και είπεν ο Κύριος προς τον Δαβίδ, Ύπαγε και πάταξον τους Φιλισταίους και σώσον την Κεειλά.
\par 3 Και είπον οι άνδρες του Δαβίδ προς αυτόν, Ιδού, ημείς ενταύθα εν τη Ιουδαία φοβούμεθα· πόσω δε μάλλον, εάν υπάγωμεν εις Κεειλά εναντίον των στρατευμάτων των Φιλισταίων;
\par 4 Και ηρώτησε πάλιν ο Δαβίδ εκ δευτέρου τον Κύριον. Και απεκρίθη προς αυτόν ο Κύριος και είπε, Σηκώθητι, κατάβα εις Κεειλά· διότι θέλω παραδώσει τους Φιλισταίους εις την χείρα σου.
\par 5 Τότε ήλθεν ο Δαβίδ και οι άνδρες αυτού εις Κεειλά, και επολέμησε προς τους Φιλισταίους και έλαβε τα κτήνη αυτών και επάταξεν αυτούς εν σφαγή μεγάλη. Και έσωσεν ο Δαβίδ τους κατοίκους της Κεειλά.
\par 6 Ότε δε Αβιάθαρ ο υιός του Αχιμέλεχ έφυγε προς τον Δαβίδ εις Κεειλά, αυτός είχε καταβή με εφόδ εν τη χειρί αυτού.
\par 7 Και απηγγέλθη προς τον Σαούλ ότι ήλθεν ο Δαβίδ εις Κεειλά. Και είπεν ο Σαούλ, Ο Θεός παρέδωκεν αυτόν εις την χείρα μου· διότι απεκλείσθη, εισελθών εις πόλιν έχουσαν πύλας και μοχλούς.
\par 8 Και συνεκάλεσεν ο Σαούλ πάντα τον λαόν εις πόλεμον, διά να καταβή εις Κεειλά, να πολιορκήση τον Δαβίδ και τους άνδρας αυτού.
\par 9 Και έμαθεν ο Δαβίδ ότι ο Σαούλ εμηχανεύετο κακόν εναντίον αυτού· και είπε προς τον Αβιάθαρ τον ιερέα, Φέρε ενταύθα το εφόδ.
\par 10 Και είπεν ο Δαβίδ, Κύριε Θεέ του Ισραήλ, μετά βεβαιότητος ήκουσεν ο δούλός σου ότι ο Σαούλ ζητεί να έλθη εις Κεειλά, διά να εξολοθρεύση την πόλιν εξ αιτίας μου·
\par 11 θέλουσι με παραδώσει εις αυτόν οι άνδρες της Κεειλά; θέλει καταβή ο Σαούλ, καθώς ήκουσεν ο δούλός σου; Κύριε Θεέ του Ισραήλ, φανέρωσον, δέομαι, προς τον δούλον σου. Και είπεν ο Κύριος, Θέλει καταβή.
\par 12 Είπε πάλιν ο Δαβίδ, Θέλουσι παραδώσει οι άνδρες της Κεειλά εμέ και τους άνδρας μου εις την χείρα του Σαούλ; Και είπεν ο Κύριος, Θέλουσι παραδώσει.
\par 13 Τότε ο Δαβίδ και οι άνδρες αυτού, έως εξακόσιοι, εσηκώθησαν και εξήλθον από Κεειλά και υπήγον όπου ηδύναντο. Και απηγγέλθη προς τον Σαούλ, ότι διεσώθη ο Δαβίδ από Κεειλά· όθεν απέσχε του να εξέλθη.
\par 14 Ο δε Δαβίδ εκάθησεν εν τη ερήμω, εν τόποις οχυροίς, και έμενεν επί τινός όρους εν τη ερήμω Ζιφ. Και αυτόν εζήτει ο Σαούλ πάσας τας ημέρας· ο Θεός όμως δεν παρέδωκεν αυτόν εις την χείρα αυτού.
\par 15 Και είδεν ο Δαβίδ ότι εξήλθεν ο Σαούλ διά να ζητή την ζωήν αυτού και ήτο ο Δαβίδ εν τη ερήμω Ζιφ, εντός του δάσους.
\par 16 Τότε εσηκώθη Ιωνάθαν, ο υιός του Σαούλ, και υπήγε προς τον Δαβίδ εις το δάσος, και ενίσχυσε την χείρα αυτού εν τω Θεώ.
\par 17 Και είπε προς αυτόν, Μη φοβού· διότι δεν θέλει σε ευρεί η χειρ Σαούλ του πατρός μου· και συ θέλεις βασιλεύσει επί τον Ισραήλ, και εγώ θέλω είσθαι δεύτερός σου· μάλιστα και Σαούλ ο πατήρ μου εξεύρει τούτο.
\par 18 Και έκαμον αμφότεροι συνθήκην ενώπιον του Κυρίου· και εκάθητο ο Δαβίδ εντός του δάσους, ο δε Ιωνάθαν ανεχώρησεν εις τον οίκον αυτού.
\par 19 Ανέβησαν δε οι Ζιφαίοι προς τον Σαούλ εις Γαβαά, λέγοντες, Δεν είναι κεκρυμμένος ο Δαβίδ εις ημάς εν οχυρώμασι εντός του δάσους, επί του βουνού Εχελά, του προς τα δεξιά Γεσιμών;
\par 20 τώρα λοιπόν, βασιλεύ, κατάβα, καθ' όλην την επιθυμίαν της ψυχής σου εις το να καταβής· και ημών έργον θέλει είσθαι να παραδώσωμεν αυτόν εις την χείρα του βασιλέως.
\par 21 Και είπεν ο Σαούλ, Ευλογημένοι σεις παρά Κυρίου, διότι ελάβετε συμπάθειαν προς εμέ·
\par 22 υπάγετε λοιπόν, βεβαιώθητε ακριβέστερα και μάθετε και ιδέτε τον τόπον αυτού, που κρύπτεται, τις είδεν αυτόν εκεί· διότι μοι είπον ότι μηχανεύεται πανουργίας·
\par 23 ιδέτε λοιπόν και μάθετε εν τίνι εκ πάντων των αποκρύφων τόπων είναι κεκρυμμένος, και επιστρέψατε προς εμέ αφού βεβαιωθήτε· και θέλω υπάγει με σάς· και εάν ήναι εν τη γη ταύτη, βεβαίως θέλω εξιχνιάσει αυτόν μεταξύ πασών των χιλιάδων του Ιούδα.
\par 24 Και εσηκώθησαν και υπήγον εις Ζιφ προ του Σαούλ· ο Δαβίδ όμως και οι άνδρες αυτού ήσαν εν τη ερήμω Μαών, εν τη πεδιάδι κατά τα δεξιά του Γεσιμών.
\par 25 Υπήγε δε ο Σαούλ και οι άνδρες αυτού να ζητήσωσιν αυτόν. Και απηγγέλθη τούτο προς τον Δαβίδ· όθεν κατέβη εις την πέτραν και εκάθητο εν τη ερήμω Μαών. Και ακούσας ο Σαούλ, έτρεξε κατόπιν του Δαβίδ εις την έρημον Μαών.
\par 26 Και ο μεν Σαούλ επορεύετο κατά τούτο το μέρος του όρους, ο δε Δαβίδ και οι άνδρες αυτού κατ' εκείνο το μέρος του όρους· και έσπευσεν ο Δαβίδ να φύγη από προσώπου του Σαούλ· πλην ο Σαούλ και οι άνδρες αυτού περιεκύκλωσαν τον Δαβίδ και τους άνδρας αυτού, διά να συλλάβωσιν αυτούς.
\par 27 Ήλθε δε μηνυτής προς τον Σαούλ, λέγων, Σπεύσον και ελθέ, διότι οι Φιλισταίοι εφώρμησαν εις την γην.
\par 28 Όθεν επέστρεψεν ο Σαούλ από του να διώκη κατόπιν του Δαβίδ, και υπήγεν εις συνάντησιν των Φιλισταίων· διά τούτο ωνόμασαν εκείνον τον τόπον, Σελά-αμμαλεκώθ.
\par 29 Ανέβη δε ο Δαβίδ εκείθεν και εκάθησεν εν οχυροίς τόποις της Εν-γαδδί.

\chapter{24}

\par 1 Και αφού επέστρεψεν ο Σαούλ από όπισθεν των Φιλισταίων, ανήγγειλαν προς αυτόν, λέγοντες, Ιδού, ο Δαβίδ είναι εν τη ερήμω Εν-γαδδί.
\par 2 Τότε έλαβεν ο Σαούλ τρεις χιλιάδας ανδρών, εκλεκτών από παντός του Ισραήλ, και υπήγε να ζητή τον Δαβίδ και τους άνδρας αυτού επί τους βράχους των αγρίων αιγών.
\par 3 Και ήλθεν εις τας μάνδρας των προβάτων επί της οδού, όπου ήτο σπήλαιον· και εισήλθεν ο Σαούλ διά να σκεπάση τους πόδας αυτού· ο δε Δαβίδ και οι άνδρες αυτού εκάθηντο εις το ενδότερον του σπηλαίου.
\par 4 Και είπον οι άνδρες του Δαβίδ προς αυτόν, Ιδού, η ημέρα περί της οποίας ο Κύριος ελάλησε προς σε, λέγων, Ιδού, εγώ θέλω παραδώσει τον εχθρόν σου εις την χείρα σου, και θέλεις κάμει εις αυτόν όπως σοι φανή καλόν. Τότε εσηκώθη ο Δαβίδ και απέκοψε κρυφίως το κράσπεδον του επενδύματος του Σαούλ.
\par 5 Και μετά ταύτα η καρδία του Δαβίδ εκτύπησεν αυτόν, επειδή είχεν αποκόψει το κράσπεδον του Σαούλ.
\par 6 Και είπε προς τους άνδρας αυτού, Μη γένοιτο εις εμέ παρά Κυρίου να κάμω το πράγμα τούτο εις τον κύριόν μου, τον κεχρισμένον του Κυρίου, να επιβάλω την χείρα μου επ' αυτόν· διότι είναι κεχρισμένος του Κυρίου.
\par 7 Και εμπόδισεν ο Δαβίδ τους άνδρας αυτού διά των λόγων τούτων και δεν αφήκεν αυτούς να σηκωθώσι κατά του Σαούλ. Σηκωθείς δε ο Σαούλ εκ του σπηλαίου, υπήγεν εις την οδόν αυτού.
\par 8 Και μετά ταύτα σηκωθείς ο Δαβίδ εξήλθεν εκ του σπηλαίου και εβόησεν όπισθεν του Σαούλ, λέγων, Κύριέ μου βασιλεύ. Και ότε έβλεψεν ο Σαούλ οπίσω αυτού, ο Δαβίδ έκυψε με το πρόσωπον εις την γην και προσεκύνησεν αυτόν.
\par 9 Και είπεν ο Δαβίδ προς τον Σαούλ, Διά τι ακούεις τους λόγους ανθρώπων λεγόντων, Ιδού, ο Δαβίδ ζητεί το κακόν σου;
\par 10 Ιδού, εν τη ημέρα ταύτη είδον οι οφθαλμοί σου τίνι τρόπω σε παρέδωκεν ο Κύριος εις την χείρα μου σήμερον, εν τω σπηλαίω· και είπον τινές να σε θανατώσω· πλην ο οφθαλμός μου σε εφείσθη· και είπα, Δεν θέλω επιβάλει την χείρα μου κατά του κυρίου μου· διότι είναι κεχρισμένος του Κυρίου.
\par 11 Ιδέ προσέτι, πάτερ μου, ιδέ μάλιστα το κράσπεδον του επενδύματός σου εν τη χειρί μου· επειδή, εκ του ότι απέκοψα το κράσπεδον του επενδύματός σου και δεν σε εθανάτωσα, γνώρισον και ιδέ ότι δεν είναι κακία ουδέ παράβασις εν τη χειρί μου και δεν ημάρτησα εναντίον σου· συ όμως θηρεύεις την ζωήν μου διά να αφαιρέσης αυτήν.
\par 12 Ας κρίνη ο Κύριος μεταξύ εμού και σου, και ας με εκδικήση ο Κύριος από σού· η χειρ μου όμως δεν θέλει είσθαι επί σέ·
\par 13 καθώς λέγει η παροιμία των αρχαίων, Εξ ανόμων εξέρχεται ανομία· όθεν η χειρ μου δεν θέλει είσθαι επί σε.
\par 14 Οπίσω τίνος εξήλθεν ο βασιλεύς του Ισραήλ; οπίσω τίνος τρέχεις συ; οπίσω κυνός νενεκρωμένου, οπίσω ενός ψύλλου.
\par 15 Ο Κύριος λοιπόν ας ήναι δικαστής και ας κρίνη μεταξύ εμού και σού· και ας ίδη, και ας δικάση την δίκην μου και ας με ελευθερώση εκ της χειρός σου.
\par 16 Και αφού ετελείωσεν ο Δαβίδ λαλών τους λόγους τούτους προς τον Σαούλ, είπεν ο Σαούλ, Η φωνή σου είναι αύτη, τέκνον μου Δαβίδ; Και ύψωσεν ο Σαούλ την φωνήν αυτού και έκλαυσε.
\par 17 Και είπε προς τον Δαβίδ, Συ είσαι δικαιότερος εμού· διότι συ ανταπέδωκας εις εμέ καλόν, εγώ δε ανταπέδωκα εις σε κακόν.
\par 18 Και συ έδειξας σήμερον με πόσην αγαθότητα εφέρθης προς εμέ· διότι ενώ με απέκλεισεν ο Κύριος εις τας χείρας σου, συ δεν με εθανάτωσας.
\par 19 Και τις, ευρών τον εχθρόν αυτού, ήθελεν αφήσει αυτόν να υπάγη την οδόν αυτού αβλαβώς; ο Κύριος λοιπόν να σοι ανταποδώση καλόν, δι' εκείνο το οποίον έκαμες εις εμέ σήμερον.
\par 20 Και τώρα, ιδού, γνωρίζω ότι βεβαίως θέλεις βασιλεύσει, και η βασιλεία του Ισραήλ θέλει στερεωθή εν τη χειρί σου.
\par 21 Τώρα λοιπόν όμοσόν μοι εις τον Κύριον, ότι δεν θέλεις εξολοθρεύσει το σπέρμα μου μετ' εμέ, και έτι δεν θέλεις αφανίσει το όνομά μου εκ του οίκου του πατρός μου.
\par 22 Και ώμοσεν ο Δαβίδ προς τον Σαούλ. Και ανεχώρησεν ο Σαούλ εις τον οίκον αυτού· ο δε Δαβίδ και οι άνδρες αυτού ανέβησαν εις το οχύρωμα.

\chapter{25}

\par 1 Απέθανε δε ο Σαμουήλ· και συνήχθησαν πας ο Ισραήλ και έκλαυσαν αυτόν, και ενεταφίασαν αυτόν εν τω οίκω αυτού εν Ραμά. Και εσηκώθη ο Δαβίδ και κατέβη εις την έρημον Φαράν.
\par 2 Ήτο δε άνθρωπός τις εν Μαών, του οποίου τα κτήματα ήσαν εν τω Καρμήλω, και ο άνθρωπος ήτο μέγας σφόδρα και είχε τρισχίλια πρόβατα και χιλίας αίγας· και εκούρευε τα πρόβατα αυτού εν τω Καρμήλω.
\par 3 Το δε όνομα του ανθρώπου ήτο Νάβαλ· και το όνομα της γυναικός αυτού Αβιγαία· και η μεν γυνή ήτο καλή εις την σύνεσιν και ώραία την όψιν· ο άνθρωπος όμως σκληρός, και κακός εις τας πράξεις αυτού· ήτο δε εκ της γενεάς του Χάλεβ.
\par 4 Και ήκουσεν ο Δαβίδ εν τη ερήμω, ότι ο Νάβαλ εκούρευε τα πρόβατα αυτού.
\par 5 Και απέστειλεν ο Δαβίδ δέκα νέους, και είπεν ο Δαβίδ προς τους νέους, Ανάβητε εις τον Κάρμηλον και υπάγετε προς τον Νάβαλ και χαιρετήσατε αυτόν εξ ονόματός μου.
\par 6 και θέλετε ειπεί, Να ήσαι πολυχρόνιος· ειρήνη και εις σε, ειρήνη και εις τον οίκόν σου, ειρήνη και εις πάντα όσα έχεις·
\par 7 και τώρα ήκουσα ότι έχεις κουρευτάς· ιδού, τους ποιμένας σου, οίτινες ήσαν μεθ' ημών, δεν εβλάψαμεν αυτούς, ουδέ εχάθη τι εις αυτούς, καθ' όλον τον καιρόν καθ' ον ήσαν εν τω Καρμήλω·
\par 8 ερώτησον τους νέους σου, και θέλουσι σοι ειπεί· ας εύρωσι λοιπόν οι νέοι ούτοι χάριν εις τους οφθαλμούς σου· διότι εις ημέραν καλήν ήλθομεν· δος, παρακαλούμεν, ό,τι έλθη εις την χείρα σου προς τους δούλους σου και προς τον υιόν σου τον Δαβίδ.
\par 9 Και ελθόντες οι νέοι του Δαβίδ ελάλησαν προς τον Νάβαλ κατά πάντας τους λόγους τούτους εν ονόματι του Δαβίδ, και έπαυσαν.
\par 10 Αλλ' ο Νάβαλ απεκρίθη προς τους δούλους του Δαβίδ και είπε, Τις είναι ο Δαβίδ; και τις ο υιός του Ιεσσαί; πολλοί είναι την σήμερον οι δούλοι, οίτινες αποσκιρτώσιν έκαστος από του κυρίου αυτού·
\par 11 θέλω λάβει λοιπόν τον άρτον μου και το ύδωρ μου και το σφακτόν μου, το οποίον έσφαξα διά τους κουρευτάς μου, και δώσει εις ανθρώπους τους οποίους δεν γνωρίζω πόθεν είναι;
\par 12 Και εστράφησαν οι νέοι του Δαβίδ εις την οδόν αυτών και ανεχώρησαν και ελθόντες απήγγειλαν προς αυτόν πάντας τους λόγους τούτους.
\par 13 Και είπεν ο Δαβίδ προς τους άνδρας αυτού, Ζώσθητε έκαστος την ρομφαίαν αυτού. Και εζώσθησαν έκαστος την ρομφαίαν αυτού· και ο Δαβίδ ομοίως εζώσθη την ρομφαίαν αυτού· και ανέβησαν κατόπιν του Δαβίδ έως τετρακόσιοι άνδρες· διακόσιοι δε έμειναν πλησίον της αποσκευής.
\par 14 Εις δε εκ των νέων απήγγειλε προς την Αβιγαίαν, την γυναίκα του Νάβαλ, λέγων, Ιδού, ο Δαβίδ απέστειλε μηνυτάς εκ της ερήμου διά να χαιρετήση τον κύριον ημών, και εκείνος απεδίωξεν αυτούς·
\par 15 οι άνδρες όμως εστάθησαν πολύ καλοί προς ημάς και δεν εβλάφθημεν ουδέ εχάσαμεν ουδέν, όσον καιρόν συνανεστράφημεν μετ' αυτών, ότε ήμεθα εν τοις αγροίς·
\par 16 ήσαν ως τείχος πέριξ ημών και νύκτα και ημέραν, καθ' όλον τον καιρόν καθ' ον ήμεθα μετ' αυτών βόσκοντες τα πρόβατα·
\par 17 τώρα λοιπόν, γνώρισον και ιδέ τι θέλεις κάμει σύ· διότι κακόν απεφασίσθη κατά του κυρίου ημών, και κατά παντός του οίκου αυτού· επειδή είναι άνθρωπος δύστροπος, ώστε ουδείς δύναται να ομιλήση προς αυτόν.
\par 18 Τότε έσπευσεν η Αβιγαία, και έλαβε διακοσίους άρτους, και δύο αγγεία οίνου, και πέντε πρόβατα ητοιμασμένα, και πέντε μέτρα σίτου πεφρυγανισμένου, και εκατόν δέσμας σταφίδος, και διακοσίας πήττας σύκων, και έθεσεν αυτά επί όνων.
\par 19 Και είπε προς τους νέους αυτής, Προπορεύεσθε έμπροσθέν μου· ιδού, εγώ έρχομαι κατόπιν σας· προς τον Νάβαλ όμως τον άνδρα αυτής δεν εφανέρωσε τούτο.
\par 20 Και καθώς αυτή, καθημένη επί του όνου, κατέβαινεν υπό την σκέπην του όρους, ιδού, ο Δαβίδ και οι άνδρες αυτού κατέβαινον προς αυτήν· και συνήντησεν αυτούς.
\par 21 είχε δε ειπεί ο Δαβίδ, Ματαίως τωόντι εφύλαξα πάντα όσα είχεν ούτος εν τη ερήμω, και δεν εχάθη ουδέν εκ πάντων των κτημάτων αυτού· και ανταπέδωκεν εις εμέ κακόν αντί καλού·
\par 22 ούτω να κάμη ο Θεός εις τους εχθρούς του Δαβίδ και ούτω να προσθέση, εάν έως το πρωΐ αφήσω εκ πάντων των πραγμάτων αυτού ουρούντα εις τοίχον.
\par 23 Και καθώς είδεν η Αβιγαία τον Δαβίδ, έσπευσε και κατέβη από του όνου και έπεσεν ενώπιον του Δαβίδ κατά πρόσωπον και προσεκύνησεν έως εδάφους.
\par 24 Και προσέπεσεν εις τους πόδας αυτού και είπεν, Επ' εμέ, επ' εμέ, κύριέ μου, ας ήναι αύτη η αδικία· και ας λαλήση, παρακαλώ, η δούλη σου εις τα ώτα σου, και άκουσον τους λόγους της δούλης σου.
\par 25 Ας μη δώση ο κύριός μου, παρακαλώ, ουδεμίαν προσοχήν εις τούτον τον δύστροπον άνθρωπον, τον Νάβαλ· διότι κατά το όνομα αυτού, τοιούτος είναι· Νάβαλ το όνομα αυτού, και αφροσύνη μετ' αυτού· εγώ δε η δούλη σου δεν είδον τους νέους του κυρίου μου, τους οποίους απέστειλας.
\par 26 Τώρα λοιπόν, κύριε μου, ζη Κύριος και ζη η ψυχή σου, ο Κύριος βεβαίως σε εκράτησεν από του να εμβής εις αίμα και να εκδικηθής διά της χειρός σου· τώρα δε οι εχθροί σου και οι ζητούντες κακόν εις τον κύριόν μου, ας ήναι ως ο Νάβαλ.
\par 27 Και τώρα αύτη η προσφορά, την οποίαν η δούλη σου έφερε προς τον κύριόν μου, ας δοθή εις τους νέους τους ακολουθούντας τον κύριόν μου.
\par 28 Συγχώρησον, παρακαλώ, το αμάρτημα της δούλης σου· διότι ο Κύριος θέλει βεβαίως κάμει εις τον κύριόν μου οίκον ασφαλή, επειδή μάχεται ο κύριός μου τας μάχας του Κυρίου, και κακία δεν ευρέθη εν σοι πώποτε.
\par 29 Αν και εσηκώθη άνθρωπος καταδιώκων σε και ζητών την ψυχήν σου, η ψυχή όμως του κυρίου μου θέλει είσθαι δεδεμένη εις τον δεσμόν της ζωής πλησίον Κυρίου του Θεού σου· τας δε ψυχάς των εχθρών σου, ταύτας θέλει εκσφενδονίσει εκ μέσου της σφενδόνης.
\par 30 Και όταν κάμη ο Κύριος εις τον κύριόν μου κατά πάντα τα αγαθά τα οποία ελάλησε περί σου, και σε καταστήση κυβερνήτην επί τον Ισραήλ,
\par 31 δεν θέλει είσθαι τούτο σκάνδαλον εις σε ουδέ πρόσκομμα καρδίας εις τον κύριόν μου, ή ότι έχυσας αίμα αναίτιον, ή ότι ο κύριός μου εξεδίκησεν αυτός εαυτόν· πλην όταν ο Κύριος αγαθοποιήση τον κύριόν μου, τότε ενθυμήθητι την δούλην σου.
\par 32 Και είπεν ο Δαβίδ προς την Αβιγαίαν, Ευλογητός Κύριος ο Θεός του Ισραήλ, όστις σε απέστειλε την ημέραν ταύτην εις συνάντησίν μου·
\par 33 και ευλογημένη η βουλή σου και ευλογημένη συ, ήτις με εφύλαξας την ημέραν ταύτην από του να εμβώ εις αίματα και να εκδικηθώ διά της χειρός μου·
\par 34 διότι αληθώς, ζη Κύριος ο Θεός του Ισραήλ, όστις με εμπόδισεν από του να σε κακοποιήσω, εάν δεν ήθελες σπεύσει να έλθης εις συνάντησίν μου, δεν ήθελε μείνει εις τον Νάβαλ έως της αυγής ουρών εις τοίχον.
\par 35 Και έλαβεν ο Δαβίδ εκ της χειρός αυτής τα όσα έφερε προς αυτόν· και είπε προς αυτήν, Ανάβα προς τον οίκόν σου εν ειρήνη· βλέπε, εισήκουσα της φωνής σου και ετίμησα το πρόσωπόν σου.
\par 36 Και ήλθεν η Αβιγαία προς τον Νάβαλ· και ιδού, είχε συμπόσιον εν τω οίκω αυτού, ως συμπόσιον βασιλέως· και η καρδία του Νάβαλ ήτο εύθυμος εν αυτώ, και ήτο εις άκρον μεθυσμένος· όθεν δεν απήγγειλε προς αυτόν ουδέν, μικρόν μέγα, έως της αυγής.
\par 37 Το πρωΐ όμως, αφού ο Νάβαλ εξεμέθυσεν, εφανέρωσε προς αυτόν η γυνή αυτού τα πράγματα ταύτα· και ενεκρώθη η καρδία αυτού εντός αυτού και έγεινεν ως λίθος.
\par 38 Και μετά δέκα ημέρας περίπου επάταξεν ο Κύριος τον Νάβαλ, και απέθανε.
\par 39 Και ότε ήκουσεν ο Δαβίδ ότι απέθανεν ο Νάβαλ, είπεν, Ευλογητός Κύριος, όστις έκρινε την κρίσιν μο περί του ονειδισμού μου του γενομένου παρά του Νάβαλ, και ημπόδισε τον δούλον αυτού από κακού· και την κακίαν του Νάβαλ έστρεψεν ο Κύριος κατά της κεφαλής αυτού. Και απέστειλεν ο Δαβίδ και ελάλησε προς την Αβιγαίαν, διά να λάβη αυτήν γυναίκα εις εαυτόν.
\par 40 Και ελθόντες οι δούλοι του Δαβίδ προς την Αβιγαίαν εις τον Κάρμηλον, ελάλησαν προς αυτήν, λέγοντες, Ο Δαβίδ απέστειλεν ημάς προς σε, διά να σε λάβη γυναίκα εις εαυτόν.
\par 41 Και εσηκώθη και προσεκύνησε κατά πρόσωπον έως εδάφους και είπεν, Ιδού, ας ήναι η δούλη σου θεράπαινα διά να πλύνη τους πόδας των δούλων του κυρίου μου.
\par 42 Και έσπευσεν η Αβιγαία και εσηκώθη και ανέβη επί του όνου, μετά πέντε κορασίων αυτής ακολουθούντων οπίσω αυτής· και υπήγε κατόπιν των απεσταλμένων του Δαβίδ και έγεινε γυνή αυτού.
\par 43 Έλαβεν ο Δαβίδ και την Αχινοάμ από Ιεζραέλ· και ήσαν αμφότεραι γυναίκες αυτού.
\par 44 Ο δε Σαούλ είχε δώσει Μιχάλ, την θυγατέρα αυτού, την γυναίκα του Δαβίδ, εις τον Φαλτί τον υιόν του Λαείς, τον από Γαλλείμ.

\chapter{26}

\par 1 Ήλθον δε οι Ζιφαίοι προς τον Σαούλ εις Γαβαά, λέγοντες, Δεν κρύπτεται ο Δαβίδ εν τω βουνώ Εχελά απέναντι Γεσιμών;
\par 2 Και εσηκώθη ο Σαούλ και κατέβη εις την έρημον Ζιφ, έχων μεθ' εαυτού τρεις χιλιάδας ανδρών εκλεκτών εκ του Ισραήλ, διά να ζητή τον Δαβίδ εν τη ερήμω Ζιφ.
\par 3 Και εστρατοπέδευσεν ο Σαούλ επί του βουνού Εχελά, του απέναντι Γεσιμών, πλησίον της οδού. Ο δε Δαβίδ εκάθητο εν τη ερήμω και είδεν ότι ο Σαούλ ήρχετο κατόπιν αυτού εις την έρημον.
\par 4 Όθεν απέστειλεν ο Δαβίδ κατασκόπους και έμαθεν ότι ο Σαούλ ήλθε τωόντι.
\par 5 Και σηκωθείς ο Δαβίδ ήλθεν εις τον τόπον όπου ο Σαούλ είχε στρατοπεδεύσει· και παρετήρησεν ο Δαβίδ τον τόπον όπου εκοιμάτο ο Σαούλ, και Αβενήρ ο υιός του Νηρ, ο αρχιστράτηγος αυτού· εκοιμάτο δε ο Σαούλ εντός του περιβόλου, και ο λαός ήτο εστρατοπεδευμένος κύκλω αυτού.
\par 6 Τότε ελάλησεν ο Δαβίδ και είπε προς τον Αχιμέλεχ τον Χετταίον και προς τον Αβισαί τον υιόν της Σερουΐας, αδελφόν του Ιωάβ, λέγων, Τις θέλει καταβή μετ' εμού προς τον Σαούλ εις το στρατόπεδον; Και είπεν ο Αβισαί, Εγώ θέλω καταβή μετά σου.
\par 7 Ήλθον λοιπόν ο Δαβίδ και ο Αβισαί διά νυκτός προς τον λαόν· και ιδού, ο Σαούλ έκειτο κοιμώμενος εντός του περιβόλου, και το δόρυ αυτού εμπεπηγμένον εις την γην προς την κεφαλήν αυτού· ο δε Αβενήρ και ο λαός εκοιμώντο κύκλω αυτού.
\par 8 Τότε είπεν ο Αβισαί προς τον Δαβίδ, Ο Θεός απέκλεισε σήμερον εις την χείρα σου τον εχθρόν σου· τώρα λοιπόν ας πατάξω αυτόν διά του δόρατος έως της γης διά μιάς· και δεν θέλω δευτερώσει επ' αυτόν.
\par 9 Αλλ' ο Δαβίδ είπε προς τον Αβισαί, Μη θανατώσης αυτόν· διότι τις επιβαλών την χείρα αυτού επί τον κεχρισμένον του Κυρίου θέλει είσθαι αθώος;
\par 10 Είπε μάλιστα ο Δαβίδ, Ζη Κύριος, ο Κύριος θέλει πατάξει αυτόν· ή η ημέρα αυτού θέλει ελθεί, και θέλει αποθάνει· θέλει καταβή εις πόλεμον και θανατωθή·
\par 11 μη γένοιτο εις εμέ παρά Κυρίου, να επιβάλω την χείρα μου επί τον κεχρισμένον του Κυρίου· λάβε όμως τώρα, παρακαλώ, το δόρυ το προς την κεφαλήν αυτού και το αγγείον του ύδατος, και ας αναχωρήσωμεν.
\par 12 Έλαβε λοιπόν ο Δαβίδ το δόρυ και το αγγείον του ύδατος από πλησίον της κεφαλής του Σαούλ· και ανεχώρησαν, και ουδείς είδε και ουδείς ενόησε και ουδείς εξύπνησε· διότι πάντες εκοιμώντο, επειδή βαθύς ύπνος παρά Κυρίου έπεσεν επ' αυτούς.
\par 13 Τότε διέβη ο Δαβίδ εις το πέραν και εστάθη επί της κορυφής του όρους μακρόθεν· ήτο δε πολύ απόστασις μεταξύ αυτών.
\par 14 Και εβόησεν ο Δαβίδ προς τον λαόν και προς τον Αβενήρ τον υιόν του Νηρ, λέγων, Δεν αποκρίνεσαι, Αβενήρ; Και απεκρίθη ο Αβενήρ και είπε, Τις είσαι συ, όστις βοάς προς τον βασιλέα;
\par 15 Και είπεν ο Δαβίδ προς τον Αβενήρ, Δεν είσαι ανήρ συ; και τις όμοιός σου μεταξύ του Ισραήλ; διά τι λοιπόν δεν φυλάττεις τον κύριόν σου τον βασιλέα; διότι εισήλθέ τις εκ του λαού διά να θανατώση τον βασιλέα τον κύριόν σου·
\par 16 δεν είναι καλόν το πράγμα τούτο, το οποίον έπραξας· ζη Κύριος, σεις είσθε άξιοι θανάτου, επειδή δε εφυλάξατε τον κύριόν σας, τον κεχρισμένον του Κυρίου. Και τώρα, ιδέτε που είναι το δόρυ του βασιλέως και το αγγείον του ύδατος· το προς την κεφαλήν αυτού.
\par 17 Και εγνώρισεν ο Σαούλ την φωνήν του Δαβίδ και είπεν, Η φωνή σου είναι, τέκνον μου Δαβίδ; Και ο Δαβίδ είπεν, Η φωνή μου, κύριέ μου βασιλεύ.
\par 18 Και είπε, Διά τι ο κύριός μου καταδιώκει ούτως οπίσω του δούλου αυτού; διότι τι έπραξα; ή τι κακόν είναι εν τη χειρί μου;
\par 19 τώρα λοιπόν ας ακούση, παρακαλώ, ο κύριός μου ο βασιλεύς τους λόγους του δούλου αυτού· εάν ο Κύριος σε διήγειρεν εναντίον μου, ας δεχθή θυσίαν· αλλ' εάν υιοί ανθρώπων, ούτοι ας ήναι επικατάρατοι ενώπιον του Κυρίου· διότι με εξέβαλον την σήμερον από του να κατοικώ εν τη κληρονομία του Κυρίου, λέγοντες, Ύπαγε, λάτρευε άλλους Θεούς·
\par 20 τώρα λοιπόν, ας μη πέση το αίμα μου εις την γην ενώπιον του Κυρίου· διότι εξήλθεν ο βασιλεύς του Ισραήλ να ζητήση ένα ψύλλον, ως όταν καταδιώκη τις πέρδικα εις τα όρη.
\par 21 Και είπεν ο Σαούλ, Ημάρτησα· επίστρεψον, τέκνον μου Δαβίδ· διότι δεν θέλω σε κακοποιήσει πλέον, επειδή η ψυχή μου εστάθη σήμερον πολύτιμος εις τους οφθαλμούς σου· ιδού, έπραξα αφρόνως και επλανήθην σφόδρα.
\par 22 Και απεκρίθη ο Δαβίδ και είπεν, Ιδού, το δόρυ του βασιλέως· και ας καταβή εις εκ των νέων και ας λάβη αυτό.
\par 23 ο δε Κύριος ας αποδώση εις έκαστον κατά την δικαιοσύνην αυτού και κατά την πίστιν αυτού· διότι σε παρέδωκεν ο Κύριος σήμερον εις την χείρα μου, πλην εγώ δεν ηθέλησα να επιβάλω την χείρα μου επί τον κεχρισμένον του Κυρίου.
\par 24 ιδού λοιπόν, καθώς η ζωή σου εστάθη σήμερον πολύτιμος εις τους οφθαλμούς μου, ούτως η ζωή μου ας σταθή πολύτιμος εις τους οφθαλμούς του Κυρίου, και ας με ελευθερώση εκ πασών των θλίψεων.
\par 25 Τότε είπεν ο Σαούλ προς τον Δαβίδ, Ευλογημένος να ήσαι, τέκνον μου Δαβίδ· βεβαίως θέλεις κατορθώσει μεγάλα και θέλεις βεβαίως υπερισχύσει. Και ο μεν Δαβίδ απήλθεν εις την οδόν αυτού, ο δε Σαούλ επέστρεψεν εις τον τόπον αυτού.

\chapter{27}

\par 1 Είπε δε ο Δαβίδ εν τη καρδία αυτού, Θέλω βεβαίως απολεσθή μίαν ημέραν διά χειρός του Σαούλ· δεν είναι τι καλήτερον δι' εμέ, παρά να διασωθώ ταχέως εις την γην των Φιλισταίων· τότε απ' εμού ο Σαούλ απελπισθείς, θέλει παραιτηθή από του να με ζητή πλέον εις πάντα τα όρια του Ισραήλ· ούτω θέλω σωθή εκ της χειρός αυτού.
\par 2 Και εσηκώθη ο Δαβίδ και διέβη, αυτός και οι εξακόσιοι άνδρες οι μετ' αυτού, προς τον Αγχούς υιόν του Μαώχ, βασιλέα της Γαθ.
\par 3 Και εκάθησεν ο Δαβίδ μετά του Αγχούς εν Γαθ, αυτός και οι άνδρες αυτού, έκαστος μετά της οικογενείας αυτού, και ο Δαβίδ μετά των δύο γυναικών αυτού, Αχινοάμ της Ιεζραηλίτιδος και Αβιγαίας της Καρμηλίτιδος γυναικός του Νάβαλ.
\par 4 Ανηγγέλθη δε προς τον Σαούλ ότι έφυγεν ο Δαβίδ εις Γαθ. όθεν δεν εζήτησε πλέον αυτόν.
\par 5 Και είπεν ο Δαβίδ προς τον Αγχούς, Εάν εύρηκα τώρα χάριν εις τους οφθαλμούς σου, ας μοι δοθή τόπος εις τινά των πόλεων της εξοχής, διά να καθήσω εκεί· διότι πως να κάθηται ο δούλός σου μετά σου εν τη βασιλευούση πόλει;
\par 6 Και έδωκεν εις αυτόν ο Αγχούς την Σικλάγ κατ' εκείνην την ημέραν· διά τούτο η Σικλάγ έμεινεν εις τους βασιλείς του Ιούδα μέχρι της σήμερον.
\par 7 Ο δε αριθμός των ημερών, τας οποίας ο Δαβίδ εκάθησεν εν τη γη των Φιλισταίων, έγεινεν εν έτος και τέσσαρες μήνες.
\par 8 Ανέβαινε δε ο Δαβίδ και οι άνδρες αυτού και έκαμνον εισδρομάς εις τους Γεσσουρίτας και Γεζραίους και Αμαληκίτας· διότι ούτοι ήσαν εκ παλαιού οι κάτοικοι της γης, κατά την είσοδον Σούρ και έως της γης Αιγύπτου.
\par 9 Και εκτύπα ο Δαβίδ την γην και δεν άφινε ζώντα ούτε άνδρα ούτε γυναίκα· και ελάμβανε πρόβατα και βόας και όνους και καμήλους και ενδύματα· και επιστρέφων ήρχετο προς τον Αγχούς.
\par 10 Και έλεγεν ο Αγχούς προς τον Δαβίδ, που εκάμετε εισδρομήν σήμερον; Και απεκρίνετο ο Δαβίδ, προς το μεσημβρινόν του Ιούδα και προς το μεσημβρινόν των Ιεραμεηλιτών και προς το μεσημβρινόν των Κεναίων.
\par 11 Και ούτε άνδρα ούτε γυναίκα δεν άφινε ζώντα ο Δαβίδ, διά να φέρη είδησιν εις Γαθ, λέγων, Μήποτε αναγγείλωσιν εναντίον ημών, λέγοντες, Ούτω κάμνει ο Δαβίδ και τοιούτος είναι ο τρόπος αυτού, καθ' όλας τας ημέρας όσας κάθηται εν τη γη των Φιλισταίων.
\par 12 Και επίστευεν ο Αγχούς τον Δαβίδ, λέγων, Αυτός έκαμεν εαυτόν διόλου μισητόν εις τον λαόν αυτού τον Ισραήλ· διά τούτο θέλει είσθαι δούλος εις εμέ πάντοτε.

\chapter{28}

\par 1 Κατ' εκείνας δε τας ημέρας συνήθροισαν οι Φιλισταίοι τα στρατεύματα αυτών προς εκστρατείαν, διά να πολεμήσωσι μετά του Ισραήλ. Και είπεν ο Αγχούς προς τον Δαβίδ, Έξευρε μετά βεβαιότητος ότι θέλεις εξέλθει μετ' εμού εις τον πόλεμον, συ και οι άνδρες σου.
\par 2 Και είπεν ο Δαβίδ προς τον Αγχούς, Θέλεις βεβαίως γνωρίσει τι θέλει κάμει ο δούλός σου. Και είπεν ο Αγχούς προς τον Δαβίδ, Διά τούτο θέλω σε κάμει αρχισωματοφύλακά μου διαπαντός.
\par 3 Απέθανε δε ο Σαμουήλ, και πας ο Ισραήλ εθρήνησεν αυτόν και ενεταφίασεν αυτόν εν Ραμά τη πόλει αυτού. Και εξέβαλεν ο Σαούλ εκ του τόπου τους έχοντας πνεύμα μαντείας και τους μάγους.
\par 4 Συνηθροίσθησαν λοιπόν οι Φιλισταίοι και ήλθον και εστρατοπέδευσαν εν Σουνήμ· και συνήθροισεν ο Σαούλ πάντα τον Ισραήλ, και εστρατοπέδευσαν εν Γελβουέ.
\par 5 Και ότε είδεν ο Σαούλ το στρατόπεδον των Φιλισταίων, εφοβήθη, και ετρόμαξεν η καρδία αυτού σφόδρα.
\par 6 Και ηρώτησεν ο Σαούλ τον Κύριον· αλλ' ο Κύριος δεν απεκρίθη προς αυτόν ούτε δι' ενυπνίων ούτε διά του Ουρίμ ούτε διά προφητών.
\par 7 Τότε είπεν ο Σαούλ προς τους δούλους αυτού, Ζητήσατέ μοι γυναίκα έχουσαν πνεύμα μαντείας, διά να υπάγω προς αυτήν και να ερωτήσω αυτήν. Και οι δούλοι αυτού είπον προς αυτόν, Ιδού, είναι εν Εν-δωρ γυνή τις έχουσα πνεύμα μαντείας.
\par 8 Και μετεσχηματίσθη ο Σαούλ και ενεδύθη άλλα ιμάτια, και υπήγεν αυτός και δύο άνδρες μετ' αυτού και ήλθον προς την γυναίκα διά νυκτός· και είπε, Μάντευσον, παρακαλώ, εις εμέ διά του πνεύματος της μαντείας και αναβίβασόν μοι όντινα σοι είπω.
\par 9 Και είπεν η γυνή προς αυτόν, Ιδού, συ εξεύρεις όσα έκαμεν ο Σαούλ, τίνι τρόπω εξωλόθρευσε τους έχοντας πνεύμα μαντείας και τους μάγους εκ του τόπου· διά τι λοιπόν συ παγιδεύεις την ζωήν μου, διά να με θανατώσωσι;
\par 10 Και ώμοσε προς αυτήν ο Σαούλ εις τον Κύριον, λέγων, Ζη Κύριος, δεν θέλει σε συμβή ουδέν κακόν διά τούτο.
\par 11 Τότε είπεν η γυνή, Τίνα να σοι αναβιβάσω; Και είπε, τον Σαμουήλ αναβίβασόν μοι.
\par 12 Και ότε είδεν γυνή τον Σαμουήλ, εβόησε μετά φωνής μεγάλης· και είπεν η γυνή προς τον Σαούλ, λέγουσα, Διά τι με ηπάτησας; και συ είσαι ο Σαούλ.
\par 13 Και είπε προς αυτήν ο βασιλεύς, Μη φοβού· τι είδες λοιπόν; Και είπεν η γυνή προς τον Σαούλ, Θεούς είδον αναβαίνοντας εκ της γης.
\par 14 Και είπε προς αυτήν, Τις είναι η μορφή αυτού; Η δε είπε, Γέρων τις αναβαίνει και είναι περιτετυλιγμένος με επένδυμα. Και εγνώρισεν ο Σαούλ ότι ήτο ο Σαμουήλ, και έκυψε κατά πρόσωπον εις την γην και προσεκύνησε.
\par 15 Και είπεν ο Σαμουήλ προς τον Σαούλ, Διά τι με παρηνώχλησας, ώστε να με κάμης να αναβώ; Και απεκρίθη ο Σαούλ, Ευρίσκομαι εν μεγάλη αμηχανία· διότι οι Φιλισταίοι πολεμούσιν εναντίον μου, και ο Θεός απεμακρύνθη απ' εμού και δεν μοι αποκρίνεται πλέον ούτε διά προφητών ούτε δι' ενυπνίων· διά τούτο σε εκάλεσα διά να φανερώσης εις εμέ τι να κάμω.
\par 16 Τότε είπεν ο Σαμουήλ, Διά τι λοιπόν ερωτάς εμέ, αφού ο Κύριος απεμακρύνθη από σου και έγεινεν εχθρός σου;
\par 17 ο Κύριος βεβαίως έκαμεν εις εαυτόν ως ελάλησε δι' εμού· διότι εξέσχισεν ο Κύριος την βασιλείαν εκ της χειρός σου και έδωκεν αυτήν εις τον πλησίον σου, τον Δαβίδ·
\par 18 επειδή δεν υπήκουσας εις την φωνήν του Κυρίου, ουδέ εξετέλεσας τον μέγαν θυμόν αυτού κατά του Αμαλήκ, διά τούτο ο Κύριος έκαμεν εις σε το πράγμα τούτο την ημέραν ταύτην·
\par 19 και θέλει παραδώσει ο Κύριος και τον Ισραήλ μετά σου εις την χείρα των Φιλισταίων· και αύριον συ και οι υιοί σου θέλετε είσθαι μετ' εμού· και το στρατόπεδον του Ισραήλ θέλει παραδώσει ο Κύριος εις την χείρα των Φιλισταίων.
\par 20 Τότε έπεσεν ο Σαούλ ευθύς όλος εξηπλωμένος κατά γής· διότι κατετρόμαξεν εκ των λόγων του Σαμουήλ· και δύναμις δεν ήτο εν αυτώ, επειδή δεν είχε φάγει άρτον όλην την ημέραν και όλην την νύκτα.
\par 21 Και ήλθεν η γυνή προς τον Σαούλ και είδεν ότι ήτο σφόδρα τεταραγμένος, και είπε προς αυτόν, Ιδού, η δούλη σου υπήκουσεν εις την φωνήν σου, και έβαλον την ζωήν μου εις την χείρα μου και υπετάχθην εις τους λόγους σου, τους οποίους ελάλησας προς εμέ·
\par 22 τώρα λοιπόν, άκουσον και συ, παρακαλώ, την φωνήν της δούλης σου, και ας βάλω ολίγον άρτον έμπροσθέν σου· και φάγε, διά να λάβης δύναμιν, επειδή υπάγεις εις οδοιπορίαν.
\par 23 Πλην δεν ήθελε, λέγων, Δεν θέλω φάγει· οι δούλοι όμως αυτού μετά της γυναικός εβίαζον αυτόν, και εισήκουσεν εις την φωνήν αυτών· και σηκωθείς από της γης, εκάθησεν επί της κλίνης.
\par 24 είχε δε η γυνή παχύ δαμάλιον εν τη οικία· και έσπευσε και έσφαξεν αυτό· και λαβούσα άλευρον, εζύμωσε και έψησεν άζυμα εξ αυτού.
\par 25 Και έφερεν έμπροσθεν του Σαούλ και έμπροσθεν των δούλων αυτού· και έφαγον. Και εσηκώθησαν και ανεχώρησαν την νύκτα εκείνην.

\chapter{29}

\par 1 Συνήθροισαν δε οι Φιλισταίοι πάντα τα στρατεύματα αυτών εις Αφέκ· και οι Ισραηλίται εστρατοπέδευσαν παρά την πηγήν την εν Ιεζραέλ.
\par 2 Και οι σατράπαι των Φιλισταίων διέβαινον κατά εκατοντάδας και χιλιάδας· ο Δαβίδ δε και οι άνδρες αυτού διέβαινον κατόπισθεν μετά του Αγχούς.
\par 3 Και είπον οι στρατηγοί των Φιλισταίων, Τι θέλουσιν ούτοι οι Εβραίοι; Και είπεν ο Αγχούς προς τους στρατηγούς των Φιλισταίων, Δεν είναι ούτος ο Δαβίδ, ο δούλος του Σαούλ βασιλέως του Ισραήλ, όστις εστάθη μετ' εμού ταύτας τας ημέρας ή τούτους τους χρόνους; και δεν εύρηκα εν αυτώ ουδέν σφάλμα, αφού ενέπεσεν εις εμέ έως της ημέρας ταύτης.
\par 4 Και ηγανάκτησαν κατ' αυτού οι στρατηγοί των Φιλισταίων· και είπον προς αυτόν οι στρατηγοί των Φιλισταίων, Απόπεμψον τον άνθρωπον τούτον, και ας επιστρέψη εις τον τόπον αυτού, τον οποίον διώρισας εις αυτόν, και ας μη καταβή μεθ' ημών εις την μάχην, μήποτε γείνη εν τη μάχη πολέμιος ημών· διότι πως ήθελε διαλλαγή ούτος μετά του κυρίου αυτού; ουχί με τας κεφαλάς των ανδρών τούτων;
\par 5 δεν είναι ούτος ο Δαβίδ, περί του οποίον έψαλλον αμοιβαίως εν τοις χοροίς, λέγοντες, Ο Σαούλ επάταξε τας χιλιάδας αυτού, Και ο Δαβίδ τας μυριάδας αυτού;
\par 6 Τότε εκάλεσεν ο Αγχούς τον Δαβίδ και είπε προς αυτόν, Ζη Κύριος, βεβαίως εστάθης ευθύς, και η έξοδός σου και η είσοδός σου μετ' εμού εν τω στρατοπέδω είναι αρεστή έμπροσθεν των οφθαλμών μου· διότι κακόν δεν εύρηκα εν σοι, αφ' ης ημέρας ήλθες προς εμέ έως της ημέρας ταύτης· αλλ' όμως εις τους οφθαλμούς των σατραπών δεν είσαι αρεστός·
\par 7 τώρα λοιπόν επίστρεψον και ύπαγε εν ειρήνη, διά να μη φέρης δυσαρέσκειαν εις τους σατράπας των Φιλισταίων.
\par 8 Και είπεν ο Δαβίδ προς τον Αγχούς, Αλλά τι έκαμα; και τι εύρηκας εν τω δούλω σου αφ' ης ημέρας είμαι ενώπιόν σου, έως της ημέρας ταύτης, ώστε να μη υπάγω να πολεμήσω εναντίον των εχθρών του κυρίου μου του βασιλέως;
\par 9 Και απεκρίθη ο Αγχούς και είπε προς τον Δαβίδ, Εξεύρω ότι είσαι αρεστός εις τους οφθαλμούς μου, ως άγγελος Θεού· πλην οι σατράπαι των Φιλισταίων είπον, Δεν θέλει αναβή μεθ' ημών εις την μάχην·
\par 10 τώρα λοιπόν σηκώθητι ενωρίς το πρωΐ, μετά των δούλων του κυρίου σου, των ελθόντων μετά σού· και καθώς σηκωθήτε ενωρίς το πρωΐ, ευθύς όταν φέγξη, αναχωρήσατε.
\par 11 Και εσηκώθη ενωρίς το πρωΐ ο Δαβίδ και οι άνδρες αυτού, διά να αναχωρήσωσι, να επιστρέψωσιν εις την γην των Φιλισταίων. Οι δε Φιλισταίοι ανέβησαν εις Ιεζραέλ.

\chapter{30}

\par 1 Και ότε ο Δαβίδ και οι άνδρες αυτού εισήλθον εις Σικλάγ την τρίτην ημέραν, οι Αμαληκίται είχον κάμει εισδρομήν εις το μεσημβρινόν και εις Σικλάγ, και είχον πατάξει την Σικλάγ και κατακαύσει αυτήν εν πυρί·
\par 2 και είχον αιχμαλωτίσει τας γυναίκας τας εν αυτή, από μικρού έως μεγάλου· δεν εθανάτωσαν ουδένα, αλλά έλαβον αυτούς και υπήγαν εις την οδόν αυτών.
\par 3 Ο δε Δαβίδ και οι άνδρες αυτού ήλθον εις την πόλιν, και ιδού, ήτο πυρπολημένη· και αι γυναίκες αυτών και οι υιοί αυτών και αι θυγατέρες αυτών ηχμαλωτισμένοι.
\par 4 Τότε ύψωσεν ο Δαβίδ και ο λαός ο μετ' αυτού την φωνήν αυτών και έκλαυσαν, εωσού δεν έμεινε πλέον εν αυτοίς δύναμις να κλαίωσι.
\par 5 Και αμφότεραι αι γυναίκες του Δαβίδ ηχμαλωτίσθησαν, Αχινοάμ η Ιεζραηλίτις, και Αβιγαία η γυνή Νάβαλ του Καρμηλίτου.
\par 6 Και εθλίβη ο Δαβίδ σφόδρα· διότι ο λαός έλεγε να λιθοβολήσωσιν αυτόν, επειδή η ψυχή παντός του λαού ήτο κατάπικρος, έκαστος διά τους υιούς αυτού και διά τας θυγατέρας αυτού· ο Δαβίδ όμως εκραταιώθη εν Κυρίω τω Θεώ αυτού.
\par 7 Και είπεν ο Δαβίδ προς Αβιάθαρ τον ιερέα, υιόν του Αχιμέλεχ, Φέρε μοι ενταύθα, παρακαλώ, το εφόδ. Και έφερεν ο Αβιάθαρ το εφόδ προς τον Δαβίδ.
\par 8 Και ηρώτησεν ο Δαβίδ τον Κύριον, λέγων, Να καταδιώξω όπισθεν τούτων των ληστών; θέλω προφθάσει αυτούς; Ο δε είπε προς αυτόν, Καταδίωξον· διότι θέλεις βεβαίως προφθάσει και αφεύκτως θέλεις ελευθερώσει πάντα.
\par 9 Τότε υπήγεν ο Δαβίδ, αυτός και οι εξακόσιοι άνδρες οι μετ' αυτού, και ήλθον έως του χειμάρρου Βοσόρ, όπου οι απομένοντες εστάθησαν.
\par 10 Ο δε Δαβίδ, αυτός και τετρακόσιοι άνδρες, κατεδίωκον, επειδή έμειναν οπίσω διακόσιοι, οίτινες αποκαμόντες δεν ηδύναντο να διαβώσι τον χείμαρρον Βοσόρ.
\par 11 Και εύρηκαν άνθρωπον Αιγύπτιον εν αγρώ και έφεραν αυτόν προς τον Δαβίδ· και έδωκαν εις αυτόν άρτον, και έφαγε, και επότισαν αυτόν ύδωρ·
\par 12 και έδωκαν εις αυτόν τμήμα πήττας σύκων και δύο βότρυς σταφίδων· και έφαγε, και επανήλθε το πνεύμα αυτού εις αυτόν· διότι δεν είχε φάγει άρτον ουδέ είχε πίει ύδωρ, τρεις ημέρας και τρεις νύκτας.
\par 13 Και είπε προς αυτόν ο Δαβίδ, Τίνος είσαι; και πόθεν είσαι; Και είπεν, Είμαι νέος Αιγύπτιος, δούλος τινός Αμαληκίτου· και με αφήκεν ο κύριός μου, επειδή ηρρώστησα τρεις ημέρας τώρα·
\par 14 ημείς εκάμαμεν εισδρομήν εις το μεσημβρινόν των Χερεθαίων και εις τα μέρη της Ιουδαίας και εις το μεσημβρινόν του Χάλεβ· και επυρπολήσαμεν την Σικλάγ.
\par 15 Και είπε προς αυτόν ο Δαβίδ, Δύνασαι να με οδηγήσης κάτω προς τους ληστάς τούτους; Ο δε είπεν, Ομοσόν μοι εις τον Θεόν, ότι δεν θέλεις με θανατώσει ούτε θέλεις με παραδώσει εις την χείρα του κυρίου μου, και θέλω σε οδηγήσει κάτω προς τούτους τους ληστάς.
\par 16 Και ότε ώδήγησεν αυτόν κάτω, ιδού, ήσαν διεσκορπισμένοι επί το πρόσωπον παντός του τόπου, τρώγοντες και πίνοντες και χορεύοντες, διά πάντα τα λάφυρα τα μεγάλα, τα οποία έλαβον εκ της γης των Φιλισταίων και εκ της γης του Ιούδα.
\par 17 Και επάταξεν αυτούς ο Δαβίδ από της αυγής μέχρι της εσπέρας της επιούσης· και δεν διεσώθη ουδέ εις εξ αυτών, πλην τετρακοσίων νέων, οίτινες εκάθηντο επί καμήλων και έφυγον.
\par 18 Και ηλευθέρωσεν ο Δαβίδ όσα ήρπασαν οι Αμαληκίται· και τας δύο γυναίκας αυτού ηλευθέρωσεν ο Δαβίδ.
\par 19 Και δεν έλειψεν εις αυτούς ούτε μικρόν ούτε μέγα, ούτε υιοί ούτε θυγατέρες ούτε λάφυρον ούτε ουδέν εκ των όσα ήρπασαν απ' αυτών· τα πάντα επανέλαβεν ο Δαβίδ.
\par 20 Και έλαβεν ο Δαβίδ πάντα τα πρόβατα και τους βόας, και φέροντες αυτά έμπροσθεν των άλλων κτηνών, έλεγον, Ταύτα είναι τα λάφυρα του Δαβίδ.
\par 21 Και ήλθεν ο Δαβίδ προς τους διακοσίους άνδρας, οίτινες είχον αποκάμει ώστε δεν ηδυνήθησαν να ακολουθήσωσι τον Δαβίδ, όθεν εκάθισεν αυτούς εις τον χείμαρρον Βοσόρ· και εξήλθον εις συνάντησιν του Δαβίδ και εις συνάντησιν του λαού του μετ' αυτού· και ότε επλησίασεν ο Δαβίδ εις τον λαόν, εχαιρέτησεν αυτούς.
\par 22 Και απεκρίθηααν πάντες οι πονηροί και διεστραμμένοι εκ των ανδρών, οίτινες υπήγαν μετά του Δαβίδ, και είπον, Επειδή ούτοι δεν ήλθον μεθ' ημών, δεν θέλομεν δώσει εις αυτούς εκ των λαφύρων, τα οποία ανελάβομεν, παρά εις έκαστον την γυναίκα αυτού και τα τέκνα αυτού· και ας λάβωσιν αυτά και ας φύγωσιν.
\par 23 Αλλ' ο Δαβίδ είπε, Δεν θέλετε κάμει ούτως, αδελφοί μου, εις εκείνα τα οποία ο Κύριος έδωκεν εις ημάς, όστις εφύλαξεν ημάς και παρέδωκεν εις την χείρα ημών τους ληστάς τους ελθόντας εναντίον ημών·
\par 24 και τις θέλει σας εισακούσει εις ταύτην την υπόθεσιν; αλλά κατά την μερίδα του καταβαίνοντος εις τον πόλεμον, ούτω θέλει είσθαι η μερίς του καθημένου πλησίον της αποσκευής· ίσα θέλουσι μοιράζεσθαι.
\par 25 Και έγεινεν ούτως απ' εκείνης της ημέρας και εις το εξής· και έκαμε τούτο νόμον και διάταγμα εν τω Ισραήλ έως της ημέρας ταύτης.
\par 26 Ότε δε ήλθεν ο Δαβίδ εις Σικλάγ, έπεμψεν εκ των λαφύρων προς τους πρεσβυτέρους Ιούδα τους φίλους αυτού, λέγων, Ιδού εις εσάς ευλογία, εκ των λαφύρων των εχθρών του Κυρίου.
\par 27 προς τους εν Βαιθήλ, και προς τους εν Ραμώθ τη μεσημβρινή, και προς τους εν Ιαθείρ,
\par 28 και προς τους εν Αροήρ, και προς τους εν Σιφμώθ, και προς τους εν Εσθεμωά,
\par 29 και προς τους εν Ραχάλ, και προς τους εν ταις πόλεσι των Ιεραμεηλιτών, και προς τους εν ταις πόλεσι των Κεναίων,
\par 30 και προς τους εν Ορμά, και προς τους εν Χωρ-ασάν, και προς τους εν Αθάχ,
\par 31 και προς τους εν Χεβρών, και προς πάντας τους τόπους, εις τους οποίους ο Δαβίδ περιήρχετο, αυτός και οι άνδρες αυτού.

\chapter{31}

\par 1 Οι δε Φιλισταίοι επολέμουν κατά του Ισραήλ· και έφυγον οι άνδρες του Ισραήλ από προσώπου των Φιλισταίων και έπεσον πεφονευμένοι εν τω όρει Γελβουέ.
\par 2 Και κατέφθασαν οι Φιλισταίοι τον Σαούλ και τους υιούς αυτού· και επάταξαν οι Φιλισταίοι τον Ιωνάθαν και τον Αβιναδάβ και τον Μελχί-σουέ, τους υιούς του Σαούλ.
\par 3 Εβάρυνε δε η μάχη επί τον Σαούλ, και επέτυχον αυτόν οι άνδρες οι τοξόται και επληγώθη βαρέως υπό των τοξοτών.
\par 4 Και είπεν ο Σαούλ προς τον οπλοφόρον αυτού, Σύρε την ρομφαίαν σου και διαπέρασόν με δι' αυτής, διά να μη έλθωσιν ούτοι οι απερίτμητοι και με διαπεράσωσι και με εμπαίξωσι· πλην ο οπλοφόρος αυτού δεν ήθελε, διότι εφοβείτο σφόδρα. Όθεν έλαβεν ο Σαούλ την ρομφαίαν και έπεσεν επ' αυτήν.
\par 5 Και ως είδεν ο οπλοφόρος αυτού ότι απέθανεν ο Σαούλ, έπεσε και αυτός επί την ρομφαίαν αυτού και απέθανε μετ' αυτού.
\par 6 ούτως απέθανεν ο Σαούλ και οι τρεις υιοί αυτού, και ο οπλοφόρος αυτού και πάντες οι άνδρες αυτού, την αυτήν εκείνην ημέραν ομού.
\par 7 Και οι άνδρες Ισραήλ, οι πέραν της κοιλάδος, και οι πέραν του Ιορδάνου, ιδόντες ότι έφυγον οι άνδρες Ισραήλ και ότι ο Σαούλ και οι υιοί αυτού απέθανον, κατέλιπον τας πόλεις και έφυγον. και ελθόντες οι Φιλισταίοι κατώκησαν εν αυταίς.
\par 8 Και την επαύριον, ότε ήλθον οι Φιλισταίοι διά να εκδύσωσι τους πεφονευμένους, εύρηκαν τον Σαούλ και τους τρεις υιούς αυτού πεπτωκότας επί το όρος Γελβουέ.
\par 9 Και απέκοψαν την κεφαλήν αυτού και εξέδυσαν τα όπλα αυτού και απέστειλαν εις την γην των Φιλισταίων κύκλω, διά να διαδώσωσι την αγγελίαν εις τον οίκον των ειδώλων αυτών και μεταξύ του λαού.
\par 10 Και ανέθεσαν τα όπλα αυτού εις τον οίκον της Ασταρώθ, και εκρέμασαν το σώμα αυτού εις το τείχος Βαιθ-σαν.
\par 11 Ακούσαντες δε περί τούτου οι κάτοικοι της Ιαβείς-γαλαάδ, τι έκαμον οι Φιλισταίοι εις τον Σαούλ,
\par 12 ηγέρθησαν πάντες οι δυνατοί άνδρες και ώδοιπόρησαν όλην την νύκτα και έλαβον το σώμα του Σαούλ και τα σώματα των υιών αυτού από του τείχους Βαιθ-σαν, και ήλθον εις Ιαβείς και έκαυσαν αυτά εκεί·
\par 13 και έλαβον τα οστά αυτών και έθαψαν υπό το δένδρον εν Ιαβείς και ενήστευσαν επτά ημέρας.


\end{document}