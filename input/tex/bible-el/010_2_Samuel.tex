\begin{document}

\title{2 Samuel}


\chapter{1}

\par 1 Μετά δε τον θάνατον του Σαούλ, αφού επέστρεψεν ο Δαβίδ από της σφαγής των Αμαληκιτών, εκάθησεν ο Δαβίδ εν Σικλάγ δύο ημέρας·
\par 2 την δε τρίτην ημέραν, ιδού, ήλθεν άνθρωπος εκ του στρατοπέδου από πλησίον του Σαούλ, έχων διεσχισμένα τα ιμάτια αυτού και χώμα επί της κεφαλής αυτού· και καθώς εισήλθε προς τον Δαβίδ, έπεσεν εις την γην και προσεκύνησε.
\par 3 Και είπε προς αυτόν ο Δαβίδ, Πόθεν έρχεσαι; Ο δε είπε προς αυτόν, Εγώ εκ του στρατοπέδου του Ισραήλ διεσώθην.
\par 4 Και είπε προς αυτόν ο Δαβίδ, Τι συνέβη; ειπέ μοι, παρακαλώ. Και απεκρίθη, Ότι έφυγεν ο λαός εκ της μάχης, και πολλοί μάλιστα εκ του λαού έπεσον και απέθανον· απέθανον δε και Σαούλ και Ιωνάθαν ο υιός αυτού.
\par 5 Και είπεν ο Δαβίδ προς τον νέον τον απαγγέλλοντα προς αυτόν, Πως εξεύρεις ότι απέθανεν ο Σαούλ, και Ιωνάθαν ο υιός αυτού;
\par 6 Και είπεν ο νέος ο απαγγέλλων προς αυτόν, Ευρέθην κατά τύχην εν τω όρει Γελβουέ, και ιδού, ο Σαούλ ήτο κεκλιμένος επί του δόρατος αυτού, και ιδού, αι άμαξαι και οι ιππείς κατέφθανον αυτόν.
\par 7 και ότε έβλεψεν εις τα οπίσω αυτού, με είδε και με εκάλεσε· και απεκρίθην, Ιδού, εγώ.
\par 8 Και είπε προς εμέ, Ποίος είσαι; Και απεκρίθην προς αυτόν, Είμαι Αμαληκίτης.
\par 9 Πάλιν είπε προς εμέ, Στήθι επάνω μου, παρακαλώ, και θανάτωσόν με· διότι σκοτοδινίασις με κατέλαβεν, επειδή η ζωή μου είναι έτι όλη εν εμοί.
\par 10 Εστάθην λοιπόν επ' αυτόν και εθανάτωσα αυτόν· επειδή ήμην βέβαιος ότι δεν ηδύνατο να ζήση αφού έπεσε· και έλαβον το διάδημα το επί της κεφαλής αυτού και το βραχιόλιον το εν τω βραχίονι αυτού, και έφερα αυτά ενταύθα προς τον κύριόν μου.
\par 11 Τότε πιάσας ο Δαβίδ τα ιμάτια αυτού, διέσχισεν αυτά· και πάντες ομοίως οι άνδρες οι μετ' αυτού.
\par 12 Και επένθησαν και έκλαυσαν και ενήστευσαν έως εσπέρας διά τον Σαούλ και διά Ιωνάθαν τον υιόν αυτού και διά τον λαόν του Κυρίου και διά τον οίκον του Ισραήλ, διότι έπεσον διά ρομφαίας.
\par 13 Είπε δε ο Δαβίδ προς τον νέον, τον απαγγέλλοντα προς αυτόν, Πόθεν είσαι; Και απεκρίθη, Είμαι υιός παροίκου τινός Αμαληκίτου.
\par 14 Και είπε προς αυτόν ο Δαβίδ, Πως δεν εφοβήθης να επιβάλης την χείρα σου διά να θανατώσης τον κεχρισμένον του Κυρίου;
\par 15 Και εκάλεσεν ο Δαβίδ ένα εκ των νέων και είπε, Πλησίασον, πέσον επ' αυτόν. Και επάταξεν αυτόν, και απέθανε.
\par 16 Και είπε προς αυτόν ο Δαβίδ, Το αίμα σου επί της κεφαλής σου· διότι το στόμα σου εμαρτύρησεν εναντίον σου, λέγων, Εγώ εθανάτωσα τον κεχρισμένον του Κυρίου.
\par 17 Και εθρήνησεν ο Δαβίδ τον θρήνον τούτον επί τον Σαούλ και επί Ιωνάθαν τον υιόν αυτού·
\par 18 και παρήγγειλε να διδάξωσι τους υιούς Ιούδα τούτο το άσμα του τόξου· ιδού, είναι γεγραμμένον εν τω βιβλίω του Ιασήρ.
\par 19 Ω δόξα του Ισραήλ, επί τους υψηλούς τόπους σου κατηκοντισμένη. Πως έπεσον οι δυνατοί.
\par 20 Μη αναγγείλητε εις την Γαθ, μη διακηρύξητε εις τας πλατείας της Ασκάλωνος, μήποτε χαρώσιν αι θυγατέρες των Φιλισταίων, μήποτε αγαλλιάσωνται αι θυγατέρες των απεριτμήτων·
\par 21 Ορη τα εν Γελβουέ, Ας μη ήναι δρόσος μηδέ βροχή εφ' υμάς, μηδέ αγροί δίδοντες απαρχάς· διότι εκεί απερρίφθη η ασπίς των ισχυρών, Η ασπίς του Σαούλ, ως να μη εχρίσθη δι' ελαίου.
\par 22 Από του αίματος των πεφονευμένων, από του στέατος των ισχυρών, το τόξον του Ιωνάθαν δεν εστρέφετο οπίσω, και η ρομφαία του Σαούλ δεν επέστρεφε κενή.
\par 23 Σαούλ και Ιωνάθαν ήσαν οι ηγαπημένοι και εράσμιοι εν τη ζωή αυτών, και εν τω θανάτω αυτών δεν εχωρίσθησαν· ήσαν ελαφρότεροι αετών, δυνατώτεροι λεόντων.
\par 24 Θυγατέρες Ισραήλ, κλαύσατε επί τον Σαούλ τον ενδύοντα υμάς κόκκινα μετά καλλωπισμών, τον επιβάλλοντα στολισμούς χρυσούς επί τα ενδύματα υμών.
\par 25 Πως έπεσον οι δυνατοί εν μέσω της μάχης· Ιωνάθαν, επί τους υψηλούς τόπους σου τετραυματισμένε.
\par 26 Περίλυπος είμαι διά σε, αδελφέ μου Ιωνάθαν· προσφιλέστατος εστάθης εις εμέ· η προς εμέ αγάπη σου ήτο εξαίσιος. Υπερέβαινε την αγάπην των γυναικών.
\par 27 Πως έπεσον οι δυνατοί, και απωλέσθησαν τα όπλα του πολέμου.

\chapter{2}

\par 1 Μετά δε ταύτα ηρώτησεν ο Δαβίδ τον Κύριον, λέγων, να αναβώ εις τινά των πόλεων Ιούδα; Ο δε Κύριος είπε προς αυτόν, Ανάβα. Και είπεν ο Δαβίδ, που να αναβώ; Ο δε είπεν, εις Χεβρών.
\par 2 Ανέβη λοιπόν εκεί ο Δαβίδ και αι δύο γυναίκες αυτού, Αχινοάμ η Ιεζραηλίτις και Αβιγαία η γυνή Νάβαλ του Καρμηλίτου.
\par 3 Και τους άνδρας τους μετ' αυτού ανεβίβασεν ο Δαβίδ, έκαστον μετά της οικογενείας αυτού· και κατώκησαν εν ταις πόλεσι Χεβρών.
\par 4 Και ήλθον οι άνδρες Ιούδα και έχρισαν εκεί τον Δαβίδ βασιλέα επί τον οίκον Ιούδα. Και απήγγειλαν προς τον Δαβίδ, λέγοντες, Οι άνδρες της Ιαβείς-γαλαάδ ήσαν οι θάψαντες τον Σαούλ.
\par 5 Και απέστειλεν ο Δαβίδ μηνυτάς προς τους άνδρας της Ιαβείς-γαλαάδ και είπε προς αυτούς, Ευλογημένοι να ήσθε παρά του Κυρίου, διότι εκάμετε το έλεος τούτο εις τον κύριόν σας, εις τον Σαούλ, και εθάψατε αυτόν
\par 6 είθε λοιπόν τώρα να κάμη ο Κύριος προς εσάς έλεος και αλήθειαν και εγώ προσέτι θέλω ανταποδώσει εις εσάς το καλόν τούτο, επειδή εκάμετε τούτο το πράγμα·
\par 7 τώρα λοιπόν, ας κραταιωθώσιν αι χείρές σας, και γίνεσθε ανδρείοι διότι ο κύριός σας ο Σαούλ απέθανε, και προσέτι ο οίκος Ιούδα έχρισαν εμέ βασιλέα εφ' εαυτών.
\par 8 Ο Αβενήρ όμως, ο υιός του Νηρ, ο αρχιστράτηγος του Σαούλ, έλαβε τον Ις-βοσθέ, υιόν του Σαούλ, και διεβίβασεν αυτόν εις Μαχαναΐμ,
\par 9 και έκαμεν αυτόν βασιλέα επί της Γαλαάδ, και επί των Ασσουριτών, και επί της Ιεζραέλ, και επί του Εφραΐμ, και επί του Βενιαμίν, και επί παντός του Ισραήλ.
\par 10 Τεσσαράκοντα ετών ήτο Ις-βοσθέ ο υιός του Σαούλ, ότε έγεινε βασιλεύς επί τον Ισραήλ· και εβασίλευσε δύο έτη· ο οίκος όμως Ιούδα ηκολούθησε τον Δαβίδ.
\par 11 Και ο αριθμός των ημερών, καθ' ας εβασίλευσεν ο Δαβίδ εν Χεβρών επί του οίκου Ιούδα, ήσαν επτά έτη και εξ μήνες.
\par 12 Εξήλθε δε Αβενήρ ο υιός του Νηρ και οι δούλοι του Ις-βοσθέ, υιού του Σαούλ, εκ Μαχαναΐμ εις Γαβαών.
\par 13 Και Ιωάβ, ο υιός της Σερουΐας, και οι δούλοι του Δαβίδ εξήλθον και συναπηντήθησαν πλησίον του υδροστασίου της Γαβαών· και εκάθησαν οι μεν εντεύθεν του υδροστασίου, οι δε εκείθεν του υδροστασίου.
\par 14 Και είπεν ο Αβενήρ προς τον Ιωάβ, Ας σηκωθώσι τώρα οι νέοι και ας παίξωσιν έμπροσθεν ημών. Και είπεν ο Ιωάβ, Ας σηκωθώσιν.
\par 15 Εσηκώθησαν λοιπόν και επέρασαν κατά αριθμόν, δώδεκα εκ του Βενιαμίν, από μέρους του Ις-βοσθέ, υιού του Σαούλ, και δώδεκα εκ των δούλων του Δαβίδ.
\par 16 Και επίασαν έκαστος τον πλησίον αυτού από της κεφαλής, και διεπέρασε την μάχαιραν αυτού εις την πλευράν του πλησίον αυτού, και έπεσον ομού· όθεν ο τόπος εκείνος ωνομάσθη Χελκάθ-ασουρείμ, όστις είναι εν Γαβαών.
\par 17 Και έγεινεν μάχη σκληροτάτη κατ' εκείνην την ημέραν· και ο Αβενήρ και οι άνδρες Ισραήλ ενικήθησαν υπό των δούλων του Δαβίδ.
\par 18 Ήσαν δε εκεί οι τρεις υιοί της Σερουΐας, Ιωάβ και Αβισαί και Ασαήλ· ο δε Ασαήλ ήτο ελαφρός τους πόδας, ως μία των δορκάδων των εν αγρώ.
\par 19 Και κατεδίωξεν ο Ασαήλ οπίσω του Αβενήρ· και τρέχων, δεν εξέκλινεν εις τα δεξιά ουδέ εις τα αριστερά, εξόπισθεν του Αβενήρ.
\par 20 Και έβλεψεν ο Αβενήρ εις τα οπίσω αυτού και είπε, Συ είσαι ο Ασαήλ; Ο δε απεκρίθη, Εγώ.
\par 21 Και είπε προς αυτόν ο Αβενήρ, Στρέψον συ εις τα δεξιά ή εις τα αριστερά, και πίασον τινά εκ των νέων και λάβε εις σεαυτόν την πανοπλίαν αυτού· πλην δεν ηθέλησεν ο Ασαήλ να εκκλίνη από όπισθεν αυτού.
\par 22 Και πάλιν είπεν ο Αβενήρ προς τον Ασαήλ, Στρέψον από όπισθέν μου· διά τι να σε κτυπήσω έως εδάφους; πως θέλω σηκώσει τότε το πρόσωπόν μου προς Ιωάβ τον αδελφόν σου;
\par 23 Αλλά δεν ήθελε να στρέψη· όθεν επάταξεν αυτόν ο Αβενήρ με το όπισθεν του δόρατος αυτού εις την πέμπτην πλευράν, και εξήλθε το δόρυ από των οπισθίων αυτού, και έπεσεν εκεί και απέθανεν εν τω αυτώ τόπω· και όσοι ήρχοντο εις τον τόπον, όπου ο Ασαήλ έπεσε και απέθανεν, ίσταντο.
\par 24 Ο δε Ιωάβ και ο Αβισαί κατεδίωκον οπίσω του Αβενήρ· και ο ήλιος έδυεν ότε αυτοί ήλθον έως του βουνού Αμμά, το οποίον είναι απέναντι Για, κατά την οδόν της ερήμου Γαβαών.
\par 25 Και συνηθροίσθησαν οι υιοί Βενιαμίν οπίσω του Αβενήρ, και έγειναν εν σώμα και εστάθησαν επί της κορυφής τινός βουνού.
\par 26 Τότε ο Αβενήρ εφώνησε προς τον Ιωάβ και είπε, Θέλει κατατρώγει ακαταπαύστως ρομφαία; δεν εξεύρεις ότι πικρία θέλει είσθαι εις το τέλος; έως πότε λοιπόν δεν θέλεις προστάξει τον λαόν να επιστρέψη από του να καταδιώκωσι τους αδελφούς αυτών;
\par 27 Και είπεν ο Ιωάβ, Ζη ο Θεός, εάν δεν ήθελες λαλήσει, βεβαίως τότε ο λαός ήθελεν αναβή το πρωΐ, έκαστος από της καταδιώξεως του αδελφού αυτού.
\par 28 Και εσάλπισεν ο Ιωάβ εν τη σάλπιγγι· και εστάθη πας ο λαός, και δεν κατεδίωκον πλέον κατόπιν του Ισραήλ ουδέ εμάχοντο πλέον.
\par 29 Και ώδοιπόρησαν ο Αβενήρ και οι άνδρες αυτού διά της πεδιάδος όλην την νύκτα εκείνην, και διέβησαν τον Ιορδάνην και επέρασαν δι' όλης της Βιθρών και ήλθον εις Μαχαναΐμ.
\par 30 Ο δε Ιωάβ επέστρεψεν από της καταδιώξεως του Αβενήρ· και ότε συνήθροισε πάντα τον λαόν, έλειπον εκ των δούλων του Δαβίδ δεκαεννέα άνδρες και ο Ασαήλ.
\par 31 Οι δούλοι δε του Δαβίδ επάταξαν εκ του Βενιαμίν και εκ των ανδρών του Αβενήρ τριακοσίους εξήκοντα άνδρας, οίτινες απέθανον.
\par 32 Και εσήκωσαν τον Ασαήλ και έθαψαν αυτόν εν τω τάφω του πατρός αυτού, τω εν Βηθλεέμ. Ο δε Ιωάβ και οι άνδρες αυτού ώδοιπόρησαν όλην την νύκτα και έφθασαν εις Χεβρών περί τα χαράγματα.

\chapter{3}

\par 1 Διήρκεσε δε πολύ ο πόλεμος μεταξύ του οίκου του Σαούλ και του οίκου του Δαβίδ. Και ο μεν Δαβίδ προέβαινε κραταιούμενος· ο δε οίκος του Σαούλ προέβαινεν εξασθενούμενος.
\par 2 Εγεννήθησαν δε εις τον Δαβίδ υιοί εν Χεβρών· και ο μεν πρωτότοκος αυτού ήτο Αμνών, εκ της Αχινοάμ της Ιεζραηλίτιδος·
\par 3 ο δε δεύτερος αυτού, Χιλεάβ, εκ της Αβιγαίας, γυναικός του Νάβαλ του Καρμηλίτου· ο δε τρίτος, Αβεσσαλώμ, υιός της Μααχά, θυγατρός του Θαλμαΐ, βασιλέως της Γεσσούρ·
\par 4 ο δε τέταρτος, Αδωνίας, υιός της Αγγείθ· και ο πέμπτος, Σεφατίας, υιός της Αβιτάλ·
\par 5 και ο έκτος, Ιθραάμ, εκ της Αιγλά, της γυναικός του Δαβίδ. Ούτοι εγεννήθησαν εις τον Δαβίδ εν Χεβρών.
\par 6 Ενώ δε εξηκολούθει ο πόλεμος μεταξύ του οίκου του Σαούλ και του οίκου του Δαβίδ, ο Αβενήρ υπεστήριζε τον οίκον του Σαούλ.
\par 7 Είχε δε ο Σαούλ παλλακήν, ονομαζομένην Ρεσφά, θυγατέρα του Αϊά· και είπεν ο Ις-βοσθέ προς τον Αβενήρ, Διά τι εισέρχεσαι προς την παλλακήν του πατρός μου;
\par 8 Και εθυμώθη σφόδρα ο Αβενήρ διά τους λόγους του Ις-βοσθέ και είπε, Κεφαλή κυνός είμαι εγώ, όστις κάμνω σήμερον έλεος προς τον οίκον Σαούλ του πατρός σου, προς τους αδελφούς αυτού και προς τους φίλους αυτού, εναντίον του Ιούδα, και δεν σε παρέδωκα εις την χείρα του Δαβίδ, ώστε να ελέγχης σήμερον αδικίαν εις εμέ περί της γυναικός ταύτης;
\par 9 ούτω να κάμη ο Θεός εις τον Αβενήρ και ούτω να προσθέση εις αυτόν, εάν, καθώς ώμοσεν ο Κύριος εις τον Δαβίδ, δεν κάμω ούτως εις αυτόν,
\par 10 να μεταβιβάσω την βασιλείαν εκ του οίκου του Σαούλ, και να στήσω τον θρόνον του Δαβίδ επί τον Ισραήλ και επί τον Ιούδαν, από Δαν έως Βηρ-σαβεέ.
\par 11 Και δεν ηδύνατο πλέον να αποκριθή λόγον προς τον Αβενήρ, επειδή εφοβείτο αυτόν.
\par 12 Τότε απέστειλεν ο Αβενήρ μηνυτάς προς τον Δαβίδ από μέρους αυτού, λέγων, Τίνος είναι η γη; λέγων προσέτι, Κάμε συνθήκην μετ' εμού, και ιδού, η χειρ μου θέλει είσθαι μετά σου, ώστε να φέρω υπό την εξουσίαν σου πάντα τον Ισραήλ.
\par 13 Ο δε είπε, Καλώς· εγώ θέλω κάμει συνθήκην μετά σού· πλην εν πράγμα ζητώ εγώ παρά σού· και είπε, Δεν θέλεις ιδεί το πρόσωπόν μου, εάν δεν φέρης έμπροσθέν μου Μιχάλ την θυγατέρα του Σαούλ, όταν έλθης να ίδης το πρόσωπόν μου.
\par 14 Και απέστειλεν ο Δαβίδ μηνυτάς προς τον Ις-βοσθέ, υιόν του Σαούλ λέγων, Απόδος την γυναίκα μου την Μιχάλ, την οποίαν ενυμφεύθην εις εμαυτόν διά εκατόν ακροβυστίας Φιλισταίων.
\par 15 Και έστειλεν ο Ις-βοσθέ και έλαβεν αυτήν παρά του ανδρός αυτής, παρά του Φαλτιήλ υιού του Λαείς.
\par 16 Και υπήγε μετ' αυτής ο ανήρ αυτής, πορευόμενος και κλαίων κατόπιν αυτής έως Βαουρείμ. Τότε είπε προς αυτόν ο Αβενήρ, Ύπαγε, επίστρεψον· και επέστρεψεν.
\par 17 Ο δε Αβενήρ συνωμίλησε μετά των πρεσβυτέρων του Ισραήλ, λέγων, Και χθές και προχθές εζητείτε τον Δαβίδ να βασιλεύση εφ' υμάς·
\par 18 τώρα λοιπόν κάμετε τούτο· διότι ο Κύριος ελάλησε περί του Δαβίδ, λέγων, Διά χειρός Δαβίδ του δούλου μου θέλω σώσει τον λαόν μου Ισραήλ εκ χειρός των Φιλισταίων και εκ χειρός πάντων των εχθρών αυτών.
\par 19 Και ελάλησε προσέτι ο Αβενήρ εις τα ώτα του Βενιαμίν· και υπήγεν ο Αβενήρ να λαλήση και εις τα ώτα του Δαβίδ εις Χεβρών, πάντα όσα ήσαν αρεστά εις τον Ισραήλ και εις πάντα τον οίκον του Βενιαμίν.
\par 20 Ήλθε λοιπόν ο Αβενήρ προς τον Δαβίδ εις Χεβρών, και μετ' αυτού είκοσι άνδρες. Και έκαμεν ο Δαβίδ εις τον Αβενήρ και εις τους άνδρας τους μετ' αυτού συμπόσιον.
\par 21 Και είπεν ο Αβενήρ προς τον Δαβίδ, Θέλω σηκωθή και υπάγει, και θέλω συνάξει πάντα τον Ισραήλ προς τον κύριόν μου τον βασιλέα, διά να κάμωσι συνθήκην μετά σου, και να βασιλεύης καθ' όλην την επιθυμίαν της ψυχής σου. Και απέστειλεν ο Δαβίδ τον Αβενήρ· και ανεχώρησεν εν ειρήνη.
\par 22 Και ιδού, οι δούλοι του Δαβίδ και ο Ιωάβ ήρχοντο από εκδρομής, και έφερον μεθ' εαυτών πολλά λάφυρα· αλλ' ο Αβενήρ δεν ήτο μετά του Δαβίδ εν Χεβρών, διότι είχεν αποστείλει αυτόν, και είχεν αναχωρήσει εν ειρήνη.
\par 23 Ότε δε ήλθεν ο Ιωάβ και άπαν το στράτευμα το μετ' αυτού, απήγγειλαν προς τον Ιωάβ, λέγοντες, Αβενήρ ο υιός του Νηρ ήλθε προς τον βασιλέα, και εξαπέστειλεν αυτόν και ανεχώρησεν εν ειρήνη.
\par 24 Τότε, εισήλθεν ο Ιωάβ προς τον βασιλέα και είπε, Τι έκαμες; ιδού, ο Αβενήρ ήλθε προς σέ· διά τι εξαπέστειλας αυτόν, και απήλθεν;
\par 25 εξεύρεις τον Αβενήρ τον υιόν του Νηρ, ότι ήλθε διά να σε απατήση και να μάθη την έξοδόν σου και την είσοδόν σου και να μάθη πάντα όσα συ πράττεις.
\par 26 Και καθώς εξήλθεν ο Ιωάβ από του Δαβίδ, έστειλε μηνυτάς κατόπιν του Αβενήρ, και επέστρεψαν αυτόν από του φρέατος Σιρά· ο Δαβίδ όμως δεν ήξευρε.
\par 27 Και ότε επέστρεψεν ο Αβενήρ εις Χεβρών, ο Ιωάβ παρεμέρισεν αυτόν εις τα πλάγια της πύλης, διά να λαλήση προς αυτόν μυστικά· και εκεί επάταξεν αυτόν υπό την πέμπτην πλευράν, και απέθανε, διά το αίμα Ασαήλ του αδελφού αυτού.
\par 28 Μετά δε ταύτα ακούσας ο Δαβίδ, είπεν, Αθώος είμαι εγώ και η βασιλεία μου, ενώπιον του Κυρίου εις τον αιώνα, από του αίματος του Αβενήρ, υιού του Νήρ·
\par 29 ας μένη επί την κεφαλήν του Ιωάβ και επί πάντα τον οίκον του πατρός αυτού· και ας μη εκλείψη από του οίκου του Ιωάβ γονόρροιος ή λεπρός ή επιστηριζόμενος επί βακτηρίαν ή πίπτων εν ρομφαία ή στερούμενος άρτου.
\par 30 Ούτως ο Ιωάβ και Αβισαί ο αδελφός αυτού εθανάτωσαν τον Αβενήρ, διότι είχε θανατώσει Ασαήλ τον αδελφόν αυτών εν Γαβαών εν τη μάχη.
\par 31 Και είπεν ο Δαβίδ προς τον Ιωάβ και προς πάντα τον λαόν τον μετ' αυτού, Διασχίσατε τα ιμάτιά σας και περιζώσθητε σάκκον και κλαύσατε έμπροσθεν του Αβενήρ. Και ο βασιλεύς Δαβίδ ηκολούθει το νεκροκράββατον.
\par 32 Και έθαψαν τον Αβενήρ εν Χεβρών· και ύψωσεν ο βασιλεύς την φωνήν αυτού και έκλαυσεν επί του τάφου του Αβενήρ· και πας ο λαός έκλαυσε.
\par 33 Και εθρήνησεν ο βασιλεύς επί τον Αβενήρ και είπεν, Απέθανεν ο Αβενήρ ως αποθνήσκει άφρων;
\par 34 αι χείρές σου δεν εδέθησαν, ουδέ οι πόδες σου ετέθησαν εν δεσμοίς· έπεσες, καθώς πίπτει τις έμπροσθεν των υιών της αδικίας. Και πας ο λαός έκλαυσε πάλιν επ' αυτόν.
\par 35 Ήλθεν έπειτα πας ο λαός διά να κάμωσι τον Δαβίδ να φάγη άρτον, ενώ ήτο έτι ημέρα· αλλ' ο Δαβίδ ώμοσε λέγων, Ούτω να κάμη ο Θεός εις εμέ και ούτω να προσθέση, εάν γευθώ άρτον ή άλλο τι, πριν δύση ο ήλιος.
\par 36 Και έμαθε τούτο πας ο λαός, και ήρεσεν εις αυτούς· καθώς ήρεσκεν εις πάντα τον λαόν ό,τι έκαμεν ο βασιλεύς.
\par 37 Διότι πας ο λαός και πας ο Ισραήλ εγνώρισαν την ημέραν εκείνην, ότι δεν ήτο από του βασιλέως το να θανατωθή Αβενήρ ο υιός του Νηρ.
\par 38 Και είπεν ο βασιλεύς προς τους δούλους αυτού, Δεν εξεύρετε ότι στρατηγός, και μέγας, έπεσε την ημέραν ταύτην εν τω Ισραήλ;
\par 39 εγώ δε είμαι την σήμερον αδύνατος, αν και εχρίσθην βασιλεύς· και ούτοι οι άνδρες οι υιοί της Σερουΐας παραπολύ δυνατοί ως προς εμέ· ο Κύριος θέλει κάμει ανταπόδοσιν εις τον εργάτην της κακίας κατά την κακίαν αυτού.

\chapter{4}

\par 1 Και ότε ήκουσεν ο υιός του Σαούλ ότι ο Αβενήρ απέθανεν εν Χεβρών, αι χείρες αυτού ενεκρώθησαν, και πάντες οι Ισραηλίται συνεταράχθησαν.
\par 2 Είχε δε ο υιός του Σαούλ δύο άνδρας, οίτινες ήσαν οπλαργηγοί, το όνομα του ενός Βαανά, και το όνομα του άλλου Ρηχάβ, υιοί Ριμμών του Βηρωθαίου, εκ των υιών Βενιαμίν· διότι και η Βηρώθ ελογίζετο του Βενιαμίν·
\par 3 οι δε Βηρωθαίοι είχον φύγει εις Γιτθαΐμ και ήσαν εκεί παροικούντες έως της ημέρας ταύτης.
\par 4 Ιωνάθαν δε, ο υιός του Σαούλ, είχεν υιόν βεβλαμμένον τους πόδας. Ήτο ηλικίας πέντε ετών ότε ήλθον αι αγγελίαι εξ Ιεζραήλ περί του Σαούλ και Ιωνάθαν, και εσήκωσεν αυτόν τροφός αυτού και έφυγεν· ενώ δε έσπευδε να φύγη, έπεσεν αυτός και εχωλώθη· το δε όνομα αυτού Μεμφιβοσθέ.
\par 5 Και υπήγαν οι υιοί Ριμμών του Βηρωθαίου, Ρηχάβ και Βαανά, και εις το καύμα της ημέρας εισήλθον εις την οικίαν του Ις-βοσθέ όστις εκοίτετο επί κλίνης το μεσημέριον·
\par 6 και εισήλθον εκεί έως του μέσου της οικίας, ως διά να λάβωσι σίτον· και εκτύπησαν αυτόν υπό την πέμπτην πλευράν· και ο Ρηχάβ και Βαανά ο αδελφός αυτού διεσώθησαν.
\par 7 Διότι ότε εισήλθον εις την οικίαν, εκείνος εκοίτετο επί της κλίνης αυτού εντός του κοιτώνος αυτού· και εκτύπησαν αυτόν και εθανάτωσαν αυτόν και απέκοψαν την κεφαλήν αυτού, και λαβόντες την κεφαλήν αυτού, ανεχώρησαν οδοιπορούντες διά της πεδιάδος όλην την νύκτα.
\par 8 Και έφερον την κεφαλήν του Ις-βοσθέ προς τον Δαβίδ εις Χεβρών και είπον προς τον βασιλέα, Ιδού, η κεφαλή του Ις-βοσθέ, υιού του Σαούλ του εχθρού σου, όστις εζήτει την ζωήν σου· και ο Κύριος έδωκεν εκδίκησιν εις τον κύριόν μου τον βασιλέα την ημέραν ταύτην, από του Σαούλ και από του σπέρματος αυτού.
\par 9 Απεκρίθη δε ο Δαβίδ προς τον Ρηχάβ και προς Βαανά τον αδελφόν αυτού, τους υιούς Ριμμών του Βηρωθαίου, και είπε προς αυτούς, Ζη Κύριος, όστις ελύτρωσε την ψυχήν μου από πάσης στενοχωρίας·
\par 10 εκείνον, όστις απήγγειλε προς εμέ, λέγων, Ιδού, απέθανεν ο Σαούλ, και εστοχάζετο εαυτόν μηνυτήν αγαθής αγγελίας, επίασα αυτόν και εθανάτωσα αυτόν εν Σικλάγ, αντί να βραβεύσω αυτόν διά την αγγελίαν αυτού·
\par 11 και πόσω μάλλον ανθρώπους πονηρούς, φονεύσαντας άνδρα δίκαιον εν τη οικία αυτού επί της κλίνης αυτού; τώρα λοιπόν δεν θέλω εκζητήσει το αίμα αυτού εκ των χειρών σας και δεν θέλω σας εξολοθρεύσει από της γης;
\par 12 Και προσέταξεν ο Δαβίδ τους νέους, και εθανάτωσαν αυτούς και έκοψαν τας χείρας αυτών και τους πόδας αυτών και εκρέμασαν αυτά επί το υδροστάσιον εν Χεβρών· την δε κεφαλήν του Ις-βοσθέ έλαβον, και έθαψαν εν τω τάφω του Αβενήρ εν Χεβρών.

\chapter{5}

\par 1 Και ήλθον πάσαι αι φυλαί του Ισραήλ προς τον Δαβίδ εις Χεβρών και είπον, λέγοντες, Ιδού, οστούν σου και σαρξ σου είμεθα ημείς·
\par 2 και πρότερον έτι, ότε ο Σαούλ εβασίλευεν εφ' ημάς, συ ήσο ο εξάγων και εισάγων τον Ισραήλ· και προς σε είπεν ο Κύριος, Συ θέλεις ποιμάνει τον λαόν μου τον Ισραήλ, και συ θέλεις είσθαι ηγεμών επί τον Ισραήλ.
\par 3 Και ήλθον πάντες οι πρεσβύτεροι του Ισραήλ προς τον βασιλέα εις Χεβρών· και έκαμεν ο βασιλεύς Δαβίδ συνθήκην μετ' αυτών εις Χεβρών ενώπιον του Κυρίου· και έχρισαν τον Δαβίδ βασιλέα επί τον Ισραήλ.
\par 4 Τριάκοντα ετών ήτο ο Δαβίδ ότε έγεινε βασιλεύς, και εβασίλευσε τεσσαράκοντα έτη·
\par 5 εν μεν Χεβρών εβασίλευσεν επί τον Ιούδαν επτά έτη και εξ μήνας· εν δε Ιερουσαλήμ εβασίλευσε τριάκοντα τρία έτη επί πάντα τον Ισραήλ και Ιούδαν.
\par 6 Και υπήγεν ο βασιλεύς και οι άνδρες αυτού εις Ιερουσαλήμ, προς τους Ιεβουσαίους, τους κατοικούντας την γήν· οίτινες ελάλησαν προς τον Δαβίδ, λέγοντες, Δεν θέλεις εισέλθει ενταύθα, εάν δεν εκβάλης τους τυφλούς και χωλούς· λέγοντες ότι ο Δαβίδ δεν ήθελε δυνηθή να εισέλθη εκεί.
\par 7 Ο Δαβίδ όμως εκυρίευσε το φρούριον Σιών· αύτη είναι η πόλις Δαβίδ.
\par 8 Και είπεν ο Δαβίδ την ημέραν εκείνην, Όστις φθάση εις τον οχετόν και πατάξη τους Ιεβουσαίους, και τους χωλούς και τους τυφλούς, τους μισουμένους υπό της ψυχής του Δαβίδ, θέλει είσθαι αρχηγός. Διά τούτο λέγουσι, Τυφλός και χωλός δεν θέλουσιν εισέλθει εις τον οίκον.
\par 9 Και κατώκησεν ο Δαβίδ εν τίνι φρουρίω και ωνόμασεν αυτό, Η πόλις Δαβίδ. Και έκαμεν ο Δαβίδ οικοδομάς κύκλω από Μιλλώ και έσω.
\par 10 Και προεχώρει ο Δαβίδ και εμεγαλύνετο, και Κύριος ο Θεός των δυνάμεων ήτο μετ' αυτού.
\par 11 Και απέστειλεν ο Χειράμ, βασιλεύς της Τύρου, πρέσβεις προς τον Δαβίδ, και ξύλα κέδρινα και ξυλουργούς και κτίστας, και ωκοδόμησαν οίκον εις τον Δαβίδ.
\par 12 Και εγνώρισεν ο Δαβίδ, ότι ο Κύριος κατέστησεν αυτόν βασιλέα επί τον Ισραήλ, και ότι ύψωσε την βασιλείαν αυτού διά τον λαόν αυτού Ισραήλ.
\par 13 Και έλαβε προσέτι ο Δαβίδ παλλακάς και γυναίκας εκ της Ιερουσαλήμ, αφού ήλθεν εκ Χεβρών· και εγεννήθησαν έτι εις τον Δαβίδ υιοί και θυγατέρες.
\par 14 ταύτα δε είναι τα ονόματα των εις αυτόν γεννηθέντων εν Ιερουσαλήμ· Σαμμουά και Σωβάβ και Νάθαν και Σολομών,
\par 15 και Ιεβάρ και Ελισουά και Νεφέγ και Ιαφιά,
\par 16 και Ελισαμά και Ελιαδά και Ελιφαλέτ.
\par 17 Ότε δε ήκουσαν οι Φιλισταίοι ότι έχρισαν τον Δαβίδ βασιλέα επί τον Ισραήλ, ανέβησαν πάντες οι Φιλισταίοι να ζητήσωσι τον Δαβίδ· και ο Δαβίδ ήκουσε περί τούτου και κατέβη εις το φρούριον.
\par 18 Και ήλθον οι Φιλισταίοι και διεχύθησαν εις την κοιλάδα Ραφαείμ.
\par 19 Και ερώτησεν ο Δαβίδ τον Κύριον, λέγων, Να αναβώ προς τους Φιλισταίους; θέλεις παραδώσει αυτούς εις την χειρά μου; Και είπεν ο Κύριος προς τον Δαβίδ, Ανάβα· διότι βεβαίως θέλω παραδώσει τους Φιλισταίους εις την χείρα σου.
\par 20 Και ήλθεν ο Δαβίδ εις Βάαλ-φερασείμ, και εκεί επάταξεν αυτούς ο Δαβίδ και είπεν, Ο Κύριος διέκοψε τους εχθρούς μου έμπροσθέν μου, καθώς διακόπτονται τα ύδατα. Διά τούτο εκαλέσθη το όνομα του τόπου εκείνου Βάαλ-φερασείμ.
\par 21 Και εκεί κατέλιπον τα είδωλα αύτών, και εσήκωσαν αυτά ο Δαβίδ και οι άνδρες αυτού.
\par 22 Και ανέβησαν πάλιν οι Φιλισταίοι και διεχύθησαν εις την κοιλάδα Ραφαείμ.
\par 23 Και ότε ηρώτησεν ο Δαβίδ τον Κύριον, είπε, Μη αναβής· στρέψον οπίσω αυτών και επίπεσον επ' αυτούς απέναντι των συκαμίνων.
\par 24 και όταν ακούσης θόρυβον διαβάσεως επί των κορυφών των συκαμίνων, τότε θέλεις σπεύσει διότι τότε ο Κύριος θέλει εξέλθει έμπροσθέν σου, διά να πατάξη το στρατόπεδον των Φιλισταίων.
\par 25 Και έκαμεν ο Δαβίδ, καθώς προσέταξεν εις αυτόν ο Κύριος· και επάταξε τους Φιλισταίους από Γαβαά έως της εισόδου Γεζέρ.

\chapter{6}

\par 1 Και πάλιν συνήθροισεν ο Δαβίδ πάντας τους εκλεκτούς εκ του Ισραήλ, τριάκοντα χιλιάδας.
\par 2 Και εσηκώθη ο Δαβίδ και υπήγε, και πας ο λαός ο μετ' αυτού, από Βάαλ του Ιούδα, διά να αναγάγη εκείθεν την κιβωτόν του Θεού, εις την οποίαν επικαλείται το Όνομα, το όνομα του Κυρίου των δυνάμεων, του καθημένου υπεράνω αυτής επί των χερουβείμ.
\par 3 Και επεβίβασαν την κιβωτόν του Θεού επί νέας αμάξης και εσήκωσαν αυτήν εκ του οίκου του Αβιναδάβ, του εν τω βουνώ· ώδήγησαν δε την άμαξαν την νέαν ο Ουζά και Αχιώ, υιοί του Αβιναδάβ.
\par 4 Και εσήκωσαν αυτήν από του οίκου του Αβιναδάβ, του εν τω βουνώ, μετά της κιβωτού του Θεού· και ο Αχιώ προεπορεύετο της κιβωτού.
\par 5 Ο δε Δαβίδ και πας ο οίκος του Ισραήλ έπαιζον έμπροσθεν του Κυρίου παν είδος οργάνων από ξύλου ελάτης και κιθάρας και ψαλτήρια και τύμπανα και σείστρα και κύμβαλα.
\par 6 Και ότε ήλθον έως του αλωνίου του Ναχών, εξήπλωσεν ο Ουζά την χείρα αυτού εις την κιβωτόν του Θεού και εκράτησεν αυτήν· διότι έσεισαν αυτήν οι βόες.
\par 7 Και εξήφθη ο θυμός του Κυρίου κατά του Ουζά· και επάταξεν αυτόν ο Θεός εκεί διά την προπέτειαν αυτού· και απέθανεν εκεί παρά την κιβωτόν του Θεού.
\par 8 Και ελυπήθη ο Δαβίδ, ότι ο Κύριος έκαμε χαλασμόν εις τον Ουζά· και εκάλεσε το όνομα του τόπου Φαρές-ουζά, έως της ημέρας ταύτης.
\par 9 Και εφοβήθη ο Δαβίδ τον Κύριον την ημέραν εκείνην και είπε, πως θέλει εισέλθει προς εμέ η κιβωτός του Κυρίου;
\par 10 Και δεν ηθέλησεν ο Δαβίδ να μετακινήση την κιβωτόν του Κυρίου προς εαυτόν εις την πόλιν Δαβίδ, αλλ' έστρεψεν αυτήν ο Δαβίδ εις τον οίκον Ωβήδ-εδώμ του Γετθαίου.
\par 11 Και εκάθησεν η κιβωτός του Κυρίου εν τω οίκω Ωβήδ-εδώμ του Γετθαίου τρεις μήνας· και ευλόγησεν ο Κύριος τον Ωβήδ-εδώμ και πάντα τον οίκον αυτού.
\par 12 Και απήγγειλαν προς τον βασιλέα Δαβίδ, λέγοντες, Ο Κύριος ευλόγησε τον οίκον του Ωβήδ-εδώμ και πάντα τα υπάρχοντα αυτού ένεκα της κιβωτού του Θεού. Τότε υπήγεν ο Δαβίδ και ανεβίβασε την κιβωτόν του Θεού εκ του οίκου του Ωβήδ-εδώμ εις την πόλιν Δαβίδ εν ευφροσύνη.
\par 13 Και ότε εβάδιζον οι βαστάζοντες την κιβωτόν του Κυρίου εξ βήματα, εθυσίαζον βουν και σιτευτόν.
\par 14 Και εχόρευεν ο Δαβίδ ενώπιον του Κυρίου εξ όλης δυνάμεως· και ήτο ο Δαβίδ περιεζωσμένος λινούν εφόδ.
\par 15 Και ο Δαβίδ και πας ο οίκος Ισραήλ ανεβίβασαν την κιβωτόν του Κυρίου εν αλαλαγμώ και εν φωνή σάλπιγγος.
\par 16 Ενώ δε η κιβωτός του Κυρίου εισήρχετο εις την πόλιν Δαβίδ, Μιχάλ, η θυγάτηρ του Σαούλ, έκυψε διά της θυρίδος, και ιδούσα τον βασιλέα Δαβίδ ορχούμενον και χορεύοντα ενώπιον του Κυρίου, εξουδένωσεν αυτόν εν τη καρδία αυτής.
\par 17 Και έφεραν την κιβωτόν του Κυρίου και έθεσαν αυτήν εις τον τόπον αυτής, εις το μέσον της σκηνής την οποίαν έστησε δι' αυτήν ο Δαβίδ· και προσέφερεν ο Δαβίδ ολοκαυτώματα και ειρηνικάς προσφοράς ενώπιον του Κυρίου.
\par 18 Και αφού ετελείωσεν ο Δαβίδ προσφέρων τα ολοκαυτώματα και τας ειρηνικάς προσφοράς, ευλόγησε τον λαόν εν ονόματι του Κυρίου των δυνάμεων.
\par 19 Και διεμοίρασεν εις πάντα τον λαόν, εις άπαν το πλήθος του Ισραήλ, από ανδρός έως γυναικός, εις έκαστον άνθρωπον εν ψωμίον και εν τμήμα κρέατος και μίαν φιάλην οίνου. Τότε πας ο λαός ανεχώρησεν, έκαστος εις την οικίαν αυτού.
\par 20 Και επέστρεψεν ο Δαβίδ διά να ευλογήση τον οίκον αυτού. Και εξελθούσα Μιχάλ, η θυγάτηρ του Σαούλ, εις συνάντησιν του Δαβίδ, είπε, Πόσον ένδοξος ήτο σήμερον ο βασιλεύς του Ισραήλ, όστις εγυμνώθη σήμερον εις τους οφθαλμούς των θεραπαινίδων των δούλων αυτού, καθώς γυμνόνεται αναισχύντως εις των μηδαμινών ανθρώπων.
\par 21 Και είπεν ο Δαβίδ προς την Μιχάλ, Ενώπιον του Κυρίου, όστις με εξέλεξεν υπέρ τον πατέρα σου και υπέρ πάντα τον οίκον αυτού, ώστε να με καταστήση ηγεμόνα επί τον λαόν του Κυρίου, επί τον Ισραήλ, ναι, ενώπιον του Κυρίου έπαιξα·
\par 22 και θέλω εξευτελισθή έτι περισσότερον και θέλω ταπεινωθή εις τους οφθαλμούς μου· και μετά των θεραπαινίδων, περί των οποίων συ ελάλησας, μετ' αυτών θέλω δοξασθή.
\par 23 Διά τούτο η Μιχάλ, η θυγάτηρ του Σαούλ, δεν εγέννησε τέκνον έως της ημέρας του θανάτου αυτής.

\chapter{7}

\par 1 Αφού δε εκάθησεν ο βασιλεύς εν τω οίκω αυτού και ανέπαυσεν αυτόν ο Κύριος πανταχόθεν από πάντων των εχθρών αυτού·
\par 2 είπεν ο βασιλεύς προς Νάθαν τον προφήτην, Ιδέ τώρα, εγώ κατοικώ εν οίκω κεδρίνω· η δε κιβωτός του Θεού κάθηται εν μέσω παραπετασμάτων.
\par 3 Και είπεν ο Νάθαν προς τον βασιλέα, Ύπαγε, κάμε παν το εν καρδία σου· διότι ο Κύριος είναι μετά σου.
\par 4 Και την νύκτα εκείνην έγεινε λόγος του Κυρίου προς τον Νάθαν, λέγων,
\par 5 Ύπαγε και ειπέ προς τον δούλον μου τον Δαβίδ, Ούτω λέγει Κύριος· συ θέλεις οικοδομήσει εις εμέ οίκον, διά να κατοικώ;
\par 6 Διότι δεν κατώκησα εν οίκω, αφ' ης ημέρας ανεβίβασα τους υιούς Ισραήλ εξ Αιγύπτου μέχρι της ημέρας ταύτης, αλλά περιηρχόμην εντός σκηνής και παραπετασμάτων.
\par 7 Πανταχού όπου περιεπάτησα μετά πάντων των υιών Ισραήλ, ελάλησα ποτέ προς τινά εκ των φυλών του Ισραήλ, εις τον οποίον προσέταξα να ποιμαίνη τον λαόν μου τον Ισραήλ, λέγων, Διά τι δεν ωκοδομήσατε εις εμέ οίκον κέδρινον;
\par 8 Τώρα λοιπόν, ούτω θέλεις ειπεί προς τον δούλον μου τον Δαβίδ· Ούτω λέγει ο Κύριος των δυνάμεων· Εγώ σε έλαβον εκ της μάνδρας, από όπισθεν των προβάτων, διά να ήσαι ηγεμών επί τον λαόν μου, επί τον Ισραήλ·
\par 9 και ήμην μετά σου πανταχού όπου περιεπάτησας, και εξωλόθρευσα πάντας τους εχθρούς σου απ' έμπροσθέν σου, και σε έκαμον ονομαστόν, κατά το όνομα των μεγάλων των επί της γής·
\par 10 και θέλω διορίσει τόπον διά τον λαόν μου τον Ισραήλ και θέλω φυτεύσει αυτούς, και θέλουσι κατοικεί εν τόπω ιδίω εαυτών και δεν θέλουσι μεταφέρεσθαι πλέον· και οι υιοί της αδικίας δεν θέλουσι καταθλίβει αυτούς πλέον ως το πρότερον,
\par 11 και ως από των ημερών καθ' ας κατέστησα κριτάς επί τον λαόν μου Ισραήλ· και θέλω σε αναπαύσει από πάντων των εχθρών σου. Ο Κύριος προσέτι αναγγέλλει προς σε ότι ο Κύριος θέλει οικοδομήσει οίκον εις σε.
\par 12 Αφού πληρωθώσιν αι ημέραι σου και κοιμηθής μετά των πατέρων σου, θέλω αναστήσει μετά σε το σπέρμα σου, το οποίον θέλει εξέλθει εκ των σπλάγχνων σου, και θέλω στερεώσει την βασιλείαν αυτού.
\par 13 αυτός θέλει οικοδομήσει οίκον εις το όνομα μου· και θέλω στερεώσει τον θρόνον της βασιλείας αυτού έως αιώνος·
\par 14 εγώ θέλω είσθαι εις αυτόν πατήρ και αυτός θέλει είσθαι εις εμέ υιός· εάν πράξη ανομίαν, θέλω σωφρονίσει αυτόν εν ράβδω ανδρών και διά μαστιγώσεων υιών ανθρώπων·
\par 15 το έλεός μου όμως δεν θέλει αφαιρεθή απ' αυτού, ως αφήρεσα αυτό από του Σαούλ, τον οποίον εξέβαλον απ' έμπροσθέν σου·
\par 16 και θέλει στερεωθή ο οίκός σου και η βασιλεία σου έμπροσθέν σου έως αιώνος· ο θρόνος σου θέλει είσθαι εστερεωμένος εις τον αιώνα.
\par 17 Κατά πάντας τους λόγους τούτους και καθ' όλην ταύτην την όρασιν, ούτως ελάλησεν ο Νάθαν προς τον Δαβίδ.
\par 18 Τότε εισήλθεν ο βασιλεύς Δαβίδ και εκάθησεν ενώπιον του Κυρίου και είπε, Τις είμαι εγώ, Κύριε Θεέ; και τις ο οίκός μου, ώστε με έφερες μέχρι τούτου.
\par 19 Αλλά και τούτο έτι εστάθη μικρόν εις τους οφθαλμούς σου, Κύριε Θεέ· και ελάλησας έτι περί του οίκου του δούλου σου διά μέλλον μακρόν. Και είναι ούτος ο τρόπος των ανθρώπων, Δέσποτα Κύριε;
\par 20 Και τι δύναται να είπη πλέον ο Δαβίδ προς σε; διότι συ γνωρίζεις τον δούλον σου, Δέσποτα Κύριε.
\par 21 Διά τον λόγον σου και κατά την καρδίαν σου έπραξας πάντα ταύτα τα μεγαλεία, διά να γνωστοποιήσης αυτά εις τον δούλον σου.
\par 22 Διά τούτο μέγας είσαι, Κύριε Θεέ· διότι δεν είναι όμοιός σου· ουδέ είναι Θεός εκτός σου, κατά πάντα όσα ηκούσαμεν με τα ώτα ημών.
\par 23 Και τι άλλο έθνος επί της γης είναι ως ο λαός σου, ως ο Ισραήλ, τον οποίον ο Θεός ήλθε να εξαγοράση διά λαόν εαυτού και διά να κάμη αυτόν ονομαστόν, και να ενεργήση υπέρ υμών πράγματα μεγάλα και θαυμαστά, υπέρ της γης σου, έμπροσθεν του λαού σου, τον οποίον ελύτρωσας διά σεαυτόν εξ Αιγύπτου, εκ των εθνών και εκ των θεών αυτών;
\par 24 Διότι εστερέωσας εις σεαυτόν τον λαόν σου Ισραήλ, διά να ήναι λαός σου εις τον αιώνα· και συ, Κύριε, έγεινες Θεός αυτών.
\par 25 και τώρα, Κύριε Θεέ, τον λόγον τον οποίον ελάλησας περί του δούλου σου και περί του οίκου αυτού ας στερεωθή ας τον αιώνα, και κάμε ως ελάλησας.
\par 26 Και ας μεγαλυνθή το όνομά σου έως αιώνος, ώστε να λέγωσιν, Ο Κύριος των δυνάμεων είναι ο Θεός επί τον Ισραήλ· και ο οίκος του δούλου σου Δαβίδ ας ήναι εστερεωμένος ενώπιόν σου.
\par 27 Διότι συ, Κύριε των δυνάμεων, Θεέ του Ισραήλ, απεκάλυψας εις τον δούλον σου, λέγων, Οίκον θέλω οικοδομήσει εις σέ· διά τούτο ο δούλός σου εύρηκε την καρδίαν αυτού ετοίμην να προσευχηθή προς σε την προσευχήν ταύτην.
\par 28 Και τώρα, Δέσποτα Κύριε, συ είσαι ο Θεός, και οι λόγοι σου θέλουσιν είσθαι αληθινοί, και συ υπεσχέθης τα αγαθά ταύτα προς τον δούλον σου·
\par 29 τώρα λοιπόν ευδόκησον να ευλογήσης τον οίκον του δούλου σου, διά να ήναι ενώπιόν σου εις τον αιώνα· διότι συ, Δέσποτα Κύριε, ελάλησας· και υπό της ευλογίας σου ας ήναι ο οίκος του δούλου σου ευλογημένος εις τον αιώνα.

\chapter{8}

\par 1 Μετά δε ταύτα επάταξεν ο Δαβίδ τους Φιλισταίους και κατετρόπωσεν αυτούς· και έλαβεν ο Δαβίδ την Μεθέγ-αμμά εκ χειρός των Φιλισταίων.
\par 2 Και επάταξε τους Μωαβίτας και διεμέτρησεν αυτούς διά σχοινίων, απλώσας αυτούς κατά γής· και διεμέτρησε διά δύο σχοινίων διά να θανατώση, και δι' ενός πλήρους σχοινίου διά να αφήση ζώντας. Ούτως οι Μωαβίται έγειναν δούλοι του Δαβίδ υποτελείς.
\par 3 Επάταξεν έτι ο Δαβίδ τον Αδαδεζέρ, υιόν του Ρεώβ, βασιλέα της Σωβά, ενώ υπήγαινε να στήση την εξουσίαν αυτού επί τον ποταμόν Ευφράτην.
\par 4 Και έλαβεν ο Δαβίδ εξ αυτού χιλίους επτακοσίους ιππείς και είκοσι χιλιάδας πεζών· και ενευροκόπησεν ο Δαβίδ πάντας τους ίππους των αμαξών, και εφύλαξεν εξ αυτών εκατόν αμάξας.
\par 5 Και ότε ήλθον οι Σύριοι της Δαμασκού διά να βοηθήσωσι τον Αδαδέζερ, βασιλέα της Σωβά, ο Δαβίδ επάταξεν εκ των Συρίων εικοσιδύο χιλιάδας ανδρών.
\par 6 Και έβαλεν ο Δαβίδ φρουράς εν τη Συρία της Δαμασκού· και οι Σύριοι έγειναν δούλοι υποτελείς του Δαβίδ. Και έσωζεν ο Κύριος τον Δαβίδ πανταχού, όπου επορεύετο.
\par 7 Και έλαβεν ο Δαβίδ τας ασπίδας τας χρυσάς, αίτινες ήσαν επί τους δούλους του Αδαδέζερ, και έφερεν αυτάς εις Ιερουσαλήμ.
\par 8 Και εκ της Βετάχ και εκ Βηρωθάϊ, πόλεων του Αδαδέζερ, ο βασιλεύς Δαβίδ έλαβε χαλκόν πολύν σφόδρα.
\par 9 Ακούσας δε ο Θοεί, βασιλεύς της Αιμάθ, ότι ο Δαβίδ επάταξε πάσαν την δύναμιν του Αδαδέζερ,
\par 10 απέστειλεν ο Θοεί Ιωράμ, τον υιόν αυτού, προς τον βασιλέα Δαβίδ, διά να χαιρετήση αυτόν και να ευλογήση αυτόν, ότι κατεπολέμησε τον Αδαδέζερ και επάταξεν αυτόν· διότι ο Αδαδέζερ ήτο πολέμιος του Θοεί. Και έφερεν ο Ιωράμ μεθ' εαυτού σκεύη αργυρά και σκεύη χρυσά και σκεύη χάλκινα·
\par 11 και ταύτα αφιέρωσεν ο βασιλεύς Δαβίδ εις τον Κύριον μετά του αργυρίου και του χρυσίου, τα οποία είχεν αφιερώσει εκ πάντων των εθνών, όσα υπέταξεν·
\par 12 εκ της Συρίας και εκ του Μωάβ και εκ των υιών Αμμών και εκ των Φιλισταίων και εκ του Αμαλήκ και εκ των λαφύρων του Αδαδέζερ, υιού του Ρεώβ, βασιλέως της Σωβά.
\par 13 Και απέκτησεν ο Δαβίδ όνομα, ότε επέστρεφε, κατατροπώσας τους Συρίους εν τη κοιλάδι του άλατος, δεκαοκτώ χιλιάδας.
\par 14 Και έβαλε φρουράς εν τη Ιδουμαία· καθ' όλην την Ιδουμαίαν έβαλε φρουράς· και πάντες οι Ιδουμαίοι έγειναν δούλοι του Δαβίδ. Και έσωζεν ο Κύριος τον Δαβίδ πανταχού όπου επορεύετο.
\par 15 Και εβασίλευσεν ο Δαβίδ επί πάντα τον Ισραήλ· και έκαμεν ο Δαβίδ κρίσιν και δικαιοσύνην εις πάντα τον λαόν αυτού.
\par 16 Και Ιωάβ, ο υιός της Σερουΐας, ήτο επί του στρατεύματος· Ιωσαφάτ δε, ο υιός του Αχιλούδ, υπομνηματογράφος·
\par 17 και Σαδώκ, ο υιός του Αχιτώβ, και Αχιμέλεχ, ο υιός του Αβιάθαρ, ιερείς· ο δε Σεραΐας, γραμματεύς.
\par 18 Και Βεναΐας, ο υιός του Ιωδαέ, ήτο επί των Χερεθαίων και επί των Φελεθαίων· οι δε υιοί του Δαβίδ ήσαν αυλάρχαι.

\chapter{9}

\par 1 Και είπεν ο Δαβίδ, Μένει τις έτι εκ του οίκου του Σαούλ, διά να κάμω έλεος προς αυτόν χάριν του Ιωνάθαν;
\par 2 Ήτο δε δούλός τις εκ του οίκου του Σαούλ, ονομαζόμενος Σιβά. Και εκάλεσαν αυτόν προς τον Δαβίδ, και είπε προς αυτόν ο βασιλεύς, Συ είσαι ο Σιβά; Ο δε είπεν, Ο δούλός σου.
\par 3 Και είπεν ο βασιλεύς, Δεν μένει τις έτι εκ του οίκου του Σαούλ, διά να κάμω προς αυτόν έλεος Θεού; Και είπεν ο Σιβά προς τον βασιλέα, Έτι υπάρχει υιός του Ιωνάθαν, βεβλαμμένος τους πόδας.
\par 4 Και είπε προς αυτόν ο βασιλεύς, Που είναι ούτος; Ο δε Σιβά είπε προς τον βασιλέα, Ιδού, είναι εν τω οίκω του Μαχείρ, υιού του Αμμιήλ, εν Λό-δεβάρ.
\par 5 Τότε έστειλεν ο βασιλεύς Δαβίδ και έλαβεν αυτόν εκ του οίκου του Μαχείρ, υιού του Αμμιήλ, εκ Λό-δεβάρ.
\par 6 Και ότε ήλθε προς τον Δαβίδ ο Μεμφιβοσθέ, υιός του Ιωνάθαν, υιού του Σαούλ, έπεσε κατά πρόσωπον αυτού και προσεκύνησε. Και είπεν ο Δαβίδ, Μεμφιβοσθέ· Ο δε είπεν, Ιδού, ο δούλός σου.
\par 7 Και είπεν ο Δαβίδ προς αυτόν, Μη φοβού· διότι βεβαίως θέλω κάμει προς σε έλεος, χάριν Ιωνάθαν του πατρός σου, και θέλω αποδώσει εις σε πάντα τα κτήματα Σαούλ του πατρός σου· και συ θέλεις τρώγει άρτον επί της τραπέζης μου διά παντός.
\par 8 Ο δε προσεκύνησεν αυτόν και είπε, Τις είναι ο δούλός σου, ώστε να επιβλέψης εις τοιούτον κύνα τεθνηκότα οποίος εγώ;
\par 9 Και εκάλεσεν ο βασιλεύς τον Σιβά, τον δούλον του Σαούλ, και είπε προς αυτόν, Πάντα όσα είχεν ο Σαούλ και πας ο οίκος αυτού έδωκα εις τον υιόν του κυρίου σου·
\par 10 θέλεις λοιπόν γεωργεί την γην δι' αυτόν, συ και οι υιοί σου, και οι δούλοί σου, και θέλεις φέρει τα εισοδήματα, διά να έχη ο υιός του κυρίου σου τροφήν να τρώγη· πλην ο Μεμφιβοσθέ, ο υιός του κυρίου σου, θέλει τρώγει διά παντός άρτον επί της τραπέζης μου. Είχε δε ο Σιβά δεκαπέντε υιούς και είκοσι δούλους.
\par 11 Ο δε Σιβά είπε προς τον βασιλέα, Κατά πάντα όσα προσέταξεν ο κύριός μου ο βασιλεύς τον δούλον αυτού, ούτω θέλει κάμει ο δούλός σου. Ο δε Μεμφιβοσθέ, είπεν ο βασιλεύς, θέλει τρώγει επί της τραπέζης μου, ως εις των υιών του βασιλέως.
\par 12 Είχε δε ο Μεμφιβοσθέ υιόν μικρόν, ονομαζόμενον Μιχά. Πάντες δε οι κατοικούντες εν τω οίκω του Σιβά ήσαν δούλοι του Μεμφιβοσθέ.
\par 13 Και ο Μεμφιβοσθέ κατώκει εν Ιερουσαλήμ· διότι έτρωγε διά παντός επί της τραπέζης του βασιλέως· ήτο δε χωλός αμφοτέρους τους πόδας.

\chapter{10}

\par 1 Μετά δε ταύτα απέθανεν ο βασιλεύς των υιών Αμμών, και εβασίλευσεν αντ' αυτού Ανούν ο υιός αυτού.
\par 2 Και είπεν ο Δαβίδ, Θέλω κάμει έλεος προς Ανούν, τον υιόν του Ναάς, επειδή ο πατήρ αυτού έκαμεν έλεος προς εμέ. Και απέστειλεν ο Δαβίδ να παρηγορήση αυτόν περί του πατρός αυτού διά χειρός των δούλων αυτού. Και ήλθον οι δούλοι του Δαβίδ εις την γην των υιών Αμμών.
\par 3 Και είπον οι άρχοντες των υιών Αμμών προς Ανούν τον κύριον αυτών, Νομίζεις ότι ο Δαβίδ τιμών τον πατέρα σου απέστειλε προς σε παρηγορητάς; δεν απέστειλεν ο Δαβίδ τους δούλους αυτού προς σε, διά να ερευνήση την πόλιν και να κατασκοπεύση αυτήν και να καταστρέψη αυτήν;
\par 4 Και επίασεν ο Ανούν τους δούλους του Δαβίδ, και εξύρισε το ήμισυ του πώγωνος αυτών και απέκοψε το ήμισυ των ιματίων αυτών μέχρι των γλουτών αυτών, και απέπεμψεν αυτούς.
\par 5 Ότε απήγγειλαν τούτο προς τον Δαβίδ, απέστειλεν εις συνάντησιν αυτών, επειδή οι άνδρες ήσαν ητιμασμένοι σφόδρα· και είπεν ο βασιλεύς, Καθήσατε εν Ιεριχώ εωσού αυξηνθώσιν οι πώγωνές σας, και επιστρέψατε.
\par 6 Βλέποντες δε οι υιοί Αμμών ότι ήσαν βδελυκτοί εις τον Δαβίδ, απέστειλαν οι υιοί Αμμών και εμίσθωσαν εκ των Συρίων Βαιθ-ρεώβ και των Συρίων Σωβά είκοσι χιλιάδας πεζών, και παρά του βασιλέως Μααχά χιλίους άνδρας, και παρά του Ις-τωβ δώδεκα χιλιάδας ανδρών.
\par 7 Και ότε ήκουσε ταύτα ο Δαβίδ, απέστειλε τον Ιωάβ και άπαν το στράτευμα των δυνατών.
\par 8 Και εξήλθον οι υιοί Αμμών και παρετάχθησαν εις πόλεμον κατά την είσοδον της πύλης· και οι Σύριοι Σωβά και Ρεώβ και Ις-τωβ και Μααχά ήσαν καθ' εαυτούς εν τη πεδιάδι.
\par 9 Βλέπων δε ο Ιωάβ ότι η μάχη παρετάχθη εναντίον αυτού έμπροσθεν και όπισθεν, εξέλεξεν εκ πάντων των εκλεκτών του Ισραήλ και παρέταξεν αυτούς εναντίον των Συρίων·
\par 10 το δε υπόλοιπον του λαού έδωκεν εις την χείρα του Αβισαί, αδελφού αυτού, και παρέταξεν αυτούς εναντίον των υιών Αμμών.
\par 11 Και είπεν, Εάν οι Σύριοι υπερισχύσωσι κατ' εμού, τότε συ θέλεις με σώσει εάν δε οι υιοί Αμμών υπερισχύσωσι κατά σου, τότε εγώ θέλω ελθεί διά να σε σώσω·
\par 12 ανδρίζου, και ας κραταιωθώμεν υπέρ του λαού ημών και υπέρ των πόλεων του Θεού ημών· ο δε Κύριος ας κάμη το αρεστόν εις τους οφθαλμούς αυτού.
\par 13 Και προσήλθεν ο Ιωάβ και ο λαός ο μετ' αυτού εις μάχην εναντίον των Συρίων· οι δε έφυγον απ' έμπροσθεν αυτού.
\par 14 Και ότε είδον οι υιοί Αμμών ότι οι Σύριοι έφυγον, τότε έφυγον και αυτοί απ' έμπροσθεν του Αβισαί και εισήλθον εις την πόλιν. Και ο Ιωάβ επέστρεψεν από των υιών Αμμών και ήλθεν εις Ιερουσαλήμ.
\par 15 Ιδόντες δε οι Σύριοι ότι κατετροπώθησαν έμπροσθεν του Ισραήλ, συνηθροίσθησαν ομού.
\par 16 Και απέστειλεν ο Αδαρέζερ και εξήγαγε τους Συρίους τους πέραν του ποταμού· και ήλθον εις Αιλάμ· και Σωβάκ, ο αρχιστράτηγος του Αδαρέζερ, προεπορεύετο έμπροσθεν αυτών.
\par 17 Και ότε απηγγέλθη προς τον Δαβίδ, συνήθροισε πάντα τον Ισραήλ και διέβη τον Ιορδάνην και ήλθεν εις Αιλάμ. Οι δε Σύριοι παρετάχθησαν εναντίον του Δαβίδ και επολέμησαν με αυτόν.
\par 18 Και έφυγον οι Σύριοι απ' έμπροσθεν του Ισραήλ· και εξωλόθρευσεν ο Δαβίδ εκ των Συρίων επτακοσίας αμάξας και τεσσαράκοντα χιλιάδας ιππέων, και Σωβάκ τον αρχιστράτηγον αυτών επάταξε, και απέθανεν εκεί.
\par 19 Και ιδόντες πάντες οι βασιλείς, οι δούλοι του Αδαρέζερ, ότι κατετροπώθησαν έμπροσθεν του Ισραήλ, έκαμον ειρήνην μετά του Ισραήλ και έγειναν δούλοι αυτών. Και οι Σύριοι εφοβούντο να βοηθήσωσι πλέον τους υιούς Αμμών.

\chapter{11}

\par 1 Εν δε τω ακολούθω έτει, καθ' ον καιρόν εκστρατεύουσιν οι βασιλείς, απέστειλεν ο Δαβίδ τον Ιωάβ και τους δούλους αυτού μετ' αυτού και πάντα τον Ισραήλ· και κατέστρεψαν τους υιούς Αμμών και επολιόρκησαν την Ραββά. Ο δε Δαβίδ έμεινεν εν Ιερουσαλήμ.
\par 2 Και προς το εσπέρας, ότε ο Δαβίδ εσηκώθη από της κλίνης αυτού, περιεπάτει επί του δώματος του βασιλικού οίκου· και είδεν από του δώματος γυναίκα λουομένην· και η γυνή ήτο ώραία την όψιν σφόδρα.
\par 3 Και απέστειλεν ο Δαβίδ και ηρεύνησε περί της γυναικός. Και είπε τις, Δεν είναι αύτη Βηθ-σαβεέ, η θυγάτηρ του Ελιάμ, η γυνή Ουρίου του Χετταίου;
\par 4 Και απέστειλεν ο Δαβίδ μηνυτάς και έλαβεν αυτήν· και ότε ήλθε προς αυτόν, εκοιμήθη μετ' αυτής, διότι είχε καθαρισθή από της ακαθαρσίας αυτής· και επέστρεψεν εις τον οίκον αυτής.
\par 5 Και συνέλαβεν η γυνή· και αποστείλασα απήγγειλε προς τον Δαβίδ και είπεν, Έγκυος είμαι.
\par 6 Και απέστειλεν ο Δαβίδ προς τον Ιωάβ, λέγων, Απόστειλόν μοι Ουρίαν τον Χετταίον. Και απέστειλεν ο Ιωάβ τον Ουρίαν προς τον Δαβίδ.
\par 7 Και ότε ήλθε προς αυτόν ο Ουρίας, ηρώτησεν ο Δαβίδ πως έχει ο Ιωάβ και πως έχει ο λαός και πως έχουσι τα του πολέμου.
\par 8 Και είπεν ο Δαβίδ προς τον Ουρίαν, Κατάβα εις τον οίκόν σου και νίψον τους πόδας σου· και εξήλθεν ο Ουρίας εκ του οίκου του βασιλέως· και κατόπιν αυτού ήλθε μερίδιον από της τραπέζης του βασιλέως.
\par 9 Αλλ' ο Ουρίας εκοιμήθη παρά την θύραν του οίκου του βασιλέως, μετά πάντων των δούλων του κυρίου αυτού και δεν κατέβη εις τον οίκον αυτού.
\par 10 Και ότε απήγγειλαν προς τον Δαβίδ, λέγοντες, Δεν κατέβη ο Ουρίας εις τον οίκον αυτού, είπεν ο Δαβίδ προς τον Ουρίαν, Συ δεν έρχεσαι εξ οδοιπορίας; διά τι δεν κατέβης εις τον οίκόν σου;
\par 11 Και είπεν ο Ουρίας προς τον Δαβίδ, Η κιβωτός και ο Ισραήλ και ο Ιούδας κατοικούσιν εν σκηναίς, και ο κύριός μου Ιωάβ και οι δούλοι του κυρίου μου, είναι εστρατοπεδευμένοι επί το πρόσωπον της πεδιάδος· και εγώ θέλω υπάγει εις τον οίκόν μου, διά να φάγω και να πίω και να κοιμηθώ μετά της γυναικός μου; ζης και ζη η ψυχή σου, δεν θέλω κάμει το πράγμα τούτο.
\par 12 Και είπεν ο Δαβίδ προς τον Ουρίαν, Μείνε ενταύθα και σήμερον, και αύριον θέλω σε εξαποστείλει. Και έμεινεν ο Ουρίας εν Ιερουσαλήμ την ημέραν εκείνην και την επαύριον.
\par 13 Και εκάλεσεν αυτόν ο Δαβίδ, και έφαγεν ενώπιον αυτού και έπιεν· και εμέθυσεν αυτόν· και το εσπέρας εξήλθε να κοιμηθή επί της κλίνης αυτού μετά των δούλων του κυρίου αυτού, πλην εις τον οίκον αυτού δεν κατέβη.
\par 14 Και το πρωΐ έγραψεν ο Δαβίδ επιστολήν προς τον Ιωάβ, και έστειλεν αυτήν διά χειρός του Ουρίου.
\par 15 Και έγραψεν εν τη επιστολή, λέγων, Θέσατε τον Ουρίαν απέναντι της σκληροτέρας μάχης· έπειτα σύρθητε απ' αυτού, διά να κτυπηθή και να αποθάνη.
\par 16 Και αφού παρετήρησε την πόλιν ο Ιωάβ, διώρισε τον Ουρίαν εις θέσιν, όπου ήξευρεν ότι ήσαν άνδρες δυνάμεως.
\par 17 Και εξήλθον οι άνδρες της πόλεως, και επολέμησαν μετά του Ιωάβ· και έπεσον εκ του λαού τινές των δούλων του Δαβίδ· εθανατώθη δε και Ουρίας ο Χετταίος.
\par 18 Και απέστειλεν ο Ιωάβ και ανήγγειλε προς τον Δαβίδ πάντα τα περί του πολέμου.
\par 19 Και προσέταξε τον μηνυτήν, λέγων, Αφού τελειώσης λαλών προς τον βασιλέα πάντα τα περί του πολέμου,
\par 20 εξαφθή ο θυμός του βασιλέως, και είπη προς σε, Διά τι επλησιάσατε εις την πόλιν μαχόμενοι; δεν ηξεύρετε ότι ήθελον τοξεύσει από του τείχους;
\par 21 τις επάταξεν Αβιμέλεχ τον υιόν του Ιερουβέσεθ; γυνή τις δεν έρριψεν επ' αυτόν τμήμα μυλοπέτρας από του τείχους, και απέθανεν εν Θαιβαίς; διά τι επλησιάσατε εις το τείχος; τότε ειπέ, Απέθανε και ο δούλός σου Ουρίας ο Χετταίος.
\par 22 Υπήγε λοιπόν ο μηνυτής και ελθών, απήγγειλε προς τον Δαβίδ πάντα εκείνα, διά τα οποία απέστειλεν αυτόν ο Ιωάβ.
\par 23 Και είπεν ο μηνυτής προς τον Δαβίδ, ότι υπερίσχυσαν καθ' ημών οι άνδρες και εξήλθον προς ημάς εις την πεδιάδα, και κατεδιώξαμεν αυτούς μέχρι της εισόδου της πύλης·
\par 24 αλλ' οι τοξόται ετόξευσαν από του τείχους επί τους δούλους σου· και τινές των δούλων του βασιλέως απέθανον, και ο δούλός σου έτι Ουρίας ο Χετταίος απέθανε.
\par 25 Τότε είπεν ο Δαβίδ προς τον μηνυτήν, Ούτω θέλεις ειπεί προς τον Ιωάβ· Μη σε ανησυχή τούτο το πράγμα· διότι η ρομφαία κατατρώγει ποτέ μεν ένα, ποτέ δε άλλον· ενίσχυσον την μάχην σου εναντίον της πόλεως και κατάστρεψον αυτήν· και συ ενθάρρυνε αυτόν.
\par 26 Ότε δε ήκουσεν η γυνή του Ουρίου, ότι Ουρίας ο ανήρ αυτής απέθανεν, επένθησε διά τον άνδρα αυτής.
\par 27 Και αφού επέρασε το πένθος, απέστειλεν ο Δαβίδ και παρέλαβεν αυτήν εις τον οίκον αυτού· και έγεινε γυνή αυτού και εγέννησεν εις αυτόν υιόν· το πράγμα όμως το οποίον έπραξεν ο Δαβίδ, εφάνη κακόν εις τους οφθαλμούς του Κυρίου.

\chapter{12}

\par 1 Και απέστειλεν ο Κύριος τον Νάθαν προς τον Δαβίδ. Και ήλθε προς αυτόν και είπε προς αυτόν, Ήσαν δύο άνδρες εν πόλει τινί, ο εις πλούσιος, ο δε άλλος πτωχός.
\par 2 Ο πλούσιος είχε ποίμνια και βουκόλια πολλά σφόδρα·
\par 3 ο δε πτωχός δεν είχεν άλλο, ειμή μίαν μικράν αμνάδα, την οποίαν ηγόρασε και έθρεψε· και εμεγάλωσε μετ' αυτού και μετά των τέκνων αυτού ομού· από του άρτου αυτού έτρωγε, και από του ποτηρίου αυτού έπινε, και εν τω κόλπω αυτού εκοιμάτο, και ήτο εις αυτόν ως θυγάτηρ.
\par 4 Ήλθε δε τις διαβάτης προς τον πλούσιον και εφειδωλεύθη να λάβη εκ των ποιμνίων αυτού και εκ των αγέλων αυτού, διά να ετοιμάση εις τον οδοιπόρον τον ελθόντα προς αυτόν, και έλαβε την αμνάδα του πτωχού και ητοίμασεν αυτήν διά τον άνθρωπον τον ελθόντα προς αυτόν.
\par 5 Και εξήφθη η οργή του Δαβίδ κατά του ανθρώπου σφόδρα· και είπε προς τον Νάθαν, Ζη Κύριος, άξιος θανάτου είναι ο άνθρωπος, όστις έπραξε τούτο·
\par 6 και θέλει πληρώσει την αμνάδα τετραπλάσιον, επειδή έπραξε το πράγμα τούτο και επειδή δεν εσπλαγχνίσθη.
\par 7 Και είπεν ο Νάθαν προς τον Δαβίδ, Συ είσαι ο άνθρωπος· ούτω λέγει Κύριος ο Θεός του Ισραήλ· Εγώ σε έχρισα βασιλέα επί τον Ισραήλ, και εγώ σε ηλευθέρωσα εκ χειρός του Σαούλ·
\par 8 και έδωκα εις σε τον οίκον του κυρίου σου και τας γυναίκας του κυρίου σου εις τον κόλπον σου, και έδωκα εις σε τον οίκον του Ισραήλ και του Ιούδα· και εάν τούτο ήτο ολίγον, ήθελον προσθέσει εις σε τοιαύτα και τοιαύτα·
\par 9 διά τι κατεφρόνησας τον λόγον του Κυρίου, ώστε να πράξης το κακόν εις τους οφθαλμούς αυτού; Ουρίαν τον Χετταίον επάταξας εν ρομφαία, και την γυναίκα αυτού έλαβες εις σεαυτόν γυναίκα, και αυτόν εθανάτωσας εν τη ρομφαία των υιών Αμμών·
\par 10 τώρα λοιπόν δεν θέλει αποσυρθή ποτέ ρομφαία εκ του οίκου σου· επειδή με κατεφρόνησας και έλαβες την γυναίκα Ουρίου του Χετταίου, διά να ήναι γυνή σου.
\par 11 Ούτω λέγει Κύριος· Ιδού, θέλω επεγείρει εναντίον σου κακά εκ του οίκου σου, και θέλω λάβει τας γυναίκάς σου έμπροσθεν των οφθαλμών σου και δώσει αυτάς εις τον πλησίον σου, και θέλει κοιμηθή μετά των γυναικών σου ενώπιον του ηλίου τούτου·
\par 12 διότι συ έπραξας κρυφίως· αλλ' εγώ θέλω κάμει τούτο το πράγμα έμπροσθεν παντός του Ισραήλ και κατέναντι του ηλίου.
\par 13 Και είπεν ο Δαβίδ προς τον Νάθαν, Ημάρτησα εις τον Κύριον. Ο δε Νάθαν είπε προς τον Δαβίδ, Και ο Κύριος παρέβλεψε το αμάρτημά σου· δεν θέλεις αποθάνει·
\par 14 επειδή όμως διά ταύτης της πράξεως έδωκας μεγάλην αφορμήν εις τους εχθρούς του Κυρίου να βλασφημώσι, διά τούτο το παιδίον το γεννηθέν εις σε εξάπαντος θέλει αποθάνει.
\par 15 Και απήλθεν ο Νάθαν εις τον οίκον αυτού. Ο δε Κύριος επάταξε το παιδίον, το οποίον εγέννησεν η γυνή του Ουρίου εις τον Δαβίδ, και ηρρώστησε.
\par 16 Και ικέτευσεν ο Δαβίδ τον Θεόν υπέρ του παιδίου· και ενήστευσεν ο Δαβίδ και εισελθών διενυκτέρευσε, κοιτόμενος κατά γης.
\par 17 Και εσηκώθησαν οι πρεσβύτεροι του οίκου αυτού, και ήλθον προς αυτόν διά να σηκώσωσιν αυτόν από της γής· πλην δεν ηθέλησεν ουδέ έφαγεν άρτον μετ' αυτών.
\par 18 Και την ημέραν την εβδόμην απέθανε το παιδίον. Και εφοβήθησαν οι δούλοι του Δαβίδ να αναγγείλωσι προς αυτόν ότι το παιδίον απέθανε· διότι έλεγον, Ιδού, ενώ έζη έτι το παιδίον, ελαλούμεν προς αυτόν, και δεν εισήκουε της φωνής ημών· πόσον λοιπόν κακόν θέλει κάμει, εάν είπωμεν προς αυτόν ότι το παιδίον απέθανεν;
\par 19 Αλλ' ιδών ο Δαβίδ ότι οι δούλοι αυτού εψιθύριζον μετ' αλλήλων, ενόησεν ο Δαβίδ ότι το παιδίον απέθανεν· όθεν είπεν ο Δαβίδ προς τους δούλους αυτού, Απέθανε το παιδίον; οι δε είπον, Απέθανε.
\par 20 Τότε εσηκώθη ο Δαβίδ από της γης και ελούσθη και ηλείφθη και ήλλαξε τα ιμάτια αυτού, και εισήλθεν εις τον οίκον του Κυρίου, και προσεκύνησεν· έπειτα εισήλθεν εις τον οίκον αυτού· και εζήτησε να φάγη και έβαλον έμπροσθεν αυτού άρτον, και έφαγεν.
\par 21 Οι δε δούλοι αυτού είπον προς αυτόν, Τι είναι τούτο, το οποίον έκαμες; ενήστευες και έκλαιες περί του παιδίου, ενώ έζη· αφού δε απέθανε το παιδίον, εσηκώθης και έφαγες άρτον.
\par 22 Και είπεν, Ενώ έτι έζη το παιδίον, ενήστευσα και έκλαυσα, διότι είπα, Τις εξεύρει; ίσως ο Θεός με ελεήση, και ζήση το παιδίον·
\par 23 αλλά τώρα απέθανε· διά τι να νηστεύω; μήπως δύναμαι να επιστρέψω αυτό πάλιν; εγώ θέλω υπάγει προς αυτό, αυτό όμως δεν θέλει αναστρέψει προς εμέ.
\par 24 Και παρηγόρησεν ο Δαβίδ την Βηθ-σαβεέ την γυναίκα αυτού, και εισήλθε προς αυτήν και εκοιμήθη μετ' αυτής, και εγέννησεν υιόν, και εκάλεσε το όνομα αυτού Σολομών· και ο Κύριος ηγάπησεν αυτόν.
\par 25 Και έστειλε διά χειρός Νάθαν του προφήτου, και εκάλεσε το όνομα αυτού Ιεδιδία, διά τον Κύριον.
\par 26 Ο δε Ιωάβ επολέμησεν εναντίον της Ραββά των υιών Αμμών, και εκυρίευσε την βασιλικήν πόλιν.
\par 27 Και απέστειλεν ο Ιωάβ μηνυτάς προς τον Δαβίδ και είπεν, Επολέμησα εναντίον της Ραββά, μάλιστα εκυρίευσα την πόλιν των υδάτων·
\par 28 τώρα λοιπόν σύναξον το επίλοιπον του λαού, και στρατοπέδευσον εναντίον της πόλεως και κυρίευσον αυτήν, διά να μη κυριεύσω εγώ την πόλιν, και ονομασθή το όνομά μου επ' αυτήν.
\par 29 Και συνήθροισεν ο Δαβίδ πάντα τον λαόν, και υπήγεν εις Ραββά και επολέμησεν εναντίον αυτής και εκυρίευσεν αυτήν·
\par 30 και έλαβε τον στέφανον του βασιλέως αυτών από της κεφαλής αυτού, το βάρος του οποίου ήτο εν τάλαντον χρυσίου με λίθους πολυτίμους· και ετέθη επί της κεφαλής του Δαβίδ· και λάφυρα της πόλεως εξέφερε πολλά σφόδρα·
\par 31 και τον λαόν τον εν αυτή εξήγαγε και έβαλεν υπό πρίονας και υπό τριβόλους σιδηρούς και υπό πελέκεις σιδηρούς, και επέρασεν αυτούς διά της καμίνου των πλίνθων. Και ούτως έκαμεν εις πάσας τας πόλεις των υιών Αμμών. Τότε επέστρεψεν ο Δαβίδ και πας ο λαός εις Ιερουσαλήμ.

\chapter{13}

\par 1 Μετά δε ταύτα Αβεσσαλώμ ο υιός του Δαβίδ είχεν αδελφήν ώραίαν, ονόματι Θάμαρ, και ηγάπησεν αυτήν Αμνών ο υιός του Δαβίδ.
\par 2 Και έπασχε τόσον ο Αμνών, ώστε ηρρώστησε διά την αδελφήν αυτού Θάμαρ· διότι ήτο παρθένος, και εφαίνετο εις τον Αμνών δυσκολώτατον να πράξη τι εις αυτήν.
\par 3 είχε δε ο Αμνών φίλον, ονομαζόμενον Ιωναδάβ, υιόν του Σαμαά, αδελφού του Δαβίδ· ήτο δε ο Ιωναδάβ άνθρωπος πανούργος σφόδρα.
\par 4 Και είπε προς αυτόν, Διά τι συ, υιέ του βασιλέως, αδυνατείς τόσον από ημέρας εις ημέραν; δεν θέλεις φανερώσει τούτο προς εμέ; Και είπε προς αυτόν ο Αμνών, Αγαπώ Θάμαρ, την αδελφήν Αβεσσαλώμ του αδελφού μου.
\par 5 Και ο Ιωναδάβ είπε προς αυτόν, Πλαγίασον επί της κλίνης σου και προσποιήθητι τον άρρωστον· και όταν ο πατήρ σου έλθη να σε ίδη, ειπέ προς αυτόν, Ας έλθη, παρακαλώ, Θάμαρ η αδελφή μου, και ας μοι δώση να φάγω, και ας ετοιμάση έμπροσθέν μου το φαγητόν, διά να ίδω και να φάγω εκ της χειρός αυτής.
\par 6 Και επλαγίασεν ο Αμνών και προσεποιήθη τον άρρωστον· και ότε ήλθεν ο βασιλεύς να ίδη αυτόν, είπεν ο Αμνών προς τον βασιλέα, Ας έλθη, παρακαλώ, Θάμαρ η αδελφή μου, και ας κάμη έμπροσθέν μου δύο κολλύρια, διά να φάγω εκ της χειρός αυτής.
\par 7 Και απέστειλεν ο Δαβίδ εις τον οίκον προς την Θάμαρ, λέγων, Ύπαγε τώρα εις τον οίκον του αδελφού σου Αμνών, και ετοίμασον εις αυτόν φαγητόν.
\par 8 Και υπήγεν η Θάμαρ εις τον οίκον του αδελφού αυτής Αμνών, όστις ήτο πλαγιασμένος· και έλαβε το άλευρον και εζύμωσε και έκαμε κολλύρια έμπροσθεν αυτού και έψησε τα κολλύρια.
\par 9 Έπειτα έλαβε το τηγάνιον και εκένωσεν αυτά έμπροσθεν αυτού· πλην δεν ηθέλησε να φάγη. Και είπεν ο Αμνών, Εκβάλετε πάντα άνθρωπον απ' έμπροσθέν μου. Και εξήλθον απ' αυτού πάντες.
\par 10 Και είπεν ο Αμνών προς την Θάμαρ, Φέρε το φαγητόν εις τον κοιτώνα, διά να φάγω εκ της χειρός σου. Και η Θάμαρ έλαβε τα κολλύρια, τα οποία έκαμε, και έφερεν εις τον κοιτώνα προς Αμνών τον αδελφόν αυτής.
\par 11 Και ότε προσέφερε προς αυτόν διά να φάγη, επίασεν αυτήν και είπε προς αυτήν, Ελθέ, κοιμήθητι μετ' εμού, αδελφή μου.
\par 12 Η δε είπε προς αυτόν, Μη, αδελφέ μου, μη με ταπεινώσης· διότι δεν πρέπει τοιούτον πράγμα να γείνη εν τω Ισραήλ· μη κάμης την αφροσύνην ταύτην·
\par 13 και εγώ πως θέλω απαλείψει το όνειδός μου; αλλά και συ θέλεις είσθαι ως εις εκ των αφρόνων εν τω Ισραήλ· τώρα λοιπόν, παρακαλώ, λάλησον προς τον βασιλέα· διότι δεν θέλει με αρνηθή εις σε.
\par 14 Δεν ηθέλησεν όμως να εισακούση της φωνής αυτής· αλλ' υπερισχύσας εκείνης, εβίασεν αυτήν και εκοιμήθη μετ' αυτής.
\par 15 Τότε ο Αμνών εμίσησεν αυτήν μίσος μέγα σφόδρα· ώστε το μίσος, με το οποίον εμίσησεν αυτήν, ήτο μεγαλήτερον παρά την αγάπην, με την οποίαν ηγάπησεν αυτήν. Και είπε προς αυτήν ο Αμνών, Σηκώθητι, ύπαγε.
\par 16 Η δε είπε προς αυτόν, Δεν είναι αιτία· το κακόν τούτο, το να με αποβάλης, είναι μεγαλήτερον του άλλου, το οποίον έπραξας εις εμέ. Δεν ηθέλησεν όμως να εισακούση αυτής.
\par 17 Και έκραξε τον νέον αυτού τον υπηρετούντα αυτόν και είπεν, Έκβαλε τώρα ταύτην απ' εμού έξω, και μόχλωσον την θύραν κατόπιν αυτής.
\par 18 Ήτο δε ενδεδυμένη χιτώνα ποικιλόχρουν· διότι αι θυγατέρες του βασιλέως, αι παρθένοι, τοιαύτα επενδύματα ενεδύοντο. Και εξέβαλεν αυτήν έξω ο υπηρέτης αυτού και εμόχλωσε την θύραν κατόπιν αυτής.
\par 19 Λαβούσα δε η Θάμαρ στάκτην επί της κεφαλής αυτής, και διασχίσασα τον εφ' αυτής χιτώνα τον ποικιλόχρουν, και βαλούσα τας χείρας αυτής επί της κεφαλής αυτής, απήρχετο, πορευομένη και κράζουσα.
\par 20 Και είπε προς αυτήν Αβεσσαλώμ ο αδελφός αυτής, Μήπως Αμνών ο αδελφός σου ευρέθη μετά σου; πλην τώρα σιώπησον, αδελφή μου· αδελφός σου είναι μη κατάθλιβε την καρδίαν σου διά το πράγμα τούτο. Η Θάμαρ λοιπόν εκάθητο χηρεύουσα εν τω οίκω του αδελφού αυτής Αβεσσαλώμ.
\par 21 Ακούσας δε ο βασιλεύς Δαβίδ πάντα ταύτα τα πράγματα, εθυμώθη σφόδρα.
\par 22 Ο δε Αβεσσαλώμ δεν ελάλησε μετά του Αμνών ούτε καλόν ούτε κακόν· διότι εμίσει ο Αβεσσαλώμ τον Αμνών, επειδή εταπείνωσε την αδελφήν αυτού Θάμαρ.
\par 23 Και μετά δύο ολόκληρα έτη, ο Αβεσσαλώμ είχε κουρευτάς εν Βαάλ-ασώρ, ήτις είναι πλησίον του Εφραΐμ, και προσεκάλεσεν ο Αβεσσαλώμ πάντας τους υιούς του βασιλέως.
\par 24 Και ήλθεν ο Αβεσσαλώμ προς τον βασιλέα και είπεν, Ιδού, τώρα, ο δούλός σου έχει κουρευτάς· ας έλθη, παρακαλώ, ο βασιλεύς και οι δούλοι αυτού μετά του δούλου σου.
\par 25 Και είπεν ο βασιλεύς προς τον Αβεσσαλώμ, Ουχί, υιέ μου, ας μη έλθωμεν τώρα πάντες, διά να μη ήμεθα βάρος εις σε. Και εβίασεν αυτόν, πλην δεν ηθέλησε να υπάγη, αλλ' ευλόγησεν αυτόν.
\par 26 Τότε είπεν ο Αβεσσαλώμ, Αν όχι, ας έλθη καν μεθ' ημών Αμνών, ο αδελφός μου. Και είπεν ο βασιλεύς προς αυτόν, Διά τι να έλθη μετά σου;
\par 27 πλην ο Αβεσσαλώμ εβίασεν αυτόν, ώστε απέστειλε μετ' αυτού τον Αμνών και πάντας τους υιούς του βασιλέως.
\par 28 Τότε προσέταξεν ο Αβεσσαλώμ τους υπηρέτας αυτού λέγων. Ιδέτε τώρα όταν ευφρανθή η καρδία του Αμνών εκ του οίνου, και είπω προς εσάς, Πατάξατε τον Αμνών, τότε θανατώσατε αυτόν· μη φοβείσθε· δεν είμαι εγώ όστις σας προστάζω; ανδρίζεσθε και γίνεσθε υιοί δυνάμεως.
\par 29 Και έκαμον οι υπηρέται του Αβεσσαλώμ προς τον Αμνών, ως προσέταξεν ο Αβεσσαλώμ. Τότε σηκωθέντες πάντες οι υιοί του βασιλέως, εκάθησαν έκαστος επί της ημιόνου αυτού και έφυγον.
\par 30 Ενώ δε ούτοι ήσαν καθ' οδόν, η φήμη έφθασε προς τον Δαβίδ, λέγουσα, Ο Αβεσσαλώμ επάταξε πάντας τους υιούς του βασιλέως, και δεν εναπελείφθη εξ αυτών ουδέ εις.
\par 31 Τότε σηκωθείς ο βασιλεύς διέσχισε τα ιμάτια αυτού και επλαγίασε κατά γής· και πάντες οι δούλοι αυτού οι περιεστώτες διέσχισαν τα ιμάτια αυτών.
\par 32 Και απεκρίθη Ιωναδάβ, ο υιός του Σαμαά, αδελφού του Δαβίδ, και είπεν, Ας μη λέγη ο κύριός μου ότι εθανατώθησαν πάντες οι νέοι, οι υιοί του βασιλέως· διότι ο Αμνών μόνος απέθανεν· επειδή ο Αβεσσαλώμ είχεν αποφασίσει τούτο, αφ' ης ημέρας εταπείνωσε Θάμαρ την αδελφήν αυτού·
\par 33 τώρα λοιπόν ας μη βάλη ο κύριός μου ο βασιλεύς το πράγμα εν τη καρδία αυτού, λέγων ότι πάντες οι υιοί του βασιλέως απέθανον· διότι ο Αμνών μόνος απέθανεν.
\par 34 Ο δε Αβεσσαλώμ έφυγε. Και υψώσας ο νέος, ο σκοπός, τους οφθαλμούς αυτού, είδε, και ιδού, λαός πολύς επορεύετο διά της οδού όπισθεν αυτού κατά το πλευρόν του όρους.
\par 35 Και είπεν ο Ιωναδάβ προς τον βασιλέα, Ιδού, οι υιοί του βασιλέως έρχονται κατά τον λόγον του δούλου σου, ούτως έγεινε.
\par 36 Και ως ετελείωσε λαλών, ιδού, οι υιοί του βασιλέως ήλθον και ύψωσαν την φωνήν αυτών και έκλαυσαν· και ο βασιλεύς έτι, και πάντες οι δούλοι αυτού έκλαυσαν κλαυθμόν μέγαν σφόδρα.
\par 37 Ο δε Αβεσσαλώμ έφυγε και υπήγε προς τον Θαλμαΐ, υιόν του Αμμιούδ, βασιλέα της Γεσσούρ· και επένθησεν ο Δαβίδ διά τον υιόν αυτού πάσας τας ημέρας.
\par 38 Ο Αβεσσαλώμ λοιπόν έφυγε και υπήγεν εις Γεσσούρ, και ήτο εκεί τρία έτη.
\par 39 Επεπόθησε δε ο βασιλεύς Δαβίδ να υπάγη προς τον Αβεσσαλώμ, διότι είχε παρηγορηθή διά τον θάνατον του Αμνών.

\chapter{14}

\par 1 Και εγνώρισεν ο Ιωάβ ο υιός της Σερουΐας, ότι η καρδία του βασιλέως ήτο εις τον Αβεσσαλώμ.
\par 2 Και απέστειλεν ο Ιωάβ εις Θεκουέ και έφερεν εκείθεν γυναίκα σοφήν, και είπε προς αυτήν, Προσποιήθητι, παρακαλώ, ότι είσαι εν πένθει και ενδύθητι ιμάτια πενθικά, και μη αλειφθής έλαιον, αλλ' έσο ως γυνή πενθούσα ήδη ημέρας πολλάς διά αποθανόντα·
\par 3 και ύπαγε προς τον βασιλέα και λάλησον προς αυτόν κατά τούτους τους λόγους. Και έβαλεν ο Ιωάβ τους λόγους εις το στόμα αυτής.
\par 4 Λαλούσα δε η γυνή η Θεκωΐτις προς τον βασιλέα, έπεσε κατά πρόσωπον αυτής επί της γης και προσεκύνησε και είπε, Σώσον, βασιλεύ.
\par 5 Και είπε προς αυτήν ο βασιλεύς, Τι έχεις; Η δε είπε, Γυνή χήρα, φευ είμαι εγώ, και απέθανεν ο ανήρ μου·
\par 6 και η δούλη σου είχε δύο υιούς, οίτινες ελογομάχησαν αμφότεροι εν τω αγρώ, και δεν ήτο ο χωρίζων αυτούς, αλλ' επάταξεν ο εις τον άλλον και εθανάτωσεν αυτόν·
\par 7 και ιδού, εσηκώθη πάσα η συγγένεια εναντίον της δούλης σου και είπον, Παράδος τον πατάξαντα τον αδελφόν αυτού, διά να θανατώσωμεν αυτόν, αντί της ζωής του αδελφού αυτού τον οποίον εφόνευσε, και να εξολοθρεύσωμεν ενταυτώ τον κληρονόμον· και ούτω θέλουσι σβέσει τον άνθρακά μου τον εναπολειφθέντα, ώστε να μη αφήσωσιν εις τον άνδρα μου όνομα μηδέ απομεινάριον επί το πρόσωπον της γης.
\par 8 Και είπεν ο βασιλεύς προς την γυναίκα, Ύπαγε εις τον οίκόν σου, και εγώ θέλω προστάξει υπέρ σου.
\par 9 Και είπεν η γυνή η Θεκωΐτις προς τον βασιλέα, Κύριέ μου βασιλεύ, επ' εμέ ας ήναι η ανομία και επί τον οίκον του πατρός μου· ο δε βασιλεύς και ο θρόνος αυτού αθώοι.
\par 10 Και είπεν ο βασιλεύς, Όστις λαλήση εναντίον σου, φέρε αυτόν προς εμέ, και δεν θέλει πλέον σε εγγίσει.
\par 11 Η δε είπεν, Ας ενθυμηθή, παρακαλώ, ο βασιλεύς Κύριον τον Θεόν σου, και ας μη αφήση τους εκδικητάς του αίματος να πληθύνωσι την φθοράν και να απολέσωσι τον υιόν μου. Ο δε είπε, Ζη Κύριος, ουδέ μία θριξ του υιού σου δεν θέλει πέσει εις την γην.
\par 12 Τότε είπεν η γυνή, Ας λαλήση, παρακαλώ, η δούλη σου λόγον προς τον κύριόν μου τον βασιλέα. Και είπε, Λάλησον.
\par 13 Και είπεν η γυνή, Και διά τι εστοχάσθης τοιούτον πράγμα κατά του λαού του Θεού; διότι ο βασιλεύς λαλεί τούτο ως άνθρωπος ένοχος, επειδή ο βασιλεύς δεν στέλλει να επαναφέρη τον εξόριστον αυτού.
\par 14 Διότι αφεύκτως θέλομεν αποθάνει, και είμεθα ως ύδωρ διακεχυμένον επί της γης, το οποίον δεν επισυνάγεται πάλιν· και ο Θεός δεν θέλει να απολεσθή ψυχή, αλλ' εφευρίσκει μέσα, ώστε ο εξόριστος να μη μένη εξωσμένος απ' αυτού.
\par 15 Τώρα διά τούτο ήλθον να λαλήσω τον λόγον τούτον προς τον κύριόν μου τον βασιλέα, διότι ο λαός με εφόβισε· και η δούλη σου είπε, θέλω τώρα λαλήσει προς τον βασιλέα· ίσως κάμη ο βασιλεύς την αίτησιν της δούλης αυτού.
\par 16 Διότι ο βασιλεύς θέλει εισακούσει, διά να ελευθερώση την δούλην αυτού εκ χειρός του ανθρώπου του ζητούντος να εξαλείψη εμέ και τον υιόν μου ενταυτώ από της κληρονομίας του Θεού.
\par 17 Είπε μάλιστα η δούλη σου, Ο λόγος του κυρίου μου του βασιλέως θέλει είσθαι τώρα παρηγορητικός· διότι ως άγγελος Θεού, ούτως είναι ο κύριός μου ο βασιλεύς, εις το να διακρίνη το καλόν και το κακόν· και Κύριος ο Θεός σου θέλει είσθαι μετά σου.
\par 18 Τότε απεκρίθη ο βασιλεύς και είπε προς την γυναίκα, Μη κρύψης απ' εμού τώρα το πράγμα, το οποίον θέλω σε ερωτήσει εγώ. Και είπεν η γυνή, Ας λαλήση, παρακαλώ, ο κύριός μου ο βασιλεύς.
\par 19 Και είπεν ο βασιλεύς, Δεν είναι εις όλον τούτο η χειρ του Ιωάβ μετά σου; Και η γυνή απεκρίθη και είπε, Ζη η ψυχή σου, κύριέ μου βασιλεύ, ουδέν εκ των όσα είπεν ο κύριός μου ο βασιλεύς δεν έκλινεν ούτε δεξιά ούτε αριστερά· διότι ο δούλός σου Ιωάβ, αυτός προσέταξεν εις εμέ, και αυτός έβαλε πάντας τους λόγους τούτους εις το στόμα της δούλης σου·
\par 20 ο δούλός σου Ιωάβ έκαμε τούτο, να μεταστρέψω την μορφήν του πράγματος τούτου· και ο κύριός μου είναι σοφός, κατά την σοφίαν αγγέλου του Θεού, εις το να γνωρίζη πάντα τα εν τη γη.
\par 21 Και είπεν ο βασιλεύς προς τον Ιωάβ, Ιδού, τώρα, έκαμα το πράγμα τούτο· ύπαγε λοιπόν, επανάφερε τον νέον, τον Αβεσσαλώμ.
\par 22 Και έπεσεν ο Ιωάβ κατά πρόσωπον αυτού εις την γην και προσεκύνησε και ευλόγησε τον βασιλέα· και είπεν ο Ιωάβ, Σήμερον ο δούλός σου γνωρίζει ότι εύρηκα χάριν εις τους οφθαλμούς σου, κύριέ μου βασιλεύ, καθότι ο βασιλεύς έκαμε τον λόγον του δούλου αυτού.
\par 23 Τότε εσηκώθη ο Ιωάβ και υπήγεν εις Γεσσούρ και έφερε τον Αβεσσαλώμ εις Ιερουσαλήμ.
\par 24 Και είπεν ο βασιλεύς, Ας επιστρέψη εις τον οίκον αυτού και ας μη ίδη το πρόσωπόν μου. Ούτως επέστρεψεν ο Αβεσσαλώμ εις τον οίκον αυτού, και δεν είδε το πρόσωπον του βασιλέως.
\par 25 Εις πάντα δε τον Ισραήλ δεν υπήρχεν άνθρωπος ούτω θαυμαζόμενος διά την ώραιότητα αυτού ως ο Αβεσσαλώμ· από του ίχνους του ποδός αυτού έως της κορυφής αυτού δεν υπήρχεν εν αυτώ ελάττωμα·
\par 26 και οπότε εκούρευε την κεφαλήν αυτού, διότι εις το τέλος εκάστου έτους εκούρευεν αυτήν· επειδή τα μαλλία εβάρυνον αυτόν διά τούτο έκοπτεν αυτά· εζύγιζε τας τρίχας της κεφαλής αυτού, και ήσαν διακοσίων σίκλων κατά το βασιλικόν ζύγιον.
\par 27 Εγεννήθησαν δε εις τον Αβεσσαλώμ τρεις υιοί και μία θυγάτηρ, ονόματι Θάμαρ· αύτη ήτο γυνή ώραιοτάτη.
\par 28 Και κατώκησεν ο Αβεσσαλώμ εν Ιερουσαλήμ δύο ολόκληρα έτη, και το πρόσωπον του βασιλέως δεν είδεν.
\par 29 Όθεν απέστειλεν ο Αβεσσαλώμ προς τον Ιωάβ, διά να πέμψη αυτόν προς τον βασιλέα· πλην δεν ηθέλησε να έλθη προς αυτόν· απέστειλε πάλιν εκ δευτέρου, αλλά δεν ηθέλησε να έλθη.
\par 30 Τότε είπε προς τους δούλους αυτού, Ιδέτε, ο αγρός του Ιωάβ είναι πλησίον του ιδικού μου, και έχει κριθήν εκεί· υπάγετε και κατακαύσατε αυτήν εν πυρί· και κατέκαυσαν οι δούλοι του Αβεσσαλώμ τον αγρόν εν πυρί.
\par 31 Και εσηκώθη ο Ιωάβ και ήλθε προς τον Αβεσσαλώμ εις την οικίαν και είπε προς αυτόν, Διά τι κατέκαυσαν οι δούλοί σου τον αγρόν μου εν πυρί;
\par 32 Ο δε Αβεσσαλώμ απεκρίθη προς τον Ιωάβ, Ιδού, απέστειλα προς σε, λέγων, Ελθέ ενταύθα, διά να σε πέμψω προς τον βασιλέα να είπης, Διά τι ήλθον από Γεσσούρ; ήθελεν είσθαι καλήτερον δι' εμέ να ήμην έτι εκεί· τώρα λοιπόν ας ίδω το πρόσωπον του βασιλέως· και αν ήναι αδικία εν εμοί, ας με θανατώση.
\par 33 Τότε ο Ιωάβ ήλθε προς τον βασιλέα και ανήγγειλε ταύτα προς αυτόν· και εκάλεσε τον Αβεσσαλώμ, και ήλθε προς τον βασιλέα, και πεσών επί πρόσωπον αυτού εις την γην, προσεκύνησεν ενώπιον του βασιλέως· και ο βασιλεύς εφίλησε τον Αβεσσαλώμ.

\chapter{15}

\par 1 Μετά δε ταύτα ητοίμασεν εις εαυτόν ο Αβεσσαλώμ αμάξας και ίππους και πεντήκοντα άνδρας, διά να τρέχωσιν έμπροσθεν αυτού.
\par 2 Και εσηκόνετο ο Αβεσσαλώμ πρωΐ, και ίστατο εις τα πλάγια της οδού της πύλης· και οπότε τις έχων διαφοράν τινά ήρχετο προς τον βασιλέα διά κρίσιν, τότε ο Αβεσσαλώμ εκάλει αυτόν προς εαυτόν και έλεγεν, Εκ ποίας πόλεως είσαι; Ο δε απεκρίνετο, Ο δούλός σου είναι εκ της δείνος φυλής του Ισραήλ.
\par 3 Και έλεγε προς αυτόν ο Αβεσσαλώμ, Ιδέ, η υπόθεσίς σου είναι καλή και ορθή· πλην δεν είναι ουδείς ο ακούων σε από μέρους του βασιλέως.
\par 4 Έλεγε προσέτι ο Αβεσσαλώμ, Τις να με εδιώριζε κριτήν του τόπου, διά να έρχηται προς εμέ πας όστις έχει διαφοράν ή κρίσιν, και να δικαιόνω αυτόν.
\par 5 Και οπότε τις επλησίαζε διά να προσκυνήση αυτόν, ήπλονε την χείρα αυτού και επίανεν αυτόν και εφίλει αυτόν.
\par 6 Και έκαμνεν ο Αβεσσαλώμ κατά τούτον τον τρόπον εις πάντα Ισραηλίτην ερχόμενον προς τον βασιλέα διά κρίσιν· και υπέκλεπτεν ο Αβεσσαλώμ τας καρδίας των ανδρών Ισραήλ.
\par 7 Και εις το τέλος τεσσαράκοντα ετών είπεν ο Αβεσσαλώμ προς τον βασιλέα, Ας υπάγω, παρακαλώ, διά να εκπληρώσω την ευχήν μου, την οποίαν ηυχήθην εις τον Κύριον, εν Χεβρών·
\par 8 διότι ο δούλός σου ηυχήθη ευχήν, ότε κατώκει εν Γεσσούρ εν Συρία, λέγων· Εάν ο Κύριος με επιστρέψη τωόντι εις Ιερουσαλήμ, τότε θέλω προσφέρει θυσίαν εις τον Κύριον.
\par 9 Και είπε προς αυτόν ο βασιλεύς, Ύπαγε εν ειρήνη. Και σηκωθείς, υπήγεν εις Χεβρών.
\par 10 Απέστειλε δε ο Αβεσσαλώμ κατασκόπους εις πάσας τας φυλάς του Ισραήλ, λέγων, Καθώς ακούσητε την φωνήν της σάλπιγγος, θέλετε ειπεί Ο Αβεσσαλώμ εβασίλευσεν εν Χεβρών.
\par 11 Και υπήγαν μετά του Αβεσσαλώμ διακόσιοι άνδρες εξ Ιερουσαλήμ, κεκλημένοι και υπήγαν εν τη απλότητι αυτών και δεν ήξευραν ουδέν.
\par 12 Και προσεκάλεσεν ο Αβεσσαλώμ Αχιτόφελ τον Γιλωναίον, τον σύμβουλον του Δαβίδ, εκ της πόλεως αυτού, εκ Γιλώ, ενώ προσέφερε τας θυσίας. Και η συνωμοσία ήτο δυνατή και ο λαός επληθύνετο αδιακόπως πλησίον του Αβεσσαλώμ.
\par 13 Ήλθε δε μηνυτής προς τον Δαβίδ λέγων, Αι καρδίαι των ανδρών Ισραήλ εστράφησαν κατόπιν του Αβεσσαλώμ.
\par 14 Και είπεν ο Δαβίδ προς πάντας τους δούλους αυτού τους μεθ' αυτού εν Ιερουσαλήμ, Σηκώθητε, και ας φύγωμεν· διότι δεν θέλομεν δυνηθή να διασωθώμεν από προσώπου του Αβεσσαλώμ· σπεύσατε να αναχωρήσωμεν, διά να μη επιταχύνη και καταφθάση ημάς και σπρώξη το κακόν εφ' ημάς και πατάξη την πόλιν εν στόματι μαχαίρας.
\par 15 Και οι δούλοι του βασιλέως είπαν προς τον βασιλέα, Εις παν, ό,τι εκλέξη ο κύριός μου ο βασιλεύς, ιδού, οι δούλοί σου.
\par 16 Και εξήλθεν ο βασιλεύς και πας ο οίκος αυτού κατόπιν αυτού. Και αφήκεν ο βασιλεύς τας δέκα γυναίκας τας παλλακάς διά να φυλάττωσι τον οίκον.
\par 17 Και εξήλθεν ο βασιλεύς και πας ο λαός κατόπιν αυτού, και εστάθησαν εις τόπον μακράν απέχοντα.
\par 18 Και πάντες οι δούλοι αυτού επορεύοντο πλησίον αυτού· και πάντες οι Χερεθαίοι και πάντες οι Φελεθαίοι και πάντες οι Γετθαίοι, εξακόσιοι άνδρες, οι ελθόντες οπίσω αυτού από Γαθ, προεπορεύοντο έμπροσθεν του βασιλέως.
\par 19 Τότε είπεν ο βασιλεύς προς Ιτταΐ τον Γετθαίον, Διά τι έρχεσαι και συ μεθ' ημών; επίστρεψον και κατοίκει μετά του βασιλέως, διότι είσαι ξένος, και μάλιστα είσαι μετωκισμένος εκ του τόπου σου·
\par 20 χθές ήλθες, και σήμερον θέλω σε κάμει να περιπλανάσαι μεθ' ημών; εγώ δε υπάγω όπου δυνηθώ· επίστρεψον και λάβε και τους αδελφούς σου· έλεος και αλήθεια μετά σου.
\par 21 Ο δε Ιτταΐ απεκρίθη προς τον βασιλέα και είπε, Ζη Κύριος, και ζη ο κύριός μου ο βασιλεύς, όπου και αν ήναι ο κύριός μου ο βασιλεύς, είτε εις θάνατον, είτε εις ζωήν, βεβαίως εκεί θέλει είσθαι και ο δούλός σου.
\par 22 Και είπεν ο Δαβίδ προς τον Ιτταΐ, Ελθέ λοιπόν, και διάβαινε. Και διέβη ο Ιτταΐ ο Γετθαίος και πάντες οι άνδρες αυτού και πάντα τα παιδία τα μετ' αυτού.
\par 23 Όλος δε ο τόπος έκλαιε μετά φωνής μεγάλης, και διέβαινε πας ο λαός· διέβη και ο βασιλεύς τον χείμαρρον Κέδρων· και πας ο λαός διέβη κατά την οδόν της ερήμου.
\par 24 Και ιδού, προσέτι ο Σαδώκ και πάντες οι Λευΐται μετ' αυτού, φέροντες την κιβωτόν της διαθήκης του Θεού· και έστησαν την κιβωτόν του Θεού· ανέβη δε ο Αβιάθαρ, αφού ετελείωσε πας ο λαός διαβαίνων από της πόλεως.
\par 25 Και είπεν ο βασιλεύς προς τον Σαδώκ, Απόστρεψον την κιβωτόν του Θεού εις την πόλιν· εάν εύρω χάριν εις τους οφθαλμούς του Κυρίου, θέλει με κάμει να επιστρέψω και να ίδω αυτήν και το κατοικητήριον αυτού·
\par 26 αλλ' εάν είπη ούτω, Δεν έχω ευαρέσκειαν εις σε, ιδού, εγώ, ας κάμη εις εμέ ό,τι φανή αρεστόν εις τους οφθαλμούς αυτού.
\par 27 Ο βασιλεύς είπεν έτι προς Σαδώκ τον ιερέα, Δεν είσαι συ ο βλέπων; επίστρεψον εις την πόλιν εν ειρήνη, και Αχιμάας ο υιός σου και Ιωνάθαν ο υιός του Αβιάθαρ, οι δύο υιοί σας μεθ' υμών·
\par 28 ιδέτε, εγώ θέλω μένει εις τας πεδιάδας της ερήμου, εωσού έλθη λόγος παρ' υμών διά να μοι αναγγείλη.
\par 29 Ο Σαδώκ λοιπόν και ο Αβιάθαρ επανέφεραν την κιβωτόν του Θεού εις Ιερουσαλήμ και έμειναν εκεί.
\par 30 Ο δε Δαβίδ ανέβαινε διά της αναβάσεως των Ελαιών, αναβαίνων και κλαίων και έχων την κεφαλήν αυτού κεκαλυμμένην και περιπατών ανυπόδητος· και πας ο λαός ο μετ' αυτού είχεν έκαστος κεκαλυμμένην την κεφαλήν αυτού, και ανέβαινον πορευόμενοι και κλαίοντες.
\par 31 Και απήγγειλαν προς τον Δαβίδ, λέγοντες, Ο Αχιτόφελ είναι μεταξύ των συνωμοτών μετά του Αβεσσαλώμ. Και είπεν ο Δαβίδ, Κύριε, δέομαί σου, διασκέδασον την βουλήν του Αχιτόφελ.
\par 32 Και ότε ήλθεν ο Δαβίδ εις την κορυφήν του όρους, όπου προσεκύνησε τον Θεόν, ιδού, ήλθεν εις συνάντησιν αυτού Χουσαΐ ο Αρχίτης, έχων διεσχισμένον τον χιτώνα αυτού και χώμα επί της ο κεφαλής αυτού.
\par 33 Και είπε προς αυτόν ο Δαβίδ, Εάν διαβής μετ' εμού, θέλεις βεβαίως είσθαι φορτίον επ' εμέ·
\par 34 εάν όμως επιστρέψης εις την πόλιν και είπης προς τον Αβεσσαλώμ, Θέλω είσθαι δούλός σου, βασιλεύ· καθώς εστάθην δούλος του πατρός σου μέχρι τούδε, ούτω θέλω είσθαι τώρα δούλός σου· τότε δύνασαι υπέρ εμού να ανατρέψης την βουλήν του Αχιτόφελ·
\par 35 και δεν είναι εκεί μετά σου ο Σαδώκ και ο Αβιάθαρ, οι ιερείς; παν ό,τι λοιπόν ήθελες ακούσει εκ του οίκου του βασιλέως, θέλεις αναγγείλει προς τον Σαδώκ και Αβιάθαρ, τους ιερείς·
\par 36 ιδού, εκεί μετ' αυτών οι δύο υιοί αυτών, Αχιμάας ο του Σαδώκ και Ιωνάθαν ο του Αβιάθαρ· και δι' αυτών θέλετε αποστέλλει προς εμέ παν ό,τι ακούσητε.
\par 37 Και καθώς εισήλθεν εις την πόλιν ο Χουσαΐ ο φίλος του Δαβίδ, ο Αβεσσαλώμ ήλθεν εις Ιερουσαλήμ.

\chapter{16}

\par 1 Και ότε ο Δαβίδ επέρασεν ολίγον κορυφήν, ιδού, Σιβά, ο υπηρέτης του Μεμφιβοσθέ, συνήντησεν αυτόν, μετά δύο όνων σαμαρωμένων, έχων επ' αυτούς διακοσίους άρτους και εκατόν βότρυς σταφίδων και εκατόν αρμαθιάς θερινών καρπών και ασκόν οίνου.
\par 2 Και είπεν ο βασιλεύς προς τον Σιβά, Διά τι φέρεις ταύτα; Ο δε Σιβά είπεν, Οι όνοι είναι διά την οικογένειαν του βασιλέως διά να επικάθηται, και οι άρτοι και οι θερινοί καρποί διά να τρώγωσιν οι νέοι· ο δε οίνος, διά να πίνωσιν όσοι ατονίσωσιν εν τη ερήμω.
\par 3 Τότε είπεν ο βασιλεύς, Και που είναι ο υιός του κυρίου σου; Και είπεν ο Σιβά προς τον βασιλέα, Ιδού, κάθηται εν Ιερουσαλήμ· διότι είπε, Σήμερον ο οίκος Ισραήλ θέλει επιστρέψει προς εμέ την βασιλείαν του πατρός μου.
\par 4 Και είπεν ο βασιλεύς προς τον Σιβά, Ιδού, ιδικά σου είναι πάντα τα υπάρχοντα του Μεμφιβοσθέ. Και είπεν ο Σιβά, Δέομαι υποκλινώς να εύρω χάριν εις τους οφθαλμούς σου, κύριέ μου βασιλεύ.
\par 5 Και ότε ήλθεν ο βασιλεύς Δαβίδ έως Βαουρείμ, ιδού, εξήρχετο εκείθεν άνθρωπος εκ της συγγενείας του οίκου του Σαούλ, ονομαζόμενος Σιμεΐ, υιός του Γηρά· και εξελθών, ήρχετο καταρώμενος.
\par 6 Και έρριπτε λίθους επί τον Δαβίδ και επί πάντας τους δούλους του βασιλέως Δαβίδ· πας δε ο λαός και πάντες οι δυνατοί ήσαν εκ δεξιών αυτού και εξ αριστερών αυτού.
\par 7 Και ούτως έλεγεν ο Σιμεΐ καταρώμενος, Έξελθε, έξελθε, ανήρ αιμάτων και ανήρ κακοποιέ·
\par 8 επέστρεψεν ο Κύριος κατά σου πάντα τα αίματα του οίκου του Σαούλ, αντί του οποίου εβασίλευσας· και παρέδωκεν ο Κύριος την βασιλείαν εις την χείρα Αβεσσαλώμ του υιού σου· και ιδού, συ επιάσθης εν τη κακία σου, διότι είσαι ανήρ αιμάτων.
\par 9 Τότε είπε προς τον βασιλέα Αβισαί ο υιός της Σερουΐας, Διά τι ούτος ο νεκρός κύων καταράται τον κύριόν μου τον βασιλέα; άφες, παρακαλώ, να περάσω και να κόψω την κεφαλήν αυτού.
\par 10 Ο δε βασιλεύς είπε, Τι μεταξύ εμού και ημών, υιοί της Σερουΐας; ας καταράται, διότι ο Κύριος είπε προς αυτόν, Καταράσθητι τον Δαβίδ. Τις λοιπόν θέλει ειπεί, Διά τι έκαμες ούτω;
\par 11 Και είπεν ο Δαβίδ προς τον Αβισαί και προς πάντας τους δούλους αυτού, Ιδού, ο υιός μου, ο εξελθών εκ των σπλάγχνων μου ζητεί την ζωήν μου· πόσω μάλλον τώρα ο Βενιαμίτης; αφήσατε αυτόν, και ας καταράται, διότι ο Κύριος προσέταξεν αυτόν·
\par 12 ίσως επιβλέψη ο Κύριος επί την θλίψιν μου, και ανταποδώση ο Κύριος εις εμέ αγαθόν αντί της κατάρας τούτου την ημέραν ταύτην.
\par 13 Και επορεύοντο ο Δαβίδ και οι άνδρες αυτού εις την οδόν, ο δε Σιμεΐ επορεύετο κατά τα πλευρά του όρους απέναντι αυτού, και κατηράτο πορευόμενος και έρριπτε λίθους κατ' αυτού και εσκόνιζε με χώμα.
\par 14 Και ήλθεν ο βασιλεύς, και πας ο λαός ο μετ' αυτού, εκλελυμένοι και ανεπαύθησαν εκεί.
\par 15 Ο δε Αβεσσαλώμ και πας ο λαός, οι άνδρες Ισραήλ, ήλθον εις Ιερουσαλήμ, και ο Αχιτόφελ μετ' αυτού.
\par 16 Και ότε ήλθε προς τον Αβεσσαλώμ Χουσαΐ ο Αρχίτης, ο φίλος του Δαβίδ, είπεν ο Χουσαΐ προς τον Αβεσσαλώμ, Ζήτω ο βασιλεύς· ζήτω ο βασιλεύς.
\par 17 Ο δε Αβεσσαλώμ είπε προς τον Χουσαΐ, τούτο είναι το έλεός σου προς τον φίλον σου; διά τι δεν υπήγες μετά του φίλου σου;
\par 18 Και είπεν ο Χουσαΐ προς τον Αβεσσαλώμ, Ουχί· αλλ' εκείνου, τον οποίον εξέλεξεν ο Κύριος και ούτος ο λαός και πάντες οι άνδρες Ισραήλ, τούτου θέλω είσθαι και μετά τούτου θέλω κατοικεί·
\par 19 και έπειτα, ποίον θέλω δουλεύει εγώ; ουχί έμπροσθεν του υιού αυτού; καθώς εδούλευσα έμπροσθεν του πατρός σου, ούτω θέλω είσθαι έμπροσθέν σου.
\par 20 Τότε είπεν ο Αβεσσαλώμ προς τον Αχιτόφελ, Συμβουλεύθητε μεταξύ σας τι θέλομεν κάμει.
\par 21 Και είπεν ο Αχιτόφελ προς τον Αβεσσαλώμ, Είσελθε εις τας παλλακάς του πατρός σου, τας οποίας αφήκε διά να φυλάττωσι τον οίκον· και θέλει ακούσει πας ο Ισραήλ, ότι έγεινες μισητός εις τον πατέρα σου· και θέλουσιν ενδυναμωθή αι χείρες πάντων των μετά σου.
\par 22 Έστησαν λοιπόν εις τον Αβεσσαλώμ σκηνήν επί του δώματος, και εισήλθεν ο Αβεσσαλώμ εις τας παλλακάς του πατρός αυτού, ενώπιον παντός του Ισραήλ.
\par 23 Και η συμβουλή του Αχιτόφελ, την οποίαν έδιδε κατ' εκείνας τας ημέρας, ήτο ως εάν τις ήθελε συμβουλευθή τον Θεόν· ούτως ενομίζετο πάσα συμβουλή του Αχιτόφελ και εις τον Δαβίδ και εις τον Αβεσσαλώμ.

\chapter{17}

\par 1 Και ο Αχιτόφελ είπε προς τον Αβεσσαλώμ, Ας εκλέξω τώρα δώδεκα χιλιάδας ανδρών και σηκωθείς, ας καταδιώξω οπίσω του Δαβίδ την νύκτα·
\par 2 και θέλω επέλθει κατ' αυτού, ενώ είναι αποκαμωμένος και εκλελυμένος τας χείρας, και θέλω κατατρομάξει αυτόν· και πας ο λαός ο μετ' αυτού θέλει φύγει, και θέλω πατάξει τον βασιλέα μεμονωμένον·
\par 3 και θέλω επιστρέψει πάντα τον λαόν προς σέ· διότι ο ανήρ, τον οποίον συ ζητείς, είναι ως εάν πάντες επέστρεφον· πας δε ο λαός θέλει είσθαι εν ειρήνη.
\par 4 Και ήρεσεν ο λόγος εις τον Αβεσσαλώμ και εις πάντας τους πρεσβυτέρους του Ισραήλ.
\par 5 Τότε είπεν ο Αβεσσαλώμ, Κάλεσον τώρα και Χουσαΐ τον Αρχίτην, και ας ακούσωμεν τι λέγει και αυτός.
\par 6 Και ότε εισήλθεν ο Χουσαΐ προς τον Αβεσσαλώμ, είπε προς αυτόν ο Αβεσσαλώμ, λέγων, Ο Αχιτόφελ ελάλησε κατά τούτον τον τρόπον· πρέπει να κάμωμεν κατά τον λόγον αυτού ή ουχί; λάλησον συ.
\par 7 Και είπεν ο Χουσαΐ προς τον Αβεσσαλώμ, Δεν είναι καλή η συμβουλή, την οποίαν έδωκεν ο Αχιτόφελ ταύτην την φοράν.
\par 8 Και είπεν ο Χουσαΐ, συ εξεύρεις τον πατέρα σου και τους άνδρας αυτού, ότι είναι δυνατοί και κατάπικροι την ψυχήν, ως άρκτος στερηθείσα των τέκνων αυτής εν τη πεδιάδι και ο πατήρ σου είναι ανήρ πολεμιστής και δεν θέλει μείνει την νύκτα μετά του λαού·
\par 9 ιδού, τώρα είναι κεκρυμμένος εν λάκκω τινί ή εν άλλω τινί τόπω· και εάν πέσωσί τινές εξ αυτών εις την αρχήν, πας όστις ακούση θέλει ειπεί, θραύσις έγεινεν εις τον λαόν, τον ακολουθούντα τον Αβεσσαλώμ·
\par 10 τότε και ο ανδρείος, του οποίου η καρδία είναι ως η καρδία του λέοντος, θέλει παντάπασι νεκρωθή· διότι πας ο Ισραήλ εξεύρει, ότι ο πατήρ σου είναι δυνατός· και οι μετ' αυτού, άνδρες δυνάμεως·
\par 11 διά ταύτα εγώ συμβουλεύω να συναχθή προς σε πας ο Ισραήλ, από Δαν έως Βηρ-σαβεέ, ως η άμμος η παρά την θάλασσαν κατά το πλήθος, και να υπάγης προσωπικώς να πολεμήσης·
\par 12 ούτω θέλομεν επέλθει κατ' αυτού εις όντινα τόπον ευρεθή, και θέλομεν πέσει επ' αυτόν ως πίπτει η δρόσος επί την γήν· ώστε εξ αυτού και εκ πάντων των ανθρώπων των μετ' αυτού δεν θέλει μείνει ουδέ είς·
\par 13 εάν δε καταφύγη εις πόλιν τινά, τότε πας ο Ισραήλ θέλει φέρει κατά της πόλεως εκείνης σχοινία, και θέλομεν σύρει αυτήν έως του χειμάρρου, ώστε να μη μείνη εκεί ουδέ λιθάριον.
\par 14 Και είπεν ο Αβεσσαλώμ και πάντες οι άνδρες Ισραήλ, Καλητέρα είναι η συμβουλή του Χουσαΐ του Αρχίτου παρά την συμβουλήν του Αχιτόφελ. Διότι ο Κύριος διέταξε να διασκεδάση την καλήν συμβουλήν του Αχιτόφελ, διά να επιφέρη ο Κύριος το κακόν επί τον Αβεσσαλώμ.
\par 15 Και είπεν ο Χουσαΐ προς τον Σαδώκ και προς τον Αβιάθαρ, τους ιερείς, Ούτω και ούτω συνεβούλευσεν ο Αχιτόφελ τον Αβεσσαλώμ και τους πρεσβυτέρους του Ισραήλ, και ούτω και ούτω συνεβούλευσα εγώ·
\par 16 τώρα λοιπόν αποστείλατε ταχέως και αναγγείλατε προς τον Δαβίδ, λέγοντες, Μη μείνης την νύκτα ταύτην εν ταις πεδιάσι της ερήμου, αλλά σπεύσον να διαπεράσης, διά να μη καταποθή ο βασιλεύς και πας ο λαός ο μετ' αυτού.
\par 17 Ο δε Ιωνάθαν και ο Αχιμάας ίσταντο πλησίον της Εν-ρωγήλ, διότι δεν ετόλμων να φανώσιν ότι εισήρχοντο εις την πόλιν· και υπήγε παιδίσκη τις και απήγγειλε προς αυτούς το πράγμα· οι δε υπήγαν και απήγγειλαν προς τον βασιλέα Δαβίδ.
\par 18 Νέος τις δε ιδών αυτούς, απήγγειλε προς τον Αβεσσαλώμ· πλην και οι δύο υπήγαν ταχέως και εισήλθον εις την οικίαν τινός εν Βαουρείμ, όστις είχε φρέαρ εν τη αυλή αυτού, και κατέβησαν εκεί.
\par 19 Και η γυνή λαβούσα κάλυμμα εξήπλωσεν επί το στόμιον του φρέατος, και έχυσεν επ' αυτό κοπανισμένον σίτον· ώστε δεν εγνώσθη το πράγμα.
\par 20 Και ελθόντες οι δούλοι του Αβεσσαλώμ εις την οικίαν προς την γυναίκα, είπον, Που είναι ο Αχιμάας και ο Ιωνάθαν; Η δε γυνή είπε προς αυτούς, Διέβησαν το ρυάκιον του ύδατος. Και αφού εζήτησαν και δεν εύρηκαν αυτούς, επέστρεψαν εις Ιερουσαλήμ.
\par 21 Αφού δε εκείνοι ανεχώρησαν, ανέβησαν εκ του φρέατος και υπήγαν και απήγγειλαν προς τον βασιλέα Δαβίδ και είπον προς τον Δαβίδ, Σηκώθητε και περάσατε ταχέως το ύδωρ· διότι ούτω συνεβούλευσεν εναντίον σας ο Αχιτόφελ.
\par 22 Τότε εσηκώθη ο Δαβίδ και πας ο λαός ο μετ' αυτού και διέβησαν τον Ιορδάνην· μέχρι του χαράγματος της ημέρας δεν έλειψεν ουδέ εις εξ αυτών, όστις δεν διέβη τον Ιορδάνην.
\par 23 Ο δε Αχιτόφελ, ιδών ότι η συμβουλή αυτού δεν εξετελέσθη, εσαμάρωσε τον όνον αυτού και σηκωθείς, ανεχώρησε προς τον οίκον αυτού, εις την πόλιν αυτού· και αφού διέταξε τα του οίκου αυτού, εκρεμάσθη και απέθανε και ετάφη εν τω τάφω του πατρός αυτού.
\par 24 Και ο Δαβίδ ήλθεν εις Μαχαναΐμ· ο δε Αβεσσαλώμ διέβη τον Ιορδάνην, αυτός και πάντες οι άνδρες Ισραήλ μετ' αυτού.
\par 25 Και κατέστησεν ο Αβεσσαλώμ αρχιστράτηγον τον Αμασά αντί του Ιωάβ. Ήτο δε ο Αμασά υιός ανδρός ονομαζομένου Ιθρά, Ισραηλίτου, όστις εισήλθε προς την Αβιγαίαν, θυγατέρα του Νάας, αδελφήν Σερουΐας, της μητρός του Ιωάβ.
\par 26 Και εστρατοπέδευσαν ο Ισραήλ και ο Αβεσσαλώμ εν γη Γαλαάδ.
\par 27 Ότε δε ήλθεν ο Δαβίδ εις Μαχαναΐμ, Σωβεί, ο υιός του Νάας από Ραββά εκ των υιών Αμμών, και Μαχείρ, ο υιός του Αμμήλ από Λό-δεβάρ, και Βαρζελλαΐ ο Γαλααδίτης από Ρωγελλίμ,
\par 28 έφεραν κλίνας και λεκάνας και σκεύη πήλινα και σίτον και κριθήν και άλευρον και σίτον πεφρυγανισμένον και κυάμους και φακήν και όσπρια πεφρυγανισμένα,
\par 29 και μέλι και βούτυρον και πρόβατα και τυρούς βοός προς τον Δαβίδ και προς τον λαόν τον μετ' αυτού, διά να φάγωσι διότι είπον, Ο λαός είναι πεινασμένος και εκλελυμένος και διψασμένος εν τη ερήμω.

\chapter{18}

\par 1 Και απηρίθμησεν ο Δαβίδ τον λαόν τον μετ' αυτού, και κατέστησεν επ' αυτούς χιλιάρχους και εκατοντάρχους.
\par 2 Και απέστειλεν ο Δαβίδ τον λαόν, εν τρίτον υπό την χείρα του Ιωάβ, και εν τρίτον υπό την χείρα του Αβισαί, υιού της Σερουΐας, αδελφού του Ιωάβ, και εν τρίτον υπό την χείρα Ιτταΐ του Γετθαίου. Και είπεν ο βασιλεύς προς τον λαόν, Θέλω βεβαίως εξέλθει και εγώ μεθ' υμών.
\par 3 Ο λαός όμως απεκρίθη, Δεν θέλεις εξέλθει διότι, εάν τραπώμεν εις φυγήν, δεν μέλει αυτούς περί ημών· ουδέ εάν το ήμισυ εξ ημών αποθάνη, δεν μέλει αυτούς περί ημών· επειδή τώρα συ είσαι ως ημείς δέκα χιλιάδες· όθεν τώρα είναι καλήτερον να ήσαι βοηθός ημών εκ της πόλεως.
\par 4 Και είπε προς αυτούς ο βασιλεύς, ό,τι σας φαίνεται καλόν, θέλω κάμει. Και εστάθη ο βασιλεύς εις το πλάγιον της πύλης· και πας ο λαός εξήρχετο κατά εκατοντάδας και κατά χιλιάδας.
\par 5 Και προσέταξεν ο βασιλεύς εις τον Ιωάβ και εις τον Αβισαί και εις τον Ιτταΐ, λέγων, Σώσατέ μοι τον νέον, τον Αβεσσαλώμ. Και πας ο λαός ήκουσεν, ενώ ο βασιλεύς προσέταττεν εις πάντας τους άρχοντας υπέρ του Αβεσσαλώμ.
\par 6 Εξήλθε λοιπόν ο λαός εις το πεδίον εναντίον του Ισραήλ· και η μάχη έγεινεν εν τω δάσει Εφραΐμ.
\par 7 Και κατετροπώθη εκεί ο λαός Ισραήλ υπό των δούλων του Δαβίδ· και έγεινεν εκεί την ημέραν εκείνην θραύσις μεγάλη, είκοσι χιλιάδων.
\par 8 διότι η μάχη έγεινεν εκεί διεσπαρμένη επί το πρόσωπον όλου του τόπου· και το δάσος κατέφαγε πλειότερον λαόν, παρ' όσον κατέφαγεν η μάχαιρα, την ημέραν εκείνην.
\par 9 Και συνήντησεν ο Αβεσσαλώμ τους δούλους του Δαβίδ. Και εκάθητο ο Αβεσσαλώμ επί ημιόνου, και εισήλθεν ο ημίονος υπό τους πυκνούς κλάδους μεγάλης δρυός, και επιάσθη η κεφαλή αυτού εις την δρυν, και εκρεμάσθη αναμέσον του ουρανού και της γής· ο δε ημίονος ο υποκάτω αυτού διεπέρασεν.
\par 10 Ιδών δε ανήρ τις, απήγγειλε προς τον Ιωάβ, και είπεν, Ιδού, είδον τον Αβεσσαλώμ κρεμάμενον εις δρυν.
\par 11 Και είπεν ο Ιωάβ προς τον άνδρα, τον απαγγείλαντα προς αυτόν, Και ιδού, είδες, και διά τι πατάξας δεν κατέβαλες αυτόν εκεί εις την γην; βεβαίως ήθελον σοι δώσει δέκα σίκλους αργυρίου και μίαν ζώνην.
\par 12 Ο δε ανήρ είπε προς τον Ιωάβ, Και χίλιοι σίκλοι αργυρίου αν ήθελον μετρηθή εις την παλάμην μου, δεν ήθελον βάλει την χείρα μου επί τον υιόν του βασιλέως· διότι εις επήκοον ημών προσέταξεν ο βασιλεύς εις σε και εις τον Αβισαί και εις τον Ιτταΐ, λέγων, Φυλάχθητε μη εγγίση μηδείς τον νέον, τον Αβεσσαλώμ·
\par 13 αλλά και εάν ήθελον πράξει δολίως εναντίον της ζωής μου, δεν κρύπτεται ουδέν από του βασιλέως· και συ ήθελες σταθή εναντίος.
\par 14 Τότε είπεν ο Ιωάβ, Δεν πρέπει να χρονοτριβώ ούτω μετά σου. Και λαβών εις την χείρα αυτού τρία βέλη, διεπέρασεν αυτά διά της καρδίας του Αβεσσαλώμ, ενώ έτι έζη εν τω μέσω της δρυός.
\par 15 Και περικυκλώσαντες δέκα νέοι, οι βαστάζοντες τα όπλα του Ιωάβ, επάταξαν τον Αβεσσαλώμ και εθανάτωσαν αυτόν.
\par 16 Και εσάλπισεν ο Ιωάβ διά της σάλπιγγος, και επέστρεψεν ο λαός από του να καταδιώκη οπίσω του Ισραήλ· διότι ανεχαίτισεν ο Ιωάβ τον λαόν.
\par 17 Και λαβόντες τον Αβεσσαλώμ, έρριψαν αυτόν εις λάκκον μέγαν εντός του δάσους· και έστησαν επ' αυτόν σωρόν λίθων μέγαν σφόδρα· και πας ο Ισραήλ έφυγεν έκαστος εις την σκηνήν αυτού.
\par 18 Έτι δε ζων ο Αβεσσαλώμ είχε λάβει και στήσει δι' εαυτόν στήλην, την εν τη κοιλάδι του βασιλέως· διότι είπεν, Δεν έχω υιόν διά να διατηρή την μνήμην του ονόματός μου· και εκάλεσε την στήλην με το όνομα αυτού· και καλείται έως της ημέρας ταύτης Στήλη του Αβεσσαλώμ.
\par 19 Τότε είπεν Αχιμάας ο υιός του Σαδώκ, Ας τρέξω τώρα και ας φέρω προς τον βασιλέα αγγελίας, ότι ο Κύριος εξεδίκησεν αυτόν εκ χειρός των εχθρών αυτού.
\par 20 Και είπε προς αυτόν ο Ιωάβ, Δεν θέλεις είσθαι την ημέραν ταύτην αγγελιαφόρος, αλλ' εις άλλην ημέραν θέλεις φέρει αγγελίας· εις ταύτην δε την ημέραν δεν θέλεις φέρει αγγελίας, επειδή ο υιός του βασιλέως απέθανε.
\par 21 Τότε είπεν ο Ιωάβ προς τον Χουσεί, Ύπαγε, απάγγειλον προς τον βασιλέα όσα είδες. Και ο Χουσεί προσεκύνησε τον Ιωάβ και έτρεξε.
\par 22 Τότε Αχιμάας ο υιός του Σαδώκ είπε πάλιν προς τον Ιωάβ, Αλλ' ό,τι και αν ήναι, ας τρέξω και εγώ, παρακαλώ, κατόπιν του Χουσεί. Ο δε Ιωάβ είπε, Διά τι θέλεις να τρέξης, τέκνον μου, ενώ δεν έχεις αρμοδίους αγγελίας;
\par 23 Αλλ' ό,τι και αν ήναι, είπεν, ας τρέξω. Τότε είπε προς αυτόν, Τρέχε. Και έτρεξεν ο Αχιμάας διά της οδού της πεδιάδος και επέρασε τον Χουσεί.
\par 24 Εκάθητο δε ο Δαβίδ μεταξύ των δύο πυλών· και ανέβη ο σκοπός εις το δώμα της πύλης, επί το τείχος, και υψώσας τους οφθαλμούς αυτού, είδε, και ιδού, άνθρωπος τρέχων μόνος.
\par 25 Και ανεβόησεν ο σκοπός και απήγγειλε προς τον βασιλέα. Και ο βασιλεύς είπεν, Εάν ήναι μόνος, έχει αγγελίας εις το στόμα αυτού. Και ήρχετο προχωρών και επλησίαζε.
\par 26 Και είδεν ο σκοπός άλλον άνθρωπον τρέχοντα· και ανεβόησεν ο σκοπός προς τον θυρωρόν, και είπεν, Ιδού, άλλος άνθρωπος τρέχων μόνος. Και είπεν ο βασιλεύς, Και ούτος είναι αγγελιαφόρος.
\par 27 Και είπεν ο σκοπός, Το τρέξιμον του πρώτου μοι φαίνεται ως το τρέξιμον του Αχιμάας, υιού του Σαδώκ. Και είπεν ο βασιλεύς, Καλός άνθρωπος είναι ούτος και έρχεται με αγαθάς αγγελίας.
\par 28 Και εβόησεν ο Αχιμάας και είπε προς τον βασιλέα, Χαίρε. και προσεκύνησε τον βασιλέα κατά πρόσωπον αυτού έως εδάφους· και είπεν, Ευλογητός Κύριος ο Θεός σου, όστις παρέδωκε τους ανθρώπους, τους σηκώσαντας την χείρα αυτών κατά του κυρίου μου του βασιλέως.
\par 29 Και είπεν ο βασιλεύς, Υγιαίνει ο νέος, ο Αβεσσαλώμ; Και απεκρίθη ο Αχιμάας, Ότε ο Ιωάβ απέστελλε τον δούλον του βασιλέως, και εμέ τον δούλον σου, είδον τον μέγαν θόρυβον, πλην δεν ήξευρον τι ήτο.
\par 30 Και είπεν ο βασιλεύς, Στρέψον, στάθητι εκεί. Και εστράφη και εστάθη.
\par 31 Και ιδού, ήλθεν ο Χουσεί· και είπεν ο Χουσεί, Αγγελίας, κύριέ μου βασιλεύ· διότι ο Κύριος σε εξεδίκησε την ημέραν ταύτην εκ χειρός πάντων των επανισταμένων επί σε.
\par 32 Και είπεν ο βασιλεύς προς τον Χουσεί, Υγιαίνει ο νέος, ο Αβεσσαλώμ; Και απεκρίθη ο Χουσεί, είθε να γείνωσιν ως ο νέος εκείνος οι εχθροί του κυρίου μου του βασιλέως, και πάντες οι επανιστάμενοι επί σε διά κακόν.
\par 33 Και εταράχθη ο βασιλεύς και ανέβη εις το υπερώον της πύλης, και έκλαυσε· και ενώ επορεύετο, έλεγεν ούτως· Υιέ μου Αβεσσαλώμ, υιέ μου, υιέ μου Αβεσσαλώμ· είθε να απέθνησκον εγώ αντί σου, Αβεσσαλώμ, υιέ μου, υιέ μου.

\chapter{19}

\par 1 Και ανηγγέλθη προς τον Ιωάβ, Ιδού, ο βασιλεύς κλαίει και πενθεί διά τον Αβεσσαλώμ.
\par 2 Και εν τη ημέρα εκείνη η σωτηρία μετεβλήθη εις πένθος εν παντί τω λαώ· διότι ήκουσεν ο λαός να λέγωσιν εν τη ημέρα εκείνη, Ο βασιλεύς είναι περίλυπος διά τον υιόν αυτού.
\par 3 Και εισήρχετο ο λαός εν τη ημέρα εκείνη κρυφίως εις την πόλιν, ως λαός όστις κρύπτεται αισχυνόμενος, όταν εν τη μάχη τραπή εις φυγήν.
\par 4 Ο δε βασιλεύς εκάλυψε το πρόσωπον αυτού, και εβόα ο βασιλεύς εν φωνή μεγάλη, Υιέ μου Αβεσσαλώμ, Αβεσσαλώμ, υιέ μου, υιέ μου.
\par 5 Και εισελθών ο Ιωάβ εις τον οίκον προς τον βασιλέα, είπε, Κατήσχυνας σήμερον τα πρόσωπα πάντων των δούλων σου, οίτινες έσωσαν σήμερον την ζωήν σου και την ζωήν των υιών σου και των θυγατέρων σου και την ζωήν των γυναικών σου και την ζωήν των παλλακών σου·
\par 6 επειδή αγαπάς τους μισούντάς σε και μισείς τους αγαπώντάς σε· διότι έδειξας σήμερον, ότι δεν είναι παρά σοι ουδέν οι άρχοντές σου και οι δούλοί σου· διότι σήμερον εγνώρισα, ότι εάν ο Αβεσσαλώμ έζη και ημείς πάντες απεθνήσκομεν σήμερον, τότε ήθελεν είσθαι αρεστόν εις σέ·
\par 7 τώρα λοιπόν σηκώθητι, έξελθε και λάλησον κατά την καρδίαν των δούλων σου· διότι ομνύω εις τον Κύριον, εάν δεν εξέλθης, δεν θέλει μείνει μετά σου την νύκτα ταύτην ουδέ είς· και τούτο θέλει είσθαι εις σε χειρότεραν υπέρ πάντα τα κακά, όσα ήλθον επί σε εκ νεότητός σου μέχρι του νυν.
\par 8 Τότε εσηκώθη ο βασιλεύς και εκάθησεν εν τη πύλη. Και ανήγγειλαν προς πάντα τον λαόν, λέγοντες, Ιδού, ο βασιλεύς κάθηται εν τη πύλη. Και ήλθε πας ο λαός έμπροσθεν του βασιλέως. Ο δε Ισραήλ έφυγεν έκαστος εις την σκηνήν αυτού.
\par 9 Και ήτο πας ο λαός εις έριδα κατά πάσας τας φυλάς του Ισραήλ, λέγοντες, Ο βασιλεύς έσωσεν ημάς εκ χειρός των εχθρών ημών· και αυτός ηλευθέρωσεν ημάς εκ χειρός των Φιλισταίων· και τώρα έφυγεν εκ του τόπου εξ αιτίας του Αβεσσαλώμ·
\par 10 ο δε Αβεσσαλώμ, τον οποίον εχρίσαμεν βασιλέα εφ' ημάς, απέθανεν εν τη μάχη· τώρα λοιπόν διά τι δεν λαλείτε να επιστρέψωμεν τον βασιλέα;
\par 11 Και απέστειλεν ο βασιλεύς Δαβίδ προς τον Σαδώκ και προς τον Αβιάθαρ, τους ιερείς, λέγων, Λαλήσατε προς τους πρεσβυτέρους του Ιούδα, λέγοντες, Διά τι είσθε οι έσχατοι εις το να επιστρέψητε τον βασιλέα εις τον οίκον αυτού; διότι οι λόγοι παντός του Ισραήλ έφθασαν προς τον βασιλέα εις τον οίκον αυτού·
\par 12 σεις είσθε αδελφοί μου, σεις οστά μου και σαρξ μου· διά τι λοιπόν είσθε οι έσχατοι εις το να επιστρέψητε τον βασιλέα;
\par 13 προς τον Αμασά μάλιστα είπατε, Δεν είσαι συ οστούν μου και σαρξ μου; ούτω να κάμη ο Θεός εις εμέ και ούτω να προσθέση, εάν δεν γείνης αρχιστράτηγος πάντοτε έμπροσθέν μου αντί του Ιωάβ.
\par 14 Και έκλινε την καρδίαν πάντων των ανδρών Ιούδα ως ενός ανθρώπου· και απέστειλαν προς τον βασιλέα, λέγοντες, Επίστρεψον συ και πάντες οι δούλοί σου.
\par 15 Επέστρεψε λοιπόν ο βασιλεύς και ήλθεν έως του Ιορδάνου. Και ο Ιούδας ήλθεν εις Γάλγαλα, διά να υπάγη εις συνάντησιν του βασιλέως, να διαβιβάση τον βασιλέα διά του Ιορδάνου.
\par 16 Έσπευσε δε Σιμεΐ ο υιός του Γηρά, ο Βενιαμίτης, εκ Βαουρείμ, και κατέβη μετά των ανδρών Ιούδα εις συνάντησιν του βασιλέως Δαβίδ.
\par 17 Και ήσαν μετ' αυτού χίλιοι άνδρες εκ του Βενιαμίν, και Σιβά ο δούλος του οίκου του Σαούλ, και οι δεκαπέντε υιοί αυτού και είκοσι δούλοι αυτού μετ' αυτού· και διέβησαν τον Ιορδάνην ενώπιον του βασιλέως.
\par 18 Έπειτα επέρασεν η λέμβος διά να διαβιβάση την οικογένειαν του βασιλέως, και να κάμη ό,τι ήθελε φανή εις αυτόν αρεστόν. Και Σιμεΐ ο υιός του Γηρά έπεσεν ενώπιον του βασιλέως, ενώ διέβαινε τον Ιορδάνην·
\par 19 και είπε προς τον βασιλέα, Ας μη λογαριάση ο κύριός μου ανομίαν εις εμέ, και μη ενθυμηθής την ανομίαν, την οποίαν έπραξεν ο δούλός σου, καθ' ην ημέραν εξήρχετο ο κύριός μου ο βασιλεύς εξ Ιερουσαλήμ, ώστε να βάλη τούτο ο βασιλεύς εν τη καρδία αυτού·
\par 20 διότι ο δούλός σου εγνώρισεν ότι εγώ ήμαρτον· και ιδού εγώ ήλθον σήμερον πρότερος παντός του οίκου Ιωσήφ, διά να καταβώ εις συνάντησιν του κυρίου μου του βασιλέως.
\par 21 Και απεκρίθη ο Αβισαί ο υιός της Σερουΐας, λέγων, Δεν πρέπει ο Σιμεΐ να θανατωθή διά τούτο, διότι κατηράσθη τον κεχρισμένον του Κυρίου;
\par 22 Αλλ' ο Δαβίδ είπε, Τι μεταξύ εμού και υμών, υιοί της Σερουΐας, ώστε γίνεσθε σήμερον επίβουλοι εις εμέ; πρέπει την ημέραν ταύτην να θανατωθή άνθρωπος εν Ισραήλ; διότι δεν γνωρίζω εγώ ότι σήμερον είμαι βασιλεύς επί τον Ισραήλ;
\par 23 Και είπεν ο βασιλεύς προς τον Σιμεΐ, Δεν θέλεις αποθάνει. Και ώμοσε προς αυτόν ο βασιλεύς.
\par 24 Και Μεμφιβοσθέ, ο υιός του Σαούλ, κατέβη εις συνάντησιν του βασιλέως· και ούτε τους πόδας αυτού είχε νίψει ούτε τον πώγωνα αυτού ευπρεπίσει ούτε τα ιμάτια αυτού είχε πλύνει, αφ' ης ημέρας ο βασιλεύς ανεχώρησε μέχρι της ημέρας καθ' ην επέστρεψεν εν ειρήνη.
\par 25 Και ότε ήλθεν εις Ιερουσαλήμ προς συνάντησιν του βασιλέως, ο βασιλεύς είπε προς αυτόν, Διά τι δεν ήλθες μετ' εμού, Μεμφιβοσθέ;
\par 26 Ο δε απεκρίθη, Κύριέ μου βασιλεύ, ο δούλός μου με ηπάτησε· διότι ο δούλός σου είπε, Θέλω στρώσει δι' εμαυτόν τον όνον, και θέλω αναβή επ' αυτόν και υπάγει προς τον βασιλέα· διότι ο δούλός σου είναι χωλός·
\par 27 και εσυκοφάντησε τον δούλον σου προς τον κύριόν μου τον βασιλέα· πλην ο κύριός μου ο βασιλεύς είναι ως άγγελος Θεού· κάμε λοιπόν το αρεστόν εις τους οφθαλμούς σου·
\par 28 διότι πας ο οίκος του πατρός μου δεν ήτο παρά άξιος θανάτου ενώπιον του κυρίου μου του βασιλέως· συ όμως κατέταξας τον δούλον σου μεταξύ εκείνων οίτινες έτρωγον επί της τραπέζης σου· και τι δίκαιον έχω εγώ πλέον, και διά τι να παραπονώμαι έτι προς τον βασιλέα;
\par 29 Και είπε προς αυτόν ο βασιλεύς, Διά τι λαλείς έτι περί των πραγμάτων σου; εγώ είπα, Συ και ο Σιβά διαμοιράσθητε τους αγρούς.
\par 30 Και είπεν ο Μεμφιβοσθέ προς τον βασιλέα, Και τα πάντα ας λάβη, αφού ο κύριός μου ο βασιλεύς επέστρεψεν εις τον οίκον αυτού εν ειρήνη.
\par 31 Και ο Βαρζελλαΐ ο Γαλααδίτης κατέβη από Ρωγελλίμ και διέβη τον Ιορδάνην μετά του βασιλέως, διά να συμπροπέμψη αυτόν έως πέραν του Ιορδάνου.
\par 32 Ήτο δε ο Βαρζελλαΐ άνθρωπος γέρων σφόδρα, ογδοήκοντα ετών ηλικίας· και διέτρεφε τον βασιλέα, ότε εκάθητο εν Μαχαναΐμ· διότι ήτο άνθρωπος μέγας σφόδρα.
\par 33 Και είπεν ο βασιλεύς προς τον Βαρζελλαΐ, Διάβα συ μετ' εμού, και θέλω σε τρέφει μετ' εμού εν Ιερουσαλήμ.
\par 34 Ο δε Βαρζελλαΐ είπε προς τον βασιλέα, Πόσαι είναι αι ημέραι των ετών της ζωής μου, ώστε να αναβώ μετά του βασιλέως εις Ιερουσαλήμ;
\par 35 είμαι σήμερον ογδοήκοντα ετών ηλικίας· δύναμαι να κάμω διάκρισιν μεταξύ καλού και κακού; δύναται ο δούλός σου να αισθανθή τι τρώγω, ή τι πίνω; δύναμαι να ακούσω πλέον την φωνήν των αδόντων ή των αδουσών; διά τι λοιπόν ο δούλός σου να ήναι έτι και φορτίον εις τον κύριόν μου τον βασιλέα;
\par 36 ο δούλός σου θέλει διαβή τον Ιορδάνην μετά του βασιλέως μέχρις ολίγου διαστήματος· και διά τι ο βασιλεύς ήθελε κάμει εις εμέ την ανταπόδοσιν ταύτην;
\par 37 ας επιστρέψη ο δούλός σου, παρακαλώ, διά να αποθάνω εν τη πόλει μου και να ενταφιασθώ πλησίον του τάφου του πατρός μου και της μητρός μου· πλην ιδού, ο δούλός σου Χιμάμ· ας διαβή μετά του κυρίου μου του βασιλέως· και κάμε εις αυτόν ό,τι φανή αρεστόν εις τους οφθαλμούς σου.
\par 38 Και είπεν ο βασιλεύς, Μετ' εμού θέλει διαβή ο Χιμάμ, και εγώ θέλω κάμει εις αυτόν ό,τι φαίνεται αρεστόν εις τους οφθαλμούς σου· και εις σε θέλω κάμει παν ό,τι ζητήσης παρ' εμού.
\par 39 Και διέβη πας ο λαός τον Ιορδάνην. Και ότε διέβη ο βασιλεύς, κατεφίλησεν ο βασιλεύς τον Βαρζελλαΐ και ευλόγησεν αυτόν· ο δε επέστρεψεν εις τον τόπον αυτού.
\par 40 Τότε διέβη ο βασιλεύς εις Γάλγαλα, και ο Χιμάμ διέβη μετ' αυτού· και πας ο λαός του Ιούδα και έτι το ήμισυ του λαού Ισραήλ διεβίβασαν τον βασιλέα.
\par 41 Και ιδού, πάντες οι άνδρες Ισραήλ ήλθον προς τον βασιλέα και είπον προς τον βασιλέα, Διά τι σε έκλεψαν οι αδελφοί ημών, οι άνδρες Ιούδα, και διεβίβασαν τον βασιλέα και την οικογένειαν αυτού, διά του Ιορδάνου, και πάντας τους άνδρας του Δαβίδ μετ' αυτού;
\par 42 Και απεκρίθησαν πάντες οι άνδρες Ιούδα προς τους άνδρας Ισραήλ, Διότι ο βασιλεύς είναι συγγενής ημών· και τι θυμόνετε διά το πράγμα τούτο; μήπως εφάγομεν τι εκ του βασιλέως; ή έδωκεν εις ημάς δώρον;
\par 43 Και απεκρίθησαν οι άνδρες Ισραήλ προς τους άνδρας Ιούδα και είπον, Ημείς έχομεν δέκα μέρη εις τον βασιλέα, και μάλιστα έχομεν εις τον Δαβίδ πλειότερον παρά σείς· διά τι λοιπόν περιφρονείτε ημάς; και δεν ελαλήσαμεν ημείς πρώτοι μεταξύ ημών περί της επιστροφής του βασιλέως ημών; Και οι λόγοι των ανδρών Ιούδα ήσαν σκληρότεροι παρά τους λόγους των ανδρών Ισραήλ.

\chapter{20}

\par 1 Συνέπεσε δε να ήναι εκεί άνθρωπός τις διεστραμμένος, ονομαζόμενος Σεβά, υιός του Βιχρεί, Βενιαμίτης· και εσάλπισε διά της σάλπιγγος και είπε, Δεν έχομεν ημείς μέρος εις τον Δαβίδ, ουδέ έχομεν κληρονομίαν εις τον υιόν του Ιεσσαί· Ισραήλ, εις τας σκηνάς αυτού έκαστος.
\par 2 Και ανέβη πας ανήρ Ισραήλ από όπισθεν του Δαβίδ, και ηκολούθησε Σεβά τον υιόν του Βιχρεί· οι δε άνδρες Ιούδα έμειναν προσκεκολλημένοι εις τον βασιλέα αυτών, από του Ιορδάνου έως Ιερουσαλήμ.
\par 3 Και ήλθεν ο Δαβίδ εις τον οίκον αυτού εις Ιερουσαλήμ· και έλαβεν ο βασιλεύς τας δέκα γυναίκας τας παλλακάς, τας οποίας είχεν αφήσει διά να φυλάττωσι τον οίκον, και έβαλεν αυτάς εις οίκον φυλάξεως και έτρεφεν αυτάς· πλην δεν εισήλθε προς αυτάς· και έμειναν αποκεκλεισμέναι μέχρι της ημέρας του θανάτου αυτών, ζώσαι εν χηρεία.
\par 4 Είπε δε ο βασιλεύς προς τον Αμασά, Σύναξον εις εμέ τους άνδρας Ιούδα εντός τριών ημερών, και συ να παρευρεθής ενταύθα.
\par 5 Και υπήγεν ο Αμασά να συνάξη τον Ιούδαν· εβράδυνεν όμως υπέρ τον ωρισμένον καιρόν, τον οποίον είχε διορίσει εις αυτόν.
\par 6 Και είπεν ο Δαβίδ προς τον Αβισαί, Τώρα ο Σεβά ο υιός του Βιχρεί θέλει κάμει εις ημάς μεγαλήτερον κακόν παρά τον Αβεσσαλώμ· λάβε συ τους δούλους του κυρίου σου και καταδίωξον οπίσω αυτού, διά να μη εύρη εις εαυτόν πόλεις οχυράς και διασωθή απ' έμπροσθεν ημών.
\par 7 Και εξήλθον οπίσω αυτού οι άνδρες του Ιωάβ και οι Χερεθαίοι και οι Φελεθαίοι και πάντες οι δυνατοί· και εξήλθον από Ιερουσαλήμ, διά να καταδιώξωσιν οπίσω του Σεβά, υιού του Βιχρεί.
\par 8 Ότε έφθασαν πλησίον της μεγάλης πέτρας, της εν Γαβαών, ο Αμασά ήλθεν εις συνάντησιν αυτών. Ο δε Ιωάβ είχε περιεζωσμένον το ιμάτιον, το οποίον ήτο ενδεδυμένος, και επ' αυτό περιεζωσμένην την μάχαιραν, κρεμαμένην εις την οσφύν αυτού εν τη θήκη αυτής· και καθώς εξήλθεν αυτός, έπεσε.
\par 9 Και είπεν ο Ιωάβ προς τον Αμασά, Υγιαίνεις, αδελφέ μου; Και επίασεν ο Ιωάβ τον Αμασά με την δεξιάν αυτού χείρα από του πώγωνος, διά να φιλήση αυτόν.
\par 10 Ο δε Αμασά δεν εφυλάχθη την μάχαιραν, ήτις ήτο εν τη χειρί του Ιωάβ· και ο Ιωάβ επάταξεν αυτόν δι' αυτής εις την πέμπτην πλευράν, και έχυσε τα εντόσθια αυτού κατά γης και δεν εδευτέρωσεν εις αυτόν· και απέθανε. Τότε ο Ιωάβ και Αβισαί ο αδελφός αυτού κατεδίωξαν οπίσω του Σεβά, υιού του εν Βιχρεί.
\par 11 Εις δε εκ των ανθρώπων του Ιωάβ εστάθη πλησίον του Αμασά και είπεν, Όστις αγαπά τον Ιωάβ, και όστις είναι του Δαβίδ, ας ακολουθή τον Ιωάβ.
\par 12 Ο δε Αμασά έκειτο αιματοκυλισμένος εκ μέσω της οδού. Και ότε είδεν ούτος ο ανήρ ότι πας ο λαός ίστατο, έσυρε τον Αμασά εκ της οδού εις τον αγρόν, και έρριψεν επ' αυτόν ιμάτιον, καθώς είδεν ότι πας ο ερχόμενος προς αυτόν ίστατο.
\par 13 Αφού μετετοπίσθη εκ της οδού, ο πας ο λαός επέρασεν οπίσω του Ιωάβ, διά να καταδιώξωσι τον Σεβά, υιόν του Βιχρεί.
\par 14 Εκείνος δε διήλθε διά πασών των φυλών του Ισραήλ εις Αβέλ και εις Βαιθ-μααχά, μετά πάντων των Βηριτών, οίτινες συνήχθησαν ομού και ηκολούθησαν αυτόν και αυτοί.
\par 15 Τότε ήλθον και επολιόρκησαν αυτόν εν Αβέλ-βαίθ-μααχά, και ύψωσαν πρόχωμα εναντίον της πόλεως, στήσαντες αυτό πλησίον του προτειχίσματος, και πας ο λαός, ο μετά του Ιωάβ, διώρυσσον το τείχος διά να κρημνίσωσιν αυτό.
\par 16 Τότε γυνή τις σοφή εβόησεν εκ της πόλεως, Ακούσατε, ακούσατε· είπατε, παρακαλώ, προς τον Ιωάβ, Πλησίασον έως ενταύθα, και θέλω λαλήσει προς σε.
\par 17 Και ότε επλησίασεν εις αυτήν, η γυνή είπε, Συ είσαι ο Ιωάβ; Ο δε απεκρίθη, Εγώ. Τότε είπε προς αυτόν, Άκουσον τους λόγους της δούλης σου. Και απεκρίθη, Ακούω.
\par 18 Και είπε, λέγουσα, Εσυνείθιζον να λέγωσι τον παλαιόν καιρόν, λέγοντες, Ας υπάγωσι να ζητήσωσι συμβουλήν εις Αβέλ· και ούτως ετελείοναν την υπόθεσιν·
\par 19 εγώ είμαι εκ των ειρηνικών και πιστών του Ισραήλ· συ ζητείς να καταστρέψης πόλιν, μάλιστα μητρόπολιν μεταξύ του Ισραήλ· διά τι θέλεις να αφανίσης την κληρονομίαν του Κυρίου;
\par 20 Και αποκριθείς ο Ιωάβ, είπε, Μη γένοιτο, μη γένοιτο εις εμέ να αφανίσω ή να καταστρέψω
\par 21 το πράγμα δεν είναι ούτως· αλλά ανήρ τις εκ του όρους Εφραΐμ, ονομαζόμενος Σεβά, υιός Βιχρεί, εσήκωσε την χείρα αυτού κατά του βασιλέως, κατά του Δαβίδ· παράδος αυτόν μόνον, και θέλω αναχωρήσει από της πόλεως. Και είπεν η γυνή προς τον Ιωάβ, Ιδού, η κεφαλή αυτού θέλει ριφθή προς σε από του τείχους.
\par 22 Και ήλθεν η γυνή προς πάντα τον λαόν λαλούσα εν τη σοφία αυτής. Και έκοψαν την κεφαλήν του Σεβά, υιού του Βιχρεί, και έρριψαν προς τον Ιωάβ. Τότε εσάλπισε διά της σάλπιγγος και διεκορπίσθησαν από της πόλεως, έκαστος εις την σκηνήν αυτού. Και ο Ιωάβ έστρεψεν εις Ιερουσαλήμ προς τον βασιλέα.
\par 23 Ήτο δε ο Ιωάβ επί παντός του στρατεύματος του Ισραήλ· ο δε Βεναΐας, ο υιός του Ιωδαέ, επί των Χερεθαίων και επί των Φελεθαίων·
\par 24 και Αδωράμ ήτο επί των φόρων· και Ιωσαφάτ, ο υιός του Αχιλούδ, υπομνηματογράφος·
\par 25 και ο Σεβά, Γραμματεύς· ο δε Σαδώκ και Αβιάθαρ, ιερείς·
\par 26 και έτι Ιράς, ο Ιαειρίτης, ήτο αυλάρχης πλησίον του Δαβίδ.

\chapter{21}

\par 1 Έγεινε δε πείνα εν ταις ημέραις του Δαβίδ τρία έτη κατά συνέχειαν· και ηρώτησεν ο Δαβίδ τον Κύριον· και ο Κύριος απεκρίθη, Τούτο έγεινεν εξ αιτίας του Σαούλ και του φονικού οίκου αυτού, διότι εθανάτωσε τους Γαβαωνίτας.
\par 2 Και εκάλεσεν ο βασιλεύς τους Γαβαωνίτας και είπε προς αυτούς· οι δε Γαβαωνίται δεν ήσαν των υιών Ισραήλ, αλλ' εκ των εναπολειφθέντων Αμορραίων· και οι υιοί Ισραήλ είχον ομόσει προς αυτούς· ο δε Σαούλ εζήτησε να θανατώση αυτούς από του ζήλου αυτού προς τους υιούς Ισραήλ και Ιούδα.
\par 3 Ο Δαβίδ λοιπόν είπε προς τους Γαβαωνίτας, Τι θέλω κάμει εις εσάς; και με τι θέλω κάμει εξιλέωσιν, διά να ευλογήσητε την κληρονομίαν του Κυρίου;
\par 4 Οι δε Γαβαωνίται είπον προς αυτόν, Ημείς ούτε περί αργυρίου ούτε περί χρυσίου έχομεν να κάμωμεν μετά του Σαούλ ή μετά του οίκου αυτού· ουδέ ζητούμεν να θανατώσης διά ημάς άνθρωπον εκ του Ισραήλ. Και είπεν, ό,τι είπητε, θέλω κάμει εις εσάς.
\par 5 Και απεκρίθησαν προς τον βασιλέα, Του ανθρώπου, όστις ηφάνισεν ημάς και όστις εμηχανεύθη να εξολοθρεύση ημάς, ώστε να μη υπάρχωμεν εις ουδέν εκ των ορίων του Ισραήλ,
\par 6 ας παραδοθώσιν εις ημάς επτά άνθρωποι εκ των υιών αυτού, και θέλομεν κρεμάσει αυτούς προς τον Κύριον εν Γαβαά του Σαούλ, του εκλεκτού του Κυρίου. Και είπεν ο βασιλεύς, Εγώ θέλω παραδώσει αυτούς.
\par 7 Τον Μεμφιβοσθέ όμως, τον υιόν του Ιωνάθαν, υιού του Σαούλ, εφείσθη ο βασιλεύς, διά τον όρκον του Κυρίου τον μεταξύ αυτών, μεταξύ του Δαβίδ και Ιωνάθαν υιού του Σαούλ.
\par 8 Έλαβε δε ο βασιλεύς τους δύο υιούς της Ρεσφά, θυγατρός του Αϊά, τους οποίους εγέννησεν εις τον Σαούλ, τον Αρμονεί και Μεμφιβοσθέ· και τους πέντε υιούς της Μιχάλ, θυγατρός του Σαούλ, τους οποίους εγέννησεν εις τον Αδριήλ, υιόν του Βαρζελλαΐ του Μεωλαθίτου·
\par 9 και παρέδωκεν αυτούς εις τας χείρας των Γαβαωνιτών, και εκρέμασαν αυτούς εις τον λόφον ενώπιον του Κυρίου· και έπεσον ομού και οι επτά και εθανατώθησαν εν ταις ημέραις του θερισμού, εν ταις πρώταις, κατά την αρχήν του θερισμού των κριθών.
\par 10 Η δε Ρεσφά, η θυγάτηρ του Αϊά, έλαβε σάκκον και έστρωσεν αυτόν εις εαυτήν επί τον βράχον, από της αρχής του θερισμού εωσού έσταξεν επ' αυτών ύδωρ εκ του ουρανού, και δεν άφινεν ούτε τα πετεινά του ουρανού να καθίσωσιν επ' αυτών την ημέραν ούτε τα θηρία του αγρού την νύκτα.
\par 11 Και ανηγγέλθη προς τον Δαβίδ τι έκαμεν η Ρεσφά, η θυγάτηρ του Αϊά, παλλακή του Σαούλ.
\par 12 Και υπήγεν ο Δαβίδ και έλαβε τα οστά του Σαούλ και τα οστά του Ιωνάθαν του υιού αυτού, παρά των ανδρών της Ιαβείς-γαλαάδ, οίτινες είχον κλέψει αυτά εκ της πλατείας Βαιθ-σαν, όπου οι Φιλισταίοι εκρέμασαν αυτούς, καθ' ην ημέραν οι Φιλισταίοι εθανάτωσαν τον Σαούλ εν Γελβουέ·
\par 13 και ανεβίβασεν εκείθεν τα οστά του Σαούλ και τα οστά Ιωνάθαν του υιού αυτού· και εσύναξαν τα οστά των κρεμασθέντων.
\par 14 Και έθαψαν τα οστά του Σαούλ και Ιωνάθαν του υιού αυτού εν γη Βενιαμίν εν Σηλά, εν τω τάφω του Κείς, του πατρός αυτού· και έκαμον πάντα όσα προσέταξεν ο βασιλεύς. Και μετά ταύτα εξιλεώθη ο Θεός προς την γην.
\par 15 Έγεινε δε πάλιν πόλεμος των Φιλισταίων μετά του Ισραήλ· και κατέβη ο Δαβίδ και οι δούλοι αυτού μετ' αυτού και επολέμησαν εναντίον των Φιλισταίων, και απέκαμεν ο Δαβίδ.
\par 16 Ο δε Ισβί-βενώβ, ο εκ των τέκνων του Ραφά, του οποίου της λόγχης το βάρος ήτο τριακόσιοι σίκλοι χαλκού, όστις ήτο περιεζωσμένος ρομφαίαν νέαν, εσκόπευε να θανατώση τον Δαβίδ.
\par 17 Εβοήθησεν όμως αυτόν Αβισαί, ο υιός της Σερουΐας, και επάταξε τον Φιλισταίον και εθανάτωσεν αυτόν. Τότε οι άνδρες του Δαβίδ ώμοσαν προς αυτόν, λέγοντες, Δεν θέλεις εξέλθει πλέον μεθ' ημών εις πόλεμον, διά να μη σβέσης τον λύχνον του Ισραήλ.
\par 18 Μετά δε ταύτα έγεινε πάλιν πόλεμος μετά των Φιλισταίων εν Γωβ, εν τω οποίω Σιββεχαΐ ο Χουσαθίτης εθανάτωσε τον Σαφ, όστις ήτο εκ των τέκνων του Ραφά·
\par 19 Και πάλιν έγεινε πόλεμος εν Γωβ μετά των Φιλισταίων, και ο Ελχανάν ο υιός του Ιαρέ-ορεγείμ, Βηθλεεμίτης, εθανάτωσε τον αδελφόν του Γολιάθ του Γετθαίου, και το ξύλον της λόγχης αυτού ήτο ως αντίον υφαντού.
\par 20 Έγεινεν έτι πόλεμος εν Γαθ, και ήτο ανήρ υπερμεγέθης, και οι δάκτυλοι των χειρών αυτού και οι δάκτυλοι των ποδών αυτού ήσαν εξ και εξ, εικοσιτέσσαρες τον αριθμόν· και ούτος έτι ήτο εκ της γενεάς του Ραφά.
\par 21 Και ωνείδισε τον Ισραήλ· και Ιωνάθαν ο υιός του Σαμαά, αδελφού του Δαβίδ, επάταξεν αυτόν.
\par 22 Οι τέσσαρες ούτοι εγεννήθησαν εις τον Ραφά εν Γαθ, και έπεσον διά χειρός του Δαβίδ και διά χειρός των δούλων αυτού.

\chapter{22}

\par 1 Και ελάλησεν ο Δαβίδ προς τον Κύριον τους λόγους της ωδής ταύτης, καθ' ην ημέραν ο Κύριος ηλευθέρωσεν αυτόν εκ χειρός πάντων των εχθρών αυτού και εκ χειρός του Σαούλ·
\par 2 και είπεν, Ο Κύριος είναι πέτρα μου και φρούριόν μου και ελευθερωτής μου·
\par 3 ο Θεός είναι ο βράχος μου· επ' αυτόν θέλω ελπίζει· η ασπίς μου και το κέρας της σωτηρίας μου, ο υψηλός πύργος μου και η καταφυγή μου, ο σωτήρ μου· συ έσωσάς με εκ της αδικίας.
\par 4 Θέλω επικαλεσθή τον αξιΰμνητον Κύριον, και εκ των εχθρών μου θέλω σωθή.
\par 5 Ότε του θανάτου τα κύματα με περιεκύκλωσαν, χείμαρροι ανομίας με κατετρόμαξαν,
\par 6 οι πόνοι του άδου με περιεκύκλωσαν, αι παγίδες του θανάτου με έφθασαν,
\par 7 εν τη στενοχωρία μου επεκαλέσθην τον Κύριον, και προς τον Θεόν μου εβόησα· και ήκουσε της φωνής μου εκ του ναού αυτού, και η κραυγή μου ήλθεν εις τα ώτα αυτού.
\par 8 Τότε εσαλεύθη και έντρομος έγεινεν η γή· τα θεμέλια του ουρανού εταράχθησαν και εσαλεύθησαν, διότι ωργίσθη.
\par 9 Καπνός ανέβαινεν εκ των μυκτήρων αυτού, και πυρ κατατρώγον εκ του στόματος αυτού· άνθρακες ανήφθησαν απ' αυτού.
\par 10 Και έκλινε τους ουρανούς και κατέβη, και γνόφος υπό τους πόδας αυτού.
\par 11 Και επέβη επί χερουβείμ και επέταξε, και εφάνη επί πτερύγων ανέμων.
\par 12 Και έθεσε σκηνήν πέριξ αυτού το σκότος, ύδατα ζοφερά, νέφη πυκνά των αέρων.
\par 13 Άνθρακες πυρός εξεκαύθησαν εκ της λάμψεως της έμπροσθεν αυτού.
\par 14 Εβρόντησεν ο Κύριος εξ ουρανού, και ο Ύψιστος έδωκε την φωνήν αυτού.
\par 15 Και απέστειλε βέλη και εσκόρπισεν αυτούς· αστραπάς, και συνετάραξεν αυτούς.
\par 16 Και εφάνησαν οι πυθμένες της θαλάσσης, ανεκαλύφθησαν τα θεμέλια της οικουμένης, εις την επιτίμησιν του Κυρίου, από του φυσήματος της πνοής των μυκτήρων αυτού.
\par 17 Εξαπέστειλεν εξ ύψους· έλαβέ με· είλκυσέ με εξ υδάτων πολλών.
\par 18 Ηλευθέρωσέ με εκ του δυνατού εχθρού μου, και εκ των μισούντων με, διότι ήσαν δυνατώτεροί μου.
\par 19 Προέφθασάν με εν τη ημέρα της θλίψεώς μου· αλλ' ο Κύριος εστάθη το αντιστήριγμά μου·
\par 20 Και εξήγαγέ με εις ευρυχωρίαν· ηλευθέρωσέ με, διότι ηυδόκησεν εις εμέ.
\par 21 Αντήμειψέ με ο Κύριος κατά την δικαιοσύνην μου· κατά την καθαρότητα των χειρών μου ανταπέδωκεν εις εμέ.
\par 22 Διότι εφύλαξα τας οδούς του Κυρίου και δεν ησέβησα εκκλίνας από του Θεού μου.
\par 23 Διότι πάσαι αι κρίσεις αυτού ήσαν έμπροσθέν μου· και από των διαταγμάτων αυτού δεν απεμακρύνθην.
\par 24 Και εστάθην άμεμπτος προς αυτόν, και εφυλάχθην από της ανομίας μου.
\par 25 Και ανταπέδωκεν εις εμέ ο Κύριος κατά την δικαιοσύνην μου, Κατά την καθαρότητά μου έμπροσθεν των οφθαλμών αυτού.
\par 26 Μετά οσίου, όσιος θέλεις είσθαι, μετά ανδρός τελείου, τέλειος θέλεις είσθαι·
\par 27 μετά καθαρού, καθαρός θέλεις είσθαι· και μετά διεστραμμένου διεστραμμένα θέλεις φερθή.
\par 28 Και θέλεις σώσει λαόν τεθλιμμένον· επί δε τους υπερηφάνους οι οφθαλμοί σου είναι, διά να ταπεινώσης αυτούς,
\par 29 διότι συ είσαι ο λύχνος μου, Κύριε· και ο Κύριος θέλει φωτίσει το σκότος μου.
\par 30 Διότι διά σου θέλω διασπάσει στράτευμα· διά του Θεού μου θέλω υπερπηδήσει τείχος.
\par 31 Του Θεού, η οδός αυτού είναι άμωμος, ο λόγος του Κυρίου είναι δεδοκιμασμένος· είναι ασπίς πάντων των ελπιζόντων επ' αυτόν.
\par 32 Διότι τις Θεός, πλην του Κυρίου; και τις φρούριον, πλην του Θεού ημών·
\par 33 ο Θεός είναι το κραταιόν οχύρωμά μου· και καθιστών άμωμον την οδόν μου.
\par 34 Κάμνει τους πόδας μου ως των ελάφων και με στήνει επί τους υψηλούς τόπους μου.
\par 35 Διδάσκει τας χείρας μου εις πόλεμον, και έκαμε τόξον χαλκούν τους βραχίονάς μου.
\par 36 Και έδωκας εις εμέ την ασπίδα της σωτηρίας σου· και η αγαθότης σου με εμεγάλυνεν.
\par 37 Συ επλάτυνας τα βήματά μου υποκάτω μου, και οι πόδες μου δεν εκλονίσθησαν.
\par 38 Κατεδίωξα τους εχθρούς μου και ηφάνισα αυτούς· και δεν επέστρεψα εωσού συνετέλεσα αυτούς.
\par 39 Και συνετέλεσα αυτούς, και δεν ηδυνήθησαν να ανεγερθώσιν· και έπεσον υπό τους πόδας μου.
\par 40 Και περιέζωσάς με δύναμιν εις πόλεμον· συνέκαμψας υποκάτω μου τους επανισταμένους επ' εμέ.
\par 41 Και έκαμες τους εχθρούς μου να στρέψωσιν εις εμέ τα νώτα, και εξωλόθρευσα τους μισούντάς με.
\par 42 Περιέβλεψαν, αλλ' ουδείς ο σώζων· εβόησαν προς τον Κύριον, και δεν εισήκουσεν αυτών.
\par 43 Και κατελέπτυνα αυτούς ως την σκόνην της γής· συνέτριψα αυτούς ως τον πηλόν της οδού και κατεπάτησα αυτούς.
\par 44 Και ηλευθέρωσάς με εκ των αντιλογιών του λαού μου· κατέστησάς με κεφαλήν εθνών· λαός, τον οποίον δεν εγνώρισα, εδούλευσεν εις εμέ.
\par 45 Ξένοι υπετάχθησαν εις εμέ· μόλις ήκουσαν, και υπήκουσαν εις εμέ.
\par 46 Ξένοι παρελύθησαν και κατετρόμαξαν εκ των αποκρύφων τόπων αυτών.
\par 47 Ζη Κύριος· και ευλογημένον το φρούριόν μου· και ας υψωθή ο Θεός, το φρούριον της σωτηρίας μου.
\par 48 Ο Θεός, ο εκδικών με και υποτάττων τους λαούς υποκάτω μου·
\par 49 Και ο εξαγαγών με εκ των εχθρών μου· συ, ναι, με υψόνεις υπεράνω των επανισταμένων επ' εμέ· ηλευθέρωσάς με από ανδρός αδίκου.
\par 50 Διά τούτο θέλω σε υμνεί, Κύριε, μεταξύ των εθνών και εις το όνομά σου θέλω ψάλλει.
\par 51 Αυτός μεγαλύνει τας σωτηρίας του βασιλέως αυτού· και κάμνει έλεος εις τον κεχρισμένον αυτού, εις τον Δαβίδ και εις το σπέρμα αυτού έως αιώνος.

\chapter{23}

\par 1 Ούτοι δε είναι οι λόγοι του Δαβίδ, οι τελευταίοι· ο Δαβίδ, ο υιός του Ιεσσαί, είπε, και ο ανήρ όστις ανεβιβάσθη εις ύψος, ο κεχρισμένος του Θεού του Ιακώβ και ο γλυκύς ψαλμωδός του Ισραήλ είπε,
\par 2 Πνεύμα Κυρίου ελάλησε δι' εμού, και ο λόγος αυτού ήλθεν επί της γλώσσης μου.
\par 3 Ο Θεός του Ισραήλ είπε προς εμέ, ο Βράχος του Ισραήλ ελάλησεν, Ο εξουσιάζων επί ανθρώπους ας ήναι δίκαιος, εξουσιάζων μετά φόβου Θεού·
\par 4 Και θέλει είσθαι ως το φως της πρωΐας, όταν ανατέλλη ο ήλιος πρωΐας ανεφέλου, ως η εκ της γης χλόη από της λάμψεως της εκ της βροχής.
\par 5 Αν και ο οίκός μου δεν είναι τοιούτος ενώπιον του Θεού, διαθήκην όμως αιώνιον έκαμε μετ' εμού, διατεταγμένην κατά πάντα και ασφαλή· όθεν τούτο είναι πάσα η σωτηρία μου και πάσα η επιθυμία· αν και δεν έκαμε να βλαστήση.
\par 6 Οι δε παράνομοι, πάντες ούτοι θέλουσιν είσθαι ως άκανθαι εξωσμέναι, διότι με χείρας δεν πιάνονται·
\par 7 Και όστις εγγίση αυτάς, πρέπει να ήναι ώπλισμένος με σίδηρον και με ξύλον λόγχης· και θέλουσι κατακαυθή εν πυρί εν τω αυτώ τόπω.
\par 8 Ταύτα είναι τα ονόματα των ισχυρών, τους οποίους είχεν ο Δαβίδ· Ιοσέβ-βασεβέθ ο Ταχμονίτης, πρώτος των τριών· ούτος ήτο Αδινώ ο Ασωναίος, όστις εθανάτωσεν οκτακοσίους εν μιά μάχη.
\par 9 Και μετ' αυτόν, Ελεάζαρ ο υιός του Δωδώ, υιού του Αχωχί, εις εκ των τριών ισχυρών μετά του Δαβίδ, ότε ωνείδισαν τους Φιλισταίους τους εκεί συνηθροισμένους εις μάχην, και οι άνδρες Ισραήλ εσύρθησαν·
\par 10 ούτος σηκωθείς, επάταξε τους Φιλισταίους, εωσού απέκαμεν η χειρ αυτού και εκολλήθη η χειρ αυτού εις την μάχαιραν· και έκαμεν ο Κύριος σωτηρίαν μεγάλην εν τη ημέρα εκείνη, και ο λαός επέστρεψεν οπίσω αυτού μόνον διά να λαφυραγωγήση.
\par 11 μετά δε τούτον Σαμμά, ο υιός του Αγαί, ο Αραρίτης· και οι μεν Φιλισταίοι είχον συναχθή εις σώμα, όπου ήτο μερίδιον αγρού πλήρες φακής, ο δε λαός έφυγεν από προσώπου των Φιλισταίων·
\par 12 ούτος δε εστηλώθη εν τω μέσω του αγρού και υπερησπίσθη αυτόν, και επάταξε τους Φιλισταίους· και ο Κύριος έκαμε σωτηρίαν μεγάλην.
\par 13 Κατέβησαν έτι τρεις εκ των τριάκοντα αρχηγών και ήλθον προς τον Δαβίδ εν καιρώ θέρους εις το σπήλαιον Οδολλάμ· το δε στρατόπεδον των Φιλισταίων εστρατοπέδευσεν εν τη κοιλάδι Ραφαείμ.
\par 14 Και ο Δαβίδ ήτο τότε εν οχυρώματι, και η φρουρά των Φιλισταίων τότε εν Βηθλεέμ.
\par 15 Και επεπόθησεν ο Δαβίδ ύδωρ και είπε, Τις ήθελε μοι δώσει να πίω ύδωρ εκ του φρέατος της Βηθλεέμ, του πλησίον της πύλης;
\par 16 Και διέσχισαν οι τρεις ισχυροί το στρατόπεδον των Φιλισταίων και ήντλησαν ύδωρ εκ του φρέατος της Βηθλεέμ, του εν τη πύλη, και λαβόντες έφεραν προς τον Δαβίδ· δεν ηθέλησεν όμως να πίη, αλλ' έκαμεν αυτό σπονδήν εις τον Κύριον·
\par 17 και είπε, Μη γένοιτο εις εμέ, Κύριε, να πράξω τούτο· το αίμα των ανδρών, των πορευθέντων μετά κινδύνου της ζωής αυτών, να πίω εγώ; Και δεν ηθέλησε να πίη. Ταύτα έκαμον οι τρεις ισχυροί.
\par 18 Και Αβισαί, ο αδελφός του Ιωάβ, υιός της Σερουΐας, ήτο πρώτος των τριών· και ούτος σείων την λόγχην αυτού εναντίον τριακοσίων, εθανάτωσεν αυτούς και απέκτησεν όνομα μεταξύ των τριών.
\par 19 Δεν εστάθη ούτος ο ενδοξότερος εκ των τριών; διά τούτο έγεινεν αρχηγός αυτών· δεν έφθασεν όμως μέχρι των τριών πρώτων.
\par 20 Και Βεναΐας, ο υιός του Ιωδαέ, υιός ανδρός δυνατού από Καβσεήλ, όστις έκαμε πολλά ανδραγαθήματα, ούτος επάταξε τους δύο λεοντώδεις άνδρας του Μωάβ· ούτος έτι κατέβη και επάταξε λέοντα εν μέσω του λάκκου εν ημέρα χιόνος.
\par 21 Ούτος έτι επάταξε τον άνδρα τον Αιγύπτιον, άνδρα ώραίον· και εν τη χειρί του Αιγυπτίου ήτο λόγχη· εκείνος δε κατέβη προς αυτόν με ράβδον, και αρπάσας την λόγχην εκ της χειρός του Αιγυπτίου, εθανάτωσεν αυτόν διά της ιδίας αυτού λόγχης.
\par 22 Ταύτα έκαμε Βεναΐας, ο υιός του Ιωδαέ, και απέκτησεν όνομα μεταξύ των τριών ισχυρών.
\par 23 Ήτο ενδοξότερος των τριάκοντα· δεν έφθασεν όμως μέχρι των τριών πρώτων· και κατέστησεν αυτόν ο Δαβίδ επί των δορυφόρων αυτού.
\par 24 Ασαήλ, ο αδελφός του Ιωάβ, ήτο μεταξύ των τριάκοντα· οίτινες ήσαν Ελχανάν, ο υιός του Δωδώ, εκ της Βηθλεέμ·
\par 25 Σαμμά ο Αρωδίτης· Ελικά ο Αρωδίτης·
\par 26 Χελής ο Φαλτίτης· Ιράς, ο υιός του Ικκής, ο Θεκωΐτης·
\par 27 Αβιέζερ ο Αναθωθίτης· Μεβουναί ο Χουσαθίτης·
\par 28 Σαλμών ο Αχωχίτης· Μααραΐ ο Νετωφαθίτης·
\par 29 Χελέβ, ο υιός του Βαανά, ο Νετωφαθίτης· Ιτταΐ, ο υιός του Ριβαί, από Γαβαά, των υιών Βενιαμίν·
\par 30 Βεναΐας ο Πιραθωνίτης· Ιδδαΐ, εκ των κοιλάδων Γαάς·
\par 31 Αβί-αλβών ο Αρβαθίτης· Αζμαβέθ ο Βαρουμίτης·
\par 32 Ελιαβά ο Σααλβωνίτης· Ιωνάθαν, εκ των υιών Ιαασήν·
\par 33 Σαμμά ο Αραρίτης· Αχιάμ, ο υιός του Σαράρ, ο Αραρίτης·
\par 34 Ελιφελέτ, ο υιός του Αασβαί, υιός του Μααχαθίτου· Ελιάμ, ο υιός του Αχιτόφελ του Γιλωναίου.
\par 35 Εσραΐ ο Καρμηλίτης· Φααραί ο Αρβίτης·
\par 36 Ιγάλ, ο υιός του Νάθαν, από Σωβά· ανί ο Γαδίτης·
\par 37 Σελέκ ο Αμμωνίτης· Νααραί ο Βηρωθαίος, ο οπλοφόρος του Ιωάβ, υιού της Σερουΐας·
\par 38 Ιράς ο Ιεθρίτης· Γαρήβ ο Ιεθρίτης·
\par 39 Ουρίας ο Χετταίος· πάντες τριάκοντα επτά.

\chapter{24}

\par 1 Και εξήφθη πάλιν η οργή του Κυρίου εναντίον του Ισραήλ, και διήγειρε τον Δαβίδ εναντίον αυτών να είπη, Ύπαγε, αρίθμησον τον Ισραήλ και τον Ιούδαν.
\par 2 Και είπεν ο βασιλεύς προς τον Ιωάβ, τον αρχηγόν του στρατεύματος, όστις ήτο μετ' αυτού· Δίελθε τώρα πάσας τας φυλάς του Ισραήλ, από Δαν έως Βηρ-σαβεέ, και απαρίθμησον τον λαόν, διά να μάθω τον αριθμόν του λαού.
\par 3 Και είπεν ο Ιωάβ προς τον βασιλέα, Είθε Κύριος ο Θεός σου να προσθέση εις τον λαόν εκατονταπλάσιον αφ' ό,τι είναι, και να ίδωσιν οι οφθαλμοί του κυρίου μου του βασιλέως· πλην διά τι ο κύριός μου ο βασιλεύς επιθυμεί το πράγμα τούτο;
\par 4 Ο λόγος όμως του βασιλέως υπερίσχυσεν επί τον Ιωάβ και επί τους αρχηγούς του στρατεύματος· και ήλθεν ο Ιωάβ και οι αρχηγοί του στρατεύματος απ' έμπροσθεν του βασιλέως, διά να απαριθμήσωσι τον λαόν τον Ισραήλ.
\par 5 Και διέβησαν τον Ιορδάνην και εστρατοπέδευσαν εν Αροήρ, εκ των δεξιών της πόλεως, της εν μέσω της φάραγγος Γαδ, και προς Ιαζήρ.
\par 6 Έπειτα ήλθον εις Γαλαάδ και εις την γην Ταχτίμ-οδσεί· και ήλθον εις Δαν-ιαάν και πέριξ, έως της Σιδώνος·
\par 7 και ήλθον εις το φρούριον της Τύρου και εις πάσας τας πόλεις των Ευαίων και των Χαναναίων· και εξήλθον κατά το νότιον του Ιούδα εις Βηρ-σαβεέ.
\par 8 Αφού δε περιώδευσαν πάσαν την γην, ήλθον εις Ιερουσαλήμ, εις το τέλος εννέα μηνών και είκοσι ημερών.
\par 9 Και έδωκεν ο Ιωάβ εις τον βασιλέα το κεφάλαιον της απαριθμήσεως του λαού· και ήσαν ο Ισραήλ οκτακόσιαι χιλιάδες άνδρες δυνάμεως σύροντες ρομφαίαν· και οι άνδρες του Ιούδα πεντακόσιαι χιλιάδες.
\par 10 Και η καρδία του Δαβίδ εκτύπησεν αυτόν, αφού απηρίθμησε τον λαόν. Και είπεν ο Δαβίδ προς τον Κύριον, Ημάρτησα σφόδρα, πράξας τούτο· και τώρα, δέομαί σου, Κύριε, αφαίρεσον την ανομίαν του δούλου σου, ότι εμωράνθην σφόδρα.
\par 11 Και ότε εσηκώθη ο Δαβίδ το πρωΐ, ο λόγος του Κυρίου ήλθε προς τον Γαδ τον προφήτην, τον βλέποντα του Δαβίδ, λέγων,
\par 12 Ύπαγε και ειπέ προς τον Δαβίδ, ούτω λέγει Κύριος· Τρία πράγματα εγώ προβάλλω εις σέ· έκλεξον εις σεαυτόν εν εκ τούτων, και θέλω σοι κάμει αυτό.
\par 13 Ήλθε λοιπόν ο Γαδ προς τον Δαβίδ και ανήγγειλε προς αυτόν και είπε προς αυτόν, Θέλεις να επέλθωσιν εις σε επτά έτη πείνης επί την γην σου; ή τρεις μήνας να φεύγης απ' έμπροσθεν των εχθρών σου και να σε διώκωσιν; ή τρεις ημέρας να ήναι θανατικόν εν τη γη σου; τώρα συλλογίσθητι, και ιδέ ποίαν απόκρισιν θέλω φέρει προς τον αποστείλαντά με.
\par 14 Και είπεν ο Δαβίδ προς τον Γαδ, Στενά μοι πανταχόθεν σφόδρα· ας πέσω λοιπόν εις την χείρα του Κυρίου, διότι είναι πολλοί οι οικτιρμοί αυτού· εις χείρα δε ανθρώπου ας μη πέσω.
\par 15 Απέστειλε λοιπόν ο Κύριος θανατικόν επί τον Ισραήλ, από πρωΐας μέχρι του διωρισμένου καιρού· και απέθανον εκ του λαού, από Δαν έως Βηρ-σαβεέ, εβδομήκοντα χιλιάδες ανδρών.
\par 16 Και ότε ο άγγελος εξέτεινε την χείρα αυτού κατά της Ιερουσαλήμ, διά να απολέση αυτήν, μετεμελήθη ο Κύριος περί του κακού, και είπε προς τον άγγελον, όστις έκαμεν εν τω λαώ την φθοράν, Αρκεί ήδη· σύρε την χείρα σου. Ήτο δε ο άγγελος του Κυρίου πλησίον του αλωνίου του Ορνά του Ιεβουσαίου.
\par 17 Και ελάλησεν ο Δαβίδ προς τον Κύριον, ότε είδε τον άγγελον τον θανατόνοντα τον λαόν, και είπεν, Ιδού, εγώ ήμαρτον και εγώ ηνόμησα· ταύτα δε τα πρόβατα τι έπραξαν; κατ' εμού λοιπόν έστω η χειρ σου και κατά του οίκου του πατρός μου.
\par 18 Και ήλθεν ο Γαδ την ημέραν εκείνην προς τον Δαβίδ και είπε προς αυτόν, Ανάβα, στήσον θυσιαστήριον εις τον Κύριον εν τω αλωνίω Ορνά του Ιεβουσαίου.
\par 19 Και ανέβη ο Δαβίδ κατά τον λόγον του Γαδ, ως προσέταξεν ο Κύριος.
\par 20 Και ανέβλεψεν ο Ορνά και είδε τον βασιλέα και τους δούλους αυτού ερχομένους προς αυτόν· και εξήλθεν ο Ορνά και προσεκύνησε τον βασιλέα κατά πρόσωπον αυτού έως εδάφους.
\par 21 Και είπεν ο Ορνά, Διά τι ήλθεν ο κύριός μου ο βασιλεύς προς τον δούλον αυτού; Και είπεν ο Δαβίδ, Διά να αγοράσω το αλώνιον παρά σου, διά να οικοδομήσω θυσιαστήριον εις τον Κύριον, και να σταθή η πληγή από του λαού.
\par 22 Και είπεν ο Ορνά προς τον Δαβίδ, Ας λάβη ο κύριός μου ο βασιλεύς και ας προσφέρη εις θυσίαν ό,τι φαίνεται αρεστόν εις τους οφθαλμούς αυτού· ιδού, οι βόες εις ολοκαύτωμα και τα αλωνικά εργαλεία και τα εργαλεία των βοών διά ξύλα.
\par 23 Τα πάντα έδωκεν ο Ορνά, ως βασιλεύς, εις τον βασιλέα. Και είπεν ο Ορνά προς τον βασιλέα, Κύριος ο Θεός σου να ευαρεστηθή εις σε.
\par 24 Και είπεν ο βασιλεύς προς τον Ορνά, Ουχί, αλλά θέλω εξάπαντος αγοράσει αυτό παρά σου διά αντιπληρωμής· διότι δεν θέλω προσφέρει ολοκαυτώματα εις Κύριον τον Θεόν μου δωρεάν. Και ηγόρασεν ο Δαβίδ το αλώνιον και τους βόας διά πεντήκοντα σίκλων αργυρίου.
\par 25 Και ωκοδόμησεν ο Δαβίδ εκεί θυσιαστήριον εις τον Κύριον, και προσέφερεν ολοκαυτώματα και ειρηνικάς προσφοράς. Και εξιλεώθη ο Κύριος προς την γην, και εστάθη η πληγή από του Ισραήλ.


\end{document}