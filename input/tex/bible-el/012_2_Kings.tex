\begin{document}

\title{2 Kings}


\chapter{1}

\par 1 Μετά δε τον θάνατον του Αχαάβ, επανεστάτησεν ο Μωάβ εναντίον του Ισραήλ.
\par 2 Και έπεσεν ο Οχοζίας διά του δρυφράκτου του υπερώου αυτού, το οποίον ήτο εν Σαμαρεία, και ηρρώστησε· και απέστειλε μηνυτάς, ειπών προς αυτούς, Υπάγετε, ερωτήσατε τον Βέελ-ζεβούλ, τον θεόν της Ακκαρών, αν έχω να αναλάβω από της αρρωστίας ταύτης.
\par 3 Αλλ' ο άγγελος Κυρίου είπε προς Ηλίαν τον Θεσβίτην, Σηκώθητι, ανάβα εις συνάντησιν των μηνυτών του βασιλέως της Σαμαρείας και ειπέ προς αυτούς, Επειδή δεν είναι Θεός εν τω Ισραήλ, διά τούτο υπάγετε να ερωτήσητε τον Βέελ-ζεβούλ, τον θεόν της Ακκαρών;
\par 4 Τώρα λοιπόν ούτω λέγει ο Κύριος· Δεν θέλεις καταβή από της κλίνης, εις την οποίαν ανέβης, αλλ' εξάπαντος θέλεις αποθάνει. Και ανεχώρησεν ο Ηλίας.
\par 5 Και επέστρεψαν οι μηνυταί προς αυτόν· ο δε είπε προς αυτούς, Διά τι επεστρέψατε;
\par 6 Και είπον προς αυτόν, Άνθρωπος τις ανέβη εις συνάντησιν ημών και είπε προς ημάς, Υπάγετε, επιστρέψατε προς τον βασιλέα, όστις σας απέστειλε, και είπατε προς αυτόν, ούτω λέγει Κύριος· Επειδή δεν είναι Θεός εν τω Ισραήλ, διά τούτο στέλλεις να ερωτήσης τον Βέελ-ζεβούλ, τον θεόν της Ακκαρών; δεν θέλεις λοιπόν καταβή από της κλίνης, εις την οποίαν ανέβης, αλλ' εξάπαντος θέλεις αποθάνει.
\par 7 Και είπε προς αυτούς, Οποία ήτο η μορφή του ανθρώπου, όστις ανέβη εις συνάντησίν σας και ελάλησε προς εσάς τους λόγους τούτους;
\par 8 Και απεκρίθησαν προς αυτόν, Άνθρωπος δασύτριχος και περιεζωσμένος την οσφύν αυτού με ζώνην δερματίνην. Και είπεν, Ηλίας ο Θεσβίτης είναι.
\par 9 Τότε απέστειλεν ο βασιλεύς προς αυτόν πεντηκόνταρχον μετά των πεντήκοντα αυτού. Και ανέβη προς αυτόν· και ιδού, εκάθητο επί της κορυφής του όρους. Και είπε προς αυτόν, Άνθρωπε του Θεού, ο βασιλεύς είπε, Κατάβα.
\par 10 Και αποκριθείς ο Ηλίας είπε προς τον πεντηκόνταρχον, Εάν εγώ ήμαι άνθρωπος του Θεού, ας καταβή πυρ εξ ουρανού και ας καταφάγη σε και τους πεντήκοντά σου. Και κατέβη πυρ εκ του ουρανού και κατέφαγεν αυτόν και τους πεντήκοντα αυτού.
\par 11 Και απέστειλε προς αυτόν πάλιν άλλον πεντηκόνταρχον μετά των πεντήκοντα αυτού. Και ελάλησε και είπε προς αυτόν, Άνθρωπε του Θεού, ούτω λέγει ο βασιλεύς· Ταχέως κατάβα.
\par 12 Και αποκριθείς ο Ηλίας είπε προς αυτούς, Εάν εγώ ήμαι άνθρωπος του Θεού, ας καταβή πυρ εξ ουρανού και ας καταφάγη σε και τους πεντήκοντά σου. Και κατέβη πυρ Θεού εξ ουρανού και κατέφαγεν αυτόν και τους πεντήκοντα αυτού.
\par 13 Και πάλιν απέστειλε τρίτον πεντηκόνταρχον μετά των πεντήκοντα αυτού. Και αναβάς ο τρίτος πεντηκόνταρχος ήλθε και εγονάτισεν έμπροσθεν του Ηλία και παρεκάλεσεν αυτόν και είπε προς αυτόν; Άνθρωπε του Θεού, ας σταθή, δέομαι, αξιοτίμητος εις τους οφθαλμούς σου η ζωή μου και η ζωή των δούλων σου τούτων των πεντήκοντα·
\par 14 ιδού, κατέβη πυρ εξ ουρανού και κατέκαυσε τους δύο πρώτους πεντηκοντάρχους μετά των πεντήκοντα αυτών· ας σταθή λοιπόν η ζωή μου αξιοτίμητος εις τους οφθαλμούς σου.
\par 15 Και είπεν ο άγγελος του Κυρίου προς τον Ηλίαν, Κατάβα μετ' αυτού· μη φοβηθής απ' αυτού. Και εσηκώθη και κατέβη μετ' αυτού προς τον βασιλέα.
\par 16 Και είπε προς αυτόν, Ούτω λέγει Κύριος· Επειδή απέστειλας μηνυτάς να ερωτήσωσι τον Βέελ-ζεβούλ, τον θεόν της Ακκαρών, ως εάν δεν ήτο Θεός εν τω Ισραήλ διά να ζητήσης τον λόγον αυτού, διά τούτο δεν θέλεις καταβή από της κλίνης, εις την οποίαν ανέβης, αλλ' εξάπαντος θέλεις αποθάνει.
\par 17 Και απέθανε κατά τον λόγον του Κυρίου, τον οποίον ελάλησεν ο Ηλίας· εβασίλευσε δε αντ' αυτού ο Ιωράμ, εν τω δευτέρω έτει του Ιωράμ, υιού του Ιωσαφάτ, βασιλέως του Ιούδα· επειδή δεν είχεν υιόν.
\par 18 Αι δε λοιπαί των πράξεων του Οχοζίου, όσας έκαμε, δεν είναι γεγραμμέναι εν τω βιβλίω των χρονικών των βασιλέων του Ισραήλ;

\chapter{2}

\par 1 Ότε δε έμελλεν ο Κύριος να αναβιβάση τον Ηλίαν εις τον ουρανόν με ανεμοστρόβιλον, ανεχώρησεν ο Ηλίας μετά του Ελισσαιέ από Γαλγάλων.
\par 2 Και είπεν ο Ηλίας προς τον Ελισσαιέ, Κάθου ενταύθα, παρακαλώ· διότι ο Κύριος με απέστειλεν έως Βαιθήλ. Και είπεν ο Ελισσαιέ, Ζη Κύριος και ζη η ψυχή σου, δεν θέλω σε αφήσει. Και κατέβησαν εις Βαιθήλ.
\par 3 Και εξήλθον οι υιοί των προφητών οι εν Βαιθήλ προς τον Ελισσαιέ και είπον προς αυτόν, Εξεύρεις ότι ο Κύριος σήμερον λαμβάνει τον κύριόν σου επάνωθεν της κεφαλής σου; Και είπε, Και εγώ εξεύρω τούτο· σιωπάτε.
\par 4 Και είπεν ο Ηλίας προς αυτόν, Ελισσαιέ, κάθου ενταύθα, παρακαλώ· διότι ο Κύριος με απέστειλεν εις Ιεριχώ. Ο δε είπε, Ζη Κύριος και ζη η ψυχή σου, δεν θέλω σε αφήσει. Και ήλθον εις Ιεριχώ.
\par 5 Και προσήλθον οι υιοί των προφητών οι εν Ιεριχώ προς τον Ελισσαιέ και είπον προς αυτόν, Εξεύρεις ότι ο Κύριος σήμερον λαμβάνει τον κύριόν σου επάνωθεν της κεφαλής σου; Και είπε, Και εγώ εξεύρω τούτο· σιωπάτε.
\par 6 Και είπεν ο Ηλίας προς αυτόν, Κάθου ενταύθα, παρακαλώ· διότι ο Κύριος με απέστειλεν εις τον Ιορδάνην. Ο δε είπε, Ζη Κύριος και ζη η ψυχή σου, δεν θέλω σε αφήσει. Και υπήγαν αμφότεροι.
\par 7 Και υπήγαν πεντήκοντα άνδρες εκ των υιών των προφητών, και εστάθησαν απέναντι μακρόθεν· εκείνοι δε οι δύο εστάθησαν επί του Ιορδάνου.
\par 8 Και έλαβεν ο Ηλίας την μηλωτήν αυτού και εδίπλωσεν αυτήν και εκτύπησε τα ύδατα, και διηρέθησαν ένθεν και ένθεν, και διέβησαν αμφότεροι διά ξηράς.
\par 9 Και ότε διέβησαν, είπεν ο Ηλίας προς τον Ελισσαιέ, Ζήτησον τι να σοι κάμω, πριν αναληφθώ από σου. Και είπεν ο Ελισσαιέ, Διπλασία μερίς του πνεύματός σου ας ήναι, παρακαλώ, επ' εμέ.
\par 10 Ο δε είπε, Σκληρόν πράγμα εζήτησας· πλην εάν με ίδης αναλαμβανόμενον από σου, θέλει γείνει εις σε ούτως· ει δε μη, δεν θέλει γείνει.
\par 11 Και ενώ αυτοί περιεπάτουν έτι λαλούντες, ιδού, άμαξα πυρός και ίπποι πυρός, και διεχώρισαν αυτούς αμφοτέρους· και ανέβη ο Ηλίας με ανεμοστρόβιλον εις τον ουρανόν.
\par 12 Ο δε Ελισσαιέ έβλεπε και εβόα, Πάτερ μου, πάτερ μου, άμαξα του Ισραήλ και ιππικόν αυτού. Και δεν είδεν αυτόν πλέον· και επίασε τα ιμάτια αυτού και διέσχισεν αυτά εις δύο τμήματα.
\par 13 Και σηκώσας την μηλωτήν του Ηλία, ήτις έπεσεν επάνωθεν εκείνου, επέστρεφε και εστάθη επί του χείλους του Ιορδάνου.
\par 14 Και λαβών την μηλωτήν του Ηλία, ήτις έπεσεν επάνωθεν εκείνου, εκτύπησε τα ύδατα και είπε, Που είναι Κύριος ο Θεός του Ηλία; Και ως εκτύπησε και αυτός τα ύδατα, διηρέθησαν ένθεν και ένθεν· και διέβη ο Ελισσαιέ.
\par 15 Και ιδόντες αυτόν οι υιοί των προφητών, οι εν Ιεριχώ εκ του απέναντι, είπον, Το πνεύμα του Ηλία επανεπαύθη επί τον Ελισσαιέ. Και ήλθον εις συνάντησιν αυτού και προσεκύνησαν αυτόν έως εδάφους.
\par 16 Και είπον προς αυτόν, Ιδού τώρα, πεντήκοντα δυνατοί άνδρες είναι μετά των δούλων σου· ας υπάγωσι, παρακαλούμεν, και ας ζητήσωσι τον κύριόν σου, μήποτε εσήκωσεν αυτόν το πνεύμα του Κυρίου και έρριψεν αυτόν επί τινός όρους ή επί τινός κοιλάδος. Και είπε, Μη αποστείλητε.
\par 17 Αλλ' αφού εβίασαν αυτόν τόσον ώστε ησχύνετο, είπεν, Αποστείλατε. Απέστειλαν λοιπόν πεντήκοντα άνδρας και εζήτησαν τρεις ημέρας, πλην δεν εύρηκαν αυτόν.
\par 18 Και ότε επέστρεψαν προς αυτόν, διότι έμεινεν εν Ιεριχώ, είπε προς αυτούς, Δεν σας είπα, Μη υπάγητε;
\par 19 Και είπον οι άνδρες της πόλεως προς τον Ελισσαιέ, Ιδού τώρα, η θέσις της πόλεως ταύτης είναι καλή, καθώς ο κύριός μου βλέπει τα ύδατα όμως είναι κακά και η γη άγονος.
\par 20 Και είπε, Φέρετέ μοι φιάλην καινήν και βάλετε άλας εις αυτήν. Και έφεραν προς αυτόν.
\par 21 Και εξήλθεν εις την πηγήν των υδάτων και έρριψε το άλας εκεί και είπεν, Ούτω λέγει Κύριος· Υγίανα τα ύδατα ταύτα· δεν θέλει είσθαι πλέον εκ τούτων θάνατος ή ακαρπία.
\par 22 Και ιάθησαν τα ύδατα έως της ημέρας ταύτης, κατά τον λόγον του Ελισσαιέ, τον οποίον ελάλησε.
\par 23 Και ανέβη εκείθεν εις Βαιθήλ· και ενώ αυτός ανέβαινεν εν τη οδώ, εξήλθον εκ της πόλεως παιδία μικρά και ενέπαιζον αυτόν και έλεγον προς αυτόν, Ανάβαινε, φαλακρέ· ανάβαινε, φαλακρέ·
\par 24 ο δε εστράφη οπίσω και ιδών αυτά, κατηράσθη αυτά εις το όνομα του Κυρίου. Και εξήλθον εκ του δάσους δύο άρκτοι και διεσπάραξαν εξ αυτών τεσσαράκοντα δύο παιδία.
\par 25 Και υπήγεν εκείθεν εις το όρος τον Κάρμηλον· και εκείθεν επέστρεψεν εις Σαμάρειαν.

\chapter{3}

\par 1 Ο δε Ιωράμ υιός του Αχαάβ εβασίλευσεν επί τον Ισραήλ εν Σαμαρεία, το δέκατον όγδοον έτος του Ιωσαφάτ βασιλέως του Ιούδα· και εβασίλευσεν έτη δώδεκα.
\par 2 Και έπραξε πονηρά ενώπιον του Κυρίου, ουχί όμως καθώς ο πατήρ αυτού και η μήτηρ αυτού· διότι εσήκωσε το άγαλμα του Βάαλ, το οποίον είχε κάμει ο πατήρ αυτού.
\par 3 Πλην ήτο προσκεκολλημένος εις τας αμαρτίας του Ιεροβοάμ υιού του Ναβάτ, όστις έκαμε τον Ισραήλ να αμαρτήση· δεν απεμακρύνθη απ' αυτών.
\par 4 Μησά δε ο βασιλεύς του Μωάβ είχε ποίμνια, και έδιδεν εις τον βασιλέα του Ισραήλ εκατόν χιλιάδας αρνίων και εκατόν χιλιάδας κριών με τα μαλλία αυτών.
\par 5 Αλλ' αφού απέθανεν ο Αχαάβ, απεστάτησεν ο βασιλεύς του Μωάβ κατά του βασιλέως του Ισραήλ.
\par 6 Και εξήλθεν ο βασιλεύς Ιωράμ εκ της Σαμαρείας κατ' εκείνον τον καιρόν και απηρίθμησε πάντα τον Ισραήλ.
\par 7 Και υπήγε και απέστειλε προς Ιωσαφάτ τον βασιλέα του Ιούδα, λέγων, Ο βασιλεύς του Μωάβ απεστάτησε κατ' εμού· έρχεσαι μετ' εμού εναντίον του Μωάβ εις πόλεμον; Ο δε είπε, Θέλω αναβή· εγώ είμαι ως συ, ο λαός μου ως ο λαός σου, οι ίπποι μου ως οι ίπποι σου.
\par 8 Και είπε, Διά ποίας οδού θέλομεν αναβή; Ο δε απεκρίθη, Διά της οδού της ερήμου Εδώμ.
\par 9 Και υπήγεν ο βασιλεύς του Ισραήλ και ο βασιλεύς του Ιούδα και ο βασιλεύς του Εδώμ· και περιήλθον οδόν επτά ημερών· και δεν ήτο ύδωρ διά το στρατόπεδον και διά τα κτήνη τα ακολουθούντα αυτούς.
\par 10 Και είπεν ο βασιλεύς του Ισραήλ, Ω βεβαίως συνεκάλεσεν ο Κύριος τους τρεις τούτους βασιλείς, διά να παραδώση αυτούς εις την χείρα του Μωάβ.
\par 11 Ο δε Ιωσαφάτ είπε, Δεν είναι εδώ προφήτης του Κυρίου, διά να ερωτήσωμεν τον Κύριον δι' αυτού; Και απεκρίθη εις εκ των δούλων του βασιλέως του Ισραήλ, και είπεν, Είναι εδώ Ελισσαιέ ο υιός του Σαφάτ, όστις επέχεεν ύδωρ εις τας χείρας του Ηλία.
\par 12 Και είπεν ο Ιωσαφάτ, Λόγος Κυρίου είναι μετ' αυτού. Και κατέβησαν προς αυτόν ο βασιλεύς του Ισραήλ και ο Ιωσαφάτ και ο βασιλεύς του Εδώμ.
\par 13 Και είπεν ο Ελισσαιέ προς τον βασιλέα του Ισραήλ, Τι είναι μεταξύ εμού και σου; ύπαγε προς τους προφήτας του πατρός σου και προς τους προφήτας της μητρός σου. Και είπε προς αυτόν ο βασιλεύς του Ισραήλ, Μή· διότι ο Κύριος συνεκάλεσε τους τρεις τούτους βασιλείς, διά να παραδώση αυτούς εις την χείρα του Μωάβ.
\par 14 Και είπεν ο Ελισσαιέ, Ζη ο Κύριος των δυνάμεων, ενώπιον του οποίου παρίσταμαι, βεβαίως εάν δεν εσεβόμην το πρόσωπον του Ιωσαφάτ βασιλέως του Ιούδα, δεν ήθελον επιβλέψει προς σε, ουδέ ήθελον σε ιδεί,
\par 15 αλλά τώρα φέρετέ μοι ψαλτωδόν. Και ενώ έψαλλεν ο ψαλτωδός, ήλθεν επ' αυτόν η χειρ του Κυρίου.
\par 16 Και είπεν, ούτω λέγει Κύριος· Κάμε την κοιλάδα ταύτην λάκκους·
\par 17 διότι ούτω λέγει Κύριος, Δεν θέλετε ιδεί άνεμον και δεν θέλετε ιδεί βροχήν· και αύτη η κοιλάς θέλει πλησθή ύδατος, και θέλει πίει, σεις και τα ποίμνιά σας και τα κτήνη σας·
\par 18 αλλά τούτο είναι μικρόν πράγμα εις τους οφθαλμούς του Κυρίου· εις την χείρα σας θέλει παραδώσει και τον Μωάβ·
\par 19 και θέλετε πατάξει πάσαν οχυράν πόλιν και πάσαν εκλεκτήν πόλιν, και θέλετε καταβάλει παν δένδρον καλόν, και εμφράξει πάσας τας πηγάς των υδάτων, και αχρειώσει με λίθους πάσαν αγαθήν μερίδα γης.
\par 20 Και το πρωΐ, ενώ ετελείτο η προσφορά, ιδού, ήλθον ύδατα από της οδού Εδώμ, και επλήσθη η γη υδάτων.
\par 21 Και ότε ήκουσαν πάντες οι Μωαβίται ότι ανέβησαν οι βασιλείς διά να πολεμήσωσιν αυτούς, συνηθροίσθησαν πάντες οι μάχαιραν περιζωννύμενοι και επάνω, και εστάθησαν επί των συνόρων.
\par 22 Και εξηγέρθησαν το πρωΐ, και καθώς ανέτειλεν ο ήλιος επί τα ύδατα, είδον οι Μωαβίται εκ του απέναντι τα ύδατα κόκκινα ως αίμα·
\par 23 και είπον, Τούτο είναι αίμα· βεβαίως οι βασιλείς επολέμησαν και εκτυπήθησαν μετ' αλλήλων· τώρα λοιπόν εις τα λάφυρα, Μωάβ.
\par 24 Και ότε ήλθον εις το στρατόπεδον του Ισραήλ, εσηκώθησαν οι Ισραηλίται και επάταξαν τους Μωαβίτας, ώστε έφυγον από προσώπου αυτών· και κτυπώντες τους Μωαβίτας εισήλθον εις την γην αυτών.
\par 25 Και κατέστρεψαν τας πόλεις· και εις πάσαν αγαθήν μερίδα γης έρριψαν έκαστος την πέτραν αυτού, και εγέμισαν αυτήν· και πάσας τας πηγάς των υδάτων ενέφραξαν, και παν δένδρον καλόν κατέβαλον· ώστε εν Κιρ-αρασέθ έμειναν οι λίθοι αυτής, και κυκλώσαντες οι σφενδονισταί επάταξαν αυτήν.
\par 26 Και ότε βασιλεύς του Μωάβ είδεν ότι η μάχη υπερίσχυεν εναντίον αυτού, έλαβε μεθ' αυτού επτακοσίους άνδρας ξιφήρεις, διά να διακόψωσι το στράτευμα, μέχρι του βασιλέως Εδώμ· πλην δεν ηδυνήθησαν.
\par 27 Τότε έλαβε τον πρωτότοκον αυτού υιόν, όστις έμελλε να βασιλεύση αντ' αυτού, και προσέφερεν αυτόν ολοκαύτωμα επί του τείχους· και έγεινεν αγανάκτησις μεγάλη εν τω Ισραήλ· και αναχωρήσαντες απ' αυτού, επέστρεψαν εις την γην αυτών.

\chapter{4}

\par 1 Γυνή δε τις εκ των γυναικών των υιών των προφητών εβόα προς τον Ελισσαιέ, λέγουσα, Ο δούλός σου ο ανήρ μου απέθανε· και συ εξεύρεις ότι ο δούλός σου εφοβείτο τον Κύριον· και ο δανειστής ήλθε να λάβη τους δύο υιούς μου εις εαυτόν διά δούλους.
\par 2 Και είπε προς αυτήν ο Ελισσαιέ, Τι να σοι κάμω; φανέρωσόν μοι τι έχεις εν τω οίκω σου; Η δε είπεν, Η δούλη σου δεν έχει ουδέν εν τω οίκω, ειμή εν αγγείον ελαίου.
\par 3 Και είπεν, Ύπαγε, δανείσθητι έξωθεν αγγεία παρά πάντων των γειτόνων σου, αγγεία κενά· δανείσθητι ουχί ολίγα·
\par 4 είσελθε έπειτα και κλείσον την θύραν όπισθέν σου και όπισθεν των υιών σου, και χύσον εκ του ελαίου εις πάντα τα σκεύη εκείνα, και τα γεμιζόμενα θες κατά μέρος.
\par 5 Ανεχώρησε λοιπόν απ' αυτού και έκλεισε θύραν όπισθεν αυτής και όπισθεν των υιών αυτής· και εκείνοι μεν επλησίαζον εις αυτήν τα αγγεία, αυτή δε ενέχεε.
\par 6 Και αφού εγέμισαν τα αγγεία, είπε προς τον υιόν αυτής, Φέρε μοι και άλλο αγγείον. Ο δε είπε προς αυτήν, Δεν είναι άλλο αγγείον. Και εστάθη το έλαιον.
\par 7 Τότε ήλθε και απήγγειλε προς τον άνθρωπον του Θεού. Και εκείνος είπεν, Ύπαγε, πώλησον το έλαιον και πλήρωσον το χρέος σου και ζήσον με το υπόλοιπον, συ και τα τέκνα σου.
\par 8 Και εν ημέρα τινί διέβαινεν ο Ελισσαιέ εις Σουνάμ, όπου ήτο γυνή τις μεγάλη, και αυτή εκράτησεν αυτόν διά να φάγη άρτον. Και οσάκις διέβαινεν, έστρεφεν εκεί διά να φάγη άρτον.
\par 9 Και είπεν η γυνή προς τον άνδρα αυτής, Ιδού τώρα, γνωρίζω ότι είναι άγιος άνθρωπος του Θεού ούτος, όστις πάντοτε διαβαίνει προς ημάς·
\par 10 ας κάμωμεν, παρακαλώ, μικρόν υπερώον επί του τοίχου· και ας βάλωμεν εκεί δι' αυτόν κλίνην και τράπεζαν και καθέδραν και λύχνον, διά να στρέφη εκεί, όταν έρχηται προς ημάς.
\par 11 Και εν ημέρα τινί ήλθεν εκεί και έστρεψεν εις το υπερώον και εκοιμήθη εκεί.
\par 12 Και είπε προς Γιεζεί τον υπηρέτην αυτού, Κάλεσον την Σουναμίτιν ταύτην. Και ότε εκάλεσεν αυτήν, εστάθη έμπροσθεν αυτού.
\par 13 Και είπε προς αυτόν, Ειπέ τώρα προς αυτήν, Ιδού, συ έλαβες πάσας ταύτας τας φροντίδας υπέρ ημών· τι να κάμω προς σε; έχεις τι να είπης προς τον βασιλέα ή προς τον αρχιστράτηγον; Η δε απεκρίθη, Εγώ κατοικώ μεταξύ του λαού μου.
\par 14 Και είπε, Τι λοιπόν να κάμω δι' αυτήν; Και ο Γιεζεί απεκρίθη, Αληθώς, αυτή δεν έχει τέκνον, και ο ανήρ αυτής είναι γέρων.
\par 15 Και είπε, Κάλεσον αυτήν. Και ότε εκάλεσεν αυτήν, εστάθη εις την θύραν.
\par 16 Και είπε, Το ερχόμενον έτος, κατά τούτον τον καιρόν, θέλεις έχει υιόν εις τας αγκάλας σου. Η δε είπε, Μη, κύριέ μου, άνθρωπε του Θεού, μη ψευσθής προς την δούλην σου.
\par 17 Και η γυνή συνέλαβε και εγέννησεν υιόν το ερχόμενον έτος, κατά τον καιρόν εκείνον τον οποίον είπε προς αυτήν ο Ελισσαιέ.
\par 18 Και ότε εμεγάλωσε το παιδίον, εξήλθεν ημέραν τινά προς τον πατέρα αυτού εις τους θεριστάς.
\par 19 Και είπε προς τον πατέρα αυτού, Την κεφαλήν μου, την κεφαλήν μου. Ο δε είπε προς τον δούλον, Λάβε αυτό προς την μητέρα αυτού.
\par 20 Και λαβών αυτό, έφερεν αυτό προς την μητέρα αυτού, και εκάθησεν επί των γονάτων αυτής μέχρι μεσημβρίας και απέθανε.
\par 21 Και ανέβη και επλαγίασεν αυτό επί της κλίνης του ανθρώπου του Θεού, και έκλεισε την θύραν επάνωθεν αυτού και εξήλθε.
\par 22 Και εκάλεσε τον άνδρα αυτής, λέγουσα, Απόστειλον προς εμέ, παρακαλώ, ένα εκ των δούλων και μίαν εκ των όνων, διά να τρέξω προς τον άνθρωπον του Θεού και να επιστρέψω.
\par 23 Ο δε είπε, Διά τι συ υπάγεις σήμερον προς αυτόν; δεν είναι νεομηνία ουδέ σάββατον. Η δε είπεν, Ειρήνη.
\par 24 Τότε έστρωσε την όνον και είπε προς τον δούλον αυτής, Σύρε και προχώρει μη παύσης εις εμέ την πορείαν, εκτός εάν σε προστάξω.
\par 25 Και υπήγε και ήλθε προς τον άνθρωπον του Θεού εις το όρος τον Κάρμηλον. Και ως είδεν ο άνθρωπος του Θεού αυτήν μακρόθεν, είπε προς τον Γιεζεί τον υπηρέτην αυτού, Ιδού, η Σουναμίτις εκείνη·
\par 26 τώρα λοιπόν, τρέξον εις συνάντησιν αυτής· και ειπέ προς αυτήν, Καλώς έχεις; καλώς έχει ο ανήρ σου; καλώς έχει το παιδίον; Η δε είπε, Καλώς.
\par 27 Και ότε ήλθε προς τον άνθρωπον του Θεού εις το όρος, επίασε τους πόδας αυτού· ο δε Γιεζεί επλησίασε διά να αποσύρη αυτήν. Ο άνθρωπος όμως του Θεού είπεν, Άφες αυτήν· διότι η ψυχή αυτής είναι κατάπικρος εν αυτή· και ο Κύριος έκρυψεν αυτό απ' εμού και δεν μοι εφανέρωσε.
\par 28 Και εκείνη είπε, Μήπως εζήτησα υιόν παρά του κυρίου μου; δεν είπα, Μη με απατάς;
\par 29 Τότε είπε προς τον Γιεζεί, Ζώσθητι την οσφύν σου και λάβε την βακτηρίαν μου εις την χείρα σου και ύπαγε· εάν απαντήσης άνθρωπον, μη χαιρετήσης αυτόν· και εάν τις σε χαιρετήση, μη αποκριθής εις αυτόν· και επίθες την βακτηρίαν μου επί το πρόσωπον του παιδίου.
\par 30 Και η μήτηρ του παιδίου είπε, Ζη Κύριος και ζη η ψυχή σου, δεν θέλω σε αφήσει. Και εσηκώθη και ηκολούθησεν αυτήν.
\par 31 Ο δε Γιεζεί επέρασεν έμπροσθεν αυτών, και επέθεσε την βακτηρίαν επί το πρόσωπον του παιδίου· πλην ουδεμία φωνή και ουδεμία ακρόασις. Όθεν επέστρεψεν εις συνάντησιν αυτού και απήγγειλε προς αυτόν, λέγων, Δεν εξύπνησε το παιδίον.
\par 32 Και ότε εισήλθεν ο Ελισσαιέ εις την οικίαν, ιδού, το παιδίον νεκρόν, πλαγιασμένον επί της κλίνης αυτού.
\par 33 Εισήλθε λοιπόν και έκλεισε την θύραν όπισθεν των δύο αυτών και προσηυχήθη εις τον Κύριον.
\par 34 Και ανέβη και επλαγίασεν επί το παιδίον, και επέθεσε το στόμα αυτού επί το στόμα εκείνου, και τους οφθαλμούς αυτού επί τους οφθαλμούς εκείνου, και τας χείρας αυτού επί τας χείρας εκείνου· και εξηπλώθη επ' αυτό· και εθερμάνθη η σαρξ του παιδίου.
\par 35 Έπειτα εσύρθη, και περιεπάτει εν τω οικήματι πότε εδώ και πότε εκεί· και ανέβη πάλιν και εξηπλώθη επ' αυτό· και το παιδίον επταρνίσθη έως επτάκις και ήνοιξε το παιδίον τους οφθαλμούς αυτού.
\par 36 Τότε εφώνησε τον Γιεζεί και είπε, Κάλεσον ταύτην την Σουναμίτιν. Και εκάλεσεν αυτήν· και ότε εισήλθε προς αυτόν, είπε, Λάβε τον υιόν σου.
\par 37 Και εκείνη εισήλθε και έπεσεν εις τους πόδας αυτού και προσεκύνησεν έως εδάφους, και εσήκωσε τον υιόν αυτής και εξήλθεν.
\par 38 Ο δε Ελισσαιέ επέστρεψεν εις Γάλγαλα· και ήτο πείνα εν τη γή· και οι υιοί των προφητών εκάθηντο έμπροσθεν αυτού· και είπε προς τον υπηρέτην αυτού, Στήσον τον λέβητα τον μέγαν και ψήσον μαγείρευμα διά τους υιούς των προφητών.
\par 39 Και εξελθών τις εις τον αγρόν διά να συνάξη χόρτα, εύρηκεν αγριοκολοκύνθην, και εσύναξεν απ' αυτής άγρια κολοκύνθια εωσού εγέμισε το ιμάτιον αυτού, και επιστρέψας, έκοψεν αυτά εις τον λέβητα του μαγειρεύματος, επειδή δεν εγνώριζον αυτά.
\par 40 Έπειτα εκένωσαν εις τους ανθρώπους διά να φάγωσι και καθώς έφαγον εκ του μαγειρεύματος, εξεφώνησαν και είπον, Άνθρωπε του Θεού, θάνατος είναι εν τω λέβητι. Και δεν ηδύναντο να φάγωσιν.
\par 41 Ο δε είπε, Φέρετε άλευρον. Και έρριψεν αυτό εις τον λέβητα. Έπειτα είπε, Κένωσον εις τον λαόν, διά να φάγωσι. Και δεν ήτο ουδέν κακόν εν τω λέβητι.
\par 42 Και ήλθεν άνθρωπός τις από Βάαλ-σαλισά, και έφερεν εις τον άνθρωπον του Θεού άρτον από των πρωτογεννημάτων, είκοσι κρίθινα ψωμία και νωπά αστάχυα σίτου, εν τω σάκκω αυτού. Και είπε, Δος εις τον λαόν, διά να φάγωσι.
\par 43 Και ο θεράπων αυτού είπε, Τι να βάλω τούτο έμπροσθεν εκατόν ανθρώπων; Ο δε είπε, Δος εις τον λαόν, διά να φάγωσι διότι ούτω λέγει Κύριος· Θέλουσι φάγει και αφήσει υπόλοιπον.
\par 44 Τότε έβαλεν έμπροσθεν αυτών, και έφαγον και αφήκαν υπόλοιπον, κατά τον λόγον του Κυρίου.

\chapter{5}

\par 1 Ο δε Νεεμάν, ο στρατηγός του βασιλέως της Συρίας, ήτο ανήρ μέγας ενώπιον του κυρίου αυτού και τιμώμενος, επειδή ο Κύριος δι' αυτού έδωκε σωτηρίαν εις την Συρίαν· και ο άνθρωπος ήτο δυνατός εν ισχύϊ, λεπρός όμως.
\par 2 Εξήλθον δε οι Σύριοι κατά τάγματα και έφερον αιχμάλωτον εκ της γης του Ισραήλ μικράν τινά κόρην· και υπηρέτει την γυναίκα του Νεεμάν.
\par 3 Και είπε προς την κυρίαν αυτής, Είθε να ήτο ο κύριός μου έμπροσθεν του προφήτου του εν Σαμαρεία, διότι ήθελεν ιατρεύσει αυτόν από της λέπρας αυτού.
\par 4 Και εισελθών ο Νεεμάν απήγγειλε προς τον κύριον αυτού, λέγων, Ούτω και ούτως ελάλησεν η κόρη η εκ της γης του Ισραήλ.
\par 5 Και είπεν ο βασιλεύς της Συρίας, Ελθέ, ύπαγε, και θέλω στείλει επιστολήν προς τον βασιλέα του Ισραήλ. Και ανεχώρησε και έλαβεν εν τη χειρί αυτού δέκα τάλαντα αργυρίου και εξ χιλιάδας χρυσούς και δέκα αλλαγάς ενδυμάτων·
\par 6 Και έφερε την επιστολήν προς τον βασιλέα του Ισραήλ, λέγουσαν, Και τώρα καθώς έλθη επιστολή αύτη προς σε, ιδού, έστειλα προς σε Νεεμάν τον δούλον μου, διά να ιατρεύσης αυτόν από της λέπρας αυτού.
\par 7 Και καθώς ανέγνωσεν ο βασιλεύς του Ισραήλ την επιστολήν, διέσχισε τα ιμάτια αυτού, και είπε, Θεός είμαι εγώ, διά να θανατόνω και να ζωοποιώ, ώστε ούτος στέλλει προς εμέ να ιατρεύσω άνθρωπον από της λέπρας αυτού; γνωρίσατε λοιπόν, παρακαλώ, και ιδέτε ότι ούτος ζητεί πρόφασιν εναντίον μου.
\par 8 Ως δε ήκουσεν ο Ελισσαιέ, ο άνθρωπος του Θεού, ότι ο βασιλεύς του Ισραήλ διέσχισε τα ιμάτια αυτού, απέστειλε προς τον βασιλέα, λέγων, Διά τι διέσχισας τα ιμάτιά σου; ας έλθη τώρα προς εμέ, και θέλει γνωρίσει ότι είναι προφήτης εν τω Ισραήλ.
\par 9 Τότε ήλθεν ο Νεεμάν μετά των ίππων αυτού και μετά της αμάξης αυτού, και εστάθη εις την θύραν της οικίας του Ελισσαιέ.
\par 10 Και απέστειλε προς αυτόν ο Ελισσαιέ μηνυτήν, λέγων, Ύπαγε και λούσθητι επτάκις εν τω Ιορδάνη, και θέλει επανέλθει η σαρξ σου εις σε, και θέλεις καθαρισθή.
\par 11 Ο δε Νεεμάν εθυμώθη και ανεχώρησε και είπεν, Ιδού, εγώ έλεγον, Θέλει βεβαίως εξέλθει προς εμέ και θέλει σταθή και επικαλεσθή το όνομα Κυρίου του Θεού αυτού, και διακινήσει την χείρα αυτού επί τον τόπον και ιατρεύσει τον λεπρόν·
\par 12 ο Αβανά και ο Φαρφάρ, ποταμοί της Δαμασκού, δεν είναι καλήτεροι υπέρ πάντα τα ύδατα του Ισραήλ; δεν ηδυνάμην να λουσθώ εν αυτοίς και να καθαρισθώ; Και στραφείς, ανεχώρησε μετά θυμού.
\par 13 Επλησίασαν δε οι δούλοι αυτού και ελάλησαν προς αυτόν και είπον· Πάτερ μου, εάν ο προφήτης ήθελε σοι ειπεί μέγα πράγμα, δεν ήθελες κάμει αυτό; πόσω μάλλον τώρα, όταν σοι λέγη, Λούσθητι και καθαρίσθητι;
\par 14 Τότε κατέβη και εβυθίσθη επτάκις εις τον Ιορδάνην, κατά τον λόγον του ανθρώπου του Θεού· και η σαρξ αυτού αποκατέστη ως σαρξ παιδίου μικρού, και εκαθαρίσθη.
\par 15 Και επέστρεψε προς τον άνθρωπον του Θεού, αυτός και πάσα η συνοδία αυτού, και ήλθε και εστάθη έμπροσθεν αυτού· και είπεν, Ιδού, τώρα εγνώρισα ότι δεν είναι Θεός εν πάση τη γη, ειμή εν τω Ισραήλ· όθεν τώρα δέχθητι, παρακαλώ, δώρον παρά του δούλου σου.
\par 16 Ο δε είπε, Ζη Κύριος, ενώπιον του οποίον παρίσταμαι, δεν θέλω δεχθή. Ο δε εβίαζεν αυτόν να δεχθή, αλλά δεν έστερξε.
\par 17 Και είπεν ο Νεεμάν, Και αν μη, ας δοθή, παρακαλώ, εις τον δούλον σου δύο ημιόνων φορτίον εκ του χώματος τούτου, διότι ο δούλός σου δεν θέλει προσφέρει εις το εξής ολοκαύτωμα ουδέ θυσίαν εις άλλους θεούς, παρά μόνον εις τον Κύριον·
\par 18 περί τούτου του πράγματος ας συγχωρήση ο Κύριος τον δούλον σου, ότι, όταν εισέρχηται ο κύριός μου εις τον οίκον του Ριμμών διά να προσκυνήση εκεί, και στηρίζηται επί την χείρα μου, και εγώ κλίνω εμαυτόν εν τω οίκω του Ριμμών, ενώ κλίνω εμαυτόν εν τω οίκω του Ριμμών, ο Κύριος ας συγχωρήση τον δούλον σου περί του πράγματος τούτου
\par 19 Και είπε προς αυτόν, Ύπαγε εν ειρήνη. Και ανεχώρησεν απ' αυτού μικρόν τι διάστημα.
\par 20 Είπε δε ο Γιεζεί, ο υπηρέτης του Ελισσαιέ του ανθρώπου του Θεού, Ιδού, εφείσθη ο κύριός μου του Νεεμάν τούτου του Συρίου, ώστε να μη λάβη εκ της χειρός αυτού εκείνο το οποίον έφερε· πλην, ζη Κύριος, εγώ θέλω τρέξει κατόπιν αυτού και θέλω λάβει τι παρ' αυτού.
\par 21 Και έτρεξεν ο Γιεζεί κατόπιν του Νεεμάν. Και ότε είδεν αυτόν ο Νεεμάν τρέχοντα κατόπιν αυτού, επήδησεν εκ της αμάξης εις συνάντησιν αυτού και είπε, Καλώς έχετε;
\par 22 Ο δε είπε, Καλώς· ο κύριός μου με απέστειλε, λέγων, Ιδού, ταύτην την ώραν ήλθον προς εμέ, εκ του όρους Εφραΐμ, δύο νέοι εκ των υιών των προφητών· δος εις αυτούς, παρακαλώ, εν τάλαντον αργυρίου και δύο αλλαγάς ενδυμάτων.
\par 23 Και είπεν ο Νεεμάν, Λάβε ευχαρίστως δύο τάλαντα. Και εβίασεν αυτόν, και έδωσε τα δύο τάλαντα του αργυρίου εις δύο θυλάκια, μετά δύο αλλαγών ενδυμάτων· και επέθεσεν αυτά εις δύο εκ των δούλων αυτού, και εβάσταζον αυτά έμπροσθεν αυτού.
\par 24 Και ότε ήλθεν εις Οφήλ, έλαβεν αυτά εκ των χειρών αυτών και εφύλαξεν εν τω οίκω· και απέλυσε τους άνδρας, και ανεχώρησαν.
\par 25 Αυτός δε εισήλθε και εστάθη έμπροσθεν του κυρίου αυτού. Και είπε προς αυτόν ο Ελισσαιέ, Πόθεν, Γιεζεί; Ο δε είπεν, Ο δούλός σου δεν υπήγε πούποτε.
\par 26 Και είπε προς αυτόν, Δεν υπήγεν η καρδία μου μετά σου, ότε ο άνθρωπος επέστρεψεν από της αμάξης αυτού εις συνάντησίν σου; είναι καιρός να λάβης αργύριον και να λάβης ιμάτια και ελαιώνας και αμπελώνας και πρόβατα και βόας και δούλους και δούλας;
\par 27 διά τούτο η λέπρα του Νεεμάν θέλει κολληθή εις σε και εις το σπέρμα σου εις τον αιώνα. Και εξήλθεν απ' έμπροσθεν αυτού λελεπρωμένος ως χιών.

\chapter{6}

\par 1 Και είπον οι υιοί των προφητών προς τον Ελισσαιέ, Ιδού τώρα, ο τόπος, εις τον οποίον ημείς κατοικούμεν ενώπιόν σου, είναι στενός δι' ημάς·
\par 2 ας υπάγωμεν, παρακαλούμεν, έως του Ιορδάνου, και εκείθεν ας λάβωμεν έκαστος μίαν δοκόν, και ας κάμωμεν εις εαυτούς εκεί τόπον, διά να κατοικώμεν εκεί. Ο δε είπεν, Υπάγετε.
\par 3 Και είπεν ο εις, Ευαρεστήθητι, παρακαλώ, να έλθης μετά των δούλων σου. Και είπε, Θέλω ελθεί.
\par 4 Και υπήγε μετ' αυτών. Και ελθόντες εις τον Ιορδάνην, έκοπτον τα ξύλα.
\par 5 Ενώ δε ο εις κατέβαλλε την δοκόν, έπεσε το σιδήριον εις το ύδωρ· και εβόησε και είπεν, Ω, κύριε· και τούτο ήτο δάνειον·
\par 6 είπε δε ο άνθρωπος του Θεού, Που έπεσε; Και έδειξε τον τόπον εις αυτόν. Τότε έκοψε σχίζαν ξύλου, και έρριψεν εκεί· και το σιδήριον επέπλευσε.
\par 7 Και είπεν, Ανάλαβε προς σεαυτόν. Και εκτείνας την χείρα αυτού, έλαβεν αυτό.
\par 8 Ο δε βασιλεύς της Συρίας επολέμει εναντίον του Ισραήλ, και συνεβουλεύθη μετά των δούλων αυτού, λέγων, Εις τον δείνα και δείνα τόπον θέλω στρατοπεδεύσει.
\par 9 Και απέστειλεν ο άνθρωπος του Θεού προς τον βασιλέα του Ισραήλ, λέγων, Φυλάχθητι να μη περάσης τον τόπον εκείνον, διότι οι Σύριοι στρατοπεδεύουσιν εκεί.
\par 10 Και απέστειλεν ο βασιλεύς του Ισραήλ εις τον τόπον, τον οποίον είπε προς αυτόν ο άνθρωπος του Θεού και παρήγγειλε περί αυτού· και προεφυλάχθη εκείθεν ουχί άπαξ ουδέ δις.
\par 11 Και εταράχθη η καρδία του βασιλέως της Συρίας διά το πράγμα τούτο· και συγκαλέσας τους δούλους αυτού, είπε προς αυτούς, Δεν θέλετε με αναγγείλει, τις εξ ημών είναι υπέρ του βασιλέως του Ισραήλ;
\par 12 Και είπεν εις εκ των δούλων αυτού, Ουδείς, κύριέ μου βασιλεύ· αλλ' ο Ελισσαιέ ο προφήτης, ο εν τω Ισραήλ, αναγγέλλει προς τον βασιλέα του Ισραήλ τους λόγους, τους οποίους λαλείς εν τω ταμείω του κοιτώνός σου.
\par 13 Και είπεν, Υπάγετε και ιδέτε που είναι, διά να στείλω να συλλάβω αυτόν. Και ανήγγειλαν προς αυτόν, λέγοντες, Ιδού, είναι εν Δωθάν.
\par 14 Και απέστειλεν εκεί ίππους και αμάξας και στράτευμα μέγα, οίτινες, ελθόντες διά νυκτός, περιεκύκλωσαν την πόλιν.
\par 15 Και ότε εξηγέρθη το πρωΐ ο υπηρέτης του ανθρώπου του Θεού και εξήλθεν, ιδού, στράτευμα είχε περικυκλωμένην την πόλιν με ίππους και αμάξας. Και είπεν ο υπηρέτης αυτού προς αυτόν, Ω, κύριε, τι θέλομεν κάμει;
\par 16 Ο δε είπε, Μη φοβού· διότι πλειότεροι είναι οι μεθ' ημών παρά τους μετ' αυτών.
\par 17 Και προσηυχήθη ο Ελισσαιέ και είπε, Κύριε, Άνοιξον, δέομαι, τους οφθαλμούς αυτού, διά να ίδη. Και ήνοιξεν ο Κύριος τους οφθαλμούς του υπηρέτου, και είδε· και ιδού, το όρος ήτο πλήρες ίππων και αμαξών πυρός περί τον Ελισσαιέ.
\par 18 Και ότε κατέβησαν προς αυτόν οι Σύριοι, προσηυχήθη ο Ελισσαιέ προς τον Κύριον και είπε, Πάταξον, δέομαι, τον λαόν τούτον με αορασίαν. Και επάταξεν αυτούς με αορασίαν, κατά τον λόγον του Ελισσαιέ.
\par 19 Και είπε προς αυτούς ο Ελισσαιέ, Δεν είναι αύτη η οδός ουδέ αύτη η πόλις· έλθετε κατόπιν μου, και θέλω σας φέρει προς τον άνθρωπον, τον οποίον ζητείτε. Και έφερεν αυτούς εις την Σαμάρειαν.
\par 20 Και ότε ήλθον εις την Σαμάρειαν, είπεν ο Ελισσαιέ, Άνοιξον, Κύριε, τους οφθαλμούς τούτων, διά να βλέπωσι. Και ήνοιξεν ο Κύριος τους οφθαλμούς αυτών, και είδον· και ιδού, ήσαν εκ τω μέσω της Σαμαρείας.
\par 21 Και ως είδεν αυτούς ο βασιλεύς του Ισραήλ, είπε προς τον Ελισσαιέ, Να πατάξω, να πατάξω, πάτερ μου;
\par 22 Ο δε είπε, Μη πατάξης· ήθελες πατάξει εκείνους, τους οποίους ηχμαλώτευσας διά της ρομφαίας σου και διά του τόξου σου; θες άρτον και ύδωρ έμπροσθεν αυτών, και ας φάγωσι και ας πίωσι και ας απέλθωσι προς τον κύριον αυτών.
\par 23 Και έθεσεν έμπροσθεν αυτών άφθονον τροφήν· και αφού έφαγον και έπιον, απέστειλεν αυτούς, και ανεχώρησαν προς τον κύριον αυτών. Και δεν ήλθον πλέον τα τάγματα της Συρίας εις την γην του Ισραήλ.
\par 24 Μετά δε ταύτα ο Βεν-αδάδ βασιλεύς της Συρίας συνήθροισεν άπαν το στράτευμα αυτού, και ανέβη και επολιόρκησε την Σαμάρειαν.
\par 25 Έγεινε δε πείνα μεγάλη εν Σαμαρεία· και ιδού, επολιόρκουν αυτήν, εωσού κεφαλή όνου επωλήθη δι' ογδοήκοντα αργύρια και το τέταρτον ενός κάβου κόπρου περιστερών διά πέντε αργύρια.
\par 26 Και ενώ διέβαινεν ο βασιλεύς του Ισραήλ επί του τείχους, γυνή τις εβόησε προς αυτόν, λέγουσα, Σώσον, κύριέ μου βασιλεύ.
\par 27 Ο δε είπεν, Εάν ο Κύριος δεν σε σώση, πόθεν θέλω σε σώσει εγώ; μη εκ του αλωνίου ή εκ του ληνού;
\par 28 Και είπε προς αυτήν ο βασιλεύς, Τι έχεις; Η δε είπε, Η γυνή αύτη μοι είπε, Δος τον υιόν σου, διά να φάγωμεν αυτόν σήμερον, και αύριον θέλομεν φάγει τον υιόν μου·
\par 29 και εβράσαμεν τον υιόν μου και εφάγομεν αυτόν· είπον δε προς αυτήν την ακόλουθον ημέραν, Δος τον υιόν σου, διά να φάγωμεν αυτόν· η δε έκρυψε τον υιόν αυτής.
\par 30 Και ως ήκουσεν ο βασιλεύς τους λόγους της γυναικός, διέρρηξε τα ιμάτια αυτού· και ενώ διέβαινεν επί του τείχους, ο λαός είδε, και ιδού, σάκκος έσωθεν επί της σαρκός αυτού.
\par 31 Και είπεν, Ούτω να κάμη εις εμέ ο Θεός και ούτω να προσθέση, εάν η κεφαλή του Ελισσαιέ υιού του Σαφάτ σταθή επάνω αυτού σήμερον.
\par 32 Ο δε Ελισσαιέ εκάθητο εν τω οίκω αυτού, και οι πρεσβύτεροι εκάθηντο μετ' αυτού· και απέστειλεν ο βασιλεύς άνδρα απ' έμπροσθεν αυτού· πριν δε έλθη προς αυτόν ο μηνυτής, αυτός είπε προς τους πρεσβυτέρους, Δεν βλέπετε ότι ούτος ο υιός του φονευτού έστειλε να αφαιρέση την κεφαλήν μου; βλέπετε, καθώς έλθη ο μηνυτής, κλείσατε την θύραν και εμποδίσατε αυτόν προς την θύραν· η φωνή των ποδών του κυρίου αυτού δεν είναι εξόπισθεν αυτού;
\par 33 Και ενώ έτι ελάλει μετ' αυτών, ιδού, κατέβη προς αυτόν ο μηνυτής· και είπεν, Ιδού, παρά Κυρίου είναι το κακόν τούτο· τι πλέον να ελπίσω εις τον Κύριον;

\chapter{7}

\par 1 Είπε δε ο Ελισσαιέ, Ακούσατε τον λόγον του Κυρίου· Ούτω λέγει Κύριος· Αύριον, περί την ώραν ταύτην, εν μέτρον σεμιδάλεως θέλει πωληθή δι' ένα σίκλον και δύο μέτρα κριθής δι' ένα σίκλον, εν τη πύλη της Σαμαρείας.
\par 2 Και απεκρίθη προς τον άνθρωπον του Θεού ο άρχων, επί του οποίου την χείρα εστηρίζετο ο βασιλεύς, και είπε, Και εάν ο Κύριος ήθελε κάμει παράθυρα εις τον ουρανόν, ηδύνατο το πράγμα τούτο να γείνη; Ο δε είπεν, Ιδού, θέλεις ιδεί με τους οφθαλμούς σου, δεν θέλεις όμως φάγει εξ αυτού.
\par 3 Ήσαν δε τέσσαρες άνδρες λεπροί εν τη εισόδω της πύλης· και είπον ο εις προς τον άλλον, Διά τι ημείς καθήμεθα εδώ εωσού αποθάνωμεν;
\par 4 εάν είπωμεν, να εισέλθωμεν εις την πόλιν, η πείνα είναι εν τη πόλει, και θέλομεν αποθάνει εκεί· εάν δε καθήμεθα εδώ, πάλιν θέλομεν αποθάνει· τώρα λοιπόν έλθετε, και ας πέσωμεν εις το στρατόπεδον των Συρίων· εάν αφήσωσιν ημάς ζώντας, θέλομεν ζήσει. και εάν θανατώσωσιν ημάς, θέλομεν αποθάνει.
\par 5 Και εσηκώθησαν ότε εσκόταζε, διά να εισέλθωσιν εις το στρατόπεδον των Συρίων· και ότε ήλθον έως του άκρου του στρατοπέδου της Συρίας, ιδού, δεν ήτο άνθρωπος εκεί.
\par 6 Διότι ο Κύριος είχε κάμει να ακουσθή εν τω στρατοπέδω των Συρίων κρότος αμαξών και κρότος ίππων, κρότος μεγάλου στρατεύματος· και είπον προς αλλήλους, Ιδού, ο βασιλεύς του Ισραήλ εμίσθωσεν εναντίον ημών τους βασιλείς των Χετταίων και τους βασιλείς των Αιγυπτίων, διά να έλθωσιν εφ' ημάς.
\par 7 Όθεν σηκωθέντες έφυγον εν τω σκότει, και εγκατέλιπον τας σκηνάς αυτών και τους ίππους αυτών και τους όνους αυτών, το στρατόπεδον όπως ήτο, και έφυγον διά την ζωήν αυτών.
\par 8 Και ότε οι λεπροί ούτοι ήλθον έως του άκρου του στρατοπέδου, εισήλθον εις μίαν σκηνήν και έφαγον και έπιον, και λαβόντες εκείθεν αργύριον και χρυσίον και ιμάτια, υπήγαν και έκρυψαν αυτά· επιστρέψαντες δε εισήλθον εις άλλην σκηνήν, και έλαβον άλλα εκείθεν και υπήγαν και έκρυψαν και ταύτα.
\par 9 Τότε είπον προς αλλήλους, Ημείς δεν κάμνομεν καλά· η ημέρα αύτη είναι ημέρα αγαθών αγγελιών, και αν ημείς σιωπώμεν και περιμένωμεν μέχρι του φωτός της αυγής, συμφορά τις θέλει επέλθει εφ' ημάς· έλθετε λοιπόν, και ας υπάγωμεν να αναγγείλωμεν ταύτα εις τον οίκον του βασιλέως.
\par 10 Ήλθον λοιπόν και εβόησαν προς τους θυρωρούς της πόλεως· και ανήγγειλαν προς αυτούς, λέγοντες, Ήλθομεν εις το στρατόπεδον των Συρίων, και ιδού, δεν ήτο εκεί άνθρωπος ουδέ φωνή ανθρώπου, ειμή ίπποι δεδεμένοι και όνοι δεδεμένοι και σκηναί καθώς ευρίσκοντο.
\par 11 Και εβόησαν οι θυρωροί και ανήγγειλαν τούτο ένδον εις τον οίκον του βασιλέως.
\par 12 Και σηκωθείς ο βασιλεύς την νύκτα, είπε προς τους δούλους αυτού, Τώρα θέλω φανερώσει προς εσάς τι έκαμον οι Σύριοι εις ημάς· εγνώρισαν ότι είμεθα πεινασμένοι και εξήλθον εκ του στρατοπέδου, διά να κρυφθώσιν εν τοις αγροίς, λέγοντες, Όταν εξέλθωσιν εκ της πόλεως, θέλομεν συλλάβει αυτούς ζώντας, και εις την πόλιν θέλομεν εισέλθει.
\par 13 Αποκριθείς δε εις εκ των δούλων αυτού είπεν, Ας λάβωσι, παρακαλώ, πέντε εκ των υπολειπομένων ίππων, οίτινες απέμειναν εν τη πόλει, ιδού, αυτοί είναι καθώς είπαν το πλήθος του Ισραήλ το εναπολειφθέν εν αυτή· ιδού, είναι καθώς άπαν το πλήθος των Ισραηλιτών οίτινες κατηναλώθησαν· και ας αποστείλωμεν διά να ίδωμεν.
\par 14 Έλαβον λοιπόν δύο ζεύγη ίππων και απέστειλεν ο βασιλεύς οπίσω του στρατοπέδου των Συρίων, λέγων, Υπάγετε και ιδέτε.
\par 15 Και υπήγαν οπίσω αυτών έως του Ιορδάνου· και ιδού, πάσα η οδός πλήρης ιματίων και σκευών, τα οποία οι Σύριοι είχον ρίψει εκ της βίας αυτών. Και επιστρέψαντες οι μηνυταί ανήγγειλαν τούτο προς τον βασιλέα.
\par 16 Και εξήλθεν ο λαός, και ήρπασαν το στρατόπεδον των Συρίων. Και επωλήθη εν μέτρον σεμιδάλεως δι' ένα σίκλον και δύο μέτρα κριθής δι' ένα σίκλον, κατά τον λόγον του Κυρίου.
\par 17 Και κατέστησεν ο βασιλεύς επί της πύλης τον άρχοντα, επί του οποίου την χείρα εστηρίζετο· και κατεπάτησεν ο λαός αυτόν εν τη πύλη, και απέθανε· καθώς ελάλησεν ο άνθρωπος του Θεού, όστις ελάλησεν ότε ο βασιλεύς κατέβη προς αυτόν.
\par 18 Και, καθώς ελάλησεν ο άνθρωπος του Θεού προς τον βασιλέα, λέγων, Δύο μέτρα κριθής δι' ένα σίκλον και εν μέτρον σεμιδάλεως δι' ένα σίκλον θέλουσιν είσθαι αύριον, περί την ώραν ταύτην, εν τη πύλη της Σαμαρείας,
\par 19 ο δε άρχων απεκρίθη προς τον άνθρωπον του Θεού και είπε, Και αν τώρα ο Κύριος ήθελε κάμει παράθυρα εις τον ουρανόν, ηδύνατο τοιούτον πράγμα να γείνη; και εκείνος είπεν, Ιδού, θέλεις ιδεί τούτο με τους οφθαλμούς σου· αλλά δεν θέλεις φάγει εξ αυτού,
\par 20 ούτω και έγεινεν εις αυτόν· διότι ο λαός κατεπάτησεν αυτόν εν τη πύλη, και απέθανε.

\chapter{8}

\par 1 Και ελάλησεν ο Ελισσαιέ προς την γυναίκα, της οποίας ανεζωοποίησε τον υιόν, λέγων, Σηκώθητι και ύπαγε, συ και ο οίκός σου, και παροίκησον όπου αν δυνηθής να παροικήσης· διότι ο Κύριος εκάλεσε την πείναν, και θέλει μάλιστα επέλθει επί την γην επτά έτη.
\par 2 Και σηκωθείσα η γυνή, έκαμε κατά τον λόγον του ανθρώπου του Θεού· και υπήγεν αυτή και ο οίκος αυτής, και παρώκησεν εν τη γη των Φιλισταίων επτά έτη.
\par 3 Μετά δε το τέλος των επτά ετών, επέστρεψεν η γυνή εκ της γης των Φιλισταίων· και εξήλθε να βοήση προς τον βασιλέα περί της οικίας αυτής και περί των αγρών αυτής.
\par 4 Και ελάλησεν ο βασιλεύς προς τον Γιεζεί, τον υπηρέτην του ανθρώπου του Θεού, λέγων, Διηγήθητί μοι, παρακαλώ, πάντα τα μεγαλεία τα οποία έκαμεν ο Ελισσαιέ.
\par 5 Και ενώ διηγείτο προς τον βασιλέα πως ανεζωοποίησε τον νεκρόν, ιδού, η γυνή, της οποίας τον υιόν είχεν αναζωοποιήσει, εβόησε προς τον βασιλέα περί της οικίας αυτής και περί των αγρών αυτής. Και είπεν ο Γιεζεί, Κύριέ μου βασιλεύ, αύτη είναι η γυνή και ούτος ο υιός αυτής, τον οποίον ανεζωοποίησεν ο Ελισσαιέ.
\par 6 Και ηρώτησεν ο βασιλεύς την γυναίκα, και αυτή διηγήθη το πράγμα προς αυτόν. Τότε έδωκεν εις αυτήν ο βασιλεύς ευνούχον, λέγων, Επίστρεψον πάντα τα πράγματα αυτής και πάντα τα προϊόντα των αγρών αυτής, αφ' ης ημέρας αφήκε την γην μέχρι του νυν.
\par 7 Ο δε Ελισσαιέ ήλθεν εις Δαμασκόν. Και Βεν-αδάδ ο βασιλεύς της Συρίας ήτο άρρωστος· και απήγγειλαν προς αυτόν, λέγοντες, Ο άνθρωπος του Θεού ήλθεν έως εδώ.
\par 8 Και είπεν ο βασιλεύς προς τον Αζαήλ, Λάβε εις την χείρα σου δώρον και ύπαγε εις συνάντησιν του ανθρώπου του Θεού και ερώτησον δι' αυτού τον Κύριον, λέγων, Θέλω αναλάβει εκ της αρρωστίας ταύτης;
\par 9 Και υπήγεν ο Αζαήλ εις συνάντησιν αυτού, λαβών δώρον εις την χείρα αυτού και από παντός αγαθού της Δαμασκού, τεσσαράκοντα καμήλων φορτίον· και ελθών εστάθη έμπροσθεν αυτού και είπεν, Ο υιός σου Βεν-αδάδ, ο βασιλεύς της Συρίας, με απέστειλε προς σε, λέγων, Θέλω αναλάβει εκ της αρρωστίας ταύτης;
\par 10 Και είπε προς αυτόν ο Ελισσαιέ, Ύπαγε, ειπέ προς αυτόν, Ναι, θέλεις αναλάβει πλην ο Κύριος έδειξεν εις εμέ ότι εξάπαντος θέλει αποθάνει.
\par 11 Και έστησε το πρόσωπον αυτού ακίνητον, εωσού ερυθρίασε· και έκλαυσεν ο άνθρωπος του Θεού.
\par 12 Και είπεν ο Αζαήλ, Διά τι κλαίεις, κύριέ μου; Ο δε απεκρίθη, Διότι εξεύρω όσα κακά θέλεις κάμει εις τους υιούς Ισραήλ· τα οχυρώματα αυτών θέλεις παραδώσει εις πυρ, και τους νέους αυτών θέλεις αποκτείνει εν ρομφαία, και τα νήπια αυτών θέλεις συντρίψει, και τας εγκυμονούσας αυτών θέλεις διασχίσει.
\par 13 Και είπεν ο Αζαήλ, Αλλά τι είναι ο δούλός σου, ο κύων, ώστε να κάμη το μέγα τούτο πράγμα; Και είπεν ο Ελισσαιέ, Ο Κύριος έδειξεν εις εμέ, ότι συ θέλεις βασιλεύσει επί της Συρίας.
\par 14 Τότε ανεχώρησεν από του Ελισσαιέ και ήλθε προς τον κύριον αυτού· ο δε είπε προς αυτόν, Τι σοι είπεν ο Ελισσαιέ; Και απεκρίθη, Μοι είπε, Ναι, θέλεις αναλάβει.
\par 15 Την δε ακόλουθον ημέραν έλαβε το σκέπασμα και εμβάψας εις ύδωρ, εξήπλωσεν επί του προσώπου αυτού· και απέθανε· και αντ' αυτού εβασίλευσεν ο Αζαήλ.
\par 16 Εν δε τω πέμπτω έτει του Ιωράμ, υιού του Αχαάβ βασιλέως του Ισραήλ, βασιλεύοντος Ιωσαφάτ επί του Ιούδα, εβασίλευσεν Ιωράμ, ο υιός του Ιωσαφάτ βασιλέως του Ιούδα.
\par 17 Τριάκοντα δύο ετών ηλικίας ήτο ότε εβασίλευσεν· εβασίλευσε δε οκτώ έτη εν Ιερουσαλήμ.
\par 18 Και περιεπάτησεν εν τη οδώ των βασιλέων του Ισραήλ, καθώς έπραξεν ο οίκος του Αχαάβ· διότι η θυγάτηρ του Αχαάβ ήτο γυνή αυτού· και έπραξε πονηρά ενώπιον του Κυρίου.
\par 19 Αλλ' ο Κύριος δεν ηθέλησε να εξολοθρεύση τον Ιούδαν, χάριν Δαβίδ του δούλου αυτού, καθώς είπε προς αυτόν ότι θέλει δώσει εις αυτόν λύχνον και εις τους υιούς αυτού εις τον αιώνα.
\par 20 Εν ταις ημέραις αυτού απεστάτησεν ο Εδώμ από της υποταγής του Ιούδα, και κατέστησαν βασιλέα εφ' εαυτών.
\par 21 Όθεν διέβη ο Ιωράμ εις Σαείρ, και πάσαι αι άμαξαι μετ' αυτού· και σηκωθείς διά νυκτός, επάταξε τους Ιδουμαίους τους κύκλω αυτού και τους αμαξάρχας· ο δε λαός έφυγον εις τας σκηνάς αυτών.
\par 22 Πλην ο Εδώμ απεστάτησεν από της υποταγής του Ιούδα, έως της ημέρας ταύτης. Τότε κατά τον αυτόν καιρόν απεστάτησεν η Λιβνά.
\par 23 Αι δε λοιπαί πράξεις του Ιωράμ και πάντα όσα έπραξε, δεν είναι γεγραμμένα εν τω βιβλίω των χρονικών των βασιλέων του Ιούδα;
\par 24 Και εκοιμήθη ο Ιωράμ μετά των πατέρων αυτού και ετάφη μετά των πατέρων αυτού εν τη πόλει Δαβίδ· εβασίλευσε δε αντ' αυτού Οχοζίας ο υιός αυτού.
\par 25 Εν τω δωδεκάτω έτει του Ιωράμ, υιού του Αχαάβ βασιλέως του Ισραήλ, εβασίλευσεν Οχοζίας, ο υιός του Ιωράμ βασιλέως του Ιούδα.
\par 26 Εικοσιδύο ετών ηλικίας ήτο ο Οχοζίας ότε εβασίλευσεν· εβασίλευσε δε εν έτος εν Ιερουσαλήμ. Και το όνομα της μητρός αυτού ήτο Γοθολία, θυγάτηρ του Αμρί, βασιλέως του Ισραήλ.
\par 27 Και περιεπάτησεν εν τη οδώ του οίκου του Αχαάβ, και έπραξε πονηρά ενώπιον του Κυρίου, καθώς ο οίκος του Αχαάβ· διότι ήτο γαμβρός του οίκου του Αχαάβ.
\par 28 Και υπήγε μετά του Ιωράμ υιού του Αχαάβ εις πόλεμον εναντίον του Αζαήλ βασιλέως της Συρίας εις Ραμώθ-γαλαάδ· και ετραυμάτισαν οι Σύριοι τον Ιωράμ.
\par 29 Και επέστρεψεν ο βασιλεύς Ιωράμ διά να ιατρευθή εν Ιεζραέλ από των τραυμάτων, τα οποία οι Σύριοι έκαμον εις αυτόν εν Ραμά, ότε επολέμει εναντίον του Αζαήλ βασιλέως της Συρίας. Οχοζίας δε ο υιός του Ιωράμ, βασιλεύς του Ιούδα, κατέβη διά να ίδη τον Ιωράμ υιόν του Αχαάβ εν Ιεζραέλ, διότι ήτο άρρωστος.

\chapter{9}

\par 1 Ελισσαιέ δε ο προφήτης εκάλεσεν ένα εκ των υιών των προφητών και είπε προς αυτόν, Περίζωσον την οσφύν σου και λάβε εις την χείρα σου την φιάλην ταύτην του ελαίου και ύπαγε εις Ραμώθ-γαλαάδ·
\par 2 και όταν εισέλθης εκεί, θέλεις ιδεί εκεί τον Ιηού, υιόν του Ιωσαφάτ, υιού του Νιμσί· και θέλεις εισέλθει και σηκώσει αυτόν εκ μέσου των αδελφών αυτού και θέλεις εισαγάγει αυτόν εις το ενδότερον δωμάτιον·
\par 3 και λαβών την φιάλην του ελαίου, θέλεις επιχέει επί την κεφαλήν αυτού και ειπεί, Ούτω λέγει Κύριος· Σε έχρισα βασιλέα επί τον Ισραήλ· τότε ανοίξας την θύραν, φύγε και μη μείνης.
\par 4 Και υπήγεν ο νέος, ο νέος ο προφήτης, εις Ραμώθ-γαλαάδ.
\par 5 Και ότε ήλθεν, ιδού, οι άρχοντες του στρατεύματος εκάθηντο· και είπεν, Έχω λόγον προς σε, ω άρχων. Και ο Ιηού είπε, Προς τίνα εκ πάντων ημών; Ο δε είπε, προς σε, ω άρχων.
\par 6 Και σηκωθείς εισήλθεν εις τον οίκον· και επέχεε το έλαιον επί την κεφαλήν αυτού και είπε προς αυτόν, Ούτω λέγει Κύριος ο Θεός του Ισραήλ· Σε έχρισα βασιλέα επί τον λαόν του Κυρίου, επί τον Ισραήλ·
\par 7 και θέλεις πατάξει τον οίκον του Αχαάβ του κυρίου σου, διά να εκδικήσω τα αίματα των δούλων μου των προφητών και τα αίματα πάντων των δούλων του Κυρίου, εκ χειρός της Ιεζάβελ·
\par 8 διότι πας ο οίκος του Αχαάβ θέλει εξολοθρευθή· και θέλω αφανίσει εκ του Αχαάβ τον ουρούντα εις τον τοίχον και τον κεκλεισμένον και τον αφειμένον εν τω Ισραήλ·
\par 9 και θέλω καταστήσει τον οίκον του Αχαάβ ως τον οίκον του Ιεροβοάμ, υιού του Ναβάτ, και ως τον οίκον του Βαασά, υιού του Αχιά·
\par 10 και την Ιεζάβελ οι κύνες θέλουσι καταφάγει εν τω αγρώ της Ιεζραέλ, και δεν θέλει είσθαι ο θάπτων αυτήν. Και ανοίξας την θύραν, έφυγε.
\par 11 Και εξήλθεν ο Ιηού προς τους δούλους του κυρίου αυτού· και είπε τις προς αυτόν, Ειρήνη; διά τι ήλθε προς σε ο παράφρων ούτος; Ο δε είπε προς αυτούς, Σεις γνωρίζετε τον άνθρωπον και το λέγειν αυτού.
\par 12 Και είπον, Ψευδές είναι· ειπέ εις ημάς, παρακαλούμεν. Ο δε είπεν, Ούτω και ούτως ελάλησε προς εμέ, λέγων, Ούτω λέγει Κύριος· σε έχρισα βασιλέα επί τον Ισραήλ.
\par 13 Τότε έσπευσαν, και λαβόντες έκαστος το ιμάτιον αυτού, έβαλον υπ' αυτόν επί του υψηλοτέρου αναβαθμού· και εσάλπισαν εν σάλπιγγι, λέγοντες, Εβασίλευσεν ο Ιηού.
\par 14 Και ο Ιηού, ο υιός του Ιωσαφάτ, υιού του Νιμσί, έκαμε συνωμοσίαν κατά του Ιωράμ. Ο δε Ιωράμ εφυλάττετο εν Ραμώθ-γαλαάδ, αυτός και άπας ο Ισραήλ, από προσώπου του Αζαήλ, βασιλέως της Συρίας.
\par 15 Είχε δε επιστρέψει ο βασιλεύς Ιωράμ διά να ιατρευθή εν Ιεζραέλ από των τραυμάτων, τα οποία οι Σύριοι έκαμον εις αυτόν, ότε επολέμει εναντίον του Αζαήλ βασιλέως της Συρίας. Και είπεν ο Ιηού· Εάν ήναι η γνώμη σας, ας μη εξέλθη μηδείς φεύγων εκ της πόλεως, διά να υπάγη να απαγγείλη τούτο εν Ιεζραέλ.
\par 16 Και ιππεύσας ο Ιηού, υπήγεν εις Ιεζραέλ· διότι ο Ιωράμ εκοίτετο εκεί. Και Οχοζίας ο βασιλεύς του Ιούδα είχε καταβή να ίδη τον Ιωράμ.
\par 17 Ίστατο δε ο σκοπός επί του πύργου εν Ιεζραήλ και, ιδών την συνοδίαν του Ιηού ερχομένου, είπε, Συνοδίαν βλέπω. Και είπεν ο Ιωράμ· Λάβε επιβάτην και πέμψον εις συνάντησιν αυτών· και ας ερωτήση, Ειρήνη;
\par 18 Υπήγε λοιπόν επιβάτης ίππου εις συνάντησιν αυτού και είπεν, Ούτω λέγει ο βασιλεύς· Ειρήνη; Και είπεν ο Ιηού, Τι σε μέλει περί ειρήνης; στρέψον οπίσω μου. Και ο σκοπός απήγγειλε, λέγων, Ο μηνυτής ήλθε μέχρις αυτών και δεν επέστρεψε.
\par 19 Και απέστειλε δεύτερον επιβάτην ίππου· όστις, ελθών προς αυτούς, είπεν, Ούτω λέγει ο βασιλεύς· Ειρήνη; Και απεκρίθη ο Ιηού, Τι σε μέλει περί ειρήνης; στρέψον οπίσω μου.
\par 20 Και απήγγειλεν ο σκοπός, λέγων, Ήλθε μέχρις αυτών και δεν επέστρεψεν· η δε πορεία είναι ως η πορεία του Ιηού υιού του Νιμσί· διότι οδεύει μανιωδώς.
\par 21 Και είπεν ο Ιωράμ, Ζεύξατε. Και έζευξαν την άμαξαν αυτού. Και εξήλθον Ιωράμ ο βασιλεύς του Ισραήλ και Οχοζίας ο βασιλεύς του Ιούδα, έκαστος εν τη αμάξη αυτού, και υπήγαν εις συνάντησιν του Ιηού, και εύρον αυτόν εν τω αγρώ Ναβουθαί του Ιεζραηλίτου.
\par 22 Και ως είδεν ο Ιωράμ τον Ιηού, είπεν, Ειρήνη, Ιηού; Ο δε απεκρίθη, Τι ειρήνη, ενόσω πληθύνονται αι πορνείαι της Ιεζάβελ της μητρός σου και αι μαγείαι αυτής;
\par 23 Και έστρεψεν ο Ιωράμ τας χείρας αυτού και έφυγε, λέγων προς τον Οχοζίαν, Δόλος, Οχοζία.
\par 24 Και δράξας ο Ιηού το τόξον αυτού, επάταξε τον Ιωράμ μεταξύ των βραχιόνων αυτού· και το βέλος εξήλθε διά της καρδίας αυτού. Ο δε εκάμφθη εν τη αμάξη αυτού.
\par 25 Και είπεν ο Ιηού προς τον Βιδκάρ, τον στρατηγόν αυτού· Λάβε και ρίψον αυτόν εις την μερίδα του αγρού του Ναβουθαί του Ιεζραηλίτου· διότι ενθυμήθητι, ότε εγώ και συ επορευόμεθα έφιπποι οπίσω Αχαάβ του πατρός αυτού, ότι ο Κύριος επρόφερε κατ' αυτού την απόφασιν ταύτην·
\par 26 Ναι, είδον χθές τα αίματα του Ναβουθαί και τα αίματα των υιών αυτού, λέγει Κύριος· και θέλω κάμει εις σε ανταπόδοσιν εν τη μερίδι ταύτη, λέγει Κύριος·-τώρα λοιπόν σήκωσον και ρίψον αυτόν εις την μερίδα ταύτην κατά τον λόγον του Κυρίου.
\par 27 Ο δε Οχοζίας βασιλεύς του Ιούδα, ως είδε τούτο, έφυγε διά της οδού της οικίας του κήπου. Και κατεδίωξεν οπίσω αυτού ο Ιηού και είπε, Πατάξατε και τούτον εν τη αμάξη αυτού. Και έκαμον ούτω, κατά την ανάβασιν Γούρ, πλησίον του Ιβλεάμ. Και έφυγεν εις Μεγιδδώ και εκεί απέθανε.
\par 28 Και έφεραν αυτόν οι δούλοι αυτού επί αμάξης εις Ιερουσαλήμ, και έθαψαν αυτόν εν τω τάφω αυτού, μετά των πατέρων αυτού, εν τη πόλει Δαβίδ.
\par 29 Εβασίλευσε δε ο Οχοζίας επί Ιούδα κατά το ενδέκατον έτος του Ιωράμ υιού του Αχαάβ.
\par 30 Και ήλθεν ο Ιηού εις Ιεζραέλ, και ακούσασα η Ιεζάβελ, έβαψε τους οφθαλμούς αυτής και εκαλλώπισε την κεφαλήν αυτής και διέκυψε διά του παραθύρου.
\par 31 Και, ενώ εισήρχετο εις την πύλην ο Ιηού, είπεν, Ευτύχησεν ο Ζιμβρί, ο φονεύσας τον κύριον αυτού;
\par 32 Ο δε, υψώσας το πρόσωπον αυτού προς το παράθυρον, είπε, Τις είναι μετ' εμού; τις; Και έκυψαν προς αυτόν δύο τρεις ευνούχοι.
\par 33 Και είπε, Ρίψατε αυτήν κάτω. Και έρριψαν αυτήν κάτω, και ερραντίσθη εκ του αίματος αυτής προς τον τοίχον και προς τους ίππους· και κατεπάτησεν αυτήν.
\par 34 Και αφού εισήλθε και έφαγε και έπιεν, είπεν, Υπάγετε να ίδητε τώρα την κατηραμένην ταύτην, και θάψατε αυτήν· διότι είναι θυγάτηρ βασιλέως.
\par 35 Και υπήγαν διά να θάψωσιν αυτήν· πλην δεν εύρηκαν εις αυτήν παρά το κρανίον και τους πόδας και τας παλάμας των χειρών.
\par 36 Και επιστρέψαντες απήγγειλαν προς αυτόν. Ο δε είπεν, Ούτος είναι ο λόγος του Κυρίου, τον οποίον ελάλησε διά του δούλου αυτού Ηλία του Θεσβίτου, λέγων, Εν τη μερίδι της Ιεζραέλ θέλουσι καταφάγει οι κύνες τας σάρκας της Ιεζάβελ·
\par 37 και το πτώμα της Ιεζάβελ θέλει είσθαι ως κοπρία επί προσώπου του αγρού εν τη μερίδι Ιεζραέλ, ώστε να μη είπωσιν, Αύτη είναι η Ιεζάβελ.

\chapter{10}

\par 1 Είχε δε ο Αχαάβ εβδομήκοντα υιούς εν Σαμαρεία. Και έγραψεν ο Ιηού επιστολάς, και απέστειλεν εις την Σαμάρειαν προς τους άρχοντας της Ιεζραέλ, προς τους πρεσβυτέρους και προς τους παιδοτρόφους του Αχαάβ, λέγων,
\par 2 Τώρα, καθώς φθάση προς εσάς η επιστολή αύτη, επειδή έχετε τους υιούς του κυρίου σας και έχετε τας αμάξας και τους ίππους και πόλιν οχυράν και όπλα,
\par 3 ιδέτε ποίος είναι ο καλήτερος και αρεστότερος μεταξύ των υιών του κυρίου σας, και καταστήσατε αυτόν επί του θρόνου του πατρός αυτού και πολεμείτε υπέρ του οίκου του κυρίου σας.
\par 4 Εκείνοι όμως εφοβήθησαν σφόδρα και είπον, Ιδού, δύο βασιλείς δεν εστάθησαν κατά πρόσωπον αυτού· και πως ημείς θέλομεν σταθή;
\par 5 Και απέστειλαν προς τον Ιηού ο επιστάτης του οίκου και ο επιστάτης της πόλεως και οι πρεσβύτεροι και οι παιδοτρόφοι, λέγοντες, Ημείς είμεθα δούλοί σου και θέλομεν κάμει παν ό,τι μας είπης· δεν θέλομεν κάμει ουδένα βασιλέα· κάμε ό,τι είναι αρεστόν εις τους οφθαλμούς σου.
\par 6 Τότε έγραψε προς αυτούς επιστολήν δευτέραν, λέγων, Εάν ήσθε εμού και εισακούητε της φωνής μου, λάβετε τας κεφαλάς των ανθρώπων, των υιών του κυρίου σας, και έλθετε προς εμέ εις Ιεζραέλ αύριον την ώραν ταύτην· οι δε υιοί του βασιλέως, εβδομήκοντα άνθρωποι, ήσαν μετά των μεγάλων της πόλεως, οίτινες ανέτρεφον αυτούς.
\par 7 Και καθώς έφθασεν η επιστολή προς αυτούς, λαβόντες τους υιούς του βασιλέως, έσφαξαν εβδομήκοντα ανθρώπους και έβαλον τας κεφαλάς αυτών εις καλάθια και έστειλαν προς αυτόν εις Ιεζραέλ.
\par 8 Και ήλθεν ο μηνυτής και ανήγγειλε προς αυτόν, λέγων, Έφεραν τας κεφαλάς των υιών του βασιλέως. Και είπε, Βάλετε αυτάς κατά δύο σωρούς, εν τη εισόδω της πύλης, έως πρωΐ.
\par 9 Και το πρωΐ εξήλθε και σταθείς είπε προς πάντα τον λαόν, Σεις είσθε δίκαιοι· ιδού, εγώ συνώμοσα εναντίον του κυρίου μου και εθανάτωσα αυτόν· αλλά πάντας τούτους τις επάταξε;
\par 10 γνωρίσατε τώρα, ότι δεν θέλει πέσει εις την γην ουδέν εκ του λόγου του Κυρίου, τον οποίον ελάλησεν ο Κύριος κατά του οίκου του Αχαάβ· διότι εξετέλεσεν ο Κύριος όσα ελάλησε διά του δούλου αυτού Ηλία.
\par 11 Και επάταξεν ο Ιηού πάντας τους εναπολειφθέντας εκ του οίκου του Αχαάβ εν Ιεζραέλ, και πάντας τους μεγάλους αυτού και τους οικείους αυτού και τους ιερείς αυτού, ώστε δεν αφήκεν εις αυτόν υπόλοιπον.
\par 12 Έπειτα σηκωθείς ανεχώρησε και ήλθεν εις Σαμάρειαν. Και εν τη οδώ, ενώ ήτο πλησίον τινός μάνδρας ποιμένων,
\par 13 εύρηκεν ο Ιηού τους αδελφούς του Οχοζίου βασιλέως του Ιούδα και είπε, Τίνες είσθε; Οι δε είπον, Είμεθα οι αδελφοί του Οχοζίου, και καταβαίνομεν να χαιρετήσωμεν τους υιούς του βασιλέως και τους υιούς της βασιλίσσης.
\par 14 Και είπε, Συλλάβετε αυτούς ζώντας. Και συνέλαβον αυτούς ζώντας και έσφαξαν αυτούς πλησίον του φρέατος της μάνδρας, τεσσαράκοντα δύο ανθρώπους· δεν αφήκαν ουδέ ένα εξ αυτών.
\par 15 Και αναχωρήσας εκείθεν, εύρηκε τον Ιωναδάβ υιόν του Ρηχάβ, ερχόμενον εις συνάντησιν αυτού· και εχαιρέτησεν αυτόν και είπε προς αυτόν, Η καρδία σου είναι ευθεία, καθώς η καρδία μου μετά της καρδίας σου; Και απεκρίθη ο Ιωναδάβ, Είναι. Εάν ναι, δος την χείρα σου. Και έδωκε την χείρα αυτού· και ανεβίβασεν αυτόν προς εαυτόν επί την άμαξαν.
\par 16 Και είπεν, Ελθέ μετ' εμού και ιδέ τον ζήλον μου υπέρ του Κυρίου. Και επεβίβασαν αυτόν εις την άμαξαν αυτού.
\par 17 Και ότε ήλθεν εις Σαμάρειαν, επάταξε πάντας τους εναπολειφθέντας εκ του Αχαάβ εν Σαμαρεία, εωσού ηφάνισεν αυτόν, κατά τον λόγον του Κυρίου, τον οποίον ελάλησε προς τον Ηλίαν.
\par 18 Τότε συνήθροισεν ο Ιηού πάντα τον λαόν και είπε προς αυτούς, Ο Αχαάβ εδούλευσε τον Βάαλ ολίγον· ο Ιηού θέλει δουλεύσει αυτόν πολύ·
\par 19 τώρα λοιπόν καλέσατε προς εμέ πάντας τους προφήτας του Βάαλ, πάντας τους λατρευτάς αυτού και πάντας τους ιερείς αυτού· ας μη λείψη μηδείς· διότι έχω θυσίαν μεγάλην εις τον Βάαλ· πας όστις λείψη, δεν θέλει ζήσει. Πλην ο Ιηού έπραξε τούτο δολίως, επί σκοπώ να εξολοθρεύση τους λατρευτάς του Βάαλ.
\par 20 Και είπεν ο Ιηού, Κηρύξατε πανήγυριν διά τον Βάαλ. Και εκήρυξαν.
\par 21 Και έπεμψεν ο Ιηού προς πάντα τον Ισραήλ· και ήλθον πάντες οι λατρευταί του Βάαλ· και δεν έμεινεν ουδείς, όστις δεν ήλθε. Και ήλθον εις τον οίκον του Βάαλ· και επλήσθη ο οίκος του Βάαλ, στόμα εις στόμα.
\par 22 Και είπε προς τον ιματιοφύλακα, Εξάγαγε ιμάτια διά πάντας τους λατρευτάς του Βάαλ. Και εξήγαγεν εις αυτούς τα ιμάτια.
\par 23 Και εισήλθεν ο Ιηού και ο Ιωναδάβ ο υιός του Ρηχάβ εις τον οίκον του Βάαλ· και είπε προς τους λατρευτάς του Βάαλ, Ερευνήσατε και ιδέτε να μη ήναι εδώ με σας μηδείς εκ των δούλων του Κυρίου, αλλά μόνον οι λατρευταί του Βάαλ.
\par 24 Και ότε εισήλθον διά να προσφέρωσι θυσίας και ολοκαυτώματα, ο Ιηού διέταξεν έξω ογδοήκοντα άνδρας και είπεν, Όστις αφήση να διασωθή τις εκ των ανθρώπων, τους οποίους εγώ έφερα εις τας χείρας σας, η ζωή αυτού θέλει είσθαι αντί της ζωής εκείνου.
\par 25 Και ως ετελείωσε προσφέρων το ολοκαύτωμα, είπεν ο Ιηού προς τους δορυφόρους και προς τους ταγματάρχας, Εισέλθετε, πατάξατε αυτούς· μηδείς ας μη εξέλθη. Και επάταξαν αυτούς οι δορυφόροι και οι ταγματάρχαι εν στόματι μαχαίρας και έρριψαν έξω· και υπήγαν έως της πόλεως του οίκου του Βάαλ.
\par 26 Και εξέβαλον τα είδωλα του οίκου του Βάαλ και κατέκαυσαν αυτά.
\par 27 Και κατεσύντριψαν το είδωλον του Βάαλ και κατεκρήμνισαν τον οίκον του Βάαλ, και έκαμον αυτόν κοπρώνα έως της ημέρας ταύτης.
\par 28 Ούτως ηφάνισεν ο Ιηού τον Βάαλ εκ του Ισραήλ.
\par 29 Πλην δεν απεμακρύνθη ο Ιηού από των αμαρτιών του Ιεροβοάμ υιού του Ναβάτ, όστις έκαμε τον Ισραήλ να αμαρτήση, από των χρυσών μόσχων των εν Βαιθήλ και των εν Δαν.
\par 30 Και είπε Κύριος προς τον Ιηού, Επειδή έπραξας καλώς εκτελέσας το αρεστόν εις τους οφθαλμούς μου, και έκαμες εις τον οίκον του Αχαάβ κατά πάντα όσα ήσαν εν τη καρδία μου, οι υιοί σου μέχρι της τετάρτης γενεάς θέλουσι καθίσει επί του θρόνου του Ισραήλ.
\par 31 Και δεν επρόσεξεν ο Ιηού να περιπατή εξ όλης της καρδίας αυτού εν τω νόμω Κυρίου του Θεού του Ισραήλ· δεν απεμακρύνθη από των αμαρτιών του Ιεροβοάμ, όστις έκαμε τον Ισραήλ να αμαρτήση.
\par 32 Εν εκείναις ταις ημέραις ήρχισεν ο Κύριος να κολοβόνη τον Ισραήλ· και επάταξεν αυτούς ο Αζαήλ εις πάντα τα όρια του Ισραήλ·
\par 33 από Ιορδάνου, προς ανατολάς ηλίου, πάσαν την γην Γαλαάδ, τους Γαδίτας και τους Ρουβηνίτας και τους Μανασσίτας από Αροήρ, της επί του χειμάρρου Αρνών, την τε Γαλαάδ και την Βασάν.
\par 34 Αι δε λοιπαί πράξεις του Ιηού και πάντα όσα έπραξε και πάντα τα κατορθώματα αυτού, δεν είναι γεγραμμένα εν τω βιβλίω των χρονικών των βασιλέων του Ισραήλ;
\par 35 Και εκοιμήθη ο Ιηού μετά των πατέρων αυτού· και έθαψαν αυτόν εν Σαμαρεία. Εβασίλευσε δε αντ' αυτού Ιωάχαζ ο υιός αυτού.
\par 36 Και ο καιρός, καθ' ον ο Ιηού εβασίλευσεν επί τον Ισραήλ εν Σαμαρεία, ήτο εικοσιοκτώ έτη.

\chapter{11}

\par 1 Γοθολία δε, η μήτηρ του Οχοζίου, ιδούσα ότι απέθανεν ο υιός αυτής, εσηκώθη και ηφάνισε παν το βασιλικόν σπέρμα.
\par 2 Ιωσαβεέ όμως, η θυγάτηρ του βασιλέως Ιωράμ, αδελφή του Οχοζίου, λαβούσα τον Ιωάς υιόν του Οχοζίου, έκλεψεν αυτόν εκ μέσου των υιών του βασιλέως των θανατουμένων, αυτόν και την τροφόν αυτού, και έβαλεν εν τω ταμείω του κοιτώνος, και έκρυψαν αυτόν από προσώπου της Γοθολίας, και δεν εθανατώθη.
\par 3 Και ήτο μετ' αυτής εν τω οίκω του Κυρίου κρυπτόμενος εξ έτη. Η δε Γοθολία εβασίλευεν επί της γης.
\par 4 Εν δε τω εβδόμω έτει ο Ιωδαέ απέστειλε και λαβών τους εκατοντάρχους μετά των ταξιάρχων και των δορυφόρων, έφερεν αυτούς προς εαυτόν εις τον οίκον του Κυρίου, και έκαμε συνθήκην μετ' αυτών και ώρκισεν αυτούς εν τω οίκω του Κυρίου· και έδειξεν εις αυτούς τον υιόν του βασιλέως.
\par 5 Και προσέταξεν εις αυτούς, λέγων, Τούτο είναι το πράγμα το οποίον θέλετε κάμει το τρίτον από σας, οι εισερχόμενοι το σάββατον, θέλετε φυλάττει την φυλακήν του βασιλικού οίκου·
\par 6 και το τρίτον θέλει είσθαι εν τη πύλη Σούρ· και το τρίτον εν τη πύλη τη όπισθεν των δορυφόρων· ούτω θέλετε φυλάττει την φυλακήν του οίκου, διά να μη παραβιασθή·
\par 7 και δύο τάγματα από σας, πάντες οι εξερχόμενοι το σάββατον, θέλουσι φυλάττει την φυλακήν του οίκου του Κυρίου περί τον βασιλέα.
\par 8 και θέλετε περικυκλόνει τον βασιλέα κύκλω, έκαστος έχων τα όπλα αυτού εν τη χειρί αυτού· και όστις εισέλθη εις τας τάξεις, ας θανατόνεται· και θέλετε είσθαι μετά του βασιλέως, όταν εξέρχηται και όταν εισέρχηται.
\par 9 Και έκαμον οι εκατόνταρχοι κατά πάντα όσα προσέταξεν Ιωδαέ ο ιερεύς· και έλαβον έκαστος τους άνδρας αυτού, τους εισερχομένους το σάββατον, μετά των εξερχομένων το σάββατον, και ήλθον προς Ιωδαέ τον ιερέα.
\par 10 Και έδωκεν ο ιερεύς εις τους εκατοντάρχους τας λόγχας και τας ασπίδας του βασιλέως Δαβίδ, τας εν τω οίκω Κυρίου.
\par 11 Και οι δορυφόροι, έχοντες έκαστος τα όπλα αυτού εν τη χειρί αυτού, παρεστάθησαν πέριξ του βασιλέως, από της δεξιάς πλευράς του οίκου έως της αριστεράς, πλησίον του θυσιαστηρίου και του ναού.
\par 12 Τότε εξήγαγε τον υιόν του βασιλέως και επέθεσεν επ' αυτόν το διάδημα και το μαρτύριον· και έκαμον αυτόν βασιλέα και έχρισαν αυτόν· και κροτήσαντες τας χείρας, είπον, Ζήτω ο βασιλεύς
\par 13 Και ακούσασα η Γοθολία την φωνήν του λαού συντρέχοντος, ήλθε προς τον λαόν εις τον οίκον του Κυρίου.
\par 14 Και είδε, και ιδού, ο βασιλεύς ίστατο πλησίον του στύλου κατά το έθος, και οι άρχοντες και οι σαλπιγκταί πλησίον του βασιλέως· και πας ο λαός της γης έχαιρε και εσάλπιζε με σάλπιγγας. Και διέρρηξεν η Γοθολία τα ιμάτια αυτής και εβόησε, Προδοσία, προδοσία
\par 15 Και προσέταξεν Ιωδαέ ο ιερεύς τους εκατοντάρχους, τους αρχηγούς του στρατεύματος, και είπε προς αυτούς, Εκβάλετε αυτήν έξω των τάξεων· και όστις ακολουθήση αυτήν, θανατώσατε αυτόν εν ρομφαία. Διότι ο ιερεύς είχεν ειπεί, Ας μη θανατωθή εντός του οίκου του Κυρίου.
\par 16 Ούτως έβαλον χείρας επ' αυτήν· και ότε ήλθεν εις την οδόν, διά της οποίας οι ίπποι έρχονται εις τον οίκον του βασιλέως, εθανατώθη εκεί.
\par 17 Και έκαμεν ο Ιωδαέ διαθήκην αναμέσον του Κυρίου και του βασιλέως και του λαού, ότι θέλουσιν είσθαι λαός του Κυρίου· και αναμέσον του βασιλέως και του λαού.
\par 18 Και εισήλθον πας ο λαός της γης εις τον οίκον του Βάαλ και εκρήμνισαν αυτόν· τα θυσιαστήρια αυτού και τα είδωλα αυτού κατεσύντριψαν ολοτελώς και Ματθάν τον ιερέα του Βάαλ εθανάτωσαν έμπροσθεν των θυσιαστηρίων. Και ο ιερεύς κατέστησεν επιτηρητάς επί τον οίκον του Κυρίου.
\par 19 Και έλαβε τους εκατοντάρχους και τους ταξιάρχους και τους δορυφόρους και πάντα τον λαόν της γής· και κατεβίβασαν τον βασιλέα εκ του οίκου του Κυρίου, και ήλθον εις τον οίκον του βασιλέως διά της οδού της πύλης των δορυφόρων. Και εκάθισεν επί του θρόνου των βασιλέων.
\par 20 Και ευφράνθη πας ο λαός της γης και η πόλις ησύχασε· την δε Γοθολίαν εθανάτωσαν εν μαχαίρα εν τω οίκω του βασιλέως.
\par 21 Επτά ετών ήτο ο Ιωάς ότε εβασίλευσε.

\chapter{12}

\par 1 Εν τω εβδόμω έτει του Ιηού εβασίλευσεν ο Ιωάς· και εβασίλευσε τεσσαράκοντα έτη εν Ιερουσαλήμ· το δε όνομα της μητρός αυτού ήτο Σιβιά εκ Βηρ-σαβεέ.
\par 2 Και έπραττεν ο Ιωάς το ευθές ενώπιον του Κυρίου, κατά πάσας τας ημέρας αυτού καθ' ας ώδήγει αυτόν Ιωδαέ ο ιερεύς.
\par 3 Οι υψηλοί όμως τόποι δεν αφηρέθησαν· ο λαός εθυσίαζεν έτι και εθυμίαζεν εν τοις υψηλοίς τόποις.
\par 4 Και είπεν ο Ιωάς προς τους ιερείς, Παν το αργύριον των αφιερωμάτων το εισφερόμενον εις τον οίκον του Κυρίου, το αργύριον εκάστου διερχομένου εις τους απαριθμουμένους, το αργύριον εκάστου κατά την εκτίμησιν αυτού, παν το αργύριον το οποίον ήθελεν ελθεί εις την καρδίαν τινός να προσφέρη εις τον οίκον του Κυρίου,
\par 5 οι ιερείς ας λαμβάνωσιν αυτό εις εαυτούς, έκαστος παρά του γνωστού αυτού· και ας επισκευάζωσι τα χαλάσματα του οίκου, πανταχού όπου ευρεθή χάλασμα.
\par 6 Πλην εν τω εικοστώ τρίτω έτει του βασιλέως Ιωάς οι ιερείς δεν είχον επισκευάσει τα χαλάσματα του οίκου.
\par 7 Όθεν εκάλεσεν ο βασιλεύς Ιωάς τον Ιωδαέ τον ιερέα και τους ιερείς και είπε προς αυτούς, Διά τι δεν επεσκευάσατε τα χαλάσματα του οίκου; τώρα λοιπόν μη λαμβάνετε πλέον αργύριον παρά των γνωστών σας, αλλά δίδετε αυτό διά τα χαλάσματα του οίκου.
\par 8 Και έστερξαν οι ιερείς να μη λαμβάνωσι πλέον αργύριον παρά του λαού και να μη επισκευάζωσι τα χαλάσματα του οίκου.
\par 9 Και έλαβεν Ιωδαέ ο ιερεύς εν κιβώτιον και ήνοιξε τρύπαν επί του σκεπάσματος αυτού, και έθεσεν αυτό πλησίον του θυσιαστηρίου, εις τα δεξιά της εισόδου του οίκου του Κυρίου· και οι ιερείς, οι φυλάττοντες την θύραν, έβαλλον εις αυτό παν το αργύριον, το εισφερόμενον εις τον οίκον του Κυρίου.
\par 10 Και ότε έβλεπον ότι ήτο πολύ το αργύριον το εν τω κιβωτίω, ο γραμματεύς του βασιλέως και ο ιερεύς ο μέγας ανέβαινον και έδενον εις σακκία και εμέτρουν το αργύριον το ευρεθέν εν τω οίκω του Κυρίου.
\par 11 Και έδιδον το αργύριον το μετρηθέν εις τας χείρας εκείνων οίτινες έκαμνον το έργον, οίτινες είχον την επιστασίαν του οίκου του Κυρίου· οι δε εξώδευον αυτό εις τους ξυλουργούς και οικοδόμους, τους δουλεύοντας εν τω οίκω του Κυρίου,
\par 12 και εις τους κτίστας και εις τους λιθοτόμους, διά να αγοράζωσι ξύλα και λίθους λατομητούς, ώστε να επισκευάζωσι τα χαλάσματα του οίκου του Κυρίου, και διά πάντα όσα εχρειάζοντο διά την επισκευήν του οίκου.
\par 13 Πλην εκ του αργυρίου του εισφερομένου εις τον οίκον του Κυρίου δεν κατεσκευάσθησαν διά τον οίκον του Κυρίου φιάλαι αργυραί, λυχνοψάλιδα, λεκάναι, σάλπιγγες, ουδέν σκεύος χρυσούν ή σκεύος αργυρούν·
\par 14 αλλ' έδιδον αυτό εις τους εργάτας, και επεσκεύαζον με αυτό τον οίκον του Κυρίου.
\par 15 Και δεν εζήτουν λογαριασμόν παρά των ανθρώπων, εις τους οποίους έδιδον το αργύριον διά να μοιρασθή εις τους εργάτας· διότι ειργάζοντο εν πίστει.
\par 16 Το αργύριον το περί ανομίας και το αργύριον το περί αμαρτίας δεν εφέροντο εις τον οίκον του Κυρίου· ταύτα ήσαν των ιερέων.
\par 17 Τότε ανέβη Αζαήλ ο βασιλεύς της Συρίας και επολέμησεν εναντίον της Γαθ, και εκυρίευσεν αυτήν· έπειτα έστησεν ο Αζαήλ το πρόσωπον αυτού διά να αναβή εναντίον της Ιερουσαλήμ.
\par 18 Και έλαβεν ο Ιωάς βασιλεύς του Ιούδα πάντα τα αφιερώματα όσα Ιωσαφάτ και Ιωράμ και Οχοζίας, οι πατέρες αυτού, βασιλείς του Ιούδα, είχον αφιερώσει, και τα ίδια αυτού αφιερώματα και παν το χρυσίον το ευρεθέν εν τοις θησαυροίς του οίκου του Κυρίου και του οίκου του βασιλέως, και έστειλεν αυτά προς τον Αζαήλ βασιλέα της Συρίας· και ανεχώρησεν από της Ιερουσαλήμ.
\par 19 Αι δε λοιπαί πράξεις του Ιωάς και πάντα όσα έπραξε, δεν είναι γεγραμμένα εν τω βιβλίω των χρονικών των βασιλέων του Ιούδα;
\par 20 Και σηκωθέντες οι δούλοι αυτού, έκαμον συνωμοσίαν και επάταξαν τον Ιωάς εν τω οίκω Μιλλώ, εν τη καταβάσει Σιλλά.
\par 21 Διότι Ιωζαχάρ ο υιός του Σιμεάθ και Ιωζαβάδ ο υιός του Σωμήρ, οι δούλοι αυτού, επάταξαν αυτόν, και απέθανε· και έθαψαν αυτόν μετά των πατέρων αυτού εν τη πόλει Δαβίδ· εβασίλευσε δε αντ' αυτού Αμασίας ο υιός αυτού.

\chapter{13}

\par 1 Εν τω εικοστώ τρίτω έτει του Ιωάς, υιού του Οχοζίου, βασιλέως του Ιούδα, εβασίλευσεν Ιωάχαζ, ο υιός του Ιηού, επί Ισραήλ εν Σαμαρεία, δεκαεπτά έτη.
\par 2 Και έπραξε πονηρά ενώπιον του Κυρίου και ηκολούθησε τας αμαρτίας του Ιεροβοάμ υιού του Ναβάτ, όστις έκαμε τον Ισραήλ να αμαρτήση· δεν απεμακρύνθη απ' αυτών.
\par 3 Και εξήφθη η οργή του Κυρίου κατά του Ισραήλ, και παρέδωκεν αυτούς εις την χείρα του Αζαήλ βασιλέως της Συρίας και εις την χείρα του Βεν-αδάδ υιού του Αζαήλ, κατά πάσας τας ημέρας.
\par 4 Και εδεήθη του Κυρίου ο Ιωάχαζ, και επήκουσεν αυτού ο Κύριος· διότι είδε την θλίψιν του Ισραήλ, ότι ο βασιλεύς της Συρίας κατέθλιβεν αυτούς.
\par 5 Και έδωκεν ο Κύριος εις τον Ισραήλ σωτήρα, και εξήλθον υποκάτωθεν της χειρός των Συρίων· και κατώκησαν οι υιοί Ισραήλ εν τοις σκηνώμασιν αυτών, ως το πρότερον.
\par 6 Πλην δεν απεμακρύνθησαν από των αμαρτιών του οίκου του Ιεροβοάμ, όστις έκαμε τον Ισραήλ να αμαρτήση· εις αυτάς περιεπάτησαν· και έτι διέμενε το άλσος εν Σαμαρεία.
\par 7 Διότι δεν έμεινεν εις τον Ιωάχαζ λαός, ειμή πεντήκοντα ιππείς και δέκα άμαξαι και δέκα χιλιάδες πεζών· διότι κατέστρεψεν αυτούς ο βασιλεύς της Συρίας και κατέστησεν αυτούς ως το χώμα το καταπατούμενον.
\par 8 Αι δε λοιπαί πράξεις του Ιωάχαζ και πάντα όσα έπραξε και τα κατορθώματα αυτού, δεν είναι γεγραμμένα εν τω βιβλίω των χρονικών των βασιλέων του Ισραήλ;
\par 9 Και εκοιμήθη ο Ιωάχαζ μετά των πατέρων αυτού, και έθαψαν αυτόν εν Σαμαρεία· εβασίλευσε δε αντ' αυτού Ιωάς ο υιός αυτού.
\par 10 Εν τω τριακοστώ εβδόμω έτει του Ιωάς βασιλέως του Ιούδα, εβασίλευσεν Ιωάς ο υιός του Ιωάχαζ επί Ισραήλ εν Σαμαρεία, δεκαέξ έτη.
\par 11 Και έπραξε πονηρά ενώπιον του Κυρίου· δεν απεμακρύνθη από πασών των αμαρτιών του Ιεροβοάμ υιού του Ναβάτ, όστις έκαμε τον Ισραήλ να αμαρτήση· εις αυτάς περιεπάτησεν.
\par 12 Αι δε λοιπαί πράξεις του Ιωάς και πάντα όσα έπραξε, τα κατορθώματα αυτού, πως επολέμησε κατά του Αμασίου βασιλέως του Ιούδα, δεν είναι γεγραμμένα εν τω βιβλίω των χρονικών των βασιλέων του Ισραήλ;
\par 13 Και εκοιμήθη ο Ιωάς μετά των πατέρων αυτού· εκάθησε δε επί του θρόνου αυτού ο Ιεροβοάμ· και ετάφη ο Ιωάς εν Σαμαρεία μετά των βασιλέων του Ισραήλ.
\par 14 Ο δε Ελισσαιέ ηρρώστησε την αρρωστίαν αυτού υπό της οποίας απέθανε. Και κατέβη προς αυτόν Ιωάς ο βασιλεύς του Ισραήλ και έκλαυσεν επί τω προσώπω αυτού και είπε, Πάτερ μου, πάτερ μου, άμαξα του Ισραήλ και ιππικόν αυτού.
\par 15 Και είπε προς αυτόν ο Ελισσαιέ, Λάβε τόξον και βέλη. Και έλαβεν εις εαυτόν τόξον και βέλη.
\par 16 Και είπε προς τον βασιλέα του Ισραήλ, Επίθες την χείρα σου επί το τόξον. Και επέθηκε την χείρα αυτού· και επέθηκεν ο Ελισσαιέ τας χείρας αυτού επί τας χείρας του βασιλέως.
\par 17 Και είπεν, Άνοιξον το παράθυρον κατά ανατολάς. Και ήνοιξε. Και είπεν ο Ελισσαιέ, Τόξευσον. Και ετόξευσε. Και είπε, το βέλος της σωτηρίας του Κυρίου και το βέλος της σωτηρίας εκ των Συρίων. Και θέλεις πατάξει τους Συρίους εν Αφέκ, εωσού συντελέσης αυτούς.
\par 18 Και είπε, Λάβε τα βέλη. Και έλαβε. Και είπε προς τον βασιλέα του Ισραήλ, Πάταξον επί την γην. Και επάταξε τρίς και εστάθη.
\par 19 Και ωργίσθη εις αυτόν ο άνθρωπος του Θεού και είπεν, Έπρεπε να πατάξης πεντάκις ή εξάκις· τότε ήθελες πατάξει τους Συρίους εωσού συντελέσης αυτούς· τώρα όμως τρίς θέλεις πατάξει τους Συρίους.
\par 20 Και απέθανεν ο Ελισσαιέ, και έθαψαν αυτόν· το δε ακόλουθον έτος τάγματα Μωαβιτών έκαμον εισβολήν εις την γην.
\par 21 Και ενώ έθαπτον άνθρωπον τινά, ιδού, είδον τάγμα· και έρριψαν τον άνθρωπον εις τον τάφον του Ελισσαιέ· και καθώς ο άνθρωπος υπήγε και ήγγισε τα οστά του Ελισσαιέ, ανέζησε και εστάθη επί τους πόδας αυτού.
\par 22 Ο δε Αζαήλ ο βασιλεύς της Συρίας, κατέθλιψε τον Ισραήλ πάσας τας ημέρας του Ιωάχαζ.
\par 23 Και ηλέησεν ο Κύριος αυτούς και ωκτείρησεν αυτούς και επέβλεψεν επ' αυτούς, διά την διαθήκην αυτού την μετά του Αβραάμ, Ισαάκ, και Ιακώβ· και δεν ηθέλησε να εξολοθρεύση αυτούς και δεν απέρριψεν αυτούς από προσώπου αυτού, μέχρι του νυν.
\par 24 Απέθανε δε ο Αζαήλ βασιλεύς της Συρίας, και εβασίλευσεν αντ' αυτού Βεν-αδάδ ο υιός αυτού.
\par 25 Και έλαβε πάλιν Ιωάς ο υιός του Ιωάχαζ εκ της χειρός του Βεν-αδάδ υιού του Αζαήλ τας πόλεις, τας οποίας ο Αζαήλ είχε λάβει εκ της χειρός Ιωάχαζ του πατρός αυτού εν τω πολέμω. Τρίς επάταξεν αυτόν ο Ιωάς και επανέλαβε τας πόλεις του Ισραήλ.

\chapter{14}

\par 1 Εν τω δευτέρω έτει του Ιωάς, υιού του Ιωάχαζ βασιλέως του Ισραήλ, εβασίλευσεν Αμασίας, ο υιός του Ιωάς βασιλέως του Ιούδα.
\par 2 Εικοσιπέντε ετών ηλικίας ήτο ότε εβασίλευσε, και εβασίλευσεν εικοσιεννέα έτη εν Ιερουσαλήμ. Το δε όνομα της μητρός αυτού ήτο Ιωαδάν εξ Ιερουσαλήμ.
\par 3 Και έπραξε το ευθές ενώπιον Κυρίου, πλην ουχί ως ο Δαβίδ ο πατήρ αυτού· έπραξε κατά πάντα όσα είχε πράξει Ιωάς ο πατήρ αυτού.
\par 4 Οι υψηλοί όμως τόποι δεν αφηρέθησαν· ο λαός εθυσίαζεν έτι και εθυμίαζεν επί τους υψηλούς τόπους.
\par 5 Ως δε η βασιλεία εκραταιώθη εν τη χειρί αυτού, εθανάτωσε τους δούλους αυτού τους θανατώσαντας τον βασιλέα τον πατέρα αυτού.
\par 6 Όμως τα τέκνα των φονευτών δεν εθανάτωσε· κατά το γεγραμμένον εν τω βιβλίω του νόμου του Μωϋσέως, όπου προσέταξεν ο Κύριος, λέγων, Οι πατέρες δεν θέλουσι θανατόνεσθαι διά τα τέκνα, ουδέ τα τέκνα θέλουσι θανατόνεσθαι διά τους πατέρας, αλλ' έκαστος θέλει θανατόνεσθαι διά το εαυτού αμάρτημα.
\par 7 Ούτος εθανάτωσεν εκ του Εδώμ δέκα χιλιάδας εν τη κοιλάδι του άλατος, και εκυρίευσε την Σελά διά πολέμου και εκάλεσε το όνομα αυτής Ιοχθεήλ μέχρι της ημέρας ταύτης.
\par 8 Τότε απέστειλεν ο Αμασίας μηνυτάς προς τον Ιωάς, υιόν του Ιωάχαζ, υιού του Ιηού βασιλέως του Ισραήλ, λέγων, Ελθέ, να ίδωμεν αλλήλους προσωπικώς.
\par 9 Και απέστειλεν ο Ιωάς βασιλεύς του Ισραήλ προς τον Αμασίαν βασιλέα του Ιούδα, λέγων, Η άκανθα η εν τω Λιβάνω απέστειλε προς την κέδρον την εν τω Λιβάνω, λέγουσα, Δος την θυγατέρα σου εις τον υιόν μου διά γυναίκα· πλην διέβη θηρίον του αγρού το εν τω Λιβάνω, και κατεπάτησε την άκανθαν·
\par 10 επάταξας τωόντι τον Εδώμ, και η καρδία σου σε ύψωσε· χαίρου την δόξαν σου καθήμενος εν τω οίκω σου· διά τι εμπλέκεσαι εις κακόν, διά το οποίον ήθελες πέσει, συ και ο Ιούδας μετά σου;
\par 11 Αλλ' ο Αμασίας δεν υπήκουσεν. Ανέβη λοιπόν ο Ιωάς βασιλεύς του Ισραήλ, και είδον αλλήλους προσωπικώς, αυτός και Αμασίας ο βασιλεύς του Ιούδα, εν Βαιθ-σεμές, ήτις είναι του Ιούδα.
\par 12 Και εκτυπήθη ο Ιούδας έμπροσθεν του Ισραήλ· και έφυγον έκαστος εις τας σκηνάς αυτού.
\par 13 Και συνέλαβεν ο Ιωάς ο βασιλεύς του Ισραήλ τον Αμασίαν βασιλέα του Ιούδα, υιόν του Ιωάς υιού του Οχοζίου, εν Βαιθ-σεμές· και ελθών εις Ιερουσαλήμ, κατηδάφισε το τείχος της Ιερουσαλήμ από της πύλης Εφραΐμ έως της πύλης της γωνίας, τετρακοσίας πήχας.
\par 14 Και λαβών παν το χρυσίον και το αργύριον και πάντα τα σκεύη τα ευρεθέντα εν τω οίκω του Κυρίου και εν τοις θησαυροίς του οίκου του βασιλέως, και ανθρώπους ενέχυρα, επέστρεψεν εις Σαμάρειαν.
\par 15 Αι δε λοιπαί πράξεις του Ιωάς, όσας έπραξε, και τα κατορθώματα αυτού, και πως επολέμησε μετά του Αμασίου βασιλέως του Ιούδα, δεν είναι γεγραμμένα εν τω βιβλίω των χρονικών των βασιλέων του Ισραήλ;
\par 16 Και εκοιμήθη ο Ιωάς μετά των πατέρων αυτού και ετάφη εν Σαμαρεία μετά των βασιλέων του Ισραήλ· εβασίλευσε δε αντ' αυτού Ιεροβοάμ ο υιός αυτού.
\par 17 Ο δε Αμασίας, ο υιός του Ιωάς, ο βασιλεύς του Ιούδα, έζησε μετά τον θάνατον του Ιωάς υιού του Ιωάχαζ, βασιλέως του Ισραήλ, δεκαπέντε έτη.
\par 18 Αι δε λοιπαί πράξεις του Αμασίου δεν είναι γεγραμμέναι εν τω βιβλίω των χρονικών των βασιλέων του Ιούδα;
\par 19 Έκαμον δε κατ' αυτού συνωμοσίαν εν Ιερουσαλήμ, και έφυγεν εις Λαχείς· απέστειλαν όμως κατόπιν αυτού εις Λαχείς και εθανάτωσαν αυτόν εκεί.
\par 20 Και έφεραν αυτόν επί ίππων, και ετάφη εν Ιερουσαλήμ μετά των πατέρων αυτού, εν τη πόλει Δαβίδ.
\par 21 Έλαβε δε πας ο λαός του Ιούδα τον Αζαρίαν, όντα ηλικίας δεκαέξ ετών, και έκαμον αυτόν βασιλέα αντί του πατρός αυτού Αμασίου.
\par 22 Και ωκοδόμησε την Ελάθ και επέστρεψεν αυτήν εις τον Ιούδα, αφού ο βασιλεύς εκοιμήθη μετά των πατέρων αυτού.
\par 23 Εν τω δεκάτω πέμπτω έτει του Αμασίου, υιού του Ιωάς, βασιλέως του Ιούδα, εβασίλευσεν εν Σαμαρεία ο Ιεροβοάμ υιός του Ιωάς, βασιλέως του Ισραήλ, έτη τεσσαράκοντα και εν.
\par 24 Και έπραξε πονηρά ενώπιον του Κυρίου· δεν απεμακρύνθη από πασών των αμαρτιών του Ιεροβοάμ υιού του Ναβάτ, όστις έκαμε τον Ισραήλ να αμαρτήση.
\par 25 Ούτος αποκατέστησε το όριον του Ισραήλ, από της εισόδου της Αιμάθ έως της θαλάσσης της πεδιάδος, κατά τον λόγον Κυρίου του Θεού του Ισραήλ, τον οποίον ελάλησε διά του δούλου αυτού Ιωνά, υιού του Αμαθί, του προφήτου, του από Γαθ-εφέρ.
\par 26 Διότι είδεν ο Κύριος την θλίψιν του Ισραήλ πικράν σφόδρα, ότι δεν ήτο ουδέν κεκλεισμένον και ουδέν αφειμένον, ουδέ ο βοηθήσων τον Ισραήλ.
\par 27 Και δεν είπεν ο Κύριος να εξαλείψη υποκάτωθεν του ουρανού το όνομα του Ισραήλ, αλλ' έσωσεν αυτούς διά χειρός του Ιεροβοάμ υιού του Ιωάς.
\par 28 Αι δε λοιπαί πράξεις του Ιεροβοάμ και πάντα όσα έπραξε και τα κατορθώματα αυτού, πως επολέμησε και πως επανέλαβε την Δαμασκόν και την Αιμάθ του Ιούδα εις τον Ισραήλ, δεν είναι γεγραμμένα εν τω βιβλίω των χρονικών των βασιλέων του Ισραήλ;
\par 29 Και εκοιμήθη ο Ιεροβοάμ μετά των πατέρων αυτού, μετά των βασιλέων του Ισραήλ· εβασίλευσε δε αντ' αυτού Ζαχαρίας ο υιός αυτού.

\chapter{15}

\par 1 Εν τω εικοστώ εβδόμω έτει του Ιεροβοάμ βασιλέως του Ισραήλ εβασίλευσεν ο Αζαρίας, υιός του Αμασίου, βασιλέως του Ιούδα.
\par 2 Δεκαέξ ετών ηλικίας ήτο ότε εβασίλευσε, και εβασίλευσε πεντήκοντα δύο έτη εν Ιερουσαλήμ· το δε όνομα της μητρός αυτού ήτο Ιεχολία, εξ Ιερουσαλήμ.
\par 3 Και έπραξε το ευθές ενώπιον του Κυρίου, κατά πάντα όσα είχε πράξει Αμασίας ο πατήρ αυτού.
\par 4 Πλην οι υψηλοί τόποι δεν αφηρέθησαν· ο λαός έτι εθυσίαζε και εθυμίαζεν επί τους υψηλούς τόπους.
\par 5 Και επάταξεν ο Κύριος τον βασιλέα, και ήτο λεπρός έως της ημέρας του θανάτου αυτού και κατώκει εν οικία αποκεχωρισμένη. Ήτο δε επί του οίκου Ιωθάμ ο υιός του βασιλέως, κρίνων τον λαόν της γης.
\par 6 Αι δε λοιπαί πράξεις του Αζαρίου και πάντα όσα έπραξε δεν είναι γεγραμμένα εν τω βιβλίω των χρονικών των βασιλέων του Ιούδα;
\par 7 Και εκοιμήθη ο Αζαρίας μετά των πατέρων αυτού· και έθαψαν αυτόν μετά των πατέρων αυτού εν πόλει Δαβίδ· εβασίλευσε δε αντ' αυτού Ιωθάμ ο υιός αυτού.
\par 8 Εν τω τριακοστώ ογδόω έτει του Αζαρίου βασιλέως του Ιούδα, Ζαχαρίας ο υιός του Ιεροβοάμ εβασίλευσεν επί τον Ισραήλ εν Σαμαρεία, εξ μήνας.
\par 9 Και έπραξε πονηρά ενώπιον του Κυρίου, ως είχον πράξει οι πατέρες αυτού· δεν απεμακρύνθη από των αμαρτιών του Ιεροβοάμ υιού του Ναβάτ, όστις έκαμε τον Ισραήλ να αμαρτήση.
\par 10 Και συνώμοσε κατ' αυτού Σαλλούμ ο υιός του Ιαβείς, και επάταξεν αυτόν κατέμπροσθεν του λαού και εθανάτωσεν αυτόν και εβασίλευσεν αντ' αυτού.
\par 11 Αι δε λοιπαί πράξεις του Ζαχαρίου, ιδού, είναι γεγραμμέναι εν τω βιβλίω των χρονικών των βασιλέων του Ισραήλ.
\par 12 Ούτος ήτο ο λόγος του Κυρίου, τον οποίον ελάλησε προς τον Ιηού, λέγων, Οι υιοί σου θέλουσι καθίσει επί του θρόνου του Ισραήλ έως τετάρτης γενεάς. Και έγεινεν ούτως.
\par 13 Εβασίλευσε δε Σαλλούμ ο υιός του Ιαβείς εν τω τριακοστώ εννάτω έτει του Οζίου βασιλέως του Ιούδα, και εβασίλευσεν ένα μήνα εν Σαμαρεία.
\par 14 Και ανέβη Μεναήμ ο υιός του Γαδεί από Θερσά, και ήλθεν εις Σαμάρειαν και εκτύπησε τον Σαλλούμ τον υιόν του Ιαβείς εν Σαμαρεία, και εθανάτωσεν αυτόν και εβασίλευσεν αντ' αυτού.
\par 15 Αι δε λοιπαί πράξεις του Σαλλούμ, και η συνωμοσία αυτού την οποίαν έκαμεν, ιδού, είναι γεγραμμέναι εν τω βιβλίω των χρονικών των βασιλέων του Ισραήλ.
\par 16 Τότε επάταξεν ο Μεναήμ την Θαψά και πάντας τους εν αυτή και τα όρια αυτής από Θερσά· επειδή δεν ήνοιξαν εις αυτόν, διά τούτο επάταξεν αυτήν· και πάσας τας εν αυτή εγκυμονούσας διέσχισεν.
\par 17 Εν τω τριακοστώ εννάτω έτει του Αζαρίου βασιλέως του Ιούδα, Μεναήμ ο υιός του Γαδεί εβασίλευσεν επί τον Ισραήλ, δέκα έτη εν Σαμαρεία.
\par 18 Και έπραξε πονηρά ενώπιον του Κυρίου· δεν απεμακρύνθη κατά πάσας τας ημέρας αυτού από των αμαρτιών του Ιεροβοάμ υιού του Ναβάτ, όστις έκαμε τον Ισραήλ να αμαρτήση.
\par 19 Τότε ήλθεν ο Φούλ βασιλεύς της Ασσυρίας εναντίον της γης. και έδωκεν ο Μεναήμ εις τον Φούλ χίλια τάλαντα αργυρίου, διά να ήναι μετ' αυτού η χειρ αυτού εις το να ενισχύση την βασιλείαν εν τη χειρί αυτού.
\par 20 Και απέσπασεν ο Μεναήμ το αργύριον από του Ισραήλ, από πάντων των δυνατών εις πλούτη, πεντήκοντα σίκλους αργυρίου αφ' εκάστου, διά να δώση εις τον βασιλέα της Ασσυρίας. Και επέστρεψεν ο βασιλεύς της Ασσυρίας και δεν εστάθη εκεί εν τη γη.
\par 21 Αι δε λοιπαί πράξεις του Μεναήμ και πάντα όσα έπραξε, δεν είναι γεγραμμένα εν τω βιβλίω των χρονικών των βασιλέων του Ισραήλ;
\par 22 Και εκοιμήθη ο Μεναήμ μετά των πατέρων αυτού· εβασίλευσε δε αντ' αυτού Φακείας ο υιός αυτού.
\par 23 Εν τω πεντηκοστώ έτει του Αζαρίου βασιλέως του Ιούδα, Φακείας ο υιός του Μεναήμ εβασίλευσεν επί τον Ισραήλ εν Σαμαρεία, δύο έτη.
\par 24 Και έπραξε πονηρά ενώπιον του Κυρίου· δεν απεμακρύνθη από των αμαρτιών του Ιεροβοάμ υιού του Ναβάτ, όστις έκαμε τον Ισραήλ να αμαρτήση.
\par 25 Και συνώμοσε κατ' αυτού ο Φεκά υιός του Ρεμαλία, ο στρατηγός αυτού, και επάταξεν αυτόν εν Σαμαρεία, εν τω παλατίω του οίκου του βασιλέως, μετά του Αργόβ και Αριέ, έχων μεθ' εαυτού και πεντήκοντα άνδρας εκ των Γαλααδιτών· και εθανάτωσεν αυτόν και εβασίλευσεν αντ' αυτού.
\par 26 Αι δε λοιπαί πράξεις του Φακείου και πάντα όσα έπραξεν, ιδού, είναι γεγραμμένα εν τω βιβλίω των χρονικών των βασιλέων του Ισραήλ.
\par 27 Εν τω πεντηκοστώ δευτέρω έτει του Αζαρίου βασιλέως του Ιούδα, Φεκά ο υιός του Ρεμαλία εβασίλευσεν επί τον Ισραήλ εν Σαμαρεία, είκοσι έτη.
\par 28 Και έπραξε πονηρά ενώπιον του Κυρίου· δεν απεμακρύνθη από των αμαρτιών του Ιεροβοάμ υιού του Ναβάτ, όστις έκαμε τον Ισραήλ να αμαρτήση.
\par 29 Εν ταις ημέραις του Φεκά βασιλέως του Ισραήλ, ήλθεν ο Θεγλάθ-φελασάρ βασιλεύς της Ασσυρίας, και εκυρίευσε την Ιϊών και την Αβέλ-βαίθ-μααχά και την Ιανώχ, και την Κεδές και την Ασώρ και την Γαλαάδ και την Γαλιλαίαν, πάσαν την γην Νεφθαλί, και μετώκισεν αυτούς εις Ασσυρίαν.
\par 30 Και έκαμεν Ωσηέ ο υιός του Ηλά συνωμοσίαν κατά του Φεκά υιού του Ρεμαλία, και επάταξεν αυτόν και εθανάτωσεν αυτόν και εβασίλευσεν αντ' αυτού, εν τω εικοστώ έτει του Ιωθάμ υιού του Οζίου.
\par 31 Αι δε λοιπαί πράξεις του Φεκά, και πάντα όσα έπραξεν, ιδού, είναι γεγραμμένα εν τω βιβλίω των χρονικών των βασιλέων του Ισραήλ.
\par 32 Εν τω δευτέρω έτει του Φεκά υιού του Ρεμαλία βασιλέως του Ισραήλ, εβασίλευσεν ο Ιωθάμ υιός του Οζίου βασιλέως του Ιούδα.
\par 33 Εικοσιπέντε ετών ηλικίας ήτο ότε εβασίλευσε, και εβασίλευσε δεκαέξ έτη εν Ιερουσαλήμ· το δε όνομα της μητρός αυτού ήτο Ιερουσά θυγάτηρ του Σαδώκ.
\par 34 Και έπραξε το ευθές ενώπιον Κυρίου· έπραξε κατά πάντα όσα έπραξεν Οζίας ο πατήρ αυτού.
\par 35 Πλην οι υψηλοί τόποι δεν αφηρέθησαν· ο λαός έτι εθυσίαζε και εθυμίαζεν επί τους υψηλούς τόπους. Ούτος ωκοδόμησε την υψηλήν πύλην του οίκου του Κυρίου.
\par 36 Αι δε λοιπαί πράξεις του Ιωθάμ και πάντα όσα έπραξε, δεν είναι γεγραμμένα εν τω βιβλίω των χρονικών των βασιλέων του Ιούδα;
\par 37 Εν ταις ημέραις εκείναις ήρχισεν ο Κύριος να εξαποστέλλη κατά του Ιούδα τον Ρεσίν βασιλέα της Συρίας και τον Φεκά υιόν του Ρεμαλία.
\par 38 Ο δε Ιωθάμ εκοιμήθη μετά των πατέρων αυτού, και ετάφη μετά των πατέρων αυτού εν πόλει Δαβίδ του πατρός αυτού· εβασίλευσε δε αντ' αυτού Άχαζ ο υιός αυτού.

\chapter{16}

\par 1 Εν τω δεκάτω εβδόμω έτει του Φεκά υιού του Ρεμαλία, εβασίλευσεν ο Άχαζ υιός του Ιωθάμ, βασιλέως του Ιούδα.
\par 2 Είκοσι ετών ηλικίας ήτο ο Άχαζ ότε εβασίλευσε, και εβασίλευσε δεκαέξ έτη εν Ιερουσαλήμ. Δεν έπραξεν όμως το ευθές ενώπιον Κυρίου του Θεού αυτού, ως ο Δαβίδ ο πατήρ αυτού.
\par 3 Αλλά περιεπάτησεν εν τη οδώ των βασιλέων του Ισραήλ, και μάλιστα διεβίβασε τον υιόν αυτού διά του πυρός, κατά τα βδελύγματα των εθνών, τα οποία ο Κύριος εξεδίωξεν απ' έμπροσθεν των υιών Ισραήλ.
\par 4 Και εθυσίαζε και εθυμίαζεν επί τους υψηλούς τόπους και επί τους λόφους και υποκάτω παντός πρασίνου δένδρου.
\par 5 Τότε ανέβησαν εις Ιερουσαλήμ διά πόλεμον Ρεσίν, ο βασιλεύς της Συρίας, και Φεκά, ο υιός του Ρεμαλία, βασιλεύς του Ισραήλ· και επολιόρκησαν τον Άχαζ, πλην δεν ηδυνήθησαν να νικήσωσι.
\par 6 Κατ' εκείνον τον καιρόν Ρεσίν ο βασιλεύς της Συρίας αποκατέστησε την Ελάθ υπό την εξουσίαν της Συρίας, και εδίωξε τους Ιουδαίους από της Ελάθ· και ελθόντες Σύριοι εις την Ελάθ, κατώκησαν εκεί έως της ημέρας ταύτης.
\par 7 Ο δε Άχαζ απέστειλε μηνυτάς προς τον Θεγλάθ-φελασάρ, βασιλέα της Ασσυρίας, λέγων, Εγώ είμαι δούλός σου και υιός σου· ανάβα και σώσον με εκ χειρός του βασιλέως της Συρίας και εκ χειρός του βασιλέως του Ισραήλ, οίτινες εσηκώθησαν εναντίον μου.
\par 8 Και έλαβεν ο Άχαζ το αργύριον και το χρυσίον το ευρεθέν εν τω οίκω του Κυρίου και εν τοις θησαυροίς του οίκου του βασιλέως, και απέστειλε δώρον εις τον βασιλέα της Ασσυρίας.
\par 9 Και εισήκουσεν αυτού ο βασιλεύς της Ασσυρίας· και ανέβη ο βασιλεύς της Ασσυρίας επί την Δαμασκόν και εκυρίευσεν αυτήν, και μετώκισε τον λαόν αυτής εις Κιρ, τον δε Ρεσίν εθανάτωσε.
\par 10 Και υπήγεν ο βασιλεύς Άχαζ εις την Δαμασκόν, προς συνάντησιν του Θεγλάθ-φελασάρ, βασιλέως της Ασσυρίας, και είδε το θυσιαστήριον το εν Δαμασκώ· και έστειλεν ο βασιλεύς Άχαζ προς τον Ουρίαν τον ιερέα το ομοίωμα του θυσιαστηρίου και τον τύπον αυτού, καθ' όλην την εργασίαν αυτού.
\par 11 Και ωκοδόμησεν Ουρίας ο ιερεύς το θυσιαστήριον, κατά η πάντα όσα ο βασιλεύς Άχαζ απέστειλεν εκ Δαμασκού. Ούτως έκαμεν Ουρίας ο ιερεύς, εωσού έλθη ο βασιλεύς Άχαζ εκ της Δαμασκού.
\par 12 Και ότε ήλθεν ο βασιλεύς εκ της Δαμασκού, είδεν ο βασιλεύς το θυσιαστήριον· και επλησίασεν ο βασιλεύς προς το θυσιαστήριον και έκαμε προσφοράν επ' αυτού.
\par 13 Και έκαυσε το ολοκαύτωμα αυτού και την εξ αλφίτων προσφοράν αυτού, και επέχεε την σπονδήν αυτού, και ερράντισε το αίμα των ειρηνικών αυτού προσφορών, επί το θυσιαστήριον.
\par 14 Και μετέφερε το χάλκινον θυσιαστήριον, το έμπροσθεν του Κυρίου, από του προσώπου του οίκου, από του μεταξύ του θυσιαστηρίου και του οίκου του Κυρίου, και έθεσεν αυτό κατά το βόρειον πλευρόν του θυσιαστηρίου.
\par 15 Και προσέταξεν ο βασιλεύς Άχαζ τον Ουρίαν τον ιερέα, λέγων, Επί το θυσιαστήριον το μέγα πρόσφερε το ολοκαύτωμα το πρωϊνόν και την εσπερινήν εξ αλφίτων προσφοράν και το ολοκαύτωμα του βασιλέως και την εξ αλφίτων προσφοράν αυτού, μετά του ολοκαυτώματος παντός του λαού της γης, και την εξ αλφίτων προσφοράν αυτών και τας σπονδάς αυτών· και ράντισον επ' αυτό άπαν το αίμα του ολοκαυτώματος και άπαν το αίμα της θυσίας· το δε χάλκινον θυσιαστήριον θέλει είσθαι εις εμέ διά να ερωτώ τον Κύριον.
\par 16 Και έκαμεν Ουρίας ο ιερεύς κατά πάντα όσα προσέταξεν ο βασιλεύς Άχαζ.
\par 17 Και απέκοψεν ο βασιλεύς Άχαζ τα συγκλείσματα των βάσεων, και εσήκωσεν επάνωθεν αυτών τον λουτήρα· και κατεβίβασε την θάλασσαν επάνωθεν των χαλκίνων βοών των υποκάτω αυτής, και έθεσεν αυτήν επί βάσιν λιθίνην.
\par 18 Και το στέγασμα του σαββάτου, το οποίον είχον οικοδομήσει εν τω οίκω, και την είσοδον του βασιλέως την έξω μετετόπισεν από του οίκου του Κυρίου, εξ αιτίας του βασιλέως της Ασσυρίας.
\par 19 Αι δε λοιπαί πράξεις του Άχαζ, όσας έπραξε, δεν είναι γεγραμμέναι εν τω βιβλίω των χρονικών των βασιλέων του Ιούδα;
\par 20 Και εκοιμήθη ο Άχαζ μετά των πατέρων αυτού και ετάφη μετά των πατέρων αυτού εν πόλει Δαβίδ· εβασίλευσε δε αντ' αυτού Εζεκίας ο υιός αυτού.

\chapter{17}

\par 1 Εν τω δωδεκάτω έτει του Άχαζ βασιλέως του Ιούδα, εβασίλευσεν Ωσηέ ο υιός του Ηλά εν Σαμαρεία επί τον Ισραήλ, εννέα έτη.
\par 2 Και έπραξε πονηρά ενώπιον του Κυρίου, πλην ουχί ως οι βασιλείς του Ισραήλ οίτινες ήσαν προ αυτού.
\par 3 Επ' αυτόν ανέβη Σαλμανασάρ ο βασιλεύς της Ασσυρίας· και έγεινεν Ωσηέ δούλος αυτού και έδιδεν εις αυτόν φόρον.
\par 4 Εύρηκε δε ο βασιλεύς της Ασσυρίας συνωμοσίαν εν τω Ωσηέ· διότι απέστειλε μηνυτάς προς τον Σω, βασιλέα της Αιγύπτου, και δεν έδωκε φόρον εις τον βασιλέα της Ασσυρίας, ως έκαμνε κατ' έτος· όθεν συνέκλεισεν αυτόν ο βασιλεύς της Ασσυρίας και έδεσεν αυτόν εν φυλακή.
\par 5 Και ανέβη ο βασιλεύς της Ασσυρίας διά πάσης της γής· και ανέβη εις την Σαμάρειαν και επολιόρκησεν αυτήν τρία έτη.
\par 6 Εν τω εννάτω έτει του Ωσηέ, ο βασιλεύς της Ασσυρίας εκυρίευσε την Σαμάρειαν και μετώκισε τον Ισραήλ εις την Ασσυρίαν, και κατώκισεν αυτούς εν Αλά και εν Αβώρ, παρά τον ποταμόν Γωζάν, και εν ταις πόλεσι των Μήδων.
\par 7 Έγεινε δε τούτο, διότι οι υιοί του Ισραήλ ημάρτησαν εις Κύριον τον Θεόν αυτών, όστις ανήγαγεν αυτούς εκ γης Αιγύπτου, υποκάτωθεν της χειρός του Φαραώ βασιλέως της Αιγύπτου, και εσεβάσβησαν άλλους θεούς,
\par 8 και περιεπάτησαν εις τα νόμιμα των εθνών, τα οποία εξεδίωξεν ο Κύριος απέμπροσθεν των υιών Ισραήλ, και τα των βασιλέων του Ισραήλ, τα οποία εθέσπισαν.
\par 9 Και έπραττον οι υιοί του Ισραήλ κρυφίως πράγματα, τα οποία δεν ήσαν ευθέα ενώπιον Κυρίου του Θεού αυτών, και ωκοδόμησαν εις εαυτούς υψηλούς τόπους εν πάσαις ταις πόλεσιν αυτών, από πύργου φυλάκων έως οχυράς πόλεως.
\par 10 Και ανήγειραν εις εαυτούς αγάλματα και άλση επί πάντα υψηλόν λόφον και υποκάτω παντός δένδρου πρασίνου.
\par 11 Και εκεί εθυμίαζον επί πάντας τους υψηλούς τόπους, καθώς τα έθνη τα οποία ο Κύριος εξεδίωξεν απέμπροσθεν αυτών· και έπραττον πονηρά πράγματα διά να παροργίζωσι τον Κύριον·
\par 12 και ελάτρευσαν τα είδωλα, περί των οποίων ο Κύριος είπε προς αυτούς, Δεν θέλετε κάμει το πράγμα τούτο.
\par 13 Και διεμαρτυρήθη ο Κύριος κατά του Ισραήλ και κατά του Ιούδα, διά χειρός πάντων των προφητών, πάντων των βλεπόντων, λέγων, Επιστρέψατε από των οδών υμών των πονηρών και φυλάττετε τας εντολάς μου, τα διατάγματά μου, κατά πάντα τον νόμον τον οποίον προσέταξα εις τους πατέρας σας και τον οποίον απέστειλα εις εσάς διά μέσου των δούλων μου των προφητών.
\par 14 Πλην αυτοί δεν υπήκουσαν, αλλ' εσκλήρυναν τον τράχηλον αυτών, ως τον τράχηλον των πατέρων αυτών, οίτινες δεν επίστευσαν εις Κύριον τον Θεόν αυτών.
\par 15 Και απέρριψαν τα διατάγματα αυτού και την διαθήκην αυτού, την οποίαν έκαμε μετά των πατέρων αυτών, και τας διαμαρτυρήσεις αυτού, τας οποίας διεμαρτυρήθη εναντίον αυτών· και υπήγαν οπίσω της ματαιότητος, και εματαιώθησαν, και οπίσω των εθνών των πέριξ αυτών, περί των οποίων ο Κύριος προσέταξεν αυτούς, να μη πράξωσιν ως εκείνα.
\par 16 Και εγκατέλιπον πάσας ταις εντολάς Κυρίου του Θεού αυτών, και έκαμον εις εαυτούς χωνευτά, δύο μόσχους, και έκαμον άλση και προσεκύνησαν πάσαν την στρατιάν του ουρανού και ελάτρευσαν τον Βάαλ.
\par 17 Και διεβίβαζον τους υιούς αυτών και τας θυγατέρας αυτών διά του πυρός, και μετεχειρίζοντο μαντείας και οιωνισμούς, και επώλησαν εαυτούς εις το να πράττωσι πονηρά ενώπιον του Κυρίου, διά να παροργίζωσιν αυτόν.
\par 18 Διά ταύτα ο Κύριος ωργίσθη σφόδρα κατά του Ισραήλ και απέβαλεν αυτούς από προσώπου αυτού· δεν εναπελείφθη, παρά μόνη η φυλή του Ιούδα.
\par 19 Και ο Ιούδας έτι δεν εφύλαξε τας εντολάς Κυρίου του Θεού αυτού, αλλά περιεπάτησαν εις τα διατάγματα του Ισραήλ, τα οποία έκαμον.
\par 20 Και απέβαλεν ο Κύριος παν το σπέρμα του Ισραήλ και κατέθλιψεν αυτούς, και παρέδωκεν αυτούς εις την χείρα των διαρπαζόντων, εωσού απέρριψεν αυτούς από προσώπου αυτού.
\par 21 Διότι απεσχίσθη ο Ισραήλ από του οίκου Δαβίδ, και έκαμον βασιλέα τον Ιεροβοάμ υιόν του Ναβάτ· και ο Ιεροβοάμ απέσπασε τον Ισραήλ εξόπισθεν του Κυρίου, και έκαμεν αυτούς να αμαρτήσωσιν αμαρτίαν μεγάλην.
\par 22 Διότι οι υιοί Ισραήλ περιεπάτησαν εν πάσαις ταις αμαρτίαις του Ιεροβοάμ, τας οποίας έπραξε· δεν απεμακρύνθησαν απ' αυτών,
\par 23 εωσού ο Κύριος απέβαλε τον Ισραήλ από προσώπου αυτού, καθώς ελάλησε διά χειρός πάντων των δούλων αυτού των προφητών. Και μετωκίσθη ο Ισραήλ από της γης αυτού εις την Ασσυρίαν, έως της ημέρας ταύτης.
\par 24 Και έφερεν ο βασιλεύς της Ασσυρίας ανθρώπους εκ Βαβυλώνος και από Χουθά και από Αυά και από Αιμάθ και από Σεφαρουΐμ, και κατώκισεν εν ταις πόλεσι της Σαμαρείας αντί των υιών Ισραήλ, και εκληρονόμησαν την Σαμάρειαν και κατώκησαν εν ταις πόλεσιν αυτής.
\par 25 Και εν τη αρχή της εκεί κατοικήσεως αυτών, δεν εφοβήθησαν τον Κύριον· και απέστειλεν ο Κύριος τους λέοντας μεταξύ αυτών, και εθανάτονον εξ αυτών.
\par 26 Και είπον προς τον βασιλέα της Ασσυρίας, λέγοντες, Τα έθνη, τα οποία μετώκισας και εκάθισας εν ταις πόλεσι της Σαμαρείας, δεν γνωρίζουσι τον νόμον του Θεού της γής· διά τούτο απέστειλε τους λέοντας μεταξύ αυτών, και ιδού, θανατόνουσιν αυτούς, επειδή δεν γνωρίζουσι τον νόμον του Θεού της γης.
\par 27 Τότε ο βασιλεύς της Ασσυρίας προσέταξε, λέγων, Φέρετε εκεί ένα των ιερέων, τους οποίους μετωκίσατε εκείθεν· και ας υπάγωσι και ας κατοικήσωσιν εκεί· και ας διδάξη αυτούς τον νόμον του Θεού της γης.
\par 28 Και εις των ιερέων, τους οποίους μετώκισαν εκ της Σαμαρείας, ήλθε και κατώκησεν εν Βαιθήλ, και εδίδασκεν αυτούς πως να φοβώνται τον Κύριον.
\par 29 Έκαστον όμως έθνος έκαμον θεούς εις εαυτούς και έθεσαν εις τους οίκους των υψηλών τόπων, τους οποίους οι Σαμαρείται έκαμον, έκαστον έθνος εν ταις πόλεσιν αυτών, όπου κατώκουν.
\par 30 Και οι άνδρες της Βαβυλώνος έκαμον την Σοκχώθ-βενώθ, οι δε άνδρες της Χουθά έκαμον την Νεργάλ, και οι άνδρες της Αιμάθ έκαμον την Ασιμά,
\par 31 και οι Αυίται έκαμον την Νιβάζ και τον Ταρτάκ, και οι Σεφαρουΐται έκαιον τους υιούς αυτών διά του πυρός εις τον Αδραμμέλεχ και Αναμμέλεχ, θεούς των Σεφαρουϊτών.
\par 32 Ούτως εφοβούντο τον Κύριον· έκαμον δε εις εαυτούς εκ των εσχάτων μεταξύ αυτών ιερείς των υψηλών τόπων, οίτινες εθυσίαζον υπέρ αυτών εν τοις οίκοις των υψηλών τόπων.
\par 33 Εφοβούντο μεν τον Κύριον, ελάτρευον όμως τους ιδίους αυτών θεούς, κατά τον τρόπον των εθνών, όθεν μετωκίσθησαν.
\par 34 Έως της ημέρας ταύτης κάμνουσι κατά τους προτέρους τρόπους· δεν φοβούνται τον Κύριον και δεν πράττουσι κατά τα διατάγματα αυτών και κατά τας κρίσεις αυτών και κατά τον νόμον και την εντολήν, την οποίαν προσέταξεν ο Κύριος εις τους υιούς Ιακώβ, τον οποίον ωνόμασεν Ισραήλ·
\par 35 και έκαμε προς αυτούς ο Κύριος διαθήκην και προσέταξεν αυτούς, λέγων, Δεν θέλετε φοβηθή άλλους θεούς, και δεν θέλετε προσκυνήσει αυτούς ουδέ λατρεύσει αυτούς ουδέ θυσιάσει εις αυτούς·
\par 36 αλλά τον Κύριον, όστις σας ανήγαγεν εκ γης Αιγύπτου μετά δυνάμεως μεγάλης και εν βραχίονι εξηπλωμένω, αυτόν θέλετε φοβείσθαι και αυτόν θέλετε προσκυνεί και εις αυτόν θέλετε θυσιάζει,
\par 37 και τα διατάγματα και τας κρίσεις και τον νόμον και την εντολήν, την οποίαν έγραψε διά σας, θέλετε προσέχει να εκτελήτε πάντοτε· άλλους δε θεούς δεν θέλετε φοβηθή·
\par 38 και την διαθήκην, την οποίαν έκαμα προς εσάς, δεν θέλετε λησμονήσει και δεν θέλετε φοβηθή άλλους θεούς·
\par 39 αλλά Κύριον τον Θεόν σας θέλετε φοβείσθαι και αυτός θέλει σας ελευθερώσει εκ χειρός πάντων των εχθρών σας.
\par 40 Πλην δεν υπήκουσαν, αλλ' έκαμνον κατά τους προτέρους τρόπους αυτών.
\par 41 Και τα έθνη ταύτα εφοβούντο μεν τον Κύριον, ελάτρευον όμως τα γλυπτά αυτών· και οι υιοί αυτών και των υιών αυτών οι υιοί, καθώς οι πατέρες αυτών έκαμνον, ούτω κάμνουσιν έως της ημέρας ταύτης.

\chapter{18}

\par 1 Εν δε τω τρίτω έτει του Ωσηέ υιού του Ηλά, βασιλέως του Ισραήλ, εβασίλευσεν Εζεκίας ο υιός του Άχαζ βασιλέως του Ιούδα.
\par 2 Εικοσιπέντε ετών ηλικίας ήτο, ότε εβασίλευσεν· εβασίλευσε δε εικοσιεννέα έτη εν Ιερουσαλήμ. Και το όνομα της μητρός αυτού ήτο Αβί, θυγάτηρ του Ζαχαρίου.
\par 3 Και έκαμε το ευθές ενώπιον του Κυρίου, κατά πάντα όσα έκαμε Δαβίδ ο πατήρ αυτού.
\par 4 Αυτός αφήρεσε τους υψηλούς τόπους και κατέθραυσε τα αγάλματα και κατέκοψε τα άλση και κατεσύντριψε τον χάλκινον όφιν, τον οποίον έκαμεν ο Μωϋσής· διότι έως των ημερών εκείνων οι υιοί του Ισραήλ εθυμίαζον εις αυτόν· και εκάλεσεν αυτόν Νεουσθάν.
\par 5 Επί Κύριον τον Θεόν του Ισραήλ ήλπισε· και δεν εστάθη μετ' αυτόν όμοιος αυτού μεταξύ πάντων των βασιλέων του Ιούδα, αλλ' ουδέ των προ αυτού·
\par 6 διότι προσεκολλήθη εις τον Κύριον· δεν απεμακρύνθη από όπισθεν αυτού, αλλ' εφύλαξε τας εντολάς αυτού, τας οποίας ο Κύριος προσέταξεν εις τον Μωϋσήν.
\par 7 Και ήτο ο Κύριος μετ' αυτού· κατευοδούτο όπου εξήρχετο· και απεστάτησε κατά του βασιλέως της Ασσυρίας και δεν εδούλευσεν εις αυτόν.
\par 8 Αυτός επάταξε τους Φιλισταίους, έως Γάζης και των ορίων αυτής, από πύργου φυλάκων έως οχυράς πόλεως.
\par 9 Εν δε τω τετάρτω έτει του βασιλέως Εζεκίου, το οποίον ήτο το έβδομον έτος του Ωσηέ, υιού του Ηλά βασιλέως του Ισραήλ, Σαλμανασάρ ο βασιλεύς της Ασσυρίας ανέβη επί την Σαμάρειαν και επολιόρκει αυτήν.
\par 10 Και εν τω τέλει τριών ετών εκυρίευσαν αυτήν· εν τω έκτω έτει του Εζεκίου, το οποίον είναι το έννατον του Ωσηέ βασιλέως του Ισραήλ, εκυριεύθη η Σαμάρεια.
\par 11 Και μετώκισεν ο βασιλεύς της Ασσυρίας τον Ισραήλ εις την Ασσυρίαν, και έθεσεν αυτούς εν Αλά και εν Αβώρ παρά τον ποταμόν Γωζάν και εν ταις πόλεσι των Μήδων·
\par 12 διότι δεν υπήκουσαν της φωνής Κυρίου του Θεού αυτών, αλλά παρέβησαν την διαθήκην αυτού, πάντα όσα προσέταξε Μωϋσής ο δούλος του Κυρίου, και δεν υπήκουσαν ουδέ έκαμον αυτά.
\par 13 Εν δε τω δεκάτω τετάρτω έτει του βασιλέως Εζεκίου, ανέβη Σενναχειρείμ ο βασιλεύς της Ασσυρίας επί πάσας τας οχυράς πόλεις του Ιούδα και εκυρίευσεν αυτάς.
\par 14 Και απέστειλεν ο Εζεκίας βασιλεύς του Ιούδα προς τον βασιλέα της Ασσυρίας εις Λαχείς, λέγων, Ημάρτησα· απόστρεψον απ' εμού· ό,τι επιβάλης επ' εμέ, θέλω βαστάσει αυτό. Και επέβαλεν ο βασιλεύς της Ασσυρίας επί τον Εζεκίαν τον βασιλέα του Ιούδα, τριακόσια τάλαντα αργυρίου και τριάκοντα τάλαντα χρυσίου.
\par 15 Και έδωκεν εις αυτόν ο Εζεκίας άπαν το αργύριον το ευρεθέν εν τω οίκω του Κυρίου και εν τοις θησαυροίς του οίκου του βασιλέως.
\par 16 Κατ' εκείνον τον καιρόν απέκοψεν ο Εζεκίας τας θύρας του ναού του Κυρίου και τους στύλους, τους οποίους Εζεκίας ο βασιλεύς του Ιούδα είχε περισκεπάσει με χρυσίον, και έδωκεν αυτό εις τον βασιλέα της Ασσυρίας.
\par 17 Και απέστειλεν ο βασιλεύς της Ασσυρίας τον Ταρτάν και τον Ραβ-σαρείς, και τον Ραβ-σάκην, από Λαχείς, προς τον βασιλέα Εζεκίαν, μετά δυνάμεως μεγάλης εις Ιερουσαλήμ· οι δε ανέβησαν και ήλθον εις την Ιερουσαλήμ. Και ότε ανέβησαν, ήλθον και εστάθησαν εν τω υδραγωγώ της άνω κολυμβήθρας, ήτις είναι εν τη μεγάλη οδώ του αγρού του γναφέως.
\par 18 Και εβόησαν προς τον βασιλέα, και εξήλθον προς αυτούς Ελιακείμ, ο υιός του Χελκίου, ο οικονόμος, και Σομνάς ο γραμματεύς και Ιωάχ, ο υιός του Ασάφ, ο υπομνηματογράφος.
\par 19 Και είπε προς αυτούς ο Ραβ-σάκης, Είπατε τώρα προς τον Εζεκίαν, Ούτω λέγει ο βασιλεύς ο μέγας, ο βασιλεύς της Ασσυρίας· Ποίον είναι το θάρρος τούτο επί το οποίον θαρρείς;
\par 20 συ λέγεις, πλην είναι λόγοι χειλέων, Έχω βουλήν και δύναμιν διά πόλεμον· αλλ' επί τίνα θαρρείς, ώστε απεστάτησας εναντίον μου;
\par 21 τώρα ιδού, συ θαρρείς επί την ράβδον του συντετριμμένου εκείνου καλάμου, επί την Αίγυπτον, επί τον οποίον εάν τις επιστηριχθή, θέλει εμπηχθή εις την χείρα αυτού και τρυπήσει αυτήν· τοιούτος είναι Φαραώ ο βασιλεύς της Αιγύπτου προς πάντας τους θαρρούντας επ' αυτόν.
\par 22 Αλλ' εάν είπητε προς εμέ, Επί Κύριον τον Θεόν ημών θαρρούμεν· δεν είναι αυτός, του οποίου τους υψηλούς τόπους και τα θυσιαστήρια αφήρεσεν ο Εζεκίας, και είπε προς τον Ιούδαν και προς την Ιερουσαλήμ, Έμπροσθεν τούτου του θυσιαστηρίου θέλετε προσκυνήσει εν Ιερουσαλήμ;
\par 23 Τώρα λοιπόν, δος ενέχυρα εις τον κύριόν μου τον βασιλέα της Ασσυρίας, και εγώ θέλω σοι δώσει δισχιλίους ίππους, αν δύνασαι από μέρους σου να δώσης επιβάτας επ' αυτούς.
\par 24 Πως λοιπόν θέλεις στρέψει οπίσω το πρόσωπον ενός τοπάρχου εκ των ελαχίστων δούλων του κυρίου μου, και ήλπισας επί την Αίγυπτον διά αμάξας και διά ιππέας;
\par 25 Και τώρα άνευ του Κυρίου ανέβην εγώ επί τον τόπον τούτον, διά να καταστρέψω αυτόν; Ο Κύριος είπε προς εμέ, Ανάβα επί την γην ταύτην και κατάστρεψον αυτήν.
\par 26 Τότε είπεν Ελιακείμ ο υιός του Χελκίου, και ο Σομνάς και ο Ιωάχ, προς τον Ραβ-σάκην, Λάλησον, παρακαλώ, προς τους δούλους σου εις την Συριακήν γλώσσαν· διότι καταλαμβάνομεν αυτήν· και μη λάλει προς ημάς Ιουδαϊστί, εις επήκοον του λαού επί του τείχους.
\par 27 Αλλ' ο Ραβ-σάκης είπε προς αυτούς, Μήπως ο κύριός μου απέστειλεν εμέ προς τον κύριόν σου ή και προς σε, διά να λαλήσω τους λόγους τούτους; δεν με απέστειλε προς τους άνδρας τους καθημένους επί του τείχους, διά να φάγωσι την κόπρον αυτών και να πίωσι το ούρον αυτών με σας;
\par 28 Τότε ο Ραβ-σάκης εστάθη και εφώνησεν Ιουδαϊστί μετά φωνής μεγάλης και ελάλησε, λέγων, Ακούσατε τον λόγον του βασιλέως του μεγάλου, του βασιλέως της Ασσυρίας.
\par 29 ούτω λέγει ο βασιλεύς· Μη σας απατά ο Εζεκίας· διότι δεν θέλει δυνηθή να σας λυτρώση εκ της χειρός αυτού·
\par 30 και μη σας κάμνη ο Εζεκίας να θαρρήτε επί τον Κύριον, λέγων, Ο Κύριος βεβαίως θέλει μας λυτρώσει, και η πόλις αύτη δεν θέλει παραδοθή εις την χείρα του βασιλέως της Ασσυρίας.
\par 31 Μη ακούετε του Εζεκίου· διότι ούτω λέγει ο βασιλεύς της Ασσυρίας. Κάμετε συμβιβασμόν μετ' εμού και εξέλθετε προς εμέ· και φάγετε έκαστος από της αμπέλου αυτού και έκαστος από της συκής αυτού, και πίετε έκαστος από των υδάτων της δεξαμενής αυτού·
\par 32 εωσού έλθω και σας λάβω εις γην ομοίαν με την γην σας, γην σίτου και οίνου, γην άρτου και αμπελώνων, γην ελαίου και μέλιτος, διά να ζήσητε και να μη αποθάνητε· και μη ακούετε του Εζεκίου, όταν σας απατά, λέγων, Ο Κύριος θέλει μας λυτρώσει.
\par 33 Μήπως ελύτρωσέ τις τωόντι εκ των θεών των εθνών την γην αυτού εκ της χειρός του βασιλέως της Ασσυρίας;
\par 34 που οι θεοί της Αιμάθ και Αρφάδ; που οι θεοί της Σεφαρουΐμ, της Ενά και της Αυά; μήπως ελύτρωσαν εκ της χειρός μου την Σαμάρειαν;
\par 35 τίνες μεταξύ πάντων των θεών των τόπων ελύτρωσαν την γην αυτών εκ της χειρός μου, ώστε και ο Κύριος να λυτρώση την Ιερουσαλήμ εκ της χειρός μου;
\par 36 Ο δε λαός εσιώπα και δεν απεκρίθη λόγον προς αυτόν· διότι ο βασιλεύς είχε προστάξει, λέγων, Μη αποκριθήτε προς αυτόν.
\par 37 Τότε Ελιακείμ, ο υιός του Χελκίου, ο οικονόμος, και Σομνάς ο γραμματεύς και Ιωάχ, ο υιός του Ασάφ, ο υπομνηματογράφος, ήλθον προς τον Εζεκίαν με διεσχισμένα ιμάτια και απήγγειλαν προς αυτόν τους λόγους του Ραβ-σάκη.

\chapter{19}

\par 1 Και ότε ήκουσεν ο βασιλεύς Εζεκίας, διέσχισε τα ιμάτια αυτού και εσκεπάσθη με σάκκον και εισήλθεν εις τον οίκον του Κυρίου.
\par 2 Και απέστειλεν Ελιακείμ τον οικονόμον και Σομνάν τον γραμματέα και τους πρεσβυτέρους των ιερέων, εσκεπασμένους με σάκκους, προς τον προφήτην Ησαΐαν, τον υιόν του Αμώς.
\par 3 Και είπον προς αυτόν, Ούτω λέγει ο Εζεκίας· Ημέρα θλίψεως και ονειδισμού και βλασφημίας η ημέρα αύτη· διότι τα τέκνα ήλθον εις την ακμήν της γέννας, πλην δύναμις δεν είναι εις την τίκτουσαν·
\par 4 είθε να ήκουσε Κύριος ο Θεός σου πάντας τους λόγους του Ραβ-σάκη, τον οποίον ο βασιλεύς της Ασσυρίας ο κύριος αυτού απέστειλε διά να ονειδίση τον ζώντα θεόν, και να υβρίση διά των λόγων, τους οποίους ήκουσε Κύριος ο Θεός σου· διά τούτο ύψωσον δέησιν υπέρ του υπολοίπου του σωζομένου.
\par 5 Και ήλθον προς τον Ησαΐαν οι δούλοι του βασιλέως Εζεκίου.
\par 6 Και είπε προς αυτούς ο Ησαΐας, Ούτω θέλετε ειπεί προς τον κύριόν σας· Ούτω λέγει Κύριος· Μη φοβού από των λόγων τους οποίους ήκουσας, διά των οποίων οι δούλοι του βασιλέως της Ασσυρίας με ωνείδισαν·
\par 7 ιδού, εγώ θέλω βάλει εις αυτόν τοιούτον πνεύμα, ώστε, ακούσας θόρυβον, θέλει επιστρέψει εις την γην αυτού· και θέλω κάμει αυτόν να πέση διά μαχαίρας εν τη γη αυτού.
\par 8 Ο Ραβ-σάκης λοιπόν επέστρεψε και εύρηκε τον βασιλέα της Ασσυρίας πολεμούντα εναντίον της Λιβνά· διότι ήκουσεν ότι έφυγεν από Λαχείς.
\par 9 Και ο βασιλεύς, ότε ήκουσε να λέγωσι περί Θιρακά του βασιλέως της Αιθιοπίας, Ιδού, εξήλθε να σε πολεμήση, απέστειλε πάλιν πρέσβεις προς τον Εζεκίαν, λέγων,
\par 10 Ούτω θέλετε ειπεί προς Εζεκίαν, τον βασιλέα του Ιούδα, λέγοντες, Ο Θεός σου, επί τον οποίον θαρρείς, ας μη σε απατά, λέγων, Η Ιερουσαλήμ δεν θέλει παραδοθή εις την χείρα του βασιλέως της Ασσυρίας·
\par 11 ιδού, συ ήκουσας τι έκαμον οι βασιλείς της Ασσυρίας εις πάντας τους τόπους, καταστρέφοντες αυτούς· και συ θέλεις λυτρωθή;
\par 12 μήπως οι θεοί των εθνών ελύτρωσαν εκείνους, τους οποίους οι πατέρες μου κατέστρεψαν, την Γωζάν και την Χαρράν και Ρεσέφ και τους υιούς του Εδέν τους εν Τελασσάρ;
\par 13 που ο βασιλεύς της Αιμάθ, και ο βασιλεύς της Αρφάδ, και ο βασιλεύς της πόλεως Σεφαρουΐμ, Ενά και Αυά;
\par 14 Και λαβών ο Εζεκίας την επιστολήν εκ της χειρός των πρέσβεων, ανέγνωσεν αυτήν· και ανέβη ο Εζεκίας εις τον οίκον του Κυρίου και εξετύλιξεν αυτήν ενώπιον του Κυρίου.
\par 15 Και προσηυχήθη ενώπιον του Κυρίου ο Εζεκίας, λέγων, Κύριε Θεέ του Ισραήλ, ο καθήμενος επί των χερουβείμ, συ αυτός είσαι ο Θεός, ο μόνος, πάντων των βασιλείων της γής· συ έκαμες τον ουρανόν και την γήν·
\par 16 κλίνον, Κύριε, το ους σου και άκουσον· άνοιξον, Κύριε, τους οφθαλμούς σου και ιδέ· και άκουσον τους λόγους του Σενναχειρείμ, όστις απέστειλε τούτον διά να ονειδίση τον ζώντα Θεόν·
\par 17 αληθώς, Κύριε, οι βασιλείς της Ασσυρίας ηρήμωσαν τα έθνη και τους τόπους αυτών,
\par 18 και έρριψαν εις το πυρ, τους θεούς αυτών· διότι δεν ήσαν θεοί, αλλ' έργον χειρών ανθρώπων, ξύλα και λίθοι· διά τούτο κατέστρεψαν αυτούς·
\par 19 τώρα λοιπόν, Κύριε Θεέ ημών, σώσον ημάς, δέομαι, εκ της χειρός αυτού· διά να γνωρίσωσι πάντα τα βασίλεια της γης, ότι συ είσαι Κύριος ο Θεός, ο μόνος.
\par 20 Τότε απέστειλεν Ησαΐας ο υιός του Αμώς προς τον Εζεκίαν, λέγων, Ούτω λέγει Κύριος ο Θεός του Ισραήλ· Ήκουσα όσα προσηυχήθης εις εμέ κατά του Σενναχειρείμ βασιλέως της Ασσυρίας.
\par 21 Ούτος είναι ο λόγος τον οποίον ο Κύριος ελάλησε περί αυτού· Σε κατεφρόνησε, σε ενέπαιξεν παρθένος, η θυγάτηρ της Σιών· οπίσω σου έσεισε κεφαλήν η θυγάτηρ της Ιερουσαλήμ.
\par 22 Τίνα ωνείδισας και εβλασφήμησας; και κατά τίνος ύψωσας φωνήν και εσήκωσας υψηλά τους οφθαλμούς σου; Κατά του Αγίου του Ισραήλ.
\par 23 Τον Κύριον ωνείδισας διά των πρέσβεών σου και είπας, Με το πλήθος των αμαξών μου ανέβην εγώ εις το ύψος των ορέων, εις τα πλευρά του Λιβάνου· και θέλω κόψει τας υψηλάς κέδρους αυτού, τας εκλεκτάς ελάτους αυτού· και θέλω εισέλθει εις τα έσχατα οικήματα αυτού, εις το δάσος του Καρμήλου αυτού·
\par 24 εγώ ανέσκαψα και έπιον ύδατα ξένα· και με το ίχνος των ποδών μου εξήρανα πάντας τους ποταμούς των πολιορκουμένων.
\par 25 Μη δεν ήκουσας ότι εγώ έκαμον τούτο παλαιόθεν, και από ημερών αρχαίων εβουλεύθην αυτό; τώρα δε εξετέλεσα τούτο, ώστε συ να ήσαι διά να καταστρέφης πόλεις ωχυρωμένας εις ερειπίων σωρούς.
\par 26 Διά τούτο οι κάτοικοι αυτών ήσαν μικράς δυνάμεως, ετρόμαξαν και κατησχύνθησαν· ήσαν ως ο χόρτος του αγρού και ως η χλόη, ως ο χόρτος των δωμάτων και ως ο σίτος ο καιόμενος πριν καλαμώση.
\par 27 Πλην εγώ εξεύρω την κατοικίαν σου και την έξοδόν σου και την είσοδόν σου και την κατ' εμού λύσσαν σου.
\par 28 Επειδή η κατ' εμού λύσσα σου και η αλαζονεία σου ανέβησαν εις τα ώτα μου, διά τούτο θέλω βάλει τον κρίκον μου εις τους μυκτήράς σου και τον χαλινόν μου εις τα χείλη σου, και θέλω σε επιστρέψει διά της οδού δι' ης ήλθες.
\par 29 Και τούτο θέλει είσθαι εις σε το σημείον· Το έτος τούτο θέλετε φάγει ό,τι είναι αυτοφυές· και το δεύτερον έτος, ό,τι εκφύεται από του αυτού· το δε τρίτον έτος, σπείρατε και θερίσατε και φυτεύσατε αμπελώνας και φάγετε τον καρπόν αυτών.
\par 30 Και το υπόλοιπον εκ του οίκου Ιούδα, το διασωθέν, θέλει ριζώσει πάλιν υποκάτωθεν και θέλει δώσει επάνω καρπούς.
\par 31 Διότι εξ Ιερουσαλήμ θέλει εξέλθει το υπόλοιπον και εκ του όρους Σιών το διασωθέν· ο ζήλος του Κυρίου των δυνάμεων θέλει εκτελέσει τούτο.
\par 32 Όθεν ούτω λέγει Κύριος περί του βασιλέως της Ασσυρίας· Δεν θέλει εισέλθει εις την πόλιν ταύτην ουδέ θέλει τοξεύσει εκεί βέλος ουδέ θέλει προβάλει κατ' αυτής ασπίδα ουδέ θέλει υψώσει εναντίον αυτής πρόχωμα.
\par 33 Διά της οδού δι' ης ήλθε, δι' αυτής θέλει επιστρέψει, και εις την πόλιν ταύτην δεν θέλει εισέλθει, λέγει Κύριος.
\par 34 Διότι θέλω υπερασπισθή την πόλιν ταύτην, ώστε να σώσω αυτήν, ένεκεν εμού και ένεκεν του δούλου μου Δαβίδ.
\par 35 Και την νύκτα εκείνην εξήλθεν ο άγγελος του Κυρίου και επάταξεν εν τω στρατοπέδω των Ασσυρίων εκατόν ογδοήκοντα πέντε χιλιάδας· και ότε εξηγέρθησαν το πρωΐ, ιδού, ήσαν πάντες σώματα νεκρά.
\par 36 Και εσηκώθη Σενναχειρείμ, ο βασιλεύς της Ασσυρίας, και έφυγε και επέστρεψε και κατώκησεν εν Νινευή.
\par 37 Και ενώ προσεκύνει εν τω οίκω Νισρώκ του θεού αυτού, Αδραμμέλεχ και Σαρασάρ οι υιοί αυτού επάταξαν αυτόν εν μαχαίρα. αυτοί δε έφυγον εις γην Αραράτ· εβασίλευσε δε αντ' αυτού Εσαραδδών ο υιός αυτού.

\chapter{20}

\par 1 Κατ' εκείνας τας ημέρας ηρρώστησεν ο Εζεκίας εις θάνατον· και ήλθε προς αυτόν Ησαΐας ο προφήτης, ο υιός του Αμώς, και είπε προς αυτόν, Ούτω λέγει Κύριος· Διάταξον περί του οίκου σου, επειδή αποθνήσκεις και δεν θέλεις ζήσει.
\par 2 Τότε έστρεψε το πρόσωπον αυτού προς τον τοίχον και προσηυχήθη εις τον Κύριον, λέγων,
\par 3 Δέομαι, Κύριε, ενθυμήθητι τώρα, πως περιεπάτησα ενώπιόν σου εν αληθεία και εν καρδία τελεία και έπραξα το αρεστόν ενώπιόν σου. Και έκλαυσεν ο Εζεκίας κλαυθμόν μέγαν.
\par 4 Και πριν εξέλθη ο Ησαΐας εις την αυλήν την μεσαίαν, έγεινε λόγος Κυρίου προς αυτόν, λέγων,
\par 5 Επίστρεψον και ειπέ προς τον Εζεκίαν τον ηγεμόνα του λαού μου, Ούτω λέγει Κύριος ο Θεός του Δαβίδ του πατρός σου· Ήκουσα την προσευχήν σου, είδον τα δάκρυά σου· ιδού, εγώ θέλω σε ιατρεύσει την τρίτην ημέραν θέλεις αναβή εις τον οίκον του Κυρίου·
\par 6 και θέλω προσθέσει εις τας ημέρας σου δεκαπέντε έτη· και θέλω ελευθερώσει σε και την πόλιν ταύτην εκ της χειρός του βασιλέως της Ασσυρίας· και θέλω υπερασπισθή την πόλιν ταύτην, ένεκεν εμού και ένεκεν του δούλου μου Δαβίδ.
\par 7 Και είπεν ο Ησαΐας, Λάβετε παλάθην σύκων. Και έλαβον και επέθεσαν αυτήν επί το έλκος, και ανέλαβε την υγείαν αυτού.
\par 8 Και είπεν ο Εζεκίας προς τον Ησαΐαν, Τι είναι το σημείον ότι ο Κύριος θέλει με ιατρεύσει, και ότι θέλω αναβή εις τον οίκον του Κυρίου την τρίτην ημέραν;
\par 9 Και είπεν ο Ησαΐας, Τούτο θέλει είσθαι εις σε το σημείον παρά Κυρίου, ότι θέλει κάμει ο Κύριος το πράγμα το οποίον ελάλησε· να προχωρήση η σκιά δέκα βαθμούς, ή να στραφή δέκα βαθμούς;
\par 10 Και απεκρίθη ο Εζεκίας, Ελαφρόν πράγμα είναι να καταβή η σκιά δέκα βαθμούς· ουχί, αλλ' ας στραφή οπίσω δέκα βαθμούς η σκιά.
\par 11 Και εβόησεν ο Ησαΐας ο προφήτης προς τον Κύριον, και έστρεψεν οπίσω την σκιάν δέκα βαθμούς, διά των βαθμών τους οποίους κατέβη διά των βαθμών του Άχαζ.
\par 12 Κατ' εκείνον τον καιρόν Βερωδάχ-βαλαδάν, ο υιός του Βαλαδάν, βασιλεύς της Βαβυλώνος, έστειλεν επιστολάς και δώρον προς τον Εζεκίαν· διότι ήκουσεν ότι ηρρώστησεν ο Εζεκίας.
\par 13 Και ηκροάσθη αυτούς ο Εζεκίας και έδειξεν εις αυτούς πάντα τον οίκον των πολυτίμων αυτού πραγμάτων, τον άργυρον και τον χρυσόν και τα αρώματα και τα πολύτιμα μύρα και όλην την οπλοθήκην αυτού και παν ό,τι ευρίσκετο εν τοις θησαυροίς αυτού· δεν ήτο ουδέν εν τω οίκω αυτού ουδέ υπό πάσαν την εξουσίαν αυτού, το οποίον ο Εζεκίας δεν έδειξεν εις αυτούς.
\par 14 Τότε ήλθεν Ησαΐας ο προφήτης προς τον βασιλέα Εζεκίαν και είπε προς αυτόν, Τι λέγουσιν ούτοι οι άνθρωποι; και πόθεν ήλθον προς σε; Και είπεν ο Εζεκίας, Από γης μακράς έρχονται, από Βαβυλώνος.
\par 15 Ο δε είπε, Τι είδον εν τω οίκω σου; Και απεκρίθη ο Εζεκίας, Είδον παν ό,τι είναι εν τω οίκω μου· δεν είναι ουδέν εν τοις θησαυροίς μου, το οποίον δεν έδειξα εις αυτούς.
\par 16 Τότε είπεν ο Ησαΐας προς τον Εζεκίαν, Άκουσον τον λόγον του Κυρίου·
\par 17 Ιδού, έρχονται ημέραι, καθ' ας παν ό,τι είναι εν τω οίκω σου και ό,τι οι πατέρες σου εναπεταμίευσαν μέχρι της ημέρας ταύτης, θέλει μετακομισθή εις την Βαβυλώνα· δεν θέλει μείνει ουδέν, λέγει Κύριος·
\par 18 και εκ των υιών σου οίτινες θέλουσιν εξέλθει από σου, τους οποίους θέλεις γεννήσει, θέλουσι λάβει και θέλουσι γείνει ευνούχοι εν τω παλατίω του βασιλέως της Βαβυλώνος.
\par 19 Τότε είπεν ο Εζεκίας προς τον Ησαΐαν, Καλός ο λόγος του Κυρίου, τον οποίον ελάλησας. Είπεν έτι, Δεν θέλει είσθαι ειρήνη και ασφάλεια εν ταις ημέραις μου;
\par 20 Αι δε λοιπαί πράξεις του Εζεκίου και πάντα τα κατορθώματα αυτού, και τίνι τρόπω έκαμε το υδροστάσιον και το υδραγωγείον και έφερε το ύδωρ εις την πόλιν, δεν είναι γεγραμμένα εν τω βιβλίω των χρονικών των βασιλέων του Ιούδα;
\par 21 Και εκοιμήθη ο Εζεκίας μετά των πατέρων αυτού· εβασίλευσε δε αντ' αυτού Μανασσής ο υιός αυτού.

\chapter{21}

\par 1 Δώδεκα ετών ηλικίας ήτο ο Μανασσής, ότε εβασίλευσεν· εβασίλευσε δε πεντήκοντα πέντε έτη εν Ιερουσαλήμ· το δε όνομα της μητρός αυτού ήτο Εφσιβά.
\par 2 Και έπραξε πονηρά ενώπιον του Κυρίου, κατά τα βδελύγματα των εθνών, τα οποία εξεδίωξεν ο Κύριος απ' έμπροσθεν των υιών Ισραήλ.
\par 3 Και ανωκοδόμησε τους υψηλούς τόπους, τους οποίους Εζεκίας ο πατήρ αυτού κατέστρεψε· και ανήγειρε θυσιαστήρια εις τον Βάαλ και έκαμεν άλσος, καθώς έκαμεν Αχαάβ ο βασιλεύς του Ισραήλ· και προσεκύνησε πάσαν την στρατιάν του ουρανού και ελάτρευσεν αυτά.
\par 4 Και ωκοδόμησε θυσιαστήρια εν τω οίκω του Κυρίου, περί του οποίου ο Κύριος είπεν, Εν Ιερουσαλήμ θέλω θέσει το όνομά μου.
\par 5 Και ωκοδόμησε θυσιαστήρια εις πάσαν την στρατιάν του ουρανού, εντός των δύο αυλών του οίκου του Κυρίου.
\par 6 Και διεβίβασε τον υιόν αυτού διά του πυρός, και προεμάντευε καιρούς, και έκαμνεν οιωνισμούς, και εσύστησεν ανταποκριτάς δαιμονίων και επαοιδούς· έπραξε πολλά πονηρά ενώπιον του Κυρίου, διά να παροργίση αυτόν.
\par 7 Και έστησε το γλυπτόν του άλσους, το οποίον έκαμεν, εν τω οίκω, περί του οποίου ο Κύριος είπε προς τον Δαβίδ και προς τον Σολομώντα τον υιόν αυτού, Εν τω οίκω τούτω και εν Ιερουσαλήμ, την οποίαν εξέλεξα από πασών των φυλών του Ισραήλ, θέλω θέσει το όνομά μου εις τον αιώνα·
\par 8 και δεν θέλω μετασαλεύσει τον πόδα του Ισραήλ από της γης, την οποίαν έδωκα εις τους πατέρας αυτών· εάν μόνον προσέξωσι να κάμνωσι κατά πάντα όσα προσέταξα εις αυτούς, και κατά πάντα τον νόμον τον οποίον ο δούλός μου Μωϋσής προσέταξεν εις αυτούς.
\par 9 Πλην δεν υπήκουσαν· και επλάνησεν αυτούς ο Μανασσής, ώστε να πράττωσι πονηρότερα παρά τα έθνη, τα οποία ο Κύριος ηφάνισεν απ' έμπροσθεν των υιών του Ισραήλ.
\par 10 Και ελάλησε Κύριος διά χειρός των δούλων αυτού των προφητών, λέγων,
\par 11 Επειδή Μανασσής ο βασιλεύς του Ιούδα έπραξε τα βδελύγματα ταύτα, πονηρότερα υπέρ πάντα όσα έπραξαν οι Αμορραίοι οι προ αυτού, και έκαμεν έτι τον Ιούδαν να αμαρτήση διά των ειδώλων αυτού,
\par 12 διά ταύτα ούτω λέγει Κύριος ο Θεός του Ισραήλ· Ιδού, εγώ επιφέρω κακόν επί την Ιερουσαλήμ και επί τον Ιούδαν, ώστε παντός ακούοντος περί αυτού θέλουσιν ηχήσει αμφότερα τα ώτα αυτού·
\par 13 και θέλω εκτείνει επί την Ιερουσαλήμ το σχοινίον της Σαμαρείας και την στάθμην του οίκου του Αχαάβ· και θέλω σπογγίσει την Ιερουσαλήμ, καθώς σπογγίζει τις τρυβλίον και σπογγίσας στρέφει άνω κάτω·
\par 14 και θέλω εγκαταλείψει το υπόλοιπον της κληρονομίας μου και παραδώσει αυτούς εις την χείρα των εχθρών αυτών· και θέλουσιν είσθαι εις διαρπαγήν και λεηλασίαν εις πάντας τους εχθρούς αυτών·
\par 15 διότι έπραξαν πονηρά ενώπιόν μου και με παρώργισαν, αφ' ης ημέρας οι πατέρες αυτών εξήλθον εξ Αιγύπτου, έως της ημέρας ταύτης.
\par 16 Και αίμα έτι αθώον έχυσεν ο Μανασσής πολύ σφόδρα, εωσού ενέπλησε την Ιερουσαλήμ απ' άκρου έως άκρου· εκτός της αμαρτίας αυτού, διά της οποίας έκαμε τον Ιούδαν να αμαρτήση, πράξας πονηρά ενώπιον του Κυρίου.
\par 17 Αι δε λοιπαί πράξεις του Μανασσή και πάντα όσα έκαμε και η αμαρτία αυτού, την οποίαν ημάρτησε, δεν είναι γεγραμμένα εν τω βιβλίω των χρονικών των βασιλέων του Ιούδα;
\par 18 Εκοιμήθη δε ο Μανασσής μετά των πατέρων αυτού, και ετάφη εν τω κήπω του οίκου αυτού εν τω κήπω Ουζά· και εβασίλευσεν αντ' αυτού Αμών ο υιός αυτού.
\par 19 Εικοσιδύο ετών ηλικίας ήτο ο Αμών ότε εβασίλευσε, και εβασίλευσε δύο έτη εν Ιερουσαλήμ· το δε όνομα της μητρός αυτού ήτο Μεσουλλεμέθ, θυγάτηρ του Αρούς από Ιοτεβά.
\par 20 Και έπραξε πονηρά ενώπιον του Κυρίου, καθώς έπραξε Μανασσής ο πατήρ αυτού.
\par 21 Και περιεπάτησεν εις πάσας τας οδούς, εις τας οποίας περιεπάτησεν ο πατήρ αυτού· και ελάτρευσε τα είδωλα, τα οποία ελάτρευσεν ο πατήρ αυτού, και προσεκύνησεν αυτά.
\par 22 Και εγκατέλιπε Κύριον τον Θεόν των πατέρων αυτού και δεν περιεπάτησεν εις την οδόν του Κυρίου.
\par 23 Και συνώμοσαν οι δούλοι του Αμών εναντίον αυτού· και εθανάτωσαν τον βασιλέα εν τω οίκω αυτού.
\par 24 Ο δε λαός της γης εθανάτωσε πάντας τους συνωμόσαντας κατά του βασιλέως Αμών· και έκαμεν ο λαός της γης Ιωσίαν τον υιόν αυτού βασιλέα αντ' αυτού.
\par 25 Αι δε λοιπαί πράξεις του Αμών, όσας έπραξε, δεν είναι γεγραμμέναι εν τω βιβλίω των χρονικών των βασιλέων του Ιούδα;
\par 26 Και έθαψαν αυτόν εν τω τάφω αυτού, εν τω κήπω Ουζά· εβασίλευσε δε αντ' αυτού Ιωσίας ο υιός αυτού.

\chapter{22}

\par 1 Οκτώ ετών ηλικίας ήτο ο Ιωσίας ότε εβασίλευσε, και εβασίλευσεν εν Ιερουσαλήμ έτη τριάκοντα και έν· το δε όνομα της μητρός αυτού ήτο Ιεδιδά, θυγάτηρ του Αδαΐου, από Βοσκάθ.
\par 2 Και έπραξε το ευθές ενώπιον του Κυρίου και περιεπάτησεν εις πάσας τας οδούς Δαβίδ του πατρός αυτού, και δεν εξέκλινε δεξιά ή αριστερά.
\par 3 Και εν τω δεκάτω ογδόω έτει του βασιλέως Ιωσία, εξαπέστειλεν ο βασιλεύς τον Σαφάν, υιόν του Αζαλίου υιού του Μεσουλλάμ, τον γραμματέα, εις τον οίκον του Κυρίου, λέγων,
\par 4 Ανάβα προς Χελκίαν τον ιερέα τον μέγαν, και ειπέ να απαριθμήση το αργύριον το εισαχθέν εις τον οίκον του Κυρίου, το οποίον οι φυλάττοντες την θύραν εσύναξαν παρά του λαού·
\par 5 και ας παραδώσωσιν αυτό εις την χείρα των ποιούντων τα έργα, των επιστατούντων εν τω οίκω του Κυρίου· οι δε ας δώσωσιν αυτό εις τους εργαζομένους τα έργα τα εν τω οίκω του Κυρίου, διά να επισκευάσωσι τα χαλάσματα του οίκου,
\par 6 εις τους ξυλουργούς και οικοδόμους και τοιχοποιούς, και διά να αγοράσωσι ξύλα και λίθους λατομητούς, διά να επισκευάσωσι τον οίκον.
\par 7 πλην δεν εγίνετο μετ' αυτών ουδείς λογαριασμός περί του διδομένου εις τας χείρας αυτών αργυρίου, διότι ειργάζοντο εν πίστει.
\par 8 Είπε δε Χελκίας ο ιερεύς ο μέγας προς Σαφάν τον γραμματέα, Εύρηκα το βιβλίον του νόμου εν τω οίκω του Κυρίου. Και έδωκεν ο Χελκίας το βιβλίον εις τον Σαφάν, και ανέγνωσεν αυτό.
\par 9 Και ήλθε Σαφάν ο γραμματεύς προς τον βασιλέα και ανέφερε λόγον προς τον βασιλέα και είπεν, Οι δούλοί σου εσύναξαν το αργύριον το ευρεθέν εν τω οίκω, και παρέδωκαν αυτό εις την χείρα των ποιούντων τα έργα, των επιστατούντων εις τον οίκον του Κυρίου.
\par 10 Και απήγγειλε Σαφάν ο γραμματεύς προς τον βασιλέα, λέγων, Χελκίας ο ιερεύς έδωκεν εις εμέ βιβλίον. Και ανέγνωσεν αυτό ο Σαφάν ενώπιον του βασιλέως.
\par 11 Και ως ήκουσεν ο βασιλεύς τους λόγους του βιβλίου του νόμου, διέσχισε τα ιμάτια αυτού.
\par 12 Και προσέταξεν ο βασιλεύς Χελκίαν τον ιερέα και Αχικάμ τον υιόν του Σαφάν και Αχβώρ τον υιόν του Μιχαΐου και Σαφάν τον γραμματέα και Ασαΐαν τον δούλον του βασιλέως, λέγων,
\par 13 Υπάγετε, ερωτήσατε τον Κύριον περί εμού και περί του λαού και περί παντός του Ιούδα, περί των λόγων του βιβλίου τούτου, το οποίον ευρέθη· διότι μεγάλη είναι η οργή του Κυρίου η εξαφθείσα εναντίον ημών, επειδή οι πατέρες ημών δεν υπήκουσαν εις τους λόγους του βιβλίου τούτου, ώστε να πράττωσι κατά πάντα τα γεγραμμένα περί ημών.
\par 14 Τότε Χελκίας ο ιερεύς, και Αχικάμ και Αχβώρ και Σαφάν και Ασαΐας, υπήγαν εις την Όλδαν την προφήτιν, την γυναίκα του Σαλλούμ υιού του Τικβά, υιού του Αράς, του ιματιοφύλακος· κατώκει δε αύτη εν Ιερουσαλήμ, κατά το Μισνέ· και ώμίλησαν μετ' αυτής.
\par 15 Και είπε προς αυτούς, Ούτω λέγει Κύριος ο Θεός του Ισραήλ· Είπατε προς τον άνθρωπον, όστις σας απέστειλε προς εμέ,
\par 16 Ούτω λέγει Κύριος· Ιδού εγώ επιφέρω κακά επί τον τόπον τούτον και επί τους κατοίκους αυτού, πάντας τους λόγους του βιβλίου, το οποίον ο βασιλεύς του Ιούδα ανέγνωσε·
\par 17 διότι με εγκατέλιπον και εθυμίασαν εις άλλους θεούς, διά να με παροργίσωσι διά πάντων των έργων των χειρών αυτών· διά τούτο θέλει εκχυθή ο θυμός μου επί τον τόπον τούτον και δεν θέλει σβεσθή.
\par 18 Προς τον βασιλέα όμως του Ιούδα, όστις σας απέστειλε να ερωτήσητε τον Κύριον, ούτω θέλετε ειπεί προς αυτόν· Ούτω λέγει Κύριος ο Θεός του Ισραήλ· Περί των λόγων τους οποίους ήκουσας,
\par 19 επειδή η καρδία σου ηπαλύνθη, και εταπεινώθης ενώπιον του Κυρίου, ότε ήκουσας όσα ελάλησα εναντίον του τόπου τούτου και εναντίον των κατοίκων αυτού, ότι θέλουσι κατασταθή ερήμωσις και κατάρα, και διέσχισας τα ιμάτιά σου και έκλαυσας ενώπιόν μου· διά τούτο και εγώ επήκουσα, λέγει Κύριος·
\par 20 ιδού λοιπόν, εγώ θέλω σε συνάξει εις τους πατέρας σου, και θέλεις συναχθή εις τον τάφον σου εν ειρήνη· και δεν θέλουσιν ιδεί οι οφθαλμοί σου πάντα τα κακά, τα οποία εγώ επιφέρω επί τον τόπον τούτον. Και έφεραν απόκρισιν προς τον βασιλέα.

\chapter{23}

\par 1 Και απέστειλεν ο βασιλεύς, και συνήγαγον προς αυτόν πάντας τους πρεσβυτέρους του Ιούδα και της Ιερουσαλήμ.
\par 2 Και ανέβη ο βασιλεύς εις τον οίκον του Κυρίου, και πάντες οι άνδρες Ιούδα και πάντες οι κάτοικοι της Ιερουσαλήμ μετ' αυτού, και οι ιερείς και οι προφήται και πας ο λαός, από μικρού έως μεγάλου· και ανέγνωσεν εις επήκοον αυτών πάντας τους λόγους του βιβλίου της διαθήκης, το οποίον ευρέθη εν τω οίκω του Κυρίου.
\par 3 Και σταθείς ο βασιλεύς πλησίον του στύλου, έκαμε την διαθήκην ενώπιον του Κυρίου, να περιπατή κατόπιν του Κυρίου και να φυλάττη τας εντολάς αυτού και τα μαρτύρια αυτού και τα διατάγματα αυτού εξ όλης καρδίας και εξ όλης ψυχής, ώστε να εκτελώσι τους λόγους της διαθήκης ταύτης, τους γεγραμμένους εν τω βιβλίω τούτω. Και πας ο λαός εστάθη εις την διαθήκην.
\par 4 Και προσέταξεν ο βασιλεύς Χελκίαν τον ιερέα τον μέγαν και τους ιερείς της δευτέρας τάξεως και τους φύλακας της πύλης, να εκβάλωσιν εκ του ναού του Κυρίου πάντα τα σκεύη, τα κατεσκευασμένα διά τον Βάαλ και διά το άλσος και διά πάσαν την στρατιάν του ουρανού· και κατέκαυσεν αυτά έξω της Ιερουσαλήμ εν τοις αγροίς Κέδρων, και μετεκόμισαν την στάκτην αυτών εις Βαιθήλ.
\par 5 Και κατήργησε τους ειδωλολάτρας ιερείς, τους οποίους οι βασιλείς του Ιούδα διώρισαν να θυμιάζωσιν εν τοις υψηλοίς τόποις, εν ταις πόλεσι του Ιούδα και εν τοις πέριξ της Ιερουσαλήμ· και τους θυμιάζοντας εις τον Βάαλ, εις τον ήλιον και εις την σελήνην και εις τα ζώδια και εις πάσαν την στρατιάν του ουρανού.
\par 6 Και εξέβαλε το άλσος εκ του οίκου του Κυρίου, έξω της Ιερουσαλήμ, εις τον χείμαρρον Κέδρων, και κατέκαυσεν αυτό εν τω χειμάρρω Κέδρων και κατελέπτυνεν αυτό εις σκόνην, και έρριψε την σκόνην αυτού επί των μνημάτων των υιών του όχλου.
\par 7 Και κατεκρήμνισε τους οίκους των σοδομιτών, τους εν τω οίκω του Κυρίου, όπου αι γυναίκες ύφαινον παραπετάσματα διά το άλσος.
\par 8 Και έφερε πάντας τους ιερείς εκ των πόλεων του Ιούδα, και εβεβήλωσε τους υψηλούς τόπους, εις τους οποίους οι ιερείς εθυμίαζον, από Γεβά έως Βηρ-σαβεέ, και κατεκρήμνισε τους υψηλούς τόπους των πυλών, των εν τη εισόδω της πύλης Ιησού του άρχοντος της πόλεως, τη εξ αριστερών της πύλης της πόλεως.
\par 9 Πλην οι ιερείς των υψηλών τόπων δεν ανέβησαν προς το θυσιαστήριον του Κυρίου εν Ιερουσαλήμ, αλλ' έτρωγον άζυμα μεταξύ των αδελφών αυτών.
\par 10 Και εβεβήλωσε τον Τοφέθ, τον εν τη φάραγγι των υιών του Εννόμ· ώστε να μη δύναται μηδείς να διαβιβάση τον υιόν αυτού ή την θυγατέρα αυτού διά του πυρός εις τον Μολόχ.
\par 11 Και αφήρεσε τους ίππους, τους οποίους οι βασιλείς του Ιούδα έστησαν εις τον ήλιον, κατά την είσοδον του οίκου του Κυρίου, πλησίον του οικήματος του Νάθαν-μελέχ του ευνούχου, το οποίον ήτο εν Φαρουρείμ, και κατέκαυσεν εν πυρί τας αμάξας του ηλίου.
\par 12 Και τα θυσιαστήρια τα επί του δώματος του υπερώου του Άχαζ, τα οποία έκαμον οι βασιλείς του Ιούδα, και τα θυσιαστήρια, τα οποία έκαμεν ο Μανασσής εν ταις δύο αυλαίς του οίκου του Κυρίου, κατέστρεψεν αυτά ο βασιλεύς και κατεκρήμνισεν εκείθεν και έρριψε την σκόνην αυτών εις τον χείμαρρον Κέδρων.
\par 13 Και τους υψηλούς τόπους τους κατά πρόσωπον της Ιερουσαλήμ, τους εν δεξιά του όρους της διαφθοράς, τους οποίους ωκοδόμησε Σολομών ο βασιλεύς του Ισραήλ διά την Αστάρτην το βδέλυγμα των Σιδωνίων, και διά τον Χεμώς το βδέλυγμα των Μωαβιτών, και διά τον Μελχώμ το βδέλυγμα των υιών Αμμών, εβεβήλωσεν ο βασιλεύς.
\par 14 Και συνέτριψε τα αγάλματα και κατέκοψε τα άλση και εγέμισε τους τόπους αυτών από οστά ανθρώπων.
\par 15 Και το θυσιαστήριον το εν Βαιθήλ και τον υψηλόν τόπον, τον οποίον έκαμεν Ιεροβοάμ ο υιός του Ναβάτ, όστις έκαμε τον Ισραήλ να αμαρτήση, και εκείνο το θυσιαστήριον και τον υψηλόν τόπον κατεχάλασε και κατέκαυσε τον υψηλόν τόπον και ελέπτυνεν αυτά εις σκόνην και το άλσος κατέκαυσεν.
\par 16 Ότε δε ο Ιωσίας εστράφη και είδε τους τάφους τους εκεί εν τω όρει, έστειλε και έλαβε τα οστά εκ των τάφων και κατέκαυσεν αυτά επί του θυσιαστηρίου, και εβεβήλωσεν αυτό· κατά τον λόγον του Κυρίου, τον οποίον εκήρυξεν ο άνθρωπος του Θεού, ο λαλήσας τους λόγους τούτους.
\par 17 Τότε είπε, Τι μνημείον είναι εκείνο το οποίον εγώ βλέπω; Και οι άνδρες της πόλεως είπον προς αυτόν, Ο τάφος του ανθρώπου του Θεού, όστις ήλθεν εξ Ιούδα και εκήρυξε τα πράγματα ταύτα, τα οποία συ έκαμες κατά του θυσιαστηρίου της Βαιθήλ.
\par 18 Και είπεν, Αφήσατε αυτόν· ας μη κινήση μηδείς τα οστά αυτού. Και διέσωσαν τα οστά αυτού, μετά των οστέων του προφήτου του ελθόντος εκ Σαμαρείας.
\par 19 Και πάντας έτι τους οίκους των υψηλών τόπων τους εν ταις πόλεσι της Σαμαρείας, τους οποίους έκαμον οι βασιλείς του Ισραήλ διά να παροργίσωσι τον Κύριον, ο Ιωσίας αφήρεσε, και έκαμεν εις αυτούς κατά πάντα τα έργα όσα έκαμεν εις Βαιθήλ.
\par 20 Και εθυσίασεν επί των θυσιαστηρίων πάντας τους ιερείς των υψηλών τόπων τους εκεί, και κατέκαυσεν επ' αυτών τα οστά των ανθρώπων και επέστρεψεν εις Ιερουσαλήμ.
\par 21 Τότε προσέταξεν ο βασιλεύς εις πάντα τον λαόν, λέγων, Κάμετε το πάσχα εις Κύριον τον Θεόν σας, κατά το γεγραμμένον εν τω βιβλίω τούτω της διαθήκης.
\par 22 Βεβαίως δεν έγεινε τοιούτον πάσχα από των ημερών των κριτών οίτινες έκρινον τον Ισραήλ, ουδέ εν πάσαις ταις ημέραις των βασιλέων του Ισραήλ και των βασιλέων του Ιούδα,
\par 23 οποίον έγεινε προς τον Κύριον εν Ιερουσαλήμ το πάσχα τούτο, κατά το δέκατον όγδοον έτος του βασιλέως Ιωσίου.
\par 24 Αφήρεσεν έτι ο Ιωσίας και τους ανταποκριτάς των δαιμονίων και τους μάντεις και τα ξόανα και τα είδωλα και πάντα τα βδελύγματα τα οποία εφαίνοντο εν τη γη του Ιούδα και εν Ιερουσαλήμ, διά να εκτελέση τους λόγους του νόμου τους γεγραμμένους εν τω βιβλίω, το οποίον εύρηκε Χελκίας ο ιερεύς εν τω οίκω του Κυρίου.
\par 25 Και όμοιος αυτού δεν υπήρξε προ αυτού βασιλεύς, όστις επέστρεψεν εις τον Κύριον εξ όλης αυτού της καρδίας και εξ όλης αυτού της ψυχής και εξ όλης αυτού της δυνάμεως, κατά πάντα τον νόμον του Μωϋσέως· ουδέ ηγέρθη μετ' αυτόν όμοιος αυτού.
\par 26 Πλην ο Κύριος δεν εστράφη από του θυμού της οργής αυτού της μεγάλης, καθ' ον εξήφθη η οργή αυτού κατά του Ιούδα, εξ αιτίας πάντων των παροργισμών, διά των οποίων παρώργισεν αυτόν ο Μανασσής.
\par 27 Και είπε Κύριος, Και τον Ιούδαν θέλω εκβάλει απ' έμπροσθέν μου, καθώς εξέβαλον τον Ισραήλ, και θέλω απορρίψει την πόλιν ταύτην, την Ιερουσαλήμ, την οποίαν εξέλεξα, και τον οίκον περί του οποίου είπα, ο όνομά μου θέλει είσθαι εκεί.
\par 28 Αι δε λοιπαί πράξεις του Ιωσίου και πάντα όσα έπραξε, δεν είναι γεγραμμένα εν τω βιβλίω των χρονικών των βασιλέων του Ιούδα;
\par 29 Εν ταις ημέραις αυτού ανέβη ο Φαραώ-νεχαώ, βασιλεύς της Αιγύπτου, κατά του βασιλέως της Ασσυρίας επί τον ποταμόν Ευφράτην. Και υπήγεν ο βασιλεύς Ιωσίας εις απάντησιν αυτού· και εκείνος, ως είδεν αυτόν, εθανάτωσεν αυτόν εν Μεγιδδώ.
\par 30 Και οι δούλοι αυτού επεβίβασαν αυτόν νεκρόν εις άμαξαν από Μεγιδδώ, και έφεραν αυτόν εις Ιερουσαλήμ, και έθαψαν αυτόν εν τω τάφω αυτού. Ο δε λαός της γης έλαβε τον Ιωάχαζ υιόν του Ιωσίου, και έχρισαν αυτόν και έκαμον αυτόν βασιλέα αντί του πατρός αυτού.
\par 31 Εικοσιτριών ετών ηλικίας ήτο ο Ιωάχαζ, ότε εβασίλευσε· και εβασίλευσε τρεις μήνας εν Ιερουσαλήμ. Το δε όνομα της μητρός αυτού ήτο Αμουτάλ, θυγάτηρ του Ιερεμίου από Λιβνά.
\par 32 Και έπραξε πονηρά ενώπιον του Κυρίου, κατά πάντα όσα έπραξαν οι πατέρες αυτού.
\par 33 Και εφυλάκισεν αυτόν ο Φαραώ-νεχαώ εν Ριβλά εν τη γη Αιμάθ, διά να μη βασιλεύη εν Ιερουσαλήμ· και κατεδίκασε την γην εις πρόστιμον εκατόν ταλάντων αργυρίου και ενός ταλάντου χρυσίου.
\par 34 Και έκαμεν ο Φαραώ-νεχαώ τον Ελιακείμ τον υιόν του Ιωσίου βασιλέα αντί Ιωσίου του πατρός αυτού, και μετήλλαξε το όνομα αυτού εις Ιωακείμ· τον δε Ιωάχαζ έλαβε και έφερεν εις Αίγυπτον, και απέθανεν εκεί.
\par 35 Ο δε Ιωακείμ έδωκεν εις τον Φαραώ το αργύριον και το χρυσίον· εφορολόγησεν όμως την γην, διά να δώση το αργύριον κατά την προσταγήν του Φαραώ· ο λαός της γης συνεισέφερε το αργύριον και το χρυσίον, έκαστος κατά την εκτίμησιν αυτού, διά να δώση εις τον Φαραώ-νεχαώ.
\par 36 Εικοσιπέντε ετών ηλικίας ήτο ο Ιωακείμ, ότε εβασίλευσεν· εβασίλευσε δε ένδεκα έτη εν Ιερουσαλήμ· το δε όνομα της μητρός αυτού ήτο Ζεβουδά, θυγάτηρ του Φεδαίου από Ρουμά.
\par 37 Και έπραξε πονηρά ενώπιον του Κυρίου, κατά πάντα όσα έπραξαν οι πατέρες αυτού.

\chapter{24}

\par 1 Εν ταις ημέραις αυτού ανέβη Ναβουχοδονόσορ ο βασιλεύς της Βαβυλώνος, και ο Ιωακείμ έγεινε δούλος αυτού τρία έτη· έπειτα εστράφη και απεστάτησε κατ' αυτού.
\par 2 Και απέστειλεν ο Κύριος εναντίον αυτού τα τάγματα των Χαλδαίων και τα τάγματα των Συρίων και τα τάγματα των Μωαβιτών και τα τάγματα των υιών Αμμών, και απέστειλεν αυτούς εναντίον του Ιούδα, διά να καταστρέψωσιν αυτόν· κατά τον λόγον του Κυρίου, τον οποίον ελάλησε διά χειρός των δούλων αυτού των προφητών.
\par 3 Τω όντι κατά προσταγήν του Κυρίου έγεινε τούτο εις τον Ιούδαν, διά να αποβάλη αυτόν από προσώπου αυτού, διά τας αμαρτίας του Μανασσή, κατά πάντα όσα έπραξε·
\par 4 και έτι διά το αθώον αίμα το οποίον έχυσε, διότι εγέμισε την Ιερουσαλήμ αίμα αθώον· και δεν ηθέλησεν ο Κύριος να συγχωρήση αυτόν.
\par 5 Αι δε λοιπαί πράξεις του Ιωακείμ και πάντα όσα έπραξε, δεν είναι γεγραμμένα εν τω βιβλίω των χρονικών των βασιλέων του Ιούδα;
\par 6 Και εκοιμήθη ο Ιωακείμ μετά των πατέρων αυτού, και εβασίλευσεν αντ' αυτού Ιωαχείν ο υιός αυτού.
\par 7 Ο δε βασιλεύς της Αιγύπτου δεν εξήλθε πλέον εκ της γης αυτού· διότι ο βασιλεύς της Βαβυλώνος έλαβεν, από του ποταμού της Αιγύπτου μέχρι του ποταμού Ευφράτου, πάντα όσα ήσαν του βασιλέως της Αιγύπτου.
\par 8 Δεκαοκτώ ετών ηλικίας ήτο ο Ιωαχείν, ότε εβασίλευσε· και εβασίλευσε τρεις μήνας εν Ιερουσαλήμ· το δε όνομα της μητρός αυτού ήτο Νεουσθά, θυγάτηρ του Ελναθάν εξ Ιερουσαλήμ.
\par 9 Και έπραξε πονηρά ενώπιον του Κυρίου, κατά πάντα όσα έπραξεν ο πατήρ αυτού.
\par 10 Κατ' εκείνον τον καιρόν ανέβησαν οι δούλοι του Ναβουχοδονόσορ βασιλέως της Βαβυλώνος επί την Ιερουσαλήμ και επολιόρκησαν την πόλιν.
\par 11 Και ήλθε Ναβουχοδονόσορ ο βασιλεύς της Βαβυλώνος κατά της πόλεως, και οι δούλοι αυτού επολιόρκουν αυτήν.
\par 12 Και εξήλθεν ο Ιωαχείν βασιλεύς του Ιούδα προς τον βασιλέα της Βαβυλώνος, αυτός και η μήτηρ αυτού και οι δούλοι αυτού και οι άρχοντες αυτού και οι ευνούχοι αυτού· και συνέλαβεν αυτόν ο βασιλεύς της Βαβυλώνος, εν τω ογδόω έτει της βασιλείας αυτού.
\par 13 Και εξήγαγεν εκείθεν πάντας τους θησαυρούς του οίκου του Κυρίου και τους θησαυρούς του οίκου του βασιλέως, και κατέκοψε πάντα τα σκεύη τα χρυσά, τα οποία έκαμε Σολομών ο βασιλεύς του Ισραήλ εν τω ναώ, του Κυρίου, καθώς ελάλησεν ο Κύριος.
\par 14 Και μετώκισε πάσαν την Ιερουσαλήμ και πάντας τους άρχοντας και πάντας τους δυνατούς πολεμιστάς, δέκα χιλιάδας αιχμαλώτων, και πάντας τους ξυλουργούς και σιδηρουργούς· δεν έμεινεν ειμή το πτωχότερον μέρος του λαού της γης.
\par 15 Και μετώκισε τον Ιωαχείν εις την Βαβυλώνα· και την μητέρα του βασιλέως και τας γυναίκας του βασιλέως και τους ευνούχους αυτού και τους δυνατούς της γης έφερεν αιχμαλώτους εξ Ιερουσαλήμ εις την Βαβυλώνα·
\par 16 και πάντας τους πολεμιστάς, επτά χιλιάδας, και τους ξυλουργούς και τους σιδηρουργούς, χιλίους, πάντας δυνατούς και επιτηδείους εις πόλεμον· και μετώκισεν αυτούς εις Βαβυλώνα ο βασιλεύς της Βαβυλώνος.
\par 17 Και έκαμεν ο βασιλεύς της Βαβυλώνος βασιλέα, αντ' αυτού, Ματθανίαν τον αδελφόν του πατρός αυτού, και μετήλλαξε το όνομα αυτού εις Σεδεκίαν.
\par 18 Ενός και είκοσι ετών ηλικίας ήτο ο Σεδεκίας, ότε εβασίλευσεν· εβασίλευσε δε ένδεκα έτη εν Ιερουσαλήμ· το δε όνομα της μητρός αυτού ήτο Αμουτάλ, θυγάτηρ του Ιερεμίου από Λιβνά.
\par 19 Και έπραξε πονηρά ενώπιον του Κυρίου, κατά πάντα όσα έπραξεν ο Ιωακείμ·
\par 20 διότι εξ οργής του Κυρίου κατά της Ιερουσαλήμ και του Ιούδα, εωσού απέρριψεν αυτούς από προσώπου αυτού, έγεινε να αποστατήση ο Σεδεκίας κατά του βασιλέως της Βαβυλώνος.

\chapter{25}

\par 1 Και εν τω ενάτω έτει της βασιλείας αυτού, τον δέκατον μήνα, την δεκάτην του μηνός, ήλθε Ναβουχοδονόσορ ο βασιλεύς της Βαβυλώνος, αυτός και παν το στράτευμα αυτού, κατά της Ιερουσαλήμ, και εστρατοπέδευσεν εναντίον αυτής· και ωκοδόμησαν περιτειχίσματα εναντίον αυτής κύκλω.
\par 2 Και η πόλις επολιορκείτο μέχρι του ενδεκάτου έτους του βασιλέως Σεδεκίου.
\par 3 Και την ενάτην του τετάρτου μηνός η πείνα υπερίσχυσεν εν τη πόλει, και δεν ήτο άρτος διά τον λαόν του τόπου.
\par 4 Και εξεπορθήθη η πόλις, και πάντες οι άνδρες του πολέμου έφυγον την νύκτα, διά της οδού της πύλης της μεταξύ των δύο τειχών, της πλησίον του βασιλικού κήπου· οι δε Χαλδαίοι ήσαν πλησίον της πόλεως κύκλω· και ο βασιλεύς υπήγε κατά την οδόν της πεδιάδος.
\par 5 Το δε στράτευμα των Χαλδαίων κατεδίωξεν οπίσω του βασιλέως, και έφθασαν αυτόν εις τας πεδιάδας της Ιεριχώ· και παν το στράτευμα αυτού διεσκορπίσθη από πλησίον αυτού.
\par 6 Και συνέλαβον τον βασιλέα και ανήγαγον αυτόν προς τον βασιλέα της Βαβυλώνος εις Ριβλά· και επρόφεραν καταδίκην επ' αυτόν.
\par 7 Και έσφαξαν τους υιούς του Σεδεκίου έμπροσθεν των οφθαλμών αυτού, και εξετύφλωσαν τους οφθαλμούς του Σεδεκίου, και δέσαντες αυτόν με δύο χαλκίνας αλύσεις, έφεραν αυτόν εις Βαβυλώνα.
\par 8 Εν δε τω πέμπτω μηνί, την εβδόμην του μηνός, του δεκάτου ενάτου έτους του Ναβουχοδονόσορ, βασιλέως της Βαβυλώνος, ήλθεν επί Ιερουσαλήμ Νεβουζαραδάν ο αρχισωματοφύλαξ, ο δούλος του βασιλέως της Βαβυλώνος·
\par 9 και κατέκαυσε τον οίκον του Κυρίου και τον οίκον του βασιλέως και πάντας τους οίκους της Ιερουσαλήμ, και πάντα μέγαν οίκον κατέκαυσεν εν πυρί.
\par 10 Και παν το στράτευμα των Χαλδαίων, το μετά του αρχισωματοφύλακος, κατεκρήμνισε τα τείχη της Ιερουσαλήμ κύκλω.
\par 11 Το δε υπόλοιπον του λαού, το εναπολειφθέν εν τη πόλει, και τους φυγόντας, οίτινες προσέφυγον προς τον βασιλέα της Βαβυλώνος, και το εναπολειφθέν του πλήθους μετώκισεν ο Νεβουζαραδάν ο αρχισωματοφύλαξ.
\par 12 Εκ των πτωχών όμως της γης αφήκεν ο αρχισωματοφύλαξ, διά αμπελουργούς και γεωργούς.
\par 13 Και τους στύλους τους χαλκίνους, τους εν τω οίκω του Κυρίου, και τας βάσεις και την χαλκίνην θάλασσαν την εν τω οίκω του Κυρίου, οι Χαλδαίοι κατέκοψαν και μετεκόμισαν τον χαλκόν αυτών εις την Βαβυλώνα.
\par 14 Έλαβον δε και τους λέβητας και τα πτυάρια και τα λυχνοψάλιδα και τα θυμιατήρια και πάντα τα σκεύη τα χάλκινα, διά των οποίων εγίνετο η υπηρεσία.
\par 15 Έλαβε προσέτι ο αρχισωματοφύλαξ και τα πυροδοχεία και τας φιάλας, ό,τι ήτο χρυσούν και ό,τι αργυρούν·
\par 16 τους δύο στύλους, την μίαν θάλασσαν και τας βάσεις, τας οποίας ο Σολομών έκαμε διά τον οίκον του Κυρίου· ο χαλκός πάντων τούτων των σκευών ήτο αζύγιστος.
\par 17 Το ύψος του ενός στύλου ήτο δεκαοκτώ πηχών, και το κιονόκρανον το επ' αυτού χάλκινον. Το δε ύψος του κιονοκράνου τριών πηχών· και το δικτυωτόν και τα ρόδια επί του κιονοκράνου κύκλω ήσαν πάντα χάλκινα· τα αυτά είχε και ο δεύτερος στύλος μετά του δικτυωτού.
\par 18 Και έλαβεν ο αρχισωματοφύλαξ Σεραΐαν τον πρώτον ιερέα και Σοφονίαν τον δεύτερον ιερέα και τους τρεις θυρωρούς·
\par 19 και εκ της πόλεως έλαβεν ένα ευνούχον, όστις ήτο επιστάτης επί των ανδρών των πολεμιστών, και πέντε άνδρας εκ των παρισταμένων έμπροσθεν του βασιλέως, τους ευρεθέντας εν τη πόλει, και τον γραμματέα τον άρχοντα των στρατευμάτων, όστις έκαμνε την στρατολογίαν του λαού της γης, και εξήκοντα άνδρας εκ του λαού της γης, τους ευρεθέντας εν τη πόλει.
\par 20 Και λαβών αυτούς Νεβουζαραδάν ο αρχισωματοφύλαξ, έφερεν αυτούς προς τον βασιλέα της Βαβυλώνος εις Ριβλά.
\par 21 Και επάταξεν αυτούς ο βασιλεύς της Βαβυλώνος και εθανάτωσεν αυτούς εν Ριβλά, εν τη γη Αιμάθ. Ούτω μετωκίσθη ο Ιούδας από της γης αυτού.
\par 22 Περί δε του λαού του εναπολειφθέντος εν τη γη Ιούδα, τους οποίους Ναβουχοδονόσορ ο βασιλεύς της Βαβυλώνος αφήκεν, επί τούτους κατέστησε Γεδαλίαν τον υιόν του Αχικάμ, υιού του Σαφάν.
\par 23 Ακούσαντες δε πάντες οι άρχοντες των στρατευμάτων, αυτοί και οι άνδρες αυτών, ότι ο βασιλεύς της Βαβυλώνος κατέστησε τον Γεδαλίαν, ήλθον προς τον Γεδαλίαν εις Μισπά, και Ισμαήλ ο υιός του Νεθανίου και Ιωανάν ο υιός του Καρηά και Σεραΐας ο υιός του Τανουμέθ ο Νετωφαθίτης και Ιααζανίας, υιός Μααχαθίτου τινός, αυτοί και οι άνδρες αυτών.
\par 24 Και ώμοσεν ο Γεδαλίας προς αυτούς και προς τους άνδρας αυτών και είπε προς αυτούς, Μη φοβείσθε να ήσθε δούλοι των Χαλδαίων. Κατοικήσατε εν τη γη και δουλεύετε τον βασιλέα της Βαβυλώνος· και θέλει είσθαι καλόν εις εσάς.
\par 25 Εν δε τω εβδόμω μηνί, Ισμαήλ ο υιός του Νεθανίου, υιού του Ελισαμά, εκ του βασιλικού σπέρματος, ήλθεν, έχων μεθ' εαυτού δέκα άνδρας, και επάταξαν τον Γεδαλίαν, ώστε απέθανε, και τους Ιουδαίους και Χαλδαίους, τους όντας μετ' αυτού εν Μισπά.
\par 26 Και εσηκώθη πας ο λαός, από μικρού έως μεγάλου, και οι άρχοντες των στρατευμάτων, και ήλθον εις την Αίγυπτον· διότι εφοβήθησαν από προσώπου των Χαλδαίων.
\par 27 Εν δε τω τριακοστώ εβδόμω έτει της μετοικεσίας του Ιωαχείν βασιλέως του Ιούδα, τον δωδέκατον μήνα, την εικοστήν εβδόμην του μηνός, ο Ευείλ-μερωδάχ βασιλεύς της Βαβυλώνος, κατά το έτος καθ' ο εβασίλευσεν, ύψωσεν εκ της φυλακής την κεφαλήν του Ιωαχείν βασιλέως του Ιούδα·
\par 28 και ελάλησεν ευμενώς μετ' αυτού, και έθεσε τον θρόνον αυτού επάνωθεν του θρόνου των βασιλέων, των μετ' αυτού εν Βαβυλώνι.
\par 29 και ήλλαξε τα ιμάτια της φυλακής αυτού· και έτρωγεν άρτον πάντοτε μετ' αυτού πάσας τας ημέρας της ζωής αυτού·
\par 30 και το σιτηρέσιον αυτού ήτο παντοτεινόν σιτηρέσιον, διδόμενον εις αυτόν παρά του βασιλέως, ημερήσιος χορηγία πάσας τας ημέρας της ζωής αυτού.


\end{document}