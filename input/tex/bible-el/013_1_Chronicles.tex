\begin{document}

\title{1 Chronicles}


\chapter{1}

\par 1 Αδάμ, Σηθ, Ενώς,
\par 2 Καϊνάν, Μααλαλεήλ, Ιάρεδ,
\par 3 Ενώχ, Μαθουσάλα, Λάμεχ,
\par 4 Νώε, Σημ, Χαμ και Ιάφεθ.
\par 5 υιοί του Ιάφεθ, Γομέρ και Μαγώγ και Μαδαΐ και Ιαυάν και Θουβάλ και Μεσέχ και Θειράς·
\par 6 και υιοί του Γομέρ, Ασχενάζ και Ριφάθ και Θωγαρμά·
\par 7 και υιοί του Ιαυάν, Ελεισά και Θαρσείς, Κιττείμ και Δωδανείμ.
\par 8 Υιοί του Χαμ, Χούς και Μισραΐμ, Φούθ και Χαναάν·
\par 9 και υιοί του Χούς, Σεβά και Αβιλά και Σαβθά και Ρααμά και Σαβθεκά· και υιοί του Ρααμά, Σεβά και Δαιδάν.
\par 10 Και ο Χούς εγέννησε τον Νεβρώδ· ούτος ήρχισε να ήναι ισχυρός επί της γης.
\par 11 Και ο Μισραΐμ εγέννησε τους Λουδείμ και τους Αναμείμ και τους Λεαβείμ και τους Ναφθουχείμ,
\par 12 και τους Πατρουσείμ και τους Χασλουχείμ, εκ των οποίων εξήλθον οι Φιλισταίοι, και τους Καφθορείμ.
\par 13 Και ο Χαναάν εγέννησε τον Σιδώνα πρωτότοκον αυτού, και τον Χετταίον,
\par 14 και τον Ιεβουσαίον και τον Αμορραίον και τον Γεργεσαίον,
\par 15 και τον Ευαίον και τον Αρουκαίον και τον Ασενναίον,
\par 16 και τον Αρβάδιον και τον Σαμαραίον και τον Αμαθαίον.
\par 17 υιοί του Σημ, Ελάμ και Ασσούρ και Αρφαξάδ και Λούδ και Αράμ· και υιοί Αράμ, Ουζ και Ουλ και Γεθέρ και Μεσέχ.
\par 18 Και ο Αρφαξάδ εγέννησε τον Σαλά, και ο Σαλά εγέννησε τον Έβερ.
\par 19 Και εις τον Έβερ εγεννήθησαν δύο υιοί· το όνομα του ενός, Φαλέγ· διότι εν ταις ημέραις αυτού διεμερίσθη η γή· το δε όνομα του αδελφού αυτού, Ιοκτάν.
\par 20 Και ο Ιοκτάν εγέννησε τον Αλμωδάδ και τον Σαλέφ και τον Ασάρ-μαβέθ και τον Ιαράχ,
\par 21 και τον Αδωράμ και τον Ουζάλ και τον Δικλά,
\par 22 και τον Εβάλ και τον Αβιμαήλ και τον Σεβά
\par 23 και τον Οφείρ, και τον Αβιλά, και τον Ιωβάβ· πάντες ούτοι ήσαν οι υιοί του Ιοκτάν.
\par 24 Σημ, Αρφαξάδ, Σαλά,
\par 25 Έβερ, Φαλέγ, Ραγαύ,
\par 26 Σερούχ, Ναχώρ, Θάρα,
\par 27 Άβραμ, όστις είναι ο Αβραάμ.
\par 28 Υιοί δε του Αβραάμ, Ισαάκ και Ισμαήλ.
\par 29 Αύται είναι αι γενεαί αυτών· Ο πρωτότοκος του Ισμαήλ, Ναβαϊώθ· έπειτα Κηδάρ και Αδβεήλ και Μιβσάμ,
\par 30 Μισμά και Δουμά, Μασσά, Αδάδ και Θαιμά,
\par 31 Ιετούρ, Ναφίς και Κεδμά· ούτοι ήσαν οι υιοί του Ισμαήλ.
\par 32 Οι δε υιοί της Χεττούρας, θεραπαίνης του Αβραάμ, ούτοι· αύτη εγέννησε τον Ζεμβράν και Ιοξάν και Μαδάν και Μαδιάμ και Ιεσβώκ και Σουά· και υιοί του Ιοξάν, Σεβά και Δαιδάν·
\par 33 και υιοί του Μαδιάμ, Γεφά και Εφέρ και Ανώχ και Αβειδά και Ελδαγά· πάντες ούτοι ήσαν υιοί της Χεττούρας.
\par 34 Και εγέννησεν ο Αβραάμ τον Ισαάκ· υιοί δε του Ισαάκ, ο Ησαύ και ο Ισραήλ.
\par 35 Υιοί του Ησαύ, Ελιφάς, Ραγουήλ και Ιεούς και Ιεγλόμ και Κορέ·
\par 36 υιοί του Ελιφάς, Θαιμάν και Ωμάρ, Σωφάρ και Γοθώμ, Κενέζ και Θαμνά και Αμαλήκ.
\par 37 Υιοί του Ραγουήλ, Ναχάθ, Ζερά, Σομέ και Μοζέ.
\par 38 Και υιοί του Σηείρ, Λωτάν και Σωβάλ και Σεβεγών και Ανά και Δησών και Εσέρ και Δισάν.
\par 39 Και υιοί του Λωτάν, Χορρί και Αιμάμ· αδελφή δε του Λωτάν, Θαμνά·
\par 40 Υιοί του Σωβάλ, Αιλάν και Μαναχάθ και Εβάλ, Σεφώ και Ωνάμ· και υιοί του Σεβεγών, Αϊέ και Ανά·
\par 41 υιοί του Ανά, Δησών· και υιοί του Δησών, Αμράν και Ασβάν και Ιθράν και Χαρράν.
\par 42 Υιοί του Εσέρ, Βαλαάν και Ζααβάν και Ιακάν· υιοί του Δισάν, Ουζ και Αράν.
\par 43 Ούτοι δε ήσαν οι βασιλείς, οι βασιλεύσαντες εν τη γη Εδώμ, πριν βασιλεύση βασιλεύς επί τους υιούς Ισραήλ· Βελά, ο υιός του Βεώρ· και το όνομα της πόλεως αυτού Δεγναβά.
\par 44 Και απέθανεν ο Βελά, και εβασίλευσεν αντ' αυτού Ιωβάβ, ο υιός του Ζερά, εκ της Βοσόρρας.
\par 45 Και απέθανεν ο Ιωβάβ, και εβασίλευσεν αντ' αυτού ο Χουσάμ, εκ της γης των Θαιμανιτών.
\par 46 Και απέθανεν ο Χουσάμ, και εβασίλευσεν αντ' αυτού Αδάδ, ο υιός του Βεδάδ, όστις επάταξε τους Μαδιανίτας εν τη πεδιάδι του Μωάβ· το δε όνομα της πόλεως αυτού Αβίθ.
\par 47 Και απέθανεν ο Αδάδ, και εβασίλευσεν αντ' αυτού Σαμλά, ο εκ Μασρεκάς.
\par 48 Και απέθανεν ο Σαμλά, και εβασίλευσεν αντ' αυτού Σαούλ, ο από Ρεχωβώθ, της παρά τον ποταμόν.
\par 49 Και απέθανεν ο Σαούλ, και εβασίλευσεν αντ' αυτού Βάαλ-χανάν, ο υιός του Αχβώρ.
\par 50 Και απέθανεν ο Βάαλ-χανάν, και εβασίλευσεν αντ' αυτού ο Αδάδ· και το όνομα της πόλεως αυτού ήτο Παί· το δε όνομα της γυναικός αυτού Μεεταβεήλ, θυγάτηρ Ματραίδ, θυγατρός Μαιζαάβ.
\par 51 Αποθανόντος δε του Αδάδ, εστάθησαν ηγεμόνες Εδώμ, ηγεμών Θαμνά, ηγεμών Αλβά, ηγεμών Ιεθέθ,
\par 52 ηγεμών Ολιβαμά, ηγεμών Ηλά, ηγεμών Φινών,
\par 53 ηγεμών Κενέζ, ηγεμών Θαιμάν, ηγεμών Μιβσάρ,
\par 54 ηγεμών Μαγεδήλ, ηγεμών Ιράμ· ούτοι εστάθησαν οι ηγεμόνες Εδώμ.

\chapter{2}

\par 1 Ούτοι είναι οι υιοί του Ισραήλ· Ρουβήν, Συμεών, Λευΐ και Ιούδας, Ισσάχαρ και Ζαβουλών,
\par 2 Δαν, Ιωσήφ και Βενιαμίν, Νεφθαλί, Γαδ, και Ασήρ.
\par 3 Υιοί του Ιούδα, Ηρ και Αυνάν και Σηλά· τρεις εγεννήθησαν εις αυτόν εκ της θυγατρός του Σουά της Χαναανίτιδος. Ήτο δε ο Ηρ, ο πρωτότοκος του Ιούδα, πονηρός ενώπιον του Κυρίου· και εθανάτωσεν αυτόν.
\par 4 Και Θάμαρ, η νύμφη αυτού, εγέννησεν εις αυτόν τον Φαρές και τον Ζαρά. Πάντες οι υιοί του Ιούδα ήσαν πέντε.
\par 5 Υιοί του Φαρές, Εσρών και Αμούλ.
\par 6 Και υιοί του Ζαρά, Ζιμβρί και Εθάν και Αιμάν και Χαλχόλ και Δαρά· πάντες πέντε.
\par 7 Και υιοί του Χαρμί, Αχάρ, ο ταράξας τον Ισραήλ, όστις έκαμε παράβασιν εις το ανάθεμα.
\par 8 Και υιοί του Εθάν, Αζαρίας.
\par 9 Υιοί δε του Εσρών, οι γεννηθέντες εις αυτόν, Ιεραμεήλ και Αράμ και Χάλεβ.
\par 10 Και Αράμ εγέννησε τον Αμμιναδάβ, και Αμμιναδάβ εγέννησε τον Ναασσών, τον άρχοντα των υιών Ιούδα.
\par 11 Και Ναασσών εγέννησε τον Σαλμά, και Σαλμά εγέννησε τον Βοόζ,
\par 12 και Βοόζ εγέννησε τον Ωβήδ, και Ωβήδ εγέννησε τον Ιεσσαί·
\par 13 και Ιεσσαί εγέννησεν Ελιάβ τον πρωτότοκον αυτού και Αβιναδάβ τον δεύτερον και Σαμμά τον τρίτον,
\par 14 Ναθαναήλ τον τέταρτον, Ραδδαί τον πέμπτον,
\par 15 Οσέμ τον έκτον, Δαβίδ τον έβδομον.
\par 16 Και αδελφαί αυτών ήσαν Σερουΐα και Αβιγαία. Και υιοί της Σερουΐας, Αβισαί και Ιωάβ και Ασαήλ, τρεις.
\par 17 Η δε Αβιγαία εγέννησε τον Αμασά· και πατήρ του Αμασά ήτο Ιεθέρ ο Ισμαηλίτης.
\par 18 Και Χάλεβ ο υιός του Εσρών εγέννησεν υιούς εκ της Αζουβά της γυναικός αυτού και εκ της Ιεριώθ· και οι υιοί αυτής ήσαν Ιεσέρ και Σωβάβ και Αρδών.
\par 19 Αποθανούσης δε της Αζουβά, ο Χάλεβ έλαβεν εις εαυτόν την Εφράθ, ήτις εγέννησεν εις αυτόν τον Ωρ.
\par 20 Και Ωρ εγέννησε τον Ουρί, και ο Ουρί εγέννησε τον Βεζελεήλ.
\par 21 Και μετά ταύτα εισήλθεν ο Εσρών προς την θυγατέρα του Μαχείρ πατρός του Γαλαάδ· και ούτος έλαβεν αυτήν ηλικίας ων εξήκοντα ετών· και εγέννησεν εις αυτόν τον Σεγούβ.
\par 22 Και Σεγούβ εγέννησε τον Ιαείρ, όστις είχεν εικοσιτρείς πόλεις εν τη γη Γαλαάδ.
\par 23 Και έλαβεν εξ αυτών Γεσσούρ και Αράμ τας κώμας Ιαείρ, την Καινάθ και τας κώμας αυτής, εξήκοντα πόλεις. Πάσαι αύται ήσαν των υιών του Μαχείρ, πατρός του Γαλαάδ.
\par 24 Και αφού απέθανεν ο Εσρών Χάλεβ-εφραθά, Αβιά η γυνή του Εσρών εγέννησεν εις αυτόν Ασχώρ τον πατέρα του Θεκουέ.
\par 25 Και οι υιοί του Ιεραμεήλ, πρωτοτόκου του Εσρών, ήσαν Αράμ ο πρωτότοκος, και Βουνά και Ορέν και Οσέμ και Αχιά.
\par 26 Ο Ιεραμεήλ έλαβε και άλλην γυναίκα, της οποίας το όνομα ήτο Ατάρα· αύτη ήτο μήτηρ του Ωνάμ.
\par 27 Και οι υιοί του Αράμ, πρωτοτόκου του Ιεραμεήλ, ήσαν Μαάς και Ιαμείν και Εκέρ.
\par 28 Και οι υιοί του Ωνάμ ήσαν Σαμμαΐ και Ιαδαέ. Και οι υιοί του Σαμμαΐ, Ναδάβ και Αβισούρ.
\par 29 Και το όνομα της γυναικός του Αβισούρ ήτο Αβιχαίλ, και εγέννησεν εις αυτόν τον Ααβάν και τον Μωλήδ.
\par 30 Και οι υιοί του Ναδάβ ήσαν Σελέδ και Απφαΐμ· απέθανε δε ο Σελέδ άτεκνος.
\par 31 Και οι υιοί του Απφαΐμ, Ιεσεί. Και οι υιοί του Ιεσεί, Σησάν. Και οι υιοί του Σησάν, Ααλαί.
\par 32 Και οι υιοί του Ιαδαέ, αδελφού του Σαμμαΐ, Ιεθέρ και Ιωνάθαν· απέθανε δε ο Ιεθέρ άτεκνος.
\par 33 Και οι υιοί του Ιωνάθαν, Φαλέθ και Ζαζά· ούτοι ήσαν οι υιοί του Ιεραμεήλ.
\par 34 Ο δε Σησάν δεν είχεν υιούς, αλλά θυγατέρας. Και είχεν ο Σησάν δούλον Αιγύπτιον, ονομαζόμενον Ιαραά·
\par 35 και έδωκεν ο Σησάν την θυγατέρα αυτού εις τον Ιαραά, τον δούλον αυτού, εις γυναίκα· και εγέννησεν εις αυτόν τον Ατθαΐ.
\par 36 Και Ατθαΐ εγέννησε τον Νάθαν, και Νάθαν εγέννησε τον Ζαβάδ,
\par 37 και Ζαβάδ εγέννησε τον Εφλάλ, και Εφλάλ εγέννησε τον Ωβήδ,
\par 38 και Ωβήδ εγέννησε τον Ιηού, και Ιηού εγέννησε τον Αζαρίαν,
\par 39 και Αζαρίας εγέννησε τον Χελής, και Χελής εγέννησε τον Ελεασά,
\par 40 και Ελεασά εγέννησε τον Σισαμαΐ, και Σισαμαΐ εγέννησε τον Σαλλούμ,
\par 41 και Σαλλούμ εγέννησε τον Ιεκαμίαν, και Ιεκαμίας εγέννησε τον Ελισαμά.
\par 42 Οι δε υιοί του Χάλεβ, αδελφού του Ιεραμεήλ, ήσαν Μησά ο πρωτότοκος αυτού, όστις ήτο ο πατήρ του Ζίφ· και οι υιοί του Μαρησά, πατρός του Χεβρών.
\par 43 Και οι υιοί του Χεβρών, Κορέ και Θαπφουά και Ρεκέμ και Σεμά.
\par 44 Και ο Σεμά εγέννησε τον Ραάμ, πατέρα του Ιορκοάμ· και ο Ρεκέμ εγέννησε τον Σαμμαΐ.
\par 45 Και ο υιός του Σαμμαΐ ήτο Μαών· ο δε Μαών ήτο πατήρ Βαιθ-σούρ.
\par 46 Και η Γεφά, παλλακή του Χάλεβ, εγέννησε τον Χαρράν, και τον Μοσά, και τον Γαζέζ. Και Χαρράν εγέννησε τον Γαζέζ.
\par 47 Και οι υιοί του Ιαδαΐ, Ρεγέμ και Ιωθάμ και Γησάν και Φελέτ και Γεφά και Σαγάφ.
\par 48 Η Μααχά, παλλακή του Χάλεβ, εγέννησε τον Σεβέρ και τον Θιρχανά.
\par 49 Εγέννησεν έτι τον Σαγάφ πατέρα Μαδμαννά, τον Σεβά πατέρα Μαχβηνά και πατέρα Γαβαά· η θυγάτηρ δε του Χάλεβ ήτο η Αχσά.
\par 50 Ούτοι ήσαν οι υιοί του Χάλεβ, υιού του Ωρ, πρωτοτόκου της Εφραθά· Σωβάλ ο πατήρ Κιριάθ-ιαρείμ,
\par 51 Σαλμά ο πατήρ Βηθλεέμ, Αρέφ ο πατήρ Βαιθ-γαδέρ.
\par 52 Και εις τον Σωβάλ τον πατέρα Κιριάθ-ιαρείμ έγειναν υιοί, ο Αροέ και Ασεί-αμενουχώθ.
\par 53 Και αι συγγένειαι Κιριάθ-ιαρείμ ήσαν οι Ιεθρίται και οι Φουθίται και οι Σουμαθίται και οι Μισραΐται. Εκ τούτων εξήλθον οι Σαραθαίοι και οι Εσθαωλαίοι.
\par 54 Οι υιοί του Σαλμά, Βηθλεέμ και οι Νετωφαθίται, Αταρώθ του οίκου Ιωάβ και οι Ζωρίται, το ήμισυ των Μαναχαθιτών,
\par 55 και αι συγγένειαι των γραμματέων, των κατοικούντων εν Ιαβής, οι Θιραθίται, οι Σιμεαθίται και οι Σουχαθίται. Ούτοι είναι οι Κεναίοι, οι εξελθόντες εκ του Αιμάθ, πατρός του οίκου Ρηχάβ.

\chapter{3}

\par 1 Ούτοι δε ήσαν οι υιοί του Δαβίδ, οι γεννηθέντες εις αυτόν εν Χεβρών· ο πρωτότοκος, Αμνών, εκ της Αχινοάμ της Ιεζραηλίτιδος· ο δεύτερος, Δανιήλ, εκ της Αβιγαίας της Καρμηλίτιδος·
\par 2 ο τρίτος, Αβεσσαλώμ, ο υιός της Μααχά, θυγατρός του Θαλμαΐ βασιλέως της Γεσσούρ· ο τέταρτος, Αδωνίας, ο υιός της Αγγείθ·
\par 3 ο πέμπτος, Σεφατίας, εκ της Αβιτάλ· ο έκτος, Ιθραάμ, εκ της Αιγλά γυναικός αυτού.
\par 4 Εξ εγεννήθησαν εις αυτόν εν Χεβρών· και εβασίλευσεν εκεί επτά έτη και εξ μήνας· εν δε Ιερουσαλήμ εβασίλευσε τριάκοντα τρία έτη.
\par 5 ούτοι δε εγεννήθησαν εις αυτόν εν Ιερουσαλήμ· Σαμαά και Σωβάβ και Νάθαν και Σολομών, τέσσαρες, εκ της Βηθ-σαβεέ θυγατρός του Αμμιήλ·
\par 6 και Ιεβάρ, και Ελισαμά και Ελιφαλέτ
\par 7 και Νωγά και Νεφέγ και Ιαφιά
\par 8 και Ελισαμά και Ελιαδά και Ελιφελέτ, εννέα·
\par 9 πάντες υιοί του Δαβίδ, πλην των υιών των παλλακών, και Θάμαρ η αδελφή αυτών.
\par 10 Υιός δε του Σολομώντος ήτο ο Ροβοάμ, υιός τούτου ο Αβιά, υιός τούτου ο Ασά, υιός τούτου ο Ιωσαφάτ,
\par 11 υιός τούτου ο Ιωράμ, υιός τούτου ο Οχοζίας, υιός τούτου ο Ιωάς,
\par 12 υιός τούτου ο Αμασίας, υιός τούτου ο Αζαρίας, υιός τούτου ο Ιωθάμ,
\par 13 υιός τούτου ο Άχαζ, υιός τούτου ο Εζεκίας, υιός τούτου ο Μανασσής,
\par 14 υιός τούτου ο Αμών, υιός τούτου ο Ιωσίας.
\par 15 Οι υιοί δε του Ιωσίου ήσαν, ο πρωτότοκος αυτού Ιωανάν· ο δεύτερος, Ιωακείμ· ο τρίτος, Σεδεκίας· ο τέταρτος, Σαλλούμ.
\par 16 Και υιοί του Ιωακείμ, Ιεχονίας ο υιός αυτού, Σεδεκίας ο υιός τούτου.
\par 17 Και υιοί του Ιεχονίου, Ασείρ, Σαλαθιήλ ο υιός τούτου,
\par 18 και Μαλχιράμ και Φεδαΐας και Σενασάρ, Ιεκαμίας, Ωσαμά και Νεδαβίας.
\par 19 Και υιοί του Φεδαΐα, Ζοροβάβελ και Σιμεΐ· και υιοί του Ζοροβάβελ, Μεσουλλάμ και Ανανίας και Σελωμείθ η αδελφή αυτών·
\par 20 και Ασσουβά και Οήλ και Βαραχίας και Ασαδίας και Ιουσάβ-εσέδ, πέντε.
\par 21 Και υιοί του Ανανία, Φελατίας και Ιεσαΐας· οι υιοί του Ρεφαΐα, οι υιοί του Αρνάν, οι υιοί του Οβαδία, οι υιοί του Σεχανία.
\par 22 Και υιοί του Σεχανία, Σεμαΐας· και υιοί του Σεμαΐα, Χαττούς και Ιγεάλ και Βαρίας και Νεαρίας και Σαφάτ, εξ.
\par 23 Και υιοί του Νεαρία, Ελιωηνάϊ και Εζεκίας και Αζρικάμ· τρεις.
\par 24 Και υιοί του Ελιωηνάϊ, Ωδαΐας και Ελιασείβ και Φελαΐας και Ακκούβ και Ιωανάν και Δαλαΐας και Ανανί, επτά.

\chapter{4}

\par 1 Υιοί του Ιούδα, Φαρές, Εσρών και Χαρμί και Ωρ και Σωβάλ.
\par 2 Και Ρεαΐα, ο υιός του Σωβάλ, εγέννησε τον Ιαάθ· και Ιαάθ εγέννησε τον Αχουμαΐ και τον Λαάδ. Αύται είναι αι συγγένειαι των Σαραθιτών.
\par 3 Και ούτοι ήσαν οι υιοί του πατρός Ητάμ· Ιεζραέλ και Ιεσμά και Ιεδβάς· και το όνομα της αδελφής αυτών Ασέλ-ελφονί·
\par 4 και Φανουήλ ο πατήρ Γεδώρ, και Εσέρ ο πατήρ Χουσά. Ούτοι είναι οι υιοί του Ωρ, πρωτοτόκου Εφραθά, πατρός Βηθλεέμ.
\par 5 Και Ασχώρ ο πατήρ Θεκουέ είχε δύο γυναίκας, Ελά και Νααρά.
\par 6 Και η μεν Νααρά εγέννησεν εις αυτόν τον Αχουζάμ και τον Εφέρ και τον Θαιμανί και τον Αχασταρί. Ούτοι ήσαν οι υιοί της Νααρά.
\par 7 Οι δε υιοί της Ελά, Σερέθ και Ιεσοάρ και Εθνάν.
\par 8 Και ο Κως εγέννησε τον Ανούβ και τον Σωβηβά, και τας συγγενείας του Αχαρήλ, υιού του Αρούμ.
\par 9 Και ο Ιαβής ήτο ενδοξότερος παρά τους αδελφούς αυτού· και η μήτηρ αυτού εκάλεσε το όνομα αυτού Ιαβής, λέγουσα, Επειδή εγέννησα αυτόν εν λύπη.
\par 10 Και επεκαλέσθη ο Ιαβής τον Θεόν του Ισραήλ, λέγων, Είθε μετ' ευλογίας να με ευλογήσης και να εκτείνης τα όριά μου, και η χειρ σου να ήναι μετ' εμού και να με φυλάττης από κακού, ώστε να μη έχω λύπην. Και εχάρισεν ο Θεός εις αυτόν όσα εζήτησε.
\par 11 Και ο Χελούβ, αδελφός του Σουά, εγέννησε τον Μεχείρ· ούτος ήτο πατήρ του Εσθών.
\par 12 Και ο Εσθών εγέννησε τον Βαιθ-ραφά και τον Φασέα και τον Θεχιννά, τον πατέρα της πόλεως Νάας· ούτοι είναι οι άνδρες Ρηχά.
\par 13 Και οι υιοί του Κενέζ, Γοθονιήλ και Σεραΐας· και οι υιοί του Γοθονιήλ, Αθάθ.
\par 14 Και ο Μεονοθαί εγέννησε τον Οφρά· και ο Σεραΐας εγέννησε τον Ιωάβ, πατέρα της κοιλάδος των τεκτόνων· διότι ήσαν τέκτονες.
\par 15 Και οι υιοί του Χάλεβ, υιού του Ιεφοννή, Ιρού, Ηλά και Ναάμ· και οι υιοί του Ηλά, Κενέζ.
\par 16 Και οι υιοί του Ιαλελεήλ, Ζιφ και Ζιφά, Θηριά και Ασαρεήλ.
\par 17 Και οι υιοί του Εζρά, Ιεθέρ και Μερέδ και Εφέρ και Ιαλών· και η γυνή του Μερέδ εγέννησε τον Μαριάμ και τον Σαμμαΐ και τον Ιεσβά τον πατέρα Εσθεμωά.
\par 18 Και η άλλη γυνή αυτού, η Ιουδαία, εγέννησε τον Ιέρεδ τον πατέρα Γεδώρ, και τον Έβερ τον πατέρα Σωχώ, και τον Ιεκουθιήλ τον πατέρα Ζανωά. Και ούτοι είναι οι υιοί της Βιθίας θυγατρός του Φαραώ, την οποίαν έλαβεν ο Μερέδ.
\par 19 Και οι υιοί της γυναικός αυτού της Οδίας, αδελφής του Ναχάμ, πατρός Κεειλά του Γαρμίτου και Εσθεμωά του Μααχαθίτου.
\par 20 Και οι υιοί του Σιμών ήσαν Αμνών και Ριννά, Βεν-ανάν και Θιλών. Και οι υιοί του Ιεσεί, Ζωχέθ και Βεν-ζωχέθ.
\par 21 Οι υιοί του Σηλά, υιού του Ιούδα, ήσαν Ηρ ο πατήρ Ληχά και Λααδά ο πατήρ Μαρησά, και αι συγγένειαι του οίκου των εργαζομένων την βύσσον, του οίκου του Ασβεά,
\par 22 και ο Ιωκείμ και οι άνδρες Χαζηβά και ο Ιωάς και ο Σαράφ, οίτινες εδέσποζον εν Μωάβ, και ο Ιασουβί-λεχέμ. Πλην ταύτα είναι αρχαία πράγματα.
\par 23 Ούτοι ήσαν οι κεραμείς και οι κατοικούντες εν Νεταΐμ και Γεδιρά· εκεί κατώκουν μετά του βασιλέως διά τας εργασίας αυτού.
\par 24 Οι υιοί του Συμεών ήσαν Νεμουήλ και Ιαμείν, Ιαρείβ, Ζερά και Σαούλ·
\par 25 Σαλλούμ υιός τούτου, Μιβσάμ υιός τούτου, Μισμά υιός τούτου.
\par 26 Και οι υιοί του Μισμά, Αμουήλ ο υιός αυτού, Ζακχούρ υιός τούτου, Σιμεΐ υιός τούτου.
\par 27 Και ο Σιμεΐ εγέννησε δεκαέξ υιούς και εξ θυγατέρας· οι αδελφοί αυτού όμως δεν είχον υιούς πολλούς, ουδέ επληθύνθησαν πάσαι αι συγγένειαι αυτών, καθώς των υιών του Ιούδα.
\par 28 Και κατώκησαν εν Βηρ-σαβεέ και Μωλαδά και Ασάρ-σουάλ,
\par 29 και εν Βαλλά και εν Ασέμ και εν Θωλάδ
\par 30 και εν Βαιθουήλ και εν Ορμά και εν Σικλάγ
\par 31 και εν Βαιθ-μαρχαβώθ και εν Ασάρ-σουσίμ και εν Βαιθ-βηρεΐ και εν Σααραείμ· Αύται ήσαν αι πόλεις αυτών έως της βασιλείας του Δαβίδ.
\par 32 Και αι κώμαι αυτών ήσαν, Ητάμ και Αείν, Ριμμών και Θοχέν και Ασάν, πέντε πόλεις·
\par 33 και πάσαι αι κώμαι αυτών αι πέριξ τούτων των πόλεων, μέχρι Βάαλ. Αύται ήσαν αι κατοικήσεις αυτών και η κατά γενεάς διαίρεσις αυτών.
\par 34 Και Μεσωβάβ και Ιαμλήχ και Ιωσά ο υιός του Αμασία,
\par 35 και Ιωήλ και Ιηού ο υιός του Ιωσιβία, υιού του Σεραΐα, υιού του Ασιήλ,
\par 36 και Ελιωηνάϊ και Ιαακωβά και Ιεσοχαΐας και Ασαΐας και Αδιήλ και Ιεσιμιήλ και Βεναΐας
\par 37 και Ζιζά ο υιός του Σιφεί, υιού του Αλλόν, υιού του Ιεδαΐα, υιού του Σιμρί, υιού του Σεμαΐα·
\par 38 ούτοι οι κατ' όνομα μνημονευθέντες ήσαν άρχοντες εις τας συγγενείας αυτών· και ο οίκος των πατέρων αυτών ηυξήθη εις πλήθος.
\par 39 Και υπήγαν έως της εισόδου Γεδώρ, προς ανατολάς της κοιλάδος, διά να ζητήσωσι βοσκήν εις τα ποίμνια αυτών·
\par 40 και εύρηκαν βοσκήν παχείαν και καλήν, και η γη ήτο ευρύχωρος και ήσυχος και ειρηνική· διότι οι πρότερον κατοικούντες εκεί ήσαν εκ του Χαμ.
\par 41 Και ούτοι οι γεγραμμένοι κατ' όνομα ήλθον εν ταις ημέραις Εζεκίου του βασιλέως Ιούδα, και επάταξαν τας σκηνάς αυτών και τους εκεί ευρεθέντας Μιναίους, και ηφάνισαν αυτούς έως της ημέρας ταύτης, και κατώκησαν αντ' αυτών· διότι ήτο εκεί βοσκή διά τα ποίμνια αυτών.
\par 42 Και εξ αυτών, εκ των υιών του Συμεών, πεντακόσιοι άνδρες υπήγαν εις το όρος Σηείρ, έχοντες επί κεφαλής αυτών τον Φελατίαν και Νεαρίαν και Ρεφαΐαν και Οζιήλ, υιούς του Ιεσεί·
\par 43 και επάταξαν το υπόλοιπον των Αμαληκιτών το διασωθέν, και κατώκησαν εκεί έως της ημέρας ταύτης.

\chapter{5}

\par 1 Οι δε υιοί του Ρουβήν πρωτοτόκου του Ισραήλ, διότι ούτος ήτο ο πρωτότοκος· επειδή όμως εμίανε την κοίτην του πατρός αυτού, τα πρωτοτόκια αυτού εδόθησαν εις τους υιούς του Ιωσήφ, υιού του Ισραήλ· πλην ουχί διά να έχη τα πρωτοτόκια ως προς την γενεαλογίαν·
\par 2 διότι ο Ιούδας υπερίσχυσεν υπέρ τους αδελφούς αυτού, ώστε εξ αυτού να εξέλθη ο ηγούμενος· τα πρωτοτόκια όμως ήσαν του Ιωσήφ·
\par 3 οι υιοί του Ρουβήν πρωτοτόκου του Ισραήλ ήσαν Ανώχ και Φαλλού, Εσρών και Χαρμί.
\par 4 υιοί του Ιωήλ, Σεμαΐας υιός τούτου, Γωγ υιός τούτου, Σιμεΐ υιός τούτου,
\par 5 Μιχά υιός τούτου, Ρεαΐα υιός τούτου, Βάαλ υιός τούτου,
\par 6 Βεηρά υιός τούτου, τον οποίον μετώκισεν ο Θελγάθ-φελνασάρ βασιλεύς της Ασσυρίας· ούτος ήτο ο αρχηγός των Ρουβηνιτών.
\par 7 Και των αδελφών αυτού κατά τας συγγενείας αυτών, ότε η γενεαλογία των γενεών αυτών απηριθμήθη, οι αρχηγοί ήσαν Ιεϊήλ και Ζαχαρίας,
\par 8 και Βελά ο υιός του Αζάζ, υιού του Σεμά, υιού του Ιωήλ· ούτος κατώκησεν εν Αροήρ και έως Νεβώ και Βάαλ-μεών·
\par 9 και προς ανατολάς κατώκησεν έως της εισόδου της ερήμου από του Ευφράτου ποταμού· διότι τα κτήνη αυτών είχον πληθυνθή εν τη γη Γαλαάδ.
\par 10 Και εν ταις ημέραις του Σαούλ έκαμον πόλεμον προς τους Αγαρηνούς, οίτινες έπεσον διά της χειρός αυτών· και κατώκησαν εν ταις σκηναίς αυτών καθ' όλον το ανατολικόν της Γαλαάδ.
\par 11 Και οι υιοί του Γαδ κατώκησαν κατέναντι αυτών, εν τη γη Βασάν έως Σαλχά·
\par 12 Ιωήλ ο αρχηγός και Σαφάμ ο δεύτερος, και Ιαναΐ και Σαφάτ, εν Βασάν.
\par 13 Και οι αδελφοί αυτών εκ του οίκου των πατέρων αυτών ήσαν, Μιχαήλ και Μεσουλλάμ και Σεβά και Ιωράμ και Ιαχάν και Ζιέ και Έβερ, επτά.
\par 14 ούτοι είναι οι υιοί του Αβιχαίλ υιού του Ουρί, υιού του Ιαροά, υιού του Γαλαάδ, υιού του Μιχαήλ, υιού του Ιεσισαΐ, υιού του Ιαδώ, υιού του Βουζ.
\par 15 Αχί ο υιός του Αβδιήλ, υιού του Γουνί, ήτο ο αρχηγός του οίκου των πατέρων αυτών.
\par 16 Και κατώκησαν εν Γαλαάδ, εν Βασάν και εν ταις κώμαις αυτής, και εν πάσι τοις περιχώροις Σαρών, έως των ορίων αυτών.
\par 17 Πάντες ούτοι απηριθμήθησαν κατά την γενεαλογίαν αυτών εν ταις ημέραις του Ιωθάμ βασιλέως του Ιούδα, και εν ταις ημέραις του Ιεροβοάμ βασιλέως του Ισραήλ.
\par 18 Οι υιοί του Ρουβήν και οι Γαδίται και το ήμισυ της φυλής του Μανασσή, εκ των δυνατών, άνδρες φέροντες ασπίδα και μάχαιραν και εντείνοντες τόξον και γεγυμνασμένοι εις πόλεμον, ήσαν τεσσαράκοντα τέσσαρες χιλιάδες και επτακόσιοι εξήκοντα, εξερχόμενοι εις πόλεμον.
\par 19 Και έκαμνον πόλεμον προς τους Αγαρηνούς και Ιετουραίους και Ναφισαίους και Νοδαβαίους.
\par 20 Και εβοηθήθησαν εναντίον αυτών, και οι Αγαρηνοί παρεδόθησαν εις τας χείρας αυτών και πάντες οι μετ' αυτών· διότι προς τον Θεόν εβόησαν εν τη μάχη, και επήκουσεν αυτών, επειδή ήλπισαν επ' αυτόν.
\par 21 Και ηχμαλώτισαν τα κτήνη αυτών, τας καμήλους αυτών πεντήκοντα χιλιάδας, και πρόβατα διακοσίας πεντήκοντα χιλιάδας, και όνους δύο χιλιάδας, και ψυχάς ανθρώπων εκατόν χιλιάδας.
\par 22 Διότι πολλοί έπεσον τεθανατωμένοι, επειδή ο πόλεμος ήτο εκ Θεού. Και κατώκησαν αντ' αυτών έως της μετοικεσίας.
\par 23 Και οι υιοί του ημίσεος της φυλής Μανασσή κατώκησαν εν τη γή· ούτοι ηύξησαν από Βασάν έως Βάαλ-ερμών και Σενείρ και έως του όρους Αερμών·
\par 24 Ούτοι δε ήσαν οι αρχηγοί του οίκου των πατέρων αυτών· Εφέρ και Ιεσεί και Ελιήλ και Αζριήλ και Ιερεμίας και Ωδουΐας και Ιαδιήλ, άνδρες δυνατοί εν ισχύϊ, άνδρες ονομαστοί, αρχηγοί του οίκου των πατέρων αυτών.
\par 25 Και εστάθησαν παραβάται κατά του Θεού των πατέρων αυτών και επόρνευσαν κατόπιν των θεών των λαών της γης, τους οποίους ο Θεός ηφάνισεν απ' έμπροσθεν αυτών.
\par 26 Διά τούτο ο Θεός του Ισραήλ διήγειρε το πνεύμα του Φούλ βασιλέως της Ασσυρίας και το πνεύμα του Θελγάθ-φελνασάρ βασιλέως της Ασσυρίας, και μετώκισεν αυτούς, τους Ρουβηνίτας και τους Γαδίτας και το ήμισυ της φυλής του Μανασσή, και έφερεν αυτούς εις Αλά και εις Αβώρ και εις Αρά και εις τον ποταμόν Γωζάν, έως της ημέρας ταύτης.

\chapter{6}

\par 1 Οι υιοί του Λευΐ ήσαν Γηρσών, Καάθ και Μεραρί.
\par 2 Και οι υιός του Καάθ, Αμράμ, Ισαάρ και Χεβρών και Οζιήλ.
\par 3 Και οι υιοί του Αμράμ, Ααρών και Μωϋσής και η Μαριάμ. Οι δε υιοί του Ααρών, Ναδάβ και Αβιούδ, Ελεάζαρ και Ιθάμαρ.
\par 4 Ο Ελεάζαρ εγέννησε τον Φινεές, ο Φινεές εγέννησε τον Αβισσουά,
\par 5 και Αβισσουά εγέννησε τον Βουκκί, και Βουκκί εγέννησε τον Οζί,
\par 6 και Οζί εγέννησε τον Ζεραΐαν, και Ζεραΐας εγέννησε τον Μεραϊώθ,
\par 7 Μεραϊώθ εγέννησε τον Αμαρίαν, και Αμαρίας εγέννησε τον Αχιτώβ,
\par 8 και Αχιτώβ εγέννησε τον Σαδώκ, και Σαδώκ εγέννησε τον Αχιμάας,
\par 9 και Αχιμάας εγέννησε τον Αζαρίαν, και Αζαρίας εγέννησε τον Ιωανάν,
\par 10 και Ιωανάν εγέννησε τον Αζαρίαν, ούτος είναι ο ιερατεύσας εν τω ναώ, τον οποίον ωκοδόμησεν ο Σολομών εν Ιερουσαλήμ·
\par 11 και Αζαρίας εγέννησε τον Αμαρίαν, και Αμαρίας εγέννησε τον Αχιτώβ,
\par 12 και Αχιτώβ εγέννησε τον Σαδώκ, και Σαδώκ εγέννησε τον Σαλλούμ,
\par 13 και Σαλλούμ εγέννησε τον Χελκίαν, και Χελκίας εγέννησε τον Αζαρίαν,
\par 14 και Αζαρίας εγέννησε τον Σεραΐαν, και Σεραΐας εγέννησε τον Ιωσεδέκ,
\par 15 και Ιωσεδέκ υπήγεν εις την μετοικεσίαν, ότε ο Κύριος έκαμε να μετοικισθή ο Ιούδας και η Ιερουσαλήμ διά χειρός του Ναβουχοδονόσορ.
\par 16 Οι υιοί του Λευΐ, Γηρσώμ, Καάθ και Μεραρί.
\par 17 Και ταύτα είναι τα ονόματα των υιών του Γηρσώμ· Λιβνί και Σιμεΐ.
\par 18 Και οι υιοί του Καάθ, Αμράμ και Ισαάρ και Χεβρών και Οζιήλ.
\par 19 Οι υιοί του Μεραρί, Μααλί και Μουσί. Και αύται είναι αι συγγένειαι των Λευϊτών κατά τας πατριάς αυτών.
\par 20 του Γηρσώμ, Λιβνί υιός τούτου, Ιαάθ υιός τούτου, Ζιμμά υιός τούτου,
\par 21 Ιωάχ υιός τούτου, Ιδδώ υιός τούτου, Ζερά υιός τούτου, Ιεθραί υιός τούτου.
\par 22 Οι υιοί του Καάθ, Αμμιναδάβ υιός αυτού, Κορέ υιός τούτου, Ασείρ υιός τούτου,
\par 23 Ελκανά υιός τούτου, και Εβιασάφ υιός τούτου, και Ασείρ υιός τούτου,
\par 24 Ταχάθ υιός τούτου, Ουριήλ υιός τούτου, Οζίας υιός τούτου, και Σαούλ υιός τούτου.
\par 25 Και οι υιοί του Ελκανά, Αμασαΐ και Αχιμώθ.
\par 26 Και Ελκανά· οι υιοί του Ελκανά, Σουφί υιός τούτου, και Ναχάθ υιός τούτου,
\par 27 Ελιάβ υιός τούτου, Ιεροάμ υιός τούτου, Ελκανά υιός τούτου.
\par 28 Και οι υιοί του Σαμουήλ, Βασνί ο πρωτότοκος και Αβιά.
\par 29 Οι υιοί του Μεραρί, Μααλί, Λιβνί υιός τούτου, Σιμεΐ υιός τούτου, Ουζά υιός τούτου,
\par 30 Σιμαά υιός τούτου, Αγγία υιός τούτου, Ασαΐας υιός τούτου.
\par 31 Και ούτοι είναι τους οποίους κατέστησεν ο Δαβίδ επί το έργον της μουσικής του οίκου του Κυρίου, αφού η κιβωτός εύρηκεν ανάπαυσιν.
\par 32 Και υπηρέτουν έμπροσθεν της σκηνής του μαρτυρίου με ψαλτωδίας, εωσού ο Σολομών ωκοδόμησε τον οίκον του Κυρίου εν Ιερουσαλήμ· και τότε κατεστάθησαν εις το υπούργημα αυτών, κατά την τάξιν αυτών.
\par 33 Και ούτοι είναι οι κατασταθέντες, μετά των τέκνων αυτών. Εκ των υιών των Κααθιτών, Αιμάν ο ψαλτωδός, υιός του Ιωήλ, υιού του Σαμουήλ,
\par 34 υιού του Ελκανά, υιού του Ιεροάμ, υιού του Ελιήλ, υιού του Θωά,
\par 35 υιού του Σούφ, υιού του Ελκανά, υιού του Μαάθ υιού του Αμασαί,
\par 36 υιού του Ελκανά υιού του Ιωήλ, υιού του Αζαρίου, υιού του Σοφονίου,
\par 37 υιού του Ταχάθ, υιού του Ασείρ, υιού του Εβιασάφ, υιού του Κορέ,
\par 38 υιού του Ισαάρ, υιού του Καάθ, υιού του Λευΐ υιού του Ισραήλ·
\par 39 και ο αδελφός αυτού Ασάφ, ο ιστάμενος εν δεξιά αυτού· Ασάφ ο υιός του Βαραχίου, υιού του Σιμεά,
\par 40 υιού του Μιχαήλ, υιού του Βαασίου, υιού του Μαλχίου,
\par 41 υιού του Εθνεί υιού του Ζερά, υιού του Αδαΐα,
\par 42 υιού του Εθάν, υιού του Ζιμμά, υιού του Σιμεΐ,
\par 43 υιού του Ιαάθ, υιού του Γηρσώμ, υιού του Λευΐ·
\par 44 και οι αδελφοί αυτών, οι υιοί του Μεραρί, οι εξ αριστερών· Εθάν ο υιός του Κεισί, υιού του Αβδί, υιού του Μαλλούχ,
\par 45 υιού του Ασαβία, υιού του Αμασία, υιού του Χελκίου,
\par 46 υιού του Αμσί, υιού του Βανί, υιού του Σαμείρ,
\par 47 υιού του Μααλί, υιού του Μουσί, υιού του Μεραρί, υιού του Λευΐ·
\par 48 και οι αδελφοί αυτών οι Λευΐται, διωρισμένοι εις πάσας τας υπηρεσίας της σκηνής του οίκου του Θεού.
\par 49 Ο δε Ααρών και οι υιοί αυτού εθυμίαζον επί το θυσιαστήριον των ολοκαυτωμάτων, και επί το θυσιαστήριον του θυμιάματος, διωρισμένοι εις πάσας τας εργασίας του αγίου των αγίων, και εις το να κάμνωσιν εξιλέωσιν υπέρ του Ισραήλ, κατά πάντα όσα προσέταξε Μωϋσής ο δούλος του Θεού.
\par 50 Και ούτοι είναι οι υιοί του Ααρών· Ελεάζαρ υιός τούτου, Φινεές υιός τούτου, Αβισσουά υιός τούτου,
\par 51 Βουκκί υιός τούτου, Οζί υιός τούτου, Ζεραΐας υιός τούτου,
\par 52 Μεραϊώθ υιός τούτου, Αμαρίας υιός τούτου, Αχιτώβ υιός τούτου,
\par 53 Σαδώκ υιός τούτου, Αχιμάας υιός τούτου.
\par 54 Και αύται είναι αι κατοικίαι αυτών κατά τας κώμας αυτών εν τοις ορίοις αυτών, των υιών του Ααρών, εκ της συγγενείας των Κααθιτών· διότι εις αυτούς έπεσεν ο κλήρος·
\par 55 και έδωκαν εις αυτούς την Χεβρών εν γη Ιούδα και τα περίχωρα αυτής κύκλω αυτής.
\par 56 Τους αγρούς όμως της πόλεως και τας κώμας αυτής έδωκαν εις τον Χάλεβ τον υιόν του Ιεφοννή.
\par 57 Εις δε τους υιούς του Ααρών έδωκαν τας πόλεις του Ιούδα, την Χεβρών, την πόλιν του καταφυγίου και την Λιβνά και τα περίχωρα αυτής και την Ιαθείρ και την Εσθεμωά και τα περίχωρα αυτής,
\par 58 και την Ηλών και τα περίχωρα αυτής, την Δεβείρ και τα περίχωρα αυτής,
\par 59 και την Ασάν και τα περίχωρα αυτής και την Βαιθ-σεμές και τα περίχωρα αυτής,
\par 60 και εκ της φυλής Βενιαμίν την Γαβαά και τα περίχωρα αυτής και την Αλεμέθ και τα περίχωρα αυτής και την Αναθώθ και τα περίχωρα αυτής· πάσαι αι πόλεις αυτών, κατά τας συγγενείας αυτών, δεκατρείς.
\par 61 Και εις τους υιούς του Καάθ, τους εναπολειφθέντας, εδόθησαν κατά κλήρον εκ της συγγενείας εκατέρας φυλής και εκ της ημισείας φυλής, της ημισείας του Μανασσή, δέκα πόλεις.
\par 62 Και εις τους υιούς του Γηρσώμ κατά τας συγγενείας αυτών, εκ της φυλής Ισσάχαρ, και εκ της φυλής Ασήρ και εκ της φυλής Νεφθαλί και εκ της φυλής του Μανασσή εν Βασάν, δεκατρείς πόλεις.
\par 63 εις τους υιούς του Μεραρί, κατά τας συγγενείας αυτών, εδόθησαν διά κλήρου εκ της φυλής Ρουβήν και εκ της φυλής Γαδ και εκ της φυλής Ζαβουλών δώδεκα πόλεις.
\par 64 Και οι υιοί Ισραήλ έδωκαν εις τους Λευΐτας τας πόλεις ταύτας και τα περίχωρα αυτών.
\par 65 Και έδωκαν κατά κλήρον, εκ της φυλής των υιών Ιούδα και εκ της φυλής των υιών Συμεών και εκ της φυλής των υιών Βενιαμίν, τας πόλεις ταύτας, ονομασθείσας κατά τα ονόματα αυτών.
\par 66 Οι δε εκ των συγγενειών των υιών Καάθ έλαβον πόλεις των ορίων αυτών εκ της φυλής Εφραΐμ.
\par 67 Και έδωκαν εις αυτούς τας πόλεις του καταφυγίου, την Συχέμ και τα περίχωρα αυτής, εν τω όρει Εφραΐμ, και την Γεζέρ και τα περίχωρα αυτής,
\par 68 και την Ιοκμεάμ και τα περίχωρα αυτής, και την Βαιθ-ωρών και τα περίχωρα αυτής,
\par 69 και την Αιαλών και τα περίχωρα αυτής, και την Γαθ-ριμμών και τα περίχωρα αυτής·
\par 70 και εκ της ημισείας φυλής Μανασσή την Ανήρ και τα περίχωρα αυτής, και την Βιλεάμ και τα περίχωρα αυτής· ταύτας έδωκαν εις τας συγγενείας των εναπολειφθέντων των υιών Καάθ.
\par 71 εις τους υιούς Γηρσώμ έδωκαν, εκ της συγγενείας της ημισείας φυλής Μανασσή, την Γωλάν εν Βασάν και τα περίχωρα αυτής, και την Ασταρώθ και τα περίχωρα αυτής·
\par 72 και εκ της φυλής Ισσάχαρ την Κεδές και τα περίχωρα αυτής, την Δαβράθ και τα περίχωρα αυτής,
\par 73 και την Ραμώθ και τα περίχωρα αυτής, και την Ανείμ και τα περίχωρα αυτής·
\par 74 και εκ της φυλής Ασήρ την Μασάλ και τα περίχωρα αυτής, και την Αβδών και τα περίχωρα αυτής,
\par 75 και την Χουκώκ και τα περίχωρα αυτής, και την Ρεώβ και τα περίχωρα αυτής·
\par 76 και εκ της φυλής Νεφθαλί την Κεδές εν Γαλιλαία και τα περίχωρα αυτής, και την Αμμών και τα περίχωρα αυτής, και την Κιριαθαΐμ και τα περίχωρα αυτής.
\par 77 Εις τους υιούς του Μεραρί τους εναπολειφθέντας έδωκαν, εκ της φυλής του Ζαβουλών, την Ριμμών και τα περίχωρα αυτής, την Θαβώρ και τα περίχωρα αυτής·
\par 78 εις δε το πέραν του Ιορδάνου πλησίον της Ιεριχώ, προς ανατολάς του Ιορδάνου, έδωκαν, εκ της φυλής Ρουβήν την Βοσόρ εν τη ερήμω και τα περίχωρα αυτής, και την Ιασά και τα περίχωρα αυτής,
\par 79 και την Κεδημώθ και τα περίχωρα αυτής, και την Μηφαάθ και τα περίχωρα αυτής·
\par 80 και εκ της φυλής Γαδ την Ραμώθ εν Γαλαάδ και τα περίχωρα αυτής, και την Μαχαναΐμ και τα περίχωρα αυτής,
\par 81 και την Εσεβών και τα περίχωρα αυτής, και την Ιαζήρ και τα περίχωρα αυτής.

\chapter{7}

\par 1 Οι δε υιοί του Ισσάχαρ ήσαν Θωλά και Φουά, Ιασούβ και Σιμβρών, τέσσαρες.
\par 2 Και οι υιοί του Θωλά, Οζί και Ρεφαΐα και Ιεριήλ και Ιαμαί και Ιεβσάμ και Σεμουήλ, αρχηγοί του οίκου των πατέρων αυτών εις τον Θωλά, δυνατοί εν ισχύϊ εις τας γενεάς αυτών· ο αριθμός αυτών ήτο εν ταις ημέραις του Δαβίδ, εικοσιδύο χιλιάδες και εξακόσιοι.
\par 3 Και οι υιοί του Οζί, Ιζραΐας· και οι υιοί του Ιζραΐα, Μιχαήλ και Οβαδία και Ιωήλ και Ιεσία, πέντε, αρχηγοί πάντες.
\par 4 Και μετ' αυτών, κατά τας γενεάς αυτών, κατά τους πατρικούς αυτών οίκους, ήσαν τάγματα παραταττόμενα εις πόλεμον, τριάκοντα εξ χιλιάδες άνδρες· διότι απέκτησαν πολλάς γυναίκας και υιούς.
\par 5 Και οι αδελφοί αυτών, μεταξύ πασών των οικογενειών του Ισσάχαρ, δυνατοί εν ισχύϊ, πάντες απαριθμηθέντες κατά τας γενεαλογίας αυτών, ογδοήκοντα επτά χιλιάδες.
\par 6 Οι υιοί του Βενιαμίν, Βελά και Βεχέρ και Ιεδιαήλ, τρεις.
\par 7 Και οι υιοί του Βελά, Εσβών και Οζί και Οζιήλ και Ιεριμώθ και Ιρί, πέντε, αρχηγοί πατρικών οίκων, δυνατοί εν ισχύϊ, απαριθμηθέντες κατά τας γενεαλογίας αυτών, ήσαν εικοσιδύο χιλιάδες και τριάκοντα τέσσαρες.
\par 8 Και οι υιοί του Βεχέρ, Ζεμιρά και Ιωάς και Ελιέζερ και Ελιωηνάϊ και Αμρί και Ιεριμώθ και Αβιά και Αναθώθ, και Αλαμέθ· πάντες ούτοι ήσαν οι υιοί του Βεχέρ.
\par 9 Και η γενεαλογική αυτών απαρίθμησις, κατά τας γενεάς αυτών, ήτο είκοσι χιλιάδες και διακόσιοι, αρχηγοί των πατρικών αυτών οίκων, δυνατοί εν ισχύϊ.
\par 10 Και οι υιοί του Ιεδιαήλ, Βαλαάν· και οι υιοί του Βαλαάν, Ιεούς και Βενιαμίν και Εχούδ και Χαναανά και Ζηθάν και Θαρσείς και Αχισσάρ·
\par 11 πάντες ούτοι οι υιοί του Ιεδιαήλ, αρχηγοί πατριών, δυνατοί εν ισχύϊ, ήσαν δεκαεπτά χιλιάδες και διακόσιοι, δυνάμενοι να εκστρατεύωσιν εις πόλεμον.
\par 12 Και Σουφίμ και Ουπίμ, υιοί του Ιρ· και υιοί του Αχήρ, Ουσίμ.
\par 13 Οι υιοί του Νεφθαλί, Ιασιήλ και Γουνί και Ιεσέρ και Σαλλούμ, υιοί της Βαλλάς.
\par 14 Οι υιοί του Μανασσή, Ασριήλ, τον οποίον η γυνή αυτού εγέννησεν· η δε παλλακή αυτού η Σύριος εγέννησε τον Μαχείρ πατέρα του Γαλαάδ·
\par 15 και ο Μαχείρ έλαβεν εις γυναίκα την αδελφήν του Ουπίμ και Σουφίμ· και το όνομα της αδελφής αυτών ήτο Μααχά· του δευτέρου δε το όνομα ήτο Σαλπαάδ· και ο Σαλπαάδ εγέννησε θυγατέρας.
\par 16 Και η Μααχά η γυνή του Μαχείρ εγέννησεν υιόν και εκάλεσε το όνομα αυτού Φαρές· και το όνομα του αδελφού αυτού ήτο Σαρές· και οι υιοί αυτού, Ουλάμ και Ρακέμ.
\par 17 Και οι υιοί του Ουλάμ, Βεδάν. Ούτοι ήσαν οι υιοί του Γαλαάδ, υιού του Μαχείρ, υιού του Μανασσή.
\par 18 Και η αδελφή αυτού Αμμολεκέθ εγέννησε τον Ισούδ και τον Αβιέζερ και τον Μααλά.
\par 19 Και οι υιοί του Σεμιδά ήσαν Αχιάν και Συχέμ και Λικχί και Ανιάμ.
\par 20 Και οι υιοί του Εφραΐμ, Σουθαλά και Βερέδ υιός τούτου, και Ταχάθ υιός τούτου, και Ελεαδά υιός τούτου, και Ταχάθ υιός τούτου,
\par 21 και Ζαβάδ υιός τούτου, και Σουθαλά υιός τούτου, και Εσέρ και Ελεάδ· εθανάτωσαν δε αυτούς οι άνδρες της Γαθ, οι γεννηθέντες εν τω τόπω, διότι κατέβησαν να λάβωσι τα κτήνη αυτών.
\par 22 Και Εφραΐμ ο πατήρ αυτών επένθησεν ημέρας πολλάς, και ήλθον οι αδελφοί αυτού διά να παρηγορήσωσιν αυτόν.
\par 23 Ύστερον εισήλθε προς την γυναίκα αυτού, ήτις συνέλαβε και εγέννησεν υιόν· και εκάλεσε το όνομα αυτού Βεριά, επειδή εγεννήθη εν συμφορά συμβάση εν τω οίκω αυτού.
\par 24 Η δε θυγάτηρ αυτού ήτο Σεερά, ήτις ωκοδόμησε την Βαιθ-ωρών, την κάτω και την άνω, και την Ουζέν-σεερά.
\par 25 Και Ρεφά ήτο υιός τούτου, και Ρεσέφ και Θελά υιοί τούτου, και Ταχάν υιός τούτου,
\par 26 Λααδάν υιός τούτου, Αμμιούδ υιός τούτου, Ελισαμά υιός τούτου,
\par 27 Ναυή υιός τούτου, Ιησούς υιός τούτου.
\par 28 Αι δε ιδιοκτησίαι αυτών και αι κατοικίαι αυτών ήσαν Βαιθήλ και αι κώμαι αυτής, και κατά ανατολάς Νααράν, και κατά δυσμάς Γέζερ και αι κώμαι αυτής, και Συχέμ και αι κώμαι αυτής, έως Γάζης και των κωμών αυτής·
\par 29 και εις τα όρια των υιών Μανασσή Βαιθ-σαν και αι κώμαι αυτής, Θαανάχ και αι κώμαι αυτής, Μεγιδδώ και αι κώμαι αυτής, Δωρ και αι κώμαι αυτής. Εν ταύταις κατώκησαν οι υιοί Ιωσήφ υιού Ισραήλ.
\par 30 Οι υιοί του Ασήρ, Ιεμνά και Ιεσσουά και Ιεσσουάϊ και Βεριά και Σερά η αδελφή αυτών.
\par 31 Και οι υιοί του Βεριά, Έβερ και Μαλχήλ, όστις είναι ο πατήρ Βιρζαβίθ.
\par 32 Και ο Έβερ εγέννησε τον Ιαφλήτ και τον Σωμήρ και τον Χωθάμ και την Σουά την αδελφήν αυτών.
\par 33 Και οι υιοί του Ιαφλήτ, Φασάχ και Βιμάλ και Ασουάθ· ούτοι είναι οι υιοί του Ιαφλήτ.
\par 34 Και οι υιοί του Σωμήρ, Αχί και Ρωγά, Ιεχουβά και Αράμ.
\par 35 Και οι υιοί Ελέμ του αδελφού αυτού, Σωφά και Ιεμνά και Σελλής και Αμάλ.
\par 36 Οι υιοί του Σωφά, Σουά και Αρνεφέρ και Σωγάλ και Βερί και Ιεμρά,
\par 37 Βοσόρ και Ωδ και Σαμμά και Σιλισά και Ιθράν και Βεηρά.
\par 38 Και οι υιοί του Ιεθέρ, Ιεφοννή και Φισπά και Αρά.
\par 39 Και οι υιοί του Ουλλά, Αράχ και Ανιήλ και Ρισιά.
\par 40 Πάντες ούτοι ήσαν οι υιοί του Ασήρ, αρχηγοί πατρικών οίκων, εκλεκτοί, δυνατοί εν ισχύϊ, πρώτοι αρχηγοί. Και ο αριθμός αυτών, κατά την γενεαλογίαν αυτών, όσοι ήσαν άξιοι να παραταχθώσιν εις μάχην, ήτο εικοσιέξ χιλιάδες άνδρες.

\chapter{8}

\par 1 Ο δε Βενιαμίν εγέννησε Βελά τον πρωτότοκον αυτού, Ασβήλ τον δεύτερον και Ααρά τον τρίτον,
\par 2 Νωά τον τέταρτον και Ραφά τον πέμπτον.
\par 3 Και οι υιοί του Βελά ήσαν, Αδδάρ και Γηρά και Αβιούδ
\par 4 και Αβισσουά και Νααμάν και Αχωά
\par 5 και Γηρά και Σεφουφάν και Ουράμ.
\par 6 Και ούτοι είναι οι υιοί του Εχούδ, οίτινες ήσαν αρχηγοί πατριών εις τους κατοικούντας την Γαβαά και μετοικισθέντας εις Μαναχάθ·
\par 7 και Νααμάν και Αχιά και Γηρά, όστις μετώκισεν αυτούς, και εγέννησε τον Ουζά και τον Αχιούδ.
\par 8 Και ο Σααραΐμ εγέννησεν υιούς εν τη γη Μωάβ, αφού απέβαλε την Ουσίμ και την Βααρά, τας γυναίκας αυτού·
\par 9 και εγέννησεν, εκ της Οδές της γυναικός αυτού, τον Ιωβάβ και τον Σιβιά και τον Μησά και τον Μαλχάμ
\par 10 και τον Ιεούς και τον Σαχιά και τον Μιρμά· ούτοι ήσαν οι υιοί αυτού, αρχηγοί πατριών.
\par 11 Εκ δε της Ουσίμ είχε γεννήσει τον Αβιτώβ και τον Ελφαάλ.
\par 12 Και οι υιοί του Ελφαάλ ήσαν Έβερ και Μισαάμ και Σαμέρ, όστις ωκοδόμησε την Ωνώ και την Λωδ και τας κώμας αυτής·
\par 13 και ο Βεριά και ο Σεμά ούτοι ήσαν αρχηγοί πατριών εις τους κατοικούντας την Αιαλών· ούτοι εξεδίωξαν τους κατοίκους της Γάθ·
\par 14 και Αχιώ, Σασάκ και Ιερεμώθ
\par 15 και Ζεβαδίας και Αράδ και Αδέρ,
\par 16 και Μιχαήλ και Ιεσπά και Ιωχά υιοί του Βεριά·
\par 17 και Ζεβαδίας και Μεσουλλάμ και Εζεκί και Έβερ
\par 18 και Ισμεραΐ και Ιεζλιά και Ιωβάβ, υιοί του Ελφαάλ·
\par 19 και Ιακείμ και Ζιχρί και Ζαβδί
\par 20 και Ελιηνάϊ και Ζιλθαΐ και Ελιήλ
\par 21 και Αδαΐας και Βεραΐα και Σιμράθ, υιοί του Σεμά·
\par 22 και Ιεσφάν και Έβερ και Ελιήλ
\par 23 και Αβδών και Ζιχρί και Ανάν
\par 24 και Ανανίας και Ελάμ και Ανθωθιά
\par 25 και Ιεφεδία και Φανουήλ υιοί του Σασάκ·
\par 26 και Σαμσεραΐ και Σεαρία και Γοθολία
\par 27 και Ιαρεσία και Ηλιά και Ζιχρί, υιοί του Ιεροάμ.
\par 28 ούτοι ήσαν αρχηγοί πατριών, αρχηγοί κατά τας γενεάς αυτών. ούτοι κατώκησαν εν Ιερουσαλήμ.
\par 29 Εν δε Γαβαών κατώκησεν ο πατήρ Γαβαών, το δε όνομα της γυναικός αυτού ήτο Μααχά·
\par 30 και ο πρωτότοκος υιός αυτού ήτο Αβδών, έπειτα Σούρ και Κείς και Βάαλ και Ναδάβ
\par 31 και Γεδώρ και Αχιώ και Ζαχέρ
\par 32 και Μικλώθ ο γεννήσας τον Σιμεά. Και ούτοι έτι κατώκησαν μετά των αδελφών αυτών εν Ιερουσαλήμ, κατέναντι των αδελφών αυτών.
\par 33 Και ο Νηρ εγέννησε τον Κείς, και Κείς εγέννησε τον Σαούλ, και Σαούλ εγέννησε τον Ιωνάθαν και τον Μαλχί-σουέ και τον Αβιναδάβ και τον Εσ-βαάλ.
\par 34 Και ο υιός του Ιωνάθαν ήτο ο Μερίβ-βαάλ· και ο Μερίβ-βαάλ εγέννησε τον Μιχά.
\par 35 Και οι υιοί του Μιχά ήσαν Φιθών και Μελέχ και Θαρεά και Άχαζ.
\par 36 Και ο Άχαζ εγέννησε τον Ιωαδά· και ο Ιωαδά εγέννησε τον Αλεμέθ και τον Αζμαβέθ και τον Ζιμβρί· και Ζιμβρί εγέννησε τον Μοσά.
\par 37 και Μοσά εγέννησε τον Βινεά· Ραφά, υιός τούτου· Ελεασά, υιός τούτου· Ασήλ, υιός τούτου.
\par 38 Και ο Ασήλ είχεν εξ υιούς, των οποίων τα ονόματα είναι ταύτα· Αζρικάμ, Βοχερού και Ισμαήλ και Σεαρία και Οβαδία και Ανάν· πάντες ούτοι ήσαν οι υιοί του Ασήλ.
\par 39 Και οι υιοί του Ησέκ του αδελφού αυτού ήσαν Ουλάμ ο πρωτότοκος αυτού, Ιεούς ο δεύτερος και Ελιφελέτ ο τρίτος.
\par 40 Και οι υιοί του Ουλάμ ήσαν άνδρες δυνατοί εν ισχύϊ, εντείνοντες τόξον και έχοντες πολλούς υιούς και υιούς υιών, εκατόν πεντήκοντα. Πάντες ούτοι ήσαν εκ των υιών Βενιαμίν.

\chapter{9}

\par 1 Ούτω πας ο Ισραήλ απηριθμήθη κατά γενεαλογίας· και ιδού, είναι καταγεγραμμένοι εν τω βιβλίω των βασιλέων του Ισραήλ και Ιούδα. Μετωκίσθησαν δε εις την Βαβυλώνα διά τας ανομίας αυτών.
\par 2 Και οι πρώτοι κάτοικοι οι εν ταις ιδιοκτησίαις αυτών, εν ταις πόλεσιν αυτών, ήσαν οι Ισραηλίται, οι ιερείς, οι Λευΐται και οι Νεθινείμ.
\par 3 Και εν Ιερουσαλήμ κατώκησαν εκ των υιών Ιούδα και εκ των υιών Βενιαμίν και εκ των υιών Εφραΐμ και Μανασσή,
\par 4 Γουθαΐ ο υιός του Αμμιούδ, υιού του Αμρί, υιού του Ιμρί, υιού του Βανί, εκ των υιών του Φαρές υιού του Ιούδα.
\par 5 Και εκ των Σηλωνιτών, Ασαΐας ο πρωτότοκος και οι υιοί αυτού.
\par 6 Και εκ των υιών του Ζερά, Ιεουήλ και οι αδελφοί αυτών, εξακόσιοι ενενήκοντα.
\par 7 Και εκ των υιών Βενιαμίν, Σαλλού ο υιός του Μεσουλλάμ, υιού του Ωδουΐα, υιού του Ασενουά,
\par 8 και ο Ιεβνιά υιός του Ιεροάμ, και ο Ηλά υιός του Οζί, υιού του Μιχρί, και ο Μεσουλλάμ υιός του Σεφατία, υιού του Ραγουήλ, υιού του Ιβνιά·
\par 9 και οι αδελφοί αυτών, κατά τας γενεάς αυτών, εννεακόσιοι πεντήκοντα εξ. Πάντες ούτοι οι άνδρες ήσαν αρχηγοί πατριών, κατά τους πατρικούς οίκους αυτών.
\par 10 Και εκ των ιερέων, Ιεδαΐας και Ιωϊαρείβ και Ιαχείν
\par 11 και Αζαρίας ο υιός του Χελκία, υιού του Μεσουλλάμ, υιού του Σαδώκ, υιού του Μεραϊώθ, υιού του Αχιτώβ, άρχων του οίκου του Θεού·
\par 12 και Αδαΐας ο υιός του Ιεροάμ, υιού του Πασχώρ, υιού του Μαλχίου, και Μαασαί ο υιός του Αδιήλ, υιού του Ιαζηρά, υιού του Μεσουλλάμ, υιού του Μεσιλλεμίθ, υιού του Ιμμήρ·
\par 13 και οι αδελφοί αυτών, αρχηγοί των πατρικών οίκων αυτών, χίλιοι επτακόσιοι εξήκοντα, δυνατοί εν ισχύϊ, άξιοι διά το έργον της υπηρεσίας του οίκου του Κυρίου.
\par 14 και εκ των Λευϊτών, Σεμαΐας ο υιός του Ασσούβ, υιού του Αζρικάμ, υιού του Ασαβία, εκ των υιών Μεραρί·
\par 15 και Βακβακάρ, Ερές και Γαλάλ και Ματθανίας ο υιός του Μιχά, υιού του Ζιχρί, υιού του Ασάφ·
\par 16 και Οβαδία ο υιός του Σεμαΐα, υιού του Γαλάλ, υιού του Ιεδουθούν, και Βαραχίας ο υιός του Ασά, υιού του Ελκανά, ο κατοικήσας εν ταις κώμαις των Νετωφαθιτών.
\par 17 οι δε θυρωροί ήσαν Σαλλούμ και Ακχούβ και Ταλμών και Αχιμάν και οι αδελφοί αυτών· ο Σαλλούμ ήτο ο άρχων·
\par 18 ούτοι μέχρι του νυν ήσαν εν τη πύλη του βασιλέως κατά ανατολάς θυρωροί κατά τα τάγματα των υιών του Λευΐ.
\par 19 Και Σαλλούμ ο υιός του Κωρή, υιού του Εβιασάφ, υιού του Κορέ, και οι αδελφοί αυτού, εκ του οίκου του πατρός αυτού, οι Κορίται, ήσαν επί το έργον της υπηρεσίας, φύλακες των πυλών της σκηνής· και οι πατέρες αυτών, εν τω στρατοπέδω του Κυρίου, ήσαν φύλακες της εισόδου.
\par 20 Και Φινεές ο υιός του Ελεάζαρ, μετά του οποίου ήτο ο Κύριος, ήτο άρχων επ' αυτούς το πρότερον.
\par 21 Ζαχαρίας ο υιός του Μεσελεμία ήτο πυλωρός της θύρας της σκηνής του μαρτυρίου.
\par 22 Πάντες ούτοι, οι εκλελεγμένοι διά να ήναι πυλωροί των θυρών, ήσαν διακόσιοι δώδεκα. Ούτοι ήσαν απηριθμημένοι κατά γενεαλογίας εν ταις κώμαις αυτών, τους οποίους ο Δαβίδ και ο Σαμουήλ ο βλέπων είχον καταστήσει εις το υπούργημα αυτών.
\par 23 Και αυτοί και οι υιοί αυτών είχον την επιστασίαν των πυλών του οίκου του Κυρίου, του οίκου της σκηνής, διά να φυλάττωσι.
\par 24 προς τους τέσσαρας ανέμους ήσαν οι πυλωροί, προς ανατολάς, προς δυσμάς, προς βορράν και προς νότον.
\par 25 Και οι αδελφοί αυτών, οι εν ταις κώμαις αυτών, έπρεπε να έρχωνται κατά επτά ημέρας εις τους διωρισμένους καιρούς μετά τούτων.
\par 26 Διότι οι Λευΐται ούτοι, οι τέσσαρες αρχιπύλωροι, έμενον εις το υπούργημα αυτών και είχον την επιστασίαν των οικημάτων και των θησαυρών του οίκου του Θεού.
\par 27 Και διενυκτέρευον πέριξ του οίκου του Θεού, διότι η φυλακή ήτο επ' αυτούς, και αυτοί έπρεπε να ανοίγωσιν αυτόν καθ' εκάστην πρωΐαν.
\par 28 Και τινές εξ αυτών είχον την επιστασίαν των λειτουργικών σκευών, διότι κατά αριθμόν εισέφερον αυτά και κατά αριθμόν εξέφερον αυτά.
\par 29 Εξ αυτών έτι ήσαν διωρισμένοι επί των άλλων σκευών και επί πάντων των σκευών των ιερών και επί της σεμιδάλεως και του οίνου και του ελαίου και του θυμιάματος και των αρωμάτων.
\par 30 Και τινές εκ των υιών των ιερέων κατεσκεύαζον το μύρον το αρωματικόν.
\par 31 Και Ματταθίας, ο εκ των Λευϊτών, ο πρωτότοκος Σαλλούμ του Κορίτου, είχε την επιστασίαν των τηγανιζομένων πραγμάτων.
\par 32 Και άλλοι εκ των αδελφών αυτών, εκ των υιών των Κααθιτών, ήσαν επί των άρτων της προθέσεως, διά να ετοιμάζωσιν αυτούς κατά σάββατον.
\par 33 Και εκ τούτων ήσαν οι ψαλτωδοί, αρχηγοί πατριών των Λευϊτών, οίτινες έμενον εν τοις οικήμασιν ελεύθεροι· διότι ενησχολούντο εις το έργον τούτο ημέραν και νύκτα.
\par 34 ούτοι ήσαν οι αρχηγοί των πατριών των Λευϊτών, κατά τας γενεάς αυτών· ούτοι οι αρχηγοί κατώκουν εν Ιερουσαλήμ.
\par 35 Και εν Γαβαών κατώκησεν ο πατήρ Γαβαών, ο Ιεχιήλ, το δε όνομα της γυναικός αυτού ήτο Μααχά·
\par 36 και ο πρωτότοκος αυτού υιός ήτο Αβδών, έπειτα Σούρ και Κείς και Βάαλ και Νηρ και Ναδάβ
\par 37 και Γεδώρ και Αχιώ και Ζαχαρίας και Μικλώθ·
\par 38 και ο Μικλώθ εγέννησε τον Σιμεάμ. Και ούτοι έτι κατώκησαν μετά των αδελφών αυτών εν Ιερουσαλήμ, αντικρύ των αδελφών αυτών.
\par 39 Και ο Νηρ εγέννησε τον Κείς, και ο Κείς εγέννησε τον Σαούλ, και ο Σαούλ εγέννησε τον Ιωνάθαν και τον Μαλχί-σουέ και τον Αβιναδάβ και τον Εσ-βαάλ.
\par 40 Και ο υιός του Ιωνάθαν ήτο ο Μερίβ-βαάλ· και ο Μερίβ-βαάλ εγέννησε τον Μιχά.
\par 41 Και οι υιοί του Μιχά ήσαν Φιθών και Μελέχ και Θαρεά,
\par 42 και Άχαζ ο γεννήσας τον Ιαρά· και Ιαρά εγέννησε τον Αλεμέθ, και τον Αζμαβέθ και τον Ζιμβρί· και Ζιμβρί εγέννησε τον Μοσά.
\par 43 και Μοσά εγέννησε τον Βινεά· και Ρεφαΐα ήτο υιός τούτου. ο Ελεασά υιός τούτου· Ασήλ υιός τούτου.
\par 44 Ο δε Ασήλ είχεν εξ υιούς, των οποίων τα ονόματα είναι ταύτα· Αζρικάμ, Βοχερού και Ισμαήλ και Σεαρία και Οβαδία και Ανάν· ούτοι ήσαν οι υιοί του Ασήλ.

\chapter{10}

\par 1 Οι δε Φιλισταίοι επολέμουν κατά του Ισραήλ· και έφυγον οι άνδρες του Ισραήλ από προσώπου των Φιλισταίων και έπεσον πεφονευμένοι εν τω όρει Γελβουέ.
\par 2 Και καταφθάσαντες οι Φιλισταίοι οπίσω του Σαούλ και οπίσω των υιών αυτού, επάταξαν οι Φιλισταίοι τον Ιωνάθαν και τον Αβιναδάβ και τον Μαλχί-σουέ, τους υιούς του Σαούλ.
\par 3 Εβάρυνε δε η μάχη επί τον Σαούλ, και επέτυχον αυτού οι τοξόται, και επληγώθη υπό των τοξοτών.
\par 4 Και είπεν ο Σαούλ προς τον οπλοφόρον αυτού, Σύρε την ρομφαίαν σου και διαπέρασόν με δι' αυτής, διά να μη έλθωσιν ούτοι οι απερίτμητοι και με εμπαίξωσι· πλην ο οπλοφόρος αυτού δεν ήθελε· διότι εφοβείτο σφόδρα. Όθεν έλαβεν ο Σαούλ την ρομφαίαν και έπεσεν επ' αυτήν.
\par 5 Και ως είδεν ο οπλοφόρος αυτού ότι απέθανεν ο Σαούλ, έπεσε και αυτός επί την ρομφαίαν και απέθανεν·
\par 6 ούτως απέθανεν ο Σαούλ και οι τρεις υιοί αυτού· και πας ο οίκος αυτού απέθανεν ομού.
\par 7 Και πάντες οι άνδρες Ισραήλ, οι εν τη κοιλάδι, ιδόντες ότι έφευγον και ότι ο Σαούλ και οι υιοί αυτού απέθανον, τότε κατέλιπον τας πόλεις αυτών και έφυγον· και ελθόντες οι Φιλισταίοι κατώκησαν εν αυταίς.
\par 8 Και την επαύριον, ότε ήλθον οι Φιλισταίοι διά να εκδύσωσι τους πεφονευμένους, εύρηκαν τον Σαούλ και τους υιούς αυτού πεπτωκότας εν τώ όρει Γελβουέ.
\par 9 Και εξέδυσαν αυτόν και έλαβον την κεφαλήν αυτού και τα όπλα αυτού και απέστειλαν εις την γην των Φιλισταίων κύκλω, διά να διαδώσωσι την αγγελίαν εις τα είδωλα αυτών και εις τον λαόν.
\par 10 Και ανέθεσαν τα όπλα αυτού εις τον οίκον των θεών αυτών· και εις τον ναόν του Δαγών προσήλωσαν την κεφαλήν αυτού.
\par 11 Ακούσαντες δε πάντες οι κάτοικοι της Ιαβείς-γαλαάδ πάντα όσα έκαμον οι Φιλισταίοι εις τον Σαούλ,
\par 12 ηγέρθησαν πάντες οι δυνατοί άνδρες και εσήκωσαν το σώμα του Σαούλ και τα σώματα των υιών αυτού και έφεραν αυτά εις Ιαβείς, και έθαψαν τα οστά αυτών υπό την δρυν εν Ιαβείς και ενήστευσαν επτά ημέρας.
\par 13 Ούτως απέθανεν ο Σαούλ, διά την ανομίαν αυτού την οποίαν ηνόμησεν εις τον Κύριον, εναντίον του λόγου του Κυρίου, τον οποίον δεν εφύλαξε· και έτι διότι εζήτησεν άνθρωπον έχοντα πνεύμα μαντείας, διά να ερωτήση,
\par 14 και δεν ηρώτησε τον Κύριον· διά τούτο εθανάτωσεν αυτόν και έστρεψε την βασιλείαν εις τον Δαβίδ τον υιόν του Ιεσσαί.

\chapter{11}

\par 1 Τότε συνήχθη πας ο Ισραήλ προς τον Δαβίδ εις Χεβρών, λέγοντες, Ιδού, οστούν σου και σαρξ σου είμεθα.
\par 2 Και πρότερον έτι και ότε εβασίλευεν ο Σαούλ, συ ήσο ο εξάγων και εισάγων τον Ισραήλ· και προς σε είπε Κύριος ο Θεός σου, συ θέλεις ποιμάνει τον λαόν μου τον Ισραήλ, και συ θέλεις είσθαι ηγεμών επί τον λαόν μου τον Ισραήλ.
\par 3 Και ήλθον πάντες οι πρεσβύτεροι του Ισραήλ προς τον βασιλέα εις Χεβρών· και έκαμεν ο Δαβίδ συνθήκην μετ' αυτών εν Χεβρών ενώπιον του Κυρίου· και έχρισαν τον Δαβίδ βασιλέα επί τον Ισραήλ, κατά τον λόγον του Κυρίου τον λαληθέντα διά του Σαμουήλ.
\par 4 Και υπήγον ο Δαβίδ και πας ο Ισραήλ εις Ιερουσαλήμ, ήτις είναι η Ιεβούς, όπου ήσαν οι Ιεβουσαίοι, οι κατοικούντες την γην.
\par 5 Και οι κάτοικοι της Ιεβούς είπον προς τον Δαβίδ, Δεν θέλεις εισέλθει ενταύθα. Αλλ' ο Δαβίδ εκυρίευσε το φρούριον Σιών, ήτις είναι η πόλις Δαβίδ.
\par 6 Και είπεν ο Δαβίδ, Όστις πρώτος πατάξη τους Ιεβουσαίους, θέλει είσθαι αρχηγός και στρατηγός. Πρώτος δε ανέβη ο Ιωάβ, ο υιός της Σερουΐας, και έγεινεν αρχηγός.
\par 7 Και κατώκησεν ο Δαβίδ εν τω φρουρίω· όθεν ωνόμασαν αυτήν πόλιν Δαβίδ.
\par 8 Και ωκοδόμησε την πόλιν κυκλόθεν από Μιλλώ και κύκλω· και επεσκεύασεν ο Ιωάβ το επίλοιπον της πόλεως.
\par 9 Και προεχώρει ο Δαβίδ μεγαλυνόμενος· και ο Κύριος των δυνάμεων ήτο μετ αυτού.
\par 10 Ούτοι δε ήσαν οι αρχηγοί των ισχυρών, τους οποίους είχεν ο Δαβίδ, οίτινες ηγωνίσθησαν μετ' αυτού διά την βασιλείαν αυτού, μετά παντός του Ισραήλ, διά να κάμωσιν αυτόν βασιλέα, κατά τον λόγον του Κυρίου τον περί του Ισραήλ.
\par 11 Και ούτος είναι ο αριθμός των ισχυρών τους οποίους είχεν ο Δαβίδ· Ιασωβεάμ ο υιός του Αχμονί, πρώτος των οπλαρχηγών. Ούτος σείων την λόγχην αυτού εναντίον τριακοσίων, εθανάτωσεν αυτούς εν μιά μάχη.
\par 12 Και μετ' αυτόν Ελεάζαρ ο υιός του Δωδώ· ο Αχωχίτης, όστις ήτο εις εκ των τριών ισχυρών.
\par 13 Ούτος ήτο μετά του Δαβίδ εν Φασ-δαμμείμ, και οι Φιλισταίοι συνηθροίσθησαν εκεί διά πόλεμον, όπου ήτο μερίδιον αγρού πλήρες κριθής· ο δε λαός έφυγεν από προσώπου των Φιλισταίων.
\par 14 Και ούτοι εστηλώθησαν εν τω μέσω του μεριδίου και ηλευθέρωσαν αυτό και επάταξαν τους Φιλισταίους· και ο Κύριος έκαμε σωτηρίαν μεγάλην.
\par 15 Κατέβησαν έτι τρεις εκ των τριάκοντα αρχηγών εις την πέτραν προς τον Δαβίδ, εις το σπήλαιον Οδολλάμ· το δε στρατόπεδον των Φιλισταίων εστρατοπέδευεν εν τη κοιλάδι Ραφαείμ.
\par 16 Και ο Δαβίδ ήτο τότε εν τω οχυρώματι και η φρουρά των Φιλισταίων τότε ο εν Βηθλεέμ.
\par 17 Και επεπόθησεν ο Δαβίδ ύδωρ και είπε, Τις ήθελε μοι δώσει να πίω ύδωρ εκ του φρέατος της Βηθλεέμ, του εν τη πύλη;
\par 18 Και οι τρεις διασχίσαντες το στρατόπεδον των Φιλισταίων, ήντλησαν ύδωρ εκ του φρέατος της Βηθλεέμ του εν τη πύλη, και λαβόντες έφεραν προς τον Δαβίδ· πλην ο Δαβίδ δεν ηθέλησε να πίη αυτό, αλλ' έκαμεν αυτό σπονδήν εις τον Κύριον,
\par 19 λέγων, Μη γένοιτο εις εμέ παρά του Θεού μου να κάμω τούτο· θέλω πίει το αίμα των ανδρών τούτων, οίτινες εξέθεσαν την ζωήν αυτών εις κίνδυνον; διότι μετά κινδύνου της ζωής αυτών έφεραν αυτό. Διά τούτο δεν ηθέλησε να πίη αυτό· ταύτα έκαμον οι τρεις ισχυροί.
\par 20 Και Αβισαί ο αδελφός του Ιωάβ, ούτος ήτο πρώτος των τριών· και ούτος σείων την λόγχην αυτού εναντίον τριακοσίων, εθανάτωσεν αυτούς και απέκτησεν όνομα μεταξύ των τριών.
\par 21 Εκ των τριών, ήτο ενδοξότερος υπέρ τους δύο και έγεινεν αρχηγός αυτών· δεν έφθασεν όμως μέχρι των τριών πρώτων.
\par 22 Βεναΐας ο υιός του Ιωδαέ, ο υιός ανδρός δυνατού από Καβσεήλ, όστις έκαμε πολλά ανδραγαθήματα, ούτος επάταξε τους δύο λεοντώδεις άνδρας του Μωάβ· ούτος έτι κατέβη και επάταξε λέοντα εν μέσω του λάκκου εν ημέρα χιόνος·
\par 23 ούτος έτι επάταξε τον άνδρα τον Αιγύπτιον, άνδρα μεγάλου αναστήματος, πεντάπηχον· και εν τη χειρί του Αιγυπτίου ήτο λόγχη ως αντίον υφαντού· κατέβη δε προς αυτόν με ράβδον, και αρπάσας την λόγχην εκ της χειρός του Αιγυπτίου εθανάτωσεν αυτόν διά της ιδίας αυτού λόγχης·
\par 24 ταύτα έκαμε Βεναΐας ο υιός του Ιωδαέ, και απέκτησεν όνομα μεταξύ των τριών ισχυρών·
\par 25 ιδού, αυτός εστάθη ενδοξότερος των τριάκοντα, δεν έφθασεν όμως μέχρι των τριών πρώτων· και κατέστησεν αυτόν ο Δαβίδ επί των δορυφόρων αυτού.
\par 26 Και οι ισχυροί των στρατευμάτων ήσαν Ασαήλ ο αδελφός του Ιωάβ, Ελχανάν ο υιός του Δωδώ εκ της Βηθλεέμ,
\par 27 Σαμμώθ ο Αρουρίτης, Χελής ο Φελωνίτης,
\par 28 Ιράς ο υιός του Ικκής ο Θεκωΐτης, Αβιέζερ ο Αναθωθίτης,
\par 29 Σιββεχαΐ ο Χουσαθίτης, Ιλαΐ ο Αχωχίτης,
\par 30 Μααραΐ ο Νετωφαθίτης, Χελέδ ο υιός του Βαανά Νετωφαθίτης,
\par 31 Ιτθαΐ ο υιός του Ριβαί εκ της Γαβαά των υιών Βενιαμίν, Βεναΐας ο Πιραθωνίτης,
\par 32 Ουραί εκ των κοιλάδων Γαάς, Αβιήλ ο Αρβαθίτης,
\par 33 Αζμαβέθ ο Βααρουμίτης, Ελιαβά ο Σααλβωνίτης,
\par 34 οι υιοί του Ασήμ του Γιζονίτου, Ιωνάθαν ο υιός του Σαγή ο Αραρίτης,
\par 35 Αχιάμ υιός του Σαχάρ ο Αραρίτης, Ελιφάλ υιός του Ουρ,
\par 36 Εφέρ ο Μεχηραθίτης, Αχιά ο Φελωνίτης,
\par 37 Εσρώ ο Καρμηλίτης, Νααραί ο υιός του Εσβαί,
\par 38 Ιωήλ ο αδελφός του Νάθαν, Μιβάρ ο υιός του Αγηρί,
\par 39 Σελέκ ο Αμμωνίτης, Νααραί ο Βηρωθαίος, ο οπλοφόρος του Ιωάβ υιού της Σερουΐας,
\par 40 Ιράς ο Ιεθρίτης, Γαρήβ Ιεθρίτης,
\par 41 Ουρίας ο Χετταίος, Ζαβάδ ο υιός του Ααλαί,
\par 42 Αδινά ο υιός του Σιζά του Ρουβηνίτου, άρχων των Ρουβηνιτών, και τριάκοντα μετ' αυτού,
\par 43 Ανάν ο υιός του Μααχά και Ιωσαφάτ ο Μιθνίτης,
\par 44 Οζίας ο Αστερωθίτης, Σαμά και Ιεχιήλ οι υιοί του Χωθάν του Αροηρίτου,
\par 45 Ιεδιαήλ ο υιός του Σιμρί και Ιωχά αδελφός αυτού ο Θισίτης,
\par 46 Ελιήλ ο Μααβίτης και Ιεριβαί και Ιωσαυϊά, οι υιοί του Ελναάμ, και Ιεθεμά ο Μωαβίτης,
\par 47 Ελιήλ και Ωβήδ και Ιασιήλ ο Μεσωβαΐτης.

\chapter{12}

\par 1 Ούτοι δε είναι οι ελθόντες προς τον Δαβίδ εις Σικλάγ, ενώ ήτο έτι κεκλεισμένος από προσώπου του Σαούλ υιού του Κείς, και ούτοι ήσαν εκ των ισχυρών, βοηθούντες εν πολέμω,
\par 2 ώπλισμένοι τόξα, μεταχειριζόμενοι και την δεξιάν και την αριστεράν εις το να τοξεύωσι λίθους και βέλη διά του τόξου, όντες εκ των αδελφών του Σαούλ, εκ του Βενιαμίν·
\par 3 ο αρχηγός Αχιέζερ, έπειτα Ιωάς, υιοί του Σεμαά του Γαβααθίτου· και Ιεζιήλ και Φελέτ, υιοί του Αζμαβέθ· και Βεραχά και Ιηού ο Αναθωθίτης·
\par 4 και Ισμαΐα ο Γαβαωνίτης, ισχυρός μεταξύ των τριάκοντα και επί των τριάκοντα· και Ιερεμίας και Ιααζιήλ και Ιωανάν και Ιωζαβάδ ο Γεδηρωθίτης,
\par 5 Ελουζαΐ και Ιεριμώθ και Βααλία και Σεμαρίας και Σεφατίας ο Αρουφίτης
\par 6 Ελκανά και Ιεσιά και Αζαρεήλ και Ιωεζέρ και Ιασωβεάμ, οι Κορίται,
\par 7 και Ιωηλά και Ζεβαδίας, οι υιοί του Ιεροάμ από Γεδώρ.
\par 8 Και εκ των Γαδιτών εχωρίσθησαν τινές και ήλθον προς τον Δαβίδ εις το οχύρωμα εν τη ερήμω, δυνατοί εν ισχύϊ, άνδρες παρατάξεως πολέμου, θυρεοφόροι και λογχοφόροι, και τα πρόσωπα αυτών πρόσωπα λέοντος, εις δε την ταχύτητα ως αι δορκάδες επί των ορέων·
\par 9 Εσέρ ο άρχων, Οβαδία ο δεύτερος, Ελιάβ ο τρίτος,
\par 10 Μισμανά ο τέταρτος, Ιερεμίας ο πέμπτος,
\par 11 Ατθαΐ ο έκτος, Ελιήλ ο έβδομος,
\par 12 Ιωανάν ο όγδοος, Ελζαβάδ ο έννατος,
\par 13 Ιερεμίας ο δέκατος, Μαχβαναί ο ενδέκατος.
\par 14 Ούτοι ήσαν εκ των υιών του Γαδ, αρχηγοί του στρατεύματος, εις ο μικρότερος επί εκατόν, και ο μεγαλήτερος επί χιλίους.
\par 15 Ούτοι ήσαν οι διαβάντες τον Ιορδάνην εν τω πρώτω μηνί, ότε πλημμυρεί επί πάσας τας όχθας αυτού· και διεσκόρπισαν πάντας τους κατοίκους των κοιλάδων προς ανατολάς και προς δυσμάς.
\par 16 Ήλθον έτι εκ των υιών Βενιαμίν και Ιούδα εις το οχύρωμα προς τον Δαβίδ.
\par 17 Και εξήλθεν ο Δαβίδ εις συνάντησιν αυτών και αποκριθείς είπε προς αυτούς, Εάν έρχησθε προς εμέ εν ειρήνη διά να μοι βοηθήσητε, η καρδία μου θέλει είσθαι ηνωμένη με σάς· αλλ' εάν διά να με προδόσητε εις τους εχθρούς μου, ενώ δεν είναι αδικία εις τας χείρας μου, ο Θεός των πατέρων ημών ας ίδη και ας ελέγξη τούτο.
\par 18 Και το Πνεύμα περιεχύθη εις τον Αμασαΐ, τον άρχοντα των τριάκοντα, και είπε, Σού είμεθα, Δαβίδ, και μετά σου, υιέ του Ιεσσαί. Ειρήνη, ειρήνη εις σε, και ειρήνη εις τους βοηθούς σου· διότι ο Θεός σου σε βοηθεί. Τότε εδέχθη αυτούς ο Δαβίδ και κατέστησεν αυτούς αρχηγούς των δυνάμεων αυτού.
\par 19 Και εκ του Μανασσή προσεχώρησαν εις τον Δαβίδ, ότε ήλθε μετά των Φιλισταίων εναντίον του Σαούλ διά να πολεμήση, πλην δεν εβοήθησαν αυτούς· διότι οι ηγεμόνες των Φιλισταίων συμβουλευθέντες απέπεμψαν αυτόν, λέγοντες, Θέλει προσχωρήσει προς τον Σαούλ τον κύριον αυτού με κίνδυνον των κεφαλών ημών.
\par 20 Ενώ επορεύετο εις Σικλάγ, προσεχώρησαν εις αυτόν εκ του Μανασσή, Αδνά και Ιωζαβάδ και Ιεδιαήλ και Μιχαήλ και Ιωζαβάδ και Ελιού και Σιλθαΐ, αρχηγοί των χιλιάδων του Μανασσή·
\par 21 και ούτοι εβοήθησαν τον Δαβίδ εναντίον των ληστών· διότι ήσαν πάντες δυνατοί εν ισχύϊ και έγειναν αρχηγοί του στρατεύματος.
\par 22 Διότι τότε από ημέρας εις ημέραν ήρχοντο προς τον Δαβίδ διά να βοηθήσωσιν αυτόν, εωσού έγεινε στρατόπεδον μέγα, ως στρατόπεδον Θεού.
\par 23 Ούτοι δε είναι οι αριθμοί των αρχηγών των ώπλισμένων διά πόλεμον, των ελθόντων προς τον Δαβίδ εις Χεβρών, διά να στρέψωσιν εις αυτόν την βασιλείαν του Σαούλ, κατά τον λόγον του Κυρίου.
\par 24 Οι υιοί του Ιούδα, θυρεοφόροι και λογχοφόροι, εξ χιλιάδες και οκτακόσιοι, ώπλισμένοι διά πόλεμον.
\par 25 Εκ των υιών Συμεών, δυνατοί εν ισχύϊ, διά πόλεμον, επτά χιλιάδες και εκατόν.
\par 26 Εκ των υιών Λευΐ, τέσσαρες χιλιάδες και εξακόσιοι.
\par 27 Και ο Ιωδαέ ήτο αρχηγός των Ααρωνιτών, και μετ' αυτού ήσαν τρεις χιλιάδες και επτακόσιοι·
\par 28 και Σαδώκ, νέος δυνατός εν ισχύϊ, και εκ του οίκου του πατρός αυτού εικοσιδύο αρχηγοί.
\par 29 Και εκ των υιών Βενιαμίν, αδελφών του Σαούλ, τρεις χιλιάδες· διότι έως τότε το μεγαλήτερον μέρος αυτών υπερησπίζετο τον οίκον του Σαούλ.
\par 30 Και εκ των υιών Εφραΐμ, είκοσι χιλιάδες και οκτακόσιοι, δυνατοί εν ισχύϊ, άνδρες ονομαστοί του οίκου των πατέρων αυτών.
\par 31 Και εκ της ημισείας φυλής του Μανασσή, δεκαοκτώ χιλιάδες· οίτινες ωνομάσθησαν κατ' όνομα, διά να έλθωσι να κάμωσι βασιλέα τον Δαβίδ.
\par 32 Και εκ των υιών Ισσάχαρ, άνδρες συνετοί εις την γνώσιν των καιρών, ώστε να γνωρίζωσι τι έπρεπε να κάμνη ο Ισραήλ· οι αρχηγοί αυτών ήσαν διακόσιοι· και πάντες οι αδελφοί αυτών υπό την διαταγήν αυτών.
\par 33 Εκ του Ζαβουλών, όσοι εξήρχοντο εις πόλεμον, παραταττόμενοι εις μάχην με πάντα τα όπλα του πολέμου, πεντήκοντα χιλιάδες, μάχιμοι εκ παρατάξεως, ουχί με διπλήν καρδίαν.
\par 34 Και εκ του Νεφθαλί, χίλιοι αρχηγοί, και μετ' αυτών θυρεοφόροι και λογχοφόροι τριάκοντα επτά χιλιάδες.
\par 35 Και εκ των Δανιτών, άνδρες παραταττόμενοι εις πόλεμον, εικοσιοκτώ χιλιάδες και εξακόσιοι.
\par 36 Και εκ του Ασήρ, όσοι εξήρχοντο εις πόλεμον, μάχιμοι εκ παρατάξεως, τεσσαράκοντα χιλιάδες.
\par 37 Και εκ του πέραν του Ιορδάνου, εκ των Ρουβηνιτών και εκ των Γαδιτών και εκ της ημισείας φυλής του Μανασσή, με πάντα τα όπλα του πολέμου διά μάχην, εκατόν είκοσι χιλιάδες.
\par 38 Πάντες ούτοι οι άνδρες οι πολεμισταί, μάχιμοι εκ παρατάξεως, ήλθον με πλήρη καρδίαν εις Χεβρών, διά να κάμωσι τον Δαβίδ βασιλέα επί πάντα τον Ισραήλ· και παν έτι το επίλοιπον του Ισραήλ ήσαν μία καρδία, διά να κάμωσι τον Δαβίδ βασιλέα.
\par 39 Και ήσαν εκεί μετά του Δαβίδ τρεις ημέρας, τρώγοντες και πίνοντες· διότι οι αδελφοί αυτών είχον κάμει ετοιμασίαν δι' αυτούς.
\par 40 Προσέτι, οι γειτονεύοντες μετ' αυτών, έως του Ισσάχαρ, και Ζαβουλών και Νεφθαλί, έφεραν τροφάς επί όνων και επί καμήλων και επί ημιόνων και επί βοών, τροφάς αλεύρου, παλάθας σύκων και σταφίδας και οίνον και έλαιον και βόας και πρόβατα, αφθόνως· διότι ήτο ευφροσύνη εν τω Ισραήλ.

\chapter{13}

\par 1 Και συνεβουλεύθη ο Δαβίδ μετά των χιλιάρχων και εκατοντάρχων, μετά πάντων των αρχηγών.
\par 2 Και είπεν ο Δαβίδ προς πάσαν την σύναξιν του Ισραήλ, Εάν σας φαίνηται καλόν και ήναι παρά Κυρίου του Θεού ημών, ας αποστείλωμεν πανταχού προς τους αδελφούς ημών, τους εναπολειφθέντας εν πάση τη γη του Ισραήλ, και μετ' αυτών προς τους ιερείς και Λευΐτας εις τας πόλεις αυτών και τα περίχωρα, διά να συναχθώσι προς ημάς·
\par 3 και ας μεταφέρωμεν προς ημάς την κιβωτόν του Θεού ημών· διότι δεν εζητήσαμεν αυτήν επί των ημερών του Σαούλ.
\par 4 Και πάσα η σύναξις είπον να κάμωσιν ούτω· διότι το πράγμα ήτο αρεστόν εις τους οφθαλμούς παντός του λαού.
\par 5 Τότε ο Δαβίδ συνήθροισε πάντα τον Ισραήλ, από Σιχώρ της Αιγύπτου έως της εισόδου Αιμάθ, διά να φέρωσι την κιβωτόν του Θεού από Κιριάθ-ιαρείμ.
\par 6 Και ανέβη ο Δαβίδ και πας ο Ισραήλ εις Βααλά, εις Κιριάθ-ιαρείμ του Ιούδα, διά να αναγάγη εκείθεν την κιβωτόν Κυρίου του Θεού, του καθημένου επί των χερουβείμ, όπου το όνομα αυτού επεκλήθη.
\par 7 Και επεβίβασαν την κιβωτόν του Θεού επί νέας αμάξης εκ του οίκου Αβιναδάβ· ώδήγησαν δε την άμαξαν ο Ουζά και Αχιώ.
\par 8 Ο δε Δαβίδ και πας ο Ισραήλ έπαιζον έμπροσθεν του Θεού εν πάση δυνάμει και με άσματα και με κιθάρας και με ψαλτήρια και με τύμπανα και με κύμβαλα και με σάλπιγγας.
\par 9 Και ότε έφθασαν έως του αλωνίου Χειδών, ο Ουζά εξήπλωσε την χείρα αυτού, διά να κρατήση την κιβωτόν· διότι οι βόες έσεισαν αυτήν.
\par 10 Και εξήφθη ο θυμός του Κυρίου κατά του Ουζά και επάταξεν αυτόν, διότι εξήπλωσε την χείρα αυτού επί την κιβωτόν· και απέθανεν εκεί ενώπιον του Θεού.
\par 11 Και ελυπήθη ο Δαβίδ, ότι ο Κύριος έκαμε χαλασμόν επί τον Ουζά· και εκάλεσε τον τόπον τούτον Φαρές-ουζά έως της ημέρας ταύτης.
\par 12 Και εφοβήθη ο Δαβίδ τον Θεόν την ημέραν εκείνην, λέγων, Πως θέλω φέρει προς εμαυτόν την κιβωτόν του Θεού;
\par 13 Και δεν μετεκίνησεν ο Δαβίδ την κιβωτόν προς εαυτόν εις την πόλιν Δαβίδ, αλλ' έστρεψεν αυτήν εις τον οίκον του Ωβήδ-εδώμ του Γετθαίου.
\par 14 Και εκάθησεν η κιβωτός του Θεού τρεις μήνας μετά της οικογενείας του Ωβήδ-εδώμ εν τω οίκω αυτού. Και ευλόγησεν ο Κύριος τον οίκον του Ωβήδ-εδώμ και πάντα όσα είχεν.

\chapter{14}

\par 1 Ο δε Χειράμ βασιλεύς της Τύρου απέστειλε πρέσβεις προς τον Δαβίδ, και ξύλα κέδρινα και κτίστας και ξυλουργούς, διά να οικοδομήσωσιν οίκον εις αυτόν.
\par 2 Και εγνώρισεν ο Δαβίδ, ότι ο Κύριος κατέστησεν αυτόν βασιλέα επί τον Ισραήλ, διότι η βασιλεία αυτού υψώθη εις ύψος, διά τον λαόν αυτού Ισραήλ.
\par 3 Και έλαβεν ο Δαβίδ έτι γυναίκας εν Ιερουσαλήμ· και εγέννησεν έτι ο Δαβίδ υιούς και θυγατέρας.
\par 4 Ταύτα δε είναι τα ονόματα των τέκνων, τα οποία εγεννήθησαν εις αυτόν εν Ιερουσαλήμ· Σαμμουά και Σωβάβ, Νάθαν και Σολομών
\par 5 και Ιεβάρ και Ελισουά και Ελφαλέτ
\par 6 και Νωγά και Νεφέγ και Ιαφιά
\par 7 και Ελισαμά και Βεελιαδά και Ελιφαλέτ.
\par 8 Ακούσαντες δε οι Φιλισταίοι ότι ο Δαβίδ εχρίσθη βασιλεύς επί πάντα τον Ισραήλ, ανέβησαν πάντες οι Φιλισταίοι να ζητήσωσι τον Δαβίδ. Και ο Δαβίδ ακούσας, εξήλθεν εναντίον αυτών.
\par 9 Και ήλθον οι Φιλισταίοι και διεχύθησαν εις την κοιλάδα Ραφαείμ.
\par 10 Και ηρώτησεν ο Δαβίδ τον Θεόν, λέγων, να αναβώ εναντίον των Φιλισταίων; και θέλεις παραδώσει συ αυτούς εις την χείρα μου; Και ο Κύριος απεκρίθη προς αυτόν, Ανάβα· διότι θέλω παραδώσει αυτούς εις την χείρα σου.
\par 11 Και ανέβησαν εις Βάαλ-φερασείμ· και εκεί επάταξεν αυτούς ο Δαβίδ. Τότε είπεν ο Δαβίδ, Ο Θεός διέκοψε τους εχθρούς μου διά χειρός μου, καθώς διακόπτονται τα ύδατα· διά τούτο εκάλεσαν το όνομα του τόπου εκείνου Βάαλ-φερασείμ.
\par 12 Και εκεί κατέλιπον τους θεούς αυτών· και ο Δαβίδ προσέταξε και κατεκαύθησαν εν πυρί.
\par 13 Οι δε Φιλισταίοι και πάλιν διεχύθησαν εις την κοιλάδα·
\par 14 όθεν πάλιν ηρώτησεν ο Δαβίδ τον Θεόν· και ο Θεός είπε προς αυτόν, Μη αναβής οπίσω αυτών· αλλά στρέψον απ' αυτών και ύπαγε επ' αυτούς απέναντι των συκαμίνων.
\par 15 Και όταν ακούσης θόρυβον διαβάσεως επί των κορυφών των συκαμίνων, τότε θέλεις εξέλθει εις την μάχην· διότι ο Θεός θέλει εξέλθει έμπροσθέν σου, διά να πατάξη το στρατόπεδον των Φιλισταίων.
\par 16 Και έκαμεν ο Δαβίδ καθώς προσέταξεν εις αυτόν ο Θεός· και επάταξαν το στρατόπεδον των Φιλισταίων από Γαβαών έως Γεζέρ.
\par 17 Και το όνομα του Δαβίδ εξήλθεν εις πάντας τους τόπους· και ο Κύριος επέφερε τον φόβον αυτού επί πάντα τα έθνη.

\chapter{15}

\par 1 Και ο Δαβίδ έκαμεν εις εαυτόν οικίας εν τη πόλει Δαβίδ, και ητοίμασε τόπον διά την κιβωτόν του Θεού και έστησε σκηνήν δι' αυτήν.
\par 2 Τότε είπεν ο Δαβίδ, Δεν πρέπει να σηκώσωσι την κιβωτόν του Θεού ειμή οι Λευΐται διότι αυτούς εξέλεξεν ο Κύριος διά να σηκόνωσι την κιβωτόν του Θεού και να λειτουργώσιν εν αυτή διαπαντός.
\par 3 Και συνήθροισεν ο Δαβίδ πάντα τον Ισραήλ εις την Ιερουσαλήμ, διά να αναβιβάσωσι την κιβωτόν του Κυρίου εις τον τόπον αυτής, τον οποίον ητοίμασε δι' αυτήν.
\par 4 Και συνήθροισεν ο Δαβίδ τους υιούς του Ααρών και τους Λευΐτας·
\par 5 εκ των υιών Καάθ, Ουριήλ τον αρχηγόν και τους αδελφούς αυτού, εκατόν είκοσι·
\par 6 εκ των υιών Μεραρί, Ασαΐαν τον αρχηγόν και τους αδελφούς αυτού, διακοσίους είκοσι·
\par 7 εκ των υιών Γηρσώμ, Ιωήλ τον αρχηγόν και τους αδελφούς αυτού, εκατόν τριάκοντα·
\par 8 εκ των υιών Ελισαφάν, Σεμαΐαν τον αρχηγόν και τους αδελφούς αυτού, διακοσίους·
\par 9 εκ των υιών Χεβρών, Ελιήλ τον αρχηγόν και τους αδελφούς αυτού, ογδοήκοντα·
\par 10 εκ των υιών Οζιήλ, Αμμιναδάβ τον αρχηγόν και τους αδελφούς αυτού, εκατόν δώδεκα.
\par 11 Και εκάλεσεν ο Δαβίδ τον Σαδώκ και τον Αβιάθαρ, τους ιερείς, και τους Λευΐτας Ουριήλ, Ασαΐαν, και Ιωήλ, Σεμαΐαν και Ελιήλ και Αμμιναδάβ,
\par 12 και είπε προς αυτούς, σεις οι άρχοντες των πατριών των Λευϊτών, αγιάσθητε σεις και οι αδελφοί σας, και αναβιβάσατε την κιβωτόν Κυρίου του Θεού του Ισραήλ εις τον τόπον τον οποίον ητοίμασα δι' αυτήν·
\par 13 διότι, επειδή σεις δεν εκάμετε τούτο την αρχήν, Κύριος ο Θεός ημών έκαμε χαλασμόν εν ημίν, καθότι δεν εζητήσαμεν αυτόν κατά το διατεταγμένον.
\par 14 Οι ιερείς λοιπόν και οι Λευΐται ηγιάσθησαν, διά να αναβιβάσωσι την κιβωτόν Κυρίου του Θεού του Ισραήλ.
\par 15 Και εσήκωσαν οι υιοί των Λευϊτών την κιβωτόν του Θεού επί ώμων με τους μοχλούς εφ' εαυτών, καθώς προσέταξεν ο Μωϋσής κατά τον λόγον του Κυρίου.
\par 16 Και είπεν ο Δαβίδ προς τους αρχηγούς των Λευϊτών να στήσωσι τους αδελφούς αυτών τους ψαλτωδούς με όργανα μουσικά, ψαλτήρια και κιθάρας και κύμβαλα, διά να ηχώσιν υψόνοντες φωνήν εν ευφροσύνη.
\par 17 Και έστησαν οι Λευΐται τον Αιμάν υιόν του Ιωήλ· και εκ των αδελφών αυτού, τον Ασάφ υιόν του Βαραχίου· και εκ των υιών Μεραρί των αδελφών αυτών, τον Εθάν υιόν του Κεισαΐα·
\par 18 και μετ' αυτών, τους δευτερεύοντας αδελφούς αυτών, Ζαχαρίαν, Βεν και Ιααζιήλ και Σεμιραμώθ και Ιεχιήλ και Ουννί, Ελιάβ και Βεναΐαν και Μαασίαν και Ματταθίαν και Ελιφελεού και Μικνεΐαν και Ωβήδ-εδώμ και Ιεϊήλ, τους πυλωρούς.
\par 19 Ούτως οι ψαλτωδοί, Αιμάν, Ασάφ και Εθάν, διωρίσθησαν διά να ηχώσι με κύμβαλα χάλκινα·
\par 20 ο δε Ζαχαρίας και Αζιήλ και Σεμιραμώθ και Ιεχιήλ και Ουννί και Ελιάβ και Μαασίας και Βεναΐας, με ψαλτήρια επί Αλαμώθ·
\par 21 και ο Ματταθίας και Ελιφελεού και Μικνεΐας και Ωβήδ-εδώμ και Ιεϊήλ και Αζαζίας, με κιθάρας επί Σεμινίθ, διά να ενισχύσωσι τον τόνον.
\par 22 Και ο Χενανίας ήτο πρωταοιδός των Λευϊτών, προεδρεύων εις το άδειν, επειδή ήτο συνετός.
\par 23 Ο δε Βαραχίας και Ελκανά ήσαν πυλωροί της κιβωτού.
\par 24 Και ο Σεβανίας και Ιωσαφάτ και Ναθαναήλ και Αμασαΐ και Ζαχαρίας και Βεναΐας και Ελιέζερ, οι ιερείς, εσάλπιζον με τας σάλπιγγας έμπροσθεν της κιβωτού του Θεού· ο δε Ωβήδ-εδώμ και Ιεχιά ήσαν πυλωροί της κιβωτού.
\par 25 Και υπήγαν ο Δαβίδ και οι πρεσβύτεροι του Ισραήλ και οι χιλίαρχοι να αναβιβάσωσι την κιβωτόν της διαθήκης του Κυρίου εκ του οίκου του Ωβήδ-εδώμ εν ευφροσύνη.
\par 26 Και ότε ο Θεός ενίσχυε τους Λευΐτας τους βαστάζοντας την κιβωτόν της διαθήκης του Κυρίου, εθυσίαζον επτά μόσχους και επτά κριούς.
\par 27 Και ο Δαβίδ ήτο ενδεδυμένος στολήν βυσσίνην, και πάντες οι Λευΐται οι βαστάζοντες την κιβωτόν και οι ψαλτωδοί και ο Χενανίας ο πρωταοιδός των ψαλτωδών· και εφόρει ο Δαβίδ εφόδ λινούν.
\par 28 Ούτω πας ο Ισραήλ ανεβίβαζε την κιβωτόν της διαθήκης του Κυρίου εν αλαλαγμώ και εν φωνή κερατίνης και εν σάλπιγξι και εν κυμβάλοις, ηχούντες εν ψαλτηρίοις και εν κιθάραις.
\par 29 Και ενώ η κιβωτός της διαθήκης του Κυρίου εισήρχετο εις την πόλιν Δαβίδ, Μιχάλ, η θυγάτηρ του Σαούλ, έκυψε διά της θυρίδος και ιδούσα τον βασιλέα Δαβίδ χορεύοντα και παίζοντα, εξουδένωσεν αυτόν εν τη καρδία αυτής.

\chapter{16}

\par 1 Και έφεραν την κιβωτόν του Θεού και έθεσαν αυτήν εν τω μέσω της σκηνής, την οποίαν έστησε δι' αυτήν ο Δαβίδ· και προσέφεραν ολοκαυτώματα και ειρηνικάς προσφοράς ενώπιον του Θεού.
\par 2 Και αφού ετελείωσεν ο Δαβίδ προσφέρων τα ολοκαυτώματα και τας ειρηνικάς προσφοράς, ευλόγησε τον λαόν εν ονόματι Κυρίου.
\par 3 Και διεμοίρασεν εις πάντα άνθρωπον εκ του Ισραήλ, από ανδρός έως γυναικός, εις έκαστον εν ψωμίον και εν τμήμα κρέατος και μίαν φιάλην οίνου.
\par 4 Και διώρισεν εκ των Λευϊτών διά να λειτουργώσιν έμπροσθεν της κιβωτού του Κυρίου, και να μνημονεύωσι και να ευχαριστώσι και να υμνώσι Κύριον τον Θεόν του Ισραήλ·
\par 5 τον Ασάφ πρώτον, και δεύτερον αυτού τον Ζαχαρίαν, έπειτα τον Ιεϊήλ και Σεμιραμώθ και Ιεχιήλ και Ματταθίαν και Ελιάβ και Βεναΐαν και Ωβήδ-εδώμ· και ο μεν Ιεϊήλ ήχει εν ψαλτηρίοις και κιθάραις, ο δε Ασάφ εν κυμβάλοις·
\par 6 ο Βεναΐας δε και ο Ιααζιήλ, οι ιερείς, εν σάλπιγξι πάντοτε έμπροσθεν της κιβωτού της διαθήκης του Θεού.
\par 7 Τότε πρώτον την ημέραν εκείνην παρέδωκεν ο Δαβίδ εις την χείρα του Ασάφ και των αδελφών αυτού τον ψαλμόν τούτον, διά να δοξολογήση τον Κύριον·
\par 8 Δοξολογείτε τον Κύριον· επικαλείσθε το όνομα αυτού· κάμετε γνωστά εις τα έθνη τα έργα αυτού.
\par 9 Ψάλλετε εις αυτόν· ψαλμωδείτε εις αυτόν· λαλείτε περί πάντων των θαυμασίων αυτού.
\par 10 Καυχάσθε εις το άγιον αυτού όνομα· ας ευφραίνηται η καρδία των εκζητούντων τον Κύριον.
\par 11 Ζητείτε τον Κύριον και την δύναμιν αυτού· εκζητείτε το πρόσωπον αυτού διαπαντός.
\par 12 Μνημονεύετε των θαυμασίων αυτού τα οποία έκαμε, των τεραστίων αυτού και των κρίσεων του στόματος αυτού,
\par 13 Σπέρμα Ισραήλ του δούλου αυτού, υιοί Ιακώβ, οι εκλεκτοί αυτού.
\par 14 Αυτός είναι Κύριος ο Θεός ημών· εν πάση τη γη είναι αι κρίσεις αυτού.
\par 15 Μνημονεύετε πάντοτε της διαθήκης αυτού, του λόγου τον οποίον προσέταξεν εις χιλίας γενεάς·
\par 16 της διαθήκης την οποίαν έκαμε προς τον Αβραάμ, και τον όρκον αυτού προς τον Ισαάκ·
\par 17 Και εβεβαίωσεν αυτόν προς τον Ιακώβ διά νόμον, προς τον Ισραήλ διά διαθήκην αιώνιον.
\par 18 Λέγων, εις σε θέλω δώσει την γην Χαναάν, μερίδα της κληρονομίας σας.
\par 19 Ενώ σεις ήσθε ολιγοστοί τον αριθμόν, ολίγοι και πάροικοι εν αυτή,
\par 20 και διήρχοντο από έθνους εις έθνος και από βασιλείου εις άλλον λαόν,
\par 21 δεν αφήκεν άνθρωπον να αδικήση αυτούς· μάλιστα υπέρ αυτών ήλεγξε βασιλείς,
\par 22 λέγων, Μη εγγίσητε τους κεχρισμένους μου, και μη κακοποιήσητε τους προφήτας μου.
\par 23 Ψάλλετε εις τον Κύριον, πάσα η γή· κηρύττετε από ημέρας εις ημέραν την σωτηρίαν αυτού.
\par 24 Αναγγείλατε εις τα έθνη την δόξαν αυτού, εις πάντας τους λαούς τα θαυμάσια αυτού.
\par 25 Διότι μέγας είναι ο Κύριος και αξιΰμνητος σφόδρα, και είναι φοβερός υπέρ πάντας τους θεούς.
\par 26 Διότι πάντες οι θεοί των εθνών είναι είδωλα· ο δε Κύριος τους ουρανούς εποίησε.
\par 27 Δόξα και μεγαλοπρέπεια είναι ενώπιον αυτού· ισχύς και αγαλλίασις εν τω τόπω αυτού.
\par 28 Απόδοτε εις τον Κύριον, πατριαί των λαών, απόδοτε εις τον Κύριον δόξαν και κράτος.
\par 29 Απόδοτε εις τον Κύριον την δόξαν του ονόματος αυτού· λάβετε προσφοράς και έλθετε ενώπιον αυτού· προσκυνήσατε τον Κύριον εν τω μεγαλοπρεπεί αγιαστηρίω αυτού.
\par 30 Φοβείσθε από προσώπου αυτού, πάσα η γή· η οικουμένη θέλει βεβαίως είσθαι εστερεωμένη, δεν θέλει σαλευθή.
\par 31 Ας ευφραίνωνται οι ουρανοί, και ας αγάλλεται η γή· και ας λέγωσι μεταξύ των εθνών, Ο Κύριος βασιλεύει.
\par 32 Ας ηχή η θάλασσα και το πλήρωμα αυτής· ας χαίρωσιν αι πεδιάδες και πάντα τα εν αυταίς.
\par 33 Τότε θέλουσιν αγάλλεσθαι τα δένδρα του δάσους εν τη παρουσία του Κυρίου· διότι έρχεται διά να κρίνη την γην.
\par 34 Δοξολογείτε τον Κύριον· διότι είναι αγαθός· διότι το έλεος αυτού μένει εις τον αιώνα.
\par 35 Και είπατε, Σώσον ημάς, Θεέ της σωτηρίας ημών, και συνάγαγε ημάς και ελευθέρωσον ημάς εκ των εθνών, διά να δοξολογώμεν το όνομά σου το άγιον, και να καυχώμεθα εις την αίνεσίν σου.
\par 36 Ευλογητός Κύριος ο Θεός του Ισραήλ απ' αιώνος και έως αιώνος. Και πας ο λαός είπεν, Αμήν, και ήνεσε τον Κύριον.
\par 37 Τότε αφήκεν εκεί έμπροσθεν της κιβωτού της διαθήκης του Κυρίου τον Ασάφ και τους αδελφούς αυτού, διά να λειτουργώσιν έμπροσθεν της κιβωτού πάντοτε, κατά το απαιτούμενον εκάστης ημέρας·
\par 38 και τον Ωβήβ-εδώμ και τους αδελφούς αυτού, εξήκοντα οκτώ· και τον Ωβήδ-εδώμ τον υιόν του Ιεδουθούν, και τον Ωσά, διά πυλωρούς·
\par 39 και τον Σαδώκ τον ιερέα και τους αδελφούς αυτού τους ιερείς, έμπροσθεν της σκηνής του Κυρίου εν τω υψηλώ τόπω τω εν Γαβαών,
\par 40 διά να προσφέρωσιν ολοκαυτώματα προς τον Κύριον επί του θυσιαστηρίου των ολοκαυτωμάτων πάντοτε πρωΐ και εσπέρας, και να κάμνωσι κατά πάντα τα γεγραμμένα εν τω νόμω του Κυρίου, τον οποίον προσέταξεν εις τον Ισραήλ·
\par 41 και μετ' αυτών τον Αιμάν και Ιεδουθούν και τους λοιπούς τους εκλελεγμένους, οίτινες διωρίσθησαν κατ' όνομα, διά να δοξολογώσι τον Κύριον, διότι το έλεος αυτού μένει εις τον αιώνα·
\par 42 και μετ' αυτών τον Αιμάν και Ιεδουθούν, με σάλπιγγας και κύμβαλα, διά εκείνους οίτινες έπρεπε να ηχώσι, και με όργανα μουσικά του Θεού. Οι δε υιοί του Ιεδουθούν ήσαν πυλωροί.
\par 43 Και απήλθε πας ο λαός, έκαστος εις την οικίαν αυτού· και επέστρεψεν ο Δαβίδ, διά να ευλογήση τον οίκον αυτού.

\chapter{17}

\par 1 Αφού δε εκάθησεν ο Δαβίδ εν τω οίκω αυτού, είπεν ο Δαβίδ προς Νάθαν τον προφήτην, Ιδού, εγώ κατοικώ εν οίκω κεδρίνω, η δε κιβωτός της διαθήκης του Κυρίου υπό παραπετάσματα.
\par 2 Και είπεν ο Νάθαν προς τον Δαβίδ, Κάμε παν το εν τη καρδία σου· διότι ο Θεός είναι μετά σου.
\par 3 Και την νύκτα εκείνην έγεινε λόγος του Θεού προς τον Νάθαν, λέγων,
\par 4 Ύπαγε και ειπέ προς τον Δαβίδ τον δούλον μου, ούτω λέγει Κύριος· Συ δεν θέλεις οικοδομήσει εις εμέ τον οίκον διά να κατοικώ·
\par 5 διότι δεν κατώκησα εν οίκω, αφ' ης ημέρας ανεβίβασα τον Ισραήλ εξ Αιγύπτου, μέχρι της ημέρας ταύτης· αλλ' ήμην από σκηνής εις σκηνήν και από κατασκηνώματος εις κατασκήνωμα.
\par 6 Πανταχού όπου περιεπάτησα μετά παντός του Ισραήλ, ελάλησα ποτέ προς τινά εκ των κριτών του Ισραήλ, τους οποίους προσέταξα να ποιμάνωσι τον λαόν μου, λέγων, Διά τι δεν ωκοδομήσατε εις εμέ οίκον κέδρινον;
\par 7 Τώρα λοιπόν ούτω θέλεις ειπεί προς τον Δαβίδ τον δούλον μου· Ούτω λέγει ο Κύριος των δυνάμεων· Εγώ σε έλαβον εκ της μάνδρας, από όπισθεν των προβάτων, διά να ήσαι ηγεμών επί τον λαόν μου τον Ισραήλ·
\par 8 και ήμην μετά σου πανταχού όπου περιεπάτησας, και εξωλόθρευσα πάντας τους εχθρούς σου απ' έμπροσθέν σου, και έκαμα εις σε όνομα, κατά το όνομα των μεγάλων των επί της γης.
\par 9 Και θέλω διορίσει τόπον διά τον λαόν μου τον Ισραήλ, και θέλω φυτεύσει αυτούς, και θέλουσι κατοικεί εν τόπω ιδίω εαυτών και δεν θέλουσι μεταφέρεσθαι πλέον· και οι υιοί της αδικίας δεν θέλουσι καταθλίβει αυτούς πλέον ως το πρότερον
\par 10 και ως από των ημερών καθ' ας κατέστησα κριτάς επί τον λαόν μου Ισραήλ. Και θέλω ταπεινώσει πάντας τους εχθρούς σου. Αναγγέλλω σοι έτι, ότι ο Κύριος θέλει οικοδομήσει οίκον εις σε.
\par 11 Και αφού πληρωθώσιν αι ημέραι σου, διά να υπάγης μετά των πατέρων σου, θέλω αναστήσει μετά σε το σπέρμα σου, το οποίον θέλει είσθαι εκ των υιών σου, και θέλω στερεώσει την βασιλείαν αυτού.
\par 12 Αυτός θέλει οικοδομήσει εις εμέ οίκον, και θέλω στερεώσει το θρόνον αυτού έως αιώνος.
\par 13 Εγώ θέλω είσθαι εις αυτόν πατήρ, και αυτός θέλει είσθαι εις εμέ υιός· και δεν θέλω αφαιρέσει το έλεός μου απ' αυτού, ως αφήρεσα αυτό απ' εκείνον όστις ήτο προ σού·
\par 14 αλλά θέλω στήσει αυτόν εν τω οίκω μου και εν τη βασιλεία μου έως του αιώνος· και ο θρόνος αυτού θέλει είσθαι εστερεωμένος εις τον αιώνα.
\par 15 Κατά πάντας τούτους τους λόγους και καθ' όλην ταύτην την όρασιν, ούτως ελάλησεν ο Νάθαν προς τον Δαβίδ.
\par 16 Τότε εισήλθεν ο βασιλεύς Δαβίδ και εκάθησεν ενώπιον του Κυρίου και είπε, Τις είμαι εγώ, Κύριε Θεέ, και τις ο οίκός μου, ώστε με έφερες μέχρι τούτου;
\par 17 Αλλά και τούτο εστάθη μικρόν εις τους οφθαλμούς σου, Θεέ· και ελάλησας περί του οίκου του δούλου σου διά μέλλον μακρόν, και επέβλεψας εις εμέ ως εις άνθρωπον υψηλού βαθμού κατά την κατάστασιν, Κύριε Θεέ.
\par 18 Τι δύναται να είπη πλέον ο Δαβίδ προς σε περί της εις τον δούλον σου τιμής; διότι συ γνωρίζεις τον δούλον σου.
\par 19 Κύριε, χάριν του δούλου σου και κατά την καρδίαν σου έκαμες πάσαν ταύτην την μεγαλωσύνην, διά να κάμης γνωστά πάντα ταύτα τα μεγαλεία.
\par 20 Κύριε, δεν είναι όμοιός σου, ουδέ είναι Θεός εκτός σου κατά πάντα όσα ηκούσαμεν με τα ώτα ημών.
\par 21 Και τι άλλο έθνος επί της γης είναι ως ο λαός σου ο Ισραήλ, τον οποίον ο Θεός ήλθε να εξαγοράση διά λαόν εαυτού, διά να κάμης εις σεαυτόν όνομα μεγαλωσύνης και τρόμου, εκβάλλων τα έθνη απ' έμπροσθεν του λαού σου, τον οποίον ελύτρωσας εξ Αιγύπτου;
\par 22 διότι τον λαόν σου τον Ισραήλ έκαμες λαόν σεαυτού εις τον αιώνα· και συ, Κύριε, έγεινες Θεός αυτών.
\par 23 Και τώρα, Κύριε, ο λόγος, τον οποίον ελάλησας περί του δούλου σου και περί του οίκου αυτού, ας στερεωθή εις τον αιώνα, και κάμε ως ελάλησας·
\par 24 και ας στερεωθή, και ας μεγαλυνθή το όνομά σου έως αιώνος, ώστε να λέγωσιν, Ο Κύριος των δυνάμεων, ο Θεός του Ισραήλ, είναι Θεός εις τον Ισραήλ· και ο οίκος Δαβίδ του δούλου σου ας ήναι εστερεωμένος ενώπιόν σου.
\par 25 Διότι συ, Θεέ μου, απεκάλυψας εις τον δούλον σου ότι θέλεις οικοδομήσει οίκον εις αυτόν· διά τούτο ο δούλός σου ενεθαρρύνθη να προσευχηθή ενώπιόν σου.
\par 26 Και τώρα, Κύριε, συ είσαι ο Θεός, και υπεσχέθης τα αγαθά ταύτα προς τον δούλον σου.
\par 27 Τώρα λοιπόν, ευδόκησον να ευλογήσης τον οίκον του δούλου σου, διά να ήναι ενώπιόν σου εις τον αιώνα· διότι συ, Κύριε, ευλόγησας, και θέλει είσθαι ευλογημένος εις τον αιώνα.

\chapter{18}

\par 1 Μετά δε ταύτα επάταξεν ο Δαβίδ τους Φιλισταίους και κατετρόπωσεν αυτούς, και έλαβε την Γαθ και τας κώμας αυτής εκ χειρός των Φιλισταίων.
\par 2 Και επάταξε τους Μωαβίτας, και έγειναν οι Μωαβίται δούλοι του Δαβίδ υποτελείς.
\par 3 Επάταξεν έτι ο Δαβίδ τον Αδαρέζερ βασιλέα της Σωβά, εν Αιμάθ, ότε επορεύετο να στήση την εξουσίαν αυτού επί τον ποταμόν Ευφράτην.
\par 4 Και έλαβεν ο Δαβίδ εξ αυτού χιλίας αμάξας και επτά χιλιάδας ιππέων και είκοσι χιλιάδας πεζών· και ενευροκόπησεν ο Δαβίδ πάντας τους ίππους των αμαξών και εφύλαξεν εξ αυτών εκατόν αμάξας.
\par 5 Και ότε ήλθον οι Σύριοι της Δαμασκού διά να βοηθήσωσι τον Αδαρέζερ βασιλέα της Σωβά, ο Δαβίδ επάταξεν εκ των Συρίων εικοσιδύο χιλιάδας ανδρών.
\par 6 Και έβαλεν ο Δαβίδ φρουράς εν τη Συρία της Δαμασκού· και οι Σύριοι έγειναν δούλοι του Δαβίδ υποτελείς. Και έσωσεν ο Κύριος τον Δαβίδ πανταχού όπου επορεύετο.
\par 7 Και έλαβεν ο Δαβίδ τας ασπίδας τας χρυσάς, αίτινες ήσαν επί τους δούλους του Αδαρέζερ, και έφερεν αυτάς εις Ιερουσαλήμ.
\par 8 Και εκ της Τιβάθ και εκ της Χούν, πόλεων του Αδαρέζερ, έλαβεν ο Δαβίδ χαλκόν πολύν σφόδρα, εκ του οποίου ο Σολομών έκαμε την χαλκίνην θάλασσαν και τους στύλους και τα σκεύη τα χάλκινα.
\par 9 Ακούσας δε ο Θοού βασιλεύς της Αιμάθ ότι επάταξεν ο Δαβίδ πάσαν την δύναμιν του Αδαρέζερ βασιλέως της Σωβά,
\par 10 απέστειλεν Αδωράμ τον υιόν αυτού προς τον βασιλέα Δαβίδ, διά να χαιρετήση αυτόν και να ευλογήση αυτόν, ότι κατεπολέμησε τον Αδαρέζερ και επάταξεν αυτόν· διότι ο Αδαρέζερ ήτο πολέμιος του Θοού· έφερε δε και παν είδος σκευών χρυσών, αργυρών και χαλκίνων.
\par 11 Και ταύτα αφιέρωσεν ο βασιλεύς Δαβίδ εις τον Κύριον, μετά του αργυρίου και του χρυσίου τα οποία έφερεν εκ πάντων των εθνών, εκ του Εδώμ και εκ του Μωάβ και εκ των υιών του Αμμών και εκ των Φιλισταίων και εκ του Αμαλήκ.
\par 12 Και ο Αβισαί ο υιός της Σερουΐας επάταξε τους Ιδουμαίους εν τη κοιλάδι του άλατος, δεκαοκτώ χιλιάδας.
\par 13 Και έβαλε φρουράς εν τη Ιδουμαία· και πάντες οι Ιδουμαίοι έγειναν δούλοι του Δαβίδ. Και έσωσεν ο Κύριος τον Δαβίδ πανταχού όπου επορεύετο.
\par 14 Και εβασίλευσεν ο Δαβίδ επί πάντα τον Ισραήλ, και έκαμνε κρίσιν και δικαιοσύνην εις πάντα τον λαόν αυτού.
\par 15 Και Ιωάβ ο υιός της Σερουΐας ήτο επί του στρατεύματος· Ιωσαφάτ δε ο υιός του Αχιλούδ, υπομνηματογράφος.
\par 16 Και Σαδώκ ο υιός του Αχιτώβ και Αβιμέλεχ ο υιός του Αβιάθαρ, ιερείς· ο δε Σουσά, γραμματεύς.
\par 17 Και Βεναΐας ο υιός του Ιωδαέ ήτο επί των Χερεθαίων και Φελεθαίων· οι δε υιοί του Δαβίδ, πρώτοι περί τον βασιλέα.

\chapter{19}

\par 1 Μετά δε ταύτα απέθανεν ο Νάας βασιλεύς των υιών Αμμών, και εβασίλευσεν αντ' αυτού ο υιός αυτού.
\par 2 Και είπεν ο Δαβίδ, Θέλω κάμει έλεος προς Ανούν τον υιόν του Νάας, επειδή ο πατήρ αυτού έκαμεν έλεος προς εμέ. Και απέστειλεν ο Δαβίδ πρέσβεις, διά να παρηγορήση αυτόν περί του πατρός αυτού. Και ήλθον οι δούλοι του Δαβίδ εις την γην των υιών Αμμών προς τον Ανούν, διά να παρηγορήσωσιν αυτόν.
\par 3 Και είπον οι άρχοντες των υιών Αμμών προς τον Ανούν, Νομίζεις ότι ο Δαβίδ τιμών τον πατέρα σου απέστειλε παρηγορητάς προς σε; δεν ήλθον οι δούλοι αυτού προς σε διά να ερευνήσωσι και να κατασκοπεύσωσι και να καταστρέψωσι τον τόπον;
\par 4 Και επίασεν ο Ανούν τους δούλους του Δαβίδ και εξύρισεν αυτούς και απέκοψε το ήμισυ των ιματίων αυτών μέχρι των γλουτών, και απέπεμψεν αυτούς.
\par 5 Υπήγαν δε και απήγγειλαν προς τον Δαβίδ περί των ανδρών. Και απέστειλεν εις συνάντησιν αυτών· επειδή οι άνδρες ήσαν ητιμασμένοι σφόδρα. Και είπεν ο βασιλεύς, Καθήσατε εν Ιεριχώ εωσού αυξηθώσιν οι πώγωνές σας, και επιστρέψατε.
\par 6 Βλέποντες δε οι υιοί Αμμών ότι ήσαν βδελυκτοί εις τον Δαβίδ, έπεμψαν ο Ανούν και οι υιοί Αμμών χίλια τάλαντα αργυρίου, διά να μισθώσωσιν εις εαυτούς αμάξας και ιππέας εκ της Μεσοποταμίας και εκ της Συρίας-μααχά και εκ της Σωβά.
\par 7 Και εμίσθωσαν εις εαυτούς τριάκοντα δύο χιλιάδας αμάξας και τον βασιλέα της Μααχά μετά του λαού αυτού, οίτινες ήλθον και εστρατοπέδευσαν κατέναντι της Μεδεβά. Και συναχθέντες οι υιοί Αμμών εκ των πόλεων αυτών, ήλθον να πολεμήσωσι.
\par 8 Και ότε ήκουσε ταύτα ο Δαβίδ, απέστειλε τον Ιωάβ και άπαν το στράτευμα των δυνατών.
\par 9 Και εξήλθον οι υιοί Αμμών και παρετάχθησαν εις πόλεμον κατά την πύλην της πόλεως· οι δε βασιλείς οι ελθόντες ήσαν καθ' εαυτούς εν τη πεδιάδι.
\par 10 Βλέπων δε ο Ιωάβ ότι η μάχη παρετάχθη εναντίον αυτού έμπροσθεν και όπισθεν, εξέλεξεν εκ πάντων των εκλεκτών του Ισραήλ και παρέταξεν αυτούς εναντίον των Συρίων.
\par 11 Το δε υπόλοιπον του λαού έδωκεν εις την χείρα του Αβισαί αδελφού αυτού, και παρετάχθησαν εναντίον των υιών Αμμών.
\par 12 Και είπεν, Εάν οι Σύριοι υπερισχύσωσι κατ' εμού, τότε συ θέλεις με σώσει· εάν δε οι υιοί Αμμών υπερισχύσωσι κατά σου, τότε εγώ θέλω σε σώσει·
\par 13 ανδρίζου, και ας κραταιωθώμεν υπέρ του λαού ημών και υπέρ των πόλεων του Θεού ημών· ο δε Κύριος ας κάμη το αρεστόν εις τους οφθαλμούς αυτού.
\par 14 Και προσήλθεν ο Ιωάβ και ο λαός ο μετ' αυτού εναντίον των Συρίων εις μάχην· οι δε έφυγον απ' έμπροσθεν αυτού.
\par 15 Και ότε είδον οι υιοί Αμμών ότι οι Σύριοι έφυγον, έφυγον και αυτοί απ' έμπροσθεν του Αβισαί του αδελφού αυτού και εισήλθον εις την πόλιν. Και ο Ιωάβ ήλθεν εις Ιερουσαλήμ.
\par 16 Ιδόντες δε οι Σύριοι ότι κατετροπώθησαν έμπροσθεν του Ισραήλ, απέστειλαν μηνυτάς και εξήγαγον τους Συρίους τους πέραν του ποταμού· και Σωφάκ, ο αρχιστράτηγος του Αδαρέζερ, επορεύετο έμπροσθεν αυτών.
\par 17 Και ότε απηγγέλθη προς τον Δαβίδ, συνήθροισε πάντα τον Ισραήλ, και διέβη τον Ιορδάνην και ήλθεν επ' αυτούς και παρετάχθη εναντίον αυτών. Και ότε παρετάχθη ο Δαβίδ εις πόλεμον εναντίον των Συρίων, επολέμησαν με αυτόν.
\par 18 Και έφυγον οι Σύριοι απ' έμπροσθεν του Ισραήλ· και εξωλόθρευσεν ο Δαβίδ εκ των Συρίων επτά χιλιάδας αμαξών και τεσσαράκοντα χιλιάδας πεζών· και Σωφάχ, τον αρχιστράτηγον, εθανάτωσε.
\par 19 Και ιδόντες οι δούλοι του Αδαρέζερ ότι κατετροπώθησαν έμπροσθεν του Ισραήλ, έκαμον ειρήνην μετά του Δαβίδ και έγειναν δούλοι αυτού· και δεν ήθελον πλέον οι Σύριοι να βοηθήσωσι τους υιούς Αμμών.

\chapter{20}

\par 1 Εν δε τω ακολούθω έτει, καθ' ον καιρόν εκστρατεύουσιν οι βασιλείς, ο Ιωάβ εξεκίνησε πάσαν την δύναμιν του στρατεύματος και έφθειρε την γην των υιών Αμμών, και ελθών επολιόρκησε την Ραββά· ο δε Δαβίδ έμεινεν εν Ιερουσαλήμ. Και επάταξεν ο Ιωάβ την Ραββά και κατέστρεψεν αυτήν.
\par 2 Και έλαβεν ο Δαβίδ τον στέφανον του βασιλέως αυτών από της κεφαλής αυτού· και ευρέθη το βάρος αυτού εν τάλαντον χρυσίου· και ήσαν επ' αυτού λίθοι πολύτιμοι και ετέθη επί την κεφαλής του Δαβίδ· και λάφυρα της πόλεως εξέφερε πολλά σφόδρα.
\par 3 Και τον λαόν τον εν αυτή εξήγαγε, και έκοψεν αυτούς με πρίονας και με τριβόλους σιδηρούς και με πελέκεις. Και ούτως έκαμεν ο Δαβίδ εις πάσας τας πόλεις των υιών Αμμών. Τότε επέστρεψεν ο Δαβίδ και πας ο λαός εις Ιερουσαλήμ.
\par 4 Μετά δε ταύτα συνεκροτήθη πόλεμος εν Γεζέρ μετά των Φιλισταίων· τότε επάταξεν ο Σιββεχαΐ ο Χουσαθίτης τον Σιφφαΐ, εκ των τέκνων του Ραφά· και κατετροπώθησαν.
\par 5 Και πάλιν έγεινε πόλεμος μετά των Φιλισταίων· και επάταξεν ο Ελχανάν ο υιός του Ιαείρ τον Λααμεί, αδελφόν του Γολιάθ του Γετθαίου, και το ξύλον της λόγχης αυτού ήτο ως αντίον υφαντού.
\par 6 Και πάλιν έγεινε πόλεμος εν Γαθ, όπου ήτο ανήρ υπερμεγέθης, και οι δάκτυλοι αυτού ήσαν εξ και εξ, εικοσιτέσσαρες, και ούτος έτι ήτο εκ της γενεάς του Ραφά.
\par 7 Και ωνείδισε τον Ισραήλ, και Ιωνάθαν ο υιός του Σαμαά, αδελφού του Δαβίδ, επάταξεν αυτόν.
\par 8 Ούτοι εγεννήθησαν εις τον Ραφά εν Γάθ· και έπεσον διά χειρός του Δαβίδ και διά χειρός των δούλων αυτού.

\chapter{21}

\par 1 Αλλ' ο Σατανάς ηγέρθη κατά του Ισραήλ, και παρεκίνησε τον Δαβίδ να απαριθμήση τον Ισραήλ.
\par 2 Και είπεν ο Δαβίδ προς τον Ιωάβ και προς τους άρχοντας του λαού, Υπάγετε, απαριθμήσατε τον Ισραήλ, από Βηρ-σαβεέ έως Δαν, και φέρετε προς εμέ, διά να μάθω, τον αριθμόν αυτών.
\par 3 Ο δε Ιωάβ απεκρίθη, Ο Κύριος να προσθέση επί τον λαόν αυτού εκατονταπλάσιον αφ' ό,τι είναι αλλά, κύριέ μου βασιλεύ, δεν είναι πάντες δούλοι του κυρίου μου; διά τι ο κύριός μου επιθυμεί τούτο; διά τι να γείνη τούτο αμάρτημα εις τον Ισραήλ;
\par 4 Ο λόγος όμως του βασιλέως υπερίσχυσεν επί τον Ιωάβ. Και ανεχώρησεν ο Ιωάβ, και περιελθών άπαντα τον Ισραήλ επέστρεψεν εις Ιερουσαλήμ.
\par 5 Και έδωκεν ο Ιωάβ το κεφάλαιον της απαριθμήσεως του λαού εις τον Δαβίδ. Και πας ο Ισραήλ ήσαν χίλιαι χιλιάδες και εκατόν χιλιάδες ανδρών συρόντων μάχαιραν· ο δε Ιούδας, τετρακόσιαι εβδομήκοντα χιλιάδες ανδρών συρόντων μάχαιραν.
\par 6 τους Λευΐτας δε και Βενιαμίτας δεν ηρίθμησε μεταξύ αυτών· διότι ο λόγος του βασιλέως ήτο βδελυκτός εις τον Ιωάβ.
\par 7 Και εφάνη κακόν εις τους οφθαλμούς του Θεού το πράγμα τούτο· όθεν επάταξε τον Ισραήλ.
\par 8 Τότε είπεν ο Δαβίδ προς τον Θεόν, Ημάρτησα σφόδρα, πράξας το πράγμα τούτο· αλλά τώρα, δέομαι, αφαίρεσον την ανομίαν του δούλου σου· διότι εμωράνθην σφόδρα.
\par 9 Και ελάλησε Κύριος προς τον Γαδ τον βλέποντα του Δαβίδ, λέγων,
\par 10 Ύπαγε και λάλησον προς τον Δαβίδ, λέγων, ούτω λέγει Κύριος· Τρία πράγματα εγώ προβάλλω εις σέ· έκλεξον εις σεαυτόν εν εκ τούτων, και θέλω σοι κάμει αυτό.
\par 11 Ήλθε λοιπόν ο Γαδ προς τον Δαβίδ και είπε προς αυτόν, Ούτω λέγει Κύριος· Έκλεξοω εις σεαυτόν,
\par 12 ή τρία έτη πείνης, ή τρεις μήνας να φθείρησαι έμπροσθεν των πολεμίων σου και να σε προφθάνη η μάχαιρα των εχθρών σου, ή τρεις ημέρας την ρομφαίαν του Κυρίου και το θανατικόν εν τη γη, και τον άγγελον του Κυρίου εξολοθρεύοντα εις πάντα τα όρια του Ισραήλ. Τώρα λοιπόν ιδέ ποίον λόγον θέλω αναφέρει προς τον αποστείλαντά με.
\par 13 Και είπεν ο Δαβίδ προς τον Γαδ, Στενά μοι πανταχόθεν σφόδρα· ας πέσω λοιπόν εις την χείρα του Κυρίου, διότι οι οικτιρμοί αυτού είναι πολλοί σφόδρα· εις χείρα δε ανθρώπου ας μη πέσω.
\par 14 Έδωκε λοιπόν ο Κύριος θανατικόν επί τον Ισραήλ· και έπεσον εκ του Ισραήλ εβδομήκοντα χιλιάδες ανδρών.
\par 15 Και απέστειλεν ο Θεός άγγελον εις Ιερουσαλήμ, διά να εξολοθρεύση αυτήν· και ενώ εξωλόθρευεν, είδεν ο Κύριος και μετεμελήθη περί του κακού, και είπε προς τον άγγελον τον εξολοθρεύοντα, Αρκεί ήδη· σύρε την χείρα σου. Ίστατο δε ο άγγελος του Κυρίου πλησίον του αλωνίου του Ορνάν του Ιεβουσαίου.
\par 16 Και υψώσας ο Δαβίδ τους οφθαλμούς αυτού, είδε τον άγγελον του Κυρίου ιστάμενον αναμέσον της γης και του ουρανού, έχοντα εν τη χειρί αυτού την ρομφαίαν αυτού γεγυμνωμένην, εκτεταμένην επί Ιερουσαλήμ· και έπεσεν ο Δαβίδ και οι πρεσβύτεροι, ενδεδυμένοι σάκκους, κατά πρόσωπον αυτών.
\par 17 Και είπεν ο Δαβίδ προς τον Θεόν, Δεν είμαι εγώ ο προστάξας να απαριθμήσωσι τον λαόν; εγώ βεβαίως είμαι ο αμαρτήσας και πράξας την κακίαν· ταύτα δε τα πρόβατα τι έπραξαν; επ' εμέ λοιπόν, Κύριε Θεέ μου, και επί τον οίκον του πατρός μου έστω η χειρ σου, και μη επί τον λαόν σου προς απώλειαν.
\par 18 Τότε ο άγγελος του Κυρίου προσέταξε τον Γαδ να είπη προς τον Δαβίδ, να αναβή ο Δαβίδ και να στήση θυσιαστήριον εις τον Κύριον εν τω αλωνίω του Ορνάν του Ιεβουσαίου.
\par 19 Και ανέβη ο Δαβίδ, κατά τον λόγον του Γαδ, τον οποίον ελάλησεν εν ονόματι Κυρίου.
\par 20 Και στραφείς ο Ορνάν είδε τον άγγελον· και εκρύφθησαν οι τέσσαρες υιοί αυτού μετ' αυτού. Ο δε Ορνάν ηλώνιζε σίτον.
\par 21 Και καθώς ήλθεν ο Δαβίδ προς τον Ορνάν, αναβλέψας ο Ορνάν και ιδών τον Δαβίδ, εξήλθεν εκ του αλωνίου και προσεκύνησε τον Δαβίδ κατά πρόσωπον έως εδάφους.
\par 22 Και είπεν ο Δαβίδ προς τον Ορνάν, Δος μοι τον τόπον του αλωνίου, διά να οικοδομήσω εν αυτώ θυσιαστήριον εις τον Κύριον· δος μοι αυτόν εις την αξίαν τιμήν· διά να σταθή η πληγή από του λαού.
\par 23 Και είπεν ο Ορνάν προς τον Δαβίδ, Λάβε αυτό εις σεαυτόν, και ας κάμη ο κύριός μου ο βασιλεύς το αρεστόν εις τους οφθαλμούς αυτού· Ιδού, δίδω τους βόας διά ολοκαύτωμα και τα αλωνικά εργαλεία διά ξύλα και τον σίτον διά προσφοράν εξ αλφίτων· τα πάντα δίδω.
\par 24 Ο δε βασιλεύς Δαβίδ είπε προς τον Ορνάν, Ουχί· αλλ' εξάπαντος θέλω αγοράσει αυτό εις την αξίαν τιμήν· διότι δεν θέλω λάβει το σον διά τον Κύριον, ουδέ θέλω προσφέρει ολοκαύτωμα δωρεάν.
\par 25 Και έδωκεν ο Δαβίδ εις τον Ορνάν, διά τον τόπον, εξακοσίους σίκλους χρυσίου κατά βάρος.
\par 26 Και ωκοδόμησεν εκεί ο Δαβίδ θυσιαστήριον εις τον Κύριον, και προσέφερεν ολοκαυτώματα και ειρηνικάς προσφοράς και επεκαλέσθη τον Κύριον· και επήκουσεν αυτού, αποστείλας εξ ουρανού πυρ επί το θυσιαστήριον της ολοκαυτώσεως.
\par 27 Και προσέταξε Κύριος τον άγγελον, και έστρεψε την ρομφαίαν αυτού εις την θήκην αυτής.
\par 28 Κατ' εκείνον τον καιρόν, ότε ο Δαβίδ είδεν ότι ο Κύριος επήκουσεν αυτού εν τω αλωνίω του Ορνάν του Ιεβουσαίου, εθυσίασεν εκεί.
\par 29 Διότι η σκηνή του Κυρίου, την οποίαν έκαμεν ο Μωϋσής εν τη ερήμω, και το θυσιαστήριον της ολοκαυτώσεως ήσαν κατά τον καιρόν εκείνον εν τω υψηλώ τόπω εν Γαβαών.
\par 30 Και δεν ηδύνατο ο Δαβίδ να υπάγη ενώπιον αυτής διά να ερωτήση τον Θεόν, επειδή εφοβείτο εξ αιτίας της ρομφαίας του αγγέλου του Κυρίου.

\chapter{22}

\par 1 Τότε είπεν ο Δαβίδ, Ούτος είναι ο οίκος Κυρίου του Θεού, και τούτο το θυσιαστήριον της ολοκαυτώσεως εις τον Ισραήλ.
\par 2 Και προσέταξεν ο Δαβίδ να συνάξωσι τους ξένους τους εν γη Ισραήλ· και κατέστησε λιθοτόμους διά να λατομήσωσι λίθους ξυστούς, προς οικοδόμησιν του οίκου του Θεού.
\par 3 Ο Δαβίδ ητοίμασε και σίδηρον πολύν, διά καρφία των θυρωμάτων των πυλών και διά τας συναρθρώσεις· και χαλκόν άφθονον αζύγιστον·
\par 4 και ξύλα κέδρινα αναρίθμητα· διότι οι Σιδώνιοι και οι Τύριοι έφερον προς τον Δαβίδ άφθονα κέδρινα ξύλα.
\par 5 Και είπεν ο Δαβίδ, Σολομών ο υιός μου είναι νέος και απαλός· ο δε οίκος όστις μέλλει να οικοδομηθή εις τον Κύριον πρέπει να ήναι εις άκρον μεγαλοπρεπής, ονομαστός και ένδοξος καθ' όλην την οικουμένην· θέλω λοιπόν κάμει ετοιμασίαν δι' αυτόν. Και έκαμεν ο Δαβίδ άφθονον ετοιμασίαν προ του θανάτου αυτού.
\par 6 Τότε εκάλεσε Σολομώντα τον υιόν αυτού και προσέταξεν εις αυτόν να οικοδομήση οίκον εις Κύριον τον Θεόν του Ισραήλ.
\par 7 Και είπεν ο Δαβίδ προς τον Σολομώντα, Υιέ μου, εγώ μεν επεθύμησα εν τη καρδία μου να οικοδομήσω οίκον εις το όνομα Κυρίου του Θεού μου·
\par 8 πλην έγεινε λόγος Κυρίου προς εμέ, λέγων, Αίμα πολύ έχυσας και πολέμους μεγάλους έκαμες· δεν θέλεις οικοδομήσει οίκον εις το όνομά μου, διότι αίματα πολλά έχυσας επί της γης ενώπιόν μου·
\par 9 ιδού, θέλει γεννηθή εις σε υιός, όστις θέλει είσθαι ανήρ αναπαύσεως· και θέλω αναπαύσει αυτόν από πάντων των εχθρών αυτού κύκλω· διότι Σολομών θέλει είσθαι το όνομα αυτού, και εν ταις ημέραις αυτού θέλω δώσει ειρήνην και ησυχίαν εις τον Ισραήλ·
\par 10 ούτος θέλει οικοδομήσει οίκον εις το όνομά μου· και ούτος θέλει είσθαι εις εμέ υιός, και εγώ πατήρ εις αυτόν· και θέλω στερεώσει τον θρόνον της βασιλείας αυτού επί τον Ισραήλ έως αιώνος.
\par 11 Τώρα, υιέ μου, ο Κύριος έστω μετά σού· και ευοδού και οικοδόμησον τον οίκον Κυρίου του Θεού σου, καθώς ελάλησε περί σου.
\par 12 Μόνον ο Κύριος να σοι δώση σοφίαν και σύνεσιν και να σε καταστήση επί τον Ισραήλ, διά να φυλάττης τον νόμον Κυρίου του Θεού σου.
\par 13 Τότε θέλεις ευοδωθή, εάν προσέχης να εκπληροίς τα διατάγματα και τας κρίσεις, τας οποίας ο Κύριος προσέταξεν εις τον Μωϋσήν περί του Ισραήλ· ενδυναμού και ανδρίζου· μη φοβού και μη πτοηθής.
\par 14 Και ιδού, εγώ κατά την πτωχείαν μου ητοίμασα διά τον οίκον του Κυρίου εκατόν χιλιάδας ταλάντων χρυσίου και χιλίας χιλιάδας ταλάντων αργυρίου· χαλκόν δε και σίδηρον αζύγιστον, διότι είναι άφθονος· ητοίμασα δε και ξύλα και λίθους· και συ πρόσθες εις ταύτα.
\par 15 Έχεις δε εργάτας εις πλήθος, λιθοτόμους και κτίστας και ξυλουργούς, και παντός είδους σοφούς εις παν έργον.
\par 16 Του χρυσού, του αργύρου και του χαλκού και του σιδήρου αριθμός δεν είναι. Σηκώθητι και κάμνε· και ο Κύριος έστω μετά σου.
\par 17 Ο Δαβίδ προσέταξεν έτι εις πάντας τους άρχοντας του Ισραήλ να βοηθήσωσι τον Σολομώντα τον υιόν αυτού, λέγων,
\par 18 Δεν είναι με σας Κύριος ο Θεός σας και έδωκεν εις εσάς ανάπαυσιν πανταχόθεν; διότι παρέδωκεν εις την χείρα μου τους κατοικούντας την γήν· και η γη υπετάχθη έμπροσθεν του Κυρίου και έμπροσθεν του λαού αυτού.
\par 19 Δότε λοιπόν την καρδίαν σας και την ψυχήν σας εις το να ζητήτε Κύριον τον Θεόν σας· και σηκώθητε και οικοδομήσατε το αγιαστήριον Κυρίου του Θεού, διά να φέρητε την κιβωτόν της διαθήκης του Κυρίου και τα άγια σκεύη του Θεού εις τον οίκον, όστις μέλλει να οικοδομηθή επί τω ονόματι του Κυρίου.

\chapter{23}

\par 1 Και αφού εγήρασεν ο Δαβίδ και ήτο πλήρης ημερών, έκαμε Σολομώντα τον υιόν αυτού βασιλέα επί τον Ισραήλ.
\par 2 Και συνήγαγε πάντας τους άρχοντας του Ισραήλ και τους ιερείς και τους Λευΐτας.
\par 3 Οι δε Λευΐται ήσαν απηριθμημένοι από ηλικίας τριάκοντα ετών και επάνω· και ο αριθμός αυτών κατά κεφαλήν αυτών, κατά άνδρα, ήτο τριάκοντα οκτώ χιλιάδες.
\par 4 Εκ τούτων εικοσιτέσσαρες χιλιάδες ήσαν εργοδιώκται εις το έργον του οίκου του Κυρίου· και εξ χιλιάδες επιστάται και κριταί·
\par 5 και τέσσαρες χιλιάδες πυλωροί· και τέσσαρες χιλιάδες υμνούντες τον Κύριον, με τα όργανα, τα οποία έκαμα, είπεν ο Δαβίδ, διά να υμνώσι τον Κύριον.
\par 6 Και διήρεσεν αυτούς ο Δαβίδ εις τάξεις κατά τους υιούς του Λευΐ, Γηρσών, Καάθ και Μεραρί.
\par 7 Εκ των Γηρσωνιτών ήσαν Λααδάν και Σιμεΐ.
\par 8 Οι υιοί του Λααδάν ήσαν Ιεχιήλ ο άρχων και Ζαιθάμ και Ιωήλ, τρεις.
\par 9 Οι υιοί του Σιμεΐ, Σελωμείθ και Αζιήλ και Χαρράν, τρεις. Ούτοι ήσαν οι αρχηγοί των πατριών του Λααδάν.
\par 10 Υιοί δε του Σιμεΐ, Ιαάθ, Ζινά και Ιεούς και Βεριά. Ούτοι ήσαν οι υιοί του Σιμεΐ, τέσσαρες.
\par 11 Και ο Ιαάθ ήτο ο αρχηγός, και Ζιζά ο δεύτερος· ο δε Ιεούς και Βεριά δεν είχον πολλούς υιούς· διά τούτο ηριθμήθησαν ομού, ως μία πατριά.
\par 12 Οι υιοί του Καάθ, Αμράμ, Ισαάρ, Χεβρών και Οζιήλ, τέσσαρες.
\par 13 Οι υιοί του Αμράμ, Ααρών και Μωϋσής· και ο Ααρών ήτο κεχωρισμένος, αυτός και οι υιοί αυτού, διά να αγιάζωσι τα αγιώτατα πράγματα πάντοτε, διά να θυμιάζωσιν ενώπιον του Κυρίου, να λειτουργώσιν εις αυτόν και να ευλογώσιν εν τω ονόματι αυτού διά παντός.
\par 14 Του δε Μωϋσέως του ανθρώπου του Θεού, οι υιοί αυτού συνηριθμήθησαν μετά της φυλής του Λευΐ.
\par 15 Οι υιοί του Μωϋσέως ήσαν Γηρσώμ και Ελιέζερ.
\par 16 Εκ των υιών του Γηρσώμ ο Σεβουήλ ήτο ο αρχηγός.
\par 17 Και οι υιοί του Ελιέζερ ήσαν Ρεαβίας ο αρχηγός· και δεν είχεν ο Ελιέζερ άλλους υιούς· του Ρεαβιά δε οι υιοί ήσαν πάμπολλοι.
\par 18 Εκ των υιών του Ισαάρ ο Σελωμείθ ήτο ο αρχηγός.
\par 19 Οι υιοί του Χεβρών ήσαν Ιερίας ο πρώτος, Αμαρίας ο δεύτερος, Ιααζιήλ ο τρίτος και Ιεκαμεάμ ο τέταρτος.
\par 20 Οι υιοί του Οζιήλ, Μιχά ο πρώτος και Ιεσιά ο δεύτερος.
\par 21 Οι υιοί του Μεραρί, Μααλί και Μουσί· οι υιοί του Μααλί, Ελεάζαρ και Κείς.
\par 22 Απέθανε δε ο Ελεάζαρ, μη έχων υιούς, αλλά θυγατέρας· και έλαβον αυτάς οι αδελφοί αυτών οι υιοί του Κείς.
\par 23 Οι υιοί του Μουσί, Μααλί και Εδέρ και Ιερεμώθ, τρεις.
\par 24 Ούτοι ήσαν οι υιοί του Λευΐ, κατά τους οίκους των πατέρων αυτών, αρχηγοί των πατριών, κατά την απαρίθμησιν αυτών, απαριθμηθέντες κατ' όνομα, κατά κεφαλήν, οίτινες έκαμνον τα έργα της υπηρεσίας του οίκου του Κυρίου, από ηλικίας είκοσι ετών και επάνω.
\par 25 Διότι ο Δαβίδ είπε, Κύριος ο Θεός του Ισραήλ έδωκεν ανάπαυσιν εις τον λαόν αυτού, και θέλει κατοικεί εν Ιερουσαλήμ διά παντός·
\par 26 οι δε Λευΐται δεν θέλουσι πλέον βαστάζει την σκηνήν και πάντα τα σκεύη αυτής διά την υπηρεσίαν αυτής.
\par 27 Όθεν κατά τους τελευταίους λόγους του Δαβίδ, οι υιοί Λευΐ ήσαν απηριθμημένοι από είκοσι ετών και επάνω·
\par 28 επειδή το έργον αυτών ήτο να παρίστανται εις τους υιούς του Ααρών, εν τη υπηρεσία του οίκου του Κυρίου, επί τας αυλάς και επί τα οικήματα και επί τον καθαρισμόν πάντων των αγίων πραγμάτων και εις το να κάμνωσι την υπηρεσίαν του οίκου του Θεού·
\par 29 και διά τους άρτους της προθέσεως και διά την σεμίδαλιν εις τας εξ αλφίτων προσφοράς και διά τα άζυμα λάγανα και διά τας τηγανίτας και διά τα φρυγανωτά και διά παν είδος μέτρου·
\par 30 και διά να ίστανται καθ' εκάστην πρωΐαν και εσπέραν, διά να υμνώσι και να δοξολογώσι τον Κύριον·
\par 31 και διά να προσφέρωσιν εις τον Κύριον πάντα τα ολοκαυτώματα εν τοις σάββασιν, εν ταις νεομηνίαις και εν ταις επισήμοις εορταίς, κατά τον αριθμόν, κατά το διατεταγμένον εις αυτούς, πάντοτε ενώπιον του Κυρίου·
\par 32 και διά να φυλάττωσι την φυλακήν της σκηνής του μαρτυρίου και την φυλακήν του αγιαστηρίου και την φυλακήν των υιών του Ααρών των αδελφών αυτών εν τη υπηρεσία του οίκου του Κυρίου.

\chapter{24}

\par 1 Και αύται ήσαν αι διαιρέσεις των υιών του Ααρών· οι υιοί του Ααρών, Ναδάβ και Αβιούδ, Ελεάζαρ και Ιθάμαρ.
\par 2 Και απέθανον ο Ναδάβ και ο Αβιούδ έμπροσθεν του πατρός αυτών, και δεν είχον υιούς· όθεν ιεράτευσαν ο Ελεάζαρ και ο Ιθάμαρ.
\par 3 Και διήρεσεν αυτούς ο Δαβίδ, τον τε Σαδώκ εκ των υιών Ελεάζαρ, και τον Αχιμέλεχ εκ των υιών του Ιθάμαρ, κατά τα χρέη αυτών εις την υπηρεσίαν αυτών.
\par 4 Ευρέθησαν δε πλειότεροι αρχηγοί εκ των υιών Ελεάζαρ, παρά εκ των υιών Ιθάμαρ· και διηρέθησαν ούτω· εκ των υιών Ελεάζαρ ήσαν δεκαέξ αρχηγοί οίκου πατέρων· και εκ των υιών Ιθάμαρ οκτώ αρχηγοί του οίκου των πατέρων αυτών.
\par 5 Διήρεσαν δε αυτούς διά κλήρων, τούτους προς εκείνους· διότι διευθυνταί του αγιαστηρίου και διευθυνταί του οίκου του Θεού ήσαν εκ των υιών Ελεάζαρ και εκ των υιών Ιθάμαρ.
\par 6 Και κατέγραψεν αυτούς Σεμαΐας ο υιός του Ναθαναήλ ο γραμματεύς, ο εκ των Λευϊτών, έμπροσθεν του βασιλέως και των αρχόντων και Σαδώκ του ιερέως και Αχιμέλεχ υιού του Αβιάθαρ και έμπροσθεν των αρχηγών των πατριών των ιερέων και Λευϊτών, λαμβανομένης μιας πατριάς εκ του Ελεάζαρ και μιας εκ του Ιθάμαρ.
\par 7 Ο πρώτος δε κλήρος εξήλθεν εις τον Ιωϊαρείβ, ο δεύτερος εις τον Ιεδαΐαν,
\par 8 ο τρίτος εις τον Χαρήμ, ο τέταρτος εις τον Σεωρήμ,
\par 9 ο πέμπτος εις τον Μαλχίαν, ο έκτος εις τον Μεϊαμείν,
\par 10 ο έβδομος εις τον Ακκώς, ο όγδοος εις τον Αβιά,
\par 11 ο ένατος εις τον Ιησούν, ο δέκατος εις τον Σεχανίαν,
\par 12 ο ενδέκατος εις τον Ελιασείβ, ο δωδέκατος εις τον Ιακείμ,
\par 13 ο δέκατος τρίτος εις τον Ουφφά, ο δέκατος τέταρτος εις τον Ιεσεβάβ,
\par 14 ο δέκατος πέμπτος εις τον Βιλγά, ο δέκατος έκτος εις τον Ιμμήρ,
\par 15 ο δέκατος έβδομος εις τον Εζείρ, ο δέκατος όγδοος εις τον Αφισής,
\par 16 ο δέκατος ένατος εις τον Πεθαΐα, ο εικοστός εις τον Ιεζεκιήλ,
\par 17 ο εικοστός πρώτος εις τον Ιαχείν, ο εικοστός δεύτερος εις τον Γαμούλ,
\par 18 ο εικοστός τρίτος εις τον Δελαΐαν, ο εικοστός τέταρτος εις τον Μααζίαν.
\par 19 Αύται ήσαν αι διατάξεις αυτών εις την υπηρεσίαν αυτών, διά να εισέρχωνται εις τον οίκον του Κυρίου κατά το διατεταγμένον εις αυτούς διά χειρός Ααρών του πατρός αυτών, ως προσέταξεν εις αυτόν Κύριος ο Θεός του Ισραήλ.
\par 20 Περί δε των επιλοίπων υιών Λευΐ· εκ των υιών Αμράμ ήτο ο Σουβαήλ, εκ των υιών Σουβαήλ ο Ιεδαΐας.
\par 21 Περί του Ρεαβιά· εκ των υιών Ρεαβιά ο πρώτος ήτο ο Ιεσία.
\par 22 Εκ των Ισααριτών ο Σελωμώθ· εκ των υιών Σελωμώθ ο Ιαάθ.
\par 23 Οι δε υιοί Χεβρών ήσαν Ιεριάς ο πρώτος, Αμαρίας ο δεύτερος, Ιααζιήλ ο τρίτος, Ιεκαμεάμ ο τέταρτος.
\par 24 Εκ των υιών Οζιήλ Μιχά· εκ των υιών του Μιχά Σαμίρ.
\par 25 Ο αδελφός του Μιχά ήτο ο Ιεσία· εκ των υιών Ιεσία ο Ζαχαρίας.
\par 26 Οι υιοί του Μεραρί ήσαν Μααλί και Μουσί· οι υιοί του Ιααζία, Βενώ.
\par 27 Οι υιοί του Μεραρί διά του Ιααζία, Βενώ και Σωάμ και Ζακχούρ και Ιβρί.
\par 28 Εκ του Μααλί ήτο ο Ελεάζαρ, όστις δεν είχεν υιούς.
\par 29 Περί δε του Κείς· οι υιοί του Κείς, ο Ιεραμεήλ.
\par 30 Και οι υιοί του Μουσί, Μααλί και Εδέρ και Ιεριμώθ. Ούτοι ήσαν οι υιοί των Λευϊτών, κατά τους οίκους των πατριών αυτών.
\par 31 Έρριψαν και ούτοι κλήρους, καθώς οι αδελφοί αυτών οι υιοί του Ααρών, έμπροσθεν του βασιλέως Δαβίδ και του Σαδώκ και του Αχιμέλεχ και των αρχηγών των πατριών των ιερέων και Λευϊτών, εξισουμένων των πρώτων πατριών μετά των αδελφών αυτών των νεωτέρων.

\chapter{25}

\par 1 Ο Δαβίδ λοιπόν και οι άρχοντες του στρατεύματος διήρεσαν εις την υπηρεσίαν τους υιούς του Ασάφ και του Αιμάν και του Ιεδουθούν, διά να υμνωδώσιν εν κιθάραις, εν ψαλτηρίοις και εν κυμβάλοις· και ο αριθμός των εργαζομένων κατά την υπηρεσίαν αυτών, ήτο,
\par 2 εκ των υιών Ασάφ, Ζακχούρ και Ιωσήφ και Νεθανίας και Ασαρηλά, υιοί Ασάφ, υπό την οδηγίαν του Ασάφ, του υμνωδούντος, κατά την διάταξιν του βασιλέως·
\par 3 του Ιεδουθούν· οι υιοί του Ιεδουθούν, Γεδαλίας και Σερί και Ιεσαΐας, Σιμεΐ, Ασαβίας και Ματταθίας, εξ, υπό την οδηγίαν του πατρός αυτών Ιεδουθούν, όστις υμνώδει εν κιθάρα, υμνών και δοξολογών τον Κύριον·
\par 4 του Αιμάν· οι υιοί του Αιμάν, Βουκκίας, Ματθανίας, Οζιήλ, Σεβουήλ και Ιεριμώθ, Ανανίας, Ανανί, Ελιαθά, Γιδδαλθί και Ρομαμθί-έζερ, Ιωσβεκασά, Μαλλωθί, Ωθίρ και Μααζιώθ·
\par 5 πάντες ούτοι ήσαν οι υιοί του Αιμάν του βλέποντος του βασιλέως εις τους λόγους του Θεού, διωρισμένοι εις το να υψόνωσι το κέρας. Έδωκε δε ο Θεός εις τον Αιμάν δεκατέσσαρας υιούς και τρεις θυγατέρας.
\par 6 Πάντες ούτοι υπό την οδηγίαν του πατρός αυτών ήσαν υμνωδούντες εν τω οίκω του Κυρίου, εν κυμβάλοις, ψαλτηρίοις και κιθάραις, διά την υπηρεσίαν του οίκου του Θεού, κατά την διάταξιν του βασιλέως προς τον Ασάφ και Ιεδουθούν και Αιμάν.
\par 7 Και έγεινεν ο αριθμός αυτών, μετά των αδελφών αυτών, των δεδιδαγμένων τα άσματα του Κυρίου, διακόσιοι ογδοήκοντα οκτώ, πάντες συνετοί.
\par 8 Και έρριψαν κλήρους περί της υπηρεσίας, εξ ίσου ο μικρός καθώς ο μεγάλος, ο διδάσκαλος καθώς ο μαθητής.
\par 9 Ο πρώτος δε κλήρος εξήλθε διά τον Ασάφ εις τον Ιωσήφ· ο δεύτερος εις τον Γεδαλίαν· ούτος και οι αδελφοί αυτού και οι υιοί αυτού ήσαν δώδεκα.
\par 10 Ο τρίτος εις τον Ζακχούρ· ούτος, οι υιοί αυτού και οι αδελφοί αυτού, δώδεκα.
\par 11 Ο τέταρτος εις τον Ισερί· ούτος, οι υιοί αυτού και οι αδελφοί αυτού, δώδεκα.
\par 12 Ο πέμπτος εις τον Νεβανίαν· ούτος, οι υιοί αυτού και οι αδελφοί αυτού, δώδεκα.
\par 13 Ο έκτος εις τον Βουκκίαν· ούτος, οι υιοί αυτού και οι αδελφοί αυτού, δώδεκα.
\par 14 Ο έβδομος εις τον Ιεσαρηλά· ούτος, οι υιοί αυτού και οι αδελφοί αυτού, δώδεκα.
\par 15 Ο όγδοος εις τον Ιεσαΐαν· ούτος, οι υιοί αυτού, και οι αδελφοί αυτού, δώδεκα.
\par 16 Ο έννατος εις τον Ματθανίαν· ούτος, οι υιοί αυτού και οι αδελφοί αυτού, δώδεκα.
\par 17 Ο δέκατος εις τον Σιμεΐ· ούτος, οι υιοί αυτού και οι αδελφοί αυτού, δώδεκα.
\par 18 Ο ενδέκατος εις τον Αζαρεήλ· ούτος, οι υιοί αυτού και οι αδελφοί αυτού, δώδεκα.
\par 19 Ο δωδέκατος εις τον Ασαβίαν· ούτος, οι υιοί αυτού και οι αδελφοί αυτού, δώδεκα.
\par 20 Ο δέκατος τρίτος εις τον Σουβαήλ· ούτος, οι υιοί αυτού και οι αδελφοί αυτού, δώδεκα.
\par 21 Ο δέκατος τέταρτος εις τον Ματταθίαν· ούτος, οι υιοί αυτού και οι αδελφοί αυτού, δώδεκα.
\par 22 Ο δέκατος πέμπτος εις τον Ιερεμώθ· ούτος, οι υιοί αυτού και οι αδελφοί αυτού, δώδεκα.
\par 23 Ο δέκατος έκτος εις τον Ανανίαν· ούτος, οι υιοί αυτού και οι αδελφοί αυτού, δώδεκα.
\par 24 Ο δέκατος έβδομος εις τον Ιωσβεκασά· ούτος, οι υιοί αυτού και οι αδελφοί αυτού, δώδεκα.
\par 25 Ο δέκατος όγδοος εις τον Ανανί· ούτος, οι υιοί αυτού και οι αδελφοί αυτού, δώδεκα.
\par 26 Ο δέκατος έννατος εις τον Μαλλωθί· ούτος, οι υιοί αυτού και οι αδελφοί αυτού, δώδεκα.
\par 27 Ο εικοστός εις τον Ελιαθά· ούτος, οι υιοί αυτού και οι αδελφοί αυτού, δώδεκα.
\par 28 Ο εικοστός πρώτος εις τον Ωθίρ· ούτος, οι υιοί αυτού και οι αδελφοί αυτού, δώδεκα.
\par 29 Ο εικοστός δεύτερος εις τον Γιδδαλθί· ούτος, οι υιοί αυτού και οι αδελφοί αυτού, δώδεκα.
\par 30 Ο εικοστός τρίτος εις τον Μααζιώθ· ούτος, οι υιοί αυτού και οι αδελφοί αυτού, δώδεκα.
\par 31 Ο εικοστός τέταρτος εις τον Ρωμαμθί-έζερ· ούτος οι υιοί αυτού και οι αδελφοί αυτού, δώδεκα.

\chapter{26}

\par 1 Περί δε των διαιρέσεων των πυλωρών· εκ των Κοριτών ήτο Μεσελεμίας ο υιός του Κορέ, εκ των υιών Ασάφ.
\par 2 Και οι υιοί του Μεσελεμία ήσαν Ζαχαρίας ο πρωτότοκος, Ιεδιαήλ ο δεύτερος, Ζεβαδίας ο τρίτος, Ιαθνιήλ ο τέταρτος,
\par 3 Ελάμ ο πέμπτος, Ιωανάν ο έκτος, Ελιωηνάϊ ο έβδομος.
\par 4 οι δε υιοί του Ωβήδ-εδώμ, Σεμαΐας ο πρωτότοκος, Ιωζαβάδ ο δεύτερος, Ιωάχ ο τρίτος και Σαχάρ ο τέταρτος και Ναθαναήλ ο πέμπτος,
\par 5 Αμμιήλ ο έκτος, Ισσάχαρ ο έβδομος, Φεουλθαΐ ο όγδοος· διότι ο Θεός ευλόγησεν αυτόν.
\par 6 Και εις Σεμαΐαν τον υιόν αυτού εγεννήθησαν υιοί, οίτινες εξουσίαζον επί του πατρικού οίκου αυτών· διότι ήσαν δυνατοί εν ισχύϊ.
\par 7 Οι υιοί του Σεμαΐα, Γοθνί και Ραφαήλ και Ωβήδ και Ελζαβάδ, των οποίων οι αδελφοί ήσαν ισχυροί, ο Ελιού, και ο Σεμαχίας.
\par 8 Πάντες ούτοι εκ των υιών του Ωβήδ-εδώμ, αυτοί και οι υιοί αυτών και οι αδελφοί αυτών, ισχυροί και άξιοι διά την υπηρεσίαν, εξήκοντα δύο ήσαν του Ωβήδ-εδώμ.
\par 9 Και ο Μεσελεμίας είχεν υιούς και αδελφούς, ισχυρούς, δεκαοκτώ.
\par 10 Και ο Ωσά, εκ των υιών του Μεραρί, είχεν υιούς· Σιμρί τον πρώτον, διότι πρωτότοκος δεν ήτο, ο πατήρ όμως αυτού έκαμεν αυτόν πρώτον·
\par 11 Χελκίαν τον δεύτερον, Τεβαλίαν τον τρίτον, Ζαχαρίαν τον τέταρτον· πάντες οι υιοί και οι αδελφοί του Ωσά ήσαν δεκατρείς.
\par 12 Μεταξύ αυτών έγειναν αι διαιρέσεις των πυλωρών· οι αρχηγοί των δυνατών είχον τας φυλακάς εξ ίσου με τους αδελφούς αυτών, διά να υπηρετώσιν εν τω οίκω του Κυρίου.
\par 13 Και έρριψαν κλήρους εξ ίσου ο μικρός καθώς και ο μεγάλος, κατ' οίκον των πατέρων αυτών, περί εκάστης πύλης.
\par 14 Και έπεσεν ο κλήρος της προς ανατολάς εις τον Σελεμίαν. Τότε έρριψαν κλήρους διά τον Ζαχαρίαν τον υιόν αυτού, σύμβουλον σοφόν· και ο κλήρος αυτού εξήλθε διά την προς βορράν.
\par 15 Εις τον Ωβήδ-εδώμ, διά την προς νότον· και εις τους υιούς αυτού, διά τον οίκον της συνάξεως.
\par 16 Εις τον Σουφίμ και Ωσά, διά την προς δυσμάς, μετά της πύλης Σαλεχέθ, πλησίον της οδού της αναβάσεως, φυλακή απέναντι φυλακής.
\par 17 Κατά ανατολάς ήσαν εξ Λευΐται, προς βορράν τέσσαρες την ημέραν, προς νότον τέσσαρες την ημέραν και προς τον οίκον της συνάξεως ανά δύο.
\par 18 Εις Παρβάρ προς δυσμάς τέσσαρες κατά την οδόν της αναβάσεως, και δύο εις Παρβάρ.
\par 19 Αύται είναι αι διαιρέσεις των πυλωρών μεταξύ των υιών Κορέ και μεταξύ των υιών Μεραρί.
\par 20 Και εκ των Λευϊτών ο Αχιά ήτο επί τους θησαυρούς του οίκου του Θεού και επί τους θησαυρούς των αφιερωμάτων.
\par 21 Περί δε των υιών του Λααδάν· οι υιοί του Γηρσωνίτου Λααδάν, αρχηγοί των πατριών του Λααδάν του Γηρσωνίτου ήσαν Ιεχιήλ.
\par 22 οι υιοί του Ιεχήλ, Ζαιθάμ και Ιωήλ ο αδελφός αυτού, οίτινες ήσαν επί τους θησαυρούς του οίκου του Κυρίου.
\par 23 Περί δε των Αμραμιτών, Ισααριτών, Χεβρωνιτών και Οζιηλιτών·
\par 24 ο μεν Σεβουήλ ο υιός του Γηρσώμ, υιού του Μωϋσέως· ήτο επιστάτης επί τους θησαυρούς.
\par 25 Οι δε αδελφοί αυτού εκ του Ελιέζερ, του οποίου ο υιός ήτο ο Ρεαβίας, και Ιεσαΐας ο υιός τούτου, και Ιωράμ ο υιός τούτου, και Ζιχρί ο υιός τούτου, και Σελωμείθ ο υιός τούτου·
\par 26 ο Σελωμείθ ούτος και οι αδελφοί αυτού ήσαν επί πάντας τους θησαυρούς των αφιερωμάτων, τα οποία Δαβίδ ο βασιλεύς και οι άρχοντες των πατριών, οι χιλίαρχοι και οι εκατόνταρχοι, και οι αρχηγοί του στρατεύματος, αφιέρωσαν.
\par 27 Εκ των πολέμων και εκ των λαφύρων έκαμον την αφιέρωσιν, διά να επισκευάζωσι τον οίκον του Κυρίου.
\par 28 Και παν ό,τι Σαμουήλ ο βλέπων, και Σαούλ ο υιός του Κείς, και Αβενήρ ο υιός του Νηρ, και Ιωάβ ο υιός της Σερουΐας, αφιέρωσαν, παν αφιέρωμα ήτο υπό την χείρα του Σελωμείθ και των αδελφών αυτού.
\par 29 Περί των Ισααριτών· ο Χενανίας και οι υιοί αυτού ήσαν διά τας έξω υποθέσεις επί του Ισραήλ, επιστάται και κριταί.
\par 30 Περί δε των Χεβρωνιτών· ο Ασαβίας και οι αδελφοί αυτού, ισχυροί, χίλιοι επτακόσιοι ήσαν έφοροι του Ισραήλ εντεύθεν του Ιορδάνου προς δυσμάς, διά πάσας τας υποθέσεις του Κυρίου και διά την υπηρεσίαν του βασιλέως.
\par 31 Μεταξύ των Χεβρωνιτών ήτο Ιερίας ο αρχηγός, μεταξύ των Χεβρωνιτών κατά τας γενεάς αυτών, κατά τας πατριάς. Εν τω τεσσαρακοστώ έτει της βασιλείας του Δαβίδ εξητάσθησαν και ευρέθησαν μεταξύ αυτών δυνατοί εν ισχύϊ, εν Ιαζήρ Γαλαάδ.
\par 32 Και οι αδελφοί αυτού, ισχυροί, ήσαν δύο χιλιάδες και επτακόσιοι αρχηγοί πατριών, τους οποίους κατέστησε Δαβίδ ο βασιλεύς επί τους Ρουβηνίτας και τους Γαδίτας και το ήμισυ της φυλής Μανασσή, διά παν πράγμα του Θεού και διά τας υποθέσεις του βασιλέως.

\chapter{27}

\par 1 Οι δε υιοί Ισραήλ κατά την απαρίθμησιν αυτών, οι αρχηγοί των πατριών και οι χιλίαρχοι και οι εκατόνταρχοι και οι αξιωματικοί αυτών οι υπηρετούντες τον βασιλέα, καθ' ύλην την τάξιν των διαιρέσεων, αίτινες εισήρχοντο και εξήρχοντο κατά μήνα εις πάντας τους μήνας του ενιαυτού, ήσαν εικοσιτέσσαρες χιλιάδες εις εκάστην διαίρεσιν.
\par 2 Επί της διαιρέσεως της πρώτης, διά τον πρώτον μήνα, ήτο Ιασωβεάμ ο υιός του Ζαβδιήλ· και εν τη διαιρέσει αυτού ήσαν εικοσιτέσσαρες χιλιάδες.
\par 3 Ούτος ήτο εκ των υιών του Φαρές, άρχων πάντων των αρχόντων των στρατευμάτων διά τον πρώτον μήνα.
\par 4 Και επί της διαιρέσεως του δευτέρου μηνός ήτο Δωδαΐ ο Αχωχίτης· και της διαιρέσεως αυτού άρχων ο Μικλώθ· εν τη διαιρέσει αυτού ήσαν ομοίως εικοσιτέσσαρες χιλιάδες.
\par 5 Ο τρίτος αρχηγός του στρατεύματος διά τον τρίτον μήνα ήτο Βεναΐας ο υιός του Ιωδαέ, πρώτος αξιωματικός· και εν τη διαιρέσει αυτού ήσαν εικοσιτέσσαρες χιλιάδες.
\par 6 Ούτος είναι ο Βεναΐας ο δυνατός μεταξύ των τριάκοντα και επί των τριάκοντα· και επί της διαιρέσεως αυτού ήτο Αμμιζαβάδ ο υιός αυτού.
\par 7 Ο τέταρτος διά τον τέταρτον μήνα Ασαήλ ο αδελφός του Ιωάβ, και μετ' αυτόν Ζεβαδίας ο υιός αυτού· και εν τη διαιρέσει αυτού εικοσιτέσσαρες χιλιάδες.
\par 8 Ο πέμπτος αρχηγός διά τον πέμπτον μήνα Σαμούθ ο Ιεζραΐτης· και εν τη διαιρέσει αυτού εικοσιτέσσαρες χιλιάδες.
\par 9 Ο έκτος διά τον έκτον μήνα Ιράς ο υιός του Ικκής ο Θεκωΐτης· και εν τη διαιρέσει αυτού εικοσιτέσσαρες χιλιάδες.
\par 10 Ο έβδομος διά τον έβδομον μήνα Χελής ο Φελωνίτης, εκ των υιών Εφραΐμ· και εν τη διαιρέσει αυτού εικοσιτέσσαρες χιλιάδες.
\par 11 Ο όγδοος διά τον όγδοον μήνα Σιββεχαΐ ο Χουσαθίτης, εκ των Ζαραϊτών· και εν τη διαιρέσει αυτού εικοσιτέσσαρες χιλιάδες.
\par 12 Ο έννατος διά τον έννατον μήνα Αβιέζερ ο Αναθωθίτης, εκ των Βενιαμιτών· και εν τη διαιρέσει αυτού εικοσιτέσσαρες χιλιάδες.
\par 13 Ο δέκατος διά τον δέκατον μήνα Μααραΐ ο Νετωφαθίτης, εκ των Ζαραϊτών· και εν τη διαιρέσει αυτού εικοσιτέσσαρες χιλιάδες.
\par 14 Ο ενδέκατος διά τον ενδέκατον μήνα Βεναΐας ο Πιραθωνίτης, εκ των υιών Εφραΐμ· και εν τη διαιρέσει αυτού εικοσιτέσσαρες χιλιάδες.
\par 15 Ο δωδέκατος διά τον δωδέκατον μήνα Χελδαΐ ο Νετωφαθίτης, εκ του Γοθονιήλ· και εν τη διαιρέσει αυτού εικοσιτέσσαρες χιλιάδες.
\par 16 Επί δε των φυλών του Ισραήλ, ο άρχων των Ρουβηνιτών ήτο Ελιέζερ ο υιός του Ζιχρί· των Συμεωνιτών, Σεφατίας ο υιός του Μααχά·
\par 17 των Λευϊτών, Ασαβίας ο υιός του Κεμουήλ· των Ααρωνιτών, ο Σαδώκ·
\par 18 του Ιούδα, ο Ελιού, εκ των αδελφών του Δαβίδ· του Ισσάχαρ, Αμρί ο υιός του Μιχαήλ·
\par 19 του Ζαβουλών, Ισμαΐας ο υιός του Οβαδία· του Νεφθαλί, Ιεριμώθ ο υιός του Αζριήλ·
\par 20 των υιών Εφραΐμ, Ιησούς ο υιός του Αζαζίου· της ημισείας φυλής Μανασσή, Ιωήλ ο υιός του Φεδαΐα·
\par 21 της ημισείας φυλής Μανασσή εν Γαλαάδ, Ιδδώ ο υιός του Ζαχαρία· του Βενιαμίν, Ιασιήλ ο υιός του Αβενήρ·
\par 22 του Δαν, Αζαρεήλ ο υιός του Ιεροάμ. Ούτοι ήσαν οι άρχοντες των φυλών του Ισραήλ.
\par 23 Πλην ο Δαβίδ δεν έλαβε τον αριθμόν αυτών από είκοσι ετών ηλικίας και κάτω· διότι ο Κύριος είπεν, ότι θέλει πληθύνει τον Ισραήλ ως τα άστρα του ουρανού.
\par 24 Ο Ιωάβ ο υιός της Σερουΐας ήρχισε να αριθμή, πλην δεν ετελείωσε, διότι έπεσε διά τούτο οργή κατά του Ισραήλ· όθεν δεν κατεχωρίσθη ο αριθμός μεταξύ των απαριθμήσεων εν τοις χρονικοίς του βασιλέως Δαβίδ.
\par 25 Επί δε των θησαυρών του βασιλέως ήτο Αζμαβέθ ο υιός του Αδήλ· και επί των θησαυρών των αγρών, των πόλεων και των κωμών και των φρουρίων, Ιωνάθαν ο υιός του Οζίου·
\par 26 και επί των εργαζομένων το έργον των αγρών διά την γεωργίαν της γης, Εζρί ο υιός του Χελούβ·
\par 27 και επί των αμπελώνων, Σιμεΐ ο Ραμαθαίος· και επί του εισοδήματος των αμπελώνων διά τας οινοθήκας, Ζαβδί ο Σιφμίτης·
\par 28 και επί των ελαιών και συκαμίνων των εν τη πεδινή, Βάαλ-ανάν ο Γεδερίτης· και επί των ελαιοθηκών, ο Ιωάς·
\par 29 και επί των βοών των βοσκομένων εν Σαρών, Σιτραΐ ο Σαρωνίτης· και επί των βοών των εν ταις κοιλάσι, Σαφάτ ο υιός του Αδλαΐ·
\par 30 και επί των καμήλων, Οβίλ ο Ισμαηλίτης· και επί των όνων, Ιεδαΐας ο Μερωνοθίτης·
\par 31 και επί των προβάτων, Ιαζίζ ο Αγαρίτης. Πάντες ούτοι ήσαν επιστάται των υπαρχόντων του βασιλέως Δαβίδ.
\par 32 Ο δε Ιωνάθαν, ο πατράδελφος του Δαβίδ, ήτο σύμβουλος, ανήρ συνετός και γραμματεύς· και Ιεχιήλ, ο υιός του Αχμονί, ήτο μετά των υιών του βασιλέως·
\par 33 και ο Αχιτόφελ, σύμβουλος του βασιλέως· και Χουσαΐ ο Αρχίτης, οικείος του βασιλέως·
\par 34 και μετά τον Αχιτόφελ, Ιωδαέ ο υιός του Βεναΐα και ο Αβιάθαρ· αρχιστράτηγος δε του βασιλέως ο Ιωάβ.

\chapter{28}

\par 1 Και συνεκάλεσεν ο Δαβίδ πάντας τους άρχοντας του Ισραήλ, τους άρχοντας των φυλών και τους άρχοντας των διαιρέσεων αίτινες υπηρέτουν τον βασιλέα, και τους χιλιάρχους και τους εκατοντάρχους και τους επιστάτας πάντων των υπαρχόντων και κτημάτων του βασιλέως και των υιών αυτού, μετά των ευνούχων και των ανδρείων, και πάντων των δυνατών εν ισχύϊ, εις Ιερουσαλήμ.
\par 2 Και σταθείς ο βασιλεύς Δαβίδ επί των ποδών αυτού, είπεν, Ακούσατέ μου, αδελφοί μου και λαέ μου· Εγώ συνέλαβον εν τη καρδία μου να οικοδομήσω οίκον αναπαύσεως διά την κιβωτόν της διαθήκης του Κυρίου και διά το υποπόδιον των ποδών του Θεού ημών· και έκαμον ετοιμασίαν διά την οικοδομήν.
\par 3 Ο Θεός όμως είπε προς εμέ, Συ δεν θέλεις οικοδομήσει οίκον εις το όνομά μου, διότι είσαι ανήρ πολέμων και αίματα έχυσας.
\par 4 Εξέλεξε δε Κύριος ο Θεός του Ισραήλ εμέ εκ παντός του οίκου του πατρός μου, διά να ήμαι βασιλεύς επί τον Ισραήλ εις τον αιώνα· διότι εξέλεξε τον Ιούδαν άρχοντα· εκ δε του οίκου του Ιούδα εξέλεξε τον οίκον του πατρός μου· μεταξύ δε των υιών του πατρός μου εμέ ηυδόκησε να κάμη βασιλέα επί πάντα τον Ισραήλ·
\par 5 και εκ πάντων των υιών μου, διότι ο Κύριος πολλούς υιούς έδωκεν εις εμέ, εξέλεξε Σολομώντα τον υιόν μου, διά να καθίση επί τον θρόνον της βασιλείας του Κυρίου, επί τον Ισραήλ.
\par 6 Και είπε προς εμέ, Σολομών ο υιός σου, αυτός θέλει οικοδομήσει τον οίκόν μου και τας αυλάς μου· διότι αυτόν εξέλεξα υιόν εις εμέ, και εγώ θέλω είσθαι πατήρ εις αυτόν·
\par 7 και θέλω στερεώσει την βασιλείαν αυτού έως αιώνος, εάν μένη σταθερός εις το να εκτελή τας εντολάς μου και τας κρίσεις μου, καθώς εν τη ημέρα ταύτη.
\par 8 Τώρα λοιπόν, ενώπιον παντός του Ισραήλ της συναγωγής του Κυρίου και εις επήκοον του Θεού ημών, προς εσάς λέγω, Φυλάττετε και ζητείτε πάσας τας εντολάς Κυρίου του Θεού σας· διά να κυριεύητε την γην ταύτην την αγαθήν, και να αφήσητε αυτήν ύστερον από σας κληρονομίαν εις τους υιούς σας διά παντός.
\par 9 Και συ, Σολομών υιέ μου, γνώρισον τον Θεόν του πατρός σου και δούλευε αυτόν εν καρδία τελεία και εν ψυχή θελούση· διότι ο Κύριος εξετάζει πάσας τας καρδίας και εξεύρει πάντας τους λογισμούς των διανοιών· εάν εκζητής αυτόν, θέλει ευρίσκεσθαι υπό σού· εάν όμως εγκαταλίπης αυτόν, θέλει σε απορρίψει διά παντός.
\par 10 Ιδέ τώρα ότι ο Κύριος σε εξέλεξε, διά να οικοδομήσης οίκον εις αγιαστήριον· ενδυναμού και εκτέλει.
\par 11 Και έδωκεν ο Δαβίδ εις τον Σολομώντα τον υιόν αυτού το σχέδιον του προνάου και των οίκων αυτού, και των θησαυροφυλακίων αυτού, και των υπερώων αυτού και των έσω δωματίων αυτού και του οίκου του ιλαστηρίου,
\par 12 και το σχέδιον πάντων όσα συνέλαβεν εν τω πνεύματι αυτού, των αυλών του οίκου του Κυρίου και πάντων των πέριξ οικημάτων, των αποθηκών του οίκου του Θεού και των αποθηκών των αφιερωμάτων·
\par 13 και των διαιρέσεων των ιερέων και Λευϊτών και παντός του έργου της υπηρεσίας του οίκου του Κυρίου, και πάντων των σκευών της υπηρεσίας του οίκου του Κυρίου.
\par 14 Έδωκε χρυσόν κατά βάρος διά τα χρυσά, διά πάντα τα σκεύη παντός είδους υπηρεσίας· και άργυρον κατά βάρος διά πάντα τα σκεύη τα αργυρά, διά πάντα τα σκεύη παντός είδους υπηρεσίας·
\par 15 και το βάρος διά τας χρυσάς λυχνίας και διά τους χρυσούς λύχνους αυτών, κατά βάρος δι' εκάστην λυχνίαν και διά τους λύχνους αυτής· και διά τας αργυράς λυχνίας κατά βάρος, διά την λυχνίαν και διά τους λύχνους αυτής, κατά την χρήσιν εκάστης λυχνίας·
\par 16 και χρυσόν κατά βάρος διά τας τραπέζας των άρτων της προθέσεως, δι' εκάστην τράπεζαν· και άργυρον διά τας τραπέζας τας αργυράς·
\par 17 και χρυσόν καθαρόν διά τας κρεάγρας και διά τας λεκάνας και διά τας φιάλας· και διά τους χρυσούς κρατήρας, κατά βάρος δι' έκαστον κρατήρα· κατά βάρος ομοίως δι' έκαστον αργυρούν κρατήρα·
\par 18 και διά το θυσιαστήριον του θυμιάματος, κεκαθαρισμένον χρυσίον κατά βάρος· και χρυσίον διά το σχέδιον της αμάξης των χερουβείμ, τα οποία εξαπλόνουσι τας πτέρυγας και σκεπάζουσι την κιβωτόν της διαθήκης του Κυρίου.
\par 19 Τα πάντα, είπεν, ο Κύριος εφανέρωσε, γράψας διά της χειρός αυτού προς εμέ πάντα τα έργα του σχεδίου.
\par 20 Και είπεν ο Δαβίδ προς Σολομώντα τον υιόν αυτού, Ενδυναμού και ανδρίζου και εκτέλει μη φοβού μηδέ πτοηθής· διότι Κύριος ο Θεός, ο Θεός μου, θέλει είσθαι μετά σού· δεν θέλει σε αφήσει ουδέ σε εγκαταλείψει, εωσού τελειώσης άπαν το έργον της υπηρεσίας του οίκου του Κυρίου.
\par 21 Και ιδού, αι διαιρέσεις των ιερέων και Λευϊτών διά πάσαν υπηρεσίαν του οίκου του Θεού· και θέλουσιν είσθαι μετά σου, διά παν έργον, πας επιστήμων, πρόθυμος εις παν είδος υπηρεσίας, και οι άρχοντες και πας ο λαός, έτοιμοι εις πάντα τα προστάγματά σου.

\chapter{29}

\par 1 Τότε είπεν ο βασιλεύς Δαβίδ προς πάσαν την σύναξιν, Σολομών ο υιός μου, τον οποίον μόνον εξέλεξεν ο Θεός, είναι έτι νέος και απαλός· το δε έργον μέγα· διότι δεν είναι διά άνθρωπον η οικοδομή, αλλά διά Κύριον τον Θεόν.
\par 2 Εγώ λοιπόν ητοίμασα καθ' όλην την δύναμίν μου διά τον οίκον του Θεού μου, τον χρυσόν διά τα χρυσά και τον άργυρον διά τα αργυρά και τον χαλκόν διά τα χάλκινα, τον σίδηρον διά τα σιδηρά και ξύλα διά τα ξύλινα, ονυχίτας λίθους και λίθους ενθέσεως, λίθους λαμπρούς και ποικίλους και παντός είδους πολυτίμους λίθους και μάρμαρα άφθονα.
\par 3 Και έτι διά τον πόθον μου εις τον οίκον του Θεού μου, και εκ των ιδίων μου υπαρχόντων έδωκα περιπλέον χρυσίον και αργύριον διά τον οίκον του Θεού μου, εκτός παντός εκείνου το οποίον ητοίμασα διά τον οίκον τον άγιον·
\par 4 τρεις χιλιάδας τάλαντα χρυσίου, εκ του χρυσίου Οφείρ, και επτά χιλιάδας τάλαντα κεκαθαρισμένου αργυρίου, διά να περισκεπάσωσι τους τοίχους των οίκων·
\par 5 το χρυσίον διά τα χρυσά, και το αργύριον διά τα αργυρά, και διά πάσαν εργασίαν γινομένην διά των χειρών των τεχνιτών. Τις λοιπόν προθυμείται να κάμη σήμερον προσφοράν εις τον Κύριον;
\par 6 Τότε οι άρχοντες των πατριών και οι άρχοντες των φυλών του Ισραήλ και οι χιλίαρχοι και οι εκατόνταρχοι και οι επίσταται των έργων του βασιλέως, επροθυμήθησαν·
\par 7 και έδωκαν διά το έργον του οίκου του Θεού, χρυσίου πεντακισχίλια τάλαντα και δέκα χιλιάδας χρυσών, και αργυρίου δέκα χιλιάδας ταλάντων, και χαλκού δεκαοκτώ χιλιάδας ταλάντων, και εκατόν χιλιάδας ταλάντων σιδήρου.
\par 8 Και εις όσους ευρέθησαν λίθοι τίμιοι, έδωκαν αυτούς εις το θησαυροφυλάκιον του οίκου του Κυρίου διά χειρός Ιεχιήλ του Γηρσωνίτου.
\par 9 Εχάρη δε ο λαός, διότι επροθυμήθησαν, επειδή με πλήρη καρδίαν προσέφεραν αυτοπροαιρέτως εις τον Κύριον· και ο βασιλεύς Δαβίδ έτι εχάρη χαράν μεγάλην.
\par 10 Και ευλόγησεν ο Δαβίδ τον Κύριον ενώπιον πάσης της συνάξεως· και είπεν ο Δαβίδ, Ευλογητός συ, Κύριε ο Θεός του Ισραήλ, ο πατήρ ημών, από του αιώνος και έως του αιώνος.
\par 11 Σού, Κύριε, είναι η μεγαλωσύνη και η δύναμις και η τιμή και η νίκη και η δόξα· διότι σου είναι πάντα τα εν ουρανώ και τα επί γής· σου η βασιλεία, Κύριε, και συ είσαι ο υψούμενος ως κεφαλή υπεράνω πάντων·
\par 12 και ο πλούτος και η δόξα παρά σου έρχονται, και συ δεσπόζεις των απάντων· και εις την χείρα σου είναι η ισχύς και η δύναμις· και εις την χείρα σου το να μεγαλύνης και να ισχυροποιής τα πάντα.
\par 13 Τώρα λοιπόν, Θεέ ημών, ημείς ευχαριστούμέν σε και υμνούμεν το ένδοξον όνομά σου.
\par 14 Αλλά τις είμαι εγώ, και τις ο λαός μου, ώστε να δυνηθώμεν να προσφέρωμεν προθύμως εις σε κατά ταύτα; διότι τα πάντα έρχονται εκ σου και εκ των σων δίδομεν εις σε.
\par 15 Διότι είμεθα ξένοι ενώπιόν σου και πάροικοι, καθώς πάντες οι πατέρες ημών· αι ημέραι ημών επί της γης είναι ως σκιά, και μονιμότης δεν υπάρχει.
\par 16 Κύριε Θεέ ημών, άπαν τούτο το πλήθος, το οποίον ητοιμάσαμεν διά να οικοδομήσωμεν οίκον εις σε διά το όνομά σου το άγιον, εκ της χειρός σου έρχεται, και σου είναι τα πάντα.
\par 17 Και γνωρίζω, Θεέ μου, ότι συ είσαι ο δοκιμάζων την καρδίαν και αρέσκεσαι εις την ευθύτητα. Εγώ εν ευθύτητι της καρδίας μου προσέφερα πάντα ταύτα· και τώρα είδον μετ' ευφροσύνης τον λαόν σου, τον ενταύθα παρόντα, ότι αυτοπροαιρέτως προσφέρει εις σε.
\par 18 Κύριε Θεέ του Αβραάμ, του Ισαάκ και του Ισραήλ, των πατέρων ημών, φύλαττε τούτο διά παντός εις τους διαλογισμούς της καρδίας του λαού σου, και κατεύθυνε την καρδίαν αυτών προς σέ·
\par 19 και δος εις τον Σολομώντα τον υιόν μου καρδίαν τελείαν, διά να φυλάττη τας εντολάς σου, τα μαρτύριά σου και τα προστάγματά σου, και να εκτελή τα πάντα και να κατασκευάση την οικοδομήν, την οποίαν προητοίμασα.
\par 20 Και είπεν ο Δαβίδ προς πάσαν την σύναξιν, Ευλογήσατε τώρα Κύριον τον Θεόν σας. Και πάσα η σύναξις ευλόγησε Κύριον τον Θεόν των πατέρων αυτών και κύψαντες, προσεκύνησαν τον Κύριον και τον βασιλέα.
\par 21 Και την ακόλουθον ημέραν εθυσίασαν θυσίας εις τον Κύριον και προσέφεραν ολοκαυτώματα προς τον Κύριον, χιλίους μόσχους, χιλίους κριούς, χίλια αρνία, και τας σπονδάς αυτών και θυσίας αφθόνους διά πάντα τον Ισραήλ·
\par 22 και έφαγον και έπιον ενώπιον του Κυρίου την ημέραν εκείνην, εν χαρά μεγάλη. Και εκήρυξαν εκ δευτέρου Σολομώντα τον υιόν του Δαβίδ βασιλέα, και έχρισαν αυτόν εις τον Κύριον, διά να ήναι άρχων, και τον Σαδώκ διά ιερέα.
\par 23 Τότε ο Σολομών εκάθησεν επί του θρόνου του Κυρίου βασιλεύς αντί Δαβίδ του πατρός αυτού, και ευημέρησε· και πας ο Ισραήλ υπήκουσεν εις αυτόν.
\par 24 Και πάντες οι άρχοντες και οι δυνατοί και πάντες έτι οι υιοί του βασιλέως Δαβίδ υπετάχθησαν εις Σολομώντα τον βασιλέα.
\par 25 Και εμεγάλυνεν ο Κύριος εις άκρον τον Σολομώντα έμπροσθεν παντός του Ισραήλ, και έθεσεν επ' αυτόν μεγαλειότητα βασιλικήν, οποία δεν εστάθη εις ουδένα βασιλέα προ αυτού εν τω Ισραήλ.
\par 26 Ούτω Δαβίδ ο υιός του Ιεσσαί εβασίλευσεν επί πάντα τον Ισραήλ·
\par 27 και ο καιρός τον οποίον εβασίλευσεν επί τον Ισραήλ ήτο τεσσαράκοντα έτη· επτά έτη εβασίλευσεν εν Χεβρών και τριάκοντα τρία εβασίλευσεν εν Ιερουσαλήμ.
\par 28 Και ετελεύτησεν εις γήρας καλόν, πλήρης ημερών, πλούτου και δόξης· και εβασίλευσεν αντ' αυτού Σολομών ο υιός αυτού.
\par 29 Αι δε πράξεις του βασιλέως Δαβίδ, αι πρώται και αι τελευταίαι ιδού, είναι γεγραμμέναι εν τω βιβλίω Σαμουήλ του βλέποντος, και εν τω βιβλίω Νάθαν του προφήτου, και εν τω βιβλίω Γαδ του βλέποντος,
\par 30 μετά πάσης αυτού της βασιλείας και της δυνάμεως αυτού και των καιρών, οίτινες παρήλθον επ' αυτόν και επί τον Ισραήλ και επί πάσας τας βασιλείας της γης.


\end{document}