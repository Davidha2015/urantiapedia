\begin{document}

\title{Ezra}


\chapter{1}

\par 1 Και εν τω πρώτω έτει Κύρου του βασιλέως της Περσίας, διά να πληρωθή ο λόγος του Κυρίου ο διά στόματος του Ιερεμίου, διήγειρεν ο Κύριος το πνεύμα του Κύρου βασιλέως της Περσίας, και διεκήρυξε δι' όλου του βασιλείου αυτού, και μάλιστα εγγράφως, λέγων,
\par 2 Ούτω λέγει Κύρος η βασιλεύς της Περσίας· πάντα τα βασίλεια της γης έδωκεν εις εμέ Κύριος ο Θεός του ουρανού· και αυτός προσέταξεν εις εμέ να οικοδομήσω εις αυτόν οίκον εν Ιερουσαλήμ, ήτις είναι εν τη Ιουδαία·
\par 3 τις εξ υμών είναι εκ παντός του λαού αυτού; ο Θεός αυτού έστω μετ' αυτού, και ας αναβή εις Ιερουσαλήμ, ήτις είναι εν τη Ιουδαία, και ας οικοδομήση τον οίκον Κυρίου του Θεού του Ισραήλ· αυτός είναι ο Θεός ο εν Ιερουσαλήμ·
\par 4 πάντα δε απολειπόμενον, εκ πάντων των τόπων, όπου παροικεί, ας βοηθήσωσιν αυτόν οι άνδρες του τόπου αυτού με αργύριον και με χρυσίον και με αγαθά και με κτήνη, εκτός της προαιρετικής προσφοράς διά τον οίκον του Θεού, τον εν Ιερουσαλήμ.
\par 5 Τότε εσηκώθησαν οι αρχηγοί των πατριών του Ιούδα και του Βενιαμίν και οι ιερείς και οι Λευΐται, μετά πάντων όσων το πνεύμα διήγειρεν ο Θεός εις το να αναβώσι διά να οικοδομήσωσι τον οίκον του Κυρίου, τον εν Ιερουσαλήμ·
\par 6 και πάντες οι πέριξ αυτών εβοήθησαν αυτούς με σκεύη αργυρά, με χρυσίον, με αγαθά και με κτήνη και με πολύτιμα πράγματα, εκτός πασών των προαιρετικών προσφορών.
\par 7 Και εξήγαγεν ο βασιλεύς Κύρος τα σκεύη του οίκου του Κυρίου, τα οποία ο Ναβουχοδονόσορ είχε φέρει από Ιερουσαλήμ και θέσει αυτά εν τω οίκω του Θεού αυτού·
\par 8 και εξήγαγε ταύτα Κύρος ο βασιλεύς της Περσίας διά χειρός του Μιθρεδάθ του θησαυροφύλακος, και ηρίθμησεν αυτά εις τον Σασαβασσάρ τον άρχοντα της Ιουδαίας.
\par 9 Και ούτος είναι ο αριθμός αυτών· τριάκοντα δίσκοι χρυσοί, χίλιοι δίσκοι αργυροί, εικοσιεννέα μάχαιραι,
\par 10 τριάκοντα φιάλαι χρυσαί, τετρακόσιαι δέκα φιάλαι αργυραί δεύτεραι, άλλα σκεύη χίλια.
\par 11 Πάντα τα σκεύη τα χρυσά και αργυρά ήσαν πεντακισχίλια και τετρακόσια· τα πάντα ανεβίβασεν ο Σασαβασσάρ μετά των αιχμαλώτων των αναβιβασθέντων από Βαβυλώνος εις Ιερουσαλήμ.

\chapter{2}

\par 1 Ούτοι δε είναι οι άνθρωποι της επαρχίας οι αναβάντες εκ της αιχμαλωσίας, εκ των μετοικισθέντων, τους οποίους Ναβουχοδονόσορ ο βασιλεύς της Βαβυλώνος μετώκισεν εις Βαβυλώνα, και επιστρέψαντες εις Ιερουσαλήμ και εις την Ιουδαίαν, έκαστος εις την πόλιν αυτού·
\par 2 οίτινες ήλθον μετά Ζοροβάβελ, Ιησού, Νεεμία, Σεραΐα, Ρεελαΐα, Μαροδοχαίου, Βιλσάν, Μισπάρ, Βιγουαί, Ρεούμ, Βαανά. Αριθμός των ανδρών του λαού του Ισραήλ·
\par 3 Υιοί Φαρώς, δισχίλιοι εκατόν εβδομήκοντα δύο.
\par 4 Υιοί Σεφατία, τριακόσιοι εβδομήκοντα δύο.
\par 5 Υιοί Αράχ, επτακόσιοι εβδομήκοντα πέντε.
\par 6 Υιοί Φαάθ-μωάβ, εκ των υιών Ιησού και Ιωάβ, δισχίλιοι οκτακόσιοι δώδεκα.
\par 7 Υιοί Ελάμ, χίλιοι διακόσιοι πεντήκοντα τέσσαρες.
\par 8 Υιοί Ζατθού, εννεακόσιοι τεσσαράκοντα πέντε.
\par 9 Υιοί Ζακχαί, επτακόσιοι εξήκοντα.
\par 10 Υιοί Βανί, εξακόσιοι τεσσαράκοντα δύο.
\par 11 Υιοί Βηβαΐ, εξακόσιοι εικοσιτρείς.
\par 12 Υιοί Αζγάδ, χίλιοι διακόσιοι εικοσιδύο.
\par 13 Υιοί Αδωνικάμ, εξακόσιοι εξήκοντα εξ.
\par 14 Υιοί Βιγουαί, δισχίλιοι πεντήκοντα εξ.
\par 15 Υιοί Αδίν, τετρακόσιοι πεντήκοντα τέσσαρες.
\par 16 Υιοί Ατήρ εκ του Εζεκίου, ενενήκοντα οκτώ.
\par 17 Υιοί Βησαί, τριακόσιοι εικοσιτρείς.
\par 18 Υιοί Ιωρά, εκατόν δώδεκα.
\par 19 Υιοί Ασούμ, διακόσιοι εικοσιτρείς.
\par 20 Υιοί Γιββάρ, ενενήκοντα πέντε.
\par 21 Υιοί Βηθλεέμ, εκατόν εικοσιτρείς.
\par 22 Άνδρες Νετωφά, πεντήκοντα εξ.
\par 23 Άνδρες Αναθώθ, εκατόν εικοσιοκτώ.
\par 24 Υιοί Αζμαβέθ, τεσσαράκοντα δύο.
\par 25 Υιοί Κιριάθ-αρείμ, Χεφειρά και Βηρώθ, επτακόσιοι τεσσαράκοντα τρεις.
\par 26 Υιοί Ραμά και Γαβαά, εξακόσιοι είκοσι και εις.
\par 27 Άνδρες Μιχμάς, εκατόν εικοσιδύο.
\par 28 Άνδρες Βαιθήλ και Γαί, διακόσιοι εικοσιτρείς.
\par 29 Υιοί Νεβώ, πεντήκοντα δύο.
\par 30 Υιοί Μαγβίς, εκατόν πεντήκοντα εξ.
\par 31 Υιοί του άλλου Ελάμ, χίλιοι διακόσιοι πεντήκοντα τέσσαρες.
\par 32 Υιοί Χαρήμ, τριακόσιοι είκοσι.
\par 33 Υιοί Λωδ, Αδίδ, και Ωνώ, επτακόσιοι εικοσιπέντε.
\par 34 Υιοί Ιεριχώ, τριακόσιοι τεσσαράκοντα πέντε.
\par 35 Υιοί Σεναά, τρισχίλιοι και εξακόσιοι τριάκοντα.
\par 36 Οι ιερείς· υιοί Ιεδαΐα, εκ του οίκου Ιησού, εννεακόσιοι εβδομήκοντα τρεις.
\par 37 Υιοί Ιμμήρ, χίλιοι πεντήκοντα δύο.
\par 38 Υιοί Πασχώρ, χίλιοι διακόσιοι τεσσαράκοντα επτά.
\par 39 Υιοί Χαρήμ, χίλιοι δεκαεπτά.
\par 40 Οι Λευΐται· υιοί Ιησού, και Καδμιήλ, εκ των υιών Ωδουΐα, εβδομήκοντα τέσσαρες.
\par 41 Οι ψαλτωδοί· υιοί Ασάφ, εκατόν εικοσιοκτώ.
\par 42 Οι υιοί των πυλωρών· υιοί Σαλλούμ, υιοί Ατήρ, υιοί Ταλμών, υιοί Ακκούβ, υιοί Ατιτά, υιοί Σωβαΐ· πάντες εκατόν τριάκοντα εννέα.
\par 43 Οι Νεθινείμ· υιοί Σιχά, υιοί Ασουφά, υιοί Ταββαώθ,
\par 44 υιοί Κηρώς, υιοί Σιαά, υιοί Φαδών,
\par 45 υιοί Λεβανά, υιοί Αγαβά, υιοί Ακκούβ,
\par 46 υιοί Αγάβ, υιοί Σαλμαί, υιοί Ανάν,
\par 47 υιοί Γιδδήλ, υιοί Γαάρ, υιοί Ρεαΐα,
\par 48 υιοί Ρεσίν, υιοί Νεκωδά, υιοί Γαζάμ,
\par 49 υιοί Ουζά, υιοί Φασέα, υιοί Βησαί,
\par 50 υιοί Ασενά, υιοί Μεουνείμ, υιοί Νεφουσείμ,
\par 51 υιοί Βακβούκ, υιοί Ακουφά, υιοί Αρούρ,
\par 52 υιοί Βασλούθ, υιοί Μεϊδά, υιοί Αρσά,
\par 53 υιοί Βαρκώς, υιοί Σισάρα, υιοί Θαμά,
\par 54 υιοί Νεσιά, υιοί Ατιφά.
\par 55 Οι υιοί των δούλων του Σολομώντος· υιοί Σωταΐ, υιοί Σωφερέθ, υιοί Φερουδά,
\par 56 υιοί Ιααλά, υιοί Δαρκών, υιοί Γιδδήλ,
\par 57 υιοί Σεφατία, υιοί Αττίλ, υιοί Φοχερέθ από Σεβαΐμ, υιοί Αμί.
\par 58 Πάντες οι Νεθινείμ, και οι υιοί των δούλων του Σολομώντος, ήσαν τριακόσιοι ενενήκοντα δύο.
\par 59 Ούτοι δε ήσαν οι αναβάντες από Θελ-μελάχ, Θελ-αρησά, Χερούβ, Αδδάν και Ιμμήρ· δεν ηδύναντο όμως να δείξωσι τον οίκον της πατριάς αυτών και το σπέρμα αυτών, αν ήσαν εκ του Ισραήλ·
\par 60 Υιοί Δαλαΐα, υιοί Τωβία, υιοί Νεκωδά, εξακόσιοι πεντήκοντα δύο·
\par 61 και εκ των υιών των ιερέων· υιοί Αβαΐα, υιοί Ακκώς, υιοί Βαρζελλαΐ, όστις έλαβε γυναίκα εκ των θυγατέρων Βαρζελλαΐ του Γαλααδίτου και ωνομάσθη κατά το όνομα αυτών.
\par 62 Ούτοι εζήτησαν την καταγραφήν αυτών μεταξύ των απαριθμηθέντων κατά γενεαλογίαν, και δεν ευρέθησαν· όθεν εξεβλήθησαν από της ιερατείας.
\par 63 Και είπε προς αυτούς ο Θιρσαθά να μη φάγωσιν από των αγιωτάτων πραγμάτων, εωσού αναστηθή ιερεύς μετά Ουρίμ και Θουμμίμ.
\par 64 Πάσα η σύναξις ομού ήσαν τεσσαράκοντα δύο χιλιάδες τριακόσιοι εξήκοντα,
\par 65 εκτός των δούλων αυτών και των θεραπαινίδων αυτών, οίτινες ήσαν επτακισχίλιοι τριακόσιοι τριάκοντα επτά· και πλην τούτων, διακόσιοι ψαλτωδοί και ψάλτριαι.
\par 66 Οι ίπποι αυτών επτακόσιοι τριάκοντα έξ· αι ημίονοι αυτών, διακόσιαι τεσσαράκοντα πέντε·
\par 67 αι κάμηλοι αυτών, τετρακόσιαι τριάκοντα πέντε· αι όνοι, εξακισχίλιαι επτακόσιαι είκοσι.
\par 68 Και τινές εκ των αρχηγών των πατριών, ότε ήλθον εις τον οίκον του Κυρίου τον εν Ιερουσαλήμ, προσέφεραν αυτοπροαιρέτως διά τον οίκον του Θεού, να ανεγείρωσιν αυτόν εν τω τόπω αυτού·
\par 69 έδωκαν κατά την δύναμιν αυτών εις το θησαυροφυλάκιον του έργου εξ μυριάδας και χιλίας δραχμάς χρυσίου και πέντε χιλιάδας μνας αργυρίου και εκατόν ιερατικούς χιτώνας.
\par 70 Ούτως οι ιερείς και οι Λευΐται και μέρος εκ του λαού και οι ψαλτωδοί και οι πυλωροί και οι Νεθινείμ κατώκησαν εν ταις πόλεσιν αυτών, και πας ο Ισραήλ εν ταις πόλεσιν αυτού.

\chapter{3}

\par 1 Και ότε έφθασεν ο έβδομος μην και οι υιοί Ισραήλ ήσαν εν ταις πόλεσι, συνηθροίσθη ο λαός ως εις άνθρωπος εις Ιερουσαλήμ.
\par 2 Και εσηκώθη Ιησούς, ο υιός του Ιωσεδέκ, και οι αδελφοί αυτού οι ιερείς, και Ζοροβάβελ ο υιός του Σαλαθιήλ και οι αδελφοί αυτού, και ωκοδόμησαν το θυσιαστήριον του Θεού του Ισραήλ, διά να προσφέρωσιν ολοκαυτώματα επ' αυτού, κατά το γεγραμμένον εν τω νόμω Μωϋσέως του ανθρώπου του Θεού·
\par 3 και έστησαν το θυσιαστήριον εν τω τόπω αυτού, καίτοι επαπειλούμενοι υπό του λαού των τόπων εκείνων· και προσέφεραν επ' αυτού ολοκαυτώματα προς τον Κύριον, ολοκαυτώματα πρωΐ και εσπέρας.
\par 4 Και έκαμον την εορτήν των σκηνών, κατά το γεγραμμένον, και τας καθημερινάς ολοκαυτώσεις κατά αριθμόν, ως ήτο διατεταγμένον κατά το καθήκον εκάστης ημέρας.
\par 5 Και μετά ταύτα προσέφεραν τα παντοτεινά ολοκαυτώματα, και των νεομηνιών και πασών των καθηγιασμένων εορτών του Κυρίου και παντός προσφέροντος αυτοπροαίρετον προσφοράν εις τον Κύριον.
\par 6 Από της πρώτης ημέρας του εβδόμου μηνός ήρχισαν να προσφέρωσιν ολοκαυτώματα προς τον Κύριον· πλην τα θεμέλια του ναού του Κυρίου δεν είχον τεθή έτι.
\par 7 Και έδωκαν αργύριον εις τους λιθοτόμους και εις τους τέκτονας· και τροφάς και ποτά και έλαιον, εις τους Σιδωνίους και εις τους Τυρίους, διά να φέρωσι ξύλα κέδρινα από του Λιβάνου εις την θάλασσαν της Ιόππης, κατά την εις αυτούς δοθείσαν άδειαν Κύρου του βασιλέως της Περσίας.
\par 8 Και εν τω δευτέρω έτει της επιστροφής αυτών προς τον οίκον του Θεού εν Ιερουσαλήμ, εν μηνί τω δευτέρω, ήρχισαν Ζοροβάβελ ο υιός του Σαλαθιήλ και Ιησούς ο υιός του Ιωσεδέκ και οι λοιποί των αδελφών αυτών, ιερείς και Λευΐται, και πάντες οι ελθόντες από της αιχμαλωσίας εις Ιερουσαλήμ· και κατέστησαν τους Λευΐτας, από είκοσι ετών ηλικίας και επάνω, διά να επισπεύδωσι το έργον του οίκου του Κυρίου.
\par 9 Και παρεστάθη ο Ιησούς, οι υιοί αυτού και οι αδελφοί αυτού, ο Καδμιήλ και οι υιοί αυτού, υιοί Ιούδα, ως εις άνθρωπος, διά να κατεπείγωσι τους εργαζομένους εν τω οίκω του Θεού· οι υιοί του Ηναδάδ, οι υιοί αυτών και οι αδελφοί αυτών οι Λευΐται.
\par 10 Και ότε έθεσαν οι οικοδόμοι τα θεμέλια του ναού του Κυρίου, εστάθησαν οι ιερείς ενδεδυμένοι, μετά σαλπίγγων, και οι Λευΐται οι υιοί του Ασάφ μετά κυμβάλων, διά να υμνώσι τον Κύριον, κατά την διαταγήν Δαβίδ του βασιλέως του Ισραήλ·
\par 11 και έψαλλον αμοιβαίως υμνούντες και ευχαριστούντες τον Κύριον, Ότι αγαθός, ότι εις τον αιώνα το έλεος αυτού επί τον Ισραήλ. Και πας ο λαός ηλάλαξαν αλαλαγμόν μέγαν, υμνούντες τον Κύριον διά την θεμελίωσιν του οίκου του Κυρίου.
\par 12 Και πολλοί εκ των ιερέων και Λευϊτών και των αρχηγών των πατριών, γέροντες, οίτινες είχον ιδεί τον πρότερον οίκον, ενώ ο οίκος ούτος εθεμελιούτο ενώπιον των οφθαλμών αυτών, έκλαιον μετά φωνής μεγάλης· πολλοί δε ηλάλαξαν εν φωνή μεγάλη μετ' ευφροσύνης.
\par 13 Και δεν διέκρινεν ο λαός την φωνήν του αλαλαγμού της ευφροσύνης από της φωνής του κλαυθμού του λαού· διότι ο λαός ηλάλαζεν αλαλαγμόν μέγαν, και η βοή ηκούετο έως από μακρόθεν.

\chapter{4}

\par 1 Οι δε εχθροί του Ιούδα και Βενιαμίν, ακούσαντες ότι οι υιοί της αιχμαλωσίας οικοδομούσι τον ναόν εις Κύριον τον Θεόν του Ισραήλ,
\par 2 ήλθον προς τον Ζοροβάβελ και προς τους αρχηγούς των πατριών και είπον προς αυτούς, Ας οικοδομήσωμεν με σάς· διότι και ημείς εκζητούμεν τον Θεόν σας, καθώς σεις, και εις αυτόν θυσιάζομεν από των ημερών του Εσαραδδών βασιλέως της Ασσούρ, όστις ανεβίβασεν ημάς εδώ.
\par 3 Ο Ζοροβάβελ όμως και ο Ιησούς και οι λοιποί των αρχηγών των πατριών του Ισραήλ, είπον προς αυτούς, Ουδέν κοινόν εις εσάς και εις ημάς, ώστε να οικοδομήσητε οίκον εις τον Θεόν ημών· αλλ' ημείς αυτοί ηνωμένοι θέλομεν οικοδομήσει εις Κύριον τον Θεόν του Ισραήλ, καθώς προσέταξεν εις ημάς ο βασιλεύς Κύρος, ο βασιλεύς της Περσίας.
\par 4 Τότε ο λαός της γης παρέλυε τας χείρας του λαού του Ιούδα και ετάραττεν αυτούς εν τη οικοδομή,
\par 5 και εμίσθονον συμβούλους εναντίον αυτών, διά να ματαιόνωσι την βουλήν αυτών, πάσας τας ημέρας Κύρου του βασιλέως της Περσίας και έως της βασιλείας Δαρείου του βασιλέως της Περσίας.
\par 6 Και επί της βασιλείας Ασσουήρου, εν αρχή της βασιλείας αυτού, έγραψαν κατηγορίαν κατά των κατοίκων της Ιουδαίας και Ιερουσαλήμ.
\par 7 Και εν ταις ημέραις του Αρταξέρξου έγραψεν ο Βισλάμ, ο Μιθρεδάθ, ο Ταβεήλ και οι λοιποί συνέταιροι αυτών προς Αρταξέρξην τον βασιλέα της Περσίας· και η επιστολή ήτο γεγραμμένη Συριστί και εξηγημένη Συριστί.
\par 8 Ρεούμ ο έπαρχος και Σαμψαί ο γραμματεύς, έγραψαν επιστολήν κατά της Ιερουσαλήμ προς Αρταξέρξην τον βασιλέα, κατά τούτον τον τρόπον·
\par 9 Ρεούμ ο έπαρχος και Σαμψαί ο γραμματεύς και οι λοιποί συνέταιροι αυτών, οι Δειναίοι, οι Αφαρσαχαίοι, οι Ταρφαλαίοι, οι Αφαρσαίοι, οι Αρχεναίοι, οι Βαβυλώνιοι, οι Σουσαναχαίοι, οι Δεαυαίοι, οι Ελαμίται
\par 10 και οι λοιποί εκ των εθνών, τα οποία ο μέγας και ένδοξος Ασεναφάρ μετεκόμισε και κατώκισεν εις τας πόλεις της Σαμαρείας, και οι λοιποί οι πέραν του ποταμού, και τα λοιπά.
\par 11 Τούτο είναι το αντίγραφον της επιστολής, την οποίαν έστειλαν προς αυτόν, προς Αρταξέρξην τον βασιλέα· οι δούλοί σου, οι άνδρες οι πέραν του ποταμού, και τα λοιπά.
\par 12 Γνωστόν έστω εις τον βασιλέα, ότι οι Ιουδαίοι οι αναβάντες από σου προς ημάς, ελθόντες εις Ιερουσαλήμ, οικοδομούσι την πόλιν, την αποστάτιδα και πονηράν, και εγείρουσι τον τοίχον και συνάπτουσι τα θεμέλια.
\par 13 Γνωστόν έστω ήδη εις τον βασιλέα, ότι εάν η πόλις αύτη οικοδομηθή και οι τοίχοι εγερθώσι, δεν θέλουσι πληρώσει φόρον, τελώνιον ή διαγώγιον· και θέλει ζημιωθή το εισόδημα των βασιλέων.
\par 14 Επειδή δε τρεφόμεθα από του παλατίου, και ήτο απρεπές διά ημάς να βλέπωμεν την ατιμίαν του βασιλέως, διά τούτο εστείλαμεν και εγνωστοποιήσαμεν προς τον βασιλέα,
\par 15 διά να γείνη έρευνα εν τω βιβλίω των υπομνημάτων των πατέρων σου· και θέλεις ευρεί εν τω βιβλίω των υπομνημάτων και γνωρίσει, ότι η πόλις αύτη είναι πόλις αποστάτις και ολέθριος εις τους βασιλείς και εις τας επαρχίας, και ότι εκ παλαιού χρόνου εκίνουν επανάστασιν εν τω μέσω αυτής, διά την οποίαν αιτίαν η πόλις αύτη κατηρημώθη.
\par 16 Γνωστοποιούμεν προς τον βασιλέα, έτι εάν η πόλις αύτη ανοικοδομηθή και οι τοίχοι αυτής ανεγερθώσι, δεν θέλεις έχει ουδέν μέρος εις το πέραν του ποταμού.
\par 17 Ο βασιλεύς απεκρίθη προς τον Ρεούμ τον έπαρχον και Σαμψαί τον γραμματέα και τους λοιπούς συνεταίρους αυτών τους κατοικούντας εν Σαμαρεία, και τους άλλους τους πέραν του ποταμού, Ειρήνη, και τα λοιπά.
\par 18 Η επιστολή, την οποίαν εστείλατε προς ημάς, ανεγνώσθη ακριβώς ενώπιόν μου.
\par 19 Και εξεδόθη διαταγή παρ' εμού, και ηρεύνησαν και εύρηκαν ότι η πόλις αύτη εκ παλαιού χρόνου επαναστατεί εναντίον των βασιλέων, και γίνονται εν αυτή στάσεις και συνωμοσίαι·
\par 20 Υπήρξαν έτι ισχυροί βασιλείς επί Ιερουσαλήμ, δεσπόζοντες επί πάντας τους πέραν του ποταμού· και επληρόνετο εις αυτούς φόρος, τελώνιον και διαγώγιον.
\par 21 Τώρα λοιπόν προστάξατε να παύσωσι τους ανθρώπους εκείνους, και η πόλις αύτη να μη οικοδομηθή, εωσού εκδοθή διαταγή παρ' εμού.
\par 22 Και προσέξατε να μη αμελήσητε να κάμητε τούτο· διά να μη αυξηθή το κακόν επί ζημία των βασιλέων.
\par 23 Ότε δε το αντίγραφον της επιστολής του βασιλέως Αρταξέρξου ανεγνώσθη ενώπιον του Ρεούμ και Σαμψαί του γραμματέως και των συνεταίρων αυτών, ανέβησαν μετά σπουδής εις Ιερουσαλήμ προς τους Ιουδαίους, και έπαυσαν αυτούς εν βία και μετά δυνάμεως.
\par 24 Και έπαυσε το έργον του οίκου του Θεού του εν Ιερουσαλήμ, και έμεινε πεπαυμένον μέχρι του δευτέρου έτους της βασιλείας Δαρείου του βασιλέως της Περσίας.

\chapter{5}

\par 1 Τότε προεφήτευσαν ο προφήτης Αγγαίος και Ζαχαρίας ο υιός του Ιδδώ, προς τους Ιουδαίους τους εν Ιουδαία και Ιερουσαλήμ, προφητεύοντες προς αυτούς εν ονόματι του Θεού του Ισραήλ.
\par 2 Και εσηκώθησαν Ζοροβάβελ ο υιός του Σαλαθιήλ και Ιησούς ο υιός του Ιωσεδέκ και ήρχισαν να οικοδομώσι τον οίκον του Θεού τον εν Ιερουσαλήμ· και μετ' αυτών οι προφήται του Θεού βοηθούντες αυτούς.
\par 3 Εν τούτω τω καιρώ ελθόντες προς αυτούς Ταθναΐ, ο έπαρχος των εντεύθεν του ποταμού, και ο Σεθάρ-βοσναΐ και οι συνέταιροι αυτών, είπον προς αυτούς ούτω· Τις προσέταξεν εις εσάς να οικοδομήτε τον οίκον τούτον και να εγείρητε τούτον τον τοίχον;
\par 4 Και τότε είπομεν προς αυτούς ποία είναι τα ονόματα των ανδρών, οίτινες οικοδομούσι την οικοδομήν ταύτην.
\par 5 Αλλ' επί τους πρεσβυτέρους των Ιουδαίων ήτο ο οφθαλμός του Θεού αυτών, και δεν ηδύναντο να παύσωσιν αυτούς, εωσού έλθη η υπόθεσις προς τον Δαρείον· και τότε έδωκαν απόκρισιν δι' επιστολής περί τούτου.
\par 6 Αντίγραφον της επιστολής, την οποίαν Ταθναΐ, ο έπαρχος των εντεύθεν του ποταμού, και ο Σεθάρ-βοσναΐ και οι συνέταιροι αυτού οι Αφαρσαχαίοι οι εντεύθεν του ποταμού, απέστειλαν προς Δαρείον τον βασιλέα.
\par 7 Απέστειλαν επιστολήν προς αυτόν, εν ή ήτο γεγραμμένον ούτως· Εις τον Δαρείον τον βασιλέα, πάσα ειρήνη.
\par 8 Γνωστόν έστω εις τον βασιλέα, ότι υπήγαμεν εις την επαρχίαν της Ιουδαίας προς τον οίκον του μεγάλου Θεού, και αυτός οικοδομείται με λίθους μεγάλους και εντίθενται ξύλα εις τους τοίχους, και το έργον τούτο προχωρεί ταχέως και ευοδούται εις τας χείρας αυτών.
\par 9 Και ερωτήσαντες εκείνους τους πρεσβυτέρους, ελαλήσαμεν προς αυτούς ούτω· Τις προσέταξεν εις εσάς να οικοδομήτε τον οίκον τούτον και να εγείρητε τούτον τον τοίχον;
\par 10 Έτι και τα ονόματα αυτών ηρωτήσαμεν, διά να σοι φανερώσωμεν και γράψωμεν προς σε τα ονόματα των ανδρών των επί κεφαλής αυτών.
\par 11 Και απεκρίθησαν προς ημάς ούτω, λέγοντες, Ημείς είμεθα οι δούλοι του Θεού του ουρανού και της γης, και οικοδομούμεν τον οίκον τον προ πολλών ήδη ετών οικοδομηθέντα, τον οποίον βασιλεύς μέγας του Ισραήλ ωκοδόμησε και ανήγειρεν·
\par 12 αφού όμως οι πατέρες ημών παρώργισαν τον Θεόν του ουρανού, παρέδωκεν αυτούς εις την χείρα του Ναβουχοδονόσορ, βασιλέως της Βαβυλώνος, του Χαλδαίου, και κατέστρεψε τον οίκον τούτον και μετώκισε τον λαόν εις την Βαβυλώνα.
\par 13 Πλην εν τω πρώτω έτει Κύρου του βασιλέως της Βαβυλώνος, ο βασιλεύς Κύρος έδωκε προσταγήν να οικοδομηθή ούτος ο οίκος του Θεού.
\par 14 Και τα σκεύη έτι τα χρυσά και αργυρά του οίκου του Θεού, τα οποία ο Ναβουχοδονόσορ έλαβεν εκ του ναού του εν Ιερουσαλήμ και έφερεν αυτά εις τον ναόν της Βαβυλώνος, ταύτα ο Κύρος ο βασιλεύς εσήκωσεν εκ του ναού της Βαβυλώνος, και παρεδόθησαν εις τον ονομαζόμενον Σασαβασσάρ, τον οποίον είχε κάμει έπαρχον·
\par 15 και είπε προς αυτόν, Λάβε τα σκεύη ταύτα, ύπαγε, φέρε αυτά εις τον ναόν τον εν Ιερουσαλήμ, και ο οίκος του Θεού ας οικοδομηθή εν τω τόπω αυτού.
\par 16 Τότε ελθών ούτος ο Σασαβασσάρ έθεσε τα θεμέλια του οίκου του Θεού, του εν Ιερουσαλήμ· και απ' εκείνου του χρόνου έως της σήμερον οικοδομείται και δεν ετελείωσε.
\par 17 Τώρα λοιπόν, εάν φαίνηται αρεστόν εις τον βασιλέα, ας γείνη έρευνα εν τω θησαυροφυλακίω του βασιλέως τω εν Βαβυλώνι, εάν ήναι αληθινόν ότι εξεδόθη διαταγή παρά Κύρου του βασιλέως να οικοδομηθή ο οίκος ούτος του Θεού εν Ιερουσαλήμ· και ας αποστείλη ο βασιλεύς προς ημάς την θέλησιν αυτού περί τούτου.

\chapter{6}

\par 1 Τότε Δαρείος ο βασιλεύς εξέδωκε διαταγήν, και ηρεύνησαν εν τοις αρχείοις, όπου κείνται οι θησαυροί εν Βαβυλώνι.
\par 2 Και ευρέθη εν Αχμεθά, εν τω παλατίω τω εν τη επαρχία των Μήδων, εις τόμος, και ήτο εν αυτώ υπόμνημα γεγραμμένον ούτως·
\par 3 Εν τω πρώτω έτει Κύρου του βασιλέως, Κύρος ο βασιλεύς εξέδωκε διαταγήν περί του οίκου του Θεού του εν Ιερουσαλήμ, Ας οικοδομηθή ο οίκος, ο τόπος εις τον οποίον προσφέρονται αι θυσίαι, και ας τεθώσι τα θεμέλια αυτού δυνατά· το ύψος αυτού εξήκοντα πήχαι, το πλάτος αυτού εξήκοντα πήχαι·
\par 4 τρεις σειραί μεγάλων λίθων, και μία σειρά ξύλων νέων· και τα αναλώματα ας δοθώσιν εκ του οίκου του βασιλέως·
\par 5 τα χρυσά έτι και τα αργυρά σκεύη του οίκου του Θεού, τα οποία ο Ναβουχοδονόσορ έλαβεν εκ του ναού του εν Ιερουσαλήμ και έφερεν εις Βαβυλώνα, ας αποδοθώσι και ας επανέλθωσιν εις τον ναόν τον εν Ιερουσαλήμ, έκαστον εις τον τόπον αυτού, και ας τεθώσιν εις τον οίκον του Θεού.
\par 6 Τώρα λοιπόν, Ταθναΐ, έπαρχε των πέραν του ποταμού, Σεθάρ-βοσναΐ, και οι συνέταιροί σας οι Αφαρσαχαίοι, οι πέραν του ποταμού, απομακρύνθητε εκείθεν·
\par 7 αφήσατε το έργον τούτου του οίκου του Θεού· ο έπαρχος των Ιουδαίων και οι πρεσβύτεροι των Ιουδαίων ας οικοδομήσωσι τον οίκον τούτον του Θεού εν τω τόπω αυτού.
\par 8 Εξεδόθη έτι απ' εμού διαταγή, τι θέλετε κάμει εις τους πρεσβυτέρους των Ιουδαίων τούτων, διά την οικοδομήν τούτου του οίκου του Θεού· εκ των υπαρχόντων του βασιλέως, εκ του φόρου των πέραν του ποταμού, θέλουσι δοθή αμέσως αναλώματα εις τους ανθρώπους τούτους, διά να μη εμποδισθώσι.
\par 9 Και ούτινος πράγματος έχουσι χρείαν, και μόσχοι και κριοί και πρόβατα, διά τας ολοκαυτώσεις του Θεού του ουρανού, σίτος, άλας, οίνος και έλαιον, κατά την αίτησιν των ιερέων των εν Ιερουσαλήμ, ας δίδωνται εις αυτούς καθ' ημέραν άνευ ελλείψεως,
\par 10 διά να προσφέρωσι θυσίας εις οσμήν ευωδίας προς τον Θεόν του ουρανού, και να προσεύχωνται υπέρ της ζωής του βασιλέως και των υιών αυτού.
\par 11 Εξεδόθη έτι παρ' εμού διαταγή περί παντός ανθρώπου, όστις παραλλάξη τον λόγον τούτον, να αποσπασθή ξύλον εκ της οικίας αυτού και να στηθή και να κρεμασθή επ' αυτό· η δε οικία αυτού ας γείνη διά τούτο κοπρών.
\par 12 Και ο Θεός, όστις κατώκισε το όνομα αυτού εκεί, ας εξολοθρεύση πάντα βασιλέα και λαόν, όστις εκτείνη την χείρα αυτού διά να παραλλάξη τι, ώστε να καταστρέψη τούτον τον οίκον του Θεού τον εν Ιερουσαλήμ. Εγώ ο Δαρείος εξέδωκα την διαταγήν· ας εκτελεσθή ταχέως.
\par 13 Τότε ο Ταθναΐ, ο έπαρχος των εντεύθεν του ποταμού, ο Σεθάρ-βοσναΐ, και οι συνέταιροι αυτών, κατά τα προσταχθέντα υπό του Δαρείου του βασιλέως, ούτως έκαμον ταχέως.
\par 14 Και οι πρεσβύτεροι των Ιουδαίων ωκοδόμουν και ευωδούντο, κατά την προφητείαν Αγγαίου του προφήτου και Ζαχαρίου υιού του Ιδδώ. Και ωκοδόμησαν και ετελείωσαν, κατά την προσταγήν του Θεού του Ισραήλ, και κατά την προσταγήν του Κύρου και Δαρείου και Αρταξέρξου βασιλέως της Περσίας.
\par 15 Και συνετελέσθη ο οίκος ούτος την τρίτην ημέραν του μηνός Αδάρ, εν τω έκτω έτει της βασιλείας Δαρείου του βασιλέως.
\par 16 Και εγκαινίασαν εν ευφροσύνη οι υιοί του Ισραήλ, οι ιερείς και οι Λευΐται, και οι λοιποί εκ των υιών της αιχμαλωσίας, τον οίκον τούτον του Θεού·
\par 17 και προσέφεραν εις τον εγκαινιασμόν του οίκου τούτου του Θεού εκατόν μόσχους, διακοσίους κριούς, τετρακόσια αρνία· και διά προσφοράν περί αμαρτίας υπέρ παντός του Ισραήλ δώδεκα τράγους, κατά τον αριθμόν των φυλών του Ισραήλ.
\par 18 Και έστησαν τους ιερείς εις τας διαιρέσεις αυτών, και τους Λευΐτας εις τα υπουργήματα αυτών, διά την υπηρεσίαν του Θεού την εν Ιερουσαλήμ, κατά το γεγραμμένον εν τω βιβλίω του Μωϋσέως.
\par 19 Και έκαμον το πάσχα οι υιοί της αιχμαλωσίας τη δεκάτη τετάρτη του πρώτου μηνός·
\par 20 διότι οι ιερείς και οι Λευΐται εκαθαρίσθησαν ομού· πάντες ήσαν κεκαθαρισμένοι, και έσφαξαν το πάσχα εις πάντας τους υιούς της αιχμαλωσίας, και εις τους αδελφούς αυτών τους ιερείς, και εις εαυτούς.
\par 21 Και έφαγον οι υιοί Ισραήλ, οι επιστρέψαντες από της αιχμαλωσίας, και πάντες οι χωρισθέντες προς αυτούς από της ακαθαρσίας των εθνών της γης, διά να εκζητήσωσι Κύριον τον Θεόν του Ισραήλ.
\par 22 Και έκαμον την εορτήν των αζύμων επτά ημέρας μετ' ευφροσύνης· διότι εύφρανεν αυτούς ο Κύριος, και έστρεψε προς αυτούς την καρδίαν του βασιλέως της Ασσυρίας, διά να ενισχύση τας χείρας αυτών εις το έργον του οίκου του Θεού, του Θεού του Ισραήλ.

\chapter{7}

\par 1 Μετά δε τα πράγματα ταύτα, επί της βασιλείας Αρταξέρξου βασιλέως της Περσίας, Έσδρας ο υιός του Σεραΐου, υιού του Αζαρία, υιού του Χελκία,
\par 2 υιού του Σαλλούμ, υιού του Σαδώκ, υιού του Αχιτώβ,
\par 3 υιού του Αμαρία, υιού του Αζαρία, υιού του Μεραϊώθ,
\par 4 υιού του Ζεραΐα, υιού του Οζί, υιού του Βουκκί,
\par 5 υιού του Αβισσουά, υιού του Φινεές, υιού του Ελεάζαρ, υιού του Ααρών του ιερέως του πρώτου,
\par 6 ούτος ο Έσδρας ανέβη από της Βαβυλώνος, ων γραμματεύς έμπειρος εις τον νόμον του Μωϋσέως, τον οποίον έδωκε Κύριος ο Θεός του Ισραήλ· και ο βασιλεύς εχάρισεν εις αυτόν πάντα τα αιτήματα αυτού, κατά την επ' αυτόν χείρα Κυρίου του Θεού αυτού.
\par 7 Ανέβησαν και τινές εκ των υιών Ισραήλ και εκ των ιερέων, και οι Λευΐται, και οι ψαλτωδοί και οι πυλωροί και οι Νεθινείμ, εις Ιερουσαλήμ, εν τω εβδόμω έτει Αρταξέρξου του βασιλέως.
\par 8 Και ήλθον εις Ιερουσαλήμ τον πέμπτον μήνα του εβδόμου έτους του βασιλέως.
\par 9 Διότι την πρώτην του πρώτον μηνός ήρχισεν ούτος να αναβαίνη από της Βαβυλώνος, και την πρώτην του πέμπτου μηνός ήλθεν εις Ιερουσαλήμ, κατά την επ' αυτόν αγαθήν χείρα του Θεού αυτού.
\par 10 Επειδή ο Έσδρας είχεν ετοιμάσει την καρδίαν αυτού εις το να εκζητή τον νόμον του Κυρίου, και να εκτελή και να διδάσκη εις τον Ισραήλ διατάγματα και κρίσεις.
\par 11 Τούτο δε είναι το αντίγραφον της επιστολής, την οποίαν ο βασιλεύς Αρταξέρξης έδωκεν εις τον Έσδραν τον ιερέα, τον γραμματέα, γραμματέα των λόγων των εντολών του Κυρίου και των διαταγμάτων αυτού προς τον Ισραήλ·
\par 12 Αρταξέρξης, βασιλεύς των βασιλέων, προς Έσδραν τον ιερέα, τον γραμματέα του νόμου του Θεού του ουρανού, τον τέλειον, και τα λοιπά.
\par 13 Εξεδόθη παρ' εμού διαταγή, ώστε πάντες οι εκ του λαού του Ισραήλ και των ιερέων αυτού και των Λευϊτών, οι εν τω βασιλείω μου, όσοι θέλουσιν αυτοπροαιρέτως να αναβώσιν εις Ιερουσαλήμ, να έλθωσι μετά σου.
\par 14 Διότι πέμπεσαι παρά του βασιλέως και των επτά συμβούλων αυτού, διά να επισκεφθής την Ιουδαίαν και Ιερουσαλήμ, κατά τον εν τη χειρί σου νόμον του Θεού σου·
\par 15 και να φέρης το αργύριον και το χρυσίον, το οποίον ο βασιλεύς και οι σύμβουλοι αυτού προσέφεραν αυτοπροαιρέτως εις τον Θεόν του Ισραήλ, του οποίου το κατοικητήριον είναι εν Ιερουσαλήμ,
\par 16 και άπαν το αργύριον και χρυσίον όσον συνάξης καθ' όλην την επαρχίαν της Βαβυλώνος, μετά των προαιρετικών προσφορών του λαού και των ιερέων, των προσφερόντων αυτοπροαιρέτως διά τον εν Ιερουσαλήμ οίκον του Θεού αυτών·
\par 17 διά ν' αγοράσης ταχέως διά του αργυρίου τούτου μόσχους, κριούς, αρνία, τας εξ αλφίτων προσφοράς αυτών και τας σπονδάς αυτών, και να προσφέρης αυτά επί το θυσιαστήριον του οίκου του Θεού σας, το εν Ιερουσαλήμ.
\par 18 Και παν ό,τι φανή αρεστόν εις σε και εις τους αδελφούς σου να κάμητε διά του υπολοίπου αργυρίου και χρυσίου, τούτο κάμετε, κατά το θέλημα του Θεού σας.
\par 19 Και τα σκεύη, τα δοθέντα εις σε διά την υπηρεσίαν του οίκου του Θεού σου, παράδος ενώπιον του Θεού της Ιερουσαλήμ.
\par 20 Και ό,τι περιπλέον χρειασθή διά τον οίκον του Θεού σου, ό,τι συμβή να εξοδεύσης, εξόδευε εκ του βασιλικού θησαυροφυλακίου.
\par 21 Και παρ' εμού, εμού του Αρταξέρξου βασιλέως, εξεδόθη διαταγή εις πάντας τους θησαυροφύλακας τους πέραν του ποταμού, παν ό,τι ζητήση παρ' υμών ο Έσδρας ο ιερεύς, ο γραμματεύς του νόμου του Θεού του ουρανού, να γίνηται πάραυτα,
\par 22 έως εκατόν ταλάντων αργυρίου, και έως εκατόν κόρων σίτου, και έως εκατόν βαθ οίνου, και έως εκατόν βαθ ελαίου, και άλας απροσδιόριστον,
\par 23 Παν ό,τι είναι προστεταγμένον παρά του Θεού του ουρανού, ας γείνη μετά σπουδής διά τον οίκον του Θεού του ουρανού· διά να μη έλθη οργή επί την βασιλείαν του βασιλέως και των υιών αυτού.
\par 24 Έτι γνωστοποιείται εις εσάς, ότι εις ουδένα εκ των ιερέων και Λευϊτών, ψαλτωδών, θυρωρών, Νεθινείμ και υπηρετών τούτου του οίκου του Θεού, δεν θέλει είσθαι νόμιμον να επιβληθή φόρος, τελώνιον ή διαγώγιον επ' αυτούς.
\par 25 Και συ, Έσδρα, κατά την εν σοι του Θεού σου σοφίαν, κατάστησον κριτάς και δικαστάς, διά να κρίνωσι πάντα τον λαόν τον πέραν του ποταμού, πάντας τους ειδότας τους νόμους του Θεού σου· και διδάσκετε τους μη ειδότας.
\par 26 Και πας όστις δεν κάμνει τον νόμον του Θεού σου και τον νόμον του βασιλέως, ας εκτελήται ταχέως κρίσις επ' αυτόν, είτε εις θάνατον, είτε εις εξορίαν, ή εις δήμευσιν υπαρχόντων, ή εις φυλακήν.
\par 27 Ευλογητός Κύριος ο Θεός των πατέρων ημών, όστις έδωκε τοιαύτα εις την καρδίαν του βασιλέως, διά να δοξάση τον οίκον του Κυρίου, τον εν Ιερουσαλήμ·
\par 28 και έκαμε να εύρω έλεος ενώπιον του βασιλέως και των συμβούλων αυτού και πάντων των αρχόντων του βασιλέως των δυνατών. Και εγώ ενισχύθην κατά την επ' εμέ χείρα Κυρίου του Θεού μου, και συνήγαγον εκ του Ισραήλ άρχοντας διά να συναναβώσι μετ' εμού.

\chapter{8}

\par 1 Ούτοι δε είναι οι αρχηγοί των πατριών αυτών, και η γενεαλογία των συναναβάντων μετ' εμού από της Βαβυλώνος, επί της βασιλείας Αρταξέρξου του βασιλέως.
\par 2 Εκ των υιών Φινεές, Γηρσώμ· εκ των υιών Ιθάμαρ, Δανιήλ· εκ των υιών Δαβίδ, Χαττούς.
\par 3 Εκ των υιών Σεχανία, του εκ των υιών Φαρώς, Ζαχαρίας· και μετ' αυτού ηριθμήθησαν κατά γενεαλογίαν τα αρσενικά εκατόν πεντήκοντα.
\par 4 Εκ των υιών του Φαάθ-μωάβ, Ελιωηνάϊ ο υιός του Ζεραΐα, και μετ' αυτού τα αρσενικά διακόσιοι.
\par 5 Εκ των υιών Σεχανία, ο υιός του Ιααζιήλ, και μετ' αυτού τα αρσενικά τριακόσιοι.
\par 6 Και εκ των υιών Αδίν, Εβέδ ο υιός του Ιωνάθαν, και μετ' αυτού τα αρσενικά πεντήκοντα.
\par 7 Και εκ των υιών Ελάμ, Ιεσαΐας ο υιός του Γοθολία, και μετ' αυτού εβδομήκοντα.
\par 8 Και εκ των υιών Σεφατία, Ζεβαδίας ο υιός του Μιχαήλ, και μετ' αυτού τα αρσενικά ογδοήκοντα.
\par 9 Εκ των υιών Ιωάβ, Οβαδία ο υιός του Ιεχιήλ, και μετ' αυτού τα αρσενικά διακόσιοι δεκαοκτώ.
\par 10 Και εκ των υιών του Σελωμείθ, ο υιός του Ιωσιφία, και μετ' αυτού τα αρσενικά εκατόν εξήκοντα.
\par 11 Και εκ των υιών Βηβαΐ, Ζαχαρίας ο υιός του Βηβαΐ, και μετ' αυτού τα αρσενικά εικοσιοκτώ.
\par 12 Και εκ των υιών Αζγάδ, Ιωανάν ο υιός του Ακκατάν, και μετ' αυτού τα αρσενικά εκατόν δέκα.
\par 13 Και εκ των υιών Αδωνικάμ οι τελευταίοι, και ταύτα τα ονόματα αυτών, Ελιφελέτ, Ιεϊήλ και Σεμαΐας, και μετ' αυτών τα αρσενικά εξήκοντα.
\par 14 Εκ δε των υιών Βιγουαί, Γουθαΐ και Ζαββούδ, και μετ' αυτών τα αρσενικά εβδομήκοντα.
\par 15 Και συνήθροισα αυτούς παρά τον ποταμόν, τον ρέοντα προς Ααβά, και εκεί κατεσκηνώσαμεν τρεις ημέρας· και παρετήρησα μεταξύ του λαού και των ιερέων και δεν εύρηκα εκεί ουδένα εκ των υιών του Λευΐ.
\par 16 Τότε απέστειλα προς τον Ελιέζερ, τον Αριήλ, τον Σεμαΐαν και τον Ελνάθαν και τον Ιαρείβ και τον Ελνάθαν και τον Νάθαν και τον Ζαχαρίαν και τον Μεσουλλάμ, τους άρχοντας· και τον Ιωϊαρίβ, και τον Ελνάθαν, συνετούς.
\par 17 Και έδωκα εις αυτούς παραγγελίαν προς τον Ιδδώ τον άρχοντα, εν τω τόπω Κασιφία, και έβαλον εις το στόμα αυτών λόγους διά να λαλήσωσι προς τον Ιδδώ και τους αδελφούς αυτού τους Νεθινείμ, εν τω τόπω Κασιφία, διά να πέμψωσι προς ημάς λειτουργούς διά τον οίκον του Θεού ημών.
\par 18 Και κατά την εφ' ημάς αγαθήν χείρα του Θεού ημών έφεραν προς ημάς άνδρα συνετόν, εκ των υιών Μααλί, υιού του Λευΐ, υιού Ισραήλ· και τον Σερεβίαν μετά των υιών αυτού και των αδελφών αυτού, δεκαοκτώ·
\par 19 και τον Ασαβίαν, και μετ' αυτού τον Ιεσαΐαν εκ των υιών Μεραρί, τους αδελφούς αυτού και τους υιούς αυτών, είκοσι·
\par 20 και εκ των Νεθινείμ, τους οποίους ο Δαβίδ και οι άρχοντες διώρισαν διά την υπηρεσίαν των Λευϊτών, διακοσίους είκοσι Νεθινείμ· πάντες ούτοι ήσαν σεσημειωμένοι κατ' όνομα.
\par 21 Τότε εκήρυξα εκεί νηστείαν παρά τον ποταμόν Ααβά, όπως ταπεινωθέντες ενώπιον του Θεού ημών, ζητήσωμεν παρ' αυτού ευθείαν οδόν διά ημάς και διά τα τέκνα ημών και διά πάντα τα υπάρχοντα ημών.
\par 22 Διότι ησχύνθην να ζητήσω παρά του βασιλέως δύναμιν και ιππείς διά να βοηθήσωσιν ημάς εναντίον του εχθρού καθ' οδόν· επειδή είχομεν ειπεί προς τον βασιλέα, λέγοντες, Η χειρ του Θεού ημών είναι προς αγαθόν επί πάντας τους ζητούντας αυτόν· το δε κράτος αυτού και η οργή αυτού επί πάντας τους εγκαταλείποντας αυτόν.
\par 23 Ενηστεύσαμεν λοιπόν και ικετεύσαμεν τον Θεόν ημών περί τούτου· και έγεινεν ίλεως προς ημάς.
\par 24 Τότε εχώρισα δώδεκα εκ των αρχόντων των ιερέων, τον Σερεβίαν, τον Ασαβίαν και μετ' αυτών δέκα εκ των αδελφών αυτών.
\par 25 Και εζύγισα εις αυτούς το αργύριον και το χρυσίον και τα σκεύη, την προσφοράν του οίκου του Θεού ημών, την οποίαν προσέφεραν ο βασιλεύς και οι σύμβουλοι αυτού και οι άρχοντες αυτού και πας ο παρευρεθείς Ισραήλ·
\par 26 εζύγισα λοιπόν και παρέδωκα εις την χείρα αυτών εξακόσια πεντήκοντα τάλαντα αργυρίου, και σκεύη αργυρά εκατόν ταλάντων, και εκατόν τάλαντα χρυσίου·
\par 27 και είκοσι φιάλας χρυσάς, χιλίων δραχμών, και δύο σκεύη εκ χαλκού στίλβοντος καλού, πολύτιμα ως χρυσίον.
\par 28 Και είπον προς αυτούς, Σεις είσθε άγιοι εις τον Κύριον, και τα σκεύη άγια· και το αργύριον και το χρυσίον αυτοπροαίρετος προσφορά εις Κύριον τον Θεόν των πατέρων σας.
\par 29 Προσέχετε και φυλάττετε αυτά, εωσού ζυγίσητε έμπροσθεν των αρχόντων των ιερέων και των Λευϊτών και των αρχόντων των πατριών του Ισραήλ, εν Ιερουσαλήμ, εντός των οικημάτων του οίκου του Κυρίου.
\par 30 Και παρέλαβον οι ιερείς και οι Λευΐται το βάρος του αργυρίου και του χρυσίου και τα σκεύη, διά να φέρωσιν αυτά εις Ιερουσαλήμ, προς τον οίκον του Θεού ημών.
\par 31 Και εσηκώθημεν από του ποταμού Ααβά την δωδεκάτην του πρώτου μηνός, διά να υπάγωμεν εις Ιερουσαλήμ· και η χειρ του Θεού ημών ήτο εφ' ημάς, και ηλευθέρωσεν ημάς εκ χειρός εχθρού και ενεδρεύοντος εν τη οδώ.
\par 32 Και ήλθομεν εις Ιερουσαλήμ· και εκαθήσαμεν εκεί τρεις ημέρας.
\par 33 Την τετάρτην δε ημέραν εζυγίσθη το αργύριον και το χρυσίον και τα σκεύη, εν τω οίκω του Θεού ημών, και παρεδόθη διά χειρός του Μερημώθ υιού του Ουρία του ιερέως· και μετ' αυτού ήτο Ελεάζαρ ο υιός του Φινεές· και μετ' αυτών Ιωζαβάδ, ο υιός του Ιησού, και Νωαδίας ο υιός του Βιννουΐ, οι Λευΐται·
\par 34 κατά αριθμόν και κατά βάρος τα πάντα· και άπαν το βάρος εγράφη εν τη ώρα εκείνη.
\par 35 Οι υιοί της μετοικεσίας, οι ελθόντες από της αιχμαλωσίας, προσέφεραν ολοκαυτώματα προς τον θεόν του Ισραήλ, δώδεκα μόσχους υπέρ παντός του Ισραήλ, ενενήκοντα εξ κριούς, εβδομήκοντα επτά αρνία, δώδεκα τράγους περί αμαρτίας, τα πάντα ολοκαύτωμα εις τον Κύριον.
\par 36 Και παρέδωκαν τα προστάγματα του βασιλέως εις τους σατράπας του βασιλέως και εις τους επάρχους τους πέραν του ποταμού· και ούτοι εβοήθησαν τον λαόν και τον οίκον του Θεού.

\chapter{9}

\par 1 Και αφού ετελέσθησαν ταύτα, προσήλθον προς εμέ οι άρχοντες, λέγοντες, Ο λαός του Ισραήλ και οι ιερείς και οι Λευΐται, δεν εχωρίσθησαν από του λαού των τόπων τούτων, και πράττουσι κατά τα βδελύγματα αυτών, των Χαναναίων, των Χετταίων, των Φερεζαίων, των Ιεβουσαίων, των Αμμωνιτών, των Μωαβιτών, των Αιγυπτίων και των Αμορραίων·
\par 2 διότι έλαβον εκ των θυγατέρων αυτών εις εαυτούς και εις τους υιούς αυτών· ώστε το σπέρμα το άγιον συνεμίχθη μετά του λαού των τόπων· και η χειρ των αρχόντων και των προεστώτων ήτο πρώτη εις την παράβασιν ταύτην.
\par 3 Και ως ήκουσα το πράγμα τούτο, διέσχισα το ιμάτιόν μου και το επένδυμά μου, και ανέσπασα τας τρίχας της κεφαλής μου και του πώγωνός μου, και εκαθήμην εκστατικός.
\par 4 Τότε συνήχθησαν προς εμέ πάντες οι τρέμοντες εις τους λόγους του Θεού του Ισραήλ, διά την παράβασιν των μετοικισθέντων· και εκαθήμην εκστατικός έως της εσπερινής προσφοράς.
\par 5 Και εν τη εσπερινή προσφορά εσηκώθην από της ταπεινώσεώς μου, και διασχίσας το ιμάτιόν μου και το επένδυμά μου, έκλινα επί τα γόνατά μου και εξέτεινα τας χείρας μου προς Κύριον τον Θεόν μου,
\par 6 και είπον, Θεέ μου, αισχύνομαι και ερυθριώ να υψώσω το πρόσωπόν μου προς σε, Θεέ μου· διότι αι ανομίαι ημών ηυξήνθησαν υπεράνω της κεφαλής, και αι παραβάσεις ημών εμεγαλύνθησαν έως των ουρανών.
\par 7 Από των ημερών των πατέρων ημών ήμεθα εν παραβάσει μεγάλη μέχρι της ημέρας ταύτης· και διά τας ανομίας ημών παρεδόθημεν, ημείς, οι βασιλείς ημών, οι ιερείς ημών, εις την χείρα των βασιλέων των τόπων, εις μάχαιραν, εις αιχμαλωσίαν και εις διαρπαγήν και εις αισχύνην προσώπου, ως είναι την ημέραν ταύτην.
\par 8 Και τώρα ως εν μιά στιγμή έγεινεν έλεος παρά Κυρίου του Θεού ημών, ώστε να διασωθή εις ημάς υπόλοιπον, και να δοθή εις ημάς στερέωσις εν τω αγίω αυτού τόπω, διά να φωτίζη ο Θεός ημών τους οφθαλμούς ημών, και να δώση εις ημάς μικράν αναψυχήν εν τη δουλεία ημών.
\par 9 Διότι δούλοι ήμεθα· και εν τη δουλεία ημών δεν εγκατέλιπεν ημάς ο Θεός ημών, αλλ' ηυδόκησε να εύρωμεν έλεος ενώπιον των βασιλέων της Περσίας, ώστε να δώση εις ημάς αναψυχήν, διά να ανεγείρωμεν τον οίκον του Θεού ημών και να ανορθώσωμεν τας ερημώσεις αυτού, και να δώση εις ημάς περιτείχισμα εν Ιούδα και εν Ιερουσαλήμ.
\par 10 Αλλά τώρα, Θεέ ημών, τι θέλομεν ειπεί μετά ταύτα; διότι εγκατελίπομεν τα προστάγματά σου,
\par 11 τα οποία προσέταξας διά χειρός των δούλων σου των προφητών, λέγων, Η γη, εις την οποίαν εισέρχεσθε διά να κληρονομήσητε αυτήν, είναι γη μεμολυσμένη με τον μολυσμόν των λαών των τόπων, με τα βδελύγματα αυτών, οίτινες εγέμισαν αυτήν, απ' άκρου έως άκρου, από των ακαθαρσιών αυτών.
\par 12 Τώρα λοιπόν τας θυγατέρας σας μη δίδετε εις τους υιούς αυτών, και τας θυγατέρας αυτών μη λαμβάνετε εις τους υιούς σας, και μη ζητείτε ποτέ την ειρήνην αυτών ή την ευτυχίαν αυτών, διά να κραταιωθήτε και να τρώγητε τα αγαθά της γης, και να αφήσητε αυτήν κληρονομίαν εις τους υιούς σας διά παντός.
\par 13 Και μετά πάντα τα επελθόντα εφ' ημάς ένεκα των πράξεων των πονηρών ημών, και της παραβάσεως ημών της μεγάλης, αφού συ, Θεέ ημών, εκρατήθης κάτω της αξίας των ανομιών ημών, και έδωκας εις ημάς ελευθέρωσιν τοιαύτην,
\par 14 πρέπει ημείς να αθετήσωμεν πάλιν τα προστάγματά σου, και να συμπενθερεύσωμεν με τον λαόν των βδελυγμάτων τούτων; δεν ήθελες οργισθή καθ' ημών, εωσού συντελέσης ημάς, ώστε να μη μείνη υπόλοιπον ή σεσωσμένον;
\par 15 Κύριε Θεέ του Ισραήλ, δίκαιος είσαι διότι εμείναμεν σεσωσμένοι, ως την ημέραν ταύτην· ιδού, ενώπιόν σου είμεθα με τας παραβάσεις ημών διότι δεν ήτο δυνατόν ένεκα τούτων να σταθώμεν ενώπιόν σου.

\chapter{10}

\par 1 Ενώ δε ο Έσδρας προσηύχετο και εξωμολογείτο, κλαίων και πεπτωκώς έμπροσθεν του οίκου του Θεού, συνήχθη προς αυτόν εκ του Ισραήλ σύναξις μεγάλη σφόδρα, άνδρες και γυναίκες και παιδία· διότι έκλαιεν ο λαός κλαυθμόν μέγαν.
\par 2 Και απεκρίθη Σεχανίας ο υιός του Ιεχιήλ, εκ των υιών Ελάμ, και είπε προς τον Έσδραν, Ημείς ηνομήσαμεν εις τον Θεόν ημών και ελάβομεν ξένας γυναίκας εκ των λαών της γής· πλην τώρα είναι ελπίς εις τον Ισραήλ περί τούτου·
\par 3 όθεν ας κάμωμεν τώρα διαθήκην προς τον Θεόν ημών, να αποβάλωμεν πάσας τας γυναίκας και τα γεννηθέντα εξ αυτών, κατά την συμβουλήν του κυρίου μου και των όσοι τρέμουσιν εις την εντολήν του Θεού ημών· και ας γείνη κατά τον νόμον·
\par 4 εγέρθητι διότι το πράγμα ανήκει εις σέ· και ημείς είμεθα μετά σού· ανδρίζου και πράττε.
\par 5 Τότε εγερθείς ο Έσδρας, ώρκισε τους άρχοντας των ιερέων, των Λευϊτών και παντός του Ισραήλ, ότι θέλουσι κάμει κατά τον λόγον τούτον. Και ώρκίσθησαν.
\par 6 Και σηκωθείς ο Έσδρας απ' έμπροσθεν του οίκου του Θεού, υπήγεν εις το οίκημα του Ιωανάν υιού του Ελιασείβ· και ότε ήλθεν εκεί, άρτον δεν έφαγεν και ύδωρ δεν έπιε· διότι ήτο εις πένθος διά την παράβασιν των μετοικισθέντων.
\par 7 Και διεκήρυξαν κατά την Ιουδαίαν και Ιερουσαλήμ προς πάντας τους υιούς της μετοικεσίας, να συναχθώσιν εις Ιερουσαλήμ·
\par 8 και πας όστις δεν έλθη εντός τριών ημερών, κατά την βουλήν των αρχόντων και πρεσβυτέρων, θέλει γείνει ανάθεμα πάσα η περιουσία αυτού, και αυτός θέλει χωρισθή από της συνάξεως των μετοικισθέντων.
\par 9 Και συνήχθησαν πάντες οι άνδρες Ιούδα και Βενιαμίν εις Ιερουσαλήμ εντός τριών ημερών. Ήτο ο ένατος μην και η εικοστή του μηνός· και πας ο λαός εκάθησεν εν τη πλατεία του οίκου του Θεού, τρέμων διά το πράγμα και διά την μεγάλην βροχήν.
\par 10 Και εγερθείς ο Έσδρας ο ιερεύς, είπε προς αυτούς, Σεις ηνομήσατε και ελάβετε γυναίκας ξένας, διά να επιπροσθέσητε εις την παράβασιν του Ισραήλ·
\par 11 τώρα λοιπόν εξομολογήθητε προς Κύριον τον Θεόν των πατέρων σας και κάμετε το θέλημα αυτού· και χωρίσθητε από των λαών της γης και από των ξένων γυναικών.
\par 12 Και απεκρίθη πάσα η σύναξις και είπον μετά φωνής μεγάλης, Καθώς ελάλησας προς ημάς, ούτω να κάμωμεν·
\par 13 ο λαός όμως είναι πολύς και ο καιρός πολύ βροχερός, και δεν δυνάμεθα να στεκώμεθα έξω, και το έργον δεν είναι μιας ημέρας ουδέ δύο· διότι είμεθα πολλοί οι αμαρτήσαντες εις τούτο το πράγμα·
\par 14 ας διορισθώσι τώρα άρχοντες ημών εν όλη τη συνάξει, και ας έλθωσι καθ' ωρισμένους καιρούς πάντες οι λαβόντες ξένας γυναίκας εις τας πόλεις ημών, και μετ' αυτών οι πρεσβύτεροι εκάστης πόλεως και οι κριταί αυτής, εωσού η φλογερά οργή του Θεού ημών διά το πράγμα τούτο αποστραφή αφ' ημών.
\par 15 Διωρίσθησαν λοιπόν εις τούτο Ιωνάθαν ο υιός του Ασαήλ, και Ιααζίας ο υιός του Τικβά· ο δε Μεσουλλάμ και ο Σαββεθαΐ, οι Λευΐται, ήσαν βοηθοί αυτών.
\par 16 Και έκαμον ούτως οι υιοί της μετοικεσίας. Και ο Έσδρας ο ιερεύς και άρχοντές τινές των πατριών, κατά τους πατρικούς οίκους αυτών, και ούτοι πάντες κατ' όνομα, εχωρίσθησαν και εκάθησαν την πρώτην ημέραν του δεκάτου μηνός διά να εξετάσωσι την υπόθεσιν.
\par 17 Και ετελείωσαν με πάντας τους άνδρας, τους λαβόντας ξένας γυναίκας, έως της πρώτης ημέρας του πρώτου μηνός.
\par 18 Και μεταξύ των υιών των ιερέων ευρέθησαν οι λαβόντες ξένας γυναίκας, εκ των υιών του Ιησού υιού του Ιωσεδέκ και των αδελφών αυτού, ο Μαασίας και ο Ελιέζερ, και ο Ιαρείβ και ο Γεδαλίας.
\par 19 Και έδωκαν τας χείρας αυτών, ότι θέλουσιν αποβάλει τας γυναίκας αυτών· και ως ένοχοι, προσέφεραν κριόν εκ του ποιμνίου διά την ανομίαν αυτών.
\par 20 Και εκ των υιών του Ιμμήρ, Ανανί και Ζεβαδίας.
\par 21 Και εκ των υιών του Χαρήμ, Μαασίας και Ηλίας και Σεμαΐας και Ιεχιήλ και Οζίας.
\par 22 Και εκ των υιών του Πασχώρ, Ελιωηνάϊ, Μαασίας, Ισμαήλ, Ναθαναήλ, Ιωζαβάδ και Ελασά.
\par 23 Εκ δε των Λευϊτών, Ιωζαβάδ και Σιμεΐ και Κελαΐας, ούτος είναι ο Κελιτά, Πεθαΐα, Ιούδας και Ελιέζερ.
\par 24 Και εκ των ψαλτωδών, Ελιασείβ· και εκ των θυρωρών, Σαλλούμ και Τελέμ και Ουρεί.
\par 25 Εκ δε του Ισραήλ, εκ των υιών Φαρώς, Ραμίας και Ιεζίας και Μαλχίας και Μιαμείν και Ελεάζαρ και Μαλχίας και Βεναΐας.
\par 26 Και εκ των υιών Ελάμ, Ματθανίας, Ζαχαρίας και Ιεχιήλ και Αβδί και Ιερεμώθ και Ηλιά.
\par 27 Και εκ των υιών Ζατθού, Ελιωηνάϊ, Ελιασείβ, Ματθανίας και Ιερεμώθ και Ζαβάδ και Αζιζά.
\par 28 Εκ δε των υιών Βηβαΐ, Ιωανάν, Ανανίας, Ζαββαΐ και Αθλαΐ.
\par 29 Και εκ των υιών Βανί, Μεσουλλάμ, Μαλλούχ και Αδαΐας, Ιασούβ και Σεάλ και Ραμώθ.
\par 30 Και εκ των υιών Φαάθ-μωάβ, Αδνά και Χελάλ, Βεναΐας, Μαασίας, Ματθανίας, Βεζελεήλ και Βιννουΐ και Μανασσής.
\par 31 Και εκ των υιών Χαρήμ, Ελιέζερ, Ιεσίας, Μαλχίας, Σεμαΐας και Συμεών,
\par 32 Βενιαμίν, Μαλλούχ και Σεμαρίας.
\par 33 Εκ των υιών Ασούμ, Ματθεναΐ, Ματταθά, Ζαβάδ, Ελιφελέτ, Ιερεμαΐ, Μανασσής και Σιμεΐ.
\par 34 Εκ των υιών Βανί, Μααδαΐας, Αμράμ και Ουήλ,
\par 35 Βεναΐας, Βεδεΐας, Χελλού,
\par 36 Βανίας, Μερημώθ, Ελιασείβ,
\par 37 Ματθανίας, Ματθεναΐ και Ιαασώ
\par 38 και Βανί και Βιννουΐ, Σιμεΐ,
\par 39 και Σελεμίας και Νάθαν και Αδαΐας,
\par 40 Μαχναδεβαΐ, Σασαΐ, Σαραΐ,
\par 41 Αζαρεήλ και Σελεμίας, Σεμαρίας,
\par 42 Σαλλούμ, Αμαρίας και Ιωσήφ.
\par 43 Εκ των υιών Νεβώ, Ιεϊήλ, Ματταθίας, Ζαβάδ, Ζεβινά, Ιαδαύ και Ιωήλ και Βεναΐας.
\par 44 Πάντες ούτοι είχον λάβει ξένας γυναίκας· και τινές εξ αυτών γυναίκας, εξ ων ετεκνοποίησαν.


\end{document}