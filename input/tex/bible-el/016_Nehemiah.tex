\begin{document}

\title{Nehemiah}


\chapter{1}

\par Λόγοι Νεεμία υιού του Αχαλία. Και εν τω μηνί Χισλεύ, εν τω εικοστώ έτει, ότε ήμην εν Σούσοις τη βασιλευούση,
\par 2 ο Ανανί, εις εκ των αδελφών μου, ήλθεν, αυτός και τινές εκ του Ιούδα, και ηρώτησα αυτούς περί των διασωθέντων Ιουδαίων, οίτινες εναπελείφθησαν εκ της αιχμαλωσίας, και περί Ιερουσαλήμ.
\par 3 Και είπον προς εμέ, Οι υπόλοιποι, οι εναπολειφθέντες εκ της αιχμαλωσίας εκεί εν τη επαρχία, είναι εν θλίψει μεγάλη, και ονειδισμώ· και το τείχος της Ιερουσαλήμ καθηρέθη, και αι πύλαι αυτής κατεκαύθησαν εν πυρί.
\par 4 Και ότε ήκουσα τους λόγους τούτους, εκάθησα και έκλαυσα και επένθησα ημέρας και ενήστευον, και προσηυχόμην ενώπιον του Θεού του ουρανού,
\par 5 και είπα, Δέομαι, Κύριε, Θεέ του ουρανού, ο μέγας και φοβερός Θεός, ο φυλάττων την διαθήκην και το έλεος προς τους αγαπώντας αυτόν και τηρούντας τας εντολάς αυτού,
\par 6 ας ήναι τώρα το ους σου προσεκτικόν και οι οφθαλμοί σου ανεωγμένοι, διά να ακούσης την προσευχήν του δούλου σου, την οποίαν ήδη προσεύχομαι ενώπιόν σου ημέραν και νύκτα υπέρ των υιών Ισραήλ των δούλων σου, και εξομολογούμαι τα αμαρτήματα των υιών Ισραήλ, τα οποία ημαρτήσαμεν εις σέ· και εγώ και ο οίκος του πατρός μου ημαρτήσαμεν.
\par 7 Όλως διεφθάρημεν ενώπιόν σου, και δεν εφυλάξαμεν τας εντολάς και τα διατάγματα και τας κρίσεις, τας οποίας προσέταξας εις τον δούλον σου τον Μωϋσήν.
\par 8 Ενθυμήθητι, δέομαι, τον λόγον, τον οποίον προσέταξας εις τον δούλον σου τον Μωϋσήν, λέγων, Εάν γείνητε παραβάται, εγώ θέλω σας διασκορπίσει μεταξύ των εθνών·
\par 9 αλλ' εάν επιστρέψητε προς εμέ και φυλάξητε τας εντολάς μου και εκτελήτε αυτάς, και αν ήναι από σας απερριμμένοι έως των εσχάτων του ουρανού, και εκείθεν θέλω συνάξει αυτούς και θέλω φέρει αυτούς εις τον τόπον, τον οποίον εξέλεξα διά να κατοικίσω το όνομά μου εκεί.
\par 10 Ούτοι δε είναι δούλοί σου και λαός σου, τους οποίους ελύτρωσας διά της δυνάμεώς σου της μεγάλης και διά της χειρός σου της κραταιάς.
\par 11 Δέομαι, Κύριε, ας ήναι ήδη το ους σου προσεκτικόν εις την προσευχήν του δούλου σου και εις την προσευχήν των δούλων σου, των θελόντων να φοβώνται το όνομά σου· και ευόδωσον, δέομαι, τον δούλον σου την ημέραν ταύτην, και χάρισον εις αυτόν έλεος ενώπιον του ανδρός τούτου. Διότι εγώ ήμην οινοχόος του βασιλέως.

\chapter{2}

\par Και εν τω μηνί Νισάν, εν τω εικοστώ έτει Αρταξέρξου του βασιλέως, ήτο οίνος έμπροσθεν αυτού· και λαβών τον οίνον, έδωκα εις τον βασιλέα. Ποτέ δε δεν είχον σκυθρωπάσει ενώπιον αυτού.
\par 2 Όθεν ο βασιλεύς είπε προς εμέ, Διά τι το πρόσωπόν σου είναι σκυθρωπόν, ενώ συ άρρωστος δεν είσαι; τούτο δεν είναι ειμή λύπη καρδίας. Τότε εφοβήθην πολύ σφόδρα.
\par 3 Και είπα προς τον βασιλέα, Ζήτω ο βασιλεύς εις τον αιώνα· διά τι το πρόσωπόν μου να μη ήναι σκυθρωπόν, ενώ η πόλις, ο τόπος των τάφων των πατέρων μου, κείται ηρημωμένος, και αι πύλαι αυτής κατηναλωμέναι υπό του πυρός;
\par 4 Τότε ο βασιλεύς είπε προς εμέ, Περί τίνος κάμνεις συ αίτησιν; Και προσηυχήθην εις τον Θεόν του ουρανού.
\par 5 Και είπα προς τον βασιλέα, Εάν ήναι αρεστόν εις τον βασιλέα, και εάν ο δούλός σου εύρηκε χάριν ενώπιόν σου, να με πέμψης εις τον Ιούδαν, εις την πόλιν των τάφων των πατέρων μου, και να ανοικοδομήσω αυτήν.
\par 6 Και είπεν ο βασιλεύς προς εμέ, καθημένης πλησίον αυτού της βασιλίσσης, Πόσον μακρά θέλει είσθαι η πορεία σου; και πότε θέλεις επιστρέψει; Και ευηρεστήθη ο βασιλεύς και με έπεμψε· και έδωκα εις αυτόν προθεσμίαν.
\par 7 Και είπα προς τον βασιλέα, Εάν ήναι αρεστόν εις τον βασιλέα, ας μοι δοθώσιν επιστολαί προς τους πέραν του ποταμού επάρχους, διά να με συμπαραπέμψωσιν, εωσού έλθω εις τον Ιούδαν·
\par 8 και επιστολή προς τον Ασάφ τον φύλακα του βασιλικού δάσους, διά να μοι δώση ξύλα να κατασκευάσω τας πύλας του φρουρίου του ναού και το τείχος της πόλεως και τον οίκον, εις τον οποίον θέλω εισέλθει. Και εχάρισεν ο βασιλεύς εις εμέ πάντα, κατά την επ' εμέ αγαθήν χείρα του Θεού μου.
\par 9 Ήλθον λοιπόν προς τους πέραν του ποταμού επάρχους και έδωκα εις αυτούς τας επιστολάς του βασιλέως. Είχε δε αποστείλει ο βασιλεύς αρχηγούς δυνάμεως και ιππείς μετ' εμού.
\par 10 Ότε δε Σαναβαλλάτ ο Ορωνίτης και Τωβίας ο δούλος, ο Αμμωνίτης, ήκουσαν, ελυπήθησαν καθ' υπερβολήν ότι ήλθεν άνθρωπος να ζητήση το καλόν των υιών Ισραήλ.
\par 11 Και ήλθον εις Ιερουσαλήμ και ήμην εκεί τρεις ημέρας.
\par 12 Και εσηκώθην την νύκτα, εγώ και ολίγοι τινές μετ' εμού· και δεν εφανέρωσα εις ουδένα τι είχε βάλει ο Θεός μου εν τη καρδία μου να κάμω εις την Ιερουσαλήμ· και άλλο κτήνος δεν ήτο μετ' εμού, ειμή το κτήνος επί του οποίου εκαθήμην.
\par 13 Και εξήλθον την νύκτα διά της πύλης της φάραγγος, και ήλθον απέναντι της πηγής του δράκοντος και προς την θύραν της κοπρίας, και παρετήρουν τα τείχη της Ιερουσαλήμ, τα οποία ήσαν κατακεκρημνισμένα, και τας πύλας αυτής κατηναλωμένας υπό του πυρός.
\par 14 Έπειτα διέβην εις την πύλην της πηγής και εις την βασιλικήν κολυμβήθραν· και δεν ήτο τόπος διά να περάση το κτήνος το υποκάτω μου.
\par 15 Και ανέβην την νύκτα διά του χειμάρρου· και αφού παρετήρησα το τείχος, εστράφην και εισήλθον διά της πύλης της φάραγγος και επέστρεψα.
\par 16 Οι δε προεστώτες δεν ήξευρον που υπήγα και τι έκαμον· ουδέ είχον φανερώσει έτι τούτο ούτε εις τους Ιουδαίους, ούτε εις τους ιερείς, ούτε εις τους προκρίτους, ούτε εις τους προεστώτας, ούτε εις τους λοιπούς τους εργαζομένους το έργον.
\par 17 Και είπα προς αυτούς, Σεις βλέπετε την δυστυχίαν εις την οποίαν είμεθα, πως η Ιερουσαλήμ κείται ηρημωμένη και αι πύλαι αυτής είναι κατηναλωμέναι υπό του πυρός· έλθετε και ας ανοικοδομήσωμεν το τείχος της Ιερουσαλήμ, διά να μη ήμεθα πλέον όνειδος.
\par 18 Και απήγγειλα προς αυτούς περί της επ' εμέ αγαθής χειρός του Θεού μου, και έτι τους λόγους του βασιλέως, τους οποίους είπε προς εμέ. Οι δε είπον, Ας σηκωθώμεν και ας οικοδομήσωμεν. Ούτως ενίσχυσαν τας χείρας αυτών προς το αγαθόν.
\par 19 Αλλ' ότε ήκουσαν ο Σαναβαλλάτ ο Ορωνίτης και Τωβίας ο δούλος, ο Αμμωνίτης, και ο Γησέμ ο Άραψ, περιεγέλασαν ημάς και περιεφρόνησαν ημάς, λέγοντες, Τι είναι το πράγμα τούτο το οποίον κάμνετε; θέλετε να επαναστατήσητε κατά του βασιλέως;
\par 20 Και εγώ απεκρίθην προς αυτούς και είπα προς αυτούς, Ο Θεός του ουρανού, αυτός θέλει ευοδώσει ημάς· διά τούτο ημείς οι δούλοι αυτού θέλομεν σηκωθή και οικοδομήσει· σεις όμως δεν έχετε μερίδα ουδέ δικαίωμα ουδέ μνημόσυνον εν Ιερουσαλήμ.

\chapter{3}

\par Τότε εσηκώθη Ελιασείβ ο ιερεύς ο μέγας, και οι αδελφοί αυτού οι ιερείς, και ωκοδόμησαν την πύλην την προβατικήν· ούτοι ηγίασαν αυτήν και έστησαν τας θύρας αυτής· και ηγίασαν αυτήν έως του πύργου Μεά, έως του πύργου Ανανεήλ.
\par 2 Και εις τα πλάγια αυτού ωκοδόμησαν οι άνδρες της Ιεριχώ. Και εις τα πλάγια αυτών ωκοδόμησε Ζακχούρ ο υιός του Ιμρί.
\par 3 Την θύραν δε την ιχθυϊκήν ωκοδόμησαν οι υιοί του Ασσεναά, οίτινες εσανίδωσαν αυτήν και έστησαν τας θύρας αυτής, τα κλείθρα αυτής και τους μοχλούς αυτής.
\par 4 Και εις τα πλάγια αυτών επεσκεύασε Μερημώθ ο υιός του Ουρία, υιού του Ακκώς. Και εις τα πλάγια αυτών επεσκεύασε Μεσουλλάμ ο υιός του Βαραχίου, υιού του Μεσηζαβεήλ. Και εις τα πλάγια αυτών επεσκεύασε Σαδώκ ο υιός του Βαανά.
\par 5 Και εις τα πλάγια αυτών επεσκεύασαν οι Θεκωΐται πλην οι πρόκριτοι αυτών δεν υπέβαλον τον τράχηλον αυτών εις το έργον του Κυρίου αυτών.
\par 6 Και την πύλην την παλαιάν επεσκεύασεν Ιωδαέ ο υιός του Φασέα, και Μεσουλλάμ ο υιός του Βεσωδία· ούτοι εσανίδωσαν αυτήν και έστησαν τας θύρας αυτής και τα κλείθρα αυτής και τους μοχλούς αυτής.
\par 7 Και εις τα πλάγια αυτών επεσκεύασε Μελαθίας ο Γαβαωνίτης και Ιαδών ο Μερωνοθίτης, άνδρες της Γαβαών και της Μισπά, υπό τον θρόνον του επάρχου των εντεύθεν του ποταμού.
\par 8 Εις τα πλάγια αυτού επεσκεύασεν Οχιήλ ο υιός του Αραχίου, εκ των χρυσοχόων. Και εις τα πλάγια αυτού επεσκεύασεν Ανανίας, ο εκ των μυρεψών· και αφήκαν την Ιερουσαλήμ έως του τείχους του πλατέος.
\par 9 Και εις τα πλάγια αυτών επεσκεύασε Ρεφαΐα ο υιός του Ωρ, ο άρχων του ημίσεος της περιχώρου της Ιερουσαλήμ.
\par 10 Και εις τα πλάγια αυτών επεσκεύασεν Ιεδαΐας ο υιός του Αρουμάφ, και απέναντι της οικίας αυτού. Και εις τα πλάγια αυτού επεσκεύασε Χαττούς ο υιός του Ασαβνία.
\par 11 Μαλχίας ο υιός του Χαρήμ, και Ασσούβ ο υιός του Φαάθ-μωάβ, επεσκεύασαν το άλλο τμήμα και τον πύργον των φούρνων.
\par 12 Και εις τα πλάγια αυτού επεσκεύασε Σαλλούμ ο υιός του Αλλωής, ο άρχων του ημίσεος της περιχώρου της Ιερουσαλήμ, αυτός και αι θυγατέρες αυτού.
\par 13 την πύλην της φάραγγος επεσκεύασεν ο Ανούν και οι κάτοικοι της Ζανωά· ούτοι ωκοδόμησαν αυτήν και έστησαν τας θύρας αυτής, τα κλείθρα αυτής και τους μοχλούς αυτής και χιλίας πήχας εις το τείχος έως της πύλης της κοπρίας.
\par 14 Την πύλην δε της κοπρίας επεσκεύασε Μαλχίας ο υιός του Ρηχάβ, ο άρχων της περιχώρου της Βαιθ-ακκερέμ· ούτος ωκοδόμησεν αυτήν και έστησε τας θύρας αυτής, τα κλείθρα αυτής και τους μοχλούς αυτής.
\par 15 Την πύλην δε της πηγής επεσκεύασε Σαλλούν ο υιός του Χολ-οζέ, ο άρχων της περιχώρου της Μισπά· ούτος ωκοδόμησεν αυτήν και εσανίδωσεν αυτήν και έστησε τας θύρας αυτής, τα κλείθρα αυτής και τους μοχλούς αυτής, και το τείχος της κολυμβήθρας του Σιλωάμ πλησίον του κήπου του βασιλέως, και έως των βαθμίδων των καταβαινουσών, από της πόλεως Δαβίδ.
\par 16 Κατόπιν αυτού επεσκεύασε Νεεμίας ο υιός του Αζβούκ, ο άρχων του ημίσεος της περιχώρου της Βαιθ-σούρ, έως απέναντι των τάφων του Δαβίδ και έως της κατασκευασθείσης κολυμβήθρας και έως του οίκου των ισχυρών.
\par 17 Κατόπιν αυτού επεσκεύασαν οι Λευΐται, Ρεούμ ο υιός του Βανί. Εις τα πλάγια αυτού επεσκεύασεν Ασαβίας, ο άρχων του ημίσεος της περιχώρου της Κεειλά, διά το μέρος αυτού.
\par 18 Κατόπιν αυτού επεσκεύασαν οι αδελφοί αυτών, Βαβαΐ ο υιός του Ηναδάδ, ο άρχων του άλλου ημίσεος της περιχώρου της Κεειλά.
\par 19 Και εις τα πλάγια αυτού επεσκεύασεν Εσέρ ο υιός του Ιησού, ο άρχων της Μισπά, άλλο τμήμα απέναντι της αναβάσεως προς την οπλοθήκην της γωνίας.
\par 20 Κατόπιν αυτού Βαρούχ ο υιός του Ζαββαΐ επεσκεύασε μετά ζήλου το άλλο τμήμα, από της γωνίας έως της θύρας του οίκου Ελιασείβ του ιερέως του μεγάλου.
\par 21 Κατόπιν αυτού επεσκεύασε Μερημώθ ο υιός του Ουρίου, υιού του Ακκώς, άλλο τμήμα, από της θύρας του οίκου του Ελιασείβ έως του τέλους του οίκου του Ελιασείβ.
\par 22 Και κατόπιν αυτού επεσκεύασαν οι ιερείς, οι κάτοικοι της περιχώρου.
\par 23 Κατόπιν αυτών επεσκεύασαν ο Βενιαμίν και ο Ασαούβ απέναντι του οίκου αυτών. Κατόπιν αυτών επεσκεύασεν Αζαρίας ο υιός του Μαασίου, υιού του Ανανίου, πλησίον του οίκου αυτού.
\par 24 Κατόπιν αυτού επεσκεύασε Βιννουΐ ο υιός του Ηναδάδ άλλο τμήμα, από του οίκου του Αζαρίου έως της καμπής, έως μάλιστα της γωνίας.
\par 25 Φαλάλ ο υιός του Ουζαΐ επεσκεύασεν απέναντι της καμπής και του πύργου του εξέχοντος από του υψηλού οίκου του βασιλέως, του πλησίον της αυλής της φυλακής. Κατόπιν αυτού Φεδαΐας ο υιός του Φαρώς.
\par 26 Οι δε Νεθινείμ κατώκουν εν Οφήλ, και επεσκεύασαν έως απέναντι της πύλης των υδάτων προς ανατολάς και του πύργου του εξέχοντος.
\par 27 Κατόπιν αυτών οι Θεκωίται επεσκεύασαν άλλο τμήμα, απέναντι του μεγάλου πύργου του εξέχοντος και έως του τείχους του Οφήλ.
\par 28 Επάνωθεν της πύλης των ίππων επεσκεύασαν οι ιερείς, έκαστος απέναντι της οικίας αυτού.
\par 29 Κατόπιν αυτών επεσκεύασε Σαδώκ ο υιός του Ιμμήρ, απέναντι της οικίας αυτού. Και κατόπιν αυτού επεσκεύασε Σεμαΐας ο υιός του Σεχανίου, ο φύλαξ της ανατολικής πύλης.
\par 30 Κατόπιν αυτού επεσκεύασεν Ανανίας ο υιός του Σελεμία, και Ανούν ο έκτος υιός του Σαλάφ, άλλο τμήμα. Κατόπιν αυτού επεσκεύασε Μεσουλλάμ ο υιός του Βαραχίου απέναντι του οικήματος αυτού.
\par 31 Κατόπιν αυτού επεσκεύασε Μαλχίας, υιός χρυσοχόου, έως της οικίας των Νεθινείμ και των μεταπρατών, απέναντι της πύλης Μιφκάδ, και έως της αναβάσεως της γωνίας.
\par 32 Και μεταξύ της αναβάσεως της γωνίας έως της προβατικής πύλης, επεσκεύασαν οι χρυσοχόοι και οι μεταπράται.

\chapter{4}

\par Ότε δε ήκουσεν ο Σαναβαλλάτ ότι ημείς οικοδομούμεν το τείχος, ωργίσθη και ηγανάκτησε πολύ και περιεγέλασε τους Ιουδαίους.
\par 2 Και ελάλησεν ενώπιον των αδελφών αυτού και του στρατεύματος της Σαμαρείας και είπε, Τι κάμνουσιν οι άθλιοι ούτοι Ιουδαίοι; θέλουσιν αφήσει αυτούς; θέλουσι θυσιάσει; θέλουσι τελειώσει εν μιά ημέρα; θέλουσιν αναζωοποιήσει εκ των σωρών του χώματος τους λίθους, και τούτους κεκαυμένους;
\par 3 Πλησίον δε αυτού ήτο Τωβίας ο Αμμωνίτης· και είπε, Και αν κτίσωσιν, αλώπηξ αναβαίνουσα θέλει καθαιρέσει το λίθινον αυτών τείχος.
\par 4 Άκουσον, Θεέ ημών· διότι μυκτηριζόμεθα· και στρέψον τον ονειδισμόν αυτών κατά της κεφαλής αυτών και κάμε αυτούς να γείνωσι λάφυρον εν γη αιχμαλωσίας·
\par 5 και μη καλύψης την ανομίαν αυτών, και η αμαρτία αυτών ας μη εξαλειφθή απ' έμπροσθέν σου· διότι προέφεραν ονειδισμούς κατά των οικοδομούντων.
\par 6 Ούτως ανωκοδομήσαμεν το τείχος· και άπαν το τείχος συνεδέθη, έως του ημίσεος αυτού· διότι ο λαός είχε καρδίαν εις το εργάζεσθαι.
\par 7 Αλλ' ότε Σαναβαλλάτ και Τωβίας και οι Άραβες και οι Αμμωνίται και οι Αζώτιοι ήκουσαν ότι τα τείχη της Ιερουσαλήμ επισκευάζονται, και ότι τα χαλάσματα ήρχισαν να φράττωνται, ωργίσθησαν σφόδρα·
\par 8 και συνώμοσαν πάντες ομού να έλθωσι να πολεμήσωσιν εναντίον της Ιερουσαλήμ, και να κάμωσιν εις αυτήν βλάβην.
\par 9 Και ημείς προσηυχήθημεν εις τον Θεόν ημών και εστήσαμεν φυλακάς εναντίον αυτών ημέραν και νύκτα, φοβούμενοι απ' αυτών.
\par 10 Και είπεν ο Ιούδας, Η δύναμις των εργατών ητόνησε, και το χώμα είναι πολύ, και ημείς δεν δυνάμεθα να οικοδομώμεν το τείχος.
\par 11 Οι δε εχθροί ημών είπον, Δεν θέλουσι μάθει ουδέ θέλουσιν ιδεί, εωσού έλθωμεν εις το μέσον αυτών και φονεύσωμεν αυτούς, και καταπαύσωμεν το έργον.
\par 12 Και ελθόντες οι Ιουδαίοι, οι κατοικούντες πλησίον αυτών, είπον προς ημάς δεκάκις, Προσέχετε από πάντων των τόπων, διά των οποίων επιστρέφετε προς ημάς.
\par 13 Όθεν έστησα εις τους χαμηλοτέρους τόπους όπισθεν του τείχους και εις τους υψηλοτέρους τόπους, έστησα τον λαόν κατά συγγενείας, με τας ρομφαίας αυτών, με τας λόγχας αυτών και με τα τόξα αυτών.
\par 14 Και είδον και εσηκώθην και είπα προς τους προκρίτους και προς τους προεστώτας και προς το επίλοιπον του λαού, Μη φοβηθήτε απ' αυτών· ενθυμείσθε τον Κύριον, τον μέγαν και φοβερόν, και πολεμήσατε υπέρ των αδελφών σας, των υιών σας και των θυγατέρων σας, των γυναικών σας και των οίκων σας.
\par 15 Και ότε οι εχθροί ημών ήκουσαν ότι το πράγμα εγνώσθη εις ημάς, και διεσκέδασεν ο Θεός την βουλήν αυτών, επεστρέψαμεν πάντες ημείς εις το τείχος, έκαστος εις το έργον αυτού.
\par 16 Και απ' εκείνης της ημέρας το ήμισυ των δούλων μου ειργάζοντο το έργον, και το ήμισυ αυτών εκράτουν τας λόγχας, τους θυρεούς και τα τόξα, τεθωρακισμένοι και οι άρχοντες ήσαν οπίσω παντός του οίκου Ιούδα.
\par 17 Οι οικοδομούντες το τείχος και οι αχθοφορούντες και οι φορτίζοντες, έκαστος διά της μιας χειρός αυτού εδούλευεν εις το έργον και διά της άλλης εκράτει το όπλον.
\par 18 Οι δε οικοδόμοι, έκαστος είχε την ρομφαίαν αυτού περιεζωσμένην εις την οσφύν αυτού και ωκοδόμει ο δε σαλπίζων εν τη σάλπιγγι ήτο πλησίον μου.
\par 19 Και είπα προς τους προκρίτους και προς τους προεστώτας και προς το επίλοιπον του λαού, το έργον είναι μέγα και πλατύ· ημείς δε είμεθα διακεχωρισμένοι επί το τείχος, ο εις μακράν του άλλου·
\par 20 εις όντινα λοιπόν τόπον ακούσητε την φωνήν της σάλπιγγος, εκεί δράμετε προς ημάς· ο Θεός ημών θέλει πολεμήσει υπέρ ημών.
\par 21 Ούτως ειργαζόμεθα το έργον· και το ήμισυ αυτών εκράτει τας λόγχας, απ' αρχής της αυγής έως της ανατολής των άστρων.
\par 22 Και κατά τον αυτόν καιρόν είπα προς τον λαόν, Έκαστος μετά του δούλου αυτού ας διανυκτερεύη εν τω μέσω της Ιερουσαλήμ, και ας ήναι την νύκτα φύλακες εις ημάς, και ας εργάζωνται την ημέραν.
\par 23 Και ούτε εγώ, ούτε οι αδελφοί μου, ούτε οι δούλοί μου, ούτε οι άνδρες της προφυλάξεως οι ακολουθούντές με, ουδείς εξ ημών εξεδύετο τα ιμάτια αυτού· μόνον διά να λούηται εξεδύετο έκαστος.

\chapter{5}

\par Ήτο δε μεγάλη κραυγή του λαού και των γυναικών αυτών κατά των αδελφών αυτών των Ιουδαίων.
\par 2 Διότι ήσαν τινές λέγοντες, Ημείς, οι υιοί ημών και αι θυγατέρες ημών, είμεθα πολλοί· όθεν ας λάβωμεν σίτον, διά να φάγωμεν και να ζήσωμεν.
\par 3 Και ήσαν τινές λέγοντες, Ημείς βάλλομεν ενέχυρον τους αγρούς ημών, τους αμπελώνας ημών και τας οικίας ημών, διά να λάβωμεν σίτον εξ αιτίας της πείνης.
\par 4 Ήσαν έτι τινές λέγοντες, Ημείς εδανείσθημεν αργύρια διά τους φόρους του βασιλέως επί τους αγρούς και επί τους αμπελώνας ημών·
\par 5 τώρα δε η σαρξ ημών είναι ως η σαρξ των αδελφών ημών, τα τέκνα ημών ως τα τέκνα αυτών· και ιδού, ημείς καθυποβάλλομεν εις δουλείαν τους υιούς ημών και τας θυγατέρας ημών διά να ήναι δούλοι, και τινές εκ των θυγατέρων ημών εφέρθησαν ήδη εις δουλείαν· και δεν είναι ουδέν εις την εξουσίαν ημών, διότι άλλοι έχουσι τους αγρούς και τους αμπελώνας ημών.
\par 6 Και ηγανάκτησα σφόδρα, ακούσας την κραυγήν αυτών και τους λόγους τούτους.
\par 7 Και εσκέφθην κατ' εμαυτόν, και επέπληξα τους προκρίτους και τους προεστώτας και είπα προς αυτούς, Σεις φορολογείτε έκαστος τον αδελφόν αυτού. Και συνεκάλεσα κατ' αυτών σύναξιν μεγάλην.
\par 8 Και είπα προς αυτούς, Ημείς κατά την δύναμιν ημών εξηγοράσαμεν τους αδελφούς ημών Ιουδαίους, τους πωληθέντας εις τα έθνη· και σεις αυτοί θέλετε πωλήσει τους αδελφούς σας; ή θέλουσι πωληθή εις ημάς; Εκείνοι δε εσιώπων και δεν εύρηκαν απόκρισιν.
\par 9 Και είπα, Δεν είναι καλόν το πράγμα το οποίον σεις κάμνετε· δεν πρέπει να περιπατήτε εν τω φόβω του Θεού ημών, διά να μη ονειδίζωσιν ημάς τα έθνη, οι εχθροί ημών;
\par 10 και εγώ έτι και οι αδελφοί μου και οι δούλοί μου εδανείσαμεν εις αυτούς χρήματα και σίτον· ας αφήσωμεν, παρακαλώ, την απαίτησιν ταύτην·
\par 11 επιστρέψατε λοιπόν εις αυτούς, ταύτην την ημέραν, τους αγρούς αυτών, τους αμπελώνας αυτών, τους ελαιώνας αυτών και τους οίκους αυτών και το εκατοστόν του αργυρίου και του σίτου, του οίνου και του ελαίου, το οποίον απαιτείτε παρ' αυτών.
\par 12 Τότε είπον, Θέλομεν αποδώσει ταύτα και δεν θέλομεν ζητήσει ουδέν παρ' αυτών· ούτω θέλομεν κάμει, καθώς συ λέγεις. Τότε εκάλεσα τους ιερείς και ώρκισα αυτούς, ότι θέλουσι κάμει κατά τον λόγον τούτον.
\par 13 Εξετίναξα έτι τον κόλπον μου, λέγων, Ούτω να εκτινάξη ο Θεός πάντα άνθρωπον από του οίκου αυτού και από του κόπου αυτού, όστις δεν εκτελέση τον λόγον τούτον, και ούτω να ήναι εκτετιναγμένος και κενός. Και είπον πάσα η σύναξις, Αμήν, και εδόξασαν τον Κύριον. Και έκαμεν ο λαός κατά τον λόγον τούτον.
\par 14 Αφ' ης δε ημέρας προσετάχθην να ήμαι κυβερνήτης αυτών εν τη γη Ιούδα, από του εικοστού έτους έως του τριακοστού δευτέρου έτους Αρταξέρξου του βασιλέως, δώδεκα έτη, εγώ και οι αδελφοί μου δεν εφάγομεν τον άρτον του κυβερνήτου.
\par 15 Οι πρότεροι όμως κυβερνήται, οι προ εμού, κατεβάρυνον τον λαόν, και ελάμβανον παρ' αυτών άρτον και οίνον, εκτός τεσσαράκοντα σίκλων αργυρίου· έτι και οι δούλοι αυτών εξουσίαζον τον λαόν· αλλ' εγώ δεν έκαμνον ούτω, φοβούμενος τον Θεόν.
\par 16 Και μάλιστα ενισχύθην εις το έργον τούτου του τείχους, και αγρόν δεν ηγοράσαμεν· και πάντες οι δούλοί μου ήσαν συνηγμένοι εκεί εις το έργον.
\par 17 Ήσαν έτι εις την τράπεζάν μου εκατόν πεντήκοντα άνδρες εκ των Ιουδαίων και των προεστώτων, και οι ερχόμενοι προς ημάς εκ των εθνών των πέριξ ημών.
\par 18 Το δε καθ' ημέραν ετοιμαζόμενον δι' εμέ ήτο εις βους και εξ εκλεκτά πρόβατα· και πτηνά ητοιμάζοντο δι' εμέ, και άπαξ εις δέκα ημέρας αφθονία από παντός είδους οίνου· και όμως δεν εζήτησα τον άρτον του κυβερνήτου· διότι η δουλεία ήτο βαρεία επί τούτον τον λαόν.
\par 19 Μνήσθητί μου, Θεέ μου, επ' αγαθώ, κατά πάντα όσα εγώ έκαμον υπέρ του λαού τούτου.

\chapter{6}

\par Καθώς δε ήκουσαν ο Σαναβαλλάτ και ο Τωβίας και ο Γησέμ ο Άραψ και οι λοιποί εκ των εχθρών ημών, ότι εγώ ωκοδόμησα το τείχος και δεν έμεινε πλέον χάλασμα εις αυτό, αν και μέχρις εκείνου του καιρού θύρας δεν έστησα επί των πυλών,
\par 2 ο Σαναβαλλάτ και ο Γησέμ απέστειλαν προς εμέ, λέγοντες, Έλθετε, και ας συναχθώμεν ομού εις τινά εκ των κωμών εν τη πεδιάδι Ωνώ. Εβουλεύοντο δε να κάμωσιν εις εμέ κακόν.
\par 3 Και απέστειλα μηνυτάς προς αυτούς, λέγων, Έργον μέγα κάμνω και δεν δύναμαι να καταβώ· διά τι να παύση το έργον, όταν εγώ αφήσας αυτό καταβώ προς εσάς;
\par 4 Και απέστειλαν προς εμέ τετράκις κατά τον τρόπον τούτον· και εγώ απεκρίθην προς αυτούς κατά τον αυτόν τρόπον.
\par 5 Τότε ο Σαναβαλλάτ απέστειλε προς εμέ τον δούλον αυτού κατά τον αυτόν τρόπον, πέμπτην φοράν, με ανοικτήν επιστολήν εις την χείρα αυτού·
\par 6 εν ή ήτο γεγραμμένον, Ηκούσθη μεταξύ των εθνών, και ο Γασμού λέγει, ότι συ και οι Ιουδαίοι βουλεύεσθε να επαναστατήσητε· διά τούτο συ οικοδομείς το τείχος, διά να γείνης βασιλεύς αυτών, κατά τους λόγους τούτους·
\par 7 έτι διώρισας προφήτας, να κηρύττωσι περί σου εν Ιερουσαλήμ, λέγοντες, Είναι βασιλεύς εν Ιούδα· και τώρα θέλει απαγγελθή προς τον βασιλέα κατά τους λόγους τούτους· ελθέ λοιπόν τώρα, και ας συμβουλευθώμεν ομού.
\par 8 Τότε απέστειλα προς αυτόν, λέγων, Δεν είναι τοιαύτα πράγματα καθώς συ λέγεις, αλλά συ πλάττεις αυτά εκ της καρδίας σου.
\par 9 Διότι πάντες ούτοι εφοβέριζον ημάς, λέγοντες, Θέλουσιν εξασθενήσει αι χείρες αυτών από του έργου, και δεν θέλει εκτελεσθή. Τώρα λοιπόν, Θεέ, κραταίωσον τας χείρας μου.
\par 10 Και εγώ υπήγα εις την οικίαν του Σεμαΐα, υιού του Δαλαΐα, υιού του Μεεταβεήλ, όστις ήτο κεκλεισμένος· και είπεν, Ας συνέλθωμεν ομού εις τον οίκον του Θεού, εντός του ναού, και ας κλείσωμεν τας θύρας του ναού· διότι αυτοί έρχονται να σε φονεύσωσι· ναι, την νύκτα έρχονται να σε φονεύσωσιν.
\par 11 Αλλ' εγώ απεκρίθην, Άνθρωπος οποίος εγώ ήθελον φύγει; και τις, οποίος εγώ, ήθελεν εισέλθει εις τον ναόν διά να σώση την ζωήν αυτού; δεν θέλω εισέλθει.
\par 12 Και ιδού, εγνώρισα ότι ο Θεός δεν απέστειλεν αυτόν να προφέρη την προφητείαν ταύτην εναντίον μου· αλλ' ότι ο Τωβίας και ο Σαναβαλλάτ εμίσθωσαν αυτόν.
\par 13 Διά τούτο ήτο μεμισθωμένος, διά να φοβηθώ και να κάμω ούτω και να αμαρτήσω, και να έχωσιν αφορμήν να κακολογήσωσι, διά να με ονειδίσωσι.
\par 14 Μνήσθητι, Θεέ μου, του Τωβία και του Σαναβαλλάτ κατά τα έργα αυτών ταύτα, και έτι της προφητίσσης Νωαδίας και των λοιπών προφητών, οίτινες με εφοβέριζον.
\par 15 Ούτω συνετελέσθη το τείχος κατά την εικοστήν πέμπτην του μηνός Ελούλ, εν πεντήκοντα δύο ημέραις.
\par 16 Και ότε ήκουσαν πάντες οι εχθροί ημών, τότε εφοβήθησαν πάντα τα έθνη τα πέριξ ημών, και εταπεινώθησαν σφόδρα εις τους οφθαλμούς εαυτών· διότι εγνώρισαν ότι παρά του Θεού ημών έγεινε το έργον τούτο.
\par 17 Προσέτι εν ταις ημέραις εκείναις οι πρόκριτοι του Ιούδα έπεμπον συνεχώς τας επιστολάς αυτών προς τον Τωβίαν, και αι του Τωβία ήρχοντο προς αυτούς.
\par 18 Διότι ήσαν εν τω Ιούδα πολλοί ώρκισμένοι εις αυτόν, επειδή ήτο γαμβρός του Σεχανία, υιού του Αράχ· και Ιωανάν ο υιός αυτού είχε λάβει την θυγατέρα του Μεσουλλάμ, υιού του Βαραχίου.
\par 19 Μάλιστα διηγούντο ενώπιόν μου τας αγαθοεργίας αυτού, και ανέφερον προς αυτόν τους λόγους μου. Και ο Τωβίας έστελλεν επιστολάς διά να με φοβερίζη.

\chapter{7}

\par Αφού δε το τείχος εκτίσθη, και έστησα τας θύρας, και διωρίσθησαν οι πυλωροί και οι ψαλτωδοί και οι Λευΐται,
\par 2 προσέταξα περί της Ιερουσαλήμ τον αδελφόν μου Ανανί και τον Ανανίαν τον άρχοντα του φρουρίου· διότι ήτο ως άνθρωπος πιστός και φοβούμενος τον Θεόν, υπέρ πολλούς.
\par 3 Και είπα προς αυτούς, Ας μη ανοίγωνται αι πύλαι της Ιερουσαλήμ εωσού θερμάνη ο ήλιος· και εκείνων έτι παρόντων, να κλείωνται αι θύραι και να ασφαλίζωνται και φυλακαί να διορίζωνται εκ των κατοίκων της Ιερουσαλήμ, έκαστος εν τη φυλακή αυτού και έκαστος απέναντι της οικίας αυτού.
\par 4 Και η πόλις ήτο ευρύχωρος και μεγάλη, ο δε λαός ολίγος εν αυτή, και οικίαι δεν ήσαν ωκοδομημέναι.
\par 5 Και έβαλεν ο Θεός μου εν τη καρδία μου να συνάξω τους προκρίτους και τους προεστώτας και τον λαόν, διά να αριθμηθώσι κατά γενεαλογίαν. Και εύρηκα βιβλίον της γενεαλογίας εκείνων, οίτινες ανέβησαν κατ' αρχάς και εύρηκα γεγραμμένον εν αυτώ.
\par 6 Ούτοι είναι οι άνθρωποι της επαρχίας, οι αναβάντες εκ της αιχμαλωσίας, εκ των μετοικισθέντων, τους οποίους μετώκισε Ναβουχοδονόσορ ο βασιλεύς της Βαβυλώνος, και επιστρέψαντες εις Ιερουσαλήμ και εις την Ιουδαίαν, έκαστος εις την πόλιν αυτού·
\par 7 οι ελθόντες μετά Ζοροβάβελ, Ιησού, Νεεμία, Αζαρία, Ρααμία, Νααμανί, Μαροδοχαίου, Βιλσάν, Μισπερέθ, Βιγουαί, Νεούμ, Βαανά. Αριθμός των ανδρών του λαού Ισραήλ·
\par 8 υιοί Φαρώς, δισχίλιοι εκατόν εβδομήκοντα δύο.
\par 9 Υιοί Σεφατία, τριακόσιοι εβδομήκοντα δύο.
\par 10 Υιοί Αράχ, εξακόσιοι πεντήκοντα δύο.
\par 11 Υιοί Φαάθ-μωάβ, εκ των υιών Ιησού και Ιωάβ, δισχίλιοι και οκτακόσιοι δεκαοκτώ.
\par 12 Υιοί Ελάμ, χίλιοι διακόσιοι πεντήκοντα τέσσαρες.
\par 13 Υιοί Ζατθού, οκτακόσιοι τεσσαράκοντα πέντε.
\par 14 Υιοί Ζακχαί, επτακόσιοι εξήκοντα.
\par 15 Υιοί Βιννουΐ, εξακόσιοι τεσσαράκοντα οκτώ.
\par 16 Υιοί Βηβαΐ, εξακόσιοι εικοσιοκτώ.
\par 17 Υιοί Αζγάδ, δισχίλιοι τριακόσιοι εικοσιδύο.
\par 18 Υιοί Αδωνικάμ, εξακόσιοι εξήκοντα επτά.
\par 19 Υιοί Βιγουαί, δισχίλιοι εξήκοντα επτά.
\par 20 Υιοί Αδίν, εξακόσιοι πεντήκοντα πέντε.
\par 21 Υιοί Ατήρ εκ του Εζεκίου, ενενήκοντα οκτώ.
\par 22 Υιοί Ασούμ, τριακόσιοι εικοσιοκτώ.
\par 23 Υιοί Βησαί, τριακόσιοι εικοσιτέσσαρες.
\par 24 Υιοί Αρίφ, εκατόν δώδεκα.
\par 25 Υιοί Γαβαών, ενενήκοντα πέντε.
\par 26 Άνδρες Βηθλεέμ και Νετωφά, εκατόν ογδοήκοντα οκτώ.
\par 27 Άνδρες Αναθώθ, εκατόν εικοσιοκτώ.
\par 28 Άνδρες Βαιθ-ασμαβέθ, τεσσαράκοντα δύο.
\par 29 Άνδρες Κιριάθ-ιαρείμ, Χεφειρά, και Βηρώθ, επτακόσιοι τεσσαράκοντα τρεις.
\par 30 Άνδρες Ραμά και Γαβαά, εξακόσιοι είκοσι και εις.
\par 31 Άνδρες Μιχμάς, εκατόν εικοσιδύο.
\par 32 Άνδρες Βαιθήλ, και Γαί, εκατόν εικοσιτρείς.
\par 33 Άνδρες της άλλης Νεβώ, πεντήκοντα δύο.
\par 34 Υιοί του άλλου Ελάμ, χίλιοι διακόσιοι πεντήκοντα τέσσαρες.
\par 35 Υιοί Χαρήμ, τριακόσιοι είκοσι.
\par 36 Υιοί Ιεριχώ, τριακόσιοι τεσσαράκοντα πέντε.
\par 37 Υιοί Λωδ, Αδίδ, και Ωνώ, επτακόσιοι είκοσι και εις.
\par 38 Υιοί Σεναά, τρισχίλιοι εννεακόσιοι τριάκοντα.
\par 39 Οι ιερείς· υιοί Ιεδαΐα, εκ του οίκου Ιησού, εννεακόσιοι εβδομήκοντα τρεις.
\par 40 Υιοί Ιμμήρ, χίλιοι πεντήκοντα δύο.
\par 41 Υιοί Πασχώρ, χίλιοι διακόσιοι τεσσαράκοντα επτά.
\par 42 Υιοί Χαρήμ, χίλιοι δεκαεπτά.
\par 43 Οι Λευΐται· υιοί Ιησού εκ του Καδμιήλ, εκ των υιών Ωδαυΐα, εβδομήκοντα τέσσαρες.
\par 44 Οι ψαλτωδοί· υιοί Ασάφ, εκατόν τεσσαράκοντα οκτώ.
\par 45 Οι πυλωροί· υιοί Σαλλούμ, υιοί Ατήρ, υιοί Ταλμών, υιοί Ακκούβ, υιοί Ατιτά, υιοί Σωβαί, εκατόν τριάκοντα οκτώ.
\par 46 Οι Νεθινείμ· υιοί Σιχά, υιοί Ασουφά, υιοί Ταββαώθ,
\par 47 υιοί Κηρώς, υιοί Σιαά, υιοί Φαδών,
\par 48 υιοί Λεβανά, υιοί Αγαβά, υιοί Σαλμαί,
\par 49 υιοί Ανάν, υιοί Γιδδήλ, υιοί Γαάρ,
\par 50 υιοί Ρεαΐα, υιοί Ρεσίν, υιοί Νεκωδά,
\par 51 υιοί Γαζάμ, υιοί Ουζά, υιοί Φασεά,
\par 52 υιοί Βησαί, υιοί Μεουνείμ, υιοί Ναφουσεσείμ,
\par 53 υιοί Βακβούκ, υιοί Ακουφά, υιοί Αρούρ,
\par 54 υιοί Βασλίθ, υιοί Μεϊδά, υιοί Αρσά,
\par 55 υιοί Βαρκώς, υιοί Σισάρα, υιοί Θαμά,
\par 56 υιοί Νεσιά, υιοί Ατιφά.
\par 57 Οι υιοί των δούλων του Σολομώντος· υιοί Σωταΐ, υιοί Σωφερέθ, υιοί Φερειδά,
\par 58 υιοί Ιααλά, υιοί Δαρκών, υιοί Γιδδήλ,
\par 59 υιοί Σεφατία, υιοί Αττίλ, υιοί Φοχερέθ από Σεβαΐμ, υιοί Αμών.
\par 60 Πάντες οι Νεθινείμ, και οι υιοί των δούλων του Σολομώντος, ήσαν τριακόσιοι ενενήκοντα δύο.
\par 61 Ούτοι δε ήσαν οι αναβάντες από Θελ-μελάχ, Θελ-αρησά, Χερούβ, Αδδών, και Ιμμήρ· δεν ηδύναντο όμως να δείξωσι τον οίκον της πατριάς αυτών και το σπέρμα αυτών, αν ήσαν εκ του Ισραήλ·
\par 62 Υιοί Δαλαΐα, υιοί Τωβία, υιοί Νεκωδά, εξακόσιοι τεσσαράκοντα δύο.
\par 63 Και εκ των ιερέων· υιοί Αβαΐα, υιοί Ακκώς, υιοί Βαρζελλαΐ, όστις έλαβε γυναίκα εκ των θυγατέρων Βαρζελλαΐ του Γαλααδίτου και ωνομάσθη κατά το όνομα αυτών.
\par 64 Ούτοι εζήτησαν την καταγραφήν αυτών μεταξύ των απαριθμηθέντων κατά γενεαλογίαν, και δεν ευρέθη· όθεν εξεβλήθησαν από της ιερατείας.
\par 65 Και είπε προς αυτούς ο Θιρσαθά, να μη φάγωσιν από των αγιωτάτων πραγμάτων, εωσού αναστηθή ιερεύς μετά Ουρίμ και Θουμμίμ.
\par 66 Πάσα η σύναξις ομού ήσαν τεσσαράκοντα δύο χιλιάδες τριακόσιοι εξήκοντα,
\par 67 εκτός των δούλων αυτών και των θεραπαινίδων αυτών, οίτινες ήσαν επτακισχίλιοι τριακόσιοι τριάκοντα επτά· και πλην τούτων διακόσιοι τεσσαράκοντα πέντε ψαλτωδοί και ψάλτριαι.
\par 68 Οι ίπποι αυτών, επτακόσιοι τριάκοντα έξ· αι ημίονοι αυτών, διακόσιαι τεσσαράκοντα πέντε·
\par 69 αι κάμηλοι, τετρακόσιαι τριάκοντα πέντε· αι όνοι, εξακισχίλιαι επτακόσιαι είκοσι.
\par 70 Και τινές εκ των αρχηγών των πατριών έδωκαν διά το έργον. Ο Θιρσαθά έδωκεν εις το θησαυροφυλάκιον χιλίας δραχμάς χρυσίου, πεντήκοντα φιάλας, πεντακοσίους τριάκοντα ιερατικούς χιτώνας.
\par 71 Και τινές εκ των αρχηγών των πατριών έδωκαν εις το θησαυροφυλάκιον του έργου είκοσι χιλιάδας δραχμάς χρυσίου και δύο χιλιάδας διακοσίας μνας αργυρίου.
\par 72 Και το δοθέν από του επιλοίπου λαού ήτο είκοσι χιλιάδες δραχμαί χρυσίου, και δισχίλιαι μναι αργυρίου, και εξήκοντα επτά ιερατικοί χιτώνες.
\par 73 Ούτως οι ιερείς και οι Λευΐται και οι πυλωροί και οι ψαλτωδοί και μέρος εκ του λαού και οι Νεθινείμ και πας ο Ισραήλ, κατώκησαν εν ταις πόλεσιν αυτών. Και ότε έφθασεν ο έβδομος μην, οι υιοί Ισραήλ ήσαν εν ταις πόλεσιν αυτών.

\chapter{8}

\par Συνήχθη δε πας ο λαός, ως εις άνθρωπος, εις την πλατείαν την έμπροσθεν της πύλης των υδάτων· και είπον προς Έσδραν τον γραμματέα, να φέρη το βιβλίον του νόμου του Μωϋσέως, τον οποίον ο Κύριος προσέταξεν εις τον Ισραήλ.
\par 2 Και την πρώτην ημέραν του εβδόμου μηνός έφερεν Έσδρας ο ιερεύς τον νόμον έμπροσθεν της συνάξεως ανδρών τε και γυναικών και πάντων των δυναμένων να εννοώσιν ακούοντες.
\par 3 Και ανέγνωσεν εν αυτώ, εν τη πλατεία τη έμπροσθεν της πύλης των υδάτων, από της αυγής μέχρι της μεσημβρίας, ενώπιον των ανδρών και των γυναικών και των δυναμένων να εννοώσι· και τα ώτα παντός του λαού προσείχον εις το βιβλίον του νόμου.
\par 4 Ίστατο δε Έσδρας ο γραμματεύς επί βήματος ξυλίνου, το οποίον έκαμον επίτηδες· και πλησίον αυτού ίστατο Ματταθίας και Σεμά και Αναΐας και Ουρίας και Χελκίας και Μαασίας, εκ δεξιών αυτού· εξ αριστερών δε αυτού Φεδαΐας και Μισαήλ και Μαλχίας και Ασούμ και Ασβαδανά, Ζαχαρίας και Μεσουλλάμ.
\par 5 Και ήνοιξεν ο Έσδρας το βιβλίον ενώπιον παντός του λαού· διότι ήτο υπεράνω παντός του λαού· και ότε ήνοιξεν αυτό, πας ο λαός ηγέρθη.
\par 6 Και ηυλόγησεν ο Έσδρας τον Κύριον, τον Θεόν τον μέγαν. Και πας ο λαός απεκρίθη, Αμήν, Αμήν, υψόνοντες τας χείρας αυτών· και κύψαντες, προσεκύνησαν τον Κύριον με τα πρόσωπα επί την γην.
\par 7 Ιησούς δε και Βανί και Σερεβίας, Ιαμείν, Ακκούβ, Σαββεθαΐ, Ωδίας, Μαασίας, Κελιτά, Αζαρίας, Ιωζαβάδ, Ανάν, Φελαΐας και οι Λευΐται εξήγουν τον νόμον εις τον λαόν· και ο λαός ίστατο εν τω τόπω αυτού.
\par 8 Και ανέγνωσαν εν τω βιβλίω του νόμου του Θεού ευκρινώς, και έδωκαν την έννοιαν και εξήγησαν τα αναγινωσκόμενα.
\par 9 Και ο Νεεμίας, ούτος είναι ο Θιρσαθά, και Έσδρας ο ιερεύς ο γραμματεύς, και οι Λευΐται οι εξηγούντες εις τον λαόν, είπον προς πάντα τον λαόν, Η ημέρα αύτη είναι αγία εις Κύριον τον Θεόν σας· μη πενθείτε μηδέ κλαίετε. Διότι πας ο λαός έκλαιεν, ως ήκουσαν τους λόγους του νόμου.
\par 10 Και είπε προς αυτούς, Υπάγετε, φάγετε παχέα και πίετε γλυκάσματα, και αποστείλατε μερίδας προς τους μη έχοντας μηδέν ητοιμασμένον· διότι η ημέρα είναι αγία εις τον Κύριον ημών· και μη λυπείσθε· διότι η χαρά του Κυρίου είναι η ισχύς σας.
\par 11 Και κατεσίγασαν οι Λευΐται πάντα τον λαόν, λέγοντες, Ησυχάζετε· διότι η ημέρα είναι αγία· και μη λυπείσθε.
\par 12 Και απήλθε πας ο λαός, διά να φάγωσι και να πίωσι και να αποστείλωσι μερίδας και να κάμωσιν ευφροσύνην μεγάλην, διότι ενόησαν τους λόγους τους οποίους εφανέρωσαν εις αυτούς.
\par 13 Και την δευτέραν ημέραν συνήχθησαν οι άρχοντες των πατριών παντός του λαού, οι ιερείς και οι Λευΐται, προς Έσδραν τον γραμματέα, διά να διδαχθώσι τους λόγους του νόμου.
\par 14 Και εύρηκαν γεγραμμένον εν τω νόμω, τον οποίον προσέταξεν ο Κύριος διά του Μωϋσέως, να κατοικήσωσιν οι υιοί Ισραήλ εν σκηναίς εν τη εορτή του εβδόμου μηνός·
\par 15 και να δημοσιεύσωσι και να διακηρύξωσιν εις πάσας τας πόλεις αυτών και εις την Ιερουσαλήμ, λέγοντες, Εξέλθετε εις το όρος και φέρετε κλάδους ελαίας και κλάδους αγριελαίας και κλάδους μυρσίνης και κλάδους φοινίκων και κλάδους δασυφύλλων δένδρων, διά να κάμητε σκηνάς, κατά το γεγραμμένον.
\par 16 Και εξελθών ο λαός έφερε, και έκαμον εις εαυτούς σκηνάς, έκαστος επί του δώματος αυτού, και εν ταις αυλαίς αυτών και εν ταις αυλαίς του οίκου του Θεού και εν τη πλατεία της πύλης των υδάτων και εν τη πλατεία της πύλης του Εφραΐμ.
\par 17 Και πάσα η σύναξις των επιστρεψάντων από της αιχμαλωσίας έκαμον σκηνάς, και εκάθησαν εν ταις σκηναίς· διότι από των ημερών Ιησού υιού του Ναυή μέχρι εκείνης της ημέρας, οι υιοί Ισραήλ δεν είχον κάμει ούτω. Και έγεινεν ευφροσύνη μεγάλη σφόδρα.
\par 18 Και καθ' εκάστην ημέραν, από της πρώτης ημέρας μέχρι της τελευταίας ημέρας, ανεγίνωσκεν εν τω βιβλίω του νόμου του Θεού. Και έκαμον εορτήν επτά ημέρας· την δε ογδόην ημέραν πάνδημον σύναξιν, κατά το διατεταγμένον.

\chapter{9}

\par Και εν τη εικοστή τετάρτη ημέρα τούτου του μηνός συνήχθησαν οι υιοί Ισραήλ με νηστείαν και με σάκκους και με χώμα εφ' εαυτούς.
\par 2 Και εχωρίσθη το σπέρμα του Ισραήλ από πάντων των ξένων· και σταθέντες εξωμολογήθησαν τας αμαρτίας αυτών και τας ανομίας των πατέρων αυτών.
\par 3 Και σταθέντες εν τω τόπω αυτών, ανέγνωσαν εν τω βιβλίω του νόμου Κυρίου του Θεού αυτών, εν τέταρτον της ημέρας· και εν τέταρτον, εξωμολογούντο και προσεκύνουν Κύριον τον Θεόν αυτών.
\par 4 Τότε εσηκώθη επί το βήμα των Λευϊτών Ιησούς και Βανί, Καδμιήλ, Σεβανίας, Βουννί, Σερεβίας, Βανί και Χανανί, και ανεβόησαν μετά φωνής μεγάλης προς Κύριον τον Θεόν αυτών.
\par 5 Και οι Λευΐται, Ιησούς και Καδμιήλ, Βανί, Ασαβνίας, Σερεβίας, Ωδίας, Σεβανίας και Πεθαΐα, είπον, Σηκώθητε, ευλογήσατε Κύριον τον Θεόν υμών από του αιώνος έως του αιώνος· και ας ήναι, Θεέ, ευλογημένον το ένδοξόν σου όνομα, το υπέρτερον πάσης ευλογίας και αινέσεως.
\par 6 Συ αυτός είσαι Κύριος μόνος· συ εποίησας τον ουρανόν, τους ουρανούς των ουρανών, και πάσαν την στρατιάν αυτών, την γην και πάντα τα επ' αυτής, τας θαλάσσας και πάντα τα εν αυταίς, και συ ζωοποιείς πάντα ταύτα· και σε προσκυνούσιν αι στρατιαί των ουρανών.
\par 7 Συ είσαι Κύριος ο Θεός, όστις εξέλεξας τον Άβραμ και εξήγαγες αυτόν από της Ουρ των Χαλδαίων, και έδωκας εις αυτόν το όνομα Αβραάμ·
\par 8 και εύρηκας την καρδίαν αυτού πιστήν ενώπιόν σου, και έκαμες προς αυτόν διαθήκην, ότι θέλεις δώσει την γην των Χαναναίων, των Χετταίων, των Αμορραίων και των Φερεζαίων και των Ιεβουσαίων και των Γεργεσαίων, ότι θέλεις δώσει αυτήν εις το σπέρμα αυτού· και εξετέλεσας τους λόγους σου· διότι δίκαιος είσαι συ.
\par 9 Και είδες την θλίψιν των πατέρων ημών εν Αιγύπτω, και ήκουσας την κραυγήν αυτών επί την Ερυθράν θάλασσαν·
\par 10 και έδειξας σημεία και τέρατα επί τον Φαραώ και επί πάντας τους δούλους αυτού και επί πάντα τον λαόν της γης αυτού· επειδή εγνώρισας ότι υπερηφανεύθησαν εναντίον αυτών. Και έκαμες εις σεαυτόν όνομα, ως την ημέραν ταύτην.
\par 11 Και διέσχισας την θάλασσαν ενώπιον αυτών, και διέβησαν διά ξηράς εν μέσω της θαλάσσης· τους δε καταδιώκοντας αυτούς έρριψας εις τα βάθη, ως λίθον εις ύδατα ισχυρά·
\par 12 και ώδήγησας αυτούς την ημέραν διά στύλου νεφέλης, την δε νύκτα διά στύλου πυρός, διά να φωτίζης εις αυτούς την οδόν δι' ης έμελλον να διέλθωσι.
\par 13 Και κατέβης επί το όρος Σινά, και ελάλησας μετ' αυτών εξ ουρανού, και έδωκας εις αυτούς ευθείας κρίσεις και αληθινούς νόμους, διατάγματα και εντολάς αγαθάς·
\par 14 και το άγιόν σου σάββατον έκαμες γνωστόν εις αυτούς, και προσέταξας εις αυτούς εντολάς και διατάγματα και νόμους, διά χειρός Μωϋσέως του δούλου σου.
\par 15 Και άρτον εξ ουρανού έδωκας εις αυτούς εις την πείναν αυτών, και ύδωρ εκ πέτρας εξήγαγες εις αυτούς εις την δίψαν αυτών· και είπας προς αυτούς να εισέλθωσι διά να κληρονομήσωσι την γην, περί ης ύψωσας την χείρα σου ότι θέλεις δώσει αυτήν εις αυτούς.
\par 16 Εκείνοι δε και οι πατέρες ημών υπερηφανεύθησαν και εσκλήρυναν τον τράχηλον αυτών και δεν υπήκουσαν εις τας εντολάς σου·
\par 17 και ηρνήθησαν να υπακούσωσι και δεν ενεθυμήθησαν τα θαυμάσιά σου τα οποία έκαμες εις αυτούς· αλλ' εσκλήρυναν τον τράχηλον αυτών, και εν τη αποστασία αυτών διώρισαν αρχηγόν διά να επιστρέψωσιν εις την δουλείαν αυτών. Αλλά συ είσαι Θεός συγχωρητικός, ελεήμων και οικτίρμων, μακρόθυμος και πολυέλεος, και δεν εγκατέλιπες αυτούς.
\par 18 Μάλιστα, ότε έκαμον εις εαυτούς χωνευτόν μόσχον και είπον, Ούτος είναι ο Θεός σου όστις σε ανήγαγεν εξ Αιγύπτου, και έπραξαν μεγάλους παροργισμούς·
\par 19 συ όμως, εν τοις οικτιρμοίς σου τοις μεγάλοις, δεν εγκατέλιπες αυτούς εν τη ερήμω· ο στύλος της νεφέλης δεν εξέκλινεν απ' αυτών την ημέραν, διά να οδηγή αυτούς εν τη οδώ, ουδέ ο στύλος του πυρός την νύκτα, διά να φωτίζη εις αυτούς και την οδόν δι' ης έμελλον να διέλθωσι.
\par 20 Και έδωκας εις αυτούς το αγαθόν σου πνεύμα, διά να συνετίζη αυτούς· και δεν εστέρησας το μάννα σου από του στόματος αυτών, και ύδωρ έδωκας εις αυτούς εις την δίψαν αυτών.
\par 21 Και τεσσαράκοντα έτη έθρεψας αυτούς εν τη ερήμω· δεν έλειψεν εις αυτούς ουδέν· τα ιμάτια αυτών δεν επαλαιώθησαν και οι πόδες αυτών δεν επρήσθησαν.
\par 22 Και έδωκας εις αυτούς βασίλεια και λαούς, και διεμέρισας εις αυτούς διά μερίδας· και εκληρονόμησαν την γην του Σηών και την γην του βασιλέως της Εσεβών και την γην του Ωγ βασιλέως της Βασάν.
\par 23 Και τους υιούς αυτών επλήθυνας ως τα άστρα του ουρανού· και έφερες αυτούς εις την γην, εις την οποίαν είπας προς τους πατέρας αυτών να εισέλθωσι, διά να κληρονομήσωσιν αυτήν.
\par 24 Και εισήλθον οι υιοί αυτών και εκληρονόμησαν την γήν· και υπέταξας έμπροσθεν αυτών τους κατοίκους της γης, τους Χαναναίους, και παρέδωκας αυτούς εις τας χείρας αυτών, και τους βασιλείς αυτών και τους λαούς της γης, διά να κάμωσιν εις αυτούς κατά την θέλησιν αυτών.
\par 25 Και εκυρίευσαν πόλεις ισχυράς και γην παχείαν, και εκληρονόμησαν οίκους πλήρεις πάντων των αγαθών, φρέατα ωρυγμένα, αμπελώνας και ελαιώνας και δένδρα κάρπιμα εν αφθονία· και έφαγον και εχορτάσθησαν και επαχύνθησαν και ενετρύφησαν, εν τη μεγάλη σου αγαθότητι.
\par 26 Και ηπείθησαν και επανεστάτησαν εναντίον σου, και έρριψαν τον νόμον σου οπίσω των νώτων αυτών, και τους προφήτας σου εφόνευσαν, οίτινες διεμαρτύροντο εναντίον αυτών διά να επιστρέψωσιν αυτούς προς σε, και έπραξαν μεγάλους παροργισμούς.
\par 27 Διά τούτο παρέδωκας αυτούς εις την χείρα των θλιβόντων αυτούς, και κατέθλιψαν αυτούς· και εν τω καιρώ της θλίψεως αυτών ανεβόησαν προς σε, και συ εισήκουσας εξ ουρανού· και κατά τους πολλούς οικτιρμούς σου έδωκας σωτήρας εις αυτούς, και έσωσαν αυτούς εκ της χειρός των θλιβόντων αυτούς.
\par 28 Αλλ' αφού ανεπαύθησαν, εστράφησαν εις το να πράττωσι πονηρά ενώπιόν σου· όθεν εγκατέλιπες αυτούς εις την χείρα των εχθρών αυτών, και εξουσίασαν αυτούς· ότε όμως επέστρεψαν και ανεβόησαν προς σε, συ εισήκουσας εξ ουρανού· και πολλάκις ηλευθέρωσας αυτούς κατά τους οικτιρμούς σου.
\par 29 Και διεμαρτυρήθης εναντίον αυτών, διά να επιστρέψης αυτούς εις τον νόμον σου· πλην αυτοί υπερηφανεύθησαν και δεν υπήκουσαν εις τας εντολάς σου, αλλ' ημάρτησαν εις τας κρίσεις σου, τας οποίας εάν τις εκτελή, θέλει ζήσει δι' αυτών· και έδωκαν νώτον απειθή και εσκλήρυναν τον τράχηλον αυτών και δεν ήκουσαν.
\par 30 Και όμως έτη πολλά παρέκτεινας επ' αυτούς, και διεμαρτυρήθης εναντίον αυτών διά του πνεύματός σου διά των προφητών σου· αλλά δεν έδωκαν ακρόασιν· διά τούτο παρέδωκας αυτούς εις την χείρα των λαών των τόπων.
\par 31 Πλην διά τους πολλούς οικτιρμούς σου δεν συνετέλεσας αυτούς, ουδέ εγκατέλιπες αυτούς· διότι Θεός οικτίρμων και ελεήμων είσαι.
\par 32 Τώρα λοιπόν, Θεέ ημών, ο μέγας, ο ισχυρός και φοβερός Θεός, ο φυλάττων την διαθήκην και το έλεος, ας μη λογισθή μικρά ενώπιόν σου πάσα η θλίψις ήτις εύρηκεν ημάς, τους βασιλείς ημών, τους άρχοντας ημών και τους ιερείς ημών και τους προφήτας ημών και τους πατέρας ημών και πάντα τον λαόν σου, από των ημερών των βασιλέων της Ασσυρίας μέχρι της ημέρας ταύτης.
\par 33 Δίκαιος βεβαίως είσαι εις πάντα τα επελθόντα εφ' ημάς· διότι συ μεν αλήθειαν έκαμες, ημείς δε ησεβήσαμεν.
\par 34 Και οι βασιλείς ημών, οι άρχοντες ημών, οι ιερείς ημών και οι πατέρες ημών, δεν εφύλαξαν τον νόμον σου και δεν έδωκαν προσοχήν εις τας εντολάς σου και εις τα μαρτύριά σου, με τα οποία διεμαρτυρήθης εναντίον αυτών.
\par 35 Διότι αυτοί, εν τη βασιλεία αυτών και εν τη μεγάλη σου αγαθωσύνη την οποίαν έδωκας εις αυτούς, και εν τη γη τη πλατεία και παχεία, την οποίαν έδωκας ενώπιον αυτών, δεν σε εδούλευσαν ουδέ εστράφησαν από των πονηρών έργων αυτών.
\par 36 Ιδού, δούλοι είμεθα την ημέραν ταύτην· και εν τη γη, την οποίαν έδωκας εις τους πατέρας ημών, διά να τρώγωσι τον καρπόν αυτής και τα αγαθά αυτής, ιδού, δούλοι είμεθα επ' αυτής·
\par 37 και αυτή δίδει πολλήν αφθονίαν εις τους βασιλείς, τους οποίους επέβαλες εφ' ημάς διά τας αμαρτίας ημών· και κατεξουσιάζουσιν επί των σωμάτων ημών και επί των κτηνών ημών κατά την αρέσκειαν αυτών· και είμεθα εν θλίψει μεγάλη.
\par 38 Όθεν διά πάντα ταύτα ημείς κάμνομεν διαθήκην πιστήν και γράφομεν αυτήν· και επισφραγίζουσιν αυτήν οι άρχοντες ημών, οι Λευΐται ημών και οι ιερείς ημών.

\chapter{10}

\par Οι δε επισφραγίσαντες ήσαν Νεεμίας ο Θιρσαθά ο υιός του Αχαλία, και Σεδεκίας,
\par 2 Σεραΐας, Αζαρίας, Ιερεμίας,
\par 3 Πασχώρ, Αμαρίας, Μαλχίας,
\par 4 Χαττούς, Σεβανίας, Μαλλούχ,
\par 5 Χαρήμ, Μερημώθ, Οβαδίας,
\par 6 Δανιήλ, Γιννεθών, Βαρούχ,
\par 7 Μεσσουλλάμ, Αβιά, Μειαμείν,
\par 8 Μααζίας, Βιλγαΐ, Σεμαΐας· ούτοι ήσαν οι ιερείς.
\par 9 Και οι Λευΐται Ιησούς ο υιός του Αζανία, Βιννουΐ εκ των υιών Ηναδάδ, Καδμιήλ·
\par 10 και οι αδελφοί αυτών, Σεβανίας, Ωδίας, Κελιτά, Φελαΐας, Ανάν,
\par 11 Μιχά, Ρεώβ, Ασαβίας,
\par 12 Ζακχούρ, Σερεβίας, Σεβανίας,
\par 13 Ωδίας, Βανί, Βενινού.
\par 14 Οι άρχοντες του λαού· Φαρώς, Φαάθ-μωάβ, Ελάμ, Ζατθού, Βανί,
\par 15 Βουννί, Αζγάδ, Βηβαΐ,
\par 16 Αδωνίας, Βιγουαί, Αδίν,
\par 17 Ατήρ, Εζεκίας, Αζούρ,
\par 18 Ωδίας, Ασούμ, Βησαί,
\par 19 Αρίφ, Αναθώθ, Νεβαΐ,
\par 20 Μαγφίας, Μεσουλλάμ, Εζείρ,
\par 21 Μεσηζαβεήλ, Σαδώκ, Ιαδδουά,
\par 22 Φελατίας, Ανάν, Αναΐας,
\par 23 Ωσηέ, Ανανίας, Ασσούβ,
\par 24 Αλλωής, Φιλεά, Σωβήκ,
\par 25 Ρεούμ, Ασαβνά, Μαασίας,
\par 26 και Αχιά, Ανάν, Γανάν,
\par 27 Μαλλούχ, Χαρήμ, Βαανά.
\par 28 Και το υπόλοιπον του λαού, οι ιερείς, οι Λευΐται, οι πυλωροί, οι ψαλτωδοί, οι Νεθινείμ, και πάντες οι αποχωρισθέντες από των λαών των τόπων προς τον νόμον του Θεού, αι γυναίκες αυτών, οι υιοί αυτών και αι θυγατέρες αυτών, πας εννοών και έχων σύνεσιν,
\par 29 ηνώθησαν μετά των αδελφών αυτών, των προκρίτων αυτών, και εισήλθον εις κατάραν και εις όρκον, να περιπατώσιν εις τον νόμον του Θεού, τον δοθέντα διά χειρός Μωϋσέως του δούλου του Θεού, και να φυλάττωσι και να εκτελώσι πάσας τας εντολάς του Κυρίου, του Κυρίου ημών, και τας κρίσεις αυτού και τα διατάγματα αυτού·
\par 30 και ότι δεν θέλομεν δώσει τας θυγατέρας ημών εις τους λαούς της γης, και τας θυγατέρας αυτών δεν θέλομεν λάβει εις τους υιούς ημών·
\par 31 και, εάν οι λαοί της γης φέρωσιν αγοράσιμα ή οποιασδήποτε τροφάς να πωλήσωσιν εν τη ημέρα του σαββάτου, ότι δεν θέλομεν λάβει ταύτα παρ' αυτών εν σαββάτω και εν ημέρα αγία· και ότι θέλομεν αφήσει το έβδομον έτος και την απαίτησιν παντός χρέους.
\par 32 Διετάξαμεν έτι εις εαυτούς να επιφορτισθώμεν να δίδωμεν κατ' έτος εν τρίτον του σίκλου διά την υπηρεσίαν του οίκου του Θεού ημών,
\par 33 διά τους άρτους της προθέσεως και διά την παντοτεινήν εξ αλφίτων προσφοράν, και διά την παντοτεινήν ολοκαύτωσιν των σαββάτων, των νεομηνιών, διά τας επισήμους εορτάς και διά τα άγια πράγματα και διά τας περί αμαρτίας προσφοράς, διά να κάμνωμεν εξιλέωσιν υπέρ του Ισραήλ, και διά παν το έργον του οίκου του Θεού ημών.
\par 34 Και ερρίψαμεν κλήρους μεταξύ των ιερέων των Λευϊτών και του λαού περί της προσφοράς των ξύλων, διά να φέρωσιν αυτά εις τον οίκον του Θεού ημών, κατά τους οίκους των πατριών ημών, εν ωρισμένοις καιροίς κατ' έτος, διά να καίωσιν επί του θυσιαστηρίου Κυρίου του Θεού ημών, κατά το γεγραμμένον εν τω νόμω·
\par 35 και διά να φέρωμεν τα πρωτογεννήματα της γης ημών και τα πρωτογεννήματα των καρπών παντός δένδρου, κατ' έτος, προς τον οίκον του Κυρίου·
\par 36 και τα πρωτότοκα των υιών ημών και των κτηνών ημών, κατά το γεγραμμένον εν τω νόμω, και τα πρωτότοκα των βοών ημών και των ποιμνίων ημών, να φέρωμεν αυτά εις τον οίκον του Θεού ημών, προς τους ιερείς τους λειτουργούντας εν τω οίκω του Θεού ημών·
\par 37 και να φέρωμεν τας απαρχάς του φυράματος ημών και τας προσφοράς ημών και τους καρπούς παντός δένδρου, οίνου και ελαίου, προς τους ιερείς, εις τα οικήματα του οίκου του Θεού ημών· και τα δέκατα της γης ημών προς τους Λευΐτας, και αυτοί οι Λευΐται να λαμβάνωσι τα δέκατα εν πάσαις ταις πόλεσι της γεωργίας ημών.
\par 38 Και ο ιερεύς ο υιός του Ααρών θέλει είσθαι μετά των Λευϊτών, όταν οι Λευΐται λαμβάνωσι τα δέκατα· και οι Λευΐται θέλουσιν αναφέρει το δέκατον των δεκάτων εις τον οίκον του Θεού ημών, εις τα οικήματα του οίκου του θησαυρού.
\par 39 Διότι οι υιοί Ισραήλ και οι υιοί Λευΐ θέλουσι φέρει τας προσφοράς του σίτου, του οίνου και του ελαίου εις τα οικήματα, όπου είναι τα σκεύη του αγιαστηρίου, και οι ιερείς οι λειτουργούντες και οι πυλωροί και οι ψαλτωδοί· και δεν θέλομεν εγκαταλείψει τον οίκον του Θεού ημών.

\chapter{11}

\par Και κατώκησαν οι άρχοντες του λαού εν Ιερουσαλήμ· και το υπόλοιπον του λαού έρριψαν κλήρους, διά να φέρωσιν ένα εκ των δέκα να κατοικήση εν Ιερουσαλήμ τη αγία πόλει, τα δε εννέα μέρη εν ταις άλλαις πόλεσι.
\par 2 Και ηυλόγησεν ο λαός πάντας τους ανθρώπους, όσοι προσέφεραν αυτοπροαιρέτως εαυτούς να κατοικήσωσιν εν Ιερουσαλήμ.
\par 3 Ούτοι δε είναι οι άρχοντες της επαρχίας οι κατοικήσαντες εν Ιερουσαλήμ· εν δε ταις πόλεσι του Ιούδα κατώκησαν έκαστος εν τη ιδιοκτησία αυτού, εν ταις πόλεσιν αυτών, ο Ισραήλ, οι ιερείς και οι Λευΐται και οι Νεθινείμ και οι υιοί των δούλων του Σολομώντος.
\par 4 Και εν Ιερουσαλήμ κατώκησάν τινές εκ των υιών Ιούδα και εκ των υιών Βενιαμίν· εκ των υιών Ιούδα, Αθαΐας ο υιός του Οζία, υιού Ζαχαρία, υιού Αμαρία, υιού Σεφατία, υιού Μααλελεήλ, εκ των υιών Φαρές·
\par 5 και Μαασίας ο υιός του Βαρούχ, υιού Χολ-οζέ, υιού Αζαΐα, υιού Αδαΐα, υιού Ιωϊαρίβ, υιού Ζαχαρία, υιού του Σηλωνί·
\par 6 πάντες οι υιοί Φαρές οι κατοικήσαντες εν Ιερουσαλήμ ήσαν τετρακόσιοι εξήκοντα οκτώ άνδρες δυνάμεως.
\par 7 Οι δε υιοί Βενιαμίν είναι ούτοι· Σαλλού ο υιός του Μεσουλλάμ, υιού Ιωάδ, υιού Φεδαΐα, υιού Κωλαΐα, υιού Μαασία, υιού Ιθιήλ, υιού Ιεσαΐα·
\par 8 και μετ' αυτών, Γαββαεί, Σαλλαΐ, εννεακόσιοι εικοσιοκτώ·
\par 9 και Ιωήλ ο υιός του Ζιχρί ήτο έφορος αυτών· ο δε Ιούδας, ο υιός του Σενουά, δεύτερος επί την πόλιν·
\par 10 Εκ των ιερέων, Ιεδαΐας ο υιός του Ιωϊαρίβ, Ιαχείν,
\par 11 Σεραΐας ο υιός του Χελκία, υιού Μεσσουλλάμ, υιού Σαδώκ, υιού Μεραϊώθ, υιού Αχιτώβ, ο άρχων του οίκου του Θεού.
\par 12 Και οι αδελφοί αυτών οι ποιούντες το έργον του οίκου ήσαν οκτακόσιοι εικοσιδύο· και Αδαΐας ο υιός του Ιεροάμ, υιού Φελαλία, υιού Αμσί, υιού Ζαχαρία, υιού Πασχώρ, υιού Μαλχία,
\par 13 και οι αδελφοί αυτού, άρχοντες πατριών, διακόσιοι τεσσαράκοντα δύο· και Αμασσαΐ ο υιός του Αζαρεήλ, υιού Ααζαΐ, υιού Μεσιλλεμώθ, υιού Ιμμήρ,
\par 14 και οι αδελφοί αυτών, άνδρες δυνατοί εν ισχύϊ, εκατόν εικοσιοκτώ· έφορος δε αυτών ήτο Ζαβδιήλ, υιός του Γεδωλείμ.
\par 15 Και εκ των Λευϊτών, Σεμαΐας ο υιός του Ασσούβ, υιού Αζρικάμ, υιού Ασαβία, υιού Βουννί·
\par 16 και Σαββεθαΐ και Ιωζαβάδ, εκ των αρχόντων των Λευϊτών, ήσαν επί των εξωτερικών έργων του οίκου του Θεού.
\par 17 Και Ματθανίας ο υιός του Μιχά, υιού Ζαβδί, υιού Ασάφ, ήτο ο εξάρχων της υμνωδίας εν τη προσευχή· και Βακβουκίας ο δεύτερος μεταξύ των αδελφών αυτού, και Αβδά ο υιός του Σαμμουά, υιού Ταλάλ, υιού Ιεδουθούν.
\par 18 Πάντες οι Λευΐται εν τη αγία πόλει ήσαν διακόσιοι ογδοήκοντα τέσσαρες.
\par 19 Οι δε πυλωροί, Ακκούβ, Ταλμών, και οι αδελφοί αυτών οι φυλάττοντες εν ταις πύλαις, ήσαν εκατόν εβδομήκοντα δύο.
\par 20 Και το υπόλοιπον του Ισραήλ, οι ιερείς και Λευΐται, ήσαν εν πάσαις ταις πόλεσιν Ιούδα, έκαστος εν τη κληρονομία αυτού.
\par 21 Οι δε Νεθινείμ κατώκησαν εν Οφήλ· και ο Σιχά και ο Γισπά ήσαν επί των Νεθινείμ.
\par 22 Και ο έφορος των Λευΐτών εν Ιερουσαλήμ ήτο Οζί, ο υιός του Βανί, υιού Ασαβία, υιού Ματθανία, υιού Μιχά. Εκ των υιών του Ασάφ, οι ψαλτωδοί ήσαν επί του έργου του οίκου του Θεού.
\par 23 Διότι ήτο προσταγή του βασιλέως περί αυτών, και διατεταγμένον μερίδιον διά τους ψαλτωδούς κατά πάσαν ημέραν.
\par 24 Και Πεθαΐα ο υιός του Μεσηζαβεήλ, εκ των υιών του Ζερά υιού του Ιούδα, ήτο επίτροπος του βασιλέως εν πάση υποθέσει περί του λαού.
\par 25 Περί δε των χωρίων, μετά των αγρών αυτών, τινές εκ των υιών Ιούδα κατώκησαν εν Κιριάθ-αρβά και ταις κώμαις αυτής, και εν Δαιβών και ταις κώμαις αυτής, και εν Ιεκαβσεήλ και τοις χωρίοις αυτής,
\par 26 και εν Ιησουά και εν Μωλαδά και εν Βαιθ-φελέτ,
\par 27 και εν Ασάρ-σουάλ και εν Βηρ-σαβεέ και ταις κώμαις αυτής,
\par 28 και εν Σικλάγ και εν Μεκονά και εν ταις κώμαις αυτής,
\par 29 και εν Εν-ριμμών και εν Σαρεά και εν Ιαρμούθ,
\par 30 Ζανωά, Οδολλάμ και τοις χωρίοις αυτών, Λαχείς και τοις αγροίς αυτής, Αζηκά και ταις κώμαις αυτής. Και κατώκησαν από Βηρ-σαβεέ έως της φάραγγος Εννόμ.
\par 31 Και οι υιοί Βενιαμίν κατώκησαν από Γεβά εν Μιχμάς και Αιϊά και Βαιθήλ και ταις κώμαις αυτής,
\par 32 εν Αναθώθ, Νωβ, Ανανία,
\par 33 Ασώρ, Ραμά, Γιτθαΐμ,
\par 34 Αδίδ, Σεβωείμ, Νεβαλλάτ,
\par 35 Λωδ, και Ωνώ, τη φάραγγι των τεκτόνων.
\par 36 Και εκ των Λευϊτών κατώκησαν διαιρέσεις εν Ιούδα και Βενιαμίν.

\chapter{12}

\par Ούτοι δε ήσαν οι ιερείς και οι Λευΐται, οι αναβάντες μετά του Ζοροβάβελ υιού του Σαλαθιήλ, και του Ιησού· Σεραΐας, Ιερεμίας, Έσδρας,
\par 2 Αμαρίας, Μαλλούχ, Χαττούς,
\par 3 Σεχανίας, Ρεούμ, Μερημώθ,
\par 4 Ιδδώ, Γιννεθώ, Αβιά,
\par 5 Μιαμείν, Μααδίας, Βιλγά,
\par 6 Σεμαΐας και Ιωϊαρίβ, Ιεδαΐας,
\par 7 Σαλλού, Αμώκ, Χελχίας, Ιεδαΐας. Ούτοι ήσαν οι αρχηγοί των ιερέων και των αδελφών αυτών εν ταις ημέραις Ιησού.
\par 8 Οι δε Λευΐται, Ιησούς, Βιννουΐ, Καδμιήλ, Σερεβίας, Ιούδας και Ματθανίας, ο επί των ύμνων, αυτός και οι αδελφοί αυτού.
\par 9 Ο δε Βακβουκίας και ο Ουννί, οι αδελφοί αυτών, ήσαν απέναντι αυτών διά τας φυλακάς.
\par 10 Και ο Ιησούς εγέννησε τον Ιωακείμ, Ιωακείμ δε εγέννησε τον Ελιασείβ, Ελιασείβ δε εγέννησε τον Ιωαδά,
\par 11 Ιωαδά δε εγέννησε τον Ιωνάθαν· Ιωνάθαν δε εγέννησε τον Ιαδδουά.
\par 12 Και εν ταις ημέραις του Ιωακείμ, ιερείς, άρχοντες πατριών, ήσαν του Σεραΐα, ο Μεραΐας· του Ιερεμία, ο Ανανίας·
\par 13 του Έσδρα, ο Μεσουλλάμ· του Αμαρία, ο Ιωανάν·
\par 14 του Μελιχού, ο Ιωνάθαν· του Σεβανία, ο Ιωσήφ·
\par 15 του Χαρήμ, ο Αδνά· του Μεραϊώθ, ο Ελκαΐ·
\par 16 του Ιδδώ, ο Ζαχαρίας· του Γιννεθών, ο Μεσουλλάμ·
\par 17 του Αβιά, ο Ζιχρί· του Μινιαμείν, και του Μωαδία, ο Φιλταΐ·
\par 18 του Βιλγά, ο Σαμμουά· του Σεμαΐα, ο Ιωνάθαν·
\par 19 και του Ιωϊαρίβ, ο Ματθεναΐ· του Ιεδαΐα, ο Οζί·
\par 20 του Σαλλαΐ, ο Καλλαΐ· του Αμώκ, ο Εβερ·
\par 21 του Χελκία, ο Ασαβίας· του Ιεδαΐα, ο Ναθαναήλ.
\par 22 Οι Λευΐται εν ταις ημέραις του Ελιασείβ, Ιωαδά και Ιωανάν και Ιαδδουά, ήσαν καταγεγραμμένοι άρχοντες πατριών· και οι ιερείς, επί της βασιλείας Δαρείου του Πέρσου.
\par 23 Οι υιοί του Λευΐ, άρχοντες των πατριών, ήσαν καταγεγραμμένοι εν τω βιβλίω των Χρονικών, μάλιστα έως των ημερών του Ιωανάν υιού του Ελιασείβ.
\par 24 Και οι άρχοντες των Λευϊτών, Ασαβίας, Σερεβίας, και Ιησούς ο υιός του Καδμιήλ, και οι αδελφοί αυτών απέναντι αυτών, διά να αινώσι και να υμνώσι κατά την προσταγήν Δαβίδ του ανθρώπου του Θεού, φυλακή απέναντι φυλακής.
\par 25 Ο Ματθανίας και Βακβουκίας, Οβαδία, Μεσουλλάμ, Ταλμών, Ακκούβ, ήσαν πυλωροί φυλάττοντες την φυλακήν εν τοις ταμείοις των πυλών.
\par 26 Ούτοι ήσαν εν ταις ημέραις του Ιωακείμ, υιού Ιησού, υιού Ιωσεδέχ, και εν ταις ημέραις Νεεμία του κυβερνήτου και του ιερέως Έσδρα του γραμματέως.
\par 27 Και εν τοις εγκαινίοις του τείχους της Ιερουσαλήμ, εζήτησαν τους Λευΐτας από πάντων των τόπων αυτών διά να φέρωσιν αυτούς εις Ιερουσαλήμ, να κάμωσι τα εγκαίνια μετ' ευφροσύνης, υμνούντες και ψάλλοντες εν κυμβάλοις, ψαλτηρίοις και εν κιθάραις.
\par 28 Και συνήχθησαν οι υιοί των ψαλτωδών και από της περιχώρου κυκλόθεν της Ιερουσαλήμ και από των χωρίων Νετωφαθί·
\par 29 και από του οίκου Γιλγάλ και από των αγρών Γεβά και Αζμαβέθ· διότι οι ψαλτωδοί ωκοδόμησαν χωρία εις εαυτούς κύκλω της Ιερουσαλήμ.
\par 30 Και εκαθαρίσθησαν οι ιερείς και οι Λευΐται, και εκαθάρισαν τον λαόν και τας πύλας και το τείχος.
\par 31 Τότε ανεβίβασα τους άρχοντας του Ιούδα επί το τείχος και έστησα δύο μεγάλους χορούς αινούντων· ο μεν επορεύετο επί τα δεξιά, επί του τείχους προς την πύλην της κοπρίας·
\par 32 και κατόπιν αυτών επορεύοντο ο Ωσαΐας και το ήμισυ των αρχόντων του Ιούδα,
\par 33 και ο Αζαρίας, ο Έσδρας και ο Μεσουλλάμ,
\par 34 ο Ιούδας και ο Βενιαμίν και ο Σεμαΐας και ο Ιερεμίας·
\par 35 και εκ των υιών των ιερέων μετά σαλπίγγων, ο Ζαχαρίας ο υιός του Ιωνάθαν, υιού του Σεμαΐα, υιού του Ματθανία, υιού του Μιχαΐα, υιού του Ζακχούρ, υιού του Ασάφ·
\par 36 και οι αδελφοί αυτού, Σεμαΐας, και Αζαρεήλ, Μιλαλαΐ, Γιλαλαΐ, Μααΐ, Ναθαναήλ, και Ιούδας, Ανανί, μετά μουσικών οργάνων Δαβίδ του ανθρώπου του Θεού, και Έσδρας ο γραμματεύς έμπροσθεν αυτών.
\par 37 Και επί την πύλην της πηγής και απέναντι αυτών, ανέβησαν διά των βαθμίδων της πόλεως Δαβίδ εις την ανάβασιν του τείχους, επάνωθεν του οίκου του Δαβίδ, και έως της πύλης των υδάτων προς ανατολάς.
\par 38 Ο δε άλλος χορός των αινούντων επορεύετο εις το απέναντι, και εγώ κατόπιν αυτών, και το ήμισυ του λαού επί του τείχους, επάνωθεν του πύργου των φούρνων και έως του τείχους του πλατέος.
\par 39 Και επάνωθεν της πύλης Εφραΐμ και επάνωθεν της παλαιάς πύλης, και επάνωθεν της ιχθυϊκής πύλης και του πύργου Ανανεήλ και του πύργου του Μεά και έως της πύλης της προβατικής· και εστάθησαν εν τη πύλη της φυλακής.
\par 40 Και εστάθησαν οι δύο χοροί των αινούντων εν τω οίκω του Θεού, και εγώ και το ήμισυ των προεστώτων μετ' εμού·
\par 41 και οι ιερείς, Ελιακείμ, Μαασίας, Μινιαμείν, Μιχαΐας, Ελιωηνάϊ, Ζαχαρίας και Ανανίας, μετά σαλπίγγων·
\par 42 και Μαασίας και Σεμαΐας και Ελεάζαρ και Οζί και Ιωανάν και Μαλχίας και Ελάμ και Εσέρ. Και οι ψαλτωδοί ύψωσαν την φωνήν αυτών, μετά του Ιεζραΐα του επιστάτου.
\par 43 Και προσέφεραν εν εκείνη τη ημέρα θυσίας μεγάλας και ευφράνθησαν· διότι ο Θεός εύφρανεν αυτούς ευφροσύνην μεγάλην. Και αι γυναίκες έτι και τα παιδία ευφράνθησαν· και η ευφροσύνη της Ιερουσαλήμ ηκούσθη έως μακρόθεν.
\par 44 Και εν τη ημέρα εκείνη διωρίσθησαν άνδρες επί των οικημάτων διά τους θησαυρούς, διά τας προσφοράς, διά τας απαρχάς και διά τα δέκατα, διά να συνάγωσιν εν αυτοίς από των αγρών των πόλεων τα νενομισμένα μερίδια διά τους ιερείς και Λευΐτας· διότι ο Ιούδας ευφράνθη εξ αιτίας των ιερέων και εξ αιτίας των Λευϊτών των παρεστώτων.
\par 45 Και οι ψαλτωδοί και οι πυλωροί εφύλαξαν την φυλακήν του Θεού αυτών και την φυλακήν του καθαρισμού, κατά την προσταγήν του Δαβίδ και Σολομώντος του υιού αυτού.
\par 46 Διότι εν ταις ημέραις του Δαβίδ και του Ασάφ ήσαν εξ αρχής πρωτοψάλται και άσματα αινέσεως και ύμνοι προς τον Θεόν.
\par 47 Και πας ο Ισραήλ εν ταις ημέραις του Ζοροβάβελ και εν ταις ημέραις του Νεεμία έδιδον τα τεταγμένα μερίδια των ψαλτωδών και των πυλωρών, κατά πάσαν ημέραν· και ηγίαζον αυτά εις τους Λευΐτας, και οι Λευΐται ηγίαζον εις τους υιούς Ααρών.

\chapter{13}

\par Εν τη ημέρα εκείνη ανεγνώσθη εν τω βιβλίω του Μωϋσέως εις τα ώτα του λαού· και ευρέθη γεγραμμένον εν αυτώ, ότι οι Αμμωνίται και οι Μωαβίται δεν έπρεπε να εισέλθωσιν εις την συναγωγήν του Θεού έως αιώνος·
\par 2 διότι δεν προϋπήντησαν τους υιούς Ισραήλ μετά άρτου και μετά ύδατος, αλλ' εμίσθωσαν τον Βαλαάμ εναντίον αυτών, διά να καταρασθή αυτούς· πλην ο Θεός ημών έτρεψε την κατάραν εις ευλογίαν.
\par 3 Και ως ήκουσαν τον νόμον, εχώρισαν από του Ισραήλ πάντα αλλογενή.
\par 4 Προ τούτου δε Ελιασείβ ο ιερεύς, όστις είχε την επιστασίαν των οικημάτων του οίκου του Θεού ημών, είχε συγγενεύσει μετά του Τωβία·
\par 5 και είχεν ετοιμάσει δι' αυτόν μέγα οίκημα, όπου πρότερον έθετον τας εξ αλφίτων προσφοράς, το λιβάνιον και τα σκεύη και τα δέκατα του σίτου, του οίνου και του ελαίου, το διατεταγμένον των Λευϊτών και των ψαλτωδών και των πυλωρών και τας προσφοράς των ιερέων.
\par 6 Πλην εν πάσι τούτοις εγώ δεν ήμην εν Ιερουσαλήμ· διότι εν τω τριακοστώ δευτέρω έτει Αρταξέρξου του βασιλέως της Βαβυλώνος ήλθον προς τον βασιλέα και μεθ' ημέρας τινάς εζήτησα παρά του βασιλέως,
\par 7 και ήλθον εις Ιερουσαλήμ και έμαθον το κακόν, το οποίον ο Ελιασείβ έκαμε χάριν του Τωβία, ετοιμάσας εις αυτόν οίκημα εν ταις αυλαίς του οίκου του Θεού.
\par 8 Και δυσηρεστήθην πολύ· και έρριψα έξω του οικήματος πάντα τα σκεύη του οίκου του Τωβία.
\par 9 Και προσέταξα, και εκαθάρισαν τα οικήματα· και επανέφερα εκεί τα σκεύη του οίκου του Θεού, τας εξ αλφίτων προσφοράς και το λιβάνιον.
\par 10 Και έμαθον ότι τα μερίδια των Λευΐτών δεν εδόθησαν εις αυτούς· διότι οι Λευΐται και οι ψαλτωδοί, οι ποιούντες το έργον, έφυγον έκαστος εις τον αγρόν αυτού.
\par 11 Και επέπληξα τους προεστώτας και είπα, Διά τι εγκατελείφθη ο οίκος του Θεού; Και εσύναξα αυτούς και αποκατέστησα αυτούς εις την θέσιν αυτών.
\par 12 Τότε έφερε πας ο Ιούδας εις τας αποθήκας το δέκατον του σίτου και του οίνου και του ελαίου.
\par 13 Και κατέστησα φύλακας επί των αποθηκών, Σελεμίαν τον ιερέα και Σαδώκ τον γραμματέα και εκ των Λευΐτών· τον Φεδαΐαν· και πλησίον αυτών, Ανάν τον υιόν του Ζακχούρ, υιού του Ματθανία· διότι ελογίζοντο πιστοί· το έργον δε αυτών ήτο να διανέμωσιν εις τους αδελφούς αυτών.
\par 14 Μνήσθητί μου, Θεέ μου, περί τούτου, και μη εξαλείψης τα ελέη μου, τα οποία έκαμα εις τον οίκον του Θεού μου και εις τας τελετάς αυτού.
\par 15 Εν εκείναις ταις ημέραις είδόν τινάς εν Ιούδα ληνοπατούντας εν σαββάτω και εισφέροντας δράγματα και επιφορτίζοντας επί όνους, και οίνον, σταφύλια και σύκα και παν είδος φορτίων, τα οποία έφερον εις Ιερουσαλήμ την ημέραν του σαββάτου· και διεμαρτυρήθην εν τη ημέρα, καθ' ην επώλουν τρόφιμα.
\par 16 Και οι Τύριοι, οι κατοικούντες εν αυτή, έφερον ιχθύας και παν είδος ωνίων και επώλουν εν σαββάτω εις τους υιούς Ιούδα και εν Ιερουσαλήμ.
\par 17 Και επέπληξα τους προκρίτους του Ιούδα και είπα προς αυτούς, Τι είναι το πράγμα τούτο το κακόν, το οποίον σεις κάμνετε, βεβηλούντες την ημέραν του σαββάτου;
\par 18 δεν έκαμνον ούτως οι πατέρες σας, και έφερεν ο Θεός ημών πάντα ταύτα τα κακά εφ' ημάς και επί την πόλιν ταύτην; αλλά σεις επαναφέρετε οργήν επί τον Ισραήλ, βεβηλούντες το σάββατον.
\par 19 Διά τούτο, ότε ήρχιζε να συσκοτάζη εις τας πύλας της Ιερουσαλήμ προ του σαββάτου, είπα, και έκλεισαν τας πύλας, και προσέταξα να μη ανοιχθώσιν έως μετά το σάββατον· και κατέστησα επί τας πύλας τινάς εκ των υπηρετών μου, διά να μη εισέλθη φορτίον την ημέραν του σαββάτου.
\par 20 Και διενυκτέρευσαν οι έμποροι και οι πωληταί παντός είδους ωνίων έξω της Ιερουσαλήμ άπαξ και δις.
\par 21 Τότε διεμαρτυρήθην εναντίον αυτών και είπα προς αυτούς, Διά τι διανυκτερεύετε έμπροσθεν του τείχους; εάν δευτερώσητε, θέλω βάλει χείρα επάνω σας. Έκτοτε δεν ήλθον εν σαββάτω.
\par 22 Και είπα προς τους Λευΐτας να καθαρίζωνται και να έρχωνται να φυλάττωσι τας πύλας, διά να αγιάζωσι την ημέραν του σαββάτου. Μνήσθητί μου, Θεέ μου, και περί τούτου, και ελέησόν με κατά το πλήθος του ελέους σου.
\par 23 Προσέτι εν ταις ημέραις εκείναις είδον τους Ιουδαίους τους λαβόντας γυναίκας Αζωτίας, Αμμωνίτιδας και Μωαβίτιδας·
\par 24 και τα τέκνα αυτών λαλούντα ήμισυ Αζωτιστί, και μη εξεύροντα να λαλήσωσιν Ιουδαϊστί αλλά κατά την γλώσσαν διαφόρων λαών.
\par 25 Και επέπληξα αυτούς και κατηράσθην αυτούς, και ερράβδισα τινάς εξ αυτών και ετριχομάδησα αυτούς, και ώρκισα αυτούς εις τον Θεόν, λέγων, Δεν θέλετε δώσει τας θυγατέρας σας εις τους υιούς αυτών, και δεν θέλετε λάβει εκ των θυγατέρων αυτών εις τους υιούς σας ή εις εαυτούς·
\par 26 δεν ημάρτησεν ούτω Σολομών ο βασιλεύς του Ισραήλ; καίτοι μεταξύ πολλών εθνών δεν υπήρξε βασιλεύς όμοιος αυτού, όστις ήτο αγαπώμενος υπό του Θεού αυτού, και έκαμεν αυτόν ο Θεός βασιλέα επί πάντα τον Ισραήλ· αλλ' όμως και αυτόν αι ξέναι γυναίκες έκαμον να αμαρτήση·
\par 27 θέλομεν λοιπόν συγκατανεύσει εις εσάς να κάμνητε άπαν τούτο το μέγα κακόν, να γίνησθε παραβάται εναντίον εις τον Θεόν ημών λαμβάνοντες ξένας γυναίκας;
\par 28 Και εις εκ των υιών του Ιωαδά, υιού του Ελιασείβ του ιερέως του μεγάλου, ήτο γαμβρός Σαναβαλλάτ του Ορωνίτου· όθεν απεδίωξα αυτόν απ' έμπροσθέν μου.
\par 29 Μνήσθητι αυτών, Θεέ μου, διότι εβεβήλωσαν την ιερατείαν και την διαθήκην της ιερατείας και των Λευϊτών.
\par 30 Και εκαθάρισα αυτούς από πάντων των ξένων, και διώρισα φυλακάς εκ των ιερέων και των Λευΐτών, έκαστον εις τα έργα αυτού·
\par 31 και διά την προσφοράν των ξύλων εν καιροίς ωρισμένοις, και διά τας απαρχάς. Μνήσθητί μου, Θεέ μου, επ' αγαθώ.


\end{document}