\begin{document}

\title{Esther}


\chapter{1}

\par Και εν ταις ημέραις του Ασσουήρου, ούτος είναι ο Ασσουήρης, ο βασιλεύων από της Ινδίας έως της Αιθιοπίας, επί εκατόν εικοσιεπτά επαρχιών·
\par 2 εν εκείναις ταις ημέραις, ότε ο βασιλεύς Ασσουήρης εκάθησεν επί τον θρόνον της βασιλείας αυτού, εν Σούσοις τη βασιλευούση,
\par 3 εν τω τρίτω έτει της βασιλείας αυτού, έκαμε συμπόσιον εις πάντας τους άρχοντας αυτού και τους δούλους αυτού· και ήτο ενώπιον αυτού η δύναμις της Περσίας και της Μηδίας, οι ευγενείς και οι άρχοντες των επαρχιών,
\par 4 ότε εδείκνυε τα πλούτη της ενδόξου βασιλείας αυτού, και την λαμπρότητα της εξόχου μεγαλειότητος αυτού, ημέρας πολλάς, εκατόν ογδοήκοντα ημέρας.
\par 5 Και αφού επληρώθησαν αι ημέραι αύται, έκαμεν ο βασιλεύς συμπόσιον εις πάντα τον λαόν τον ευρεθέντα εν Σούσοις τη βασιλευούση, από μεγάλου έως μικρού, επτά ημέρας, εν τη αυλή του κήπου του βασιλικού παλατίου·
\par 6 όπου ήσαν παραπετάσματα λευκά, πράσινα και κυανά, κρεμάμενα διά σχοινίων βυσσίνων και πορφυρών διά κρίκων αργυρών εις στύλους εκ μαρμάρου· κλίναι χρυσαί και αργυραί ήσαν επί λιθόστρωτον εκ πορφυρίτου και κυανού και λευκού και μέλανος μαρμάρου.
\par 7 Εκέρνων δε εις σκεύη χρυσά, ηλλάσσοντο δε τα σκεύη διαδοχικώς, και ήτο οίνος βασιλικός εν αφθονία, κατά την μεγαλοπρέπειαν του βασιλέως.
\par 8 Η δε πόσις ήτο κεκανονισμένη· ουδείς εβίαζε· διότι ο βασιλεύς προσέταξεν ούτως εις πάντας τους οικονόμους του παλατίου αυτού, να κάμνωσι κατά την ευχαρίστησιν εκάστου.
\par 9 Και Αστίν έτι η βασίλισσα έκαμεν εις τας γυναίκας συμπόσιον εν τω οίκω τω βασιλικώ του βασιλέως Ασσουήρου.
\par 10 Εν τη εβδόμη δε ημέρα, ότε η καρδία του βασιλέως ήτο εύθυμος εκ του οίνου, προσέταξε τον Μεουμάν, τον Βηζαθά, τον Αρβωνά, τον Βηγθά και Αβαγθά, τον Ζεθάρ και τον Χαρκάς, τους επτά ευνούχους τους υπηρετούντας ενώπιον του βασιλέως Ασσουήρου,
\par 11 να φέρωσι την Αστίν την βασίλισσαν ενώπιον του βασιλέως, μετά του βασιλικού διαδήματος, διά να δείξη το κάλλος αυτής εις τους λαούς και εις τους άρχοντας· διότι ήτο ώραία την όψιν.
\par 12 Η βασίλισσα όμως Αστίν ηρνήθη να έλθη κατά την προσταγήν του βασιλέως, την διά των ευνούχων. Όθεν ο βασιλεύς εθυμώθη σφόδρα, και η οργή αυτού εξήφθη εν εαυτώ.
\par 13 Τότε είπεν ο βασιλεύς προς τους σοφούς, τους γνωρίζοντας τους καιρούς· διότι τοιαύτη ήτο η συνήθεια του βασιλέως προς πάντας τους γνωρίζοντας τον νόμον και την κρίσιν·
\par 14 πλησίον δε αυτού ήτο ο Καρσένα, ο Σεθάρ, ο Αδμαθά, ο Θαρσείς, ο Μερές, ο Μαρσενά και ο Μεμουκάν, οι επτά άρχοντες της Περσίας και της Μηδίας, οίτινες έβλεπον το πρόσωπον του βασιλέως και είχον την προεδρίαν εν τω βασιλείω·
\par 15 Τι αρμόζει να κάμωμεν προς την βασίλισσαν Αστίν κατά τον νόμον, διότι δεν εξετέλεσε την διά των ευνούχων προσταγήν του βασιλέως Ασσουήρου;
\par 16 Και απεκρίθη ο Μεμουκάν ενώπιον του βασιλέως και των αρχόντων, Η βασίλισσα Αστίν δεν ημάρτησε μόνον εις τον βασιλέα, αλλά και εις πάντας τους άρχοντας και εις πάντας τους λαούς τους εν πάσαις ταις επαρχίαις του βασιλέως Ασσουήρου·
\par 17 διότι η πράξις της βασιλίσσης θέλει διαδοθή εις πάσας τας γυναίκας, ώστε θέλουσι καταφρονεί τους άνδρας αυτών έμπροσθεν των οφθαλμών αυτών, όταν διαφημισθή ότι ο βασιλεύς Ασσουήρης προσέταξε την βασίλισσαν Αστίν να φερθή ενώπιον αυτού, και δεν ήλθε·
\par 18 και την σήμερον αι δέσποιναι της Περσίας και της Μηδίας, όσαι ήκουσαν περί της πράξεως της βασιλίσσης, θέλουσι λαλήσει ούτω προς πάντας τους άρχοντας του βασιλέως· εντεύθεν μεγάλη περιφρόνησις και οργή·
\par 19 εάν λοιπόν ήναι αρεστόν εις τον βασιλέα, ας εξέλθη παρ' αυτού βασιλική διαταγή, και ας γραφή μεταξύ των νόμων των Περσών και των Μήδων, διά να ήναι αμετάθετος, να μη έλθη πλέον η Αστίν ενώπιον του βασιλέως Ασσουήρου· και ας δώση ο βασιλεύς την βασιλικήν αυτής αξίαν εις άλλην καλητέραν αυτής·
\par 20 και όταν το πρόσταγμα του βασιλέως, το οποίον θέλει κάμει, δημοσιευθή διά παντός του βασιλείου αυτού, διότι είναι μέγα, πάσαι αι γυναίκες θέλουσιν αποδίδει τιμήν εις τους άνδρας αυτών, από μεγάλου έως μικρού.
\par 21 Και ο λόγος ήρεσεν εις τον βασιλέα και εις τους άρχοντας· και έκαμεν ο βασιλεύς κατά τον λόγον του Μεμουκάν·
\par 22 και έστειλε γράμματα εις πάσας τας επαρχίας του βασιλέως, εις εκάστην επαρχίαν κατά το γράφειν αυτής, και προς έκαστον λαόν κατά την γλώσσαν αυτού, να ήναι πας ανήρ κύριος εν τη οικία αυτού, και να λαλή κατά την γλώσσαν του λαού αυτού.

\chapter{2}

\par Μετά τα πράγματα ταύτα, αφού κατεπραΰνθη ο θυμός του βασιλέως Ασσουήρου, ενεθυμήθη την Αστίν, και τι είχε κάμει αυτή και τι είχεν αποφασισθή εναντίον αυτής.
\par 2 Και είπον οι δούλοι του βασιλέως, οι υπηρετούντες αυτόν, Ας ζητηθώσι διά τον βασιλέα νέαι παρθένοι, ώραίαι την όψιν·
\par 3 και ας διορίση ο βασιλεύς εφόρους εν πάσαις ταις επαρχίαις του βασιλείου αυτού, και να συνάξωσιν εις τα Σούσα την βασιλεύουσαν πάσας τας νέας παρθένους τας ώραίας την όψιν εις τον γυναικώνα, υπό την τήρησιν του Ήγαϊ ευνούχου του βασιλέως, του φύλακος των γυναικών· και ας δοθώσιν εις αυτάς τα προς καθαρισμόν αυτών·
\par 4 και η νέα, ήτις αρέση εις τον βασιλέα, ας ήναι βασίλισσα αντί της Αστίν. Και το πράγμα ήρεσεν εις τον βασιλέα, και έκαμεν ούτω.
\par 5 Ήτο δε εν Σούσοις τη βασιλευούση άνθρωπός τις Ιουδαίος, ονομαζόμενος Μαροδοχαίος, υιός του Ιαείρ, υιού του Σιμεΐ, υιού του Κείς, Βενιαμίτης·
\par 6 όστις είχε μετοικισθή από Ιερουσαλήμ μετά των αιχμαλώτων, οίτινες μετωκίσθησαν μετά του Ιεχονία βασιλέως του Ιούδα, τους οποίους μετώκισε Ναβουχοδονόσορ ο βασιλεύς της Βαβυλώνος.
\par 7 Και ούτος ανέτρεφε την Αδασσά, ήτις είναι η Εσθήρ, την θυγατέρα του θείου αυτού· διότι δεν είχεν ούτε πατέρα ούτε μητέρα· και το κοράσιον ήτο ευειδές και ώραίον· το οποίον ο Μαροδοχαίος, ότε ο πατήρ αυτής και η μήτηρ απέθανον, ανέλαβε διά θυγατέρα αυτού.
\par 8 Ότε δε ηκούσθη το πρόσταγμα του βασιλέως και η διαταγή αυτού, και ότε πολλά κοράσια συνήχθησαν εις τα Σούσα την βασιλεύουσαν υπό την τήρησιν του Ήγαϊ, εφέρθη και η Εσθήρ εις τον οίκον του βασιλέως υπό την τήρησιν του Ήγαϊ, του φύλακος των γυναικών.
\par 9 Και το κοράσιον ήρεσεν εις αυτόν και εύρηκε χάριν ενώπιον αυτού, ώστε έσπευσε να δώση εις αυτήν τα προς καθαρισμόν αυτής και την μερίδα αυτής· έδωκε δε εις αυτήν και τα επτά κοράσια τα διωρισμένα εκ του οίκου του βασιλέως· και μετέφερεν αυτήν και τα κοράσια αυτής εις το καλήτερον μέρος του γυναικώνος.
\par 10 Η Εσθήρ δεν εφανέρωσε τον λαόν αυτής ουδέ την συγγένειαν αυτής· διότι ο Μαροδοχαίος είχε προστάξει εις αυτήν να μη φανερώση.
\par 11 Και καθ' εκάστην ημέραν περιεπάτει ο Μαροδοχαίος έμπροσθεν της αυλής του γυναικώνος, διά να μανθάνη πως είχεν η Εσθήρ και τι έγεινεν εις αυτήν.
\par 12 Ότε δε έφθανεν η σειρά εκάστου κορασίου διά να εισέλθη προς τον βασιλέα Ασσουήρην, αφού ήθελε σταθή δώδεκα μήνας κατά το ήθος των γυναικών, διότι ούτω συνεπληρούντο αι ημέραι του καθαρισμού αυτών, εξ μήνας περιηλείφοντο με έλαιον σμύρνινον, και εξ μήνας με αρώματα και με άλλα καθαριστικά των γυναικών·
\par 13 και ούτως εισήρχετο το κοράσιον προς τον βασιλέα· παν ό,τι έλεγεν, εδίδετο εις αυτήν, διά να λάβη μεθ' εαυτής εκ του γυναικώνος εις τον οίκον του βασιλέως.
\par 14 Το εσπέρας εισήρχετο και το πρωΐ επέστρεφεν εις τον δεύτερον γυναικώνα, υπό την τήρησιν του Σαασγάζ, ευνούχου του βασιλέως, όστις εφύλαττε τας παλλακίδας· δεν εισήρχετο πλέον εις τον βασιλέα, ειμή εάν ήθελεν αυτήν ο βασιλεύς, και εκαλείτο ονομαστί.
\par 15 Ότε λοιπόν έφθασεν η σειρά διά να εισέλθη προς τον βασιλέα η Εσθήρ, η θυγάτηρ του Αβιχαίλ, θείου του Μαροδοχαίου, την οποίαν έλαβε διά θυγατέρα αυτού, δεν εζήτησεν άλλο παρ' ό,τι διώρισεν ο Ήγαϊ ο ευνούχος του βασιλέως, ο φύλαξ των γυναικών. Και η Εσθήρ εύρισκε χάριν ενώπιον πάντων των βλεπόντων αυτήν.
\par 16 Η Εσθήρ λοιπόν εφέρθη προς τον βασιλέα Ασσουήρην εις τον βασιλικόν αυτού οίκον, τον δέκατον μήνα, ούτος είναι ο μην Τεβέθ, εν τω εβδόμω έτει της βασιλείας αυτού.
\par 17 Και ηγάπησεν ο βασιλεύς την Εσθήρ υπέρ πάσας τας γυναίκας, και εύρηκε χάριν και έλεος ενώπιον αυτού υπέρ πάσας τας παρθένους· και επέθηκε το βασιλικόν διάδημα επί την κεφαλήν αυτής και έκαμεν αυτήν βασίλισσαν αντί της Αστίν.
\par 18 Τότε έκαμεν ο βασιλεύς συμπόσιον μέγα εις πάντας τους άρχοντας αυτού και τους δούλους αυτού, το συμπόσιον της Εσθήρ· και έκαμεν άφεσιν εις τας επαρχίας και έδωκε δώρα κατά την βασιλικήν μεγαλοπρέπειαν.
\par 19 Και ότε αι παρθένοι συνήχθησαν την δευτέραν φοράν, τότε εκάθησεν ο Μαροδοχαίος εν τη βασιλική πύλη.
\par 20 Η Εσθήρ δεν εφανέρωσε την συγγένειαν αυτής ούτε τον λαόν αυτής, καθώς προσέταξεν εις αυτήν ο Μαροδοχαίος· διότι η Εσθήρ έκαμνε την προσταγήν του Μαροδοχαίου, καθώς ότε ανετρέφετο πλησίον αυτού.
\par 21 Εν εκείναις ταις ημέραις, ενώ ο Μαροδοχαίος εκάθητο εν τη βασιλική πύλη, δύο εκ των ευνούχων του βασιλέως, Βιχθάν και Θερές, εκ των φυλαττόντων την είσοδον, ωργίσθησαν και εζήτουν να επιβάλωσι χείρα επί τον βασιλέα Ασσουήρην.
\par 22 Και το πράγμα έγεινε γνωστόν εις τον Μαροδοχαίον, και ανήγγειλεν αυτό προς Εσθήρ την βασίλισσαν· η δε Εσθήρ είπεν αυτό προς τον βασιλέα εξ ονόματος του Μαροδοχαίου.
\par 23 Και γενομένης εξετάσεως περί του πράγματος, ευρέθη ούτως· όθεν εκρεμάσθησαν αμφότεροι εις ξύλον· και εγράφη εν τω βιβλίω των χρονικών ενώπιον του βασιλέως.

\chapter{3}

\par Μετά τα πράγματα ταύτα εμεγάλυνεν ο βασιλεύς Ασσουήρης τον Αμάν, τον υιόν του Αμμεδαθά του Αγαγίτου, και ύψωσεν αυτόν και έθεσε τον θρόνον αυτού υπεράνω πάντων των αρχόντων των περί αυτόν.
\par 2 Και πάντες οι δούλοι του βασιλέως, οι εν τη βασιλική πύλη, έκλινον και προσεκύνουν τον Αμάν· διότι ούτω προσέταξεν ο βασιλεύς περί αυτού. Ο Μαροδοχαίος όμως δεν έκλινε και δεν προσεκύνει αυτόν.
\par 3 Και είπον οι δούλοι του βασιλέως, οι εν τη βασιλική πύλη, προς τον Μαροδοχαίον, Διά τι συ παραβαίνεις την προσταγήν του βασιλέως;
\par 4 Αφού δε καθ' ημέραν έλεγον προς αυτόν, και εκείνος δεν υπήκουεν εις αυτούς, απήγγειλαν τούτο προς τον Αμάν, διά να ίδωσιν αν οι λόγοι του Μαροδοχαίου ήσαν στερεοί· διότι είχε φανερώσει προς αυτούς ότι ήτο Ιουδαίος.
\par 5 Και ότε ο Αμάν είδεν ότι ο Μαροδοχαίος δεν έκλινε και δεν προσεκύνει αυτόν, ενεπλήσθη θυμού ο Αμάν.
\par 6 Και εστοχάσθη ταπεινόν να βάλη χείρα επί μόνον τον Μαροδοχαίον· διότι είχον φανερώσει προς αυτόν τον λαόν του Μαροδοχαίου· όθεν εζήτει ο Αμάν να αφανίση πάντας τους Ιουδαίους τους εν παντί τω βασιλείω του Ασσουήρου, τον λαόν του Μαροδοχαίου.
\par 7 Και εν τω πρώτω μηνί, ούτος είναι ο μην Νισάν, εν τω δωδεκάτω έτει του βασιλέως Ασσουήρου, έρριψαν φούρ, ήγουν κλήρον, ενώπιον του Αμάν, από ημέρας εις ημέραν και από μηνός εις μήνα, μέχρι του δωδεκάτου μηνός, ούτος είναι ο μην Αδάρ.
\par 8 Και είπεν ο Αμάν προς τον βασιλέα Ασσουήρην, Υπάρχει τις λαός διεσπαρμένος και διακεχωρισμένος μεταξύ των λαών κατά πάσας τας επαρχίας του βασιλείου σου· και οι νόμοι αυτών διάφοροι των νόμων πάντων των λαών, και δεν φυλάττουσι τους νόμους του βασιλέως· όθεν δεν αρμόζει εις τον βασιλέα να υποφέρη αυτούς·
\par 9 εάν ήναι αρεστόν εις τον βασιλέα, ας γραφή να εξολοθρευθώσι και εγώ θέλω μετρήσει δέκα χιλιάδας ταλάντων αργυρίου εις τας χείρας των οικονόμων διά να φέρωσιν εις τα θησαυροφυλάκια του βασιλέως.
\par 10 Και εκβαλών ο βασιλεύς το δακτυλίδιον αυτού από της χειρός αυτού, έδωκεν αυτό εις τον Αμάν τον υιόν του Αμμεδαθά του Αγαγίτου, τον εχθρόν των Ιουδαίων,
\par 11 Και είπεν ο βασιλεύς προς τον Αμάν· το αργύριον δίδεται εις σε, και ο λαός, διά να κάμης εις αυτόν όπως σοι αρέσκει.
\par 12 Και προσεκλήθησαν οι γραμματείς του βασιλέως την δεκάτην τρίτην ημέραν του πρώτου μηνός, και εγράφη κατά πάντα όσα προσέταξεν ο Αμάν, προς τους σατράπας του βασιλέως και προς τους διοικητάς τους κατά πάσαν επαρχίαν και προς τους άρχοντας εκάστου λαού πάσης επαρχίας κατά το γράφειν αυτών, και προς έκαστον λαόν κατά την γλώσσαν αυτών· εν ονόματι του βασιλέως Ασσουήρου εγράφη και εσφραγίσθη με το δακτυλίδιον του βασιλέως.
\par 13 Και εστάλησαν γράμματα διά ταχυδρόμων εις πάσας τας επαρχίας του βασιλέως, διά να αφανίσωσι, να φονεύσωσι και να εξολοθρεύσωσι πάντας τους Ιουδαίους, νέους και γέροντας, νήπια και γυναίκας, εν μιά ημέρα, την δεκάτην τρίτην του δωδεκάτου μηνός, ούτος είναι ο μην Αδάρ, και να διαρπάσωσι τα υπάρχοντα αυτών.
\par 14 Το αντίγραφον της επιστολής, το προς διάδοσιν του προστάγματος κατά πάσαν επαρχίαν, εδημοσιεύθη προς πάντας τους λαούς, διά να ήναι έτοιμοι εν εκείνη τη ημέρα.
\par 15 Οι ταχυδρόμοι εξήλθον, σπεύδοντες διά την προσταγήν του βασιλέως, και η διαταγή εξεδόθη εν Σούσοις τη βασιλευούση. Ο δε βασιλεύς και ο Αμάν εκάθησαν να συμποσιάσωσιν· η δε πόλις Σούσα ήτο εν αμηχανία.

\chapter{4}

\par Και μαθών Μαροδοχαίος άπαν το γινόμενον, διέσχισε τα ιμάτια αυτού και ενεδύθη σάκκον εν σποδώ και εξήλθεν εις το μέσον της πόλεως και εβόα μετά βοής μεγάλης και πικράς·
\par 2 και ήλθεν έως έμπροσθεν της βασιλικής πύλης· διότι ουδείς ηδύνατο να εισέλθη εις την βασιλικήν πύλην ενδεδυμένος σάκκον.
\par 3 Και κατά πάσαν επαρχίαν, όπου έφθασεν η προσταγή του βασιλέως και το διάταγμα αυτού, ήτο μέγα πένθος μεταξύ των Ιουδαίων, και νηστεία και θρήνος και ολολυγμός· πολλοί εκοίτοντο με σάκκον και σποδόν.
\par 4 Εισήλθον δε αι θεράπαιναι της Εσθήρ και οι ευνούχοι αυτής, και απήγγειλαν τούτο προς αυτήν. Και εταράχθη σφόδρα η βασίλισσα· και έπεμψεν ιμάτια διά να ενδύσωσι τον Μαροδοχαίον και να εκβάλωσι τον σάκκον αυτού απ' αυτού· και δεν εδέχθη.
\par 5 Τότε εκάλεσεν η Εσθήρ τον Αθάχ, εκ των ευνούχων του βασιλέως, τον οποίον είχε διορίσει εις την υπηρεσίαν αυτής, και προσέταξεν εις αυτόν περί του Μαροδοχαίου, διά να μάθη τι τούτο, και διά τι τούτο.
\par 6 Και εξήλθεν ο Αθάχ προς τον Μαροδοχαίον εις την πλατείαν της πόλεως, την απέναντι της βασιλικής πύλης.
\par 7 Και εφανέρωσε προς αυτόν ο Μαροδοχαίος άπαν το γεγονός εις αυτόν, και το ποσόν του αργυρίου το οποίον ο Αμάν υπεσχέθη να μετρήση εις τα θησαυροφυλάκια του βασιλέως διά τους Ιουδαίους· διά να απολέση αυτούς.
\par 8 Και έδωκεν εις αυτόν το αντίγραφον του γράμματος της διαταγής, της εκδοθείσης εν Σούσοις διά να αφανίσωσιν αυτούς, διά να δείξη αυτό εις την Εσθήρ, και να απαγγείλη προς αυτήν και να παραγγείλη εις αυτήν να εισέλθη προς τον βασιλέα, να παρακαλέση αυτόν και να κάμη αίτησιν προς αυτόν υπέρ του λαού αυτής.
\par 9 Και ήλθεν ο Αθάχ και απήγγειλε προς την Εσθήρ τους λόγους του Μαροδοχαίου.
\par 10 Η δε Εσθήρ ελάλησε προς τον Αθάχ και έδωκεν εις αυτόν προσταγήν προς τον Μαροδοχαίον,
\par 11 Πάντες οι δούλοι του βασιλέως, και ο λαός των επαρχιών του βασιλέως, εξεύρουσιν, ότι όστις, ανήρ ή γυνή, εισέλθη προς τον βασιλέα εις την ενδοτέραν αυλήν άκλητος, εις νόμος αυτού είναι να θανατόνηται, εκτός εκείνου προς τον οποίον ο βασιλεύς εκτείνει το χρυσούν σκήπτρον διά να ζήση· αλλ' εγώ δεν προσεκλήθην να εισέλθω προς τον βασιλέα ήδη τριάκοντα ημέρας.
\par 12 Και απήγγειλαν προς τον Μαροδοχαίον τους λόγους της Εσθήρ.
\par 13 Τότε ο Μαροδοχαίος παρήγγειλε ν' αποκριθώσι προς την Εσθήρ, Μη στοχάζεσαι εν σεαυτή ότι συ εκ πάντων των Ιουδαίων θέλεις σωθή εν τω οίκω του βασιλέως·
\par 14 διότι εάν σιωπήσης διόλου εν τω καιρώ τούτω, θέλει ελθεί άλλοθεν αναψυχή και σωτηρία εις τους Ιουδαίους, συ δε και ο οίκος του πατρός σου θέλετε απολεσθή· και τις εξεύρει εάν συ ήλθες εις την βασιλείαν διά τοιούτον καιρόν οποίος ούτος;
\par 15 Τότε προσέταξεν η Εσθήρ να αποκριθώσι προς τον Μαροδοχαίον·
\par 16 Ύπαγε, σύναξον πάντας τους Ιουδαίους τους ευρισκομένους εν Σούσοις, και νηστεύσατε υπέρ εμού και μη φάγητε και μη πίητε τρεις ημέρας, νύκτα και ημέραν· και εγώ και αι θεράπαιναί μου θέλομεν νηστεύσει ομοίως· και ούτω θέλω εισέλθει προς τον βασιλέα, το οποίον δεν είναι κατά τον νόμον· και αν απολεσθώ, ας απολεσθώ.
\par 17 Και απελθών ο Μαροδοχαίος έκαμε κατά πάντα όσα προσέταξεν εις αυτόν η Εσθήρ.

\chapter{5}

\par Την τρίτην δε ημέραν ενδυθείσα η Εσθήρ την βασιλικήν στολήν εστάθη εν τη εσωτέρα αυλή του βασιλικού οίκου, απέναντι του οίκου του βασιλέως· και ο βασιλεύς εκάθητο επί του βασιλικού θρόνου αυτού εν τω βασιλικώ οίκω, απέναντι της πύλης του οίκου.
\par 2 Και ως είδεν ο βασιλεύς την Εσθήρ την βασίλισσαν ισταμένην εν τη αυλή, εύρηκε χάριν ενώπιον αυτού· και εξέτεινεν ο βασιλεύς προς την Εσθήρ το χρυσούν σκήπτρον το εν τη χειρί αυτού· και επλησίασεν η Εσθήρ και ήγγισε το άκρον του σκήπτρου.
\par 3 Και είπε προς αυτήν ο βασιλεύς, Τι θέλεις, βασίλισσα Εσθήρ; και τις η αίτησίς σου; και έως του ημίσεος της βασιλείας θέλει δοθή εις σε.
\par 4 Και απεκρίθη η Εσθήρ, Εάν ήναι αρεστόν εις τον βασιλέα, ας έλθη ο βασιλεύς και ο Αμάν την ημέραν ταύτην εις το συμπόσιον, το οποίον ητοίμασα δι' αυτόν.
\par 5 Και είπεν ο βασιλεύς, Επισπεύσατε τον Αμάν, διά να κάμη τον λόγον της Εσθήρ. Και ήλθον ο βασιλεύς και ο Αμάν εις το συμπόσιον, το οποίον έκαμεν η Εσθήρ.
\par 6 Και είπεν ο βασιλεύς προς την Εσθήρ επί του συμποσίου του οίνου, Τι το ζήτημά σου; και θέλει δοθή εις σέ· και τις η αίτησίς σου; και έως του ημίσεος της βασιλείας εάν ζητήσης, θέλει γείνει.
\par 7 Τότε αποκριθείσα η Εσθήρ είπε, το ζήτημά μου και η αίτησίς μου είναι·
\par 8 Εάν εύρηκα χάριν ενώπιον του βασιλέως, και εάν ήναι αρεστόν εις τον βασιλέα να εκτελέση το ζήτημά μου και να κάμη την αίτησίν μου, ας έλθη ο βασιλεύς και ο Αμάν εις το συμπόσιον το οποίον θέλω ετοιμάσει δι' αυτούς· και αύριον θέλω κάμει κατά τον λόγον του βασιλέως.
\par 9 Τότε εξήλθεν ο Αμάν την ημέραν εκείνην περιχαρής και εύθυμος την καρδίαν· αλλ' ότε ο Αμάν είδε τον Μαροδοχαίον εν τη πύλη του βασιλέως, ότι δεν εσηκώθη ουδέ εκινήθη δι' αυτόν, ενεπλήσθη ο Αμάν θυμού κατά του Μαροδοχαίου.
\par 10 Αλλ' ο Αμάν εκράτησεν εαυτόν· και εισελθών εις τον οίκον αυτού έστειλε και εκάλεσε τους φίλους αυτού και Ζερές την γυναίκα αυτού,
\par 11 και διηγήθη προς αυτούς ο Αμάν περί της δόξης του πλούτου αυτού και του πλήθους των τέκνων αυτού, και πόσον ο βασιλεύς εμεγάλυνεν αυτόν, και τίνι τρόπω ύψωσεν αυτόν υπεράνω των αρχόντων και των δούλων του βασιλέως.
\par 12 Και είπεν ο Αμάν, Μάλιστα η βασίλισσα Εσθήρ δεν προσεκάλεσεν εις το συμπόσιον το οποίον έκαμεν, ειμή εμέ, μετά του βασιλέως· και αύριον έτι είμαι προσκεκλημένος προς αυτήν μετά του βασιλέως·
\par 13 πλην πάντα ταύτα δεν με ωφελούσιν, ενόσω βλέπω τον Μαροδοχαίον τον Ιουδαίον καθήμενον εν τη πύλη του βασιλέως.
\par 14 Και είπε προς αυτόν Ζερές η γυνή αυτού και πάντες οι φίλοι αυτού, Ας κατασκευασθή ξύλον πεντήκοντα πηχών το ύψος, και το πρωΐ ειπέ προς τον βασιλέα να κρεμασθή ο Μαροδοχαίος επ' αυτό· τότε ύπαγε περιχαρής μετά του βασιλέως εις το συμπόσιον. Και το πράγμα ήρεσεν εις τον Αμάν, και προσέταξε να ετοιμασθή το ξύλον.

\chapter{6}

\par Εν εκείνη τη νυκτί ο ύπνος έφυγεν από του βασιλέως· και προσέταξε να φέρωσι το βιβλίον των υπομνημάτων των χρονικών· και ανεγινώσκοντο ενώπιον του βασιλέως.
\par 2 Και ευρέθη γεγραμμένον ότι ο Μαροδοχαίος απήγγειλε περί του Βιχθάν και Θερές, δύο εκ των ευνούχων του βασιλέως, θυρωρών, οίτινες εζήτησαν να επιβάλωσι χείρα επί τον βασιλέα Ασσουήρην.
\par 3 Και είπεν ο βασιλεύς, Ποία τιμή και αξιοπρέπεια έγεινεν εις τον Μαροδοχαίον διά τούτο; Και είπον οι δούλοι του βασιλέως οι υπηρετούντες αυτόν, Δεν έγεινεν ουδέν εις αυτόν.
\par 4 Και είπεν ο βασιλεύς, Τις είναι εν τη αυλή; είχε δε ελθεί ο Αμάν εις την εξωτέραν αυλήν του βασιλικού οίκου, διά να είπη προς τον βασιλέα να κρεμάση τον Μαροδοχαίον εις το ξύλον το οποίον ητοίμασε δι' αυτόν.
\par 5 Και είπον προς αυτόν οι δούλοι του βασιλέως, Ιδού, ο Αμάν ίσταται εν τη αυλή. Και είπεν ο βασιλεύς, Ας εισέλθη.
\par 6 Και ότε εισήλθεν ο Αμάν, είπε προς αυτόν ο βασιλεύς, Τι πρέπει να γείνη εις τον άνθρωπον, τον οποίον ευαρεστείται ο βασιλεύς να τιμήση; Ο δε Αμάν εστοχάσθη εν τη καρδία αυτού, εις ποίον άλλον ο βασιλεύς ήθελεν ευαρεστηθή να κάμη τιμήν, παρά εις εμέ;
\par 7 Απεκρίθη λοιπόν ο Αμάν προς τον βασιλέα, Περί του ανθρώπου, τον οποίον ο βασιλεύς ευαρεστείται να τιμήση,
\par 8 ας φέρωσι την βασιλικήν στολήν, την οποίαν ο βασιλεύς ενδύεται, και τον ίππον επί του οποίου ο βασιλεύς ιππεύει, και να τεθή το βασιλικόν διάδημα επί της κεφαλής αυτού·
\par 9 και η στολή αύτη και ο ίππος ας δοθώσιν εις την χείρα τινός εκ των μεγαλητέρων αρχόντων του βασιλέως, διά να στολίση τον άνθρωπον τον οποίον ο βασιλεύς ευαρεστείται να τιμήση· και φέρων αυτόν έφιππον διά των οδών της πόλεως ας κηρύττη έμπροσθεν αυτού, ούτω θέλει γίνεσθαι εις τον άνθρωπον, τον οποίον ο βασιλεύς ευαρεστείται να τιμήση.
\par 10 Και είπεν ο βασιλεύς προς τον Αμάν, Σπεύσον, λάβε την στολήν και τον ίππον, ως είπας, και κάμε ούτως εις τον Μαροδοχαίον τον Ιουδαίον τον καθήμενον εν τη βασιλική πύλη· ας μη λείψη μηδέν εκ πάντων όσα είπας.
\par 11 Και έλαβεν ο Αμάν την στολήν και τον ίππον, και εστόλισε τον Μαροδοχαίον και έφερεν αυτόν έφιππον διά των οδών της πόλεως, κηρύττων έμπροσθεν αυτού, ούτω θέλει γίνεσθαι εις τον άνθρωπον, τον οποίον ο βασιλεύς ευαρεστείται να τιμήση.
\par 12 Και επανήλθεν ο Μαροδοχαίος εις την πύλην του βασιλέως· ο δε Αμάν έσπευσε προς τον οίκον αυτού περίλυπος και έχων την κεφαλήν αυτού κεκαλυμμένην.
\par 13 Και διηγήθη ο Αμάν προς Ζερές την γυναίκα αυτού και προς πάντας τους φίλους αυτού παν ό,τι συνέβη εις αυτόν. Και είπον προς αυτόν οι σοφοί αυτού και Ζερές η γυνή αυτού, Εάν ο Μαροδοχαίος, έμπροσθεν του οποίου ήρχισας να εκπίπτης, ήναι εκ του σπέρματος των Ιουδαίων, δεν θέλεις κατισχύσει εναντίον αυτού, αλλ' εξάπαντος θέλεις πέσει έμπροσθεν αυτού.
\par 14 Ενώ ελάλουν έτι μετ' αυτού, έφθασαν οι ευνούχοι του βασιλέως και έσπευσαν να φέρωσι τον Αμάν εις το συμπόσιον, το οποίον ητοίμασεν η Εσθήρ.

\chapter{7}

\par Ήλθον λοιπόν ο βασιλεύς και ο Αμάν να συμποσιάσωσι μετά της Εσθήρ της βασιλίσσης.
\par 2 Και είπε πάλιν ο βασιλεύς προς την Εσθήρ την δευτέραν ημέραν επί του συμποσίου του οίνου, Τι το ζήτημά σου, βασίλισσα Εσθήρ; και θέλει δοθή εις σέ· και τις η αίτησίς σου; και έως του ημίσεος της βασιλείας εάν ζητήσης, θέλει γείνει.
\par 3 Τότε απεκρίθη η Εσθήρ η βασίλισσα και είπεν, Εάν εύρηκα χάριν ενώπιόν σου, βασιλεύ, και εάν ήναι αρεστόν εις τον βασιλέα, η ζωή μου ας μοι δοθή εις το ζήτημά μου και ο λαός μου εις την αίτησίν μου·
\par 4 διότι επωλήθημεν, εγώ και ο λαός μου, εις απώλειαν, εις σφαγήν και εις όλεθρον· και εάν ηθέλομεν πωληθή ως δούλοι και δούλαι ήθελον σιωπήσει, αν και ο εχθρός δεν ηδύνατο να αναπληρώση την ζημίαν του βασιλέως.
\par 5 Τότε απεκρίθη ο βασιλεύς Ασσουήρης και είπε προς την Εσθήρ την βασίλισσαν, Τις είναι αυτός και που είναι εκείνος, όστις ετόλμησε να κάμη ούτω;
\par 6 Και είπεν η Εσθήρ, Ο εναντίος και εχθρός είναι ούτος ο αχρείος Αμάν. Τότε εταράχθη ο Αμάν ενώπιον του βασιλέως και της βασιλίσσης.
\par 7 Και σηκωθείς ο βασιλεύς από του συμποσίου του οίνου ωργισμένος υπήγεν εις τον κήπον του παλατίου· ο δε Αμάν εστάθη, διά να ζητήση την ζωήν αυτού παρά της Εσθήρ της βασιλίσσης· διότι είδεν ότι κακόν ήτο αποφασισμένον εναντίον αυτού παρά του βασιλέως.
\par 8 Και επέστρεψεν ο βασιλεύς από του κήπου του παλατίου εις τον οίκον του συμποσίου του οίνου· ο δε Αμάν ήτο πεπτωκώς επί της κλίνης εφ' ης ήτο η Εσθήρ. Και είπεν ο βασιλεύς, Θέλει έτι και την βασίλισσαν να βιάση έμπροσθέν μου εν τω οίκω; Ο λόγος εξήλθεν εκ του στόματος του βασιλέως και εσκέπασαν το πρόσωπον του Αμάν.
\par 9 Και είπεν ο Αρβονά, εις εκ των ευνούχων, ενώπιον του βασιλέως, Ιδού, και το ξύλον πεντήκοντα πηχών το ύψος, το οποίον ο Αμάν έκαμε διά τον Μαροδοχαίον, τον λαλήσαντα αγαθά υπέρ του βασιλέως, ίσταται εν τη οικία του Αμάν. Και είπεν ο βασιλεύς, Κρεμάσατε αυτόν επ' αυτού.
\par 10 Και εκρέμασαν τον Αμάν επί του ξύλου, το οποίον ητοίμασε διά τον Μαροδοχαίον. Και κατέπαυσεν ο θυμός του βασιλέως.

\chapter{8}

\par Εν τη ημέρα εκείνη ο βασιλεύς Ασσουήρης έδωκεν εις την Εσθήρ την βασίλισσαν τον οίκον του Αμάν, του εχθρού των Ιουδαίων. Και ήλθεν ο Μαροδοχαίος ενώπιον του βασιλέως· διότι η Εσθήρ εφανέρωσε τι ήτο αυτής.
\par 2 Και εκβαλών ο βασιλεύς το δακτυλίδιον αυτού, το οποίον αφήρεσεν από του Αμάν, έδωκεν αυτό εις τον Μαροδοχαίον. Και κατέστησεν η Εσθήρ τον Μαροδοχαίον επί τον οίκον του Αμάν.
\par 3 Και ελάλησε πάλιν η Εσθήρ ενώπιον του βασιλέως, και προσέπεσεν εις τους πόδας αυτού και ικέτευσεν αυτόν μετά δακρύων να ακυρώση την κακίαν του Αμάν του Αγαγίτου, και την σκευωρίαν αυτού την οποίαν εσκευώρησε κατά των Ιουδαίων.
\par 4 Και εξέτεινεν ο βασιλεύς το χρυσούν σκήπτρον προς την Εσθήρ. Τότε σηκωθείσα η Εσθήρ εστάθη ενώπιον του βασιλέως,
\par 5 και είπεν, Εάν ήναι αρεστόν εις τον βασιλέα, και εάν εύρηκα χάριν ενώπιον αυτού, και το πράγμα φαίνηται ορθόν εις τον βασιλέα και αρέσκηται εις εμέ, ας γραφή να ανακαλεσθώσι τα γράμματα τα σκευωρηθέντα υπό του Αμάν του υιού Αμμεδαθά του Αγαγίτου, τα οποία έγραψε διά να απολεσθώσιν οι Ιουδαίοι οι εν πάσαις ταις επαρχίαις του βασιλέως·
\par 6 διότι πως δύναμαι να υποφέρω να ίδω το κακόν, το οποίον θέλει ευρεί τον λαόν μου; ή πως δύναμαι να υποφέρω να ίδω τον αφανισμόν της συγγενείας μου;
\par 7 Τότε είπεν ο βασιλεύς Ασσουήρης προς Εσθήρ την βασίλισσαν και προς τον Μαροδοχαίον τον Ιουδαίον, Ιδού, έδωκα εις την Εσθήρ τον οίκον του Αμάν και αυτόν εκρέμασαν επί του ξύλου, διότι εξήπλωσε την χείρα αυτού κατά των Ιουδαίων·
\par 8 σεις λοιπόν γράψατε υπέρ των Ιουδαίων, όπως φαίνεται εις εσάς καλόν, εν ονόματι του βασιλέως, και σφραγίσατε με το βασιλικόν δακτυλίδιον· διότι το γράμμα το γεγραμμένον εν ονόματι του βασιλέως και εσφραγισμένον με το βασιλικόν δακτυλίδιον, είναι αμετάτρεπτον.
\par 9 Και προσεκλήθησαν οι γραμματείς του βασιλέως εν τω καιρώ εκείνω εν τω τρίτω μηνί, ούτος είναι ο μην Σιβάν, την εικοστήν τρίτην αυτού· και εγράφη κατά πάντα όσα ο Μαροδοχαίος προσέταξε, προς τους Ιουδαίους και προς τους σατράπας και διοικητάς και άρχοντας των επαρχιών των από Ινδίας έως Αιθιοπίας, εκατόν εικοσιεπτά επαρχιών, εις πάσαν επαρχίαν κατά το γράφειν αυτής και προς πάντα λαόν κατά την γλώσσαν αυτού και προς τους Ιουδαίους κατά το γράφειν αυτών και κατά την γλώσσαν αυτών.
\par 10 Και έγραψεν εν ονόματι του βασιλέως Ασσουήρου και εσφράγισεν αυτό με το βασιλικόν δακτυλίδιον και εξαπέστειλε τα γράμματα διά ταχυδρόμων εφίππων, ιππαζόντων επί ταχυπόδων και γενναίων ημιόνων·
\par 11 δι' ων επέτρεπεν ο βασιλεύς εις τους Ιουδαίους τους κατά πάσαν πόλιν, να συναχθώσι και να σταθώσιν υπέρ της ζωής αυτών, να απολέσωσι, να φονεύσωσι και να αφανίσωσι πάσαν την δύναμιν του λαού και της επαρχίας των καταθλιβόντων αυτούς, παιδία και γυναίκας, και τα λάφυρα αυτών να διαρπάσωσιν,
\par 12 εν μιά ημέρα, κατά πάσας τας επαρχίας του βασιλέως Ασσουήρου, τη δεκάτη τρίτη του δωδεκάτου μηνός, ούτος είναι ο μην Αδάρ.
\par 13 Το αντίγραφον της επιστολής, το προς διάδοσιν του προστάγματος κατά πάσαν επαρχίαν, εδημοσιεύθη προς πάντας τους λαούς, διά να ήναι οι Ιουδαίοι έτοιμοι κατ' εκείνην την ημέραν να εκδικηθώσιν εναντίον των εχθρών αυτών.
\par 14 Και εξήλθον οι ταχυδρόμοι, ιππάζοντες επί ταχυπόδων ημιόνων, σπεύδοντες και κατεπειγόμενοι υπό της προσταγής του βασιλέως. Και η διαταγή εξεδόθη εν Σούσοις τη βασιλευούση.
\par 15 Ο δε Μαροδοχαίος εξήλθεν απ' έμπροσθεν του βασιλέως εν στολή βασιλική κυανή και λευκή και φορών μέγαν στέφανον χρυσούν και επένδυμα βύσσινον και πορφυρούν· και η πόλις Σούσα έχαιρε και ευφραίνετο.
\par 16 Εις τους Ιουδαίους ήτο φως και αγαλλίασις και χαρά και δόξα.
\par 17 Και εν πάση επαρχία και εν πάση πόλει, όπου ήλθε του βασιλέως το πρόσταγμα και η διαταγή, έγεινεν εις τους Ιουδαίους χαρά και αγαλλίασις, ευωχία και ημέρα αγαθή. Και πολλοί εκ των λαών της γης έγειναν Ιουδαίοι διότι ο φόβος των Ιουδαίων επέπεσεν επ' αυτούς.

\chapter{9}

\par Εν δε τω δωδεκάτω μηνί, ούτος είναι ο μην Αδάρ, τη δεκάτη τρίτη ημέρα του αυτού, ότε το πρόσταγμα του βασιλέως και η διαταγή αυτού ήτο πλησίον να εκτελεσθή, εν τη ημέρα καθ' ην οι εχθροί των Ιουδαίων ήλπιζον να κατακρατήσωσιν αυτών, αν και ετράπη εις το εναντίον, διότι οι Ιουδαίοι κατεκράτησαν των μισούντων αυτούς,
\par 2 συνήχθησαν οι Ιουδαίοι εν ταις πόλεσιν αυτών κατά πάσας τας επαρχίας του βασιλέως Ασσουήρου, διά να επιβάλωσι χείρα επί τους ζητούντας το κακόν αυτών· και ουδείς ηδυνήθη να αντισταθή εις αυτούς, διότι ο φόβος αυτών επέπεσεν επί πάντας τους λαούς.
\par 3 Και πάντες οι άρχοντες των επαρχιών και οι σατράπαι και οι διοικηταί και οι οικονόμοι του βασιλέως εβοήθουν τους Ιουδαίους· διότι ο φόβος του Μαροδοχαίου επέπεσεν επ' αυτούς·
\par 4 επειδή ο Μαροδοχαίος ήτο μέγας εν τω οίκω του βασιλέως και η φήμη αυτού διεδόθη εις πάσας τας επαρχίας· διότι ο άνθρωπος ο Μαροδοχαίος προέβαινε μεγαλυνόμενος.
\par 5 Και επάταξαν οι Ιουδαίοι πάντας τους εχθρούς αυτών με πάταγμα ρομφαίας και σφαγήν και όλεθρον, και έκαμον εις τους μισούντας αυτούς όπως ήθελον.
\par 6 Και εν Σούσοις τη βασιλευούση εφόνευσαν οι Ιουδαίοι και απώλεσαν πεντακοσίους άνδρας.
\par 7 Και τον Φαρσανδαθά και τον Δαλφών και τον Ασπαθά
\par 8 και τον Ποραθά και τον Αδαλία και τον Αριδαθά
\par 9 και τον Φαρμαστά και τον Αρισαΐ και τον Αριδαΐ και τον Βαϊεζαθά,
\par 10 τους δέκα υιούς του Αμάν υιού του Αμμεδαθά, του εχθρού των Ιουδαίων, εφόνευσαν· επί λάφυρα όμως δεν έβαλον την χείρα αυτών.
\par 11 Εν τη ημέρα εκείνη ο αριθμός των φονευθέντων εν Σούσοις τη βασιλευούση εφέρθη ενώπιον του βασιλέως.
\par 12 Και είπεν ο βασιλεύς προς Εσθήρ την βασίλισσαν, Εν Σούσοις τη βασιλευούση εφόνευσαν οι Ιουδαίοι και απώλεσαν πεντακοσίους άνδρας και τους δέκα υιούς του Αμάν· εν ταις λοιπαίς επαρχίαις του βασιλέως τι έκαμον; τώρα τι το ζήτημά σου; και θέλει δοθή εις σέ· και τις έτι η αίτησίς σου; και θέλει γείνει.
\par 13 Και είπεν η Εσθήρ, Εάν ήναι αρεστόν εις τον βασιλέα, ας δοθή εις τους Ιουδαίους τους εν Σούσοις, να κάμωσι και αύριον κατά την διαταγήν της ημέρας ταύτης· και τους δέκα υιούς του Αμάν να κρεμάσωσιν επί ξύλων.
\par 14 Και προσέταξεν ο βασιλεύς να γείνη ούτω· και εξεδόθη διαταγή εν Σούσοις· και εκρέμασαν τους δέκα υιούς του Αμάν.
\par 15 Και συνήχθησαν οι Ιουδαίοι οι εν Σούσοις και την δεκάτην τετάρτην του μηνός Αδάρ και εφόνευσαν τριακοσίους άνδρας εν Σούσοις· επί λάφυρα όμως δεν έβαλον την χείρα αυτών.
\par 16 Οι δε άλλοι Ιουδαίοι, οι εν ταις επαρχίαις του βασιλέως, συνήχθησαν και εστάθησαν υπέρ της ζωής αυτών, και έλαβον ανάπαυσιν από των εχθρών αυτών και εφόνευσαν εκ των μισούντων αυτούς εβδομήκοντα πέντε χιλιάδας· επί τα λάφυρα όμως δεν έβαλον την χείρα αυτών·
\par 17 την δεκάτην τρίτην ημέραν του μηνός Αδάρ· και την δεκάτην τετάρτην ημέραν του αυτού ανεπαύθησαν και έκαμον ταύτην ημέραν συμποσίου και ευφροσύνης.
\par 18 Οι δε Ιουδαίοι οι εν Σούσοις συνήχθησαν την δεκάτην τρίτην αυτού και την δεκάτην τετάρτην αυτού· την δε δεκάτην πέμπτην του αυτού ανεπαύθησαν και έκαμον ταύτην ημέραν συμποσίου και ευφροσύνης.
\par 19 Διά τούτο οι Ιουδαίοι οι χωρικοί οι κατοικούντες εν ταις ατειχίστοις πόλεσιν έκαμνον την δεκάτην τετάρτην ημέραν του μηνός Αδάρ ημέραν ευφροσύνης και συμποσίου και ημέραν αγαθήν, και απέστελλον μερίδας προς αλλήλους.
\par 20 Και έγραψεν ο Μαροδοχαίος τα πράγματα ταύτα και απέστειλεν επιστολάς προς πάντας τους Ιουδαίους τους εν πάσαις ταις επαρχίαις τον βασιλέως Ασσουήρου, τους πλησίον και τους μακράν,
\par 21 προσδιορίζων εις αυτούς να φυλάττωσι την δεκάτην τετάρτην ημέραν του μηνός Αδάρ και την δεκάτην πέμπτην του αυτού καθ' έκαστον έτος,
\par 22 ως τας ημέρας καθ' ας οι Ιουδαίοι ανεπαύθησαν από των εχθρών αυτών, και τον μήνα καθ' ον η λύπη αυτών ετράπη εις αυτούς εις χαράν και το πένθος εις ημέραν αγαθήν· ώστε να κάμνωσιν αυτάς ημέρας συμποσίου και ευφροσύνης και να αποστέλλωσι μερίδας προς αλλήλους και δώρα προς τους πτωχούς.
\par 23 Και εδέχθησαν οι Ιουδαίοι εκείνο το οποίον ήρχισαν να κάμνωσι και εκείνο το οποίον έγραψεν ο Μαροδοχαίος προς αυτούς·
\par 24 διότι ο Αμάν ο υιός του Αμμεδαθά, ο Αγαγίτης, ο εχθρός πάντων των Ιουδαίων, εσκευώρησε κατά των Ιουδαίων να απολέση αυτούς, και έρριψε Φούρ, ήγουν κλήρον, διά να αναλώση αυτούς και να αφανίση αυτούς·
\par 25 Ότε όμως ήλθεν αυτή η Εσθήρ ενώπιον του βασιλέως, προσέταξε δι' επιστολών να τραπή κατά της κεφαλής αυτού η κακή αυτού σκευωρία, την οποίαν εσκευώρησε κατά των Ιουδαίων, και εκρέμασαν επί του ξύλου αυτόν και τους υιούς αυτού.
\par 26 Διά τούτο ωνόμασαν τας ημέρας ταύτας Φουρείμ εκ του ονόματος Φούρ. Όθεν διά πάντας τους λόγους της επιστολής ταύτης, και δι' εκείνο το οποίον είδον περί του πράγματος τούτου και το οποίον συνέβη εις αυτούς,
\par 27 διέταξαν οι Ιουδαίοι, και εδέχθησαν εφ' εαυτούς και επί το σπέρμα αυτών και επί πάντας τους προστιθεμένους εις αυτούς, να μη λείψωσι ποτέ από του να φυλάττωσι τας δύο ταύτας ημέρας, κατά το γεγραμμένον περί αυτών και κατά τον καιρόν αυτών εκάστου έτους·
\par 28 και αι ημέραι αύται να μνημονεύωνται και να φυλάττωνται εν πάση γενεά, εκάστη συγγενεία, εκάστη επαρχία, και εκάστη πόλει και αι ημέραι αύται Φουρείμ να μη εκλείψωσιν εκ μέσου των Ιουδαίων, και να μη παύση το μνημόσυνον αυτών από του σπέρματος αυτών.
\par 29 Τότε η Εσθήρ η βασίλισσα, η θυγάτηρ του Αβιχαίλ, και ο Μαροδοχαίος ο Ιουδαίος, έγραψαν εκ δευτέρου μεθ' όλου του κύρους, διά να στερεώσωσι ταύτα τα περί Φουρείμ γεγραμμένα.
\par 30 Και έπεμψεν επιστολάς προς πάντας τους Ιουδαίους, εις τας εκατόν εικοσιεπτά επαρχίας του βασιλείου του Ασσουήρου, με λόγους ειρήνης και αληθείας,
\par 31 διά να στερεώση τας ημέρας ταύτας Φουρείμ εν τοις καιροίς αυτών, καθώς προσδιώρισαν εις αυτούς ο Μαροδοχαίος ο Ιουδαίος και Εσθήρ η βασίλισσα, και καθώς διώρισαν, εφ' εαυτούς και επί το σπέρμα αυτών, την υπόθεσιν των νηστειών και της κραυγής αυτών.
\par 32 Και διά διαταγής της Εσθήρ εκυρώθη η υπόθεσις αύτη των Φουρείμ, και εγράφη εν βιβλίω.

\chapter{10}

\par Και επέβαλεν ο βασιλεύς Ασσουήρης φόρον επί την γην και τας νήσους της θαλάσσης.
\par 2 Πάσαι δε αι πράξεις της δυνάμεως αυτού και της ισχύος αυτού, και η περιγραφή μεγαλειότητος του Μαροδοχαίου, εις ην ο βασιλεύς προεβίβασεν αυτόν, δεν είναι γεγραμμένα εν τω βιβλίω των χρονικών των βασιλέων της Μηδίας και Περσίας;
\par 3 Διότι ο Μαροδοχαίος ο Ιουδαίος εστάθη δεύτερος μετά τον βασιλέα Ασσουήρην και μέγας μεταξύ των Ιουδαίων και αγαπώμενος υπό του πλήθους των αδελφών αυτού, ζητών το καλόν του λαού αυτού και λαλών ειρήνην περί παντός του σπέρματος αυτού.


\end{document}