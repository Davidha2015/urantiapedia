\begin{document}

\title{Job}


\chapter{1}

\par Άνθρωπος τις ήτο εν τη γη της Αυσίτιδος ονομαζόμενος Ιώβ· και ο άνθρωπος ούτος ήτο άμεμπτος και ευθύς και φοβούμενος τον Θεόν και απεχόμενος από κακού.
\par 2 Και εγεννήθησαν εις αυτόν επτά υιοί και τρεις θυγατέρες.
\par 3 Και ήσαν τα κτήνη αυτού επτακισχίλια πρόβατα και τρισχίλιαι κάμηλοι και πεντακόσια ζεύγη βοών και πεντακόσιαι όνοι και πλήθος πολύ υπηρετών· και ήτο ο άνθρωπος εκείνος ο μεγαλήτερος πάντων των κατοίκων της Ανατολής.
\par 4 Και υπήγαινον οι υιοί αυτού και έκαμνον συμπόσια εν ταις οικίαις αυτών, έκαστος κατά την ημέραν αυτού, και έστελλον και προσεκάλουν τας τρεις αδελφάς αυτών διά να τρώγωσι και να πίνωσι μετ' αυτών.
\par 5 Και ότε ετελείονον αι ημέραι του συμποσίου, έστελλεν ο Ιώβ και ηγίαζεν αυτούς, και εξεγειρόμενος πρωΐ προσέφερεν ολοκαυτώματα κατά τον αριθμόν πάντων αυτών· διότι έλεγεν ο Ιώβ, Μήπως οι υιοί μου ημάρτησαν και εβλασφήμησαν τον Θεόν εν τη καρδία αυτών. Ούτως έκαμνεν ο Ιώβ, πάντοτε.
\par 6 Ημέραν δε τινά ήλθον οι υιοί του Θεού διά να παρασταθώσιν ενώπιον του Κυρίου, και μεταξύ αυτών ήλθε και ο Σατανάς.
\par 7 Και είπεν ο Κύριος προς τον Σατανάν, Πόθεν έρχεσαι; Και ο Σατανάς απεκρίθη προς τον Κύριον και είπε, Περιελθών την γην και εμπεριπατήσας εν αυτή πάρειμι.
\par 8 Και είπεν ο Κύριος προς τον Σατανάν, Έβαλες τον νούν σου επί τον δούλον μου Ιώβ, ότι δεν υπάρχει όμοιος αυτού εν τη γη, άνθρωπος άμεμπτος και ευθύς, φοβούμενος τον Θεόν και απεχόμενος από κακού;
\par 9 Και απεκρίθη ο Σατανάς προς τον Κύριον και είπε, Μήπως δωρεάν φοβείται ο Ιώβ τον Θεόν;
\par 10 δεν περιέφραξας κυκλόθεν αυτόν και την οικίαν αυτού και πάντα όσα έχει; τα έργα των χειρών αυτού ευλόγησας, και τα κτήνη αυτού επληθύνθησαν επί της γής·
\par 11 πλην τώρα έκτεινον την χείρα σου και έγγισον πάντα όσα έχει, διά να ίδης εάν δεν σε βλασφημήση κατά πρόσωπον.
\par 12 Και είπεν ο Κύριος προς τον Σατανάν, Ιδού, εις την χείρα σου πάντα όσα έχει· μόνον επ' αυτόν μη επιβάλης την χείρα σου. Και εξήλθεν ο Σατανάς απ' έμπροσθεν του Κυρίου.
\par 13 Ημέραν δε τινά οι υιοί αυτού και αι θυγατέρες αυτού έτρωγον και έπινον οίνον εν τη οικία του αδελφού αυτών του πρωτοτόκου.
\par 14 Και ήλθε μηνυτής προς τον Ιώβ και είπεν, Οι βόες ηροτρίαζον και αι όνοι έβοσκον πλησίον αυτών·
\par 15 και επέπεσαν οι Σαβαίοι και ήρπασαν αυτά· και τους δούλους επάταξαν εν στόματι μαχαίρας· και εγώ μόνος διεσώθην διά να σοι απαγγείλω.
\par 16 Ενώ ούτος έτι ελάλει, ήλθε και άλλος και είπε, Πυρ Θεού έπεσεν εξ ουρανού και έκαυσε τα πρόβατα και τους δούλους και κατέφαγεν αυτούς· και εγώ μόνος διεσώθην διά να σοι απαγγείλω.
\par 17 Ενώ ούτος έτι ελάλει, ήλθε και άλλος και είπεν, Οι Χαλδαίοι έκαμον τρεις λόχους και εφώρμησαν εις τας καμήλους και ήρπασαν αυτάς· και τους δούλους επάταξαν εν στόματι μαχαίρας· και εγώ μόνος διεσώθην διά να σοι απαγγείλω.
\par 18 Ενώ ούτος έτι ελάλει, ήλθε και άλλος και είπεν, Οι υιοί σου και αι θυγατέρες σου έτρωγον και έπινον οίνον εν τη οικία του αδελφού αυτών του πρωτοτόκου·
\par 19 και ιδού, ήλθε μέγας άνεμος εκ του πέραν της ερήμου και προσέβαλε τας τέσσαρας γωνίας του οίκου και έπεσεν επί τα παιδία, και απέθανον· και εγώ μόνος διεσώθην διά να σοι απαγγείλω.
\par 20 Τότε σηκωθείς ο Ιώβ διέσχισε το επένδυμα αυτού και εξύρισε την κεφαλήν αυτού και έπεσεν επί την γην και προσεκύνησε,
\par 21 και είπε, Γυμνός εξήλθον εκ κοιλίας μητρός μου και γυμνός θέλω επιστρέψει εκεί· ο Κύριος έδωκε και ο Κύριος αφήρεσεν· είη το όνομα Κυρίου ευλογημένον.
\par 22 Εν πάσι τούτοις δεν ημάρτησεν ο Ιώβ και δεν έδωκεν αφροσύνην εις τον Θεόν.

\chapter{2}

\par Ημέραν δε τινά ήλθον οι υιοί του Θεού διά να παρασταθώσιν ενώπιον του Κυρίου· και μεταξύ αυτών ήλθε και ο Σατανάς, διά να παρασταθή ενώπιον του Κυρίου.
\par 2 Και είπεν ο Κύριος προς τον Σατανάν, Πόθεν έρχεσαι; Και ο Σατανάς απεκρίθη προς τον Κύριον και είπε, Περιελθών την γην και εμπεριπατήσας εν αυτή πάρειμι.
\par 3 Και είπεν ο Κύριος προς τον Σατανάν, Έβαλες τον νούν σου επί τον δούλον μου Ιώβ, ότι δεν υπάρχει όμοιος αυτού εν τη γη, άνθρωπος άμεμπτος και ευθύς, φοβούμενος τον Θεόν και απεχόμενος από κακού; και έτι κρατεί την ακεραιότητα αυτού, αν και με παρώξυνας κατ' αυτού, διά να εξολοθρεύσω αυτόν άνευ αιτίας.
\par 4 Και απεκρίθη ο Σατανάς προς τον Κύριον και είπε, Δέρμα υπέρ δέρματος, και πάντα όσα έχει ο άνθρωπος θέλει δώσει υπέρ της ζωής αυτού·
\par 5 πλην τώρα έκτεινον την χείρα σου και έγγισον τα οστά αυτού και την σάρκα αυτού, διά να ίδης εάν δεν σε βλασφημήση κατά πρόσωπον.
\par 6 Και είπεν ο Κύριος προς τον Σατανάν, Ιδού, αυτός εις την χείρα σου· μόνον την ζωήν αυτού φύλαξον.
\par 7 Τότε εξήλθεν ο Σατανάς απ' έμπροσθεν του Κυρίου και επάταξε τον Ιώβ με έλκος κακόν από του ίχνους των ποδών αυτού έως της κορυφής αυτού.
\par 8 Και έλαβεν εις εαυτόν όστρακον, διά να ξύηται με αυτό· και εκάθητο εν μέσω της σποδού.
\par 9 Τότε είπε προς αυτόν γυνή αυτού, Έτι κρατείς την ακεραιότητά σου; Βλασφήμησον τον Θεόν και απόθανε.
\par 10 Ο δε είπε προς αυτήν, Ελάλησας ως λαλεί μία εκ των αφρόνων γυναικών· τα αγαθά μόνον θέλομεν δεχθή εκ του Θεού, και τα κακά δεν θέλομεν δεχθή; Εν πάσι τούτοις δεν ημάρτησεν ο Ιώβ με τα χείλη αυτού.
\par 11 Ακούσαντες δε οι τρεις φίλοι του Ιώβ πάντα ταύτα τα κακά τα επελθόντα επ' αυτόν, ήλθον έκαστος εκ του τόπου αυτού· Ελιφάς ο Θαιμανίτης και Βιλδάδ ο Σαυχίτης και Σωφάρ ο Νααμαθίτης· διότι είχον συμφωνήσει να έλθωσιν ομού, διά να συλλυπηθώσιν αυτόν και να παρηγορήσωσιν αυτόν.
\par 12 Και ότε εσήκωσαν τους οφθαλμούς αυτών μακρόθεν και δεν εγνώρισαν αυτόν, ύψωσαν την φωνήν αυτών και έκλαυσαν· και διέσχισαν έκαστος το ιμάτιον αυτού και έρριψαν χώμα επί τας κεφαλάς αυτών προς τον ουρανόν.
\par 13 Και εκάθησαν μετ' αυτού κατά γης επτά ημέρας και επτά νύκτας, και ουδείς ελάλησε λόγον προς αυτόν, διότι έβλεπον ότι ο πόνος αυτού ήτο μέγας σφόδρα.

\chapter{3}

\par Μετά ταύτα ήνοιξεν ο Ιώβ το στόμα αυτού, και κατηράσθη την ημέραν αυτού.
\par 2 Και ελάλησεν ο Ιώβ και είπεν·
\par 3 Είθε να χαθή η ημέρα καθ' ην εγεννήθην, και η νυξ καθ' ην είπον, Εγεννήθη αρσενικόν.
\par 4 Η ημέρα εκείνη να ήναι σκότος· ο Θεός να μη αναζητήση αυτήν άνωθεν, και να μη φέγξη επ' αυτήν φως.
\par 5 Σκότος και σκιά θανάτου να αμαυρώσωσιν αυτήν· γνόφος να επικάθηται επ' αυτήν. Να επέλθωσιν επ' αυτήν ως πικροτάτην ημέραν.
\par 6 Την νύκτα εκείνην να κατακρατήση σκότος· να μη συναφθή με τας ημέρας του έτους· να μη εισέλθη εις τον αριθμόν των μηνών.
\par 7 Ιδού, έρημος να ήναι η νυξ εκείνη· φωνή χαρμόσυνος να μη επέλθη επ' αυτήν.
\par 8 Να καταρασθώσιν αυτήν οι καταρώμενοι τας ημέρας, οι έτοιμοι να ανεγείρωσι το πένθος αυτών.
\par 9 Να σκοτισθώσι τα άστρα της εσπέρας αυτής· να προσμένη το φως, και να μη έρχηται· και να μη ίδη τα βλέφαρα της αυγής·
\par 10 διότι δεν έκλεισε τας θύρας της κοιλίας της μητρός μου, και δεν έκρυψε την θλίψιν από των οφθαλμών μου.
\par 11 Διά τι δεν απέθανον από μήτρας; και δεν εξέπνευσα άμα εξήλθον εκ της κοιλίας;
\par 12 Διά τι με υπεδέχθησαν τα γόνατα; ή διά τι οι μαστοί διά να θηλάσω;
\par 13 Διότι τώρα ήθελον κοιμάσθαι και ησυχάζει· ήθελον υπνώττει· τότε ήθελον είσθαι εις ανάπαυσιν,
\par 14 μετά βασιλέων και βουλευτών της γης, οικοδομούντων εις εαυτούς ερημώσεις·
\par 15 ή μετά αρχόντων, οίτινες έχουσι χρυσίον, οίτινες εγέμισαν τους οίκους αυτών αργυρίου·
\par 16 ή ως εξάμβλωμα κεκρυμμένον δεν ήθελον υπάρχει, ως βρέφη μη ιδόντα φως.
\par 17 Εκεί οι ασεβείς παύουσιν από του να ταράττωσι, και εκεί αναπαύονται οι κεκοπιασμένοι·
\par 18 εκεί αναπαύονται ομού οι αιχμάλωτοι· δεν ακούουσι φωνήν καταδυνάστου·
\par 19 εκεί είναι ο μικρός και ο μέγας· και ο δούλος, ελεύθερος του κυρίου αυτού.
\par 20 Διά τι εδόθη φως εις τον δυστυχή, και ζωή εις τον πεπικραμένον την ευχήν,
\par 21 οίτινες ποθούσι τον θάνατον και δεν επιτυγχάνουσιν, αν και ανορύττωσιν αυτόν μάλλον παρά κεκρυμμένους θησαυρούς,
\par 22 οίτινες υπερχαίρουσιν, υπερευφραίνονται, όταν εύρωσι τον τάφον;
\par 23 Διά τι εδόθη φως εις άνθρωπον, του οποίου η οδός είναι κεκρυμμένη, και τον οποίον ο Θεός περιέκλεισε;
\par 24 Διότι προ του φαγητού μου έρχεται ο στεναγμός μου, και οι βρυγμοί μου εκχέονται ως ύδατα.
\par 25 Επειδή εκείνο, το οποίον εφοβούμην, συνέβη εις εμέ, και εκείνο, το οποίον ετρόμαζον, ήλθεν επ' εμέ.
\par 26 Δεν είχον ειρήνην ουδέ ανάπαυσιν ουδέ ησυχίαν· οργή επήλθεν επ' εμέ.

\chapter{4}

\par Τότε Ελιφάς ο Θαιμανίτης απεκρίθη και είπεν·
\par 2 Εάν επιχειρισθώμεν να λαλήσωμεν προς σε, θέλεις δυσαρεστηθή; αλλά τις δύναται να κρατηθή από του να ομιλήση;
\par 3 Ιδού, συ ενουθέτησας πολλούς· και χείρας αδυνάτους ενίσχυσας.
\par 4 Οι λόγοι σου υπεστήριξαν τους κλονιζομένους, και γόνατα κάμπτοντα ενεδυνάμωσας.
\par 5 Τώρα δε ήλθεν επί σε τούτο, και βαρυθυμείς· σε εγγίζει, και ταράττεσαι.
\par 6 Ο φόβος σου δεν είναι το θάρρος σου, και η ευθύτης των οδών σου η ελπίς σου;
\par 7 Ενθυμήθητι, παρακαλώ· τις αθώος ων απωλέσθη; και που εξωλοθρεύθησαν οι ευθείς;
\par 8 Καθώς εγώ είδον, όσοι ηροτρίασαν ανομίαν και έσπειραν ασέβειαν, θερίζουσιν αυτάς·
\par 9 εξολοθρεύονται υπό του φυσήματος του Θεού, και από της πνοής των μυκτήρων αυτού αφανίζονται·
\par 10 ο βρυγμός του λέοντος και η φωνή του αγρίου λέοντος και το γαυρίαμα των σκύμνων, εσβέσθησαν·
\par 11 ο λέων απόλλυται δι' έλλειψιν θηράματος, και οι σκύμνοι της λεαίνας διασκορπίζονται.
\par 12 Και λόγος ήλθεν επ' εμέ κρυφίως, και το ωτίον μου έλαβέ τι παρ' αυτού.
\par 13 Εν μέσω των στοχασμών διά τα οράματα της νυκτός, ότε βαθύς ύπνος πίπτει επί τους ανθρώπους,
\par 14 Φρίκη συνέλαβέ με και τρόμος, και μεγάλως τα οστά μου συνέσεισε.
\par 15 Και πνεύμα διήλθεν απ' έμπροσθέν μου, αι τρίχες του σώματός μου ανεσηκώθησαν·
\par 16 εστάθη, αλλ' εγώ δεν διέκρινα την μορφήν αυτού· σχήμα εφάνη έμπροσθεν των οφθαλμών μου· ήκουσα λεπτόν φύσημα και φωνήν λέγουσαν,
\par 17 Ο άνθρωπος θέλει είσθαι δικαιότερος του Θεού; θέλει είσθαι ο άνθρωπος καθαρώτερος του Ποιητού αυτού;
\par 18 Ιδού, αυτός δεν εμπιστεύεται εις τους δούλους αυτού, και εν τοις αγγέλοις αυτού βλέπει ελάττωμα·
\par 19 πόσω μάλλον εις τους κατοικούντας οικίας πηλίνας, αίτινες έχουσι το θεμέλιον αυτών εν τω χώματι και αφανίζονται έμπροσθεν του σαρακίου;
\par 20 Από πρωΐ έως εσπέρας φθείρονται· χωρίς να νοήση τις, αφανίζονται διά παντός.
\par 21 Το μεγαλείον αυτών το εν αυτοίς δεν παρέρχεται; Αποθνήσκουσιν, αλλ' ουχί εν σοφία.

\chapter{5}

\par Κάλεσον τώρα, εάν τις σοι αποκριθή; και προς τίνα των αγίων θέλεις αποβλέψει;
\par 2 Διότι η οργή φονεύει τον άφρονα, και η αγανάκτησις θανατόνει τον μωρόν.
\par 3 Εγώ είδον τον άφρονα ριζούμενον· αλλ' ευθύς προείπα κατηραμένην την κατοικίαν αυτού.
\par 4 Οι υιοί αυτού είναι μακράν από της σωτηρίας, και καταπιέζονται έμπροσθεν της πύλης, και ουδείς ο ελευθερών·
\par 5 των οποίων τον θερισμόν κατατρώγει ο πεινών, και αρπάζει αυτόν εκ των ακανθών και την περιουσίαν αυτών καταπίνει ο διψών.
\par 6 Διότι εκ του χώματος δεν εξέρχεται η θλίψις, ουδέ η λύπη βλαστάνει εκ της γής·
\par 7 αλλ' ο άνθρωπος γεννάται διά την λύπην, και οι νεοσσοί των αετών διά να πετώσιν υψηλά.
\par 8 Αλλ' εγώ τον Θεόν θέλω επικαλεσθή, και εν τω Θεώ θέλω εναποθέσει την υπόθεσίν μου·
\par 9 όστις κάμνει μεγαλεία ανεξιχνίαστα, θαυμάσια αναρίθμητα·
\par 10 όστις δίδει βροχήν επί το πρόσωπον της γης, και πέμπει ύδατα επί το πρόσωπον των αγρών·
\par 11 όστις υψόνει τους ταπεινούς, και ανεγείρει εις σωτηρίαν τους τεθλιμμένους·
\par 12 όστις διασκεδάζει τας βουλάς των πανούργων, και δεν δύνανται αι χείρες αυτών να εκτελέσωσι την επιχείρησιν αυτών·
\par 13 όστις συλλαμβάνει τους σοφούς εν τη πανουργία αυτών· και η βουλή των δολίων ανατρέπεται·
\par 14 την ημέραν απαντώσι σκότος, και εν μεσημβρία ψηλαφώσι καθώς εν νυκτί.
\par 15 Τον πτωχόν όμως λυτρόνει εκ της ρομφαίας, εκ του στόματος αυτών και εκ της χειρός του ισχυρού.
\par 16 Και ο πτωχός έχει ελπίδα, της δε ανομίας το στόμα εμφράττεται.
\par 17 Ιδού, μακάριος ο άνθρωπος, τον οποίον ελέγχει ο Θεός· διά τούτο μη καταφρόνει την παιδείαν του Παντοδυνάμου·
\par 18 διότι αυτός πληγόνει και επιδένει· κτυπά, και αι χείρες αυτού ιατρεύουσιν.
\par 19 Εν εξ θλίψεσι θέλει σε ελευθερώσει· και εν τη εβδόμη δεν θέλει σε εγγίσει κακόν.
\par 20 Εν τη πείνη θέλει σε λυτρώσει εκ θανάτου· και εν πολέμω εκ χειρός ρομφαίας.
\par 21 Από μάστιγος γλώσσης θέλεις είσθαι πεφυλαγμένος· και δεν θέλεις φοβηθή από του επερχομένου ολέθρου.
\par 22 Τον όλεθρον και την πείναν θέλεις καταγελά· και δεν θέλεις φοβηθή από των θηρίων της γης.
\par 23 Διότι θέλεις έχει συμμαχίαν μετά των λίθων της πεδιάδος· και τα θηρία του αγρού θέλουσιν ειρηνεύει μετά σου.
\par 24 Και θέλεις γνωρίσει ότι ειρήνη είναι εν τη σκηνή σου, και θέλεις επισκεφθή την κατοικίαν σου, και δεν θέλει σοι λείπει ουδέν.
\par 25 Και θέλεις γνωρίσει ότι είναι πολύ το σπέρμα σου, και οι έκγονοί σου ως η βοτάνη της γης.
\par 26 Θέλεις ελθεί εις τον τάφον εν βαθεί γήρατι, καθώς συσσωρεύεται η θημωνία του σίτου εν τω καιρώ αυτής.
\par 27 Ιδού, τούτο εξιχνιάσαμεν, ούτως έχει· άκουσον αυτό και γνώρισον εν σεαυτώ.

\chapter{6}

\par Ο δε Ιώβ απεκρίθη και είπεν·
\par 2 Είθε να εζυγίζετο τωόντι η λύπη μου, και η συμφορά μου να ετίθετο όλη ομού εν τη πλάστιγγι.
\par 3 Επειδή τώρα ήθελεν είσθαι βαρυτέρα υπέρ την άμμον της θαλάσσης· διά τούτο οι λόγοι μου καταπίνονται.
\par 4 Διότι τα βέλη του Παντοδυνάμου είναι εντός μου, των οποίων το φαρμάκιον εκπίνει το πνεύμά μου· οι τρόμοι του Θεού παρατάττονται εναντίον μου.
\par 5 Ογκάται ο άγριος όνος παρά τη χλόη; ή μυκάται ο βους παρά τη φάτνη αυτού;
\par 6 Τρώγεται το άνοστον χωρίς άλατος; ή υπάρχει γεύσις εν τω λευκώματι του ωού;
\par 7 Τα πράγματα, τα οποία η ψυχή μου απεστρέφετο να εγγίση, έγειναν ως το αηδές φαγητόν μου.
\par 8 Είθε να απελάμβανον την αίτησίν μου, και να μοι έδιδεν ο Θεός την επιθυμίαν μου.
\par 9 Και να ήθελεν ευδοκήσει ο Θεός να με αφανίση· να απολύση την χείρα αυτού και να με κόψη.
\par 10 Και θέλει είσθαι έτι η παρηγορία μου, ότι, και αν καταναλωθώ εν τη θλίψει και αυτός δεν με λυπηθή, εγώ δεν έκρυψα τους λόγους του Αγίου.
\par 11 Ποία η δύναμίς μου, ώστε να εγκαρτερώ; και ποίον το τέλος μου, ώστε να υποφέρη η ψυχή μου;
\par 12 Μήπως η δύναμίς μου είναι δύναμις λίθων; ή η σαρξ μου χαλκός;
\par 13 Μήπως δεν εξέλιπεν εν εμοί η βοήθειά μου και απεμακρύνθη απ' εμού η σωτηρία;
\par 14 Εις τον τεθλιμμένον έλεος πρέπει παρά του φίλου αυτού· αλλ' αυτός εγκατέλιπε τον φόβον του Παντοδυνάμου.
\par 15 Οι αδελφοί μου εφέρθησαν απατηλώς ως χείμαρρος, ως ρεύμα χειμάρρων παρήλθον·
\par 16 οίτινες θολόνονται εκ του πάγου, εις τους οποίους διαλύεται η χιών·
\par 17 όταν θερμανθώσιν, εκλείπουσιν· όταν γείνη θερμότης, εξαλείφονται από του τόπου αυτών.
\par 18 Τα ίχνη της πορείας αυτών συστρέφονται· καταντώσιν εις το μηδέν και χάνονται·
\par 19 τα πλήθη της Θαιμά εθεώρουν, οι συνοδοιπόροι της Σεβά περιέμενον αυτούς·
\par 20 Εψεύσθησαν της ελπίδος αυτών· ήλθον εκεί και ενετράπησαν.
\par 21 Τώρα και σεις είσθε ως αυτοί· είδετε την πληγήν μου και ετρομάξατε.
\par 22 Μήπως εγώ είπα, Φέρετε προς εμέ; ή, Δότε δώρον εις εμέ από της περιουσίας υμών;
\par 23 ή, Ελευθερώσατέ με εκ της χειρός του εχθρού; ή, Λυτρώσατέ με εκ της χειρός των ισχυρών;
\par 24 Διδάξατέ με, και εγώ θέλω σιωπήσει· και δείξατέ μοι κατά τι έσφαλα.
\par 25 Πόσον ισχυροί είναι οι ορθοί λόγοι· αλλ' ο έλεγχός σας, τι αποδεικνύει;
\par 26 Φαντάζεσθε να ελέγξητε λόγους, ενώ αι ομιλίαι του απηλπισμένου είναι ως άνεμος;
\par 27 Τωόντι, σεις επιπίπτετε επί τον ορφανόν, και σκάπτετε λάκκον εις τον φίλον σας.
\par 28 Τώρα λοιπόν ευαρεστήθητε να εμβλέψητε εις εμέ, διότι έμπροσθεν υμών κείται αν εγώ ψεύδωμαι.
\par 29 Επιστρέψατε, παρακαλώ· ας μη γείνη αδικία· ναι, επιστρέψατε πάλιν· η δικαιοσύνη μου είναι εν τούτω.
\par 30 Υπάρχει αδικία εν τη γλώσση μου; δεν δύναται ο ουρανίσκος μου να διακρίνη τα διεφθαρμένα;

\chapter{7}

\par Δεν είναι εκστρατεία ο βίος του ανθρώπου επί της γης; αι ημέραι αυτού ως ημέραι μισθωτού;
\par 2 Καθώς ο δούλος επιποθεί την σκιάν, και καθώς ο μισθωτός αναμένει τον μισθόν αυτού,
\par 3 ούτως εγώ έλαβον διά κληρονομίαν μήνας ματαιότητος, και οδυνηραί νύκτες διωρίσθησαν εις εμέ.
\par 4 Όταν πλαγιάζω, λέγω, Πότε θέλω εγερθή, και θέλει περάσει η νυξ; και είμαι πλήρης ανησυχίας έως της αυγής·
\par 5 Η σαρξ μου είναι περιενδεδυμένη σκώληκας και βώλους χώματος· το δέρμα μου διασχίζεται και ρέει.
\par 6 Αι ημέραι μου είναι ταχύτεραι της κερκίδος του υφαντού, και χάνονται άνευ ελπίδος.
\par 7 Ενθυμήθητι ότι η ζωή μου είναι άνεμος· ο οφθαλμός μου δεν θέλει επιστρέψει διά να ίδη αγαθόν.
\par 8 Ο οφθαλμός του βλέποντός με δεν θέλει με ιδεί πλέον· οι οφθαλμοί σου είναι επ' εμέ, και εγώ δεν υπάρχω.
\par 9 Καθώς το νέφος διαλύεται και χάνεται ούτως ο καταβαίνων εις τον τάφον δεν θέλει επαναβή·
\par 10 δεν θέλει επιστρέψει πλέον εις τον οίκον αυτού, και ο τόπος αυτού δεν θέλει γνωρίσει αυτόν πλέον.
\par 11 Διά τούτο εγώ δεν θέλω κρατήσει το στόμα μου· θέλω λαλήσει εν τη αγωνία του πνεύματός μου· θέλω θρηνολογήσει εν τη πικρία της ψυχής μου.
\par 12 Θάλασσα είμαι ή κήτος, ώστε έθεσας επ' εμέ φυλακήν;
\par 13 Όταν λέγω, Η κλίνη μου θέλει με παρηγορήσει, η κοίτη μου θέλει ελαφρώσει το παράπονόν μου,
\par 14 τότε με φοβίζεις με όνειρα και με καταπλήττεις με οράσεις·
\par 15 και η ψυχή μου εκλέγει αγχόνην και θάνατον, παρά τα οστά μου.
\par 16 Αηδίασα· δεν θέλω ζήσει εις τον αιώνα· λείψον απ' εμού· διότι αι ημέραι μου είναι ματαιότης.
\par 17 Τι είναι ο άνθρωπος, ώστε μεγαλύνεις αυτόν, και βάλλεις τον νούν σου επ' αυτόν;
\par 18 Και επισκέπτεσαι αυτόν κατά πάσαν πρωΐαν και δοκιμάζεις αυτόν κατά πάσαν στιγμήν;
\par 19 Έως πότε δεν θέλεις συρθή απ' εμού και δεν θέλεις με αφήσει, έως να καταπίω τον σίελόν μου;
\par 20 Ημάρτησα· τι δύναμαι να κάμω εις σε, διατηρητά του ανθρώπου; διά τι με έθεσας σημάδιόν σου, και είμαι βάρος εις εμαυτόν;
\par 21 Και διά τι δεν συγχωρείς την παράβασίν μου και αφαιρείς την ανομίαν μου; διότι μετ' ολίγον θέλω κοιμάσθαι εν τω χώματι· και το πρωΐ θέλεις με ζητήσει, και δεν θέλω υπάρχει.

\chapter{8}

\par Και απεκρίθη Βιλδάδ ο Σαυχίτης και είπεν·
\par 2 Έως πότε θέλεις λαλεί ταύτα; και οι λόγοι του στόματός σου θέλουσιν είσθαι ως άνεμος σφοδρός;
\par 3 Μήπως ο Θεός ανατρέπει την κρίσιν; ή ο Παντοδύναμος ανατρέπει το δίκαιον;
\par 4 Εάν οι υιοί σου ημάρτησαν εις αυτόν, παρέδωκεν αυτούς εις την χείρα της ανομίας αυτών.
\par 5 Εάν συ ήθελες ζητήσει τον Θεόν πρωΐ, και ήθελες δεηθή του Παντοδυνάμου·
\par 6 εάν ήσο καθαρός και ευθύς, βεβαίως τώρα ήθελεν εγερθή διά σε, και ήθελεν ευτυχεί η κατοικία της δικαιοσύνης σου.
\par 7 Και αν η αρχή σου ήτο μικρά, τα ύστερά σου όμως ήθελον μεγαλυνθή σφόδρα.
\par 8 Επειδή ερώτησον, παρακαλώ, περί των προτέρων γενεών, και ερεύνησον ακριβώς περί των πατέρων αυτών·
\par 9 διότι ημείς είμεθα χθεσινοί, και δεν εξεύρομεν ουδέν, επειδή αι ημέραι ημών επί της γης είναι σκιά·
\par 10 δεν θέλουσι σε διδάξει αυτοί, και σοι ειπεί και προφέρει λόγους εκ της καρδίας αυτών;
\par 11 Θάλλει ο πάπυρος άνευ πηλού; αυξάνει ο σχοίνος άνευ ύδατος;
\par 12 Ενώ είναι έτι πράσινος και αθέριστος, ξηραίνεται προ παντός χόρτου.
\par 13 Ούτως είναι αι οδοί πάντων των λησμονούντων τον Θεόν· και η ελπίς του υποκριτού θέλει χαθή·
\par 14 η ελπίς αυτού θέλει κοπή, και το θάρρος αυτού θέλει είσθαι ιστός αράχνης.
\par 15 Θέλει επιστηριχθή επί την οικίαν αυτού, πλην αυτή δεν θέλει σταθή· θέλει κρατήσει αυτήν, πλην δεν θέλει ανορθωθή.
\par 16 Είναι χλωρός έμπροσθεν του ηλίου, και ο κλάδος αυτού απλόνεται εις τον κήπον αυτού.
\par 17 Αι ρίζαι αυτού περιπλέκονται εις τον σωρόν των λίθων, και εκλέγει τον πετρώδη τόπον.
\par 18 Εάν εξαλειφθή από του τόπου αυτού, τότε θέλει αρνηθή αυτόν, λέγων, Δεν σε είδον.
\par 19 Ιδού, αύτη είναι η χαρά της οδού αυτού, και εκ του χώματος άλλοι θέλουσι αναβλαστήσει.
\par 20 Ιδού, ο Θεός δεν θέλει απορρίψει τον άμεμπτον, ουδέ θέλει πιάσει την χείρα των κακοποιών·
\par 21 εωσού γεμίση το στόμα σου από γέλωτος, και τα χείλη σου αλαλαγμού.
\par 22 Οι μισούντές σε θέλουσιν ενδυθή αισχύνην· και η κατοικία των ασεβών δεν θέλει υπάρχει.

\chapter{9}

\par Και απεκρίθη ο Ιώβ και είπεν·
\par 2 Αληθώς εξεύρω ότι ούτως έχει· αλλά πως ο άνθρωπος θέλει δικαιωθή ενώπιον του Θεού;
\par 3 Εάν θελήση να διαδικασθή μετ' αυτού δεν δύναται να αποκριθή προς αυτόν εν εκ χιλίων.
\par 4 Είναι σοφός την καρδίαν και κραταιός την δύναμιν· τις εσκληρύνθη εναντίον αυτού και ευτύχησεν;
\par 5 Αυτός μετακινεί τα όρη, και δεν γνωρίζουσι τις έστρεψεν αυτά εν τη οργή αυτού.
\par 6 Αυτός σείει την γην από του τόπου αυτής, και οι στύλοι αυτής σαλεύονται.
\par 7 Αυτός προστάζει τον ήλιον, και δεν ανατέλλει· και κρύπτει υπό σφραγίδα τα άστρα.
\par 8 Αυτός μόνος εκτείνει τους ουρανούς και πατεί επί τα ύψη της θαλάσσης.
\par 9 Αυτός κάμνει τον Αρκτούρον, τον Ωρίωνα και την Πλειάδα και τα ταμεία του νότου.
\par 10 Αυτός κάμνει μεγαλεία ανεξιχνίαστα και θαυμάσια αναρίθμητα.
\par 11 Ιδού, διαβαίνει πλησίον μου, και δεν βλέπω αυτόν· διέρχεται, και δεν εννοώ αυτόν.
\par 12 Ιδού, αφαιρεί· τις θέλει εμποδίσει αυτόν; τις θέλει ειπεί προς αυτόν, Τι κάμνεις;
\par 13 Εάν ο Θεός δεν σύρη την οργήν αυτού, οι επηρμένοι βοηθοί καταβάλλονται υποκάτω αυτού.
\par 14 Πόσον ολιγώτερον εγώ ήθελον αποκριθή προς αυτόν, εκλέγων τους προς αυτόν λόγους μου;
\par 15 προς τον οποίον, και αν ήμην δίκαιος, δεν ήθελον αποκριθή, αλλ' ήθελον ζητήσει έλεος παρά του Κριτού μου.
\par 16 Εάν κράξω, και μοι αποκριθή, δεν ήθελον πιστεύσει ότι εισήκουσε της φωνής μου.
\par 17 Διότι με κατασυντρίβει με ανεμοστρόβιλον και πληθύνει τας πληγάς μου αναιτίως.
\par 18 Δεν με αφίνει να αναπνεύσω, αλλά με χορτάζει από πικρίας.
\par 19 Εάν πρόκηται περί δυνάμεως, ιδού, είναι δυνατός· και εάν περί κρίσεως, τις θέλει μαρτυρήσει υπέρ εμού;
\par 20 Εάν ήθελον να δικαιώσω εμαυτόν, το στόμα μου ήθελε με καταδικάσει· εάν ήθελον ειπεί, είμαι άμεμπτος, ήθελε με αποδείξει διεφθαρμένον.
\par 21 Και αν ήμην άμεμπτος, δεν ήθελον φροντίσει περί εμαυτού· ήθελον καταφρονήσει την ζωήν μου.
\par 22 Εν τούτο είναι, διά τούτο είπα, αυτός αφανίζει τον άμεμπτον και τον ασεβή.
\par 23 Και αν η μάστιξ αυτού θανατόνη ευθύς, γελά όμως εις την δοκιμασίαν των αθώων.
\par 24 Η γη παρεδόθη εις τας χείρας του ασεβούς· αυτός σκεπάζει τα πρόσωπα των κριτών αυτής· αν ουχί αυτός, που και τις είναι;
\par 25 Αι δε ημέραι μου είναι ταχυδρόμου ταχύτεραι· φεύγουσι και δεν βλέπουσι καλόν.
\par 26 Παρήλθον ως πλοία σπεύδοντα· ως αετός πετώμενος επί το θήραμα.
\par 27 Εάν είπω, Θέλω λησμονήσει το παράπονόν μου, θέλω παραιτήσει το πένθος μου και παρηγορηθή·
\par 28 τρομάζω διά πάσας τας θλίψεις μου, γνωρίζων ότι δεν θέλεις με αθωώσει.
\par 29 Είμαι ασεβής· διά τι λοιπόν να κοπιάζω εις μάτην;
\par 30 Εάν λουσθώ εν ύδατι χιόνος και επιμελώς αποκαθαρίσω τας χείρας μου·
\par 31 συ όμως θέλεις με βυθίσει εις τον βόρβορον, ώστε και αυτά μου τα ιμάτια θέλουσι με βδελύττεσθαι.
\par 32 Διότι δεν είναι άνθρωπος ως εγώ, διά να αποκριθώ προς αυτόν, και να έλθωμεν εις κρίσιν ομού.
\par 33 Δεν υπάρχει μεσίτης μεταξύ ημών, διά να βάλη την χείρα αυτού επ' αμφοτέρους ημάς.
\par 34 Ας απομακρύνη απ' εμού την ράβδον αυτού, και ο φόβος αυτού ας μη με εκπλήττη·
\par 35 τότε θέλω λαλήσει και δεν θέλω φοβηθή αυτόν· διότι ούτω δεν είμαι εν εμαυτώ.

\chapter{10}

\par Η ψυχή μου εβαρύνθη την ζωήν μου· θέλω παραδοθή εις το παράπονόν μου· θέλω λαλήσει εν τη πικρία της ψυχής μου.
\par 2 Θέλω ειπεί προς τον Θεόν, μη με καταδικάσης· δείξόν μοι διά τι με δικάζεις.
\par 3 Είναι καλόν εις σε να καταθλίβης, να καταφρονής το έργον των χειρών σου και να ευοδόνης την βουλήν των ασεβών;
\par 4 Σαρκός οφθαλμούς έχεις; ή βλέπεις καθώς βλέπει άνθρωπος;
\par 5 Ανθρώπινος είναι ο βίος σου; ή τα έτη σου ως ημέραι ανθρώπου,
\par 6 ώστε αναζητείς την ανομίαν μου και ανερευνάς την αμαρτίαν μου;
\par 7 Ενώ εξεύρεις ότι δεν ησέβησα· και δεν υπάρχει ο ελευθερών εκ των χειρών σου.
\par 8 Αι χείρές σου με εμόρφωσαν και με έπλασαν όλον κύκλω· και με καταστρέφεις.
\par 9 Ενθυμήθητι, δέομαι, ότι ως πηλόν με έκαμες· και εις χώμα θέλεις με επιστρέψει.
\par 10 Δεν με ήμελξας ως γάλα και με έπηξας ως τυρόν;
\par 11 Δέρμα και σάρκα με ενέδυσας και με οστά και νεύρα με περιέφραξας.
\par 12 Ζωήν και έλεος εχάρισας εις εμέ, και η επίσκεψίς σου εφύλαξε το πνεύμά μου·
\par 13 ταύτα όμως έκρυπτες εν τη καρδία σου· εξεύρω ότι τούτο ήτο μετά σου.
\par 14 Εάν αμαρτήσω, με παραφυλάττεις, και από της ανομίας μου δεν θέλεις με αθωώσει.
\par 15 Εάν ασεβήσω, ουαί εις εμέ· και εάν ήμαι δίκαιος, δεν δύναμαι να σηκώσω την κεφαλήν μου· είμαι πλήρης ατιμίας· ιδέ λοιπόν την θλίψιν μου,
\par 16 διότι αυξάνει. Με κυνηγείς ως άγριος λέων· και επιστρέφων δεικνύεσαι θαυμαστός κατ' εμού.
\par 17 Ανανεόνεις τους μάρτυράς σου εναντίον μου, και πληθύνεις την οργήν σου κατ' εμού· αλλαγαί στρατεύματος γίνονται επ' εμέ.
\par 18 Διά τι λοιπόν με εξήγαγες εκ της μήτρας; είθε να εξέπνεον, και οφθαλμός να μη με έβλεπεν.
\par 19 Ήθελον είσθαι ως μη υπάρξας· ήθελον φερθή εκ της μήτρας εις τον τάφον.
\par 20 Αι ημέραι μου δεν είναι ολίγαι; παύσον λοιπόν, και άφες με, διά να αναλάβω ολίγον,
\par 21 πριν υπάγω όθεν δεν θέλω επιστρέψει, εις γην σκότους και σκιάς θανάτου·
\par 22 γην γνοφεράν, ως το σκότος της σκιάς του θανάτου, όπου τάξις δεν είναι, και το φως είναι ως το σκότος.

\chapter{11}

\par Και απεκρίθη ο Σωφάρ ο Νααμαθίτης και είπε·
\par 2 Δεν δίδεται απόκρισις εις το πλήθος των λόγων; και ο πολύλογος θέλει δικαιωθή;
\par 3 Αι φλυαρίαι σου θέλουσιν αποστομώσει τους ανθρώπους; και όταν περιγελάς, δεν θέλει σε καταισχύνει τις;
\par 4 Διότι είπες, Η ομιλία μου είναι καθαρά, και είμαι καθαρός ενώπιόν σου.
\par 5 Αλλ' είθε να ελάλει ο Θεός και να ήνοιγε τα χείλη αυτού εναντίον σου.
\par 6 Και να σοι εφανέρονε τα κρύφια της σοφίας, ότι είναι διπλάσια των όσα γνωρίζονται. Έξευρε λοιπόν, ότι ο Θεός απαιτεί από σου ολιγώτερον της ανομίας σου.
\par 7 Δύνασαι να εξιχνιάσης τα βάθη του Θεού; δύνασαι να εξιχνιάσης τον Παντοδύναμον με εντέλειαν;
\par 8 Ταύτα είναι ως τα ύψη του ουρανού· τι δύνασαι να κάμης; είναι βαθύτερα του άδου· τι δύνασαι να γνωρίσης;
\par 9 Το μέτρον αυτών είναι μακρότερον της γης, και πλατύτερον της θαλάσσης.
\par 10 Εάν θελήση να χαλάση και να κλείση, ή να συνάξη, τότε τις δύναται να εμποδίση αυτόν;
\par 11 Διότι αυτός γνωρίζει την ματαιότητα των ανθρώπων, και βλέπει την ασέβειαν· και δεν θέλει εξετάσει;
\par 12 Ο δε μάταιος άνθρωπος υπερηφανεύεται, και γεννάται ο άνθρωπος άγριον ονάριον.
\par 13 Εάν συ ετοιμάσης την καρδίαν σου και εκτείνης τας χείρας σου προς αυτόν·
\par 14 εάν την ανομίαν, την εν χερσί σου, απομακρύνης και δεν αφίνης να κατοικήση ασέβεια εν ταις σκηναίς σου·
\par 15 τότε βεβαίως θέλεις υψώσει το πρόσωπόν σου ακηλίδωτον· μάλιστα θέλεις είσθαι σταθερός και δεν θέλεις φοβείσθαι.
\par 16 Διότι συ θέλεις λησμονήσει την θλίψιν· θέλεις ενθυμηθή αυτήν ως ύδατα διαρρεύσαντα·
\par 17 και ο καιρός σου θέλει ανατείλει λαμπρότερος της μεσημβρίας· και εάν επέλθη σκότος επί σε, πάλιν θέλεις γείνει ως η αυγή·
\par 18 και θέλεις είσθαι ασφαλής, διότι υπάρχει ελπίς εις σέ· ναι, θέλεις σκάπτει διά την σκηνήν σου και θέλεις κοιμάσθαι εν ασφαλεία·
\par 19 θέλεις πλαγιάζει, και ουδείς θέλει σε τρομάζει· και πολλοί θέλουσιν ικετεύει το πρόσωπόν σου.
\par 20 Των δε ασεβών οι οφθαλμοί θέλουσι μαρανθή, και καταφύγιον θέλει λείψει απ' αυτών, και η ελπίς αυτών θέλει είσθαι να εκπνεύσωσι.

\chapter{12}

\par Ο δε Ιώβ απεκρίθη και είπε·
\par 2 Σεις είσθε αληθώς οι άνθρωποι, και με σας θέλει τελευτήσει η σοφία.
\par 3 Και εγώ έχω σύνεσιν ως και υμείς· δεν είμαι κατώτερος υμών· και τις δεν γνωρίζει τοιαύτα πράγματα;
\par 4 Έγεινα χλεύη εις τον πλησίον μου, όστις επικαλούμαι τον Θεόν, και μοι αποκρίνεται. Ο δίκαιος και άμεμπτος περιγελάται.
\par 5 Ο κινδυνεύων να ολισθήση με τους πόδας είναι εις τον στοχασμόν του ευτυχούντος ως λύχνος καταπεφρονημένος.
\par 6 Αι σκηναί των ληστών ευτυχούσι, και οι παροργίζοντες τον Θεόν είναι εν ασφαλεία, εις τας χείρας των οποίων ο Θεός φέρει αφθονίαν.
\par 7 Αλλ' ερώτησον τώρα τα ζώα, και θέλουσι σε διδάξει· και τα πετεινά του ουρανού, και θέλουσι σοι απαγγείλει·
\par 8 ή λάλησον προς την γην, και θέλει σε διδάξει· και οι ιχθύες της θαλάσσης θέλουσι σοι διηγηθή.
\par 9 Τις εκ πάντων τούτων δεν γνωρίζει, ότι η χειρ του Κυρίου έκαμε ταύτα;
\par 10 Εν τη χειρί του οποίου είναι ψυχή πάντων των ζώντων και η πνοή πάσης ανθρωπίνης σαρκός.
\par 11 Το ωτίον δεν διακρίνει τους λόγους; και ο ουρανίσκος λαμβάνει γεύσιν του φαγητού αυτού;
\par 12 Η σοφία είναι μετά των γερόντων, και η σύνεσις εν τη μακρότητι των ημερών.
\par 13 Εν αυτώ είναι η σοφία και η δύναμις· αυτός έχει βουλήν και σύνεσιν.
\par 14 Ιδού, καταστρέφει, και δεν ανοικοδομείται· κλείει κατά του ανθρώπου, και ουδείς ο ανοίγων.
\par 15 Ιδού, κρατεί τα ύδατα, και ξηραίνονται· πάλιν εξαποστέλλει αυτά, και καταστρέφουσι την γην.
\par 16 Μετ' αυτού είναι η δύναμις και η σοφία· αυτού είναι ο απατώμενος και ο απατών.
\par 17 Παραδίδει λάφυρον τους βουλευτάς και μωραίνει τους κριτάς.
\par 18 Λύει την ζώνην των βασιλέων και περιζώνει την οσφύν αυτών με σχοινίον.
\par 19 Παραδίδει λάφυρον τους άρχοντας και καταστρέφει τους ισχυρούς.
\par 20 Αφαιρεί τον λόγον των δεινών ρητόρων, και σηκόνει την σύνεσιν από των πρεσβυτέρων.
\par 21 Εκχέει καταφρόνησιν επί τους άρχοντας, και λύει την ζώνην των ισχυρών.
\par 22 Αποκαλύπτει εκ του σκότους βαθέα πράγματα, και εξάγει εις φως την σκιάν του θανάτου.
\par 23 Μεγαλύνει τα έθνη και αφανίζει αυτά· πλατύνει τα έθνη και συστέλλει αυτά.
\par 24 Αφαιρεί την καρδίαν από των αρχηγών των λαών της γης, και κάμνει αυτούς να περιπλανώνται εν ερήμω αβάτω·
\par 25 ψηλαφώσιν εν σκότει χωρίς φωτός, και κάμνει αυτούς να παραφέρωνται ως ο μεθύων.

\chapter{13}

\par Ιδού, ταύτα πάντα είδεν ο οφθαλμός μου· το ωτίον μου ήκουσε και ενόησε ταύτα.
\par 2 Καθώς γνωρίζετε σεις, γνωρίζω και εγώ· δεν είμαι κατώτερος υμών.
\par 3 Αλλ' όμως θέλω λαλήσει προς τον Παντοδύναμον, και επιθυμώ να διαλεχθώ μετά του Θεού.
\par 4 Σεις δε είσθε εφευρεταί ψεύδους· είσθε πάντες ιατροί ανωφελείς.
\par 5 Είθε να εσιωπάτε παντάπασι και τούτο ήθελεν είσθαι εις εσάς σοφία.
\par 6 Ακούσατε τώρα τους λόγους μου, και προσέξατε εις τας δικαιολογίας των χειλέων μου.
\par 7 Θέλετε λαλεί άδικα υπέρ του Θεού; και θέλετε προφέρει δόλια υπέρ αυτού;
\par 8 Θέλετε κάμει προσωποληψίαν υπέρ αυτού; θέλετε δικολογήσει υπέρ του Θεού;
\par 9 Είναι καλόν να σας εξιχνιάση; ή καθώς άνθρωπος περιγελά άνθρωπον, θέλετε περιγελά αυτόν;
\par 10 Εξάπαντος θέλει σας εξελέγξει, εάν κρυφίως προσωποληπτήτε.
\par 11 Το μεγαλείον αυτού δεν θέλει σας τρομάξει, και ο φόβος αυτού πέσει εφ' υμάς;
\par 12 τα απομνημονεύματά σας ισοδυναμούσι με κονιορτόν, τα προπύργιά σας με προπύργια χώματος.
\par 13 Σιωπήσατε, αφήσατέ με, διά να λαλήσω εγώ, και ας έλθη επ' εμέ ό,τι δήποτε.
\par 14 διά τι πιάνω τας σάρκας μου με τους οδόντας μου και βάλλω την ζωήν μου εις την χείρα μου;
\par 15 Και αν με θανατόνη, εγώ θέλω ελπίζει εις αυτόν· πλην θέλω υπερασπισθή τας οδούς μου ενώπιον αυτού.
\par 16 Αυτός μάλιστα θέλει είσθαι η σωτηρία μου· διότι δεν θέλει ελθεί ενώπιον αυτού υποκριτής.
\par 17 Ακροάσθητε προσεκτικώς τον λόγον μου, και την παράστασίν μου με τα ώτα σας.
\par 18 Ιδού τώρα, διέταξα την κρίσιν μου· εξεύρω ότι εγώ θέλω δικαιωθή.
\par 19 Τις είναι εκείνος όστις θέλει αντιδιαλεχθή μετ' εμού, διά να σιωπήσω τώρα και να εκπνεύσω;
\par 20 Μόνον δύο μη κάμης εις εμέ· τότε δεν θέλω κρυφθή από του προσώπου σου·
\par 21 την χείρα σου απομάκρυνον απ' εμού· και ο φόβος σου ας μη με τρομάξη.
\par 22 Έπειτα κάλεσον, και εγώ θέλω αποκριθή· ή ας λαλήσω, και αποκρίθητί μοι.
\par 23 Πόσαι είναι αι ανομίαι μου και αι αμαρτίαι μου; φανέρωσόν μοι το έγκλημά μου και την αμαρτίαν μου.
\par 24 Διά τι κρύπτεις το πρόσωπόν σου και με θεωρείς ως εχθρόν σου;
\par 25 Θέλεις κατατρίψει φύλλον φερόμενον υπό του ανέμου; και θέλεις κατατρέξει άχυρον ξηρόν;
\par 26 Διότι γράφεις πικρίας εναντίον μου, και αποδίδεις εις εμέ τας ανομίας της νεότητός μου·
\par 27 και βάλλεις τους πόδας μου εις δεσμά, και παραφυλάττεις πάσας τας οδούς μου· σημειόνεις τα ίχνη των ποδών εμού·
\par 28 όστις φθείρεται ως πράγμα σεσηπός, ως ένδυμα σκωληκόβρωτον.

\chapter{14}

\par Άνθρωπος γεγεννημένος εκ γυναικός είναι ολιγόβιος και πλήρης ταραχής·
\par 2 αναβλαστάνει ως άνθος και κόπτεται· φεύγει ως σκιά και δεν διαμένει.
\par 3 Και επί τοιούτον ανοίγεις τους οφθαλμούς σου, και με φέρεις εις κρίσιν μετά σου;
\par 4 Τις δύναται να εξαγάγη καθαρόν από ακαθάρτου; ουδείς.
\par 5 Επειδή αι ημέραι αυτού είναι προσδιωρισμέναι, ο αριθμός των μηνών αυτού ευρίσκεται παρά σοι, και συ έθεσας τα όρια αυτού, και δεν δύναται να υπερβή αυτά,
\par 6 απόστρεψον απ' αυτού, διά να ησυχάση, εωσού χαίρων εκπληρώση ως μισθωτός την ημέραν αυτού.
\par 7 Διότι περί του δένδρου, εάν κοπή, είναι ελπίς ότι θέλει αναβλαστήσει, και ότι ο τρυφερός αυτού βλαστός δεν θέλει εκλείψει.
\par 8 Και αν η ρίζα αυτού παλαιωθή εν τη γη και ο κορμός αυτού αποθάνη εν τω χώματι,
\par 9 όμως διά της οσμής του ύδατος θέλει αναβλαστήσει και θέλει εκβάλει κλάδους ως νεόφυτον.
\par 10 Αλλ' ο άνθρωπος αποθνήσκει και παρέρχεται· και ο άνθρωπος εκπνέει, και που είναι;
\par 11 Καθώς τα ύδατα εκλείπουσιν εκ της θαλάσσης και ο ποταμός στειρεύει και ξηραίνεται,
\par 12 ούτως ο άνθρωπος, αφού κοιμηθή, δεν ανίσταται· εωσού οι ουρανοί μη υπάρξωσι, δεν θέλουσιν εξυπνήσει, και δεν θέλουσιν εγερθή εκ του ύπνου αυτών.
\par 13 Είθε να με έκρυπτες εν τω τάφω, να με εσκέπαζες εωσού παρέλθη η οργή σου, να προσδιώριζες εις εμέ προθεσμίαν, και τότε να με ενθυμηθής
\par 14 Εάν αποθάνη ο άνθρωπος, θέλει αναζήσει; πάσας τας ημέρας της εκστρατείας μου θέλω περιμένει, εωσού έλθη η απαλλαγή μου.
\par 15 Θέλεις καλέσει, και εγώ θέλω σοι αποκριθή· θέλεις επιβλέψει εις το έργον των χειρών σου.
\par 16 Διότι τώρα αριθμείς τα διαβήματά μου· δεν παραφυλάττεις τας αμαρτίας μου;
\par 17 Η παράβασίς μου είναι επεσφραγισμένη εν βαλαντίω, και επισημειόνεις την ανομίαν μου.
\par 18 Βεβαίως το μεν όρος πίπτον εξουδενούται, ο δε βράχος μετακινείται από του τόπου αυτού.
\par 19 Τα ύδατα τρώγουσι τας πέτρας· αι πλημμύραι αυτών παρασύρουσι το χώμα της γής· ούτω συ καταστρέφεις την ελπίδα του ανθρώπου,
\par 20 υπερισχύεις πάντοτε εναντίον αυτού, και αυτός παρέρχεται· μεταβάλλεις την όψιν αυτού και αποπέμπεις αυτόν.
\par 21 Οι υιοί αυτού υψούνται, και αυτός δεν εξεύρει· και ταπεινούνται, και αυτός δεν εννοεί ουδέν περί αυτών.
\par 22 Μόνον η σαρξ αυτού επ' αυτού θέλει πονεί, και η ψυχή αυτού εν αυτώ θέλει πενθεί.

\chapter{15}

\par Τότε απεκρίθη Ελιφάς ο Θαιμανίτης και είπεν·
\par 2 Έπρεπε σοφός να προφέρη στοχασμούς μάταιους και να γεμίζη την κοιλίαν αυτού από ανατολικού ανέμου;
\par 3 Έπρεπε να φιλονεική διά λόγων ματαίων και ομιλιών ανωφελών;
\par 4 Βεβαίως συ απορρίπτεις τον φόβον και αποκλείεις την δέησιν ενώπιον του Θεού.
\par 5 Διότι το στόμα σου αποδεικνύει την ανομίαν σου, και εξέλεξας την γλώσσαν των πανούργων.
\par 6 Το στόμα σου σε καταδικάζει, και ουχί εγώ· και τα χείλη σου καταμαρτυρούσιν εναντίον σου.
\par 7 Μη πρώτος άνθρωπος εγεννήθης; ή προ των βουνών επλάσθης;
\par 8 Μήπως ήκουσας τας βουλάς του Θεού; και εξήντλησας εις σεαυτόν την σοφίαν;
\par 9 Τι εξεύρεις, και δεν εξεύρομεν; τι εννοείς, και δεν εννοούμεν;
\par 10 Υπάρχουσι και μεταξύ ημών πολιοί και γέροντες, γεροντότεροι του πατρός σου.
\par 11 Αι παρηγορίαι του Θεού φαίνονται μικρόν πράγμα εις σε; ή έχεις τι απόκρυφον εν σεαυτώ;
\par 12 Διά τι σε αποπλανά η καρδία σου; και διά τι παραφέρονται οι οφθαλμοί σου,
\par 13 ώστε στρέφεις το πνεύμά σου κατά του Θεού και αφίνεις να εξέρχωνται τοιούτοι λόγοι εκ του στόματός σου;
\par 14 Τι είναι ο άνθρωπος, ώστε να ήναι καθαρός; και ο γεγεννημένος εκ γυναικός, ώστε να ήναι δίκαιος;
\par 15 Ιδού, εις τους αγίους αυτού δεν εμπιστεύεται· και οι ουρανοί δεν είναι καθαροί εις τους οφθαλμούς αυτού·
\par 16 πόσω μάλλον βδελυρός και ακάθαρτος είναι ο άνθρωπος, ο πίνων ανομίαν ως ύδωρ;
\par 17 Εγώ θέλω σε διδάξει· άκουσόν μου· τούτο βεβαίως είδον και θέλω φανερώσει,
\par 18 το οποίον οι σοφοί ανήγγειλαν παρά των πατέρων αυτών, και δεν έκρυψαν·
\par 19 εις τους οποίους μόνους εδόθη η γη, και ξένος δεν επέρασε διά μέσου αυτών.
\par 20 Ο ασεβής βασανίζεται πάσας τας ημέρας, και αριθμητά έτη είναι πεφυλαγμένα διά τον τύραννον.
\par 21 Ήχος φόβου είναι εις τα ώτα αυτού· εν μέσω ειρήνης θέλει επέλθει επ' αυτόν ο εξολοθρευτής.
\par 22 Δεν πιστεύει ότι θέλει επιστρέψει εκ του σκότους, και περιμένει την μάχαιραν.
\par 23 Περιπλανάται διά άρτον, και που; εξεύρει ότι η ημέρα του σκότους είναι ετοίμη πλησίον αυτού.
\par 24 Θλίψις και στενοχωρία θέλουσι καταπλήττει αυτόν· θέλουσιν υπερισχύσει κατ' αυτού, ως βασιλεύς εις μάχην παρεσκευασμένος·
\par 25 διότι εξήπλωσε την χείρα αυτού κατά του Θεού και ηλαζονεύθη κατά του Παντοδυνάμου·
\par 26 ώρμησε κατ' αυτού με τράχηλον επηρμένον, με την πεπυκνωμένην ράχιν των ασπίδων αυτού·
\par 27 διότι εσκέπασε το πρόσωπον αυτού με το πάχος αυτού και υπερεπάχυνε τα πλευρά αυτού·
\par 28 και κατώκησεν εις πόλεις ερήμους, εις οίκους ακατοικήτους, ετοίμους διά σωρούς.
\par 29 δεν θέλει πλουτισθή, ουδέ θέλουσι διαμένει τα υπάρχοντα αυτού, ουδέ θέλει εκτανθή η αφθονία αυτών επί την γην.
\par 30 Δεν θέλει χωρισθή εκ του σκότους· φλόξ θέλει ξηράνει τους βλαστούς αυτού, και με την πνοήν του στόματος αυτού θέλει απέλθει.
\par 31 Ας μη πιστεύση εις την ματαιότητα ο ηπατημένος, διότι ματαιότης θέλει είσθαι η αμοιβή αυτού.
\par 32 Προ του καιρού αυτού θέλει φθαρή, και ο κλάδος αυτού δεν θέλει πρασινίσει.
\par 33 Θέλει αποβάλει την άωρον σταφυλήν αυτού ως η άμπελος, και θέλει ρίψει το άνθος αυτού ως η ελαία.
\par 34 Διότι η σύναξις των υποκριτών θέλει ερημωθή, και πυρ θέλει καταφάγει τας σκηνάς της δωροληψίας.
\par 35 Συλλαμβάνουσι πονηρίαν και γεννώσι ματαιότητα, και η καρδία αυτών μηχανάται δόλον.

\chapter{16}

\par Τότε ο Ιώβ απεκρίθη και είπε·
\par 2 Πολλά τοιαύτα ήκουσα· άθλιοι παρηγορηταί είσθε πάντες.
\par 3 Έχουσι τέλος αι ματαιολογίαι; ή τι σε ενθαρρύνει εις το να αποκρίνησαι;
\par 4 Και εγώ εδυνάμην να λαλήσω καθώς σείς· εάν η ψυχή σας ήτο εις τον τόπον της ψυχής μου, ηδυνάμην να επισωρεύσω λόγους εναντίον σας, και να κινήσω εναντίον σας την κεφαλήν μου.
\par 5 Ήθελον σας ενισχύσει με το στόμα μου, και η κίνησις των χειλέων μου ήθελε σας ανακουφίσει.
\par 6 Αν λαλώ, ο πόνος μου δεν ανακουφίζεται· και αν σιωπώ, ποία ελάττωσις γίνεται εις εμέ;
\par 7 Αλλά τώρα με υπερεβάρυνεν· ηρήμωσας πάσαν την συνοδίαν μου.
\par 8 Και αι ρυτίδες με τας οποίας με εσημείωσας, είναι μαρτυρία· και η ισχνότης μου ανισταμένη εις εμέ, μαρτυρεί επί του προσώπου μου.
\par 9 Με διασπαράττει ο εχθρός μου εν τω θυμώ αυτού και με μισεί· τρίζει τους οδόντας αυτού εναντίον μου· οξύνει τους οφθαλμούς αυτού επ' εμέ.
\par 10 Ανοίγουσι το στόμα αυτών κατ' εμού· με τύπτουσι κατά της σιαγόνος υβριστικώς· συνήχθησαν ομού επ' εμέ.
\par 11 Ο Θεός με παρέδωκεν εις τον άδικον, και με έρριψεν εις χείρας ασεβών.
\par 12 Ήμην εν ησυχία, και με κατεσπάραξε· και πιάσας με από του τραχήλου, με κατεσύντριψε, και με έθεσε σκοπόν αυτού.
\par 13 Οι τοξόται αυτού με περιεκύκλωσαν· διαπερά τα νεφρά μου, και δεν φείδεται· εκχέει την χολήν μου επί την γην.
\par 14 Με συντρίβει με πληγήν επί πληγήν· έδραμεν επ' εμέ ως γίγας.
\par 15 Σάκκον έρραψα επί το δέρμα μου, και εμόλυνα το κέρας μου με χώμα.
\par 16 Το πρόσωπόν μου κατεκάη υπό του κλαυθμού, και σκιά θανάτου είναι επί των βλεφάρων μου·
\par 17 ενώ αδικία δεν υπάρχει εν ταις χερσί μου, και η προσευχή μου είναι καθαρά.
\par 18 Ω γη, μη σκεπάσης το αίμα μου, και ας μη υπάρχη τόπος διά την κραυγήν μου,
\par 19 και τώρα, ιδού, ο μάρτυς μου είναι εν τω ουρανώ, και η μαρτυρία μου εν τοις υψίστοις.
\par 20 Οι φίλοι μου είναι οι εμπαίζοντές με· ο οφθαλμός μου σταλάζει δάκρυα προς τον Θεόν.
\par 21 Να ήτο δυνατόν να διαδικάζηταί τις προς τον Θεόν, ως άνθρωπος προς τον πλησίον αυτού.
\par 22 Διότι ήλθον τα ηριθμημένα έτη· και θέλω υπάγει την οδόν, οπόθεν δεν θέλω επιστρέψει.

\chapter{17}

\par Το πνεύμά μου φθείρεται, αι ημέραι μου σβύνονται, οι τάφοι είναι έτοιμοι δι' εμέ.
\par 2 Δεν είναι χλευασταί πλησίον μου; και δεν διανυκτερεύει ο οφθαλμός μου εν ταις πικρίαις αυτών;
\par 3 Ασφάλισόν με, δέομαι· γενού εις εμέ εγγυητής πλησίον σου· τις ήθελεν εγγυηθή εις εμέ;
\par 4 Διότι συ έκρυψας την καρδίαν αυτών από συνέσεως· διά τούτο δεν θέλεις υψώσει αυτούς.
\par 5 Του λαλούντος με απάτην προς τους φίλους, και οι οφθαλμοί των τέκνων αυτού θέλουσι τήκεσθαι.
\par 6 Και με κατέστησε παροιμίαν των λαών· και ενώπιον αυτών κατεστάθην όνειδος.
\par 7 Και ο οφθαλμός μου εμαράνθη υπό της θλίψεως, και πάντα τα μέλη μου έγειναν ως σκιά.
\par 8 Οι ευθείς θέλουσι θαυμάσει εις τούτο, και ο αθώος θέλει διεγερθή κατά του υποκριτού.
\par 9 Ο δε δίκαιος θέλει κρατεί την οδόν αυτού, και ο καθαρός τας χείρας θέλει επαυξήσει την δύναμιν αυτού.
\par 10 Σεις δε πάντες επιστράφητε, και έλθετε τώρα· διότι ουδένα συνετόν θέλω ευρεί μεταξύ σας.
\par 11 Αι ημέραι μου παρήλθον, εκόπησαν οι σκοποί μου, αι επιθυμίαι της καρδίας μου.
\par 12 Την νύκτα μετέβαλον εις ημέραν· το φως είναι πλησίον του σκότους.
\par 13 Εάν προσμένω, ο τάφος είναι η κατοικία μου· έστρωσα την κλίνην μου εν τω σκότει.
\par 14 Εβόησα προς την φθοράν, Είσαι, πατήρ μου· προς τον σκώληκα, Μήτηρ μου και αδελφή μου είσαι.
\par 15 Και που τώρα η ελπίς μου; και την ελπίδα μου τις θέλει ιδεί;
\par 16 εις το βάθος του άδου θέλει καταβή· βεβαίως θέλει αναπαυθή μετ' εμού εν τω χώματι.

\chapter{18}

\par Και απεκρίθη Βιλδάδ ο Σαυχίτης και είπεν·
\par 2 Έως πότε δεν θέλετε τελειώσει τους λόγους; προσέξατε, και έπειτα θέλομεν λαλήσει.
\par 3 Διά τι λογιζόμεθα ως τετράποδα, και εξαχρειούμεθα έμπροσθέν σας;
\par 4 Ω διασπαράττων την ψυχήν σου εν τω θυμώ σου, διά σε η γη θέλει εγκαταλειφθή; και ο βράχος θέλει μετακινηθή από του τόπου αυτού;
\par 5 Βεβαίως το φως των ασεβών θέλει σβεσθή, και ο σπινθήρ του πυρός αυτών δεν θέλει αναλάμψει·
\par 6 το φως θέλει είσθαι σκότος εν τη σκηνή αυτού, και ο λύχνος αυτού άνωθεν αυτού θέλει σβεσθή·
\par 7 τα βήματα της δυνάμεως αυτού θέλουσι συσταλθή, και η βουλή αυτού θέλει κατακρημνίσει αυτόν.
\par 8 Διότι με τους εαυτού πόδας ερρίφθη εις δίκτυον, και περιπατεί επί βρόχων.
\par 9 Παγίς θέλει συλλάβει αυτόν από της πτέρνας· ο κλέπτης θέλει υπερισχύσει κατ' αυτού.
\par 10 Η παγίς αυτού είναι κεκρυμμένη εν τη γη, και η ενέδρα αυτού επί της οδού.
\par 11 Τρόμοι θέλουσι φοβίζει αυτόν κυκλόθεν, και θέλουσι καταδιώκει αυτόν κατά πόδας.
\par 12 Η δύναμις αυτού θέλει λιμοκτονήσει, και όλεθρος θέλει είσθαι έτοιμος εις την πλευράν αυτού.
\par 13 Πρωτότοκος θάνατος θέλει καταφάγει το κάλλος του δέρματος αυτού· το κάλλος αυτού θέλει καταφάγει.
\par 14 Το θάρρος αυτού θέλει εκριζωθή από της σκηνής αυτού, και αυτός θέλει συρθή προς τον βασιλέα των τρόμων.
\par 15 Ούτοι θέλουσι κατοικήσει εν τη σκηνή αυτού, ήτις δεν είναι πλέον αυτού· θείον θέλει διασπαρή επί την κατοικίαν αυτού.
\par 16 Υποκάτωθεν αι ρίζαι αυτού θέλουσι ξηρανθή, και επάνωθεν θέλει κοπή ο κλάδος αυτού.
\par 17 Το μνημόσυνον αυτού θέλει εξαλειφθή από της γης, και δεν θέλει υπάρχει πλέον το όνομα αυτού εν ταις πλατείαις.
\par 18 Θέλει εξωσθή από του φωτός εις το σκότος, και θέλει εκβληθή από του κόσμου.
\par 19 Δεν θέλει έχει ούτε υιόν ούτε έγγονον μεταξύ του λαού αυτού, ουδέ υπόλοιπον εν ταις κατοικίαις αυτού.
\par 20 Οι μεταγενέστεροι θέλουσιν εκπλαγή διά την ημέραν αυτού, καθώς οι προγενέστεροι έλαβον φρίκην.
\par 21 Βεβαίως τοιαύται είναι αι κατοικίαι του ασεβούς, και ούτος ο τόπος του μη γνωρίζοντος τον Θεόν.

\chapter{19}

\par Και απεκρίθη ο Ιώβ και είπεν·
\par 2 Έως πότε θέλετε θλίβει την ψυχήν μου, και θέλετε με κατασυντρίβει με λόγους;
\par 3 Δεκάκις ήδη με ωνειδίσατε· δεν αισχύνεσθε να σκληρύνησθε εναντίον μου;
\par 4 Και εάν τωόντι έσφαλα, το σφάλμα μου μένει εν εμοί.
\par 5 Αλλ' εάν θέλητε εξάπαντος να μεγαλυνθήτε εναντίον μου, και να ρίπτητε κατ' εμού το όνειδός μου,
\par 6 μάθετε τώρα ότι ο Θεός με κατέστρεψε, και με περιεκύκλωσε με το δίκτυον αυτού.
\par 7 Ιδού, φωνάζω, Αδικία· αλλά δεν εισακούομαι· επικαλούμαι, αλλ' ουδεμία κρίσις.
\par 8 Έφραξε την οδόν μου, και δεν δύναμαι να περάσω, και έθεσε σκότος εις τας τρίβους μου.
\par 9 Με εξέδυσε την δόξαν μου, και αφήρεσε τον στέφανον της κεφαλής μου.
\par 10 Με ηφάνισε πανταχόθεν, και χάνομαι· και εξερρίζωσε την ελπίδα μου ως δένδρον.
\par 11 Και εξήψε κατ' εμού τον θυμόν αυτού, και με στοχάζεται ως εχθρόν αυτού.
\par 12 Τα τάγματα αυτού ήλθον ομού και ητοίμασαν την οδόν αυτών εναντίον μου, και εστρατοπέδευσαν πέριξ της σκηνής μου.
\par 13 Απεμάκρυνεν απ' εμού τους αδελφούς μου, και ηλλοτριώθησαν όλως απ' εμού οι γνώριμοί μου.
\par 14 Οι πλησίον μου με αφήκαν, και οι γνωστοί μου με ελησμόνησαν.
\par 15 Οι κατοικούντες εν τω οίκω μου και αι θεράπαιναί μου με στοχάζονται ως ξένον· ξένος κατεστάθην εις τους οφθαλμούς αυτών.
\par 16 Καλώ τον υπηρέτην μου, και δεν αποκρίνεται· με το στόμα μου ικέτευσα αυτόν.
\par 17 Η πνοή μου έγεινε ξένη εις την γυναίκα μου, και αι παρακλήσεις μου εις τα τέκνα της κοιλίας μου.
\par 18 Και αυτά τα παιδάρια με κατεφρόνησαν· εσηκώθην, και ελάλησαν εναντίον μου.
\par 19 Πάντες οι μυστικοί φίλοι μου με εβδελύχθησαν· και εκείνοι, τους οποίους ηγάπησα, εστράφησαν εναντίον μου.
\par 20 Τα οστά μου εκολλήθησαν εις το δέρμα μου και εις την σάρκα μου και διεσώθην με το δέρμα των οδόντων μου.
\par 21 Ελεήσατέ με, ελεήσατέ με, σεις φίλοι μου· διότι χειρ Θεού με επλήγωσε.
\par 22 Διά τι με κατατρέχετε ως ο Θεός, και δεν εχορτάσθητε από των σαρκών μου;
\par 23 Ω και να εγράφοντο οι λόγοι μου· να ενετυπούντο εν βιβλίω·
\par 24 να ενεχαράττοντο επί βράχον διά σιδηράς γραφίδος και μολύβδου διαπαντός
\par 25 Διότι εξεύρω ότι ζη ο Λυτρωτής μου, και θέλει εγερθή εν τοις εσχάτοις καιροίς επί της γής·
\par 26 και αφού μετά το δέρμα μου το σώμα τούτο φθαρή, πάλιν με την σάρκα μου θέλω ιδή τον Θεόν·
\par 27 τον οποίον αυτός εγώ θέλω ιδεί, και θέλουσι θεωρήσει οι οφθαλμοί μου, και ουχί άλλος· οι νεφροί μου κατατήκονται εν τω κόλπω μου.
\par 28 Αλλά σεις έπρεπε να είπητε, Διά τι κατατρέχομεν αυτόν; επειδή η ρίζα του πράγματος ευρίσκεται εν εμοί.
\par 29 Φοβήθητε την ρομφαίαν· διότι η ρομφαία είναι ο εκδικητής των ανομιών, διά να γνωρίσητε ότι υπάρχει κρίσις.

\chapter{20}

\par Και απεκρίθη Σωφάρ ο Νααμαθίτης και είπε·
\par 2 Διά τούτο οι στοχασμοί μου με κινούσιν εις το να αποκριθώ, και διά τούτο σπεύδω.
\par 3 Ήκουσα την εις εμέ ονειδιστικήν επίπληξιν, και το πνεύμα της συνέσεως μου με κάμνει να αποκριθώ.
\par 4 Δεν γνωρίζεις τούτο παλαιόθεν αφ' ότου ο άνθρωπος ετέθη επί της γης,
\par 5 ότι ο θρίαμβος των ασεβών είναι ολιγοχρόνιος, και η χαρά του υποκριτού στιγμαία.
\par 6 Και αν το μεγαλείον αυτού αναβή εις τους ουρανούς και η κεφαλή αυτού φθάση έως των νεφελών,
\par 7 θέλει αφανισθή διαπαντός ως κόπρος αυτού· όσοι έβλεπον αυτόν θέλουσι λέγει, Που εκείνος;
\par 8 θέλει πετάξει ως όνειρον και δεν θέλει ευρεθή· και, ως όρασις της νυκτός θέλει εξαφανισθή.
\par 9 Και ο οφθαλμός όστις έβλεπεν αυτόν δεν θέλει ιδεί αυτόν πλέον· και ο τόπος αυτού δεν θέλει πλέον γνωρίσει αυτόν.
\par 10 Οι υιοί αυτού θέλουσι ζητήσει την εύνοιαν των πτωχών, και αι χείρες αυτού θέλουσιν επιστρέψει τα αγαθά αυτών.
\par 11 Τα οστά αυτού γέμουσιν από των αμαρτημάτων της νεότητος αυτού, και θέλουσι κοιμηθή μετ' αυτού εν χώματι.
\par 12 Αν και η κακία ήναι γλυκεία εν τω στόματι αυτού, κρύπτη αυτήν υπό την γλώσσαν αυτού·
\par 13 αν και περιθάλπη αυτήν και δεν αφίνη αυτήν, αλλά κρατή αυτήν εν τω μέσω του ουρανίσκου αυτού·
\par 14 όμως η τροφή αυτού θέλει αλλοιωθή εις τα εντόσθια αυτού· χολή ασπίδων θέλει γείνει εν αυτώ.
\par 15 Τα πλούτη όσα κατέπιε, θέλει εξεμέσει· ο Θεός θέλει εκσπάσει αυτά από της κοιλίας αυτού.
\par 16 Φαρμάκιον ασπίδων θέλει θηλάσει· γλώσσα εχίδνης θέλει θανατώσει αυτόν.
\par 17 Δεν θέλει ιδεί τους ποταμούς, τους ρύακας τους ρέοντας μέλι και βούτυρον.
\par 18 Εκείνο, διά το οποίον εκοπίασε, θέλει αποδώσει και δεν θέλει καταπίει αυτό· κατά την απόκτησιν θέλει γείνει η απόδοσις αυτού, και δεν θέλει χαρή.
\par 19 Διότι κατέθλιψεν, εγκατέλιπε τους πένητας· ήρπασεν οικίαν, την οποίαν δεν ωκοδόμησε.
\par 20 Βεβαίως δεν θέλει γνωρίσει ανάπαυσιν εν τη κοιλία αυτού· δεν θέλει διασώσει ουδέν εκ των επιθυμητών αυτού.
\par 21 Δεν θέλει μείνει εις αυτόν ουδέν προς τροφήν· όθεν δεν θέλει ελπίσει επί τα αγαθά αυτού.
\par 22 Εν τη πλήρει αφθονία αυτού θέλει επέλθει επ' αυτόν στενοχωρία· πάσα η δύναμις της ταλαιπωρίας θέλει επιπέσει επ' αυτόν.
\par 23 Ενώ καταγίνεται να εμπλήση την κοιλίαν αυτού, ο Θεός θέλει αποστείλει τον θυμόν της οργής αυτού επ' αυτόν, και θέλει επιβρέξει αυτόν κατ' αυτού ενώ τρώγει.
\par 24 Ενώ φεύγει το όπλον το σιδηρούν, το χάλκινον τόξον θέλει διαπεράσει αυτόν.
\par 25 Το βέλος σύρεται και διαπερά το σώμα, και η αστράπτουσα ακμή εξέρχεται εκ της χολής αυτού. Τρόμοι είναι επ' αυτόν,
\par 26 παν σκότος κρύπτεται εν τοις ταμείοις αυτού· πυρ άσβεστον θέλει κατατρώγει αυτόν· όσοι εναπελείφθησαν εν τη σκηνή αυτού θέλουσι δυστυχεί.
\par 27 Ο ουρανός θέλει ανακαλύψει την ανομίαν αυτού· και η γη θέλει σηκωθή κατ' αυτού.
\par 28 Η περιουσία του οίκου αυτού θέλει αφανισθή· θέλει διαρρεύσει εν τη ημέρα της κατ' αυτού οργής.
\par 29 Αύτη είναι η παρά του Θεού μερίς του ασεβούς ανθρώπου, και η κληρονομία η διωρισμένη εις αυτόν παρά του Θεού.

\chapter{21}

\par Και απεκρίθη ο Ιώβ και είπεν·
\par 2 Ακούσατε μετά προσοχής την ομιλίαν μου, και τούτο ας ήναι αντί των παρηγοριών σας.
\par 3 Υποφέρετέ με να λαλήσω· και αφού λαλήσω, εμπαίζετε.
\par 4 Μη εις άνθρωπον παραπονούμαι εγώ; διά τι λοιπόν να μη ταραχθή το πνεύμά μου;
\par 5 Εμβλέψατε εις εμέ και θαυμάσατε, και βάλετε χείρα επί στόματος.
\par 6 Μόνον να ενθυμηθώ, ταράττομαι, και τρόμος κυριεύει την σάρκα μου.
\par 7 Διά τι οι ασεβείς ζώσι, γηράσκουσι, μάλιστα ακμάζουσιν εις πλούτη;
\par 8 Το σπέρμα αυτών στερεούται έμπροσθεν αυτών μετ' αυτών, και τα έκγονα αυτών έμπροσθεν των οφθαλμών αυτών.
\par 9 Αι οικίαι αυτών είναι ασφαλείς από φόβου· και ράβδος Θεού δεν είναι επ' αυτούς.
\par 10 Ο βους αυτών συλλαμβάνει και δεν αποτυγχάνει· η δάμαλις αυτών τίκτει και δεν αποβάλλει.
\par 11 Απολύουσι τα τέκνα αυτών ως πρόβατα, και τα παιδία αυτών σκιρτώσι.
\par 12 Λαμβάνουσι το τύμπανον και την κιθάραν και ευφραίνονται εις τον ήχον του οργάνου.
\par 13 Διάγουσι τας ημέρας αυτών εν αγαθοίς και εν μιά στιγμή καταβαίνουσιν εις τον άδην.
\par 14 Και λέγουσι προς τον Θεόν, απόστηθι αφ' ημών, διότι δεν θέλομεν να γνωρίσωμεν τας οδούς σου·
\par 15 τι είναι ο Παντοδύναμος διά να δουλεύωμεν αυτόν; και τι ωφελούμεθα επικαλούμενοι αυτόν;
\par 16 Ιδού, τα αγαθά αυτών δεν είναι εν τη χειρί αυτών· μακράν απ' εμού η βουλή των ασεβών.
\par 17 Ποσάκις σβύνεται ο λύχνος των ασεβών, και έρχεται η καταστροφή αυτών επ' αυτούς Ο Θεός διαμοιράζει εις αυτούς ωδίνας εν τη οργή αυτού.
\par 18 Είναι ως άχυρον έμπροσθεν του ανέμου· και ως κονιορτός, τον οποίον αρπάζει ο ανεμοστρόβιλος.
\par 19 Ο Θεός φυλάττει την ποινήν της ανομίας αυτών διά τους υιούς αυτών· ανταποδίδει εις αυτούς, και θέλουσι γνωρίσει τούτο.
\par 20 Οι οφθαλμοί αυτών θέλουσιν ιδεί την καταστροφήν αυτών, και θέλουσι πίει από του θυμού του Παντοδυνάμου.
\par 21 Διότι ο ασεβής ποίαν ηδονήν έχει μεθ' εαυτόν εν τω οίκω αυτού, αφού κοπή εις το μέσον ο αριθμός των μηνών αυτού;
\par 22 Θέλει διδάξει τις τον Θεόν γνώσιν; και αυτός κρίνει τους υψηλούς.
\par 23 Ο μεν αποθνήσκει εν τω άκρω της ευδαιμονίας αυτού, ενώ είναι κατά πάντα ευτυχής και ήσυχος·
\par 24 τα πλευρά αυτού είναι πλήρη πάχους, και τα οστά αυτού ποτίζονται μυελόν.
\par 25 Ο δε αποθνήσκει εν πικρία ψυχής, και ποτέ δεν έφαγεν εν ευφροσύνη.
\par 26 Θέλουσι κοίτεσθαι ομού εν τω χώματι, και σκώληκες θέλουσι σκεπάσει αυτούς.
\par 27 Ιδού, γνωρίζω τους διαλογισμούς σας, και τας πονηρίας τας οποίας μηχανάσθε κατ' εμού.
\par 28 Διότι λέγετε, Που ο οίκος του άρχοντος; και που η σκηνή της κατοικήσεως των ασεβών;
\par 29 Δεν ηρωτήσατε τους διαβαίνοντας την οδόν; και τα σημεία αυτών δεν καταλαμβάνετε;
\par 30 Ότι ο ασεβής φυλάττεται εις ημέραν αφανισμού, εις ημέραν οργής φέρεται.
\par 31 Τις θέλει φανερώσει έμπροσθεν αυτού την οδόν αυτού; και τις θέλει ανταποδώσει εις αυτόν ό,τι αυτός έπραξε;
\par 32 και αυτός θέλει φερθή εις τον τάφον, και θέλει διαμένει εν τω μνήματι.
\par 33 Οι βώλοι της κοιλάδος θέλουσιν είσθαι γλυκείς εις αυτόν, και πας άνθρωπος θέλει υπάγει κατόπιν αυτού, καθώς αναρίθμητοι προπορεύονται αυτού.
\par 34 Πως λοιπόν με παρηγορείτε ματαίως, αφού εις τας αποκρίσεις σας μένει ψεύδος;

\chapter{22}

\par Και απεκρίθη Ελιφάς ο Θαιμανίτης και είπε·
\par 2 Δύναται άνθρωπος να ωφελήση τον Θεόν, διότι φρόνιμος ων δύναται να ωφελή εαυτόν;
\par 3 Είναι ευχαρίστησις εις τον Παντοδύναμον, εάν ήσαι δίκαιος; ή κέρδος, εάν καθιστάς αμέμπτους τας οδούς σου;
\par 4 Μήπως φοβούμενός σε θέλει σε ελέγξει και θέλει ελθεί εις κρίσιν μετά σου;
\par 5 Η κακία σου δεν είναι μεγάλη; και αι ανομίαι σου άπειροι;
\par 6 Διότι έλαβες ενέχυρον παρά του αδελφού σου αναιτίως και εστέρησας τους γυμνούς από του ενδύματος αυτών.
\par 7 Δεν επότισας ύδωρ τον διψώντα, και ηρνήθης άρτον εις τον πεινώντα.
\par 8 Ο δε ισχυρός άνθρωπος απελάμβανε την γήν· και ο περίβλεπτος κατώκει εν αυτή.
\par 9 Χήρας απέβαλες αβοηθήτους, και οι βραχίονες των ορφανών συνετρίβησαν υπό σου.
\par 10 Διά τούτο παγίδες σε περιεκύκλωσαν, και φόβος αιφνίδιος σε ταράττει·
\par 11 και σκότος, ώστε δεν βλέπεις· και πλημμύρα υδάτων σε σκεπάζει.
\par 12 Δεν είναι ο Θεός εν τοις υψηλοίς του ουρανού; και θεώρησον το ύψος των άστρων, πόσον υψηλά είναι
\par 13 Και συ λέγεις, Τι γνωρίζει ο Θεός; δύναται να κρίνη διά του γνόφου;
\par 14 Νέφη αποκρύπτουσιν αυτόν, και δεν βλέπει, και τον γύρον του ουρανού διαπορεύεται.
\par 15 Μήπως θέλεις φυλάξει την παντοτεινήν οδόν, την οποίαν επάτησαν οι άνομοι;
\par 16 Οίτινες αφηρπάσθησαν αώρως, και το θεμέλιον αυτών κατεπόντισε χείμαρρος·
\par 17 οίτινες είπον προς τον Θεόν, απόστηθι αφ' ημών· και τι θέλει κάμει ο Παντοδύναμος εις αυτούς;
\par 18 Αλλ' αυτός ενέπλησεν αγαθών τους οίκους αυτών· πλην μακράν απ' εμού η βουλή των ασεβών.
\par 19 Οι δίκαιοι βλέπουσι και αγάλλονται· και οι αθώοι μυκτηρίζουσιν αυτούς.
\par 20 Η μεν περιουσία ημών δεν ηφανίσθη, το υπόλοιπον όμως αυτών κατατρώγει πυρ.
\par 21 Οικειώθητι λοιπόν μετ' αυτού και έσο εν ειρήνη· ούτω θέλει ελθεί καλόν εις σε.
\par 22 Δέχθητι λοιπόν τον νόμον εκ του στόματος αυτού, και βάλε τους λόγους αυτού εν τη καρδία σου.
\par 23 Εάν επιστρέψης προς τον Παντοδύναμου, θέλεις ανοικοδομηθή, εκδιώξας την ανομίαν μακράν από των σκηνών σου.
\par 24 Και θέλεις επισωρεύσει το χρυσίον ως χώμα και το χρυσίον του Οφείρ ως τας πέτρας των χειμάρρων.
\par 25 Και ο Παντοδύναμος θέλει είσθαι ο υπερασπιστής σου, και θέλεις έχει πλήθος αργυρίου.
\par 26 Διότι τότε θέλεις ευφραίνεσθε εις τον Παντοδύναμον, και θέλεις υψώσει το πρόσωπόν σου προς τον Θεόν.
\par 27 Θέλεις δεηθή αυτού, και θέλει σου εισακούσει, και θέλεις αποδώσει τας ευχάς σου.
\par 28 Και ό,τι αποφασίσης, θέλει κατορθούσθαι εις σέ· και το φως θέλει φέγγει επί τας οδούς σου.
\par 29 Όταν ταπεινωθή τις, τότε θέλεις ειπεί, Είναι ύψωσις· διότι θέλει σώσει τον κεκυφότα τους οφθαλμούς.
\par 30 Θέλει σώσει και τον μη αθώον· ναι, διά της καθαρότητος των χειρών σου θέλει σωθή.

\chapter{23}

\par Και απεκρίθη ο Ιώβ και είπε·
\par 2 Και την σήμερον το παράπονόν μου είναι πικρόν· η πληγή μου είναι βαρυτέρα του στεναγμού μου.
\par 3 Είθε να ήξευρον που να εύρω αυτόν· ήθελον υπάγει έως του θρόνου αυτού·
\par 4 ήθελον εκθέσει κρίσιν ενώπιον αυτού, και ήθελον εμπλήσει το στόμα μου αποδείξεων·
\par 5 ήθελον γνωρίσει τους λόγους τους οποίους ήθελε μοι αποκριθή, και ήθελον νοήσει τι ήθελε μοι ειπεί.
\par 6 Μη εν πλήθει δυνάμεως θέλει διαμάχεσθαι μετ' εμού; ουχί· αλλ' ήθελε βάλει εις εμέ προσοχήν.
\par 7 Τότε ηδύνατο ο δίκαιος να διαλεχθή μετ' αυτού· και ήθελον ελευθερωθή διαπαντός από του κριτού μου.
\par 8 Ιδού, υπάγω εμπρός, αλλά δεν είναι· και οπίσω, αλλά δεν βλέπω αυτόν·
\par 9 εις τα αριστερά, όταν εργάζηται, αλλά δεν δύναμαι να ίδω αυτόν. Κρύπτεται εις τα δεξιά, και δεν βλέπω αυτόν.
\par 10 Γνωρίζει όμως την οδόν μου· με εδοκίμασε· θέλω εξέλθει ως χρυσίον.
\par 11 Ο πους μου ενέμεινεν εις τα βήματα αυτού· εφύλαξα την οδόν αυτού και δεν εξέκλινα·
\par 12 την εντολήν των χειλέων αυτού, και δεν ωπισθοδρόμησα· διετήρησα τους λόγους του στόματος αυτού, μάλλον παρά την αναγκαίαν μου τροφήν.
\par 13 Διότι αυτός είναι εν μιά βουλή· και τις δύναται να αποστρέψη αυτόν; και ό,τι επιθυμεί η ψυχή αυτού, κάμνει.
\par 14 Διότι εκτελεί το ορισθέν εις εμέ· και πολλά τοιαύτα είναι μετ' αυτού.
\par 15 Διά τούτο καταπλήττομαι από προσώπου αυτού· συλλογίζομαι και φρίττω απ' αυτού·
\par 16 διότι ο Θεός εμαλάκωσε την καρδίαν μου, και ο Παντοδύναμος με κατέπληξεν·
\par 17 επειδή δεν απεκόπην προ του σκότους, και δεν έκρυψε τον γνόφον από του προσώπου μου.

\chapter{24}

\par Επειδή οι καιροί δεν είναι κεκρυμμένοι από του Παντοδυνάμου, διά τι οι γνωρίζοντες αυτόν δεν βλέπουσι τας ημέρας αυτού;
\par 2 Μετακινούσιν όρια· αρπάζουσι ποίμνια και ποιμαίνουσιν·
\par 3 αφαιρούσι την όνον των ορφανών· λαμβάνουσι τον βουν της χήρας εις ενέχυρον·
\par 4 εξωθούσι τους ενδεείς από της οδού· οι πτωχοί της γης ομού κρύπτονται.
\par 5 Ιδού, ως άγριοι όνοι εν τη ερήμω, εξέρχονται εις τα έργα αυτών εγειρόμενοι πρωΐ διά αρπαγήν· η έρημος δίδει τροφήν δι' αυτούς και διά τα τέκνα αυτών.
\par 6 Θερίζουσιν αγρόν μη όντα εαυτών, και τρυγώσιν άμπελον αδικίας.
\par 7 Κάμνουσι τους γυμνούς να νυκτερεύωσιν άνευ ιματίου, και δεν έχουσι σκέπασμα εις το ψύχος.
\par 8 Υγραίνονται εκ των βροχών των ορέων και εναγκαλίζονται τον βράχον, μη έχοντες καταφύγιον.
\par 9 Εκείνοι αρπάζουσι τον ορφανόν από του μαστού, και λαμβάνουσιν ενέχυρον παρά του πτωχού·
\par 10 κάμνουσιν αυτόν να υπάγη γυμνός άνευ ιματίου, και οι βαστάζοντες τα χειρόβολα μένουσι πεινώντες.
\par 11 Οι εκπιέζοντες το έλαιον εντός των τοίχων αυτών και πατούντες τους ληνούς αυτών, διψώσιν.
\par 12 Άνθρωποι στενάζουσιν εκ της πόλεως, και η ψυχή των πεπληγωμένων βοά· αλλ' ο Θεός δεν επιθέτει εις αυτούς αφροσύνην.
\par 13 Ούτοι είναι εκ των ανθισταμένων εις το φώς· δεν γνωρίζουσι τας οδούς αυτού, και δεν μένουσιν εν ταις τρίβοις αυτού.
\par 14 Ο φονεύς εγειρόμενος την αυγήν φονεύει τον πτωχόν και τον ενδεή, την δε νύκτα γίνεται ως κλέπτης.
\par 15 Ο οφθαλμός ομοίως του μοιχού παραφυλάττει το νύκτωμα, λέγων, Οφθαλμός δεν θέλει με ιδεί· και καλύπτει το πρόσωπον αυτού.
\par 16 Εν τω σκότει διατρυπώσι τας οικίας, τας οποίας την ημέραν εσημείωσαν δι' εαυτούς. Δεν γνωρίζουσι φώς·
\par 17 διότι η αυγή είναι εις πάντας αυτούς σκιά θανάτου· εάν τις γνωρίση αυτούς, είναι τρόμοι σκιάς θανάτου.
\par 18 Είναι ελαφροί επί το πρόσωπον των υδάτων· η μερίς αυτών είναι κατηραμένη επί της γής· δεν βλέπουσι την οδόν των αμπέλων.
\par 19 Η ξηρασία και η θερμότης αρπάζουσι τα ύδατα της χιόνος, ο δε τάφος τους αμαρτωλούς.
\par 20 Η μήτρα θέλει λησμονήσει αυτούς· ο σκώληξ θέλει βόσκεσθαι επ' αυτούς· δεν θέλουσιν ελθεί πλέον εις ενθύμησιν· και η αδικία θέλει συντριφθή ως ξύλον.
\par 21 Κακοποιούσι την στείραν την άτεκνον· και δεν αγαθοποιούσι την χήραν·
\par 22 και κατακρατούσι τους δυνατούς διά της δυνάμεως αυτών· εγείρονται, και δεν είναι ουδείς ασφαλής εν τη ζωή αυτού.
\par 23 Έδωκε μεν ο Θεός εις αυτούς ασφάλειαν και αναπαύονται· όμως οι οφθαλμοί αυτού είναι επί τας οδούς αυτών.
\par 24 Υψόνονται ολίγον καιρόν και δεν υπάρχουσι, και καταβάλλονται ως πάντες· σηκόνονται εκ του μέσου και αποκόπτονται ως η κεφαλή των ασταχύων·
\par 25 και εάν τώρα δεν ήναι ούτω, τις θέλει με διαψεύσει και εξουθενίσει τους λόγους μου;

\chapter{25}

\par Και απεκρίθη Βιλδάδ ο Σαυχίτης και είπεν·
\par 2 Εξουσία και φόβος είναι μετ' αυτού· εκτελεί ειρήνην εις τα ύψη αυτού.
\par 3 Υπάρχει αριθμός των στρατευμάτων αυτού; και επί τίνα δεν ανατέλλει το φως αυτού;
\par 4 Πως λοιπόν δύναται άνθρωπος να δικαιωθή ενώπιον του Θεού; ή πως δύναται να ήναι καθαρός ο γεγεννημένος εκ γυναικός;
\par 5 Ιδού, και αυτή η σελήνη δεν είναι λαμπρά, και οι αστέρες δεν είναι καθαροί ενώπιον αυτού.
\par 6 Πόσον ολιγώτερον ο άνθρωπος, σαπρία; και ο υιός του ανθρώπου, ο σκώληξ;

\chapter{26}

\par Και απεκρίθη ο Ιώβ και είπε·
\par 2 Πόσον εβοήθησας τον αδύνατον· έσωσας βραχίονα ανίσχυρον.
\par 3 Πόσον συνεβούλευσας τον άσοφον και εντελή σύνεσιν έδειξας
\par 4 Προς τίνα απήγγειλας τους λόγους; και τίνος πνοή εξήλθεν από σου;
\par 5 Οι νεκροί τρέμουσιν αυτόν υποκάτωθεν των υδάτων, και οι συγκατοικούντες μετ' αυτών.
\par 6 Γυμνός ο άδης έμπροσθεν αυτού, και η απώλεια δεν έχει σκέπασμα.
\par 7 Εκτείνει τον βορέαν επί το κενόν· κρεμά την γην επί το μηδέν.
\par 8 Δεσμεύει τα ύδατα εις τας νεφέλας αυτού· και η νεφέλη δεν σχίζεται υποκάτω αυτών.
\par 9 Σκεπάζει το πρόσωπον του θρόνου αυτού· εκτείνει το νέφος αυτού επ' αυτόν.
\par 10 Περιεκύκλωσε τα ύδατα με όρια, έως της συντελείας του φωτός και του σκότους.
\par 11 Οι στύλοι του ουρανού τρέμουσι και εξίστανται από της επιτιμήσεως αυτού.
\par 12 Ταράττει την θάλασσαν διά της δυνάμεως αυτού, και διά της συνέσεως αυτού καταδαμάζει την υπερηφανίαν αυτής.
\par 13 Διά του πνεύματος αυτού εκόσμησε τους ουρανούς· η χειρ αυτού εσχημάτισε τον συστρεφόμενον όφιν.
\par 14 Ιδού, ταύτα είναι μέρη των οδών αυτού· αλλά πόσον ελάχιστον πράγμα ακούομεν περί αυτού; την δε βροντήν της δυνάμεως αυτού τις δύναται να εννοήση;

\chapter{27}

\par Και εξηκολούθησεν ο Ιώβ την παραβολήν αυτού και είπε·
\par 2 Ζη ο Θεός, ο αποβαλών την κρίσιν μου, και ο Παντοδύναμος, ο πικράνας την ψυχήν μου,
\par 3 ότι πάντα τον χρόνον ενόσω η πνοή μου είναι εν εμοί και το πνεύμα του Θεού εις τους μυκτήράς μου,
\par 4 τα χείλη μου δεν θέλουσι λαλήσει αδικίαν και η γλώσσα μου δεν θέλει μελετήσει δόλον.
\par 5 Μη γένοιτο εις εμέ να σας δικαιώσω· έως να εκπνεύσω, δεν θέλω απομακρύνει την ακεραιότητά μου απ' εμού.
\par 6 Θέλω κρατεί την δικαιοσύνην μου και δεν θέλω αφήσει αυτήν· η καρδία μου δεν θέλει με ελέγξει ενόσω ζω.
\par 7 Ο εχθρός μου να ήναι ως ο ασεβής και ο ανιστάμενος κατ' εμού ως ο παράνομος.
\par 8 Διότι τις η ελπίς του υποκριτού, αν και επλεονέκτησεν, όταν ο Θεός αποσπά την ψυχήν αυτού;
\par 9 Άραγε θέλει ακούσει ο Θεός την κραυγήν αυτού, όταν επέλθη επ' αυτόν συμφορά;
\par 10 Θέλει ευφραίνεσθαι εις τον Παντοδύναμον; θέλει επικαλείσθαι τον Θεόν εν παντί καιρώ;
\par 11 θέλω σας διδάξει τι είναι εν τη χειρί του Θεού· ό,τι είναι παρά τω Παντοδυνάμω, δεν θέλω κρύψει αυτό.
\par 12 Ιδού, σεις πάντες είδετε· διά τι λοιπόν είσθε όλως τόσον μάταιοι;
\par 13 Τούτο είναι παρά Θεού η μερίς του ασεβούς ανθρώπου, και η κληρονομία των δυναστών, την οποίαν θέλουσι λάβει παρά του Παντοδυνάμου.
\par 14 Εάν οι υιοί αυτού πολλαπλασιασθώσιν, είναι διά την ρομφαίαν· και οι έκγονοι αυτού δεν θέλουσι χορτασθή άρτον.
\par 15 Οι εναπολειφθέντες αυτού θέλουσι ταφή εν θανάτω· και αι χήραι αυτού δεν θέλουσι κλαύσει.
\par 16 Και αν επισωρεύση αργύριον ως το χώμα και ετοιμάση ιμάτια ως τον πηλόν·
\par 17 δύναται μεν να ετοιμάση, πλην ο δίκαιος θέλει ενδυθή αυτά· και ο αθώος θέλει διαμοιρασθή το αργύριον.
\par 18 Οικοδομεί τον οίκον αυτού ως το σαράκιον, και ως καλύβην, την οποίαν κάμνει ο αγροφύλαξ.
\par 19 Πλαγιάζει πλούσιος, πλην δεν θέλει συναχθή· ανοίγει τους οφθαλμούς αυτού και δεν υπάρχει.
\par 20 Τρόμοι συλλαμβάνουσιν αυτόν ως ύδατα, ανεμοστρόβιλος αρπάζει αυτόν την νύκτα.
\par 21 Σηκόνει αυτόν ανατολικός άνεμος, και υπάγει· και αποσπά αυτόν από του τόπου αυτού.
\par 22 Διότι ο Θεός θέλει ρίψει κατ' αυτού συμφοράς και δεν θέλει φεισθή· από της χειρός αυτού σπεύδει να φύγη.
\par 23 Θέλουσι κροτήσει τας χείρας αυτών επ' αυτόν, και θέλουσι συρίξει αυτόν από του τόπου αυτών.

\chapter{28}

\par Βεβαίως είναι τόπος του αργυρίου όθεν εξάγεται, και τόπος του χρυσίου όπου καθαρίζεται·
\par 2 ο σίδηρος λαμβάνεται εκ της γης και ο χαλκός χύνεται εκ της πέτρας.
\par 3 Βάλλει μεν ο άνθρωπος όρια εις το σκότος και ανιχνεύει τα πάντα μέχρι τελειότητος· τους λίθους του σκότους και της σκιάς του θανάτου.
\par 4 Χείμαρρος εξορμά εκ του τόπου όπου κατοικεί· ύδατα αδοκίμαστα υπό του ποδός· ταύτα ολιγοστεύουσι και αναχωρούσιν από των ανθρώπων.
\par 5 Περί δε της γης, εξ αυτής εξέρχεται ο άρτος και υποκάτωθεν αυτής ανασκάπτεται ως υπό πυρός·
\par 6 οι λίθοι αυτής είναι τόπος σαπφείρων· και εν αυτή χώμα χρυσίου.
\par 7 Την οδόν εκείνην δεν γνωρίζει πτηνόν και οφθαλμός γυπός δεν είδεν αυτήν·
\par 8 τα θηρία δεν επάτησαν αυτήν, ο άγριος λέων δεν επέρασε δι' αυτής.
\par 9 Εκτείνει την χείρα αυτού επί τον σκληρόν βράχον· ανατρέπει τα όρη από της ρίζης.
\par 10 Εγκόπτει ποταμούς μεταξύ των βράχων· και ο οφθαλμός αυτού ανακαλύπτει παν πολύτιμον.
\par 11 Δεσμεύει των ποταμών την πλημμύραν· και το κεκρυμμένον εκφέρει εις φως.
\par 12 Αλλ' η σοφία πόθεν θέλει ευρεθή; και που είναι ο τόπος της συνέσεως;
\par 13 Ο άνθρωπος δεν γνωρίζει την τιμήν αυτής· και δεν ευρίσκεται εν τη γη των ζώντων.
\par 14 Η άβυσσος λέγει, δεν είναι εν εμοί· και η θάλασσα λέγει, δεν είναι μετ' εμού.
\par 15 Δεν δύναται να δοθή χρυσίον αντ' αυτής· και αργύριον δεν δύναται να ζυγισθή εις αντάλλαγμα αυτής.
\par 16 Δεν δύναται να εκτιμηθή με το χρυσίον του Οφείρ, με τον πολύτιμον όνυχα και σάπφειρον.
\par 17 Το χρυσίον και ο κρύσταλλος δεν δύναται να εξισωθώσι με αυτήν· και αντάλλαγμα αυτής να γείνη με σκεύη καθαρωτάτου χρυσίου.
\par 18 Δεν θέλει μνημονευθή κοράλλιον, ή μαργαρίται· διότι η τιμή της σοφίας είναι υπερτέρα των πολυτίμων λίθων.
\par 19 Το τοπάζιον της Αιθιοπίας δεν θέλει εξισωθή με αυτήν· δεν θέλει εκτιμηθή με καθαρόν χρυσίον.
\par 20 Πόθεν λοιπόν έρχεται η σοφία; και που είναι ο τόπος της συνέσεως;
\par 21 Είναι βεβαίως κεκρυμμένη από των οφθαλμών πάντων των ζώντων, και εσκεπασμένη από των πτηνών του ουρανού.
\par 22 Η απώλεια και ο θάνατος λέγουσι, Διά των ώτων ημών ηκούσαμεν την φήμην αυτής.
\par 23 Ο Θεός εννοεί την οδόν αυτής, και αυτός γνωρίζει τον τόπον αυτής.
\par 24 Επειδή αυτός θεωρεί έως των περάτων της γης, βλέπει υποκάτω παντός του ουρανού,
\par 25 διά να ζυγίζη το βάρος των ανέμων, και να σταθμίζη τα ύδατα με μέτρον.
\par 26 Ότε έκαμε νόμον διά την βροχήν και οδόν διά την αστραπήν της βροντής,
\par 27 τότε είδε και εφανέρωσεν αυτήν· ητοίμασεν αυτήν και μάλιστα εξιχνίασεν αυτήν.
\par 28 Και είπε προς τον άνθρωπον, Ιδού, ο φόβος του Κυρίου, ούτος είναι η σοφία, και η αποχή από του κακού σύνεσις.

\chapter{29}

\par Και εξηκολούθησεν ο Ιώβ την παραβολήν αυτού και είπεν·
\par 2 Ω να ήμην ως εις τους παρελθόντας μήνας, ως εν ταις ημέραις ότε ο Θεός με εφύλαττεν·
\par 3 ότε ο λύχνος αυτού έφεγγεν επί της κεφαλής μου, και διά του φωτός αυτού περιεπάτουν εν τω σκότει·
\par 4 καθώς ήμην εν ταις ημέραις της ακμής μου, ότε η εύνοια του Θεού ήτο επί την σκηνήν μου·
\par 5 ότε ο Παντοδύναμος ήτο μετ' εμού, και τα παιδία μου κύκλω μου·
\par 6 ότε έπλυνον τα βήματά μου με βούτυρον, και ο βράχος εξέχεε δι' εμέ ποταμούς ελαίου·
\par 7 ότε διά της πόλεως εξηρχόμην εις την πύλην, ητοίμαζον την καθέδραν μου εν τη πλατεία
\par 8 Οι νέοι με έβλεπον και εκρύπτοντο· και οι γέροντες εγειρόμενοι ίσταντο.
\par 9 Οι άρχοντες έπαυον ομιλούντες και έβαλλον χείρα επί το στόμα αυτών.
\par 10 Η φωνή των εγκρίτων εκρατείτο, και η γλώσσα αυτών εκολλάτο εις τον ουρανίσκον αυτών.
\par 11 Ότε το ωτίον ήκουε και με εμακάριζε, και ο οφθαλμός έβλεπε και εμαρτύρει υπέρ εμού·
\par 12 διότι ηλευθέρουν τον πτωχόν βοώντα και τον ορφανόν τον μη έχοντα βοηθόν.
\par 13 Η ευλογία του απολλυμένου ήρχετο επ' εμέ· και την καρδίαν της χήρας εύφραινον.
\par 14 Εφόρουν δικαιοσύνην και ενεδυόμην την ευθύτητά μου ως επενδύτην και διάδημα.
\par 15 Ήμην οφθαλμός εις τον τυφλόν και πους εις τον χωλόν εγώ.
\par 16 Ήμην πατήρ εις τους πτωχούς, και την δίκην την οποίαν δεν εγνώριζον εξιχνίαζον.
\par 17 Και συνέτριβον τους κυνόδοντας του αδίκου και απέσπων το θήραμα από των οδόντων αυτού.
\par 18 Τότε έλεγον, θέλω αποθάνει εν τη φωλεά μου και ως την άμμον θέλω πολλαπλασιάσει τας ημέρας μου.
\par 19 Η ρίζα μου ήτο ανοικτή προς τα ύδατα, και η δρόσος διενυκτέρευεν επί των κλάδων μου.
\par 20 Η δόξα μου ανενεούτο εν εμοί, και το τόξον μου εκρατύνετο εν τη χειρί μου.
\par 21 Με ηκροάζοντο προσέχοντες και εις την συμβουλήν μου εσιώπων.
\par 22 Μετά τους λόγους μου δεν προσέθετον ουδέν, και η ομιλία μου εστάλαζεν επ' αυτούς.
\par 23 Και με περιέμενον ως την βροχήν· και ήσαν κεχηνότες ως διά την όψιμον βροχήν.
\par 24 Εγέλων προς αυτούς, και δεν επίστευον· και την φαιδρότητα του προσώπου μου δεν άφινον να πέση.
\par 25 Εάν ηρεσκόμην εις την οδόν αυτών, εκαθήμην πρώτος, και κατεσκήνουν ως βασιλεύς εν τω στρατεύματι, ως ο παρηγορών τους τεθλιμμένους.

\chapter{30}

\par Αλλά τώρα οι νεώτεροί μου την ηλικίαν με περιγελώσι, των οποίων τους πατέρας δεν ήθελον καταδεχθή να βάλω μετά των κυνών του ποιμνίου μου.
\par 2 Και εις τι τωόντι ηδύνατο να με ωφελήση η δύναμις των χειρών αυτών, εις τους οποίους η ισχύς εξέλιπε;
\par 3 Δι' ένδειαν και πείναν ήσαν απομεμονωμένοι· έφευγον εις γην άνυδρον, σκοτεινήν, ηφανισμένην και έρημον·
\par 4 έκοπτον μολόχην πλησίον των θάμνων και την ρίζαν των αρκεύθων διά τροφήν αυτών.
\par 5 Ήσαν εκ μέσου δεδιωγμένοι· εφώναζον επ' αυτούς ως κλέπτας.
\par 6 Κατώκουν εν τοις κρημνοίς των χειμάρρων, ταις τρύπαις της γης και τοις βρόχοις.
\par 7 Μεταξύ των θάμνων ωγκώντο· υποκάτω των ακανθών συνήγοντο·
\par 8 άφρονες και δύσφημοι, εκδεδιωγμένοι εκ της γης.
\par 9 Και τώρα εγώ είμαι το τραγώδιον αυτών, είμαι και η παροιμία αυτών.
\par 10 Με βδελύττονται, απομακρύνονται απ' εμού, και δεν συστέλλονται να πτύωσιν εις το πρόσωπόν μου.
\par 11 Επειδή ο Θεός διέλυσε την υπεροχήν μου και με έθλιψεν, απέρριψαν και αυτοί τον χαλινόν έμπροσθέν μου.
\par 12 Εκ δεξιών ανίστανται οι νέοι· απωθούσι τους πόδας μου, και ετοιμάζουσι κατ' εμού τας ολεθρίους οδούς αυτών.
\par 13 Ανατρέπουσι την οδόν μου, επαυξάνουσι την συμφοράν μου, χωρίς να έχωσι βοηθόν.
\par 14 Εφορμώσιν ως σφοδρά πλημμύρα, επί της ερημώσεώς μου περικυλίονται.
\par 15 Τρόμοι εστράφησαν επ' εμέ· καταδιώκουσι την ψυχήν μου ως άνεμος· και η σωτηρία μου παρέρχεται ως νέφος.
\par 16 Και τώρα η ψυχή μου εξεχύθη εντός μου· ημέραι θλίψεως με κατέλαβον.
\par 17 Την νύκτα τα οστά μου διεπεράσθησαν εν εμοί, και τα νεύρά μου δεν αναπαύονται.
\par 18 Υπό της σφοδράς δυνάμεως ηλλοιώθη το ένδυμά μου· με περισφίγγει ως το περιλαίμιον του χιτώνος μου.
\par 19 Με έρριψεν εις τον πηλόν, και ωμοιώθην με χώμα και κόνιν.
\par 20 Κράζω προς σε, και δεν μοι αποκρίνεσαι· ίσταμαι, και με παραβλέπεις.
\par 21 Έγεινες ανελεήμων προς εμέ· διά της κραταιάς χειρός σου με μαστιγόνεις.
\par 22 Με εσήκωσας επί τον άνεμον· με επεβίβασας και διέλυσας την ουσίαν μου.
\par 23 Εξεύρω μεν ότι θέλεις με φέρει εις θάνατον και τον οίκον τον προσδιωρισμένον εις πάντα ζώντα.
\par 24 Αλλά δεν θέλει εκτείνει χείρα εις τον τάφον, εάν κράζωσι προς αυτόν όταν αφανίζη.
\par 25 Δεν έκλαυσα εγώ διά τον όντα εν ημέραις σκληραίς, και ελυπήθη η ψυχή μου διά τον πτωχόν;
\par 26 Ενώ περιέμενον το καλόν, τότε ήλθε το κακόν· και ενώ ανέμενον το φως, τότε ήλθε το σκότος.
\par 27 Τα εντόσθιά μου ανέβρασαν και δεν ανεπαύθησαν· ημέραι θλίψεως με προέφθασαν.
\par 28 Περιεπάτησα μελαγχροινός ουχί υπό ηλίου· εσηκώθην, εβοήσα εν συνάξει.
\par 29 Έγεινα αδελφός των δρακόντων και σύντροφος των στρουθοκαμήλων.
\par 30 Το δέρμα μου εμαύρισεν επ' εμέ, και τα οστά μου κατεκαύθησαν υπό της φλογώσεως.
\par 31 Η δε κιθάρα μου μετεβλήθη εις πένθος και το όργανόν μου εις φωνήν κλαιόντων.

\chapter{31}

\par Έκαμον συνθήκην μετά των οφθαλμών μου· και πως να έχω τον στοχασμόν μου επί παρθένον;
\par 2 και τι το μερίδιον παρά Θεού άνωθεν; και η κληρονομία του Παντοδυνάμου εκ των υψηλών;
\par 3 Ουχί αφανισμός διά τον ασεβή; και ταλαιπωρία διά τους εργάτας της ανομίας;
\par 4 δεν βλέπει αυτός τας οδούς μου και απαριθμεί πάντα τα βήματά μου;
\par 5 Εάν περιεπάτησα με ψεύδος, ή ο πους μου έσπευσεν εις δόλον,
\par 6 ας με ζυγίση διά της στάθμης της δικαιοσύνης και ας γνωρίση ο Θεός την ακεραιότητά μου·
\par 7 αν το βήμά μου εξετράπη από της οδού και η καρδία μου επηκολούθησε τους οφθαλμούς μου, και αν κηλίς προσεκολλήθη εις τας χείρας μου·
\par 8 να σπείρω, και άλλος να φάγη· και να εκριζωθώσιν οι έκγονοί μου.
\par 9 Αν η καρδία μου ηπατήθη υπό γυναικός, ή παρεμόνευσα εις την θύραν του πλησίον μου,
\par 10 η γυνή μου να αλέση δι' άλλον, και άλλοι να πέσωσιν επ' αυτήν.
\par 11 Διότι μιαρόν ανόμημα τούτο και αμάρτημα κατάδικον·
\par 12 διότι είναι πυρ κατατρώγον μέχρις αφανισμού, και ήθελεν εκριζώσει πάντα τα γεννήματά μου.
\par 13 Αν κατεφρόνησα την κρίσιν του δούλου μου ή της δούλης μου, ότε διεφέροντο προς εμέ,
\par 14 τι θέλω κάμει τότε, όταν εγερθή ο Θεός; και όταν επισκεφθή, τι θέλω αποκριθή προς αυτόν;
\par 15 Ο ποιήσας εμέ εν τη κοιλία, δεν εποίησε και εκείνον; και δεν εμόρφωσεν ημάς ο αυτός εν τη μήτρα;
\par 16 Αν ηρνήθην την επιθυμίαν των πτωχών, ή εμάρανα τους οφθαλμούς της χήρας,
\par 17 ή έφαγον μόνος τον άρτον μου, και ο ορφανός δεν έφαγεν εξ αυτού·
\par 18 διότι ο μεν εκ νεότητος μου ετρέφετο μετ' εμού, ως μετά πατρός, την δε εκ κοιλίας της μητρός μου ωδήγησα·
\par 19 αν είδον τινά απολλύμενον δι' έλλειψιν ενδύματος ή πτωχόν χωρίς σκεπάσματος,
\par 20 αν οι νεφροί αυτού δεν με ευλόγησαν και δεν εθερμάνθη με το μαλλίον των προβάτων μου,
\par 21 αν εσήκωσα την χείρα μου κατά του ορφανού, βλέπων ότι υπερίσχυον εν τη πύλη,
\par 22 να πέση ο βραχίων μου εκ του ώμου, και η χειρ μου να συντριφθή εκ του αγκώνος.
\par 23 Διότι ο παρά του Θεού όλεθρος ήτο εις εμέ φρίκη και διά την μεγαλειότητα αυτού δεν ήθελον δυνηθή να ανθέξω.
\par 24 Αν έθεσα εις το χρυσίον την ελπίδα μου, ή είπα προς το καθαρόν χρυσίον, Συ είσαι το θάρρος μου,
\par 25 αν ευφράνθην διότι ο πλούτος μου ήτο μέγας και διότι η χειρ μου εύρηκεν αφθονίαν,
\par 26 αν εθεώρουν τον ήλιον αναλάμποντα ή την σελήνην περιπατούσαν εν τη λαμπρότητι αυτής,
\par 27 και η καρδία μου εθέλχθη κρυφίως, ή με το στόμα μου εφίλησα την χείρα μου,
\par 28 και τούτο ήθελεν είσθαι ανόμημα κατάδικον· διότι ήθελον αρνηθή τον Θεόν τον Ύψιστον.
\par 29 Αν εχάρην εις τον αφανισμόν του μισούντός με, ή επεχάρην ότε εύρηκεν αυτόν κακόν·
\par 30 διότι ουδέ αφήκα το στόμα μου να αμαρτήση, ευχόμενος κατάραν εις την ψυχήν αυτού·
\par 31 αν οι άνθρωποι της σκηνής μου δεν είπον, τις θέλει δείξει άνθρωπον μη χορτασθέντα από των κρεάτων αυτού;
\par 32 Ο ξένος δεν διενυκτέρευεν έξω· ήνοιγον την θύραν μου εις τον οδοιπόρον·
\par 33 αν εσκέπασα την παράβασίν μου ως ο Αδάμ, κρύπτων την ανομίαν μου εν τω κόλπω μου·
\par 34 διότι μήπως εφοβούμην μέγα πλήθος, ή με ετρόμαζεν η καταφρόνησις των οικογενειών, ώστε να σιωπήσω και να μη εκβώ εκ της θύρας;
\par 35 Ω να ήτο τις να με ήκουεν. Ιδού, η επιθυμία μου είναι να απεκρίνετο ο Παντοδύναμος εις εμέ, και ο αντίδικός μου να έγραφε βιβλίον.
\par 36 Βεβαίως ήθελον βαστάσει αυτό επί του ώμου μου, ήθελον περιδέσει αυτό στέφανον επ' εμέ·
\par 37 ήθελον φανερώσει προς αυτόν τον αριθμόν των βημάτων μου· ως άρχων ήθελον πλησιάσει εις αυτόν.
\par 38 Αν ο αγρός μου καταβοά εναντίον μου και κλαίωσιν ομού οι αύλακες αυτού,
\par 39 αν έφαγον τον καρπόν αυτόν χωρίς μισθόν, ή έκαμον να εκβή η ψυχή των γεωργών αυτού,
\par 40 Ας φυτρώσωσι τρίβολοι αντί σίτου και ζιζάνια αντί κριθής. Ετελείωσαν οι λόγοι του Ιώβ.

\chapter{32}

\par Έπαυσαν δε και οι τρεις ούτοι άνθρωποι αποκρινόμενοι προς τον Ιώβ, διότι ήτο δίκαιος εις τους οφθαλμούς αυτού.
\par 2 Τότε εξήφθη ο θυμός του Ελιού, υιού του Βαραχιήλ του Βουζίτου, εκ της συγγενείας του Αράμ· κατά του Ιώβ εξήφθη ο θυμός αυτού, διότι εδικαίονεν εαυτόν μάλλον παρά τον Θεόν.
\par 3 Και κατά των τριών αυτού φίλων εξήφθη ο θυμός αυτού, διότι δεν εύρηκαν απόκρισιν και κατεδίκασαν τον Ιώβ.
\par 4 Ο δε Ελιού περιέμενε να λαλήση προς τον Ιώβ, διότι εκείνοι ήσαν γεροντότεροι αυτού.
\par 5 Ότε δε ο Ελιού είδεν, ότι δεν ήτο απόκρισις εν τω στόματι των τριών ανδρών, εξήφθη ο θυμός αυτού.
\par 6 και απεκρίθη ο Ελιού ο υιός του Βαραχιήλ του Βουζίτου και είπεν· Εγώ είμαι νέος την ηλικίαν, και σεις γέροντες· διά τούτο εφοβήθην και συνεστάλην να σας φανερώσω την γνώμην μου.
\par 7 Εγώ είπα, Αι ημέραι ας λαλήσωσι και το πλήθος των ετών ας διδάξη, σοφίαν.
\par 8 Βεβαίως είναι πνεύμα εν τω ανθρώπω η έμπνευσις όμως του Παντοδυνάμου συνετίζει αυτόν.
\par 9 Οι μεγαλήτεροι δεν είναι πάντοτε σοφοί· ούτε οι γέροντες νοούσι κρίσιν.
\par 10 Διά τούτο είπα, Ακούσατέ μου· θέλω φανερώσει και εγώ την γνώμην μου.
\par 11 Ιδού, επρόσμενα τους λόγους σας· ηκροάσθην τα επιχειρήματά σας, εωσού εξετάσητε τους λόγους.
\par 12 Και σας παρετήρουν, και ιδού, ουδείς εξ υμών ηδυνήθη να καταπείση τον Ιώβ, αποκρινόμενος εις τους λόγους αυτού·
\par 13 διά να μη είπητε, Ημείς ευρήκαμεν σοφίαν. Ο Θεός θέλει καταβάλει αυτόν, ουχί άνθρωπος.
\par 14 Εκείνος δε δεν διηύθυνε λόγους προς εμέ· και δεν θέλω αποκριθή προς αυτόν κατά τας ομιλίας σας.
\par 15 Εκείνοι ετρόμαξαν, δεν απεκρίθησαν πλέον· έχασαν τους λόγους αυτών.
\par 16 Και περιέμενον, επειδή δεν ελάλουν· αλλ' ίσταντο· δεν απεκρίνοντο πλέον.
\par 17 Ας αποκριθώ και εγώ το μέρος μου· ας φανερώσω και εγώ την γνώμην μου.
\par 18 Διότι είμαι πλήρης λόγων· το πνεύμα εντός μου με αναγκάζει.
\par 19 Ιδού, η κοιλία μου είναι ως οίνος όστις δεν ηνοίχθη· είναι ετοίμη να σπάση, ως ασκοί γλεύκους.
\par 20 Θέλω λαλήσει διά να αναπνεύσω· θέλω ανοίξει τα χείλη μου και αποκριθή.
\par 21 Μη γένοιτο να γείνω προσωπολήπτης, μηδέ να κολακεύσω άνθρωπον.
\par 22 Διότι δεν εξεύρω να κολακεύω· ο Ποιητής μου ήθελε με αναρπάσει ευθύς.

\chapter{33}

\par Διά τούτο, Ιώβ, άκουσον τώρα τας ομιλίας μου, και ακροάσθητι πάντας τους λόγους μου.
\par 2 Ιδού, τώρα ήνοιξα το στόμα μου· η γλώσσα μου λαλεί εν τω στόματί μου.
\par 3 Οι λόγοι μου θέλουσιν είσθαι κατά την ευθύτητα της καρδίας μου· και τα χείλη μου θέλουσι προφέρει γνώσιν καθαράν.
\par 4 Το Πνεύμα του Θεού με έκαμε και η πνοή του Παντοδυνάμου με εζωοποίησεν.
\par 5 Εάν δύνασαι, αποκρίθητί μοι· παρατάχθητι έμπροσθέν μου· στήθι.
\par 6 Ιδού, εγώ είμαι κατά τον λόγόν σου από μέρους του Θεού· εκ πηλού είμαι και εγώ μεμορφωμένος.
\par 7 Ιδού, ο τρόμος μου δεν θέλει σε ταράξει, ουδέ η χειρ μου θέλει είσθαι βαρεία επί σε.
\par 8 Συ τωόντι είπας εις τα ώτα μου, και ήκουσα την φωνήν των λόγων σου,
\par 9 Είμαι καθαρός χωρίς αμαρτίας· είμαι αθώος· και ανομία δεν υπάρχει εν εμοί·
\par 10 ιδού, ευρίσκει αφορμάς εναντίον μου· με νομίζει εχθρόν αυτού·
\par 11 βάλλει τους πόδας μου εν τω ξύλω· παραφυλάττει πάσας τας οδούς μου.
\par 12 Ιδού, κατά τούτο δεν είσαι δίκαιος· θέλω αποκριθή προς σε, διότι ο Θεός είναι μεγαλήτερος του ανθρώπου.
\par 13 Διά τι αντιμάχεσαι προς αυτόν; διότι δεν δίδει λόγον περί ουδεμιάς των πράξεων αυτού.
\par 14 Διότι ο Θεός λαλεί άπαξ και δις, αλλ' ο άνθρωπος δεν προσέχει.
\par 15 Εν ενυπνίω, εν οράσει νυκτερινή, ότε βαθύς ύπνος πίπτει επί τους ανθρώπους, ότε υπνώττουσιν επί της κλίνης·
\par 16 τότε ανοίγει τα ώτα των ανθρώπων, και επισφραγίζει την προς αυτούς νουθεσίαν·
\par 17 διά να αποστρέψη τον άνθρωπον από των πράξεων αυτού και να εκβάλη την υπερηφανίαν εκ του ανθρώπου.
\par 18 Προλαμβάνει την ψυχήν αυτού από του λάκκου και την ζωήν αυτού από του να διαπερασθή υπό ρομφαίας.
\par 19 Πάλιν, τιμωρείται με πόνους επί της κλίνης αυτού, και το πλήθος των οστέων αυτού με δυνατούς πόνους·
\par 20 ώστε η ζωή αυτού αποστρέφεται τον άρτον και η ψυχή αυτού το επιθυμητόν φαγητόν·
\par 21 η σαρξ αυτού αναλίσκεται, ώστε δεν φαίνεται, και τα οστά αυτού τα αφανή εξέχουσιν·
\par 22 η δε ψυχή αυτού πλησίαζει εις τον λάκκον και η ζωή αυτού εις τους φονευτάς.
\par 23 Εάν ήναι μηνυτής μετ' αυτού ή ερμηνευτής, εις μεταξύ χιλίων, διά να αναγγείλη προς τον άνθρωπον την ευθύτητα αυτού·
\par 24 τότε θέλει είσθαι ίλεως εις αυτόν και θέλει ειπεί, Λύτρωσον αυτόν από του να καταβή εις τον λάκκον· εγώ εύρηκα εξιλασμόν.
\par 25 Η σαρξ αυτού θέλει είσθαι ανθηροτέρα νηπίου· θέλει επιστρέψει εις τας ημέρας της νεότητος αυτού·
\par 26 θέλει δεηθή του Θεού και θέλει ευνοήσει προς αυτόν· και θέλει βλέπει το πρόσωπον αυτού εν χαρά· και θέλει αποδώσει εις τον άνθρωπον την δικαιοσύνην αυτού.
\par 27 Θέλει βλέπει προς τους ανθρώπους και θέλει λέγει, Ημάρτησα και διέστρεψα το ορθόν, και δεν με ωφέλησεν·
\par 28 αλλ' αυτός ελύτρωσε την ψυχήν μου από του να υπάγη εις τον λάκκον· και η ζωή μου θέλει ιδεί το φως.
\par 29 Ιδού, πάντα ταύτα εργάζεται ο Θεός δις και τρίς μετά του ανθρώπου,
\par 30 διά να αποστρέψη την ψυχήν αυτού από του λάκκου, ώστε να φωτισθή εν τω φωτί των ζώντων.
\par 31 Πρόσεχε, Ιώβ, άκουσόν μου· σιώπα, και εγώ θέλω λαλήσει.
\par 32 Εάν έχης τι να είπης, αποκρίθητί μοι· λάλησον, διότι επιθυμώ να δικαιωθής.
\par 33 Ει δε μη, συ άκουσόν μου· σιώπα και θέλω σε διδάξει σοφίαν.

\chapter{34}

\par Επανελάβε δε ο Ελιού και είπεν·
\par 2 Ακούσατε τους λόγους μου, ω σοφοί· και δότε ακρόασιν εις εμέ, οι νοήμονες·
\par 3 Διότι το ωτίον δοκιμάζει τους λόγους, ο δε ουρανίσκος γεύεται το φαγητόν.
\par 4 Ας εκλέξωμεν εις εαυτούς κρίσιν· ας γνωρίσωμεν μεταξύ ημών τι το καλόν.
\par 5 Διότι ο Ιώβ είπεν, Είμαι δίκαιος· και ο Θεός αφήρεσε την κρίσιν μου·
\par 6 εψεύσθην εις την κρίσιν μου· η πληγή μου είναι ανίατος, άνευ παραβάσεως.
\par 7 Τις άνθρωπος ως ο Ιώβ, όστις καταπίνει τον χλευασμόν ως ύδωρ·
\par 8 και υπάγει εν συνοδία μετά των εργατών της ανομίας, και περιπατεί μετά ανθρώπων ασεβών;
\par 9 Διότι είπεν, ουδέν ωφελεί τον άνθρωπον το να ευαρεστή εις τον Θεόν.
\par 10 Διά τούτο ακούσατέ μου, άνδρες συνετοί· μη γένοιτο να υπάρχη εις τον Θεόν αδικία, και εις τον Παντοδύναμον ανομία.
\par 11 Επειδή κατά το έργον του ανθρώπου θέλει αποδώσει εις αυτόν, και θέλει κάμει έκαστον να εύρη κατά την οδόν αυτού.
\par 12 Ναι, βεβαίως ο Θεός δεν θέλει πράξει ασεβώς, ουδέ θέλει διαστρέψει ο Παντοδύναμος την κρίσιν.
\par 13 Τις κατέστησεν αυτόν επιτηρητήν της γης; ή τις διέταξε πάσαν την οικουμένην;
\par 14 Εάν βάλη την καρδίαν αυτού επί τον άνθρωπον, θέλει σύρει εις εαυτόν το πνεύμα αυτού και την πνοήν αυτού·
\par 15 πάσα σαρξ θέλει εκπνεύσει ομού, και ο άνθρωπος θέλει επιστρέψει εις το χώμα.
\par 16 Εάν τώρα έχης σύνεσιν· άκουσον τούτο· ακροάθητι της φωνής των λόγων μου.
\par 17 Μήπως κυβερνά ο μισών την ευθύτητα; και θέλεις καταδικάσει τον κατ' εξοχήν δίκαιον;
\par 18 όστις λέγει προς βασιλέα, Είσαι ασεβής, προς άρχοντας, Είσθε κακοί;
\par 19 Όστις δεν προσωποληπτεί εις άρχοντας ουδέ αποβλέπει εις τον πλούσιον μάλλον παρά εις τον πτωχόν; επειδή πάντες ούτοι είναι έργον των χειρών αυτού.
\par 20 Εν μιά στιγμή θέλουσιν αποθάνει, και το μεσονύκτιον ο λαός θέλει ταραχθή και θέλει παρέλθει· και ο ισχυρός θέλει αναρπαχθή, ουχί υπό χειρός.
\par 21 Διότι οι οφθαλμοί αυτού είναι επί τας οδούς του ανθρώπου, Και βλέπει πάντα τα βήματα αυτού.
\par 22 Δεν είναι σκότος ουδέ σκιά θανάτου, όπου οι εργάται της ανομίας να κρυφθώσιν.
\par 23 Επειδή δεν θέλει αφήσει πλέον τον άνθρωπον να έλθη εις κρίσιν μετά του Θεού.
\par 24 Θέλει συντρίψει αναριθμήτους ισχυρούς και βάλει άλλους αντ' αυτών
\par 25 διότι γνωρίζει τα έργα αυτών, και ανατρέπει αυτούς την νύκτα, και συντρίβονται.
\par 26 Κτυπά αυτούς ως ασεβείς εν τω τόπω των θεατών·
\par 27 επειδή εξέκλιναν απ' αυτού και δεν εθεώρησαν ουδεμίαν των οδών αυτού·
\par 28 και έκαμον να έλθη προς αυτόν η κραυγή των πτωχών, και ήκουσε την φωνήν των τεθλιμμένων.
\par 29 Και όταν αυτός δίδη ησυχίαν, τις θέλει διαταράξει αυτήν; και όταν κρύπτη το πρόσωπον αυτού, τις δύναται να ίδη αυτόν; είτε επί έθνος είτε επί άνθρωπον ομού·
\par 30 ώστε να μη βασιλεύη υποκριτής, διά να μη παγιδεύηται ο λαός.
\par 31 Βεβαίως πρέπει να λέγη τις προς τον Θεόν, Έπαθον, δεν θέλω πλέον πράξει κακώς·
\par 32 ό,τι δεν βλέπω, συ δίδαξόν με· εάν έπραξα ανομίαν, δεν θέλω πράξει πλέον.
\par 33 Αλλά μήπως θέλει γείνει κατά τον στοχασμόν σου; είτε συ αποβάλης είτε εκλέξης, αυτός θέλει ανταποδώσει, και ουχί εγώ· λέγε λοιπόν ό,τι εξεύρεις.
\par 34 Άνδρες συνετοί θέλουσιν ειπεί προς εμέ, και ο σοφός άνθρωπος όστις με ακούει,
\par 35 Ο Ιώβ δεν ελάλησεν εν γνώσει, και οι λόγοι αυτού δεν ήσαν μετά συνέσεως.
\par 36 Η επιθυμία μου είναι, ο Ιώβ να εξετασθή έως τέλους· επειδή απεκρίθη ως οι άνθρωποι οι ασεβείς.
\par 37 Διότι εις την αμαρτίαν αυτού προσθέτει ασέβειαν· καυχάται μεταξύ ημών, και πολλαπλασιάζει τους λόγους αυτού εναντίον του Θεού.

\chapter{35}

\par Και επανέλαβεν ο Ελιού και είπε·
\par 2 Στοχάζεσαι ότι είναι ορθόν τούτο, το οποίον είπας, Είμαι δικαιότερος του Θεού;
\par 3 Διότι είπας, Τις ωφέλεια θέλει είσθαι εις σε; Τι κέρδος θέλω λάβει εκ τούτου μάλλον παρά εκ της αμαρτίας μου;
\par 4 Εγώ θέλω αποκριθή προς σε και προς τους φίλους σου μετά σου.
\par 5 Ανάβλεψον εις τους ουρανούς και ιδέ· και θεώρησον τα νέφη, πόσον υψηλότερά σου είναι.
\par 6 Εάν αμαρτάνης, τι πράττεις κατ' αυτού; ή αν αι παραβάσεις σου πολλαπλασιασθώσι, τι κατορθόνεις κατ' αυτού;
\par 7 Εάν ήσαι δίκαιος, τι θέλεις δώσει εις αυτόν; ή τι θέλει λάβει εκ της χειρός σου;
\par 8 Η ασέβειά σου δύναται να βλάψη άνθρωπον ως σέ· και η δικαιοσύνη σου δύναται να ωφελήση υιόν ανθρώπου.
\par 9 Εκ του πλήθους των καταθλιβόντων καταβοώσι· κραυγάζουσιν ένεκεν του βραχίονος των ισχυρών·
\par 10 Αλλ' ουδείς λέγει, που είναι ο Θεός ο Ποιητής μου, όστις δίδει άσματα εις την νύκτα,
\par 11 Όστις συνετίζει ημάς υπέρ τα κτήνη της γης, και σοφίζει ημάς υπέρ τα πετεινά του ουρανού;
\par 12 Εκεί βοώσι διά την υπερηφανίαν των πονηρών, δεν θέλει όμως αποκριθή.
\par 13 Ο Θεός βεβαίως δεν θέλει εισακούσει της ματαιολογίας, ουδέ θέλει επιβλέψει ο Παντοδύναμος εις αυτήν·
\par 14 πόσον ολιγώτερον όταν συ λέγης, ότι δεν θέλεις ιδεί αυτόν· η κρίσις όμως είναι ενώπιον αυτού· όθεν έχε το θάρρος σου επ' αυτόν.
\par 15 Αλλά τώρα, επειδή δεν επεσκέφθη εν τω θυμώ αυτού και δεν παρετήρησε μετά μεγάλης αυστηρότητος,
\par 16 διά τούτο ο Ιώβ ανοίγει το στόμα αυτού ματαίως· επισωρεύει λόγους εν αγνωσία.

\chapter{36}

\par Και ο Ελιού εξηκολούθησε και είπεν·
\par 2 Υπόμεινόν με ολίγον, και θέλω σε διδάξει· διότι έχω έτι λόγους υπέρ του Θεού.
\par 3 Θέλω λάβει τα επιχειρήματά μου μακρόθεν, και θέλω αποδώσει δικαιοσύνην εις τον Ποιητήν μου·
\par 4 διότι οι λόγοι μου επ' αληθείας δεν θέλουσιν είσθαι ψευδείς· πλησίον σου είναι ο τέλειος κατά την γνώσιν.
\par 5 Ιδού, ο Θεός είναι ισχυρός, όμως δεν καταφρονεί ουδένα· ισχυρός εις δύναμιν σοφίας.
\par 6 Δεν θέλει ζωοποιήσει τον ασεβή· εις δε τους πτωχούς δίδει το δίκαιον.
\par 7 Δεν αποσύρει τους οφθαλμούς αυτού από των δικαίων, αλλά και μετά βασιλέων βάλλει αυτούς επί θρόνου· μάλιστα καθίζει αυτούς διαπαντός, και είναι υψωμένοι.
\par 8 Και εάν ήθελον είσθαι δεδεμένοι με δεσμά και πιασθή με σχοινία θλίψεως,
\par 9 τότε φανερόνει εις αυτούς τα έργα αυτών και τας παραβάσεις αυτών, ότι υπερηύξησαν,
\par 10 και ανοίγει το ωτίον αυτών εις διδασκαλίαν, και από της ανομίας προστάζει να επιστρέψωσιν.
\par 11 Εάν υπακούσωσι και δουλεύσωσι, θέλουσι τελειώσει τας ημέρας αυτών εν αγαθοίς και τα έτη αυτών εν ευφροσύναις.
\par 12 Αλλ' εάν δεν υπακούσωσι, θέλουσι διαπερασθή υπό ρομφαίας και θέλουσι τελευτήσει εν αγνωσία.
\par 13 Οι δε υποκριταί την καρδίαν επισωρεύουσιν οργήν· δεν θέλουσι βοήσει όταν δέση αυτούς·
\par 14 αυτοί αποθνήσκουσιν εν τη νεότητι, και η ζωή αυτών τελειόνει μεταξύ των ασελγών.
\par 15 Λυτρόνει τον τεθλιμμένον εν τη θλίψει αυτού και ανοίγει τα ώτα αυτών εν συμφορά·
\par 16 και ούτως ήθελε σε εκβάλει από της στενοχωρίας εις ευρυχωρίαν, όπου δεν υπάρχει στενοχωρία· και το παρατιθέμενον επί της τραπέζης σου θέλει είσθαι πλήρες πάχους.
\par 17 Αλλά συ εξεπλήρωσας δίκην ασεβούς· δίκη και κρίσις θέλουσι σε καταλάβει.
\par 18 Επειδή υπάρχει θυμός, πρόσεχε μη σε εξαφανίση διά της προσβολής αυτού· τότε ουδέ μέγα λύτρον ήθελε σε λυτρώσει.
\par 19 Θέλει αποβλέψει εις τα πλούτη σου, ούτε εις χρυσίον ούτε εις πάσαν την ισχύν της δυνάμεως;
\par 20 Μη επιπόθει την νύκτα, καθ' ην οι λαοί εκκόπτονται εν τω τόπω αυτών.
\par 21 Πρόσεχε, μη στραφής προς την ανομίαν· διότι συ προέκρινας τούτο μάλλον παρά την θλίψιν.
\par 22 Ιδού, ο Θεός είναι υψωμένος διά της δυνάμεως αυτού· τις διδάσκει ως αυτός;
\par 23 Τις διώρισεν εις αυτόν την οδόν αυτού; ή τις δύναται να είπη, Έπραξας ανομίαν;
\par 24 Ενθυμού να μεγαλύνης το έργον αυτού, το οποίον θεωρούσιν οι άνθρωποι.
\par 25 Πας άνθρωπος βλέπει αυτό· ο άνθρωπος θεωρεί αυτό μακρόθεν.
\par 26 Ιδού, ο Θεός είναι μέγας και ακατανόητος εις ημάς, και ο αριθμός των ετών αυτού ανεξερεύνητος.
\par 27 Όταν ανασύρη τας ρανίδας του ύδατος, αυταί καταχέουσιν εκ των ατμών αυτού βροχήν,
\par 28 την οποίαν τα νέφη ραίνουσιν· αφθόνως σταλάζουσιν επί τον άνθρωπον.
\par 29 Δύναταί τις έτι να εννοήση τας εφαπλώσεις των νεφελών, τον κρότον της σκηνής αυτού;
\par 30 Ιδού, εφαπλόνει το φως αυτού επ' αυτήν και σκεπάζει τους πυθμένας της θαλάσσης·
\par 31 επειδή δι' αυτών δικάζει τους λαούς και δίδει τροφήν αφθόνως.
\par 32 Εν ταις παλάμαις αυτού κρύπτει την αστραπήν· και προστάζει αυτήν εις ό,τι έχει να απαντήση.
\par 33 Παραγγέλλει εις αυτήν υπέρ του φίλου αυτού, κατά δε του ασεβούς ετοιμάζει οργήν.

\chapter{37}

\par Εις τούτο έτι η καρδία μου τρέμει και εκπηδά από του τόπου αυτής.
\par 2 Ακούσατε προσεκτικώς την τρομεράν φωνήν αυτού και τον ήχον τον εξερχόμενον εκ του στόματος αυτού.
\par 3 Αποστέλλει αυτήν υποκάτω παντός του ουρανού και το φως αυτού επί τα έσχατα της γης.
\par 4 Κατόπιν αυτού βοά φωνή· βροντά με την φωνήν της μεγαλωσύνης αυτού· και δεν θέλει στήσει αυτά, αφού η φωνή αυτού ακουσθή.
\par 5 Ο Θεός βροντά θαυμασίως με την φωνήν αυτού· κάμνει μεγαλεία, και δεν εννοούμεν.
\par 6 Διότι λέγει προς την χιόνα, γίνου επί την γήν· και προς την ψεκάδα και προς τον υετόν της δυνάμεως αυτού.
\par 7 Κατασφραγίζει την χείρα παντός ανθρώπου· διά να γνωρίσωσι πάντες οι άνθρωποι το έργον αυτού.
\par 8 Τότε τα θηρία εισέρχονται εις τα σπήλαια και κατασκηνούσιν εν τοις τόποις αυτών.
\par 9 Εκ του νότου έρχεται ο ανεμοστρόβιλος, και το ψύχος εκ του βορρά.
\par 10 Εκ του φυσήματος του Θεού δίδεται πάγος· και το πλάτος των υδάτων στερεούται.
\par 11 Πάλιν η γαλήνη διασκεδάζει την νεφέλην· το φως αυτού διασκορπίζει τα νέφη·
\par 12 και αυτά περιφέρονται κύκλω υπό τας οδηγίας αυτού, διά να κάμνωσι παν ό,τι προστάζει εις αυτά επί το πρόσωπον της οικουμένης·
\par 13 κάμνει αυτά να έρχωνται, ή διά παιδείαν, ή διά την γην αυτού, ή διά έλεος.
\par 14 Ακροάσθητι τούτο, Ιώβ· στάθητι και συλλογίσθητι τα θαυμάσια του Θεού.
\par 15 Εννοείς πως ο Θεός διατάττει αυτά, και κάμνει να λάμπη το φως της νεφέλης αυτού;
\par 16 Εννοείς τα ζυγοσταθμίσματα των νεφών, τα θαυμάσια του τελείου κατά την γνώσιν;
\par 17 Διά τι τα ενδύματά σου είναι θερμά, όταν αναπαύη την γην διά του νότου;
\par 18 Εξήπλωσας μετ' αυτού το στερέωμα το δυνατόν ως κάτοπτρον χυτόν;
\par 19 Δίδαξον ημάς τι να είπωμεν προς αυτόν· ημείς δεν δυνάμεθα να διατάξωμεν τους λόγους ημών εξ αιτίας του σκότους.
\par 20 Θέλει αναγγελθή προς αυτόν, εάν εγώ λαλώ; εάν λαλήση άνθρωπος, βεβαίως θέλει καταποθή.
\par 21 Τώρα δε οι άνθρωποι δεν δύνανται να ατενίσωσιν εις το λαμπρόν φως, το εν τω στερεώματι, αφού ο άνεμος περάση και καθαρίση αυτό,
\par 22 και χρυσαυγής καιρός έλθη από βορρά. Φοβερά δόξα υπάρχει εν τω Θεώ.
\par 23 Τον Παντοδύναμον, δεν δυνάμεθα να εννοήσωμεν αυτόν· είναι υπέροχος κατά την δύναμιν και κατά την κρίσιν και κατά το πλήθος της δικαιοσύνης, δεν καταθλίβει.
\par 24 Διά τούτο οι άνθρωποι φοβούνται αυτόν· ουδείς σοφός την καρδίαν δύναται να εννοήση αυτόν.

\chapter{38}

\par Τότε απεκρίθη ο Κύριος προς τον Ιώβ εκ του ανεμοστροβίλου και είπε·
\par 2 Τις ούτος, όστις σκοτίζει την βουλήν μου διά λόγων ασυνέτων;
\par 3 Ζώσον ήδη την οσφύν σου ως ανήρ· διότι θέλω σε ερωτήσει, και φανέρωσόν μοι.
\par 4 Που ήσο ότε εθεμελίονον την γην; απάγγειλον, εάν έχης σύνεσιν.
\par 5 Τις έθεσε τα μέτρα αυτής, εάν εξεύρης; ή τις ήπλωσε στάθμην επ' αυτήν;
\par 6 Επί τίνος είναι εστηριγμένα τα θεμέλια αυτής; ή τις έθεσε τον ακρογωνιαίον λίθον αυτής,
\par 7 ότε τα άστρα της αυγής έψαλλον ομού και πάντες οι υιοί του Θεού ηλάλαζον;
\par 8 ή τις συνέκλεισε την θάλασσαν με θύρας, ότε εξορμώσα εξήλθεν εκ μήτρας;
\par 9 ότε περιέβαλον αυτήν με νεφέλην και με ομίχλην εσπαργάνωσα αυτήν,
\par 10 και περιώρισα αυτήν διά προστάγματός μου, και έβαλον μοχλούς και πύλας,
\par 11 και είπα, Έως αυτού θέλεις έρχεσθαι και δεν θέλεις υπερβή· και εδώ θέλει συντρίβεσθαι η υπερηφανία των κυμάτων σου;
\par 12 Προσέταξας συ την πρωΐαν επί των ημερών σου; έδειξας εις την αυγήν τον τόπον αυτής,
\par 13 διά να πιάση τα έσχατα της γης, ώστε οι κακούργοι να εκτιναχθώσιν απ' αυτής;
\par 14 Αυτή μεταμορφούται ως πηλός σφραγιζόμενος· και τα πάντα παρουσιάζονται ως στολή.
\par 15 Και το φως των ασεβών αφαιρείται απ' αυτών, ο δε βραχίων των υπερηφάνων συντρίβεται.
\par 16 Εισήλθες έως των πηγών της θαλάσσης; ή περιεπάτησας εις εξιχνίασιν της αβύσσου;
\par 17 Ηνοίχθησαν εις σε του θανάτου αι πύλαι; ή είδες τας θύρας της σκιάς του θανάτου;
\par 18 Εγνώρισας το πλάτος της γης; απάγγειλον, εάν ενόησας πάντα ταύτα.
\par 19 Που είναι η οδός της κατοικίας του φωτός; και του σκότους, που είναι ο τόπος αυτού,
\par 20 διά να συλλάβης αυτό εις το όριον αυτού και να γνωρίσης τας τρίβους της οικίας αυτού;
\par 21 Γνωρίζεις αυτό, διότι τότε εγεννήθης; ή διότι ο αριθμός των ημερών σου είναι πολύς;
\par 22 Εισήλθες εις τους θησαυρούς της χιόνος; ή είδες τους θησαυρούς της χαλάζης,
\par 23 τους οποίους φυλάττω διά τον καιρόν της θλίψεως διά την ημέραν της μάχης και του πολέμου;
\par 24 Διά τίνος οδού διαδίδεται το φως, ή ο ανατολικός άνεμος διαχέεται επί την γην;
\par 25 Τις ήνοιξε ρύακας διά τας ραγδαίας βροχάς, ή δρόμον διά την αστραπήν της βροντής,
\par 26 διά να φέρη βροχήν επί γην ακατοίκητον, εις έρημον, όπου άνθρωπος δεν υπάρχει,
\par 27 διά να χορτάση την άβατον και ακατοίκητον, και να αναβλαστήση τον βλαστόν της χλόης;
\par 28 Έχει πατέρα η βροχή; ή τις εγέννησε τας σταγόνας της δρόσου;
\par 29 Από μήτρας τίνος εξέρχεται ο πάγος; και την πάχνην του ουρανού, τις εγέννησε;
\par 30 Τα ύδατα σκληρύνονται ως λίθος, και το πρόσωπον της αβύσσου πηγνύεται.
\par 31 Δύνασαι να δεσμεύσης τας γλυκείας επιρροάς της Πλειάδος ή να λύσης τα δεσμά τον Ωρίωνος;
\par 32 Δύνασαι να εκβάλης τα Ζώδια εις τον καιρόν αυτών; ή δύνασαι να οδηγήσης τον Αρκτούρον μετά των υιών αυτού;
\par 33 Γνωρίζεις τους νόμους του ουρανού; δύνασαι να διατάξης τας επιρροάς αυτού επί την γην;
\par 34 Δύνασαι να υψώσης την φωνήν σου εις τα νέφη, διά να σε σκεπάση αφθονία υδάτων;
\par 35 Δύνασαι να αποστείλης αστραπάς, ώστε να εξέλθωσι και να είπωσι προς σε, Ιδού, ημείς;
\par 36 Τις έβαλε σοφίαν εντός του ανθρώπου; ή τις έδωκε σύνεσιν εις την καρδίαν αυτού;
\par 37 Τις δύναται να αριθμήση τα νέφη διά σοφίας; ή τις δύναται να κενόνη τα δοχεία του ουρανού,
\par 38 διά να χωνευθή το χώμα εις σύμπηξιν και οι βώλοι να συγκολλώνται;
\par 39 Θέλεις κυνηγήσει θήραμα διά τον λέοντα; ή χορτάσει την όρεξιν των σκύμνων,
\par 40 όταν κοίτωνται εν τοις σπηλαίοις και κάθηνται εις τους κρυπτήρας διά να ενεδρεύωσι;
\par 41 Τις ετοιμάζει εις τον κόρακα την τροφήν αυτού, όταν οι νεοσσοί αυτού κράζωσι προς τον Θεόν, περιπλανώμενοι δι' έλλειψιν τροφής;

\chapter{39}

\par Γνωρίζεις τον καιρόν του τοκετού των αγρίων αιγών του βράχου; δύνασαι να σημειώσης πότε γεννώσιν αι έλαφοι;
\par 2 Δύνασαι να αριθμήσης τους μήνας τους οποίους πληρούσιν; ή γνωρίζεις τον καιρόν του τοκετού αυτών;
\par 3 Αυταί συγκάμπτονται, γεννώσι τα παιδία αυτών, ελευθερόνονται από των ωδίνων αυτών.
\par 4 Τα τέκνα αυτών ενδυναμούνται, αυξάνουσιν εν τη πεδιάδι· εξέρχονται και δεν επιστρέφουσι πλέον εις αυτάς.
\par 5 Τις εξαπέστειλεν ελεύθερον τον άγριον όνον; ή τις έλυσε τους δεσμούς αυτού;
\par 6 του οποίου οικίαν έκαμον την έρημον, και την αλμυρίδα κατοικίαν αυτού.
\par 7 Καταγελά του θορύβου της πόλεως· δεν ακούει την κραυγήν του εργοδιώκτου.
\par 8 Κατασκοπεύει τα όρη διά βοσκήν αυτού, και υπάγει ζητών κατόπιν παντός είδους χλόης.
\par 9 Θέλει ευχαριστηθή ο μονόκερως να σε δουλεύη, ή θέλει διανυκτερεύσει εν τη φάτνη σου;
\par 10 Δύνασαι να δέσης τον μονόκερων με τον δεσμόν αυτού προς αροτρίασιν; ή θέλει ομαλίζει τας πεδιάδας οπίσω σου;
\par 11 Θέλεις βάλει το θάρρος σου εις αυτόν, διότι η δύναμις αυτού είναι μεγάλη; ή θέλεις αφήσει την εργασίαν σου επ' αυτόν;
\par 12 Θέλεις εμπιστευθή εις αυτόν να σοι φέρη τον σπόρον σου και να συνάξη αυτόν εν τω αλωνίω σου;
\par 13 Έδωκας συ τας ώραίας πτέρυγας εις τους ταώνας; ή πτέρυγας και πτερά εις την στρουθοκάμηλον;
\par 14 ήτις αφίνει τα ωά αυτής εις την γην και θάλπει αυτά επί του χώματος,
\par 15 και λησμονεί ότι ο πους ενδέχεται να συντρίψη αυτά, ή το θηρίον του αγρού να καταπατήση αυτά·
\par 16 σκληρύνεται κατά των τέκνων αυτής, ως να μη ήσαν αυτής· ματαίως εκοπίασε, μη φοβουμένη·
\par 17 διότι ο Θεός εστέρησεν αυτήν από σοφίας και δεν εμοίρασεν εις αυτήν σύνεσιν·
\par 18 οσάκις σηκόνεται όρθιος, καταγελά του ίππου και του αναβάτου αυτού.
\par 19 Συ έδωκας δύναμιν εις τον ίππον; περιενέδυσας τον τράχηλον αυτού με βροντήν;
\par 20 συ κάμνεις αυτόν να πηδά ως ακρίς; το γαυρίαμα των μυκτήρων αυτού είναι τρομερόν·
\par 21 ανασκάπτει εν τη κοιλάδι και αγάλλεται εις την δύναμιν αυτού· εξέρχεται εις απάντησιν των όπλων·
\par 22 καταγελά του φόβου και δεν τρομάζει· ουδέ στρέφει από προσώπου ρομφαίας·
\par 23 η φαρέτρα κροταλίζει κατ' αυτού, η εξαστράπτουσα λόγχη και το δόρυ.
\par 24 Καταπίνει την γην εν αγριότητι και μανία· και δεν πιστεύει ότι ηχεί σάλπιγξ·
\par 25 άμα δε τη φωνή της σάλπιγγος, λέγει, Α, α και μακρόθεν οσφραίνεται την μάχην, την κραυγήν των στρατηγών και τον αλαλαγμόν.
\par 26 Διά της σοφίας σου πετά ο ιέραξ και απλόνει τας πτέρυγας αυτού προς νότον;
\par 27 Εις την προσταγήν σου ανυψούται ο αετός και κάμνει την φωλεάν αυτού εν τοις υψηλοίς;
\par 28 Κατοικεί επί βράχου και διατρίβει, επί αποτόμου βράχου και επί αβάτων τόπων·
\par 29 εκείθεν αναζητεί τροφήν· οι οφθαλμοί αυτού σκοπεύουσι μακρόθεν·
\par 30 και οι νεοσσοί αυτού αίμα πίνουσι· και όπου πτώματα, εκεί και αυτός.

\chapter{40}

\par Ο Κύριος απεκρίθη έτι προς τον Ιώβ και είπεν·
\par 2 Ο διαδικαζόμενος προς τον Παντοδύναμον θέλει διδάξει αυτόν; ο ελέγχων τον Θεόν ας αποκριθή προς τούτο.
\par 3 Τότε ο Ιώβ απεκρίθη προς τον Κύριον και είπεν·
\par 4 Ιδού, εγώ είμαι ουτιδανός· τι δύναμαι να αποκριθώ προς σε; θέλω βάλει την χείρα μου επί το στόμα μου·
\par 5 άπαξ ελάλησα και δεν θέλω αποκριθή πλέον· μάλιστα, δίς· αλλά δεν θέλω επιπροσθέσει.
\par 6 Τότε απεκρίθη ο Κύριος προς τον Ιώβ εκ του ανεμοστροβίλου και είπε·
\par 7 Ζώσον ήδη ως ανήρ την οσφύν σου· εγώ θέλω σε ερωτήσει, και απάγγειλόν μοι.
\par 8 Θέλεις άρα αναιρέσει την κρίσιν μου; θέλεις με καταδικάσει, διά να δικαιωθής;
\par 9 Έχεις βραχίονα ως ο Θεός; ή δύνασαι να βροντάς με φωνήν ως αυτός;
\par 10 Στολίσθητι τώρα μεγαλοπρέπειαν και υπεροχήν· και ενδύθητι δόξαν και ώραιότητα.
\par 11 Έκχεε τας φλόγας της οργής σου· και βλέπε πάντα υπερήφανον και ταπείνονε αυτόν.
\par 12 Βλέπε πάντα υπερήφανον· κρήμνιζε αυτόν· και καταπάτει τους ασεβείς εν τω τόπω αυτών.
\par 13 Κρύψον αυτούς ομού εν τω χώματι· κάλυψον τα πρόσωπα αυτών εν αφανεία.
\par 14 Τότε και εγώ θέλω ομολογήσει προς σε, ότι η δεξιά σου δύναται να σε σώση.
\par 15 Ιδού τώρα, ο Βεεμώθ, τον οποίον έκαμα μετά σου, τρώγει χόρτον ως βους.
\par 16 Ιδού τώρα, η δύναμις αυτού είναι εν τοις νεφροίς αυτού και η ισχύς αυτού εν τω ομφαλώ της κοιλίας αυτού.
\par 17 Υψόνει την ουράν αυτού ως κέδρον· τα νεύρα των μηρών αυτού είναι συμπεπλεγμένα.
\par 18 Τα οστά αυτού είναι χάλκινοι σωλήνες· τα οστά αυτού ως μοχλοί σιδήρου.
\par 19 Τούτο είναι το αριστούργημα του Θεού· ο ποιήσας αυτόν δύναται να πλησιάση εις αυτόν την ρομφαίαν αυτού.
\par 20 Διότι τα όρη προμηθεύουσιν εις αυτόν την τροφήν, όπου παίζουσι πάντα τα θηρία του αγρού.
\par 21 Πλαγιάζει υποκάτω των σκιερών δένδρων, υπό την σκέπην των καλάμων και εν τοις βάλτοις.
\par 22 Τα σκιερά δένδρα σκεπάζουσιν αυτόν με την σκιάν αυτών· αι ιτέαι των ρυάκων περικαλύπτουσιν αυτόν.
\par 23 Ιδού, εάν πλημμυρίση ποταμός, δεν σπεύδει να φύγη· έχει θάρρος, και αν ο Ιορδάνης προσβάλλη εις το στόμα αυτού.
\par 24 Δύναταί τις φανερά να συλλάβη αυτόν; ή διά παγίδων να διατρυπήση την ρίνα αυτού;

\chapter{41}

\par Δύνασαι να σύρης έξω τον Λευϊάθαν διά αγκίστρου; ή να περιδέσης την γλώσσαν αυτού με φορβιάν;
\par 2 Δύνασαι να βάλης χαλινόν εις την ρίνα αυτού; ή να τρυπήσης την σιαγόνα αυτού με άκανθαν;
\par 3 Θέλει πληθύνει προς σε ικεσίας; θέλει σοι λαλήσει μετά γλυκύτητος;
\par 4 Θέλει κάμει συνθήκην μετά σου; θέλεις πάρει αυτόν διά δούλον παντοτεινόν;
\par 5 Θέλεις παίζει μετ' αυτού ως μετά πτηνού; ή θέλεις δέσει αυτόν διά τας θεραπαίνας σου;
\par 6 Θέλουσι κάμει οι φίλοι συμπόσιον εξ αυτού; θέλουσι μοιράσει αυτόν μεταξύ των εμπόρων;
\par 7 Δύνασαι να γεμίσης το δέρμα αυτού με βέλη; ή την κεφαλήν αυτού με αλιευτικά καμάκια;
\par 8 Βάλε την χείρα σου επ' αυτόν· ενθυμήθητι τον πόλεμον· μη κάμης πλέον τούτο.
\par 9 Ιδού, η ελπίς να πιάση τις αυτόν είναι ματαία· δεν ήθελε μάλιστα εκπλαγή εις την θεωρίαν αυτού;
\par 10 Ουδείς είναι τόσον τολμηρός ώστε να εγείρη αυτόν· και τις δύναται να σταθή έμπροσθεν εμού;
\par 11 Τις πρότερον έδωκεν εις εμέ και να ανταποδόσω; τα υποκάτω παντός του ουρανού είναι εμού.
\par 12 Δεν θέλω σιωπήσει τα μέλη αυτού ουδέ την δύναμιν ουδέ την ευάρεστον αυτού συμμετρίαν.
\par 13 Τις να εξιχνιάση την επιφάνειαν του ενδύματος αυτού; τις να εισέλθη εντός των διπλών σιαγόνων αυτού;
\par 14 Τις δύναται να ανοίξη τας πύλας του προσώπου αυτού; οι οδόντες αυτού κύκλω είναι τρομεροί.
\par 15 Αι ισχυραί ασπίδες αυτού είναι το εγκαύχημα αυτού, συγκεκλεισμέναι ομού διά σφιγκτού σφραγίσματος·
\par 16 η μία ενούται μετά της άλλης, ώστε ουδέ αήρ δύναται να περάση δι' αυτών·
\par 17 είναι προσκεκολλημέναι η μία μετά της άλλης· συνέχονται ούτως, ώστε δεν δύνανται να αποσπασθώσιν.
\par 18 Εις τον πταρνισμόν αυτού λάμπει φως, και οι οφθαλμοί αυτού είναι ως τα βλέφαρα της αυγής.
\par 19 Εκ του στόματος αυτού εξέρχονται λαμπάδες καιόμεναι και σπινθήρες πυρός εξακοντίζονται.
\par 20 Εκ των μυκτήρων αυτού εξέρχεται καπνός, ως εξ αγγείου κοχλάζοντος ή λέβητος.
\par 21 Η πνοή αυτού ανάπτει άνθρακας, και φλόξ εξέρχεται εκ του στόματος αυτού·
\par 22 Εν τω τραχήλω αυτού κατοικεί δύναμις, και τρόμος προπορεύεται έμπροσθεν αυτού.
\par 23 Τα στρώματα της σαρκός αυτού είναι συγκεκολλημένα· είναι στερεά επ' αυτόν· δεν δύνανται να σαλευθώσιν.
\par 24 Η καρδία αυτού είναι στερεά ως λίθος· σκληρά μάλιστα ως η κάτω μυλόπετρα.
\par 25 Ότε ανεγείρεται, φρίττουσιν οι δυνατοί, και εκ του φόβου παραφρονούσιν.
\par 26 Η ρομφαία του συναπαντώντος αυτόν δεν δύναται να ανθέξη· η λόγχη, το δόρυ, ουδέ ο θώραξ.
\par 27 Θεωρεί τον σίδηρον ως άχυρον, τον χαλκόν ως ξύλον σαθρόν.
\par 28 Τα βέλη δεν δύνανται να τρέψωσιν αυτόν εις φυγήν· αι πέτραι της σφενδόνης είναι εις αυτόν ως στυπίον.
\par 29 Τα ακόντια λογίζονται ως στυπίον· γελά εις το σείσμα της λόγχης.
\par 30 Οξείς λίθοι κοίτονται υποκάτω αυτού· υποστρόνει τα αγκυλωτά σώματα επί πηλού.
\par 31 Κάμνει την άβυσσον ως λέβητα να κοχλάζη· καθιστά την θάλασσαν ως σκεύος μυρεψού.
\par 32 Αφίνει οπίσω την πορείαν φωτεινήν· ήθελέ τις υπολάβει την άβυσσον ως πολιάν.
\par 33 Επί της γης δεν υπάρχει όμοιον αυτού, δεδημιουργημένον ούτως άφοβον.
\par 34 Περιορά πάντα τα υψηλά· είναι βασιλεύς επί πάντας τους υιούς της υπερηφανίας.

\chapter{42}

\par Τότε απεκρίθη ο Ιώβ προς τον Κύριον και είπεν·
\par 2 Εξεύρω ότι δύνασαι τα πάντα, και ουδείς στοχασμός σου δύναται να εμποδισθή.
\par 3 Τις ούτος ο κρύπτων την βουλήν ασυνέτως; Εγώ λοιπόν προέφερα εκείνο, το οποίον δεν ενόουν. Πράγματα υπερθαύμαστα δι' εμέ, τα οποία δεν εγνώριζον.
\par 4 Άκουσον, δέομαι· και εγώ θέλω λαλήσει· θέλω σε ερωτήσει, και συ δίδαξόν με.
\par 5 Ήκουον περί σου με την ακοήν του ωτίου, αλλά τώρα ο οφθαλμός μου σε βλέπει·
\par 6 διά τούτο βδελύττομαι εμαυτόν, και μετανοώ εν χώματι και σποδώ.
\par 7 Αφού δε ο Κύριος ελάλησε τους λόγους τούτους προς τον Ιώβ, είπεν ο Κύριος προς Ελιφάς τον Θαιμανίτην, Ο θυμός μου εξήφθη κατά σου και κατά των δύο φίλων σου· διότι δεν ελαλήσατε περί εμού το ορθόν ως ο δούλός μου Ιώβ·
\par 8 διά τούτο λάβετε τώρα εις εαυτούς επτά μόσχους και επτά κριούς και υπάγετε προς τον δούλον μου Ιώβ, και προσφέρετε ολοκαύτωμα υπέρ εαυτών· ο δε Ιώβ ο δούλός μου θέλει ικετεύσει υπέρ υμών· διότι θέλω δεχθή το πρόσωπον αυτού· διά να μη πράξω με σας κατά την αφροσύνην σας· διότι δεν ελαλήσατε περί εμού το ορθόν ως ο δούλός μου Ιώβ.
\par 9 Και υπήγον Ελιφάς ο Θαιμανίτης και Βιλδάδ ο Σαυχίτης και Σωφάρ ο Νααμαθίτης, και έκαμον ως προσέταξεν εις αυτούς ο Κύριος· ο δε Κύριος εδέχθη το πρόσωπον του Ιώβ.
\par 10 Και έστρεψεν ο Κύριος την αιχμαλωσίαν του Ιώβ, αφού προσηυχήθη υπέρ των φίλων αυτού· και έδωκεν ο Κύριος εις τον Ιώβ διπλάσια πάντων των όσα είχε πρότερον.
\par 11 Τότε ήλθον προς αυτόν πάντες οι αδελφοί αυτού και πάσαι αι αδελφαί αυτού και πάντες οι γνωρίζοντες αυτόν πρότερον, και έφαγον άρτον μετ' αυτού εν τω οίκω αυτού· και συνέκλαυσαν με αυτόν και παρηγόρησαν αυτόν περί παντός του κακού, το οποίον ο Κύριος επέφερεν επ' αυτόν· και έδωκαν έκαστος εις αυτόν εν αργύριον και έκαστος εν χρυσούν ενώτιον.
\par 12 Και ευλόγησεν ο Κύριος τα έσχατα του Ιώβ μάλλον παρά τα πρώτα· ώστε απέκτησε δεκατέσσαρας χιλιάδας προβάτων και εξακισχιλίας καμήλους και χίλια ζεύγη βοών και χιλίας όνους.
\par 13 Εγεννήθησαν έτι εις αυτόν επτά υιοί και τρεις θυγατέρες·
\par 14 και εκάλεσε το όνομα της πρώτης Ιεμιμά· και το όνομα της δευτέρας Κεσιά· και το όνομα της τρίτης Κερέν-αππούχ·
\par 15 και δεν ευρίσκοντο εφ' όλης της γης γυναίκες ώραίαι ως αι θυγατέρες του Ιώβ· και ο πατήρ αυτών έδωκεν εις αυτάς κληρονομίαν μεταξύ των αδελφών αυτών.
\par 16 Μετά ταύτα έζησεν ο Ιώβ εκατόν τεσσαράκοντα έτη, και είδε τους υιούς αυτού και τους υιούς των υιών αυτού, τετάρτην γενεάν.
\par 17 και ετελεύτησεν ο Ιώβ, γέρων και πλήρης ημερών.


\end{document}