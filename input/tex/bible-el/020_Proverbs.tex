\begin{document}

\title{Proverbs}


\chapter{1}

\par 1 Παροιμίαι Σολομώντος, υιού του Δαβίδ, βασιλέως του Ισραήλ,
\par 2 διά να γνωρίση τις σοφίαν και παιδείαν, διά να νοήση λόγους φρονήσεως,
\par 3 διά να λάβη διδασκαλίαν συνέσεως, δικαιοσύνης και κρίσεως και ευθύτητος,
\par 4 διά να δώση νόησιν εις τους απλούς, και εις τον νέον μάθησιν και διάγνωσιν.
\par 5 Ο σοφός ακούων θέλει γείνει σοφώτερος, και ο νοήμων θέλει αποκτήσει επιστήμην κυβερνήσεως·
\par 6 ώστε να εννοή παροιμίαν και σκοτεινόν λόγον, ρήσεις σοφών και αινίγματα αυτών.
\par 7 Αρχή σοφίας φόβος Κυρίου· οι άφρονες καταφρονούσι την σοφίαν και την διδασκαλίαν.
\par 8 Άκουε, υιέ μου, την διδασκαλίαν του πατρός σου, και μη απορρίψης τον νόμον της μητρός σου.
\par 9 Διότι ταύτα θέλουσιν είσθαι στέφανος χαρίτων εις την κορυφήν σου και περιδέραιον περί τον τράχηλόν σου.
\par 10 Υιέ μου, εάν θελήσωσιν οι αμαρτωλοί να σε δελεάσωσι, μη θελήσης·
\par 11 εάν είπωσιν, Ελθέ μεθ' ημών, ας ενεδρεύσωμεν δι' αίμα, ας επιβουλευθώμεν αναιτίως τον αθώον,
\par 12 Ας καταπίωμεν αυτούς ζώντας, ως ο άδης, και ολοκλήρους ως τους καταβαίνοντας εις τον λάκκον·
\par 13 θέλομεν ευρεί παν πολύτιμον αγαθόν, θέλομεν γεμίσει τους οίκους ημών από λαφύρων·
\par 14 θες τον κλήρόν σου μεταξύ ημών, εν βαλάντιον ας ήναι εις πάντας ημάς·
\par 15 υιέ μου, μη περιπατήσης εν οδώ μετ' αυτών· άπεχε τον πόδα σου από των τρίβων αυτών·
\par 16 διότι οι πόδες αυτών τρέχουσιν εις το κακόν, και σπεύδουσιν εις το να χύσωσιν αίμα.
\par 17 Διότι ματαίως εξαπλόνεται δίκτυον έμπροσθεν των οφθαλμών παντός πτερωτού.
\par 18 Διότι ούτοι ενεδρεύουσι κατά του ιδίου αυτών αίματος, επιβουλεύονται τας εαυτών ψυχάς·
\par 19 Τοιαύται είναι αι οδοί παντός πλεονέκτου· η πλεονεξία αφαιρεί την ζωήν των κυριευομένων υπ' αυτής.
\par 20 Η σοφία φωνάζει έξω, εκπέμπει την φωνήν αυτής εν ταις πλατείαις·
\par 21 Κράζει επί κεφαλής των αγορών, εν ταις εισόδοις των πυλών· απαγγέλλει τους λόγους αυτής διά της πόλεως, λέγουσα,
\par 22 Έως πότε, μωροί, θέλετε αγαπά την μωρίαν, και οι χλευασταί θέλουσιν ηδύνεσθαι εις τους χλευασμούς αυτών, και οι άφρονες θέλουσι μισεί την γνώσιν;
\par 23 Επιστρέψατε προς τους ελέγχους μου· ιδού, εγώ θέλω εκχέει το πνεύμά μου εφ' υμάς, θέλω σας κάμει να νοήσητε τους λόγους μου.
\par 24 Επειδή εγώ έκραζον, και σεις δεν υπηκούετε· εξέτεινον την χείρα μου, και ουδείς προσείχεν·
\par 25 Αλλά κατεφρονείτε πάσας τας συμβουλάς μου και τους ελέγχους μου δεν εδέχεσθε·
\par 26 διά τούτο και εγώ θέλω επιγελάσει εις τον όλεθρόν σας· θέλω καταχαρή, όταν επέλθη ο φόβος σας.
\par 27 Όταν ο φόβος σας επέλθη ως ερήμωσις και η καταστροφή σας εφορμήση ως ανεμοστρόβιλος, όταν η θλίψις και η στενοχωρία έλθωσιν εφ' υμάς·
\par 28 τότε θέλουσι με επικαλεσθή, αλλά δεν θέλω αποκριθή· επιμόνως θέλουσι με εκζητήσει, αλλά δεν θέλουσι με ευρεί.
\par 29 Διότι εμίσησαν την γνώσιν και τον φόβον του Κυρίου δεν εξέλεξαν·
\par 30 δεν ηθέλησαν τας συμβουλάς μου· κατεφρόνησαν πάντας τους ελέγχους μου·
\par 31 διά τούτο θέλουσι φάγει από των καρπών της οδού αυτών και θέλουσι χορτασθή από των κακοβουλιών αυτών.
\par 32 Διότι η αποστασία των μωρών θέλει θανατώσει αυτούς, και η αμεριμνησία των αφρόνων θέλει αφανίσει αυτούς.
\par 33 Όστις όμως ακούει εμού, θέλει κατοικήσει εν ασφαλεία· και θέλει ησυχάζει, μη φοβούμενος κακόν.

\chapter{2}

\par 1 Υιέ μου, εάν δεχθής τους λόγους μου και ταμιεύσης τας εντολάς μου παρά σεαυτώ,
\par 2 ώστε να προσέξη το ωτίον σου εις την σοφίαν, να κλίνης την καρδίαν σου εις την σύνεσιν·
\par 3 και εάν επικαλεσθής την φρόνησιν, και υψώσης την φωνήν σου εις την σύνεσιν·
\par 4 εάν ζητήσης αυτήν ως αργύριον και εξερευνήσης αυτήν ως κεκρυμμένους θησαυρούς,
\par 5 τότε θέλεις εννοήσει τον φόβον του Κυρίου και θέλεις ευρεί την επίγνωσιν του Θεού.
\par 6 Διότι ο Κύριος δίδει σοφίαν· εκ του στόματος αυτού εξέρχεται γνώσις και σύνεσις.
\par 7 Αποταμιεύει σωτηρίαν εις τους ευθείς· είναι ασπίς εις τους περιπατούντας εν ακεραιότητι,
\par 8 υπερασπίζων τας οδούς της δικαιοσύνης και φυλάττων την οδόν των οσίων αυτού.
\par 9 Τότε θέλεις εννοήσει δικαιοσύνην και κρίσιν και ευθύτητα, πάσαν οδόν αγαθήν.
\par 10 Εάν η σοφία εισέλθη εις την καρδίαν σου και η γνώσις ηδύνη την ψυχήν σου,
\par 11 ορθή βουλή θέλει σε φυλάττει, σύνεσις θέλει σε διατηρεί·
\par 12 διά να σε ελευθερόνη από της οδού της πονηράς, από ανθρώπου λαλούντος δόλια,
\par 13 οίτινες εγκαταλείπουσι τας οδούς της ευθύτητος, διά να περιπατώσιν εν ταις οδοίς του σκότους·
\par 14 οίτινες ηδύνονται εις το να κάμνωσι κακόν, χαίρουσιν εις τας διαστροφάς της κακίας,
\par 15 των οποίων αι οδοί είναι σκολιαί και αι πορείαι αυτών διεστραμμέναι·
\par 16 διά να σε ελευθερόνη από ξένης γυναικός, από αλλοτρίας κολακευούσης με τους λόγους αυτής,
\par 17 ήτις εγκατέλιπε τον επιστήθιον της νεότητος αυτής και ελησμόνησε την διαθήκην του Θεού αυτής.
\par 18 Διότι ο οίκος αυτής καταβιβάζει εις τον θάνατον, και τα βήματα αυτής εις τους νεκρούς·
\par 19 πάντες οι εισερχόμενοι προς αυτήν δεν επιστρέφουσιν ουδέ αναλαμβάνουσι τας οδούς της ζωής·
\par 20 διά να περιπατής εν τη οδώ των αγαθών και να φυλάττης τας τρίβους των δικαίων.
\par 21 Διότι οι ευθείς θέλουσι κατοικήσει την γην, και οι τέλειοι θέλουσιν εναπολειφή εν αυτή.
\par 22 Οι δε ασεβείς θέλουσιν εκκοπή από της γης, και οι παράνομοι θέλουσιν εκριζωθή απ' αυτής.

\chapter{3}

\par 1 Υιέ μου, μη λησμονής τους νόμους μου, και η καρδία σου ας φυλάττη τας εντολάς μου.
\par 2 Διότι μακρότητα ημερών και έτη ζωής και ειρήνην θέλουσι προσθέσει εις σε.
\par 3 Έλεος και αλήθεια ας μη σε εγκαταλίπωσι· δέσον αυτάς περί τον τράχηλόν σου· εγχάραξον αυτάς επί την πλάκα της καρδίας σου·
\par 4 ούτω θέλεις ευρεί χάριν και εύνοιαν ενώπιον Θεού και ανθρώπων.
\par 5 Έλπιζε επί Κύριον εξ όλης σου της καρδίας, και μη επιστηρίζεσαι εις την σύνεσίν σου·
\par 6 εν πάσαις ταις οδοίς σου αυτόν γνώριζε, και αυτός θέλει διευθύνει τα διαβήματά σου.
\par 7 Μη φαντάζεσαι σεαυτόν σοφόν· φοβού τον Κύριον και έκκλινον από κακού.
\par 8 Τούτο θέλει είσθαι ίασις εις τα νεύρά σου και μυέλωσις εις τα οστά σου.
\par 9 Τίμα τον Κύριον από των υπαρχόντων σου και από των απαρχών πάντων των γεννημάτων σου·
\par 10 και θέλουσιν εμπλησθή αι σιτοθήκαι σου από αφθονίας και οι ληνοί σου θέλουσιν εκχειλίζει από νέου οίνου.
\par 11 Υιέ μου, μη καταφρόνει την παιδείαν του Κυρίου και μη αθύμει ελεγχόμενος υπ' αυτού.
\par 12 Διότι ο Κύριος ελέγχει όντινα αγαπά, καθώς και ο πατήρ τον υιόν, εις τον οποίον ευαρεστείται.
\par 13 Μακάριος ο άνθρωπος, όστις εύρηκε σοφίαν, και ο άνθρωπος, όστις απέκτησε σύνεσιν·
\par 14 Διότι το εμπόριον αυτής είναι καλήτερον παρά το εμπόριον του αργυρίου και το κέρδος αυτής παρά χρυσίον καθαρόν.
\par 15 Είναι τιμιωτέρα πολυτίμων λίθων· και πάντα όσα επιθυμήσης δεν είναι αντάξια αυτής.
\par 16 Μακρότης ημερών είναι εν τη δεξιά αυτής· εν τη αριστερά αυτής, πλούτος και δόξα.
\par 17 Αι οδοί αυτής είναι οδοί τερπναί και πάσαι αι τρίβοι αυτής ειρήνη.
\par 18 Είναι δένδρον ζωής εις τους εναγκαλιζομένους αυτήν· και μακάριοι οι κρατούντες αυτήν.
\par 19 Διά της σοφίας εθεμελίωσεν ο Κύριος, εστερέωσε τους ουρανούς εν συνέσει.
\par 20 Διά της γνώσεως αυτού αι άβυσσοι ηνοίχθησαν και τα νέφη σταλάζουσι δρόσον.
\par 21 Υιέ μου, ας μη απομακρυνθώσι ταύτα από των οφθαλμών σου· φύλαττε ορθήν βουλήν και φρόνησιν·
\par 22 και θέλει είσθαι ζωή εις την ψυχήν σου και χάρις εις τον τράχηλόν σου.
\par 23 Τότε θέλεις περιπατεί ασφαλώς την οδόν σου, και ο πους σου δεν θέλει προσκόψει.
\par 24 Όταν πλαγιάζης, δεν θέλεις τρομάζει· μάλιστα θέλεις πλαγιάζει, και ο ύπνος σου θέλει είσθαι γλυκύς.
\par 25 Δεν θέλεις τρομάξει από αιφνιδίου φόβου ουδέ από του ολέθρου των ασεβών, όταν επέλθη·
\par 26 Διότι ο Κύριος θέλει είσθαι η ελπίς σου, και θέλει φυλάξει τον πόδα σου από του να πιασθή.
\par 27 Μη αρνηθής το καλόν προς εκείνους, εις τους οποίους πρέπει, όταν ήναι εν τη χειρί σου να κάμνης αυτό.
\par 28 Μη είπης προς τον πλησίον σου, Ύπαγε και επανάστρεψον και αύριον θέλω σοι δώσει· ενώ έχεις τούτο παρά σεαυτώ.
\par 29 Μη μηχανεύου κακόν κατά του πλησίον σου, ενώ πεποιθώς κατοικεί μετά σου.
\par 30 Μη μάχου τινά αναιτίως, εάν δεν έκαμε κακόν εις σε.
\par 31 Μη ζήλευε τον βίαιον άνθρωπον και μη εκλέξης μηδεμίαν εκ των οδών αυτού·
\par 32 διότι ο Κύριος βδελύττεται τον σκολιόν· το δε απόρρητον αυτού φανερόνεται εις τους δικαίους.
\par 33 Κατάρα Κυρίου εν τω οίκω του ασεβούς· ευλογεί δε την κατοικίαν των δικαίων.
\par 34 Βεβαίως αυτός αντιτάττεται εις τους υπερηφάνους· εις δε τους ταπεινούς δίδει χάριν.
\par 35 Οι σοφοί θέλουσι κληρονομήσει δόξαν· το δε ύψος των αφρόνων θέλει είσθαι η ατιμία.

\chapter{4}

\par 1 Ακούσατε, τέκνα, παιδείαν πατρός, και προσέχετε να μάθητε σύνεσιν.
\par 2 Διότι δίδω εις εσάς καλήν διδασκαλίαν· μη εγκαταλίπητε τον νόμον μου.
\par 3 Διότι και εγώ εστάθην υιός του πατρός μου, αγαπητός και μονογενής ενώπιον της μητρός μου·
\par 4 και με εδίδασκε και μοι έλεγεν, Ας κρατή η καρδία σου τους λόγους μου· φύλαττε τας εντολάς μου και θέλεις ζήσει.
\par 5 Απόκτησον σοφίαν, απόκτησον σύνεσιν· μη λησμονήσης αυτήν, μηδέ εκκλίνης από των λόγων του στόματός μου·
\par 6 μη εγκαταλίπης αυτήν, και θέλει σε περιφυλάττει· αγάπα αυτήν, και θέλει σε διατηρεί.
\par 7 Η σοφία είναι το πρώτιστον· απόκτησον σοφίαν· και υπέρ πάσαν απόκτησίν σου απόκτησον σύνεσιν.
\par 8 Ανάλαβε αυτήν και θέλει σε υψώσει· θέλει σε δοξάσει, όταν εναγκαλισθής αυτήν.
\par 9 Θέλει επιθέσει επί την κεφαλήν σου στέφανον χαρίτων· θέλει σοι δώσει διάδημα δόξης.
\par 10 Άκουε, υιέ μου, και δέχθητι τους λόγους μου· και θέλουσι πληθυνθή τα έτη της ζωής σου.
\par 11 Σε διδάσκω την οδόν της σοφίας· σε εμβιβάζω εις τρίβους ευθείας.
\par 12 Όταν περιπατής, τα βήματά σου δεν θέλουσιν είσθαι εστενοχωρημένα· και όταν τρέχης, δεν θέλεις προσκόψει.
\par 13 Δράξον την παιδείαν, μη αφήσης αυτήν· φύλαττε αυτήν, διότι είναι η ζωή σου.
\par 14 Μη εισέλθης εις την τρίβον των ασεβών, και μη υπάγης εις την οδόν των πονηρών.
\par 15 Απόφευγε αυτήν, μη περάσης δι' αυτής, έκκλινον απ' αυτής και διάβα.
\par 16 Διότι αυτοί δεν κοιμώνται, εάν δεν κακοποιήσωσι· και ο ύπνος αυτών αφαιρείται, εάν δεν υποσκελίσωσιν.
\par 17 Επειδή τρώγουσιν άρτον ασεβείας και πίνουσιν οίνον δυναστείας.
\par 18 Η οδός όμως των δικαίων είναι ως το λαμπρόν φως, το φέγγον επί μάλλον και μάλλον, εωσού γείνη τελεία ημέρα.
\par 19 Η οδός των ασεβών είναι ως το σκότος· δεν γνωρίζουσι που προσκόπτουσιν.
\par 20 Υιέ μου, πρόσεχε εις τας ρήσεις μου· κλίνον το ωτίον σου εις τα λόγιά μου.
\par 21 Ας μη απομακρυνθώσιν από των οφθαλμών σου· φύλαττε αυτά εν τη καρδία σου·
\par 22 διότι είναι ζωή εις τους ευρίσκοντας αυτά και ίασις εις πάσαν αυτών την σάρκα.
\par 23 Μετά πάσης φυλάξεως φύλαττε την καρδίαν σου· διότι εκ ταύτης προέρχονται αι εκβάσεις της ζωής.
\par 24 Απόβαλε από σου σκολιότητα στόματος, και διαστροφήν χειλέων απομάκρυνον από σου.
\par 25 Οι οφθαλμοί σου ας βλέπωσιν ορθά, και τα βλέφαρά σου ας κατευθύνωνται έμπροσθέν σου.
\par 26 Στάθμιζε το βάδισμα των ποδών σου, και πάσαι αι οδοί σου θέλουσι κατευθυνθή.
\par 27 Μη εκκλίνης δεξιά ή αριστερά· απόστρεψον τον πόδα σου από κακού.

\chapter{5}

\par 1 Υιέ μου, πρόσεχε εις την σοφίαν μου, κλίνον το ωτίον σου εις την σύνεσίν μου·
\par 2 διά να τηρής φρόνησιν και τα χείλη σου να φυλάττωσι γνώσιν.
\par 3 Διότι τα χείλη της αλλοτρίας γυναικός στάζουσιν ως κηρήθρα μέλιτος, και ο ουρανίσκος αυτής είναι μαλακώτερος ελαίου·
\par 4 το τέλος όμως αυτής είναι πικρόν ως αψίνθιον, οξύ ως μάχαιρα δίστομος.
\par 5 Οι πόδες αυτής καταβαίνουσιν εις θάνατον· τα βήματα αυτής καταντώσιν εις τον άδην.
\par 6 διά να μη γνωρίσης την οδόν της ζωής, αι πορείαι αυτής είναι άστατοι και ουχί ευδιάγνωστοι.
\par 7 Ακούσατέ μου λοιπόν τώρα, τέκνα, και μη αποστραφήτε τους λόγους του στόματός μου.
\par 8 Απομάκρυνον την οδόν σου απ' αυτής, και μη πλησιάσης εις την θύραν του οίκου αυτής,
\par 9 διά να μη δώσης την τιμήν σου εις άλλους και τα έτη σου εις τους ανελεήμονας·
\par 10 διά να μη χορτασθώσι ξένοι από της περιουσίας σου και οι κόποι σου έλθωσιν εις οίκον αλλοτρίου,
\par 11 και συ στενάζης εις τα έσχατά σου, όταν η σαρξ σου και το σώμα σου καταναλωθώσι,
\par 12 και λέγης, Πως εμίσησα την παιδείαν, και η καρδία μου κατεφρόνησε τους ελέγχους,
\par 13 και δεν υπήκουσα εις την φωνήν των διδασκόντων με, ουδέ έκλινα το ωτίον μου εις τους νουθετούντάς με.
\par 14 Παρ' ολίγον έπεσον εις παν κακόν, εν μέσω της συνάξεως και της συναγωγής.
\par 15 Πίνε ύδατα εκ της δεξαμενής σου και πηγάζοντα εκ του φρέατός σου·
\par 16 Ας εκχέωνται έξω αι πηγαί σου, και τα ρυάκια των υδάτων σου εις τας πλατείας·
\par 17 σου μόνου ας ήναι αυτά, και ουχί ξένων μετά σού·
\par 18 η πηγή σου ας ήναι ευλογημένη· και ευφραίνου μετά της γυναικός της νεότητός σου.
\par 19 Ας ήναι εις σε ως έλαφος ερασμία και δορκάς κεχαριτωμένη· ας σε ποτίζωσιν οι μαστοί αυτής εν παντί καιρώ· ευφραίνου πάντοτε εις την αγάπην αυτής.
\par 20 Και διά τι, υιέ μου, θέλεις θέλγεσθαι υπό ξένης και θέλεις εναγκαλίζεσθαι κόλπον αλλοτρίας;
\par 21 Διότι του ανθρώπου αι οδοί είναι ενώπιον των οφθαλμών του Κυρίου, και σταθμίζει πάσας τας πορείας αυτού.
\par 22 Αι ίδιαι αυτού ανομίαι θέλουσι συλλάβει τον ασεβή, και με τα σχοινία της αμαρτίας αυτού θέλει σφίγγεσθαι.
\par 23 Ούτος θέλει αποθάνει απαίδευτος και εκ του πλήθους της αφροσύνης αυτού θέλει περιπλανάσθαι.

\chapter{6}

\par 1 Υιέ μου, εάν έγεινας εγγυητής διά τον φίλον σου, εάν έδωκας την χείρα σου εις ξένον,
\par 2 επαγιδεύθης διά των λόγων του στόματός σου, επιάσθης διά των λόγων του στόματός σου·
\par 3 Κάμε λοιπόν τούτο, υιέ μου, και σώζου, επειδή ήλθες εις τας χείρας του φίλου σου· ύπαγε, μη αποκάμης, και βίαζε τον φίλον σου.
\par 4 Μη δώσης ύπνον εις τους οφθαλμούς σου, μηδέ νυσταγμόν εις τα βλεφαρά σου·
\par 5 Σώζου, ως δορκάδιον εκ χειρός του κυνηγού και ως πτηνόν εκ χειρός του ιξευτού.
\par 6 Ύπαγε προς τον μύρμηκα, ω οκνηρέ· παρατήρησον τας οδούς αυτού και γίνου σοφός·
\par 7 όστις μη έχων άρχοντα, επιστάτην ή κυβερνήτην,
\par 8 ετοιμάζει την τροφήν αυτού το θέρος, συνάγει τας τροφάς αυτού εν τω θερισμώ.
\par 9 Έως πότε θέλεις κοιμάσθαι, οκνηρέ; πότε θέλεις σηκωθή εκ του ύπνου σου;
\par 10 Ολίγος ύπνος, ολίγος νυσταγμός, ολίγη συμπλοκή των χειρών εις τον ύπνον·
\par 11 Έπειτα η πτωχεία σου έρχεται ως ταχυδρόμος, και η ένδειά σου ως ανήρ ένοπλος.
\par 12 Ο αχρείος άνθρωπος, ο κακότροπος άνθρωπος, περιπατεί με στόμα διεστραμμένον·
\par 13 Κάμνει νεύμα διά των οφθαλμών αυτού, σημαίνει διά των ποδών αυτού, διδάσκει διά των δακτύλων αυτού·
\par 14 μετά διεστραμμένης καρδίας μηχανάται κακά εν παντί καιρώ· εγείρει έριδας·
\par 15 διά τούτο εξαίφνης θέλει επέλθει η απώλεια αυτού· εξαίφνης θέλει συντριφθή ανιάτως.
\par 16 Ταύτα τα εξ μισεί ο Κύριος, επτά μάλιστα βδελύττεται η ψυχή αυτού·
\par 17 οφθαλμούς υπερηφάνους, γλώσσαν ψευδή και χείρας εκχεούσας αίμα αθώον,
\par 18 καρδίαν μηχανευομένην λογισμούς κακούς, πόδας τρέχοντας ταχέως εις το κακοποιείν,
\par 19 μάρτυρα ψευδή λαλούντα ψεύδος και τον εμβάλλοντα έριδας μεταξύ αδελφών.
\par 20 Υιέ μου, φύλαττε την εντολήν του πατρός σου, και μη απορρίψης τον νόμον της μητρός σου.
\par 21 Περίαψον αυτά διαπαντός επί της καρδίας σου, περίδεσον αυτά περί τον τράχηλόν σου.
\par 22 Όταν περιπατής, θέλει σε οδηγεί· όταν κοιμάσαι, θέλει σε φυλάττει· και όταν εξυπνήσης, θέλει συνομιλεί μετά σου.
\par 23 Διότι λύχνος είναι η εντολή και φως ο νόμος, και οι έλεγχοι της παιδείας οδός ζωής·
\par 24 διά να σε φυλάττωσιν από κακής γυναικός, από κολακείας γλώσσης γυναικός αλλοτρίας.
\par 25 Μη ορεχθής το κάλλος αυτής εν τη καρδία σου· και ας μη σε θηρεύση διά των βλεφάρων αυτής.
\par 26 Διότι εξ αιτίας γυναικός πόρνης καταντά τις έως τμήματος άρτου, η δε μοιχαλίς θηρεύει την πολύτιμον ψυχήν.
\par 27 Δύναταί τις να βάλη πυρ εις τον κόλπον αυτού, και τα ιμάτια αυτού να μη καώσι;
\par 28 Δύναταί τις να περιπατήση επ' ανθράκων πυρός, και οι πόδες αυτού να μη κατακαώσιν;
\par 29 Ούτω και ο εισερχόμενος προς την γυναίκα του πλησίον αυτού· όστις εγγίζει αυτήν, δεν θέλει αθωωθή.
\par 30 Τον κλέπτην δεν αποστρέφονται, εάν κλέπτη διά να χορτάση την ψυχήν αυτού, όταν πεινά·
\par 31 αλλ' εάν πιασθή, θέλει αποδώσει επταπλάσια· θέλει δώσει πάντα τα υπάρχοντα της οικίας αυτού.
\par 32 Όστις όμως μοιχεύει με γυναίκα, είναι ενδεής φρενών· απώλειαν φέρει εις την ψυχήν αυτού, όστις πράττει τούτο.
\par 33 Πληγάς και ατιμίαν θέλει υποφέρει· και το όνειδος αυτού δεν θέλει εξαλειφθή.
\par 34 Διότι η ζηλοτυπία είναι μανία του ανδρός, και δεν θέλει δείξει έλεος εις την ημέραν της εκδικήσεως.
\par 35 Δεν θέλει δεχθή ουδέν λύτρον· ουδέ θέλει εξιλεωθή, και αν πολλαπλασιάσης τα δώρα.

\chapter{7}

\par 1 Υιέ μου, φύλαττε τους λόγους μου και ταμίευσον τας εντολάς μου παρά σεαυτώ.
\par 2 Φύλαττε τας εντολάς μου, και θέλεις ζήσει· και τον νόμον μου, ως την κόρην των οφθαλμών σου.
\par 3 Δέσον αυτά επί τους δακτύλους σου, εγχάραξον αυτά επί την πλάκα της καρδίας σου.
\par 4 Ειπέ προς την σοφίαν; συ είσαι αδελφή μου· και κάλεσον την φρόνησιν συγγενή σου·
\par 5 διά να σε φυλάττωσιν από ξένης γυναικός, από αλλοτρίας κολακευούσης διά των λόγων αυτής.
\par 6 Επειδή από του παραθύρου της οικίας μου έκυψα διά του δικτυωτού μου·
\par 7 και είδον μεταξύ των αφρόνων, παρετήρησα μεταξύ των νεανίσκων, νέον ενδεή φρενών·
\par 8 όστις διέβαινε διά της πλατείας, πλησίον της γωνίας αυτής, και διήρχετο την οδόν προς την οικίαν αυτής,
\par 9 εν τω εσπερινώ σκότει της ημέρας, εν τω σκοτασμώ της νυκτός και τω γνόφω·
\par 10 και ιδού, συναπαντά αυτόν γυνή έχουσα σχήμα πορνικόν, και καρδίαν δολιόφρονα,
\par 11 φλύαρος και αναιδής· οι πόδες αυτής δεν μένουσιν εν τω οίκω αυτής·
\par 12 τώρα είναι έξω, τώρα εν ταις πλατείαις, και ενεδρεύει πλησίον πάσης γωνίας.
\par 13 Και πιάνει αυτόν και φιλεί αυτόν και με αναιδές πρόσωπον λέγει προς αυτόν,
\par 14 Έχω θυσίας ειρηνικάς· σήμερον απέδωκα τας ευχάς μου·
\par 15 διά τούτο εξήλθον εις απάντησίν σου, ποθούσα το πρόσωπόν σου, και σε εύρηκα·
\par 16 έστρωσα την κλίνην μου με πέπλους, με τάπητας πεποικιλμένους, με νήματα της Αιγύπτου·
\par 17 εθυμίασα την κλίνην μου με σμύρναν, αλόην και κινάμωμον·
\par 18 ελθέ, ας μεθυσθώμεν από έρωτος μέχρι της αυγής· ας εντρυφήσωμεν εις έρωτας·
\par 19 διότι δεν είναι ο ανήρ εν τη οικία αυτού, υπήγεν εις οδόν μακράν·
\par 20 έλαβε βαλάντιον αργυρίου εν τη χειρί αυτού· εν ωρισμένω καιρώ θέλει επανέλθει εις την οικίαν αυτού.
\par 21 Διά της πολλής αυτής τέχνης απεπλάνησεν αυτόν· διά της κολακείας των χειλέων αυτής είλκυσεν αυτόν.
\par 22 Ευθύς ακολουθεί αυτήν κατόπιν, καθώς ο βους υπάγει εις την σφαγήν, ή καθώς η έλαφος πηδά εις τον βρόχον,
\par 23 εωσού βέλος διαπεράση το ήπαρ αυτής· καθώς το πτηνόν σπεύδει εις την παγίδα και δεν εξεύρει ότι είναι εναντίον της ζωής αυτού.
\par 24 Τώρα λοιπόν ακούσατέ μου, τέκνα, και προσέχετε εις τους λόγους του στόματός μου.
\par 25 Ας μη εκκλίνη εις τας οδούς αυτής η καρδία σου, μη παρεκτραπής εις τας τρίβους αυτής.
\par 26 Διότι πολλούς έκαμε να πέσωσι πεπληγωμένοι, και δυνατοί είναι οι φονευθέντες υπ' αυτής.
\par 27 Οδοί άδου είναι ο οίκος αυτής, καταβαίνουσαι εις τα ταμεία του θανάτου.

\chapter{8}

\par 1 Δεν κράζει η σοφία; και δεν εκπέμπει την φωνήν αυτής η σύνεσις;
\par 2 Ίσταται επί της κορυφής των υψηλών τόπων, υπέρ την οδόν, εν τω μέσω των τριόδων.
\par 3 Κράζει πλησίον των πυλών, εν τη εισόδω της πόλεως, εν τη εισόδω των θυρών·
\par 4 προς εσάς, άνθρωποι, κράζω· και η φωνή μου εκπέμπεται προς τους υιούς των ανθρώπων.
\par 5 Απλοί, νοήσατε φρόνησιν· και άφρονες, αποκτήσατε νοήμονα καρδίαν.
\par 6 Ακούσατε· διότι θέλω λαλήσει πράγματα έξοχα, και τα χείλη μου θέλουσι προφέρει ορθά.
\par 7 Διότι αλήθειαν θέλει λαλήσει ο λάρυγξ μου· τα δε χείλη μου βδελύττονται την ασέβειαν.
\par 8 Πάντες οι λόγοι του στόματός μου είναι μετά δικαιοσύνης· δεν υπάρχει εν αυτοίς δόλιον διεστραμμένον·
\par 9 Πάντες είναι σαφείς εις τον νοούντα και ορθοί εις τους ευρίσκοντας γνώσιν.
\par 10 Λάβετε την παιδείαν μου, και μη αργύριον· και γνώσιν, μάλλον παρά χρυσίον εκλεκτόν.
\par 11 Διότι η σοφία είναι καλητέρα λίθων πολυτίμων· και πάντα τα επιθυμητά πράγματα δεν είναι αντάξια αυτής.
\par 12 Εγώ η σοφία κατοικώ μετά της φρονήσεως, και εφευρίσκω γνώσιν συνετών βουλευμάτων.
\par 13 Ο φόβος του Κυρίου είναι να μισή τις το κακόν· αλαζονείαν και αυθάδειαν και πονηράν οδόν και διεστραμμένον στόμα εγώ μισώ.
\par 14 Εμού είναι η βουλή και η ασφάλεια· εγώ είμαι η σύνεσις· εμού η δύναμις.
\par 15 Δι' εμού οι βασιλείς βασιλεύουσι, και οι άρχοντες θεσπίζουσι δικαιοσύνην.
\par 16 Δι' εμού οι ηγεμόνες ηγεμονεύουσι, και οι μεγιστάνες, πάντες οι κριταί της γής·
\par 17 Εγώ τους εμέ αγαπώντας αγαπώ· και οι ζητούντές με θέλουσι με ευρεί.
\par 18 Πλούτος και δόξα είναι μετ' εμού, αγαθά διαμένοντα και δικαιοσύνη.
\par 19 Οι καρποί μου είναι καλήτεροι χρυσίου και χρυσίου καθαρού· και τα γεννήματά μου, εκλεκτού αργυρίου.
\par 20 Περιπατώ εν οδώ δικαιοσύνης, αναμέσον των τρίβων της κρίσεως,
\par 21 διά να κάμω τους αγαπώντάς με να κληρονομήσωσιν αγαθά, και να γεμίσω τους θησαυρούς αυτών.
\par 22 Ο Κύριος με είχεν εν τη αρχή των οδών αυτού, προ των έργων αυτού, απ' αιώνος.
\par 23 Προ του αιώνος με έχρισεν, απ' αρχής, πριν υπάρξη η γη.
\par 24 Εγεννήθην ότε δεν ήσαν αι άβυσσοι, ότε δεν υπήρχον αι πηγαί αι αναβρύουσαι ύδατα·
\par 25 Πριν τα όρη θεμελιωθώσι, προ των λόφων, εγώ εγεννήθην·
\par 26 ενώ δεν είχεν έτι κάμει την γην ούτε πεδιάδας, ούτε κορυφάς χωμάτων της οικουμένης.
\par 27 Ότε ητοίμαζε τους ουρανούς, εγώ ήμην εκεί· ότε περιέγραφε καμάραν υπεράνω του προσώπου της αβύσσου·
\par 28 ότε εστερέονε τον αιθέρα επάνω· ότε ωχύρονε τας πηγάς της αβύσσου·
\par 29 ότε επέβαλλε τον νόμον αυτού εις την θάλασσαν, να μη παραβώσι τα ύδατα το πρόσταγμα αυτού· ότε διέταττε τα θεμέλια της γής·
\par 30 τότε ήμην πλησίον αυτού δημιουργούσα· και εγώ ήμην καθ' ημέραν η τρυφή αυτού, ευφραινομένη πάντοτε ενώπιον αυτού,
\par 31 ευφραινομένη εν τη οικουμένη της γης αυτού· και η τρυφή μου ήτο μετά των υιών των ανθρώπων.
\par 32 Τώρα λοιπόν ακούσατέ μου, ω τέκνα· διότι μακάριοι οι φυλάττοντες τας οδούς μου.
\par 33 Ακούσατε παιδείαν και γένεσθε σοφοί, και μη αποδοκιμάζετε αυτήν.
\par 34 Μακάριος ο άνθρωπος, όστις μου ακούση, αγρυπνών καθ' ημέραν εν ταις πύλαις μου, περιμένων εις τους παραστάτας των θυρών μου·
\par 35 διότι όστις εύρη εμέ, θέλει ευρεί ζωήν, και θέλει λάβει χάριν παρά Κυρίου.
\par 36 Όστις όμως αμαρτήση εις εμέ, την εαυτού ψυχήν αδικεί· πάντες οι μισούντές με αγαπώσι θάνατον.

\chapter{9}

\par 1 Η σοφία ωκοδόμησε τον οίκον αυτής, ελατόμησε τους στύλους αυτής επτά·
\par 2 έσφαξε τη σφάγια αυτής, εκέρασε τον οίνον αυτής, και ητοίμασε την τράπεζαν αυτής·
\par 3 απέστειλε τας θεραπαίνας αυτής, κηρύττει επί των υψηλών τόπων της πόλεως,
\par 4 Όστις είναι άφρων, ας στραφή εδώ· και, προς τους ενδεείς φρενών, λέγει προς αυτούς,
\par 5 Έλθετε, φάγετε από του άρτου μου, και πίετε από του οίνου τον οποίον εκέρασα·
\par 6 αφήσατε την αφροσύνην και ζήσατε· και κατευθύνθητε εν τη οδώ της συνέσεως.
\par 7 Ο νουθετών χλευαστήν λαμβάνει εις εαυτόν ατιμίαν· και ο ελέγχων τον ασεβή λαμβάνει εις εαυτόν μώμον.
\par 8 Μη έλεγχε χλευαστήν, διά να μη σε μισήση· έλεγχε σοφόν, και θέλει σε αγαπήσει.
\par 9 Δίδε αφορμήν εις τον σοφόν και θέλει γείνει σοφώτερος· δίδασκε τον δίκαιον και θέλει αυξηθή εις μάθησιν.
\par 10 Αρχή σοφίας φόβος Κυρίου· και επίγνωσις αγίων φρόνησις.
\par 11 Διότι δι' εμού αι ημέραι σου θέλουσι πολλαπλασιασθή, και έτη ζωής θέλουσι προστεθή εις σε.
\par 12 Εάν γείνης σοφός, θέλεις είσθαι σοφός διά σεαυτόν· και εάν γείνης χλευαστής, συ μόνος θέλεις πάσχει.
\par 13 Γυνή άφρων, θρασεία, ανόητος και μη γνωρίζουσα μηδέν·
\par 14 κάθηται εν τη θύρα της οικίας αυτής επί θρόνου, εν τοις υψηλοίς τόποις της πόλεως,
\par 15 προσκαλούσα τους διαβάτας τους κατευθυνομένους εις την οδόν αυτών·
\par 16 όστις είναι άφρων, ας στραφή εδώ· και προς τον ενδεή φρενών, λέγει προς αυτόν,
\par 17 Τα κλοπιμαία ύδατα είναι γλυκέα, και ο κρύφιος άρτος είναι ηδύς.
\par 18 Αλλ' αυτός αγνοεί ότι εκεί είναι οι νεκροί, και εις τα βάθη του άδου οι κεκλημένοι αυτής.

\chapter{10}

\par 1 Παροιμίαι Σολομώντος. Υιός σοφός ευφραίνει πατέρα· υιός δε άφρων είναι λύπη της μητρός αυτού.
\par 2 Οι θησαυροί της ανομίας δεν ωφελούσιν· η δε δικαιοσύνη ελευθερόνει εκ θανάτου.
\par 3 Ο Κύριος δεν θέλει λιμοκτονήσει ψυχήν δικαίου· ανατρέπει δε την περιουσίαν των ασεβών.
\par 4 Η οκνηρά χειρ πτωχείαν φέρει· πλουτίζει δε η χειρ του επιμελούς.
\par 5 Ο συνάγων εν τω θέρει είναι υιός συνέσεως· ο δε κοιμώμενος εν τω θερισμώ υιός αισχύνης.
\par 6 Ευλογία επί την κεφαλήν του δικαίου· το στόμα δε των ασεβών αδικία καλύπτει.
\par 7 Η μνήμη του δικαίου είναι μετ' ευλογίας· το δε όνομα των ασεβών σήπεται.
\par 8 Ο σοφός την καρδίαν θέλει δέχεσθαι εντολάς· ο δε μωρός τα χείλη θέλει υποσκελισθή.
\par 9 Ο περιπατών εν ακεραιότητι περιπατεί ασφαλώς· ο δε διαστρέφων τας οδούς αυτού θέλει γνωρισθή.
\par 10 Όστις νεύει διά του οφθαλμού, προξενεί οδύνην· ο δε μωρός τα χείλη θέλει υποσκελισθή.
\par 11 Το στόμα του δικαίου είναι πηγή ζωής· το στόμα δε των ασεβών αδικία καλύπτει.
\par 12 Το μίσος διεγείρει έριδας· αλλ' η αγάπη καλύπτει πάντα τα σφάλματα.
\par 13 Εις τα χείλη του συνετού ευρίσκεται η σοφία· η δε ράβδος είναι διά την ράχιν του ενδεούς φρενών.
\par 14 Οι σοφοί αποταμιεύουσι γνώσιν· το στόμα δε του προπετούς είναι πλησίον απωλείας.
\par 15 Τα αγαθά του πλουσίου είναι η οχυρά αυτού πόλις· καταστροφή δε των πενήτων πτωχεία αυτών.
\par 16 Τα έργα του δικαίου είναι εις ζωήν· το προϊόν του ασεβούς εις αμαρτίαν.
\par 17 Ο φυλάττων την παιδείαν ευρίσκεται εν οδώ ζωής· ο δε εγκαταλείπων τον έλεγχον αποπλανάται.
\par 18 Όστις καλύπτει μίσος υπό χείλη ψευδή, και όστις προφέρει συκοφαντίαν, είναι άφρων.
\par 19 Εν τη πολυλογία δεν λείπει αμαρτία· αλλ' όστις κρατεί τα χείλη αυτού, είναι συνετός.
\par 20 Η γλώσσα του δικαίου αργύριον εκλεκτόν· η καρδία των ασεβών πράγμα μηδαμινόν.
\par 21 Τα χείλη του δικαίου βόσκουσι πολλούς· οι δε άφρονες αποθνήσκουσι δι' έλλειψιν φρενών.
\par 22 Η ευλογία του Κυρίου πλουτίζει, και λύπη δεν θέλει προστεθή εις αυτήν.
\par 23 Ως γέλως είναι εις τον άφρονα να πράττη κακόν· η δε σοφία είναι ανδρός συνετού.
\par 24 Ο φόβος του ασεβούς θέλει επέλθει επ' αυτόν· η επιθυμία δε των δικαίων θέλει εκπληρωθή.
\par 25 Καθώς παρέρχεται ο ανεμοστρόβιλος, ούτως ο ασεβής δεν υπάρχει· ο δε δίκαιος θέλει είσθαι τεθεμελιωμένος εις τον αιώνα.
\par 26 Καθώς το όξος εις τους οδόντας και ο καπνός εις τους οφθαλμούς, ούτως είναι ο οκνηρός εις τους αποστέλλοντας αυτόν.
\par 27 Ο φόβος του Κυρίου προσθέτει ημέρας· τα δε έτη των ασεβών θέλουσιν ελαττωθή.
\par 28 Η προσδοκία των δικαίων θέλει είσθαι ευφροσύνη· η ελπίς όμως των ασεβών θέλει απολεσθή.
\par 29 Η οδός του Κυρίου είναι οχύρωμα εις τον άμεμπτον, όλεθρος δε εις τους εργάτας της ανομίας.
\par 30 Ο δίκαιος εις τον αιώνα δεν θέλει σαλευθή· οι δε ασεβείς δεν θέλουσι κατοικήσει την γην.
\par 31 Το στόμα του δικαίου αναδίδει σοφίαν· η δε ψευδής γλώσσα θέλει εκκοπή.
\par 32 Τα χείλη του δικαίου γνωρίζουσι το ευχάριστον· το στόμα δε των ασεβών τα διεστραμμένα.

\chapter{11}

\par 1 Δολία πλάστιγξ βδέλυγμα εις τον Κύριον· δίκαιον δε ζύγιον ευαρέστησις αυτού.
\par 2 Όπου εισέλθη υπερηφανία, εισέρχεται και καταισχύνη· η δε σοφία είναι μετά των ταπεινών.
\par 3 Η ακεραιότης των ευθέων θέλει οδηγεί αυτούς· η δε υπουλότης των σκολιών θέλει απολέσει αυτούς.
\par 4 Τα πλούτη δεν ωφελούσιν εν ημέρα οργής· η δε δικαιοσύνη ελευθερόνει εκ θανάτου.
\par 5 Η δικαιοσύνη του ακεραίου θέλει ορθοτομήσει την οδόν αυτού· ο δε ασεβής θέλει πέσει διά της ασεβείας αυτού.
\par 6 Η δικαιοσύνη των ευθέων θέλει ελευθερώσει αυτούς· οι δε παραβάται θέλουσι συλληφθή εν τη κακία αυτών.
\par 7 Όταν ο ασεβής άνθρωπος αποθνήσκη, η ελπίς αυτού απόλλυται· απόλλυται και η προσδοκία των ανόμων.
\par 8 Ο δίκαιος ελευθερόνεται εκ της θλίψεως, αντ' αυτού δε εισέρχεται ο ασεβής.
\par 9 Ο υποκριτής διά του στόματος αφανίζει τον πλησίον αυτού· αλλ' οι δίκαιοι θέλουσιν ελευθερωθή διά της γνώσεως.
\par 10 Εις την ευόδωσιν των δικαίων η πόλις ευφραίνεται· και εις τον όλεθρον των ασεβών αγάλλεται.
\par 11 Διά της ευλογίας των ευθέων υψόνεται πόλις· διά του στόματος δε των ασεβών καταστρέφεται.
\par 12 Ο ενδεής φρενών περιφρονεί τον πλησίον αυτού· ο δε φρόνιμος άνθρωπος σιωπά.
\par 13 Ο σπερμολόγος περιέρχεται αποκαλύπτων τα μυστικά· ο δε την ψυχήν πιστός κρύπτει το πράγμα.
\par 14 Όπου δεν είναι κυβέρνησις, ο λαός πίπτει· εκ του πλήθους δε των συμβούλων προέρχεται σωτηρία.
\par 15 Όστις εγγυάται δι' άλλον, θέλει πάθει κακόν· και όστις μισεί την εγγύησιν, είναι ασφαλής.
\par 16 Η εύκοσμος γυνή απολαμβάνει τιμήν· οι δε καρτερικοί απολαμβάνουσι πλούτη.
\par 17 Ο ελεήμων άνθρωπος αγαθοποιεί την ψυχήν αυτού· ο δε ανελεήμων θλίβει την σάρκα αυτού.
\par 18 Ο ασεβής εργάζεται έργον ψευδές· εις δε τον σπείροντα δικαιοσύνην θέλει είσθαι μισθός ασφαλής.
\par 19 Καθώς η δικαιοσύνη τείνει εις ζωήν, ούτως ο κυνηγών το κακόν τρέχει εις τον θάνατον αυτού.
\par 20 Οι διεστραμμένοι την καρδίαν είναι βδέλυγμα εις τον Κύριον· αλλ' οι άμεμπτοι την οδόν είναι δεκτοί εις αυτόν.
\par 21 Και χειρ με χείρα εάν συνάπτηται, ο ασεβής δεν θέλει μένει ατιμώρητος· το δε σπέρμα των δικαίων θέλει ελευθερωθή.
\par 22 Ως έρρινον χρυσούν εις χοίρου μύτην, ούτω γυνή ώραία χωρίς φρονήσεως.
\par 23 Η επιθυμία των δικαίων είναι μόνον το καλόν· η προσδοκία δε των ασεβών οργή.
\par 24 Οι μεν σκορπίζουσι, και όμως περισσεύονται· οι δε παρά το δέον φείδονται, και όμως έρχονται εις ένδειαν.
\par 25 Η αγαθοποιός ψυχή θέλει παχυνθή· και όστις ποτίζει, θέλει ποτισθή και αυτός.
\par 26 Όστις κρατεί σίτον, θέλει είσθαι λαοκατάρατος· ευλογία δε θέλει είσθαι επί την κεφαλήν του πωλούντος.
\par 27 Όστις προθυμείται εις το καλόν, θέλει απολαύσει χάριν· αλλ' όστις ζητεί το κακόν, θέλει επέλθει επ' αυτόν.
\par 28 Όστις ελπίζει επί τον πλούτον αυτού, ούτος θέλει πέσει· οι δε δίκαιοι ως βλαστός θέλουσιν ανθήσει.
\par 29 Όστις ταράττει τον οίκον αυτού, θέλει κληρονομήσει άνεμον· και ο άφρων θέλει είσθαι δούλος εις τον φρόνιμον.
\par 30 Ο καρπός του δικαίου είναι δένδρον ζωής· και όστις κερδίζει ψυχάς, είναι σοφός.
\par 31 Αν ο δίκαιος παιδεύηται επί της γης, πολλώ μάλλον ο ασεβής και ο αμαρτωλός.

\chapter{12}

\par 1 Όστις αγαπά παιδείαν, αγαπά γνώσιν· αλλ' όστις μισεί έλεγχον, είναι άφρων.
\par 2 Ο καλός ευρίσκει χάριν παρά Κυρίου· τον δε μηχανευόμενον κακά θέλει καταδικάσει.
\par 3 Δεν θέλει στερεωθή άνθρωπος διά της ανομίας· η ρίζα δε των δικαίων θέλει μένει ασάλευτος.
\par 4 Η ενάρετος γυνή είναι στέφανος εις τον άνδρα αυτής· η δε προξενούσα αισχύνην είναι ως σαπρία εις τα οστά αυτού.
\par 5 Οι λογισμοί των δικαίων είναι ευθύτης· αι δε βουλαί των ασεβών δόλος.
\par 6 Οι λόγοι των ασεβών ενεδρεύουσιν αίμα· το δε στόμα των ευθέων θέλει ελευθερώσει αυτούς.
\par 7 Οι ασεβείς καταστρέφονται και δεν υπάρχουσιν· ο οίκος δε των δικαίων θέλει διαμένει.
\par 8 Ο άνθρωπος εγκωμιάζεται κατά την σύνεσιν αυτού· ο δε διεστραμμένος την καρδίαν θέλει είσθαι εις καταφρόνησιν.
\par 9 Καλήτερος ο άνθρωπος ο μη τιμώμενος και επαρκών εις εαυτόν, παρά ο κενοδοξών και στερούμενος άρτου.
\par 10 Ο δίκαιος επιμελείται την ζωήν του κτήνους αυτού· τα δε σπλάγχνα των ασεβών είναι ανελεήμονα.
\par 11 Ο εργαζόμενος την γην αυτού θέλει χορτασθή άρτον· ο δε ακολουθών τους ματαιόφρονας είναι ενδεής φρενών.
\par 12 Ο ασεβής ζητεί την υπεράσπισιν των κακών· αλλ' η ρίζα του δικαίου αναδίδει.
\par 13 Δι' αμαρτίαν χειλέων παγιδεύεται ο ασεβής· ο δε δίκαιος εξέρχεται εκ στενοχωρίας.
\par 14 Εκ των καρπών του στόματος αυτού ο άνθρωπος θέλει εμπλησθή αγαθών· και η αμοιβή των χειρών του ανθρώπου θέλει επιστρέψει εις αυτόν.
\par 15 Η οδός του άφρονος είναι ορθή εις τους οφθαλμούς αυτού· ο δε ακούων συμβουλάς είναι σοφός.
\par 16 Ο άφρων φανερόνει ευθύς την οργήν αυτού· ο δε φρόνιμος σκεπάζει το όνειδος αυτού.
\par 17 Ο λαλών αλήθειαν αναγγέλλει το δίκαιον· ο δε ψευδομάρτυς δόλον.
\par 18 Ο φλύαρος είναι ως τραύματα μαχαίρας· η δε γλώσσα των σοφών, ίασις.
\par 19 Τα χείλη της αληθείας θέλουσιν είσθαι σταθερά διαπαντός· η δε ψευδής γλώσσα μόνον στιγμιαία.
\par 20 Δόλος είναι εν τη καρδία των μηχανευομένων κακά· ευφροσύνη δε εις τους βουλευομένους ειρήνην.
\par 21 Ουδεμία βλάβη θέλει συμβή εις τον δίκαιον· οι δε ασεβείς θέλουσιν εμπλησθή κακών.
\par 22 Ψευδή χείλη βδέλυγμα εις τον Κύριον· οι δε ποιούντες αλήθειαν είναι δεκτοί εις αυτόν.
\par 23 Ο φρόνιμος άνθρωπος καλύπτει γνώσιν· η δε καρδία των αφρόνων διακηρύττει μωρίαν.
\par 24 Η χειρ των επιμελών θέλει εξουσιάζει· οι δε οκνηροί θέλουσιν είσθαι υποτελείς.
\par 25 Η λύπη εν τη καρδία του ανθρώπου ταπεινόνει αυτήν· ο δε καλός λόγος ευφραίνει αυτήν.
\par 26 Ο δίκαιος υπερέχει του πλησίον αυτού· η δε οδός των ασεβών πλανά αυτούς.
\par 27 Ο οκνηρός δεν επιτυγχάνει του θηράματος αυτού· τα δε υπάρχοντα του επιμελούς ανθρώπου είναι πολύτιμα.
\par 28 Εν τη οδώ της δικαιοσύνης είναι ζωή· και η πορεία της οδού αυτής δεν φέρει εις θάνατον.

\chapter{13}

\par 1 Ο σοφός υιός δέχεται την διδασκαλίαν του πατρός· ο δε χλευαστής δεν ακούει έλεγχον.
\par 2 Εκ των καρπών του στόματος αυτού ο άνθρωπος θέλει φάγει αγαθά· η δε ψυχή των ανόμων αδικίαν.
\par 3 Ο φυλάττων το στόμα αυτού διαφυλάττει την ζωήν αυτού· ο δε ανοίγων προπετώς τα χείλη αυτού θέλει απολεσθή.
\par 4 Η ψυχή του οκνηρού επιθυμεί και δεν έχει· η δε ψυχή των επιμελών θέλει χορτασθή.
\par 5 Ο δίκαιος μισεί λόγον ψευδή· ο δε ασεβής καθίσταται δυσώδης και άτιμος.
\par 6 Η δικαιοσύνη φυλάττει τον τέλειον την οδόν· η δε ασέβεια καταστρέφει τον αμαρτωλόν.
\par 7 Υπάρχει άνθρωπος όστις κάμνει τον πλούσιον, και δεν έχει ουδέν· και άλλος όστις κάμνει τον πτωχόν, και έχει πλούτον πολύν.
\par 8 Το λύτρον της ψυχής του ανθρώπου είναι ο πλούτος αυτού· ο δε πτωχός δεν ακούει επίπληξιν.
\par 9 Το φως των δικαίων είναι φαιδρόν· ο δε λύχνος των ασεβών θέλει σβεσθή.
\par 10 Μόνον από της υπερηφανίας προέρχεται η έρις· η δε σοφία είναι μετά των δεχομένων συμβουλάς.
\par 11 Τα εκ ματαιότητος πλούτη θέλουσιν ελαττωθή· ο δε συνάγων με την χείρα αυτού θέλει αυξηνθή.
\par 12 Η ελπίς αναβαλλομένη ατονίζει την καρδίαν· το δε ποθούμενον, όταν έρχηται, είναι δένδρον ζωής.
\par 13 Ο καταφρονών τον λόγον θέλει αφανισθή· ο δε φοβούμενος την εντολήν, ούτος θέλει ανταμειφθή.
\par 14 Ο νόμος του σοφού είναι πηγή ζωής, απομακρύνων από παγίδων θανάτου.
\par 15 Σύνεσις αγαθή δίδει χάριν· η δε οδός των παρανόμων φέρει εις όλεθρον.
\par 16 Πας φρόνιμος πράττει μετά γνώσεως· ο δε άφρων ανακαλύπτει μωρίαν.
\par 17 Ο κακός μηνυτής πίπτει εις δυστυχίαν· ο δε πιστός πρέσβυς είναι ίασις.
\par 18 Πτωχεία και αισχύνη θέλουσιν είσθαι εις τον αποβάλλοντα την διδασκαλίαν· ο δε φυλάττων τον έλεγχον θέλει τιμηθή.
\par 19 Επιθυμία εκπληρωθείσα ευφραίνει την ψυχήν· εις δε τους άφρονας είναι βδελυρόν να εκκλίνωσιν από του κακού.
\par 20 Ο περιπατών μετά σοφών θέλει είσθαι σοφός· ο δε σύντροφος των αφρόνων θέλει απολεσθή.
\par 21 Κακόν παρακολουθεί τους αμαρτωλούς· εις δε τους δικαίους θέλει ανταποδοθή καλόν.
\par 22 Ο αγαθός αφίνει κληρονομίαν εις υιούς υιών· ο πλούτος δε του αμαρτωλού θησαυρίζεται διά τον δίκαιον.
\par 23 Πολλήν τροφήν δίδει ο αγρός των πτωχών· τινές δε δι' έλλειψιν κρίσεως αφανίζονται.
\par 24 Ο φειδόμενος της ράβδου αυτού μισεί τον υιόν αυτού· αλλ' ο αγαπών αυτόν παιδεύει αυτόν εν καιρώ.
\par 25 Ο δίκαιος τρώγει μέχρι χορτασμού της ψυχής αυτού· η δε κοιλία των ασεβών θέλει στερείσθαι.

\chapter{14}

\par 1 Αι σοφαί γυναίκες οικοδομούσι τον οίκον αυτών· η δε άφρων κατασκάπτει αυτόν διά των χειρών αυτής.
\par 2 Ο περιπατών εν τη ευθύτητι αυτού φοβείται τον Κύριον· ο δε σκολιός τας οδούς αυτού καταφρονεί αυτόν.
\par 3 Εν στόματι άφρονος είναι η ράβδος της υπερηφανίας· τα δε χείλη των σοφών θέλουσι φυλάττει αυτούς.
\par 4 Όπου δεν είναι βόες, η αποθήκη είναι κενή· η δε αφθονία των γεννημάτων είναι εκ της δυνάμεως του βοός.
\par 5 Ο αληθής μάρτυς δεν θέλει ψεύδεσθαι· ο δε ψευδής μάρτυς εκχέει ψεύδη.
\par 6 Ο χλευαστής ζητεί σοφίαν και δεν ευρίσκει· εις δε τον συνετόν είναι εύκολος η μάθησις.
\par 7 Ύπαγε κατέναντι του άφρονος ανθρώπου και δεν θέλεις ευρεί χείλη συνέσεως.
\par 8 Η σοφία του φρονίμου είναι να γνωρίζη την οδόν αυτού· η δε μωρία των αφρόνων αποπλάνησις.
\par 9 Οι άφρονες γελώσιν εις την ανομίαν· εν μέσω δε των ευθέων είναι χάρις.
\par 10 Η καρδία του ανθρώπου γνωρίζει την πικρίαν της ψυχής αυτού· και ξένος δεν συμμετέχει της χαράς αυτής.
\par 11 Η οικία των ασεβών θέλει αφανισθή· η δε σκηνή των ευθέων θέλει ανθεί.
\par 12 Υπάρχει οδός, ήτις φαίνεται ορθή εις τον άνθρωπον, αλλά τα τέλη αυτής φέρουσιν εις θάνατον.
\par 13 Έτι και εις τον γέλωτα πονεί η καρδία· και το τέλος της χαράς είναι λύπη.
\par 14 Ο διεφθαρμένος την καρδίαν θέλει εμπλησθή από των οδών αυτού· ο δε αγαθός άνθρωπος αφ' εαυτού.
\par 15 Ο απλούς πιστεύει εις πάντα λόγον· ο δε φρόνιμος προσέχει εις τα βήματα αυτού.
\par 16 Ο σοφός φοβείται και φεύγει από του κακού· αλλ' ο άφρων προχωρεί και θρασύνεται.
\par 17 Ο οξύθυμος πράττει αστοχάστως· και ο κακόβουλος άνθρωπος είναι μισητός.
\par 18 Οι άφρονες κληρονομούσι μωρίαν· οι δε φρόνιμοι στεφανούνται σύνεσιν.
\par 19 Οι κακοί υποκλίνουσιν έμπροσθεν των αγαθών, και οι ασεβείς εις τας πύλας των δικαίων.
\par 20 Ο πτωχός μισείται και υπό του πλησίον αυτού· του δε πλουσίου οι φίλοι πολλοί.
\par 21 Ο καταφρονών τον πλησίον αυτού αμαρτάνει· ο δε ελεών τους πτωχούς είναι μακάριος.
\par 22 Δεν πλανώνται οι βουλευόμενοι κακόν; έλεος όμως και αλήθεια θέλει είσθαι εις τους βουλευομένους αγαθόν.
\par 23 Εν παντί κόπω υπάρχει κέρδος· η δε φλυαρία των χειλέων φέρει μόνον εις ένδειαν.
\par 24 Τα πλούτη των σοφών είναι στέφανος εις αυτούς· των δε αφρόνων η υπεροχή μωρία.
\par 25 Ο αληθής μάρτυς ελευθερόνει ψυχάς· ο δε δόλιος εκχέει ψεύδη.
\par 26 Εν τω φόβω του Κυρίου είναι ελπίς ισχυρά· και εις τα τέκνα αυτού θέλει υπάρχει καταφύγιον.
\par 27 Ο φόβος του Κυρίου είναι πηγή ζωής, απομακρύνων από παγίδων θανάτου.
\par 28 Εν τω πλήθει του λαού είναι η δόξα του βασιλέως· εν δε τη ελλείψει του λαού ο αφανισμός του ηγεμονεύοντος.
\par 29 Ο μακρόθυμος έχει μεγάλην φρόνησιν· ο δε οξύθυμος ανεγείρει την αφροσύνην αυτού.
\par 30 Η υγιαίνουσα καρδία είναι ζωή της σαρκός· ο δε φθόνος σαπρία των οστέων.
\par 31 Ο καταθλίβων τον πένητα ονειδίζει τον Ποιητήν αυτού· ο δε τιμών αυτόν ελεεί τον πτωχόν.
\par 32 Ο ασεβής εκτινάσσεται εν τη ασεβεία αυτού· ο δε δίκαιος και εν τω θανάτω αυτού έχει ελπίδα.
\par 33 Εν τη καρδία του συνετού επαναπαύεται σοφία· εν μέσω δε των αφρόνων φανερούται.
\par 34 Η δικαιοσύνη υψόνει έθνος· η δε αμαρτία είναι όνειδος λαών.
\par 35 Εύνοια του βασιλέως είναι προς φρόνιμον δούλον· θυμός δε αυτού προς τον προξενούντα αισχύνην.

\chapter{15}

\par 1 Η γλυκεία απόκρισις καταπραΰνει θυμόν· αλλ' ο λυπηρός λόγος διεγείρει οργήν.
\par 2 Η γλώσσα των σοφών καλλωπίζει την γνώσιν· το στόμα δε των αφρόνων εξερεύγεται μωρίαν.
\par 3 Οι οφθαλμοί του Κυρίου είναι εν παντί τόπω, παρατηρούντες κακούς και αγαθούς.
\par 4 Η υγιαίνουσα γλώσσα είναι δένδρον ζωής· η δε διεστραμμένη, σύντριψις εις το πνεύμα.
\par 5 Ο άφρων καταφρονεί την διδασκαλίαν του πατρός αυτού· ο δε φυλάττων έλεγχον είναι φρόνιμος.
\par 6 Εν τω οίκω του δικαίου είναι θησαυρός πολύς· εις δε το εισόδημα του ασεβούς διασκορπισμός.
\par 7 Τα χείλη των σοφών διαδίδουσι γνώσιν· αλλ' η καρδία των αφρόνων δεν είναι ούτως.
\par 8 Η θυσία των ασεβών είναι βδέλυγμα εις τον Κύριον· αλλ' η δέησις των ευθέων ευπρόσδεκτος εις αυτόν.
\par 9 Βδέλυγμα είναι εις τον Κύριον η οδός του ασεβούς· αγαπά δε τον θηρεύοντα την δικαιοσύνην.
\par 10 Η διδασκαλία είναι δυσάρεστος εις τον εγκαταλείποντα την οδόν· ο μισών τον έλεγχον θέλει τελευτήσει.
\par 11 Ο άδης και η απώλεια είναι έμπροσθεν του Κυρίου· πόσω μάλλον αι καρδίαι των υιών των ανθρώπων;
\par 12 Ο χλευαστής δεν αγαπά τον ελέγχοντα αυτόν, ουδέ θέλει υπάγει προς τους σοφούς.
\par 13 Καρδία ευφραινομένη ιλαρύνει το πρόσωπον· υπό δε της λύπης της καρδίας καταθλίβεται το πνεύμα.
\par 14 Η καρδία του συνετού ζητεί γνώσιν· το δε στόμα των αφρόνων βόσκει μωρίαν.
\par 15 Πάσαι αι ημέραι του τεθλιμμένου είναι κακαί· ο δε ευφραινόμενος την καρδίαν έχει ευωχίαν παντοτεινήν.
\par 16 Καλήτερον το ολίγον εν φόβω Κυρίου, παρά θησαυροί πολλοί και ταραχή εν αυτοίς.
\par 17 Καλήτερον ξενισμός λαχάνων μετά αγάπης, παρά μόσχος σιτευτός μετά μίσους.
\par 18 Ο θυμώδης άνθρωπος διεγείρει μάχας· ο δε μακρόθυμος καταπαύει έριδας.
\par 19 Η οδός του οκνηρού είναι ως πεφραγμένη από ακάνθας· αλλ' η οδός των ευθέων είναι εξωμαλισμένη.
\par 20 Υιός σοφός ευφραίνει πατέρα· ο δε μωρός άνθρωπος καταφρονεί την μητέρα αυτού.
\par 21 Η μωρία είναι χαρά εις τον ενδεή φρενών· ο δε συνετός άνθρωπος περιπατεί ορθώς.
\par 22 Όπου συμβούλιον δεν υπάρχει, οι σκοποί ματαιόνονται· εν δε τω πλήθει των συμβούλων στερεόνονται.
\par 23 Χαρά εις τον άνθρωπον διά την απόκρισιν του στόματος αυτού, και λόγος εν καιρώ, πόσον καλός είναι.
\par 24 Η οδός της ζωής εις τον συνετόν είναι προς τα άνω, διά να εκκλίνη από του άδου κάτω.
\par 25 Ο Κύριος καταστρέφει τον οίκον των υπερηφάνων· στερεόνει δε το όριον της χήρας.
\par 26 Οι λογισμοί του πονηρού είναι βδέλυγμα εις τον Κύριον· των δε καθαρών οι λόγοι ευάρεστοι.
\par 27 Ο δωρολήπτης ταράττει τον οίκον αυτού· αλλ' όστις μισεί τα δώρα θέλει ζήσει.
\par 28 Η καρδία του δικαίου προμελετά διά να αποκριθή· το δε στόμα των ασεβών εξερεύγεται κακά.
\par 29 Ο Κύριος είναι μακράν από των ασεβών· εισακούει δε της δεήσεως των δικαίων.
\par 30 Το φως των οφθαλμών ευφραίνει την καρδίαν· και η καλή φήμη παχύνει τα οστά.
\par 31 Το ωτίον, το οποίον ακούει τον έλεγχον της ζωής, διατρίβει μεταξύ των σοφών.
\par 32 Όστις απωθεί την διδασκαλίαν, αποστρέφεται την ψυχήν αυτού· αλλ' όστις ακούει τον έλεγχον, αποκτά σύνεσιν.
\par 33 Ο φόβος του Κυρίου είναι διδασκαλία σοφίας· και η ταπείνωσις προπορεύεται της δόξης.

\chapter{16}

\par 1 Του ανθρώπου είναι αι προπαρασκευαί της καρδίας· παρά δε του Κυρίου η απόκρισις της γλώσσης.
\par 2 Πάσαι αι οδοί του ανθρώπου φαίνονται ορθαί εις τους οφθαλμούς αυτού· πλην ο Κύριος σταθμίζει τα πνεύματα.
\par 3 Αφιέρονε τα έργα σου εις τον Κύριον, και αι βουλαί σου θέλουσι στερεωθή.
\par 4 Ο Κύριος έκαμε τα πάντα δι' εαυτόν, έτι και τον ασεβή διά την ημέραν την κακήν.
\par 5 Βδέλυγμα εις τον Κύριον είναι πας υψηλοκάρδιος· και χειρ με χείρα αν συνάπτηται, δεν θέλει μένει ατιμώρητος.
\par 6 Διά χάριτος και αληθείας καθαρίζεται η ανομία· και διά του φόβου του Κυρίου εκκλίνουσιν οι άνθρωποι από του κακού.
\par 7 Όταν ο Κύριος αρέσκηται εις τας οδούς του ανθρώπου, και τους εχθρούς αυτού ειρηνεύει μετ' αυτού.
\par 8 Καλήτερον ολίγον μετά δικαιοσύνης, παρά εισοδήματα μεγάλα μετά αδικίας.
\par 9 Η καρδία του ανθρώπου βουλεύεται την οδόν αυτού· αλλ' ο Κύριος διευθύνει τα βήματα αυτού.
\par 10 Χρησμός είναι εις τα χείλη του βασιλέως· το στόμα αυτού δεν σφάλλει εν τη κρίσει.
\par 11 Δικαία στάθμη και πλάστιγξ είναι του Κυρίου· πάντα τα ζύγια του σακκίου είναι έργου αυτού.
\par 12 Βδέλυγμα είναι εις τους βασιλείς να πράττωσιν ανομίαν· διότι ο θρόνος στερεόνεται μετά της δικαιοσύνης.
\par 13 Τα δίκαια χείλη είναι ευπρόσδεκτα εις τους βασιλείς, και αγαπώσι τον λαλούντα ορθά.
\par 14 Θυμός βασιλέως είναι άγγελος θανάτου· αλλ' ο σοφός άνθρωπος καταπραΰνει αυτόν.
\par 15 Εις το φως του προσώπου του βασιλέως είναι ζωή· και η εύνοια αυτού είναι ως νέφος οψίμου βροχής.
\par 16 Πόσον καλητέρα είναι η απόκτησις της σοφίας παρά το χρυσίον και προκριτωτέρα η απόκτησις της συνέσεως παρά το αργύριον
\par 17 Η οδός των ευθέων είναι να εκκλίνωσιν από του κακού· όστις φυλάττει την οδόν αυτού, διατηρεί την ψυχήν αυτού.
\par 18 Η υπερηφανία προηγείται του ολέθρου, και υψηλοφροσύνη του πνεύματος προηγείται της πτώσεως.
\par 19 Καλήτερον να ήναι τις ταπεινόφρων μετά των ταπεινών, παρά να μοιράζη λάφυρα μετά των υπερηφάνων.
\par 20 Ο συνετός εις τα πράγματα θέλει ευρεί καλόν· και ο ελπίζων επί τον Κύριον είναι μακάριος.
\par 21 Ο σοφός την καρδίαν θέλει ονομάζεσθαι φρόνιμος· και η γλυκύτης των χειλέων προσθέτει μάθησιν.
\par 22 Η σύνεσις είναι πηγή ζωής εις τον έχοντα αυτήν· η δε παιδεία των αφρόνων μωρία.
\par 23 Η καρδία του σοφού συνετίζει το στόμα αυτού, και εις τα χείλη αυτού προσθέτει μάθησιν.
\par 24 Κηρήθρα μέλιτος οι ευάρεστοι λόγοι· γλυκύτης εις την ψυχήν και ίασις εις τα οστά.
\par 25 Υπάρχει οδός ήτις φαίνεται ορθή εις τον άνθρωπον, αλλά τα τέλη αυτής είναι οδοί θανάτου.
\par 26 Ο εργαζόμενος εργάζεται δι' εαυτόν· διότι το στόμα αυτού αναγκάζει αυτόν.
\par 27 Ο αχρείος άνθρωπος σκάπτει κακόν· και εις τα χείλη αυτού είναι ως πυρ καίον.
\par 28 Ο διεστραμμένος άνθρωπος διασπείρει έριδας· και ο ψιθυριστής διαχωρίζει τους στενωτέρους φίλους.
\par 29 Ο βίαιος άνθρωπος αποπλανά τον πλησίον αυτού και φέρει αυτόν εις οδόν ουχί καλήν·
\par 30 Κλείων τους οφθαλμούς αυτού μηχανάται διεστραμμένα· δαγκάνων τα χείλη αυτού εκτελεί το κακόν.
\par 31 Η πολιά είναι στέφανος δόξης, ευρισκομένη εν τη οδώ της δικαιοσύνης.
\par 32 Καλήτερος ο μακρόθυμος παρά τον δυνατόν· και ο εξουσιάζων το πνεύμα αυτού παρά τον εκπορθούντα πόλιν.
\par 33 Ο κλήρος ρίπτεται εις την κάλπην· όλη όμως η κρίσις αυτού είναι παρά Κυρίου.

\chapter{17}

\par 1 Καλήτερον ξηρόν ψωμίον και ειρήνη μετ' αυτού, παρά οίκον πλήρη θυμάτων μετά έριδος.
\par 2 Ο φρόνιμος υπηρέτης θέλει εξουσιάζει επί υιού αισχύνης και θέλει συμμοιρασθή την κληρονομίαν μεταξύ αδελφών.
\par 3 Το χωνευτήριον δοκιμάζει τον άργυρον και η κάμινος τον χρυσόν, ο δε Κύριος τας καρδίας.
\par 4 Ο κακοποιός υπακούει εις τα άνομα χείλη· ο ψεύστης δίδει ακρόασιν εις την κακήν γλώσσαν.
\par 5 Όστις περιγελά τον πτωχόν, ονειδίζει τον Ποιητήν αυτού· όστις χαίρει εις συμφοράς, δεν θέλει μείνει ατιμώρητος.
\par 6 Τέκνα τέκνων είναι ο στέφανος των γερόντων· και η δόξα των τέκνων οι πατέρες αυτών.
\par 7 Χείλη υπεροχής δεν αρμόζουσιν εις τον άφρονα· πολύ ολιγώτερον χείλη ψεύδους εις τον άρχοντα.
\par 8 Το δώρον είναι ως λίθος πολύτιμος εις τους οφθαλμούς του δωροδοκουμένου· όπου τούτο εμφανισθή, κατορθόνει.
\par 9 Όστις κρύπτει παράβασιν, ζητεί φιλίαν· αλλ' όστις επαναλέγει το πράγμα, χωρίζει τους στενωτέρους φίλους.
\par 10 Περισσότερον τύπτει ο έλεγχος τον φρόνιμον, παρά εκατόν μαστιγώσεις τον άφρονα.
\par 11 Ο κακός ζητεί μόνον στάσεις· διά τούτο άγγελος σκληρός θέλει πεμφθή κατ' αυτού.
\par 12 Ας απαντήση τον άνθρωπον άρκτος στερηθείσα των τέκνων αυτής και ουχί άφρων εν τη μωρία αυτού.
\par 13 Όστις αποδίδει κακόν αντί καλού, κακόν δεν θέλει αναχωρήσει από του οίκου αυτού.
\par 14 Όστις αρχίζει φιλονεικίαν, είναι ως ο εκφράττων ύδατα· όθεν παύσον από της φιλονεικίας πριν εξαφθή.
\par 15 Ο δικαιόνων τον ασεβή και ο καταδικάζων τον δίκαιον, αμφότεροι είναι βδέλυγμα εις τον Κύριον.
\par 16 Τι χρησιμεύουσι τα χρήματα εις την χείρα του άφρονος, διά να αγοράση σοφίαν, αφού δεν έχει γνώσιν;
\par 17 Εν παντί καιρώ αγαπά ο φίλος, και ο αδελφός γεννάται διά καιρόν ανάγκης.
\par 18 Άνθρωπος ενδεής φρενών δίδει χείρα και εγγυάται διά τον φίλον αυτού.
\par 19 Ο αγαπών έριδας αγαπά αμαρτήματα· ο υπερυψόνων την πύλην αυτού ζητεί όλεθρον.
\par 20 Ο σκολιός την καρδίαν δεν ευρίσκει καλόν· και ο διεστραμμένος την γλώσσαν αυτού πίπτει εις συμφοράν.
\par 21 Όστις γεννά άφρονα, γεννά αυτόν διά λύπην αυτού· και ο πατήρ του ανοήτου δεν απολαμβάνει χαράν.
\par 22 Η ευφραινομένη καρδία δίδει ευεξίαν ως ιατρικόν· το δε κατατεθλιμμένον πνεύμα ξηραίνει τα οστά.
\par 23 Ο ασεβής δέχεται δώρον από του κόλπου, διά να διαστρέψη τας οδούς της κρίσεως.
\par 24 Επί του προσώπου του συνετού είναι σοφία· αλλ' οι οφθαλμοί του άφρονος βλέπουσιν εις τα άκρα της γης.
\par 25 Ο άφρων υιός είναι βαρυθυμία εις τον πατέρα αυτού και πικρία εις την γεννήσασαν αυτόν.
\par 26 Δεν είναι ποτέ καλόν να επιβάλληται ποινή εις τον δίκαιον, να επιβουλεύηταί τις τους άρχοντας διά την ευθύτητα αυτών.
\par 27 Ο κρατών τους λόγους αυτού είναι γνωστικός· ο μακρόθυμος άνθρωπος είναι φρόνιμος.
\par 28 Και αυτός ο άφρων, όταν σιωπά λογίζεται σοφός· και ο κλείων τα χείλη αυτού, συνετός.

\chapter{18}

\par 1 Ο ιδιογνώμων ζητεί κατά την επιθυμίαν αυτού και εναντιόνεται εις παν ό,τι είναι ορθόν.
\par 2 Ο άφρων δεν ηδύνεται εις την σύνεσιν, αλλ' εις ό,τι φαντάζεται η καρδία αυτού.
\par 3 Όταν έρχηται ο ασεβής, έρχεται και η καταφρόνησις· και μετά του ονείδους, η ατιμία.
\par 4 Οι λόγοι του στόματος του ανθρώπου είναι ύδατα βαθέα· και η πηγή της σοφίας, χείμαρρος αναπηδών.
\par 5 Δεν είναι καλόν να προσωποληπτή τις τον ασεβή, διά να ανατρέπη το δίκαιον εν τη κρίσει.
\par 6 Τα χείλη του άφρονος εμβαίνουσιν εις έριδας, και το στόμα αυτού προσκαλεί ραπίσματα.
\par 7 Το στόμα του άφρονος είναι ο αφανισμός αυτού, και τα χείλη αυτού παγίς εις την ψυχήν αυτού.
\par 8 Οι λόγοι του ψιθυριστού καταπίνονται ηδέως και καταβαίνουσιν έως των ενδομύχων της κοιλίας.
\par 9 Ο οκνηρός εις το έργον αυτού είναι βεβαίως αδελφός του ασώτου.
\par 10 Το όνομα του Κυρίου είναι πύργος οχυρός· ο δίκαιος, καταφεύγων εις αυτόν, είναι εν ασφαλεία.
\par 11 Τα αγαθά του πλουσίου είναι η οχυρά αυτού πόλις, και φαντάζεται αυτά ως υψηλόν τείχος.
\par 12 Προ του αφανισμού υψόνεται η καρδία του ανθρώπου· και η ταπείνωσις προπορεύεται της δόξης.
\par 13 Το να αποκρίνηταί τις πριν ακούση, είναι εις αυτόν αφροσύνη και όνειδος.
\par 14 Το πνεύμα του ανθρώπου θέλει υποστηρίζει την αδυναμίαν αυτού· αλλά το κατατεθλιμμένον πνεύμα τις δύναται να υποφέρη;
\par 15 Η καρδία του φρονίμου αποκτά σύνεσιν· και το ωτίον των σοφών ζητεί γνώσιν.
\par 16 Το δώρον του ανθρώπου ανοίγει τόπον εις αυτόν, και φέρει αυτόν έμπροσθεν των μεγάλων.
\par 17 Ο πρωτολογών εν τη κρίσει αυτού φαίνεται δίκαιος· αλλ' ο αντίδικος αυτού έρχεται και εξελέγχει αυτόν.
\par 18 Ο κλήρος παύει τας αντιλογίας και αποφασίζει μεταξύ των δυνατών.
\par 19 Αδελφός δυσαρεστηθείς υποτάσσεται δυσκολώτερα παρά οχυρά πόλις· αι δε διαφοραί αυτών είναι ως μοχλοί φρουρίου.
\par 20 Εκ των καρπών του στόματος του ανθρώπου θέλει χορτασθή η κοιλία αυτού· από του προϊόντος των χειλέων αυτού θέλει εμπλησθή.
\par 21 Θάνατος και ζωή είναι εις την χείρα της γλώσσης· και οι αγαπώντες αυτήν θέλουσι φάγει τους καρπούς αυτής.
\par 22 Όστις εύρηκε γυναίκα, εύρηκεν αγαθόν και απήλαυσε χάριν παρά Κυρίου.
\par 23 Ο πένης λαλεί μετά ικεσιών· αλλ' ο πλούσιος αποκρίνεται μετά σκληρότητος.
\par 24 Ο άνθρωπος ο έχων φίλους πρέπει να φέρηται φιλικώς· και υπάρχει φίλος στενώτερος αδελφού.

\chapter{19}

\par 1 Καλήτερος ο πτωχός ο περιπατών εν τη ακεραιότητι αυτού, παρά τον πλούσιον τον διεστραμμένον τα χείλη αυτού και όντα άφρονα.
\par 2 Ψυχή άνευ γνώσεως βεβαίως δεν είναι καλόν· και όστις σπεύδει με τους πόδας, προσκόπτει.
\par 3 Η αφροσύνη του ανθρώπου διαστρέφει την οδόν αυτού· και η καρδία αυτού αγανακτεί κατά του Κυρίου.
\par 4 Ο πλούτος προσθέτει φίλους πολλούς· ο δε πτωχός εγκαταλείπεται υπό του φίλου αυτού.
\par 5 Ο ψευδής μάρτυς δεν θέλει μείνει ατιμώρητος· και ο λαλών ψεύδη δεν θέλει εκφύγει.
\par 6 Πολλοί κολακεύουσι το πρόσωπον του άρχοντος· και πας τις είναι φίλος του διδόντος ανθρώπου.
\par 7 Τον πτωχόν μισούσι πάντες οι αδελφοί αυτού· πόσω μάλλον θέλουσιν αποφεύγει αυτόν οι φίλοι αυτού; αυτός ακολουθεί φωνάζων· αλλ' εκείνοι δεν αποκρίνονται.
\par 8 Όστις αποκτά σοφίαν, αγαπά την ψυχήν αυτού· όστις φυλάττει φρόνησιν, θέλει ευρεί καλόν.
\par 9 Ο ψευδής μάρτυς δεν θέλει μείνει ατιμώρητος· και ο λαλών ψεύδη θέλει απολεσθή.
\par 10 Η τρυφή δεν αρμόζει εις άφρονα· πολύ ολιγώτερον εις δούλον, να εξουσιάζη επ' αρχόντων.
\par 11 Η φρόνησις του ανθρώπου συστέλλει τον θυμόν αυτού· και είναι δόξα αυτού να παραβλέπη την παράβασιν.
\par 12 Η οργή του βασιλέως είναι ως βρυχηθμός λέοντος· η δε εύνοια αυτού ως δρόσος επί τον χόρτον.
\par 13 Ο άφρων υιός είναι όλεθρος εις τον πατέρα αυτού· και αι έριδες της γυναικός είναι ακατάπαυστον στάξιμον.
\par 14 Οίκος και πλούτη κληρονομούνται εκ των πατέρων· αλλ' η φρόνιμος γυνή παρά Κυρίου δίδεται.
\par 15 Η οκνηρία ρίπτει εις βαθύν ύπνον· και η άεργος ψυχή θέλει πεινά.
\par 16 Ο φυλάττων την εντολήν φυλάττει την ψυχήν αυτού· ο δε καταφρονών τας οδούς αυτού θέλει απολεσθή.
\par 17 Ο ελεών πτωχόν δανείζει εις τον Κύριον· και θέλει γείνει εις αυτόν η ανταπόδοσις αυτού.
\par 18 Παίδευε τον υιόν σου ενόσω είναι ελπίς· αλλά μη διεγείρης την ψυχήν σου, ώστε να θανατώσης αυτόν.
\par 19 Ο οργίλος θέλει λάβει ποινήν· διότι και αν ελευθερώσης αυτόν, πάλιν θέλεις κάμει το αυτό.
\par 20 Άκουε συμβουλήν και δέχου διδασκαλίαν διά να γείνης σοφός εις τα έσχατά σου.
\par 21 Είναι πολλοί λογισμοί εν τη καρδία του ανθρώπου· η βουλή όμως του Κυρίου, εκείνη θέλει μένει.
\par 22 Τιμή του ανθρώπου είναι η αγαθότης αυτού· και καλήτερος ο πτωχός παρά τον ψεύστην.
\par 23 Ο φόβος του Κυρίου φέρει ζωήν, και ο φοβούμενος αυτόν θέλει πλαγιάζει κεχορτασμένος· κακόν δεν θέλει συναπαντήσει.
\par 24 Ο οκνηρός εμβάπτει την χείρα αυτού εις το τρυβλίον, και δεν θέλει ουδέ εις το στόμα αυτού να επιστρέψη αυτήν.
\par 25 Εάν μαστιγώσης τον χλευαστήν, ο απλούς θέλει γείνει προσεκτικός· και εάν ελέγξης τον φρόνιμον, θέλει εννοήσει γνώσιν.
\par 26 Όστις ατιμάζει τον πατέρα και απωθεί την μητέρα, είναι υιός προξενών αισχύνην και όνειδος.
\par 27 Παύσον, υιέ μου, να ακούης διδασκαλίαν παρεκτρέπουσαν από των λόγων της γνώσεως.
\par 28 Ο ασεβής μάρτυς χλευάζει το δίκαιον· και το στόμα των ασεβών καταπίνει ανομίαν.
\par 29 Κρίσεις ετοιμάζονται διά τους χλευαστάς, και ραβδισμοί διά την ράχιν των αφρόνων.

\chapter{20}

\par 1 Ο οίνος είναι χλευαστής, και τα σίκερα στασιαστικά· και όστις δελεάζεται υπό τούτων, δεν είναι φρόνιμος.
\par 2 Απειλή βασιλέως είναι βρυχηθμός λέοντος· όστις παροξύνει αυτόν, αμαρτάνει εις την ιδίαν αυτού ζωήν.
\par 3 Τιμή είναι εις τον άνθρωπον να παύη από της έριδος· πας δε άφρων εμπλέκεται εις ταύτην.
\par 4 Ο οκνηρός δεν θέλει να αροτριά εξ αιτίας του χειμώνος· διά τούτο θέλει ζητεί εν τω θέρει και δεν θέλει λαμβάνει.
\par 5 Η βουλή εν τη καρδία του ανθρώπου είναι ως ύδατα βαθέα· αλλ' ο συνετός άνθρωπος θέλει ανασύρει αυτήν.
\par 6 Πολλοί άνθρωποι κηρύττουσιν έκαστος την καλοκαγαθίαν αυτού· αλλά τις θέλει εύρη άνθρωπον πιστόν;
\par 7 Ο δίκαιος περιπατεί εν τη ακεραιότητι αυτού· και τα τέκνα αυτού είναι μακάρια μετ' αυτόν.
\par 8 Βασιλεύς, καθήμενος επί θρόνου κρίσεως, διασκεδάζει παν κακόν διά των οφθαλμών αυτού.
\par 9 Τις δύναται να είπη, Εκαθάρισα την καρδίαν μου, είμαι καθαρός από των αμαρτιών μου;
\par 10 Ζύγια διάφορα, μέτρα διάφορα, είναι αμφότερα βδέλυγμα εις τον Κύριον.
\par 11 Γνωρίζεται και αυτό το παιδίον εκ των πράξεων αυτού, αν τα έργα αυτού είναι καθαρά, και αν ευθέα.
\par 12 Το ωτίον ακούει και ο οφθαλμός βλέπει· αλλ' ο Κύριος έκαμεν αμφότερα.
\par 13 Μη αγάπα τον ύπνον, διά να μη έλθης εις πτωχείαν· άνοιξον τους οφθαλμούς σου και θέλεις χορτασθή άρτου.
\par 14 Κακόν, κακόν, λέγει ο αγοραστής· αλλ' αφού αναχωρήση, τότε καυχάται.
\par 15 Υπάρχει χρυσίον και πλήθος μαργαριτών· τα χείλη όμως της γνώσεως είναι το πολύτιμου κειμήλιον.
\par 16 Λάβε το ιμάτιον του εγγυωμένου διά ξένον· και λάβε ενέχυρον απ' αυτού, εγγυωμένου περί ξένων πραγμάτων.
\par 17 Ο άρτος του ψεύδους είναι γλυκύς εις τον άνθρωπον· μετά ταύτα όμως το στόμα αυτού θέλει γεμισθή χαλίκων.
\par 18 Οι σκοποί στερεόνονται διά της συμβουλής· και μετά καλήν σκέψιν κάμνε πόλεμον.
\par 19 Ο σπερμολόγος περιερχόμενος αποκαλύπτει τα μυστικά· διά τούτο μη σμίγου μετά του πλατύνοντος τα χείλη αυτού.
\par 20 Ο λύχνος του κακολογούντος τον πατέρα αυτού ή την μητέρα αυτού θέλει σβεσθή εν βαθεί σκότει.
\par 21 Κληρονομία αποκτηθείσα ταχέως την αρχήν, εις το τέλος δεν ευλογείται.
\par 22 Μη είπης, Θέλω ανταποδώσει κακόν· ανάμενε τον Κύριον και θέλει σε σώσει.
\par 23 Ζύγια διάφορα είναι βδέλυγμα εις τον Κύριον· και η δολία πλάστιγξ δεν είναι καλόν.
\par 24 Τα διαβήματα του ανθρώπου διευθύνονται υπό του Κυρίου· πως λοιπόν ο άνθρωπος ήθελε γνωρίσει την εαυτού οδόν;
\par 25 Παγίς είναι εις τον άνθρωπον, προπετώς να λαλή· περί ιερών και μετά τας ευχάς να σκέπτηται.
\par 26 Ο σοφός βασιλεύς λικμίζει τους ασεβείς και στρέφει τον τροχόν επ' αυτούς.
\par 27 Λύχνος του Κυρίου είναι το πνεύμα του ανθρώπου, το οποίον διερευνά πάντα τα ενδόμυχα της καρδίας.
\par 28 Έλεος και αλήθεια διαφυλάττουσι τον βασιλέα· και ο θρόνος αυτού υποστηρίζεται υπό του ελέους.
\par 29 Καύχημα των νέων είναι η δύναμις αυτών· και δόξα των γερόντων η πολιά.
\par 30 Τα μελανίσματα των πληγών λευκαίνουσι τον κακόν· και τα κτυπήματα τα ενδόμυχα της καρδίας.

\chapter{21}

\par 1 Η καρδία του βασιλέως είναι εν τη χειρί του Κυρίου ως ρεύματα υδάτων· όπου θέλει στρέφει αυτήν.
\par 2 Πάσαι αι οδοί του ανθρώπου φαίνονται ορθαί εις τους οφθαλμούς αυτού· πλην ο Κύριος σταθμίζει τας καρδίας.
\par 3 Να κάμνη τις δικαιοσύνην και κρίσιν είναι αρεστότερον εις τον Κύριον παρά θυσίαν.
\par 4 Το επηρμένον όμμα και η αλαζών καρδία, ο λύχνος των ασεβών, είναι αμαρτία.
\par 5 Οι λογισμοί του επιμελούς φέρουσι βεβαίως εις αφθονίαν· παντός δε προπετούς βεβαίως εις ένδειαν.
\par 6 Το αποκτάν θησαυρούς διά ψευδούς γλώσσης είναι ματαιότης άστατος των ζητούντων θάνατον.
\par 7 Αι αρπαγαί των ασεβών θέλουσιν εξολοθρεύσει αυτούς· διότι αρνούνται να πράττωσι το δίκαιον.
\par 8 Η οδός του διεφθαρμένου ανθρώπου είναι σκολιά· του δε καθαρού το έργον είναι ευθές.
\par 9 Καλήτερον να κατοική τις εν γωνία δώματος, παρά εν οίκω ευρυχώρω μετά γυναικός φιλέριδος.
\par 10 Η ψυχή του ασεβούς επιθυμεί κακόν· ο πλησίον αυτού δεν ευρίσκει χάριν εις τους οφθαλμούς αυτού.
\par 11 Όταν ο χλευαστής τιμωρήται, ο απλούς γίνεται σοφώτερος· και ο σοφός διδασκόμενος λαμβάνει γνώσιν.
\par 12 Ο δίκαιος συλλογίζεται την οικίαν του ασεβούς, όταν οι ασεβείς κατακρημνίζωνται εις την κακίαν αυτών.
\par 13 Όστις εμφράττει τα ώτα αυτού εις την κραυγήν του πτωχού, θέλει φωνάξει και αυτός και δεν θέλει εισακουσθή.
\par 14 Δώρον κρύφιον καταπραΰνει θυμόν· και χάρισμα εις τον κόλπον βαλλόμενον, οργήν ισχυράν.
\par 15 Χαρά είναι εις τον δίκαιον να κάμνη κρίσιν· όλεθρος δε εις τους εργάτας της ανομίας.
\par 16 Άνθρωπος αποπλανώμενος από της οδού της συνέσεως θέλει κατασκηνώσει εν τη συνάξει των τεθανατωμένων.
\par 17 Ο αγαπών ευθυμίαν θέλει κατασταθή πένης· ο αγαπών οίνον και μύρα δεν θέλει πλουτήσει.
\par 18 Ο ασεβής θέλει είσθαι αντίλυτρον του δικαίου, και των ευθέων ο παραβάτης.
\par 19 Καλήτερον να κατοική τις εν γη ερήμω παρά μετά γυναικός φιλέριδος και θυμώδους.
\par 20 Θησαυρός πολύτιμος και μύρα ευρίσκονται εν τω οίκω του σοφού· ο δε άφρων άνθρωπος καταδαπανά αυτά.
\par 21 Ο θηρεύων δικαιοσύνην και έλεος θέλει ευρεί ζωήν, δικαιοσύνην και δόξαν.
\par 22 Ο σοφός εκπορθεί την πόλιν των δυνατών και καταβάλλει το οχύρωμα του θάρρους αυτής.
\par 23 Όστις φυλάττει το στόμα αυτού και την γλώσσαν αυτού, φυλάττει την ψυχήν αυτού από στενοχωριών.
\par 24 Υπερήφανος και αλαζών χλευαστής καλείται, όστις πράττει μετά θυμού αλαζονείας.
\par 25 Αι επιθυμίαι του οκνηρού θανατόνουσιν αυτόν· διότι αι χείρες αυτού δεν θέλουσι να εργάζωνται·
\par 26 επιθυμεί όλην την ημέραν επιθυμίας· ο δε δίκαιος δίδει και δεν φείδεται.
\par 27 Η θυσία των ασεβών είναι βδέλυγμα· πολλώ μάλλον όταν προσφέρωσιν αυτήν μετά πονηρίας.
\par 28 Ο ψευδής μάρτυς θέλει απολεσθή· ο δε άνθρωπος όστις υπακούει θέλει λαλεί πάντοτε.
\par 29 Ο ασεβής άνθρωπος σκληρύνει το πρόσωπον αυτού· αλλ' ο ευθύς κατευθύνει τας οδούς αυτού.
\par 30 Δεν είναι σοφία ούτε σύνεσις ούτε βουλή εναντίον του Κυρίου.
\par 31 Ο ίππος ετοιμάζεται διά την ημέραν της μάχης· η σωτηρία όμως είναι παρά Κυρίου.

\chapter{22}

\par 1 Προτιμότερον όνομα καλόν παρά πλούτη μεγάλα, χάρις αγαθή παρά αργύριον και χρυσίον.
\par 2 Πλούσιος και πτωχός συναπαντώνται· ο Κύριος είναι ο Ποιητής αμφοτέρων τούτων.
\par 3 Ο φρόνιμος προβλέπει το κακόν και κρύπτεται· οι άφρονες όμως προχωρούσι και τιμωρούνται.
\par 4 Η αμοιβή της ταπεινώσεως και του φόβου του Κυρίου είναι πλούτος και δόξα και ζωή.
\par 5 Τρίβολοι και παγίδες είναι εν τη οδώ του σκολιού· όστις φυλάττει την ψυχήν αυτού, θέλει είσθαι μακράν απ' αυτών.
\par 6 Δίδαξον το παιδίον εν αρχή της οδού αυτού· και δεν θέλει απομακρυνθή απ' αυτής ουδέ όταν γηράση.
\par 7 Ο πλούσιος εξουσιάζει τους πτωχούς· και ο δανειζόμενος είναι δούλος του δανείζοντος.
\par 8 Ο σπείρων ανομίαν θέλει θερίσει συμφοράς· και η ράβδος της ύβρεως αυτού θέλει εκλείψει.
\par 9 Ο έχων όμμα αγαθόν θέλει ευλογηθή· διότι δίδει εκ του άρτου αυτού εις τον πτωχόν.
\par 10 Εκδίωξον τον χλευαστήν και θέλει συνεξέλθει η φιλονεικία, και η έρις και η ύβρις θέλουσι παύσει.
\par 11 Όστις αγαπά την καθαρότητα της καρδίας, διά την χάριν των χειλέων αυτού ο βασιλεύς θέλει είσθαι φίλος αυτού.
\par 12 Οι οφθαλμοί του Κυρίου περιφρουρούσι την γνώσιν· ανατρέπει δε τας υποθέσεις του παρανόμου.
\par 13 Ο οκνηρός λέγει, Λέων είναι έξω· εν τω μέσω των πλατειών θέλω φονευθή.
\par 14 Στόμα γυναικός αλλοτρίας είναι λάκκος βαθύς· ο μισούμενος υπό Κυρίου θέλει εμπέσει εις αυτόν.
\par 15 Η ανοησία είναι συνδεδεμένη μετά της καρδίας του παιδίου· η ράβδος της παιδείας θέλει αποχωρίσει αυτήν απ' αυτού.
\par 16 Όστις καταθλίβει τον πτωχόν διά να αυξήση τα πλούτη αυτού, και όστις δίδει εις τον πλούσιον, θέλει ελθεί βεβαίως εις ένδειαν.
\par 17 Κλίνον το ωτίον σου και άκουε τους λόγους των σοφών, και προσκόλλησον την καρδίαν σου εις την γνώσιν μου·
\par 18 διότι είναι τερπνοί, εάν φυλάττη αυτούς εν τη καρδία σου· και θέλουσι συναρμόζεσθαι ομού επί των χειλέων σου.
\par 19 Διά να ήναι το θάρρος σου επί τον Κύριον, εδίδαξα ταύτα εις σε την ημέραν ταύτην, μάλιστα εις σε.
\par 20 Δεν έγραψα εις σε πολλάκις διά συμβουλών και γνώσεων,
\par 21 διά να σε κάμω να γνωρίσης την βεβαιότητα των λόγων της αληθείας, ώστε να αποκρίνησαι λόγους αληθείας προς τους εξαποστέλλοντάς σε;
\par 22 Μη γυμνόνης τον πτωχόν, διότι είναι πτωχός· μηδέ κατάθλιβε εις την πύλην τον δυστυχούντα·
\par 23 διότι ο Κύριος θέλει εκδικάσει την δίκην αυτών· και θέλει γυμνώσει την ψυχήν των γυμνωσάντων αυτούς.
\par 24 Μη κάμνε φιλίαν μετά ανθρώπου θυμώδους· και μετά ανθρώπου οργίλου μη συμπεριπάτει·
\par 25 μήποτε μάθης τας οδούς αυτού, και λάβης παγίδα εις την ψυχήν σου.
\par 26 Μη έσο εκ των διδόντων χείρα, εκ των εγγυωμένων διά χρέη.
\par 27 Εάν δεν έχης πόθεν να πληρώσης, διά τι να πάρωσι την κλίνην σου υποκάτωθέν σου;
\par 28 Μη μετακίνει όρια αρχαία, τα οποία έθεσαν οι πατέρες σου.
\par 29 Είδες άνθρωπον επιτήδειον εις τα έργα αυτού; αυτός θέλει παρασταθή ενώπιον βασιλέων· δεν θέλει παρασταθή ενώπιον ουτιδανών.

\chapter{23}

\par 1 Όταν καθήσης να φάγης μετά άρχοντος, παρατήρει επιμελώς τα παρατιθέμενα έμπροσθέν σου·
\par 2 και βάλε μάχαιραν εις τον λαιμόν σου, εάν ήσαι αδηφάγος·
\par 3 μη επιθύμει τα εδέσματα αυτού· διότι ταύτα είναι τροφή δολιότητος.
\par 4 Μη μερίμνα διά να γείνης πλούσιος· άπεχε από της σοφίας σου.
\par 5 Θέλεις επιστήσει τους οφθαλμούς σου εις το μη υπάρχον; διότι ο πλούτος κατασκευάζει βεβαίως εις εαυτόν πτέρυγας ως αετού και πετά προς τον ουρανόν.
\par 6 Μη τρώγε τον άρτον του φθονερού, μηδέ επιθύμει τα εδέσματα αυτού·
\par 7 διότι καθώς φρονεί εν τη ψυχή αυτού, τοιούτος είναι· φάγε και πίε, λέγει προς σέ· αλλ' η καρδία αυτού δεν είναι μετά σου.
\par 8 Το ψωμίον, το οποίον έφαγες, θέλεις εξεμέσει και θέλεις χάσει τας γλυκείας συνομιλίας σου.
\par 9 Μη λάλει εις τα ώτα του άφρονος· διότι θέλει καταφρονήσει την σοφίαν των λόγων σου.
\par 10 Μη μετακίνει όρια αρχαία· και μη εισέλθης εις τους αγρούς των ορφανών·
\par 11 διότι ο Λυτρωτής αυτών είναι ισχυρός· αυτός θέλει εκδικάσει την δίκην αυτών εναντίον σου.
\par 12 Προσκόλλησον την καρδίαν σου εις την παιδείαν και τα ώτα σου εις τους λόγους της γνώσεως.
\par 13 Μη φείδου να παιδεύης το παιδίον· διότι εάν κτυπήσης αυτό διά της ράβδου, δεν θέλει αποθάνει·
\par 14 συ κτυπών αυτό διά της ράβδου, θέλεις ελευθερώσει την ψυχήν αυτού εκ του άδου.
\par 15 Υιέ μου, εάν η καρδία σου γείνη σοφή, θέλει ευφραίνεσθαι και η καρδία εμού·
\par 16 και τα νεφρά μου θέλουσιν αγάλλεσθαι, όταν τα χείλη σου λαλώσιν ορθά.
\par 17 Ας μη ζηλεύη η καρδία σου τους αμαρτωλούς· αλλ' έσο εν τω φόβω του Κυρίου όλην την ημέραν·
\par 18 διότι βεβαίως είναι αμοιβή, και η ελπίς σου δεν θέλει εκκοπή.
\par 19 Άκουε συ, υιέ μου, και γίνου σοφός, και κατεύθυνε την καρδίαν σου εις την οδόν.
\par 20 Μη έσο μεταξύ οινοποτών, μεταξύ κρεοφάγων ασώτων·
\par 21 διότι ο μέθυσος και ο άσωτος θέλουσι πτωχεύσει· και ο υπνώδης θέλει ενδυθή ράκη.
\par 22 Υπάκουε εις τον πατέρα σου, όστις σε εγέννησε· και μη καταφρόνει την μητέρα σου, όταν γηράση.
\par 23 Αγόραζε την αλήθειαν και μη πώλει· την σοφίαν και την παιδείαν και την σύνεσιν.
\par 24 Ο πατήρ του δικαίου θέλει χαρή σφόδρα· και όστις γεννά σοφόν υιόν, θέλει ευφραίνεσθαι εις αυτόν.
\par 25 Ο πατήρ σου και η μήτηρ σου θέλουσιν ευφραίνεσθαι· μάλιστα εκείνη, ήτις σε εγέννησε, θέλει χαίρει.
\par 26 Υιέ μου, δος την καρδίαν σου εις εμέ, και ας προσέχωσιν οι οφθαλμοί σου εις τας οδούς μου·
\par 27 διότι η πόρνη είναι λάκκος βαθύς· και η αλλοτρία γυνή στενόν φρέαρ.
\par 28 Αυτή προσέτι ενεδρεύει ως ληστής και πληθύνει τους παραβάτας μεταξύ των ανθρώπων.
\par 29 Εις τίνα είναι ουαί; εις τίνα στεναγμοί; εις τίνα έριδες; εις τίνα ματαιολογίαι; εις τίνα κτυπήματα άνευ αιτίας; εις τίνα φλόγωσις οφθαλμών;
\par 30 Εις τους εγχρονίζοντας εν τω οίνω· εις εκείνους οίτινες διάγουσιν ανιχνεύοντες οινοποσίας.
\par 31 Μη θεώρει τον οίνον ότι κοκκινίζει, ότι δίδει το χρώμα αυτού εις το ποτήριον, ότι καταβαίνει ευαρέστως.
\par 32 Εν τω τέλει αυτού δάκνει ως όφις και κεντρόνει ως βασιλίσκος·
\par 33 οι οφθαλμοί σου θέλουσι κυττάξει αλλοτρίας γυναίκας, και η καρδία σου θέλει λαλήσει αισχρά·
\par 34 και θέλεις είσθαι ως κοιμώμενος εν μέσω θαλάσσης, και ως κοιτώμενος επί κορυφής, καταρτίου·
\par 35 με έτυπτον, θέλεις ειπεί, και δεν επόνεσα· με έδειραν, και δεν ησθάνθην· πότε θέλω εγερθή, διά να υπάγω να ζητήσω αυτόν πάλιν;

\chapter{24}

\par 1 Μη ζήλευε τους κακούς ανθρώπους, μηδέ επιθύμει να ήσαι μετ' αυτών·
\par 2 διότι η καρδία αυτών μελετά καταδυνάστευσιν, και τα χείλη αυτών λαλούσι κακουργίας.
\par 3 Διά της σοφίας οικοδομείται οίκος και διά της συνέσεως στερεόνεται.
\par 4 Και διά της γνώσεως τα ταμεία θέλουσι γεμισθή από παντός πολυτίμου και ευφροσύνου πλούτου.
\par 5 Ο σοφός άνθρωπος ισχύει, και ο άνθρωπος ο φρόνιμος αυξάνει δύναμιν.
\par 6 Διότι διά σοφών βουλών θέλεις κάμει τον πόλεμόν σου· εκ του πλήθους δε των συμβούλων προέρχεται σωτηρία.
\par 7 Η σοφία είναι παραπολύ υψηλή διά τον άφρονα· δεν θέλει ανοίξει το στόμα αυτού εν τη πύλη.
\par 8 Όστις μελετά να πράξη κακόν, θέλει ονομασθή ανήρ κακεντρεχής.
\par 9 Η μελέτη της αφροσύνης είναι αμαρτία· και ο χλευαστής βδέλυγμα εις τους ανθρώπους.
\par 10 Εάν μικροψυχήσης εν τη ημέρα της συμφοράς, μικρά είναι η δύναμίς σου.
\par 11 Ελευθέρονε τους συρομένους εις θάνατον, και μη αποσύρου από των όντων εις ακμήν σφαγής.
\par 12 Εάν είπης, Ιδού, ημείς δεν εξεύρομεν τούτο· δεν γνωρίζει ο σταθμίζων τας καρδίας; και ο φυλάττων την ψυχήν σου και αποδίδων εις έκαστον κατά τα έργα αυτού, δεν εξεύρει;
\par 13 Υιέ μου, φάγε μέλι, διότι είναι καλόν· και κηρήθραν, διότι είναι γλυκεία επί του ουρανίσκον σου·
\par 14 Τοιαύτη θέλει είσθαι εις την ψυχήν σου η γνώσις της σοφίας· όταν εύρης αυτήν, τότε θέλεις λάβει αμοιβήν, και η ελπίς σου δεν θέλει εκκοπή.
\par 15 Μη στήνε παγίδα, ω άνομε, κατά της κατοικίας του δικαίου· μη ταράξης τον τόπον της αναπαύσεως αυτού·
\par 16 διότι ο δίκαιος πίπτει επτάκις και σηκόνεται· αλλ' οι ασεβείς θέλουσι πέσει εις όλεθρον.
\par 17 Εις την πτώσιν του εχθρού σου μη χαρής· και εις το ολίσθημα αυτού ας μη ευφραίνεται η καρδία σου·
\par 18 Μήποτε ο Κύριος ίδη και φανή τούτο κακόν εις τους οφθαλμούς αυτού και μεταστρέψη τον θυμόν αυτού απ' αυτού.
\par 19 Μη αγανάκτει περί των πονηρευομένων· μη ζήλευε τους ασεβείς·
\par 20 διότι δεν θέλει έχει τέλος αγαθόν ο κακός· ο λύχνος των ασεβών θέλει σβεσθή.
\par 21 Υιέ μου, φοβού τον Κύριον και τον βασιλέα· και μη έχε συγκοινωνίαν μετά στασιαστών·
\par 22 διότι η συμφορά αυτών θέλει επέλθει εξαίφνης· και τις γνωρίζει αμφοτέρων τας τιμωρίας;
\par 23 Ταύτα προσέτι είναι διά τους σοφούς. Η προσωποληψία εν τη κρίσει δεν είναι καλόν.
\par 24 Τον λέγοντα προς τον ασεβή, Είσαι δίκαιος, τούτον οι λαοί θέλουσι καταρασθή και τα έθνη θέλουσι βδελύττεσθαι·
\par 25 αλλ' εις τους ελέγχοντας αυτόν θέλει είσθαι χάρις, και ευλογία αγαθών θέλει ελθεί επ' αυτούς.
\par 26 Όστις αποκρίνεται λόγους ορθούς, είναι ως ο φιλών τα χείλη.
\par 27 Διάταττε το έργον σου έξω και προετοίμαζε αυτό εις σεαυτόν εν τω αγρώ· και έπειτα οικοδόμησον τον οίκόν σου.
\par 28 Μη ήσο μάρτυς άδικος κατά του πλησίον σου, μηδέ απάτα διά των χειλέων σου.
\par 29 Μη είπης, Καθώς έκαμεν εις εμέ, ούτω θέλω κάμει εις αυτόν· θέλω αποδώσει εις τον άνθρωπον κατά το έργον αυτού.
\par 30 Διέβαινον διά του αγρού του οκνηρού και διά του αμπελώνος του ανθρώπου του ενδεούς φρενών·
\par 31 και ιδού, πανταχού είχον βλαστήσει άκανθαι· κνίδαι είχον σκεπάσει το πρόσωπον αυτού, και το λιθόφραγμα αυτού ήτο κατακεκρημνισμένον.
\par 32 Τότε εγώ θεωρήσας εσυλλογίσθην εν τη καρδία μου· είδον, και έλαβον διδασκαλίαν.
\par 33 Ολίγος ύπνος, ολίγος νυσταγμός, ολίγη συμπλοκή των χειρών εις τον ύπνον·
\par 34 έπειτα η πτωχεία σου έρχεται ως ταχυδρόμος, και η ένδειά σου ως ανήρ ένοπλος.

\chapter{25}

\par 1 Και αύται είναι παροιμίαι του Σολομώντος, τας οποίας συνέλεξαν οι άνθρωποι του Εζεκίου, βασιλέως του Ιούδα.
\par 2 Δόξα του Θεού είναι να καλύπτη το πράγμα· δόξα δε των βασιλέων να εξιχνιάζωσι το πράγμα.
\par 3 Ο ουρανός κατά το ύψος και η γη κατά το βάθος και η καρδία των βασιλέων είναι ανεξερεύνητα.
\par 4 Αφαίρεσον την σκωρίαν από του αργύρου, και σκεύος θέλει εξέλθει εις τον χρυσοχόον·
\par 5 αφαίρεσον τους ασεβείς απ' έμπροσθεν του βασιλέως, και ο θρόνος αυτού θέλει στερεωθή εν δικαιοσύνη.
\par 6 Μη αλαζονεύου έμπροσθεν του βασιλέως, και μη ίστασαι εν τω τόπω των μεγάλων·
\par 7 Διότι καλήτερον να σοι είπωσιν, Ανάβα εδώ, παρά να καταβιβασθής επί παρουσία του άρχοντος, τον οποίον είδον οι οφθαλμοί σου.
\par 8 Μη εξέλθης εις έριδα ταχέως· μήποτε εν τω τέλει απορήσης τι να κάμης, όταν ο πλησίον σου σε καταισχύνη.
\par 9 Εκδίκασον την δίκην σου μετά του πλησίον σου· και μη ανακάλυπτε το μυστικόν άλλου·
\par 10 Μήποτε ο ακούων σε ονειδίση και η καταισχύνη σου δεν εξαλειφθή.
\par 11 Λόγος λαληθείς πρεπόντως είναι μήλα χρυσά εις ποικίλματα αργυρά.
\par 12 Ως ενώτιον χρυσούν και στολίδιον καθαρού χρυσίου, είναι ο σοφός ο ελέγχων ωτίον υπήκοον.
\par 13 Ως το ψύχος της χιόνος εν καιρώ του θερισμού, ούτως είναι ο πιστός πρέσβυς εις τους αποστέλλοντας αυτόν· διότι αναπαύει την ψυχήν των κυρίων αυτού.
\par 14 Ο καυχώμενος εις δώρον ψευδές ομοιάζει σύννεφα και άνεμον χωρίς βροχής.
\par 15 Δι' υπομονής πείθεται ο ηγεμών· και η γλυκεία γλώσσα συντρίβει οστά.
\par 16 Εύρηκας μέλι; φάγε όσον σοι είναι αρκετόν, μήποτε υπερεμπλησθής απ' αυτού και εξεμέσης αυτό.
\par 17 Σπανίως βάλε τον πόδα σου εις τον οίκον του πλησίον σου, μήποτε σε βαρυνθή και σε μισήση.
\par 18 Ο άνθρωπος, όστις μαρτυρεί κατά του πλησίον αυτού μαρτυρίαν ψευδή, είναι ως ρόπαλον και μάχαιρα και βέλος οξύ.
\par 19 Πίστις προς άπιστον εν ημέρα συμφοράς είναι ως οδόντιον σεσηπός και πους εξηρθρωμένος.
\par 20 Ως ο εκδυόμενος ιμάτιον εν ημέρα ψύχους και το όξος επί νίτρον, ούτως είναι ο ψάλλων άσματα εις λελυπημένην καρδίαν.
\par 21 Εάν πεινά ο εχθρός σου, δος εις αυτόν άρτον να φάγη· και εάν διψά, πότισον αυτόν ύδωρ·
\par 22 διότι θέλεις σωρεύσει άνθρακας πυρός επί την κεφαλήν αυτού, και ο Κύριος θέλει σε ανταμείψει.
\par 23 Ο βορράς άνεμος εκδιώκει την βροχήν· το δε ωργισμένον πρόσωπον την υποψιθυρίζουσαν γλώσσαν.
\par 24 Καλήτερον να κατοική τις εν γωνία δώματος, παρά εν οίκω ευρυχώρω μετά γυναικός φιλέριδος.
\par 25 Ως ύδωρ ψυχρόν εις ψυχήν διψώσαν, ούτως είναι αγγελίαι αγαθαί από μακρυνής γης.
\par 26 Ο δίκαιος σφάλλων έμπροσθεν του ασεβούς είναι ως πηγή θολερά και βρύσις διαφθαρείσα.
\par 27 Καθώς δεν είναι καλόν να τρώγη τις πολύ μέλι, ούτω δεν είναι ένδοξον να ζητή την ιδίαν αυτού δόξαν.
\par 28 Όστις δεν κρατεί το πνεύμα αυτού, είναι ως πόλις κατηδαφισμένη και ατείχιστος.

\chapter{26}

\par 1 Καθώς η χιών εν τω θέρει και καθώς η βροχή εν τω θερισμώ, ούτως εις τον άφρονα η τιμή δεν αρμόζει.
\par 2 Ως περιφέρεται το στρουθίον, ως περιπετά η χελιδών, ούτως η άδικος κατάρα δεν θέλει επιφθάσει.
\par 3 Μάστιξ διά τον ίππον, κημός διά τον όνον, και ράβδος διά την ράχιν των αφρόνων.
\par 4 Μη αποκρίνου εις τον άφρονα κατά την αφροσύνην αυτού, διά να μη γείνης και συ όμοιος αυτού.
\par 5 Αποκρίνου εις τον άφρονα κατά την αφροσύνην αυτού, διά να μη ήναι σοφός εις τους οφθαλμούς αυτού.
\par 6 Όστις αποστέλλει μήνυμα διά χειρός του άφρονος, αποκόπτει τους πόδας αυτού και πίνει ζημίαν.
\par 7 Ως τα σκέλη του χωλού κρέμονται ανωφελή, ούτως είναι και παροιμία εν τω στόματι των αφρόνων.
\par 8 Ως ο δεσμεύων λίθον εις σφενδόνην, ούτως είναι όστις δίδει τιμήν εις τον άφρονα.
\par 9 Ως η άκανθα ωθουμένη εις την χείρα του μεθύσου, ούτως είναι η παροιμία εν τω στόματι των αφρόνων.
\par 10 Ο δυνάστης μιαίνει τα πάντα και μισθόνει τους άφρονας, μισθόνει και τους παραβάτας.
\par 11 Ως ο κύων επιστρέφει εις τον εμετόν αυτού, ούτως ο άφρων επαναλαμβάνει την αφροσύνην αυτού.
\par 12 Είδες άνθρωπον νομίζοντα εαυτόν σοφόν; μάλλον ελπίς είναι εκ του άφρονος παρά εξ αυτού.
\par 13 Ο οκνηρός λέγει, Λέων είναι εν τη οδώ, λέων εν ταις πλατείαις.
\par 14 Ως η θύρα περιστρέφεται επί τας στρόφιγγας αυτής, ούτως ο οκνηρός επί την κλίνην αυτού.
\par 15 Ο οκνηρός εμβάπτει την χείρα αυτού εις το τρυβλίον και βαρύνεται να επιστρέψη αυτήν εις το στόμα αυτού.
\par 16 Ο οκνηρός νομίζει εαυτόν σοφώτερον παρά επτά σοφούς γνωμοδότας.
\par 17 Όστις διαβαίνων ανακατόνεται εις έριδα μη ανήκουσαν εις αυτόν, ομοιάζει τον πιάνοντα κύνα από των ωτίων.
\par 18 Ως ο μανιακός όστις ρίπτει φλόγας, βέλη και θάνατον,
\par 19 ούτως είναι ο άνθρωπος, όστις απατά τον πλησίον αυτού και λέγει, δεν έκαμον εγώ παίζων;
\par 20 Όπου δεν είναι ξύλα, το πυρ σβύνεται· και όπου δεν είναι ψιθυριστής, η έρις ησυχάζει.
\par 21 Οι άνθρακες διά την ανθρακιάν και τα ξύλα διά το πυρ, και ο φίλερις άνθρωπος διά να εξάπτη έριδας.
\par 22 Οι λόγοι του ψιθυριστού καταπίνονται ηδέως, και καταβαίνουσιν εις τα ενδόμυχα της κοιλίας.
\par 23 Τα ένθερμα χείλη μετά πονηράς καρδίας είναι ως σκωρία αργύρου επικεχρισμένη επί πήλινον αγγείον.
\par 24 Όστις μισεί, υποκρίνεται με τα χείλη αυτού, και μηχανεύεται δόλον εν τη καρδία αυτού.
\par 25 Όταν ομιλή χαριέντως, μη πίστευε αυτόν· διότι έχει επτά βδελύγματα εν τη καρδία αυτού.
\par 26 Όστις σκεπάζει το μίσος διά δόλου, η πονηρία αυτού θέλει φανερωθή εν μέσω της συνάξεως.
\par 27 Όστις σκάπτει λάκκον, θέλει πέσει εις αυτόν· και ο λίθος θέλει επιστρέψει επί τον κυλίοντα αυτόν.
\par 28 Η ψευδής γλώσσα μισεί τους υπ' αυτής καταθλιβομένους· και το απατηλόν στόμα εργάζεται καταστροφήν.

\chapter{27}

\par 1 Μη καυχάσαι εις την αύριον ημέραν· διότι δεν εξεύρεις τι θέλει γεννήσει η ημέρα.
\par 2 Ας σε επαινή άλλος και μη το στόμα σου· ξένος, και μη τα χείλη σου.
\par 3 Βαρύς είναι ο λίθος και δυσβάστακτος η άμμος· αλλ' η οργή του άφρονος είναι βαρυτέρα των δύο.
\par 4 Ο θυμός είναι σκληρός και η οργή οξεία· αλλά τις δύναται να σταθή έμπροσθεν της ζηλοτυπίας;
\par 5 Ο φανερός έλεγχος είναι καλήτερος παρά κρυπτομένη αγάπη·
\par 6 πληγαί φίλου είναι πισταί· φιλήματα δε εχθρών πολυάριθμα.
\par 7 Κεχορτασμένη ψυχή αποστρέφεται την κηρήθραν· εις δε την πεινασμένην ψυχήν παν πικρόν φαίνεται γλυκύ.
\par 8 Ως το πτηνόν το αποπλανώμενόν από της φωλεάς αυτού, ούτως είναι ο άνθρωπος ο αποπλανώμενος από του τόπου αυτού.
\par 9 Τα μύρα και τα θυμιάματα ευφραίνουσι την καρδίαν, και η γλυκύτης του φίλου διά της εγκαρδίου συμβουλής.
\par 10 Τον φίλον σου και τον φίλον του πατρός σου μη εγκαταλίπης· εις δε τον οίκον του αδελφού σου μη εισέλθης εν τη ημέρα της συμφοράς σου· διότι καλήτερον είναι γείτων πλησίον παρά αδελφός μακράν.
\par 11 Υιέ μου, γίνου σοφός και εύφραινε την καρδίαν μου, διά να έχω τι να αποκρίνωμαι προς τον ονειδίζοντά με.
\par 12 Ο φρόνιμος προβλέπει το κακόν και κρύπτεται· οι άφρονες εξακολουθούσι και τιμωρούνται.
\par 13 Λάβε το ιμάτιον του εγγυωμένου διά ξένον και λάβε ενέχυρον απ' αυτού, εγγυωμένου περί ξένων πραγμάτων.
\par 14 Ο εγειρόμενος το πρωΐ και ευλογών μετά μεγάλης φωνής τον πλησίον αυτού θέλει λογισθή ως καταρώμενος αυτόν.
\par 15 Ακατάπαυστον στάξιμον εν ημέρα βροχερά, και φίλερις γυνή είναι όμοια·
\par 16 ο κρύπτων αυτήν κρύπτει τον άνεμον· και το μύρον εν τη δεξιά αυτού κρυπτόμενον φωνάζει.
\par 17 Ο σίδηρος ακονίζει τον σίδηρον· και ο άνθρωπος ακονίζει το πρόσωπον του φίλου αυτού.
\par 18 Ο φυλάττων την συκήν θέλει φάγει τον καρπόν αυτής· και ο φυλάττων τον κύριον αυτού θέλει τιμηθή.
\par 19 Καθώς εις το ύδωρ ανταποκρίνεται πρόσωπον εις πρόσωπον, ούτω καρδία ανθρώπου εις άνθρωπον.
\par 20 Ο άδης και η απώλεια δεν χορταίνουσι· και οι οφθαλμοί του ανθρώπου δεν χορταίνουσιν.
\par 21 Ο άργυρος δοκιμάζεται διά του χωνευτηρίου και ο χρυσός διά της καμίνου· ο δε άνθρωπος διά του στόματος των εγκωμιαζόντων αυτόν.
\par 22 Και αν κοπανίσης διά κοπάνου τον άφρονα εν ιγδίω μεταξύ σίτου κοπανιζομένου, η αφροσύνη αυτού δεν θέλει χωρισθή απ' αυτού.
\par 23 Πρόσεχε να γνωρίζης την κατάστασιν των ποιμνίων σου, και επιμελού καλώς τας αγέλας σου·
\par 24 Διότι ο πλούτος δεν μένει διαπαντός· ουδέ το διάδημα από γενεάς εις γενεάν.
\par 25 Ο χόρτος βλαστάνει και η χλόη αναφαίνεται, και τα χόρτα των ορέων συνάγονται.
\par 26 Τα αρνία είναι διά τα ενδύματά σου, και οι τράγοι διά την πληρωμήν του αγρού.
\par 27 Και θέλεις έχει άφθονον γάλα αιγών διά την τροφήν σου, διά την τροφήν του οίκου σου και την ζωήν των θεραπαινών σου.

\chapter{28}

\par 1 Οι ασεβείς φεύγουσιν ουδενός διώκοντος· οι δε δίκαιοι έχουσι θάρρος ως λέων.
\par 2 Διά τα αμαρτήματα του τόπου πολλοί είναι οι άρχοντες αυτού· δι' ανθρώπου όμως συνετού και νοήμονος το πολίτευμα αυτού θέλει διαρκεί.
\par 3 Πτωχός άνθρωπος και δυναστεύων πτωχούς είναι ως βροχή κατακλύζουσα, ήτις δεν δίδει άρτον.
\par 4 Όσοι εγκαταλείπουσι τον νόμον, εγκωμιάζουσι τους ασεβείς· αλλ' οι φυλάττοντες τον νόμον αντιμάχονται εις αυτούς.
\par 5 Οι κακοί άνθρωποι δεν θέλουσι νοήσει κρίσιν· αλλ' οι ζητούντες τον Κύριον θέλουσι νοήσει τα πάντα.
\par 6 Καλήτερος ο πτωχός ο περιπατών εν τη ακεραιότητι αυτού, παρά τον διεστραμμένον τας οδούς αυτού, και αν ήναι πλούσιος.
\par 7 Ο φυλάττων τον νόμον είναι υιός συνετός· ο δε φίλος των ασώτων καταισχύνει τον πατέρα αυτού.
\par 8 Ο αυξάνων την περιουσίαν αυτού διά τόκου και πλεονεξίας συνάγει αυτήν διά τον ελεούντα τους πτωχούς.
\par 9 Του εκκλίνοντος το ωτίον αυτού από του να ακούη τον νόμον, και αυτή η προσευχή αυτού θέλει είσθαι βδέλυγμα.
\par 10 Ο αποπλανών τους ευθείς εις οδόν κακήν αυτός θέλει πέσει εις τον ίδιον αυτού λάκκον· αλλ' οι άμεμπτοι θέλουσι κληρονομήσει αγαθά.
\par 11 Ο πλούσιος άνθρωπος νομίζει εαυτόν σοφόν· αλλ' ο συνετός πτωχός εξελέγχει αυτόν.
\par 12 Όταν οι δίκαιοι θριαμβεύωσι, μεγάλη είναι η δόξα· αλλ' όταν οι ασεβείς υψόνωνται, οι άνθρωποι κρύπτονται.
\par 13 Ο κρύπτων τας αμαρτίας αυτού δεν θέλει ευοδωθή· ο δε εξομολογούμενος και παραιτών αυτάς θέλει ελεηθή.
\par 14 Μακάριος ο άνθρωπος ο φοβούμενος πάντοτε· όστις όμως σκληρύνει την καρδίαν αυτού, θέλει πέσει εις συμφοράν.
\par 15 Λέων βρυχώμενος και άρκτος πεινώσα είναι διοικητής ασεβής επί λαόν πενιχρόν.
\par 16 Ο ηγεμών ο στερούμενος συνέσεως πληθύνει τας καταδυναστείας· ο δε μισών την αρπαγήν θέλει μακρύνει τας ημέρας αυτού.
\par 17 Ο άνθρωπος ο ένοχος αίματος ανθρώπου θέλει σπεύσει εις τον λάκκον· ουδείς θέλει κρατήσει αυτόν.
\par 18 Ο περιπατών εν ακεραιότητι θέλει σωθή· ο δε διεστραμμένος εν ταις οδοίς αυτού θέλει πέσει διά μιας.
\par 19 Ο εργαζόμενος την γην αυτού θέλει χορτασθή άρτον· ο δε ακολουθών τους ματαιόφρονας θέλει εμπλησθή πτωχείας.
\par 20 Ο πιστός άνθρωπος θέλει έχει πολλήν ευλογίαν· αλλ' όστις σπεύδει να πλουτήση, δεν θέλει μείνει ατιμώρητος.
\par 21 Να ήναι τις προσωπολήπτης, δεν είναι καλόν· διότι ο τοιούτος άνθρωπος δι' εν κομμάτιον άρτου θέλει ανομήσει.
\par 22 Ο έχων πονηρόν οφθαλμόν σπεύδει να πλουτήση, και δεν καταλαμβάνει ότι η ένδεια θέλει ελθεί επ' αυτόν.
\par 23 Ο ελέγχων άνθρωπον ύστερον θέλει ευρεί περισσοτέραν χάριν, παρά τον κολακεύοντα διά της γλώσσης.
\par 24 Ο κλέπτων τον πατέρα αυτού ή την μητέρα αυτού, και λέγων, Τούτο δεν είναι αμαρτία, αυτός είναι σύντροφος του ληστού.
\par 25 Ο αλαζών την καρδίαν διεγείρει έριδας· ο δε θαρρών επί Κύριον θέλει παχυνθή.
\par 26 Ο θαρρών επί την ιδίαν αυτού καρδίαν είναι άφρων· αλλ' ο περιπατών εν σοφία, ούτος θέλει σωθή.
\par 27 Όστις δίδει εις τους πτωχούς, δεν θέλει ελθεί εις ένδειαν· αλλ' όστις αποστρέφει τους οφθαλμούς αυτού, θέλει έχει πολλάς κατάρας.
\par 28 Όταν οι ασεβείς υψόνωνται, οι άνθρωποι κρύπτονται· αλλ' εν τη απωλεία εκείνων οι δίκαιοι πληθύνονται.

\chapter{29}

\par 1 Άνθρωπος όστις ελεγχόμενος σκληρύνει τον τράχηλον, εξαίφνης θέλει αφανισθή και χωρίς ιάσεως.
\par 2 Όταν οι δίκαιοι μεγαλυνθώσιν, ο λαός ευφραίνεται· αλλ' όταν ο ασεβής εξουσιάζη, στενάζει ο λαός.
\par 3 Όστις αγαπά την σοφίαν, ευφραίνει τον πατέρα αυτού· αλλ' όστις συναναστρέφεται με πόρνας, φθείρει την περιουσίαν αυτού.
\par 4 Ο βασιλεύς διά της δικαιοσύνης στερεόνει τον τόπον· αλλ' ο δωρολήπτης καταστρέφει αυτόν.
\par 5 Ο άνθρωπος όστις κολακεύει τον πλησίον αυτού, εκτείνει δίκτυον έμπροσθεν των βημάτων αυτού.
\par 6 Ο κακός άνθρωπος παγιδεύεται εν τη ανομία· αλλ' ο δίκαιος ψάλλει και ευφραίνεται.
\par 7 Ο δίκαιος λαμβάνει γνώσιν της κρίσεως των πενήτων· ο ασεβής δεν νοεί γνώσιν.
\par 8 Οι χλευασταί άνθρωποι καταφλέγουσι την πόλιν· αλλ' οι σοφοί αποστρέφουσι την οργήν.
\par 9 Ο σοφός άνθρωπος, διαφερόμενος μετά του άφρονος ανθρώπου, είτε οργίζεται, είτε γελά, δεν ευρίσκει ανάπαυσιν.
\par 10 Οι άνδρες των αιμάτων μισούσι τον άμεμπτον· αλλ' οι ευθείς εκζητούσι την ζωήν αυτού.
\par 11 Ο άφρων εκθέτει όλην αυτού την ψυχήν· ο δε σοφός αναχαιτίζει αυτήν εις τα οπίσω.
\par 12 Εάν ο διοικητής προσέχη εις λόγους ψευδείς, πάντες οι υπηρέται αυτού γίνονται ασεβείς.
\par 13 Πένης και δανειστής συναπαντώνται· ο Κύριος φωτίζει αμφοτέρων τους οφθαλμούς.
\par 14 Βασιλέως κρίνοντος τους πτωχούς εν αληθεία, ο θρόνος αυτού θέλει στερεωθή διαπαντός.
\par 15 Η ράβδος και ο έλεγχος δίδουσι σοφίαν· παιδίον δε απολελυμένόν καταισχύνει την μητέρα αυτού.
\par 16 Όταν οι ασεβείς πληθύνωνται, η ανομία περισσεύει· αλλ' οι δίκαιοι θέλουσιν ιδεί την πτώσιν αυτών.
\par 17 Παίδευε τον υιόν σου και θέλει φέρει ανάπαυσιν εις σέ· και θέλει φέρει ηδονήν εις την ψυχήν σου.
\par 18 Όπου δεν υπάρχει όρασις, ο λαός διαφθείρεται· είναι δε μακάριος ο φυλάττων τον νόμον.
\par 19 Ο δούλος διά λόγων δεν θέλει διορθωθή· επειδή καταλαμβάνει μεν, αλλά δεν υπακούει.
\par 20 Είδες άνθρωπον ταχύν εις τους λόγους αυτού; περισσοτέρα ελπίς είναι εκ του άφρονος παρά εξ αυτού.
\par 21 Εάν τις ανατρέφη παιδιόθεν τον δούλον αυτού τρυφηλώς, τέλος πάντων θέλει κατασταθή υιός.
\par 22 Ο θυμώδης άνθρωπος εξάπτει έριδα, και ο οργίλος άνθρωπος πληθύνει ανομίας.
\par 23 Η υπερηφανία του ανθρώπου θέλει ταπεινώσει αυτόν· ο δε ταπεινόφρων απολαμβάνει τιμήν.
\par 24 Ο συμμεριστής του κλέπτου μισεί την εαυτού ψυχήν· ακούει τον όρκον και δεν ομολογεί.
\par 25 Ο φόβος του ανθρώπου στήνει παγίδα· ο δε πεποιθώς επί Κύριον θέλει είσθαι εν ασφαλεία.
\par 26 Πολλοί ζητούσι το πρόσωπον του ηγεμόνος· αλλ' η του ανθρώπου κρίσις είναι παρά Κυρίου.
\par 27 Ο άδικος άνθρωπος είναι βδέλυγμα εις τους δικαίους· και ο ευθύς εις την οδόν αυτού, βδέλυγμα εις τους ασεβείς.

\chapter{30}

\par 1 Οι λόγοι του Αγούρ, υιού του Ιακαί· τουτέστιν ο χρησμός, τον οποίον ο άνθρωπος ελάλησε προς τον Ιθιήλ, προς τον Ιθιήλ και τον Ούκαλ.
\par 2 Βεβαίως εγώ είμαι ο αφρονέστερος των ανθρώπων, και φρόνησις ανθρώπου δεν υπάρχει εν εμοί·
\par 3 και δεν έμαθον την σοφίαν, ούτε εξεύρω την γνώσιν των αγίων.
\par 4 Τις ανέβη εις τον ουρανόν και κατέβη; τις συνήγαγε τον άνεμον εν ταις χερσίν αυτού; τις εδέσμευσε τα ύδατα εν ιματίω; τις εστερέωσε πάντα τα άκρα της γης; τι το όνομα αυτού; και τι το όνομα του υιού αυτού, εάν εξεύρης;
\par 5 Πας λόγος Θεού είναι δεδοκιμασμένος· είναι ασπίς εις τους πεποιθότας επ' αυτόν.
\par 6 Μη προσθέσης εις τους λόγους αυτού· μήποτε σε εξελέγξη, και ευρεθής ψεύστης.
\par 7 Δύο ζητώ παρά σού· μη αρνηθής ταύτα εις εμέ πριν αποθάνω.
\par 8 Ματαιότητα και λόγον ψευδή απομάκρυνε απ' εμού· πτωχείαν και πλούτον μη δώσης εις εμέ· τρέφε με με αυτάρκη τροφήν.
\par 9 Μήποτε χορτασθώ και σε αρνηθώ και είπω, Τις είναι ο Κύριος; ή μήποτε ευρεθείς πτωχός κλέψω και λάβω το όνομα του Θεού μου επί ματαίω.
\par 10 Μη καταλάλει υπηρέτην προς τον κύριον αυτού· μήποτε σε καταρασθή και ευρεθής ένοχος.
\par 11 Υπάρχει γενεά, ήτις καταράται τον πατέρα αυτής και δεν ευλογεί την μητέρα αυτής·
\par 12 Υπάρχει γενεά καθαρά εις τους οφθαλμούς αυτής, αλλά δεν είναι πεπλυμένη από της ακαθαρσίας αυτής.
\par 13 Υπάρχει γενεά, της οποίας πόσον υψηλοί είναι οι οφθαλμοί και τα βλέφαρα αυτής επηρμένα.
\par 14 Υπάρχει γενεά, της οποίας οι οδόντες είναι ρομφαίαι και οι μυλόδοντες μάχαιραι, διά να κατατρώγωσι τους πτωχούς της γης και τους ενδεείς εκ μέσου των ανθρώπων.
\par 15 Η βδέλλα έχει δύο θυγατέρας, αίτινες φωνάζουσι, Φέρε, φέρε. Τα τρία ταύτα δεν χορταίνουσι ποτέ, μάλιστα τέσσαρα δεν λέγουσι ποτέ, Αρκεί.
\par 16 Ο άδης, και η στείρα μήτρα· η γη, ήτις δεν χορταίνει από ύδατος, και το πυρ, το οποίον δεν λέγει, Αρκεί.
\par 17 Τον οφθαλμόν, όστις εμπαίζει τον πατέρα αυτού και καταφρονεί να υπακούση εις την μητέρα αυτού, οι κόρακες της φάραγγος θέλουσιν εκβάλει και οι νεοσσοί των αετών θέλουσι φάγει.
\par 18 Τα τρία ταύτα είναι θαυμαστά εις εμέ, μάλιστα τέσσαρα δεν εννοώ·
\par 19 Τα ίχνη του αετού εις τον ουρανόν· τα ίχνη του όφεως επί του βράχου· τα ίχνη του πλοίου εν μέσω της θαλάσσης· και τα ίχνη του ανθρώπου εν τη νεότητι.
\par 20 Τοιαύτη είναι η οδός της μοιχαλίδος γυναικός· τρώγει και σπογγίζει το στόμα αυτής, και λέγει, Δεν έπραξα ανομίαν.
\par 21 Διά τρία η γη ταράττεται, μάλιστα διά τέσσαρα, τα οποία δεν δύναται να υποφέρη·
\par 22 Διά τον δούλον, όταν βασιλεύση· και τον άφρονα, όταν χορτασθή άρτον·
\par 23 διά την μισητήν γυναίκα, όταν υπανδρευθή· και την δούλην, όταν εκδιώξη την κυρίαν αυτής.
\par 24 Τα τέσσαρα ταύτα είναι ελάχιστα επί της γης, είναι όμως σοφώτατα·
\par 25 οι μύρμηκες, οίτινες είναι λαός αδύνατος αλλ' εν τω θέρει ετοιμάζουσι την τροφήν αυτών·
\par 26 οι χοιρογρύλλιοι, οίτινες είναι λαός ανίσχυρος αλλά κάμνουσι τους οίκους αυτών επί βράχου·
\par 27 αι ακρίδες, αίτινες δεν έχουσι βασιλέα αλλ' εκβαίνουσι πάσαι ομού κατά τάγματα·
\par 28 ο ασκάλαβος, όστις βαστάζεται εν ταις χερσίν αυτού, και διατρίβει εν τοις παλατίοις των βασιλέων.
\par 29 Τα τρία ταύτα βαδίζουσι καλώς, μάλιστα τέσσαρα περιπατούσιν ευπρεπώς·
\par 30 Ο λέων, όστις είναι ο ισχυρότερος των ζώων, και δεν στρέφει από προσώπου τινός·
\par 31 Ο αλέκτωρ, ο τράγος έτι· και ο βασιλεύς, περικεκυκλωμένος υπό του λαού αυτού.
\par 32 Εάν έπραξας αφρόνως υψόνων σεαυτόν, και εάν εβουλεύθης κακόν, βάλε χείρα επί στόματος.
\par 33 Διότι όστις κτυπά το γάλα, εκβάλλει βούτυρον· και όστις εκθλίβει την ρίνα, εκβάλλει αίμα· και όστις ερεθίζει οργήν, εξάγει μάχας.

\chapter{31}

\par 1 Οι λόγοι του βασιλέως Λεμουήλ, ο χρησμός, τον οποίον η μήτηρ αυτού εδίδαξεν αυτόν.
\par 2 Τι, υιέ μου; και τι, τέκνον της κοιλίας μου; και τι, υιέ των ευχών μου;
\par 3 Μη δώσης τας δυνάμεις σου εις τας γυναίκας, μηδέ τας οδούς σου εις τας αφανιστρίας των βασιλέων.
\par 4 Δεν είναι των βασιλέων, Λεμουήλ, δεν είναι των βασιλέων να πίνωσιν οίνον, ουδέ των ηγεμόνων, σίκερα·
\par 5 μήποτε πιόντες λησμονήσωσι τον νόμον και διαστρέψωσι την κρίσιν τινός τεθλιμμένου.
\par 6 Δίδετε σίκερα εις τους τεθλιμμένους, και οίνον εις τους πεπικραμένους την ψυχήν·
\par 7 διά να πίωσι και να λησμονήσωσι την πτωχείαν αυτών και να μη ενθυμώνται πλέον την δυστυχίαν αυτών.
\par 8 Άνοιγε το στόμα σου υπέρ του αφώνου, υπέρ της κρίσεως πάντων των εγκαταλελειμμένων.
\par 9 Άνοιγε το στόμα σου, κρίνε δικαίως, και υπερασπίζου τον πτωχόν και τον ενδεή.
\par 10 Γυναίκα ενάρετον τις θέλει ευρεί; διότι η τοιαύτη είναι πολύ τιμιωτέρα υπέρ τους μαργαρίτας.
\par 11 Η καρδία του ανδρός αυτής θαρρεί επ' αυτήν, και δεν θέλει στερείσθαι αφθονίας.
\par 12 Θέλει φέρει εις αυτόν καλόν και ουχί κακόν, πάσας τας ημέρας της ζωής αυτής.
\par 13 Ζητεί μαλλίον και λινάριον και εργάζεται ευχαρίστως με τας χείρας αυτής.
\par 14 Είναι ως τα πλοία των εμπόρων· φέρει την τροφήν αυτής από μακρόθεν.
\par 15 Και εγείρεται ενώ είναι έτι νυξ και δίδει τροφήν εις τον οίκον αυτής, και έργα εις τας θεραπαίνας αυτής.
\par 16 Θεωρεί αγρόν και αγοράζει αυτόν· εκ του καρπού των χειρών αυτής φυτεύει αμπελώνα.
\par 17 Ζώνει την οσφύν αυτής με δύναμιν, και ενισχύει τους βραχίονας αυτής.
\par 18 Αισθάνεται ότι το εμπόριον αυτής είναι καλόν· ο λύχνος αυτής δεν σβύνεται την νύκτα.
\par 19 Βάλλει τας χείρας αυτής εις το αδράκτιον και κρατεί εν τη χειρί αυτής την ηλακάτην.
\par 20 Ανοίγει την χείρα αυτής εις τους πτωχούς και εκτείνει τας χείρας αυτής προς τους ενδεείς.
\par 21 Δεν φοβείται την χιόνα διά τον οίκον αυτής· διότι πας ο οίκος αυτής είναι ενδεδυμένοι διπλά.
\par 22 Κάμνει εις εαυτήν σκεπάσματα· το ένδυμα αυτής είναι βύσσος και πορφύρα.
\par 23 Ο ανήρ αυτής γνωρίζεται εν ταις πύλαις, όταν κάθηται μεταξύ των πρεσβυτέρων του τόπου.
\par 24 Κάμνει λεπτόν πανίον και πωλεί· και δίδει ζώνας εις τους εμπόρους.
\par 25 Ισχύν και ευπρέπειαν είναι ενδεδυμένη· και ευφραίνεται διά τον μέλλοντα καιρόν.
\par 26 Ανοίγει το στόμα αυτής εν σοφία· και επί της γλώσσης αυτής είναι νόμος ευμενείας.
\par 27 Επαγρυπνεί εις την κυβέρνησιν του οίκου αυτής και άρτον οκνηρίας δεν τρώγει.
\par 28 Τα τέκνα αυτής σηκόνονται και μακαρίζουσιν αυτήν· ο ανήρ αυτής, και επαινεί αυτήν·
\par 29 Πολλαί θυγατέρες εφέρθησαν αξίως, αλλά συ υπερέβης πάσας.
\par 30 Ψευδής είναι η χάρις και μάταιον το κάλλος· η γυνή η φοβουμένη τον Κύριον, αυτή θέλει επαινείσθαι.
\par 31 Δότε εις αυτήν εκ του καρπού των χειρών αυτής· και τα έργα αυτής ας επαινώσιν αυτήν εν ταις πύλαις.


\end{document}