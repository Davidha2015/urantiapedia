\begin{document}

\title{Song of Solomon}


\chapter{1}

\par Το Άσμα των Ασμάτων, το του Σολομώντος.
\par 2 Ας με φιλήση με τα φιλήματα του στόματος αυτού. Διότι η αγάπη σου είναι καλητέρα παρά τον οίνον.
\par 3 Διά την ευωδίαν των καλών μύρων σου, το όνομά σου είναι μύρον εκκεχυμένον· διά τούτο αι νεάνιδες σε αγαπώσιν.
\par 4 Ελκυσόν με· θέλομεν δράμει κατόπιν σου· ο βασιλεύς με εισήγαγεν εις τα ταμεία αυτού· θέλομεν αγάλλεσθαι και ευφραίνεσθαι εις σε, θέλομεν ενθυμείσθαι την αγάπην σου μάλλον παρά οίνον· οι έχοντες ευθύτητα σε αγαπώσι.
\par 5 Μέλαινα είμαι, πλην εύχαρις, θυγατέρες της Ιερουσαλήμ· ως τα σκηνώματα του Κηδάρ, ως τα παραπετάσματα του Σολομώντος.
\par 6 Μη βλέπετε εις εμέ, ότι είμαι μεμελανωμένη, επειδή ο ήλιος με έκαυσεν· οι υιοί της μητρός μου ωργίσθησαν κατ' εμού· με έβαλον φύλακα εις τους αμπελώνας· τον ίδιόν μου αμπελώνα όμως δεν εφύλαξα.
\par 7 Απάγγειλόν μοι, συ, τον οποίον αγαπά η ψυχή μου, Που ποιμαίνεις, που αναπαύεις το ποίμνιον την μεσημβρίαν· διά τι να γείνω ως περικεκαλυμμένη μεταξύ των ποιμνίων των συντρόφων σου;
\par 8 Εάν δεν γνωρίζης τούτο αφ' εαυτής, ώραία μεταξύ των γυναικών, έξελθε συ κατόπιν εις τα ίχνη του ποιμνίου, και ποίμαινε τα ερίφιά σου πλησίον των σκηνών των βοσκών.
\par 9 Με τας ίππους των αμαξών του Φαραώ σε εξωμοίωσα, ηγαπημένη μου.
\par 10 Αι σιαγόνες σου είναι ώραίαι με τας σειράς των μαργαριτών, και ο τράχηλός σου με τα περιδέρραια.
\par 11 Θέλομεν κάμει εις σε αλύσεις χρυσάς με στίγματα αργυρίου.
\par 12 Ενόσω ο βασιλεύς κάθηται εις την τράπεζαν αυτού, ο νάρδος μου διαχέει την οσμήν αυτού.
\par 13 Δεμάτιον σμύρνης είναι εις εμέ ο αγαπητός μου· θέλει διανυκτερεύει μεταξύ των μαστών μου.
\par 14 Ο αγαπητός μου είναι εις εμέ ως βότρυς κύπρινος εις τους αμπελώνας του Εν-γαδδί.
\par 15 Ιδού, είσαι ώραία, αγαπητή μου· ιδού, είσαι ώραία· οι οφθαλμοί σου είναι ως περιστερών.
\par 16 Ιδού, είσαι ώραίος, αγαπητέ μου, ναι, εύχαρις· και η κλίνη ημών είναι ευθαλής.
\par 17 Αι δοκοί των οίκων ημών είναι κέδροι, τα σανιδώματα ημών εκ κυπαρίσσου.

\chapter{2}

\par Εγώ είμαι το άνθος του Σαρών και το κρίνον των κοιλάδων.
\par 2 Καθώς το κρίνον μεταξύ των ακανθών, ούτως είναι η αγαπητή μου μεταξύ των νεανίδων.
\par 3 Καθώς η μηλέα μεταξύ των δένδρων του δάσους, ούτως είναι ο αγαπητός μου μεταξύ των νεανίσκων· επεθύμησα την σκιάν αυτού και εκάθησα υπ' αυτήν, και ο καρπός αυτού ήτο γλυκύς εις τον ουρανίσκον μου.
\par 4 Με έφερεν εις τον οίκον του οίνου, και η σημαία αυτού επ' εμέ η αγάπη.
\par 5 Υποστηρίξατέ με με γλυκίσματα δυναμωτικά, αναψύξατέ με με μήλα· διότι είμαι τετρωμένη υπό αγάπης.
\par 6 Η αριστερά αυτού είναι υπό την κεφαλήν μου, και η δεξιά αυτού με εναγκαλίζεται.
\par 7 Σας ορκίζω, θυγατέρες Ιερουσαλήμ, εις τας δορκάδας και εις τας ελάφους του αγρού, να μη εξεγείρητε μηδέ να εξυπνήσητε την αγάπην μου, εωσού θελήση.
\par 8 Φωνή του αγαπητού μου Ιδού, αυτός έρχεται πηδών επί τα όρη, σκιρτών επί τους λόφους.
\par 9 Ο αγαπητός μου είναι όμοιος με δορκάδα ή με σκύμνον ελάφου· ιδού, ίσταται όπισθεν του τοίχου ημών, κυττάζει έξω διά των θυρίδων, προκύπτει διά των δικτυωτών.
\par 10 Αποκρίνεται ο αγαπητός μου και λέγει προς εμέ, Σηκώθητι, αγαπητή μου, ώραία μου, και ελθέ·
\par 11 Διότι ιδού, ο χειμών παρήλθεν, η βροχή διέβη, απήλθε·
\par 12 τα άνθη φαίνονται εν τη γή· ο καιρός του άσματος έφθασε, και η φωνή της τρυγόνος ηκούσθη εν τη γη ημών·
\par 13 η συκή εξέφερε τους ολύνθους αυτής, και αι άμπελοι με τα άνθη της σταφυλής διαδίδουσιν ευωδίαν· σηκώθητι, αγαπητή μου, ώραία μου, και ελθέ·
\par 14 Ω περιστερά μου, ήτις είσαι εν ταις σχισμαίς του βράχου, εν τοις αποκρύφοις των κρημνών, δείξόν μοι την όψιν σου, κάμε με να ακούσω την φωνήν σου· διότι η φωνή σου είναι γλυκεία και η όψις σου ώραία.
\par 15 Πιάσατε εις ημάς τας αλώπεκας, τας μικράς αλώπεκας, αίτινες αφανίζουσι τας αμπέλους· διότι αι άμπελοι ημών ανθούσιν.
\par 16 Ο αγαπητός μου είναι εις εμέ και εγώ εις αυτόν· ποιμαίνει μεταξύ των κρίνων.
\par 17 Εωσού πνεύση η αύρα της ημέρας και φύγωσιν αι σκιαί, επίστρεψον, αγαπητέ μου· γίνου όμοιος με δορκάδα ή με σκύμνον ελάφου επί τα όρη τα διεσχισμένα.

\chapter{3}

\par Την νύκτα επί της κλίνης μου εζήτησα εκείνον, τον οποίον αγαπά η ψυχή μου· εζήτησα αυτόν και δεν εύρηκα αυτόν.
\par 2 Θέλω σηκωθή τώρα και περιέλθει την πόλιν, εν ταις αγοραίς και εν ταις πλατείαις· θέλω ζητήσει εκείνον, τον οποίον αγαπά η ψυχή μου· εζήτησα αυτόν και δεν εύρηκα αυτόν.
\par 3 Με εύρηκαν οι φύλακες οι περιερχόμενοι την πόλιν. Μη είδετε εκείνον, τον οποίον αγαπά η ψυχή μου;
\par 4 Αφού ολίγον επέρασα απ' αυτών, εύρηκα εκείνον, τον οποίον αγαπά η ψυχή μου· επίασα αυτόν και δεν αφήκα αυτόν, εωσού εισήγαγον αυτόν εις τον οίκον της μητρός μου, και εις τον κοιτώνα της συλλαβούσης με.
\par 5 Σας ορκίζω, θυγατέρες Ιερουσαλήμ, εις τας δορκάδας και εις τας ελάφους του αγρού, να μη εξεγείρητε μηδέ να εξυπνήσητε την αγάπην μου, εωσού θελήση.
\par 6 Τις αύτη, η αναβαίνουσα από της ερήμου ως στύλοι καπνού, τεθυμιαμένη με σμύρναν και λίβανον, με πάσαν αρωματικήν σκόνην του μυρεψού;
\par 7 Ιδού, η κλίνη του Σολομώντος· εξήκοντα δυνατοί άνδρες είναι περί αυτήν, εκ των δυνατών του Ισραήλ·
\par 8 Πάντες ούτοι κρατούσι ρομφαίαν, δεδιδαγμένοι πόλεμον· έκαστος έχει την ρομφαίαν αυτού επί τον μηρόν αυτού διά νυκτερινούς φόβους.
\par 9 Ο βασιλεύς Σολομών έκαμεν εις εαυτόν φορείον εκ ξύλων του Λιβάνου·
\par 10 τους στύλους αυτού έκαμεν εξ αργύρου, το ανακλιντήριον αυτού εκ χρυσού, την στρωμνήν αυτού εκ πορφύρας· το μέσον αυτού ήτο εγκεκοσμημένον ερασμίως υπό των θυγατέρων της Ιερουσαλήμ.
\par 11 Εξέλθετε και ιδέτε, θυγατέρες Σιών, τον βασιλέα Σολομώντα εν τω διαδήματι, με το οποίον έστειλεν αυτόν η μήτηρ αυτού εν τη ημέρα της νυμφεύσεως αυτού και εν τη ημέρα της ευφροσύνης της καρδίας αυτού.

\chapter{4}

\par Ιδού, είσαι ώραία, αγαπητή μου· ιδού, είσαι ώραία· οι οφθαλμοί σου είναι ως περιστερών μεταξύ των πλοκάμων σου· τα μαλλία σου είναι ως ποίμνιον αιγών, καταβαινόντων από του όρους Γαλαάδ.
\par 2 Οι οδόντες σου είναι ως ποίμνιον προβάτων κεκουρευμένων, αναβαινόντων από της λούσεως, τα οποία πάντα γεννώσι δίδυμα, και δεν υπάρχει άτεκνον μεταξύ αυτών·
\par 3 τα χείλη σου ως ταινία ερυθρά, και η λαλιά σου εύχαρις· αι παρειαί σου ως τμήμα ροϊδίου μεταξύ των πλοκάμων σου·
\par 4 Ο τράχηλός σου ως ο πύργος του Δαβίδ, ο ωκοδομημένος διά οπλοθήκην, επί του οποίου κρέμανται χίλιοι θυρεοί, πάντες ασπίδες ισχυρών·
\par 5 οι δύο μαστοί σου ως δύο σκύμνοι δορκάδος δίδυμοι, βόσκοντες μεταξύ των κρίνων.
\par 6 Εωσού πνεύση η αύρα της ημέρας και φύγωσιν αι σκιαί, εγώ θέλω υπάγει εις το όρος της σμύρνης, και εις τον λόφον του θυμιάματος.
\par 7 Όλη ώραία είσαι, αγαπητή μου· και μώμος δεν υπάρχει εν σοι.
\par 8 Ελθέ μετ' εμού από του Λιβάνου, νύμφη από του Λιβάνου μετ' εμού· βλέψον από της κορυφής του Αμανά, από της κορυφής του Σενείρ και του Αερμών, από των φωλεών των λεόντων, από των ορέων των παρδάλεων.
\par 9 Έτρωσας την καρδίαν μου, αδελφή μου, νύμφη· έτρωσας την καρδίαν μου, δι' ενός των οφθαλμών σου, δι' ενός πλοκάμου του τραχήλου σου.
\par 10 Πόσον ώραία είναι η αγάπη σου, αδελφή μου, νύμφη πόσον καλητέρα η αγάπη σου παρά τον οίνον και η οσμή των μύρων σου παρά πάντα τα αρώματα
\par 11 Τα χείλη σου, νύμφη, στάζουσιν ως κηρήθρα· μέλι και γάλα είναι υπό την γλώσσαν σου· και η οσμή των ιματίων σου ως οσμή του Λιβάνου.
\par 12 Κήπος κεκλεισμένος είναι η αδελφή μου, η νύμφη μου· βρύσις κεκλεισμένη, πηγή εσφραγισμένη.
\par 13 Οι βλαστοί σου είναι παράδεισος ροϊδίων, μετά εκλεκτών καρπών· κύπρος μετά νάρδου·
\par 14 νάρδος και κρόκος· κάλαμος και κιννάμωμον, μετά πάντων των δένδρων του θυμιάματος· σμύρνα και αλόη, μετά πάντων των πρωτίστων αρωμάτων·
\par 15 πηγή κήπων, φρέαρ ύδατος ζώντος, και ρύακες από του Λιβάνου.
\par 16 Εγέρθητι, Βορρά· και έρχου, Νότε· πνεύσον εις τον κήπόν μου· διά να εκχυθώσι τα αρώματα αυτού. Ας έλθη ο αγαπητός μου εις τον κήπον αυτού, και ας φάγη τους εξαιρέτους καρπούς αυτού.

\chapter{5}

\par Ήλθον εις τον κήπόν μου, αδελφή μου, νύμφη· ετρύγησα την σμύρναν μου μετά των αρωμάτων μου· έφαγον την κηρήθραν μου μετά του μέλιτός μου· έπιον τον οίνον μου μετά του γάλακτός μου· Φάγετε, φίλοι· πίετε, ναι, πίετε αφθόνως, αγαπητοί.
\par 2 Εγώ κοιμώμαι, αλλ' η καρδία μου αγρυπνεί· φωνή του αγαπητού μου· κρούει· Ανοιξόν μοι, αδελφή μου, αγαπητή μου, περιστερά μου, αμώμητέ μου· διότι η κεφαλή μου εγέμισεν από δρόσου, οι βόστρυχοί μου από ψεκάδων της νυκτός.
\par 3 Εξεδύθην τον χιτώνα μου· πως να ενδυθώ αυτόν; ένιψα τους πόδας μου· πως θέλω μολύνει αυτούς;
\par 4 Ο αγαπητός μου εισήξε την χείρα αυτού διά της τρύπης της θύρας, και τα σπλάγχνα μου εταράχθησαν δι' αυτόν.
\par 5 Εγώ εσηκώθην διά να ανοίξω εις τον αγαπητόν μου· και αι χείρές μου έσταζον σμύρναν, και οι δάκτυλοί μου σμύρναν σταλακτήν, επί τας λαβάς του μοχλού.
\par 6 Εγώ ήνοιξα εις τον αγαπητόν μου· αλλ' ο αγαπητός μου εσύρθη, έφυγεν· η ψυχή μου ελιποθύμησεν εις τον λόγον αυτού· εζήτησα αυτόν και δεν εύρηκα αυτόν, εφώνησα αυτόν και δεν μοι απεκρίθη.
\par 7 Με εύρηκαν οι φύλακες οι περιερχόμενοι την πόλιν, με εκτύπησαν, με επλήγωσαν· οι φύλακες των τειχών αφήρεσαν απ' εμού το ιμάτιόν μου.
\par 8 Σας ορκίζω, θυγατέρες Ιερουσαλήμ, εάν εύρητε τον αγαπητόν μου, Τι θέλετε ειπεί προς αυτόν; Ότι είμαι τετρωμένη υπό αγάπης.
\par 9 Τι διαφέρει άλλου αγαπητού ο αγαπητός σου, ω ώραία μεταξύ των γυναικών; τι διαφέρει άλλου αγαπητού ο αγαπητός σου, και ώρκισας ημάς ούτως;
\par 10 Ο αγαπητός μου είναι λευκός και ερυθρός, διακρινόμενος μεταξύ μυριάδων·
\par 11 Η κεφαλή αυτού είναι χρυσίον δεδοκιμασμένον, οι πλόκαμοι αυτού κλάδοι φοινίκων, μέλανες ως κόραξ·
\par 12 οι οφθαλμοί αυτού ως περιστερών επί των ρυάκων των υδάτων, λελουμένοι εν γάλακτι, καθήμενοι ως λίθοι ενθέσεως·
\par 13 Αι σιαγόνες αυτού ως πρασιαί αρωμάτων, ως αλώνια φυτών μυρεψικών· τα χείλη αυτού ως κρίνα, στάζοντα σμύρναν σταλακτήν·
\par 14 Αι χείρες αυτού δακτυλίδια χρυσά, πεπληρωμένα με βηρύλλιον· η κοιλία αυτού ελεφάντινον τεχνούργημα, περικεκοσμημένον με σαπφείρους·
\par 15 αι κνήμαι αυτού στύλοι μαρμάρινοι, εστηριγμένοι επί βάσεων καθαρού χρυσίου· το είδος αυτού ως Λίβανος· έξοχος ως κέδροι.
\par 16 Ο ουρανίσκος αυτού είναι γλυκασμοί· και αυτός όλος επιθυμητός. Ούτος είναι ο αγαπητός μου, και ούτος ο φίλος μου, θυγατέρες Ιερουσαλήμ.

\chapter{6}

\par Που υπήγεν ο αγαπητός σου, ω ωραία μεταξύ των γυναικών; που εστράφη ο αγαπητός σου; και θέλομεν ζητήσει αυτόν μετά σου.
\par 2 Ο αγαπητός μου κατέβη εις τον κήπον αυτού, εις τας πρασιάς των αρωμάτων, διά να ποιμαίνη εν τοις κήποις και να συνάγη κρίνα.
\par 3 Εγώ είμαι του αγαπητού μου, και εμού ο αγαπητός μου· ποιμαίνει μεταξύ των κρίνων.
\par 4 Είσαι ώραία, αγαπητή μου, ως Θερσά, εύχαρις ως η Ιερουσαλήμ, τρομερά ως στράτευμα με σημαίας.
\par 5 Απόστρεψον τους οφθαλμούς σου απεναντίον μου, διότι με κατέπληξαν· τα μαλλία σου είναι ως ποίμνιον αιγών καταβαινόντων από Γαλαάδ.
\par 6 Οι οδόντες σου είναι ως ποίμνιον προβάτων, αναβαινόντων από της λούσεως, τα οποία πάντα γεννώσι δίδυμα, και δεν υπάρχει άτεκνον μεταξύ αυτών·
\par 7 αι παρειαί σου ως τμήμα ροϊδίου μεταξύ των πλοκάμων σου.
\par 8 Εξήκοντα βασίλισσαι είναι και ογδοήκοντα παλλακαί, και νεάνιδες αναρίθμητοι·
\par 9 μία είναι η περιστερά μου, η αμώμητός μου· αυτή είναι η μόνη της μητρός αυτής· είναι η εκλεκτή της τεκούσης αυτήν. Είδον αυτήν αι θυγατέρες και εμακάρισαν αυτήν· αι βασίλισσαι και αι παλλακαί, και επήνεσαν αυτήν.
\par 10 Τις αύτη, η προκύπτουσα ως αυγή, ώραία ως η σελήνη, λάμπουσα ως ο ήλιος, τρομερά ως στράτευμα με σημαίας;
\par 11 Κατέβην εις τον κήπον των καρυών διά να ίδω την χλόην της κοιλάδος, να ίδω εάν εβλάστησεν η άμπελος και εξήνθησαν αι ροϊδιαί.
\par 12 Χωρίς να αισθανθώ, η ψυχή μου με κατέστησεν ως τας αμάξας του Αμινναδίβ.
\par 13 Επίστρεψον, επίστρεψον, ω Σουλαμίτις· επίστρεψον, επίστρεψον, διά να σε θεωρήσωμεν. Τι θέλετε ιδεί εις την Σουλαμίτιν; Ως χορόν δύο στρατοπέδων;

\chapter{7}

\par Πόσον ώραία είναι τα βήματά σου με τα σανδάλια, θύγατερ του ηγεμόνος το τόρνευμα των μηρών σου είναι όμοιον με περιδέραιον, έργον χειρών καλλιτέχνου.
\par 2 Ο ομφαλός σου κρατήρ τορνευτός, πλήρης κεκερασμένου οίνου· η κοιλία σου θημωνία σίτου περιπεφραγμένη με κρίνους·
\par 3 οι δύο σου μαστοί ως δύο σκύμνοι δορκάδος δίδυμοι·
\par 4 ο τράχηλός σου ως πύργος ελεφάντινος· οι οφθαλμοί σου ως αι κολυμβήθραι εν Εσεβών, προς την πύλην Βαθ-ραββίμ· η μύτη σου ως ο πύργος του Λιβάνου, βλέπων προς την Δαμασκόν·
\par 5 Η κεφαλή σου επί σε ως Κάρμηλος, και η κόμη της κεφαλής σου ως πορφύρα· ο βασιλεύς είναι δεδεμένος εις τους πλοκάμους σου.
\par 6 Πόσον ώραία και πόσον επιθυμητή είσαι, αγαπητή, διά τας τρυφάς.
\par 7 Τούτο το ανάστημά σου ομοιάζει με φοίνικα, και οι μαστοί σου με βότρυας.
\par 8 Είπα, Θέλω αναβή εις τον φοίνικα, θέλω πιάσει τα βάϊα αυτού· και ιδού, οι μαστοί σου θέλουσιν είσθαι ως βότρυες της αμπέλου, και η οσμή της ρινός σου ως μήλα·
\par 9 και ο ουρανίσκος σου ως ο καλός οίνος ρέων ηδέως διά τον αγαπητόν μου, και κάμνων να λαλώσι τα χείλη των κοιμωμένων.
\par 10 Εγώ είμαι του αγαπητού μου, και η επιθυμία αυτού είναι προς εμέ.
\par 11 Ελθέ, αγαπητέ μου, ας εξέλθωμεν εις τον αγρόν· ας διανυκτερεύσωμεν εν ταις κώμαις.
\par 12 Ας εξημερωθώμεν εις τους αμπελώνας· ας ίδωμεν εάν εβλάστησεν η άμπελος, εάν ήνοιξε το άνθος της σταφυλής και εξήνθησαν αι ροϊδιαί· εκεί θέλω δώσει την αγάπην μου εις σε.
\par 13 Οι μανδραγόραι έδωκαν οσμήν, και εν ταις θύραις ημών είναι παν είδος καρπών αρεστών, νέων και παλαιών, τους οποίους εφύλαξα, αγαπητέ μου, διά σε.

\chapter{8}

\par Είθε να ήσο ως αδελφός μου, θηλάσας τους μαστούς της μητρός μου. Ευρίσκουσά σε έξω ήθελον σε φιλήσει, και δεν ήθελον με καταφρονήσει.
\par 2 Ήθελον σε σύρει και σε εισάξει εις τον οίκον της μητρός μου, διά να με διδάξης· ήθελον σε ποτίσει οίνον αρωματικόν και χυμόν του ροϊδίου μου.
\par 3 Η αριστερά αυτού ήθελεν είσθαι υπό την κεφαλήν μου, και η δεξιά αυτού ήθελε με εναγκαλισθή.
\par 4 Σας ορκίζω, θυγατέρες Ιερουσαλήμ, να μη εξεγείρητε μηδέ να εξυπνήσητε την αγάπην μου, εωσού θελήση.
\par 5 Τις αύτη η αναβαίνουσα από της ερήμου, επιστηριζομένη επί τον αγαπητόν αυτής; Εγώ σε εξύπνησα υπό την μηλέαν· εκεί σε εκοιλοπόνησεν η μήτηρ σου· εκεί σε εγέννησεν η τεκούσά σε.
\par 6 Θέσον με, ως σφραγίδα, επί την καρδίαν σου, ως σφραγίδα επί τον βραχίονά σου· διότι η αγάπη είναι ισχυρά ως ο θάνατος· η ζηλοτυπία σκληρά ως ο άδης· αι φλόγες αυτής φλόγες πυρός, ανάφλεξις ορμητικωτάτη.
\par 7 Ύδατα πολλά δεν δύνανται να σβέσωσι την αγάπην, ουδέ ποταμοί δύνανται να πνίξωσιν αυτήν· εάν τις δώση πάντα τα υπάρχοντα του οίκου αυτού διά την αγάπην, παντελώς θέλουσι καταφρονήσει αυτά.
\par 8 Ημείς έχομεν αδελφήν μικράν, και μαστούς δεν έχει· τι θέλομεν κάμει εις την αδελφήν ημών την ημέραν καθ' ην γείνη λόγος περί αυτής;
\par 9 Εάν ήναι τείχος, θέλομεν οικοδομήσει επ' αυτήν παλάτιον αργυρούν· και εάν ήναι θύρα, θέλομεν περιασφαλίσει αυτήν με σανίδας κεδρίνας.
\par 10 Εγώ είμαι τείχος, και οι μαστοί μου ως πύργοι· τότε ήμην εις τους οφθαλμούς αυτού ως ευρίσκουσα ειρήνην.
\par 11 Ο Σολομών είχεν αμπελώνα εν Βάαλ-χαμών· έδωκε τον αμπελώνα εις φύλακας· έκαστος έπρεπε να φέρη διά τον καρπόν αυτού χίλια αργύρια.
\par 12 Ο αμπελών εμού είναι έμπροσθέν μου· τα χίλια ας ήναι διά σε, Σολομών, και διακόσια διά τους φυλάττοντας τον καρπόν αυτού.
\par 13 Ω συ η καθημένη εν τοις κήποις, οι σύντροφοι προσέχουσιν εις την φωνήν σου· κάμε με να ακούσω αυτήν.
\par 14 Φεύγε, αγαπητέ μου, και γίνου όμοιος με δορκάδα ή με σκύμνον ελάφου επί τα όρη των αρωμάτων.


\end{document}