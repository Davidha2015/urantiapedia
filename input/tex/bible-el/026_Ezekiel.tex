\begin{document}

\title{Ezekiel}


\chapter{1}

\par 1 Εν τω τριακοστώ έτει, τω τετάρτω μηνί, τη πέμπτη του μηνός, ενώ ήμην μεταξύ των αιχμαλώτων παρά τον ποταμόν Χεβάρ, ηνοίχθησαν οι ουρανοί και είδον οράματα του Θεού·
\par 2 τη πέμπτη του μηνός του έτους τούτου, του πέμπτου της αιχμαλωσίας του βασιλέως Ιωαχείν,
\par 3 έγεινε ρητώς λόγος Κυρίου προς τον Ιεζεκιήλ υιόν του Βουζεί τον ιερέα, εν τη γη των Χαλδαίων παρά τον ποταμόν Χεβάρ, και εκεί εστάθη η χειρ του Κυρίου επ' αυτόν.
\par 4 Και είδον και ιδού, ανεμοστρόβιλος ήρχετο από βορρά, νέφος μέγα και πυρ συστρεφόμενον· πέριξ δε τούτου λάμψις και εκ μέσου αυτού εφαίνετο ως όψις ηλέκτρου, εκ μέσου του πυρός.
\par 5 Και εκ μέσου αυτού εφαίνετο τεσσάρων ζώων ομοίωμα. Και η θέα αυτών ήτο αύτη· είχον ομοίωμα ανθρώπου.
\par 6 Και έκαστον είχε τέσσαρα πρόσωπα και έκαστον αυτών είχε τέσσαρας πτέρυγας.
\par 7 Και οι πόδες αυτών ήσαν πόδες ορθοί, και το ίχνος του ποδός αυτών όμοιον με ίχνος ποδός μόσχου· και εσπινθηροβόλουν ως όψις χαλκού στίλβοντος.
\par 8 Και είχον χείρας ανθρώπου υποκάτωθεν των πτερύγων αυτών, εις τα τέσσαρα αυτών μέρη· και τα τέσσαρα είχον τα πρόσωπα αυτών και τας πτέρυγας αυτών.
\par 9 Αι πτέρυγες αυτών συνείχοντο η μία μετά της άλλης· δεν εστρέφοντο ενώ εβάδιζον· κατέναντι του προσώπου αυτών επορεύοντο έκαστον.
\par 10 Περί δε του ομοιώματος του προσώπου αυτών, τα τέσσαρα είχον πρόσωπον ανθρώπου, και πρόσωπον λέοντος κατά το δεξιόν μέρος· και τα τέσσαρα είχον πρόσωπον βοός κατά το αριστερόν· είχον και τα τέσσαρα πρόσωπον αετού.
\par 11 Και τα πρόσωπα αυτών και αι πτέρυγες αυτών ήσαν διηρημέναι προς τα άνω· δύο εκάστου συνείχοντο η μία μετά της άλλης και δύο εκάλυπτον τα σώματα αυτών.
\par 12 Και επορεύοντο έκαστον κατέναντι του προσώπου αυτού· όπου το πνεύμα εφέρετο, εκεί εβάδιζον· ενώ εβάδιζον, δεν εστρέφοντο.
\par 13 Περί δε του ομοιώματος των ζώων, η θέα αυτών ήτο ως καιόμενοι άνθρακες πυρός, ως θέα λαμπάδων· τούτο συνεστρέφετο μεταξύ των ζώων· και ήτο το πυρ λαμπρόν, και αστραπή εξήρχετο από του πυρός.
\par 14 Και τα ζώα έτρεχον και επέστρεφον ως η θέα της αστραπής.
\par 15 Και ως είδον τα ζώα, ιδού, τροχός εις επί την γην, πλησίον των ζώων εις τα τέσσαρα αυτών πρόσωπα.
\par 16 Η θέα των τροχών και η εργασία αυτών ήσαν ως όψις βηρύλλου· και οι τέσσαρες είχον το αυτό ομοίωμα· και η θέα αυτών και η εργασία αυτών ήσαν ως εάν ήτο τροχός εν μέσω τροχού.
\par 17 Ότε εβάδιζον, εκινούντο κατά τα τέσσαρα αυτών πλάγια· δεν εστρέφοντο ενώ εβάδιζον.
\par 18 Οι δε κύκλοι αυτών ήσαν τόσον υψηλοί, ώστε έκαμνον φόβον· και οι κύκλοι αυτών πλήρεις οφθαλμών κύκλω των τεσσάρων τούτων.
\par 19 Και ότε τα ζώα επορεύοντο, επορεύοντο οι τροχοί πλησίον αυτών· και ότε τα ζώα υψόνοντο από της γης, υψόνοντο και οι τροχοί.
\par 20 Όπου ήτο να υπάγη το πνεύμα, εκεί επορεύοντο, εκεί το πνεύμα ήτο να υπάγη· και οι τροχοί υψόνοντο απέναντι τούτων, διότι το πνεύμα των ζώων ήτο εν τοις τροχοίς.
\par 21 Ότε εκείνα επορεύοντο, επορεύοντο και ούτοι· και ότε εκείνα ίσταντο, ίσταντο και ούτοι· ότε δε εκείνα υψόνοντο από της γης, και οι τροχοί υψόνοντο απέναντι αυτών, διότι το πνεύμα των ζώων ήτο εν τοις τροχοίς.
\par 22 Και το ομοίωμα του στερεώματος του επάνωθεν της κεφαλής των ζώων ήτο ως όψις φοβερού κρυστάλλου, εξηπλωμένου υπέρ τας κεφαλάς αυτών.
\par 23 Υποκάτω δε του στερεώματος ήσαν εκτεταμέναι αι πτέρυγες αυτών, η μία προς την άλλην· δύο είχεν έκαστον, με τας οποίας εκάλυπτον τα σώματα αυτών.
\par 24 Και ότε επορεύοντο, ήκουον τον ήχον των πτερύγων αυτών, ως ήχον υδάτων πολλών, ως φωνήν του Παντοδυνάμου, και την φωνήν της λαλιάς ως φωνήν στρατοπέδου· ότε ίσταντο, κατεβίβαζον τας πτέρυγας αυτών.
\par 25 Και εγίνετο φωνή άνωθεν εκ του στερεώματος του υπέρ την κεφαλήν αυτών· ότε ίσταντο, κατεβίβαζον τας πτέρυγας αυτών.
\par 26 Υπεράνωθεν δε του στερεώματος του υπέρ την κεφαλήν αυτών εφαίνετο ομοίωμα θρόνου, ως θέα λίθου σαπφείρου· και επί του ομοιώματος του θρόνου ομοίωμα ως θέα ανθρώπου καθημένου επ' αυτόν άνωθεν.
\par 27 Και είδον ως όψιν ηλέκτρου, ως θέαν πυρός εν αυτώ κύκλω, από της θέας της οσφύος αυτού και επάνω· και από της θέας της οσφύος αυτού και κάτω είδον ως θέαν πυρός, και είχε λάμψιν κύκλω.
\par 28 Ως η θέα του τόξου, του γινομένον εν τη νεφέλη εν ημέρα βροχής, ούτως ήτο η θέα της λάμψεως κύκλω. Αύτη ήτο η θέα του ομοιώματος της δόξης του Κυρίου. Και ότε είδον, έπεσον επί πρόσωπόν μου και ήκουσα φωνήν λαλούντος.

\chapter{2}

\par 1 Και είπε προς εμέ, Υιέ ανθρώπου, στήθι επί τους πόδας σου, και θέλω λαλήσει προς σε.
\par 2 Και καθώς ελάλησε προς εμέ, εισήλθεν εις εμέ το πνεύμα και με έστησεν επί τους πόδας μου, και ήκουσα τον λαλούντα προς εμέ.
\par 3 Και είπε προς εμέ, Υιέ ανθρώπου, εγώ σε εξαποστέλλω προς τους υιούς Ισραήλ, προς έθνη αποστατικά, τα οποία απεστάτησαν απ' εμού· αυτοί και οι πατέρες αυτών εστάθησαν παραβάται εναντίον μου έως ταύτης της σήμερον ημέρας·
\par 4 και είναι υιοί σκληροπρόσωποι και σκληροκάρδιοι. Εγώ σε εξαποστέλλω προς αυτούς, και θέλεις ειπεί προς αυτούς, Ούτω λέγει Κύριος ο Θεός.
\par 5 Και εάν τε ακούσωσιν, εάν τε απειθήσωσι, διότι είναι οίκος αποστάτης, θέλουσιν όμως γνωρίσει ότι εστάθη προφήτης εν μέσω αυτών.
\par 6 Και συ, υιέ ανθρώπου, μη φοβηθής απ' αυτών και από των λόγων αυτών μη δειλιάσης, διότι είναι άκανθαι και σκόλοπες μετά σου, και κατοικείς μεταξύ σκορπίων· μη φοβηθής από των λόγων αυτών και από προσώπου αυτών μη τρομάξης, διότι είναι οίκος αποστάτης.
\par 7 Και θέλεις λαλήσει τους λόγους μου προς αυτούς, εάν τε ακούσωσιν, εάν τε απειθήσωσι· διότι είναι αποστάται.
\par 8 Συ όμως, υιέ ανθρώπου, άκουε τούτο, το οποίον εγώ λαλώ προς σέ· μη γείνης αποστάτης ως ο αποστάτης οίκος· άνοιξον το στόμα σου και φάγε τούτο, το οποίον εγώ δίδω εις σε.
\par 9 Και είδον και ιδού, χειρ εξηπλωμένη προς εμέ, και ιδού, εν αυτή τόμος βιβλίου.
\par 10 Και εξετύλιξεν αυτόν ενώπιόν μου· και ήτο γεγραμμένος έσωθεν και έξωθεν, και εν αυτώ γεγραμμένοι κλαυθμοί και θρηνωδίαι και ουαί.

\chapter{3}

\par 1 Και είπε προς εμέ, Υιέ ανθρώπου, φάγε τούτο, το οποίον ευρίσκεις· φάγε τούτον τον τόμον και ύπαγε να λαλήσης προς τον οίκον Ισραήλ.
\par 2 Και ήνοιξα το στόμα μου και με εψώμισε τον τόμον εκείνον.
\par 3 Και είπε προς εμέ, Υιέ ανθρώπου, ας φάγη η κοιλία σου και ας εμπλησθώσι τα εντόσθιά σου από του τόμου τούτου, τον οποίον εγώ δίδω εις σε. Και έφαγον και έγεινεν εν τω στόματί μου ως μέλι υπό της γλυκύτητος.
\par 4 Και είπε προς εμέ, Υιέ ανθρώπου, ύπαγε, είσελθε εις τον οίκον του Ισραήλ και λάλησον τους λόγους μου προς αυτούς.
\par 5 Διότι δεν εξαποστέλλεσαι προς λαόν βαθύχειλον και βαρύγλωσσον αλλά προς τον οίκον Ισραήλ·
\par 6 ουχί προς λαούς πολλούς βαθυχείλους και βαρυγλώσσους, των οποίων τους λόγους δεν εννοείς. Και προς τοιούτους εάν σε εξαπέστελλον, ούτοι ήθελον σου εισακούσει.
\par 7 Ο οίκος όμως Ισραήλ δεν θέλει να σου ακούση· διότι δεν θέλουσι να εισακούωσιν εμού· επειδή πας ο οίκος Ισραήλ είναι σκληρομέτωπος και σκληροκάρδιος.
\par 8 Ιδού, έκαμον το πρόσωπόν σου δυνατόν εναντίον των προσώπων αυτών και το μέτωπόν σου, δυνατόν εναντίον των μετώπων αυτών.
\par 9 Ως αδάμαντα σκληρότερον χάλικος έκαμον το μέτωπόν σου· μη φοβηθής αυτούς και μη τρομάξης από προσώπου αυτών, διότι είναι οίκος αποστάτης.
\par 10 Και είπε προς εμέ, Υιέ ανθρώπου, πάντας τους λόγους μου, τους οποίους θέλω λαλήσει προς σε, λάβε εν τη καρδία σου και άκουσον με τα ώτα σου.
\par 11 Και ύπαγε, είσελθε προς τους αιχμαλωτισθέντας, προς τους υιούς του λαού σου, και λάλησον προς αυτούς και ειπέ προς αυτούς, Ούτω λέγει Κύριος ο Θεός, εάν τε ακούσωσιν, εάν τε απειθήσωσι.
\par 12 Και με εσήκωσε το πνεύμα, και ήκουσα όπισθέν μου φωνήν μεγάλης συγκινήσεως λεγόντων, Ευλογημένη η δόξα του Κυρίου εκ του τόπου αυτού.
\par 13 Και ήκουσα τον ήχον των πτερύγων των ζώων, αίτινες συνείχοντο η μία μετά της άλλης, και τον ήχον των τροχών απέναντι τούτων και φωνήν μεγάλης συγκινήσεως.
\par 14 Και με ύψωσε το πνεύμα και με έλαβε και υπήγα εν πικρία και εν αγανακτήσει του πνεύματός μου· πλην η χειρ του Κυρίου ήτο κραταιά επ' εμέ.
\par 15 Και ήλθον προς τους μετοικισθέντας εις Τελαβίβ, τους κατοικούντας παρά τον ποταμόν Χεβάρ, και εκάθησα όπου εκείνοι εκάθηντο και παρέμεινα εκεί μεταξύ αυτών επτά ημέρας εκστατικός.
\par 16 Και μετά τας επτά ημέρας έγεινε λόγος Κυρίου προς εμέ, λέγων,
\par 17 Υιέ ανθρώπου, σε κατέστησα φύλακα επί τον οίκον Ισραήλ· άκουσον λοιπόν λόγον εκ του στόματός μου και νουθέτησον αυτούς παρ' εμού.
\par 18 Όταν λέγω προς τον άνομον, Εξάπαντος θέλεις θανατωθή, και συ δεν νουθετήσης αυτόν και δεν λαλήσης διά να αποτρέψης τον άνομον από της οδού αυτού της ανόμου, ώστε να σώσης την ζωήν αυτού, εκείνος μεν ο άνομος θέλει αποθάνει εν τη ανομία αυτού· πλην εκ της χειρός σου θέλω ζητήσει το αίμα αυτού.
\par 19 Αλλ' εάν συ μεν νουθετήσης τον άνομον, αυτός όμως δεν επιστρέφη από της ανομίας αυτού και από της οδού αυτού της ανόμου, εκείνος μεν θέλει αποθάνει εν τη ανομία αυτού, συ δε ηλευθέρωσας την ψυχήν σου.
\par 20 Πάλιν, εάν ο δίκαιος εκτραπή από της δικαιοσύνης αυτού και πράξη ανομίαν, και εγώ θέσω πρόσκομμα έμπροσθεν αυτού· εκείνος θέλει αποθάνει· επειδή δεν έδωκας εις αυτόν νουθεσίαν θέλει αποθάνει εν τη αμαρτία αυτού, και η δικαιοσύνη αυτού, την οποίαν έκαμε, δεν θέλει μνημονευθή· πλην εκ της χειρός σου θέλω ζητήσει το αίμα αυτού.
\par 21 Εάν όμως συ νουθετήσης τον δίκαιον διά να μη αμαρτήση και αυτός δεν αμαρτήση, ο δίκαιος θέλει βεβαίως ζήσει, διότι ενουθετήθη· και συ ηλευθέρωσας την ψυχήν σου.
\par 22 Και εστάθη εκεί η χειρ του Κυρίου επ' εμέ και είπε προς εμέ, Σηκώθητι, έξελθε εις την πεδιάδα και εκεί θέλω λαλήσει προς σε.
\par 23 Και εσηκώθην και εξήλθον εις την πεδιάδα και ιδού, η δόξα του Κυρίου ίστατο εκεί, ως η δόξα την οποίαν είδον παρά τον ποταμόν Χεβάρ· και έπεσον επί πρόσωπόν μου.
\par 24 Και εισήλθε το πνεύμα εις εμέ και με έστησεν επί τους πόδας μου και ελάλησε προς εμέ και μοι είπεν, Ύπαγε, κλείσθητι εντός της οικίας σου.
\par 25 Διότι, όσον περί σου, υιέ ανθρώπου, ιδού, θέλουσι βάλει επί σε δεσμά και θέλουσι σε δέσει με αυτά και δεν θέλεις εξέλθει εις το μέσον αυτών.
\par 26 Και την γλώσσαν σου θέλω κολλήσει προς τον λάρυγγά σου και θέλεις γείνει άλαλος· και δεν θέλεις είσθαι προς αυτούς ανήρ ελέγχων, διότι είναι οίκος αποστάτης.
\par 27 Πλην όταν λαλήσω προς σε, θέλω ανοίξει το στόμα σου και θέλεις ειπεί προς αυτούς, Ούτω λέγει Κύριος ο Θεός· Ο ακούων ας ακούη· και ο απειθών ας απειθή· διότι είναι οίκος αποστάτης.

\chapter{4}

\par 1 Και συ, υιέ ανθρώπου, λάβε εις σεαυτόν κεραμίδα και θες αυτήν έμπροσθέν σου και διάγραψον επ' αυτής πόλιν, την Ιερουσαλήμ·
\par 2 και στήσον πολιορκίαν εναντίον αυτής και οικοδόμησον προμαχώνας εναντίον αυτής και ύψωσον προχώματα εναντίον αυτής, θες έτι στρατόπεδον εναντίον αυτής και στήσον κριούς κύκλω εναντίον αυτής.
\par 3 Και λάβε εις σεαυτόν πλάκα σιδηράν και θες αυτήν ως τοίχον σιδηρούν μεταξύ σου και της πόλεως, και στήριξον το πρόσωπόν σου εναντίον αυτής και θέλει πολιορκηθή και θέλεις βάλει πολιορκίαν εναντίον αυτής· τούτο θέλει είσθαι σημείον εις τον οίκον Ισραήλ.
\par 4 Και συ πλαγίασον επί την αριστεράν σου πλευράν και θες την ανομίαν του οίκου Ισραήλ επ' αυτήν· κατά τον αριθμόν των ημερών, καθ' ας θέλεις πλαγιάσει επ' αυτήν, θέλεις βαστάσει την ανομίαν αυτών.
\par 5 Διότι εγώ επί σε έθεσα τα έτη της ανομίας αυτών κατά τον αριθμόν των ημερών, τριακοσίας ενενήκοντα ημέρας· και θέλεις βαστάσει την ανομίαν του οίκου Ισραήλ.
\par 6 Και αφού τελειώσης ταύτας, πλαγίασον πάλιν επί την πλευράν σου την δεξιάν, και θέλεις βαστάσει την ανομίαν του οίκου Ιούδα τεσσαράκοντα ημέρας· εκάστην μίαν ημέραν προσδιώρισα εις σε αντί ενός έτους.
\par 7 Και θέλεις στηρίξει το πρόσωπόν σου προς την πολιορκίαν της Ιερουσαλήμ και ο βραχίων σου θέλει είσθαι γυμνός και θέλεις προφητεύσει εναντίον αυτής.
\par 8 Και ιδού, θέλω βάλει επί σε δεσμά, και δεν θέλεις στραφή από της μιας σου πλευράς εις την άλλην, εωσού τελειώσης τας ημέρας της πολιορκίας σου.
\par 9 Και συ λάβε εις σεαυτόν σίτον και κριθήν και κυάμους και φακήν και κέγχρον και άρακον, και θες αυτά εις εν αγγείον και κάμε εξ αυτών άρτους εις σεαυτόν· κατά τον αριθμόν των ημερών καθ' ας θέλεις πλαγιάσει επί την πλευράν σου, τριακοσίας και ενενήκοντα ημέρας θέλεις τρώγει εκ τούτων.
\par 10 Και το φαγητόν σου, το οποίον θέλεις τρώγει εκ τούτων, θέλει είσθαι με ζύγιον, είκοσι σίκλων την ημέραν· από καιρού έως καιρού θέλεις τρώγει εξ αυτών.
\par 11 Και ύδωρ με μέτρον θέλεις πίνει, το έκτον ενός ίν· από καιρού έως καιρού θέλεις πίνει.
\par 12 Και θέλεις τρώγει αυτούς ως κριθίνους εγκρυφίας, και θέλεις ψήνει αυτούς με κόπρον εξερχομένην από ανθρώπου, έμπροσθεν των οφθαλμών αυτών.
\par 13 Και είπε Κύριος, Ούτω θέλουσι φάγει οι υιοί Ισραήλ τον άρτον αυτών μεμολυσμένον μεταξύ των εθνών, όπου θέλω διασκορπίσει αυτούς.
\par 14 Και εγώ είπα, Α, Κύριε Θεέ· ιδού, ψυχή μου δεν εμολύνθη, επειδή από νεότητός μου έως του νυν δεν έφαγον θνησιμαίον ή θηριάλωτον, ουδέ εισήλθε ποτέ εις το στόμα μου κρέας βδελυκτόν.
\par 15 Και είπε προς εμέ, Ιδέ, έδωκα εις σε κόπρον βοός αντί κόπρου ανθρωπίνης, και με ταύτην θέλεις ψήσει τον άρτον σου.
\par 16 Και είπε προς εμέ, Υιέ ανθρώπου, ιδού, εγώ θέλω συντρίψει το υποστήριγμα του άρτου εν Ιερουσαλήμ· και θέλουσι τρώγει άρτον με ζύγιον και εν στενοχωρία, και θέλουσι πίνει ύδωρ με μέτρον και εν αγωνία·
\par 17 διά να καταντήσωσιν ενδεείς άρτου και ύδατος· και θέλουσιν εκθαμβείσθαι προς αλλήλους, και θέλουσιν αναλωθή διά τας ανομίας αυτών.

\chapter{5}

\par 1 Και συ, υιέ ανθρώπου, λάβε εις σεαυτόν μάχαιραν κοπτεράν· θέλεις λάβει εις σεαυτόν ξυράφιον κουρέως και θέλεις περάσει αυτό επί την κεφαλήν σου και επί τον πώγωνά σου. Λάβε έπειτα εις σεαυτόν πλάστιγγας ζυγίων και διαίρεσον αυτά.
\par 2 Το τρίτον θέλεις καύσει εν πυρί εν τω μέσω της πόλεως, ενώ αι ημέραι της πολιορκίας συμπληρούνται· και το τρίτον θέλεις λάβει και κατακόψει κύκλω αυτής εν μαχαίρα· και το τρίτον θέλεις διασκορπίσει εις τον αέρα· και εγώ θέλω γυμνώσει μάχαιραν όπισθεν αυτών.
\par 3 Και εκ τούτων θέλεις λάβει έτι ολίγας τινάς και δέσει αυτάς εις τα κράσπεδά σου.
\par 4 Έπειτα λάβε έτι εκ τούτων και ρίψον αυτάς εις το μέσον του πυρός και κατάκαυσον αυτάς εν πυρί· εντεύθεν θέλει εξέλθει πυρ εις πάντα τον οίκον Ισραήλ.
\par 5 Ούτω λέγει Κύριος ο Θεός· Αύτη είναι η Ιερουσαλήμ· εγώ έθεσα αυτήν εν μέσω των εθνών και των πέριξ αυτής τόπων.
\par 6 Αλλ' αυτή μετήλλαξε τας κρίσεις μου εις ανομίαν χειρότερα παρά τα έθνη, και τα διατάγματά μου χειρότερα παρά τους τόπους τους πέριξ αυτής· διότι απέρριψαν τας κρίσεις μου και τα διατάγματά μου· δεν περιεπάτησαν εν αυτοίς.
\par 7 Όθεν ούτω λέγει Κύριος ο Θεός· Επειδή σεις υπερέβητε τα έθνη τα πέριξ υμών και δεν περιεπατήσατε εν τοις διατάγμασί μου και τας κρίσεις μου δεν εξετελέσατε αλλ' ουδέ κατά τας κρίσεις των εθνών των πέριξ υμών επράξατε,
\par 8 Διά τούτο ούτω λέγει Κύριος ο Θεός· Ιδού, και εγώ είμαι εναντίον σου και θέλω εκτελέσει κρίσεις εν μέσω σου ενώπιον των εθνών.
\par 9 Και θέλω κάμει εις σε εκείνο το οποίον δεν έκαμον, ουδέ θέλω κάμει ποτέ όμοιον τούτου, διά πάντα τα βδελύγματά σου.
\par 10 Διά τούτο οι πατέρες θέλουσι φάγει τα τέκνα αυτών εν μέσω σου και τα τέκνα θέλουσι φάγει τους πατέρας αυτών· και θέλω εκτελέσει κρίσεις εις σέ· άπαν δε το υπόλοιπόν σου θέλω διασκορπίσει εις πάντα άνεμον.
\par 11 Διά τούτο, ζω εγώ, λέγει Κύριος ο Θεός, εξάπαντος, επειδή συ εμίανας τα άγιά μου με πάσας τας μιαράς πράξεις σου και με πάντα τα βδελύγματά σου, και εγώ λοιπόν θέλω σε συντρίψει· και ο οφθαλμός μου δεν θέλει φεισθή, και εγώ δεν θέλω σε ελεήσει.
\par 12 Το τρίτον σου θέλουσιν αποθάνει υπό λοιμού και θέλουσιν αναλωθή εν μέσω σου υπό πείνης· και το τρίτον θέλουσι πέσει κύκλω σου υπό ρομφαίας· το δε άλλο τρίτον θέλω διασκορπίσει εις πάντα άνεμον και θέλω γυμνώσει μάχαιραν όπισθεν αυτών.
\par 13 Και θέλει συντελεσθή ο θυμός μου και θέλω αναπαύσει την οργήν μου επ' αυτούς και θέλω ευχαριστηθή· και θέλουσι γνωρίσει ότι εγώ ο Κύριος ελάλησα εν τω ζήλω μου, όταν συντελέσω κατ' αυτών την οργήν μου.
\par 14 Και θέλω σε καταστήσει έρημον και όνειδος μεταξύ των εθνών των κύκλω σου, ενώπιον παντός διαβαίνοντος.
\par 15 Και θέλεις είσθαι όνειδος και παίγνιον, διδασκαλία και θάμβος, εις τα έθνη τα κύκλω σου, όταν εκτελέσω κρίσεις εις σε εν θυμώ και εν οργή και μετ' επιτιμήσεων οργής· εγώ ο Κύριος ελάλησα.
\par 16 Όταν εξαποστείλω επ' αυτούς τα κακά βέλη της πείνης τα εξολοθρευτικά, τα οποία θέλω εξαποστείλει διά να σας εξολοθρεύσω, θέλω επαυξήσει έτι την πείναν εις εσάς και θέλω συντρίψει εις εσάς το υποστήριγμα του άρτου.
\par 17 Και θέλω εξαποστείλει εφ' υμάς πείναν και θηρία κακά και θέλουσι σε ορφανίσει, και λοιμός και αίμα θέλουσι περάσει διά σου, και θέλω φέρει ρομφαίαν επί σέ· εγώ ο Κύριος ελάλησα.

\chapter{6}

\par 1 Και έγεινε λόγος Κυρίου προς εμέ, λέγων,
\par 2 Υιέ ανθρώπου, στήριξον το πρόσωπόν σου προς τα όρη του Ισραήλ και προφήτευσον εναντίον αυτών,
\par 3 και ειπέ, Ορη του Ισραήλ, ακούσατε τον λόγον Κυρίου του Θεού· Ούτω λέγει Κύριος ο Θεός προς τα όρη και προς τα βουνά, προς τους ρύακας και προς τας κοιλάδας. Ιδού, εγώ, εγώ θέλω φέρει εφ' υμάς ρομφαίαν, και θέλω καταστρέψει τους υψηλούς τόπους σας.
\par 4 Και τα θυσιαστήριά σας θέλουσιν αφανισθή και τα είδωλά σας θέλουσι συντριφθή, και τους τετραυματισμένους σας θέλω καταβάλει έμπροσθεν των ξοάνων σας.
\par 5 Και θέλω στρώσει τα πτώματα των υιών Ισραήλ έμπροσθεν των ξοάνων αυτών, και θέλω διασκορπίσει τα οστά σας κύκλω των θυσιαστηρίων σας.
\par 6 κατά πάσαν κατοίκησίν σας αι πόλεις θέλουσιν ερημωθή, και οι υψηλοί τόποι θέλουσιν αφανισθή, ώστε τα θυσιαστήριά σας να ερημωθώσι και να αφανισθώσι, και τα ξόανά σας να συντριφθώσι και να εκλείψωσι, και τα είδωλά σας να πέσωσι κατακεκομμένα, και τα έργα σας να εξαλειφθώσι.
\par 7 Και οι τετραυματισμένοι θέλουσι πέσει εν μέσω υμών, και θέλετε γνωρίσει ότι εγώ είμαι ο Κύριος.
\par 8 Θέλω όμως αφήσει υπόλοιπον, διά να έχητε τινάς εκφυγόντας την μάχαιραν μεταξύ των εθνών, όταν διασκορπισθήτε εις τους τόπους.
\par 9 Και όσοι από σας εκφύγωσι, θέλουσι με ενθυμείσθαι μεταξύ των εθνών, όπου θέλουσι φερθή αιχμάλωτοι, όταν φέρω εις συντριβήν την πορνικήν αυτών καρδίαν ήτις εξέκλινεν απ' εμού, και τους οφθαλμούς αυτών τους εκπορνεύοντας κατόπιν των ξοάνων αυτών· και θέλουσιν αποστρέφεσθαι εαυτούς δι' όσας κακίας έπραξαν εν πάσι τοις βδελύγμασιν αυτών.
\par 10 Και θέλουσι γνωρίσει ότι εγώ ο Κύριος δεν ελάλησα ματαίως, ότι ήθελον κάμει εις αυτούς τα κακά ταύτα.
\par 11 Ούτω λέγει Κύριος ο Θεός· Κρότησον με την χείρα σου και κτύπησον με τον πόδα σου και ειπέ, Ουαί διά πάντα τα κακά βδελύγματα του οίκου Ισραήλ· διότι θέλουσι πέσει υπό μαχαίρας, υπό πείνης και υπό λοιμού.
\par 12 Ο μακράν θέλει αποθάνει υπό λοιμού και ο πλησίον θέλει πέσει υπό μαχαίρας, ο δε εναπολειφθείς και ο πολιορκούμενος θέλει αποθάνει υπό πείνης· ούτω θέλω συντελέσει την οργήν μου επ' αυτούς.
\par 13 Και θέλετε γνωρίσει ότι εγώ είμαι ο Κύριος, όταν οι τραυματίαι αυτών κήνται μεταξύ των ξοάνων αυτών κύκλω των θυσιαστηρίων αυτών, επί πάντα υψηλόν λόφον, επί πάσας τας κορυφάς των ορέων και υποκάτω παντός δένδρον πρασίνου και υποκάτω πάσης δασυφύλλου δρυός, του τόπου όπου προσέφερον οσμήν ευωδίας εις πάντα τα ξόανα αυτών.
\par 14 Και θέλω εκτείνει την χείρα μου επ' αυτούς και καταστήσει έρημον την γην, ερημοτέραν μάλιστα παρά την έρημον Διβλαθά, εις πάσας αυτών τας κατοικήσεις· και θέλουσι γνωρίσει ότι εγώ είμαι ο Κύριος.

\chapter{7}

\par 1 Και έγεινε λόγος Κυρίου προς εμέ, λέγων,
\par 2 Και συ, υιέ ανθρώπου, άκουσον· ούτω λέγει Κύριος ο Θεός προς την γην του Ισραήλ. Τέλος, το τέλος ήλθεν επί τα τέσσαρα άκρα της γης.
\par 3 Το τέλος ήλθε τώρα επί σε και θέλω αποστείλει επί σε την οργήν μου και θέλω σε κρίνει κατά τας οδούς σου και θέλω ανταποδώσει επί σε πάντα τα βδελύγματά σου.
\par 4 Και ο οφθαλμός μου δεν θέλει σε φεισθή και δεν θέλω ελεήσει· αλλά θέλω ανταποδώσει επί σε τας οδούς σου, και θέλουσιν είσθαι εν μέσω σου τα βδελύγματά σου· και θέλετε γνωρίσει ότι εγώ είμαι ο Κύριος.
\par 5 Ούτω λέγει Κύριος ο Θεός· Κακόν, εν κακόν, ιδού, έρχεται·
\par 6 τέλος ήλθε, το τέλος ήλθεν· εξηγέρθη κατά σού· ιδού, έφθασεν.
\par 7 Η πρωΐα ήλθεν επί σε, κάτοικε της γής· ο καιρός ήλθεν, η ημέρα της καταστροφής επλησίασε και ουχί αγαλλίασις των ορέων.
\par 8 Τώρα ευθύς θέλω εκχέει την οργήν μου επί σε και θέλω συντελέσει τον θυμόν μου επί σέ· και θέλω σε κρίνει κατά τας οδούς σου και ανταποδώσει επί σε πάντα τα βδελύγματά σου.
\par 9 Και ο οφθαλμός μου δεν θέλει φεισθή και δεν θέλω ελεήσει· κατά τας οδούς σου θέλω σοι ανταποδώσει, και θέλουσιν είσθαι τα βδελύγματά σου εν μέσω σου· και θέλετε γνωρίσει ότι εγώ είμαι ο Κύριος ο πατάσσων.
\par 10 Ιδού, η ημέρα, ιδού, ήλθεν· η πρωΐα εφάνη· η ράβδος ήνθησεν· η υπερηφανία εβλάστησεν.
\par 11 Η βία ηυξήνθη εις ράβδον ανομίας· ουδείς εξ αυτών θέλει μείνει ούτε εκ του πλήθους αυτών ούτε εκ των θορυβούντων εξ αυτών· και δεν θέλει υπάρχει ο πενθών δι ' αυτούς.
\par 12 Ο καιρός ήλθεν, η ημέρα επλησίασεν· ας μη χαίρη ο αγοράζων και ας μη θρηνή ο πωλών, διότι είναι οργή επί παν το πλήθος αυτής.
\par 13 Διότι ο πωλητής δεν θέλει επιστρέψει εις το πωληθέν, αν και ευρίσκηται έτι μεταξύ των ζώντων· επειδή η όρασις η περί παντός του πλήθους αυτών δεν θέλει στραφή οπίσω· και ουδείς θέλει στερεώσει εαυτόν, του οποίου η ζωή είναι εν τη ανομία αυτού.
\par 14 Εσάλπισαν εν σάλπιγγι και ητοιμάσθησαν τα πάντα· πλην ουδείς υπάγει εις τον πόλεμον, διότι η οργή μου είναι επί παν το πλήθος αυτής.
\par 15 Η μάχαιρα είναι έξωθεν και ο λοιμός και η πείνα έσωθεν· ο εν τω αγρώ θέλει τελευτήσει εν μαχαίρα, τον δε εν τη πόλει, η πείνα και ο λοιμός θέλουσι καταφάγει αυτόν.
\par 16 Και όσοι εξ αυτών εκφύγωσι, θέλουσι διασωθή και θέλουσιν είσθαι επί των ορέων ως αι περιστεραί των κοιλάδων, θρηνούντες πάντες ούτοι, έκαστος διά τας ανομίας αυτού.
\par 17 Πάσαι αι χείρες θέλουσι παραλυθή και πάντα τα γόνατα θέλουσι ρεύσει ως ύδωρ.
\par 18 Και θέλουσι περιζωσθή σάκκον και φρίκη θέλει καλύψει αυτούς· και αισχύνη θέλει είσθαι επί πάντα τα πρόσωπα και φαλάκρωμα επί πάσας τας κεφαλάς αυτών.
\par 19 Το αργύριον αυτών θέλουσι ρίψει εις τας οδούς, και το χρυσίον αυτών θέλει είσθαι ως ακαθαρσία· το αργύριον αυτών και το χρυσίον αυτών δεν θέλουσι δυνηθή να λυτρώσωσιν αυτούς εν τη ημέρα της οργής του Κυρίου· δεν θέλουσι χορτάσει τας ψυχάς αυτών και δεν θέλουσι γεμίσει τας κοιλίας αυτών, διότι έγεινε το πρόσκομμα της ανομίας αυτών.
\par 20 Επειδή την δόξαν του στολισμού αυτών, μετεχειρίσθησαν αυτήν εις υπερηφανίαν και έκαμον εξ αυτής τας εικόνας των βδελυγμάτων αυτών, τα μισητά αυτών· διά τούτο εγώ καθιστώ αυτήν εις αυτούς ακαθαρσίαν.
\par 21 Και θέλω παραδώσει αυτήν εις χείρας αλλοτρίων διάρπαγμα και εις τους ασεβείς της γης λάφυρον, και θέλουσι βεβηλώσει αυτήν.
\par 22 Και θέλω αποστρέψει το πρόσωπόν μου απ' αυτών, και θέλουσι βεβηλώσει το άδυτόν μου· και οι λεηλάται θέλουσιν εμβή εις αυτό και βεβηλώσει αυτό.
\par 23 Κάμε άλυσον, διότι η γη είναι πλήρης από κρίσεως αιμάτων και η πόλις πλήρης καταδυναστείας.
\par 24 Διά τούτο θέλω φέρει τους κακίστους των εθνών και θέλουσι κληρονομήσει τας οικίας αυτών· και θέλω καταβάλει την έπαρσιν των ισχυρών· και τα άγια αυτών θέλουσι βεβηλωθή.
\par 25 Όλεθρος επέρχεται· και θέλουσι ζητήσει ειρήνην και δεν θέλει υπάρχει.
\par 26 Συμφορά επί συμφοράν θέλει έρχεσθαι και αγγελία θέλει φθάνει επ' αγγελίαν· τότε θέλουσι ζητήσει παρά προφήτου όρασιν· και θέλει χαθή ο νόμος από του ιερέως και η βουλή από των πρεσβυτέρων.
\par 27 Ο βασιλεύς θέλει πενθήσει και ο άρχων θέλει ενδυθή αφανισμόν και αι χείρες του λαού της γης θέλουσι παραλυθή· κατά τας οδούς αυτών θέλω κάμει εις αυτούς και κατά τας κρίσεις αυτών θέλω κρίνει αυτούς, και θέλουσι γνωρίσει ότι εγώ είμαι ο Κύριος.

\chapter{8}

\par 1 Και εν τω έκτω έτει, τω έκτω μηνί, τη πέμπτη του μηνός, ενώ εγώ εκαθήμην εν τω οίκω μου και οι πρεσβύτεροι του Ιούδα εκάθηντο έμπροσθέν μου, χειρ Κυρίου του Θεού έπεσεν εκεί επ' εμέ.
\par 2 Και είδον και ιδού, ομοίωμα ως θέα πυρός· από της θέας της οσφύος αυτού και κάτω πυρ, και από της οσφύος αυτού και επάνω ως θέα λάμψεως, ως όψις ηλέκτρου.
\par 3 Και εξήπλωσεν ομοίωμα χειρός, και με επίασεν από της κόμης της κεφαλής μου και με ύψωσε το πνεύμα μεταξύ της γης και του ουρανού και με έφερε δι' οραμάτων Θεού εις Ιερουσαλήμ, εις την θύραν της εσωτέρας πύλης της βλεπούσης προς βορράν, όπου ίστατο το είδωλον της ζηλοτυπίας, το παροξύνον εις ζηλοτυπίαν.
\par 4 Και ιδού, η δόξα του Θεού του Ισραήλ ήτο εκεί, κατά το όραμα το οποίον είδον εν τη πεδιάδι.
\par 5 Και είπε προς εμέ, Υιέ ανθρώπου, ύψωσον τώρα τους οφθαλμούς σου προς την οδόν την προς βορράν. Και ύψωσα τους οφθαλμούς μου προς την οδόν την προς βορράν και ιδού, κατά το βόρειον μέρος εν τη πύλη του θυσιαστηρίου το είδωλον τούτο της ζηλοτυπίας κατά την είσοδον.
\par 6 Τότε είπε προς εμέ, Υιέ ανθρώπου, βλέπεις συ τι κάμνουσιν ούτοι; τα μεγάλα βδελύγματα, τα οποία ο οίκος Ισραήλ κάμνει εδώ, διά να απομακρυνθώ από των αγίων μου; πλην στρέψον έτι, θέλεις ιδεί μεγαλήτερα βδελύγματα.
\par 7 Και με έφερεν εις την πύλην της αυλής· και είδον και ιδού, μία οπή εν τω τοίχω.
\par 8 Και είπε προς εμέ, Υιέ ανθρώπου, σκάψον τώρα εν τω τοίχω· και έσκαψα εν τω τοίχω και ιδού, μία θύρα.
\par 9 Και είπε προς εμέ, Είσελθε και ιδέ τα πονηρά βδελύγματα, τα οποία ούτοι κάμνουσιν εδώ.
\par 10 Και εισήλθον και είδον· και ιδού, παν ομοίωμα ερπετών και βδελυκτών ζώων και πάντα τα είδωλα του οίκου Ισραήλ, εζωγραφημένα επί τον τοίχον κύκλω κύκλω.
\par 11 Και ίσταντο έμπροσθεν αυτών εβδομήκοντα άνδρες εκ των πρεσβυτέρων του οίκου Ισραήλ· εν μέσω δε αυτών ίστατο Ιααζανίας ο υιός του Σαφάν· και εκράτει έκαστος εν τη χειρί αυτού το θυμιατήριον αυτού· και ανέβαινε πυκνόν νέφος θυμιάματος.
\par 12 Και είπε προς εμέ, Υιέ ανθρώπου, είδες τι κάμνουσιν εν τω σκότει οι πρεσβύτεροι του οίκου Ισραήλ, έκαστος εν τω κρυπτώ οικήματι των εικόνων αυτού; διότι είπον, Ο Κύριος δεν μας βλέπει· ο Κύριος εγκατέλιπε την γην.
\par 13 Και είπε προς εμέ, Στρέψον έτι· θέλεις ιδεί μεγαλήτερα βδελύγματα, τα οποία ούτοι κάμνουσι.
\par 14 Και με έφερεν εις τα πρόθυρα της πύλης του οίκου του Κυρίου της προς βορράν, και ιδού, εκεί εκάθηντο γυναίκες θρηνούσαι τον Θαμμούζ.
\par 15 Και είπε προς εμέ, Είδες, υιέ ανθρώπου; Στρέψον έτι· θέλεις ιδεί μεγαλήτερα βδελύγματα παρά ταύτα.
\par 16 Και με εισήγαγεν εις την εσωτέραν αυλήν του οίκου του Κυρίου· και ιδού, εν τη θύρα του ναού του Κυρίου, μεταξύ της στοάς και του θυσιαστηρίου, περίπου είκοσιπέντε άνδρες με τα νώτα αυτών προς τον ναόν του Κυρίου και τα πρόσωπα αυτών προς ανατολάς, και προσεκύνουν τον ήλιον κατά ανατολάς.
\par 17 Και είπε προς εμέ, είδες, υιέ ανθρώπου; Μικρόν είναι τούτο εις τον οίκον Ιούδα, να κάμνωσι τα βδελύγματα, τα οποία ούτοι κάμνουσιν ενταύθα; ώστε εγέμισαν την γην από καταδυναστείας και εξέκλιναν διά να με παροργίσωσι· και ιδού, βάλλουσι τον κλάδον εις τους μυκτήρας αυτών.
\par 18 Και εγώ λοιπόν θέλω φερθή μετ' οργής· ο οφθαλμός μου δεν θέλει φεισθή ουδέ θέλω ελεήσει· και όταν κράξωσιν εις τα ώτα μου μετά φωνής μεγάλης, δεν θέλω εισακούσει αυτούς.

\chapter{9}

\par 1 Και έκραξεν εις τα ώτα μου μετά φωνής μεγάλης· λέγων, Ας πλησιάσωσιν οι τεταγμένοι κατά της πόλεως, έκαστος έχων το όπλον αυτού της εξολοθρεύσεως εν τη χειρί αυτού.
\par 2 Και ιδού, εξ άνδρες ήρχοντο από της οδού της υψηλοτέρας πύλης της βλεπούσης προς βορράν, έκαστος έχων εν τη χειρί αυτού όπλον κατασυντριμμού· και εν τω μέσω αυτών εις άνθρωπος ενδεδυμένος λινά με γραμματέως καλαμάριον εν τη οσφύϊ αυτού· και εισελθόντες εστάθησαν πλησίον του χαλκίνου θυσιαστηρίου.
\par 3 Και η δόξα του Θεού του Ισραήλ ανέβη επάνωθεν των χερουβείμ, επάνωθεν των οποίων ήτο, εις το κατώφλιον του οίκου· και εφώνησε προς τον άνδρα τον ενδεδυμένον τα λινά, τον έχοντα εν τη οσφύϊ αυτού το καλαμάριον του γραμματέως·
\par 4 και είπε Κύριος προς αυτόν, Δίελθε διά της πόλεως, διά της Ιερουσαλήμ, και κάμε σημείον επί των μετώπων των ανδρών, των στεναζόντων και βοώντων διά πάντα τα βδελύγματα τα γινόμενα εν μέσω αυτής.
\par 5 Προς δε τους άλλους είπεν, ακούοντος εμού, Διέλθετε κατόπιν αυτού διά της πόλεως και πατάξατε· ας μη φεισθή ο οφθαλμός σας και μη ελεήσητε·
\par 6 γέροντας, νέους και παρθένους και νήπια και γυναίκας, φονεύσατε μέχρις εξαλείψεως· εις πάντα όμως άνθρωπον εφ' ου είναι το σημείον μη πλησιάσητε· και αρχίσατε από του αγιαστηρίου μου. Και ήρχισαν από των ανδρών των πρεσβυτέρων των έμπροσθεν του οίκου.
\par 7 Και είπε προς αυτούς, Μιάνατε τον οίκον και γεμίσατε τας αυλάς από τραυματιών· εξέλθετε. Και εξήλθον και επάταξαν εν τη πόλει.
\par 8 Ενώ δε ούτοι επάτασσον αυτούς, εναπολειφθείς εγώ έπεσον επί πρόσωπόν μου και ανεβόησα και είπα, Οίμοι, Κύριε Θεέ· συ εξαλείφεις άπαν το υπόλοιπον του Ισραήλ, εκχέων την οργήν σου επί την Ιερουσαλήμ;
\par 9 Και είπε προς εμέ, Η ανομία του οίκου Ισραήλ και Ιούδα υπερεμεγαλύνθη σφόδρα και η γη είναι πλήρης αιμάτων· και πόλις πλήρης διαφθοράς· διότι λέγουσιν, Ο Κύριος εγκατέλιπε την γην, και, Ο Κύριος δεν βλέπει.
\par 10 Και εγώ λοιπόν δεν θέλει φεισθή ο οφθαλμός μου και δεν θέλω ελεήσει· κατά της κεφαλής αυτών θέλω ανταποδώσει τας οδούς αυτών.
\par 11 Και ιδού, ο ανήρ ο ενδεδυμένος τα λινά, ο έχων εν τη οσφύϊ αυτού το καλαμάριον, έφερεν απόκρισιν, λέγων, Έκαμον καθώς προσέταξας εις εμέ.

\chapter{10}

\par 1 Έπειτα είδον και ιδού, εν τω στερεώματι τω άνωθεν της κεφαλής των χερουβείμ εφαίνετο υπεράνω αυτών ως λίθος σάπφειρος, κατά την θέαν ομοιώματος θρόνου.
\par 2 Και ελάλησε προς τον άνδρα τον ενδεδυμένον τα λινά και είπεν, Είσελθε μεταξύ των τροχών, υποκάτω των χερουβείμ, και γέμισον την χείρα σου άνθρακας πυρός εκ μέσου των χερουβείμ και διασκόρπισον αυτούς επί την πόλιν. Και εισήλθεν ενώπιόν μου.
\par 3 Τα δε χερουβείμ ίσταντο εν δεξιοίς του οίκου, ότε εισήρχετο ο ανήρ· και η νεφέλη εγέμισε την εσωτέραν αυλήν.
\par 4 Και η δόξα του Κυρίου υψώθη άνωθεν των χερουβείμ κατά το κατώφλιον του οίκου· και ενέπλησε τον οίκον η νεφέλη και η αυλή ενεπλήσθη από της λάμψεως της δόξης του Κυρίου.
\par 5 Και ο ήχος των πτερύγων των χερουβείμ ηκούετο έως της εξωτέρας αυλής, ως φωνή του Παντοδυνάμου Θεού, οπόταν λαλή.
\par 6 Και ότε προσέταξε τον άνδρα τον ενδεδυμένον τα λινά, λέγων, Λάβε πυρ εκ μέσου των τροχών, εκ μέσου των χερουβείμ, τότε εισήλθε και εστάθη πλησίον των τροχών.
\par 7 Και εν χερούβ εξέτεινε την χείρα αυτού εκ μέσου των χερουβείμ, προς το πυρ το εν τω μέσω των χερουβείμ, και έλαβεν εκ τούτου και έθεσεν εις τας χείρας του ενδεδυμένου τα λινά· ο δε έλαβεν αυτό και εξήλθεν.
\par 8 Εφαίνετο δε ομοίωμα χειρός ανθρώπου εις τα χερουβείμ υπό τας πτέρυγας αυτών.
\par 9 Και είδον και ιδού, τέσσαρες τροχοί πλησίον των χερουβείμ, εις τροχός πλησίον ενός χερούβ και εις τροχός πλησίον άλλον χερούβ, και η θέα των τροχών ήτο ως όψις βηρύλλου λίθου.
\par 10 Περί δε της θέας αυτών, και οι τέσσαρες είχον το αυτό ομοίωμα, ως εάν ήτο τροχός εν μέσω τροχού.
\par 11 Ενώ εβάδιζον, επορεύοντο κατά τα τέσσαρα αυτών πλάγια· δεν εστρέφοντο ενώ εβάδιζον, αλλ' εις όντινα τόπον ο πρώτος απευθύνετο, ηκολούθουν αυτόν οι άλλοι· δεν εστρέφοντο ενώ εβάδιζον.
\par 12 Όλον δε το σώμα αυτών και τα νώτα αυτών και αι χείρες αυτών και αι πτέρυγες αυτών και οι τροχοί, οι τέσσαρες αυτών τροχοί, ήσαν κύκλω πλήρεις οφθαλμών.
\par 13 Περί δε των τροχών, ούτοι εκαλούντο, ακούοντος εμού, Γαλγάλ.
\par 14 Και έκαστον είχε τέσσαρα πρόσωπα· το πρόσωπον του ενός πρόσωπον χερούβ, και το πρόσωπον του δευτέρου πρόσωπον ανθρώπου, και του τρίτου πρόσωπον λέοντος, και του τετάρτου πρόσωπον αετού.
\par 15 Και τα χερουβείμ υψώθησαν τούτο είναι το ζώον, το οποίον είδον παρά τον ποταμόν Χεβάρ.
\par 16 Και ότε τα χερουβείμ επορεύοντο, επορεύοντο οι τροχοί πλησίον αυτών και ότε τα χερουβείμ ύψονον τας πτέρυγας αυτών διά να ανυψωθώσιν από της γης, και αυτοί οι τροχοί δεν εξέκλινον από πλησίον αυτών.
\par 17 Ότε δε ίσταντο, και εκείνοι ίσταντο· και ότε ανυψούντο, και εκείνοι ανυψούντο μετ' αυτών διότι το πνεύμα των ζώων ήτο εν αυτοίς.
\par 18 Και η δόξα του Κυρίου εξήλθεν από του κατωφλίου του οίκου και εστάθη επί των χερουβείμ.
\par 19 Και τα χερουβείμ ύψωσαν τας πτέρυγας αυτών και ανυψώθησαν από της γης ενώπιόν μου ότε εξήλθον, ήσαν και οι τροχοί πλησίον αυτών· και εστάθησαν εν τη θύρα της ανατολικής πύλης του οίκου του Κυρίου· και η δόξα του Θεού του Ισραήλ ήτο επ' αυτών υπεράνωθεν.
\par 20 Τούτο είναι το ζώον, το οποίον είδον υποκάτω του Θεού του Ισραήλ παρά τον ποταμόν Χεβάρ· και εγνώρισα ότι ήσαν χερουβείμ.
\par 21 Έκαστον είχεν ανά τέσσαρα πρόσωπα και έκαστον τέσσαρας πτέρυγας και ομοίωμα χειρών ανθρώπου υπό τας πτέρυγας αυτών.
\par 22 Τα δε πρόσωπα αυτών ήσαν κατά το ομοίωμα, τα αυτά πρόσωπα, τα οποία είδον παρά τον ποταμόν Χεβάρ, η θέα αυτών και αυτά· επορεύοντο δε έκαστον κατέναντι του προσώπου αυτού.

\chapter{11}

\par 1 Και με ανέλαβε το πνεύμα και με έφερεν εις την ανατολικήν πύλην του οίκου του Κυρίου, την βλέπουσαν προς ανατολάς· και ιδού, εν τη θύρα της πύλης εικοσιπέντε άνδρες, και μεταξύ αυτών είδον τον Ιααζανίαν υιόν του Αζώρ και τον Φελατίαν υιόν του Βεναΐα, άρχοντας του λαού.
\par 2 Και είπε Κύριος προς εμέ, Υιέ ανθρώπου, ούτοι είναι οι άνδρες οι διαλογιζόμενοι αδικίαν και συμβουλεύοντες κακήν συμβουλήν εις την πόλιν ταύτην,
\par 3 οι λέγοντες, Δεν είναι πλησίον· ας κτίσωμεν οικίας· αύτη η πόλις είναι ο λέβης και ημείς το κρέας.
\par 4 Διά τούτο προφήτευσον εναντίον αυτών, προφήτευσον, υιέ ανθρώπου.
\par 5 Και πνεύμα Κυρίου έπεσεν επ' εμέ και μοι είπε, Λάλησον Ούτω λέγει Κύριος· κατά τούτον τον τρόπον ελαλήσατε, οίκος Ισραήλ διότι τα διαβούλια του πνεύματός σας, εγώ εξεύρω αυτά.
\par 6 Επληθύνατε τους πεφονευμένους σας εν τη πόλει ταύτη, και εγεμίσατε τας οδούς αυτής από πεφονευμένων.
\par 7 Όθεν ούτω λέγει Κύριος ο Θεός οι πεφονευμένοι σας, τους οποίους εθέσατε εν μέσω αυτής, ούτοι είναι το κρέας και αύτη η πόλις ο λέβης· σας όμως θέλω εκβάλει εκ μέσου αυτής.
\par 8 Την μάχαιραν εφοβήθητε· και μάχαιραν θέλω φέρει εφ' υμάς, λέγει Κύριος ο Θεός.
\par 9 Και θέλω σας εκβάλει εκ μέσου αυτής και θέλω σας παραδώσει εις χείρας αλλοφύλων· και θέλω εκτελέσει εφ' υμάς κρίσεις.
\par 10 Υπό ρομφαίας θέλετε πέσει· εν τοις ορίοις του Ισραήλ θέλω σας κρίνει· και θέλετε γνωρίσει ότι εγώ είμαι ο Κύριος.
\par 11 Η πόλις αύτη δεν θέλει είσθαι εις εσάς ο λέβης ουδέ σεις θέλετε είσθαι εν μέσω αυτής το κρέας εν τοις ορίοις του Ισραήλ θέλω σας κρίνει
\par 12 και θέλετε γνωρίσει ότι εγώ είμαι ο Κύριος· διότι δεν περιεπατήσατε εν τοις διατάγμασί μου ουδέ εξετελέσατε τας κρίσεις μου, αλλ' επράξατε κατά τας κρίσεις των εθνών των κύκλω υμών.
\par 13 Ενώ δ' εγώ προεφήτευον, απέθανεν ο Φελατίας ο υιός του Βεναΐα. Τότε έπεσον επί πρόσωπόν μου και ανεβόησα μετά φωνής μεγάλης και είπα, Οίμοι, Κύριε Θεέ συντέλειαν θέλεις να κάμης συ του υπολοίπου του Ισραήλ;
\par 14 Και έγεινε λόγος Κυρίου προς εμέ, λέγων,
\par 15 Υιέ ανθρώπου, οι αδελφοί σου, οι αδελφοί σου, οι άνδρες της συγγενείας σου, και σύμπας ο οίκος Ισραήλ, είναι εκείνοι προς τους οποίους είπον οι κατοικούντες την Ιερουσαλήμ, Απομακρύνθητε από του Κυρίου εις ημάς εδόθη αύτη η γη διά κληρονομίαν.
\par 16 Διά τούτο ειπέ, Ούτω λέγει Κύριος ο Θεός· Αν και απέρριψα αυτούς μακράν μεταξύ των εθνών, αν και διεσκόρπισα αυτούς εις τους τόπους, θέλω είσθαι όμως εις αυτούς ως μικρόν αγιαστήριον, εν τοις τόποις όπου υπάγωσι.
\par 17 Διά τούτο ειπέ, Ούτω λέγει Κύριος ο Θεός· Και θέλω σας συναθροίσει από των λαών και θέλω σας συνάξει εκ των τόπων όπου ήσθε διεσκορπισμένοι και θέλω σας δώσει την γην Ισραήλ.
\par 18 Και ελθόντες εκεί θέλουσι σηκώσει απ' αυτής πάντα τα βδελύγματα αυτής και πάντα τα μιαρά αυτής.
\par 19 Και θέλω δώσει εις αυτούς καρδίαν μίαν και πνεύμα νέον θέλω βάλει εν υμίν· και αποσπάσας την λιθίνην καρδίαν από της σαρκός αυτών θέλω δώσει εις αυτούς καρδίαν σαρκίνην,
\par 20 διά να περιπατώσιν εν τοις διατάγμασί μου και να φυλάττωσι τας κρίσεις μου και να εκτελώσιν αυτάς· και θέλουσιν είσθαι λαός μου και εγώ θέλω είσθαι Θεός αυτών.
\par 21 Εκείνων δε των οποίων η καρδία περιπατεί κατά την επιθυμίαν των βδελυγμάτων αυτών και των μιαρών αυτών, τας οδούς τούτων θέλω ανταποδώσει κατά της κεφαλής αυτών, λέγει Κύριος ο Θεός.
\par 22 Τότε τα χερουβείμ ύψωσαν τας πτέρυγας αυτών και οι τροχοί ανέβαινον πλησίον αυτών· και η δόξα του Θεού του Ισραήλ ήτο επ' αυτών υπεράνωθεν.
\par 23 Και η δόξα του Κυρίου ανέβη εκ μέσου της πόλεως και εστάθη επί το όρος το προς ανατολάς της πόλεως.
\par 24 Και με ανέλαβε το πνεύμα και δι' οράματος με έφερεν εν πνεύματι Θεού εις την γην των Χαλδαίων, προς τους αιχμαλώτους. Τότε το όραμα, το οποίον είδον, απήλθεν απ' εμού.
\par 25 Και ελάλησα προς τους αιχμαλώτους πάντα τα πράγματα όσα έδειξεν ο Κύριος εις εμέ.

\chapter{12}

\par 1 Και έγεινε λόγος Κυρίου προς εμέ, λέγων,
\par 2 Υιέ ανθρώπου, συ κατοικείς εν μέσω οίκου αποστάτου, οίτινες οφθαλμούς έχουσι διά να βλέπωσι, και δεν βλέπουσιν· ώτα έχουσι διά να ακούωσι, και δεν ακούουσι· διότι είναι οίκος αποστάτης.
\par 3 Διά τούτο, συ, υιέ ανθρώπου, ετοίμασον εις σεαυτόν αποσκευήν μετοικισμού, και μετοικίσθητι την ημέραν ενώπιον αυτών· και θέλεις μετοικισθή από του τόπου σου εις άλλον τόπον ενώπιον αυτών· ίσως προσέξωσιν, αν και ήναι οίκος αποστάτης.
\par 4 Και θέλεις εκφέρει την αποσκευήν σου την ημέραν ενώπιον αυτών, ως αποσκευήν μετοικισμού· και συ θέλεις εξέλθει το εσπέρας ενώπιον αυτών, ως οι εξερχόμενοι εις μετοικισμόν.
\par 5 Ενώπιον αυτών κάμε διόρυγμα εν τω τοίχω και έκφερε δι' αυτού.
\par 6 Ενώπιον αυτών θέλεις σηκώσει αυτήν επ' ώμων, και θέλεις εκφέρει, ενώ σκοτάζει· θέλεις σκεπάσει το πρόσωπόν σου και δεν θέλεις ιδεί την γήν· διότι σε έδωκα σημείον εις τον οίκον Ισραήλ.
\par 7 Και έκαμον ως προσετάχθην· έφερα έξω την αποσκευήν μου την ημέραν ως αποσκευήν μετοικισμού, και το εσπέρας έκαμον εις εμαυτόν διόρυγμα εν τω τοίχω διά της χειρός· εξέφερα αυτήν ενώ εσκόταζεν, ενώπιον αυτών εσήκωσα αυτήν επ' ώμων.
\par 8 Και το πρωΐ έγεινε λόγος Κυρίου προς εμέ λέγων,
\par 9 Υιέ ανθρώπου, ο οίκος Ισραήλ, ο οίκος ο αποστάτης, δεν είπε προς σε, Συ τι κάμνεις;
\par 10 ειπέ προς αυτούς, Ούτω λέγει Κύριος ο Θεός· Το φορτίον τούτο αποβλέπει τον άρχοντα τον εν Ιερουσαλήμ και άπαντα τον οίκον Ισραήλ, οίτινες είναι μεταξύ αυτών.
\par 11 Ειπέ, Εγώ είμαι το σημείόν σας· καθώς εγώ έκαμον, ούτω θέλει γείνει εις αυτούς· εις μετοικεσίαν και εις αιχμαλωσίαν θέλουσιν υπάγει.
\par 12 Και ο άρχων ο μεταξύ αυτών θέλει φορτωθή επ' ώμων, ενώ σκοτάζει, και θέλει εκφέρει· θέλουσι διορύξει τον τοίχον διά να εκφέρωσι δι' αυτού· θέλει σκεπάσει το πρόσωπον αυτού, διά να μη ίδη την γην με τους οφθαλμούς αυτού.
\par 13 Θέλω όμως εξαπλώσει το δίκτυόν μου επ' αυτόν, και θέλει πιασθή εις τα βρόχιά μου· και θέλω φέρει αυτόν εις την Βαβυλώνα, την γην των Χαλδαίων· αλλά δεν θέλει ιδεί αυτήν και εκεί θέλει αποθάνει.
\par 14 Και θέλω διασπείρει εις πάντα άνεμον πάντας τους περί αυτόν διά να βοηθώσιν αυτόν και πάσας τας δυνάμεις αυτού· και θέλω γυμνώσει μάχαιραν όπισθεν αυτών.
\par 15 Και θέλουσι γνωρίσει ότι εγώ είμαι ο Κύριος, όταν διασκορπίσω αυτούς μεταξύ των εθνών και διασπείρω αυτούς εις τους τόπους.
\par 16 Θέλω όμως αφήσει ολίγους τινάς εξ αυτών από της ρομφαίας, από της πείνης και από του λοιμού, διά να διηγώνται πάντα τα βδελύγματα αυτών μεταξύ των εθνών, όπου υπάγωσι· και θέλουσι γνωρίσει ότι εγώ είμαι ο Κύριος.
\par 17 Και έγεινε λόγος Κυρίου προς εμέ, λέγων,
\par 18 Υιέ ανθρώπου, φάγε τον άρτον σου μετά τρόμου και πίε το ύδωρ σου μετά φρίκης και αγωνίας.
\par 19 Και ειπέ προς τον λαόν της γης, Ούτω λέγει Κύριος ο Θεός περί των κατοίκων της Ιερουσαλήμ και περί της γης του Ισραήλ. Θέλουσι φάγει τον άρτον αυτών μετά αγωνίας και θέλουσι πίει το ύδωρ αυτών μετά εκστάσεως· διότι η γη αυτής θέλει ερημωθή από του πληρώματος αυτής, διά την ανομίαν πάντων των κατοικούντων εν αυτή·
\par 20 και αι πόλεις αι κατοικούμεναι θέλουσιν ερημωθή και η γη θέλει αφανισθή, και θέλετε γνωρίσει ότι εγώ είμαι ο Κύριος.
\par 21 Και έγεινε λόγος Κυρίου προς εμέ, λέγων,
\par 22 Υιέ ανθρώπου, τις αύτη η παροιμία, την οποίαν έχετε εν γη Ισραήλ, λέγοντες, Αι ημέραι μακρύνονται και πάσα όρασις εχάθη;
\par 23 Ειπέ διά τούτο προς αυτούς, Ούτω λέγει Κύριος ο Θεός· Θέλω κάμει την παροιμίαν ταύτην να παύση, και πλέον δεν θέλουσι παροιμιάζεσθαι αυτήν εν τω Ισραήλ· αλλ' ειπέ προς αυτούς, Πλησιάζουσιν αι ημέραι και η εκπλήρωσις πάσης οράσεως·
\par 24 διότι δεν θέλει είσθαι πλέον ουδεμία όρασις ψευδής ουδέ μάντευμα κολακευτικόν εν μέσω του οίκου Ισραήλ.
\par 25 Διότι εγώ είμαι ο Κύριος· εγώ θέλω λαλήσει και ο λόγος τον οποίον θέλω λαλήσει θέλει εκτελεσθή· δεν θέλει πλέον μακρυνθή· διότι εν ταις ημέραις υμών, οίκος αποστάτης, θέλω λαλήσει λόγον και εκτελέσει αυτόν, λέγει Κύριος ο Θεός.
\par 26 Και έγεινε λόγος Κυρίου προς εμέ, λέγων,
\par 27 Υιέ ανθρώπου, ιδού, ο οίκος Ισραήλ λέγουσιν, Η όρασις, την οποίαν ούτος βλέπει, εκτείνεται εις ημέρας πολλάς και προφητεύει περί χρόνων μακρών.
\par 28 Διά τούτο ειπέ προς αυτούς, Ούτω λέγει Κύριος ο Θεός· Ουδείς των λόγων μου θέλει πλέον μακρυνθή αλλ' ο λόγος τον οποίον ελάλησα θέλει εκτελεσθή, λέγει Κύριος ο Θεός.

\chapter{13}

\par 1 Και έγεινε λόγος Κυρίου προς εμέ, λέγων,
\par 2 Υιέ ανθρώπου, προφήτευσον επί τους προφήτας του Ισραήλ τους προφητεύοντας και ειπέ προς τους προφητεύοντας εξ ιδίας αυτών καρδίας, Ακούσατε τον λόγον του Κυρίου.
\par 3 Ούτω λέγει Κύριος ο Θεός· Ουαί εις τους προφήτας τους μωρούς, τους περιπατούντας οπίσω του πνεύματος αυτών, και δεν είδον ουδεμίαν όρασιν.
\par 4 Ισραήλ, οι προφήταί σου είναι ως αι αλώπεκες εν ταις ερήμοις.
\par 5 Δεν ανέβητε εις τας χαλάστρας ουδέ ανεγείρατε τα περιφράγματα υπέρ του οίκου Ισραήλ, διά να σταθή εν τη μάχη την ημέραν του Κυρίου.
\par 6 Είδον ματαιότητας και μαντείας ψευδείς, αίτινες λέγουσιν, Ο Κύριος λέγει· και ο Κύριος δεν απέστειλεν αυτούς· και έκαμον τους ανθρώπους να ελπίζωσιν ότι ο λόγος αυτών ήθελε πληρωθή.
\par 7 Δεν είδετε οράσεις ματαίας και ελαλήσατε μαντείας ψευδείς και λέγετε, Ο Κύριος είπεν, ενώ εγώ δεν ελάλησα;
\par 8 Όθεν ούτω λέγει Κύριος ο Θεός· Επειδή ελαλήσατε ματαιότητας και είδετε ψεύδη, διά τούτο, ιδού, εγώ είμαι εναντίον σας, λέγει Κύριος ο Θεός.
\par 9 Και η χειρ μου θέλει είσθαι επί τους προφήτας τους βλέποντας ματαιότητας και μαντεύοντας ψεύδη· δεν θέλουσιν είσθαι εν τη βουλή του λαού μου και εν τη καταγραφή του οίκου του Ισραήλ δεν θέλουσι καταγραφή ουδέ θέλουσιν εισέλθει εις γην Ισραήλ, και θέλετε γνωρίσει ότι εγώ είμαι Κύριος ο Θεός.
\par 10 Επειδή, ναι, επειδή επλάνησαν τον λαόν μου, λέγοντες, Ειρήνη· και δεν υπάρχει ειρήνη· και ο εις έκτιζε τοίχον και ιδού, οι άλλοι περιήλειφον αυτόν με πηλόν αμάλακτον·
\par 11 ειπέ προς τους αλείφοντας με πηλόν αμάλακτον, ότι θέλει πέσει· θέλει γείνει βροχή κατακλύζουσα· και σεις, λίθοι χαλάζης, θέλετε πέσει κατ' αυτού και άνεμος θυελλώδης θέλει σχίσει αυτόν.
\par 12 Ιδού, όταν ο τοίχος πέση, δεν θέλουσιν ειπεί προς εσάς, Που είναι αλοιφή, με την οποίαν ηλείψατε αυτόν;
\par 13 Διά τούτο, ούτω λέγει Κύριος ο Θεός· θέλω εξάπαντος σχίσει αυτόν εν τη οργή μου δι' ανέμου θυελλώδους· και εν τω θυμώ μου θέλει γείνει βροχή κατακλύζουσα και εν τη οργή μου λίθοι φοβεράς χαλάζης, διά να καταστρέψωσιν αυτόν.
\par 14 Και θέλω ανατρέψει τον τοίχον, τον οποίον ηλείψατε με πηλόν αμάλακτον και θέλω κατεδαφίσει αυτόν, και θέλουσιν ανακαλυφθή τα θεμέλια αυτού, και θέλει πέσει και σεις θέλετε συναπολεσθή εν μέσω αυτού, και θέλετε γνωρίσει ότι εγώ είμαι ο Κύριος.
\par 15 Και θέλω συντελέσει τον θυμόν μου επί τον τοίχον και επί τους αλείψαντας αυτόν με πηλόν αμάλακτον, και θέλω ειπεί προς εσάς, Ο τοίχος δεν υπάρχει ουδέ οι αλείψαντες αυτόν,
\par 16 οι προφήται του Ισραήλ, οι προφητεύοντες περί της Ιερουσαλήμ και βλέποντες οράματα ειρήνης περί αυτής, και δεν υπάρχει ειρήνη, λέγει Κύριος ο Θεός.
\par 17 Και συ, υιέ ανθρώπου, στήριξον το πρόσωπόν σου επί τας θυγατέρας του λαού σου, τας προφητευούσας εξ ιδίας αυτών καρδίας· και προφήτευσον κατ' αυτών,
\par 18 και ειπέ, Ούτω λέγει Κύριος ο Θεός· Ουαί εις εκείνας, αίτινες συρράπτουσι προσκεφάλαια διά πάντα αγκώνα χειρός και κάμνουσι καλύπτρας επί την κεφαλήν πάσης ηλικίας, διά να δελεάζωσι ψυχάς. Τας ψυχάς του λαού μου δελεάζετε και θέλετε σώσει τας εαυτών ψυχάς;
\par 19 Και θέλετε με βεβηλόνει μεταξύ του λαού μου διά μίαν δράκα κριθής και διά κομμάτια άρτου, ώστε να θανατόνητε ψυχάς αίτινες δεν έπρεπε να αποθάνωσι, και να σώζητε ψυχάς αίτινες δεν έπρεπε να ζώσι, ψευδόμεναι προς τον λαόν μου, τον ακούοντα ψεύδη;
\par 20 διά τούτο ούτω λέγει Κύριος ο Θεός· Ιδού, εγώ είμαι εναντίον εις τα προσκεφάλαιά σας, με τα οποία δελεάζετε τας ψυχάς, διά να πετώσι προς εσάς, και θέλω διαρρήξει αυτά από των βραχιόνων σας, και θέλω αφήσει τας ψυχάς να φύγωσι, τας ψυχάς τας οποίας σεις δελεάζετε διά να πετώσι προς εσάς.
\par 21 Και θέλω διαρρήξει τας καλύπτρας σας και ελευθερώσει τον λαόν μου εκ της χειρός σας, και δεν θέλουσιν είσθαι πλέον εις την χείρα σας διά να δελεάζωνται· και θέλετε γνωρίσει ότι εγώ είμαι ο Κύριος.
\par 22 Διότι με τα ψεύδη εθλίψατε την καρδίαν του δικαίου, τον οποίον εγώ δεν ελύπησα· και ενισχύσατε τας χείρας του κακούργου, ώστε να μη επιστρέψη από της οδού αυτού της πονηράς, διά να σώσω την ζωήν αυτού.
\par 23 Διά τούτο δεν θέλετε ιδεί πλέον ματαιότητα και δεν θέλετε μαντεύσει μαντείας· και θέλω ελευθερώσει τον λαόν μου εκ της χειρός σας· και θέλετε γνωρίσει έτι εγώ είμαι ο Κύριος.

\chapter{14}

\par 1 Και ήλθον προς εμέ τινές εκ των πρεσβυτέρων του Ισραήλ και εκάθησαν έμπροσθέν μου.
\par 2 Και έγεινε λόγος Κυρίου προς εμέ, λέγων,
\par 3 Υιέ ανθρώπου, οι άνδρες ούτοι ανεβίβασαν τα είδωλα αυτών εις τας καρδίας αυτών και έθεσαν το πρόσκομμα της ανομίας αυτών έμπροσθεν του προσώπου αυτών· ήθελον εκζητηθή τωόντι παρ' αυτών;
\par 4 Διά τούτο λάλησον προς αυτούς και ειπέ προς αυτούς, Ούτω λέγει Κύριος ο Θεός· εις πάντα άνθρωπον εκ του οίκου Ισραήλ, όστις αναβιβάση τα είδωλα αυτού εις την καρδίαν αυτού και θέση το πρόσκομμα της ανομίας αυτού έμπροσθεν του προσώπου αυτού και έλθη προς τον προφήτην, εγώ ο Κύριος θέλω αποκριθή προς αυτόν ερχόμενον, κατά το πλήθος των ειδώλων αυτού·
\par 5 διά να πιάσω τον οίκον Ισραήλ από της καρδίας αυτών, επειδή πάντες απηλλοτριώθησαν απ' εμού διά των ειδώλων αυτών.
\par 6 Διά τούτο ειπέ προς τον οίκον Ισραήλ, Ούτω λέγει Κύριος ο Θεός· Μετανοήσατε και επιστρέψατε από των ειδώλων σας και αποστρέψατε τα πρόσωπά σας από πάντων των βδελυγμάτων σας.
\par 7 Διότι εις πάντα άνθρωπον εκ του οίκου Ισραήλ και εκ των ξένων των παροικούντων εν τω Ισραήλ, όστις απαλλοτριωθή απ' εμού και αναβιβάση τα είδωλα αυτού εις την καρδίαν αυτού και θέση το πρόσκομμα της ανομίας αυτού έμπροσθεν του προσώπου αυτού και έλθη προς τον προφήτην διά να ερωτήση αυτόν περί εμού, εγώ ο Κύριος θέλω αποκριθή προς αυτόν περί εμού·
\par 8 και θέλω στήσει το πρόσωπόν μου εναντίον του ανθρώπου εκείνου και θέλω κάμει αυτόν σημείον και παροιμίαν και θέλω εκκόψει αυτόν εκ μέσου του λαού μου· και θέλετε γνωρίσει ότι εγώ είμαι ο Κύριος.
\par 9 Και εάν ο προφήτης πλανηθή και λαλήση λόγον, εγώ ο Κύριος επλάνησα τον προφήτην εκείνον· και θέλω εκτείνει την χείρα μου επ' αυτόν και εξολοθρεύσει αυτόν εκ μέσου του λαού μου Ισραήλ.
\par 10 Και θέλουσι λάβει την ποινήν της ανομίας αυτών· η ποινή του προφήτου θέλει είσθαι ως η ποινή του ερωτώντος·
\par 11 διά να μη αποπλανάται πλέον ο οίκος Ισραήλ απ' εμού, και να μη μιαίνωνται πλέον με πάσας τας παραβάσεις αυτών, αλλά να ήναι λαός μου και εγώ να ήμαι Θεός αυτών, λέγει Κύριος ο Θεός.
\par 12 Και έγεινε λόγος Κυρίου προς εμέ, λέγων,
\par 13 Υιέ ανθρώπου, όταν γη τις αμαρτήση εις εμέ με παράβασιν βαρείαν, τότε θέλω εκτείνει την χείρα μου επ' αυτήν και συντρίψει το υποστήριγμα του άρτου αυτής, και θέλω εξαποστείλει την πείναν εναντίον αυτής και εκκόψει άνθρωπον και κτήνος απ' αυτής·
\par 14 και εάν οι τρεις ούτοι άνδρες, Νώε, Δανιήλ και Ιώβ, ήσαν εν μέσω αυτής, μόνοι ούτοι ήθελον σώσει τας ψυχάς αυτών διά την δικαιοσύνην αυτών, λέγει Κύριος ο Θεός.
\par 15 Και εάν ήθελον επιφέρει κατά της γης θηρία κακά και έφθειρον αυτήν, ώστε να αφανισθή, ώστε να μη δύναταί τις να περάση δι' αυτής εξ αιτίας των θηρίων,
\par 16 και οι τρεις ούτοι άνδρες ευρίσκοντο εν μέσω αυτής, ζω εγώ, λέγει Κύριος ο Θεός, δεν ήθελον σώσει ούτε υιούς ούτε θυγατέρας· μόνοι ούτοι ήθελον σωθή, η δε γη ήθελεν αφανισθή.
\par 17 Η και εάν ήθελον επιφέρει ρομφαίαν επί την γην εκείνην και ειπεί, Ρομφαία, δίελθε διά της γης, ώστε να εκκόψω απ' αυτής άνθρωπον και κτήνος,
\par 18 και οι τρεις ούτοι άνδρες ευρίσκοντο εν μέσω αυτής, ζω εγώ, λέγει Κύριος ο Θεός, δεν ήθελον σώσει υιούς και θυγατέρας αλλ' αυτοί μόνοι ήθελον σωθή.
\par 19 Η εάν ήθελον επιφέρει θανατικόν επί την γην εκείνην και εκχέει την οργήν μου επ' αυτήν με αίμα, ώστε να εκκόψω απ' αυτής άνθρωπον και κτήνος,
\par 20 και ευρίσκοντο εν μέσω αυτής Νώε, Δανιήλ και Ιώβ, ζω εγώ, λέγει Κύριος ο Θεός, δεν ήθελον σώσει ούτε υιόν ούτε θυγατέρα· ούτοι μόνοι ήθελον σώσει τας ψυχάς αυτών διά την δικαιοσύνην αυτών.
\par 21 Διότι ούτω λέγει Κύριος ο Θεός· Πόσω μάλλον λοιπόν, όταν εξαποστείλω τας τέσσαρας δεινάς κρίσεις μου επί της Ιερουσαλήμ, την ρομφαίαν και την πείναν και τα κακά θηρία και το θανατικόν, ώστε να εκκόψω απ' αυτής άνθρωπον και κτήνος;
\par 22 Πλην ιδού, θέλουσι μένει εν αυτή λείψανα τινά, διασεσωσμένοι τινές, υιοί και θυγατέρες· ιδού, ούτοι θέλουσιν εξέλθει προς εσάς και θέλετε ιδεί τας οδούς αυτών και τας πράξεις αυτών· και θέλετε παρηγορηθή διά τα κακά, τα οποία επέφερα επί την Ιερουσαλήμ, διά πάντα όσα επέφερα επ' αυτήν.
\par 23 Και ούτοι θέλουσι σας παρηγορήσει, όταν ίδητε τας οδούς αυτών και τας πράξεις αυτών· και θέλετε γνωρίσει ότι εγώ δεν έκαμον χωρίς αιτίας πάντα όσα έκαμον εν αυτή, λέγει Κύριος ο Θεός.

\chapter{15}

\par 1 Και έγεινε λόγος Κυρίου προς εμέ, λέγων,
\par 2 Υιέ ανθρώπου, τι ήθελεν είσθαι το ξύλον της αμπέλου προς παν άλλο ξύλον, τα κλήματα προς παν ό,τι είναι εν τοις ξύλοις του δρυμού;
\par 3 Ήθελον λάβει απ' αυτής ξύλον διά να μεταχειρισθώσιν εις εργασίαν; ή ήθελον λάβει απ' αυτής πάσσαλον, διά να κρεμάσωσιν εις αυτόν σκεύος τι;
\par 4 Ιδού, ρίπτεται εις το πυρ διά να καταναλωθή· το πυρ κατατρώγει και τα δύο άκρα αυτού και το μέσον αυτού κατακαίεται. θέλει είσθαι χρήσιμον εις εργασίαν;
\par 5 Ιδού, ότε ήτο ακέραιον, δεν εχρησίμευεν εις εργασίαν· πόσον ολιγώτερον θέλει είσθαι χρήσιμον εις εργασίαν, αφού το πυρ κατέφαγεν αυτό και εκάη;
\par 6 Διά τούτο ούτω λέγει Κύριος ο Θεός· Καθώς είναι το ξύλον της αμπέλου εν τοις ξύλοις του δρυμού, το οποίον παρέδωκα εις το πυρ διά να καταναλωθή, ούτω θέλω παραδώσει τους κατοικούντας την Ιερουσαλήμ.
\par 7 Και θέλω στήσει το πρόσωπόν μου εναντίον αυτών· εκ του πυρός θέλουσιν εξέλθει και το πυρ θέλει καταφάγει αυτούς· και όταν στήσω το πρόσωπόν μου εναντίον αυτών, θέλετε γνωρίσει ότι εγώ είμαι ο Κύριος.
\par 8 Και θέλω παραδώσει την γην εις αφανισμόν, διότι έγειναν παραβάται, λέγει Κύριος ο Θεός.

\chapter{16}

\par 1 Και έγεινε λόγος Κυρίου προς εμέ, λέγων,
\par 2 Υιέ ανθρώπου, κάμε την Ιερουσαλήμ να γνωρίση τα βδελύγματα αυτής,
\par 3 και ειπέ, Ούτω λέγει Κύριος ο Θεός προς την Ιερουσαλήμ· Η ρίζα σου και η γέννησίς σου είναι εκ της γης των Χαναναίων· ο πατήρ σου Αμορραίος και η μήτηρ σου Χετταία.
\par 4 Εις δε την γέννησίν σου, καθ' ην ημέραν εγεννήθης, ο ομφαλός σου δεν εκόπη και εν ύδατι δεν ελούσθης, διά να καθαρισθής, και με άλας δεν ηλατίσθης και εν σπαργάνοις δεν εσπαργανώθης.
\par 5 Οφθαλμός δεν σε εφείσθη, διά να κάμη εις σε τι εκ τούτων, ώστε να σε σπλαγχνισθή· αλλ' ήσο απερριμμένη εις το πρόσωπον της πεδιάδος, εν τη αποστροφή της ψυχής σου, καθ' ην ημέραν εγεννήθης.
\par 6 Και ότε διέβην από πλησίον σου και σε είδον κυλιομένην εν τω αίματί σου, είπα προς σε ευρισκομένην εν τω αίματί σου, Ζήθι· ναι, είπα προς σε ευρισκομένην εν τω αίματί σου, Ζήθι.
\par 7 Και σε έκαμον μυριοπλάσιον, ως την χλόην του αγρού, και ηυξήνθης και εμεγαλύνθης και έφθασας εις το άκρον της ώραιότητος· οι μαστοί σου εμορφώθησαν και αι τρίχες σου ανεφύησαν· ήσο όμως γυμνή και ασκέπαστος.
\par 8 Και ότε διέβην από πλησίον σου και σε είδον, ιδού, η ηλικία σου ήτο ηλικία έρωτος· και απλώσας το κράσπεδόν μου επί σε, εσκέπασα την ασχημοσύνην σου· και ώμοσα προς σε και εισήλθον εις συνθήκην μετά σου, λέγει Κύριος ο Θεός, και έγεινες εμού.
\par 9 Και σε έλουσα εν ύδατι και απέπλυνα το αίμα σου από σου και σε έχρισα εν ελαίω.
\par 10 Και σε ενέδυσα κεντητά και σε υπέδησα με σανδάλια υακίνθινα και σε περιέζωσα με βύσσον και σε εφόρεσα μεταξωτά.
\par 11 Και σε εστόλισα με στολίδια και περιέθεσα εις τας χείρας σου βραχιόλια και περιδέραιον επί τον τράχηλόν σου.
\par 12 Και έβαλον έρρινα εις τους μυκτήράς σου και ενώτια εις τα ώτα σου και στέφανον δόξης επί την κεφαλήν σου.
\par 13 Και εστολίσθης με χρυσίον και αργύριον, και τα ιμάτιά σου ήσαν βύσσινα και μεταξωτά και κεντητά· σεμίδαλιν και μέλι και έλαιον έτρωγες· και έγεινες ώραία σφόδρα και ευημέρησας μέχρι βασιλείας.
\par 14 Και εξήλθεν η φήμη σου μεταξύ των εθνών διά το κάλλος σου· διότι ήτο τέλειον διά του στολισμού μου, τον οποίον έθεσα επί σε, λέγει Κύριος ο Θεός.
\par 15 Συ όμως εθαρρεύθης εις το κάλλος σου, και επορνεύθης διά την φήμην σου και εξέχεας την πορνείαν σου εις πάντα διαβάτην, γινομένη αυτού.
\par 16 Και έλαβες εκ των ιματίων σου και εστόλισας τους υψηλούς σου τόπους με ποικίλα χρώματα και εξεπορνεύθης απ' αυτών· τοιαύτα δεν έγειναν ουδέ θέλουσι γείνει.
\par 17 Και έλαβες τα σκεύη της λαμπρότητός σου, τα εκ του χρυσίου μου και τα εκ του αργυρίου μου, τα οποία έδωκα εις σε, και έκαμες εις σεαυτήν εικόνας αρσενικάς και εξεπορνεύθης με αυτάς·
\par 18 και έλαβες τα κεντητά σου ιμάτια και εσκέπασας αυτάς· και έθεσας έμπροσθεν αυτών το έλαιόν μου και το θυμίαμά μου.
\par 19 Και τον άρτον μου, τον οποίον έδωκα εις σε, την σεμίδαλιν και το έλαιον και το μέλι, με τα οποία σε έτρεφον, έθεσας και ταύτα έμπροσθεν αυτών εις οσμήν ευωδίας· ούτως έγεινε, λέγει Κύριος ο Θεός.
\par 20 Και έλαβες τους υιούς σου και τας θυγατέρας σου, τας οποίας εγέννησας εις εμέ, και ταύτα εθυσίασας εις αυτάς, διά να αναλωθώσιν εν τω πυρί· μικρόν έργον των πορνεύσεών σου ήτο τούτο,
\par 21 ότι έσφαξας τα τέκνα μου και παρέδωκας αυτά διά να διαβιβάσωσιν αυτά διά του πυρός εις τιμήν αυτών;
\par 22 Και εν πάσι τοις βδελύγμασί σου και ταις πορνείαις σου δεν ενεθυμήθης ταις ημέρας της νεότητός σου, ότε ήσο γυμνή και ασκέπαστος, κυλιομένη εν τω αίματί σου.
\par 23 Και μετά πάσας τας κακίας σου, Ουαί, ουαί εις σε, λέγει Κύριος ο Θεός,
\par 24 έκτισας και εις σεαυτήν οίκημα πορνικόν και έκαμες εις σεαυτήν πορνοστάσιον εν πάση πλατεία.
\par 25 Εις πάσαν αρχήν οδού ωκοδόμησας το πορνοστάσιόν σου και έκαμες το κάλλος σου βδελυκτόν και ήνοιξας τους πόδας σου εις πάντα διαβάτην, και επλήθυνας την πορνείαν σου.
\par 26 Και εξεπορνεύθης με τους Αιγυπτίους τους πλησιοχώρους σου, τους μεγαλοσάρκους· και επολλαπλασίασας την πορνείαν σου, διά να με παροργίσης.
\par 27 Ιδού λοιπόν, εξήπλωσα την χείρα μου επί σε, και αφήρεσα τα νενομισμένα σου, και σε παρέδωκα εις την θέλησιν εκείνων αίτινες σε εμίσουν, των θυγατέρων των Φιλισταίων, αίτινες εντρέπονται διά την οδόν σου την αισχράν.
\par 28 Και εξεπορνεύθης με τους Ασσυρίους, διότι ήσο άπληστος· ναι, εξεπορνεύθης με αυτούς και έτι δεν εχορτάσθης.
\par 29 Και επολλαπλασίασας την πορνείαν σου εν γη Χαναάν μέχρι των Χαλδαίων· και ουδέ ούτως εχορτάσθης.
\par 30 Πόσον διεφθάρη η καρδία σου, λέγει Κύριος ο Θεός, επειδή πράττεις πάντα ταύτα, έργα της πλέον αναισχύντου πόρνης.
\par 31 Διότι έκτισας το πορνικόν οίκημά σου εν τη αρχή πάσης οδού, και έκαμες το πορνοστάσιόν σου εν πάση πλατεία· και δεν εστάθης ως πόρνη, καθότι κατεφρόνησας μίσθωμα,
\par 32 αλλ' ως γυνή μοιχαλίς, αντί του ανδρός αυτής δεχομένη ξένους.
\par 33 Εις πάσας τας πόρνας δίδουσι μίσθωμα· αλλά συ τα μισθώματά σου δίδεις εις πάντας τους εραστάς σου και διαφθείρεις αυτούς, διά να εισέρχωνται προς σε πανταχόθεν επί τη πορνεία σου.
\par 34 Και γίνεται εις σε το ανάπαλιν των άλλων γυναικών εν ταις πορνείαις σου· διότι δεν σε ακολουθεί ουδείς διά να πράξη πορνείαν· καθότι συ δίδεις μίσθωμα και μίσθωμα δεν δίδεται εις σε, κατά τούτο γίνεται εις σε το ανάπαλιν.
\par 35 Διά τούτο, άκουσον, πόρνη, τον λόγον του Κυρίου·
\par 36 ούτω λέγει Κύριος ο Θεός· Επειδή εξέχεας τον χαλκόν σου, και η γύμνωσίς σου εξεσκεπάσθη εν ταις πορνείαις σου προς τους εραστάς σου και προς πάντα τα είδωλα των βδελυγμάτων σου, και διά το αίμα των τέκνων σου, τα οποία προσέφερες εις αυτά·
\par 37 διά τούτο ιδού, εγώ συνάγω πάντας τους εραστάς σου, μεθ' ων κατετρύφησας, και πάντας όσους ηγάπησας, μετά πάντων των μισηθέντων υπό σού· και θέλω συνάξει αυτούς επί σε πανταχόθεν και θέλω αποκαλύψει την αισχύνην σου εις αυτούς, και θέλουσιν ιδεί όλην την γύμνωσίν σου.
\par 38 Και θέλω σε κρίνει κατά την κρίσιν των μοιχαλίδων και εκχεουσών αίμα· και θέλω σε παραδώσει εις αίμα μετ' οργής και ζηλοτυπίας.
\par 39 Και θέλω σε παραδώσει εις την χείρα αυτών· και θέλουσι κατασκάψει το πορνικόν οίκημά σου και κατεδαφίσει τους υψηλούς τόπους σου θέλουσιν ότι σε εκδύσει τα ιμάτιά σου και αφαιρέσει τους στολισμούς της λαμπρότητός σου και θέλουσι σε αφήσει γυμνήν και ασκέπαστον.
\par 40 Και θέλουσι φέρει επί σε όχλους, οίτινες θέλουσι σε λιθοβολήσει με λίθους και σε διαπεράσει με τα ξίφη αυτών.
\par 41 Και θέλουσι κατακαύσει εν πυρί τας οικίας σου, και θέλουσιν εκτελέσει επί σε κρίσεις ενώπιον πολλών γυναικών· και θέλω σε κάμει να παύσης από της πορνείας, και δεν θέλεις δίδει του λοιπού μίσθωμα.
\par 42 Και θέλω αναπαύσει τον θυμόν μου επί σε, και η ζηλοτυπία μου θέλει σηκωθή από σου, και θέλω ησυχάσει και δεν θέλω οργισθή πλέον.
\par 43 Επειδή δεν ενεθυμήθης τας ημέρας της νεότητός σου, αλλά με παρώξυνας εν πάσι τούτοις, διά τούτο ιδού, και εγώ θέλω ανταποδώσει τας οδούς σου επί της κεφαλής σου, λέγει Κύριος ο Θεός· και δεν θέλεις κάμει κατά την ασέβειαν ταύτην επί πάσι τοις βδελύγμασί σου.
\par 44 Ιδού, πας ο παροιμιαζόμενος θέλει παροιμιασθή κατά σου, λέγων, κατά την μητέρα η θυγάτηρ αυτής.
\par 45 Συ είσαι η θυγάτηρ της μητρός σου, της αποβαλούσης τον άνδρα αυτής και τα τέκνα αυτής· και είσαι η αδελφή των αδελφών σου, αίτινες απέβαλον τους άνδρας αυτών και τα τέκνα αυτών· η μήτηρ σας ήτο Χετταία και ο πατήρ σας Αμορραίος.
\par 46 Και η αδελφή σου η πρεσβυτέρα είναι η Σαμάρεια, αυτή και αι θυγατέρες αυτής, αι κατοικούσαι εν τοις αριστεροίς σου· η δε νεωτέρα αδελφή σου, η κατοικούσα εν τοις δεξιοίς σου, τα Σόδομα και αι θυγατέρες αυτής.
\par 47 Συ όμως δεν περιεπάτησας κατά τας οδούς αυτών και δεν έπραξας κατά τα βδελύγματα αυτών· αλλ' ως εάν ήτο τούτο πολύ μικρόν, υπερέβης αυτών την διαφθοράν εν πάσαις ταις οδοίς σου.
\par 48 Ζω εγώ, λέγει Κύριος ο Θεός, η αδελφή σου Σόδομα δεν έπραξεν, αυτή και αι θυγατέρες αυτής, ως έπραξας συ και αι θυγατέρες σου.
\par 49 Ιδού, αύτη ήτο η ανομία της αδελφής σου Σοδόμων, υπερηφανία, πλησμονή άρτου και αφθονία τρυφηλότητος, αυτής και των θυγατέρων αυτής· τον πτωχόν δε και τον ενδεή δεν εβοήθει
\par 50 και υψούντο και έπραττον βδελυρά ενώπιόν μου· όθεν, καθώς είδον ταύτα, ηφάνισα αυτάς.
\par 51 Και η Σαμάρεια δεν ημάρτησεν ουδέ το ήμισυ των αμαρτημάτων σου· αλλά συ επλήθυνας τα βδελύγματά σου υπέρ εκείνας και εδικαίωσας τας αδελφάς σου με πάντα τα βδελύγματά σου, τα οποία έπραξας.
\par 52 Συ λοιπόν, ήτις έκρινες τας αδελφάς σου, βάσταζε την καταισχύνην σου· ένεκα των αμαρτημάτων σου, με τα οποία κατεστάθης βδελυρωτέρα εκείνων, εκείναι είναι δικαιότεραί σου· όθεν αισχύνθητι και συ και βάσταζε την καταισχύνην σου, ότι εδικαίωσας τας αδελφάς σου.
\par 53 Όταν φέρω οπίσω τους αιχμαλώτους αυτών, τους αιχμαλώτους Σοδόμων και των θυγατέρων αυτής και τους αιχμαλώτους της Σαμαρείας και των θυγατέρων αυτής, τότε θέλω επιστρέψει και τους αιχμαλώτους της αιχμαλωσίας σου μεταξύ αυτών·
\par 54 διά να βαστάζης την ατιμίαν σου και να καταισχύνησαι διά πάντα όσα έπραξας και να ήσαι παρηγορία εις αυτάς.
\par 55 Όταν η αδελφή σου Σόδομα και αι θυγατέρες αυτής επιστρέψωσιν εις την προτέραν αυτών κατάστασιν, και η Σαμάρεια και αι θυγατέρες αυτής επιστρέψωσιν εις την προτέραν αυτών κατάστασιν, τότε θέλεις επιστρέψει συ και αι θυγατέρες σου εις την προτέραν σας κατάστασιν.
\par 56 Διότι η αδελφή σου Σόδομα δεν ανεφέρθη εκ του στόματός σου εν ταις ημέραις της υπερηφανίας σου,
\par 57 πριν ανακαλυφθή η κακία σου, καθώς ανεκαλύφθη εν καιρώ του γενομένου εις σε ονείδους υπό των θυγατέρων της Συρίας και πασών των πέριξ αυτής, των θυγατέρων των Φιλισταίων, αίτινες σε ελεηλάτησαν πανταχόθεν.
\par 58 Συ εβάστασας την ασέβειάν σου και τα βδελύγματά σου, λέγει Κύριος.
\par 59 Διότι ούτω λέγει Κύριος ο Θεός· Εγώ θέλω κάμει εις σε καθώς έκαμες συ, ήτις κατεφρόνησας τον όρκον, παραβαίνουσα την διαθήκην.
\par 60 Αλλ' όμως θέλω ενθυμηθή την διαθήκην μου την γενομένην προς σε εν ταις ημέραις της νεότητός σου, και θέλω στήσει εις σε διαθήκην αιώνιον.
\par 61 Τότε θέλεις ενθυμηθή τας οδούς σου και αισχυνθή, όταν δεχθής τας αδελφάς σου, τας πρεσβυτέρας σου και τας νεωτέρας σου· και θέλω δώσει αυτάς εις σε διά θυγατέρας, ουχί όμως κατά την διαθήκην σου.
\par 62 Και εγώ θέλω στήσει την διαθήκην μου προς σε, και θέλεις γνωρίσει έτι εγώ είμαι ο Κύριος·
\par 63 διά να ενθυμηθής, και να αισχυνθής και να μη ανοίξης πλέον το στόμα σου υπό της εντροπής σου, όταν εξιλεωθώ προς σε διά πάντα όσα έπραξας, λέγει Κύριος ο Θεός.

\chapter{17}

\par 1 Και έγεινε λόγος Κυρίου προς εμέ, λέγων,
\par 2 Υιέ ανθρώπου, πρόβαλε αίνιγμα και παροιμιάσθητι παροιμίαν προς τον οίκον Ισραήλ·
\par 3 και ειπέ, Ούτω λέγει Κύριος ο Θεός· Ο αετός ο μέγας ο μεγαλοπτέρυγος, ο μακρός εις την έκτασιν, ο πλήρης πτερών ποικιλοχρόων, ήλθεν εις τον Λίβανον και έλαβε τον υψηλότερον κλάδον της κέδρου·
\par 4 απέκοψε τα άκρα των τρυφερών αυτού κλάδων και έφερεν αυτά εις γην εμπορικήν· έθεσεν αυτά εις πόλιν εμπόρων.
\par 5 Και έλαβεν από του σπέρματος της γης και έθεσεν αυτό εις πεδίον σπόριμον· πλησίον πολλών υδάτων έφερεν αυτό· ως ιτέαν έθεσεν αυτό.
\par 6 Και εβλάστησε και έγεινεν άμπελος πλατεία, χαμηλή εις το ανάστημα, της οποίας τα κλήματα εστρέφοντο προς αυτόν και αι ρίζαι αυτής ήσαν υποκάτω αυτού· και έγεινεν άμπελος και έκαμε κλήματα και εξέδωκε βλαστούς.
\par 7 Ήτο και άλλος αετός μέγας, ο μεγαλοπτέρυγος και πολύπτερος· και ιδού, η άμπελος αύτη εξέτεινε τας ρίζας αυτής προς αυτόν, και ήπλωσε τους κλάδους αυτής προς αυτόν, διά να ποτίση αυτήν διά των αυλακίων της φυτεύσεως αυτής.
\par 8 Ήτο πεφυτευμένη εν γη καλή πλησίον υδάτων πολλών, διά να κάμη βλαστούς και να φέρη καρπόν, ώστε να γείνη άμπελος αγαθή.
\par 9 Ειπέ, Ούτω λέγει Κύριος ο Θεός· θέλει ευοδωθή; δεν θέλει ανασπάσει αυτός τας ρίζας αυτής και κόψει τον καρπόν αυτής, ώστε να ξηρανθή; θέλει ξηρανθή κατά πάντα τα φύλλα του βλαστήματος αυτής, χωρίς μάλιστα μεγάλης δυνάμεως ή πολλού λαού, διά να εκσπάση αυτήν εκ των ριζών αυτής.
\par 10 Ναι, ιδού, φυτευθείσα θέλει ευοδωθή; δεν θέλει ξηρανθή ολοκλήρως, ως όταν εγγίση αυτήν ο ανατολικός άνεμος; θέλει ξηρανθή εν ταις αύλαξιν όπου εβλάστησε.
\par 11 Και έγεινε λόγος Κυρίου προς εμέ, λέγων,
\par 12 Ειπέ τώρα προς τον οίκον τον αποστάτην· δεν εννοείτε τι δηλούσι ταύτα; ειπέ, Ιδού, ο βασιλεύς της Βαβυλώνος ήλθεν εις Ιερουσαλήμ, και έλαβε τον βασιλέα αυτής και τους άρχοντας αυτής, και έφερεν αυτούς μεθ' εαυτού εις Βαβυλώνα·
\par 13 και έλαβεν από του σπέρματος του βασιλικού και έκαμε συνθήκην μετ' αυτού και έκαμεν αυτόν να ορκισθή· έλαβε και τους δυνατούς του τόπου,
\par 14 διά να ταπεινωθή το βασίλειον, ώστε να μη ανορθωθή, διά να φυλάττη την συνθήκην αυτού, ώστε να στηρίζη αυτήν.
\par 15 Απεστάτησεν όμως απ' αυτού, εξαποστείλας πρέσβεις εαυτού εις την Αίγυπτον, διά να δώσωσιν εις αυτόν ίππους και λαόν πολύν. Θέλει ευοδωθή; θέλει διασωθή ο πράττων ταύτα; ή παραβαίνων την συνθήκην θέλει διασωθή;
\par 16 Ζω εγώ, λέγει Κύριος ο Θεός, βεβαίως εν τω τόπω του βασιλέως του βασιλεύσαντος αυτόν, του οποίου τον όρκον κατεφρόνησε και του οποίου την συνθήκην παρέβη, μετ' αυτού εν μέσω της Βαβυλώνος θέλει τελευτήσει.
\par 17 Και δεν θέλει κάμει υπέρ αυτού ουδέν εν τω πολέμω ο Φαραώ, με το δυνατόν στράτευμα και με το μέγα πλήθος, υψόνων προχώματα και οικοδομών προμαχώνας, διά να απολέση πολλάς ψυχάς.
\par 18 Διότι κατεφρόνησε τον όρκον παραβαίνων την συνθήκην· και ιδού, επειδή, αφού έδωκε την χείρα αυτού, έπραξε πάντα ταύτα, δεν θέλει διασωθή.
\par 19 Διά τούτο ούτω λέγει Κύριος ο Θεός· Ζω εγώ, βεβαίως τον όρκον μου τον οποίον κατεφρόνησε, και την συνθήκην μου την οποίαν παρέβη, κατά της κεφαλής αυτού θέλω ανταποδώσει αυτά.
\par 20 Και θέλω εξαπλώσει το δίκτυόν μου επ' αυτόν και θέλει πιασθή εις τα βρόχιά μου· και θέλω φέρει αυτόν εις Βαβυλώνα, και εκεί θέλω κριθή μετ' αυτού περί της ανομίας αυτού, την οποίαν ηνόμησεν εις εμέ.
\par 21 Και πάντες οι φυγάδες αυτού μετά πάντων των ταγμάτων αυτού θέλουσι πέσει εν μαχαίρα, και οι εναπολειφθέντες θέλουσι διασκορπισθή εις πάντα άνεμον· και θέλετε γνωρίσει ότι εγώ ο Κύριος ελάλησα.
\par 22 Ούτω λέγει Κύριος ο Θεός· Και θέλω λάβει εγώ εκ του υψηλοτέρου κλάδου της υψηλής κέδρου και φυτεύσει· θέλω κόψει εγώ εκ της κορυφής των νέων αυτού κλώνων ένα τρυφερόν και φυτεύσει επί όρους υψηλού και εξόχου·
\par 23 επί του υψηλού όρους του Ισραήλ θέλω φυτεύσει αυτόν, και θέλει εκφέρει κλάδους και καρποφορήσει και θέλει γείνει κέδρος μεγάλη και υποκάτω αυτής θέλουσι κατασκηνώσει παν όρνεον και παν πτηνόν· υπό την σκιάν των κλάδων αυτής θέλουσι κατασκηνώσει.
\par 24 Και πάντα τα δένδρα του αγρού θέλουσι γνωρίσει, ότι εγώ ο Κύριος εταπείνωσα το δένδρον το υψηλόν, ύψωσα το δένδρον το ταπεινόν, κατεξήρανα το δένδρον το χλωρόν, και έκαμον το δένδρον το ξηρόν να αναθάλλη. Εγώ ο Κύριος ελάλησα και εξετέλεσα.

\chapter{18}

\par 1 Και έγεινε λόγος Κυρίου προς εμέ, λέγων,
\par 2 Τι εννοείτε σεις, οι παροιμιαζόμενοι την παροιμίαν ταύτην περί της γης του Ισραήλ, λέγοντες, Οι πατέρες έφαγον όμφακα και οι οδόντες των τέκνων ημωδίασαν;
\par 3 Ζω εγώ, λέγει Κύριος ο Θεός, δεν θέλετε πλέον παροιμιασθή την παροιμίαν ταύτην εν τω Ισραήλ.
\par 4 Ιδού, πάσαι αι ψυχαί είναι εμού· ως η ψυχή του πατρός, ούτω και η ψυχή του υιού εμού είναι· ψυχή η αμαρτήσασα, αυτή θέλει αποθάνει.
\par 5 Αλλ' όστις είναι δίκαιος και πράττει κρίσιν και δικαιοσύνην,
\par 6 δεν τρώγει επί των ορέων και δεν σηκόνει τους οφθαλμούς αυτού προς τα είδωλα του οίκου Ισραήλ, και δεν μιαίνει την γυναίκα του πλησίον αυτού και δεν πλησιάζει εις γυναίκα ούσαν εν τη ακαθαρσία αυτής,
\par 7 και δεν καταδυναστεύει άνθρωπον, επιστρέφει εις τον χρεωφειλέτην το ενέχυρον αυτού, δεν αρπάζει βιαίως, δίδει τον άρτον αυτού εις τον πεινώντα και καλύπτει με ιμάτιον τον γυμνόν,
\par 8 δεν δίδει επί τόκω και δεν λαμβάνει προσθήκην, αποστρέφει την χείρα αυτού από αδικίας, κάμνει δικαίαν κρίσιν αναμέσον ανθρώπου και ανθρώπου,
\par 9 περιπατεί εν τοις διατάγμασί μου και φυλάττει τας κρίσεις μου, διά να κάμνη αλήθειαν, ούτος είναι δίκαιος, θέλει βεβαίως ζήσει, λέγει Κύριος ο Θεός.
\par 10 Εάν όμως γεννήση υιόν κλέπτην, χύνοντα αίμα και πράττοντά τι εκ των τοιούτων,
\par 11 και όστις δεν κάμνει πάντα ταύτα, αλλά και επί των ορέων τρώγει και την γυναίκα του πλησίον αυτού μιαίνει,
\par 12 τον πτωχόν και ενδεή καταδυναστεύει, αρπάζει βιαίως, δεν επιστρέφει το ενέχυρον και σηκόνει τους οφθαλμούς αυτού προς τα είδωλα και πράττει βδελύγματα,
\par 13 δίδει επί τόκω και λαμβάνει προσθήκην, ούτος θέλει ζήσει; δεν θέλει ζήσει· πάντα ταύτα τα βδελύγματα έπραξεν· εξάπαντος θέλει θανατωθή· το αίμα αυτού θέλει είσθαι επ' αυτόν.
\par 14 Εάν δε γεννήση υιόν, όστις βλέπων πάντα τα αμαρτήματα του πατρός αυτού, τα οποία έπραξε, προσέχει και δεν πράττει τοιαύτα,
\par 15 δεν τρώγει επί των ορέων και δεν σηκόνει τους οφθαλμούς αυτού προς τα είδωλα του οίκου Ισραήλ και δεν μιαίνει την γυναίκα του πλησίον αυτού,
\par 16 και δεν καταδυναστεύει άνθρωπον, δεν κατακρατεί το ενέχυρον και δεν αρπάζει βιαίως, δίδει τον άρτον αυτού εις τον πεινώντα και καλύπτει με ιμάτιον τον γυμνόν,
\par 17 αποστρέφει την χείρα αυτού από του πτωχού, τόκον και προσθήκην δεν λαμβάνει, εκτελεί τας κρίσεις μου, περιπατεί εν τοις διατάγμασί μου, ούτος δεν θέλει θανατωθή διά την ανομίαν του πατρός αυτού, εξάπαντος θέλει ζήσει.
\par 18 Ο πατήρ αυτού, επειδή σκληρώς κατεδυνάστευσεν, ήρπασε βιαίως τον αδελφόν αυτού και έπραξε μεταξύ του λαού αυτού ό,τι δεν είναι καλόν, ιδού, ούτος θέλει αποθάνει εν τη ανομία αυτού.
\par 19 Σεις όμως λέγετε, Διά τι; ο υιός δεν βαστάζει την ανομίαν του πατρός; Αφού ο υιός έκαμε κρίσιν και δικαιοσύνην, και εφύλαξε πάντα τα διατάγματά μου και εξετέλεσεν αυτά, εξάπαντος θέλει ζήσει.
\par 20 Η ψυχή η αμαρτάνουσα, αυτή θέλει αποθάνει· ο υιός δεν θέλει βαστάσει την ανομίαν του πατρός και ο πατήρ δεν θέλει βαστάσει την ανομίαν του υιού· η δικαιοσύνη του δικαίου θέλει είσθαι επ' αυτόν και η ανομία του ανόμου θέλει είσθαι επ' αυτόν.
\par 21 Αλλ' εάν ο άνομος επιστραφή από πασών των αμαρτιών αυτού, τας οποίας έπραξε, και φυλάξη πάντα τα διατάγματά μου και πράξη κρίσιν και δικαιοσύνην, εξάπαντος θέλει ζήσει, δεν θέλει αποθάνει·
\par 22 πάσαι αι ανομίαι αυτού, τας οποίας έπραξε, δεν θέλουσι μνημονευθή εις αυτόν· εν τη δικαιοσύνη αυτού, την οποίαν έπραξε, θέλει ζήσει.
\par 23 Μήπως εγώ θέλω τωόντι τον θάνατον του ανόμου, λέγει Κύριος ο Θεός, και ουχί το να επιστρέψη από των οδών αυτού και να ζήση;
\par 24 Όταν όμως ο δίκαιος επιστραφή από της δικαιοσύνης αυτού και πράξη αδικίαν και πράξη κατά πάντα τα βδελύγματα τα οποία ο άνομος πράττει, τότε θέλει ζήσει; Πάσα η δικαιοσύνη αυτού, την οποίαν έκαμε, δεν θέλει μνημονευθή· εν τη ανομία αυτού την οποίαν ηνόμησε και εν τη αμαρτία αυτού, την οποίαν ημάρτησεν, εν αυταίς θέλει αποθάνει.
\par 25 Σεις όμως λέγετε, Η οδός του Κυρίου δεν είναι ευθεία. Ακούσατε τώρα, οίκος Ισραήλ· Η οδός μου δεν είναι ευθεία; ουχί αι οδοί υμών διεστραμμέναι;
\par 26 Όταν ο δίκαιος επιστραφή από της δικαιοσύνης αυτού και πράξη αδικίαν και αποθάνη εν αυτή, διά την αδικίαν αυτού την οποίαν έπραξε θέλει αποθάνει.
\par 27 Και όταν ο άνομος επιστραφή από της ανομίας αυτού, την οποίαν έπραξε, και πράξη κρίσιν και δικαιοσύνην, ούτος θέλει φυλάξει ζώσαν την ψυχήν αυτού.
\par 28 Επειδή εσυλλογίσθη και επέστρεψεν από πασών των ανομιών αυτού, τας οποίας έπραξε, θέλει εξάπαντος ζήσει, δεν θέλει αποθάνει.
\par 29 Αλλ' ο οίκος Ισραήλ λέγει, Η οδός του Κυρίου δεν είναι ευθεία· οίκος Ισραήλ, αι οδοί μου δεν είναι ευθείαι; ουχί αι οδοί υμών διεστραμμέναι;
\par 30 Διά τούτο, οίκος Ισραήλ, θέλω σας κρίνει, έκαστον κατά τας οδούς αυτού, λέγει Κύριος ο Θεός. Μετανοήσατε και επιστρέψατε από πασών των ανομιών υμών, και δεν θέλει είσθαι εις εσάς η ανομία εις απώλειαν.
\par 31 Απορρίψατε αφ' υμών πάσας τας ανομίας υμών, τας οποίας ηνομήσατε εις εμέ, και κάμετε εις εαυτούς νέαν καρδίαν και νέον πνεύμα· και διά τι να αποθάνητε, οίκος Ισραήλ;
\par 32 Διότι εγώ δεν θέλω τον θάνατον του αποθνήσκοντος, λέγει Κύριος ο Θεός· διά τούτο επιστρέψατε και ζήσατε.

\chapter{19}

\par 1 Και συ ανάλαβε θρήνον διά τους ηγεμόνας του Ισραήλ,
\par 2 και ειπέ, Τι είναι η μήτηρ σου; Λέαινα· κείται μεταξύ λεόντων, έθρεψε τα βρέφη αυτής εν μέσω σκύμνων.
\par 3 Και ανέθρεψεν εν εκ των βρεφών αυτής και έγεινε σκύμνος και έμαθε να αρπάζη το θήραμα· ανθρώπους έτρωγε.
\par 4 Και τα έθνη ήκουσαν περί αυτού· επιάσθη εν τω λάκκω αυτών, και έφεραν αυτόν με αλύσεις εις την γην της Αιγύπτου.
\par 5 Και ιδούσα ότι η ελπίς αυτής εματαιώθη και εχάθη, έλαβεν εν άλλο εκ των βρεφών αυτής και έκαμεν αυτό σκύμνον.
\par 6 Και αναστρεφόμενον εν μέσω των λεόντων έγεινε σκύμνος και έμαθε να αρπάζη θήραμα· ανθρώπους έτρωγε.
\par 7 Και εγνώρισε τα παλάτια αυτών και ερήμονε τας πόλεις αυτών· και ήτο ηφανισμένη η γη και το πλήρωμα αυτής από του ήχου του βρυχήματος αυτού.
\par 8 Και τα έθνη παρετάχθησαν εναντίον αυτού κυκλόθεν εκ των επαρχιών και ήπλωσαν κατ' αυτού τα βρόχια αυτών, και επιάσθη εν τω λάκκω αυτών.
\par 9 Και έβαλον αυτόν με αλύσεις εις κλωβίον και έφεραν αυτόν προς τον βασιλέα της Βαβυλώνος· εν δεσμωτηρίω εισήγαγον αυτόν, διά να μη ακουσθή πλέον φωνή αυτού επί τα όρη του Ισραήλ.
\par 10 Η μήτηρ σου, καθ' ομοίωσίν σου, ήτο ως άμπελος πεφυτευμένη πλησίον των υδάτων· έγεινε καρποφόρος και πλήρης κλάδων διά τα πολλά ύδατα.
\par 11 Και έγειναν εις αυτήν ράβδοι ισχυραί διά σκήπτρα των κρατούντων· και ο κορμός αυτής υψώθη εν μέσω των πυκνών κλάδων, και έγεινε περίβλεπτος κατά το ύψος αυτής μεταξύ του πλήθους των βλαστών αυτής.
\par 12 Απεσπάσθη όμως μετά θυμού, ερρίφθη κατά γης, και ανατολικός άνεμος κατεξήρανε τον καρπόν αυτής· αι ισχυραί αυτής ράβδοι συνεθλάσθησαν και εξηράνθησαν· πυρ κατέφαγεν αυτάς.
\par 13 Και τώρα είναι πεφυτευμένη εν ερήμω, εν ξηρά και ανύδρω γη.
\par 14 Και εξήλθε πυρ από ράβδου τινός εκ των κλάδων αυτής και κατέφαγε τον καρπόν αυτής, ώστε δεν υπήρχε πλέον εν αυτή ράβδος ισχυρά διά σκήπτρον ηγεμονίας· ούτος είναι ο θρήνος και θέλει είσθαι εις θρήνον.

\chapter{20}

\par 1 Και εν τω εβδόμω έτει, τω πέμπτω μηνί, τη δεκάτη του μηνός, ήλθον τινές εκ των πρεσβυτέρων του Ισραήλ διά να επερωτήσωσι τον Κύριον, και εκάθησαν έμπροσθέν μου.
\par 2 Και έγεινε λόγος Κυρίου προς εμέ, λέγων,
\par 3 Υιέ ανθρώπου, λάλησον προς τους πρεσβυτέρους του Ισραήλ και ειπέ προς αυτούς, Ούτω λέγει Κύριος ο Θεός· Ήλθετε διά να με επερωτήσητε; Ζω εγώ, λέγει Κύριος ο Θεός, δεν θέλω επερωτηθή από σας.
\par 4 Θέλεις κρίνει αυτούς; υιέ ανθρώπου, θέλεις κρίνει; δείξον εις αυτούς τα βδελύγματα των πατέρων αυτών·
\par 5 και ειπέ προς αυτούς, Ούτω λέγει Κύριος ο Θεός· Εν τη ημέρα καθ' ην εξέλεξα τον Ισραήλ και ύψωσα την χείρα μου προς το σπέρμα του οίκου Ιακώβ και εγνωρίσθην εις αυτούς εν Αιγύπτω και ύψωσα την χείρα μου προς αυτούς, λέγων, Εγώ είμαι Κύριος ο Θεός σας,
\par 6 εν εκείνη τη ημέρα ύψωσα την χείρα μου προς αυτούς ότι θέλω εξαγάγει αυτούς εκ γης Αιγύπτου εις γην την οποίαν προέβλεψα δι' αυτούς, γην ρέουσαν γάλα και μέλι, ήτις είναι η δόξα πασών των γαιών.
\par 7 Και είπα προς αυτούς, Απορρίψατε έκαστος τα βδελύγματα των οφθαλμών αυτού και μη μιαίνεσθε με τα είδωλα της Αιγύπτου· εγώ είμαι Κύριος ο Θεός σας.
\par 8 Αυτοί όμως απεστάτησαν απ' εμού και δεν ηθέλησαν να μου ακούσωσι· δεν απέρριψαν έκαστος τα βδελύγματα των οφθαλμών αυτών και δεν εγκατέλιπον τα είδωλα της Αιγύπτου. Τότε είπα να εκχέω τον θυμόν μου επ' αυτούς, διά να συντελέσω την οργήν μου εναντίον αυτών εν μέσω της γης Αιγύπτου.
\par 9 Πλην ένεκεν του ονόματός μου, διά να μη βεβηλωθή ενώπιον των εθνών, μεταξύ των οποίων ήσαν και έμπροσθεν των οποίων εγνωρίσθην εις αυτούς, έκαμον τούτο, να εξαγάγω αυτούς εκ γης Αιγύπτου.
\par 10 Και εξήγαγον αυτούς εκ γης Αιγύπτου και έφερα αυτούς εις την έρημον·
\par 11 και έδωκα εις αυτούς τα διατάγματά μου και έκαμον εις αυτούς γνωστάς τας κρίσεις μου, τας οποίας κάμνων ο άνθρωπος θέλει ζήσει δι' αυτών.
\par 12 Και τα σάββατά μου έδωκα έτι εις αυτούς, διά να ήναι μεταξύ εμού και αυτών σημείον, ώστε να γνωρίζωσιν ότι εγώ είμαι ο Κύριος ο αγιάζων αυτούς.
\par 13 Αλλ' ο οίκος Ισραήλ απεστάτησεν απ' εμού εν τη ερήμω· εν τοις διατάγμασί μου δεν περιεπάτησαν και τας κρίσεις μου απέρριψαν, τας οποίας κάμνων ο άνθρωπος θέλει ζήσει δι' αυτών· και τα σάββατά μου εβεβήλωσαν σφόδρα· τότε είπα να εκχέω τον θυμόν μου επ' αυτούς εν τη ερήμω, διά να εξολοθρεύσω αυτούς.
\par 14 Πλην έκαμον τούτο ένεκεν του ονόματός μου, διά να μη βεβηλωθή ενώπιον των εθνών, έμπροσθεν των οποίων εξήγαγον αυτούς.
\par 15 Και εγώ ύψωσα ότι προς αυτούς την χείρα μου εν τη ερήμω, ότι δεν θέλω φέρει αυτούς εις την γην, την οποίαν έδωκα εις αυτούς, γην ρέουσαν γάλα και μέλι, ήτις είναι δόξα πασών των γαιών·
\par 16 διότι τας κρίσεις μου απέρριψαν και εν τοις διατάγμασί μου δεν περιεπάτησαν και τα σάββατά μου εβεβήλωσαν· διότι αι καρδίαι αυτών επορεύοντο κατόπιν των ειδώλων αυτών.
\par 17 Και εφείσθη ο οφθαλμός μου επ' αυτούς, ώστε να μη εξαλείψω αυτούς, και δεν συνετέλεσα αυτούς εν τη ερήμω.
\par 18 Αλλ' είπα προς τα τέκνα αυτών εν τη ερήμω, Μη περιπατείτε εν τοις διατάγμασι των πατέρων σας και μη φυλάττετε τας κρίσεις αυτών και μη μιαίνεσθε με τα είδωλα αυτών·
\par 19 εγώ είμαι Κύριος ο Θεός σας· εν τοις διατάγμασί μου περιπατείτε· και τας κρίσεις μου φυλάττετε και εκτελείτε αυτάς·
\par 20 και αγιάζετε τα σάββατά μου, και ας ήναι μεταξύ εμού και υμών σημείον, ώστε να γνωρίζητε ότι εγώ είμαι Κύριος ο Θεός σας.
\par 21 Τα τέκνα όμως απεστάτησαν απ' εμού· εν τοις διατάγμασί μου δεν περιεπάτησαν και τας κρίσεις μου δεν εφύλαξαν, ώστε να εκτελώσιν αυτάς, τας οποίας κάμνων ο άνθρωπος θέλει ζήσει δι' αυτών· τα σάββατά μου εβεβήλωσαν· τότε είπα να εκχέω τον θυμόν μου επ' αυτούς, διά να συντελέσω την οργήν μου εναντίον αυτών εν τη ερήμω.
\par 22 Και απέστρεψα την χείρα μου και έκαμον τούτο ένεκεν του ονόματός μου, διά να μη βεβηλωθή ενώπιον των εθνών, έμπροσθεν των οποίων εξήγαγον αυτούς.
\par 23 Ύψωσα έτι εγώ την χείρα μου προς αυτούς εν τη ερήμω, ότι ήθελον διασκορπίσει αυτούς μεταξύ των εθνών και διασπείρει αυτούς εις τους τόπους·
\par 24 διότι τας κρίσεις μου δεν εξετέλεσαν και τα διατάγματά μου απέρριψαν και τα σάββατά μου εβεβήλωσαν, και οι οφθαλμοί αυτών ήσαν κατόπιν των ειδώλων των πατέρων αυτών.
\par 25 Διά τούτο και εγώ έδωκα εις αυτούς διατάγματα ουχί καλά και κρίσεις, διά των οποίων δεν ήθελον ζήσει·
\par 26 και εμίανα αυτούς εις τας προσφοράς αυτών, εις το ότι διεβίβαζον διά του πυρός παν διανοίγον μήτραν, διά να ερημώσω αυτούς, ώστε να γνωρίσωσιν ότι εγώ είμαι ο Κύριος.
\par 27 Διά τούτο, υιέ ανθρώπου, λάλησον προς τον οίκον Ισραήλ και ειπέ προς αυτούς, Ούτω λέγει Κύριος ο Θεός· κατά τούτο ότι οι πατέρες σας ύβρισαν εις εμέ, κάμνοντες παράβασιν εναντίον μου.
\par 28 Διότι αφού έφερα αυτούς εις την γην, περί της οποίας ύψωσα την χείρα μου ότι θέλω δώσει αυτήν εις αυτούς, τότε ενέβλεψαν εις πάντα λόφον υψηλόν και εις παν δένδρον κατάσκιον, και εκεί προσέφεραν τας θυσίας αυτών και έστησαν εκεί τας παροργιστικάς προσφοράς αυτών, και έθεσαν εκεί οσμήν ευωδίας αυτών και έκαμον εκεί τας σπονδάς αυτών.
\par 29 Και είπα προς αυτούς, Τι δηλοί ο υψηλός τόπος, εις τον οποίον σεις υπάγετε; και το όνομα αυτού εκλήθη Βαμά, έως της σήμερον.
\par 30 Διά τούτο ειπέ προς τον οίκον Ισραήλ, Ούτω λέγει Κύριος ο Θεός· Ενώ σεις μιαίνεσθε εν τη οδώ των πατέρων σας και εκπορνεύετε κατόπιν των βδελυγμάτων αυτών
\par 31 και μιαίνεσθε με πάντα τα είδωλά σας έως της σήμερον, προσφέροντες τα δώρα σας, διαβιβάζοντες τους υιούς σας διά του πυρός, και εγώ θέλω επερωτηθή από σας, οίκος Ισραήλ; Ζω εγώ, λέγει Κύριος ο Θεός, δεν θέλω επερωτηθή από σας.
\par 32 Και εκείνο το οποίον διαβουλεύεσθε, ουδόλως θέλει γείνει· διότι σεις λέγετε, Θέλομεν είσθαι ως τα έθνη, ως αι οικογένειαι των τόπων, εις το να λατρεύωμεν ξύλα και λίθους.
\par 33 Ζω εγώ, λέγει Κύριος ο Θεός, εξάπαντος εν χειρί κραταιά και εν βραχίονι εξηπλωμένω και εν θυμώ εκχεομένω θέλω βασιλεύσει εφ' υμάς.
\par 34 Και θέλω σας εξαγάγει εκ των λαών και θέλω σας συνάξει εκ των τόπων, εις τους οποίους είσθε διεσκορπισμένοι, εν χειρί κραταιά και εν βραχίονι εξηπλωμένω και εν θυμώ εκχεομένω.
\par 35 Και θέλω σας φέρει εις την έρημον των λαών και εκεί θέλω κριθή με σας πρόσωπον προς πρόσωπον·
\par 36 καθώς εκρίθην με τους πατέρας σας εν τη ερήμω της γης Αιγύπτου, ούτω θέλω σας κρίνει, λέγει Κύριος ο Θεός.
\par 37 Και θέλω σας περάσει υπό την ράβδον και θέλω σας φέρει εις τους δεσμούς της διαθήκης.
\par 38 Και θέλω εκκαθαρίσει εκ μέσου υμών τους αποστάτας και τους ασεβήσαντας εις εμέ· θέλω εκβάλει αυτούς εκ της γης της παροικίας αυτών και δεν θέλουσιν εισέλθει εις γην Ισραήλ, και θέλετε γνωρίσει ότι εγώ είμαι ο Κύριος.
\par 39 Σεις δε, οίκος Ισραήλ, ούτω λέγει Κύριος ο Θεός· Υπάγετε, λατρεύετε έκαστος τα είδωλα αυτού, και του λοιπού, εάν δεν θέλητε να μου ακούητε· και μη βεβηλόνετε πλέον το όνομά μου το άγιον με τα δώρα σας και με τα είδωλά σας.
\par 40 Διότι επί του όρους του αγίου μου, επί του υψηλού όρους του Ισραήλ, λέγει Κύριος ο Θεός, εκεί πας ο οίκος του Ισραήλ, πάντες οι εν τη γη θέλουσι με λατρεύσει· εκεί θέλω δεχθή αυτούς και εκεί θέλω ζητήσει τας προσφοράς σας και τας απαρχάς των δώρων σας με πάντα τα άγιά σας.
\par 41 Θέλω σας δεχθή με οσμήν ευωδίας, όταν σας εξαγάγω εκ των λαών και σας συνάξω εκ των τόπων εις τους οποίους διεσκορπίσθητε· και θέλω αγιασθή εις εσάς ενώπιον των εθνών.
\par 42 Και θέλετε γνωρίσει ότι εγώ είμαι ο Κύριος, όταν σας φέρω εις γην Ισραήλ, εις γην περί της οποίας ύψωσα την χείρα μου ότι θέλω δώσει αυτήν εις τους πατέρας σας.
\par 43 Και εκεί θέλετε ενθυμηθή τας οδούς σας και πάντα τα έργα σας, εις τα οποία εμιάνθητε· και θέλετε αποστραφή αυτοί εαυτούς έμπροσθεν των οφθαλμών σας, διά πάντα τα κακά σας όσα επράξατε.
\par 44 Και θέλετε γνωρίσει ότι εγώ είμαι ο Κύριος, όταν κάμω ούτως εις εσάς ένεκεν του ονόματός μου, ουχί κατά τας πονηράς οδούς σας ουδέ κατά τα διεφθαρμένα έργα σας, οίκος Ισραήλ, λέγει Κύριος ο Θεός.
\par 45 Και έγεινε λόγος Κυρίου προς εμέ, λέγων,
\par 46 Υιέ ανθρώπου, στήριξον το πρόσωπόν σου προς μεσημβρίαν και στάλαξον λόγον προς μεσημβρίαν και προφήτευσον κατά του δάσους της μεσημβρινής πεδιάδος·
\par 47 και ειπέ προς το δάσος της μεσημβρίας, Άκουσον τον λόγον του Κυρίου· Ούτω λέγει Κύριος ο Θεός· Ιδού, εγώ θέλω ανάψει πυρ εν σοι, και θέλει καταφάγει εν σοι παν δένδρον χλωρόν και παν δένδρον ξηρόν· η φλόξ η εξαφθείσα δεν θέλει σβεσθή, και παν πρόσωπον από μεσημβρίας μέχρι βορρά θέλει καυθή εν αυτώ.
\par 48 Και πάσα σαρξ θέλει ιδεί, ότι εγώ ο Κύριος εξέκαυσα αυτό· δεν θέλει σβεσθή.
\par 49 Και εγώ είπα, Φευ Κύριε Θεέ αυτοί λέγουσι περί εμού, δεν λαλεί ούτος παροιμίας;

\chapter{21}

\par 1 Και έγεινε λόγος Κυρίου προς εμέ, λέγων,
\par 2 Υιέ ανθρώπου, στήριξον το πρόσωπόν σου προς Ιερουσαλήμ και στάλαξον λόγον προς τους αγίους τόπους και προφήτευσον κατά της γης Ισραήλ
\par 3 και ειπέ προς την γην Ισραήλ, Ούτω λέγει Κύριος· Ιδού, εγώ είμαι εναντίον σου και θέλω σύρει την μάχαιράν μου εκ της θήκης αυτής και θέλω εκκόψει από σου τον δίκαιον και τον ασεβή.
\par 4 Και επειδή θέλω εκκόψει από σου τον δίκαιον και τον ασεβή, διά τούτο θέλει εξέλθει η μάχαιρά μου εκ της θήκης αυτής εναντίον πάσης σαρκός, από μεσημβρίας μέχρι βορρά·
\par 5 και θέλουσι γνωρίσει πάσα σαρξ ότι εγώ ο Κύριος έσυρα την μάχαιράν μου εκ της θήκης αυτής· δεν θέλει επιστρέψει πλέον.
\par 6 Διά τούτο συ, υιέ ανθρώπου, στέναξον μετά συντριμμού της οσφύος σου, και μετά πικρίας στέναξον ενώπιον αυτών.
\par 7 Και όταν είπωσι προς σε, διά τι στενάζεις συ; θέλεις αποκριθή· διά την αγγελίαν, ότι έρχεται· και πάσα καρδία θέλει λυώσει, και πάσαι αι χείρες θέλουσι παραλυθή, και παν πνεύμα θέλει λιποθυμήσει, και πάντα τα γόνατα θέλουσι ρεύσει ως ύδωρ· ιδού, έρχεται και θέλει γείνει, λέγει Κύριος ο Θεός.
\par 8 Και έγεινε λόγος Κυρίου προς εμέ, λέγων,
\par 9 Υιέ ανθρώπου, προφήτευσον και ειπέ, Ούτω λέγει Κύριος· Ειπέ, Ρομφαία, ρομφαία ακονίζεται και μάλιστα στιλβούται·
\par 10 ακονίζεται, διά να κάμη σφαγήν· στιλβούται, διά να αστράπτη. Δυνάμεθα λοιπόν να ήμεθα εύθυμοι; αύτη είναι η ράβδος του υιού μου, η καταφρονούσα παν ξύλον.
\par 11 Και έδωκεν αυτήν να στιλβωθή, διά να κρατήται εν τη χειρί· η ρομφαία αύτη είναι ηκονισμένη και εστιλβωμένη, διά να δοθή εις την χείρα του σφαγέως.
\par 12 Βόησον και ολόλυξον, υιέ ανθρώπου· διότι αύτη είναι εναντίον του λαού μου, είναι εναντίον πάντων των αρχόντων του Ισραήλ·. τρόμος θέλει επιπέσει επί τον λαόν μου διά την ρομφαίαν· διά τούτο κτύπησον επί τον μηρόν σου.
\par 13 Διότι εξέτασις είναι· και τι; βεβαίως και η καταφρονούσα ράβδος δεν θέλει υπάρχει, λέγει Κύριος ο Θεός.
\par 14 Διά τούτο συ, υιέ ανθρώπου, προφήτευσον και κρότησον χείρα επί χείρα και ας διπλασιασθή η ρομφαία, ας τριπλασιασθή, η ρομφαία των τετραυματισμένων· αύτη είναι η ρομφαία των μεγάλων τραυματιών, ήτις θέλει διαπεράσει έως των ενδομύχων αυτών.
\par 15 Επέφερα την κοπήν της ρομφαίας επί πάσας τας πύλας αυτών, διά να λυώση πάσα καρδία και να πληθυνθή ο όλεθρος. Ουαί· ητοιμάσθη διά να εξαστράπτη, ηκονίσθη διά σφαγήν.
\par 16 Συσφίγχθητι, ρομφαία, επιτέθητι δεξιά, αριστερά, όπου στραφή το πρόσωπόν σου.
\par 17 Και εγώ έτι θέλω κροτήσει χείρα μου επί χείρα μου και θέλω αναπαύσει τον θυμόν μου· εγώ ο Κύριος ελάλησα.
\par 18 Και έγεινε λόγος Κυρίου προς εμέ, λέγων,
\par 19 Και συ, υιέ ανθρώπου, διόρισον εις σεαυτόν δύο οδούς, διά να διέλθη η ρομφαία του βασιλέως της Βαβυλώνος, και αμφότεραι θέλουσιν εξέρχεσθαι από της αυτής γής· και κάμε τόπον, κάμε αυτόν εν τη αρχή της οδού της πόλεως.
\par 20 Διόρισον οδόν διά να διέλθη η ρομφαία εις την Ραββά των υιών Αμμών και εις την Ιουδαίαν προς την Ιερουσαλήμ την ωχυρωμένην.
\par 21 Διότι ο βασιλεύς της Βαβυλώνος εστάθη εις τον διαχωρισμόν, εν τη αρχή των δύο οδών, διά να ερωτήση τους μάντεις· ανεκάτωσε τα μαντικά βέλη, ηρώτησε τα γλυπτά, παρετήρησε το ήπαρ.
\par 22 Προς την δεξιάν αυτού έγεινεν ο χρησμός διά την Ιερουσαλήμ, διά να στήση τους κριούς, διά να ανοίξη το στόμα επί σφαγήν, να υψώση την φωνήν μετά αλαλαγμού, να στήση κριούς εναντίον των πυλών, να κάμη προχώματα, να οικοδομήση προμαχώνας.
\par 23 Πλην τούτο θέλει είσθαι εις αυτούς ως μαντεία ματαία, εις τους οφθαλμούς εκείνων, οίτινες έκαμον όρκους προς αυτούς· αυτός όμως θέλει ενθυμίσει αυτούς την ανομίαν αυτών, διά να πιασθώσι.
\par 24 Διά τούτο ούτω λέγει Κύριος ο Θεός. Επειδή εκάμετε να έλθη εις ενθύμησιν η ανομία σας, ότε ανεκαλύφθησαν αι παραβάσεις σας, ώστε να φανερωθώσι τα αμαρτήματά σας εις πάσας τας πράξεις σας· επειδή ήλθετε εις ενθύμησιν, θέλετε γείνει χειριάλωτοι.
\par 25 Και συ, βέβηλε ασεβή, ηγεμών του Ισραήλ, του οποίου ήλθεν η ημέρα, ότε η ανομία έφθασεν εις πέρας,
\par 26 ούτω λέγει Κύριος ο Θεός· Σήκωσον το διάδημα και αφαίρεσον το στέμμα· αυτό δεν θέλει είσθαι τοιούτον· ο ταπεινός θέλει υψωθή και ο υψηλός θέλει ταπεινωθή.
\par 27 Θέλω ανατρέψει, ανατρέψει, ανατρέψει αυτό, και δεν θέλει υπάρχει εωσού έλθη εκείνος, εις ον ανήκει· και εις τούτον θέλω δώσει αυτό.
\par 28 Και συ, υιέ ανθρώπου, προφήτευσον και ειπέ, Ούτω λέγει Κύριος ο Θεός περί των υιών Αμμών και περί του ονειδισμού αυτών, και ειπέ, Η ρομφαία, η ρομφαία είναι γεγυμνωμένη, διά την σφαγήν εστιλβωμένη, διά να εξολοθρεύση εξαστράπτουσα,
\par 29 ενώ βλέπουσι ματαίας οράσεις περί σου, ενώ μαντεύουσι ψεύδος εις σε, διά να σε βάλωσιν επί τον τράχηλον των τετραυματισμένων, των ασεβών, των οποίων η ημέρα ήλθεν, ότε η ανομία αυτών έφθασεν εις πέρας.
\par 30 Επίστρεψον αυτήν εις την θήκην αυτής. Εν τω τόπω όπου εκτίσθης, εν τη γη της γεννήσεώς σου, θέλω σε κρίνει.
\par 31 Και θέλω εκχέει την οργήν μου επί σε εν τω πυρί της οργής μου θέλω εμφυσήσει επί σέ· και θέλω σε παραδώσει εις χείρας ανδρών αγρίων, τεκταινόντων όλεθρον.
\par 32 Τροφή πυρός θέλεις γείνει· το αίμα σου θέλει είσθαι εν τω μέσω της γης σου· δεν θέλει είσθαι πλέον μνήμη περί σού· διότι εγώ ο Κύριος ελάλησα.

\chapter{22}

\par 1 Και έγεινε λόγος Κυρίου προς εμέ; λέγων,
\par 2 Και συ, υιέ ανθρώπου, θέλεις κρίνει, θέλεις κρίνει την πόλιν των αιμάτων; και θέλεις παραστήσει εις αυτήν πάντα τα βδελύγματα αυτής;
\par 3 Ειπέ λοιπόν, Ούτω λέγει Κύριος ο Θεός· Ω πόλις εκχέουσα αίματα μέσω εαυτής, διά να έλθη ο καιρός αυτής, και κατασκευάζουσα είδωλα εναντίον εαυτής, διά να μιαίνηται,
\par 4 έγεινας ένοχος εν τω αίματί σου, το οποίον εξέχεας, και εμιάνθης εν τοις ειδώλοις σου, τα οποία κατεσκεύασας, και έκαμες να πλησιάσωσιν αι ημέραι σου, και ήλθες μέχρι των ετών σου· διά τούτο σε κατέστησα όνειδος εις τα έθνη και παίγνιον εις πάντας τους τόπους.
\par 5 Οι πλησίον και οι μακράν από σου θέλουσιν εμπαίξει σε, μεμολυσμένη κατά το όνομα, μεγάλη κατά τας συμφοράς.
\par 6 Ιδού, οι άρχοντες του Ισραήλ ήσαν εν σοι, διά να χύνωσιν αίμα, έκαστος κατά την δύναμιν αυτού.
\par 7 Εν σοι κατεφρόνουν πατέρα και μητέρα· εν μέσω σου εφέροντο απατηλώς προς τον ξένον· εν σοι κατεδυνάστευον τον ορφανόν και την χήραν.
\par 8 Τα άγιά μου κατεφρόνησας και τα σάββατά μου εβεβήλωσας.
\par 9 Εν σοι ήσαν άνδρες συκοφάνται διά να χύνωσιν αίμα, και εν σοι έτρωγον επί των ορέων, εν μέσω σου πράττουσιν ανοσιουργίας.
\par 10 Εν σοι εξεσκέπασαν αισχύνην πατρός, εν σοι εταπείνωσαν την αποκεχωρισμένην εν τη ακαθαρσία αυτής.
\par 11 Και ο μεν έπραξε βδελυρίαν μετά της γυναικός του πλησίον αυτού, ο δε εμίανεν ανοσίως την νύμφην αυτού, και άλλος εν σοι εταπείνωσε την αδελφήν αυτού, την θυγατέρα του πατρός αυτού.
\par 12 Εν σοι ελάμβανον δώρα διά να εκχέωσιν αίμα· έλαβες τόκον και προσθήκην και δι' απάτης ησχροκέρδησας από των πλησίον σου, και ελησμόνησας εμέ, λέγει Κύριος ο Θεός.
\par 13 Ιδού, διά τούτο εκρότησα τας χείρας μου επί τη αισχροκερδεία σου, την οποίαν έπραξας, και επί τω αίματί σου, το οποίον ήτο εν μέσω σου.
\par 14 Θέλει ανθέξει η καρδία σου; ή θέλουσιν έχει δύναμιν αι χείρές σου, εν ημέραις καθ' ας εγώ θέλω ενεργήσει εναντίον σου; εγώ ο Κύριος ελάλησα και θέλω εκτελέσει.
\par 15 Και θέλω σε διασκορπίσει εν τοις έθνεσι και σε διασπείρει εις τους τόπους και θέλω εξαλείψει από σου την ακαθαρσίαν σου.
\par 16 Και θέλεις βεβηλωθή εν σοι ενώπιον των εθνών, και θέλεις γνωρίσει ότι εγώ είμαι ο Κύριος.
\par 17 Και έγεινε λόγος Κυρίου προς εμέ, λέγων,
\par 18 Υιέ ανθρώπου, ο οίκος Ισραήλ έγεινεν εις εμέ σκωρία· πάντες είναι χαλκός και κασσίτερος και σίδηρος και μόλυβδος εν τω μέσω του χωνευτηρίου· είναι σκωρίαι αργύρου.
\par 19 Διά τούτο ούτω λέγει Κύριος ο Θεός· Επειδή πάντες σεις εγείνετε σκωρίαι, ιδού, διά τούτο θέλω σας συνάξει εις το μέσον της Ιερουσαλήμ·
\par 20 καθώς συνάγουσιν εις το μέσον του χωνευτηρίου τον άργυρον και τον χαλκόν και τον σίδηρον και τον μόλυβδον και τον κασσίτερον, διά να φυσήσωσι το πυρ επ' αυτά ώστε να διαλύσωσιν αυτά, ούτως εν τω θυμώ μου και εν τη οργή μου θέλω σας συνάξει και θέλω σας βάλει εκεί και διαλύσει.
\par 21 Θέλω εξάπαντος σας συνάξει, και εν τω πυρί της οργής μου θέλω εμφυσήσει εφ' υμάς και θέλετε διαλυθή εν τω μέσω αυτής.
\par 22 Καθώς ο άργυρος διαλύεται εν μέσω του χωνευτηρίου, ούτω θέλετε διαλυθή εν μέσω αυτής· και θέλετε γνωρίσει ότι εγώ ο Κύριος εξέχεα την οργήν μου εφ' υμάς.
\par 23 Και έγεινε λόγος Κυρίου προς εμέ, λέγων,
\par 24 Υιέ ανθρώπου, ειπέ προς αυτήν· Συ είσαι η γη, ήτις δεν εκαθαρίσθη, και δεν έγεινε βροχή επ' αυτής εν τη ημέρα της οργής.
\par 25 Εν μέσω αυτής είναι συνωμοσία των προφητών αυτής· ως λέοντες ωρυόμενοι, αρπάζοντες το θήραμα, κατατρώγουσι ψυχάς· έλαβον θησαυρούς και πολύτιμα πράγματα· επλήθυναν τας χήρας αυτής εν τω μέσω αυτής.
\par 26 Οι ιερείς αυτής ηθέτησαν τον νόμον μου και εβεβήλωσαν τα άγιά μου· μεταξύ αγίου και βεβήλου δεν έκαμον διαφοράν και μεταξύ ακαθάρτου και καθαρού δεν έκαμον διάκρισιν, και έκρυπτον τους οφθαλμούς αυτών από των σαββάτων μου, και εβεβηλούμην εν μέσω αυτών.
\par 27 Οι άρχοντες αυτής είναι εν μέσω αυτής ως λύκοι αρπάζοντες το θήραμα, διά να εκχέωσιν αίμα, διά να αφανίζωσι ψυχάς, διά να αισχροκερδήσωσιν αισχροκέρδειαν.
\par 28 Και οι προφήται αυτής περήλειφον αυτούς με πηλόν αμάλακτον, βλέποντες οράσεις ματαίας και μαντεύοντες προς αυτούς ψεύδη, λέγοντες, Ούτω λέγει Κύριος ο Θεός· ενώ ο Κύριος δεν ελάλησεν.
\par 29 Ο λαός της γης μετεχειρίζετο απάτην και έκαμνεν αρπαγάς και κατεδυνάστευε τον πτωχόν και τον ενδεή και τον ξένον ηπάτα άνευ κρίσεως.
\par 30 Και εζήτησα μεταξύ αυτών άνδρα, όστις να ανεγείρη το περίφραγμα και να σταθή εν τη χαλάστρα ενώπιόν μου υπέρ της γης, διά να μη εξολοθρεύσω αυτήν· και δεν εύρηκα.
\par 31 Διά τούτο εξέχεα την οργήν μου επ' αυτούς· κατηνάλωσα αυτούς εν τω πυρί της οργής μου· τας οδούς αυτών ανταπέδωκα επί τας κεφαλάς αυτών, λέγει Κύριος ο Θεός.

\chapter{23}

\par 1 Και έγεινε λόγος Κυρίου προς εμέ, λέγων,
\par 2 Υιέ ανθρώπου, ήσαν δύο γυναίκες, θυγατέρες της αυτής μητρός·
\par 3 και εξεπορνεύθησαν εν Αιγύπτω· εξεπορνεύθησαν εν τη νεότητι αυτών· εκεί επιέσθησαν τα στήθη αυτών και εκεί συνεθλίβησαν οι παρθενικοί αυτών μαστοί.
\par 4 Τα δε ονόματα αυτών ήσαν, Οολά η πρεσβυτέρα και Οολιβά η αδελφή αυτής· και αύται έγειναν εμού και εγέννησαν υιούς και θυγατέρας. Ήσαν λοιπόν τα ονόματα αυτών Σαμάρεια η Οολά και Ιερουσαλήμ η Οολιβά.
\par 5 Και η Οολά εξεπορνεύθη, ενώ ήτο εμού, και παρεφρόνησε διά τους εραστάς αυτής, τους Ασσυρίους τους γείτονας αυτής,
\par 6 ενδεδυμένους κυανά, ταξιάρχους και άρχοντας, πάντας ερασμίους νέους, ιππείς ιππεύοντας εφ' ίππων.
\par 7 Και έπραξε τας πορνείας αυτής μετ' αυτών, οίτινες πάντες ήσαν οι εκλεκτοί των Ασσυρίων, και μετά πάντων εκείνων, διά τους οποίους παρεφρόνησεν· εν πάσι τοις ειδώλοις αυτών εμιαίνετο.
\par 8 Και δεν αφήκε την πορνείαν αυτής την εξ Αιγύπτου· διότι μετ' αυτής εκοιμώντο εν τη νεότητι αυτής και ούτοι επίεζον τα παρθενικά αυτής στήθη και εξέχεον την πορνείαν αυτών επ' αυτήν.
\par 9 Διά τούτο παρέδωκα αυτήν εις τας χείρας των εραστών αυτής, εις τας χείρας των Ασσυρίων, διά τους οποίους παρεφρόνησεν.
\par 10 Ούτοι ανεκάλυψαν την αισχύνην αυτής· έλαβον τους υιούς αυτής και τας θυγατέρας αυτής και αυτήν εν ρομφαία απέκτειναν, και έγεινε περιβόητος μεταξύ των γυναικών, και εξετέλεσαν κρίσιν επ' αυτήν.
\par 11 Ότε δε η αδελφή αυτής Οολιβά είδε τούτο, διεφθάρη εν τη παραφροσύνη αυτής υπέρ εκείνην και εν ταις πορνείαις αυτής υπέρ τας πορνείας της αδελφής αυτής·
\par 12 παρεφρόνησε διά τους Ασσυρίους τους γείτονας αυτής, ταξιάρχους και άρχοντας ενδεδυμένους πολυτελώς, ιππείς ιππεύοντας εφ' ίππων, πάντας νέους ερασμίους.
\par 13 Και είδον ότι εμιάνθη· μίαν οδόν έχουσιν αμφότεραι.
\par 14 Προσέθεσεν έτι εις τας εαυτής πορνείας· διότι ως είδεν άνδρας εζωγραφημένους επί του τοίχου, εικόνας Χαλδαίων, οίτινες ήσαν εζωγραφημένοι με μίλτον,
\par 15 περιεζωσμένους ζώνας επί τας οσφύας αυτών, φορούντας τιάρας βεβαμμένας επί τας κεφαλάς αυτών, πάντας έχοντας όψιν αρχόντων, ομοίους με τους Βαβυλωνίους της γης των Χαλδαίων, εν ή εγεννήθησαν.
\par 16 και ως είδεν αυτούς με τους οφθαλμούς αυτής, παρεφρόνησε δι' αυτούς και εξαπέστειλε προς αυτούς πρέσβεις εις την Χαλδαίαν.
\par 17 Και οι Βαβυλώνιοι ήλθον προς αυτήν εις την κοίτην του έρωτος και εμίαναν αυτήν με την πορνείαν αυτών και εμιάνθη μετ' αυτών και η ψυχή αυτής απεξενώθη απ' αυτών.
\par 18 Και απεκάλυψε τας πορνείας αυτής και εξεσκέπασε την αισχύνην αυτής· τότε η ψυχή μου απεξενώθη απ' αυτής, καθώς η ψυχή μου είχεν αποξενωθή από της αδελφής αυτής.
\par 19 Διότι επλήθυνε τας πορνείας αυτής, ανακαλούσα εις μνήμην τας ημέρας της νεότητος αυτής, ότε επορνεύετο εν γη Αιγύπτου.
\par 20 Και παρεφρόνησε διά τους εραστάς αυτής, των οποίων η σαρξ είναι σαρξ όνων και η ρεύσις αυτών ρεύσις ίππων.
\par 21 Και ενεθυμήθης την ακολασίαν της νεότητός σου, ότε τα στήθη σου επιέζοντο υπό των Αιγυπτίων, διά τους μαστούς της νεότητός σου.
\par 22 Διά τούτο, Οολιβά, ούτω λέγει Κύριος ο Θεός· Ιδού, εγώ θέλω εγείρει τους εραστάς σου εναντίον σου, αφ' ων η ψυχή σου απεξενώθη, και θέλω φέρει αυτούς εναντίον σου πανταχόθεν·
\par 23 τους Βαβυλωνίους και πάντας τους Χαλδαίους, Φεκώδ και Σωέ και Κωέ, πάντας τους Ασσυρίους μετ' αυτών· οίτινες πάντες είναι εράσμιοι νέοι, ταξίαρχοι και ηγεμόνες, στρατάρχαι και ονομαστοί, πάντες ιππεύοντες εφ' ίππων.
\par 24 Και θέλουσιν ελθεί εναντίον σου μετά αρμάτων, μετά αμαξών και τροχών και μετά πλήθους λαών, και θέλουσι θέσει κύκλω εναντίον σου θυρεούς και ασπίδας και περικεφαλαίας· και θέλω θέσει ενώπιον αυτών κρίσιν και θέλουσι σε κρίνει κατά τας κρίσεις αυτών.
\par 25 Και θέλω στήσει τον ζήλον μου εναντίον σου και θέλουσι φερθή προς σε μετ' οργής· θέλουσιν εκκόψει την μύτην σου και τα ώτα σου· και το υπόλοιπόν σου θέλει πέσει εν μαχαίρα· ούτοι θέλουσι λάβει τους υιούς σου και τας θυγατέρας σου· το δε υπόλοιπόν σου θέλει καταφαγωθή υπό πυρός.
\par 26 Θέλουσιν έτι σε εκδύσει τα ιμάτιά σου και αφαιρέσει τους στολισμούς της λαμπρότητός σου.
\par 27 Και θέλω παύσει από σου την ακολασίαν σου και την πορνείαν σου την εκ γης Αιγύπτου· και δεν θέλεις σηκώσει τους οφθαλμούς σου προς αυτούς, και δεν θέλεις ενθυμηθή πλέον την Αίγυπτον.
\par 28 Διότι ούτω λέγει Κύριος ο Θεός· Ιδού, θέλω σε παραδώσει εις την χείρα εκείνων τους οποίους μισείς, εις την χείρα εκείνων αφ' ων απεξενώθη η ψυχή σου.
\par 29 Και θέλουσι φερθή προς σε με μίσος, και θέλουσι λάβει πάντας τους κόπους σου, και θέλουσι σε εγκαταλείψει γυμνήν και ασκέπαστον· και η αισχύνη της πορνείας σου θέλει αποκαλυφθή και η ακολασία σου και αι πορνείαί σου.
\par 30 Ταύτα θέλω κάμει εις σε, επειδή επορνεύθης κατόπιν των εθνών, επειδή εμιάνθης εν τοις ειδώλοις αυτών.
\par 31 Εν τη οδώ της αδελφής σου περιεπάτησας· διά τούτο θέλω δώσει εις την χείρα σου το ποτήριον αυτής.
\par 32 Ούτω λέγει Κύριος ο Θεός· Το ποτήριον της αδελφής σου θέλεις πίει, το βαθύ και πλατύ· θέλεις είσθαι γέλως και παίγνιον· το ποτήριον τούτο χωρεί πολύ.
\par 33 Θέλεις εμπλησθή από μέθης και θλίψεως, με το ποτήριον της εκπλήξεως και του αφανισμού, με το ποτήριον της αδελφής σου Σαμαρείας.
\par 34 Και θέλεις πίει αυτό και στραγγίσει, και θέλεις συντρίψει τα όστρακα αυτού, και θέλεις διασπαράξει τα στήθη σου· διότι εγώ ελάλησα, λέγει Κύριος ο Θεός.
\par 35 Διά τούτο ούτω λέγει Κύριος ο Θεός· Επειδή με ελησμόνησας και με απέρριψας οπίσω των νώτων σου, βάστασον λοιπόν και συ την ακολασίαν σου και τας πορνείας σου.
\par 36 Και είπε Κύριος προς εμέ· Υιέ ανθρώπου, θέλεις κρίνει την Οολά και την Οολιβά; απάγγειλον λοιπόν προς αυτάς τα βδελύγματα αυτών·
\par 37 ότι εμοιχεύοντο, και είναι αίμα εν ταις χερσίν αυτών, και εμοιχεύοντο μετά των ειδώλων αυτών, και ότι διεβίβαζον χάριν αυτών τα τέκνα αυτών, τα οποία εγέννησαν εις εμέ, διά του πυρός εις κατανάλωσιν.
\par 38 Έπραξαν έτι τούτο εις εμέ· εμίαναν τα άγιά μου εν τη αυτή ημέρα και εβεβήλωσαν τα σάββατά μου.
\par 39 Διότι ότε έσφαξαν τα τέκνα αυτών εις τα είδωλα αυτών, τότε εισήρχοντο την αυτήν ημέραν εις τα άγιά μου, διά να βεβηλόνωσιν αυτά· και ιδού, ούτως έπραττον εν μέσω του οίκου μου.
\par 40 Και προσέτι ειπέ ότι σεις επέμψατε προς άνδρας, διά να έλθωσι μακρόθεν, προς τους οποίους εστάλη πρέσβυς, και ιδού, ήλθον· διά τους οποίους ελούσθης, βάψας τους οφθαλμούς σου και εστολίσθης με στολισμούς.
\par 41 Και εκάθησας επί κλίνης μεγαλοπρεπούς και έμπροσθεν αυτής ήτο τράπεζα ητοιμασμένη, εφ' ης έθεσας το θυμίαμά μου και το έλαιόν μου.
\par 42 Και ήσαν εν αυτή φωναί πλήθους αγαλλομένου· και μετά των ανδρών του όχλου εισήγοντο Σαβαίοι εκ της ερήμου, φορούντες βραχιόλια επί τας χείρας αυτών και ωραίους στεφάνους επί τας κεφαλάς αυτών.
\par 43 Τότε είπα προς την καταγηράσασαν εν μοιχείαις, Τώρα κάμνουσι πορνείας μετ' αυτής και αυτή μετ' εκείνων
\par 44 και ούτοι εισήρχοντο προς αυτήν, καθώς εισέρχονται προς γυναίκα πόρνην· ούτως εισήρχοντο προς την Οολά και προς την Οολιβά, τας ακολάστους γυναίκας.
\par 45 Διά τούτο άνδρες δίκαιοι, ούτοι θέλουσι κρίνει αυτάς, κατά την κρίσιν των μοιχαλίδων και κατά την κρίσιν των εκχεουσών αίμα· επειδή είναι μοιχαλίδες και αίμα είναι εν ταις χερσίν αυτών.
\par 46 Όθεν ούτω λέγει Κύριος ο Θεός· Θέλω αναβιβάσει επ' αυτάς όχλον και θέλω παραδώσει αυτάς εις ταραχήν και διαρπαγήν.
\par 47 Και ο όχλος θέλει λιθοβολήσει αυτάς με λίθους και κατακόψει αυτάς με τα ξίφη αυτών· θέλουσι φονεύσει τους υιούς αυτών και τας θυγατέρας αυτών και τας οικίας αυτών θέλουσι κατακαύσει εν πυρί.
\par 48 Ούτω θέλω παύσει την ακολασίαν από της γης, διά να μάθωσι πάσαι αι γυναίκες να μη πράττωσι κατά τας ακολασίας σας.
\par 49 Και θέλουσιν ανταποδώσει τας ακολασίας υμών εφ' υμάς, και θέλετε βαστάσει τας αμαρτίας των ειδώλων σας· και θέλετε γνωρίσει ότι εγώ είμαι Κύριος ο Θεός.

\chapter{24}

\par 1 Και εν τω εννάτω έτει, τω δεκάτω μηνί, τη δεκάτη του μηνός, έγεινε λόγος Κυρίου προς εμέ, λέγων,
\par 2 Υιέ ανθρώπου, γράψον εις σεαυτόν το όνομα της ημέρας, αυτής ταύτης της ημέρας· διότι ο βασιλεύς της Βαβυλώνος παρετάχθη κατά της Ιερουσαλήμ εν αυτή ταύτη τη ημέρα.
\par 3 Και πρόφερε παραβολήν προς τον αποστάτην οίκον· και ειπέ προς αυτούς, Ούτω λέγει Κύριος ο Θεός· Στήσον τον λέβητα; στήσον, και έτι χύσον ύδωρ εις αυτόν·
\par 4 συνάγαγε εις αυτόν τα τμήματα αυτού, παν τμήμα καλόν, τον μηρόν και τον ώμον· γέμισον αυτόν από των εκλεκτών οστέων.
\par 5 Λάβε εκ των εκλεκτών του ποιμνίου και στίβασον έτι τα οστά κάτω αυτού· βράσον αυτά καλώς και ας εψηθώσι και αυτά τα οστά αυτού εν αυτώ.
\par 6 Διότι ούτω λέγει Κύριος ο Θεός· Ουαί εις την πόλιν των αιμάτων, εις τον λέβητα, του οποίου η σκωρία είναι εν αυτώ και του οποίου η σκωρία δεν εξήλθεν απ' αυτού. Έκβαλε κατά σειράν τα τμήματα αυτής· κλήρος ας μη πέση επ' αυτήν.
\par 7 Διότι το αίμα αυτής είναι εν μέσω αυτής· επί λειόπετραν εξέθεσεν αυτό· δεν έχυσεν αυτό επί την γην, ώστε να σκεπασθή με χώμα.
\par 8 Διά να κάμω να αναβή θυμός εις εκτέλεσιν εκδικήσεως, θέλω εκθέσει το αίμα αυτής επί λειόπετραν, διά να μη σκεπασθή.
\par 9 Διά τούτο ούτω λέγει Κύριος ο Θεός· Ουαί εις την πόλιν των αιμάτων· και εγώ θέλω μεγαλύνει την πυράν.
\par 10 Επισώρευσον τα ξύλα, άναψον το πυρ, κατανάλωσον τα κρέατα και διάλυσον αυτά, ας καώσι και τα οστά.
\par 11 Τότε στήσον αυτόν κενόν επί τους άνθρακας αυτόν, διά να πυρωθή ο χαλκός αυτού και να καή και να λυώση εν αυτώ η ακαθαρσία αυτού, να καταναλωθή η σκωρία αυτού.
\par 12 Ματαίως εδοκιμάσθη με κόπους, και η μεγάλη αυτής σκωρία δεν εξήλθεν απ' αυτής, η σκωρία αυτής εν τω πυρί.
\par 13 Εν τη ακαθαρσία σου υπάρχει μιαρότης· επειδή εγώ σε εκαθάρισα και δεν εκαθαρίσθης, δεν θέλεις πλέον καθαρισθή από της ακαθαρσίας σου, εωσού αναπαύσω τον θυμόν μου επί σε.
\par 14 Εγώ ο Κύριος ελάλησα· θέλει γείνει και θέλω εκτελέσει αυτό· δεν θέλω στραφή οπίσω και δεν θέλω φεισθή και δεν θέλω μεταμεληθή· κατά τας οδούς σου και κατά τας πράξεις σου θέλουσι σε κρίνει, λέγει Κύριος ο Θεός.
\par 15 Και έγεινε λόγος Κυρίου προς εμέ, λέγων,
\par 16 Υιέ ανθρώπου, ιδού, εγώ θέλω αφαιρέσει από σου διά μιας πληγής το επιθύμημα των οφθαλμών σου· και μη πενθήσης και μη κλαύσης και ας μη ρεύσωσι τα δάκρυά σου·
\par 17 κρατήθητι από στεναγμών, μη κάμης πένθος νεκρών, δέσον την τιάραν σου επί την κεφαλήν σου, και βάλε εις τους πόδας σου τα υποδήματά σου, και μη καλύψης τα χείλη σου, και άρτον ανδρών μη φάγης.
\par 18 Και ελάλησα προς τον λαόν το πρωΐ, και το εσπέρας απέθανεν η γυνή μου· και έκαμον το πρωΐ ως προσετάχθην.
\par 19 Και είπεν ο λαός προς εμέ, Δεν θέλεις απαγγείλει προς υμάς τι δηλούσιν εις υμάς ταύτα, τα οποία κάμνεις;
\par 20 Και απεκρίθην προς αυτούς, λόγος Κυρίου έγεινε προς εμέ λέγων,
\par 21 Ειπέ προς τον οίκον Ισραήλ, Ούτω λέγει Κύριος ο Θεός· Ιδού, θέλω βεβηλώσει τα άγιά μου, το καύχημα της δυνάμεώς σας, τα επιθυμήματα των οφθαλμών σας και τα περιπόθητα των ψυχών σας· και οι υιοί σας και αι θυγατέρες σας, όσους αφήκατε, εν ρομφαία θέλουσι πέσει.
\par 22 Και θέλετε κάμει καθώς εγώ έκαμον· δεν θέλετε καλύψει τα χείλη σας και άρτον ανδρών δεν θέλετε φάγει.
\par 23 Και αι τιάραι σας θέλουσιν είσθαι επί των κεφαλών σας και τα υποδήματά σας εις τους πόδας σας· δεν θέλετε πενθήσει ουδέ κλαύσει· αλλά θέλετε λυώσει διά τας ανομίας σας και θέλετε στενάξει ο εις προς τον άλλον.
\par 24 Και ο Ιεζεκιήλ θέλει είσθαι σημείον εις εσάς· κατά πάντα όσα έκαμε θέλετε κάμει· όταν τούτο έλθη, τότε θέλετε γνωρίσει ότι εγώ είμαι Κύριος ο Θεός.
\par 25 Περί δε σου, υιέ ανθρώπου, εν εκείνη τη ημέρα, όταν αφαιρέσω απ' αυτών την ισχύν αυτών, την χαράν της δόξης αυτών, τα επιθυμήματα των οφθαλμών αυτών και το θάρρος των ψυχών αυτών, τους υιούς αυτών και τας θυγατέρας αυτών,
\par 26 εν τη ημέρα εκείνη ο διασωθείς δεν θέλει ελθεί προς σε, διά να αναγγείλη ταύτα εις τα ώτα σου;
\par 27 Εν εκείνη τη ημέρα το στόμα σου θέλει ανοιχθή προς τον διασωθέντα και θέλεις λαλήσει και δεν θέλεις είσθαι πλέον άλαλος· και θέλεις είσθαι εις αυτούς σημείον· και θέλουσι γνωρίσει ότι εγώ είμαι ο Κύριος.

\chapter{25}

\par 1 Και έγεινε λόγος Κυρίου προς εμέ, λέγων,
\par 2 Υιέ ανθρώπου, στήριξον το πρόσωπόν σου επί τους υιούς Αμμών και προφήτευσον κατ' αυτών·
\par 3 και ειπέ προς τους υιούς Αμμών, Ακούσατε τον λόγον Κυρίου του Θεού· ούτω λέγει Κύριος ο Θεός· Επειδή επιλέγεις εις τα άγιά μου, Εύγε, διότι εβεβηλώθησαν, και εις την γην του Ισραήλ, διότι ηφανίσθη, και εις τον οίκον Ιούδα, διότι υπήγαν εις αιχμαλωσίαν,
\par 4 διά τούτο, ιδού, θέλω σε παραδώσει προς κληρονομίαν εις τους υιούς της ανατολής, και θέλουσι θέσει τας επαύλεις αυτών εν σοι και θέλουσι κάμει τας κατασκηνώσεις αυτών εν σοί· ούτοι θέλουσι φάγει τους καρπούς σου και ούτοι θέλουσι πίει το γάλα σου.
\par 5 Και θέλω καταστήσει την Ραββά σταύλον καμήλων και την γην των υιών Αμμών μάνδραν προβάτων· και θέλετε γνωρίσει ότι εγώ είμαι ο Κύριος.
\par 6 Διότι ούτω λέγει Κύριος ο Θεός· Επειδή επεκρότησας χείρας και εκτύπησας με τον πόδα και εν όλη τη περιφρονήσει της καρδίας σου εχάρης κατά της γης Ισραήλ,
\par 7 διά τούτο, ιδού, θέλω εκτείνει την χείρα μου επί σε και θέλω σε παραδώσει εις διαρπαγήν των εθνών και θέλω σε εκκόψει από των λαών και σε εξαφανίσει από των τόπων· θέλω σε εξολοθρεύσει· και θέλεις γνωρίσει ότι εγώ είμαι ο Κύριος.
\par 8 Ούτω λέγει Κύριος ο Θεός· Επειδή ο Μωάβ και ο Σηείρ λέγουσιν, Ιδού, ο οίκος Ιούδα είναι ως πάντα τα έθνη·
\par 9 διά τούτο, ιδού, θέλω ανοίξει την πλευράν του Μωάβ από των πόλεων, από των πόλεων αυτού, από των άκρων αυτού, την δόξαν της γης, την Βαιθ-ιεσιμώθ, Βάαλ-μεών και Κιριαθαΐμ,
\par 10 εις τους υιούς της ανατολής, κατά των υιών Αμμών, και θέλω παραδώσει αυτήν εις κληρονομίαν, διά να μη μνημονεύωνται οι υιοί Αμμών μεταξύ των εθνών.
\par 11 Και θέλω εκτελέσει κρίσεις επί τον Μωάβ· και θέλουσι γνωρίσει ότι εγώ είμαι ο Κύριος.
\par 12 Ούτω λέγει Κύριος ο Θεός· Επειδή ο Εδώμ έπραξεν εκδικητικώς εις τον οίκον Ιούδα και έβρισε βαρέως και εξεδικήθη εναντίον αυτών,
\par 13 διά τούτο ούτω λέγει Κύριος ο Θεός· θέλω λοιπόν εκτείνει την χείρα μου επί τον Εδώμ, και θέλω εκκόψει απ' αυτού άνθρωπον και κτήνος και θέλω εξαφανίσει αυτόν από Θαιμάν, και θέλουσι πέσει εν ρομφαία έως Δαιδάν.
\par 14 Και θέλω ενεργήσει την εκδίκησίν μου επί τον Εδώμ διά χειρός τον λαού μου Ισραήλ· και θέλουσι κάμει εις τον Εδώμ κατά τον θυμόν μου και κατά την οργήν μου· και θέλουσι γνωρίσει την εκδίκησίν μου, λέγει Κύριος ο Θεός.
\par 15 Ούτω λέγει Κύριος ο Θεός· Επειδή οι Φιλισταίοι εφέρθησαν εκδικητικώς και έκαμον εκδίκησιν περιφρονούντες εκ ψυχής, ώστε να φέρωσιν όλεθρον διά παλαιόν μίσος,
\par 16 διά τούτο ούτω λέγει Κύριος ο Θεός· Ιδού, εγώ θέλω εκτείνει την χείρα μου επί τους Φιλισταίους και θέλω εκκόψει τους Χερεθαίους και εξαφανίσει το υπόλοιπον των λιμένων της θαλάσσης·
\par 17 και θέλω κάμει επ' αυτούς μεγάλην εκδίκησιν εν ελεγμοίς θυμού· και θέλουσι γνωρίσει ότι εγώ είμαι ο Κύριος, όταν εκτελέσω την εκδίκησίν μου επ' αυτούς.

\chapter{26}

\par 1 Και εν τω ενδεκάτω έτει, τη πρώτη του μηνός, έγεινε λόγος Κυρίου προς εμέ, λέγων,
\par 2 Υιέ ανθρώπου, επειδή η Τύρος είπε κατά της Ιερουσαλήμ, Εύγε, συνετρίβη η πύλη των λαών· εστράφη προς εμέ· θέλω γεμισθή, διότι ηρημώθη·
\par 3 διά τούτο ούτω λέγει Κύριος ο Θεός· Ιδού, εγώ είμαι εναντίον σου, Τύρος, και θέλω επεγείρει εναντίον σου έθνη πολλά, ως επεγείρει η θάλασσα τα κύματα αυτής.
\par 4 Και θέλουσι καταστρέψει τα τείχη της Τύρου και κατεδαφίσει τους πύργους αυτής· και θέλω ξύσει το χώμα αυτής απ' αυτής και καταστήσει αυτήν ως λειόπετραν.
\par 5 Θέλει είσθαι διά να εξαπλόνωσι δίκτυα εν μέσω της θαλάσσης· διότι εγώ ελάλησα, λέγει Κύριος ο Θεός· και θέλει κατασταθή διαρπαγή των εθνών.
\par 6 Και αι κώμαι αυτής, αι εν τη πεδιάδι, θέλουσιν εξολοθρευθή εν μαχαίρα· και θέλουσι γνωρίσει ότι εγώ είμαι ο Κύριος.
\par 7 Διότι ούτω λέγει Κύριος ο Θεός· Ιδού, θέλω φέρει κατά της Τύρου τον Ναβουχοδονόσορ βασιλέα της Βαβυλώνος, βασιλέα βασιλέων, από βορρά, μεθ' ίππων και μετά αρμάτων και μεθ' ιππέων και συνάξεως και λαού πολλού.
\par 8 Ούτος θέλει εξολοθρεύσει εν μαχαίρα τας κώμας σου εν τη πεδιάδι· και θέλει εγείρει προμαχώνας εναντίον σου και θέλει κάμει προχώματα εναντίον σου και υψώσει κατά σου ασπίδας.
\par 9 Και θέλει στήσει τας πολεμικάς μηχανάς αυτού επί τα τείχη σου και με τους πελέκεις αυτού θέλει καταβάλει τους πύργους σου.
\par 10 Από του πλήθους των ίππων αυτού ο κονιορτός αυτών θέλει σε σκεπάσει· τα τείχη σου θέλουσι σεισθή από του ήχου των ιππέων και των τροχών και των αμαξών, όταν εισέρχωνται εις τας πύλας σου, καθώς εισέρχονται εις πόλιν εκπορθουμένην.
\par 11 Με τας οπλάς των ίππων αυτού θέλει καταπατήσει πάσας τας οδούς σου· τον λαόν σου θέλει θανατώσει εν μαχαίρα, και οι ισχυροί σου φρουροί θέλουσι καταβληθή εις την γην.
\par 12 Και θέλουσι διαρπάσει τα πλούτη σου και λαφυραγωγήσει τα εμπορεύματά σου· και θέλουσι καταβάλει τα τείχη σου και κρημνίσει τους οίκους σου τους ώραίους· και θέλουσι ρίψει εις το μέσον των υδάτων τους λίθους σου και τα ξύλα σου και το χώμα σου.
\par 13 Και θέλω παύσει τον θόρυβον των ασμάτων σου, και η φωνή των κιθαρών σου δεν θέλει ακουσθή πλέον·
\par 14 και θέλω σε καταστήσει ως λειόπετραν· θέλεις είσθαι διά να εξαπλόνωσι δίκτυα· δεν θέλεις πλέον οικοδομηθή· διότι εγώ ο Κύριος ελάλησα, λέγει Κύριος ο Θεός.
\par 15 Ούτω λέγει Κύριος ο Θεός προς την Τύρον· δεν θέλουσι σεισθή αι νήσοι εις τον ήχον της πτώσεώς σου, όταν οι τραυματίαι σου στενάζωσιν, όταν η σφαγή γίνηται εν μέσω σου;
\par 16 Τότε πάντες οι ηγεμόνες της θαλάσσης θέλουσι καταβή από των θρόνων αυτών, και θέλουσιν εκβάλει τας χλαμύδας αυτών και εκδυθή τα κεντητά ιμάτια αυτών· θέλουσιν ενδυθή τρόμον· κατά γης θέλουσι καθήσει και τρέμει κατά πάσαν στιγμήν και εκπλήττεσθαι διά σε.
\par 17 Και αναλαβόντες θρήνον διά σε θέλουσι λέγει προς σε, Πως κατεστράφης, η κατοικουμένη υπό θαλασσοπόρων, η περίφημος πόλις, ήτις ήσο ισχυρά εν θαλάσση, συ και οι κάτοικοί σου, οίτινες διέδιδον τον τρόμον αυτών εις πάντας τους ενοικούντας εν αυτή.
\par 18 Τώρα αι νήσοι θέλουσι τρέμει εν τη ημέρα της πτώσεώς σου, ναι, αι νήσοι αι εν τη θαλάσση θέλουσι ταραχθή εν τη αφανεία σου.
\par 19 Διότι ούτω λέγει Κύριος ο Θεός· Όταν σε καταστήσω πόλιν ηρημωμένην ως τας πόλεις τας μη κατοικουμένας, όταν επιφέρω επί σε την άβυσσον και σε σκεπάσωσιν ύδατα πολλά,
\par 20 όταν σε καταβιβάσω μετά των καταβαινόντων εις λάκκον, προς λαόν αιώνιον, και σε θέσω εις τα κατώτατα της γης, εις τόπους ερήμους απ' αιώνος, μετά των καταβαινόντων εις λάκκον, διά να μη κατοικηθής, και όταν αποκαταστήσω δόξαν εν τη γη των ζώντων,
\par 21 θέλω σε καταστήσει τρόμον και δεν θέλεις υπάρχει· και θέλεις ζητηθή και δεν θέλεις ευρεθή πλέον εις τον αιώνα, λέγει Κύριος ο Θεός.

\chapter{27}

\par 1 Και έγεινε λόγος Κυρίου προς εμέ, λέγων,
\par 2 Και συ, υιέ ανθρώπου, ανάλαβε θρήνον διά την Τύρον,
\par 3 και ειπέ προς την Τύρον την κειμένην εν τη εισόδω της θαλάσσης, την εμπορευομένην μετά των λαών εν πολλαίς νήσοις, Ούτω λέγει Κύριος ο Θεός· Τύρος, συ είπας, Εγώ είμαι πλήρης εις το κάλλος.
\par 4 Τα όριά σου είναι εν τη καρδία των θαλασσών, οι οικοδόμοι σου έκαμον πλήρες το κάλλος σου.
\par 5 Έκτισαν πάντα τα πλευρά των πλοίων σου εξ ελάτων από Σενείρ· έλαβον κέδρους εκ του Λιβάνου διά να κάμωσι κατάρτια εις σε.
\par 6 Εκ των δρυών της Βασάν έκαμον τα κωπία σου· έκαμον τα καθίσματά σου εξ ελέφαντος, εν πύξω από των νήσων των Κητιαίων.
\par 7 Λεπτόν λινόν εξ Αιγύπτου κεντητόν εξήπλονες εις σεαυτήν διά πανία· κυανούν και πορφυρούν εκ των νήσων Ελεισά ήτο το επισκήνωμά σου.
\par 8 Οι κάτοικοι της Σιδώνος και Αρβάδ ήσαν οι κωπηλάται σου· οι σοφοί σου, Τύρος, οι όντες εν σοι, αυτοί ήσαν οι κυβερνήται των πλοίων σου.
\par 9 Οι πρεσβύτεροι της Γεβάλ και οι σοφοί αυτής ήσαν εν σοι οι επισκευασταί των χαλασμάτων σου· πάντα τα πλοία της θαλάσσης και οι ναύται αυτών ήσαν εν σοι, διά να εμπορεύωνται το εμπόριόν σου.
\par 10 Πέρσαι και Λύδιοι και Λίβυες ήσαν εν τοις στρατεύμασί σου οι άνδρες σου οι πολεμισταί· ασπίδας και περικεφαλαίας εκρέμων εις σέ· ούτοι επεδείκνυον την μεγαλοπρέπειάν σου.
\par 11 Οι άνδρες της Αρβάδ μετά του στρατεύματός σου ήσαν κύκλω επί τα τείχη σου, και οι Γαμμαδίται επί τους πύργους σου· εκρέμων τας ασπίδας αυτών επί τα τείχη σου κύκλω· ούτοι συνεπλήρουν το κάλλος σου.
\par 12 Η Θαρσείς εμπορεύετο μετά σου εις πλήθος παντός πλούτου· με άργυρον, σίδηρον, κασσίτερον και μόλυβδον εμπορεύοντο εν ταις αγοραίς σου.
\par 13 Ιαυάν, Θουβάλ και Μεσέχ ήσαν έμποροί σου· εν τη αγορά σου εμπορεύοντο ψυχάς ανθρώπων και σκεύη χάλκινα.
\par 14 Από δε του οίκου Θωγαρμά εμπορεύοντο εν ταις αγοραίς σου ίππους και ιππέας και ημιόνους.
\par 15 Οι άνδρες της Δαιδάν ήσαν έμποροί σου· πολλών νήσων το εμπόριον ήτο εν τη χειρί σου· έφερον εις σε οδόντας ελεφάντων και έβενον εις ανταλλαγήν.
\par 16 Η Συρία εμπορεύετο μετά σου διά το πλήθος των εργασιών σου· έδιδεν εις τας αγοράς σου σμάραγδον, πορφύραν και κεντητά και βύσσον και κοράλλιον και αχάτην.
\par 17 Ο Ιούδας και η γη Ισραήλ ήσαν έμποροί σου· έδιδον εις την αγοράν σου σίτον του Μιννίθ και στακτήν και μέλι και έλαιον και βάλσαμον.
\par 18 Η Δαμασκός εμπορεύετο μετά σου εις το πλήθος των εργασιών σου, εις το πλήθος παντός πλούτου· εις οίνον της Χελβών και εις λευκά όρια.
\par 19 Και Δαν και Ιαυάν και Μωσέλ έδιδον εις τας αγοράς σου σίδηρον ειργασμένον, κασίαν και κάλαμον αρωματικόν· ταύτα ήσαν μεταξύ των πραγματειών σου.
\par 20 Η Δαιδάν εμπορεύετο μετά σου εις πολύτιμα υφάσματα διά αμάξας.
\par 21 Η Αραβία και πάντες οι άρχοντες Κηδάρ ήσαν έμποροί σου, εμπορευόμενοι μετά σου εις αρνία και κριούς και τράγους.
\par 22 Οι έμποροι της Σαβά και Ρααμά ήσαν έμποροί σου, δίδοντες εις τας αγοράς σου παν εξαίρετον άρωμα και πάντα λίθον τίμιον και χρυσίον.
\par 23 Χαρράν και Χαναά και Εδέν, οι έμποροι της Σαβά, ο Ασσούρ και ο Χιλμάδ, εμπορεύοντο μετά σου.
\par 24 Ούτοι ήσαν έμποροί σου εις παν είδος, εις κυανά ενδύματα και κεντητά και εις κιβώτια πλουσίων στολισμάτων, δεδεμένα με σχοινία και κατεσκευασμένα εκ κέδρου, μεταξύ των άλλων σου πραγματειών.
\par 25 Τα πλοία της Θαρσείς υπερείχον εις το εμπόριόν σου, και ήσο πλήρης, και εστάθης ενδοξοτάτη εν τη καρδία των θαλασσών.
\par 26 Οι κωπηλάται σου σε έφερον εις ύδατα πολλά· αλλ' ο άνεμος ο ανατολικός σε συνέτριψεν εν τη καρδία των θαλασσών.
\par 27 Τα πλούτη σου και αι αγοραί σου, το εμπόριόν σου, οι ναύταί σου και οι κυβερνήταί σου, οι επισκευασταί των πλοίων σου και οι εμπορευόμενοι το εμπόριόν σου, και πάντες οι άνδρες σου οι πολεμισταί οι εν σοι και παν το άθροισμά σου το εν μέσω σου, θέλουσι πέσει εν τη καρδία των θαλασσών, την ημέραν της πτώσεώς σου.
\par 28 Τα προάστεια θέλουσι σεισθή εις τον ήχον της κραυγής των κυβερνητών σου.
\par 29 Και πάντες οι κωπηλάται, οι ναύται, πάντες οι κυβερνήται της θαλάσσης, θέλουσι καταβή εκ των πλοίων αυτών, θέλουσι σταθή επί της γης,
\par 30 και θέλουσι κραυγάσει με την φωνήν αυτών επί σε, και θέλουσι βοήσει πικρά και ρίψει χώμα επί τας κεφαλάς αυτών και κατακυλισθή εν τη σποδώ.
\par 31 Και θέλουσι φαλακρωθή ολοκλήρως διά σε και περιζωσθή σάκκον και κλαύσει διά σε με πικρίαν ψυχής, οδυρόμενοι πικρώς.
\par 32 Και εν τω οδυρμώ αυτών θέλουσιν αναλάβει θρήνον διά σε και θέλουσι θρηνωδήσει, λέγοντες περί σου, Τις ως η Τύρος, ως η καταστραφείσα εν μέσω της θαλάσσης;
\par 33 Ότε αι πραγματείαί σου εξήρχοντο εκ των θαλασσών, εχόρταινες πολλούς λαούς· με το πλήθος του πλούτου σου και του εμπορίου σου επλούτιζες τους βασιλείς της γης.
\par 34 Τώρα συνετρίβης εν ταις θαλάσσαις, εν τω βάθει των υδάτων· το εμπόριόν σου και παν το άθροισμά σου έπεσον εν μέσω σου.
\par 35 Πάντες οι κάτοικοι των νήσων θέλουσιν εκπλαγή διά σε και οι βασιλείς αυτών θέλουσι κατατρομάξει, θέλουσιν ωχριάσει τα πρόσωπα.
\par 36 Οι έμποροι μεταξύ των εθνών θέλουσι συρίξει επί σέ· φρίκη θέλεις είσθαι και δεν θέλεις υπάρξει έως αιώνος.

\chapter{28}

\par 1 Και έγεινε λόγος Κυρίου προς εμέ, λέγων,
\par 2 Υιέ ανθρώπου, ειπέ προς τον ηγεμόνα της Τύρου, Ούτω λέγει Κύριος ο Θεός· Επειδή υψώθη η καρδία σου και είπας, Εγώ είμαι θεός, επί της καθέδρας του Θεού κάθημαι, εν τη καρδία των θαλασσών· ενώ είσαι άνθρωπος αλλ' ουχί Θεός· και έκαμες την καρδίαν σου ως καρδίαν Θεού·
\par 3 ιδού, συ είσαι σοφώτερος του Δανιήλ· ουδέν μυστήριον είναι κεκρυμμένον από σού·
\par 4 διά της σοφίας σου και διά της συνέσεώς σου έκαμες εις σεαυτόν δύναμιν και απέκτησας εν τοις θησαυροίς σου χρυσίον και αργύριον·
\par 5 διά της μεγάλης σοφίας σου ηύξησας τα πλούτη σου διά του εμπορίου, και η καρδία σου υψώθη διά την δύναμίν σου·
\par 6 διά τούτο, ούτω λέγει Κύριος ο Θεός· Επειδή έκαμες την καρδίαν σου ως καρδίαν Θεού,
\par 7 ιδού, διά τούτο θέλω φέρει εναντίον σου ξένους, τους τρομερωτέρους των εθνών· και θέλουσιν εκσπάσει τα ξίφη αυτών κατά του κάλλους της σοφίας σου και θέλουσι μολύνει την λαμπρότητά σου.
\par 8 Θέλουσι σε καταβιβάσει εις τον λάκκον, και θέλεις τελευτήσει με τον θάνατον των πεφονευμένων εν τη καρδία των θαλασσών.
\par 9 Θέλεις λέγει έτι ενώπιον του φονεύοντός σε, Εγώ είμαι θεός, άνθρωπος ων και ουχί θεός, εν ταις χερσί του φονεύοντός σε;
\par 10 Θάνατον απεριτμήτων θέλεις θανατωθή διά χειρός των ξένων· διότι εγώ ελάλησα, λέγει Κύριος ο Θεός.
\par 11 Και έγεινε λόγος Κυρίου προς εμέ, λέγων,
\par 12 Υιέ ανθρώπου, ανάλαβε θρήνον επί τον βασιλέα της Τύρου και ειπέ προς αυτόν, Ούτω λέγει Κύριος ο Θεός· Συ επεσφράγισας τα πάντα, είσαι πλήρης σοφίας και τέλειος εις κάλλος.
\par 13 Εστάθης εν Εδέμ τω παραδείσω του Θεού· ήσο περιεσκεπασμένος υπό παντός λίθου τιμίου, υπό σαρδίου, τοπαζίου και αδάμαντος, βηρυλλίου, όνυχος και ιάσπεως, σαπφείρου, σμαράγδου και άνθρακος και χρυσίου· η υπηρεσία των τυμπάνων σου και των αυλών σου ήτο ητοιμασμένη διά σε την ημέραν καθ' ην εκτίσθης.
\par 14 Ησο χερούβ κεχρισμένον, διά να επισκιάζης· και εγώ σε έστησα· ήσο εν τω όρει τω αγίω του Θεού· περιεπάτεις εν μέσω λίθων πυρίνων.
\par 15 Ησο τέλειος εν ταις οδοίς σου αφ' ης ημέρας εκτίσθης, εωσού ευρέθη αδικία εν σοι.
\par 16 Εκ του πλήθους του εμπορίου σου ενέπλησαν το μέσον σου από ανομίας και ήμαρτες· διά τούτο θέλω σε απορρίψει ως βέβηλον από του όρους του Θεού, και θέλω σε καταστρέψει εν μέσω των πυρίνων λίθων, χερούβ επισκιάζον.
\par 17 Η καρδία σου υψώθη διά το κάλλος σου· έφθειρας την σοφίαν σου διά την λαμπρότητά σου· θέλω σε ρίψει κατά γής· θέλω σε εκθέσει ενώπιον των βασιλέων, διά να βλέπωσιν εις σε.
\par 18 Εβεβήλωσας τα ιερά σου διά το πλήθος των αμαρτιών σου, διά τας αδικίας του εμπορίου σου· διά τούτο θέλω εκβάλει πυρ εκ μέσου σου, το οποίον θέλει σε καταφάγει· και θέλω σε καταστήσει σποδόν επί της γης, ενώπιον πάντων των βλεπόντων σε.
\par 19 Πάντες οι γνωρίζοντές σε μεταξύ των λαών θέλουσιν εκπλαγή διά σέ· φρίκη θέλεις είσθαι και δεν θέλεις υπάρξει έως αιώνος.
\par 20 Και έγεινε λόγος Κυρίου προς εμέ, λέγων,
\par 21 Υιέ ανθρώπου, στήριξον το πρόσωπόν σου επί την Σιδώνα, και προφήτευσον κατ' αυτής
\par 22 και ειπέ, Ούτω λέγει Κύριος ο Θεός· Ιδού, εγώ είμαι εναντίον σου, Σιδών· και θέλω δοξασθή εν μέσω σου· και θέλουσι γνωρίσει ότι εγώ είμαι ο Κύριος, όταν εκτελέσω κρίσεις εις αυτήν και αγιασθώ εν αυτή.
\par 23 Διότι θέλω εξαποστείλει εις αυτήν θανατικόν και αίμα εν ταις οδοίς αυτής· και οι τετραυματισμένοι θέλουσι πέσει εν μέσω αυτής διά μαχαίρας ελθούσης επ' αυτήν κυκλόθεν· και θέλουσι γνωρίσει ότι εγώ είμαι ο Κύριος.
\par 24 Και δεν θέλει είσθαι πλέον εν τω οίκω Ισραήλ σκόλοψ πικρίας και άκανθα οδύνης εκ πάντων των πέριξ αυτών των καταφρονούντων αυτούς· και θέλουσι γνωρίσει ότι εγώ είμαι Κύριος ο Θεός.
\par 25 Ούτω λέγει Κύριος ο Θεός· Όταν συνάξω τον οίκον Ισραήλ εκ των λαών, μεταξύ των οποίων είναι διεσκορπισμένοι, και αγιασθώ εν αυτοίς ενώπιον των εθνών, τότε θέλουσι κατοικήσει εν τη γη αυτών, την οποίαν έδωκα εις τον δούλον μου τον Ιακώβ.
\par 26 Και θέλουσι κατοικήσει εν αυτή εν ασφαλεία και θέλουσιν οικοδομήσει οικίας και φυτεύσει αμπελώνας· ναι, θέλουσι κατοικήσει εν ασφαλεία, όταν εκτελέσω κρίσεις επί πάντας τους καταφρονήσαντας αυτούς κυκλόθεν αυτών· και θέλουσι γνωρίσει ότι εγώ είμαι Κύριος ο Θεός αυτών.

\chapter{29}

\par 1 Εν τω δεκάτω έτει τω δεκάτω μηνί, τη δωδεκάτη του μηνός, έγεινε λόγος Κυρίου προς εμέ, λέγων,
\par 2 Υιέ ανθρώπου, στήριξον το πρόσωπόν σου επί Φαραώ τον βασιλέα της Αιγύπτου και προφήτευσον κατ' αυτού και καθ' όλης της Αιγύπτου·
\par 3 λάλησον και ειπέ, Ούτω λέγει Κύριος ο Θεός· Ιδού, εγώ είμαι εναντίον σου, Φαραώ βασιλεύ Αιγύπτου, μεγάλε δράκων, κοιτόμενε εν μέσω των ποταμών αυτού· όστις είπας, Ο ποταμός μου είναι εμού και εγώ έκαμον αυτόν δι' εμαυτόν.
\par 4 Και θέλω βάλει άγκιστρα εις τας σιαγόνας σου, και θέλω προσκολλήσει τους ιχθύας του ποταμού σου εις τα λέπη σου, και θέλω σε ανασύρει εκ μέσου των ποταμών σου· και πάντες οι ιχθύες των ποταμών σου θέλουσι προσκολληθή εις τα λέπη σου.
\par 5 Και θέλω σε εκρίψει εν τη ερήμω, σε και πάντας τους ιχθύας των ποταμών σου· θέλεις πέσει επί πρόσωπον της πεδιάδος· δεν θέλεις συναχθή ουδέ περισταλθή· εις τα θηρία της γης και εις τα πετεινά του ουρανού σε παρέδωκα εις βρώσιν·
\par 6 και πάντες οι κατοικούντες την Αίγυπτον θέλουσι γνωρίσει ότι εγώ είμαι ο Κύριος· διότι εστάθησαν ράβδος καλαμίνη εις τον οίκον Ισραήλ.
\par 7 Ότε σε επίασαν με την χείρα, συνετρίβης και ετρύπησας όλον τον ώμον αυτών· και ότε εστηρίχθησαν επί σε, συνεθλάσθης και συνέκαμψας πάσας τας οσφύας αυτών.
\par 8 Διά τούτο ούτω λέγει Κύριος ο Θεός· Ιδού, θέλω φέρει ρομφαίαν επί σε και θέλω εκκόψει από σου άνθρωπον και κτήνος.
\par 9 Και η γη της Αιγύπτου θέλει είσθαι θάμβος και ερημία· και θέλουσι γνωρίσει ότι εγώ είμαι ο Κύριος· διότι είπεν, Ο ποταμός είναι εμού και εγώ έκαμον αυτόν.
\par 10 Διά τούτο ιδού, εγώ είμαι εναντίον σου και εναντίον των ποταμών σου· και θέλω κάμει την γην της Αιγύπτου όλως έρημον και θάμβος, από Μιγδώλ μέχρι Συήνης και μέχρι των ορίων της Αιθιοπίας.
\par 11 Πούς ανθρώπου δεν θέλει διέλθει δι' αυτής ουδέ πους κτήνους θέλει διέλθει δι' αυτής ουδέ θέλει κατοικηθή τεσσαράκοντα έτη.
\par 12 Και θέλω κάμει την γην της Αιγύπτου θάμβος, εν μέσω των ηρημωμένων τόπων, και αι πόλεις αυτής εν μέσω των πόλεων των ηρημωμένων θέλουσιν είσθαι θάμβος τεσσαράκοντα έτη· και θέλω διασπείρει τους Αιγυπτίους μεταξύ των εθνών και διασκορπίσει αυτούς εις τους τόπους.
\par 13 Πλην ούτω λέγει Κύριος ο Θεός· Εν τω τέλει των τεσσαράκοντα ετών θέλω συνάξει τους Αιγυπτίους εκ των λαών, εις τους οποίους ήσαν διεσκορπισμένοι·
\par 14 και θέλω επαναγάγει τους αιχμαλώτους της Αιγύπτου και επιστρέψει αυτούς εις την γην Παθρώς, εις την γην της καταγωγής αυτών· και θέλουσιν είσθαι εκεί βασίλειον ποταπόν.
\par 15 Θέλει είσθαι το ποταπώτερον των βασιλείων· και δεν θέλει υψωθή πλέον επί τα έθνη· διότι θέλω ελαττώσει αυτούς, διά να μη δεσπόζωσιν επί τα έθνη.
\par 16 Και δεν θέλει είσθαι πλέον το θάρρος του οίκου Ισραήλ, αναμιμνήσκον την ανομίαν αυτών, αποβλεπόντων οπίσω αυτών· και θέλουσι γνωρίσει ότι εγώ είμαι Κύριος ο Θεός.
\par 17 Και εν τω εικοστώ εβδόμω έτει, τω πρώτω μηνί, τη πρώτη του μηνός, έγεινε λόγος Κυρίου προς εμέ, λέγων,
\par 18 Υιέ ανθρώπου, Ναβουχοδονόσορ ο βασιλεύς της Βαβυλώνος εδούλευσε το στράτευμα αυτού δουλείαν μεγάλην κατά της Τύρου· πάσα κεφαλή εφαλακρώθη και πας ώμος εξεδάρθη· μισθόν όμως διά την Τύρον δεν έλαβεν ούτε αυτός ούτε το στράτευμα αυτού διά την δουλείαν, την οποίαν εδούλευσε κατ' αυτής·
\par 19 διά τούτο ούτω λέγει Κύριος ο Θεός· Ιδού, εγώ δίδω την γην της Αιγύπτου εις τον Ναβουχοδονόσορ βασιλέα της Βαβυλώνος· και θέλει σηκώσει το πλήθος αυτής και θέλει λεηλατήσει την λεηλασίαν αυτής και λαφυραγωγήσει τα λάφυρα αυτής· και τούτο θέλει είσθαι ο μισθός εις το στράτευμα αυτού.
\par 20 Έδωκα εις αυτόν την γην της Αιγύπτου διά τον κόπον αυτού, με τον οποίον εδούλευσε κατ' αυτής, επειδή ηγωνίσθησαν δι' εμέ, λέγει Κύριος ο Θεός.
\par 21 Εν εκείνη τη ημέρα θέλω κάμει να βλαστήση το κέρας του οίκου Ισραήλ, και θέλω σε κάμει να ανοίξης στόμα εν μέσω αυτών· και θέλουσι γνωρίσει ότι εγώ είμαι ο Κύριος.

\chapter{30}

\par 1 Και έγεινε λόγος Κυρίου προς εμέ, λέγων,
\par 2 Υιέ ανθρώπου, προφήτευσον και ειπέ, Ούτω λέγει Κύριος ο Θεός. Ολολύζετε, Ουαί, διά την ημέραν.
\par 3 Διότι πλησίον είναι η ημέρα, ναι, η ημέρα του Κυρίου είναι πλησίον, ημέρα νεφώδης· ο καιρός των εθνών θέλει είσθαι.
\par 4 Και η μάχαιρα θέλει ελθεί, επί την Αίγυπτον και μέγας τρόμος θέλει είσθαι εν τη Αιθιοπία, όταν οι τετραυματισμένοι πέσωσιν εν Αιγύπτω, και θέλουσι λάβει το πλήθος αυτής και θέλουσι καταστρέψει τα θεμέλια αυτής.
\par 5 Αιθίοπες και Λίβυες και Λύδιοι και πάντες οι σύμμικτοι λαοί, και ο Χούβ και οι υιοί της συμμάχου γης, θέλουσι πέσει μετ' αυτών εν μαχαίρα.
\par 6 Ούτω λέγει Κύριος· Θέλουσι πέσει και οι υποστηρίζοντες την Αίγυπτον, και η υπερηφανία της δυνάμεως αυτής θέλει καταβληθή· από Μιγδώλ μέχρι Συήνης θέλουσι πέσει εν αυτή διά μαχαίρας, λέγει Κύριος ο Θεός.
\par 7 Και θέλουσιν αφανισθή εν μέσω των ηφανισμένων τόπων, και αι πόλεις αυτής θέλουσιν είσθαι εν μέσω των ηρημωμένων πόλεων.
\par 8 Και θέλουσι γνωρίσει ότι εγώ είμαι ο Κύριος, όταν βάλω πυρ εις την Αίγυπτον και συντριφθώσι πάντες οι βοηθούντες αυτήν.
\par 9 Εν εκείνη τη ημέρα θέλουσιν εξέλθει απ' εμού μηνυταί εν πλοίοις, διά να εκπλήξωσι τους αμερίμνους Αιθίοπας· και τρόμος μέγας θέλει επέλθει επ' αυτούς, καθώς εν τη ημέρα της Αιγύπτου· διότι, ιδού, έρχεται.
\par 10 Ούτω λέγει Κύριος ο Θεός· Και θέλω απολέσει το πλήθος της Αιγύπτου διά χειρός του Ναβουχοδονόσορ βασιλέως της Βαβυλώνος.
\par 11 Αυτός και ο λαός αυτού μετ' αυτού, οι τρομερώτεροι των εθνών, θέλουσι φερθή διά να αφανίσωσι την γήν· και θέλουσιν εκσπάσει τας ρομφαίας αυτών κατά της Αιγύπτου και γεμίσει την γην από τετραυματισμένων.
\par 12 Και θέλω ξηράνει τους ποταμούς και παραδώσει την γην εις χείρας κακών, και θέλω αφανίσει την γην και το πλήρωμα αυτής διά χειρός των ξένων· εγώ ο Κύριος ελάλησα.
\par 13 Ούτω λέγει Κύριος ο Θεός· Και θέλω καταστρέψει τα ξόανα και εξαλείψει τα είδωλα από Νωφ, και δεν θέλει υπάρχει πλέον άρχων εκ της γης της Αιγύπτου, και θέλω εμβάλει φόβον εις την γην της Αιγύπτου.
\par 14 Και θέλω αφανίσει την Παθρώς και βάλει πυρ εις την Τάνιν και εκτελέσει κρίσεις εν Νω.
\par 15 Και θέλω εκχέει τον θυμόν μου επί Σιν την ισχύν της Αιγύπτου, και θέλω εκκόψει το πλήθος της Νω.
\par 16 Και θέλω βάλει πυρ εις την Αίγυπτον· η Σιν θέλει λάβει μέγαν τρόμον και η Νω θέλει διασπαραχθή και η Νωφ θέλει είσθαι καθ' ημέραν εν αγωνία.
\par 17 Οι νεανίσκοι της Αβήν και της Πι-βεσέθ θέλουσι πέσει εν μαχαίρα, και αύται θέλουσιν υπάγει εις αιχμαλωσίαν.
\par 18 Και εν Τάφνης η ημέρα θέλει συσκοτάσει, όταν συντρίψω εκεί τα σκήπτρα της Αιγύπτου· και η έπαρσις της δυνάμεως αυτής θέλει παύσει εν αυτή· ταύτην δε, νέφος θέλει σκεπάσει αυτήν, και αι θυγατέρες αυτής θέλουσιν υπάγει εις αιχμαλωσίαν.
\par 19 Και θέλω εκτελέσει κρίσεις επί την Αίγυπτον· και θέλουσι γνωρίσει ότι εγώ είμαι ο Κύριος.
\par 20 Και εν τω ενδεκάτω έτει, τω πρώτω μηνί, τη εβδόμη του μηνός, έγεινε λόγος Κυρίου προς εμέ λέγων,
\par 21 Υιέ ανθρώπου, συνέθλασα τον βραχίονα του Φαραώ βασιλέως της Αιγύπτου· και ιδού, δεν θέλει επιδεθή προς θεραπείαν, ώστε να περιτυλίξωσιν αυτόν με επιδέσματα διά να δοθή εις αυτόν δύναμις να κρατή μάχαιραν.
\par 22 Διά τούτο ούτω λέγει Κύριος ο Θεός· Ιδού, εγώ είμαι εναντίον του Φαραώ βασιλέως της Αιγύπτου και θέλω συνθλάσει τους βραχίονας αυτού, τον δυνατόν και τον συντεθλασμένον· και θέλω κάμει την μάχαιραν να εκπέση από της χειρός αυτού.
\par 23 Και θέλω διασπείρει τους Αιγυπτίους μεταξύ των εθνών και διασκορπίσει αυτούς εις τους τόπους.
\par 24 Και θέλω ενισχύσει τους βραχίονας του βασιλέως της Βαβυλώνος και θέλω δώσει την ρομφαίαν μου εις την χείρα αυτού, τους δε βραχίονας του Φαραώ θέλω συνθλάσει και θέλει στενάξει έμπροσθεν αυτού με στεναγμούς τετραυματισμένου.
\par 25 Τους βραχίονας όμως του βασιλέως της Βαβυλώνος θέλω ενισχύσει, οι δε βραχίονες του Φαραώ θέλουσι πέσει· και θέλουσι γνωρίσει ότι εγώ είμαι ο Κύριος, όταν δώσω την ρομφαίαν μου εις την χείρα του βασιλέως της Βαβυλώνος· και θέλει εκτείνει αυτήν επί την γην της Αιγύπτου.
\par 26 Και θέλω διασπείρει τους Αιγυπτίους μεταξύ των εθνών και διασκορπίσει αυτούς εις τους τόπους· και θέλουσι γνωρίσει ότι εγώ είμαι ο Κύριος.

\chapter{31}

\par 1 Και εν τω ενδεκάτω έτει, τω τρίτω μηνί, τη πρώτη του μηνός, έγεινε λόγος Κυρίου προς εμέ, λέγων,
\par 2 Υιέ ανθρώπου, ειπέ προς τον Φαραώ βασιλέα της Αιγύπτου και προς το πλήθος αυτού· Με ποίον ώμοιώθης εν τη μεγαλειότητί σου;
\par 3 Ιδού, ο Ασσύριος ήτο κέδρος εν τω Λιβάνω με κλάδους ωραίους, και πυκνός την σκιάν και υψηλός το μέγεθος, και η κορυφή αυτού ήτο εν μέσω κλάδων πυκνών.
\par 4 Τα ύδατα ηύξησαν αυτόν, η άβυσσος ύψωσεν αυτόν με τους ποταμούς αυτής τους ρέοντας κύκλω των φυτών αυτού, και εξέπεμπε τους ρύακας αυτής εις πάντα τα δένδρα του αγρού.
\par 5 Όθεν το ύψος εαυτού ανέβη υπεράνω πάντων των δένδρων του αγρού και οι κλώνοι αυτού επλήθυναν και οι κλάδοι αυτού εξετάνθησαν διά το πλήθος των υδάτων, ενώ εβλάστανε.
\par 6 Πάντα τα πετεινά του ουρανού εφώλευον εν τοις κλώνοις αυτού, και πάντα τα ζώα του αγρού εγέννων υπό τους κλάδους αυτού· υπό δε την σκιάν αυτού κατώκουν πάντα τα μεγάλα έθνη.
\par 7 Ήτο λοιπόν ώραίος κατά το μέγεθος αυτού και κατά την έκτασιν των κλάδων αυτού, διότι αι ρίζαι αυτού ήσαν πλησίον υδάτων πολλών.
\par 8 Αι κέδροι εν τω παραδείσω του Θεού δεν ηδύναντο να κρύψωσιν αυτόν· αι έλατοι δεν εξισούντο με τους κλώνους αυτού, και αι κάστανοι δεν εξισούντο με τους κλάδους αυτού· ουδέν δένδρον εν τω παραδείσω του Θεού ώμοίαζεν αυτόν κατά την ώραιότητα αυτού.
\par 9 Έκαμον αυτόν ώραίον κατά το πλήθος των κλάδων αυτού, ώστε πάντα τα δένδρα της Εδέμ, τα εν τω παραδείσω του Θεού, εζήλευον αυτόν.
\par 10 Διά τούτο ούτω λέγει Κύριος ο Θεός· Επειδή ύψωσας σεαυτόν υψηλά, και επειδή εσήκωσε την κορυφήν αυτού μεταξύ των πυκνών κλώνων και η καρδία αυτού επήρθη εις το ύψος αυτού,
\par 11 διά τούτο παρέδωκα αυτόν εις την χείρα του δυνάστου των εθνών, όστις θέλει φερθή αξίως προς αυτόν· απέβαλον αυτόν διά την ασέβειαν αυτού.
\par 12 Και ξένοι, οι τρομερώτεροι των εθνών, έκοψαν αυτόν και εγκατέλιπον αυτόν· οι κλάδοι αυτού έπεσον επί τα όρη και εν πάσαις ταις φάραγξι και οι κλώνοι αυτού συνετρίφθησαν υπό πάντων των ποταμών της γης, και πάντες οι λαοί της γης κατέβησαν από της σκιάς αυτού και εγκατέλιπον αυτόν.
\par 13 Επί του πτώματος αυτού θέλουσιν επικάθησθαι πάντα τα πετεινά του ουρανού και επί τους κλάδους αυτού θέλουσιν είσθαι πάντα τα ζώα του αγρού·
\par 14 διά να μη υψωθή εν τω ύψει αυτού ουδέν εκ των δένδρων των υδάτων μηδέ να σηκώσωσι την κορυφήν αυτών μεταξύ των πυκνών κλάδων, και εκ πάντων των πινόντων ύδωρ, ουδέν εκ τούτων να μη στέκηται εν τω ύψει αυτού· διότι πάντα παρεδόθησαν εις τον θάνατον, εις τα κατώτατα της γης, εν μέσω των υιών των ανθρώπων, μετά των καταβαινόντων εις λάκκον.
\par 15 Ούτω λέγει Κύριος ο Θεός· Καθ' ην ημέραν κατέβη εις τον άδην, έκαμον να γείνη πένθος· εσκέπασα την άβυσσον δι' αυτόν και εμπόδισα τους ποταμούς αυτής και τα μεγάλα ύδατα εκρατήθησαν· και έκαμον να πενθήση ο Λίβανος δι' αυτόν και πάντα τα δένδρα του αγρού εμαράνθησαν δι' αυτόν.
\par 16 Έκαμον τα έθνη να σεισθώσιν εις τον ήχον της πτώσεως αυτού, ότε κατεβίβασα αυτόν εις τον άδην μετά των καταβαινόντων εις λάκκον· και πάντα τα δένδρα της Εδέμ, τα εκλεκτά και τα καλά του Λιβάνου, πάντα τα πίνοντα ύδωρ, παρηγορήθησαν εν τοις κατωτάτοις της γης.
\par 17 Και αυτοί ότι κατέβησαν εις τον άδην μετ' αυτού, προς τους τεθανατωμένους εν μαχαίρα· και όσοι ήσαν ο βραχίων αυτού, οι κατοικούντες υπό την σκιάν αυτού εν μέσω των εθνών.
\par 18 Με ποίον ώμοιώθης ούτως εν τη δόξη και εν τη μεγαλειότητι, μεταξύ των δένδρων της Εδέμ; θέλεις όμως καταβιβασθή μετά των δένδρων της Εδέμ εις τα κατώτατα της γής· θέλεις κοίτεσθαι εν μέσω των απεριτμήτων μετά των τεθανατωμένων εν μαχαίρα· ούτος είναι ο Φαραώ και άπαν το πλήθος αυτού, λέγει Κύριος ο Θεός.

\chapter{32}

\par 1 Και εν τω δωδεκάτω έτει, τω δωδεκάτω μηνί, τη πρώτη του μηνός, έγεινε λόγος Κυρίου προς εμέ, λέγων,
\par 2 Υιέ ανθρώπου, ανάλαβε θρήνον επί τον Φαραώ βασιλέα της Αιγύπτου και ειπέ προς αυτόν, Ωμοιώθης με σκύμνον λέοντος μεταξύ των εθνών και είσαι ως δράκων εν ταις θαλάσσαις· και εφώρμησας εις τους ποταμούς σου και ετάραττες τα ύδατα με τους πόδας σου και κατεπάτεις τους ποταμούς αυτών.
\par 3 Ούτω λέγει Κύριος ο Θεός· διά τούτο θέλω εξαπλώσει το δίκτυόν μου επί σε με άθροισμα πολλών λαών, και θέλουσι σε ανασύρει εν τη σαγήνη μου.
\par 4 Και θέλω σε εγκαταλείψει εν τη γη, θέλω σε εκρίψει επί το πρόσωπον της πεδιάδος, και θέλω επικαθίσει επί σε πάντα τα πετεινά του ουρανού και χορτάσει από σου τα θηρία πάσης της γης.
\par 5 Και θέλω εκθέσει τας σάρκας σου επί τα όρη, και εμπλήσει τας κοιλάδας από των σωρών του πτώματός σου.
\par 6 Και την γην, όπου πλέεις, θέλω ποτίσει με το αίμα σου έως των ορέων· και οι ποταμοί θέλουσιν εμπλησθή από σου.
\par 7 Και όταν σε αποσβέσω, θέλω περικαλύψει τον ουρανόν και συσκοτάσει τους αστέρας αυτού· θέλω περικαλύψει εν νεφέλη τον ήλιον και σελήνη δεν θέλει φέγγει το φως αυτής.
\par 8 Πάντας τους λαμπρούς φωστήρας του ουρανού θέλω συσκοτάσει επί σε, και θέλω επιβάλει σκότος επί την γην σου, λέγει Κύριος ο Θεός.
\par 9 Και θέλω κάμει να φρίξη η καρδία πολλών λαών, όταν φέρω τον συντριμμόν σου μεταξύ των εθνών, εις τόπους τους οποίους δεν εγνώρισας.
\par 10 Και θέλω κάμει πολλούς λαούς να εκπλαγώσι διά σε και οι βασιλείς αυτών θέλουσι φρίξει σφόδρα διά σε, όταν διασείσω την ρομφαίαν μου ενώπιον αυτών· και θέλουσι τρέμει κατά πάσαν στιγμήν, έκαστος διά την ζωήν αυτού, εν τη ημέρα της πτώσεώς σου.
\par 11 Διότι ούτω λέγει Κύριος ο Θεός· Η ρομφαία του βασιλέως της Βαβυλώνος θέλει ελθεί επί σε.
\par 12 Εν μαχαίραις ισχυρών θέλω καταβάλει το πλήθός σου· πάντες ούτοι είναι οι τρομερώτεροι των εθνών· και θέλουσι πορθήσει την έπαρσιν της Αιγύπτου και άπαν το πλήθος αυτής θέλει καταστραφή.
\par 13 Και θέλω εξαφανίσει πάντα τα κτήνη αυτής από πλησίον υδάτων πολλών, και δεν θέλει πλέον ταράξει αυτά πους ανθρώπου και ίχνος κτήνους δεν θέλει ταράξει αυτά.
\par 14 Τότε θέλω ησυχάσει τα ύδατα αυτών και κάμει τους ποταμούς αυτών να ρέωσιν ως έλαιον, λέγει Κύριος ο Θεός.
\par 15 Όταν κάμω την γην της Αιγύπτου θάμβος, και ερημωθή η γη από του πληρώματος αυτής, όταν πατάξω πάντας τους κατοικούντας εν αυτή, τότε θέλουσι γνωρίσει ότι εγώ είμαι ο Κύριος.
\par 16 Ούτος είναι ο θρήνος, με τον οποίον θέλουσι θρηνήσει αυτήν· αι θυγατέρες των εθνών θέλουσι θρηνήσει αυτήν· θέλουσι θρηνήσει διά την Αίγυπτον και δι' άπαν το πλήθος αυτής, λέγει Κύριος ο Θεός.
\par 17 Και εν τω δωδεκάτω έτει τη δεκάτη πέμπτη του μηνός, έγεινε λόγος Κυρίου προς εμέ, λέγων,
\par 18 Υιέ ανθρώπου, θρήνησον διά το πλήθος της Αιγύπτου και καταβίβασον αυτούς, αυτήν και τας θυγατέρας των ισχυρών εθνών, εις τα κατώτατα της γης, μετά των καταβαινόντων εις λάκκον.
\par 19 Τίνος είσαι ώραιοτέρα; κατάβηθι και κοίτου μετά των απεριτμήτων.
\par 20 Θέλουσι πέσει εν μέσω των τεθανατωμένων εν μαχαίρα· εις την μάχαιραν παρεδόθη αυτή· σύρετε αυτήν και πάντα τα πλήθη αυτής.
\par 21 Οι ισχυρότεροι μεταξύ των δυνατών θέλουσι λαλήσει προς αυτόν εκ μέσου του άδου μετά των βοηθούντων αυτόν· κατέβησαν, κοίτονται απερίτμητοι, τεθανατωμένοι εν μαχαίρα.
\par 22 Εκεί είναι ο Ασσούρ και άπαν το άθροισμα αυτού· οι τάφοι αυτού είναι κύκλω αυτού· πάντες ούτοι τεθανατωμένοι, πεπτωκότες εν μαχαίρα.
\par 23 Διότι οι τάφοι αυτού είναι τεθειμένοι εις τα βάθη του λάκκου και το άθροισμα αυτού κύκλω του τάφου αυτού· πάντες ούτοι τεθανατωμένοι, πεπτωκότες εν μαχαίρα, οίτινες διέδιδον τρόμον εις την γην των ζώντων.
\par 24 Εκεί είναι ο Ελάμ και άπαν το πλήθος αυτού κύκλω του τάφου αυτού· πάντες ούτοι τεθανατωμένοι, πεπτωκότες εν μαχαίρα, καταβάντες απερίτμητοι εις τα κατώτατα της γης, οίτινες διέδιδον τον τρόμον αυτών εις την γην των ζώντων· και έλαβον την καταισχύνην αυτών μετά των καταβαινόντων εις λάκκον.
\par 25 Έθεσαν εις αυτόν κλίνην μετά παντός του πλήθους αυτού εν μέσω των τεθανατωμένων· οι τάφοι αυτού είναι κύκλω αυτού· πάντες ούτοι απερίτμητοι, τεθανατωμένοι εν μαχαίρα, αν και διεδόθη ο τρόμος αυτών εις την γην των ζώντων· και έλαβον την καταισχύνην αυτών μετά των καταβαινόντων εις λάκκον· ετέθη εν μέσω των τεθανατωμένων.
\par 26 Εκεί είναι ο Μεσέχ, ο Θουβάλ και άπαν το πλήθος αυτού· οι τάφοι αυτού είναι κύκλω αυτού· πάντες ούτοι απερίτμητοι, τεθανατωμένοι εν μαχαίρα, αν και διέδωκαν τον τρόμον αυτών εις την γην των ζώντων.
\par 27 Πλην δεν κοίτονται μετά των πεσόντων ισχυρών εκ των απεριτμήτων, οίτινες κατέβησαν εις τον άδην μετά των πολεμικών αυτών όπλων· και έθεσαν τας μαχαίρας αυτών υπό τας κεφαλάς αυτών· αλλ' αι ανομίαι αυτών θέλουσιν είσθαι επί τα οστά αυτών, αν και ήσαν τρόμος των ισχυρών εν τη γη των ζώντων.
\par 28 Ναι, συ θέλεις συντριφθή εν μέσω των απεριτμήτων, και θέλεις κοίτεσθαι μετά των τεθανατωμένων εν μαχαίρα.
\par 29 Εκεί είναι ο Εδώμ, οι βασιλείς αυτού και πάντες οι ηγεμόνες αυτού, οίτινες μετά της δυνάμεως αυτών ετέθησαν μεταξύ των τεθανατωμένων εν μαχαίρα· ούτοι θέλουσι κοίτεσθαι μετά των απεριτμήτων και μετά των καταβαινόντων εις λάκκον.
\par 30 Εκεί είναι οι ηγεμόνες του βορρά, πάντες ούτοι, και πάντες οι Σιδώνιοι, οίτινες κατέβησαν μετά των τεθανατωμένων, εν τω τρόμω αυτών, κατησχυμμένοι εν τη δυνάμει αυτών· και κοίτονται απερίτμητοι μετά των τεθανατωμένων εν μαχαίρα, και έλαβον την καταισχύνην αυτών μετά των καταβαινόντων εις λάκκον.
\par 31 Ο Φαραώ θέλει ιδεί αυτούς και παρηγορηθή δι' άπαν το πλήθος αυτού, ο Φαραώ και άπαν το στράτευμα αυτού, οι τεθανατωμένοι εν μαχαίρα, λέγει Κύριος ο Θεός.
\par 32 Διότι έδωκα τον τρόμον μου εις την γην των ζώντων· και θέλει κοίτεσθαι εν μέσω των απεριτμήτων μετά των τεθανατωμένων εν μαχαίρα· ο Φαραώ και άπαν το πλήθος αυτού, λέγει Κύριος ο Θεός.

\chapter{33}

\par 1 Και έγεινε λόγος Κυρίου προς εμέ, λέγων,
\par 2 Υιέ ανθρώπου, λάλησον προς τους υιούς του λαού σου και ειπέ προς αυτούς· Όταν επιφέρω την ρομφαίαν επί γην τινά και ο λαός της γης λάβη άνθρωπον τινά εκ μέσου αυτού και θέσωσιν αυτόν φύλακα εις εαυτούς,
\par 3 και αυτός, ιδών την ρομφαίαν επερχομένην επί την γην, σαλπίση εν σάλπιγγι και σημάνη εις τον λαόν,
\par 4 τότε όστις ακούση την φωνήν της σάλπιγγος και δεν φυλαχθή, εάν η ρομφαία ελθούσα καταλάβη αυτόν, το αίμα αυτού θέλει είσθαι επί την κεφαλήν αυτού.
\par 5 Ήκουσε την φωνήν της σάλπιγγος και δεν εφυλάχθη· το αίμα αυτού θέλει είσθαι επ' αυτόν. Όστις όμως φυλαχθή, θέλει διασώσει την ζωήν αυτού.
\par 6 Αλλ' εάν ο φύλαξ, ιδών την ρομφαίαν επερχομένην, δεν σαλπίση εν τη σάλπιγγι και ο λαός δεν φυλαχθή, η δε ρομφαία ελθούσα καταλάβη τινά εξ αυτών, ούτος μεν κατελήφθη διά την ανομίαν αυτού, πλην το αίμα αυτού θέλω εκζητήσει εκ της χειρός του φύλακος.
\par 7 Και συ, υιέ ανθρώπου, εγώ σε έθεσα φύλακα επί τον οίκον Ισραήλ· άκουσον λοιπόν λόγον εκ του στόματός μου και νουθέτησον αυτούς παρ' εμού·
\par 8 Όταν λέγω εις τον άνομον, Άνομε, θέλεις εξάπαντος θανατωθή· και συ δεν λαλήσης διά να αποτρέψης τον άνομον από της οδού αυτού, εκείνος μεν ο άνομος θέλει αποθάνει εν τη ανομία αυτού, πλην εκ της χειρός σου θέλω εκζητήσει το αίμα αυτού.
\par 9 Αλλ' εάν συ αποτρέπης τον άνομον από της οδού αυτού διά να επιστρέψη απ' αυτής, και δεν επιστρέψη από της οδού αυτού, εκείνος μεν θέλει αποθάνει εν τη ανομία αυτού, συ δε ηλευθέρωσας την ψυχήν σου.
\par 10 Διά τούτο, συ, υιέ ανθρώπου, ειπέ προς τον οίκον Ισραήλ· Ούτω σεις ελαλήσατε, λέγοντες, Εάν αι παραβάσεις ημών και αι αμαρτίαι ημών ήναι εφ' ημάς, και ημείς είμεθα απωλεσμένοι δι' αυτάς, πως θέλομεν ζήσει;
\par 11 Ειπέ προς αυτούς· Ζω εγώ, λέγει Κύριος ο Θεός, δεν θέλω τον θάνατον του αμαρτωλού, αλλά να επιστρέψη ο ασεβής από της οδού αυτού και να ζή· επιστρέψατε, επιστρέψατε από των οδών υμών των πονηρών· διά τι να αποθάνητε, οίκος Ισραήλ;
\par 12 Διά τούτο συ, υιέ ανθρώπου, ειπέ προς τους υιούς του λαού σου, Η δικαιοσύνη του δικαίου δεν θέλει ελευθερώσει αυτόν εν τη ημέρα της παραβάσεως αυτού, και ο ασεβής δεν θέλει πέσει διά την ασέβειαν αυτού, καθ' ην ημέραν επιστρέψη από της ασεβείας αυτού, και ο δίκαιος δεν θέλει δυνηθή να ζήση διά την δικαιοσύνην αυτού, καθ' ην ημέραν αμαρτήση.
\par 13 Όταν είπω προς τον δίκαιον ότι θέλει εξάπαντος ζήσει, και αυτός θαρρών εις την δικαιοσύνην αυτού πράξη αδικίαν, άπασα η δικαιοσύνη αυτού δεν θέλει μνημονευθή· και εν τη αδικία αυτού την οποίαν έπραξεν, εν αυτή θέλει αποθάνει.
\par 14 Και όταν λέγω προς τον ασεβή, Εξάπαντος θέλεις αποθάνει, ο δε επιστρέψας από της αμαρτίας αυτού πράξη κρίσιν και δικαιοσύνην,
\par 15 αποδώση το ενέχυρον ο ασεβής, επιστρέψη το ηρπαγμένον, περιπατή εν τοις διατάγμασι της ζωής μη πράττων αδικίαν, θέλει εξάπαντος ζήσει, δεν θέλει αποθάνει·
\par 16 πάσαι αι αμαρτίαι αυτού, τας οποίας ημάρτησε, δεν θέλουσι πλέον μνημονευθή εις αυτόν· έκαμε κρίσιν και δικαιοσύνην· θέλει εξάπαντος ζήσει.
\par 17 Οι υιοί όμως του λαού σου λέγουσιν, Η οδός του Κυρίου δεν είναι ευθεία. Αλλά τούτων αυτών η οδός δεν είναι ευθεία.
\par 18 Όταν ο δίκαιος επιστρέψη από της δικαιοσύνης αυτού και πράξη αδικίαν, διά τούτο μάλιστα θέλει αποθάνει.
\par 19 Και όταν ο άνομος επιστρέψη από της ανομίας αυτού και πράξη κρίσιν και δικαιοσύνην, αυτός θέλει ζήσει διά τούτο.
\par 20 Σεις όμως λέγετε, Η οδός του Κυρίου δεν είναι ευθεία· οίκος Ισραήλ, θέλω σας κρίνει έκαστον κατά τας οδούς αυτού.
\par 21 Και εν τω δωδεκάτω έτει της αιχμαλωσίας ημών, τω δεκάτω μηνί, τη πέμπτη του μηνός, ήλθε προς εμέ διασεσωσμένος τις εξ Ιερουσαλήμ, λέγων, Ηλώθη η πόλις.
\par 22 Και η χειρ του Κυρίου εστάθη επ' εμέ το εσπέρας πριν έλθη ο διασεσωσμένος, και ήνοιξε το στόμα μου εωσού ήλθε προς εμέ το πρωΐ· και ανοιχθέντος του στόματός μου δεν εσιώπησα πλέον.
\par 23 Και έγεινε λόγος Κυρίου προς εμέ, λέγων,
\par 24 Υιέ ανθρώπου, οι κατοικούντες εκείνας τας ερημώσεις εν τη γη Ισραήλ λαλούσι, λέγοντες, Εις ήτο ο Αβραάμ και εκληρονόμησε την γήν· ημείς δε είμεθα πολλοί· εις ημάς εδόθη η γη διά κληρονομίαν.
\par 25 Διά τούτο ειπέ προς αυτούς, Ούτω λέγει Κύριος ο Θεός· σεις τρώγετε κρέας εν αίματι και σηκόνετε τους οφθαλμούς σας προς τα είδωλά σας και χύνετε αίμα, και θέλετε κληρονομήσει την γην;
\par 26 Σεις στηρίζεσθε επί την ρομφαίαν σας, εργάζεσθε βδελύγματα και μιαίνετε έκαστος την γυναίκα του πλησίον αυτού, και θέλετε κληρονομήσει την γην;
\par 27 Ειπέ ούτω προς αυτούς· Ούτω λέγει Κύριος ο Θεός· Ζω εγώ, οι εν ταις ερημώσεσι θέλουσιν εξάπαντος πέσει εν μαχαίρα, και τον επί το πρόσωπον της πεδιάδος, θέλω παραδώσει αυτόν εις τα θηρία διά να καταφάγωσιν αυτόν, οι δε εν τοις φρουρίοις και εν τοις σπηλαίοις θέλουσιν αποθάνει υπό θανατικού.
\par 28 Διότι θέλω παραδώσει εις όλεθρον και ερήμωσιν την γην, και η έπαρσις της δυνάμεως αυτής θέλει καταβληθή, και τα όρη του Ισραήλ θέλουσιν ερημωθή, ώστε να μη υπάρχη ο διαβαίνων.
\par 29 Και θέλουσι γνωρίσει ότι εγώ είμαι ο Κύριος, όταν παραδώσω εις όλεθρον και ερήμωσιν την γην, διά πάντα τα βδελύγματα αυτών τα οποία έπραξαν.
\par 30 Και συ, υιέ ανθρώπου, οι υιοί του λαού σου λαλούσιν εναντίον σου παρά τα τείχη και εν ταις θύραις των οικιών, και λαλούσι προς αλλήλους, έκαστος προς τον αδελφόν αυτού, λέγοντες, Έλθετε λοιπόν και ακούσατε τις ο λόγος ο εξερχόμενος παρά Κυρίου.
\par 31 Και έρχονται προς σε, καθώς συνάγεται ο λαός, και κάθηται έμπροσθέν σου ο λαός μου και ακούουσι τους λόγους σου, αλλά δεν κάμνουσιν αυτούς· διότι εν τω στόματι αυτών δεικνύουσι πολλήν αγάπην, η καρδία όμως αυτών υπάγει κατόπιν της αισχροκερδείας αυτών.
\par 32 Και ιδού, συ είσαι προς αυτούς ως ερωτικόν άσμα ανθρώπου ηδυφώνου και παίζοντος όργανα καλώς, διότι ακούουσι τους λόγους σου αλλά δεν κάμνουσιν αυτούς.
\par 33 Πλην όταν έλθη τούτο, και ιδού, έρχεται, τότε θέλουσι γνωρίσει ότι εστάθη προφήτης εν μέσω αυτών.

\chapter{34}

\par 1 Και έγεινε λόγος Κυρίου προς εμέ, λέγων,
\par 2 Υιέ ανθρώπου, προφήτευσον επί τους ποιμένας του Ισραήλ· προφήτευσον και ειπέ προς αυτούς, Ούτω λέγει Κύριος ο Θεός προς τους ποιμένας· Ουαί εις τους ποιμένας του Ισραήλ, οίτινες βόσκουσιν εαυτούς· οι ποιμένες δεν βόσκουσι τα ποίμνια;
\par 3 Σεις τρώγετε το πάχος και ενδύεσθε το μαλλίον, σφάζετε τα παχέα· δεν βόσκετε τα ποίμνια.
\par 4 Δεν ενισχύσατε το ασθενές και δεν ιατρεύσατε το κακώς έχον και δεν εκάμετε επίδεσμα εις το συντετριμμένον και δεν επανεφέρατε το πεπλανημένον και δεν εζητήσατε το απολωλός· αλλά εν βία και εν σκληρότητι εδεσπόζετε επ' αυτά.
\par 5 Και διεσκορπίσθησαν, επειδή δεν υπήρχε ποιμήν, και έγειναν κατάβρωμα εις πάντα τα θηρία του αγρού και διεσκορπίσθησαν.
\par 6 Τα πρόβατά μου περιεπλανώντο επί παν όρος και επί πάντα λόφον υψηλόν, και επί παν το πρόσωπον της γης ήσαν διεσκορπισμένα τα πρόβατά μου, και δεν υπήρχεν ο ερευνών ουδέ ο ζητών.
\par 7 Διά τούτο, ακούσατε, ποιμένες, τον λόγον του Κυρίου·
\par 8 Ζω εγώ, λέγει Κύριος ο Θεός, εξάπαντος, επειδή τα πρόβατά μου έγειναν λάφυρον και τα πρόβατά μου έγειναν κατάβρωμα πάντων των θηρίων του αγρού δι' έλλειψιν ποιμένος, και δεν εζήτησαν οι ποιμένες μου τα πρόβατά μου αλλ' οι ποιμένες εβόσκησαν εαυτούς και δεν εβόσκησαν τα πρόβατά μου,
\par 9 διά τούτο, ακούσατε, ποιμένες, τον λόγον του Κυρίου·
\par 10 Ούτω λέγει Κύριος ο Θεός· Ιδού, εγώ είμαι εναντίον των ποιμένων, και θέλω εκζητήσει τα πρόβατά μου εκ της χειρός αυτών και θέλω παύσει αυτούς από του να ποιμαίνωσι τα πρόβατα· και δεν θέλουσι πλέον βόσκει εαυτούς οι ποιμένες, διότι θέλω ελευθερώσει εκ του στόματος αυτών τα πρόβατά μου και δεν θέλουσιν είσθαι κατάβρωμα εις αυτούς.
\par 11 Διότι ούτω λέγει Κύριος ο Θεός· Ιδού, εγώ, εγώ θέλω και αναζητήσει τα πρόβατά μου και επισκεφθή αυτά.
\par 12 Καθώς ο ποιμήν επισκέπτεται το ποίμνιον αυτού, καθ' ην ημέραν ευρίσκεται εν μέσω των προβάτων αυτού διεσκορπισμένων, ούτω θέλω επισκεφθή τα πρόβατά μου και θέλω ελευθερώσει αυτά εκ πάντων των τόπων, όπου ήσαν διεσκορπισμένα, εν ημέρα νεφώδει και ζοφερά.
\par 13 Και θέλω εξαγάγει αυτά εκ των λαών και συνάξει αυτά εκ των τόπων και φέρει αυτά εις την γην αυτών και βοσκήσει αυτά επί τα όρη του Ισραήλ, πλησίον των ποταμών και επί πάντα τα κατοικούμενα της γης.
\par 14 Θέλω βοσκήσει αυτά εν αγαθή νομή, και η μάνδρα αυτών θέλει είσθαι επί των υψηλών ορέων του Ισραήλ· εκεί θέλουσιν αναπαύεσθαι εν μάνδρα καλή, και θέλουσι βόσκεσθαι εν παχεία νομή επί των ορέων του Ισραήλ.
\par 15 Εγώ θέλω βοσκήσει τα πρόβατά μου και εγώ θέλω αναπαύσει αυτά, λέγει Κύριος ο Θεός.
\par 16 Θέλω εκζητήσει το απολωλός και επαναφέρει το πεπλανημένον και επιδέσει το συντετριμμένον και ενισχύσει το ασθενές· το παχύ όμως και το ισχυρόν θέλω καταστρέψει· εν δικαιοσύνη θέλω βοσκήσει αυτά.
\par 17 Και περί υμών, ποίμνιόν μου, ούτω λέγει Κύριος ο Θεός· Ιδού, εγώ θέλω κρίνει αναμέσον προβάτου και προβάτου, αναμέσον κριών και τράγων.
\par 18 Μικρόν είναι εις εσάς, ότι εβοσκήσατε την καλήν βοσκήν, το δε επίλοιπον της βοσκής σας κατεπατείτε με τους πόδας σας; και ότι επίνετε καθαρόν ύδωρ, το δε επίλοιπον εταράττετε με τους πόδας σας;
\par 19 τα δε πρόβατά μου έβοσκον το καταπεπατημένον με τους πόδας σας και έπινον το τεταραγμένον με τους πόδας σας.
\par 20 Διά τούτο ούτω λέγει προς αυτά Κύριος ο Θεός· Ιδού, εγώ, εγώ θέλω η κρίνει αναμέσον προβάτου παχέος και αναμέσον προβάτου ισχνού.
\par 21 Επειδή απωθείτε με πλευρά και με ώμους και κερατίζετε διά των κεράτων σας πάντα τα ασθενή, εωσού διεσκορπίσατε αυτά εις τα έξω,
\par 22 διά τούτο θέλω σώσει τα πρόβατά μου και δεν θέλουσιν είσθαι πλέον λάφυρον· και θέλω κρίνει αναμέσον προβάτου και προβάτου.
\par 23 Και θέλω καταστήσει επ' αυτά ένα ποιμένα και θέλει ποιμαίνει αυτά, τον δούλον μου Δαβίδ· αυτός θέλει ποιμαίνει αυτά και αυτός θέλει είσθαι ποιμήν αυτών.
\par 24 Και εγώ ο Κύριος θέλω είσθαι Θεός αυτών και ο δούλός μου Δαβίδ άρχων εν μέσω αυτών· εγώ ο Κύριος ελάλησα.
\par 25 Και θέλω κάμει προς αυτά διαθήκην ειρήνης· και θέλω αφανίσει από της γης τα πονηρά θηρία· και θέλουσι κατοικήσει ασφαλώς εν τη ερήμω και κοιμάσθαι εν τοις δρυμοίς.
\par 26 Και θέλω καταστήσει ευλογίαν αυτά και τα πέριξ του όρους μου, και θέλω καταβιβάζει την βροχήν εν τω καιρώ αυτής· βροχή ευλογίας θέλει είσθαι.
\par 27 Και τα δένδρα του αγρού θέλουσιν αποδίδει τον καρπόν αυτών και η γη θέλει δίδει το προϊόν αυτής και θέλουσιν είσθαι ασφαλείς εν τη γη αυτών· και θέλουσι γνωρίσει ότι εγώ είμαι ο Κύριος, όταν συντρίψω τα δεσμά του ζυγού αυτών και ελευθερώσω αυτούς εκ της χειρός των καταδουλωσάντων αυτούς.
\par 28 Και δεν θέλουσιν είσθαι πλέον λάφυρον εις τα έθνη, και τα θηρία της γης δεν θέλουσι κατατρώγει αυτούς· αλλά θέλουσι κατοικεί ασφαλώς και δεν θέλει υπάρχει ο εκφοβών.
\par 29 Και θέλω αναστήσει εις αυτούς φυτόν ονομαστόν, και δεν θέλουσι πλέον φθείρεσθαι υπό πείνης εν τη γη και δεν θέλουσι φέρει πλέον την ύβριν των εθνών.
\par 30 Και θέλουσι γνωρίσει ότι εγώ Κύριος ο Θεός αυτών είμαι μετ' αυτών και αυτοί, ο οίκος Ισραήλ, λαός μου, λέγει Κύριος ο Θεός.
\par 31 Και σεις, πρόβατά μου, τα πρόβατα της βοσκής μου, σεις είσθε άνθρωποι, και εγώ ο Θεός σας, λέγει Κύριος ο Θεός.

\chapter{35}

\par 1 Και έγεινε λόγος Κυρίου προς εμέ, λέγων,
\par 2 Υιέ ανθρώπου, στήριξον το πρόσωπόν σου επί το όρος Σηείρ και προφήτευσον επ' αυτό·
\par 3 και ειπέ προς αυτό, Ούτω λέγει Κύριος ο Θεός· Ιδού, όρος Σηείρ, εγώ είμαι εναντίον σου· και θέλω εκτείνει την χείρα μου κατά σου, και θέλω σε παραδώσει εις όλεθρον και ερήμωσιν.
\par 4 Θέλω αφανίσει τας πόλεις σου και συ θέλεις είσθαι ερήμωσις, και θέλεις γνωρίσει ότι εγώ είμαι ο Κύριος.
\par 5 Επειδή εφύλαξας παλαιόν μίσος και παρέδωκας τους υιούς Ισραήλ εις χείρα ρομφαίας εν τω καιρώ της θλίψεως αυτών, ότε η ανομία αυτών έφθασεν εις το άκρον,
\par 6 διά τούτο, ζω εγώ, λέγει Κύριος ο Θεός, θέλω σε παραδώσει εις αίμα και αίμα θέλει σε καταδιώκει· επειδή δεν εμίσησας το αίμα, αίμα λοιπόν θέλει σε καταδιώκει·
\par 7 και θέλω παραδώσει εις παντελή ερήμωσιν το όρος Σηείρ και θέλω εξαλείψει απ' αυτού τον διαβαίνοντα και τον επιστρέφοντα.
\par 8 Και θέλω γεμίσει τα όρη αυτού από των τεθανατωμένων αυτού· εν τοις όρεσί σου και εν ταις φάραγξί σου και εν πάσι τοις ποταμοίς σου θέλουσι πέσει οι τεθανατωμένοι εν μαχαίρα.
\par 9 Θέλω σε καταστήσει ερημίαν αιώνιον, και αι πόλεις σου δεν θέλουσι κατοικηθή· και θέλετε γνωρίσει ότι εγώ είμαι ο Κύριος.
\par 10 Επειδή είπας, τα δύο ταύτα έθνη και οι δύο ούτοι τόποι θέλουσιν είσθαι εμού και ημείς θέλομεν κληρονομήσει αυτά, αν και ο Κύριος εστάθη εκεί,
\par 11 διά τούτο, ζω εγώ, λέγει Κύριος ο Θεός, θέλω κάμει κατά τον θυμόν σου και κατά τον φθόνον σου, τον οποίον εξετέλεσας διά το προς αυτούς μίσός σου, και θέλω γνωσθή εις αυτούς όταν σε κρίνω.
\par 12 Και θέλεις γνωρίσει ότι εγώ ο Κύριος ήκουσα πάσας τας βλασφημίας σου, τας οποίας επρόφερες κατά των ορέων του Ισραήλ, λέγων, αυτά ηρημώθησαν, εις ημάς εδόθησαν διά τροφήν.
\par 13 Και με το στόμα υμών εμεγαλορρημονήσατε κατ' εμού και επληθύνατε τους λόγους υμών κατ' εμού· εγώ ήκουσα.
\par 14 Ούτω λέγει Κύριος ο Θεός· Όταν πάσα η γη ευφραίνηται, έρημον θέλω καταστήσει σε.
\par 15 Καθώς ευφράνθης επί την κληρονομίαν τον οίκον Ισραήλ διότι ηφανίσθη, ούτω θέλω κάμει εις σέ· θέλεις ερημωθή, όρος Σηείρ και πας ο Εδώμ, πας αυτός· και θέλουσι γνωρίσει ότι εγώ είμαι ο Κύριος.

\chapter{36}

\par 1 Και συ, υιέ ανθρώπου, προφήτευσον επί τα όρη Ισραήλ και ειπέ, Ορη του Ισραήλ, ακούσατε τον λόγον του Κυρίου·
\par 2 Ούτω λέγει Κύριος ο Θεός· Επειδή ο εχθρός είπεν εναντίον σας, Εύγε, οι αιώνιοι υψηλοί τόποι έγειναν κληρονομία ημών,
\par 3 διά τούτο προφήτευσον και ειπέ, Ούτω λέγει Κύριος ο Θεός· Επειδή ηρήμωσαν και κατέπιον εσάς κυκλόθεν, διά να γείνητε κληρονομία εις το υπόλοιπον των εθνών, και κατεστάθητε λάλημα της γλώσσης και όνειδος των λαών·
\par 4 διά τούτο, όρη του Ισραήλ, ακούσατε τον λόγον Κυρίου του Θεού· Ούτω λέγει Κύριος ο Θεός προς τα όρη και προς τα βουνά, προς τους χειμάρρους και προς τας φάραγγας και προς τους ηρημωμένους και ηφανισμένους τόπους και προς τας εγκαταλελειμμένας πόλεις, αίτινες έγειναν λάφυρον και εμπαιγμός εις το υπόλοιπον των πέριξ εθνών·
\par 5 διά τούτο ούτω λέγει Κύριος ο Θεός· Εξάπαντος εν τω πυρί του ζήλου μου ελάλησα κατά του υπολοίπου των εθνών και κατά παντός του Εδώμ, οίτινες έκαμον την γην μου κληρονομίαν εαυτών εν χαρά όλης της καρδίας αυτών και εν περιφρονήσει ψυχής, διά να εκθέσωσιν αυτήν εις λάφυρον.
\par 6 Διά τούτο προφήτευσον επί την γην Ισραήλ, και ειπέ προς τα όρη και προς τα βουνά, προς τους χειμάρρους και προς τας φάραγγας, Ούτω λέγει Κύριος ο Θεός· Ιδού, εγώ ελάλησα εν τω ζήλω μου και εν τω θυμώ μου, διότι εβαστάσατε την ύβριν των εθνών·
\par 7 διά τούτο ούτω λέγει Κύριος ο Θεός· Εγώ ύψωσα την χείρα μου· εξάπαντος τα έθνη τα πέριξ υμών, αυτά θέλουσι βαστάσει την αισχύνην αυτών.
\par 8 Σεις δε, όρη του Ισραήλ, θέλετε εκβλαστήσει τους κλάδους σας και θέλετε δώσει τον καρπόν σας εις τον λαόν μου Ισραήλ, διότι πλησιάζουσι να έλθωσι.
\par 9 Διότι ιδού, εγώ επιβλέπω εφ' υμάς και θέλω στραφή προς υμάς, και θέλετε αροτριασθή και σπαρθή.
\par 10 Και θέλω πληθύνει εφ' υμών ανθρώπους, άπαντα τον οίκον Ισραήλ, άπαντα αυτόν· και αι πόλεις θέλουσι κατοικηθή και αι ερημώσεις θέλουσιν οικοδομηθή.
\par 11 Και θέλω πληθύνει εφ' υμών ανθρώπους και κτήνη και θέλουσιν αυξηθή και καρποφορήσει· και θέλω σας κατοικίσει ως ήσθε πρότερον και αγαθοποιήσει μάλλον παρά τας αρχάς σας· και θέλετε γνωρίσει ότι εγώ είμαι ο Κύριος.
\par 12 Και θέλω κάμει να περιπατώσιν εφ' υμών άνθρωποι, ο λαός μου Ισραήλ· και θέλουσι σας κληρονομήσει, και θέλετε είσθαι κληρονομία αυτών, και του λοιπού δεν θέλετε πλέον ατεκνώσει αυτούς.
\par 13 Ούτω λέγει Κύριος ο Θεός· Επειδή είπον προς εσάς, Συ είσαι γη κατατρώγουσα ανθρώπους και ατεκνόνουσα τους λαούς σου,
\par 14 διά τούτο δεν θέλεις πλέον κατατρώγει ανθρώπους ουδέ ατεκνώσει πλέον τους λαούς σου, λέγει Κύριος ο Θεός.
\par 15 Και δεν θέλω πλέον κάμει να ακουσθή εν σοι η ύβρις των εθνών, και δεν θέλεις φέρει πλέον τον ονειδισμόν των λαών, και δεν θέλεις κάμει πλέον τους λαούς σου να ατεκνωθώσι, λέγει Κύριος ο Θεός.
\par 16 Και έγεινε λόγος Κυρίου προς εμέ, λέγων,
\par 17 Υιέ ανθρώπου, ότε ο οίκος Ισραήλ κατώκησαν εν τη γη αυτών, εμίαναν αυτήν διά της οδού αυτών και διά των πράξεων αυτών· η οδός αυτών ήτο έμπροσθέν μου ως ακαθαρσία αποκεχωρισμένης.
\par 18 Διά τούτο εξέχεα τον θυμόν μου επ' αυτούς, διά το αίμα, το οποίον έχυσαν επί την γην, και διά τα είδωλα αυτών, με τα οποία εμόλυναν αυτήν·
\par 19 και διέσπειρα αυτούς μεταξύ των εθνών και ήσαν διεσκορπισμένοι εν τοις τόποις· κατά την οδόν αυτών και κατά τα έργα αυτών έκρινα αυτούς.
\par 20 Και ότε εισήλθον εις τα έθνη, όπου ήλθον, εβεβήλωσαν το όνομά μου το άγιον, ενώ ελέγετο περί αυτών, Ούτοι είναι ο λαός του Κυρίου και εκ της γης αυτού εξήλθον.
\par 21 Εσπλαγχνίσθην όμως ένεκεν του αγίου ονόματός μου, το οποίον ο οίκος Ισραήλ εβεβήλωσε μεταξύ των εθνών εις τα οποία ήλθον.
\par 22 Διά τούτο ειπέ προς τον οίκον Ισραήλ, Ούτω λέγει Κύριος ο Θεός· Εγώ δεν κάμνω τούτο ένεκεν υμών, οίκος Ισραήλ, αλλ' ένεκεν του αγίου ονόματός μου, το οποίον εβεβηλώσατε μεταξύ των εθνών, εις τα οποία ήλθετε.
\par 23 Και θέλω αγιάσει το όνομά μου το μέγα, το βεβηλωθέν μεταξύ των εθνών, το οποίον εβεβηλώσατε εν μέσω αυτών· και θέλουσι γνωρίσει τα έθνη ότι εγώ είμαι ο Κύριος, λέγει Κύριος ο Θεός, όταν αγιασθώ εν υμίν έμπροσθεν των οφθαλμών αυτών.
\par 24 Διότι θέλω σας λάβει εκ μέσου των εθνών και θέλω σας συνάξει εκ πάντων των τόπων και σας φέρει εις την γην υμών.
\par 25 Και θέλω ράνει εφ' υμών καθαρόν ύδωρ και θέλετε καθαρισθή· από πασών των ακαθαρσιών σας και από πάντων των ειδώλων σας θέλω σας καθαρίσει.
\par 26 Και θέλω δώσει εις εσάς καρδίαν νέαν, και πνεύμα νέον θέλω εμβάλει εν υμίν, και αποσπάσας την λιθίνην καρδίαν από της σαρκός σας θέλω δώσει εις εσάς καρδίαν σαρκίνην.
\par 27 Και θέλω εμβάλει εν υμίν το Πνεύμα μου και σας κάμει να περιπατήτε εν τοις διατάγμασί μου και να φυλάττητε τας κρίσεις μου και να εκτελήτε αυτάς.
\par 28 Και θέλετε κατοικήσει εν τη γη, την οποίαν έδωκα εις τους πατέρας σας· και θέλετε είσθαι λαός μου και εγώ θέλω είσθαι Θεός σας.
\par 29 Και θέλω σας σώσει από πασών των ακαθαρσιών σας· και θέλω ανακαλέσει τον σίτον και πληθύνει αυτόν, και δεν θέλω πλέον επιφέρει εις εσάς πείναν.
\par 30 Και θέλω πληθύνει τον καρπόν των δένδρων και τα γεννήματα του αγρού, διά να μη λάβητε πλέον ονειδισμόν πείνης μεταξύ των εθνών.
\par 31 Και θέλετε ενθυμηθή τας οδούς υμών τας πονηράς και τα έργα υμών τα μη αγαθά, και θέλετε αποστραφή αυτοί εαυτούς έμπροσθεν των οφθαλμών σας διά τας ανομίας σας και διά τα βδελύγματά σας.
\par 32 Εγώ δεν κάμνω ταύτα ένεκεν υμών, λέγει Κύριος ο Θεός, ας ήναι γνωστόν εις εσάς· αισχύνθητε και εντράπητε διά τας οδούς σας, οίκος Ισραήλ.
\par 33 Ούτω λέγει Κύριος ο Θεός· Καθ' ην ημέραν σας καθαρίσω από πασών των ανομιών σας, θέλω κάμει έτι να κατοικηθώσιν αι πόλεις, και θέλουσιν οικοδομηθή αι ερημώσεις.
\par 34 Και η γη η ηφανισμένη θέλει γεωργηθή, αντί να κήται ηφανισμένη ενώπιον παντός διαβαίνοντος.
\par 35 Και θέλουσι λέγει, Η γη αύτη, ήτις ήτο ηφανισμένη, κατεστάθη ως ο παράδεισος της Εδέμ, και αι πόλεις αι ηρημωμέναι και ηφανισμέναι και κατηδαφισμέναι ωχυρώθησαν, κατωκίσθησαν.
\par 36 Και τα έθνη τα εναπολειφθέντα κύκλω υμών θέλουσι γνωρίσει ότι εγώ ο Κύριος ωκοδόμησα τα κατηδαφισμένα και εφύτευσα τα ηφανισμένα· εγώ ο Κύριος ελάλησα, και θέλω εκτελέσει.
\par 37 Ούτω λέγει Κύριος ο Θεός· Και τούτο θέλει ζητηθή παρ' εμού εκ του οίκου Ισραήλ να κάμω εις αυτούς, να πληθύνω αυτούς με ανθρώπους ως ποίμνιον προβάτων.
\par 38 Ως το άγιον ποίμνιον, ως το ποίμνιον της Ιερουσαλήμ εν ταις επισήμοις εορταίς αυτής, ούτως αι πόλεις αι ηρημωμέναι θέλουσι γείνει πλήρεις ποιμνίων ανθρώπων· και θέλουσι γνωρίσει ότι εγώ είμαι ο Κύριος.

\chapter{37}

\par 1 Χείρ Κυρίου εστάθη επ' εμέ· και με εξήγαγεν ο Κύριος διά πνεύματος και με έθεσεν εν μέσω πεδιάδος και αυτή ήτο πλήρης οστέων.
\par 2 Και με έκαμε να διέλθω πλησίον αυτών κύκλω· και ιδού, ήσαν πολλά σφόδρα επί το πρόσωπον της πεδιάδος· και ιδού, ήσαν κατάξηρα.
\par 3 Και είπε προς εμέ, Υιέ ανθρώπου, δύνανται τα οστά ταύτα να αναζήσωσι; Και είπα, Κύριε Θεέ, συ εξεύρεις.
\par 4 Και είπε προς εμέ, Προφήτευσον επί τα οστά ταύτα και ειπέ προς αυτά, Τα οστά τα ξηρά, ακούσατε τον λόγον του Κυρίου·
\par 5 Ούτω λέγει Κύριος ο Θεός προς τα οστά ταύτα· Ιδού, εγώ θέλω εμβάλει εις εσάς πνεύμα και θέλετε αναζήσει·
\par 6 και θέλω βάλει εφ' υμάς νεύρα και αναγάγει σάρκα εφ' υμάς και περισκεπάσει υμάς με δέρμα, και θέλω εμβάλει εις εσάς πνεύμα και θέλετε αναζήσει και θέλετε γνωρίσει ότι εγώ είμαι ο Κύριος.
\par 7 Και προεφήτευσα, ως προσετάχθην· και καθώς προεφήτευσα, έγεινεν ήχος, και ιδού, σεισμός, και τα οστά συνήλθον ομού, οστούν μετά του οστού αυτού.
\par 8 Και είδον και ιδού, νεύρα και σάρκες ανεφύησαν επ' αυτά και δέρμα περιεσκέπασεν αυτά επάνω· πνεύμα όμως δεν ήτο εν αυτοίς.
\par 9 Και είπε προς εμέ, προφήτευσον επί το πνεύμα, προφήτευσον, υιέ ανθρώπου, και ειπέ προς το πνεύμα, Ούτω λέγει Κύριος ο Θεός· Ελθέ, πνεύμα, εκ των τεσσάρων ανέμων και εμφύσησον επί τους πεφονευμένους τούτους και ας αναζήσωσι.
\par 10 Και προεφήτευσα, ως προσετάχθην· και το πνεύμα εισήλθεν εις αυτούς και ανέζησαν και εστάθησαν επί τους πόδας αυτών, στράτευμα μέγα σφόδρα.
\par 11 Και είπε προς εμέ, Υιέ ανθρώπου, τα οστά ταύτα είναι πας ο οίκος Ισραήλ· ιδού, ούτοι λέγουσι, τα οστά ημών εξηράνθησαν και η ελπίς ημών εχάθη· ημείς ηφανίσθημεν.
\par 12 Διά τούτο προφήτευσον και ειπέ προς αυτούς, Ούτω λέγει Κύριος ο Θεός· Ιδού, λαέ μου, εγώ ανοίγω τους τάφους σας και θέλω σας αναβιβάσει εκ των τάφων σας, θέλω σας φέρει εις την γην του Ισραήλ.
\par 13 Και θέλετε γνωρίσει ότι εγώ είμαι ο Κύριος, όταν, λαέ μου, ανοίξω τους τάφους σας και σας αναβιβάσω εκ των τάφων σας.
\par 14 Και θέλω δώσει το πνεύμά μου εις εσάς και θέλετε αναζήσει και θέλω σας θέσει εν τη γη υμών, και θέλετε γνωρίσει ότι εγώ ο Κύριος ελάλησα και εξετέλεσα, λέγει Κύριος.
\par 15 Και έγεινε λόγος Κυρίου προς εμέ, λέγων,
\par 16 Και συ, υιέ ανθρώπου, λάβε εις σεαυτόν ράβδον μίαν και γράψον επ' αυτήν περί του Ιούδα και περί των υιών Ισραήλ των συνακολούθων αυτού· λάβε και άλλην ράβδον και γράψον επ' αυτήν περί του Ιωσήφ, της ράβδον του Εφραΐμ, και παντός του οίκου Ισραήλ των συνακολούθων αυτού.
\par 17 Και σύναψον αυτάς εις σεαυτόν μίαν προς μίαν εις ράβδον μίαν και θέλουσι γείνει μία εν τη χειρί σου.
\par 18 Και όταν οι υιοί του λαού σου είπωσι προς σε, λέγοντες, Δεν θέλεις απαγγείλει εις ημάς τι δηλούσιν εις σε ταύτα;
\par 19 ειπέ προς αυτούς, Ούτω λέγει Κύριος ο Θεός· Ιδού, εγώ θέλω λάβει την ράβδον του Ιωσήφ, την εν τη χειρί του Εφραΐμ, και των φυλών του Ισραήλ των συνακολούθων αυτού, και θέλω βάλει εκείνας μετά ταύτης, της ράβδου του Ιούδα, και κάμει αυτάς μίαν ράβδον, και θέλουσιν είσθαι μία εν τη χειρί μου.
\par 20 Και αι ράβδοι, επί τας οποίας έγραψας, θέλουσιν είσθαι εν τη χειρί σου ενώπιον αυτών.
\par 21 Και ειπέ προς αυτούς, Ούτω λέγει Κύριος ο Θεός· Ιδού, εγώ θέλω λάβει τους υιούς Ισραήλ εκ μέσου των εθνών όπου υπήγον, και θέλω συνάξει αυτούς πανταχόθεν και φέρει αυτούς εις την γην αυτών.
\par 22 Και θέλω κάμει αυτούς εν έθνος εν τη γη, επί των ορέων του Ισραήλ· και εις βασιλεύς θέλει είσθαι βασιλεύς επί πάντας αυτούς· και δεν θέλουσιν είσθαι πλέον δύο έθνη και δεν θέλουσιν είσθαι του λοιπού διηρημένοι πλέον εις δύο βασίλεια·
\par 23 και δεν θέλουσι μιαίνεσθαι πλέον εν τοις ειδώλοις αυτών ουδέ εν τοις βδελύγμασιν αυτών ουδέ εν πάσαις ταις παραβάσεσιν αυτών· αλλά θέλω σώσει αυτούς εκ πασών των κατοικήσεων αυτών, εν αις ημάρτησαν, και θέλω καθαρίσει αυτούς· και θέλουσιν είσθαι λαός μου και εγώ θέλω είσθαι Θεός αυτών.
\par 24 Και Δαβίδ ο δούλός μου θέλει είσθαι βασιλεύς επ' αυτούς· και θέλει είσθαι επί πάντας αυτούς εις ποιμήν· και θέλουσι περιπατεί εν ταις κρίσεσί μου και θέλουσι φυλάττει τα διατάγματά μου και εκτελεί αυτά.
\par 25 Και θέλουσι κατοικεί εν τη γη, την οποίαν έδωκα εις τον δούλον μου τον Ιακώβ, όπου κατώκησαν οι πατέρες σας· και εν αυτή θέλουσι κατοικεί, αυτοί και τα τέκνα αυτών και τα τέκνα των τέκνων αυτών, έως αιώνος· και Δαβίδ ο δούλός μου θέλει είσθαι άρχων αυτών εις τον αιώνα.
\par 26 Και θέλω κάμει προς αυτούς διαθήκην ειρήνης· αύτη θέλει είσθαι διαθήκη αιώνιος προς αυτούς· και θέλω στηρίξει αυτούς και πληθύνει αυτούς, και θέλω θέσει το αγιαστήριόν μου εν μέσω αυτών εις τον αιώνα.
\par 27 Και η σκηνή μου θέλει είσθαι εν μέσω αυτών, και θέλω είσθαι Θεός αυτών και αυτοί θέλουσιν είσθαι λαός μου.
\par 28 Και θέλουσι γνωρίσει τα έθνη ότι εγώ ο Κύριος είμαι ο αγιάζων τον Ισραήλ, όταν το αγιαστήριόν μου ήναι εν μέσω αυτών εις τον αιώνα.

\chapter{38}

\par 1 Και έγεινε λόγος Κυρίου προς εμέ, λέγων,
\par 2 Υιέ ανθρώπου, στήριξον το πρόσωπόν σου επί Γωγ, την γην του Μαγώγ, του ηγεμόνος της Ρως, Μεσέχ και Θουβάλ, και προφήτευσον κατ' αυτού,
\par 3 και ειπέ, Ούτω λέγει Κύριος ο Θεός· Ιδού, εγώ είμαι εναντίον σου, Γωγ, ηγεμών της Ρως, Μεσέχ και Θουβάλ·
\par 4 και θέλω σε περιστρέψει και βάλει άγκιστρα εις τας σιαγόνας σου, και θέλω εκβάλει σε και πάσαν την δύναμίν σου, ίππους και ιππέας, πάντας τούτους εντελώς ώπλισμένους, μέγα άθροισμα μετά θυρεών και ασπίδων, πάντας τούτους μεταχειριζομένους μαχαίρας.
\par 5 Πέρσας, Αιθίοπας και Λίβυας μετ' αυτών· πάντας τούτους μετ' ασπίδων και περικεφαλαιών·
\par 6 τον Γομέρ και πάντα τα τάγματα αυτού, τον οίκον Θωγαρμά από των εσχάτων του βορρά και πάντα τα τάγματα αυτού και πολλούς λαούς μετά σου.
\par 7 Ετοιμάσθητι και ετοίμασον σεαυτόν, συ και παν το άθροισμά σου το συναθροισθέν εις σε, και έσο φύλαξ εις αυτούς·
\par 8 μετά πολλάς ημέρας θέλει γείνει επίσκεψις εις σέ· εν τοις εσχάτοις χρόνοις θέλεις ελθεί εις την γην, ήτις ηλευθερώθη εκ της μαχαίρας και συνήχθη εκ πολλών λαών εναντίον των ορέων του Ισραήλ, τα οποία κατεστάθησαν έρημα διαπαντός· αυτός όμως μετεφέρθη εκ μέσου των λαών, και θέλουσι κατοικήσει πάντες ασφαλώς.
\par 9 Και θέλεις αναβή και ελθεί ως ανεμοζάλη· θέλεις είσθαι ως νέφος, διά να σκεπάσης την γην, συ και πάντα τα τάγματά σου και πολύς λαός μετά σου.
\par 10 Ούτω λέγει Κύριος ο Θεός· Και εν εκείνη τη ημέρα θέλουσιν αναβή πράγματα επί την καρδίαν σου και θέλεις βουλευθή βουλάς πονηράς·
\par 11 και θέλεις ειπεί, Θέλω αναβή εις γην πόλεων ατειχίστων· θέλω ελθεί προς ησυχάζοντας, κατοικούντας εν ασφαλεία, πάντας τούτους κατοικούντας πόλεις ατειχίστους και μη εχούσας μοχλούς και πύλας·
\par 12 διά να λεηλατήσης λεηλασίαν και να λαφυραγωγήσης λάφυρον, διά να επαναστρέψης την χείρα σου επί ερημώσεις κατοικισθείσας και επί λαόν συνηγμένον εκ των εθνών αποκτήσαντα κτήνη και αγαθά, κατοικούντα εν μέσω της γης.
\par 13 Σεβά και Δαιδάν και οι έμποροι της Θαρσείς, μετά πάντων των σκύμνων αυτής, θέλουσιν ειπεί προς σε, Ηλθες να λεηλατήσης λεηλασίαν; συνήθροισας το πλήθός σου διά να λαφυραγωγήσης λάφυρον; διά να αρπάσης αργύριον και χρυσίον, διά να λάβης κτήνη και αγαθά, διά να κάμης λείαν μεγάλην;
\par 14 Διά τούτο, υιέ ανθρώπου, προφήτευσον και ειπέ προς τον Γωγ, Ούτω λέγει Κύριος ο Θεός· Εν εκείνη τη ημέρα, ότε ο λαός μου Ισραήλ θέλει κατοικεί εν ασφαλεία, συ δεν θέλεις μάθει τούτο;
\par 15 Και θέλεις ελθεί εκ του τόπου σου, εκ των εσχάτων του βορρά, συ και πολλοί λαοί μετά σου, άπαντες αναβάται ίππων, πλήθος μέγα και δύναμις πολλή·
\par 16 και θέλεις αναβή εναντίον του λαού μου Ισραήλ ως νέφος, διά να σκεπάσης την γήν· τούτο θέλει είσθαι εν ταις εσχάταις ημέραις· και θέλω σε φέρει εναντίον της γης μου, διά να με γνωρίσωσι τα έθνη, όταν αγιασθώ εν σοι, Γωγ, ενώπιον αυτών.
\par 17 Ούτω λέγει Κύριος ο Θεός· Συ είσαι εκείνος, περί του οποίου ελάλησα εν ταις αρχαίαις ημέραις, διά των δούλων μου των προφητών του Ισραήλ, οίτινες προεφήτευσαν εν εκείναις ταις ημέραις διά πολλών ετών, ότι έμελλον να σε φέρω εναντίον αυτών;
\par 18 Αλλ' εν εκείνη τη ημέρα, εν τη ημέρα καθ' ην ο Γωγ έλθη εναντίον της γης Ισραήλ, η οργή μου θέλει αναβή επί το πρόσωπόν μου, λέγει Κύριος ο Θεός.
\par 19 Διότι εν τω ζήλω μου, εν τω πυρί της οργής μου ελάλησα, Εξάπαντος εν τη ημέρα εκείνη θέλει είσθαι σεισμός μέγας εν γη Ισραήλ·
\par 20 και οι ιχθύες της θαλάσσης και τα πετεινά του ουρανού και τα θηρία του αγρού και πάντα τα ερπετά τα έρποντα επί της γης και πάντες οι άνθρωποι οι επί του προσώπου της γης θέλουσι σεισθή από της παρουσίας μου· και τα όρη θέλουσιν ανατραπή και οι πύργοι θέλουσι πέσει και παν τείχος θέλει κατεδαφισθή.
\par 21 Και θέλω καλέσει εναντίον αυτού μάχαιραν κατά πάντα τα όρη μου, λέγει Κύριος ο Θεός· η μάχαιρα εκάστου ανθρώπου θέλει είσθαι κατά του αδελφού αυτού.
\par 22 Και θέλω ελθεί εις κρίσιν εναντίον αυτού εν λοιμώ και εν αίματι· και θέλω βρέξει επ' αυτόν και επί τα τάγματα αυτού και επί τον πολύν λαόν τον μετ' αυτού βροχήν κατακλυσμού και λίθους χαλάζης, πυρ και θείον.
\par 23 Και θέλω μεγαλυνθή και αγιασθή, και θέλω γνωρισθή ενώπιον πολλών εθνών και θέλουσι γνωρίσει ότι εγώ είμαι ο Κύριος.

\chapter{39}

\par 1 Και συ, υιέ ανθρώπου, προφήτευσον κατά του Γωγ και ειπέ, Ούτω λέγει Κύριος ο Θεός· Ιδού, εγώ είμαι εναντίον σου, Γωγ, ηγεμών της Ρως, Μεσέχ και Θουβάλ·
\par 2 και θέλω σε περιστρέψει και σε περιπλανήσει, και θέλω σε αναβιβάσει εκ των εσχάτων του βορρά και φέρει επί τα όρη του Ισραήλ·
\par 3 και θέλω εκτινάξει το τόξον σου από της αριστεράς σου χειρός και κάμει τα βέλη σου να εκπέσωσιν από της δεξιάς σου χειρός.
\par 4 Θέλεις πέσει επί των ορέων του Ισραήλ, συ και πάντα τα τάγματά σου και οι λαοί οι μετά σού· θέλω σε δώσει εις τα πτερωτά όρνεα παντός είδους και εις τα θηρία του αγρού, εις κατάβρωμα·
\par 5 θέλεις πέσει επί του προσώπου του αγρού· διότι εγώ ελάλησα, λέγει Κύριος ο Θεός. Και θέλω αποστείλει πυρ επί τον Μαγώγ και μεταξύ των κατοικούντων εν ασφαλεία, τας νήσους· και θέλουσι γνωρίσει ότι εγώ είμαι ο Κύριος.
\par 6 Και θέλω κάμει το όνομά μου το άγιον γνωστόν εν μέσω του λαού μου Ισραήλ.
\par 7 Και δεν θέλω αφήσει να βεβηλώσωσι πλέον το όνομά μου το άγιον· και θέλουσι γνωρίσει τα έθνη, ότι εγώ είμαι ο Κύριος, ο Άγιος εν Ισραήλ·
\par 8 Ιδού, ήλθε και έγεινε, λέγει Κύριος ο Θεός· αύτη είναι η ημέρα, περί της οποίας ελάλησα.
\par 9 Και οι κατοικούντες τας πόλεις του Ισραήλ θέλουσιν εξέλθει και θέλουσι βάλει εις το πυρ και καύσει τα όπλα και τας ασπίδας και τους θυρεούς, τα τόξα και τα βέλη και τα ακόντια και τας λόγχας· και θέλουσι καίει με αυτά πυρ επτά έτη·
\par 10 και δεν θέλουσι λάβει ξύλα εκ του αγρού ουδέ θέλουσι κόψει εκ των δρυμών, διότι θέλουσι καίει πυρ εκ των όπλων· και θέλουσι λεηλατήσει τους λεηλατήσαντας αυτούς και λαφυραγωγήσει τους λαφυραγωγήσαντας αυτούς, λέγει Κύριος ο Θεός.
\par 11 Και εν εκείνη τη ημέρα θέλω δώσει εις τον Γωγ τόπον ταφής εκεί εν Ισραήλ, την φάραγγα των διαβατών, προς ανατολάς της θαλάσσης· και αυτή θέλει κλείει την οδόν των διαβαινόντων· και εκεί θέλουσι χώσει τον Γωγ και άπαν το πλήθος αυτού· και θέλουσιν ονομάσει αυτήν, Η φάραγξ του Αμών-γωγ.
\par 12 Και ο οίκος Ισραήλ θέλει χόνει αυτούς επτά μήνας, διά να καθαρίσωσι την γην.
\par 13 Και άπας ο λαός της γης θέλει χόνει αυτούς· και θέλει είσθαι εις αυτούς ονομαστή η ημέρα καθ' ην εδοξάσθην, λέγει Κύριος ο Θεός.
\par 14 Και θέλουσι διαχωρίσει άνδρας, οίτινες περιερχόμενοι ακαταπαύστως την γην θέλουσι θάπτει με την βοήθειαν των διαβατών τους μείναντας επί του προσώπου της γης, διά να καθαρίσωσιν αυτήν· μετά το τέλος των επτά μηνών θέλουσι κάμει ακριβή αναζήτησιν.
\par 15 Και εκ των διαβατών των διαβαινόντων την γην, όταν τις ίδη οστούν ανθρώπου, τότε θέλει στήνει σημείον πλησίον αυτού, εωσού οι ενταφιασταί θάψωσιν αυτό εν τη φάραγγι του Αμών-γωγ.
\par 16 Και της πόλεως δε το όνομα θέλει είσθαι Αμωνά. Ούτω θέλουσι καθαρίσει την γην.
\par 17 Και συ, υιέ ανθρώπου, ούτω λέγει Κύριος ο Θεός· Ειπέ προς τα όρνεα παντός είδους και προς πάντα τα θηρία του αγρού, συνάχθητε και έλθετε· συναθροίσθητε πανταχόθεν εις την θυσίαν μου, την οποίαν εγώ εθυσίασα διά σας, θυσίαν μεγάλην επί των ορέων του Ισραήλ, διά να φάγητε σάρκα και να πίητε αίμα.
\par 18 Θέλετε φάγει την σάρκα των ισχυρών και πίει το αίμα των αρχόντων της γης, των κριών, των αρνίων και των τράγων και των μόσχων, πάντων σιτευτών της Βασάν·
\par 19 και θέλετε φάγει πάχος εις χορτασμόν και πίει αίμα εις μέθην εκ της θυσίας μου την οποίαν εθυσίασα διά σάς·
\par 20 και θέλετε χορτασθή επί της τραπέζης μου, από ίππων και αναβατών, από ισχυρών και από παντός ανδρός πολεμιστού, λέγει Κύριος ο Θεός.
\par 21 Και θέλω θέσει την δόξαν μου μεταξύ των εθνών, και πάντα τα έθνη θέλουσιν ιδεί την κρίσιν μου την οποίαν εξετέλεσα και την χείρα μου, την οποίαν επέβαλον επ' αυτά.
\par 22 Και θέλει γνωρίσει ο οίκος Ισραήλ ότι εγώ είμαι Κύριος ο Θεός αυτών, από της ημέρας ταύτης και εις το εξής.
\par 23 Και τα έθνη θέλουσι γνωρίσει ότι ο οίκος Ισραήλ ηχμαλωτίσθη διά την ανομίαν αυτών· επειδή εστάθησαν παραβάται προς εμέ, διά τούτο έκρυψα το πρόσωπόν μου απ' αυτών και παρέδωκα αυτούς εις την χείρα των εχθρών αυτών· και έπεσον πάντες εν μαχαίρα.
\par 24 Κατά τας ακαθαρσίας αυτών και κατά τας παραβάσεις αυτών έπραξα εις αυτούς, και έκρυψα απ' αυτών το πρόσωπόν μου.
\par 25 Διά τούτο ούτω λέγει Κύριος ο Θεός· Τώρα θέλω επιστρέψει την αιχμαλωσίαν του Ιακώβ και ελεήσει άπαντα τον οίκον Ισραήλ, και θέλω είσθαι ζηλότυπος διά το όνομά μου το άγιον,
\par 26 και θέλουσι βαστάσει την αισχύνην αυτών και πάσας τας παραβάσεις αυτών, διά των οποίων έγειναν παραβάται προς εμέ, ότε κατώκουν ασφαλώς εν τη γη αυτών και δεν υπήρχεν ο εκφοβών.
\par 27 Όταν επαναφέρω αυτούς εκ των λαών και συνάξω αυτούς εκ των τόπων των εχθρών αυτών και αγιασθώ εν αυτοίς ενώπιον εθνών πολλών,
\par 28 τότε θέλουσι γνωρίσει ότι εγώ είμαι Κύριος ο Θεός αυτών, όταν, αφού κάμω αυτούς να φερθώσιν εις αιχμαλωσίαν μεταξύ των εθνών, συνάξω αυτούς εις την γην αυτών και δεν αφήσω εξ αυτών πλέον εκεί υπόλοιπον·
\par 29 και δεν θέλω κρύψει πλέον το πρόσωπόν μου απ' αυτών, διότι εξέχεα το πνεύμά μου επί τον οίκον Ισραήλ, λέγει Κύριος ο Θεός.

\chapter{40}

\par 1 Εν τω εικοστώ πέμπτω έτει της αιχμαλωσίας ημών, εν τη αρχή του έτους, τη δεκάτη του μηνός, τω δεκάτω τετάρτω έτει μετά την άλωσιν της πόλεως, εν τη αυτή ημέρα χειρ Κυρίου εστάθη επ' εμέ και με έφερεν εκεί.
\par 2 Δι' οραμάτων του Θεού με έφερεν εις γην Ισραήλ και με έθεσεν επί όρους υψηλοτάτου, εφ' ου ήτο προς μεσημβρίαν ως οικοδομή πόλεως.
\par 3 Και με έφερεν εκεί και ιδού, άνθρωπος του οποίου η θέα ήτο ως θέα χαλκού, και είχεν εν τη χειρί αυτού νήμα λινούν και μέτρον καλάμινον, και αυτός ίστατο εν τη πύλη.
\par 4 Και ο άνθρωπος είπε προς εμέ, Υιέ ανθρώπου, ιδέ με τους οφθαλμούς σου και άκουσον με τα ώτα σου και θέσον την καρδίαν σου επί πάντα όσα εγώ σοι δείξω· διότι, διά να σοι δείξω ταύτα, εισήχθης ενταύθα· απάγγειλον πάντα όσα βλέπεις προς τον οίκον Ισραήλ.
\par 5 Και ιδού, περίβολος έξωθεν του οίκου κύκλω, και εν τη χειρί του ανθρώπου μέτρον καλάμινον εξ πηχών και μιας παλάμης· και εμέτρησε το πλάτος του οικοδομήματος, ένα κάλαμον, και το ύψος, ένα κάλαμον.
\par 6 Τότε ήλθε προς την πύλην την βλέπουσαν κατά ανατολάς, και ανέβη τας βαθμίδας αυτής και εμέτρησε το κατώφλιον της πύλης, πλάτος ενός καλάμου, και το ανώφλιον, πλάτος ενός καλάμου.
\par 7 Και έκαστον οίκημα ήτο μακρόν ένα κάλαμον και πλατύ ένα κάλαμον· και μεταξύ των οικημάτων ήσαν πέντε πήχαι· και το κατώφλιον της πύλης πλησίον της στοάς της προς την πύλην την έσωθεν ήτο ενός καλάμου.
\par 8 Τότε εμέτρησε την στοάν της πύλης της έσωθεν και ήτο ενός καλάμου.
\par 9 Έπειτα εμέτρησε την στοάν της πύλης, οκτώ πήχας, και τα μέτωπα αυτής, δύο πήχας· η στοά δε της πύλης ήτο έσωθεν.
\par 10 Και τα οικήματα της πύλης προς ανατολάς ήσαν τρία εντεύθεν και τρία εκείθεν· και τα τρία ενός μέτρου, και τα μέτωπα είχον εν μέτρον, εντεύθεν και εκείθεν.
\par 11 Και εμέτρησε το πλάτος της εισόδου της πύλης, δέκα πήχας, και το μήκος της πύλης, δεκατρείς πήχας.
\par 12 Ήτο δε έμπροσθεν των οικημάτων διάστημα μία πήχη εντεύθεν και διάστημα μία πήχη εκείθεν· και τα οικήματα ήσαν εξ πηχών εντεύθεν και εξ πηχών εκείθεν.
\par 13 Έπειτα εμέτρησε την πύλην από της στέγης του ενός οικήματος μέχρι της στέγης του άλλου· το πλάτος ήτο εικοσιπέντε πηχών και θύρα απέναντι θύρας.
\par 14 Και έκαμε τα μέτωπα εξήκοντα πηχών μέχρι του μετώπου της αυλής κύκλω κύκλω του πυλώνος.
\par 15 Και από του προσώπου της πύλης της εισόδου έως του προσώπου της στοάς της εσωτέρας πύλης ήσαν πεντήκοντα πήχαι.
\par 16 Και ήσαν παράθυρα αδιόρατα εις τα οικήματα και εις τα μέτωπα αυτών έσωθεν της πύλης κύκλω κύκλω, και ωσαύτως εις τας στοάς· ήσαν παράθυρα και έσωθεν κύκλω κύκλω, εφ' εκάστου δε μετώπου φοίνικες.
\par 17 Και με έφερεν εις την εξωτέραν αυλήν και ιδού, θάλαμοι και λιθόστρωτον κατεσκευασμένον εν τη αυλή κύκλω κύκλω· τριάκοντα θάλαμοι επί του λιθοστρώτου.
\par 18 Και το λιθόστρωτον το επί τα πλάγια των πυλών, κατά το μήκος των πυλών, ήτο το κατώτερον λιθόστρωτον.
\par 19 Και εμέτρησε το πλάτος από του προσώπου της κατωτέρας πύλης μέχρι του προσώπου της εσωτέρας αυλής έξωθεν, εκατόν πήχας προς ανατολάς και προς βορράν.
\par 20 Και την πύλην της εξωτέρας αυλής την βλέπουσαν προς βορράν εμέτρησε, το μήκος αυτής και το πλάτος αυτής.
\par 21 Και τα οικήματα αυτής ήσαν τρία εντεύθεν και τρία εκείθεν, και τα μέτωπα αυτής και τα τόξα αυτής ήσαν κατά το μέτρον της πρώτης πύλης, το μήκος αυτής πεντήκοντα πηχών και το πλάτος αυτής εικοσιπέντε πηχών.
\par 22 Και τα παράθυρα αυτών και τα τόξα αυτών και οι φοίνικες αυτών ήσαν κατά το μέτρον της πύλης της βλεπούσης προς ανατολάς· και ανέβαινον προς αυτήν δι' επτά βαθμίδων· και τα τόξα αυτής ήσαν έμπροσθεν αυτών.
\par 23 Και η πύλη της εσωτέρας αυλής ήτο απέναντι της πύλης της προς βορράν και προς ανατολάς· και εμέτρησεν από πύλης εις πύλην, εκατόν πήχας.
\par 24 Και με έφερε προς νότον, και ιδού, πύλη βλέπουσα προς νότον· και εμέτρησε τα μέτωπα αυτής και τα τόξα αυτής, κατά τα αυτά μέτρα.
\par 25 Και ήσαν παράθυρα εις αυτήν και εις τα τόξα αυτής κύκλω κύκλω, ως τα παράθυρα εκείνα· το μήκος πεντήκοντα πηχών και το πλάτος εικοσιπέντε πηχών.
\par 26 Και η ανάβασις αυτής ήτο επτά βαθμίδες και τα τόξα αυτής ήσαν έμπροσθεν αυτών· και είχε φοίνικας, ένα εντεύθεν και ένα εκείθεν, επί των μετώπων αυτής.
\par 27 Και ήτο πύλη εις την εσωτέραν αυλήν προς νότον· και εμέτρησεν από πύλης εις πύλην, προς νότον, εκατόν πήχας.
\par 28 Και με έφερεν εις την εσωτέραν αυλήν διά της νοτίου πύλης· και εμέτρησε την νότιον πύλην κατά τα αυτά μέτρα.
\par 29 Και τα οικήματα αυτής και τα μέτωπα αυτής και τα τόξα αυτής κατά τα αυτά μέτρα· και ήσαν παράθυρα εις αυτήν και εις τα τόξα αυτής κύκλω κύκλω· πεντήκοντα πηχών το μήκος και εικοσιπέντε πηχών το πλάτος.
\par 30 Και τα τόξα κύκλω κύκλω ήσαν εικοσιπέντε πηχών το μήκος και πέντε πηχών το πλάτος.
\par 31 Και τα τόξα αυτής ήσαν προς την εξωτέραν αυλήν, και φοίνικες επί των μετώπων αυτής, και οκτώ βαθμίδες η ανάβασις αυτής.
\par 32 Και με έφερεν εις την εσωτέραν πύλην προς ανατολάς, και εμέτρησε την πύλην κατά τα αυτά μέτρα.
\par 33 Και τα οικήματα αυτής και τα μέτωπα αυτής και τα τόξα αυτής ήσαν κατά τα αυτά μέτρα· και ήσαν παράθυρα εις αυτήν και εις τα τόξα αυτής κύκλω κύκλω, πεντήκοντα πηχών το μήκος και εικοσιπέντε πηχών το πλάτος.
\par 34 Και τα τόξα αυτής ήσαν προς την εξωτέραν αυλήν, και φοίνικες επί των μετώπων αυτής, εντεύθεν και εκείθεν, και οκτώ βαθμίδες η ανάβασις αυτής.
\par 35 Και με έφερεν εις την βόρειον πύλην και εμέτρησεν αυτήν κατά τα αυτά μέτρα·
\par 36 τα οικήματα αυτής, τα μέτωπα αυτής και τα τόξα αυτής και τα παράθυρα αυτής κύκλω κύκλω· το μήκος πεντήκοντα πηχών και το πλάτος εικοσιπέντε πηχών.
\par 37 Και τα μέτωπα αυτής ήσαν προς την εξωτέραν αυλήν, και φοίνικες επί των μετώπων αυτής, εντεύθεν και εκείθεν, και οκτώ βαθμίδες η ανάβασις αυτής.
\par 38 Και οι θάλαμοι και αι είσοδοι αυτής ήσαν πλησίον των μετώπων των πυλών, όπου έπλυνον το ολοκαύτωμα.
\par 39 Και εν τη στοά της πύλης ήσαν δύο τράπεζαι εντεύθεν και δύο τράπεζαι εκείθεν, διά να σφάζωσιν επ' αυτών το ολοκαύτωμα και την περί αμαρτίας προσφοράν και την περί ανομίας προσφοράν.
\par 40 Και εις το έξω πλάγιον, καθώς ανέβαινέ τις προς την είσοδον της βορείου πύλης, ήσαν δύο τράπεζαι, και εις το άλλο πλάγιον, το προς την στοάν της πύλης, δύο τράπεζαι.
\par 41 Τέσσαρες τράπεζαι ήσαν εντεύθεν και τέσσαρες τράπεζαι εκείθεν παρά τα πλάγια της πύλης· οκτώ τράπεζαι, εφ' ων έσφαζον τα θύματα.
\par 42 Και αι τέσσαρες τράπεζαι του ολοκαυτώματος ήσαν εκ λίθου πελεκητού, μιας πήχης και ημίσεος το μήκος και μιας πήχης και ημίσεος το πλάτος και μιας πήχης το ύψος· και επ' αυτών έθετον τα εργαλεία, δι' ων έσφαζον το ολοκαύτωμα και την θυσίαν.
\par 43 Και έσωθεν ήσαν άγκιστρα, μιας παλάμης το πλάτος, προσηλωμένα κύκλω κύκλω· επί δε των τραπεζών έθετον το κρέας των προσφορών.
\par 44 Και έξωθεν της πύλης της εσωτέρας ήσαν οι θάλαμοι των μουσικών, εν τη εσωτέρα αυλή τη επί τα πλάγια της βορείου πύλης· και τα πρόσωπα αυτών ήσαν προς νότον, εν επί το πλάγιον της ανατολικής πύλης βλέπον προς βορράν.
\par 45 Και είπε προς εμέ, Ο θάλαμος ούτος ο βλέπων προς νότον είναι διά τους ιερείς τους φυλάττοντας την φυλακήν του οίκου,
\par 46 ο δε θάλαμος ο βλέπων προς βορράν είναι διά τους ιερείς τους φυλάττοντας την φυλακήν του θυσιαστηρίου· ούτοι είναι οι υιοί Σαδώκ, μεταξύ των υιών Λευΐ, οίτινες προσέρχονται εις τον Κύριον, διά να λειτουργώσιν εις αυτόν.
\par 47 Και εμέτρησε την αυλήν, μήκος εκατόν πηχών και πλάτος εκατόν πηχών, εις τετράγωνον· και το θυσιαστήριον ήτο έμπροσθεν του οίκου.
\par 48 Και με έφερεν εις την στοάν του οίκου και εμέτρησεν έκαστον μέτωπον της στοάς, πέντε πηχών εντεύθεν και πέντε πηχών εκείθεν· και το πλάτος της πύλης τριών πηχών εντεύθεν και τριών πηχών εκείθεν.
\par 49 το μήκος της στοάς ήτο είκοσι πηχών και το πλάτος ένδεκα πηχών· και με έφερε διά των βαθμίδων, δι' ων ανέβαινον εις αυτήν· και ήσαν στύλοι παρά τα μέτωπα, εις εντεύθεν και εις εκείθεν.

\chapter{41}

\par 1 Έπειτα με έφερεν εις τον ναόν και εμέτρησε τα μέτωπα, εξ πήχας το πλάτος εντεύθεν και εξ πήχας το πλάτος εκείθεν, το πλάτος της σκηνής.
\par 2 Και το πλάτος της εισόδου ήτο δέκα πηχών· και τα πλευρά της θύρας πέντε πηχών εντεύθεν και πέντε πηχών εκείθεν· και εμέτρησε το μήκος αυτού, τεσσαράκοντα πήχας, και το πλάτος είκοσι πήχας.
\par 3 Και εισήλθεν εις το εσώτερον και εμέτρησε το μέτωπον της θύρας, δύο πήχας, και την θύραν, εξ πήχας, και το πλάτος της θύρας, επτά πήχας.
\par 4 Έπειτα εμέτρησε το μήκος τούτου, είκοσι πήχας, και το πλάτος είκοσι πήχας, έμπροσθεν του ναού· και είπε προς εμέ, τούτο είναι το άγιον των αγίων.
\par 5 Και εμέτρησε τον τοίχον του οίκου, εξ πήχας· και το πλάτος εκάστου των εις τα πλάγια οικημάτων, τέσσαρας πήχας, κύκλω κύκλω του οίκου κύκλω.
\par 6 Και τα πλάγια οικήματα ήσαν ανά τρία, οίκημα επί οικήματος, και τριάκοντα κατά τάξιν· και εισεχώρουν εις τον τοίχον του ναού, εκτισμένον κύκλω κύκλω διά τα πλάγια οικήματα, διά να κρατώνται στερεά, χωρίς να επιστηρίζωνται όμως επί τον τοίχον του οίκου.
\par 7 Και ο οίκος επλατύνετο, και ήτο κλίμαξ ελικοειδής αναβαίνουσα εις τα πλάγια οικήματα· διότι η ελικοειδής κλίμαξ του οίκου ανέβαινε προς τα άνω κύκλω κύκλω του οίκου· όθεν ο οίκος εγίνετο πλατύτερος προς τα άνω, και ούτως ηύξανεν από του κατωτάτου πατώματος έως του ανωτάτου διά των μέσων.
\par 8 Και είδον το ύψος του οίκου κύκλω κύκλω· τα θεμέλια των πλαγίων οικημάτων ήσαν εις ολόκληρος κάλαμος εξ πηχών διάστημα.
\par 9 Το πλάτος του τοίχου διά τα έξωθεν πλάγια οικήματα ήτο πέντε πηχών· και το εναπολειφθέν κενόν ήτο ο τόπος των έσωθεν πλαγίων οικημάτων.
\par 10 Και μεταξύ των θαλάμων ήτο διάστημα είκοσι πηχών κύκλω κύκλω, περί τον οίκον.
\par 11 Και αι θύραι των πλαγίων οικημάτων ήσαν προς το μέρος το εναπολειφθέν μία θύρα προς βορράν και μία θύρα προς νότον· και το πλάτος του εναπολειφθέντος μέρους ήτο πέντε πηχών κύκλω κύκλω.
\par 12 Η δε οικοδομή η κατά πρόσωπον του κεχωρισμένου μέρους, προς το δυτικόν πλάγιον, ήτο εβδομήκοντα πηχών το πλάτος· και ο τοίχος της οικοδομής, πέντε πηχών το πάχος κύκλω κύκλω· το δε μήκος αυτής ενενήκοντα πηχών.
\par 13 Και εμέτρησε τον τοίχον, εξ εκατόν πηχών το μήκος· και το κεχωρισμένον μέρος και την οικοδομήν και τους τοίχους αυτής, εκατόν πηχών το μήκος·
\par 14 και το πλάτος του προσώπου του οίκου και του κεχωρισμένου μέρους προς ανατολάς, εκατόν πηχών.
\par 15 Και εμέτρησε το μήκος της οικοδομής της κατά πρόσωπον του κεχωρισμένου μέρους όπισθεν αυτού, και τας στοάς αυτού εντεύθεν και εκείθεν, εκατόν πηχών, και τον ενδότερον ναόν και τα πρόθυρα της αυλής·
\par 16 τους παραστάτας της θύρας και τα αόρατα παράθυρα και τας στοάς κύκλω κατά τα τρία αυτών πατώματα, κατά πρόσωπον της θύρας, εστρωμένα με ξύλον κύκλω κύκλω· και το έδαφος έως των παραθύρων και τα παράθυρα ήσαν εσκεπασμένα·
\par 17 έως επάνωθεν της θύρας και έως του εσωτέρου οίκου και έξωθεν και δι' όλου του τοίχου κύκλω έσωθεν και έξωθεν, κατά τα μέτρα.
\par 18 Και ήτο ειργασμένον με χερουβείμ και με φοίνικας, ώστε φοίνιξ ήτο μεταξύ χερούβ και χερούβ, και έκαστον χερούβ είχε δύο πρόσωπα·
\par 19 και πρόσωπον ανθρώπου προς τον φοίνικα εντεύθεν και πρόσωπον λέοντος προς τον φοίνικα εκείθεν· ούτως ήτο ειργασμένον δι' όλου του οίκου κύκλω κύκλω.
\par 20 Από του εδάφους έως επάνωθεν της θύρας ήσαν ειργασμένα χερουβείμ και φοίνικες και εις τον τοίχον του ναού.
\par 21 Οι παραστάται του ναού ήσαν τετράγωνοι και το πρόσωπον του αγιαστηρίου, η θέα του ενός ως η θέα του άλλου.
\par 22 Το ξύλινον θυσιαστήριον ήτο τριών πηχών το ύψος, το δε μήκος αυτού δύο πηχών· και τα κέρατα αυτού και το μήκος αυτού και οι τοίχοι αυτού ήσαν εκ ξύλου· και είπε προς εμέ, Αύτη είναι η τράπεζα η ενώπιον του Κυρίου.
\par 23 Και ο ναός και το αγιαστήριον είχον δύο θυρώματα.
\par 24 Και τα θυρώματα είχον δύο φύλλα έκαστον, δύο στρεφόμενα φύλλα· δύο εις το εν θύρωμα και δύο φύλλα εις το άλλο.
\par 25 Και ήσαν ειργασμένα επ' αυτών, επί των θυρωμάτων του ναού, χερουβείμ και φοίνικες, καθώς ήσαν ειργαμένα επί των τοίχων· και ήσαν δοκοί ξύλιναι επί το πρόσωπον της στοάς έξωθεν.
\par 26 Και ήσαν παράθυρα αδιόρατα και φοίνικες εντεύθεν και εκείθεν εις τα πλάγια της στοάς και επί τα πλάγια οικήματα του οίκου και δοκοί ξύλιναι.

\chapter{42}

\par 1 Και με εξήγαγεν εις την αυλήν την εξωτέραν κατά την οδόν την προς βορράν· και με έφερεν εις τον θάλαμον τον απέναντι του κεχωρισμένου μέρους κατά τον κατά πρόσωπον της οικοδομής, προς βορράν.
\par 2 κατά πρόσωπον του μήκους, το οποίον ήτο εκατόν πηχών, ήτο η βόρειος θύρα, το δε πλάτος πεντήκοντα πηχών.
\par 3 Απέναντι των είκοσι πηχών, αίτινες ήσαν διά την εσωτέραν αυλήν, και απέναντι του λιθοστρώτου του διά την εξωτέραν αυλήν, ήτο στοά αντικρύ στοάς τριπλή.
\par 4 Και κατά πρόσωπον των θαλάμων ήτο περίπατος δέκα πηχών το πλάτος, και προς τα έσω οδός μιας πήχης· και αι θύραι αυτών ήσαν προς βορράν.
\par 5 Οι δε ανώτατοι θάλαμοι ήσαν στενώτεροι, επειδή αι κάτω στοαί και αι μεσαίαι της οικοδομής εξείχον μάλλον παρά εκείνους.
\par 6 Διότι ούτοι ήσαν εις τρία πατώματα, δεν είχον όμως στύλους ως τους στύλους των αυλών· διά τούτο η οικοδομή εστενούτο μάλλον παρά το κατώτατον και το μεσαίον από της γης.
\par 7 Και ο τοίχος ο έξωθεν απέναντι των θαλάμων, προς την εξωτέραν αυλήν κατά πρόσωπον των θαλάμων, είχε μήκος πεντήκοντα πηχών.
\par 8 Διότι το μήκος των θαλάμων των εν τη εξωτέρα αυλή ήτο πεντήκοντα πηχών· και ιδού, κατά πρόσωπον του ναού ήσαν εκατόν πήχαι.
\par 9 Κάτωθεν δε των θαλάμων τούτων ήτο η είσοδος κατά ανατολάς, καθώς υπάγει τις προς αυτούς από της αυλής της εξωτέρας.
\par 10 Οι θάλαμοι ήσαν εις το πάχος του τοίχου της αυλής προς ανατολάς, κατά πρόσωπον του κεχωρισμένου μέρους και κατά πρόσωπον της οικοδομής.
\par 11 Και η οδός η κατά πρόσωπον αυτών ήτο κατά την θέαν των θαλάμων των προς βορράν· είχον ίσον μήκος με εκείνους, ίσον πλάτος με εκείνους· και πάσαι αι έξοδοι αυτών ήσαν και κατά τας διατάξεις εκείνων και κατά τας θύρας εκείνων.
\par 12 Και κατά τας θύρας των βαλάμων των προς νότον ήτο θύρα εις την αρχήν της οδού, της οδού κατ' ευθείαν απέναντι του τοίχου προς ανατολάς, καθώς εμβαίνει τις προς αυτά.
\par 13 Και είπε προς εμέ, Οι βόρειοι θάλαμοι και οι νότιοι θάλαμοι οι κατά πρόσωπον του κεχωρισμένου μέρους, ούτοι είναι θάλαμοι άγιοι, όπου οι ιερείς οι πλησιάζοντες εις τον Κύριον θέλουσι τρώγει τα αγιώτατα· εκεί θέλουσι θέτει τα αγιώτατα και την προσφοράν την εξ αλφίτων και την περί αμαρτίας προσφοράν και την περί ανομίας προσφοράν, διότι ο τόπος είναι άγιος.
\par 14 Όταν οι ιερείς εισέρχωνται εκεί, δεν θέλουσιν εξέρχεσθαι από του αγίου τόπου εις την αυλήν την εξωτέραν, αλλ' εκεί θέλουσιν αποθέτει τα ενδύματα αυτών, με τα οποία λειτουργούσι, διότι είναι άγια· και θέλουσιν ενδύεσθαι άλλα ενδύματα, και τότε θέλουσι πλησιάζει εις ό,τι είναι του λαού.
\par 15 Αφού δε ετελείωσε τα μέτρα του έσωθεν οίκου, με εξήγαγε προς την πύλην την βλέπουσαν κατά ανατολάς και εμέτρησεν αυτόν κύκλω κύκλω.
\par 16 Εμέτρησε την ανατολικήν πλευράν με το καλάμινον μέτρον, πεντακοσίους καλάμους, με το καλάμινον μέτρον κύκλω.
\par 17 Εμέτρησε την βόρειον πλευράν, πεντακοσίους καλάμους, με το καλάμινον μέτρον κύκλω.
\par 18 Εμέτρησε την νότιον πλευράν, πεντακοσίους καλάμους, με το καλάμινον μέτρον.
\par 19 Εστράφη έπειτα προς την δυτικήν πλευράν και εμέτρησε πεντακοσίους καλάμους με το καλάμινον μέτρον.
\par 20 Εμέτρησεν αυτόν κατά τας τέσσαρας πλευράς· είχε τοίχον κύκλω κύκλω, πεντακοσίων καλάμων το μήκος και πεντακοσίων το πλάτος, διά να κάμνη χώρισμα μεταξύ του αγίου και του βεβήλου τόπου.

\chapter{43}

\par 1 Και με έφερεν εις την πύλην, την πύλην την βλέπουσαν κατά ανατολάς.
\par 2 Και ιδού, η δόξα του Θεού του Ισραήλ ήρχετο από της οδού της ανατολής· και η φωνή αυτού ως φωνή υδάτων πολλών· και η γη έλαμπεν από της δόξης αυτού.
\par 3 Και η θέα την οποίαν είδον ήτο κατά την θέαν, κατά την θέαν την οποίαν είδον, ότε ήλθον να χαλάσω την πόλιν· και αι θέαι ήσαν κατά την θέαν, την οποίαν είδον παρά τον ποταμόν Χεβάρ· και έπεσον επί πρόσωπόν μου.
\par 4 Και η δόξα του Κυρίου εισήλθεν εις τον οίκον διά της οδού της πύλης της βλεπούσης κατά ανατολάς.
\par 5 Και με εσήκωσε το πνεύμα και με έφερεν εις την αυλήν την εσωτέραν· και ιδού, ο οίκος ήτο πλήρης της δόξης του Κυρίου.
\par 6 Και ήκουσα φωνήν λαλούντος προς εμέ εκ του οίκου· και ο άνθρωπος ίστατο πλησίον μου.
\par 7 Και είπε προς εμέ, Υιέ ανθρώπου, τον τόπον του θρόνου μου και τον τόπον του ίχνους των ποδών μου, όπου θέλω κατοικεί εν μέσω των υιών Ισραήλ εις τον αιώνα, και το όνομά μου το άγιον, δεν θέλει πλέον βεβηλώσει ο οίκος Ισραήλ, ούτε αυτοί ούτε οι βασιλείς αυτών, με τας πορνείας αυτών ουδέ με τα πτώματα των βασιλέων αυτών ουδέ με τους υψηλούς αυτών τόπους.
\par 8 Θέτοντες τα κατώφλια αυτών πλησίον των κατωφλίων μου και τους παραστάτας αυτών πλησίον των παραστατών μου, ώστε δεν ήτο παρά ο τοίχος μεταξύ εμού και αυτών, εβεβήλουν ούτω το όνομά μου το άγιον με τα βδελύγματα αυτών, τα οποία έπραττον· διά τούτο ηνάλωσα αυτούς εν τω θυμώ μου.
\par 9 Τώρα ας απομακρύνωσιν απ' εμού τας πορνείας αυτών και τα πτώματα των βασιλέων αυτών, και θέλω κατοικεί εν μέσω αυτών εις τον αιώνα.
\par 10 Συ, υιέ ανθρώπου, δείξον τον οίκον τούτον εις τον οίκον Ισραήλ, διά να εντραπώσι διά τας ανομίας αυτών· και ας μετρήσωσι το σχέδιον.
\par 11 Και εάν εντραπώσι διά πάντα όσα έπραξαν, δείξον εις αυτούς την μορφήν του οίκου και την διάταξιν αυτού και τας εξόδους αυτού και τας εισόδους αυτού και πάσαν την μορφήν αυτού και πάσας τας διατάξεις αυτού και πάσαν την μορφήν αυτού και πάντα τον νόμον αυτού, και διάγραψον αυτόν ενώπιον αυτών, διά να φυλάξωσιν όλην την μορφήν αυτού και πάσας τας διατάξεις αυτού και να εκτελώσιν αυτάς.
\par 12 Ούτος είναι ο νόμος του οίκου· επί της κορυφής του όρους, όλον το όριον αυτού κύκλω κύκλω θέλει είσθαι αγιώτατον. Ιδού, ούτος είναι ο νόμος του οίκου.
\par 13 Και ταύτα είναι τα μέτρα του θυσιαστηρίου εις πήχας· η πήχη είναι μία πήχη κοινή και παλάμη· το μεν κοίλωμα αυτού θέλει είσθαι μία πήχη και το πλάτος μία πήχη, το δε γείσωμα αυτού εις τα χείλη αυτού κύκλω μία σπιθαμή· και τούτο θέλει είσθαι το ανώτερον μέρος του θυσιαστηρίου.
\par 14 Από δε του κοιλώματος του προς την γην έως της κατωτέρας οφρύος θέλει είσθαι δύο πήχαι και το πλάτος μία πήχη· και από της οφρύος της μικροτέρας έως της οφρύος της μεγαλητέρας, τέσσαρες πήχαι, και το πλάτος μία πήχη.
\par 15 Και το θυσιαστήριον θέλει είσθαι τεσσάρων πηχών το ύψος· από δε του θυσιαστηρίου και επάνω θέλουσιν είσθαι τέσσαρα κέρατα.
\par 16 Και το θυσιαστήριον θέλει είσθαι δώδεκα πηχών το μήκος και δώδεκα το πλάτος, τετράγωνον εις τας τέσσαρας πλευράς αυτού.
\par 17 Και η οφρύς θέλει είσθαι δεκατεσσάρων πηχών το μήκος και δεκατεσσάρων το πλάτος εις τας τέσσαρας πλευράς αυτής· και το γείσωμα κύκλω αυτής μισή πήχη· και το κοίλωμα αυτής κύκλω μία πήχη· και αι βαθμίδες αυτής θέλουσι βλέπει προς ανατολάς.
\par 18 Και είπε προς εμέ, Υιέ ανθρώπου, ούτω λέγει Κύριος ο Θεός· Αύται είναι αι διατάξεις του θυσιαστηρίου καθ' ην ημέραν κατασκευάσωσιν αυτό, διά να προσφέρωσιν επ' αυτού ολοκαύτωμα και να ραντίζωσιν επ' αυτό αίμα.
\par 19 Και θέλεις δώσει εις τους ιερείς τους Λευΐτας, τους όντας εκ του σπέρματος Σαδώκ, τους πλησιάζοντάς με διά να λειτουργώσιν εις εμέ, λέγει Κύριος ο Θεός, μόσχον βοός διά προσφοράν περί αμαρτίας.
\par 20 Και θέλεις λάβει από του αίματος αυτού και βάλει επί τα τέσσαρα κέρατα αυτού και επί τας τέσσαρας γωνίας της οφρύος και επί το γείσωμα κύκλω· και θέλεις καθαρίσει αυτό και κάμει εξιλέωσιν περί αυτού.
\par 21 Και θέλεις λάβει τον μόσχον τον διά προσφοράν περί αμαρτίας, και θέλουσι καύσει αυτόν εν τω διωρισμένω τόπω του οίκου έξω του αγιαστηρίου.
\par 22 Και την δευτέραν ημέραν θέλεις προσφέρει τράγον εξ αιγών άμωμον διά προσφοράν περί αμαρτίας· και θέλουσι καθαρίσει το θυσιαστήριον, ως εκαθάρισαν διά του μόσχου.
\par 23 Αφού τελειώσης καθαρίζων αυτό, θέλεις προσφέρει μόσχον βοός άμωμον και κριόν εκ του ποιμνίου άμωμον.
\par 24 Και θέλεις προσφέρει αυτά ενώπιον του Κυρίου, και οι ιερείς θέλουσι ρίψει όλας επ' αυτά και θέλουσιν ολοκαυτώσει αυτά ολοκαύτωμα εις τον Κύριον.
\par 25 Επτά ημέρας θέλεις ετοιμάζει καθ' εκάστην τράγον διά προσφοράν περί αμαρτίας· και θέλουσιν ετοιμάζει μόσχον βοός και κριόν εκ του ποιμνίου, αμώμους.
\par 26 Επτά ημέρας θέλουσι κάμνει εξιλέωσιν περί του θυσιαστηρίου και καθαρίζει αυτό· και αυτοί θέλουσι καθιερωθή.
\par 27 Και αφού συμπληρωθώσιν αι ημέραι, από της ογδόης ημέρας και εφεξής θέλουσι προσφέρει οι ιερείς τα ολοκαυτώματά σας επί του θυσιαστηρίου και τας ειρηνικάς προσφοράς σας· και εγώ θέλω σας δεχθή, λέγει Κύριος ο Θεός.

\chapter{44}

\par 1 Και με επέστρεψε κατά την οδόν της εξωτέρας πύλης του αγιαστηρίου της βλεπούσης κατά ανατολάς· και αύτη ήτο κεκλεισμένη.
\par 2 Και είπε Κύριος προς εμέ, Η πύλη αύτη θέλει είσθαι κεκλεισμένη, δεν θέλει ανοιχθή, και άνθρωπος δεν θέλει εισέλθει δι' αυτής· διότι Κύριος ο Θεός του Ισραήλ εισήλθε δι' αυτής, διά τούτο θέλει είσθαι κεκλεισμένη.
\par 3 Αύτη θέλει είσθαι διά τον άρχοντα· ο άρχων, ούτος θέλει καθήσει εν αυτή διά να φάγη άρτον ενώπιον του Κυρίου· θέλει εισέλθει διά της οδού της στοάς της πύλης ταύτης και διά της αυτής οδού θέλει εξέλθει.
\par 4 Και με έφερε κατά την οδόν της βορείου πύλης κατέναντι του οίκου· και είδον και ιδού, ο οίκος του Κυρίου ήτο πλήρης της δόξης του Κυρίου· και έπεσον επί πρόσωπόν μου.
\par 5 Και είπε Κύριος προς εμέ, Υιέ ανθρώπου, πρόσεξον εν τη καρδία σου και ιδέ με τους οφθαλμούς σου και άκουσον με τα ώτα σου πάντα όσα εγώ λαλώ προς σε περί πασών των διατάξεων του οίκου του Κυρίου και περί πάντων των νόμων αυτού· και παρατήρησον καλώς την είσοδον του οίκου, μετά πασών των εξόδων του αγιαστηρίου.
\par 6 Και θέλεις ειπεί προς τους απειθείς, προς τον οίκον Ισραήλ, Ούτω λέγει Κύριος ο Θεός· Οίκος Ισραήλ, αρκέσθητε εις πάντα τα βδελύγματα υμών,
\par 7 ότι εισήξατε αλλογενείς, απεριτμήτους την καρδίαν και απεριτμήτους την σάρκα, διά να ήναι εν τω αγιαστηρίω μου, να βεβηλόνωσιν αυτό, τον οίκόν μου, όταν προσφέρητε τον άρτον μου, το πάχος και το αίμα, ενώ παραβαίνουσι την διαθήκην μου εξ αιτίας πάντων των βδελυγμάτων σας.
\par 8 Και δεν εφυλάξατε σεις την φυλακήν των αγίων μου, αλλά κατεστήσατε επί του αγιαστηρίου μου φύλακας της φυλακής μου αντί υμών.
\par 9 Ούτω λέγει Κύριος ο Θεός· Ουδείς αλλογενής απερίτμητος την καρδίαν και απερίτμητος την σάρκα θέλει εισέρχεσθαι εις το αγιαστήριόν μου, εκ πάντων των αλλογενών των μεταξύ των υιών Ισραήλ·
\par 10 αλλ' οι Λευΐται, οίτινες απεστάτησαν απ' εμού ότε ο Ισραήλ απεπλανάτο, αποπλανηθέντες απ' εμού κατόπιν των ειδώλων αυτών, και θέλουσι βαστάσει την ανομίαν αυτών.
\par 11 Και θέλουσιν είσθαι λειτουργοί εν τω αγιαστηρίω μου, επιστατούντες επί των πυλών του οίκου και φυλάττοντες τον οίκον· αυτοί θέλουσι σφάζει εις τον λαόν τα ολοκαυτώματα και τας θυσίας, και αυτοί θέλουσιν ίστασθαι ενώπιον αυτών διά να υπηρετώσιν εις αυτούς.
\par 12 Διότι υπηρέτουν εις αυτούς έμπροσθεν των ειδώλων αυτών και ήσαν πρόσκομμα ανομίας εις τον οίκον Ισραήλ· διά τούτο εγώ ύψωσα την χείρα μου εναντίον αυτών, λέγει Κύριος ο Θεός, και θέλουσι βαστάσει την ανομίαν αυτών.
\par 13 Και δεν θέλουσι με πλησιάζει διά να ιερατεύωσιν εις εμέ και δεν θέλουσι πλησιάζει εις ουδέν από των αγίων μου, εις τα άγια των αγίων· αλλά θέλουσι βαστάσει την αισχύνην αυτών και τα βδελύγματα αυτών, τα οποία έπραξαν.
\par 14 Και θέλω καταστήσει αυτούς φύλακας της φυλακής του οίκου διά πάσαν την υπηρεσίαν αυτού και διά πάντα όσα θέλουσι γίνεσθαι εν αυτώ.
\par 15 Οι δε ιερείς οι Λευΐται, οι υιοί Σαδώκ, οι φυλάξαντες την φυλακήν του αγιαστηρίου μου, ότε οι υιοί Ισραήλ απεπλανώντο απ' εμού, ούτοι θέλουσι με πλησιάζει διά να λειτουργώσιν εις εμέ, και θέλουσιν ίστασθαι ενώπιόν μου διά να προσφέρωσιν εις εμέ το πάχος και το αίμα, λέγει Κύριος ο Θεός·
\par 16 ούτοι θέλουσιν εισέρχεσθαι εις το αγιαστήριόν μου και ούτοι θέλουσι πλησιάζει εις την τράπεζάν μου, διά να λειτουργώσιν εις εμέ και θέλουσι φυλάττει την φυλακήν μου.
\par 17 Και όταν εισέρχωνται εις τας πύλας της εσωτέρας αυλής, θέλουσιν ενδύεσθαι λινά ιμάτια· και δεν θέλει είσθαι μαλλίον επ' αυτών, ενώ λειτουργούσιν εις τας πύλας της εσωτέρας αυλής και ένδον.
\par 18 Θέλουσιν έχει τιάρας λινάς επί τας κεφαλάς αυτών και θέλουσιν έχει λινά περισκελή επί τας οσφύας αυτών· δεν θέλουσι περιζώννυσθαι ουδέν προξενούν ιδρώτα.
\par 19 Και όταν εξέρχωνται εις την αυλήν την εξωτέραν, εις την αυλήν την εξωτέραν προς τον λαόν, θέλουσιν εκδύεσθαι τα ενδύματα αυτών, με τα οποία ελειτούργουν, και θέτει αυτά εις τους αγίους θαλάμους, και θέλουσιν ενδύεσθαι άλλα ενδύματα· και δεν θέλουσιν αγιάζει τον λαόν με τα ενδύματα αυτών.
\par 20 Και δεν θέλουσι ξυρίζει τας κεφαλάς αυτών και δεν θέλουσιν αφίνει την κόμην αυτών να αυξάνηται· μόνον θέλουσι κουρεύει τας κεφαλάς αυτών.
\par 21 Και οίνον δεν θέλει πίνει ουδείς ιερεύς, όταν εισέρχηται εις την εσωτέραν αυλήν.
\par 22 Και χήραν και αποβεβλημένην δεν θέλουσι λαμβάνει εις εαυτούς διά γυναίκα· αλλά θέλουσι λαμβάνει παρθένον εκ του σπέρματος του οίκου Ισραήλ ή χήραν χηρεύουσαν ιερέως.
\par 23 Και θέλουσι διδάσκει τον λαόν μου την διαφοράν μεταξύ αγίου και βεβήλου, και θέλουσι κάμνει αυτούς να διακρίνωσι μεταξύ ακαθάρτου και καθαρού.
\par 24 Και εν ταις αμφισβητήσεσιν ούτοι θέλουσιν ίστασθαι διά να κρίνωσι· κατά τας κρίσεις μου θέλουσι κρίνει αυτάς και θέλουσι φυλάττει τα νόμιμά μου και τα διατάγματά μου εν πάσαις ταις εορταίς μου· και θέλουσιν αγιάζει τα σάββατά μου.
\par 25 Και δεν θέλουσιν εισέρχεσθαι εις νεκρόν ανθρώπου διά να μιανθώσιν· ειμή διά πατέρα ή διά μητέρα ή διά υιόν ή διά θυγατέρα, δι' αδελφόν ή διά αδελφήν μη υπανδρευθείσαν, διά τούτους θέλουσι μιαίνεσθαι.
\par 26 Αφού δε ο μεμιασμένος καθαρισθή, θέλουσιν αριθμεί εις αυτόν επτά ημέρας.
\par 27 Και την ημέραν, καθ' ην εισέρχεται εις το αγιαστήριον, εις την αυλήν την εσωτέραν, διά να λειτουργήση εν τω αγιαστηρίω, θέλει προσφέρει την περί αμαρτίας προσφοράν αυτού, λέγει Κύριος ο Θεός.
\par 28 Και τούτο θέλει είσθαι εις αυτούς διά κληρονομίαν· εγώ είμαι η κληρονομία αυτών· και ιδιοκτησίαν δεν θέλετε δίδει εις αυτούς εν τω Ισραήλ· εγώ είμαι η ιδιοκτησία αυτών.
\par 29 Θέλουσι τρώγει την εξ αλφίτων προσφοράν και την περί αμαρτίας προσφοράν και την περί ανομίας προσφοράν· και παν αφιέρωμα μεταξύ του Ισραήλ θέλει είσθαι αυτών.
\par 30 Και αι απαρχαί πάντων των πρωτογεννημάτων και πάσα υψουμένη προσφορά πάντων εκ παντός είδους των υψουμένων προσφορών σας θέλουσιν είσθαι των ιερέων· και την απαρχήν της ζύμης σας θέλετε δίδει εις τον ιερέα, διά να επαναπαύη ευλογίαν εις τους οίκους σας.
\par 31 Οι ιερείς δεν θέλουσι τρώγει ουδέν θνησιμαίον ή θηριάλωτον, είτε πτηνόν είτε κτήνος.

\chapter{45}

\par 1 Και όταν κληρόνητε την γην εις κληρονομίαν, θέλετε χωρίσει μερίδα εις τον Κύριον, μερίδα αγίαν εκ της γής· το μήκος θέλει είσθαι μήκος εικοσιπέντε χιλιάδων καλάμων και το πλάτος δέκα χιλιάδων· τούτο θέλει είσθαι άγιον κατά πάντα τα όρια αυτού κύκλω.
\par 2 Εκ τούτου θέλουσιν είσθαι διά το αγιαστήριον πεντακόσιαι κατά μήκος με πεντακοσίας κατά πλάτος, τετράγωνον κύκλω, και πεντήκοντα πήχαι κύκλω διά τα προάστεια αυτού.
\par 3 Κατά τούτο λοιπόν το μέτρον θέλεις μετρήσει μήκος εικοσιπέντε χιλιάδων και πλάτος δέκα χιλιάδων, και εν τούτω θέλει είσθαι το αγιαστήριον, το άγιον των αγίων.
\par 4 Τούτο θέλει είσθαι εκ της γης, αγία μερίς διά τους ιερείς, τους λειτουργούντας εν τω αγιαστηρίω, τους πλησιάζοντας διά να λειτουργώσιν εις τον Κύριον· και θέλει είσθαι εις αυτούς τόπος διά οικίας και τόπος άγιος διά το αγιαστήριον.
\par 5 Και εικοσιπέντε χιλιάδας μήκους και δέκα χιλιάδας πλάτους θέλουσιν έχει οι Λευΐται δι' εαυτούς, οι υπηρέται του οίκου, διά ιδιοκτησίαν μετά είκοσι θαλάμων.
\par 6 Και θέλετε δώσει διά ιδιοκτησίαν της πόλεως πέντε χιλιάδας πλάτους και εικοσιπέντε χιλιάδας μήκους πλησίον της αγίας μερίδος· τούτο θέλει είσθαι δι' άπαντα τον οίκον Ισραήλ·
\par 7 και διά τον άρχοντα θέλει είσθαι μερίς εντεύθεν και εκείθεν της αγίας μερίδος και της ιδιοκτησίας της πόλεως, κατά πρόσωπον της αγίας μερίδος και κατά πρόσωπον της ιδιοκτησίας της πόλεως, από του δυτικού προς δυσμάς και από του ανατολικού προς ανατολάς· και το μήκος θέλει είσθαι πλησίον μιας εκάστης των μερίδων, από του δυτικού ορίου προς το ανατολικόν όριον.
\par 8 Εις γην θέλει είσθαι εις αυτόν η ιδιοκτησία, εν τω Ισραήλ· και οι άρχοντές μου δεν θέλουσι πλέον καταβλίβει τον λαόν μου· το υπόλοιπον δε της γης θέλουσι δώσει εις τον οίκον Ισραήλ κατά τας φυλάς αυτών.
\par 9 Ούτω λέγει Κύριος ο Θεός· Αρκεί εις εσάς, άρχοντες του Ισραήλ· απομακρύνατε την βίαν και αρπαγήν και κάμνετε κρίσιν και δικαιοσύνην· σηκώσατε τας καταδυναστείας σας από του λαού μου, λέγει Κύριος ο Θεός.
\par 10 Δικαίαν πλάστιγγα θέλετε έχει και δίκαιον εφά και δίκαιον βαθ.
\par 11 Το εφά και το βαθ θέλουσιν είσθαι του αυτού μέτρου, ώστε το βαθ να περιλαμβάνη το δέκατον του χομόρ και το εφά το δέκατον του χομόρ· το μέτρον αυτού θέλει είσθαι κατά το χομόρ.
\par 12 Και ο σίκλος θέλει είσθαι είκοσι γερά· είκοσι σίκλοι, εικοσιπέντε σίκλοι, δεκαπέντε σίκλοι, θέλει είσθαι η μνα σας.
\par 13 Η υψουμένη προσφορά, την οποίαν θέλετε προσφέρει, είναι αύτη· Το έκτον του εφά ενός χομόρ σίτου· και θέλετε δίδει το έκτον του εφά ενός χομόρ κριθής.
\par 14 Περί δε του διατάγματος του ελαίου, ενός βαθ ελαίου, θέλετε προσφέρει το δέκατον του βαθ διά εν κορ, το οποίον είναι εν χομόρ εκ δέκα βάθ· διότι δέκα βαθ είναι εν χομόρ.
\par 15 Και εκ του ποιμνίου εν πρόβατον από των διακοσίων, από των παχειών βοσκών του Ισραήλ, διά προσφοράν εξ αλφίτων και διά ολοκαύτωμα και διά ειρηνικάς προσφοράς, διά να κάμνη εξιλέωσιν υπέρ αυτών, λέγει Κύριος ο Θεός.
\par 16 Πας ο λαός της γης θέλει δίδει ταύτην την υψουμένην προσφοράν εις τον άρχοντα εν τω Ισραήλ.
\par 17 Εις δε τον άρχοντα ανήκει να δίδη τα ολοκαυτώματα και τας εξ αλφίτων προσφοράς και τας σπονδάς, εν ταις εορταίς και εν ταις νεομηνίαις και εν τοις σάββασι κατά πάσας τας πανηγύρεις του οίκου Ισραήλ· αυτός θέλει ετοιμάζει την περί αμαρτίας προσφοράν και την εξ αλφίτων προσφοράν και το ολοκαύτωμα και τας ειρηνικάς προσφοράς, διά να κάμνη εξιλέωσιν υπέρ του οίκου Ισραήλ.
\par 18 Ούτω λέγει Κύριος ο Θεός· Εν τω πρώτω μηνί, τη πρώτη του μηνός, θέλεις λαμβάνει μόσχον βοός άμωμον και θέλεις καθαρίζει το αγιαστήριον·
\par 19 και ο ιερεύς θέλει λαμβάνει από του αίματος της περί αμαρτίας προσφοράς και θέλει θέτει επί τους παραστάτας του οίκου και επί τας τέσσαρας γωνίας της οφρύος του θυσιαστηρίου και επί τους παραστάτας της πύλης της εσωτέρας αυλής.
\par 20 Και ούτω θέλεις κάμνει τη εβδόμη του μηνός υπέρ παντός αμαρτάνοντος εξ αγνοίας και υπέρ του απλού· ούτω θέλετε κάμνει εξιλέωσιν υπέρ του οίκου.
\par 21 Εν τω πρώτω μηνί, τη δεκάτη τετάρτη ημέρα του μηνός, θέλει είσθαι εις εσάς το πάσχα, εορτή επτά ημερών· άζυμα θέλουσι τρώγει.
\par 22 Και κατ' εκείνην την ημέραν ο άρχων θέλει ετοιμάζει υπέρ εαυτού και υπέρ παντός του λαού της γης μόσχον διά προσφοράν περί αμαρτίας.
\par 23 Και εν ταις επτά ημέραις της εορτής θέλει κάμνει ολοκαυτώματα εις τον Κύριον, επτά μόσχους και επτά κριούς αμώμους καθ' ημέραν εν ταις επτά ημέραις, και τράγον εξ αιγών καθ' ημέραν διά προσφοράν περί αμαρτίας.
\par 24 Και θέλει ετοιμάζει προσφοράν εξ αλφίτων ενός εφά διά τον μόσχον και ενός εφά διά τον κριόν και ενός ιν ελαίου εις το εφά.
\par 25 Εν τω εβδόμω μηνί, τη δεκάτη πέμπτη ημέρα του μηνός, θέλει κάμνει εν τη εορτή κατά τα αυτά επτά ημέρας, κατά την προσφοράν την περί αμαρτίας, κατά τα ολοκαυτώματα και κατά την εξ αλφίτων προσφοράν και κατά το έλαιον.

\chapter{46}

\par 1 Ούτω λέγει Κύριος ο Θεός· Η πύλη εσωτέρας αυλής, η βλέπουσα προς ανατολάς, θέλει είσθαι κεκλεισμένη τας εξ εργασίμους ημέρας. την δε ημέραν του σαββάτου θέλει ανοίγεσθαι και την ημέραν της νεομηνίας θέλει ανοίγεσθαι.
\par 2 Και ο άρχων θέλει εισέλθει διά της οδού της στοάς της πύλης της έξωθεν και θέλει ίστασθαι πλησίον του παραστάτου της πύλης, και οι ιερείς θέλουσιν ετοιμάζει το ολοκαύτωμα αυτού και τας ειρηνικάς προσφοράς αυτού, και αυτός θέλει προσκυνήσει επί το κατώφλιον της πύλης· τότε θέλει εξέλθει· η πύλη όμως δεν θέλει κλεισθή έως εσπέρας.
\par 3 Ο λαός της γης θέλει προσκυνεί ωσαύτως εις την είσοδον της πύλης ταύτης ενώπιον του Κυρίου εν τοις σάββασι και εν ταις νεομηνίαις.
\par 4 Το δε ολοκαύτωμα, το οποίον ο άρχων θέλει προσφέρει εις τον Κύριον την ημέραν του σαββάτου, θέλει είσθαι εξ αρνία άμωμα, και κριός άμωμος.
\par 5 Και η εξ αλφίτων προσφορά θέλει είσθαι εν εφά δι' ένα κριόν· η δε εξ αλφίτων προσφορά διά τα αρνία, όσον προαιρείται να δώση· και εν ιν ελαίου δι' εν εφά.
\par 6 Και την ημέραν της νεομηνίας θέλει είσθαι μόσχος βοός άμωμος και εξ αρνία και κριός· άμωμα θέλουσιν είσθαι.
\par 7 Και θέλει ετοιμάζει προσφοράν εξ αλφίτων, εν εφά διά τον μόσχον και εν εφά διά τον κριόν· διά δε τα αρνία, όσον είναι ικανή η χειρ αυτού· και εν ιν ελαίου δι' εν εφά.
\par 8 Και όταν ο άρχων εισέρχηται, θέλει εισέρχεσθαι διά της οδού της στοάς της πύλης ταύτης και θέλει εξέρχεσθαι διά της οδού της αυτής.
\par 9 Όταν όμως ο λαός της γης έρχηται ενώπιον του Κυρίου εν ταις επισήμοις εορταίς, ο εισερχόμενος διά της οδού της βορείου πύλης διά να προσκυνήση θέλει εξέρχεσθαι διά της οδού της νοτίου πύλης· και ο εισερχόμενος διά της οδού της νοτίου πύλης θέλει εξέρχεσθαι διά της οδού της βορείου πύλης· δεν θέλει επιστρέφει διά της οδού της πύλης, δι' ης εισήλθεν, αλλά θέλει εξέρχεσθαι διά της απέναντι.
\par 10 Και ο άρχων εν τω μέσω αυτών εισερχομένων θέλει εισέρχεσθαι, και εξερχομένων θέλει εξέρχεσθαι.
\par 11 Και εν ταις εορταίς και εν ταις πανηγύρεσιν η εξ αλφίτων προσφορά θέλει είσθαι εν εφά διά τον μόσχον και εν εφά διά τον κριόν, διά δε τα αρνία, όσον προαιρείται να δώση· και εν ιν ελαίου δι' εν εφά.
\par 12 Όταν δε ο άρχων ετοιμάζη αυτοπροαίρετον ολοκαύτωμα ή προσφοράς ειρηνικάς αυτοπροαιρέτους εις τον Κύριον, τότε θέλουσιν ανοίγει εις αυτόν την πύλην την βλέπουσαν κατά ανατολάς, και θέλει ετοιμάζει το ολοκαύτωμα αυτού και τας ειρηνικάς προσφοράς αυτού, καθώς κάμνει εν τη ημέρα του σαββάτου· τότε θέλει εξέρχεσθαι, και μετά την εξέλευσιν αυτού θέλουσι κλείει την πύλην.
\par 13 Θέλεις δε ετοιμάζει καθ' ημέραν ολοκαύτωμα εις τον Κύριον εξ αρνίου ενιαυσίου αμώμου· καθ' εκάστην πρωΐαν θέλεις ετοιμάζει αυτό.
\par 14 Και θέλεις ετοιμάζει δι' αυτό προσφοράν εξ αλφίτων καθ' εκάστην πρωΐαν, το έκτον του εφά και έλαιον το τρίτον του ιν, διά να αναμιγνύης αυτό μετά της σεμιδάλεως· προσφοράν εξ αλφίτων εις τον Κύριον διά παντός κατά πρόσταγμα αιώνιον.
\par 15 Και θέλουσιν ετοιμάζει το αρνίον και την εξ αλφίτων προσφοράν και το έλαιον καθ' εκάστην πρωΐαν, ολοκαύτωμα παντοτεινόν.
\par 16 Ούτω λέγει Κύριος ο Θεός· Εάν ο άρχων δώση δώρον εις τινά εκ των υιών αυτού, τούτο θέλει είσθαι κληρονομία αυτού· των υιών αυτού είναι· ιδιοκτησία αυτών θέλει είσθαι εν κληρονομία.
\par 17 Αλλ' εάν δώση δώρον εκ της κληρονομίας αυτού εις τινά εκ των δούλων αυτού τότε θέλει είσθαι αυτού έως του έτους της αφέσεως· μετά τούτο θέλει επιστρέφει εις τον άρχοντα· διότι η κληρονομία αυτού είναι των υιών αυτού· αυτών θέλει είσθαι.
\par 18 Ο δε άρχων δεν θέλει λαμβάνει εκ της κληρονομίας του λαού, εκβάλλων αυτούς διά καταδυναστείας εκ της ιδιοκτησίας αυτών· εκ της ιδιοκτησίας αυτού θέλει κληροδοτήσει τους υιούς αυτού, διά να μη διασκορπίζηται ο λαός μου έκαστος εκ της ιδιοκτησίας αυτού.
\par 19 Έπειτα με έφερε διά της εισόδου της εις τα πλάγια της πύλης προς τους αγίους θαλάμους των ιερέων τους βλέποντας προς βορράν· και ιδού, εκεί τόπος εις το ενδότερον προς δυσμάς.
\par 20 Και είπε προς εμέ, ούτος είναι ο τόπος, όπου οι ιερείς θέλουσι βράζει την περί ανομίας προσφοράν και την περί αμαρτίας προσφοράν, όπου θέλουσι ψήνει την εξ αλφίτων προσφοράν, διά να μη εκφέρωσιν αυτά εις την αυλήν την εξωτέραν, διά να αγιάσωσι τον λαόν.
\par 21 Και με εξήγαγεν εις την αυλήν την εξωτέραν και με περιέφερεν εις τας τέσσαρας γωνίας της αυλής· και ιδού, αυλή εν εκάστη γωνία της αυλής.
\par 22 κατά τας τέσσαρας γωνίας της αυλής ήσαν αυλαί ηνωμέναι, τεσσαράκοντα πηχών το μήκος και τριάκοντα το πλάτος· αι τέσσαρες αύται γωνίαι ήσαν του αυτού μέτρου.
\par 23 Και ήτο σειρά οικοδομών κύκλω αυτών, κύκλω των τεσσάρων αυτών· και ήσαν μαγειρεία κατεσκευασμένα υποκάτω των σειρών κύκλω.
\par 24 Και είπε προς εμέ, Ταύτα είναι τα οικήματα των μαγείρων, όπου οι υπηρέται του οίκου θέλουσι βράζει τας θυσίας του λαού.

\chapter{47}

\par 1 Και με επέστρεψεν εις την θύραν του οίκου· και ιδού, ύδατα εξερχόμενα κάτωθεν από του κατωφλίου του οίκου προς ανατολάς· διότι το μέτωπον του οίκου ήτο προς ανατολάς, και τα ύδατα κατέβαινον κάτωθεν από του δεξιού πλαγίου του οίκου, κατά το νότιον του θυσιαστηρίου.
\par 2 Και με εξήγαγε διά της οδού της πύλης της προς βορράν και με έφερε κύκλω διά της έξωθεν οδού προς την πύλην την εξωτέραν, διά της οδού της βλεπούσης προς ανατολάς· και ιδού, τα ύδατα έρρεον από του δεξιού πλαγίου.
\par 3 Και ο άνθρωπος, όστις είχε το μέτρον εν τη χειρί αυτού, εξελθών προς ανατολάς εμέτρησε χιλίας πήχας και με διεβίβασε διά των υδάτων· τα ύδατα ήσαν έως των αστραγάλων.
\par 4 Και εμέτρησε χιλίας και με διεβίβασε διά των υδάτων· τα ύδατα ήσαν έως των γονάτων. Πάλιν εμέτρησε χιλίας και με διεβίβασε· τα ύδατα ήσαν έως της οσφύος.
\par 5 Έπειτα εμέτρησε χιλίας· και ήτο ποταμός, τον οποίον δεν ηδυνάμην να διαβώ, διότι τα ύδατα ήσαν υψωμένα, ύδατα κολυμβήματος, ποταμός αδιάβατος.
\par 6 Και είπε προς εμέ, Είδες, υιέ ανθρώπου; Τότε με έφερε και με επέστρεψεν εις το χείλος του ποταμού.
\par 7 Και ότε επέστρεψα, ιδού, κατά το χείλος του ποταμού δένδρα πολλά σφόδρα, εντεύθεν και εντεύθεν.
\par 8 Και είπε προς εμέ, τα ύδατα ταύτα εξέρχονται προς την ανατολικήν γην και καταβαίνουσιν εις την πεδινήν και εισέρχονται εις την θάλασσαν· και όταν εκχυθώσιν εις την θάλασσαν, τα ύδατα αυτής θέλουσιν ιαθή.
\par 9 Και παν έμψυχον έρπον, εις όσα μέρη ήθελον επέλθει ούτοι οι ποταμοί, θέλει ζή· και θέλει είσθαι εκεί πλήθος ιχθύων πολύ σφόδρα, επειδή τα ύδατα ταύτα έρχονται εκεί· διότι θέλουσιν ιαθή· και θέλουσι ζη τα πάντα, όπου ο ποταμός έρχεται.
\par 10 Και οι αλιείς θέλουσιν ίστασθαι επ' αυτήν από Εν-γαδδί έως Εν-εγλαΐμ· εκεί θέλουσιν εξαπλόνει τα δίκτυα· οι ιχθύες αυτών θέλουσιν είσθαι κατά τα είδη αυτών ως οι ιχθύες της μεγάλης θαλάσσης, πολλοί σφόδρα.
\par 11 Οι ελώδεις όμως τόποι αυτής και οι βαλτώδεις αυτής δεν θέλουσιν ιαθή· θέλουσιν είσθαι διωρισμένοι διά άλας.
\par 12 Πλησίον δε του ποταμού επί του χείλους αυτού, εντεύθεν και εντεύθεν, θέλουσιν αυξάνεσθαι δένδρα παντός είδους διά τροφήν, των οποίων τα φύλλα δεν θέλουσι μαραίνεσθαι και ο καρπός αυτών δεν θέλει εκλείψει· νέος καρπός θέλει γεννάσθαι καθ' έκαστον μήνα, διότι τα ύδατα αυτού εξέρχονται από του αγιαστηρίου· και ο καρπός αυτών θέλει είσθαι διά τροφήν και το φύλλον αυτών διά ιατρείαν.
\par 13 Ούτω λέγει Κύριος ο Θεός· ταύτα θέλουσιν είσθαι τα όρια, διά των οποίων θέλετε κληρονομήσει την γην κατά τας δώδεκα φυλάς του Ισραήλ· ο Ιωσήφ θέλει έχει δύο μερίδας.
\par 14 Σεις δε θέλετε κληρονομήσει αυτήν, έκαστος καθώς ο αδελφός αυτού· περί της οποίας ύψωσα την χείρα μου ότι θέλω δώσει αυτήν εις τους πατέρας σας· και η γη αύτη θέλει κληρωθή εις εσάς εις κληρονομίαν.
\par 15 Και τούτο θέλει είσθαι το όριον της γης προς το βόρειον πλάγιον, από της θαλάσσης της μεγάλης, κατά την οδόν της Εθλών, καθώς υπάγει τις εις Σεδάδ,
\par 16 Αιμάθ, Βηρωθά, Σιβραΐμ, ήτις είναι αναμέσον του ορίου της Δαμασκού και του ορίου της Αιμάθ, Ασάρ-αττιχών, η πλησίον των ορίων της Αυράν.
\par 17 Και το όριον από της θαλάσσης θέλει είσθαι Ασάρ-ενάν, το όριον της Δαμασκού, και το βόρειον το κατά βορράν, και το όριον της Αιμάθ. Και τούτο είναι το βόρειον πλευρόν.
\par 18 Το δε ανατολικόν πλευρόν θέλετε μετρήσει από Αυράν και από Δαμασκού και από Γαλαάδ και από της γης του Ισραήλ κατά τον Ιορδάνην, από του ορίου του προς την θάλασσαν την ανατολικήν. Και τούτο είναι το ανατολικόν πλευρόν.
\par 19 Το δε μεσημβρινόν πλευρόν προς νότον, από Θαμάρ έως των υδάτων της Μεριβά Κάδης, κατά την έκτασιν του χειμάρρου έως της μεγάλης θαλάσσης. Και τούτο είναι το νότιον πλευρόν προς μεσημβρίαν.
\par 20 Το δε δυτικόν πλευρόν θέλει είσθαι η μεγάλη θάλασσα από του ορίου, εωσού έλθη τις κατέναντι της Αιμάθ. Τούτο είναι το δυτικόν πλευρόν.
\par 21 Ούτω θέλετε διαιρέσει την γην ταύτην μεταξύ σας κατά τας φυλάς του Ισραήλ.
\par 22 Και θέλετε κληρώσει αυτήν εις εαυτούς διά κληρονομίαν, μετά των ξένων των παροικούντων μεταξύ σας, όσοι γεννήσωσιν υιούς εν μέσω σας· και θέλουσιν είσθαι εις εσάς ως αυτόχθονες μεταξύ των υιών Ισραήλ· θέλουσιν έχει μεθ' υμών κληρονομίαν μεταξύ των φυλών Ισραήλ.
\par 23 Και εις ην τινά φυλήν παροική ο ξένος, εκεί θέλετε δώσει εις αυτόν την κληρονομίαν αυτού, λέγει Κύριος ο Θεός.

\chapter{48}

\par 1 Ταύτα δε είναι τα ονόματα των φυλών· από του βορείου άκρου, κατά την οδόν της Εθλών, καθώς υπάγει τις εις Αιμάθ, Ασάρ-ενάν, το όριον της Δαμασκού προς βορράν, κατά το μέρος της Αιμάθ· και ταύτα είναι το ανατολικόν αυτού πλευρόν και το δυτικόν· του Δαν, εν μερίδιον.
\par 2 Και πλησίον του ορίου του Δαν, από του ανατολικού πλευρού έως του δυτικού πλευρού, του Ασήρ, εν.
\par 3 Και πλησίον του ορίου του Ασήρ, από του ανατολικού πλευρού έως του δυτικού πλευρού, του Νεφθαλί, εν.
\par 4 Και πλησίον του ορίου του Νεφθαλί, από του ανατολικού πλευρού έως του δυτικού πλευρού, του Μανασσή, εν.
\par 5 Και πλησίον του ορίου του Μανασσή, από του ανατολικού πλευρού έως του δυτικού πλευρού, του Εφραΐμ, εν.
\par 6 Και πλησίον του ορίου του Εφραΐμ, από του ανατολικού πλευρού έως του δυτικού πλευρού, του Ρουβήν, εν.
\par 7 Και πλησίον του ορίου του Ρουβήν, από του ανατολικού πλευρού έως του δυτικού πλευρού, του Ιούδα, εν.
\par 8 Και πλησίον του ορίου του Ιούδα, από του ανατολικού πλευρού έως του δυτικού πλευρού, θέλει είσθαι το μερίδιον, το οποίον θέλετε αφιερώσει από εικοσιπέντε χιλιάδων καλάμων εις πλάτος, κατά δε το μήκος ως εν των άλλων μεριδίων, από του ανατολικού πλευρού έως του δυτικού πλευρού· και το αγιαστήριον θέλει είσθαι εν μέσω αυτού.
\par 9 Η μερίς, την οποίαν θέλετε αφιερώσει εις τον Κύριον, θέλει είσθαι από είκοσιπέντε χιλιάδων κατά μήκος και δέκα χιλιάδων κατά πλάτος.
\par 10 Και δι' αυτούς, διά τους ιερείς, θέλει είσθαι αύτη η αγία μερίς, προς βορράν είκοσιπέντε χιλιάδων κατά μήκος και προς δυσμάς δέκα χιλιάδων κατά πλάτος και προς ανατολάς δέκα χιλιάδων κατά πλάτος και προς νότον είκοσιπέντε χιλιάδων κατά μήκος· και το αγιαστήριον του Κυρίου θέλει είσθαι εν μέσω αυτού.
\par 11 Αύτη θέλει είσθαι διά τους ιερείς τους καθιερωθέντας, εκ των υιών Σαδώκ, τους φυλάξαντας την φυλακήν μου, τους μη αποπλανηθέντας εις την αποπλάνησιν των υιών Ισραήλ, καθώς απεπλανήθησαν οι Λευΐται.
\par 12 Και αύτη η αφιερωθείσα μερίς της γης θέλει είσθαι εις αυτούς αγιωτάτη, πλησίον του ορίου των Λευϊτών.
\par 13 Και πλησίον του ορίου των ιερέων θέλουσιν έχει οι Λευΐται είκοσιπέντε χιλιάδας κατά μήκος και δέκα χιλιάδας κατά πλάτος· όλον το μήκος θέλει είσθαι είκοσιπέντε χιλιάδων και το πλάτος δέκα χιλιάδων.
\par 14 Και δεν θέλουσι πωλήσει εξ αυτού ουδέ θέλουσιν αλλάξει ουδέ θέλουσιν απαλλοτριώσει τα πρωτογεννήματα της γής· διότι είναι άγιον εις τον Κύριον.
\par 15 Αι δε πέντε χιλιάδες αι περισσεύουσαι εις το πλάτος απέναντι των είκοσιπέντε χιλιάδων θέλουσιν είσθαι τόπος βέβηλος διά την πόλιν, προς κατοίκησιν και διά προάστεια· και η πόλις θέλει είσθαι εν μέσω αυτού.
\par 16 Και ταύτα θέλουσιν είσθαι τα μέτρα αυτής· το βόρειον πλευρόν τέσσαρες χιλιάδες και πεντακόσιαι και το μεσημβρινόν πλευρόν τέσσαρες χιλιάδες και πεντακόσιαι και κατά το ανατολικόν πλευρόν τέσσαρες χιλιάδες και πεντακόσιαι και το δυτικόν πλευρόν τέσσαρες χιλιάδες και πεντακόσιαι.
\par 17 Και τα προάστεια της πόλεως θέλουσιν είσθαι προς βορράν διακόσιαι πεντήκοντα και προς νότον διακόσιαι πεντήκοντα και προς ανατολάς διακόσιαι πεντήκοντα και προς δυσμάς διακόσιαι πεντήκοντα.
\par 18 Και το επίλοιπον κατά μήκος το συνεχόμενον μετά της αγίας μερίδος, δέκα χιλιάδες προς ανατολάς και δέκα χιλιάδες προς δυσμάς, και θέλει συνέχεσθαι μετά της αγίας μερίδος, και τα γεννήματα αυτού θέλουσιν είσθαι διά τροφήν των υπηρετούντων την πόλιν.
\par 19 Και οι υπηρετούντες την πόλιν θέλουσιν υπηρετεί αυτήν εκ πασών των φυλών του Ισραήλ.
\par 20 Άπαν το αφιέρωμα θέλει είσθαι εικοσιπέντε χιλιάδων μετά εικοσιπέντε χιλιάδων· τετράγωνον θέλετε αφιερώσει την αγίαν μερίδα, μετά της ιδιοκτησίας της πόλεως.
\par 21 Και το υπόλοιπον θέλει είσθαι διά τον άρχοντα, εντεύθεν και εντεύθεν της αγίας μερίδος, και της ιδιοκτησίας της πόλεως, απέναντι των εικοσιπέντε χιλιάδων του αφιερώματος κατά το ανατολικόν όριον, και προς δυσμάς απέναντι των εικοσιπέντε χιλιάδων κατά το δυτικόν όριον, πλησίον των μερίδων του άρχοντος. Ούτω θέλει είσθαι αγία μερίς· και το αγιαστήριον του οίκου εν μέσω αυτού.
\par 22 Και εκ της ιδιοκτησίας των Λευϊτών και εκ της ιδιοκτησίας της πόλεως, αίτινες είναι εν μέσω του ανήκοντος εις τον άρχοντα, μεταξύ του ορίου του Ιούδα και του ορίου του Βενιαμίν, τούτο θέλει είσθαι του άρχοντος.
\par 23 Περί δε των επιλοίπων φυλών, από του ανατολικού πλευρού έως του δυτικού πλευρού, του Βενιαμίν, εν μερίδιον.
\par 24 Και πλησίον του ορίου του Βενιαμίν, από του ανατολικού πλευρού έως του δυτικού πλευρού, του Συμεών, εν.
\par 25 Και πλησίον του ορίου του Συμεών, από του ανατολικού πλευρού έως του δυτικού πλευρού, του Ισσάχαρ, εν.
\par 26 Και πλησίον του ορίου του Ισσάχαρ, από του ανατολικού πλευρού έως του δυτικού πλευρού, του Ζαβουλών, εν.
\par 27 Και πλησίον του ορίου του Ζαβουλών, από του ανατολικού πλευρού έως του δυτικού πλευρού, του Γαδ, εν.
\par 28 Και πλησίον του ορίου του Γαδ κατά το μεσημβρινόν πλευρόν προς νότον, το όριον θέλει είσθαι από Θαμάρ έως των υδάτων της Μεριβά Κάδης, κατά τον χείμαρρον έως της μεγάλης θαλάσσης.
\par 29 Αύτη είναι η γη, την οποίαν θέλετε κληρώσει εις τας φυλάς του Ισραήλ διά κληρονομίαν, και αύται είναι αι μερίδες αυτών, λέγει Κύριος ο Θεός.
\par 30 Και αύτη είναι η έκτασις της πόλεως η προς βορράν, τέσσαρες χιλιάδες και πεντακόσια μέτρα.
\par 31 Και αι πύλαι της πόλεως θέλουσιν είσθαι κατά τα ονόματα των φυλών Ισραήλ· τρεις πύλαι προς βορράν· η πύλη του Ρουβήν μία, η πύλη του Ιούδα μία, πύλη του Λευΐ μία.
\par 32 Και κατά το ανατολικόν πλευρόν τέσσαρες χιλιάδες και πεντακόσια μέτρα· και τρεις πύλαι· και η πύλη του Ιωσήφ μία, η πύλη του Βενιαμίν μία, η πύλη του Δαν μία.
\par 33 Και κατά το μεσημβρινόν πλευρόν τέσσαρες χιλιάδες και πεντακόσια μέτρα, και τρεις πύλαι· η πύλη του Συμεών μία, η πύλη του Ισσάχαρ μία, η πύλη του Ζαβουλών μία.
\par 34 Κατά το δυτικόν πλευρόν τέσσαρες χιλιάδες και πεντακόσια· αι πύλαι αυτών τρείς· η πύλη του Γαδ μία, η πύλη του Ασήρ μία, η πύλη του Νεφθαλί μία.
\par 35 Η περιφέρεια ήτο δεκαοκτώ χιλιάδων μέτρων. Και το όνομα της πόλεως απ' εκείνης της ημέρας θέλει είσθαι, Ο Κύριος εκεί.


\end{document}