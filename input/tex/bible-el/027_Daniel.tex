\begin{document}

\title{Δανιήλ}


\chapter{1}

\par 1 Εν τω τρίτω έτει της βασιλείας του Ιωακείμ, βασιλέως του Ιούδα, ήλθε Ναβουχοδονόσορ ο βασιλεύς της Βαβυλώνος εις Ιερουσαλήμ και επολιόρκησεν αυτήν.
\par 2 Και παρέδωκε Κύριος εις την χείρα αυτού τον Ιωακείμ, βασιλέα του Ιούδα, και μέρος των σκευών του οίκου του Θεού· και έφερεν αυτά εις γην Σενναάρ, εις τον οίκον του θεού αυτού· και εισήγαγε τα σκεύη εις το θησαυροφυλάκιον του θεού αυτού.
\par 3 Και είπεν ο βασιλεύς προς τον Ασφενάζ τον αρχιευνούχον αυτού, να φέρη εκ των υιών Ισραήλ και εκ του σπέρματος του βασιλικού και εκ των αρχόντων
\par 4 νεανίσκους μη έχοντας μηδένα μώμον και ωραίους την όψιν και νοήμονας εν πάση σοφία και ειδήμονας πάσης γνώσεως και έχοντας φρόνησιν και δυναμένους να ίστανται εν τω παλατίω του βασιλέως και να διδάσκη αυτούς τα γράμματα και την γλώσσαν των Χαλδαίων.
\par 5 Και διέταξεν εις αυτούς ο βασιλεύς καθημερινήν μερίδα από των βασιλικών εδεσμάτων και από του οίνου, τον οποίον αυτός έπινε· και αφού ανατραφώσι τρία έτη, να ίστανται μετά ταύτα ενώπιον του βασιλέως.
\par 6 Και μεταξύ τούτων ήσαν, εκ των υιών Ιούδα, Δανιήλ, Ανανίας, Μισαήλ και Αζαρίας·
\par 7 εις τους οποίους ο αρχιευνούχος επέθηκεν ονόματα· και τον μεν Δανιήλ ωνόμασε Βαλτασάσαρ· τον δε Ανανίαν Σεδράχ· τον δε Μισαήλ Μισάχ· και τον Αζαρίαν Αβδέ-νεγώ.
\par 8 Αλλ' ο Δανιήλ έβαλεν εν τη καρδία αυτού να μη μιανθή από των εδεσμάτων του βασιλέως ουδέ από του οίνου τον οποίον εκείνος έπινε· διά τούτο παρεκάλεσε τον αρχιευνούχον να μη μιανθή.
\par 9 Και έκαμεν ο Θεός τον Δανιήλ να εύρη χάριν και έλεος ενώπιον του αρχιευνούχου.
\par 10 Και είπεν ο αρχιευνούχος προς τον Δανιήλ, Εγώ φοβούμαι τον κύριόν μου τον βασιλέα, όστις διέταξε το φαγητόν σας και το ποτόν σας, μήποτε ίδη τα πρόσωπά σας σκυθρωπότερα παρά των νεανίσκων των συνομηλίκων σας, και ενοχοποιήσητε την κεφαλήν μου εις τον βασιλέα.
\par 11 Και είπεν ο Δανιήλ προς τον Αμελσάρ, τον οποίον ο αρχιευνούχος κατέστησεν επί τον Δανιήλ, τον Ανανίαν, τον Μισαήλ και τον Αζαρίαν,
\par 12 Δοκίμασον, παρακαλώ, τους δούλους σου δέκα ημέρας· και ας δοθώσιν εις ημάς όσπρια να τρώγωμεν και ύδωρ να πίνωμεν·
\par 13 έπειτα ας θεωρηθώσι τα πρόσωπα ημών ενώπιόν σου και το πρόσωπον των νεανίσκων, οίτινες τρώγουσιν από των εδεσμάτων του βασιλέως· και όπως ίδης, κάμε με τους δούλους σου.
\par 14 Και εισήκουσεν αυτών εις τούτο το πράγμα και εδοκίμασεν αυτούς δέκα ημέρας.
\par 15 Και μετά το τέλος των δέκα ημερών τα πρόσωπα αυτών εφάνησαν ώραιότερα και παχύτερα εις την σάρκα παρά πάντων των νεανίσκων, οίτινες έτρωγον τα εδέσματα του βασιλέως.
\par 16 Και αφήρει ο Αμελσάρ το φαγητόν αυτών και τον οίνον τον οποίον έπρεπε να πίνωσι και έδιδεν εις αυτούς όσπρια.
\par 17 Και εις τους τέσσαρας τούτους νεανίσκους έδωκεν ο Θεός γνώσιν και σύνεσιν εις πάσαν μάθησιν και σοφίαν, και κατέστησε τον Δανιήλ νοήμονα εις πάσαν όρασιν και ενύπνιον.
\par 18 Και εν τω τέλει των ημερών, ότε ο βασιλεύς είπε να εισάξωσιν αυτούς, ο αρχιευνούχος εισήξεν αυτούς ενώπιον του Ναβουχοδονόσορ.
\par 19 Και ελάλησε μετ' αυτών ο βασιλεύς· και δεν ευρέθη μεταξύ πάντων αυτών όμοιος του Δανιήλ, του Ανανία, του Μισαήλ και του Αζαρία, και ίσταντο ενώπιον του βασιλέως.
\par 20 Και εν πάση υποθέσει σοφίας και νοήσεως, περί της οποίας ο βασιλεύς ηρώτησεν αυτούς, εύρηκεν αυτούς δεκαπλασίως καλητέρους παρά πάντας τους μάγους και επαοιδούς, όσοι ήσαν εν παντί τω βασιλείω αυτού.
\par 21 Και διέμενεν ο Δανιήλ ούτως έως του πρώτου έτους Κύρου του βασιλέως.

\chapter{2}

\par 1 Και εν τω δευτέρω έτει της βασιλείας του Ναβουχοδονόσορ, ο Ναβουχοδονόσορ ενυπνιάσθη ενύπνια, και εταράχθη το πνεύμα αυτού και ο ύπνος αυτού έφυγεν απ' αυτού.
\par 2 Και είπεν ο βασιλεύς να καλέσωσι τους μάγους και τους επαοιδούς και τους γόητας και τους Χαλδαίους, διά να φανερώσωσι προς τον βασιλέα τα ενύπνια αυτού. Ήλθον λοιπόν και εστάθησαν έμπροσθεν του βασιλέως.
\par 3 Και είπε προς αυτούς ο βασιλεύς, Ενυπνιάσθην ενύπνιον και το πνεύμά μου εταράχθη εις το να γνωρίσω το ενύπνιον.
\par 4 Και ελάλησαν οι Χαλδαίοι προς τον βασιλέα Συριστί, λέγοντες, Βασιλεύ, ζήθι εις τον αιώνα· ειπέ το ενύπνιον προς τους δούλους σου και ημείς θέλομεν φανερώσει την ερμηνείαν.
\par 5 Ο βασιλεύς απεκρίθη και είπε προς τους Χαλδαίους, το πράγμα διέφυγεν απ' εμού· εάν δεν κάμητε γνωστόν εις εμέ το ενύπνιον και την ερμηνείαν αυτού, θέλετε καταμελισθή και αι οικίαι σας θέλουσι γείνει κοπρώνες·
\par 6 αλλ' εάν φανερώσητε το ενύπνιον και την ερμηνείαν αυτού, θέλετε λάβει παρ' εμού δώρα και αμοιβάς και τιμήν μεγάλην· το ενύπνιον λοιπόν και την ερμηνείαν αυτού φανερώσατε εις εμέ.
\par 7 Απεκρίθησαν εκ δευτέρου και είπον, Ας είπη ο βασιλεύς το ενύπνιον προς τους δούλους αυτού, και ημείς θέλομεν φανερώσει την ερμηνείαν αυτού.
\par 8 Ο βασιλεύς απεκρίθη και είπε, Επ' αληθείας καταλαμβάνω ότι σεις θέλετε να εξαγοράζητε τον καιρόν, βλέποντες ότι διέφυγεν απ' εμού το πράγμα.
\par 9 Αλλ' εάν δεν κάμητε γνωστόν εις εμέ το ενύπνιον, αύτη μόνη η απόφασις είναι διά σάς· διότι συνεβουλεύθητε να είπητε ψευδείς και διεφθαρμένους λόγους έμπροσθέν μου, εωσού παρέλθη ο καιρός· είπατέ μοι λοιπόν το ενύπνιον και θέλω γνωρίσει ότι δύνασθε να φανερώσητε εις εμέ και την ερμηνείαν αυτού.
\par 10 Απεκρίθησαν οι Χαλδαίοι έμπροσθεν του βασιλέως και είπον, δεν υπάρχει άνθρωπος επί της γης δυνάμενος να φανερώση το πράγμα του βασιλέως· καθώς δεν υπάρχει ουδείς βασιλεύς, άρχων ή διοικητής, όστις να ζητή τοιαύτα πράγματα παρά μάγου ή επαοιδού ή Χαλδαίου·
\par 11 και το πράγμα το οποίον ο βασιλεύς ζητεί είναι μέγα, και δεν είναι άλλος δυνάμενος να φανερώση αυτό έμπροσθεν του βασιλέως, εκτός των θεών, των οποίων η κατοικία δεν είναι μετά σαρκός.
\par 12 Διά τούτο εθυμώθη ο βασιλεύς και ωργίσθη σφόδρα και είπε να απολέσωσι πάντας τους σοφούς της Βαβυλώνος.
\par 13 Και εξήλθεν η απόφασις και οι σοφοί εθανατόνοντο· εζήτησαν δε και τον Δανιήλ και τους συντρόφους αυτού, διά να θανατώσωσιν αυτούς.
\par 14 Και απεκρίθη ο Δανιήλ μετά φρονήσεως και σοφίας προς τον Αριώχ τον αρχισωματοφύλακα του βασιλέως, όστις εξήλθε διά να θανατώση τους σοφούς της Βαβυλώνος,
\par 15 απεκρίθη και είπε προς τον Αριώχ, τον άρχοντα του βασιλέως, Διά τι η βιαία αύτη απόφασις παρά του βασιλέως; Και ο Αριώχ εφανέρωσε το πράγμα προς τον Δανιήλ.
\par 16 Και εισήλθεν ο Δανιήλ και παρεκάλεσε τον βασιλέα να δώση καιρόν εις αυτόν και ήθελε φανερώσει την ερμηνείαν προς τον βασιλέα.
\par 17 Και υπήγεν ο Δανιήλ εις τον οίκον αυτού και εγνωστοποίησε το πράγμα προς τον Ανανίαν, προς τον Μισαήλ και προς τον Αζαρίαν, τους συντρόφους αυτού,
\par 18 διά να ζητήσωσιν έλεος παρά του Θεού του ουρανού περί του μυστηρίου τούτου, ώστε να μη απολεσθή ο Δανιήλ και οι σύντροφοι αυτού μετά των επιλοίπων σοφών της Βαβυλώνος.
\par 19 Και το μυστήριον απεκαλύφθη προς τον Δανιήλ δι' οράματος της νυκτός. Τότε ευλόγησεν ο Δανιήλ τον Θεόν του ουρανού.
\par 20 Και ελάλησεν ο Δανιήλ και είπεν, Είη το όνομα του Θεού ευλογημένον από του αιώνος και έως του αιώνος· διότι αυτού είναι η σοφία και η δύναμις·
\par 21 και αυτός μεταβάλλει τους καιρούς και τους χρόνους· καθαιρεί βασιλείς και καθιστά βασιλείς· δίδει σοφίαν εις τους σοφούς και γνώσιν εις τους συνετούς.
\par 22 Αυτός αποκαλύπτει τα βαθέα και τα κεκρυμμένα· γνωρίζει τα εν τω σκότει και το φως κατοικεί μετ' αυτού.
\par 23 Σε, Θεέ των πατέρων μου, ευχαριστώ και σε δοξολογώ, όστις μοι έδωκας σοφίαν και δύναμιν, και έκαμες γνωστόν εις εμέ ό,τι εδεήθημεν παρά σου. Διότι συ έκαμες γνωστήν εις ημάς του βασιλέως την υπόθεσιν.
\par 24 Υπήγε λοιπόν ο Δανιήλ προς τον Αριώχ, τον οποίον ο βασιλεύς διέταξε να απολέση τους σοφούς της Βαβυλώνος· υπήγε και είπε προς αυτόν ούτω· Μη απολέσης τους σοφούς της Βαβυλώνος· είσαξόν με ενώπιον του βασιλέως και εγώ θέλω φανερώσει την ερμηνείαν προς τον βασιλέα.
\par 25 Και εισήξεν ο Αριώχ μετά σπουδής τον Δανιήλ ενώπιον του βασιλέως και είπε προς αυτόν ούτως, Εύρηκα άνδρα εκ των υιών της αιχμαλωσίας του Ιούδα, όστις θέλει φανερώσει την ερμηνείαν εις τον βασιλέα.
\par 26 Απεκρίθη ο βασιλεύς και είπε προς τον Δανιήλ, του οποίου το όνομα ήτο Βαλτασάσαρ, Είσαι ικανός να φανερώσης προς εμέ το ενύπνιον το οποίον είδον και την ερμηνείαν αυτού;
\par 27 Απεκρίθη ο Δανιήλ ενώπιον του βασιλέως και είπε, Το μυστήριον, περί του οποίου ο βασιλεύς επερωτά, δεν δύνανται σοφοί, επαοιδοί, μάγοι, μάντεις, να φανερώσωσι προς τον βασιλέα·
\par 28 αλλ' είναι Θεός εν τω ουρανώ, όστις αποκαλύπτει μυστήρια και κάμνει γνωστόν εις τον βασιλέα Ναβουχοδονόσορ, τι μέλλει γενέσθαι εν ταις εσχάταις ημέραις. Το ενύπνιόν σου και αι οράσεις της κεφαλής σου επί της κλίνης σου είναι αύται·
\par 29 βασιλεύ, οι διαλογισμοί σου ανέβησαν εις τον νούν σου επί της κλίνης σου, περί του τι μέλλει γενέσθαι μετά ταύτα· και ο αποκαλύπτων μυστήρια έκαμε γνωστόν εις σε τι μέλλει γενέσθαι.
\par 30 Πλην όσον το κατ' εμέ, το μυστήριον τούτο δεν απεκαλύφθη προς εμέ διά σοφίας, την οποίαν έχω εγώ μάλλον παρά πάντας τους ζώντας, αλλά διά να φανερωθή η ερμηνεία προς τον βασιλέα και διά να γνωρίσης τους διαλογισμούς της καρδίας σου.
\par 31 Συ, βασιλεύ, εθεώρεις και ιδού, εικών μεγάλη· εξαίσιος ήτο εκείνη η εικών και υπέροχος η λάμψις αυτής, ισταμένης ενώπιόν σου, και η μορφή αυτής φοβερά.
\par 32 Η κεφαλή της εικόνος εκείνης ήτο εκ χρυσού καθαρού, το στήθος αυτής και οι βραχίονες αυτής εξ αργύρου, η κοιλία αυτής και οι μηροί αυτής εκ χαλκού,
\par 33 αι κνήμαι αυτής εκ σιδήρου, οι πόδες αυτής μέρος μεν εκ σιδήρου, μέρος δε εκ πηλού.
\par 34 Εθεώρεις εωσού απεκόπη λίθος άνευ χειρών, και εκτύπησε την εικόνα επί τους πόδας αυτής τους εκ σιδήρου και πηλού και κατεσύντριψεν αυτούς.
\par 35 Τότε ο σίδηρος, ο πηλός, ο χαλκός, ο άργυρος και ο χρυσός κατεσυντρίφθησαν ομού και έγειναν ως λεπτόν άχυρον αλωνίου θερινού· και ο άνεμος εσήκωσεν αυτά και ουδείς τόπος ευρέθη αυτών· ο δε λίθος ο κτυπήσας την εικόνα έγεινεν όρος μέγα και εγέμισεν όλην την γην.
\par 36 Τούτο είναι το ενύπνιον· και την ερμηνείαν αυτού θέλομεν ειπεί ενώπιον του βασιλέως.
\par 37 Συ, βασιλεύ, είσαι βασιλεύς βασιλέων· διότι ο Θεός του ουρανού έδωκεν εις σε βασιλείαν, δύναμιν και ισχύν και δόξαν.
\par 38 Και πάντα τόπον, όπου κατοικούσιν οι υιοί των ανθρώπων, τα θηρία του αγρού και τα πετεινά του ουρανού, έδωκεν εις την χείρα σου και σε κατέστησε κύριον επί πάντων τούτων· συ είσαι η κεφαλή εκείνη η χρυσή.
\par 39 Και μετά σε θέλει αναστηθή άλλη βασιλεία κατωτέρα σου και τρίτη άλλη βασιλεία εκ χαλκού, ήτις θέλει κυριεύσει επί πάσης της γης.
\par 40 Και τετάρτη βασιλεία θέλει σταθή ισχυρά ως ο σίδηρος· καθώς ο σίδηρος κατακόπτει και καταλεπτύνει τα πάντα· μάλιστα καθώς ο σίδηρος ο συντρίβων τα πάντα, ούτω θέλει κατακόπτει και κατασυντρίβει.
\par 41 Περί δε του ότι είδες τους πόδας και τους δακτύλους, μέρος μεν εκ πηλού κεραμέως, μέρος δε εκ σιδήρου, θέλει είσθαι βασιλεία διηρημένη· πλην θέλει μένει τι εν αυτή εκ της δυνάμεως του σιδήρου, καθώς είδες τον σίδηρον αναμεμιγμένον μετά αργιλλώδους πηλού.
\par 42 Και καθώς οι δάκτυλοι των ποδών ήσαν μέρος εκ σιδήρου και μέρος εκ πηλού, ούτως η βασιλεία θέλει είσθαι κατά μέρος ισχυρά και κατά μέρος εύθραυστος.
\par 43 Και καθώς είδες τον σίδηρον αναμεμιγμένον μετά του αργιλλώδους πηλού, ούτω θέλουσιν αναμιχθή διά σπέρματος ανθρώπων· πλην δεν θέλουσιν είσθαι κεκολλημένοι ο εις μετά του άλλου, καθώς ο σίδηρος δεν μιγνύεται μετά του πηλού.
\par 44 Και εν ταις ημέραις των βασιλέων εκείνων, θέλει αναστήσει ο Θεός του ουρανού βασιλείαν, ήτις εις τον αιώνα δεν θέλει φθαρή· και η βασιλεία αύτη δεν θέλει περάσει εις άλλον λαόν· θέλει κατασυντρίψει και συντελέσει πάσας ταύτας τας βασιλείας, αυτή δε θέλει διαμένει εις τους αιώνας,
\par 45 καθώς είδες ότι απεκόπη λίθος εκ του όρους άνευ χειρών και κατεσύντριψε τον σίδηρον, τον χαλκόν, τον πηλόν, τον άργυρον και τον χρυσόν· ο Θεός ο μέγας έκαμε γνωστόν εις τον βασιλέα ό,τι θέλει γείνει μετά ταύτα· και αληθινόν είναι το ενύπνιον και πιστή η ερμηνεία αυτού.
\par 46 Τότε ο βασιλεύς Ναβουχοδονόσορ έπεσεν επί πρόσωπον και προσεκύνησε τον Δανιήλ και προσέταξε να προσφέρωσιν εις αυτόν προσφοράν και θυμιάματα.
\par 47 Και αποκριθείς ο βασιλεύς προς τον Δανιήλ, είπεν, Επ' αληθείας, ο Θεός σας, αυτός είναι Θεός θεών και Κύριος των βασιλέων και όστις αποκαλύπτει μυστήρια· διότι ηδυνήθης να αποκαλύψης το μυστήριον τούτο.
\par 48 Τότε ο βασιλεύς εμεγάλυνε τον Δανιήλ και δώρα μεγάλα και πολλά έδωκεν εις αυτόν και κατέστησεν αυτόν κύριον επί πάσης της επαρχίας της Βαβυλώνος και αρχιδιοικητήν επί πάντας τους σοφούς της Βαβυλώνος.
\par 49 Και εζήτησεν ο Δανιήλ παρά του βασιλέως και κατέστησε τον Σεδράχ, τον Μισάχ και τον Αβδέ-νεγώ επί τας υποθέσεις της επαρχίας της Βαβυλώνος· ο δε Δανιήλ ευρίσκετο εν τη αυλή του βασιλέως.

\chapter{3}

\par 1 Ναβουχοδονόσορ ο βασιλεύς έκαμεν εικόνα χρυσήν, το ύψος αυτής εξήκοντα πηχών και το πλάτος αυτής εξ πηχών· και έστησεν αυτήν εν τη πεδιάδι Δουρά, εν τη επαρχία της Βαβυλώνος.
\par 2 Και απέστειλε Ναβουχοδονόσορ ο βασιλεύς να συνάξη τους σατράπας, τους διοικητάς και τους τοπάρχας, τους κριτάς, τους θησαυροφύλακας, τους συμβούλους, τους νομοδιδασκάλους και πάντας τους άρχοντας των επαρχιών, διά να έλθωσιν εις τα εγκαίνια της εικόνος, την οποίαν έστησε Ναβουχοδονόσορ ο βασιλεύς.
\par 3 Και οι σατράπαι, οι διοικηταί και οι τοπάρχαι, οι κριταί, οι θησαυροφύλακες, οι σύμβουλοι, οι νομοδιδάσκαλοι και πάντες οι άρχοντες των επαρχιών συνήχθησαν εις τα εγκαίνια της εικόνος, την οποίαν έστησε Ναβουχοδονόσορ ο βασιλεύς· και εστάθησαν έμπροσθεν της εικόνος, την οποίαν έστησεν ο Ναβουχοδονόσορ.
\par 4 Και κήρυξ εβόα μεγαλοφώνως, Εις εσάς προστάττεται, λαοί, έθνη και γλώσσαι,
\par 5 καθ' ην ώραν ακούσητε τον ήχον της σάλπιγγος, της σύριγγος, της κιθάρας, της σαμβύκης, του ψαλτηρίου, της συμφωνίας και παντός είδους μουσικής, πεσόντες προσκυνήσατε την εικόνα την χρυσήν, την οποίαν έστησε Ναβουχοδονόσορ ο βασιλεύς·
\par 6 και όστις δεν πέση και προσκυνήση, την αυτήν ώραν θέλει ριφθή εις το μέσον της καμίνου του πυρός της καιομένης.
\par 7 Διά τούτο ότε ήκουσαν πάντες οι λαοί τον ήχον της σάλπιγγος, της σύριγγος, της κιθάρας, της σαμβύκης, του ψαλτηρίου και παντός είδους μουσικής, πίπτοντες πάντες οι λαοί, τα έθνη και αι γλώσσαι προσεκύνουν την εικόνα την χρυσήν, την οποίαν έστησε Ναβουχοδονόσορ ο βασιλεύς.
\par 8 Χαλδαίοι δε τινές προσήλθον τότε και διέβαλον τους Ιουδαίους·
\par 9 και είπον λέγοντες προς τον βασιλέα Ναβουχοδονόσορ, Βασιλεύ, ζήθι εις τον αιώνα.
\par 10 Συ, βασιλεύ, εξέδωκας πρόσταγμα, πας άνθρωπος, όστις ακούση τον ήχον της σάλπιγγος, της σύριγγος, της κιθάρας, της σαμβύκης, του ψαλτηρίου και της συμφωνίας και παντός είδους μουσικής, να πέση και να προσκυνήση την εικόνα την χρυσήν·
\par 11 και όστις δεν πέση και προσκυνήση, να ριφθή εις το μέσον της καμίνου του πυρός της καιομένης.
\par 12 Είναι άνδρες τινές Ιουδαίοι, τους οποίους κατέστησας επί τας υποθέσεις της επαρχίας της Βαβυλώνος, ο Σεδράχ, ο Μισάχ και ο Αβδέ-νεγώ· ούτοι οι άνθρωποι, βασιλεύ, δεν σε εσεβάσθησαν· τους θεούς σου δεν λατρεύουσι και την εικόνα την χρυσήν, την οποίαν έστησας, δεν προσκυνούσι.
\par 13 Τότε ο Ναβουχοδονόσορ μετά θυμού και οργής προσέταξε να φέρωσι τον Σεδράχ, Μισάχ και Αβδέ-νεγώ. Και έφεραν τους ανθρώπους τούτους ενώπιον του βασιλέως.
\par 14 Και αποκριθείς ο Ναβουχοδονόσορ είπε προς αυτούς, Τωόντι, Σεδράχ, Μισάχ και Αβδέ-νεγώ, τους θεούς μου δεν λατρεύετε και την εικόνα την χρυσήν, την οποίαν έστησα, δεν προσκυνείτε;
\par 15 τώρα λοιπόν εάν ήσθε έτοιμοι, οπόταν ακούσητε τον ήχον της σάλπιγγος, της σύριγγος, της κιθάρας, της σαμβύκης, του ψαλτηρίου και της συμφωνίας και παντός είδους μουσικής, να πέσητε και να προσκυνήσητε την εικόνα την οποίαν έκαμα, καλώς· εάν όμως δεν προσκυνήσητε, θέλετε ριφθή την αυτήν ώραν εις το μέσον της καμίνου του πυρός της καιομένης· και τις είναι εκείνος ο Θεός, όστις θέλει σας ελευθερώσει εκ των χειρών μου;
\par 16 Απεκρίθησαν ο Σεδράχ, ο Μισάχ και ο Αβδέ-νεγώ και είπον προς τον βασιλέα, Ναβουχοδονόσορ, ημείς δεν έχομεν χρείαν να σοι αποκριθώμεν περί του πράγματος τούτου.
\par 17 Εάν ήναι ούτως, ο Θεός ημών, τον οποίον ημείς λατρεύομεν, είναι δυνατός να μας ελευθερώση εκ της καμίνου του πυρός της καιομένης· και εκ της χειρός σου, βασιλεύ, θέλει μας ελευθερώσει.
\par 18 Αλλά και αν ουχί, ας ήναι γνωστόν εις σε, βασιλεύ, ότι τους θεούς σου δεν λατρεύομεν και την εικόνα την χρυσήν, την οποίαν έστησας, δεν προσκυνούμεν.
\par 19 Τότε ο Ναβουχοδονόσορ επλήσθη θυμού και η όψις του προσώπου αυτού ηλλοιώθη κατά του Σεδράχ, του Μισάχ και του Αβδέ-νεγώ· και λαλήσας προσέταξε να εκκαύσωσι την κάμινον επταπλασίως μάλλον παρ' όσον εφαίνετο καιομένη.
\par 20 Και προσέταξε τους δυνατωτέρους άνδρας του στρατεύματος αυτού να δέσωσι τον Σεδράχ, Μισάχ και Αβδέ-νεγώ, και να ρίψωσιν αυτούς εις την κάμινον του πυρός την καιομένην.
\par 21 Τότε οι άνδρες εκείνοι εδέθησαν μετά των σαλβαρίων αυτών, των τιαρών αυτών και των περικνημίδων αυτών και των άλλων ενδυμάτων αυτών και ερρίφθησαν εις το μέσον της καμίνου του πυρός της καιομένης.
\par 22 Επειδή δε η προσταγή του βασιλέως ήτο κατεπείγουσα και η κάμινος εξεκαύθη εις υπερβολήν, η φλόξ του πυρός εθανάτωσε τους άνδρας εκείνους, οίτινες εσήκωσαν τον Σεδράχ, Μισάχ και Αβδέ-νεγώ.
\par 23 Ούτοι δε οι τρεις άνδρες, ο Σεδράχ, Μισάχ και Αβδέ-νεγώ, έπεσον δεμένοι εις το μέσον της καμίνου του πυρός της καιομένης.
\par 24 Ο δε Ναβουχοδονόσορ ο βασιλεύς εξεπλάγη· και σηκωθείς μετά σπουδής ελάλησε και είπε προς τους μεγιστάνας αυτού, δεν ερρίψαμεν τρεις άνδρας δεδεμένους εις το μέσον του πυρός; οι δε απεκρίθησαν και είπον προς τον βασιλέα, Αληθώς, βασιλεύ.
\par 25 Και αποκριθείς είπεν, Ιδού, εγώ βλέπω τέσσαρας άνδρας λελυμένους, περιπατούντας εν μέσω του πυρός, και βλάβη δεν είναι εις αυτούς, και η όψις του τετάρτου είναι ομοία με Υιόν Θεού.
\par 26 Τότε πλησιάσας ο Ναβουχοδονόσορ εις το στόμα της καμίνου του πυρός της καιομένης ελάλησε και είπε, Σεδράχ, Μισάχ και Αβδέ-νεγώ, δούλοι του Θεού του Υψίστου, εξέλθετε και έλθετε. Τότε ο Σεδράχ, Μισάχ και Αβδέ-νεγώ εξήλθον εκ μέσου του πυρός.
\par 27 Και συναχθέντες οι σατράπαι, οι διοικηταί και οι τοπάρχαι και οι μεγιστάνες του βασιλέως είδον τους άνδρας τούτους, ότι επί των σωμάτων αυτών το πυρ δεν ίσχυσε και θριξ της κεφαλής αυτών δεν εκάη και τα σαλβάρια αυτών δεν παρήλλαξαν ουδέ οσμή πυρός επέρασεν επ' αυτούς.
\par 28 Τότε ελάλησεν ο Ναβουχοδονόσορ και είπεν, Ευλογητός ο Θεός του Σεδράχ, Μισάχ και Αβδέ-νεγώ, όστις απέστειλε τον άγγελον αυτού και ηλευθέρωσε τους δούλους αυτού, οίτινες ήλπισαν επ' αυτόν και παρήκουσαν τον λόγον του βασιλέως και παρέδωκαν τα σώματα αυτών, διά να μη λατρεύσωσι μηδέ να προσκυνήσωσιν άλλον θεόν εκτός του Θεού αυτών.
\par 29 Διά τούτο εκδίδω πρόσταγμα, ότι πας λαός, έθνος και γλώσσα, ήτις λαλήση κακόν εναντίον του Θεού του Σεδράχ, Μισάχ και Αβδέ-νεγώ, θέλει καταμελισθή, και αι οικίαι αυτών θέλουσι γείνει κοπρώνες· διότι άλλος Θεός δεν είναι δυνάμενος να ελευθερώση ούτω.
\par 30 Τότε ο βασιλεύς προεβίβασε τον Σεδράχ, Μισάχ και Αβδέ-νεγώ εις την επαρχίαν της Βαβυλώνος.

\chapter{4}

\par 1 Ναβουχοδονόσορ ο βασιλεύς, προς πάντας τους λαούς, έθνη και γλώσσας τους κατοικούντας επί πάσης της γής· Ειρήνη ας πληθυνθή εις εσάς.
\par 2 Τα σημεία και τα θαυμάσια, τα οποία έκαμεν εις εμέ ο Θεός ο Ύψιστος, ήρεσεν ενώπιόν μου να αναγγείλω.
\par 3 Πόσον είναι μεγάλα τα σημεία αυτού· και πόσον ισχυρά τα θαυμάσια αυτού· η βασιλεία αυτού είναι βασιλεία αιώνιος και η εξουσία αυτού εις γενεάν και γενεάν.
\par 4 Εγώ ο Ναβουχοδονόσορ ήμην αναπαυόμενος εν τω οίκω μου και ακμάζων εν τω παλατίω μου.
\par 5 Είδον ενύπνιον, το οποίον με κατέπληξε, και οι διαλογισμοί μου επί της κλίνης μου και αι οράσεις της κεφαλής μου με ετάραξαν.
\par 6 Διά τούτο εξέδωκα πρόσταγμα να εισαχθώσιν ενώπιόν μου πάντες οι σοφοί της Βαβυλώνος, διά να φανερώσωσιν εις εμέ την ερμηνείαν του ενυπνίου.
\par 7 Τότε εισήλθον οι μάγοι, οι επαοιδοί, οι Χαλδαίοι και οι μάντεις· και εγώ είπα το ενύπνιον έμπροσθεν αυτών, αλλά δεν μοι εφανέρωσαν την ερμηνείαν αυτού.
\par 8 Ύστερον δε ήλθεν ο Δανιήλ ενώπιόν μου, του οποίου το όνομα ήτο Βαλτασάσαρ κατά το όνομα του Θεού μου, και εις τον οποίον είναι το πνεύμα των αγίων θεών· και έμπροσθεν τούτου είπα το ενύπνιον, λέγων,
\par 9 Βαλτασάσαρ, άρχων των μάγων, επειδή εγνώρισα ότι το πνεύμα των αγίων θεών είναι εν σοι, και ουδέν κρυπτόν είναι δύσκολον εις σε, ειπέ τας οράσεις του ενυπνίου μου, το οποίον είδον, και την ερμηνείαν αυτού.
\par 10 Ιδού αι οράσεις της κεφαλής μου επί της κλίνης μου· Έβλεπον και ιδού, δένδρον εν μέσω της γης και το ύψος αυτού μέγα.
\par 11 Το δένδρον εμεγαλύνθη και ενεδυναμώθη και το ύψος αυτού έφθανεν έως του ουρανού, και η θέα αυτού έως των περάτων πάσης της γης.
\par 12 Τα φύλλα αυτού ήσαν ώραία και ο καρπός αυτού πολύς και εν αυτώ ήτο τροφή πάντων· υπό την σκιάν αυτού ανεπαύοντο τα θηρία του αγρού, και εν τοις κλάδοις αυτού κατεσκήνουν τα πετεινά του ουρανού, και εξ αυτού ετρέφετο πάσα σαρξ.
\par 13 Είδον εν ταις οράσεσι της κεφαλής μου επί της κλίνης μου και ιδού, φύλαξ και άγιος κατέβη εκ του ουρανού,
\par 14 και εφώνησε μεγαλοφώνως και είπεν ούτω· Κόψατε το δένδρον και αποκόψατε τους κλάδους αυτού· εκτινάξατε τα φύλλα αυτού και διασκορπίσατε τον καρπόν αυτού· ας φύγωσι τα θηρία υποκάτωθεν αυτού και τα πετεινά από των κλάδων αυτού·
\par 15 το στέλεχος όμως των ριζών αυτού αφήσατε εν τη γη, και τούτο με δεσμόν σιδηρούν και χαλκούν, εν τω τρυφερώ χόρτω του αγρού· και θέλει βρέχεσθαι με την δρόσον του ουρανού και η μερίς αυτού θέλει είσθαι μετά των θηρίων εν τω χόρτω της γής·
\par 16 η καρδία αυτού θέλει μεταβληθή εκ της ανθρωπίνης και θέλει δοθή εις αυτόν καρδία θηρίου· και επτά καιροί θέλουσι παρέλθει επ' αυτόν.
\par 17 Το πράγμα τούτο είναι διά προστάγματος των φυλάκων και η υπόθεσις διά του λόγου των αγίων· ώστε να γνωρίσωσιν οι ζώντες, ότι ο Ύψιστος είναι Κύριος της βασιλείας των ανθρώπων, και εις όντινα θέλει δίδει αυτήν, και το εξουθένημα των ανθρώπων καθιστά επ' αυτήν.
\par 18 Τούτο το ενύπνιον είδον εγώ ο Ναβουχοδονόσορ ο βασιλεύς· και συ, Βαλτασάσαρ, ειπέ την ερμηνείαν αυτού· διότι πάντες οι σοφοί του βασιλείου μου δεν είναι ικανοί να φανερώσωσι προς εμέ την ερμηνείαν· συ δε είσαι ικανός· διότι το πνεύμα των αγίων θεών είναι εν σοι.
\par 19 Τότε ο Δανιήλ, του οποίου το όνομα ήτο Βαλτασάσαρ, έμεινεν εκστατικός έως μιας ώρας, και οι διαλογισμοί αυτού ετάραττον αυτόν. Ο βασιλεύς ελάλησε και είπε, Βαλτασάσαρ, ας μη σε ταράττη το ενύπνιον ή η ερμηνεία αυτού. Ο Βαλτασάσαρ απεκρίθη και είπε, Κύριέ μου, το ενύπνιον ας επέλθη επί τους μισούντάς σε και η ερμηνεία αυτού επί τους εχθρούς σου.
\par 20 Το δένδρον, το οποίον είδες, το αυξηθέν και ενδυναμωθέν, του οποίου το ύψος έφθανεν έως του ουρανού και η θέα αυτού επί πάσαν την γην,
\par 21 και τα φύλλα αυτού ήσαν ώραία και ο καρπός αυτού πολύς, και τροφή πάντων ήτο εν αυτώ, και υποκάτω αυτού κατώκουν τα θηρία του αγρού, εν δε τοις κλάδοις αυτού κατεσκήνουν τα πετεινά του ουρανού,
\par 22 συ είσαι το δένδρον τούτο, βασιλεύ, όστις εμεγαλύνθης και ενεδυναμώθης· και η μεγαλωσύνη σου υψώθη και έφθασεν έως του ουρανού και η εξουσία σου έως των περάτων της γης.
\par 23 Περί δε του ότι είδεν ο βασιλεύς φύλακα και άγιον καταβαίνοντα εκ του ουρανού και λέγοντα, Κόψατε το δένδρον και καταστρέψατε αυτό· μόνον το στέλεχος των ριζών αυτού αφήσατε εν τη γη, και τούτο με δεσμόν σιδηρούν και χαλκούν, εν τω τρυφερώ χόρτω του αγρού· και ας βρέχηται υπό της δρόσου του ουρανού και μετά των θηρίων του αγρού ας ήναι η μερίς αυτού, εωσού παρέλθωσιν επτά καιροί επ' αυτό·
\par 24 αύτη είναι η ερμηνεία, βασιλεύ, και αύτη η απόφασις του Υψίστου, ήτις έφθασεν επί τον κύριόν μου τον βασιλέα·
\par 25 και θέλεις διωχθή εκ των ανθρώπων και μετά των θηρίων του αγρού θέλει είσθαι η κατοικία σου, και θέλεις τρώγει χόρτον ως οι βόες και υπό της δρόσου του ουρανού θέλεις βρέχεσθαι· και επτά καιροί θέλουσι παρέλθει επί σε, εωσού γνωρίσης ότι ο Ύψιστος είναι Κύριος της βασιλείας των ανθρώπων και εις όντινα θέλει, δίδει αυτήν.
\par 26 Περί δε του ότι προσετάχθη να αφήσωσι το στέλεχος των ριζών του δένδρου· το βασίλειόν σου θέλει στερεωθή εν σοι, αφού γνωρίσης την ουράνιον εξουσίαν.
\par 27 Διά τούτο, βασιλεύ, ας γείνη δεκτή η συμβουλή μου προς σε, και έκκοψον τας αμαρτίας σου διά δικαιοσύνης και τας ανομίας σου διά οικτιρμών πενήτων· ίσως και διαρκέση η ευημερία σου.
\par 28 Πάντα ταύτα ήλθον επί τον Ναβουχοδονόσορ τον βασιλέα.
\par 29 Εν τω τέλει δώδεκα μηνών, ενώ περιεπάτει επί του βασιλικού παλατίου της Βαβυλώνος,
\par 30 ελάλησεν ο βασιλεύς και είπε, Δεν είναι αύτη η Βαβυλών η μεγάλη, την οποίαν εγώ ωκοδόμησα διά καθέδραν του βασιλείου με την ισχύν της δυνάμεώς μου και εις τιμήν της δόξης μου;
\par 31 Ο λόγος ήτο έτι εν τω στόματι του βασιλέως και έγεινε φωνή εξ ουρανού λέγουσα, Προς σε αναγγέλλεται, Ναβουχοδονόσορ βασιλεύ· η βασιλεία παρήλθεν από σού·
\par 32 και θέλεις εκδιωχθή εκ των ανθρώπων και μετά των θηρίων του αγρού θέλει είσθαι η κατοικία σου· χόρτον ως οι βόες θέλεις τρώγει, και επτά καιροί θέλουσι παρέλθει επί σε, εωσού γνωρίσης ότι ο Ύψιστος είναι Κύριος της βασιλείας των ανθρώπων, και εις όντινα θέλει, δίδει αυτήν.
\par 33 Εν αυτή τη ώρα ο λόγος εξετελέσθη επί τον Ναβουχοδονόσορ· και εξεδιώχθη εκ των ανθρώπων και χόρτον ως οι βόες έτρωγε και υπό της δρόσου του ουρανού το σώμα αυτού εβρέχετο, εωσού αι τρίχες αυτού ηυξήνθησαν ως αετών πτερά και οι όνυχες αυτού ως ορνέων.
\par 34 Και εν τέλει των ημερών, εγώ ο Ναβουχοδονόσορ εσήκωσα τους οφθαλμούς μου προς τον ουρανόν και αι φρένες μου επέστρεψαν εις εμέ και ευλόγησα τον Ύψιστον και ήνεσα και εδόξασα τον ζώντα εις τον αιώνα, του οποίου η εξουσία είναι εξουσία αιώνιος και η βασιλεία αυτού εις γενεάν και γενεάν,
\par 35 και πάντες οι κάτοικοι της γης λογίζονται ενώπιον αυτού ως ουδέν, και κατά την θέλησιν αυτού πράττει εις το στράτευμα του ουρανού και εις τους κατοίκους της γης, και δεν υπάρχει ο εμποδίζων την χείρα αυτού ή ο λέγων προς αυτόν, Τι έκαμες;
\par 36 Εν τω αυτώ καιρώ αι φρένες μου επέστρεψαν εις εμέ· και προς δόξαν της βασιλείας μου επανήλθεν εις εμέ η λαμπρότης μου και η μορφή μου και οι αυλικοί μου και οι μεγιστάνές μου με εζήτουν, και εστερεώθην εν τη βασιλεία μου και μεγαλειότης περισσοτέρα προσετέθη εις εμέ.
\par 37 Τώρα εγώ ο Ναβουχοδονόσορ αινώ και υπερυψώ και δοξάζω τον βασιλέα του ουρανού, διότι πάντα τα έργα αυτού είναι αλήθεια και αι οδοί αυτού κρίσις, και τους περιπατούντας εν τη υπερηφανία δύναται να ταπεινώση.

\chapter{5}

\par 1 Βαλτάσαρ ο βασιλεύς έκαμε συμπόσιον μέγα εις χιλίους εκ των μεγιστάνων αυτού και έπινεν οίνον ενώπιον των χιλίων.
\par 2 Και εν τη γεύσει του οίνου προσέταξεν ο Βαλτάσαρ να φέρωσι τα σκεύη τα χρυσά και τα αργυρά, τα οποία Ναβουχοδονόσορ ο πατήρ αυτού αφήρεσεν εκ του ναού του εν Ιερουσαλήμ, διά να πίωσιν εν αυτοίς ο βασιλεύς και οι μεγιστάνες αυτού, αι γυναίκες αυτού και αι παλλακαί αυτού.
\par 3 Και εφέρθησαν τα σκεύη τα χρυσά, τα οποία αφηρέθησαν εκ του ναού του οίκου του Θεού του εν Ιερουσαλήμ· και έπινον εν αυτοίς ο βασιλεύς και οι μεγιστάνες αυτού, αι γυναίκες αυτού και αι παλλακαί αυτού.
\par 4 Έπινον οίνον και ήνεσαν τους θεούς τους χρυσούς και αργυρούς, τους χαλκούς, τους σιδηρούς, τους ξυλίνους και τους λιθίνους.
\par 5 Εν αυτή τη ώρα εξήλθον δάκτυλοι χειρός ανθρώπου και έγραψαν κατέναντι της λυχνίας επί το κονίαμα του τοίχου του παλατίου του βασιλέως· και ο βασιλεύς έβλεπε την παλάμην της χειρός, ήτις έγραψε.
\par 6 Τότε η όψις του βασιλέως ηλλοιώθη και οι διαλογισμοί αυτού συνετάραττον αυτόν, ώστε οι σύνδεσμοι της οσφύος αυτού διελύοντο και τα γόνατα αυτού συνεκρούοντο.
\par 7 Και εβόησεν ο βασιλεύς μεγαλοφώνως να εισάξωσι τους επαοιδούς, τους Χαλδαίους και τους μάντεις. Τότε ο βασιλεύς ελάλησε και είπε προς τους σοφούς της Βαβυλώνος, Όστις αναγνώση την γραφήν ταύτην και μοι δείξη την ερμηνείαν αυτής, θέλει ενδυθή πορφύραν, και η άλυσος η χρυσή θέλει τεθή περί τον τράχηλον αυτού και θέλει είσθαι ο τρίτος άρχων του βασιλείου.
\par 8 Τότε εισήλθον πάντες οι σοφοί του βασιλέως· πλην δεν ηδύναντο να αναγνώσωσι την γραφήν ουδέ την ερμηνείαν αυτής να φανερώσωσι προς τον βασιλέα.
\par 9 Και ο βασιλεύς Βαλτάσαρ εταράχθη μεγάλως και ηλλοιώθη εν αυτώ η όψις αυτού και οι μεγιστάνες αυτού συνεταράχθησαν.
\par 10 Η βασίλισσα εκ των λόγων του βασιλέως και των μεγιστάνων αυτού εισήλθεν εις τον οίκον του συμποσίου· και ελάλησεν η βασίλισσα και είπε, Βασιλεύ, ζήθι εις τον αιώνα· μη σε ταράττωσιν οι διαλογισμοί σου και η όψις σου ας μη αλλοιούται.
\par 11 Υπάρχει άνθρωπος εν τω βασιλείω σου, εις τον οποίον είναι το πνεύμα των αγίων θεών· και εν ταις ημέραις του πατρός σου φως και σύνεσις και σοφία, ως η σοφία των θεών, ευρέθησαν εν αυτώ, τον οποίον ο βασιλεύς Ναβουχοδονόσορ ο πατήρ σου, ο βασιλεύς ο πατήρ σου, κατέστησεν άρχοντα των μάγων, των επαοιδών, των Χαλδαίων και των μάντεων.
\par 12 Διότι πνεύμα έξοχον και γνώσις και σύνεσις, ερμηνεία ενυπνίων και εξήγησις αινιγμάτων και λύσις αποριών, ευρέθησαν εν αυτώ τω Δανιήλ, τον οποίον ο βασιλεύς μετωνόμασε Βαλτασάσαρ· τώρα λοιπόν ας προσκληθή ο Δανιήλ, και θέλει δείξει την ερμηνείαν.
\par 13 Τότε εισήχθη ο Δανιήλ έμπροσθεν του βασιλέως. Και ο βασιλεύς ελάλησε και είπε προς τον Δανιήλ, Συ είσαι ο Δανιήλ εκείνος, όστις είσαι εκ των υιών της αιχμαλωσίας του Ιούδα, τους οποίους έφερεν εκ της Ιουδαίας ο βασιλεύς ο πατήρ μου;
\par 14 Ήκουσα τωόντι περί σου, ότι το πνεύμα των θεών είναι εν σοι και φως και σύνεσις και σοφία έξοχος ευρέθησαν εν σοι.
\par 15 Και τώρα εισήλθον έμπροσθέν μου οι σοφοί και οι επαοιδοί, διά να αναγνώσωσι την γραφήν ταύτην και να φανερώσωσιν εις εμέ την ερμηνείαν αυτής, πλην δεν ηδυνήθησαν να δείξωσι του πράγματος την ερμηνείαν.
\par 16 Και εγώ ήκουσα περί σου, ότι δύνασαι να ερμηνεύης και να λύης απορίας· τώρα λοιπόν, εάν δυνηθής να αναγνώσης την γραφήν και να φανερώσης προς εμέ την ερμηνείαν αυτής θέλεις ενδυθή πορφύραν και η άλυσος η χρυσή θέλει τεθή περί τον τράχηλόν σου και θέλεις είσθαι ο τρίτος άρχων του βασιλείου.
\par 17 Τότε ο Δανιήλ απεκρίθη και είπεν έμπροσθεν του βασιλέως, Τα δώρα σου ας ήναι εν σοι και δος εις άλλον τας αμοιβάς σου· εγώ δε θέλω αναγνώσει την γραφήν εις τον βασιλέα και θέλω φανερώσει την ερμηνείαν προς αυτόν.
\par 18 Βασιλεύ, ο Θεός ο ύψιστος έδωκεν εις τον Ναβουχοδονόσορ τον πατέρα σου βασιλείαν και μεγαλειότητα και δόξαν και τιμήν.
\par 19 Και διά την μεγαλειότητα, την οποίαν έδωκεν εις αυτόν, πάντες οι λαοί, έθνη και γλώσσαι έτρεμον και εφοβούντο έμπροσθεν αυτού· όντινα ήθελεν εφόνευε και όντινα ήθελεν εφύλαττε ζώντα και όντινα ήθελεν ύψωνε και όντινα ήθελεν εταπείνονεν·
\par 20 αλλ' ότε η καρδία αυτού επήρθη και ο νούς αυτού εσκληρύνθη εν τη υπερηφανία, κατεβιβάσθη από του βασιλικού θρόνου αυτού και αφηρέθη η δόξα αυτού απ' αυτού·
\par 21 και εξεδιώχθη εκ των υιών των ανθρώπων, και η καρδία αυτού έγεινεν ως των θηρίων και η κατοικία αυτού ήτο μετά των αγρίων όνων· με χόρτον ως οι βόες ετρέφετο και το σώμα αυτού εβρέχετο υπό της δρόσου του ουρανού· εωσού εγνώρισεν ότι ο Θεός ο ύψιστος είναι Κύριος της βασιλείας των ανθρώπων και όντινα θέλει, στήνει επ' αυτήν.
\par 22 Και συ ο υιός αυτού, ο Βαλτάσαρ, δεν εταπείνωσας την καρδίαν σου, ενώ εγνώριζες πάντα ταύτα·
\par 23 αλλ' υψώθης εναντίον του Κυρίου του ουρανού· και τα σκεύη του οίκου αυτού έφεραν έμπροσθέν σου, και επίνετε οίνον εξ αυτών και συ και οι μεγιστάνές σου, αι γυναίκές σου και αι παλλακαί σου· και εδοξολόγησας τους θεούς τους αργυρούς και τους χρυσούς, τους χαλκούς, τους σιδηρούς, τους ξυλίνους και τους λιθίνους, οίτινες δεν βλέπουσιν ουδέ ακούουσιν ουδέ νοούσι· τον δε Θεόν, εις του οποίου την χείρα είναι η πνοή σου και εις την εξουσίαν αυτού πάσαι αι οδοί σου, δεν εδόξασας.
\par 24 Διά τούτο εστάλη απ' έμπροσθεν αυτού η παλάμη της χειρός και ενεχαράχθη η γραφή αύτη.
\par 25 Και αύτη είναι η γραφή ήτις ενεχαράχθη· Μενέ, Μενέ, Θεκέλ, Ο υ φ α ρ σ ί ν.
\par 26 Αύτη είναι η ερμηνεία του πράγματος· Μενέ, εμέτρησεν ο Θεός την βασιλείαν σου και ετελείωσεν αυτήν·
\par 27 Θεκέλ, εζυγίσθης εν τη πλάστιγγι και ευρέθης ελλιπής·
\par 28 Φερές, διηρέθη η βασιλεία σου και εδόθη εις τους Μήδους και Πέρσας.
\par 29 Τότε προσέταξεν ο Βαλτάσαρ και ενέδυσαν τον Δανιήλ την πορφύραν, και περιέθηκαν την άλυσον την χρυσήν περί τον τράχηλον αυτού, και διεκήρυξαν περί αυτού, να ήναι ο τρίτος άρχων του βασιλείου.
\par 30 Την αυτήν νύκτα εφονεύθη ο Βαλτάσαρ ο βασιλεύς των Χαλδαίων.
\par 31 Και Δαρείος ο Μήδος έλαβε την βασιλείαν, ων περίπου ετών εξήκοντα δύο.

\chapter{6}

\par 1 Αρεστόν εφάνη εις τον Δαρείον να καταστήση επί του βασιλείου εκατόν είκοσι σατράπας, διά να ήναι εφ' όλου του βασιλείου·
\par 2 και επ' αυτούς τρεις προέδρους, εις των οποίων ήτο ο Δανιήλ, διά ν' αποδίδωσι λόγον εις αυτούς οι σατράπαι ούτοι, και ο βασιλεύς να μη ζημιόνηται.
\par 3 Τότε ο Δανιήλ ούτος προετιμήθη υπέρ τους προέδρους και σατράπας, διότι πνεύμα έξοχον ήτο εν αυτώ· και ο βασιλεύς εστοχάσθη να καταστήση αυτόν εφ' όλου του βασιλείου.
\par 4 Οι δε πρόεδροι και οι σατράπαι εζήτουν να εύρωσι πρόφασιν κατά του Δανιήλ εκ των υποθέσεων της βασιλείας· πλην δεν ηδύναντο να εύρωσιν ουδεμίαν πρόφασιν ουδέ αμάρτημα· διότι ήτο πιστός, και δεν ευρέθη εν αυτώ ουδέν σφάλμα ουδέ αμάρτημα.
\par 5 Και είπον οι άνθρωποι ούτοι, δεν θέλομεν ευρεί πρόφασιν κατά του Δανιήλ τούτου, εκτός εάν εύρωμέν τι εναντίον αυτού εκ του νόμου του Θεού αυτού.
\par 6 Τότε οι πρόεδροι και οι σατράπαι ούτοι συνήχθησαν εις τον βασιλέα και είπον ούτω προς αυτόν· Βασιλεύ Δαρείε, ζήθι εις τον αιώνα.
\par 7 Πάντες οι πρόεδροι του βασιλείου, οι διοικηταί και οι σατράπαι, οι αυλικοί και οι τοπάρχαι, συνεβουλεύθησαν να εκδοθή βασιλικόν ψήφισμα και να στηριχθή απαγόρευσις, ότι όστις κάμη αίτησίν τινά παρ' οποιουδήποτε θεού ή ανθρώπου, έως τριάκοντα ημερών, εκτός παρά σου, βασιλεύ, ούτος να ριφθή εις τον λάκκον των λεόντων·
\par 8 τώρα λοιπόν, βασιλεύ, κάμε την απαγόρευσιν και υπόγραψον το ψήφισμα, διά να μη αλλαχθή, κατά τον νόμον των Μήδων και Περσών, όστις δεν ακυρούται.
\par 9 Όθεν ο βασιλεύς Δαρείος υπέγραψε την γραφήν και την απαγόρευσιν.
\par 10 Και ο Δανιήλ, καθώς έμαθεν ότι υπεγράφη η γραφή, εισήλθεν εις τον οίκον αυτού· και έχων τας θυρίδας του κοιτώνος αυτού ανεωγμένας προς την Ιερουσαλήμ, έπιπτεν επί τα γόνατα αυτού τρίς της ημέρας, προσευχόμενος και δοξολογών ενώπιον του Θεού αυτού, καθώς έκαμνε πρότερον.
\par 11 Τότε οι άνθρωποι εκείνοι συνήχθησαν και εύρηκαν τον Δανιήλ κάμνοντα αίτησιν και ικετεύοντα ενώπιον του Θεού αυτού.
\par 12 Όθεν προσελθόντες ελάλησαν έμπροσθεν του βασιλέως περί της βασιλικής απαγορεύσεως λέγοντες, Δεν υπέγραψας απόφασιν, ότι πας άνθρωπος, όστις κάμη αίτησιν παρ' οποιουδήποτε θεού ή ανθρώπου, έως τριάκοντα ημερών, εκτός παρά σου, βασιλεύ, θέλει ριφθή εις τον λάκκον των λεόντων; Ο βασιλεύς απεκρίθη και είπεν, Αληθινός είναι ο λόγος, κατά τον νόμον των Μήδων και Περσών, όστις δεν ακυρούται.
\par 13 Τότε απεκρίθησαν και είπον έμπροσθεν του βασιλέως, Ο Δανιήλ εκείνος, ο εκ των υιών της αιχμαλωσίας του Ιούδα, δεν σε σέβεται, βασιλεύ, ουδέ την απόφασιν την οποίαν υπέγραψας, αλλά κάμνει την δέησιν αυτού τρίς της ημέρας.
\par 14 Τότε ο βασιλεύς, ως ήκουσε τους λόγους, ελυπήθη πολύ επ' αυτώ και εφρόντιζεν εγκαρδίως περί του Δανιήλ να ελευθερώση αυτόν, και ηγωνίζετο μέχρι της δύσεως του ηλίου διά να λυτρώση αυτόν.
\par 15 Τότε οι άνθρωποι εκείνοι συνήχθησαν εις τον βασιλέα και είπον προς αυτόν, Έξευρε, βασιλεύ, ότι ο νόμος των Μήδων και Περσών είναι, ουδεμία απαγόρευσις ούτε διαταγή, την οποίαν ο βασιλεύς κάμη, να ακυρούται.
\par 16 Τότε ο βασιλεύς προσέταξε και έφεραν τον Δανιήλ και έρριψαν αυτόν εις τον λάκκον των λεόντων. Ελάλησε δε ο βασιλεύς και είπε προς τον Δανιήλ, Ο Θεός σου, τον οποίον συ λατρεύεις ακαταπαύστως, αυτός θέλει σε ελευθερώσει.
\par 17 Και εφέρθη εις λίθος και επετέθη επί το στόμα του λάκκου, και ο βασιλεύς εσφράγισεν αυτόν διά της ιδίας αυτού σφραγίδος και διά της σφραγίδος των μεγιστάνων αυτού, διά να μη αλλοιωθή μηδέν περί του Δανιήλ.
\par 18 Τότε ο βασιλεύς υπήγεν εις το παλάτιον αυτού και διενυκτέρευσε νηστικός και δεν εφέρθησαν έμπροσθεν αυτού όργανα μουσικά, και ο ύπνος αυτού έφυγεν απ' αυτού.
\par 19 Εξηγέρθη δε ο βασιλεύς πολλά ενωρίς το πρωΐ και υπήγε μετά σπουδής εις τον λάκκον των λεόντων.
\par 20 Και ότε ήλθεν εις τον λάκκον, εφώνησε μετά φωνής κλαυθμηράς προς τον Δανιήλ· και ελάλησεν ο βασιλεύς και είπε προς τον Δανιήλ, Δανιήλ, δούλε του Θεού του ζώντος, ο Θεός σου, τον οποίον συ λατρεύεις ακαταπαύστως, ηδυνήθη να σε ελευθερώση εκ των λεόντων;
\par 21 Τότε ελάλησεν ο Δανιήλ προς τον βασιλέα, Βασιλεύ, ζήθι εις τον αιώνα.
\par 22 Ο Θεός μου απέστειλε τον άγγελον αυτού και έφραξε τα στόματα των λεόντων και δεν με έβλαψαν, διότι αθωότης ευρέθη εν εμοί ενώπιον αυτού, και έτι ενώπιόν σου, βασιλεύ, πταίσμα δεν έπραξα.
\par 23 Τότε ο βασιλεύς μεγάλως εχάρη επ' αυτώ και προσέταξε να αναβιβάσωσι τον Δανιήλ εκ του λάκκου. Και ανεβιβάσθη ο Δανιήλ εκ του λάκκου και ουδεμία βλάβη ηυρέθη εν αυτώ, διότι είχε πίστιν εις τον Θεόν αυτού.
\par 24 Τότε ο βασιλεύς προσέταξε και έφεραν τους ανθρώπους εκείνους, οίτινες διέβαλον τον Δανιήλ, και έρριψαν εις τον λάκκον των λεόντων αυτούς, τα τέκνα αυτών και τας γυναίκας αυτών· και πριν φθάσωσιν εις το βάθος του λάκκου, οι λέοντες συνήρπασαν αυτούς και κατεσύντριψαν πάντα τα οστά αυτών.
\par 25 Τότε έγραψε Δαρείος ο βασιλεύς προς πάντας τους λαούς, έθνη και γλώσσας, τους κατοικούντας επί πάσης της γης, Ειρήνη ας πληθυνθή εις εσάς.
\par 26 Διαταγή εξεδόθη παρ' εμού, εν όλω τω κράτει της βασιλείας μου, να τρέμωσιν οι άνθρωποι και να φοβώνται ενώπιον του Θεού του Δανιήλ· διότι αυτός είναι Θεός ζων και διαμένων εις τον αιώνα, και η βασιλεία αυτού δεν θέλει φθαρή και η εξουσία αυτού θέλει είσθαι μέχρι τέλους·
\par 27 αυτός ο ελευθερωτής και σωτήρ και ποιών σημεία και τεράστια εν τω ουρανώ και επί της γης, όστις ηλευθέρωσε τον Δανιήλ εκ της δυνάμεως των λεόντων.
\par 28 Και ευημέρησεν αυτός ο Δανιήλ εν τη βασιλεία του Δαρείου και εν τη βασιλεία Κύρου του Πέρσου.

\chapter{7}

\par 1 Εν τω πρώτω έτει του Βαλτάσαρ βασιλέως της Βαβυλώνος ο Δανιήλ είδεν ενύπνιον και οράσεις της κεφαλής αυτού επί της κλίνης αυτού· τότε έγραψε το ενύπνιον και διηγήθη το κεφάλαιον των λόγων.
\par 2 Ο Δανιήλ ελάλησε και είπεν, Εγώ εθεώρουν εν τω οράματί μου την νύκτα και ιδού οι τέσσαρες άνεμοι του ουρανού συνεφώρμησαν επί την θάλασσαν την μεγάλην.
\par 3 Και τέσσαρα θηρία μεγάλα ανέβησαν εκ της θαλάσσης, διαφέροντα απ' αλλήλων.
\par 4 Το πρώτον ήτο ως λέων και είχε πτέρυγας αετού· εθεώρουν εωσού απεσπάσθησαν αι πτέρυγες αυτού, και εσηκώθη από της γης και εστάθη επί τους πόδας ως άνθρωπος, και καρδία ανθρώπου εδόθη εις αυτό.
\par 5 Και ιδού, έπειτα θηρίον δεύτερον όμοιον με άρκτον, και εσηκώθη κατά το εν πλάγιον, και είχε τρεις πλευράς εν τω στόματι αυτού μεταξύ των οδόντων αυτού· και έλεγον ούτω προς αυτό· Σηκώθητι, κατάφαγε σάρκας πολλάς.
\par 6 Μετά τούτο εθεώρουν και ιδού, έτερον ως λεοπάρδαλις, έχον επί τα νώτα αυτού τέσσαρας πτέρυγας πτηνού· το θηρίον είχεν έτι τέσσαρας κεφαλάς· και εδόθη εξουσία εις αυτό.
\par 7 Μετά τούτο είδον εν τοις οράμασι της νυκτός και ιδού, θηρίον τέταρτον, τρομερόν και καταπληκτικόν και ισχυρόν σφόδρα· και είχε μεγάλους σιδηρούς οδόντας· κατέτρωγε και κατεσύντριβε και κατεπάτει το υπόλοιπον με τους πόδας αυτού· και αυτό ήτο διάφορον πάντων των θηρίων των προ αυτού· και είχε δέκα κέρατα.
\par 8 Παρετήρουν τα κέρατα και ιδού, έτερον μικρόν κέρας ανέβη μεταξύ αυτών, έμπροσθεν του οποίου τρία εκ των πρώτων κεράτων εξερριζώθησαν· και ιδού, εν τω κέρατι τούτω ήσαν οφθαλμοί ως οφθαλμοί ανθρώπου και στόμα λαλούν πράγματα μεγάλα.
\par 9 Εθεώρουν έως ότου οι θρόνοι ετέθησαν και ο Παλαιός των ημερών εκάθησε, του οποίου το ένδυμα ήτο λευκόν ως χιών και αι τρίχες της κεφαλής αυτού ως μαλλίον καθαρόν· ο θρόνος αυτού ήτο ως φλόξ πυρός, οι τροχοί αυτού ως πυρ καταφλέγον.
\par 10 Ποταμός πυρός εξήρχετο και διεχέετο απ' έμπροσθεν αυτού· χίλιαι χιλιάδες υπηρέτουν εις αυτόν και μύριαι μυριάδες παρίσταντο ενώπιον αυτού· το κριτήριον εκάθησε και τα βιβλία ανεώχθησαν.
\par 11 Εθεώρουν τότε εξ αιτίας της φωνής των μεγάλων λόγων, τους οποίους το κέρας ελάλει, εθεώρουν εωσού εθανατώθη το θηρίον και το σώμα αυτού απωλέσθη και εδόθη εις καύσιν πυρός.
\par 12 Περί δε των λοιπών θηρίων, η εξουσία αυτών αφηρέθη· πλην παράτασις ζωής εδόθη εις αυτά έως καιρού και χρόνου.
\par 13 Είδον εν οράμασι νυκτός και ιδού, ως Υιός ανθρώπου ήρχετο μετά των νεφελών του ουρανού και έφθασεν έως του Παλαιού των ημερών και εισήγαγον αυτόν ενώπιον αυτού.
\par 14 Και εις αυτόν εδόθη η εξουσία και η δόξα και η βασιλεία, διά να λατρεύωσιν αυτόν πάντες οι λαοί, τα έθνη και αι γλώσσαι· η εξουσία αυτού είναι εξουσία αιώνιος, ήτις δεν θέλει παρέλθει, και η βασιλεία αυτού, ήτις δεν θέλει φθαρή.
\par 15 Έφριξε το πνεύμα εμού του Δανιήλ εντός του σώματός μου και αι οράσεις της κεφαλής μου με ετάραττον.
\par 16 Επλησίασα εις ένα των παρισταμένων και εζήτουν να μάθω παρ' αυτού την αλήθειαν πάντων τούτων. Και ελάλησε προς εμέ και μοι εφανέρωσε την ερμηνείαν των πραγμάτων.
\par 17 Ταύτα τα μεγάλα θηρία, τα οποία είναι τέσσαρα, είναι τέσσαρες βασιλείς, οίτινες θέλουσιν εγερθή εκ της γης.
\par 18 Αλλ' οι άγιοι του Υψίστου θέλουσι παραλάβει την βασιλείαν και θέλουσιν έχει το βασίλειον εις τον αιώνα και εις τον αιώνα του αιώνος.
\par 19 Τότε ήθελον να μάθω την αλήθειαν περί του τετάρτου θηρίου, το οποίον ήτο διάφορον από πάντων των άλλων, καθ' υπερβολήν τρομερόν, του οποίου οι οδόντες ήσαν σιδηροί και οι όνυχες αυτού χάλκινοι· κατέτρωγε, κατεσύντριβε και κατεπάτει το υπόλοιπον με τους πόδας αυτού·
\par 20 και περί των δέκα κεράτων, τα οποία ήσαν εν τη κεφαλή αυτού και περί του άλλου, το οποίον ανέβη και έμπροσθεν του οποίου έπεσον τρία· περί του κέρατος λέγω εκείνου, το οποίον είχεν οφθαλμούς και στόμα λαλούν μεγάλα πράγματα, του οποίου η όψις ήτο ρωμαλεωτέρα παρά των συντρόφων αυτού.
\par 21 Εθεώρουν, και το κέρας εκείνο έκαμνε πόλεμον μετά των αγίων και υπερίσχυε κατ' αυτών·
\par 22 εωσού ήλθεν ο Παλαιός των ημερών και εδόθη η κρίσις εις τους αγίους του Υψίστου· και ο καιρός έφθασε και οι άγιοι έλαβον την βασιλείαν.
\par 23 Και εκείνος είπε, το θηρίον το τέταρτον θέλει είσθαι η τετάρτη βασιλεία επί της γης, ήτις θέλει διαφέρει από πασών των βασιλειών, και θέλει καταφάγει πάσαν την γην και θέλει καταπατήσει αυτήν και κατασυντρίψει αυτήν.
\par 24 Και τα δέκα κέρατα είναι δέκα βασιλείς, οίτινες θέλουσιν εγερθή εκ της βασιλείας ταύτης· και κατόπιν αυτών άλλος θέλει εγερθή· και αυτός θέλει διαφέρει των πρώτων και θέλει υποτάξει τρεις βασιλείς.
\par 25 Και θέλει λαλήσει λόγους εναντίον του Υψίστου, και θέλει κατατρέχει τους αγίους του Υψίστου, και θέλει διανοηθή να μεταβάλλη καιρούς και νόμους· και θέλουσι δοθή εις την χείρα αυτού μέχρι καιρού και καιρών και ημίσεος καιρού.
\par 26 Κριτήριον όμως θέλει καθήσει, και θέλει αφαιρεθή η εξουσία αυτού, διά να φθαρή και να αφανισθή έως τέλους.
\par 27 Και η βασιλεία και η εξουσία και η μεγαλωσύνη των βασιλειών των υποκάτω παντός του ουρανού θέλει δοθή εις τον λαόν των αγίων του Υψίστου, του οποίου η βασιλεία είναι βασιλεία αιώνιος, και πάσαι αι εξουσίαι θέλουσι λατρεύσει και υπακούσει εις αυτόν.
\par 28 Έως ενταύθα είναι το τέλος του πράγματος. Όσον δι' εμέ τον Δανιήλ, οι διαλογισμοί μου πολύ με ετάραττον και η όψις μου ηλλοιώθη εν εμοί· πλην συνετήρησα το πράγμα εν τη καρδία μου.

\chapter{8}

\par 1 Εν τω τρίτω έτει της βασιλείας του βασιλέως Βαλτάσαρ όρασις εφάνη εις εμέ, εις εμέ τον Δανιήλ, μετά την εις εμέ φανείσαν πρότερον.
\par 2 Και είδον εν τη οράσει· και ότε είδον, ήμην εν Σούσοις τη βασιλευούση τη εν τη επαρχία Ελάμ· και είδον εν τη οράσει και εγώ ήμην πλησίον του ποταμού Ουλαΐ.
\par 3 Και εσήκωσα τους οφθαλμούς μου και είδον και ιδού, ίστατο έμπροσθεν του ποταμού κριός εις έχων κέρατα· και τα κέρατα ήσαν υψηλά, το εν όμως υψηλότερον του άλλου· και το υψηλότερον εξεφύτρωσεν ύστερον.
\par 4 Είδον τον κριόν κερατίζοντα προς δύσιν και προς βορράν και προς νότον· και ουδέν θηρίον ηδύνατο να σταθή έμπροσθεν αυτού και δεν υπήρχεν ο ελευθερών εκ της χειρός αυτού· αλλ' έκαμνε κατά την θέλησιν αυτού και εμεγαλύνθη.
\par 5 Ενώ δε εγώ εσκεπτόμην, ιδού, τράγος ήρχετο από της δύσεως επί πρόσωπον πάσης της γης και δεν ήγγιζε το έδαφος· και ο τράγος είχε κέρας περίβλεπτον μεταξύ των οφθαλμών αυτού.
\par 6 Και ήλθεν έως του κριού του έχοντος τα δύο κέρατα, τον οποίον είδον ιστάμενον έμπροσθεν του ποταμού, και έδραμε προς αυτόν εν τη ορμή της δυνάμεως αυτού.
\par 7 Και είδον αυτόν ότι επλησίασεν εις τον κριόν και εξηγριώθη κατ' αυτού και εκτύπησε τον κριόν και συνέτριψε τα δύο κέρατα αυτού· και δεν ήτο δύναμις εν τω κριώ να σταθή έμπροσθεν αυτού, αλλ' έρριψεν αυτόν κατά γης και κατεπάτησεν αυτόν· και δεν υπήρχεν ο ελευθερών τον κριόν εκ της χειρός αυτού.
\par 8 Διά τούτο ο τράγος εμεγαλύνθη σφόδρα· και ότε ενεδυναμώθη, συνετρίβη το κέρας το μέγα· και αντ' αυτού ανέβησαν τέσσαρα άλλα περίβλεπτα προς τους τέσσαρας ανέμους του ουρανού.
\par 9 Και εκ του ενός εξ αυτών εξήλθεν εν κέρας μικρόν, το οποίον εμεγαλύνθη καθ' υπερβολήν προς τον νότον και προς την ανατολήν και προς την γην της δόξης·
\par 10 και εμεγαλύνθη έως του στρατεύματος του ουρανού· και έρριψεν εις την γην μέρος εκ της στρατιάς και εκ των αστέρων και κατεπάτησεν αυτά·
\par 11 μάλιστα εμεγαλύνθη έως κατά του άρχοντος του στρατεύματος· και αφήρεσεν απ' αυτού την παντοτεινήν θυσίαν, και το άγιον κατοικητήριον αυτού κατεβλήθη·
\par 12 και το στράτευμα παρεδόθη εις αυτόν μετά της παντοτεινής θυσίας εξ αιτίας της παραβάσεως, και έρριψε κατά γης την αλήθειαν· και έπραξε και ευωδώθη.
\par 13 Τότε ήκουσα αγίου τινός λαλούντος· και άλλος άγιος έλεγε προς τον δείνα λαλούντα, Έως πότε θέλει διαρκεί όρασις η περί της παντοτεινής θυσίας και της παραβάσεως, ήτις φέρει την ερήμωσιν, και το αγιαστήριον και το στράτευμα παραδίδονται εις καταπάτησιν;
\par 14 Και είπε προς εμέ, Έως δύο χιλιάδων και τριακοσίων ημερονυκτίων· τότε το αγιαστήριον θέλει καθαρισθή.
\par 15 Και ότε εγώ ο Δανιήλ είδον την όρασιν και εζήτουν την έννοιαν, τότε ιδού, εστάθη έμπροσθέν μου ως θέα ανθρώπου·
\par 16 και ήκουσα φωνήν ανθρώπου εν μέσω του Ουλαΐ, ήτις έκραξε και είπε, Γαβριήλ, κάμε τον άνθρωπον τούτον να εννοήση την όρασιν.
\par 17 Και ήλθε πλησίον όπου ιστάμην· και ότε ήλθεν, ετρόμαξα και έπεσον επί πρόσωπόν μου· ο δε είπε προς εμέ, Εννόησον, υιέ ανθρώπου· διότι η όρασις είναι διά τους εσχάτους καιρούς.
\par 18 Και ενώ ελάλει προς εμέ, εγώ ήμην βεβυθισμένος εις βαθύν ύπνον με το πρόσωπόν μου επί την γήν· πλην με ήγγισε και με έκαμε να σταθώ όρθιος.
\par 19 Και είπεν, Ιδού, εγώ θέλω σε κάμει να γνωρίσης τι θέλει συμβή εν τοις εσχάτοις της οργής· διότι εν τω ωρισμένω καιρώ θέλει είσθαι το τέλος.
\par 20 Ο κριός, τον οποίον είδες, ο έχων τα δύο κέρατα, είναι οι βασιλείς της Μηδίας και της Περσίας.
\par 21 Και ο τριχωτός τράγος είναι ο βασιλεύς της Ελλάδος· και το κέρας το μέγα, το μεταξύ των οφθαλμών αυτού, αυτός είναι ο πρώτος βασιλεύς.
\par 22 Το δε ότι συνετρίβη και ανέβησαν τέσσαρα αντ' αυτού, δηλοί ότι τέσσαρα βασίλεια θέλουσιν εγερθή εκ του έθνους τούτου· πλην ουχί κατά την δύναμιν αυτού.
\par 23 Και εν τοις εσχάτοις καιροίς της βασιλείας αυτών, όταν αι ανομίαι φθάσωσιν εις το πλήρες, θέλει εγερθή βασιλεύς σκληροπρόσωπος και συνετός εις πανουργίας.
\par 24 Και η δύναμις αυτού θέλει είσθαι ισχυρά, ουχί όμως εξ ιδίας αυτού δυνάμεως· και θέλει αφανίζει εξαισίως και θέλει ευοδούσθαι και κατορθόνει και θέλει αφανίζει τους ισχυρούς και τον λαόν τον άγιον.
\par 25 Και διά της πανουργίας αυτού θέλει κάμει να ευοδούται η απάτη εν τη χειρί αυτού· και θέλει μεγαλυνθή εν τη καρδία αυτού και εν ειρήνη θέλει αφανίσει πολλούς· και θέλει σηκωθή κατά του Άρχοντος των αρχόντων· πλην θέλει συντριφθή άνευ χειρός.
\par 26 Και η ρηθείσα όρασις περί των ημερονυκτίων είναι αληθής· συ λοιπόν σφράγισον την όρασιν, διότι είναι διά ημέρας πολλάς.
\par 27 Και εγώ ο Δανιήλ ελιποθύμησα και ήμην ασθενής ημέρας τινάς· μετά ταύτα εσηκώθην και έκαμνον τα έργα του βασιλέως· εθαύμαζον δε διά την όρασιν και δεν υπήρχεν ο εννοών.

\chapter{9}

\par 1 Εν τω πρώτω έτει του Δαρείου, του υιού του Ασσουήρου, εκ του σπέρματος των Μήδων, όστις εβασίλευσεν επί το βασίλειον των Χαλδαίων,
\par 2 εν τω πρώτω έτει της βασιλείας αυτού, εγώ ο Δανιήλ ενόησα εν τοις βιβλίοις τον αριθμόν των ετών, περί των οποίων ο λόγος του Κυρίου έγεινε προς Ιερεμίαν τον προφήτην, ότι ήθελον συμπληρωθή εβδομήκοντα έτη εις τας ερημώσεις της Ιερουσαλήμ.
\par 3 Και έστρεψα το πρόσωπόν μου προς Κύριον τον Θεόν, διά να κάμω προσευχήν και δεήσεις εν νηστεία και σάκκω και σποδώ·
\par 4 και εδεήθην προς Κύριον τον Θεόν μου και εξωμολογήθην και είπον, Ω Κύριε, ο μέγας και φοβερός Θεός, ο φυλάττων την διαθήκην και το έλεος προς τους αγαπώντας αυτόν και τηρούντας τας εντολάς αυτού·
\par 5 ημαρτήσαμεν και ηνομήσαμεν και ησεβήσαμεν και απεστατήσαμεν και εξεκλίναμεν από των εντολών σου και από των κρίσεών σου.
\par 6 Και δεν υπηκούσαμεν εις τους δούλους σου τους προφήτας, οίτινες ελάλουν εν τω ονόματί σου προς τους βασιλείς ημών, τους άρχοντας ημών και τους πατέρας ημών, και προς πάντα τον λαόν της γης.
\par 7 Εις σε, Κύριε, είναι η δικαιοσύνη, εις ημάς δε η αισχύνη του προσώπου, ως εν τη ημέρα ταύτη, εις τους άνδρας του Ιούδα και εις τους κατοίκους της Ιερουσαλήμ και εις πάντα τον Ισραήλ, τους εγγύς και τους μακράν, κατά πάντας τους τόπους όπου εδίωξας αυτούς, διά την παράβασιν αυτών, την οποίαν παρέβησαν εις σε.
\par 8 Κύριε, εις ημάς είναι η αισχύνη του προσώπου, εις τους βασιλείς ημών, τους άρχοντας ημών και εις τους πατέρας ημών, οίτινες ημαρτήσαμεν εις σε.
\par 9 Εις Κύριον τον Θεόν ημών είναι οι οικτιρμοί και αι αφέσεις· διότι απεστατήσαμεν απ' αυτού,
\par 10 και δεν υπηκούσαμεν εις την φωνήν Κυρίου του Θεού ημών, να περιπατώμεν εν τοις νόμοις αυτού, τους οποίους έθεσεν ενώπιον ημών διά των δούλων αυτού των προφητών.
\par 11 Και πας ο Ισραήλ παρέβη τον νόμον σου και εξέκλινε διά να μη υπακούη εις την φωνήν σου· διά τούτο εξεχύθη εφ' ημάς η κατάρα και ο όρκος ο γεγραμμένος εν τω νόμω του Μωϋσέως, δούλου του Θεού· διότι ημαρτήσαμεν εις αυτόν.
\par 12 Και εβεβαίωσε τους λόγους αυτού, τους οποίους ελάλησεν εναντίον ημών και εναντίον των κριτών ημών, οίτινες μας έκρινον, φέρων εφ' ημάς κακόν μέγα· διότι δεν έγεινεν υποκάτω παντός του ουρανού, ως έγεινεν εν Ιερουσαλήμ.
\par 13 Ως είναι γεγραμμένον εν τω νόμω Μωϋσέως, άπαν το κακόν τούτο ήλθεν εφ' ημάς· πλην δεν εδεήθημεν ενώπιον Κυρίου του Θεού ημών, διά να επιστρέψωμεν από των ανομιών ημών και να προσέξωμεν εις την αλήθειάν σου·
\par 14 διά τούτο ο Κύριος εγρηγόρησεν επί το κακόν και έφερεν αυτό εφ' ημάς· διότι δίκαιος είναι Κύριος ο Θεός ημών εν πάσι τοις έργοις αυτού, όσα πράττει· επειδή ημείς δεν υπηκούσαμεν εις την φωνήν αυτού.
\par 15 Και τώρα, Κύριε ο Θεός ημών, όστις εξήγαγες τον λαόν σου εκ γης Αιγύπτου εν χειρί κραταιά και έκαμες εις σεαυτόν όνομα, ως εν τη ημέρα ταύτη, ημαρτήσαμεν, ησεβήσαμεν.
\par 16 Κύριε, κατά πάσας τας δικαιοσύνας σου ας αποστραφή, δέομαι, ο θυμός σου και η οργή σου από της πόλεώς σου Ιερουσαλήμ, του όρους του αγίου σου· διότι διά τας αμαρτίας ημών και διά τας ανομίας των πατέρων ημών η Ιερουσαλήμ και ο λαός σου κατεστάθημεν όνειδος εις πάντας τους πέριξ ημών.
\par 17 Τώρα λοιπόν εισάκουσον, Θεέ ημών, την προσευχήν του δούλου σου και τας δεήσεις αυτού, και επίλαμψον το πρόσωπόν σου, ένεκεν του Κυρίου, επί το ηρημωμένον αγιαστήριόν σου.
\par 18 Κλίνον, Θεέ μου, το ωτίον σου και άκουσον· άνοιξον τους οφθαλμούς σου και ιδέ τας ερημώσεις ημών και την πόλιν, επί την οποίαν εκλήθη το όνομά σου· διότι ημείς δεν προσφέρομεν τας ικεσίας ημών ενώπιόν σου διά τας δικαιοσύνας ημών, αλλά διά τους πολλούς οικτιρμούς σου.
\par 19 Κύριε, εισάκουσον· Κύριε, συγχώρησον· Κύριε, ακροάσθητι και κάμε· μη χρονίσης, ένεκέν σου, Θεέ μου· διότι το όνομά σου εκλήθη επί την πόλιν σου και επί τον λαόν σου.
\par 20 Και ενώ εγώ ελάλουν έτι και προσηυχόμην και εξωμολογούμην την αμαρτίαν μου και την αμαρτίαν του λαού μου Ισραήλ, και προσέφερον την ικεσίαν μου ενώπιον Κυρίου του Θεού μου περί του όρους του αγίου του Θεού μου,
\par 21 και ενώ εγώ ελάλουν έτι εν τη προσευχή, ο ανήρ Γαβριήλ, τον οποίον είδον εν τη οράσει κατ' αρχάς, πετών ταχέως με ήγγισε περί την ώραν της εσπερινής θυσίας·
\par 22 και με συνέτισε και ελάλησε μετ' εμού και είπε, Δανιήλ, τώρα εξήλθον διά να σε κάμω να λάβης σύνεσιν.
\par 23 Εν τη αρχή των ικεσιών σου εξήλθεν η προσταγή και εγώ ήλθον να δείξω τούτο εις σέ· διότι είσαι σφόδρα αγαπητός· διά τούτο εννόησον τον λόγον και κατάλαβε την οπτασίαν.
\par 24 Εβδομήκοντα εβδομάδες διωρίσθησαν επί τον λαόν σου και επί την πόλιν την αγίαν σου, διά να συντελεσθή η παράβασις και να τελειώσωσιν αι αμαρτίαι, και να γείνη εξιλέωσις περί της ανομίας και να εισαχθή δικαιοσύνη αιώνιος και να σφραγισθή όρασις και προφητεία και να χρισθή ο Άγιος των αγίων.
\par 25 Γνώρισον λοιπόν και κατάλαβε ότι από της εξελεύσεως της προσταγής του να ανοικοδομηθή η Ιερουσαλήμ έως του Χριστού του ηγουμένου θέλουσιν είσθαι εβδομάδες επτά και εβδομάδες εξήκοντα δύο· θέλει οικοδομηθή πάλιν η πλατεία και το τείχος, μάλιστα εν καιροίς στενοχωρίας.
\par 26 Και μετά τας εξήκοντα δύο εβδομάδας θέλει εκκοπή ο Χριστός, πλην ουχί δι' εαυτόν· και ο λαός του ηγουμένου, όστις θέλει ελθεί, θέλει αφανίσει την πόλιν και το αγιαστήριον· και το τέλος αυτής θέλει ελθεί μετά κατακλυσμού, και έως του τέλους του πολέμου είναι διωρισμένοι αφανισμοί.
\par 27 Και θέλει στερεώσει την διαθήκην εις πολλούς εν μιά εβδομάδι· και εν τω ημίσει της εβδομάδος θέλει παύσει η θυσία και η προσφορά, και επί το πτερύγιον του Ιερού θέλει είσθαι το βδέλυγμα της ερημώσεως, και έως της συντελείας του καιρού θέλει δοθή διορία επί την ερήμωσιν.

\chapter{10}

\par 1 Εν τω τρίτω έτει του Κύρου, βασιλέως της Περσίας, απεκαλύφθη λόγος εις τον Δανιήλ, του οποίου το όνομα εκλήθη Βαλτασάσαρ· και ο λόγος ήτο αληθινός και η δύναμις των λεγομένων μεγάλη· και κατέλαβε τον λόγον και εννόησε την οπτασίαν.
\par 2 Εν ταις ημέραις εκείναις εγώ ο Δανιήλ ήμην πενθών τρεις ολοκλήρους εβδομάδας.
\par 3 Άρτον επιθυμητόν δεν έφαγον και κρέας και οίνος δεν εισήλθεν εις το στόμα μου ουδέ ήλειψα εμαυτόν παντελώς, μέχρι συμπληρώσεως τριών ολοκλήρων εβδομάδων.
\par 4 Και την εικοστήν τετάρτην ημέραν του πρώτου μηνός, ενώ ήμην παρά την όχθην του μεγάλου ποταμού, όστις είναι ο Τίγρις,
\par 5 εσήκωσα τους οφθαλμούς μου και είδον και ιδού, εις άνθρωπος ενδεδυμένος λινά και αι οσφύες αυτού ήσαν περιεζωσμέναι με χρυσίον καθαρόν του Ουφάζ,
\par 6 το δε σώμα αυτού ήτο ως βηρύλλιον, και το πρόσωπον αυτού ως θέα αστραπής, και οι οφθαλμοί αυτού ως λαμπάδες πυρός, και οι βραχίονες αυτού και οι πόδες αυτού ως όψις χαλκού στίλβοντος, και η φωνή των λόγων αυτού ως φωνή όχλου.
\par 7 Και μόνος εγώ ο Δανιήλ είδον την όρασιν· οι δε άνδρες οι όντες μετ' εμού δεν είδον την όρασιν· αλλά τρόμος μέγας επέπεσεν επ' αυτούς και έφυγον διά να κρυφθώσιν.
\par 8 Εγώ λοιπόν έμεινα μόνος και είδον την όρασιν την μεγάλην ταύτην, και δεν απέμεινεν ισχύς εν εμοί· και η ακμή μου μετεστράφη εν εμοί εις μαρασμόν και δεν έμεινεν ισχύς εν εμοί.
\par 9 Ήκουσα όμως την φωνήν των λόγων αυτού· και ενώ ήκουον την φωνήν των λόγων αυτού, εγώ ήμην βεβυθισμένος εις βαθύν ύπνον επί πρόσωπόν μου και το πρόσωπόν μου επί την γην.
\par 10 Και ιδού, χειρ με ήγγισε και με ήγειρεν επί τα γόνατά μου και τας παλάμας των χειρών μου.
\par 11 Και είπε προς εμέ, Δανιήλ, ανήρ σφόδρα αγαπητέ, εννόησον τους λόγους, τους οποίους εγώ λαλώ προς σε, και στήθι ορθός· διότι προς σε απεστάλην τώρα. Και ότε ελάλησε προς εμέ τον λόγον τούτον, εσηκώθην έντρομος.
\par 12 Και είπε προς εμέ, Μη φοβού, Δανιήλ· διότι από της πρώτης ημέρας, καθ' ην έδωκας την καρδίαν σου εις το να εννοής και κακουχήσαι ενώπιον του Θεού σου, εισηκούσθησαν οι λόγοι σου και εγώ ήλθον εις τους λόγους σου.
\par 13 Πλην ο άρχων της βασιλείας της Περσίας ανθίστατο εις εμέ εικοσιμίαν ημέραν· αλλ' ιδού, ο Μιχαήλ, εις των πρώτων αρχόντων, ήλθε διά να μοι βοηθήση· και εγώ έμεινα εκεί πλησίον των βασιλέων της Περσίας.
\par 14 Και ήλθον να σε κάμω να καταλάβης τι θέλει συμβή εις τον λαόν σου εν ταις εσχάταις ημέραις· διότι η όρασις είναι έτι διά πολλάς ημέρας.
\par 15 Και ενώ ελάλει τοιούτους λόγους προς εμέ, έβαλον το πρόσωπόν μου προς την γην και έγεινα άφωνος.
\par 16 Και ιδού, ως θέα υιού ανθρώπου ήγγισε τα χείλη μου· τότε ήνοιξα το στόμα μου και ελάλησα και είπον προς τον ιστάμενον έμπροσθέν μου, Κύριέ μου, εξ αιτίας της οράσεως συνεστράφησαν τα εντόσθιά μου εν εμοί και δεν έμεινεν ισχύς εν εμοί.
\par 17 Και πως δύναται ο δούλος τούτου του κυρίου μου να λαλήση μετά του κυρίου μου τούτου; εν εμοί βεβαίως από του νυν δεν υπάρχει ουδεμία ισχύς αλλ' ουδέ πνοή έμεινεν εν εμοί.
\par 18 Και με ήγγισε πάλιν ως θέα ανθρώπου και με ενίσχυσε,
\par 19 και είπε, Μη φοβού, ανήρ σφόδρα αγαπητέ· ειρήνη εις σέ· ανδρίζου και ίσχυε. Και ενώ ελάλει προς εμέ, ενισχύθην και είπον, Ας λαλήση ο κύριός μου· διότι με ενίσχυσας.
\par 20 Και είπεν, Εξεύρεις διά τι ήλθον προς σε; τώρα δε θέλω επιστρέψει να πολεμήσω μετά του άρχοντος της Περσίας· και όταν εξέλθω, ιδού, ο άρχων της Ελλάδος θέλει ελθεί.
\par 21 Πλην θέλω σοι αναγγείλει το γεγραμμένον εν τη γραφή της αληθείας· και δεν είναι ουδείς ο αγωνιζόμενος μετ' εμού υπέρ τούτων, ειμή Μιχαήλ ο άρχων υμών.

\chapter{11}

\par 1 Και εγώ εν τω πρώτω έτει Δαρείου του Μήδου ιστάμην διά να κραταιώσω και να ενδυναμώσω αυτόν.
\par 2 Και τώρα θέλω σοι αναγγείλει την αλήθειαν. Ιδού, ότι τρεις βασιλείς θέλουσιν εγερθή εν τη Περσία· και ο τέταρτος θέλει είσθαι πολύ πλουσιώτερος παρά πάντας· και αφού κραταιωθή εν τω πλούτω αυτού, θέλει διεγείρει το παν εναντίον του βασιλείου της Ελλάδος.
\par 3 Και θέλει σηκωθή βασιλεύς δυνατός και θέλει εξουσιάζει εν δυνάμει μεγάλη και κάμει κατά την θέλησιν αυτού.
\par 4 Και καθώς σταθή, θέλει συντριφθή η βασιλεία αυτού και θέλει διαιρεθή εις τους τέσσαρας ανέμους του ουρανού· πλην ουχί εις τους απογόνους αυτού, ουδέ κατά την εξουσίαν αυτού, με την οποίαν εξουσίασε· διότι η βασιλεία αυτού θέλει εκριζωθή και διαμερισθή εις άλλους, εκτός τούτων.
\par 5 Και ο βασιλεύς του νότου θέλει ισχύσει, και εις εκ των αρχόντων αυτού· και θέλει ισχύσει υπέρ αυτόν και θέλει εξουσιάσει· η εξουσία αυτού θέλει είσθαι εξουσία μεγάλη.
\par 6 Και μετά έτη θέλουσι συζευχθή· και η θυγάτηρ του βασιλέως του νότου θέλει ελθεί προς τον βασιλέα του βορρά, διά να κάμη συμφιλίωσιν· πλην αυτή δεν θέλει αναχαιτίσει την δύναμιν του βραχίονος ουδέ το σπέρμα αυτού θέλει σταθή· αλλά θέλει παραδοθή αυτή και οι φέροντες αυτήν και το γεννηθέν εξ αυτής και ο ενισχύων αυτήν εν καιροίς.
\par 7 Εκ του βλαστού όμως των ριζών αυτής θέλει σηκωθή τις αντ' αυτού, και ελθών μετά δυνάμεως θέλει εισέλθει εις τα οχυρώματα του βασιλέως του βορρά και θέλει ενεργήσει εναντίον αυτών και υπερισχύσει·
\par 8 και προσέτι θέλει φέρει αιχμαλώτους εις την Αίγυπτον τους θεούς αυτών, μετά των χωνευτών αυτών, μετά των πολυτίμων σκευών αυτών, των αργυρών και των χρυσών· και αυτός θέλει σταθή έτη τινά υπέρ τον βασιλέα του βορρά.
\par 9 Εκείνος δε θέλει εισέλθει εις το βασίλειον του βασιλέως του νότου, πλην θέλει επιστρέψει εις την γην αυτού.
\par 10 Οι δε υιοί αυτού θέλουσιν εγερθή εις πόλεμον και συνάξει πλήθος δυνάμεων πολλών· και εις εξ αυτών θέλει ελθεί εν ορμή και πλημμυρήσει και διαβή· και θέλει επανέλθει και εγερθή εις μάχην έως του οχυρώματος αυτού.
\par 11 Και ο βασιλεύς του νότου θέλει εξαγριωθή και θέλει εξέλθει και πολεμήσει μετ' αυτού, μετά του βασιλέως του βορρά· όστις θέλει παρατάξει πλήθος πολύ· το πλήθος όμως θέλει παραδοθή εις την χείρα αυτού.
\par 12 Και αφού πατάξη το πλήθος, η καρδία αυτού θέλει υψωθή· και θέλει καταβάλει μυριάδας, πλην δεν θέλει κραταιωθή.
\par 13 Και ο βασιλεύς του βορρά θέλει επιστρέψει και θέλει παρατάξει πλήθος περισσότερον παρά το πρώτον, και θέλει ελθεί εν ορμή εν τω τέλει των ωρισμένων ετών μετά δυνάμεως μεγάλης και μετά πλούτου πολλού.
\par 14 Και εν τοις καιροίς εκείνοις πολλοί θέλουσι σηκωθή εναντίον του βασιλέως του νότου· και οι λυμεώνες εκ του λαού σου θέλουσιν επαρθή διά να εκπληρώσωσιν όρασιν· πλην θέλουσι πέσει.
\par 15 Και ο βασιλεύς του βορρά θέλει ελθεί και υψώσει πρόχωμα και κυριεύσει τας οχυράς πόλεις· και οι βραχίονες του νότου δεν θέλουσιν αντισταθή ουδέ το πλήθος των εκλεκτών αυτού, και δεν θέλει είσθαι δύναμις προς αντίστασιν.
\par 16 Και ο ερχόμενος εναντίον αυτού θέλει κάμει κατά την θέλησιν αυτού και δεν θέλει είσθαι ο ανθιστάμενος εις αυτόν· και θέλει σταθή εν τη γη της δόξης, ήτις θέλει αναλωθή υπό των χειρών αυτού.
\par 17 Και θέλει στηρίξει το πρόσωπον αυτού εις το να εισέλθη μετά της δυνάμεως παντός του βασιλείου αυτού, και ευθύτης θέλει είσθαι μετ' αυτού· και θέλει ενεργήσει· και θέλει δώσει εις αυτόν θυγατέρα γυναικών, διαφθείρων αυτήν· πλην αυτή δεν θέλει σταθή ουδέ θέλει είσθαι υπέρ αυτού.
\par 18 Έπειτα θέλει στρέψει το πρόσωπον αυτού προς τας νήσους και θέλει κυριεύσει πολλάς· αλλ' ηγεμών τις θέλει παύσει το εξ αυτού όνειδος· εκτός τούτου θέλει επιστρέψει το όνειδος αυτού επ' αυτόν.
\par 19 Τότε θέλει στρέψει το πρόσωπον αυτού προς τα οχυρώματα της γης αυτού· πλην θέλει προσκόψει και πέσει και δεν θέλει ευρεθή.
\par 20 Και αντ' αυτού θέλει σηκωθή τύραννος, όστις θέλει κάμει να παρέλθη η δόξα του βασιλείου· πλην εν ολίγαις ημέραις θέλει αφανισθή και ουχί εν οργή ουδέ εν μάχη.
\par 21 Και αντ' αυτού θέλει σηκωθή εξουθενημένος τις, εις τον οποίον δεν θέλουσι δώσει τιμήν βασιλικήν· αλλά θέλει ελθεί ειρηνικώς και κυριεύσει το βασίλειον εν κολακείαις.
\par 22 Και οι βραχίονες του κατακλύζοντος θέλουσι κατακλυσθή έμπροσθεν αυτού και θέλουσι συντριφθή, και αυτός έτι ο άρχων της διαθήκης.
\par 23 Και μετά την συμμαχίαν την μετ' αυτού θέλει φέρεσθαι δολίως· διότι θέλει αναβή και υπερισχύσει μετά ολίγον λαού.
\par 24 Θέλει ελθεί μάλιστα ειρηνικώς επί τους παχυτέρους τόπους της επαρχίας, και θέλει κάμει ό,τι δεν έκαμον οι πατέρες αυτού ουδέ οι πατέρες των πατέρων αυτού· θέλει διαμοιράσει μεταξύ αυτών διάρπαγμα και λάφυρα και πλούτη, και θέλει μηχανευθή τας μηχανάς αυτού κατά των οχυρωμάτων και τούτο μέχρι καιρού.
\par 25 Και θέλει διεγείρει την δύναμιν αυτού και την καρδίαν αυτού εναντίον του βασιλέως του νότου μετά δυνάμεως μεγάλης· και ο βασιλεύς του νότου θέλει εγερθή εις πόλεμον μετά δυνάμεως μεγάλης και ισχυράς σφόδρα· πλην δεν θέλει δυνηθή να σταθή, διότι θέλουσι μηχανευθή μηχανάς κατ' αυτού.
\par 26 Και οι τρώγοντες τα εδέσματα αυτού θέλουσι συντρίψει αυτόν, και το στράτευμα αυτού θέλει πλημμυρήσει, και πολλοί θέλουσι πέσει πεφονευμένοι.
\par 27 Αι δε καρδίαι αμφοτέρων τούτων των βασιλέων θέλουσιν είσθαι εν τη πονηρία και θέλουσι λαλεί ψεύδη εν τη αυτή τραπέζη· αλλά τούτο δεν θέλει ευδοκιμήσει, επειδή έτι το τέλος θέλει είσθαι εν τω ωρισμένω καιρώ.
\par 28 Τότε θέλει επιστρέψει εις την γην αυτού μετά μεγάλου πλούτου· και η καρδία αυτού θέλει είσθαι εναντίον της διαθήκης της αγίας· και θέλει ενεργήσει και θέλει επιστρέψει εις την γην αυτού.
\par 29 Εν τω ωρισμένω καιρώ θέλει επιστρέψει και ελθεί προς τον νότον· πλην η εσχάτη φορά δεν θέλει είσθαι ως η πρώτη,
\par 30 διότι τα πλοία των Κηττιαίων θέλουσιν ελθεί εναντίον αυτού, και θέλει ταπεινωθή και επιστρέψει και θυμωθή εναντίον της διαθήκης της αγίας· και θέλει ενεργήσει και επιστρέψει και θέλει συνεννοηθή μετά των εγκαταλιπόντων την διαθήκην την αγίαν.
\par 31 Και βραχίονες θέλουσιν εγερθή εξ αυτού, και θέλουσι βεβηλώσει το αγιαστήριον της δυνάμεως και αφαιρέσει την παντοτεινήν θυσίαν και στήσει το βδέλυγμα της ερημώσεως.
\par 32 Και τους ανομούντας εις την διαθήκην θέλει διαφθείρει εν κολακείαις· ο λαός όμως, όστις γνωρίζει τον Θεόν αυτού, θέλει ισχύσει και κατορθώσει.
\par 33 Και οι συνετοί του λαού θέλουσι διδάξει πολλούς· πλην θέλουσι πέσει διά ρομφαίας και διά φλογός, δι' αιχμαλωσίας και διά λαφυραγωγίας, πολλών ημερών.
\par 34 Και όταν πέσωσι, θέλουσι βοηθηθή μικράν βοήθειαν· πολλοί όμως θέλουσι προστεθή εις αυτούς εν κολακείαις.
\par 35 Και εκ των συνετών θέλουσι πέσει, διά να δοκιμασθώσι και να καθαρισθώσι και να λευκανθώσιν, έως του εσχάτου καιρού· διότι και τούτο θέλει γείνει εν τω ωρισμένω καιρώ.
\par 36 Και ο βασιλεύς θέλει κάμει κατά την θέλησιν αυτού, και θέλει υψωθή και μεγαλυνθή υπεράνω παντός Θεού, και θέλει μεγαλορρημονήσει κατά του Θεού των θεών, και θέλει ευημερεί, εωσού συντελεσθή η οργή· διότι το ωρισμένον θέλει γείνει.
\par 37 Και δεν θέλει φροντίζει περί των θεών των πατέρων αυτού ουδέ περί επιθυμίας γυναικών ουδέ θέλει φροντίζει περί ουδενός θεού, διότι θέλει μεγαλυνθή υπεράνω πάντων.
\par 38 Τον δε θεόν Μαουζείμ θέλει δοξάσει επί του τόπου αυτού· και θεόν, τον οποίον οι πατέρες αυτού δεν εγνώρισαν, θέλει τιμήσει με χρυσόν και άργυρον και με πολυτίμους λίθους και με πράγματα επιθυμητά.
\par 39 Ούτω θέλει κάμει εις τα οχυρώματα Μαουζείμ μετά θεού αλλοτρίου· όσοι γνωρίσωσιν αυτόν, εις αυτούς θέλει πληθύνει δόξαν και θέλει κάμει αυτούς να εξουσιάσωσιν επί πολλών, και την γην θέλει διαμοιράσει με τιμήν.
\par 40 Και εν τω εσχάτω καιρώ ο βασιλεύς του νότου θέλει συγκερατισθή μετ' αυτού· και ο βασιλεύς του βορρά θέλει ελθεί εναντίον αυτού ως ανεμοστρόβιλος, μετά αμαξών και μετά ιππέων και μετά πολλών πλοίων· και θέλει ελθεί εις τους τόπους και θέλει πλημμυρήσει και διαβή·
\par 41 θέλει εισέλθει έτι εις την γην της δόξης και πολλοί τόποι θέλουσι καταστραφή· ούτοι όμως θέλουσι διασωθή εκ της χειρός αυτού, ο Εδώμ και ο Μωάβ και οι πρώτοι των υιών Αμμών.
\par 42 Και θέλει εκτείνει την χείρα αυτού επί τους τόπους· και η γη της Αιγύπτου δεν θέλει εκφύγει.
\par 43 Και θέλει κυριεύσει τους θησαυρούς του χρυσίου και του αργυρίου και πάντα τα επιθυμητά της Αιγύπτου· και οι Λίβυες και οι Αιθίοπες θέλουσιν είσθαι κατόπιν των βημάτων αυτού.
\par 44 Πλην αγγελίαι εκ της ανατολής και εκ του βορρά θέλουσι ταράξει αυτόν· διά τούτο θέλει εκβή μετά θυμού μεγάλου, διά να αφανίση και να εξολοθρεύση πολλούς.
\par 45 Και θέλει στήσει τας σκηνάς της βασιλικής αυτού κατοικήσεως μεταξύ των θαλασσών επί του ενδόξου όρους της αγιότητος· πλην θέλει ελθεί εις το τέλος αυτού και δεν θέλει υπάρχει ο βοηθών αυτόν.

\chapter{12}

\par 1 Και εν τω καιρώ εκείνω θέλει εγερθή Μιχαήλ, ο άρχων ο μέγας, ο ιστάμενος υπέρ των υιών του λαού σου· και θέλει είσθαι καιρός θλίψεως, οποία ποτέ δεν έγεινεν αφού υπήρξεν έθνος, μέχρις εκείνου του καιρού· και εν τω καιρώ εκείνω θέλει διασωθή ο λαός σου, πας όστις ευρεθή γεγραμμένος εν τω βιβλίω.
\par 2 Και πολλοί εκ των κοιμωμένων εν τω χώματι της γης θέλουσιν εξεγερθή, οι μεν εις αιώνιον ζωήν, οι δε εις ονειδισμόν και εις καταισχύνην αιώνιον.
\par 3 Και οι συνετοί θέλουσιν εκλάμψει ως η λαμπρότης του στερεώματος· και οι επιστρέφοντες πολλούς εις δικαιοσύνην ως οι αστέρες, εις τους αιώνας των αιώνων.
\par 4 Και συ, Δανιήλ, έγκλεισον τους λόγους και σφράγισον το βιβλίον, έως του εσχάτου καιρού· τότε πολλοί θέλουσι περιτρέχει και η γνώσις θέλει πληθυνθή.
\par 5 Και εγώ ο Δανιήλ εθεώρησα και ιδού, ίσταντο δύο άλλοι, εις εντεύθεν επί του χείλους του ποταμού και εις εκείθεν επί του χείλους του ποταμού.
\par 6 Και είπεν ο εις προς τον άνδρα τον ενδεδυμένον λινά, όστις ήτο επάνωθεν των υδάτων του ποταμού, Έως πότε θέλει είσθαι το τέλος των θαυμασίων τούτων;
\par 7 Και ήκουσα τον άνδρα τον ενδεδυμένον λινά, όστις ήτο επάνωθεν των υδάτων του ποταμού, ότε ύψωσε την δεξιάν αυτού και την αριστεράν αυτού εις τον ουρανόν και ώμοσεν εις τον ζώντα εις τον αιώνα, ότι θέλει είσθαι εις καιρόν, καιρούς και ήμισυ καιρού· και όταν συντελεσθή ο διασκορπισμός της δυνάμεως του αγίου λαού, πάντα ταύτα θέλουσιν εκπληρωθή.
\par 8 Και εγώ ήκουσα, αλλά δεν ενόησα· τότε είπον, Κύριέ μου, ποίον το τέλος τούτων;
\par 9 Και είπε, Ύπαγε, Δανιήλ· διότι οι λόγοι είναι κεκλεισμένοι και εσφραγισμένοι έως του εσχάτου καιρού.
\par 10 Πολλοί θέλουσι καθαρισθή και λευκανθή και δοκιμασθή· και οι ασεβείς θέλουσιν ασεβεί· και ουδείς εκ των ασεβών θέλει νοήσει· αλλ' οι συνετοί θέλουσι νοήσει.
\par 11 Και από του καιρού, καθ' ον η παντοτεινή θυσία αφαιρεθή και το βδέλυγμα της ερημώσεως στηθή, θέλουσιν είσθαι ημέραι χίλιαι διακόσιαι και ενενήκοντα.
\par 12 Μακάριος όστις υπομείνη και φθάση εις ημέρας χιλίας τριακοσίας και τριάκοντα πέντε.
\par 13 Αλλά συ ύπαγε, έως του τέλους· και θέλεις αναπαυθή και θέλεις σταθή εν τω κλήρω σου εις το τέλος των ημερών.


\end{document}