\begin{document}

\title{Joel}


\chapter{1}

\par 1 Ο λόγος του Κυρίου ο γενόμενος προς Ιωήλ τον υιόν του Φαθουήλ.
\par 2 Ακούσατε τούτο, οι πρεσβύτεροι, και δότε ακρόασιν, πάντες οι κατοικούντες την γήν· έγεινε τούτο εν ταις ημέραις υμών ή εν ταις ημέραις των πατέρων υμών;
\par 3 Διηγήθητε προς τα τέκνα σας περί τούτου και τα τέκνα σας προς τα τέκνα αυτών και τα τέκνα αυτών προς άλλην γενεάν.
\par 4 ό,τι αφήκεν η κάμπη, κατέφαγεν η ακρίς· και ό,τι αφήκεν η ακρίς, κατέφαγεν ο βρούχος· και ό,τι αφήκεν ο βρούχος, κατέφαγεν η ερυσίβη.
\par 5 Ανανήψατε, μέθυσοι, και κλαύσατε, και ολολύξατε, πάντες οι οινοπόται, διά τον νέον οίνον· διότι αφηρέθη από του στόματός σας.
\par 6 Επειδή έθνος ανέβη επί την γην μου, ισχυρόν και αναρίθμητον, του οποίου οι οδόντες είναι οδόντες λέοντος, και έχει μυλόδοντας σκύμνου.
\par 7 Έθεσε την άμπελόν μου εις αφανισμόν και τας συκάς μου εις θραύσιν· όλως εξελέπισεν αυτήν και απέρριψε· τα κλήματα αυτής έμειναν λευκά.
\par 8 Θρήνησον ως νύμφη περιεζωσμένη σάκκον διά τον άνδρα της νεότητος αυτής.
\par 9 Η προσφορά και η σπονδή αφηρέθη από του οίκου του Κυρίου· πενθούσιν οι ιερείς, οι λειτουργοί του Κυρίου.
\par 10 Ηρημώθη η πεδιάς, πενθεί η γή· διότι ηφανίσθη ο σίτος, εξηράνθη ο νέος οίνος, εξέλιπε το έλαιον.
\par 11 Αισχύνθητε, γεωργοί· ολολύξατε, αμπελουργοί, διά τον σίτον και διά την κριθήν· διότι ο θερισμός του αγρού απωλέσθη.
\par 12 Η άμπελος εξηράνθη και η συκή εξέλιπεν· η ροϊδιά και ο φοίνιξ και η μηλέα, πάντα τα δένδρα του αγρού εξηράνθησαν, ώστε εξέλιπεν η χαρά από των υιών των ανθρώπων.
\par 13 Περιζώσθητε και θρηνείτε, ιερείς· ολολύζετε, λειτουργοί του θυσιαστηρίου· έλθετε, διανυκτερεύσατε εν σάκκω, λειτουργοί του Θεού μου· διότι η προσφορά και η σπονδή επαύθη από του οίκου του Θεού σας.
\par 14 Αγιάσατε νηστείαν, κηρύξατε σύναξιν επίσημον, συνάξατε τους πρεσβυτέρους, πάντας τους κατοίκους του τόπου, εις τον οίκον Κυρίου του Θεού σας· και βοήσατε προς τον Κύριον,
\par 15 Οίμοι διά την ημέραν εκείνην· διότι η ημέρα του Κυρίου επλησίασε και θέλει ελθεί ως όλεθρος από του Παντοδυνάμου.
\par 16 Δεν αφηρέθησαν αι τροφαί απ' έμπροσθεν των οφθαλμών ημών, η ευφροσύνη και η χαρά από του οίκου του Θεού ημών;
\par 17 Οι σπόροι φθείρονται υπό τους βώλους αυτών, αι σιτοθήκαι ηρημώθησαν, αι αποθήκαι εχαλάσθησαν· διότι ο σίτος εξηράνθη.
\par 18 Πως στενάζουσι τα κτήνη· αδημονούσιν αι αγέλαι των βοών, διότι δεν έχουσι βοσκήν· ναι, τα ποίμνια των προβάτων ηφανίσθησαν.
\par 19 Κύριε, προς σε θέλω βοήσει· διότι το πυρ κατηνάλωσε τας βοσκάς της ερήμου και η φλόξ κατέκαυσε πάντα τα δένδρα του αγρού.
\par 20 Τα κτήνη έτι της πεδιάδος χάσκουσι προς σέ· διότι εξηράνθησαν οι ρύακες των υδάτων και πυρ κατέφαγε τας βοσκάς της ερήμου.

\chapter{2}

\par 1 Σαλπίσατε σάλπιγγα εν Σιών, και αλαλάξατε εν τω όρει τω αγίω μου· ας τρομάξωσι πάντες οι κατοικούντες την γήν· διότι έρχεται η ημέρα του Κυρίου, διότι είναι εγγύς·
\par 2 ημέρα σκότους και γνόφου, ημέρα νεφέλης και ομίχλης· ως αυγή εξαπλούται επί τα όρη λαός πολύς και ισχυρός· όμοιος αυτού δεν εστάθη απ' αιώνος ουδέ μετ' αυτόν θέλει σταθή πλέον ποτέ εις γενεάς γενεών.
\par 3 Πυρ κατατρώγει έμπροσθεν αυτού και φλόξ κατακαίει όπισθεν αυτού· η γη είναι ως ο παράδεισος της Εδέμ έμπροσθεν αυτού, και όπισθεν αυτού πεδιάς ηφανισμένη· και βεβαίως δεν θέλει εκφύγει απ' αυτού ουδέν.
\par 4 Η θέα αυτών είναι ως θέα ίππων, και ως ιππείς, ούτω θέλουσι τρέχει.
\par 5 Ως κρότος αμαξών θέλουσι πηδά επί τας κορυφάς των ορέων, ως ήχος φλογός πυρός, ήτις κατατρώγει την καλάμην, ως ισχυρός λαός παρατεταγμένος εις μάχην.
\par 6 Ενώπιον αυτού οι λαοί θέλουσι κατατρομάξει· πάντα τα πρόσωπα θέλουσιν αποσβολωθή.
\par 7 Θέλουσι τρέξει ως μαχηταί, ως άνδρες πολεμισταί θέλουσιν αναβή το τείχος, και θέλουσιν υπάγει έκαστος εις την οδόν αυτού και δεν θέλουσι χαλάσει τας τάξεις αυτών.
\par 8 Και δεν θέλουσι σπρώξει ο εις τον άλλον· θέλουσι περιπατεί έκαστος εις την οδόν αυτού, και πίπτοντες επί τα βέλη δεν θέλουσι πληγωθή.
\par 9 Θέλουσι περιτρέχει εν τη πόλει, θέλουσι δράμει επί το τείχος, θέλουσιν αναβαίνει επί τας οικίας, θέλουσιν εμβαίνει διά των θυρίδων ως κλέπτης.
\par 10 Η γη θέλει σεισθή έμπροσθεν αυτών, οι ουρανοί θέλουσι τρέμει, ο ήλιος και η σελήνη θέλουσι συσκοτάσει, και τα άστρα θέλουσι σύρει οπίσω το φέγγος αυτών.
\par 11 Και ο Κύριος θέλει εκπέμψει την φωνήν αυτού έμπροσθεν του στρατεύματος αυτού· διότι το στρατόπεδον αυτού είναι μέγα σφόδρα, διότι ο εκτελών τον λόγον αυτού είναι ισχυρός, διότι η ημέρα του Κυρίου είναι μεγάλη και τρομερά σφόδρα και τις δύναται να υποφέρη αυτήν;
\par 12 Και τώρα διά τούτο, λέγει Κύριος, επιστρέψατε προς εμέ εξ όλης της καρδίας υμών και εν νηστεία και εν κλαυθμώ και εν πένθει.
\par 13 Και διαρρήξατε την καρδίαν σας και μη τα ιμάτιά σας και επιστρέψατε προς Κύριον τον Θεόν σας· διότι είναι ελεήμων και οικτίρμων, μακρόθυμος και πολυέλεος και μεταμελούμενος διά το κακόν.
\par 14 Τις οίδεν, αν θέλη επιστρέψει και μεταμεληθή και αφήσει ευλογίαν κατόπιν αυτού, προσφοράν και σπονδήν εις Κύριον τον Θεόν υμών;
\par 15 Σαλπίσατε σάλπιγγα εν Σιών, αγιάσατε νηστείαν, κηρύξατε σύναξιν επίσημον.
\par 16 Συναθροίσατε τον λαόν, αγιάσατε την σύναξιν, συνάξατε τους πρεσβυτέρους, συναθροίσατε τα νήπια και τα θηλάζοντα μαστούς· ας εξέλθη ο νυμφίος εκ του κοιτώνος αυτού και η νύμφη εκ του θαλάμου αυτής.
\par 17 Ας κλαύσωσιν οι ιερείς, οι λειτουργοί του Κυρίου, μεταξύ της στοάς και του θυσιαστηρίου, και ας είπωσι, Φείσαι, Κύριε, του λαού σου και μη δώσης την κληρονομίαν σου εις όνειδος, ώστε να κυριεύσωσιν αυτούς τα έθνη· διά τι να είπωσι μεταξύ των λαών, Που είναι ο Θεός αυτών;
\par 18 Και ο Κύριος θέλει ζηλοτυπήσει διά την γην αυτού και θέλει φεισθή του λαού αυτού.
\par 19 Ναι, ο Κύριος θέλει αποκριθή και ειπεί προς τον λαόν αυτού, Ιδού, εγώ θέλω εξαποστείλει προς υμάς τον σίτον και τον οίνον και το έλαιον και θέλετε εμπλησθή απ' αυτών, και δεν θέλω σας κάμει πλέον όνειδος μεταξύ των εθνών.
\par 20 Αλλά θέλω απομακρύνει από σας τον εκ του βορρά πολέμιον, και θέλω εξώσει αυτόν εις γην άνυδρον και έρημον, με το πρόσωπον αυτού προς την ανατολικήν θάλασσαν, το δε όπισθεν αυτού μέρος προς την θάλασσαν την δυτικήν, και η δυσωδία αυτού θέλει αναβή και η κακή οσμή αυτού θέλει υψωθή, διότι έπραξε μεγάλα.
\par 21 Μη φοβού, γή· χαίρε και ευφραίνου· διότι ο Κύριος θέλει κάμει μεγαλεία.
\par 22 Μη τρομάζετε, κτήνη της πεδιάδος· διότι αι βοσκαί της ερήμου βλαστάνουσι, διότι το δένδρον φέρει τον καρπόν αυτού, η συκή και η άμπελος εκδίδουσι την δύναμιν αυτών.
\par 23 Και, τα τέκνα της Σιών, χαίρετε και ευφραίνεσθε εις Κύριον τον Θεόν σας· διότι έδωκεν εις εσάς την πρώϊμον βροχήν εγκαίρως και θέλει βρέξει εις εσάς βροχήν πρώϊμον και όψιμον ως πρότερον.
\par 24 Και τα αλώνια θέλουσι γεμισθή από σίτου και οι ληνοί θέλουσιν υπερεκχειλίσει από οίνου και ελαίου.
\par 25 Και θέλω αναπληρώσει εις εσάς τα έτη, τα οποία κατέφαγεν η ακρίς, ο βρούχος και η ερυσίβη και η κάμπη, το στράτευμά μου το μέγα, το οποίον εξαπέστειλα εναντίον σας.
\par 26 Και θέλετε φάγει αφθόνως και χορτασθή και αινέσει το όνομα Κυρίου του Θεού σας· όστις έκαμε θαυμάσια με σάς· και ο λαός μου δεν θέλει καταισχυνθή εις τον αιώνα.
\par 27 Και θέλετε γνωρίσει ότι εγώ είμαι εν μέσω του Ισραήλ και εγώ είμαι Κύριος ο Θεός σας και δεν είναι άλλος ουδείς· και ο λαός μου δεν θέλει καταισχυνθή εις τον αιώνα.
\par 28 Και μετά ταύτα θέλω εκχέει το πνεύμά μου επί πάσαν σάρκα· και θέλουσι προφητεύσει οι υιοί σας και αι θυγατέρες σας· οι πρεσβύτεροί σας θέλουσιν ενυπνιασθή ενύπνια, οι νεανίσκοι σας θέλουσιν ιδεί οράσεις.
\par 29 Και έτι επί τους δούλους μου και επί τας δούλας μου εν ταις ημέραις εκείναις θέλω εκχέει το πνεύμά μου.
\par 30 Και θέλω δείξει τέρατα εν τοις ουρανοίς και επί της γης, αίμα και πυρ και ατμίδα καπνού.
\par 31 Ο ήλιος θέλει μεταστραφή εις σκότος και η σελήνη εις αίμα, πριν έλθη η ημέρα του Κυρίου η μεγάλη και επιφανής.
\par 32 Και πας όστις επικαλεσθή το όνομα του Κυρίου, θέλει σωθή· διότι εν τω όρει Σιών και εν Ιερουσαλήμ θέλει είσθαι σωτηρία, καθώς είπεν ο Κύριος, και εις τους υπολοίπους τους οποίους ο Κύριος θέλει προσκαλέσει.

\chapter{3}

\par 1 Διότι, ιδού, εν ταις ημέραις εκείναις και εν τω καιρώ εκείνω, όταν επιστρέψω τους αιχμαλώτους του Ιούδα και της Ιερουσαλήμ,
\par 2 θέλω συνάξει έτι πάντα τα έθνη και θέλω καταβιβάσει αυτά εις την κοιλάδα του Ιωσαφάτ, και θέλω κριθή μετ' αυτών εκεί υπέρ του λαού μου και της κληρονομίας μου Ισραήλ, τον οποίον διέσπειραν μεταξύ των εθνών και διεμοιράσθησαν την γην μου·
\par 3 και έρριψαν κλήρους διά τον λαόν μου· και έδωκαν παιδίον διά πόρνην και επώλουν κοράσιον διά οίνον και έπινον.
\par 4 Και έτι τι έχετε σεις να κάμητε μετ' εμού, Τύρε και Σιδών και πάντα τα όρια της Παλαιστίνης; θέλετε μοι ανταποδώσει ανταπόδομα; εάν σεις ανταποδώσητε εις εμέ, χωρίς αργοπορίας ταχέως θέλω επιστρέψει το ανταπόδομά σας επί την κεφαλήν σας.
\par 5 Διότι ελάβετε το αργύριόν μου και το χρυσίον μου, και τα εκλεκτά μου αγαθά εφέρετε εις τους ναούς σας.
\par 6 Τους δε υιούς Ιούδα και τους υιούς Ιερουσαλήμ επωλήσατε εις τους υιούς των Ελλήνων, διά να απομακρύνητε αυτούς από των ορίων αυτών.
\par 7 Ιδού, εγώ θέλω εγείρει αυτούς από του τόπου όπου επωλήσατε αυτούς, και θέλω επιστρέψει το ανταπόδομά σας επί την κεφαλήν σας.
\par 8 Και θέλω πωλήσει τους υιούς σας και τας θυγατέρας σας εις την χείρα των υιών Ιούδα, και θέλουσι πωλήσει αυτούς εις τους Σαβαίους, εις έθνος μακράν απέχον· διότι ο Κύριος ελάλησε.
\par 9 Κηρύξατε τούτο εν τοις έθνεσιν, αγιάσατε πόλεμον, διεγείρατε τους μαχητάς, ας πλησιάσωσιν, ας αναβαίνωσι πάντες οι άνδρες του πολέμου·
\par 10 σφυρηλατήσατε τα υνία σας εις ρομφαίας και τα δρέπανά σας εις λόγχας· ο αδύνατος ας λέγη, Εγώ είμαι δυνατός·
\par 11 Συναθροίσθητε και έλθετε κυκλόθεν, πάντα τα έθνη, και συνάχθητε ομού· εκεί θέλει καταστρέψει ο Κύριος τους ισχυρούς σου.
\par 12 Ας εγερθώσι και ας αναβώσι τα έθνη εις την κοιλάδα του Ιωσαφάτ· διότι εκεί θέλω καθήσει διά να κρίνω πάντα τα έθνη τα κυκλόθεν.
\par 13 Βάλετε δρέπανον, διότι ο θερισμός είναι ώριμος· έλθετε, κατάβητε· διότι ο ληνός είναι πλήρης, τα υπολήνια υπερεκχειλίζουσιν· επειδή η κακία αυτών είναι μεγάλη.
\par 14 Πλήθη, πλήθη εις την κοιλάδα της δίκης· διότι εγγύς είναι η ημέρα του Κυρίου εις την κοιλάδα της δίκης.
\par 15 Ο ήλιος και σελήνη θέλουσι συσκοτάσει και οι αστέρες θέλουσι σύρει οπίσω το φέγγος αυτών.
\par 16 Ο δε Κύριος θέλει βρυχήσει εκ Σιών και εκπέμψει την φωνήν αυτού εξ Ιερουσαλήμ· και οι ουρανοί και η γη θέλουσι σεισθή· αλλ' ο Κύριος θέλει είσθαι το καταφύγιον του λαού αυτού και η ισχύς των υιών Ισραήλ.
\par 17 Ούτω θέλετε γνωρίσει ότι εγώ είμαι Κύριος ο Θεός σας, ο κατοικών εν Σιών, τω όρει τω αγίω μου· τότε η Ιερουσαλήμ θέλει είσθαι αγία και αλλογενείς δεν θέλουσι διέλθει δι' αυτής πλέον.
\par 18 Και εν τη ημέρα εκείνη τα όρη θέλουσι σταλάξει γλεύκος και οι λόφοι θέλουσι ρέει γάλα και πάντες οι ρύακες του Ιούδα θέλουσι ρέει ύδατα και πηγή θέλει εξέλθει εκ του οίκου του Κυρίου και θέλει ποτίζει την φάραγγα του Σιττείμ.
\par 19 Η Αίγυπτος θέλει είσθαι ηρημωμένη και ο Εδώμ θέλει είσθαι έρημος άβατος διά τας εις τους υιούς Ιούδα αδικίας, διότι έχυσαν αίμα αθώον εν τη γη αυτών.
\par 20 Η δε Ιουδαία θέλει κατοικείσθαι εις τον αιώνα και η Ιερουσαλήμ εις γενεάς γενεών.
\par 21 Και θέλω καθαρίσει το αίμα αυτών, το οποίον δεν εκαθάρισα· διότι ο Κύριος κατοικεί εν Σιών.


\end{document}