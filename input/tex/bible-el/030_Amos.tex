\begin{document}

\title{Amos}


\chapter{1}

\par Οι λόγοι του Αμώς, όστις ήτο εκ των βοσκών της Θεκουέ, τους οποίους είδε περί του Ισραήλ εν ταις ημέραις Οζίου βασιλέως του Ιούδα, και εν ταις ημέραις Ιεροβοάμ, υιού του Ιωάς βασιλέως του Ισραήλ, δύο έτη προ του σεισμού.
\par 2 Και είπεν, Ο Κύριος θέλει βρυχήσει εκ Σιών και θέλει εκπέμψει την φωνήν αυτού εξ Ιερουσαλήμ· και αι κατοικίαι των ποιμένων θέλουσι πενθήσει, και η κορυφή του Καρμήλου θέλει ξηρανθή.
\par 3 Ούτω λέγει Κύριος· διά τας τρεις παραβάσεις της Δαμασκού και διά τας τέσσαρας δεν θέλω αποστρέψει την τιμωρίαν αυτής, διότι ηλώνισαν την Γαλαάδ με τριβόλους σιδηρούς·
\par 4 αλλά θέλω εξαποστείλει πυρ εις τον οίκον Αζαήλ και θέλει καταφάγει τα παλάτια του Βεν-αδάδ.
\par 5 Και θέλω συντρίψει τους μοχλούς της Δαμασκού και εξολοθρεύσει τον κάτοικον από της πεδιάδος Αβέν και τον κρατούντα το σκήπτρον από του οίκου Εδέν, και ο λαός της Συρίας θέλει φερθή αιχμάλωτος εις Κιρ, λέγει Κύριος.
\par 6 Ούτω λέγει Κύριος· Διά τας τρεις παραβάσεις της Γάζης και διά τας τέσσαρας δεν θέλω αποστρέψει την τιμωρίαν αυτής· διότι ηχμαλώτισαν τον λαόν μου αιχμαλωσίαν τελείαν, διά να παραδώσωσιν αυτούς εις τον Εδώμ·
\par 7 αλλά θέλω εξαποστείλει πυρ εις το τείχος της Γάζης και θέλει καταφάγει τα παλάτια αυτής.
\par 8 Και θέλω εξολοθρεύσει τον κάτοικον από της Αζώτου και τον κρατούντα το σκήπτρον από της Ασκάλωνος, και θέλω στρέψει την χείρα μου εναντίον της Ακκαρών και το υπόλοιπον των Φιλισταίων θέλει απολεσθή, λέγει Κύριος ο Θεός.
\par 9 Ούτω λέγει Κύριος· Διά τας τρεις παραβάσεις της Τύρου και διά τας τέσσαρας δεν θέλω αποστρέψει την τιμωρίαν αυτής· διότι παρέδωκαν τον λαόν μου εις αιχμαλωσίαν τελείαν εις τον Εδώμ και δεν ενεθυμήθησαν την αδελφικήν συνθήκην·
\par 10 αλλά θέλω εξαποστείλει πυρ εις το τείχος της Τύρου και θέλει καταφάγει τα παλάτια αυτής.
\par 11 Ούτω λέγει Κύριος· Διά τας τρεις παραβάσεις του Εδώμ και διά τας τέσσαρας δεν θέλω αποστρέψει την τιμωρίαν αυτού, διότι κατεδίωξε τον αδελφόν αυτού εν ρομφαία και ηθέτησε την ευσπλαγχνίαν αυτού, και ο θυμός αυτού κατεσπάραττεν ακαταπαύστως και εφύλαττε την οργήν αυτού παντοτεινά·
\par 12 αλλά θέλω εξαποστείλει πυρ επί Θαιμάν και θέλει καταφάγει τα παλάτια της Βοσόρρας.
\par 13 Ούτω λέγει Κύριος· Διά τας τρεις παραβάσεις των υιών Αμμών και διά τας τέσσαρας δεν θέλω αποστρέψει την τιμωρίαν αυτού, διότι διέσχιζον τας εγκυμονούσας της Γαλαάδ, διά να πλατύνωσι το όριον αυτών·
\par 14 αλλά θέλω ανάψει πυρ εις το τείχος της Ραββά και θέλει καταφάγει τα παλάτια αυτής, μετά κραυγής εν τη ημέρα της μάχης, μετά ανεμοστροβίλου εν τη ημέρα της θυέλλης.
\par 15 Και ο βασιλεύς αυτών θέλει υπάγει εις αιχμαλωσίαν, αυτός και οι άρχοντες αυτού ομού, λέγει Κύριος.

\chapter{2}

\par Ούτω λέγει Κύριος· Διά τας τρεις παραβάσεις του Μωάβ και διά τας τέσσαρας δεν θέλω αποστρέψει την τιμωρίαν αυτού· διότι κατέκαυσε τα οστά του βασιλέως του Εδώμ μέχρι κονίας·
\par 2 αλλά θέλω εξαποστείλει πυρ επί τον Μωάβ και θέλει καταφάγει τα παλάτια της Κιριώθ· και ο Μωάβ θέλει αποθάνει μετά θορύβου, μετά κραυγής, μετ' ήχου σάλπιγγος.
\par 3 Και θέλω εξολοθρεύσει τον κριτήν εκ μέσου αυτού, και θέλω αποκτείνει πάντας τους άρχοντας αυτού μετ' αυτού, λέγει Κύριος.
\par 4 Ούτω λέγει Κύριος· Διά τας τρεις παραβάσεις του Ιούδα και διά τας τέσσαρας δεν θέλω αποστρέψει την τιμωρίαν αυτού, διότι κατεφρόνησαν τον νόμον του Κυρίου και δεν εφύλαξαν τα προστάγματα αυτού και επλάνησαν αυτούς τα μάταια αυτών, κατόπιν των οποίων περιεπάτησαν οι πατέρες αυτών·
\par 5 αλλά θέλω εξαποστείλει πυρ επί τον Ιούδαν και θέλει καταφάγει τα παλάτια της Ιερουσαλήμ.
\par 6 Ούτω λέγει Κύριος· Διά τας τρεις παραβάσεις του Ισραήλ και διά τας τέσσαρας δεν θέλω αποστρέψει την τιμωρίαν αυτού διότι επώλησαν τον δίκαιον δι' αργύριον και τον πένητα διά ζεύγος υποδημάτων·
\par 7 οίτινες ποθούσι να βλέπωσι την κόνιν της γης επί την κεφαλήν των πτωχών και εκκλίνουσι την οδόν των πενήτων· και υιός και πατήρ αυτού υπάγουσι προς την αυτήν παιδίσκην, διά να βεβηλόνωσι το όνομα το άγιόν μου·
\par 8 και πλαγιάζουσι πλησίον παντός θυσιαστηρίου επί ενδυμάτων ενεχυριασμένων, και πίνουσιν εν τω οίκω των θεών αυτών τον οίνον των καταδυναστευομένων.
\par 9 Αλλ' εγώ εξωλόθρευσα τον Αμορραίον απ' έμπροσθεν αυτών, του οποίου το ύψος ήτο ως το ύψος των κέδρων και αυτός ισχυρός ως αι δρύς· και ηφάνισα τον καρπόν αυτού επάνωθεν και τας ρίζας αυτού υποκάτωθεν.
\par 10 Και εγώ σας ανεβίβασα εκ γης Αιγύπτου και σας περιέφερον τεσσαράκοντα έτη διά της ερήμου, διά να κληρονομήσητε την γην του Αμορραίου.
\par 11 Και ανέστησα εκ των υιών σας διά προφήτας και εκ των νεανίσκων σας διά Ναζηραίους. Δεν είναι ούτως, υιοί Ισραήλ; λέγει Κύριος.
\par 12 Σεις δε εποτίζετε τους Ναζηραίους οίνον και προσετάξατε τους προφήτας λέγοντες, Μη προφητεύσητε.
\par 13 Ιδού, εγώ θέλω καταθλίψει υμάς εν τω τόπω υμών, καθώς καταθλίβεται η άμαξα η πλήρης δραγμάτων.
\par 14 Και η φυγή θέλει χαθή από του δρομέως και ο ανδρείος δεν θέλει στερεώσει την ισχύν αυτού και ο ισχυρός δεν θέλει διασώσει την ψυχήν αυτού,
\par 15 και ο τοξότης δεν θέλει δυνηθή να σταθή· και ο ταχύπους να εκφύγη και ο ιππεύς να σώση την ζωήν αυτού,
\par 16 και ο μεταξύ των δυνατών γενναιοκάρδιος γυμνός θέλει φύγει εν εκείνη τη ημέρα, λέγει Κύριος.

\chapter{3}

\par Ακούσατε τον λόγον τούτον, τον οποίον ελάλησεν ο Κύριος εναντίον σας, υιοί Ισραήλ, εναντίον παντός του γένους, το οποίον ανεβίβασα εκ γης Αιγύπτου λέγων,
\par 2 Εσάς μόνον εγνώρισα εκ πάντων των γενών της γής· διά τούτο θέλω σας τιμωρήσει διά πάσας τας ανομίας σας.
\par 3 Δύνανται δύο να περιπατήσωσιν ομού, εάν δεν ήναι σύμφωνοι;
\par 4 Θέλει βρυχήσει ο λέων εν τω δρυμώ, εάν δεν έχη θήραν; θέλει εκπέμψει την φωνήν αυτού ο σκύμνος από της κατοικίας αυτού, αν δεν επίασέ τι;
\par 5 Δύναται πτηνόν να πέση εις παγίδα επί της γης, όπου δεν είναι βρόχος δι' αυτό; ήθελε σηκωθή παγίς εκ της γης, χωρίς να πιασθή τι;
\par 6 Δύναται να ηχήση σάλπιγξ εν πόλει και ο λαός να μη πτοηθή; δύναται να γείνη συμφορά εν πόλει και ο Κύριος να μη έκαμεν αυτήν;
\par 7 Βεβαίως Κύριος ο Θεός δεν θέλει κάμει ουδέν, χωρίς να αποκαλύψη το απόκρυφον αυτού εις τους δούλους αυτού τους προφήτας.
\par 8 Ο λέων εβρύχησε· τις δεν θέλει φοβηθή; Κύριος ο Θεός ελάλησε· τις δεν θέλει προφητεύσει;
\par 9 Κηρύξατε προς τα παλάτια της Αζώτου και προς τα παλάτια της γης της Αιγύπτου και είπατε, Συνάχθητε επί τα όρη της Σαμαρείας και ιδέτε τους μεγάλους θορύβους εν μέσω αυτής και τας καταδυναστείας εν μέσω αυτής,
\par 10 διότι δεν εξεύρουσι να πράττωσι το ορθόν, λέγει Κύριος, οι θησαυρίζοντες αδικίαν και αρπαγήν εν τοις παλατίοις αυτών.
\par 11 Διά τούτο ούτω λέγει Κύριος ο Θεός· Εχθρός θέλει περικυκλώσει την γην σου και θέλει καταβάλει την ισχύν σου από σου και τα παλάτιά σου θέλουσι διαρπαγή.
\par 12 Ούτω λέγει Κύριος· Καθώς ο ποιμήν αποσπά από του στόματος του λέοντος δύο σκέλη ή λοβόν ωτίου, ούτω θέλουσιν αποσπασθή οι υιοί Ισραήλ, οι κατοικούντες εν Σαμαρεία από της γωνίας της κλίνης και εν Δαμασκώ από της στρωμνής.
\par 13 Ακούσατε και διαμαρτυρήθητε προς τον οίκον Ιακώβ, λέγει Κύριος ο Θεός, ο Θεός των δυνάμεων,
\par 14 ότι καθ' ην ημέραν επισκεφθώ τας παραβάσεις του Ισραήλ επ' αυτόν, θέλω επισκεφθή και τα θυσιαστήρια της Βαιθήλ, και τα κέρατα του θυσιαστηρίου θέλουσιν εκκοπή και πέσει κατά γης.
\par 15 Και θέλω πατάξει τον χειμερινόν οίκον μετά του θερινού οίκου, και οι οίκοι οι ελεφάντινοι θέλουσιν απολεσθή και οι οίκοι οι μεγάλοι θέλουσιν αφανισθή, λέγει Κύριος.

\chapter{4}

\par Ακούσατε τον λόγον τούτον, δαμάλεις της Βασάν, αι εν τω όρει της Σαμαρείας, αι καταδυναστεύουσαι τους πτωχούς, αι καταθλίβουσαι τους πένητας, αι λέγουσαι προς τους κυρίους αυτών, Φέρετε και ας πίωμεν.
\par 2 Κύριος ο Θεός ώμοσεν εις την αγιότητα αυτού ότι ιδού, ημέραι έρχονται εις υμάς, καθ' ας θέλουσι σας πιάσει με άγκιστρα και τους απογόνους σας με καμάκια αλιευτικά.
\par 3 Και θέλετε εξέλθει από τας χαλάστρας εκάστη απ' ευθείας ενώπιον αυτής, και θέλετε απορρίψει πάντα τα του παλατίου, λέγει Κύριος.
\par 4 Έλθετε εις Βαιθήλ και ασεβήσατε· εν Γαλγάλοις πληθύνατε την ασέβειαν· και φέρετε τας θυσίας σας κατά πάσαν πρωΐαν, τα δέκατά σας κατά πάσαν τριετίαν.
\par 5 Και προσφέρετε εις θυσίαν ευχαριστίας άρτον ένζυμον, και κηρύξατε τας αυτοπροαιρέτους προσφοράς· αναγγείλατε αυτάς· διότι ούτως αγαπάτε, υιοί Ισραήλ, λέγει Κύριος ο Θεός.
\par 6 Και εγώ ότι σας έδωκα πείναν εν πάσαις ταις πόλεσιν υμών και έλλειψιν άρτου εν πάσι τοις τόποις υμών, και δεν επεστρέψατε προς εμέ, λέγει Κύριος.
\par 7 Και εγώ προσέτι εκράτησα την βροχήν από σας, ότε έμενον τρεις μήνες έτι έως του θέρους· και έβρεξα επί μίαν πόλιν και επί άλλην πόλιν δεν έβρεξα· μία μερίς εβράχη και η μερίς, επί την οποίαν δεν έβρεξεν εξηράνθη.
\par 8 Ούτω δύο τρεις πόλεις υπήγαν περιπλανώμεναι εις μίαν πόλιν να πίωσιν ύδωρ και δεν εχορτάσθησαν, και δεν επεστρέψατε προς εμέ, λέγει Κύριος.
\par 9 Σας επάταξα με ανεμοφθορίαν και ερυσίβην· το πλήθος των κήπων σας και των αμπελώνων σας και των συκεώνων σας και των ελαιώνων σας κατέφαγεν η κάμπη, και δεν επεστρέψατε προς εμέ, λέγει Κύριος.
\par 10 Εξαπέστειλα εφ' ημάς θανατικόν κατά τον τρόπον της Αιγύπτου· τους νεανίσκους σας εθανάτωσα εν ρομφαία, αιχμαλωτίσας και τους ίππους σας· και ανεβίβασα την δυσωδίαν των στρατοπέδων σας έως των μυκτήρων σας, και δεν επεστρέψατε προς εμέ, λέγει Κύριος.
\par 11 Σας κατέστρεψα, καθώς ο Θεός κατέστρεψε τα Σόδομα και τα Γόμορρα, και εγείνετε ως δαυλός απεσπασμένος από της πυρκαϊάς, και δεν επεστρέψατε προς εμέ, λέγει Κύριος.
\par 12 Διά τούτο ούτω θέλω κάμει εις σε, Ισραήλ· όθεν, επειδή θέλω κάμει τούτο εις σε, ετοιμάσθητι να απαντήσης τον Θεόν σου, Ισραήλ.
\par 13 Διότι ιδού, ο μορφών τα όρη και κατασκευάζων τον άνεμον και απαγγέλλων προς τον άνθρωπον τις είναι ο στοχασμός αυτού, ο ποιών την αυγήν σκότος και επιβαίνων επί τα ύψη της γης, Κύριος ο Θεός των δυνάμεων είναι το όνομα αυτού.

\chapter{5}

\par Ακούσατε τον λόγον τούτον, τον θρήνον τον οποίον εγώ αναλαμβάνω εναντίον σας, οίκος Ισραήλ.
\par 2 Επεσε· δεν θέλει σηκωθή πλέον η παρθένος του Ισραήλ· είναι ερριμμένη επί της γης αυτής· δεν υπάρχει ο ανιστών αυτήν.
\par 3 Διότι ούτω λέγει Κύριος ο Θεός· Η πόλις, εξ ης εξήρχοντο χίλιοι, θέλει μείνει με εκατόν· και εξ ης εξήρχοντο εκατόν, θέλει μείνει με δέκα εν τω οίκω Ισραήλ.
\par 4 Διότι ούτω λέγει Κύριος προς τον οίκον του Ισραήλ· Εκζητήσατέ με και θέλετε ζήσει.
\par 5 Και μη εκζητείτε την Βαιθήλ και μη εισέρχεσθε εις Γάλγαλα και μη διαβαίνετε εις Βηρ-σαβεέ· διότι τα Γάλγαλα θέλουσιν υπάγει εξάπαντος εις αιχμαλωσίαν και η Βαιθήλ θέλει καταντήσει εις το μηδέν.
\par 6 Εκζητήσατε τον Κύριον και θέλετε ζήσει, μήπως εφορμήση ως πυρ επί τον οίκον Ιωσήφ και καταφάγη αυτόν και δεν υπάρχη ο σβύνων την Βαιθήλ.
\par 7 Σεις, οι μεταστρέφοντες την κρίσιν εις αψίνθιον και απορρίπτοντες κατά γης την δικαιοσύνην,
\par 8 εκζητήσατε τον ποιούντα την Πλειάδα και τον Ωρίωνα και μετατρέποντα την σκιάν του θανάτου εις αυγήν και σκοτίζοντα την ημέραν εις νύκτα, τον προσκαλούντα τα ύδατα της θαλάσσης και εκχέοντα αυτά επί το πρόσωπον της γής· Κύριος είναι το όνομα αυτού·
\par 9 τον εγείροντα αφανισμόν κατά του ισχυρού και επάγοντα αφανισμόν εις τα οχυρώματα.
\par 10 Μισούσι τον ελέγχοντα εν τη πύλη και βδελύττονται τον λαλούντα εν ευθύτητι.
\par 11 Όθεν, επειδή καταθλίβετε τον πτωχόν και λαμβάνετε απ' αυτού φόρον σίτου, αν και ωκοδομήσατε οίκους λαξευτούς, δεν θέλετε όμως κατοικήσει εν αυτοίς· αν και εφυτεύσατε αμπελώνας επιθυμητούς, δεν θέλετε όμως πίει τον οίνον αυτών.
\par 12 Διότι γνωρίζω τας πολλάς ασεβείας σας και τας ισχυράς αμαρτίας σας οίτινες καταθλίβετε τον δίκαιον, δωροδοκείσθε και καταδυναστεύετε τους πτωχούς εν τη πύλη.
\par 13 Διά τούτο ο συνετός θέλει σιωπά εν τω καιρώ εκείνω· διότι είναι καιρός κακός.
\par 14 Εκζητήσατε το καλόν και ουχί το κακόν, διά να ζήσητε· και ούτω Κύριος ο Θεός των δυνάμεων θέλει είσθαι μεθ' υμών, καθώς είπετε.
\par 15 Μισείτε το κακόν και αγαπάτε το καλόν και αποκαταστήσατε την κρίσιν εν τη πύλη· ίσως ο Κύριος ο Θεός των δυνάμεων ελεήση το υπόλοιπον του Ιωσήφ.
\par 16 Διά τούτο Κύριος ο Θεός των δυνάμεων, ο Κύριος, λέγει ούτως· Οδυρμός εν πάσαις ταις πλατείαις· και εν πάσαις ταις οδοίς θέλουσι λέγει, Ουαί, ουαί· και θέλουσι κράζει τον γεωργόν εις πένθος και τους επιτηδείους θρηνωδούς εις οδυρμόν.
\par 17 Και εν πάσαις ταις αμπέλοις οδυρμός· διότι θέλω περάσει διά μέσου σου, λέγει Κύριος.
\par 18 Ουαί εις τους επιθυμούντας την ημέραν του Κυρίου· προς τι θέλει είσθαι αύτη διά σας; η ημέρα του Κυρίου είναι σκότος και ουχί φως.
\par 19 Είναι ως εάν έφευγεν άνθρωπος απ' έμπροσθεν λέοντος και άρκτος απήντα αυτόν, ή ως εάν εισήρχετο εις οίκον και επιστηρίξαντα την χείρα αυτού εις τον τοίχον, εδάγκανεν αυτόν όφις.
\par 20 Δεν θέλει είσθαι σκότος η ημέρα του Κυρίου και ουχί φως; μάλιστα ζόφος και φέγγος μη έχουσα;
\par 21 Εμίσησα, απεστράφην τας εορτάς σας, και δεν θέλω οσφρανθή εν ταις πανηγύρεσιν υμών.
\par 22 Εάν μοι προσφέρητε τα ολοκαυτώματα και τας θυσίας σας, δεν θέλω δεχθή αυτάς και δεν θέλω επιβλέψει εις τας ειρηνικάς θυσίας των σιτευτών σας.
\par 23 Αφαίρεσον απ' εμού τον ήχον των ωδών σου, και το άσμα των οργάνων σου δεν θέλω ακούσει.
\par 24 Αλλ' η κρίσις ας καταρρέη ως ύδωρ και η δικαιοσύνη ως αένναος χείμαρρος.
\par 25 Μήποτε θυσίας και προσφοράς προσεφέρετε εις εμέ, οίκος Ισραήλ, τεσσαράκοντα έτη εν τη ερήμω;
\par 26 Μάλιστα ανελάβετε την σκηνήν του Μολόχ σας και τον Χιούν, τον αστέρα του θεού σας, τα είδωλα υμών, τα οποία εκάμετε εις αυτούς.
\par 27 Διά τούτο θέλω σας μετοικίσει επέκεινα της Δαμασκού, λέγει Κύριος· ο Θεός των δυνάμεων είναι το όνομα αυτού.

\chapter{6}

\par Ουαί εις τους αμεριμνούντας εν Σιών και πεποιθότας επί το όρος της Σαμαρείας, τα διαφημιζόμενα ως έξοχα μεταξύ των εθνών και εις τα οποία ήλθεν ο οίκος Ισραήλ.
\par 2 Διάβητε εις Χαλνέ και ιδέτε· και εκείθεν διέλθετε εις Αιμάθ την μεγάλην· έπειτα κατάβητε εις την Γαθ των Φιλισταίων· είναι αύται καλήτεραι παρά τα βασίλεια ταύτα; το όριον αυτών μεγαλήτερον παρά το όριόν σας;
\par 3 Οίτινες θέτετε μακράν την κακήν ημέραν και φέρετε πλησίον την καθέδραν της αρπαγής·
\par 4 οίτινες πλαγιάζετε επί κλίνας ελεφαντίνας και εξαπλόνεσθε επί τας στρωμνάς σας και τρώγετε τα αρνία εκ του ποιμνίου και τους μόσχους εκ μέσου της αγέλης,
\par 5 οίτινες ψάλλετε εν τη φωνή της λύρας, εφευρίσκετε εις εαυτούς όργανα μουσικής καθώς ο Δαβίδ,
\par 6 οίτινες πίνετε τον οίνον με φιάλας και χρίεσθε με τα εξαίρετα μύρα· διά δε τον συντριμμόν του Ιωσήφ δεν θλίβεσθε.
\par 7 Διά τούτο τώρα ούτοι θέλουσιν υπάγει εις αιχμαλωσίαν μετά των πρώτων αιχμαλωτισθησομένων, και η αγαλλίασις των εξηπλωμένων εν τω συμποσίω θέλει αφαιρεθή.
\par 8 Κύριος ο Θεός ώμοσεν εις εαυτόν, Κύριος ο Θεός των δυνάμεων λέγει, Εγώ βδελύττομαι την έπαρσιν του Ιακώβ και εμίσησα τα παλάτια αυτού· διά τούτο θέλω παραδώσει την πόλιν και το πλήρωμα αυτής.
\par 9 Και δέκα άνθρωποι εάν εναπολειφθώσιν εν μιά οικία, θέλουσιν αποθάνει.
\par 10 Και ο σηκόνων έκαστον αυτών θείος ή ο καίων αυτόν, διά να εκβάλη τα οστά εκ του οίκου, θέλει ειπεί προς τον ευρισκόμενον εις τα ενδότερα της οικίας, Είναί τις έτι μετά σου; Και αυτός θέλει ειπεί, Ουχί. Τότε θέλει ειπεί, Σιώπα· διότι δεν είναι πλέον καιρός να αναφέρωμεν το όνομα του Κυρίου.
\par 11 Διότι ιδού, ο Κύριος προστάττει και θέλει πατάξει τον οίκον τον μέγαν με συντριμμούς και τον οίκον τον μικρόν με διαρρήξεις.
\par 12 Δύνανται να τρέξωσιν οι ίπποι επί βράχον; δύναταί τις να αροτριάση εκεί με βόας; σεις όμως μετεστρέψατε την κρίσιν εις χολήν και τον καρπόν της δικαιοσύνης εις αψίνθιον·
\par 13 σεις οι ευφραινόμενοι εις μηδαμινά, οι λέγοντες, Δεν απεκτήσαμεν εις εαυτούς δόξαν διά της δυνάμεως ημών;
\par 14 Αλλ' ιδού, εγώ θέλω επαναστήσει έθνος εναντίον σας, οίκος Ισραήλ, λέγει Κύριος ο Θεός των δυνάμεων· και θέλουσι σας καταθλίψει από εισόδου Αιμάθ έως του ποταμού της ερήμου.

\chapter{7}

\par Ούτως έδειξεν εις εμέ Κύριος ο Θεός· και ιδού, εμόρφωσεν ακρίδας εν τη αρχή της βλαστήσεως του δευτέρου χόρτου, και ιδού, ήτο ο δεύτερος χόρτος μετά τον θερισμόν του βασιλέως.
\par 2 Και ότε ετελείωσαν να τρώγωσι τον χόρτον της γης, τότε είπα, Κύριε Θεέ, γενού ίλεως, δέομαι· τις θέλει αναστήσει τον Ιακώβ; διότι είναι ολιγοστός.
\par 3 Ο Κύριος μετεμελήθη εις τούτο· δεν θέλει γείνει, λέγει Κύριος.
\par 4 Ούτως έδειξεν εις εμέ Κύριος ο Θεός· και ιδού, Κύριος ο Θεός καλεί εις δίκην διά πυρός και το πυρ κατέφαγε την άβυσσον την μεγάλην και κατέφαγε μέρος της γης.
\par 5 Τότε είπα, Κύριε Θεέ, παύσον, δέομαι· τις θέλει αναστήσει τον Ιακώβ; διότι είναι ολιγοστός.
\par 6 Ο Κύριος μετεμελήθη εις τούτο· Και τούτο δεν θέλει γείνει, λέγει Κύριος ο Θεός.
\par 7 Ούτως έδειξεν εις εμέ, και ιδού, ο Κύριος ίστατο επί τοίχου εκτισμένου με στάθμην, έχων εν τη χειρί αυτού στάθμην.
\par 8 Και είπε Κύριος προς εμέ, Τι βλέπεις συ, Αμώς; Και είπα, Στάθμην. Τότε είπεν ο Κύριος, Ιδού, εγώ θέλω βάλει στάθμην εις το μέσον του λαού μου Ισραήλ· δεν θέλω πλέον παρατρέξει αυτόν του λοιπού.
\par 9 Και οι βωμοί του Ισαάκ θέλουσιν ερημωθή και τα αγιαστήρια του Ισραήλ θέλουσιν αφανισθή· και θέλω σηκωθή εναντίον του οίκου Ιεροβοάμ εν ρομφαία.
\par 10 Τότε ο Αμασίας ο ιερεύς της Βαιθήλ εξαπέστειλε προς Ιεροβοάμ τον βασιλέα του Ισραήλ, λέγων, Ο Αμώς συνώμοσεν εναντίον σου εν μέσω του οίκου Ισραήλ· ο τόπος δεν δύναται να υποφέρη πάντας τους λόγους αυτού·
\par 11 διότι ούτω λέγει ο Αμώς· Ο Ιεροβοάμ θέλει τελευτήσει διά ρομφαίας, ο δε Ισραήλ βεβαίως θέλει φερθή αιχμάλωτος εκ της γης αυτού.
\par 12 Τότε είπεν ο Αμασίας προς τον Αμώς, Ω συ ο βλέπων, ύπαγε, φύγε εις την γην Ιούδα και εκεί τρώγε άρτον και εκεί προφήτευε·
\par 13 εν δε τη Βαιθήλ μη προφητεύσης πλέον, διότι είναι αγιαστήριον του βασιλέως και είναι οίκος του βασιλείου.
\par 14 Και απεκρίθη ο Αμώς και είπε προς τον Αμασίαν, δεν ήμην εγώ προφήτης ουδέ υιός προφήτου εγώ, αλλ' ήμην βοσκός και συνάζων συκάμινα·
\par 15 και ο Κύριος με έλαβεν από όπισθεν του ποιμνίου και είπε Κύριος προς εμέ, Ύπαγε, προφήτευσον εις τον λαόν μου Ισραήλ.
\par 16 Τώρα λοιπόν άκουε τον λόγον του Κυρίου. Συ λέγεις, Μη προφήτευε κατά του Ισραήλ και μη στάλαζε λόγον κατά του οίκου Ισαάκ.
\par 17 Διά τούτο ούτω λέγει Κύριος· Η γυνή σου θέλει είσθαι πόρνη εν τη πόλει, και οι υιοί σου και αι θυγατέρες σου θέλουσι πέσει διά ρομφαίας, και η γη σου θέλει μερισθή διά σχοινίου, και συ θέλεις τελευτήσει εν γη ακαθάρτω· ο δε Ισραήλ βεβαίως θέλει φερθή αιχμάλωτος εκ της γης αυτού.

\chapter{8}

\par Ούτως έδειξεν εις εμέ Κύριος ο Θεός· και ιδού, κάνιστρον καρπού θερινού.
\par 2 Και είπε, Τι βλέπεις συ, Αμώς; Και είπα, Κάνιστρον καρπού θερινού. Τότε είπε Κύριος προς εμέ, Ήλθε το τέλος επί τον λαόν μου Ισραήλ· δεν θέλω πλέον παρατρέξει αυτόν του λοιπού.
\par 3 Και τα άσματα του ναού θέλουσιν είσθαι ολολυγμοί εν τη ημέρα εκείνη, λέγει Κύριος ο Θεός· πολλά πτώματα θέλουσιν είσθαι εν παντί τόπω· θέλουσιν εκρίψει αυτά εν σιωπή.
\par 4 Ακούσατε τούτο, οι ροφούντες τους πένητας και οι αφανίζοντες τους πτωχούς του τόπου,
\par 5 λέγοντες, Πότε θέλει παρέλθει ο μην, διά να πωλήσωμεν γεννήματα; και το σάββατον, διά να ανοίξωμεν σίτον, σμικρύνοντες το εφά και μεγαλύνοντες τον σίκλον και νοθεύοντες τα ζύγια της απάτης;
\par 6 διά να αγοράσωμεν τους πτωχούς με αργύριον και τον πένητα διά ζεύγος υποδημάτων, και να πωλήσωμεν τα σκύβαλα του σίτου;
\par 7 Ο Κύριος ώμοσεν εις την δόξαν του Ιακώβ, λέγων, Βεβαίως δεν θέλω λησμονήσει ποτέ ουδέν εκ των έργων αυτών.
\par 8 Η γη δεν θέλει ταραχθή διά τούτο και πενθήσει πας ο κατοικών εν αυτή; και δεν θέλει υπερεκχειλίσει όλη ως ποταμός και δεν θέλει απορριφθή και καταποντισθή ως υπό του ποταμού της Αιγύπτου;
\par 9 Και εν τη ημέρα εκείνη, λέγει Κύριος ο Θεός, θέλω κάμει τον ήλιον να δύση εν καιρώ μεσημβρίας και θέλω συσκοτάσει την γην εν φωτεινή ημέρα.
\par 10 Και θέλω μεταστρέψει τας εορτάς σας εις πένθος και πάντα τα άσματά σας εις θρήνον, και θέλω αναβιβάσει σάκκον επί πάσαν οσφύν και φαλάκρωμα επί πάσαν κεφαλήν, και θέλω καταστήσει αυτόν ως τον πενθούντα υιόν μονογενή και το τέλος αυτού θέλει είσθαι ως ημέρα πικρίας.
\par 11 Ιδού, έρχονται ημέραι, λέγει Κύριος ο Θεός, και θέλω εξαποστείλει πείναν επί την γήν· ουχί πείναν άρτου ουδέ δίψαν ύδατος, αλλ' ακροάσεως των λόγων του Κυρίου.
\par 12 Και θέλουσι περιπλανάσθαι από θαλάσσης έως θαλάσσης, και από βορρά έως ανατολής θέλουσι περιτρέχει, ζητούντες τον λόγον του Κυρίου, και δεν θέλουσιν ευρεί.
\par 13 Εν τη ημέρα εκείνη θέλουσι λιποθυμήσει αι ώραίαι παρθένοι και οι νεανίσκοι υπό δίψης,
\par 14 οι ομνύοντες εις την αμαρτίαν της Σαμαρείας και οι λέγοντες, Ζη ο Θεός σου, Δαν, και, Ζη η οδός της Βηρσαβεέ, και θέλουσι πέσει και δεν θέλουσι σηκωθή πλέον.

\chapter{9}

\par Είδον τον Κύριον ιστάμενον επί του θυσιαστηρίου, και είπε, Πάταξον το ανώφλιον της πύλης, διά να σεισθώσι τα προπύλαια, και σύντριψον αυτά κατά της κεφαλής πάντων τούτων· τους δε υπολοίπους αυτών θέλω θανατώσει εν ρομφαία· ουδείς εξ αυτών φεύγων θέλει διαφύγει και ουδείς εξ αυτών σωζόμενος θέλει διασωθή.
\par 2 Εάν σκάψωσιν έως άδου, εκείθεν η χειρ μου θέλει ανασπάσει αυτούς· και εάν αναβώσιν εις τον ουρανόν, εκείθεν θέλω κατάξει αυτούς.
\par 3 Και εάν κρυφθώσιν εν τη κορυφή του Καρμήλου, εκείθεν θέλω εξερευνήσει και συλλάβει αυτούς· και εάν κρυφθώσιν από των οφθαλμών μου εις τα βάθη της θαλάσσης, εκεί θέλω προστάξει τον δράκοντα και θέλει δαγκάσει αυτούς.
\par 4 Και εάν υπάγωσιν εις αιχμαλωσίαν έμπροσθεν των εχθρών αυτών, εκείθεν θέλω προστάξει την μάχαιραν και θέλει θανατώσει αυτούς· και θέλω στήσει τους οφθαλμούς μου επ' αυτούς διά κακόν και ουχί διά καλόν.
\par 5 Διότι Κύριος ο Θεός των δυνάμεων είναι, όστις εγγίζει την γην και τήκεται, και πάντες οι κατοικούντες εν αυτή θέλουσι πενθήσει· και θέλει υπερεκχειλίσει όλη ως ποταμός και θέλει καταποντισθή ως υπό του ποταμού της Αιγύπτου.
\par 6 Αυτός είναι ο οικοδομών τα υπερώα αυτού εν τω ουρανώ και θεμελιών τον θόλον αυτού επί της γης, ο προσκαλών τα ύδατα της θαλάσσης και εκχέων αυτά επί το πρόσωπον της γής· Κύριος το όνομα αυτού.
\par 7 δεν είσθε εις εμέ ως υιοί Αιθιόπων, σεις υιοί Ισραήλ; λέγει Κύριος· δεν ανεβίβασα τον Ισραήλ εκ γης Αιγύπτου και τους Φιλισταίους από Καφθόρ και τους Συρίους από Κιρ;
\par 8 Ιδού, οι οφθαλμοί Κυρίου του Θεού είναι επί το βασίλειον το αμαρτωλόν, και θέλω αφανίσει αυτό από προσώπου της γής· πλην ότι δεν θέλω αφανίσει ολοτελώς τον οίκον Ιακώβ, λέγει Κύριος.
\par 9 Διότι ιδού, εγώ θέλω προστάξει και θέλω λικμήσει τον οίκον Ισραήλ μεταξύ πάντων των εθνών, καθώς λικμάται ο σίτος εν τω κοσκίνω, και δεν θέλει πέσει κόκκος επί την γην.
\par 10 Υπό ρομφαίας θέλουσιν αποθάνει πάντες οι αμαρτωλοί του λαού μου, οι λέγοντες, Δεν θέλει μας εγγίσει ουδέ μας καταφθάσει το κακόν.
\par 11 Εν τη ημέρα εκείνη θέλω αναστήσει την σκηνήν του Δαβίδ την πεπτωκυίαν, και θέλω φράξει τας χαλάστρας αυτής, και θέλω ανεγείρει τα ερείπια αυτής, και θέλω ανοικοδομήσει αυτήν ως εν ταις αρχαίαις ημέραις·
\par 12 διά να κληρονομήσωσι το υπόλοιπον του Εδώμ και πάντα τα έθνη, επί τα οποία καλείται το όνομά μου, λέγει Κύριος, ο ποιών ταύτα.
\par 13 Ιδού, έρχονται ημέραι, λέγει Κύριος, και ο αροτρεύς θέλει φθάσει τον θεριστήν και ο ληνοβάτης τον σπείροντα τον σπόρον, και τα όρη θέλουσι σταλάξει γλεύκος και πάντες οι βουνοί θέλουσι ρέει αγαθά.
\par 14 Και θέλω επιστρέψει τους αιχμαλώτους του λαού μου Ισραήλ, και θέλουσιν ανοικοδομήσει τας πόλεις τας ηρημωμένας και κατοικήσει· και θέλουσι φυτεύσει αμπελώνας και πίει τον οίνον αυτών, και θέλουσι κάμει κήπους και φάγει τον καρπόν αυτών.
\par 15 Και θέλω φυτεύσει αυτούς επί την γην αυτών, και δεν θέλουσιν εκσπασθή πλέον από της γης αυτών, την οποίαν έδωκα εις αυτούς, λέγει Κύριος ο Θεός σου.


\end{document}