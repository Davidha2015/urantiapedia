\begin{document}

\title{Ιωνάς}


\chapter{1}

\par 1 Και έγεινε λόγος Κυρίου προς Ιωνάν τον υιόν του Αμαθί, λέγων,
\par 2 Σηκώθητι, ύπαγε εις Νινευή, την πόλιν την μεγάλην, και κήρυξον κατ' αυτής· διότι η ασέβεια αυτών ανέβη ενώπιόν μου.
\par 3 Και εσηκώθη ο Ιωνάς διά να φύγη εις Θαρσείς από προσώπου Κυρίου και κατέβη εις Ιόππην· και εύρηκε πλοίον πορευόμενον εις Θαρσείς, και έδωκε τον ναύλον αυτού και επέβη εις αυτό, διά να υπάγη μετ' αυτών εις Θαρσείς από προσώπου Κυρίου.
\par 4 Αλλ' ο Κύριος εξήγειρεν άνεμον μέγαν επί την θάλασσαν, και έγεινε κλύδων μέγας εν τη θαλάσση και το πλοίον εκινδύνευε να συντριφθή.
\par 5 Και εφοβήθησαν οι ναύται και ανεβόησαν έκαστος προς τον θεόν αυτού και έκαμον εκβολήν των εν τω πλοίω σκευών εις την θάλασσαν, διά να ελαφρωθή απ' αυτών· ο δε Ιωνάς κατέβη εις το κοίλωμα του πλοίου και επλαγίασε και εκοιμάτο βαθέως.
\par 6 Και επλησίασε προς αυτόν ο πλοίαρχος και είπε προς αυτόν, Τι κοιμάσαι συ; σηκώθητι, επικαλού τον Θεόν σου, ίσως ο Θεός μας ενθυμηθή και δεν χαθώμεν.
\par 7 Και είπον έκαστος προς τον πλησίον αυτού, Έλθετε και ας ρίψωμεν κλήρους, διά να γνωρίσωμεν τίνος ένεκεν το κακόν τούτο είναι εφ' ημάς. Και έρριψαν κλήρους και έπεσεν ο κλήρος επί τον Ιωνάν.
\par 8 Τότε είπον προς αυτόν, Ειπέ τώρα προς ημάς, τίνος ένεκεν το κακόν τούτο ήλθεν εφ' ημάς; Τι είναι το έργον σου; και πόθεν έρχεσαι; τις ο τόπος σου; και εκ τίνος λαού είσαι;
\par 9 Ο δε είπε προς αυτούς, Εγώ είμαι Εβραίος· και σέβομαι Κύριον τον Θεόν του ουρανού, όστις εποίησε την θαλάσσαν και την ξηράν.
\par 10 Τότε εφοβήθησαν οι άνθρωποι φόβον μέγαν και είπον προς αυτόν, Τι είναι τούτο, το οποίον έκαμες; διότι εγνώρισαν οι άνθρωποι, ότι έφευγεν από προσώπου Κυρίου, επειδή είχεν αναγγείλει τούτο προς αυτούς.
\par 11 Και είπον προς αυτόν, Τι να σε κάμωμεν, διά να ησυχάση η θάλασσα αφ' ημών; διότι η θάλασσα εκλυδωνίζετο επί το μάλλον.
\par 12 Και είπε προς αυτούς, Σηκώσατέ με και ρίψατέ με εις την θάλασσαν, και η θάλασσα θέλει ησυχάσει αφ' υμών· διότι εγώ γνωρίζω, ότι εξ αιτίας εμού έγεινεν ο μέγας ούτος κλύδων εφ' υμάς.
\par 13 Οι άνθρωποι όμως εκωπηλάτουν δυνατά διά να επιστρέψωσι προς την ξηράν· αλλά δεν εδύναντο, διότι η θάλασσα εκλυδωνίζετο επί το μάλλον κατ' αυτών.
\par 14 Όθεν ανεβόησαν προς τον Κύριον και είπον, Δεόμεθα, Κύριε, δεόμεθα, ας μη χαθώμεν διά την ζωήν του ανθρώπου τούτου και μη επιβάλης εφ' ημάς αίμα αθώον· διότι συ, Κύριε, έκαμες ως ήθελες.
\par 15 Και εσήκωσαν τον Ιωνάν και έρριψαν αυτόν εις την θάλασσαν και η θάλασσα εστάθη από του θυμού αυτής.
\par 16 Τότε οι άνθρωποι εφοβήθησαν τον Κύριον φόβον μέγαν και προσέφεραν θυσίαν εις τον Κύριον και έκαμον ευχάς.
\par 17 Και διέταξε Κύριος μέγα κήτος να καταπίη τον Ιωνάν. Και ήτο ο Ιωνάς εν τη κοιλία του κήτους τρεις ημέρας και τρεις νύκτας.

\chapter{2}

\par 1 Και προσηυχήθη Ιωνάς προς Κύριον τον Θεόν αυτού εκ της κοιλίας του κήτους,
\par 2 Και είπεν, Εβόησα εν τη θλίψει μου προς τον Κύριον, και εισήκουσέ μου· εκ κοιλίας άδου εβόησα, και ήκουσας της φωνής μου.
\par 3 Διότι με έρριψας εις τα βάθη, εις την καρδίαν της θαλάσσης, και ρεύματα με περιεκύκλωσαν· πάσαι αι τρικυμίαι σου και τα κύματά σου διήλθον επάνωθέν μου.
\par 4 Και εγώ είπα, Απερρίφθην απ' έμπροσθεν των οφθαλμών σου· όμως θέλω επιβλέψει πάλιν εις τον ναόν τον άγιόν σου.
\par 5 Τα ύδατα με περιεκύκλωσαν έως της ψυχής, η άβυσσος με περιέκλεισε, τα φύκια περιετυλίχθησαν περί την κεφαλήν μου.
\par 6 Κατέβην εις τα έσχατα των ορέων· οι μοχλοί της γης είναι επάνωθέν μου διαπαντός· αλλ' ανέβη η ζωή μου από της φθοράς, Κύριε Θεέ μου·
\par 7 Ενώ ήτο εκλείπουσα εν εμοί η ψυχή μου, ενεθυμήθην τον Κύριον· και η προσευχή μου εισήλθε προς σε, εις τον ναόν τον άγιόν σου.
\par 8 Οι φυλάττοντες ματαιότητας ψεύδους εγκαταλείπουσι το έλεος αυτών.
\par 9 Αλλ' εγώ θέλω θυσιάσει προς σε μετά φωνής αινέσεως· θέλω αποδώσει όσα ηυχήθην· η σωτηρία είναι παρά του Κυρίου.
\par 10 Και προσέταξεν ο Κύριος το κήτος και εξήμεσε τον Ιωνάν επί την ξηράν.

\chapter{3}

\par 1 Και έγεινε λόγος Κυρίου προς Ιωνάν εκ δευτέρου, λέγων,
\par 2 Σηκώθητι, ύπαγε εις Νινευή, την πόλιν την μεγάλην, και κήρυξον προς αυτήν το κήρυγμα, το οποίον εγώ λαλώ προς σε.
\par 3 Και εσηκώθη ο Ιωνάς και υπήγεν εις Νινευή κατά τον λόγον του Κυρίου. Η δε Νινευνή ήτο πόλις μεγάλη σφόδρα, οδού τριών ημερών·
\par 4 Και ήρχισεν ο Ιωνάς να διέρχηται εις την πόλιν οδόν μιας ημέρας και εκήρυξε και είπεν, Έτι τεσσαράκοντα ημέραι και η Νινευή θέλει καταστραφή.
\par 5 Και οι άνδρες της Νινευή επίστευσαν εις τον Θεόν και εκήρυξαν νηστείαν και ενεδύθησαν σάκκους από μεγάλου αυτών έως μικρού αυτών·
\par 6 διότι ο λόγος είχε φθάσει προς τον βασιλέα της Νινευή και εσηκώθη από του θρόνου αυτού και αφήρεσε την στολήν αυτού επάνωθεν εαυτού και εσκεπάσθη με σάκκον και εκάθησεν επί σποδού.
\par 7 Και διεκηρύχθη και εγνωστοποιήθη εν τη Νινευή διά ψηφίσματος του βασιλέως και των μεγιστάνων αυτού και ελαλήθη, οι άνθρωποι και τα κτήνη, οι βόες και τα πρόβατα, να μη γευθώσι μηδέν, μηδέ να βοσκήσωσι, μηδέ ύδωρ να πίωσιν·
\par 8 αλλ' άνθρωπος και κτήνος να σκεπασθώσι με σάκκους και να φωνάξωσιν ισχυρώς προς τον Θεόν· και ας επιστρέψωσιν έκαστος από της οδού αυτού της πονηράς και από της αδικίας, ήτις είναι εν ταις χερσίν αυτών.
\par 9 Τις εξεύρει αν επιστρέψη και μεταμεληθή ο Θεός και επιστρέψη από της οργής του θυμού αυτού και δεν απολεσθώμεν;
\par 10 Και είδεν ο Θεός τα έργα αυτών, ότι επέστρεψαν από της οδού αυτών της πονηράς· και μετεμελήθη ο Θεός περί του κακού, το οποίον είπε να κάμη εις αυτούς· και δεν έκαμεν αυτό.

\chapter{4}

\par 1 Και ελυπήθη ο Ιωνάς λύπην μεγάλην και ηγανάκτησε.
\par 2 Και προσηυχήθη προς τον Κύριον και είπεν, Ω Κύριε, δεν ήτο ούτος ο λόγος μου, ενώ έτι ήμην εν τη πατρίδι μου; διά τούτο προέλαβον να φύγω εις Θαρσείς· διότι εγνώριζον ότι συ είσαι Θεός ελεήμων και οικτίρμων, μακρόθυμος και πολυέλεος και μετανοών διά το κακόν.
\par 3 Και τώρα, Κύριε, λάβε, δέομαί σου, την ψυχήν μου απ' εμού· διότι είναι κάλλιον εις εμέ να αποθάνω παρά να ζω.
\par 4 Και είπε Κύριος, Είναι καλόν να αγανακτής;
\par 5 Και εξήλθεν Ιωνάς από της πόλεως και εκάθησε κατά το ανατολικόν μέρος της πόλεως, και εκεί έκαμεν εις εαυτόν καλύβην και εκάθητο υποκάτω αυτής εν τη σκιά, εωσού ίδη τι έμελλε να γείνη εις την πόλιν.
\par 6 Και διέταξε Κύριος ο Θεός κολοκύνθην και έκαμε να αναβή επάνωθεν του Ιωνά, διά να ήναι σκιά υπεράνω της κεφαλής αυτού, διά να ανακουφίση αυτόν από της θλίψεως αυτού. Και εχάρη ο Ιωνάς διά την κολοκύνθην χαράν μεγάλην.
\par 7 Και διέταξεν ο Θεός σκώληκα, ότε εχάραξεν η αυγή της επαύριον· και επάταξε την κολοκύνθην και εξηράνθη.
\par 8 Και καθώς ανέτειλεν ο ήλιος, διέταξεν ο Θεός άνεμον ανατολικόν καυστικόν· και προσέβαλεν ο ήλιος επί την κεφαλήν του Ιωνά, ώστε ώλιγοψύχησε· και εζήτησεν εν τη ψυχή αυτού να αποθάνη, και είπεν, Είναι κάλλιον εις εμέ να αποθάνω παρά να ζω.
\par 9 Και είπεν ο Θεός προς τον Ιωνάν, είναι καλόν να αγανακτής διά την κολοκύνθην; Και είπε, Καλόν είναι να αγανακτώ έως θανάτου.
\par 10 Και είπε Κύριος, Συ ελυπήθης υπέρ της κολοκύνθης, διά την οποίαν δεν εκοπίασας, αλλ' ουδέ έκαμες αυτήν να αυξήση, ήτις εγεννήθη εν μιά νυκτί και εν μιά νυκτί εχάθη.
\par 11 Και εγώ δεν έπρεπε να λυπηθώ υπέρ της Νινευή, της πόλεως της μεγάλης, εν ή υπάρχουσι πλειότεροι των δώδεκα μυριάδων ανθρώπων, οίτινες δεν διακρίνουσι την δεξιάν αυτών από της αριστεράς αυτών, και κτήνη πολλά;


\end{document}