\begin{document}

\title{Μιχαίας}


\chapter{1}

\par 1 Ο λόγος του Κυρίου ο γενόμενος προς Μιχαίαν τον Μωρασθίτην εν ταις ημέραις Ιωάθαμ, Άχαζ και Εζεκίου, βασιλέων του Ιούδα, τον οποίον είδε περί Σαμαρείας και Ιερουσαλήμ.
\par 2 Ακούσατε, πάντες οι λαοί· πρόσεχε, γη, και το πλήρωμα αυτής, και ας ήναι Κύριος ο Θεός μάρτυς εις εσάς, ο Κύριος εκ του ναού του αγίου αυτού.
\par 3 Διότι ιδού, ο Κύριος εξέρχεται εκ του τόπου αυτού και θέλει καταβή και πατήσει επί τα ύψη της γης.
\par 4 Και τα όρη θέλουσιν αναλύσει υποκάτω αυτού και αι κοιλάδες θέλουσι διασχισθή ως κηρός από προσώπου πυρός και ως ύδατα καταφερόμενα εις κατήφορον.
\par 5 Διά την ασέβειαν του Ιακώβ είναι άπαν τούτο και διά τας αμαρτίας του οίκου Ισραήλ. Τις είναι η ασέβεια του Ιακώβ; ουχί η Σαμάρεια; και τίνες οι υψηλοί τόποι του Ιούδα; ουχί η Ιερουσαλήμ;
\par 6 Διά τούτο θέλω καταστήσει την Σαμάρειαν εις σωρούς λίθων αγρού, όπου φυτεύεται αμπελών, και θέλω κατακυλίσει τους λίθους αυτής εις την κοιλάδα και ανακαλύψει τα θεμέλια αυτής.
\par 7 Και πάντα τα γλυπτά αυτής θέλουσι κατακοπή, και πάντα τα μισθώματα αυτής θέλουσι κατακαή εν πυρί, και πάντα τα είδωλα αυτής θέλω εξαφανίσει· διότι από μισθού πορνείας συνήγαγεν αυτά και εις μισθόν πορνείας θέλουσιν επιστρέψει.
\par 8 Διά τούτο θέλω θρηνήσει και ολολύξει, θέλω υπάγει εκδεδυμένος και γυμνός, θέλω κάμει θρήνον ως θώων και πένθος ως στρουθοκαμήλων.
\par 9 Διότι η πληγή αυτής είναι ανίατος, διότι ήλθεν έως του Ιούδα, έφθασεν έως της πύλης του λαού μου, έως της Ιερουσαλήμ.
\par 10 Μη αναγγείλητε τούτο εις Γαθ, μη πενθήσητε πένθος· εν Βηθ-αφρά κυλίσθητι εις την κόνιν.
\par 11 Διάβηθι, η κάτοικος της Σαφίρ, έχουσα γυμνήν την αισχύνην σου· η κάτοικος της Σαανάν ας μη εξέλθη· το πένθος της Βαιθ-εζήλ θέλει λάβει από σας την αρχήν αυτού.
\par 12 Διότι η κάτοικος της Μαρώθ ελυπήθη διά τα αγαθά αυτής, επειδή κατέβη κακόν από του Κυρίου εις την πύλην της Ιερουσαλήμ.
\par 13 Κάτοικε της Λαχείς, ζεύξον την άμαξαν εις τον ταχύν ίππον· συ, η αρχή της αμαρτίας εις την θυγατέρα της Σιών· διότι αι ασέβειαι του Ισραήλ εν σοι ευρέθησαν.
\par 14 Διά τούτο θέλεις δώσει έγγραφον ελευθερώσεως εις την Μορέσεθ-γάθ· οι οίκοι του Αχζίβ θέλουσι ματαιώσει τας ελπίδας των βασιλέων του Ισραήλ.
\par 15 Θέλω έτι φέρει κληρονόμον εις σε, κάτοικε της Μαρησά· θέλει ελθεί έως Οδολλάμ, της δόξης του Ισραήλ.
\par 16 Φαλακρώθητι και κείρον την κεφαλήν σου διά τα τέκνα σου τα τρυφερά· πλάτυνον την φαλακρότητά σου ως αετός, διότι ηχμαλωτίσθησαν από σου.

\chapter{2}

\par 1 Ουαί εις τους διαλογιζομένους ανομίαν και μηχανευομένους κακόν εν ταις κλίναις αυτών· μόλις φέγγει η αυγή και πράττουσιν αυτό, διότι είναι εν τη δυνάμει της χειρός αυτών.
\par 2 Και επιθυμούσιν αγρούς και λαμβάνουσι διά της βίας, και οίκους και αρπάζουσιν αυτούς· ούτω διαρπάζουσιν άνθρωπον και τον οίκον αυτού, ναι, άνθρωπον και την κληρονομίαν αυτού.
\par 3 διά τούτο ούτω λέγει Κύριος· Ιδού, εναντίον του γένους τούτου εγώ βουλεύομαι κακόν, εκ του οποίου δεν θέλετε ελευθερώσει τους τραχήλους σας ουδέ θέλετε περιπατεί υπερηφάνως, διότι ο καιρός ούτος είναι κακός.
\par 4 Εν τη ημέρα εκείνη θέλει ληφθή παροιμία εναντίον σας, και θέλει θρηνήσει ο θρηνών με θρήνον και ειπεί, Διόλου ηφανίσθημεν· ηλλοίωσε την μερίδα του λαού μου· πως απεμάκρυνεν αυτήν απ' εμού· αντί να αποδώση, διεμέρισε τους αγρούς ημών.
\par 5 Διά τούτο συ δεν θέλεις έχει τινά βάλλοντα σχοινίον διά κλήρον, εν τη συνάξει του Κυρίου.
\par 6 Μη προφητεύετε, οι προφητεύοντες· δεν θέλουσι προφητεύσει εις αυτούς· η αισχύνη αυτών δεν θέλει απομακρυνθή.
\par 7 Ω συ, ο καλούμενος οίκος Ιακώβ, εσμικρύνθη το πνεύμα του Κυρίου; είναι τοιαύτα τα επιτηδεύματα αυτού; οι λόγοι μου δεν κάμνουσι καλόν εις τον ορθώς περιπατούντα;
\par 8 Και πρότερον ο λαός μου επανέστη ως εχθρός· το επένδυμα μετά του χιτώνος αρπάζετε από των διαβαινόντων αφόβως, των επιστρεφόντων από του πολέμου.
\par 9 Τας γυναίκας του λαού μου εξώσατε από των τερπνών αυτών οίκων· από των τέκνων αυτών αφηρέσατε την δόξαν μου διαπαντός.
\par 10 Σηκώθητε και αναχωρήσατε, διότι αύτη δεν είναι η ανάπαυσίς σας· επειδή εμιάνθη, θέλει σας αφανίσει, μάλιστα εν σκληρώ αφανισμώ.
\par 11 Εάν τις περιπατή κατά το πνεύμα αυτού και λαλή ψεύδη, λέγων, Θέλω προφητεύσει εις σε περί οίνου και σίκερα, ούτος βεβαίως θέλει είσθαι ο προφήτης του λαού τούτου.
\par 12 Θέλω βεβαίως σε συνάξει όλον Ιακώβ· θέλω βεβαίως συλλέξει το υπόλοιπον του Ισραήλ· θέλω θέσει αυτούς ομού ως πρόβατα της Βοσόρρας, ως ποίμνιον εν μέσω της μάνδρας αυτών· μέγαν θόρυβον θέλουσι κάμει εκ του πλήθους των ανθρώπων.
\par 13 Ο διαρρηγνύων ανέβη έμπροσθεν αυτών· διέρρηξαν και διέβησαν διά της πύλης και εξήλθον δι' αυτής· και ο βασιλεύς αυτών θέλει διαβή έμπροσθεν αυτών και ο Κύριος επί κεφαλής αυτών.

\chapter{3}

\par 1 Και είπα, Ακούσατε τώρα, αρχηγοί του Ιακώβ και άρχοντες του οίκου Ισραήλ· δεν ανήκει εις εσάς να γνωρίζητε την κρίσιν;
\par 2 Οι μισούντες το καλόν και αγαπώντες το κακόν, οι αποσπώντες το δέρμα αυτών επάνωθεν αυτών και την σάρκα αυτών από των οστών αυτών,
\par 3 οι κατατρώγοντες έτι την σάρκα του λαού μου και εκδείροντες το δέρμα αυτών επάνωθεν αυτών και συντρίβοντες τα οστά αυτών και κατακόπτοντες αυτά ως διά χύτραν και ως κρέας εν μέσω λέβητος.
\par 4 Τότε θέλουσι βοήσει προς τον Κύριον, πλην δεν θέλει εισακούσει αυτούς· θέλει μάλιστα κρύψει το πρόσωπον αυτού απ' αυτών εν τω καιρώ εκείνω, διότι εφέρθησαν κακώς εις τας πράξεις αυτών.
\par 5 Ούτω λέγει Κύριος περί των προφητών, οίτινες πλανώσι τον λαόν μου, οίτινες δαγκάνοντες διά των οδόντων αυτών φωνάζουσιν, Ειρήνη· και εάν τις δεν βάλλη τι εις το στόμα αυτών, κηρύττουσιν εναντίον αυτού πόλεμον.
\par 6 Διά τούτο νυξ θέλει είσθαι εις εσάς αντί οράσεως και σκότος εις εσάς αντί μαντείας· και ο ήλιος θέλει δύσει επί τους προφήτας και η ημέρα θέλει συσκοτάσει επ' αυτούς.
\par 7 Τότε θέλουσι καταισχυνθή οι βλέποντες και θέλουσιν εντραπή οι μάντεις· και θέλουσι σκεπάσει τα χείλη αυτών πάντες ούτοι, διότι δεν είναι απόκρισις Θεού.
\par 8 Αλλ' εγώ βεβαίως είμαι πλήρης δυνάμεως διά του πνεύματος του Κυρίου και κρίσεως και ισχύος, διά να απαγγείλω εις τον Ιακώβ την παράβασιν αυτού και εις τον Ισραήλ την αμαρτίαν αυτού.
\par 9 Ακούσατε λοιπόν τούτο, αρχηγοί του οίκου Ιακώβ και άρχοντες του οίκου Ισραήλ, οι βδελυττόμενοι την κρίσιν και διαστρέφοντες πάσαν ευθύτητα,
\par 10 οι οικοδομούντες την Σιών εν αίματι και την Ιερουσαλήμ εν ανομία.
\par 11 Οι άρχοντες αυτής κρίνουσι με δώρα και οι ιερείς αυτής διδάσκουσιν επί μισθώ και οι προφήται αυτής μαντεύουσιν επί αργυρίω και επαναπαύονται επί τον Κύριον, λέγοντες, Δεν είναι ο Κύριος εν μέσω ημών; κακόν δεν θέλει ελθεί εφ' ημάς.
\par 12 Διά τούτο η Σιών εξ αιτίας σας θέλει αροτριασθή ως αγρός, και η Ιερουσαλήμ θέλει γείνει σωροί λίθων, και το όρος του οίκου ως υψηλοί τόποι δρυμού.

\chapter{4}

\par 1 Και εν ταις εσχάταις ημέραις το όρος του οίκου του Κυρίου θέλει στηριχθή επί της κορυφής των ορέων και υψωθή υπεράνω των βουνών, και λαοί θέλουσι συρρέει εις αυτό.
\par 2 Και έθνη πολλά θέλουσιν υπάγει και ειπεί, Έλθετε και ας αναβώμεν εις το όρος του Κυρίου και εις τον οίκον του Θεού του Ιακώβ· και θέλει διδάξει ημάς τας οδούς αυτού, και θέλομεν περιπατήσει εν ταις τρίβοις αυτού· διότι εκ Σιών θέλει εξέλθει νόμος και λόγος Κυρίου εξ Ιερουσαλήμ.
\par 3 Και θέλει κρίνει αναμέσον λαών πολλών και θέλει ελέγξει έθνη ισχυρά, έως εις μακράν· και θέλουσι σφυρηλατήσει τας μαχαίρας αυτών διά υνία και τας λόγχας αυτών διά δρέπανα· δεν θέλει σηκώσει μάχαιραν έθνος εναντίον έθνους ουδέ θέλουσι μάθει πλέον τον πόλεμον.
\par 4 Και θέλουσι κάθησθαι έκαστος υπό την άμπελον αυτού και υπό την συκήν αυτού, και δεν θέλει υπάρχει ο εκφοβών· διότι το στόμα του Κυρίου των δυνάμεων ελάλησε.
\par 5 Διότι πάντες οι λαοί θέλουσι περιπατεί έκαστος εν τω ονόματι του θεού αυτού· ημείς δε θέλομεν περιπατεί εν τω ονόματι Κυρίου του Θεού ημών εις τον αιώνα και εις τον αιώνα.
\par 6 Εν τη ημέρα εκείνη, λέγει Κύριος, θέλω συνάξει την χωλαίνουσαν και θέλω εισδεχθή την εξωσμένην και εκείνην, την οποίαν έθλιψα.
\par 7 Και θέλω κάμει την χωλαίνουσαν υπόλοιπον και την αποβεβλημένην έθνος ισχυρόν, και ο Κύριος θέλει βασιλεύει επ' αυτούς εν τω όρει Σιών, από του νυν και έως του αιώνος.
\par 8 Και συ, πύργε του ποιμνίου, οχύρωμα της θυγατρός Σιών, εις σε θέλει ελθεί η πρώτη εξουσία· ναι, θέλει ελθεί το βασίλειον εις την θυγατέρα της Ιερουσαλήμ.
\par 9 Διά τι τώρα κραυγάζεις δυνατά; δεν είναι βασιλεύς εν σοι; ηφανίσθη ο σύμβουλός σου, ώστε σε κατέλαβον ωδίνες ως τικτούσης;
\par 10 Κοιλοπόνει και αγωνίζου, θυγάτηρ Σιών, ως η τίκτουσα, διότι τώρα θέλεις εξέλθει εκ της πόλεως και θέλεις κατοικήσει εν αγρώ και θέλεις υπάγει έως της Βαβυλώνος· εκεί θέλεις ελευθερωθή, εκεί θέλει σε εξαγοράσει ο Κύριος εκ της χειρός των εχθρών σου.
\par 11 Τώρα δε συνήχθησαν εναντίον σου έθνη πολλά λέγοντα, Ας μιανθή και ας επιβλέπη ο οφθαλμός ημών επί την Σιών.
\par 12 Αλλ' αυτοί δεν γνωρίζουσι τους λογισμούς του Κυρίου ουδέ εννοούσι την βουλήν αυτού, ότι συνήγαγεν αυτούς ως δράγματα αλωνίου.
\par 13 Σηκώθητι και αλώνιζε, θυγάτηρ Σιών, διότι θέλω κάμει το κέρας σου σιδηρούν και τας οπλάς σου θέλω κάμει χαλκάς, και θέλεις κατασυντρίψει λαούς πολλούς· και θέλω αφιερώσει τα διαρπάγματα αυτών εις τον Κύριον και την περιουσίαν αυτών εις τον Κύριον πάσης της γης.

\chapter{5}

\par 1 Συναθροίσθητι τώρα εις τάγματα, θυγάτηρ ταγμάτων· έθεσε πολιορκίαν εναντίον ημών· θέλουσι πατάξει τον κριτήν του Ισραήλ εν ράβδω κατά της σιαγόνος.
\par 2 Και συ, Βηθλεέμ Εφραθά, η μικρά ώστε να ήσαι μεταξύ των χιλιάδων του Ιούδα, εκ σου θέλει εξέλθει εις εμέ ανήρ διά να ήναι ηγούμενος εν τω Ισραήλ· του οποίου αι έξοδοι είναι απ' αρχής, από ημερών αιώνος.
\par 3 Διά τούτο θέλει αφήσει αυτούς, έως του καιρού καθ' ον η τίκτουσα θέλει γεννήσει· τότε το υπόλοιπον των αδελφών αυτού θέλει επιστρέψει εις τους υιούς Ισραήλ.
\par 4 Και θέλει σταθή και ποιμάνει εν τη ισχύϊ του Κυρίου, εν τη μεγαλειότητι του ονόματος Κυρίου του Θεού αυτού· και θέλουσι κατοικήσει· διότι τώρα θέλει μεγαλυνθή έως των άκρων της γης.
\par 5 Και ούτος θέλει είσθαι ειρήνη. Όταν ο Ασσύριος έλθη εις την γην ημών και όταν πατήση εις τα παλάτια ημών, τότε θέλομεν επεγείρει κατ' αυτού επτά ποιμένας και οκτώ άρχοντας ανθρώπων·
\par 6 και θέλουσι ποιμάνει την γην της Ασσυρίας εν ρομφαία και την γην του Νεβρώδ εν ταις εισόδοις αυτού· και θέλει ελευθερώσει ημάς εκ του Ασσυρίου, όταν έλθη εις την γην ημών και όταν πατήση εν τοις ορίοις ημών.
\par 7 Και το υπόλοιπον του Ιακώβ θέλει είσθαι εν μέσω λαών πολλών ως δρόσος από Κυρίου, ως ρανίδες επί χόρτου, όστις δεν προσμένει παρά ανθρώπου ουδέ ελπίζει επί υιούς ανθρώπων.
\par 8 Και το υπόλοιπον του Ιακώβ θέλει είσθαι μεταξύ εθνών, εν μέσω λαών πολλών, ως λέων μεταξύ κτηνών του δρυμού, ως σκύμνος μεταξύ ποιμνίων προβάτων, όστις διαβαίνων καταπατεί και διασπαράττει και δεν υπάρχει ο ελευθερών.
\par 9 Η χειρ σου θέλει υψωθή επί τους εναντίους σου, και πάντες οι εχθροί σου θέλουσιν εκκοπή.
\par 10 Και εν τη ημέρα εκείνη, λέγει Κύριος, θέλω εξολοθρεύσει τους ίππους σου εκ μέσου σου, και θέλω απολέσει τας αμάξας σου.
\par 11 Και θέλω εξολοθρεύσει τας πόλεις της γης σου, και κατεδαφίσει πάντα τα οχυρώματά σου.
\par 12 Και θέλω εξολοθρεύσει τας μαγείας από της χειρός σου, και δεν θέλεις έχει πλέον μάντεις.
\par 13 Και θέλω εξολοθρεύσει τα γλυπτά σου και τα είδωλά σου εκ μέσου σου, και δεν θέλεις λατρεύσει πλέον το έργον των χειρών σου.
\par 14 Και θέλω ανασπάσει τα άλση σου εκ μέσου σου, και θέλω αφανίσει τας πόλεις σου.
\par 15 Και θέλω κάμει εκδίκησιν μετά θυμού και μετ' οργής επί τα έθνη, τα οποία δεν μου εισήκουσαν.

\chapter{6}

\par 1 Ακούσατε τώρα ό,τι λέγει ο Κύριος Σηκώθητι, διαδικάσθητι έμπροσθεν των ορέων, και ας ακούσωσιν οι βουνοί την φωνήν σου.
\par 2 Ακούσατε, όρη, την κρίσιν του Κυρίου, και σεις, τα ισχυρά θεμέλια της γης διότι ο Κύριος έχει κρίσιν μετά του λαού αυτού και θέλει διαδικασθή μετά του Ισραήλ.
\par 3 Λαέ μου, τι σοι έκαμα; και εις τι σε παρηνώχλησα; μαρτύρησον κατ' εμού.
\par 4 Διότι σε ανεβίβασα εκ γης Αιγύπτου και σε ελύτρωσα εξ οίκου δουλείας και εξαπέστειλα έμπροσθέν σου τον Μωϋσήν, τον Ααρών και την Μαριάμ.
\par 5 Λαέ μου, ενθυμήθητι τώρα τι εβουλεύθη Βαλάκ ο βασιλεύς του Μωάβ και τι απεκρίθη προς αυτόν Βαλαάμ ο του Βεώρ από Σιττείμ έως Γαλγάλων, διά να γνωρίσητε την δικαιοσύνην του Κυρίου.
\par 6 Με τι θέλω ελθεί ενώπιον του Κυρίου, να προσκυνήσω ενώπιον του υψίστου Θεού; θέλω ελθεί ενώπιον αυτού με ολοκαυτώματα, με μόσχους ενιαυσίους;
\par 7 Θέλει ευαρεστηθή ο Κύριος εις χιλιάδας κριών ή εις μυριάδας ποταμών ελαίου; θέλω δώσει τον πρωτότοκόν μου διά την παράβασίν μου, τον καρπόν της κοιλίας μου διά την αμαρτίαν της ψυχής μου;
\par 8 Αυτός σοι έδειξεν, άνθρωπε, τι το καλόν και τι ζητεί ο Κύριος παρά σου, ειμή να πράττης το δίκαιον και να αγαπάς έλεος και να περιπατής ταπεινώς μετά του Θεού σου;
\par 9 Η φωνή του Κυρίου κράζει προς την πόλιν, και η σοφία θέλει φοβείσθαι το όνομά σου· ακούσατε την ράβδον και τις διώρισεν αυτήν.
\par 10 Υπάρχουσιν έτι οι θησαυροί της ασεβείας εν τω οίκω του ασεβούς και το ελλιπές μέτρον το βδελυκτόν;
\par 11 να δικαιώσω αυτούς με τας ασεβείς πλάστιγγας και με το σακκίον των δολίων ζυγίων;
\par 12 Διότι οι πλούσιοι αυτής είναι πλήρεις αδικίας, και οι κάτοικοι αυτής ελάλησαν ψεύδη, και η γλώσσα αυτών είναι απατηλή εν τω στόματι αυτών.
\par 13 Και εγώ λοιπόν πατάξας θέλω σε αδυνατίσει, θέλω σε ερημώσει εξ αιτίας των αμαρτιών σου.
\par 14 Συ θέλεις τρώγει και δεν θέλεις χορτάζεσθαι, και η πείνά σου θέλει είσθαι εν μέσω σου και θέλεις φύγει αλλά δεν θέλεις διασώσει, και ό,τι διέσωσας, θέλω παραδώσει εις την ρομφαίαν.
\par 15 Συ θέλεις σπείρει και δεν θέλεις θερίσει συ θέλεις πιέσει ελαίας και δεν θέλεις αλειφθή με έλαιον, και γλεύκος και δεν θέλεις πίει οίνον.
\par 16 Διότι εφυλάχθησαν τα διατάγματα του Αμρί και πάντα τα έργα του οίκου του Αχαάβ και επορεύθητε εν ταις βουλαίς αυτών διά να σε παραδώσω εις αφανισμόν και τους κατοίκους αυτής εις συριγμόν και θέλετε βαστάσει το όνειδος του λαού μου.

\chapter{7}

\par 1 Ουαί εις εμέ, διότι είμαι ως επικαρπολογία θέρους, ως επιφυλλίς τρυγητού δεν υπάρχει βότρυς διά να φάγη τις η ψυχή μου επεθύμησε τας απαρχάς των καρπών.
\par 2 Ο όσιος απωλέσθη εκ της γης και ο ευθύς δεν υπάρχει μεταξύ των ανθρώπων πάντες ενεδρεύουσι διά αίμα κυνηγούσιν έκαστος τον αδελφόν αυτού.
\par 3 Εις το να κακοποιώσιν ετοιμάζουσι τας χείρας αυτών ο άρχων απαιτεί και ο κριτής κρίνει επί μισθώ· και ο μεγάλος προφέρει την πονηράν αυτού επιθυμίαν, την οποίαν συμπεριστρεφόμενοι εκπληρούσιν.
\par 4 Ο καλήτερος αυτών είναι ως άκανθα· ο ευθύς οξύτερος φραγμού ακανθώδους· η ημέρα των φυλάκων σου, η επίσκεψίς σου έφθασε· τώρα θέλει είσθαι η αμηχανία αυτών.
\par 5 Μη εμπιστεύεσθε εις φίλον, μη θαρρείτε εις οικείον· φύλαττε τας θύρας του στόματός σου από της συγκαθευδούσης εν τω κόλπω σου·
\par 6 διότι ο υιός περιφρονεί τον πατέρα, η θυγάτηρ επανίσταται κατά της μητρός αυτής, η νύμφη κατά της πενθεράς αυτής· οι εχθροί του ανθρώπου είναι οι άνθρωποι της εαυτού οικίας.
\par 7 Εγώ δε θέλω επιβλέψει επί Κύριον· θέλω προσμείνει τον Θεόν της σωτηρίας μου· ο Θεός μου θέλει μου εισακούσει.
\par 8 Μη ευφραίνου εις εμέ, η εχθρά μου· αν και έπεσα, θέλω σηκωθή· αν και εκάθησα εν σκότει, ο Κύριος θέλει είσθαι φως εις εμέ.
\par 9 Θέλω υποφέρει την οργήν του Κυρίου, διότι ημάρτησα εις αυτόν, εωσού διαδικάση την δίκην μου και κάμη την κρίσιν μου· θέλει με εξάξει εις το φως, θέλω ιδεί την δικαιοσύνην αυτού.
\par 10 Και θέλει ιδεί η εχθρά μου, και αισχύνη θέλει περικαλύψει αυτήν, ήτις λέγει προς εμέ, Που είναι Κύριος ο Θεός σου; οι οφθαλμοί μου θέλουσιν ιδεί αυτήν· τώρα θέλει είσθαι εις καταπάτημα ως ο πηλός των οδών.
\par 11 Καθ' ην ημέραν τα τείχη σου μέλλουσι να κτισθώσι, την ημέραν εκείνην θέλει διαδοθή εις μακράν το πρόσταγμα.
\par 12 Την ημέραν εκείνην θέλουσιν ελθεί έως εις σε από της Ασσυρίας και των πόλεων της Αιγύπτου και από της Αιγύπτου έως του ποταμού και από θαλάσσης έως θαλάσσης και από όρους έως όρους.
\par 13 Και η γη θέλει ερημωθή εξ αιτίας των κατοικούντων αυτήν, διά τον καρπόν των πράξεων αυτών.
\par 14 Ποίμαινε τον λαόν σου εν τη ράβδω σου, το ποίμνιον της κληρονομίας σου, το οποίον κατοικεί μεμονωμένον εν τω δάσει, εν μέσω του Καρμήλου· ας νέμωνται την Βασάν και την Γαλαάδ καθώς εν ταις αρχαίαις ημέραις.
\par 15 Καθώς εν ταις ημέραις της εξόδου σου εκ γης Αιγύπτου θέλω δείξει εις αυτόν θαυμάσια.
\par 16 Τα έθνη θέλουσιν ιδεί και θέλουσι καταισχυνθή διά πάσαν την ισχύν αυτών· θέλουσιν επιθέσει την χείρα επί το στόμα, τα ώτα αυτών θέλουσι κωφωθή.
\par 17 Θέλουσι γλείφει το χώμα ως όφεις, ως τα ερπετά της γης θέλουσι σύρεσθαι από των τρυπών αυτών· θέλουσιν εκπλαγή εις Κύριον τον Θεόν ημών και θέλουσι φοβηθή από σου.
\par 18 Τις Θεός όμοιός σου, συγχωρών ανομίαν και παραβλέπων την παράβασιν του υπολοίπου της κληρονομίας αυτού; δεν φυλάττει την οργήν αυτού διαπαντός, διότι αυτός αρέσκεται εις το έλεος.
\par 19 Θέλει επιστρέψει, θέλει ευσπλαγχνισθή ημάς, θέλει καταστρέψει τας ανομίας ημών· και θέλεις ρίψει πάσας τας αμαρτίας αυτών εις τα βάθη της θαλάσσης.
\par 20 Θέλεις εκτελέσει αλήθειαν εις τον Ιακώβ, έλεος εις τον Αβραάμ, καθώς ώμοσας εις τους πατέρας ημών από των αρχαίων ημερών.


\end{document}