\begin{document}

\title{Habakkuk}


\chapter{1}

\par 1 Η όρασις, την οποίαν είδεν Αββακούμ ο προφήτης.
\par 2 Έως πότε, Κύριε, θέλω κράζει, και δεν θέλεις εισακούει; θέλω βοά προς σε, Αδικία· και δεν θέλεις σώζει;
\par 3 Διά τι με κάμνεις να βλέπω ανομίαν και να θεωρώ ταλαιπωρίαν και αρπαγήν και αδικίαν έμπροσθέν μου; και υπάρχουσι διεγείροντες έριδα και φιλονεικίαν.
\par 4 Διά τούτο ο νόμος είναι αργός, και δεν εξέρχεται κρίσις τελεία· επειδή ο ασεβής καταδυναστεύει τον δίκαιον, διά τούτο εξέρχεταί κρίσις διεστραμμένη.
\par 5 Ιδέτε μεταξύ των εθνών και επιβλέψατε και θαυμάσατε μεγάλως, διότι εγώ θέλω πράξει έργον εν ταις ημέραις σας, το οποίον δεν θέλετε πιστεύσει, εάν τις διηγηθή αυτό.
\par 6 Διότι, ιδού, εγώ εξεγείρω τους Χαλδαίους, το έθνος το πικρόν και ορμητικόν, το οποίον θέλει διέλθει το πλάτος του τόπου, διά να κληρονομήση κατοικίας ουχί εαυτού.
\par 7 Είναι φοβεροί και τρομεροί· η κρίσις αυτών και η εξουσία αυτών θέλει προέρχεσθαι εξ αυτών.
\par 8 Και οι ίπποι αυτών είναι ταχύτεροι παρδάλεων και οξύτεροι λύκων της εσπέρας· και οι ιππείς αυτών θέλουσι διαχυθή και οι ιππείς αυτών θέλουσιν ελθεί από μακρόθεν· θέλουσι πετάξει ως αετός σπεύδων εις βρώσιν,
\par 9 πάντες θέλουσιν ελθεί επί αρπαγή· η όψις των προσώπων αυτών είναι ως ο ανατολικός άνεμος, και θέλουσι συνάξει τους αιχμαλώτους ως άμμον.
\par 10 Και θέλουσι περιπαίζει τους βασιλείς, και οι άρχοντες θέλουσιν είσθαι παίγνιον εις αυτούς· θέλουσιν εμπαίζει παν οχύρωμα· διότι θέλουσιν επισωρεύσει χώμα και θέλουσι κυριεύσει αυτό.
\par 11 Τότε το πνεύμα αυτού θέλει αλλοιωθή, και θέλει υπερβή παν όριον και θέλει ασεβεί, αποδίδων την ισχύν αυτού ταύτην εις τον θεόν αυτού.
\par 12 Δεν είσαι συ απ' αιώνος, Κύριε Θεέ μου, ο Άγιός μου; δεν θέλομεν αποθάνει. Συ, Κύριε, διέταξας αυτούς διά κρίσιν· και συ, Ισχυρέ, κατέστησας αυτούς εις παιδείαν ημών.
\par 13 Οι οφθαλμοί σου είναι καθαρώτεροι παρά ώστε να βλέπης τα πονηρά, και δεν δύνασαι να επιβλέπης εις την ανομίαν· διά τι επιβλέπεις εις τους παρανόμους και σιωπάς, όταν ο ασεβής καταπίνη τον δικαιότερον εαυτού,
\par 14 και κάμνεις τους ανθρώπους ως τους ιχθύας της θαλάσσης, ως τα ερπετά, τα μη έχοντα άρχοντα εφ' εαυτών;
\par 15 Ανασύρουσι πάντας διά του αγκίστρου, έλκουσιν αυτούς εις το δίκτυον αυτών και συνάγουσιν αυτούς εις την σαγήνην αυτών· διά τούτο ευφραίνονται και χαίρουσι.
\par 16 Διά τούτο θυσιάζουσιν εις το δίκτυον αυτών και καίουσι θυμίαμα εις την σαγήνην αυτών· διότι δι' αυτών η μερίς αυτών είναι παχεία και το φαγητόν αυτών εκλεκτόν.
\par 17 Μη διά τούτο θέλουσι πάντοτε εκκενόνει το δίκτυον αυτών; και δεν θέλουσι φείδεσθαι φονεύοντες πάντοτε τα έθνη;

\chapter{2}

\par 1 Επί της σκοπιάς μου θέλω σταθή και θέλω στηλωθή επί του πύργου, και θέλω αποσκοπεύει διά να ίδω τι θέλει λαλήσει προς εμέ και τι θέλω αποκριθή προς τον ελέγχοντά με.
\par 2 Και απεκρίθη προς εμέ ο Κύριος και είπε, Γράψον την όρασιν και έκθεσον αυτήν επί πινακιδίων, ώστε τρέχων να αναγινώσκη τις αυτήν·
\par 3 διότι η όρασις μένει έτι εις ωρισμένον καιρόν, αλλ' εις το τέλος θέλει λαλήσει και δεν θέλει ψευσθή· αν και αργοπορή, πρόσμεινον αυτήν· διότι βεβαίως θέλει ελθεί, δεν θέλει βραδύνει.
\par 4 Ιδού, η ψυχή αυτού επήρθη, δεν είναι ευθεία εν αυτώ· ο δε δίκαιος θέλει ζήσει διά της πίστεως αυτού.
\par 5 Και μάλιστα είναι προπετής εξ αιτίας του οίνου, ανήρ αλαζών, ουδέ ησυχάζει· όστις πλατύνει την ψυχήν αυτού ως άδης και είναι ως ο θάνατος και δεν χορταίνει, αλλά συνάγει εις εαυτόν πάντα τα έθνη και συλλαμβάνει εις εαυτόν πάντας τους λαούς.
\par 6 Δεν θέλουσι λάβει πάντες ούτοι παραβολήν κατ' αυτού και παροιμίαν εμπαικτικήν εναντίον αυτού; και ειπεί, Ουαί εις τον πληθύνοντα το μη εαυτού· έως πότε; και εις τον επιβαρύνοντα εαυτόν με παχύν πηλόν.
\par 7 Δεν θέλουσι σηκωθή εξαίφνης οι δάκνοντές σε και εξεγερθή οι ταλαιπωρούντές σε και θέλεις είσθαι προς αυτούς εις διαρπαγήν;
\par 8 Επειδή συ ελαφυραγώγησας έθνη πολλά, άπαν το υπόλοιπον των λαών θέλουσι σε λαφυραγωγήσει, εξ αιτίας των αιμάτων των ανθρώπων και της αδικίας της γης, της πόλεως και πάντων των κατοικούντων εν αυτή.
\par 9 Ουαί εις τον πλεονεκτούντα πλεονεξίαν κακήν διά τον οίκον αυτού, διά να θέση την φωλεάν αυτού υψηλά, διά να ελευθερωθή εκ χειρός του κακού.
\par 10 Εβουλεύθης αισχύνην εις τον οίκόν σου, εξολοθρεύων πολλούς λαούς, και ημάρτησας κατά της ψυχής σου.
\par 11 Διότι ο λίθος από του τοίχου θέλει βοήσει και τα ξυλοδέματα θέλουσιν αποκριθή προς αυτόν.
\par 12 Ουαί εις τον οικοδομούντα πόλιν εν αίμασι και θεμελιούντα πόλιν εν αδικίαις.
\par 13 Ιδού, δεν είναι τούτο παρά του Κυρίου των δυνάμεων, να μοχθώσιν οι λαοί διά το πυρ και τα έθνη να αποκάμνωσι διά την ματαιότητα;
\par 14 Διότι η γη θέλει είσθαι πλήρης της γνώσεως της δόξης του Κυρίου, καθώς τα ύδατα σκεπάζουσι την θάλασσαν.
\par 15 Ουαί εις τον ποτίζοντα τον πλησίον αυτού, εις σε όστις προσφέρεις την φιάλην σου και προσέτι μεθύεις αυτόν, διά να θεωρής την γύμνωσιν αυτών.
\par 16 Ενεπλήσθης αισχύνης αντί δόξης· πίε και συ, και ας ανακαλυφθή η ακροβυστία σου· το ποτήριον της δεξιάς του Κυρίου θέλει στραφή προς σε, και εμετός ατιμίας θέλει είσθαι επί την δόξαν σου.
\par 17 Διότι η προς τον Λίβανον αδικία σου θέλει σε καλύψει, και η φθορά των θηρίων η καταπτοήσασα αυτά θέλει σε πτοήσει, εξ αιτίας των αιμάτων των ανθρώπων και της αδικίας της γης, της πόλεως και πάντων των κατοικούντων εν αυτή.
\par 18 Τις η ωφέλεια του γλυπτού, ότι ο μορφωτής αυτού έγλυψεν αυτό; του χωνευτού και του διδασκάλου του ψεύδους, ότι ο κατασκευάσας θαρρεί εις το έργον αυτού, ώστε να κάμνη είδωλα άφωνα;
\par 19 Ουαί εις τον λέγοντα προς το ξύλον, Εξεγείρου· εις τον άφωνον λίθον, Σηκώθητι· αυτό θέλει διδάξει; Ιδού, αυτό είναι περιεσκεπασμένον με χρυσόν και άργυρον, και δεν είναι πνοή παντελώς εν αυτώ.
\par 20 Αλλ' ο Κύριος είναι εν τω ναώ τω αγίω αυτού· σιώπα ενώπιον αυτού, πάσα η γη.

\chapter{3}

\par 1 Προσευχή Αββακούμ του προφήτου επί Σιγιωνώθ.
\par 2 Κύριε, ήκουσα την ακοήν σου και εφοβήθην· Κύριε, ζωοποίει το έργον σου εν μέσω των ετών· Εν μέσω των ετών γνωστοποίει, αυτό· εν τη οργή σου μνήσθητι ελέους.
\par 3 Ο Θεός ήλθεν από Θαιμάν και ο Άγιος από του όρους Φαράν· Διάψαλμα. εκάλυψεν ουρανούς η δόξα αυτού, και της αινέσεως αυτού ήτο πλήρης η γή·
\par 4 Και η λάμψις αυτού ήτο ως το φώς· ακτίνες εξήρχοντο εκ της χειρός αυτού, και εκεί ήτο ο κρυψών της ισχύος αυτού.
\par 5 Έμπροσθεν αυτού προεπορεύετο ο θάνατος, και αστραπαί εξήρχοντο υπό τους πόδας αυτού.
\par 6 Εστάθη και διεμέτρησε την γήν· επέβλεψε και διέλυσε τα έθνη· και τα όρη τα αιώνια συνετρίβησαν, οι αιώνιοι βουνοί εταπεινώθησαν· αι οδοί αυτού είναι αιώνιοι.
\par 7 Είδον τας σκηνάς της Αιθιοπίας εν θλίψει· ετρόμαξαν τα παραπετάσματα της γης Μαδιάμ.
\par 8 Μήπως ωργίσθη ο Κύριος κατά των ποταμών; μήπως ήτο ο θυμός σου κατά των ποταμών; ή η οργή σου κατά της θαλάσσης, ώστε επέβης επί τους ίππους σου και επί τας αμάξας σου προς σωτηρίαν;
\par 9 Εσύρθη έξω το τόξον σου, καθώς μεθ' όρκου ανήγγειλας εις τας φυλάς. Διάψαλμα. Συ διέσχισας την γην εις ποταμούς.
\par 10 Σε είδον τα όρη και ετρόμαξαν. Κατακλυσμός υδάτων επήλθεν· η άβυσσος ανέπεμψε την φωνήν αυτής, ανύψωσε τας χείρας αυτής.
\par 11 Ο ήλιος και η σελήνη εστάθησαν εν τω κατοικητηρίω αυτών· εν τω φωτί των βελών σου περιεπάτουν, εν τη λάμψει της αστραπτούσης λόγχης σου.
\par 12 Εν αγανακτήσει διήλθες την γην, εν θυμώ κατεπάτησας τα έθνη.
\par 13 Εξήλθες εις σωτηρίαν του λαού σου, εις σωτηρίαν του χριστού σου· επάταξας τον αρχηγόν του οίκου των ασεβών, απεκάλυψας τα θεμέλια έως βάθους. Διάψαλμα.
\par 14 Διεπέρασας με τας λόγχας αυτού την κεφαλήν των στραταρχών αυτού· εφώρμησαν ως ανεμοστρόβιλος διά να μη διασκορπίσωσιν· η αγαλλίασις αυτών ήτο ως εάν έμελλον κρυφίως να καταφάγωσι τον πτωχόν.
\par 15 Διέβης διά της θαλάσσης μετά των ίππων σου, διά σωρών υδάτων πολλών.
\par 16 Ήκουσα, και τα εντόσθιά μου συνεταράχθησαν· τα χείλη μου έτρεμον εις την φωνήν· η σαθρότης εισήλθεν εις τα οστά μου, και υποκάτω μου έλαβον τρόμον· πλην εν τη ημέρα της θλίψεως θέλω αναπαυθή, όταν αναβή κατά του λαού ο μέλλων να εκπορθήση αυτόν.
\par 17 Αν και η συκή δεν θέλει βλαστήσει, μηδέ θέλει είσθαι καρπός εν ταις αμπέλοις· ο κόπος της ελαίας θέλει ματαιωθή, και οι αγροί δεν θέλουσι δώσει τροφήν· το ποίμνιον θέλει εξολοθρευθή από της μάνδρας, και δεν θέλουσιν είσθαι βόες εν τοις σταύλοις·
\par 18 Εγώ όμως θέλω ευφραίνεσθαι εις τον Κύριον, θέλω χαίρει εις τον Θεόν της σωτηρίας μου.
\par 19 Κύριος ο Θεός είναι η δύναμίς μου, και θέλει κάμει τους πόδας μου ως των ελάφων· και θέλει με κάμει να περιπατώ επί τους υψηλούς τόπους μου. Εις τον πρώτον μουσικόν επί Νεγινώθ.


\end{document}