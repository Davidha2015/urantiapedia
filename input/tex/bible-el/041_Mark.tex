\begin{document}

\title{Mark}


\chapter{1}

\par Αρχή του ευαγγελίου του Ιησού Χριστού, Υιού του Θεού.
\par 2 Καθώς είναι γεγραμμένον εν τοις προφήταις· Ιδού, εγώ αποστέλλω τον άγγελόν μου προ προσώπου σου, όστις θέλει κατασκευάσει την οδόν σου έμπροσθέν σου·
\par 3 Φωνή βοώντος εν τη ερήμω, ετοιμάσατε την οδόν του Κυρίου, ευθείας κάμετε τας τρίβους αυτού.
\par 4 Ήτο ο Ιωάννης βαπτίζων εν τη ερήμω και κηρύττων βάπτισμα μετανοίας εις άφεσιν αμαρτιών.
\par 5 Και εξήρχοντο προς αυτόν όλος ο τόπος της Ιουδαίας και οι Ιεροσολυμίται, και εβαπτίζοντο πάντες εν τω Ιορδάνη ποταμώ υπ' αυτού, εξομολογούμενοι τας αμαρτίας αυτών.
\par 6 Ήτο δε ο Ιωάννης ενδεδυμένος τρίχας καμήλου και έχων ζώνην δερματίνην περί την οσφύν αυτού, και τρώγων ακρίδας και μέλι άγριον.
\par 7 Και εκήρυττε, λέγων· Έρχεται ο ισχυρότερός μου οπίσω μου, του οποίου δεν είμαι άξιος σκύψας να λύσω το λωρίον των υποδημάτων αυτού.
\par 8 Εγώ μεν σας εβάπτισα εν ύδατι, αυτός δε θέλει σας βαπτίσει εν Πνεύματι Αγίω.
\par 9 Και εν εκείναις ταις ημέραις ήλθεν ο Ιησούς από Ναζαρέτ της Γαλιλαίας και εβαπτίσθη υπό Ιωάννου εις τον Ιορδάνην.
\par 10 Και ευθύς ενώ ανέβαινεν από του ύδατος, είδε τους ουρανούς σχιζομένους και το Πνεύμα καταβαίνον ως περιστεράν επ' αυτόν·
\par 11 και φωνή έγεινεν εκ των ουρανών· Συ είσαι ο Υιός μου ο αγαπητός, εις τον οποίον ευηρεστήθην.
\par 12 Και ευθύς το Πνεύμα εκβάλλει αυτόν εις την έρημον·
\par 13 και ήτο εκεί εν τη ερήμω ημέρας τεσσαράκοντα πειραζόμενος υπό του Σατανά, και ήτο μετά των θηρίων, και οι άγγελοι υπηρέτουν αυτόν.
\par 14 Αφού δε παρεδόθη ο Ιωάννης, ήλθεν ο Ιησούς εις την Γαλιλαίαν κηρύττων το ευαγγέλιον της βασιλείας του Θεού
\par 15 και λέγων ότι επληρώθη ο καιρός και επλησίασεν η βασιλεία του Θεού· μετανοείτε και πιστεύετε εις το ευαγγέλιον.
\par 16 Περιπατών δε παρά την θάλασσαν της Γαλιλαίας, είδε τον Σίμωνα και Ανδρέαν τον αδελφόν αυτού, ρίπτοντας δίκτυον εις την θάλασσαν· διότι ήσαν αλιείς·
\par 17 και είπε προς αυτούς ο Ιησούς· Έλθετε οπίσω μου, και θέλω σας κάμει να γείνητε αλιείς ανθρώπων.
\par 18 Και ευθύς αφήσαντες τα δίκτυα αυτών, ηκολούθησαν αυτόν.
\par 19 Και προχωρήσας εκείθεν ολίγον, είδεν Ιάκωβον τον του Ζεβεδαίου και Ιωάννην τον αδελφόν αυτού, και αυτούς εν τω πλοίω επισκευάζοντας τα δίκτυα,
\par 20 και ευθύς εκάλεσεν αυτούς. Και αφήσαντες τον πατέρα αυτών Ζεβεδαίον εν τω πλοίω μετά των μισθωτών, υπήγον οπίσω αυτού.
\par 21 Και εισέρχονται εις Καπερναούμ· και ευθύς εν τω σαββάτω εισελθών ο Ιησούς εις την συναγωγήν εδίδασκε.
\par 22 Και εξεπλήττοντο διά την διδαχήν αυτού· διότι εδίδασκεν αυτούς ως έχων εξουσίαν, και ουχί ως οι γραμματείς.
\par 23 Και ήτο εν τη συναγωγή αυτών άνθρωπος έχων πνεύμα ακάθαρτον, και ανέκραξε,
\par 24 λέγων· Φευ, τι είναι μεταξύ ημών και σου, Ιησού Ναζαρηνέ; ήλθες να μας απολέσης; σε γνωρίζω τις είσαι, ο Άγιος του Θεού.
\par 25 Και επετίμησεν αυτό ο Ιησούς, λέγων· Σιώπα και έξελθε εξ αυτού.
\par 26 Και το πνεύμα το ακάθαρτον, αφού εσπάραξεν αυτόν και έκραξε μετά φωνής μεγάλης, εξήλθεν εξ αυτού.
\par 27 Και πάντες εξεπλάγησαν, ώστε συνεζήτουν προς αλλήλους, λέγοντες· Τι είναι τούτο; τις αύτη η νέα διδαχή, διότι μετά εξουσίας προστάζει και τα ακάθαρτα πνεύματα, και υπακούουσιν εις αυτόν;
\par 28 Εξήλθε δε η φήμη αυτού ευθύς εις όλην την περίχωρον της Γαλιλαίας.
\par 29 Και ευθύς εξελθόντες εκ της συναγωγής, ήλθον εις την οικίαν Σίμωνος και Ανδρέου μετά του Ιακώβου και Ιωάννου.
\par 30 Η δε πενθερά του Σίμωνος ήτο κατάκοιτος πάσχουσα πυρετόν. Και ευθύς ελάλησαν προς αυτόν περί αυτής.
\par 31 Και πλησιάσας ήγειρεν αυτήν πιάσας την χείρα αυτής, και αφήκεν αυτήν ο πυρετός ευθύς, και υπηρέτει αυτούς.
\par 32 Αφού δε έγεινεν εσπέρα, ότε έδυσεν ο ήλιος, έφεραν προς αυτόν πάντας τους πάσχοντας και τους δαιμονιζομένους·
\par 33 και η πόλις όλη ήτο συνηγμένη έμπροσθεν της θύρας·
\par 34 και εθεράπευσε πολλούς πάσχοντας διαφόρους αρρωστίας, και δαιμόνια πολλά εξέβαλε, και δεν άφινε τα δαιμόνια να λαλώσιν, επειδή εγνώριζον αυτόν.
\par 35 Και το πρωΐ ενώ ήτο όρθρος βαθύς, σηκωθείς εξήλθε· και υπήγεν εις έρημον τόπον και εκεί προσηύχετο.
\par 36 Και έδραμον κατόπιν αυτού ο Σίμων και οι μετ' αυτού,
\par 37 και ευρόντες αυτόν λέγουσι προς αυτόν ότι πάντες σε ζητούσι.
\par 38 Και λέγει προς αυτούς· Ας υπάγωμεν εις τας πλησίον κωμοπόλεις, διά να κηρύξω και εκεί· επειδή διά τούτο εξήλθον.
\par 39 Και εκήρυττεν εν ταις συναγωγαίς αυτών εις όλην την Γαλιλαίαν και εξέβαλλε τα δαιμόνια.
\par 40 Και έρχεται προς αυτόν λεπρός παρακαλών αυτόν και γονυπετών έμπροσθεν αυτού και λέγων προς αυτόν ότι, εάν θέλης, δύνασαι να με καθαρίσης.
\par 41 Ο δε Ιησούς σπλαγχνισθείς, εξέτεινε την χείρα και ήγγισεν αυτόν και λέγει προς αυτόν· Θέλω, καθαρίσθητι.
\par 42 Και ως είπε τούτο, ευθύς έφυγεν απ' αυτού η λέπρα, και εκαθαρίσθη.
\par 43 Και προστάξας αυτόν εντόνως, ευθύς απέπεμψεν αυτόν
\par 44 και λέγει προς αυτόν· Πρόσεχε μη είπης προς μηδένα μηδέν, αλλ' ύπαγε, δείξον σεαυτόν εις τον ιερέα και πρόσφερε περί του καθαρισμού σου όσα προσέταξεν ο Μωϋσής διά μαρτυρίαν εις αυτούς.
\par 45 Αλλ' εκείνος εξελθών ήρχισε να κηρύττη πολλά και να διαφημίζη τον λόγον, ώστε πλέον δεν ηδύνατο αυτός να εισέλθη φανερά εις πόλιν, αλλ' ήτο έξω εν ερήμοις τόποις· και ήρχοντο προς αυτόν πανταχόθεν.

\chapter{2}

\par Και μεθ' ημέρας πάλιν εισήλθεν εις Καπερναούμ και ηκούσθη ότι είναι εις οίκον.
\par 2 Και ευθύς συνήχθησαν πολλοί, ώστε δεν εχώρουν πλέον αυτούς ουδέ τα πρόθυρα· και εκήρυττεν εις αυτούς τον λόγον.
\par 3 Και έρχονται προς αυτόν φέροντες παραλυτικόν, βασταζόμενον υπό τεσσάρων·
\par 4 και μη δυνάμενοι να πλησιάσωσιν εις αυτόν εξ αιτίας του όχλου, εχάλασαν την στέγην όπου ήτο, και διατρυπήσαντες καταβιβάζουσι τον κράββατον, εφ' ου κατέκειτο ο παραλυτικός.
\par 5 Ιδών δε ο Ιησούς την πίστιν αυτών, λέγει προς τον παραλυτικόν· Τέκνον, συγκεχωρημέναι είναι εις σε αι αμαρτίαι σου.
\par 6 Ήσαν δε τινές των γραμματέων εκεί καθήμενοι και διαλογιζόμενοι εν ταις καρδίαις αυτών·
\par 7 Διά τι ούτος λαλεί τοιαύτας βλασφημίας; τις δύναται να συγχωρή αμαρτίας ειμή εις, ο Θεός;
\par 8 Και ευθύς νοήσας ο Ιησούς διά του πνεύματος αυτού ότι ούτω διαλογίζονται καθ' εαυτούς, είπε προς αυτούς· Διά τι διαλογίζεσθε ταύτα εν ταις καρδίαις σας;
\par 9 τι είναι ευκολώτερον, να είπω προς τον παραλυτικόν, Συγκεχωρημέναι είναι αι αμαρτίαι σου, ή να είπω, Σηκώθητι και έπαρε τον κράββατόν σου και περιπάτει;
\par 10 αλλά διά να γνωρίσητε ότι έξουσίαν έχει ο Υιός του ανθρώπου επί της γης να συγχωρή αμαρτίας λέγει προς τον παραλυτικόν·
\par 11 Προς σε λέγω, Σηκώθητι και έπαρε τον κράββατόν σου και ύπαγε εις τον οίκόν σου.
\par 12 Και ηγέρθη ευθύς και σηκώσας τον κράββατον, εξήλθεν ενώπιον πάντων, ώστε εξεπλήττοντο πάντες και εδόξαζον τον Θεόν, λέγοντες ότι ουδέποτε είδομεν τοιαύτα.
\par 13 Και εξήλθε πάλιν παρά την θάλασσαν· και πας ο όχλος ήρχετο προς αυτόν, και εδίδασκεν αυτούς.
\par 14 Και διαβαίνων είδε Λευΐν τον του Αλφαίου καθήμενον εις το τελώνιον, και λέγει προς αυτόν· Ακολούθει με. Και σηκωθείς ηκολούθησεν αυτόν.
\par 15 Και ενώ εκάθητο εις την τράπεζαν εν τη οικία αυτού, συνεκάθηντο και πολλοί τελώναι και αμαρτωλοί μετά του Ιησού και των μαθητών αυτού· διότι ήσαν πολλοί, και ηκολούθησαν αυτόν.
\par 16 Οι δε γραμματείς και Φαρισαίοι, ιδόντες αυτόν τρώγοντα μετά των τελωνών και αμαρτωλών, έλεγον προς τους μαθητάς αυτού· Διά τι μετά των τελωνών και αμαρτωλών τρώγει και πίνει;
\par 17 Και ακούσας ο Ιησούς, λέγει προς αυτούς· Δεν έχουσι χρείαν ιατρού οι υγιαίνοντες, αλλ' οι πάσχοντες· δεν ήλθον διά να καλέσω δικαίους αλλά αμαρτωλούς εις μετάνοιαν.
\par 18 Οι μαθηταί δε του Ιωάννου και οι των Φαρισαίων ενήστευον. Και έρχονται και λέγουσι προς αυτόν· Διά τι οι μαθηταί του Ιωάννου και οι των Φαρισαίων νηστεύουσιν, οι δε μαθηταί σου δεν νηστεύουσι;
\par 19 Και είπε προς αυτούς ο Ιησούς· Μήπως δύνανται οι υιοί του νυμφώνος, ενόσω ο νυμφίος είναι μετ' αυτών, να νηστεύωσιν; όσον καιρόν έχουσι τον νυμφίον μεθ' εαυτών, δεν δύνανται να νηστεύωσι·
\par 20 θέλουσιν όμως ελθεί ημέραι, όταν αφαιρεθή απ' αυτών ο νυμφίος, και τότε θέλουσι νηστεύσει εν εκείναις ταις ημέραις.
\par 21 Και ουδείς ράπτει επίρραμμα αγνάφου πανίου επί ιμάτιον παλαιόν· ει δε μη, το αναπλήρωμα αυτού το νέον αφαιρεί από του παλαιού, και γίνεται σχίσμα χειρότερον.
\par 22 Και ουδείς βάλλει οίνον νέον εις ασκούς παλαιούς· ει δε μη, ο οίνος ο νέος διασχίζει τους ασκούς, και ο οίνος εκχέεται και οι ασκοί φθείρονται· αλλά πρέπει οίνος νέος να βάλληται εις ασκούς νέους.
\par 23 Και ότε διέβαινεν εν σαββάτω διά των σπαρτών, οι μαθηταί αυτού, ενώ ώδευον, ήρχισαν να ανασπώσι τα αστάχυα.
\par 24 Και οι Φαρισαίοι έλεγον προς αυτόν· Ιδού, διά τι πράττουσιν εν τοις σάββασιν εκείνο, το οποίον δεν συγχωρείται;
\par 25 Και αυτός έλεγε προς αυτούς· Ποτέ δεν ανεγνώσατε τι έπραξεν ο Δαβίδ, ότε έλαβε χρείαν και επείνασεν αυτός και οι μετ' αυτού;
\par 26 πως εισήλθεν εις τον οίκον του Θεού επί Αβιάθαρ του αρχιερέως, και έφαγε τους άρτους της προθέσεως, τους οποίους δεν είναι συγκεχωρημένον ειμή εις τους ιερείς να φάγωσι, και έδωκε και εις τους όντας μετ' αυτού;
\par 27 Και έλεγε προς αυτούς· το σάββατον έγεινε διά τον άνθρωπον, ουχί ο άνθρωπος διά το σάββατον·
\par 28 ώστε ο Υιός του ανθρώπου κύριος είναι και του σαββάτου.

\chapter{3}

\par Και εισήλθε πάλιν εις την συναγωγήν· και ήτο εκεί άνθρωπος έχων εξηραμμένην την χείρα.
\par 2 Και παρετήρουν αυτόν αν εν τω σαββάτω θέλη θεραπεύσει αυτόν, διά να κατηγορήσωσιν αυτόν.
\par 3 Και λέγει προς τον άνθρωπον τον έχοντα εξηραμμένην την χείρα· Σηκώθητι εις το μέσον.
\par 4 Και λέγει προς αυτούς· Είναι συγκεχωρημένον εν σαββάτω να αγαθοποιήση τις ή να κακοποιήση; να σώση ψυχήν ή να θανατώση; οι δε εσιώπων.
\par 5 Και περιβλέψας αυτούς μετ' οργής, λυπούμενος διά την πώρωσιν της καρδίας αυτών, λέγει προς τον άνθρωπον· Έκτεινον την χείρα σου. Και εξέτεινε, και αποκατεστάθη η χειρ αυτού υγιής ως η άλλη.
\par 6 Και εξελθόντες οι Φαρισαίοι συνεβουλεύθησαν ευθύς μετά των Ηρωδιανών κατ' αυτού, διά να απολέσωσιν αυτόν.
\par 7 Και ο Ιησούς ανεχώρησε μετά των μαθητών αυτού προς την θάλασσαν· και ηκολούθησαν αυτόν πολύ πλήθος από της Γαλιλαίας και από της Ιουδαίας
\par 8 και από Ιεροσολύμων και από της Ιδουμαίας και από πέραν του Ιορδάνου και οι περί Τύρον και Σιδώνα, πλήθος πολύ, ακούσαντες όσα έπραττεν, ήλθον προς αυτόν.
\par 9 Και είπε προς τους μαθητάς αυτού να μένη πλησίον αυτού εν πλοιάριον εξ αιτίας του όχλου, διά να μη συνθλίβωσιν αυτόν·
\par 10 διότι εθεράπευσε πολλούς, ώστε έπιπτον επ' αυτόν διά να εγγίσωσιν αυτόν όσοι είχον αρρωστίας·
\par 11 και τα πνεύματα τα ακάθαρτα, ότε εθεώρουν αυτόν, προσέπιπτον εις αυτόν και έκραζον, λέγοντα ότι συ είσαι ο Υιός του Θεού.
\par 12 Και πολλά επετίμα αυτά διά να μη φανερώσωσιν αυτόν.
\par 13 Και αναβαίνει εις το όρος και προσκαλεί όσους αυτός ήθελε, και υπήγον προς αυτόν.
\par 14 Και εξέλεξε δώδεκα, διά να ήναι μετ' αυτού και διά να αποστέλλη αυτούς να κηρύττωσι
\par 15 και να έχωσιν εξουσίαν να θεραπεύωσι τας νόσους και να εκβάλλωσι τα δαιμόνια·
\par 16 Σίμωνα, τον οποίον επωνόμασε Πέτρον,
\par 17 και Ιάκωβον τον του Ζεβεδαίου και Ιωάννην τον αδελφόν του Ιακώβου· και επωνόμασεν αυτούς Βοανεργές, το οποίον σημαίνει Υιοί βροντής·
\par 18 και Ανδρέαν και Φίλιππον και Βαρθολομαίον και Ματθαίον και Θωμάν και Ιάκωβον τον του Αλφαίου και Θαδδαίον και Σίμωνα τον Κανανίτην
\par 19 και Ιούδαν τον Ισκαριώτην, όστις και παρέδωκεν αυτόν.
\par 20 Και έρχονται εις οίκον τινά· και συναθροίζεται πάλιν όχλος, ώστε αυτοί δεν ηδύναντο μηδέ να φάγωσιν άρτον.
\par 21 Και ότε ήκουσαν οι συγγενείς αυτού, εξήλθον διά να πιάσωσιν αυτόν· διότι έλεγον ότι είναι έξω εαυτού.
\par 22 Και οι γραμματείς, οίτινες κατέβησαν από Ιεροσολύμων, έλεγον ότι έχει Βεελζεβούλ, και ότι διά του άρχοντος των δαιμονίων εκβάλλει τα δαιμόνια.
\par 23 Και προσκαλέσας αυτούς, έλεγε προς αυτούς διά παραβολών· Πως δύναται Σατανάς να εκβάλλη Σατανάν;
\par 24 και εάν βασιλεία διαιρεθή καθ' εαυτής, η βασιλεία εκείνη δεν δύναται να σταθή·
\par 25 και εάν οικία διαιρεθή καθ' εαυτής, η οικία εκείνη δεν δύναται να σταθή.
\par 26 Και αν ο Σατανάς εσηκώθη καθ' εαυτού και διηρέθη, δεν δύναται να σταθή, αλλ' έχει τέλος.
\par 27 Ουδείς δύναται να αρπάση τα σκεύη του δυνατού, εισελθών εις την οικίαν αυτού, εάν πρώτον δεν δέση τον δυνατόν, και τότε θέλει διαρπάσει την οικίαν αυτού.
\par 28 Αληθώς σας λέγω ότι πάντα τα αμαρτήματα θέλουσι συγχωρηθή εις τους υιούς των ανθρώπων και αι βλασφημίαι, όσας βλασφημήσωσιν·
\par 29 όστις όμως βλασφημήση εις το Πνεύμα το Άγιον, δεν έχει συγχώρησιν εις τον αιώνα, αλλ' είναι ένοχος αιωνίου καταδίκης·
\par 30 διότι έλεγον, Πνεύμα ακάθαρτον έχει.
\par 31 Έρχονται λοιπόν οι αδελφοί και η μήτηρ αυτού, και σταθέντες έξω απέστειλαν προς αυτόν και έκραζον αυτόν.
\par 32 Και εκάθητο όχλος περί αυτόν· είπον δε προς αυτόν· Ιδού, η μήτηρ σου και οι αδελφοί σου έξω σε ζητούσι.
\par 33 Και απεκρίθη προς αυτούς, λέγων· Τις είναι η μήτηρ μου ή οι αδελφοί μου;
\par 34 Και περιβλέψας κύκλω προς τους καθημένους περί αυτόν, λέγει· Ιδού η μήτηρ μου και οι αδελφοί μου·
\par 35 διότι όστις κάμη το θέλημα του Θεού, ούτος είναι αδελφός μου και αδελφή μου και μήτηρ.

\chapter{4}

\par Και πάλιν ήρχισε να διδάσκη πλησίον της θαλάσσης· και συνήχθη προς αυτόν όχλος πολύς, ώστε εισελθών εις το πλοίον εκάθητο εις την θάλασσαν· και πας ο όχλος ήτο επί της γης πλησίον της θαλάσσης.
\par 2 Και εδίδασκεν αυτούς διά παραβολών πολλά, και έλεγε προς αυτούς εν τη διδαχή αυτού·
\par 3 Ακούετε· ιδού, εξήλθεν ο σπείρων διά να σπείρη.
\par 4 Και ενώ έσπειρεν, άλλο μεν έπεσε παρά την οδόν, και ήλθον τα πετεινά του ουρανού και κατέφαγον αυτό.
\par 5 Άλλο δε έπεσεν επί το πετρώδες, όπου δεν είχε γην πολλήν, και ευθύς ανεφύη, διότι δεν είχε βάθος γης,
\par 6 και ότε ανέτειλεν ο ήλιος εκαυματίσθη, και επειδή δεν είχε ρίζαν εξηράνθη.
\par 7 Και άλλο έπεσεν εις τας ακάνθας, και ανέβησαν αι άκανθαι και συνέπνιξαν αυτό, και καρπόν δεν έδωκε·
\par 8 και άλλο έπεσεν εις την γην την καλήν και έδιδε καρπόν αναβαίνοντα και αυξάνοντα, και έδωκεν εν τριάκοντα και εν εξήκοντα και εν εκατόν.
\par 9 Και έλεγε προς αυτούς· Ο έχων ώτα διά να ακούη, ας ακούη.
\par 10 Ότε δε έμεινε καταμόνας, ηρώτησαν αυτόν οι περί αυτόν μετά των δώδεκα περί της παραβολής.
\par 11 Και έλεγε προς αυτούς· Εις εσάς εδόθη να γνωρίσητε το μυστήριον της βασιλείας του Θεού· εις εκείνους δε τους έξω διά παραβολών τα πάντα γίνονται,
\par 12 διά να βλέπωσι βλέποντες και να μη ίδωσι, και να ακούωσιν ακούοντες και να μη νοήσωσι, μήποτε επιστρέψωσι και συγχωρηθώσιν εις αυτούς τα αμαρτήματα.
\par 13 Και λέγει προς αυτούς· Δεν εξεύρετε την παραβολήν ταύτην, και πως θέλετε γνωρίσει πάσας τας παραβολάς;
\par 14 Ο σπείρων τον λόγον σπείρει.
\par 15 Οι δε παρά την οδόν είναι ούτοι, εις τους οποίους σπείρεται ο λόγος, και όταν ακούσωσιν, ευθύς έρχεται ο Σατανάς, και αφαιρεί τον λόγον τον εσπαρμένον εν ταις καρδίαις αυτών.
\par 16 Και ομοίως οι επί τα πετρώδη σπειρόμενοι είναι ούτοι, οίτινες όταν ακούσωσι τον λόγον, ευθύς μετά χαράς δέχονται αυτόν,
\par 17 δεν έχουσιν όμως ρίζαν εν εαυτοίς, αλλ' είναι πρόσκαιροι· έπειτα όταν γείνη θλίψις ή διωγμός διά τον λόγον, ευθύς σκανδαλίζονται.
\par 18 Και οι εις τας ακάνθας σπειρόμενοι είναι ούτοι, οίτινες ακούουσι τον λόγον,
\par 19 και αι μέριμναι του αιώνος τούτου και η απάτη του πλούτου και αι επιθυμίαι των άλλων πραγμάτων εισερχόμεναι συμπνίγουσι τον λόγον, και γίνεται άκαρπος.
\par 20 Και οι εις την γην την καλήν σπαρέντες είναι ούτοι, οίτινες ακούουσι τον λόγον και παραδέχονται και καρποφορούσιν εν τριάκοντα και εν εξήκοντα και εν εκατόν.
\par 21 Και έλεγε προς αυτούς· Μήπως ο λύχνος έρχεται διά να τεθή υπό τον μόδιον ή υπό την κλίνην; ουχί διά να τεθή επί τον λυχνοστάτην;
\par 22 διότι δεν είναι τι κρυπτόν, το οποίον δεν θέλει φανερωθή, ουδ' έγεινε τι απόκρυφον, το οποίον δεν θέλει ελθεί εις το φανερόν.
\par 23 Όστις έχει ώτα διά να ακούη, ας ακούη.
\par 24 Και έλεγε προς αυτούς· Προσέχετε τι ακούετε. Με οποίον μέτρον μετρείτε, θέλει μετρηθή εις εσάς, και θέλει γείνει προσθήκη εις εσάς τους ακούοντας.
\par 25 Διότι όστις έχει, θέλει δοθή εις αυτόν· και όστις δεν έχει, και εκείνο το οποίον έχει θέλει αφαιρεθή απ' αυτού.
\par 26 Και έλεγεν· Ούτως είναι η βασιλεία του Θεού, ως εάν άνθρωπος ρίψη τον σπόρον επί της γης,
\par 27 και κοιμάται και σηκόνηται νύκτα και ημέραν, και ο σπόρος βλαστάνη και αυξάνη καθώς αυτός δεν εξεύρει.
\par 28 Διότι αφ' εαυτής η γη καρποφορεί, πρώτον χόρτον, έπειτα αστάχυον, έπειτα πλήρη σίτον εν τω ασταχύω.
\par 29 Όταν δε ωριμάση ο καρπός, ευθύς αποστέλλει το δρέπανον, διότι ήλθεν ο θερισμός.
\par 30 Έτι έλεγε· Με τι να ομοιώσωμεν την βασιλείαν του Θεού; ή με ποίαν παραβολήν να παραβάλωμεν αυτήν;
\par 31 Είναι ομοία με κόκκον σινάπεως, όστις, όταν σπαρή επί της γης, είναι μικρότερος πάντων των σπερμάτων των επί της γής·
\par 32 αφού δε σπαρή, αναβαίνει και γίνεται μεγαλήτερος πάντων των λαχάνων και κάμνει κλάδους μεγάλους, ώστε υπό την σκιάν αυτού δύνανται τα πετεινά του ουρανού να κατασκηνώσι.
\par 33 Και διά τοιούτων πολλών παραβολών ελάλει προς αυτούς τον λόγον, καθώς ηδύναντο να ακούωσι,
\par 34 χωρίς δε παραβολής δεν ελάλει προς αυτούς· κατ ιδίαν όμως εξήγει πάντα εις τους μαθητάς αυτού.
\par 35 Και λέγει προς αυτούς εν εκείνη τη ημέρα, ότε έγεινεν εσπέρα· Ας διέλθωμεν εις το πέραν.
\par 36 Και αφήσαντες τον όχλον, παραλαμβάνουσιν αυτόν ως ήτο εν τω πλοίω και άλλα δε πλοιάρια ήσαν μετ' αυτού.
\par 37 Και γίνεται μέγας ανεμοστρόβιλος και τα κύματα εισέβαλλον εις το πλοίον, ώστε αυτό ήδη εγεμίζετο.
\par 38 Και αυτός ήτο επί της πρύμνης κοιμώμενος επί το προσκεφάλαιον· και εξυπνούσιν αυτόν και λέγουσι προς αυτόν· Διδάσκαλε, δεν σε μέλει ότι χανόμεθα;
\par 39 Και σηκωθείς επετίμησε τον άνεμον και είπε προς την θάλασσαν· Σιώπα, ησύχασον. Και έπαυσεν ο άνεμος, και έγεινε γαλήνη μεγάλη.
\par 40 Και είπε προς αυτούς· Διά τι είσθε ούτω δειλοί; πως δεν έχετε πίστιν;
\par 41 Και εφοβήθησαν φόβον μέγαν και έλεγον προς αλλήλους· Τις λοιπόν είναι ούτος, ότι και ο άνεμος και η θάλασσα υπακούουσιν εις αυτόν;

\chapter{5}

\par Και ήλθον εις το πέραν της θαλάσσης εις την χώραν των Γαδαρηνών.
\par 2 Και ως εξήλθεν εκ του πλοίου, ευθύς απήντησεν αυτόν εκ των μνημείων άνθρωπος έχων πνεύμα ακάθαρτον,
\par 3 όστις είχε την κατοικίαν εν τοις μνημείοις, και ουδείς ηδύνατο να δέση αυτόν ουδέ με αλύσεις,
\par 4 διότι πολλάκις είχε δεθή με ποδόδεσμα και με αλύσεις, και διεσπάσθησαν υπ' αυτού αι αλύσεις και τα ποδόδεσμα συνετρίφθησαν, και ουδείς ίσχυε να δαμάση αυτόν·
\par 5 και διά παντός νύκτα και ημέραν ήτο εν τοις όρεσι και εν τοις μνημείοις, κράζων και κατακόπτων εαυτόν με λίθους.
\par 6 Ιδών δε τον Ιησούν από μακρόθεν, έδραμε και προσεκύνησεν αυτόν,
\par 7 και κράξας μετά φωνής μεγάλης είπε· Τι είναι μεταξύ εμού και σου, Ιησού, Υιέ του Θεού του υψίστου; ορκίζω σε εις τον Θεόν, μη με βασανίσης.
\par 8 Διότι έλεγε προς αυτόν· Έξελθε από του ανθρώπου το πνεύμα το ακάθαρτον.
\par 9 Και ηρώτησεν αυτόν· Τι είναι το όνομά σου; Και απεκρίθη λέγων· Λεγεών είναι το όνομά μου, διότι πολλοί είμεθα.
\par 10 Και παρεκάλει αυτόν πολλά να μη αποστείλη αυτούς έξω της χώρας.
\par 11 Ήτο δε εκεί προς τα όρη αγέλη μεγάλη χοίρων βοσκομένη.
\par 12 και παρεκάλεσαν αυτόν πάντες οι δαίμονες, λέγοντες· Πέμψον ημάς εις τους χοίρους, διά να εισέλθωμεν εις αυτούς.
\par 13 Και ο Ιησούς ευθύς επέτρεψεν εις αυτούς. Και εξελθόντα τα πνεύματα τα ακάθαρτα εισήλθον εις τους χοίρους· και ώρμησεν η αγέλη κατά του κρημνού εις την θάλασσαν· ήσαν δε έως δύο χιλιάδες· και επνίγοντο εν τη θαλάσση.
\par 14 Οι δε βόσκοντες τους χοίρους έφυγον και ανήγγειλαν εις την πόλιν και εις τους αγρούς· και εξήλθον διά να ίδωσι τι είναι το γεγονός.
\par 15 Και έρχονται προς τον Ιησούν, και θεωρούσι τον δαιμονιζόμενον, όστις είχε τον λεγεώνα, καθήμενον και ενδεδυμένον και σωφρονούντα, και εφοβήθησαν.
\par 16 Και διηγήθησαν προς αυτούς οι ιδόντες πως έγεινε το πράγμα εις τον δαιμονιζόμενον, και περί των χοίρων.
\par 17 Και ήρχισαν να παρακαλώσιν αυτόν να αναχωρήση από των ορίων αυτών.
\par 18 Και ότε εισήλθεν εις το πλοίον, παρεκάλει αυτόν ο δαιμονισθείς να ήναι μετ' αυτού.
\par 19 Πλην ο Ιησούς δεν αφήκεν αυτόν, αλλά λέγει προς αυτόν· Ύπαγε εις τον οίκόν σου προς τους οικείους σου και ανάγγειλον προς αυτούς όσα ο Κύριος σοι έκαμε και σε ηλέησε.
\par 20 Και ανεχώρησε και ήρχισε να κηρύττη εν τη Δεκαπόλει όσα έκαμεν εις αυτόν ο Ιησούς, και πάντες εθαύμαζον.
\par 21 Και αφού ο Ιησούς διεπέρασε πάλιν εν τω πλοίω εις το πέραν, συνήχθη προς αυτόν όχλος πολύς, και ήτο πλησίον της θαλάσσης.
\par 22 Και ιδού, έρχεται εις των αρχισυναγώγων, ονόματι Ιάειρος, και ιδών αυτόν πίπτει προς τους πόδας αυτού
\par 23 και παρεκάλει αυτόν πολλά, λέγων ότι το θυγάτριόν μου πνέει τα λοίσθια· να έλθης και να βάλης τας χείρας σου επ' αυτήν, διά να σωθή και θέλει ζήσει.
\par 24 Και υπήγε μετ' αυτού· και ηκολούθει αυτόν όχλος πολύς, και συνέθλιβον αυτόν.
\par 25 Και γυνή τις, έχουσα ρύσιν αίματος δώδεκα έτη
\par 26 και πολλά παθούσα υπό πολλών ιατρών και δαπανήσασα πάσαν την περιουσίαν αυτής και μηδέν ωφεληθείσα, αλλά μάλλον εις το χείρον ελθούσα,
\par 27 ακούσασα περί του Ιησού, ήλθε μεταξύ του όχλου όπισθεν και ήγγισε το ιμάτιον αυτού·
\par 28 διότι έλεγεν ότι και αν τα ιμάτια αυτού εγγίσω, θέλω σωθή.
\par 29 Και ευθύς εξηράνθη η πηγή του αίματος αυτής, και ησθάνθη εν τω σώματι αυτής ότι ιατρεύθη από της μάστιγος.
\par 30 Και ευθύς ο Ιησούς, νοήσας εν εαυτώ την δύναμιν την εξελθούσαν απ' αυτού, στραφείς εν τω όχλω έλεγε· Τις ήγγισε τα ιμάτιά μου;
\par 31 Και έλεγον προς αυτόν οι μαθηταί αυτόν· Βλέπεις τον όχλον συνθλίβοντά σε, και λέγεις τις μου ήγγισε;
\par 32 Και περιέβλεπε διά να ίδη την πράξασαν τούτο.
\par 33 Η δε γυνή, φοβηθείσα και τρέμουσα, επειδή ήξευρε τι έγεινεν επ' αυτήν, ήλθε και προσέπεσεν εις αυτόν και είπε προς αυτόν πάσαν την αλήθειαν.
\par 34 Ο δε είπε προς αυτήν· Θύγατερ, η πίστις σου σε έσωσεν· ύπαγε εις ειρήνην και έσο υγιής από της μάστιγός σου.
\par 35 Ενώ αυτός ελάλει έτι, έρχονται από του αρχισυναγώγου, λέγοντες ότι η θυγάτηρ σου απέθανε· τι πλέον ενοχλείς τον Διδάσκαλον;
\par 36 Ο δε Ιησούς, ευθύς ότε ήκουσε τον λόγον λαλούμενον, λέγει προς τον αρχισυνάγωγον· Μη φοβού, μόνον πίστευε.
\par 37 Και δεν αφήκεν ουδένα να ακολουθήση αυτόν ειμή τον Πέτρον και Ιάκωβον και Ιωάννην τον αδελφόν Ιακώβου.
\par 38 Και έρχεται εις τον οίκον του αρχισυναγώγου και βλέπει θόρυβον, κλαίοντας και αλαλάζοντας πολλά,
\par 39 και εισελθών λέγει προς αυτούς· Τι θορυβείσθε και κλαίετε; το παιδίον δεν απέθανεν, αλλά κοιμάται.
\par 40 Και κατεγέλων αυτού. Ο δε, αφού εξέβαλεν άπαντας, παραλαμβάνει τον πατέρα του παιδίου και την μητέρα και τους μεθ' εαυτού και εισέρχεται όπου έκειτο το παιδίον,
\par 41 και πιάσας την χείρα του παιδίου, λέγει προς αυτήν· Ταλιθά, κούμι· το οποίον μεθερμηνευόμενον είναι, Κοράσιον, σοι λέγω, σηκώθητι.
\par 42 Και ευθύς εσηκώθη το κοράσιον και περιεπάτει· διότι ήτο ετών δώδεκα. Και εξεπλάγησαν με έκπληξιν μεγάλην.
\par 43 Και παρήγγειλεν εις αυτούς πολλά να μη μάθη μηδείς τούτο και είπε να δοθή εις αυτήν να φάγη.

\chapter{6}

\par Και εξήλθεν εκείθεν και ήλθεν εις την πατρίδα αυτού· και ακολουθούσιν αυτόν οι μαθηταί αυτού.
\par 2 Και ότε ήλθε το σάββατον, ήρχισε να διδάσκη εν τη συναγωγή· και πολλοί ακούοντες εξεπλήττοντο και έλεγον· Πόθεν εις τούτον ταύτα; και τις η σοφία η δοθείσα εις αυτόν, ώστε και θαύματα τοιαύτα γίνονται διά των χειρών αυτού;
\par 3 δεν είναι ούτος ο τέκτων, ο υιός της Μαρίας, αδελφός δε του Ιακώβου και Ιωσή και Ιούδα και Σίμωνος; και δεν είναι αι αδελφαί αυτού ενταύθα παρ' ημίν; Και εσκανδαλίζοντο εν αυτώ.
\par 4 Έλεγε δε προς αυτούς ο Ιησούς ότι δεν είναι προφήτης άνευ τιμής ειμή εν τη πατρίδι αυτού και μεταξύ των συγγενών και εν τη οικία αυτού.
\par 5 Και δεν ηδύνατο εκεί ουδέν θαύμα να κάμη, ειμή ότι επί ολίγους αρρώστους επιθέσας τας χείρας εθεράπευσεν αυτούς·
\par 6 και εθαύμαζε διά την απιστίαν αυτών. Και περιήρχετο τας κώμας κύκλω διδάσκων.
\par 7 Και προσκαλέσας τους δώδεκα, ήρχισε να αποστέλλη αυτούς δύο, και έδιδεν εις αυτούς εξουσίαν κατά των πνευμάτων των ακαθάρτων,
\par 8 και παρήγγειλεν εις αυτούς να μη βαστάζωσι μηδέν εις την οδόν ειμή ράβδον μόνον, μη σακκίον, μη άρτον, μη χαλκόν εις την ζώνην,
\par 9 αλλά να ήναι υποδεδεμένοι σανδάλια και να μη ενδύωνται δύο χιτώνας.
\par 10 Και έλεγε προς αυτούς· Όπου εάν εισέλθητε εις οικίαν, εκεί μένετε εωσού εξέλθητε εκείθεν.
\par 11 Και όσοι δεν σας δεχθώσι μηδέ σας ακούσωσιν, εξερχόμενοι εκείθεν εκτινάξατε τον κονιορτόν τον υποκάτω των ποδών σας διά μαρτυρίαν εις αυτούς. Αληθώς σας λέγω, ελαφροτέρα θέλει είσθαι η τιμωρία εις τα Σόδομα ή Γόμορρα εν ημέρα κρίσεως, παρά εις την πόλιν εκείνην.
\par 12 Και εξελθόντες εκήρυττον να μετανοήσωσι,
\par 13 και εξέβαλλον πολλά δαιμόνια και ήλειφον πολλούς αρρώστους με έλαιον και εθεράπευον.
\par 14 Και ήκουσεν ο βασιλεύς Ηρώδης· διότι φανερόν έγεινε το όνομα αυτού· και έλεγεν ότι Ιωάννης ο Βαπτιστής ανέστη εκ νεκρών, και διά τούτο ενεργούσιν αι δυνάμεις εν αυτώ.
\par 15 Άλλοι έλεγον ότι ο Ηλίας είναι· άλλοι δε έλεγον ότι προφήτης είναι ή ως εις των προφητών.
\par 16 Ακούσας δε ο Ηρώδης είπεν ότι ούτος είναι ο Ιωάννης, τον οποίον εγώ απεκεφάλισα· αυτός ανέστη εκ νεκρών.
\par 17 Διότι αυτός ο Ηρώδης απέστειλε και επίασε τον Ιωάννην και έδεσεν αυτόν εν τη φυλακή διά την Ηρωδιάδα την γυναίκα Φιλίππου του αδελφού αυτού, επειδή είχε λάβει αυτήν εις γυναίκα.
\par 18 Διότι ο Ιωάννης έλεγε προς τον Ηρώδην ότι δεν σοι είναι συγκεχωρημένον να έχης την γυναίκα του αδελφού σου.
\par 19 Η δε Ηρωδιάς εμίσει αυτόν και ήθελε να θανατώση αυτόν, και δεν ηδύνατο.
\par 20 Διότι ο Ηρώδης εφοβείτο τον Ιωάννην, γνωρίζων αυτόν άνδρα δίκαιον και άγιον, και διεφύλαττεν αυτόν και έκαμνε πολλά ακούων αυτού και ευχαρίστως ήκουεν αυτού.
\par 21 Και ότε ήλθεν αρμόδιος ημέρα, καθ' ην ο Ηρώδης έκαμνεν εν τοις γενεθλίοις αυτού δείπνον εις τους μεγιστάνας αυτού και εις τους χιλιάρχους και τους πρώτους της Γαλιλαίας,
\par 22 και εισήλθεν η θυγάτηρ αυτής της Ηρωδιάδος και εχόρευσε και ήρεσεν εις τον Ηρώδην και τους συγκαθημένους, είπεν ο βασιλεύς προς το κοράσιον· Ζήτησόν με ό,τι αν θέλης, και θέλω σοι δώσει.
\par 23 Και ώμοσε προς αυτήν ότι θέλω σοι δώσει ό,τι με ζητήσης, έως του ημίσεος της βασιλείας μου.
\par 24 Η δε εξελθούσα είπε προς την μητέρα αυτής· Τι να ζητήσω; Η δε είπε· Την κεφαλήν Ιωάννου του Βαπτιστού.
\par 25 Και ευθύς εισελθούσα μετά σπουδής εις τον βασιλέα, εζήτησε λέγουσα· Θέλω να μοι δώσης πάραυτα επί πίνακι την κεφαλήν Ιωάννου του Βαπτιστού.
\par 26 Και ο βασιλεύς, αν και ελυπήθη πολύ, διά τους όρκους όμως και τους συγκαθημένους δεν ηθέλησε να απορρίψη την αίτησιν αυτής.
\par 27 Και ευθύς αποστείλας ο βασιλεύς δήμιον, προσέταξε να φερθή η κεφαλή αυτού. Ο δε απελθών απεκεφάλισεν αυτόν εν τη φυλακή
\par 28 και έφερε την κεφαλήν αυτού επί πίνακι και έδωκεν αυτήν εις το κοράσιον, και το κοράσιον έδωκεν αυτήν εις την μητέρα αυτής.
\par 29 Και ακούσαντες οι μαθηταί αυτού, ήλθον και εσήκωσαν το πτώμα αυτού και έθεσαν αυτό εν μνημείω.
\par 30 Και συνάγονται οι απόστολοι προς τον Ιησούν και απήγγειλαν προς αυτόν πάντα, και όσα έπραξαν και όσα εδίδαξαν.
\par 31 Και είπε προς αυτούς· Έλθετε σεις αυτοί κατ' ιδίαν εις τόπον έρημον και αναπαύεσθε ολίγον· διότι ήσαν πολλοί οι ερχόμενοι και οι υπάγοντες, και ουδέ να φάγωσιν ηυκαίρουν·
\par 32 και υπήγον εις έρημον τόπον με το πλοίον κατ' ιδίαν.
\par 33 Και είδον αυτούς υπάγοντας οι όχλοι, και πολλοί εγνώρισαν αυτόν και συνέδραμον εκεί πεζοί από πασών των πόλεων και φθάσαντες προ αυτών συνήχθησαν πλησίον αυτού.
\par 34 Εξελθών δε ο Ιησούς, είδε πολύν όχλον και εσπλαγχνίσθη δι' αυτούς, επειδή ήσαν ως πρόβατα μη έχοντα ποιμένα, και ήρχισε να διδάσκη αυτούς πολλά.
\par 35 Και επειδή είχεν ήδη παρέλθει ώρα πολλή, προσελθόντες προς αυτόν οι μαθηταί αυτού, λέγουσιν ότι έρημος είναι ο τόπος και παρήλθεν ήδη πολλή ώρα·
\par 36 απόλυσον αυτούς, διά να υπάγωσιν εις τους πέριξ αγρούς και κώμας και αγοράσωσιν εις εαυτούς άρτους· διότι δεν έχουσι τι να φάγωσιν.
\par 37 Ο δε αποκριθείς είπε προς αυτούς· Δότε σεις εις αυτούς να φάγωσι. Και λέγουσι προς αυτόν· Να υπάγωμεν να αγοράσωμεν διακοσίων δηναρίων άρτους και να δώσωμεν εις αυτούς να φάγωσιν;
\par 38 Ο δε λέγει προς αυτούς· Πόσους άρτους έχετε; υπάγετε και ίδετε. Και αφού είδον, λέγουσι· Πέντε, και δύο οψάρια.
\par 39 Και προσέταξεν αυτούς να καθίσωσι πάντας επί του χλωρού χόρτου συμπόσια συμπόσια.
\par 40 Και εκάθησαν πρασιαί ανά εκατόν και ανά πεντήκοντα.
\par 41 Και λαβών τους πέντε άρτους και τα δύο οψάρια, αναβλέψας εις τον ουρανόν ηυλόγησε και κατέκοψε τους άρτους και έδιδεν εις τους μαθητάς αυτού διά να βάλωσιν έμπροσθεν αυτών, και τα δύο οψάρια εμοίρασεν εις πάντας.
\par 42 Και έφαγον πάντες και εχορτάσθησαν.
\par 43 Και εσήκωσαν από των κλασμάτων δώδεκα κοφίνους πλήρεις και από των οψαρίων.
\par 44 Ήσαν δε οι φαγόντες τους άρτους έως πεντακισχίλιοι άνδρες.
\par 45 Και ευθύς ηνάγκασε τους μαθητάς αυτού να εμβώσιν εις το πλοίον και να προϋπάγωσιν εις το πέραν προς Βηθσαϊδάν, εωσού αυτός απολύση τον όχλον·
\par 46 και απολύσας αυτούς, υπήγεν εις το όρος να προσευχηθή.
\par 47 Και ότε έγεινεν εσπέρα, το πλοίον ήτο εν τω μέσω της θαλάσσης και αυτός μόνος επί της γης.
\par 48 Και είδεν αυτούς βασανιζομένους εις το να κωπηλατώσι· διότι ήτο ο άνεμος εναντίος εις αυτούς· και περί την τετάρτην φυλακήν της νυκτός έρχεται προς αυτούς περιπατών επί της θαλάσσης, και ήθελε να περάση αυτούς.
\par 49 Οι δε ιδόντες αυτόν περιπατούντα επί της θαλάσσης ενόμισαν ότι είναι φάντασμα και ανέκραξαν·
\par 50 διότι πάντες είδον αυτόν και εταράχθησαν. Και ευθύς ελάλησε μετ' αυτών και λέγει προς αυτούς· Θαρσείτε, εγώ είμαι, μη φοβείσθε.
\par 51 Και ανέβη προς αυτούς εις το πλοίον, και έπαυσεν ο άνεμος· και εξεπλήττοντο καθ' εαυτούς λίαν καθ' υπερβολήν και εθαύμαζον.
\par 52 Διότι δεν ενόησαν εκ των άρτων, επειδή η καρδία αυτών ήτο πεπωρωμένη.
\par 53 Και διαπεράσαντες ήλθον εις την γην Γεννησαρέτ και ελιμενίσθησαν.
\par 54 Και ότε εξήλθον εκ του πλοίου, ευθύς γνωρίσαντες αυτόν,
\par 55 έδραμον εις πάντα τα περίχωρα εκείνα και ήρχισαν να περιφέρωσιν επί των κραββάτων τους αρρώστους, όπου ήκουον ότι είναι εκεί.
\par 56 Και όπου εισήρχετο εις κώμας ή πόλεις ή αγρούς, έθετον εις τας αγοράς τους ασθενείς και παρεκάλουν αυτόν να εγγίσωσι καν το κράσπεδον του ιματίου αυτού· και όσοι ήγγιζον αυτόν, εθεραπεύοντο.

\chapter{7}

\par Και συνάγονται προς αυτόν οι Φαρισαίοι και τινές των γραμματέων, ελθόντες από Ιεροσολύμων·
\par 2 και ιδόντες τινάς των μαθητών αυτού τρώγοντας άρτους με χείρας μεμολυσμένας, τουτέστιν ανίπτους, εμέμφθησαν αυτούς·
\par 3 διότι οι Φαρισαίοι και πάντες οι Ιουδαίοι, εάν δεν νίψωσι μέχρι του αγκώνος τας χείρας, δεν τρώγουσι, κρατούντες την παράδοσιν των πρεσβυτέρων·
\par 4 και επιστρέψαντες από της αγοράς, εάν δεν νιφθώσι, δεν τρώγουσιν· είναι και άλλα πολλά, τα οποία παρέλαβον να φυλάττωσι, πλύματα ποτηρίων και ξεστών και σκευών χαλκίνων και κλινών·
\par 5 έπειτα ερωτώσιν αυτόν οι Φαρισαίοι και οι γραμματείς· Διατί οι μαθηταί σου δεν περιπατούσι κατά την παράδοσιν των πρεσβυτέρων, αλλά με χείρας ανίπτους τρώγουσι τον άρτον;
\par 6 Ο δε αποκριθείς είπε προς αυτούς· ότι καλώς προεφήτευσεν ο Ησαΐας περί υμών των υποκριτών, ως είναι γεγραμμένον· Ούτος ο λαός διά των χειλέων με τιμά, η δε καρδία αυτών μακράν απέχει απ' εμού.
\par 7 Εις μάτην δε με σέβονται, διδάσκοντες διδασκαλίας εντάλματα ανθρώπων.
\par 8 Διότι αφήσαντες την εντολήν του Θεού, κρατείτε την παράδοσιν των ανθρώπων, πλύματα ξεστών και ποτηρίων, και άλλα παρόμοια τοιαύτα πολλά κάμνετε.
\par 9 Και έλεγε προς αυτούς· Καλώς αθετείτε την εντολήν του Θεού, διά να φυλάττητε την παράδοσίν σας.
\par 10 Διότι ο Μωϋσής είπε· Τίμα τον πατέρα σου και την μητέρα σου. καί· Ο κακολογών πατέρα ή μητέρα εξάπαντος να θανατόνηται·
\par 11 σεις όμως λέγετε· Εάν άνθρωπος είπη προς τον πατέρα ή προς την μητέρα, Κορβάν, τουτέστι δώρον, είναι ό,τι ήθελες ωφεληθή εξ εμού, αρκεί,
\par 12 και δεν αφίνετε πλέον αυτόν να κάμη ουδέν εις τον πατέρα αυτού ή εις την μητέρα αυτού,
\par 13 ακυρούντες τον λόγον του Θεού χάριν της παραδόσεώς σας, την οποίαν παρεδώκατε· και κάμνετε παρόμοια τοιαύτα πολλά.
\par 14 Και προσκαλέσας πάντα τον όχλον, έλεγε προς αυτούς· Ακούετέ μου πάντες και νοείτε.
\par 15 Δεν είναι ουδέν εισερχόμενον έξωθεν του ανθρώπου εις αυτόν, το οποίον δύναται να μολύνη αυτόν, αλλά τα εξερχόμενα απ' αυτού, εκείνα είναι τα μολύνοντα τον άνθρωπον.
\par 16 Ο έχων ώτα διά να ακούη, ας ακούη.
\par 17 Και ότε εισήλθεν εις οίκον από του όχλου, ηρώτων αυτόν οι μαθηταί αυτού περί της παραβολής.
\par 18 Και λέγει προς αυτούς· Ούτω και σεις ασύνετοι είσθε; δεν καταλαμβάνετε ότι παν το έξωθεν εισερχόμενον εις τον άνθρωπον δεν δύναται να μολύνη αυτόν;
\par 19 διότι δεν εισέρχεται εις την καρδίαν αυτού, αλλ' εις την κοιλίαν, και εξέρχεται εις τον αφεδρώνα, καθαρίζον πάντα τα φαγητά.
\par 20 Έλεγε δε ότι το εξερχόμενον εκ του ανθρώπου, εκείνο μολύνει τον άνθρωπον.
\par 21 Διότι έσωθεν εκ της καρδίας των ανθρώπων εξέρχονται οι διαλογισμοί οι κακοί, μοιχείαι, πορνείαι, φόνοι,
\par 22 κλοπαί, πλεονεξίαι, πονηρίαι, δόλος, ασέλγεια, βλέμμα πονηρόν· βλασφημία, υπερηφανία, αφροσύνη·
\par 23 πάντα ταύτα τα πονηρά έσωθεν εξέρχονται και μολύνουσι τον άνθρωπον.
\par 24 Και σηκωθείς εκείθεν υπήγεν εις τα μεθόρια Τύρου και Σιδώνος. Και εισελθών εις την οικίαν, δεν ήθελε να μάθη τούτο μηδείς, δεν ηδυνήθη όμως να κρυφθή.
\par 25 Διότι ακούσασα περί αυτού γυνή τις, της οποίας το θυγάτριον είχε πνεύμα ακάθαρτον, ήλθε και προσέπεσεν εις τους πόδας αυτού·
\par 26 ήτο δε η γυνή Ελληνίς, Συροφοίνισσα το γένος· και παρεκάλει αυτόν να εκβάλη το δαιμόνιον εκ της θυγατρός αυτής.
\par 27 Ο δε Ιησούς είπε προς αυτήν· Άφες πρώτον να χορτασθώσι τα τέκνα· διότι δεν είναι καλόν να λάβη τις τον άρτον των τέκνων και να ρίψη εις τα κυνάρια.
\par 28 Η δε απεκρίθη και λέγει προς αυτόν· Ναι, Κύριε· αλλά και τα κυνάρια υποκάτω της τραπέζης τρώγουσιν από των ψιχίων των παιδίων.
\par 29 Και είπε προς αυτήν· Διά τούτον τον λόγον ύπαγε· εξήλθε το δαιμόνιον από της θυγατρός σου.
\par 30 Και ότε υπήγεν εις τον οίκον αυτής, εύρεν ότι το δαιμόνιον εξήλθε και την θυγατέρα κειμένην επί της κλίνης.
\par 31 Και πάλιν εξελθών εκ των ορίων Τύρου και Σιδώνος ήλθε προς την θάλασσαν της Γαλιλαίας ανά μέσον των ορίων της Δεκαπόλεως.
\par 32 Και φέρουσι προς αυτόν κωφόν μογιλάλον και παρακαλούσιν αυτόν να επιθέση την χείρα επ' αυτόν.
\par 33 Και παραλαβών αυτόν κατ' ιδίαν από του όχλου έβαλε τους δακτύλους αυτού εις τα ώτα αυτού, και πτύσας ήγγισε την γλώσσαν αυτού,
\par 34 και αναβλέψας εις τον ουρανόν, εστέναξε και λέγει προς αυτόν· Εφφαθά, τουτέστιν Ανοίχθητι.
\par 35 Και ευθύς ηνοίχθησαν τα ώτα αυτού και ελύθη ο δεσμός της γλώσσης αυτού, και ελάλει ορθώς.
\par 36 Και παρήγγειλεν εις αυτούς να μη είπωσι τούτο εις μηδένα· πλην όσον αυτός παρήγγελλεν εις αυτούς, τόσον περισσότερον εκείνοι εκήρυττον.
\par 37 Και εξεπλήττοντο καθ' υπερβολήν, λέγοντες· Καλώς έπραξε τα πάντα· και τους κωφούς κάμνει να ακούωσι και τους αλάλους να λαλώσι.

\chapter{8}

\par Εν εκείναις ταις ημέραις, επειδή ήτο πάμπολυς όχλος και δεν είχον τι να φάγωσι, προσκαλέσας ο Ιησούς τους μαθητάς αυτού λέγει προς αυτούς·
\par 2 Σπλαγχνίζομαι διά τον όχλον, ότι τρεις ήδη ημέρας μένουσι πλησίον μου και δεν έχουσι τι να φάγωσι·
\par 3 και εάν απολύσω αυτούς νήστεις εις τους οίκους αυτών, θέλουσιν αποκάμει καθ' οδόν· διότι τινές εξ αυτών ήλθον μακρόθεν.
\par 4 Και απεκρίθησαν προς αυτόν οι μαθηταί αυτού· Πόθεν θέλει τις δυνηθή να χορτάση τούτους από άρτων εδώ επί της ερημίας;
\par 5 Και ηρώτησεν αυτούς· Πόσους άρτους έχετε; Οι δε είπον· Επτά.
\par 6 Και προσέταξε τον όχλον να καθήσωσιν επί της γής· και λαβών τους επτά άρτους, αφού ευχαρίστησεν, έκοψε και έδιδεν εις τους μαθητάς αυτού διά να βάλωσιν έμπροσθεν του όχλου· και έβαλον.
\par 7 Είχον και ολίγα οψαράκια· και ευλογήσας είπε να βάλωσι και αυτά.
\par 8 Έφαγον δε και εχορτάσθησαν, και εσήκωσαν περισσεύματα κλασμάτων επτά σπυρίδας.
\par 9 Ήσαν δε οι φαγόντες ως τετρακισχίλιοι· και απέλυσεν αυτούς.
\par 10 Και ευθύς εμβάς εις το πλοίον μετά των μαθητών αυτού, ήλθεν εις τα μέρη Δαλμανουθά.
\par 11 Και εξήλθον οι Φαρισαίοι και ήρχισαν να κάμνωσιν ερωτήσεις προς αυτόν, και εζήτουν παρ' αυτού σημείον από του ουρανού, πειράζοντες αυτόν.
\par 12 Τότε αναστενάξας εκ καρδίας αυτού, λέγει· Διά τι η γενεά αύτη σημείον ζητεί; αληθώς σας λέγω, δεν θέλει δοθή εις την γενεάν ταύτην σημείον.
\par 13 Και αφήσας αυτούς εισήλθε πάλιν εις το πλοίον και απήλθεν εις το πέραν.
\par 14 Ελησμόνησαν δε να λάβωσιν άρτους και δεν είχον μεθ' εαυτών εν τω πλοίω ειμή ένα άρτον.
\par 15 Και παρήγγελλεν εις αυτούς, λέγων· Βλέπετε, προσέχετε από της ζύμης των Φαρισαίων και της ζύμης του Ηρώδου.
\par 16 Και διελογίζοντο προς αλλήλους, λέγοντες ότι άρτους δεν έχομεν.
\par 17 Νοήσας δε ο Ιησούς, λέγει προς αυτούς· Τι διαλογίζεσθε ότι δεν έχετε άρτους; έτι δεν νοείτε ουδέ καταλαμβάνετε; έτι πεπωρωμένην έχετε την καρδίαν σας;
\par 18 οφθαλμούς έχοντες δεν βλέπετε, και ώτα έχοντες δεν ακούετε; και δεν ενθυμείσθε;
\par 19 ότε έκοψα τους πέντε άρτους εις τους πεντακισχιλίους, πόσους κοφίνους πλήρεις κλασμάτων εσηκώσατε; Λέγουσι προς αυτόν· δώδεκα.
\par 20 Και ότε τους επτά εις τους τετρακισχιλίους, πόσας σπυρίδας πλήρεις κλασμάτων εσηκώσατε; Οι δε είπον· Επτά.
\par 21 Και έλεγε προς αυτούς· Πως δεν καταλαμβάνετε;
\par 22 Και έρχεται εις Βηθσαϊδάν. Και φέρουσι προς αυτόν τυφλόν και παρακαλούσιν αυτόν να εγγίση αυτόν.
\par 23 Και πιάσας την χείρα του τυφλού, έφερεν αυτόν έξω της κώμης και πτύσας εις τα όμματα αυτού, επέθεσεν επ' αυτόν τας χείρας και ηρώτα αυτόν αν βλέπη τι.
\par 24 Και αναβλέψας έλεγε· Βλέπω τους ανθρώπους, ό,τι ως δένδρα βλέπω περιπατούντας.
\par 25 Έπειτα πάλιν επέθεσε τας χείρας επί τους οφθαλμούς αυτού και έκαμεν αυτόν να αναβλέψη, και αποκατεστάθη η όρασις αυτού, και είδε καθαρώς άπαντας.
\par 26 Και απέστειλεν αυτόν εις τον οίκον αυτού, λέγων· Μηδέ εις την κώμην εισέλθης μηδέ είπης τούτο εις τινά εν τη κώμη.
\par 27 Και εξήλθεν ο Ιησούς και οι μαθηταί αυτού εις τας κώμας της Καισαρείας Φιλίππου· και καθ' οδόν ηρώτα τους μαθητάς αυτού, λέγων προς αυτούς· Τίνα με λέγουσιν οι άνθρωποι ότι είμαι;
\par 28 Οι δε απεκρίθησαν· Ιωάννην τον Βαπτιστήν, και άλλοι τον Ηλίαν, άλλοι δε ένα των προφητών.
\par 29 Και αυτός λέγει προς αυτούς· Αλλά σεις τίνα με λέγετε ότι είμαι; Και αποκριθείς ο Πέτρος, λέγει προς αυτόν· Συ είσαι ο Χριστός.
\par 30 Και παρήγγειλεν αυστηρώς εις αυτούς να μη λέγωσιν εις μηδένα περί αυτού.
\par 31 Και ήρχισε να διδάσκη αυτούς ότι πρέπει ο Υιός του ανθρώπου να πάθη πολλά, και να καταφρονηθή από των πρεσβυτέρων και αρχιερέων και γραμματέων, και να θανατωθή, και μετά τρεις ημέρας να αναστηθή·
\par 32 και ελάλει τον λόγον παρρησία. Και παραλαβών αυτόν ο Πέτρος κατ' ιδίαν, ήρχισε να επιτιμά αυτόν.
\par 33 Ο δε επιστραφείς και ιδών τους μαθητάς αυτού, επετίμησε τον Πέτρον λέγων· Ύπαγε οπίσω μου, Σατανά· διότι δεν φρονείς τα του Θεού, αλλά τα των ανθρώπων.
\par 34 Και προσκαλέσας τον όχλον μετά των μαθητών αυτού, είπε προς αυτούς· Όστις θέλει να έλθη οπίσω μου, ας απαρνηθή εαυτόν και ας σηκώση τον σταυρόν αυτού, και ας με ακολουθή.
\par 35 Διότι όστις θέλει να σώση την ζωήν αυτού, θέλει απολέσει αυτήν· και όστις απολέση την ζωήν αυτού ένεκεν εμού και του ευαγγελίου, ούτος θέλει σώσει αυτήν.
\par 36 Επειδή τι θέλει ωφελήσει τον άνθρωπον, εάν κερδήση τον κόσμον όλον και ζημιωθή την ψυχήν αυτού;
\par 37 Η τι θέλει δώσει ο άνθρωπος εις ανταλλαγήν της ψυχής αυτού;
\par 38 Διότι όστις αισχυνθή δι' εμέ και διά τους λόγους μου εν τη γενεά ταύτη τη μοιχαλίδι και αμαρτωλώ, και ο Υιός του ανθρώπου θέλει αισχυνθή δι' αυτόν, όταν έλθη εν τη δόξη του Πατρός αυτού μετά των αγγέλων.

\chapter{9}

\par Και έλεγε προς αυτούς· Αληθώς, σας λέγω ότι είναι τινές των εδώ ισταμένων, οίτινες δεν θέλουσι γευθή θάνατον, εωσού ίδωσι την βασιλείαν του Θεού ελθούσαν μετά δυνάμεως.
\par 2 Και μεθ' ημέρας εξ παραλαμβάνει ο Ιησούς τον Πέτρον και τον Ιάκωβον και τον Ιωάννην και αναβιβάζει αυτούς εις όρος υψηλόν κατ' ιδίαν μόνους· και μετεμορφώθη έμπροσθεν αυτών·
\par 3 και τα ιμάτια αυτού έγειναν στιλπνά, λευκά λίαν ως χιών, οποία λευκαντής επί της γης δεν δύναται να λευκάνη.
\par 4 Και εφάνη εις αυτούς ο Ηλίας μετά του Μωϋσέως, και ήσαν συλλαλούντες μετά του Ιησού.
\par 5 Και αποκριθείς ο Πέτρος λέγει προς τον Ιησούν· Ραββί, καλόν είναι να ήμεθα εδώ· και ας κάμωμεν τρεις σκηνάς, διά σε μίαν και διά τον Μωϋσήν μίαν και διά τον Ηλίαν μίαν.
\par 6 Διότι δεν ήξευρε τι να είπη· επειδή ήσαν πεφοβισμένοι.
\par 7 Και νεφέλη επεσκίασεν αυτούς, και ήλθε φωνή εκ της νεφέλης, λέγουσα· Ούτος είναι ο Υιός μου ο αγαπητός· αυτού ακούετε.
\par 8 Και εξαίφνης περιβλέψαντες, δεν είδον πλέον ουδένα, αλλά τον Ιησούν μόνον μεθ' εαυτών.
\par 9 Ενώ δε κατέβαινον από του όρους, παρήγγειλεν εις αυτούς να μη διηγηθώσιν εις μηδένα όσα είδον, ειμή όταν ο Υιός του ανθρώπου αναστηθή εκ νεκρών.
\par 10 Και εφύλαξαν τον λόγον εν εαυτοίς, συζητούντες προς αλλήλους τι είναι το να αναστηθή εκ νεκρών.
\par 11 Και ηρώτων αυτόν λέγοντες, Διά τι λέγουσιν οι γραμματείς ότι πρέπει να έλθη ο Ηλίας πρώτον;
\par 12 Ο δε αποκριθείς είπε προς αυτούς· Ο Ηλίας μεν ελθών πρώτον αποκαθιστά πάντα· και ότι είναι γεγραμμένον περί του Υιού του ανθρώπου ότι πρέπει να πάθη πολλά και να εξουδενωθή·
\par 13 σας λέγω όμως ότι και ο Ηλίας ήλθε, και έπραξαν εις αυτόν όσα ηθέλησαν, καθώς είναι γεγραμμένον περί αυτού.
\par 14 Και ότε ήλθε προς τους μαθητάς, είδε περί αυτούς όχλον πολύν και γραμματείς κάμνοντας συζητήσεις μετ' αυτών.
\par 15 Και ευθύς πας ο όχλος ιδών αυτόν έγεινεν έκθαμβος και προστρέχοντες ησπάζοντο αυτόν.
\par 16 Και ηρώτησε τους γραμματείς· Τι συζητείτε μετ' αυτών;
\par 17 Και αποκριθείς εις εκ του όχλου, είπε· Διδάσκαλε, έφερα προς σε τον υιόν μου, έχοντα πνεύμα άλαλον.
\par 18 Και όπου πιάση αυτόν σπαράττει αυτόν, και αφρίζει και τρίζει τους οδόντας αυτού και ξηραίνεται· και είπον προς τους μαθητάς σου να εκβάλωσιν αυτό, αλλά δεν ηδυνήθησαν.
\par 19 Εκείνος δε αποκριθείς προς αυτόν, λέγει· Ω γενεά άπιστος, έως πότε θέλω είσθαι μεθ' υμών; έως πότε θέλω υπομένει υμάς; φέρετε αυτόν προς εμέ.
\par 20 Και έφεραν αυτόν προς αυτόν. Και ως είδεν αυτόν, ευθύς το πνεύμα εσπάραξεν αυτόν, και πεσών επί της γης εκυλίετο αφρίζων.
\par 21 Και ηρώτησε τον πατέρα αυτού· Πόσος καιρός είναι αφού τούτο έγεινεν εις αυτόν; Ο δε είπε· Παιδιόθεν.
\par 22 Και πολλάκις αυτόν και εις πυρ έρριψε και εις ύδατα, διά να απολέση αυτόν· αλλ' εάν δύνασαί τι, βοήθησον ημάς, σπλαγχνισθείς εφ' ημάς.
\par 23 Ο δε Ιησούς είπε προς αυτόν· Το εάν δύνασαι να πιστεύσης, πάντα είναι δυνατά εις τον πιστεύοντα.
\par 24 Και ευθύς κράξας ο πατήρ του παιδίου μετά δακρύων, έλεγε· Πιστεύω, Κύριε· βοήθει εις την απιστίαν μου.
\par 25 Ιδών δε ο Ιησούς ότι επισυντρέχει όχλος, επετίμησε το πνεύμα το ακάθαρτον, λέγων προς αυτό· το πνεύμα το άλαλον και κωφόν, εγώ σε προστάζω, Έξελθε απ' αυτού και μη εισέλθης πλέον εις αυτόν.
\par 26 Και το πνεύμα κράξαν και πολλά σπαράξαν αυτόν, εξήλθε, και έγεινεν ως νεκρός, ώστε πολλοί έλεγον ότι απέθανεν.
\par 27 Ο δε Ιησούς πιάσας αυτόν από της χειρός ήγειρεν αυτόν, και εσηκώθη.
\par 28 Και ότε εισήλθεν εις οίκον, οι μαθηταί αυτού ηρώτων αυτόν κατ' ιδίαν, Διά τι ημείς δεν ηδυνήθημεν να εκβάλωμεν αυτό;
\par 29 Και είπε προς αυτούς· Τούτο το γένος δεν δύναται να εξέλθη δι' ουδενός άλλου τρόπου ειμή διά προσευχής και νηστείας.
\par 30 Και εξελθόντες εκείθεν διέβαινον διά της Γαλιλαίας, και δεν ήθελε να μάθη τούτο ουδείς.
\par 31 Διότι εδίδασκε τους μαθητάς αυτού και έλεγε προς αυτούς ότι ο Υιός του ανθρώπου παραδίδεται εις χείρας ανθρώπων, και θέλουσι θανατώσει αυτόν, και θανατωθείς την τρίτην ημέραν θέλει αναστηθή.
\par 32 Εκείνοι όμως δεν ηνόουν τον λόγον και εφοβούντο να ερωτήσωσιν αυτόν.
\par 33 Και ήλθεν εις Καπερναούμ· και ότε εισήλθεν εις την οικίαν, ηρώτα αυτούς· Τι διελογίζεσθε καθ' οδόν προς αλλήλους;
\par 34 Οι δε εσιώπων· διότι καθ' οδόν διελέχθησαν προς αλλήλους τις είναι μεγαλήτερος.
\par 35 Και καθήσας εκάλεσε τους δώδεκα και λέγει προς αυτούς· Όστις θέλει να ήναι πρώτος, θέλει είσθαι πάντων έσχατος και πάντων υπηρέτης.
\par 36 Και λαβών παιδίον έστησεν αυτό εν τω μέσω αυτών, και εναγκαλισθείς αυτό είπε προς αυτούς·
\par 37 Όστις δεχθή εν των τοιούτων παιδίων εις το όνομά μου, εμέ δέχεται· και όστις δεχθή εμέ, δεν δέχεται εμέ, αλλά τον αποστείλαντά με.
\par 38 Απεκρίθη δε προς αυτόν ο Ιωάννης, λέγων· Διδάσκαλε, είδομέν τινά εκβάλλοντα δαιμόνια εις το όνομά σου, όστις δεν ακολουθεί ημάς, και ημποδίσαμεν αυτόν, διότι δεν ακολουθεί ημάς.
\par 39 Ο δε Ιησούς είπε· Μη εμποδίζετε αυτόν· διότι δεν είναι ουδείς όστις θέλει κάμει θαύμα εις το όνομά μου και θέλει δυνηθή ευθύς να με κακολογήση.
\par 40 Επειδή όστις δεν είναι καθ' ημών, είναι υπέρ ημών.
\par 41 Διότι όστις σας ποτίση ποτήριον ύδατος εις το όνομά μου, επειδή είσθε του Χριστού, αληθώς σας λέγω, δεν θέλει χάσει τον μισθόν αυτού.
\par 42 Και όστις σκανδαλίση ένα των μικρών των πιστευόντων εις εμέ, συμφέρει εις αυτόν καλήτερον να περιτεθή μύλου πέτρα περί τον τράχηλον αυτού και να ριφθή εις την θάλασσαν.
\par 43 Και εάν σε σκανδαλίζη η χειρ σου, απόκοψον αυτήν· καλήτερόν σοι είναι να εισέλθης εις την ζωήν κουλλός, παρά έχων τας δύο χείρας να απέλθης εις την γέενναν, εις το πυρ το άσβεστον,
\par 44 όπου ο σκώληξ αυτών δεν τελευτά και το πυρ δεν σβύνεται.
\par 45 Και εάν ο πους σου σε σκανδαλίζη, απόκοψον αυτόν· καλήτερόν σοι είναι να εισέλθης εις την ζωήν χωλός, παρά έχων τους δύο πόδας να ριφθής εις την γέενναν, εις το πυρ το άσβεστον,
\par 46 όπου ο σκώληξ αυτών δεν τελευτά και το πυρ δεν σβύνεται.
\par 47 Και εάν ο οφθαλμός σου σε σκανδαλίζη, έκβαλε αυτόν· καλήτερόν σοι είναι να εισέλθης μονόφθαλμος εις την βασιλείαν του Θεού, παρά έχων δύο οφθαλμούς να ριφθής εις την γέενναν του πυρός,
\par 48 όπου ο σκώληξ αυτών δεν τελευτά και το πυρ δεν σβύνεται.
\par 49 Διότι πας τις με πυρ θέλει αλατισθή, και πάσα θυσία με άλας θέλει αλατισθή.
\par 50 Καλόν το άλας· αλλ' εάν το άλας γείνη ανάλατον, με τι θέλετε αρτύσει αυτό; έχετε άλας εν εαυτοίς και ειρηνεύετε εν αλλήλοις.

\chapter{10}

\par Και σηκωθείς εκείθεν έρχεται εις τα όρια της Ιουδαίας διά του πέραν του Ιορδάνου, και συνέρχονται πάλιν όχλοι προς αυτόν, και ως εσυνείθιζε, πάλιν εδίδασκεν αυτούς.
\par 2 Και προσελθόντες οι Φαρισαίοι, ηρώτησαν αυτόν αν συγχωρήται εις άνδρα να χωρισθή την γυναίκα αυτού, πειράζοντες αυτόν.
\par 3 Ο δε αποκριθείς είπε προς αυτούς· τι προσέταξεν εις εσάς ο Μωϋσής;
\par 4 Οι δε είπον· Ο Μωϋσής συνεχώρησε να γράψη έγγραφον διαζυγίου και να χωρισθή αυτήν.
\par 5 Και αποκριθείς ο Ιησούς είπε προς αυτούς· Διά την σκληροκαρδίαν σας έγραψεν εις εσάς την εντολήν ταύτην·
\par 6 απ' αρχής όμως της κτίσεως άρσεν και θήλυ εποίησεν αυτούς ο Θεός·
\par 7 ένεκεν τούτου θέλει αφήσει άνθρωπος τον πατέρα αυτού και την μητέρα, και θέλει προσκολληθή εις την γυναίκα αυτού,
\par 8 και θέλουσιν είσθαι οι δύο εις σάρκα μίαν. Ώστε δεν είναι πλέον δύο, αλλά μία σάρξ·
\par 9 εκείνο λοιπόν, το οποίον ο Θεός συνέζευξεν, άνθρωπος ας μη χωρίζη.
\par 10 Και εν τη οικία πάλιν οι μαθηταί αυτού ηρώτησαν αυτόν περί του αυτού,
\par 11 και λέγει προς αυτούς· Όστις χωρισθή την γυναίκα αυτού και νυμφευθή άλλην, πράττει μοιχείαν εις αυτήν·
\par 12 και εάν γυνή χωρισθή τον άνδρα αυτής και συζευχθή με άλλον, μοιχεύεται.
\par 13 Και έφεραν προς αυτόν παιδία, διά να εγγίση αυτά· οι δε μαθηταί επέπληττον τους φέροντας.
\par 14 Ιδών δε ο Ιησούς ηγανάκτησε και είπε προς αυτούς· Αφήσατε τα παιδία να έρχωνται προς εμέ, και μη εμποδίζετε αυτά· διότι των τοιούτων είναι η βασιλεία του Θεού.
\par 15 Αληθώς σας λέγω, Όστις δεν δεχθή την βασιλείαν του Θεού ως παιδίον, δεν θέλει εισέλθει εις αυτήν.
\par 16 Και εναγκαλισθείς αυτά, έθετε τας χείρας επ' αυτά και ηυλόγει αυτά.
\par 17 Ενώ δε εξήρχετο εις την οδόν, έδραμέ τις και γονυπετήσας έμπροσθεν αυτού, ηρώτα αυτόν· Διδάσκαλε αγαθέ, τι να κάμω διά να κληρονομήσω ζωήν αιώνιον;
\par 18 Και ο Ιησούς είπε προς αυτόν· Τι με λέγεις αγαθόν; Ουδείς αγαθός ειμή εις, ο Θεός.
\par 19 Τας εντολάς εξεύρεις· Μη μοιχεύσης, Μη φονεύσης, Μη κλέψης, Μη ψευδομαρτυρήσης, Μη αποστερήσης, Τίμα τον πατέρα σου και την μητέρα.
\par 20 Ο δε αποκριθείς είπε προς αυτόν· Διδάσκαλε, ταύτα πάντα εφύλαξα εκ νεότητός μου.
\par 21 Και ο Ιησούς εμβλέψας εις αυτόν, ηγάπησεν αυτόν και είπε προς αυτόν· Εν σοι λείπει· ύπαγε, πώλησον όσα έχεις και δος εις τους πτωχούς, και θέλεις έχει θησαυρόν εν ουρανώ, και ελθέ, ακολούθει μοι, σηκώσας τον σταυρόν.
\par 22 Εκείνος όμως σκυθρωπάσας διά τον λόγον, ανεχώρησε λυπούμενος· διότι είχε κτήματα πολλά.
\par 23 Και περιβλέψας ο Ιησούς, λέγει προς τους μαθητάς αυτού· Πόσον δυσκόλως θέλουσιν εισέλθει εις την βασιλείαν του Θεού οι έχοντες τα χρήματα.
\par 24 Οι δε μαθηταί εξεπλήττοντο διά τους λόγους αυτού. Και ο Ιησούς πάλιν αποκριθείς λέγει προς αυτούς· Τέκνα, πόσον δύσκολον είναι να εισέλθωσιν εις την βασιλείαν του Θεού οι έχοντες το θάρρος αυτών εις τα χρήματα.
\par 25 Ευκολώτερον είναι κάμηλος να περάση διά της τρύπης της βελόνης παρά πλούσιος να εισέλθη εις την βασιλείαν του Θεού.
\par 26 Εκείνοι δε σφόδρα εξεπλήττοντο, λέγοντες προς εαυτούς· Και τις δύναται να σωθή;
\par 27 Εμβλέψας δε εις αυτούς ο Ιησούς, λέγει· Παρά ανθρώποις είναι αδύνατον, αλλ' ουχί παρά τω Θεώ· διότι τα πάντα είναι δυνατά παρά τω Θεώ.
\par 28 Και ήρχισεν ο Πέτρος να λέγη προς αυτόν· Ιδού, ημείς αφήκαμεν πάντα και σε ηκολουθήσαμεν.
\par 29 Αποκριθείς δε ο Ιησούς είπεν· Αληθώς σας λέγω, δεν είναι ουδείς όστις, αφήσας οικίαν ή αδελφούς ή αδελφάς ή πατέρα ή μητέρα ή γυναίκα ή τέκνα ή αγρούς ένεκεν εμού και του ευαγγελίου,
\par 30 δεν θέλει λάβει εκατονταπλασίονα τώρα εν τω καιρώ τούτω, οικίας και αδελφούς και αδελφάς και μητέρας και τέκνα και αγρούς μετά διωγμών, και εν τω ερχομένω αιώνι ζωήν αιώνιον.
\par 31 Πολλοί όμως πρώτοι θέλουσιν είσθαι έσχατοι και οι έσχατοι πρώτοι.
\par 32 Ήσαν δε εν τη οδώ αναβαίνοντες εις Ιεροσόλυμα· και ο Ιησούς προεπορεύετο αυτών, και εθαύμαζον και ακολουθούντες εφοβούντο. Και παραλαβών πάλιν τους δώδεκα, ήρχισε να λέγη προς αυτούς τα μέλλοντα να συμβώσιν εις αυτόν,
\par 33 ότι ιδού, αναβαίνομεν εις Ιεροσόλυμα και ο Υιός του ανθρώπου θέλει παραδοθή εις τους αρχιερείς και εις τους γραμματείς, και θέλουσι καταδικάσει αυτόν εις θάνατον και θέλουσι παραδώσει αυτόν εις τα έθνη,
\par 34 και θέλουσιν εμπαίξει αυτόν και μαστιγώσει αυτόν και θέλουσιν εμπτύσει εις αυτόν και θανατώσει αυτόν, και την τρίτην ημέραν θέλει αναστηθή.
\par 35 Τότε έρχονται προς αυτόν ο Ιάκωβος και Ιωάννης, οι υιοί του Ζεβεδαίου, λέγοντες· Διδάσκαλε, θέλομεν να κάμης εις ημάς ό,τι ζητήσωμεν.
\par 36 Ο δε είπε προς αυτούς· Τι θέλετε να κάμω εις εσάς;
\par 37 Οι δε είπον προς αυτόν· Δος εις ημάς να καθήσωμεν εις εκ δεξιών σου και εις εξ αριστερών σου εν τη δόξη σου.
\par 38 Ο δε Ιησούς είπε προς αυτούς· Δεν εξεύρετε τι ζητείτε. Δύνασθε να πίητε το ποτήριον, το οποίον εγώ πίνω, και να βαπτισθήτε το βάπτισμα, το οποίον εγώ βαπτίζομαι;
\par 39 Οι δε είπον προς αυτόν· Δυνάμεθα. Ο δε Ιησούς είπε προς αυτούς· το μεν ποτήριον, το οποίον εγώ πίνω, θέλετε πίει, και το βάπτισμα το οποίον εγώ βαπτίζομαι, θέλετε βαπτισθή·
\par 40 το να καθήσητε όμως εκ δεξιών μου και εξ αριστερών μου δεν είναι εμού να δώσω, αλλ' εις όσους είναι ητοιμασμένον.
\par 41 Και ακούσαντες οι δέκα ήρχισαν να αγανακτώσι περί Ιακώβου και Ιωάννου.
\par 42 Ο δε Ιησούς προσκαλέσας αυτούς, λέγει προς αυτούς· Εξεύρετε ότι οι νομιζόμενοι άρχοντες των εθνών κατακυριεύουσιν αυτά και οι μεγάλοι αυτών κατεξουσιάζουσιν αυτά·
\par 43 ούτως όμως δεν θέλει είσθαι εν υμίν, αλλ' όστις θέλει να γείνη μέγας εν υμίν, θέλει είσθαι υπηρέτης υμών,
\par 44 και όστις εξ υμών θέλει να γείνη πρώτος, θέλει είσθαι δούλος πάντων·
\par 45 διότι ο Υιός του ανθρώπου δεν ήλθε διά να υπηρετηθή, αλλά διά να υπηρετήση και να δώση την ζωήν αυτού λύτρον αντί πολλών.
\par 46 Και έρχονται εις Ιεριχώ. Και ενώ εξήρχετο από της Ιεριχώ αυτός και οι μαθηταί αυτού και όχλος ικανός, ο υιός του Τιμαίου Βαρτίμαιος ο τυφλός εκάθητο παρά την οδόν ζητών.
\par 47 Και ακούσας ότι είναι Ιησούς ο Ναζωραίος, ήρχισε να κράζη και να λέγη· Υιέ του Δαβίδ Ιησού, ελέησόν με.
\par 48 Και επέπληττον αυτόν πολλοί διά να σιωπήση· αλλ' εκείνος πολλώ μάλλον έκραζεν· Υιέ του Δαβίδ, ελέησόν με.
\par 49 Και σταθείς ο Ιησούς, είπε να κραχθή· και κράζουσι τον τυφλόν, λέγοντες προς αυτόν· Θάρσει, σηκώθητι· σε κράζει.
\par 50 Και εκείνος απορρίψας το ιμάτιον αυτού, εσηκώθη και ήλθε προς τον Ιησούν.
\par 51 Και αποκριθείς λέγει προς αυτόν ο Ιησούς· Τι θέλεις να σοι κάμω; Και ο τυφλός είπε προς αυτόν· Ραββουνί, να αναβλέψω.
\par 52 Ο δε Ιησούς είπε προς αυτόν· Ύπαγε, η πίστις σου σε έσωσε. Και ευθύς ανέβλεψε και ηκολούθει τον Ιησούν εν τη οδώ.

\chapter{11}

\par Και ότε πλησιάζουσιν εις Ιερουσαλήμ εις Βηθφαγή και Βηθανίαν προς το όρος των Ελαιών, αποστέλλει δύο των μαθητών αυτού
\par 2 και λέγει προς αυτούς· Υπάγετε εις την κώμην την κατέναντι υμών, και ευθύς εισερχόμενοι εις αυτήν θέλετε ευρεί πωλάριον δεδεμένον, επί του οποίου ουδείς άνθρωπος εκάθησε· λύσατε αυτό και φέρετε.
\par 3 Και εάν τις είπη προς εσάς· Διά τι κάμνετε τούτο; είπατε ότι ο Κύριος έχει χρείαν αυτού, και ευθύς θέλει αποστείλει αυτό εδώ.
\par 4 Και υπήγον και εύρον το πωλάριον δεδεμένον προς την θύραν έξω επί της διόδου, και λύουσιν αυτό.
\par 5 Και τινές των εκεί ισταμένων έλεγον προς αυτούς· Τι κάμνετε λύοντες το πωλάριον;
\par 6 Οι δε είπον προς αυτούς καθώς παρήγγειλεν ο Ιησούς, και αφήκαν αυτούς.
\par 7 Και έφεραν το πωλάριον προς τον Ιησούν και έβαλον επ' αυτού τα ιμάτια αυτών, και εκάθησεν επ' αυτού.
\par 8 Πολλοί δε έστρωσαν τα ιμάτια αυτών εις την οδόν, άλλοι δε έκοπτον κλάδους από των δένδρων και έστρωνον εις την οδόν.
\par 9 Και οι προπορευόμενοι και οι ακολουθούντες έκραζον, λέγοντες· Ωσαννά, ευλογημένος ο ερχόμενος εν ονόματι Κυρίου.
\par 10 Ευλογημένη η ερχομένη βασιλεία εν ονόματι Κυρίου του πατρός ημών Δαβίδ· Ωσαννά εν τοις υψίστοις.
\par 11 Και εισήλθεν ο Ιησούς εις Ιεροσόλυμα και εις το ιερόν· και αφού περιέβλεψε πάντα, επειδή η ώρα ήτο ήδη προς εσπέραν, εξήλθεν εις Βηθανίαν μετά των δώδεκα.
\par 12 Και τη επαύριον, αφού εξήλθον από Βηθανίας, επείνασε·
\par 13 και ιδών μακρόθεν συκήν έχουσαν φύλλα, ήλθεν αν τυχόν εύρη τι εν αυτή· και ελθών επ' αυτήν ουδέν εύρεν ειμή φύλλα· διότι δεν ήτο καιρός σύκων.
\par 14 Και αποκριθείς ο Ιησούς είπε προς αυτήν· Μηδείς πλέον εις τον αιώνα να μη φάγη καρπόν από σου. Και ήκουον τούτο οι μαθηταί αυτού.
\par 15 Και έρχονται εις Ιεροσόλυμα· και εισελθών ο Ιησούς εις το ιερόν, ήρχισε να εκβάλλη τους πωλούντας και αγοράζοντας εν τω ιερώ, και τας τραπέζας των αργυραμοιβών και τα καθίσματα των πωλούντων τας περιστεράς ανέτρεψε,
\par 16 και δεν άφινε να περάση τις σκεύος διά του ιερού,
\par 17 και εδίδασκε, λέγων προς αυτούς· Δεν είναι γεγραμμένον, ότι Ο οίκός μου θέλει ονομάζεσθαι οίκος προσευχής διά πάντα τα έθνη; σεις δε εκάμετε αυτόν σπήλαιον ληστών.
\par 18 Και ήκουσαν οι γραμματείς και οι αρχιερείς και εζήτουν πως να απολέσωσιν αυτόν· διότι εφοβούντο αυτόν, επειδή πας ο όχλος εξεπλήττετο εις την διδαχήν αυτού.
\par 19 Και ότε έγεινεν εσπέρα, εξήρχετο έξω της πόλεως.
\par 20 Και το πρωΐ διαβαίνοντες είδον την συκήν εξηραμμένην εκ ριζών.
\par 21 Και ενθυμηθείς ο Πέτρος, λέγει προς αυτόν· Ραββί, ίδε, η συκή, την οποίαν κατηράσθης, εξηράνθη.
\par 22 Και αποκριθείς ο Ιησούς, λέγει προς αυτούς· Έχετε πίστιν Θεού.
\par 23 Διότι αληθώς σας λέγω ότι όστις είπη προς το όρος τούτο, Σηκώθητε και ρίφθητι εις την θάλασσαν, και δεν διστάση εν τη καρδία αυτού, αλλά πιστεύση ότι εκείνα τα οποία λέγει γίνονται, θέλει γείνει εις αυτόν ό,τι εάν είπη.
\par 24 Διά τούτο σας λέγω, Πάντα όσα προσευχόμενοι ζητείτε, πιστεύετε ότι λαμβάνετε, και θέλει γείνει εις εσάς.
\par 25 Και όταν ίστασθε προσευχόμενοι, συγχωρείτε εάν έχητέ τι κατά τινός, διά να συγχωρήση εις εσάς και ο Πατήρ σας ο εν τοις ουρανοίς τα αμαρτήματά σας.
\par 26 Αλλ' εάν σεις δεν συγχωρήτε, ουδέ ο Πατήρ σας ο εν τοις ουρανοίς θέλει συγχωρήσει τα αμαρτήματά σας.
\par 27 Και έρχονται πάλιν εις Ιεροσόλυμα· και ενώ περιεπάτει εν τω ιερώ, έρχονται προς αυτόν οι αρχιερείς και οι γραμματείς και οι πρεσβύτεροι
\par 28 και λέγουσι προς αυτόν· Εν ποία εξουσία πράττεις ταύτα; και τις σοι έδωκε την εξουσίαν ταύτην, διά να πράττης ταύτα;
\par 29 Ο δε Ιησούς αποκριθείς είπε προς αυτούς· Θέλω σας ερωτήσει και εγώ ένα λόγον, και αποκρίθητέ μοι, και θέλω σας ειπεί εν ποία εξουσία πράττω ταύτα.
\par 30 Το βάπτισμα του Ιωάννου εξ ουρανού ήτο ή εξ ανθρώπων; αποκρίθητέ μοι.
\par 31 Και διελογίζοντο καθ' εαυτούς, λέγοντες· Εάν είπωμεν, Εξ ουρανού, θέλει ειπεί· Διά τι λοιπόν δεν επιστεύσατε εις αυτόν;
\par 32 Αλλ' εάν είπωμεν, Εξ ανθρώπων; εφοβούντο τον λαόν· διότι πάντες είχον τον Ιωάννην ότι ήτο τωόντι προφήτης.
\par 33 Και αποκριθέντες λέγουσι προς τον Ιησούν· Δεν εξεύρομεν. Και ο Ιησούς αποκριθείς λέγει προς αυτούς· Ουδέ εγώ λέγω προς υμάς εν ποία εξουσία πράττω ταύτα.

\chapter{12}

\par Και ήρχισε να λέγη προς αυτούς διά παραβολών· Άνθρωπος τις εφύτευσεν αμπελώνα και περιέβαλεν εις αυτόν φραγμόν και έσκαψεν υπολήνιον και ωκοδόμησε πύργον, και εμίσθωσεν αυτόν εις γεωργούς και απεδήμησε.
\par 2 Και εν τω καιρώ των καρπών απέστειλε προς τους γεωργούς δούλον, διά να λάβη παρά των γεωργών από του καρπού του αμπελώνος.
\par 3 Εκείνοι δε πιάσαντες αυτόν έδειραν και απέπεμψαν κενόν.
\par 4 Και πάλιν απέστειλε προς αυτούς άλλον δούλον· και εκείνον λιθοβολήσαντες, επλήγωσαν την κεφαλήν αυτού και απέπεμψαν ητιμωμένον.
\par 5 Και πάλιν απέστειλεν άλλον· και εκείνον εφόνευσαν, και πολλούς άλλους, τους μεν έδειραν, τους δε εφόνευσαν.
\par 6 Έτι λοιπόν έχων ένα υιόν, αγαπητόν αυτού, απέστειλε και αυτόν προς αυτούς έσχατον, λέγων ότι θέλουσιν εντραπή τον υιόν μου.
\par 7 Εκείνοι δε οι γεωργοί είπον προς αλλήλους ότι ούτος είναι ο κληρονόμος· έλθετε, ας φονεύσωμεν αυτόν, και θέλει είσθαι ημών η κληρονομία.
\par 8 Και πιάσαντες αυτόν εφόνευσαν και έρριψαν έξω του αμπελώνος.
\par 9 Τι λοιπόν θέλει κάμει ο κύριος του αμπελώνος; Θέλει ελθεί και απολέσει τους γεωργούς και θέλει δώσει τον αμπελώνα εις άλλους.
\par 10 Ουδέ την γραφήν ταύτην δεν ανεγνώσατε, Ο λίθος, τον οποίον απεδοκίμασαν οι οικοδομούντες, ούτος έγεινε κεφαλή γωνίας·
\par 11 παρά Κυρίου έγεινεν αύτη και είναι θαυμαστή εν οφθαλμοίς ημών.
\par 12 Και εζήτουν να πιάσωσιν αυτόν και εφοβήθησαν τον όχλον· επειδή ενόησαν ότι προς αυτούς είπε την παραβολήν· και αφήσαντες αυτόν ανεχώρησαν.
\par 13 Και αποστέλλουσι προς αυτόν τινάς των Φαρισαίων και των Ηρωδιανών, διά να παγιδεύσωσιν αυτόν εις λόγον.
\par 14 Και εκείνοι ελθόντες, λέγουσι προς αυτόν· Διδάσκαλε, εξεύρομεν ότι είσαι αληθής και δεν σε μέλει περί ουδενός· διότι δεν βλέπεις εις πρόσωπον ανθρώπων, αλλ' επ' αληθείας την οδόν του Θεού διδάσκεις. Είναι συγκεχωρημένον να δώσωμεν δασμόν εις τον Καίσαρα, ή ουχί; να δώσωμεν ή να μη δώσωμεν;
\par 15 Ο δε γνωρίσας την υπόκρισιν αυτών, είπε προς αυτούς· τι με πειράζετε; φέρετέ μοι δηνάριον διά να ίδω.
\par 16 Και εκείνοι έφεραν. Και λέγει προς αυτούς· Τίνος είναι η εικών αύτη και η επιγραφή; οι δε είπον προς αυτόν· Του Καίσαρος.
\par 17 Και αποκριθείς ο Ιησούς είπε προς αυτούς· Απόδοτε τα του Καίσαρος εις τον Καίσαρα και τα του Θεού εις τον Θεόν. Και εθαύμασαν δι' αυτόν.
\par 18 Και έρχονται προς αυτόν Σαδδουκαίοι, οίτινες λέγουσιν ότι δεν είναι ανάστασις, και ηρώτησαν αυτόν, λέγοντες·
\par 19 Διδάσκαλε, ο Μωϋσής μας έγραψεν ότι εάν αποθάνη τινός ο αδελφός και αφήση γυναίκα και τέκνα δεν αφήση, να λάβη ο αδελφός αυτού την γυναίκα αυτού και να εξαναστήση σπέρμα εις τον αδελφόν αυτού.
\par 20 Ήσαν λοιπόν επτά αδελφοί. Και ο πρώτος έλαβε γυναίκα, και αποθνήσκων δεν αφήκε σπέρμα·
\par 21 και έλαβεν αυτήν ο δεύτερος και απέθανε, και ουδέ αυτός αφήκε σπέρμα· και ο τρίτος ωσαύτως.
\par 22 Και έλαβον αυτήν οι επτά, και δεν αφήκαν σπέρμα. Τελευταία πάντων απέθανε και η γυνή.
\par 23 Εν τη αναστάσει λοιπόν, όταν αναστηθώσι, τίνος αυτών θέλει είσθαι γυνή; διότι και οι επτά έλαβον αυτήν γυναίκα.
\par 24 Και αποκριθείς ο Ιησούς, είπε προς αυτούς· Δεν πλανάσθε διά τούτο, μη γνωρίζοντες τας γραφάς μηδέ την δύναμιν του Θεού;
\par 25 Διότι όταν αναστηθώσιν εκ νεκρών ούτε νυμφεύουσιν ούτε νυμφεύονται, αλλ' είναι ως άγγελοι οι εν τοις ουρανοίς.
\par 26 Περί δε των νεκρών ότι ανίστανται, δεν ανεγνώσατε εν τη βίβλω του Μωϋσέως, πως είπε προς αυτόν ο Θεός επί της βάτου, λέγων, Εγώ είμαι ο Θεός του Αβραάμ και ο Θεός του Ισαάκ και ο Θεός του Ιακώβ;
\par 27 Δεν είναι ο Θεός νεκρών, αλλά Θεός ζώντων· σεις λοιπόν πλανάσθε πολύ.
\par 28 Και προσελθών εις των γραμματέων, όστις ήκουσεν αυτούς συζητούντας, γνωρίζων ότι καλώς απεκρίθη προς αυτούς, ηρώτησεν αυτόν· Ποία εντολή είναι πρώτη πασών;
\par 29 Ο δε Ιησούς απεκρίθη προς αυτόν ότι πρώτη πασών των εντολών είναι· Άκουε Ισραήλ, Κύριος ο Θεός ημών είναι εις Κύριος·
\par 30 και θέλεις αγαπά Κύριον τον Θεόν σου εξ όλης της καρδίας σου, και εξ όλης της ψυχής σου, και εξ όλης της διανοίας σου, και εξ όλης της δυνάμεώς σου· αύτη είναι η πρώτη εντολή.
\par 31 Και δευτέρα ομοία, αύτη· Θέλεις αγαπά τον πλησίον σου ως σεαυτόν. Μεγαλητέρα τούτων άλλη εντολή δεν είναι.
\par 32 Και είπε προς αυτόν ο γραμματεύς· Καλώς, Διδάσκαλε, αληθώς είπας ότι είναι εις Θεός, και δεν είναι άλλος εκτός αυτού·
\par 33 και το να αγαπά τις αυτόν εξ όλης της καρδίας και εξ όλης της συνέσεως και εξ όλης της ψυχής και εξ όλης της δυνάμεως, και το να αγαπά τον πλησίον ως εαυτόν, είναι πλειότερον πάντων των ολοκαυτωμάτων και των θυσιών.
\par 34 Και ο Ιησούς, ιδών αυτόν ότι φρονίμως απεκρίθη, είπε προς αυτόν· Δεν είσαι μακράν από της βασιλείας του Θεού. Και ουδείς πλέον ετόλμα να ερωτήση αυτόν.
\par 35 Και αποκριθείς ο Ιησούς, έλεγε διδάσκων εν τω ιερώ· Πως λέγουσιν οι γραμματείς ότι ο Χριστός είναι υιός του Δαβίδ;
\par 36 Διότι αυτός ο Δαβίδ είπε διά του Πνεύματος του Αγίου· Είπεν ο Κύριος προς τον Κύριόν μου, Κάθου εκ δεξιών μου, εωσού θέσω τους εχθρούς σου υποπόδιον των ποδών σου.
\par 37 Αυτός λοιπόν ο Δαβίδ λέγει αυτόν Κύριον· και πόθεν είναι υιός αυτού; Και ο πολύς όχλος ήκουεν αυτόν ευχαρίστως.
\par 38 Και έλεγε προς αυτούς εν τη διδαχή αυτού· Προσέχετε από των γραμματέων, οίτινες θέλουσι να περιπατώσιν εστολισμένοι και αγαπώσι τους ασπασμούς εν ταις αγοραίς
\par 39 και πρωτοκαθεδρίας εν ταις συναγωγαίς και τους πρώτους τόπους εν τοις δείπνοις.
\par 40 Οίτινες κατατρώγουσι τας οικίας των χηρών, και τούτο επί προφάσει ότι κάμνουσι μακράς προσευχάς· ούτοι θέλουσι λάβει μεγαλητέραν καταδίκην.
\par 41 Και καθήσας ο Ιησούς απέναντι του γαζοφυλακίου, εθεώρει πως ο όχλος έβαλλε χαλκόν εις το γαζοφυλάκιον. Και πολλοί πλούσιοι έβαλλον πολλά·
\par 42 και ελθούσα μία χήρα πτωχή έβαλε δύο λεπτά, τουτέστιν, ένα κοδράντην.
\par 43 Και προσκαλέσας τους μαθητάς αυτού, λέγει προς αυτούς· Αληθώς σας λέγω ότι η χήρα αύτη η πτωχή έβαλε περισσότερον πάντων, όσοι έβαλον εις το γαζοφυλάκιον·
\par 44 διότι πάντες εκ του περισσεύοντος εις αυτούς έβαλον· αύτη όμως εκ του υστερήματος αυτής έβαλε πάντα όσα είχεν, όλην την περιουσίαν αυτής.

\chapter{13}

\par Και ενώ εξήρχετο εκ του ιερού, λέγει προς αυτόν εις των μαθητών αυτού· Διδάσκαλε, ίδε οποίοι λίθοι και οποίαι οικοδομαί.
\par 2 Και ο Ιησούς αποκριθείς είπε προς αυτόν· Βλέπεις ταύτας τας μεγάλας οικοδομάς; δεν θέλει αφεθή λίθος επί λίθον, όστις να μη κατακρημνισθή.
\par 3 Και ενώ εκάθητο εις το όρος των Ελαιών κατέναντι του ιερού, ηρώτων αυτόν κατ' ιδίαν ο Πέτρος και Ιάκωβος και Ιωάννης και Ανδρέας.
\par 4 Ειπέ προς ημάς πότε θέλουσι γείνει ταύτα, και τι το σημείον όταν ταύτα πάντα μέλλωσι να συντελεσθώσιν;
\par 5 Ο δε Ιησούς αποκριθείς προς αυτούς, ήρχισε να λέγη· Βλέπετε μη σας πλανήση τις.
\par 6 Διότι πολλοί θέλουσιν ελθεί εν τω ονόματί μου, λέγοντες ότι εγώ είμαι, και πολλούς θέλουσι πλανήσει.
\par 7 Όταν δε ακούσητε πολέμους και φήμας πολέμων, μη ταράττεσθε· διότι πρέπει να γείνωσι ταύτα, αλλά δεν είναι έτι το τέλος.
\par 8 Διότι θέλει εγερθή έθνος επί έθνος και βασιλεία επί βασιλείαν, και θέλουσι γείνει σεισμοί κατά τόπους και θέλουσι γείνει πείναι και ταραχαί. Ταύτα είναι αρχαί ωδίνων.
\par 9 Σεις δε προσέχετε εις εαυτούς. Διότι θέλουσι σας παραδώσει εις συνέδρια, και εις συναγωγάς θέλετε δαρθή, και ενώπιον ηγεμόνων και βασιλέων θέλετε σταθή ένεκεν εμού προς μαρτυρίαν εις αυτούς·
\par 10 και πρέπει πρώτον να κηρυχθή το ευαγγέλιον εις πάντα τα έθνη.
\par 11 Όταν δε σας φέρωσι διά να σας παραδώσωσι, μη προμεριμνάτε τι θέλετε λαλήσει, μηδέ μελετάτε, αλλ' ό,τι δοθή εις εσάς εν εκείνη τη ώρα, τούτο λαλείτε· διότι δεν είσθε σεις οι λαλούντες, αλλά το Πνεύμα το Άγιον.
\par 12 Θέλει δε παραδώσει αδελφός αδελφόν εις θάνατον και πατήρ τέκνον, και θέλουσιν επαναστή τέκνα επί γονείς και θέλουσι θανατώσει αυτούς.
\par 13 Και θέλετε είσθαι μισούμενοι υπό πάντων διά το όνομά μου· ο δε υπομείνας έως τέλους, ούτος θέλει σωθή.
\par 14 Όταν δε ίδητε το βδέλυγμα της ερημώσεως, το λαληθέν υπό Δανιήλ του προφήτου, ιστάμενον όπου δεν πρέπει -ο αναγινώσκων ας εννοή- τότε οι εν τη Ιουδαία ας φεύγωσιν εις τα όρη·
\par 15 και ο επί του δώματος ας μη καταβή εις την οικίαν, μηδ' ας εισέλθη διά να λάβη τι εκ της οικίας αυτού,
\par 16 και όστις είναι εις τον αγρόν, ας μη επιστρέψη εις τα οπίσω διά να λάβη το ιμάτιον αυτού.
\par 17 Ουαί δε εις τας εγκυμονούσας και τας θηλαζούσας εν εκείναις ταις ημέραις.
\par 18 Προσεύχεσθε δε διά να μη γείνη η φυγή υμών εν χειμώνι.
\par 19 Διότι αι ημέραι εκείναι θέλουσιν είσθαι θλίψις τοιαύτη, οποία δεν έγεινεν απ' αρχής της κτίσεως, την οποίαν έκτισεν ο Θεός έως του νυν, ουδέ θέλει γείνει.
\par 20 Και εάν ο Κύριος δεν ήθελε συντέμει τας ημέρας εκείνας, δεν ήθελε σωθή ουδεμία σάρξ· αλλά διά τους εκλεκτούς, τους οποίους εξέλεξε, συνέτεμε τας ημέρας.
\par 21 Και τότε εάν τις είπη προς υμάς, Ιδού, εδώ είναι ο Χριστός, ή, Ιδού, εκεί, μη πιστεύσητε.
\par 22 Διότι θέλουσιν εγερθή ψευδόχριστοι και ψευδοπροφήται και θέλουσι δείξει σημεία και τέρατα, διά να αποπλανώσιν, ει δυνατόν, και τους εκλεκτούς.
\par 23 Σεις όμως προσέχετε· ιδού, σας προείπον πάντα.
\par 24 Αλλ' εν εκείναις ταις ημέραις, μετά την θλίψιν εκείνην, ο ήλιος θέλει σκοτισθή και η σελήνη δεν θέλει δώσει το φέγγος αυτής
\par 25 και οι αστέρες του ουρανού θέλουσι πίπτει και αι δυνάμεις αι εν τοις ουρανοίς θέλουσι σαλευθή.
\par 26 Και τότε θέλουσιν ιδεί τον Υιόν του ανθρώπου ερχόμενον εν νεφέλαις μετά δυνάμεως πολλής και δόξης.
\par 27 Και τότε θέλει αποστείλει τους αγγέλους αυτού και συνάξει τους εκλεκτούς αυτού εκ των τεσσάρων ανέμων, απ' άκρου της γης έως άκρου του ουρανού.
\par 28 Από δε της συκής μάθετε την παραβολήν. Όταν ο κλάδος αυτής γείνη ήδη απαλός και εκβλαστάνη τα φύλλα, εξεύρετε ότι πλησίον είναι το θέρος·
\par 29 ούτω και σεις, όταν ίδητε ταύτα γινόμενα, εξεύρετε ότι πλησίον είναι επί τας θύρας.
\par 30 Αληθώς σας λέγω ότι δεν θέλει παρέλθει η γενεά αύτη, εωσού γείνωσι πάντα ταύτα.
\par 31 Ο ουρανός και η γη θέλουσι παρέλθει, οι δε λόγοι μου δεν θέλουσι παρέλθει.
\par 32 Περί δε της ημέρας εκείνης και της ώρας ουδείς γινώσκει, ουδέ οι άγγελοι οι εν ουρανώ, ουδέ ο Υιός, ειμή ο Πατήρ.
\par 33 Προσέχετε, αγρυπνείτε και προσεύχεσθε· διότι δεν εξεύρετε πότε είναι ο καιρός.
\par 34 Επειδή τούτο θέλει είσθαι ως άνθρωπος αποδημών, όστις αφήκε την οικίαν αυτού και έδωκεν εις τους δούλους αυτού την εξουσίαν και εις έκαστον το έργον αυτού και εις τον θυρωρόν προσέταξε να αγρυπνή.
\par 35 Αγρυπνείτε λοιπόν· διότι δεν εξεύρετε πότε έρχεται ο κύριος της οικίας, την εσπέραν ή το μεσονύκτιον ή όταν φωνάζη ο αλέκτωρ ή το πρωΐ·
\par 36 μήποτε ελθών εξαίφνης, σας εύρη κοιμωμένους.
\par 37 Και όσα λέγω προς εσάς προς πάντας λέγω· Αγρυπνείτε.

\chapter{14}

\par Μετά δε δύο ημέρας ήτο το πάσχα και τα άζυμα. Και εζήτουν οι αρχιερείς και οι γραμματείς πως να συλλάβωσιν αυτόν με δόλον και να θανατώσωσιν.
\par 2 Έλεγον δε, Μη εν τη εορτή, μήποτε γείνη θόρυβος του λαού.
\par 3 Και ενώ αυτός ήτο εν Βηθανία εν τη οικία Σίμωνος του λεπρού, και εκάθητο εις την τράπεζαν, ήλθε γυνή έχουσα αλάβαστρον μύρου νάρδου καθαράς πολυτίμου, και συντρίψασα το αλάβαστρον, έχυσε το μύρον επί της κεφαλής αυτού.
\par 4 Ήσαν δε τινές αγανακτούντες καθ' εαυτούς και λέγοντες· Διά τι έγεινεν η απώλεια αύτη του μύρου;
\par 5 διότι ηδύνατο τούτο να πωληθή υπέρ τριακόσια δηνάρια και να δοθώσιν εις τους πτωχούς· και ωργίζοντο κατ' αυτής.
\par 6 Αλλ' ο Ιησούς είπεν· Αφήσατε αυτήν· διά τι ενοχλείτε αυτήν; καλόν έργον έπραξεν εις εμέ.
\par 7 Διότι τους πτωχούς πάντοτε έχετε μεθ' εαυτών, και όταν θέλητε, δύνασθε να ευεργετήσητε αυτούς· εμέ όμως πάντοτε δεν έχετε.
\par 8 ό,τι ηδύνατο αύτη έπραξε· προέλαβε να αλείψη με μύρον το σώμα μου διά τον ενταφιασμόν.
\par 9 Αληθώς σας λέγω, Όπου αν κηρυχθή το ευαγγέλιον τούτο εις όλον τον κόσμον, και εκείνο το οποίον έπραξεν αύτη θέλει λαληθή εις μνημόσυνον αυτής.
\par 10 Τότε ο Ιούδας ο Ισκαριώτης, εις των δώδεκα, υπήγε προς τους αρχιερείς, διά να παραδώση αυτόν εις αυτούς.
\par 11 Εκείνοι δε ακούσαντες εχάρησαν και υπεσχέθησαν να δώσωσιν εις αυτόν αργύρια· και εζήτει πως να παραδώση αυτόν εν ευκαιρία.
\par 12 Και τη πρώτη ημέρα των αζύμων, ότε εθυσίαζον το πάσχα, λέγουσι προς αυτόν οι μαθηταί αυτού· Που θέλεις να υπάγωμεν και να ετοιμάσωμεν διά να φάγης το πάσχα;
\par 13 Και αποστέλλει δύο των μαθητών αυτού και λέγει προς αυτούς· Υπάγετε εις την πόλιν, και θέλει σας απαντήσει άνθρωπος βαστάζων σταμνίον ύδατος· ακολουθήσατε αυτόν,
\par 14 και όπου εισέλθη, είπατε προς τον οικοδεσπότην ότι ο Διδάσκαλος λέγει· Που είναι το κατάλυμα, όπου θέλω φάγει το πάσχα μετά των μαθητών μου;
\par 15 Και αυτός θέλει σας δείξει ανώγεον μέγα εστρωμένον έτοιμον· εκεί ετοιμάσατε εις ημάς.
\par 16 Και εξήλθον οι μαθηταί αυτού και ήλθον εις την πόλιν, και εύρον καθώς είπε προς αυτούς, και ητοίμασαν το πάσχα.
\par 17 Και ότε έγεινεν εσπέρα, έρχεται μετά των δώδεκα·
\par 18 και ενώ εκάθηντο εις την τράπεζαν και έτρωγον, είπεν ο Ιησούς· Αληθώς σας λέγω ότι εις εξ υμών θέλει με παραδώσει, όστις τρώγει μετ' εμού.
\par 19 Οι δε ήρχισαν να λυπώνται και να λέγωσι προς αυτόν εις έκαστος· Μήπως εγώ; και άλλος· Μήπως εγώ;
\par 20 Ο δε αποκριθείς είπε προς αυτούς· Εις εκ των δώδεκα, ο εμβάπτων μετ' εμού εις το πινάκιον την χείρα.
\par 21 Ο μεν Υιός του ανθρώπου υπάγει, καθώς είναι γεγραμμένον περί αυτού· ουαί δε εις τον άνθρωπον εκείνον, διά του οποίου ο Υιός του ανθρώπου παραδίδεται· καλόν ήτο εις τον άνθρωπον εκείνον, αν δεν ήθελε γεννηθή.
\par 22 Και ενώ έτρωγον, λαβών ο Ιησούς άρτον ευλογήσας έκοψε και έδωκεν εις αυτούς και είπε· λάβετε, φάγετε· τούτο είναι το σώμα μου.
\par 23 Και λαβών το ποτήριον, ευχαρίστησε και έδωκεν εις αυτούς, και έπιον εξ αυτού πάντες.
\par 24 Και είπε προς αυτούς· Τούτο είναι το αίμα μου το της καινής διαθήκης, το περί πολλών εκχυνόμενον.
\par 25 Αληθώς σας λέγω ότι δεν θέλω πίει πλέον εκ του γεννήματος της αμπέλου έως της ημέρας εκείνης, όταν πίνω αυτό νέον εν τη βασιλεία του Θεού.
\par 26 Και αφού ύμνησαν, εξήλθον εις το όρος των ελαιών,
\par 27 Και λέγει προς αυτούς ο Ιησούς ότι πάντες θέλετε σκανδαλισθή εν εμοί την νύκτα ταύτην· διότι είναι γεγραμμένον, Θέλω πατάξει τον ποιμένα και θέλουσι διασκορπισθή τα πρόβατα·
\par 28 αφού όμως αναστηθώ, θέλω υπάγει πρότερον υμών εις την Γαλιλαίαν.
\par 29 Ο δε Πέτρος είπε προς αυτόν· Και εάν πάντες σκανδαλισθώσιν, εγώ όμως ουχί.
\par 30 Και λέγει προς αυτόν ο Ιησούς· Αληθώς σοι λέγω ότι σήμερον την νύκτα ταύτην, πριν ο αλέκτωρ φωνάξη δις, τρίς θέλεις με απαρνηθή.
\par 31 Ο δε έτι μάλλον έλεγεν· Εάν γείνη χρεία να συναποθάνω μετά σου, δεν θέλω σε απαρνηθή. Ωσαύτως δε και πάντες έλεγον.
\par 32 Και έρχονται εις χωρίον ονομαζόμενον Γεθσημανή, και λέγει προς τους μαθητάς αυτού· Καθήσατε εδώ, εωσού προσευχηθώ·
\par 33 και παραλαμβάνει τον Πέτρον και τον Ιάκωβον και Ιωάννην μεθ' εαυτού, και ήρχισε να εκθαμβήται και να αδημονή.
\par 34 Και λέγει προς αυτούς· Περίλυπος είναι η ψυχή μου έως θανάτου· μείνατε εδώ και αγρυπνείτε.
\par 35 Και προχωρήσας ολίγον, έπεσεν επί της γης και προσηύχετο να παρέλθη αν ήναι δυνατόν απ' αυτού η ώρα εκείνη,
\par 36 και έλεγεν· Αββά ο Πατήρ, πάντα είναι δυνατά εις σέ· απομάκρυνον απ' εμού το ποτήριον τούτο. Ουχί όμως ό,τι θέλω εγώ, αλλ' ό,τι συ.
\par 37 Και έρχεται και ευρίσκει αυτούς κοιμωμένους και λέγει προς τον Πέτρον· Σίμων, κοιμάσαι; δεν ηδυνήθης μίαν ώραν να αγρυπνήσης;
\par 38 αγρυπνείτε και προσεύχεσθε, διά να μη εισέλθητε εις πειρασμόν· το μεν πνεύμα πρόθυμον, η δε σαρξ ασθενής.
\par 39 Και πάλιν υπήγε και προσηυχήθη, ειπών τον αυτόν λόγον.
\par 40 Και επιστρέψας εύρεν αυτούς πάλιν κοιμωμένους· διότι οι οφθαλμοί αυτών ήσαν βεβαρημένοι και δεν ήξευρον τι να αποκριθώσι προς αυτόν.
\par 41 Και έρχεται την τρίτην φοράν και λέγει προς αυτούς· Κοιμάσθε το λοιπόν και αναπαύεσθε. Αρκεί· ήλθεν η ώρα· ιδού, παραδίδεται ο Υιός του άνθρωπου εις τας χείρας των αμαρτωλών.
\par 42 Εγέρθητε, υπάγωμεν· ιδού, ο παραδίδων με επλησίασε.
\par 43 Και ευθύς, ενώ ελάλει έτι, έρχεται ο Ιούδας, εις εκ των δώδεκα, και μετ' αυτού όχλος πολύς μετά μαχαιρών και ξύλων, παρά των αρχιερέων και των γραμματέων και των πρεσβυτέρων.
\par 44 Ο δε παραδίδων αυτόν είχε δώσει εις αυτούς σημείον, λέγων· Όντινα φιλήσω, αυτός είναι· πιάσατε αυτόν και φέρετε ασφαλώς.
\par 45 Και ότε ήλθεν, ευθύς πλησιάσας εις αυτόν λέγει· Ραββί, Ραββί, και κατεφίλησεν αυτόν.
\par 46 Και εκείνοι επέβαλον επ' αυτόν τας χείρας αυτών και επίασαν αυτόν.
\par 47 Εις δε τις των παρεστώτων σύρας την μάχαιραν, εκτύπησε τον δούλον του αρχιερέως και απέκοψε το ωτίον αυτού.
\par 48 Και αποκριθείς ο Ιησούς είπε προς αυτούς· Ως επί ληστήν εξήλθετε μετά μαχαιρών και ξύλων να με συλλάβητε;
\par 49 καθ' ημέραν ήμην πλησίον υμών εν τω ιερώ διδάσκων, και δεν με επιάσατε, πλην τούτο έγεινε διά να πληρωθώσιν αι γραφαί.
\par 50 Και αφήσαντες αυτόν πάντες έφυγον.
\par 51 Και εις τις νεανίσκος ηκολούθει αυτόν, περιτετυλιγμένος σινδόνα εις το γυμνόν σώμα αυτού· και πιάνουσιν αυτόν οι νεανίσκοι.
\par 52 Ο δε αφήσας την σινδόνα, έφυγεν απ' αυτών γυμνός.
\par 53 Και έφεραν τον Ιησούν προς τον αρχιερέα· και συνέρχονται προς αυτόν πάντες οι αρχιερείς και οι πρεσβύτεροι και οι γραμματείς.
\par 54 Και ο Πέτρος από μακρόθεν ηκολούθησεν αυτόν έως ένδον της αυλής του αρχιερέως, και συνεκάθητο μετά των υπηρετών και εθερμαίνετο εις το πυρ.
\par 55 Οι δε αρχιερείς και όλον το συνέδριον εζήτουν κατά του Ιησού μαρτυρίαν, διά να θανατώσωσιν αυτόν, και δεν εύρισκον.
\par 56 Διότι πολλοί εψευδομαρτύρουν κατ' αυτού, αλλ' αι μαρτυρίαι δεν ήσαν σύμφωνοι.
\par 57 Και τινές σηκωθέντες εψευδομαρτύρουν κατ' αυτού, λέγοντες
\par 58 ότι Ημείς ηκούσαμεν αυτόν λέγοντα, ότι Εγώ θέλω χαλάσει τον ναόν τούτον τον χειροποίητον και διά τριών ημερών άλλον αχειροποίητον θέλω οικοδομήσει.
\par 59 Πλην ουδέ ούτως ήτο σύμφωνος μαρτυρία αυτών.
\par 60 Και σηκωθείς ο αρχιερεύς εις το μέσον, ηρώτησε τον Ιησούν, λέγων· Δεν αποκρίνεσαι ουδέν; τι μαρτυρούσιν ούτοι κατά σου;
\par 61 Ο δε εσιώπα και δεν απεκρίθη ουδέν. Πάλιν ο αρχιερεύς ηρώτα αυτόν, λέγων προς αυτόν· Συ είσαι ο Χριστός ο Υιός του Ευλογητού;
\par 62 Ο δε Ιησούς είπεν· Εγώ είμαι· και θέλετε ιδεί τον Υιόν του ανθρώπου καθήμενον εκ δεξιών της δυνάμεως και ερχόμενον μετά των νεφελών του ουρανού.
\par 63 Τότε ο αρχιερεύς, διασχίσας τα ιμάτια αυτού, λέγει· Τι χρείαν έχομεν πλέον μαρτύρων;
\par 64 ηκούσατε την βλασφημίαν· τι σας φαίνεται; Οι δε πάντες κατέκριναν αυτόν ότι είναι ένοχος θανάτου.
\par 65 Και ήρχισάν τινές να εμπτύωσιν εις αυτόν και να περικαλύπτωσι το πρόσωπον αυτού και να γρονθίζωσιν αυτόν και να λέγωσι προς αυτόν· Προφήτευσον· και οι υπηρέται έτυπτον αυτόν με ραπίσματα.
\par 66 Και ενώ ήτο ο Πέτρος εν τη αυλή κάτω, έρχεται μία των θεραπαινίδων του αρχιερέως,
\par 67 και ότε είδε τον Πέτρον θερμαινόμενον, εμβλέψασα εις αυτόν, λέγει· Και συ έσο μετά του Ναζαρηνού Ιησού.
\par 68 Ο δε ηρνήθη, λέγων· Δεν εξεύρω ουδέ καταλαμβάνω τι συ λέγεις. Και εξήλθεν έξω εις το προαύλιον, και ο αλέκτωρ εφώναξε.
\par 69 Και η θεράπαινα ιδούσα αυτόν πάλιν, ήρχισε να λέγη προς τους παρεστώτας ότι ούτος εξ αυτών είναι.
\par 70 Ο δε πάλιν ηρνείτο. Και μετ' ολίγον πάλιν οι παρεστώτες έλεγον προς τον Πέτρον· Αληθώς εξ αυτών είσαι· διότι Γαλιλαίος είσαι και η λαλιά σου ομοιάζει.
\par 71 Εκείνος δε ήρχισε να αναθεματίζη και να ομνύη ότι δεν εξεύρω τον άνθρωπον τούτον, τον οποίον λέγετε.
\par 72 Και ο αλέκτωρ εφώναξεν εκ δευτέρου. Και ενεθυμήθη ο Πέτρος τον λόγον, τον οποίον είπε προς αυτόν ο Ιησούς, ότι Πριν ο αλέκτωρ φωνάξη δις, θέλεις με αρνηθή τρίς. Και ήρχισε να κλαίη πικρώς.

\chapter{15}

\par Και ευθύς το πρωΐ συνεβουλεύθησαν οι αρχιερείς μετά των πρεσβυτέρων και γραμματέων και όλον το συνέδριον, και δέσαντες τον Ιησούν έφεραν και παρέδωκαν εις τον Πιλάτον.
\par 2 Και ηρώτησεν αυτόν ο Πιλάτος· Συ είσαι ο βασιλεύς των Ιουδαίων; Ο δε αποκριθείς είπε προς αυτόν· Συ λέγεις.
\par 3 Και κατηγόρουν αυτόν οι αρχιερείς πολλά.
\par 4 Ο δε Πιλάτος πάλιν ηρώτησεν αυτόν, λέγων· Δεν αποκρίνεσαι ουδέν; ίδε πόσα σου καταμαρτυρούσιν.
\par 5 Ο δε Ιησούς έτι δεν απεκρίθη ουδέν, ώστε ο Πιλάτος εθαύμαζε.
\par 6 Κατά δε την εορτήν απέλυεν εις αυτούς ένα δέσμιον, όντινα εζήτουν·
\par 7 ήτο δε ο λεγόμενος Βαραββάς δεδεμένος μετά των συνωμοτών, οίτινες εν τη στάσει έπραξαν φόνον.
\par 8 Και αναβοήσας ο όχλος, ήρχισε να ζητή να κάμη καθώς πάντοτε έκαμνεν εις αυτούς.
\par 9 Ο δε Πιλάτος απεκρίθη προς αυτούς, λέγων· Θέλετε να σας απολύσω τον βασιλέα των Ιουδαίων;
\par 10 Επειδή ήξευρεν ότι διά φθόνον παρέδωκαν αυτόν οι αρχιερείς.
\par 11 Οι αρχιερείς όμως διήγειραν τον όχλον να ζητήσωσι να απολύση εις αυτούς μάλλον τον Βαραββάν.
\par 12 Και ο Πιλάτος αποκριθείς πάλιν, είπε προς αυτούς· Τι λοιπόν θέλετε να κάμω τούτον, τον οποίον λέγετε βασιλέα των Ιουδαίων;
\par 13 Οι δε πάλιν έκραξαν· Σταύρωσον αυτόν.
\par 14 Ο δε Πιλάτος έλεγε προς αυτούς· Και τι κακόν έπραξεν; οι δε περισσότερον έκραξαν· Σταύρωσον αυτόν.
\par 15 Ο Πιλάτος λοιπόν, θέλων να κάμη εις τον όχλον το αρεστόν, απέλυσεν εις αυτούς τον Βαραββάν και παρέδωκε τον Ιησούν, αφού εμαστίγωσεν αυτόν, διά να σταυρωθή.
\par 16 Οι δε στρατιώται έφεραν αυτόν ένδον της αυλής, το οποίον είναι το πραιτώριον, και συγκαλούσιν όλον το τάγμα των στρατιωτών·
\par 17 και ενδύουσιν αυτόν πορφύραν και πλέξαντες ακάνθινον στέφανον, βάλλουσι περί την κεφαλήν αυτού,
\par 18 και ήρχισαν να χαιρετώσιν αυτόν, λέγοντες· Χαίρε, βασιλεύ των Ιουδαίων·
\par 19 και έτυπτον την κεφαλήν αυτού με κάλαμον και ενέπτυον εις αυτόν, και γονυπετούντες προσεκύνουν αυτόν.
\par 20 Και αφού ενέπαιξαν αυτόν, εξέδυσαν αυτόν την πορφύραν και ενέδυσαν αυτόν τα ιμάτια αυτού και έφεραν αυτόν έξω, διά να σταυρώσωσιν αυτόν.
\par 21 Και αγγαρεύουσι τινά Σίμωνα Κυρηναίον διαβαίνοντα, ενώ ήρχετο από του αγρού, τον πατέρα του Αλεξάνδρου και Ρούφου, διά να σηκώση τον σταυρόν αυτού.
\par 22 Και φέρουσιν αυτόν εις τον τόπον Γολγοθά, το οποίον μεθερμηνευόμενον είναι, Κρανίου τόπος.
\par 23 Και έδιδον εις αυτόν να πίη οίνον μεμιγμένον με σμύρναν· αλλ' εκείνος δεν έλαβε.
\par 24 Και αφού εσταύρωσαν αυτόν, διεμερίζοντο τα ιμάτια αυτού, βάλλοντες κλήρον επ' αυτά τι έκαστος να λάβη.
\par 25 Ήτο δε ώρα τρίτη και εσταύρωσαν αυτόν.
\par 26 Και η επιγραφή της κατηγορίας αυτού ήτο επιγεγραμμένη, Ο βασιλεύς των Ιουδαίων.
\par 27 Και μετ' αυτού σταυρόνουσι δύο ληστάς, ένα εκ δεξιών και ένα εξ αριστερών αυτού.
\par 28 Και επληρώθη η γραφή η λέγουσα· Και μετά ανόμων ελογίσθη.
\par 29 Και οι διαβαίνοντες εβλασφήμουν αυτόν, κινούντες τας κεφαλάς αυτών και λέγοντες· Ουά, ο χαλών τον ναόν και διά τριών ημερών οικοδομών,
\par 30 σώσον σεαυτόν και κατάβα από του σταυρού.
\par 31 Ομοίως δε και οι αρχιερείς, εμπαίζοντες προς αλλήλους μετά των γραμματέων, έλεγον· Άλλους έσωσεν, εαυτόν δεν δύναται να σώση.
\par 32 Ο Χριστός ο βασιλεύς του Ισραήλ ας καταβή τώρα από του σταυρού, διά να ίδωμεν και πιστεύσωμεν. Και οι συνεσταυρωμένοι μετ' αυτού ωνείδιζον αυτόν.
\par 33 Ότε δε ήλθεν η έκτη ώρα, σκότος έγεινεν εφ' όλην την γην έως ώρας εννάτης·
\par 34 και την ώραν την εννάτην εβόησεν ο Ιησούς μετά φωνής μεγάλης, λέγων· Ελωΐ, Ελωΐ, λαμά σαβαχθανί; το οποίον μεθερμηνευόμενον είναι, Θεέ μου, Θεέ μου, διά τι με εγκατέλιπες;
\par 35 Και τινές των παρεστώτων ακούσαντες, έλεγον· Ιδού, τον Ηλίαν φωνάζει.
\par 36 Δραμών δε εις και γεμίσας σπόγγον από όξους και περιθέσας αυτόν εις κάλαμον, επότιζεν αυτόν, λέγων· Αφήσατε, ας ίδωμεν αν έρχηται ο Ηλίας να καταβιβάση αυτόν.
\par 37 Ο δε Ιησούς, εκβαλών φωνήν μεγάλην, εξέπνευσε.
\par 38 Και το καταπέτασμα του ναού εσχίσθη εις δύο από άνωθεν έως κάτω.
\par 39 Ιδών δε ο εκατόνταρχος ο παριστάμενος απέναντι αυτού ότι ούτω κράξας εξέπνευσεν, είπεν· Αληθώς ο άνθρωπος ούτος ήτο Υιός Θεού.
\par 40 Ήσαν δε και γυναίκες από μακρόθεν θεωρούσαι, μεταξύ των οποίων ήτο και Μαρία η Μαγδαληνή και Μαρία η μήτηρ του Ιακώβου του μικρού και του Ιωσή, και η Σαλώμη,
\par 41 αίτινες και ότε ήτο εν τη Γαλιλαία ηκολούθουν αυτόν και υπηρέτουν αυτόν, και άλλαι πολλαί, αίτινες συνανέβησαν μετ' αυτού εις Ιεροσόλυμα.
\par 42 Και ότε έγεινεν ήδη εσπέρα, διότι ήτο παρασκευή, τουτέστι προσάββατον,
\par 43 ήλθεν Ιωσήφ ο από Αριμαθαίας, έντιμος βουλευτής, όστις και αυτός περιέμενε την βασιλείαν του Θεού, και τολμήσας εισήλθε προς τον Πιλάτον και εζήτησε το σώμα του Ιησού.
\par 44 Ο δε Πιλάτος εθαύμασεν αν ήδη απέθανε· και προσκαλέσας τον εκατόνταρχον, ηρώτησεν αυτόν αν προ πολλού απέθανε·
\par 45 και μαθών παρά του εκατοντάρχου, εχάρισε το σώμα εις τον Ιωσήφ.
\par 46 Και ούτος, αγοράσας σινδόνα και καταβιβάσας αυτόν, ετύλιξε με την σινδόνα και έθεσεν αυτόν εν μνημείω, το οποίον ήτο λελατομημένον εκ πέτρας, και προσεκύλισε λίθον επί την θύραν του μνημείου.
\par 47 Η δε Μαρία η Μαγδαληνή και Μαρία η μήτηρ του Ιωσή έβλεπον που τίθεται.

\chapter{16}

\par Και αφού επέρασε το σάββατον, Μαρία η Μαγδαληνή και Μαρία η μήτηρ του Ιακώβου και η Σαλώμη ηγόρασαν αρώματα, διά να έλθωσι και αλείψωσιν αυτόν.
\par 2 Και πολλά πρωΐ της πρώτης ημέρας της εβδομάδος έρχονται εις το μνημείον, ότε ανέτειλεν ο ήλιος.
\par 3 Και έλεγον προς εαυτάς· Τις θέλει αποκυλίσει εις ημάς τον λίθον εκ της θύρας του μνημείου;
\par 4 Και αναβλέψασαι θεωρούσιν ότι ο λίθος ήτο αποκεκυλισμένος· διότι ήτο μέγας σφόδρα.
\par 5 Και εισελθούσαι εις το μνημείον είδον νεανίσκον καθήμενον εις τα δεξιά, ενδεδυμένον στολήν λευκήν, και ετρόμαξαν.
\par 6 Ο δε λέγει προς αυτάς· Μη τρομάζετε· Ιησούν ζητείτε τον Ναζαρηνόν τον εσταυρωμένον· ανέστη, δεν είναι εδώ· ιδού ο τόπος, όπου έθεσαν αυτόν.
\par 7 Αλλ' υπάγετε, είπατε προς τους μαθητάς αυτού και προς τον Πέτρον ότι υπάγει πρότερον υμών εις την Γαλιλαίαν· εκεί θέλετε ιδεί αυτόν, καθώς είπε προς εσάς.
\par 8 Και εξελθούσαι ταχέως, έφυγον από του μνημείου· είχε δε αυτάς τρόμος και έκστασις, και δεν είπον ουδέν προς ουδένα· διότι εφοβούντο.
\par 9 Αφού δε ανέστη το πρωΐ της πρώτης της εβδομάδος, εφάνη πρώτον εις την Μαρίαν την Μαγδαληνήν, εξ ης είχεν εκβάλει επτά δαιμόνια.
\par 10 Εκείνη υπήγε και απήγγειλε προς εκείνους, οίτινες είχον σταθή μετ' αυτού, ενώ επένθουν και έκλαιον.
\par 11 Και εκείνοι, ακούσαντες ότι ζη και εθεάθη υπ' αυτής, δεν επίστευσαν.
\par 12 Μετά δε ταύτα εφανερώθη εν άλλη μορφή εις δύο εξ αυτών, ενώ περιεπάτουν και επορεύοντο εις τον αγρόν.
\par 13 Και εκείνοι υπήγαν και απήγγειλαν προς τους λοιπούς· αλλ' ουδέ εις εκείνους επίστευσαν.
\par 14 Ύστερον εφάνη εις τους ένδεκα, ενώ εκάθηντο εις την τράπεζαν, και ωνείδισε την απιστίαν αυτών και σκληροκαρδίαν, διότι δεν επίστευσαν εις τους ιδόντας αυτόν αναστάντα.
\par 15 Και είπε προς αυτούς· Υπάγετε εις όλον τον κόσμον και κηρύξατε το ευαγγέλιον εις όλην την κτίσιν.
\par 16 Όστις πιστεύση και βαπτισθή θέλει σωθή, όστις όμως απιστήση θέλει κατακριθή.
\par 17 Σημεία δε εις τους πιστεύσαντας θέλουσι παρακολουθεί ταύτα, Εν τω ονόματί μου θέλουσιν εκβάλλει δαιμόνια· θέλουσι λαλεί νέας γλώσσας·
\par 18 όφεις θέλουσι πιάνει· και εάν θανάσιμόν τι πίωσι, δεν θέλει βλάψει αυτούς· επί αρρώστους θέλουσιν επιθέσει τας χείρας, και θέλουσιν ιατρεύεσθαι.
\par 19 Ο μεν λοιπόν Κύριος, αφού ελάλησεν προς αυτούς, ανελήφθη εις τον ουρανόν και εκάθισεν εκ δεξιών του Θεού.
\par 20 Εκείνοι δε εξελθόντες εκήρυξαν πανταχού, συνεργούντος του Κυρίου και βεβαιούντος το κήρυγμα διά των επακολουθούντων θαυμάτων. Αμήν.


\end{document}