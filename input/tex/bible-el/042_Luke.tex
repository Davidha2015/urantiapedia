\begin{document}

\title{Luke}


\chapter{1}

\par 1 Επειδή πολλοί επεχείρησαν να συντάξωσι διήγησιν περί των μετά πληροφορίας βεβαιωμένων εις ημάς πραγμάτων,
\par 2 καθώς παρέδοσαν εις ημάς οι απ' αρχής γενόμενοι αυτόπται και υπηρέται του λόγου,
\par 3 εφάνη και εις εμέ εύλογον, όστις διηρεύνησα πάντα εξ αρχής ακριβώς, να σοι γράψω κατά σειράν περί τούτων, κράτιστε Θεόφιλε,
\par 4 διά να γνωρίσης την βεβαιότητα των πραγμάτων, περί των οποίων κατηχήθης.
\par 5 Υπήρξεν επί των ημερών Ηρώδου, του βασιλέως της Ιουδαίας, ιερεύς τις το όνομα Ζαχαρίας εκ της εφημερίας Αβιά, και η γυνή αυτού ήτο εκ των θυγατέρων του Ααρών, και το όνομα αυτής Ελισάβετ.
\par 6 Ήσαν δε αμφότεροι δίκαιοι ενώπιον του Θεού, περιπατούντες εν πάσαις ταις εντολαίς και τοις δικαιώμασι του Κυρίου άμεμπτοι.
\par 7 Και δεν είχον τέκνον, καθότι η Ελισάβετ ήτο στείρα, και αμφότεροι ήσαν προβεβηκότες εις την ηλικίαν αυτών.
\par 8 Ενώ δε ιεράτευεν αυτός εν τη τάξει της εφημερίας αυτού ενώπιον του Θεού,
\par 9 κατά το έθος της ιερατείας έπεσεν εις αυτόν ο κλήρος να θυμιάση εισελθών εις τον ναόν του Κυρίου·
\par 10 και παν το πλήθος του λαού προσηύχετο έξω εν τη ώρα του θυμιάματος.
\par 11 Εφάνη δε εις αυτόν άγγελος Κυρίου, ιστάμενος εκ δεξιών του θυσιαστηρίου του θυμιάματος·
\par 12 και ο Ζαχαρίας ιδών εταράχθη, και φόβος επέπεσεν επ' αυτόν.
\par 13 Είπε δε προς αυτόν ο άγγελος· Μη φοβού, Ζαχαρία· διότι εισηκούσθη η δέησίς σου, και η γυνή σου Ελισάβετ θέλει γεννήσει υιόν εις σε, και θέλεις καλέσει το όνομα αυτού Ιωάννην.
\par 14 και θέλει είσθαι εις σε χαρά και αγαλλίασις, και πολλοί θέλουσι χαρή διά την γέννησιν αυτού.
\par 15 Διότι θέλει είσθαι μέγας ενώπιον του Κυρίου, και οίνον και σίκερα δεν θέλει πίει, και θέλει πληρωθή Πνεύματος Αγίου έτι εκ κοιλίας της μητρός αυτού,
\par 16 και πολλούς των υιών Ισραήλ θέλει επιστρέψει εις Κύριον τον Θεόν αυτών.
\par 17 Και αυτός θέλει ελθεί προ προσώπου αυτού εν πνεύματι και δυνάμει Ηλίου, διά να επιστρέψη τας καρδίας των πατέρων εις τα τέκνα και τους απειθείς εις την φρόνησιν των δικαίων, διά να ετοιμάση εις τον Κύριον λαόν προδιατεθειμένον.
\par 18 Και είπεν ο Ζαχαρίας προς τον άγγελον· Πως θέλω γνωρίσει τούτο; διότι εγώ είμαι γέρων, και η γυνή μου προβεβηκυία εις την ηλικίαν αυτής.
\par 19 Και αποκριθείς ο άγγελος, είπε προς αυτόν· Εγώ είμαι Γαβριήλ ο παριστάμενος ενώπιον του Θεού, και απεστάλην διά να λαλήσω προς σε και να σε ευαγγελίσω ταύτα.
\par 20 Και ιδού, θέλεις είσθαι σιωπών και μη δυνάμενος να λαλήσης έως της ημέρας, καθ' ην θέλουσι γείνει ταύτα, διότι δεν επίστευσας εις τους λόγους μου, οίτινες θέλουσιν εκπληρωθή εις τον καιρόν αυτών.
\par 21 Και ο λαός περιέμενε τον Ζαχαρίαν, και εθαύμαζον ότι εβράδυνεν εν τω ναώ.
\par 22 Ότε δε εξήλθε, δεν ηδύνατο να λαλήση προς αυτούς· και ενόησαν ότι οπτασίαν είδεν εν τω ναώ· και αυτός έκαμνεν εις αυτούς νεύματα και διέμενε κωφός.
\par 23 Και αφού ετελείωσαν αι ημέραι της λειτουργίας αυτού, απήλθεν εις τον οίκον αυτού.
\par 24 Μετά δε ταύτας τας ημέρας συνέλαβεν Ελισάβετ η γυνή αυτού, και έκρυπτεν εαυτήν πέντε μήνας, λέγουσα
\par 25 ότι ούτως έκαμεν εις εμέ ο Κύριος εν ταις ημέραις, καθ' ας επέβλεψε να αφαιρέση το όνειδός μου μεταξύ των ανθρώπων.
\par 26 Εν δε τω μηνί τω έκτω απεστάλη ο άγγελος Γαβριήλ υπό του Θεού εις πόλιν της Γαλιλαίας ονομαζομένην Ναζαρέτ,
\par 27 προς παρθένον ηρραβωνισμένην με άνδρα ονομαζόμενον Ιωσήφ, εξ οίκου Δαβίδ, και το όνομα της παρθένου Μαριάμ.
\par 28 Και εισελθών ο άγγελος προς αυτήν, είπε· Χαίρε, κεχαριτωμένη· ο Κύριος μετά σού· ευλογημένη συ εν γυναιξίν.
\par 29 Εκείνη δε ιδούσα διεταράχθη διά τον λόγον αυτού, και διελογίζετο οποίος τάχα ήτο ο ασπασμός ούτος.
\par 30 Και είπεν ο άγγελος προς αυτήν· Μη φοβού, Μαριάμ· διότι εύρες χάριν παρά τω Θεώ.
\par 31 Και ιδού, θέλεις συλλάβει εν γαστρί και θέλεις γεννήσει υιόν και θέλεις καλέσει το όνομα αυτού Ιησούν.
\par 32 Ούτος θέλει είσθαι μέγας και Υιός Υψίστου θέλει ονομασθή, και θέλει δώσει εις αυτόν Κύριος ο Θεός τον θρόνον Δαβίδ του πατρός αυτού,
\par 33 και θέλει βασιλεύσει επί τον οίκον του Ιακώβ εις τους αιώνας, και της βασιλείας αυτού δεν θέλει είσθαι τέλος.
\par 34 Είπε δε η Μαριάμ προς τον άγγελον. Πως θέλει είσθαι τούτο, επειδή άνδρα δεν γνωρίζω;
\par 35 Και αποκριθείς ο άγγελος είπε προς αυτήν· Πνεύμα Άγιον θέλει επέλθει επί σε, και δύναμις του Υψίστου θέλει σε επισκιάσει· διά τούτο και το γεννώμενον εκ σου άγιον θέλει ονομασθή Υιός Θεού.
\par 36 και ιδού, Ελισάβετ η συγγενής σου και αυτή συνέλαβεν υιόν εις το γήρας αυτής, και ούτος είναι μην έκτος εις αυτήν την καλουμένην στείραν·
\par 37 διότι ουδέν πράγμα θέλει είσθαι αδύνατον παρά τω Θεώ.
\par 38 Είπε δε η Μαριάμ· Ιδού, η δούλη του Κυρίου· γένοιτο εις εμέ κατά τον λόγον σου. Και ανεχώρησεν απ' αυτής ο άγγελος.
\par 39 Σηκωθείσα δε η Μαριάμ εν ταις ημέραις ταύταις, υπήγε μετά σπουδής εις την ορεινήν εις πόλιν Ιούδα,
\par 40 και εισήλθεν εις τον οίκον Ζαχαρίου και ησπάσθη την Ελισάβετ.
\par 41 Και ως ήκουσεν η Ελισάβετ τον ασπασμόν της Μαρίας, εσκίρτησε το βρέφος εν τη κοιλία αυτής· και επλήσθη Πνεύματος Αγίου η Ελισάβετ
\par 42 και ανεφώνησε μετά φωνής μεγάλης και είπεν· Ευλογημένη συ εν γυναιξί και ευλογημένος ο καρπός της κοιλίας σου.
\par 43 Και πόθεν μοι τούτο, να έλθη η μήτηρ του Κυρίου μου προς με;
\par 44 Διότι ιδού, καθώς ήλθεν η φωνή του ασπασμού σου εις τα ώτα μου, εσκίρτησεν εν αγαλλιάσει το βρέφος εν τη κοιλία μου.
\par 45 Και μακαρία η πιστεύσασα, διότι θέλει γείνει εκπλήρωσις των λαληθέντων προς αυτήν παρά Κυρίου.
\par 46 Και είπεν η Μαριάμ· Μεγαλύνει η ψυχή μου τον Κύριον
\par 47 και ηγαλλίασε το πνεύμά μου εις τον Θεόν τον Σωτήρά μου,
\par 48 διότι επέβλεψεν επί την ταπείνωσιν της δούλης αυτού. Επειδή ιδού, από του νυν θέλουσι με μακαρίζει πάσαι αι γενεαί·
\par 49 διότι έκαμεν εις εμέ μεγαλεία ο δυνατός και άγιον το όνομα αυτού,
\par 50 και το έλεος αυτού εις γενεάς γενεών επί τους φοβουμένους αυτόν.
\par 51 Ενήργησε κραταιώς διά του βραχίονος αυτού· διεσκόρπισε τους υπερηφάνους κατά τα διανοήματα της καρδίας αυτών.
\par 52 Εκρήμνισε δυνάστας από θρόνων και ύψωσε ταπεινούς,
\par 53 πεινώντας ενέπλησεν από αγαθά και πλουτούντας εξαπέστειλε κενούς.
\par 54 Εβοήθησεν Ισραήλ τον δούλον αυτού, ενθυμηθείς το έλεος αυτού,
\par 55 Καθώς ελάλησε προς τους πατέρας ημών, προς τον Αβραάμ και προς το σπέρμα αυτού εις τον αιώνα.
\par 56 Έμεινε δε η Μαριάμ μετ' αυτής ως τρεις μήνας και υπέστρεψεν εις τον οίκον αυτής.
\par 57 Εις δε την Ελισάβετ συνεπληρώθη ο καιρός του να γεννήση, και εγέννησεν υιόν.
\par 58 Και ήκουσαν οι γείτονες και οι συγγενείς αυτής ότι εμεγάλυνεν ο Κύριος το έλεος αυτού προς αυτήν, και συνέχαιρον αυτήν.
\par 59 Και εν τη ογδόη ημέρα, ήλθον διά να περιτέμωσι το παιδίον, και ωνόμαζον αυτό κατά το όνομα του πατρός αυτού Ζαχαρίαν.
\par 60 Και αποκριθείσα η μήτηρ αυτού, είπεν· Ουχί, αλλ' Ιωάννης θέλει ονομασθή.
\par 61 Και είπον προς αυτήν ότι ουδείς υπάρχει εν τη συγγενεία σου, όστις καλείται με το όνομα τούτο.
\par 62 Ηρώτων δε διά νευμάτων τον πατέρα αυτού τι όνομα ήθελε να δοθή εις αυτό.
\par 63 Και ζητήσας πινακίδιον έγραψε, λέγων· Ιωάννης είναι το όνομα αυτού· και εθαύμασαν πάντες.
\par 64 Ηνοίχθη δε το στόμα αυτού πάραυτα και η γλώσσα αυτού, και ελάλει ευλογών τον Θεόν.
\par 65 Και έπεσε φόβος επί πάντας τους γείτονας αυτών, και καθ' όλην την ορεινήν της Ιουδαίας διελαλούντο πάντα τα πράγματα ταύτα,
\par 66 και πάντες οι ακούσαντες έβαλον αυτά εν τη καρδία αυτών, λέγοντες· Τι άρα θέλει είσθαι το παιδίον τούτο; και χειρ Κυρίου ήτο μετ' αυτού.
\par 67 Και Ζαχαρίας ο πατήρ αυτού επλήσθη Πνεύματος Αγίου και προεφήτευσε, λέγων·
\par 68 Ευλογητός Κύριος ο Θεός του Ισραήλ, διότι επεσκέφθη και έκαμε λύτρωσιν εις τον λαόν αυτού,
\par 69 και ανήγειρεν εις ημάς κέρας σωτηρίας εν τω οίκω Δαβίδ του δούλου αυτού,
\par 70 καθώς ελάλησε διά στόματος των αγίων, των απ' αιώνος προφητών αυτού,
\par 71 σωτηρίαν εκ των εχθρών ημών και εκ της χειρός πάντων των μισούντων ημάς,
\par 72 διά να εκπληρώση το έλεος αυτού προς τους πατέρας ημών και να ενθυμηθή την αγίαν διαθήκην αυτού,
\par 73 τον όρκον, τον οποίον ώμοσε προς Αβραάμ τον πατέρα ημών, ότι θέλει δώσει εις ημάς
\par 74 να ελευθερωθώμεν εκ της χειρός των εχθρών ημών και να λατρεύωμεν αυτόν αφόβως
\par 75 εν οσιότητι και δικαιοσύνη ενώπιον αυτού πάσας τας ημέρας της ζωής ημών.
\par 76 Και συ, παιδίον, προφήτης του Υψίστου θέλεις ονομασθή. Διότι θέλεις προπορευθή προ προσώπου του Κυρίου εις το να ετοιμάσης τας οδούς αυτού,
\par 77 εις το να δώσης γνώσιν σωτηρίας εις τον λαόν αυτού διά της αφέσεως των αμαρτιών αυτών
\par 78 διά σπλάγχνα ελέους του Θεού ημών με τα οποία επεσκέφθη ημάς ανατολή εξ ύψους,
\par 79 διά να φωτίση τους καθημένους εν σκότει και σκιά θανάτου, ώστε να κατευθύνη τους πόδας ημών εις οδόν ειρήνης.
\par 80 Το δε παιδίον ηύξανε και εδυναμούτο κατά το πνεύμα, και ήτο εν ταις ερήμοις έως της ημέρας καθ' ην έμελλε να αναδειχθή προς τον Ισραήλ.

\chapter{2}

\par 1 Εν εκείναις δε ταις ημέραις εξήλθε διάταγμα παρά του Καίσαρος Αυγούστου να απογραφή πάσα η οικουμένη.
\par 2 Αύτη η απογραφή έγεινε πρώτη, ότε ηγεμόνευε της Συρίας ο Κυρήνιος.
\par 3 Και ήρχοντο πάντες να απογράφωνται, έκαστος εις την εαυτού πόλιν.
\par 4 Ανέβη δε και Ιωσήφ από της Γαλιλαίας εκ της πόλεως Ναζαρέτ εις την Ιουδαίαν εις την πόλιν του Δαβίδ, ήτις καλείται Βηθλεέμ, επειδή αυτός ήτο εκ του οίκου και της πατριάς του Δαβίδ,
\par 5 διά να απογραφή μετά της Μαριάμ της ηρραβωνισμένης με αυτόν εις γυναίκα, ήτις ήτο έγκυος.
\par 6 Και ενώ ήσαν εκεί, επληρώθησαν αι ημέραι του να γεννήση·
\par 7 και εγέννησε τον υιόν αυτής τον πρωτότοκον, και εσπαργάνωσεν αυτόν και κατέκλινεν αυτόν εν τη φάτνη, διότι δεν ήτο τόπος δι' αυτούς εν τω καταλύματι.
\par 8 Και ποιμένες ήσαν κατά το αυτό μέρος διανυκτερεύοντες εν τοις αγροίς και φυλάττοντες φυλακάς της νυκτός επί το ποίμνιον αυτών.
\par 9 Και ιδού, άγγελος Κυρίου εξαίφνης εφάνη εις αυτούς, και δόξα Κυρίου έλαμψε περί αυτούς, και εφοβήθησαν φόβον μέγαν.
\par 10 Και είπε προς αυτούς ο άγγελος· Μη φοβείσθε· διότι ιδού, ευαγγελίζομαι εις εσάς χαράν μεγάλην, ήτις θέλει είσθαι εις πάντα τον λαόν,
\par 11 διότι σήμερον εγεννήθη εις εσάς εν πόλει Δαβίδ σωτήρ, όστις είναι Χριστός Κύριος.
\par 12 Και τούτο θέλει είσθαι το σημείον εις εσάς· θέλετε ευρεί βρέφος εσπαργανωμένον, κείμενον εν τη φάτνη.
\par 13 Και εξαίφνης μετά του αγγέλου εφάνη πλήθος στρατιάς ουρανίου υμνούντων τον Θεόν και λεγόντων·
\par 14 Δόξα εν υψίστοις Θεώ και επί γης ειρήνη, εν ανθρώποις ευδοκία.
\par 15 Και καθώς οι άγγελοι ανεχώρησαν απ' αυτών εις τον ουρανόν, οι άνθρωποι οι ποιμένες είπον προς αλλήλους. Ας υπάγωμεν λοιπόν έως Βηθλεέμ και ας ίδωμεν το πράγμα τούτο το γεγονός, το οποίον ο Κύριος εφανέρωσεν εις ημάς.
\par 16 Και ήλθον μετά σπουδής και εύρον την τε Μαριάμ και τον Ιωσήφ και το βρέφος κείμενον εν τη φάτνη.
\par 17 Και ιδόντες, διεκήρυξαν τον λόγον τον λαληθέντα προς αυτούς περί του παιδίου τούτου·
\par 18 και πάντες οι ακούσαντες εθαύμασαν περί των λαληθέντων υπό των ποιμένων προς αυτούς.
\par 19 Η δε Μαριάμ εφύλαττε πάντας τους λόγους τούτους, διαλογιζομένη περί αυτών εν τη καρδία αυτής.
\par 20 Και υπέστρεψαν οι ποιμένες, δοξάζοντες και υμνούντες τον Θεόν διά πάντα όσα ήκουσαν και είδον, καθώς ελαλήθησαν προς αυτούς.
\par 21 Και ότε επληρώθησαν αι οκτώ ημέραι διά να περιτέμωσι το παιδίον, εκλήθη το όνομα αυτού Ιησούς, το ονομασθέν υπό του αγγέλου πριν συλληφθή εν τη κοιλία.
\par 22 Και ότε επληρώθησαν αι ημέραι του καθαρισμού αυτής κατά τον νόμον του Μωϋσέως, ανεβίβασαν αυτόν εις Ιεροσόλυμα διά να παραστήσωσιν εις τον Κύριον,
\par 23 καθώς είναι γεγραμμένον εν τω νόμω του Κυρίου, ότι παν αρσενικόν διανοίγον μήτραν θέλει κληθή άγιον εις τον Κύριον,
\par 24 και διά να προσφέρωσι θυσίαν κατά το ειρημένον εν τω νόμω του Κυρίου, ζεύγος τρυγόνων ή δύο νεοσσούς περιστερών.
\par 25 Και ιδού, ήτο άνθρωπός τις εν Ιερουσαλήμ, ονομαζόμενος Συμεών, και ο άνθρωπος ούτος ήτο δίκαιος και ευλαβής, προσμένων την παρηγορίαν του Ισραήλ, και Πνεύμα Άγιον ήτο επ' αυτόν·
\par 26 και ήτο εις αυτόν αποκεκαλυμμένον υπό του Πνεύματος του Αγίου ότι δεν θέλει ιδεί θάνατον, πριν ίδη τον Χριστόν του Κυρίου.
\par 27 Και ήλθε διά του Πνεύματος εις το ιερόν· και ότε οι γονείς εισέφεραν το παιδίον Ιησούν διά να κάμωσι περί αυτού κατά την συνήθειαν του νόμου,
\par 28 αυτός εδέχθη αυτό εις τας αγκάλας αυτού και ευλόγησε τον Θεόν και είπε·
\par 29 Νυν απολύεις τον δούλον σου, δέσποτα, κατά το ρήμα σου, εν ειρήνη·
\par 30 διότι είδον οι οφθαλμοί μου το σωτήριόν σου,
\par 31 το οποίον ητοίμασας ενώπιον πάντων των λαών,
\par 32 φως εις φωτισμόν των εθνών και δόξαν του λαού σου Ισραήλ.
\par 33 Και ο Ιωσήφ και η μήτηρ αυτού εθαύμαζον διά τα λεγόμενα περί αυτού.
\par 34 Και ευλόγησεν αυτούς ο Συμεών, και είπε προς Μαριάμ την μητέρα αυτού· Ιδού, ούτος κείται εις πτώσιν και ανάστασιν πολλών εν τω Ισραήλ και εις σημείον αντιλεγόμενον.
\par 35 Και σου δε αυτής την ψυχήν ρομφαία θέλει διαπεράσει, διά να ανακαλυφθώσιν οι διαλογισμοί πολλών καρδιών.
\par 36 Και υπήρχέ τις Άννα προφήτις, θυγάτηρ Φανουήλ, εκ της φυλής Ασήρ· αύτη ήτο πολύ προβεβηκυία εις ηλικίαν, ήτις έζησε μετά του ανδρός αυτής επτά έτη από της παρθενίας αυτής,
\par 37 και αύτη ήτο χήρα ως ετών ογδοήκοντα τεσσάρων, ήτις δεν απεμακρύνετο από του ιερού, νύκτα και ημέραν λατρεύουσα τον Θεόν εν νηστείαις και προσευχαίς·
\par 38 και αύτη φθάσασα εν αυτή τη ώρα, εδοξολόγει τον Κύριον και ελάλει περί αυτού προς πάντας τους προσμένοντας λύτρωσιν εν Ιερουσαλήμ.
\par 39 Και αφού ετελείωσαν πάντα τα κατά τον νόμον του Κυρίου, υπέστρεψαν εις την Γαλιλαίαν, εις την πόλιν αυτών Ναζαρέτ.
\par 40 Το δε παιδίον ηύξανε και εδυναμούτο κατά το πνεύμα πληρούμενον σοφίας, και χάρις Θεού ήτο επ' αυτό.
\par 41 Επορεύοντο δε οι γονείς αυτού κατ' έτος εις Ιερουσαλήμ εν τη εορτή του πάσχα.
\par 42 Και ότε έγεινεν ετών δώδεκα, αφού ανέβησαν εις Ιεροσόλυμα κατά το έθος της εορτής
\par 43 και ετελείωσαν τας ημέρας, ενώ αυτοί υπέστρεφον, το παιδίον ο Ιησούς έμεινεν οπίσω εν Ιερουσαλήμ, και δεν ενόησεν ο Ιωσήφ και η μήτηρ αυτού.
\par 44 Νομίσαντες δε ότι αυτός ήτο εν τη συνοδία, ήλθον μιας ημέρας οδόν και ανεζήτουν αυτόν μεταξύ των συγγενών και των γνωρίμων.
\par 45 Και μη ευρόντες αυτόν, υπέστρεψαν εις Ιερουσαλήμ ζητούντες αυτόν.
\par 46 Και μετά τρεις ημέρας εύρον αυτόν εν τω ιερώ καθήμενον εν μέσω των διδασκάλων και ακούοντα αυτόν και ερωτώντα αυτούς.
\par 47 Εξίσταντο δε πάντες οι ακούοντες αυτόν διά την σύνεσιν και τας αποκρίσεις αυτού.
\par 48 Και ιδόντες αυτόν εξεπλάγησαν, και είπε προς αυτόν η μήτηρ αυτού· Τέκνον, διά τι έπραξας εις ημάς ούτως; ιδού, ο πατήρ σου και εγώ καταλυπούμενοι σε εζητούμεν.
\par 49 Και είπε προς αυτούς· Διά τι με εζητείτε; δεν ηξεύρετε ότι πρέπει να ήμαι εις τα του Πατρός μου;
\par 50 Και αυτοί δεν ενόησαν τον λόγον, τον οποίον ελάλησε προς αυτούς.
\par 51 Και κατέβη μετ' αυτών και ήλθεν εις Ναζαρέτ, και ήτο υποτασσόμενος εις αυτούς. Η δε μήτηρ αυτού εφύλαττε πάντας τους λόγους τούτους εν τη καρδία αυτής.
\par 52 Και ο Ιησούς προέκοπτεν εις σοφίαν και ηλικίαν και χάριν παρά Θεώ και ανθρώποις.

\chapter{3}

\par 1 Εν δε τω δεκάτω πέμπτω έτει της ηγεμονίας Τιβερίου Καίσαρος, ότε ο Πόντιος Πιλάτος ηγεμόνευε της Ιουδαίας, και τετράρχης της Γαλιλαίας ήτο ο Ηρώδης, Φίλιππος δε ο αδελφός αυτού τετράρχης της Ιτουραίας και της Τραχωνίτιδος χώρας, και ο Λυσανίας τετράρχης της Αβιληνής,
\par 2 επί αρχιερέων Άννα και Καϊάφα, έγεινε λόγος Θεού προς Ιωάννην, τον υιόν του Ζαχαρίου, εν τη ερήμω,
\par 3 και ήλθεν εις πάσαν την περίχωρον του Ιορδάνου, κηρύττων βάπτισμα μετανοίας εις άφεσιν αμαρτιών,
\par 4 ως είναι γεγραμμένον εν τω βιβλίω των λόγων Ησαΐου του προφήτου, λέγοντος· Φωνή βοώντος εν τη ερήμω, ετοιμάσατε την οδόν του Κυρίου, ευθείας κάμετε τας τρίβους αυτού.
\par 5 πάσα φάραγξ θέλει γεμισθή και παν όρος και βουνός θέλει ταπεινωθή, και τα σκολιά θέλουσι γείνει ευθέα και αι τραχείαι οδοί ομαλαί,
\par 6 και πάσα σαρξ θέλει ιδεί το σωτήριον του Θεού.
\par 7 Έλεγε δε προς τους όχλους τους εξερχομένους διά να βαπτισθώσιν υπ' αυτού· Γεννήματα εχιδνών, τις έδειξεν εις εσάς να φύγητε από της μελλούσης οργής;
\par 8 Κάμετε λοιπόν καρπούς αξίους της μετανοίας, και μη αρχίσητε να λέγητε καθ' εαυτούς, Πατέρα έχομεν τον Αβραάμ· διότι σας λέγω ότι δύναται ο Θεός εκ των λίθων τούτων να αναστήση τέκνα εις τον Αβραάμ.
\par 9 Ήδη δε και η αξίνη κείται προς την ρίζαν των δένδρων· παν λοιπόν δένδρον μη κάμνον καρπόν καλόν εκκόπτεται και εις πυρ βάλλεται.
\par 10 Και ηρώτων αυτόν οι όχλοι, λέγοντες· Τι λοιπόν θέλομεν κάμει;
\par 11 Αποκριθείς δε λέγει προς αυτούς. Ο έχων δύο χιτώνας ας μεταδώση εις τον μη έχοντα, και ο έχων τροφάς ας κάμη ομοίως.
\par 12 Ήλθον δε και τελώναι διά να βαπτισθώσι, και είπον προς αυτόν· Διδάσκαλε, τι θέλομεν κάμει;
\par 13 Ο δε είπε προς αυτούς· Μη εισπράττετε μηδέν περισσότερον παρά το διατεταγμένον εις εσάς.
\par 14 Ηρώτων δε αυτόν και στρατιωτικοί, λέγοντες· Και ημείς τι θέλομεν κάμει; Και είπε προς αυτούς· Μη βιάσητε μηδένα μηδέ συκοφαντήσητε, και αρκείσθε εις τα σιτηρέσιά σας.
\par 15 Ενώ δε επρόσμενεν ο λαός, και διελογίζοντο πάντες εν ταις καρδίαις αυτών περί του Ιωάννου, μήποτε αυτός είναι ο Χριστός,
\par 16 απεκρίθη ο Ιωάννης προς πάντας, λέγων· Εγώ μεν σας βαπτίζω εν ύδατι· έρχεται όμως ο ισχυρότερός μου, του οποίου δεν είμαι άξιος να λύσω το λωρίον των υποδημάτων αυτού· αυτός θέλει σας βαπτίσει εν Πνεύματι Αγίω και πυρί.
\par 17 Του οποίου το πτυάριον είναι εν τη χειρί αυτού και θέλει διακαθαρίσει το αλώνιον αυτού, και θέλει συνάξει τον σίτον εις την αποθήκην αυτού, το δε άχυρον θέλει κατακαύσει εν πυρί ασβέστω.
\par 18 Και άλλα πολλά προτρέπων ευηγγελίζετο τον λαόν.
\par 19 Ο δε Ηρώδης ο τετράρχης, ελεγχόμενος υπ' αυτού περί της Ηρωδιάδος, της γυναικός Φιλίππου του αδελφού αυτού, και περί πάντων των κακών όσα έπραξεν ο Ηρώδης,
\par 20 προσέθεσε και τούτο επί πάσι και κατέκλεισε τον Ιωάννην εν τη φυλακή.
\par 21 Αφού δε εβαπτίσθη πας ο λαός, βαπτισθέντος και του Ιησού και προσευχομένου, ηνοίχθη ο ουρανός
\par 22 και κατέβη το Πνεύμα το Άγιον εν σωματική μορφή ως περιστερά επ' αυτόν, και έγεινε φωνή εκ του ουρανού, λέγουσα· Συ είσαι ο Υιός μου ο αγαπητός, εις σε ευηρεστήθην.
\par 23 Και αυτός ο Ιησούς ήρχιζε να ήναι ως τριάκοντα ετών, ων καθώς ενομίζετο, υιός Ιωσήφ, του Ηλί,
\par 24 του Ματθάτ, του Λευΐ, του Μελχί, του Ιαννά, του Ιωσήφ,
\par 25 του Ματταθίου, του Αμώς, του Ναούμ, του Εσλί, του Ναγγαί,
\par 26 του Μαάθ, του Ματταθίου, του Σεμεΐ, του Ιωσήφ, του Ιούδα,
\par 27 του Ιωαννά, του Ρησά, του Ζοροβάβελ, του Σαλαθιήλ, του Νηρί,
\par 28 του Μελχί, του Αδδί, του Κωσάμ, του Ελμωδάμ, του Ηρ,
\par 29 του Ιωσή, του Ελιέζερ, του Ιωρείμ, του Ματθάτ, του Λευΐ,
\par 30 του Συμεών, του Ιούδα, του Ιωσήφ, του Ιωνάν, του Ελιακείμ,
\par 31 του Μελεά, του Μαϊνάν, του Ματταθά, του Ναθάν, του Δαβίδ,
\par 32 του Ιεσσαί, του Ωβήδ, του Βοόζ, του Σαλμών, του Ναασσών,
\par 33 του Αμιναδάβ, του Αράμ, του Εσρώμ, του Φαρές, του Ιούδα,
\par 34 του Ιακώβ, του Ισαάκ, του Αβραάμ, του Θάρα, του Ναχώρ,
\par 35 του Σερούχ, του Ραγαύ, του Φαλέκ, του Έβερ, του Σαλά,
\par 36 του Καϊνάν, του Αρφαξάδ, του Σημ, του Νώε, του Λάμεχ,
\par 37 του Μαθουσάλα, του Ενώχ, του Ιαρέδ, του Μαλελεήλ, του Καϊνάν,
\par 38 του Ενώς, του Σηθ, του Αδάμ, του Θεού.

\chapter{4}

\par 1 Ο δε Ιησούς, πλήρης Πνεύματος Αγίου, υπέστρεψεν από τον Ιορδάνην και εφέρετο διά του Πνεύματος εις την έρημον,
\par 2 πειραζόμενος υπό του διαβόλου ημέρας τεσσαράκοντα, και δεν έφαγεν ουδέν τας ημέρας εκείνας· αφού δε αύται ετελείωσαν, ύστερον επείνασε.
\par 3 Και είπε προς αυτόν ο διάβολος· Εάν είσαι Υιός του Θεού, ειπέ προς τον λίθον τούτον να γείνη άρτος.
\par 4 Και απεκρίθη ο Ιησούς προς αυτόν, λέγων· είναι γεγραμμένον ότι με άρτον μόνον δεν θέλει ζήσει ο άνθρωπος, αλλά με πάντα λόγον Θεού.
\par 5 Και αναβιβάσας αυτόν ο διάβολος εις όρος υψηλόν, έδειξεν εις αυτόν πάντα τα βασίλεια της οικουμένης εν μιά στιγμή χρόνου,
\par 6 και είπε προς αυτόν ο διάβολος· εις σε θέλω δώσει άπασαν την εξουσίαν ταύτην και την δόξαν αυτών, διότι εις εμέ είναι παραδεδομένη, και εις όντινα θέλω δίδω αυτήν.
\par 7 Συ λοιπόν εάν προσκυνήσης ενώπιόν μου, σου θέλουσιν είσθαι πάντα.
\par 8 Και αποκριθείς προς αυτόν, είπεν ο Ιησούς· Ύπαγε οπίσω μου, Σατανά· διότι είναι γεγραμμένον, θέλεις προσκυνήσει Κύριον τον Θεόν σου και αυτόν μόνον θέλεις λατρεύσει.
\par 9 Και έφερεν αυτόν εις Ιερουσαλήμ και έστησεν αυτόν επί το πτερύγιον του ιερού και είπε προς αυτόν· Εάν είσαι ο Υιός του Θεού, ρίψον σεαυτόν εντεύθεν κάτω·
\par 10 διότι είναι γεγραμμένον ότι εις τους αγγέλους αυτού θέλει προστάξει περί σου, διά να σε διαφυλάξωσι,
\par 11 και ότι θέλουσι σε σηκόνει επί των χειρών αυτών, διά να μη προσκόψης προς λίθον τον πόδα σου.
\par 12 Και αποκριθείς είπε προς αυτόν ο Ιησούς ότι είναι ειρημένον, δεν θέλεις πειράσει Κύριον τον Θεόν σου.
\par 13 Και αφού ετελείωσε πάντα πειρασμόν ο διάβολος, απεμακρύνθη απ' αυτού μέχρι καιρού.
\par 14 Και ο Ιησούς υπέστρεψεν εν τη δυνάμει του Πνεύματος εις την Γαλιλαίαν· και εξήλθε φήμη περί αυτού καθ' όλην την περίχωρον.
\par 15 Και αυτός εδίδασκεν εν ταις συναγωγαίς αυτών, δοξαζόμενος υπό πάντων.
\par 16 Και ήλθεν εις την Ναζαρέτ, όπου ήτο ανατεθραμμένος, και εισήλθε κατά την συνήθειαν αυτού εις την συναγωγήν εν τη ημέρα του σαββάτου και εσηκώθη να αναγνώση.
\par 17 Και εδόθη εις αυτόν το βιβλίον Ησαΐου του προφήτου, και ανοίξας το βιβλίον εύρε τον τόπον, όπου ήτο γεγραμμένον·
\par 18 Πνεύμα Κυρίου είναι επ' εμέ, διά τούτο με έχρισε· με απέστειλε διά να ευαγγελίζωμαι προς τους πτωχούς, διά να ιατρεύσω τους συτετριμμένους την καρδίαν, να κηρύξω προς τους αιχμαλώτους ελευθερίαν και προς τους τυφλούς ανάβλεψιν, να αποστείλω τους συντεθλασμένους εν ελευθερία,
\par 19 διά να κηρύξω ευπρόσδεκτον Κυρίου ενιαυτόν.
\par 20 Και κλείσας το βιβλίον, απέδωκεν εις τον υπηρέτην και εκάθησε· πάντων δε οι οφθαλμοί των εν τη συναγωγή ήσαν ατενίζοντες εις αυτόν.
\par 21 Και ήρχισε να λέγη προς αυτούς ότι σήμερον επληρώθη η γραφή αύτη εις τα ώτα υμών.
\par 22 Και πάντες εμαρτύρουν εις αυτόν και εθαύμαζον διά τους λόγους της χάριτος τους εξερχομένους εκ του στόματος αυτού και έλεγον· Δεν είναι ούτος ο υιός του Ιωσήφ;
\par 23 Και είπε προς αυτούς· Βεβαίως θέλετε με ειπεί την παραβολήν ταύτην· Ιατρέ, θεράπευσον σεαυτόν· όσα ηκούσαμεν ότι έγειναν εν τη Καπερναούμ, κάμε και εδώ εν τη πατρίδι σου.
\par 24 Είπε δέ· Αληθώς σας λέγω ότι ουδείς προφήτης είναι δεκτός εν τη πατρίδι αυτού.
\par 25 Και επ' αληθείας σας λέγω, Πολλαί χήραι ήσαν εν τω Ισραήλ επί των ημερών Ηλίου, ότε εκλείσθη ο ουρανός επί έτη τρία και μήνας εξ, καθ' ον καιρόν έγεινε πείνα μεγάλη εφ' όλην την γην,
\par 26 και προς ουδεμίαν αυτών επέμφθη ο Ηλίας, ειμή εις Σαρεπτά της Σιδώνος προς γυναίκα χήραν.
\par 27 Και πολλοί λεπροί ήσαν επί Ελισαίου του προφήτου εν τω Ισραήλ, και ουδείς αυτών εκαθαρίσθη, ειμή Νεεμάν ο Σύρος.
\par 28 Και επλήσθησαν πάντες θυμού εν τη συναγωγή, ακούοντες ταύτα,
\par 29 και σηκωθέντες εξέβαλον αυτόν έξω της πόλεως και έφεραν αυτόν έως της οφρύος του όρους, επί του οποίου η πόλις αυτών ήτο ωκοδομημένη, διά να κατακρημνίσωσιν αυτόν·
\par 30 αυτός όμως περάσας διά μέσου αυτών επορεύετο.
\par 31 Και κατέβη εις Καπερναούμ, πόλιν της Γαλιλαίας, και εδίδασκεν αυτούς εν τοις σάββασι·
\par 32 και εξεπλήττοντο διά την διδαχήν αυτού, διότι ο λόγος αυτού ήτο μετά εξουσίας.
\par 33 Και εν τη συναγωγή ήτο άνθρωπος έχων πνεύμα δαιμονίου ακαθάρτου, και ανέκραξε μετά φωνής μεγάλης,
\par 34 λέγων· Φευ, τι είναι μεταξύ υμών και σου, Ιησού Ναζαρηνέ; ήλθες να απολέσης ημάς; Σε γνωρίζω τις είσαι, ο Άγιος του Θεού.
\par 35 Και επετίμησεν αυτό ο Ιησούς, λέγων· Σιώπα και έξελθε εξ αυτού. Και το δαιμόνιον έρριψεν αυτόν εις το μέσον και εξήλθεν απ' αυτού, χωρίς να βλάψη αυτόν παντελώς.
\par 36 Και εξεπλάγησαν πάντες και συνελάλουν προς αλλήλους, λέγοντες· Τις είναι ο λόγος ούτος, ότι μετά εξουσίας και δυνάμεως προστάζει τα ακάθαρτα πνεύματα, και εξέρχονται;
\par 37 και διεδίδετο φήμη περί αυτού εις πάντα τόπον της περιχώρου.
\par 38 Σηκωθείς δε εκ της συναγωγής, εισήλθεν εις την οικίαν του Σίμωνος. Η πενθερά δε του Σίμωνος εκρατείτο υπό πυρετού μεγάλου, και παρεκάλεσαν αυτόν περί αυτής.
\par 39 Και σταθείς επάνω αυτής επετίμησε τον πυρετόν, και αφήκεν αυτήν και παρευθύς σηκωθείσα υπηρέτει αυτούς.
\par 40 Ενώ δε έδυεν ο ήλιος, πάντες όσοι είχον ασθενούντας υπό διαφόρων νόσων έφεραν αυτούς προς αυτόν· εκείνος δε επιθέσας τας χείρας εις ένα έκαστον αυτών εθεράπευσεν αυτούς.
\par 41 Εξήρχοντο δε και δαιμόνια από πολλών, κράζοντα και λέγοντα ότι Συ είσαι ο Χριστός ο Υιός του Θεού. Και επιτιμών αυτά δεν άφινε να λαλώσιν, επειδή εγνώριζον αυτόν ότι είναι ο Χριστός.
\par 42 Και ότε έγεινεν ημέρα, εξελθών υπήγεν εις έρημον τόπον και οι όχλοι εζήτουν αυτόν, και ήλθον έως αυτού και εκράτουν αυτόν διά να μη αναχωρήση απ' αυτών.
\par 43 Ο δε είπε προς αυτούς ότι Και εις τας άλλας πόλεις πρέπει να ευαγγελίσω την βασιλείαν του Θεού επειδή εις τούτο είμαι απεσταλμένος.
\par 44 Και εκήρυττεν εν ταις συναγωγαίς της Γαλιλαίας.

\chapter{5}

\par 1 Ενώ δε ο όχλος συνέθλιβεν αυτόν διά να ακούη τον λόγον του Θεού, αυτός ίστατο πλησίον της λίμνης Γεννησαρέτ,
\par 2 και είδε δύο πλοία ιστάμενα παρά την λίμνην οι δε αλιείς αποβάντες απ' αυτών εξέπλυναν τα δίκτυα.
\par 3 Εμβάς δε εις εν των πλοίων, το οποίον ήτο του Σίμωνος, παρεκάλεσεν αυτόν να απομακρύνη αυτό ολίγον από της γης, και καθήσας εδίδασκεν εκ του πλοίου τους όχλους.
\par 4 Καθώς δε έπαυσε λαλών, είπε προς τον Σίμωνα· Επανάγαγε το πλοίον εις τα βαθέα και ρίψατε τα δίκτυα υμών διά να οψαρεύσητε.
\par 5 Και αποκριθείς ο Σίμων, είπε προς αυτόν· Διδάσκαλε, δι' όλης της νυκτός κοπιάσαντες δεν επιάσαμεν ουδέν· αλλ' όμως επί τω λόγω σου θέλω ρίψει το δίκτυον.
\par 6 Και αφού έκαμον τούτο, συνέκλεισαν πλήθος πολύ ιχθύων και διεσχίζετο το δίκτυον αυτών.
\par 7 Και έκαμον νεύμα εις τους συντρόφους τους εν τω άλλω πλοίω, διά να έλθωσι να βοηθήσωσιν αυτούς· και ήλθον και εγέμισαν αμφότερα τα πλοία, ώστε εβυθίζοντο.
\par 8 Ιδών δε ο Σίμων Πέτρος, προσέπεσε προς τα γόνατα του Ιησού, λέγων· Έξελθε απ' εμού, διότι είμαι άνθρωπος αμαρτωλός, Κύριε.
\par 9 Επειδή έκπληξις κατέλαβεν αυτόν και πάντας τους μετ' αυτού διά την άγραν των ιχθύων, την οποίαν συνέλαβον,
\par 10 ομοίως δε και τον Ιάκωβον και Ιωάννην, τους υιούς του Ζεβεδαίου, οίτινες ήσαν σύντροφοι του Σίμωνος. Και είπε προς τον Σίμωνα ο Ιησούς· Μη φοβού· από του νυν ανθρώπους θέλεις αγρεύει.
\par 11 Και αφού έφεραν τα πλοία επί την γην, αφήσαντες άπαντα ηκολούθησαν αυτόν.
\par 12 Και ενώ ήτο εν μιά των πόλεων ιδού, άνθρωπος πλήρης λέπρας· και ιδών τον Ιησούν, έπεσε κατά πρόσωπον και παρεκάλεσεν αυτόν, λέγων· Κύριε, εάν θέλης, δύνασαι να με καθαρίσης.
\par 13 Και εκτείνας την χείρα, ήγγισεν αυτόν και είπε· Θέλω, καθαρίσθητι. Και ευθύς η λέπρα έφυγεν απ' αυτού.
\par 14 Και αυτός παρήγγειλεν αυτόν να μη είπη τούτο προς μηδένα, αλλ' ύπαγε, λέγει, και δείξον σεαυτόν εις τον ιερέα και πρόσφερε περί του καθαρισμού σου, καθώς προσέταξεν ο Μωϋσής, διά μαρτυρίαν εις αυτούς.
\par 15 Αλλ' έτι μάλλον διήρχετο η φήμη περί αυτού, και συνηθροίζοντο όχλοι πολλοί, διά να ακούωσι και να θεραπεύωνται υπ' αυτού από των ασθενειών αυτών·
\par 16 αυτός δε απεσύρετο εις τας ερήμους και προσηύχετο.
\par 17 Και εν μιά των ημερών, ενώ αυτός εδίδασκεν, εκάθηντο Φαρισαίοι και νομοδιδάσκαλοι, οίτινες είχον ελθεί εκ πάσης κώμης της Γαλιλαίας και Ιουδαίας και Ιερουσαλήμ· και δύναμις Κυρίου ήτο εις το να ιατρεύη αυτούς.
\par 18 Και ιδού, άνδρες φέροντες επί κλίνης άνθρωπον, όστις ήτο παραλυτικός, και εζήτουν να φέρωσιν αυτόν έσω και να θέσωσιν ενώπιον αυτού·
\par 19 και μη ευρόντες διά ποίας εισόδου να φέρωσιν αυτόν έσω εξ αιτίας του όχλου, ανέβησαν επί το δώμα και διά των κεραμίδων κατεβίβασαν αυτόν μετά του κλινιδίου εις το μέσον έμπροσθεν του Ιησού.
\par 20 Και ιδών την πίστιν αυτών, είπε προς αυτόν· Άνθρωπε, συγκεχωρημέναι είναι εις σε αι αμαρτίαι σου.
\par 21 Και ήρχισαν να διαλογίζωνται οι γραμματείς και οι Φαρισαίοι, λέγοντες· Τις είναι ούτος, όστις λαλεί βλασφημίας; τις δύναται να συγχωρή αμαρτίας ειμή μόνος ο Θεός;
\par 22 Νοήσας δε ο Ιησούς τους διαλογισμούς αυτών, απεκρίθη και είπε προς αυτούς· Τι διαλογίζεσθε εν ταις καρδίαις σας;
\par 23 τι είναι ευκολώτερον, να είπω, Συγκεχωρημέναι είναι εις σε αι αμαρτίαι σου, ή να είπω, Σηκώθητι και περιπάτει;
\par 24 αλλά διά να γνωρίσητε ότι εξουσίαν έχει ο Υιός του ανθρώπου επί της γης να συγχωρή αμαρτίας, είπε προς τον παραλυτικόν· Προς σε λέγω, Σηκώθητι και σήκωσον το κλινίδιόν σου και ύπαγε εις τον οίκόν σου.
\par 25 Και παρευθύς εγερθείς ενώπιον αυτών, εσήκωσε το κλινίδιον εφ' ου κατέκειτο και ανεχώρησεν εις τον οίκον αυτού, δοξάζων τον Θεόν.
\par 26 Και έκστασις κατέλαβεν άπαντας και εδόξαζον τον Θεόν, και επλήσθησαν φόβου, λέγοντες ότι είδομεν παράδοξα σήμερον.
\par 27 Και μετά ταύτα εξήλθε και είδε τελώνην τινά Λευΐν το όνομα, καθήμενον εις το τελώνιον, και είπε προς αυτόν· Ακολούθει μοι.
\par 28 Και αφήσας άπαντα, εσηκώθη και ηκολούθησεν αυτόν.
\par 29 Και έκαμεν εις αυτόν ο Λευΐς υποδοχήν μεγάλην εν τη οικία αυτού, και ήτο πλήθος πολύ τελωνών και άλλων, οίτινες εκάθηντο μετ' αυτών εις την τράπεζαν.
\par 30 Και εγόγγυζον οι γραμματείς αυτών και οι Φαρισαίοι προς τους μαθητάς αυτού, λέγοντες· Διά τι μετά τελωνών και αμαρτωλών τρώγετε και πίνετε;
\par 31 Και αποκριθείς ο Ιησούς, είπε προς αυτούς· Δεν έχουσι χρείαν ιατρού οι υγιαίνοντες, αλλ' οι πάσχοντες.
\par 32 Δεν ήλθον διά να καλέσω δικαίους, αλλά αμαρτωλούς εις μετάνοιαν.
\par 33 Οι δε είπον προς αυτόν· Διά τι οι μαθηταί του Ιωάννου νηστεύουσι συχνά και κάμνουσι δεήσεις, ομοίως και οι των Φαρισαίων, οι δε ιδικοί σου τρώγουσι και πίνουσιν;
\par 34 Ο δε είπε προς αυτούς· Μήπως δύνασθε να κάμητε τους υιούς του νυμφώνος να νηστεύωσιν, ενόσω είναι μετ' αυτών ο νυμφίος;
\par 35 θέλουσιν όμως ελθεί ημέραι, όταν αφαιρεθή απ' αυτών ο νυμφίος· τότε θέλουσι νηστεύει εν εκείναις ταις ημέραις.
\par 36 Έλεγε δε και παραβολήν προς αυτούς, ότι ουδείς βάλλει επίρραμμα ιματίου νέου επί ιμάτιον παλαιόν ει δε μη, και το νέον σχίζει και με το παλαιόν δεν συμφωνεί το επίρραμμα το από του νέου.
\par 37 Και ουδείς βάλλει οίνον νέον εις ασκούς παλαιούς ει δε μη, ο νέος οίνος θέλει σχίσει τους ασκούς, και αυτός θέλει εκχυθή και οι ασκοί θέλουσι φθαρή
\par 38 αλλά πρέπει να βάλληται ο νέος οίνος εις ασκούς νέους, και αμφότερα διατηρούνται.
\par 39 Και ουδείς αφού πίη οίνον παλαιόν, θέλει ευθύς νέον· διότι λέγει· Ο παλαιός είναι καλήτερος.

\chapter{6}

\par 1 Κατά δε το δευτερόπρωτον σάββατον διέβαινεν αυτός διά των σπαρτών και οι μαθηταί αυτού ανέσπων τα στάχυα και έτρωγον, τρίβοντες με τας χείρας.
\par 2 Τινές δε των Φαρισαίων είπον προς αυτούς· Διά τι πράττετε ό,τι δεν συγχωρείται να πράττηται εν τοις σάββασι;
\par 3 Και αποκριθείς προς αυτούς, είπεν ο Ιησούς· Ουδέ τούτο δεν ανεγνώσατε, το οποίον έπραξεν ο Δαβίδ, οπότε επείνασεν αυτός και οι μετ' αυτού όντες;
\par 4 πως εισήλθεν εις τον οίκον του Θεού και έλαβε τους άρτους της προθέσεως και έφαγε και έδωκε και εις τους μετ' αυτού, τους οποίους δεν είναι συγκεχωρημένον να φάγωσιν ειμή μόνοι οι ιερείς;
\par 5 Και έλεγε προς αυτούς ότι ο Υιός του ανθρώπου κύριος είναι και του σαββάτου.
\par 6 Και πάλιν εν άλλω σαββάτω εισήλθεν εις την συναγωγήν και εδίδασκε· και ήτο εκεί άνθρωπος, του οποίου η δεξιά χειρ ήτο ξηρά.
\par 7 Παρετήρουν δε αυτόν οι γραμματείς και οι Φαρισαίοι, αν εν τω σαββάτω θέλη θεραπεύσει, διά να εύρωσι κατηγορίαν κατ' αυτού.
\par 8 Αυτός όμως εγνώριζε τους διαλογισμούς αυτών και είπε προς τον άνθρωπον τον έχοντα ξηράν την χείρα· Σηκώθητι και στήθι εις το μέσον. Και εκείνος σηκωθείς εστάθη.
\par 9 Είπε λοιπόν ο Ιησούς προς αυτούς· Θέλω σας ερωτήσει τι είναι συγκεχωρημένον, να αγαθοποιήση τις εν τοις σάββασιν ή να κακοποιήση; να σώση ψυχήν ή να απολέση;
\par 10 Και περιβλέψας πάντας αυτούς, είπε προς τον άνθρωπον· Έκτεινον την χείρα σου. Ο δε έκαμεν ούτω, και αποκατεστάθη η χειρ αυτού υγιής ως η άλλη.
\par 11 Αυτοί δε επλήσθησαν μανίας και συνωμίλουν προς αλλήλους τι να κάμωσιν εις τον Ιησούν.
\par 12 Εν εκείναις δε ταις ημέραις εξήλθεν εις το όρος να προσευχηθή, και διενυκτέρευεν εν τη προσευχή του Θεού.
\par 13 Και ότε έγεινεν ημέρα, έκραξε τους μαθητάς αυτού και εξέλεξεν εξ αυτών δώδεκα, τους οποίους και ωνόμασεν αποστόλους,
\par 14 τον Σίμωνα, τον οποίον και ωνόμασε Πέτρον, και Ανδρέαν τον αδελφόν αυτού, Ιάκωβον και Ιωάννην, Φίλιππον και Βαρθολομαίον,
\par 15 Ματθαίον και Θωμάν, Ιάκωβον τον του Αλφαίου και Σίμωνα τον καλούμενον Ζηλωτήν,
\par 16 Ιούδαν τον αδελφόν Ιακώβου, και Ιούδαν τον Ισκαριώτην, όστις και έγεινε προδότης,
\par 17 και καταβάς μετ' αυτών εστάθη επί τόπου πεδινού, και παρήσαν όχλος μαθητών αυτού και πλήθος πολύ του λαού από πάσης της Ιουδαίας και Ιερουσαλήμ και της παραλίας Τύρου και Σιδώνος, οίτινες ήλθον διά να ακούσωσιν αυτόν και να ιατρευθώσιν από των νόσων αυτών,
\par 18 και οι ενοχλούμενοι υπό πνευμάτων ακαθάρτων, και εθεραπεύοντο.
\par 19 Και πας ο όχλος εζήτει να εγγίζη αυτόν, διότι δύναμις εξήρχετο παρ' αυτού και ιάτρευε πάντας.
\par 20 Και αυτός σηκώσας τους οφθαλμούς αυτού εις τους μαθητάς αυτού, έλεγε· Μακάριοι σεις οι πτωχοί, διότι υμετέρα είναι η βασιλεία του Θεού.
\par 21 Μακάριοι οι πεινώντες τώρα, διότι θέλετε χορτασθή. Μακάριοι οι κλαίοντες τώρα, διότι θέλετε γελάσει.
\par 22 Μακάριοι είσθε, όταν σας μισήσωσιν οι άνθρωποι, και όταν σας αφορίσωσι και ονειδίσωσι και εκβάλωσι το όνομά σας ως κακόν ένεκεν του Υιού του άνθρώπου.
\par 23 Χαίρετε εν εκείνη τη ημέρα και σκιρτήσατε· διότι ιδού, ο μισθός σας είναι πολύς εν τω ουρανώ· επειδή ούτως έπραττον εις τους προφήτας οι πατέρες αυτών.
\par 24 Πλην ουαί εις εσάς τους πλουσίους, διότι απηλαύσατε την παρηγορίαν σας.
\par 25 Ουαί εις εσάς, οι κεχορτασμένοι, διότι θέλετε πεινάσει. Ουαί εις εσάς, οι γελώντες τώρα, διότι θέλετε πενθήσει και κλαύσει.
\par 26 Ουαί εις εσάς, όταν πάντες οι άνθρωποι σας ευφημήσωσι· διότι ούτως έπραττον εις τους ψευδοπροφήτας οι πατέρες αυτών.
\par 27 Αλλά προς εσάς τους ακούοντας λέγω· Αγαπάτε τους εχθρούς σας, αγαθοποιείτε εκείνους, οίτινες σας μισούσιν,
\par 28 ευλογείτε εκείνους, οίτινες σας καταρώνται, και προσεύχεσθε υπέρ εκείνων, οίτινες σας βλάπτουσιν.
\par 29 Εις τον τύπτοντά σε επί την σιαγόνα πρόσφερε και την άλλην, και από του αφαιρούντος το ιμάτιόν σου μη εμποδίσης και τον χιτώνα.
\par 30 Εις πάντα δε τον ζητούντα παρά σου δίδε, και από του αφαιρούντος τα σα μη απαίτει.
\par 31 Και καθώς θέλετε να πράττωσιν εις εσάς οι άνθρωποι, και σεις πράττετε ομοίως εις αυτούς.
\par 32 Και εάν αγαπάτε τους αγαπώντάς σας, ποία χάρις χρεωστείται εις εσάς; διότι και οι αμαρτωλοί αγαπώσι τους αγαπώντας αυτούς.
\par 33 Και εάν αγαθοποιήτε τους αγαθοποιούντάς σας, ποία χάρις χρεωστείται εις εσάς; διότι και οι αμαρτωλοί το αυτό πράττουσι.
\par 34 Και εάν δανείζητε εις εκείνους, παρ' ων ελπίζετε πάλιν να λάβητε, ποία χάρις χρεωστείται εις εσάς; διότι και οι αμαρτωλοί εις αμαρτωλούς δανείζουσι διά να λάβωσι πάλιν τα ίσα.
\par 35 Πλην αγαπάτε τους εχθρούς σας και αγαθοποιείτε και δανείζετε, μηδεμίαν απολαβήν ελπίζοντες, και θέλει είσθαι ο μισθός σας πολύς, και θέλετε είσθαι υιοί του Υψίστου· διότι αυτός είναι αγαθός προς τους αχαρίστους και κακούς.
\par 36 Γίνεσθε λοιπόν οικτίρμονες, καθώς και ο Πατήρ σας είναι οικτίρμων.
\par 37 Και μη κρίνετε, και δεν θέλετε κριθή· μη καταδικάζετε, και δεν θέλετε καταδικασθή· συγχωρείτε, και θέλετε συγχωρηθή·
\par 38 δίδετε, και θέλει δοθή εις εσάς· μέτρον καλόν, πεπιεσμένον και συγκεκαθισμένον και υπερεκχυνόμενον θέλουσι δώσει εις τον κόλπον σας. Διότι με το αυτό μέτρον, με το οποίον μετρείτε, θέλει αντιμετρηθή εις εσάς.
\par 39 Είπε δε παραβολήν προς αυτούς, Μήπως δύναται τυφλός να οδηγή τυφλόν; δεν θέλουσι πέσει αμφότεροι εις βόθρον;
\par 40 Δεν είναι μαθητής ανώτερος του διδασκάλου αυτού· πας δε τετελειοποιημένος θέλει είσθαι ως ο διδάσκαλος αυτού.
\par 41 Και διά τι βλέπεις το ξυλάριον το εν τω οφθαλμώ του αδελφού σου, την δε δοκόν την εν τω ιδίω σου οφθαλμώ δεν παρατηρείς;
\par 42 ή πως δύνασαι να λέγης προς τον αδελφόν σου· Αδελφέ, άφες να εκβάλω το ξυλάριον το εν τω οφθαλμώ σου, ενώ συ δεν βλέπεις την δοκόν την εν τω οφθαλμώ σου; Υποκριτά, έκβαλε πρώτον την δοκόν εκ του οφθαλμού σου, και τότε θέλεις ιδεί καθαρώς διά να εκβάλης το ξυλάριον το εν τω οφθαλμώ του αδελφού σου.
\par 43 Διότι δεν είναι δένδρον καλόν, το οποίον κάμνει καρπόν σαπρόν, ουδέ δένδρον σαπρόν, το οποίον κάμνει καρπόν καλόν·
\par 44 επειδή έκαστον δένδρον εκ του καρπού αυτού γνωρίζεται. Διότι δεν συνάγουσιν εξ ακανθών σύκα, ουδέ τρυγώσιν εκ βάτου σταφύλια.
\par 45 Ο αγαθός άνθρωπος εκ του αγαθού θησαυρού της καρδίας αυτού εκφέρει το αγαθόν, και ο κακός άνθρωπος εκ του κακού θησαυρού της καρδίας αυτού εκφέρει το κακόν· διότι εκ του περισσεύματος της καρδίας λαλεί το στόμα αυτού.
\par 46 Διά τι δε με κράζετε, Κύριε, Κύριε, και δεν πράττετε όσα λέγω;
\par 47 Πας όστις έρχεται προς εμέ και ακούει τους λόγους μου και κάμνει αυτούς, θέλω σας δείξει με ποίον είναι όμοιος·
\par 48 είναι όμοιος με άνθρωπον οικοδομούντα οικίαν, όστις έσκαψε και εβάθυνε και έβαλε θεμέλιον επί την πέτραν· ότε δε έγεινε πλημμύρα, προσέβαλεν ο ποταμός κατά της οικίας εκείνης και δεν ηδυνήθη να σαλεύση αυτήν· διότι ήτο τεθεμελιωμένη επί την πέτραν.
\par 49 Όστις όμως ακούση και δεν κάμη, είναι όμοιος με άνθρωπον οικοδομήσαντα οικίαν επί την γην χωρίς θεμέλιον· κατά της οποίας προσέβαλεν ο ποταμός, και ευθύς έπεσε, και έγεινεν ο κρημνισμός της οικίας εκείνης μέγας.

\chapter{7}

\par 1 Αφού δε ετελείωσε πάντας τους λόγους αυτού εις τας ακοάς του λαού, εισήλθεν εις Καπερναούμ.
\par 2 Εκατοντάρχου δε τινός δούλος, όστις ήτο πολύτιμος εις αυτόν, κακώς έχων έμελλε να αποθάνη.
\par 3 Και ακούσας περί του Ιησού, απέστειλε προς αυτόν πρεσβυτέρους των Ιουδαίων, παρακαλών αυτόν να έλθη να διασώση τον δούλον αυτού.
\par 4 Οι δε ελθόντες προς τον Ιησούν, παρεκάλουν αυτόν επιμόνως, λέγοντες ότι είναι άξιος εκείνος, εις τον οποίον θέλεις κάμει τούτο·
\par 5 διότι αγαπά το έθνος υμών, και την συναγωγήν αυτός ωκοδόμησεν εις ημάς.
\par 6 Ο δε Ιησούς επορεύετο μετ' αυτών. Ενώ δε απείχεν ήδη ου μακράν από της οικίας, έπεμψε προς αυτόν ο εκατόνταρχος φίλους, λέγων προς αυτόν· Κύριε, μη ενοχλείσαι· διότι δεν είμαι άξιος να εισέλθης υπό την στέγην μου·
\par 7 όθεν ουδέ εμαυτόν έκρινα άξιον να έλθω προς σέ· αλλά ειπέ λόγον, και θέλει ιατρευθή ο δούλός μου.
\par 8 Διότι και εγώ είμαι άνθρωπος υποκείμενος εις εξουσίαν, έχων υπ' εμαυτόν στρατιώτας, και λέγω προς τούτον, Ύπαγε, και υπάγει; και προς άλλον, Έρχου, και έρχεται, και προς τον δούλον μου, Κάμε τούτο, και κάμνει.
\par 9 Ακούσας δε ταύτα ο Ιησούς εθαύμασεν αυτόν, και στραφείς προς τον όχλον τον ακολουθούντα αυτόν, είπε· Σας λέγω, Ουδέ εν τω Ισραήλ εύρον τοσαύτην πίστιν.
\par 10 Και υποστρέψαντες οι απεσταλμένοι εις τον οίκον, εύρον τον ασθενή δούλον υγιαίνοντα.
\par 11 Την δε ακόλουθον ημέραν επορεύετο ο Ιησούς εις πόλιν ονομαζομένην Ναΐν· και συνεπορεύοντο μετ' αυτού ικανοί εκ των μαθητών αυτού και όχλος πολύς.
\par 12 Ως δε επλησίασεν εις την πύλην της πόλεως, ιδού, εφέρετο έξω νεκρός υιός μονογενής της μητρός αυτού, και αύτη ήτο χήρα, και όχλος πολύς της πόλεως ήτο μετ' αυτής.
\par 13 Και ιδών αυτήν ο Κύριος, εσπλαγχνίσθη δι' αυτήν και είπε προς αυτήν· Μη κλαίε·
\par 14 και πλησιάσας ήγγισε το νεκροκράββατον, οι δε βαστάζοντες εστάθησαν, και είπε· Νεανίσκε, προς σε λέγω, σηκώθητι.
\par 15 Και ανεκάθησεν ο νεκρός και ήρχισε να λαλή, και έδωκεν αυτόν εις την μητέρα αυτού.
\par 16 Κατέλαβε δε άπαντας φόβος και εδόξαζον τον Θεόν, λέγοντες ότι προφήτης μέγας ηγέρθη εν ημίν, και ότι επεσκέφθη ο Θεός τον λαόν αυτού.
\par 17 Και εξήλθεν ο λόγος ούτος περί αυτού εν όλη τη Ιουδαία και εν πάσι τοις περιχώροις.
\par 18 Και απήγγειλαν προς τον Ιωάννην οι μαθηταί αυτού περί πάντων τούτων.
\par 19 Και προσκαλέσας ο Ιωάννης δύο τινάς των μαθητών αυτού, έπεμψε προς τον Ιησούν, λέγων· Συ είσαι ο ερχόμενος, ή άλλον προσδοκώμεν;
\par 20 Και ελθόντες προς αυτόν οι άνθρωποι, είπον· Ιωάννης ο Βαπτιστής απέστειλεν ημάς προς σε, λέγων· Συ είσαι ο ερχόμενος, ή άλλον προσδοκώμεν;
\par 21 Εν αυτή δε τη ώρα εθεράπευσε πολλούς από νόσων και μαστίγων και πνευμάτων πονηρών, και εις τυφλούς πολλούς εχάρισε το βλέπειν.
\par 22 Και αποκριθείς ο Ιησούς, είπε προς αυτούς· Υπάγετε και απαγγείλατε προς τον Ιωάννην όσα είδετε και ηκούσατε· ότι τυφλοί αναβλέπουσι, χωλοί περιπατούσι, λεπροί καθαρίζονται, κωφοί ακούουσι, νεκροί εγείρονται, πτωχοί ευαγγελίζονται·
\par 23 και μακάριος είναι όστις δεν σκανδαλισθή εν εμοί.
\par 24 Αφού δε ανεχώρησαν οι απεσταλμένοι του Ιωάννου, ήρχισε να λέγη προς τους όχλους περί του Ιωάννου· Τι εξήλθετε εις την έρημον να ίδητε; κάλαμον υπό ανέμου σαλευόμενον;
\par 25 Αλλά τι εξήλθετε να ίδητε; άνθρωπον ενδεδυμένον μαλακά ιμάτια; ιδού, οι λαμπρώς ενδεδυμένοι και τρυφώντες ευρίσκονται εν τοις βασιλικοίς παλατίοις.
\par 26 Αλλά τι εξήλθετε να ίδητε; προφήτην; Ναι, σας λέγω, και περισσότερον προφήτου.
\par 27 Ούτος είναι, περί του οποίου είναι γεγραμμένον, Ιδού, εγώ αποστέλλω τον άγγελόν μου προ προσώπου σου, Όστις θέλει κατασκευάσει την οδόν σου έμπροσθέν σου.
\par 28 Διότι σας λέγω, μεταξύ των γεννηθέντων εκ γυναικών ουδείς προφήτης είναι μεγαλήτερος Ιωάννου του βαπτιστού· πλην ο μικρότερος εν τη βασιλεία του Θεού είναι μεγαλήτερος αυτού.
\par 29 Και πας ο λαός ακούσας και οι τελώναι εδικαίωσαν τον Θεόν, βαπτισθέντες το βάπτισμα του Ιωάννου.
\par 30 Οι δε Φαρισαίοι και οι νομικοί ηθέτησαν εις εαυτούς την βουλήν του Θεού, μη βαπτισθέντες υπ' αυτού.
\par 31 Και είπεν ο Κύριος· Με τι λοιπόν να ομοιώσω τους ανθρώπους της γενεάς ταύτης; και με τι είναι όμοιοι;
\par 32 Είναι όμοιοι με παιδία καθήμενα εν τη αγορά και φωνάζοντα προς άλληλα και λέγοντα· Αυλόν σας επαίξαμεν, και δεν εχορεύσατε· σας εθρηνωδήσαμεν, και δεν εκλαύσατε.
\par 33 Διότι ήλθεν Ιωάννης ο Βαπτιστής μήτε άρτον τρώγων μήτε οίνον πίνων, και λέγετε· Δαιμόνιον έχει.
\par 34 Ήλθεν ο Υιός του ανθρώπου τρώγων και πίνων, και λέγετε· Ιδού άνθρωπος φάγος και οινοπότης, φίλος τελωνών και αμαρτωλών.
\par 35 Και εδικαιώθη η σοφία από πάντων των τέκνων αυτής.
\par 36 Παρεκάλει δε αυτόν εις εκ των Φαρισαίων να φάγη μετ' αυτού· και εισελθών εις την οικίαν του Φαρισαίου, εκάθησεν εις την τράπεζαν.
\par 37 Και ιδού, γυνή τις εν τη πόλει, ήτις ήτο αμαρτωλή, μαθούσα ότι κάθηται εις την τράπεζαν εν τη οικία του Φαρισαίου, έφερεν αλάβαστρον μύρου
\par 38 και σταθείσα πλησίον των ποδών αυτού οπίσω κλαίουσα, ήρχισε να βρέχη τους πόδας αυτού με τα δάκρυα και εσπόγγιζε με τας τρίχας της κεφαλής αυτής και κατεφίλει τους πόδας αυτού και ήλειφε με το μύρον.
\par 39 Ιδών δε ο Φαρισαίος ο καλέσας αυτόν, είπε καθ' εαυτόν λέγων· Ούτος, εάν ήτο προφήτης, ήθελε γνωρίζει τις και οποία είναι η γυνή, ήτις εγγίζει αυτόν, ότι είναι αμαρτωλή.
\par 40 Και αποκριθείς ο Ιησούς, είπε προς αυτόν· Σίμων, έχω να σοι είπω τι. Ο δε λέγει· Διδάσκαλε, ειπέ.
\par 41 Είχε τις δανειστάς δύο χρεωφειλέτας· ο εις εχρεώστει δηνάρια πεντακόσια, ο δε άλλος πεντήκοντα.
\par 42 Και επειδή δεν είχον να αποδώσωσιν, εχάρισεν αυτά εις αμφοτέρους. Τις λοιπόν εξ αυτών, ειπέ, θέλει αγαπήσει αυτόν περισσότερον;
\par 43 Αποκριθείς δε ο Σίμων, είπε· Νομίζω ότι εκείνος, εις τον οποίον εχάρισε το περισσότερον. Ο δε είπε προς αυτόν· Ορθώς έκρινας.
\par 44 Και στραφείς προς την γυναίκα, είπε προς τον Σίμωνα· Βλέπεις ταύτην την γυναίκα; Εισήλθον εις την οικίαν σου, ύδωρ διά τους πόδας μου δεν έδωκας· αύτη δε με τα δάκρυα έβρεξε τους πόδας μου και με τας τρίχας της κεφαλής αυτής εσπόγγισε.
\par 45 Φίλημα δεν μοι έδωκας· αύτη δε, αφ' ης εισήλθον, δεν έπαυσε καταφιλούσα τους πόδας μου.
\par 46 Με έλαιον την κεφαλήν μου δεν ήλειψας· αύτη δε με μύρον ήλειψε τους πόδας μου.
\par 47 Διά τούτο σοι λέγω, συγκεχωρημέναι είναι αι αμαρτίαι αυτής αι πολλαί, διότι ηγάπησε πολύ· εις όντινα δε συγχωρείται ολίγον, ολίγον αγαπά.
\par 48 Και είπε προς αυτήν· Συγκεχωρημέναι είναι αι αμαρτίαι σου.
\par 49 Και ήρχισαν οι συγκαθήμενοι εις την τράπεζαν να λέγωσι καθ' εαυτούς· Τις είναι ούτος, όστις και αμαρτίας συγχωρεί;
\par 50 Είπε δε προς την γυναίκα· Η πίστις σου σε έσωσεν· ύπαγε εις ειρήνην.

\chapter{8}

\par 1 Και μετά ταύτα διήρχετο αυτός πάσαν πόλιν και κώμην, κηρύττων και ευαγγελιζόμενος την βασιλείαν του Θεού, και οι δώδεκα ήσαν μετ' αυτού,
\par 2 και γυναίκες τινές, αίτινες ήσαν τεθεραπευμέναι από πνευμάτων πονηρών και ασθενειών, Μαρία η καλουμένη Μαγδαληνή, εκ της οποίας είχον εκβή επτά δαιμόνια,
\par 3 και Ιωάννα η γυνή του Χουζά, επιτρόπου του Ηρώδου, και Σουσάννα και άλλαι πολλαί, αίτινες διηκόνουν αυτόν από των υπαρχόντων αυτών.
\par 4 Επειδή δε συνέτρεχεν όχλος πολύς και ήρχοντο προς αυτόν από πάσης πόλεως, είπε διά παραβολής·
\par 5 Εξήλθεν ο σπείρων, διά να σπείρη τον σπόρον αυτού. Και ενώ έσπειρεν, άλλο μεν έπεσε παρά την οδόν και κατεπατήθη, και τα πετεινά του ουρανού κατέφαγον αυτό·
\par 6 άλλο δε έπεσεν επί την πέτραν και αναφυέν εξηράνθη, διότι δεν είχεν ικμάδα·
\par 7 και άλλο έπεσεν εις το μέσον των ακανθών, και συμφυτρώσασαι αι άκανθαι απέπνιξαν αυτό·
\par 8 και άλλο έπεσεν επί την γην την αγαθήν, και αναφυέν έκαμε καρπόν εκατονταπλασίονα. Ταύτα λέγων, εφώναζεν· Ο έχων ώτα διά να ακούη, ας ακούη.
\par 9 Ηρώτων δε αυτόν οι μαθηταί αυτού, λέγοντες· Τι σημαίνει η παραβολή αύτη;
\par 10 Ο δε είπεν· Εις εσάς εδόθη να γνωρίσητε τα μυστήρια της βασιλείας του Θεού, εις δε τους λοιπούς διά παραβολών, διά να μη βλέπωσιν ενώ βλέπουσι και να μη καταλαμβάνωσιν ενώ ακούουσιν.
\par 11 Αύτη δε είναι η παραβολή· Ο σπόρος είναι ο λόγος του Θεού·
\par 12 οι δε σπειρόμενοι παρά την οδόν είναι οι ακούοντες, έπειτα έρχεται ο διάβολος και αφαιρεί τον λόγον από της καρδίας αυτών, διά να μη πιστεύσωσι και σωθώσιν.
\par 13 Οι δε επί της πέτρας είναι εκείνοι οίτινες, όταν ακούσωσι, μετά χαράς δέχονται τον λόγον, και ούτοι ρίζαν δεν έχουσιν, οίτινες προς καιρόν πιστεύουσι και εν καιρώ πειρασμού αποστατούσι.
\par 14 Το δε πεσόν εις τας ακάνθας, ούτοι είναι εκείνοι οίτινες ήκουσαν, και υπό μεριμνών και πλούτου και ηδονών του βίου υπάγουσι και συμπνίγονται και δεν τελεσφορούσι.
\par 15 Το δε εις την καλήν γην, ούτοι είναι εκείνοι, οίτινες ακούσαντες τον λόγον, κρατούσιν εν καρδία καλή και αγαθή και καρποφορούσιν εν υπομονή.
\par 16 Ουδείς δε λύχνον ανάψας, σκεπάζει αυτόν με σκεύος και θέτει υποκάτω κλίνης, αλλά θέτει επί του λυχνοστάτου, διά να βλέπωσι το φως οι εισερχόμενοι.
\par 17 Διότι δεν υπάρχει κρυπτόν, το οποίον δεν θέλει γείνει φανερόν; ουδέ απόκρυφον, το οποίον δεν θέλει γείνει γνωστόν και ελθεί εις το φανερόν.
\par 18 Προσέχετε λοιπόν πως ακούετε· διότι όστις έχει, θέλει δοθή εις αυτόν, και όστις δεν έχει, και εκείνο το οποίον νομίζει ότι έχει θέλει αφαιρεθή απ' αυτού.
\par 19 Ήλθον δε προς αυτόν η μήτηρ και οι αδελφοί αυτού και δεν ηδύναντο διά τον όχλον να πλησιάσωσιν αυτόν.
\par 20 Και απηγγέλθη προς αυτόν υπό τινών λεγόντων· Η μήτηρ σου και οι αδελφοί σου ίστανται έξω θέλοντες να σε ίδωσιν.
\par 21 Ο δε αποκριθείς είπε προς αυτούς· Μήτηρ μου και αδελφοί μου είναι ούτοι, οι ακούοντες τον λόγον του Θεού και πράττοντες αυτόν.
\par 22 Και εν μιά των ημερών εισήλθεν εις πλοίον αυτός και οι μαθηταί αυτού, και είπε προς αυτούς· Ας διέλθωμεν εις το πέραν της λίμνης· και εσηκώθησαν.
\par 23 Ενώ δε έπλεον, απεκοιμήθη. Και κατέβη ανεμοστρόβιλος εις την λίμνην, και εγεμίζετο το πλοίον και εκινδύνευον.
\par 24 Προσελθόντες δε εξύπνησαν αυτόν, λέγοντες· Επιστάτα, Επιστάτα, χανόμεθα. Ο δε σηκωθείς επετίμησε τον άνεμον και την ταραχήν του ύδατος, και έπαυσαν, και έγεινε γαλήνη.
\par 25 Είπε δε προς αυτούς, που είναι η πίστις σας; Και φοβηθέντες εθαύμασαν, λέγοντες προς αλλήλους· Τις λοιπόν είναι ούτος, ότι και τους ανέμους προστάζει και το ύδωρ, και υπακούουσιν εις αυτόν;
\par 26 Και κατέπλευσαν εις την χώραν των Γαδαρηνών, ήτις είναι αντιπέραν της Γαλιλαίας.
\par 27 Και καθώς εξήλθεν επί την γην, υπήντησεν αυτόν άνθρωπός τις εκ της πόλεως, όστις είχε δαιμόνια από χρόνων πολλών, και ιμάτιον δεν ενεδύετο και εν οικία δεν έμενεν, αλλ' εν τοις μνήμασιν.
\par 28 Ιδών δε τον Ιησούν, ανέκραξε και προσέπεσεν εις αυτόν και μετά φωνής μεγάλης είπε· Τι είναι μεταξύ εμού και σου, Ιησού, Υιέ του Θεού του Υψίστου; δέομαί σου, μη με βασανίσης.
\par 29 Διότι προσέταξεν εις το πνεύμα το ακάθαρτον να εξέλθη από του ανθρώπου. Επειδή προ πολλών χρόνων είχε συναρπάσει αυτόν, και εδεσμεύετο με αλύσεις και εφυλάττετο με ποδόδεσμα και διασπών τα δεσμά, εφέρετο υπό του δαίμονος εις τας ερήμους.
\par 30 Και ηρώτησεν αυτόν ο Ιησούς, λέγων· Τι είναι το όνομά σου; Ο δε είπε· Λεγεών· διότι δαιμόνια πολλά εισήλθον εις αυτόν·
\par 31 και παρεκάλουν αυτόν να μη προστάξη αυτά να απέλθωσιν εις την άβυσσον.
\par 32 Ήτο δε εκεί αγέλη χοίρων πολλών βοσκομένων εν τω όρει· και παρεκάλουν αυτόν να επιτρέψη εις αυτά να εισέλθωσιν εις εκείνους· και επέτρεψεν εις αυτά.
\par 33 Εξελθόντα δε τα δαιμόνια από του ανθρώπου, εισήλθον εις τους χοίρους, και ώρμησεν η αγέλη κατά του κρημνού εις την λίμνην και απεπνίγη.
\par 34 Ιδόντες δε οι βοσκοί το γενόμενον έφυγον, και απελθόντες απήγγειλαν εις την πόλιν και εις τους αγρούς.
\par 35 Και εξήλθον διά να ίδωσι το γεγονός, και ήλθον προς τον Ιησούν και εύρον τον άνθρωπον, εκ του οποίου είχον εξέλθει τα δαιμόνια, καθήμενον παρά τους πόδας του Ιησού, ενδεδυμένον και σωφρονούντα· και εφοβήθησαν.
\par 36 Διηγήθησαν δε προς αυτούς και οι ιδόντες πως εσώθη ο δαιμονιζόμενος.
\par 37 Και άπαν το πλήθος της περιχώρου των Γαδαρηνών παρεκάλεσαν αυτόν να αναχωρήση απ' αυτών, διότι κατείχοντο υπό μεγάλου φόβου, αυτός δε εμβάς εις το πλοίον υπέστρεψεν.
\par 38 Ο δε άνθρωπος, εκ του οποίου είχον εξέλθει τα δαιμόνια, παρεκάλει αυτόν να ήναι μετ' αυτού· ο Ιησούς όμως απέλυσεν αυτόν, λέγων.
\par 39 Επίστρεψον εις τον οίκόν σου και διηγού όσα έκαμεν εις σε ο Θεός· και ανεχώρησε κηρύττων καθ' όλην την πόλιν όσα έκαμεν εις αυτόν ο Ιησούς.
\par 40 Ότε δε υπέστρεψεν ο Ιησούς, υπεδέχθη αυτόν ο όχλος· διότι πάντες ήσαν περιμένοντες αυτόν.
\par 41 Και ιδού, ήλθεν άνθρωπος ονομαζόμενος Ιάειρος, όστις ήτο άρχων της συναγωγής και πεσών εις τους πόδας του Ιησού, παρεκάλει αυτόν να εισέλθη εις τον οίκον αυτού,
\par 42 διότι είχε θυγατέρα μονογενή ως ετών δώδεκα, και αύτη απέθνησκεν. Ενώ δε επορεύετο, οι όχλοι συνέθλιβον αυτόν.
\par 43 Και γυνή τις έχουσα ρύσιν αίματος δώδεκα έτη, ήτις δαπανήσασα εις ιατρούς όλον τον βίον αυτής δεν ηδυνήθη να θεραπευθή υπ' ουδενός,
\par 44 πλησιάσασα όπισθεν ήγγισε το άκρον του ιματίου αυτού, και παρευθύς εστάθη η ρύσις του αίματος αυτής.
\par 45 Και είπεν ο Ιησούς· Τις μου ήγγισε; και ενώ ηρνούντο πάντες, είπεν ο Πέτρος και οι μετ' αυτού· Επιστάτα, οι όχλοι σε συμπιέζουσι και σε συνθλίβουσι, και λέγεις· Τις μου ήγγισεν;
\par 46 Ο δε Ιησούς είπε· Μου ήγγισέ τις· διότι εγώ ενόησα ότι εξήλθε δύναμις απ' εμού.
\par 47 Ιδούσα δε η γυνή ότι δεν εκρύφθη, ήλθε τρέμουσα και προσπεσούσα εις αυτόν, απήγγειλε προς αυτόν ενώπιον παντός του λαού διά ποίαν αιτίαν ήγγισεν αυτόν, και ότι παρευθύς ιατρεύθη.
\par 48 Ο δε είπε προς αυτήν· Θάρρει, θύγατερ, η πίστις σου σε έσωσεν· ύπαγε εις ειρήνην.
\par 49 Ενώ δε ελάλει έτι, έρχεταί τις παρά του αρχισυναγώγου, λέγων προς αυτόν ότι απέθανεν η θυγάτηρ σου· μη ενόχλει τον Διδάσκαλον.
\par 50 Ο δε Ιησούς ακούσας απεκρίθη προς αυτόν, λέγων· Μη φοβού· μόνον πίστευε, και θέλει σωθή.
\par 51 Και ότε εισήλθεν εις την οικίαν, δεν αφήκεν ουδένα να εισέλθη ειμή τον Πέτρον και Ιάκωβον και Ιωάννην και τον πατέρα της κόρης και την μητέρα.
\par 52 Έκλαιον δε πάντες και εθρήνουν αυτήν. Ο δε είπε· Μη κλαίετε· δεν απέθανεν, αλλά κοιμάται.
\par 53 Και κατεγέλων αυτόν, εξεύροντες ότι απέθανεν.
\par 54 Αλλ' αυτός εκβαλών έξω πάντας και πιάσας την χείρα αυτής, εφώναξε λέγων· Κοράσιον, σηκώθητι.
\par 55 Και υπέστρεψε το πνεύμα αυτής, και ανέστη παρευθύς, και προσέταξε να δοθή εις αυτήν να φάγη.
\par 56 Και εξεπλάγησαν οι γονείς αυτής. Ο δε παρήγγειλεν εις αυτούς να μη είπωσιν εις μηδένα το γεγονός.

\chapter{9}

\par 1 Συγκαλέσας δε τους δώδεκα μαθητάς αυτού, έδωκεν εις αυτούς δύναμιν και εξουσίαν κατά πάντων των δαιμονίων και να θεραπεύωσι νόσους·
\par 2 και απέστειλεν αυτούς διά να κηρύττωσι την βασιλείαν του Θεού και να ιατρεύωσι τους ασθενούντας,
\par 3 και είπε προς αυτούς· Μη βαστάζετε μηδέν εις την οδόν, μήτε ράβδους μήτε σακκίον μήτε άρτον μήτε αργύριον μήτε να έχητε ανά δύο χιτώνας.
\par 4 Και εις ήντινα οικίαν εισέλθητε, εκεί μένετε και εκείθεν εξέρχεσθε.
\par 5 Και όσοι δεν σας δεχθώσιν, εξερχόμενοι από της πόλεως εκείνης αποτινάξατε και τον κονιορτόν από των ποδών σας διά μαρτυρίαν κατ' αυτών.
\par 6 Εξερχόμενοι δε διήρχοντο από κώμης εις κώμην, κηρύττοντες το ευαγγέλιον και θεραπεύοντες πανταχού.
\par 7 Ήκουσε δε Ηρώδης ο τετράρχης πάντα τα γινόμενα υπ' αυτού, και ήτο εν απορία, διότι ελέγετο υπό τινών ότι ο Ιωάννης ανέστη εκ νεκρών·
\par 8 υπό τινών δε ότι ο Ηλίας εφάνη, υπ' άλλων δε, ότι ανέστη εις των αρχαίων προφητών.
\par 9 Και είπεν ο Ηρώδης· Τον Ιωάννην εγώ απεκεφάλισα· τις δε είναι ούτος, περί του οποίου εγώ ακούω τοιαύτα; και εζήτει να ίδη αυτόν.
\par 10 Και υποστρέψαντες οι απόστολοι, διηγήθησαν προς αυτόν όσα έπραξαν. Και παραλαβών αυτούς απεσύρθη κατ' ιδίαν εις τόπον έρημον πόλεώς τινός ονομαζομένης Βηθσαϊδά.
\par 11 Οι δε όχλοι νοήσαντες ηκολούθησαν αυτόν, και δεχθείς αυτούς ελάλει προς αυτούς περί της βασιλείας του Θεού, και τους έχοντας χρείαν θεραπείας ιάτρευεν.
\par 12 Η δε ημέρα ήρχισε να κλίνη· και προσελθόντες οι δώδεκα, είπον προς αυτόν· Απόλυσον τον όχλον, διά να υπάγωσιν εις τας πέριξ κώμας και τους αγρούς και να καταλύσωσι και να εύρωσι τροφάς, διότι εδώ είμεθα εν ερήμω τόπω.
\par 13 Και είπε προς αυτούς· Δότε σεις εις αυτούς να φάγωσιν. Οι δε είπον· Ημείς δεν έχομεν πλειότερον παρά πέντε άρτους και δύο ιχθύας, εκτός εάν υπάγωμεν ημείς και αγοράσωμεν τροφάς δι' όλον τον λαόν τούτον·
\par 14 διότι ήσαν ως πεντακισχίλιοι άνδρες· και είπε προς τους μαθητάς αυτού· Καθίσατε αυτούς κατά αθροίσματα ανά πεντήκοντα.
\par 15 Και έπραξαν ούτω, και εκάθησαν άπαντας.
\par 16 Λαβών δε τους πέντε άρτους και τους δύο ιχθύας, ανέβλεψεν εις τον ουρανόν και ευλόγησεν αυτούς και κατέκοψε, και έδιδεν εις τους μαθητάς διά να βάλλωσιν έμπροσθεν του όχλου.
\par 17 Και έφαγον και εχορτάσθησαν πάντες, και εσηκώθη το περισσεύσαν εις αυτούς εκ των κλασμάτων δώδεκα κοφίνια.
\par 18 Και ενώ αυτός προσηύχετο καταμόνας, ήσαν μετ' αυτού οι μαθηταί, και ηρώτησεν αυτούς λέγων· Τίνα με λέγουσιν οι όχλοι ότι είμαι;
\par 19 οι δε αποκριθέντες είπον· Ιωάννην τον Βαπτιστήν, άλλοι δε Ηλίαν, άλλοι δε ότι ανέστη τις των αρχαίων προφητών.
\par 20 Είπε δε προς αυτούς· Σεις δε τίνα με λέγετε ότι είμαι; και αποκριθείς ο Πέτρος είπε· Τον Χριστόν του Θεού.
\par 21 Ο δε προσέταξεν αυτούς σφοδρώς και παρήγγειλε να μη είπωσιν εις μηδένα τούτο,
\par 22 ειπών ότι πρέπει ο Υιός του ανθρώπου να πάθη πολλά και να καταφρονηθή από των πρεσβυτέρων και αρχιερέων και γραμματέων, και να θανατωθή και τη τρίτη ημέρα να αναστηθή.
\par 23 Έλεγε δε προς πάντας· Εάν τις θέλη να έλθη οπίσω μου, ας απαρνηθή εαυτόν και ας σηκώση τον σταυρόν αυτού καθ' ημέραν και ας με ακολουθή.
\par 24 Διότι όστις θέλει να σώση την ζωήν αυτού, θέλει απολέσει αυτήν· και όστις απολέση την ζωήν αυτού ένεκεν ομού, ούτος θέλει σώσει αυτήν.
\par 25 Επειδή τι ωφελείται ο άνθρωπος, εάν κερδήση τον κόσμον όλον, εαυτόν δε απολέση ή ζημιωθή;
\par 26 Διότι όστις επαισχυνθή δι' εμέ και τους λόγους μου, διά τούτον ο Υιός του ανθρώπου θέλει επαισχυνθή, όταν έλθη εν τη δόξη αυτού και του Πατρός και των αγίων αγγέλων.
\par 27 Λέγω δε προς εσάς αληθώς, Είναι τινές των εδώ ισταμένων, οίτινες δεν θέλουσι γευθή θάνατον, εωσού ίδωσι την βασιλείαν του Θεού.
\par 28 Μετά δε τους λόγους τούτους παρήλθον έως οκτώ ημέραι, και παραλαβών τον Πέτρον και Ιωάννην και Ιάκωβον, ανέβη εις το όρος διά να προσευχηθή.
\par 29 Και ενώ προσηύχετο, ηλλοιώθη η όψις του προσώπου αυτού και τα ιμάτια αυτού έγειναν λευκά εξαστράπτοντα.
\par 30 και ιδού, άνδρες δύο συνελάλουν μετ' αυτού, οίτινες ήσαν Μωϋσής και Ηλίας,
\par 31 οίτινες φανέντες εν δόξη, έλεγον τον θάνατον αυτού, τον οποίον έμελλε να εκπληρώση εν Ιερουσαλήμ.
\par 32 Ο δε Πέτρος και οι μετ' αυτού ήσαν βεβαρημένοι υπό του ύπνου· και ότε εξύπνησαν, είδον την δόξαν αυτού και τους δύο άνδρας τους ισταμένους μετ' αυτού.
\par 33 Και ενώ αυτοί εχωρίζοντο απ' αυτού, είπεν ο Πέτρος προς τον Ιησούν· Επιστάτα, καλόν είναι να ήμεθα εδώ· και ας κάμωμεν τρεις σκηνάς, μίαν διά σε και διά τον Μωϋσήν μίαν και μίαν διά τον Ηλίαν, μη εξεύρων τι λέγει.
\par 34 Ενώ δε αυτός έλεγε ταύτα, ήλθε νεφέλη και επεσκίασεν αυτούς· και εφοβήθησαν ότε εισήλθον εις την νεφέλην·
\par 35 και έγεινε φωνή εκ της νεφέλης, λέγουσα· Ούτος είναι ο Υιός μου ο αγαπητός· αυτού ακούετε.
\par 36 Και αφού έγεινεν η φωνή, ευρέθη ο Ιησούς μόνος· και αυτοί εσιώπησαν και προς ουδένα είπον εν εκείναις ταις ημέραις ουδέν εξ όσων είδον.
\par 37 Την δε ακόλουθον ημέραν, ότε κατέβησαν από του όρους, υπήντησεν αυτόν όχλος πολύς.
\par 38 Και ιδού, άνθρωπός τις εκ του όχλου ανέκραξε, λέγων· Διδάσκαλε, δέομαί σου, επίβλεψον επί τον υιόν μου, διότι μονογενής μου είναι·
\par 39 και ιδού, δαιμόνιον πιάνει αυτόν, και εξαίφνης κράζει και σπαράττει αυτόν μετά αφρού, και μόλις αναχωρεί απ' αυτού, συντρίβον αυτόν·
\par 40 και παρεκάλεσα τους μαθητάς σου διά να εκβάλωσιν αυτό, και δεν ηδυνήθησαν.
\par 41 Αποκριθείς δε ο Ιησούς, είπεν· Ω γενεά άπιστος και διεστραμμένη, έως πότε θέλω είσθαι μεθ' υμών και θέλω υπομένει υμάς; φέρε τον υιόν σου εδώ.
\par 42 Και ενώ αυτός έτι προσήρχετο, έρριψεν αυτόν κάτω το δαιμόνιον και κατεσπάραξεν· ο δε Ιησούς επετίμησε το πνεύμα το ακάθαρτον και ιάτρευσε το παιδίον και απέδωκεν αυτό εις τον πατέρα αυτού.
\par 43 Εξεπλήττοντο δε πάντες επί την μεγαλειότητα του Θεού. Και ενώ πάντες εθαύμαζον διά πάντα όσα έκαμεν ο Ιησούς, είπε προς τους μαθητάς αυτού·
\par 44 Βάλετε σεις εις τα ώτα σας τους λόγους τούτους· διότι ο Υιός του ανθρώπου μέλλει να παραδοθή εις χείρας ανθρώπων.
\par 45 Εκείνοι όμως δεν ενόουν τον λόγον τούτον, και ήτο αποκεκρυμμένος απ' αυτών, διά να μη νοήσωσιν αυτόν, και εφοβούντο να ερωτήσωσιν αυτόν περί του λόγου τούτου.
\par 46 Εισήλθε δε εις αυτούς διαλογισμός, τις τάχα εξ αυτών ήτο μεγαλήτερος.
\par 47 Ο δε Ιησούς, ιδών τον διαλογισμόν της καρδίας αυτών, επίασε παιδίον και έστησεν αυτό πλησίον εαυτού
\par 48 και είπε προς αυτούς· Όστις δεχθή τούτο το παιδίον εις το όνομά μου, εμέ δέχεται, και όστις δεχθή εμέ, δέχεται τον αποστείλαντά με· διότι ο υπάρχων μικρότερος μεταξύ πάντων υμών ούτος θέλει είσθαι μέγας.
\par 49 Αποκριθείς δε ο Ιωάννης, είπεν· Επιστάτα, είδομέν τινά εκβάλλοντα τα δαιμόνια εν τω ονόματί σου, και ημποδίσαμεν αυτόν, διότι δεν ακολουθεί μεθ' ημών.
\par 50 Και είπε προς αυτόν ο Ιησούς· Μη εμποδίζετε· διότι όστις δεν είναι καθ' ημών, είναι υπέρ ημών.
\par 51 Και ότε συνεπληρούντο αι ημέραι διά να αναληφθή, τότε αυτός έκαμε στερεάν απόφασιν να υπάγη εις Ιερουσαλήμ.
\par 52 Και απέστειλεν έμπροσθεν αυτού μηνυτάς, οίτινες πορευθέντες εισήλθον εις κώμην Σαμαρειτών, διά να κάμωσιν ετοιμασίαν εις αυτόν.
\par 53 Και δεν εδέχθησαν αυτόν, διότι εφαίνετο ότι επορεύετο εις Ιερουσαλήμ.
\par 54 Ιδόντες δε οι μαθηταί αυτού Ιάκωβος και Ιωάννης, είπον· Κύριε, θέλεις να είπωμεν να καταβή πυρ από του ουρανού και να αφανίση αυτούς, καθώς και ο Ηλίας έκαμε;
\par 55 Στραφείς δε επέπληξεν αυτούς και είπε· δεν εξεύρετε ποίου πνεύματος είσθε σείς·
\par 56 διότι ο Υιός του ανθρώπου δεν ήλθε να απολέση ψυχάς ανθρώπων, αλλά να σώση. Και υπήγον εις άλλην κώμην.
\par 57 Ενώ δε επορεύοντο, είπε τις προς αυτόν καθ' οδόν· Θέλω σε ακολουθήσει όπου αν υπάγης, Κύριε.
\par 58 Και είπε προς αυτόν ο Ιησούς· Αι αλώπεκες έχουσι φωλεάς και τα πετεινά του ουρανού κατοικίας, ο δε Υιός του ανθρώπου δεν έχει που να κλίνη την κεφαλήν.
\par 59 Είπε δε προς άλλον· Ακολούθει μοι. Ο δε είπε· Κύριε, συγχώρησόν μοι να υπάγω πρώτον να θάψω τον πατέρα μου.
\par 60 Και ο Ιησούς είπε προς αυτόν· Άφες τους νεκρούς να θάψωσι τους εαυτών νεκρούς· συ δε απελθών κήρυττε την βασιλείαν του Θεού.
\par 61 Είπε δε και άλλος· θέλω σε ακολουθήσει, Κύριε· πρώτον όμως συγχώρησόν μοι να αποχαιρετήσω τους εις τον οίκόν μου.
\par 62 Και είπε προς αυτόν ο Ιησούς· Ουδείς βαλών την χείρα αυτού επί άροτρον και βλέπων εις τα οπίσω είναι αρμόδιος διά την βασιλείαν του Θεού.

\chapter{10}

\par 1 Μετά δε ταύτα διώρισεν ο Κύριος και άλλους εβδομήκοντα, και απέστειλεν αυτούς ανά δύο έμπροσθεν αυτού εις πάσαν πόλιν και τόπον, όπου έμελλεν αυτός να υπάγη.
\par 2 Έλεγε λοιπόν προς αυτούς· Ο μεν θερισμός είναι πολύς, οι δε εργάται ολίγοι· παρακαλέσατε λοιπόν τον Κύριον του θερισμού να αποστείλη εργάτας εις τον θερισμόν αυτού.
\par 3 Υπάγετε· ιδού, εγώ σας αποστέλλω ως αρνία εν μέσω λύκων.
\par 4 Μη βαστάζετε βαλάντιον, μη σακκίον, μηδέ υποδήματα, και μηδένα χαιρετήσητε κατά την οδόν.
\par 5 Εις ήντινα δε οικίαν εισέρχησθε, πρώτον λέγετε· Ειρήνη εις τον οίκον τούτον.
\par 6 Και εάν μεν ήναι εκεί υιός ειρήνης, θέλει αναπαυθή επ' αυτόν η ειρήνη σας· ει δε μη, θέλει επιστρέψει εις εσάς.
\par 7 Εν αυτή δε τη οικία μένετε τρώγοντες και πίνοντες τα παρ' αυτών διδόμενα· διότι ο εργάτης είναι άξιος του μισθού αυτού· μη μεταβαίνετε εξ οικίας εις οικίαν.
\par 8 Και εις ήντινα πόλιν εισέρχησθε και σας δέχωνται, τρώγετε τα παρατιθέμενα εις εσάς,
\par 9 και θεραπεύετε τους εν αυτή ασθενείς και λέγετε προς αυτούς· Επλησίασεν εις εσάς η βασιλεία του Θεού.
\par 10 Εις ήντινα όμως πόλιν εισέρχησθε και δεν σας δέχωνται, εξελθόντες εις τας πλατείας αυτής, είπατε·
\par 11 Και τον κονιορτόν, όστις εκολλήθη εις ημάς εκ της πόλεώς σας, εκτινάσσομεν εις εσάς· πλην τούτο γινώσκετε, ότι επλησίασεν εις εσάς η βασιλεία του Θεού.
\par 12 Σας λέγω δε ότι εν τη ημέρα εκείνη ελαφροτέρα θέλει είσθαι η τιμωρία εις τα Σόδομα παρά εις την πόλιν εκείνην.
\par 13 Ουαί εις σε, Χοραζίν, ουαί εις σε, Βηθσαϊδά· διότι εάν εν τη Τύρω και Σιδώνι ήθελον γείνει τα θαύματα τα γενόμενα εν τω μέσω υμών, προ πολλού ήθελον μετανοήσει καθήμεναι εν σάκκω και σποδώ.
\par 14 Πλην εις την Τύρον και Σιδώνα ελαφροτέρα θέλει είσθαι η τιμωρία εν τη κρίσει παρά εις εσάς.
\par 15 Και συ, Καπερναούμ, ήτις υψώθης έως του ουρανού, θέλεις καταβιβασθή έως άδου.
\par 16 Όστις ακούει εσάς εμέ ακούει, και όστις αθετεί εσάς εμέ αθετεί, ο δε αθετών εμέ αθετεί τον αποστείλαντά με.
\par 17 Υπέστρεψαν δε οι εβδομήκοντα μετά χαράς, λέγοντες· Κύριε, και τα δαιμόνια υποτάσσονται εις ημάς εν τω ονόματί σου.
\par 18 Είπε δε προς αυτούς· Εθεώρουν τον Σατανάν ως αστραπήν εκ του ουρανού πεσόντα.
\par 19 Ιδού, δίδω εις εσάς την εξουσίαν του να πατήτε επάνω όφεων και σκορπίων και επί πάσαν την δύναμιν του εχθρού, και ουδέν θέλει σας βλάψει.
\par 20 Πλην εις τούτο μη χαίρετε, ότι τα πνεύματα υποτάσσονται εις εσάς· αλλά χαίρετε μάλλον ότι τα ονόματά σας εγράφησαν εν τοις ουρανοίς.
\par 21 Εν αυτή τη ώρα ηγαλλιάσθη κατά το πνεύμα ο Ιησούς και είπεν· Ευχαριστώ σοι, Πάτερ, Κύριε του ουρανού και της γης, ότι απέκρυψας ταύτα από σοφών και συνετών και απεκάλυψας αυτά εις νήπια· ναι, ω Πάτερ, διότι ούτως έγεινεν αρεστόν έμπροσθέν σου.
\par 22 Πάντα παρεδόθησαν εις εμέ υπό του Πατρός μου· και ουδείς γινώσκει τις είναι ο Υιός, ειμή ο Πατήρ, και τις είναι ο Πατήρ, ειμή ο Υιός και εις όντινα θέλη ο Υιός να αποκαλύψη αυτόν.
\par 23 Και στραφείς προς τους μαθητάς, είπε κατ' ιδίαν· Μακάριοι οι οφθαλμοί οι βλέποντες όσα βλέπετε.
\par 24 Διότι σας λέγω ότι πολλοί προφήται και βασιλείς επεθύμησαν να ίδωσιν όσα σεις βλέπετε, και δεν είδον, και να ακούσωσιν όσα ακούετε, και δεν ήκουσαν.
\par 25 Και ιδού, νομικός τις εσηκώθη πειράζων αυτόν και λέγων· Διδάσκαλε, τι πράξας θέλω κληρονομήσει ζωήν αιώνιον;
\par 26 Ο δε είπε προς αυτόν· Εν τω νόμω τι είναι γεγραμμένον; πως αναγινώσκεις;
\par 27 Ο δε αποκριθείς είπε· Θέλεις αγαπά Κύριον τον Θεόν σου εξ όλης της καρδίας σου και εξ όλης της ψυχής σου και εξ όλης της δυνάμεώς σου και εξ όλης της διανοίας σου, και τον πλησίον σου ως σεαυτόν.
\par 28 Είπε δε προς αυτόν· Ορθώς απεκρίθης· τούτο κάμνε και θέλεις ζήσει.
\par 29 Αλλ' εκείνος, θέλων να δικαιώση εαυτόν, είπε προς τον Ιησούν· Και τις είναι ο πλησίον μου;
\par 30 Και αποκριθείς ο Ιησούς είπεν· Άνθρωπος τις κατέβαινεν από Ιερουσαλήμ εις Ιεριχώ και περιέπεσεν εις ληστάς· οίτινες και γυμνώσαντες αυτόν και καταπληγώσαντες, ανεχώρησαν αφήσαντες αυτόν ημιθανή.
\par 31 Κατά συγκυρίαν δε ιερεύς τις κατέβαινε δι' εκείνης της οδού, και ιδών αυτόν επέρασεν από το άλλο μέρος.
\par 32 Ομοίως δε και Λευΐτης, φθάσας εις τον τόπον, ελθών και ιδών επέρασεν από το άλλο μέρος.
\par 33 Σαμαρείτης δε τις οδοιπορών ήλθεν εις τον τόπον όπου ήτο, και ιδών αυτόν εσπλαγχνίσθη,
\par 34 και πλησιάσας έδεσε τας πληγάς αυτού επιχέων έλαιον και οίνον, και επιβιβάσας αυτόν επί το κτήνος αυτού, έφερεν αυτόν εις ξενοδοχείον και επεμελήθη αυτού·
\par 35 και την επαύριον, ότε εξήρχετο, εκβαλών δύο δηνάρια έδωκεν εις τον ξενοδόχον και είπε προς αυτόν· Επιμελήθητι αυτού, και ό,τι συ δαπανήσης περιπλέον, εγώ όταν επανέλθω θέλω σοι αποδώσει.
\par 36 Τις λοιπόν εκ των τριών τούτων σοι φαίνεται ότι έγεινε πλησίον του εμπεσόντος εις τους ληστάς;
\par 37 Ο δε είπεν· Ο ποιήσας το έλεος εις αυτόν· Είπε λοιπόν προς αυτόν ο Ιησούς· Ύπαγε και συ, κάμνε ομοίως.
\par 38 Ενώ δε απήρχοντο, αυτός εισήλθεν εις κώμην τινά· και γυνή τις ονομαζομένη Μάρθα υπεδέχθη αυτόν εις τον οίκον αυτής.
\par 39 Και αύτη είχεν αδελφήν καλουμένην Μαρίαν, ήτις και καθήσασα παρά τους πόδας του Ιησού, ήκουε τον λόγον αυτού.
\par 40 Η δε Μάρθα ενησχολείτο εις πολλήν υπηρεσίαν· και ελθούσα έμπροσθεν αυτού είπε· Κύριε, δεν σε μέλει ότι η αδελφή μου με αφήκε μόνην να υπηρετώ; είπε λοιπόν προς αυτήν να μοι βοηθήση.
\par 41 Αποκριθείς δε ο Ιησούς, είπε προς αυτήν· Μάρθα, Μάρθα, μεριμνάς και αγωνίζεσαι περί πολλά·
\par 42 πλην ενός είναι χρεία· η Μαρία όμως εξέλεξε την αγαθήν μερίδα, ήτις δεν θέλει αφαιρεθή απ' αυτής.

\chapter{11}

\par 1 Και ενώ αυτός προσηύχετο εν τόπω τινί, καθώς έπαυσεν, είπε τις των μαθητών αυτού προς αυτόν· Κύριε, δίδαξον ημάς να προσευχώμεθα, καθώς και ο Ιωάννης εδίδαξε τους μαθητάς αυτού.
\par 2 Είπε δε προς αυτούς· Όταν προσεύχησθε, λέγετε· Πάτερ ημών ο εν τοις ουρανοίς, αγιασθήτω το όνομά σου, ελθέτω η βασιλεία σου, γενηθήτω το θέλημά σου ως εν ουρανώ, και επί της γής·
\par 3 τον άρτον ημών τον επιούσιον δίδε εις ημάς καθ' ημέραν·
\par 4 και συγχώρησον εις ημάς τας αμαρτίας ημών, διότι και ημείς συγχωρούμεν εις πάντα αμαρτάνοντα εις ημάς· και μη φέρης ημάς εις πειρασμόν, αλλ' ελευθέρωσον ημάς από του πονηρού.
\par 5 Και είπε προς αυτούς· Εάν τις εξ υμών έχη φίλον, και υπάγη προς αυτόν το μεσονύκτιον και είπη προς αυτόν· Φίλε, δάνεισόν μοι τρεις άρτους,
\par 6 επειδή ήλθε φίλος μου προς εμέ εξ οδοιπορίας, και δεν έχω τι να βάλω έμπροσθεν αυτού.
\par 7 Και εκείνος αποκριθείς έσωθεν είπη· Μη με ενόχλει· η θύρα είναι ήδη κεκλεισμένη και τα παιδία μου είναι μετ' εμού εις την κλίνην· δεν δύναμαι να σηκωθώ και να σοι δώσω.
\par 8 Σας λέγω· Και αν δεν σηκωθή και δώση εις αυτόν, διότι είναι φίλος αυτού, τουλάχιστον διά την αναίδειαν αυτού θέλει σηκωθή και δώσει εις αυτόν όσα χρειάζεται.
\par 9 Και εγώ σας λέγω· Αιτείτε και θέλει σας δοθή· ζητείτε και θέλετε ευρεί, κρούετε και θέλει σας ανοιχθή.
\par 10 Διότι πας ο αιτών λαμβάνει, και ο ζητών ευρίσκει, και εις τον κρούοντα θέλει ανοιχθή.
\par 11 Και εάν τις εξ υμών ήναι πατήρ, και ο υιός αυτού ζητήση άρτον, μήπως θέλει δώσει εις αυτόν λίθον; και εάν οψάριον, μήπως αντί οψαρίου θέλει δώσει εις αυτόν όφιν;
\par 12 ή και αν ζητήση ωόν, μήπως θέλει δώσει εις αυτόν σκορπίον;
\par 13 εάν λοιπόν σεις, πονηροί όντες, εξεύρετε να δίδητε καλάς δόσεις εις τα τέκνα σας, πόσω μάλλον ο Πατήρ ο ουράνιος θέλει δώσει Πνεύμα Άγιον εις τους αιτούντας παρ' αυτού;
\par 14 Και εξέβαλλε δαιμόνιον, και αυτό ήτο κωφόν· αφού δε εξήλθε το δαιμόνιον, ελάλησεν ο κωφός, και εθαύμασαν οι όχλοι.
\par 15 Τινές όμως εξ αυτών είπον· Διά του Βεελζεβούλ του άρχοντος των δαιμονίων εκβάλλει τα δαιμόνια.
\par 16 Άλλοι δε πειράζοντες εζήτουν παρ' αυτού σημείον εξ ουρανού.
\par 17 Πλην αυτός νοήσας τους διαλογισμούς αυτών, είπε προς αυτούς· Πάσα βασιλεία διαιρεθείσα καθ' εαυτής ερημούται, και οίκος διαιρεθείς καθ' εαυτού πίπτει.
\par 18 Εάν λοιπόν και ο Σατανάς διηρέθη καθ' εαυτού, πως θέλει σταθή η βασιλεία αυτού, επειδή λέγετε ότι εγώ εκβάλλω τα δαιμόνια διά του Βεελζεβούλ.
\par 19 Αλλ' εάν εγώ διά του Βεελζεβούλ εκβάλλω τα δαιμόνια, οι υιοί σας διά τίνος εκβάλλουσι; διά τούτο αυτοί θέλουσιν είσθαι κριταί σας.
\par 20 Αλλ' εάν διά του δακτύλου του Θεού εκβάλλω τα δαιμόνια, άρα έφθασεν εις εσάς η βασιλεία του Θεού.
\par 21 Όταν ο ισχυρός καθωπλισμένος φυλάττη την εαυτού αυλήν, τα υπάρχοντα αυτού είναι εν ειρήνη·
\par 22 όταν όμως ο ισχυρότερος αυτού επελθών νικήση αυτόν, αφαιρεί την πανοπλίαν αυτού, εις την οποίαν εθάρρει, και διαμοιράζει τα λάφυρα αυτού.
\par 23 Όστις δεν είναι μετ' εμού είναι κατ' εμού, και όστις δεν συνάγει μετ' εμού σκορπίζει.
\par 24 Όταν το ακάθαρτον πνεύμα εξέλθη από του ανθρώπου, διέρχεται δι' ανύδρων τόπων και ζητεί ανάπαυσιν, και μη ευρίσκον λέγει· ας υποστρέψω εις τον οίκον μου όθεν εξήλθον·
\par 25 και ελθόν ευρίσκει αυτόν σεσαρωμένον και εστολισμένον.
\par 26 Τότε υπάγει και παραλαμβάνει επτά άλλα πνεύματα πονηρότερα εαυτού, και εισελθόντα κατοικούσιν εκεί, και γίνονται τα έσχατα του ανθρώπου εκείνου χειρότερα των πρώτων.
\par 27 Και ενώ αυτός έλεγε ταύτα, γυνή τις εκ του όχλου υψώσασα φωνήν, είπε προς αυτόν· Μακαρία η κοιλία ήτις σε εβάστασε, και οι μαστοί, τους οποίους εθήλασας.
\par 28 Αυτός δε είπε· Μακάριοι μάλλον οι ακούοντες τον λόγον του Θεού και φυλάττοντες αυτόν.
\par 29 Και ενώ οι όχλοι συνηθροίζοντο, ήρχισε να λέγη· Η γενεά αύτη είναι πονηρά· σημείον ζητεί, και σημείον δεν θέλει δοθή εις αυτήν ειμή το σημείον Ιωνά του προφήτου.
\par 30 Διότι καθώς ο Ιωνάς έγεινε σημείον εις τους Νινευΐτας, ούτω θέλει είσθαι και ο Υιός του ανθρώπου εις την γενεάν ταύτην.
\par 31 Η βασίλισσα του νότου θέλει σηκωθή εν τη κρίσει μετά των ανθρώπων της γενεάς ταύτης και θέλει κατακρίνει αυτούς, διότι ήλθεν εκ των περάτων της γης διά να ακούση την σοφίαν του Σολομώντος, και ιδού, πλειότερον του Σολομώντος είναι εδώ.
\par 32 Οι άνδρες της Νινευΐ θέλουσιν αναστηθή εν τη κρίσει μετά της γενεάς ταύτης και θέλουσι κατακρίνει αυτήν, διότι μετενόησαν εις το κήρυγμα του Ιωνά, και ιδού, πλειότερον του Ιωνά είναι εδώ.
\par 33 Ουδείς δε λύχνον ανάψας θέτει εις τόπον απόκρυφον ουδέ υπό τον μόδιον, αλλ' επί τον λυχνοστάτην, διά να βλέπωσι το φως οι εισερχόμενοι.
\par 34 Ο λύχνος του σώματος είναι ο οφθαλμός· όταν λοιπόν ο οφθαλμός σου ήναι καθαρός, και όλον το σώμα σου είναι φωτεινόν· αλλ' όταν ήναι πονηρός, και το σώμα σου είναι σκοτεινόν.
\par 35 Πρόσεχε λοιπόν μήποτε το φως το εν σοι ήναι σκότος.
\par 36 Εάν λοιπόν όλον το σώμα σου ήναι φωτεινόν, μη έχον τι μέρος σκοτεινόν, θέλει είσθαι φωτεινόν όλον, καθώς όταν ο λύχνος σε φωτίζη διά της λάμψεως.
\par 37 Και αφού ελάλησε ταύτα, Φαρισαίός τις παρεκάλει αυτόν να γευματίση εν τω οίκω αυτού· εισελθών δε εκάθησεν εις την τράπεζαν.
\par 38 Ο δε Φαρισαίος ιδών εθαύμασεν ότι δεν ενίφθη πρώτον πριν του γεύματος.
\par 39 Και ο Κύριος είπε προς αυτόν· Τώρα σεις οι Φαρισαίοι το έξωθεν του ποτηρίου και του πινακίου καθαρίζετε, το δε εσωτερικόν σας γέμει αρπαγής και πονηρίας.
\par 40 Άφρονες, εκείνος όστις έκαμε το έξωθεν δεν έκαμε και το έσωθεν;
\par 41 Πλην δότε ελεημοσύνην τα υπάρχοντα υμών, και ιδού, τα πάντα είναι καθαρά εις εσάς.
\par 42 Αλλ' ουαί εις εσάς τους Φαρισαίους, διότι αποδεκατίζετε το ηδύοσμον και το πήγανον και παν λάχανον, και παραβλέπετε την κρίσιν και την αγάπην του Θεού· ταύτα έπρεπε να κάμητε και εκείνα να μη αφήσητε.
\par 43 Ουαί εις εσάς τους Φαρισαίους, διότι αγαπάτε την πρωτοκαθεδρίαν εν ταις συναγωγαίς και τους ασπασμούς εν ταις αγοραίς.
\par 44 Ουαί εις εσάς, γραμματείς και Φαρισαίοι, υποκριταί, διότι είσθε ως τα μνημεία, τα οποία δεν φαίνονται, και οι άνθρωποι οι περιπατούντες επάνω δεν γνωρίζουσιν.
\par 45 Αποκριθείς δε τις των νομικών, λέγει προς αυτόν· Διδάσκαλε, ταύτα λέγων και ημάς υβρίζεις.
\par 46 Ο δε είπε· Και εις εσάς τους νομικούς ουαί, διότι φορτίζετε τους ανθρώπους φορτία δυσβάστακτα, και σεις με ένα των δακτύλων σας δεν εγγίζετε τα φορτία.
\par 47 Ουαί εις εσάς, διότι οικοδομείτε τα μνημεία των προφητών, οι δε πατέρες σας εφόνευσαν αυτούς.
\par 48 Άρα μαρτυρείτε και συμφωνείτε εις τα έργα των πατέρων σας, διότι αυτοί μεν εφόνευσαν αυτούς, σεις δε οικοδομείτε τα μνημεία αυτών.
\par 49 Διά τούτο και η σοφία του Θεού είπε· Θέλω αποστείλει εις αυτούς προφήτας και αποστόλους, και εξ αυτών θέλουσι φονεύσει και εκδιώξει,
\par 50 διά να εκζητηθή το αίμα πάντων των προφητών, το εκχυνόμενον από της αρχής του κόσμου, από της γενεάς ταύτης,
\par 51 από του αίματος του Άβελ έως του αίματος Ζαχαρίου του φονευθέντος μεταξύ του θυσιαστηρίου και του ναού· ναι, σας λέγω, θέλει εκζητηθή από της γενεάς ταύτης.
\par 52 Ουαί εις εσάς τους νομικούς, διότι αφηρέσατε το κλειδίον της γνώσεως· σεις δεν εισήλθετε και τους εισερχομένους ημποδίσατε.
\par 53 Ενώ δε αυτός έλεγε ταύτα προς αυτούς, ήρχισαν οι γραμματείς και οι Φαρισαίοι να διεγείρωσιν αυτόν σφόδρα και να βιάζωσιν αυτόν να ομιλήση, ερωτώντες περί πολλών,
\par 54 ενεδρεύοντες αυτόν και ζητούντες να αρπάσωσί τι από του στόματος αυτού, διά να κατηγορήσωσιν αυτόν.

\chapter{12}

\par 1 Εν τω μεταξύ αφού συνηθροίσθησαν αι μυριάδες του όχλου, ώστε κατεπάτουν αλλήλους, ήρχισε να λέγη προς τους μαθητάς αυτού πρώτον· Προσέχετε εις εαυτούς από της ζύμης των Φαρισαίων, ήτις είναι υπόκρισις.
\par 2 Αλλά δεν είναι ουδέν κεκαλυμμένον, το οποίον δεν θέλει ανακαλυφθή, και κρυπτόν, το οποίον δεν θέλει γνωρισθή·
\par 3 όθεν όσα είπετε εν τω σκότει εν τω φωτί θέλουσιν ακουσθή, και ό,τι ελαλήσατε προς το ωτίον εν τοις ταμείοις θέλει κηρυχθή επί των δωμάτων.
\par 4 Λέγω δε προς εσάς τους φίλους μου· Μη φοβηθήτε από των αποκτεινόντων το σώμα και μετά ταύτα μη δυναμένων περισσότερόν τι να πράξωσι.
\par 5 Θέλω δε σας δείξει ποίον να φοβηθήτε· Φοβήθητε εκείνον, όστις αφού αποκτείνη, έχει εξουσίαν να ρίψη εις την γέενναν· ναι, σας λέγω, τούτον φοβήθητε.
\par 6 Δεν πωλούνται πέντε στρουθία διά δύο ασσάρια; και εν εξ αυτών δεν είναι λελησμονημένον ενώπιον του Θεού·
\par 7 αλλά και αι τρίχες της κεφαλής υμών είναι πάσαι ηριθμημέναι. Μη φοβείσθε λοιπόν· από πολλών στρουθίων διαφέρετε.
\par 8 Σας λέγω δέ· Πας όστις με ομολογήση έμπροσθεν των ανθρώπων, και ο Υιός του ανθρώπου θέλει ομολογήσει αυτόν έμπροσθεν των αγγέλων του Θεού·
\par 9 όστις δε με αρνηθή ενώπιον των ανθρώπων, και ο Υιός του ανθρώπου θέλει αρνηθή αυτόν ενώπιον των αγγέλων του Θεού.
\par 10 Και πας όστις θέλει ειπεί λόγον κατά του Υιού του ανθρώπου, θέλει συγχωρηθή εις αυτόν· όστις όμως βλασφημήση κατά του Αγίου Πνεύματος, εις αυτόν δεν θέλει συγχωρηθή.
\par 11 Όταν δε σας φέρωσιν εις τας συναγωγάς και τας αρχάς και τας εξουσίας, μη μεριμνάτε πως ή τι να απολογηθήτε, ή τι να είπητε·
\par 12 διότι το Άγιον Πνεύμα θέλει σας διδάξει εν αυτή τη ώρα τι πρέπει να είπητε.
\par 13 Είπε δε τις προς αυτόν εκ του όχλου· Διδάσκαλε, ειπέ προς τον αδελφόν μου να μοιρασθή μετ' εμού την κληρονομίαν.
\par 14 Ο δε είπε προς αυτόν· Άνθρωπε, τις με κατέστησε δικαστήν ή μεριστήν εφ' υμάς;
\par 15 Και είπε προς αυτούς· Προσέχετε και φυλάττεσθε από της πλεονεξίας· διότι εάν τις έχη περισσά, η ζωή αυτού δεν συνίσταται εκ των υπαρχόντων αυτού.
\par 16 Είπε δε προς αυτούς παραβολήν, λέγων· Ανθρώπου τινός πλουσίου ηυτύχησαν τα χωράφια·
\par 17 Και διελογίζετο εν εαυτώ λέγων· Τι να κάμω, διότι δεν έχω που να συνάξω τους καρπούς μου;
\par 18 Και είπε· Τούτο θέλω κάμει· θέλω χαλάσει τας αποθήκας μου και θέλω οικοδομήσει μεγαλητέρας και συνάξει εκεί πάντα τα γεννήματά μου και τα αγαθά μου,
\par 19 και θέλω ειπεί προς την ψυχήν μου· Ψυχή, έχεις πολλά αγαθά εναποτεταμιευμένα δι' έτη πολλά· αναπαύου, φάγε, πίε, ευφραίνου.
\par 20 Είπε δε προς αυτόν ο Θεός· Άφρον, ταύτην την νύκτα την ψυχήν σου απαιτούσιν από σού· όσα δε ητοίμασας, τίνος θέλουσιν είσθαι;
\par 21 Ούτω θέλει είσθαι όστις θησαυρίζει εις εαυτόν και δεν πλουτεί εις Θεόν.
\par 22 Είπε δε προς τους μαθητάς αυτού· Διά τούτο λέγω προς εσάς, Μη μεριμνάτε διά την ζωήν σας, τι να φάγητε, μηδέ διά το σώμα, τι να ενδυθήτε.
\par 23 Η ζωή είναι τιμιώτερον της τροφής και το σώμα του ενδύματος.
\par 24 Παρατηρήσατε τους κόρακας, ότι δεν σπείρουσιν ουδέ θερίζουσιν, οίτινες δεν έχουσι ταμείον ουδέ αποθήκην, και ο Θεός τρέφει αυτούς· πόσω μάλλον σεις διαφέρετε των πτηνών.
\par 25 Και τις εξ υμών μεριμνών δύναται να προσθέση εις το ανάστημα αυτού μίαν πήχυν;
\par 26 Εάν λοιπόν ουδέ το ελάχιστον δύνασθε, τι μεριμνάτε περί των λοιπών;
\par 27 Παρατηρήσατε τα κρίνα πως αυξάνουσι· δεν κοπιάζουσιν ουδέ κλώθουσι· σας λέγω όμως, ουδέ ο Σολομών εν πάση τη δόξη αυτού ενεδύθη ως εν τούτων.
\par 28 Αλλ' εάν τον χόρτον, όστις σήμερον είναι εν τω αγρώ και αύριον ρίπτεται εις κλίβανον, ο Θεός ενδύη ούτω, πόσω μάλλον εσάς, ολιγόπιστοι.
\par 29 Και σεις μη ζητείτε τι να φάγητε ή τι να πίητε, και μη ήσθε μετέωροι·
\par 30 διότι ταύτα πάντα ζητούσι τα έθνη του κόσμου· υμών δε ο Πατήρ εξεύρει ότι έχετε χρείαν τούτων·
\par 31 πλην ζητείτε την βασιλείαν του Θεού, και ταύτα πάντα θέλουσι σας προστεθή.
\par 32 Μη φοβού, μικρόν ποίμνιον· διότι ο Πατήρ σας ηυδόκησε να σας δώση την βασιλείαν.
\par 33 Πωλήσατε τα υπάρχοντά σας και δότε ελεημοσύνην. Κάμετε εις εαυτούς βαλάντια τα οποία δεν παλαιούνται, θησαυρόν εν τοις ουρανοίς όστις δεν εκλείπει, όπου κλέπτης δεν πλησιάζει ουδέ ο σκώληξ διαφθείρει·
\par 34 διότι όπου είναι ο θησαυρός σας, εκεί θέλει είσθαι και η καρδία σας.
\par 35 Ας ήναι αι οσφύες σας περιεζωσμέναι και οι λύχνοι καιόμενοι·
\par 36 και σεις όμοιοι με ανθρώπους, οίτινες προσμένουσι τον κύριον αυτών, πότε θέλει επιστρέψει εκ των γάμων, διά να ανοίξωσιν ευθύς εις αυτόν όταν έλθη και κρούση.
\par 37 Μακάριοι οι δούλοι εκείνοι, τους οποίους ελθών ο κύριος θέλει ευρεί αγρυπνούντας. Αληθώς σας λέγω, ότι θέλει περιζωσθή και καθίσει αυτούς εις την τράπεζαν, και ελθών εις το μέσον θέλει υπηρετήσει αυτούς.
\par 38 Και εάν έλθη εν τη δευτέρα φυλακή και εν τη τρίτη φυλακή έλθη και εύρη ούτω, μακάριοι είναι οι δούλοι εκείνοι.
\par 39 Τούτο δε γινώσκετε, ότι εάν ήξευρεν ο οικοδεσπότης ποίαν ώραν ο κλέπτης έρχεται, ήθελεν αγρυπνήσει και δεν ήθελεν αφήσει να διορυχθή ο οίκος αυτού.
\par 40 Και σεις λοιπόν γίνεσθε έτοιμοι· διότι καθ' ην ώραν δεν στοχάζεσθε, έρχεται ο Υιός του ανθρώπου.
\par 41 Είπε δε προς αυτόν ο Πέτρος· Κύριε, προς ημάς λέγεις την παραβολήν ταύτην ή και προς πάντας;
\par 42 Και ο Κύριος είπε· Τις λοιπόν είναι ο πιστός οικονόμος και φρόνιμος, τον οποίον θέλει καταστήσει ο κύριος αυτού επί των υπηρετών αυτού, διά να δίδη εν καιρώ την διωρισμένην τροφήν;
\par 43 Μακάριος ο δούλος εκείνος, τον οποίον ελθών ο κύριος αυτού θέλει ευρεί πράττοντα ούτως.
\par 44 Αληθώς σας λέγω, ότι θέλει καταστήσει αυτόν επί πάντων των υπαρχόντων αυτού.
\par 45 Εάν δε είπη ο δούλος εκείνος εν τη καρδία αυτού, Βραδύνει να έλθη ο κύριός μου· και αρχίση να δέρη τους δούλους και τας δούλας, και να τρώγη και να πίνη και να μεθύη,
\par 46 θέλει ελθεί ο κύριος του δούλου εκείνου, καθ' ην ημέραν δεν προσμένει και καθ' ην ώραν δεν εξεύρει, και θέλει αποχωρίσει αυτόν, και το μέρος αυτού θέλει θέσει μετά των απίστων.
\par 47 Εκείνος δε ο δούλος, όστις γνωρίσας το θέλημα του κυρίου αυτού δεν ητοίμασεν ουδέ έκαμε κατά το θέλημα αυτού, θέλει δαρθή πολύ·
\par 48 όστις όμως μη γνωρίσας έπραξεν άξια δαρμών, θέλει δαρθή ολίγον· εις πάντα δε, εις τον οποίον εδόθη πολύ, πολύ θέλει ζητηθή παρ' αυτού, και εις όντινα ενεπιστεύθη πολύ, περισσότερον θέλουσιν απαιτήσει παρ' αυτού.
\par 49 Πυρ ήλθον να βάλω εις την γην, και τι θέλω, εάν ήδη ανήφθη;
\par 50 Βάπτισμα δε έχω να βαπτισθώ, και πως στενοχωρούμαι εωσού εκτελεσθή.
\par 51 Νομίζετε ότι ήλθον να δώσω ειρήνην εν τη γη; ουχί, σας λέγω, αλλά διαχωρισμόν.
\par 52 Διότι από του νυν θέλουσιν είσθαι πέντε εν οίκω ενί διακεχωρισμένοι, οι τρεις κατά των δύο και οι δύο κατά των τριών·
\par 53 Θέλει διαχωρισθή πατήρ κατά υιού και υιός κατά πατρός, μήτηρ κατά θυγατρός και θυγάτηρ κατά μητρός, πενθερά κατά της νύμφης αυτής και νύμφη κατά της πενθεράς αυτής.
\par 54 Έλεγε και προς τους όχλους· Όταν ίδητε την νεφέλην ανυψουμένην από δυσμών, ευθύς λέγετε, Βροχή έρχεται, και γίνεται ούτω·
\par 55 και όταν νότον πνέοντα, λέγετε ότι καύσων θέλει είσθαι, και γίνεται.
\par 56 Υποκριταί, το πρόσωπον της γης και του ουρανού εξεύρετε να διακρίνητε, τον δε καιρόν τούτον πως δεν διακρίνετε;
\par 57 Διά τι δε και αφ' εαυτών δεν κρίνετε το δίκαιον;
\par 58 Ενώ λοιπόν υπάγεις μετά του αντιδίκου σου προς τον άρχοντα, προσπάθησον καθ' οδόν να απαλλαχθής απ' αυτού, μήποτε σε σύρη προς τον κριτήν, και ο κριτής σε παραδώση εις τον υπηρέτην, και ο υπηρέτης σε βάλη εις φυλακήν.
\par 59 Σοι λέγω, δεν θέλεις εξέλθει εκείθεν, εωσού αποδώσης και το έσχατον λεπτόν.

\chapter{13}

\par 1 Κατ' εκείνον δε τον καιρόν ήλθον τινές, απαγγέλλοντες προς αυτόν περί των Γαλιλαίων, των οποίων το αίμα ο Πιλάτος έμιξε με τας θυσίας αυτών.
\par 2 Και αποκριθείς ο Ιησούς, είπε προς αυτούς· Νομίζετε ότι οι Γαλιλαίοι ούτοι ήσαν αμαρτωλοί υπέρ πάντας τους Γαλιλαίους, διότι έπαθον τοιαύτα;
\par 3 Ουχί, σας λέγω, αλλ' εάν δεν μετανοήτε, πάντες ομοίως θέλετε απολεσθή.
\par 4 Η εκείνοι οι δεκαοκτώ, επί τους οποίους έπεσεν ο πύργος εν τω Σιλωάμ και εθανάτωσεν αυτούς, νομίζετε ότι ούτοι ήσαν αμαρτωλοί υπέρ πάντας τους ανθρώπους τους κατοικούντας εν Ιερουσαλήμ;
\par 5 Ουχί, σας λέγω, αλλ' εάν δεν μετανοήτε, πάντες ομοίως θέλετε απολεσθή.
\par 6 Έλεγε δε ταύτην την παραβολήν· Είχε τις συκήν πεφυτευμένην εν τω αμπελώνι αυτού, και ήλθε ζητών καρπόν εν αυτή και δεν εύρε.
\par 7 Και είπε προς τον αμπελουργόν· Ιδού, τρία έτη έρχομαι ζητών καρπόν εν τη συκή ταύτη και δεν ευρίσκω· έκκοψον αυτήν· διά τι καταργεί και την γην;
\par 8 Ο δε αποκριθείς λέγει προς αυτόν· Κύριε, άφες αυτήν και τούτο το έτος, έως ότου σκάψω περί αυτήν και βάλω κοπρίαν·
\par 9 και εάν μεν κάμη καρπόν, καλώς· ει δε μη, θέλεις εκκόψει αυτήν μετά ταύτα.
\par 10 Εδίδασκε δε εν μιά των συναγωγών το σάββατον.
\par 11 Και ιδού, γυνή τις είχε πνεύμα ασθενείας δεκαοκτώ έτη και ήτο συγκύπτουσα και δεν ηδύνατο παντελώς να ανακύψη.
\par 12 Ιδών δε αυτήν ο Ιησούς, εφώναξε και είπε προς αυτήν· Γύναι, ηλευθερωμένη είσαι από της ασθενείας σου·
\par 13 και έθεσεν επ' αυτήν τας χείρας· και παρευθύς ανωρθώθη και εδόξαζε τον Θεόν.
\par 14 Αποκριθείς δε ο αρχισυνάγωγος, αγανακτών ότι εις το σάββατον εθεράπευσεν ο Ιησούς, έλεγε προς τον όχλον· Εξ ημέραι είναι, εις τας οποίας πρέπει να εργάζησθε· εν ταύταις λοιπόν ερχόμενοι θεραπεύεσθε, και μη τη ημέρα του σαββάτου.
\par 15 Απεκρίθη λοιπόν προς αυτόν ο Κύριος και είπεν· Υποκριτά, δεν λύει έκαστος υμών εν τω σαββάτω τον βουν αυτού ή τον όνον από της φάτνης και φέρων ποτίζει;
\par 16 αύτη δε, ούσα θυγάτηρ του Αβραάμ, την οποίαν ο Σατανάς έδεσεν, ιδού, δεκαοκτώ έτη, δεν έπρεπε να λυθή από του δεσμού τούτου τη ημέρα του σαββάτου;
\par 17 Και ενώ, αυτός έλεγε ταύτα, κατησχύνοντο πάντες οι εναντίοι αυτού, και πας ο όχλος έχαιρε δι' όλα τα ένδοξα έργα τα γινόμενα υπ' αυτού.
\par 18 Έλεγε δέ· Με τι είναι ομοία η βασιλεία του Θεού, και με τι να ομοιώσω αυτήν;
\par 19 Είναι ομοία με κόκκον σινάπεως, τον οποίον λαβών άνθρωπος έρριψεν εις τον κήπον αυτού· και ηύξησε και έγεινε δένδρον μέγα, και τα πετεινά του ουρανού κατεσκήνωσαν εν τοις κλάδοις αυτού.
\par 20 Και πάλιν είπε· Με τι να ομοιώσω την βασιλείαν του Θεού;
\par 21 Είναι ομοία με προζύμιον, το οποίον λαβούσα γυνή ενέκρυψεν εις τρία μέτρα αλεύρου, εωσού ανέβη όλον το φύραμα.
\par 22 Και διήρχετο τας πόλεις και κώμας διδάσκων και οδοιπορών εις Ιερουσαλήμ.
\par 23 Είπε δε τις προς αυτόν· Κύριε, ολίγοι άρα είναι οι σωζόμενοι; Ο δε είπε προς αυτούς·
\par 24 Αγωνίζεσθε να εισέλθητε διά της στενής πύλης· διότι πολλοί, σας λέγω, θέλουσι ζητήσει να εισέλθωσι και δεν θέλουσι δυνηθή.
\par 25 Αφού σηκωθή ο οικοδεσπότης και αποκλείση την θύραν, και αρχίσητε να στέκησθε έξω και να κρούητε την θύραν, λέγοντες· Κύριε, Κύριε, άνοιξον εις ημάς· και εκείνος αποκριθείς σας είπη, δεν σας εξεύρω πόθεν είσθε·
\par 26 τότε θέλετε αρχίσει να λέγητε· Εφάγομεν έμπροσθέν σου και επίομεν, και εν ταις πλατείαις ημών εδίδαξας.
\par 27 Και θέλει ειπεί· Σας λέγω, δεν σας εξεύρω πόθεν είσθε· φύγετε απ' εμού πάντες οι εργάται της αδικίας.
\par 28 Εκεί θέλει είσθαι ο κλαυθμός και ο τριγμός των οδόντων, όταν ίδητε τον Αβραάμ και Ισαάκ και Ιακώβ και πάντας τους προφήτας εν τη βασιλεία του Θεού, εαυτούς δε εκβαλλομένους έξω.
\par 29 Και θέλουσιν ελθεί από ανατολών και δυσμών και από βορρά και νότου και θέλουσι καθήσει εν τη βασιλεία του Θεού.
\par 30 Και ιδού, είναι έσχατοι, οίτινες θέλουσιν είσθαι πρώτοι, και είναι πρώτοι, οίτινες θέλουσιν είσθαι έσχατοι.
\par 31 Κατ' εκείνην την ημέραν προσήλθον τινές Φαρισαίοι, λέγοντες προς αυτόν· Έξελθε και αναχώρησον εντεύθεν, διότι ο Ηρώδης θέλει να σε θανατώση.
\par 32 Και είπε προς αυτούς· Υπάγετε και είπατε προς την αλώπεκα ταύτην· Ιδού, εκβάλλω δαιμόνια και κάμνω θεραπείας σήμερον και αύριον, και την τρίτην ημέραν τελειούμαι.
\par 33 Πλην πρέπει εγώ σήμερον και αύριον και την εφεξής ημέραν να υπάγω· διότι δεν είναι δυνατόν προφήτης να απολεσθή έξω της Ιερουσαλήμ.
\par 34 Ιερουσαλήμ, Ιερουσαλήμ, η φονεύουσα τους προφήτας και λιθοβολούσα τους απεσταλμένους προς αυτήν, ποσάκις ηθέλησα να συνάξω τα τέκνα σου καθ' ον τρόπον η όρνις τα ορνίθια εαυτής υπό τας πτέρυγας, και δεν ηθελήσατε.
\par 35 Ιδού, σας αφίνεται ο οίκός σας έρημος· αληθώς δε σας λέγω ότι δεν θέλετε με ιδεί, εωσού έλθη ο καιρός ότε θέλετε ειπεί· Ευλογημένος ο ερχόμενος εν ονόματι Κυρίου.

\chapter{14}

\par 1 Και ότε ήλθεν αυτός εις τον οίκον τινός των αρχόντων των Φαρισαίων το σάββατον διά να φάγη άρτον, εκείνοι παρετήρουν αυτόν.
\par 2 Και ιδού, άνθρωπός τις υδρωπικός ήτο έμπροσθεν αυτού.
\par 3 Και αποκριθείς ο Ιησούς, είπε προς τους νομικούς και Φαρισαίους, λέγων· Είναι τάχα συγκεχωρημένον να θεραπεύη τις εν τω σαββάτω;
\par 4 Οι δε εσιώπησαν. Και πιάσας ιάτρευσεν αυτόν και απέλυσε.
\par 5 Και αποκριθείς προς αυτούς είπε· Τίνος υμών ο όνος ή ο βους θέλει πέσει εις φρέαρ, και δεν θέλει ευθύς ανασύρει αυτόν εν τη ημέρα του σαββάτου;
\par 6 Και δεν ηδυνήθησαν να αποκριθώσιν εις αυτόν προς ταύτα.
\par 7 Είπε δε παραβολήν προς τους κεκλημένους, επειδή παρετήρει πως εξέλεγον τας πρωτοκαθεδρίας, λέγων προς αυτούς.
\par 8 Όταν προσκληθής υπό τινός εις γάμους, μη καθήσης εις τον πρώτον τόπον, μήποτε είναι προσκεκλημένος υπ' αυτού εντιμότερός σου,
\par 9 και ελθών εκείνος, όστις εκάλεσε σε και αυτόν, σοι είπη· Δος τόπον εις τούτον· και τότε αρχίσης με αισχύνην να λαμβάνης τον έσχατον τόπον.
\par 10 Αλλ' όταν προσκληθής, ύπαγε και κάθησον εις τον έσχατον τόπον, διά να σοι είπη όταν έλθη εκείνος, όστις σε εκάλεσε· Φίλε, ανάβα ανωτέρω· τότε θέλεις έχει δόξαν ενώπιον των συγκαθημένων μετά σου.
\par 11 Διότι πας ο υψών εαυτόν θέλει ταπεινωθή και ο ταπεινών εαυτόν θέλει υψωθή.
\par 12 Έλεγε δε και προς εκείνον, όστις προσεκάλεσεν αυτόν. Όταν κάμνης γεύμα ή δείπνον, μη προσκάλει τους φίλους σου μηδέ τους αδελφούς σου μηδέ τους συγγενείς σου μηδέ γείτονας πλουσίους, μήποτε και αυτοί σε αντικαλέσωσι, και γείνη εις σε ανταπόδοσις.
\par 13 Αλλ' όταν κάμνης υποδοχήν, προσκάλει πτωχούς, βεβλαμμένους, χωλούς, τυφλούς,
\par 14 και θέλεις είσθαι μακάριος, διότι δεν έχουσι να σοι ανταποδώσωσιν· επειδή η ανταπόδοσις θέλει γείνει εις σε εν τη αναστάσει των δικαίων.
\par 15 Ακούσας δε ταύτα εις των συγκαθημένων, είπε προς αυτόν· Μακάριος όστις φάγη άρτον εν τη βασιλεία του Θεού.
\par 16 Ο δε είπε προς αυτόν· Άνθρωπος τις έκαμε δείπνον μέγα και εκάλεσε πολλούς·
\par 17 και απέστειλε τον δούλον αυτού τη ώρα του δείπνου διά να είπη προς τους κεκλημένους· Έρχεσθε, επειδή πάντα είναι ήδη έτοιμα.
\par 18 Και ήρχισαν πάντες με μίαν γνώμην να παραιτώνται. Ο πρώτος είπε προς αυτόν· Αγρόν ηγόρασα, και έχω ανάγκην να εξέλθω και να ίδω αυτόν· παρακαλώ σε, έχε με παρητημένον.
\par 19 Και άλλος είπεν· Ηγόρασα πέντε ζεύγη βοών, και υπάγω να δοκιμάσω αυτά· παρακαλώ σε, έχε με παρητημένον.
\par 20 και άλλος είπε· Γυναίκα ενυμφεύθην, και διά τούτο δεν δύναμαι να έλθω.
\par 21 Και ελθών ο δούλος εκείνος, απήγγειλε προς τον κύριον αυτού ταύτα. Τότε οργισθείς ο οικοδεσπότης, είπε προς τον δούλον αυτού· Έξελθε ταχέως εις τας πλατείας και τας οδούς της πόλεως, και εισάγαγε εδώ τους πτωχούς και βεβλαμμένους και χωλούς και τυφλούς.
\par 22 Και είπεν ο δούλος· Κύριε, έγεινεν ως προσέταξας, και είναι έτι τόπος.
\par 23 Και είπεν ο κύριος προς τον δούλον· Έξελθε εις τας οδούς και φραγμούς και ανάγκασον να εισέλθωσι, διά να γεμισθή ο οίκός μου.
\par 24 Διότι σας λέγω ότι ουδείς των ανδρών εκείνων των κεκλημένων θέλει γευθή του δείπνου μου.
\par 25 Ήρχοντο δε μετ' αυτού όχλοι πολλοί. Και στραφείς είπε προς αυτούς·
\par 26 Εάν τις έρχηται προς εμέ και δεν μισή τον πατέρα αυτού και την μητέρα και την γυναίκα και τα τέκνα και τους αδελφούς και τας αδελφάς, έτι δε και την εαυτού ζωήν, δεν δύναται να ήναι μαθητής μου.
\par 27 Και όστις δεν βαστάζει τον σταυρόν αυτού και έρχεται οπίσω μου, δεν δύναται να ήναι μαθητής μου.
\par 28 Διότι τις εξ υμών, θέλων να οικοδομήση πύργον, δεν κάθηται πρώτον και λογαριάζει την δαπάνην, αν έχη τα αναγκαία διά να τελειώση αυτόν;
\par 29 μήποτε αφού βάλη θεμέλιον και δεν δύναται να τελειώση αυτόν, αρχίσωσι πάντες οι βλέποντες να εμπαίζωσιν αυτόν,
\par 30 λέγοντες· Ότι ούτος ο άνθρωπος ήρχισε να οικοδομή και δεν ηδυνήθη να τελειώση.
\par 31 Η τις βασιλεύς υπάγων να πολεμήση άλλον βασιλέα δεν κάθηται πρότερον και σκέπτεται εάν ήναι δυνατός με δέκα χιλιάδας να απαντήση τον ερχόμενον κατ' αυτού με είκοσι χιλιάδας;
\par 32 Ει δε μη, ενώ αυτός είναι έτι μακράν, αποστέλλει πρέσβεις και ζητεί ειρήνην.
\par 33 Ούτω λοιπόν πας όστις εξ υμών δεν απαρνείται πάντα τα εαυτού υπάρχοντα, δεν δύναται να ήναι μαθητής μου.
\par 34 Καλόν το άλας· αλλ' εάν το άλας διαφθαρή, με τι θέλει αρτυθή;
\par 35 δεν είναι πλέον χρήσιμον ούτε διά την γην ούτε διά την κοπρίαν· έξω ρίπτουσιν αυτό. Ο έχων ώτα διά να ακούη ας ακούη.

\chapter{15}

\par 1 Επλησίαζον δε εις αυτόν πάντες οι τελώναι και οι αμαρτωλοί, διά να ακούωσιν αυτόν.
\par 2 Και διεγόγγυζον οι Φαρισαίοι και οι γραμματείς, λέγοντες ότι ούτος αμαρτωλούς δέχεται και συντρώγει μετ' αυτών.
\par 3 Είπε δε προς αυτούς την παραβολήν ταύτην, λέγων·
\par 4 Τις άνθρωπος εξ υμών εάν έχη εκατόν πρόβατα και χάση εν εξ αυτών, δεν αφίνει τα ενενήκοντα εννέα εν τη ερήμω και υπάγει ζητών το απολωλός, εωσού εύρη αυτό;
\par 5 Και ευρών αυτό, βάλλει επί τους ώμους αυτού χαίρων.
\par 6 Και ελθών εις τον οίκον, συγκαλεί τους φίλους και τους γείτονας, λέγων προς αυτούς· Συγχάρητέ μοι, διότι εύρον το πρόβατόν μου το απολωλός.
\par 7 Σας λέγω ότι ούτω θέλει είσθαι χαρά εν τω ουρανώ διά ένα αμαρτωλόν μετανοούντα μάλλον παρά διά ενενήκοντα εννέα δικαίους, οίτινες δεν έχουσι χρείαν μετανοίας.
\par 8 Η τις γυνή έχουσα δέκα δραχμάς, εάν χάση δραχμήν μίαν, δεν ανάπτει λύχνον και σαρόνει την οικίαν και ζητεί επιμελώς, έως ότου εύρη αυτήν;
\par 9 και αφού εύρη, συγκαλεί τας φίλας και τας γείτονας, λέγουσα· Συγχάρητέ μοι, διότι εύρον την δραχμήν την οποίαν έχασα.
\par 10 Ούτω, σας λέγω, χαρά γίνεται ενώπιον των αγγέλων του Θεού διά ένα αμαρτωλόν μετανοούντα.
\par 11 Είπε δέ· Άνθρωπος τις είχε δύο υιούς.
\par 12 Και είπεν ο νεώτερος αυτών προς τον πατέρα· Πάτερ, δος μοι το ανήκον μέρος της περιουσίας. Και διεμοίρασεν εις αυτούς τα υπάρχοντα αυτού.
\par 13 Και μετ' ολίγας ημέρας συνάξας πάντα ο νεώτερος υιός, απεδήμησεν εις χώραν μακράν και εκεί διεσκόρπισε την περιουσίαν αυτού ζων ασώτως.
\par 14 Αφού δε εδαπάνησε πάντα, έγεινε πείνα μεγάλη εν τη χώρα εκείνη, και αυτός ήρχισε να στερήται.
\par 15 Τότε υπήγε και προσεκολλήθη εις ένα των πολιτών της χώρας εκείνης, όστις έπεμψεν αυτόν εις τους αγρούς αυτού διά να βόσκη χοίρους.
\par 16 Και επεθύμει να γεμίση την κοιλίαν αυτού από των ξυλοκεράτων, τα οποία έτρωγον οι χοίροι, και ουδείς έδιδεν εις αυτόν.
\par 17 Ελθών δε εις εαυτόν, είπε· Πόσοι μισθωτοί του πατρός μου περισσεύουσιν άρτον, και εγώ χάνομαι υπό της πείνης.
\par 18 Σηκωθείς θέλω υπάγει προς τον πατέρα μου και θέλω ειπεί προς αυτόν· Πάτερ, ήμαρτον εις τον ουρανόν και ενώπιόν σου·
\par 19 και δεν είμαι πλέον άξιος να ονομασθώ υιός σου· κάμε με ως ένα των μισθωτών σου.
\par 20 Και σηκωθείς ήλθε προς τον πατέρα αυτού. Ενώ, δε απείχεν έτι μακράν, είδεν αυτόν ο πατήρ αυτού και εσπλαγχνίσθη, και δραμών επέπεσεν επί τον τράχηλον αυτού και κατεφίλησεν αυτόν.
\par 21 είπε δε προς αυτόν ο υιός· Πάτερ, ήμαρτον εις τον ουρανόν και ενώπιόν σου, και δεν είμαι πλέον άξιος να ονομασθώ υιός σου.
\par 22 Και ο πατήρ είπε προς τους δούλους αυτού· Φέρετε έξω την στολήν την πρώτην και ενδύσατε αυτόν, και δότε δακτυλίδιον εις την χείρα αυτού και υποδήματα εις τους πόδας,
\par 23 και φέροντες τον μόσχον τον σιτευτόν σφάξατε, και φαγόντες ας ευφρανθώμεν,
\par 24 διότι ούτος ο υιός μου νεκρός ήτο και ανέζησε, και απολωλώς ήτο και ευρέθη. Και ήρχισαν να ευφραίνωνται.
\par 25 Ήτο δε ο πρεσβύτερος αυτού υιός εν τω αγρώ· και καθώς ερχόμενος επλησίασεν εις την οικίαν, ήκουσε συμφωνίαν και χορούς,
\par 26 και προσκαλέσας ένα των δούλων, ηρώτα τι είναι ταύτα.
\par 27 Ο δε είπε προς αυτόν ότι ο αδελφός σου ήλθε· και έσφαξεν ο πατήρ σου τον μόσχον τον σιτευτόν, διότι απήλαυσεν αυτόν υγιαίνοντα.
\par 28 Και ωργίσθη και δεν ήθελε να εισέλθη. Εξήλθε λοιπόν ο πατήρ αυτού και παρεκάλει αυτόν.
\par 29 Ο δε αποκριθείς είπε προς τον πατέρα· Ιδού, τόσα έτη σε δουλεύω, και ποτέ εντολήν σου δεν παρέβην, και εις εμέ ουδέ ερίφιον έδωκάς ποτέ διά να ευφρανθώ μετά των φίλων μου.
\par 30 Ότε δε ο υιός σου ούτος, ο καταφαγών σου τον βίον μετά πορνών, ήλθεν, έσφαξας δι' αυτόν τον μόσχον τον σιτευτόν.
\par 31 Ο δε είπε προς αυτόν· Τέκνον, συ πάντοτε μετ' εμού είσαι, και πάντα τα εμά σα είναι·
\par 32 έπρεπε δε να ευφρανθώμεν και να χαρώμεν, διότι ο αδελφός σου ούτος νεκρός ήτο και ανέζησε, και απολωλώς ήτο και ευρέθη.

\chapter{16}

\par 1 Έλεγε δε και προς τους μαθητάς αυτού· Ήτο άνθρωπός τις πλούσιος, όστις είχεν οικονόμον, και ούτος κατηγορήθη προς αυτόν ως διασκορπίζων τα υπάρχοντα αυτού.
\par 2 Και κράξας αυτόν, είπε προς αυτόν· Τι είναι τούτο το οποίον ακούω περί σου; δος τον λογαριασμόν της οικονομίας σου· διότι δεν θέλεις δυνηθή πλέον να ήσαι οικονόμος.
\par 3 Είπε δε καθ' εαυτόν ο οικονόμος· Τι να κάμω, επειδή ο κύριός μου αφαιρεί απ' εμού την οικονομίαν; να σκάπτω δεν δύναμαι, να ζητώ εντρέπομαι·
\par 4 ενόησα τι πρέπει να κάμω, διά να με δεχθώσιν εις τους οίκους αυτών, όταν αποβληθώ της οικονομίας.
\par 5 Και προσκαλέσας ένα έκαστον των χρεωφειλετών του κυρίου αυτού, είπε προς τον πρώτον· Πόσον χρεωστείς εις τον κύριόν μου;
\par 6 Ο δε είπεν· Εκατόν μέτρα ελαίου. Και είπε προς αυτόν· Λάβε το έγγραφόν σου και καθήσας ταχέως γράψον πεντήκοντα.
\par 7 Έπειτα είπε προς άλλον· Συ δε πόσον χρεωστείς; Ο δε είπεν· Εκατόν μόδια σίτου. Και λέγει προς αυτόν· Λάβε το έγγραφόν σου και γράψον ογδοήκοντα.
\par 8 Και επήνεσεν ο κύριος τον άδικον οικονόμον, ότι φρονίμως έπραξε· διότι οι υιοί του αιώνος τούτου είναι φρονιμώτεροι εις την εαυτών γενεάν παρά τους υιούς του φωτός.
\par 9 Και εγώ σας λέγω· Κάμετε εις εαυτούς φίλους εκ του μαμωνά της αδικίας, διά να σας δεχθώσιν εις τας αιωνίους σκηνάς, όταν εκλείψητε.
\par 10 Ο εν τω ελαχίστω πιστός και εν τω πολλώ πιστός είναι, και ο εν τω ελαχίστω άδικος και εν τω πολλώ άδικος είναι.
\par 11 Εάν λοιπόν εις τον άδικον μαμωνά δεν εφάνητε πιστοί, τον αληθινόν πλούτον τις θέλει σας εμπιστευθή;
\par 12 Και εάν εις το ξένον δεν εφάνητε πιστοί, τις θέλει σας δώσει το ιδικόν σας;
\par 13 Ουδείς δούλος δύναται να δουλεύη δύο κυρίους διότι ή τον ένα θέλει μισήσει και τον άλλον θέλει αγαπήσει· ή εις τον ένα θέλει προσκολληθή και τον άλλον θέλει καταφρονήσει. Δεν δύνασθε να δουλεύητε Θεόν και μαμωνά.
\par 14 Ήκουον δε ταύτα πάντα και οι Φαρισαίοι, φιλάργυροι όντες, και περιεγέλων αυτόν.
\par 15 Και είπε προς αυτούς· Σεις είσθε οι δικαιόνοντες εαυτούς ενώπιον των ανθρώπων, ο Θεός όμως γνωρίζει τας καρδίας σας· διότι εκείνο, το οποίον μεταξύ των ανθρώπων είναι υψηλόν, βδέλυγμα είναι ενώπιον του Θεού.
\par 16 Ο νόμος και οι προφήται έως Ιωάννου υπήρχον· από τότε η βασιλεία του Θεού ευαγγελίζεται, και πας τις βιάζεται να εισέλθη εις αυτήν.
\par 17 Ευκολώτερον δε είναι ο ουρανός και η γη να παρέλθωσι παρά μία κεραία του νόμου να πέση.
\par 18 Πας όστις χωρίζεται την γυναίκα αυτού και νυμφεύεται άλλην, μοιχεύει, και πας όστις νυμφεύεται κεχωρισμένην από ανδρός, μοιχεύει.
\par 19 Ήτο δε άνθρωπός τις πλούσιος και ενεδύετο πορφύραν και στολήν βυσσίνην, ευφραινόμενος καθ' ημέραν μεγαλοπρεπώς.
\par 20 Ήτο δε πτωχός τις ονομαζόμενος Λάζαρος, όστις έκειτο πεπληγωμένος πλησίον της πύλης αυτού
\par 21 και επεθύμει να χορτασθή από των ψιχίων των πιπτόντων από της τραπέζης του πλουσίου· αλλά και οι κύνες ερχόμενοι έγλειφον τας πληγάς αυτού.
\par 22 Απέθανε δε ο πτωχός και εφέρθη υπό των αγγέλων εις τον κόλπον του Αβραάμ· απέθανε δε και ο πλούσιος και ετάφη.
\par 23 Και εν τω άδη υψώσας τους οφθαλμούς αυτού, ενώ ήτο εν βασάνοις, βλέπει τον Αβραάμ από μακρόθεν και τον Λάζαρον εν τοις κόλποις αυτού.
\par 24 Και αυτός φωνάξας είπε· Πάτερ Αβραάμ, ελέησόν με και πέμψον τον Λάζαρον, διά να βάψη το άκρον του δακτύλου αυτού εις ύδωρ και να καταδροσίση την γλώσσαν μου, διότι βασανίζομαι εν τη φλογί ταύτη·
\par 25 είπε δε ο Αβραάμ· Τέκνον, ενθυμήθητι ότι απέλαβες συ τα αγαθά σου εν τη ζωή σου, και ο Λάζαρος ομοίως τα κακά· τώρα ούτος μεν παρηγορείται, συ δε βασανίζεσαι·
\par 26 και εκτός τούτων πάντων, μεταξύ ημών και υμών χάσμα μέγα είναι εστηριγμένον, ώστε οι θέλοντες να διαβώσιν εντεύθεν προς εσάς να μη δύνανται, μηδέ οι εκείθεν να διαπερώσι προς υμάς.
\par 27 Είπε δέ· παρακαλώ σε λοιπόν, πάτερ, να πέμψης αυτόν εις τον οίκον του πατρός μου·
\par 28 διότι έχω πέντε αδελφούς· διά να μαρτυρήση εις αυτούς, ώστε να μη έλθωσι και αυτοί εις τον τόπον τούτον της βασάνου.
\par 29 Λέγει προς αυτόν ο Αβραάμ, Έχουσι τον Μωϋσήν και τους προφήτας· ας ακούσωσιν αυτούς.
\par 30 Ο δε είπεν· Ουχί, πάτερ Αβραάμ, αλλ' εάν τις από νεκρών υπάγη προς αυτούς, θέλουσι μετανοήσει.
\par 31 Είπε δε προς αυτόν· Εάν τον Μωϋσήν και τους προφήτας δεν ακούωσιν, ουδέ εάν τις αναστηθή εκ νεκρών θέλουσι πεισθή.

\chapter{17}

\par 1 Είπε δε προς τους μαθητάς· Αδύνατον είναι να μη έλθωσι τα σκάνδαλα· πλην ουαί εις εκείνον, διά του οποίου έρχονται.
\par 2 Συμφέρει εις αυτόν να κρεμασθή περί τον τράχηλον αυτού μύλου πέτρα και να ριφθή εις την θάλασσαν, παρά να σκανδαλίση ένα των μικρών τούτων.
\par 3 Προσέχετε εις εαυτούς. Εάν δε ο αδελφός σου αμαρτήση εις σε, επίπληξον αυτόν· και εάν μετανοήση, συγχώρησον αυτόν.
\par 4 και εάν επτάκις της ημέρας αμαρτήση εις σε, και επτάκις της ημέρας επιστρέψη προς σε λέγων· Μετανοώ, θέλεις συγχωρήσει αυτόν.
\par 5 Και είπον οι απόστολοι προς τον Κύριον· Αύξησον εις ημάς την πίστιν.
\par 6 Ο δε Κύριος είπεν· Εάν έχετε πίστιν ως κόκκον σινάπεως, ηθέλετε ειπεί εις την συκάμινον ταύτην, Εκριζώθητι και φυτεύθητι εις την θάλασσαν· και ήθελε σας υπακούσει.
\par 7 Τις δε από σας έχων δούλον αροτριώντα ή ποιμαίνοντα, θέλει ειπεί προς αυτόν, ευθύς αφού έλθη εκ του αγρού· Ύπαγε, κάθησον να φάγης,
\par 8 και δεν θέλει ειπεί προς αυτόν· Ετοίμασον τι να δειπνήσω, και περιζωσθείς υπηρέτει με, εωσού φάγω και πίω, και μετά ταύτα θέλεις φάγει και πίει συ;
\par 9 Μήπως γνωρίζει χάριν εις τον δούλον εκείνον, διότι έκαμε τα διαταχθέντα εις αυτόν; δεν μοι φαίνεται.
\par 10 Ούτω και σεις, όταν κάμητε πάντα τα διαταχθέντα εις εσάς, λέγετε ότι δούλοι αχρείοι είμεθα, επειδή εκάμαμεν ό,τι εχρεωστούμεν να κάμωμεν.
\par 11 Και ότε αυτός επορεύετο εις την Ιερουσαλήμ, διέβαινε διά μέσου της Σαμαρείας και Γαλιλαίας.
\par 12 Και ενώ εισήρχετο εις τινά κώμην, απήντησαν αυτόν δέκα άνθρωποι λεπροί, οίτινες εστάθησαν μακρόθεν,
\par 13 και αυτοί ύψωσαν φωνήν, λέγοντες· Ιησού, Επιστάτα, ελέησον ημάς.
\par 14 Και ιδών είπε προς αυτούς· Υπάγετε και δείξατε εαυτούς εις τους ιερείς. Και ενώ, επορεύοντο, εκαθαρίσθησαν.
\par 15 Εις δε εξ αυτών, ιδών ότι ιατρεύθη, υπέστρεψε μετά φωνής μεγάλης δοξάζων τον Θεόν,
\par 16 και έπεσε κατά πρόσωπον εις τους πόδας αυτού, ευχαριστών αυτόν· και αυτός ήτο Σαμαρείτης.
\par 17 Αποκριθείς δε ο Ιησούς είπε· Δεν εκαθαρίσθησαν οι δέκα; οι δε εννέα που είναι;
\par 18 Δεν ευρέθησαν άλλοι να υποστρέψωσι διά να δοξάσωσι τον Θεόν ειμή ο αλλογενής ούτος;
\par 19 Και είπε προς αυτόν· Σηκωθείς ύπαγε· η πίστις σου σε έσωσεν.
\par 20 Ερωτηθείς δε υπό των Φαρισαίων, πότε έρχεται η βασιλεία του Θεού, απεκρίθη προς αυτούς και είπε· Δεν έρχεται η βασιλεία του Θεού ούτως ώστε να παρατηρήται·
\par 21 ουδέ θέλουσιν ειπεί· Ιδού, εδώ είναι, ή Ιδού εκεί· διότι ιδού, η βασιλεία του Θεού είναι εντός υμών.
\par 22 Είπε δε προς τους μαθητάς· θέλουσιν ελθεί ημέραι, ότε θέλετε επιθυμήσει να ίδητε μίαν των ημερών του Υιού του ανθρώπου, και δεν θέλετε ιδεί.
\par 23 και θέλουσι σας ειπεί· Ιδού, εδώ είναι, ή Ιδού εκεί· μη υπάγητε μηδ' ακολουθήσητε.
\par 24 Διότι ως η αστραπή η αστράπτουσα εκ της υπ' ουρανόν λάμπει εις την υπ' ουρανόν, ούτω θέλει είσθαι και ο Υιός του ανθρώπου εν τη ημέρα αυτού.
\par 25 Πρώτον όμως πρέπει αυτός να πάθη πολλά και να καταφρονηθή από της γενεάς ταύτης.
\par 26 Και καθώς έγεινεν εν ταις ημέραις του Νώε, ούτω θέλει είσθαι και εν ταις ημέραις του Υιού του ανθρώπου·
\par 27 έτρωγον, έπινον, ενύμφευον, ενυμφεύοντο, μέχρι της ημέρας καθ' ην ο Νώε εισήλθεν εις την κιβωτόν, και ήλθεν ο κατακλυσμός και απώλεσεν άπαντας.
\par 28 Ομοίως και καθώς έγεινεν εν ταις ημέραις του Λώτ· έτρωγον, έπινον, ηγόραζον, επώλουν, εφύτευον, ωκοδόμουν·
\par 29 καθ' ην δε ημέραν εξήλθεν ο Λωτ από Σοδόμων, έβρεξε πυρ και θείον απ' ουρανού και απώλεσεν άπαντας.
\par 30 Ωσαύτως θέλει είσθαι καθ' ην ημέραν ο Υιός του ανθρώπου θέλει φανερωθή.
\par 31 Κατ' εκείνην την ημέραν όστις ευρεθή επί του δώματος και τα σκεύη αυτού εν τη οικία, ας μη καταβή διά να λάβη αυτά, και όστις εν τω αγρώ ομοίως ας μη επιστρέψη εις τα οπίσω.
\par 32 Ενθυμείσθε την γυναίκα του Λωτ.
\par 33 Όστις ζητήση να σώση την ζωήν αυτού, θέλει απολέσει αυτήν, και όστις απολέση αυτήν, θέλει διαφυλάξει αυτήν.
\par 34 Σας λέγω, Εν τη νυκτί εκείνη θέλουσιν είσθαι δύο επί μιας κλίνης, ο εις παραλαμβάνεται και ο άλλος αφίνεται·
\par 35 δύο γυναίκες θέλουσιν αλέθει ομού, η μία παραλαμβάνεται και η άλλη αφίνεται·
\par 36 δύο θέλουσιν είσθαι εν τω αγρώ, ο εις παραλαμβάνεται και ο άλλος αφίνεται.
\par 37 Και αποκριθέντες λέγουσι προς αυτόν· Που, Κύριε; Ο δε είπε προς αυτούς· Όπου είναι το σώμα, εκεί θέλουσι συναχθή οι αετοί.

\chapter{18}

\par 1 Έλεγε δε και παραβολήν προς αυτούς περί του ότι πρέπει πάντοτε να προσεύχωνται και να μη αποκάμνωσι,
\par 2 λέγων· Κριτής τις ήτο εν τινί πόλει, όστις τον Θεόν δεν εφοβείτο και άνθρωπον δεν εντρέπετο.
\par 3 Ήτο δε χήρα τις εν εκείνη τη πόλει και ήρχετο προς αυτόν, λέγουσα· Εκδίκησόν με από του αντιδίκου μου.
\par 4 Και μέχρι τινός δεν ηθέλησε· μετά δε ταύτα είπε καθ' εαυτόν· Αν και τον Θεόν δεν φοβώμαι και άνθρωπον δεν εντρέπωμαι,
\par 5 τουλάχιστον επειδή με ενοχλεί η χήρα αύτη, ας εκδικήσω αυτήν, διά να μη έρχηται πάντοτε και με βασανίζη.
\par 6 Και είπεν ο Κύριος· Ακούσατε τι λέγει ο άδικος κριτής·
\par 7 ο δε Θεός δεν θέλει κάμει την εκδίκησιν των εκλεκτών αυτού των βοώντων προς αυτόν ημέραν και νύκτα, αν και μακροθυμή δι' αυτούς;
\par 8 σας λέγω ότι θέλει κάμει την εκδίκησιν αυτών ταχέως. Πλην ο Υιός του ανθρώπου, όταν έλθη, άρα γε θέλει ευρεί την πίστιν επί της γης;
\par 9 Είπε δε και προς τινάς, τους θαρρούντας εις εαυτούς ότι είναι δίκαιοι και καταφρονούντας τους λοιπούς, την παραβολήν ταύτην·
\par 10 Άνθρωποι δύο ανέβησαν εις το ιερόν διά να προσευχηθώσιν, ο εις Φαρισαίος και ο άλλος τελώνης.
\par 11 Ο Φαρισαίος σταθείς προσηύχετο καθ' εαυτόν ταύτα· Ευχαριστώ σοι, Θεέ, ότι δεν είμαι καθώς οι λοιποί άνθρωποι, άρπαγες, άδικοι, μοιχοί, ή και καθώς ούτος ο τελώνης·
\par 12 νηστεύω δις της εβδομάδος, αποδεκατίζω πάντα όσα έχω.
\par 13 Και ο τελώνης μακρόθεν ιστάμενος, δεν ήθελεν ουδέ τους οφθαλμούς να υψώση εις τον ουρανόν, αλλ' έτυπτεν εις το στήθος αυτού, λέγων· Ο Θεός, ιλάσθητί μοι τω αμαρτωλώ.
\par 14 Σας λέγω, Κατέβη ούτος εις τον οίκον αυτού δεδικαιωμένος μάλλον παρά εκείνος· διότι πας ο υψών εαυτόν θέλει ταπεινωθή, ο δε ταπεινών εαυτόν θέλει υψωθή.
\par 15 Έφερον δε προς αυτόν και τα βρέφη, διά να εγγίζη αυτά· ιδόντες δε οι μαθηταί, επέπληξαν αυτούς.
\par 16 Ο Ιησούς όμως προσκαλέσας αυτά, είπεν· Αφήσατε τα παιδία να έρχωνται προς εμέ, και μη εμποδίζετε αυτά· διότι των τοιούτων είναι η βασιλεία του Θεού.
\par 17 Αληθώς σας λέγω, Όστις δεν δεχθή την βασιλείαν του Θεού ως παιδίον, δεν θέλει εισέλθει εις αυτήν.
\par 18 Και άρχων τις ηρώτησεν αυτόν λέγων· Διδάσκαλε αγαθέ, τι να πράξω διά να κληρονομήσω ζωήν αιώνιον;
\par 19 Και ο Ιησούς είπε προς αυτόν· Τι με λέγεις αγαθόν; ουδείς αγαθός ειμή εις ο Θεός.
\par 20 Τας εντολάς εξεύρεις· Μη μοιχεύσης, Μη φονεύσης, Μη κλέψης, Μη ψευδομαρτυρήσης, Τίμα τον πατέρα σου και την μητέρα σου.
\par 21 Ο δε είπε· Ταύτα πάντα εφύλαξα εκ νεότητός μου.
\par 22 Ακούσας δε ταύτα ο Ιησούς, είπε προς αυτόν· Έτι εν σοι λείπει· πάντα όσα έχεις πώλησον και διαμοίρασον εις πτωχούς, και θέλεις έχει θησαυρόν εν ουρανώ, και ελθέ, ακολούθει μοι.
\par 23 Ο δε ακούσας ταύτα έγεινε περίλυπος διότι ήτο πλούσιος σφόδρα.
\par 24 Ιδών δε αυτόν ο Ιησούς περίλυπον γενόμενον, είπε· Πως δυσκόλως θέλουσιν εισέλθει εις την βασιλείαν του Θεού οι έχοντες τα χρήματα·
\par 25 διότι ευκολώτερον είναι να περάση κάμηλος διά τρύπης βελόνης, παρά πλούσιος να εισέλθη εις την βασιλείαν του Θεού.
\par 26 Είπον δε οι ακούσαντες· Και τις δύναται να σωθή;
\par 27 Ο δε είπε· Τα αδύνατα παρά ανθρώποις είναι δυνατά παρά τω Θεώ.
\par 28 Είπε δε ο Πέτρος· Ιδού, ημείς αφήκαμεν πάντα και σε ηκολουθήσαμεν.
\par 29 Ο δε είπε προς αυτούς· Αληθώς σας λέγω ότι δεν είναι ουδείς, όστις αφήκεν οικίαν ή γονείς ή αδελφούς ή γυναίκα ή τέκνα ένεκεν της βασιλείας του Θεού,
\par 30 όστις δεν θέλει απολαύσει πολλαπλάσια εν τω καιρώ τούτω και εν τω ερχομένω αιώνι ζωήν αιώνιον.
\par 31 Παραλαβών δε τους δώδεκα, είπε προς αυτούς· Ιδού, αναβαίνομεν εις Ιεροσόλυμα, και θέλουσιν εκτελεσθή πάντα τα γεγραμμένα διά των προφητών εις τον Υιόν του ανθρώπου.
\par 32 Διότι θέλει παραδοθή εις τα έθνη και θέλει εμπαιχθή και υβρισθή και εμπτυσθή,
\par 33 και μαστιγώσαντες θέλουσι θανατώσει αυτόν, και τη τρίτη ημέρα θέλει αναστηθή.
\par 34 Και αυτοί δεν ενόησαν ουδέν εκ τούτων, και ήτο ο λόγος ούτος κεκρυμμένος απ' αυτών, και δεν ενόουν τα λεγόμενα.
\par 35 Ότε δε επλησίαζεν εις την Ιεριχώ, τυφλός τις εκάθητο παρά την οδόν ζητών·
\par 36 ακούσας δε όχλον διαβαίνοντα, ηρώτα τι είναι τούτο.
\par 37 Απήγγειλαν δε προς αυτόν ότι Ιησούς ο Ναζωραίος διαβαίνει.
\par 38 Και εφώναξε λέγων· Ιησού, υιέ του Δαβίδ, ελέησόν με.
\par 39 Και οι προπορευόμενοι επέπληττον αυτόν διά να σιωπήση· αλλ' αυτός πολλώ μάλλον έκραζεν· Υιέ του Δαβίδ, ελέησόν με.
\par 40 Σταθείς δε ο Ιησούς, προσέταξε να φερθή προς αυτόν. Και αφού επλησίασεν, ηρώτησεν αυτόν
\par 41 λέγων· Τι θέλεις να σοι κάμω; Ο δε είπε· Κύριε, να αναβλέψω.
\par 42 Και ο Ιησούς είπε προς αυτόν· Ανάβλεψον· η πίστις σου σε έσωσε.
\par 43 Και παρευθύς ανέβλεψε και ηκολούθει αυτόν δοξάζων τον Θεόν· και πας ο λαός ιδών ήνεσε τον Θεόν.

\chapter{19}

\par 1 Και εισελθών διήρχετο την Ιεριχώ·
\par 2 και ιδού, άνθρωπος ονομαζόμενος Ζακχαίος, όστις ήτο αρχιτελώνης, και ούτος ήτο πλούσιος,
\par 3 και εζήτει να ίδη τον Ιησούν τις είναι, και δεν ηδύνατο διά τον όχλον, διότι ήτο μικρός το ανάστημα.
\par 4 και δραμών εμπρός ανέβη επί συκομορέαν διά να ίδη αυτόν· επειδή δι' εκείνης της οδού έμελλε να περάση.
\par 5 Και ως ήλθεν εις τον τόπον ο Ιησούς, αναβλέψας είδεν αυτόν και είπε προς αυτόν· Ζακχαίε, κατάβα ταχέως· διότι σήμερον πρέπει να μείνω εν τω οίκω σου.
\par 6 Και κατέβη ταχέως και υπεδέχθη αυτόν μετά χαράς.
\par 7 Και ιδόντες άπαντες εγόγγυζον, λέγοντες ότι εις αμαρτωλόν άνθρωπον εισήλθε να καταλύση.
\par 8 Σταθείς δε ο Ζακχαίος, είπε προς τον Κύριον· Ιδού, τα ημίση των υπαρχόντων μου, Κύριε, δίδω εις τους πτωχούς, και εάν εσυκοφάντησά τινά εις τι, αποδίδω τετραπλούν.
\par 9 Είπε δε προς αυτόν ο Ιησούς ότι, Σήμερον έγεινε σωτηρία εις τον οίκον τούτον, καθότι και αυτός υιός του Αβραάμ είναι.
\par 10 Διότι ο Υιός του ανθρώπου ήλθε να ζητήση και να σώση το απολωλός.
\par 11 Και ενώ αυτοί ήκουον ταύτα, προσθέσας είπε παραβολήν, διότι ήτο πλησίον της Ιερουσαλήμ και αυτοί ενόμιζον ότι η βασιλεία του Θεού έμελλεν ευθύς να φανή·
\par 12 είπε λοιπόν· Άνθρωπος τις ευγενής υπήγεν εις χώραν μακράν διά να λάβη εις εαυτόν βασιλείαν και να υποστρέψη.
\par 13 Και καλέσας δέκα δούλους εαυτού, έδωκεν εις αυτούς δέκα μνας και είπε προς αυτούς· Πραγματευθήτε εωσού έλθω.
\par 14 Οι συμπολίται αυτού όμως εμίσουν αυτόν και απέστειλαν κατόπιν αυτού πρέσβεις, λέγοντες· Δεν θέλομεν τούτον να βασιλεύση εφ' ημάς.
\par 15 Και αφού υπέστρεψε λαβών την βασιλείαν, είπε να προσκληθώσι προς αυτόν οι δούλοι εκείνοι, εις τους οποίους έδωκε το αργύριον, διά να μάθη τι εκέρδησεν έκαστος.
\par 16 Και ήλθεν ο πρώτος, λέγων· Κύριε, η μνα σου εκέρδησε δέκα μνας.
\par 17 Και είπε προς αυτόν· Εύγε, αγαθέ δούλε· επειδή εις το ελάχιστον εφάνης πιστός, έχε εξουσίαν επάνω δέκα πόλεων.
\par 18 Και ήλθεν ο δεύτερος, λέγων· Κύριε, η μνα σου έκαμε πέντε μνας.
\par 19 Είπε δε και προς τούτον· Και συ γενού εξουσιαστής επάνω πέντε πόλεων.
\par 20 Ήλθε και άλλος, λέγων· Κύριε, ιδού η μνα σου, την οποίαν είχον πεφυλαγμένην εν μανδηλίω.
\par 21 Διότι σε εφοβούμην, επειδή είσαι άνθρωπος αυστηρός· λαμβάνεις ό,τι δεν κατέβαλες, και θερίζεις ό,τι δεν έσπειρας.
\par 22 Και λέγει προς αυτόν· Εκ του στόματός σου θέλω σε κρίνει, πονηρέ δούλε· ήξευρες ότι εγώ είμαι άνθρωπος αυστηρός, λαμβάνων ό,τι δεν κατέβαλον, και θερίζων ό,τι δεν έσπειρα·
\par 23 διά τι λοιπόν δεν έδωκας το αργύριόν μου εις την τράπεζαν, ώστε εγώ ελθών ήθελον συνάξει αυτό μετά του τόκου;
\par 24 Και είπε προς τους παρεστώτας· Αφαιρέσατε απ' αυτού την μναν και δότε εις τον έχοντα τας δέκα μνας.
\par 25 Και είπον προς αυτόν· Κύριε, έχει δέκα μνας.
\par 26 Διότι σας λέγω ότι εις πάντα τον έχοντα θέλει δοθή, από δε του μη έχοντος και ό,τι έχει θέλει αφαιρεθή απ' αυτού.
\par 27 Πλην τους εχθρούς μου εκείνους, οίτινες δεν με ηθέλησαν να βασιλεύσω επ' αυτούς, φέρετε εδώ και κατασφάξατε έμπροσθέν μου.
\par 28 Και ειπών ταύτα, προεχώρει αναβαίνων εις Ιεροσόλυμα.
\par 29 Και ως επλησίασεν εις Βηθφαγή και Βηθανίαν, προς το όρος το καλούμενον Ελαιών, απέστειλε δύο των μαθητών αυτού,
\par 30 ειπών· Υπάγετε εις την κατέναντι κώμην, εις την οποίαν εμβαίνοντες θέλετε ευρεί πωλάριον δεδεμένον, επί του οποίου ουδείς άνθρωπος εκάθησέ ποτε· λύσατε αυτό και φέρετε.
\par 31 Και εάν τις σας ερωτήση, Διά τι λύετε αυτό ούτω θέλετε ειπεί προς αυτόν, Ότι ο Κύριος έχει χρείαν αυτού.
\par 32 Υπήγαν δε οι απεσταλμένοι και εύρον καθώς είπε προς αυτούς·
\par 33 και ενώ έλυον το πωλάριον, είπον προς αυτούς οι κύριοι αυτού· Διά τι λύετε το πωλάριον;
\par 34 Οι δε είπον· Ο Κύριος έχει χρείαν αυτού,
\par 35 και έφεραν αυτό προς τον Ιησούν· και ρίψαντες επί το πωλάριον τα ιμάτια αυτών, επεκάθισαν τον Ιησούν.
\par 36 Ενώ δε επορεύετο, υπέστρωνον τα ιμάτια αυτών εις την οδόν.
\par 37 Και ότε επλησίαζεν ήδη εις την κατάβασιν του όρους των Ελαιών, ήρχισαν άπαν το πλήθος των μαθητών χαίροντες να υμνώσι τον Θεόν μεγαλοφώνως διά πάντα τα θαύματα, τα οποία είδον,
\par 38 λέγοντες· Ευλογημένος ο ερχόμενος Βασιλεύς εν ονόματι του Κυρίου· ειρήνη εν ουρανώ, και δόξα εν υψίστοις.
\par 39 Και τινές των Φαρισαίων από του όχλου είπον προς αυτόν· Διδάσκαλε, επίπληξον τους μαθητάς σου.
\par 40 Και αποκριθείς είπε προς αυτούς· Σας λέγω ότι εάν ούτοι σιωπήσωσιν, οι λίθοι θέλουσι φωνάξει.
\par 41 Και ότε επλησίασεν, ιδών την πόλιν έκλαυσεν επ' αυτήν,
\par 42 λέγων, Είθε να εγνώριζες και συ, τουλάχιστον εν τη ημέρα σου ταύτη, τα προς ειρήνην σου αποβλέποντα· αλλά τώρα εκρύφθησαν από των οφθαλμών σου·
\par 43 διότι θέλουσιν ελθεί ημέραι επί σε και οι εχθροί σου θέλουσι κάμει χαράκωμα περί σε, και θέλουσι σε περικυκλώσει και θέλουσι σε στενοχωρήσει πανταχόθεν,
\par 44 και θέλουσι κατεδαφίσει σε και τα τέκνα σου εν σοι, και δεν θέλουσιν αφήσει εν σοι λίθον επί λίθον, διότι δεν εγνώρισας τον καιρόν της επισκέψεώς σου.
\par 45 Και εισελθών εις το ιερόν, ήρχισε να εκβάλλη τους πωλούντας εν αυτώ και αγοράζοντας,
\par 46 λέγων προς αυτούς· Είναι γεγραμμένον, Ο οίκός μου είναι οίκος προσευχής· σεις δε εκάμετε αυτόν σπήλαιον ληστών.
\par 47 Και εδίδασκε καθ' ημέραν εν τω ιερώ οι δε αρχιερείς και οι γραμματείς και οι πρώτοι του λαού εζήτουν να απολέσωσιν αυτόν.
\par 48 Και δεν εύρισκον το τι να πράξωσι· διότι πας ο λαός ήτο προσηλωμένος εις το να ακούη αυτόν.

\chapter{20}

\par 1 Και εν μιά των ημερών εκείνων, ενώ αυτός εδίδασκε τον λαόν εν τω ιερώ, και ευηγγελίζετο, ήλθον εξαίφνης οι αρχιερείς και οι γραμματείς μετά των πρεσβυτέρων
\par 2 και είπον προς αυτόν, λέγοντες· Ειπέ προς ημάς εν ποία εξουσία πράττεις ταύτα, ή τις είναι όστις σοι έδωκε την εξουσίαν ταύτην;
\par 3 Αποκριθείς δε είπε προς αυτούς· Θέλω σας ερωτήσει και εγώ ένα λόγον, και είπατέ μοι·
\par 4 το βάπτισμα του Ιωάννου εξ ουρανού ήτο ή εξ ανθρώπων;
\par 5 Οι δε εσυλλογίσθησαν καθ' εαυτούς λέγοντες, ότι Εάν είπωμεν, Εξ ουρανού, θέλει ειπεί, Διά τι λοιπόν δεν επιστεύσατε εις αυτόν;
\par 6 Εάν δε είπωμεν, Εξ ανθρώπων, πας ο λαός θέλει μας λιθοβολήσει· επειδή είναι πεπεισμένοι ότι ο Ιωάννης είναι προφήτης.
\par 7 Και απεκρίθησαν ότι δεν εξεύρουσι πόθεν ήτο.
\par 8 Και ο Ιησούς είπε προς αυτούς· Ουδέ εγώ σας λέγω εν ποία εξουσία πράττω ταύτα.
\par 9 Ήρχισε δε να λέγη προς τον λαόν την παραβολήν ταύτην· Άνθρωπος τις εφύτευσεν αμπελώνα, και εμίσθωσεν αυτόν εις γεωργούς, και απεδήμησε πολύν καιρόν.
\par 10 Και εν τω καιρώ των καρπών απέστειλε προς τους γεωργούς δούλον διά να δώσωσιν εις αυτόν από του καρπού του αμπελώνος· οι γεωργοί όμως δείραντες αυτόν εξαπέστειλαν κενόν·
\par 11 Και πάλιν έπεμψεν άλλον δούλον. Πλην αυτοί δείραντες και εκείνον και ατιμάσαντες εξαπέστειλαν κενόν.
\par 12 Και πάλιν έπεμψε τρίτον. Αλλ' εκείνοι και τούτον πληγώσαντες απεδίωξαν.
\par 13 Είπε δε ο κύριος του αμπελώνος· Τι να κάμω; ας πέμψω τον υιόν μου τον αγαπητόν· ίσως ιδόντες τούτον θέλουσιν εντραπεί.
\par 14 Πλην ιδόντες αυτόν οι γεωργοί, διελογίζοντο καθ' εαυτούς λέγοντες· Ούτος είναι ο κληρονόμος· έλθετε ας φονεύσωμεν αυτόν, διά να γείνη ημών η κληρονομία.
\par 15 Και εκβαλόντες αυτόν έξω του αμπελώνος, εφόνευσαν· Τι λοιπόν θέλει κάμει εις αυτούς ο κύριος του αμπελώνος;
\par 16 Θέλει ελθεί και απολέσει τους γεωργούς τούτους, και θέλει δώσει τον αμπελώνα εις άλλους. Ακούσαντες δε είπον· Μη γένοιτο.
\par 17 Ο δε εμβλέψας εις αυτούς είπε· Τι λοιπόν είναι τούτο το γεγραμμένον, Ο λίθος, τον οποίον απεδοκίμασαν οι οικοδομούντες, ούτος έγεινε κεφαλή γωνίας;
\par 18 Πας όστις πέση επί τον λίθον εκείνον θέλει συντριφθή· εις όντινα δε επιπέση, θέλει κατασυντρίψει αυτόν.
\par 19 Και εζήτησαν οι αρχιερείς και οι γραμματείς να βάλωσιν επ' αυτόν τας χείρας εν αυτή τη ώρα, πλην εφοβήθησαν τον λαόν· διότι ηνόησαν ότι προς αυτούς είπε την παραβολήν ταύτην.
\par 20 Και παραφυλάξαντες απέστειλαν ενεδρευτάς, υποκρινομένους ότι είναι δίκαιοι, επί σκοπώ να πιάσωσιν αυτόν από λόγου, διά να παραδώσωσιν αυτόν εις την αρχήν και εις την εξουσίαν του ηγεμόνος.
\par 21 Και ηρώτησαν αυτόν λέγοντες· Διδάσκαλε, εξεύρομεν ότι ορθώς ομιλείς και διδάσκεις και δεν βλέπεις εις πρόσωπον, αλλ' επ' αληθείας την οδόν του Θεού διδάσκεις·
\par 22 είναι συγκεχωρημένον εις ημάς να δώσωμεν φόρον εις τον Καίσαρα ή ουχί;
\par 23 Εννοήσας δε την πανουργίαν αυτών, είπε προς αυτούς· Τι με πειράζετε;
\par 24 δείξατέ μοι δηνάριον· τίνος εικόνα έχει και επιγραφήν; Και αποκριθέντες είπον· Του Καίσαρος.
\par 25 Ο δε είπε προς αυτούς· Απόδοτε λοιπόν τα του Καίσαρος εις τον Καίσαρα και τα του Θεού εις τον Θεόν.
\par 26 Και δεν ηδυνήθησαν να πιάσωσιν αυτόν από λόγου έμπροσθεν του λαού, και θαυμάσαντες διά την απόκρισιν αυτού εσιώπησαν.
\par 27 Προσελθόντες δε τινές των Σαδδουκαίων, οίτινες αρνούνται ότι είναι ανάστασις, ηρώτησαν αυτόν,
\par 28 λέγοντες· Διδάσκαλε, ο Μωϋσής μας έγραψεν· Εάν τινός ο αδελφός αποθάνη έχων γυναίκα, και ούτος αποθάνη άτεκνος, να λάβη ο αδελφός αυτού την γυναίκα και να εξαναστήση σπέρμα εις τον αδελφόν αυτού.
\par 29 Ήσαν λοιπόν επτά αδελφοί· και ο πρώτος λαβών γυναίκα, απέθανεν άτεκνος·
\par 30 και έλαβεν ο δεύτερος την γυναίκα, και ούτος απέθανεν άτεκνος·
\par 31 και ο τρίτος έλαβεν αυτήν· ωσαύτως δε και οι επτά· και δεν αφήκαν τέκνα, και απέθανον.
\par 32 Ύστερον δε πάντων απέθανε και η γυνή.
\par 33 Εν τη αναστάσει λοιπόν τίνος αυτών γίνεται γυνή; διότι και οι επτά έλαβον αυτήν γυναίκα.
\par 34 Και ο Ιησούς αποκριθείς είπε προς αυτούς· οι υιοί του αιώνος τούτου νυμφεύουσι και νυμφεύονται·
\par 35 οι δε καταξιωθέντες να απολαύσωσιν εκείνον τον αιώνα και την εκ νεκρών ανάστασιν ούτε νυμφεύουσιν ούτε νυμφεύονται·
\par 36 διότι ούτε να αποθάνωσι πλέον δύνανται· επειδή είναι ισάγγελοι και είναι υιοί του Θεού, όντες υιοί της αναστάσεως.
\par 37 Ότι δε εγείρονται οι νεκροί, και ο Μωϋσής εφανέρωσεν επί της βάτου, ότε λέγει Κύριον τον Θεόν του Αβραάμ και τον Θεόν του Ισαάκ και τον Θεόν του Ιακώβ.
\par 38 Ο δε Θεός δεν είναι νεκρών, αλλά ζώντων· διότι πάντες ζώσι εν αυτώ.
\par 39 Αποκριθέντες δε τινές των γραμματέων είπον· Διδάσκαλε, καλώς είπας.
\par 40 Και δεν ετόλμων πλέον να ερωτώσιν αυτόν ουδέν.
\par 41 Είπε δε προς αυτούς· Πως λέγουσι τον Χριστόν ότι είναι υιός του Δαβίδ;
\par 42 Και αυτός ο Δαβίδ λέγει εν τη βίβλω των ψαλμών· Είπεν ο Κύριος προς τον Κύριόν μου, κάθου εκ δεξιών μου,
\par 43 εωσού θέσω τους εχθρούς σου υποπόδιον των ποδών σου.
\par 44 Ο Δαβίδ λοιπόν ονομάζει αυτόν Κύριον· και πως είναι υιός αυτού;
\par 45 Και ενώ ήκουε πας ο λαός, είπε προς τους μαθητάς αυτού·
\par 46 Προσέχετε από των γραμματέων, οίτινες θέλουσι να περιπατώσιν εστολισμένοι και αγαπώσιν ασπασμούς εν ταις αγοραίς και πρωτοκαθεδρίας εν ταις συναγωγαίς και τους πρώτους τόπους εν τοις δείπνοις,
\par 47 οίτινες κατατρώγουσι τας οικίας των χηρών, και τούτο επί προφάσει ότι κάμνουσι μακράς προσευχάς· ούτοι θέλουσι λάβει μεγαλητέραν καταδίκην.

\chapter{21}

\par 1 Αναβλέψας δε είδε τους πλουσίους, τους βάλλοντας τα δώρα αυτών εις το γαζοφυλάκιον·
\par 2 είδε δε και χήραν τινά πτωχήν, βάλλουσαν εκεί δύο λεπτά,
\par 3 και είπεν· Αληθώς σας λέγω ότι η πτωχή αύτη χήρα έβαλε περισσότερον πάντων·
\par 4 διότι άπαντες ούτοι εκ του περισσεύματος αυτών έβαλον εις τα δώρα του Θεού, αύτη όμως εκ του υστερήματος αυτής έβαλεν όλην την περιουσίαν όσην είχε.
\par 5 Και ενώ τινές έλεγον περί του ιερού ότι είναι εστολισμένον με λίθους ωραίους και αφιερώματα, είπε·
\par 6 Ταύτα, τα οποία θεωρείτε, θέλουσιν ελθεί ημέραι, εις τας οποίας δεν θέλει αφεθή λίθος επί λίθον, όστις δεν θέλει κατακρημνισθή.
\par 7 Ηρώτησαν δε αυτόν, λέγοντες· Διδάσκαλε, πότε λοιπόν θέλουσι γείνει ταύτα και τι το σημείον, όταν μέλλωσι ταύτα να γείνωσιν;
\par 8 Ο δε είπε· Βλέπετε μη πλανηθήτε· διότι πολλοί θέλουσιν ελθεί εν τω ονόματί μου, λέγοντες ότι Εγώ είμαι και Ο καιρός επλησίασε. Μη υπάγητε λοιπόν οπίσω αυτών.
\par 9 Όταν δε ακούσητε πολέμους και ακαταστασίας, μη φοβηθήτε· διότι πρέπει ταύτα να γείνωσι πρώτον, αλλά δεν είναι ευθύς το τέλος.
\par 10 Τότε έλεγε προς αυτούς· θέλει εγερθή έθνος επί έθνος και βασιλεία επί βασιλείαν,
\par 11 και θέλουσι γείνει κατά τόπους σεισμοί μεγάλοι και πείναι και λοιμοί, και θέλουσιν είσθαι φόβητρα και σημεία μεγάλα από του ουρανού.
\par 12 Προ δε τούτων πάντων θέλουσιν επιβάλει εφ' υμάς τας χείρας αυτών, και θέλουσι σας καταδιώξει, παραδίδοντες εις συναγωγάς και φυλακάς, φερομένους έμπροσθεν βασιλέων και ηγεμόνων ένεκεν του ονόματός μου·
\par 13 και τούτο θέλει αποβή εις εσάς προς μαρτυρίαν.
\par 14 Βάλετε λοιπόν εις τας καρδίας σας να μη προμελετάτε τι να απολογηθήτε·
\par 15 διότι εγώ θέλω σας δώσει στόμα και σοφίαν, εις την οποίαν δεν θέλουσι δυνηθή να αντιλογήσωσιν ουδέ να αντισταθώσι πάντες οι εναντίοι σας.
\par 16 Θέλετε δε παραδοθή και υπό γονέων και αδελφών και συγγενών και φίλων, και θέλουσι θανατώσει τινάς εξ υμών,
\par 17 και θέλετε είσθαι μισούμενοι υπό πάντων διά το όνομά μου·
\par 18 πλην θριξ εκ της κεφαλής σας δεν θέλει χαθή·
\par 19 διά της υπομονής σας αποκτήσατε τας ψυχάς σας.
\par 20 Όταν δε ίδητε την Ιερουσαλήμ περικυκλουμένην υπό στρατοπέδων, τότε γνωρίσατε ότι επλησίασεν η ερήμωσις αυτής.
\par 21 Τότε οι όντες εν τη Ιουδαία ας φεύγωσιν εις τα όρη, και οι εν μέσω αυτής ας αναχωρώσιν έξω, και οι εν τοις αγροίς ας μη εμβαίνωσιν εις αυτήν,
\par 22 διότι ημέραι εκδικήσεως είναι αύται, διά να πληρωθώσι πάντα τα γεγραμμένα.
\par 23 Ουαί δε εις τας εγκυμονούσας και τας θηλαζούσας εν εκείναις ταις ημέραις· διότι θέλει είσθαι μεγάλη στενοχωρία επί της γης και οργή κατά του λαού τούτου,
\par 24 και θέλουσι πέσει εν στόματι μαχαίρας και θέλουσι φερθή αιχμάλωτοι εις πάντα τα έθνη, και η Ιερουσαλήμ θέλει είσθαι πατουμένη υπό εθνών, εωσού εκπληρωθώσιν οι καιροί των εθνών.
\par 25 Και θέλουσιν είσθαι σημεία εν τω ηλίω και τη σελήνη και τοις άστροις, και επί της γης στενοχωρία εθνών εν απορία, και θέλει ηχεί η θάλασσα και τα κύματα,
\par 26 οι άνθρωποι θέλουσιν αποψυχεί εκ του φόβου και προσδοκίας των επερχομένων δεινών εις την οικουμένην· διότι αι δυνάμεις των ουρανών θέλουσι σαλευθή.
\par 27 Και τότε θέλουσιν ιδεί τον Υιόν του ανθρώπου ερχόμενον εν νεφέλη μετά δυνάμεως και δόξης πολλής.
\par 28 Όταν δε ταύτα αρχίσωσι να γίνωνται, ανακύψατε και σηκώσατε τας κεφαλάς σας, διότι πλησιάζει η απολύτρωσίς σας.
\par 29 Και είπε προς αυτούς παραβολήν· Ίδετε την συκήν και πάντα τα δένδρα.
\par 30 Όταν ήδη ανοίξωσι, βλέποντες γνωρίζετε αφ' εαυτών ότι ήδη το θέρος είναι πλησίον.
\par 31 Ούτω και σεις, όταν ίδητε ταύτα γινόμενα, εξεύρετε ότι είναι πλησίον η βασιλεία του Θεού.
\par 32 Αληθώς σας λέγω ότι δεν θέλει παρέλθει η γενεά αύτη, εωσού γείνωσι πάντα ταύτα.
\par 33 Ο ουρανός και η γη θέλουσι παρέλθει, οι δε λόγοι μου δεν θέλουσι παρέλθει.
\par 34 Προσέχετε δε εις εαυτούς μήποτε βαρυνθώσιν αι καρδίαι σας από κραιπάλης και μέθης και μεριμνών βιωτικών, και επέλθη αιφνίδιος εφ' υμάς η ημέρα εκείνη·
\par 35 διότι ως παγίς θέλει επέλθει επί πάντας τους καθημένους επί πρόσωπον πάσης της γης.
\par 36 Αγρυπνείτε λοιπόν δεόμενοι εν παντί καιρώ, διά να καταξιωθήτε να εκφύγητε πάντα ταύτα τα μέλλοντα να γείνωσι και να σταθήτε έμπροσθεν του Υιού του ανθρώπου.
\par 37 Και τας μεν ημέρας εδίδασκεν εν τω ιερώ, τας δε νύκτας εξερχόμενος διενυκτέρευεν εις το όρος το ονομαζόμενον Ελαιών·
\par 38 και πας ο λαός από του όρθρου συνήρχετο προς αυτόν εν τω ιερώ διά να ακούη αυτόν.

\chapter{22}

\par 1 Επλησίαζε δε η εορτή των αζύμων, λεγομένη Πάσχα.
\par 2 Και εζήτουν οι αρχιερείς και οι γραμματείς το πως να θανατώσωσιν αυτόν διότι φοβούντο τον λαόν.
\par 3 Εισήλθε δε ο Σατανάς εις τον Ιούδαν τον επονομαζόμενον Ισκαριώτην, όντα εκ του αριθμού των δώδεκα,
\par 4 και υπήγε και συνελάλησε μετά των αρχιερέων και των στρατηγών το πως να παραδώση αυτόν εις αυτούς.
\par 5 Και εχάρησαν και συνεφώνησαν να δώσωσιν εις αυτόν αργύριον·
\par 6 και έδωκεν υπόσχεσιν και εζήτει ευκαιρίαν να παραδώση αυτόν εις αυτούς χωρίς θορύβου.
\par 7 Ήλθε δε ημέρα των αζύμων, καθ' ην έπρεπε να θυσιάσωσι το πάσχα,
\par 8 και απέστειλε τον Πέτρον και Ιωάννην, ειπών· Υπάγετε και ετοιμάσατε εις ημάς το πάσχα, διά να φάγωμεν.
\par 9 Οι δε είπον προς αυτόν· Που θέλεις να ετοιμάσωμεν;
\par 10 Ο δε είπε προς αυτούς· Ιδού, όταν εισέλθητε εις την πόλιν, θέλει σας συναπαντήσει άνθρωπος βαστάζων σταμνίον ύδατος· ακολουθήσατε αυτόν εις την οικίαν όπου εισέρχεται.
\par 11 Και θέλετε ειπεί προς τον οικοδεσπότην της οικίας· Ο Διδάσκαλος σοι λέγει, Που είναι το κατάλυμα, όπου θέλω φάγει το πάσχα μετά των μαθητών μου;
\par 12 και εκείνος θέλει σας δείξει ανώγεον μέγα εστρωμένον· εκεί ετοιμάσατε.
\par 13 Αφού δε υπήγον, εύρον καθώς είπε προς αυτούς, και ητοίμασαν το πάσχα.
\par 14 Και ότε ήλθεν η ώρα, εκάθησεν εις την τράπεζαν, και οι δώδεκα απόστολοι μετ' αυτού.
\par 15 Και είπε προς αυτούς· Πολύ επεθύμησα να φάγω το πάσχα τούτο με σας προ του να πάθω·
\par 16 διότι σας λέγω, ότι δεν θέλω φάγει πλέον εξ αυτού, εωσού εκπληρωθή εν τη βασιλεία του Θεού.
\par 17 Και λαβών το ποτήριον, ευχαρίστησε και είπε· Λάβετε τούτο και διαμοιράσατε εις αλλήλους·
\par 18 διότι σας λέγω ότι δεν θέλω πίει από του γεννήματος της αμπέλου, εωσού έλθη η βασιλεία του Θεού.
\par 19 Και λαβών άρτον, ευχαριστήσας έκοψε και έδωκεν εις αυτούς, λέγων· Τούτο είναι το σώμα μου το υπέρ υμών διδόμενον· τούτο κάμνετε εις την ιδικήν μου ανάμνησιν.
\par 20 Ωσαύτως και το ποτήριον, αφού εδείπνησαν, λέγων· Τούτο το ποτήριον είναι η καινή διαθήκη εν τω αίματί μου, το υπέρ υμών εκχυνόμενον.
\par 21 Πλην ιδού, η χειρ εκείνου όστις με παραδίδει, είναι μετ' εμού επί της τραπέζης.
\par 22 Και ο μεν Υιός του ανθρώπου υπάγει κατά το ωρισμένον· πλην ουαί εις τον άνθρωπον εκείνον, δι' ου παραδίδεται.
\par 23 Και αυτοί ήρχισαν να συζητώσι προς αλλήλους το ποίος τάχα ήτο εξ αυτών, όστις έμελλε να κάμη τούτο.
\par 24 Έγεινε δε και φιλονεικία μεταξύ αυτών, περί του τις εξ αυτών νομίζεται ότι είναι μεγαλήτερος.
\par 25 Ο δε είπε προς αυτούς· οι βασιλείς των εθνών κυριεύουσιν αυτά, και οι εξουσιάζοντες αυτά ονομάζονται ευεργέται.
\par 26 Σεις όμως ουχί ούτως, αλλ' ο μεγαλήτερος μεταξύ σας ας γείνη ως ο μικρότερος, και ο προϊστάμενος ως ο υπηρετών.
\par 27 Διότι τις είναι μεγαλήτερος, ο καθήμενος εις την τράπεζαν ή ο υπηρετών; ουχί ο καθήμενος; αλλ' εγώ είμαι εν μέσω υμών ως ο υπηρετών.
\par 28 Σεις δε είσθε οι διαμείναντες μετ' εμού εν τοις πειρασμοίς μου·
\par 29 όθεν εγώ ετοιμάζω εις εσάς βασιλείαν, ως ο Πατήρ μου ητοίμασεν εις εμέ,
\par 30 διά να τρώγητε και να πίνητε επί της τραπέζης μου εν τη βασιλεία μου, και να καθήσητε επί θρόνων, κρίνοντες τας δώδεκα φυλάς του Ισραήλ.
\par 31 Είπε δε ο Κύριος· Σίμων, Σίμων, ιδού, ο Σατανάς σας εζήτησε διά να σας κοσκινίση ως τον σίτον·
\par 32 πλην εγώ εδεήθην περί σου διά να μη εκλείψη η πίστις σου· και συ, όταν ποτέ επιστρέψης, στήριξον τους αδελφούς σου.
\par 33 Ο δε είπε προς αυτόν· Κύριε, έτοιμος είμαι μετά σου να υπάγω και εις φυλακήν και εις θάνατον.
\par 34 Ο δε είπε· Σοι λέγω, Πέτρε, δεν θέλει φωνάξει σήμερον ο αλέκτωρ, πριν απαρνηθής τρίς ότι δεν με γνωρίζεις.
\par 35 Και είπε προς αυτούς· Ότε σας απέστειλα χωρίς βαλαντίου και σακκίου και υποδημάτων, μήπως εστερήθητέ τινός; οι δε είπον· Ουδενός.
\par 36 Είπε λοιπόν προς αυτούς· Αλλά τώρα όστις έχει βαλάντιον ας λάβη αυτό μεθ' εαυτού, ομοίως και σακκίον, και όστις δεν έχει ας πωλήση το ιμάτιον αυτού και ας αγοράση μάχαιραν.
\par 37 Διότι σας λέγω ότι έτι τούτο το γεγραμμένον πρέπει να εκτελεσθή εις εμέ, το, Και μετά ανόμων ελογίσθη. Διότι τα περί εμού γεγραμμένα λαμβάνουσι τέλος.
\par 38 Οι δε είπον· Κύριε, ιδού, ήδη δύο μάχαιραι. Ο δε είπε προς αυτούς· Ικανόν είναι.
\par 39 Και εξελθών υπήγε κατά την συνήθειαν εις το όρος των Ελαιών· ηκολούθησαν δε αυτόν και οι μαθηταί αυτού.
\par 40 Αφού δε ήλθεν εις τον τόπον, είπε προς αυτούς· Προσεύχεσθε, διά να μη εισέλθητε εις πειρασμόν.
\par 41 Και αυτός εχωρίσθη απ' αυτών ως λίθου βολήν, και γονατίσας προσηύχετο,
\par 42 λέγων· Πάτερ, εάν θέλης να απομακρύνης το ποτήριον τούτο απ' εμού· πλην ουχί το θέλημά μου, αλλά το σον ας γείνη.
\par 43 Εφάνη δε εις αυτόν άγγελος απ' ουρανού ενισχύων αυτόν.
\par 44 Και ελθών εις αγωνίαν, προσηύχετο θερμότερον, έγεινε δε ο ιδρώς αυτού ως θρόμβοι αίματος καταβαίνοντες εις την γην.
\par 45 Και σηκωθείς από της προσευχής, ήλθε προς τους μαθητάς αυτού και εύρεν αυτούς κοιμωμένους από της λύπης,
\par 46 και είπε προς αυτούς· Τι κοιμάσθε; σηκώθητε και προσεύχεσθε, διά να μη εισέλθητε εις πειρασμόν.
\par 47 Ενώ δε αυτός ελάλει έτι, ιδού όχλος, και ο λεγόμενος Ιούδας, εις των δώδεκα, ήρχετο προ αυτών και επλησίασεν εις τον Ιησούν, διά να φιλήση αυτόν.
\par 48 Ο δε Ιησούς είπε προς αυτόν· Ιούδα, με φίλημα παραδίδεις τον Υιόν του ανθρώπου;
\par 49 Ιδόντες δε οι περί αυτόν τι έμελλε να γείνη, είπον προς αυτόν· Κύριε, να κτυπήσωμεν με την μάχαιραν;
\par 50 Και εκτύπησεν εις εξ αυτών τον δούλον του αρχιερέως και απέκοψεν αυτού το ωτίον το δεξιόν.
\par 51 Αποκριθείς δε ο Ιησούς, είπεν· Αφήσατε έως τούτου· και πιάσας το ωτίον αυτού ιάτρευσεν αυτόν.
\par 52 Είπε δε ο Ιησούς προς τους ελθόντας επ' αυτόν αρχιερείς και στρατηγούς του ιερού και πρεσβυτέρους. Ως επί ληστήν εξήλθετε μετά μαχαιρών και ξύλων;
\par 53 καθ' ημέραν ήμην μεθ' υμών εν τω ιερώ και δεν ηπλώσατε τας χείρας επ' εμέ. Αλλ' αύτη είναι η ώρα σας και η εξουσία του σκότους.
\par 54 Συλλαβόντες δε αυτόν, έφεραν και εισήγαγον αυτόν εις τον οίκον του αρχιερέως. Ο δε Πέτρος ηκολούθει μακρόθεν.
\par 55 Αφού δε ανάψαντες πυρ εν τω μέσω της αυλής συνεκάθησαν, εκάθητο ο Πέτρος εν μέσω αυτών.
\par 56 Ιδούσα δε αυτόν μία τις δούλη καθήμενον προς το φως και ενατενίσασα εις αυτόν, είπε· Και ούτος ήτο μετ' αυτού.
\par 57 Ο δε ηρνήθη, λέγων· Γύναι, δεν γνωρίζω αυτόν.
\par 58 Και μετ' ολίγον άλλος τις ιδών αυτόν, είπε· Και συ εξ αυτών είσαι. Ο δε Πέτρος είπεν· Άνθρωπε, δεν είμαι.
\par 59 Και αφού επέρασεν ως μία ώρα, άλλος τις διϊσχυρίζετο, λέγων· Επ' αληθείας και ούτος μετ' αυτού ήτο· διότι Γαλιλαίος είναι.
\par 60 Είπε δε ο Πέτρος· Άνθρωπε, δεν εξεύρω τι λέγεις. Και παρευθύς, ενώ αυτός ελάλει έτι, εφώναξεν ο αλέκτωρ.
\par 61 Και στραφείς ο Κύριος ενέβλεψεν εις τον Πέτρον, και ενεθυμήθη ο Πέτρος τον λόγον του Κυρίου, ότι είπε προς αυτόν ότι πριν φωνάξη ο αλέκτωρ, θέλεις με απαρνηθή τρίς.
\par 62 Και εξελθών έξω ο Πέτρος έκλαυσε πικρώς.
\par 63 Και οι άνδρες οι κρατούντες τον Ιησούν ενέπαιζον αυτόν δέροντες,
\par 64 και περικαλύψαντες αυτόν ερράπιζον το πρόσωπον αυτού και ηρώτων αυτόν, λέγοντες· Προφήτευσον τις είναι όστις σε εκτύπησε;
\par 65 Και άλλα πολλά βλασφημούντες έλεγον εις αυτόν.
\par 66 Και καθώς έγεινεν ημέρα, συνήχθη το πρεσβυτέριον του λαού, αρχιερείς τε και γραμματείς, και ανεβίβασαν αυτόν εις το συνέδριον αυτών, λέγοντες·
\par 67 Συ είσαι ο Χριστός; ειπέ προς ημάς· είπε δε προς αυτούς. Εάν σας είπω, δεν θέλετε πιστεύσει,
\par 68 εάν δε και ερωτήσω, δεν θέλετε μοι αποκριθή ουδέ θέλετε με απολύσει·
\par 69 από του νυν θέλει είσθαι ο Υιός του ανθρώπου καθήμενος εκ δεξιών της δυνάμεως του Θεού.
\par 70 Είπον δε πάντες· Συ λοιπόν είσαι ο Υιός του Θεού; Ο δε είπε προς αυτούς· Σεις λέγετε ότι εγώ είμαι.
\par 71 Οι δε είπον· Τι χρείαν έχομεν πλέον μαρτυρίας; διότι ημείς αυτοί ηκούσαμεν από του στόματος αυτού.

\chapter{23}

\par 1 Τότε εσηκώθη άπαν το πλήθος αυτών και έφεραν αυτόν προς τον Πιλάτον.
\par 2 Και ήρχισαν να κατηγορώσιν αυτόν, λέγοντες· Τούτον εύρομεν διαστρέφοντα το έθνος και εμποδίζοντα το να δίδωσι φόρους εις τον Καίσαρα, λέγοντα εαυτόν ότι είναι Χριστός βασιλεύς.
\par 3 Ο δε Πιλάτος ηρώτησεν αυτόν, λέγων· Συ είσαι ο βασιλεύς των Ιουδαίων; Ο δε αποκριθείς προς αυτόν, είπε· Συ λέγεις.
\par 4 Και ο Πιλάτος είπε προς τους αρχιερείς και τους όχλους· Ουδέν έγκλημα ευρίσκω εν τω ανθρώπω τούτω.
\par 5 Οι δε επέμενον λέγοντες ότι Ταράττει τον λαόν, διδάσκων καθ' όλην την Ιουδαίαν, αρχίσας από της Γαλιλαίας έως εδώ.
\par 6 Ο δε Πιλάτος ακούσας Γαλιλαίαν ηρώτησεν αν ο άνθρωπος ήναι Γαλιλαίος,
\par 7 και μαθών ότι είναι εκ της επικρατείας του Ηρώδου, έπεμψεν αυτόν προς τον Ηρώδην, όστις ήτο και αυτός εν Ιεροσολύμοις εν ταύταις ταις ημέραις.
\par 8 Ο δε Ηρώδης, ιδών τον Ιησούν, εχάρη πολύ· διότι ήθελε προ πολλού να ίδη αυτόν, επειδή ήκουε πολλά περί αυτού και ήλπιζε να ίδη τι θαύμα γινόμενον υπ' αυτού.
\par 9 Ηρώτα δε αυτόν με λόγους πολλούς· πλην αυτός δεν απεκρίθη προς αυτόν ουδέν.
\par 10 Ίσταντο δε οι αρχιερείς και οι γραμματείς, κατηγορούντες αυτόν εντόνως.
\par 11 Αφού δε ο Ηρώδης μετά των στρατευμάτων αυτού εξουθένησεν αυτόν και ενέπαιξεν, ενέδυσεν αυτόν λαμπρόν ιμάτιον και έπεμψεν αυτόν πάλιν προς τον Πιλάτον.
\par 12 Εν αυτή δε τη ημέρα ο Πιλάτος και ο Ηρώδης έγειναν φίλοι μετ' αλλήλων· διότι πρότερον ήσαν εις έχθραν προς αλλήλους.
\par 13 Ο δε Πιλάτος, συγκαλέσας τους αρχιερείς και τους άρχοντας και τον λαόν,
\par 14 είπε προς αυτούς· Εφέρατε προς εμέ τον άνθρωπον τούτον ως στασιάζοντα τον λαόν, και ιδού, εγώ ενώπιόν σας ανακρίνας δεν εύρον εν τω ανθρώπω τούτω ουδέν έγκλημα εξ όσων κατηγορείτε κατ' αυτού,
\par 15 αλλ' ουδέ ο Ηρώδης, διότι σας έπεμψα προς αυτόν· και ιδού, ουδέν άξιον θανάτου είναι πεπραγμένον υπ' αυτού.
\par 16 Αφού λοιπόν παιδεύσω αυτόν, θέλω απολύσει.
\par 17 Έπρεπε δε αναγκαίως να απολύη εις αυτούς ένα εν τη εορτή.
\par 18 Πάντες δε ομού ανέκραξαν, λέγοντες· Σήκωσον τούτον, απόλυσον δε εις ημάς τον Βαραββάν·
\par 19 όστις διά στάσιν τινά γενομένην εν τη πόλει και διά φόνον ήτο βεβλημένος εις φυλακήν.
\par 20 Πάλιν λοιπόν ο Πιλάτος ελάλησε προς αυτούς, θέλων να απολύση τον Ιησούν.
\par 21 Οι δε εφώναζον, λέγοντες· Σταύρωσον, σταύρωσον αυτόν.
\par 22 Ο δε και τρίτην φοράν είπε προς αυτούς· Και τι κακόν έπραξεν ούτος; ουδεμίαν αιτίαν θανάτου εύρον εν αυτώ· αφού λοιπόν παιδεύσω αυτόν, θέλω απολύσει.
\par 23 Αλλ' εκείνοι επέμενον, με φωνάς μεγάλας ζητούντες να σταυρωθή, και αι φωναί αυτών και των αρχιερέων υπερίσχυον.
\par 24 Και ο Πιλάτος απεφάσισε να γείνη το ζήτημα αυτών,
\par 25 και απέλυσεν εις αυτούς τον διά στάσιν και φόνον βεβλημένον εις την φυλακήν, τον οποίον εζήτουν, τον δε Ιησούν παρέδωκεν εις το θέλημα αυτών.
\par 26 Και καθώς έφεραν αυτόν έξω, επίασαν Σίμωνα τινά Κυρηναίον, ερχόμενον από του αγρού, και έθεσαν επάνω αυτού τον σταυρόν, διά να φέρη αυτόν όπισθεν του Ιησού.
\par 27 Ηκολούθει δε αυτόν πολύ πλήθος του λαού και γυναικών, αίτινες και ωδύροντο και εθρήνουν αυτόν.
\par 28 Στραφείς δε προς αυτάς ο Ιησούς, είπε· θυγατέρες της Ιερουσαλήμ, μη κλαίετε δι' εμέ, αλλά δι' εαυτάς κλαίετε και διά τα τέκνα σας.
\par 29 Διότι ιδού, έρχονται ημέραι καθ' ας θέλουσιν ειπεί· Μακάριαι αι στείραι και αι κοιλίαι, αίτινες δεν εγέννησαν, και οι μαστοί, οίτινες δεν εθήλασαν.
\par 30 Τότε θέλουσιν αρχίσει να λέγωσιν εις τα όρη, Πέσετε εφ' ημάς, και εις τα βουνά, Σκεπάσατε ημάς·
\par 31 διότι εάν εις το υγρόν ξύλον πράττωσι ταύτα, τι θέλει γείνει εις το ξηρόν;
\par 32 Εφέροντο δε και άλλοι δύο μετ' αυτού, οίτινες ήσαν κακούργοι διά να θανατωθώσι.
\par 33 Και ότε ήλθον εις τον τόπον τον ονομαζόμενον Κρανίον, εκεί εσταύρωσαν αυτόν και τους κακούργους, τον μεν εκ δεξιών, τον δε εξ αριστερών.
\par 34 Ο δε Ιησούς έλεγε· Πάτερ, συγχώρησον αυτούς· διότι δεν εξεύρουσι τι πράττουσι. Διαμεριζόμενοι δε τα ιμάτια αυτού, έβαλον κλήρον.
\par 35 Και ίστατο ο λαός θεωρών. Ενέπαιζον δε και οι άρχοντες μετ' αυτών, λέγοντες· Άλλους έσωσεν, ας σώση αυτόν, εάν ούτος ήναι ο Χριστός ο εκλεκτός του Θεού.
\par 36 Ενέπαιζον δε αυτόν και οι στρατιώται, πλησιάζοντες και προσφέροντες όξος εις αυτόν
\par 37 και λέγοντες· Εάν συ ήσαι ο βασιλεύς των Ιουδαίων, σώσον σεαυτόν.
\par 38 Ήτο δε και επιγραφή γεγραμμένη επάνωθεν αυτού με γράμματα Ελληνικά και Ρωμαϊκά και Εβραϊκά· Ούτος εστίν ο Βασιλεύς των Ιουδαίων.
\par 39 Εις δε των κρεμασθέντων κακούργων εβλασφήμει αυτόν, λέγων· Εάν συ ήσαι ο Χριστός, σώσον σεαυτόν και ημάς.
\par 40 Αποκριθείς δε ο άλλος, επέπληττεν αυτόν, λέγων· Ουδέ τον Θεόν δεν φοβείσαι συ, όστις είσαι εν τη αυτή καταδίκη;
\par 41 και ημείς μεν δικαίως· διότι άξια των όσα επράξαμεν απολαμβάνομεν· ούτος όμως ουδέν άτοπον έπραξε.
\par 42 Και έλεγε προς τον Ιησούν· Μνήσθητί μου, Κύριε, όταν έλθης εν τη βασιλεία σου.
\par 43 Και είπε προς αυτόν ο Ιησούς· Αληθώς σοι λέγω, σήμερον θέλεις είσθαι μετ' εμού εν τω παραδείσω.
\par 44 Ήτο δε ως έκτη ώρα και έγεινε σκότος εφ' όλην την γην έως ώρας εννάτης,
\par 45 και εσκοτίσθη ο ήλιος και εσχίσθη εις το μέσον το καταπέτασμα του ναού·
\par 46 και φωνάξας με φωνήν μεγάλην ο Ιησούς είπε· Πάτερ, εις χείρας σου παραδίδω το πνεύμά μου· και ταύτα ειπών εξέπνευσεν.
\par 47 Ιδών δε ο εκατόνταρχος το γενόμενον, εδόξασε τον Θεόν, λέγων· Όντως ο άνθρωπος ούτος ήτο δίκαιος.
\par 48 Και πάντες οι όχλοι οι συνελθόντες εις την θεωρίαν ταύτην, βλέποντες τα γενόμενα, υπέστρεφον τύπτοντες τα στήθη αυτών.
\par 49 Ίσταντο δε μακρόθεν πάντες οι γνωστοί αυτού, και αι γυναίκες αίτινες συνηκολούθησαν αυτόν από της Γαλιλαίας, και έβλεπον ταύτα.
\par 50 Και ιδού, ανήρ τις Ιωσήφ το όνομα, όστις ήτο βουλευτής, ανήρ αγαθός και δίκαιος,
\par 51 ούτος δεν ήτο σύμφωνος με την βουλήν και την πράξιν αυτών, από Αριμαθαίας πόλεως των Ιουδαίων, όστις και αυτός περιέμενε την βασιλείαν του Θεού,
\par 52 ούτος ελθών προς τον Πιλάτον, εζήτησε το σώμα του Ιησού,
\par 53 και καταβιβάσας αυτό ετύλιξεν αυτό με σινδόνα και έθεσεν αυτό εν μνημείω λελατομημένω· όπου ουδείς έτι είχεν ενταφιασθή.
\par 54 Και ήτο ημέρα παρασκευή, και εξημέρονε σάββατον.
\par 55 Ηκολούθησαν δε και γυναίκες, αίτινες είχον ελθεί μετ' αυτού από της Γαλιλαίας, και είδον το μνημείον και πως ετέθη το σώμα αυτού.
\par 56 Και αφού υπέστρεψαν ητοίμασαν αρώματα και μύρα. Και το μεν σάββατον ησύχασαν κατά την εντολήν.

\chapter{24}

\par 1 Την δε πρώτην ημέραν της εβδομάδος, ενώ ήτο όρθρος βαθύς, ήλθον εις το μνήμα φέρουσαι τα οποία ητοίμασαν αρώματα, και άλλαι τινές μετ' αυτών.
\par 2 Εύρον δε τον λίθον αποκεκυλισμένον από του μνημείου,
\par 3 και εισελθούσαι δεν εύρον το σώμα του Κυρίου Ιησού.
\par 4 Και ενώ ήσαν εν απορία περί τούτου, ιδού, δύο άνδρες εστάθησαν έμπροσθεν αυτών με ιμάτια αστράπτοντα.
\par 5 Καθώς δε αύται εφοβήθησαν και έκλινον το πρόσωπον εις την γην, είπον προς αυτάς· Τι ζητείτε τον ζώντα μετά των νεκρών;
\par 6 δεν είναι εδώ, αλλ' ανέστη· ενθυμήθητε πως ελάλησε προς εσάς, ενώ ήτο έτι εν τη Γαλιλαία,
\par 7 λέγων ότι πρέπει ο Υιός του ανθρώπου να παραδοθή εις χείρας ανθρώπων αμαρτωλών και να σταυρωθή και την τρίτην ημέραν να αναστηθή.
\par 8 Και ενεθυμήθησαν τους λόγους αυτού.
\par 9 Και αφού υπέστρεψαν από του μνημείου, απήγγειλαν ταύτα πάντα προς τους ένδεκα και πάντας τους λοιπούς.
\par 10 Ήσαν δε η Μαγδαληνή Μαρία και Ιωάννα και Μαρία η μήτηρ του Ιακώβου και αι λοιπαί μετ' αυτών, αίτινες έλεγον ταύτα προς τους αποστόλους.
\par 11 Και οι λόγοι αυτών εφάνησαν ενώπιον αυτών ως φλυαρία, και δεν επίστευον εις αυτάς.
\par 12 Ο δε Πέτρος σηκωθείς έδραμεν εις το μνημείον, και παρακύψας βλέπει τα σάβανα κείμενα μόνα, και ανεχώρησε, θαυμάζων καθ' εαυτόν το γεγονός.
\par 13 Και ιδού, δύο εξ αυτών επορεύοντο εν αυτή τη ημέρα εις κώμην ονομαζομένην Εμμαούς, απέχουσαν εξήκοντα στάδια από Ιερουσαλήμ.
\par 14 Και αυτοί ωμίλουν προς αλλήλους περί πάντων των συμβεβηκότων τούτων.
\par 15 Και ενώ ωμίλουν και συνδιελέγοντο, πλησιάσας και αυτός ο Ιησούς επορεύετο μετ' αυτών·
\par 16 αλλ' οι οφθαλμοί αυτών εκρατούντο διά να μη γνωρίσωσιν αυτόν.
\par 17 Είπε δε προς αυτούς· Τίνες είναι οι λόγοι ούτοι, τους οποίους συνομιλείτε προς αλλήλους περιπατούντες, και είσθε σκυθρωποί;
\par 18 Αποκριθείς δε ο εις, ονομαζόμενος Κλεόπας, είπε προς αυτόν· Συ μόνος παροικείς εν Ιερουσαλήμ και δεν έμαθες τα γενόμενα εν αυτή εν ταις ημέραις ταύταις;
\par 19 Και είπε προς αυτούς· Ποία; Οι δε είπον προς αυτόν· Τα περί Ιησού του Ναζωραίου, όστις εστάθη ανήρ προφήτης δυνατός εν έργω και λόγω ενώπιον του Θεού και παντός του λαού,
\par 20 και πως παρέδωκαν αυτόν οι αρχιερείς και οι άρχοντες ημών εις καταδίκην θανάτου και εσταύρωσαν αυτόν.
\par 21 Ημείς δε ηλπίζομεν ότι αυτός είναι ο μέλλων να λυτρώση τον Ισραήλ· αλλά και προς τούτοις πάσι τρίτη ημέρα είναι σήμερον αύτη, αφού έγειναν ταύτα.
\par 22 Αλλά και γυναίκές τινές εξ ημών εξέπληξαν ημάς, αίτινες υπήγον την αυγήν εις το μνημείον,
\par 23 και μη ευρούσαι το σώμα αυτού, ήλθον λέγουσαι ότι είδον και οπτασίαν αγγέλων, οίτινες λέγουσιν ότι αυτός ζη.
\par 24 Και τινές των υμετέρων υπήγον εις το μνημείον και εύρον ούτω, καθώς και αι γυναίκες είπον, αυτόν όμως δεν είδον.
\par 25 Και αυτός είπε προς αυτούς· Ω ανόητοι και βραδείς την καρδίαν εις το να πιστεύητε εις πάντα όσα ελάλησαν οι προφήται·
\par 26 δεν έπρεπε να πάθη ταύτα ο Χριστός και να εισέλθη εις την δόξαν αυτού;
\par 27 Και αρχίσας από Μωϋσέως και από πάντων των προφητών, διηρμήνευεν εις αυτούς τα περί εαυτού γεγραμμένα εν πάσαις ταις γραφαίς.
\par 28 Και επλησίασαν εις την κώμην όπου επορεύοντο, και αυτός προσεποιείτο ότι υπάγει μακρότερα·
\par 29 και παρεβίασαν αυτόν, λέγοντες· Μείνον μεθ' ημών, διότι πλησιάζει η εσπέρα και έκλινεν η ημέρα. Και εισήλθε διά να μείνη μετ' αυτών.
\par 30 Και αφού εκάθησε μετ' αυτών εις την τράπεζαν, λαβών τον άρτον ευλόγησε και κόψας έδιδεν εις αυτούς.
\par 31 Αυτών δε διηνοίχθησαν οι οφθαλμοί, και εγνώρισαν αυτόν. Και αυτός έγεινεν άφαντος απ' αυτών.
\par 32 Και είπον προς αλλήλους· Δεν εκαίετο εν υμίν η καρδία ημών, ότε ελάλει προς ημάς καθ' οδόν και μας εξήγει τας γραφάς;
\par 33 Και σηκωθέντες τη αυτή ώρα υπέστρεψαν εις Ιερουσαλήμ, και εύρον συνηθροισμένους τους ένδεκα και τους μετ' αυτών,
\par 34 οίτινες έλεγον ότι όντως ανέστη ο Κύριος και εφάνη εις τον Σίμωνα.
\par 35 Και αυτοί διηγούντο τα εν τη οδώ και πως εγνωρίσθη εις αυτούς, ενώ έκοπτε τον άρτον.
\par 36 Ενώ δε ελάλουν ταύτα, αυτός ο Ιησούς εστάθη εν μέσω αυτών και λέγει προς αυτούς· Ειρήνη υμίν.
\par 37 Εκείνοι δε εκπλαγέντες και έμφοβοι γενόμενοι ενόμιζον ότι έβλεπον πνεύμα.
\par 38 Και είπε προς αυτούς· Διά τι είσθε τεταραγμένοι; και διά τι αναβαίνουσιν εις τας καρδίας σας διαλογισμοί;
\par 39 ίδετε τας χείρας μου και τους πόδας μου, ότι αυτός εγώ είμαι· ψηλαφήσατέ με και ίδετε, διότι πνεύμα σάρκα και οστέα δεν έχει, καθώς εμέ θεωρείτε έχοντα.
\par 40 Και τούτο ειπών, έδειξεν εις αυτούς τας χείρας και τους πόδας.
\par 41 Ενώ δε αυτοί ηπίστουν έτι από της χαράς και εθαύμαζον, είπε προς αυτούς· Έχετε τι φαγώσιμον ενταύθα;
\par 42 Οι δε έδωκαν εις αυτόν μέρος οπτού ιχθύος και από κηρήθραν μέλιτος.
\par 43 Και λαβών ενώπιον αυτών έφαγεν.
\par 44 Είπε δε προς αυτούς· Ούτοι είναι οι λόγοι, τους οποίους ελάλησα προς υμάς ότε ήμην έτι μεθ' υμών, ότι πρέπει να πληρωθώσι πάντα τα γεγραμμένα εν τω νόμω του Μωϋσέως και προφήταις και ψαλμοίς περί εμού.
\par 45 Τότε διήνοιξεν αυτών τον νούν, διά να καταλάβωσι τας γραφάς·
\par 46 και είπε προς αυτούς ότι ούτως είναι γεγραμμένον και ούτως έπρεπε να πάθη ο Χριστός και να αναστηθή εκ νεκρών τη τρίτη ημέρα,
\par 47 και να κηρυχθή εν τω ονόματι αυτού μετάνοια και άφεσις αμαρτιών εις πάντα τα έθνη, γινομένης αρχής από Ιερουσαλήμ.
\par 48 Σεις δε είσθε μάρτυρες τούτων.
\par 49 Και ιδού, εγώ αποστέλλω την επαγγελίαν του Πατρός μου εφ' υμάς· σεις δε καθήσατε εν τη πόλει Ιερουσαλήμ εωσού ενδυθήτε δύναμιν εξ ύψους.
\par 50 Και έφερεν αυτούς έξω έως εις Βηθανίαν, και υψώσας τας χείρας αυτού ευλόγησεν αυτούς.
\par 51 Και ενώ ευλόγει αυτούς, απεχωρίσθη απ' αυτών και ανεφέρετο εις τον ουρανόν.
\par 52 Και αυτοί προσκυνήσαντες αυτόν, υπέστρεψαν εις Ιερουσαλήμ μετά χαράς μεγάλης,
\par 53 και ήσαν διαπαντός εν τω ιερώ, αινούντες και ευλογούντες τον Θεόν. Αμήν.


\end{document}