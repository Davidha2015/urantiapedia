\begin{document}

\title{Ιωάννης}


\chapter{1}

\par 1 Εν αρχή ήτο ο Λόγος, και ο Λόγος ήτο παρά τω Θεώ, και Θεός ήτο ο Λόγος.
\par 2 Ούτος ήτο εν αρχή παρά τω Θεώ.
\par 3 Πάντα δι' αυτού έγειναν, και χωρίς αυτού δεν έγεινεν ουδέ εν, το οποίον έγεινεν.
\par 4 Εν αυτώ ήτο ζωή, και η ζωή ήτο το φως των ανθρώπων.
\par 5 Και το φως εν τη σκοτία φέγγει και η σκοτία δεν κατέλαβεν αυτό.
\par 6 Υπήρξεν άνθρωπος απεσταλμένος παρά Θεού, ονομαζόμενος Ιωάννης·
\par 7 ούτος ήλθεν εις μαρτυρίαν, διά να μαρτυρήση περί του φωτός, διά να πιστεύσωσι πάντες δι' αυτού.
\par 8 Δεν ήτο εκείνος το φως, αλλά διά να μαρτυρήση περί του φωτός.
\par 9 Ήτο το φως το αληθινόν, το οποίον φωτίζει πάντα άνθρωπον ερχόμενον εις τον κόσμον.
\par 10 Ήτο εν τω κόσμω, και ο κόσμος έγεινε δι' αυτού, και ο κόσμος δεν εγνώρισεν αυτόν.
\par 11 Εις τα ίδια ήλθε, και οι ίδιοι δεν εδέχθησαν αυτόν.
\par 12 Όσοι δε εδέχθησαν αυτόν, εις αυτούς έδωκεν εξουσίαν να γείνωσι τέκνα Θεού, εις τους πιστεύοντας εις το όνομα αυτού·
\par 13 οίτινες ουχί εξ αιμάτων ουδέ εκ θελήματος σαρκός ουδέ εκ θελήματος ανδρός, αλλ' εκ Θεού εγεννήθησαν.
\par 14 Και ο Λόγος έγεινε σαρξ και κατώκησε μεταξύ ημών, και είδομεν την δόξαν αυτού, δόξαν ως μονογενούς παρά του Πατρός, πλήρης χάριτος και αληθείας.
\par 15 Ο Ιωάννης μαρτυρεί περί αυτού και εφώναξε, λέγων· Ούτος ήτο περί ου είπον, Ο οπίσω μου ερχόμενος είναι ανώτερος μου, διότι ήτο πρότερός μου.
\par 16 Και πάντες ημείς ελάβομεν εκ του πληρώματος αυτού και χάριν αντί χάριτος·
\par 17 διότι και ο νόμος εδόθη διά του Μωϋσέως· η δε χάρις και αλήθεια έγεινε διά Ιησού Χριστού.
\par 18 Ουδείς είδε ποτέ τον Θεόν· ο μονογενής Υιός, ο ων εις τον κόλπον του Πατρός, εκείνος εφανέρωσεν αυτόν.
\par 19 Και αύτη είναι η μαρτυρία του Ιωάννου, ότε απέστειλαν οι Ιουδαίοι εξ Ιεροσολύμων ιερείς και Λευΐτας διά να ερωτήσωσιν αυτόν· Συ τις είσαι;
\par 20 Και ώμολόγησε και δεν ηρνήθη· και ώμολόγησεν ότι δεν είμαι εγώ ο Χριστός.
\par 21 Και ηρώτησαν αυτόν· Τι λοιπόν; Ηλίας είσαι συ; και λέγει, δεν είμαι. Ο προφήτης είσαι συ; και απεκρίθη, Ουχί.
\par 22 Είπον λοιπόν προς αυτόν· Τις είσαι; διά να δώσωμεν απόκρισιν εις τους αποστείλαντας ημάς· τι λέγεις περί σεαυτού;
\par 23 Απεκρίθη· Εγώ είμαι φωνή βοώντος εν τη ερήμω, ευθύνατε την οδόν του Κυρίου, καθώς είπεν Ησαΐας ο προφήτης.
\par 24 Οι δε απεσταλμένοι ήσαν εκ των Φαρισαίων·
\par 25 και ηρώτησαν αυτόν και είπον προς αυτόν· Διά τι λοιπόν βαπτίζεις, εάν συ δεν είσαι ο Χριστός ούτε ο Ηλίας ούτε ο προφήτης;
\par 26 Απεκρίθη προς αυτούς ο Ιωάννης λέγων· Εγώ βαπτίζω εν ύδατι· εν μέσω δε υμών ίσταται εκείνος, τον οποίον σεις δεν γνωρίζετε·
\par 27 αυτός είναι ο οπίσω μου ερχόμενος, όστις είναι ανώτερός μου, του οποίου εγώ δεν είμαι άξιος να λύσω το λωρίον του υποδήματος αυτού.
\par 28 Ταύτα έγειναν εν Βηθαβαρά πέραν του Ιορδάνου, όπου ήτο ο Ιωάννης βαπτίζων.
\par 29 Τη επαύριον βλέπει ο Ιωάννης τον Ιησούν ερχόμενον προς αυτόν και λέγει· Ιδού, ο Αμνός του Θεού ο αίρων την αμαρτίαν του κόσμου.
\par 30 Ούτος είναι περί ου εγώ είπον· Οπίσω μου έρχεται ανήρ, όστις είναι ανώτερός μου, διότι ήτο πρότερός μου.
\par 31 Και εγώ δεν εγνώριζον αυτόν, αλλά διά να φανερωθή εις τον Ισραήλ, διά τούτο ήλθον εγώ βαπτίζων εν τω ύδατι.
\par 32 Και εμαρτύρησεν ο Ιωάννης, λέγων ότι Είδον το Πνεύμα καταβαίνον ως περιστεράν εξ ουρανού και έμεινεν επ' αυτόν.
\par 33 Και εγώ δεν εγνώριζον αυτόν· αλλ' ο πέμψας με διά να βαπτίζω εν ύδατι εκείνος μοι είπεν· εις όντινα ίδης το Πνεύμα καταβαίνον και μένον επ' αυτόν, ούτος είναι ο βαπτίζων εν Πνεύματι Αγίω.
\par 34 Και εγώ είδον και εμαρτύρησα, ότι ούτος είναι ο Υιός του Θεού.
\par 35 Τη επαύριον πάλιν ίστατο ο Ιωάννης και δύο εκ των μαθητών αυτού,
\par 36 και εμβλέψας εις τον Ιησούν περιπατούντα, λέγει· Ιδού, ο Αμνός του Θεού.
\par 37 Και ήκουσαν αυτόν οι δύο μαθηταί λαλούντα και ηκολούθησαν τον Ιησούν.
\par 38 Στραφείς δε ο Ιησούς και ιδών αυτούς ακολουθούντας, λέγει προς αυτούς· Τι ζητείτε; Οι δε είπον προς αυτόν, Ραββί, το οποίον ερμηνευόμενον λέγεται, Διδάσκαλε, που μένεις;
\par 39 Λέγει προς αυτούς· Έλθετε και ίδετε, ήλθον και είδον που μένει, και έμειναν παρ' αυτώ την ημέραν εκείνην· η δε ώρα ήτο ως δεκάτη.
\par 40 Ήτο Ανδρέας ο αδελφός του Σίμωνος Πέτρου εις εκ των δύο, οίτινες ήκουσαν περί αυτού παρά του Ιωάννου και ηκολούθησαν αυτόν.
\par 41 Ούτος πρώτος ευρίσκει τον εαυτού αδελφόν Σίμωνα και λέγει προς αυτόν· Ευρήκαμεν τον Μεσσίαν, το οποίον μεθερμηνευόμενον είναι ο Χριστός.
\par 42 Και έφερεν αυτόν προς τον Ιησούν. Εμβλέψας δε εις αυτόν ο Ιησούς είπε· Συ είσαι Σίμων, ο υιός του Ιωνά· συ θέλεις ονομασθή Κηφάς, το οποίον ερμηνεύεται Πέτρος.
\par 43 Τη επαύριον ηθέλησεν ο Ιησούς να εξέλθη εις την Γαλιλαίαν· και ευρίσκει τον Φίλιππον και λέγει προς αυτόν· Ακολούθει μοι.
\par 44 Ήτο δε ο Φίλιππος από Βηθσαϊδά, εκ της πόλεως Ανδρέου και Πέτρου.
\par 45 Ευρίσκει Φίλιππος τον Ναθαναήλ και λέγει προς αυτόν· Εκείνον τον οποίον έγραψεν ο Μωϋσής εν τω νόμω και οι προφήται ευρήκαμεν, Ιησούν τον υιόν του Ιωσήφ τον από Ναζαρέτ.
\par 46 Και είπε προς αυτόν ο Ναθαναήλ· Εκ Ναζαρέτ δύναται να προέλθη τι αγαθόν; Λέγει προς αυτόν ο Φίλιππος, Έρχου και ίδε.
\par 47 Είδεν ο Ιησούς τον Ναθαναήλ ερχόμενον προς αυτόν και λέγει περί αυτού· Ιδού, αληθώς Ισραηλίτης, εις τον οποίον δόλος δεν υπάρχει.
\par 48 Λέγει προς αυτόν ο Ναθαναήλ· Πόθεν με γινώσκεις; Απεκρίθη ο Ιησούς και είπε προς αυτόν· Πριν ο Φίλιππος σε φωνάξη, όντα υποκάτω της συκής, είδόν σε.
\par 49 Απεκρίθη ο Ναθαναήλ και λέγει προς αυτόν· Ραββί, συ είσαι ο Υιός του Θεού, συ είσαι ο βασιλεύς του Ισραήλ.
\par 50 Απεκρίθη ο Ιησούς και είπε προς αυτόν· Επειδή σοι είπον· είδόν σε υποκάτω της συκής, πιστεύεις; μεγαλήτερα τούτων θέλεις ιδεί.
\par 51 Και λέγει προς αυτόν· Αληθώς, αληθώς σας λέγω· από του νυν θέλετε ιδεί τον ουρανόν ανεωγμένον και τους αγγέλους του Θεού αναβαίνοντας και καταβαίνοντας επί τον Υιόν του ανθρώπου.

\chapter{2}

\par 1 Και την τρίτην ημέραν έγεινε γάμος εν Κανά της Γαλιλαίας, και ήτο η μήτηρ του Ιησού εκεί.
\par 2 Προσεκλήθη δε και ο Ιησούς και οι μαθηταί αυτού εις τον γάμον.
\par 3 Και επειδή έλειψεν ο οίνος, λέγει η μήτηρ του Ιησού προς αυτόν· Οίνον δεν έχουσι.
\par 4 Λέγει προς αυτήν ο Ιησούς· Τι είναι μεταξύ εμού και σου, γύναι; δεν ήλθεν έτι η ώρα μου.
\par 5 Λέγει η μήτηρ αυτού προς τους υπηρέτας· ό,τι σας λέγει, κάμετε.
\par 6 Ήσαν δε εκεί υδρίαι λίθιναι εξ κείμεναι κατά το έθος του καθαρισμού των Ιουδαίων, χωρούσαι εκάστη δύο ή τρία μέτρα.
\par 7 Λέγει προς αυτούς ο Ιησούς· Γεμίσατε τας υδρίας ύδατος. Και εγέμισαν αυτάς έως άνω.
\par 8 Και λέγει προς αυτούς· Αντλήσατε τώρα και φέρετε προς τον αρχιτρίκλινον. Και έφεραν.
\par 9 Καθώς δε ο αρχιτρίκλινος εγεύθη το ύδωρ εις οίνον μεταβεβλημένον και δεν ήξευρε πόθεν είναι, οι υπηρέται όμως ήξευρον οι αντλήσαντες το ύδωρ φωνάζει τον νυμφίον ο αρχιτρίκλινος
\par 10 και λέγει προς αυτόν· Πας άνθρωπος πρώτον τον καλόν οίνον βάλλει, και αφού πίωσι πολύ, τότε τον κατώτερον· συ εφύλαξας τον καλόν οίνον έως τώρα.
\par 11 Ταύτην την αρχήν των θαυμάτων έκαμεν ο Ιησούς εν Κανά της Γαλιλαίας και εφανέρωσε την δόξαν αυτού, και επίστευσαν εις αυτόν οι μαθηταί αυτού.
\par 12 Μετά τούτο κατέβη εις Καπερναούμ αυτός και η μήτηρ αυτού και οι αδελφοί αυτού και οι μαθηταί αυτού, και εκεί έμειναν ουχ πολλάς ημέρας.
\par 13 Επλησίαζε δε το πάσχα των Ιουδαίων, και ανέβη εις Ιεροσόλυμα ο Ιησούς.
\par 14 Και εύρεν εν τω ιερώ, τους πωλούντας βόας και πρόβατα και περιστεράς, και τους αργυραμοιβούς καθημένους.
\par 15 Και ποιήσας μάστιγα εκ σχοινίων, εδίωξε πάντας εκ του ιερού και τα πρόβατα και τους βόας, και τα νομίσματα των αργυραμοιβών έχυσε και τας τραπέζας ανέτρεψε,
\par 16 και προς τους πωλούντας τας περιστεράς είπε· Σηκώσατε ταύτα εντεύθεν· μη κάμνετε τον οίκον του Πατρός μου οίκον εμπορίου.
\par 17 Τότε ενεθυμήθησαν οι μαθηταί αυτού ότι είναι γεγραμμένον, Ο ζήλος του οίκου σου με κατέφαγεν.
\par 18 Απεκρίθησαν λοιπόν οι Ιουδαίοι και είπον προς αυτόν· Τι σημείον δεικνύεις εις ημάς, διότι κάμνεις ταύτα;
\par 19 Απεκρίθη ο Ιησούς και είπε προς αυτούς· Χαλάσατε τον ναόν τούτον, και διά τριών ημερών θέλω εγείρει αυτόν.
\par 20 Και οι Ιουδαίοι είπον· Εις τεσσαράκοντα και εξ έτη ωκοδομήθη ο ναός ούτος, και συ θέλεις εγείρει αυτόν εις τρεις ημέρας;
\par 21 Εκείνος όμως έλεγε περί του ναού του σώματος αυτού.
\par 22 Ότε λοιπόν ηγέρθη εκ νεκρών, ενεθυμήθησαν οι μαθηταί αυτού ότι τούτο έλεγε προς αυτούς, και επίστευσαν εις την γραφήν και εις τον λόγον, τον οποίον είπεν ο Ιησούς.
\par 23 Και ενώ ήτο εν Ιεροσολύμοις κατά την εορτήν του πάσχα, πολλοί επίστευσαν εις το όνομα αυτού, βλέποντες αυτού τα θαύματα, τα οποία έκαμνεν.
\par 24 Αυτός δε ο Ιησούς δεν ενεπιστεύετο εις αυτούς, διότι εγνώριζε πάντας,
\par 25 και διότι δεν είχε χρείαν διά να μαρτυρήση τις περί του ανθρώπου· επειδή αυτός εγνώριζε τι ήτο εντός του ανθρώπου.

\chapter{3}

\par 1 Ήτο δε άνθρωπός τις εκ των Φαρισαίων, Νικόδημος ονομαζόμενος, άρχων των Ιουδαίων.
\par 2 Ούτος ήλθε προς τον Ιησούν διά νυκτός και είπε προς αυτόν· Ραββί, εξεύρομεν ότι από Θεού ήλθες διδάσκαλος· διότι ουδείς δύναται να κάμνη τα σημεία ταύτα, τα οποία συ κάμνεις, εάν δεν ήναι ο Θεός μετ' αυτού.
\par 3 Απεκρίθη ο Ιησούς και είπε προς αυτόν· Αληθώς, αληθώς σοι λέγω, εάν τις δεν γεννηθή άνωθεν, δεν δύναται να ίδη την βασιλείαν του Θεού.
\par 4 Λέγει προς αυτόν ο Νικόδημος· Πως δύναται άνθρωπος να γεννηθή γέρων ων; μήποτε δύναται να εισέλθη δευτέραν φοράν εις την κοιλίαν της μητρός αυτού και να γεννηθή;
\par 5 Απεκρίθη ο Ιησούς· Αληθώς, αληθώς σοι λέγω, εάν τις δεν γεννηθή εξ ύδατος και Πνεύματος, δεν δύναται να εισέλθη εις την βασιλείαν του Θεού.
\par 6 Το γεγεννημένον εκ της σαρκός είναι σαρξ και το γεγεννημένον εκ του Πνεύματος είναι πνεύμα.
\par 7 Μη θαυμάσης ότι σοι είπον, Πρέπει να γεννηθήτε άνωθεν.
\par 8 Ο άνεμος όπου θέλει πνέει, και την φωνήν αυτού ακούεις, αλλά δεν εξεύρεις πόθεν έρχεται και που υπάγει· ούτως είναι πας, όστις εγεννήθη εκ του Πνεύματος.
\par 9 Απεκρίθη ο Νικόδημος και είπε προς αυτόν· Πως δύνανται να γείνωσι ταύτα;
\par 10 Απεκρίθη ο Ιησούς και είπε προς αυτόν· Συ είσαι ο διδάσκαλος του Ισραήλ και ταύτα δεν εξεύρεις;
\par 11 Αληθώς, αληθώς σοι λέγω ότι εκείνο το οποίον εξεύρομεν λαλούμεν και εκείνο το οποίον είδομεν μαρτυρούμεν, και την μαρτυρίαν ημών δεν δέχεσθε.
\par 12 Εάν τα επίγεια σας είπον και δεν πιστεύητε, πως, εάν σας είπω τα επουράνια, θέλετε πιστεύσει;
\par 13 Και ουδείς ανέβη εις τον ουρανόν ειμή ο καταβάς εκ του ουρανού, ο Υιός του ανθρώπου, ο ων εν τω ουρανώ.
\par 14 Και καθώς ο Μωϋσής ύψωσε τον όφιν εν τη ερήμω, ούτω πρέπει να υψωθή ο Υιός του ανθρώπου,
\par 15 διά να μη απολεσθή πας ο πιστεύων εις αυτόν, αλλά να έχη ζωήν αιώνιον.
\par 16 Διότι τόσον ηγάπησεν ο Θεός τον κόσμον, ώστε έδωκε τον Υιόν αυτού τον μονογενή, διά να μη απολεσθή πας ο πιστεύων εις αυτόν, αλλά να έχη ζωήν αιώνιον.
\par 17 Επειδή δεν απέστειλεν ο Θεός τον Υιόν αυτού εις τον κόσμον διά να κρίνη τον κόσμον, αλλά διά να σωθή ο κόσμος δι' αυτού.
\par 18 Όστις πιστεύει εις αυτόν δεν κρίνεται, όστις όμως δεν πιστεύει είναι ήδη κεκριμένος, διότι δεν επίστευσεν εις το όνομα του μονογενούς Υιού του Θεού.
\par 19 Και αύτη είναι η κρίσις, ότι το φως ήλθεν εις τον κόσμον, και οι άνθρωποι ηγάπησαν το σκότος μάλλον παρά το φώς· διότι ήσαν πονηρά τα έργα αυτών.
\par 20 Επειδή πας, όστις πράττει φαύλα, μισεί το φως και δεν έρχεται εις το φως, διά να μη ελεγχθώσι τα έργα αυτού·
\par 21 όστις όμως πράττει την αλήθειαν, έρχεται εις το φως, διά να φανερωθώσι τα έργα αυτού ότι επράχθησαν κατά Θεόν.
\par 22 Μετά ταύτα ήλθεν ο Ιησούς και οι μαθηταί αυτού εις την γην της Ιουδαίας, και εκεί διέτριβε μετ' αυτών και εβάπτιζεν.
\par 23 Ήτο δε και ο Ιωάννης βαπτίζων εν Αινών πλησίον του Σαλείμ, διότι ήσαν εκεί ύδατα πολλά, και ήρχοντο και εβαπτίζοντο·
\par 24 Επειδή ο Ιωάννης δεν ήτο έτι βεβλημένος εις την φυλακήν.
\par 25 Έγεινε λοιπόν συζήτησις περί καθαρισμού παρά των μαθητών του Ιωάννου με Ιουδαίους τινάς.
\par 26 Και ήλθον προς τον Ιωάννην και είπον προς αυτόν· Ραββί, εκείνος όστις ήτο μετά σου πέραν του Ιορδάνου, εις τον οποίον συ εμαρτύρησας, ιδού, ούτος βαπτίζει και πάντες έρχονται προς αυτόν.
\par 27 Απεκρίθη ο Ιωάννης και είπε· Δεν δύναται ο άνθρωπος να λαμβάνη ουδέν, εάν δεν ήναι δεδομένον εις αυτόν εκ του ουρανού.
\par 28 Σεις αυτοί είσθε μάρτυρές μου ότι είπον· Δεν είμαι εγώ ο Χριστός, αλλ' ότι είμαι απεσταλμένος έμπροσθεν εκείνου.
\par 29 Όστις έχει την νύμφην είναι νυμφίος· ο δε φίλος του νυμφίου, ο ιστάμενος και ακούων αυτόν, χαίρει καθ' υπερβολήν διά την φωνήν του νυμφίου. Αύτη λοιπόν η χαρά η ιδική μου επληρώθη.
\par 30 Εκείνος πρέπει να αυξάνη, εγώ δε να ελαττόνωμαι.
\par 31 Ο ερχόμενος άνωθεν είναι υπεράνω πάντων. Ο ων εκ της γης εκ της γης είναι και εκ της γης λαλεί· ο ερχόμενος εκ του ουρανού είναι υπεράνω πάντων,
\par 32 και εκείνο το οποίον είδε και ήκουσε, τούτο μαρτυρεί, και ουδείς δέχεται την μαρτυρίαν αυτού.
\par 33 Όστις δεχθή την μαρτυρίαν αυτού επεσφράγισεν ότι ο Θεός είναι αληθής.
\par 34 Διότι εκείνος, τον οποίον απέστειλεν ο Θεός, τους λόγους του Θεού λαλεί· επειδή ο Θεός δεν δίδει εις αυτόν το Πνεύμα με μέτρον.
\par 35 Ο Πατήρ αγαπά τον Υιόν και πάντα έδωκεν εις την χείρα αυτού.
\par 36 Όστις πιστεύει εις τον Υιόν έχει ζωήν αιώνιον· όστις όμως απειθεί εις τον Υιόν δεν θέλει ιδεί ζωήν, αλλ' η οργή του Θεού μένει επάνω αυτού.

\chapter{4}

\par 1 Καθώς λοιπόν έμαθεν ο Κύριος ότι ήκουσαν οι Φαρισαίοι ότι ο Ιησούς πλειοτέρους μαθητάς κάμνει και βαπτίζει παρά ο Ιωάννης-
\par 2 αν και ο Ιησούς αυτός δεν εβάπτιζεν, αλλ' οι μαθηταί αυτού-
\par 3 αφήκε την Ιουδαίαν και απήλθε πάλιν εις την Γαλιλαίαν.
\par 4 Έπρεπε δε να περάση διά της Σαμαρείας.
\par 5 Έρχεται λοιπόν εις πόλιν της Σαμαρείας λεγομένην Σιχάρ, πλησίον του αγρού, τον οποίον έδωκεν ο Ιακώβ εις τον Ιωσήφ τον υιόν αυτού.
\par 6 Ήτο δε εκεί πηγή του Ιακώβ. Ο Ιησούς λοιπόν κεκοπιακώς εκ της οδοιπορίας εκάθητο ούτως εις την πηγήν. Ώρα ήτο περίπου έκτη.
\par 7 Έρχεται γυνή τις εκ της Σαμαρείας, διά να αντλήση ύδωρ. Λέγει προς αυτήν ο Ιησούς· Δος μοι να πίω.
\par 8 Διότι οι μαθηταί αυτού είχον υπάγει εις την πόλιν, διά να αγοράσωσι τροφάς.
\par 9 Λέγει λοιπόν προς αυτόν η γυνή η Σαμαρείτις· Πως συ, Ιουδαίος ων, ζητείς να πίης παρ' εμού, ήτις είμαι γυνή Σαμαρείτις; Διότι δεν συγκοινωνούσιν οι Ιουδαίοι με τους Σαμαρείτας.
\par 10 Απεκρίθη ο Ιησούς και είπε προς αυτήν· Εάν ήξευρες την δωρεάν του Θεού, και τις είναι ο λέγων σοι, Δος μοι να πίω, συ ήθελες ζητήσει παρ' αυτού, και ήθελε σοι δώσει ύδωρ ζων.
\par 11 Λέγει προς αυτόν η γυνή· Κύριε, ούτε άντλημα έχεις, και το φρέαρ είναι βαθύ· πόθεν λοιπόν έχεις το ύδωρ το ζων;
\par 12 μήπως συ είσαι μεγαλήτερος του πατρός ημών Ιακώβ, όστις έδωκεν εις ημάς το φρέαρ, και αυτός έπιεν εξ αυτού και οι υιοί αυτού και τα θρέμματα αυτού;
\par 13 Απεκρίθη ο Ιησούς και είπε προς αυτήν· Πας όστις πίνει εκ του ύδατος τούτου θέλει διψήσει πάλιν·
\par 14 όστις όμως πίη εκ του ύδατος, το οποίον εγώ θέλω δώσει εις αυτόν, δεν θέλει διψήσει εις τον αιώνα, αλλά το ύδωρ, το οποίον θέλω δώσει εις αυτόν, θέλει γείνει εν αυτώ πηγή ύδατος αναβλύζοντος εις ζωήν αιώνιον.
\par 15 Λέγει προς αυτόν η γυνή· Κύριε, δος μοι τούτο το ύδωρ, διά να μη διψώ μηδέ να έρχωμαι εδώ να αντλώ.
\par 16 Λέγει προς αυτήν ο Ιησούς· Ύπαγε, κάλεσον τον άνδρα σου και ελθέ εδώ.
\par 17 Απεκρίθη η γυνή και είπε· Δεν έχω άνδρα. Λέγει προς αυτήν ο Ιησούς· Καλώς είπας ότι δεν έχω άνδρα·
\par 18 διότι πέντε άνδρας έλαβες, και εκείνος, τον οποίον έχεις τώρα, δεν είναι ανήρ σου· τούτο αληθές είπας.
\par 19 Λέγει προς αυτόν η γυνή· Κύριε, βλέπω ότι συ είσαι προφήτης.
\par 20 Οι πατέρες ημών εις τούτο το όρος προσεκύνησαν, και σεις λέγετε ότι εν τοις Ιεροσολύμοις είναι ο τόπος όπου πρέπει να προσκυνώμεν.
\par 21 Λέγει προς αυτήν ο Ιησούς· Γύναι, πίστευσόν μοι ότι έρχεται ώρα, ότε ούτε εις το όρος τούτο ούτε εις τα Ιεροσόλυμα θέλετε προσκυνήσει τον Πατέρα.
\par 22 Σεις προσκυνείτε εκείνο το οποίον δεν εξεύρετε, ημείς προσκυνούμεν εκείνο το οποίον εξεύρομεν, διότι η σωτηρία είναι εκ των Ιουδαίων.
\par 23 Πλην έρχεται ώρα, και ήδη είναι, ότε οι αληθινοί προσκυνηταί θέλουσι προσκυνήσει τον Πατέρα εν πνεύματι και αληθεία· διότι ο Πατήρ τοιούτους ζητεί τους προσκυνούντας αυτόν.
\par 24 Ο Θεός είναι Πνεύμα, και οι προσκυνούντες αυτόν εν πνεύματι και αληθεία πρέπει να προσκυνώσι.
\par 25 Λέγει προς αυτόν η γυνή· Εξεύρω ότι έρχεται ο Μεσσίας, ο λεγόμενος Χριστός· όταν έλθη εκείνος, θέλει αναγγείλει εις ημάς πάντα.
\par 26 Λέγει προς αυτήν ο Ιησούς· Εγώ είμαι, ο λαλών σοι.
\par 27 Και επάνω εις τούτο ήλθον οι μαθηταί αυτού και εθαύμασαν ότι ελάλει μετά γυναικός· ουδείς όμως είπε, Τι ζητείς; ή Τι λαλείς μετ' αυτής;
\par 28 Αφήκε λοιπόν η γυνή την υδρίαν αυτής και υπήγεν εις την πόλιν και λέγει προς τους ανθρώπους·
\par 29 Έλθετε να ίδητε άνθρωπον, όστις μοι είπε πάντα όσα έπραξα· μήπως ούτος είναι ο Χριστός;
\par 30 Εξήλθον λοιπόν εκ της πόλεως και ήρχοντο προς αυτόν.
\par 31 Εν δε τω μεταξύ οι μαθηταί παρεκάλουν αυτόν λέγοντες· Ραββί, φάγε.
\par 32 Ο δε είπε προς αυτούς. Εγώ έχω φαγητόν να φάγω, το οποίον σεις δεν εξεύρετε.
\par 33 Έλεγον λοιπόν οι μαθηταί προς αλλήλους· Μήπως τις έφερε προς αυτόν να φάγη;
\par 34 Λέγει προς αυτούς ο Ιησούς· Το εμόν φαγητόν είναι να πράττω το θέλημα του πέμψαντός με και να τελειώσω το έργον αυτού.
\par 35 Δεν λέγετε σεις ότι τέσσαρες μήνες είναι έτι και ο θερισμός έρχεται; Ιδού, σας λέγω, υψώσατε τους οφθαλμούς σας και ίδετε τα χωράφια, ότι είναι ήδη λευκά προς θερισμόν.
\par 36 Και ο θερίζων λαμβάνει μισθόν και συνάγει καρπόν εις ζωήν αιώνιον, διά να χαίρη ομού και ο σπείρων και ο θερίζων.
\par 37 Διότι κατά τούτο αληθεύει ο λόγος, ότι άλλος είναι ο σπείρων και άλλος ο θερίζων.
\par 38 Εγώ σας απέστειλα να θερίζητε εκείνο, εις το οποίον σεις δεν εκοπιάσατε· άλλοι εκοπίασαν, και σεις εισήλθετε εις τον κόπον αυτών.
\par 39 Εξ εκείνης δε της πόλεως πολλοί των Σαμαρειτών επίστευσαν εις αυτόν διά τον λόγον της γυναικός, μαρτυρούσης ότι μοι είπε πάντα όσα έπραξα.
\par 40 Καθώς λοιπόν ήλθον προς αυτόν οι Σαμαρείται, παρεκάλουν αυτόν να μείνη παρ' αυτοίς· και έμεινεν εκεί δύο ημέρας.
\par 41 Και πολύ πλειότεροι επίστευσαν διά τον λόγον αυτού,
\par 42 και προς την γυναίκα έλεγον, ότι δεν πιστεύομεν πλέον διά τον λόγον σου· επειδή ημείς ηκούσαμεν, και γνωρίζομεν ότι ούτος είναι αληθώς ο Σωτήρ του κόσμου, ο Χριστός.
\par 43 Μετά δε τας δύο ημέρας εξήλθεν εκείθεν και υπήγεν εις την Γαλιλαίαν.
\par 44 Διότι αυτός ο Ιησούς εμαρτύρησεν ότι προφήτης εν τη πατρίδι αυτού δεν έχει τιμήν.
\par 45 Ότε λοιπόν ήλθεν εις την Γαλιλαίαν, εδέχθησαν αυτόν οι Γαλιλαίοι, ιδόντες πάντα όσα έκαμεν εν Ιεροσολύμοις κατά την εορτήν· διότι και αυτοί ήλθον εις την εορτήν.
\par 46 Ήλθε λοιπόν ο Ιησούς πάλιν εις την Κανά της Γαλιλαίας, όπου έκαμε το ύδωρ οίνον. Και ήτο τις βασιλικός άνθρωπος, του οποίου ο υιός ησθένει εν Καπερναούμ·
\par 47 ούτος ακούσας ότι ο Ιησούς ήλθεν εκ της Ιουδαίας εις την Γαλιλαίαν, υπήγε προς αυτόν και παρεκάλει αυτόν να καταβή και να ιατρεύση τον υιόν αυτού· διότι έμελλε να αποθάνη.
\par 48 Είπε λοιπόν ο Ιησούς προς αυτόν· Εάν δεν ίδητε σημεία και τέρατα, δεν θέλετε πιστεύσει.
\par 49 Λέγει προς αυτόν ο βασιλικός· Κύριε, κατάβα πριν αποθάνη το παιδίον μου.
\par 50 Λέγει προς αυτόν ο Ιησούς· Ύπαγε, ο υιός σου ζη. Και επίστευσεν ο άνθρωπος εις τον λόγον, τον οποίον είπε προς αυτόν ο Ιησούς, και ανεχώρει.
\par 51 Ενώ δε ούτος ήδη κατέβαινεν, απήντησαν αυτόν οι δούλοι αυτού και απήγγειλαν λέγοντες ότι ο υιός σου ζη.
\par 52 Ηρώτησε λοιπόν αυτούς την ώραν, καθ' ην έγεινε καλήτερα. Και είπον προς αυτόν ότι Χθες την εβδόμην ώραν αφήκεν αυτόν ο πυρετός.
\par 53 Ενόησε λοιπόν ο πατήρ ότι έγεινε τούτο κατ' εκείνην την ώραν, καθ' ην ο Ιησούς είπε προς αυτόν ότι Ο υιός σου ζή· και επίστευσεν αυτός και όλη η οικία αυτού.
\par 54 Τούτο πάλιν δεύτερον θαύμα έκαμεν ο Ιησούς, αφού ήλθεν εκ της Ιουδαίας εις την Γαλιλαίαν.

\chapter{5}

\par 1 Μετά ταύτα ήτο εορτήν των Ιουδαίων, και ανέβη ο Ιησούς εις Ιεροσόλυμα.
\par 2 Είναι δε εν τοις Ιεροσολύμοις πλησίον της προβατικής πύλης κολυμβήθρα, η επονομαζομένη Εβραϊστί Βηθεσδά, έχουσα πέντε στοάς.
\par 3 Εν ταύταις κατέκειτο πλήθος πολύ των ασθενούντων, τυφλών, χωλών, ξηρών, οίτινες περιέμενον την κίνησιν του ύδατος.
\par 4 Διότι άγγελος κατέβαινε κατά καιρόν εις την κολυμβήθραν και ετάραττε το ύδωρ· όστις λοιπόν εισήρχετο πρώτος μετά την ταραχήν του ύδατος, εγίνετο υγιής από οποιανδήποτε νόσον έπασχεν.
\par 5 Ήτο δε εκεί άνθρωπός τις τριάκοντα οκτώ έτη πάσχων ασθένειαν.
\par 6 Τούτον ιδών ο Ιησούς κατακείμενον, και εξεύρων ότι πολύν ήδη καιρόν πάσχει, λέγει προς αυτόν· Θέλεις να γείνης υγιής;
\par 7 Απεκρίθη προς αυτόν ο ασθενών· Κύριε, άνθρωπον δεν έχω, διά να με βάλη εις την κολυμβήθραν, όταν ταραχθή το ύδωρ· ενώ δε έρχομαι εγώ, άλλος προ εμού καταβαίνει.
\par 8 Λέγει προς αυτόν ο Ιησούς· Εγέρθητι, σήκωσον τον κράββατόν σου και περιπάτει.
\par 9 Και ευθύς έγεινεν ο άνθρωπος υγιής και εσήκωσε τον κράββατον αυτού, και περιεπάτει. Ήτο δε σάββατον εκείνην την ημέραν.
\par 10 Έλεγον λοιπόν οι Ιουδαίοι προς τον τεθεραπευμένον· Σάββατον είναι· Δεν σοι είναι συγκεχωρημένον να σηκώσης τον κράββατον.
\par 11 Απεκρίθη προς αυτούς· Ο ιατρεύσας με, εκείνος μοι είπε· Σήκωσον τον κράββατόν σου, και περιπάτει.
\par 12 Ηρώτησαν λοιπόν αυτόν· Τις είναι ο άνθρωπος, όστις σοι είπε, Σήκωσον τον κράββατόν σου και περιπάτει;
\par 13 Ο δε ιατρευθείς δεν ήξευρε τις είναι· διότι ο Ιησούς υπεξήλθεν, επειδή ήτο όχλος πολύς εν τω τόπω.
\par 14 Μετά ταύτα ευρίσκει αυτόν ο Ιησούς εν τω ιερώ και είπε προς αυτόν· Ιδού, έγεινες υγιής· μηκέτι αμάρτανε, διά να μη σοι γείνη τι χειρότερον.
\par 15 Υπήγε λοιπόν ο άνθρωπος και ανήγγειλε προς τους Ιουδαίους ότι ο Ιησούς είναι ο ιατρεύσας αυτόν.
\par 16 Και διά τούτο κατέτρεχον τον Ιησούν οι Ιουδαίοι και εζήτουν να θανατώσωσιν αυτόν, διότι έκαμνε ταύτα εν σαββάτω.
\par 17 Ο δε Ιησούς απεκρίθη προς αυτούς· Ο Πατήρ μου εργάζεται έως τώρα, και εγώ εργάζομαι.
\par 18 Διά τούτο λοιπόν μάλλον εζήτουν οι Ιουδαίοι να θανατώσωσιν αυτόν, διότι ουχί μόνον παρέβαινε το σάββατον, αλλά και Πατέρα εαυτού έλεγε τον Θεόν, ίσον με τον Θεόν κάμνων εαυτόν.
\par 19 Απεκρίθη λοιπόν ο Ιησούς και είπε προς αυτούς· Αληθώς, αληθώς σας λέγω, δεν δύναται ο Υιός να πράττη ουδέν αφ' εαυτού, εάν δεν βλέπη τον Πατέρα πράττοντα τούτο· επειδή όσα εκείνος πράττει, ταύτα και ο Υιός πράττει ομοίως.
\par 20 Διότι ο Πατήρ αγαπά τον Υιόν και δεικνύει εις αυτόν πάντα όσα αυτός πράττει, και μεγαλήτερα τούτων έργα θέλει δείξει εις αυτόν, διά να θαυμάζητε σεις.
\par 21 Επειδή καθώς ο Πατήρ εγείρει τους νεκρούς και ζωοποιεί, ούτω και ο Υιός ούστινας θέλει ζωοποιεί.
\par 22 Επειδή ουδέ κρίνει ο Πατήρ ουδένα, αλλ' εις τον Υιόν έδωκε πάσαν την κρίσιν,
\par 23 διά να τιμώσι πάντες τον Υιόν καθώς τιμώσι τον Πατέρα. Ο μη τιμών τον Υιόν δεν τιμά τον Πατέρα τον πέμψαντα αυτόν.
\par 24 Αληθώς, αληθώς σας λέγω ότι ο ακούων τον λόγον μου και πιστεύων εις τον πέμψαντά με έχει ζωήν αιώνιον, και εις κρίσιν δεν έρχεται, αλλά μετέβη εκ του θανάτου εις την ζωήν.
\par 25 Αληθώς, αληθώς σας λέγω ότι έρχεται ώρα, και ήδη είναι, ότε οι νεκροί θέλουσιν ακούσει την φωνήν του Υιού του Θεού, και οι ακούσαντες θέλουσι ζήσει.
\par 26 Διότι καθώς ο Πατήρ έχει ζωήν εν εαυτώ, ούτως έδωκε και εις τον Υιόν να έχη ζωήν εν εαυτώ·
\par 27 και εξουσίαν έδωκεν εις αυτόν να κάμνη και κρίσιν, διότι είναι Υιός ανθρώπου.
\par 28 Μη θαυμάζετε τούτο· διότι έρχεται ώρα, καθ' ην πάντες οι εν τοις μνημείοις θέλουσιν ακούσει την φωνήν αυτού,
\par 29 και θέλουσιν εξέλθει οι πράξαντες τα αγαθά εις ανάστασιν ζωής, οι δε πράξαντες τα φαύλα εις ανάστασιν κρίσεως.
\par 30 Δεν δύναμαι εγώ να κάμνω απ' εμαυτού ουδέν. Καθώς ακούω κρίνω, και η κρίσις η εμή δικαία είναι· διότι δεν ζητώ το θέλημα το εμόν, αλλά το θέλημα του πέμψαντός με Πατρός.
\par 31 Εάν εγώ μαρτυρώ περί εμαυτού, η μαρτυρία μου δεν είναι αληθής.
\par 32 Άλλος είναι ο μαρτυρών περί εμού, και εξεύρω ότι είναι αληθής η μαρτυρία, την οποίαν μαρτυρεί περί εμού.
\par 33 Σεις απεστείλατε προς τον Ιωάννην, και εμαρτύρησεν εις την αλήθειαν·
\par 34 εγώ δε παρά ανθρώπου δεν λαμβάνω την μαρτυρίαν, αλλά λέγω ταύτα διά να σωθήτε σεις.
\par 35 Εκείνος ήτο ο λύχνος ο καιόμενος και φέγγων, και σεις ηθελήσατε να αγαλλιασθήτε προς ώραν εις το φως αυτού.
\par 36 Αλλ' εγώ έχω την μαρτυρίαν μεγαλητέραν της του Ιωάννου· διότι τα έργα, τα οποία μοι έδωκεν ο Πατήρ διά να τελειώσω αυτά, αυτά τα έργα, τα οποία εγώ πράττω, μαρτυρούσι περί εμού ότι ο Πατήρ με απέστειλε·
\par 37 και ο πέμψας με Πατήρ, αυτός εμαρτύρησε περί εμού. Ούτε φωνήν αυτού ηκούσατε πώποτε ούτε όψιν αυτού είδετε.
\par 38 Και τον λόγον αυτού δεν έχετε μένοντα εν εαυτοίς, διότι σεις δεν πιστεύετε εις τούτον, τον οποίον εκείνος απέστειλεν.
\par 39 Ερευνάτε τας γραφάς, διότι σεις νομίζετε ότι εν αυταίς έχετε ζωήν αιώνιον· και εκείναι είναι αι μαρτυρούσαι περί εμού·
\par 40 πλην δεν θέλετε να έλθητε προς εμέ, διά να έχητε ζωήν.
\par 41 Δόξαν παρά ανθρώπων δεν λαμβάνω·
\par 42 αλλά σας εγνώρισα ότι την αγάπην του Θεού δεν έχετε εν εαυτοίς·
\par 43 εγώ ήλθον εν τω ονόματι του Πατρός μου, και δεν με δέχεσθε· εάν άλλος έλθη εν τω ονόματι εαυτού, εκείνον θέλετε δεχθή.
\par 44 Πως δύνασθε σεις να πιστεύσητε, οίτινες λαμβάνετε δόξαν ο εις παρά του άλλου, και δεν ζητείτε την δόξαν την παρά του μόνου Θεού;
\par 45 Μη νομίζετε ότι εγώ θέλω σας κατηγορήσει προς τον Πατέρα· υπάρχει ο κατήγορός σας ο Μωϋσής, εις τον οποίον σεις ηλπίσατε.
\par 46 Διότι εάν επιστεύετε εις τον Μωϋσήν, ηθέλετε πιστεύσει εις εμέ· επειδή περί εμού εκείνος έγραψεν.
\par 47 Εάν δε εις τα γεγραμμένα εκείνου δεν πιστεύητε, πως θέλετε πιστεύσει εις τους ιδικούς μου λόγους;

\chapter{6}

\par 1 Μετά ταύτα ανεχώρησεν ο Ιησούς πέραν της θαλάσσης της Γαλιλαίας της Τιβεριάδος·
\par 2 και ηκολούθει αυτόν όχλος πολύς, διότι έβλεπον τα θαύματα αυτού, τα οποία έκαμνεν επί των ασθενούντων.
\par 3 Ανέβη δε εις το όρος ο Ιησούς και εκεί εκάθητο μετά των μαθητών αυτού.
\par 4 Επλησίαζε δε το πάσχα, η εορτή των Ιουδαίων.
\par 5 Υψώσας λοιπόν ο Ιησούς τους οφθαλμούς και ιδών ότι πολύς όχλος έρχεται προς αυτόν, λέγει προς τον Φίλιππον· Πόθεν θέλομεν αγοράσει άρτους, διά να φάγωσιν ούτοι;
\par 6 Έλεγε δε τούτο δοκιμάζων αυτόν· διότι αυτός ήξευρε τι έμελλε να κάμη.
\par 7 Απεκρίθη προς αυτόν ο Φίλιππος· Διακοσίων δηναρίων άρτοι δεν αρκούσιν εις αυτούς, διά να λάβη ολίγον τι έκαστος αυτών.
\par 8 Λέγει προς αυτόν εις εκ των μαθητών αυτού, Ανδρέας ο αδελφός Σίμωνος Πέτρου·
\par 9 Εδώ είναι εν παιδάριον, το οποίον έχει πέντε άρτους κριθίνους και δύο οψάρια· αλλά ταύτα τι είναι εις τοσούτους;
\par 10 Είπε δε ο Ιησούς· Κάμετε τους ανθρώπους να καθήσωσιν· ήτο δε χόρτος πολύς εν τω τόπω. Εκάθησαν λοιπόν οι άνδρες τον αριθμόν έως πεντακισχίλιοι.
\par 11 Και έλαβεν ο Ιησούς τους άρτους και ευχαριστήσας διεμοίρασεν εις τους μαθητάς, οι δε μαθηταί εις τους καθημένους· ομοίως και εκ των οψαρίων όσον ήθελον.
\par 12 Αφού δε εχορτάσθησαν, λέγει προς τους μαθητάς αυτούς· Συνάξατε τα περισσεύσαντα κλάσματα, διά να μη χαθή τίποτε.
\par 13 Εσύναξαν λοιπόν και εγέμισαν δώδεκα κοφίνους κλασμάτων εκ των πέντε άρτων των κριθίνων, τα οποία επερίσσευσαν εις τους φαγόντας.
\par 14 Οι άνθρωποι λοιπόν, ιδόντες το θαύμα, το οποίον έκαμεν ο Ιησούς, έλεγον ότι Ούτος είναι αληθώς ο προφήτης ο μέλλων να έλθη εις τον κόσμον.
\par 15 Ο Ιησούς λοιπόν γνωρίσας ότι μέλλουσι να έλθωσι και να αρπάσωσιν αυτόν, διά να κάμωσιν αυτόν βασιλέα, ανεχώρησε πάλιν εις το όρος αυτός μόνος.
\par 16 Καθώς δε έγεινεν εσπέρα, κατέβησαν οι μαθηταί αυτού εις την θάλασσαν,
\par 17 και εμβάντες εις το πλοίον, ήρχοντο πέραν της θαλάσσης εις Καπερναούμ. Και είχεν ήδη γείνει σκότος και ο Ιησούς δεν είχεν ελθεί προς αυτούς,
\par 18 και η θάλασσα υψόνετο, επειδή έπνεε δυνατός άνεμος.
\par 19 Αφού λοιπόν εκωπηλάτησαν ως εικοσιπέντε ή τριάκοντα στάδια βλέπουσι τον Ιησούν περιπατούντα επί της θαλάσσης και πλησιάζοντα εις το πλοίον, και εφοβήθησαν.
\par 20 Εκείνος δε λέγει προς αυτούς· Εγώ είμαι· μη φοβείσθε.
\par 21 Ήθελον λοιπόν να λάβωσιν αυτόν εις το πλοίον, και παρευθύς το πλοίον έφθασεν εις την γην, εις την οποίαν υπήγαινον.
\par 22 Τη επαύριον ο όχλος ο ιστάμενος πέραν της θαλάσσης ότε είδεν ότι πλοιάριον άλλο δεν ήτο εκεί ειμή εν, εκείνο εις το οποίον εισήλθον οι μαθηταί αυτού, και ότι ο Ιησούς δεν εισήλθε μετά των μαθητών αυτού εις το πλοιάριον, αλλά μόνοι οι μαθηταί αυτού ανεχώρησαν·
\par 23 ήλθον δε άλλα πλοιάρια εκ της Τιβεριάδος πλησίον του τόπου, όπου έφαγον τον άρτον, αφού ο Κύριος ευχαρίστησεν·
\par 24 ότε λοιπόν είδεν ο όχλος ότι ο Ιησούς δεν είναι εκεί, ουδέ οι μαθηταί αυτού, εισήλθον και αυτοί εις τα πλοία και ήλθον εις Καπερναούμ ζητούντες τον Ιησούν.
\par 25 Και ευρόντες αυτόν πέραν της θαλάσσης, είπον προς αυτόν· Ραββί, πότε ήλθες εδώ;
\par 26 Απεκρίθη προς αυτούς ο Ιησούς και είπεν· Αληθώς, αληθώς σας λέγω, με ζητείτε, ουχί διότι είδετε θαύματα, αλλά διότι εφάγετε εκ των άρτων και εχορτάσθητε.
\par 27 Εργάζεσθε μη διά την τροφήν την φθειρομένην, αλλά διά την τροφήν την μένουσαν εις ζωήν αιώνιον, την οποίαν ο Υιός του ανθρώπου θέλει σας δώσει· διότι τούτον εσφράγισεν ο Πατήρ ο Θεός.
\par 28 Είπον λοιπόν προς αυτόν· Τι να κάμωμεν, διά να εργαζώμεθα τα έργα του Θεού;
\par 29 Απεκρίθη ο Ιησούς και είπε προς αυτούς· Τούτο είναι το έργον του Θεού, να πιστεύσητε εις τούτον, τον οποίον εκείνος απέστειλε.
\par 30 Τότε είπον προς αυτόν· Τι σημείον λοιπόν κάμνεις συ, διά να ίδωμεν και πιστεύσωμεν εις σε; τι εργάζεσαι;
\par 31 οι πατέρες ημών έφαγον το μάννα εν τη ερήμω, καθώς είναι γεγραμμένον· Άρτον εκ του ουρανού έδωκεν εις αυτούς να φάγωσιν.
\par 32 Είπε λοιπόν προς αυτούς ο Ιησούς· Αληθώς, αληθώς σας λέγω, δεν έδωκεν εις εσάς τον άρτον εκ του ουρανού ο Μωϋσής, αλλ' ο Πατήρ μου σας δίδει τον άρτον εκ του ουρανού τον αληθινόν.
\par 33 Διότι ο άρτος του Θεού είναι ο καταβαίνων εκ του ουρανού και δίδων ζωήν εις τον κόσμον.
\par 34 Είπον λοιπόν προς αυτόν· Κύριε, πάντοτε δος εις ημάς τον άρτον τούτον.
\par 35 Και είπε προς αυτούς ο Ιησούς· Εγώ είμαι ο άρτος της ζωής· όστις έρχεται προς εμέ, δεν θέλει πεινάσει, και όστις πιστεύει εις εμέ, δεν θέλει διψήσει πώποτε.
\par 36 Πλην σας είπον ότι και με είδετε και δεν πιστεύετε.
\par 37 Παν ό,τι μοι δίδει ο Πατήρ, προς εμέ θέλει ελθεί, και τον ερχόμενον προς εμέ δεν θέλω εκβάλει έξω·
\par 38 διότι κατέβην εκ του ουρανού, ουχί διά να κάμω το θέλημα το εμόν, αλλά το θέλημα του πέμψαντός με.
\par 39 Τούτο δε είναι το θέλημα του πέμψαντός με Πατρός, παν ό,τι μοι έδωκε να μη απολέσω ουδέν εξ αυτού, αλλά να αναστήσω αυτό εν τη εσχάτη ημέρα.
\par 40 Και τούτο είναι το θέλημα του πέμψαντός με, πας όστις βλέπει τον Υιόν και πιστεύει εις αυτόν να έχη ζωήν αιώνιον, και εγώ θέλω αναστήσει αυτόν εν τη εσχάτη ημέρα.
\par 41 Εγόγγυζον λοιπόν οι Ιουδαίοι περί αυτού ότι είπεν, Εγώ είμαι ο άρτος ο καταβάς εκ του ουρανού,
\par 42 και έλεγον· δεν είναι ούτος Ιησούς ο υιός του Ιωσήφ, του οποίου ημείς γνωρίζομεν τον πατέρα και την μητέρα; πως λοιπόν λέγει ούτος ότι εκ του ουρανού κατέβην;
\par 43 Απεκρίθη λοιπόν ο Ιησούς και είπε προς αυτούς· Μη γογγύζετε μεταξύ σας.
\par 44 Ουδείς δύναται να έλθη προς εμέ, εάν δεν ελκύση αυτόν ο Πατήρ ο πέμψας με, και εγώ θέλω αναστήσει αυτόν εν τη εσχάτη ημέρα.
\par 45 Είναι γεγραμμένον εν τοις προφήταις· Και πάντες θέλουσιν είσθαι διδακτοί του Θεού. Πας λοιπόν, όστις ακούση παρά του Πατρός και μάθη, έρχεται προς εμέ·
\par 46 ουχί ότι είδε τις τον Πατέρα, ειμή εκείνος όστις είναι παρά του Θεού, ούτος είδε τον Πατέρα.
\par 47 Αληθώς αληθώς, σας λέγω, Ο πιστεύων εις εμέ έχει ζωήν αιώνιον.
\par 48 Εγώ είμαι ο άρτος της ζωής.
\par 49 Οι πατέρες σας έφαγον το μάννα εν τη ερήμω και απέθανον·
\par 50 ούτος είναι ο άρτος ο καταβαίνων εκ του ουρανού, διά να φάγη τις εξ αυτού και να μη αποθάνη.
\par 51 Εγώ είμαι ο άρτος ο ζων, ο καταβάς εκ του ουρανού. Εάν τις φάγη εκ τούτου του άρτου, θέλει ζήσει εις τον αιώνα. Και ο άρτος δε τον οποίον εγώ θέλω δώσει, είναι η σαρξ μου την οποίαν εγώ θέλω δώσει υπέρ της ζωής του κόσμου.
\par 52 Εμάχοντο λοιπόν προς αλλήλους Ιουδαίοι, λέγοντες· Πως δύναται ούτος να δώση εις ημάς να φάγωμεν την σάρκα αυτού;
\par 53 Είπε λοιπόν εις αυτούς ο Ιησούς· Αληθώς, αληθώς σας λέγω, Εάν δεν φάγητε την σάρκα του υιού του ανθρώπου και πίητε το αίμα αυτού, δεν έχετε ζωήν εν εαυτοίς.
\par 54 Όστις τρώγει την σάρκα μου και πίνει το αίμα μου, έχει ζωήν αιώνιον, και εγώ θέλω αναστήσει αυτόν εν τη εσχάτη ημέρα.
\par 55 Διότι η σαρξ μου αληθώς είναι τροφή, και το αίμα μου αληθώς είναι πόσις.
\par 56 Όστις τρώγει την σάρκα μου και πίνει το αίμα μου εν εμοί μένει, και εγώ εν αυτώ.
\par 57 Καθώς με απέστειλεν ο ζων Πατήρ και εγώ ζω διά τον Πατέρα, ούτω και όστις με τρώγει θέλει ζήσει και εκείνος δι' εμέ.
\par 58 Ούτος είναι ο άρτος ο καταβάς εκ του ουρανού, ουχί καθώς οι πατέρες σας έφαγον το μάννα και απέθανον· όστις τρώγει τούτον τον άρτον θέλει ζήσει εις τον αιώνα.
\par 59 Ταύτα είπεν εν τη συναγωγή, διδάσκων εν Καπερναούμ.
\par 60 Πολλοί λοιπόν εκ των μαθητών αυτού ακούσαντες, είπον· Σκληρός είναι ούτος ο λόγος· τις δύναται να ακούη αυτόν;
\par 61 Νοήσας δε ο Ιησούς εν εαυτώ ότι γογγύζουσι περί τούτου οι μαθηταί αυτού, είπε προς αυτούς· Τούτο σας σκανδαλίζει;
\par 62 εάν λοιπόν θεωρήτε τον Υιόν του ανθρώπου αναβαίνοντα όπου ήτο το πρότερον;
\par 63 το πνεύμα είναι εκείνο το οποίον ζωοποιεί, η σαρξ δεν ωφελεί ουδέν· οι λόγοι, τους οποίους εγώ λαλώ προς εσάς, πνεύμα είναι και ζωή είναι.
\par 64 Πλην είναι τινές από σας, οίτινες δεν πιστεύουσι. Διότι ήξευρεν εξ αρχής ο Ιησούς, τίνες είναι οι μη πιστεύοντες και τις είναι ο μέλλων να παραδώση αυτόν.
\par 65 Και έλεγε· Διά τούτο σας είπον ότι ουδείς δύναται να έλθη προς εμέ, εάν δεν είναι δεδομένον εις αυτόν εκ του Πατρός μου.
\par 66 Έκτοτε πολλοί των μαθητών αυτού εστράφησαν εις τα οπίσω και δεν περιεπάτουν πλέον μετ' αυτού.
\par 67 Είπε λοιπόν ο Ιησούς προς τους δώδεκα· Μήπως και σεις θέλετε να υπάγητε;
\par 68 Απεκρίθη λοιπόν προς αυτόν ο Σίμων Πέτρος· Κύριε, προς τίνα θέλομεν υπάγει; λόγους ζωής αιωνίου έχεις·
\par 69 και ημείς επιστεύσαμεν και εγνωρίσαμεν ότι συ είσαι ο Χριστός ο Υιός του Θεού του ζώντος.
\par 70 Απεκρίθη προς αυτούς ο Ιησούς· Δεν εξέλεξα εγώ εσάς τους δώδεκα και εις από σας είναι διάβολος;
\par 71 Έλεγε δε τον Ιούδαν του Σίμωνος τον Ισκαριώτην· διότι ούτος, εις ων εκ των δώδεκα, έμελλε να παραδώση αυτόν.

\chapter{7}

\par 1 Και περιεπάτει ο Ιησούς μετά ταύτα εν τη Γαλιλαία· διότι δεν ήθελε να περιπατή εν τη Ιουδαία, επειδή οι Ιουδαίοι εζήτουν να θανατώσωσιν αυτόν.
\par 2 Επλησίαζε δε η εορτή των Ιουδαίων, η σκηνοπηγία.
\par 3 Είπον λοιπόν προς αυτόν οι αδελφοί αυτού· Μετάβηθι εντεύθεν και ύπαγε εις την Ιουδαίαν, διά να ίδωσι και οι μαθηταί σου τα έργα σου, τα οποία κάμνεις·
\par 4 διότι ουδείς πράττει τι κρυφίως και ζητεί αυτός να ήναι φανερός. Εάν πράττης ταύτα, φανέρωσον σεαυτόν εις τον κόσμον.
\par 5 Διότι ουδέ οι αδελφοί αυτού επίστευον εις αυτόν.
\par 6 Λέγει λοιπόν προς αυτούς ο Ιησούς· Ο καιρός ο ιδικός μου δεν ήλθεν έτι, ο δε καιρός ο ιδικός σας είναι πάντοτε έτοιμος.
\par 7 Δεν δύναται ο κόσμος να μισή εσάς· εμέ όμως μισεί, διότι εγώ μαρτυρώ περί αυτού ότι τα έργα αυτού είναι πονηρά.
\par 8 Σεις ανάβητε εις την εορτήν ταύτην· εγώ δεν αναβαίνω έτι εις την εορτήν ταύτην, διότι ο καιρός μου δεν επληρώθη έτι.
\par 9 Και αφού είπε ταύτα προς αυτούς, έμεινεν εν τη Γαλιλαία.
\par 10 Αφού δε ανέβησαν οι αδελφοί αυτού, τότε και αυτός ανέβη εις την εορτήν, ουχί φανερώς αλλά κρυφίως πως.
\par 11 Οι Ιουδαίοι λοιπόν εζήτουν αυτόν εν τη εορτή και έλεγον· Που είναι εκείνος;
\par 12 Και ήτο πολύς γογγυσμός περί αυτού μεταξύ των όχλων. Άλλοι μεν έλεγον ότι είναι καλός· άλλοι δε έλεγον, Ουχί, αλλά πλανά τον όχλον.
\par 13 Ουδείς όμως ελάλει παρρησία περί αυτού διά τον φόβον των Ιουδαίων.
\par 14 Και ενώ η εορτή ήτο ήδη περί τα μέσα, ανέβη ο Ιησούς εις το ιερόν και εδίδασκε.
\par 15 Και εθαύμαζον οι Ιουδαίοι, λέγοντες· Πως ούτος εξεύρει γράμματα, ενώ δεν έμαθεν;
\par 16 Απεκρίθη προς αυτούς ο Ιησούς και είπεν· Η ιδική μου διδαχή δεν είναι εμού, αλλά του πέμψαντός με.
\par 17 Εάν τις θέλη να κάμη το θέλημα αυτού, θέλει γνωρίσει περί της διδαχής, αν ήναι εκ του Θεού ή αν εγώ λαλώ απ' εμαυτού.
\par 18 Όστις λαλεί αφ' εαυτού, ζητεί την δόξαν την ιδικήν αυτού, όστις όμως ζητεί την δόξαν του πέμψαντος αυτόν, ούτος είναι αληθής, και αδικία εν αυτώ δεν υπάρχει.
\par 19 Ο Μωϋσής δεν σας έδωκε τον νόμον; και ουδείς από σας εκπληροί τον νόμον. Διά τι ζητείτε να μη θανατώσητε;
\par 20 Απεκρίθη ο όχλος και είπε· Δαιμόνιον έχεις· τις ζητεί να σε θανατώση;
\par 21 Απεκρίθη ο Ιησούς και είπε προς αυτούς· Εν έργον έκαμον, και πάντες θαυμάζετε.
\par 22 Διά τούτο ο Μωϋσής σας έδωκε την περιτομήν, ουχί ότι είναι εκ του Μωϋσέως, αλλ' εκ των πατέρων, και εν σαββάτω περιτέμνετε άνθρωπον.
\par 23 Εάν λαμβάνη άνθρωπος περιτομήν εν σαββάτω, διά να μη λυθή ο νόμος του Μωϋσέως, οργίζεσθε κατ' εμού διότι έκαμον ολόκληρον άνθρωπον υγιή εν σαββάτω;
\par 24 Μη κρίνετε κατ' όψιν, αλλά την δικαίαν κρίσιν κρίνατε.
\par 25 Έλεγον λοιπόν τινές εκ των Ιεροσολυμιτών· Δεν είναι ούτος, τον οποίον ζητούσι να θανατώσωσι;
\par 26 Και ιδού, παρρησία λαλεί, και δεν λέγουσι προς αυτόν ουδέν. Μήπως τωόντι εγνώρισαν οι άρχοντες ότι ούτος είναι αληθώς ο Χριστός;
\par 27 Αλλά τούτον εξεύρομεν πόθεν είναι· ο δε Χριστός όταν έρχεται, ουδείς γινώσκει πόθεν είναι.
\par 28 Εφώναξε λοιπόν ο Ιησούς, διδάσκων εν τω ιερώ, και είπε· Και εμέ εξεύρετε και πόθεν είμαι εξεύρετε· και απ' εμαυτού δεν ήλθον, αλλ' είναι αληθινός ο πέμψας με, τον οποίον σεις δεν εξεύρετε·
\par 29 εγώ όμως εξεύρω αυτόν, διότι παρ' αυτού είμαι και εκείνος με απέστειλεν.
\par 30 Εζήτουν λοιπόν να πιάσωσιν αυτόν, και ουδείς επέβαλεν επ' αυτόν την χείρα, διότι δεν είχεν ελθεί έτι η ώρα αυτού.
\par 31 Πολλοί δε εκ του όχλου επίστευσαν εις αυτόν και έλεγον ότι ο Χριστός όταν έλθη, μήπως θέλει κάμει θαύματα πλειότερα τούτων, τα οποία ούτος έκαμεν;
\par 32 Ήκουσαν οι Φαρισαίοι τον όχλον ότι εγόγγυζε ταύτα περί αυτού, και απέστειλαν οι Φαρισαίοι και οι αρχιερείς υπηρέτας διά να πιάσωσιν αυτόν.
\par 33 Είπε λοιπόν προς αυτούς ο Ιησούς· Έτι ολίγον καιρόν είμαι μεθ' υμών, και υπάγω προς τον πέμψαντά με.
\par 34 Θέλετε με ζητήσει και δεν θέλετε με ευρεί· και όπου είμαι εγώ, σεις δεν δύνασθε να έλθητε.
\par 35 Είπον λοιπόν οι Ιουδαίοι προς αλλήλους· Που μέλλει ούτος να υπάγη, ώστε ημείς δεν θέλομεν ευρεί αυτόν; Μήπως μέλλει να υπάγη εις τους διεσπαρμένους μεταξύ των Ελλήνων και να διδάσκη τους Έλληνας;
\par 36 Τις είναι ούτος ο λόγος τον οποίον είπε, Θέλετε με ζητήσει και δεν θέλετε με ευρεί, και, όπου είμαι εγώ, σεις δεν δύνασθε να έλθητε;
\par 37 Κατά δε την τελευταίαν ημέραν την μεγάλην της εορτής ίστατο ο Ιησούς και έκραξε λέγων· Εάν τις διψά, ας έρχηται προς εμέ και ας πίνη.
\par 38 Όστις πιστεύει εις εμέ, καθώς είπεν η γραφή, ποταμοί ύδατος ζώντος θέλουσι ρεύσει εκ της κοιλίας αυτού.
\par 39 Τούτο δε είπε περί του Πνεύματος, το οποίον έμελλον να λαμβάνωσιν οι πιστεύοντες εις αυτόν· διότι δεν ήτο έτι δεδομένον Πνεύμα Άγιον, επειδή ο Ιησούς έτι δεν εδοξάσθη.
\par 40 Πολλοί λοιπόν εκ του όχλου ακούσαντες τον λόγον, έλεγον· Ούτος είναι αληθώς ο προφήτης.
\par 41 Άλλοι έλεγον· Ούτος είναι ο Χριστός. Άλλοι δε έλεγον· Μη γαρ εκ της Γαλιλαίας έρχεται ο Χριστός;
\par 42 Δεν είπεν η γραφή ότι εκ του σπέρματος του Δαβίδ και από της κώμης Βηθλεέμ, όπου ήτο ο Δαβίδ, έρχεται ο Χριστός;
\par 43 Σχίσμα λοιπόν έγεινε μεταξύ του όχλου δι' αυτόν.
\par 44 Τινές δε εξ αυτών ήθελον να πιάσωσιν αυτόν, αλλ' ουδείς επέβαλεν επ' αυτόν τας χείρας.
\par 45 Ήλθον λοιπόν οι υπηρέται προς τους αρχιερείς και Φαρισαίους, και εκείνοι είπον προς αυτούς· Διά τι δεν εφέρετε αυτόν;
\par 46 Απεκρίθησαν οι υπηρέται· Ουδέποτε ελάλησεν άνθρωπος ούτω, καθώς ούτος ο άνθρωπος.
\par 47 Απεκρίθησαν λοιπόν προς αυτούς οι Φαρισαίοι· Μήπως και σεις επλανήθητε;
\par 48 Μήπως τις εκ των αρχόντων επίστευσεν εις αυτόν ή εκ των Φαρισαίων;
\par 49 Αλλ' ο όχλος ούτος, όστις δεν γνωρίζει τον νόμον, είναι επικατάρατοι.
\par 50 Λέγει ο Νικόδημος προς αυτούς, ο ελθών προς αυτόν διά νυκτός, εις ων εξ αυτών.
\par 51 Μήπως ο νόμος ημών κρίνει τον άνθρωπον, εάν δεν ακούση παρ' αυτού πρότερον και μάθη τι πράττει;
\par 52 Απεκρίθησαν και είπον προς αυτόν· Μήπως και συ εκ της Γαλιλαίας είσαι; ερεύνησον και ίδε ότι προφήτης εκ της Γαλιλαίας δεν ηγέρθη.
\par 53 Και υπήγεν έκαστος εις τον οίκον αυτού.

\chapter{8}

\par 1 Ο δε Ιησούς υπήγεν εις το όρος των Ελαιών.
\par 2 Και την αυγήν ήλθε πάλιν εις το ιερόν, και πας ο λαός ήρχετο προς αυτόν· και καθήσας εδίδασκεν αυτούς.
\par 3 Φέρουσι δε προς αυτόν οι γραμματείς και οι Φαρισαίοι γυναίκα συλληφθείσαν επί μοιχεία, και στήσαντες αυτήν εν τω μέσω,
\par 4 λέγουσι προς αυτόν· Διδάσκαλε, αύτη η γυνή συνελήφθη επ' αυτοφώρω μοιχευομένη.
\par 5 Εν δε τω νόμω ο Μωϋσής προσέταξεν ημάς να λιθοβολώνται αι τοιαύται· συ λοιπόν τι λέγεις;
\par 6 Έλεγον δε τούτο δοκιμάζοντες αυτόν, διά να έχωσι ίνα κατηγορώσιν αυτόν. Ο δε Ιησούς κύψας κάτω, έγραφε διά του δακτύλου εις την γην.
\par 7 Και επειδή επέμενον ερωτώντες αυτόν, ανακύψας είπε προς αυτούς· Όστις από σας είναι αναμάρτητος, πρώτος ας ρίψη τον λίθον επ' αυτήν.
\par 8 Και πάλιν κύψας κάτω έγραφεν εις την γην.
\par 9 Εκείνοι δε ακούσαντες, εξήρχοντο εις έκαστος, αρχίσαντες από των πρεσβυτέρων έως των εσχάτων· και έμεινε μόνος ο Ιησούς και η γυνή ισταμένη εν τω μέσω.
\par 10 Ανακύψας δε ο Ιησούς, είπε προς αυτήν· Γύναι, που είναι εκείνοι οι κατήγοροί σου; δεν σε κατεδίκασεν ουδείς;
\par 11 Και εκείνη είπεν· Ουδείς, Κύριε. Και ο Ιησούς είπε προς αυτήν· Ουδέ εγώ σε καταδικάζω· ύπαγε, και εις το εξής μη αμάρτανε.
\par 12 Πάλιν λοιπόν ο Ιησούς ελάλησε προς αυτούς λέγων· Εγώ είμαι το φως του κόσμου· όστις ακολουθεί εμέ δεν θέλει περιπατήσει εις το σκότος, αλλά θέλει έχει το φως της ζωής.
\par 13 Είπον λοιπόν προς αυτόν οι Φαρισαίοι· Συ περί σεαυτού μαρτυρείς· η μαρτυρία σου δεν είναι αληθής.
\par 14 Απεκρίθη ο Ιησούς και είπε προς αυτούς· Και αν εγώ μαρτυρώ περί εμαυτού, η μαρτυρία μου είναι αληθής, διότι εξεύρω πόθεν ήλθον και που υπάγω· σεις όμως δεν εξεύρετε πόθεν έρχομαι και που υπάγω.
\par 15 Σεις κατά την σάρκα κρίνετε· εγώ δεν κρίνω ουδένα.
\par 16 Αλλά και εάν εγώ κρίνω, η κρίσις η εμή είναι αληθής, διότι μόνος δεν είμαι, αλλ' εγώ και ο Πατήρ ο πέμψας με.
\par 17 Και εν τω νόμω δε υμών είναι γεγραμμένον ότι δύο ανθρώπων η μαρτυρία είναι αληθινή.
\par 18 Εγώ είμαι ο μαρτυρών περί εμαυτού, και ο πέμψας με Πατήρ μαρτυρεί περί εμού.
\par 19 Έλεγον λοιπόν προς αυτόν· Που είναι ο Πατήρ σου; Απεκρίθη ο Ιησούς· Ούτε εμέ εξεύρετε ούτε τον Πατέρα μου· εάν ηξεύρετε εμέ, ηθέλετε εξεύρει και τον Πατέρα μου.
\par 20 Τούτους τους λόγους ελάλησεν ο Ιησούς εν τω θησαυροφυλακίω, διδάσκων εν τω ιερώ, και ουδείς επίασεν αυτόν, διότι δεν είχεν ελθεί έτι η ώρα αυτού.
\par 21 Είπε λοιπόν πάλιν προς αυτούς ο Ιησούς· Εγώ υπάγω και θέλετε με ζητήσει, και θέλετε αποθάνει εν τη αμαρτία υμών· όπου εγώ υπάγω, σεις δεν δύνασθε να έλθητε.
\par 22 Έλεγον λοιπόν οι Ιουδαίοι· Μήπως θέλει θανατώσει εαυτόν, και διά τούτο λέγει, Όπου εγώ υπάγω, σεις δεν δύνασθε να έλθητε;
\par 23 Και είπε προς αυτούς· Σεις είσθε εκ των κάτω, εγώ είμαι εκ των άνω· σεις είσθε εκ του κόσμου τούτου, εγώ δεν είμαι εκ του κόσμου τούτου.
\par 24 Σας είπον λοιπόν ότι θέλετε αποθάνει εν ταις αμαρτίαις υμών· διότι εάν δεν πιστεύσητε ότι εγώ είμαι, θέλετε αποθάνει εν ταις αμαρτίαις υμών.
\par 25 Έλεγον λοιπόν προς αυτόν· Συ τις είσαι; και είπε προς αυτούς ο Ιησούς· ό,τι σας λέγω απ' αρχής.
\par 26 Πολλά έχω να λέγω και να κρίνω περί υμών· αλλ' ο πέμψας με είναι αληθής, και εγώ όσα ήκουσα παρ' αυτού, ταύτα λέγω εις τον κόσμον.
\par 27 δεν ενόησαν ότι έλεγε προς αυτούς περί του Πατρός.
\par 28 Είπε λοιπόν προς αυτούς ο Ιησούς· Όταν υψώσητε τον Υιόν του άνθρώπου, τότε θέλετε γνωρίσει ότι εγώ είμαι, και απ' εμαυτού δεν κάμνω ουδέν, αλλά καθώς με εδίδαξεν ο Πατήρ μου, ταύτα λαλώ.
\par 29 Και ο πέμψας με είναι μετ' εμού· δεν με αφήκεν ο Πατήρ μόνον, διότι εγώ κάμνω πάντοτε τα αρεστά εις αυτόν.
\par 30 Ενώ ελάλει ταύτα, πολλοί επίστευσαν εις αυτόν.
\par 31 Έλεγε λοιπόν ο Ιησούς προς τους Ιουδαίους τους πιστεύσαντας εις αυτόν· Εάν σεις μείνητε εν τω λόγω τω εμώ, είσθε αληθώς μαθηταί μου,
\par 32 και θέλετε γνωρίσει την αλήθειαν, και η αλήθεια θέλει σας ελευθερώσει.
\par 33 Απεκρίθησαν προς αυτόν· Σπέρμα του Αβραάμ είμεθα, και δεν εγείναμεν δούλοι εις ουδένα πώποτε· πως συ λέγεις ότι θέλετε γείνει ελεύθεροι;
\par 34 Απεκρίθη προς αυτούς ο Ιησούς· Αληθώς, αληθώς σας λέγω ότι πας όστις πράττει την αμαρτίαν δούλος είναι της αμαρτίας.
\par 35 Ο δε δούλος δεν μένει πάντοτε εν τη οικία· ο υιός μένει πάντοτε.
\par 36 Εάν λοιπόν ο Υιός σας ελευθερώση, όντως ελεύθεροι θέλετε είσθαι.
\par 37 Εξεύρω ότι είσθε σπέρμα του Αβραάμ· αλλά ζητείτε να με θανατώσητε, διότι ο λόγος ο εμός δεν χωρεί εις εσάς.
\par 38 Εγώ λαλώ ό,τι είδον πλησίον του Πατρός μου· και σεις ομοίως κάμνετε ό,τι είδετε πλησίον του πατρός σας.
\par 39 Απεκρίθησαν και είπον προς αυτόν· Ο πατήρ ημών είναι ο Αβραάμ. Λέγει προς αυτούς ο Ιησούς· Εάν ήσθε τέκνα του Αβραάμ, τα έργα του Αβραάμ ηθέλετε κάμνει.
\par 40 Τώρα δε ζητείτε να με θανατώσητε, άνθρωπον όστις σας ελάλησα την αλήθειαν, την οποίαν ήκουσα παρά του Θεού· τούτο ο Αβραάμ δεν έκαμε.
\par 41 Σεις κάμνετε τα έργα του πατρός σας. Είπον λοιπόν προς αυτόν· Ημείς δεν εγεννήθημεν εκ πορνείας· ένα Πατέρα έχομεν, τον Θεόν.
\par 42 Είπε λοιπόν προς αυτούς ο Ιησούς· Εάν ο Θεός ήτο Πατήρ σας, ηθέλετε αγαπά εμέ· διότι εγώ εκ του Θεού εξήλθον και έρχομαι· επειδή δεν ήλθον απ' εμαυτού, αλλ' εκείνος με απέστειλε.
\par 43 Διά τι δεν γνωρίζετε την λαλιάν μου; διότι δεν δύνασθε να ακούητε τον λόγον μου.
\par 44 Σεις είσθε εκ πατρός του διαβόλου και τας επιθυμίας του πατρός σας θέλετε να πράττητε. Εκείνος ήτο απ' αρχής ανθρωποκτόνος και δεν μένει εν τη αληθεία, διότι αλήθεια δεν υπάρχει εν αυτώ· όταν λαλή το ψεύδος, εκ των ιδίων λαλεί, διότι είναι ψεύστης και ο πατήρ αυτού του ψεύδους.
\par 45 Εγώ δε διότι λέγω την αλήθειαν, δεν με πιστεύετε.
\par 46 Τις από σας με ελέγχει περί αμαρτίας; εάν δε αλήθειαν λέγω, διά τι σεις δεν με πιστεύετε;
\par 47 Όστις είναι εκ του Θεού, τους λόγους του Θεού ακούει· διά τούτο σεις δεν ακούετε, διότι εκ του Θεού δεν είσθε.
\par 48 Απεκρίθησαν λοιπόν οι Ιουδαίοι και είπον προς αυτόν· Δεν λέγομεν ημείς καλώς ότι Σαμαρείτης είσαι συ και δαιμόνιον έχεις;
\par 49 Απεκρίθη ο Ιησούς· Εγώ δαιμόνιον δεν έχω, αλλά τιμώ τον Πατέρα μου, και σεις με ατιμάζετε.
\par 50 Και εγώ δεν ζητώ την δόξαν μου· υπάρχει ο ζητών και κρίνων.
\par 51 Αληθώς, αληθώς σας λέγω· Εάν τις φυλάξη τον λόγον μου, θάνατον δεν θέλει ιδεί εις τον αιώνα.
\par 52 Είπον λοιπόν προς αυτόν οι Ιουδαίοι· Τώρα κατελάβομεν ότι δαιμόνιον έχεις. Ο Αβραάμ απέθανε και οι προφήται, και συ λέγεις· Εάν τις φυλάξη τον λόγον μου, δεν θέλει γευθή θάνατον εις τον αιώνα.
\par 53 Μήπως συ είσαι μεγαλήτερος του πατρός ημών Αβραάμ, όστις απέθανε; και οι προφήται απέθανον· συ τίνα κάμνεις σεαυτόν;
\par 54 Απεκρίθη ο Ιησούς· Εάν εγώ δοξάζω εμαυτόν, η δόξα μου είναι ουδέν· ο Πατήρ μου είναι όστις με δοξάζει, τον οποίον σεις λέγετε ότι είναι Θεός σας.
\par 55 Και δεν εγνωρίσατε αυτόν εγώ όμως γνωρίζω αυτόν· και εάν είπω ότι δεν γνωρίζω αυτόν, θέλω είσθαι όμοιός σας ψεύστης· αλλά γνωρίζω αυτόν και τον λόγον αυτού φυλάττω.
\par 56 Ο Αβραάμ ο πατήρ σας είχεν αγαλλίασιν να ίδη την ημέραν την εμήν και είδε και εχάρη.
\par 57 Είπον λοιπόν οι Ιουδαίοι προς αυτόν· Πεντήκοντα έτη δεν έχεις έτι, και είδες τον Αβραάμ;
\par 58 Είπε προς αυτούς ο Ιησούς· Αληθώς, αληθώς σας λέγω· Πριν γείνη ο Αβραάμ, εγώ είμαι.
\par 59 Εσήκωσαν λοιπόν λίθους διά να ρίψωσι κατ' αυτού· πλην ο Ιησούς εκρύβη και εξήλθεν εκ του ιερού περάσας διά μέσον αυτών, και ούτως ανεχώρησε.

\chapter{9}

\par 1 Και ενώ ανεχώρει, είδεν άνθρωπον τυφλόν εκ γενετής.
\par 2 Και ηρώτησαν αυτόν οι μαθηταί αυτού, λέγοντες· Ραββί, τις ήμαρτεν, ούτος ή οι γονείς αυτού, ώστε να γεννηθή τυφλός;
\par 3 Απεκρίθη ο Ιησούς· Ούτε ούτος ήμαρτεν ούτε οι γονείς αυτού, αλλά διά να φανερωθώσι τα έργα του Θεού εν αυτώ.
\par 4 Εγώ πρέπει να εργάζωμαι τα έργα του πέμψαντός με, εωσού είναι ημέρα· έρχεται νυξ ότε ουδείς δύναται να εργάζηται.
\par 5 Ενόσω είμαι εν τω κόσμω, είμαι φως του κόσμου.
\par 6 Αφού είπε ταύτα, έπτυσε χαμαί και έκαμε πηλόν εκ του πτύσματος και επέχρισε τον πηλόν επί τους οφθαλμούς του τυφλού
\par 7 και είπε προς αυτόν· Ύπαγε, νίφθητι εις την κολυμβήθραν του Σιλωάμ, το οποίον ερμηνεύεται απεσταλμένος. Υπήγε λοιπόν και ενίφθη, και ήλθε βλέπων.
\par 8 Οι δε γείτονες και όσοι έβλεπον αυτόν πρότερον ότι ήτο τυφλός έλεγον δεν είναι ούτος, όστις εκάθητο και εζήτει;
\par 9 Άλλοι έλεγον ότι ούτος είναι· άλλοι δε ότι όμοιος αυτού είναι. Εκείνος έλεγεν ότι εγώ είμαι.
\par 10 Έλεγον λοιπόν προς αυτόν· Πως ηνοίχθησαν οι οφθαλμοί σου;
\par 11 Απεκρίθη εκείνος και είπεν· Άνθρωπος λεγόμενος Ιησούς έκαμε πηλόν και επέχρισε τους οφθαλμούς μου και μοι είπεν· Ύπαγε εις την κολυμβήθραν του Σιλωάμ και νίφθητι· αφού δε υπήγα και ενίφθην, ανέβλεψα.
\par 12 Είπον λοιπόν προς αυτόν· Που είναι εκείνος; Λέγει· Δεν εξεύρω.
\par 13 Φέρουσιν αυτόν τον ποτέ τυφλόν προς τους Φαρισαίους.
\par 14 Ήτο δε σάββατον, ότε έκαμε τον πηλόν ο Ιησούς και ήνοιξε τους οφθαλμούς αυτού.
\par 15 Πάλιν λοιπόν ηρώτων αυτόν και οι Φαρισαίοι πως ανέβλεψε. Και εκείνος είπε προς αυτούς· Πηλόν έβαλεν επί τους οφθαλμούς μου, και ενίφθην, και βλέπω.
\par 16 Έλεγον λοιπόν τινές εκ των Φαρισαίων· Ούτος ο άνθρωπος δεν είναι παρά του Θεού, διότι δεν φυλάττει το σάββατον. Άλλοι έλεγον· Πως δύναται άνθρωπος αμαρτωλός να κάμνη τοιαύτα θαύματα; Και ήτο σχίσμα μεταξύ αυτών.
\par 17 Λέγουσι πάλιν προς τον τυφλόν· Συ τι λέγεις περί αυτού, επειδή ήνοιξε τους οφθαλμούς σου; Και εκείνος είπεν ότι προφήτης είναι.
\par 18 Δεν επίστευσαν λοιπόν οι Ιουδαίοι περί αυτού ότι ήτο τυφλός και ανέβλεψεν, έως ότου εφώναξαν τους γονείς αυτού του αναβλέψαντος
\par 19 και ηρώτησαν αυτούς, λέγοντες· Ούτος είναι ο υιός σας, τον οποίον σεις λέγετε ότι εγεννήθη τυφλός; πως λοιπόν βλέπει τώρα;
\par 20 Απεκρίθησαν προς αυτούς οι γονείς αυτού και είπον· Εξεύρομεν ότι ούτος είναι ο υιός ημών και ότι εγεννήθη τυφλός·
\par 21 Πως δε βλέπει τώρα δεν εξεύρομεν, ή τις ήνοιξε τους οφθαλμούς αυτού ημείς δεν εξεύρομεν· αυτός ηλικίαν έχει, αυτόν ερωτήσατε, αυτός περί εαυτού θέλει λαλήσει.
\par 22 Ταύτα είπον οι γονείς αυτού, διότι εφοβούντο τους Ιουδαίους· επειδή ήδη είχον συμφωνήσει οι Ιουδαίοι, εάν τις ομολογήση αυτόν Χριστόν, να γείνη αποσυνάγωγος.
\par 23 Διά τούτο οι γονείς αυτού είπον ότι ηλικίαν έχει, αυτόν ερωτήσατε.
\par 24 Εφώναξαν λοιπόν εκ δευτέρου τον άνθρωπον, όστις ήτο τυφλός, και είπον προς αυτόν· Δόξασον τον Θεόν· ημείς εξεύρομεν ότι ο άνθρωπος ούτος είναι αμαρτωλός.
\par 25 Απεκρίθη λοιπόν εκείνος και είπεν· Αν ήναι αμαρτωλός δεν εξεύρω· εν εξεύρω, ότι ήμην τυφλός και τώρα βλέπω.
\par 26 Είπον δε προς αυτόν πάλιν· τι σοι έκαμε; πως ήνοιξε τους οφθαλμούς σου;
\par 27 Απεκρίθη προς αυτούς· Σας είπον ήδη, και δεν ηκούσατε· διά τι πάλιν θέλετε να ακούητε; μήπως και σεις θέλετε να γείνητε μαθηταί αυτού;
\par 28 Ελοιδόρησαν λοιπόν αυτόν και είπον· Συ είσαι μαθητής εκείνου· ημείς δε του Μωϋσέως είμεθα μαθηταί.
\par 29 Ημείς εξεύρομεν ότι προς τον Μωϋσήν ελάλησεν ο Θεός· τούτον όμως δεν εξεύρομεν πόθεν είναι.
\par 30 Απεκρίθη ο άνθρωπος και είπε προς αυτούς· Εν τούτω μάλιστα είναι το θαυμαστόν, ότι σεις δεν εξεύρετε πόθεν είναι, και ήνοιξέ μου τους οφθαλμούς.
\par 31 Εξεύρομεν δε ότι αμαρτωλούς ο Θεός δεν ακούει, αλλ' εάν τις ήναι θεοσεβής και κάμνη το θέλημα αυτού, τούτον ακούει.
\par 32 Εκ του αιώνος δεν ηκούσθη ότι ήνοιξέ τις οφθαλμούς γεγεννημένου τυφλού.
\par 33 Εάν ούτος δεν ήτο παρά Θεού, δεν ηδύνατο να κάμη ουδέν.
\par 34 Απεκρίθησαν και είπον προς αυτόν· Συ εγεννήθης όλος εν αμαρτίαις, και συ διδάσκεις ημάς; και εξέβαλον αυτόν έξω.
\par 35 Ήκουσεν ο Ιησούς ότι εξέβαλον αυτόν έξω, και ευρών αυτόν είπε προς αυτόν· Συ πιστεύεις εις τον Υιόν του Θεού;
\par 36 Απεκρίθη εκείνος και είπε· Τις είναι, Κύριε, διά να πιστεύσω εις αυτόν;
\par 37 Και ο Ιησούς είπε προς αυτόν· Και είδες αυτόν και ο λαλών μετά σου εκείνος είναι.
\par 38 Ο δε είπε· Πιστεύω, Κύριε· και προσεκύνησεν αυτόν.
\par 39 Και είπεν ο Ιησούς· Εγώ διά κρίσιν ήλθον εις τον κόσμον τούτον, διά να βλέπωσιν οι μη βλέποντες και να γείνωσι τυφλοί οι βλέποντες.
\par 40 Και ήκουσαν ταύτα όσοι εκ των Φαρισαίων ήσαν μετ' αυτού, και είπον προς αυτόν· Μήπως και ημείς είμεθα τυφλοί;
\par 41 Είπε προς αυτούς ο Ιησούς· Εάν ήσθε τυφλοί, δεν ηθέλετε έχει αμαρτίαν· τώρα όμως λέγετε ότι βλέπομεν· η αμαρτία σας λοιπόν μένει.

\chapter{10}

\par 1 Αληθώς, αληθώς σας λέγω, όστις δεν εισέρχεται διά της θύρας εις την αυλήν των προβάτων, αλλά αναβαίνει αλλαχόθεν, εκείνος είναι κλέπτης και ληστής·
\par 2 όστις όμως εισέρχεται διά της θύρας, είναι ποιμήν των προβάτων.
\par 3 Εις τούτον ο θυρωρός ανοίγει, και τα πρόβατα την φωνήν αυτού ακούουσι, και τα εαυτού πρόβατα κράζει κατ' όνομα και εξάγει αυτά.
\par 4 Και όταν εκβάλη τα εαυτού πρόβατα, υπάγει έμπροσθεν αυτών, και τα πρόβατα ακολουθούσιν αυτόν, διότι γνωρίζουσι την φωνήν αυτού.
\par 5 Ξένον όμως δεν θέλουσιν ακολουθήσει, αλλά θέλουσι φύγει απ' αυτού, διότι δεν γνωρίζουσι την φωνήν των ξένων.
\par 6 Ταύτην την παραβολήν είπε προς αυτούς ο Ιησούς· εκείνοι όμως δεν ενόησαν τι ήσαν ταύτα, τα οποία ελάλει προς αυτούς.
\par 7 Είπε λοιπόν πάλιν προς αυτούς ο Ιησούς· Αληθώς, αληθώς σας λέγω ότι εγώ είμαι η θύρα των προβάτων.
\par 8 Πάντες όσοι ήλθον προ εμού κλέπται είναι και λησταί· αλλά δεν ήκουσαν αυτούς τα πρόβατα.
\par 9 Εγώ είμαι η θύρα· δι' εμού εάν τις εισέλθη, θέλει σωθή και θέλει εισέλθει και εξέλθει και θέλει ευρεί βοσκήν.
\par 10 Ο κλέπτης δεν έρχεται, ειμή διά να κλέψη και θύση και απολέση· εγώ ήλθον διά να έχωσι ζωήν και να έχωσιν αυτήν εν αφθονία.
\par 11 Εγώ είμαι ο ποιμήν ο καλός. Ο ποιμήν ο καλός την ψυχήν αυτού βάλλει υπέρ των προβάτων·
\par 12 ο δε μισθωτός και μη ων ποιμήν, του οποίου δεν είναι τα πρόβατα ιδικά του, θεωρεί τον λύκον ερχόμενον και αφίνει τα πρόβατα και φεύγει· και ο λύκος αρπάζει αυτά και σκορπίζει τα πρόβατα.
\par 13 Ο δε μισθωτός φεύγει, διότι είναι μισθωτός και δεν μέλει αυτόν περί των προβάτων.
\par 14 Εγώ είμαι ο ποιμήν ο καλός, και γνωρίζω τα εμά και γνωρίζομαι υπό των εμών,
\par 15 καθώς με γνωρίζει ο Πατήρ και εγώ γνωρίζω τον Πατέρα, και την ψυχήν μου βάλλω υπέρ των προβάτων.
\par 16 Και άλλα πρόβατα έχω, τα οποία δεν είναι εκ της αυλής ταύτης· και εκείνα πρέπει να συνάξω, και θέλουσιν ακούσει την φωνήν μου, και θέλει γείνει μία ποίμνη, εις ποιμήν.
\par 17 Διά τούτο ο Πατήρ με αγαπά, διότι εγώ βάλλω την ψυχήν μου, διά να λάβω αυτήν πάλιν.
\par 18 Ουδείς αφαιρεί αυτήν απ' εμού, αλλ' εγώ βάλλω αυτήν απ' εμαυτού· εξουσίαν έχω να βάλω αυτήν, και εξουσίαν έχω πάλιν να λάβω αυτήν· ταύτην την εντολήν έλαβον παρά του Πατρός μου.
\par 19 Σχίσμα λοιπόν έγεινε πάλιν μεταξύ των Ιουδαίων διά τους λόγους τούτους.
\par 20 Και έλεγον πολλοί εξ αυτών· Δαιμόνιον έχει και είναι μαινόμενος· τι ακούετε αυτόν;
\par 21 Άλλοι έλεγον· Ούτοι οι λόγοι δεν είναι δαιμονιζομένου· μήπως δύναται δαιμόνιον να ανοίγη οφθαλμούς τυφλών;
\par 22 Έγειναν δε τα εγκαίνια εν Ιεροσολύμοις, και ήτο χειμών·
\par 23 και ο Ιησούς περιεπάτει εν τω ιερώ εν τη στοά του Σολομώντος.
\par 24 Περιεκύκλωσαν λοιπόν αυτόν οι Ιουδαίοι και έλεγον προς αυτόν· Έως πότε κρατείς εν αμφιβολία την ψυχήν ημών; εάν συ ήσαι ο Χριστός, ειπέ προς ημάς παρρησία.
\par 25 Απεκρίθη προς αυτούς ο Ιησούς· Σας είπον, και δεν πιστεύετε. Τα έργα, τα οποία εγώ κάμνω εν τω ονόματι του Πατρός μου, ταύτα μαρτυρούσι περί εμού·
\par 26 αλλά σεις δεν πιστεύετε· διότι δεν είσθε εκ των προβάτων των εμών, καθώς σας είπον.
\par 27 Τα πρόβατα τα εμά ακούουσι την φωνήν μου, και εγώ γνωρίζω αυτά, και με ακολουθούσι.
\par 28 Και εγώ δίδω εις αυτά ζωήν αιώνιον, και δεν θέλουσιν απολεσθή εις τον αιώνα, και ουδείς θέλει αρπάσει αυτά εκ της χειρός μου.
\par 29 Ο Πατήρ μου, όστις μοι έδωκεν αυτά, είναι μεγαλήτερος πάντων, και ουδείς δύναται να αρπάση εκ της χειρός του Πατρός μου.
\par 30 Εγώ και ο Πατήρ εν είμεθα.
\par 31 Επίασαν λοιπόν πάλιν οι Ιουδαίοι λίθους, διά να λιθοβολήσωσιν αυτόν.
\par 32 Απεκρίθη προς αυτούς ο Ιησούς· Πολλά καλά έργα έδειξα εις εσάς εκ του Πατρός μου· διά ποίον έργον εξ αυτών με λιθοβολείτε;
\par 33 Απεκρίθησαν προς αυτόν οι Ιουδαίοι, λέγοντες· Περί καλού έργου δεν σε λιθοβολούμεν, αλλά περί βλασφημίας, και διότι συ άνθρωπος ων κάμνεις σεαυτόν Θεόν.
\par 34 Απεκρίθη προς αυτούς ο Ιησούς· Δεν είναι γεγραμμένον εν τω νόμω υμών, Εγώ είπα, θεοί είσθε;
\par 35 Εάν εκείνους είπε θεούς, προς τους οποίους έγεινεν ο λόγος του Θεού, και δεν δύναται να αναιρεθή η γραφή,
\par 36 εκείνον, τον οποίον ο Πατήρ ηγίασε και απέστειλεν εις τον κόσμον, σεις λέγετε ότι βλασφημείς, διότι είπον, Υιός του Θεού είμαι;
\par 37 Εάν δεν κάμνω τα έργα του Πατρός μου, μη πιστεύετε εις εμέ·
\par 38 αλλ' εάν κάμνω, αν και εις εμέ δεν πιστεύητε, πιστεύσατε εις τα έργα, διά να γνωρίσητε και πιστεύσητε ότι ο Πατήρ είναι εν εμοί και εγώ εν αυτώ.
\par 39 Εζήτουν λοιπόν πάλιν να πιάσωσιν αυτόν· και εξέφυγεν εκ της χειρός αυτών.
\par 40 Και υπήγε πάλιν πέραν του Ιορδάνου, εις τον τόπον όπου εβάπτιζε κατ' αρχάς ο Ιωάννης, και έμεινεν εκεί.
\par 41 Και πολλοί ήλθον προς αυτόν και έλεγον ότι ο Ιωάννης μεν ουδέν θαύμα έκαμε, πάντα όμως όσα είπεν ο Ιωάννης περί τούτου, ήσαν αληθινά.
\par 42 Και εκεί επίστευσαν πολλοί εις αυτόν.

\chapter{11}

\par 1 Ήτο δε τις ασθενής Λάζαρος από Βηθανίας, εκ της κώμης της Μαρίας και Μάρθας της αδελφής αυτής.
\par 2 Η δε Μαρία ήτο η αλείψασα τον Κύριον με μύρον και σπογγίσασα τους πόδας αυτού με τας τρίχας αυτής, της οποίας ο αδελφός Λάζαρος ησθένει.
\par 3 Απέστειλαν λοιπόν αι αδελφαί προς αυτόν, λέγουσαι· Κύριε, ιδού, εκείνος τον οποίον αγαπάς, ασθενεί.
\par 4 Και ακούσας ο Ιησούς είπεν· Αύτη η ασθένεια δεν είναι προς θάνατον, αλλ' υπέρ της δόξης του Θεού, διά να δοξασθή ο Υιός του Θεού δι' αυτής.
\par 5 Ηγάπα δε ο Ιησούς την Μάρθαν και την αδελφήν αυτής και τον Λάζαρον.
\par 6 Καθώς λοιπόν ήκουσεν ότι ασθενεί, τότε μεν έμεινε δύο ημέρας εν τω τόπω όπου ήτο·
\par 7 έπειτα μετά τούτο λέγει προς τους μαθητάς· Ας υπάγωμεν εις την Ιουδαίαν πάλιν.
\par 8 Λέγουσι προς αυτόν οι μαθηταί· Ραββί, τώρα εζήτουν να σε λιθοβολήσωσιν οι Ιουδαίοι, και πάλιν υπάγεις εκεί;
\par 9 Απεκρίθη ο Ιησούς· Δεν είναι δώδεκα αι ώραι της ημέρας; εάν τις περιπατή εν τη ημέρα, δεν προσκόπτει, διότι βλέπει το φως του κόσμου τούτου·
\par 10 εάν τις όμως περιπατή εν τη νυκτί, προσκόπτει, διότι το φως δεν είναι εν αυτώ.
\par 11 Ταύτα είπε, και μετά τούτο λέγει προς αυτούς· Λάζαρος ο φίλος ημών εκοιμήθη· αλλά υπάγω διά να εξυπνήσω αυτόν.
\par 12 Είπον λοιπόν οι μαθηταί αυτού· Κύριε, αν εκοιμήθη, θέλει σωθή.
\par 13 Αλλ' ο Ιησούς είχεν ειπεί περί του θανάτου αυτού· εκείνοι όμως ενόμισαν ότι λέγει περί της κοιμήσεως του ύπνου.
\par 14 Τότε λοιπόν είπε προς αυτούς ο Ιησούς παρρησία· Ο Λάζαρος απέθανε.
\par 15 Και χαίρω διά σας, διά να πιστεύσητε, διότι δεν ήμην εκεί· αλλ' ας υπάγωμεν προς αυτόν.
\par 16 Είπε δε ο Θωμάς, ο λεγόμενος Δίδυμος προς τους συμμαθητάς· Ας υπάγωμεν και ημείς, διά να αποθάνωμεν μετ' αυτού.
\par 17 Ελθών λοιπόν ο Ιησούς εύρεν αυτόν τέσσαρας ημέρας έχοντα ήδη εν τω μνημείω.
\par 18 Ήτο δε η Βηθανία πλησίον των Ιεροσολύμων, απέχουσα ως δεκαπέντε στάδια.
\par 19 Και πολλοί εκ των Ιουδαίων είχον ελθεί προς την Μάρθαν και Μαρίαν, διά να παρηγορήσωσιν αυτάς περί του αδελφού αυτών.
\par 20 Η Μάρθα λοιπόν, καθώς ήκουσεν ότι ο Ιησούς έρχεται, υπήντησεν αυτόν· η δε Μαρία εκάθητο εν τω οίκω.
\par 21 Είπε λοιπόν η Μάρθα προς τον Ιησούν· Κύριε, εάν ήσο εδώ, ο αδελφός μου δεν ήθελεν αποθάνει.
\par 22 Πλην και τώρα εξεύρω ότι όσα ζητήσης παρά του Θεού, θέλει σοι δώσει ο Θεός.
\par 23 Λέγει προς αυτήν ο Ιησούς· Ο αδελφός σου θέλει αναστηθή.
\par 24 Λέγει προς αυτόν η Μάρθα· Εξεύρω ότι θέλει αναστηθή εν τη αναστάσει εν τη εσχάτη ημέρα.
\par 25 Είπε προς αυτήν ο Ιησούς· Εγώ είμαι η ανάστασις και η ζωή· ο πιστεύων εις εμέ, και αν αποθάνη, θέλει ζήσει·
\par 26 και πας όστις ζη και πιστεύει εις εμέ δεν θέλει αποθάνει εις τον αιώνα. Πιστεύεις τούτο;
\par 27 Λέγει προς αυτόν· Ναι, Κύριε, εγώ επίστευσα ότι συ είσαι ο Χριστός, ο Υιός του Θεού, ο ερχόμενος εις τον κόσμον.
\par 28 Και αφού είπε ταύτα, υπήγε και εφώναξε Μαρίαν την αδελφήν αυτής κρυφίως και είπεν· Ο Διδάσκαλος ήλθε και σε κράζει.
\par 29 Εκείνη, καθώς ήκουσε, σηκόνεται ταχέως και έρχεται προς αυτόν.
\par 30 Δεν είχε δε ελθεί ο Ιησούς έτι εις την κώμην, αλλ' ήτο εν τω τόπω, όπου υπήντησεν αυτόν η Μάρθα.
\par 31 Οι Ιουδαίοι λοιπόν, οι όντες μετ' αυτής εν τη οικία και παρηγορούντες αυτήν, ιδόντες την Μαρίαν ότι εσηκώθη ταχέως και εξήλθεν, ηκολούθησαν αυτήν, λέγοντες ότι υπάγει εις το μνημείον, διά να κλαύση εκεί.
\par 32 Η Μαρία λοιπόν καθώς ήλθεν όπου ήτο ο Ιησούς, ιδούσα αυτόν έπεσεν εις τους πόδας αυτού, λέγουσα προς αυτόν· Κύριε, εάν ήσο εδώ, ο αδελφός μου δεν ήθελεν αποθάνει.
\par 33 Ο δε Ιησούς, καθώς είδεν αυτήν κλαίουσαν και τους ελθόντας μετ' αυτής Ιουδαίους κλαίοντας, εστέναξεν εν τη ψυχή αυτού και εταράχθη,
\par 34 και είπε· Που εβάλετε αυτόν; Λέγουσι προς αυτόν· Κύριε, ελθέ και ίδε.
\par 35 Εδάκρυσεν ο Ιησούς.
\par 36 Έλεγον λοιπόν οι Ιουδαίοι· Ιδέ πόσον ηγάπα αυτόν.
\par 37 Τινές δε εξ αυτών είπον· Δεν ηδύνατο ούτος, όστις ήνοιξε τους οφθαλμούς του τυφλού, να κάμη ώστε και ούτος να μη αποθάνη;
\par 38 Ο Ιησούς λοιπόν, πάλιν στενάζων εν εαυτώ, έρχεται εις το μνημείον· ήτο δε σπήλαιον, και έκειτο λίθος επ' αυτού.
\par 39 Λέγει ο Ιησούς· Σηκώσατε τον λίθον. Λέγει προς αυτόν η αδελφή του αποθανόντος η Μάρθα· Κύριε, όζει ήδη· διότι είναι τεσσάρων ημερών.
\par 40 Λέγει προς αυτήν ο Ιησούς· Δεν σοι είπον ότι εάν πιστεύσης, θέλεις ιδεί την δόξαν του Θεού;
\par 41 Εσήκωσαν λοιπόν τον λίθον, όπου έκειτο ο αποθανών. Ο δε Ιησούς, υψώσας τους οφθαλμούς άνω, είπε· Πάτερ, ευχαριστώ σοι ότι μου ήκουσας.
\par 42 Και εγώ εγνώριζον ότι πάντοτε μου ακούεις· αλλά διά τον όχλον τον περιεστώτα είπον τούτο, διά να πιστεύσωσιν ότι συ με απέστειλας.
\par 43 Και ταύτα ειπών, μετά φωνής μεγάλης εκραύγασε· Λάζαρε, ελθέ έξω.
\par 44 Και εξήλθεν ο τεθνηκώς, δεδεμένος τους πόδας και τας χείρας με τα σάβανα, και το πρόσωπον αυτού ήτο περιδεδεμένον με σουδάριον. Λέγει προς αυτούς ο Ιησούς· Λύσατε αυτόν και αφήσατε να υπάγη.
\par 45 Πολλοί λοιπόν εκ των Ιουδαίων, οίτινες είχον ελθεί εις την Μαρίαν και είδον όσα έκαμεν ο Ιησούς, επίστευσαν εις αυτόν.
\par 46 Τινές δε εξ αυτών απήλθον προς τους Φαρισαίους και είπον προς αυτούς όσα έκαμεν ο Ιησούς.
\par 47 Συνεκρότησαν λοιπόν συνέδριον οι αρχιερείς και οι Φαρισαίοι και έλεγον· Τι κάμνομεν, διότι ούτος ο άνθρωπος πολλά θαύματα κάμνει.
\par 48 Εάν αφήσωμεν αυτόν ούτω, πάντες θέλουσι πιστεύσει εις αυτόν, και θέλουσιν ελθεί οι Ρωμαίοι και αφανίσει και τον τόπον ημών και το έθνος.
\par 49 Εις δε τις εξ αυτών, ο Καϊάφας, όστις ήτο αρχιερεύς του ενιαυτού εκείνου, είπε προς αυτούς· Σεις δεν εξεύρετε τίποτε,
\par 50 ουδέ συλλογίζεσθε ότι μας συμφέρει να αποθάνη εις άνθρωπος υπέρ του λαού και να μη απολεσθή όλον το έθνος.
\par 51 Τούτο δε αφ' εαυτού δεν είπεν, αλλ' αρχιερεύς ων του ενιαυτού εκείνου προεφήτευσεν ότι έμελλεν ο Ιησούς να αποθάνη υπέρ του έθνους,
\par 52 και ουχί μόνον υπέρ του έθνους, αλλά και διά να συνάξη εις εν τα τέκνα του Θεού τα διεσκορπισμένα.
\par 53 Απ' εκείνης λοιπόν της ημέρας συνεβουλεύθησαν, διά να θανατώσωσιν αυτόν.
\par 54 Όθεν ο Ιησούς δεν περιεπάτει πλέον παρρησία μεταξύ των Ιουδαίων, αλλ' ανεχώρησεν εκείθεν εις τον τόπον πλησίον της ερήμου, εις πόλιν λεγομένην Εφραΐμ, και εκεί διέτριβε μετά των μαθητών αυτού.
\par 55 Επλησίαζε δε το πάσχα των Ιουδαίων, και πολλοί ανέβησαν εκ του τόπου εκείνου εις Ιεροσόλυμα προ του πάσχα, διά να καθαρίσωσιν εαυτούς.
\par 56 Εζήτουν λοιπόν τον Ιησούν και έλεγον προς αλλήλους ιστάμενοι εν τω ιερώ· Τι σας φαίνεται ότι δεν θέλει ελθεί εις την εορτήν;
\par 57 Είχον δε δώσει προσταγήν και οι αρχιερείς και οι Φαρισαίοι, εάν τις μάθη που είναι, να μηνύση, διά να πιάσωσιν αυτόν.

\chapter{12}

\par 1 Ο Ιησούς λοιπόν προ εξ ημερών του πάσχα ήλθεν εις Βηθανίαν, όπου ήτο ο Λάζαρος ο αποθανών, τον οποίον ανέστησεν εκ νεκρών.
\par 2 Και έκαμαν εις αυτόν δείπνον εκεί, και η Μάρθα υπηρέτει· ο δε Λάζαρος ήτο εις εκ των συγκαθημένων μετ' αυτού.
\par 3 Τότε η Μαρία, λαβούσα μίαν λίτραν μύρου νάρδου καθαράς πολυτίμου, ήλειψε τους πόδας του Ιησού και με τας τρίχας αυτής εσπόγγισε τους πόδας αυτού· η δε οικία επλήσθη εκ της οσμής του μύρου.
\par 4 Λέγει λοιπόν εις εκ των μαθητών αυτού, ο Ιούδας Σίμωνος ο Ισκαριώτης, όστις έμελλε να παραδώση αυτόν·
\par 5 Διά τι τούτο το μύρον δεν επωλήθη τριακόσια δηνάρια και εδόθη εις τους πτωχούς;
\par 6 Είπε δε τούτο ουχί διότι έμελεν αυτόν περί των πτωχών, αλλά διότι ήτο κλέπτης και είχε το γλωσσόκομον και εβάσταζε τα βαλλόμενα εις αυτό.
\par 7 Είπε λοιπόν ο Ιησούς· Άφες αυτήν, εις την ημέραν του ενταφιασμού μου εφύλαξεν αυτό.
\par 8 Διότι τους πτωχούς πάντοτε έχετε μεθ' εαυτών, εμέ όμως πάντοτε δεν έχετε.
\par 9 Έμαθε δε όχλος πολύς εκ των Ιουδαίων ότι είναι εκεί, και ήλθον ουχί διά τον Ιησούν μόνον, αλλά διά να ίδωσι και τον Λάζαρον, τον οποίον ανέστησεν εκ νεκρών.
\par 10 Συνεβουλεύθησαν δε οι αρχιερείς, διά να θανατώσωσι και τον Λάζαρον,
\par 11 διότι πολλοί εκ των Ιουδαίων δι' αυτόν υπήγαινον και επίστευον εις τον Ιησούν.
\par 12 Τη επαύριον όχλος πολύς ο ελθών εις την εορτήν, ακούσαντες ότι έρχεται ο Ιησούς εις Ιεροσόλυμα,
\par 13 έλαβον τα βαΐα των φοινίκων και εξήλθον εις υπάντησιν αυτού και έκραζον· Ωσαννά, ευλογημένος ο ερχόμενος εν ονόματι Κυρίου, ο βασιλεύς του Ισραήλ.
\par 14 Ευρών δε ο Ιησούς ονάριον, εκάθησεν επ' αυτό, καθώς είναι γεγραμμένον·
\par 15 Μη φοβού, θύγατερ Σιών· ιδού, ο βασιλεύς σου έρχεται καθήμενος επί πώλου όνου.
\par 16 Ταύτα όμως δεν ενόησαν οι μαθηταί αυτού κατ' αρχάς, αλλ' ότε εδοξάσθη ο Ιησούς, τότε ενεθυμήθησαν ότι ταύτα ήσαν γεγραμμένα δι' αυτόν, και ταύτα έκαμον εις αυτόν.
\par 17 Εμαρτύρει λοιπόν ο όχλος, ο ων μετ' αυτού ότε εφώναξε τον Λάζαρον εκ του μνημείου και ανέστησεν αυτόν εκ νεκρών.
\par 18 Διά τούτο και υπήντησεν αυτόν ο όχλος, διότι ήκουσεν ότι έκαμε το θαύμα τούτο.
\par 19 Οι Φαρισαίοι λοιπόν είπον προς αλλήλους· Βλέπετε ότι δεν ωφελείτε ουδέν; ιδού, ο κόσμος οπίσω αυτού υπήγεν.
\par 20 Ήσαν δε τινές Έλληνες μεταξύ των αναβαινόντων διά να προσκυνήσωσιν εν τη εορτή.
\par 21 Ούτοι λοιπόν ήλθον προς τον Φίλιππον τον από Βηθσαϊδά της Γαλιλαίας, και παρεκάλουν αυτόν, λέγοντες· Κύριε, θέλομεν να ίδωμεν τον Ιησούν.
\par 22 Έρχεται ο Φίλιππος και λέγει προς τον Ανδρέαν, και πάλιν ο Ανδρέας και ο Φίλιππος λέγουσι προς τον Ιησούν.
\par 23 Ο δε Ιησούς απεκρίθη προς αυτούς λέγων· Ήλθεν η ώρα διά να δοξασθή ο Υιός του ανθρώπου.
\par 24 Αληθώς, αληθώς σας λέγω, Εάν ο κόκκος του σίτου δεν πέση εις την γην και αποθάνη, αυτός μόνος μένει· εάν όμως αποθάνη, πολύν καρπόν φέρει.
\par 25 Όστις αγαπά την ψυχήν αυτού, θέλει απολέση αυτήν, και όστις μισεί την ψυχήν αυτού εν τω κόσμω τούτω, εις ζωήν αιώνιον θέλει φυλάξει αυτήν.
\par 26 Εάν εμέ υπηρετή τις, εμέ ας ακολουθή, και όπου είμαι εγώ, εκεί θέλει είσθαι και ο υπηρέτης ο εμός· και εάν τις εμέ υπηρετή, θέλει τιμήσει αυτόν ο Πατήρ.
\par 27 Τώρα η ψυχή μου είναι τεταραγμένη· και τι να είπω; Πάτερ, σώσον με εκ της ώρας ταύτης. Αλλά διά τούτο ήλθον εις την ώραν ταύτην.
\par 28 Πάτερ, δόξασόν σου το όνομα. Ήλθε λοιπόν φωνή εκ του ουρανού· Και εδόξασα και πάλιν θέλω δοξάσει.
\par 29 Ο όχλος λοιπόν ο παρεστώς και ακούσας έλεγεν ότι έγεινε βροντή· άλλοι έλεγον· Άγγελος ελάλησε προς αυτόν.
\par 30 Απεκρίθη ο Ιησούς και είπεν· Η φωνή αύτη δεν έγεινε δι' εμέ, αλλά διά σας.
\par 31 Τώρα είναι κρίσις του κόσμου τούτου, τώρα ο άρχων του κόσμου τούτου θέλει εκβληθή έξω.
\par 32 Και εγώ εάν υψωθώ εκ της γης, θέλω ελκύσει πάντας προς εμαυτόν.
\par 33 Τούτο δε έλεγε, δεικνύων με ποίον θάνατον έμελλε να αποθάνη.
\par 34 Απεκρίθη προς αυτόν ο όχλος· Ημείς ηκούσαμεν εκ του νόμου Ότι ο Χριστός μένει εις τον αιώνα, και πως συ λέγεις Ότι πρέπει να υψωθή ο Υιός του ανθρώπου; τις είναι ούτος ο Υιός του ανθρώπου;
\par 35 Είπε λοιπόν προς αυτούς ο Ιησούς· Έτι ολίγον καιρόν το φως είναι μεθ' υμών· περιπατείτε ενόσω έχετε το φως, διά να μη σας καταφθάση το σκότος· και όστις περιπατεί εν τω σκότει δεν εξεύρει που υπάγει.
\par 36 Ενόσω έχετε το φως, πιστεύετε εις το φως, διά να γείνητε υιοί του φωτός. Ταύτα ελάλησεν ο Ιησούς, και απελθών εκρύφθη απ' αυτών.
\par 37 Αλλ' ενώ έκαμε τόσα θαύματα έμπροσθεν αυτών, δεν επίστευον εις αυτόν·
\par 38 διά να πληρωθή ο λόγος του προφήτου Ησαΐου, τον οποίον είπε· Κύριε, τις επίστευσεν εις το κήρυγμα ημών; και ο βραχίων του Κυρίου εις τίνα απεκαλύφθη;
\par 39 Διά τούτο δεν ηδύναντο να πιστεύωσι διότι πάλιν είπεν ο Ησαΐας·
\par 40 Ετύφλωσε τους οφθαλμούς αυτών και εσκλήρυνε την καρδίαν αυτών, διά να μη ίδωσι με τους οφθαλμούς και νοήσωσι με την καρδίαν και επιστρέψωσι, και ιατρεύσω αυτούς.
\par 41 Ταύτα είπεν ο Ησαΐας, ότε είδε την δόξαν αυτού και ελάλησε περί αυτού.
\par 42 Αλλ' όμως και εκ των αρχόντων πολλοί επίστευσαν εις αυτόν, πλην διά τους Φαρισαίους δεν ώμολόγουν, διά να μη γείνωσιν αποσυνάγωγοι.
\par 43 Διότι ηγάπησαν την δόξαν των ανθρώπων μάλλον παρά την δόξαν του Θεού.
\par 44 Ο δε Ιησούς έκραξε και είπεν· Ο πιστεύων εις εμέ δεν πιστεύει εις εμέ, αλλ' εις τον πέμψαντά με,
\par 45 και ο θεωρών εμέ θεωρεί τον πέμψαντά με.
\par 46 Εγώ ήλθον φως εις τον κόσμον, διά να μη μείνη εν τω σκότει πας ο πιστεύων εις εμέ.
\par 47 Και εάν τις ακούση τους λόγους μου και δεν πιστεύση, εγώ δεν κρίνω αυτόν· διότι δεν ήλθον διά να κρίνω τον κόσμον, αλλά διά να σώσω τον κόσμον.
\par 48 Ο αθετών εμέ και μη δεχόμενος τους λόγους μου, έχει τον κρίνοντα αυτόν· ο λόγος, τον οποίον ελάλησα, εκείνος θέλει κρίνει αυτόν εν τη εσχάτη ημέρα·
\par 49 διότι εγώ απ' εμαυτού δεν ελάλησα, αλλ' ο πέμψας με Πατήρ αυτός μοι έδωκεν εντολήν τι να είπω και τι να λαλήσω·
\par 50 και εξεύρω ότι η εντολή αυτού είναι ζωή αιώνιος. Όσα λοιπόν λαλώ εγώ, καθώς μοι είπεν ο Πατήρ, ούτω λαλώ.

\chapter{13}

\par 1 Προ δε της εορτής του πάσχα εξεύρων ο Ιησούς ότι ήλθεν η ώρα αυτού διά να μεταβή εκ του κόσμου τούτου προς τον Πατέρα, αγαπήσας τους ιδικούς του τους εν τω κόσμω, μέχρι τέλους ηγάπησεν αυτούς.
\par 2 Και αφού έγεινε δείπνος, ο δε διάβολος είχεν ήδη βάλει εις την καρδίαν του Ιούδα Σίμωνος του Ισκαριώτου να παραδώση αυτόν,
\par 3 εξεύρων ο Ιησούς ότι πάντα έδωκεν εις αυτόν ο Πατήρ εις τας χείρας, και ότι από του Θεού εξήλθε και προς τον Θεόν υπάγει,
\par 4 εγείρεται εκ του δείπνου και εκδύεται τα ιμάτια αυτού, και λαβών προσόψιον διεζώσθη·
\par 5 έπειτα βάλλει ύδωρ εις τον νιπτήρα, και ήρχισε να νίπτη τους πόδας των μαθητών και να σπογγίζη με το προσόψιον, με το οποίον ήτο διεζωσμένος.
\par 6 Έρχεται λοιπόν προς τον Σίμωνα Πέτρον, και λέγει προς αυτόν εκείνος· Κύριε, συ μου νίπτεις τους πόδας;
\par 7 Απεκρίθη ο Ιησούς και είπε προς αυτόν· Εκείνο, το οποίον εγώ κάμνω, συ δεν εξεύρεις τώρα, θέλεις όμως γνωρίσει μετά ταύτα.
\par 8 Λέγει προς αυτόν ο Πέτρος· Δεν θέλεις νίψει τους πόδας μου εις τον αιώνα. Απεκρίθη προς αυτόν ο Ιησούς· Εάν δεν σε νίψω, δεν έχεις μέρος μετ' εμού.
\par 9 Λέγει προς αυτόν ο Σίμων Πέτρος· Κύριε, μη τους πόδας μου μόνον, αλλά και τας χείρας και την κεφαλήν.
\par 10 Λέγει προς αυτόν ο Ιησούς· Ο λελουμένος δεν έχει χρείαν ειμή τους πόδας να νιφθή, αλλ' είναι όλος καθαρός· και σεις είσθε καθαροί, αλλ' ουχί πάντες.
\par 11 Διότι ήξευρεν εκείνον, όστις έμελλε να παραδώση αυτόν· διά τούτο είπε· Δεν είσθε πάντες καθαροί.
\par 12 Αφού λοιπόν ένιψε τους πόδας αυτών και έλαβε τα ιμάτια αυτού, καθήσας πάλιν είπε προς αυτούς· Εξεύρετε τι έκαμον εις εσάς;
\par 13 Σεις με φωνάζετε, Ο Διδάσκαλος και ο Κύριος, και καλώς λέγετε, διότι είμαι.
\par 14 Εάν λοιπόν εγώ, ο Κύριος και ο Διδάσκαλος, σας ένιψα τους πόδας, και σεις χρεωστείτε να νίπτητε τους πόδας αλλήλων.
\par 15 Διότι παράδειγμα έδωκα εις εσάς, διά να κάμνητε και σεις, καθώς εγώ έκαμον εις εσάς.
\par 16 Αληθώς, αληθώς σας λέγω, δεν είναι δούλος ανώτερος του κυρίου αυτού, ουδέ απόστολος ανώτερος του πέμψαντος αυτόν.
\par 17 Εάν εξεύρητε ταύτα, μακάριοι είσθε εάν κάμνητε αυτά.
\par 18 Δεν λέγω τούτο περί πάντων υμών· εγώ εξεύρω ποίους εξέλεξα· αλλά διά να πληρωθή η γραφή, Ο τρώγων μετ' εμού τον άρτον εσήκωσεν επ' εμέ την πτέρναν αυτού.
\par 19 Από του νυν σας λέγω τούτο πριν γείνη, διά να πιστεύσητε όταν γείνη, ότι εγώ είμαι.
\par 20 Αληθώς, αληθώς σας λέγω, όστις δέχεται όντινα πέμψω, εμέ δέχεται, και όστις δέχεται, εμέ δέχεται τον πέμψαντά με.
\par 21 Αφού είπε ταύτα ο Ιησούς, εταράχθη την ψυχήν και εμαρτύρησε και είπεν· Αληθώς, αληθώς σας λέγω ότι εις εξ υμών θέλει με παραδώσει.
\par 22 Έβλεπον λοιπόν εις αλλήλους οι μαθηταί, απορούντες περί τίνος λέγει.
\par 23 Εκάθητο δε κεκλιμένος εις τον κόλπον του Ιησού εις των μαθητών αυτού, τον οποίον ηγάπα ο Ιησούς.
\par 24 Νεύει λοιπόν προς τούτον ο Σίμων Πέτρος διά να ερωτήση τις είναι εκείνος, περί του οποίου λέγει.
\par 25 Και πεσών εκείνος επί το στήθος του Ιησού, λέγει προς αυτόν· Κύριε, τις είναι;
\par 26 Αποκρίνεται ο Ιησούς· Εκείνος είναι, εις τον οποίον εγώ βάψας το ψωμίον θέλω δώσει. Και εμβάψας το ψωμίον δίδει εις τον Ιούδαν Σίμωνος τον Ισκαριώτην.
\par 27 Και μετά το ψωμίον τότε εισήλθεν εις εκείνον ο Σατανάς. Λέγει λοιπόν προς αυτόν ο Ιησούς· ό,τι κάμνεις, κάμε ταχύτερον.
\par 28 Τούτο όμως ουδείς των καθημένων ενόησε προς τι είπε προς αυτόν.
\par 29 Διότι τινές ενόμιζον, επειδή ο Ιούδας είχε το γλωσσόκομον, ότι λέγει προς αυτόν ο Ιησούς, Αγόρασον όσων έχομεν χρείαν διά την εορτήν, ή να δώση τι εις τους πτωχούς.
\par 30 Λαβών λοιπόν εκείνος το ψωμίον, εξήλθεν ευθύς· ήτο δε νυξ.
\par 31 Ότε λοιπόν εξήλθε, λέγει ο Ιησούς· Τώρα εδοξάσθη ο Υιός του άνθρώπου, και ο Θεός εδοξάσθη εν αυτώ.
\par 32 Εάν ο Θεός εδοξάσθη εν αυτώ, και ο Θεός θέλει δοξάσει αυτόν εν εαυτώ και ευθύς θέλει δοξάσει αυτόν.
\par 33 Τεκνία, έτι ολίγον είμαι μεθ' υμών. Θέλετε με ζητήσει, και καθώς είπον προς τους Ιουδαίους ότι όπου υπάγω εγώ, σεις δεν δύνασθε να έλθητε, και προς εσάς λέγω τώρα.
\par 34 Εντολήν καινήν σας δίδω, Να αγαπάτε αλλήλους, καθώς εγώ σας ηγάπησα και σεις να αγαπάτε αλλήλους.
\par 35 Εκ τούτου θέλουσι γνωρίσει πάντες ότι είσθε μαθηταί μου, εάν έχητε αγάπην προς αλλήλους.
\par 36 Λέγει προς αυτόν ο Σίμων Πέτρος· Κύριε, που υπάγεις; Απεκρίθη εις αυτόν ο Ιησούς· Όπου υπάγω, δεν δύνασαι τώρα να με ακολουθήσης, ύστερον όμως θέλεις με ακολουθήσει.
\par 37 Λέγει προς αυτόν ο Πέτρος· Κύριε, διατί δεν δύναμαι να σε ακολουθήσω τώρα; την ψυχήν μου θέλω βάλει υπέρ σου.
\par 38 Απεκρίθη προς αυτόν ο Ιησούς· Την ψυχήν σου θέλεις βάλει υπέρ εμού; αληθώς, αληθώς σοι λέγω, δεν θέλει φωνάξει ο αλέκτωρ, εωσού με απαρνηθής τρίς.

\chapter{14}

\par 1 Ας μη ταράττηται η καρδία σας· πιστεύετε εις τον Θεόν, και εις εμέ πιστεύετε.
\par 2 Εν τη οικία του Πατρός μου είναι πολλά οικήματα· ει δε μη, ήθελον σας ειπεί· υπάγω να σας ετοιμάσω τόπον·
\par 3 και αφού υπάγω και σας ετοιμάσω τόπον, πάλιν έρχομαι και θέλω σας παραλάβει προς εμαυτόν, διά να είσθε και σεις, όπου είμαι εγώ.
\par 4 Και όπου εγώ υπάγω εξεύρετε, και την οδόν εξεύρετε.
\par 5 Λέγει προς αυτόν ο Θωμάς· Κύριε, δεν εξεύρομεν που υπάγεις· και πως δυνάμεθα να εξεύρωμεν την οδόν;
\par 6 Λέγει προς αυτόν ο Ιησούς· Εγώ είμαι η οδός και η αλήθεια και η ζωή· ουδείς έρχεται προς τον Πατέρα, ειμή δι' εμού.
\par 7 Εάν εγνωρίζετε εμέ, και τον Πατέρα μου ηθέλετε γνωρίσει. Και από του νυν γνωρίζετε αυτόν και είδετε αυτόν.
\par 8 Λέγει προς αυτόν ο Φίλιππος· Κύριε, δείξον εις ημάς τον Πατέρα και αρκεί εις ημάς.
\par 9 Λέγει προς αυτόν ο Ιησούς· Τόσον καιρόν είμαι μεθ' υμών, και δεν με εγνώρισας, Φίλιππε; όστις είδεν εμέ είδε τον Πατέρα· και πως συ λέγεις, Δείξον εις ημάς τον Πατέρα;
\par 10 Δεν πιστεύεις ότι εγώ είμαι εν τω Πατρί και ο Πατήρ είναι εν εμοί; τους λόγους, τους οποίους εγώ λαλώ προς υμάς, απ' εμαυτού δεν λαλώ· αλλ' ο Πατήρ ο μένων εν εμοί αυτός εκτελεί τα έργα.
\par 11 Πιστεύετέ μοι ότι εγώ είμαι εν τω Πατρί και ο Πατήρ είναι εν εμοί· ει δε μη, διά τα έργα αυτά πιστεύετέ μοι.
\par 12 Αληθώς, αληθώς σας λέγω, όστις πιστεύει εις εμέ, τα έργα τα οποία κάμνω και εκείνος θέλει κάμει, και μεγαλήτερα τούτων θέλει κάμει, διότι εγώ υπάγω προς τον Πατέρα μου,
\par 13 και ό,τι αν ζητήσητε εν τω ονόματί μου, θέλω κάμει τούτο, διά να δοξασθή ο Πατήρ εν τω Υιώ.
\par 14 Εάν ζητήσητέ τι εν τω ονόματί μου, εγώ θέλω κάμει αυτό.
\par 15 Εάν με αγαπάτε, τας εντολάς μου φυλάξατε.
\par 16 Και εγώ θέλω παρακαλέσει τον Πατέρα, και θέλει σας δώσει άλλον Παράκλητον, διά να μένη μεθ' υμών εις τον αιώνα,
\par 17 το Πνεύμα της αληθείας, το οποίον ο κόσμος δεν δύναται να λάβη, διότι δεν βλέπει αυτό ουδέ γνωρίζει αυτό· σεις όμως γνωρίζετε αυτό, διότι μένει μεθ' υμών και εν υμίν θέλει είσθαι.
\par 18 Δεν θέλω σας αφήσει ορφανούς· έρχομαι προς εσάς.
\par 19 Έτι ολίγον και ο κόσμος πλέον δεν με βλέπει, σεις όμως με βλέπετε, διότι εγώ ζω και σεις θέλετε ζη.
\par 20 Εν εκείνη τη ημέρα σεις θέλετε γνωρίσει, ότι εγώ είμαι εν τω Πατρί μου και σεις εν εμοί και εγώ εν υμίν.
\par 21 Ο έχων τας εντολάς μου και φυλάττων αυτάς, εκείνος είναι ο αγαπών με· ο δε αγαπών με θέλει αγαπηθή υπό του Πατρός μου, και εγώ θέλω αγαπήσει αυτόν και θέλω φανερώσει εμαυτόν εις αυτόν.
\par 22 Λέγει προς αυτόν ο Ιούδας, ουχί ο Ισκαριώτης· Κύριε, τι συμβαίνει ότι μέλλεις να φανερώσης σεαυτόν εις ημάς και ουχί εις τον κόσμον;
\par 23 Απεκρίθη ο Ιησούς και είπε προς αυτόν· Εάν τις με αγαπά, τον λόγον μου θέλει φυλάξει, και ο Πατήρ μου θέλει αγαπήσει αυτόν, και προς αυτόν θέλομεν ελθεί και εν αυτώ θέλομεν κατοικήσει.
\par 24 Ο μη αγαπών με τους λόγους μου δεν φυλάττει· και ο λόγος, τον οποίον ακούετε, δεν είναι ιδικός μου, αλλά του πέμψαντός με Πατρός.
\par 25 Ταύτα ελάλησα προς εσάς ενώ ευρίσκομαι μεθ' υμών·
\par 26 ο δε Παράκλητος, το Πνεύμα το Άγιον, το οποίον θέλει πέμψει ο Πατήρ εν τω ονόματί μου, εκείνος θέλει σας διδάξει πάντα και θέλει σας υπενθυμίσει πάντα όσα είπον προς εσάς.
\par 27 Ειρήνην αφίνω εις εσάς, ειρήνην την εμήν δίδω εις εσάς· ουχί καθώς ο κόσμος δίδει, σας δίδω εγώ. Ας μη ταράττηται η καρδία σας μηδέ ας δειλιά.
\par 28 Ηκούσατε ότι εγώ σας είπον, Υπάγω και έρχομαι προς εσάς. Εάν με ηγαπάτε, ηθέλετε χαρή ότι είπον, Υπάγω προς τον Πατέρα· διότι ο Πατήρ μου είναι μεγαλήτερός μου·
\par 29 και τώρα σας είπον πριν γείνη, διά να πιστεύσητε όταν γείνη.
\par 30 Δεν θέλω πλέον λαλήσει πολλά μεθ' υμών· διότι έρχεται ο άρχων του κόσμου τούτου· και δεν έχει ουδέν εν εμοί.
\par 31 Αλλά διά να γνωρίση ο κόσμος ότι αγαπώ τον Πατέρα, και καθώς με προσέταξεν ο Πατήρ, ούτω κάμνω. Εγέρθητε, ας υπάγωμεν εντεύθεν.

\chapter{15}

\par 1 Εγώ είμαι η άμπελος η αληθινή, και ο Πατήρ μου είναι ο γεωργός.
\par 2 Παν κλήμα εν εμοί μη φέρον καρπόν, εκκόπτει αυτό, και παν το φέρον καρπόν, καθαρίζει αυτό, διά να φέρη πλειότερον καρπόν.
\par 3 Τώρα σεις είσθε καθαροί διά τον λόγον τον οποίον ελάλησα προς εσάς.
\par 4 Μείνατε εν εμοί, και εγώ εν υμίν. Καθώς το κλήμα δεν δύναται να φέρη καρπόν αφ' εαυτού, εάν δεν μείνη εν τη αμπέλω, ούτως ουδέ σεις, εάν δεν μείνητε εν εμοί.
\par 5 Εγώ είμαι η άμπελος, σεις τα κλήματα. Ο μένων εν εμοί και εγώ εν αυτώ, ούτος φέρει καρπόν πολύν, διότι χωρίς εμού δεν δύνασθε να κάμητε ουδέν.
\par 6 Εάν τις δεν μείνη εν εμοί, ρίπτεται έξω ως το κλήμα και ξηραίνεται, και συνάγουσιν αυτά και ρίπτουσιν εις το πυρ, και καίονται.
\par 7 Εάν μείνητε εν εμοί και οι λόγοι μου μείνωσιν εν υμίν, θέλετε ζητεί ό,τι αν θέλητε, και θέλει γείνει εις εσάς.
\par 8 Εν τούτω δοξάζεται ο Πατήρ μου, εις το να φέρητε καρπόν πολύν· και ούτω θέλετε είσθαι μαθηταί μου.
\par 9 Καθώς εμέ ηγάπησεν ο Πατήρ, και εγώ ηγάπησα εσάς· μείνατε εν τη αγάπη μου.
\par 10 Εάν τας εντολάς μου φυλάξητε, θέλετε μείνει εν τη αγάπη μου, καθώς εγώ εφύλαξα τας εντολάς του Πατρός μου και μένω εν τη αγάπη αυτού.
\par 11 Ταύτα ελάλησα προς εσάς διά να μείνη εν υμίν η χαρά μου και η χαρά υμών να ήναι πλήρης.
\par 12 Αύτη είναι η εντολή μου, να αγαπάτε αλλήλους, καθώς σας ηγάπησα.
\par 13 Μεγαλητέραν ταύτης αγάπην δεν έχει ουδείς, του να βάλη τις την ψυχήν αυτού υπέρ των φίλων αυτού.
\par 14 Σεις είσθε φίλοι μου, εάν κάμνητε όσα εγώ σας παραγγέλλω.
\par 15 Δεν σας λέγω πλέον δούλους, διότι ο δούλος δεν εξεύρει τι κάμνει ο κύριος αυτού· εσάς δε είπον φίλους, διότι πάντα όσα ήκουσα παρά του Πατρός μου, εφανέρωσα εις εσάς.
\par 16 Σεις δεν εξελέξατε εμέ, αλλ' εγώ εξέλεξα εσάς, και σας διέταξα διά να υπάγητε σεις και να κάμητε καρπόν, και ο καρπός σας να μένη, ώστε, ό,τι αν ζητήσητε παρά του Πατρός εν τω ονόματί μου, να σας δώση αυτό.
\par 17 Ταύτα σας παραγγέλλω, να αγαπάτε αλλήλους.
\par 18 Εάν ο κόσμος σας μισή, εξεύρετε ότι εμέ πρότερον υμών εμίσησεν.
\par 19 Εάν ήσθε εκ του κόσμου, ο κόσμος ήθελεν αγαπά το ιδικόν του· επειδή όμως δεν είσθε εκ του κόσμου, αλλ' εγώ σας εξέλεξα εκ του κόσμου, διά τούτο σας μισεί ο κόσμος.
\par 20 Ενθυμείσθε τον λόγον, τον οποίον εγώ είπον προς εσάς· Δεν είναι δούλος μεγαλήτερος του κυρίου αυτού. Εάν εμέ εδίωξαν, και σας θέλουσι διώξει· εάν τον λόγον μου εφύλαξαν, και τον υμέτερον θέλουσι φυλάξει.
\par 21 Αλλά ταύτα πάντα θέλουσι κάμει εις εσάς διά το όνομά μου, διότι δεν εξεύρουσι τον πέμψαντά με.
\par 22 Εάν δεν ήλθον και ελάλησα προς αυτούς, αμαρτίαν δεν ήθελον έχει· τώρα όμως δεν έχουσι πρόφασιν περί της αμαρτίας αυτών.
\par 23 Ο μισών εμέ και τον Πατέρα μου μισεί.
\par 24 Εάν δεν έκαμον μεταξύ αυτών τα έργα, τα οποία ουδείς άλλος έκαμεν, αμαρτίαν δεν ήθελον έχει· αλλά τώρα και είδον και εμίσησαν και εμέ και τον Πατέρα μου.
\par 25 Αλλά τούτο έγεινε διά να πληρωθή ο λόγος, ο γεγραμμένος εν τω νόμω αυτών, Ότι εμίσησάν με δωρεάν.
\par 26 Όταν όμως έλθη ο Παράκλητος, τον οποίον εγώ θέλω πέμψει προς εσάς παρά του Πατρός, το Πνεύμα της αληθείας, το οποίον εκπορεύεται παρά του Πατρός, εκείνος θέλει μαρτυρήσει περί εμού.
\par 27 Αλλά και σεις μαρτυρείτε, διότι απ' αρχής μετ' εμού είσθε.

\chapter{16}

\par 1 Ταύτα ελάλησα προς εσάς διά να μη σκανδαλισθήτε.
\par 2 Θέλουσι σας κάμει αποσυναγώγους· μάλιστα έρχεται ώρα, καθ' ην πας όστις σας θανατώση θέλει νομίσει ότι προσφέρει λατρείαν εις τον Θεόν.
\par 3 Και ταύτα θέλουσι σας κάμει, διότι δεν εγνώρισαν τον Πατέρα ουδέ εμέ.
\par 4 Αλλά ταύτα είπον προς εσάς διά να ενθυμήσθε αυτά, όταν έλθη η ώρα, ότι εγώ είπον προς εσάς. Δεν είπον δε ταύτα προς εσάς εξ αρχής, διότι ήμην μεθ' υμών.
\par 5 Τώρα δε υπάγω προς τον πέμψαντά με, και ουδείς εξ υμών με ερωτά· Που υπάγεις;
\par 6 Αλλ' επειδή ελάλησα προς εσάς ταύτα, η λύπη εγέμισε την καρδίαν σας.
\par 7 Εγώ όμως την αλήθειαν σας λέγω· συμφέρει εις εσάς να απέλθω εγώ. Διότι εάν δεν απέλθω, ο Παράκλητος δεν θέλει ελθεί προς εσάς· αλλ' αφού απέλθω, θέλω πέμψει αυτόν προς εσάς·
\par 8 και ελθών εκείνος θέλει ελέγξει τον κόσμον περί αμαρτίας και περί δικαιοσύνης και περί κρίσεως·
\par 9 περί αμαρτίας μεν, διότι δεν πιστεύουσιν εις εμέ·
\par 10 περί δικαιοσύνης δε, διότι υπάγω προς τον Πατέρα μου και πλέον δεν με βλέπετε·
\par 11 περί δε κρίσεως, διότι ο άρχων του κόσμου τούτου εκρίθη.
\par 12 Έτι πολλά έχω να είπω προς εσάς, δεν δύνασθε όμως τώρα να βαστάζητε αυτά.
\par 13 Όταν δε έλθη εκείνος, το Πνεύμα της αληθείας, θέλει σας οδηγήσει εις πάσαν την αλήθειαν· διότι δεν θέλει λαλήσει αφ' εαυτού, αλλ' όσα αν ακούση θέλει λαλήσει, και θέλει σας αναγγείλει τα μέλλοντα.
\par 14 Εκείνος θέλει δοξάσει εμέ, διότι εκ του εμού θέλει λάβει και αναγγείλει προς εσάς.
\par 15 Πάντα όσα έχει ο Πατήρ, εμού είναι· διά τούτο είπον ότι εκ του εμού θέλει λάβει και αναγγείλει προς εσάς.
\par 16 Ολίγον έτι και δεν με βλέπετε, και πάλιν ολίγον και θέλετε με ιδεί, διότι εγώ υπάγω προς τον Πατέρα.
\par 17 Τότε τινές εκ των μαθητών αυτού είπον προς αλλήλους· Τι είναι τούτο, το οποίον μας λέγει, Ολίγον και δεν με βλέπετε, και πάλιν ολίγον και θέλετε με ιδεί, και, Ότι εγώ υπάγω προς τον Πατέρα;
\par 18 Έλεγον λοιπόν· Τούτο τι είναι, το οποίον λέγει το ολίγον; Δεν εξεύρομεν τι λαλεί.
\par 19 Ενόησε λοιπόν ο Ιησούς ότι ήθελον να ερωτήσωσιν αυτόν, και είπε προς αυτούς· Περί τούτου συζητείτε μετ' αλλήλων ότι είπον, Ολίγον και δεν με βλέπετε, και πάλιν ολίγον και θέλετε με ιδεί;
\par 20 Αληθώς, αληθώς σας λέγω ότι σεις θέλετε κλαύσει και θρηνήσει, ο δε κόσμος θέλει χαρή· και σεις θέλετε λυπηθή, η λύπη σας όμως θέλει μεταβληθή εις χαράν.
\par 21 Η γυνή όταν γεννά, λύπην έχει, διότι ήλθεν ώρα αυτής· αφού όμως γεννήση το παιδίον, δεν ενθυμείται πλέον την θλίψιν, διά την χαράν ότι εγεννήθη άνθρωπος εις τον κόσμον.
\par 22 Και σεις λοιπόν τώρα μεν έχετε λύπην· πάλιν όμως θέλω σας ιδεί, και θέλει χαρή η καρδία σας, και την χαράν σας ουδείς αφαιρεί από σας.
\par 23 Και εν εκείνη τη ημέρα δεν θέλετε ζητήσει παρ' εμού ουδέν. Αληθώς, αληθώς σας λέγω ότι όσα αν αιτήσητε παρά του Πατρός εν τω ονόματί μου, θέλει σας δώσει.
\par 24 Έως τώρα δεν ητήσατε ουδέν εν τω ονόματί μου· αιτείτε και θέλετε λαμβάνει, διά να ήναι πλήρης η χαρά σας.
\par 25 Ταύτα διά παροιμιών ελάλησα προς εσάς· αλλ' έρχεται ώρα, ότε δεν θέλω σας λαλήσει πλέον διά παροιμιών, αλλά παρρησία θέλω σας αναγγείλει περί του Πατρός.
\par 26 Εν εκείνη τη ημέρα θέλετε ζητήσει εν τω ονόματί μου· και δεν σας λέγω ότι εγώ θέλω παρακαλέσει τον Πατέρα περί υμών·
\par 27 διότι αυτός ο Πατήρ σας αγαπά, επειδή σεις ηγαπήσατε εμέ και επιστεύσατε ότι εγώ παρά του Θεού εξήλθον.
\par 28 Εξήλθον παρά του Πατρός και ήλθον εις τον κόσμον· πάλιν αφίνω τον κόσμον και υπάγω προς τον Πατέρα.
\par 29 Λέγουσι προς αυτόν οι μαθηταί αυτού· Ιδού, τώρα παρρησία λαλείς και, ουδεμίαν παροιμίαν λέγεις.
\par 30 Τώρα γνωρίζομεν ότι εξεύρεις πάντα και δεν έχεις χρείαν να σε ερωτά τις. Εκ τούτου πιστεύομεν ότι από Θεού εξήλθες.
\par 31 Απεκρίθη προς αυτούς ο Ιησούς· Τώρα πιστεύετε;
\par 32 Ιδού, έρχεται ώρα, και ήδη ήλθε, να σκορπισθήτε έκαστος εις τα ίδια και να αφήσητε εμέ μόνον· αλλά δεν είμαι μόνος, διότι ο Πατήρ είναι μετ' εμού.
\par 33 Ταύτα ελάλησα προς εσάς, διά να έχητε ειρήνην εν εμοί. Εν τω κόσμω θέλετε έχει θλίψιν· αλλά θαρσείτε, εγώ ενίκησα τον κόσμον.

\chapter{17}

\par 1 Ταύτα ελάλησεν ο Ιησούς, και ύψωσε τους οφθαλμούς αυτού εις τον ουρανόν και είπε· Πάτερ, ήλθεν η ώρα· δόξασον τον Υιόν σου, διά να σε δοξάση και ο Υιός σου,
\par 2 καθώς έδωκας εις αυτόν εξουσίαν πάσης σαρκός, διά να δώση ζωήν αιώνιον εις πάντας όσους έδωκας εις αυτόν.
\par 3 Αύτη δε είναι η αιώνιος ζωή, το να γνωρίζωσι σε τον μόνον αληθινόν Θεόν και τον οποίον απέστειλας Ιησούν Χριστόν.
\par 4 Εγώ σε εδόξασα επί της γης, το έργον ετελείωσα, το οποίον μοι έδωκας διά να κάμω·
\par 5 και τώρα δόξασόν με συ, Πάτερ, πλησίον σου με την δόξαν, την οποίαν είχον παρά σοι πριν γείνη ο κόσμος.
\par 6 Εφανέρωσα το όνομά σου εις τους ανθρώπους, τους οποίους μοι έδωκας εκ του κόσμου. Ιδικοί σου ήσαν και εις εμέ έδωκας αυτούς, και τον λόγον σου εφύλαξαν.
\par 7 Τώρα εγνώρισαν ότι πάντα όσα μοι έδωκας παρά σου είναι·
\par 8 διότι τους λόγους, τους οποίους μοι έδωκας, έδωκα εις αυτούς, και αυτοί εδέχθησαν και εγνώρισαν αληθώς ότι παρά σου εξήλθον, και επίστευσαν ότι συ με απέστειλας.
\par 9 Εγώ περί αυτών παρακαλώ· δεν παρακαλώ περί του κόσμου, αλλά περί εκείνων, τους οποίους μοι έδωκας, διότι ιδικοί σου είναι.
\par 10 Και τα εμά πάντα σα είναι και τα σα εμά, και εδοξάσθην εν αυτοίς.
\par 11 Και δεν είμαι πλέον εν τω κόσμω, αλλ' ούτοι είναι εν τω κόσμω, και εγώ έρχομαι προς σε. Πάτερ άγιε, φύλαξον αυτούς εν τω ονόματί σου, τους οποίους μοι έδωκας, διά να ήναι εν καθώς ημείς.
\par 12 Ότε ήμην μετ' αυτών εν τω κόσμω, εγώ εφύλαττον αυτούς εν τω ονόματί σου· εκείνους τους οποίους μοι έδωκας εφύλαξα, και ουδείς εξ αυτών απωλέσθη ειμή ο υιός της απωλείας, διά να πληρωθή η γραφή.
\par 13 Τώρα δε έρχομαι προς σε, και ταύτα λαλώ εν τω κόσμω διά να έχωσι την χαράν μου πλήρη εν εαυτοίς.
\par 14 Εγώ έδωκα εις αυτούς τον λόγον σου, και ο κόσμος εμίσησεν αυτούς, διότι δεν είναι εκ του κόσμου, καθώς εγώ δεν είμαι εκ του κόσμου.
\par 15 Δεν παρακαλώ να σηκώσης αυτούς εκ του κόσμου, αλλά να φυλάξης αυτούς εκ του πονηρού.
\par 16 Εκ του κόσμου δεν είναι, καθώς εγώ δεν είμαι εκ του κόσμου.
\par 17 Αγίασον αυτούς εν τη αληθεία σου· ο λόγος ο ιδικός σου είναι αλήθεια.
\par 18 Καθώς εμέ απέστειλας εις τον κόσμον, και εγώ απέστειλα αυτούς εις τον κόσμον·
\par 19 και υπέρ αυτών εγώ αγιάζω εμαυτόν, διά να ήναι και αυτοί ηγιασμένοι εν τη αληθεία.
\par 20 Και δεν παρακαλώ μόνον περί τούτων, αλλά και περί των πιστευσόντων εις εμέ διά του λόγου αυτών·
\par 21 διά να ήναι πάντες εν, καθώς συ, Πάτερ, είσαι εν εμοί και εγώ εν σοι, να ήναι και αυτοί εν ημίν εν, διά να πιστεύση ο κόσμος ότι συ με απέστειλας.
\par 22 Και εγώ την δόξαν την οποίαν μοι έδωκας έδωκα εις αυτούς, διά να ήναι εν καθώς ημείς είμεθα εν,
\par 23 εγώ εν αυτοίς και συ εν εμοί, διά να ήναι τετελειωμένοι εις εν, και να γνωρίζη ο κόσμος ότι συ με απέστειλας και ηγάπησας αυτούς καθώς εμέ ηγάπησας.
\par 24 Πάτερ, εκείνους τους οποίους μοι έδωκας, θέλω, όπου είμαι εγώ, να ήναι και εκείνοι μετ' εμού, διά να θεωρώσι την δόξαν μου, την οποίαν μοι έδωκας, διότι με ηγάπησας προ καταβολής κόσμου.
\par 25 Πάτερ δίκαιε, και ο κόσμος δεν σε εγνώρισεν, εγώ δε σε εγνώρισα, και ούτοι εγνώρισαν ότι συ με απέστειλας.
\par 26 Και εφανέρωσα εις αυτούς το όνομά σου και θέλω φανερώσει, διά να ήναι η αγάπη, με την οποίαν με ηγάπησας, εν αυτοίς, και εγώ εν αυτοίς.

\chapter{18}

\par 1 Αφού είπε ταύτα ο Ιησούς, εξήλθε μετά των μαθητών αυτού πέραν του χειμάρρου των Κέδρων, όπου ήτο κήπος, εις τον οποίον εισήλθεν αυτός και οι μαθηταί αυτού.
\par 2 Ήξευρε δε τον τόπον και Ιούδας ο παραδίδων αυτόν· διότι πολλάκις συνήλθεν εκεί ο Ιησούς μετά των μαθητών αυτού.
\par 3 Ο Ιούδας λοιπόν, λαβών το τάγμα και εκ των αρχιερέων και Φαρισαίων υπηρέτας, έρχεται εκεί μετά φανών και λαμπάδων και όπλων.
\par 4 Ο δε Ιησούς, εξεύρων πάντα τα ερχόμενα επ' αυτόν, εξήλθε και είπε προς αυτούς· Τίνα ζητείτε;
\par 5 Απεκρίθησαν προς αυτόν· Ιησούν τον Ναζωραίον. Λέγει προς αυτούς ο Ιησούς· Εγώ είμαι. Ίστατο δε μετ' αυτών και Ιούδας ο παραδίδων αυτόν.
\par 6 Καθώς λοιπόν είπε προς αυτούς ότι εγώ είμαι, απεσύρθησαν εις τα οπίσω και έπεσον χαμαί.
\par 7 Πάλιν λοιπόν ηρώτησεν αυτούς· Τίνα ζητείτε; Οι δε είπον· Ιησούν τον Ναζωραίον.
\par 8 Απεκρίθη ο Ιησούς· Σας είπον ότι εγώ είμαι. Εάν λοιπόν εμέ ζητήτε, αφήσατε τούτους να υπάγωσι·
\par 9 διά να πληρωθή ο λόγος, τον οποίον είπεν, Ότι εξ εκείνων τους οποίους μοι έδωκας, δεν απώλεσα ουδένα.
\par 10 Τότε ο Σίμων Πέτρος έχων μάχαιραν έσυρεν αυτήν και εκτύπησε τον δούλον του αρχιερέως και απέκοψεν αυτού το ωτίον το δεξιόν· ήτο δε το όνομα του δούλου Μάλχος.
\par 11 Είπε λοιπόν ο Ιησούς προς τον Πέτρον· Βάλε την μάχαιράν σου εις την θήκην· το ποτήριον, το οποίον μοι έδωκεν ο Πατήρ, δεν θέλω πίει αυτό;
\par 12 Το τάγμα λοιπόν και ο χιλίαρχος και οι υπηρέται των Ιουδαίων συνέλαβον τον Ιησούν και έδεσαν αυτόν,
\par 13 και έφεραν αυτόν εις τον Άνναν πρώτον· διότι ήτο πενθερός του Καϊάφα, όστις ήτο αρχιερεύς του ενιαυτού εκείνου.
\par 14 Ήτο δε ο Καϊάφας ο συμβουλεύσας τους Ιουδαίους ότι συμφέρει να απολεσθή εις άνθρωπος υπέρ του λαού.
\par 15 Ηκολούθει δε τον Ιησούν ο Σίμων Πέτρος και ο άλλος μαθητής. Ο δε μαθητής εκείνος ήτο γνωστός εις τον αρχιερέα και εισήλθε μετά του Ιησού εις την αυλήν του αρχιερέως.
\par 16 Ο δε Πέτρος ίστατο έξω πλησίον της θύρας. Εξήλθε λοιπόν ο μαθητής ο άλλος, όστις ήτο γνωστός εις τον αρχιερέα, και ώμίλησεν εις την θυρωρόν, και εισήγαγε τον Πέτρον.
\par 17 Λέγει λοιπόν η δούλη η θυρωρός προς τον Πέτρον· Μήπως και συ είσαι εκ των μαθητών του ανθρώπου τούτου; Λέγει εκείνος· Δεν είμαι.
\par 18 Ίσταντο δε οι δούλοι και οι υπηρέται, οίτινες είχον κάμει ανθρακιάν, διότι ήτο ψύχος, και εθερμαίνοντο· και μετ' αυτών ίστατο ο Πέτρος και εθερμαίνετο.
\par 19 Ο αρχιερεύς λοιπόν ηρώτησε τον Ιησούν περί των μαθητών αυτού και περί της διδαχής αυτού.
\par 20 Απεκρίθη προς αυτόν ο Ιησούς· Εγώ παρρησία ελάλησα εις τον κόσμον· εγώ πάντοτε εδίδαξα εν τη συναγωγή και εν τω ιερώ, όπου οι Ιουδαίοι συνέρχονται πάντοτε, και εν κρυπτώ δεν ελάλησα ουδέν.
\par 21 Τι με ερωτάς; ερώτησον τους ακούσαντας, τι ελάλησα προς αυτούς· ιδού, ούτοι εξεύρουσιν όσα είπον εγώ.
\par 22 Ότε δε είπε ταύτα, εις των υπηρετών ιστάμενος πλησίον έδωκε ράπισμα εις τον Ιησούν, ειπών· Ούτως αποκρίνεσαι προς τον αρχιερέα;
\par 23 Απεκρίθη προς αυτόν ο Ιησούς· Εάν κακώς ελάλησα, μαρτύρησον περί του κακού· εάν δε καλώς, τι με δέρεις;
\par 24 Είχε δε αποστείλει αυτόν ο Άννας δεδεμένον προς Καϊάφαν τον αρχιερέα.
\par 25 Ο δε Σίμων Πέτρος ίστατο και εθερμαίνετο· είπον λοιπόν προς αυτόν· Μήπως και συ εκ των μαθητών αυτού είσαι; Ηρνήθη εκείνος και είπε· Δεν είμαι.
\par 26 Λέγει εις εκ των δούλων του αρχιερέως, όστις ήτο συγγενής εκείνου, του οποίου ο Πέτρος απέκοψε το ωτίον· Δεν σε είδον εγώ εν τω κήπω μετ' αυτού;
\par 27 Πάλιν λοιπόν ηρνήθη ο Πέτρος, και ευθύς εφώναξεν ο αλέκτωρ.
\par 28 Φέρουσι λοιπόν τον Ιησούν από του Καϊάφα εις το πραιτώριον· ήτο δε πρωΐ· και αυτοί δεν εισήλθον εις το πραιτώριον, διά να μη μιανθώσιν, αλλά διά να φάγωσι το πάσχα.
\par 29 Εξήλθε λοιπόν ο Πιλάτος προς αυτούς και είπε· Τίνα κατηγορίαν φέρετε κατά του ανθρώπου τούτου;
\par 30 Απεκρίθησαν και είπον προς αυτόν· Εάν ούτος δεν ήτο κακοποιός, δεν ηθέλομεν σοι παραδώσει αυτόν.
\par 31 Είπε λοιπόν προς αυτούς ο Πιλάτος· Λάβετε αυτόν σεις και κατά τον νόμον σας κρίνατε αυτόν. Είπον δε προς αυτόν οι Ιουδαίοι· Ημείς δεν έχομεν εξουσίαν να θανατώσωμεν ουδένα.
\par 32 Διά να πληρωθή ο λόγος του Ιησού, τον οποίον είπε, δεικνύων με ποίον θάνατον έμελλε να αποθάνη.
\par 33 Εισήλθε πάλιν εις το πραιτώριον ο Πιλάτος και εφώναξε τον Ιησούν και είπε προς αυτόν· Συ είσαι ο βασιλεύς των Ιουδαίων;
\par 34 Απεκρίθη προς αυτόν ο Ιησούς· Αφ' εαυτού λέγεις συ τούτο, ή άλλοι σοι είπον περί εμού;
\par 35 Απεκρίθη ο Πιλάτος· Μήπως εγώ είμαι Ιουδαίος; το έθνος το ιδικόν σου και οι αρχιερείς σε παρέδωκαν εις εμέ· τι έκαμες;
\par 36 Απεκρίθη ο Ιησούς· Η βασιλεία η εμή δεν είναι εκ του κόσμου τούτου· εάν η βασιλεία η εμή ήτο εκ του κόσμου τούτου, οι υπηρέται μου ήθελον αγωνίζεσθαι, διά να μη παραδωθώ εις τους Ιουδαίους· τώρα δε η βασιλεία η εμή δεν είναι εντεύθεν.
\par 37 Και ο Πιλάτος είπε προς αυτόν· Λοιπόν βασιλεύς είσαι συ; Απεκρίθη ο Ιησούς· Συ λέγεις ότι βασιλεύς είμαι εγώ. Εγώ διά τούτο εγεννήθην και διά τούτο ήλθον εις τον κόσμον, διά να μαρτυρήσω εις την αλήθειαν. Πας όστις είναι εκ της αληθείας ακούει την φωνήν μου.
\par 38 Λέγει προς αυτόν ο Πιλάτος· Τι είναι αλήθεια; Και τούτο ειπών, πάλιν εξήλθε προς τους Ιουδαίους και λέγει προς αυτούς· Εγώ δεν ευρίσκω ουδέν έγκλημα εν αυτώ·
\par 39 είναι δε συνήθεια εις εσάς να σας απολύσω ένα εν τω πάσχα· θέλετε λοιπόν να σας απολύσω τον βασιλέα των Ιουδαίων;
\par 40 Πάλιν λοιπόν εκραύγασαν πάντες, λέγοντες· Μη τούτον, αλλά τον Βαραββάν. Ήτο δε ο Βαραββάς ληστής.

\chapter{19}

\par 1 Τότε λοιπόν έλαβεν ο Πιλάτος τον Ιησούν και εμαστίγωσε.
\par 2 Και οι στρατιώται, πλέξαντες στέφανον εξ ακανθών, έθεσαν επί της κεφαλής αυτού και ενέδυσαν αυτόν ιμάτιον πορφυρούν
\par 3 και έλεγον· Χαίρε βασιλεύ των Ιουδαίων· και έδιδον εις αυτόν ραπίσματα.
\par 4 Εξήλθε δε πάλιν έξω ο Πιλάτος και λέγει προς αυτούς· Ιδού, σας φέρω αυτόν έξω, διά να γνωρίσητε ότι ουδέν έγκλημα ευρίσκω εν αυτώ.
\par 5 Εξήλθε λοιπόν ο Ιησούς έξω, φορών τον ακάνθινον στέφανον και το πορφυρούν ιμάτιον, και λέγει προς αυτούς ο Πιλάτος· Ιδέ ο άνθρωπος.
\par 6 Ότε δε είδον αυτόν οι αρχιερείς και οι υπηρέται, εκραύγασαν λέγοντες· Σταύρωσον, σταύρωσον αυτόν. Λέγει προς αυτούς ο Πιλάτος· Λάβετε αυτόν σεις και σταυρώσατε· διότι εγώ δεν ευρίσκω εν αυτώ έγκλημα.
\par 7 Απεκρίθησαν προς αυτόν οι Ιουδαίοι· Ημείς νόμον έχομεν, και κατά τον νόμον ημών πρέπει να αποθάνη, διότι έκαμεν εαυτόν Υιόν του Θεού.
\par 8 Ότε δε ήκουσεν ο Πιλάτος τούτον τον λόγον, μάλλον εφοβήθη,
\par 9 και εισήλθε πάλιν εις το πραιτώριον, και λέγει προς τον Ιησούν· Πόθεν είσαι συ; Ο δε Ιησούς απόκρισιν δεν έδωκεν εις αυτόν.
\par 10 Λέγει λοιπόν προς αυτόν ο Πιλάτος· Προς εμέ δεν λαλείς; δεν εξεύρεις ότι εξουσίαν έχω να σε σταυρώσω και εξουσίαν έχω να σε απολύσω;
\par 11 Απεκρίθη ο Ιησούς· Δεν είχες ουδεμίαν εξουσίαν κατ' εμού, εάν δεν σοι ήτο δεδομένον άνωθεν· διά τούτο ο παραδίδων με εις σε έχει μεγαλητέραν αμαρτίαν.
\par 12 Έκτοτε εζήτει ο Πιλάτος να απολύση αυτόν· οι Ιουδαίοι όμως έκραζον, λέγοντες· Εάν τούτον απολύσης, δεν είσαι φίλος του Καίσαρος. Πας όστις κάμνει εαυτόν βασιλέα αντιλέγει εις τον Καίσαρα.
\par 13 Ο Πιλάτος λοιπόν, ακούσας τούτον τον λόγον, έφερεν έξω τον Ιησούν και εκάθησεν επί του βήματος εις τον τόπον λεγόμενον Λιθόστρωτον, Εβραϊστί δε Γαβαθθά.
\par 14 Ήτο δε παρασκευή του πάσχα και ώρα περίπου έκτη· και λέγει προς τους Ιουδαίους· Ιδού ο βασιλεύς σας.
\par 15 Οι δε εκραύγασαν· Άρον, άρον, σταύρωσον αυτόν. Λέγει προς αυτούς ο Πιλάτος· Τον βασιλέα σας να σταυρώσω; Απεκρίθησαν οι αρχιερείς· Δεν έχομεν βασιλέα ειμή Καίσαρα.
\par 16 Τότε λοιπόν παρέδωκεν αυτόν εις αυτούς διά να σταυρωθή. Και παρέλαβον τον Ιησούν και απήγαγον·
\par 17 και βαστάζων τον σταυρόν αυτού, εξήλθεν εις τον λεγόμενον Κρανίου τόπον, όστις λέγεται Εβραϊστί Γολγοθά,
\par 18 όπου εσταύρωσαν αυτόν και μετ' αυτού άλλους δύο εντεύθεν και εντεύθεν, μέσον δε τον Ιησούν.
\par 19 Έγραψε δε και τίτλον ο Πιλάτος και έθεσεν επί του σταυρού· ήτο δε γεγραμμένον· Ιησούς ο Ναζωραίος ο Βασιλεύς των Ιουδαίων.
\par 20 Και τούτον τον τίτλον ανέγνωσαν πολλοί των Ιουδαίων, διότι ήτο πλησίον της πόλεως ο τόπος, όπου εσταυρώθη ο Ιησούς· και ήτο γεγραμμένον Εβραϊστί, Ελληνιστί, Ρωμαϊστί.
\par 21 Έλεγον λοιπόν προς τον Πιλάτον οι αρχιερείς των Ιουδαίων· Μη γράφε, Ο βασιλεύς των Ιουδαίων· αλλ' ότι εκείνος είπε, Βασιλεύς είμαι των Ιουδαίων.
\par 22 Απεκρίθη ο Πιλάτος· Ο γέγραφα, γέγραφα.
\par 23 Οι στρατιώται λοιπόν, αφού εσταύρωσαν τον Ιησούν, έλαβον τα ιμάτια αυτού και έκαμον τέσσαρα μερίδια, εις έκαστον στρατιώτην εν μερίδιον, και τον χιτώνα· ήτο δε ο χιτών άρραφος, από άνωθεν όλος υφαντός.
\par 24 Είπον λοιπόν προς αλλήλους· Ας μη σχίσωμεν αυτόν, αλλ' ας ρίψωμεν λαχνόν περί αυτού τίνος θέλει είσθαι· διά να πληρωθή η γραφή η λέγουσα· Διεμερίσθησαν τα ιμάτιά μου εις εαυτούς, και επί τον ιματισμόν μου έβαλον κλήρον· οι μεν λοιπόν στρατιώται ταύτα έκαμον.
\par 25 Ίσταντο δε πλησίον εις τον σταυρόν του Ιησού η μήτηρ αυτού και η αδελφή της μητρός αυτού, Μαρία η γυνή του Κλωπά και Μαρία η Μαγδαληνή.
\par 26 Ο Ιησούς λοιπόν, ως είδε την μητέρα και τον μαθητήν παριστάμενον, τον οποίον ηγάπα, λέγει προς την μητέρα αυτού· Γύναι, ιδού ο υιός σου.
\par 27 Έπειτα λέγει προς τον μαθητήν· Ιδού η μήτηρ σου. Και απ' εκείνης της ώρας έλαβεν αυτήν ο μαθητής εις την οικίαν αυτού.
\par 28 Μετά τούτο γινώσκων ο Ιησούς ότι πάντα ήδη ετελέσθησαν διά να πληρωθή η γραφή, λέγει· Διψώ.
\par 29 Έκειτο δε εκεί αγγείον πλήρες όξους· και εκείνοι γεμίσαντες σπόγγον από όξους και περιθέσαντες εις ύσσωπον προσέφεραν εις το στόμα αυτού.
\par 30 Ότε λοιπόν έλαβε το όξος ο Ιησούς, είπε, Τετέλεσται· και κλίνας την κεφαλήν παρέδωκε το πνεύμα.
\par 31 Οι δε Ιουδαίοι, διά να μη μείνωσιν επί του σταυρού τα σώματα εν τω σαββάτω, επειδή ήτο παρασκευή· διότι ήτο μεγάλη εκείνη η ημέρα του σαββάτου· παρεκάλεσαν τον Πιλάτον διά να συνθλασθώσιν αυτών τα σκέλη, και να σηκωθώσιν.
\par 32 Ήλθον λοιπόν οι στρατιώται, και του μεν πρώτου συνέθλασαν τα σκέλη και του άλλου του συσταυρωθέντος μετ' αυτού·
\par 33 εις δε τον Ιησούν ελθόντες, ως είδον αυτόν ήδη τεθνηκότα, δεν συνέθλασαν αυτού τα σκέλη,
\par 34 αλλ' εις των στρατιωτών εκέντησε με λόγχην την πλευράν αυτού, και ευθύς εξήλθεν αίμα και ύδωρ.
\par 35 Και ο ιδών μαρτυρεί, και αληθινή είναι η μαρτυρία αυτού, και εκείνος εξεύρει ότι αλήθειαν λέγει, διά να πιστεύσητε σεις.
\par 36 Διότι έγειναν ταύτα, διά να πληρωθή η γραφή, Οστούν αυτού δεν θέλει συντριφθή.
\par 37 Και πάλιν άλλη γραφή λέγει· Θέλουσιν επιβλέψει εις εκείνον, τον οποίον εξεκέντησαν.
\par 38 Μετά δε ταύτα Ιωσήφ ο από Αριμαθαίας, όστις ήτο μαθητής του Ιησού, κεκρυμμένος όμως διά τον φόβον των Ιουδαίων, παρεκάλεσε τον Πιλάτον να σηκώση το σώμα του Ιησού· και ο Πιλάτος έδωκεν άδειαν. Ήλθε λοιπόν και εσήκωσε το σώμα του Ιησού.
\par 39 Ήλθε δε και ο Νικόδημος, όστις είχεν ελθεί προς τον Ιησούν διά νυκτός κατ' αρχάς, φέρων μίγμα σμύρνης και αλόης έως εκατόν λίτρας.
\par 40 Έλαβον λοιπόν το σώμα του Ιησού και έδεσαν αυτό με σάβανα μετά των αρωμάτων, καθώς είναι συνήθεια εις τους Ιουδαίους να ενταφιάζωσιν.
\par 41 Ήτο δε εν τω τόπω όπου εσταυρώθη κήπος, και εν τω κήπω μνημείον νέον, εις το οποίον ουδείς έτι είχε τεθή.
\par 42 Εκεί λοιπόν έθεσαν τον Ιησούν διά την παρασκευήν των Ιουδαίων, διότι ήτο πλησίον το μνημείον.

\chapter{20}

\par 1 Την δε πρώτην της εβδομάδος Μαρία η Μαγδαληνή έρχεται εις το μνημείον το πρωΐ, ενώ έτι ήτο σκότος, και βλέπει τον λίθον σηκωμένον εκ του μνημείου.
\par 2 Τρέχει λοιπόν και έρχεται προς τον Σίμωνα Πέτρον και προς τον άλλον μαθητήν, τον οποίον ηγάπα ο Ιησούς, και λέγει προς αυτούς· Εσήκωσαν τον Κύριον εκ του μνημείου, και δεν εξεύρομεν που έθεσαν αυτόν.
\par 3 Εξήλθε λοιπόν ο Πέτρος και ο άλλος μαθητής και ήρχοντο εις το μνημείον.
\par 4 Έτρεχον δε οι δύο ομού· και ο άλλος μαθητής προέτρεξε ταχύτερον του Πέτρου και ήλθε πρώτος εις το μνημείον,
\par 5 και παρακύψας βλέπει κείμενα τα σάβανα, δεν εισήλθεν όμως.
\par 6 Έρχεται λοιπόν ο Σίμων Πέτρος ακολουθών αυτόν, και εισήλθεν εις το μνημείον και θεωρεί τα σάβανα κείμενα,
\par 7 και το σουδάριον, το οποίον ήτο επί της κεφαλής αυτού, κείμενον ουχί ομού με τα σάβανα, αλλά χωριστά τετυλιγμένον εις ένα τόπον.
\par 8 Τότε λοιπόν εισήλθε και ο άλλος μαθητής ο ελθών πρώτος εις το μνημείον, και είδε και επίστευσε·
\par 9 διότι δεν ενόουν έτι την γραφήν ότι πρέπει αυτός να αναστηθή εκ νεκρών.
\par 10 Ανεχώρησαν λοιπόν πάλιν εις τα ίδια οι μαθηταί.
\par 11 Η δε Μαρία ίστατο πλησίον του μνημείου κλαίουσα έξω. Ενώ λοιπόν έκλαιεν, έκυψεν εις το μνημείον·
\par 12 και βλέπει δύο αγγέλους με λευκά ιμάτια καθημένους, ένα προς την κεφαλήν και ένα προς τους πόδας, εκεί όπου έκειτο το σώμα του Ιησού.
\par 13 Και λέγουσι προς αυτήν εκείνοι· Γύναι, τι κλαίεις; Λέγει προς αυτούς· Διότι εσήκωσαν τον Κύριόν μου, και δεν εξεύρω που έθεσαν αυτόν.
\par 14 Και αφού είπε ταύτα, εστράφη εις τα οπίσω και θεωρεί τον Ιησούν ιστάμενον, και δεν ήξευρεν ότι είναι ο Ιησούς.
\par 15 Λέγει προς αυτήν ο Ιησούς· Γύναι, τι κλαίεις; τίνα ζητείς; Εκείνη νομίζουσα ότι είναι ο κηπουρός, λέγει προς αυτόν· Κύριε, εάν συ εσήκωσας αυτόν, ειπέ μοι που έθεσας αυτόν, και εγώ θέλω σηκώσει αυτόν.
\par 16 Λέγει προς αυτήν ο Ιησούς· Μαρία. Εκείνη στραφείσα λέγει προς αυτόν· Ραββουνί, το οποίον λέγεται, Διδάσκαλε.
\par 17 Λέγει προς αυτήν ο Ιησούς· Μη μου άπτου· διότι δεν ανέβην έτι προς τον Πατέρα μου. Αλλ' ύπαγε προς τους αδελφούς μου και ειπέ προς αυτούς· Αναβαίνω προς τον Πατέρα μου και Πατέρα σας και Θεόν μου και Θεόν σας.
\par 18 Έρχεται Μαρία η Μαγδαληνή και απαγγέλλει προς τους μαθητάς ότι είδε τον Κύριον και ότι είπε ταύτα προς αυτήν.
\par 19 Το εσπέρας λοιπόν της ημέρας εκείνης της πρώτης της εβδομάδος, ενώ αι θύραι ήσαν κεκλεισμέναι, όπου οι μαθηταί ήσαν συνηγμένοι διά τον φόβον των Ιουδαίων, ήλθεν ο Ιησούς και εστάθη εις το μέσον, και λέγει προς αυτούς· Ειρήνη υμίν.
\par 20 Και τούτο ειπών έδειξεν εις αυτούς τας χείρας και την πλευράν αυτού. Εχάρησαν λοιπόν οι μαθηταί ιδόντες τον Κύριον.
\par 21 Είπε δε πάλιν προς αυτούς ο Ιησούς· Ειρήνη υμίν· καθώς με απέστειλεν ο Πατήρ, και εγώ πέμπω εσάς.
\par 22 Και τούτο ειπών, ενεφύσησε και λέγει προς αυτούς· Λάβετε Πνεύμα Άγιον.
\par 23 Αν τινών συγχωρήσητε τας αμαρτίας, είναι συγκεχωρημέναι εις αυτούς, αν τινών κρατήτε, είναι κεκρατημέναι.
\par 24 Θωμάς δε, εις εκ των δώδεκα, ο λεγόμενος Δίδυμος, δεν ήτο μετ' αυτών ότε ήλθεν ο Ιησούς.
\par 25 Έλεγον λοιπόν προς αυτόν οι άλλοι μαθηταί· Είδομεν τον Κύριον. Ο δε είπε προς αυτούς· Εάν δεν ίδω εν ταις χερσίν αυτού τον τύπον των ήλων και βάλω τον δάκτυλόν μου εις τον τύπον των ήλων, και βάλω την χείρα μου εις την πλευράν αυτού, δεν θέλω πιστεύσει.
\par 26 Και μεθ' ημέρας οκτώ πάλιν ήσαν έσω οι μαθηταί αυτού και Θωμάς μετ' αυτών. Έρχεται ο Ιησούς, ενώ αι θύραι ήσαν κεκλεισμέναι, και εστάθη εις το μέσον και είπεν· Ειρήνη υμίν.
\par 27 Έπειτα λέγει προς τον Θωμάν· Φέρε τον δάκτυλόν σου εδώ και ίδε τας χείρας μου, και φέρε την χείρα σου και βάλε εις την πλευράν μου, και μη γίνου άπιστος αλλά πιστός.
\par 28 Και απεκρίθη ο Θωμάς και είπε προς αυτόν· Ο Κύριός μου και ο Θεός μου.
\par 29 Λέγει προς αυτόν ο Ιησούς· Επειδή με είδες, Θωμά, επίστευσας· μακάριοι όσοι δεν είδον και επίστευσαν.
\par 30 Και άλλα πολλά θαύματα έκαμεν ο Ιησούς ενώπιον των μαθητών αυτού, τα οποία δεν είναι γεγραμμένα εν τω βιβλίω τούτω·
\par 31 ταύτα δε εγράφησαν διά να πιστεύσητε ότι ο Ιησούς είναι ο Χριστός ο Υιός του Θεού, και πιστεύοντες να έχητε ζωήν εν τω ονόματι αυτού.

\chapter{21}

\par 1 Μετά ταύτα εφανέρωσεν εαυτόν πάλιν ο Ιησούς εις τους μαθητάς επί της θαλάσσης της Τιβεριάδος· εφανέρωσε δε ούτως.
\par 2 Ήσαν ομού Σίμων Πέτρος και Θωμάς ο λεγόμενος Δίδυμος και Ναθαναήλ ο από Κανά της Γαλιλαίας, και οι υιοί του Ζεβεδαίου και άλλοι δύο εκ των μαθητών αυτού.
\par 3 Λέγει προς αυτούς Σίμων Πέτρος· Υπάγω να αλιεύσω. Λέγουσι προς αυτόν· Ερχόμεθα και ημείς μετά σου. Εξήλθον και ανέβησαν εις το πλοίον ευθύς, και κατ' εκείνην την νύκτα δεν επίασαν ουδέν.
\par 4 Αφού δε έγεινεν ήδη πρωΐ, εστάθη ο Ιησούς εις τον αιγιαλόν· δεν εγνώριζον όμως οι μαθηταί ότι είναι ο Ιησούς.
\par 5 Λέγει λοιπόν προς αυτούς ο Ιησούς· Παιδία, μήπως έχετέ τι προσφάγιον; Απεκρίθησαν προς αυτόν· Ουχί.
\par 6 Ο δε είπε προς αυτούς· Ρίψατε το δίκτυον εις τα δεξιά μέρη του πλοίου και θέλετε ευρεί. Έρριψαν λοιπόν και δεν ηδυνήθησαν πλέον να σύρωσιν αυτό από του πλήθους των ιχθύων.
\par 7 Λέγει λοιπόν προς τον Πέτρον ο μαθητής εκείνος, τον οποίον ηγάπα ο Ιησούς· Ο Κύριος είναι. Ο δε Σίμων Πέτρος, ακούσας ότι είναι ο Κύριος, εζώσθη το επένδυμα· διότι ήτο γυμνός· και έρριψεν εαυτόν εις την θάλασσαν.
\par 8 Οι δε άλλοι μαθηταί ήλθον με το πλοιάριον· διότι δεν ήσαν μακράν από της γης, αλλ' έως διακοσίας πήχας· σύροντες το δίκτυον των ιχθύων.
\par 9 Καθώς λοιπόν απέβησαν εις την γην, βλέπουσιν ανθρακιάν κειμένην και οψάριον επικείμενον και άρτον.
\par 10 Λέγει προς αυτούς ο Ιησούς· Φέρετε από των οψαρίων, τα οποία επιάσατε τώρα.
\par 11 Ανέβη Σίμων Πέτρος και έσυρε το δίκτυον επί της γης, γέμον ιχθύων μεγάλων εκατόν πεντήκοντα τριών· και ενώ ήσαν τόσοι, δεν εσχίσθη το δίκτυον.
\par 12 Λέγει προς αυτούς ο Ιησούς· Έλθετε, γευματίσατε. Ουδείς όμως των μαθητών ετόλμα να εξετάση αυτόν, Συ τις είσαι, εξεύροντες ότι είναι ο Κύριος.
\par 13 Έρχεται λοιπόν ο Ιησούς και λαμβάνει τον άρτον και δίδει εις αυτούς, και το οψάριον ομοίως.
\par 14 Αύτη ήτο ήδη τρίτη φορά, καθ' ην ο Ιησούς εφανερώθη εις τους μαθητάς αυτού, αφού ηγέρθη εκ νεκρών.
\par 15 Αφού λοιπόν εγευμάτισαν, λέγει προς τον Σίμωνα Πέτρον ο Ιησούς· Σίμων Ιωνά, αγαπάς με περισσότερον τούτων; Λέγει προς αυτόν· Ναι, Κύριε, συ εξεύρεις ότι σε αγαπώ. Λέγει προς αυτόν· Βόσκε τα αρνία μου.
\par 16 Λέγει προς αυτόν πάλιν δευτέραν φοράν· Σίμων Ιωνά, αγαπάς με; Λέγει προς αυτόν· Ναι, Κύριε, συ εξεύρεις ότι σε αγαπώ. Λέγει προς αυτόν· Ποίμαινε τα πρόβατά μου.
\par 17 Λέγει προς αυτόν την τρίτην φοράν· Σίμων Ιωνά, αγαπάς με; Ελυπήθη ο Πέτρος ότι είπε προς αυτόν την τρίτην φοράν· Αγαπάς με; και είπε προς αυτόν· Κύριε, συ εξεύρεις τα πάντα, συ γνωρίζεις ότι σε αγαπώ. Λέγει προς αυτόν ο Ιησούς· Βόσκε τα πρόβατά μου.
\par 18 Αληθώς, αληθώς σοι λέγω, ότε ήσο νεώτερος, εζώννυες σεαυτόν και περιεπάτεις όπου ήθελες· αφού όμως γηράσης, θέλεις εκτείνει τας χείρας σου, και άλλος θέλει σε ζώσει, και θέλει σε φέρει όπου δεν θέλεις.
\par 19 Είπε δε τούτο δεικνύων με ποίον θάνατον μέλλει να δοξάση τον Θεόν. Και τούτο ειπών λέγει προς αυτόν· Ακολούθει μοι.
\par 20 Στραφείς δε ο Πέτρος, βλέπει ακολουθούντα τον μαθητήν, τον οποίον ηγάπα ο Ιησούς, όστις και ανέπεσεν εν τω δείπνω επί το στήθος αυτού και είπε· Κύριε, τις είναι ο παραδίδων σε;
\par 21 Τούτον ιδών ο Πέτρος λέγει προς τον Ιησούν· Κύριε, ούτος δε τι;
\par 22 Λέγει προς αυτόν ο Ιησούς· Εάν αυτόν θέλω να μένη εωσού έλθω, τι προς σε; συ ακολούθει μοι.
\par 23 Διεδόθη λοιπόν ο λόγος ούτος εις τους αδελφούς ότι ο μαθητής εκείνος δεν αποθνήσκει· ο Ιησούς όμως δεν είπε προς αυτόν ότι δεν αποθνήσκει, αλλ' εάν θέλω αυτόν να μένη εωσού έλθω, τι προς σε;
\par 24 Ούτος είναι ο μαθητής ο μαρτυρών περί τούτων και γράψας ταύτα, και εξεύρομεν ότι είναι αληθής η μαρτυρία αυτού.
\par 25 Είναι δε και άλλα πολλά όσα έκαμεν ο Ιησούς, τα οποία εάν γραφθώσι καθ' εν, ουδ' αυτός ο κόσμος νομίζω θέλει χωρήσει τα γραφόμενα βιβλία. Αμήν.


\end{document}