\begin{document}

\title{Πράξεις Αποστόλων}


\chapter{1}

\par 1 Τον μεν πρώτον λόγον έκαμον, ω Θεόφιλε, περί πάντων όσα ήρχισεν ο Ιησούς να κάμνη και να διδάσκη,
\par 2 μέχρι της ημέρας καθ' ην ανελήφθη, αφού διά Πνεύματος Αγίου έδωκεν εντολάς εις τους αποστόλους, τους οποίους εξέλεξεν·
\par 3 εις τους οποίους και εφανέρωσεν εαυτόν ζώντα μετά το πάθος αυτού διά πολλών τεκμηρίων, εμφανιζόμενος εις αυτούς τεσσαράκοντα ημέρας και λέγων τα περί της βασιλείας του Θεού.
\par 4 Και συνερχόμενος μετ' αυτών, παρήγγειλε να μη απομακρυνθώσιν από Ιεροσολύμων, αλλά να περιμένωσι την επαγγελίαν του Πατρός, την οποίαν ηκούσατε, είπε, παρ' εμού.
\par 5 Διότι ο μεν Ιωάννης εβάπτισεν εν ύδατι, σεις όμως θέλετε βαπτισθή εν Πνεύματι Αγίω ουχί μετά πολλάς ταύτας ημέρας.
\par 6 Εκείνοι λοιπόν συνελθόντες ηρώτων αυτόν, λέγοντες· Κύριε, τάχα εν τω καιρώ τούτω αποκαθιστάνεις την βασιλείαν εις τον Ισραήλ;
\par 7 Είπε δε προς αυτούς· Δεν ανήκει εις εσάς να γνωρίζητε τους χρόνους ή τους καιρούς, τους οποίους ο Πατήρ έθεσεν εν τη ιδία αυτού εξουσία,
\par 8 αλλά θέλετε λάβει δύναμιν, όταν επέλθη το Άγιον Πνεύμα εφ' υμάς, και θέλετε είσθαι εις εμέ μάρτυρες και εν Ιερουσαλήμ και εν πάση τη Ιουδαία και Σαμαρεία και έως εσχάτου της γης.
\par 9 Και αφού είπε ταύτα, βλεπόντων αυτών ανελήφθη, και νεφέλη υπέλαβεν αυτόν από των οφθαλμών αυτών.
\par 10 Και ενώ ήσαν ατενίζοντες εις τον ουρανόν ότε αυτός ανέβαινεν, ιδού, άνδρες δύο με ιμάτια λευκά εστάθησαν πλησίον αυτών,
\par 11 οίτινες και είπον· Άνδρες Γαλιλαίοι, τι ίστασθε εμβλέποντες εις τον ουρανόν; ούτος ο Ιησούς, όστις ανελήφθη αφ' υμών εις τον ουρανόν, θέλει ελθεί ούτω καθ' ον τρόπον είδετε αυτόν πορευόμενον εις τον ουρανόν.
\par 12 Τότε υπέστρεψαν εις Ιερουσαλήμ από του όρους του καλουμένου Ελαιώνος, το οποίον είναι πλησίον της Ιερουσαλήμ, απέχον οδόν σαββάτου.
\par 13 Και ότε εισήλθον, ανέβησαν εις το ανώγεον, όπου είχον το κατάλυμα, ο Πέτρος και Ιάκωβος και Ιωάννης και Ανδρέας, Φίλιππος και Θωμάς, Βαρθολομαίος και Ματθαίος, Ιάκωβος Αλφαίου και Σίμων ο Ζηλωτής και Ιούδας Ιακώβου.
\par 14 Ούτοι πάντες ενέμενον ομοθυμαδόν εις την προσευχήν και την δέησιν μετά των γυναικών και Μαρίας της μητρός του Ιησού και μετά των αδελφών αυτού.
\par 15 Και εν ταις ημέραις ταύταις σηκωθείς ο Πέτρος εις το μέσον των μαθητών, είπεν· ήτο δε ο αριθμός των εκεί παρόντων ως εκατόν είκοσιν·
\par 16 Άνδρες αδελφοί, έπρεπε να πληρωθή η γραφή αύτη, την οποίαν προείπε το Πνεύμα το Άγιον διά στόματος του Δαβίδ περί του Ιούδα, όστις έγεινεν οδηγός εις τους συλλαβόντας τον Ιησούν,
\par 17 διότι ήτο συνηριθμημένος με ημάς και έλαβε την μερίδα της διακονίας ταύτης.
\par 18 Ούτος λοιπόν απέκτησεν αγρόν εκ του μισθού της αδικίας, και πεσών πρόμυττα εσχίσθη εις το μέσον, και εξεχύθησαν όλα τα εντόσθια αυτού·
\par 19 και έγεινε γνωστόν εις πάντας τους κατοικούντας την Ιερουσαλήμ, ώστε ο αγρός εκείνος ωνομάσθη εν τη ιδία αυτών διαλέκτω Ακελδαμά, τουτέστιν, αγρός αίματος.
\par 20 Διότι είναι γεγραμμένον εν τω βιβλίω των Ψαλμών· Ας γείνη η κατοικία αυτού έρημος και ας μη ήναι ο κατοικών εν αυτή· και, Άλλος ας λάβη την επισκοπήν αυτού.
\par 21 Πρέπει λοιπόν εκ των ανδρών, οίτινες συνήλθον μεθ' ημών καθ' όλον τον καιρόν, καθ' ον εισήλθε και εξήλθε προς ημάς ο Κύριος Ιησούς,
\par 22 αρχίσας από του βαπτίσματος του Ιωάννου έως της ημέρας καθ' ην ανελήφθη αφ' ημών, εις εκ τούτων να γείνη μεθ' ημών μάρτυς της αναστάσεως αυτού.
\par 23 Και έστησαν δύο, Ιωσήφ τον καλούμενον Βαρσαβάν, όστις επωνομάσθη Ιούστος, και Ματθίαν.
\par 24 Και προσευχηθέντες είπον· Συ, Κύριε, καρδιογνώστα πάντων, ανάδειξον εκ των δύο τούτων ένα, όντινα εξέλεξας,
\par 25 διά να λάβη την μερίδα της διακονίας ταύτης και αποστολής, εκ της οποίας εξέπεσεν ο Ιούδας διά να απέλθη εις τον τόπον αυτού.
\par 26 Και έδωκαν τους κλήρους αυτών, και έπεσεν ο κλήρος εις τον Ματθίαν, και συγκατεψηφίσθη μετά των ένδεκα αποστόλων.

\chapter{2}

\par 1 Και ότε ήλθεν η ημέρα της Πεντηκοστής, ήσαν άπαντες ομοθυμαδόν εν τω αυτώ τόπω.
\par 2 Και εξαίφνης έγεινεν ήχος εκ του ουρανού ως ανέμου βιαίως φερομένου, και εγέμισεν όλον τον οίκον όπου ήσαν καθήμενοι·
\par 3 και εφάνησαν εις αυτούς διαμεριζόμεναι γλώσσαι ως πυρός, και εκάθησεν επί ένα έκαστον αυτών,
\par 4 και επλήσθησαν άπαντες Πνεύματος Αγίου, και ήρχισαν να λαλώσι ξένας γλώσσας, καθώς το Πνεύμα έδιδεν εις αυτούς να λαλώσιν.
\par 5 Ήσαν δε κατοικούντες εν Ιερουσαλήμ Ιουδαίοι, άνδρες ευλαβείς από παντός έθνους των υπό τον ουρανόν·
\par 6 και καθώς έγεινεν η φωνή αύτη, συνήλθε το πλήθος και συνεταράχθη, διότι ήκουον αυτούς εις έκαστος λαλούντας με την ιδίαν αυτού διάλεκτον.
\par 7 Εξεπλήττοντο δε πάντες και εθαύμαζον, λέγοντες προς αλλήλους· Ιδού, πάντες ούτοι οι λαλούντες δεν είναι Γαλιλαίοι;
\par 8 Και πως ημείς ακούομεν έκαστος εν τη ιδία ημών διαλέκτω, εν ή εγεννήθημεν;
\par 9 Πάρθοι και Μήδοι και Ελαμίται και οι κατοικούντες την Μεσοποταμίαν, την Ιουδαίαν τε και Καππαδοκίαν, τον Πόντον και την Ασίαν,
\par 10 την Φρυγίαν τε και την Παμφυλίαν, την Αίγυπτον και τα μέρη της Λιβύης της κατά την Κυρήνην και οι παρεπιδημούντες Ρωμαίοι, Ιουδαίοί τε και προσήλυτοι,
\par 11 Κρήτες και Άραβες, ακούομεν αυτούς λαλούντας εν ταις γλώσσαις ημών τα μεγαλεία του Θεού.
\par 12 Εθαύμαζον δε πάντες και ηπόρουν, άλλος προς άλλον λέγοντες· Τι σημαίνει τούτο;
\par 13 Άλλοι δε χλευάζοντες έλεγον ότι είναι μεστοί από γλυκύν οίνον.
\par 14 Σταθείς δε ο Πέτρος μετά των ένδεκα, ύψωσε την φωνήν αυτού και ελάλησε προς αυτούς· Άνδρες Ιουδαίοι και πάντες οι κατοικούντες την Ιερουσαλήμ, τούτο ας ήναι γνωστόν εις εσάς και ακούσατε τους λόγους μου.
\par 15 Διότι ούτοι δεν είναι μεθυσμένοι, καθώς σεις νομίζετε· διότι είναι τρίτη ώρα της ημέρας·
\par 16 αλλά τούτο είναι το ρηθέν διά του προφήτου Ιωήλ·
\par 17 Και εν ταις εσχάταις ημέραις, λέγει ο Θεός, Θέλω εκχέει από του Πνεύματός μου επί πάσαν σάρκα, και θέλουσι προφητεύσει οι υιοί σας και αι θυγατέρες σας, και οι νεανίσκοι σας θέλουσιν ιδεί οράσεις, και οι πρεσβύτεροί σας θέλουσιν ενυπνιασθή ενύπνια·
\par 18 και έτι επί τους δούλους μου και επί τας δούλας μου εν ταις ημέραις εκείναις θέλω εκχέει από του Πνεύματός μου, και θέλουσι προφητεύσει·
\par 19 και θέλω δείξει τέρατα εν τω ουρανώ άνω και σημεία επί της γης κάτω, αίμα και πυρ και ατμίδα καπνού·
\par 20 ο ήλιος θέλει μεταστραφή εις σκότος και η σελήνη εις αίμα, πριν έλθη η ημέρα του Κυρίου η μεγάλη και επιφανής.
\par 21 Και πας όστις αν επικαλεσθή το όνομα του Κυρίου, θέλει σωθή.
\par 22 Άνδρες Ισραηλίται, ακούσατε τους λόγους τούτους· τον Ιησούν τον Ναζωραίον, άνδρα αποδεδειγμένον προς εσάς από του Θεού διά θαυμάτων και τεραστίων και σημείων, τα οποία ο Θεός έκαμε δι' αυτού εν μέσω υμών, καθώς και σεις εξεύρετε,
\par 23 τούτον λαβόντες παραδεδομένον κατά την ωρισμένην βουλήν και πρόγνωσιν του Θεού, διά χειρών ανόμων σταυρώσαντες εθανατώσατε·
\par 24 τον οποίον ο Θεός ανέστησε, λύσας τας ωδίνας του θανάτου, διότι δεν ήτο δυνατόν να κρατήται υπ' αυτού.
\par 25 Επειδή ο Δαβίδ λέγει περί αυτού· Έβλεπον τον Κύριον ενώπιόν μου διαπαντός, διότι είναι εκ δεξιών μου διά να μη σαλευθώ.
\par 26 Διά τούτο ευφράνθη η καρδία μου και ηγαλλίασεν η γλώσσα μου· έτι δε και η σαρξ μου θέλει αναπαυθή επ' ελπίδι.
\par 27 Διότι δεν θέλεις εγκαταλείψει την ψυχήν μου εν τω άδη ουδέ θέλεις αφήσει τον όσιόν σου να ίδη διαφθοράν.
\par 28 Εφανέρωσας εις εμέ οδούς ζωής, θέλεις με χορτάσει από ευφροσύνης διά του προσώπου σου.
\par 29 Άνδρες αδελφοί, δύναμαι να σας είπω μετά παρρησίας περί του πατριάρχου Δαβίδ ότι και ετελεύτησε και ετάφη, και το μνήμα αυτού είναι παρ' ημίν μέχρι της ημέρας ταύτης.
\par 30 Επειδή λοιπόν ήτο προφήτης και ήξευρεν ότι μεθ' όρκου ώμοσε προς αυτόν ο Θεός, ότι εκ του καρπού της οσφύος αυτού θέλει αναστήσει κατά σάρκα τον Χριστόν διά να καθίση αυτόν επί του θρόνου αυτού,
\par 31 προϊδών ελάλησε περί της αναστάσεως του Χριστού ότι δεν εγκατελείφθη η ψυχή αυτού εν τω άδη ουδέ η σαρξ αυτού είδε διαφθοράν.
\par 32 Τούτον τον Ιησούν ανέστησεν ο Θεός, του οποίου πάντες ημείς είμεθα μάρτυρες.
\par 33 Αφού λοιπόν υψώθη διά της δεξιάς του Θεού και έλαβε παρά του Πατρός την επαγγελίαν του Αγίου Πνεύματος, εξέχεε τούτο, το οποίον τώρα σεις βλέπετε και ακούετε.
\par 34 Διότι ο Δαβίδ δεν ανέβη εις τους ουρανούς, λέγει όμως αυτός, Είπεν ο Κύριος προς τον Κύριόν μου, κάθου εκ δεξιών μου,
\par 35 εωσού θέσω τους εχθρούς σου υποπόδιον των ποδών σου.
\par 36 Βεβαίως λοιπόν ας εξεύρη πας ο οίκος του Ισραήλ ότι ο Θεός Κύριον και Χριστόν έκαμεν αυτόν τούτον τον Ιησούν, τον οποίον σεις εσταυρώσατε.
\par 37 Αφού δε ήκουσαν ταύτα, ήλθεν εις κατάνυξιν η καρδία αυτών, και είπον προς τον Πέτρον και τους λοιπούς αποστόλους· Τι πρέπει να κάμωμεν, άνδρες αδελφοί;
\par 38 Και ο Πέτρος είπε προς αυτούς· Μετανοήσατε, και ας βαπτισθή έκαστος υμών εις το όνομα του Ιησού Χριστού εις άφεσιν αμαρτιών, και θέλετε λάβει την δωρεάν του Αγίου Πνεύματος.
\par 39 Διότι προς εσάς είναι η επαγγελία και προς τα τέκνα σας και προς πάντας τους εις μακράν, όσους αν προσκαλέση Κύριος ο Θεός ημών.
\par 40 Και με άλλους πολλούς λόγους διεμαρτύρετο και προέτρεπε, λέγων, Σώθητε από της διεστραμμένης ταύτης γενεάς.
\par 41 Εκείνοι λοιπόν μετά χαράς δεχθέντες τον λόγον αυτού εβαπτίσθησαν, και προσετέθησαν εν εκείνη τη ημέρα έως τρεις χιλιάδες ψυχαί.
\par 42 Και ενέμενον εν τη διδαχή των αποστόλων και εν τη κοινωνία και εν τη κλάσει του άρτου και εν ταις προσευχαίς.
\par 43 Κατέλαβε δε πάσαν ψυχήν φόβος, και πολλά τεράστια και σημεία εγίνοντο διά των αποστόλων.
\par 44 Και πάντες οι πιστεύοντες ήσαν ομού και είχον τα πάντα κοινά,
\par 45 και τα κτήματα και τα υπάρχοντα αυτών επώλουν και διεμοίραζον αυτά εις πάντας, καθ' ην έκαστος είχε χρείαν.
\par 46 Και καθ' ημέραν εμμένοντες ομοθυμαδόν εν τω ιερώ και κόπτοντες τον άρτον κατ' οίκους, μετελάμβανον την τροφήν εν αγαλλιάσει και απλότητι καρδίας,
\par 47 δοξολογούντες τον Θεόν και ευρίσκοντες χάριν ενώπιον όλου του λαού. Ο δε Κύριος προσέθετε καθ' ημέραν εις την εκκλησίαν τους σωζομένους.

\chapter{3}

\par 1 Ανέβαινον δε ομού ο Πέτρος και Ιωάννης εις το ιερόν κατά την ώραν της προσευχής την εννάτην.
\par 2 Και άνθρωπός τις χωλός υπάρχων εκ κοιλίας μητρός αυτού εβαστάζετο, τον οποίον έθετον καθ' ημέραν προς την θύραν του ιερού την λεγομένην Ωραίαν, διά να ζητή ελεημοσύνην παρά των εισερχομένων εις το ιερόν·
\par 3 ούτος ιδών τον Πέτρον και Ιωάννην μέλλοντας να εισέλθωσιν εις το ιερόν, εζήτει να λάβη ελεημοσύνην.
\par 4 Ατενίσας δε εις αυτόν ο Πέτρος μετά του Ιωάννου, είπε· Βλέψον εις ημάς.
\par 5 Και εκείνος έβλεπεν αυτούς μετά προσοχής, προσμένων να λάβη τι παρ' αυτών.
\par 6 Ο δε Πέτρος είπεν· Αργύριον και χρυσίον εγώ δεν έχω· αλλ' ό,τι έχω, τούτο σοι δίδω· εν τω ονόματι του Ιησού Χριστού του Ναζωραίου σηκώθητι και περιπάτει.
\par 7 Και πιάσας αυτόν από της δεξιάς χειρός εσήκωσε· και παρευθύς εστερεώθησαν αι βάσεις και τα σφυρά των ποδών αυτού,
\par 8 και αναπηδήσας εστάθη όρθιος και περιεπάτει, και εισήλθε μετ' αυτών εις το ιερόν περιπατών και πηδών και δοξάζων τον Θεόν.
\par 9 Και είδεν αυτόν πας ο λαός περιπατούντα και δοξάζοντα τον Θεόν·
\par 10 και εγνώριζον αυτόν ότι ούτος ήτο ο καθήμενος διά ελεημοσύνην εις την Ωραίαν πύλην του ιερού, και επλήσθησαν από θάμβους και εκστάσεως διά το γεγονός εις αυτόν.
\par 11 Και ενώ ο ιατρευθείς χωλός εκράτει τον Πέτρον και Ιωάννην, συνέδραμε προς αυτούς πας ο λαός εις την στοάν την λεγομένην Σολομώντος έκθαμβοι.
\par 12 Ιδών δε ο Πέτρος, απεκρίθη προς τον λαόν· Άνδρες Ισραηλίται, τι θαυμάζετε διά τούτο, ή τι ατενίζετε εις ημάς, ως εάν εκάμομεν από ιδίας ημών δυνάμεως ή ευσεβείας να περιπατή αυτός;
\par 13 Ο Θεός του Αβραάμ και Ισαάκ και Ιακώβ, ο Θεός των πατέρων ημών, εδόξασε τον Υιόν αυτού Ιησούν, τον οποίον σεις παρεδώκατε και ηρνήθητε αυτόν ενώπιον του Πιλάτου, ενώ εκείνος έκρινε να απολύση αυτόν.
\par 14 Σεις όμως τον άγιον και δίκαιον ηρνήθητε, και εζητήσατε άνδρα φονέα να χαρισθή εις εσάς,
\par 15 τον δε αρχηγόν της ζωής εθανατώσατε, τον οποίον ο Θεός ανέστησεν εκ νεκρών, του οποίου ημείς είμεθα μάρτυρες.
\par 16 Και διά της εις το όνομα αυτού πίστεως τούτον, τον οποίον θεωρείτε και γνωρίζετε, το όνομα αυτού εστερέωσε, και η πίστις η δι' αυτού έδωκεν εις αυτόν την τελείαν ταύτην υγείαν ενώπιον πάντων υμών.
\par 17 Και τώρα, αδελφοί, εξεύρω ότι επράξατε κατά άγνοιαν, καθώς και οι άρχοντές σας·
\par 18 ο δε Θεός όσα προείπε διά στόματος πάντων των προφητών αυτού ότι ο Χριστός έμελλε να πάθη, εξεπλήρωσεν ούτω.
\par 19 Μετανοήσατε λοιπόν και επιστρέψατε, διά να εξαλειφθώσιν αι αμαρτίαι σας, διά να έλθωσι καιροί αναψυχής από της παρουσίας του Κυρίου,
\par 20 και αποστείλη τον προκεκηρυγμένον εις εσάς Ιησούν Χριστόν,
\par 21 τον οποίον πρέπει να δεχθή ο ουρανός μέχρι των καιρών της αποκαταστάσεως πάντων, όσα ελάλησεν ο Θεός απ' αιώνος διά στόματος πάντων των αγίων αυτού προφητών.
\par 22 Διότι ο Μωϋσής είπε προς τους πατέρας Ότι Κύριος ο Θεός σας θέλει αναστήσει εις εσάς προφήτην εκ των αδελφών σας ως εμέ· αυτού θέλετε ακούει κατά πάντα όσα αν λαλήση προς εσάς.
\par 23 Και πάσα ψυχή, ήτις δεν ακούση του προφήτου εκείνου, θέλει εξολοθρευθή εκ του λαού.
\par 24 Και πάντες δε οι προφήται από Σαμουήλ και των καθεξής, όσοι ελάλησαν, προανήγγειλαν και τας ημέρας ταύτας.
\par 25 Σεις είσθε υιοί των προφητών και της διαθήκης, την οποίαν έκαμεν ο Θεός προς τους πατέρας ημών, λέγων προς τον Αβραάμ· Και εν τω σπέρματί σου θέλουσιν ευλογηθή πάσαι αι φυλαί της γης.
\par 26 Προς εσάς πρώτον ο Θεός αναστήσας τον Υιόν αυτού Ιησούν απέστειλεν αυτόν διά να σας ευλογή όταν επιστρέφητε έκαστος από των πονηριών υμών.

\chapter{4}

\par 1 Ενώ δε αυτοί ελάλουν προς τον λαόν, ήλθον επ' αυτούς οι ιερείς και ο στρατηγός του ιερού και οι Σαδδουκαίοι,
\par 2 αγανακτούντες διότι εδίδασκον τον λαόν και εκήρυττον διά του Ιησού την εκ νεκρών ανάστασιν·
\par 3 και επέβαλον επ' αυτούς τας χείρας και έθεσαν υπό φύλαξιν έως της αύριον, διότι ήτο ήδη εσπέρα.
\par 4 Πολλοί δε των ακουσάντων τον λόγον επίστευσαν, και έγεινεν ο αριθμός των ανδρών ως πέντε χιλιάδες.
\par 5 Και τη επαύριον συνήχθησαν εις την Ιερουσαλήμ οι άρχοντες αυτών και οι πρεσβύτεροι και οι γραμματείς,
\par 6 και Άννας ο αρχιερεύς και Καϊάφας και Ιωάννης και Αλέξανδρος και όσοι ήσαν εκ γένους αρχιερατικού.
\par 7 Και στήσαντες αυτούς εις το μέσον, ηρώτων· Διά ποίας δυνάμεως ή διά ποίου ονόματος επράξατε τούτο σεις;
\par 8 Τότε ο Πέτρος, πλησθείς Πνεύματος Αγίου, είπε προς αυτούς· Άρχοντες του λαού και πρεσβύτεροι του Ισραήλ,
\par 9 εάν ημείς ανακρινώμεθα σήμερον διά ευεργεσίαν προς άνθρωπον ασθενούντα, διά ποίας δυνάμεως ούτος ιατρεύθη,
\par 10 ας ήναι γνωστόν εις πάντας υμάς και εις πάντα τον λαόν του Ισραήλ ότι διά του ονόματος του Ιησού Χριστού του Ναζωραίου, τον οποίον σεις εσταυρώσατε, τον οποίον ο Θεός ανέστησεν εκ νεκρών, διά τούτου παρίσταται ούτος ενώπιον υμών υγιής.
\par 11 Ούτος είναι ο λίθος ο εξουθενηθείς εφ' υμών των οικοδομούντων, όστις έγεινε κεφαλή γωνίας.
\par 12 Και δεν υπάρχει δι' ουδενός άλλου η σωτηρία· διότι ούτε όνομα άλλο είναι υπό τον ουρανόν δεδομένον μεταξύ των ανθρώπων, διά του οποίου πρέπει να σωθώμεν.
\par 13 Θεωρούντες δε την παρρησίαν του Πέτρου και Ιωάννου, και πληροφορηθέντες ότι είναι άνθρωποι αγράμματοι και ιδιώται, εθαύμαζον και ανεγνώριζον αυτούς ότι ήσαν μετά του Ιησού·
\par 14 βλέποντες δε τον άνθρωπον τον τεθεραπευμένον ιστάμενον μετ' αυτών, δεν είχον ουδέν να αντείπωσι.
\par 15 Προστάξαντες δε αυτούς να απέλθωσιν έξω του συνεδρίου, συνεβουλεύθησαν προς αλλήλους,
\par 16 λέγοντες· Τι θέλομεν κάμει εις τους ανθρώπους τούτους; επειδή ότι μεν έγεινε δι' αυτών γνωστόν θαύμα, είναι φανερόν εις πάντας τους κατοικούντας την Ιερουσαλήμ, και δεν δυνάμεθα να αρνηθώμεν τούτο·
\par 17 αλλά διά να μη διαδοθή περισσότερον εις τον λαόν, ας απειλήσωμεν αυτούς αυστηρώς να μη λαλώσι πλέον εν τω ονόματι τούτω προς μηδένα άνθρωπον.
\par 18 Και καλέσαντες αυτούς, παρήγγειλαν εις αυτούς να μη λαλώσι καθόλου μηδέ να διδάσκωσιν εν τω ονόματι του Ιησού.
\par 19 Ο δε Πέτρος και Ιωάννης αποκριθέντες προς αυτούς, είπον· Αν ήναι δίκαιον ενώπιον του Θεού να ακούωμεν εσάς μάλλον παρά τον Θεόν, κρίνατε.
\par 20 Διότι ημείς δεν δυνάμεθα να μη λαλώμεν όσα είδομεν και ηκούσαμεν.
\par 21 Οι δε, πάλιν απειλήσαντες αυτούς απέλυσαν, μη ευρίσκοντες το πως να τιμωρήσωσιν αυτούς, διά τον λαόν, διότι πάντες εδόξαζον τον Θεόν διά το γεγονός.
\par 22 Επειδή ο άνθρωπος, εις τον οποίον έγεινε το θαύμα τούτο της θεραπείας, ήτο περισσότερον των τεσσαράκοντα ετών.
\par 23 Και αφού απελύθησαν, ήλθον προς τους οικείους και απήγγειλαν όσα είπον προς αυτούς οι αρχιερείς και οι πρεσβύτεροι.
\par 24 Οι δε ακούσαντες, ομοθυμαδόν ύψωσαν την φωνήν προς τον Θεόν και είπον· Δέσποτα, συ είσαι ο Θεός, όστις έκαμες τον ουρανόν και την γην και την θάλασσαν και πάντα τα εν αυτοίς,
\par 25 όστις είπας διά στόματος Δαβίδ του δούλου σου· Διά τι εφρύαξαν τα έθνη και οι λαοί εμελέτησαν μάταια;
\par 26 παρεστάθησαν οι βασιλείς της γης και οι άρχοντες συνήχθησαν ομού κατά του Κυρίου και κατά του Χριστού αυτού.
\par 27 Διότι συνήχθησαν επ' αληθείας εναντίον του αγίου Παιδός σου Ιησού, τον οποίον έχρισας, και ο Ηρώδης και ο Πόντιος Πιλάτος μετά των εθνών και των λαών του Ισραήλ,
\par 28 διά να κάμωσιν όσα η χειρ σου και η βουλή σου προώρισε να γείνωσι·
\par 29 και τώρα, Κύριε, βλέψον εις τας απειλάς αυτών και δος εις τους δούλους σου να λαλώσι τον λόγον σου μετά πάσης παρρησίας,
\par 30 εκτείνων την χείρα σου εις θεραπείαν και γινομένων σημείων και τεραστίων διά του ονόματος του αγίου Παιδός σου Ιησού.
\par 31 Μετά δε την δέησιν αυτών εσείσθη ο τόπος όπου ήσαν συνηγμένοι, και επλήσθησαν άπαντες Πνεύματος Αγίου και ελάλουν τον λόγον του Θεού μετά παρρησίας.
\par 32 Του δε πλήθους των πιστευσάντων η καρδία και η ψυχή ήτο μία, και ουδέ εις έλεγεν ότι είναι εαυτού τι εκ των υπαρχόντων αυτού αλλ' είχον τα πάντα κοινά.
\par 33 Και μετά δυνάμεως μεγάλης απέδιδον οι απόστολοι την μαρτυρίαν της αναστάσεως του Κυρίου Ιησού, και χάρις μεγάλη ήτο επί πάντας αυτούς.
\par 34 Επειδή ουδέ ήτο τις μεταξύ αυτών ενδεής· διότι όσοι ήσαν κτήτορες αγρών ή οικιών, πωλούντες έφερον τας τιμάς των πωλουμένων
\par 35 και έθετον εις τους πόδας των αποστόλων· και διεμοιράζετο εις έκαστον κατά την χρείαν την οποίαν είχε.
\par 36 Και ο Ιωσής, ο επονομασθείς υπό των αποστόλων Βαρνάβας, το οποίον μεθερμηνευόμενον είναι υιός παρηγορίας Λευΐτης, Κύπριος το γένος,
\par 37 έχων αγρόν επώλησε και έφερε τα χρήματα και έθεσεν εις τους πόδας των αποστόλων.

\chapter{5}

\par 1 Άνθρωπος δε τις Ανανίας το όνομα μετά της γυναικός αυτού Σαπφείρης επώλησε κτήμα
\par 2 και εκράτησεν από της τιμής, εν γνώσει και της γυναικός αυτού, και φέρων μέρος τι έθεσεν εις τους πόδας των αποστόλων.
\par 3 Είπε δε ο Πέτρος· Ανανία, διά τι εγέμισεν ο Σατανάς την καρδίαν σου, ώστε να ψευσθής εις το Πνεύμα το Άγιον και να κρατήσης από της τιμής του αγρού;
\par 4 Ενώ έμενε, δεν ήτο σου; και αφού επωλήθη, δεν ήτο εν τη εξουσία σου; διά τι έβαλες εν τη καρδία σου το πράγμα τούτο; δεν εψεύσθης εις ανθρώπους, αλλ' εις τον Θεόν.
\par 5 Ενώ δε ήκουεν ο Ανανίας τους λόγους τούτους, έπεσε και εξεψύχησε, και επέπεσε φόβος μέγας επί πάντας τους ακούοντας ταύτα.
\par 6 Σηκωθέντες δε οι νεώτεροι, ετύλιξαν αυτόν και εκβαλόντες έθαψαν.
\par 7 Μετά δε περίπου τρεις ώρας εισήλθεν η γυνή αυτού, μη εξεύρουσα το γεγονός.
\par 8 Και απεκρίθη προς αυτήν ο Πέτρος· Ειπέ μοι, διά τόσον επωλήσατε τον αγρόν; Και εκείνη είπε· Ναι, διά τόσον.
\par 9 Και ο Πέτρος είπε προς αυτήν· Διά τι συνεφωνήσατε να πειράζητε το Πνεύμα του Κυρίου; ιδού, εις την θύραν οι πόδες των θαψάντων τον άνδρα σου και θέλουσιν εκβάλει και σε.
\par 10 Και έπεσε παρευθύς εις τους πόδας αυτού και εξεψύχησεν· εισελθόντες δε οι νεανίσκοι, εύρον αυτήν νεκράν και εκβαλόντες έθαψαν πλησίον του ανδρός αυτής.
\par 11 Και επέπεσε φόβος μέγας εφ' όλην την εκκλησίαν και επί πάντας τους ακούοντας ταύτα.
\par 12 Πολλά δε σημεία και τέρατα εγίνοντο εν τω λαώ διά των χειρών των αποστόλων· και ήσαν ομοθυμαδόν άπαντες εν τη στοά του Σολομώντος.
\par 13 Εκ δε των λοιπών ουδείς ετόλμα να προσκολληθή εις αυτούς, ο λαός όμως εμεγάλυνεν αυτούς·
\par 14 και προσετίθεντο μάλλον πιστεύοντες εις τον Κύριον, πλήθη ανδρών τε και γυναικών,
\par 15 ώστε έφερον έξω εις τας πλατείας τους ασθενείς και έθετον επί κλινών και κραββάτων, διά να επισκιάση καν η σκιά του Πέτρου ερχομένου τινά εξ αυτών.
\par 16 Συνήρχετο δε και το πλήθος των πέριξ πόλεων εις Ιερουσαλήμ φέροντες ασθενείς και ενοχλουμένους υπό πνευμάτων ακαθάρτων, οίτινες εθεραπεύοντο άπαντες.
\par 17 Και σηκωθείς ο αρχιερεύς και πάντες οι μετ' αυτού, οίτινες ήσαν αίρεσις των Σαδδουκαίων, επλήσθησαν ζήλου
\par 18 και επέβαλον τας χείρας αυτών επί τους αποστόλους, και έβαλον αυτούς εις δημοσίαν φυλακήν.
\par 19 Άγγελος όμως Κυρίου διά της νυκτός ήνοιξε τας θύρας της φυλακής, και εκβαλών αυτούς είπεν·
\par 20 Υπάγετε, και σταθέντες λαλείτε εν τω ιερώ προς τον λαόν πάντας τους λόγους της ζωής ταύτης.
\par 21 Και ακούσαντες εισήλθον την αυγήν εις το ιερόν και εδίδασκον. Ελθών δε ο αρχιερεύς και οι μετ' αυτού, συνεκάλεσαν το συνέδριον και όλην την γερουσίαν των υιών του Ισραήλ και έστειλαν εις το δεσμωτήριον, διά να φέρωσιν αυτούς.
\par 22 Οι δε υπηρέται ελθόντες δεν εύρον αυτούς εν τη φυλακή, και επιστρέψαντες απήγγειλαν,
\par 23 λέγοντες ότι το μεν δεσμωτήριον εύρομεν κεκλεισμένον μετά πάσης ασφαλείας, και τους φύλακας ισταμένους έξω έμπροσθεν των θυρών, ανοίξαντες δε ουδένα εύρομεν έσω.
\par 24 Ως δε ήκουσαν τους λόγους τούτους και ο ιερεύς και ο στρατηγός του ιερού και οι αρχιερείς, ήσαν εν απορία περί αυτών εις τι έμελλε να καταντήση τούτο.
\par 25 Και ελθών τις απήγγειλε προς αυτούς, λέγων ότι ιδού, οι άνθρωποι, τους οποίους εβάλετε εις την φυλακήν, ίστανται εν τω ιερώ και διδάσκουσι τον λαόν.
\par 26 Τότε υπήγεν ο στρατηγός μετά των υπηρετών και έφερεν αυτούς, ουχί μετά βίας· διότι εφοβούντο τον λαόν, μη λιθοβοληθώσι.
\par 27 Και αφού έφεραν αυτούς, έστησαν εν τω συνεδρίω. Και ηρώτησεν αυτούς ο αρχιερεύς
\par 28 λέγων· Δεν σας παρηγγείλαμεν ρητώς να μη διδάσκητε εν τω ονόματι τούτω; και ιδού, εγεμίσατε την Ιερουσαλήμ από της διδαχής σας, και θέλετε να φέρητε εφ' ημάς το αίμα του ανθρώπου τούτου.
\par 29 Αποκριθείς δε ο Πέτρος και οι απόστολοι, είπον· Πρέπει να πειθαρχώμεν εις τον Θεόν μάλλον παρά εις τους ανθρώπους.
\par 30 Ο Θεός των πατέρων ημών ανέστησε τον Ιησούν, τον οποίον σεις εθανατώσατε κρεμάσαντες επί ξύλου·
\par 31 τούτον ο Θεός ύψωσε διά της δεξιάς αυτού Αρχηγόν και Σωτήρα, διά να δώση μετάνοιαν εις τον Ισραήλ και άφεσιν αμαρτιών.
\par 32 Και ημείς είμεθα μάρτυρες αυτού περί των λόγων τούτων, και το Πνεύμα δε το Άγιον, το οποίον έδωκεν ο Θεός εις τους πειθαρχούντας εις αυτόν.
\par 33 Οι δε ακούσαντες έτριζον τους οδόντας και εβουλεύοντο να θανατώσωσιν αυτούς.
\par 34 Σηκωθείς δε εν τω συνεδρίω Φαρισαίός τις Γαμαλιήλ το όνομα, νομοδιδάσκαλος τιμώμενος υπό παντός του λαού, προσέταξε να εκβάλωσι τους αποστόλους δι' ολίγην ώραν,
\par 35 και είπε προς αυτούς· Άνδρες Ισραηλίται, προσέχετε εις εαυτούς περί των ανθρώπων τούτων τι μέλλετε να πράξητε.
\par 36 Διότι προ τούτων των ημερών εσηκώθη ο Θευδάς, λέγων εαυτόν ότι είναι μέγας τις, εις τον οποίον προσεκολλήθη αριθμός ανδρών έως τετρακοσίων· όστις εφονεύθη, και πάντες όσοι επείθοντο εις αυτόν διελύθησαν και κατήντησαν εις ουδέν.
\par 37 Μετά τούτον εσηκώθη Ιούδας ο Γαλιλαίος εν ταις ημέραις της απογραφής και έσυρεν οπίσω αυτού αρκετόν λαόν· και εκείνος απωλέσθη, και πάντες όσοι επείθοντο εις αυτόν διεσκορπίσθησαν.
\par 38 Και τώρα σας λέγω, απέχετε από των ανθρώπων τούτων και αφήσατε αυτούς· διότι εάν η βουλή αύτη ή το έργον τούτο ήναι εξ ανθρώπων, θέλει ματαιωθή·
\par 39 εάν όμως ήναι εκ Θεού, δεν δύνασθε να ματαιώσητε αυτό, και προσέχετε μήπως ευρεθήτε και θεομάχοι. Και επείσθησαν εις αυτόν,
\par 40 και προσκαλέσαντες τους αποστόλους, έδειραν και παρήγγειλαν να μη λαλώσιν εν τω ονόματι του Ιησού, και απέλυσαν αυτούς.
\par 41 Εκείνοι λοιπόν ανεχώρουν από προσώπου του συνεδρίου, χαίροντες ότι υπέρ του ονόματος αυτού ηξιώθησαν να ατιμασθώσι.
\par 42 Και πάσαν ημέραν εν τω ιερώ και κατ' οίκον δεν έπαυον διδάσκοντες και ευαγγελιζόμενοι τον Ιησούν Χριστόν.

\chapter{6}

\par 1 Εν δε ταις ημέραις ταύταις, ότε επληθύνοντο οι μαθηταί, έγεινε γογγυσμός των Ελληνιστών κατά των Εβραίων, ότι αι χήραι αυτών παρεβλέποντο εν τη καθημερινή διακονία.
\par 2 Τότε οι δώδεκα, προσκαλέσαντες το πλήθος των μαθητών, είπον· Δεν είναι πρέπον να αφήσωμεν ημείς τον λόγον του Θεού και να διακονώμεν εις τραπέζας.
\par 3 Σκέφθητε λοιπόν, αδελφοί, να εκλέξητε εξ υμών επτά άνδρας μαρτυρουμένους, πλήρεις Πνεύματος Αγίου και σοφίας, τους οποίους ας καταστήσωμεν επί της χρείας ταύτης·
\par 4 ημείς δε θέλομεν εμμένει εν τη προσευχή και τη διακονία του λόγου.
\par 5 Και ήρεσεν ο λόγος ενώπιον παντός του πλήθους· και εξέλεξαν τον Στέφανον, άνδρα πλήρη πίστεως και Πνεύματος Αγίου, και Φίλιππον και Πρόχορον και Νικάνορα και Τίμωνα και Παρμενάν και Νικόλαον, προσήλυτον Αντιοχέα,
\par 6 τους οποίους έστησαν ενώπιον των αποστόλων και προσευχηθέντες επέθεσαν επ' αυτούς τας χείρας.
\par 7 Και ο λόγος του Θεού ηύξανε, και επληθύνετο ο αριθμός των μαθητών εν Ιερουσαλήμ σφόδρα, και πολύ πλήθος των ιερέων υπήκουον εις την πίστιν.
\par 8 Ο δε Στέφανος, πλήρης πίστεως και δυνάμεως, έκαμνε τέρατα και σημεία μεγάλα εν τω λαώ.
\par 9 Και εσηκώθησάν τινές των εκ της συναγωγής της λεγομένης Λιβερτίνων και Κυρηναίων και Αλεξανδρέων και των από Κιλικίας και Ασίας, φιλονεικούντες με τον Στέφανον,
\par 10 και δεν ηδύναντο να αντισταθώσιν εις την σοφίαν και εις το πνεύμα, με το οποίον ελάλει.
\par 11 Τότε έβαλον κρυφίως ανθρώπους, λέγοντας ότι ηκούσαμεν αυτόν λαλούντα λόγια βλάσφημα κατά του Μωϋσέως και του Θεού·
\par 12 και διήγειραν τον λαόν και τους πρεσβυτέρους και τους γραμματείς, και επελθόντες ήρπασαν αυτόν και έφεραν εις το συνέδριον,
\par 13 και έστησαν μάρτυρας ψευδείς, λέγοντας· Ο άνθρωπος ούτος δεν παύει λαλών λόγια βλάσφημα κατά του αγίου τούτου τόπου και του νόμου·
\par 14 διότι ηκούσαμεν αυτόν λέγοντα, ότι Ιησούς ο Ναζωραίος ούτος θέλει καταλύσει τον τόπον τούτον και αλλάξει τα έθιμα, τα οποία παρέδωκεν εις ημάς ο Μωϋσής.
\par 15 Και ατενίσαντες εις αυτόν πάντες οι καθήμενοι εν τω συνεδρίω, είδον το πρόσωπον αυτού ως πρόσωπον αγγέλου.

\chapter{7}

\par 1 Είπε δε ο αρχιερεύς· Τωόντι ούτως έχουσι ταύτα;
\par 2 Ο δε είπεν· Άνδρες αδελφοί και πατέρες, ακούσατε. Ο Θεός της δόξης εφάνη εις τον πατέρα ημών Αβραάμ ότε ήτο εν τη Μεσοποταμία, πριν κατοικήση εν Χαρράν,
\par 3 και είπε προς αυτόν· Έξελθε εκ της γης σου και εκ της συγγενείας σου, και ελθέ εις γην, την οποίαν θέλω σοι δείξει.
\par 4 Τότε εξελθών εκ της γης των Χαλδαίων κατώκησεν εν Χαρράν· και εκείθεν μετά τον θάνατον του πατρός αυτού μετώκισεν αυτόν εις την γην ταύτην, εις την οποίαν σεις κατοικείτε τώρα·
\par 5 και δεν έδωκεν εις αυτόν κληρονομίαν εν αυτή ουδέ βήμα ποδός, υπεσχέθη δε ότι θέλει δώσει αυτήν κτήμα εις αυτήν και εις το σπέρμα αυτού μετ' αυτόν, ενώ δεν είχε τέκνον.
\par 6 Ελάλησε δε προς αυτόν ο Θεός ούτως, ότι το σπέρμα αυτού θέλει είσθαι πάροικον εν γη ξένη, και θέλουσι δουλώσει αυτό και καταθλίψει τετρακόσια έτη·
\par 7 και το έθνος, εις το οποίον θέλουσι δουλωθή, εγώ θέλω κρίνει, είπεν ο Θεός· και μετά ταύτα θέλουσιν εξέλθει και θέλουσι με λατρεύσει εν τω τόπω τούτω.
\par 8 Και έδωκεν εις αυτόν διαθήκην περιτομής· και ούτως εγέννησε τον Ισαάκ και περιέτεμεν αυτόν τη ογδόη ημέρα, και ο Ισαάκ εγέννησε τον Ιακώβ, και ο Ιακώβ τους δώδεκα πατριάρχας.
\par 9 Και οι πατριάρχαι, φθονήσαντες τον Ιωσήφ, επώλησαν εις την Αίγυπτον. Ο Θεός όμως ήτο μετ' αυτού,
\par 10 και ηλευθέρωσεν αυτόν εκ πασών των θλίψεων αυτού και έδωκεν εις αυτόν χάριν και σοφίαν ενώπιον Φαραώ του βασιλέως της Αιγύπτου, όστις κατέστησεν αυτόν κυβερνήτην επί της Αιγύπτου και όλου του οίκου αυτού.
\par 11 Ήλθε δε πείνα εφ' όλην την γην της Αιγύπτου και Χαναάν και θλίψις μεγάλη, και δεν εύρισκον τροφάς οι πατέρες ημών.
\par 12 Ακούσας δε ο Ιακώβ ότι υπήρχε σίτος εν Αιγύπτω, εξαπέστειλε πρώτην φοράν τους πατέρας ημών·
\par 13 και εν τη δευτέρα ανεγνωρίσθη ο Ιωσήφ εις τους αδελφούς αυτού, και εφανερώθη εις τον Φαραώ το γένος του Ιωσήφ.
\par 14 Αποστείλας δε ο Ιωσήφ, εκάλεσε προς εαυτόν τον πατέρα αυτού Ιακώβ και πάσαν την συγγένειαν αυτού εβδομήκοντα πέντε ψυχάς.
\par 15 Και κατέβη ο Ιακώβ εις Αίγυπτον και ετελεύτησεν εκεί αυτός και οι πατέρες ημών,
\par 16 και μετεκομίσθησαν εις Συχέμ και ετέθησαν εν τω μνήματι, το οποίον ηγόρασεν ο Αβραάμ με τιμήν αργυρίου παρά των υιών του Εμμώρ πατρός του Συχέμ.
\par 17 Καθώς δε επλησίαζεν ο καιρός της επαγγελίας, την οποίαν ώμοσεν ο Θεός προς τον Αβραάμ, ηύξησεν ο λαός και επληθύνθη εν Αιγύπτω,
\par 18 εωσού εσηκώθη βασιλεύς άλλος, όστις δεν ήξευρε τον Ιωσήφ.
\par 19 Ούτος δολιευθείς το γένος ημών, κατέθλιψε τους πατέρας ημών, ώστε να κάμη να ρίπτωνται τα βρέφη αυτών, διά να μη ζωογονώνται·
\par 20 εν τούτω τω καιρώ εγεννήθη ο Μωϋσής, και είχε θείον κάλλος· όστις ανετράφη τρεις μήνας εν τω οίκω του πατρός αυτού.
\par 21 Αφού δε ερρίφθη, ανέλαβεν αυτόν η θυγάτηρ του Φαραώ και ανέθρεψεν αυτόν διά να ήναι υιός αυτής.
\par 22 Και εδιδάχθη ο Μωϋσής πάσαν την σοφίαν των Αιγυπτίων και ήτο δυνατός εν λόγοις και εν έργοις.
\par 23 Ενώ δε ετελείονε το τεσσαρακοστόν έτος της ηλικίας αυτού, ήλθεν εις την καρδίαν αυτού να επισκεφθή τους αδελφούς αυτού, τους υιούς Ισραήλ.
\par 24 Και ιδών τινά αδικούμενον, υπερησπίσθη αυτόν και έκαμεν εκδίκησιν υπέρ του καταθλιβομένου, πατάξας τον Αιγύπτιον.
\par 25 Ενόμιζε δε ότι οι αδελφοί αυτού ήθελον νοήσει ότι ο Θεός διά της χειρός αυτού δίδει εις αυτούς σωτηρίαν· εκείνοι όμως δεν ενόησαν.
\par 26 Την δε ακόλουθον ημέραν εφάνη εις αυτούς, ενώ εμάχοντο, και παρεκίνησεν αυτούς εις ειρήνην, ειπών· Άνθρωποι, αδελφοί είσθε σείς· διά τι αδικείτε αλλήλους;
\par 27 Ο δε αδικών τον πλησίον απέσπρωξεν αυτόν, ειπών· Τις σε κατέστησεν άρχοντα και δικαστήν εφ' ημάς;
\par 28 Μήπως θέλεις συ να με φονεύσης, καθ' ον τρόπον εφόνευσας χθές τον Αιγύπτιον;
\par 29 Τότε ο Μωϋσής έφυγε διά τον λόγον τούτον και έγεινε πάροικος εν γη Μαδιάμ, όπου εγέννησε δύο υιούς.
\par 30 Και αφού συνεπληρώθησαν τεσσαράκοντα έτη, εφάνη εις αυτόν άγγελος Κυρίου εν τη ερήμω του όρους Σινά εν μέσω φλογός καιομένης βάτου.
\par 31 Ο δε Μωϋσής ιδών εθαύμασε διά το όραμα· και ενώ επλησίαζε διά να παρατηρήση, ήλθε φωνή Κυρίου προς αυτόν·
\par 32 Εγώ είμαι ο Θεός των πατέρων σου, ο Θεός του Αβραάμ και ο Θεός του Ισαάκ και ο Θεός του Ιακώβ. Έντρομος δε γενόμενος ο Μωϋσής, δεν ετόλμα να παρατηρήση.
\par 33 Και είπε προς αυτόν ο Κύριος· Λύσον το υπόδημα των ποδών σου· διότι ο τόπος, επί του οποίου ίστασαι, είναι γη αγία.
\par 34 Είδον, είδον την ταλαιπωρίαν του λαού μου του εν Αιγύπτω και ήκουσα τον στεναγμόν αυτών και κατέβην διά να ελευθερώσω αυτούς· και τώρα ελθέ, θέλω σε αποστείλει εις Αίγυπτον.
\par 35 Τούτον τον Μωϋσήν τον οποίον ηρνήθησαν ειπόντες· Τις σε κατέστησεν άρχοντα και δικαστήν; τούτον ο Θεός απέστειλεν αρχηγόν και λυτρωτήν διά χειρός του αγγέλου του φανέντος εις αυτόν εν τη βάτω.
\par 36 Ούτος εξήγαγεν αυτούς, αφού έκαμε τέρατα και σημεία εν γη Αιγύπτου και εν τη Ερυθρά θαλάσση και εν τη ερήμω τεσσαράκοντα έτη.
\par 37 Ούτος είναι ο Μωϋσής, όστις είπε προς τους υιούς του Ισραήλ· προφήτην εκ των αδελφών σας θέλει σας αναστήσει Κύριος ο Θεός σας, ως εμέ· αυτού θέλετε ακούσει.
\par 38 Ούτος είναι όστις εν τη εκκλησία εν τη ερήμω εστάθη μετά του αγγέλου του λαλούντος προς αυτόν εν τω όρει Σινά και μετά των πατέρων ημών, και παρέλαβε λόγια ζωοποιά διά να δώση εις ημάς.
\par 39 Εις τον οποίον οι πατέρες ημών δεν ηθέλησαν να υπακούσωσιν, αλλ' απέβαλον και εστράφησαν εν ταις καρδίαις αυτών εις Αίγυπτον
\par 40 ειπόντες προς τον Ααρών· Κάμε εις ημάς θεούς, οίτινες θέλουσι προπορεύεσθαι ημών· διότι ούτος ο Μωϋσής, όστις εξήγαγεν ημάς εξ Αιγύπτου, δεν εξεύρομεν τι συνέβη εις αυτόν.
\par 41 Και κατεσκεύασαν μόσχον εν ταις ημέραις εκείναις και προσέφεραν θυσίαν εις το είδωλον και ευφραίνοντο εις τα έργα των χειρών αυτών.
\par 42 Όθεν εστράφη ο Θεός και παρέδωκεν αυτούς εις το να λατρεύσωσι την στρατιάν του ουρανού, καθώς είναι γεγραμμένον εν τω βιβλίω των προφητών. Μήπως προσεφέρατε εις εμέ σφάγια και θυσίας τεσσαράκοντα έτη εν τη ερήμω, οίκος Ισραήλ;
\par 43 Μάλιστα ανελάβετε την σκηνήν του Μολόχ και το άστρον του Θεού σας Ρεμφάν, τους τύπους, τους οποίους εκάμετε διά να προσκυνήτε αυτούς· διά τούτο θέλω σας μετοικίσει επέκεινα της Βαβυλώνος.
\par 44 Η σκηνή του μαρτυρίου ήτο μετά των πατέρων ημών εν τη ερήμω, καθώς διέταξεν εκείνος, όστις ελάλει προς τον Μωϋσήν, να κατασκευάση αυτήν κατά τον τύπον τον οποίον είχεν ιδεί·
\par 45 την οποίαν και παραλαβόντες οι πατέρες ημών, έφεραν μετά του Ιησού εις την κατακτηθείσαν γην των εθνών, τα οποία ο Θεός έξωσεν απ' έμπροσθεν των πατέρων ημών, έως των ημερών του Δαβίδ·
\par 46 όστις εύρε χάριν ενώπιον του Θεού και ηυχήθη να εύρη κατοικίαν διά τον Θεόν του Ιακώβ.
\par 47 Ο Σολομών δε ωκοδόμησεν εις αυτόν οίκον.
\par 48 Αλλ' ο Ύψιστος δεν κατοικεί εν χειροποιήτοις ναοίς, καθώς ο προφήτης λέγει·
\par 49 Ο ουρανός είναι θρόνος μου, η δε γη υποπόδιον των ποδών μου· ποίον οίκον θέλετε οικοδομήσει δι' εμέ, λέγει Κύριος, ή ποίος ο τόπος της αναπαύσεώς μου;
\par 50 Η χειρ μου δεν έκαμε ταύτα πάντα;
\par 51 Σκληροτράχηλοι και απερίτμητοι την καρδίαν και τα ώτα, σεις πάντοτε αντιφέρεσθε κατά του Πνεύματος του Αγίου· καθώς οι πατέρες σας, ούτω και σεις.
\par 52 Τίνα των προφητών δεν εδίωξαν οι πατέρες σας; μάλιστα εφόνευσαν εκείνους, οίτινες προκατήγγειλαν περί της ελεύσεως του δικαίου, του οποίου σεις εγείνατε τώρα προδόται και φονείς·
\par 53 οίτινες ελάβετε τον νόμον εκ διαταγών αγγέλων και δεν εφυλάξατε.
\par 54 Ακούοντες δε ταύτα, κατεκόπτοντο τας καρδίας αυτών και έτριζον τους οδόντας κατ' αυτού.
\par 55 Ο δε Στέφανος, πλήρης ων Πνεύματος Αγίου, ατενίσας εις τον ουρανόν, είδε την δόξαν του Θεού και τον Ιησούν ιστάμενον εκ δεξιών του Θεού
\par 56 και είπεν· Ιδού, θεωρώ τους ουρανούς ανεωγμένους και τον Υιόν του ανθρώπου ιστάμενον εκ δεξιών του Θεού.
\par 57 Τότε φωνάξαντες μετά φωνής μεγάλης, έφραξαν τα ώτα αυτών και ώρμησαν ομοθυμαδόν επ' αυτόν,
\par 58 και εκβαλόντες έξω της πόλεως ελιθοβόλουν. Και οι μάρτυρες απέθεσαν τα ιμάτια αυτών εις τους πόδας νεανίου τινός ονομαζομένου Σαύλου.
\par 59 Και ελιθοβόλουν τον Στέφανον, επικαλούμενον και λέγοντα· Κύριε Ιησού, δέξαι το πνεύμά μου.
\par 60 Και γονατίσας εφώναξε μετά φωνής μεγάλης· Κύριε, μη λογαριάσης εις αυτούς την αμαρτίαν ταύτην. Και τούτο ειπών εκοιμήθη.

\chapter{8}

\par 1 Ο δε Σαύλος ήτο σύμφωνος εις τον φόνον αυτού. Και έγεινεν εν εκείνη τη ημέρα διωγμός μέγας κατά της εκκλησίας της εν Ιεροσολύμοις και πάντες διεσπάρησαν εις τους τόπους της Ιουδαίας και Σαμαρείας, πλην των αποστόλων.
\par 2 Έφεραν δε τον Στέφανον εις τον τάφον άνδρες ευλαβείς και έκαμον θρήνον μέγαν επ' αυτόν.
\par 3 Ο δε Σαύλος εκακοποίει την εκκλησίαν, εμβαίνων εις πάσαν οικίαν και σύρων άνδρας και γυναίκας, παρέδιδεν εις την φυλακήν.
\par 4 Οι μεν λοιπόν διασπαρέντες διήλθον ευαγγελιζόμενοι τον λόγον.
\par 5 Ο δε Φίλιππος, καταβάς εις την πόλιν της Σαμαρείας, εκήρυττεν εις αυτούς τον Χριστόν.
\par 6 Και οι όχλοι προσείχον ομοθυμαδόν εις τα λεγόμενα υπό του Φιλίππου, ακούοντες και βλέποντες τα θαύματα, τα οποία έκαμνε.
\par 7 Διότι εκ πολλών εχόντων πνεύματα ακάθαρτα εξήρχοντο αυτά φωνάζοντα μετά μεγάλης φωνής, και πολλοί παραλυτικοί και χωλοί εθεραπεύθησαν,
\par 8 και έγεινε χαρά μεγάλη εν εκείνη τη πόλει.
\par 9 Άνθρωπος δε τις Σίμων ονομαζόμενος προϋπήρχεν εν τη πόλει, κάμνων μαγείας και εκπλήττων τον λαόν της Σαμαρείας, λέγων εαυτόν ότι είναι μέγας τις·
\par 10 εις τον οποίον έδιδον προσοχήν πάντες από μικρού έως μεγάλου, λέγοντες· Ούτος είναι η δύναμις του Θεού η μεγάλη.
\par 11 Έδιδον δε προσοχήν εις αυτόν, διότι είχεν εκπλήξει αυτούς πολύν καιρόν με τας μαγείας.
\par 12 Ότε όμως επίστευσαν εις τον Φίλιππον ευαγγελιζόμενον τα περί της βασιλείας του Θεού και του ονόματος του Ιησού Χριστού, εβαπτίζοντο άνδρες τε και γυναίκες.
\par 13 Ο δε Σίμων και αυτός επίστευσε, και βαπτισθείς έμενε πάντοτε μετά του Φιλίππου, και θεωρών σημεία και θαύματα μεγάλα γινόμενα εξεπλήττετο.
\par 14 Ακούσαντες δε οι απόστολοι οι εν Ιεροσολύμοις ότι η Σαμάρεια εδέχθη τον λόγον του Θεού, απέστειλαν προς αυτούς τον Πέτρον και Ιωάννην·
\par 15 Οίτινες καταβάντες προσηυχήθησαν περί αυτών διά να λάβωσι Πνεύμα Αγιον·
\par 16 διότι δεν είχεν έτι επιπέσει επ' ουδένα εξ αυτών, αλλά μόνον ήσαν βεβαπτισμένοι εις το όνομα του Κυρίου Ιησού.
\par 17 Τότε επέθετον τας χείρας επ' αυτούς, και ελάμβανον Πνεύμα Άγιον.
\par 18 Ιδών δε ο Σίμων ότι διά της επιθέσεως των χειρών των αποστόλων δίδεται το Πνεύμα το Άγιον, προσέφερεν εις αυτούς χρήματα,
\par 19 λέγων· Δότε και εις εμέ την εξουσίαν ταύτην, ώστε εις όντινα επιθέσω τας χείρας να λαμβάνη Πνεύμα Άγιον.
\par 20 Και ο Πέτρος είπε προς αυτόν· το αργύριόν σου ας ήναι μετά σου εις απώλειαν, διότι ενόμισας ότι η δωρεά του Θεού αποκτάται διά χρημάτων.
\par 21 Συ δεν έχεις μερίδα ουδέ κλήρον εν τω λόγω τούτω· διότι η καρδία σου δεν είναι ευθεία ενώπιον του Θεού.
\par 22 Μετανόησον λοιπόν από της κακίας σου ταύτης και δεήθητι του Θεού, ίσως συγχωρηθή εις σε η επίνοια της καρδίας σου·
\par 23 επειδή σε βλέπω ότι είσαι εις χολήν πικρίας και δεσμόν αδικίας.
\par 24 Αποκριθείς δε ο Σίμων, είπε· Δεήθητε σεις υπέρ εμού προς τον Κύριον, διά να μη έλθη επ' εμέ μηδέν εξ όσων είπετε.
\par 25 Εκείνοι λοιπόν, αφού εμαρτύρησαν και ελάλησαν τον λόγον του Κυρίου, υπέστρεψαν εις Ιερουσαλήμ, κηρύξαντες το ευαγγέλιον και εν πολλαίς κώμαις των Σαμαρειτών.
\par 26 Άγγελος δε Κυρίου ελάλησε προς τον Φίλιππον, λέγων· Σηκώθητι και ύπαγε προς μεσημβρίαν εις την οδόν την καταβαίνουσαν από Ιερουσαλήμ εις Γάζαν· αύτη είναι έρημος.
\par 27 Και σηκωθείς υπήγε. Και ιδού, άνθρωπος Αιθίοψ ευνούχος, άρχων της Κανδάκης της βασιλίσσης των Αιθιόπων, όστις ήτο επί πάντων των θησαυρών αυτής, ούτος είχεν ελθεί διά να προσκυνήση εις Ιερουσαλήμ,
\par 28 και υπέστρεφε και καθήμενος επί της αμάξης αυτού, ανεγίνωσκε τον προφήτην Ησαΐαν.
\par 29 Είπε δε το Πνεύμα προς τον Φίλιππον· Πλησίασον και προσκολλήθητι εις την άμαξαν ταύτην.
\par 30 Και ο Φίλιππος έδραμε πλησίον και ήκουσεν αυτόν αναγινώσκοντα τον προφήτην Ησαΐαν και είπεν· Άραγε γινώσκεις α αναγινώσκεις;
\par 31 Ο δε είπε· Και πως ήθελον δυνηθή, εάν δεν με οδηγήση τις; Και παρεκάλεσε τον Φίλιππον να αναβή και να καθήση μετ' αυτού.
\par 32 Το δε χωρίον της γραφής, το οποίον ανεγίνωσκεν, ήτο τούτο. Εφέρθη ως πρόβατον επί σφαγήν· και ως αρνίον έμπροσθεν του κείροντος αυτό άφωνον, ούτω δεν ανοίγει το στόμα αυτού.
\par 33 Εν τη ταπεινώσει αυτού αφηρέθη η κρίσις αυτού· την δε γενεάν αυτού τις θέλει διηγηθή; διότι σηκόνεται από της γης η ζωή αυτού.
\par 34 Αποκριθείς δε ο ευνούχος προς τον Φίλιππον, είπε· Παρακαλώ σε, περί τίνος λέγει τούτο ο προφήτης; περί εαυτού περί άλλου τινός;
\par 35 Και ανοίξας ο Φίλιππος το στόμα αυτού και αρχίσας από της γραφής ταύτης, ευηγγελίσατο εις αυτόν τον Ιησούν.
\par 36 Και καθώς εξηκολούθουν την οδόν, ήλθον εις το ύδωρ, και λέγει ο ευνούχος· Ιδού ύδωρ· τι με εμποδίζει να βαπτισθώ;
\par 37 Και ο Φίλιππος είπεν· Εάν πιστεύης εξ όλης της καρδίας, δύνασαι. Και αποκριθείς είπε· Πιστεύω ότι ο Ιησούς Χριστός είναι ο Υιός του Θεού.
\par 38 Και προσέταξε να σταθή η άμαξα, και κατέβησαν αμφότεροι εις το ύδωρ, ο Φίλιππος και ο ευνούχος, και εβάπτισεν αυτόν.
\par 39 Ότε δε ανέβησαν εκ του ύδατος, Πνεύμα Κυρίου ήρπασε τον Φίλιππον, και δεν είδεν αυτόν πλέον ο ευνούχος· αλλ' επορεύετο την οδόν αυτού χαίρων.
\par 40 Ο δε Φίλιππος ευρέθη εις Άζωτον, και διερχόμενος εκήρυττεν εις πάσας τας πόλεις, εωσού ήλθεν εις Καισάρειαν.

\chapter{9}

\par 1 Ο δε Σαύλος, πνέων έτι απειλήν και φόνον κατά των μαθητών του Κυρίου, ήλθε προς τον αρχιερέα
\par 2 και εζήτησε παρ' αυτού επιστολάς εις Δαμασκόν προς τας συναγωγάς, όπως εάν εύρη τινάς εκ της οδού ταύτης, άνδρας τε και γυναίκας, φέρη δεδεμένους εις Ιερουσαλήμ.
\par 3 Ενώ δε πορευόμενος επλησίαζεν εις την Δαμασκόν, εξαίφνης ήστραψε περί αυτόν φως από του ουρανού,
\par 4 και πεσών επί την γην, ήκουσε φωνήν λέγουσαν προς αυτόν· Σαούλ, Σαούλ, τι με διώκεις;
\par 5 Και είπε· Τις είσαι, Κύριε; Και ο Κύριος είπεν· Εγώ είμαι ο Ιησούς, τον οποίον συ διώκεις· σκληρόν σοι είναι να λακτίζης προς κέντρα.
\par 6 Ο δε τρέμων και έκθαμβος γενόμενος, είπε· Κύριε, τι θέλεις να κάμω; Και ο Κύριος είπε προς αυτόν· Σηκώθητι και είσελθε εις την πόλιν, και θέλει σοι λαληθή τι πρέπει να κάμης.
\par 7 Οι δε άνδρες οι συνοδεύοντες αυτόν ίσταντο άφωνοι, ακούοντες μεν την φωνήν, μηδένα όμως βλέποντες.
\par 8 Εσηκώθη δε ο Σαύλος από της γης, και έχων ανεωγμένους τους οφθαλμούς αυτού δεν έβλεπεν ουδένα· και χειραγωγούντες αυτόν εισήγαγον εις Δαμασκόν.
\par 9 Και ήτο τρεις ημέρας χωρίς να βλέπη, και δεν έφαγεν ουδέ έπιεν.
\par 10 Ήτο δε τις μαθητής εν Δαμασκώ Ανανίας ονομαζόμενος, και είπε προς αυτόν ο Κύριος δι' οράματος· Ανανία· Ο δε είπεν· Ιδού εγώ, Κύριε.
\par 11 Και ο Κύριος είπε προς αυτόν· Σηκωθείς ύπαγε εις την οδόν την ονομαζομένην Ευθείαν και ζήτησον εν τη οικία του Ιούδα τινά Σαύλον ονομαζόμενον Ταρσέα· διότι ιδού, προσεύχεται,
\par 12 και είδε δι' οράματος άνθρωπον Ανανίαν ονομαζόμενον ότι εισήλθε και έθεσεν επ' αυτόν την χείρα, διά να αναβλέψη.
\par 13 Απεκρίθη δε ο Ανανίας· Κύριε, ήκουσα από πολλών περί του ανδρός τούτου, όσα κακά έπραξεν εις τους αγίους σου εν Ιερουσαλήμ·
\par 14 και εδώ έχει εξουσίαν παρά των αρχιερέων να δέση πάντας τους επικαλουμένους το όνομά σου.
\par 15 Είπε δε προς αυτόν ο Κύριος· Ύπαγε, διότι ούτος είναι σκεύος εκλογής εις εμέ, διά να βαστάση το όνομά μου ενώπιον εθνών και βασιλέων και των υιών Ισραήλ·
\par 16 επειδή εγώ θέλω δείξει εις αυτόν όσα πρέπει να πάθη υπέρ του ονόματός μου.
\par 17 Υπήγε δε ο Ανανίας και εισήλθεν εις την οικίαν, και επιθέσας επ' αυτόν τας χείρας είπε· Σαούλ αδελφέ, ο Κύριος με απέστειλεν, ο Ιησούς όστις εφάνη εις σε εν τη οδώ καθ' ην ήρχου, διά να αναβλέψης και να πλησθής Πνεύματος Αγίου.
\par 18 Και ευθύς έπεσον από των οφθαλμών αυτού ως λέπη, και ανέβλεψεν ευθύς, και σηκωθείς εβαπτίσθη.
\par 19 Και λαβών τροφήν εδυναμώθη. Διέτριψε δε ο Σαύλος ημέρας τινάς μετά των εν Δαμασκώ μαθητών,
\par 20 και ευθύς εκήρυττεν εν ταις συναγωγαίς τον Χριστόν ότι ούτος είναι ο Υιός του Θεού.
\par 21 Εξεπλήττοντο δε πάντες οι ακούοντες και έλεγον· Δεν είναι ούτος, όστις εξωλόθρευσεν εν Ιερουσαλήμ τους επικαλουμένους το όνομα τούτο και εδώ διά τούτο είχεν ελθεί διά να φέρη αυτούς δεδεμένους προς τους αρχιερείς;
\par 22 Ο δε Σαύλος μάλλον ενεδυναμούτο και συνέχεε τους Ιουδαίους τους κατοικούντας εν Δαμασκώ, αποδεικνύων ότι ούτος είναι ο Χριστός.
\par 23 Και αφού παρήλθον ημέραι ικαναί, συνεβουλεύθησαν οι Ιουδαίοι να θανατώσωσιν αυτόν·
\par 24 εγνωστοποιήθη δε εις τον Σαύλον η επιβουλή αυτών. Και παρεφύλαττον τας πύλας ημέραν και νύκτα, διά να θανατώσωσιν αυτόν·
\par 25 λαβόντες δε αυτόν οι μαθηταί, διά νυκτός κατεβίβασαν διά του τείχους κρεμάσαντες εντός σπυρίδος.
\par 26 Και ελθών ο Σαύλος εις Ιερουσαλήμ επροσπάθει να προσκολληθή εις τους μαθητάς· πλην πάντες εφοβούντο αυτόν, μη πιστεύοντες ότι είναι μαθητής.
\par 27 Ο Βαρνάβας δε παραλαβών αυτόν έφερε προς τους αποστόλους, και διηγήθη προς αυτούς πως είδε τον Κύριον εν τη οδώ και ότι ελάλησε προς αυτόν, και πως εν Δαμασκώ, εκήρυξε μετά παρρησίας εν τω ονόματι του Ιησού.
\par 28 Και ήτο μετ' αυτών εν Ιερουσαλήμ εισερχόμενος και εξερχόμενος και μετά παρρησίας κηρύττων εν τω ονόματι του Κυρίου Ιησού,
\par 29 και ελάλει και εφιλονείκει μετά των Ελληνιστών· εκείνοι δε κατεγίνοντο εις το να θανατώσωσιν αυτόν.
\par 30 Μαθόντες δε οι αδελφοί, κατεβίβασαν αυτόν εις Καισάρειαν και εξαπέστειλαν αυτόν εις Ταρσόν.
\par 31 Αι μεν λοιπόν εκκλησίαι καθ' όλην την Ιουδαίαν και Γαλιλαίαν και Σαμάρειαν είχον ειρήνην, οικοδομούμεναι και περιπατούσαι εν τω φόβω του Κυρίου, και διά της παρηγορίας του Αγίου Πνεύματος επληθύνοντο.
\par 32 Ο δε Πέτρος, διερχόμενος διά πάντων, κατέβη και προς τους αγίους τους κατοικούντας την Λύδδαν.
\par 33 Και εύρεν άνθρωπον τινά Αινέαν το όνομα, όστις ήτο παραλυτικός, από ετών οκτώ κατακείμενος επί κραββάτου.
\par 34 Και είπε προς αυτόν ο Πέτρος· Αινέα, σε ιατρεύει Ιησούς ο Χριστός· σηκώθητι και στρώσον την κλίνην σου. Και ευθύς εσηκώθη.
\par 35 Και είδον αυτόν πάντες οι κατοικούντες την Λύδδαν και τον Σάρωνα, οίτινες επέστρεψαν εις τον Κύριον.
\par 36 Και εν Ιόππη ήτο τις μαθήτρια ονόματι Ταβιθά, ήτις διερμηνευομένη λέγεται Δορκάς· αύτη ήτο πλήρης αγαθών έργων και ελεημοσυνών, τας οποίας έκαμνε·
\par 37 κατ' εκείνας δε τας ημέρας συνέβη ασθενήσασα να αποθάνη· και λούσαντες αυτήν έθεσαν εις ανώγεον.
\par 38 Και επειδή η Λύδδα ήτο πλησίον της Ιόππης, ακούσαντες οι μαθηταί ότι ο Πέτρος είναι εν αυτή, απέστειλαν προς αυτόν δύο άνδρας, παρακαλούντες να μη βραδύνη να περάση έως εις αυτούς.
\par 39 Και σηκωθείς ο Πέτρος, υπήγε μετ' αυτών· τον οποίον ελθόντα ανεβίβασαν εις το ανώγεον, και παρεστάθησαν ενώπιον αυτού πάσαι αι χήραι, κλαίουσαι και δεικνύουσαι χιτώνας και ιμάτια, όσα η Δορκάς ειργάζετο ότε ήτο μετ' αυτών.
\par 40 Ο δε Πέτρος, εκβαλών έξω πάντας, εγονάτισε και προσηυχήθη και στραφείς προς το σώμα, είπε· Ταβιθά, ανάστηθι. Η δε ήνοιξε τους οφθαλμούς αυτής και ιδούσα τον Πέτρον ανεκάθησεν.
\par 41 Ο δε έδωκε χείρα εις αυτήν και εσήκωσεν αυτήν, και φωνάξας τους αγίους και τας χήρας παρέστησεν αυτήν ζώσαν.
\par 42 Έγεινε δε τούτο γνωστόν καθ' όλην την Ιόππην, και πολλοί επίστευσαν εις τον Κύριον.
\par 43 Και ο Πέτρος έμεινεν ικανάς ημέρας εν Ιόππη παρά τινί Σίμωνι βυρσοδέψη.

\chapter{10}

\par 1 Ήτο δε τις άνθρωπος εν Καισαρεία ονόματι Κορνήλιος, εκατόνταρχος εκ του τάγματος του λεγομένου Ιταλικού,
\par 2 ευσεβής και φοβούμενος τον Θεόν μετά παντός του οίκου αυτού, όστις και έκαμνεν ελεημοσύνας εις τον λαόν πολλάς και εδέετο του Θεού διαπαντός·
\par 3 ούτος είδε φανερά δι' οράματος περί την εννάτην ώραν της ημέρας άγγελον του Θεού, ότι εισήλθε προς αυτόν και είπε προς αυτόν· Κορνήλιε.
\par 4 Ο δε ατενίσας εις αυτόν και έμφοβος γενόμενος, είπε· Τι είναι, Κύριε; Και είπε προς αυτόν· Αι προσευχαί σου και αι ελεημοσύναι σου ανέβησαν εις μνημόσυνόν σου ενώπιον του Θεού.
\par 5 Και τώρα πέμψον εις Ιόππην ανθρώπους και προσκάλεσον τον Σίμωνα, όστις επονομάζεται Πέτρος·
\par 6 ούτος ξενίζεται παρά τινί Σίμωνι βυρσοδέψη, έχοντι οικίαν πλησίον της θαλάσσης. Ούτος θέλει σοι λαλήσει τι πρέπει να κάμνης.
\par 7 Καθώς δε ανεχώρησεν ο άγγελος ο λαλών προς τον Κορνήλιον, εφώναξε δύο εκ των υπηρετών αυτού και ένα στρατιώτην ευσεβή εκ των διαμενόντων πάντοτε πλησίον αυτού,
\par 8 και διηγηθείς προς αυτούς τα πάντα, απέστειλεν αυτούς εις την Ιόππην.
\par 9 Τη δε επαύριον, ενώ εκείνοι ώδοιπόρουν και επλησίαζον εις την πόλιν, ανέβη ο Πέτρος εις το δώμα διά να προσευχηθή περί την έκτην ώραν.
\par 10 Και πεινάσας ήθελε να φάγη· ενώ δε ητοίμαζον, επήλθεν επ' αυτόν έκστασις,
\par 11 και θεωρεί τον ουρανόν ανεωγμένον και καταβαίνον επ' αυτόν σκεύος τι ως σινδόνα μεγάλην, το οποίον ήτο δεδεμένον από των τεσσάρων άκρων και κατεβιβάζετο επί την γην,
\par 12 εντός του οποίου υπήρχον πάντα τα τετράποδα της γης και τα θηρία και τα ερπετά και τα πετεινά του ουρανού.
\par 13 Και έγεινε φωνή προς αυτόν· Σηκωθείς, Πέτρε, σφάξον και φάγε·
\par 14 Ο δε Πέτρος είπε· Μη γένοιτο, Κύριε· διότι ουδέποτε έφαγον ουδέν βέβηλον ή ακάθαρτον.
\par 15 Και πάλιν εκ δευτέρου έγεινε φωνή προς αυτόν· Όσα ο Θεός εκαθάρισε, συ μη λέγε βέβηλα.
\par 16 Έγεινε δε τούτο τρίς, και πάλιν ανελήφθη το σκεύος εις τον ουρανόν.
\par 17 Ενώ δε ο Πέτρος ήτο εν απορία καθ' εαυτόν τι εσήμαινε το όραμα, το οποίον είδεν, ιδού, οι άνθρωποι οι απεσταλμένοι παρά του Κορνηλίου ερωτήσαντες και μαθόντες την οικίαν του Σίμωνος έφθασαν εις την πύλην,
\par 18 και φωνάξαντες ηρώτων αν ο Σίμων ο επονομαζόμενος Πέτρος ξενίζεται ενταύθα.
\par 19 Και ενώ ο Πέτρος διελογίζετο περί του οράματος, είπε προς αυτόν το Πνεύμα· Ιδού, τρεις άνθρωποι σε ζητούσι·
\par 20 σηκωθείς λοιπόν κατάβηθι και ύπαγε μετ' αυτών, μηδόλως διστάζων, διότι εγώ απέστειλα αυτούς.
\par 21 Καταβάς δε ο Πέτρος προς τους ανθρώπους τους απεσταλμένους προς αυτόν από του Κορνηλίου, είπεν· Ιδού, εγώ είμαι εκείνος τον οποίον ζητείτε· τις η αιτία διά την οποίαν ήλθετε;
\par 22 Οι δε είπον· Κορνήλιος ο εκατόνταρχος, ανήρ δίκαιος και φοβούμενος τον Θεόν και μαρτυρούμενος υπό όλου του έθνους των Ιουδαίων, διετάχθη θεόθεν υπό αγίου αγγέλου να σε προσκαλέση εις τον οίκον αυτού και να ακούση λόγους παρά σου.
\par 23 Προσκαλέσας λοιπόν αυτούς έσω, εφιλοξένησε. Τη δε επαύριον εξήλθεν ο Πέτρος μετ' αυτών, και τινές των αδελφών των από της Ιόππης υπήγον μετ' αυτόν,
\par 24 και τη επαύριον εισήλθον εις την Καισάρειαν. Ο δε Κορνήλιος περιέμενεν αυτούς, συγκαλέσας τους συγγενείς αυτού και τους οικείους φίλους.
\par 25 Ως δε εισήλθεν ο Πέτρος, ελθών ο Κορνήλιος εις συνάντησιν αυτού, έπεσεν εις τους πόδας αυτού και προσεκύνησεν.
\par 26 Αλλ' ο Πέτρος εσήκωσεν αυτόν, λέγων· Σηκώθητι· και εγώ αυτός άνθρωπος είμαι.
\par 27 Και συνομιλών μετ' αυτού εισήλθε και ευρίσκει συνηγμένους πολλούς,
\par 28 και είπε προς αυτούς· Σεις εξεύρετε ότι είναι ασυγχώρητον εις άνθρωπον Ιουδαίον να συναναστρέφηται ή να πλησιάζη εις αλλόφυλον· ο Θεός όμως έδειξεν εις εμέ να μη λέγω μηδένα άνθρωπον βέβηλον ή ακάθαρτον·
\par 29 όθεν και προσκληθείς ήλθον χωρίς αντιλογίας. Ερωτώ λοιπόν διά τίνα λόγον με προσεκαλέσατε;
\par 30 Και ο Κορνήλιος είπε· Από τεσσάρων ημερών ήμην νηστεύων μέχρι της ώρας ταύτης, και την εννάτην ώραν προσηυχόμην εν τω οίκω μου· και ιδού, εστάθη ενώπιόν μου ανήρ με ενδύματα λαμπρά,
\par 31 και λέγει· Κορνήλιε, εισηκούσθη η προσευχή σου και αι ελεημοσύναι σου εμνημονεύθησαν ενώπιον του Θεού.
\par 32 Πέμψον λοιπόν εις Ιόππην και προσκάλεσον τον Σίμωνα, όστις επονομάζεται Πέτρος· ούτος ξενίζεται εν τη οικία Σίμωνος του βυρσοδέψου πλησίον της θαλάσσης· όστις ελθών θέλει σοι λαλήσει.
\par 33 Ευθύς λοιπόν έπεμψα προς σε, και συ έκαμες καλά ότι ήλθες. Τώρα λοιπόν ημείς πάντες παριστάμεθα ενώπιον του Θεού, διά να ακούσωμεν πάντα όσα προσετάχθησαν εις σε υπό του Θεού.
\par 34 Τότε ο Πέτρος ανοίξας το στόμα είπεν· Επ' αληθείας γνωρίζω ότι δεν είναι προσωπολήπτης ο Θεός,
\par 35 αλλ' εν παντί έθνει, όστις φοβείται αυτόν και εργάζεται δικαιοσύνην, είναι δεκτός εις αυτόν.
\par 36 Τον λόγον, τον οποίον απέστειλε προς τους υιούς Ισραήλ ευαγγελιζόμενος ειρήνην διά Ιησού Χριστού· ούτος είναι ο Κύριος πάντων·
\par 37 τον λόγον τούτον σεις εξεύρετε, όστις εκηρύχθη καθ' όλην την Ιουδαίαν, αρχίσας από της Γαλιλαίας, μετά το βάπτισμα, το οποίον εκήρυξεν ο Ιωάννης,
\par 38 πως ο Θεός έχρισε τον Ιησούν τον από Ναζαρέτ με Πνεύμα Άγιον και με δύναμιν, όστις διήλθεν ευεργετών και θεραπεύων πάντας τους καταδυναστευομένους υπό του διαβόλου, διότι ο Θεός ήτο μετ' αυτού·
\par 39 και ημείς είμεθα μάρτυρες πάντων όσα έκαμε και εν τη γη των Ιουδαίων και εν Ιερουσαλήμ· τον οποίον εφόνευσαν κρεμάσαντες επί ξύλου.
\par 40 Τούτον ο Θεός ανέστησε την τρίτην ημέραν και έκαμεν αυτόν να εμφανισθή
\par 41 ουχί εις πάντα τον λαόν, αλλ' εις μάρτυρας τους προδιωρισμένους υπό του Θεού, εις ημάς, οίτινες συνεφάγομεν και συνεπίομεν μετ' αυτού, αφού ανέστη εκ νεκρών·
\par 42 και παρήγγειλεν εις ημάς να κηρύξωμεν προς τον λαόν και να μαρτυρήσωμεν ότι αυτός είναι ο ωρισμένος υπό του Θεού κριτής ζώντων και νεκρών.
\par 43 Εις τούτον πάντες οι προφήται μαρτυρούσιν, ότι διά του ονόματος αυτού θέλει λάβει άφεσιν αμαρτιών πας ο πιστεύων εις αυτόν.
\par 44 Ενώ έτι ελάλει ο Πέτρος τους λόγους τούτους, επήλθε το Πνεύμα το Άγιον επί πάντας τους ακούοντας τον λόγον.
\par 45 Και εξεπλάγησαν οι εκ περιτομής πιστοί, όσοι ήλθον μετά του Πέτρου, ότι η δωρεά του Αγίου Πνεύματος εξεχύθη και επί τα έθνη·
\par 46 διότι ήκουον αυτούς λαλούντας γλώσσας και μεγαλύνοντας τον Θεόν. Τότε απεκρίθη ο Πέτρος·
\par 47 Μήπως δύναταί τις να εμποδίση το ύδωρ, ώστε να μη βαπτισθώσιν ούτοι, οίτινες έλαβον το Πνεύμα το Άγιον καθώς και ημείς;
\par 48 Και προσέταξεν αυτούς να βαπτισθώσιν εις το όνομα του Κυρίου. Τότε παρεκάλεσαν αυτόν να διαμείνη ημέρας τινάς.

\chapter{11}

\par 1 Ήκουσαν δε οι απόστολοι και οι αδελφοί οι όντες εν τη Ιουδαία ότι και τα έθνη εδέχθησαν τον λόγον του Θεού.
\par 2 Και ότε ανέβη ο Πέτρος εις Ιεροσόλυμα, εφιλονείκουν μετ' αυτού οι εκ περιτομής,
\par 3 λέγοντες ότι Εισήλθες προς ανθρώπους απεριτμήτους και συνέφαγες μετ' αυτών.
\par 4 Ο δε Πέτρος ήρχισε και εξέθετε προς αυτούς τα γενόμενα κατά σειράν, λέγων·
\par 5 Εγώ ήμην προσευχόμενος εν τη πόλει Ιόππη, και είδον όραμα εν εκστάσει, σκεύος τι καταβαίνον ως σινδόνα μεγάλην, ήτις δεδεμένη από των τεσσάρων άκρων κατεβιβάζετο εκ του ουρανού και ήλθε μέχρις εμού·
\par 6 εις την οποίαν ατενίσας παρετήρουν και είδον τα τετράποδα της γης και τα θηρία και τα ερπετά και τα πετεινά του ουρανού.
\par 7 Και ήκουσα φωνήν λέγουσαν προς εμέ· Σηκωθείς, Πέτρε, σφάξον και φάγε.
\par 8 Και είπον· Μη γένοιτο, Κύριε, διότι ουδέν βέβηλον ή ακάθαρτον εισήλθε ποτέ εις το στόμα μου.
\par 9 Και η φωνή μοι απεκρίθη εκ δευτέρου εκ του ουρανού· Όσα ο Θεός εκαθάρισε, συ μη λέγε βέβηλα.
\par 10 Έγεινε δε τούτο τρίς, και πάλιν ανεσύρθησαν άπαντα εις τον ουρανόν.
\par 11 Και ιδού, τη αυτή ώρα τρεις άνθρωποι έφθασαν εις την οικίαν, εν ή ήμην, απεσταλμένοι προς εμέ από Καισαρείας.
\par 12 Είπε δε προς εμέ το Πνεύμα να υπάγω μετ' αυτών, μηδόλως διστάζων. Ήλθον δε μετ' εμού και οι εξ ούτοι αδελφοί, και εισήλθομεν εις τον οίκον του ανθρώπου,
\par 13 και απήγγειλε προς ημάς πως είδε τον άγγελον εν τω οίκω αυτού, ότι εστάθη και είπε προς αυτόν· Απόστειλον ανθρώπους εις Ιόππην και προσκάλεσον τον Σίμωνα τον επονομαζόμενον Πέτρον,
\par 14 όστις θέλει λαλήσει προς σε λόγους, δι' ων θέλεις σωθή συ και πας ο οίκός σου.
\par 15 Και ενώ ήρχισα να λαλώ, το Πνεύμα το Άγιον επήλθεν επ' αυτούς καθώς και εφ' ημάς κατ' αρχάς.
\par 16 Τότε ενεθυμήθην τον λόγον του Κυρίου, ότι έλεγεν· Ιωάννης μεν εβάπτισεν εν ύδατι, σεις όμως θέλετε βαπτισθή εν Πνεύματι Αγίω.
\par 17 Εάν λοιπόν ο Θεός έδωκεν εις αυτούς την ίσην δωρεάν ως και εις ημάς, διότι επίστευσαν εις τον Κύριον Ιησούν Χριστόν, εγώ τις ήμην ώστε να δυνηθώ να εμποδίσω τον Θεόν;
\par 18 Ακούσαντες δε ταύτα ησύχασαν και εδόξαζον τον Θεόν, λέγοντες· Και εις τα έθνη λοιπόν έδωκεν ο Θεός την μετάνοιαν εις ζωήν.
\par 19 Οι μεν λοιπόν διασκορπισθέντες εκ του διωγμού του γενομένου διά τον Στέφανον, επέρασαν έως Φοινίκης και Κύπρου και Αντιοχείας, εις μηδένα κηρύττοντες τον λόγον, ειμή μόνον εις Ιουδαίους.
\par 20 Ήσαν δε τινές εξ αυτών άνδρες Κύπριοι και Κυρηναίοι, οίτινες εισελθόντες εις Αντιόχειαν, ελάλουν προς τους Ελληνιστάς, ευαγγελιζόμενοι τον Κύριον Ιησούν.
\par 21 Και ήτο χειρ Κυρίου μετ' αυτών, και πολύ πλήθος πιστεύσαντες επέστρεψαν εις τον Κύριον.
\par 22 Ηκούσθη δε ο λόγος περί αυτών εις τα ώτα της εκκλησίας της εν Ιεροσολύμοις, και εξαπέστειλαν τον Βαρνάβαν, διά να περάση έως Αντιοχείας·
\par 23 όστις ελθών και ιδών την χάριν του Θεού, εχάρη και παρεκίνει πάντας να εμμένωσιν εν σταθερότητι καρδίας εις τον Κύριον,
\par 24 επειδή ήτο ανήρ αγαθός και πλήρης Πνεύματος Αγίου και πίστεως· και προσετέθη εις τον Κύριον πλήθος ικανόν.
\par 25 Τότε εξήλθεν εις Ταρσόν ο Βαρνάβας, διά να αναζητήση τον Σαύλον,
\par 26 και ευρών αυτόν, έφερεν αυτόν εις Αντιόχειαν. Και συνελθόντες εις την εκκλησίαν εν ολόκληρον έτος εδίδαξαν πλήθος ικανόν, και πρώτον εν Αντιοχεία ωνομάσθησαν οι μαθηταί Χριστιανοί.
\par 27 Εν εκείναις δε ταις ημέραις κατέβησαν από Ιεροσολύμων προφήται εις Αντιόχειαν·
\par 28 σηκωθείς δε εις εξ αυτών ονόματι Άγαβος, εφανέρωσε διά του Πνεύματος ότι έμελλε να γείνη μεγάλη πείνα καθ' όλην την οικουμένην· ήτις και έγεινεν επί Κλαυδίου Καίσαρος.
\par 29 Όθεν οι μαθηταί απεφάσισαν, έκαστος αυτών κατά την εαυτού κατάστασιν, να πέμψωσι βοήθειαν προς τους αδελφούς τους κατοικούντας εν τη Ιουδαία·
\par 30 το οποίον και έκαμον αποστείλαντες αυτήν προς τους πρεσβυτέρους διά χειρός Βαρνάβα και Σαύλου.

\chapter{12}

\par 1 Κατ' εκείνον δε τον καιρόν επεχείρησεν Ηρώδης ο βασιλεύς να κακοποιήση τινάς από της εκκλησίας.
\par 2 Εφόνευσε δε διά μαχαίρας Ιάκωβον τον αδελφόν του Ιωάννου.
\par 3 Και ιδών ότι ήτο αρεστόν εις τους Ιουδαίους, προσέθεσε να συλλάβη και τον Πέτρον· ήσαν δε αι ημέραι των αζύμων·
\par 4 τον οποίον και πιάσας έβαλεν εις φυλακήν, παραδώσας αυτόν εις τέσσαρας τετράδας στρατιωτών διά να φυλάττωσιν αυτόν, θέλων μετά το πάσχα να παραστήση αυτόν εις τον λαόν.
\par 5 Ο μεν λοιπόν Πέτρος εφυλάττετο εν τη φυλακή· εγίνετο δε υπό της εκκλησίας ακατάπαυστος προσευχή προς τον Θεόν υπέρ αυτού.
\par 6 Ότε δε έμελλεν ο Ηρώδης να παραστήση αυτόν, την νύκτα εκείνην ο Πέτρος εκοιμάτο μεταξύ δύο στρατιωτών δεδεμένος με δύο αλύσεις, και φύλακες έμπροσθεν της θύρας εφύλαττον το δεσμωτήριον.
\par 7 Και ιδού, άγγελος Κυρίου ήλθεν εξαίφνης και φως έλαμψεν εν τω οικήματι· κτυπήσας δε την πλευράν του Πέτρου εξύπνησεν αυτόν, λέγων· Σηκώθητι ταχέως. Και έπεσον αι αλύσεις αυτού εκ των χειρών.
\par 8 Και είπεν ο άγγελος προς αυτόν· Περιζώσθητι και υπόδησον τα σανδάλια σου. Και έκαμεν ούτω. Και λέγει προς αυτόν· Φόρεσον το ιμάτιόν σου και ακολούθει μοι.
\par 9 Και εξελθών ηκολούθει αυτόν, και δεν ήξευρεν ότι το γινόμενον διά του αγγέλου ήτο αληθινόν, αλλ' ενόμιζεν ότι βλέπει όραμα.
\par 10 Αφού δε επέρασαν πρώτην και δευτέραν φρουράν, ήλθον εις την πύλην την σιδηράν την φέρουσαν εις την πόλιν, ήτις αφ' εαυτής ηνοίχθη εις αυτούς, και εξελθόντες διεπέρασαν οδόν μίαν, και ευθύς ο άγγελος ανεχώρησεν απ' αυτού.
\par 11 Και ο Πέτρος συνελθών εις εαυτόν, είπε· Τώρα γνωρίζω αληθώς ότι Κύριος εξαπέστειλε τον άγγελον αυτού και με ηλευθέρωσεν εκ της χειρός του Ηρώδου και όλης της ελπίδος του λαού των Ιουδαίων.
\par 12 Και αφού εσκέφθη, ήλθεν εις την οικίαν Μαρίας της μητρός του Ιωάννου του επονομαζομένου Μάρκου, όπου ήσαν ικανοί συνηθροισμένοι και προσευχόμενοι.
\par 13 Ότε δε ο Πέτρος έκρουσε την θύραν του προαυλίου, προσήλθε θεράπαινα ονομαζομένη Ρόδη, διά να ακούση,
\par 14 και γνωρίσασα την φωνήν του Πέτρου από της χαράς δεν ήνοιξε την πύλην, αλλ' έτρεξε και απήγγειλεν ότι ο Πέτρος ίσταται έμπροσθεν της πύλης.
\par 15 Οι δε είπον προς αυτήν· Παραφρονείς. Εκείνη όμως διϊσχυρίζετο ότι ούτως έχει. Οι δε έλεγον· Ο άγγελος αυτού είναι.
\par 16 Ο δε Πέτρος επέμενε κρούων. Και ανοίξαντες είδον αυτόν και εξεπλάγησαν.
\par 17 Και σείσας εις αυτούς την χείρα διά να σιωπήσωσι, διηγήθη προς αυτούς πως ο Κύριος εξήγαγεν αυτόν εκ της φυλακής, και είπεν· Απαγγείλατε ταύτα προς τον Ιάκωβον και τους αδελφούς. Και εξελθών υπήγεν εις άλλον τόπον.
\par 18 Αφού δε εξημέρωσεν, ήτο ταραχή ουκ ολίγη μεταξύ των στρατιωτών τι άρα έγεινεν ο Πέτρος.
\par 19 Ο δε Ηρώδης, αφού εζήτησεν αυτόν και δεν εύρεν, ανακρίνας τους φύλακας προσέταξε να θανατωθώσι, και καταβάς από της Ιουδαίας εις την Καισάρειαν, διέτριβεν εκεί.
\par 20 Ήτο δε ο Ηρώδης σφόδρα ωργισμένος κατά των Τυρίων και Σιδωνίων· ήλθον δε προς αυτόν ομοθυμαδόν, και πείσαντες τον Βλάστον τον επί του κοιτώνος του βασιλέως, εζήτουν ειρήνην, διότι ο τόπος αυτών ετρέφετο από του βασιλικού.
\par 21 Και εν ημέρα ωρισμένη ενδυθείς ο Ηρώδης βασιλικήν στολήν και καθήσας επί του θρόνου, εδημηγόρει προς αυτούς.
\par 22 Ο δε λαός επεφώνει· Θεού φωνή και ουχί ανθρώπου.
\par 23 Και παρευθύς επάταξεν αυτόν άγγελος Κυρίου, διότι δεν έδωκε την δόξαν εις τον Θεόν, και γενόμενος σκωληκόβρωτος εξεψύχησεν.
\par 24 Ο δε λόγος του Θεού ηύξανε και επληθύνετο.
\par 25 Ο δε Βαρνάβας και ο Σαύλος υπέστρεψαν εξ Ιερουσαλήμ αφού εξεπλήρωσαν την διακονίαν αυτών, παραλαβόντες μεθ' εαυτών και τον Ιωάννην τον επονομασθέντα Μάρκον.

\chapter{13}

\par 1 Ήσαν δε εν Αντιοχεία εν τη υπαρχούση εκκλησία προφήταί τινές και διδάσκαλοι, ο Βαρνάβας και Συμεών ο καλούμενος Νίγερ, και Λούκιος ο Κυρηναίος, και Μαναήν ο συνανατραφείς μετά του Ηρώδου του τετράρχου, και ο Σαύλος.
\par 2 Και ενώ υπηρέτουν εις τον Κύριον και ενήστευον, είπε το Πνεύμα το Αγιον· Χωρίσατε εις εμέ τον Βαρνάβαν και τον Σαύλον διά το έργον, εις το οποίον προσεκάλεσα αυτούς.
\par 3 Τότε αφού ενήστευσαν και προσευχήθησαν και επέθεσαν τας χείρας επ' αυτούς, απέστειλαν.
\par 4 Ούτοι λοιπόν πεμφθέντες υπό του Πνεύματος του Αγίου, κατέβησαν εις την Σελεύκειαν και εκείθεν απέπλευσαν εις την Κύπρον,
\par 5 και ότε ήλθον εις την Σαλαμίνα, εκήρυττον τον λόγον του Θεού εν ταις συναγωγαίς των Ιουδαίων· είχον δε και τον Ιωάννην υπηρέτην.
\par 6 Και αφού διήλθον την νήσον μέχρι της Πάφου, εύρον τινά μάγον ψευδοπροφήτην Ιουδαίον ονομαζόμενον Βαριησούν,
\par 7 όστις ήτο μετά του ανθυπάτου Σεργίου Παύλου, ανδρός συνετού. Ούτος προσκαλέσας τον Βαρνάβαν και Σαύλον, εζήτησε να ακούση τον λόγον του Θεού·
\par 8 ανθίστατο δε εις αυτούς Ελύμας ο μάγος, διότι ούτω μεθερμηνεύεται το όνομα αυτού, ζητών να αποτρέψη τον ανθύπατον από της πίστεως.
\par 9 Πλην ο Σαύλος, ο και Παύλος, πλησθείς Πνεύματος Αγίου και ατενίσας εις αυτόν,
\par 10 είπεν· Ω πλήρης παντός δόλου και πάσης ραδιουργίας, υιέ του διαβόλου, εχθρέ πάσης δικαιοσύνης, δεν θέλεις παύσει διαστρέφων τας ευθείας οδούς του Κυρίου;
\par 11 Και τώρα ιδού, χειρ του Κυρίου είναι κατά σου, και θέλεις είσθαι τυφλός, μη βλέπων τον ήλιον μέχρι καιρού. Και παρευθύς επέπεσεν επ' αυτόν αμαύρωσις και σκότος, και περιστρεφόμενος εζήτει χειραγωγούς.
\par 12 Τότε ιδών ο ανθύπατος το γεγονός επίστευσεν, εκπληττόμενος εις την διδαχήν του Κυρίου.
\par 13 Αποπλεύσαντες δε από της Πάφου ο Παύλος και οι περί αυτόν ήλθον εις την Πέργην της Παμφυλίας· ο δε Ιωάννης, χωρισθείς απ' αυτών, υπέστρεψεν εις τα Ιεροσόλυμα.
\par 14 Αυτοί δε περάσαντες από της Πέργης, έφθασαν εις Αντιόχειαν της Πισιδίας, και εισελθόντες εις την συναγωγήν τη ημέρα του σαββάτου εκάθησαν.
\par 15 Και μετά την ανάγνωσιν του νόμου και των προφητών απέστειλαν εις αυτούς οι αρχισυνάγωγοι, λέγοντες· Άνδρες αδελφοί, εάν έχητε λόγον τινά προτροπής εις τον λαόν, λέγετε.
\par 16 Σηκωθείς δε ο Παύλος και σείσας την χείρα, είπεν· Άνδρες Ισραηλίται και οι φοβούμενοι τον Θεόν, ακούσατε.
\par 17 Ο Θεός του λαού τούτου Ισραήλ εξέλεξε τους πατέρας ημών και ύψωσε τον λαόν παροικούντα εν γη Αιγύπτου, και μετά βραχίονος υψηλού εξήγαγεν αυτούς εξ αυτής,
\par 18 και έως τεσσαράκοντα έτη υπέφερε τους τρόπους αυτών εν τη ερήμω,
\par 19 και αφού κατέστρεψεν επτά έθνη εν γη Χαναάν, διεμέρισεν εις αυτούς κατά κλήρον την γην αυτών.
\par 20 Και μετά ταύτα ως τετρακόσια και πεντήκοντα περίπου έτη έδωκεν εις αυτούς κριτάς έως Σαμουήλ του προφήτου.
\par 21 Και έπειτα εζήτησαν βασιλέα, και έδωκεν εις αυτούς ο Θεός τον Σαούλ, υιόν του Κις, άνδρα εκ της φυλής Βενιαμίν, τεσσαράκοντα έτη·
\par 22 και μεταστήσας αυτόν, ανέστησεν εις αυτούς βασιλέα τον Δαβίδ, περί του οποίου και είπε μαρτυρήσας· Εύρον Δαβίδ τον του Ιεσσαί, άνδρα κατά την καρδίαν μου, όστις θέλει κάμει πάντα τα θελήματά μου.
\par 23 Από του σπέρματος τούτου ο Θεός κατά την επαγγελίαν αυτού ανέστησεν εις τον Ισραήλ σωτήρα τον Ιησούν,
\par 24 αφού ο Ιωάννης προ της ελεύσεως αυτού προεκήρυξε βάπτισμα μετανοίας εις πάντα τον λαόν του Ισραήλ.
\par 25 Και ενώ ο Ιωάννης ετελείονε τον δρόμον αυτού, έλεγε· Τίνα με στοχάζεσθε ότι είμαι; δεν είμαι εγώ, αλλ' ιδού, έρχεται μετ' εμέ εκείνος, του οποίου δεν είμαι άξιος να λύσω το υπόδημα των ποδών.
\par 26 Άνδρες αδελφοί, υιοί του γένους του Αβραάμ και οι εν υμίν φοβούμενοι τον Θεόν, προς εσάς απεστάλη ο λόγος της σωτηρίας ταύτης.
\par 27 Διότι οι κατοικούντες εν Ιερουσαλήμ και οι άρχοντες αυτών, μη γνωρίσαντες τούτον μηδέ τας ρήσεις των προφητών, τας αναγινωσκομένας κατά παν σάββατον, επλήρωσαν αυτάς κρίναντες τούτον,
\par 28 και μη ευρόντες μηδεμίαν αιτίαν θανάτου, εζήτησαν παρά του Πιλάτου να θανατωθή.
\par 29 Αφού δε ετελείωσαν πάντα τα περί αυτού γεγραμμένα, καταβιβάσαντες αυτόν από του ξύλου έθεσαν εις μνημείον.
\par 30 Ο Θεός όμως ανέστησεν αυτόν εκ νεκρών·
\par 31 όστις εφάνη επί πολλάς ημέρας εις τους μετ' αυτού αναβάντας από της Γαλιλαίας εις Ιερουσαλήμ, οίτινες είναι μάρτυρες αυτού προς τον λαόν.
\par 32 Και ημείς ευαγγελιζόμεθα προς εσάς την γενομένην εις τους πατέρας επαγγελίαν,
\par 33 ότι ταύτην ο Θεός εξεπλήρωσεν εις ημάς τα τέκνα αυτών, αναστήσας τον Ιησούν, ως είναι γεγραμμένον και εν τω ψαλμώ τω δευτέρω· Υιός μου είσαι συ, εγώ σήμερον σε εγέννησα.
\par 34 Ότι δε ανέστησεν αυτόν εκ νεκρών, μη μέλλοντα πλέον να υποστρέψη εις την διαφθοράν, λέγει ούτως, ότι θέλω σας δώσει τα ελέη του Δαβίδ τα πιστά.
\par 35 Διά τούτο και εν άλλω ψαλμώ λέγει· Δεν θέλεις αφήσει τον όσιόν σου να ίδη διαφθοράν.
\par 36 Διότι ο μεν Δαβίδ, αφού υπηρέτησε την βουλήν του Θεού εν τη γενεά αυτού, εκοιμήθη και προσετέθη εις τους πατέρας αυτού και είδε διαφθοράν·
\par 37 εκείνος όμως, τον οποίον ο Θεός ανέστησε, δεν είδε διαφθοράν.
\par 38 Έστω λοιπόν γνωστόν εις εσάς, άνδρες αδελφοί, ότι διά τούτου κηρύττεται προς εσάς άφεσις αμαρτιών.
\par 39 Και από πάντων, αφ' όσων δεν ηδυνήθητε διά του νόμου του Μωϋσέως να δικαιωθήτε, διά τούτου πας ο πιστεύων δικαιούται.
\par 40 Βλέπετε λοιπόν μη επέλθη εφ' υμάς το λαληθέν υπό των προφητών·
\par 41 Ίδετε, οι καταφρονηταί, και θαυμάσατε και αφανίσθητε, διότι έργον εγώ εργάζομαι εν ταις ημέραις υμών, έργον, εις το οποίον δεν θέλετε πιστεύσει, εάν τις διηγηθή εις εσάς.
\par 42 Ενώ δε εξήρχοντο εκ της συναγωγής των Ιουδαίων, παρεκάλουν τα έθνη να κηρυχθώσιν εις αυτούς οι λόγοι ούτοι το ακόλουθον σάββατον.
\par 43 Και αφού ελύθη η συναγωγή, πολλοί εκ των Ιουδαίων και των ευσεβών προσηλύτων ηκολούθησαν τον Παύλον και τον Βαρνάβαν, οίτινες λαλούντες προς αυτούς, έπειθον αυτούς να εμμένωσιν εις την χάριν του Θεού.
\par 44 το δε ερχόμενον σάββατον σχεδόν όλη η πόλις συνήχθη διά να ακούσωσι τον λόγον του Θεού.
\par 45 Ιδόντες δε οι Ιουδαίοι τα πλήθη, επλήσθησαν φθόνου και ηναντιούντο εις τα υπό του Παύλου λεγόμενα, αντιλέγοντες και βλασφημούντες.
\par 46 Ο Παύλος δε και ο Βαρνάβας, λαλούντες μετά παρρησίας, είπον· Εις εσάς πρώτον ήτο αναγκαίον να λαληθή ο λόγος του Θεού· αλλ' επειδή απορρίπτετε αυτόν και δεν κρίνετε εαυτούς αξίους της αιωνίου ζωής, ιδού, στρεφόμεθα εις τα έθνη·
\par 47 διότι ούτω προσέταξεν ημάς ο Κύριος, λέγων· Σε έθεσα φως των εθνών, διά να ήσαι προς σωτηρίαν έως εσχάτου της γης.
\par 48 Και οι εθνικοί ακούσαντες έχαιρον και εδόξαζον τον λόγον του Κυρίου, και επίστευσαν όσοι ήσαν ωρισμένοι διά την αιώνιον ζωήν·
\par 49 και ο λόγος του Κυρίου διεδίδετο δι' όλου του τόπου.
\par 50 Οι δε Ιουδαίοι παρεκίνησαν τας ευλαβείς και επισήμους γυναίκας και τους πρώτους της πόλεως και διήγειραν διωγμόν κατά του Παύλου και του Βαρνάβα, και εξέβαλον αυτούς από των ορίων αυτών.
\par 51 Εκείνοι δε εκτινάξαντες τον κονιορτόν των ποδών αυτών επ' αυτούς, ήλθον εις το Ικόνιον.
\par 52 Και οι μαθηταί επληρούντο χαράς και Πνεύματος Αγίου.

\chapter{14}

\par 1 Εν δε τω Ικονίω εισελθόντες ομού εις την συναγωγήν των Ιουδαίων, ελάλησαν ούτως ώστε επίστευσε πολύ πλήθος Ιουδαίων τε και Ελλήνων.
\par 2 Όσοι δε Ιουδαίοι δεν επείθοντο παρώξυναν και διέστρεψαν τας ψυχάς των εθνικών κατά των αδελφών.
\par 3 Ικανόν λοιπόν καιρόν διέτριψαν λαλούντες μετά παρρησίας περί του Κυρίου, όστις εμαρτύρει εις τον λόγον της χάριτος αυτού, και έδιδε να γίνωνται σημεία και τέρατα διά των χειρών αυτών.
\par 4 Εσχίσθη δε το πλήθος της πόλεως, και οι μεν ήσαν μετά των Ιουδαίων, οι δε μετά των αποστόλων.
\par 5 Και ότε ώρμησαν οι εθνικοί και οι Ιουδαίοι μετά των αρχόντων αυτών εις το να υβρίσωσι και να λιθοβολήσωσιν αυτούς,
\par 6 εννοήσαντες κατέφυγον εις τας πόλεις της Λυκαονίας Λύστραν και Δέρβην και τα περίχωρα,
\par 7 και εκεί εκήρυττον το ευαγγέλιον.
\par 8 Εν δε τοις Λύστροις εκάθητο ανήρ τις αδύνατος τους πόδας, χωλός υπάρχων εκ κοιλίας μητρός αυτού, όστις ποτέ δεν είχε περιπατήσει.
\par 9 Ούτος ήκουε τον Παύλον λαλούντα· όστις ατενίσας εις αυτόν και ιδών ότι έχει πίστιν διά να σωθή,
\par 10 είπε μετά μεγάλης φωνής· Σηκώθητι επί τους πόδας σου ορθός. Και επήδα και περιεπάτει.
\par 11 Οι δε όχλοι, ιδόντες τούτο το οποίον έκαμεν ο Παύλος, ύψωσαν την φωνήν αυτών, λέγοντες Λυκαονιστί· Οι θεοί ομοιωθέντες με ανθρώπους κατέβησαν προς ημάς.
\par 12 Και ωνόμαζον τον μεν Βαρνάβαν Δία, τον δε Παύλον Ερμήν, επειδή αυτός ήτο ο αρχηγός του λόγου.
\par 13 Και ο ιερεύς του Διός, του όντος έμπροσθεν της πόλεως αυτών, έφερε ταύρους και στέμματα εις τας πύλας μετά του όχλου και ήθελε να προσφέρη θυσίαν.
\par 14 Ακούσαντες δε οι απόστολοι Βαρνάβας και Παύλος, διέσχισαν τα ιμάτια αυτών και επήδησαν εις το μέσον του όχλου, κράζοντες
\par 15 και λέγοντες· Άνδρες, τι κάμνετε ταύτα; και ημείς είμεθα άνθρωποι ομοιοπαθείς με σας, κηρύττοντες προς εσάς να επιστρέψητε από τούτων των ματαίων προς τον Θεόν τον ζώντα, όστις έκαμε τον ουρανόν και την γην και την θάλασσαν και πάντα τα εν αυτοίς·
\par 16 όστις εν ταις παρελθούσαις γενεαίς αφήκε πάντα τα έθνη να περιπατώσιν εν ταις οδοίς αυτών.
\par 17 καίτοι δεν αφήκεν αμαρτύρητον εαυτόν αγαθαποιών, δίδων εις ημάς ουρανόθεν βροχάς και καιρούς καρποφόρους, γεμίζων τροφής και ευφροσύνης τας καρδίας ημών.
\par 18 Και ταύτα λέγοντες μόλις εμπόδισαν τους όχλους, ώστε να μη προσφέρωσι θυσίαν εις αυτούς.
\par 19 Εν τούτω δε ήλθον Ιουδαίοι εξ Αντιοχείας και Ικονίου, και πείσαντες τους όχλους και λιθοβολήσαντες τον Παύλον, έσυραν έξω της πόλεως, νομίσαντες ότι απέθανεν.
\par 20 Ότε δε περιεκύκλωσαν αυτόν οι μαθηταί, σηκωθείς εισήλθεν εις την πόλιν και τη επαύριον εξήλθε μετά του Βαρνάβα εις Δέρβην.
\par 21 Και αφού εκήρυξαν το ευαγγέλιον εν τη πόλει εκείνη και εμαθήτευσαν ικανούς, υπέστρεψαν εις την Λύστραν και Ικόνιον και Αντιόχειαν,
\par 22 επιστηρίζοντες τας ψυχάς των μαθητών, προτρέποντες να εμμένωσιν εις την πίστιν, και διδάσκοντες ότι διά πολλών θλίψεων πρέπει να εισέλθωμεν εις την βασιλείαν του Θεού.
\par 23 Και αφού εχειροτόνησαν εις αυτούς πρεσβυτέρους κατά πάσαν εκκλησίαν, προσευχηθέντες με νηστείας, αφιέρωσαν αυτούς εις τον Κύριον, εις τον οποίον είχον πιστεύσει.
\par 24 Και διελθόντες την Πισιδίαν ήλθον εις Παμφυλίαν,
\par 25 και κηρύξαντες τον λόγον εν Πέργη, κατέβησαν εις Αττάλειαν,
\par 26 και εκείθεν απέπλευσαν εις Αντιόχειαν, όθεν ήσαν παραδεδομένοι εις την χάριν του Θεού διά το έργον, το οποίον εξετέλεσαν.
\par 27 Ελθόντες δε και συνάξαντες την εκκλησίαν, ανήγγειλαν όσα έκαμεν ο Θεός δι' αυτών, και ότι ήνοιξεν εις τα έθνη θύραν πίστεως.
\par 28 Και διέτριβον εκεί ουκ ολίγον καιρόν μετά των μαθητών.

\chapter{15}

\par 1 Και τινές κατελθόντες από της Ιουδαίας εδίδασκον τους αδελφούς, ότι εάν δεν περιτέμνησθε κατά το έθος του Μωϋσέως, δεν δύνασθε να σωθήτε.
\par 2 Γενομένης λοιπόν αντιστάσεως και συζητήσεως ουκ ολίγης υπό του Παύλου και Βαρνάβα προς αυτούς, ενέκριναν να αναβή ο Παύλος και ο Βαρνάβας και τινές άλλοι εξ αυτών προς τους αποστόλους και πρεσβυτέρους εις Ιερουσαλήμ περί του ζητήματος τούτου.
\par 3 Εκείνοι λοιπόν προπεμφθέντες υπό της εκκλησίας, διήρχοντο την Φοινίκην και Σαμάρειαν, εκδιηγούμενοι την επιστροφήν των εθνών, και επροξένουν χαράν μεγάλην εις πάντας τους αδελφούς.
\par 4 Ότε δε ήλθον εις Ιερουσαλήμ, υπεδέχθησαν υπό της εκκλησίας και των αποστόλων και των πρεσβυτέρων, και ανήγγειλαν όσα ο Θεός έκαμε δι' αυτών.
\par 5 Εσηκώθησαν δε τινές των από της αιρέσεως των Φαρισαίων, οίτινες είχον πιστεύσει, και έλεγον ότι πρέπει να περιτέμνωμεν αυτούς και να παραγγέλλωμεν να φυλάττωσι τον νόμον του Μωϋσέως.
\par 6 Και συνήχθησαν οι απόστολοι και οι πρεσβύτεροι, διά να σκεφθώσι περί του πράγματος τούτου.
\par 7 Μετά δε πολλήν συζήτησιν σηκωθείς ο Πέτρος, είπε προς αυτούς· Άνδρες αδελφοί, σεις εξεύρετε ότι απ' αρχής ο Θεός εξέλεξε μεταξύ ημών διά του στόματός μου να ακούσωσι τα έθνη τον λόγον του ευαγγελίου και να πιστεύσωσι.
\par 8 Και ο καρδιογνώστης Θεός έδωκεν εις αυτούς μαρτυρίαν, χαρίσας εις αυτούς το Πνεύμα το Άγιον καθώς και εις ημάς,
\par 9 και δεν έκαμεν ουδεμίαν διάκρισιν μεταξύ ημών και αυτών, καθαρίσας τας καρδίας αυτών διά της πίστεως.
\par 10 Τώρα λοιπόν διά τι πειράζετε τον Θεόν, επιβάλλοντες ζυγόν εις τον τράχηλον των μαθητών, τον οποίον ούτε οι πατέρες ημών ούτε ημείς δεν ηδυνήθημεν να βαστάσωμεν;
\par 11 Αλλά διά της χάριτος του Κυρίου Ιησού Χριστού πιστεύομεν ότι θέλομεν σωθή καθ' ον τρόπον και εκείνοι.
\par 12 Εσιώπησε δε παν το πλήθος και ήκουον τον Βαρνάβαν και τον Παύλον εξιστορούντας όσα σημεία και τέρατα έκαμεν ο Θεός δι' αυτών μεταξύ των εθνών.
\par 13 Και αφού αυτοί εσιώπησαν, απεκρίθη ο Ιάκωβος, λέγων· Άνδρες αδελφοί, ακούσατέ μου.
\par 14 Ο Συμεών εφανέρωσε τίνι τρόπω κατ' αρχάς ο Θεός επεσκέφθη τα έθνη ώστε να λάβη εξ αυτών λαόν διά το όνομα αυτού.
\par 15 Και με τούτο συμφωνούσιν οι λόγοι των προφητών, καθώς είναι γεγραμμένον·
\par 16 Μετά ταύτα θέλω επιστρέψει και θέλω ανοικοδομήσει την σκηνήν του Δαβίδ την πεπτωκυίαν, και τα κατηδαφισμένα αυτής θέλω ανοικοδομήσει και θέλω ανορθώσει αυτήν,
\par 17 διά να εκζητήσωσι τον Κύριον οι λοιποί των ανθρώπων, και πάντα τα έθνη, επί τα οποία καλείται το όνομά μου, λέγει Κύριος ο ποιών ταύτα πάντα.
\par 18 Απ' αιώνος είναι γνωστά εις τον Θεόν πάντα τα έργα αυτού.
\par 19 Όθεν εγώ κρίνω να μη παρενοχλώμεν τους από των εθνών επιστρέφοντας εις τον Θεόν,
\par 20 αλλά να γράφωμεν προς αυτούς να απέχωσιν από των μιασμάτων των ειδώλων και από της πορνείας και του πνικτού και του αίματος.
\par 21 Διότι ο Μωϋσής από γενεάς αρχαίας έχει εν πάση πόλει τους κηρύττοντας αυτόν εν ταις συναγωγαίς, αναγινωσκόμενος κατά παν σάββατον.
\par 22 Τότε εφάνη εύλογον εις τους αποστόλους και εις τους πρεσβυτέρους μεθ' όλης της εκκλησίας να εκλέξωσιν εξ αυτών άνδρας και να πέμψωσιν εις Αντιόχειαν μετά του Παύλου και Βαρνάβα, Ιούδαν τον επονομαζόμενον Βαρσαβάν και Σίλαν, άνδρας προεστώτας μεταξύ των αδελφών,
\par 23 και έγραψαν διά χειρός αυτών ταύτα· Οι απόστολοι και οι πρεσβύτεροι και οι αδελφοί προς τους εξ εθνών αδελφούς τους κατά την Αντιόχειαν και Συρίαν και Κιλικίαν, χαίρειν.
\par 24 Επειδή ηκούσαμεν ότι τινές εξ ημών εξελθόντες σας ετάραξαν με λόγους και διαστρέφουσι τας ψυχάς σας, λέγοντες να περιτέμνησθε και να φυλάττητε τον νόμον, εις τους οποίους ημείς δεν παρηγγείλαμεν τούτο,
\par 25 εφάνη εύλογον εις ημάς, συνελθόντας ομοθυμαδόν, να εκλέξωμεν άνδρας και να πέμψωμεν προς εσάς μετά των αγαπητών ημών Βαρνάβα και Παύλου,
\par 26 ανθρώπων οίτινες παρέδωκαν τας ψυχάς αυτών υπέρ του ονόματος του Κυρίου ημών Ιησού Χριστού.
\par 27 Απεστείλαμεν λοιπόν τον Ιούδαν και τον Σίλαν διά να σας απαγγείλωσι και αυτοί διά στόματος τα αυτά.
\par 28 Διότι εφάνη εύλογον εις το Άγιον Πνεύμα και εις ημάς να μη επιβάλλωμεν εις εσάς μηδέν πλειότερον βάρος εκτός των αναγκαίων τούτων,
\par 29 να απέχητε από ειδωλοθύτων και αίματος και πνικτού και πορνείας· από των οποίων φυλάττοντες εαυτούς θέλετε πράξει καλώς. Έρρωσθε.
\par 30 Ούτοι μεν λοιπόν απολυθέντες ήλθον εις Αντιόχειαν, και συνάξαντες το πλήθος ενεχείρησαν την επιστολήν.
\par 31 Αναγνώσαντες δε αυτήν, εχάρησαν διά την γενομένην παρηγορίαν.
\par 32 Ο Ιούδας δε και ο Σίλας, όντες και αυτοί προφήται, παρηγόρησαν τους αδελφούς διά λόγων πολλών και επεστήριξαν αυτούς.
\par 33 Και αφού διέτριψαν εκεί καιρόν τινά, απεστάλησαν εν ειρήνη από των αδελφών προς τους αποστόλους.
\par 34 Εις τον Σίλαν όμως εφάνη εύλογον να μείνη έτι αυτού.
\par 35 Ο δε Παύλος και Βαρνάβας διέτριβον εν Αντιοχεία, διδάσκοντες και κηρύττοντες μετά και άλλων πολλών τον λόγον του Κυρίου.
\par 36 Μετά δε τινάς ημέρας είπεν ο Παύλος προς τον Βαρνάβαν· Ας επιστρέψωμεν τώρα και ας επισκεφθώμεν τους αδελφούς ημών κατά πάσαν πόλιν, εν αις εκηρύξαμεν τον λόγον του Κυρίου, πως έχουσι.
\par 37 Και ο μεν Βαρνάβας εστοχάσθη να συμπαραλάβη τον Ιωάννην τον λεγόμενον Μάρκον·
\par 38 ο Παύλος όμως έκρινεν άξιον, τον αποχωρισθέντα από αυτών από της Παμφυλίας και μη συνακολουθήσαντα αυτούς εις το έργον, τούτον να μη συμπαραλάβωσι.
\par 39 Συνέβη λοιπόν ερεθισμός, ώστε απεχωρίσθησαν απ' αλλήλων, και ο μεν Βαρνάβας, παραλαβών τον Μάρκον, εξέπλευσεν εις Κύπρον.
\par 40 Ο δε Παύλος, εκλέξας τον Σίλαν, εξήλθε, παραδοθείς υπό των αδελφών εις την χάριν του Θεού.
\par 41 Και διήρχετο την Συρίαν και Κιλικίαν, επιστηρίζων τας εκκλησίας.

\chapter{16}

\par 1 Κατήντησε δε εις Δέρβην και Λύστραν. Και ιδού, ήτο εκεί μαθητής τις ονόματι Τιμόθεος, υιός γυναικός τινός Ιουδαίας πιστής, πατρός δε Έλληνος,
\par 2 όστις είχε καλήν μαρτυρίαν υπό των εν Λύστροις και Ικονίω αδελφών.
\par 3 Τούτον ηθέλησεν ο Παύλος να εξέλθη μεθ' εαυτού, και λαβών αυτόν περιέτεμε διά τους Ιουδαίους τους όντας εν τοις τόποις εκείνοις· επειδή εγνώριζον πάντες τον πατέρα αυτού ότι ήτο Έλλην.
\par 4 Ως δε διήρχοντο τας πόλεις, παρέδιδον εις αυτούς διαταγάς να φυλάττωσι τα δόγματα τα εγκεκριμένα υπό των αποστόλων και των πρεσβυτέρων των εν Ιερουσαλήμ.
\par 5 Αι μεν λοιπόν εκκλησίαι εστερεούντο εις την πίστιν και ηυξάνοντο τον αριθμόν καθ' ημέραν.
\par 6 Διελθόντες δε την Φρυγίαν και την γην της Γαλατίας, επειδή εμποδίσθησαν υπό του Αγίου Πνεύματος να κηρύξωσι τον λόγον εν τη Ασία,
\par 7 ήλθον κατά την Μυσίαν και εδοκίμαζον να υπάγωσι προς την Βιθυνίαν· πλην δεν αφήκεν αυτούς το Πνεύμα.
\par 8 Περάσαντες δε την Μυσίαν κατέβησαν εις Τρωάδα.
\par 9 Και όραμα εφάνη διά νυκτός εις τον Παύλον. Ανήρ τις Μακεδών ίστατο, παρακαλών αυτόν και λέγων· Διάβα εις Μακεδονίαν και βοήθησον ημάς.
\par 10 Και ως είδε το όραμα, ευθύς εζητήσαμεν να υπάγωμεν εις την Μακεδονίαν, συμπεραίνοντες ότι ο Κύριος προσκαλεί ημάς, διά να κηρύξωμεν το ευαγγέλιον προς αυτούς.
\par 11 Αποπλεύσαντες λοιπόν από της Τρωάδος, επεράσαμεν κατ' ευθείαν εις Σαμοθράκην και την ακόλουθον ημέραν εις Νεάπολιν
\par 12 και εκείθεν εις Φιλίππους, ήτις είναι πρώτη πόλις του μέρους εκείνου της Μακεδονίας, αποικία Ρωμαϊκή. Και διετρίβομεν εν τη πόλει ταύτη ημέρας τινάς·
\par 13 και τη ημέρα του σαββάτου εξήλθομεν έξω της πόλεως πλησίον του ποταμού, όπου εσυνειθίζετο να γίνηται προσευχή, και καθήσαντες ελαλούμεν προς τας εκεί συνελθούσας γυναίκας.
\par 14 Και γυνή τις Λυδία το όνομα, πωλήτρια πορφύρας εκ πόλεως Θυατείρων, σεβομένη τον Θεόν, ήκουε, της οποίας ο Κύριος διήνοιξε την καρδίαν διά να προσέχη εις τα λαλούμενα υπό του Παύλου.
\par 15 Αφού δε εβαπτίσθη αυτή και ο οίκος αυτής, παρεκάλεσε λέγουσα· Εάν με εκρίνατε ότι είμαι πιστή εις τον Κύριον, εισέλθετε εις τον οίκόν μου και μείνατε· και μας εβίασεν.
\par 16 Ενώ δε επορευόμεθα εις την προσευχήν, απήντησεν ημάς δούλη τις έχουσα πνεύμα πύθωνος, ήτις έδιδε πολύ κέρδος εις τους κυρίους αυτής μαντευομένη.
\par 17 Αύτη ακολουθήσασα τον Παύλον και ημάς έκραζε, λέγουσα· Ούτοι οι άνθρωποι είναι δούλοι του Θεού του Υψίστου, οίτινες κηρύττουσι προς ημάς οδόν σωτηρίας.
\par 18 Τούτο δε έκαμνεν επί πολλάς ημέρας. Βαρυνθείς δε ο Παύλος και στραφείς, είπε προς το πνεύμα, Προστάζω σε εν τω ονόματι του Ιησού Χριστού να εξέλθης απ' αυτής. Και εξήλθε την αυτήν ώραν.
\par 19 Ιδόντες δε οι κύριοι αυτής ότι εξήλθεν η ελπίς του κέρδους αυτών, πιάσαντες τον Παύλον και τον Σίλαν, έσυραν εις την αγοράν προς τους άρχοντας,
\par 20 και φέροντες αυτούς προς τους στρατηγούς, είπον· Ούτοι οι άνθρωποι εκταράττουσι την πόλιν ημών, Ιουδαίοι όντες,
\par 21 και διδάσκουσιν έθιμα, τα οποία δεν είναι εις ημάς συγκεχωρημένον να παραδεχώμεθα μηδέ να πράττωμεν, Ρωμαίοι όντες.
\par 22 Και συνεφώρμησεν ο όχλος κατ' αυτών. Και οι στρατηγοί διασχίσαντες αυτών τα ιμάτια, προσέταττον να ραβδίζωσιν αυτούς,
\par 23 και αφού έδωκαν εις αυτούς πολλούς ραβδισμούς, έβαλον εις φυλακήν, παραγγείλαντες τον δεσμοφύλακα να φυλάττη αυτούς ασφαλώς·
\par 24 όστις λαβών τοιαύτην παραγγελίαν, έβαλεν αυτούς εις την εσωτέραν φυλακήν και συνέκλεισε τους πόδας αυτών εις το ξύλον.
\par 25 Κατά δε το μεσονύκτιον ο Παύλος και ο Σίλας προσευχόμενοι ύμνουν τον Θεόν· και ηκροάζοντο αυτούς οι δέσμιοι.
\par 26 Και εξαίφνης έγεινε σεισμός μέγας, ώστε εσαλεύθησαν τα θεμέλια του δεσμωτηρίου, και παρευθύς ηνοίχθησαν πάσαι αι θύραι και ελύθησαν πάντων τα δεσμά.
\par 27 Εξυπνήσας δε ο δεσμοφύλαξ και ιδών ανεωγμένας τας θύρας της φυλακής, έσυρε μάχαιραν και έμελλε να θανατώση εαυτόν, νομίζων ότι έφυγον οι δέσμιοι.
\par 28 Πλην ο Παύλος έκραξε μετά φωνής μεγάλης, λέγων· Μη πράξης μηδέν κακόν εις σεαυτόν· διότι πάντες είμεθα εδώ.
\par 29 Ζητήσας δε φώτα εισεπήδησε, και έντρομος γενόμενος έπεσεν έμπροσθεν του Παύλου και του Σίλα,
\par 30 και εκβαλών αυτούς έξω, είπε· Κύριοι, τι πρέπει να κάμω διά να σωθώ;
\par 31 Οι δε είπον· Πίστευσον εις τον Κύριον Ιησούν Χριστόν, και θέλεις σωθή, συ και ο οίκός σου.
\par 32 Και ελάλησαν προς αυτόν τον λόγον του Κυρίου και προς πάντας τους εν τη οικία αυτού.
\par 33 Και παραλαβών αυτούς εν εκείνη τη ώρα της νυκτός, έλουσε τας πληγάς αυτών και εβαπτίσθη ευθύς αυτός και πάντες οι αυτού,
\par 34 και αναβιβάσας αυτούς εις τον οίκον αυτού παρέθηκε τράπεζαν, και ευφράνθη πανοικί πιστεύσας εις το Θεόν.
\par 35 Αφού δε έγεινεν ημέρα, έστειλαν οι στρατηγοί τους ραβδούχους, λέγοντες· Απόλυσον τους ανθρώπους εκείνους.
\par 36 Και ο δεσμοφύλαξ απήγγειλε τους λόγους τούτους προς τον Παύλον, λέγων ότι οι στρατηγοί έστειλαν διά να απολυθήτε· τώρα λοιπόν εξέλθετε και υπάγετε εν ειρήνη.
\par 37 Αλλ' ο Παύλος είπε προς αυτούς· Αφού έδειραν ημάς δημοσία χωρίς να καταδικασθώμεν, ανθρώπους Ρωμαίους όντας, έβαλον εις φυλακήν· και τώρα μας εκβάλλουσι κρυφίως; ουχί βεβαίως, αλλ' αυτοί ας έλθωσι και ας μας εκβάλωσιν.
\par 38 Ανήγγειλαν δε προς τους στρατηγούς οι ραβδούχοι τους λόγους τούτους· και εφοβήθησαν ακούσαντες ότι είναι Ρωμαίοι,
\par 39 και ελθόντες παρεκάλεσαν αυτούς, και αφού εξέβαλον, παρεκάλουν αυτούς να εξέλθωσιν εκ της πόλεως.
\par 40 Οι δε εξελθόντες εκ της φυλακής, υπήγον εις τον οίκον της Λυδίας, και ιδόντες τους αδελφούς, παρηγόρησαν αυτούς και ανεχώρησαν.

\chapter{17}

\par 1 Διοδεύσαντες δε την Αμφίπολιν και Απολλωνίαν, ήλθαν εις Θεσσαλονίκην, όπου ήτο η συναγωγή των Ιουδαίων.
\par 2 Και κατά την συνήθειάν του ο Παύλος εισήλθε προς αυτούς, και τρία σάββατα διελέγετο μετ' αυτών από των γραφών,
\par 3 εξηγών και αποδεικνύων ότι έπρεπε να πάθη ο Χριστός και να αναστηθή εκ νεκρών και ότι ούτος είναι ο Χριστός Ιησούς, τον οποίον εγώ σας κηρύττω.
\par 4 Και τινές εξ αυτών επείσθησαν και ηνώθησαν μετά του Παύλου και του Σίλα, και εκ των θεοσεβών Ελλήνων πολύ πλήθος και εκ των πρώτων γυναικών ουκ ολίγαι.
\par 5 Φθονήσαντες δε οι μη πειθόμενοι Ιουδαίοι και λαβόντες μεθ' εαυτών κακούς τινάς ανθρώπους εκ των χυδαίων και οχλαγωγήσαντες, εθορύβουν την πόλιν και εφορμήσαντες εις την οικίαν του Ιάσονος, εζήτουν αυτούς διά να φέρωσιν εις τον δήμον·
\par 6 μη ευρόντες δε αυτούς, έσυραν τον Ιάσονα και τινάς αδελφούς επί τους πολιτάρχας, βοώντες ότι οι αναστατώσαντες την οικουμένην, ούτοι ήλθον και εδώ,
\par 7 τους οποίους υπεδέχθη ο Ιάσων· και πάντες ούτοι πράττουσιν εναντίον των προσταγμάτων του Καίσαρος, λέγοντες ότι είναι βασιλεύς άλλος, ο Ιησούς.
\par 8 Ετάραξαν δε τον όχλον και τους πολιτάρχας ακούοντας ταύτα,
\par 9 και λαβόντες εγγύησιν παρά του Ιάσονος και των λοιπών, απέλυσαν αυτούς.
\par 10 Οι δε αδελφοί ευθύς διά της νυκτός εξέπεμψαν τον τε Παύλον και τον Σίλαν εις Βέροιαν, οίτινες ελθόντες υπήγον εις την συναγωγήν των Ιουδαίων.
\par 11 Ούτοι δε ήσαν ευγενέστεροι παρά τους εν Θεσσαλονίκη, καθότι εδέχθησαν τον λόγον μετά πάσης προθυμίας, εξετάζοντες καθ' ημέραν τας γραφάς αν ούτως έχωσι ταύτα.
\par 12 Πολλοί μεν λοιπόν εξ αυτών επίστευσαν, και εκ των επισήμων Ελληνίδων γυναικών και εκ των ανδρών ουκ ολίγοι.
\par 13 Ως δε έμαθον οι από της Θεσσαλονίκης Ιουδαίοι ότι και εν τη Βεροία εκηρύχθη υπό του Παύλου ο λόγος του Θεού, ήλθον και εκεί και ετάραττον τους όχλους.
\par 14 Και ευθύς τότε οι αδελφοί εξαπέστειλαν τον Παύλον να υπάγη έως εις την θάλασσαν· ο Σίλας δε και ο Τιμόθεος έμειναν εκεί.
\par 15 Οι δε συνοδεύοντες τον Παύλον έφεραν αυτόν έως Αθηνών, και αφού έλαβον παραγγελίαν προς τον Σίλαν και Τιμόθεον να έλθωσι προς αυτόν όσον τάχιστα, ανεχώρησαν.
\par 16 Ενώ δε περιέμενεν αυτούς ο Παύλος εν ταις Αθήναις, το πνεύμα αυτού παρωξύνετο εν αυτώ, επειδή έβλεπε την πόλιν γέμουσαν ειδώλων.
\par 17 Διελέγετο λοιπόν εν τη συναγωγή μετά των Ιουδαίων και μετά των θεοσεβών και εν τη αγορά καθ' εκάστην ημέραν μετά των τυχόντων.
\par 18 Τινές δε των Επικουρίων και των Στωϊκών φιλοσόφων συνήρχοντο εις λόγους μετ' αυτού, και οι μεν έλεγον· Τι θέλει τάχα ο σπερμολόγος ούτος να είπη; οι δέ· Ξένων θεών κήρυξ φαίνεται ότι είναι· διότι εκήρυττε προς αυτούς τον Ιησούν και την ανάστασιν.
\par 19 Και πιάσαντες αυτόν έφεραν εις τον Άρειον Πάγον, λέγοντες· Δυνάμεθα να μάθωμεν τις αύτη η νέα διδαχή, ήτις κηρύττεται υπό σου;
\par 20 διότι φέρεις εις τας ακοάς ημών παράδοξά τινα· θέλομεν λοιπόν να μάθωμεν τι σημαίνουσι ταύτα.
\par 21 Πάντες δε οι Αθηναίοι και οι επιδημούντες ξένοι εις ουδέν άλλο ηυκαίρουν παρά εις το να λέγωσι και να ακούωσι τι νεώτερον.
\par 22 Σταθείς δε ο Παύλος εν μέσω του Αρείου Πάγου, είπεν· Άνδρες Αθηναίοι, κατά πάντα σας βλέπω εις άκρον θεολάτρας.
\par 23 Διότι ενώ διηρχόμην και ανεθεώρουν τα σεβάσματά σας, εύρον και βωμόν, εις τον οποίον είναι επιγεγραμμένον, Αγνώστω Θεώ. Εκείνον λοιπόν, τον οποίον αγνοούντες λατρεύετε, τούτον εγώ κηρύττω προς εσάς.
\par 24 Ο Θεός, όστις έκαμε τον κόσμον και πάντα τα εν αυτώ, ούτος Κύριος ων του ουρανού και της γης, δεν κατοικεί εν χειροποιήτοις ναοίς,
\par 25 ουδέ λατρεύεται υπό χειρών ανθρώπων ως έχων χρείαν τινός, επειδή αυτός δίδει εις πάντας ζωήν και πνοήν και τα πάντα·
\par 26 και έκαμεν εξ ενός αίματος παν έθνος ανθρώπων, διά να κατοικώσιν εφ' όλου του προσώπου της γης, και διώρισε τους προδιατεταγμένους καιρούς και τα οροθέσια της κατοικίας αυτών,
\par 27 διά να ζητώσι τον Κύριον, ίσως δυνηθώσι να ψηλαφήσωσιν αυτόν και να εύρωσιν, αν και δεν είναι μακράν από ενός εκάστου ημών.
\par 28 Διότι εν αυτώ ζώμεν και κινούμεθα και υπάρχομεν, καθώς και τινές των ποιητών σας είπον· Διότι και γένος είμεθα τούτου.
\par 29 Γένος λοιπόν όντες του Θεού, δεν πρέπει να νομίζωμεν τον Θεόν ότι είναι όμοιος με χρυσόν ή άργυρον ή λίθον, κεχαραγμένα διά τέχνης και επινοίας ανθρώπου.
\par 30 Τους καιρούς λοιπόν της αγνοίας παραβλέψας ο Θεός, τώρα παραγγέλλει εις πάντας τους ανθρώπους πανταχού να μετανοώσι,
\par 31 διότι προσδιώρισεν ημέραν εν ή μέλλει να κρίνη την οικουμένην εν δικαιοσύνη, διά ανδρός τον οποίον διώρισε, και έδωκεν εις πάντας βεβαίωσιν περί τούτου, αναστήσας αυτόν εκ νεκρών.
\par 32 Ακούσαντες δε ανάστασιν νεκρών, οι μεν εχλεύαζον, οι δε είπον· Περί τούτου θέλομεν σε ακούσει πάλιν.
\par 33 Και ούτως ο Παύλος εξήλθεν εκ μέσου αυτών.
\par 34 Τινές δε άνδρες προσεκολλήθησαν εις αυτόν και επίστευσαν, μεταξύ των οποίων ήτο και Διονύσιος ο Αρεοπαγίτης και γυνή τις ονόματι Δάμαρις και άλλοι μετ' αυτών.

\chapter{18}

\par 1 Μετά δε ταύτα αναχωρήσας ο Παύλος εκ των Αθηνών, ήλθεν εις Κόρινθον·
\par 2 και ευρών τινά Ιουδαίον ονόματι Ακύλαν, γεγεννημένον εν Πόντω, νεωστί ελθόντα από της Ιταλίας, και Πρίσκιλλαν την γυναίκα αυτού, διότι ο Κλαύδιος είχε διατάξει να αναχωρήσωσι πάντες οι Ιουδαίοι εκ της Ρώμης, προσήλθε προς αυτούς,
\par 3 και επειδή ήτο ομότεχνος, έμενε παρ' αυτοίς και ειργάζετο· διότι ήσαν σκηνοποιοί την τέχνην.
\par 4 Διελέγετο δε εν τη συναγωγή κατά παν σάββατον και έπειθεν Ιουδαίους και Έλληνας.
\par 5 Ότε δε κατέβησαν από της Μακεδονίας ο τε Σίλας και ο Τιμόθεος, ο Παύλος συνεσφίγγετο κατά το πνεύμα διαμαρτυρόμενος προς τους Ιουδαίους ότι ο Ιησούς είναι ο Χριστός.
\par 6 Και επειδή αυτοί ηναντιούντο και εβλασφήμουν, εκτινάξας τα ιμάτια αυτού είπε προς αυτούς· το αίμα σας επί την κεφαλήν σας· εγώ είμαι καθαρός· από του νυν θέλω υπάγει εις τα έθνη.
\par 7 Και μεταβάς εκείθεν ήλθεν εις την οικίαν τινός ονομαζομένου Ιούστου, όστις εσέβετο τον Θεόν, του οποίου η οικία συνείχετο με την συναγωγήν.
\par 8 Κρίσπος δε ο αρχισυνάγωγος επίστευσεν εις τον Κύριον μεθ' όλου του οίκου αυτού, και πολλοί των Κορινθίων ακούοντες επίστευον και εβαπτίζοντο.
\par 9 Και ο Κύριος είπεν εν νυκτί προς τον Παύλον δι' οράματος· Μη φοβού, αλλά ομίλει και μη σιωπήσης,
\par 10 διότι εγώ είμαι μετά σου, και ουδείς θέλει επιβάλει χείρα επί σε διά να σε κακοποιήση, διότι έχω λαόν πολύν εν τη πόλει ταύτη.
\par 11 Και εκάθησεν εκεί εν έτος και μήνας εξ, διδάσκων μεταξύ αυτών τον λόγον του Θεού.
\par 12 Ότε δε ο Γαλλίων ήτο ανθύπατος της Αχαΐας, οι Ιουδαίοι εσηκώθησαν ομοθυμαδόν κατά του Παύλου και έφεραν αυτόν εις το δικαστήριον,
\par 13 λέγοντες ότι ούτος πείθει τους ανθρώπους να λατρεύωσι τον Θεόν παρά τον νόμον.
\par 14 Και ότε έμελλεν ο Παύλος να ανοίξη το στόμα, είπεν ο Γαλλίων προς τους Ιουδαίους· Εάν μεν ήτο τι αδίκημα ή ραδιούργημα πονηρόν, ω Ιουδαίοι, ευλόγως ήθελον σας υποφέρει·
\par 15 εάν δε ήναι ζήτημα περί λέξεων και ονομάτων και του νόμου υμών, θεωρήσατε σείς· διότι εγώ κριτής τούτων δεν θέλω να γείνω.
\par 16 Και απεδίωξεν αυτούς από του δικαστηρίου.
\par 17 Πιάσαντες δε πάντες οι Έλληνες Σωσθένην τον αρχισυνάγωγον, έτυπτον έμπροσθεν του δικαστηρίου· και παντελώς δεν έμελε τον Γαλλίωνα περί τούτων.
\par 18 Ο δε Παύλος, αφού προσέμεινεν έτι ημέρας ικανάς, αποχαιρετήσας τους αδελφούς, εξέπλευσεν εις την Συρίαν, και μετ' αυτού η Πρίσκιλλα και ο Ακύλας, αφού εξύρισε την κεφαλήν εν Κεγχρεαίς· διότι είχεν ευχήν.
\par 19 Και κατήντησεν εις Έφεσον, και αφήκεν εκείνους αυτού, αυτός δε εισελθών εις την συναγωγήν, συνδιελέχθη μετά των Ιουδαίων.
\par 20 Και παρακαλούμενος υπ' αυτών να μείνη πλειότερον καιρόν παρ' αυτοίς, δεν συγκατένευσεν,
\par 21 αλλά απεχαιρέτησεν αυτούς ειπών· Πρέπει εξάπαντος να κάμω την ερχομένην εορτήν εις Ιεροσόλυμα, θέλω δε επιστρέψει πάλιν προς εσάς, του Θεού θέλοντος. Και απέπλευσεν από της Εφέσου,
\par 22 και αποβάς εις Καισάρειαν, ανέβη εις Ιερουσαλήμ, και χαιρετήσας την εκκλησίαν κατέβη εις Αντιόχειαν,
\par 23 και διατρίψας καιρόν τινά, εξήλθε και διήρχετο κατά σειράν την γην της Γαλατίας και την Φρυγίαν, επιστηρίζων πάντας τους μαθητάς.
\par 24 Ιουδαίος δε τις ονόματι Απολλώς, Αλεξανδρεύς το γένος, ανήρ λόγιος, κατήντησεν εις Έφεσον, όστις ήτο δυνατός εν ταις γραφαίς.
\par 25 Ούτος ήτο κατηχημένος την οδόν του Κυρίου, και ζέων κατά το πνεύμα, ελάλει και εδίδασκεν ακριβώς τα περί του Κυρίου, γινώσκων μόνον το βάπτισμα του Ιωάννου.
\par 26 Και ούτος ήρχισε να λαλή μετά παρρησίας εν τη συναγωγή. Ακούσαντες δε αυτόν ο Ακύλας και Πρίσκιλλα, παρέλαβον αυτόν και εξέθεσαν εις αυτόν ακριβέστερα την οδόν του Θεού.
\par 27 Επειδή δε ήθελε να περάση εις την Αχαΐαν, οι αδελφοί έγραψαν προς τους μαθητάς, προτρέποντες να δεχθώσιν αυτόν· όστις ελθών, ωφέλησε πολύ τους πιστεύσαντας διά της χάριτος·
\par 28 διότι εντόνως εξήλεγχε τους Ιουδαίους, δημοσία αποδεικνύων διά των γραφών ότι ο Ιησούς είναι ο Χριστός.

\chapter{19}

\par 1 Ενώ δε ο Απολλώς ήτο εν Κορίνθω, ο Παύλος αφού επέρασε τα ανωτερικά μέρη ήλθεν εις Εφεσον· και ευρών τινάς μαθητάς,
\par 2 είπε προς αυτούς· Ελάβετε Πνεύμα Άγιον αφού επιστεύσατε; οι δε είπον προς αυτόν· Αλλ' ουδέ αν υπάρχη Πνεύμα Άγιον ηκούσαμεν.
\par 3 Και είπε προς αυτούς· Εις τι λοιπόν εβαπτίσθητε; Οι δε είπον· Εις το βάπτισμα του Ιωάννου.
\par 4 Και είπεν ο Παύλος· Ο Ιωάννης μεν εβάπτισε βάπτισμα μετανοίας, λέγων προς τον λαόν να πιστεύσωσιν εις τον ερχόμενον μετ' αυτόν, τουτέστιν εις τον Χριστόν Ιησούν.
\par 5 Ακούσαντες δε εβαπτίσθησαν εις το όνομα του Κυρίου Ιησού.
\par 6 Και αφού ο Παύλος επέθηκεν επ' αυτών τας χείρας, ήλθε το Πνεύμα το Άγιον επ' αυτούς, και ελάλουν γλώσσας και προεφήτευον.
\par 7 Ήσαν δε πάντες ούτοι άνδρες έως δώδεκα.
\par 8 Και εισελθών εις την συναγωγήν ελάλει μετά παρρησίας, διαλεγόμενος τρεις μήνας και πείθων εις τα περί της βασιλείας του Θεού.
\par 9 Επειδή όμως τινές εσκληρύνοντο και δεν επείθοντο, κακολογούντες την οδόν του Κυρίου ενώπιον του πλήθους, απομακρυνθείς απ' αυτών, απεχώρισε τους μαθητάς, διαλεγόμενος καθ' ημέραν εν τω σχολείω τινός, όστις ελέγετο Τύραννος.
\par 10 Έγεινε δε τούτο επί δύο έτη, ώστε πάντες οι κατοικούντες την Ασίαν ήκουσαν τον λόγον του Κυρίου Ιησού, Ιουδαίοί τε και Έλληνες.
\par 11 Και ο Θεός έκαμνε διά των χειρών του Παύλου θαύματα μεγάλα,
\par 12 ώστε και επί τους ασθενείς εφέροντο από του σώματος αυτού μανδήλια ή περιζώματα και έφευγον απ' αυτών αι ασθένειαι, και τα πνεύματα τα πονηρά εξήρχοντο απ' αυτών.
\par 13 Και τινές από των περιερχομένων εξορκιστών Ιουδαίων επεχείρησαν να προφέρωσιν επί τους έχοντας τα πνεύματα τα πονηρά το όνομα του Κυρίου Ιησού, λέγοντες· Σας ορκίζομεν εις τον Ιησούν, τον οποίον ο Παύλος κηρύττει.
\par 14 Και οι πράττοντες τούτο ήσαν επτά τινές υιοί Ιουδαίου αρχιερέως ονομαζομένου Σκευά.
\par 15 Αποκριθέν δε το πνεύμα το πονηρόν, είπε· Τον Ιησούν γνωρίζω και τον Παύλον εξεύρω· σεις δε τίνες είσθε;
\par 16 Και πηδήσας επ' αυτούς ο άνθρωπος, εις τον οποίον ήτο το πνεύμα το πονηρόν, και νικήσας αυτούς, ίσχυσε κατ' αυτών, ώστε γυμνοί και τετραυματισμένοι έφυγον εκ του οίκου εκείνου.
\par 17 Και τούτο έγεινε γνωστόν εις πάντας, Ιουδαίους τε και Έλληνας, τους κατοικούντας την Έφεσον, και επέπεσε φόβος επί πάντας αυτούς, και εμεγαλύνετο το όνομα του Κυρίου Ιησού·
\par 18 και πολλοί των πιστευσάντων ήρχοντο εξομολογούμενοι και φανερόνοντες τας πράξεις αυτών.
\par 19 Πολλοί δε και εξ εκείνων, οίτινες έκαμνον τας μαγείας, φέροντες τα βιβλία αυτών κατέκαιον ενώπιον πάντων· και αριθμήσαντες τας τιμάς αυτών, εύρον πεντήκοντα χιλιάδας αργυρίου.
\par 20 Ούτω κραταιώς ηύξανε και ίσχυεν ο λόγος του Κυρίου.
\par 21 Ως δε ετελέσθησαν ταύτα, ο Παύλος απεφάσισεν εν εαυτώ, αφού διέλθη την Μακεδονίαν και Αχαΐαν, να υπάγη εις την Ιερουσαλήμ, ειπών ότι αφού υπάγω εκεί, πρέπει να ίδω και την Ρώμην.
\par 22 Και αποστείλας εις την Μακεδονίαν δύο των υπηρετούντων αυτόν, Τιμόθεον και Έραστον, αυτός έμεινε καιρόν τινά εν τη Ασία.
\par 23 Έγεινε δε κατ' εκείνον τον καιρόν ταραχή ουκ ολίγη περί ταύτης της οδού.
\par 24 Διότι αργυροκόπος τις ονόματι Δημήτριος, κατασκευάζων ναούς αργυρούς της Αρτέμιδος, επροξένει εις τους τεχνίτας ουκ ολίγον κέρδος·
\par 25 τους οποίους συναθροίσας και τους εργαζομένους τα τοιαύτα, είπεν· Άνδρες, εξεύρετε ότι εκ ταύτης της εργασίας προέρχεται η ευπορία ημών,
\par 26 και θεωρείτε και ακούετε ότι πολύν λαόν ου μόνον της Εφέσου, αλλά σχεδόν πάσης της Ασίας ο Παύλος ούτος έπεισε και μετέβαλε, λέγων ότι δεν είναι θεοί οι διά χειρών κατασκευαζόμενοι.
\par 27 Και ου μόνον η τέχνη ημών αύτη κινδυνεύει να εξουδενωθή, αλλά και το ιερόν της μεγάλης θεάς Αρτέμιδος να λογισθή εις ουδέν, και μέλλει μάλιστα να καταστραφή η μεγαλειότης αυτής, την οποίαν όλη η Ασία και η οικουμένη σέβεται.
\par 28 Ακούσαντες δε και εμπλησθέντες θυμού, έκραζον λέγοντες· Μεγάλη η Άρτεμις των Εφεσίων.
\par 29 Και η πόλις όλη επλήσθη ταραχής, και ώρμησαν ομοθυμαδόν εις το θέατρον, αφού συνήρπασαν τον Γάϊον και Αρίσταρχον τους Μακεδόνας, συνοδοιπόρους του Παύλου.
\par 30 Ενώ δε ο Παύλος ήθελε να εισέλθη εις τον δήμον, οι μαθηταί δεν άφινον αυτόν,
\par 31 τινές δε και εκ των Ασιαρχών, όντες φίλοι αυτού, έστειλαν προς αυτόν και παρεκάλουν να μη εκτεθή εις το θέατρον.
\par 32 Άλλοι μεν λοιπόν έκραζον άλλο τι και άλλοι άλλο· διότι η σύναξις ήτο συγκεχυμένη, και οι πλειότεροι δεν ήξευρον διά τι είχον συναχθή.
\par 33 Εκ δε του όχλου προήγαγον τον Αλέξανδρον, διά να λαλήση, επειδή οι Ιουδαίοι επρόβαλον αυτόν· και ο Αλέξανδρος σείσας την χείρα ήθελε να απολογηθή προς τον δήμον.
\par 34 Αφού δε εγνώρισαν ότι είναι Ιουδαίος, έγεινε μία φωνή εκ πάντων των κραζόντων, έως δύο ώρας· Μεγάλη η Άρτεμις των Εφεσίων.
\par 35 Καθησυχάσας δε ο γραμματεύς τον όχλον, λέγει· Άνδρες Εφέσιοι, και τις άνθρωπος είναι όστις δεν εξεύρει ότι η πόλις των Εφεσίων είναι λάτρις της μεγάλης θεάς Αρτέμιδος και του Διοπετούς αγάλματος;
\par 36 Επειδή λοιπόν ταύτα είναι αναντίρρητα, πρέπει σεις να ησυχάζητε και να μη πράττητε μηδέν προπετές.
\par 37 Διότι εφέρετε τους άνδρας τούτους, οίτινες ούτε ιερόσυλοι είναι ούτε την θεάν σας βλασφημούσιν.
\par 38 Εάν μεν λοιπόν ο Δημήτριος και οι συντεχνίται αυτού έχωσι διαφοράν μετά τινός, υπάρχουσι δικάσιμοι ημέραι και υπάρχουσιν ανθύπατοι, ας εγκαλέσωσιν αλλήλους.
\par 39 Εάν δε ζητήτε τι περί άλλων πραγμάτων, εν τη νομίμω συνελεύσει θέλει διαλυθή.
\par 40 Διότι κινδυνεύομεν να κατηγορηθώμεν ως στασιασταί διά την σημερινήν ταραχήν, χωρίς να υπάρχη μηδεμία αιτία, διά της οποίας θέλομεν δυνηθή να δικαιολογήσωμεν τον θόρυβον τούτον.
\par 41 Και ειπών ταύτα, απέλυσε την συνέλευσιν.

\chapter{20}

\par 1 Αφού δε έπαυσεν ο θόρυβος, προσκαλέσας ο Παύλος τους μαθητάς και ασπασθείς, εξήλθε διά να υπάγη εις την Μακεδονίαν.
\par 2 Και διαπεράσας τα μέρη εκείνα και προτρέψας αυτούς διά λόγων πολλών, ήλθεν εις την Ελλάδα·
\par 3 και αφού διέτριψε τρεις μήνας, επειδή έγεινε κατ' αυτού επιβουλή υπό των Ιουδαίων, ενώ έμελλε να αποπλεύση εις την Συρίαν, ενεκρίθη να επιστρέψη διά της Μακεδονίας.
\par 4 Συνηκολούθει δε αυτόν μέχρι της Ασίας Σώπατρος ο Βεροιαίος και εκ των Θεσσαλονικέων Αρίσταρχος και Σεκούνδος και Γάϊος ο εκ Δέρβης και ο Τιμόθεος, Ασιανοί δε ο Τυχικός και ο Τρόφιμος.
\par 5 Ούτοι ελθόντες πρότεροι περιέμενον ημάς εις την Τρωάδα·
\par 6 ημείς δε εξεπλεύσαμεν από Φιλίππων μετά τας ημέρας των αζύμων και εις πέντε ημέρας ήλθομεν προς αυτούς εις την Τρωάδα, όπου διετρίψαμεν ημέρας επτά.
\par 7 Και τη πρώτη ημέρα της εβδομάδος ενώ οι μαθηταί ήσαν συνηγμένοι διά την κλάσιν του άρτου, ο Παύλος διελέγετο προς αυτούς, μέλλων να αναχωρήση τη επαύριον, και παρέτεινε τον λόγον μέχρι μεσονυκτίου.
\par 8 Ήσαν δε λαμπάδες ικαναί εις το ανώγεον, όπου ήσαν συνηγμένοι.
\par 9 Και νεανίας τις ονόματι Εύτυχος, καθήμενος επί του παραθύρου, κατεφέρετο εις ύπνον βαθύν, ενώ ο Παύλος διελέγετο εκτεταμένως, και κυριευθείς υπό του ύπνου έπεσε κάτω από του τρίτου πατώματος και εσήκωσαν αυτόν νεκρόν.
\par 10 Καταβάς δε ο Παύλος, έπεσεν επ' αυτόν και εναγκαλισθείς είπε· Μη θορυβείσθε· διότι η ψυχή αυτού είναι εν αυτώ.
\par 11 Αφού δε ανέβη και έκοψεν άρτον και εγεύθη και ώμίλησεν ικανώς μέχρι της αυγής· μετά ταύτα ανεχώρησε.
\par 12 Τον δε νέον έφεραν ζώντα και παρηγορήθησαν καθ' υπερβολήν.
\par 13 Ημείς δε καταβάντες πρότεροι εις το πλοίον, απεπλεύσαμεν εις την Άσσον, μέλλοντες να αναλάβωμεν εκείθεν τον Παύλον· επειδή ούτως είχε διατάξει, μέλλων αυτός να υπάγη πεζός.
\par 14 Και καθώς συνήντησεν ημάς εις την Άσσον, αναλαβόντες αυτόν ήλθομεν εις Μιτυλήνην·
\par 15 και εκείθεν αποπλεύσαντες κατηντήσαμεν την επιούσαν αντικρύ Χίου· την δε άλλην εφθάσαμεν εις Σάμον, και μείναντες εν τω Τρωγυλλίω την ακόλουθον ημέραν ήλθομεν εις Μίλητον.
\par 16 Διότι ο Παύλος έκρινε να παραπλεύση την Έφεσον, διά να μη συμβή εις αυτόν να χρονοτριβήση εν τη Ασία· διότι έσπευδεν, αν ήτο δυνατόν εις αυτόν, να ευρεθή την ημέραν της Πεντηκοστής εις Ιεροσόλυμα.
\par 17 Πέμψας δε από της Μιλήτου εις Έφεσον, προσεκάλεσε τους πρεσβυτέρους της εκκλησίας.
\par 18 Και ότε ήλθον προς αυτόν, είπε προς αυτούς· Σεις εξεύρετε, από της πρώτης ημέρας αφ' ης επάτησα εις την Ασίαν, πως επέρασα μεθ' υμών όλον τον χρόνον,
\par 19 δουλεύων τον Κύριον μετά πάσης ταπεινοφροσύνης και μετά πολλών δακρύων και πειρασμών, οίτινες μοι συνέβησαν εν ταις επιβουλαίς των Ιουδαίων,
\par 20 ότι δεν υπέκρυψα ουδέν των συμφερόντων, ώστε να μη αναγγείλω αυτό προς εσάς και να σας διδάξω δημοσία και κατ' οίκους,
\par 21 διαμαρτυρόμενος προς Ιουδαίους τε και Έλληνας την εις τον Θεόν μετάνοιαν και την πίστιν την εις τον Κύριον ημών Ιησούν Χριστόν.
\par 22 Και τώρα ιδού, εγώ δεδεμένος τω πνεύματι υπάγω εις Ιερουσαλήμ, μη γνωρίζων τα μέλλοντα να συμβώσιν εις εμέ εν αυτή,
\par 23 πλην ότι το Πνεύμα το Άγιον μαρτυρεί εν πάση πόλει λέγον, ότι δεσμά και θλίψεις με περιμένουσι.
\par 24 Δεν φροντίζω όμως περί ουδενός τούτων ουδέ έχω πολύτιμον την ζωήν μου, ως το να τελειώσω τον δρόμον μου μετά χαράς και την διακονίαν, την οποίαν έλαβον παρά του Κυρίου Ιησού, να διακηρύξω το ευαγγέλιον της χάριτος του Θεού.
\par 25 Και τώρα ιδού, εγώ εξεύρω ότι πλέον δεν θέλετε ιδεί το πρόσωπόν μου σεις πάντες, μεταξύ των οποίων διήλθον κηρύττων την βασιλείαν του Θεού.
\par 26 Όθεν μαρτύρομαι προς εσάς εν τη σήμερον ημέρα, ότι εγώ είμαι καθαρός από του αίματος πάντων·
\par 27 διότι δεν συνεστάλην να αναγγείλω προς εσάς πάσαν την βουλήν του Θεού.
\par 28 Προσέχετε λοιπόν εις εαυτούς και εις όλον το ποίμνιον, εις το οποίον το Πνεύμα το Άγιον σας έθεσεν επισκόπους, διά να ποιμαίνητε την εκκλησίαν του Θεού, την οποίαν απέκτησε διά του ιδίου αυτού αίματος.
\par 29 Διότι εγώ εξεύρω τούτο, ότι μετά την αναχώρησίν μου θέλουσιν εισέλθει εις εσάς λύκοι βαρείς μη φειδόμενοι του ποιμνίου·
\par 30 και εξ υμών αυτών θέλουσι σηκωθή άνθρωποι λαλούντες διεστραμμένα, διά να αποσπώσι τους μαθητάς οπίσω αυτών.
\par 31 Διά τούτο αγρυπνείτε, ενθυμούμενοι ότι τρία έτη νύκτα και ημέραν δεν έπαυσα νουθετών μετά δακρύων ένα έκαστον.
\par 32 Και τώρα, αδελφοί, σας αφιερόνω εις τον Θεόν και εις τον λόγον της χάριτος αυτού, όστις δύναται να εποικοδομήση και να δώση εις εσάς κληρονομίαν μεταξύ πάντων των ηγιασμένων.
\par 33 Αργύριον ή χρυσίον ή ιμάτιον ουδενός επεθύμησα·
\par 34 σεις δε αυτοί εξεύρετε ότι εις τας χρείας μου και εις τους όντας μετ' εμού αι χείρες αύται υπηρέτησαν.
\par 35 Κατά πάντα υπέδειξα εις εσάς ότι ούτω κοπιάζοντες πρέπει να βοηθήτε τους ασθενείς και να ενθυμήσθε τους λόγους του Κυρίου Ιησού, ότι αυτός είπε· Μακάριον είναι να δίδη τις μάλλον παρά να λαμβάνη.
\par 36 Και αφού είπε ταύτα, γονατίσας προσηυχήθη μετά πάντων αυτών.
\par 37 Έγεινε δε πολύς κλαυθμός πάντων, και πεσόντες επί τον τράχηλον του Παύλου κατεφίλουν αυτόν,
\par 38 υπερλυπούμενοι μάλιστα διά τον λόγον τον οποίον είπεν, ότι δεν θέλουσιν ιδεί πλέον το πρόσωπον αυτού. Και προέπεμπον αυτόν εις το πλοίον.

\chapter{21}

\par 1 Καθώς δε αποσπασθέντες απ' αυτών απεπλεύσαμεν, ήλθομεν κατ' ευθείαν εις την Κων, την δε ακόλουθον ημέραν εις την Ρόδον, και εκείθεν εις Πάταρα.
\par 2 Και ευρόντες πλοίον μέλλον να περάση εις Φοινίκην, επέβημεν εις αυτό και απεπλεύσαμεν.
\par 3 Και αφού διεκρίναμεν μακρόθεν την Κύπρον και αφήκαμεν αυτήν αριστερά, επλέομεν εις Συρίαν, και κατέβημεν εις Τύρον· διότι εκεί έμελλε το πλοίον να εκβάλη το φορτίον αυτού.
\par 4 Και ευρόντες τους μαθητάς, εμείναμεν αυτού επτά ημέρας· οίτινες έλεγον προς τον Παύλον διά του Πνεύματος να μη αναβή εις Ιερουσαλήμ.
\par 5 Αφού δε ετελειώσαμεν τας ημέρας εκείνας, εξελθόντες επορευόμεθα και προέπεμπον ημάς πάντες συν γυναιξί και τέκνοις έως έξω της πόλεως, και γονατίσαντες επί τον αιγιαλόν προσηυχήθημεν,
\par 6 και ασπασθέντες αλλήλους επέβημεν εις το πλοίον, εκείνοι δε υπέστρεψαν εις τα ίδια.
\par 7 Και ημείς τελειώσαντες τον πλούν από Τύρου κατηντήσαμεν εις Πτολεμαΐδα, και ασπασθέντες τους αδελφούς εμείναμεν παρ' αυτοίς μίαν ημέραν.
\par 8 Τη δε επαύριον, ο Παύλος και οι περί αυτόν αναχωρήσαντες, ήλθομεν εις Καισάρειαν· και εισελθόντες εις τον οίκον Φιλίππου του Ευαγγελιστού, του όντος εκ των επτά, εμείναμεν παρ' αυτώ.
\par 9 Είχε δε ούτος τέσσαρας θυγατέρας παρθένους, αίτινες προεφήτευον.
\par 10 Και ενώ διετρίβομεν εκεί ημέρας πολλάς, κατέβη από της Ιουδαίας προφήτης τις ονόματι Άγαβος,
\par 11 και ελθών προς ημάς, έλαβε την ζώνην του Παύλου και δέσας τας χείρας εαυτού και τους πόδας είπε· Ταύτα λέγει το Πνεύμα το Αγιον· Τον άνδρα, του οποίου είναι η ζώνη αύτη, ούτω θέλουσι δέσει εν Ιερουσαλήμ οι Ιουδαίοι και θέλουσι παραδώσει εις τας χείρας των εθνών.
\par 12 Και ως ηκούσαμεν ταύτα, παρεκαλούμεν αυτόν και ημείς και οι εντόπιοι να μη αναβή εις Ιερουσαλήμ.
\par 13 Ο Παύλος όμως απεκρίθη· Τι κάμνετε, κλαίοντες και καταθλίβοντες την καρδίαν μου; επειδή εγώ ουχί μόνον να δεθώ, αλλά και να αποθάνω εις Ιερουσαλήμ είμαι έτοιμος υπέρ του ονόματος του Κυρίου Ιησού.
\par 14 Και επειδή δεν επείθετο, ησυχάσαμεν ειπόντες· Ας γείνη το θέλημα του Κυρίου.
\par 15 Μετά δε τας ημέρας ταύτας ετοιμάσαντες την αποσκευήν ημών, ανεβαίνομεν εις Ιερουσαλήμ·
\par 16 ήλθον δε μεθ' ημών και τινές των μαθητών εκ της Καισαρείας, φέροντες Μνάσωνά τινά Κύπριον, παλαιόν μαθητήν, παρά τω οποίω εμέλλομεν να ξενισθώμεν.
\par 17 Και αφού ήλθομεν εις Ιεροσόλυμα, μετά χαράς εδέχθησαν ημάς οι αδελφοί.
\par 18 Την δε ακόλουθον ημέραν υπήγεν ο Παύλος μεθ' ημών προς τον Ιάκωβον, και ήλθον πάντες οι πρεσβύτεροι.
\par 19 Και ασπασθείς αυτούς, διηγείτο καθ' εν έκαστον όσα έκαμεν ο Θεός μεταξύ των εθνών διά της διακονίας αυτού.
\par 20 Εκείνοι δε ακούσαντες εδόξαζον τον Κύριον, και είπον προς αυτόν· Βλέπεις, αδελφέ, πόσαι μυριάδες είναι εκ των Ιουδαίων οίτινες επίστευσαν, και πάντες είναι ζηλωταί του νόμου.
\par 21 Έμαθον δε περί σου ότι διδάσκεις πάντας τους μεταξύ των εθνών Ιουδαίους να αποστατήσωσιν από του Μωϋσέως, λέγων να μη περιτέμνωσι τα τέκνα αυτών μηδέ να περιπατώσι κατά τα έθιμα.
\par 22 Τι είναι λοιπόν; μέλλει βεβαίως να συναχθή πλήθος· διότι θέλουσιν ακούσει ότι ήλθες.
\par 23 Κάμε λοιπόν τούτο, το οποίον σοι λέγομεν· Ευρίσκονται παρ' ημίν τέσσαρες άνδρες, οίτινες έχουσιν ευχήν εφ' εαυτών·
\par 24 παράλαβε τούτους και καθαρίσθητι μετ' αυτών και δαπάνησον δι' αυτούς διά να ξυρισθώσι την κεφαλήν, και να γνωρίσωσι πάντες ότι δεν υπάρχει ουδέν εκ των όσα έμαθον περί σου, αλλ' ακολουθείς και συ φυλάττων τον νόμον.
\par 25 Περί δε των εθνών, τα οποία επίστευσαν, ημείς εγράψαμεν, αποφασίσαντες να μη φυλάττωσι μηδέν τοιούτον, παρά μόνον να απέχωσιν από του ειδωλοθύτου και του αίματος και πνικτού και πορνείας.
\par 26 Τότε ο Παύλος παραλαβών τους άνδρας, την ακόλουθον ημέραν καθαρισθείς μετ' αυτών εισήλθεν εις το ιερόν, διαγγέλλων πότε εκπληρούνται αι ημέραι του καθαρισμού, ότε θέλει γείνει προσφορά υπέρ ενός εκάστου αυτών.
\par 27 Ως δε έμελλον αι επτά ημέραι να συντελεσθώσιν, οι από της Ασίας Ιουδαίοι ιδόντες αυτόν εν τω ιερώ, ετάραξαν πάντα τον όχλον και έβαλον τας χείρας επ' αυτόν,
\par 28 κράζοντες· Άνδρες Ισραηλίται, βοηθείτε· ούτος είναι ο άνθρωπος, όστις διδάσκει πάντας πανταχού εναντίον του λαού και του νόμου και του τόπου τούτου· προς τούτοις δε εισήγαγε και Έλληνας εις το ιερόν και εβεβήλωσε τον άγιον τούτον τόπον·
\par 29 διότι είχον ιδεί προλαβόντως Τρόφιμον τον Εφέσιον μετ' αυτού εν τη πόλει, τον οποίον ενόμιζον ότι ο Παύλος εισήγαγεν εις το ιερόν.
\par 30 Και εκινήθη η πόλις όλη και έγεινε συρροή του λαού, και πιάσαντες τον Παύλον έσυρον αυτόν έξω του ιερού, και ευθύς εκλείσθησαν αι θύραι.
\par 31 Ενώ δε εζήτουν να θανατώσωσιν αυτόν, ανέβη η φήμη εις τον χιλίαρχον του τάγματος, ότι όλη η Ιερουσαλήμ είναι τεταραγμένη·
\par 32 όστις παραλαβών ευθύς στρατιώτας και εκατοντάρχους, έδραμε κάτω προς αυτούς. Οι δε ιδόντες τον χιλίαρχον και τους στρατιώτας, έπαυσαν να τύπτωσι τον Παύλον.
\par 33 Τότε πλησιάσας ο χιλίαρχος, επίασεν αυτόν και προσέταξε να δεθή με δύο αλύσεις, και ηρώτα τις ήτο και τι είχε πράξει.
\par 34 Και εφώναζον μεταξύ του όχλου άλλοι άλλο τι και άλλοι άλλο· μη δυνάμενος δε διά τον θόρυβον να μάθη το βέβαιον, προσέταξε να φερθή εις το φρούριον.
\par 35 Ότε δε έφθασεν εις τας βαθμίδας, συνέβη να βαστάζηται υπό των στρατιωτών διά την βίαν του όχλου·
\par 36 επειδή το πλήθος του λαού ηκολούθει, κράζον· Σήκωσον αυτόν.
\par 37 Ενώ δε έμελλεν ο Παύλος να εισαχθή εις το φρούριον, λέγει προς τον χιλίαρχον· Μοι είναι συγκεχωρημένον να σοι είπω τι; Ο δε είπεν· Εξεύρεις Ελληνικά;
\par 38 δεν είσαι συ τάχα ο Αιγύπτιος, ο προ των ημερών τούτων διεγείρας εις αποστασίαν και εκβαλών εις την έρημον τους τετρακισχιλίους άνδρας φονείς;
\par 39 Και ο Παύλος είπεν· Εγώ είμαι άνθρωπος Ιουδαίος εκ της Ταρσού, πολίτης επισήμου πόλεως της Κιλικίας και σε παρακαλώ, δος μοι την άδειαν να λαλήσω προς τον λαόν.
\par 40 Και αφού έδωκεν εις αυτόν την άδειαν, ο Παύλος, σταθείς επί των βαθμίδων, έσεισε την χείρα εις τον λαόν· και γενομένης σιωπής μεγάλης, ελάλησεν εις την Εβραϊκήν διάλεκτον, λέγων·

\chapter{22}

\par 1 Άνδρες αδελφοί και πατέρες, ακούσατέ με απολογούμενον τώρα προς εσάς.
\par 2 Ακούσαντες δε ότι ελάλει προς αυτούς εις την Εβραϊκήν διάλεκτον, έδειξαν περισσοτέραν ησυχίαν. Και είπεν·
\par 3 Εγώ μεν είμαι άνθρωπος Ιουδαίος, γεγεννημένος εν Ταρσώ της Κιλικίας, ανατεθραμμένος δε εν τη πόλει ταύτη παρά τους πόδας του Γαμαλιήλ, πεπαιδευμένος κατά την ακρίβειαν του πατροπαραδότου νόμου, ζηλωτής ων του Θεού, καθώς πάντες σεις είσθε σήμερον·
\par 4 όστις κατέτρεξα μέχρι θανάτου ταύτην την οδόν, δεσμεύων και παραδίδων εις φυλακάς άνδρας τε και γυναίκας,
\par 5 καθώς και ο αρχιερεύς μαρτυρεί εις εμέ και όλον το πρεσβυτέριον· παρά των οποίων και επιστολάς λαβών προς τους αδελφούς, επορευόμην εις Δαμασκόν διά να φέρω δεδεμένους εις Ιερουσαλήμ και τους εκεί όντας, διά να τιμωρηθώσιν.
\par 6 Ενώ δε οδοιπορών επλησίαζον εις την Δαμασκόν, περί την μεσημβρίαν εξαίφνης έστραψε περί εμέ φως πολύ εκ του ουρανού,
\par 7 και έπεσον εις το έδαφος και ήκουσα φωνήν λέγουσαν προς εμέ· Σαούλ, Σαούλ, τι με διώκεις;
\par 8 Εγώ δε απεκρίθην· Τις είσαι, Κύριε; Και είπε προς εμέ· Εγώ είμαι Ιησούς ο Ναζωραίος, τον οποίον συ διώκεις.
\par 9 Οι όντες δε μετ' εμού το μεν φως είδον και κατεφοβήθησαν, την φωνήν όμως του λαλούντος προς εμέ δεν ήκουσαν.
\par 10 Και είπον· Τι να κάμω, Κύριε; Και ο Κύριος είπε προς εμέ· Σηκωθείς ύπαγε εις Δαμασκόν, και εκεί θέλει σοι λαληθή περί πάντων όσα είναι διωρισμένα να κάμης.
\par 11 Και επειδή εκ της λαμπρότητος του φωτός εκείνου δεν έβλεπον, χειραγωγούμενος υπό των όντων μετ' εμού ήλθον εις Δαμασκόν.
\par 12 Ανανίας δε τις, άνθρωπος ευσεβής κατά τον νόμον, μαρτυρούμενος υπό πάντων των εκεί κατοικούντων Ιουδαίων,
\par 13 ήλθε προς εμέ και σταθείς επάνω μου μοι, είπε· Σαούλ αδελφέ, ανάβλεψον. Και εγώ τη αυτή ώρα ανέβλεψα εις αυτόν.
\par 14 Ο δε είπεν· Ο Θεός των πατέρων ημών σε διώρισε να γνωρίσης το θέλημα αυτού και να ίδης τον δίκαιον και να ακούσης φωνήν εκ του στόματος αυτού,
\par 15 διότι θέλεις είσθαι μάρτυς περί αυτού προς πάντας τους ανθρώπους των όσα είδες και ήκουσας.
\par 16 Και τώρα τι βραδύνεις; σηκωθείς βαπτίσθητι και απολούσθητι από των αμαρτιών σου, επικαλεσθείς το όνομα του Κυρίου.
\par 17 Αφού δε υπέστρεψα εις Ιερουσαλήμ, ενώ προσηυχόμην εν τω ιερώ, ήλθον εις έκστασιν
\par 18 και είδον αυτόν λέγοντα προς εμέ· Σπεύσον και έξελθε ταχέως εξ Ιερουσαλήμ, διότι δεν θέλουσι παραδεχθή την περί εμού μαρτυρίαν σου.
\par 19 Και εγώ είπον· Κύριε, αυτοί εξεύρουσιν ότι εγώ εφυλάκιζον και έδερον εν ταις συναγωγαίς τους πιστεύοντας εις σέ·
\par 20 και ότε εχύνετο το αίμα Στεφάνου του μάρτυρός σου, και εγώ ήμην παρών και συνεφώνουν εις τον φόνον αυτού και εφύλαττον τα ιμάτια των φονευόντων αυτόν.
\par 21 Και είπε προς εμέ· Ύπαγε, διότι εγώ θέλω σε εξαποστείλει εις έθνη μακράν.
\par 22 Και μέχρι τούτου του λόγου ήκουον αυτόν· τότε δε ύψωσαν την φωνήν αυτών, λέγοντες· Σήκωσον από της γης τον τοιούτον· διότι δεν πρέπει να ζη.
\par 23 Και επειδή αυτοί εκραύγαζον και ετίναζον τα ιμάτια και έρριπτον κονιορτόν εις τον αέρα,
\par 24 ο χιλίαρχος προσέταξε να φερθή εις το φρούριον, παραγγείλας να εξετασθή διά μαστίγων, διά να γνωρίση διά ποίαν αιτίαν εφώναζον ούτω κατ' αυτού.
\par 25 Και καθώς εξήπλωσεν αυτόν δεδεμένον με τα λωρία, ο Παύλος είπε προς τον παρεστώτα εκατόνταρχον· Είναι τάχα νόμιμον εις εσάς άνθρωπον Ρωμαίον και ακατάκριτον να μαστιγόνητε;
\par 26 Ακούσας δε ο εκατόνταρχος, υπήγε και απήγγειλε προς τον χιλίαρχον, λέγων· Βλέπε τι μέλλεις να κάμης· διότι ο άνθρωπος ούτος είναι Ρωμαίος.
\par 27 Προσελθών δε ο χιλίαρχος, είπε προς αυτόν· Λέγε μοι, συ Ρωμαίος είσαι; Ο δε είπε· Ναι.
\par 28 Και απεκρίθη ο χιλίαρχος· Εγώ διά πολλών χρημάτων απέκτησα ταύτην την πολιτογράφησιν. Ο δε Παύλος είπεν· Αλλ' εγώ και εγεννήθην Ρωμαίος.
\par 29 Ευθύς λοιπόν απεσύρθησαν απ' αυτού οι μέλλοντες να βασανίσωσιν αυτόν. Και ο χιλίαρχος έτι εφοβήθη γνωρίσας ότι είναι Ρωμαίος, και διότι είχε δέσει αυτόν.
\par 30 Τη δε επαύριον θέλων να μάθη το βέβαιον, περί τίνος κατηγορείται παρά των Ιουδαίων, έλυσεν αυτόν από των δεσμών, και προσέταξε να έλθωσιν οι αρχιερείς και όλον το συνέδριον αυτών και καταβιβάσας τον Παύλον, έστησεν έμπροσθεν αυτών.

\chapter{23}

\par 1 Ατενίσας δε ο Παύλος εις το συνέδριον, είπεν· Άνδρες αδελφοί, εγώ έζησα ενώπιον του Θεού μετά πάσης καλής συνειδήσεως μέχρι ταύτης της ημέρας.
\par 2 Ο δε αρχιερεύς Ανανίας προσέταξε τους παρεστώτας πλησίον αυτού να κτυπήσωσι το στόμα αυτού.
\par 3 Τότε ο Παύλος είπε προς αυτόν· Ο Θεός μέλλει να σε κτυπήση, τοίχε ασβεστωμένε· και συ κάθησαι να με κρίνης κατά τον νόμον, και παρανομών προστάζεις να με κτυπώσιν;
\par 4 Οι δε παρεστώτες είπον· Τον αρχιερέα του Θεού λοιδορείς;
\par 5 Και ο Παύλος είπε· Δεν ήξευρον, αδελφοί, ότι είναι αρχιερεύς· διότι είναι γεγραμμένον. Άρχοντα του λαού σου δεν θέλεις κακολογήσει.
\par 6 Εννοήσας δε ο Παύλος ότι το εν μέρος είναι Σαδδουκαίων, το δε άλλο Φαρισαίων, έκραξεν εν τω συνεδρίω. Άνδρες αδελφοί, εγώ είμαι Φαρισαίος, υιός Φαρισαίου· περί ελπίδος και αναστάσεως νεκρών εγώ κρίνομαι.
\par 7 Και ότε ελάλησε τούτο, έγεινε διαίρεσις των Φαρισαίων και των Σαδδουκαίων, και διηρέθη το πλήθος.
\par 8 Διότι οι μεν Σαδδουκαίοι λέγουσιν ότι δεν είναι ανάστασις ουδέ άγγελος ουδέ πνεύμα, οι δε Φαρισαίοι ομολογούσιν αμφότερα.
\par 9 Και έγεινε κραυγή μεγάλη, και σηκωθέντες οι γραμματείς του μέρους των Φαρισαίων διεμάχοντο, λέγοντες· Ουδέν κακόν ευρίσκομεν εν τω ανθρώπω τούτω· αν δε ελάλησε προς αυτόν πνεύμα ή άγγελος, ας μη θεομαχώμεν.
\par 10 Και επειδή έγεινε μεγάλη διαίρεσις, φοβηθείς ο χιλίαρχος μη διασπαραχθή ο Παύλος υπ' αυτών, προσέταξε να καταβή το στράτευμα και να αρπάση αυτόν εκ μέσου αυτών και να φέρη εις το φρούριον.
\par 11 Την δε ερχομένην νύκτα επιφανείς εις αυτόν ο Κύριος, είπε· Θάρρει, Παύλε, διότι καθώς εμαρτύρησας τα περί εμού εις Ιερουσαλήμ, ούτω πρέπει να μαρτυρήσης και εις Ρώμην.
\par 12 Και ότε έγεινεν ημέρα, τινές των Ιουδαίων συνομώσαντες ανεθεμάτισαν εαυτούς, λέγοντες μήτε να φάγωσι μήτε να πίωσιν, εωσού φονεύσωσι τον Παύλον·
\par 13 ήσαν δε πλειότεροι των τεσσαράκοντα οι πράξαντες την συνωμοσίαν ταύτην·
\par 14 οίτινες ελθόντες προς τους αρχιερείς και τους πρεσβυτέρους, είπον· Με ανάθεμα ανεθεματίσαμεν εαυτούς, να μη γευθώμεν μηδέν εωσού φονεύσωμεν τον Παύλον.
\par 15 Τώρα λοιπόν σεις μετά του συνεδρίου μηνύσατε προς τον χιλίαρχον, να καταβιβάση αυτόν αύριον προς εσάς, ως μέλλοντας να μάθητε ακριβέστερον τα περί αυτού· ημείς δε, πριν αυτός πλησιάση, είμεθα έτοιμοι να φονεύσωμεν αυτόν.
\par 16 Ακούσας δε την ενέδραν ο υιός της αδελφής του Παύλου, υπήγε και εισελθών εις το φρούριον απήγγειλε προς τον Παύλον.
\par 17 Και ο Παύλος προσκαλέσας ένα των εκατοντάρχων, είπε· Φέρε τον νέον τούτον προς τον χιλίαρχον· διότι έχει τι να απαγγείλη προς αυτόν.
\par 18 Εκείνος λοιπόν παραλαβών αυτόν, έφερε προς τον χιλίαρχον και λέγει· Ο δέσμιος Παύλος με έκραξε και με παρεκάλεσε να φέρω τον νέον τούτον προς σε, διότι έχει τι να σοι λαλήση.
\par 19 Πιάσας δε αυτόν από της χειρός ο χιλίαρχος και αποσυρθείς κατ' ιδίαν, ηρώτησε, Τι είναι εκείνο, το οποίον έχεις να μοι απαγγείλης;
\par 20 Ο δε είπεν ότι οι Ιουδαίοι συνεφώνησαν να σε παρακαλέσωσι να καταβιβάσης αύριον τον Παύλον εις το συνέδριον, ως θέλοντες να μάθωσι τι ακριβέστερον περί αυτού.
\par 21 Συ λοιπόν μη πεισθής εις αυτούς, διότι ενεδρεύουσιν αυτόν πλειότεροι των τεσσαράκοντα άνδρες εξ αυτών, οίτινες ανεθεμάτισαν εαυτούς μήτε να φάγωσι μήτε να πίωσιν, εωσού φονεύσωσιν αυτόν· και τώρα είναι έτοιμοι, προσμένοντες την παρά σου υπόσχεσιν.
\par 22 Ο χιλίαρχος λοιπόν απέλυσε τον νέον, παραγγείλας, Να μη είπης εις μηδένα ότι εφανέρωσας ταύτα εις εμέ.
\par 23 Και προσκαλέσας δύο τινάς των εκατοντάρχων, είπεν· Ετοιμάσατε διακοσίους στρατιώτας διά να υπάγωσιν έως Καισαρείας, και εβδομήκοντα ιππείς και διακοσίους λογχοφόρους, από τρίτης ώρας της νυκτός,
\par 24 ετοιμάσατε και ζώα, διά να επικαθίσωσι τον Παύλον και φέρωσιν ασφαλώς προς Φήλικα τον ηγεμόνα·
\par 25 και έγραψεν επιστολήν περιέχουσαν τον τύπον τούτον·
\par 26 Κλαύδιος Λυσίας προς τον κράτιστον ηγεμόνα Φήλικα, χαίρειν.
\par 27 Τον άνδρα τούτον, συλληφθέντα υπό των Ιουδαίων και μέλλοντα να φονευθή υπ' αυτών, επελθών μετά του στρατεύματος έσωσα αυτόν, μαθών ότι είναι Ρωμαίος.
\par 28 Θέλων δε να μάθω αιτίαν, διά την οποίαν εκατηγόρουν αυτόν, κατεβίβασα αυτόν εις το συνέδριον αυτών·
\par 29 και εύρον αυτόν εγκαλούμενον περί ζητημάτων του νόμου αυτών, μη έχοντα όμως μηδέν έγκλημα άξιον θανάτου ή δεσμών.
\par 30 Και επειδή εμηνύθη προς εμέ ότι μέλλει να γείνη εις τον άνθρωπον επιβουλή υπό των Ιουδαίων, ευθύς έπεμψα αυτόν προς σε, παραγγείλας και εις τους κατηγόρους να είπωσιν ενώπιον σου τα κατ' αυτού. Υγίαινε.
\par 31 Οι μεν λοιπόν στρατιώται κατά την δοθείσαν εις αυτούς προσταγήν αναλαβόντες τον Παύλον, έφεραν διά της νυκτός εις την Αντιπατρίδα,
\par 32 την δε επαύριον, αφήσαντες τους ιππείς να υπάγωσι μετ' αυτού, υπέστρεψαν εις το φρούριον·
\par 33 οίτινες εισελθόντες εις την Καισάρειαν και εγχειρίσαντες την επιστολήν εις τον ηγεμόνα, παρέστησαν και τον Παύλον εις αυτόν.
\par 34 Ο δε ηγεμών, αφού ανέγνωσε την επιστολήν και ηρώτησεν εκ ποίας επαρχίας είναι και ήκουσεν ότι είναι από Κιλικίας,
\par 35 Θέλω σε ακροασθή, είπεν, όταν και οι κατήγοροί σου έλθωσι· και προσέταξε να φυλάττηται εν τω πραιτωρίω του Ηρώδου.

\chapter{24}

\par 1 Μετά δε πέντε ημέρας κατέβη ο αρχιερεύς Ανανίας μετά των πρεσβυτέρων και μετά τινός Τερτύλλου ρήτορος, οίτινες ενεφανίσθησαν εις τον ηγεμόνα κατά του Παύλου.
\par 2 Προσκληθέντος δε αυτού, ήρχισε να κατηγορή ο Τέρτυλλος, λέγων· Επειδή απολαμβάνομεν διά σου πολλήν ησυχίαν και γίνονται εις το έθνος τούτο λαμπρά πράγματα διά της προνοίας σου,
\par 3 κατά πάντα και πανταχού ευγνωμονούμεν, κράτιστε Φήλιξ, μετά πάσης ευχαριστίας.
\par 4 Αλλά διά να μη σε απασχολώ περισσότερον, παρακαλώ να ακούσης ημάς συντόμως με την επιείκειάν σου.
\par 5 Επειδή εύρομεν τον άνθρωπον τούτον ότι είναι φθοροποιός και διεγείρει στάσιν μεταξύ όλων των κατά την οικουμένην Ιουδαίων, και είναι πρωτοστάτης της αιρέσεως των Ναζωραίων,
\par 6 όστις και τον ναόν εδοκίμασε να βεβηλώση, τον οποίον και εκρατήσαμεν και κατά τον ημέτερον νόμον ηθελήσαμεν να κρίνωμεν.
\par 7 Ελθών όμως Λυσίας ο χιλίαρχος απέσπασεν αυτόν μετά πολλής βίας εκ των χειρών ημών,
\par 8 προστάξας τους κατηγόρους αυτού να έλθωσιν ενώπιόν σου· παρά του οποίου θέλεις δυνηθή εξετάσας αυτός να μάθης περί πάντων τούτων, περί των οποίων ημείς κατηγορούμεν αυτόν.
\par 9 Συνωμολόγησαν δε και οι Ιουδαίοι, λέγοντες ότι ταύτα ούτως έχουσι.
\par 10 Τότε ο Παύλος, αφού ο ηγεμών ένευσεν εις αυτόν να ομιλήση, απεκρίθη· Επειδή σε γνωρίζω ότι εκ πολλών ετών είσαι κριτής εις το έθνος τούτο, απολογούμαι περί εμαυτού προθυμότερον,
\par 11 διότι δύνασαι να πληροφορηθής ότι δεν είναι πλειότεραι των δώδεκα ημερών αφού εγώ ανέβην διά να προσκυνήσω εν Ιερουσαλήμ·
\par 12 και ούτε εν τω ιερώ εύρον εμέ διαλεγόμενον μετά τινός ή οχλαγωγούντα, ούτε εν ταις συναγωγαίς ούτε εν τη πόλει·
\par 13 ουδέ δύνανται να φέρωσιν αποδείξεις περί όσων με κατηγορούσι τώρα.
\par 14 Ομολογώ δε τούτο εις σε, ότι κατά την οδόν, την οποίαν ούτοι λέγουσιν αίρεσιν, ούτω λατρεύω τον Θεόν των πατέρων μου, πιστεύων εις πάντα τα γεγραμμένα εν τω νόμω και εν τοις προφήταις,
\par 15 ελπίδα έχων εις τον Θεόν, την οποίαν και αυτοί ούτοι προσμένουσιν, ότι μέλλει να γείνη ανάστασις νεκρών, δικαίων τε και αδίκων·
\par 16 εις τούτο δε εγώ σπουδάζω, εις το να έχω άπταιστον συνείδησιν προς τον Θεόν και προς τους ανθρώπους διαπαντός.
\par 17 Μετά πολλά δε έτη ήλθον διά να κάμω εις το έθνος μου ελεημοσύνας και προσφοράς·
\par 18 εν τω μεταξύ δε τούτων Ιουδαίοί τινές εκ της Ασίας εύρόν με κεκαθαρισμένον εν τω ιερώ, ουχί μετά όχλου ουδέ μετά θορύβου,
\par 19 οίτινες έπρεπε να παρασταθώσιν ενώπιόν σου και να με κατηγορήσωσιν, εάν είχόν τι κατ' εμού.
\par 20 Η αυτοί ούτοι ας είπωσιν εάν εύρον εν εμοί τι αδίκημα, ότε παρεστάθην ενώπιον του συνεδρίου,
\par 21 εκτός εάν ήναι περί ταύτης της μιας φωνής, την οποίαν εφώναξα ιστάμενος μεταξύ αυτών, ότι περί αναστάσεως νεκρών εγώ κρίνομαι σήμερον από σας.
\par 22 Ακούσας δε ταύτα ο Φήλιξ ανέβαλε την κρίσιν αυτών, επειδή ήξευρεν ακριβέστερα τα περί της οδού ταύτης, και είπεν· Όταν Λυσίας ο χιλίαρχος καταβή, θέλω αποφασίσει περί της διαφοράς σας,
\par 23 και διέταξε τον εκατόνταρχον να φυλάττηται ο Παύλος και να έχη άνεσιν και να μη εμποδίζωσι μηδένα εκ των οικείων αυτού να υπηρετή ή να έρχηται προς αυτόν.
\par 24 Μετά δε ημέρας τινάς ελθών ο Φήλιξ μετά της Δρουσίλλης της γυναικός αυτού, ήτις ήτο Ιουδαία, μετεκάλεσε τον Παύλον και ήκουσε παρ' αυτού περί της εις Χριστόν πίστεως.
\par 25 Ενώ δε αυτός ωμίλει περί δικαιοσύνης και εγκρατείας και περί της μελλούσης κρίσεως, ο Φήλιξ γενόμενος έμφοβος απεκρίθη· Κατά το παρόν ύπαγε, και όταν λάβω καιρόν θέλω σε μετακαλέσει,
\par 26 εν τούτω δε και ήλπιζεν ότι θέλουσι δοθή εις αυτόν χρήματα υπό του Παύλου, διά να απολύση αυτόν· όθεν και συχνότερα μετακαλών αυτόν ωμίλει μετ' αυτού.
\par 27 Μετά δε την συμπλήρωσιν δύο ετών ο Φήλιξ έλαβε διάδοχον τον Πόρκιον Φήστον· και θέλων να κάμη χάριν εις τους Ιουδαίους ο Φήλιξ, αφήκε τον Παύλον δεδεμένον.

\chapter{25}

\par 1 Ο Φήστος λοιπόν, αφού ήλθεν εις την επαρχίαν, μετά τρεις ημέρας ανέβη εις Ιεροσόλυμα από της Καισαρείας.
\par 2 Ενεφανίσθησαν δε εις αυτόν ο αρχιερεύς και οι πρώτοι των Ιουδαίων κατά του Παύλου και παρεκάλουν αυτόν,
\par 3 ζητούντες χάριν κατ' αυτού, να μεταφέρη αυτόν εις Ιερουσαλήμ, ενεδρεύοντες να φονεύσωσιν αυτόν καθ' οδόν.
\par 4 Ο δε Φήστος απεκρίθη ότι ο Παύλος φυλάττεται εν Καισαρεία, και ότι αυτός ταχέως μέλλει να αναχωρήση εκείσε.
\par 5 Όθεν οι δυνατοί μεταξύ σας, είπεν, ας καταβώσι μετ' εμού, και εάν υπάρχη τι εν τω ανθρώπω τούτω, ας κατηγορήσωσιν αυτόν.
\par 6 Και αφού διέτριψε μεταξύ αυτών υπέρ τας δέκα ημέρας, κατέβη εις Καισάρειαν, και τη επαύριον καθήσας επί του βήματος, προσέταξε να φερθή ο Παύλος.
\par 7 Και αφού ήλθε, παρεστάθησαν οι καταβάντες από Ιεροσολύμων Ιουδαίοι, επιφέροντες κατά του Παύλου πολλάς και βαρείας κατηγορίας, τας οποίας δεν ηδύναντο να αποδείξωσιν·
\par 8 απολογουμένου εκείνου ότι ούτε εις τον νόμον των Ιουδαίων ούτε εις το ιερόν ούτε εις τον Καίσαρα έπραξα τι αμάρτημα.
\par 9 Ο δε Φήστος, θέλων να κάμη χάριν εις τους Ιουδαίους, αποκριθείς προς τον Παύλον είπε· Θέλεις να αναβής εις Ιεροσόλυμα και εκεί να κριθής περί τούτων ενώπιόν μου;
\par 10 Και ο Παύλος είπεν· Επί του βήματος του Καίσαρος παρίσταμαι, όπου πρέπει να κριθώ. Δεν ηδίκησα κατ' ουδέν τους Ιουδαίους, καθώς και συ γνωρίζεις κάλλιστα·
\par 11 διότι εάν αδικώ και έπραξα τι άξιον θανάτου, δεν φεύγω τον θάνατον· αλλ' εάν δεν υπάρχη ουδέν εξ όσων ούτοι με κατηγορούσιν, ουδείς δύναται να με χαρίση εις αυτούς· τον Καίσαρα επικαλούμαι.
\par 12 Τότε ο Φήστος, συνομιλήσας μετά του συμβουλίου, απεκρίθη· Τον Καίσαρα επικαλείσαι, προς τον Καίσαρα θέλεις υπάγει.
\par 13 Και αφού παρήλθον ημέραι τινές, Αγρίππας ο βασιλεύς και η Βερνίκη ήλθον εις Καισάρειαν διά να χαιρετήσωσι τον Φήστον.
\par 14 Ενώ δε διέτριβον εκεί ημέρας πολλάς, ο Φήστος ανέφερε προς τον βασιλέα τα περί του Παύλου, λέγων· Είναι τις άνθρωπος αφημένος εδώ δέσμιος υπό του Φήλικος,
\par 15 περί του οποίου, ότε υπήγα εις Ιεροσόλυμα, οι αρχιερείς και οι πρεσβύτεροι των Ιουδαίων ενεφανίσθησαν εις εμέ, ζητούντες καταδίκην εναντίον αυτού·
\par 16 προς τους οποίους απεκρίθην ότι δεν είναι συνήθεια εις τους Ρωμαίους να παραδίδωσι κατά χάριν ουδένα άνθρωπον εις θάνατον, πριν ο κατηγορούμενος έχη τους κατηγόρους κατά πρόσωπον και λάβη καιρόν απολογίας περί του εγκλήματος.
\par 17 Αφού λοιπόν αυτοί συνήλθον εδώ, χωρίς να κάμω μηδεμίαν αναβολήν την ακόλουθον ημέραν καθήσας επί του βήματος, προσέταξα να φερθή ο άνθρωπος·
\par 18 περί του οποίου οι κατήγοροι παρασταθέντες δεν επέφεραν ουδεμίαν κατηγορίαν εξ όσων εγώ υπενόουν,
\par 19 αλλ' είχον κατ' αυτού ζητήματά τινά περί της ιδίας αυτών δεισιδαιμονίας και περί τινός Ιησού αποθανόντος, τον οποίον ο Παύλος έλεγεν ότι ζη.
\par 20 Απορών δε εγώ εις την περί τούτου ζήτησιν, έλεγον αν θέλη να υπάγη εις Ιερουσαλήμ και εκεί να κριθή περί τούτων.
\par 21 Αλλ' επειδή ο Παύλος επεκαλέσθη να φυλαχθή εις την κρίσιν του Σεβαστού, προσέταξα να φυλάττηται, εωσού πέμψω αυτόν προς τον Καίσαρα.
\par 22 Ο δε Αγρίππας είπε προς τον Φήστον· Ήθελον και εγώ να ακούσω τον άνθρωπον. Και εκείνος· Αύριον, είπε, θέλεις ακούσει αυτόν.
\par 23 Την επαύριον λοιπόν, ότε ήλθεν ο Αγρίππας και η Βερνίκη μετά μεγάλης πομπής και εισήλθον εις το ακροατήριον μετά των χιλιάρχων και των εξόχων ανδρών της πόλεως, προσέταξεν ο Φήστος, και εφέρθη ο Παύλος.
\par 24 Τότε λέγει ο Φήστος· Αγρίππα βασιλεύ και πάντες οι συμπαρευρισκόμενοι μεθ' ημών, θεωρείτε τούτον, περί του οποίου όλον το πλήθος των Ιουδαίων με ώμίλησαν και εν Ιεροσολύμοις και εδώ, καταβοώντες ότι αυτός δεν πρέπει πλέον να ζη.
\par 25 Εγώ δε επειδή εύρον ότι δεν έπραξεν ουδέν άξιον θανάτου, και αυτός ούτος επεκαλέσθη τον Σεβαστόν, απεφάσισα να πέμψω αυτόν.
\par 26 Περί του οποίου δεν έχω ουδέν βέβαιον να γράψω προς τον κύριόν μου· όθεν έφερα αυτόν ενώπιόν σας, και μάλιστα ενώπιον σου, βασιλεύ Αγρίππα, διά να έχω τι να γράψω, αφού γείνη η ανάκρισις.
\par 27 Διότι μοι φαίνεται άλογον, πέμπων δέσμιον, να μη φανερώσω και τα κατ' αυτού εγκλήματα.

\chapter{26}

\par 1 Ο δε Αγρίππας είπε προς τον Παύλον. Έχεις την άδειαν να ομιλήσης υπέρ σεαυτού. Τότε ο Παύλος εκτείνας την χείρα, απελογείτο·
\par 2 Μακάριον νομίζω εμαυτόν, βασιλεύ Αγρίππα, μέλλων να απολογηθώ ενώπιόν σου σήμερον περί πάντων εις όσα εγκαλούμαι υπό των Ιουδαίων,
\par 3 μάλιστα επειδή γνωρίζεις πάντα τα παρά τοις Ιουδαίοις έθιμα και ζητήματα· όθεν δέομαί σου να με ακούσης μετά μακροθυμίας.
\par 4 Την εκ νεότητος λοιπόν ζωήν μου, την οποίαν απ' αρχής έζησα μεταξύ του έθνους μου εν Ιεροσολύμοις, εξεύρουσι πάντες οι Ιουδαίοι,
\par 5 επειδή με γνωρίζουσιν εξ αρχής, εάν θέλωσι να μαρτυρήσωσιν, ότι κατά την ακριβεστάτην αίρεσιν της θρησκείας ημών έζησα Φαρισαίος.
\par 6 Και τώρα παρίσταμαι κρινόμενος διά την ελπίδα της επαγγελίας της γενομένης υπό του Θεού προς τους πατέρας ημών,
\par 7 εις την οποίαν το δωδεκάφυλον ημών γένος, λατρεύον εκτενώς τον Θεόν νύκτα και ημέραν, ελπίζει να καταντήση· περί ταύτης της ελπίδος εγκαλούμαι, βασιλεύ Αγρίππα, υπό των Ιουδαίων.
\par 8 Τι απίστευτον κρίνεται εις εσάς, ότι ο Θεός ανιστά νεκρούς;
\par 9 Εγώ μεν εστοχάσθην κατ' εμαυτόν ότι έπρεπε να πράξω πολλά εναντία εις το όνομα του Ιησού του Ναζωραίου·
\par 10 το οποίον και έπραξα εν Ιεροσολύμοις, και πολλούς των αγίων εγώ κατέκλεισα εις φυλακάς, λαβών την εξουσίαν παρά των αρχιερέων, και ότε εφονεύοντο έδωκα ψήφον κατ' αυτών.
\par 11 Και εν πάσαις ταις συναγωγαίς πολλάκις τιμωρών αυτούς ηνάγκαζον να βλασφημώσι, και καθ' υπερβολήν μαινόμενος εναντίον αυτών κατεδίωκον έως και εις τας έξω πόλεις.
\par 12 Εν τούτοις δε, ότε ηρχόμην εις την Δαμασκόν μετ' εξουσίας και επιτροπής της παρά των αρχιερέων,
\par 13 εν τω μέσω της ημέρας είδον καθ' οδόν, βασιλεύ, φως ουρανόθεν υπερβαίνον την λαμπρότητα του ηλίου, το οποίον έλαμψε περί εμέ και τους οδοιπορούντας μετ' εμού·
\par 14 και ενώ κατεπέσομεν πάντες εις την γην, ήκουσα φωνήν λαλούσαν προς με και λέγουσαν εις την Εβραϊκήν διάλεκτον· Σαούλ Σαούλ, τι με διώκεις; σκληρόν σοι είναι να λακτίζης προς κέντρα.
\par 15 Εγώ δε είπον· Τις είσαι, Κύριε; Και εκείνος είπεν· Εγώ είμαι ο Ιησούς, τον οποίον συ διώκεις.
\par 16 Αλλά σηκώθητι και στήθι επί τους πόδας σου· επειδή διά τούτο εφάνην εις σε, διά να σε καταστήσω υπηρέτην και μάρτυρα και όσων είδες και περί όσων θέλω φανερωθή εις σε,
\par 17 εκλέγων σε εκ του λαού και των εθνών, εις τα οποία τώρα σε αποστέλλω
\par 18 διά να ανοίξης τους οφθαλμούς αυτών, ώστε να επιστρέψωσιν από του σκότους εις το φως και από της εξουσίας του Σατανά προς τον Θεόν, διά να λάβωσιν άφεσιν αμαρτιών και κληρονομίαν μεταξύ των ηγιασμένων διά της εις εμέ πίστεως.
\par 19 Όθεν, βασιλεύ Αγρίππα, δεν έγεινα απειθής εις την ουράνιον οπτασίαν,
\par 20 αλλ' εκήρυττον πρώτον εις τους εν Δαμασκώ και Ιεροσολύμοις και εις πάσαν την γην της Ιουδαίας, και έπειτα εις τα έθνη, να μετανοώσι και να επιστρέφωσιν εις τον Θεόν, πράττοντες έργα άξια της μετανοίας.
\par 21 Διά ταύτα οι Ιουδαίοι συλλαβόντες με εν τω ιερώ, επεχείρουν να με φονεύσωσιν.
\par 22 Αξιωθείς όμως της βοηθείας της παρά του Θεού, ίσταμαι έως της ημέρας ταύτης μαρτυρών προς μικρόν τε και μεγάλον, μη λέγων μηδέν εκτός των όσα ελάλησαν οι προφήται και ο Μωϋσής ότι έμελλον να γείνωσιν,
\par 23 ότι ο Χριστός έμελλε να πάθη, ότι πρώτος αναστάς εκ νεκρών μέλλει να κηρύξη φως εις τον λαόν και εις τα έθνη.
\par 24 Ενώ δε αυτός απελογείτο ταύτα, ο Φήστος είπε με μεγάλην φωνήν· Μαίνεσαι, Παύλε, τα πολλά γράμματα σε καταφέρουσιν εις μανίαν.
\par 25 Ο δε, Δεν μαίνομαι, είπε, κράτιστε Φήστε, αλλά προφέρω λόγους αληθείας και νοός υγιαίνοντος.
\par 26 Διότι έχει γνώσιν περί τούτων ο βασιλεύς, προς τον οποίον και λαλώ μετά παρρησίας· επειδή είμαι πεπεισμένος ότι δεν λανθάνει αυτόν ουδέν τούτων, διότι τούτο δεν είναι πεπραγμένον εν γωνία.
\par 27 Πιστεύεις, βασιλεύ Αγρίππα, εις τους προφήτας; εξεύρω ότι πιστεύεις.
\par 28 Και ο Αγρίππας είπε προς τον Παύλον· Παρ' ολίγον με πείθεις να γείνω Χριστιανός.
\par 29 Και ο Παύλος είπεν· Ήθελον εύχεσθαι προς τον Θεόν, ουχί μόνον συ, αλλά και πάντες οι σήμερον ακούοντές με, να γείνωσι και παρ' ολίγον και παρά πολύ τοιούτοι οποίος και εγώ είμαι, παρεκτός των δεσμών τούτων.
\par 30 Και αφού αυτός είπε ταύτα, εσηκώθη ο βασιλεύς και ο ηγεμών και η Βερνίκη και οι συγκαθήμενοι μετ' αυτών,
\par 31 και αναχωρήσαντες ελάλουν προς αλλήλους, λέγοντες ότι ουδέν άξιον θανάτου ή δεσμών πράττει ο άνθρωπος ούτος.
\par 32 Ο δε Αγρίππας είπε προς τον Φήστον· Ο άνθρωπος ούτος ηδύνατο να απολυθή, εάν δεν είχεν επικαλεσθή τον Καίσαρα.

\chapter{27}

\par 1 Αφού δε απεφασίσθη να αποπλεύσωμεν εις την Ιταλίαν, παρέδωκαν τον Παύλον και τινάς άλλους δεσμίους εις εκατόνταρχον Ιούλιον ονομαζόμενον, εκ του τάγματος του Σεβαστού λεγομένου.
\par 2 Και αφού επέβημεν εις πλοίον Αδραμυττηνόν, εσηκώθημεν μέλλοντες να παραπλεύσωμεν τους κατά την Ασίαν τόπους, έχοντες μεθ' ημών Αρίσταρχον τον Μακεδόνα τον εκ Θεσσαλονίκης·
\par 3 και την άλλην ημέραν εφθάσαμεν εις Σιδώνα· και ο Ιούλιος φιλανθρώπως φερόμενος προς τον Παύλον επέτρεψεν εις αυτόν να υπάγη προς τους φίλους αυτού και να λάβη περίθαλψιν.
\par 4 Και εκείθεν σηκωθέντες υπεπλεύσαμεν την Κύπρον, επειδή ήσαν εναντίοι οι άνεμοι,
\par 5 και διαπλεύσαντες το πέλαγος της Κιλικίας και Παμφυλίας, ήλθομεν εις τα Μύρα της Λυκίας.
\par 6 Και εκεί ευρών ο εκατόνταρχος πλοίον Αλεξανδρινόν, το οποίον έπλεεν εις την Ιταλίαν, επεβίβασεν ημάς εις αυτό·
\par 7 βραδυπλοούντες δε ικανάς ημέρας και μόλις φθάσαντες εις την Κνίδον, επειδή δεν μας άφινεν ο άνεμος, υπεπλεύσαμεν την Κρήτην κατά την Σαλμώνην,
\par 8 και μόλις παραπλεύσαντες αυτήν, ήλθομεν εις τόπον τινά ονομαζόμενον Καλούς Λιμένας, πλησίον του οποίου ήτο η πόλις Λασαία.
\par 9 Επειδή δε παρήλθεν ικανός καιρός και ο πλούς ήτο ήδη επικίνδυνος, διότι και η νηστεία είχεν ήδη παρέλθει, συνεβούλευεν ο Παύλος,
\par 10 λέγων προς αυτούς· Άνδρες, βλέπω ότι ο πλούς μέλλει να γείνη με κακοπάθειαν και πολλήν ζημίαν ουχί μόνον του φορτίου και του πλοίου, αλλά και των ψυχών ημών.
\par 11 Αλλ' ο εκατόνταρχος επείθετο μάλλον εις τον κυβερνήτην και εις τον ναύκληρον παρά εις τα υπό του Παύλου λεγόμενα.
\par 12 Και επειδή ο λιμήν δεν ήτο επιτήδειος εις παραχειμασίαν, οι πλειότεροι εγνωμοδότησαν να σηκωθώσι και εκείθεν, ώστε φθάσαντες αν ηδύναντο εις Φοίνικα, λιμένα της Κρήτης βλέποντα προς τον λίβα άνεμον και προς τον χώρον, να παραχειμάσωσιν εκεί.
\par 13 Και ότε έπνευσεν ολίγον νότος, νομίσαντες ότι επέτυχον του σκοπού, ανέσυραν την άγκυραν και παρέπλεον πλησίον την Κρήτην.
\par 14 Πλην μετ' ολίγον προσέβαλε κατ' αυτής άνεμος τυφωνικός ο λεγόμενος Ευροκλύδων.
\par 15 Και επειδή το πλοίον συνηρπάσθη και δεν ηδύνατο να αντέχη προς τον άνεμον, αφεθέντες εφερόμεθα.
\par 16 Και τρέξαντες υπό νησίδιόν τι ονομαζόμενον Κλαύδην, μόλις ηδυνήθημεν να βάλωμεν εις την εξουσίαν μας την λέμβον,
\par 17 την οποίαν αφού ανέλαβον μετεχειρίζοντο βοηθήματα, ζώνοντες υποκάτωθεν το πλοίον· και φοβούμενοι μη εκπέσωσιν εις την Σύρτιν, κατεβίβασαν τα πανία και εφέροντο ούτως.
\par 18 Και επειδή εχειμαζόμεθα σφοδρώς, την ακόλουθον ημέραν έκαμνον χύσιν,
\par 19 και την τρίτην με τας ιδίας ημών χείρας ερρίψαμεν τα σκεύη του πλοίου·
\par 20 και επειδή διά πολλών ημερών δεν εφαίνοντο ούτε ήλιος ούτε άστρα, και χειμών βαρύς επέκειτο, πάσα ελπίς σωτηρίας αφηρείτο πλέον αφ' ημών.
\par 21 Μετά δε πολυήμερον ασιτίαν σταθείς ο Παύλος εν τω μέσω αυτών, είπεν· Έπρεπεν, ω άνδρες, να μου υπακούσητε και να μη σηκωθήτε από της Κρήτης και ούτως ηθέλομεν αποφύγει την κακοπάθειαν ταύτην και την ζημίαν.
\par 22 Αλλά και ήδη σας παραινώ να έχητε θάρρος· διότι εξ υμών ουδεμία ψυχή δεν θέλει χαθή, ειμή μόνον το πλοίον.
\par 23 Διότι την νύκτα ταύτην εφάνη εις εμέ άγγελος του Θεού, του οποίου είμαι, τον οποίον και λατρεύω,
\par 24 λέγων· μη φοβού, Παύλε· πρέπει να παρασταθής ενώπιον του Καίσαρος· και ιδού, ο Θεός σοι εχάρισε πάντας τους πλέοντας μετά σου.
\par 25 Διά τούτο θαρρείτε, άνδρες· διότι πιστεύω εις τον Θεόν ότι ούτω θέλει γείνει, καθ' ον τρόπον ελαλήθη προς εμέ.
\par 26 Πρέπει δε να πέσωμεν εις νήσόν τινά.
\par 27 Ότε δε ήλθεν η δεκάτη τετάρτη νυξ, ενώ παρεφερόμεθα εν τη Αδριατική θαλάσση, περί το μέσον της νυκτός εσυμπέραινον οι ναύται ότι πλησιάζουσιν εις τόπον τινά.
\par 28 Και ρίψαντες την βολίδα εύρον είκοσι οργυιάς, και αφού επροχώρησαν ολίγον διάστημα, ρίψαντες και πάλιν την βολίδα εύρον οργυιάς δεκαπέντε·
\par 29 και φοβούμενοι μήπως πέσωμεν έξω εις τραχείς τόπους, ρίψαντες τέσσαρας αγκύρας από της πρύμνης, ηύχοντο να γείνη ημέρα.
\par 30 Επειδή δε οι ναύται εζήτουν να φύγωσιν εκ του πλοίου και κατεβίβασαν την λέμβον εις την θάλασσαν, επί προφάσει ότι έμελλον να εκτείνωσιν αγκύρας εκ της πρώρας,
\par 31 ο Παύλος είπε προς τον εκατόνταρχον και προς τους στρατιώτας· Εάν ούτοι δεν μείνωσιν εν τω πλοίω, σεις δεν δύνασθε να σωθήτε.
\par 32 Τότε οι στρατιώται απέκοψαν τα σχοινία της λέμβου και αφήκαν αυτήν να πέση έξω.
\par 33 Έως δε να εξημερώση, ο Παύλος παρεκάλει πάντας να λάβωσι τροφήν τινά, λέγων· Δεκατέσσαρας ημέρας σήμερον προσδοκώντες διαμένετε νηστικοί, και δεν εφάγετε ουδέν.
\par 34 Διά τούτο σας παρακαλώ να λάβητε τροφήν· διότι τούτο είναι αναγκαίον προς την σωτηρίαν σας· επειδή ουδενός από σας δεν θέλει πέσει θριξ εκ της κεφαλής.
\par 35 Αφού δε είπε ταύτα και έλαβεν άρτον, ευχαρίστησε τον Θεόν ενώπιον πάντων και κόψας ήρχισε να τρώγη.
\par 36 Λαβόντες δε πάντες θάρρος, έλαβον και αυτοί τροφήν·
\par 37 ήμεθα δε εν τω πλοίω ψυχαί όλαι διακόσιαι εβδομήκοντα εξ.
\par 38 Αφού δε εχορτάσθησαν από τροφής ελάφρυνον το πλοίον, ρίπτοντες τον σίτον εις την θάλασσαν.
\par 39 Και ότε έγεινεν ημέρα, δεν εγνώριζον την γην, παρετήρουν όμως κόλπον τινά έχοντα αιγιαλόν, εις τον οποίον εβουλεύθησαν, αν ηδύναντο, να εξώσωσι το πλοίον.
\par 40 Και κόψαντες τας αγκύρας, αφήκαν το πλοίον εις την θάλασσαν, λύσαντες ενταυτώ τους δεσμούς των πηδαλίων, και υψώσαντες τον αρτέμονα προς τον άνεμον, κατηυθύνοντο εις τον αιγιαλόν.
\par 41 Περιπεσόντες δε εις τόπον, όπου συνήρχοντο δύο θάλασσαι, έρριψαν έξω το πλοίον, και η μεν πρώρα εκάθησε και έμεινεν ασάλευτος, η δε πρύμνη διελύετο υπό της βίας των κυμάτων.
\par 42 Εβουλεύθησαν δε οι στρατιώται να θανατώσωσι τους δεσμίους, διά να μη φύγη μηδείς κολυμβήσας.
\par 43 Αλλ' ο εκατόνταρχος, θέλων να διασώση τον Παύλον, εμπόδισεν αυτούς από του σκοπού και προσέταξεν, όσοι ηδύναντο να κολυμβώσι να ριφθώσι πρώτοι και να εκβώσιν εις την γην,
\par 44 οι δε λοιποί άλλοι μεν επί σανίδων, άλλοι δε επί τινών λειψάνων του πλοίου, και ούτω διεσώθησαν πάντες εις την γην.

\chapter{28}

\par 1 Και αφού διεσώθησαν, τότε εγνώρισαν ότι η νήσος ονομάζεται Μελίτη.
\par 2 Οι δε βάρβαροι έδειξαν εις ημάς ου την τυχούσαν φιλανθρωπίαν· διότι ανάψαντες πυράν, υπεδέχθησαν πάντας ημάς διά την επικειμένην βροχήν και διά το ψύχος.
\par 3 Ότε δε ο Παύλος, συσσωρεύσας πλήθος φρυγάνων, έβαλεν επί την πυράν, έχιδνα εξελθούσα εκ της θερμότητος προσεκολλήθη εις την χείρα αυτού.
\par 4 Ως δε είδον οι βάρβαροι το θηρίον κρεμάμενον εκ της χειρός αυτού, έλεγον προς αλλήλους· Βεβαίως φονεύς είναι ο άνθρωπος ούτος, τον οποίον διασωθέντα εκ της θαλάσσης η θεία δίκη δεν αφήκε να ζη.
\par 5 Και αυτός μεν απετίναξε το θηρίον εις το πυρ και δεν έπαθεν ουδέν κακόν·
\par 6 εκείνοι δε επρόσμενον ότι έμελλε να πρησθή ή εξαίφνης να πέση κάτω νεκρός. Αφού όμως επρόσμενον πολλήν ώραν και έβλεπον ότι ουδέν κακόν εγίνετο εις αυτόν, μεταβαλόντες στοχασμόν έλεγον ότι είναι Θεός.
\par 7 Εις τα πέριξ δε του τόπου εκείνου ήσαν κτήματα του πρώτου της νήσου ονομαζομένου Ποπλίου, όστις αναδεχθείς ημάς, εξένισε φιλοφρόνως τρεις ημέρας.
\par 8 Συνέβη δε να ήναι κατάκειτος ο πατήρ του Ποπλίου, πάσχων πυρετόν και δυσεντερίαν· προς τον οποίον εισελθών ο Παύλος και προσευχηθείς, επέθεσεν επ' αυτόν τας χείρας και ιάτρευσεν αυτόν.
\par 9 Τούτου λοιπόν γενομένου και οι λοιποί, όσοι είχον ασθενείας εν τη νήσω, προσήρχοντο και εθεραπεύοντο·
\par 10 οίτινες και με τιμάς πολλάς ετίμησαν ημάς και ότε εμέλλομεν να αναχωρήσωμεν, εφωδίασαν με τα χρειώδη.
\par 11 Μετά δε τρεις μήνας απεπλεύσαμεν επί πλοίου Αλεξανδρινού, με σημαίαν των Διοσκούρων, το οποίον είχε παραχειμάσει εν τη νήσω,
\par 12 και φθάσαντες εις τας Συρακούσας, εμείναμεν τρεις ημέρας·
\par 13 εκείθεν δε περιπλεύσαντες κατηντήσαμεν εις Ρήγιον, και μετά μίαν ημέραν, πνεύσαντος νότου, την δευτέραν ημέραν ήλθομεν εις Ποτιόλους·
\par 14 όπου ευρόντες αδελφούς, παρεκαλέσθημεν να μείνωμεν παρ' αυτοίς επτά ημέρας, και ούτως ήλθομεν εις την Ρώμην.
\par 15 Εκείθεν δε ακούσαντες οι αδελφοί τα περί ημών, εξήλθον εις απάντησιν ημών έως του Αππίου Φόρου και των Τριών Ταβερνών, τους οποίους ιδών ο Παύλος, ηυχαρίστησε τον Θεόν και έλαβε θάρρος.
\par 16 Ότε δε ήλθομεν εις Ρώμην, ο εκατόνταρχος παρέδωκε τους δεσμίους εις τον στρατοπεδάρχην· εις τον Παύλον όμως συνεχωρήθη να μένη καθ' εαυτόν μετά του στρατιώτου, όστις εφύλαττεν αυτόν.
\par 17 Μετά δε τρεις ημέρας συνεκάλεσεν ο Παύλος τους όντας των Ιουδαίων πρώτους· και αφού συνήλθον, έλεγε προς αυτούς· Άνδρες αδελφοί, εγώ ουδέν εναντίον πράξας εις τον λαόν ή εις τα έθιμα τα πατρώα, παρεδόθην εξ Ιεροσολύμων δέσμιος εις τας χείρας των Ρωμαίων·
\par 18 οίτινες αφού με ανέκριναν, ήθελον να με απολύσωσι, διότι ουδεμία αιτία θανάτου υπήρχεν εν εμοί.
\par 19 Επειδή δε αντέλεγον οι Ιουδαίοι, ηναγκάσθην να επικαλεσθώ τον Καίσαρα, ουχί ως έχων να κατηγορήσω κατά τι το έθνος μου.
\par 20 Διά ταύτην λοιπόν την αιτίαν σας εκάλεσα, διά να σας ίδω και ομιλήσω· διότι ένεκα της ελπίδος του Ισραήλ φορώ ταύτην την άλυσιν.
\par 21 Οι δε είπον προς αυτόν· Ημείς ούτε γράμματα ελάβομεν περί σου από της Ιουδαίας, ούτε ελθών τις εκ των αδελφών απήγγειλεν ή ελάλησέ τι κακόν περί σου.
\par 22 Επιθυμούμεν δε να ακούσωμεν παρά σου τι φρονείς διότι περί της αιρέσεως ταύτης είναι γνωστόν εις ημάς ότι πανταχού αντιλέγεται.
\par 23 Και αφού διώρισαν εις αυτόν ημέραν, ήλθον προς αυτόν πολλοί εις το κατάλυμα, εις τους οποίους εξέθεσε διά μαρτυριών την βασιλείαν του Θεού και έπειθεν αυτούς εις τα περί του Ιησού από τε του νόμου του Μωϋσέως και των προφητών από πρωΐ έως εσπέρας.
\par 24 Και άλλοι μεν επείθοντο εις τα λεγόμενα, άλλοι δε ηπίστουν.
\par 25 Ασύμφωνοι δε όντες προς αλλήλους ανεχώρουν, αφού ο Παύλος είπεν ένα λόγον, ότι καλώς ελάλησε το Πνεύμα το Άγιον προς τους πατέρας ημών διά Ησαΐου του προφήτου,
\par 26 λέγον· Ύπαγε προς τον λαόν τούτον και ειπέ· Με την ακοήν θέλετε ακούσει και δεν θέλετε εννοήσει, και βλέποντες θέλετε ιδεί και δεν θέλετε καταλάβει·
\par 27 διότι επαχύνθη η καρδία του λαού τούτου, και με τα ώτα βαρέως ήκουσαν και τους οφθαλμούς αυτών έκλεισαν, μήποτε ίδωσι με τους οφθαλμούς και ακούσωσι με τα ώτα και νοήσωσι με την καρδίαν και επιστρέψωσι, και ιατρεύσω αυτούς.
\par 28 Γνωστόν λοιπόν έστω εις εσάς ότι εις τα έθνη απεστάλη το σωτήριον του Θεού, αυτοί και θέλουσιν ακούσει.
\par 29 Και αφού είπε ταύτα ανεχώρησαν οι Ιουδαίοι έχοντες πολλήν συζήτησιν προς αλλήλους.
\par 30 Έμεινε δε ο Παύλος δύο ολόκληρα έτη εν ιδιαιτέρα μισθωτή οικία και εδέχετο πάντας τους ερχομένους προς αυτόν,
\par 31 κηρύττων την βασιλείαν του Θεού και διδάσκων μετά πάσης παρρησίας ακωλύτως τα περί του Κυρίου Ιησού Χριστού.


\end{document}