\begin{document}

\title{Romans}


\chapter{1}

\par Παύλος, δούλος Ιησού Χριστού, προσκεκλημένος απόστολος, κεχωρισμένος διά το ευαγγέλιον του Θεού,
\par 2 το οποίον προϋπεσχέθη διά των προφητών αυτού εν ταις αγίαις γραφαίς,
\par 3 περί του Υιού αυτού, όστις εγεννήθη εκ σπέρματος Δαβίδ κατά σάρκα,
\par 4 και απεδείχθη Υιός Θεού εν δυνάμει κατά το πνεύμα της αγιωσύνης διά της εκ νεκρών αναστάσεως, Ιησού Χριστού του Κυρίου ημών,
\par 5 διά του οποίου ελάβομεν χάριν και αποστολήν εις υπακοήν πίστεως πάντων των εθνών υπέρ του ονόματος αυτού,
\par 6 μεταξύ των οποίων είσθε και σεις προσκεκλημένοι του Ιησού Χριστού,
\par 7 προς πάντας τους όντας εν Ρώμη αγαπητούς του Θεού, προσκεκλημένους αγίους, χάρις είη υμίν και ειρήνη από Θεού Πατρός ημών και Κυρίου Ιησού Χριστού.
\par 8 Πρώτον μεν ευχαριστώ τον Θεόν μου διά Ιησού Χριστού υπέρ πάντων υμών, διότι η πίστις σας κηρύττεται εν όλω τω κόσμω.
\par 9 Επειδή μάρτυς μου είναι ο Θεός, τον οποίον λατρεύω διά του πνεύματός μου εν τω ευαγγελίω του Υιού αυτού, ότι αδιαλείπτως σας ενθυμούμαι,
\par 10 δεόμενος πάντοτε εν ταις προσευχαίς μου να αξιωθώ ήδη ποτέ διά του θελήματος του Θεού να έλθω προς εσάς.
\par 11 Διότι επιποθώ να σας ίδω, διά να σας μεταδώσω χάρισμά τι πνευματικόν προς στήριξιν υμών,
\par 12 τούτο δε είναι, το να συμπαρηγορηθώ μεταξύ σας διά της κοινής πίστεως υμών τε και εμού.
\par 13 Δεν θέλω δε να αγνοήτε, αδελφοί, ότι πολλάκις εμελέτησα να έλθω προς εσάς, εμποδίσθην όμως μέχρι τούδε, διά να απολαύσω καρπόν τινά και μεταξύ σας, καθώς και μεταξύ των λοιπών εθνών.
\par 14 Χρεώστης είμαι προς Ελληνάς τε και βαρβάρους, σοφούς τε και ασόφους·
\par 15 ούτω πρόθυμος είμαι το κατ' εμέ να κηρύξω το ευαγγέλιον και προς εσάς τους εν Ρώμη.
\par 16 Διότι δεν αισχύνομαι το ευαγγέλιον του Χριστού· επειδή είναι δύναμις Θεού προς σωτηρίαν εις πάντα τον πιστεύοντα Ιουδαίόν τε, πρώτον και Έλληνα.
\par 17 Διότι δι' αυτού αποκαλύπτεται η δικαιοσύνη του Θεού εκ πίστεως εις πίστιν, καθώς είναι γεγραμμένον· Ο δε δίκαιος θέλει ζήσει εκ πίστεως.
\par 18 Διότι οργή Θεού αποκαλύπτεται απ' ουρανού επί πάσαν ασέβειαν και αδικίαν ανθρώπων, οίτινες κατακρατούσι την αλήθειαν εν αδικία.
\par 19 Επειδή ό,τι δύναται να γνωρισθή περί Θεού είναι φανερόν εν αυτοίς, διότι ο Θεός εφανέρωσε τούτο προς αυτούς.
\par 20 Επειδή τα αόρατα αυτού βλέπονται φανερώς από κτίσεως κόσμου νοούμενα διά των ποιημάτων, η τε αΐδιος αυτού δύναμις και η θειότης, ώστε αυτοί είναι αναπολόγητοι.
\par 21 Διότι γνωρίσαντες τον Θεόν, δεν εδόξασαν ως Θεόν ουδέ ευχαρίστησαν, αλλ' εματαιώθησαν εν τοις διαλογισμοίς αυτών, και εσκοτίσθη η ασύνετος αυτών καρδία·
\par 22 λέγοντες ότι είναι σοφοί εμωράνθησαν,
\par 23 και ήλλαξαν την δόξαν του αφθάρτου Θεού εις ομοίωμα εικόνος φθαρτού ανθρώπου και πετεινών και τετραπόδων και ερπετών.
\par 24 Διά τούτο και παρέδωκεν αυτούς ο Θεός διά των επιθυμιών των καρδιών αυτών εις ακαθαρσίαν, ώστε να ατιμάζωνται τα σώματα αυτών μεταξύ αυτών.
\par 25 Οίτινες μετήλλαξαν την αλήθειαν του Θεού εις το ψεύδος, και εσεβάσθησαν και ελάτρευσαν την κτίσιν μάλλον παρά τον κτίσαντα, όστις είναι ευλογητός εις τους αιώνας· αμήν.
\par 26 Διά τούτο παρέδωκεν αυτούς ο Θεός εις πάθη ατιμίας· διότι και αι γυναίκες αυτών μετήλλαξαν την φυσικήν χρήσιν εις την παρά φύσιν·
\par 27 ομοίως δε και οι άνδρες, αφήσαντες την φυσικήν χρήσιν της γυναικός, εξεκαύθησαν εις την επιθυμίαν αυτών προς αλλήλους, πράττοντες την ασχημοσύνην άρσενες εις άρσενας και απολαμβάνοντες εις εαυτούς την πρέπουσαν αντιμισθίαν της πλάνης αυτών.
\par 28 Και καθώς απεδοκίμασαν το να έχωσιν επίγνωσιν του Θεού, παρέδωκεν αυτούς ο Θεός εις αδόκιμον νούν, ώστε να πράττωσι τα μη πρέποντα,
\par 29 πλήρεις όντες πάσης αδικίας, πορνείας, πονηρίας, πλεονεξίας, κακίας, γέμοντες φθόνου, φόνου, έριδος, δόλου, κακοηθείας·
\par 30 ψιθυρισταί, κατάλαλοι, μισόθεοι, υβρισταί, υπερήφανοι, αλαζόνες, εφευρεταί κακών, απειθείς εις τους γονείς,
\par 31 ασύνετοι, παραβάται συνθηκών, άσπλαγχνοι, αδιάλλακτοι, ανελεήμονες·
\par 32 οίτινες ενώ γνωρίζουσι την δικαιοσύνην του Θεού, ότι οι πράττοντες τα τοιαύτα είναι άξιοι θανάτου, ουχί μόνον πράττουσιν αυτά, αλλά και συνευδοκούσιν εις τους πράττοντας.

\chapter{2}

\par Διά τούτο αναπολόγητος είσαι, ω άνθρωπε, πας όστις κρίνεις· διότι εις ό,τι κρίνεις τον άλλον, σεαυτόν κατακρίνεις· επειδή τα αυτά πράττεις συ ο κρίνων.
\par 2 Εξεύρομεν δε ότι η κρίσις του Θεού είναι κατά αλήθειαν εναντίον των πραττόντων τα τοιαύτα.
\par 3 Και νομίζεις τούτο, ω άνθρωπε, συ ο κρίνων τους πράττοντας τα τοιαύτα και πράττων αυτά, ότι θέλεις εκφύγει την κρίσιν του Θεού;
\par 4 Η καταφρονείς τον πλούτον της χρηστότητος αυτού και της υπομονής και της μακροθυμίας, αγνοών ότι η χρηστότης του Θεού σε φέρει εις μετάνοιαν;
\par 5 διά δε την σκληρότητά σου και αμετανόητον καρδίαν θησαυρίζεις εις σεαυτόν οργήν εν τη ημέρα της οργής και της αποκαλύψεως της δικαιοκρισίας του Θεού,
\par 6 όστις θέλει αποδώσει εις έκαστον κατά τα έργα αυτού,
\par 7 εις μεν τους ζητούντας δι' υπομονής έργου αγαθού, δόξαν και τιμήν και αφθαρσίαν ζωήν αιώνιον,
\par 8 εις δε τους φιλονείκους και απειθούντας μεν εις την αλήθειαν, πειθομένους δε εις την αδικίαν θέλει είσθαι θυμός και οργή,
\par 9 θλίψις και στενοχωρία επί πάσαν ψυχήν ανθρώπου του εργαζομένου το κακόν, Ιουδαίου τε πρώτον και Ελληνος·
\par 10 δόξα δε και τιμή και ειρήνη εις πάντα τον εργαζόμενον το αγαθόν, Ιουδαίόν τε πρώτον και Ελληνα·
\par 11 επειδή δεν είναι προσωποληψία παρά τω Θεώ.
\par 12 Διότι όσοι ημάρτησαν χωρίς νόμου, θέλουσι και απολεσθή χωρίς νόμου· και όσοι ημάρτησαν υπό νόμον, θέλουσι κριθή διά νόμου.
\par 13 Διότι δεν είναι δίκαιοι παρά τω Θεώ οι ακροαταί του νόμου, αλλ' οι εκτελεσταί του νόμου θέλουσι δικαιωθή.
\par 14 Επειδή όταν οι εθνικοί οι μη έχοντες νόμον πράττωσιν εκ φύσεως τα του νόμου, ούτοι νόμον μη έχοντες είναι νόμος εις εαυτούς,
\par 15 οίτινες δεικνύουσι το έργον του νόμου γεγραμμένον εν ταις καρδίαις αυτών, έχοντες συμμαρτυρούσαν την συνείδησιν αυτών και τους λογισμούς κατηγορούντας ή και απολογουμένους μεταξύ αλλήλων,
\par 16 εν τη ημέρα ότε θέλει κρίνει ο Θεός τα κρυπτά των ανθρώπων διά του Ιησού Χριστού κατά το ευαγγέλιόν μου.
\par 17 Ιδού, συ επονομάζεσαι Ιουδαίος και επαναπαύεσαι εις τον νόμον και καυχάσαι εις τον Θεόν,
\par 18 και γνωρίζεις το θέλημα αυτού και διακρίνεις τα διαφέροντα, διδασκόμενος υπό του νόμου,
\par 19 και έχεις πεποίθησιν εις σεαυτόν ότι είσαι οδηγός τυφλών, φως των εν σκότει,
\par 20 παιδευτής αφρόνων, διδάσκαλος νηπίων, έχων τον τύπον της γνώσεως και της αληθείας εν τω νόμω.
\par 21 Ο διδάσκων λοιπόν άλλον σεαυτόν δεν διδάσκεις; ο κηρύττων να μη κλέπτωσι κλέπτεις;
\par 22 ο λέγων να μη μοιχεύωσι μοιχεύεις; ο βδελυττόμενος τα είδωλα ιεροσυλείς;
\par 23 ο καυχώμενος εις τον νόμον, ατιμάζεις τον Θεόν διά της παραβάσεως του νόμου;
\par 24 Διότι το όνομα του Θεού εξ αιτίας σας βλασφημείται μεταξύ των εθνών, καθώς είναι γεγραμμένον.
\par 25 Επειδή ωφελεί μεν η περιτομή, εάν εκτελής τον νόμον· εάν όμως ήσαι παραβάτης του νόμου, η περιτομή σου έγεινεν ακροβυστία.
\par 26 Εάν λοιπόν ο απερίτμητος φυλάττη τα διατάγματα του νόμου, η ακροβυστία αυτού δεν θέλει λογισθή αντί περιτομής;
\par 27 και ο εκ φύσεως απερίτμητος, εκτελών τον νόμον, θέλει κρίνει σε όστις, έχων το γράμμα του νόμου και την περιτομήν, είσαι παραβάτης του νόμου.
\par 28 Διότι Ιουδαίος δεν είναι ο εν τω φανερώ Ιουδαίος, ουδέ περιτομή η εν τω φανερώ η γινομένη εν σαρκί,
\par 29 αλλ' Ιουδαίος είναι ο εν τω κρυπτώ Ιουδαίος, και περιτομή η της καρδίας κατά πνεύμα, ουχί κατά γράμμα, του οποίου ο έπαινος είναι ουχί εξ ανθρώπων, αλλ' εκ του Θεού.

\chapter{3}

\par Τις λοιπόν η υπεροχή του Ιουδαίου, ή τις η ωφέλεια της περιτομής;
\par 2 Πολλή κατά πάντα τρόπον. Πρώτον μεν διότι εις τους Ιουδαίους ενεπιστεύθησαν τα λόγια του Θεού.
\par 3 Επειδή αν τινές δεν επίστευσαν, τι εκ τούτου; μήπως η απιστία αυτών θέλει καταργήσει την πίστιν του Θεού;
\par 4 Μη γένοιτο. Αλλ' έστω ο Θεός αληθής, πας δε άνθρωπος ψεύστης, καθώς είναι γεγραμμένον· Διά να δικαιωθής εν τοις λόγοις σου και να νικήσης, όταν κρίνησαι.
\par 5 Εάν δε η αδικία ημών δεικνύη την δικαιοσύνην του Θεού, τι θέλομεν ειπεί; μήπως είναι άδικος ο Θεός ο επιφέρων την οργήν; ως άνθρωπος λαλώ.
\par 6 Μη γένοιτο· επειδή πως θέλει κρίνει ο Θεός τον κόσμον;
\par 7 Διότι εάν η αλήθεια του Θεού επερίσσευσε προς δόξαν αυτού διά του εμού ψεύσματος, διά τι πλέον εγώ κρίνομαι ως αμαρτωλός,
\par 8 και καθώς βλασφημούμεθα και καθώς κηρύττουσί τινές, ότι ημείς λέγομεν, Διά τι να μη πράττωμεν τα κακά, διά να έλθωσι τα αγαθά; των οποίων η κατάκρισις είναι δικαία.
\par 9 Τι λοιπόν; υπερέχομεν των εθνικών; Ουχί βεβαίως· διότι προεξηλέγξαμεν Ιουδαίους τε και Έλληνας, ότι είναι πάντες υπό αμαρτίαν,
\par 10 καθώς είναι γεγραμμένον Ότι δεν υπάρχει δίκαιος ουδέ εις,
\par 11 δεν υπάρχει τις έχων σύνεσιν· δεν υπάρχει τις εκζητών τον Θεόν.
\par 12 Πάντες εξέκλιναν, ομού εξηχρειώθησαν· δεν υπάρχει ο πράττων αγαθόν, δεν υπάρχει ουδέ εις.
\par 13 Τάφος ανεώγμένος είναι ο λάρυγξ αυτών, με τας γλώσσας αυτών ελάλουν δόλια· φαρμάκιον ασπίδων είναι υπό τα χείλη αυτών·
\par 14 των οποίων το στόμα γέμει κατάρας και πικρίας·
\par 15 οι πόδες αυτών είναι ταχείς εις το να χύσωσιν αίμα·
\par 16 ερήμωσις και ταλαιπωρία εν ταις οδοίς αυτών,
\par 17 Και οδόν ειρήνης δεν εγνώρισαν.
\par 18 Δεν είναι φόβος Θεού έμπροσθεν των οφθαλμών αυτών.
\par 19 Εξεύρομεν δε ότι όσα λέγει ο νόμος λαλεί προς τους υπό τον νόμον, διά να εμφραχθή παν στόμα και να γείνη πας ο κόσμος υπόδικος εις τον Θεόν,
\par 20 διότι εξ έργων νόμου δεν θέλει δικαιωθή ουδεμία σαρξ ενώπιον αυτού· επειδή διά του νόμου γίνεται η γνώρισις της αμαρτίας.
\par 21 Τώρα δε χωρίς νόμου η δικαιοσύνη του Θεού εφανερώθη, μαρτυρουμένη υπό του νόμου και των προφητών,
\par 22 δικαιοσύνη δε του Θεού διά πίστεως Ιησού Χριστού εις πάντας και επί πάντας τους πιστεύοντας· διότι δεν υπάρχει διαφορά·
\par 23 επειδή πάντες ήμαρτον και υστερούνται της δόξης του Θεού,
\par 24 δικαιούνται δε δωρεάν με την χάριν αυτού διά της απολυτρώσεως της εν Χριστώ Ιησού,
\par 25 τον οποίον ο Θεός προέθετο μέσον εξιλεώσεως διά της πίστεως εν τω αίματι αυτού, προς φανέρωσιν της δικαιοσύνης αυτού διά την άφεσιν των προγενομένων αμαρτημάτων διά της μακροθυμίας του Θεού,
\par 26 προς φανέρωσιν της δικαιοσύνης αυτού εν τω παρόντι καιρώ, διά να ήναι αυτός δίκαιος και να δικαιόνη τον πιστεύοντα εις τον Ιησούν.
\par 27 Που λοιπόν η καύχησις; Εκλείσθη έξω. Διά ποίου νόμου; των έργων; Ουχί, αλλά διά του νόμου της πίστεως.
\par 28 Συμπεραίνομεν λοιπόν ότι ο άνθρωπος δικαιούται διά της πίστεως χωρίς των έργων του νόμου.
\par 29 Ή των Ιουδαίων μόνον είναι ο Θεός; Ουχί δε και των εθνών; Ναι, και των εθνών,
\par 30 επειδή εις είναι ο Θεός όστις θέλει δικαιώσει την περιτομήν εκ πίστεως και την ακροβυστίαν διά της πίστεως.
\par 31 Νόμον λοιπόν καταργούμεν διά της πίστεως; μη γένοιτο, αλλά νόμον συνιστώμεν.

\chapter{4}

\par Τι λοιπόν θέλομεν ειπεί ότι απήλαυσεν Αβραάμ ο πατήρ ημών κατά σάρκα;
\par 2 Διότι εάν ο Αβραάμ εδικαιώθη εκ των έργων, έχει καύχημα, αλλ' ουχί ενώπιον του Θεού.
\par 3 Επειδή τι λέγει η γραφή; Και επίστευσεν Αβραάμ εις τον Θεόν, και ελογίσθη εις αυτόν εις δικαιοσύνην.
\par 4 Εις δε τον εργαζόμενον ο μισθός δεν λογίζεται ως χάρις, αλλ' ως χρέος·
\par 5 εις τον μη εργαζόμενον όμως, πιστεύοντα δε εις τον δικαιούντα τον ασεβή, η πίστις αυτού λογίζεται εις δικαιοσύνην,
\par 6 καθώς και ο Δαβίδ λέγει τον μακαρισμόν του ανθρώπου, εις τον οποίον ο Θεός λογίζεται δικαιοσύνην, χωρίς έργων·
\par 7 Μακάριοι εκείνοι, των οποίων συνεχωρήθησαν αι ανομίαι και των οποίων εσκεπάσθησαν αι αμαρτίαι·
\par 8 μακάριος ο άνθρωπος, εις τον οποίον ο Κύριος δεν θέλει λογίζεσθαι αμαρτίαν.
\par 9 Ούτος λοιπόν ο μακαρισμός γίνεται διά τους περιτετμημένους ή και διά τους απεριτμήτους; διότι λέγομεν ότι η πίστις ελογίσθη εις τον Αβραάμ εις δικαιοσύνην.
\par 10 Πως λοιπόν ελογίσθη; ότε ήτο εν περιτομή ή εν ακροβυστία; Ουχί εν περιτομή αλλ' εν ακροβυστία·
\par 11 και έλαβε το σημείον της περιτομής, σφραγίδα της δικαιοσύνης της εκ πίστεως της εν τη ακροβυστία, διά να ήναι αυτός πατήρ πάντων των πιστευόντων ενώ υπάρχουσιν εν τη ακροβυστία, διά να λογισθή και εις αυτούς η δικαιοσύνη,
\par 12 και πατήρ της περιτομής, ουχί μόνον εις τους περιτετμημένους, αλλά και εις τους περιπατούντας εις τα ίχνη της πίστεως του πατρός ημών Αβραάμ της εν τη ακροβυστία.
\par 13 Επειδή η επαγγελία προς τον Αβραάμ ή προς το σπέρμα αυτού, ότι έμελλε να ήναι κληρονόμος του κόσμου, δεν έγεινε διά του νόμου, αλλά διά της δικαιοσύνης της εκ πίστεως.
\par 14 Διότι εάν ήναι κληρονόμοι οι εκ του νόμου, η πίστις εματαιώθη και κατηργήθη η επαγγελία·
\par 15 επειδή ο νόμος επιφέρει οργήν· διότι όπου δεν υπάρχει νόμος, ουδέ παράβασις υπάρχει.
\par 16 Διά τούτο εκ πίστεως η κληρονομία, διά να ήναι κατά χάριν, ώστε η επαγγελία να ήναι βεβαία εις άπαν το σπέρμα, ουχί μόνον το εκ του νόμου, αλλά και το εκ της πίστεως του Αβραάμ, όστις είναι πατήρ πάντων ημών,
\par 17 καθώς είναι γεγραμμένον, ότι πατέρα πολλών εθνών σε κατέστησα, ενώπιον του Θεού εις τον οποίον επίστευσε, του ζωοποιούντος τους νεκρούς και καλούντος τα μη όντα ως όντα·
\par 18 όστις καίτοι μη έχων ελπίδα επίστευσεν επ' ελπίδι, ότι έμελλε να γείνη πατήρ πολλών εθνών κατά το λαληθέν· Ούτω θέλει είσθαι το σπέρμα σου·
\par 19 και μη ασθενήσας κατά την πίστιν δεν εσυλλογίσθη το σώμα αυτού ότι ήτο ήδη νενεκρωμένον, εκατονταετής περίπου ων, και την νέκρωσιν της μήτρας της Σάρρας·
\par 20 ουδέ εδίστασεν εις την επαγγελίαν του Θεού διά της απιστίας, αλλ' ενεδυναμώθη εις την πίστιν, δοξάσας τον Θεόν,
\par 21 και πεποιθώς ότι εκείνο, το οποίον υπεσχέθη, είναι δυνατός και να εκτελέση.
\par 22 Διά τούτο και ελογίσθη εις αυτόν εις δικαιοσύνην.
\par 23 Δεν εγράφη δε δι' αυτόν μόνον, ότι ελογίσθη εις αυτόν,
\par 24 αλλά και δι' ημάς, εις τους οποίους μέλλει να λογισθή, τους πιστεύοντας εις τον αναστήσαντα εκ νεκρών Ιησούν τον Κύριον ημών,
\par 25 όστις παρεδόθη διά τας αμαρτίας ημών και ανέστη διά την δικαίωσιν ημών.

\chapter{5}

\par Δικαιωθέντες λοιπόν εκ πίστεως, έχομεν ειρήνην προς τον Θεόν διά του Κυρίου ημών Ιησού Χριστού,
\par 2 διά του οποίου ελάβομεν και την είσοδον διά της πίστεως εις την χάριν ταύτην, εις την οποίαν ιστάμεθα και καυχώμεθα εις την ελπίδα της δόξης του Θεού.
\par 3 Και ουχί μόνον τούτο, αλλά και καυχώμεθα εις τας θλίψεις, γινώσκοντες ότι θλίψις εργάζεται υπομονήν,
\par 4 η δε υπομονή δοκιμήν, η δε δοκιμή ελπίδα,
\par 5 η δε ελπίς δεν καταισχύνει, διότι αγάπη του Θεού είναι εκκεχυμένη εν ταις καρδίαις ημών διά Πνεύματος Αγίου του δοθέντος εις ημάς.
\par 6 Επειδή ο Χριστός, ότε ήμεθα έτι ασθενείς, απέθανε κατά τον ωρισμένον καιρόν υπέρ των ασεβών.
\par 7 Διότι μόλις υπέρ δικαίου θέλει αποθάνει τις· επειδή υπέρ του αγαθού ίσως και τολμά τις να αποθάνη·
\par 8 αλλ' ο Θεός δεικνύει την εαυτού αγάπην εις ημάς, διότι ενώ ημείς ήμεθα έτι αμαρτωλοί, ο Χριστός απέθανεν υπέρ ημών.
\par 9 Πολλώ μάλλον λοιπόν αφού εδικαιώθημεν τώρα διά του αίματος αυτού, θέλομεν σωθή από της οργής δι' αυτού.
\par 10 Διότι εάν εχθροί όντες εφιλιώθημεν με τον Θεόν διά του θανάτου του Υιού αυτού, πολλώ, μάλλον φιλιωθέντες θέλομεν σωθή διά της ζωής αυτού·
\par 11 και ουχί μόνον τούτο, αλλά και καυχώμενοι εις τον Θεόν διά του Κυρίου ημών Ιησού Χριστού, διά του οποίου ελάβομεν τώρα την φιλίωσιν.
\par 12 Διά τούτο καθώς δι' ενός ανθρώπου η αμαρτία εισήλθεν εις τον κόσμον και διά της αμαρτίας ο θάνατος, και ούτω διήλθεν ο θάνατος εις πάντας ανθρώπους, επειδή πάντες ήμαρτον·
\par 13 διότι μέχρι του νόμου ήτο εν τω κόσμω η αμαρτία, αμαρτία όμως δεν λογίζεται όταν δεν ήναι νόμος·
\par 14 αλλ' εβασίλευσεν ο θάνατος από Αδάμ μέχρι Μωϋσέως και επί τους μη αμαρτήσαντας κατά την ομοιότητα της παραβάσεως του Αδάμ, όστις είναι τύπος του μέλλοντος.
\par 15 Πλην δεν είναι καθώς το αμάρτημα, ούτω και το χάρισμα· διότι αν διά το αμάρτημα του ενός απέθανον οι πολλοί, πολύ περισσότερον η χάρις του Θεού και η δωρεά διά της χάριτος του ενός ανθρώπου Ιησού Χριστού επερίσσευσεν εις τους πολλούς.
\par 16 Και η δωρεά δεν είναι καθώς η δι' ενός αμαρτήσαντος γενομένη κατάκρισις· διότι η κρίσις εκ του ενός έγεινεν εις κατάκρισιν των πολλών, το δε χάρισμα εκ πολλών αμαρτημάτων έγεινεν εις δικαίωσιν.
\par 17 Διότι αν και διά το αμάρτημα του ενός ο θάνατος εβασίλευσε διά του ενός, πολύ περισσότερον οι λαμβάνοντες την αφθονίαν της χάριτος και της δωρεάς της δικαιοσύνης θέλουσι βασιλεύσει εν ζωή διά του ενός Ιησού Χριστού.
\par 18 Καθώς λοιπόν δι' ενός αμαρτήματος ήλθε κατάκρισις εις πάντας ανθρώπους, ούτω και διά μιας δικαιοσύνης ήλθεν εις πάντας ανθρώπους δικαίωσις εις ζωήν.
\par 19 Διότι καθώς διά της παρακοής του ενός ανθρώπου οι πολλοί κατεστάθησαν αμαρτωλοί, ούτω και διά της υπακοής του ενός οι πολλοί θέλουσι κατασταθή δίκαιοι.
\par 20 Παρεισήλθε δε ο νόμος διά να περισσεύση το αμάρτημα. Και όπου επερίσσευσεν η αμαρτία, υπερεπερίσσευσεν η χάρις,
\par 21 ίνα καθώς εβασίλευσεν η αμαρτία διά του θανάτου, ούτω και η χάρις βασιλεύση διά της δικαιοσύνης εις ζωήν αιώνιον διά Ιησού Χριστού του Κυρίου ημών.

\chapter{6}

\par Τι λοιπόν θέλομεν ειπεί; θέλομεν επιμένει εν τη αμαρτία, διά να περισσεύση η χάρις;
\par 2 Μη γένοιτο· ημείς, οίτινες απεθάνομεν κατά την αμαρτίαν, πως θέλομεν ζήσει πλέον εν αυτή;
\par 3 Η αγνοείτε ότι όσοι εβαπτίσθημεν εις Χριστόν Ιησούν, εις τον θάνατον αυτού εβαπτίσθημεν;
\par 4 Συνετάφημεν λοιπόν μετ' αυτού διά του βαπτίσματος εις τον θάνατον, ίνα καθώς ο Χριστός ανέστη εκ νεκρών διά της δόξης του Πατρός, ούτω και ημείς περιπατήσωμεν εις νέαν ζωήν.
\par 5 Διότι εάν εγείναμεν σύμφυτοι με αυτόν κατά την ομοιότητα του θανάτου αυτού, θέλομεν είσθαι και κατά την ομοιότητα της αναστάσεως,
\par 6 τούτο γινώσκοντες, ότι ο παλαιός ημών άνθρωπος συνεσταυρώθη, διά να καταργηθή το σώμα της αμαρτίας, ώστε να μη ήμεθα πλέον δούλοι της αμαρτίας·
\par 7 διότι ο αποθανών ηλευθερώθη από της αμαρτίας.
\par 8 Εάν δε απεθάνομεν μετά του Χριστού, πιστεύομεν ότι και θέλομεν συζήσει μετ' αυτού,
\par 9 γινώσκοντες ότι ο Χριστός αναστάς εκ νεκρών δεν αποθνήσκει πλέον, θάνατος αυτόν δεν κυριεύει πλέον.
\par 10 Διότι καθ' ο απέθανεν, απέθανεν άπαξ διά την αμαρτίαν, αλλά καθ' ο ζη, ζη εις τον Θεόν.
\par 11 Ούτω και σεις φρονείτε εαυτούς ότι είσθε νεκροί μεν κατά την αμαρτίαν, ζώντες δε εις τον Θεόν διά Ιησού Χριστού του Κυρίου ημών.
\par 12 Ας μη βασιλεύη λοιπόν η αμαρτία εν τω θνητώ υμών σώματι, ώστε κατά τας επιθυμίας αυτού να υπακούητε εις αυτήν,
\par 13 μηδέ παριστάνετε τα μέλη σας όπλα αδικίας εις την αμαρτίαν, αλλά παραστήσατε εαυτούς εις τον Θεόν ως ζώντας εκ νεκρών, και τα μέλη σας όπλα δικαιοσύνης εις τον Θεόν.
\par 14 Διότι η αμαρτία δεν θέλει σας κυριεύσει· επειδή δεν είσθε υπό νόμον, αλλ' υπό χάριν.
\par 15 Τι λοιπόν; θέλομεν αμαρτήσει διότι δεν είμεθα υπό νόμον, αλλ' υπό χάριν; μη γένοιτο.
\par 16 Δεν εξεύρετε ότι εις όντινα παριστάνετε εαυτούς δούλους προς υπακοήν, είσθε δούλοι εκείνου εις τον οποίον υπακούετε, ή της αμαρτίας προς θάνατον ή της υπακοής προς δικαιοσύνην;
\par 17 Χάρις όμως εις τον Θεόν, διότι υπήρχετε δούλοι της αμαρτίας, πλην υπηκούσατε εκ καρδίας εις τον τύπον της διδαχής, εις τον οποίον παρεδόθητε,
\par 18 ελευθερωθέντες δε από της αμαρτίας, εδουλώθητε εις την δικαιοσύνην·
\par 19 ανθρωπίνως λέγω διά την ασθένειαν της σαρκός σας. Διότι καθώς παρεστήσατε τα μέλη σας δούλα εις την ακαθαρσίαν και εις την ανομίαν προς την ανομίαν, ούτω τώρα παραστήσατε τα μέλη σας δούλα εις την δικαιοσύνην προς αγιασμόν.
\par 20 Διότι ότε υπήρχετε δούλοι της αμαρτίας, υπήρχετε ελεύθεροι από της δικαιοσύνης.
\par 21 Τίνα λοιπόν καρπόν είχετε τότε εξ εκείνων των έργων, διά τα οποία τώρα αισχύνεσθε; διότι το τέλος εκείνων είναι θάνατος.
\par 22 Αλλά τώρα ελευθερωθέντες από της αμαρτίας και δουλωθέντες εις τον Θεόν, έχετε τον καρπόν σας εις αγιασμόν, το δε τέλος ζωήν αιώνιον.
\par 23 Διότι ο μισθός της αμαρτίας είναι θάνατος, το δε χάρισμα του Θεού ζωή αιώνιος διά Ιησού Χριστού του Κυρίου ημών.

\chapter{7}

\par Η αγνοείτε, αδελφοί, διότι λαλώ προς γινώσκοντας τον νόμον, ότι ο νόμος έχει κυριότητα επί του ανθρώπου εφ' όσον χρόνον ζη;
\par 2 Διότι η ύπανδρος γυνή είναι δεδεμένη διά του νόμου με τον άνδρα ζώντα· εάν δε αποθάνη ο ανήρ, απαλλάττεται από του νόμου του ανδρός.
\par 3 Άρα λοιπόν εάν ζώντος του ανδρός συζευχθή με άλλον άνδρα, θέλει είσθαι μοιχαλίς· εάν όμως αποθάνη ο ανήρ, είναι ελευθέρα από του νόμου, ώστε να μη ήναι μοιχαλίς εάν συζευχθή με άλλον άνδρα.
\par 4 Λοιπόν, αδελφοί μου, και σεις εθανατώθητε ως προς τον νόμον διά του σώματος του Χριστού, διά να συζευχθήτε με άλλον, τον αναστάντα εκ νεκρών, διά να καρποφορήσωμεν εις τον Θεόν.
\par 5 Διότι ότε ήμεθα εν τη σαρκί, τα πάθη των αμαρτιών τα διά του νόμου ενηργούντο εν τοις μέλεσιν ημών, διά να καρποφορήσωμεν εις τον θάνατον·
\par 6 τώρα όμως απηλλάχθημεν από του νόμου, αποθανόντος εκείνου, υπό του οποίου εκρατούμεθα, διά να δουλεύωμεν κατά το νέον πνεύμα και ουχί κατά το παλαιόν γράμμα.
\par 7 Τι λοιπόν θέλομεν ειπεί; ο νόμος είναι αμαρτία; Μη γένοιτο. Αλλά την αμαρτίαν δεν εγνώρισα, ειμή διά του νόμου· διότι και την επιθυμίαν δεν ήθελον γνωρίσει, εάν ο νόμος δεν έλεγε· Μη επιθυμήσης.
\par 8 Αφορμήν δε λαβούσα η αμαρτία διά της εντολής, εγέννησεν εν εμοί πάσαν επιθυμίαν· διότι χωρίς του νόμου η αμαρτία είναι νεκρά.
\par 9 Και εγώ έζων ποτέ χωρίς νόμου· αλλ' ότε ήλθεν η εντολή, ανέζησεν αμαρτία, εγώ δε απέθανον·
\par 10 και η εντολή, ήτις εδόθη προς ζωήν, αυτή ευρέθη εν εμοί προς θάνατον.
\par 11 Διότι η αμαρτία, λαβούσα αφορμήν διά της εντολής, με εξηπάτησε και δι' αυτής με εθανάτωσεν.
\par 12 Ώστε ο μεν νόμος είναι άγιος, και η εντολή αγία και δικαία και αγαθή.
\par 13 το αγαθόν λοιπόν έγεινεν εις εμέ θάνατος; μη γένοιτο. Αλλ' η αμαρτία, διά να φανή αμαρτία, προξενούσα εις εμέ θάνατον διά του αγαθού, ώστε να γείνη καθ' υπερβολήν αμαρτωλός αμαρτία διά της εντολής.
\par 14 Διότι εξεύρομεν ότι ο νόμος είναι πνευματικός· εγώ δε είμαι σαρκικός, πεπωλημένος υπό την αμαρτίαν.
\par 15 Διότι εκείνο, το οποίον πράττω, δεν γνωρίζω· επειδή εκείνο το οποίον θέλω τούτο δεν πράττω, αλλ' εκείνο το οποίον μισώ τούτο πράττω.
\par 16 Εάν δε εκείνο το οποίον δεν θέλω τούτο πράττω, συμφωνώ με τον νόμον, ότι είναι καλός.
\par 17 Τώρα δε δεν πράττω πλέον τούτο εγώ, αλλ' η αμαρτία η κατοικούσα εν εμοί.
\par 18 Διότι εξεύρω ότι δεν κατοικεί εν εμοί, τουτέστιν εν τη σαρκί μου, αγαθόν· επειδή το θέλειν πάρεστιν εις εμέ, το πράττειν όμως το καλόν δεν ευρίσκω·
\par 19 διότι δεν πράττω το αγαθόν, το οποίον θέλω· αλλά το κακόν, το οποίον δεν θέλω, τούτο πράττω.
\par 20 Εάν δε εγώ πράττω εκείνο το οποίον δεν θέλω, δεν εργάζομαι αυτό πλέον εγώ, αλλ' η αμαρτία η κατοικούσα εν εμοί.
\par 21 Ευρίσκω λοιπόν τον νόμον τούτον ότι, ενώ εγώ θέλω να πράττω το καλόν, πάρεστιν εις εμέ το κακόν·
\par 22 διότι ηδύνομαι μεν εις τον νόμον του Θεού κατά τον εσωτερικόν άνθρωπον,
\par 23 βλέπω όμως εν τοις μέλεσί μου άλλον νόμον αντιμαχόμενον εις τον νόμον του νοός μου, και αιχμαλωτίζοντά με εις τον νόμον της αμαρτίας, τον όντα εν τοις μέλεσί μου.
\par 24 Ταλαίπωρος άνθρωπος εγώ· τις θέλει με ελευθερώσει από του σώματος του θανάτου τούτου;
\par 25 Ευχαριστώ εις τον Θεόν διά Ιησού Χριστού του Κυρίου ημών. Άρα λοιπόν αυτός εγώ με τον νούν μεν δουλεύω εις τον νόμον του Θεού, με την σάρκα δε εις τον νόμον της αμαρτίας.

\chapter{8}

\par Δεν είναι τώρα λοιπόν ουδεμία κατάκρισις εις τους εν Χριστώ, Ιησού, τους μη περιπατούντας κατά την σάρκα, αλλά κατά το πνεύμα.
\par 2 Διότι ο νόμος του Πνεύματος της ζωής εν Χριστώ Ιησού με ηλευθέρωσεν από του νόμου της αμαρτίας και του θανάτου.
\par 3 Επειδή το αδύνατον εις τον νόμον, καθότι ήτο ανίσχυρος διά της σαρκός, ο Θεός πέμψας τον εαυτού Υιόν με ομοίωμα σαρκός αμαρτίας και περί αμαρτίας, κατέκρινε την αμαρτίαν εν τη σαρκί,
\par 4 διά να πληρωθή η δικαιοσύνη του νόμου εις ημάς τους μη περιπατούντας κατά την σάρκα, αλλά κατά το πνεύμα·
\par 5 διότι οι ζώντες κατά την σάρκα τα της σαρκός φρονούσιν, οι δε κατά το πνεύμα τα του πνεύματος.
\par 6 Επειδή το φρόνημα της σαρκός είναι θάνατος, το δε φρόνημα του πνεύματος ζωή και ειρήνη·
\par 7 διότι το φρόνημα της σαρκός είναι έχθρα εις τον Θεόν· επειδή εις τον νόμον του Θεού δεν υποτάσσεται· αλλ' ουδέ δύναται·
\par 8 όσοι δε είναι της σαρκός δεν δύνανται να αρέσωσιν εις τον Θεόν.
\par 9 Σεις όμως δεν είσθε της σαρκός, αλλά του πνεύματος, εάν το Πνεύμα του Θεού κατοική εν υμίν. Αλλ' εάν τις δεν έχη το Πνεύμα του Χριστού, ούτος δεν είναι αυτού.
\par 10 Εάν δε ο Χριστός ήναι εν υμίν, το μεν σώμα είναι νεκρόν διά την αμαρτίαν, το δε πνεύμα ζωή διά την δικαιοσύνην.
\par 11 Εάν δε κατοική εν υμίν το Πνεύμα του αναστήσαντος τον Ιησούν εκ νεκρών, ο αναστήσας τον Χριστόν εκ νεκρών θέλει ζωοποιήσει και τα θνητά σώματα υμών διά του Πνεύματος αυτού του κατοικούντος εν υμίν.
\par 12 Άρα λοιπόν, αδελφοί, είμεθα χρεώσται ουχί εις την σάρκα, ώστε να ζώμεν κατά σάρκα·
\par 13 διότι εάν ζήτε κατά την σάρκα, μέλλετε να αποθάνητε· αλλ' εάν διά του Πνεύματος θανατόνητε τας πράξεις του σώματος, θέλετε ζήσει.
\par 14 Επειδή όσοι διοικούνται υπό του Πνεύματος του Θεού, ούτοι είναι υιοί του Θεού.
\par 15 Διότι δεν ελάβετε πνεύμα δουλείας, διά να φοβήσθε πάλιν, αλλ' ελάβετε πνεύμα υιοθεσίας, διά του οποίου κράζομεν· Αββά, ο Πατήρ.
\par 16 Αυτό το Πνεύμα συμμαρτυρεί με το πνεύμα ημών ότι είμεθα τέκνα Θεού.
\par 17 Εάν δε τέκνα και κληρονόμοι, κληρονόμοι μεν Θεού, συγκληρονόμοι δε Χριστού, εάν συμπάσχωμεν, διά να γείνωμεν και συμμέτοχοι της δόξης αυτού.
\par 18 Επειδή φρονώ ότι τα παθήματα του παρόντος καιρού δεν είναι άξια να συγκριθώσι με την δόξαν την μέλλουσαν να αποκαλυφθή εις ημάς.
\par 19 Διότι η μεγάλη προσδοκία της κτίσεως προσμένει την φανέρωσιν των υιών του Θεού.
\par 20 Επειδή η κτίσις υπετάχθη εις την ματαιότητα, ουχί εκουσίως, αλλά διά τον υποτάξαντα αυτήν,
\par 21 επ' ελπίδι ότι και αυτή η κτίσις θέλει ελευθερωθή από της δουλείας της φθοράς και μεταβή εις την ελευθερίαν της δόξης των τέκνων του Θεού.
\par 22 Επειδή εξεύρομεν ότι πάσα η κτίσις συστενάζει και συναγωνιά έως του νύν·
\par 23 και ουχί μόνον αυτή, αλλά και αυτοί οίτινες έχομεν την απαρχήν του Πνεύματος, και ημείς αυτοί στενάζομεν εν εαυτοίς περιμένοντες την υιοθεσίαν, την απολύτρωσιν του σώματος ημών.
\par 24 Διότι με την ελπίδα εσώθημεν· ελπίς δε ήτις βλέπεται δεν είναι ελπίς· διότι εκείνο, το οποίον βλέπει τις, διά τι και ελπίζει;
\par 25 Εάν δε ελπίζωμεν εκείνο, το οποίον δεν βλέπομεν, διά της υπομονής περιμένομεν αυτό.
\par 26 Ωσαύτως δε και το Πνεύμα συμβοηθεί εις τας ασθενείας ημών· επειδή το τι να προσευχηθώμεν ως πρέπει δεν εξεύρομεν, αλλ' αυτό το Πνεύμα ικετεύει υπέρ ημών διά στεναγμών αλαλήτων·
\par 27 ο δε ερευνών τας καρδίας εξεύρει τι είναι το φρόνημα του Πνεύματος, ότι κατά Θεόν ικετεύει υπέρ των αγίων.
\par 28 Εξεύρομεν δε ότι πάντα συνεργούσι προς το αγαθόν εις τους αγαπώντας τον Θεόν, εις τους κεκλημένους κατά τον προορισμόν αυτού·
\par 29 διότι όσους προεγνώρισε, τούτους και προώρισε συμμόρφους της εικόνος του Υιού αυτού, διά να ήναι αυτός πρωτότοκος μεταξύ πολλών αδελφών·
\par 30 όσους δε προώρισε, τούτους και εκάλεσε, και όσους εκάλεσε, τούτους και εδικαίωσε, και όσους εδικαίωσε, τούτους και εδόξασε.
\par 31 Τι λοιπόν θέλομεν ειπεί προς ταύτα; Εάν ο Θεός ήναι υπέρ ημών, τις θέλει είσθαι καθ' ημών;
\par 32 Επειδή όστις τον ίδιον εαυτού Υιόν δεν εφείσθη, αλλά παρέδωκεν αυτόν υπέρ πάντων ημών, πως και μετ' αυτού δεν θέλει χαρίσει εις ημάς τα πάντα;
\par 33 Τις θέλει εγκαλέσει τους εκλεκτούς του Θεού; Θεός είναι ο δικαιών·
\par 34 τις θέλει είσθαι ο κατακρίνων; Χριστός ο αποθανών, μάλλον δε και αναστάς, όστις και είναι εν τη δεξιά του Θεού, όστις και μεσιτεύει υπέρ ημών.
\par 35 Τις θέλει μας χωρίσει από της αγάπης του Χριστού; θλίψις ή στενοχωρία ή διωγμός ή πείνα ή γυμνότης ή κίνδυνος ή μάχαιρα;
\par 36 Καθώς είναι γεγραμμένον, Ότι ένεκα σου θανατούμεθα όλην την ημέραν. Ελογίσθημεν ως πρόβατα σφαγής.
\par 37 Αλλ' εις πάντα ταύτα υπερνικώμεν διά του αγαπήσαντος ημάς.
\par 38 Επειδή είμαι πεπεισμένος ότι ούτε θάνατος ούτε ζωή ούτε άγγελοι ούτε αρχαί ούτε δυνάμεις ούτε παρόντα ούτε μέλλοντα
\par 39 ούτε ύψωμα ούτε βάθος ούτε άλλη τις κτίσις θέλει δυνηθή να χωρίση ημάς από της αγάπης του Θεού της εν Χριστώ Ιησού τω Κυρίω ημών.

\chapter{9}

\par Αλήθειαν λέγω εν Χριστώ, δεν ψεύδομαι, έχων συμμαρτυρούσαν με εμέ την συνείδησίν μου εν Πνεύματι Αγίω,
\par 2 ότι έχω λύπην μεγάλην και αδιάλειπτον οδύνην εν τη καρδία μου.
\par 3 Διότι ηυχόμην αυτός εγώ να ήμαι ανάθεμα από του Χριστού υπέρ των αδελφών μου, των κατά σάρκα συγγενών μου,
\par 4 οίτινες είναι Ισραηλίται, των οποίων είναι η υιοθεσία και η δόξα και αι διαθήκαι και η νομοθεσία και η λατρεία και αι επαγγελίαι,
\par 5 των οποίων είναι οι πατέρες, και εκ των οποίων εγεννήθη ο Χριστός το κατά σάρκα, ο ων επί πάντων Θεός ευλογητός εις τους αιώνας· αμήν.
\par 6 Αλλά δεν είναι δυνατόν ότι εξέπεσεν ο λόγος του Θεού. Διότι πάντες οι εκ του Ισραήλ δεν είναι ούτοι Ισραήλ,
\par 7 ουδέ διότι είναι σπέρμα του Αβραάμ, διά τούτο είναι πάντες τέκνα, αλλ' Εν τω Ισαάκ θέλει κληθή εις σε σπέρμα.
\par 8 Τουτέστι, τα τέκνα της σαρκός ταύτα δεν είναι τέκνα Θεού, αλλά τα τέκνα της επαγγελίας λογίζονται διά σπέρμα.
\par 9 Διότι ο λόγος της επαγγελίας είναι ούτος· Κατά τον καιρόν τούτον θέλω ελθεί και η Σάρρα θέλει έχει υιόν.
\par 10 Και ουχί μόνον τούτο, αλλά και η Ρεβέκκα, ότε συνέλαβε δύο εξ ενός ανδρός, Ισαάκ του πατρός ημών·
\par 11 διότι πριν έτι γεννηθώσι τα παιδία, και πριν πράξωσί τι αγαθόν ή κακόν, διά να μένη ο κατ' εκλογήν προορισμός του Θεού, ουχί εκ των έργων, αλλ' εκ του καλούντος,
\par 12 ερρέθη προς αυτήν ότι ο μεγαλήτερος θέλει δουλεύσει εις τον μικρότερον,
\par 13 καθώς είναι γεγραμμένον· Τον Ιακώβ ηγάπησα, τον δε Ησαύ εμίσησα.
\par 14 Τι λοιπόν θέλομεν ειπεί; Μήπως είναι αδικία εις τον Θεόν; μη γένοιτο.
\par 15 Διότι προς τον Μωϋσήν λέγει· θέλω ελεήσει όντινα ελεώ, και θέλω οικτειρήσει όντινα οικτείρω.
\par 16 Άρα λοιπόν δεν είναι του θέλοντος ουδέ του τρέχοντος, αλλά του ελεούντος Θεού.
\par 17 Διότι η γραφή λέγει προς τον Φαραώ ότι δι' αυτό τούτο σε εξήγειρα, διά να δείξω εν σοι την δύναμίν μου, και διά να διαγγελθή το όνομά μου εν πάση τη γη.
\par 18 Άρα λοιπόν όντινα θέλει ελεεί και όντινα θέλει σκληρύνει.
\par 19 Θέλεις λοιπόν μοι ειπεί· Διά τι πλέον μέμφεται; εις το θέλημα αυτού τις εναντιούται;
\par 20 Αλλά μάλιστα συ, ω άνθρωπε, τις είσαι, όστις ανταποκρίνεσαι προς τον Θεόν; Μήπως το πλάσμα θέλει ειπεί προς τον πλάσαντα, Διά τι με έκαμες ούτως;
\par 21 Η δεν έχει εξουσίαν ο κεραμεύς του πηλού, από του αυτού μίγματος να κάμη άλλο μεν σκεύος εις τιμήν, άλλο δε εις ατιμίαν;
\par 22 Τι δε, αν ο Θεός, θέλων να δείξη την οργήν αυτού και να κάμη γνωστήν την δύναμιν αυτού, υπέφερε μετά πολλής μακροθυμίας σκεύη οργής κατεσκευασμένα εις απώλειαν,
\par 23 και διά να γνωστοποιήση τον πλούτον της δόξης αυτού επί σκεύη ελέους, τα οποία προητοίμασεν εις δόξαν,
\par 24 ημάς τους οποίους εκάλεσεν ουχί μόνον εκ των Ιουδαίων αλλά και εκ των εθνών;
\par 25 Καθώς και εν τω Ωσηέ λέγει· Θέλω καλέσει λαόν μου τον ου λαόν μου, και ηγαπημένην την ουκ ηγαπημένην·
\par 26 και εν τω τόπω, όπου ερρέθη προς αυτούς, δεν είσθε λαός μου, εκεί θέλουσι καλεσθή υιοί Θεού ζώντος.
\par 27 Ο δε Ησαΐας κράζει υπέρ του Ισραήλ· Αν και ο αριθμός των υιών Ισραήλ ήναι ως η άμμος της θαλάσσης, το υπόλοιπον αυτών θέλει σωθή·
\par 28 διότι θέλει τελειώσει και συντέμει λογαριασμόν μετά δικαιοσύνης, επειδή συντετμημένον λογαριασμόν θέλει κάμει ο Κύριος επί της γης.
\par 29 Και καθώς προείπεν ο Ησαΐας· Εάν ο Κύριος Σαβαώθ δεν ήθελεν αφήσει εις ημάς σπέρμα, ως τα Σόδομα ηθέλομεν γείνει και με τα Γόμορρα ηθέλομεν ομοιωθή.
\par 30 Τι λοιπόν θέλομεν ειπεί; Ότι τα έθνη τα μη ζητούντα δικαιοσύνην έφθασαν εις δικαιοσύνην, δικαιοσύνην δε την εκ πίστεως,
\par 31 ο δε Ισραήλ ζητών νόμον δικαιοσύνης, εις νόμον δικαιοσύνης δεν έφθασε.
\par 32 Διά τι; Επειδή δεν εζήτει αυτήν εκ πίστεως, αλλ' ως εκ των έργων του νόμου· διότι προσέκοψαν εις τον λίθον του προσκόμματος,
\par 33 καθώς είναι γεγραμμένον· Ιδού, θέτω εν Σιών λίθον προσκόμματος και πέτραν σκανδάλου, και πας ο πιστεύων επ' αυτόν δεν θέλει καταισχυνθή.

\chapter{10}

\par Αδελφοί, η επιθυμία της καρδίας μου και η δέησις η προς τον Θεόν υπέρ του Ισραήλ είναι διά την σωτηρίαν αυτών·
\par 2 διότι μαρτυρώ περί αυτών ότι έχουσι ζήλον Θεού, αλλ' ουχί κατ' επίγνωσιν.
\par 3 Επειδή μη γνωρίζοντες την δικαιοσύνην του Θεού, και ζητούντες να συστήσωσι την ιδίαν αυτών δικαιοσύνην, δεν υπετάχθησαν εις την δικαιοσύνην του Θεού.
\par 4 Επειδή το τέλος του νόμου είναι ο Χριστός προς δικαιοσύνην εις πάντα τον πιστεύοντα.
\par 5 Διότι ο Μωϋσής γράφει την δικαιοσύνην την εκ του νόμου, λέγων ότι ο άνθρωπος ο κάμνων ταύτα θέλει ζήσει δι' αυτών·
\par 6 η εκ πίστεως όμως δικαιοσύνη λέγει ούτω· Μη είπης εν τη καρδία σου, Τις θέλει αναβή εις τον ουρανόν; τουτέστι διά να καταβιβάση τον Χριστόν.
\par 7 ή, Τις θέλει καταβή εις την άβυσσον; τουτέστι διά να αναβιβάση τον Χριστόν εκ νεκρών.
\par 8 Αλλά τι λέγει; Πλησίον σου είναι ο λόγος, εν τω στόματί σου και εν τη καρδία σου· τουτέστιν ο λόγος της πίστεως, τον οποίον κηρύττομεν.
\par 9 Ότι εάν ομολογήσης διά του στόματός σου τον Κύριον Ιησούν, και πιστεύσης εν τη καρδία σου ότι ο Θεός ανέστησεν αυτόν εκ νεκρών, θέλεις σωθή·
\par 10 διότι με την καρδίαν πιστεύει τις προς δικαιοσύνην, και με το στόμα γίνεται ομολογία προς σωτηρίαν.
\par 11 Διότι λέγει η γραφή· Πας ο πιστεύων επ' αυτόν δεν θέλει καταισχυνθή.
\par 12 Επειδή δεν είναι διαφορά Ιουδαίου τε και Ελληνος· διότι ο αυτός Κύριος είναι πάντων, πλούσιος προς πάντας τους επικαλουμένους αυτόν·
\par 13 διότι Πας όστις επικαλεσθή το όνομα του Κυρίου θέλει σωθή.
\par 14 Πως λοιπόν θέλουσιν επικαλεσθή εκείνον, εις τον οποίον δεν επίστευσαν; και πως θέλουσι πιστεύσει εις εκείνον, περί του οποίου δεν ήκουσαν; και πως θέλουσιν ακούσει χωρίς να υπάρχη ο κηρύττων;
\par 15 Και πως θέλουσι κηρύξει, εάν δεν αποσταλώσι; Καθώς είναι γεγραμμένον· Πόσον ωραίοι οι πόδες των ευαγγελιζομένων ειρήνην, των ευαγγελιζομένων τα αγαθά.
\par 16 Αλλά δεν υπήκουσαν πάντες εις το ευαγγέλιον. Διότι ο Ησαΐας λέγει· Κύριε, τις επίστευσεν εις το κήρυγμα ημών;
\par 17 Άρα η πίστις είναι εξ ακοής, η δε ακοή διά του λόγου του Θεού.
\par 18 Λέγω όμως, Μη δεν ήκουσαν; Μάλιστα εις πάσαν την γην εξήλθεν ο φθόγγος αυτών, Και εις τα πέρατα της οικουμένης οι λόγοι αυτών.
\par 19 Αλλά λέγω, Μη δεν εγνώρισεν ο Ισραήλ; Πρώτος ο Μωϋσής λέγει· Εγώ θέλω σας παροξύνει εις ζηλοτυπίαν με τους μη έθνος, Θέλω σας παροργίσει με έθνος ασύνετον.
\par 20 Ο δε Ησαΐας αποτολμά και λέγει· Ευρέθην παρά των μη ζητούντων με, εφανερώθην εις τους μη ερωτώντας περί εμού.
\par 21 Προς δε τον Ισραήλ λέγει· Όλην την ημέραν εξέτεινα τας χείρας μου προς λαόν απειθούντα και αντιλέγοντα.

\chapter{11}

\par Λέγω λοιπόν, Μήπως απέρριψεν ο Θεός τον λαόν αυτού; Μη γένοιτο· διότι και εγώ Ισραηλίτης είμαι, εκ σπέρματος Αβραάμ, εκ φυλής Βενιαμίν.
\par 2 Δεν απέρριψεν ο Θεός τον λαόν αυτού, τον οποίον προεγνώρισεν. Η δεν εξεύρετε τι λέγει η γραφή περί του Ηλία; πως ομιλεί προς τον Θεόν κατά του Ισραήλ, λέγων·
\par 3 Κύριε, τους προφήτας σου εθανάτωσαν και τα θυσιαστήριά σου κατέσκαψαν, και εγώ εναπελείφθην μόνος, και ζητούσι την ψυχήν μου.
\par 4 Αλλά τι αποκρίνεται προς αυτόν ο Θεός; Αφήκα εις εμαυτόν επτά χιλιάδας ανδρών, οίτινες δεν έκλιναν γόνυ εις τον Βάαλ.
\par 5 Ούτω λοιπόν και επί του παρόντος καιρού απέμεινε κατάλοιπόν τι κατ' εκλογήν χάριτος.
\par 6 Εάν δε κατά χάριν, δεν είναι πλέον εξ έργων· επειδή τότε η χάρις δεν γίνεται πλέον χάρις. Εάν δε εξ έργων, δεν είναι πλέον χάρις· επειδή το έργον δεν είναι πλέον έργον.
\par 7 Τι λοιπόν; Ο Ισραήλ δεν επέτυχεν εκείνο το οποίον ζητεί, οι εκλεκτοί όμως επέτυχον· οι δε λοιποί ετυφλώθησαν,
\par 8 καθώς είναι γεγραμμένον· Έδωκεν εις αυτούς ο Θεός πνεύμα νυσταγμού, οφθαλμούς διά να μη βλέπωσι και ώτα διά να μη ακούωσιν, έως της σήμερον ημέρας.
\par 9 Και ο Δαβίδ λέγει· Ας γείνη η τράπεζα αυτών εις παγίδα και εις βρόχον και εις σκάνδαλον και εις ανταπόδομα εις αυτούς·
\par 10 ας σκοτισθώσιν οι οφθαλμοί αυτών διά να μη βλέπωσι, και τον νώτον αυτών διαπαντός κύρτωσον.
\par 11 Λέγω λοιπόν, Μήπως έπταισαν διά να πέσωσι; Μη γένοιτο· αλλά διά της πτώσεως αυτών έγεινεν η σωτηρία εις τα έθνη, διά να κινήση αυτούς εις ζηλοτυπίαν.
\par 12 Και εάν η πτώσις αυτών ήναι πλούτος του κόσμου και ελάττωσις αυτών πλούτος των εθνών, πόσω μάλλον το πλήρωμα αυτών;
\par 13 Διότι προς εσάς τα έθνη λέγω, Εφ' όσον μεν είμαι εγώ απόστολος των εθνών, την διακονίαν μου δοξάζω,
\par 14 ίσως κινήσω εις ζηλοτυπίαν αυτούς, οίτινες είναι σαρξ μου και σώσω τινάς εξ αυτών.
\par 15 Διότι εάν η αποβολή αυτών ήναι φιλίωσις του κόσμου, τι θέλει είσθαι η πρόσληψις αυτών ειμή ζωή εκ νεκρών;
\par 16 Και εάν η ζύμη ήναι αγία, είναι και το φύραμα· και εάν η ρίζα ήναι αγία, είναι και οι κλάδοι.
\par 17 Αλλ' εάν τινές των κλάδων απεκόπησαν, συ δε αγριελαία ούσα ενεκεντρίσθης μεταξύ αυτών και έγεινες συγκοινωνός της ρίζης και της παχύτητος της ελαίας,
\par 18 μη κατακαυχάσαι εναντίον των κλάδων· εάν δε κατακαυχάσαι, συ δεν βαστάζεις την ρίζαν, αλλ' η ρίζα σε.
\par 19 Θέλεις ειπεί λοιπόν· Απεκόπησαν οι κλάδοι, διά να εγκεντρισθώ εγώ.
\par 20 Καλώς· διά την απιστίαν απεκόπησαν, συ δε διά της πίστεως ίστασαι· μη υψηλοφρόνει, αλλά φοβού·
\par 21 διότι εάν ο Θεός δεν εφείσθη τους φυσικούς κλάδους, πρόσεχε μήπως δεν φεισθή μηδέ σε.
\par 22 Ιδέ λοιπόν την χρηστότητα και την αυστηρότητα του Θεού, επί μεν τους πεσόντας την αυστηρότητα, επί σε δε την χρηστότητα, εάν επιμείνης εις την χρηστότητα· διότι άλλως και συ θέλεις αποκοπή.
\par 23 Και εκείνοι δε, εάν δεν επιμείνωσιν εις την απιστίαν, θέλουσιν εγκεντρισθή· διότι δυνατός είναι ο Θεός πάλιν να εγκεντρίση αυτούς.
\par 24 Επειδή εάν συ απεκόπης από της φυσικής αγριελαίας και παρά φύσιν ενεκεντρίσθης εις καλλιελαίαν, πόσω μάλλον ούτοι οι φυσικοί θέλουσιν εγκεντρισθή εις την ιδίαν αυτών ελαίαν.
\par 25 Διότι δεν θέλω να αγνοήτε, αδελφοί, το μυστήριον τούτο, διά να μη υψηλοφρονήτε, ότι τύφλωσις κατά μέρος έγεινεν εις τον Ισραήλ, εωσού εισέλθη το πλήρωμα των εθνών,
\par 26 και ούτω πας ο Ισραήλ θέλει σωθή, καθώς είναι γεγραμμένον· Θέλει ελθεί εκ Σιών ο λυτρωτής και θέλει αποστρέψει τας ασεβείας από του Ιακώβ·
\par 27 Και αύτη είναι η παρ' εμού διαθήκη προς αυτούς, Όταν αφαιρέσω τας αμαρτίας αυτών.
\par 28 Κατά μεν το ευαγγέλιον, είναι εχθροί διά σας, κατά δε την εκλογήν αγαπητοί διά τους πατέρας.
\par 29 Διότι ανεπίδεκτα μεταμελείας είναι τα χαρίσματα και η πρόσκλησις του Θεού.
\par 30 Διότι καθώς και σεις ηπειθήσατέ ποτέ εις τον Θεόν, τώρα όμως ηλεήθητε εν τη απειθεία τούτων,
\par 31 ούτω και ούτοι ηπείθησαν τώρα εν τω υμετέρω ελέει, διά να ελεηθώσι και αυτοί·
\par 32 διότι ο Θεός συνέκλεισε τους πάντας εις την απείθειαν, διά να ελεήση τους πάντας.
\par 33 Ω βάθος πλούτου και σοφίας και γνώσεως Θεού. Πόσον ανεξερεύνητοι είναι αι κρίσεις αυτού και ανεξιχνίαστοι αι οδοί αυτού.
\par 34 Διότι τις εγνώρισε τον νούν του Κυρίου; ή τις έγεινε σύμβουλος αυτού;
\par 35 ή τις έδωκε τι πρώτος εις αυτόν, διά να γείνη εις αυτόν ανταπόδοσις;
\par 36 Επειδή εξ αυτού και δι' αυτού και εις αυτόν είναι τα πάντα. Αυτώ, η δόξα εις τους αιώνας. Αμήν.

\chapter{12}

\par Σας παρακαλώ λοιπόν, αδελφοί, διά των οικτιρμών του Θεού, να παραστήσητε τα σώματά σας θυσίαν ζώσαν, αγίαν, ευάρεστον εις τον Θεόν, ήτις είναι η λογική σας λατρεία,
\par 2 και μη συμμορφόνεσθε με τον αιώνα τούτον, αλλά μεταμορφόνεσθε διά της ανακαινίσεως του νοός σας, ώστε να δοκιμάζητε τι είναι το θέλημα του Θεού, το αγαθόν και ευάρεστον και τέλειον.
\par 3 Διότι λέγω διά της χάριτος της εις εμέ δοθείσης προς πάντα όστις είναι μεταξύ σας, να μη φρονή υψηλότερα παρ' ό,τι πρέπει να φρονή, αλλά να φρονή ώστε να σωφρονή, κατά το μέτρον της πίστεως, το οποίον ο Θεός εμοίρασεν εις έκαστον.
\par 4 Διότι καθώς έχομεν εν ενί σώματι μέλη πολλά, πάντα δε τα μέλη δεν έχουσι το αυτό έργον,
\par 5 ούτω και ημείς οι πολλοί εν σώμα είμεθα εν Χριστώ, ο δε καθείς μέλη αλλήλων.
\par 6 Έχοντες δε χαρίσματα διάφορα κατά την δοθείσαν εις ημάς χάριν, είτε προφητείαν, ας προφητεύωμεν κατά την αναλογίαν της πίστεως,
\par 7 είτε διακονίαν, ας καταγινώμεθα εις την διακονίαν, είτε διδάσκει τις, ας καταγίνηται εις την διδασκαλίαν,
\par 8 είτε προτρέπει τις, εις την προτροπήν· ο μεταδίδων, ας μεταδίδη εν απλότητι, ο προϊστάμενος ας προΐσταται μετ' επιμελείας, ο ελεών ας ελεή εν ιλαρότητι.
\par 9 Η αγάπη ας ήναι ανυπόκριτος. Αποστρέφεσθε το πονηρόν, προσκολλάσθε εις το αγαθόν,
\par 10 γίνεσθε προς αλλήλους φιλόστοργοι διά της φιλαδελφίας, προλαμβάνοντες να τιμάτε αλλήλους,
\par 11 εις την σπουδήν άοκνοι, κατά το πνεύμα ζέοντες, τον Κύριον δουλεύοντες,
\par 12 εις την ελπίδα χαίροντες, εις την θλίψιν υπομένοντες, εις την προσευχήν προσκαρτερούντες,
\par 13 εις τας χρείας των αγίων μεταδίδοντες, την φιλοξενίαν ακολουθούντες.
\par 14 Ευλογείτε τους καταδιώκοντας υμάς, ευλογείτε και μη καταράσθε.
\par 15 Χαίρετε μετά χαιρόντων και κλαίετε μετά κλαιόντων.
\par 16 Έχετε προς αλλήλους το αυτό φρόνημα. Μη υψηλοφρονείτε, αλλά συγκαταβαίνετε εις τους ταπεινούς. Μη φαντάζεσθε εαυτούς φρονίμους.
\par 17 Εις μηδένα μη ανταποδίδετε κακόν αντί κακού· προνοείτε τα καλά ενώπιον πάντων ανθρώπων·
\par 18 ει δυνατόν, όσον το αφ' υμών ειρηνεύετε μετά πάντων ανθρώπων.
\par 19 Μη εκδικήτε εαυτούς, αγαπητοί, αλλά δότε τόπον τη οργή· διότι είναι γεγραμμένον· εις εμέ ανήκει η εκδίκησις, εγώ θέλω κάμει ανταπόδοσιν, λέγει Κύριος.
\par 20 Εάν λοιπόν πεινά ο εχθρός σου, τρέφε αυτόν, εάν διψά, πότιζε αυτόν· διότι πράττων τούτο θέλεις σωρεύσει άνθρακας πυρός επί την κεφαλήν αυτού.
\par 21 Μη νικάσαι υπό του κακού, αλλά νίκα διά του αγαθού το κακόν.

\chapter{13}

\par Πάσα ψυχή ας υποτάσσηται εις τας ανωτέρας εξουσίας. Διότι δεν υπάρχει εξουσία ειμή από Θεού· αι δε ούσαι εξουσίαι υπό του Θεού είναι τεταγμέναι.
\par 2 Ώστε ο εναντιούμενος εις την εξουσίαν εναντιούται εις την διαταγήν του Θεού· οι δε εναντιούμενοι θέλουσι λάβει εις εαυτούς καταδίκην.
\par 3 Διότι οι άρχοντες δεν είναι φόβος των αγαθών έργων, αλλά των κακών. Θέλεις δε να μη φοβήσαι την εξουσίαν; πράττε το καλόν, και θέλεις έχει έπαινον παρ' αυτής·
\par 4 επειδή ο άρχων είναι του Θεού υπηρέτης εις σε προς το καλόν. Εάν όμως πράττης το κακόν, φοβού· διότι δεν φορεί ματαίως την μάχαιραν· επειδή του Θεού υπηρέτης είναι, εκδικητής διά να εκτελή την οργήν κατά του πράττοντος το κακόν.
\par 5 Διά τούτο είναι ανάγκη να υποτάσσησθε ουχί μόνον διά την οργήν, αλλά και διά την συνείδησιν.
\par 6 Επειδή διά τούτο πληρόνετε και φόρους· διότι υπηρέται του Θεού είναι εις αυτό τούτο ενασχολούμενοι.
\par 7 Απόδοτε λοιπόν εις πάντας τα οφειλόμενα, εις όντινα οφείλετε τον φόρον τον φόρον, εις όντινα τον δασμόν τον δασμόν, εις όντινα τον φόβον τον φόβον, εις όντινα την τιμήν την τιμήν.
\par 8 Εις μηδένα μη οφείλετε μηδέν ειμή το να αγαπάτε αλλήλους· διότι ο αγαπών τον άλλον εκπληροί τον νόμον.
\par 9 Επειδή το, Μη μοιχεύσης, μη φονεύσης, μη κλέψης, μη ψευδομαρτυρήσης, μη επιθυμήσης, και πάσα άλλη εντολή, εν τούτω τω λόγω συμπεριλαμβάνεται, εν τώ· Θέλεις αγαπά τον πλησίον σου ως σεαυτόν.
\par 10 Η αγάπη κακόν δεν κάμνει εις τον πλησίον· είναι λοιπόν εκπλήρωσις του νόμου η αγάπη.
\par 11 Και μάλιστα, εξεύροντες τον καιρόν, ότι είναι ήδη ώρα να εγερθώμεν εκ του ύπνου· διότι είναι πλησιεστέρα εις ημάς η σωτηρία παρ' ότε επιστεύσαμεν.
\par 12 Η νυξ προεχώρησεν, η δε ημέρα επλησίασεν· ας απορρίψωμεν λοιπόν τα έργα του σκότους και ας ενδυθώμεν τα όπλα του φωτός.
\par 13 Ας περιπατήσωμεν ευσχημόνως ως εν ημέρα, μη εις συμπόσια και μέθας, μη εις κοίτας και ασελγείας, μη εις έριδα και φθόνον·
\par 14 αλλ' ενδύθητε τον Κύριον Ιησούν Χριστόν, και μη φροντίζετε περί της σαρκός εις το να εκτελήτε τας επιθυμίας αυτής.

\chapter{14}

\par Τον δε ασθενούντα κατά την πίστιν προσδέχεσθε, ουχί εις φιλονεικίας διαλογισμών.
\par 2 Άλλος μεν πιστεύει ότι δύναται να τρώγη πάντα, ο δε ασθενών τρώγει λάχανα.
\par 3 Ο τρώγων ας μη καταφρονή τον μη τρώγοντα, και ο μη τρώγων ας μη κρίνη τον τρώγοντα· διότι ο Θεός προσεδέχθη αυτόν.
\par 4 Συ τις είσαι όστις κρίνεις ξένον δούλον; εις τον ίδιον αυτού κύριον ίσταται ή πίπτει· θέλει όμως σταθή, διότι ο Θεός είναι δυνατός να στήση αυτόν.
\par 5 Άλλος μεν κρίνει μίαν ημέραν αγιωτέραν παρά άλλην ημέραν, άλλος δε κρίνει ίσην πάσαν ημέραν. Ας ήναι έκαστος πεπληροφορημένος εις τον ίδιον αυτού νούν.
\par 6 Ο παρατηρών την ημέραν παρατηρεί αυτήν διά τον Κύριον, και ο μη παρατηρών την ημέραν διά τον Κύριον δεν παρατηρεί αυτήν. Ο τρώγων διά τον Κύριον τρώγει· διότι ευχαριστεί εις τον Θεόν. Και ο μη τρώγων διά τον Κύριον δεν τρώγει, και ευχαριστεί εις τον Θεόν.
\par 7 Διότι ουδείς εξ ημών ζη δι' εαυτόν και ουδείς αποθνήσκει δι' εαυτόν.
\par 8 Επειδή εάν τε ζώμεν, διά τον Κύριον ζώμεν· εάν τε αποθνήσκωμεν, διά τον Κύριον αποθνήσκομεν. Εάν τε λοιπόν ζώμεν, εάν τε αποθνήσκωμεν, του Κυρίου είμεθα.
\par 9 Επειδή διά τούτο ο Χριστός και απέθανε και ανέστη και ανέζησε, διά να ήναι Κύριος και νεκρών και ζώντων.
\par 10 Συ δε διά τι κρίνεις τον αδελφόν σου; ή και συ διά τι εξουθενείς τον αδελφόν σου; επειδή πάντες ημείς θέλομεν παρασταθή εις το βήμα του Χριστού.
\par 11 Διότι είναι γεγραμμένον· Ζω εγώ, λέγει Κύριος, ότι εις εμέ θέλει κάμψει παν γόνυ, και πάσα γλώσσα θέλει δοξολογήσει τον Θεόν.
\par 12 Άρα λοιπόν έκαστος ημών περί εαυτού θέλει δώσει λόγον εις τον Θεόν.
\par 13 Λοιπόν ας μη κρίνωμεν πλέον αλλήλους, αλλά τούτο κρίνατε μάλλον, το να μη βάλλητε πρόσκομμα εις τον αδελφόν ή σκάνδαλον.
\par 14 Εξεύρω και είμαι πεπεισμένος εν Κυρίω Ιησού ότι ουδέν υπάρχει ακάθαρτον αφ' εαυτού ειμή εις τον όστις στοχάζεταί τι ότι είναι ακάθαρτον, εις εκείνον είναι ακάθαρτον.
\par 15 Εάν όμως ο αδελφός σου λυπήται διά φαγητόν, δεν περιπατείς πλέον κατά αγάπην· μη φέρε εις απώλειαν με το φαγητόν σου εκείνον, υπέρ του οποίου ο Χριστός απέθανεν.
\par 16 Ας μη βλασφημήται λοιπόν το αγαθόν σας.
\par 17 Διότι η βασιλεία του Θεού δεν είναι βρώσις και πόσις, αλλά δικαιοσύνη και ειρήνη και χαρά εν Πνεύματι Αγίω·
\par 18 επειδή ο δουλεύων εν τούτοις τον Χριστόν ευαρεστεί εις τον Θεόν και ευδοκιμεί παρά τοις ανθρώποις.
\par 19 Άρα λοιπόν ας ζητώμεν τα προς την ειρήνην και τα προς την οικοδομήν αλλήλων.
\par 20 Μη κατάστρεφε το έργον του Θεού διά φαγητόν. Πάντα μεν είναι καθαρά, κακόν όμως είναι εις τον άνθρωπον όστις τρώγει με σκάνδαλον.
\par 21 Καλόν είναι το να μη φάγης κρέας μηδέ να πίης οίνον μηδέ να πράξης τι, εις το οποίον ο αδελφός σου προσκόπτει ή σκανδαλίζεται ή ασθενεί.
\par 22 Συ πίστιν έχεις; έχε αυτήν εντός σου ενώπιον του Θεού· μακάριος όστις δεν κατακρίνει εαυτόν εις εκείνο, το οποίον αποδέχεται.
\par 23 Όστις όμως αμφιβάλλει, κατακρίνεται, εάν φάγη, διότι δεν τρώγει εκ πίστεως· και παν ό,τι δεν γίνεται εκ πίστεως, είναι αμαρτία.

\chapter{15}

\par Οφείλομεν δε ημείς οι δυνατοί να βαστάζωμεν τα ασθενήματα των αδυνάτων, και να μη αρέσκωμεν εις εαυτούς.
\par 2 Αλλ' έκαστος ημών ας αρέσκη εις τον πλησίον διά το καλόν προς οικοδομήν·
\par 3 επειδή και ο Χριστός δεν ήρεσεν εις εαυτόν, αλλά καθώς είναι γεγραμμένον, Οι ονειδισμοί των ονειδιζόντων σε επέπεσον επ' εμέ.
\par 4 Διότι όσα προεγράφησαν, διά την διδασκαλίαν ημών προεγράφησαν, διά να έχωμεν την ελπίδα διά της υπομονής και της παρηγορίας των γραφών.
\par 5 Ο δε Θεός της υπομονής και της παρηγορίας είθε να σας δώση να φρονήτε το αυτό εν αλλήλοις κατά Χριστόν Ιησούν,
\par 6 διά να δοξάζητε ομοθυμαδόν εν ενί στόματι τον Θεόν και Πατέρα του Κυρίου ημών Ιησού Χριστού.
\par 7 Διά τούτο προσδέχεσθε αλλήλους, καθώς και ο Χριστός προσεδέχθη ημάς εις δόξαν Θεού.
\par 8 Λέγω δε ότι ο Ιησούς Χριστός έγεινε διάκονος της περιτομής υπέρ της αληθείας του Θεού, διά να βεβαιώση τας προς τους πατέρας επαγγελίας,
\par 9 και διά να δοξάσωσι τα έθνη τον Θεόν διά το έλεος αυτού, καθώς είναι γεγραμμένον· Διά τούτο θέλω σε υμνεί μεταξύ των εθνών· και εις το όνομά σου θέλω ψάλλει.
\par 10 Και πάλιν λέγει· Ευφράνθητε, έθνη, μετά του λαού αυτού.
\par 11 Και πάλιν· Αινείτε τον Κύριον, πάντα τα έθνη, και δοξολογείτε αυτόν πάντες οι λαοί.
\par 12 Και πάλιν ο Ησαΐας λέγει· Θέλει είσθαι η ρίζα του Ιεσσαί, Και ο ανιστάμενος διά να βασιλεύη επί τα έθνη· εις αυτόν τα έθνη θέλουσιν ελπίσει.
\par 13 Ο δε Θεός της ελπίδος είθε να σας εμπλήση πάσης χαράς και ειρήνης διά της πίστεως, ώστε να περισσεύητε εις την ελπίδα διά της δυνάμεως του Πνεύματος του Αγίου.
\par 14 Είμαι δε, αδελφοί μου, και αυτός εγώ πεπεισμένος διά σας, ότι και σεις είσθε πλήρεις αγαθωσύνης, πεπληρωμένοι πάσης γνώσεως, δυνάμενοι και αλλήλους να νουθετήτε.
\par 15 Σας έγραψα όμως, αδελφοί, τολμηρότερον οπωσούν, ως υπενθυμίζων υμάς, διά την χάριν την δοθείσαν εις εμέ υπό του Θεού
\par 16 εις το να ήμαι υπηρέτης του Ιησού Χριστού προς τα έθνη, ιερουργών το ευαγγέλιον του Θεού, διά να γείνη η προσφορά των εθνών ευπρόσδεκτος, ηγιασμένη διά του Πνεύματος του Αγίου.
\par 17 Έχω λοιπόν καύχησιν εν Χριστώ Ιησού διά τα προς τον Θεόν·
\par 18 διότι δεν θέλω τολμήσει να είπω τι εξ εκείνων, τα οποία δεν έκαμεν ο Χριστός δι' εμού προς υπακοήν των εθνών λόγω και έργω,
\par 19 με δύναμιν σημείων και τεράτων, με δύναμιν του Πνεύματος του Θεού, ώστε από Ιερουσαλήμ και κύκλω μέχρι της Ιλλυρίας εξεπλήρωσα το κήρυγμα του ευαγγελίου του Χριστού,
\par 20 ούτω δε εφιλοτιμήθην να κηρύττω το ευαγγέλιον, ουχί όπου ωνομάσθη ο Χριστός, διά να μη οικοδομώ επί ξένου θεμελίου·
\par 21 αλλά καθώς είναι γεγραμμένον· Εκείνοι προς τους οποίους δεν ανηγγέλθη περί αυτού θέλουσιν ιδεί, και εκείνοι οίτινες δεν ήκουσαν θέλουσι νοήσει.
\par 22 Διά τούτο και εμποδιζόμην πολλάκις να έλθω προς εσάς·
\par 23 τώρα όμως μη έχων πλέον τόπον εν τοις κλίμασι τούτοις, επιποθών δε από πολλών ετών να έλθω προς εσάς,
\par 24 όταν υπάγω εις την Ισπανίαν, θέλω ελθεί προς εσάς· διότι ελπίζω διαβαίνων να σας ιδώ και να προπεμφθώ εκεί από σας, αφού πρώτον οπωσούν σας χορτασθώ.
\par 25 Τώρα δε υπάγω εις Ιερουσαλήμ, εκπληρών την διακονίαν εις τους αγίους.
\par 26 Διότι ευηρεστήθησαν η Μακεδονία και Αχαΐα να κάμωσί τινά βοήθειαν εις τους πτωχούς των αγίων των εν Ιερουσαλήμ.
\par 27 Ευηρεστήθησαν τωόντι, και είναι οφειλέται αυτών. Διότι εάν τα έθνη έγειναν συγκοινωνοί αυτών εις τα πνευματικά, χρεωστούσι να υπηρετήσωσιν αυτούς και εις τα σωματικά.
\par 28 Αφού λοιπόν εκτελέσω τούτο και επισφραγίσω εις αυτούς τον καρπόν τούτον, θέλω περάσει δι' υμών εις την Ισπανίαν.
\par 29 Εξεύρω δε ότι ερχόμενος προς εσάς, θέλω ελθεί με αφθονίαν της ευλογίας του ευαγγελίου του Χριστού.
\par 30 Σας παρακαλώ δε, αδελφοί, διά του Κυρίου ημών Ιησού Χριστού και διά της αγάπης του Πνεύματος, να συναγωνισθήτε μετ' εμού, προσευχόμενοι υπέρ εμού προς τον Θεόν,
\par 31 διά να ελευθερωθώ από των εν τη Ιουδαία απειθούντων, και διά να γείνη ευπρόσδεκτος εις τους αγίους η εις την Ιερουσαλήμ διακονία μου,
\par 32 διά να έλθω μετά χαράς προς εσάς διά θελήματος του Θεού και να συναναπαυθώ με σας.
\par 33 Ο δε Θεός της ειρήνης είη μετά πάντων υμών· αμήν.

\chapter{16}

\par Συνιστώ δε εις εσάς Φοίβην την αδελφήν ημών, ήτις είναι διάκονος της εκκλησίας της εν Κεγχρεαίς,
\par 2 διά να δεχθήτε αυτήν εν Κυρίω αξίως των αγίων και να παρασταθήτε εις αυτήν εις ό,τι πράγμα έχει χρείαν υμών· διότι και αύτη εστάθη προστάτις πολλών και εμού αυτού.
\par 3 Ασπάσθητε την Πρίσκιλλαν και τον Ακύλαν, τους συνεργούς μου εν Χριστώ Ιησού,
\par 4 οίτινες υπέρ της ζωής μου υπέβαλον υπό την μάχαιραν τον τράχηλον αυτών, τους οποίους ουχί εγώ μόνος ευχαριστώ, αλλά και πάσαι αι εκκλησίαι των εθνών.
\par 5 Ασπάσθητε και την κατ' οίκον αυτών εκκλησίαν. Ασπάσθητε Επαίνετον τον αγαπητόν μου, όστις είναι απαρχή της Αχαΐας εις τον Χριστόν.
\par 6 Ασπάσθητε την Μαριάμ, ήτις πολλά εκοπίασε δι' ημάς.
\par 7 Ασπάσθητε τον Ανδρόνικον και Ιουνίαν, τους συγγενείς μου και συναιχμαλώτους μου, οίτινες είναι επίσημοι μεταξύ των αποστόλων, οίτινες και προ εμού ήσαν εις τον Χριστόν.
\par 8 Ασπάσθητε τον Αμπλίαν τον αγαπητόν μου εν Κυρίω.
\par 9 Ασπάσθητε τον Ουρβανόν τον συνεργόν ημών εν Χριστώ, και τον Στάχυν τον αγαπητόν μου.
\par 10 Ασπάσθητε τον Απελλήν τον δεδοκιμασμένον εν Χριστώ. Ασπάσθητε τους εκ της οικογενείας του Αριστοβούλου.
\par 11 Ασπάσθητε τον Ηρωδίωνα τον συγγενή μου. Ασπάσθητε τους εκ της οικογενείας του Ναρκίσσου, τους όντας εν Κυρίω.
\par 12 Ασπάσθητε την Τρύφαιναν και την Τρυφώσαν, αίτινες κοπιάζουσιν εν Κυρίω. Ασπάσθητε Περσίδα την αγαπητήν, ήτις πολλά εκοπίασεν εν Κυρίω.
\par 13 Ασπάσθητε τον Ρούφον, τον εκλεκτόν εν Κυρίω, και την μητέρα αυτού και εμού.
\par 14 Ασπάσθητε τον Ασύγκριτον, τον Φλέγοντα, τον Ερμάν, τον Πατρόβαν, τον Ερμήν και τους μετ' αυτών αδελφούς.
\par 15 Ασπάσθητε τον Φιλόλογον και την Ιουλίαν, τον Νηρέα και την αδελφήν αυτού, και τον Ολυμπάν και πάντας τους αγίους τους μετ' αυτών.
\par 16 Ασπάσθητε αλλήλους εν φιλήματι αγίω. Σας ασπάζονται αι εκκλησίαι του Χριστού.
\par 17 Σας παρακαλώ δε, αδελφοί, να προσέχητε τους ποιούντας τας διχοστασίας και τα σκάνδαλα εναντίον της διδαχής, την οποίαν σεις εμάθετε, και απομακρύνεσθε απ' αυτών.
\par 18 Διότι οι τοιούτοι δεν δουλεύουσι τον Κύριον ημών Ιησούν Χριστόν, αλλά την εαυτών κοιλίαν, και διά λόγων καλών και κολακευτικών εξαπατώσι τας καρδίας των ακάκων·
\par 19 διότι η υπακοή σας διεφημίσθη εις πάντας. Όσον λοιπόν διά σας χαίρω· θέλω δε να ήσθε σοφοί μεν εις το αγαθόν, απλοί δε εις το κακόν.
\par 20 Ο δε Θεός της ειρήνης ταχέως θέλει συντρίψει τον Σατανάν υπό τους πόδας σας. Η χάρις του Κυρίου ημών Ιησού Χριστού είη μεθ' υμών. Αμήν.
\par 21 Σας ασπάζονται ο Τιμόθεος ο συνεργός μου, και Λούκιος και Ιάσων και Σωσίπατρος οι συγγενείς μου.
\par 22 Σας ασπάζομαι εν Κυρίω εγώ ο Τέρτιος, ο γράψας την επιστολήν.
\par 23 Σας ασπάζεται ο Γάϊος ο φιλοξενών εμέ και την εκκλησίαν όλην. Σας ασπάζεται Έραστος ο οικονόμος της πόλεως και Κούαρτος ο αδελφός.
\par 24 Η χάρις του Κυρίου ημών Ιησού Χριστού είη μετά πάντων υμών· αμήν.
\par 25 Εις δε τον δυνάμενον να σας στηρίξη κατά το ευαγγέλιόν μου και το κήρυγμα του Ιησού Χριστού, κατά την αποκάλυψιν του μυστηρίου του σεσιωπημένου μεν από χρόνων αιωνίων,
\par 26 φανερωθέντος δε τώρα διά προφητικών γραφών κατ' επιταγήν του αιωνίου Θεού και γνωρισθέντος εις πάντα τα έθνη προς υπακοήν πίστεως,
\par 27 εις τον μόνον σοφόν Θεόν έστω η δόξα διά Ιησού Χριστού εις τους αιώνας· αμήν.


\end{document}