\begin{document}

\title{Προς Κορινθίους Β'}


\chapter{1}

\par 1 Παύλος, απόστολος Ιησού Χριστού διά θελήματος Θεού, και Τιμόθεος ο αδελφός, προς την εκκλησίαν του Θεού την ούσαν εν Κορίνθω μετά πάντων των αγίων των όντων εν όλη τη Αχαΐα·
\par 2 χάρις υμίν και ειρήνη από Θεού Πατρός ημών και Κυρίου Ιησού Χριστού.
\par 3 Ευλογητός ο Θεός και Πατήρ του Κυρίου ημών Ιησού Χριστού, ο Πατήρ των οικτιρμών και Θεός πάσης παρηγορίας,
\par 4 ο παρηγορών ημάς εν πάση τη θλίψει ημών, διά να δυνάμεθα ημείς να παρηγορώμεν τους εν πάση θλίψει διά της παρηγορίας, με την οποίαν παρηγορούμεθα ημείς αυτοί υπό του Θεού·
\par 5 διότι καθώς περισσεύουσι τα παθήματα του Χριστού εις ημάς, ούτω διά του Χριστού περισσεύει και η παρηγορία ημών.
\par 6 Και είτε θλιβόμεθα, θλιβόμεθα υπέρ της παρηγορίας σας και σωτηρίας της ενεργουμένης διά της υπομονής των αυτών παθημάτων, τα οποία και ημείς πάσχομεν· είτε παρηγορούμεθα, παρηγορούμεθα υπέρ της παρηγορίας σας και σωτηρίας· και η ελπίς, την οποίαν έχομεν, είναι βεβαία υπέρ υμών·
\par 7 επειδή εξεύρομεν ότι καθώς είσθε κοινωνοί των παθημάτων, ούτω και της παρηγορίας.
\par 8 Διότι δεν θέλομεν να αγνοήτε, αδελφοί, περί της θλίψεως ημών, ήτις συνέβη εις ημάς εν τη Ασία, ότι καθ' υπερβολήν εστενοχωρήθημεν υπέρ δύναμιν, ώστε απηλπίσθημεν και του ζήν·
\par 9 αλλ' ημείς αυτοί εν εαυτοίς, ελάβομεν την απόφασιν του θανάτου, διά να μη έχωμεν την πεποίθησιν εις εαυτούς, αλλ' εις τον Θεόν τον εγείροντα τους νεκρούς·
\par 10 όστις ηλευθέρωσεν ημάς εκ τοσούτου μεγάλου θανάτου και ελευθερόνει, εις τον οποίον ελπίζομεν ότι και ότι θέλει ελευθερώσει,
\par 11 ενώ και σεις συνεργείτε υπέρ ημών διά της δεήσεως, διά να γείνη εκ πολλών προσώπων ευχαριστία υπέρ ημών διά το δοθέν εις ημάς χάρισμα διά πολλών.
\par 12 Διότι το καύχημα ημών είναι τούτο, η μαρτυρία της συνειδήσεως ημών, ότι εν απλότητι και ειλικρινεία Θεού, ουχί εν σοφία σαρκική, αλλ' εν χάριτι Θεού επολιτεύθημεν εν τω κόσμω, περισσότερον δε προς εσάς.
\par 13 Διότι δεν σας γράφομεν άλλο, παρ' εκείνα τα οποία αναγινώσκετε ή και γνωρίζετε, ελπίζω δε ότι και έως τέλους θέλετε γνωρίσει.
\par 14 Καθώς και μας εγνωρίσατε κατά μέρος, ότι είμεθα καύχημα εις εσάς, καθώς σεις εις ημάς, εν τη ημέρα του Κυρίου Ιησού.
\par 15 Και με ταύτην την πεποίθησιν ήθελον να έλθω προς εσάς πρότερον, διά να έχητε δευτέραν χάριν,
\par 16 και δι' υμών να διαβώ εις Μακεδονίαν, και πάλιν από Μακεδονίας να έλθω προς εσάς και από σας να προπεμφθώ εις την Ιουδαίαν.
\par 17 Τούτο λοιπόν βουλευόμενος μήπως τάχα μετεχειρίσθην ελαφρότητα; ή όσα βουλεύομαι, κατά σάρκα βουλεύομαι, διά να ήναι εις εμέ το ναι ναι, και το ου ου;
\par 18 Αλλ' όμως πιστός ο Θεός ότι ο λόγος ημών ο λαληθείς προς εσάς δεν έγεινε ναι και ου.
\par 19 Διότι ο Υιός του Θεού Ιησούς Χριστός ο κηρυχθείς μεταξύ σας δι' ημών, δι' εμού και του Σιλουανού και του Τιμοθέου, δεν έγεινε ναι και ου, αλλά ναι έγεινεν εν αυτώ.
\par 20 Διότι πάσαι αι επαγγελίαι του Θεού είναι εν αυτώ, το ναι και εν αυτώ το αμήν, προς δόξαν του Θεού δι' ημών.
\par 21 Ο δε βεβαιών ημάς μεθ' υμών εις Χριστόν και ο χρίσας ημάς είναι ο Θεός,
\par 22 όστις και εσφράγισεν ημάς και έδωκε τον αρραβώνα του Πνεύματος εν ταις καρδίαις ημών.
\par 23 Εγώ δε μάρτυρα τον Θεόν επικαλούμαι εις την ψυχήν μου, ότι φειδόμενος υμών δεν ήλθον έτι εις Κόρινθον.
\par 24 Ουχί διότι έχομεν εξουσίαν επί της πίστεώς σας, αλλ' είμεθα συνεργοί της χαράς σας· επειδή εν τη πίστει στέκεσθε.

\chapter{2}

\par 1 Απεφάσισα δε τούτο κατ' εμαυτόν, το να μη έλθω πάλιν προς εσάς με λύπην.
\par 2 Διότι εάν εγώ σας λυπώ, και τις είναι ο ευφραίνων εμέ ειμή ο λυπούμενος υπ' εμού;
\par 3 Και έγραψα προς εσάς τούτο αυτό, ώστε όταν έλθω να μη έχω λύπην απ' εκείνων, αφ' ων έπρεπε να έχω χαράν, έχων πεποίθησιν εις πάντας υμάς ότι η χαρά μου είναι πάντων υμών.
\par 4 Διότι εκ πολλής θλίψεως και στενοχωρίας καρδίας έγραψα προς εσάς μετά πολλών δακρύων, ουχί διά να λυπηθήτε, αλλά διά να γνωρίσητε την αγάπην, ην έχω περισσοτέρως εις εσάς.
\par 5 Αλλ' εάν τις ελύπησε, δεν ελύπησεν εμέ, ειμή κατά μέρος, διά να μη επιβαρύνω πάντας υμάς.
\par 6 Αρκετόν είναι εις τον τοιούτον αύτη η επίπληξις η υπό των πλειοτέρων·
\par 7 ώστε το εναντίον πρέπει μάλλον να συγχωρήσητε αυτόν, και να παρηγορήσητε, διά να μη καταποθή ο τοιούτος υπό της υπερβαλλούσης λύπης.
\par 8 Διά τούτο σας παρακαλώ να βεβαιώσητε προς αυτόν την αγάπην σας.
\par 9 Επειδή διά τούτο και έγραψα, διά να γνωρίσω την δοκιμασίαν σας, αν ήσθε κατά πάντα υπήκοοι·
\par 10 εις όντινα δε συγχωρείτέ τι, συγχωρώ και εγώ· διότι εάν εγώ συνεχώρησά τι, εις όντινα συνεχώρησα, διά σας έκαμον τούτο ενώπιον του Χριστού,
\par 11 διά να μη υπερισχύση καθ' ημών ο Σατανάς· διότι δεν αγνοούμεν τα διανοήματα αυτού.
\par 12 Ότε δε ήλθον εις την Τρωάδα διά να κηρύξω το ευαγγέλιον του Χριστού, και ηνοίχθη εις εμέ θύρα εν Κυρίω,
\par 13 δεν έλαβον άνεσιν εις το πνεύμά μου, διότι δεν εύρον Τίτον τον αδελφόν μου, αλλ' αποχαιρετήσας αυτούς εξήλθον εις Μακεδονίαν.
\par 14 Πλην χάρις εις τον Θεόν, όστις πάντοτε κάμνει ημάς να βριαμβεύωμεν διά του Χριστού και φανερόνει εν παντί τόπω δι' ημών την οσμήν της γνώσεως αυτού·
\par 15 διότι του Χριστού ευωδία είμεθα προς τον Θεόν εις τους σωζομένους και εις τους απολλυμένους·
\par 16 εις τούτους μεν οσμή θανάτου διά θάνατον, εις εκείνους δε οσμή ζωής διά ζωήν. Και προς ταύτα τις είναι ικανός;
\par 17 Διότι ημείς καθώς οι πολλοί δεν καπηλεύομεν τον λόγον του Θεού, αλλ' ως από ειλικρινείας, αλλ' ως από Θεού κατενώπιον του Θεού λαλούμεν εν Χριστώ.

\chapter{3}

\par 1 Αρχίζομεν πάλιν να συνιστώμεν εαυτούς; ή μήπως έχομεν χρείαν, καθώς τινές, συστατικών επιστολών προς εσάς ή συστατικών από σας;
\par 2 Σεις είσθε η επιστολή ημών, εγγεγραμμένη εν ταις καρδίαις ημών, γινωσκομένη και αναγινωσκομένη υπό πάντων ανθρώπων,
\par 3 και φανερόνεσθε ότι είσθε επιστολή Χριστού, γενομένη διά της διακονίας ημών, εγγεγραμμένη ουχί με μελάνην, αλλά με το Πνεύμα του Θεού του ζώντος, ουχί εις πλάκας λιθίνας, αλλ' εις πλάκας σαρκίνας της καρδίας.
\par 4 Τοιαύτην δε πεποίθησιν έχομεν διά του Χριστού προς τον Θεόν.
\par 5 Ουχί διότι είμεθα ικανοί αφ' εαυτών να νοήσωμέν τι ως εξ ημών αυτών, αλλ' ικανότης ημών είναι εκ του Θεού,
\par 6 όστις και έκαμεν ημάς ικανούς να ήμεθα διάκονοι της καινής διαθήκης, ουχί του γράμματος, αλλά του πνεύματος· διότι το γράμμα θανατόνει, το δε πνεύμα ζωοποιεί.
\par 7 Αλλ' εάν η διακονία του θανάτου η εν γράμμασιν εντετυπωμένη εις λίθους έγεινεν ένδοξος, ώστε οι υιοί Ισραήλ δεν ηδύναντο να ενατενίσωσιν εις το πρόσωπον του Μωϋσέως διά την δόξαν του προσώπου αυτού την μέλλουσαν να καταργηθή,
\par 8 πως η διακονία του Πνεύματος δεν θέλει είσθαι μάλλον ένδοξος;
\par 9 διότι αν η διακονία της κατακρίσεως ήναι δόξα, πολλώ μάλλον η διακονία της δικαιοσύνης υπερέχει κατά την δόξαν.
\par 10 Διότι ουδέ εδοξάσθη εν τούτω τω μέρει το δεδοξασμένον ένεκεν της υπερβαλλούσης δόξης.
\par 11 Επειδή εάν το μέλλον να καταργηθή ήτο ένδοξον, πολλώ μάλλον το μένον είναι ένδοξον.
\par 12 Έχοντες λοιπόν τοιαύτην ελπίδα πολλήν παρρησίαν μεταχειριζόμεθα,
\par 13 και ουχί καθώς ο Μωϋσής έβαλλε κάλυμμα επί το πρόσωπον αυτού διά να μη ατενίσωσιν οι υιοί Ισραήλ εις το τέλος του μέλλοντος να καταργηθή.
\par 14 Αλλ' ετυφλώθησαν αι διάνοιαι αυτών. Διότι έως της σήμερον το αυτό κάλυμμα μένει εν τη αναγνώσει της παλαιάς διαθήκης, μη ανακαλυπτόμενον, επειδή καταργείται διά του Χριστού,
\par 15 αλλ' έως σήμερον, όταν αναγινώσκηται ο Μωϋσής, κάλυμμα κείται επί της καρδίας αυτών·
\par 16 όταν όμως επιστρέψη προς τον Κύριον, θέλει αφαιρεθή το κάλυμμα.
\par 17 Ο δε Κύριος είναι το Πνεύμα· και όπου είναι το Πνεύμα του Κυρίου, εκεί ελευθερία.
\par 18 Ημείς δε πάντες βλέποντες ως εν κατόπτρω την δόξαν του Κυρίου με ανακεκαλυμμένον πρόσωπον, μεταμορφούμεθα εις την αυτήν εικόνα από δόξης εις δόξαν, καθώς από του Πνεύματος του Κυρίου.

\chapter{4}

\par 1 Διά τούτο, έχοντες την διακονίαν ταύτην, καθώς ηλεήθημεν, δεν αποκάμνομεν,
\par 2 αλλ' απηρνήθημεν τα κρυπτά της αισχύνης, μη περιπατούντες εν πανουργία μηδέ δολόνοντες τον λόγον του Θεού, αλλά με την φανέρωσιν της αληθείας συνιστώντες εαυτούς προς πάσαν συνείδησιν ανθρώπων ενώπιον του Θεού.
\par 3 Εάν δε και ήναι το ευαγγέλιον ημών κεκαλυμμένον, εις τους απολλυμένους είναι κεκαλυμμένον,
\par 4 των οποίων απίστων όντων ο Θεός του κόσμου τούτου ετύφλωσε τον νούν, διά να μη επιλάμψη εις αυτούς ο φωτισμός του ευαγγελίου της δόξης του Χριστού, όστις είναι εικών του Θεού.
\par 5 Διότι ημείς δεν κηρύττομεν εαυτούς, αλλά τον Χριστόν Ιησούν τον Κύριον, εαυτούς δε δούλους υμών διά τον Ιησούν.
\par 6 Διότι ο Θεός ο ειπών να λάμψη φως εκ του σκότους, είναι όστις έλαμψεν εν ταις καρδίαις ημών προς φωτισμόν της γνώσεως της δόξης του Θεού διά του προσώπου του Ιησού Χριστού.
\par 7 Έχομεν δε τον θησαυρόν τούτον εις οστράκινα σκεύη, διά να ήναι η υπερβολή της δυνάμεως του Θεού και ουχί εξ ημών,
\par 8 κατά πάντα θλιβόμενοι αλλ' ουχί στενοχωρούμενοι, απορούμενοι αλλ' ουχί απελπιζόμενοι,
\par 9 διωκόμενοι αλλ' ουχί εγκαταλειπόμενοι, καταβαλλόμενοι αλλ' ουχί απολλύμενοι,
\par 10 πάντοτε την νέκρωσιν του Κυρίου Ιησού περιφέροντες εν τω σώματι, διά να φανερωθή εν τω σώματι ημών και η ζωή του Ιησού.
\par 11 Διότι ημείς οι ζώντες παραδιδόμεθα πάντοτε εις τον θάνατον διά τον Ιησούν, διά να φανερωθή και η ζωή του Ιησού εν τη θνητή ημών σαρκί.
\par 12 Ώστε ο μεν θάνατος ενεργείται εν ημίν, η δε ζωή εν υμίν.
\par 13 Έχοντες δε το αυτό πνεύμα της πίστεως κατά το γεγραμμένον, Επίστευσα, διό ελάλησα, και ημείς πιστεύομεν, διό και λαλούμεν,
\par 14 εξεύροντες ότι ο αναστήσας τον Κύριον Ιησούν θέλει αναστήσει και ημάς διά του Ιησού και παραστήσει μεθ' υμών.
\par 15 Διότι τα πάντα είναι διά σας, ώστε η χάρις, πλεονάσασα διά την ευχαριστίαν των πλειοτέρων, να περισσεύση εις την δόξαν του Θεού.
\par 16 Διά τούτο δεν αποκάμνομεν, αλλ' εάν και ο εξωτερικός ημών άνθρωπος φθείρηται, ο εσωτερικός όμως ανανεούται καθ' εκάστην ημέραν.
\par 17 Διότι η προσωρινή ελαφρά θλίψις ημών εργάζεται εις ημάς καθ' υπερβολήν εις υπερβολήν αιώνιον βάρος δόξης,
\par 18 επειδή ημείς δεν ενατενίζομεν εις τα βλεπόμενα, αλλ' εις τα μη βλεπόμενα· διότι τα βλεπόμενα είναι πρόσκαιρα, τα δε μη βλεπόμενα αιώνια.

\chapter{5}

\par 1 Διότι εξεύρομεν ότι εάν η επίγειος οικία του σκηνώματος ημών χαλασθή, έχομεν εκ του Θεού οικοδομήν, οικίαν αχειροποίητον, αιώνιον εν τοις ουρανοίς.
\par 2 Επειδή εν τούτω στενάζομεν, επιποθούντες να επενδυθώμεν το κατοικητήριον ημών το ουράνιον,
\par 3 αν και ενδυθέντες αυτό δεν θέλωμεν ευρεθή γυμνοί.
\par 4 Διότι όσοι είμεθα εν τούτω τω σκηνώματι στενάζομεν υπό το βάρος αυτού· επειδή θέλομεν ουχί να εκδυθώμεν, αλλά να επενδυθώμεν, διά να καταποθή το θνητόν υπό της ζωής.
\par 5 Εκείνος δε, όστις έπλασεν ημάς δι' αυτό τούτο, είναι ο Θεός, όστις και έδωκεν εις ημάς τον αρραβώνα του Πνεύματος.
\par 6 Έχοντες λοιπόν το θάρρος πάντοτε και εξεύροντες ότι ενόσω ενδημούμεν εν τω σώματι αποδημούμεν από του Κυρίου·
\par 7 διότι περιπατούμεν διά πίστεως, ουχί διά της όψεως·
\par 8 θαρρούμεν δε και επιθυμούμεν μάλλον να αποδημήσωμεν από του σώματος και να ενδημήσωμεν προς τον Κύριον.
\par 9 Όθεν και φιλοτιμούμεθα, είτε ενδημούντες είτε αποδημούντες, να ήμεθα ευάρεστοι εις αυτόν.
\par 10 Διότι πρέπει πάντες να εμφανισθώμεν έμπροσθεν του βήματος του Χριστού, διά να ανταμειφθή έκαστος κατά τα πεπραγμένα διά του σώματος καθ' α έπραξεν, είτε αγαθόν είτε κακόν.
\par 11 Εξεύροντες λοιπόν τον φόβον του Κυρίου, τους μεν ανθρώπους καταπείθομεν, εις τον Θεόν δε είμεθα φανεροί, ελπίζω δε ότι και εις τας συνειδήσεις σας είμεθα φανεροί.
\par 12 Διότι δεν συνιστώμεν πάλιν εαυτούς εις εσάς, αλλά σας δίδομεν αφορμήν καυχήματος υπέρ ημών, διά να έχητε λόγον προς τους καυχωμένους με το πρόσωπον και ουχί με την καρδίαν.
\par 13 Διότι είτε έξω εαυτών είμεθα, διά τον Θεόν είμεθα, είτε έμφρονες είμεθα, διά σας είμεθα.
\par 14 Επειδή η αγάπη του Χριστού συσφίγγει ημάς, διότι κρίνομεν τούτο, ότι εάν εις απέθανεν υπέρ πάντων, άρα οι πάντες απέθανον·
\par 15 και απέθανεν υπέρ πάντων, διά να μη ζώσι πλέον δι' εαυτούς οι ζώντες, αλλά διά τον αποθανόντα και αναστάντα υπέρ αυτών.
\par 16 Ώστε ημείς από του νυν δεν γνωρίζομεν ουδένα κατά σάρκα· αν δε και εγνωρίσαμεν κατά σάρκα τον Χριστόν, αλλά τώρα πλέον δεν γνωρίζομεν.
\par 17 Όθεν εάν τις ήναι εν Χριστώ είναι νέον κτίσμα· τα αρχαία παρήλθον, ιδού, τα πάντα έγειναν νέα.
\par 18 Τα δε πάντα είναι εκ του Θεού, όστις διήλλαξεν ημάς προς εαυτόν διά του Ιησού Χριστού και έδωκεν εις ημάς την διακονίαν της διαλλαγής,
\par 19 δηλονότι ο Θεός ήτο εν τω Χριστώ διαλλάσσων τον κόσμον προς εαυτόν, μη λογαριάζων εις αυτούς τα πταίσματα αυτών, και ενεπιστεύθη εις ημάς τον λόγον της διαλλαγής.
\par 20 Υπέρ του Χριστού λοιπόν είμεθα πρέσβεις, ως εάν σας παρεκάλει ο Θεός δι' ημών· δεόμεθα λοιπόν υπέρ του Χριστού, διαλλάγητε προς τον Θεόν·
\par 21 διότι τον μη γνωρίσαντα αμαρτίαν έκαμεν υπέρ ημών αμαρτίαν, διά να γείνωμεν ημείς δικαιοσύνη του Θεού δι' αυτού.

\chapter{6}

\par 1 Όντες δε συνεργοί αυτού, παρακαλούμεν ενταυτώ να μη δεχθήτε την χάριν του Θεού ματαίως·
\par 2 διότι λέγει· Εν καιρώ δεκτώ επήκουσά σου και εν ημέρα σωτηρίας σε εβοήθησα· ιδού, τώρα καιρός ευπρόσδεκτος, ιδού, τώρα ημέρα σωτηρίας·
\par 3 μη δίδοντες μηδέν πρόσκομμα κατ' ουδέν, διά να μη προσαφθή μώμος εις την διακονίαν,
\par 4 αλλά εν παντί συνιστώντες εαυτούς ως υπηρέται Θεού, εν υπομονή πολλή, εν θλίψεσιν, εν ανάγκαις, εν στενοχωρίαις,
\par 5 εν ραβδισμοίς, εν φυλακαίς, εν ακαταστασίαις, εν κόποις, εν αγρυπνίαις, εν νηστείαις,
\par 6 εν καθαρότητι, εν γνώσει, εν μακροθυμία, εν χρηστότητι, εν Πνεύματι Αγίω, εν αγάπη ανυποκρίτω,
\par 7 εν λόγω αληθείας, εν δυνάμει Θεού, διά των όπλων της δικαιοσύνης των δεξιών και αριστερών,
\par 8 διά δόξης και ατιμίας, διά δυσφημίας και ευφημίας, ως πλάνοι όμως αληθείς,
\par 9 ως αγνοούμενοι αλλά καλώς γνωριζόμενοι, ως αποθνήσκοντες αλλ ιδού, ζώμεν, ως παιδευόμενοι αλλά μη θανατούμενοι,
\par 10 ως λυπούμενοι πάντοτε όμως χαίροντες, ως πτωχοί πολλούς όμως πλουτίζοντες, ως μηδέν έχοντες και τα πάντα κατέχοντες.
\par 11 Το στόμα ημών ηνοίχθη προς εσάς, Κορίνθιοι, η καρδία ημών επλατύνθη·
\par 12 δεν έχετε στενοχωρίαν εν ημίν, αλλ' έχετε στενοχωρίαν εν τοις σπλάγχνοις υμών·
\par 13 την αυτήν λοιπόν αντιμισθίαν αποδίδοντες, ως προς τέκνα λαλώ, πλατύνθητε και σεις.
\par 14 Μη ομοζυγείτε με τους απίστους· διότι τίνα μετοχήν έχει η δικαιοσύνη με την ανομίαν; τίνα δε κοινωνίαν το φως προς το σκότος;
\par 15 Τίνα δε συμφωνίαν ο Χριστός με τον Βελίαλ; ή τίνα μερίδα ο πιστός με τον άπιστον;
\par 16 Τίνα δε συμβίβασιν ο ναός του Θεού με τα είδωλα; διότι σεις είσθε ναός Θεού ζώντος, καθώς είπεν ο Θεός ότι θέλω κατοικεί εν αυτοίς και περιπατεί, και θέλω είσθαι Θεός αυτών, και αυτοί θέλουσιν είσθαι λαός μου.
\par 17 Διά τούτο. Εξέλθετε εκ μέσου αυτών και αποχωρίσθητε, λέγει Κύριος, και μη εγγίσητε ακάθαρτον, και εγώ θέλω σας δεχθή,
\par 18 και θέλω είσθαι Πατήρ σας, και σεις θέλετε είσθαι υιοί μου και θυγατέρες, λέγει Κύριος παντοκράτωρ.

\chapter{7}

\par 1 Έχοντες λοιπόν, αγαπητοί, ταύτας τας επαγγελίας, ας καθαρίσωμεν εαυτούς από παντός μολυσμού σαρκός και πνεύματος, εκπληρούντες αγιωσύνην εν φόβω Θεού.
\par 2 Δέχθητε ημάς εν υμίν· ουδένα ηδικήσαμεν, ουδένα εφθείραμεν, εις ουδένα εστάθημεν πλεονέκται.
\par 3 Δεν λέγω τούτο προς κατάκρισίν σας· διότι προείπον ότι είσθε εν ταις καρδίαις ημών, ώστε να συναποθάνωμεν και να συζώμεν.
\par 4 Πολλήν παρρησίαν έχω προς εσάς, πολλήν καύχησιν έχω διά σάς· είμαι πλήρης παρηγορίας, έχω υπερπερισσεύουσαν την χαράν εις όλην την θλίψιν ημών.
\par 5 Διότι αφού ήλθομεν εις Μακεδονίαν ουδεμίαν άνεσιν έλαβεν η σαρξ ημών, αλλά κατά πάντα εθλιβόμεθα· έξωθεν μάχαι, έσωθεν φόβοι.
\par 6 Αλλ' ο Θεός ο παρηγορών τους ταπεινούς παρηγόρησεν ημάς διά της παρουσίας του Τίτου·
\par 7 και ουχί μόνον διά της παρουσίας αυτού, αλλά και διά της παρηγορίας, την οποίαν παρηγορήθη διά σας, αναγγέλλων προς ημάς τον μέγαν πόθον σας, τον οδυρμόν σας, τον ζήλον σας υπέρ εμού, ώστε περισσότερον εχάρην,
\par 8 διότι εάν και σας ελύπησα διά της επιστολής, δεν μετανοώ, αν και μετενόουν· επειδή βλέπω ότι η επιστολή εκείνη, αν και προς ώραν, σας ελύπησε.
\par 9 Τώρα χαίρω, ουχί ότι ελυπήθητε, αλλ' ότι ελυπήθητε προς μετάνοιαν· διότι ελυπήθητε κατά Θεόν, διά να μη ζημιωθήτε εξ ημών εις ουδέν.
\par 10 Διότι η κατά Θεόν λύπη γεννά μετάνοιαν προς σωτηρίαν αμεταμέλητον· η λύπη όμως του κόσμου γεννά θάνατον.
\par 11 Διότι ιδού, αυτό τούτο, το ότι ελυπήθητε κατά Θεόν, πόσην σπουδήν εγέννησεν εις εσάς, αλλά απολογίαν, αλλά αγανάκτησιν, αλλά φόβον, αλλά πόθον, αλλά ζήλον, αλλ' εκδίκησιν. Κατά πάντα απεδείξατε εαυτούς ότι είσθε καθαροί εις τούτο το πράγμα.
\par 12 Λοιπόν, αν και σας έγραψα, δεν έκαμον τούτο διά τον αδικήσαντα, ουδέ διά τον αδικηθέντα, αλλά διά να φανερωθή προς εσάς η σπουδή ημών, την οποίαν έχομεν διά σας ενώπιον του Θεού.
\par 13 Διά τούτο παρηγορήθημεν διά την παρηγορίαν σας, και έτι περισσότερον εχάρημεν διά την χαράν του Τίτου, ότι ανεπαύθη το πνεύμα αυτού παρά πάντων υμών·
\par 14 διότι εάν εκαυχήθην τι προς αυτόν διά σας, δεν κατησχύνθην, αλλά καθώς σας ελαλήσαμεν πάντα εν αληθεία, ούτω και η καύχησις ημών η προς τον Τίτον έγεινεν αλήθεια.
\par 15 Και η αγάπη αυτού αυξάνει περισσότερον προς εσάς, όταν ενθυμήται την υπακοήν πάντων υμών, πως μετά φόβου και τρόμου εδέχθητε αυτόν.
\par 16 Χαίρω λοιπόν ότι κατά πάντα έχω θάρρος εις εσάς.

\chapter{8}

\par 1 Γνωστοποιούμεν δε εις εσάς, αδελφοί, την χάριν του Θεού την δεδομένην εις τας εκκλησίας της Μακεδονίας,
\par 2 ότι η περισσεία της χαράς αυτών, ενώ εδοκίμαζον μεγάλην θλίψιν, και η βαθεία πτωχεία αυτών ανέδειξαν εκ περισσού τον πλούτον της ελευθερότητος αυτών·
\par 3 διότι υπήρξαν κατά δύναμιν, μαρτυρώ τούτο, και υπέρ δύναμιν αυτοπροαίρετοι,
\par 4 παρακαλούντες ημάς μετά πολλής παρακλήσεως να δεχθώμεν την χάριν και την κοινωνίαν της διακονίας της εις τους αγίους,
\par 5 και ουχί μόνον καθώς ηλπίσαμεν, αλλ' εαυτούς έδωκαν πρώτον εις τον Κύριον και εις ημάς διά θελήματος του Θεού,
\par 6 ώστε παρεκαλέσαμεν τον Τίτον, καθώς ήρχισεν, ούτω και να τελειώση προς εσάς και την χάριν ταύτην.
\par 7 Καθώς λοιπόν περισσεύετε εν παντί, εν πίστει και λόγω και γνώσει και πάση σπουδή και της προς ημάς αγάπης σας, ούτω σπουδάσατε να περισσεύσητε και εν ταύτη τη χάριτι.
\par 8 Δεν λέγω τούτο κατ' επιταγήν, αλλά διά να δοκιμάσω διά της σπουδής των άλλων και την γνησιότητα της αγάπης σας·
\par 9 διότι εξεύρετε την χάριν του Κυρίου ημών Ιησού Χριστού, ότι πλούσιος ων επτώχευσε διά σας, διά να πλουτήσητε σεις με την πτωχείαν εκείνου.
\par 10 Και εις τούτο γνώμην δίδω· διότι τούτο συμφέρει εις εσάς, οίτινες ηρχίσατε από πέρυσιν ουχί μόνον το να κάμητε, αλλά και το να θέλητε·
\par 11 τώρα δε τελειώσατε και το να κάμητε, ώστε καθώς υπήρξεν η προθυμία του θέλειν, ούτω να υπάρχη και το τελειώσαι αφ' όσα έχετε.
\par 12 Διότι εάν προϋπάρχη η προθυμία, είναι τις ευπρόσδεκτος καθ' όσα έχει, ουχί καθ' όσα δεν έχει.
\par 13 Επειδή δεν θέλω να ήναι εις άλλους άνεσις, εις εσάς δε στενοχωρία
\par 14 αλλά να γείνη εν ισότητι, ώστε εν τω παρόντι καιρώ το περίσσευμά σας να αναπληρώση την στέρησιν εκείνων, διά να χρησιμεύση και το περίσσευμα εκείνων εις την στέρησίν σας, ώστε να γείνη ισότης,
\par 15 καθώς είναι γεγραμμένον· Όστις είχε συνάξει πολύ δεν ελάμβανε πλειότερον, και όστις ολίγον δεν ελάμβανεν ολιγώτερον.
\par 16 Χάρις δε εις τον Θεόν τον δίδοντα εις την καρδίαν του Τίτου την αυτήν σπουδήν διά σας,
\par 17 διότι την μεν προτροπήν εδέχθη, προθυμότερος δε ων ανεχώρησε προς εσάς αυτοπροαίρετος.
\par 18 Επέμψαμεν δε μετ' αυτού τον αδελφόν, του οποίου ο εν τω ευαγγελίω έπαινος γίνεται κατά πάσας τας εκκλησίας·
\par 19 και ουχί μόνον τούτο, αλλά και εψηφίσθη υπό των εκκλησιών συνοδοιπόρος ημών μετά της δωρεάς ταύτης της διακονουμένης υφ' ημών προς την δόξαν αυτού του Κυρίου και προς ένδειξιν της προθυμίας σας·
\par 20 φοβούμενοι τούτο, μη προσάψη τις εις ημάς μώμον εν τη αφθονία ταύτη τη διακονουμένη υφ' ημών,
\par 21 προνοούντες τα καλά ουχί μόνον ενώπιον του Κυρίου, αλλά και ενώπιον των ανθρώπων.
\par 22 Επέμψαμεν δε μετ' αυτών τον αδελφόν ημών, τον οποίον πολλάκις εδοκιμάσαμεν εν πολλοίς ότι είναι πρόθυμος, τώρα δε πολύ προθυμότερος διά την πολλήν πεποίθησιν την προς εσάς.
\par 23 Όσον μεν περί Τίτου, είναι κοινωνός εμού και εις εσάς συνεργός· όσον δε περί των αδελφών ημών, είναι απόστολοι των εκκλησιών, δόξα Χριστού.
\par 24 Την ένδειξιν λοιπόν της αγάπης σας και της καυχήσεως ημών την οποίαν έχομεν διά σας, δείξατε προς αυτούς και ενώπιον των εκκλησιών.

\chapter{9}

\par 1 Διότι περί της διακονίας της εις τους αγίους περιττόν είναι εις εμέ να σας γράφω.
\par 2 Επειδή εξεύρω την προθυμίαν σας, την οποίαν καυχώμαι περί υμών προς τους Μακεδόνας, ότι η Αχαΐα ητοιμάσθη από πέρυσι· και ο ζήλος σας διήγειρε πολλούς.
\par 3 Έπεμψα δε τους αδελφούς, διά να μη ματαιωθή ως προς τούτο η διά σας καύχησις ημών· διά να ήσθε, καθώς έλεγον, ητοιμασμένοι,
\par 4 μήπως, εάν έλθωσι μετ' εμού Μακεδόνες και σας εύρωσιν ανετοίμους, καταισχυνθώμεν ημείς, διά να μη λέγωμεν σεις, εις την πεποίθησιν ταύτην της καυχήσεως.
\par 5 Αναγκαίον λοιπόν εστοχάσθην να παρακαλέσω τους αδελφούς να έλθωσι πρότερον εις εσάς και να προετοιμάσωσι την προϋποσχεθείσαν ελεημοσύνην σας, ώστε να ήναι ετοίμη αύτη, ούτως ως ελεημοσύνη και ουχί ως πλεονεξία.
\par 6 Τούτο δε λέγω, ότι ο σπείρων με φειδωλίαν και με φειδωλίαν θέλει θερίσει, και ο σπείρων με αφθονίαν και με αφθονίαν θέλει θερίσει.
\par 7 Έκαστος κατά την προαίρεσιν της καρδίας αυτού, ουχί με λύπην ή εξ ανάγκης· διότι τον ιλαρόν δότην αγαπά ο Θεός.
\par 8 Δυνατός δε είναι ο Θεός να περισσεύση πάσαν χάριν εις εσάς, ώστε έχοντες πάντοτε εν παντί πάσαν αυτάρκειαν να περισσεύητε εις παν έργον αγαθόν,
\par 9 καθώς είναι γεγραμμένον· Εσκόρπισεν, έδωκεν εις τους πένητας· η δικαιοσύνη αυτού μένει εις τον αιώνα.
\par 10 Ο δε χορηγών σπόρον εις τον σπείροντα και άρτον προς τροφήν είθε να χορηγήση και να πληθύνη τον σπόρον σας και να αυξήση τα γεννήματα της δικαιοσύνης σας·
\par 11 πλουτιζόμενοι κατά πάντα εις πάσαν ελευθεριότητα, ήτις εργάζεται δι' ημών ευχαριστίαν εις τον Θεόν.
\par 12 Διότι η διακονία της υπηρεσίας ταύτης ουχί μόνον προσαναπληροί τας στερήσεις των αγίων, αλλά και περισσεύει διά πολλών ευχαριστιών προς τον Θεόν·
\par 13 επειδή δοκιμάζοντες την διακονίαν ταύτην δοξάζουσι τον Θεόν διά την υποταγήν της εις το ευαγγέλιον του Χριστού ομολογίας σας και διά την ελευθεριότητα της προς αυτούς και προς πάντας μεταδόσεως,
\par 14 και διά της υπέρ υμών δεήσεως αυτών, οίτινες σας επιποθούσι διά την προς εσάς υπερβάλλουσαν χάριν του Θεού.
\par 15 Χάρις δε εις τον Θεόν διά την ανεκδιήγητον αυτού δωρεάν.

\chapter{10}

\par 1 Αυτός δε εγώ ο Παύλος σας παρακαλώ διά της πραότητος και επιεικείας του Χριστού, όστις παρών μεν είμαι ταπεινός μεταξύ σας, απών δε λαμβάνω θάρρος προς εσάς·
\par 2 σας παρακαλώ δε όταν έλθω, να μη λάβω θάρρος με την πεποίθησιν εκείνην, με την οποίαν στοχάζομαι να τολμήσω εναντίον τινών, οίτινες θεωρούσιν ημάς ως κατά σάρκα περιπατούντας.
\par 3 Διότι αν και περιπατώμεν εν σαρκί, δεν πολεμούμεν όμως κατά σάρκα·
\par 4 διότι τα όπλα του πολέμου ημών δεν είναι σαρκικά, αλλά δυνατά συν Θεώ προς καθαίρεσιν οχυρωμάτων·
\par 5 επειδή καθαιρούμεν λογισμούς και παν ύψωμα επαιρόμενον εναντίον της γνώσεως του Θεού, και αιχμαλωτίζομεν παν νόημα εις την υπακοήν του Χριστού,
\par 6 και είμεθα έτοιμοι να εκδικήσωμεν πάσαν παρακοήν, όταν γείνη πλήρης η υπακοή σας.
\par 7 Τα κατά πρόσωπον βλέπετε. Εάν τις έχη πεποίθησιν εις εαυτόν ότι είναι του Χριστού, ας συλλογίζηται τούτο πάλιν αφ' εαυτού, ότι καθώς αυτός είναι του Χριστού, ούτω και ημείς είμεθα του Χριστού.
\par 8 Διότι εάν και περισσότερόν τι καυχηθώ διά την εξουσίαν ημών, την οποίαν έδωκεν εις ημάς ο Κύριος εις οικοδομήν και ουχί εις καθαίρεσίν σας, δεν θέλω αισχυνθή,
\par 9 διά να μη φανώ ότι θέλω να σας εκφοβίζω διά των επιστολών.
\par 10 Διότι αι μεν επιστολαί, λέγει τις, είναι βαρείαι και ισχυραί, η δε παρουσία του σώματος ασθενής και ο λόγος εξουθενημένος.
\par 11 Τούτο ας παρατηρή ο τοιούτος, ότι οποίοι είμεθα εις τον λόγον διά των επιστολών απόντες, τοιούτοι και παρόντες εις το έργον.
\par 12 Διότι δεν τολμώμεν να συναριθμήσωμεν ή να συγκρίνωμεν εαυτούς προς τινάς εκ των συνιστώντων εαυτούς· αλλ' αυτοί καθ' εαυτούς μετρούντες εαυτούς και προς εαυτούς συγκρίνοντες εαυτούς ανοηταίνουσιν.
\par 13 Αλλ' ημείς δεν θέλομεν καυχηθή εις τα άμετρα, αλλά κατά το μέτρον του κανόνος, το οποίον εμοίρασεν εις ημάς ο Θεός, μέτρον ώστε να φθάσωμεν έως και εις εσάς.
\par 14 Διότι δεν υπερεκτείνομεν εαυτούς ως μη φθάσαντες εις εσάς. επειδή έως και εις εσάς εφθάσαμεν διά του ευαγγελίου του Χριστού,
\par 15 και δεν καυχώμεθα εις τα άμετρα εις ξένους κόπους, αλλ' έχομεν ελπίδα, ότι αυξανομένης της πίστεώς σας, θέλομεν μεγαλυνθή εις εσάς εκ περισσού κατά τον κανόνα ημών,
\par 16 ώστε να κηρύξωμεν το ευαγγέλιον και εις τους επέκεινα υμών τόπους, ουχί να καυχηθώμεν εις τα εν αλλοτρίω κανόνι έτοιμα.
\par 17 Αλλ' όστις καυχάται, εν Κυρίω ας καυχάται·
\par 18 διότι δεν είναι δόκιμος όστις συνιστά αυτός εαυτόν, αλλ' εκείνος τον οποίον ο Κύριος συνιστά.

\chapter{11}

\par 1 Είθε να υποφέρητε ολίγον τι την αφροσύνην μου· αλλά και υποφέρετέ με.
\par 2 Διότι είμαι ζηλότυπος προς εσάς κατά ζηλοτυπίαν Θεού· επειδή σας ηρραβώνισα με ένα άνδρα, διά να σας παραστήσω παρθένον αγνήν εις τον Χριστόν.
\par 3 φοβούμαι όμως μήπως, καθώς ο όφις εξηπάτησε την Εύαν διά της πανουργίας αυτού, διαφθαρή ούτως ο νούς σας, εκπεσών από της απλότητος της εις τον Χριστόν.
\par 4 Διότι εάν ο ερχόμενος κηρύττη προς εσάς άλλον Ιησούν, τον οποίον ημείς δεν εκηρύξαμεν, ή λαμβάνητε άλλο πνεύμα, το οποίον δεν ελάβετε, ή άλλο ευαγγέλιον, το οποίον δεν εδέχθητε, καλώς ηθέλετε υποφέρει αυτόν.
\par 5 Αλλά στοχάζομαι ότι δεν είμαι εις ουδέν κατώτερος των πρωτίστων αποστόλων.
\par 6 Εάν δε και ήμαι ιδιώτης κατά τον λόγον, αλλ' ουχί κατά την γνώσιν, αλλ' εν παντί τρόπω εφανερώθημεν κατά πάντα εις εσάς.
\par 7 Η έπραξα αμαρτίαν ταπεινόνων εμαυτόν διά να υψωθήτε σεις, διότι σας εκήρυξα δωρεάν το ευαγγέλιον του Θεού;
\par 8 Άλλας εκκλησίας εγύμνωσα λαβών τα αναγκαία διά την υπηρεσίαν σας,
\par 9 και ότε ήμην παρών εις εσάς και εστερήθην, δεν κατεβάρυνα ουδένα· διότι την στέρησίν μου προσανεπλήρωσαν οι αδελφοί ελθόντες από Μακεδονίας· και κατά πάντα εφύλαξα εμαυτόν και θέλω φυλάξει αβαρή προς εσάς.
\par 10 Είναι αλήθεια του Χριστού εν εμοί ότι η καύχησις αύτη δεν θέλει αποκλεισθή εις εμέ εν τοις τόποις της Αχαΐας.
\par 11 Διά τι; διότι δεν σας αγαπώ; ο Θεός γινώσκει.
\par 12 ό,τι δε κάμνω, τούτο και θέλω κάμνει, διά να εκκόψω την αφορμήν των θελόντων αφορμήν, ίνα ευρεθώσιν εις εκείνο, διά το οποίον καυχώνται, τοιούτοι καθώς και ημείς.
\par 13 Διότι οι τοιούτοι είναι ψευδαπόστολοι, εργάται δόλιοι, μετασχηματιζόμενοι εις αποστόλους Χριστού.
\par 14 Και ουδέν θαυμαστόν· διότι αυτός ο Σατανάς μετασχηματίζεται εις άγγελον φωτός.
\par 15 Δεν είναι λοιπόν μέγα αν και οι διάκονοι αυτού μετασχηματίζωνται εις διακόνους δικαιοσύνης, των οποίων το τέλος θέλει είσθαι κατά τα έργα αυτών.
\par 16 Πάλιν λέγω, Μηδείς ας μη με στοχασθή ότι είμαι άφρων· ει δε μη, δέχθητέ με καν ως άφρονα, διά να καυχηθώ και εγώ ολίγον τι.
\par 17 ό,τι λαλώ, εις τούτο το θάρρος της καυχήσεως, δεν λαλώ κατά τον Κύριον, αλλ' ως άφρων.
\par 18 Επειδή πολλοί καυχώνται κατά την σάρκα, θέλω καυχηθή και εγώ.
\par 19 Διότι σεις ευχαρίστως υποφέρετε τους άφρονας, όντες φρόνιμοι·
\par 20 επειδή υποφέρετε, εάν τις σας καταδουλόνη, εάν τις σας κατατρώγη, εάν τις λαμβάνη τα υμών, εάν τις επαίρηται, εάν τις σας κτυπά εις το πρόσωπον.
\par 21 Κατά ατιμίαν λέγω, ως να ήμεθα ημείς ασθενείς. Αλλ' εις ό,τι τολμά τις, αφρόνως ομιλώ, τολμώ και εγώ.
\par 22 Εβραίοι είναι; και εγώ· Ισραηλίται είναι; και εγώ· σπέρμα Αβραάμ είναι; και εγώ·
\par 23 υπηρέται του Χριστού είναι; παραφρονών λαλώ, πλειότερον εγώ· εις κόπους περισσότερον, εις πληγάς καθ' υπερβολήν, εις φυλακάς περισσότερον, εις θανάτους πολλάκις.
\par 24 Υπό των Ιουδαίων πεντάκις έλαβον πληγάς τεσσαράκοντα παρά μίαν,
\par 25 τρίς ερραβδίσθην, άπαξ ελιθοβολήθην, τρίς εναυάγησα, εν ημερονύκτιον εν τω βυθώ έκαμον.
\par 26 εις οδοιπορίας πολλάκις, εις κινδύνους ποταμών, κινδύνους ληστών, κινδύνους εκ του γένους, κινδύνους εξ εθνών, κινδύνους εν πόλει, κινδύνους εν ερημία, κινδύνους εν θαλάσση, κινδύνους εν ψευδαδέλφοις.
\par 27 εν κόπω και μόχθω, εν αγρυπνίαις πολλάκις, εν πείνη και δίψη, εν νηστείαις πολλάκις, εν ψύχει και γυμνότητι·
\par 28 εκτός των εξωτερικών ο καθ' ημέραν επικείμενος εις εμέ αγών, η μέριμνα πασών των εκκλησιών.
\par 29 Τις ασθενεί, και δεν ασθενώ; τις σκανδαλίζεται, και εγώ δεν φλέγομαι;
\par 30 Εάν πρέπη να καυχώμαι, θέλω καυχηθή εις τα της ασθενείας μου.
\par 31 Ο Θεός και Πατήρ του Κυρίου ημών Ιησού Χριστού, ο ων ευλογητός εις τους αιώνας, γνωρίζει ότι δεν ψεύδομαι.
\par 32 Εν Δαμασκώ ο εθνάρχης του βασιλέως Αρέτα εφρούρει την πόλιν των Δαμασκηνών, θέλων να με πιάση,
\par 33 και διά θυρίδος από του τείχους κατεβιβάσθην εν κοφίνω και εξέφυγον τας χείρας αυτού.

\chapter{12}

\par 1 Να καυχώμαι βέβαια δεν μοι συμφέρει· διότι θέλω ελθεί εις οπτασίας και αποκαλύψεις Κυρίου.
\par 2 Γνωρίζω άνθρωπον εν Χριστώ προ ετών δεκατεσσάρων, είτε εντός του σώματος δεν εξεύρω, είτε εκτός του σώματος δεν εξεύρω, ο Θεός εξεύρει· ότι ηρπάγη ο τοιούτος έως τρίτου ουρανού.
\par 3 Και γνωρίζω τον τοιούτον άνθρωπον, είτε εντός του σώματος είτε εκτός του σώματος δεν εξεύρω, ο Θεός εξεύρει,
\par 4 ότι ηρπάγη εις τον παράδεισον και ήκουσεν ανεκλάλητα λόγια, τα οποία δεν συγχωρείται εις άνθρωπον να λαλήση.
\par 5 Υπέρ του τοιούτου θέλω καυχηθή, υπέρ δε εμαυτού δεν θέλω καυχηθή ειμή εις τας ασθενείας μου.
\par 6 Διότι εάν θελήσω να καυχηθώ, δεν θέλω είσθαι άφρων, επειδή αλήθειαν θέλω ειπεί· συστέλλομαι όμως μη στοχασθή τις εις εμέ ανώτερόν τι αφ' ό,τι με βλέπει ή ακούει τι εξ εμού.
\par 7 Και διά να μη υπεραίρωμαι διά την υπερβολήν των αποκαλύψεων, μοι εδόθη σκόλοψ εις την σάρκα, άγγελος Σατάν διά να με ραπίζη, διά να μη υπεραίρωμαι.
\par 8 Περί τούτου τρίς παρεκάλεσα τον Κύριον διά να απομακρυνθή απ' εμού·
\par 9 και μοι είπεν· Αρκεί εις σε η χάρις μου· διότι η δύναμίς μου εν αδυναμία δεικνύεται τελεία. Με άκραν λοιπόν ευχαρίστησιν θέλω καυχηθή μάλλον εις τας αδυναμίας μου, διά να κατοικήση εν εμοί η δύναμις του Χριστού.
\par 10 Όθεν ευαρεστούμαι εις τας αδυναμίας, εις τας ύβρεις, εις τας ανάγκας, εις τους διωγμούς, εις τας στενοχωρίας, υπέρ του Χριστού· διότι όταν ήμαι αδύνατος, τότε είμαι δυνατός.
\par 11 Έγεινα άφρων καυχώμενος· σεις με ηναγκάσατε. Διότι έπρεπεν εγώ να συνιστώμαι από σάς· επειδή εις ουδέν υπήρξα κατώτερος των πρωτίστων αποστόλων, αν και ήμαι μηδέν.
\par 12 Τα μεν σημεία του αποστόλου ενηργήθησαν μεταξύ σας εν πάση υπομονή, διά θαυμάτων και τεραστίων και δυνάμεων.
\par 13 Διότι κατά τι εμείνατε κατώτεροι των λοιπών εκκλησιών, ειμή ότι αυτός εγώ δεν σας κατεβάρυνα; συγχωρήσατέ μοι την αδικίαν ταύτην.
\par 14 Ιδού, τρίτην φοράν είμαι έτοιμος να έλθω προς εσάς, και δεν θέλω σας καταβαρύνει· διότι δεν ζητώ τα υμών, αλλ' υμάς. Διότι δεν χρεωστούσι τα τέκνα να θησαυρίζωσι διά τους γονείς, αλλ' οι γονείς διά τα τέκνα.
\par 15 Εγώ δε με άκραν χαράν θέλω δαπανήσει και όλως δαπανηθή υπέρ των ψυχών σας, αν και ενώ σας αγαπώ περισσότερον, αγαπώμαι ολιγώτερον.
\par 16 Έστω όμως, εγώ δεν σας κατεβάρυνα, αλλά πανούργος ων, σας επίασα με δόλον.
\par 17 Μήπως διά τινός εξ εκείνων, τους οποίους έστειλα προς εσάς, δι' αυτού επλεονέκτησα από σας;
\par 18 Παρεκάλεσα τον Τίτον, και μετ' αυτού απέστειλα τον αδελφόν· μήπως ο Τίτος επλεονέκτησέ τι από σας; ουχί με το αυτό πνεύμα περιεπατήσαμεν; ουχί εις τα αυτά ίχνη;
\par 19 Πάλιν νομίζετε ότι απολογούμεθα προς εσάς; ενώπιον του Θεού λαλούμεν εν Χριστώ· πράττομεν δε τα πάντα, αγαπητοί, διά την οικοδομήν σας.
\par 20 Διότι φοβούμαι μήπως ελθών δεν σας εύρω οποίους θέλω, και εγώ ευρεθώ εις εσάς οποίον δεν θέλετε, μήπως ήναι μεταξύ σας έριδες, ζηλοτυπίαι, θυμοί, μάχαι, καταλαλιαί, ψιθυρισμοί, αλαζονείαι, ακαταστασίαι,
\par 21 μήπως πάλιν όταν έλθω προς εσάς, με ταπεινώση ο Θεός μου και πενθήσω πολλούς των προαμαρτησάντων και μη μετανοησάντων διά την ακαθαρσίαν και πορνείαν και ασέλγειαν, την οποίαν έπραξαν.

\chapter{13}

\par 1 Τρίτην ταύτην φοράν έρχομαι προς εσάς· επί στόματος δύο μαρτύρων και τριών θέλει βεβαιούσθαι πας λόγος.
\par 2 Προείπον και προλέγω, ως παρών την δευτέραν φοράν, και τώρα απών γράφω προς τους προαμαρτήσαντας και τους λοιπούς πάντας, ότι εάν έλθω πάλιν, δεν θέλω φεισθή·
\par 3 επειδή ζητείτε δοκιμήν του δι' εμού λαλούντος Χριστού, όστις δεν είναι ασθενής προς εσάς, αλλ' είναι δυνατός μεταξύ σας.
\par 4 Διότι αν εσταυρώθη εξ ασθενείας, ζη όμως εκ δυνάμεως Θεού. διότι και ημείς ασθενούμεν εν αυτώ, πλην εκ δυνάμεως Θεού θέλομεν ζήσει μετ' αυτού εις εσάς.
\par 5 Εαυτούς εξετάζετε αν ήσθε εν τη πίστει, εαυτούς δοκιμάζετε. Η δεν γνωρίζετε εαυτούς ότι ο Ιησούς Χριστός είναι εν υμίν; εκτός εάν ήσθε αδόκιμοι κατά τι.
\par 6 Ελπίζω δε ότι θέλετε γνωρίσει ότι ημείς δεν είμεθα αδόκιμοι.
\par 7 Εύχομαι δε εις τον Θεόν να μη πράξητε μηδέν κακόν, ουχί διά να φανώμεν ημείς δόκιμοι, αλλά διά να πράττητε σεις το καλόν, ημείς δε ας ήμεθα ως αδόκιμοι.
\par 8 Διότι δεν δυνάμεθα να πράξωμέν τι κατά της αληθείας, αλλ' υπέρ της αληθείας.
\par 9 Επειδή χαίρομεν όταν ημείς ασθενώμεν, σεις δε ήσθε δυνατοί· τούτο μάλιστα και ευχόμεθα, την τελειοποίησίν σας.
\par 10 Διά τούτο ταύτα γράφω απών, διά να μη φερθώ αποτόμως παρών κατά την εξουσίαν, την οποίαν μοι έδωκεν ο Κύριος προς οικοδομήν και ουχί προς καθαίρεσιν.
\par 11 Λοιπόν, αδελφοί, χαίρετε, τελειοποιείσθε, παραμυθείσθε, φρονείτε το αυτό, ειρηνεύετε· και ο Θεός της αγάπης και της ειρήνης θέλει είσθαι μεθ' υμών.
\par 12 Ασπάσθητε αλλήλους εν φιλήματι αγίω.
\par 13 Σας ασπάζονται πάντες οι άγιοι.
\par 14 Η χάρις του Κυρίου Ιησού Χριστού και η αγάπη του Θεού και η κοινωνία του Αγίου Πνεύματος είη μετά πάντων υμών· αμήν.


\end{document}