\begin{document}

\title{Galatians}


\chapter{1}

\par Παύλος απόστολος ουχί από ανθρώπων, ουδέ δι' ανθρώπου, αλλά διά Ιησού Χριστού και Θεού Πατρός του αναστήσαντος αυτόν εκ νεκρών,
\par 2 και πάντες οι μετ' εμού αδελφοί, προς τας εκκλησίας της Γαλατίας·
\par 3 χάρις είη υμίν και ειρήνη από Θεού Πατρός και Κυρίου ημών Ιησού Χριστού,
\par 4 όστις έδωκεν εαυτόν διά τας αμαρτίας ημών, διά να ελευθερώση ημάς εκ του παρόντος πονηρού αιώνος κατά το θέλημα του Θεού και Πατρός ημών,
\par 5 εις τον οποίον έστω η δόξα εις τους αιώνας των αιώνων· αμήν.
\par 6 Θαυμάζω ότι τόσον ταχέως μεταφέρεσθε από εκείνου, όστις σας εκάλεσε διά της χάριτος του Χριστού, εις άλλο ευαγγέλιον,
\par 7 το οποίον δεν είναι άλλο, αλλ' υπάρχουσί τινές, οι οποίοι σας ταράττουσι και θέλουσι να μετατρέψωσι το ευαγγέλιον του Χριστού.
\par 8 Αλλά και εάν ημείς ή άγγελος εξ ουρανού σας κηρύττη άλλο ευαγγέλιον παρά εκείνο, το οποίον σας εκηρύξαμεν, ας ήναι ανάθεμα.
\par 9 Καθώς προείπομεν, και τώρα πάλιν λέγω· Εάν τις σας κηρύττη άλλο ευαγγέλιον παρά εκείνο, το οποίον παρελάβετε, ας ήναι ανάθεμα.
\par 10 Διότι τώρα ανθρώπους πείθω ή τον Θεόν; ή ζητώ να αρέσκω εις ανθρώπους; διότι εάν ακόμη ήρεσκον εις ανθρώπους, δεν ήθελον είσθαι δούλος Χριστού.
\par 11 Αλλά σας γνωστοποιώ, αδελφοί, ότι το ευαγγέλιον το κηρυχθέν υπ' εμού δεν είναι ανθρώπινον·
\par 12 διότι ουδ' εγώ παρέλαβον αυτό παρά ανθρώπου ούτε εδιδάχθην, αλλά δι' αποκαλύψεως Ιησού Χριστού.
\par 13 Διότι ηκούσατε την ποτέ διαγωγήν μου εν τω Ιουδαϊσμώ, ότι καθ' υπερβολήν εδίωκον την εκκλησίαν του Θεού και εκακοποίουν αυτήν,
\par 14 και προέκοπτον εις τον Ιουδαϊσμόν υπέρ πολλούς συνηλικιώτας εν τω γένει μου, περισσότερον ζηλωτής υπάρχων των πατρικών μου παραδόσεων.
\par 15 Ότε δε ηυδόκησεν ο Θεός, ο προσδιορίσας με εκ κοιλίας μητρός μου και καλέσας διά της χάριτος αυτού,
\par 16 να αποκαλύψη τον Υιόν αυτού εν εμοί, διά να κηρύττω αυτόν μεταξύ των εθνών, ευθύς δεν συνεβουλεύθην σάρκα και αίμα,
\par 17 ουδέ ανέβην εις Ιεροσόλυμα προς τους προ εμού αποστόλους, αλλ' απήλθον εις Αραβίαν και πάλιν υπέστρεψα εις Δαμασκόν.
\par 18 Έπειτα μετά έτη τρία ανέβην εις Ιεροσόλυμα, διά να γνωρίσω προσωπικώς τον Πέτρον, και έμεινα παρ' αυτώ ημέρας δεκαπέντε·
\par 19 άλλον δε των αποστόλων δεν είδον, ειμή Ιάκωβον τον αδελφόν του Κυρίου.
\par 20 Όσα δε σας γράφω, ιδού, ενώπιον του Θεού ομολογώ ότι δεν ψεύδομαι.
\par 21 Έπειτα ήλθον εις τους τόπους της Συρίας και της Κιλικίας.
\par 22 Και ήμην προσωπικώς αγνοούμενος εις τας εκκλησίας της Ιουδαίας τας εν Χριστώ·
\par 23 ήκουον δε μόνον ότι ο ποτέ διώκων ημάς, τώρα κηρύττει την πίστιν, την οποίαν ποτέ κατεπολέμει,
\par 24 και εδόξαζον τον Θεόν δι' εμέ.

\chapter{2}

\par Έπειτα μετά δεκατέσσαρα έτη πάλιν ανέβην εις Ιεροσόλυμα μετά του Βαρνάβα, συμπαραλαβών και τον Τίτον·
\par 2 ανέβην δε κατά αποκάλυψιν· και παρέστησα προς αυτούς το ευαγγέλιον, το οποίον κηρύττω μεταξύ των εθνών, κατ' ιδίαν δε προς τους επισημοτέρους, μήπως τρέχω ή έτρεξα εις μάτην.
\par 3 Αλλ' ουδέ ο Τίτος ο μετ' εμού, Έλλην ων, ηναγκάσθη να περιτμηθή,
\par 4 αλλά διά τους παρεισάκτους ψευδαδέλφους, οίτινες παρεισήλθον διά να κατασκοπεύσωσι την ελευθερίαν ημών, την οποίαν έχομεν εν Χριστώ Ιησού, διά να μας καταδουλώσωσιν·
\par 5 εις τους οποίους ουδέ προς ώραν υπεχωρήσαμεν υποτασσόμενοι, διά να διαμείνη εις εσάς η αλήθεια του ευαγγελίου.
\par 6 Περί δε των νομιζομένων ότι είναι τι, οποίοι ποτέ και αν ήσαν, ουδόλως φροντίζω· ο Θεός δεν βλέπει εις πρόσωπον ανθρώπου· διότι εις εμέ οι επισημότεροι δεν προσέθεσαν ουδέν περισσότερον,
\par 7 αλλά το εναντίον, αφού είδον ότι ενεπιστεύθην να κηρύττω το ευαγγέλιον προς τους απεριτμήτους καθώς ο Πέτρος προς τους περιτετμημένους·
\par 8 διότι ο ενεργήσας εις τον Πέτρον, ώστε να αποσταλή προς τους περιτετμημένους, ενήργησε και εις εμέ, προς τους εθνικούς·
\par 9 και αφού εγνώρισαν την χάριν την δοθείσαν εις εμέ Ιάκωβος και Κηφάς και Ιωάννης, οι θεωρούμενοι ότι είναι στύλοι, δεξιάς έδωκαν κοινωνίας εις εμέ και εις τον Βαρνάβαν, διά να υπάγωμεν ημείς μεν εις τα έθνη, αυτοί δε εις τους περιτετμημένους·
\par 10 μόνον μας παρήγγειλαν να ενθυμώμεθα τους πτωχούς, το οποίον και εσπούδασα αυτό τούτο να κάμω.
\par 11 Ότε δε ήλθεν ο Πέτρος εις την Αντιόχειαν, ηναντιώθην εις αυτόν κατά πρόσωπον, διότι ήτο αξιόμεμπτος.
\par 12 Επειδή πριν έλθωσί τινές από του Ιακώβου, συνέτρωγε με τους εθνικούς· ότε δε ήλθον, συνεστέλλετο και απεχώριζεν εαυτόν, φοβούμενος τους εκ περιτομής.
\par 13 Και μετ' αυτού συνυπεκρίθησαν και οι λοιποί Ιουδαίοι, ώστε και ο Βαρνάβας συμπαρεσύρθη εις την υπόκρισιν αυτών.
\par 14 Αλλ' ότε εγώ είδον ότι δεν ορθοποδούσι προς την αλήθειαν του ευαγγελίου, είπον προς τον Πέτρον έμπροσθεν πάντων· Εάν συ Ιουδαίος ων ζης εθνικώς και ουχί Ιουδαϊκώς, διά τι αναγκάζεις τους εθνικούς να ιουδαΐζωσιν;
\par 15 Ημείς εκ γεννήσεως Ιουδαίοι όντες και ουχί εκ των εθνών αμαρτωλοί,
\par 16 εξεύροντες ότι δεν δικαιούται άνθρωπος εξ έργων νόμου ειμή διά πίστεως Ιησού Χριστού, και ημείς επιστεύσαμεν εις τον Ιησούν Χριστόν, διά να δικαιωθώμεν εκ πίστεως Χριστού και ουχί εξ έργων νόμου, διότι δεν θέλει δικαιωθή εξ έργων νόμου ουδείς άνθρωπος.
\par 17 Αλλ' εάν ζητούντες να δικαιωθώμεν εις τον Χριστόν ευρέθημεν και ημείς αμαρτωλοί, άρα ο Χριστός αμαρτίας είναι διάκονος; Μη γένοιτο.
\par 18 Διότι εάν όσα κατέστρεψα ταύτα πάλιν οικοδομώ, παραβάτην δεικνύω εμαυτόν.
\par 19 Διότι εγώ διά του νόμου απέθανον εις τον νόμον, διά να ζήσω εις τον Θεόν.
\par 20 Μετά του Χριστού συνεσταυρώθην· ζω δε ουχί πλέον εγώ, αλλ' ο Χριστός ζη εν εμοί· καθ' ο δε τώρα ζω εν σαρκί, ζω εν τη πίστει του Υιού του Θεού, όστις με ηγάπησε και παρέδωκεν εαυτόν υπέρ εμού.
\par 21 Δεν αθετώ την χάριν του Θεού· διότι αν η δικαίωσις γίνηται διά του νόμου, άρα ο Χριστός εις μάτην απέθανε.

\chapter{3}

\par Ω ανόητοι Γαλάται, τις σας εβάσκανεν, ώστε να μη πείθησθε εις την αλήθειαν σεις, έμπροσθεν εις τους οφθαλμούς των οποίων ο Ιησούς Χριστός, διεγράφη εσταυρωμένος μεταξύ σας;
\par 2 Τούτο μόνον θέλω να μάθω από σάς· Εξ έργων νόμου ελάβετε το Πνεύμα, ή εξ ακοής της πίστεως;
\par 3 τόσον ανόητοι είσθε; αφού ηρχίσατε με το Πνεύμα, τώρα τελειόνετε με την σάρκα;
\par 4 εις μάτην επάθετε τόσα; αν μόνον εις μάτην.
\par 5 Εκείνος λοιπόν όστις χορηγεί εις εσάς το Πνεύμα και ενεργεί θαύματα μεταξύ σας, εξ έργων νόμου κάμνει ταύτα ή εξ ακοής πίστεως;
\par 6 καθώς ο Αβραάμ επίστευσεν εις τον Θεόν, και ελογίσθη εις αυτόν εις δικαιοσύνην.
\par 7 Εξεύρετε λοιπόν ότι οι όντες εκ πίστεως, ούτοι είναι υιοί του Αβραάμ.
\par 8 Προϊδούσα δε η γραφή ότι εκ πίστεως δικαιόνει τα έθνη ο Θεός, προήγγειλεν εις τον Αβραάμ ότι θέλουσιν ευλογηθή εν σοι πάντα τα έθνη.
\par 9 Ώστε οι όντες εκ πίστεως ευλογούνται μετά του πιστού Αβραάμ.
\par 10 Διότι όσοι είναι εξ έργων νόμου, υπό κατάραν είναι· επειδή είναι γεγραμμένον· Επικατάρατος πας όστις δεν εμμένει εν πάσι τοις γεγραμμένοις εν τω βιβλίω του νόμου, ώστε να πράξη αυτά.
\par 11 Ότι δε ουδείς δικαιούται διά του νόμου ενώπιον του Θεού, είναι φανερόν, διότι ο δίκαιος θέλει ζήσει εκ πίστεως,
\par 12 ο δε νόμος δεν είναι εκ πίστεως· αλλ' ο άνθρωπος ο πράττων αυτά θέλει ζήσει δι' αυτών.
\par 13 Ο Χριστός εξηγόρασεν ημάς εκ της κατάρας του νόμου, γενόμενος κατάρα υπέρ ημών· διότι είναι γεγραμμένον· Επικατάρατος πας ο κρεμάμενος επί ξύλου.
\par 14 Διά να έλθη εις τα έθνη η ευλογία του Αβραάμ διά Ιησού Χριστού, ώστε να λάβωμεν την επαγγελίαν του Πνεύματος διά της πίστεως.
\par 15 Αδελφοί, κατά άνθρωπον ομιλώ· όμως και ανθρώπου διαθήκην κεκυρωμένην ουδείς αθετεί ή προσθέτει εις αυτήν.
\par 16 Προς δε τον Αβραάμ ελαλήθησαν αι επαγγελίαι και προς το σπέρμα αυτού· δεν λέγει, Και προς τα σπέρματα, ως περί πολλών, αλλ' ως περί ενός, Και προς το σπέρμα σου, όστις είναι ο Χριστός.
\par 17 Τούτο δε λέγω· ότι διαθήκην προκεκυρωμένην εις τον Χριστόν υπό του Θεού ο μετά έτη τετρακόσια τριάκοντα γενόμενος νόμος δεν ακυρόνει, ώστε να καταργήση την επαγγελίαν.
\par 18 Διότι εάν η κληρονομία ήναι εκ νόμου, δεν είναι πλέον εξ επαγγελίας· αλλ' εις τον Αβραάμ δι' επαγγελίας εχάρισε ταύτην ο Θεός.
\par 19 Διά τι λοιπόν εδόθη ο νόμος; Εξ αιτίας των παραβάσεων προσετέθη, εωσού έλθη το σπέρμα, προς το οποίον έγεινεν η επαγγελία, διαταχθείς δι' αγγέλων διά χειρός μεσίτου·
\par 20 ο δε μεσίτης δεν είναι ενός, ο Θεός όμως είναι εις.
\par 21 Ο νόμος λοιπόν εναντίος των επαγγελιών του Θεού είναι; Μη γένοιτο. Διότι εάν ήθελε δοθή νόμος δυνάμενος να ζωοποιήση, η δικαιοσύνη ήθελεν είσθαι τωόντι εκ του νόμου·
\par 22 η γραφή όμως συνέκλεισε τα πάντα υπό την αμαρτίαν, διά να δοθή η επαγγελία εκ πίστεως Ιησού Χριστού εις τους πιστεύοντας.
\par 23 Πριν δε έλθη η πίστις, εφρουρούμεθα υπό τον νόμον συγκεκλεισμένοι εις την πίστιν, ήτις έμελλε να αποκαλυφθή.
\par 24 Ώστε ο νόμος έγεινε παιδαγωγός ημών εις τον Χριστόν, διά να δικαιωθώμεν εκ πίστεως.
\par 25 αφού όμως ήλθεν η πίστις, δεν είμεθα πλέον υπό παιδαγωγόν.
\par 26 Διότι πάντες είσθε υιοί Θεού διά της πίστεως της εν Χριστώ Ιησού·
\par 27 επειδή όσοι εβαπτίσθητε εις Χριστόν, Χριστόν ενεδύθητε.
\par 28 Δεν είναι πλέον Ιουδαίος ουδέ Έλλην, δεν είναι δούλος ουδέ ελεύθερος, δεν είναι άρσεν και θήλυ· διότι πάντες σεις είσθε εις εν Χριστώ Ιησού·
\par 29 εάν δε ήσθε του Χριστού, άρα είσθε σπέρμα του Αβραάμ και κατά την επαγγελίαν κληρονόμοι.

\chapter{4}

\par Λέγω δε, εφ' όσον χρόνον ο κληρονόμος είναι νήπιος, δεν διαφέρει δούλου, αν και ήναι κύριος πάντων,
\par 2 αλλ' είναι υπό επιτρόπους και οικονόμους μέχρι της προθεσμίας υπό του πατρός.
\par 3 Ούτω και ημείς, ότε ήμεθα νήπιοι, υπό τα στοιχεία του κόσμου ήμεθα δεδουλωμένοι·
\par 4 ότε όμως ήλθε το πλήρωμα του χρόνου, εξαπέστειλεν ο Θεός τον Υιόν αυτού, όστις εγεννήθη εκ γυναικός και υπετάγη εις τον νόμον,
\par 5 διά να εξαγοράση τους υπό νόμον, διά να λάβωμεν την υιοθεσίαν.
\par 6 Και επειδή είσθε υιοί, εξαπέστειλεν ο Θεός το Πνεύμα του Υιού αυτού εις τας καρδίας σας, το οποίον κράζει· Αββά, ο Πατήρ.
\par 7 Όθεν δεν είσαι πλέον δούλος αλλ' υιός· εάν δε υιός, και κληρονόμος του Θεού διά του Χριστού.
\par 8 Αλλά τότε μεν μη γνωρίζοντες τον Θεόν, εδουλεύσατε εις τους μη φύσει όντας Θεούς·
\par 9 τώρα δε αφού εγνωρίσατε τον Θεόν, μάλλον δε εγνωρίσθητε υπό του Θεού, πως επιστρέφετε πάλιν εις τα ασθενή και πτωχά στοιχεία, εις τα οποία πάλιν ως πρότερον θέλετε να δουλεύητε;
\par 10 Ημέρας παρατηρείτε και μήνας και καιρούς και ενιαυτούς.
\par 11 Φοβούμαι διά σας, μήπως ματαίως εκοπίασα εις εσάς.
\par 12 Γίνεσθε ως εγώ, διότι και εγώ είμαι καθώς σεις, αδελφοί, σας παρακαλώ, ουδόλως με ηδικήσατε·
\par 13 εξεύρετε δε ότι πρότερον σας εκήρυξα το ευαγγέλιον εν ασθενεία της σαρκός,
\par 14 και δεν εξουθενήσατε ουδ' απερρίψατε τον πειρασμόν μου τον εν τη σαρκί μου, αλλά με εδέχθητε ως άγγελον Θεού, ως Χριστόν Ιησούν.
\par 15 Τις λοιπόν ήτο ο μακαρισμός σας; επειδή μαρτυρώ προς εσάς ότι ει δυνατόν τους οφθαλμούς σας ηθέλετε εκβάλει και δώσει εις εμέ.
\par 16 Εχθρός σας έγεινα λοιπόν, διότι σας λέγω την αλήθειαν;
\par 17 Δεικνύουσι ζήλον προς εσάς, ουχί όμως καλόν, αλλά θέλουσι να σας αποκλείσωσι, διά να έχητε σεις ζήλον προς αυτούς.
\par 18 Καλόν δε είναι να ήσθε ζηλωταί προς το καλόν πάντοτε και ουχί μόνον όταν ευρίσκωμαι μεταξύ σας.
\par 19 Τεκνία μου, διά τους οποίους πάλιν είμαι εις ωδίνας, εωσού μορφωθή ο Χριστός εν υμίν·
\par 20 ήθελον δε να παρευρίσκωμαι μεταξύ σας τώρα και να αλλάξω την φωνήν μου, διότι απορώ διά σας.
\par 21 Είπατέ μοι οι θέλοντες να ήσθε υπό νόμον· τον νόμον δεν ακούετε;
\par 22 Διότι είναι γεγραμμένον ότι ο Αβραάμ εγέννησε δύο υιούς, ένα εκ της δούλης και ένα εκ της ελευθέρας.
\par 23 Αλλ' ο μεν εκ της δούλης εγεννήθη κατά σάρκα, ο δε εκ της ελευθέρας διά της επαγγελίας·
\par 24 Τα οποία είναι κατά αλληγορίαν· διότι αύται είναι αι δύο διαθήκαι, μία μεν από του όρους Σινά, η γεννώσα προς δουλείαν, ήτις είναι η Άγαρ.
\par 25 Διότι το Άγαρ είναι το όρος Σινά εν τη Αραβία, και ταυτίζεται με την σημερινήν Ιερουσαλήμ, είναι δε εις δουλείαν μετά των τέκνων αυτής·
\par 26 η δε άνω Ιερουσαλήμ είναι ελευθέρα, ήτις είναι μήτηρ πάντων ημών.
\par 27 Διότι είναι γεγραμμένον· Ευφράνθητι, στείρα η μη τίκτουσα, έκβαλε φωνήν και βόησον, η μη ωδίνουσα· διότι τα τέκνα της ερήμου είναι πλειότερα παρά τα τέκνα της εχούσης τον άνδρα.
\par 28 Ημείς δε, αδελφοί, καθώς ο Ισαάκ επαγγελίας τέκνα είμεθα.
\par 29 Αλλά καθώς τότε ο κατά σάρκα γεννηθείς εδίωκε τον κατά πνεύμα, ούτω και τώρα.
\par 30 Αλλά τι λέγει η γραφή; Έκβαλε την δούλην και τον υιόν αυτής· διότι δεν θέλει κληρονομήσει ο υιός της δούλης μετά του υιού της ελευθέρας.
\par 31 Λοιπόν, αδελφοί, δεν είμεθα της δούλης τέκνα, αλλά της ελευθέρας.

\chapter{5}

\par Εν τη ελευθερία λοιπόν, με την οποίαν ηλευθέρωσεν ημάς ο Χριστός, μένετε σταθεροί, και μη υποβληθήτε πάλιν εις ζυγόν δουλείας.
\par 2 Ιδού, εγώ ο Παύλος σας λέγω ότι εάν περιτέμνησθε, ο Χριστός δεν θέλει σας ωφελήσει ουδέν.
\par 3 Μαρτύρομαι δε πάλιν προς πάντα άνθρωπον περιτεμνόμενον, ότι είναι χρεώστης να εκτελή όλον τον νόμον.
\par 4 Απεχωρίσθητε από του Χριστού όσοι δικαιόνεσθε διά του νόμου, εξεπέσατε από της χάριτος·
\par 5 διότι ημείς διά του Πνεύματος προσδοκώμεν εκ πίστεως την ελπίδα της δικαιώσεως.
\par 6 Διότι εν Χριστώ Ιησού ούτε περιτομή έχει ισχύν τινά, ούτε ακροβυστία, αλλά πίστις δι' αγάπης ενεργουμένη.
\par 7 Ετρέχετε καλώς· τις σας ημπόδισεν ώστε να μη πείθησθε εις την αλήθειαν;
\par 8 Η κατάπεισις αύτη δεν είναι εξ εκείνου, όστις σας καλεί.
\par 9 Ολίγη ζύμη καθιστά όλον το φύραμα ένζυμον.
\par 10 Εγώ έχω πεποίθησιν εις εσάς διά του Κυρίου ότι δεν θέλετε φρονήσει ουδέν άλλο· όστις όμως σας ταράττει, αυτός θέλει υποφέρει την ποινήν, οποίος και αν ήναι.
\par 11 Εγώ δε, αδελφοί, εάν ακόμη κηρύττω περιτομήν, διά τι πλέον κατατρέχομαι; άρα κατηργήθη το σκάνδαλον του σταυρού.
\par 12 Είθε να αποκοπώσιν οι ταράττοντές σας.
\par 13 Διότι σεις, αδελφοί, προσεκλήθητε εις ελευθερίαν· μόνον μη μεταχειρίζεσθε την ελευθερίαν εις αφορμήν της σαρκός, αλλά διά της αγάπης δουλεύετε αλλήλους.
\par 14 Διότι όλος ο νόμος εις ένα λόγον συμπληρούται, εις τον, Θέλεις αγαπά τον πλησίον σου ως σεαυτόν.
\par 15 Εάν όμως δάκνητε και κατατρώγητε αλλήλους, προσέχετε μη υπ' αλλήλων αφανισθήτε.
\par 16 Λέγω λοιπόν, Περιπατείτε κατά το Πνεύμα και δεν θέλετε εκπληροί την επιθυμίαν της σαρκός.
\par 17 Διότι η σαρξ επιθυμεί εναντία του Πνεύματος, το δε Πνεύμα εναντία της σαρκός· ταύτα δε αντίκεινται προς άλληλα, ώστε εκείνα, τα οποία θέλετε, να μη πράττητε.
\par 18 Αλλ' εάν οδηγήσθε υπό του Πνεύματος, δεν είσθε υπό νόμον.
\par 19 Φανερά δε είναι τα έργα της σαρκός, τα οποία είναι μοιχεία, πορνεία, ακαθαρσία, ασέλγεια,
\par 20 ειδωλολατρεία, φαρμακεία, έχθραι, έριδες, ζηλοτυπίαι, θυμοί, μάχαι, διχοστασίαι, αιρέσεις,
\par 21 φθόνοι, φόνοι, μέθαι, κώμοι, και τα όμοια τούτων, περί των οποίων σας προλέγω, καθώς και προείπον, ότι οι τα τοιαύτα πράττοντες βασιλείαν Θεού δεν θέλουσι κληρονομήσει.
\par 22 Ο δε καρπός του Πνεύματος είναι αγάπη, χαρά, ειρήνη, μακροθυμία, χρηστότης, αγαθωσύνη, πίστις,
\par 23 πραότης, εγκράτεια· κατά των τοιούτων δεν υπάρχει νόμος.
\par 24 Όσοι δε είναι του Χριστού εσταύρωσαν την σάρκα ομού με τα πάθη και τας επιθυμίας.
\par 25 Εάν ζώμεν κατά το Πνεύμα, ας περιπατώμεν και κατά το Πνεύμα.
\par 26 Μη γινώμεθα κενόδοξοι, αλλήλους ερεθίζοντες, αλλήλους φθονούντες.

\chapter{6}

\par Αδελφοί, και εάν άνθρωπος απερισκέπτως πέση εις κανέν αμάρτημα, σεις οι πνευματικοί διορθόνετε τον τοιούτον με πνεύμα πραότητος, προσέχων εις σεαυτόν, μη και συ πειρασθής.
\par 2 Αλλήλων τα βάρη βαστάζετε και ούτως εκπληρώσατε τον νόμον του Χριστού.
\par 3 Διότι εάν τις νομίζη ότι είναι τι ενώ είναι μηδέν, εαυτόν εξαπατά.
\par 4 Αλλ' έκαστος ας εξετάζη το εαυτού έργον, και τότε εις εαυτόν μόνον θέλει έχει το καύχημα και ουχί εις τον άλλον·
\par 5 διότι έκαστος το εαυτού φορτίον θέλει βαστάσει.
\par 6 Ο δε κατηχούμενος τον λόγον ας κάμνη τον κατηχούντα μέτοχον εις πάντα τα αγαθά αυτού.
\par 7 Μη πλανάσθε, ο Θεός δεν εμπαίζεται· επειδή ό,τι αν σπείρη ο άνθρωπος, τούτο και θέλει θερίσει·
\par 8 διότι ο σπείρων εις την σάρκα εαυτού θέλει θερίσει εκ της σαρκός φθοράν, αλλ' ο σπείρων εις το Πνεύμα θέλει θερίσει εκ του Πνεύματος ζωήν αιώνιον.
\par 9 Ας μη αποκάμνωμεν δε πράττοντες το καλόν· διότι εάν δεν αποκάμνωμεν, θέλομεν θερίσει εν τω δέοντι καιρώ.
\par 10 Άρα λοιπόν ενόσω έχομεν καιρόν, ας εργαζώμεθα το καλόν προς πάντας, μάλιστα δε προς τους οικείους της πίστεως.
\par 11 Ίδετε πόσον μακράν επιστολήν σας έγραψα με την χείρα μου.
\par 12 Όσοι θέλουσι να αρέσκωσι κατά την σάρκα, ούτοι σας αναγκάζουσι να περιτέμνησθε, μόνον διά να μη διώκωνται διά τον σταυρόν του Χριστού.
\par 13 Διότι ουδέ οι περιτεμνόμενοι αυτοί φυλάττουσι τον νόμον· αλλά θέλουσι να περιτέμνησθε σεις, διά να έχωσι καύχησιν εις την σάρκα σας.
\par 14 Εις εμέ δε μη γένοιτο να καυχώμαι ειμή εις τον σταυρόν του Κυρίου ημών Ιησού Χριστού, διά του οποίου ο κόσμος εσταυρώθη ως προς εμέ και εγώ ως προς τον κόσμον.
\par 15 Διότι εν Χριστώ Ιησού ούτε περιτομή ισχύει τι ούτε ακροβυστία, αλλά νέα κτίσις.
\par 16 Και όσοι περιπατήσωσι κατά τον κανόνα τούτον, ειρήνη επ' αυτούς και έλεος, και επί τον Ισραήλ του Θεού.
\par 17 Εις το εξής μηδείς ας μη δίδη εις εμέ ενόχλησιν· διότι εγώ βαστάζω τα στίγματα του Κυρίου Ιησού εν τω σώματί μου.
\par 18 Η χάρις του Κυρίου ημών Ιησού Χριστού είη μετά του πνεύματος υμών, αδελφοί· αμήν.


\end{document}