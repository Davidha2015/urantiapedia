\begin{document}

\title{Φιλιππησίους}


\chapter{1}

\par 1 Παύλος, απόστολος Ιησού Χριστού διά θελήματος Θεού, προς τους αγίους τους όντας εν Εφέσω και πιστούς εν Χριστώ Ιησού·
\par 2 χάρις είη υμίν και ειρήνη από Θεού Πατρός ημών και Κυρίου Ιησού Χριστού.
\par 3 Ευλογητός ο Θεός και Πατήρ του Κυρίου ημών Ιησού Χριστού, ο ευλογήσας ημάς εν πάση ευλογία πνευματική εις τα επουράνια διά Χριστού,
\par 4 καθώς εξέλεξεν ημάς δι' αυτού προ καταβολής κόσμου, διά να ήμεθα άγιοι και άμωμοι ενώπιον αυτού διά της αγάπης,
\par 5 προορίσας ημάς εις υιοθεσίαν διά Ιησού Χριστού εις εαυτόν, κατά την ευδοκίαν του θελήματος αυτού,
\par 6 εις έπαινον της δόξης της χάριτος αυτού, με την οποίαν εχαρίτωσεν ημάς διά του ηγαπημένου αυτού,
\par 7 διά του οποίου έχομεν την απολύτρωσιν διά του αίματος αυτού, την άφεσιν των αμαρτημάτων, κατά τον πλούτον της χάριτος αυτού,
\par 8 καθ' ην επερίσσευσεν εις ημάς εν πάση σοφία και φρονήσει,
\par 9 γνωστοποιήσας εις ημάς το μυστήριον του θελήματος αυτού κατά την ευδοκίαν αυτού, την οποίαν προέθετο εν εαυτώ,
\par 10 εις οικονομίαν του πληρώματος των καιρών, να συγκεφαλαιώση τα πάντα εν τω Χριστώ και τα εν τοις ουρανοίς και τα επί της γης.
\par 11 Εν αυτώ, εις τον οποίον και ελάβομεν κληρονομίαν, προορισθέντες κατά την πρόθεσιν του ενεργούντος τα πάντα κατά την βουλήν του θελήματος αυτού,
\par 12 διά να ήμεθα εις έπαινον της δόξης αυτού ημείς οι προελπίσαντες εις τον Χριστόν·
\par 13 εις τον οποίον και σεις ηλπίσατε, ακούσαντες τον λόγον της αληθείας, το ευαγγέλιον της σωτηρίας σας, εις τον οποίον και πιστεύσαντες εσφραγίσθητε με το Πνεύμα το Άγιον της επαγγελίας,
\par 14 όστις είναι ο αρραβών της κληρονομίας ημών, μέχρι της απολυτρώσεως του αποκτηθέντος λαού αυτού, εις έπαινον της δόξης αυτού.
\par 15 Διά τούτο και εγώ, ακούσας την εις τον Κύριον Ιησούν πίστιν σας και την εις πάντας τους αγίους αγάπην,
\par 16 δεν παύομαι ευχαριστών τον Θεόν υπέρ υμών, μνημονεύων υμάς εν ταις προσευχαίς μου,
\par 17 διά να σας δώση ο Θεός του Κυρίου ημών Ιησού Χριστού, ο Πατήρ της δόξης, πνεύμα σοφίας και αποκαλύψεως εις επίγνωσιν αυτού,
\par 18 ώστε να φωτισθώσιν οι οφθαλμοί του νοός σας, εις το να γνωρίσητε ποία είναι η ελπίς της προσκλήσεως αυτού, και τις ο πλούτος της δόξης της κληρονομίας αυτού εις τους αγίους,
\par 19 και τι το υπερβάλλον μέγεθος της δυνάμεως αυτού προς ημάς τους πιστεύοντας κατά την ενέργειαν του κράτους της ισχύος αυτού,
\par 20 την οποίαν ενήργησεν εν τω Χριστώ, αναστήσας αυτόν εκ νεκρών, και εκάθισεν εκ δεξιών αυτού εν τοις επουρανίοις,
\par 21 υπεράνω πάσης αρχής και εξουσίας και δυνάμεως και κυριότητος και παντός ονόματος ονομαζομένου ου μόνον εν τω αιώνι τούτω, αλλά και εν τω μέλλοντι·
\par 22 και πάντα υπέταξεν υπό τους πόδας αυτού, και έδωκεν αυτόν κεφαλήν υπεράνω πάντων εις την εκκλησίαν,
\par 23 ήτις είναι το σώμα αυτού, το πλήρωμα του τα πάντα εν πάσι πληρούντος.

\chapter{2}

\par 1 Και εσάς όντας νεκρούς διά τας παραβάσεις και τας αμαρτίας εζωοποίησεν,
\par 2 εις τας οποίας περιεπατήσατέ ποτέ κατά το πολίτευμα του κόσμου τούτου, κατά τον άρχοντα της εξουσίας του αέρος, του πνεύματος το οποίον ενεργεί την σήμερον εις τους υιούς της απειθείας·
\par 3 μεταξύ των οποίων και ημείς πάντες ανεστράφημέν ποτέ κατά τας επιθυμίας της σαρκός ημών, πράττοντες τα θελήματα της σαρκός και των διαλογισμών, και ήμεθα εκ φύσεως τέκνα οργής, ως και οι λοιποί·
\par 4 ο Θεός όμως πλούσιος ων εις έλεος, διά την πολλήν αγάπην αυτού με την οποίαν ηγάπησεν ημάς,
\par 5 και ενώ ήμεθα νεκροί διά τα αμαρτήματα, εζωοποίησεν ημάς μετά του Χριστού· κατά χάριν είσθε σεσωσμένοι·
\par 6 και συνανέστησε και συνεκάθισεν εν τοις επουρανίοις διά Ιησού Χριστού,
\par 7 διά να δείξη εις τους επερχομένους αιώνας τον υπερβάλλοντα πλούτον της χάριτος αυτού διά της προς ημάς αγαθότητος εν Χριστώ Ιησού.
\par 8 Διότι κατά χάριν είσθε σεσωσμένοι διά της πίστεως· και τούτο δεν είναι από σας, Θεού το δώρον·
\par 9 ουχί εξ έργων, διά να μη καυχηθή τις.
\par 10 Διότι αυτού ποίημα είμεθα, κτισθέντες εν Χριστώ Ιησού προς έργα καλά, τα οποία προητοίμασεν ο Θεός διά να περιπατήσωμεν εν αυτοίς.
\par 11 Διά τούτο ενθυμείσθε ότι σεις οι ποτέ εθνικοί κατά σάρκα, οι λεγόμενοι ακροβυστία υπό της λεγομένης περιτομής της χειροποιήτου εν τη σαρκί,
\par 12 ότι ήσθε εν τω καιρώ εκείνω χωρίς Χριστού, απηλλοτριωμένοι από της πολιτείας του Ισραήλ και ξένοι των διαθηκών της επαγγελίας, ελπίδα μη έχοντες και όντες εν τω κόσμω χωρίς Θεού.
\par 13 Τώρα όμως διά του Ιησού Χριστού σεις οι ποτέ όντες μακράν εγείνετε πλησίον διά του αίματος του Χριστού.
\par 14 Διότι αυτός είναι η ειρήνη ημών, όστις έκαμε τα δύο εν και έλυσε το μεσότοιχον του φραγμού,
\par 15 καταργήσας την έχθραν εν τη σαρκί αυτού, τον νόμον των εντολών των εν τοις διατάγμασι, διά να κτίση εις εαυτόν τους δύο εις ένα νέον άνθρωπον, φέρων ειρήνην,
\par 16 και να συνδιαλλάξη αμφοτέρους εις εν σώμα προς τον Θεόν διά του σταυρού, θανατώσας δι' αυτού την έχθραν.
\par 17 Και ελθών εκήρυξεν ευαγγέλιον ειρήνης εις εσάς τους μακράν και εις τους πλησίον,
\par 18 διότι δι' αυτού έχομεν αμφότεροι την είσοδον προς τον Πατέρα δι' ενός Πνεύματος.
\par 19 Άρα λοιπόν δεν είσθε πλέον ξένοι και πάροικοι αλλά συμπολίται των αγίων και οικείοι του Θεού,
\par 20 εποικοδομηθέντες επί το θεμέλιον των αποστόλων και προφητών, όντος ακρογωνιαίου λίθου αυτού του Ιησού Χριστού·
\par 21 εν τω οποίω πάσα η οικοδομή συναρμολογουμένη αυξάνεται εις ναόν άγιον εν Κυρίω·
\par 22 εν τω οποίω και σεις συνοικοδομείσθε εις κατοικητήριον του Θεού διά του Πνεύματος.

\chapter{3}

\par 1 Διά τούτο εγώ ο Παύλος, ο δέσμιος του Ιησού Χριστού υπέρ υμών των εθνικών,
\par 2 επειδή ηκούσατε την οικονομίαν της χάριτος του Θεού της δοθείσης εις εμέ υπέρ υμών,
\par 3 ότι δι' αποκαλύψεως εφανέρωσεν εις εμέ το μυστήριον, καθώς προέγραψα συντόμως,
\par 4 εξ ων δύνασθε αναγινώσκοντες να νοήσητε την εν τω μυστηρίω του Χριστού γνώσιν μου,
\par 5 το οποίον εν άλλαις γενεαίς δεν εγνωστοποιήθη εις τους υιούς των ανθρώπων, καθώς τώρα απεκαλύφθη διά Πνεύματος εις τους αγίους αυτού αποστόλους και προφήτας,
\par 6 να ήναι τα έθνη συγκληρονόμα και σύσσωμα και συμμέτοχα της επαγγελίας αυτού εν τω Χριστώ διά του ευαγγελίου,
\par 7 του οποίου έγεινα υπηρέτης κατά την δωρεάν της χάριτος του Θεού, την δοθείσαν εις εμέ κατά την ενέργειαν της δυνάμεως αυτού.
\par 8 Εις εμέ τον πλέον ελάχιστον πάντων των αγίων εδόθη η χάρις αύτη, να ευαγγελίσω μεταξύ των εθνών τον ανεξιχνίαστον πλούτον του Χριστού
\par 9 και να φωτίσω πάντας, ποία είναι η κοινωνία του μυστηρίου του αποκεκρυμμένου από των αιώνων εν τω Θεώ όστις έκτισε τα πάντα διά του Ιησού Χριστού,
\par 10 διά να γνωρισθή τώρα διά της εκκλησίας εν τοις επουρανίοις εις τας αρχάς και τας εξουσίας η πολυποίκιλος σοφία του Θεού,
\par 11 κατά την αιώνιον πρόθεσιν, την οποίαν έκαμεν εν Χριστώ Ιησού τω Κυρίω ημών,
\par 12 διά του οποίου έχομεν την παρρησίαν και την είσοδον με πεποίθησιν διά της εις αυτόν πίστεως.
\par 13 Διά τούτο σας παρακαλώ να μη αθυμήτε διά τας υπέρ υμών θλίψεις μου, το οποίον είναι δόξα υμών.
\par 14 Διά τούτο κάμπτω τα γόνατά μου προς τον Πατέρα του Κυρίου ημών Ιησού Χριστού,
\par 15 εκ του οποίου πάσα πατριά εν ουρανοίς και επί γης ονομάζεται,
\par 16 διά να δώση εις εσάς κατά τον πλούτον της δόξης αυτού, να κραταιωθήτε εν δυνάμει διά του Πνεύματος αυτού εις τον εσωτερικόν άνθρωπον,
\par 17 διά να κατοικήση ο Χριστός διά της πίστεως εν ταις καρδίαις υμών,
\par 18 ώστε να δυνηθήτε, ερριζωμένοι και τεθεμελιωμένοι εν αγάπη, να καταλάβητε μετά πάντων των αγίων τι το πλάτος και μήκος και βάθος και ύψος,
\par 19 και να γνωρίσητε την αγάπην του Χριστού την υπερβαίνουσαν πάσαν γνώσιν, διά να πληρωθήτε με όλον το πλήρωμα του Θεού.
\par 20 Εις δε τον δυνάμενον υπερεκπερισσού να κάμη υπέρ πάντα όσα ζητούμεν ή νοούμεν, κατά την δύναμιν την ενεργουμένην εν ημίν,
\par 21 εις αυτόν έστω η δόξα εν τη εκκλησία διά Ιησού Χριστού εις πάσας τας γενεάς του αιώνος των αιώνων· αμήν.

\chapter{4}

\par 1 Σας παρακαλώ λοιπόν εγώ ο δέσμιος εν Κυρίω να περιπατήσητε αξίως της προσκλήσεως, καθ' ην προσεκλήθητε,
\par 2 μετά πάσης ταπεινοφροσύνης και πραότητος, μετά μακροθυμίας, υποφέροντες αλλήλους εν αγάπη,
\par 3 σπουδάζοντες να διατηρήτε την ενότητα του Πνεύματος διά του συνδέσμου της ειρήνης.
\par 4 Εν σώμα και εν Πνεύμα, καθώς και προσεκλήθητε με μίαν ελπίδα της προσκλήσεώς σας·
\par 5 εις Κύριος, μία πίστις, εν βάπτισμα·
\par 6 εις Θεός και Πατήρ πάντων, ο ων επί πάντων και διά πάντων και εν πάσιν υμίν.
\par 7 Εις ένα δε έκαστον ημών εδόθη η χάρις κατά το μέτρον της δωρεάς του Χριστού.
\par 8 Διά τούτο λέγει· Αναβάς εις ύψος, ηχμαλώτευσεν αιχμαλωσίαν και έδωκε χαρίσματα εις τους ανθρώπους.
\par 9 Το δε ανέβη τι είναι ειμή ότι και κατέβη πρώτον εις τα κατώτερα μέρη της γης;
\par 10 Ο καταβάς αυτός είναι και ο αναβάς υπεράνω πάντων των ουρανών, διά να πληρώση τα πάντα.
\par 11 Και αυτός έδωκεν άλλους μεν αποστόλους, άλλους δε προφήτας, άλλους δε ευαγγελιστάς, άλλους δε ποιμένας και διδασκάλους,
\par 12 προς την τελειοποίησιν των αγίων, διά το έργον της διακονίας, διά την οικοδομήν του σώματος του Χριστού,
\par 13 εωσού καταντήσωμεν πάντες εις την ενότητα της πίστεως και της επιγνώσεως του Υιού του Θεού, εις άνδρα τέλειον, εις μέτρον ηλικίας του πληρώματος του Χριστού,
\par 14 διά να μη ήμεθα πλέον νήπιοι, κυματιζόμενοι και περιφερόμενοι με πάντα άνεμον της διδασκαλίας, διά της δολιότητος των ανθρώπων, διά της πανουργίας εις το μεθοδεύεσθαι την πλάνην,
\par 15 αλλά αληθεύοντες εις την αγάπην να αυξήσωμεν εις αυτόν κατά πάντα, όστις είναι η κεφαλή, ο Χριστός,
\par 16 εξ ου παν το σώμα συναρμολογούμενον και συνδεόμενον διά πάσης συναφείας των συνεργούντων μελών, κατά την ανάλογον ενέργειαν ενός εκάστου μέρους κάμνει την αύξησιν του σώματος προς οικοδομήν εαυτού εν αγάπη.
\par 17 Τούτο λοιπόν λέγω και μαρτύρομαι διά του Κυρίου, να μη περιπατήτε πλέον καθώς και τα λοιπά έθνη περιπατούσιν εν τη ματαιότητι του νοός αυτών,
\par 18 εσκοτισμένοι την διάνοιαν, απηλλοτριωμένοι όντες από της ζωής του Θεού διά την άγνοιαν την ούσαν εν αυτοίς, διά την πώρωσιν της καρδίας αυτών,
\par 19 οίτινες αναισθητούντες, παρέδωκαν εαυτοίς εις την ασέλγειαν, διά να εργάζωνται πάσαν ακαθαρσίαν ακορέστως.
\par 20 Σεις όμως δεν εμάθετε ούτω τον Χριστόν,
\par 21 επειδή αυτόν ηκούσατε και εις αυτόν εδιδάχθητε, καθώς είναι η αλήθεια εν τω Ιησού·
\par 22 να απεκδυθήτε τον παλαιόν άνθρωπον τον κατά την προτέραν διαγωγήν, τον φθειρόμενον κατά τας απατηλάς επιθυμίας,
\par 23 και να ανανεόνησθε εις το πνεύμα του νοός σας
\par 24 και να ενδυθήτε τον νέον άνθρωπον, τον κτισθέντα κατά Θεόν εν δικαιοσύνη και οσιότητι της αληθείας.
\par 25 Όθεν απορρίψαντες το ψεύδος, λαλείτε αλήθειαν έκαστος μετά του πλησίον αυτού· διότι είμεθα μέλη αλλήλων.
\par 26 Οργίζεσθε και μη αμαρτάνετε· ο ήλιος ας μη δύη επί τον παροργισμόν σας,
\par 27 μήτε δίδετε τόπον εις τον διάβολον.
\par 28 Ο κλέπτων ας μη κλέπτη πλέον, μάλλον δε ας κοπιάζη εργαζόμενος το καλόν με τας χείρας αυτού, διά να έχη να μεταδίδη εις τον χρείαν έχοντα.
\par 29 Μηδείς λόγος σαπρός ας μη εξέρχηται εκ του στόματός σας, αλλ' όστις είναι καλός προς οικοδομήν της χρείας, διά να δώση χάριν εις τους ακούοντας.
\par 30 Και μη λυπείτε το Πνεύμα το Άγιον του Θεού, με το οποίον εσφραγίσθητε διά την ημέραν της απολυτρώσεως.
\par 31 Πάσα πικρία και θυμός και οργή και κραυγή και βλασφημία ας αφαιρεθή από σας μετά πάσης κακίας·
\par 32 γίνεσθε δε εις αλλήλους χρηστοί, εύσπλαγχνοι, συγχωρούντες αλλήλους, καθώς ο Θεός συνεχώρησεν εσάς διά του Χριστού.

\chapter{5}

\par 1 Γίνεσθε λοιπόν μιμηταί του Θεού ως τέκνα αγαπητά,
\par 2 και περιπατείτε εν αγάπη, καθώς και ο Χριστός ηγάπησεν ημάς και παρέδωκεν εαυτόν υπέρ ημών προσφοράν και θυσίαν εις τον Θεόν εις οσμήν ευωδίας.
\par 3 Πορνεία δε και πάσα ακαθαρσία ή πλεονεξία μηδέ ας ονομάζηται μεταξύ σας, καθώς πρέπει εις αγίους,
\par 4 μηδέ αισχρότης και μωρολογία ή βωμολοχία, τα οποία είναι απρεπή, αλλά μάλλον ευχαριστία.
\par 5 Διότι τούτο εξεύρετε, ότι πας πόρνος ή ακάθαρτος ή πλεονέκτης, όστις είναι ειδωλολάτρης, δεν έχει κληρονομίαν εν τη βασιλεία του Χριστού και Θεού.
\par 6 Μηδείς ας μη σας απατά με ματαίους λόγους· επειδή διά ταύτα έρχεται η οργή του Θεού επί τους υιούς της απειθείας.
\par 7 Μη γίνεσθε λοιπόν συμμέτοχοι αυτών.
\par 8 Διότι ήσθε ποτέ σκότος, τώρα όμως φως εν Κυρίω· περιπατείτε ως τέκνα φωτός·
\par 9 διότι ο καρπός του Πνεύματος είναι εν πάση αγαθωσύνη και δικαιοσύνη και αληθεία·
\par 10 εξετάζοντες τι είναι ευάρεστον εις τον Κύριον.
\par 11 Και μη συγκοινωνείτε εις τα έργα τα άκαρπα του σκότους, μάλλον δε και ελέγχετε·
\par 12 διότι τα κρυφίως γινόμενα υπ' αυτών αισχρόν έστι και λέγειν·
\par 13 τα δε πάντα ελεγχόμενα υπό του φωτός γίνονται φανερά· επειδή παν το φανερούμενον φως είναι.
\par 14 Διά τούτο λέγει· Σηκώθητι ο κοιμώμενος και ανάστηθι εκ των νεκρών, και θέλει σε φωτίσει ο Χριστός.
\par 15 Προσέχετε λοιπόν πως να περιπατήτε ακριβώς, μη ως άσοφοι, αλλ' ως σοφοί,
\par 16 εξαγοραζόμενοι τον καιρόν, διότι αι ημέραι είναι πονηραί.
\par 17 Διά τούτο μη γίνεσθε άφρονες, αλλά νοείτε τι είναι το θέλημα του Κυρίου.
\par 18 Και μη μεθύσκεσθε με οίνον, εις τον οποίον είναι ασωτία, αλλά πληρούσθε διά του Πνεύματος,
\par 19 λαλούντες μεταξύ σας με ψαλμούς και ύμνους και ωδάς πνευματικάς, άδοντες και ψάλλοντες εν τη καρδία υμών εις τον Κύριον,
\par 20 ευχαριστούντες πάντοτε υπέρ πάντων εις τον Θεόν και Πατέρα εν ονόματι του Κυρίου ημών Ιησού Χριστού,
\par 21 υποτασσόμενοι εις αλλήλους εν φόβω Θεού.
\par 22 Αι γυναίκες, υποτάσσεσθε εις τους άνδρας σας ως εις τον Κύριον,
\par 23 διότι ο ανήρ είναι κεφαλή της γυναικός, καθώς και ο Χριστός κεφαλή της εκκλησίας, και αυτός είναι σωτήρ του σώματος.
\par 24 Αλλά καθώς η εκκλησία υποτάσσεται εις τον Χριστόν, ούτω και αι γυναίκες ας υποτάσσωνται εις τους άνδρας αυτών κατά πάντα.
\par 25 Οι άνδρες, αγαπάτε τας γυναίκάς σας, καθώς και ο Χριστός ηγάπησε την εκκλησίαν και παρέδωκεν εαυτόν υπέρ αυτής,
\par 26 διά να αγιάση αυτήν, καθαρίσας με το λουτρόν του ύδατος διά του λόγου,
\par 27 διά να παραστήση αυτήν εις εαυτόν ένδοξον εκκλησίαν, μη έχουσαν κηλίδα ή ρυτίδα ή τι των τοιούτων, αλλά διά να ήναι αγία και άμωμος.
\par 28 Ούτω χρεωστούσιν οι άνδρες να αγαπώσι τας εαυτών γυναίκας ως τα εαυτών σώματα. Όστις αγαπά την εαυτού γυναίκα εαυτόν αγαπά·
\par 29 διότι ουδείς εμίσησέ ποτέ την εαυτού σάρκα, αλλ' εκτρέφει και περιθάλπει αυτήν, καθώς και ο Κύριος την εκκλησίαν·
\par 30 επειδή μέλη είμεθα του σώματος αυτού, εκ της σαρκός αυτού και εκ των οστέων αυτού.
\par 31 Διά τούτο θέλει αφήσει ο άνθρωπος τον πατέρα αυτού και την μητέρα και θέλει προσκολληθή εις την γυναίκα αυτού, και θέλουσιν είσθαι οι δύο εις σάρκα μίαν.
\par 32 Το μυστήριον τούτο είναι μέγα, εγώ δε λέγω τούτο περί Χριστού και περί της εκκλησίας.
\par 33 Πλην και σεις οι καθ' ένα έκαστος την εαυτού γυναίκα ούτως ας αγαπά ως εαυτόν, η δε γυνή ας σέβηται τον άνδρα.

\chapter{6}

\par 1 Τα τέκνα, υπακούετε εις τους γονείς σας εν Κυρίω· διότι τούτο είναι δίκαιον.
\par 2 Τίμα τον πατέρα σου και την μητέρα, ήτις είναι εντολή πρώτη με επαγγελίαν,
\par 3 διά να γείνη εις σε καλόν και να ήσαι μακροχρόνιος επί της γης.
\par 4 Και οι πατέρες, μη παροργίζετε τα τέκνα σας, αλλ' εκτρέφετε αυτά εν παιδεία Κυρίου.
\par 5 Οι δούλοι, υπακούετε εις τους κατά σάρκα κυρίους σας μετά φόβου και τρόμου εν απλότητι της καρδίας σας ως εις τον Χριστόν,
\par 6 μη κατ' οφθαλμοδουλείαν ως ανθρωπάρεσκοι, αλλ' ως δούλοι του Χριστού, εκπληρούντες το θέλημα του Θεού εκ ψυχής,
\par 7 μετ' ευνοίας δουλεύοντες εις τον Κύριον και ουχί εις ανθρώπους,
\par 8 εξεύροντες ότι έκαστος ό,τι καλόν πράξη, τούτο θέλει λάβει παρά του Κυρίου, είτε δούλος είτε ελεύθερος.
\par 9 Και οι κύριοι, τα αυτά πράττετε προς αυτούς, αφίνοντες την απειλήν, εξεύροντες ότι και σεις αυτοί έχετε Κύριον εν ουρανοίς, και προσωποληψία δεν υπάρχει παρ' αυτώ.
\par 10 Το λοιπόν, αδελφοί μου, ενδυναμούσθε εν Κυρίω και εν τω κράτει της ισχύος αυτού.
\par 11 Ενδύθητε την πανοπλίαν του Θεού, διά να δυνηθήτε να σταθήτε εναντίον εις τας μεθοδείας του διαβόλου·
\par 12 διότι δεν είναι η πάλη ημών εναντίον εις αίμα και σάρκα, αλλ' εναντίον εις τας αρχάς, εναντίον εις τας εξουσίας, εναντίον εις τους κοσμοκράτορας του σκότους του αιώνος τούτου· εναντίον εις τα πνεύματα της πονηρίας εν τοις επουρανίοις.
\par 13 Διά τούτο αναλάβετε την πανοπλίαν του Θεού, διά να δυνηθήτε να αντισταθήτε εν τη ημέρα τη πονηρά και αφού καταπολεμήσητε τα πάντα, να σταθήτε.
\par 14 Σταθήτε λοιπόν περιεζωσμένοι την οσφύν σας με αλήθειαν και ενδεδυμένοι τον θώρακα της δικαιοσύνης
\par 15 και έχοντες υποδεδημένους τους πόδας με την ετοιμασίαν του ευαγγελίου της ειρήνης·
\par 16 επί πάσι δε αναλάβετε την ασπίδα της πίστεως, διά της οποίας θέλετε δυνηθή να σβέσητε πάντα τα βέλη του πονηρού τα πεπυρωμένα·
\par 17 και λάβετε την περικεφαλαίαν της σωτηρίας και την μάχαιραν του Πνεύματος, ήτις είναι ο λόγος του Θεού,
\par 18 προσευχόμενοι εν παντί καιρώ μετά πάσης προσευχής και δεήσεως διά του Πνεύματος, και εις αυτό τούτο αγρυπνούντες με πάσαν προσκαρτέρησιν και δέησιν υπέρ πάντων των αγίων,
\par 19 και υπέρ εμού, διά να δοθή εις εμέ λόγος να ανοίξω το στόμα μου μετά παρρησίας, διά να κάμω γνωστόν το μυστήριον του ευαγγελίου,
\par 20 υπέρ του οποίου είμαι πρέσβυς, φορών άλυσιν, διά να λαλήσω περί αυτού μετά παρρησίας καθώς πρέπει να λαλήσω.
\par 21 Αλλά διά να εξεύρητε και σεις τα κατ' εμέ, τι κάμνω, τα πάντα θέλει σας φανερώσει ο Τυχικός ο αγαπητός αδελφός και πιστός διάκονος εν Κυρίω,
\par 22 τον οποίον έπεμψα προς εσάς δι' αυτό τούτο, διά να μάθητε τα περί ημών και να παρηγορήση τας καρδίας σας.
\par 23 Ειρήνη εις τους αδελφούς και αγάπη μετά πίστεως από Θεού Πατρός και Κυρίου Ιησού Χριστού.
\par 24 Η χάρις είη μετά πάντων των αγαπώντων τον Κύριον ημών Ιησούν Χριστόν εν καθαρότητι· αμήν.


\end{document}