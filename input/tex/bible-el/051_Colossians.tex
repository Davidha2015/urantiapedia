\begin{document}

\title{Colossians}


\chapter{1}

\par Παύλος, απόστολος Ιησού Χριστού διά θελήματος Θεού, και Τιμόθεος ο αδελφός,
\par 2 προς τους αγίους και πιστούς εν Χριστώ αδελφούς τους εν Κολοσσαίς· χάρις είη υμίν και ειρήνη από Θεού Πατρός ημών και Κυρίου Ιησού Χριστού.
\par 3 Ευχαριστούμεν τον Θεόν και Πατέρα του Κυρίου ημών Ιησού Χριστού, προσευχόμενοι πάντοτε υπέρ υμών,
\par 4 ακούσαντες την εις τον Ιησούν Χριστόν πίστιν σας και την εις πάντας τους αγίους αγάπην,
\par 5 διά την ελπίδα την αποτεταμιευμένην διά σας εν τοις ουρανοίς, την οποίαν προηκούσατε εν τω λόγω της αληθείας του ευαγγελίου,
\par 6 το οποίον ήλθεν εις εσάς, καθώς και εις όλον τον κόσμον, και καρποφορεί καθώς και εις εσάς, αφ' ης ημέρας ηκούσατε και εγνωρίσατε την χάριν του Θεού εν αληθεία,
\par 7 καθώς και εμάθετε από Επαφρά του αγαπητού συνδούλου ημών, όστις είναι διά σας, πιστός διάκονος του Χριστού,
\par 8 όστις και εφανέρωσεν εις ημάς την εν Πνεύματι αγάπην σας.
\par 9 Διά τούτο και ημείς, αφ' ης ημέρας ηκούσαμεν, δεν παύομεν προσευχόμενοι διά σας και δεόμενοι να εμπλησθήτε από της επιγνώσεως του θελήματος αυτού μετά πάσης σοφίας και πνευματικής συνέσεως,
\par 10 διά να περιπατήσητε αξίως του Κυρίου, ευαρεστούντες κατά πάντα, καρποφορούντες εις παν έργον αγαθόν και αυξανόμενοι εις την επίγνωσιν του Θεού,
\par 11 ενδυναμούμενοι εν πάση δυνάμει κατά το κράτος της δόξης αυτού εις πάσαν υπομονήν και μακροθυμίαν,
\par 12 μετά χαράς ευχαριστούντες τον Πατέρα, όστις έκαμεν ημάς αξίους της μερίδος του κλήρου των αγίων εν τω φωτί,
\par 13 όστις ηλευθέρωσεν ημάς εκ της εξουσίας του σκότους και μετέφερεν εις την βασιλείαν του αγαπητού αυτού Υιού·
\par 14 εις τον οποίον έχομεν την απολύτρωσιν διά του αίματος αυτού, την άφεσιν των αμαρτιών·
\par 15 όστις είναι εικών του Θεού του αοράτου, πρωτότοκος πάσης κτίσεως,
\par 16 επειδή δι' αυτού εκτίσθησαν τα πάντα, τα εν τοις ουρανοίς και τα επί της γης, τα ορατά και τα αόρατα, είτε θρόνοι είτε κυριότητες είτε αρχαί είτε εξουσίαι· τα πάντα δι' αυτού και εις αυτόν εκτίσθησαν·
\par 17 και αυτός είναι προ πάντων, και τα πάντα συντηρούνται δι' αυτού,
\par 18 και αυτός είναι η κεφαλή του σώματος, της εκκλησίας· όστις είναι αρχή, πρωτότοκος εκ των νεκρών, διά να γείνη αυτός πρωτεύων εις τα πάντα,
\par 19 διότι εν αυτώ ηυδόκησεν ο Πατήρ να κατοικήση παν το πλήρωμα
\par 20 και δι' αυτού να συνδιαλλάξη τα πάντα προς εαυτόν, ειρηνοποιήσας διά του αίματος του σταυρού αυτού, δι' αυτού, είτε τα επί της γης είτε τα εν τοις ουρανοίς.
\par 21 Και σας, οίτινες ήσθε ποτέ απηλλοτριωμένοι και εχθροί κατά την διάνοιαν με τα έργα τα πονηρά,
\par 22 τώρα όμως διήλλαξε προς εαυτόν διά του σώματος της σαρκός αυτού διά του θανάτου, διά να σας παραστήση ενώπιον αυτού αγίους και αμώμους και ανεγκλήτους,
\par 23 εάν επιμένητε εις την πίστιν, τεθεμελιωμένοι και στερεοί και μη μετακινούμενοι από της ελπίδος του ευαγγελίου, το οποίον ηκούσατε, του κηρυχθέντος εις πάσαν την κτίσιν την υπό τον ουρανόν, του οποίου εγώ ο Παύλος έγεινα υπηρέτης.
\par 24 Τώρα χαίρω εις τα παθήματά μου διά σας, και ανταναπληρώ τα υστερήματα των θλίψεων του Χριστού εν τη σαρκί μου υπέρ του σώματος αυτού, το οποίον είναι η εκκλησία,
\par 25 της οποίας εγώ έγεινα υπηρέτης κατά την οικονομίαν του Θεού την εις εμέ δοθείσαν διά σας, διά να εκπληρώσω το κήρυγμα του λόγου του Θεού,
\par 26 το μυστήριον, το οποίον ήτο αποκεκρυμμένον από των αιώνων και από των γενεών, τώρα δε εφανερώθη εις τους αγίους αυτού,
\par 27 εις τους οποίους ηθέλησεν ο Θεός να φανερώση τις ο πλούτος της δόξης του μυστηρίου τούτου εις τα έθνη, όστις είναι ο Χριστός εις εσάς, η ελπίς της δόξης·
\par 28 τον οποίον ημείς κηρύττομεν, νουθετούντες πάντα άνθρωπον και διδάσκοντες πάντα άνθρωπον εν πάση σοφία, διά να παραστήσωμεν πάντα άνθρωπον τέλειον εν Χριστώ Ιησού·
\par 29 εις το οποίον και κοπιάζω, αγωνιζόμενος κατά την ενέργειαν αυτού την ενεργουμένην εν εμοί μετά δυνάμεως.

\chapter{2}

\par Διότι θέλω να εξεύρητε οποίον μέγαν αγώνα έχω διά σας και τους εν Λαοδικεία και τους όσοι δεν είδον το πρόσωπόν μου σωματικώς,
\par 2 διά να παρηγορηθώσιν αι καρδίαι αυτών, ενωθέντων ομού εν αγάπη και εις πάντα πλούτον της πληροφορίας της συνέσεως, ώστε να γνωρίσωσι το μυστήριον του Θεού και Πατρός και του Χριστού,
\par 3 εν τω οποίω είναι κεκρυμμένοι πάντες οι θησαυροί της σοφίας και της γνώσεως.
\par 4 Λέγω δε τούτο, διά να μη σας εξαπατά τις με πιθανολογίαν·
\par 5 διότι αν και κατά το σώμα ήμαι απών, με το πνεύμα όμως είμαι μεθ' υμών, χαίρων και βλέπων την τάξιν σας και την σταθερότητα της εις Χριστόν πίστεώς σας.
\par 6 Καθώς λοιπόν παρελάβετε τον Χριστόν Ιησούν τον Κύριον, εν αυτώ περιπατείτε,
\par 7 ερριζωμένοι και εποικοδομούμενοι εν αυτώ και στερεούμενοι εν τη πίστει καθώς εδιδάχθητε, περισσεύοντες εν αυτή μετά ευχαριστίας.
\par 8 Βλέπετε μη σας εξαπατήση τις διά της φιλοσοφίας και της ματαίας απάτης, κατά την παράδοσιν των ανθρώπων, κατά τα στοιχεία του κόσμου και ουχί κατά Χριστόν·
\par 9 διότι εν αυτώ κατοικεί παν το πλήρωμα της θεότητος σωματικώς,
\par 10 και είσθε πλήρεις εν αυτώ, όστις είναι η κεφαλή πάσης αρχής και εξουσίας,
\par 11 εις τον οποίον και περιετμήθητε με περιτομήν αχειροποίητον, απεκδυθέντες το σώμα των αμαρτιών της σαρκός διά της περιτομής του Χριστού,
\par 12 συνταφέντες μετ' αυτού εν τω βαπτίσματι, διά του οποίου και συνανέστητε διά της πίστεως της ενεργείας του Θεού, όστις ανέστησεν αυτόν εκ των νεκρών.
\par 13 Και εσάς, όντας νεκρούς εις τα αμαρτήματα και την ακροβυστίαν της σαρκός σας, συνεζωοποίησε μετ' αυτού, συγχωρήσας εις εσάς πάντα τα πταίσματα,
\par 14 εξαλείψας το καθ' ημών χειρόγραφον, συνιστάμενον εις διατάγματα, το οποίον ήτο εναντίον εις ημάς, και αφήρεσεν αυτό εκ του μέσου, προσηλώσας αυτό επί του σταυρού·
\par 15 και απογυμνώσας τας αρχάς και τας εξουσίας, παρεδειγμάτισε παρρησία, θριαμβεύσας κατ' αυτών επ' αυτού.
\par 16 Ας μη σας κρίνη λοιπόν μηδείς διά φαγητόν ή διά ποτόν ή διά λόγον εορτής ή νεομηνίας ή σαββάτων,
\par 17 τα οποία είναι σκιά των μελλόντων, το σώμα όμως είναι του Χριστού.
\par 18 Ας μη σας στερήση μηδείς του βραβείου με προσποίησιν ταπεινοφροσύνης και με θρησκείαν των αγγέλων, εμβατεύων εις πράγματα τα οποία δεν είδε, ματαίως φυσιούμενος υπό του νοός της σαρκός αυτού,
\par 19 και μη κρατών την κεφαλήν, τον Χριστόν, εκ του οποίου όλον το σώμα διά των αρμών και συνδέσμων διατηρούμενον και συνδεόμενον αυξάνει κατά την αύξησιν του Θεού.
\par 20 Εάν λοιπόν απεθάνετε μετά του Χριστού από των στοιχείων του κόσμου, διά τι ως ζώντες εν τω κόσμω υπόκεισθε εις διατάγματα,
\par 21 Μη πιάσης, μη γευθής, μη εγγίσης,
\par 22 τα οποία πάντα φθείρονται διά της χρήσεως, κατά τα εντάλματα και τας διδασκαλίας των ανθρώπων;
\par 23 τα οποία έχουσι φαινόμενον μόνον σοφίας εις εθελοθρησκείαν και ταπεινοφροσύνην και σκληραγωγίαν του σώματος, εις ουδεμίαν τιμήν έχοντα την ευχαρίστησιν της σαρκός.

\chapter{3}

\par Εάν λοιπόν συνανέστητε μετά του Χριστού, τα άνω ζητείτε, όπου είναι ο Χριστός καθήμενος εν δεξιά του Θεού,
\par 2 τα άνω φρονείτε, μη τα επί της γης.
\par 3 Διότι απεθάνετε, και η ζωή σας είναι κεκρυμμένη μετά του Χριστού εν τω Θεώ·
\par 4 όταν ο Χριστός, η ζωή ημών, φανερωθή, τότε και σεις μετ' αυτού θέλετε φανερωθή εν δόξη.
\par 5 Νεκρώσατε λοιπόν τα μέλη σας τα επί της γης, πορνείαν, ακαθαρσίαν, πάθος, επιθυμίαν κακήν και την πλεονεξίαν, ήτις είναι ειδωλολατρεία,
\par 6 διά τα οποία έρχεται η οργή του Θεού επί τους υιούς της απειθείας,
\par 7 εις τα οποία και σεις περιεπατήσατέ ποτέ, ότε εζήτε εν αυτοίς·
\par 8 τώρα όμως απορρίψατε και σεις ταύτα πάντα, οργήν, θυμόν, κακίαν, βλασφημίαν, αισχρολογίαν εκ του στόματός σας·
\par 9 μη ψεύδεσθε εις αλλήλους, αφού απεξεδύθητε τον παλαιόν άνθρωπον μετά των πράξεων αυτού
\par 10 και ενεδύθητε τον νέον, τον ανακαινιζόμενον εις επίγνωσιν κατά την εικόνα του κτίσαντος αυτόν,
\par 11 όπου δεν είναι Έλλην και Ιουδαίος, περιτομή και ακροβυστία, βάρβαρος, Σκύθης, δούλος, ελεύθερος, αλλά τα πάντα και εν πάσιν είναι ο Χριστός.
\par 12 Ενδύθητε λοιπόν, ως εκλεκτοί του Θεού άγιοι και ηγαπημένοι, σπλάγχνα οικτιρμών, χρηστότητα, ταπεινοφροσύνην, πραότητα, μακροθυμίαν,
\par 13 υποφέροντες αλλήλους και συγχωρούντες εις αλλήλους, εάν τις έχη παράπονον κατά τινος· καθώς και ο Χριστός συνεχώρησεν εις εσάς, ούτω και σείς·
\par 14 και εν πάσι τούτοις ενδύθητε την αγάπην, ήτις είναι σύνδεσμος της τελειότητος.
\par 15 Και η ειρήνη του Θεού ας βασιλεύη εν ταις καρδίαις υμών, εις την οποίαν και προσεκλήθητε εις εν σώμα· και γίνεσθε ευγνώμονες.
\par 16 Ο λόγος του Χριστού ας κατοική εν υμίν πλουσίως μετά πάσης σοφίας· διδάσκοντες και νουθετούντες αλλήλους με ψαλμούς και ύμνους και ωδάς πνευματικάς, εν χάριτι ψάλλοντες εκ της καρδίας υμών προς τον Κύριον.
\par 17 Και παν ό,τι αν πράττητε εν λόγω ή εν έργω, πάντα εν τω ονόματι του Κυρίου Ιησού πράττετε, ευχαριστούντες δι' αυτού τον Θεόν και Πατέρα.
\par 18 Αι γυναίκες, υποτάσσεσθε εις τους άνδρας σας, καθώς πρέπει εν Κυρίω.
\par 19 Οι άνδρες, αγαπάτε τας γυναίκάς σας και μη ήσθε πικροί προς αυτάς.
\par 20 Τα τέκνα, υπακούετε εις τους γονείς κατά πάντα· διότι τούτο είναι ευάρεστον εις τον Κύριον.
\par 21 Οι πατέρες, μη ερεθίζετε τα τέκνα σας, διά να μη μικροψυχώσιν.
\par 22 Οι δούλοι, υπακούετε κατά πάντα εις τους κατά σάρκα κυρίους σας, ουχί με οφθαλμοδουλείας ως ανθρωπάρεσκοι, αλλά με απλότητα καρδίας, φοβούμενοι τον Θεόν.
\par 23 Και παν ό,τι αν πράττητε, εκ ψυχής εργάζεσθε, ως εις τον Κύριον και ουχί εις ανθρώπους,
\par 24 εξεύροντες ότι από του Κυρίου θέλετε λάβει την ανταπόδοσιν της κληρονομίας· διότι εις τον Κύριον Χριστόν δουλεύετε.
\par 25 Όστις όμως αδικεί, θέλει λάβει την αμοιβήν της αδικίας αυτού, και δεν υπάρχει προσωποληψία.

\chapter{4}

\par Οι κύριοι, αποδίδετε εις τους δούλους σας το δίκαιον και το ίσον, εξεύροντες ότι και σεις έχετε Κύριον εν ουρανοίς.
\par 2 Εμμένετε εις την προσευχήν, αγρυπνούντες εις αυτήν μετά ευχαριστίας,
\par 3 προσευχόμενοι ενταυτώ και περί ημών, να ανοίξη εις ημάς ο Θεός θύραν του λόγου, διά να λαλήσωμεν το μυστήριον του Χριστού, διά το οποίον και είμαι δεδεμένος,
\par 4 διά να φανερώσω αυτό καθώς πρέπει να λαλήσω.
\par 5 Περιπατείτε εν φρονήσει προς τους έξω, εξαγοραζόμενοι τον καιρόν.
\par 6 Ο λόγος σας ας ήναι πάντοτε με χάριν, ηρτυμένος με άλας, διά να εξεύρητε πως πρέπει να αποκρίνησθε προς ένα έκαστον.
\par 7 Τα κατ' εμέ πάντα θέλει σας φανερώσει ο Τυχικός ο αγαπητός αδελφός και πιστός διάκονος και σύνδουλος εν Κυρίω,
\par 8 τον οποίον έπεμψα προς εσάς δι' αυτό τούτο, διά να μάθη την κατάστασίν σας και να παρηγορήση τας καρδίας σας,
\par 9 μετά του Ονησίμου του πιστού και αγαπητού αδελφού, όστις είναι από σάς· θέλουσι σας φανερώσει πάντα τα εδώ.
\par 10 Σας ασπάζεται Αρίσταρχος ο συναιχμάλωτός μου και Μάρκος ο ανεψιός του Βαρνάβα, περί του οποίου ελάβετε παραγγελίας· εάν έλθη προς εσάς, υποδέχθητε αυτόν,
\par 11 και Ιησούς ο λεγόμενος Ιούστος, οίτινες είναι εκ της περιτομής, ούτοι μόνοι είναι συνεργοί μου εις την βασιλείαν του Θεού, οίτινες έγειναν εις εμέ παρηγορία.
\par 12 Σας ασπάζεται ο Επαφράς, όστις είναι από σας, ο δούλος του Χριστού, πάντοτε αγωνιζόμενος διά σας εν ταις προσευχαίς, διά να σταθήτε τέλειοι και πλήρεις εις παν θέλημα του Θεού·
\par 13 διότι μαρτυρώ περί αυτού ότι έχει ζήλον πολύν διά σας και τους εν Λαοδικεία και τους εν Ιεραπόλει.
\par 14 Σας ασπάζεται Λουκάς ο ιατρός ο αγαπητός και ο Δημάς.
\par 15 Ασπάσθητε τους εν Λαοδικεία αδελφούς και τον Νυμφάν και την κατ' οίκον αυτού εκκλησίαν·
\par 16 και αφού αναγνωσθή μεταξύ σας η επιστολή, κάμετε να αναγνωσθή και εν τη εκκλησία των Λαοδικέων, και την εκ Λαοδικείας να αναγνώσητε και σεις.
\par 17 Και είπατε προς τον Αρχιππον· Πρόσεχε εις την διακονίαν, την οποίαν παρέλαβες εν Κυρίω, διά να εκπληροίς αυτήν.
\par 18 Ο ασπασμός εγράφη με την χείρα εμού του Παύλου. Ενθυμείσθε τα δεσμά μου. Η χάρις είη μεθ' υμών· αμήν.


\end{document}