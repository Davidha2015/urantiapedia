\begin{document}

\title{1 Timothy}


\chapter{1}

\par 1 Παύλος, απόστολος Ιησού Χριστού, κατ' επιταγήν Θεού του Σωτήρος ημών και Κυρίου Ιησού Χριστού της ελπίδος ημών,
\par 2 προς Τιμόθεον, το γνήσιον τέκνον εις την πίστιν· είη χάρις, έλεος, ειρήνη από Θεού Πατρός ημών και Χριστού Ιησού του Κυρίου ημών.
\par 3 Καθώς σε παρεκάλεσα απερχόμενος εις Μακεδονίαν, να προσμείνης εν Εφέσω, διά να παραγγείλης εις τινάς να μη ετεροδιδασκαλώσι
\par 4 μηδέ να προσέχωσιν εις μύθους και γενεαλογίας απεράντους, αίτινες προξενούσι φιλονεικίας μάλλον παρά την εις την πίστιν οικοδομήν του Θεού, ούτω πράττε·
\par 5 το δε τέλος της παραγγελίας είναι αγάπη εκ καθαράς καρδίας και συνειδήσεως αγαθής και πίστεως ανυποκρίτου,
\par 6 από των οποίων αποπλανηθέντες τινές εξετράπησαν εις ματαιολογίαν.
\par 7 Θέλοντες να ήναι νομοδιδάσκαλοι, ενώ δεν νοούσιν ούτε όσα λέγουσιν ούτε περί τίνων διϊσχυρίζονται.
\par 8 Εξεύρομεν δε ότι ο νόμος είναι καλός, εάν τις μεταχειρίζηται αυτόν νομίμως,
\par 9 γνωρίζων τούτο, ότι ο νόμος δεν ετέθη διά τον δίκαιον, αλλά διά τους ανόμους και ανυποτάκτους, τους ασεβείς και αμαρτωλούς, τους ανοσίους και βεβήλους, τους πατροκτόνους και μητροκτόνους, τους ανδροφόνους,
\par 10 πόρνους, αρσενοκοίτας, ανδραποδιστάς, ψεύστας, επιόρκους, και ει τι άλλο αντιβαίνει εις την υγιαίνουσαν διδασκαλίαν,
\par 11 κατά το ευαγγέλιον της δόξης του μακαρίου Θεού, το οποίον εγώ ενεπιστεύθην.
\par 12 Και ευχαριστώ τον ενδυναμώσαντά με Ιησούν Χριστόν τον Κύριον ημών, ότι ενέκρινε πιστόν και έταξεν εις την διακονίαν εμέ,
\par 13 τον πρότερον όντα βλάσφημον και διώκτην και υβριστήν· ηλεήθην όμως, διότι αγνοών έπραξα εν απιστία,
\par 14 αλλ' υπερεπερίσσευσεν η χάρις του Κυρίου ημών μετά πίστεως και αγάπης της εν Χριστώ Ιησού.
\par 15 Πιστός ο λόγος και πάσης αποδοχής άξιος, ότι ο Ιησούς Χριστός ήλθεν εις τον κόσμον διά να σώση τους αμαρτωλούς, των οποίων πρώτος είμαι εγώ·
\par 16 αλλά διά τούτο ηλεήθην, διά να δείξη ο Ιησούς Χριστός εις εμέ πρώτον την πάσαν μακροθυμίαν, εις παράδειγμα των μελλόντων να πιστεύωσιν εις αυτόν εις ζωήν αιώνιον.
\par 17 εις δε τον βασιλέα των αιώνων, τον άφθαρτον, τον αόρατον, τον μόνον σοφόν Θεόν, είη τιμή και δόξα εις τους αιώνας των αιώνων· αμήν.
\par 18 Ταύτην την παραγγελίαν παραδίδω εις σε, τέκνον Τιμόθεε, κατά τας προγενομένας προφητείας περί σου, να στρατεύης κατ' αυτάς την καλήν στρατείαν,
\par 19 έχων πίστιν και αγαθήν συνείδησιν, την οποίαν τινές αποβαλόντες εναυάγησαν εις την πίστιν·
\par 20 εκ των οποίων είναι ο Υμέναιος και Αλέξανδρος, τους οποίους παρέδωκα εις τον Σατανάν, διά να μάθωσι να μη βλασφημώσι.

\chapter{2}

\par 1 Παρακαλώ λοιπόν πρώτον πάντων να κάμνητε δεήσεις, προσευχάς, παρακλήσεις, ευχαριστίας υπέρ πάντων ανθρώπων,
\par 2 υπέρ βασιλέων και πάντων των όντων εν αξιώμασι, διά να διάγωμεν βίον ατάραχον και ησύχιον εν πάση ευσεβεία και σεμνότητι.
\par 3 Διότι τούτο είναι καλόν και ευπρόσδεκτον ενώπιον του σωτήρος ημών Θεού,
\par 4 όστις θέλει να σωθώσι πάντες οι άνθρωποι και να έλθωσιν εις επίγνωσιν της αληθείας.
\par 5 Διότι είναι εις Θεός, εις και μεσίτης Θεού και ανθρώπων, άνθρωπος Ιησούς Χριστός,
\par 6 όστις έδωκεν εαυτόν αντίλυτρον υπέρ πάντων, μαρτυρίαν γενομένην εν ωρισμένοις καιροίς,
\par 7 εις το οποίον ετάχθην εγώ κήρυξ και απόστολος, αλήθειαν λέγω εν Χριστώ, δεν ψεύδομαι, διδάσκαλος των εθνών εις την πίστιν και εις την αλήθειαν.
\par 8 Θέλω λοιπόν να προσεύχωνται οι άνδρες εν παντί τόπω, υψόνοντες καθαράς χείρας χωρίς οργής και δισταγμού.
\par 9 Ωσαύτως και αι γυναίκες με στολήν σεμνήν, με αιδώ και σωφροσύνην να στολίζωσιν εαυτάς, ουχί με πλέγματα ή χρυσόν ή μαργαρίτας ή ενδυμασίαν πολυτελή,
\par 10 αλλά το οποίον πρέπει εις γυναίκας επαγγελλομένας θεοσέβειαν, με έργα αγαθά.
\par 11 Η γυνή ας μανθάνη εν ησυχία μετά πάσης υποταγής·
\par 12 εις γυναίκα όμως δεν συγχωρώ να διδάσκη, μηδέ να αυθεντεύη επί του ανδρός, αλλά να ησυχάζη.
\par 13 Διότι ο Αδάμ πρώτος επλάσθη, έπειτα η Εύα·
\par 14 και ο Αδάμ δεν ηπατήθη, αλλ' η γυνή απατηθείσα έγεινε παραβάτις·
\par 15 θέλει όμως σωθή διά της τεκνογονίας, εάν μείνωσιν εις την πίστιν και αγάπην και αγιασμόν μετά σωφροσύνης.

\chapter{3}

\par 1 Πιστός ο λόγος· Εάν τις ορέγηται επισκοπήν, καλόν έργον επιθυμεί.
\par 2 Πρέπει λοιπόν ο επίσκοπος να ήναι άμεμπτος, μιας γυναικός ανήρ, άγρυπνος, σώφρων, κόσμιος, φιλόξενος, διδακτικός,
\par 3 ουχί μέθυσος, ουχί πλήκτης, ουχί αισχροκερδής, αλλ' επιεικής, άμαχος, αφιλάργυρος,
\par 4 κυβερνών καλώς τον εαυτού οίκον, έχων τα τέκνα αυτού εις υποταγήν μετά πάσης σεμνότητος·
\par 5 διότι εάν τις δεν εξεύρη να κυβερνά τον εαυτού οίκον, πως θέλει επιμεληθή την εκκλησίαν του Θεού;
\par 6 να μη ήναι νεοκατήχητος, διά να μη υπερηφανευθή και πέση εις την καταδίκην του διαβόλου.
\par 7 Πρέπει δε αυτός να έχη και παρά των έξωθεν μαρτυρίαν καλήν, διά να μη πέση εις ονειδισμόν και παγίδα του διαβόλου.
\par 8 Οι διάκονοι ωσαύτως πρέπει να ήναι σεμνοί, ουχί δίγλωσσοι, ουχί δεδομένοι εις οίνον πολύν, ουχί αισχροκερδείς,
\par 9 έχοντες το μυστήριον της πίστεως μετά καθαράς συνειδήσεως.
\par 10 Και ούτοι δε ας δοκιμάζωνται πρώτον, έπειτα ας γίνωνται διάκονοι, εάν ήναι άμεμπτοι.
\par 11 Αι γυναίκες ωσαύτως σεμναί, ουχί κατάλαλοι, εγκρατείς, πισταί κατά πάντα.
\par 12 Οι διάκονοι ας ήναι μιας γυναικός άνδρες, κυβερνώντες καλώς τα τέκνα αυτών και τους οίκους αυτών.
\par 13 Διότι οι καλώς διακονήσαντες αποκτώσιν εις εαυτούς βαθμόν καλόν και πολλήν παρρησίαν εις την πίστιν την εις τον Ιησούν Χριστόν.
\par 14 Ταύτα σοι γράφω, ελπίζων να έλθω προς σε ταχύτερον·
\par 15 αλλ' εάν βραδύνω, διά να εξεύρης πως πρέπει να πολιτεύησαι εν τω οίκω του Θεού, όστις είναι η εκκλησία του Θεού του ζώντος, ο στύλος και το εδραίωμα της αληθείας.
\par 16 Και αναντιρρήτως το μυστήριον της ευσεβείας είναι μέγα· ο Θεός εφανερώθη εν σαρκί, εδικαιώθη εν πνεύματι, εφάνη εις αγγέλους, εκηρύχθη εις τα έθνη, επιστεύθη εις τον κόσμον, ανελήφθη εν δόξη.

\chapter{4}

\par 1 Το δε Πνεύμα ρητώς λέγει ότι εν υστέροις καιροίς θέλουσιν αποστατήσει τινές από της πίστεως, προσέχοντες εις πνεύματα πλάνης και εις διδασκαλίας δαιμονίων,
\par 2 διά της υποκρίσεως ψευδολόγων, εχόντων την εαυτών συνείδησιν κεκαυτηριασμένην,
\par 3 εμποδιζόντων τον γάμον, προσταζόντων αποχήν βρωμάτων, τα οποία ο Θεός έκτισε διά να μεταλαμβάνωσι μετά ευχαριστίας οι πιστοί και οι γνωρίσαντες την αλήθειαν.
\par 4 Διότι παν κτίσμα Θεού είναι καλόν, και ουδέν απορρίψιμον, όταν λαμβάνηται μετά ευχαριστίας·
\par 5 διότι αγιάζεται διά του λόγου του Θεού και διά της προσευχής.
\par 6 Ταύτα συμβουλεύων εις τους αδελφούς, θέλεις είσθαι καλός διάκονος του Ιησού Χριστού, εντρεφόμενος εν τοις λόγοις της πίστεως και της καλής διδασκαλίας, την οποίαν παρηκολούθησας.
\par 7 Τους δε βεβήλους και γραώδεις μύθους παραιτού και γύμναζε σεαυτόν εις την ευσέβειαν·
\par 8 διότι η σωματική γυμνασία είναι προς ολίγον ωφέλιμος αλλ' η ευσέβεια είναι προς πάντα ωφέλιμος, έχουσα επαγγελίαν της παρούσης ζωής και της μελλούσης.
\par 9 Πιστός ο λόγος και πάσης αποδοχής άξιος·
\par 10 επειδή διά τούτο και κοπιάζομεν και ονειδιζόμεθα, διότι ελπίζομεν εις τον ζώντα Θεόν, όστις είναι ο Σωτήρ πάντων ανθρώπων, μάλιστα των πιστών.
\par 11 Παράγγελλε ταύτα και δίδασκε.
\par 12 Μηδείς ας μη καταφρονή την νεότητά σου, αλλά γίνου τύπος των πιστών εις λόγον, εις συναναστροφήν, εις αγάπην, εις πνεύμα, εις πίστιν, εις καθαρότητα.
\par 13 Έως να έλθω, καταγίνου εις την ανάγνωσιν, εις την προτροπήν, εις την διδασκαλίαν·
\par 14 μη αμέλει το χάρισμα, το οποίον είναι εν σοι, το οποίον εδόθη εις σε διά προφητείας μετά επιθέσεως των χειρών του πρεσβυτερίου.
\par 15 Ταύτα μελέτα, εις ταύτα μένε, διά να ήναι φανερά εις πάντας η προκοπή σου.
\par 16 Πρόσεχε εις σεαυτόν και εις την διδασκαλίαν, επίμενε εις αυτά· διότι τούτο πράττων και σεαυτόν θέλεις σώσει και τους ακούοντάς σε.

\chapter{5}

\par 1 Πρεσβύτερον μη επιπλήξης, αλλά πρότρεπε ως πατέρα, τους νεωτέρους ως αδελφούς,
\par 2 τας πρεσβυτέρας ως μητέρας, τας νεωτέρας ως αδελφάς μετά πάσης καθαρότητος.
\par 3 Τας χήρας τίμα τας αληθώς χήρας.
\par 4 Εάν δε τις χήρα έχη τέκνα ή έκγονα, ας μανθάνωσι πρώτον να καθιστώσιν ευσεβή τον ίδιον αυτών οίκον και να αποδίδωσιν αμοιβάς εις τους προγόνους αυτών. Διότι τούτο είναι καλόν και ευπρόσδεκτον ενώπιον του Θεού.
\par 5 Η δε αληθώς χήρα και μεμονωμένη ελπίζει επί τον Θεόν και εμμένει εις τας δεήσεις και τας προσευχάς νύκτα και ημέραν·
\par 6 η δεδομένη όμως εις τας ηδονάς ενώ ζη είναι νεκρά.
\par 7 Και ταύτα παράγγελλε, διά να ήναι άμεμπτοι.
\par 8 Αλλ' εάν τις δεν προνοή περί των εαυτού και μάλιστα των οικείων, ηρνήθη την πίστιν και είναι απίστου χειρότερος.
\par 9 Ας καταγράφηται χήρα ουχί ολιγώτερον των εξήκοντα ετών, ήτις υπήρξεν ενός ανδρός γυνή,
\par 10 ήτις μαρτυρείται διά τα καλά αυτής έργα, εάν ανέθρεψε τέκνα, εάν περιέθαλψε ξένους, εάν πόδας αγίων ένιψεν, εάν θλιβομένους εβοήθησεν, εάν επηκολούθησεν εις παν έργον αγαθόν.
\par 11 Τας δε νεωτέρας χήρας απόβαλλε· διότι αφού εντρυφήσωσι κατά του Χριστού, θέλουσι να υπανδρεύωνται,
\par 12 έχουσαι την καταδίκην, διότι ηθέτησαν την πρώτην πίστιν·
\par 13 και ενταυτώ μανθάνουσι να ήναι αργαί, περιερχόμεναι τας οικίας, και ουχί μόνον αργαί, αλλά και φλύαροι και περίεργοι, λαλούσαι τα μη πρέποντα.
\par 14 Θέλω λοιπόν αι νεώτεραι να υπανδρεύωνται, να τεκνοποιώσι, να κυβερνώσιν οίκον, να μη δίδωσι μηδεμίαν αφορμήν εις τον εναντίον να λοιδορή.
\par 15 Διότι εξετράπησαν ήδη τινές οπίσω του Σατανά.
\par 16 Εάν τις πιστός ή πιστή έχη χήρας, ας προμηθεύη εις αυτάς τα αναγκαία, και ας μη επιβαρύνηται η εκκλησία, διά να δύναται να βοηθή τας αληθώς χήρας.
\par 17 Οι καλώς προϊστάμενοι πρεσβύτεροι ας αξιόνωνται διπλής τιμής, μάλιστα όσοι κοπιάζουσιν εις λόγον και διδασκαλίαν·
\par 18 διότι λέγει η γραφή· Δεν θέλεις εμφράξει το στόμα βοός αλωνίζοντος· και, Άξιος είναι ο εργάτης του μισθού αυτού.
\par 19 Κατηγορίαν εναντίον πρεσβυτέρου μη παραδέχου, εκτός διά στόματος δύο ή τριών μαρτύρων.
\par 20 Τους αμαρτάνοντας έλεγχε ενώπιον πάντων, διά να έχωσι φόβον και οι λοιποί.
\par 21 Διαμαρτύρομαι ενώπιον του Θεού και του Κυρίου Ιησού Χριστού και των εκλεκτών αγγέλων, να φυλάξης ταύτα, χωρίς προτιμήσεως, μηδέν πράττων κατά χάριν.
\par 22 Μη επίθετε χείρας ταχέως εις μηδένα, μηδέ γίνου κοινωνός αλλοτρίων αμαρτιών· φύλαττε σεαυτόν καθαρόν.
\par 23 Μη υδροπότει πλέον, αλλά μεταχειρίζου ολίγον οίνον διά τον στόμαχόν σου και τας συχνάς σου ασθενείας.
\par 24 Τινών ανθρώπων αι αμαρτίαι είναι φανεραί, και προπορεύονται αυτών εις την κρίσιν, εις τινάς δε και επακολουθούσιν·
\par 25 ωσαύτως και τα καλά έργα τινών είναι φανερά, και όσα είναι κατ' άλλον τρόπον δεν δύνανται να κρυφθώσι.

\chapter{6}

\par 1 Όσοι είναι υπό ζυγόν δουλείας, ας νομίζωσι τους κυρίους αυτών αξίους πάσης τιμής, διά να μη βλασφημήται το όνομα του Θεού και η διδασκαλία.
\par 2 Οι δε έχοντες πιστούς κυρίους ας μη καταφρονώσιν αυτούς, διότι είναι αδελφοί, αλλά προθυμότερον ας δουλεύωσι, διότι είναι πιστοί και αγαπητοί οι απολαμβάνοντες την ευεργεσίαν. Ταύτα δίδασκε και νουθέτει.
\par 3 Εάν τις ετεροδιδασκαλή και δεν ακολουθή τους υγιαίνοντας λόγους του Κυρίου ημών Ιησού Χριστού και την διδασκαλίαν την κατ' ευσέβειαν,
\par 4 είναι τετυφωμένος και δεν εξεύρει ουδέν, αλλά νοσεί περί συζητήσεις και λογομαχίας, εκ των οποίων προέρχεται φθόνος, έρις, βλασφημίαι, υπόνοιαι πονηραί,
\par 5 μάταιαι συνδιαλέξεις ανθρώπων διεφθαρμένων τον νούν και απεστερημένων της αληθείας, νομιζόντων την ευσέβειαν ότι είναι πλουτισμός. Απομακρύνου από των τοιούτων.
\par 6 Μέγας δε πλουτισμός είναι η ευσέβεια μετά αυταρκείας.
\par 7 Διότι δεν εφέραμεν ουδέν εις τον κόσμον, φανερόν ότι ουδέ δυνάμεθα να εκφέρωμέν τι·
\par 8 έχοντες δε διατροφάς και σκεπάσματα, ας αρκώμεθα εις ταύτα.
\par 9 Όσοι δε θέλουσι να πλουτώσι πίπτουσιν εις πειρασμόν και παγίδα και εις επιθυμίας πολλάς ανοήτους και βλαβεράς, αίτινες βυθίζουσι τους ανθρώπους εις όλεθρον και απώλειαν.
\par 10 Διότι ρίζα πάντων των κακών είναι η φιλαργυρία, την οποίαν τινές ορεγόμενοι απεπλανήθησαν από της πίστεως και διεπέρασαν εαυτούς με οδύνας πολλάς.
\par 11 Συ όμως, ω άνθρωπε του Θεού, ταύτα φεύγε· ζήτει δε δικαιοσύνην, ευσέβειαν, πίστιν, αγάπην, υπομονήν, πραότητα.
\par 12 Αγωνίζου τον καλόν αγώνα της πίστεως· κράτει την αιώνιον ζωήν, εις την οποίαν και προσεκλήθης και ώμολόγησας την καλήν ομολογίαν ενώπιον πολλών μαρτύρων.
\par 13 Σε παραγγέλλω ενώπιον του Θεού του ζωοποιούντος τα πάντα και του Ιησού Χριστού του μαρτυρήσαντος ενώπιον του Ποντίου Πιλάτου την καλήν ομολογίαν,
\par 14 να φυλάξης την εντολήν αμόλυντον, άμεμπτον, μέχρι της επιφανείας του Κυρίου ημών Ιησού Χριστού,
\par 15 την οποίαν εν τοις ωρισμένοις καιροίς θέλει δείξει ο μακάριος και μόνος Δεσπότης, ο Βασιλεύς των βασιλευόντων, και Κύριος των κυριευόντων,
\par 16 όστις μόνος έχει την αθανασίαν, κατοικών φως απρόσιτον, τον οποίον ουδείς των ανθρώπων είδεν ουδέ δύναται να ίδη· εις τον οποίον έστω τιμή και κράτος αιώνιον· αμήν.
\par 17 Εις τους πλουσίους του κόσμου τούτου παράγγελλε να μη υψηλοφρονώσι, μηδέ να ελπίζωσιν επί την αδηλότητα του πλούτου, αλλ' επί τον Θεόν τον ζώντα, όστις δίδει εις ημάς πλουσίως πάντα εις απόλαυσιν,
\par 18 να αγαθοεργώσι, να πλουτώσιν εις έργα καλά, να ήναι ευμετάδοτοι, κοινωνικοί,
\par 19 θησαυρίζοντες εις εαυτούς θεμέλιον καλόν εις το μέλλον, διά να απολαύσωσι την αιώνιον ζωήν.
\par 20 Ω Τιμόθεε, την παρακαταθήκην φύλαξον, αποστρεφόμενος τας βεβήλους ματαιολογίας και τας αντιλογίας της ψευδωνύμου γνώσεως,
\par 21 την οποίαν τινές επαγγελλόμενοι επλανήθησαν κατά την πίστιν. Η χάρις είη μετά σού· αμήν.


\end{document}