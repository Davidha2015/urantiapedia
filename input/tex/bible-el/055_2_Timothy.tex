\begin{document}

\title{2 Τιμόθεο}


\chapter{1}

\par 1 Παύλος, απόστολος Ιησού Χριστού διά θελήματος Θεού κατά την επαγγελίαν της ζωής της εν Χριστώ Ιησού,
\par 2 προς Τιμόθεον το αγαπητόν τέκνον· είη χάρις, έλεος, ειρήνη από Θεού Πατρός και Χριστού Ιησού του Κυρίου ημών.
\par 3 Ευχαριστώ τον Θεόν, τον οποίον λατρεύω από προγόνων μετά καθαράς συνειδήσεως, ότι αδιαλείπτως σε ενθυμούμαι εν ταις δεήσεσί μου νύκτα και ημέραν,
\par 4 επιποθών να σε ίδω, ενθυμούμενος τα δάκρυά σου, διά να εμπλησθώ χαράς,
\par 5 ανακαλών εις την μνήμην μου την εν σοι ανυπόκριτον πίστιν, ήτις πρώτον κατώκησεν εν τη μάμμη σου Λωΐδι και εν τη μητρί σου Ευνίκη, είμαι δε πεπεισμένος ότι και εν σοι.
\par 6 Διά την οποίαν αιτίαν σε υπενθυμίζω να αναζωπυρής το χάρισμα του Θεού, το οποίον είναι εν σοι διά της επιθέσεως των χειρών μου·
\par 7 διότι δεν έδωκεν εις ημάς ο Θεός πνεύμα δειλίας, αλλά δυνάμεως και αγάπης και σωφρονισμού.
\par 8 Μη αισχυνθής λοιπόν την μαρτυρίαν του Κυρίου ημών μηδέ εμέ τον δέσμιον αυτού, αλλά συγκακοπάθησον μετά του ευαγγελίου με την δύναμιν του Θεού,
\par 9 όστις έσωσεν ημάς και εκάλεσε με κλήσιν αγίαν, ουχί κατά τα έργα ημών, αλλά κατά την εαυτού πρόθεσιν και χάριν, την δοθείσαν εις ημάς εν Χριστώ Ιησού προ χρόνων αιωνίων,
\par 10 φανερωθείσαν δε τώρα διά της επιφανείας του Σωτήρος ημών Ιησού Χριστού, όστις κατήργησε μεν τον θάνατον, έφερε δε εις φως την ζωήν και την αφθαρσίαν διά του ευαγγελίου,
\par 11 εις το οποίον ετάχθην εγώ κήρυξ και απόστολος και διδάσκαλος των εθνών.
\par 12 Διά την οποίαν αιτίαν και πάσχω ταύτα, πλην δεν επαισχύνομαι· διότι εξεύρω εις τίνα επίστευσα, και είμαι πεπεισμένος ότι είναι δυνατός να φυλάξη την παρακαταθήκην μου μέχρις εκείνης της ημέρας.
\par 13 Κράτει το υπόδειγμα των υγιαινόντων λόγων, τους οποίους ήκουσας παρ' εμού, μετά πίστεως και αγάπης της εν Χριστώ Ιησού.
\par 14 Την καλήν παρακαταθήκην φύλαξον διά του Πνεύματος του Αγίου του ενοικούντος εν ημίν.
\par 15 Εξεύρεις τούτο, ότι με απεστράφησαν πάντες οι εν τη Ασία, εκ των οποίων είναι ο Φύγελλος και ο Ερμογένης.
\par 16 Είθε ο Κύριος να δώση έλεος εις τον οίκον του Ονησιφόρου, διότι πολλάκις με παρηγόρησε και δεν επησχύνθη την άλυσίν μου,
\par 17 αλλ' ότε ήλθεν εις την Ρώμην, με εζήτησε μετά σπουδής πολλής και με εύρεν·
\par 18 είθε ο Κύριος να δώση εις αυτόν να εύρη έλεος παρά Κυρίου εν εκείνη τη ημέρα· και όσας διακονίας έκαμεν εν Εφέσω, συ εξεύρεις καλήτερα.

\chapter{2}

\par 1 Συ λοιπόν, τέκνον μου, ενδυναμού διά της χάριτος της εν Χριστώ Ιησού,
\par 2 και όσα ήκουσας παρ' εμού διά πολλών μαρτύρων, ταύτα παράδος εις πιστούς ανθρώπους, οίτινες θέλουσιν είσθαι ικανοί και άλλους να διδάξωσι.
\par 3 Συ λοιπόν κακοπάθησον ως καλός στρατιώτης Ιησού Χριστού.
\par 4 Ουδείς στρατευόμενος εμπλέκεται εις τας βιωτικάς υποθέσεις, διά να αρέση εις τον στρατολογήσαντα.
\par 5 Εάν δε και αγωνίζηταί τις, δεν στεφανούται, εάν νομίμως δεν αγωνισθή.
\par 6 Ο κοπιάζων γεωργός πρέπει πρώτος να μεταλαμβάνη από των καρπών.
\par 7 Εννόει εκείνα τα οποία λέγω· είθε δε να σοι δώση ο Κύριος σύνεσιν εις πάντα.
\par 8 Ενθυμού τον εκ σπέρματος Δαβίδ Ιησούν Χριστόν, τον αναστάντα εκ νεκρών, κατά το ευαγγέλιόν μου.
\par 9 Διά το οποίον κακοπαθώ μέχρι δεσμών ως κακούργος· αλλ' ο λόγος του Θεού δεν δεσμεύεται.
\par 10 Διά τούτο πάντα υπομένω διά τους εκλεκτούς, διά να απολαύσωσι και αυτοί την σωτηρίαν την εν Χριστώ Ιησού μετά δόξης αιωνίου.
\par 11 Πιστός ο λόγος· διότι εάν συναπεθάνομεν, θέλομεν και συζήσει·
\par 12 εάν υπομένωμεν, θέλομεν και συμβασιλεύσει· εάν αρνώμεθα αυτόν, και εκείνος θέλει αρνηθή ημάς·
\par 13 εάν απιστώμεν, εκείνος μένει πιστός· να αρνηθή εαυτόν δεν δύναται.
\par 14 Ταύτα υπενθύμιζε, διαμαρτυρόμενος ενώπιον του Κυρίου να μη λογομαχώσι, το οποίον δεν είναι εις ουδέν χρήσιμον, αλλά φέρει καταστροφήν των ακουόντων.
\par 15 Σπούδασον να παραστήσης σεαυτόν δόκιμον εις τον Θεόν, εργάτην ανεπαίσχυντον, ορθοτομούντα τον λόγον της αληθείας.
\par 16 Τας δε βεβήλους ματαιοφωνίας φεύγε· διότι θέλουσι προχωρήσει εις πλειοτέραν ασέβειαν,
\par 17 και ο λόγος αυτών θέλει κατατρώγει ως γάγγραινα· εκ των οποίων είναι ο Υμέναιος και ο Φιλητός,
\par 18 οίτινες απεπλανήθησαν από της αληθείας, λέγοντες ότι έγεινεν ήδη η ανάστασις, και ανατρέπουσι την πίστιν τινών.
\par 19 Το στερεόν όμως θεμέλιον του Θεού μένει, έχον την σφραγίδα ταύτην· Γνωρίζει ο Κύριος τους όντας αυτού, καί· Ας απομακρυνθή από της αδικίας πας όστις ονομάζει το όνομα του Κυρίου.
\par 20 Εν μεγάλη δε οικία δεν είναι μόνον σκεύη χρυσά και αργυρά, αλλά και ξύλινα και οστράκινα, και άλλα μεν προς χρήσιν τιμίαν, άλλα δε προς άτιμον.
\par 21 Εάν λοιπόν καθαρίση τις εαυτόν από τούτων, θέλει είσθαι σκεύος τιμίας χρήσεως, ηγιασμένον και εύχρηστον εις τον δεσπότην, ητοιμασμένον εις παν έργον αγαθόν.
\par 22 Τας δε νεανικάς επιθυμίας φεύγε και ζήτει την δικαιοσύνην, την πίστιν, την αγάπην, την ειρήνην μετά των επικαλουμένων τον Κύριον εκ καθαράς καρδίας.
\par 23 Τας δε μωράς και απαιδεύτους φιλονεικίας παραιτού, εξεύρων ότι γεννώσι μάχας·
\par 24 ο δε δούλος του Κυρίου δεν πρέπει να μάχηται, αλλά να ήναι πράος προς πάντας, διδακτικός, ανεξίκακος,
\par 25 διδάσκων μετά πραότητος τους αντιφρονούντας, μήποτε δώση εις αυτούς ο Θεός μετάνοιαν, ώστε να γνωρίσωσι την αλήθειαν,
\par 26 και να ανανήψωσιν από της παγίδος του διαβόλου, υπό του οποίου είναι πεπαγιδευμένοι εις το θέλημα εκείνου.

\chapter{3}

\par 1 Γίνωσκε δε τούτο, ότι εν ταις εσχάταις ημέραις θέλουσιν ελθεί καιροί κακοί·
\par 2 διότι θέλουσιν είσθαι οι άνθρωποι φίλαυτοι, φιλάργυροι, αλαζόνες, υπερήφανοι, βλάσφημοι, απειθείς εις τους γονείς, αχάριστοι, ανόσιοι,
\par 3 άσπλαγχνοι, αδιάλλακτοι, συκοφάνται, ακρατείς, ανήμεροι, αφιλάγαθοι,
\par 4 προδόται, προπετείς, τετυφωμένοι, φιλήδονοι μάλλον παρά φιλόθεοι,
\par 5 έχοντες μεν μορφήν ευσεβείας, ηρνημένοι δε την δύναμιν αυτής. Και τούτους φεύγε.
\par 6 Διότι εκ τούτων είναι εκείνοι, οίτινες εισχωρούσιν εις τας οικίας και αιχμαλωτίζουσι τα γυναικάρια τα πεφορτισμένα αμαρτίας, συρόμενα υπό διαφόρων επιθυμιών,
\par 7 τα οποία πάντοτε μανθάνουσι και ποτέ δεν δύνανται να έλθωσιν εις την γνώσιν της αληθείας.
\par 8 Και καθ' ον τρόπον ο Ιαννής και Ιαμβρής αντέστησαν εις τον Μωϋσήν, ούτω και αυτοί ανθίστανται εις την αλήθειαν, άνθρωποι διεφθαρμένοι τον νούν, αδόκιμοι εις την πίστιν.
\par 9 Αλλά δεν θέλουσι προκόψει πλειότερον· διότι η ανοησία αυτών θέλει γείνει κατάδηλος εις πάντας, καθώς και η εκείνων έγεινε.
\par 10 Συ όμως παρηκολούθησας την διδασκαλίαν μου, την διαγωγήν, την πρόθεσιν, την πίστιν, την μακροθυμίαν, την αγάπην, την υπομονήν,
\par 11 τους διωγμούς, τα παθήματα, οποία μοι συνέβησαν εν Αντιοχεία, εν Ικονίω, εν Λύστροις· οποίους διωγμούς υπέφερα, και εκ πάντων με ηλευθέρωσεν ο Κύριος.
\par 12 Και πάντες δε οι θέλοντες να ζώσιν ευσεβώς εν Χριστώ Ιησού θέλουσι διωχθή.
\par 13 Πονηροί δε άνθρωποι και γόητες θέλουσι προκόψει εις το χείρον, πλανώντες και πλανώμενοι.
\par 14 Αλλά συ μένε εις εκείνα, τα οποία έμαθες και επιστώθης, εξεύρων παρά τίνος έμαθες,
\par 15 και ότι από βρέφους γνωρίζεις τα ιερά γράμματα, τα δυνάμενα να σε σοφίσωσιν εις σωτηρίαν διά της πίστεως της εν Χριστώ Ιησού.
\par 16 Όλη η γραφή είναι θεόπνευστος και ωφέλιμος προς διδασκαλίαν, προς έλεγχον, προς επανόρθωσιν, προς εκπαίδευσιν την μετά της δικαιοσύνης,
\par 17 διά να ήναι τέλειος ο άνθρωπος του Θεού, ητοιμασμένος εις παν έργον αγαθόν

\chapter{4}

\par 1 Διαμαρτύρομαι λοιπόν εγώ ενώπιον του Θεού και του Κυρίου Ιησού Χριστού, όστις μέλλει να κρίνη ζώντας και νεκρούς εν τη επιφανεία αυτού και τη βασιλεία αυτού,
\par 2 κήρυξον τον λόγον, επίμενε εγκαίρως ακαίρως, έλεγξον, επίπληξον, πρότρεψον, μετά πάσης μακροθυμίας και διδαχής.
\par 3 Διότι θέλει ελθεί καιρός ότε δεν θέλουσιν υποφέρει την υγιαίνουσαν διδασκαλίαν, αλλά θέλουσιν επισωρεύσει εις εαυτούς διδασκάλους κατά τας ιδίας αυτών επιθυμίας, γαργαλιζόμενοι την ακοήν,
\par 4 και από μεν της αληθείας θέλουσιν αποστρέψει την ακοήν αυτών, εις δε τους μύθους θέλουσιν εκτραπή.
\par 5 Συ δε αγρύπνει εις πάντα, κακοπάθησον, εργάσθητι έργον ευαγγελιστού, την διακονίαν σου κάμε πλήρη.
\par 6 Διότι εγώ γίνομαι ήδη σπονδή και ο καιρός της αναχωρήσεώς μου έφθασε.
\par 7 Τον αγώνα τον καλόν ηγωνίσθην, τον δρόμον ετελείωσα, την πίστιν διετήρησα·
\par 8 του λοιπού μένει εις εμέ ο της δικαιοσύνης στέφανος, τον οποίον ο Κύριος θέλει μοι αποδώσει εν εκείνη τη ημέρα, ο δίκαιος κριτής, και ου μόνον εις εμέ, αλλά και εις πάντας όσοι επιποθούσι την επιφάνειαν αυτού.
\par 9 Σπούδασον να έλθης προς εμέ ταχέως·
\par 10 διότι ο Δημάς με εγκατέλιπεν, αγαπήσας τον παρόντα κόσμον, και απήλθεν εις θεσσαλονίκην, ο Κρήσκης εις Γαλατίαν, ο Τίτος εις Δαλματίαν·
\par 11 ο Λουκάς είναι μόνος μετ' εμού. Τον Μάρκον παραλαβών φέρε μετά σού· διότι μοι είναι χρήσιμος εις την διακονίαν.
\par 12 Τον δε Τυχικόν απέστειλα εις Έφεσον.
\par 13 Τον φελόνην, τον οποίον αφήκα εν Τρωάδι παρά τω Κάρπω, ερχόμενος φέρε, και τα βιβλία, μάλιστα τας μεμβράνας.
\par 14 Ο Αλέξανδρος ο χαλκεύς πολλά κακά μοι έκαμεν· ο Κύριος να αποδώση εις αυτόν κατά τα έργα αυτού·
\par 15 τον οποίον και συ φυλάττου· διότι πολύ ανθίσταται εις τους λόγους ημών.
\par 16 Εν τη πρώτη απολογία μου δεν με παρεστάθη ουδείς, αλλά πάντες με εγκατέλιπον· είθε να μη λογαριασθή εις αυτούς·
\par 17 αλλ' ο Κύριος με παρεστάθη και με ενεδυνάμωσε, διά να πληρωθή δι' εμού το κήρυγμα και να ακούσωσι πάντα τα έθνη· και ηλευθερώθην εκ του στόματος του λέοντος.
\par 18 Και θέλει με ελευθερώσει ο Κύριος από παντός έργου πονηρού και θέλει με διασώσει διά την επουράνιον βασιλείαν αυτού· εις τον οποίον έστω η δόξα εις τους αιώνας των αιώνων· αμήν.
\par 19 Ασπάσθητι την Πρίσκαν και τον Ακύλαν και τον οίκον του Ονησιφόρου.
\par 20 Ο Έραστος έμεινεν εν Κορίνθω, τον δε Τρόφιμον αφήκα εν Μιλήτω ασθενή.
\par 21 Σπούδασον να έλθης προ του χειμώνος. Ασπάζεταί σε ο Εύβουλος και Πούδης και Λίνος και η Κλαυδία και οι αδελφοί πάντες.
\par 22 Ο Κύριος Ιησούς Χριστός είη μετά του πνεύματός σου. Η χάρις μεθ' υμών· αμήν.


\end{document}