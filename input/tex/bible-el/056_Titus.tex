\begin{document}

\title{Τίτο}


\chapter{1}

\par 1 Παύλος, δούλος Θεού, απόστολος δε Ιησού Χριστού κατά την πίστιν των εκλεκτών του Θεού και την επίγνωσιν της αληθείας της κατ' ευσέβειαν
\par 2 επ' ελπίδι ζωής αιωνίου, την οποίαν υπεσχέθη ο αψευδής Θεός προ χρόνων αιωνίων,
\par 3 εφανέρωσε δε εν καιροίς ωρισμένοις τον λόγον αυτού διά του κηρύγματος, το οποίον ενεπιστεύθην εγώ κατ' επιταγήν του σωτήρος ημών Θεού,
\par 4 προς Τίτον, γνήσιον τέκνον κατά κοινήν ημών πίστιν· είη χάρις, έλεος, ειρήνη από Θεού Πατρός και Κυρίου Ιησού Χριστού του Σωτήρος ημών.
\par 5 Διά τούτο σε αφήκα εν Κρήτη, διά να διορθώσης τα ελλείποντα και να καταστήσης εν πάση πόλει πρεσβυτέρους, καθώς εγώ σε διέταξα,
\par 6 όστις είναι ανέγκλητος, μιας γυναικός ανήρ, έχων τέκνα πιστά, μη κατηγορούμενα ως άσωτα ή ανυπότακτα.
\par 7 Διότι πρέπει ο επίσκοπος να ήναι ανέγκλητος, ως οικονόμος Θεού, μη αυθάδης, μη οργίλος, μη μέθυσος, μη πλήκτης, μη αισχροκερδής,
\par 8 αλλά φιλόξενος, φιλάγαθος, σώφρων, δίκαιος, όσιος, εγκρατής,
\par 9 προσκεκολλημένος εις τον πιστόν λόγον της διδασκαλίας, διά να ήναι δυνατός και να προτρέπη διά της υγιαινούσης διδασκαλίας και να εξελέγχη τους αντιλέγοντας.
\par 10 Διότι υπάρχουσι πολλοί και ανυπότακτοι ματαιολόγοι και φρενοπλάνοι, μάλιστα οι εκ της περιτομής,
\par 11 τους οποίους πρέπει να αποστομόνωμεν, οίτινες ανατρέπουσιν ολοκλήρους οίκους, διδάσκοντες όσα δεν πρέπει, χάριν αισχρού κέρδους.
\par 12 Είπε τις αυτών προφήτης ίδιος αυτών· Οι Κρήτες είναι πάντοτε ψεύσται, κακά θηρία, γαστέρες αργαί.
\par 13 Η μαρτυρία αύτη είναι αληθινή. Διά την οποίαν αιτίαν έλεγχε αυτούς αποτόμως, διά να υγιαίνωσιν εν τη πίστει,
\par 14 και να μη προσέχωσιν εις Ιουδαϊκούς μύθους και εντολάς ανθρώπων αποστρεφομένων την αλήθειαν.
\par 15 Εις μεν τους καθαρούς πάντα είναι καθαρά· εις δε τους μεμιασμένους και απίστους ουδέν καθαρόν, αλλά και ο νούς αυτών και η συνείδησις είναι μεμιασμένα.
\par 16 Ομολογούσιν ότι γνωρίζουσι τον Θεόν, με τα έργα όμως αρνούνται, βδελυκτοί όντες και απειθείς και εις παν έργον αγαθόν αδόκιμοι.

\chapter{2}

\par 1 Συ όμως λάλει όσα πρέπουσιν εις την υγιαίνουσαν διδασκαλίαν.
\par 2 Οι γέροντες να ήναι άγρυπνοι, σεμνοί, σώφρονες, υγιαίνοντες εν τη πίστει, τη αγάπη, τη υπομονή.
\par 3 Αι γραίαι ωσαύτως να έχωσι τρόπον ιεροπρεπή, μη κατάλαλοι, μη δεδουλωμέναι εις πολλήν οινοποσίαν, να ήναι διδάσκαλοι των καλών,
\par 4 διά να νουθετώσι τας νέας να ήναι φίλανδροι, φιλότεκνοι,
\par 5 σώφρονες, καθαραί, οικοφύλακες, αγαθαί, ευπειθείς εις τους ιδίους αυτών άνδρας, διά να μη βλασφημήται ο λόγος του Θεού.
\par 6 Τους νεωτέρους ωσαύτως νουθέτει να σωφρονώσι,
\par 7 δεικνύων κατά πάντα σεαυτόν τύπον των καλών έργων, φυλάττων εν τη διδασκαλία αδιαφθορίαν, σεμνότητα,
\par 8 λόγον υγιή και ακατάκριτον, διά να εντραπή ο εναντίος, μη έχων να λέγη διά σας μηδέν κακόν.
\par 9 Τους δούλους να υποτάσσωνται εις τους εαυτών δεσπότας, να ευαρεστώσιν εις αυτούς κατά πάντα, να μη αντιλέγωσι,
\par 10 να μη σφετερίζωνται τα αλλότρια, αλλά να δεικνύωσι πάσαν πίστιν αγαθήν, διά να στολίζωσι κατά πάντα την διδασκαλίαν του σωτήρος ημών Θεού.
\par 11 Διότι εφανερώθη η χάρις του Θεού η σωτήριος εις πάντας ανθρώπους,
\par 12 διδάσκουσα ημάς να αρνηθώμεν την ασέβειαν και τας κοσμικάς επιθυμίας και να ζήσωμεν σωφρόνως και δικαίως και ευσεβώς εν τω παρόντι αιώνι,
\par 13 προσμένοντες την μακαρίαν ελπίδα και επιφάνειαν της δόξης του μεγάλου Θεού και Σωτήρος ημών Ιησού Χριστού,
\par 14 όστις έδωκεν εαυτόν υπέρ ημών, διά να μας λυτρώση από πάσης ανομίας και να μας καθαρίση εις εαυτόν λαόν εκλεκτόν, ζηλωτήν καλών έργων.
\par 15 Ταύτα λάλει και πρότρεπε και έλεγχε μετά πάσης εξουσίας· ας μη σε περιφρονή μηδείς.

\chapter{3}

\par 1 Υπενθύμιζε αυτούς να υποτάσσωνται εις τας αρχάς και εξουσίας, να πειθαρχώσι, να ήναι έτοιμοι εις παν έργον αγαθόν,
\par 2 να μη βλασφημώσι μηδένα, να ήναι άμαχοι, συμβιβαστικοί, να δεικνύωσι προς πάντας ανθρώπους πάσαν πραότητα.
\par 3 Διότι ήμεθά ποτέ και ημείς ανόητοι, απειθείς, πλανώμενοι, δουλεύοντες εις διαφόρους επιθυμίας και ηδονάς, ζώντες εν κακία και φθόνω, μισητοί και μισούντες αλλήλους.
\par 4 Αλλ' ότε εφανερώθη η χρηστότης και η φιλανθρωπία του Σωτήρος ημών Θεού,
\par 5 ουχί εξ έργων δικαιοσύνης τα οποία επράξαμεν ημείς, αλλά κατά το έλεος αυτού έσωσεν ημάς διά λουτρού παλιγγενεσίας και ανακαινίσεως του Αγίου Πνεύματος,
\par 6 το οποίον εξέχεε πλουσίως εφ' ημάς διά Ιησού Χριστού του Σωτήρος ημών,
\par 7 ίνα δικαιωθέντες διά της χάριτος εκείνου, γείνωμεν κληρονόμοι κατά την ελπίδα της αιωνίου ζωής.
\par 8 Πιστός ο λόγος, και θέλω ταύτα να διαβεβαιοίς, διά να φροντίζωσιν οι πιστεύσαντες εις τον Θεόν να προΐστανται καλών έργων. Ταύτα είναι τα καλά και ωφέλιμα εις τους ανθρώπους·
\par 9 μωράς δε φιλονεικίας και γενεαλογίας και έριδας και μάχας νομικάς φεύγε, διότι είναι ανωφελείς και μάταιαι.
\par 10 Αιρετικόν άνθρωπον μετά μίαν και δευτέραν νουθεσίαν παραιτού,
\par 11 εξεύρων ότι διεφθάρη ο τοιούτος και αμαρτάνει, ων αυτοκατάκριτος.
\par 12 Όταν πέμψω προς σε τον Αρτεμάν ή τον Τυχικόν, σπούδασον να έλθης προς με εις Νικόπολιν· διότι εκεί απεφάσισα να παραχειμάσω.
\par 13 Ζηνάν τον νομικόν και τον Απολλώ πρόπεμψον επιμελώς, διά να μη λείπη εις αυτούς μηδέν.
\par 14 Ας μανθάνωσι δε και οι ημέτεροι να προΐστανται καλών έργων εις τας αναγκαίας χρείας, διά να μη ήναι άκαρποι.
\par 15 Ασπάζονταί σε πάντες οι μετ' εμού· ασπάσθητι τους αγαπώντας ημάς εν πίστει. Η χάρις είη μετά πάντων υμών. Αμήν.


\end{document}