\begin{document}

\title{Hebrews}


\chapter{1}

\par 1 Ο Θεός, αφού ελάλησε το πάλαι προς τους πατέρας ημών διά των προφητών πολλάκις και πολυτρόπως,
\par 2 εν ταις εσχάταις ταύταις ημέραις ελάλησε προς ημάς διά του Υιού, τον οποίον έθεσε κληρονόμον πάντων, δι' ου έκαμε και τους αιώνας·
\par 3 όστις ων απαύγασμα της δόξης και χαρακτήρ της υποστάσεως αυτού, και βαστάζων τα πάντα με τον λόγον της δυνάμεως αυτού, αφού δι' εαυτού έκαμε καθαρισμόν των αμαρτιών ημών, εκάθησεν εν δεξιά της μεγαλωσύνης εν υψηλοίς,
\par 4 τοσούτον ανώτερος των αγγέλων γενόμενος, όσον εξοχώτερον υπέρ αυτούς όνομα εκληρονόμησε.
\par 5 Διότι προς τίνα των αγγέλων είπε ποτε· Υιός μου είσαι συ, Εγώ σήμερον σε εγέννησα; και πάλιν· Εγώ θέλω είσθαι εις αυτόν Πατήρ, και αυτός θέλει είσθαι εις εμέ Υιός;
\par 6 Όταν δε πάλιν εισαγάγη τον πρωτότοκον εις την οικουμένην, λέγει· Και ας προσκυνήσωσιν εις αυτόν πάντες οι άγγελοι του Θεού.
\par 7 Και περί μεν των αγγέλων λέγει· Ο ποιών τους αγγέλους αυτού πνεύματα, και τους λειτουργούς αυτού πυρός φλόγα·
\par 8 περί δε του Υιού· Ο θρόνος σου, ω Θεέ, είναι εις τον αιώνα του αιώνος· σκήπτρον ευθύτητος είναι το σκήπτρον της βασιλείας σου.
\par 9 Ηγάπησας δικαιοσύνην και εμίσησας ανομίαν· διά τούτο έχρισέ σε, ο Θεός, ο Θεός σου, έλαιον αγαλλιάσεως υπέρ τους μετόχους σου·
\par 10 καί· Συ κατ' αρχάς, Κύριε, την γην εθεμελίωσας, και έργα των χειρών σου είναι οι ουρανοί·
\par 11 αυτοί θέλουσιν απολεσθή, συ δε διαμένεις· και πάντες ως ιμάτιον θέλουσι παλαιωθή,
\par 12 και ως περιένδυμα θέλεις τυλίξει αυτούς, και θέλουσιν αλλαχθή· Συ όμως είσαι ο αυτός, και τα έτη σου δεν θέλουσιν εκλείψει.
\par 13 Προς τίνα δε των αγγέλων είπε ποτε· Κάθου εκ δεξιών μου, εωσού θέσω τους εχθρούς σου υποπόδιον των ποδών σου;
\par 14 Δεν είναι πάντες λειτουργικά πνεύματα εις υπηρεσίαν αποστελλόμενα διά τους μέλλοντας να κληρονομήσωσι σωτηρίαν;

\chapter{2}

\par 1 Διά τούτο πρέπει ημείς να προσέχωμεν περισσότερον εις όσα ηκούσαμεν, διά να μη εκπέσωμέν ποτέ.
\par 2 Διότι εάν ο λόγος ο λαληθείς δι' αγγέλων έγεινε βέβαιος, και πάσα παράβασις και παρακοή έλαβε δικαίαν μισθαποδοσίαν,
\par 3 πως ημείς θέλομεν εκφύγει, εάν αμελήσωμεν τόσον μεγάλην σωτηρίαν; ήτις αρχίσασα να λαλήται διά του Κυρίου, εβεβαιώθη εις ημάς υπό των ακουσάντων,
\par 4 και ο Θεός συνεπεμαρτύρει με σημεία και τέρατα και με διάφορα θαύματα και με διανομάς του Αγίου Πνεύματος κατά την θέλησιν αυτού.
\par 5 Διότι δεν υπέταξεν εις αγγέλους την οικουμένην την μέλλουσαν, περί της οποίας λαλούμεν.
\par 6 Εμαρτύρησε δε τις εν τινί μέρει, λέγων· Τι είναι ο άνθρωπος, ώστε να ενθυμήσαι αυτόν, Η ο υιός του ανθρώπου, ώστε να επισκέπτησαι αυτόν;
\par 7 Έκαμες αυτόν ολίγον τι κατώτερον των αγγέλων, με δόξαν και τιμήν εστεφάνωσας αυτόν και κατέστησας αυτόν επί τα έργα των χειρών σου·
\par 8 Πάντα υπέταξας υποκάτω των ποδών αυτού. Διότι υποτάξας εις αυτόν τα πάντα, δεν αφήκεν ουδέν ανυπότακτον εις αυτόν. Τώρα όμως δεν βλέπομεν έτι τα πάντα υποτεταγμένα εις αυτόν·
\par 9 τον δε ολίγον τι παρά τους αγγέλους ηλαττωμένον Ιησούν βλέπομεν διά το πάθημα του θανάτου με δόξαν και τιμήν εστεφανωμένον, διά να γευθή θάνατον υπέρ παντός ανθρώπου διά της χάριτος του Θεού.
\par 10 Διότι έπρεπεν εις αυτόν, διά τον οποίον είναι τα πάντα και διά του οποίου έγειναν τα πάντα, φέρων εις την δόξαν πολλούς υιούς, να κάμη τέλειον τον αρχηγόν της σωτηρίας αυτών διά των παθημάτων.
\par 11 Επειδή και ο αγιάζων και οι αγιαζόμενοι εξ ενός είναι πάντες· δι' ην αιτίαν δεν επαισχύνεται να ονομάζη αυτούς αδελφούς,
\par 12 λέγων· Θέλω απαγγείλει το όνομά σου προς τους αδελφούς μου, εν μέσω εκκλησίας θέλω σε υμνήσει·
\par 13 και πάλιν· Εγώ θέλω έχει την πεποίθησίν μου επ' αυτόν· και πάλιν· Ιδού, εγώ και τα παιδία, τα οποία μοι έδωκεν ο Θεός.
\par 14 Επειδή λοιπόν τα παιδία εμέθεξαν από σαρκός και αίματος, και αυτός παρομοίως μετέλαβεν από των αυτών, διά να καταργήση διά του θανάτου τον έχοντα το κράτος του θανάτου, τουτέστι τον διάβολον,
\par 15 και ελευθερώση εκείνους, όσοι διά τον φόβον του θανάτου ήσαν διά παντός του βίου υποκείμενοι εις την δουλείαν.
\par 16 Διότι βεβαίως δεν ανέλαβεν αγγέλων φύσιν, αλλά σπέρματος Αβραάμ ανέλαβεν.
\par 17 Όθεν έπρεπε να ομοιωθή κατά πάντα με τους αδελφούς, διά να γείνη ελεήμων και πιστός αρχιερεύς εις τα προς τον Θεόν, διά να κάμνη εξιλέωσιν υπέρ των αμαρτιών του λαού.
\par 18 Επειδή καθ' ότι αυτός έπαθε πειρασθείς, δύναται να βοηθήση τους πειραζομένους.

\chapter{3}

\par 1 Όθεν, αδελφοί άγιοι, ουρανίου προσκλήσεως μέτοχοι, κατανοήσατε τον απόστολον και αρχιερέα της ομολογίας ημών τον Ιησούν Χριστόν,
\par 2 όστις ήτο πιστός εις τον καταστήσαντα αυτόν, καθώς και ο Μωϋσής εις όλον τον οίκον αυτού.
\par 3 Επειδή ούτος ηξιώθη πλειοτέρας δόξης παρά τον Μωϋσήν, καθ' όσον έχει τιμήν πλειοτέραν παρά τον οίκον ο κατασκευάσας αυτόν.
\par 4 Διότι πας οίκος κατασκευάζεται υπό τινός, ο δε κατασκευάσας τα πάντα είναι ο Θεός.
\par 5 Και ο μεν Μωϋσής υπήρξε πιστός εις όλον τον οίκον αυτού ως θεράπων, εις μαρτυρίαν των λαληθησομένων,
\par 6 ο δε Χριστός ως Υιός επί τον οίκον αυτού, του οποίου ημείς είμεθα οίκος, εάν κρατήσωμεν μέχρι τέλους βεβαίαν την παρρησίαν και το καύχημα της ελπίδος.
\par 7 Διά τούτο, καθώς λέγει το Πνεύμα το Αγιον· Σήμερον, εάν ακούσητε της φωνής αυτού,
\par 8 μη σκληρύνητε τας καρδίας σας ως εν τω παραπικρασμώ κατά την ημέραν του πειρασμού εν τη ερήμω,
\par 9 όπου οι πατέρες σας με επείραααν, με εδοκίμασαν και είδον τα έργα μου τεσσαράκοντα έτη·
\par 10 διά τούτο δυσηρεστήθην εις την γενεάν εκείνην και είπον· Πάντοτε πλανώνται εν τη καρδία αυτών και αυτοί δεν εγνώρισαν τας οδούς μου·
\par 11 ούτως ώμοσα εν τη οργή μου, δεν θέλουσιν εισέλθει εις την κατάπαυσίν μου·
\par 12 προσέχετε, αδελφοί, να μη υπάρχη εις μήδενα από σας πονηρά καρδία απιστίας, ώστε να αποστατήση από Θεού ζώντος,
\par 13 αλλά προτρέπετε αλλήλους καθ' εκάστην ημέραν, ενόσω ονομάζεται το σήμερον, διά να μη σκληρυνθή τις εξ υμών διά της απάτης της αμαρτίας·
\par 14 διότι μέτοχοι εγείναμεν του Χριστού, εάν κρατήσωμεν μέχρι τέλους βεβαίαν την αρχήν της πεποιθήσεως,
\par 15 ενώ λέγεται· Σήμερον, εάν ακούσητε της φωνής αυτού, μη σκληρύνητε τας καρδίας σας ως εν τω παραπικρασμώ.
\par 16 Διότι τινές, αφού ήκουσαν, παρεπίκραναν αυτόν αλλ' ουχί πάντες οι εξελθόντες εξ Αιγύπτου διά του Μωϋσέως.
\par 17 Εις τίνας δε παρωργίσθη τεσσαράκοντα έτη; ουχί εις τους αμαρτήσαντας, των οποίων τα κώλα έπεσον εν τη ερήμω;
\par 18 Προς τίνας δε ώμοσεν ότι δεν θέλουσιν εισέλθει εις την κατάπαυσιν αυτού, ειμή προς τους απειθήσαντας;
\par 19 Και βλέπομεν ότι διά απιστίαν δεν ηδυνήθησαν να εισέλθωσι.

\chapter{4}

\par 1 Ας φοβηθώμεν λοιπόν μήποτε, ενώ μένει εις ημάς επαγγελία να εισέλθωμεν εις την κατάπαυσιν αυτού, φανή τις εξ υμών ότι υστερήθη αυτής.
\par 2 Διότι ημείς ευηγγελίσθημεν, καθώς και εκείνοι· αλλά δεν ωφέλησεν εκείνους ο λόγος, τον οποίον ήκουσαν, επειδή δεν ήτο εις τους ακούσαντας ηνωμένος με την πίστιν.
\par 3 Διότι εισερχόμεθα εις την κατάπαυσιν ημείς οι πιστεύσαντες, καθώς είπεν· Ούτως ώμοσα εν τη οργή μου, δεν θέλουσιν εισέλθει εις την κατάπαυσίν μου· αν και τα έργα αυτού ετελείωσαν από καταβολής κόσμου.
\par 4 Διότι είπεν εν μέρει τινί περί της εβδόμης ούτω· Και κατέπαυσεν ο Θεός εν τη ημέρα τη εβδόμη από πάντων των έργων αυτού·
\par 5 και εν τούτω πάλιν· Δεν θέλουσιν εισέλθει εις την κατάπαυσίν μου.
\par 6 Επειδή λοιπόν μένει να εισέλθωσί τινές εις αυτήν, και οι πρότερον ευαγγελισθέντες δεν εισήλθον δι' απείθειαν
\par 7 πάλιν διορίζει ημέραν τινά, Σήμερον, λέγων διά του Δαβίδ, μετά τοσούτον καιρόν, καθώς είρηται· Σήμερον, εάν της φωνής αυτού ακούσητε, μη σκληρύνητε τας καρδίας σας.
\par 8 Διότι εάν ο Ιησούς σου Ναυή είχε δώσει εις αυτούς κατάπαυσιν, δεν ήθελε μετά ταύτα λαλεί περί άλλης ημέρας.
\par 9 Άρα μένει κατάπαυσις εις τον λαόν του Θεού.
\par 10 Διότι ο εισελθών εις την κατάπαυσιν αυτού και αυτός κατέπαυσεν από των έργων αυτού, καθώς ο Θεός από των εαυτού.
\par 11 Ας σπουδάσωμεν λοιπόν να εισέλθωμεν εις εκείνην την κατάπαυσιν, διά να μη πέση τις εις το αυτό παράδειγμα της απειθείας.
\par 12 Διότι ο λόγος του Θεού είναι ζων και ενεργός και κοπτερώτερος υπέρ πάσαν δίστομον μάχαιραν και διέρχεται μέχρι διαιρέσεως ψυχής τε και πνεύματος, αρμών τε και μυελών, και διερευνά τους διαλογισμούς και τας εννοίας της καρδίας·
\par 13 και δεν είναι ουδέν κτίσμα αφανές ενώπιον αυτού, αλλά πάντα είναι γυμνά και τετραχηλισμένα εις τους οφθαλμούς αυτού, προς ον έχομεν να δώσωμεν λόγον.
\par 14 Έχοντες λοιπόν αρχιερέα μέγαν, όστις διήλθε τους ουρανούς, Ιησούν τον Υιόν του Θεού, ας κρατώμεν την ομολογίαν.
\par 15 Διότι δεν έχομεν αρχιερέα μη δυνάμενον να συμπαθήση εις τας ασθενείας ημών, αλλά πειρασθέντα κατά πάντα καθ' ομοιότητα ημών χωρίς αμαρτίας.
\par 16 Ας πλησιάζωμεν λοιπόν μετά παρρησίας εις τον θρόνον της χάριτος, διά να λάβωμεν έλεος και να εύρωμεν χάριν προς βοήθειαν εν καιρώ χρείας.

\chapter{5}

\par 1 Διότι πας αρχιερεύς, εξ ανθρώπων λαμβανόμενος, υπέρ ανθρώπων καθίσταται εις τα προς τον Θεόν, διά να προσφέρη δώρα τε και θυσίας υπέρ αμαρτιών,
\par 2 δυνάμενος να συμπαθή εις τους αγνοούντας και πλανωμένους, διότι και αυτός είναι περιενδεδυμένος ασθένειαν·
\par 3 και διά ταύτην χρεωστεί, καθώς περί του λαού, ούτω και περί εαυτού να προσφέρη θυσίαν υπέρ αμαρτιών.
\par 4 Και ουδείς λαμβάνει την τιμήν ταύτην εις εαυτόν, αλλ' ο καλούμενος υπό του Θεού, καθώς και ο Ααρών.
\par 5 Ούτω και ο Χριστός δεν εδόξασεν εαυτόν διά να γείνη αρχιερεύς, αλλ' ο λαλήσας προς αυτόν· Υιός μου είσαι συ, εγώ σήμερον σε εγέννησα·
\par 6 καθώς και αλλαχού λέγει· Συ είσαι ιερεύς εις τον αιώνα κατά την τάξιν Μελχισεδέκ.
\par 7 Όστις εν ταις ημέραις της σαρκός αυτού, αφού μετά κραυγής δυνατής και δακρύων προσέφερε δεήσεις και ικεσίας προς τον δυνάμενον να σώζη αυτόν εκ του θανάτου, και εισηκούσθη διά την ευλάβειαν αυτού,
\par 8 καίτοι ων Υιός, έμαθε την υπακοήν αφ' όσων έπαθε,
\par 9 και γενόμενος τέλειος, κατεστάθη αίτιος σωτηρίας αιωνίου εις πάντας τους υπακούοντας εις αυτόν,
\par 10 ονομασθείς υπό του Θεού αρχιερεύς κατά την τάξιν Μελχισεδέκ·
\par 11 Περί του οποίου πολλά έχομεν να είπωμεν και δυσερμήνευτα, διότι εγείνετε νωθροί τας ακοάς.
\par 12 Επειδή ενώ ως προς τον καιρόν έπρεπε να ήσθε διδάσκαλοι, πάλιν έχετε χρείαν του να σας διδάσκη τις τα αρχικά στοιχεία των λόγων του Θεού, και κατηντήσατε να έχητε χρείαν γάλακτος και ουχί στερεάς τροφής.
\par 13 Διότι πας ο μετέχων γάλακτος είναι άπειρος του λόγου της δικαιοσύνης· επειδή είναι νήπιος·
\par 14 των τελείων όμως είναι η στερεά τροφή, οίτινες διά την έξιν έχουσι τα αισθητήρια γεγυμνασμένα εις το να διακρίνωσι το καλόν και το κακόν.

\chapter{6}

\par 1 Διά τούτο αφήσαντες την αρχικήν διδασκαλίαν του Χριστού, ας φερώμεθα προς την τελειότητα, χωρίς να βάλλωμεν εκ νέου θεμέλιον μετανοίας από νεκρών έργων και πίστεως εις Θεόν,
\par 2 της διδαχής των βαπτισμών και της επιθέσεως των χειρών, και της αναστάσεως των νεκρών και της κρίσεως της αιωνίου.
\par 3 Και τούτο θέλομεν κάμει, εάν επιτρέπη ο Θεός.
\par 4 Διότι αδύνατον είναι οι άπαξ φωτισθέντες και γευθέντες της επουρανίου δωρεάς και γενόμενοι μέτοχοι του Αγίου Πνεύματος
\par 5 και γευθέντες τον καλόν λόγον του Θεού και τας δυνάμεις του μέλλοντος αιώνος,
\par 6 και έπειτα παραπεσόντες, αδύνατον να ανακαινισθώσι πάλιν εις μετάνοιαν, ανασταυρούντες εις εαυτούς τον Υιόν του Θεού και καταισχύνοντες.
\par 7 Διότι γη, ήτις πίνει την πολλάκις ερχομένην επ' αυτής βροχήν και γεννά βοτάνην ωφέλιμον εις εκείνους, διά τους οποίους και γεωργείται, μεταλαμβάνει ευλογίαν παρά Θεού·
\par 8 όταν όμως εκφύη ακάνθας και τριβόλους, είναι αδόκιμος και πλησίον κατάρας, της οποίας το τέλος είναι να καυθή.
\par 9 Περί υμών δε, αν και λαλώμεν ούτως, αγαπητοί, είμεθα πεπεισμένοι ότι έχετε τα καλήτερα και συνεχόμενα με την σωτηρίαν.
\par 10 Διότι δεν είναι άδικος ο Θεός, ώστε να λησμονήση το έργον σας και τον κόπον της αγάπης, την οποίαν εδείξατε εις το όνομα αυτού, υπηρετήσαντες τους αγίους και υπηρετούντες.
\par 11 Επιθυμούμεν δε να δεικνύη έκαστος υμών την αυτήν σπουδήν προς την πληροφορίαν της ελπίδος μέχρι τέλους,
\par 12 διά να μη γείνητε νωθροί, αλλά μιμηταί των διά πίστεως και μακροθυμίας κληρονομούντων τας επαγγελίας.
\par 13 Διότι ο Θεός, δίδων επαγγελίαν εις τον Αβραάμ, επειδή δεν είχε να ομόση εις ουδένα μεγαλήτερον, ώμοσεν εις εαυτόν,
\par 14 λέγων· Βεβαίως ευλογών θέλω σε ευλογήσει και πληθύνων θέλω σε πληθύνει·
\par 15 και ούτω προσμείνας με υπομονήν, απήλαυσε την επαγγελίαν.
\par 16 Διότι οι μεν άνθρωποι ομνύουσιν εις τον μεγαλήτερον, και ο όρκος είναι εις αυτούς τέλος πάσης αντιλογίας προς βεβαίωσιν.
\par 17 Εις το οποίον ο Θεός, θέλων να δείξη περισσότερον προς τους κληρονόμους της επαγγελίας το αμετάθετον της βουλής αυτού, μετεχειρίσθη μέσον τον όρκον,
\par 18 ώστε διά δύο πραγμάτων αμεταθέτων, εις τα οποία είναι αδύνατον να ψευσθή ο Θεός, να έχωμεν ισχυράν παρηγορίαν οι καταφυγόντες εις το να κρατήσωμεν την προκειμένην ελπίδα·
\par 19 την οποίαν έχομεν ως άγκυραν της ψυχής ασφαλή τε και βεβαίαν και εισερχομένην εις το εσωτερικόν του καταπετάσματος,
\par 20 όπου ο Ιησούς εισήλθεν υπέρ ημών πρόδρομος, γενόμενος αρχιερεύς εις τον αιώνα κατά την τάξιν Μελχισεδέκ.

\chapter{7}

\par 1 Διότι ούτος ο Μελχισεδέκ, βασιλεύς Σαλήμ, ιερεύς του Θεού του Υψίστου, όστις συνήντησε τον Αβραάμ επιστρέφοντα από της καταστροφής των βασιλέων και ηυλόγησεν αυτόν,
\par 2 εις ον ο Αβραάμ εχώρισε και δέκατον από πάντων των λαφύρων, όστις πρώτον μεν ερμηνεύεται βασιλεύς δικαιοσύνης, έπειτα δε βασιλεύς Σαλήμ, το οποίον είναι βασιλεύς ειρήνης,
\par 3 απάτωρ, αμήτωρ, αγενεαλόγητος, μη έχων μήτε αρχήν ημερών μήτε τέλος ζωής, αλλ' αφωμοιωμένος με τον Υιόν του Θεού, μένει ιερεύς πάντοτε.
\par 4 Στοχασθήτε δε πόσον μέγας ήτο ούτος, εις ον ο Αβραάμ ο πατριάρχης έδωκε και δέκατον εκ των λαφύρων.
\par 5 Και όσοι μεν εκ των υιών του Λευΐ λαμβάνουσι την ιερατείαν, έχουσιν εντολήν να αποδεκατόνωσι τον λαόν κατά τον νόμον, τουτέστι τους αδελφούς αυτών, καίτοι εξελθόντας εκ της οσφύος του Αβραάμ·
\par 6 εκείνος δε όστις δεν εγενεαλογείτο εξ αυτών, εδεκάτωσε τον Αβραάμ, και ηυλόγησε τον έχοντα τας επαγγελίας·
\par 7 χωρίς δε τινός αντιλογίας το μικρότερον ευλογείται υπό του μεγαλητέρου.
\par 8 Και εδώ μεν θνητοί άνθρωποι λαμβάνουσι δέκατα, εκεί δε λαμβάνει ο μαρτυρούμενος ότι ζη.
\par 9 Και διά να είπω ούτω, διά του Αβραάμ και ο Λευΐ, όστις ελάμβανε δέκατα, απεδεκατώθη.
\par 10 Διότι εν τη οσφύϊ του πατρός αυτού ήτο έτι, ότε συνήντησεν αυτόν ο Μελχισεδέκ.
\par 11 Εάν λοιπόν η τελειότης υπήρχε διά Λευϊτικής ιερωσύνης· διότι ο λαός επ' αυτής έλαβε τον νόμον· τις χρεία πλέον να εγερθή άλλος ιερεύς κατά την τάξιν Μελχισεδέχ, και ουχί να λέγηται κατά την τάξιν Ααρών;
\par 12 Διότι μετατιθεμένης της ιερωσύνης, εξ ανάγκης και νόμου μετάθεσις γίνεται.
\par 13 Επειδή εκείνος, περί του οποίου λέγονται ταύτα, άλλης φυλής μετείχεν, εξ ης ουδείς επλησίασεν εις το θυσιαστήριον.
\par 14 Επειδή είναι πρόδηλον ότι εξ Ιούδα ανέτειλεν ο Κύριος ημών, εις την οποίαν φυλήν ο Μωϋσής ουδέν περί ιερωσύνης ελάλησε.
\par 15 Και περισσότερον έτι κατάδηλον είναι, διότι κατά την ομοιότητα του Μελχισεδέκ εγείρεται άλλος ιερεύς,
\par 16 όστις δεν έγεινε κατά νόμον σαρκικής εντολής αλλά κατά δύναμιν ζωής ατελευτήτου·
\par 17 διότι μαρτυρεί λέγων ότι Συ είσαι ιερεύς εις τον αιώνα κατά την τάξιν Μελχισεδέκ.
\par 18 Διότι αθέτησις μεν γίνεται της προηγουμένης εντολής διά το ασθενές και ανωφελές αυτής·
\par 19 επειδή ο νόμος ουδέν έφερεν εις το τέλειον, έγεινε δε επεισαγωγή ελπίδος καλητέρας, διά της οποίας πλησιάζομεν εις τον Θεόν.
\par 20 Και καθ' όσον δεν έγεινεν ιερεύς χωρίς ορκωμοσίας·
\par 21 διότι εκείνοι έγειναν ιερείς χωρίς ορκωμοσίας, ούτος δε μετά ορκωμοσίας διά του λέγοντος προς αυτόν· Ώμοσε Κύριος, και δεν θέλει μεταμεληθή· Συ είσαι ιερεύς εις τον αιώνα κατά την τάξιν Μελχισεδέκ·
\par 22 κατά τοσούτον ανωτέρας διαθήκης εγγυητής έγεινεν ο Ιησούς.
\par 23 Και εκείνοι μεν έγειναν πολλοί ιερείς, επειδή ημποδίζοντο υπό του θανάτου να παραμένωσιν·
\par 24 εκείνος όμως, επειδή μένει εις τον αιώνα, έχει αμετάθετον την ιερωσύνην·
\par 25 όθεν δύναται και να σώζη εντελώς τους προσερχομένους εις τον Θεόν δι' αυτού, ζων πάντοτε διά να μεσιτεύση υπέρ αυτών.
\par 26 Διότι τοιούτος αρχιερεύς έπρεπεν εις ημάς, όσιος, άκακος, αμίαντος, κεχωρισμένος από των αμαρτωλών και υψηλότερος των ουρανών γενόμενος,
\par 27 όστις δεν έχει καθ' ημέραν ανάγκην, ως οι αρχιερείς να προσφέρη πρότερον θυσίας υπέρ των ιδίων αυτού αμαρτιών, έπειτα υπέρ των του λαού· διότι άπαξ έκαμε τούτο, ότε προσέφερεν εαυτόν.
\par 28 Διότι ο νόμος καθιστά αρχιερείς ανθρώπους έχοντας αδυναμίαν· ο λόγος όμως της ορκωμοσίας της μετά τον νόμον κατέστησε τον Υιόν, όστις είναι τετελειωμένος εις τον αιώνα.

\chapter{8}

\par 1 Κεφάλαιον δε των λεγομένων είναι τούτο, Τοιούτον έχομεν αρχιερέα, όστις εκάθησεν εν δεξιά του θρόνου της μεγαλωσύνης εν τοις ουρανοίς,
\par 2 λειτουργός των αγίων και της σκηνής της αληθινής, την οποίαν κατεσκεύασεν ο Κύριος, και ουχί άνθρωπος.
\par 3 Διότι πας αρχιερεύς καθίσταται διά να προσφέρη δώρα και θυσίας· όθεν είναι αναγκαίον να έχη και ούτός τι, το οποίον να προσφέρη.
\par 4 Επειδή εάν ήτο επί γης, ουδέ ήθελεν είσθαι ιερεύς, διότι υπήρχον οι ιερείς οι προσφέροντες τα δώρα κατά τον νόμον,
\par 5 οίτινες λειτουργούσιν εις υπόδειγμα και σκιάν των επουρανίων, καθώς ελαλήθη προς τον Μωϋσήν ότε έμελλε να κατασκευάση την σκηνήν· διότι Πρόσεχε, λέγει, να κάμης πάντα κατά τον τύπον τον δειχθέντα εις σε εν τω όρει.
\par 6 Τώρα όμως ο Χριστός έλαβεν εξοχωτέραν λειτουργίαν, καθόσον είναι και ανωτέρας διαθήκης μεσίτης, ήτις ενομοθετήθη με ανωτέρας επαγγελίας.
\par 7 Διότι εάν η πρώτη εκείνη ήτο άμεμπτος, δεν ήθελε ζητείσθαι τόπος διά την δευτέραν.
\par 8 Διότι μεμφόμενος αυτούς λέγει· Ιδού, έρχονται ημέραι, λέγει Κύριος, και θέλω συντελέσει επί τον οίκον του Ισραήλ και επί τον οίκον του Ιούδα διαθήκην καινήν,
\par 9 ουχί κατά την διαθήκην, την οποίαν έκαμον προς τους πατέρας αυτών, καθ' ην ημέραν επίασα αυτούς από της χειρός διά να εξαγάγω αυτούς εκ γης Αιγύπτου· διότι αυτοί δεν ενέμειναν εις την διαθήκην μου, και εγώ ημέλησα αυτούς, λέγει Κύριος.
\par 10 Διότι αύτη είναι η διαθήκη, την οποίαν θέλω κάμει προς τον οίκον του Ισραήλ μετά τας ημέρας εκείνας, λέγει Κύριος· Θέλω δώσει τους νόμους μου εις την διάνοιαν αυτών, και θέλω γράψει αυτούς επί της καρδίας αυτών, και θέλω είσθαι εις αυτούς Θεός, και αυτοί θέλουσιν είσθαι εις εμέ λαός.
\par 11 Και δεν θέλουσι διδάσκει έκαστος τον πλησίον αυτού και έκαστος τον αδελφόν αυτού, λέγων· Γνώρισον τον Κύριον· διότι πάντες θέλουσι με γνωρίζει από μικρού έως μεγάλου αυτών·
\par 12 διότι θέλω είσθαι ίλεως εις τας αδικίας αυτών, και τας αμαρτίας αυτών και τας ανομίας αυτών δεν θέλω ενθυμείσθαι πλέον.
\par 13 Λέγων δε καινήν, έκαμε παλαιάν την πρώτην· το δε παλαιούμενον και γηράσκον είναι πλησίον αφανισμού.

\chapter{9}

\par 1 Είχε μεν λοιπόν και η πρώτη σκηνή διατάξεις λατρείας και το άγιον το κοσμικόν.
\par 2 Διότι κατεσκευάσθη σκηνή η πρώτη, εις την οποίαν ήτο και η λυχνία και η τράπεζα και η πρόθεσις των άρτων, ήτις λέγεται Άγια.
\par 3 Μετά δε το δεύτερον καταπέτασμα ήτο σκηνή η λεγομένη Άγια αγίων,
\par 4 έχουσα χρυσούν θυμιατήριον και την κιβωτόν της διαθήκης πανταχόθεν περικεκαλυμμένην με χρυσίον, εν ή ήτο στάμνος χρυσή, έχουσα το μάννα, και η ράβδος του Ααρών η βλαστήσασα και αι πλάκες της διαθήκης,
\par 5 υπεράνω δε αυτής ήσαν Χερουβείμ δόξης κατασκιάζοντα το ιλαστήριον· περί των οποίων δεν είναι τώρα χρεία να λέγωμεν κατά μέρος.
\par 6 Όντων δε τούτων ούτω κατεσκευασμένων, εις μεν την πρώτην σκηνήν εισέρχονται διαπαντός οι ιερείς εκτελούντες τας λατρείας,
\par 7 εις δε την δευτέραν άπαξ του ενιαυτού εισέρχεται μόνος ο αρχιερεύς, ουχί χωρίς αίματος, το οποίον προσφέρει υπέρ εαυτού και των εξ αγνοίας αμαρτημάτων του λαού,
\par 8 και τούτο εδηλοποίει το Πνεύμα το Άγιον, ότι δεν ήτο πεφανερωμένη η εις τα άγια οδός, επειδή η πρώτη σκηνή ίστατο έτι·
\par 9 ήτις ήτο τύπος εις τον τότε παρόντα καιρόν, καθ' ον προσεφέροντο δώρα και θυσίαι, αίτινες δεν ηδύναντο να κάμωσι τέλειον κατά την συνείδησιν τον λατρεύοντα,
\par 10 επειδή ήσαν διατεταγμένα μόνον εις βρώματα και πόματα και διαφόρους βαπτισμούς και διατάξεις σαρκικάς, μέχρι καιρού διορθώσεως.
\par 11 Ελθών δε ο Χριστός αρχιερεύς των μελλόντων αγαθών διά της μεγαλητέρας και τελειοτέρας σκηνής, ουχί χειροποιήτου, τουτέστιν ουχί ταύτης της κατασκευής,
\par 12 ουδέ δι' αίματος τράγων και μόσχων, αλλά διά του ιδίου αυτού αίματος, εισήλθεν άπαξ εις τα άγια, αποκτήσας αιωνίαν λύτρωσιν.
\par 13 Διότι εάν το αίμα των ταύρων και τράγων και η σποδός της δαμάλεως ραντίζουσα τους μεμολυσμένους αγιάζη προς την καθαρότητα της σαρκός,
\par 14 πόσω μάλλον το αίμα του Χριστού, όστις διά του Πνεύματος του αιωνίου προσέφερεν εαυτόν άμωμον εις τον Θεόν, θέλει καθαρίσει την συνείδησίν σας από νεκρών έργων εις το να λατρεύητε τον ζώντα Θεόν;
\par 15 Και διά τούτο είναι μεσίτης διαθήκης καινής, ίνα διά του θανάτου, όστις έγεινε προς απολύτρωσιν των επί της πρώτης διαθήκης παραβάσεων, λάβωσιν οι κεκλημένοι την επαγγελίαν της αιωνίου κληρονομίας.
\par 16 Διότι όπου είναι διαθήκη, ανάγκη να υπάρχη θάνατος εκείνου, όστις έκαμε την διαθήκην·
\par 17 διότι η διαθήκη επί τεθνεώτων είναι βεβαία, επειδή ποτέ δεν ισχύει, ενόσω ζη ο διαθέτης.
\par 18 Όθεν ουδέ η πρώτη δεν ήτο εγκαινιασμένη χωρίς αίματος·
\par 19 διότι αφού πάσα εντολή του νόμου ελαλήθη υπό του Μωϋσέως προς πάντα τον λαόν, λαβών το αίμα των μόσχων και των τράγων με ύδωρ και μαλλίον κόκκινον και ύσσωπον, ερράντισε και αυτό το βιβλίον και πάντα τον λαόν,
\par 20 λέγων· Τούτο είναι το αίμα της διαθήκης, την οποίαν διέταξεν εις εσάς ο Θεός·
\par 21 και την σκηνήν δε και πάντα τα σκεύη της υπηρεσίας με το αίμα ομοίως ερράντισε.
\par 22 Και σχεδόν με αίμα καθαρίζονται πάντα κατά τον νόμον, και χωρίς χύσεως αίματος δεν γίνεται άφεσις.
\par 23 Ανάγκη λοιπόν ήτο οι μεν τύποι των επουρανίων να καθαρίζωνται διά τούτων, αυτά όμως τα επουράνια με θυσίας ανωτέρας παρά ταύτας.
\par 24 Διότι ο Χριστός δεν εισήλθεν εις χειροποίητα άγια, αντίτυπα των αληθινών, αλλ' εις αυτόν τον ουρανόν, διά να εμφανισθή τώρα ενώπιον του Θεού υπέρ ημών·
\par 25 ουδέ διά να προσφέρη πολλάκις εαυτόν, καθώς ο αρχιερεύς εισέρχεται εις τα άγια κατ' ενιαυτόν με ξένον αίμα·
\par 26 διότι έπρεπε τότε πολλάκις να πάθη από καταβολής κόσμου· τώρα δε άπαξ εις το τέλος των αιώνων εφανερώθη, διά να αθετήση την αμαρτίαν διά της θυσίας εαυτού.
\par 27 Και καθώς είναι αποφασισμένον εις τους ανθρώπους άπαξ να αποθάνωσι, μετά δε τούτο είναι κρίσις,
\par 28 ούτω και ο Χριστός, άπαξ προσφερθείς διά να σηκώση τας αμαρτίας πολλών, θέλει φανή εκ δευτέρου χωρίς αμαρτίας εις τους προσμένοντας αυτόν διά σωτηρίαν.

\chapter{10}

\par 1 Διότι ο νόμος, έχων σκιάν των μελλόντων αγαθών, ουχί αυτήν την εικόνα των πραγμάτων, δεν δύναταί ποτέ διά των αυτών θυσιών, τας οποίας προσφέρουσι κατ' ενιαυτόν πάντοτε να τελειοποιήση τους προσερχομένους·
\par 2 επειδή τότε δεν ήθελον παύσει να προσφέρωνται, διότι οι λατρευταί άπαξ καθαρισθέντες, δεν ήθελον έχει πλέον ουδεμίαν συνείδησιν αμαρτιών·
\par 3 αλλ' εν αυταίς γίνεται κατ' ενιαυτόν ανάμνησις αμαρτιών·
\par 4 διότι αδύνατον είναι αίμα ταύρων και τράγων να αφαιρή αμαρτίας.
\par 5 Διά τούτο εισερχόμενος εις τον κόσμον, λέγει· Θυσίαν και προσφοράν δεν ηθέλησας, αλλ' ητοίμασας εις εμέ σώμα·
\par 6 εις ολοκαυτώματα και προσφοράς περί αμαρτίας δεν ευηρεστήθης·
\par 7 τότε είπον· Ιδού, έρχομαι, εν τω τόμω του βιβλίου είναι γεγραμμένον περί εμού, διά να κάμω, ω Θεέ, το θέλημά σου.
\par 8 Αφού είπεν ανωτέρω ότι θυσίαν και προσφοράν και ολοκαυτώματα και προσφοράς περί αμαρτίας δεν ηθέλησας ουδέ ευηρεστήθης εις αυτάς, αίτινες προσφέρονται κατά τον νόμον,
\par 9 τότε είπεν· Ιδού, έρχομαι διά να κάμω, ω Θεέ, το θέλημά σου. Αναιρεί το πρώτον, διά να συστήση το δεύτερον.
\par 10 Με το οποίον θέλημα είμεθα ηγιασμένοι διά της προσφοράς του σώματος του Ιησού Χριστού άπαξ γενομένης.
\par 11 Και πας μεν ιερεύς ίσταται καθ' ημέραν λειτουργών και τας αυτάς πολλάκις προσφέρων θυσίας, αίτινες ποτέ δεν δύνανται να αφαιρέσωσιν αμαρτίας·
\par 12 αλλ' αυτός αφού προσέφερε μίαν θυσίαν υπέρ αμαρτιών, εκάθησε διαπαντός εν δεξιά του Θεού,
\par 13 προσμένων του λοιπού εωσού τεθώσιν οι εχθροί αυτού υποπόδιον των ποδών αυτού.
\par 14 Διότι με μίαν προσφοράν ετελειοποίησε διά παντός τους αγιαζομένους.
\par 15 Μαρτυρεί δε εις ημάς και το Πνεύμα το Αγιον· διότι αφού είπε πρότερον,
\par 16 Αύτη είναι η διαθήκη, την οποίαν θέλω κάμει προς αυτούς μετά τας ημέρας εκείνας, λέγει ο Κύριος· Θέλω δώσει τους νόμους μου εις τας καρδίας αυτών και θέλω γράψει αυτούς επί των διανοιών αυτών, προσθέτει,
\par 17 Και τας αμαρτίας αυτών και τας ανομίας αυτών δεν θέλω ενθυμείσθαι πλέον.
\par 18 Όπου δε είναι άφεσις τούτων, δεν είναι πλέον προσφορά περί αμαρτίας.
\par 19 Έχοντες λοιπόν, αδελφοί, παρρησίαν να εισέλθωμεν εις τα άγια διά του αίματος του Ιησού,
\par 20 διά νέας και ζώσης οδού, την οποίαν καθιέρωσεν εις ημάς διά του καταπετάσματος, τουτέστι της σαρκός αυτού,
\par 21 και έχοντες ιερέα μέγαν επί τον οίκον του Θεού,
\par 22 ας πλησιάζωμεν μετά αληθινής καρδίας εν πληροφορία πίστεως, έχοντες τας καρδίας ημών κεκαθαρμένας από συνειδήσεως πονηράς και λελουμένοι το σώμα με ύδωρ καθαρόν·
\par 23 ας κρατώμεν την ομολογίαν της ελπίδος ασάλευτον· διότι πιστός ο υποσχεθείς·
\par 24 και ας φροντίζωμεν περί αλλήλων, παρακινούντες εις αγάπην και καλά έργα,
\par 25 μη αφίνοντες το να συνερχώμεθα ομού, καθώς είναι συνήθεια εις τινάς, αλλά προτρέποντες αλλήλους, και τοσούτω μάλλον, όσον βλέπετε πλησιάζουσαν την ημέραν.
\par 26 Διότι εάν ημείς αμαρτάνωμεν εκουσίως, αφού ελάβομεν την γνώσιν της αληθείας, δεν απολείπεται πλέον θυσία περί αμαρτιών,
\par 27 αλλά φοβερά τις απεκδοχή κρίσεως και έξαψις πυρός, το οποίον μέλλει να κατατρώγη τους εναντίους.
\par 28 Εάν τις αθετήση τον νόμον του Μωϋσέως, επί δύο ή τριών μαρτύρων αποθνήσκει χωρίς έλεος·
\par 29 πόσον στοχάζεσθε χειροτέρας τιμωρίας θέλει κριθή άξιος ο καταπατήσας τον Υιόν του Θεού και νομίσας κοινόν το αίμα της διαθήκης, με το οποίον ηγιάσθη, και υβρίσας το Πνεύμα της χάριτος;
\par 30 Διότι εξεύρομεν τον ειπόντα· Εις εμέ ανήκει η εκδίκησις, εγώ θέλω κάμει ανταπόδοσιν, λέγει Κύριος· και πάλιν· Ο Κύριος θέλει κρίνει τον λαόν αυτού.
\par 31 Φοβερόν είναι το να πέση τις εις χείρας Θεού ζώντος.
\par 32 Αναφέρετε δε εις την μνήμην σας τας προτέρας ημέρας, εν αις αφού εφωτίσθητε, υπεμείνατε μέγαν αγώνα παθημάτων·
\par 33 ποτέ μεν θεατριζόμενοι με ονειδισμούς και θλίψεις, ποτέ δε γινόμενοι κοινωνοί των τα τοιαύτα παθόντων.
\par 34 Διότι εδείξατε συμπάθειαν εις τα δεσμά μου και εδέχθητε μετά χαράς την αρπαγήν των υπαρχόντων σας, εξεύροντες ότι έχετε εις εαυτούς περιουσίαν εν ουρανοίς καλητέραν και διαμένουσαν.
\par 35 Μη αποβάλητε λοιπόν την παρρησίαν σας, ήτις έχει μισθαποδοσίαν μεγάλην.
\par 36 Διότι έχετε χρείαν υπομονής, διά να κάμητε το θέλημα του Θεού και να λάβητε την επαγγελίαν.
\par 37 Διότι έτι ολίγον καιρόν, και θέλει ελθεί ο ερχόμενος και δεν θέλει βραδύνει.
\par 38 Ο δε δίκαιος θέλει ζήσει εκ πίστεως· και εάν τις συρθή οπίσω, η ψυχή μου δεν ευαρεστείται εις αυτόν.
\par 39 Ημείς όμως δεν είμεθα εκ των συρομένων οπίσω προς απώλειαν, αλλ' εκ των πιστευόντων προς σωτηρίαν της ψυχής.

\chapter{11}

\par 1 Είναι δε η πίστις ελπιζομένων πεποίθησις, βεβαίωσις πραγμάτων μη βλεπομένων.
\par 2 Διότι διά ταύτης έλαβον καλήν μαρτυρίαν οι πρεσβύτεροι.
\par 3 Διά πίστεως εννοούμεν ότι οι αιώνες εκτίθησαν με τον λόγον του Θεού, ώστε τα βλεπόμενα δεν έγειναν εκ φαινομένων.
\par 4 Διά πίστεως ο Άβελ προσέφερε προς τον Θεόν καλητέραν θυσίαν παρά τον Κάϊν, διά της οποίας εμαρτυρήθη ότι ήτο δίκαιος, επειδή ο Θεός έδωκε μαρτυρίαν περί των δώρων αυτού, και δι' αυτής καίτοι αποθανών έτι λαλεί.
\par 5 Διά πίστεως μετετέθη ο Ενώχ, διά να μη ίδη θάνατον, και δεν ευρίσκετο, διότι μετέθεσεν αυτόν ο Θεός· επειδή προ της μεταθέσεως αυτού εμαρτυρήθη ότι ευηρέστησεν εις τον Θεόν·
\par 6 χωρίς δε πίστεως αδύνατον είναι να ευαρεστήση τις εις αυτόν· διότι ο προσερχόμενος εις τον Θεόν πρέπει να πιστεύη ότι είναι και γίνεται μισθαποδότης εις τους εκζητούντας αυτόν.
\par 7 Διά πίστεως ο Νώε, ειδοποιηθείς θεόθεν περί των μη βλεπομένων έτι, εφοβήθη και κατεσκεύασε κιβωτόν προς σωτηρίαν του οίκου αυτού, δι' ης κατέκρινε τον κόσμον και έγεινε κληρονόμος της διά πίστεως δικαιοσύνης.
\par 8 Διά πίστεως υπήκουσεν ο Αβραάμ, ότε εκαλείτο να εξέλθη εις τον τόπον τον οποίον έμελλε να λάβη εις κληρονομίαν, και εξήλθε μη εξεύρων που υπάγει.
\par 9 Διά πίστεως παρώκησεν εις την γην της επαγγελίας ως ξένην, κατοικήσας εν σκηναίς μετά Ισαάκ και Ιακώβ των συγκληρονόμων της αυτής επαγγελίας·
\par 10 διότι περιέμενε την πόλιν την έχουσαν τα θεμέλια, της οποίας τεχνίτης και δημιουργός είναι ο Θεός.
\par 11 Διά πίστεως και αυτή η Σάρρα έλαβε δύναμιν εις το να συλλάβη σπέρμα και παρά καιρόν ηλικίας εγέννησεν, επειδή εστοχάσθη πιστόν τον υποσχεθέντα.
\par 12 Διά τούτο και εξ ενός, μάλιστα νενεκρωμένου, εγεννήθησαν καθώς τα άστρα του ουρανού κατά το πλήθος, και ως η άμμος η παρά το χείλος της θαλάσσης, ήτις δεν δύναται να αριθμηθή.
\par 13 Εν πίστει απέθανον ούτοι πάντες, μη λαβόντες τας επαγγελίας, αλλά μακρόθεν ιδόντες αυτάς και πεισθέντες και εγκολπωθέντες και ομολογήσαντες ότι είναι ξένοι και παρεπίδημοι επί της γης.
\par 14 Διότι οι λέγοντες τοιαύτα δεικνύουσιν ότι ζητούσι πατρίδα.
\par 15 Και εάν μεν ενεθυμούντο εκείνην, εξ ης εξήλθον, ήθελον ευρεί καιρόν να επιστρέψωσι·
\par 16 τώρα όμως επιθυμούσι καλητέραν, τουτέστιν επουράνιον. Διά τούτο ο Θεός δεν επαισχύνεται αυτούς να λέγηται Θεός αυτών, διότι ητοίμασε δι' αυτούς πόλιν.
\par 17 Διά πίστεως ο Αβραάμ, ότε εδοκιμάζετο, προσέφερε τον Ισαάκ, και τον μονογενή αυτού προσέφερεν εκείνος όστις ανεδέχθη τας επαγγελίας,
\par 18 προς τον οποίον ελαλήθη ότι εν Ισαάκ θέλει κληθή εις σε σπέρμα,
\par 19 συλλογισθείς ότι ο Θεός δύναται και εκ νεκρών να ανεγείρη· εξ ων και έλαβεν αυτόν οπίσω παραβολικώς.
\par 20 Διά πίστεως ο Ισαάκ ηυλόγησε τον Ιακώβ και τον Ησαύ περί των μελλόντων.
\par 21 Διά πίστεως ο Ιακώβ αποθνήσκων ηυλόγησεν έκαστον των υιών του Ιωσήφ και προσεκύνησεν επιστηριζόμενος επί το άκρον της ράβδου αυτού.
\par 22 Διά πίστεως ο Ιωσήφ αποθνήσκων προανήγγειλε περί της εξόδου των υιών Ισραήλ και παρήγγειλε περί των οστέων αυτού.
\par 23 Διά πίστεως ο Μωϋσής, αφού εγεννήθη, εκρύφθη τρεις μήνας υπό των γονέων αυτού, διότι είδον κεχαριτωμένον το παιδίον, και δεν εφοβήθησαν το διάταγμα του βασιλέως.
\par 24 Διά πίστεως ο Μωϋσής, αφού εμεγάλωσεν, ηρνήθη να λέγηται υιός της θυγατρός του Φαραώ,
\par 25 προκρίνας μάλλον να κακουχήται με τον λαόν του Θεού παρά να έχη πρόσκαιρον απόλαυσιν αμαρτίας,
\par 26 κρίνας τον υπέρ του Χριστού ονειδισμόν μεγαλήτερον πλούτον παρά τους εν Αιγύπτω θησαυρούς· διότι απέβλεπεν εις την μισθαποδοσίαν.
\par 27 Διά πίστεως αφήκε την Αίγυπτον, μη φοβηθείς τον θυμόν του βασιλέως· διότι ως βλέπων τον αόρατον ενεκαρτέρησε.
\par 28 Διά πίστεως έκαμε το πάσχα και την πρόσχυσιν του αίματος, διά να μη εγγίση αυτούς ο εξολοθρεύων τα πρωτότοκα.
\par 29 Διά πίστεως διέβησαν την Ερυθράν θάλασσαν ως διά ξηράς, την οποίαν δοκιμάσαντες οι Αιγύπτιοι κατεποντίσθησαν.
\par 30 Διά πίστεως έπεσον τα τείχη της Ιεριχώ, αφού εκυκλώθησαν επί επτά ημέρας.
\par 31 Διά πίστεως η πόρνη Ραάβ δεν συναπωλέσθη με τους απειθήσαντας, δεχθείσα τους κατασκόπους με ειρήνην.
\par 32 Και τι έτι να λέγω; Διότι θέλει με λείψει ο καιρός διηγούμενον περί Γεδεών, Βαράκ τε και Σαμψών και Ιεφθάε, Δαβίδ τε και Σαμουήλ και των προφητών,
\par 33 οίτινες διά της πίστεως κατεπολέμησαν βασιλείας, ειργάσθησαν δικαιοσύνην, επέτυχον τας επαγγελίας, έφραξαν στόματα λεόντων,
\par 34 έσβεσαν δύναμιν πυρός, έφυγον στόματα μαχαίρας, ενεδυναμώθησαν από ασθενείας, έγειναν ισχυροί εν πολέμω, έτρεψαν εις φυγήν στρατεύματα αλλοτρίων.
\par 35 Έλαβον γυναίκες τους νεκρούς αυτών αναστηθέντας· άλλοι δε εβασανίσθησαν, μη δεχθέντες την απολύτρωσιν, διά να αξιωθώσι καλητέρας αναστάσεως·
\par 36 άλλοι δε εδοκίμασαν εμπαιγμούς και μάστιγας, έτι δε και δεσμά και φυλακήν·
\par 37 ελιθοβολήθησαν, επριονίσθησαν, επειράσθησαν, με σφαγήν μαχαίρας απέθανον, περιεπλανήθησαν με δέρματα προβάτων, με δέρματα αιγών· υστερούμενοι, θλιβόμενοι, κακουχούμενοι,
\par 38 των οποίων δεν ήτο άξιος ο κόσμος, πλανώμενοι εν ερημίαις και όρεσι και σπηλαίοις και ταις τρύπαις της γης.
\par 39 Και ούτοι πάντες αν και έλαβον καλήν μαρτυρίαν διά της πίστεως, δεν απήλαυσαν την επαγγελίαν,
\par 40 διότι ο Θεός προέβλεψε καλύτερόν τι περί ημών, διά να μη λάβωσι την τελειότητα χωρίς ημών.

\chapter{12}

\par 1 Λοιπόν και ημείς, περικυκλωμένοι όντες υπό τοσούτου νέφους μαρτύρων, ας απορρίψωμεν παν βάρος και την ευκόλως εμπεριπλέκουσαν ημάς αμαρτίαν, και ας τρέχωμεν μεθ' υπομονής τον προκείμενον εις ημάς αγώνα,
\par 2 αποβλέποντες εις τον Ιησούν, τον αρχηγόν και τελειωτήν της πίστεως, όστις υπέρ της χαράς της προκειμένης εις αυτόν υπέφερε σταυρόν, καταφρονήσας την αισχύνην, και εκάθησεν εν δεξιά του θρόνου του Θεού.
\par 3 Διότι συλλογίσθητε τον υπομείναντα υπό των αμαρτωλών τοιαύτην αντιλογίαν εις εαυτόν, διά να μη αποκάμητε χαυνούμενοι κατά τας ψυχάς σας.
\par 4 Δεν αντεστάθητε έτι μέχρις αίματος αγωνιζόμενοι κατά της αμαρτίας,
\par 5 και ελησμονήσατε την νουθεσίαν, ήτις λαλεί προς εσάς ως προς υιούς, λέγουσα· Υιέ μου, μη καταφρονής την παιδείαν του Κυρίου, μηδέ αθυμής ελεγχόμενος υπ' αυτού.
\par 6 Διότι όντινα αγαπά Κύριος παιδεύει και μαστιγόνει πάντα υιόν, τον οποίον παραδέχεται.
\par 7 Εάν υπομένητε την παιδείαν, ο Θεός φέρεται προς εσάς ως προς υιούς· διότι τις υιός είναι, τον οποίον δεν παιδεύει ο πατήρ;
\par 8 Εάν όμως ήσθε χωρίς παιδείαν, της οποίας έγειναν μέτοχοι πάντες, άρα είσθε νόθοι και ουχί υιοί,
\par 9 έπειτα τους μεν κατά σάρκα πατέρας ημών είχομεν παιδευτάς και εσεβόμεθα αυτούς· δεν θέλομεν υποταχθή πολλώ μάλλον εις τον Πατέρα των πνευμάτων και ζήσει;
\par 10 Διότι εκείνοι μεν προς ολίγας ημέρας επαίδευον ημάς κατά την αρέσκειαν αυτών, ο δε προς το συμφέρον ημών, διά να γείνωμεν μέτοχοι της αγιότητος αυτού.
\par 11 Πάσα δε παιδεία προς μεν το παρόν δεν φαίνεται ότι είναι πρόξενος χαράς, αλλά λύπης, ύστερον όμως αποδίδει εις τους γυμνασθέντας δι' αυτής καρπόν ειρηνικόν δικαιοσύνης.
\par 12 Διά τούτο ανορθώσατε τας κεχαυνωμένας χείρας και τα παραλελυμένα γόνατα,
\par 13 και κάμετε εις τους πόδας σας ευθείας οδούς, διά να μη εκτραπή το χωλόν, αλλά μάλλον να θεραπευθή.
\par 14 Ζητείτε ειρήνην μετά πάντων, και τον αγιασμόν, χωρίς του οποίου ουδείς θέλει ιδεί τον Κύριον,
\par 15 παρατηρούντες μήπως υστερήταί τις από της χάριτος του Θεού, μήπως ρίζα τις πικρίας αναφύουσα φέρη ενόχλησιν και διά ταύτης μιανθώσι πολλοί,
\par 16 μήπως ήναι τις πόρνος ή βέβηλος καθώς ο Ησαύ, όστις διά μίαν βρώσιν επώλησε τα πρωτοτόκια αυτού.
\par 17 Επειδή εξεύρετε ότι και μετέπειτα, θέλων να κληρονομήση την ευλογίαν, απεδοκιμάσθη, διότι δεν εύρε τόπον μετανοίας, αν και εξεζήτησεν αυτήν μετά δακρύων.
\par 18 Διότι δεν προσήλθετε εις όρος ψηλαφώμενον και καιόμενον με πυρ και εις ζόφον και σκότος και ανεμοστρόβιλον
\par 19 και εις σάλπιγγος ήχον και φωνήν λόγων, την οποίαν οι ακούσαντες παρεκάλεσαν να μη λαληθή πλέον προς αυτούς ο λόγος·
\par 20 διότι δεν υπέφερον το προσταττόμενον· Και ζώον εάν εγγίση το όρος, θέλει λιθοβοληθή ή με βέλη θέλει κατατοξευθή·
\par 21 και τόσον φοβερόν ήτο το φαινόμενον, ώστε ο Μωϋσής είπε· Κατάφοβος είμαι και έντρομος·
\par 22 αλλά προσήλθετε εις όρος Σιών και εις πόλιν Θεού ζώντος, την επουράνιον Ιερουσαλήμ, και εις μυριάδας αγγέλων,
\par 23 εις πανήγυριν και εκκλησίαν πρωτοτόκων καταγεγραμμένων εν τοις ουρανοίς, και εις Θεόν κριτήν πάντων, και εις πνεύματα δικαίων οίτινες έλαβον την τελειότητα,
\par 24 και εις νέας διαθήκης μεσίτην Ιησούν, και εις αίμα καθαρισμού το οποίον λαλεί καλήτερα παρά το του Άβελ.
\par 25 Προσέχετε μη καταφρονήσητε τον λαλούντα. Διότι αν εκείνοι δεν απέφυγον, καταφρονήσαντες τον λαλούντα προς αυτούς επί της γης, πολλώ μάλλον ημείς εάν αποστραφώμεν τον λαλούντα από των ουρανών·
\par 26 του οποίου η φωνή την γην εσάλευσε τότε, τώρα δε υπεσχέθη, λέγων· Έτι άπαξ εγώ σείω ουχί μόνον την γην, αλλά και τον ουρανόν.
\par 27 Το δε έτι άπαξ δηλοί των σαλευομένων την μετάθεσιν ως χειροποιήτων, διά να μείνωσι τα μη σαλευόμενα.
\par 28 Διά τούτο παραλαμβάνοντες βασιλείαν ασάλευτον, ας κρατώμεν την χάριν, διά της οποίας να λατρεύωμεν ευαρέστως τον Θεόν με σέβας και ευλάβειαν.
\par 29 Διότι ο Θεός ημών είναι πυρ καταναλίσκον.

\chapter{13}

\par 1 Η φιλαδελφία ας μένη.
\par 2 Την φιλοξενίαν μη λησμονείτε· επειδή διά ταύτης τινές εφιλοξένησαν αγγέλους μη γνωρίζοντες.
\par 3 Ενθυμείσθε τους δεσμίους ως συνδέσμιοι, τους ταλαιπωρουμένους ως όντες και σεις εν σώματι.
\par 4 Τίμιος έστω ο γάμος εις πάντας και η κοίτη αμίαντος· τους δε πόρνους και μοιχούς θέλει κρίνει ο Θεός.
\par 5 Ο τρόπος σας έστω αφιλάργυρος, αρκείσθε εις τα παρόντα, διότι αυτός είπε· Δεν θέλω σε αφήσει ουδέ σε εγκαταλείψει·
\par 6 ώστε ημείς θαρρούντες να λέγωμεν· Ο Κύριος βοηθός μου, και δεν θέλω φοβηθή· τι να μοι κάμη άνθρωπος;
\par 7 Ενθυμείσθε τους προεστώτάς σας, οίτινες ελάλησαν προς εσάς τον λόγον του Θεού, των οποίων μιμείσθε την πίστιν, έχοντες προ οφθαλμών το αποτέλεσμα του πολιτεύματος αυτών.
\par 8 Ο Ιησούς Χριστός είναι ο αυτός χθές και σήμερον και εις τους αιώνας.
\par 9 Μη πλανάσθε με διδαχάς ποικίλας και ξένας· διότι καλόν είναι με την χάριν να στερεόνηται η καρδία, ουχί με βρώματα, εις τα οποία όσοι περιεπάτησαν δεν ωφελήθησαν.
\par 10 Έχομεν θυσιαστήριον, εξ ου δεν έχουσιν εξουσίαν να φάγωσιν οι λατρεύοντες εις την σκηνήν.
\par 11 Διότι των ζώων, των οποίων το αίμα εισφέρεται εις τα άγια διά του αρχιερέως περί αμαρτίας, τούτων τα σώματα κατακαίονται έξω του στρατοπέδου.
\par 12 Όθεν και ο Ιησούς, διά να αγιάση τον λαόν διά του ιδίου αυτού αίματος, έξω της πύλης έπαθεν.
\par 13 Ας εξερχώμεθα λοιπόν προς αυτόν έξω του στρατοπέδου, τον ονειδισμόν αυτού φέροντες·
\par 14 διότι δεν έχομεν εδώ πόλιν διαμένουσαν, αλλά την μέλλουσαν επιζητούμεν.
\par 15 Δι' αυτού λοιπόν ας αναφέρωμεν πάντοτε εις τον Θεόν θυσίαν αινέσεως, τουτέστι καρπόν χειλέων ομολογούντων το όνομα αυτού.
\par 16 Την δε αγαθοποιΐαν και το μεταδοτικόν μη λησμονείτε, διότι εις τοιαύτας θυσίας ευαρεστείται ο Θεός.
\par 17 Πείθεσθε εις τους προεστώτάς σας και υπακούετε· διότι αυτοί αγρυπνούσιν υπέρ των ψυχών σας ως μέλλοντες να αποδώσωσι λόγον· διά να κάμνωσι τούτο μετά χαράς και μη στενάζοντες· διότι τούτο δεν σας ωφελεί.
\par 18 Προσεύχεσθε περί ημών· διότι είμεθα πεπεισμένοι ότι έχομεν καλήν συνείδησιν, θέλοντες να πολιτευώμεθα κατά πάντα καλώς.
\par 19 Περισσότερον δε παρακαλώ να κάμητε τούτο, διά να αποκατασταθώ εις εσάς ταχύτερα.
\par 20 Ο δε Θεός της ειρήνης, όστις ανεβίβασεν εκ των νεκρών τον μέγαν ποιμένα των προβάτων διά του αίματος της αιωνίου διαθήκης, τον Κύριον ημών Ιησούν,
\par 21 είθε να σας κάμη τελείους εις παν έργον αγαθόν, διά να εκτελήτε το θέλημα αυτού, ενεργών εν υμίν το ευάρεστον ενώπιον αυτού διά του Ιησού Χριστού, εις τον οποίον είη η δόξα εις τους αιώνας των αιώνων· αμήν.
\par 22 Σας παρακαλώ δε, αδελφοί, υποφέρετε τον λόγον της νουθεσίας· διότι εν συντομία σας έγραψα.
\par 23 Εξεύρετε ότι ο αδελφός Τιμόθεος απελύθη της φυλακής, μετά του οποίου, εάν έλθη ταχύτερα, θέλω σας ιδεί.
\par 24 Ασπάσθητε πάντας τους προεστώτάς σας και πάντας τους αγίους. Σας ασπάζονται οι από της Ιταλίας.
\par 25 Η χάρις είη μετά πάντων υμών· αμήν.


\end{document}