\begin{document}

\title{Επιστολή Πέτρου Α'}


\chapter{1}

\par 1 Πέτρος, απόστολος Ιησού Χριστού, προς τους παρεπιδήμους τους διεσπαρμένους εις Πόντον, Γαλατίαν, Καππαδοκίαν, Ασίαν και Βιθυνίαν,
\par 2 εκλεκτούς κατά πρόγνωσιν Θεού Πατρός, διά του αγιασμού του Πνεύματος, εις υπακοήν και ραντισμόν του αίματος του Ιησού Χριστού· πληθυνθείη χάρις και ειρήνη εις εσάς.
\par 3 Ευλογητός ο Θεός και Πατήρ του Κυρίου ημών Ιησού Χριστού, όστις κατά το πολύ έλεος αυτού ανεγέννησεν ημάς εις ελπίδα ζώσαν διά της αναστάσεως του Ιησού Χριστού εκ νεκρών,
\par 4 εις κληρονομίαν άφθαρτον και αμίαντον και αμάραντον, πεφυλαγμένην εν τοις ουρανοίς δι' ημάς,
\par 5 οίτινες με την δύναμιν του Θεού φυλαττόμεθα διά της πίστεως, εις σωτηρίαν ετοίμην να αποκαλυφθή εν τω εσχάτω καιρώ·
\par 6 διά το οποίον αγαλλιάσθε, αν και τώρα ολίγον, εάν χρειασθή, λυπηθήτε εν διαφόροις πειρασμοίς,
\par 7 ίνα η δοκιμή της πίστεώς σας, πολύ τιμιωτέρα ούσα παρά το χρυσίον το φθειρόμενον διά πυρός δε δοκιμαζόμενον, ευρεθή εις έπαινον και τιμήν και δόξαν όταν φανερωθή ο Ιησούς Χριστός,
\par 8 τον οποίον αν και δεν είδετε αγαπάτε, εις τον οποίον, αν και τώρα δεν βλέπητε αυτόν, πιστεύοντες όμως αγαλλιάσθε με χαράν ανεκλάλητον και ένδοξον,
\par 9 απολαμβάνοντες το τέλος της πίστεώς σας, την σωτηρίαν των ψυχών.
\par 10 Περί της οποίας σωτηρίας εξεζήτησαν και εξηρεύνησαν οι προφήται οι προφητεύσαντες περί της χάριτος, ήτις έμελλε να έλθη εις εσάς·
\par 11 ερευνώντες εις τίνα ή ποίον καιρόν εφανέρονε το εν αυτοίς Πνεύμα του Χριστού, ότε προεμαρτύρει τα πάθη του Χριστού και τας μετά ταύτα δόξας·
\par 12 εις τους οποίους απεκαλύφθη ότι ουχί δι' εαυτούς, αλλά δι' ημάς υπηρέτουν αυτά, τα οποία τώρα ανηγγέλθησαν προς εσάς διά των κηρυξάντων το ευαγγέλιον εις εσάς εν Πνεύματι Αγίω τω αποσταλέντι απ' ουρανού, εις τα οποία επιθυμούσιν οι άγγελοι να παρακύψωσι.
\par 13 Διά τούτο αναζωσθέντες τας οσφύας της διανοίας σας, εγκρατεύεσθε και έχετε τελείαν ελπίδα εις την χάριν την ερχομένην εις εσάς, όταν αποκαλυφθή ο Ιησούς Χριστός,
\par 14 ως τέκνα υπακοής μη συμμορφούμενοι με τας προτέρας επιθυμίας, τας οποίας είχετε εν αγνοία υμών,
\par 15 αλλά καθώς είναι άγιος εκείνος, όστις σας εκάλεσεν, ούτω και σεις γίνεσθε άγιοι εν πάση διαγωγή·
\par 16 διότι είναι γεγραμμένον· Άγιοι γίνεσθε, διότι εγώ είμαι άγιος.
\par 17 Και εάν επικαλήσθε Πατέρα τον κρίνοντα απροσωπολήπτως κατά το έργον εκάστου, διάγετε μετά φόβου τον καιρόν της παροικίας σας,
\par 18 εξεύροντες ότι δεν ελυτρώθητε από της ματαίας πατροπαραδότου διαγωγής υμών διά φθαρτών, αργυρίου ή χρυσίου,
\par 19 αλλά διά του τιμίου αίματος του Χριστού, ως αμνού αμώμου και ασπίλου,
\par 20 όστις ήτο μεν προωρισμένος προ καταβολής κόσμου, εφανερώθη δε εν τοις εσχάτοις καιροίς διά σας,
\par 21 τους πιστεύοντας δι' αυτού εις τον Θεόν, τον αναστήσαντα αυτόν εκ νεκρών και δόντα εις αυτόν δόξαν, ώστε η πίστις σας και η ελπίς να ήναι εις τον Θεόν.
\par 22 Καθαρίσαντες λοιπόν τας ψυχάς σας με την υπακοήν της αληθείας διά του Πνεύματος προς φιλαδελφίαν ανυπόκριτον, αγαπήσατε ενθέρμως αλλήλους εκ καθαράς καρδίας,
\par 23 επειδή ανεγεννήθητε ουχί εκ φθαρτού σπέρματος, αλλά αφθάρτου, διά του λόγου του Θεού του ζώντος και μένοντος εις τον αιώνα.
\par 24 Διότι Πάσα σαρξ είναι ως χόρτος, και πάσα δόξα ανθρώπου ως άνθος χόρτου. Εξηράνθη ο χόρτος, και το άνθος αυτού εξέπεσεν.
\par 25 Ο λόγος όμως του Κυρίου μένει εις τον αιώνα. Και ούτος είναι ο λόγος ο ευαγγελισθείς εις εσάς.

\chapter{2}

\par 1 Απορρίψαντες λοιπόν πάσαν κακίαν και πάντα δόλον και υποκρίσεις και φθόνους και πάσας καταλαλιάς,
\par 2 επιποθήσατε ως νεογέννητα βρέφη το λογικόν άδολον γάλα, διά να αυξηθήτε δι' αυτού,
\par 3 επειδή εγεύθητε ότι αγαθός ο Κύριος.
\par 4 Εις τον οποίον προσερχόμενοι, ως εις λίθον ζώντα, υπό μεν των ανθρώπων αποδεδοκιμασμένον, παρά δε τω Θεώ εκλεκτόν, έντιμον,
\par 5 και σεις, ως λίθοι ζώντες, οικοδομείσθε οίκος πνευματικός, ιεράτευμα άγιον, διά να προσφέρητε πνευματικάς θυσίας ευπροσδέκτους εις τον Θεόν διά Ιησού Χριστού·
\par 6 διά τούτο και περιέχεται εν τη γραφή· Ιδού, θέτω εν Σιών λίθον ακρογωνιαίον, εκλεκτόν, έντιμον, και ο πιστεύων επ' αυτόν δεν θέλει καταισχυνθή.
\par 7 Εις εσάς λοιπόν τους πιστεύοντας είναι η τιμή, εις δε τους απειθούντας ο λίθος, τον οποίον απεδοκίμασαν οι οικοδομούντες, ούτος έγεινε κεφαλή γωνίας
\par 8 και λίθος προσκόμματος και πέτρα σκανδάλου· οίτινες προσκόπτουσιν εις τον λόγον, όντες απειθείς, εις το οποίον και ήσαν προσδιωρισμένοι·
\par 9 σεις όμως είσθε γένος εκλεκτόν, βασίλειον ιεράτευμα, έθνος άγιον, λαός τον οποίον απέκτησεν ο Θεός, διά να εξαγγείλητε τας αρετάς εκείνου, όστις σας εκάλεσεν εκ του σκότους εις το θαυμαστόν αυτού φώς·
\par 10 οι ποτέ μη όντες λαός, τώρα δε λαός του Θεού, οι ποτέ μη ηλεημένοι, τώρα δε ελεηθέντες.
\par 11 Αγαπητοί, σας παρακαλώ ως ξένους και παρεπιδήμους, να απέχητε από των σαρκικών επιθυμιών, αίτινες στρατεύονται κατά της ψυχής,
\par 12 να έχητε καλήν την διαγωγήν σας μεταξύ των εθνών, ίνα ενώ σας καταλαλούσιν ως κακοποιούς, εκ των καλών έργων, όταν ίδωσιν αυτά, δοξάσωσι τον Θεόν εν τη ημέρα της επισκέψεως.
\par 13 Υποτάχθητε λοιπόν εις πάσαν ανθρωπίνην διάταξιν διά τον Κύριον· είτε εις βασιλέα, ως υπερέχοντα,
\par 14 είτε εις ηγεμόνας, ως δι' αυτού πεμπομένους εις εκδίκησιν μεν κακοποιών, έπαινον δε αγαθοποιών·
\par 15 διότι ούτως είναι το θέλημα του Θεού, αγαθοποιούντες να αποστομόνητε την αγνωσίαν των αφρόνων ανθρώπων·
\par 16 ως ελεύθεροι, και μη ως έχοντες την ελευθερίαν επικάλυμμα της κακίας, αλλ' ως δούλοι του Θεού.
\par 17 Πάντας τιμήσατε, την αδελφότητα αγαπάτε, τον Θεόν φοβείσθε, τον βασιλέα τιμάτε.
\par 18 Οι οικέται υποτάσσεσθε εν παντί φόβω εις τους κυρίους σας, ου μόνον εις τους αγαθούς και επιεικείς, αλλά και εις τους διεστραμμένους.
\par 19 Διότι τούτο είναι χάρις, το να υποφέρη τις λύπας διά την εις τον Θεόν συνείδησιν, πάσχων αδίκως.
\par 20 Διότι ποία δόξα είναι, εάν αμαρτάνοντες και ραπιζόμενοι υπομένητε; εάν όμως αγαθοποιούντες και πάσχοντες υπομένητε, τούτο είναι χάρις παρά τω Θεώ.
\par 21 Διότι εις τούτο προσεκλήθητε, επειδή και ο Χριστός έπαθεν υπέρ υμών, αφίνων παράδειγμα εις υμάς διά να ακολουθήσητε τα ίχνη αυτού·
\par 22 όστις αμαρτίαν δεν έκαμεν, ουδέ ευρέθη δόλος εν τω στόματι αυτού.
\par 23 Όστις λοιδορούμενος δεν αντελοιδόρει, πάσχων δεν ηπείλει, αλλά παρέδιδεν εαυτόν εις τον κρίνοντα δικαίως·
\par 24 όστις τας αμαρτίας ημών αυτός εβάστασεν εν τω σώματι αυτού επί του ξύλου, διά να ζήσωμεν εν τη δικαιοσύνη, αποθανόντες κατά τας αμαρτίας· με του οποίου την πληγήν ιατρεύθητε.
\par 25 Διότι υπήρχετε ως πρόβατα πλανώμενα, αλλά τώρα επεστράφητε εις τον ποιμένα και επίσκοπον των ψυχών σας.

\chapter{3}

\par 1 Ομοίως αι γυναίκες, υποτάσσεσθε εις τους άνδρας υμών, ίνα και εάν τινές απειθώσιν εις τον λόγον, κερδηθώσιν άνευ του λόγου διά της διαγωγής των γυναικών,
\par 2 αφού ίδωσι την μετά φόβου καθαράν διαγωγήν σας.
\par 3 Των οποίων ο στολισμός ας ήναι ουχί ο εξωτερικός, ο του πλέγματος των τριχών και της περιθέσεως των χρυσίων ή της ενδύσεως των ιματίων,
\par 4 αλλ' ο κρυπτός άνθρωπος της καρδίας, κεκοσμημένος με την αφθαρσίαν του πράου και ησυχίου πνεύματος, το οποίον ενώπιον του Θεού είναι πολύτιμον.
\par 5 Διότι ούτω ποτέ και αι άγιαι γυναίκες αι ελπίζουσαι επί τον Θεόν εστόλιζον εαυτάς, υποτασσόμεναι εις τους άνδρας αυτών,
\par 6 καθώς η Σάρρα υπήκουσεν εις τον Αβραάμ, καλούσα αυτόν κύριον· της οποίας σεις εγεννήθητε τέκνα, αγαθοποιούσαι και μη φοβούμεναι μηδεμίαν πτόησιν.
\par 7 Οι άνδρες ομοίως, συνοικείτε με τας γυναίκάς σας εν φρονήσει, αποδίδοντες τιμήν εις το γυναικείον γένος ως εις σκεύος ασθενέστερον, και ως εις συγκληρονόμους της χάριτος της ζωής, διά να μη εμποδίζωνται αι προσευχαί σας.
\par 8 Τελευταίον δε, γίνεσθε πάντες ομόφρονες, συμπαθείς, φιλάδελφοι, εύσπλαγχνοι, φιλόφρονες,
\par 9 μη αποδίδοντες κακόν αντί κακού ή λοιδορίαν αντί λοιδορίας, αλλά το εναντίον ευλογούντες, επειδή εξεύρετε ότι εις τούτο προσεκλήθητε, διά να κληρονομήσητε ευλογίαν.
\par 10 Διότι Όστις θέλει να αγαπά την ζωήν και ίδη ημέρας αγαθάς ας παύση την γλώσσαν αυτού από κακού και τα χείλη αυτού από του να λαλώσι δόλον,
\par 11 ας εκκλίνη από κακού και ας πράξη αγαθόν, ας ζητήση ειρήνην και ας ακολουθήση αυτήν·
\par 12 διότι οι οφθαλμοί του Κυρίου είναι επί τους δικαίους και τα ώτα αυτού εις την δέησιν αυτών, το δε πρόσωπον του Κυρίου είναι κατά των πραττόντων κακά.
\par 13 Και τις θέλει σας κακοποιήσει, εάν γείνητε μιμηταί του αγαθού;
\par 14 Αλλ' εάν και πάσχητε διά την δικαιοσύνην, είσθε μακάριοι· τον δε φόβον αυτών μη φοβηθήτε μηδέ ταραχθήτε,
\par 15 αλλά αγιάσατε Κύριον τον Θεόν εν ταις καρδίαις υμών, και εστέ πάντοτε έτοιμοι εις απολογίαν μετά πραότητος και φόβου προς πάντα τον ζητούντα από σας λόγον περί της ελπίδος της εν υμίν,
\par 16 έχοντες συνείδησιν αγαθήν, ίνα, ενώ σας καταλαλώσιν ως κακοποιούς, καταισχυνθώσιν οι συκοφαντούντες την καλήν σας εν Χριστώ διαγωγήν.
\par 17 Διότι καλήτερον να πάσχητε, εάν ήναι ούτω το θέλημα του Θεού, αγαθοποιούντες παρά κακοποιούντες.
\par 18 Επειδή και ο Χριστός άπαξ έπαθε διά τας αμαρτίας, ο δίκαιος υπέρ των αδίκων, διά να φέρη ημάς προς τον Θεόν, θανατωθείς μεν κατά την σάρκα, ζωοποιηθείς δε διά του πνεύματος·
\par 19 διά του οποίου πορευθείς εκήρυξε και προς τα πνεύματα τα εν τη φυλακή,
\par 20 τα οποία ηπείθησάν ποτέ, ότε η μακροθυμία του Θεού επρόσμενέ ποτέ αυτούς εν ταις ημέραις του Νώε, ενώ κατεσκευάζετο η κιβωτός, εις ην ολίγαι, τουτέστιν οκτώ, ψυχαί διεσώθησαν δι' ύδατος.
\par 21 Του οποίου αντίτυπον ον το βάπτισμα, σώζει και ημάς την σήμερον, ουχί αποβολή της ακαθαρσίας της σαρκός, αλλά μαρτυρία της αγαθής συνειδήσεως εις Θεόν, διά της αναστάσεως του Ιησού Χριστού,
\par 22 όστις είναι εν δεξιά του Θεού πορευθείς εις τον ουρανόν, και εις ον υπετάχθησαν άγγελοι και εξουσίαι και δυνάμεις.

\chapter{4}

\par 1 Επειδή λοιπόν ο Χριστός έπαθεν υπέρ ημών κατά σάρκα, οπλίσθητε και σεις το αυτό φρόνημα, διότι ο παθών κατά σάρκα έπαυσεν από της αμαρτίας,
\par 2 διά να ζήσητε τον εν σαρκί επίλοιπον χρόνον, ουχί πλέον εν ταις επιθυμίαις των ανθρώπων, αλλ' εν τω θελήματι του Θεού.
\par 3 Διότι αρκετός είναι εις ημάς ο παρελθών καιρός του βίου, ότε επράξαμεν το θέλημα των εθνών, περιπατήσαντες εν ασελγείαις, επιθυμίαις, οινοποσίαις, κώμοις, συμποσίοις και αθεμίτοις ειδωλολατρείαις·
\par 4 και διά τούτο παραξενεύονται ότι σεις δεν συντρέχετε με αυτούς εις την αυτήν εκχείλισιν της ασωτίας, και σας βλασφημούσιν·
\par 5 οίτινες θέλουσιν αποδώσει λόγον εις εκείνον, όστις είναι έτοιμος να κρίνη ζώντας και νεκρούς.
\par 6 Επειδή διά τούτο εκηρύχθη το ευαγγέλιον και προς τους νεκρούς, διά να κριθώσι μεν κατά ανθρώπους εν σαρκί, να ζώσι δε κατά Θεόν εν πνεύματι.
\par 7 Πάντων δε το τέλος επλησίασε. Φρονίμως λοιπόν διάγετε και αγρυπνείτε εις τας προσευχάς·
\par 8 προ πάντων δε έχετε ένθερμον την εις αλλήλους αγάπην, διότι η αγάπη θέλει καλύψει πλήθος αμαρτιών·
\par 9 γίνεσθε φιλόξενοι εις αλλήλους χωρίς γογγυσμών·
\par 10 έκαστος κατά το χάρισμα, το οποίον έλαβεν, υπηρετείτε κατά τούτο εις αλλήλους ως καλοί οικονόμοι της πολυειδούς χάριτος του Θεού·
\par 11 εάν τις λαλή, ας λαλή ως λαλών λόγια Θεού· εάν τις υπηρετή, ας υπηρετή ως υπηρετών εκ της δυνάμεως, την οποίαν χορηγεί ο Θεός· διά να δοξάζηται εν πάσιν ο Θεός διά Ιησού Χριστού, εις τον οποίον είναι η δόξα και το κράτος εις τους αιώνας των αιώνων· αμήν.
\par 12 Αγαπητοί, μη παραξενεύεσθε διά τον βασανισμόν τον γινόμενον εις εσάς προς δοκιμασίαν, ως εάν συνέβαινεν εις εσάς παράδοξόν τι,
\par 13 αλλά καθότι είσθε κοινωνοί των παθημάτων του Χριστού, χαίρετε, ίνα και όταν η δόξα αυτού φανερωθή χαρήτε αγαλλιώμενοι.
\par 14 Εάν ονειδίζησθε διά το όνομα του Χριστού, είσθε μακάριοι, διότι το Πνεύμα της δόξης και το του Θεού αναπαύεται εφ' υμάς· κατά μεν αυτούς βλασφημείται, κατά δε υμάς δοξάζεται.
\par 15 Διότι μηδείς υμών ας μη πάσχη ως φονεύς ή κλέπτης ή κακοποιός ή ως περιεργαζόμενος τα αλλότρια.
\par 16 αλλ' εάν πάσχη ως Χριστιανός, ας μη αισχύνηται, αλλ' ας δοξάζη τον Θεόν κατά τούτο.
\par 17 Διότι έφθασεν ο καιρός του να αρχίση η κρίσις από του οίκου του Θεού· και αν αρχίζη πρώτον αφ' ημών, τι θέλει είσθαι το τέλος των απειθούντων εις το ευαγγέλιον του Θεού;
\par 18 και αν ο δίκαιος μόλις σώζηται, ο ασεβής και αμαρτωλός που θέλει φανή;
\par 19 Ώστε και οι πάσχοντες κατά το θέλημα του Θεού ας εμπιστεύωνται τας εαυτών ψυχάς εις αυτόν, ως εις πιστόν δημιουργόν εν αγαθοποιΐα.

\chapter{5}

\par 1 Τους μεταξύ σας πρεσβυτέρους παρακαλώ εγώ ο συμπρεσβύτερος και μάρτυς των παθημάτων τον Χριστού, ο και κοινωνός της δόξης, ήτις μέλλει να αποκαλυφθή,
\par 2 ποιμάνατε το μεταξύ σας ποίμνιον του Θεού, επισκοπούντες μη αναγκαστικώς αλλ' εκουσίως, μηδέ αισχροκερδώς αλλά προθύμως,
\par 3 μηδέ ως κατακυριεύοντες την κληρονομίαν του Θεού, αλλά τύποι γινόμενοι του ποιμνίου.
\par 4 Και όταν φανερωθή ο αρχιποιμήν, θέλετε λάβει τον αμαράντινον στέφανον της δόξης.
\par 5 Ομοίως οι νεώτεροι υποτάχθητε εις τους πρεσβυτέρους. Πάντες δε υποτασσόμενοι εις αλλήλους ενδύθητε την ταπεινοφροσύνην· διότι ο Θεός αντιτάσσεται εις τους υπερηφάνους, εις δε τους ταπεινούς δίδει χάριν.
\par 6 Ταπεινώθητε λοιπόν υπό την κραταιάν χείρα του Θεού, διά να σας υψώση εν καιρώ,
\par 7 και πάσαν την μέριμναν υμών ρίψατε επ' αυτόν, διότι αυτός φροντίζει περί υμών.
\par 8 Εγκρατεύθητε, αγρυπνήσατε· διότι ο αντίδικός σας διάβολος ως λέων ωρυόμενος περιέρχεται ζητών τίνα να καταπίη·
\par 9 εις τον οποίον αντιστάθητε μένοντες στερεοί εις την πίστιν, εξεύροντες ότι τα αυτά παθήματα γίνονται εις τους αδελφούς σας τους εν τω κόσμω.
\par 10 Ο δε Θεός πάσης χάριτος, όστις εκάλεσεν ημάς εις την αιώνιον αυτού δόξαν διά του Χριστού Ιησού, αφού πάθητε ολίγον, αυτός να σας τελειοποιήση, στηρίξη, ενισχύση, θεμελιώση.
\par 11 Εις αυτόν είη η δόξα και το κράτος εις τους αιώνας των αιώνων· αμήν.
\par 12 Σας έγραψα εν βραχυλογία διά του Σιλουανού του πιστού αδελφού, ως φρονώ, προτρέπων και επιμαρτυρών ότι αύτη είναι η αληθινή χάρις του Θεού, εις την οποίαν στέκεσθε.
\par 13 Σας ασπάζεται η εν Βαβυλώνι συνεκλεκτή εκκλησία και Μάρκος, ο υιός μου.
\par 14 Ασπάσθητε αλλήλους εν φιλήματι αγάπης. Ειρήνη εις πάντας υμάς τους εν Χριστώ Ιησού· αμήν.


\end{document}