\begin{document}

\title{2 Peter}


\chapter{1}

\par Συμεών Πέτρος, δούλος και απόστολος του Ιησού Χριστού, προς τους όσοι έλαχον ισότιμον με ημάς πίστιν εις την δικαιοσύνην του Θεού ημών και Σωτήρος Ιησού Χριστού·
\par 2 χάρις και ειρήνη πληθυνθείη εις εσάς διά της επιγνώσεως του Θεού και του Ιησού του Κυρίου ημών.
\par 3 Καθώς η θεία δύναμις αυτού εχάρισεν εις ημάς πάντα τα προς ζωήν και ευσέβειαν διά της επιγνώσεως του καλέσαντος ημάς διά της δόξης αυτού και αρετής,
\par 4 διά των οποίων εδωρήθησαν εις ημάς αι μέγισται και τίμιαι επαγγελίαι, ίνα διά τούτων γείνητε κοινωνοί θείας φύσεως, αποφυγόντες την εν τω κόσμω υπάρχουσαν διά της επιθυμίας διαφθοράν.
\par 5 Και δι' αυτό δε τούτο καταβαλόντες πάσαν σπουδήν, προσθέσατε εις την πίστιν σας την αρετήν, εις δε την αρετήν την γνώσιν,
\par 6 εις δε την γνώσιν την εγκράτειαν, εις δε την εγκράτειαν την υπομονήν, εις δε την υπομονήν την ευσέβειαν,
\par 7 εις δε την ευσέβειαν την φιλαδελφίαν, εις δε την φιλαδελφίαν την αγάπην.
\par 8 Διότι, εάν ταύτα υπάρχωσιν εις εσάς και περισσεύωσι, σας καθιστώσιν ουχί αργούς ουδέ ακάρπους εις την επίγνωσιν του Κυρίου ημών Ιησού Χριστού·
\par 9 επειδή εις όντινα δεν υπάρχουσι ταύτα, τυφλός είναι, μυωπάζει και ελησμόνησε τον καθαρισμόν των παλαιών αυτού αμαρτιών.
\par 10 Διά τούτο, αδελφοί, επιμελήθητε περισσότερον να κάμητε βεβαίαν την κλήσιν και την εκλογήν σας· διότι ταύτα κάμνοντες δεν θέλετε πταίσει ποτέ.
\par 11 Διότι ούτω θέλει σας δοθή πλουσίως η είσοδος εις την αιώνιον βασιλείαν του Κυρίου ημών και Σωτήρος Ιησού Χριστού.
\par 12 Όθεν δεν θέλω αμελήσει να σας υπενθυμίζω πάντοτε περί τούτων, καίτοι ειδότας και εστηριγμένους εις την παρούσαν αλήθειαν.
\par 13 Στοχάζομαι όμως δίκαιον, εφ' όσον είμαι εν τούτω τω σκηνώματι, να σας διεγείρω διά της υπενθυμίσεως,
\par 14 επειδή εξεύρω ότι εντός ολίγον θέλω αποθέσει το σκήνωμά μου, καθώς και ο Κύριος ημών Ιησούς Χριστός μοι εφανέρωσε.
\par 15 Θέλω όμως επιμεληθή, ώστε σεις και μετά την αποβίωσίν μου να δύνασθε πάντοτε να ενθυμήσθε αυτά.
\par 16 Διότι σας εγνωστοποιήσαμεν την δύναμιν και παρουσίαν του Κυρίου ημών Ιησού Χριστού, ουχί μύθους σοφιστικούς ακολουθήσαντες, αλλ' αυτόπται γενόμενοι της εκείνου μεγαλειότητος.
\par 17 Διότι έλαβε παρά Θεού Πατρός τιμήν και δόξαν, ότε ήλθεν εις αυτόν τοιαύτη φωνή υπό της μεγαλοπρεπούς δόξης, Ούτος είναι ο Υιός μου ο αγαπητός, εις τον οποίον εγώ ευηρεστήθην·
\par 18 και ταύτην την φωνήν ημείς ηκούσαμεν εξ ουρανού ελθούσαν, όντες μετ' αυτού εν τω όρει τω αγίω.
\par 19 Και έχομεν βεβαιότερον τον προφητικόν λόγον, εις τον οποίον κάμνετε καλά να προσέχητε ως εις λύχνον φέγγοντα εν σκοτεινώ τόπω, εωσού έλθη η αυγή της ημέρας και ο φωσφόρος ανατείλη εν ταις καρδίαις υμών·
\par 20 τούτο πρώτον εξεύροντες, ότι ουδεμία προφητεία της γραφής γίνεται εξ ιδίας του προφητεύοντος διασαφήσεως·
\par 21 διότι δεν ήλθε ποτέ προφητεία εκ θελήματος ανθρώπου, αλλ' υπό του Πνεύματος του Αγίου κινούμενοι ελάλησαν οι άγιοι άνθρωποι του Θεού.

\chapter{2}

\par Υπήρξαν όμως και ψευδοπροφήται μεταξύ του λαού, καθώς και μεταξύ σας θέλουσιν είσθαι ψευδοδιδάσκαλοι, οίτινες θέλουσι παρεισάξει αιρέσεις απωλείας, αρνούμενοι και τον αγοράσαντα αυτούς δεσπότην, επισύροντες εις εαυτούς ταχείαν απώλειαν·
\par 2 και πολλοί θέλουσιν εξακολουθήσει εις τας απωλείας αυτών, διά τους οποίους η οδός της αληθείας θέλει βλασφημηθή·
\par 3 και διά πλεονεξίαν θέλουσι σας εμπορευθή με πλαστούς λόγους, των οποίων η καταδίκη έκπαλαι δεν μένει αργή, και η απώλεια αυτών δεν νυστάζει.
\par 4 Διότι εάν ο Θεός δεν εφείσθη αγγέλους αμαρτήσαντας, αλλά ρίψας αυτούς εις τον τάρταρον δεδεμένους με αλύσεις σκότους, παρέδωκε διά να φυλάττωνται εις κρίσιν,
\par 5 και εάν τον παλαιόν κόσμον δεν εφείσθη, αλλά φέρων κατακλυσμόν επί τον κόσμον των ασεβών εφύλαξεν όγδοον τον Νώε, κήρυκα της δικαιοσύνης,
\par 6 και κατέκρινεν εις καταστροφήν τας πόλεις των Σοδόμων και της Γομόρρας και ετέφρωσε, καταστήσας παράδειγμα των μελλόντων να ασεβώσι,
\par 7 και ηλευθέρωσε τον δίκαιον Λωτ καταθλιβόμενον υπό της ασελγούς διαγωγής των ανόμων·
\par 8 διότι ο δίκαιος, κατοικών μεταξύ αυτών, δι' οράσεως και ακοής, εβασάνιζεν από ημέρας εις ημέραν την δικαίαν αυτού ψυχήν διά τα άνομα έργα αυτών·
\par 9 εξεύρει ο Κύριος να ελευθερόνη εκ του πειρασμού τους ευσεβείς, τους δε αδίκους να φυλάττη εις την ημέραν της κρίσεως, διά να κολάζωνται,
\par 10 μάλιστα δε τους οπίσω της σαρκός ακολουθούντας με επιθυμίαν ακαθαρσίας και καταφρονούντας την εξουσίαν. Είναι τολμηταί, αυθάδεις, δεν τρέμουσι βλασφημούντες τα αξιώματα,
\par 11 ενώ οι άγγελοι, μεγαλήτεροι όντες εις ισχύν και δύναμιν, δεν φέρουσι κατ' αυτών βλάσφημον κρίσιν ενώπιον του Κυρίου.
\par 12 Ούτοι όμως, ως άλογα φυσικά ζώα γεγεννημένα διά άλωσιν και φθοράν, βλασφημούσι περί πραγμάτων τα οποία αγνοούσι, και θέλουσι καταφθαρή εν τη ιδία αυτών διαφθορά,
\par 13 και θέλουσι λάβει τον μισθόν της αδικίας αυτών· στοχάζονται ηδονήν την καθημερινήν τρυφήν, είναι σπίλοι και μώμοι, εντρυφώσιν εν ταις απάταις αυτών, συμποσιάζουσι με σας,
\par 14 έχουσιν οφθαλμούς μεστούς μοιχείας και μη παυομένους από της αμαρτίας, δελεάζουσι ψυχάς αστηρίκτους, έχουσι την καρδίαν γεγυμνασμένην εις πλεονεξίας, είναι τέκνα κατάρας·
\par 15 αφήσαντες την ευθείαν οδόν, επλανήθησαν και ηκολούθησαν την οδόν του Βαλαάμ υιού του Βοσόρ, όστις ηγάπησε τον μισθόν της αδικίας,
\par 16 ηλέγχθη όμως διά την ιδίαν αυτού παρανομίαν, άφωνον υποζύγιον με φωνήν ανθρώπου λαλήσαν ημπόδισε την παραφροσύνην του προφήτου.
\par 17 Ούτοι είναι πηγαί άνυδροι, νεφέλαι υπό ανεμοστροβίλου ελαυνόμεναι, διά τους οποίους το ζοφερόν σκότος φυλάττεται εις τον αιώνα.
\par 18 Διότι λαλούντες υπερήφανα λόγια ματαιότητος, δελεάζουσι με τας επιθυμίας της σαρκός, με τας ασελγείας εκείνους οίτινες τωόντι απέφυγον τους εν πλάνη ζώντας,
\par 19 επαγγελλόμενοι εις αυτούς ελευθερίαν, ενώ αυτοί είναι δούλοι της διαφθοράς· διότι από όντινα νικάταί τις, τούτου και δούλος γίνεται.
\par 20 Επειδή εάν αφού απέφυγον τα μολύσματα του κόσμου διά της επιγνώσεως του Κυρίου και Σωτήρος Ιησού Χριστού, ενεπλέχθησαν πάλιν εις ταύτα και νικώνται, έγειναν εις αυτούς τα έσχατα χειρότερα των πρώτων.
\par 21 Επειδή καλήτερον ήτο εις αυτούς να μη γνωρίσωσι την οδόν της δικαιοσύνης, παρά αφού εγνώρισαν να επιστρέψωσιν εκ της παραδοθείσης εις αυτούς αγίας εντολής.
\par 22 Συνέβη δε εις αυτούς το της αληθινής παροιμίας, Ο κύων επέστρεψεν εις το ίδιον αυτού εξέρασμα, και, Ο χοίρος λουσθείς επέστρεψεν εις το κύλισμα του βορβόρου.

\chapter{3}

\par Δευτέραν ήδη ταύτην την επιστολήν σας γράφω, αγαπητοί, με τας οποίας διεγείρω δι' υπενθυμίσεως την ειλικρινή σας διάνοιαν,
\par 2 διά να ενθυμηθήτε τους λόγους τους προλαληθέντας υπό των αγίων προφητών και την παραγγελίαν ημών των αποστόλων του Κυρίου και Σωτήρος·
\par 3 τούτο πρώτον γνωρίζοντες, ότι θέλουσιν ελθεί εν ταις εσχάταις ημέραις εμπαίκται, περιπατούντες κατά τας ιδίας αυτών επιθυμίας
\par 4 και λέγοντες· Που είναι η υπόσχεσις της παρουσίας αυτού; διότι αφ' ης ημέρας οι πατέρες εκοιμήθησαν, τα πάντα διαμένουσιν ούτως απ' αρχής της κτίσεως.
\par 5 Διότι εκουσίως αγνοούσι τούτο, ότι με τον λόγον του Θεού οι ουρανοί έγειναν έκπαλαι και η γη συνεστώσα εξ ύδατος και δι' ύδατος,
\par 6 διά των οποίων ο τότε κόσμος απωλέσθη κατακλυσθείς υπό του ύδατος·
\par 7 οι δε σημερινοί ουρανοί και η γη διά του αυτού λόγου είναι αποτεταμιευμένοι, φυλαττόμενοι διά το πυρ εις την ημέραν της κρίσεως και της απωλείας των ασεβών ανθρώπων.
\par 8 Εν δε τούτο ας μη σας λανθάνη, αγαπητοί, ότι παρά Κυρίω μία ημέρα είναι ως χίλια έτη και χίλια έτη ως ημέρα μία.
\par 9 Δεν βραδύνει ο Κύριος την υπόσχεσιν αυτού, ως τινές λογίζονται τούτο βραδύτητα, αλλά μακροθυμεί εις ημάς, μη θέλων να απολεσθώσι τινές, αλλά πάντες να έλθωσιν εις μετάνοιαν.
\par 10 Θέλει δε ελθεί η ημέρα του Κυρίου ως κλέπτης εν νυκτί, καθ' ην οι ουρανοί θέλουσι παρέλθει με συριγμόν, τα στοιχεία δε πυρακτούμενα θέλουσι διαλυθή, και η γη και τα εν αυτή έργα θέλουσι κατακαή.
\par 11 Επειδή λοιπόν πάντα ταύτα διαλύονται, οποίοι πρέπει να ήσθε σεις εις πολίτευμα άγιον και ευσέβειαν,
\par 12 προσμένοντες και σπεύδοντες εις την παρουσίαν της ημέρας του Θεού, καθ' ην οι ουρανοί πυρούμενοι θέλουσι διαλυθή και τα στοιχεία πυρακτούμενα θέλουσι χωνευθή;
\par 13 Κατά δε την υπόσχεσιν αυτού νέους ουρανούς και νέαν γην προσμένομεν, εν οις δικαιοσύνη κατοικεί.
\par 14 Διά τούτο, αγαπητοί, ταύτα προσμένοντες, σπουδάσατε να ευρεθήτε άσπιλοι και αμώμητοι ενώπιον αυτού εν ειρήνη,
\par 15 και νομίζετε σωτηρίαν την μακροθυμίαν του Κυρίου ημών, καθώς και ο αγαπητός ημών αδελφός Παύλος έγραψε προς εσάς κατά την δοθείσαν εις αυτόν σοφίαν,
\par 16 ως και εν πάσαις ταις επιστολαίς αυτού, λαλών εν αυταίς περί τούτων, μεταξύ των οποίων είναι τινά δυσνόητα, τα οποία οι αμαθείς και αστήρικτοι στρεβλόνουσιν, ως και τας λοιπάς γραφάς προς την ιδίαν αυτών απώλειαν.
\par 17 Σεις λοιπόν, αγαπητοί, προγνωρίζοντες ταύτα φυλάττεσθε, διά να μη παρασυρθήτε με την πλάνην των ανόμων και εκπέσητε από τον στηριγμόν σας,
\par 18 αυξάνεσθε δε εις την χάριν και εις την γνώσιν του Κυρίου ημών και Σωτήρος Ιησού Χριστού· εις αυτόν έστω η δόξα και νυν και εις ημέραν αιώνος· αμήν.


\end{document}