\begin{document}

\title{Ιούδα}


\chapter{1}

\par 1 ΙΟΥΔΑΣ δούλος Ιησού Χριστού, αδελφός δε Ιακώβου, προς τους κλητούς τους ηγιασμένους υπό Θεού Πατρός, και τετηρημένους υπό του Ιησού Χριστού
\par 2 έλεος πληθυνθείη εις εσάς, και ειρήνη, και αγάπη.
\par 3 Αγαπητοί, επειδή καταβάλλω πάσαν σπουδήν να σας γράψω περί της κοινής σωτηρίας, έλαβον ανάγκην να σας γράψω, προτρέπων εις το να αγωνίζησθε δια την πίστιν, ήτις άπαξ παρεδόθη εις τους αγίους.
\par 4 Διότι εισεχώρησαν λαθραίως τινές άνθρωποι, οίτινες ήσαν παλαιόθεν προγεγραμμένοι εις ταύτην την καταδίκην, ασεβείς, μεταστρέφοντες την χάριν του Θεού ημών εις ασέλγειαν, και αρνούμενοι τον μόνον δεσπότην Θεόν και Κύριον ημών Ιησούν Χριστόν.
\par 5 Θέλω δε να σας υπενθυμίσω, αν και σεις εγνωρίσατε ήδη τούτο, ότι ο Κύριος, αφού έσωσε τον λαόν εκ γης Αιγύπτου, απώλεσεν ύστερον τους μη πιστεύσαντας
\par 6 και αγγέλους οίτινες δεν εφύλαξαν την εαυτών αξίαν, αλλά κατέλιπον το ίδιον αυτών κατοικητήριον, εφύλαξε με παντοτεινά δεσμά υποκάτω του σκότους, δια την κρίσιν της μεγάλης ημέρας
\par 7 καθώς τα Σόδομα και τα Γόμορρα, και αι πέριξ αυτών πόλεις, εις την πορνείαν παραδοθείσαι κατά τον όμοιον με τούτους τρόπον, και ακολουθούσαι οπίσω άλλης σαρκός, πρόκεινται παράδειγμα τιμωρούμεναι με το αιώνιον πυρ.
\par 8 Ομοίως και ούτοι ενυπνιαζόμενοι, την μεν σάρκα μιαίνουσι, την δε εξουσίαν καταφρονούσι, και τα αξιώματα βλασφημούσιν.
\par 9 Ο δε Μιχαήλ ο αρχάγγελος, ότε αγωνιζόμενος με τον διάβολον εφιλονείκει περί του σώματος του Μωϋσέως, δεν ετόλμησε να επιφέρη εναντίον αυτού κατηγορίαν βλάσφημον, αλλ' είπεν, Ο Κύριος να σε επιτιμήση.
\par 10 Ούτοι δε όσα μεν δεν εξεύρουσι βλασφημούσιν, όσα δε φυσικώς ως τα άλογα ζώα εξεύρουσιν, εις ταύτα φθείρονται.
\par 11 Ουαί εις αυτούς διότι περιεπάτησαν εις την οδόν του Κάϊν, και χάριν μισθού εξεχύθησαν εις την πλάνην του Βαλαάμ, και απωλέσθησαν εις την αντιλογίαν του Κορέ.
\par 12 Ούτοι είναι κηλίδες εις τας αγάπας σας, συμποσιάζοντες αφόβως, βόσκοντες εαυτούς, νεφέλαι άνυδροι υπό ανέμων περιφερόμεναι, δένδρα φθινοπωρινά άκαρπα, δις αποθανόντα, εκριζωθέντα,
\par 13 κύματα άγρια θαλάσσης επαφρίζοντα τας ιδίας αυτών αισχύνας, αστέρες πλανήται, δια τους οποίους το ζοφερόν σκότος είναι τετηρημένον εις τον αιώνα.
\par 14 Προεφήτευσε δε περί τούτων και ο Ενώχ, έβδομος από Αδάμ, λέγων, "Ιδού, ήλθεν ο Κύριος με μυριάδας αγίων αυτού,
\par 15 δια να κάμη κρίσιν κατά πάντων, και να ελέγξη πάντας τους ασεβείς εξ αυτών, δια πάντα τα έργα της ασεβείας αυτών, τα οποία έπραξαν και δια πάντα τα σκληρά τα οποία ελάλησαν κατ' αυτού αμαρτωλοί ασεβείς."
\par 16 Ούτοι είναι γογγυσταί, μεμψίμοιροι, περιπατούντες κατά τας επιθυμίας αυτών και το στόμα αυτών λαλεί υπερήφανα, και κολακεύουσι πρόσωπα χάριν ωφωλείας.
\par 17 Αλλά σεις, αγαπητοί, ενθυμήθητε τους λόγους τους προειρημένους υπό των αποστόλων του Κυρίου ημών Ιησού Χριστού,
\par 18 ότι σας έλεγον, ότι "εν εσχάτω καιρώ θέλουσιν είσθαι εμπαίκται, περιπατούντες κατά τας ασεβείς επιθυμίας αυτών."
\par 19 Ούτοι είναι οι αποχωρίζοντες εαυτούς, ζωώδεις, Πνεύμα μη έχοντες.
\par 20 Σεις όμως, αγαπητοί, εποικοδομούντες εαυτούς επί την αγιωτάτην πίστιν σας, προσευχόμενοι εν Πνεύματι Αγίω,
\par 21 φυλάξατε εαυτούς εις την αγάπην του Θεού, προσμένοντες το έλεος του Κυρίου ημών Ιησού Χριστού εις ζωήν αιώνιον.
\par 22 Και άλλους μεν ελεείτε, κάμνοντες διάκρισιν,
\par 23 άλλους δε σώζετε μετά φόβου, αρπάζοντες αυτούς εκ του πυρός, μισούντες και τον χιτώνα τον μεμολυσμένον από της σαρκός.
\par 24 Εις δε τον δυνάμενον να σας φυλάξη απταίστους, και να σας στήση κατενώπιον της δόξης αυτού αμώμους εν αγαλλιάσει,
\par 25 εις τον μόνον σοφόν Θεόν τον σωτήρα ημών, είη δόξα και μεγαλωσύνη, κράτος και εξουσία, και νυν και εις πάντας τους αιώνας. Αμήν.


\end{document}