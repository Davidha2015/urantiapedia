\begin{document}

\title{Αποκάλυψις Ιωάννου}


\chapter{1}

\par 1 Αποκάλυψις Ιησού Χριστού, την οποίαν έδωκεν εις αυτόν ο Θεός, διά να δείξη εις τους δούλους αυτού όσα πρέπει να γείνωσι ταχέως, και εφανέρωσεν αυτά αποστείλας διά του αγγέλου αυτού εις τον δούλον αυτού Ιωάννην,
\par 2 όστις εμαρτύρησε τον λόγον του Θεού και την μαρτυρίαν του Ιησού Χριστού και όσα είδε.
\par 3 Μακάριος ο αναγινώσκων και οι ακούοντες τους λόγους της προφητείας και φυλάττοντες τα γεγραμμένα εν αυτή· διότι ο καιρός είναι πλησίον.
\par 4 Ο Ιωάννης προς τας επτά εκκλησίας τας εν τη Ασία· χάρις υμίν και ειρήνη από του ο ων και ο ην και ο ερχόμενος· και από των επτά πνευμάτων, τα οποία είναι ενώπιον του θρόνου αυτού,
\par 5 και από του Ιησού Χριστού, όστις είναι ο μάρτυς ο πιστός, ο πρωτότοκος εκ των νεκρών και ο άρχων των βασιλέων της γης. Εις τον αγαπήσαντα ημάς και λούσαντα ημάς από των αμαρτιών ημών με το αίμα αυτού,
\par 6 και όστις έκαμεν ημάς βασιλείς και ιερείς εις τον Θεόν και Πατέρα αυτού, εις αυτόν είη η δόξα και το κράτος εις τους αιώνας των αιώνων· αμήν.
\par 7 Ιδού, έρχεται μετά των νεφελών, και θέλει ιδεί αυτόν πας οφθαλμός και εκείνοι οίτινες εξεκέντησαν αυτόν, και θέλουσι θρηνήσει επ' αυτόν πάσαι αι φυλαί της γης. Ναι, αμήν.
\par 8 Εγώ είμαι το Α και το Ω, αρχή και τέλος, λέγει ο Κύριος, ο ων και ο ην και ο ερχόμενος, ο παντοκράτωρ.
\par 9 Εγώ ο Ιωάννης, ο και αδελφός σας και συγκοινωνός εις την θλίψιν και εις την βασιλείαν και την υπομονήν του Ιησού Χριστού, ήμην εν τη νήσω τη καλουμένη Πάτμω διά τον λόγον του Θεού και διά την μαρτυρίαν του Ιησού Χριστού.
\par 10 Κατά την κυριακήν ημέραν ήλθον εις έκστασιν πνευματικήν, και ήκουσα οπίσω μου φωνήν μεγάλην ως σάλπιγγος,
\par 11 ήτις έλεγεν· Εγώ είμαι το Α και το Ω, ο πρώτος και ο έσχατος· και, ό,τι βλέπεις, γράψον εις βιβλίον και πέμψον εις τας επτά εκκλησίας, τας εν τη Ασία, εις Έφεσον και εις Σμύρνην και εις Πέργαμον και εις Θυάτειρα και εις Σάρδεις και εις Φιλαδέλφειαν και εις Λαοδίκειαν.
\par 12 Και εστράφην να ίδω την φωνήν, ήτις ελάλησε μετ' εμού· και στραφείς είδον επτά λυχνίας χρυσάς,
\par 13 και εν μέσω των επτά λυχνιών είδον ένα όμοιον με υιόν ανθρώπου, ενδεδυμένον ποδήρη χιτώνα και περιεζωσμένον πλησίον των μαστών ζώνην χρυσήν.
\par 14 Η δε κεφαλή αυτού και αι τρίχες ήσαν λευκαί και ως μαλλίον λευκόν, ως χιών· και οι οφθαλμοί αυτού ως φλόξ πυρός,
\par 15 και οι πόδες αυτού όμοιοι με χαλκολίβανον, ως εν καμίνω πεπυρωμένοι, και η φωνή αυτού ως φωνή υδάτων πολλών,
\par 16 και είχεν εν τη δεξιά αυτού χειρί επτά αστέρας, και εκ του στόματος αυτού του εξήρχετο ρομφαία δίστομος οξεία, και η όψις αυτού έλαμπεν ως ο ήλιος λάμπει εν τη δυνάμει αυτού.
\par 17 Και ότε είδον αυτόν, έπεσα προς τους πόδας αυτού ως νεκρός, και επέθηκε την δεξιάν αυτού χείρα επ' εμέ; λέγων μοι· Μη φοβού· εγώ είμαι ο πρώτος και ο έσχατος
\par 18 και ο ζων, και έγεινα νεκρός, και ιδού, είμαι ζων εις τους αιώνας των αιώνων, αμήν, και έχω τα κλειδία του άδου και του θανάτου.
\par 19 Γράψον όσα είδες και όσα είναι και όσα μέλλουσι να γείνωσι μετά ταύτα·
\par 20 το μυστήριον των επτά αστέρων, τους οποίους είδες εν τη δεξιά μου, και τας επτά λυχνίας τας χρυσάς. Οι επτά αστέρες είναι οι άγγελοι των επτά εκκλησιών, και αι επτά λυχνίαι, τας οποίας είδες, είναι αι επτά εκκλησίαι.

\chapter{2}

\par 1 Προς τον άγγελον της εκκλησίας της Εφέσου γράψον. Ταύτα λέγει ο κρατών τους επτά αστέρας εν τη δεξιά αυτού, ο περιπατών εν μέσω των επτά λυχνιών των χρυσών·
\par 2 Εξεύρω τα έργα σου και τον κόπον σου και την υπομονήν σου, και ότι δεν δύνασαι να υποφέρης τους κακούς, και εδοκίμασας τους λέγοντας ότι είναι απόστολοι, και δεν είναι, και εύρες αυτούς ψευδείς·
\par 3 και υπέφερες και έχεις υπομονήν και διά το όνομά μου εκοπίασας, και δεν απέκαμες.
\par 4 Πλην έχω τι κατά σου, διότι την αγάπην σου την πρώτην αφήκας.
\par 5 Ενθυμού λοιπόν πόθεν εξέπεσες και μετανόησον και κάμε τα πρώτα έργα· ει δε μη, έρχομαι προς σε ταχέως και θέλω κινήσει την λυχνίαν σου εκ του τόπου αυτής, εάν δεν μετανοήσης.
\par 6 Έχεις όμως τούτο, ότι μισείς τα έργα των Νικολαϊτών, τα οποία και εγώ μισώ.
\par 7 Όστις έχει ωτίον ας ακούση τι λέγει το Πνεύμα προς τας εκκλησίας. Εις τον νικώντα θέλω δώσει εις αυτόν να φάγη εκ του ξύλου της ζωής, το οποίον είναι εν μέσω του παραδείσου του Θεού.
\par 8 Και προς τον άγγελον της εκκλησίας των Σμυρναίων γράψον· Ταύτα λέγει ο πρώτος και ο έσχατος, όστις έγεινε νεκρός και έζησεν·
\par 9 Εξεύρω τα έργα σου και την θλίψιν και την πτωχείαν· είσαι όμως πλούσιος· και την βλασφημίαν των λεγόντων εαυτούς ότι είναι Ιουδαίοι και δεν είναι, αλλά συναγωγή του Σατανά.
\par 10 Μη φοβού μηδέν εκ των όσα μέλλεις να πάθης. Ιδού, ο διάβολος μέλλει να βάλη τινάς εξ υμών εις φυλακήν διά να δοκιμασθήτε, και θέλετε έχει θλίψιν δέκα ημερών. Γίνου πιστός μέχρι θανάτου, και θέλω σοι δώσει τον στέφανον της ζωής.
\par 11 Όστις έχει ωτίον ας ακούση τι λέγει το Πνεύμα προς τας εκκλησίας. Ο νικών δεν θέλει αδικηθή εκ του θανάτου του δευτέρου.
\par 12 Και προς τον άγγελον της εν Περγάμω εκκλησίας γράψον· Ταύτα λέγει ο έχων την ρομφαίαν την δίστομον, την οξείαν·
\par 13 Εξεύρω τα έργα σου και που κατοικείς· όπου είναι ο θρόνος του Σατανά· και κρατείς το όνομά μου, και δεν ηρνήθης την πίστιν μου και εν ταις ημέραις, εν αις υπήρχεν Αντίπας ο μάρτυς μου ο πιστός, όστις εφονεύθη παρ' υμίν, όπου κατοικεί ο Σατανάς.
\par 14 Έχω όμως κατά σου ολίγα, διότι έχεις εκεί τινάς κρατούντας την διδαχήν του Βαλαάμ, όστις εδίδασκε τον Βαλάκ να βάλη σκάνδαλον ενώπιον των υιών Ισραήλ, ώστε να φάγωσιν ειδωλόθυτα και να πορνεύσωσιν.
\par 15 Ούτως έχεις και συ τινάς κρατούντας την διδαχήν των Νικολαϊτών, το οποίον μισώ.
\par 16 Μετανόησον· ει δε μη, έρχομαι προς σε ταχέως και θέλω πολεμήσει προς αυτούς με την ρομφαίαν του στόματός μου.
\par 17 Όστις έχει ωτίον, ας ακούση τι λέγει το Πνεύμα προς τας εκκλησίας. Εις τον νικώντα θέλω δώσει εις αυτόν να φάγη από του μάννα του κεκρυμμένου, και θέλω δώσει εις αυτόν ψήφον λευκήν, και επί την ψήφον όνομα νέον γεγραμμένον, το οποίον ουδείς γνωρίζει ειμή ο λαμβάνων.
\par 18 Και προς τον άγγελον της εν Θυατείροις εκκλησίας γράψον· Ταύτα λέγει ο Υιός του Θεού, ο έχων τους οφθαλμούς αυτού ως φλόγα πυρός, και οι πόδες αυτού είναι όμοιοι με χαλκολίβανον·
\par 19 Εξεύρω τα έργα σου και την αγάπην και την διακονίαν και την πίστιν και την υπομονήν σου και τα έργα σου και τα έσχατα, ότι είναι πλειότερα των πρώτων.
\par 20 Έχω όμως κατά σου ολίγα, διότι αφίνεις την γυναίκα Ιεζάβελ, ήτις λέγει εαυτήν προφήτιν, να διδάσκη και να πλανά τους δούλους μου εις το να πορνεύωσι και να τρώγωσιν ειδωλόθυτα.
\par 21 Και έδωκα εις αυτήν καιρόν να μετανοήση εκ της πορνείας αυτής, και δεν μετενόησεν.
\par 22 Ιδού, εγώ βάλλω αυτήν εις κλίνην και τους μοιχεύοντας μετ' αυτής εις θλίψιν μεγάλην, εάν δεν μετανοήσωσιν εκ των έργων αυτών,
\par 23 και τα τέκνα αυτής θέλω αποκτείνει με θάνατον, και θέλουσι γνωρίσει πάσαι αι εκκλησίαι, ότι εγώ είμαι ο ερευνών νεφρούς και καρδίας, και θέλω σας δώσει εις έκαστον κατά τα έργα σας.
\par 24 Λέγω δε προς εσάς και προς τους λοιπούς τους εν Θυατείροις, όσοι δεν έχουσι την διδαχήν ταύτην και οίτινες δεν εγνώρισαν τα βάθη του Σατανά, ως λέγουσι· Δεν θέλω βάλει εφ' υμάς άλλο βάρος·
\par 25 πλην εκείνο, το οποίον έχετε, κρατήσατε εωσού έλθω.
\par 26 Και όστις νικά και όστις φυλάττει μέχρι τέλους τα έργα μου, θέλω δώσει εις αυτόν εξουσίαν επί των εθνών,
\par 27 και θέλει ποιμάνει αυτούς εν ράβδω σιδηρά, θέλουσι συντριφθή ως τα σκεύη του κεραμέως, καθώς και εγώ έλαβον παρά του Πατρός μου,
\par 28 και θέλω δώσει εις αυτόν τον αστέρα τον πρωϊνόν.
\par 29 Όστις έχει ωτίον ας ακούση τι λέγει το Πνεύμα προς τας εκκλησίας.

\chapter{3}

\par 1 Και προς τον άγγελον της εν Σάρδεσιν εκκλησίας γράψον· Ταύτα λέγει ο έχων τα επτά πνεύματα του Θεού και τους επτά αστέρας. Εξεύρω τα έργα σου, ότι το όνομα έχεις ότι ζης και είσαι νεκρός.
\par 2 Γίνου άγρυπνος και στήριξον τα λοιπά, τα οποία μέλλουσι να αποθάνωσι· διότι δεν εύρηκα τα έργα σου τέλεια ενώπιον του Θεού.
\par 3 Ενθυμού λοιπόν πως έλαβες και ήκουσας, και φύλαττε αυτά και μετανόησον. Εάν λοιπόν δεν αγρυπνήσης, θέλω ελθεί επί σε ως κλέπτης, και δεν θέλεις γνωρίσει ποίαν ώραν θέλω ελθεί επί σε.
\par 4 Έχεις ολίγα ονόματα και εν Σάρδεσι, τα οποία δεν εμόλυναν τα ιμάτια αυτών, και θέλουσι περιπατήσει μετ' εμού με λευκά, διότι είναι άξιοι.
\par 5 Ο νικών, ούτος θέλει ενδυθή ιμάτια λευκά, και δεν θέλω εξαλείψει το όνομα αυτού εκ του βιβλίου της ζωής, και θέλω ομολογήσει το όνομα αυτού ενώπιον του Πατρός μου και ενώπιον των αγγέλων αυτού.
\par 6 Όστις έχει ωτίον, ας ακούση τι λέγει το Πνεύμα προς τας εκκλησίας.
\par 7 Και προς τον άγγελον της εν Φιλαδελφεία εκκλησίας γράψον· Ταύτα λέγει ο άγιος, ο αληθινός, ο έχων το κλειδίον του Δαβίδ, όστις ανοίγει και ουδείς κλείει, και κλείει και ουδείς ανοίγει·
\par 8 Εξεύρω τα έργα σου· ιδού, έθεσα ενώπιόν σου θύραν ανεωγμένην, και ουδείς δύναται να κλείση αυτήν· διότι έχεις μικράν δύναμιν και εφύλαξας τον λόγον μου και δεν ηρνήθης το όνομά μου.
\par 9 Ιδού, θέλω κάμει τους εκ της συναγωγής του Σατανά, οίτινες λέγουσιν εαυτούς ότι είναι Ιουδαίοι, και δεν είναι, αλλά ψεύδονται ιδού, θέλω κάμει αυτούς να έλθωσι και να προσκυνήσωσιν ενώπιον των ποδών σου και να γνωρίσωσιν ότι εγώ σε ηγάπησα.
\par 10 Επειδή εφύλαξας τον λόγον της υπομονής μου, και εγώ θέλω σε φυλάξει εκ της ώρας του πειρασμού, ήτις μέλλει να έλθη επί της οικουμένης όλης, διά να δοκιμάση τους κατοικούντας επί της γης.
\par 11 Ιδού, έρχομαι ταχέως· κράτει εκείνο το οποίον έχεις, διά να μη λάβη μηδείς τον στέφανόν σου.
\par 12 Όστις νικά, θέλω κάμει αυτόν στύλον εν τω ναώ του Θεού μου, και δεν θέλει εξέλθει πλέον έξω, και θέλω γράψει επ' αυτόν το όνομα του Θεού μου και το όνομα της πόλεως του Θεού μου, της νέας Ιερουσαλήμ, ήτις καταβαίνει εκ του ουρανού από του Θεού μου, και το όνομά μου το νέον.
\par 13 Όστις έχει ωτίον, ας ακούση τι λέγει το Πνεύμα προς τας εκκλησίας.
\par 14 Και προς τον άγγελον της εκκλησίας των Λαοδικέων γράψον· Ταύτα λέγει ο Αμήν, ο μάρτυς ο πιστός και αληθινός, η αρχή της κτίσεως του Θεού.
\par 15 Εξεύρω τα έργα σου, ότι ούτε ψυχρός είσαι ούτε ζεστός· είθε να ήσο ψυχρός ή ζεστός·
\par 16 ούτως, επειδή είσαι χλιαρός και ούτε ψυχρός ούτε ζεστός, μέλλω να σε εξεμέσω εκ του στόματός μου.
\par 17 Διότι λέγεις ότι πλούσιος είμαι και επλούτησα και δεν έχω χρείαν ουδενός, και δεν εξεύρεις ότι συ είσαι ο ταλαίπωρος και ελεεινός και πτωχός και τυφλός και γυμνός·
\par 18 συμβουλεύω σε να αγοράσης παρ' εμού χρυσίον δεδοκιμασμένον εκ πυρός διά να πλουτήσης, και ιμάτια λευκά διά να ενδυθής και να μη φανερωθή η αισχύνη της γυμνότητός σου, και χρίσον τους οφθαλμούς σου με κολλούριον διά να βλέπης.
\par 19 Εγώ όσους αγαπώ, ελέγχω και παιδεύω· γενού λοιπόν ζηλωτής και μετανόησον.
\par 20 Ιδού, ίσταμαι εις την θύραν και κρούω· εάν τις ακούση της φωνής μου και ανοίξη την θύραν, θέλω εισέλθει προς αυτόν και θέλω δειπνήσει μετ' αυτού και αυτός μετ' εμού.
\par 21 Όστις νικά, θέλω δώσει εις αυτόν να καθήση μετ' εμού εν τω θρόνω μου, καθώς και εγώ ενίκησα και εκάθησα μετά του Πατρός μου εν τω θρόνω αυτού.
\par 22 Όστις έχει ωτίον, ας ακούση τι λέγει το Πνεύμα προς τας εκκλησίας.

\chapter{4}

\par 1 Μετά ταύτα είδον, και ιδού, θύρα ανεωγμένη εν τω ουρανώ, και η φωνή η πρώτη, την οποίαν ήκουσα ως σάλπιγγος λαλούσης μετ' εμού, έλεγεν· Ανάβα εδώ και θέλω σοι δείξει όσα πρέπει να γείνωσι μετά ταύτα.
\par 2 Και ευθύς ήλθον εις πνευματικήν έκστασιν· και ιδού, θρόνος έκειτο εν τω ουρανώ, και επί του θρόνου ήτο τις καθήμενος.
\par 3 και ο καθήμενος ήτο όμοιος κατά την θέαν με λίθον ίασπιν και σάρδινον· και ήτο ίρις κύκλω του θρόνου ομοία κατά την θέαν με σμάραγδον.
\par 4 Και κύκλω του θρόνου ήσαν θρόνοι εικοσιτέσσαρες· και επί τους θρόνους είδον καθημένους τους εικοσιτέσσαρας πρεσβυτέρους, ενδεδυμένους ιμάτια λευκά, και είχον επί τας κεφαλάς αυτών στεφάνους χρυσούς.
\par 5 Και εκ του θρόνου εξήρχοντο αστραπαί και βρονταί και φωναί· και ήσαν επτά λαμπάδες πυρός καιόμεναι έμπροσθεν του θρόνου, αίτινες είναι τα επτά πνεύματα του Θεού·
\par 6 και έμπροσθεν του θρόνου ήτο θάλασσα υαλίνη, ομοία με κρύσταλλον· και εν τω μέσω του θρόνου και κύκλω του θρόνου τέσσαρα ζώα γέμοντα οφθαλμών έμπροσθεν και όπισθεν.
\par 7 Και το ζώον το πρώτον ήτο όμοιον με λέοντα, και το δεύτερον ζώον όμοιον με μοσχάριον, και το τρίτον ζώον είχε το πρόσωπον ως άνθρωπος, και το τέταρτον ζώον ήτο όμοιον με αετόν πετώμενον.
\par 8 Και τα τέσσαρα ζώα είχον έκαστον χωριστά ανά εξ πτέρυγας κυκλόθεν και έσωθεν ήσαν γέμοντα οφθαλμών, και δεν παύουσιν ημέραν και νύκτα λέγοντα· Άγιος, άγιος, άγιος Κύριος ο Θεός ο παντοκράτωρ, ο ην και ο ων και ο ερχόμενος.
\par 9 Και όταν προσφέρωσι τα ζώα δόξαν και τιμήν και ευχαριστίαν εις τον καθήμενον επί του θρόνου, εις τον ζώντα εις τους αιώνας των αιώνων,
\par 10 οι εικοσιτέσσαρες πρεσβύτεροι θέλουσι πέσει ενώπιον του καθημένου επί του θρόνου, και θέλουσι προσκυνήσει τον ζώντα εις τους αιώνας των αιώνων, και θέλουσι βάλει τους στεφάνους αυτών ενώπιον του θρόνου, λέγοντες·
\par 11 Άξιος είσαι, Κύριε, να λάβης την δόξαν και την τιμήν και την δύναμιν, διότι συ έκτισας τα πάντα, και διά το θέλημά σου υπάρχουσι και εκτίσθησαν.

\chapter{5}

\par 1 Και είδον εν τη δεξιά του καθημένου επί του θρόνου βιβλίον γεγραμμένον έσωθεν και όπισθεν, κατεσφραγισμένον με σφραγίδας επτά.
\par 2 Και είδον άγγελον ισχυρόν κηρύττοντα μετά φωνής μεγάλης· Τις είναι άξιος να ανοίξη το βιβλίον και να λύση τας σφραγίδας αυτού;
\par 3 Και ουδείς ηδύνατο εν τω ουρανώ, ουδέ επί της γης ουδέ υποκάτω της γης να ανοίξη το βιβλίον ουδέ να βλέπη αυτό.
\par 4 Και εγώ έκλαιον πολλά, ότι ουδείς ευρέθη άξιος να ανοίξη και να αναγνώση το βιβλίον ούτε να βλέπη αυτό.
\par 5 Και εις εκ των πρεσβυτέρων μοι λέγει· Μη κλαίε· ιδού, υπερίσχυσεν ο λέων, όστις είναι εκ της φυλής Ιούδα, η ρίζα του Δαβίδ, να ανοίξη το βιβλίον και να λύση τας επτά σφραγίδας αυτού.
\par 6 Και είδον και ιδού εν μέσω του θρόνου και των τεσσάρων ζώων και εν μέσω των πρεσβυτέρων Αρνίον ιστάμενον ως εσφαγμένον, έχον κέρατα επτά και οφθαλμούς επτά, οίτινες είναι τα επτά πνεύματα του Θεού τα απεσταλμένα εις πάσαν την γην.
\par 7 Και ήλθε και έλαβε το βιβλίον εκ της δεξιάς του καθημένου επί του θρόνου.
\par 8 Και ότε έλαβε το βιβλίον, τα τέσσαρα ζώα και οι εικοσιτέσσαρες πρεσβύτεροι έπεσον ενώπιον του Αρνίου, έχοντες έκαστος κιθάρας και φιάλας χρυσάς πλήρεις θυμιαμάτων, αίτινες είναι αι προσευχαί των αγίων·
\par 9 και ψάλλουσι νέαν ωδήν, λέγοντες· Άξιος είσαι να λάβης το βιβλίον και να ανοίξης τας σφραγίδας αυτού, διότι εσφάγης και ηγόρασας ημάς εις τον Θεόν διά του αίματός σου εκ πάσης φυλής και γλώσσης και λαού και έθνους,
\par 10 και έκαμες ημάς εις τον Θεόν ημών βασιλείς και ιερείς, και θέλομεν βασιλεύσει επί της γης.
\par 11 Και είδον και ήκουσα φωνήν αγγέλων πολλών κυκλόθεν του θρόνου και των ζώων και των πρεσβυτέρων, και ήτο ο αριθμός αυτών μυριάδες μυριάδων και χιλιάδες χιλιάδων,
\par 12 λέγοντες μετά φωνής μεγάλης· Άξιον είναι το Αρνίον το εσφαγμένον να λάβη την δύναμιν και πλούτον και σοφίαν και ισχύν και τιμήν και δόξαν και ευλογίαν.
\par 13 Και παν κτίσμα, το οποίον είναι εν τω ουρανώ και επί της γης και υποκάτω της γης και όσα είναι εν τη θαλάσση και πάντα τα εν αυτοίς, ήκουσα ότι έλεγον· Εις τον καθήμενον επί του θρόνου και εις το Αρνίον έστω η ευλογία και η τιμή και η δόξα και το κράτος εις τους αιώνας των αιώνων.
\par 14 Και τα τέσσαρα ζώα έλεγον· Αμήν· και οι εικοσιτέσσαρες πρεσβύτεροι έπεσαν και προσεκύνησαν τον ζώντα εις τους αιώνας των αιώνων.

\chapter{6}

\par 1 Και είδον, ότε ήνοιξε το Αρνίον μίαν εκ των σφραγίδων, και ήκουσα εν εκ των τεσσάρων ζώων λέγον ως φωνήν βροντής· Έρχου και βλέπε.
\par 2 Και είδον, και ιδού, ίππος λευκός· και ο καθήμενος επ' αυτόν είχε τόξον· και εδόθη εις αυτόν στέφανος, και εξήλθε νικών και διά να νικήση.
\par 3 Και ότε ήνοιξε την δευτέραν σφραγίδα, ήκουσα το δεύτερον ζώον λέγον· Έρχου και βλέπε.
\par 4 Και εξήλθεν άλλος ίππος κόκκινος, και εις τον καθήμενον επ' αυτόν εδόθη να σηκώση την ειρήνην από της γης, και να σφάξωσιν αλλήλους, και εδόθη εις αυτόν μάχαιρα μεγάλη.
\par 5 Και ότε ήνοιξε την τρίτην σφραγίδα, ήκουσα το τρίτον ζώον λέγον· Έρχου και βλέπε. Και είδον, και ιδού, ίππος μέλας, και ο καθήμενος επ' αυτόν είχε ζυγαρίαν εν τη χειρί αυτού.
\par 6 Και ήκουσα φωνήν εν μέσω των τεσσάρων ζώων λέγουσαν· Μία χοίνιξ σίτου δι' εν δηνάριον και τρεις χοίνικες κριθής δι' εν δηνάριον, και το έλαιον και τον οίνον μη βλάψης.
\par 7 Και ότε ήνοιξε την σφραγίδα την τετάρτην, ήκουσα φωνήν του τετάρτου ζώου λέγουσαν· Έρχου και βλέπε.
\par 8 Και είδον, και ιδού, ίππος ωχρός, και ο καθήμενος επάνω αυτού ωνομάζετο θάνατος, και ο Άδης ηκολούθει μετ' αυτού· και εδόθη εις αυτούς εξουσία επί το τέταρτον της γης, να θανατώσωσι με ρομφαίαν και με πείναν και με θάνατον και με τα θηρία της γης.
\par 9 Και ότε ήνοιξε την πέμπτην σφραγίδα, είδον υποκάτω του θυσιαστηρίου τας ψυχάς των εσφαγμένων διά τον λόγον του Θεού και διά την μαρτυρίαν, την οποίαν είχον.
\par 10 Και έκραξαν μετά φωνής μεγάλης, λέγοντες· Έως πότε, ω Δέσποτα άγιε και αληθινέ, δεν κρίνεις και εκδικείς το αίμα ημών από των κατοικούντων επί της γης;
\par 11 Και εδόθησαν εις έκαστον στολαί λευκαί, και ερρέθη προς αυτούς να αναπαυθώσιν έτι ολίγον καιρόν, εωσού συμπληρωθώσι και οι σύνδουλοι αυτών και οι αδελφοί αυτών οι μέλλοντες να φονευθώσιν ως και αυτοί.
\par 12 Και είδον, ότε ήνοιξε την σφραγίδα την έκτην, και ιδού, έγεινε σεισμός μέγας, και ο ήλιος έγεινε μέλας ως σάκκος τρίχινος και η σελήνη έγεινεν ως αίμα,
\par 13 και οι αστέρες του ουρανού έπεσαν εις την γην, καθώς η συκή ρίπτει τα άωρα σύκα αυτής, σειομένη υπό μεγάλου ανέμου,
\par 14 και ο ουρανός απεχωρίσθη ως βιβλίον τυλιγμένον, και παν όρος και νήσος εκινήθησαν εκ των τόπων αυτών·
\par 15 και οι βασιλείς της γης και οι μεγιστάνες και οι πλούσιοι και οι χιλίαρχοι και οι δυνατοί και πας δούλος και πας ελεύθερος έκρυψαν εαυτούς εις τα σπήλαια και εις τας πέτρας των ορέων,
\par 16 και λέγουσι προς τα όρη και προς τας πέτρας· Πέσατε εφ' ημάς και κρύψατε ημάς από προσώπου του καθημένου επί του θρόνου και από της οργής του Αρνίου,
\par 17 διότι ήλθεν η ημέρα η μεγάλη της οργής αυτού, και τις δύναται να σταθή;

\chapter{7}

\par 1 Και μετά ταύτα είδον τέσσαρας αγγέλους ισταμένους επί τας τέσσαρας γωνίας της γης, κρατούντας τους τέσσαρας ανέμους της γης, διά να μη πνέη άνεμος επί της γης μήτε επί της θαλάσσης μήτε επί παν δένδρον.
\par 2 Και είδον άλλον άγγελον ότι ανέβη από ανατολής ηλίου, έχων σφραγίδα του Θεού του ζώντος, και έκραξε μετά φωνής μεγάλης προς τους τέσσαρας αγγέλους, εις τους οποίους εδόθη να βλάψωσι την γην και την θάλασσαν,
\par 3 λέγων· Μη βλάψητε την γην μήτε την θάλασσαν μήτε τα δένδρα, εωσού σφραγίσωμεν τους δούλους του Θεού ημών επί των μετώπων αυτών.
\par 4 Και ήκουσα τον αριθμόν των εσφραγισμένων· εκατόν τεσσαράκοντα τέσσαρες χιλιάδες ήσαν εσφραγισμένοι εκ πάσης φυλής των υιών Ισραήλ·
\par 5 εκ φυλής Ιούδα δώδεκα χιλιάδες εσφραγισμένοι· εκ φυλής Ρουβήν δώδεκα χιλιάδες εσφραγισμένοι· εκ φυλής Γαδ δώδεκα χιλιάδες εσφραγισμένοι·
\par 6 εκ φυλής Ασήρ δώδεκα χιλιάδες εσφραγισμένοι·εκ φυλής Νεφθαλείμ δώδεκα χιλιάδες εσφραγισμένοι· εκ φυλής Μανασσή δώδεκα χιλιάδες εσφραγισμένοι.
\par 7 εκ φυλής Συμεών δώδεκα χιλιάδες εσφραγισμένοι· εκ φυλής Λευΐ δώδεκα χιλιάδες εσφραγισμένοι· εκ φυλής Ισσάχαρ δώδεκα χιλιάδες εσφραγισμένοι·
\par 8 εκ φυλής Ζαβουλών δώδεκα χιλιάδες εσφραγισμένοι· εκ φυλής Ιωσήφ δώδεκα χιλιάδες εσφραγισμένοι· εκ φυλής Βενιαμίν δώδεκα χιλιάδες εσφραγισμένοι·
\par 9 Μετά ταύτα είδον, και ιδού, όχλος πολύς, τον οποίον ουδείς ηδύνατο να αριθμήση, εκ παντός έθνους και φυλών και λαών και γλωσσών, οίτινες ίσταντο ενώπιον του θρόνου και ενώπιον του Αρνίου, ενδεδυμένοι στολάς λευκάς, έχοντες φοίνικας εν ταις χερσίν αυτών.
\par 10 και κράζοντες μετά φωνής μεγάλης έλεγον· Η σωτηρία είναι του Θεού ημών, του καθημένου επί του θρόνου, και του Αρνίου.
\par 11 Και πάντες οι άγγελοι ίσταντο κύκλω του θρόνου και των πρεσβυτέρων και των τεσσάρων ζώων, και έπεσαν κατά πρόσωπον ενώπιον του θρόνου και προσεκύνησαν τον Θεόν
\par 12 λέγοντες· Αμήν· η ευλογία και δόξα και η σοφία και η ευχαριστία και η τιμή και η δύναμις και η ισχύς ανήκει εις τον Θεόν ημών εις τους αιώνας των αιώνων· αμήν.
\par 13 Και απεκρίθη εις εκ των πρεσβυτέρων, λέγων προς εμέ· Ούτοι οι ενδεδυμένοι τας στολάς τας λευκάς τίνες είναι και πόθεν ήλθον;
\par 14 Και είπα προς αυτόν· Κύριε, συ εξεύρεις. Και είπε προς εμέ· Ούτοι είναι οι ερχόμενοι εκ της θλίψεως της μεγάλης, και έπλυναν τας στολάς αυτών και ελεύκαναν αυτάς εν τω αίματι του Αρνίου.
\par 15 Διά τούτο είναι ενώπιον του θρόνου του Θεού και λατρεύουσιν αυτόν ημέραν και νύκτα εν τω ναώ αυτού, και ο καθήμενος επί του θρόνου θέλει κατασκηνώσει επ' αυτούς.
\par 16 Δεν θέλουσι πεινάσει πλέον ουδέ θέλουσι διψήσει πλέον, ουδέ θέλει πέσει επ' αυτούς ο ήλιος ουδέ κανέν καύμα,
\par 17 διότι το Αρνίον το αναμέσον του θρόνου θέλει ποιμάνει αυτούς και οδηγήσει αυτούς εις ζώσας πηγάς υδάτων, και θέλει εξαλείψει ο Θεός παν δάκρυον από των οφθαλμών αυτών.

\chapter{8}

\par 1 Και ότε ήνοιξε την σφραγίδα την εβδόμην, έγεινε σιωπή εν τω ουρανώ, έως ημίσειαν ώραν.
\par 2 Και είδον τους επτά αγγέλους, οίτινες ίσταντο ενώπιον του Θεού, και εδόθησαν εις αυτούς επτά σάλπιγγες.
\par 3 Και ήλθεν άλλος άγγελος και εστάθη έμπροσθεν του θυσιαστηρίου, κρατών θυμιατήριον χρυσούν, και εδόθησαν εις αυτόν θυμιάματα πολλά, διά να προσφέρη με τας προσευχάς πάντων των αγίων επί το θυσιαστήριον το χρυσούν το ενώπιον του θρόνου.
\par 4 Και ανέβη ο καπνός των θυμιαμάτων με τας προσευχάς πάντων των αγίων εκ της χειρός του αγγέλου ενώπιον του Θεού.
\par 5 Και έλαβεν ο άγγελος το θυμιατήριον και εγέμισεν αυτό εκ του πυρός του θυσιαστηρίου και έρριψεν εις την γην. Και έγειναν φωναί και βρονταί και αστραπαί και σεισμός.
\par 6 Και οι επτά άγγελοι, οι έχοντες τας επτά σάλπιγγας, ητοίμασαν εαυτούς διά να σαλπίσωσι.
\par 7 Και ο πρώτος άγγελος εσάλπισε, και έγεινε χάλαζα και πυρ μεμιγμένα με αίμα, και ερρίφθησαν εις την γήν· και το τρίτον των δένδρων κατεκάη και πας χλωρός χόρτος κατεκάη.
\par 8 Και ο δεύτερος άγγελος εσάλπισε, και ως όρος μέγα καιόμενον με πυρ ερρίφθη εις την θάλασσαν, και το τρίτον της θαλάσσης έγεινεν αίμα,
\par 9 και απέθανε το τρίτον των εμψύχων κτισμάτων των εν τη θαλάσση και το τρίτον των πλοίων διεφθάρη.
\par 10 Και ο τρίτος άγγελος εσάλπισε, και έπεσεν εκ του ουρανού αστήρ μέγας καιόμενος ως λαμπάς, και έπεσεν επί το τρίτον των ποταμών, και επί τας πηγάς των υδάτων·
\par 11 και το όνομα του αστέρος λέγεται Αψινθος· και έγεινε το τρίτον των υδάτων άψινθος, και πολλοί άνθρωποι απέθανον εκ των υδάτων, διότι επικράνθησαν.
\par 12 Και ο τέταρτος άγγελος εσάλπισε, και εκτυπήθη το τρίτον του ηλίου και το τρίτον της σελήνης και το τρίτον των αστέρων, διά να σκοτισθή το τρίτον αυτών, και η ημέρα να χάση το τρίτον του φωτισμού αυτής, και η νυξ ομοίως.
\par 13 Και είδον και ήκουσα ένα άγγελον πετώμενον εις το μεσουράνημα, όστις έλεγε μετά φωνής μεγάλης· Ουαί, ουαί, ουαί εις τους κατοικούντας επί της γης διά τας λοιπάς φωνάς της σάλπιγγος των τριών αγγέλων των μελλόντων να σαλπίσωσι.

\chapter{9}

\par 1 Και ο πέμπτος άγγελος εσάλπισε· και είδον ότι έπεσεν εις την γην αστήρ εκ του ουρανού, και εδόθη εις αυτόν το κλειδίον του φρέατος της αβύσσου.
\par 2 Και ήνοιξε το φρέαρ της αβύσσου, και ανέβη καπνός εκ του φρέατος ως καπνός καμίνου μεγάλης, και εσκοτίσθη ο ήλιος και ο αήρ εκ του καπνού του φρέατος.
\par 3 Και εκ του καπνού εξήλθον ακρίδες εις την γην, και εδόθη εις αυτάς εξουσία ως έχουσιν εξουσίαν οι σκορπίοι της γής·
\par 4 και ερρέθη προς αυτάς να μη βλάψωσι τον χόρτον της γης μηδέ κανέν χλωρόν μηδέ κανέν δένδρον, ειμή τους ανθρώπους μόνους, οίτινες δεν έχουσι την σφραγίδα του Θεού επί των μετώπων αυτών.
\par 5 Και εδόθη εις αυτάς να μη θανατώσωσιν αυτούς, αλλά να βασανισθώσι πέντε μήνας· και ο βασανισμός αυτών ήτο ως βασανισμός σκορπίου, όταν κτυπήση άνθρωπον.
\par 6 Και εν ταις ημέραις εκείναις θέλουσι ζητήσει οι άνθρωποι τον θάνατον και δεν θέλουσιν ευρεί αυτόν, και θέλουσιν επιθυμήσει να αποθάνωσι, και ο θάνατος θέλει φύγει απ' αυτών.
\par 7 Και αι μορφαί των ακρίδων ήσαν όμοιαι με ίππους ητοιμασμένους εις πόλεμον, και επί τας κεφαλάς αυτών ήσαν ως στέφανοι όμοιοι με χρυσόν, και τα πρόσωπα αυτών ως πρόσωπα ανθρώπων.
\par 8 Και είχον τρίχας ως τρίχας γυναικών, και οι οδόντες αυτών ήσαν ως λεόντων,
\par 9 και είχον θώρακας ως θώρακας σιδηρούς, και η φωνή των πτερύγων αυτών ήτο ως φωνή αμαξών ίππων πολλών τρεχόντων εις πόλεμον.
\par 10 Και είχον ουράς ομοίας με σκορπίους και ήσαν κέντρα εις τας ουράς αυτών, και η εξουσία αυτών ήτο να βλάψωσι τους ανθρώπους πέντε μήνας.
\par 11 Και είχον εφ' εαυτών βασιλέα τον άγγελον της αβύσσου, όστις Εβραϊστί ονομάζεται Αβαδδών, και εις την Ελληνικήν έχει όνομα Απολλύων.
\par 12 Η ουαί η μία απήλθεν· ιδού, έρχονται έτι δύο ουαί μετά ταύτα.
\par 13 Και ο έκτος άγγελος εσάλπισε· και ήκουσα μίαν φωνήν εκ των τεσσάρων κεράτων του θυσιαστηρίου του χρυσού του ενώπιον του Θεού,
\par 14 λέγουσαν προς τον έκτον άγγελον, όστις είχε την σάλπιγγα· Λύσον τους τέσσαρας αγγέλους τους δεδεμένους εις τον μέγαν ποταμόν Ευφράτην.
\par 15 Και ελύθησαν οι τέσσαρες άγγελοι, οι ητοιμασμένοι εις την ώραν και ημέραν και μήνα και ενιαυτόν, διά να θανατώσωσι το τρίτον των ανθρώπων.
\par 16 Και ο αριθμός των στρατευμάτων του ιππικού ήτο δύο μυριάδες μυριάδων· και ήκουσα τον αριθμόν αυτών.
\par 17 Και ούτως είδον τους ίππους εν τη οράσει και τους καθημένους επ' αυτών, ότι είχον θώρακας πυρίνους και υακινθίνους και θειώδεις· και αι κεφαλαί των ίππων ήσαν ως κεφαλαί λεόντων, και εκ των στομάτων αυτών εξήρχετο πυρ και καπνός και θείον.
\par 18 Υπό των τριών τούτων εθανατώθησαν το τρίτον των ανθρώπων· εκ του πυρός και εκ του καπνού και εκ του θείου του εξερχομένου εκ των στομάτων αυτών.
\par 19 Διότι αι εξουσίαι αυτών είναι εν τω στόματι αυτών, επειδή αι ουραί αυτών είναι όμοιαι με όφεις, έχουσαι κεφαλάς, και με αυτάς βλάπτουσι.
\par 20 Και οι λοιποί των ανθρώπων, οίτινες δεν εθανατώθησαν με τας πληγάς ταύτας, ούτε μετενόησαν από των έργων των χειρών αυτών, ώστε να μη προσκυνήσωσι τα δαιμόνια και τα είδωλα τα χρυσά και τα αργυρά και τα χάλκινα και τα λίθινα και τα ξύλινα, τα οποία ούτε να βλέπωσι δύνανται ούτε να ακούωσιν ούτε να περιπατώσι,
\par 21 και δεν μετενόησαν εκ των φόνων αυτών ούτε εκ των φαρμακειών αυτών ούτε εκ της πορνείας αυτών ούτε εκ των κλοπών αυτών.

\chapter{10}

\par 1 Και είδον άλλον άγγελον ισχυρόν καταβαίνοντα εκ του ουρανού, ενδεδυμένον νεφέλην, και ήτο ίρις επί της κεφαλής αυτού, και το πρόσωπον αυτού ως ο ήλιος, και οι πόδες αυτού ως στύλοι πυρός,
\par 2 και είχεν εν τη χειρί αυτού βιβλιάριον ανεωγμένον. Και έθεσε τον πόδα αυτού τον δεξιόν επί την θάλασσαν, τον δε αριστερόν επί την γην,
\par 3 και έκραξε μετά φωνής μεγάλης καθώς βρυχάται ο λέων. Και ότε έκραξεν, ελάλησαν αι επτά βρονταί τας εαυτών φωνάς.
\par 4 Και ότε ελάλησαν αι επτά βρονταί τας φωνάς εαυτών, έμελλον να γράφω· και ήκουσα φωνήν εκ του ουρανού λέγουσαν προς εμέ. Σφράγισον εκείνα, τα οποία ελάλησαν αι επτά βρονταί, και μη γράψης ταύτα.
\par 5 Και ο άγγελος, τον οποίον είδον ιστάμενον επί της θαλάσσης και επί της γης, εσήκωσε την χείρα αυτού εις τον ουρανόν
\par 6 και ώμοσεν εις τον ζώντα εις τους αιώνας των αιώνων, όστις έκτισε τον ουρανόν και τα εν αυτώ, και την γην και τα εν αυτή και την θάλασσαν και τα εν αυτή, ότι καιρός δεν θέλει είσθαι έτι,
\par 7 αλλ' εν ταις ημέραις της φωνής του εβδόμου αγγέλου, όταν μέλλη να σαλπίση, τότε θέλει τελεσθή το μυστήριον του Θεού, καθώς εφανέρωσε προς τους εαυτού δούλους τους προφήτας.
\par 8 Και η φωνή, την οποίαν ήκουσα εκ του ουρανού, πάλιν ελάλει μετ' εμού και έλεγεν· Ύπαγε, λάβε το βιβλιάριον το ανεωγμένον εν τη χειρί του αγγέλου του ισταμένου επί της θαλάσσης και επί της γης.
\par 9 Και υπήγα προς τον άγγελον, λέγων προς αυτόν, Δος μοι το βιβλιάριον. Και λέγει προς εμέ· Λάβε και κατάφαγε αυτό, και θέλει πικράνει την κοιλίαν σου, πλην εν τω στόματί σου θέλει είσθαι γλυκύ ως μέλι.
\par 10 Και έλαβον το βιβλιάριον εκ της χειρός του αγγέλου και κατέφαγον αυτό· και ήτο εν τω στόματί μου ως μέλι γλυκύ· και ότε έφαγον αυτό επικράνθη η κοιλία μου.
\par 11 Και μοι λέγει· Πρέπει πάλιν να προφητεύσης περί λαών και εθνών και γλωσσών και βασιλέων πολλών.

\chapter{11}

\par 1 Και μοι εδόθη κάλαμος όμοιος με ράβδον, και ο άγγελος ίστατο λέγων· Σηκώθητι και μέτρησον τον ναόν του Θεού και το θυσιαστήριον και τους προσκυνούντας εν αυτώ.
\par 2 Την αυλήν όμως την έξωθεν του ναού άφες έξω και μη μετρήσης αυτήν, διότι εδόθη εις τα έθνη, και την πόλιν την αγίαν θέλουσι πατήσει τεσσαράκοντα δύο μήνας.
\par 3 Και θέλω δώσει εις τους δύο μάρτυράς μου να προφητεύσωσι χιλίας διακοσίας εξήκοντα ημέρας, ενδεδυμένοι σάκκους.
\par 4 Ούτοι είναι αι δύο ελαίαι και αι δύο λυχνίαι, αι ιστάμεναι ενώπιον του Θεού της γης.
\par 5 Και εάν τις θέλη να βλάψη αυτούς, εξέρχεται πυρ εκ του στόματος αυτών και κατατρώγει τους εχθρούς αυτών· και εάν τις θέλη να βλάψη αυτούς, ούτω πρέπει αυτός να θανατωθή.
\par 6 Ούτοι έχουσιν εξουσίαν να κλείσωσι τον ουρανόν, διά να μη βρέχη βροχή εν ταις ημέραις της προφητείας αυτών, και έχουσιν εξουσίαν επί των υδάτων να μεταβάλλωσιν αυτά εις αίμα και να πατάξωσι την γην με πάσαν πληγήν, οσάκις εάν θελήσωσι.
\par 7 Και όταν τελειώσωσι την μαρτυρίαν αυτών, το θηρίον το αναβαίνον εκ της αβύσσου θέλει κάμει πόλεμον με αυτούς και θέλει νικήσει αυτούς και θανατώσει αυτούς.
\par 8 Και τα πτώματα αυτών θέλουσι κείσθαι επί της πλατείας της πόλεως της μεγάλης, ήτις καλείται πνευματικώς Σόδομα και Αίγυπτος, όπου και ο Κύριος ημών εσταυρώθη.
\par 9 Και οι άνθρωποι εκ των λαών και φυλών και γλωσσών και εθνών θέλουσι βλέπει τα πτώματα αυτών ημέρας τρεις και ήμισυ, και δεν θέλουσιν αφήσει τα πτώματα αυτών να τεθώσιν εις μνήματα.
\par 10 Και οι κατοικούντες επί της γης θέλουσι χαρή δι' αυτούς και ευφρανθή και θέλουσι πέμψει δώρα προς αλλήλους, διότι ούτοι οι δύο προφήται εβασάνισαν τους κατοικούντας επί της γης.
\par 11 Και μετά τας τρεις ημέρας και ήμισυ εισήλθεν εις αυτούς πνεύμα ζωής εκ του Θεού, και εστάθησαν επί τους πόδας αυτών, και φόβος μέγας έπεσεν επί τους θεωρούντας αυτούς.
\par 12 Και ήκουσαν φωνήν μεγάλην εκ του ουρανού, λέγουσαν προς αυτούς· Ανάβητε εδώ. Και ανέβησαν εις τον ουρανόν εν τη νεφέλη, και είδον αυτούς οι εχθροί αυτών.
\par 13 Και κατ' εκείνην την ώραν έγεινε σεισμός μέγας, και έπεσε το δέκατον της πόλεως, και εθανατώθησαν εν τω σεισμώ ονόματα ανθρώπων χιλιάδες επτά, και οι λοιποί έγειναν έμφοβοι και έδωκαν δόξαν εις τον Θεόν του ουρανού.
\par 14 Η ουαί η δευτέρα απήλθεν· ιδού, η ουαί η τρίτη έρχεται ταχέως.
\par 15 Και ο έβδομος άγγελος εσάλπισε και έγειναν φωναί μεγάλαι εν τω ουρανώ, λέγουσαι· Αι βασιλείαι του κόσμου έγειναν του Κυρίου ημών και του Χριστού αυτού, και θέλει βασιλεύσει εις τους αιώνας των αιώνων.
\par 16 Και οι εικοσιτέσσαρες πρεσβύτεροι, οι καθήμενοι ενώπιον του Θεού επί τους θρόνους αυτών, έπεσαν κατά πρόσωπον αυτών και προσεκύνησαν τον Θεόν,
\par 17 λέγοντες· Ευχαριστούμέν σοι, Κύριε Θεέ παντοκράτωρ, ο ων και ο ην και ο ερχόμενος, διότι έλαβες την δύναμίν σου την μεγάλην και εβασίλευσας,
\par 18 και τα έθνη ωργίσθησαν, και ήλθεν η οργή σου και ο καιρός των νεκρών διά να κριθώσι και να δώσης τον μισθόν εις τους δούλους σου τους προφήτας και εις τους αγίους και εις τους φοβουμένους το όνομά σου, τους μικρούς και τους μεγάλους, και να διαφθείρης τους διαφθείροντας την γην.
\par 19 Και ηνοίχθη ο ναός του Θεού εν τω ουρανώ, και εφάνη η κιβωτός της διαθήκης αυτού εν τω ναώ αυτού, και έγειναν αστραπαί και φωναί και βρονταί και σεισμός και χάλαζα μεγάλη.

\chapter{12}

\par 1 Και σημείον μέγα εφάνη εν τω ουρανώ, γυνή ενδεδυμένη τον ήλιον, και η σελήνη υποκάτω των ποδών αυτής, και επί της κεφαλής αυτής στέφανος αστέρων δώδεκα·
\par 2 και έγκυος ούσα έκραζε κοιλοπονούσα και βασανιζομένη διά να γεννήση.
\par 3 Και εφάνη άλλο σημείον εν τω ουρανώ, και ιδού, δράκων μέγας κόκκινος, έχων κεφαλάς επτά και κέρατα δέκα, και επί τας κεφαλάς αυτού διαδήματα επτά,
\par 4 και η ουρά αυτού έσυρε το τρίτον των αστέρων του ουρανού και έρριψεν αυτούς εις την γην. Και ο δράκων εστάθη ενώπιον της γυναικός της μελλούσης να γεννήση, διά να καταφάγη το τέκνον αυτής, όταν γεννήση.
\par 5 Και εγέννησε παιδίον άρρεν, το οποίον μέλλει να ποιμάνη πάντα τα έθνη εν ράβδω σιδηρά· και το τέκνον αυτής ηρπάσθη προς τον Θεόν και τον θρόνον αυτού.
\par 6 Και η γυνή έφυγεν εις την έρημον, όπου έχει τόπον ητοιμασμένον από του Θεού, διά να τρέφωσιν αυτήν εκεί ημέρας χιλίας διακοσίας εξήκοντα.
\par 7 Και έγεινε πόλεμος εν τω ουρανώ· ο Μιχαήλ και οι άγγελοι αυτού επολέμησαν κατά του δράκοντος· και ο Δράκων επολέμησε και οι άγγελοι αυτού,
\par 8 και δεν υπερίσχυσαν, ουδέ ευρέθη πλέον τόπος αυτών εν τω ουρανώ.
\par 9 Και ερρίφθη ο δράκων ο μέγας, ο όφις ο αρχαίος, ο καλούμενος Διάβολος και ο Σατανάς, ο πλανών την οικουμένην όλην, ερρίφθη εις την γην, και οι άγγελοι αυτού ερρίφθησαν μετ' αυτού.
\par 10 Και ήκουσα φωνήν μεγάλην λέγουσαν εν τω ουρανώ· Τώρα έγεινεν η σωτηρία και δύναμις και η βασιλεία του Θεού ημών και η εξουσία τον Χριστού αυτού, διότι κατερρίφθη ο κατήγορος των αδελφών ημών, ο κατηγορών αυτούς ενώπιον του Θεού ημών ημέραν και νύκτα.
\par 11 Και αυτοί ενίκησαν αυτόν διά το αίμα του Αρνίου και διά τον λόγον της μαρτυρίας αυτών, και δεν ηγάπησαν την ψυχήν αυτών μέχρι θανάτου.
\par 12 Διά τούτο ευφραίνεσθε οι ουρανοί και οι κατοικούντες εν αυτοίς· ουαί εις τους κατοικούντας την γην και την θάλασσαν, διότι κατέβη ο διάβολος εις εσάς έχων θυμόν μέγαν, επειδή γνωρίζει ότι ολίγον καιρόν έχει.
\par 13 Και ότε είδεν ο δράκων ότι ερρίφθη εις την γην, εδίωξε την γυναίκα, ήτις εγέννησε τον άρρενα.
\par 14 Και εδόθησαν εις την γυναίκα δύο πτέρυγες του αετού του μεγάλου, διά να πετά εις την έρημον εις τον τόπον αυτής, όπου τρέφεται εκεί καιρόν και καιρούς και ήμισυ καιρού από προσώπου του όφεως.
\par 15 Και έρριψεν ο όφις οπίσω της γυναικός εκ του στόματος αυτού ύδωρ ως ποταμόν, διά να κάμη να σύρη αυτήν ο ποταμός.
\par 16 Και εβοήθησεν την γυναίκα και ήνοιξεν η γη, το στόμα αυτής και κατέπιε τον ποταμόν, τον οποίον έρριψεν ο δράκων εκ του στόματος αυτού.
\par 17 Και ωργίσθη ο δράκων κατά της γυναικός και υπήγε να κάμη πόλεμον με τους λοιπούς του σπέρματος αυτής, τους φυλάττοντας τας εντολάς του Θεού και έχοντας την μαρτυρίαν του Ιησού Χριστού.

\chapter{13}

\par 1 Και εστάθην επί την άμμον της θαλάσσης· και είδον θηρίον αναβαίνον εκ της θαλάσσης, το οποίον είχε κεφαλάς επτά και κέρατα δέκα, και επί των κεράτων αυτού δέκα διαδήματα και επί τας κεφαλάς αυτού όνομα βλασφημίας.
\par 2 Και το θηρίον, το οποίον είδον, ήτο όμοιον με πάρδαλιν, και οι πόδες αυτού ως άρκτου, και το στόμα αυτού ως στόμα λέοντος· και έδωκεν εις αυτό ο δράκων την δύναμιν αυτού και τον θρόνον αυτού και εξουσίαν μεγάλην.
\par 3 Και είδον μίαν των κεφαλών αυτού ως πεπληγωμένην θανατηφόρως· και η θαναφόρος πληγή αυτού εθεραπεύθη, και εθαύμασεν όλη η γη οπίσω του θηρίου,
\par 4 και προσεκύνησαν τον δράκοντα, όστις έδωκεν εξουσίαν εις το θηρίον, και προσεκύνησαν το θηρίον, λέγοντες· Τις όμοιος με το θηρίον; τις δύναται να πολεμήση με αυτό;
\par 5 και εδόθη εις αυτό στόμα λαλούν μεγάλα και βλαφημίας· και εδόθη εις αυτό εξουσία να κάμη πόλεμον τεσσαράκοντα δύο μήνας.
\par 6 Και ήνοιξε το στόμα αυτού εις βλαφημίαν εναντίον του Θεού, να βλαφημήση το όνομα αυτού και την σκηνήν αυτού και τους κατοικούντας εν τω ουρανώ.
\par 7 Και εδόθη εις αυτό να κάμη πόλεμον με τους αγίους, και να νικήση αυτούς, και εδόθη εις αυτό εξουσία επί πάσαν φυλήν και γλώσσαν και έθνος.
\par 8 Και θέλουσι προσκυνήσει αυτό πάντες οι κατοικούντες επί της γης, των οποίων τα ονόματα δεν εγράφησαν εν τω βιβλίω της ζωής του Αρνίου του εσφαγμένου από καταβολής κόσμου.
\par 9 Όστις έχει ωτίον, ας ακούση.
\par 10 Όστις φέρει εις αιχμαλωσίαν, εις αιχμαλωσίαν υπάγει. Όστις φονεύση με μάχαιραν, πρέπει αυτός να φονευθή με μάχαιραν. Εδώ είναι η υπομονή και η πίστις των αγίων.
\par 11 Και είδον άλλο θηρίον αναβαίνον εκ της γης, και είχε κέρατα δύο όμοια με αρνίου, και ελάλει ως δράκων.
\par 12 Και ενήργει όλην την εξουσίαν του πρώτου θηρίου ενώπιον αυτού. Και έκαμε την γην και τους κατοικούντας εν αυτή να προσκυνήσωσι το θηρίον το πρώτον, του οποίου εθεραπεύθη η θανατηφόρος πληγή.
\par 13 Και έκαμνε σημεία μεγάλα, ώστε και πυρ έκαμνε να καταβαίνη εκ του ουρανού εις την γην ενώπιον των ανθρώπων.
\par 14 Και επλάνα τους κατοικούντας επί της γης διά τα σημεία, τα οποία εδόθησαν εις αυτό να κάμη ενώπιον του θηρίου, λέγον προς τους κατοικούντας επί της γης να κάμωσιν εικόνα εις το θηρίον, το οποίον έχει την πληγήν της μαχαίρας και έζησε.
\par 15 Και εδόθη εις αυτό να δώση πνεύμα εις την εικόνα του θηρίου, ώστε και να λαλήση η εικών του θηρίου και να κάμη, όσοι δεν προσκυνήσωσι την εικόνα του θηρίου, να θανατωθώσι.
\par 16 Και έκαμνε πάντας, τους μικρούς και τους μεγάλους και τους πλουσίους και τους πτωχούς και τους ελευθέρους και τους δούλους, να λάβωσι χάραγμα επί της χειρός αυτών της δεξιάς ή επί των μετώπων αυτών,
\par 17 και να μη δύναται μηδείς να αγοράση ή να πωλήση, ειμή ο έχων το χάραγμα, ή το όνομα του θηρίου ή τον αριθμόν του ονόματος αυτού.
\par 18 Εδώ είναι η σοφία· όστις έχει τον νούν, ας λογαριάση τον αριθμόν του θηρίου, διότι είναι αριθμός ανθρώπου· και ο αριθμός αυτού είναι χξς'.

\chapter{14}

\par 1 Και είδον, και ιδού Αρνίον ιστάμενον επί το όρος Σιών, και μετ' αυτού εκατόν τεσσαράκοντα τέσσαρες χιλιάδες, έχουσαι το όνομα του Πατρός αυτού γεγραμμένον επί των μετώπων αυτών.
\par 2 Και ήκουσα φωνήν εκ του ουρανού ως φωνήν υδάτων πολλών και ως φωνήν βροντής μεγάλης· και ήκουσα φωνήν κιθαρωδών οίτινες εκιθάριζον με τας κιθάρας αυτών.
\par 3 Και έψαλλον ως ωδήν νέαν ενώπιον του θρόνου και ενώπιον των τεσσάρων ζώων και των πρεσβυτέρων· και ουδείς ηδύνατο να μάθη την ωδήν, ειμή αι εκατόν τεσσαράκοντα τέσσαρες χιλιάδες, οι ηγορασμένοι από της γης.
\par 4 Ούτοι είναι οι μη μολυνθέντες με γυναίκας· διότι παρθένοι είναι. Ούτοι είναι οι ακολουθούντες το Αρνίον όπου αν υπάγη. Ούτοι ηγοράσθησαν από των ανθρώπων απαρχή εις τον Θεόν και εις το Αρνίον·
\par 5 και εν τω στόματι αυτών δεν ευρέθη δόλος, διότι είναι άμωμοι ενώπιον του θρόνου Θεού,
\par 6 Και είδον άλλον άγγελον πετώμενον εις το μεσουράνημα, όστις είχεν ευαγγέλιον αιώνιον, διά να κηρύξη εις τους κατοικούντας επί της γης και εις παν έθνος και φυλήν και γλώσσαν και λαόν,
\par 7 και έλεγε μετά φωνής μεγάλης· Φοβήθητε τον Θεόν και δότε δόξαν εις αυτόν, διότι ήλθεν η ώρα της κρίσεως αυτού, και προσκυνήσατε τον ποιήσαντα τον ουρανόν και την γην και την θάλασσαν και τας πηγάς των υδάτων.
\par 8 Και άλλος άγγελος ηκολούθησε, λέγων· Έπεσεν, έπεσε Βαβυλών η πόλις η μεγάλη, διότι εκ του οίνου του θυμού της πορνείας αυτής επότισε πάντα τα έθνη.
\par 9 Και τρίτος άγγελος ηκολούθησεν αυτούς, λέγων μετά φωνής μεγάλης· Όστις προσκυνεί το θηρίον και την εικόνα αυτού και λαμβάνει χάραγμα επί του μετώπου αυτού ή επί της χειρός αυτού,
\par 10 και αυτός θέλει πίει εκ του οίνου του θυμού του Θεού του κεκερασμένου ακράτου εν τω ποτηρίω της οργής αυτού, και θέλει βασανισθή με πυρ και θείον ενώπιον των αγίων αγγέλων και ενώπιον του Αρνίου.
\par 11 Και ο καπνός του βασανισμού αυτών αναβαίνει εις αιώνας αιώνων, και δεν έχουσιν ανάπαυσιν ημέραν και νύκτα όσοι προσκυνούσι το θηρίον και την εικόνα αυτού και όστις λαμβάνει το χάραγμα του ονόματος αυτού.
\par 12 Εδώ είναι η υπομονή των αγίων, εδώ οι φυλάττοντες τας εντολάς του Θεού και την πίστιν του Ιησού.
\par 13 Και ήκουσα φωνήν εκ του ουρανού λέγουσαν προς εμέ· Γράψον, Μακάριοι οι νεκροί, οίτινες αποθνήσκουσιν εν Κυρίω από του νυν. Ναι, λέγει το Πνεύμα, διά να αναπαυθώσιν από των κόπων αυτών, και τα έργα αυτών ακολουθούσι με αυτούς.
\par 14 Και είδον, και ιδού, νεφέλη λευκή, και επί της νεφέλης εκάθητό τις όμοιος με υιόν ανθρώπου, έχων επί της κεφαλής αυτού στέφανον χρυσούν και εν τη χειρί αυτού δρέπανον κοπτερόν.
\par 15 Και άλλος άγγελος εξήλθεν εκ του ναού, κράζων μετά μεγάλης φωνής προς τον καθήμενον επί της νεφέλης. Πέμψον το δρέπανόν σου και θέρισον, διότι ήλθεν εις σε η ώρα του να θερίσης, επειδή εξηράνθη ο θερισμός της γης.
\par 16 Και ο καθήμενος επί της νεφέλης έβαλε το δρέπανον αυτού επί την γην, και εθερίσθη η γη.
\par 17 Και άλλος άγγελος εξήλθεν εκ του ναού του εν τω ουρανώ, έχων και αυτός δρέπανον κοπτερόν.
\par 18 Και άλλος άγγελος εξήλθεν εκ του θυσιαστηρίου, έχων εξουσίαν επί του πυρός, και εφώναξε μετά κραυγής μεγάλης προς τον έχοντα το δρέπανον το κοπτερόν, λέγων· Πέμψον το δρέπανόν σου το κοπτερόν και τρύγησον τους βότρυας της αμπέλου της γης, διότι ωρίμασαν τα σταφύλια αυτής.
\par 19 Και έβαλεν ο άγγελος το δρέπανον αυτού εις την γην και ετρύγησε την άμπελον της γης και έρριψε τα τρυγηθέντα εις τον μεγάλον ληνόν του θυμού του Θεού.
\par 20 Και επατήθη ο ληνός έξω της πόλεως, και εξήλθεν αίμα εκ του ληνού έως των χαλινών των ίππων εις διάστημα χιλίων εξακοσίων σταδίων.

\chapter{15}

\par 1 Και είδον άλλο σημείον εν τω ουρανώ μέγα και θαυμαστόν, αγγέλους επτά, οίτινες είχον τας επτά εσχάτας πληγάς, διότι εν αυταίς ετελέσθη ο θυμός του Θεού.
\par 2 Και είδον ως θάλασσαν υαλίνην μεμιγμένην με πυρ, και εκείνους οίτινες ενίκησαν κατά του θηρίου και κατά της εικόνος αυτού και κατά του χαράγματος αυτού και κατά του αριθμού του ονόματος αυτού ισταμένους επί την θάλασσαν την υαλίνην, έχοντας κιθάρας του Θεού.
\par 3 Και έψαλλον την ωδήν Μωϋσέως του δούλου του Θεού και την ωδήν του Αρνίου, λέγοντες· Μεγάλα και θαυμαστά τα έργα σου, Κύριε Θεέ παντοκράτωρ· δίκαιαι και αληθιναί αι οδοί σου, βασιλεύ των αγίων.
\par 4 Τις δεν θέλει σε φοβηθή, Κύριε, και δοξάσει το όνομά σου; διότι είσαι μόνος όσιος, διότι πάντα τα έθνη θέλουσιν ελθεί και προσκυνήσει ενώπιόν σου, διότι αι κρίσεις σου εφανερώθησαν.
\par 5 Και μετά ταύτα είδον και ιδού, ηνοίχθη ο ναός της σκηνής του μαρτυρίου εν τω ουρανώ,
\par 6 και εξήλθον εκ του ναού οι επτά άγγελοι, έχοντες τας επτά πληγάς, ενδεδυμένοι λινά καθαρά και λαμπρά και περιεζωσμένοι περί τα στήθη ζώνας χρυσάς.
\par 7 Και εν εκ των τεσσάρων ζώων έδωκεν εις τους επτά αγγέλους επτά φιάλας χρυσάς, πλήρεις του θυμού του Θεού του ζώντος εις τους αιώνας των αιώνων.
\par 8 Και εγεμίσθη ο ναός από καπνού εκ της δόξης του Θεού και εκ της δυνάμεως αυτού· και ουδείς ηδύνατο να εισέλθη εις τον ναόν, εωσού τελειώσωσιν αι επτά πληγαί των επτά αγγέλων.

\chapter{16}

\par 1 Και ήκουσα φωνήν μεγάλην εκ του ναού λέγουσαν προς τους επτά αγγέλους· Υπάγετε και εκχέατε εις την γην τας φιάλας του θυμού του Θεού.
\par 2 Και υπήγεν ο πρώτος και εξέχεε την φιάλην αυτού επί την γήν· και έγεινεν έλκος κακόν και πονηρόν εις τους ανθρώπους, τους έχοντας το χάραγμα του θηρίου και τους προσκυνούντας την εικόνα αυτού.
\par 3 Και ο δεύτερος άγγελος εξέχεε την φιάλην αυτού εις την θάλασσαν· και έγεινεν αίμα ως νεκρού, και πάσα ψυχή ζώσα απέθανεν εν τη θαλάσση.
\par 4 Και ο τρίτος άγγελος εξέχεε την φιάλην αυτού εις τους ποταμούς και εις τας πηγάς των υδάτων· και έγεινεν αίμα.
\par 5 Και ήκουσα τον άγγελον των υδάτων λέγοντα· Δίκαιος είσαι, Κύριε, ο ων και ο ην και ο όσιος, διότι έκρινας ταύτα·
\par 6 επειδή αίμα αγίων και προφητών εξέχεαν, και αίμα έδωκας εις αυτούς να πίωσι· διότι άξιοι είναι.
\par 7 Και ήκουσα άλλον εκ του θυσιαστηρίου λέγοντα· Ναι, Κύριε Θεέ παντοκράτωρ, αληθιναί και δίκαιαι αι κρίσεις σου.
\par 8 Και ο τέταρτος άγγελος εξέχεε την φιάλην αυτού επί τον ήλιον· και εδόθη εις αυτόν να καυματίση τους ανθρώπους με πυρ.
\par 9 Και εκαυματίσθησαν οι άνθρωποι καύμα μέγα, και εβλασφήμησαν το όνομα του Θεού του έχοντος εξουσίαν επί τας πληγάς ταύτας, και δεν μετενόησαν ώστε να δώσωσι δόξαν εις αυτόν.
\par 10 Και ο πέμπτος άγγελος εξέχεε την φιάλην αυτού επί τον θρόνον του θηρίου· και έγεινεν η βασιλεία αυτού πλήρης σκότους, και εμάσσουν τας γλώσσας αυτών εκ του πόνου,
\par 11 και εβλασφήμησαν τον Θεόν του ουρανού διά τους πόνους αυτών και διά τα έλκη αυτών, και δεν μετενόησαν από των έργων αυτών.
\par 12 Και ο έκτος άγγελος εξέχεε την φιάλην αυτού επί τον ποταμόν τον μέγαν τον Ευφράτην· και εξηράνθη το ύδωρ αυτού, διά να ετοιμασθή η οδός των βασιλέων των από ανατολών ηλίου.
\par 13 Και είδον τρία ακάθαρτα πνεύματα όμοια με βατράχους εξερχόμενα εκ του στόματος του δράκοντος και εκ του στόματος του θηρίου και εκ του στόματος του ψευδοπροφήτου·
\par 14 διότι είναι πνεύματα δαιμόνων εκτελούντα σημεία, τα οποία εκπορεύονται προς τους βασιλείς της γης και της οικουμένης όλης, διά να συνάξωσιν αυτούς εις τον πόλεμον της ημέρας εκείνης της μεγάλης του Θεού του παντοκράτορος.
\par 15 Ιδού, έρχομαι ως κλέπτης· μακάριος όστις αγρυπνεί και φυλάττει τα ιμάτια αυτού, διά να μη περιπατή γυμνός και βλέπωσι την ασχημοσύνην αυτού.
\par 16 Και συνήθροισεν αυτούς εις τον τόπον τον καλούμενον Εβραϊστί Αρμαγεδδών.
\par 17 Και ο έβδομος άγγελος εξέχεε την φιάλην αυτού εις τον αέρα· και εξήλθε φωνή μεγάλη από του ναού του ουρανού από του θρόνου, λέγουσα· Ετελέσθη.
\par 18 Και έγειναν φωναί και βρονταί και αστραπαί, και έγεινε σεισμός μέγας, οποίος δεν έγεινεν αφού οι άνθρωποι υπήρξαν επί της γης, τόσον πολλά μεγάλος σεισμός.
\par 19 Και διηρέθη η πόλις η μεγάλη εις τρία μέρη, και αι πόλεις των εθνών έπεσον. Και Βαβυλών η μεγάλη ήλθεν εις ενθύμησιν ενώπιον του Θεού διά να δώση εις αυτήν το ποτήριον του οίνου του θυμού της οργής αυτού.
\par 20 Και πάσα νήσος έφυγε και τα όρη δεν ευρέθησαν.
\par 21 Και χάλαζα μεγάλη έως ενός ταλάντου κατέβαινεν εκ του ουρανού επί τους ανθρώπους· και εβλασφήμησαν οι άνθρωποι τον Θεόν διά την πληγήν της χαλάζης, διότι η πληγή αυτής ήτο μεγάλη σφόδρα.

\chapter{17}

\par 1 Και ήλθεν εις εκ των επτά αγγέλων των εχόντων τας επτά φιάλας, και ελάλησε μετ' εμού, λέγων μοι· Ελθέ, θέλω σοι δείξει την κρίσιν της πόρνης της μεγάλης της καθημένης επί των υδάτων των πολλών,
\par 2 μετά της οποίας επόρνευσαν οι βασιλείς της γης και εμεθύσθησαν οι κατοικούντες την γην εκ του οίνου της πορνείας αυτής.
\par 3 Και με έφερεν εν πνεύματι εις έρημον. Και είδον γυναίκα καθημένην επί θηρίον κόκκινον, γέμον ονομάτων βλασφημίας, έχον κεφαλάς επτά και κέρατα δέκα.
\par 4 Και η γυνή ήτο ενδεδυμένη πορφύραν και κόκκινον και κεχρυσωμένη με χρυσόν και λίθους τιμίους και μαργαρίτας, έχουσα εν τη χειρί αυτής χρυσούν ποτήριον γέμον βδελυγμάτων και ακαθαρσίας της πορνείας αυτής,
\par 5 και επί το μέτωπον αυτής ήτο όνομα γεγραμμένον· Μυστήριον, Βαβυλών η μεγάλη, η μήτηρ των πορνών και των βδελυγμάτων της γης.
\par 6 Και είδον την γυναίκα μεθύουσαν εκ του αίματος των αγίων και εκ του αίματος των μαρτύρων του Ιησού. Και ιδών αυτήν, εθαύμασα θαυμασμόν μέγαν.
\par 7 Και μοι είπεν ο άγγελος. Διά τι εθαύμασας; εγώ θέλω σοι ειπεί το μυστήριον της γυναικός και του θηρίου του βαστάζοντος αυτήν, το οποίον έχει τας επτά κεφαλάς και τα δέκα κέρατα.
\par 8 Το θηρίον, το οποίον είδες, ήτο και δεν είναι, και μέλλει να αναβή εκ της αβύσσου και να υπάγη εις απώλειαν· και θέλουσι θαυμάσει οι κατοικούντες επί της γης, των οποίων τα ονόματα δεν είναι γεγραμμένα εν τω βιβλίω της ζωής από καταβολής κόσμου, βλέποντες το θηρίον, το οποίον ήτο και δεν είναι, αν και ήναι.
\par 9 Εδώ είναι ο νούς ο έχων σοφίαν. Αι επτά κεφαλαί είναι επτά όρη, όπου η γυνή κάθηται επ' αυτών·
\par 10 και είναι επτά βασιλείς· οι πέντε έπεσαν, και ο εις είναι, ο άλλος δεν ήλθεν έτι, και όταν έλθη, ολίγον πρέπει να μείνη.
\par 11 Και το θηρίον, το οποίον ήτο και δεν είναι, είναι και αυτός ο όγδοος, και είναι εκ των επτά, και υπάγει εις απώλειαν.
\par 12 Και τα δέκα κέρατα, τα οποία είδες, είναι δέκα βασιλείς, οίτινες βασιλείαν δεν έλαβον έτι, αλλά μίαν ώραν λαμβάνουσιν εξουσίαν ως βασιλείς μετά του θηρίου.
\par 13 Ούτοι έχουσι μίαν γνώμην και θέλουσι παραδώσει εις το θηρίον την δύναμιν και την εξουσίαν εαυτών.
\par 14 Ούτοι θέλουσι πολεμήσει με το Αρνίον, και το Αρνίον θέλει νικήσει αυτούς, διότι είναι Κύριος των κυρίων και Βασιλεύς των βασιλέων, και όσοι είναι μετ' αυτού είναι κλητοί και εκλεκτοί και πιστοί.
\par 15 Και μοι λέγει· Τα ύδατα, τα οποία είδες, όπου η πόρνη κάθηται, είναι λαοί και όχλοι και έθνη και γλώσσαι.
\par 16 Και τα δέκα κέρατα, τα οποία είδες επί το θηρίον, ούτοι θέλουσι μισήσει την πόρνην και θέλουσι κάμει αυτήν ηρημωμένην και γυμνήν, και τας σάρκας αυτής θέλουσι φάγει, και αυτήν θέλουσι κατακαύσει εν πυρί.
\par 17 Διότι ο Θεός έδωκεν εις τας καρδίας αυτών να κάμωσι την γνώμην αυτού, και να γείνωσι της αυτής γνώμης και να δώσωσι την βασιλείαν αυτών εις το θηρίον, εωσού εκτελεσθώσιν οι λόγοι του Θεού.
\par 18 Και η γυνή, την οποίαν είδες, είναι η πόλις η μεγάλη, η έχουσα βασιλείαν επί των βασιλέων.

\chapter{18}

\par 1 Και μετά ταύτα είδον άγγελον καταβαίνοντα εκ του ουρανού, όστις είχεν εξουσίαν μεγάλην, και η γη εφωτίσθη εκ της δόξης αυτού,
\par 2 και έκραξε δυνατά μετά φωνής μεγάλης, λέγων· Έπεσεν, έπεσε Βαβυλών η μεγάλη, και έγεινε κατοικητήριον δαιμόνων και φυλακή παντός πνεύματος ακαθάρτου και φυλακή παντός ορνέου ακαθάρτου και μισητού·
\par 3 διότι εκ του οίνου του θυμού της πορνείας αυτής έπιον πάντα τα έθνη, και οι βασιλείς της γης επόρνευσαν μετ' αυτής και οι έμποροι της γης επλούτησαν εκ της υπερβολής της εντρυφήσεως αυτής.
\par 4 Και ήκουσα άλλην φωνήν εκ του ουρανού, λέγουσαν· Εξέλθετε εξ αυτής ο λαός μου, διά να μη συγκοινωνήσητε εις τας αμαρτίας αυτής, και να μη λάβητε εκ των πληγών αυτής·
\par 5 διότι αι αμαρτίαι αυτής έφθασαν έως του ουρανού, και ενεθυμήθη ο Θεός τα αδικήματα αυτής.
\par 6 Απόδοτε εις αυτήν ως και αυτή απέδωκεν εις εσάς, και διπλασιάσατε εις αυτήν διπλάσια κατά τα έργα αυτής· με το ποτήριον, με το οποίον εκέρασε, διπλάσιον κεράσατε εις αυτήν·
\par 7 όσον εδόξασεν εαυτήν και κατετρύφησε, τόσον βασανισμόν και πένθος δότε εις αυτήν. Διότι λέγει εν τη καρδία αυτής, Κάθημαι βασίλισσα και χήρα δεν είμαι και πένθος δεν θέλω ιδεί,
\par 8 διά τούτο εν μιά ημέρα θέλουσιν ελθεί αι πληγαί αυτής, θάνατος και πένθος και πείνα, και θέλει κατακαυθή εν πυρί· διότι ισχυρός είναι Κύριος ο Θεός ο κρίνων αυτήν.
\par 9 Και θέλουσι κλαύσει αυτήν και πενθήσει δι' αυτήν οι βασιλείς της γης, οι πορνεύσαντες και κατατρυφήσαντες μετ' αυτής, όταν βλέπωσι τον καπνόν της πυρπολήσεως αυτής,
\par 10 από μακρόθεν ιστάμενοι διά τον φόβον του βασανισμού αυτής, λέγοντες· Ουαί, ουαί, η πόλις η μεγάλη, Βαβυλών, η πόλις η ισχυρά, διότι εν μιά ώρα ήλθεν η κρίσις σου.
\par 11 Και οι έμποροι της γης κλαίουσι και πενθούσι δι' αυτήν, διότι ουδείς αγοράζει πλέον τας πραγματείας αυτών,
\par 12 πραγματείας χρυσού και αργύρου και λίθων τιμίων και μαργαριτών και βύσσου και πορφύρας και μετάξης και κοκκίνου και παν ξύλον αρωματικόν και παν σκεύος ελεφάντινον και παν σκεύος εκ ξύλου πολυτίμου και χαλκού και σιδήρου και μαρμάρου,
\par 13 και κινάμωμον και θυμιάματα και μύρον και λίβανον και οίνον και έλαιον και σεμίδαλιν και σίτον και κτήνη και πρόβατα και ίππους και αμάξας και ανδράποδα και ψυχάς ανθρώπων.
\par 14 Και τα οπωρικά της επιθυμίας της ψυχής σου έφυγον από σου, και πάντα τα παχέα και τα λαμπρά έφυγον από σου, και πλέον δεν θέλεις ευρεί αυτά.
\par 15 Οι έμποροι τούτων, οι πλουτήσαντες απ' αυτής, θέλουσι σταθή από μακρόθεν διά τον φόβον του βασανισμού αυτής, κλαίοντες και πενθούντες,
\par 16 και λέγοντες· Ουαί, ουαί, η πόλις η μεγάλη· η ενδεδυμένη βύσσινον και πορφυρούν και κόκκινον και κεχρυσωμένη με χρυσόν και λίθους τιμίους και μαργαρίτας,
\par 17 διότι εν μιά ώρα ηρημώθη ο τοσούτος πλούτος. Και πας πλοίαρχος και παν το πλήθος το επί των πλοίων και ναύται και όσοι εμπορεύονται διά της θαλάσσης, εστάθησαν από μακρόθεν,
\par 18 και έκραζον βλέποντες τον καπνόν της πυρπολήσεως αυτής, λέγοντες· Ποία πόλις εστάθη ομοία με την πόλιν την μεγάλην;
\par 19 Και έβαλον χώμα επί τας κεφαλάς αυτών και έκραζον κλαίοντες και πενθούντες, λέγοντες· Ουαί, ουαί, η πόλις η μεγάλη, εν ή επλούτησαν εκ της αφθονίας αυτής πάντες οι έχοντες πλοία εν τη θαλάσση· διότι εν μιά ώρα ηρημώθη.
\par 20 Ευφραίνου επ' αυτήν, ουρανέ, και οι άγιοι απόστολοι και οι προφήται, διότι έκρινεν ο Θεός την κρίσιν σας εναντίον αυτής.
\par 21 Και εσήκωσεν εις άγγελος ισχυρός λίθον, ως μυλόπετραν μεγάλην, και έρριψεν εις την θάλασσαν, λέγων· Ούτω με ορμήν θέλει ριφθή η Βαβυλών η μεγάλη πόλις, και δεν θέλει ευρεθή πλέον.
\par 22 Και φωνή κιθαρωδών και μουσικών και αυλητών και σαλπιστών δεν θέλει ακουσθή πλέον εν σοι, και πας τεχνίτης πάσης τέχνης δεν θέλει ευρεθή πλέον εν σοι, και φωνή μύλου δεν θέλει ακουσθή πλέον εν σοι,
\par 23 και φως λύχνου δεν θέλει φέγγει πλέον εν σοι, και φωνή νυμφίου και νύμφης δεν θέλει ακουσθή πλέον εν σοί· διότι οι έμποροί σου ήσαν οι μεγιστάνες της γης, διότι με την γοητείαν σου επλανήθησαν πάντα τα έθνη,
\par 24 και εν αυτή ευρέθη αίμα προφητών και αγίων και πάντων των εσφαγμένων επί της γης.

\chapter{19}

\par 1 Και μετά ταύτα ήκουσα ως φωνήν μεγάλην όχλου πολλού εν τω ουρανώ, λέγοντος· Αλληλούϊα· η σωτηρία και η δόξα και η τιμή και η δύναμις ανήκουσιν εις Κύριον τον Θεόν ημών,
\par 2 διότι αληθιναί και δίκαιαι είναι αι κρίσεις αυτού· διότι έκρινε την πόρνην την μεγάλην, ήτις έφθειρε την γην με την πορνείαν αυτής, και εξεδίκησεν εκ της χειρός αυτής το αίμα των δούλων αυτού.
\par 3 Και εκ δευτέρου είπον· Αλληλούϊα· και ο καπνός αυτής αναβαίνει εις τους αιώνας των αιώνων.
\par 4 Και έπεσον οι εικοσιτέσσαρες πρεσβύτεροι και τα τέσσαρα ζώα και προσεκύνησαν τον Θεόν τον καθήμενον επί του θρόνου λέγοντες· Αμήν, αλληλούϊα.
\par 5 Και εξήλθεν εκ του θρόνου φωνή, λέγουσα· Αινείτε τον Θεόν ημών, πάντες οι δούλοι αυτού και οι φοβούμενοι αυτόν και οι μικροί και οι μεγάλοι.
\par 6 Και ήκουσα ως φωνήν όχλου πολλού, και ως φωνήν υδάτων πολλών, και ως φωνήν βροντών ισχυρών, λεγόντων· Αλληλούϊα· διότι εβασίλευσε Κύριος ο Θεός ο παντοκράτωρ.
\par 7 Ας χαίρωμεν και ας αγαλλιώμεθα και ας δώσωμεν την δόξαν εις αυτόν, διότι ήλθεν ο γάμος του Αρνίου, και η γυνή αυτού ητοίμασεν εαυτήν.
\par 8 Και εδόθη εις αυτήν να ενδυθή βύσσινον καθαρόν και λαμπρόν· διότι το βύσσινον είναι τα δικαιώματα των αγίων.
\par 9 Και λέγει προς εμέ· Γράψον, Μακάριοι οι κεκλημένοι εις το δείπνον του γάμου του Αρνίου. Και λέγει προς εμέ· Ούτοι είναι οι αληθινοί λόγοι του Θεού.
\par 10 Και έπεσον έμπροσθεν των ποδών αυτού, διά να προσκυνήσω αυτόν. Και λέγει μοι· Πρόσεχε μη κάμης τούτο· εγώ είμαι σύνδουλός σου και των αδελφών σου, οίτινες έχουσι την μαρτυρίαν του Ιησού· τον Θεόν προσκύνησον· διότι η μαρτυρία του Ιησού είναι το πνεύμα της προφητείας.
\par 11 Και είδον τον ουρανόν ανεωγμένον, και ιδού ίππος λευκός, και ο καθήμενος επ' αυτόν εκαλείτο Πιστός και Αληθινός, και κρίνει και πολεμεί εν δικαιοσύνη.
\par 12 Οι δε οφθαλμοί αυτού ήσαν ως φλόξ πυρός, και επί της κεφαλής αυτού διαδήματα πολλά, και είχεν όνομα γεγραμμένον, το οποίον ουδείς γνωρίζει ειμή αυτός,
\par 13 και ήτο ενδεδυμένος ιμάτιον βεβαμμένον με αίμα, και καλείται το όνομα αυτού· ο Λόγος του Θεού.
\par 14 Και τα στρατεύματα τα εν τω ουρανώ ηκολούθουν αυτόν εφ' ίππων λευκών, ενδεδυμένοι βύσσινον λευκόν και καθαρόν.
\par 15 Και εκ του στόματος αυτού εξέρχεται ρομφαία κοπτερά, διά να κτυπά με αυτήν τα έθνη· και αυτός θέλει ποιμάνει αυτούς εν ράβδω σιδηρά· και αυτός πατεί τον ληνόν του οίνου του θυμού και της οργής του Θεού του παντοκράτορος·
\par 16 και επί το ιμάτιον και επί τον μηρόν αυτού έχει γεγραμμένον το όνομα, Βασιλεύς βασιλέων και Κύριος κυρίων.
\par 17 Και είδον ένα άγγελον ιστάμενον εν τω ηλίω, και έκραξε μετά φωνής μεγάλης, λέγων προς πάντα τα όρνεα τα πετώμενα εις το μεσουράνημα· Έλθετε και συνάγεσθε εις το δείπνον του μεγάλου Θεού,
\par 18 διά να φάγητε σάρκας βασιλέων και σάρκας χιλιάρχων και σάρκας ισχυρών και σάρκας ίππων και των καθημένων επ' αυτών και σάρκας πάντων ελευθέρων και δούλων και μικρών και μεγάλων.
\par 19 Και είδον το θηρίον και τους βασιλείς της γης και τα στρατεύματα αυτών συνηγμένα, διά να κάμωσι πόλεμον με τον καθήμενον επί του ίππου και με το στράτευμα αυτού.
\par 20 Και επιάσθη το θηρίον και μετά τούτου ο ψευδοπροφήτης, όστις έκαμε τα σημεία ενώπιον αυτού, με τα οποία επλάνησε τους λαβόντας το χάραγμα του θηρίου και τους προσκυνούντας την εικόνα αυτού· ζώντες ερρίφθησαν οι δύο εις την λίμνην του πυρός, την καιομένην με το θείον.
\par 21 Και οι λοιποί εφονεύθησαν με την ρομφαίαν του καθημένου επί του ίππου, την εξερχομένην εκ του στόματος αυτού· και πάντα τα όρνεα εχορτάσθησαν εκ των σαρκών αυτών.

\chapter{20}

\par 1 Και είδον άγγελον καταβαίνοντα εκ του ουρανού, όστις είχε το κλειδίον της αβύσσου και άλυσιν μεγάλην εν τη χειρί αυτού.
\par 2 Και επίασε τον δράκοντα, τον όφιν τον αρχαίον, όστις είναι Διάβολος και Σατανάς, και έδεσεν αυτόν χίλια έτη,
\par 3 και έρριψεν αυτόν εις την άβυσσον και έκλεισεν αυτόν και εσφράγισεν επάνω αυτού, διά να μη πλανήση τα έθνη πλέον, εωσού πληρωθώσι τα χίλια έτη· και μετά ταύτα πρέπει να λυθή ολίγον καιρόν.
\par 4 Και είδον θρόνους, και εκάθησαν επ' αυτών, και κρίσις εδόθη εις αυτούς και είδον τας ψυχάς των πεπελεκισμένων διά την μαρτυρίαν του Ιησού και διά τον λόγον του Θεού, και οίτινες δεν προσεκύνησαν το θηρίον ούτε την εικόνα αυτού, και δεν έλαβον το χάραγμα επί το μέτωπον αυτών και επί την χείρα αυτών· και έζησαν και εβασίλευσαν μετά του Χριστού τα χίλια έτη.
\par 5 Οι δε λοιποί των νεκρών δεν ανέζησαν, εωσού πληρωθώσι τα χίλια έτη. Αύτη είναι η ανάστασις η πρώτη.
\par 6 Μακάριος και άγιος, όστις έχει μέρος εις την πρώτην ανάστασιν· επί τούτων ο θάνατος ο δεύτερος δεν έχει εξουσίαν, αλλά θέλουσιν είσθαι ιερείς του Θεού και του Χριστού και θέλουσι βασιλεύσει μετ' αυτού χίλια έτη.
\par 7 Και όταν πληρωθώσι τα χίλια έτη, θέλει λυθή ο Σατανάς εκ της φυλακής αυτού,
\par 8 και θέλει εξέλθει, διά να πλανήση τα έθνη τα εις τας τέσσαρας γωνίας της γης, τον Γωγ και τον Μαγώγ, διά να συνάξη αυτούς εις πόλεμον, των οποίων ο αριθμός είναι ως η άμμος της θαλάσσης.
\par 9 Και ανέβησαν επί το πλάτος της γης και περιεκύκλωσαν το στρατόπεδον των αγίων και την πόλιν την ηγαπημένην· και κατέβη πυρ από του Θεού εκ του ουρανού και κατέφαγεν αυτούς·
\par 10 και ο διάβολος ο πλανών αυτούς ερρίφθη εις την λίμνην του πυρός και του θείου, όπου είναι το θηρίον και ο ψευδοπροφήτης, και θέλουσι βασανίζεσθαι ημέραν και νύκτα εις τους αιώνας των αιώνων.
\par 11 Και είδον θρόνον λευκόν μέγαν και τον καθήμενον επ' αυτού, από προσώπου του οποίου έφυγεν η γη και ο ουρανός, και δεν ευρέθη τόπος δι' αυτά.
\par 12 Και είδον τους νεκρούς, μικρούς και μεγάλους, ισταμένους ενώπιον του Θεού, και τα βιβλία ηνοίχθησαν· και βιβλίον άλλο ηνοίχθη, το οποίον είναι της ζωής· και εκρίθησαν οι νεκροί εκ των γεγραμμένων εν τοις βιβλίοις κατά τα έργα αυτών.
\par 13 Και έδωκεν η θάλασσα τους εν αυτή νεκρούς, και ο θάνατος και ο άδης έδωκαν τους εν αυτοίς νεκρούς, και εκρίθησαν έκαστος κατά τα έργα αυτών.
\par 14 Και ο θάνατος και ο άδης ερρίφθησαν εις την λίμνην του πυρός· ούτος είναι ο δεύτερος θάνατος.
\par 15 Και όστις δεν ευρέθη γεγραμμένος εν τω βιβλίω της ζωής, ερρίφθη εις την λίμνην του πυρός.

\chapter{21}

\par 1 Και είδον ουρανόν νέον και γην νέαν· διότι ο πρώτος ουρανός και η πρώτη γη παρήλθε, και η θάλασσα δεν υπάρχει πλέον.
\par 2 Και εγώ ο Ιωάννης είδον την πόλιν την αγίαν, την νέαν Ιερουσαλήμ καταβαίνουσαν από του Θεού εκ του ουρανού, ητοιμασμένην ως νύμφην κεκοσμημένην διά τον άνδρα αυτής.
\par 3 Και ήκουσα φωνήν μεγάλην εκ του ουρανού, λέγουσαν· Ιδού, η σκηνή του Θεού μετά των ανθρώπων, και θέλει σκηνώσει μετ' αυτών, και αυτοί θέλουσιν είσθαι λαοί αυτού, και αυτός ο Θεός θέλει είσθαι μετ' αυτών Θεός αυτών·
\par 4 και θέλει εξαλείψει ο Θεός παν δάκρυον από των οφθαλμών αυτών, και ο θάνατος δεν θέλει υπάρχει πλέον, ούτε πένθος ούτε κραυγή ούτε πόνος δεν θέλουσιν υπάρχει πλέον· διότι τα πρώτα παρήλθον.
\par 5 Και είπεν ο καθήμενος επί του θρόνου· Ιδού, κάμνω νέα τα πάντα. Και λέγει προς εμέ· Γράψον, διότι ούτοι οι λόγοι είναι αληθινοί και πιστοί.
\par 6 Και είπε προς εμέ· Ετελέσθη. Εγώ είμαι το Α και το Ω, η αρχή και το τέλος. Εγώ θέλω δώσει εις τον διψώντα εκ της πηγής του ύδατος της ζωής δωρεάν.
\par 7 Ο νικών θέλει κληρονομήσει τα πάντα, και θέλω είσθαι εις αυτόν Θεός και αυτός θέλει είσθαι εις εμέ υιός.
\par 8 Οι δε δειλοί και άπιστοι και βδελυκτοί και φονείς και πόρνοι και μάγοι και ειδωλολάτραι και πάντες οι ψεύσται θέλουσιν έχει την μερίδα αυτών εν τη λίμνη τη καιομένη με πυρ και θείον· ούτος είναι ο δεύτερος θάνατος.
\par 9 Και ήλθε προς εμέ εις των επτά αγγέλων των εχόντων τας επτά φιάλας τας πλήρεις από των επτά εσχάτων πληγών, και ελάλησε μετ' εμού, λέγων· Ελθέ, θέλω σοι δείξει την νύμφην, του Αρνίου την γυναίκα.
\par 10 Και με έφερεν εν πνεύματι επί όρος μέγα και υψηλόν, και μοι έδειξε την πόλιν την μεγάλην, την αγίαν Ιερουσαλήμ, καταβαίνουσαν εκ του ουρανού από του Θεού,
\par 11 έχουσαν την δόξαν του Θεού· και η λαμπρότης αυτής ήτο ομοία με λίθον πολύτιμον, ως λίθον ίασπιν κρυσταλλίζοντα·
\par 12 και είχε τείχος μέγα και υψηλόν, είχε και δώδεκα πυλώνας, και εις τους πυλώνας δώδεκα αγγέλους, και ονόματα επιγεγραμμένα, τα οποία είναι των δώδεκα φυλών των υιών Ισραήλ.
\par 13 Προς ανατολάς πυλώνες τρεις, προς βορράν πυλώνες τρεις, προς νότον πυλώνες τρεις, προς δυσμάς πυλώνες τρεις.
\par 14 Και το τείχος της πόλεως είχε θεμέλια δώδεκα, και εν αυτοίς τα ονόματα των δώδεκα αποστόλων του Αρνίου.
\par 15 Και ο λαλών μετ' εμού είχε κάλαμον χρυσούν, διά να μετρήση την πόλιν και τους πυλώνας αυτής και το τείχος αυτής.
\par 16 Και πόλις κείται τετράγωνος, και το μήκος αυτής είναι τοσούτον όσον και το πλάτος. Και εμέτρησε την πόλιν με τον κάλαμον έως δώδεκα χιλιάδας σταδίων· το μήκος και το πλάτος και το ύψος αυτής είναι ίσα.
\par 17 Και εμέτρησε το τείχος αυτής, εκατόν τεσσαράκοντα τεσσάρων πηχών, κατά το μέτρον του ανθρώπου, ήγουν του αγγέλου.
\par 18 Και η οικοδόμησις του τείχους αυτής ήτο ίασπις, και η πόλις χρυσίον καθαρόν, ομοία με ύαλον καθαρόν.
\par 19 Και τα θεμέλια του τείχους της πόλεως ήσαν κεκοσμημένα με πάντα λίθον πολύτιμον· το πρώτον θεμέλιον ίασπις, το δεύτερον σάπφειρος, το τρίτον χαλκηδών, το τέταρτον σμάραγδος,
\par 20 το πέμπτον σαρδόνυξ, το έκτον σάρδιος, το έβδομον χρυσόλιθος, το όγδοον βήρυλλος, το έννατον τοπάζιον, το δέκατον χρυσόπρασος, το ενδέκατον υάκινθος, το δωδέκατον αμέθυστος.
\par 21 Και οι δώδεκα πυλώνες ήσαν δώδεκα μαργαρίται· έκαστος των πυλώνων ήτο εξ ενός μαργαρίτου και η πλατεία της πόλεως χρυσίον καθαρόν ως ύαλος διαφανής.
\par 22 Και ναόν δεν είδον εν αυτή· διότι ναός αυτής είναι ο Κύριος ο Θεός ο παντοκράτωρ και το Αρνίον.
\par 23 Και η πόλις δεν έχει χρείαν του ηλίου ουδέ της σελήνης, διά να φέγγωσιν εν αυτή· διότι η δόξα του Θεού εφώτισεν αυτήν, και ο λύχνος αυτής είναι το Αρνίον.
\par 24 Και τα έθνη των σωζομένων θέλουσι περιπατεί εν τω φωτί αυτής· και οι βασιλείς της γης φέρουσι την δόξαν και την τιμήν αυτών εις αυτήν.
\par 25 Και οι πυλώνες αυτής δεν θέλουσι κλεισθή την ημέραν· διότι νυξ δεν θέλει είσθαι εκεί.
\par 26 Και θέλουσι φέρει την δόξαν και την τιμήν των εθνών εις αυτήν.
\par 27 Και δεν θέλει εισέλθει εις αυτήν ουδέν το οποίον μιαίνει και προξενεί βδέλυγμα και ψεύδος, αλλά μόνον οι γεγραμμένοι εν τω βιβλίω της ζωής του Αρνίου.

\chapter{22}

\par 1 Και μοι έδειξε καθαρόν ποταμόν ύδατος της ζωής λαμπρόν ως κρύσταλλον, εξερχόμενον εκ του θρόνου του Θεού και του Αρνίου.
\par 2 Εν τω μέσω της πλατείας αυτής και του ποταμού εντεύθεν και εντεύθεν ήτο το δένδρον της ζωής, φέρον καρπούς δώδεκα, καθ' έκαστον μήνα κάμνον τον καρπόν αυτού, και τα φύλλα του δένδρου είναι εις θεραπείαν των εθνών.
\par 3 Και ουδέν ανάθεμα θέλει είσθαι πλέον· και ο θρόνος του Θεού και του Αρνίου θέλει είσθαι εν αυτή, και οι δούλοι αυτού θέλουσι λατρεύσει αυτόν
\par 4 και θέλουσιν ιδεί το πρόσωπον αυτού, και το όνομα αυτού θέλει είσθαι επί των μετώπων αυτών.
\par 5 Και νυξ δεν θέλει είσθαι εκεί, και δεν έχουσι χρείαν λύχνου και φωτός ηλίου, διότι Κύριος ο Θεός φωτίζει αυτούς, και θέλουσι βασιλεύσει εις τους αιώνας των αιώνων.
\par 6 Και είπε προς εμέ· Ούτοι οι λόγοι είναι πιστοί και αληθινοί· και Κύριος ο Θεός των αγίων προφητών απέστειλε τον άγγελον αυτού, διά να δείξη εις τους δούλους αυτού τα όσα πρέπει να γείνωσι ταχέως.
\par 7 Ιδού, έρχομαι ταχέως. Μακάριος όστις φυλάττει τους λόγους της προφητείας του βιβλίου τούτου.
\par 8 Και εγώ ο Ιωάννης είμαι ο ιδών ταύτα και ακούσας. Και ότε ήκουσα και είδον, έπεσα να προσκυνήσω έμπροσθεν των ποδών του αγγέλου του δεικνύοντος εις εμέ ταύτα.
\par 9 Και λέγει προς εμέ· Πρόσεχε μη κάμης τούτο· διότι εγώ είμαι σύνδουλός σου και των αδελφών σου των προφητών και των φυλαττόντων τους λόγους του βιβλίου τούτου· τον Θεόν προσκύνησον.
\par 10 Και λέγει προς εμέ· Μη σφραγίσης τους λόγους της προφητείας του βιβλίου τούτου· διότι ο καιρός είναι εγγύς.
\par 11 Όστις αδικεί ας αδικήση έτι, και όστις είναι μεμολυσμένος ας μολυνθή έτι, και ο δίκαιος ας γείνη έτι δίκαιος, και ο άγιος ας γείνη έτι άγιος.
\par 12 Και ιδού, έρχομαι ταχέως, και ο μισθός μου είναι μετ' εμού, διά να αποδώσω εις έκαστον ως θέλει είσθαι το έργον αυτού.
\par 13 Εγώ είμαι το Α και το Ω, αρχή και τέλος, ο πρώτος και ο έσχατος.
\par 14 Μακάριοι οι πράττοντες τας εντολάς αυτού, διά να έχωσιν εξουσίαν επί το δένδρον της ζωής και να εισέλθωσι διά των πυλώνων εις την πόλιν.
\par 15 Έξω δε είναι οι κύνες και οι μάγοι και οι πόρνοι και οι φονείς και οι ειδωλολάτραι και πας ο αγαπών και πράττων το ψεύδος.
\par 16 Εγώ ο Ιησούς έπεμψα τον άγγελόν μου να μαρτυρήση εις εσάς ταύτα εις τας εκκλησίας. Εγώ είμαι η ρίζα και το γένος του Δαβίδ, ο αστήρ ο λαμπρός και ορθρινός.
\par 17 Και το Πνεύμα και η νύμφη λέγουσιν· Ελθέ. Και όστις ακούει, ας είπη. Ελθέ. Και όστις διψά, ας έλθη, και όστις θέλει, ας λαμβάνη δωρεάν το ύδωρ της ζωής.
\par 18 Διότι μαρτύρομαι εις πάντα ακούοντα τους λόγους της προφητείας του βιβλίου τούτου· Εάν τις επιθέση εις ταύτα, ο Θεός θέλει επιθέσει εις αυτόν τας πληγάς τας γεγραμμένας εν τω βιβλίω τούτω·
\par 19 και εάν τις αφαιρέση από των λόγων του βιβλίου της προφητείας ταύτης, ο Θεός θέλει αφαιρέσει το μέρος αυτού από του βιβλίου της ζωής και από της πόλεως της αγίας και των γεγραμμένων εν τω βιβλίω τούτω.
\par 20 Λέγει ο μαρτυρών ταύτα· Ναι, έρχομαι ταχέως. Αμήν, ναι, έρχου, Κύριε Ιησού.
\par 21 Η χάρις του Κυρίου ημών Ιησού Χριστού είη μετά πάντων υμών· αμήν.


\end{document}