\begin{document}

\title{Letter of Aristeas}

\chapter{1}

\par \textit{At the time of the Jewish Captivity in Egypt, Ptolemy Philadelphus reveals himself as the first great bibliophile. He desires to have all the books in the world in his library; in order to get the Laws of Moses he offers to trade 100,000 captives for that work exclaiming, “It is a small boon indeed!”}

\par 1 SINCE I have collected material for a memorable history of my visit to Eleazar the High Priest of the Jews, and because you, Philocrates, as you lose no opportunity of reminding me, have set great store upon receiving an account of the motives and object of my mission, I have attempted to draw up clear exposition of the matter for you, for I perceive that you possess a natural love of learning, a quality which is the highest possession of man—to be constantly attempting ‘to add to his stock of knowledge and acquirements’ whether through the study of history or by actually participating in the events themselves.

\par 2 It is by this means, by taking up into itself the noblest elements, that the soul is established in purity, and having fixed its aim on piety, the noblest goal of all, it uses this as its infallible guide and so acquires a definite purpose.

\par 3 It was my devotion to the pursuit of religious knowledge that led me to undertake the embassy to the man I have mentioned, who was held in the highest esteem by his own citizens and by others, both for his virtue and his majesty, and who had in his possession documents of the highest value to the Jews in his own country and in foreign lands for the interpretation of the divine law, for their laws are written on leather parchments in Jewish characters.

\par 4 This embassy then I undertook with enthusiasm, having first of all found an opportunity of pleading with the king on behalf of the Jewish captives who had been transported from Judea to Egypt by the king's father, when he first obtained possession of this city and conquered the land of Egypt.

\par 5 It is worth while that I should tell you this story, too, since I am convinced that you, with your disposition towards holiness and your sympathy with men who are living in accordance with the holy law, will all the more readily listen to the account which I purpose to set forth, since you yourself have lately come to us from the island and are anxious to hear everything that tends to build up the soul.

\par 6 On a former occasion too, I sent you a record of the facts which I thought worth relating about the Jewish race,—the record which I had obtained from the most learned high priests of the most learned land of Egypt.

\par 7 As you are so eager to acquire the knowledge of those things which can benefit the mind, I' feel it incumbent upon me to impart to you all the information in my power.

\par 8 I should feel the same duty towards all who possessed the same disposition but I feel it especially towards you since you have aspirations which are so noble, and since you are not only my brother in character, no less than in blood, but are one with me as well in the pursuit of goodness.

\par 9 For neither the pleasure derived from gold nor any other of the possessions which are prized by shallow minds confers the same benefit as the pursuit of culture and the study which we expend in securing it.

\par 10 But that I may not weary you by a too lengthy introduction, I will proceed at once to the substance of my narrative.

\par 11 Demetrius of Phalerum, the president of the king's library, received vast sum of money, for the purpose of collecting together, as far as he possibly could, all the books in the world.

\par 12 By means of purchase and transcription, he carried out, to the best of his ability, the purpose of the king.

\par 13 On one occasion when I was present he was asked, How many thousand books are there in the library? and he replied, ‘More than two hundred thousand, O king, and I shall make endeavour in the immediate future to gather together the remainder also, so that the total of five hundred thousand may be reached. I am told that the laws of the Jews are worth transcribing and deserve a place in your library!’

\par 14 ‘What is to prevent you from doing this?’ replied the king. ‘Everything that is necessary has been placed at your disposal!’

\par 15 ‘They need to be translated,’ answered Demetrius ‘for in the country of the Jews they use a peculiar alphabet (just as the Egyptians, too, have a special form of letters) and speak a peculiar dialect.’

\par 16 ‘They are supposed to use the Syriac tongue, but this is not the case; their language is quite different.’

\par 17 And the king when he understood all the facts of the case ordered a letter to be written to the Jewish High Priest that his purpose (which has already been described) might be accomplished.

\par 18 Thinking that the time had come to press the demand, which I had often laid before Sosibius of Tarentum and Andreas, the chief of the bodyguard, for the emancipation of the Jews who had been transported from Judea by the king's father—for when by a combination of good fortune and courage he had brought his attack on the whole district of Coele-Syria and Phoenicia to a successful issue, in the process of terrorising the country into subjection, he transported some of his foes and others he reduced to captivity.

\par 19 The number of those whom he transported from the country of the Jews to Egypt amounted to no less than a hundred thousand.

\par 20 Of these he armed thirty thousand picked men and settled them in garrisons in the country districts.

\par 21 (And even before this time large numbers of Jews had come into Egypt with the Persian, and in an earlier period still others had been sent to Egypt to help Psammetichus in his campaign against the king of the Ethiopians. But these were nothing like so numerous as the captives whom Ptolemy the son of Lagus transported.)

\par 22 As I have already said Ptolemy picked out the best of these, the men who were in the prime of life and distinguished for their courage, and armed them, but the great mass of the others, those who were too old or too young for this purpose, and the women too, he reduced to slavery, not that he wished to do this of his own free will, but he was compelled by his soldiers who claimed them as a reward for the services which they had rendered in war.

\par 23 Having, as has already been stated, obtained an opportunity for securing their emancipation, I addressed the king with the following arguments. ‘Let us not be so unreasonable as to allow our deeds to give the lie to our words.’

\par 24 ‘Since the law which we wish not only to transcribe but also to translate belongs to the whole Jewish race, what justification shall we be able to find for our embassy while such vast numbers of them remain in a state of slavery in your kingdom?’

\par 25 ‘In the perfection and wealth of your clemency release those who are held in such miserable bondage, since as I have been at pains to discover, the God who gave them their law is the God who maintains your kingdom.’

\par 26 ‘They worship the same God—the Lord and Creator of the Universe, as all other men, as we ourselves, O king, though we call him by different names, such as Zeus 1 or Dis.’

\par 27 ‘This name was very appropriately bestowed upon him by our first ancestors, in order to signify that He, through whom all things are endowed with life and come into being, is necessarily the Rider and Lord of the Universe.’

\par 28 ‘Set all mankind an example of magnanimity by releasing those who are held in bondage.’

\par 29 After a brief interval, while I was offering up an earnest prayer to God that He would so dispose the mind of the king that all the captives might be set at liberty—(for the human race, being the creation of God, is swayed and influenced by Him.

\par 30 Therefore with many divers prayers I called upon Him who ruleth the heart that the king might be constrained to grant my request.

\par 31 For I had great hopes with regard to the salvation of the men since I was assured that God would grant a fulfilment of my prayer.

\par 32 For when men from pure motives plan some action in the interest of righteousness and the performance of noble deeds, Almighty God brings their efforts and purposes to a successful issue)—the king raised his head and looking up at me with a cheerful countenance asked, ‘How many thousands do you think they will number?’

\par 33 Andreas, who was standing near, replied, ‘A little more than a hundred thousand.’

\par 34 ‘It is a small boon indeed,’ said the king, ‘that Aristeas asks of us!’

\par 35 Then Sosibius and some others who were present said ‘Yes, but it will be a fit tribute to your magnanimity for you to offer the enfranchisement of these men as an act of devotion to the supreme God.’

\par 36 ‘You have been greatly honoured by Almighty God and exalted above all your forefathers in glory and it is only fitting that you should render to Him the greatest thank-offering in your power.’

\par 37 Extremely pleased with these arguments he gave orders that an addition should be made to the wages of the soldiers by the amount of the redemption money, that twenty drachmae should be paid to the owners for every slave, that a public order should be issued and that registers of the captives should be attached to it.

\par 38 He showed the greatest enthusiasm in the business, for it was God who had brought our purpose to fulfilment in its entirety and constrained him to redeem not only those who had come into Egypt with the army of his father but any who had come before that time or had been subsequently brought into the kingdom.

\par 39 It was pointed out to him that the ransom money would exceed four hundred talents.

\par 40 I think it will be useful to insert a copy of the decree, for in this way the magnanimity of the king, who was empowered by God to save such vast multitudes, will be made clearer and more manifest.

\par 41 The decree of the king ran as follows: ‘All who served in the army of our father in the campaign against Syria and Phoenicia and in the attack upon the country of the Jews and became possessed of Jewish captives and brought them back to the city of Alexandria and the land of Egypt or sold them to others—and in the same way any captives who were in our land before that time or were brought hither afterwards—all who possess such captives are required to set them at liberty at once, receiving twenty drachmae per head as ransom money.’

\par 42 ‘The soldiers will receive this money as a gift added to their wages, the others from the king's treasury.’

\par 43 ‘We think that it was against our father's will and against all propriety that they should have been made captives and that the devastation of their land and the transportation of the Jews to Egypt was an act of military wantonness.’

\par 44 ‘The spoil which fell to the soldiers on the field of battle was all the booty which they should have claimed.’

\par 45 ‘To reduce the people to slavery in addition was an act of absolute injustice.’

\par 46 ‘Wherefore, since it is acknowledged that we are accustomed to render justice to all men and especially to those who are unfairly in a condition of servitude, and since we strive to deal fairly with all men according to the demands of justice and piety, we have decreed, in reference to the persons of the Jews who are in any condition of bondage in any part of our dominion, that those who possess them shall receive the stipulated sum of money and set them at liberty and that no man shall show any tardiness in discharging his obligations.’

\par 47 ‘Within three days after the publication of this decree, they must make lists of slaves for the officers appointed to carry out our will, and immediately produce the persons of the captives.’

\par 48 ‘For we consider that it will be advantageous to us and to our affairs that the matter should be brought to a conclusion.’

\par 49 ‘Any one who likes may give information about any who disobey the decree, on condition that if the man is proved guilty he will become his slave; his property, however, will be handed over to the royal treasury.’

\par 50 When the decree was brought to be read over to the king for his approval, it contained all the other provisions except the phrase ‘any captives who were in the land before that time or were brought hither afterwards,’ and in his magnanimity and the largeness of his heart the king inserted this clause and gave orders that the grant of money required for the redemption should be deposited in full with the paymasters of the forces and the royal bankers, and so the matter was decided and the decree ratified within seven days.

\par 51 The grant for the redemption amounted to more than six hundred and sixty talents; for many infants at the breast were emancipated together with their mothers.

\par 52 When the question was raised whether the sum of twenty talents was to be paid for these, the king ordered that it should be done, and thus he carried out his decision in the most comprehensive way.

\par \textit{Footnotes}
\par \textit{143:1 A important comparison of God and Zeus.}

\chapter{2}

\par \textit{Showing how the most careful records were kept of affairs of state. Government Red Tape. A committee of six is appointed to go to the High Priest in Jerusalem and arrange for the exchange. Aristeas is put in charge of the delegation.}

\par 1 WHEN this had been done, he ordered Demetrius to draw up a memorial with regard to the transcription of the Jewish books.

\par 2 For all affairs of state used to be carried out by means of decrees and with the most pains-taking accuracy by these Egyptian kings, and nothing was done in a slipshod or haphazard fashion.

\par 3 And so I have inserted copies of the memorial and the letters, the number of the presents sent and the nature of each, since every one of them excelled in magnificence and technical skill.

\par 4 The following is a copy of the memorial. The Memory of Demetrius to the great king. ‘Since you have given me instructions O king, that the books which are needed to complete your library should be collected together, and that those which are defective should be repaired, I have devoted myself with the utmost care to the fulfilment of your wishes, and I now have the following proposal to lay before you.’

\par 5 ‘The books of the law of the Jews (with some few others) are absent from the library.’

\par 6 ‘They are written in the Hebrew characters and language and have been carelessly interpreted, and do not represent the original text as I am informed by those who know; for they have never had a king's care to protect them.’

\par 7 ‘It is necessary that these should be made accurate for your library since the law which they contain, inasmuch as it is of divine origin, is full of wisdom and free from all blemish.’

\par 8 ‘For this reason literary men and poets and the mass of historical writers have held aloof from referring to these books and the men who have lived and are living in accordance with them, because their conception of life is so sacred and religious, as Hecataeus of Abdera says.’

\par 9 ‘If it please you, O king, a letter shall be written to the High Priest in Jerusalem, asking him to send six elders out of every tribe—men who have lived the noblest life and are most skilled in their law—that we may find out the points in which the majority of them are in agreement, and so having obtained an accurate translation may place it in a conspicuous place in a manner worthy of the work itself and your purpose.’

\par 10 ‘May continual prosperity be yours!’

\par 11 When this memorial had been presented, the king ordered a letter to be written to Eleazar on the matter, giving also an account of the emancipation of the Jewish captives.

\par 12 And he gave fifty talents weight of gold and seventy talents of silver and a large quantity of precious stones to make bowls and vials and a table and libation cups.

\par 13 He also gave orders to those who had the custody of his coffers to allow the artificers to make a selection of any materials they might require for the purpose, and that a hundred talents in money should be sent to provide sacrifices for the temple and for other needs.

\par 14 I shall give you a full account of the workmanship after I have set before you copies of the letters. The letter of the king ran as follows:

\par 15 ‘King Ptolemy sends greeting and salutation to the High Priest Eleazar.’

\par 16 ‘Since there are many Jews settled in our realm who were carried off from Jerusalem by the Persians at the time of their power and many more who came with my father into Egypt as captives—large numbers of these he placed in the army and paid them higher wages than usual, and when he had proved the loyalty of their leaders he built fortresses and placed them in their charge that the native Egyptians might be intimidated by them.’

\par 17 ‘And I, when I ascended the throne, adopted a kindly attitude towards all my subjects, and more particularly to those who were citizens of yours—I have set at liberty more than a hundred thousand captives, paying their owners the appropriate market price for them, and if ever evil has been done to your people through the passions of the mob, I have made them reparation.’

\par 18 ‘The motive which prompted my action has been the desire to act piously and render unto the supreme God a thank-offering for maintaining my kingdom in peace and great glory in all the world.’

\par 19 ‘Moreover those of your people who were in the prime of life I have drafted into my army, and those who were fit to be attached to my person and worthy of the confidence of the court, I have established in official positions.’

\par 20 ‘Now since I am anxious to show my gratitude to these men and to the Jews throughout the world and to the generations yet to come, I have determined that your law shall be translated from the Hebrew tongue which is in use amongst you into the Greek language, that these books may be added to the other royal books in my library.’

\par 21 ‘It will be a kindness on your part and a reward for my zeal if you will select six elders from each of your tribes, men of noble life and skilled in your law and able to interpret it, that in questions of dispute we may be able to discover the verdict in which the majority agree, for the investigation is of the highest possible importance.’

\par 22 ‘I hope to win great renown by the accomplishment of this work.’

\par 23 ‘I have sent Andreas, the chief of my bodyguard and Aristeas—men whom I hold in high esteem—to lay the matter before you and present you with a hundred talents of silver, the first-fruits of my offering for the temple and the sacrifices and other religious rites.’

\par 24 ‘If you will write to me concerning your wishes in these matters, you will confer a great favour upon me and afford me a new pledge of friendship, for all your wishes shall be carried out as speedily as possible. Farewell!’

\par 25 To this letter Eleazar replied appropriately as follows: ‘Eleazar the High Priest sends greetings to King Ptolemy his true friend.’

\par 26 ‘My highest wishes are for your welfare and the welfare of Queen Arsinoe, your sister, and your children.’

\par 27 ‘I also am well. I have received your letter and am greatly rejoiced by your purpose and your noble counsel.’

\par 28 ‘I summoned together the whole people and read it to them that they might know of your devotion to our God.’

\par 29 ‘I showed them too the cups which you sent, twenty of gold and thirty of silver, the five bowls and the table of dedication, and the hundred talents of silver for the offering of the sacrifices and providing the things of which the temple stands in need.’

\par 30 ‘These gifts were brought to me by Andreas, one of your most honoured servants, and by Aristeas, both good men and true, distinguished by their learning, and worthy in every way to be the representatives of your high principles and righteous purposes.’

\par 31 ‘These men imparted to me your message and received from me an answer in agreement with your letter. I will consent to everything which is advantageous to you even though your request is very unusual.’

\par 32 ‘For you have bestowed upon our citizens great and never to be forgotten benefits in many ways.’

\par 33 ‘Immediately therefore I offered sacrifices on behalf of you, your sister, your children, and your friends, and all the people prayed that your plans might prosper continually, and that Almighty God might preserve your kingdom in peace with honour, and that the translation of the holy law might prove advantageous to you and be carried out successfully.’

\par 34 ‘In the presence of all the people I selected six elders from each tribe, good men and true, and I have sent them to you with a copy of our law.’

\par 35 ‘It will be a kindness, O righteous king, if you will give instruction that as soon as the translation of the law is completed, the men shall be restored again to us in safety. Farewell!’

\par 36 The following are the names of the elders: Of the first tribe, Joseph, Ezekiah, Zachariah, John, Ezekiah, Elisha.

\par 37 Of the second tribe, Judas, Simon, Samuel, Adaeus, Mattathias, Eschlemias.

\par 38 Of the third tribe, Nehemia, Joseph, Theodosius, Baseas, Ornias, Dakis.

\par 39 Of the fourth tribe, Jonathan, Abraeus, Elisha, Ananias, Chabrias. . . .

\par 40 Of the fifth tribe, Isaac, Jacob, Jesus, Sabbataeus, Simon, Levi.

\par 41 Of the sixth tribe, Judas, Joseph, Simon, Zacharias, Samuel, Selemas.

\par 42 Of the seventh tribe, Sabbataeus, Zedekiah, Jacob, Isaac, Jesias, Natthaeus.

\par 43 Of the eighth tribe, Theodosius, Jason, Jesus, Theodotus, John, Jonathan.

\par 44 Of the ninth tribe, Theophilus, Abraham, Arsamos, Jason, Endemias, Daniel.

\par 45 Of the tenth tribe, Jeremiah, Eleazar, Zachariah, Baneas, Elisha, Dathaeus.

\par 46 Of the eleventh tribe, Samuel, Joseph, Judas, Jonathes, Chabu, Dositheus.

\par 47 Of the twelfth tribe, Isaelus, John, Theodosius, Arsamos, Abietes, Ezekiel.

\par 48 They were seventy-two in all. Such was the answer which Eleazar and his friends gave to the king's letter.

\chapter{3}

\par \textit{In which is described the most exquisite and beautiful table ever produced. Also other rich gifts, Interesting in the light of recent excavations in Egypt.}

\par 1 I WILL now Proceed to redeem my promise and give a description of the works of art.

\par 2 They were wrought with exceptional skill, for the king spared no expense and personally superintended the workmen individually.

\par 3 They could not therefore scamp any part of the work or finish it off negligently.

\par 4 First of all I will give you a description of the table.

\par 5 The king was anxious that this piece of work should be of exceptionally large dimensions, and he caused enquiries to be made of the Jews in the locality with regard to the size of the table already in the temple at Jerusalem.

\par 6 And when they described the measurements, he proceeded to ask whether he might make a larger structure.

\par 7 And some of the priests and the other Jews replied that there was nothing to prevent him.

\par 8 And he said that he was anxious to make it five times the size, but he hesitated lest it should prove useless for the temple services.

\par 9 He was desirous that his gift should not merely be stationed in the temple, for it would afford him much greater pleasure if the men whose duty it was to offer the fitting sacrifices were able to do so appropriately on the table which he had made.

\par 10 He did not suppose that it was owing to lack of gold that the former table had been made of small size, but there seems to have been, he said, some reason why it was made of—this dimension.

\par 11 For had the order been given, there would have been no lack of means.

\par 12 Wherefore we must not transgress or go beyond the proper measure.

\par 13 At the same time he ordered them to press into service all the manifold forms of art, for he was a man of the most lofty conceptions and nature had endowed him with a keen imagination which enabled him to picture the appearance which would be presented by the finished work.

\par 14 He gave orders too, that where there were no instructions laid down in the Jewish Scriptures, everything should be made as beautiful as possible.—

\par 15 When such instructions were laid down, they were to be carried out to the letter.

\par 16 They made the table two cubits 1 long, one cubit broad, one and a half cubits high fashioning it of pure solid gold.

\par 17 What I am describing was not thin gold laid over another foundation, but the whole structure was of massive gold welded together.

\par 18 And they made a border of a hand's breadth round about it.

\par 19 And there was a wreath of wave-work, engraved in relief in the form of ropes marvellously wrought on its three sides.

\par 20 For it was triangular in shape and the style of the work was exactly the same on each of the sides, so that whichever side they were turned, they presented the same appearance.

\par 21 Of the two sides under the border, the one which sloped down to the table was a very beautiful piece of work, but it was the outer side which attracted the gaze of the spectator.

\par 22 Now the upper edge of the two sides, being elevated, was sharp since, as we have said, the rim was three-sided, from whatever point of view one approached it.

\par 23 And there were layers of precious stones on it in the midst of the embossed cord-work, and they were interwoven with one another by an inimitable artistic device.

\par 24 For the sake of security they were all fixed by golden needles which were inserted in perforations in the stones.

\par 25 At the sides they were clamped together by fastenings to hold them firm.

\par 26 On the part of the border round the table which slanted upwards and met the eyes, there was wrought a pattern of eggs in precious stones, elaborately engraved by a continuous piece of fluted relief-work, closely connected together round the whole table.

\par 27 And under the stones which had been arranged to represent eggs the artists made a crown containing all kinds of fruits, having at its top clusters of grapes and ears of corn, dates also and apples, and pomegranates and the like, conspicuously arranged.

\par 28 These fruits were wrought out of precious stones, of the same colour as the fruits themselves and they fastened them edgeways round all the sides of the table with a band of gold.

\par 29 And after the crown of fruit had been put on, underneath there was inserted another pattern of eggs in precious stones, and other fluting and embossed work, that both sides of the table might be used, according to the wishes of the owners and for this reason the wave-work and the border were extended down to the feet of the table.

\par 30 They made and fastened under the whole width of the table a massive plate four fingers thick, that the feet might be inserted into it, and clamped fast with linch-pins which fitted into sockets under the border, so that which ever side of the table people preferred, might be used.

\par 31 Thus it became manifestly clear that the work was intended to be used either way.

\par 32 On the table itself they engraved a ‘maeander,’ having precious stones standing out in the middle of it, rubies and emeralds and an onyx too and many other kinds of stones which excel in beauty.

\par 33 And next to the ‘maeander’ there was placed a wonderful piece of network, which made the centre of the table appear like a rhomboid in shape, and on it a crystal and amber, as it is called, had been wrought, which produced an incomparable impression on the beholders.

\par 34 They made the feet of the table with heads like lilies, so that they seemed to be like lilies bending down beneath the table, and the parts which were visible represented leaves which stood upright.

\par 35 The basis of the foot on the ground consisted of a ruby and measured a hand's breadth high all round.

\par 36 It had the appearance of a shoe and was eight fingers broad.

\par 37 Upon it the whole expanse of the foot rested.

\par 38 And they made the foot appear like ivy growing out of the stone, interwoven with akanthus and surrounded with a vine which encircled it with clusters of grapes, which were worked in stones, up to the top of the foot.

\par 39 All the four feet were made in the same style, and everything was wrought and fitted so skilfully, and such remarkable skill and knowledge were expended upon making it true to nature, that when the air was stirred by a breath of wind, movement was imparted to the leaves, and everything was fashioned to correspond with the actual reality which it represented.

\par 40 And they made the top of the table in three parts like a triptychon, and they were so fitted and dovetailed together with spigots along the whole breadth of the work, that the meeting of the joints could not be seen or even discovered.

\par 41 The thickness of the table was not less than half a cubit, so that the whole work must have cost many talents.

\par 42 For since the king did not wish to add to its size he expended on the details the same sum of money which would have been required if the table could have been of larger dimensions.

\par 43 And everything was completed in accordance with his plan, in a most wonderful and remarkable way, with inimitable art and incomparable beauty.

\par 44 Of the mixing bowls, two were wrought in gold, and from the base to the middle were engraved with relief work in the pattern of scales, and between the scales Precious stones were inserted with great artistic skill.

\par 45 Then there was a ‘maeander’ a cubit in height, with its surface. wrought out of precious stones of many colours, displaying great artistic effort and beauty.

\par 46 Upon this there was a mosaic, worked in the form of a rhombus, having a net-like appearance and reaching right up to the brim.

\par 47 In the middle, small shields which were made of different precious stones, placed alternately, and varying in kind, not less than four fingers broad, enhanced the beauty of their appearance.

\par 48 On the top of the brim there was an ornament of lilies in bloom, and intertwining clusters of grapes were engraven all round.

\par 49 Such then was the construction of the golden bowls, and they held more than two firkins each.

\par 50 The silver bowls had a smooth surface, and were wonderfully made as if they were intended for looking-glasses, so that everything which was brought near to them was reflected even more clearly than in mirrors.

\par 51 But it is impossible to describe the real impression which these works of art produced upon the mind when they were finished.

\par 52 For, when these vessels had been completed and placed side by side, first a silver bowl and then a golden, then another silver, and then another golden, the appearance they presented is altogether indescribable, and those who came to see them were not able to tear themselves from the brilliant sight and entrancing spectacle.

\par 53 The impressions produced by the spectacle were various in kind.

\par 54 When men looked at the golden vessels, and their minds made a complete survey of each detail of workmanship, their souls were thrilled with wonder.

\par 55 Again when a man wished to direct his gaze to the silver vessels, as they stood before him, everything seemed to flash with light round about the place where he was standing, and afforded a still greater delight to the onlookers.

\par 56 So that it is really impossible to describe the artistic beauty of the works.

\par 57 The golden vials they engraved in the centre with vine wreaths.

\par 58 And about the rims they wove a wreath of ivy and myrtle and olive in relief work and inserted precious stones in it.

\par 59 The other parts of the relief work they wrought in different patterns, since they made it a point of honour to complete everything in a way worthy of the majesty of the king.

\par 60 In a word it may be said that neither in the king's treasury nor in any other, were there any works which equalled these in costliness or in artistic skill.

\par 61 For the king spent no little thought upon them, for he loved to gain glory for the excellence of his designs.

\par 62 For oftentimes he would neglect his official business, and spend his time with the artists in his anxiety that they should complete everything in a manner worthy of the place to which the gifts were to be sent.

\par 63 So everything was carried out on a grand scale, in a manner worthy of the king who sent the gifts and of the high priest who was the ruler of the land.

\par 64 There was no stint of precious stones, for not less than five thousand were used and they were all of large size.

\par 65 The most exceptional artistic skill was employed, so that the cost of the stones and the workmanship was five times as much as that of the gold.

\par \textit{Footnotes}

\par \textit{148:1 A cubit is 18 inches.}

\chapter{4}

\par \textit{Vivid details of the sacrifice. The unerring accuracy of the priests is notable. A savage orgy. A description of the temple and its water-works.}

\par 1 I HAVE given you this description of the presents because I thought it was necessary.

\par 2 The next point in the narrative is an account of our journey to Eleazar, but I will first of all give you a description of the whole country.

\par 3 When we arrived in the land of the Jews we saw the city situated in the middle of the whole of Judea on the top of a mountain of considerable altitude.

\par 4 On the summit the temple had been built in all its splendour.

\par 5 It was surrounded by three walls more than seventy cubits high and in length and breadth corresponding to the structure of the edifice.

\par 6 All the buildings were characterised by a magnificence and costliness quite unprecedented.

\par 7 It was obvious that no expense had been spared on the door and the fastenings, which connected it with the door-posts, and the stability of the lintel.

\par 8 The style of the curtain too was thoroughly in proportion to that of the entrance.

\par 9 Its fabric owing to the draught of wind was in perpetual motion, and as this motion was communicated from the bottom and the curtain bulged out to its highest extent, it afforded a pleasant spectacle from which a man could scarcely tear himself away.

\par 10 The construction of the altar was in keeping with the place itself and with the burnt offerings which were consumed by fire upon it, and the approach to it was on a similar scale.

\par 11 There was a gradual slope up to it, conveniently arranged for the purpose of decency, and the ministering priests were robed in linen garments, down to their ankles.

\par 12 The Temple faces the east and its back is toward the west.

\par 13 The whole of the floor is paved with stones and slopes down to the appointed places, that water may be conveyed to wash away the blood from the sacrifices, for many thousand beasts are sacrificed there on the feast days.

\par 14 And there is an inexhaustible supply of water, because an abundant natural spring gushes up from within the temple area.

\par 15 There are moreover wonderful and indescribable cisterns underground, as they pointed out to me, at a distance of five furlongs all round the site of the temple, and each of them has countless pipes so that the different streams converge together.

\par 16 And all these were fastened with lead at the bottom and at the sidewalls, and over them a great quantity of plaster had been spread, and every part of the work had been most carefully carried out.

\par 17 There are many openings for water at the base of the altar which are invisible to all except to those who are engaged in the ministration, so that all the blood of the sacrifices which is collected in great quantities is washed away in the twinkling of an eye.

\par 18 Such is my opinion with regard to the character of the reservoirs and I will now show you how it was confirmed.

\par 19 They led me more than four furlongs outside the city and bade me peer down towards a certain spot and listen to the noise that was made by the meeting of the waters, so that the great size of the reservoirs became manifest to me, as has already been pointed out.

\par 20 The ministration of the priests is in every way unsurpassed both for its physical endurance and for its orderly and silent service.

\par 21 For they all work spontaneously, though it entails much painful exertion, and each one has a special task allotted to him.

\par 22 The service is carried on without interruption—some provide the wood, others the oil, others the fine wheat flour, others the spices; others again bring the pieces of flesh for the burnt offering, exhibiting a wonderful, degree of strength.

\par 23 For they take up with both hands the limbs of a calf, each of them weighing more than two talents, and throw them with each hand in a wonderful way on to the high place of the altar and never miss placing them on the proper spot.

\par 24 In the same way the pieces of the sheep and also of the goats are wonderful both for their weight and their fatness.

\par 25 For those, whose business it is, always select the beasts which are without blemish and specially fat, and thus the sacrifice which I have described, is carried out.

\par 26 There is a special place set apart for them to rest in, where those who are relieved from duty sit.

\par 27 When this takes place, those who have already rested and are ready to resume their duties rise up spontaneously since there is no one to give orders with regard to the arrangement of the sacrifices.

\par 28 The most complete silence reigns so that one might imagine that there was not a single person present, though there are actually seven hundred men engaged in the work, besides the vast number of those who are occupied in bringing up the sacrifices.

\par 29 Everything is carried out with reverence and in a way worthy of the great God.

\par 30 We were greatly astonished, when we saw Eleazar engaged in the ministration, at the mode of his dress, and the majesty of his appearance, which was revealed in the robe which he wore and the precious stones upon his person.

\par 31 There were golden bells upon the garment which reached down to his feet, giving forth a peculiar kind of melody, and on both sides of them there were pomegranates with variegated flowers of a wonderful hue.

\par 32 He was girded with a girdle of conspicuous beauty, woven in the most beautiful colours.

\par 33 On his breast he wore the oracle of God, as it is called, on which twelve stones, of different kinds, were inset, fastened together with gold, containing the names of the leaders of the tribes, according to their original order, each one flashing forth in an indescribable way its own particular colour.

\par 34 On his head he wore a tiara, as it is called, and upon this in the middle of his forehead an inimitable turban, the royal diadem full of glory with the name of God inscribed in sacred letters on a plate of gold . . . having been judged worthy to wear these emblems in the ministrations.

\par 35 Their appearance created such awe and confusion of mind as to make one feel that one had come into the presence of a man who belonged to a different world.

\par 36 I am convinced that any one who takes part in the spectacle which I have described will he filled with astonishment and indescribable wonder and be profoundly affected in his mind at the thought of the sanctity which is attached to each detail of the service.

\par 37 But in order that we might gain complete information, we ascended to the summit of the neighboring citadel and looked around us.

\par 38 It is situated in a very lofty spot, and is fortified with many towers, which have been built up to the very top, of immense stones, with the object, as we were informed, of guarding the temple precincts, so that if there were an attack, or an insurrection or an onslaught of the enemy, no one would be able to force an entrance within the walls that surround the temple.

\par 39 On the towers of the citadel engines of war were placed and different kinds of machines, and the position was much higher than the circle of walls which I have mentioned.

\par 40 The towers were guarded too by most trusty men who had given the utmost proof of their loyalty to their country.

\par 41 these men were never allowed to leave the citadel, except on feast days and then only in detachments, nor did they permit any stranger to enter it.

\par 42 They were also very careful when any command came from the chief officer to admit any visitors to inspect the place, as our own experience taught us.

\par 43 They were very reluctant to admit us—though we were but two unarmed men—to view the offering of the sacrifices.

\par 44 And they asserted that they were bound by an oath when the trust was committed to them, for they had all sworn and were bound to carry out the oath sacredly to the letter, that though they were five hundred in number they would not permit more than five men to enter at one time.

\par 45 The citadel was the special protection of the temple and its founder had fortified it so strongly that it might efficiently protect it.

\chapter{5}

\par \textit{A description of the city and the countryside. Compare Verse 11 with conditions of today. Verses 89-41 reveal how the ancients estimate a scholar and a gentleman.}

\par 1 THE size of the city is of moderate dimensions.

\par 2 It is about forty furlongs 1 in circumference, as far as one could conjecture.

\par 3 It has its towers arranged in the shape of a theatre, with thoroughfares leading between them now the crossroads of the lower towers are visible but those of the upper towers are more frequented.

\par 4 For the ground ascends, since the city is built upon a mountain.

\par 5 There are steps too which lead up to the crossroads, and some people are always going up, and others down and they keep as far apart from each other as possible on the road because of those who are bound by the rules of purity, lest they should touch anything which is unlawful.

\par 6 It was not without reason that the original founders of the city built it in due proportions, for they possessed clear insight with regard to what was required.

\par 7 For the country is extensive and beautiful.

\par 8 Some parts of it are level, especially the districts which belong to Samaria, as it is called, and which border on the land of the Idumeans, other parts are mountainous, especially those which are contiguous to the land of Judea.

\par 9 The people therefore are bound to devote themselves to agriculture and the cultivation of the soil that by this means they may have a plentiful supply of crops.

\par 10 In this way cultivation of every kind is carried on and an abundant harvest reaped in the whole of the aforesaid land.

\par 11 The cities which are large and enjoy a corresponding prosperity are well-populated, but they neglect the country districts, since all men are inclined to a life of enjoyment, for every one has a natural tendency towards the pursuit of pleasure.

\par 12 The same thing happened in Alexandria, which excels all cities in size and prosperity.

\par 13 Country people by migrating from the rural districts and settling in the city brought agriculture into disrepute: and so to prevent them from settling in the city, the king issued orders that they should not stay in it for more than twenty days. 2

\par 14 And in the same way he gave the judges written instructions, that if it was necessary to issue a summons against any one who lived in the country, the case must be settled within five days.

\par 15 And since he considered the matter one of great importance, he appointed also legal officers for every district with their assistants, that the farmers and their advocates might not in the interests of business empty the granaries of the city, I mean, of the produce of husbandry.

\par 16 I have permitted this digression because it was Eleazar who pointed out with great clearness the points which have been mentioned.

\par 17 For great is the energy which they expend on the tillage of the soil.

\par 18 For the land is thickly planted with multitudes of olive trees, with crops of corn and pulse, with vines too, and there is abundance of honey.

\par 19 Other kinds of fruit trees and dates do not count compared with these.

\par 20 There are cattle of all kinds in great quantities and a rich pasturage for them.

\par 21 Wherefore they rightly recognise that the country districts need a large population, and the relations between the city and the villages are properly regulated.

\par 22 A great quantity of spices and precious stones and gold is brought into the country by the Arabs.

\par 23 For the country is well adapted not only for agriculture but also for commerce, and the city is rich in the arts and lacks none of the merchandise which is brought across the sea.

\par 24 It possesses too suitable and commodious harbours at Askalon, Joppa, and Gaza, as well m at Ptolemais which was founded by the King and holds a central position compared with the other places named, being not far distant from any of them.

\par 25 The country produces everything in abundance, since it is well watered in all directions and well protected from storms.

\par 26 The river Jordan, as it is called, which never runs dry, flows through the land.

\par 27 Originally the country contained not less than 60 million acres—though afterwards the neighbouring peoples made incursions against it—and 600,000 men were settled upon it in farms of a hundred acres each.

\par 28 The river like the Nile rises in harvest-time and irrigates a large portion of the land.

\par 29 Near the district belonging to the people of Ptolemais it issues into another river and this flows out into the sea.

\par 30 Other mountain torrents, as they are called, flow down into the plain and encompass the parts about Gaza and the district of Ashdod.

\par 31 The country is encircled by a natural fence and is very difficult to attack and cannot be assailed by large forces, owing to the narrow passes, with theft overhanging precipices and deep ravines, and the rugged character of the mountainous regions which surround all the land.

\par 32 We were told that from the neighbouring mountains of Arabia copper and iron were formerly obtained.

\par 33 This was stopped, however, at the time of the Persian rule, since the authorities of the time spread abroad a false report that the working of the mines was useless and expensive in order to prevent their country from being destroyed by the mining in these districts and possibly taken away from them owing to the Persian rule, since by the assistance of this false report they found an excuse for entering the district.

\par 34 I have now, my dear brother Philocrates, given you all the essential information upon this subject in brief form.

\par 35 I shall describe the work of translation in the sequel.

\par 36 The High Priest selected men of the finest character and the highest culture, such as one would expect from their noble parentage.

\par 37 They were men who had not only acquired proficiency in Jewish literature but had studied most carefully that of the Greeks as well.

\par 38 They were specially qualified therefore for serving on embassies and they undertook this duty whenever it was necessary.

\par 39 They possessed a great facility for conferences and the discussion of problems connected with the law.

\par 40 They espoused the middle course—and this is always the best course to pursue.

\par 41 They abjured the rough and uncouth manner, but they were altogether above pride and never assumed an air of superiority over others, and in conversation they were ready to listen and give an appropriate answer to every question.

\par 42 And all of them carefully observed this rule and were anxious above everything else to excel each other in its observance and they were all of them worthy of their leader and of his virtue.

\par 43 And one could observe how they loved Eleazar by their unwillingness to be torn away from him and how he loved them.

\par 44 Far besides the letter which he wrote to the king concerning their safe return, he also earnestly besought Andreas to work for the same end and urged me, too, to assist to the best of my ability.

\par 45 And although we promised to give our best attention to the matter, he said that he was still greatly distressed, for he knew that the king out of the goodness of his nature considered it his highest privilege, whenever he heard of a man who was superior to his fellows in culture and wisdom, to summon him to his court.

\par 46 For I have heard of a fine saying of his to the effect that by securing just and prudent men about his person he would secure the greatest protection for his kingdom, since such friends would unreservedly give him the most beneficial advice.

\par 47 And the men who were now being sent to him by Eleazar undoubtedly possessed these qualities.

\par 48 And he frequently asserted upon oath that he would never let the men go if it were merely some private interest of his own that constituted the impelling motive-but it was for the common advantage of all the citizens that he was sending them.

\par 49 For, he explained, the good life consists in the keeping of the enactments of the law, and this end is achieved much more by hearing than by reading.

\par 50 From this and other similar statements it was clear what his feelings towards them were.

\par \textit{Footnotes}

\par \textit{154:1 A furlong is 1/8 mile (i. e. 220 yards).}

\par \textit{154:2 This account of the measures adopted at Alexandria to prevent the depopulation of the countryside through migrations into the city is an interesting revelation that the question was as acute 2000 years ago as it is today.}

\chapter{6}

\par \textit{Explanations of the customs of the people showing what is meant by the word, “Unclean.” The essence and origin of the “God-Belief.” Verses 48-44 give a picturesque description of the Divinity of physiology.}

\par 1 IT is worth while to mention briefly the information which he gave in reply to our questions.

\par 2 For I suppose that most people feel a curiosity with regard to some of the enactments in the law, especially those about meats and drinks and animals recognised as unclean.

\par 3 When we asked why, since there is but one form of creation, some animals are regarded as unclean for eating, and others unclean even to the touch (for though the law is scrupulous on most points, it is specially scrupulous on such matters as these) he began his reply as follows:

\par 4 ‘You observe,’ he said, ‘what an effect our modes of life and our associations produce upon us; by associating with the bad, men catch their depravities and become miserable throughout their life; but if they live with the wise and prudent, they find the means of escaping from ignorance and amending their lives.’

\par 5 Our lawgiver first of all laid down the principles of piety and righteousness and inculcated them point by point, not merely by prohibitions but by the use of examples as well, demonstrating the injurious effects of sin and the punishments inflicted by God upon the guilty.

\par 6 For he proved first of all that there is only one God and that his power is manifested throughout the universe, since every place is filled with his sovereignty and none of the things which are wrought in secret by men upon the earth escapes His knowledge.

\par 7 For all that a man does and all that is to come to pass in the future are manifest to Him.

\par 8 Working out these truths carefully and having made them plain, he showed that even if a man should think of doing evil—to say nothing of actually effecting it,—he would not escape detection, for he made it clear that the power of God pervaded the whole of the law.

\par 9 Beginning from his starting point, he went on to show that all mankind except ourselves believe in the existence of many gods, though they themselves are much more powerful than the beings whom they vainly worship.

\par 10 For when they have made statues of stone and wood, they say that they are the images of those who have invented something useful for life and they worship them, though they have clear proof that they possess no feeling.

\par 11 For it would be utterly foolish to suppose that any one became a god in virtue of his inventions.

\par 12 For the inventors simply took certain objects already created and by combining them together, showed that they possessed a fresh utility: they did not themselves create the substance of the thing, and so it is a vain and foolish thing for people to make gods of men like themselves.

\par 13 For in our times there are many who are much more inventive and much more learned than the men of former days who have been deified, and yet they would never come to worship them.

\par 14 The makers and authors of these myths think that they are the wisest of the Greeks.

\par 15 Why need we speak of other infatuated people, Egyptians and the like, who place their reliance upon wild beasts and most kinds of creeping things and cattle, and worship them, and offer sacrifices to them both while living and when dead?

\par 16 Now our Lawgiver being a wise man and specially endowed by God to understand all things, took a comprehensive view of each particular detail, and fenced us round with impregnable ramparts and walls of iron, that we might not mingle at all with any of the other nations, but remain pure in body and soul, free from all vain imaginations, worshipping the one Almighty God above the whole creation.

\par 17 Hence the leading Egyptian priests having looked carefully into many matters, and being cognizant with our affairs, call us “men of God.”

\par 18 This is a title which does not belong to the rest of mankind but only to those who worship the true God.

\par 19 The rest are men not of God but of meats and drinks and clothing.

\par 20 For their whole disposition leads them to find solace in these things are reckoned of no account, but throughout their things.

\par 21 Among our people such whole life their main consideration is the sovereignty of God.

\par 22 Therefore lest we should be corrupted by any abomination, or our lives be perverted by evil communications, he hedged us round on all sides by rules of purity, affecting alike what we eat, or drink, or touch, or hear, or see.

\par 23 For though, speaking generally, all things are alike in their natural constitution, since they are all governed by one and the same power, yet there is a deep reason in each individual case why we abstain from the use of certain things and enjoy the common use of others.

\par 24 For the sake of illustration I will run over one or two points and explain them to you.

\par 25 For you must not fall into the degrading idea that it was out of regard to mice and weasels and other such things that Moses drew up his laws with such exceeding care. 1

\par 26 All these ordinances were made for the sake of righteousness to aid the quest for virtue and the perfecting of character.

\par 27 For all the birds that we use are tame and distinguished by their cleanliness, feeding on various kinds of grain and pulse, such as for instance pigeons, turtle-doves, locusts, partridges, geese also, and all other birds of this class.

\par 28 But the birds which are forbidden you will find to be wild and carnivorous, tyrannising over the others by the strength which they possess, and cruelly obtaining food by preying of the tame birds enumerated above.

\par 29 And not only so, but they seize lambs and kids, and injure human beings too, whether dead or alive, and so by naming them unclean, he gave a sign by means of them that those, for whom the

legislation was ordained, must practise righteousness in their hearts and not tyrannise over any one in reliance upon their own strength nor rob them of anything, but steer their course of life in accordance with justice, just as the tame birds, already mentioned, consume the different kinds of pulse that grow upon the earth and do not tyrannise to the destruction of their own kindred.

\par 30 Our legislator taught us therefore that it is by such methods as these that indications are given to the wise, that they must be just and effect nothing by violence, and refrain from tyrannising over others in reliance upon their own strength.

\par 31 For since it is considered unseemly even to touch such unclean animals, as have been mentioned, on account of their particular habits, ought we not to take every precaution lest our own characters should be destroyed to the same extent?

\par 32 Wherefore all the rules which he has laid down with regard to what is permitted in the case of these birds and other animals, he has enacted with the object of teaching us a moral lesson.

\par 33 For the division of the hoof and the separation of the claws are intended to teach us that we must discriminate between our individual actions with a view to the practice of virtue.

\par 34 For the strength of our whole body and its activity depend upon our shoulders and limbs.

\par 35 Therefore he compels us to recognise that we must perform all our actions with discrimination according to the standard of righteousness,—more especially because we have been distinctly separated from the rest of mankind.

\par 36 For most other men defile themselves by promiscuous intercourse, thereby working great iniquity, and whole countries and cities pride themselves upon such vices.

\par 37 For they not only have intercourse with men but they defile their own mothers and even their daughters.

\par 38 But we have been kept separate from such sins.

\par 39 And the people who have been separated in the aforementioned way are also characterised by the Lawgiver as possessing the gift of memory.

\par 40 For all animals “which are cloven-footed and chew the cud” represent to the initiated the symbol of memory.

\par 41 For the act of chewing the cud is nothing else than the reminiscence of life and existence.

\par 42 For life is wont to be sustained by means of food, wherefor he exhorts us in the Scripture also in these words: “Thou shalt surely remember the Lord that wrought in thee those great and wonderful things.”

\par 43 For when they are properly conceived, they are manifestly great and glorious; first the construction of the body and the disposition of the food and the separation of each individual limb and, for more, the organisation of the senses, the operation and invisible movement of the mind, the rapidity of its particular actions and its discovery of the arts, display an infinite resourcefulness.

\par 44 Wherefore he exhorts us to remember that the aforesaid parts are kept together by the divine power with consummate skill.

\par 45 For he has marked out every time and place that we may continually remember the God who rules and preserves us.

\par 46 For in the matter of meats and drinks he bids us first of all offer part as a sacrifice and then forthwith enjoy our meal.

\par 47 Moreover, upon our garments he has given us a symbol of remembrance, and in like manner he has ordered us to put the divine oracles upon our gates and doors as a remembrance of God.

\par 48 And upon our hands, too, he expressly orders the symbol to be fastened, clearly showing that we ought to perform every act in righteousness, remembering our own creation, and above all the fear of God.

\par 49 He bids men also, when lying down to sleep and rising tip again, to meditate upon the works of God, not only in word, but by observing distinctly the change and impression produced upon them, when they are going to sleep, and also their waking, how divine and incomprehensible the change from one of these states to the other is.

\par 50 The excellency of the analogy in regard to discrimination and memory has now been pointed out to you, according to our interpretation of “the cloven hoof and the chewing of the cud.”

\par 51 For our laws have not been drawn up at random or in accordance with the first casual thought that occurred to the mind, but with a view to truth and the indication of right reason.

\par 52 For by means of the directions which he gives with regard to meats and drinks and particular cases of touching, he bids us neither to do nor listen to anything thoughtlessly nor to resort to injustice by the abuse of the power of reason.

\par 53 In the case of the wild animals, too, the same principle may be discovered.

\par 54 For the character of the weasel and of mice and such animals as these, which are expressly mentioned, is destructive.

\par 55 Mice defile and damage everything, not only for their own food but even to the extent of rendering absolutely useless to man whatever it falls in their way to damage.

\par 56 The weasel class, too, is peculiar: for besides what has been said, it has a characteristic which is defiling: It conceives through the ears and brings forth through the mouth.

\par 57 And it is for this reason that a like practice is declared unclean in men.

\par 58 For by embodying in speech all that they receive through the ears, they involve others in evils and work no ordinary impurity, being themselves altogether defiled by the pollution of impiety.

\par 59 And your king, as we are informed, does quite right in destroying such men.

\par 60 Then I said ‘I suppose you mean the informers, for he constantly exposes them to tortures and to painful forms of death.’

\par 61 ‘Yes,’ he replied, ‘these are the men I mean; for to watch for men's destruction is an unholy thing’.

\par 62 And our law forbids us to injure any one either by word or deed.

\par 63 My brief account of these matters ought to have convinced you, that all our regulations have been drawn up with a view to righteousness, and that nothing has been enacted in the Scripture thoughtlessly or without due reason, but its purpose is to enable us throughout our whole life and in all our actions to practise righteousness before all men, being mindful of Almighty God.

\par 64 And so concerning meats and things unclean, creeping things, and wild beasts, the whole system aims at righteousness and righteous relationships between man and man.

\par 65 He seemed to me to have made a good defence on all the points; for in reference also to the calves and rams and goats which are offered, he said that it was necessary to take them from the herds and flocks, and sacrifice tame animals and offer nothing wild, that the offerers of the sacrifices might understand the symbolic meaning of the lawgiver and not be under the influence of an arrogant self-consciousness.

\par 66 For he, who offers a sacrifice, makes an offering also of his own soul in all its moods.

\par 67 I think that these particulars with regard to our discussion are worth narrating, and on account of the sanctity and natural meaning of the law, I have been induced to explain them to you clearly, Philocrates, because of your own devotion to learning.

\par \textit{Footnotes}

\par \textit{158:1 Compare this quaint idea with 1 Corinthians, IX, 9.}

\chapter{7}

\par \textit{The arrival of the envoys with the manuscript of the precious book and gifts. Preparations for a royal banquet. The host immediately upon being seated at table entertains his guests with questions and answers. Some sage comments on sociology.}

\par 1 AND Eleazar, after offering the sacrifice, and selecting the envoys, and preparing many gifts for the king, despatched us on our journey in great security.

\par 2 And when we reached Alexandria, the king was at once informed of our arrival.

\par 3 On our admission to the palace, Andreas and I warmly greeted the king and handed over to him the letter written by Eleazar.

\par 4 The king was very anxious to meet the envoys, and gave orders that all the other officials should be dismissed and the envoys summoned to his presence at once.

\par 5 Now this excited general surprise, for it is customary for those who come to seek an audience with the king on matters of importance to be admitted to his presence on the fifth day, while envoys from kings or very important cities with difficulty secure admission to the Court in thirty days—but these men he counted worthy of greater honour, since he held their master in such high esteem, and so he immediately dismissed those whose presence he regarded as superfluous and continued walking about until they came in and he was able to welcome them.

\par 6 When they entered with the gifts which had been sent with them and the valuable parchments, on which the law was inscribed in gold in Jewish characters, for the parchment was wonderfully prepared and the connexion between the pages had been so effected as to be invisible, the king as soon as he saw them began to ask them about the books.

\par 7 And when they had taken the rolls out of their coverings and unfolded the pages, the king stood still for a long time and then making obeisance about seven times, he said:

\par 8 ‘I thank you, my friends, and I thank him that sent you still more, and most of all God, whose oracles these are.’

\par 9 And when all, the envoys and the others who were present as well, shouted out at one time and with one voice: ‘God save the King!’ he burst into tears of joy.

\par 10 For his exaltation of soul and the sense of the overwhelming honour which had been paid him compelled him to weep over his good fortune.

\par 11 He commanded them to put the rolls back in their places and then after saluting the men, said: ‘It was right, men of God, that I should first of all pay my reverence to the books for the sake of which I summoned you here and then when I had done that, to extend the right-hand of friendship to you.’

\par 12 ‘It was for this reason that I did this first.’

\par 13 ‘I have enacted that this day, on which you arrived, shall be kept as a great day and it will be celebrated annually throughout my life time.’

\par 14 ‘It happens also that it is the anniversary of my naval victory over Antigonus. Therefore I shall be glad to feast with you to-day.’

\par 15 ‘Everything that you may have occasion to use,’ he said, ‘shall be prepared for you in a befitting manner and for me also with you.’

\par 16 After they had expressed their delight, he gave orders that the best quarters near the citadel should be assigned to them, and that preparations should be made for the banquet.

\par 17 And Nicanor summoned the lord high steward, Dorotheus, who was the special officer appointed to look after the Jews, and commanded him to make the necessary preparation for each one.

\par 18 For this arrangement had been made by the king and it is an arrangement which you see maintained to-day.

\par 19 For as many cities as have special customs in the matter of drinking, eating, and reclining, have special officers appointed to look after their requirements.

\par 20 And whenever they come to visit the kings, preparations are made in accordance with their own customs, in order that there may be no discomfort to disturb the enjoyment of their visit.

\par 21 The. same precaution was taken in the case of the Jewish envoys.

\par 22 Now Dorotheus who was the patron appointed to look after Jewish guests was a very conscientious man.

\par 23 All the stores which were under his control and set apart for the reception of such guests, he brought out for the feast.

\par 24 He arranged the seats in two rows in accordance with the king's instructions.

\par 25 For he had ordered him to make half the men sit at his right hand and the rest behind him, in order that he might not withhold from them the highest possible honour.

\par 26 When they had taken their seats he instructed Dorotheus to carry out everything in accordance with the customs which were in use amongst his Jewish guests.

\par 27 Therefore he dispensed with the services of the sacred heralds and the sacrificing priests and the others who were accustomed to offer the prayers, and called upon one of our number, Eleazar, the oldest of the Jewish priests, to offer prayer instead.

\par 28 And he rose up and made a remarkable prayer. ‘May Almighty God enrich you, O king, with all the good things which He has made and may He grant you and your wife and your children and your comrades the continual possession of them as long as you live!’

\par 29 At these words a loud and joyous applause broke out which lasted for a considerable time, and then they turned to the enjoyment of the banquet which had been prepared.

\par 30 All the arrangements for service at table were carried out in accordance with the injunction of Dorotheus.

\par 31 Among the attendants were the royal pages and others who held places of honour at the king's court.

\par 32 Taking an opportunity afforded by a pause in the banquet the king asked the envoy who sat in the seat of honour (for they were arranged according to seniority), how he could keep his kingdom unimpaired to the end?

\par 33 After pondering for a moment he replied, ‘You could best establish its security if you were to imitate the unceasing benignity of God. For if you exhibit clemency and inflict mild punishments upon those who deserve them in accordance with their deserts, you will turn them from evil and lead them to repentance.’ 1

\par 34 The king praised the answer and then asked the next man, how he could do everything for the best in all his actions?

\par 35 And he replied, ‘If a man maintains a just bearing towards all, he will always act rightly on every occasion, remembering that every thought is known to God. If you take the fear of God as your starting-point, you will never miss the goal.’

\par 36 The king complimented this man, too, upon his answer and asked another, how he could have friends like-minded with himself?

\par 37 He replied, ‘If they see you studying the interests of the multitudes over whom you rule; you will do well to observe how God bestows his benefits on the human race, providing for them health and food and—all other things in due season.’

\par 38 After expressing his agreement with the reply, the king asked the next guest, how in giving audiences and passing judgments he could gain the praise even of those who failed to win their suit?

\par 39 And he said, ‘If you are fair in speech to all alike and never act insolently nor tyrannically in your treatment of offenders. And you will do this if you watch the method by which God acts. The petitions of the worthy are always fulfilled, while those who fail to obtain an answer to their prayers are informed by means of dreams or events of what was harmful in their requests and that God does not smite them according to their sins or the greatness of His strength, but acts with forbearance towards them.’

\par 40 The king praised the man warmly for his answer and asked the next in order, how he could be invincible in military affairs?

\par 41 And he replied, ‘If he did not trust entirely to his multitudes or his warlike forces, but called upon God continually to bring his enterprises to a successful issue, while he himself. discharged all his duties in the spirit of justice.’

\par 42 Welcoming this answer, he asked another how he might become. an object of dread to his enemies.

\par 43 And he replied, ‘If while maintaining a vast supply of arms and forces he remembered that these things were powerless to achieve a permanent and conclusive result. For even God instils fear into the minds of men by granting reprieves and making merely a display of the greatness of his power.’

\par 44 This man the king praised and then said to the next, ‘What is the highest good in life?’

\par 45 And he answered, ‘To know that God is Lord of the Universe, and that in our finest achievements it is not we who attain success but God who by his power brings all things to fulfilment and leads us to the goal.’

\par 46 The king exclaimed that the man had answered well and then asked the next how he could keep all his possessions intact and finally hand them down to his successors in the same condition?

\par 47 And he answered, ‘By praying constantly to God that you may be inspired with high motives in all your undertakings and by warning your descendants not to be dazzled by fame or wealth, for it is God who bestows all these gifts and men never by themselves win the supremacy.’

\par 48 The king expressed his agreement with the answer and inquired of the next guest, how he could bear with equanimity whatever befell him?

\par 49 And he said, ‘If you have a firm grasp of the thought that all men are appointed by God to share the greatest evil as well as the greatest good, since it is impossible for one who is a man to be exempt from these. But God to whom we ought always to pray, inspires us with courage to endure.’

\par 50 Delighted with the man's reply, the king said that all their answers had been good. ‘I will put a question to one other,’ he added, ‘and then I will stop for the present: that we may turn our attention to the enjoyment of the feast and spend a pleasant time.’

\par 51 Thereupon he asked the man, ‘What is the true aim of courage?’

\par 52 And he answered, ‘If a right plan is carried out in the hour of danger in accordance with the original intention. For all things are accomplished by God to your advantage, O king, since your purpose is good.’

\par 53 When all had signified by their applause their agreement with the answer, the king said to the philosophers (for not a few of them were present), ‘It is my opinion that these men excel in virtue and possess extraordinary knowledge, since on the spur of the moment they have given fitting answers to these questions which I have put to them, and have all made God the starting-point of their words.’

\par 54 And Menedemus, the philosopher of Eretria, said, ‘True, O King—for since the universe is managed by providence and since we rightly perceive that man is the creation of God, it follows that all power and beauty of speech proceed from God.’

\par 55 When the king had nodded his assent to this sentiment, the speaking ceased and they proceeded to enjoy themselves. When evening came on, the banquet ended.

\par \textit{Footnotes}

\par \textit{162:1 Compare this attitude toward criminals with that of the so-called modern humanitarian view. Also Bee Chapter VIII. 11.}

\chapter{8}

\par \textit{More questions and answers. Note Verse 20 with its reference to flying through the air written in 150 B. C.}

\par 1 ON the following day they sat down to table again and continued the banquet according to the same arrangements.

\par 2 When the king thought that a fitting opportunity had arrived to put inquiries to his guests, he proceeded to ask further questions of the men who sat next in order to those who had given answers on the previous day.

\par 3 He began to open the conversation with the eleventh man, for there were ten who had been asked questions on the former occasion.

\par 4 When silence was established, he asked how he could continue to be rich?

\par 5 After a brief reflection, the man who had been asked the question replied—‘If he did nothing unworthy of his position, never acted licentiously, never lavished expense on empty and vain pursuits, but by acts of benevolence made all his subjects well disposed towards himself. For it is God who is the author of all good things and Him man must needs obey.’

\par 6 The king bestowed praise upon him and then asked another how he could maintain the truth?

\par 7 In reply to the question he said, ‘By recognizing that a lie brings great disgrace upon all men, and more especially upon kings. For since they have the power to do whatever they wish, why should they resort to lies? In addition to this you must always remember, O King, that God is a lover of the truth.’

\par 8 The king received the answer with great delight and looking at another said, ‘What is the teaching of wisdom?’

\par 9 And the other replied, ‘As you wish that no evil should befall you, but to be a partaker of all good things, so you should act on the same principle towards your subjects and offenders, and you should mildly admonish the noble and good. For God draws all men to Himself by his benignity.’

\par 10 The king praised him and asked the next in order how he could be the friend of men?

\par 11 And he replied, ‘By observing that the human race increases and is born with much trouble and great suffering: wherefore you must not lightly punish or inflict torments upon them, since you know that the life of men is made up of pains and penalties. For if you understood everything you would be filled with pity, for God also it pitiful!’

\par 12 The king received the answer with approbation and inquired of the next, ‘What is the most essential qualification for ruling?’

\par 13 ‘To keep oneself,’ he answered, ‘free from bribery and to practise sobriety during the greater part of one’s life, to honour rightousness above all things, and to make friends of men of this type. For God, too, is a lover of justice! Having signified his approval, the king said to another, ‘What is the true mark of piety?’

\par 14 And he replied, ‘To perceive that God constantly works in the Universe and knows all things, and no man who acts unjustly and works wickedness can escape His notice. As God is the benefactor of the whole world, so you, too, must imitate Him and be void of offence!’

\par 15 The king signified his agreement and said to another, ‘What is the essence of kingship?’

\par 16 And he replied, ‘To rule oneself well and not to be led astray by wealth or fame to immoderate or unseemly desires, this is the true way of ruling if you reason the matter well out. For all that you really need is yours, and God is free from need and benignant withal. Let your thoughts be such as become a man, and desire not many things but only such as are necessary for ruling!’

\par 17 The king praised him and asked another man, how his deliberations might be for the best?

\par 18 And he replied, ‘If he constantly set justice before him in everything and thought that injustice was equivalent to deprivation of life. For God always promises the highest blessings to the just!’

\par 19 Having praised him, the king asked the next, how he could be free from disturbing thoughts in his sleep?

\par 20 And he replied, ‘You have asked me a. question which is very difficult to answer, for we cannot bring our true selves into play during the hours for sleep, but are held fast in these by imaginations that cannot be controlled by reason. For our souls possess the feeling that they actually see the things that enter into our consciousness during sleep. But we make a mistake if we suppose that we are actually sailing on the sea in boats or flying through the air 1 or travelling to other regions or anything else of the kind. And yet we actually do imagine such things to be taking place.’

\par 21 So far as it is possible for me to decide, I have reached the following conclusion. You must in every possible way, O King, govern your words and actions by the rule of piety that you may have the consciousness that you are maintaining virtue and that you never choose to gratify yourself at the expense of reason and never by abusing your power do despite to righteousness.

\par 22 For the mind mostly busies itself in sleep with the same things with, which it occupies itself when awake. And he who has all his thoughts and actions set towards the noblest ends establishes himself in righteousness both when he is awake and when he is asleep. Wherefore. you must be steadfast in the constant discipline of self.

\par 23 The king bestowed praise on the man and said to another—‘Since you are the tenth to answer, when you have spoken, we will devote ourselves to the banquet.’ And then he put the question, how can I avoid doing anything unworthy of myself?

\par 24 And he replied, ‘Look always to your own fame and your own supreme position, that you may speak and think only such things as are consistent therewith, knowing that all your subjects think and talk about you. For you must not appear to be worse than the actors, who study carefully the rôle, which it is necessary for them to play, and shape all their actions in accordance with it. You are not acting a part, but are really a king, since God has bestowed upon you a royal authority in keeping with your character.’

\par 25 When the king had applauded loud and long in the most, gracious way, the guests were urged to seek repose. So when the conversation ceased, they devoted themselves to the next course of the feast.

\par 26 On the following day, the same arrangement was observed, and when the king found an opportunity of putting questions to the men, he questioned the first of those who had been left over for the next interrogation, What is the highest form of government?

\par 27 And he replied, ‘To rule oneself and not to be carried away by impulses. For all men possess a certain natural bent of mind. It is probable that most men have an inclination towards food and drink and pleasure, and kings a bent towards the acquisition of territory and great renown. But it is good that there should be moderation in all things.’

\par 28 ‘What God gives, that you must take and keep, but never yearn for things that are beyond your reach.’

\par 29 Pleased with these words, the king asked the next, how he could be free from envy?

\par 30 And he after a brief pause replied, ‘If you consider first of all that it is God who bestows on all kings glory and great wealth and no one is king by his own power. All men wish to share this glory but cannot, since it is the gift of God!’

\par 31 ‘The king praised the man in a long speech and then asked another, how he could despise his enemies?’

\par 32 And he replied, ‘if you show kindness to all men and win their friendship, you need fear no one. To be popular with all men is the best of good gifts to receive from God!’

\par 33 Having praised this answer the king ordered the next man to reply to the question, how he could maintain his great renown?

\par 34 And he replied that ‘If you are generous and large-hearted in bestowing kindness and acts of grace upon others, you will never lose your renown, but if you wish the aforesaid graces to continue yours, you must call upon God continually.’

\par 35 The king expressed his approval and asked the next, To whom ought a man to show liberality?

\par 36 And he replied, ‘All men acknowledge that we ought to show liberality to those who are well disposed towards us, but I think that we ought to show the same keen spirit of generosity to those who are opposed to us that by this means we may win them over to the right and to what is advantageous to ourselves. But we must pray to God that this may be accomplished, for he rules the minds of all men.’

\par 37 Having expressed his agreement with the answer, the king asked the sixth to reply to the question, to whom ought we to exhibit gratitude?

\par 38 And he replied, ‘To our parents continually, for God has given us a most important commandment with regard to the honour due to parents. In the next place He reckons the attitude of friend towards friend for He speaks of “a friend which is as thine own soul.” You do well in trying to bring all men into friendship with yourself.’

\par 39 The king spoke kindly to him and then asked the next, What is it that resembles beauty in value?

\par 40 And he said, ‘Piety, for it is the pre-eminent form of beauty, and its power lies in love, which is the gift of God. This you have already acquired and with it all the blessings of life.’

\par 41 The king in the most gracious way applauded the answer and asked another, how, if he were to fail, he could regain his reputation again in the same degree?

\par 42 And he said, ‘It is not possible for you to fail, for you have sown in all men the seeds of gratitude which produce a harvest of goodwill, and this is mightier than the strongest weapons and guarantees the greatest security. But if any man does fail, he must never again do those things which caused his failure, but he must form friendships and act justly. For it is the gift of God to be able to do good actions and not the contrary.’

\par 43 Delighted with these words, the king asked another, how he could be free from grief?

\par 44 And he replied, ‘If he never injured any one, but did good to everybody and followed the pathway of righteousness, for its fruits bring freedom from grief. But we must pray to God that unexpected evils such as death or disease or pain or anything of this kind may not come upon us and injure us. But since you are devoted to piety, no such misfortune will ever come upon you.’

\par 45 The king bestowed great praise upon him and asked the tenth, What is the highest form of glory?

\par 46 And he said, ‘To honour God, and this is done not with gifts and sacrifices but with purity of soul and holy conviction, since all things are fashioned and governed by God in accordance with His will. Of this purpose you are in constant possession as all men can she from your achievements in the past and in the present.’

\par 47 With loud voice the king greeted them all and spoke kindly to them, and all those who were present expressed their approval, especially the philosophers. For they were far superior to them [i. e. the philosophers] both in conduct and in argument, since they always made God their starting-point.

\par 48 After this the king to show his good feeling proceeded to drink the health of his guests.

\par \textit{Footnotes}

\par \textit{165:1 Written about 150 B. C.!}

\chapter{9}

\par \textit{Verse 8 epitomizes the value of knowledge. Verse 28, parental affection. Note especially the question in Verse 26 and the answer. Also note the question in Verse 47 and the answer. This is sage advice for business men.}

\par 1 ON the following day the same arrangements were e for the banquet, and the king, as soon as an opportunity occurred, began to put questions to the men who sat next to those who had already responded, and he said to the first ‘Is wisdom capable of being taught?’

\par 2 And he said, ‘The soul is so constituted that it is able by the divine power to receive all the good and reject the contrary.’

\par 3 The king expressed approval and asked the next man, What is it that is most beneficial to health?

\par 4 And he said, ‘Temperance, and it is not possible to acquire this unless God create a disposition towards it.’

\par 5 The king spoke kindly to the man and said to another, ‘How can a man worthily pay the debt of gratitude to his parents?’

\par 6 And he said, ‘By never causing them pain, and this is not possible unless God dispose the mind to the pursuit of the noblest ends.’

\par 7 The king expressed agreement and asked the next, how he could become an eager listener?

\par 8 And he said, ‘By remembering that all knowledge is useful, because it enables you by the help of God in a time of emergency to select some of the things which you have learned and apply them to the crisis which confronts you. And so the efforts of men are fulfilled by the assistance of God.’

\par 9 The king praised him and asked the next How he could avoid doing anything contrary to law?

\par 10 And he said, ‘If you recognize that it is God who has put the thoughts into the hearts of the lawgivers that the lives of men might be preserved, you will follow them.’

\par 11 The king acknowledged the man's answer and said to another, ‘What is the advantage of kinship?’

\par 12 And he replied, ‘If we consider that we ourselves are afflicted by the misfortunes which fall upon our relatives and if their sufferings become our own—then the strength of kinship is apparent at once, for it is only when such feeling is shown that we shall win honour and esteem in their eyes. For help, when it is linked with kindliness, is of itself a bond which is altogether indissoluble. And in the day of their prosperity we must not crave their possessions, but must pray God to bestow all manner of good upon them.’

\par 13 And having accorded to him the same praise as to the rest, the king asked another, how he could attain freedom from fear?

\par 14 And he said, ‘When the mind is conscious that it has wrought no evil, and when God directs it to all noble counsels.’

\par 15 The king expressed his approval and asked another, how he could always maintain a right judgement?

\par 16 And he replied, ‘If he constantly set before his eyes the misfortunes which befall men and recognized that it is God who takes away prosperity from some and brings others to great honour and glory.’

\par 17 The king gave a kindly reception to the man and asked the next to answer the question, how he could avoid a life of ease and pleasure?

\par 18 And he replied, ‘If he continually remembered that he was the ruler of a great empire and the lord of vast multitudes, and that his mind ought not to be occupied with other things, but, he ought always to be considering how he could best promote their welfare. He must pray, too, to God that no duty might be neglected.’

\par 19 Having bestowed praise upon him, the king asked the tenth, how he could recognize those who were dealing treacherously with him?

\par 20 And he replied to the question, ‘If he observed whether the bearing of those about him was natural and whether they maintained the proper rule of precedence at receptions and councils, and in their general intercourse, never going beyond the bounds of propriety in congratulations or in other matters of deportment. But God will incline your mind, O King, to all that is noble.’

\par 21 When the king had expressed his loud approval and praised them all individually (amid the plaudits of all who were present), they turned to the enjoyment of the feast.

\par 22 And on the next day, when the opportunity offered, the king asked the next man, What is the grossest form of neglect?

\par 23 And he replied, ‘If a man does not care for his children and devote every effort to their education. For we always pray to God not so much for ourselves as for our children that every blessing may be theirs. Our desire that our children may possess self-control is only realized by the power of God.’

\par 24 The king said that he had spoken well and then asked another, how he could be patriotic?

\par 25 ‘By keeping before your mind,’ he replied, ‘the thought that it is good to live and die in one's own country. Residence abroad 1 brings contempt upon the poor and shame upon the rich as though they had been banished for a crime. If you bestow benefits upon all, as you continually do, God will give you favour with all and you will be accounted patriotic.’

\par 26 After listening to this man, the king asked the next in order, how he could live amicably with his wife?

\par 27 And he answered, ‘By recognizing that womankind are by nature headstrong and energetic in the pursuit of their own desires, and subject to sudden changes of opinion through fallacious reasoning, and their nature is essentially weak. It is necessary to deal wisely with them and not to provoke strife. For the successful conduct of life. the steersman must know the goal toward which he ought to direct his course. It is only by calling upon the help of God that men can steer a true course of life at all times.’

\par 28 The king expressed his agreement and asked the next, how he could be free from error?

\par 29 And he replied, ‘If you always act with deliberation and never give credence to slanders, but prove for yourself the things that are said to you and decide by your own judgement the requests which are made to you and carry out everything in the light of your judgement, you will be free from error, O King. But the knowledge and practice of these things is the work of the Divine power.’

\par 30 Delighted with these words, the king asked another, how he could be free from wrath?

\par 31 And he said in reply to the question, ‘If he recognized that he had power over all even to inflict death upon them, if he gave way to wrath, and that it would be useless and pitiful if he, just because he was lord, deprived many of life.’

\par 32 ‘What need was there for wrath, when all men were in subjection and no one was hostile to him? It is necessary to recognize that God rules the whole world in the spirit of kindness and without wrath at all, and you,’ said he, ‘O King, must of necessity copy His example.’

\par 33 The king said that he had answered well and then inquired of the next man, What is good counsel?

\par 34 ‘To act well at all times and with due reflection,’ he explained, ‘comparing what is advantageous to our own policy with the injurious effects that would result from the adoption of the opposite view, in order that by weighing every point we may be well advised and our purpose may be accomplished. And most important of all, by the power of God every plan of yours will find fulfilment because you practise piety.’

\par 35 The king said that this man had answered well, and asked another, What is philosophy?

\par 36 And he explained, ‘To deliberate well in reference to any question that emerges and never to be carried away by impulses, but to ponder over the injuries that result from the passions, and to act rightly as the circumstances demand, practising moderation. But we must pray to God to instil into our mind a regard for these things.’

\par 37 The king signified his consent and asked another, how he could meet with recognition when traveling abroad?

\par 38 ‘By being fair to all men,’ he replied, ‘and by appearing to be inferior rather than superior to those amongst whom he was traveling. For it is a recognized principle that God by His very nature accepts the humble. And the human race loves those who are willing to be in subjection to them.’

\par 39 Having expressed his approval at this reply, the king asked another, how he could build in such a way that his structures would endure after him?

\par 40 And he replied to the question, ‘If his creations were on a great and noble scale, so that the beholders would spare them for their beauty, and if he never dismissed any of those who wrought such works and never compelled others to minister to his needs without wages.’

\par 41 For observing how God provides for the human race, granting them health and mental capacity and. all other gifts, he himself should follow His example by rendering to men a recompense for their arduous toil. 1 For it is the deeds that are wrought in righteousness that abide continually!

\par 42 The king said that this man, too, had answered well and asked the tenth, What is the fruit of wisdom?

\par 43 And he replied, ‘That a man should be conscious in himself that he has wrought no evil and that he should live his life in the truth. Since it is from these, O mighty King, that the greatest joy and steadfastness of soul and strong faith in God accrue to you if you rule your realm in piety.’

\par 44 And when they heard the answer they all shouted with loud acclaim, and afterwards the king in the fullness of his joy began to drink their healths.

\par 45 And on the next day the banquet followed the same course as on previous occasions, and when the opportunity presented itself the king proceeded to put questions to the remaining guests, and he said to the first, ‘How can a man keep himself from pride?’

\par 46 And he replied, ‘If he maintains equality and remembers on all occasions that he is a man ruling over men. And God brings the proud to nought, and exalts the meek and humble!’

\par 47 The king spoke kindly to him and asked the next, Whom ought a man to select as his counsellors?

\par 48 And he replied, ‘Those who have been tested in many affairs and maintain unmingled goodwill towards him and partake of his own disposition. And God manifests Himself to those who are worthy that these ends may be attained.’

\par 49 The king praised him and asked another, What is the most necessary possession for a king?

\par 50 ‘The friendship and love of his subjects,’ he replied, ‘for it is through this that the bond of goodwill is rendered indissoluble. And it is God who ensures that this may come to pass in accordance with your wish.’

\par 51 The king praised him and inquired of another, What is goal of speech? And he replied, ‘To convince your opponent by showing him his mistakes in a well-ordered army of arguments.’

\par 52 ‘For in this way you will win your hearer, not by opposing him, but by bestowing praise upon him with a view to persuading him. And it is by the power of God that persuasion is accomplished.’

\par 53 The king said that he had given a good answer, and asked another, how he could live amicably with the many different races who formed the population of his kingdom?

\par 54 ‘By acting the proper part towards each,’ he replied, ‘and taking righteousness as your guide, as you are now doing with the help of the insight which God bestows upon you.’

\par 55 The king was delighted by this reply, and asked another, ‘Under what circumstances ought a man to suffer grief?’

\par 56 ‘In the misfortunes that befall our friends,’ he replied, ‘when we see that they are protracted and irremediable. Reason does not allow us to grieve for those who are dead and set free from evil, but all men do grieve over them because they think only of themselves and their own advantage. It is by the power of God alone that we can escape all evil!’

\par 57 The king said that he had given a fitting answer, and asked another, how is reputation lost?

\par 58 And he replied, ‘When pride and unbounded self-confidence hold sway, dishonour and loss of reputation are engendered. For God is the Lord of all reputation and bestows it where He will.’

\par 59 The king gave his confirmation to the answer, and asked the next man, To whom ought men to entrust themselves?

\par 60 ‘To those,’ he replied, ‘who serve you from goodwill and not from fear or self-interest, thinking only of their own gain. For the one is the sign of love, the other the mark of ill will and time-serving.’

\par 61 ‘For the man who is always watching for his own gain is a traitor at heart. But you possess the affection of all your subjects by the help of the good counsel which God bestows upon you.’

\par 62 The king said that he had answered wisely, and asked another, What is it that keeps a kingdom safe?

\par 63 And he replied to the question, ‘Care and forethought that no evil may be wrought by those who are placed in a position of authority over the people, and this you always do by the help of God who inspires you with grave judgement.’

\par 64 The king spoke words of encouragement to him, and asked another, What is it that maintains gratitude and honour?

\par 65 And he replied, ‘Virtue, for it is the creator of good deeds, and by it evil is destroyed, even as you exhibit nobility of character towards all by the gift which God bestows upon you.’

\par 66 The king graciously acknowledged the answer and asked the eleventh (since there were two more than seventy), how he could in time of war maintain tranquillity of soul?

\par 67 And he replied, ‘By remembering that he had done no evil to any of his subjects, and that all would fight for him in return for the benefits which they had received, knowing that even if they lose their lives, you will care for those dependent on them. For you never fail to make reparation to any—such is the kind-heartedness with which God has inspired you.’

\par 68 The king loudly applauded them all and spoke very kindly to them and then drank a long draught to the health of each, giving himself up to enjoyment, and lavishing the most generous and joyous friendship upon his guests.

\par \textit{Footnotes}

\par \textit{169:1 There were foreign residents in those days too.}

\par \textit{170:1 The policy of a fair wage for a fair day's work is here seen to be not so modern as we sometimes think in what we are pleased to call this enlightened age.}

\chapter{10}

\par \textit{The questions and answers continue. Showing how the army officers ought to be selected. What man is worthy of admiration and other problems of daily life as true today as 2000 years ago. Verses 15-17 are notable for recommending the theatre. Verses 2i-22 describe the wisdom of electing a president or having a king.}

\par 1 ON the seventh day much more extensive preparations were made, and many others were present from the different cities (among them a large number of ambassadors).

\par 2 When an opportunity occurred, the king asked the first of those who had not yet been questioned, how he could avoid being deceived by fallacious reasoning?

\par 3 And he replied, ‘By noticing carefully the speaker, the thing spoken, and the subject under discussion, and by putting the same questions again after an interval in different forms. But to possess an alert mind and to be able to form a sound judgement in every case is one of the good gifts of God, and you possess it, O King.’

\par 4 The king loudly applauded the answer and asked another, Why is it that the majority of men never become virtuous?

\par 5 ‘Because,’ he replied, ‘all men are by nature intemperate and inclined to pleasure. Hence, injustice springs up and a flood of avarice. The habit of virtue is a hindrance to those who are devoted to a life of pleasure because it enjoins upon them the preference of temperance and righteousness. For it is God who is the master of these things.’

\par 6 The king said that he had answered well, and asked, What ought kings to obey? And he said, ‘The laws, in order that by righteous enactments they may restore the lives of men. Even as you by such conduct in obedience to the Divine command have laid up in store for yourself a perpetual memorial.’

\par 7 The king said that this man, too, had spoken well, and asked the next, Whom ought we to appoint as governors?

\par 8 And he replied, ‘All who hate wickedness, and imitating your own conduct act righteously that they may maintain a good reputation constantly. For this is what you do, O mighty King,’ he said, ‘and it is God who has bestowed upon you the crown of righteousness.’

\par 9 The king loudly acclaimed the answer and then looking at the next man said, ‘Whom ought we to appoint as officers over the forces?’

\par 10 And he explained, ‘Those who excel in courage and righteousness and those who are more anxious about the safety of their men than to gain a victory by risking their lives through rashness. For as God acts well towards all men, so too you in imitation of Him are the benefactor of all your subjects.’

\par 11 The king said that he had given a good answer and asked another, What man is worthy of admiration?

\par 12 And he replied, ‘The man who is furnished with reputation and wealth and power and possesses a soul equal to it all. You yourself show by your actions that you are most worthy of admiration through the help of God who makes you care for these things.’

\par 13 The king expressed his approval and said to another, ‘To what affairs ought kings to devote most time?’

\par 14 And he replied, ‘To reading and the study of the records of official journeys, which are written in reference to the various kingdoms, with a view to the reformation and preservation of the subjects. And it is by such activity that you have attained to a glory which has never been approached by others, through the help of God who fulfils all your desires.’

\par 15 The king spoke enthusiastically to the man and asked another, how ought a man to occupy himself during his hours of relaxation and recreation?

\par 16 And he replied, ‘To watch those plays which can be acted with propriety and to set before one's eyes scenes taken from life and enacted with dignity and decency is profitable and appropriate.’

\par 17 ‘For there is some edification to be found even in these amusements, for often some desirable lesson is taught by the most insignificant affairs of life. But by practising the utmost propriety in all your actions, you have shown that you are a philosopher and you are honoured by God on account of your virtue.’

\par 18 The king, pleased with the words which had just been spoken, said to the ninth man, how ought a man to conduct himself at banquets?

\par 19 And he replied, ‘You should summon to your side men of learning and those who are able to give you useful hints with regard to the affairs of your kingdom and the lives of your subjects (for you could not find any theme more suitable or more educative than this) since such men are dear to God because they have trained their minds to contemplate the noblest themes—as you indeed are doing yourself, since all your actions are directed by God.’

\par 20 Delighted with the reply, the king inquired of the next man, What is best for the people? That a private citizen should be made king over them or a member of the royal family?

\par 21 And he replied, ‘He who is best by nature. For kings who come of royal lineage are often harsh and severe towards their subjects. And still more is this the case with some of those who have risen from the ranks of private citizens, who after having experienced evil and borne their share of poverty, when they rule over multitudes turn out to be more cruel than the godless tyrants.’

\par 22 ‘But, as I have said, a good nature which has been properly trained is capable of ruling, and you are a great king, not so much because you excel in the glory of your rule and your wealth but rather because you have surpassed all men in clemency and philanthropy, thanks to God who has endowed you with these qualities.’

\par 23 The king spent some time in praising this man and then asked the last of all, What is the greatest achievement in ruling an empire?

\par 24 And he replied, ‘That the subjects should continually dwell in a state of peace, and that justice should be speedily administered in cases of dispute.’

\par 25 ‘These results are achieved through the influence of the ruler, when he is a man who hates evil and loves the good and devotes his energies to saving the lives of men, just as you consider injustice the worst form of evil and by your just administration have fashioned for yourself an undying reputation, since God bestows upon you a mind which is pure and untainted by any evil.’

\par 26 And when he ceased, loud and joyful applause broke out for some considerable time. When it stopped the king took a cup and gave a toast in honour of all his guests and the words which they had uttered.

\par 27 ‘Then in conclusion he said, I have derived the greatest benefit from your presence. I have profited much by the wise caching which you have given me in reference to the art of ruling.’'¡

\par 28 Then he ordered that three talents of silver should be presented to each of them, and appointed one of his slaves to deliver over the money.

\par 29 All at once shouted their approval, and the banquet became a scene of joy, while the king gave himself up to a continuous round of festivity.

\chapter{11}

\par \textit{For a comment on ancient stenography, see Verse 7. The translation is submitted for approval and accepted as read, and (Verse 23) a rising vote of approval is taken and unanimously carried.}

\par 1 I HAVE written at length and must crave your pardon, Philocrates.

\par 2 I was astonished beyond measure at the men and the way in which on the spur of the moment they gave answers which really needed a long time to devise.

\par 3 For though the questioner had given great thought to each particular question, those who replied one after the other had their answers to the questions ready at once and so they seemed to me and to all who were present and especially to the philosophers to be worthy of admiration.

\par 4 And I suppose that the thing will seem incredible to those who will read my narrative in the future.

\par 5 But it is unseemly to misrepresent facts which are recorded in the public archives.

\par 6 And it would not be right for me to transgress in such a matter as this. I tell the story just as it happened, conscientiously avoiding any error.

\par 7 I was so impressed by the force of their utterances, that I made an effort to consult those whose business it was to make a record of all that happened at the royal audiences and banquets.

\par 8 For it is the custom, as you know, from the moment the king begins to transact business until the time when he retires to rest, for a record to be taken of all his sayings and doings—a most excellent and useful arrangement.

\par 9 For on the following day the minutes of the doings and sayings of the previous day are read over before business commence, and if there has been any irregularity, the matter is at once set right.

\par 10 I obtained therefore, as has been said, accurate information from the public records, and I have set forth the facts in proper order since I know how eager you are to obtain useful information.

\par 11 Three days later Demetrius took the men and passing along the sea-wall, seven stadia long, to the island, crossed the bridge and made for the northern districts of Pharos.

\par 12 There he assembled them in a house, which had been built upon the sea-shore, of great beauty and in a secluded situation, and invited them to carry out the work of translation, since everything that they needed for the purpose was placed at their disposal.

\par 13 So they set to work comparing their several results and making them agree, and whatever they agreed upon was suitably copied out under the direction of Demetrius.

\par 14 And the session lasted until the ninth hour; after this they were set free to minister to their physical needs.

\par 15 Everything they wanted was furnished for them on a lavish scale. In addition to this Dorotheus made the same preparations for them daily as were made for the king himself—for thus he had been commanded by the king.

\par 16 In the early morning they appeared daily at the Court, and after saluting the king went back to their own place.

\par 17 And as is the custom of all the Jews, they washed their hands in the sea and prayed to God and then devoted themselves to reading and translating the particular passage upon which they were engaged, and I put the question to them, Why it was that they washed their hands before they prayed?

\par 18 And they explained that it was a token that they had done no evil (for every form of activity is wrought by means of the hands) since in their noble and holy way they regard everything as a symbol of righteousness and truth.

\par 19 As I have already said, they met together daily in the place which was delightful for its quiet and its brightness and applied themselves to their task.

\par 20 And it so chanced that the work of translation was completed in seventy-two days, just as if this had been arranged of set purpose.

\par 21 When the work was completed, Demetrius collected together the Jewish population in the place where the translation had been made, and read it over to all, in the presence of the translators, who met with a great reception also from the people, because of the great benefits which they had conferred upon them.

\par 22 They bestowed warm praise upon Demetrius, too, and urged him to have the whole law transcribed and present a copy to their leaders.

\par 23 After the books had been read, the priests and the elders of the translators and the Jewish community and the leaders of the people stood up and said, that since so excellent and sacred and accurate a translation had been made, it was only right that it should remain as it was and no alteration should be made in it

\par 24 And when the whole company expressed their approval, they bade them pronounce a curse in accordance with their custom upon anyone who should make any alteration either by adding anything or changing in any way whatever any of the words which had been written or making any omission.

\par 25 This was a very wise precaution to ensure that the book might be preserved for all the future time unchanged.

\par 26 When the matter was reported to the king, he rejoiced greatly, for he felt that the design which he had formed had been safely carried out.

\par 27 The whole book was read over to him and he was greatly astonished at the spirit of the lawgiver.

\par 28 And he said to Demetrius, “How is it that none of the historians or the poets have ever thought it worth their while to allude to such a wonderful achievement?”

\par 29 And he replied, ‘Because the law is sacred and of divine origin. And some of those who formed the intention of dealing with it have been smitten by God and therefore desisted from their purpose.’

\par 30 He said that he had heard from Theopompus that he had been driven out of his mind for more than thirty days because he intended to insert in his history some of the incidents from the earlier and somewhat unreliable translations of the law.

\par 31 When he had recovered a little, he besought God to make it clear to him why the misfortune had befallen him.

\par 32 And it was revealed to him in a dream, that from idle curiosity he was wishing to communicate sacred truths to common men, and that if he desisted he would recover his health.

\par 33 I have heard, too, from the lips of Theodektes, one of the tragic poets, that when he was about to adapt some of the incidents recorded in the book for one of his plays, he was affected with cataract in both his eyes.

\par 34 And when he perceived the reason why the misfortune had befallen him, he prayed to God for many days and was afterwards restored.

\par 35 And after, the king, as I have already said, had received the explanation of Demetrius on this point, he did homage and ordered that great care should be taken of the books, and that they should be sacredly guarded.

\par 36 And he urged the translators to visit him frequently after their return to Judea, for it was only right, he said, that he should now send them home.

\par 37 But when they came back, he would treat them as friends, as was right, and they would receive rich presents from him.

\par 38 He ordered preparations to be made for them to return home, and treated them most munificently.

\par 39 He presented each one of them with three robes of the finest sort, two talents of gold, a sideboard weighing one talent, all the furniture for three couches.

\par 40 And with the escort he sent Eleazar ten couches with silver legs and all the necessary equipment, a sideboard worth thirty talents, ten robes, purple, and a magnificent crown, and a hundred pieces of the finest woven linen, also bowls and dishes, and two golden beakers to be dedicated to God.

\par 41 He urged him also in a letter that if any of the men preferred to come back to him, not to hinder them.

\par 42 For he counted it a great privilege to enjoy the society of such learned men, and he would rather lavish his wealth upon them than upon vanities.

\par 43 And now Philocrates, you have the complete story in accordance with my promise.

\par 44 I think that you find greater pleasure in these matters than in the writings of the mythologists.

\par 45 For you are devoted to the study of those things which can benefit the soul, and spend much time upon it. I shall attempt to narrate whatever other events are worth recording, that by perusing them you may secure the highest reward for your zeal.


\end{document}