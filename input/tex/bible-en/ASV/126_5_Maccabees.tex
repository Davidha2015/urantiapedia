\begin{document}

\title{5 Maccabees}


\chapter{1}

\par \textit {The attempt of Heliodorus on the treasury. Ir was ordained by the kings of the Grecian}

1 It was ordained by the kings of the Grecian Gentiles that large sums of money should be sent into the holy city’ every year, and should be delivered to the priests, that they might add it to the treasury of the house of God, as money for the receivers of alms [orphans ] and for widows. 

2 Now Seleucus was king in Macedonia: and he had a friend, one of his captains, who was called Heliodorus. This man was sent to spoil the treasury, and to take whatever money was therein. 

3 When this was noised abroad, it created great grief among the citizens; and they were afraid lest Heliodorus should proceed to greater lengths; 

4 as they had not sufficient power to prevent him from executing his orders. 

5 Wherefore they all fly to God ,for aid, and ordained a general fast, and supplicated with humility, bowing of the knees, and great wailing; 

6 putting on sackcloth, and rolling themselves in ashes, with Onias the high priest and the other princes — and elders, even to the common people, and women and children. 

7 And on the next day Heliodorus came into the house of God, with a train of followers; and entered into the house with his foot soldiers, he himself being on horseback, and was in search of the money. 

8 But the great and good God sent a loud, terrible voice upon him; and he saw a person armed with weapons of war, riding on a large horse, and advancing against him: 

9 wherefore he was seized with fear and trembling: and that person came up to him, and pulled him off from his saddle, and struck him with violence to the earth. 

10 So that being exceedingly terror-struck, and frightened out of his senses, he became dumb. 

11 But when his attendants saw what had befallen him, and could perceive no one who had done these things unto him, they carried him with all haste down to his own house: 

12 and he remained during several days, neither speaking nor taking any food. 

13 Wherefore the chief men of his friends went to Onias the priest, beseeching him to be appeased towards him, and to implore the great and good God that He would not punish him. 

14 Which thing Onias did; and Heliodorus was healed of his disease. 

15 And he saw in a vision the person, whom he had seen in the sanctuary, commanding him to go to Onias the priest, and to salute him, and pay him becoming honour; telling him, that the great and good God had heard his prayers, and had healed him at Onias’ request. 

16 Heliodorus therefore hastened to Onias the priest, whom falling down he saluted; and gave him money of various kinds!, requesting him to add it to that which was in the treasury. 

17 Then he went from Jerusalem into the country of Macedonia, and related to king Seleucus what had happened to him; entreating that he would not compel him to become his representative at Jerusalem. 

18 Wherefore the king wondered at the things which Heliodorus mentioned to him; and commanded him to publish them to the world. 

19 And he took care that his men should be removed and sent away from Jerusalem, increasing the gifts which he used to send thither annually, on account of what had befallen Heliodorus. 

20 And the kings added more to the money which they ordered to be given to the priests, that it might be spent on the orphans and widows; also to that which was to be spent on the sacrifices. 

\chapter{2}

\par \textit{The history of the translation of the twenty-four books out of the Hebrew tongue into the Greek tongue, for Ptolemy king of Egypt.}

1 There was a man of Macedon named Ptolemy, endued with knowledge and understanding; whom, as he dwelt in Egypt, the Egyptians made king over the country of Egypt. 

2 Wherefore he, being possessed with a desire of seeking out various knowledge, collected all the books of wise men from every quarter. 

3 And being anxious to obtain “the Twenty-four Books,” he wrote to the high priest in Jerusalem, to send him seventy elders from among those who were most skilled in those books;; and he sent to the priest a letter, with a present. 

4 So when the king’s letter came to the priest, he chose out seventy learned men, and sent them, together with a man named Eleazar, one excelling in religion, science, and learning: who departed into Egypt. 

5 And when their approach was made known to the king, he commanded seventy lodgings to be prepared, and the men to be there entertained. 

6 He also ordered a secretary to be appointed for each one, who should take down the interpretation of these books in the Greek character and language. 

7 He likewise forbade that any one of these should hold communication with any of his fellows; lest they should agree together to make any change in those books. 

8 So the secretaries took down from every one of them the translation of “the Twenty-four Books.” 

9 And when the translations were finished, Eleazar brought them to the king; and compared them together in his presence: on which comparison, they were found to agree. 

10 Upon which the king was exceeding glad, and ordered a large sum of money to be divided amongst the party. But Eleazar hanase(fi he rewarded with a munificent recompense. 

11 He also on that day set free every captive which was found in Egypt, of the tribe of Judah and of Benjamin, that they might return to their own country Syria. 

12 The number of them was about one hundred and thirty thousand. 

13 Moreover, he ordered money to be distributed among them, so that several denarii came to the share of each person; who, receiving these, departed into their own land. 

14 Then he commanded a great table to be made of the purest gold, which should be large enough to contain a representation of the whole land of Egypt, and a picture of the Nile, from the commencement of its stream to the end of it in Egypt, with its various divisions through the country, and how it laves the whole land. 

15 He also ordered the table to be set with many precious stones. 

16 And this table was made; and its carving was finished, and it was set with precious stones: and it was carried into the city of Jerusalem, a present to the magnificent house. 

17 And, arriving in safety, it was placed in the house, according to the king’s command. And truly men never be held its like, for the beauty of the pictures, and the excellence of the workmanship. 

\chapter{3}

\par \textit{The history of the Jews. A relation of what befell the Jews under king Antiochus; and what battles took place between them and his captains; and to what lengths he at last proceeded.}

1 There was a certain man of the kings of Macedon, who was called Antiochus; among whose deeds was this: 

2 that when Ptolemy the abovementioned king of Egypt was dead, he went with his armies to attack the second Ptolemy. And, having conquered and slain Ptolemy, he won his country? Egypt, and took possession of it. 

3 From hence, as his affairs gained 3 an accession of strength, he subdued a great part of the earth; the king of Persia and others paying him obedience. 

4 Wherefore his heart was lifted up: and being puffed up with pride, he commanded images to be made after his own likeness; that men should worship them, to his glorifying and honour. 

5 And when these were made, he sent messengers into all the regions of his empire, commanding them to be worshipped and adored. T’o these commands the nations assented, fearing and dreading his tyranny. 

6 Now there were at that time in Judea three men, the very worst of all mortals; and each of them had, as it were, a connexion in the same sort of vice. The name of one of these three was Menelaus; of the second, Simeon; of the third, Alcimus. 

7 And about that time there appeared" certain images, which the citizens of Jerusalem beheld in the air for the space of forty days: they were the appearances of men riding on fiery horses fighting with each other. 

8 So those impious men went to Antiochus, to obtain from him some authority, that they might perpetrate with ease whatever they wished, of whoredom, and plundering of men’s goods; and in short, might rule over the rest, and might keep them in subjection. And they said to him, 

9 “O king, there have appeared lately in the air over Jerusalem fiery horsemen, contending with each other: and on that account the Hebrews have rejoiced, saying, “that this portended the death of king Antiochus.” 

10 Which words the king believing, being filled with rage, he marched to Jerusalem in the shortest possible time; and came upon the nation not at all forewarned of his approach. 

11 And his men attacked the inhabitants, and slew them with the sword, making a very great slaughter; many also they wounded, and a great multitude they led into captivity. 

12 But some escaping fled into the mountains and woods, where they continued a long time, feeding upon herbs. 

13 After this, Antiochus determined to depart from the country. 

14 But the evil which he had done to the nation did not suffice him: but he left as his substitute a man named Felix, enjoining him to compel the Jews to worship his image, and to eat swine’s flesh. 

15 Which Felix did, sending for the people to obey the king in the things which he had commanded him. 

16 But they refused to do the things whereunto they were called; wherefore he slew a great multitude of them; preserving those wicked orn and their family, and raising their dignity.

\chapter{4}

\par \textit{The history of the death of Eleazar the priest}

1 Afterwards was seized Eleazar, who had gone with the doctors unto Ptolemy”, and was then a very old man, ninety years of age; and he was placed before Felix; 

2 who said to him, “Eleazar, truly you are a wise and prudent man; and indeed I have loved you for many years, and therefore I should not wish “your death: 

3 therefore obey the king, and worship his image, and eat of his sacrifices, and depart in safety.” 

4 To whom Eleazar replied; “I am not about to forsake my obedience to God, in order to obey the king.” 

5 And Felix, coming up, whispered to him, “Take care to send for some one to bring you flesh from your own offerings, which place upon my table: 

6 and eat some part of that in the presence of the people, that they may know that you have obeyed the king: and you will save your life, without any harm done to your religion.” 

7 Eleazar answered him, “I do not obey God under any kind of fraud, but rather I will endure this your violence. For inasmuch as I am an old man of ninety years, my bones are now weakened, and my body has wasted away. 

8 If I therefore shall with a brave spirit endure those torments, from which even the bravest young men shrink back in fear; my people and the youths of my nation will bravely imitate me, and will say; 

9 ‘How is it that we may not endure the pains, which one, who is inferior to us in strength, and less substantial in flesh and bones, has undergone?’ 

10 which indeed will be better for me, than to deceive them by a feigned obedience to the king: 

11 for they will then say, ‘If that decrepid old man, wise and prudent as he is, is clinging to life and overcome by the pain of temporary matters, abdicating his religion; truly that will be lawful for us which was lawful for him, since he is an old man and a wise one, and one whom we ought to follow.’ 

12 Wherefore I would rather die, leaving to them a constancy in religion and patience against tyranny; than live; after having weakened their constancy in obeying their Lord and following his commands; so that through me “they may be rendered happy, not unhappy.” 

13 Now when.Felix had heard the determination of Eleazar, he was violently enraged with him, and commanded him to be tortured in a variety of ways: so that he came into the most desperate mortal struggle, and said; 

14 “Thou, O God, knowest that I might have delivered myself from the troubles into which I have fallen, by obeying another rather than Thee. 

15 This however I have not done; but I have preferred obeying Thee, and have esteemed all the violence offered me as light, for the sake of constaney in obedience to Thee. 

16 And now I think little of the things which have happened to me according to thy good pleasure, and support them as well as I can. 

17 I therefore pray Thee, that Thou wilt accept this from me, and cause me to die before I become weaker in endurance.” 

18 And God heard his prayers;.and immediately he died. 

19 But he left his people devoted to the worship of their God, and endued with a sound fortitude, and perseverance in religion, and patience to bear up against the trials which awaited them. 

\chapter{5}

\par \textit{The history of the death of the seven brethren.}

1 After this, seven brothers were seized, and their mother; and they were sent to the king; for he had not yet gone far away from Jerusalem. 

2 And when they had been carried to the king, one of them was brought into his presence; whom he ordered to renounce his religion: 

3 but he refusing said to him, “If you think to teach us the truth for the first time, the matter is not so: 

4 for truth is that which we have learned from our fathers, and by which we have bound ourselves to embrace the worship of God only, and constantly to observe the law; and from this we in no wise will depart.” 

5 And king Antiechus was angry at these words, and commanded an iron frying pan to be brought, and to be placed on the fire. 

6 Then he ordered the young man’s tongue to be cut out, and his hands and feet to be cut off, and the skin of his head to be flayed off, and to be placed in the pan: and they did so to him. 

7 Then he commanded a large brazen caldron to be brought and set over the fire, into which the srest of his body was thrown. 

8 And when the man was near dying, he ordered the fire to be removed from him, that he might be tortured the longer: intending by these acts to terrify his mother and his brethren. 

9 But in fact by this he gave them additional courage and strength, to maintain their religion with constancy, and to bear all those torments which tyranny could inflict upon them. 

10 So when the first was dead, the second was brought before him: to whom some of the attendants said, “Obey those orders which the king will give you, lest you perish even as your brother perished.” 

11 But he answered, “I am not weaker in spirit than my brother, nor behind him in my faith. Bring forward your fire and sword; and do not diminish ought of that which you did to my brother.” And they did to him as had been done to his brother. 

12 And he called out to the king, and said to him; “Hear, O thou monster of cruelty towards men, and know that thou gainest nothing of ours except our bodies; but our souls thou dost by no means obtain; and these shortly will go to their Creator, 

13 whom He will restore to their bodies, when He shall raise to life> the dead men of his nation and the slain ones of his people.” 

14 And the third was brought out; who beckoning with his hand said to the king; “Why dost thou frighten us, O enemy? 

15 know that this is sent upon us from heaven, which also we undergo as such, giving thanks to God, and from Him we hope for our reward.” 

16 And the king, and those who stood near him, admired the courage of the youth, and the firmness of his mind, and his fair discourse. Then he gave orders, and he was slain. 

17 And the fourth was brought out, who said: “For God’s religion we set our lives to sale, and hire them out, that we may require payment from Him, on that day when you shall have no excuse in the judgment, and shall not be able to endure your tortures.” 

18 The king commanded, and he was put to death. 

19 And the fifth was. brought out, who said to him; “Think not within thyself that God has forsaken ‘us, because of the things which He has sent upon us. 

20 But truly his will is, to shew us honour and love by these things; and He will avenge us of thee and of thy posterity.” 

21 And the king commanded, and he was slain. 

22 And the sixth was brought out, who said; “I confess indeed my offences to God, but I believe that they shall be forgiven me through his my death. 

23 But you have now opposed God, by slaying those who embrace His religion: and surely He will repay you according to your works, and will root you out from his earth.” And he gave orders for him, and he was slain. 

24 And the seventh was brought out, who was a boy. 

25 Then his mother arose, fearless and unmoved, and looked upon¢ the corpses of her children: 

26 and she said, “My sons, I know not how I conceived each one of you, when I did conceive him. Nor had I the power of giving him breath; or of bringing him forth to the light of this world; or of bestowing on him courage and understanding: 

27 but indeed the great and good God himself formed him according to his own will: and gave to him a form according to his good pleasure: 

28 and brought him into the world by his power; appointing to him a term of life, and good rules, and a dispensation of religion, as it pleaseth Him. 

29 But you now have sold to God your bodies which he himself formed, and your souls which he created: and you have acquiesced in his judgments which he hath decreed. 

30 Wherefore, happy are ye, in the things which happily you have obtained; and blessed are ye, for the things in which you “have been victorious.” 

31 Now Antiochus had supposed, when he beheld her rise up, that she had done this through being overcome by fear for her child; and he wholly thought that she was about to enjoin him obedience to the king, that he might not perish as his brethren had perished. 

32 But when he had heard her words, he was ashamed, and blushed, and commanded the boy to be brought to him; that he might exhort him, and persuade him to love life, and deter him from death: 

33 lest all those should be seen to oppose his authority, and very many others should follow their example. 

34 Therefore, when he was brought to him, he exhorted him by discourse, and promised him riches, and sware to him that he would make him viceroy to himself. 

35 But when the boy was not at all moved 35 by his words, and took no heed of them; the king turned to his mother, and said to her; 

36 “Happy woman, pity this thy son, whom alone thou hast surviving; and exhort him to comply with my orders, and to escape those sufferings which have happened to his brethren.” 

37 And she said, “Bring him hither, that I may exhort him in the words of God.” 

38 And when they had brought him to her, she went aside from the crowd: then she kissed him, and laughed to scorn the things which had been said to her by Antiochus: 

39 and then said to him; My son, come now, be obedient to me, because I have brought you forth, and suckled you, and educated you, and taught you divine religion. 

40 “Look up now to the heaven, and the earth, and the water, and the fire; and understand that the one true God himself created these; and formed man of flesh and blood, who lives a short time, and then will die. 

41 Wherefore fear the true God, who dieth not: and obey the true Being, 

42 who changeth not his promises: and fear not this mere giant 4: and die for God’s religion, as your brothers have died. 

43 For if you could see, my son, their honourable dwelling-place, and the light of their habitation, and to what glory they have attained, you would not endure not to follow them: 

44 and in truth I also hope that the great and good God will prepare me, and that I shall closely follow you.” 

45 Then said the boy; “Know ye that I well obey God, and will not obey the commands of Antiochus: wherefore, delay not to let me follow my brothers; hinder me not from departing to the place whither they have gone.” 

46 Then to the king he said; “Woe to thee from God! whither wilt thou fly from Him? where wilt thou seek for refuge? or whose help wilt thou implore, that He may not take vengeance on thee? 

47 Truly thou hast done us a kindness, when thou hadst designed to do us evil: thou hast done evil to thine own soul, and hast destroyed it, while thou thoughtest to do it good. 

48 Now we are on our way to a life which death shall never follow; and shall dwell in light which dark‘mess shall never put away. 

49 But your dwelling shall be in the infernal regions, with exquisite punishments from God. 

50 And I trust, that the wrath of God will depart from his people, on account of what we have suffered for them: 

51 but that you He will torment in this world, and bring you to a wretched death; and that afterwards you will depart into eternal torments.” 

52 And Antiochus was angry, seeing that the boy opposed his authority; wherefore he commanded him to be tortured even more than his brothers. And this was done, and he died. 

53 But their mother intreated God, and besought Him that she might follow her sons; and immediately she died. 

54 Then Antiochus departed into his country Macedonia: and he wrote to Felix, and to the other governors‘ in Syria, that they should slay all the Jews, except those who should embrace his religion. 

55 And his servants obeyed his command, putting a multitude of men to death. 

\chapter{6}

\par \textit{The history of Mattathias the high priest, the son of Jochanan, who is the son of Hesmai the priest}

1 A Certain man named Mattathias, the son of Jochanan, fled to one of the mountains which were fortified. And the men who were scattered abroad fled thither to him: and some concealed themselves in secluded places. 

2 But after that Antiochus had departed to a greater distance from the country, Mattathias sent his son Judas secretly into the cities of Judah; 

3 to certify them of his own and his people’s health, and to desire that as many as were inspired with courage, magnanimity, and zeal for religion, for their wives, and their children, should come unto him. 

4 And certain of the higher orders of the people, who had stayed behind, went out to him: who, when they were come to him, said to them; 

5 “Nothing is left to us, but prayer to God, and confidence in Him, and a fight with our enemies, if perhaps God will give us assistance and the victory over them.” 

6 And the people assented to the opinion of Mattathias, and they acted according to it. 

7 And it was told to Felix; and he marched against them with a great army. 

8 And word was brought to him, while on his march, that about a thousand of the people of the Jews, men and _women mixed, were assembled together, and dwelling in a certain cave, that they might be enabled to preserve their own way of worship. 

9 And he turned aside to them with some part of his troops, sending the commanders of his men with the rest of the army against Mattathias. 

10 Now Felix demanded from those who were in the cave, that they should come out to him, and consent to enter 11 into his religion; but they refused. 

11 Whereupon he threatened that he would put smoke under them; and they endured that, and did not come out to him; and he put smoke under them, and they all died.

12 And when the generals of his army were marching against Mattathias, and came even to him, he being ready for battle; 

13 one of the generals, of noble blood, went to him, proposing to him to obey the king, and that he should not oppose his authority; so that he himself might live, and those who were with him, and might not perish. 

14 To whom he said; “I indeed obey God the true king: but do you obey your king, and do whatsoever seems good to you.” And he ceased from speaking. 

15 And they began to lay snares for him. 

16 And there came a certain man, of the worst of the Jews who were with them, and excited them to march against him and to prepare war. 

17 And Mattathias rushed on him with his drawn sword, and cut off the Jew’s head: then he struck the general’, to whom the Jew was speaking, and slew him also. 

18 But Mattathias’ companions, seeing what he 18 had done, hastened to him; and they burst into the camp of the enemy, slaying great numbers of them, and put them to flight: afterwards they pursued the fugitives, until they slew the whole of them. 

19 After this, Mattathias blew the trumpet, and proclaimed an expedition against Felix. And he and his companions entered into the land of Judah, and took possession of very many of their cities. 

20 And the most high God gave them rest by his hands from the generals of Antiochus: and they returned to the observance of their own religion: and the bands of their enemies retreated from before them. 

\chapter{7}

\par \textit{The account of the death of Mattathias, and the acts of Judas his son after him.}

1 Now Mattathias became infirm. And when he was near to death, he called his sons, who were five, and said unto them: 

2 “I know of a certainty that very many and great wars will be kindled in the land of Judah, for the sake [or, by reason] of those matters for which the great and good God has stirred us up to wage war against our enemies. 

3 But I charge you that you fear God, and trust in him, and be zealous of the law, and the sanctuary, and the people also; 

4 and prepare yourselves to wage war against its enemies: and fear not death, because, without doubt, this is decreed unto all men. 

5 So that, if God shall make you victorious, you have at once obtained that which you were longing for: but if you fall, that is no loss to you in his sight.” 

6 And Mattathias died and was buried; and his sons did according to that which he had commanded them. And they agreed to make their brother Judas their leader. 

7 Now Judas their brother was the best in counsel, and bravest in strength of them all. 

8 And an army was sent against them by Felix”, under a man who was called Seron’, whom Judas with his company put to flight, and he slew great numbers. 

9 And the fame of Judas was spread abroad, and increased greatly in the ears of men: and all the nations which were round about him feared him exceedingly. 

10 And it was told to king Antiochus what Mattathias and his son Judas had done. 

11 News of this came also to the king of the Persians; so that he played false with Antiochus, departing from his friendship, following the example of Judas. 

12 Which giving Antiochus a great deal of uneasiness, he called to him one of his household officers named Lysias‘, a stout and brave man, and said to him; 

13 I have now determined to go into the land of Persia to make war; and I wish to leave behind me my son in my stead; and to take with me the half of my army, and to leave the remainder with my son: 

14 and behold I have given to you the governance of my son, and the governance of the men whom I leave with him. 

15 And verily you know what Mattathias and Judas have done to my friends and my subjects. 

16 Wherefore, send one to lead a powerful army into the land of Judah; and command him to attack the land of Judah with the sword, and to root them out, and to demolish their dwellings, and to destroy all traces of them.” 

17 Then Antiochus departed into the country of Persia. 

18 But Lysias made ready three hardy and brave generals, skilled in war; of whom one was named. Ptolemy, a second Nicanor, and the third Gorgias. 

19 And with them he sent forty thousand chosen troops and seven thousand horsemen. He also charged them to bring with them an army of Syrians, and Philistines; and ordered them to root out the Jews entirely. 

20 And they marched forth, carrying with them a multitude of merchants, that they might sell to them the captives which they were about to obtain from among the Jews. 

21 But tidings of this came to Judas the son of Mattathias; and he went to the house of the great and good God; 

22 and assembled his men, and enjoined them a fast, and supplications, and prayers to the great and good God; and charged that they should beseech Him for victory against their enemies; which thing they did. 

23 After this, Judas collecting his men, appointed over each thousand a chief, and likewise over each hundred, and over each fifty, and over each ten. 

24 Then he commanded proclamation to be niade by trumpet throughout his army, that whosoever was fearful, and whomsoever God commanded to be dismissed from the army, he should return home. 

25 And great numbers returned; and there remained with them seven thousand stout and brave men, skilled in wars and accustomed thereto; nor had any one of them ever fled: and they marched against their enemies. 

26 But when they had drawn nigh to them, Judas prayed to his Lord, intreating Him that He would turn away from him the malice of his enemy; and that He would assist him, and render him victorious. 

27 Then he commanded the priests to sound the trumpets, which they did: and all his men called upon God, and rushed upon the army of Nicanor. 

28 And God gave them victory over them, and they turned him and his men to flight, killing of them nine thousand men, and the rest were dispersed. 

29 And Judas and his company returned to Nicanor’s camp, and made spoil of it; and plundered very much property of the merchants, and sent it to be divided among the sick. 

30 This battle took place on the sixth day of the week; wherefore Judas and his men remained on the same spot until the sabbath-day had passed. 

31 Then they marched against Ptolemy and Gorgias, whom they found and defeated, and gained a victory over them, slaying twenty thousand of their troops. 

32 And Ptolemy and Gorgias fled; whom Judas and his company pursued; yet he could not overtake them, because they betook themselves into a city of two idols, and fortified themselves therein with the remnant of their army. 

33 And Judas attacked Felix; and he was put to flight before him. And Judas pursued him. Who, coming to a certain house which was nigh at hand, entered into it and closed the doors, for it was a fortified house. 

34 And Judas commanded, and he set fire to it; and the house was burned, and Felix was burned in it. So Judas took vengeance on him for Hleazar and the others whom Feelix had put to death. 

35 Afterwards the people returned to the slain, and took their spoils and their armour; but the best of the prey they sent into the Holy Land. 

36 But Nicanor departed in disguise unknown, and returned to Lysias, and told him all which had happened to him and his company. 

\chapter{8}

\par \textit{The relation of Antiochus’ return, and of his going into the land of Judah, and of the disease which fell on him, of which he died in his journey.}

1 But Antiochus returned out of the country of Persia, flying, with his army disbanded. 

2 And when he had learned what had hap1+ pened to his army which Lysias had sent forth, and to all his men, he went out with a large army, marching to the land of Judah. 

3 Now when in his progress he had reached the middle of his journey, God smote his troops with most mighty weapons: 

4 but this could not stop him from his journey; but he persisted in it, uttering all sorts of insolence against God, and saying that no one could turn him aside, nor hinder him from his determined purposes. 

5 Wherefore the great and good 5 God smote him also with ulcers which attacked the whole of his body: but even yet he did not desist, nor refrain from his journey; 

6 but was more filled with wrath, and inflamed with an eager desire to obtain what he had determined on, and to carry his resolution into effect. 

7 Now there were in his army very many elephants. It so happened that one of these ran away, and made a bellowing: upon which the horses which were drawing the couch on which Antiochus lay, ran off, and threw_him out. 

8 And, as he was fat and corpulent, his limbs were bruised, and some of his joints were dislocated. 

9 And the bad smell of his ulcers, which already sent forth a foetid odour, was so much increased, that neither he himself could longer endure it, nor could those who came near him. 

10 So when he fell, his servants took him up, and carried him upon their shoulders: but as the foul smell grew worse, they threw him down and departed to a distance. 

11 Therefore, perceiving the evils which surrounded him, he believed for certain that all that punishment had come upon him from the great and good God; by reason of the injury and the tyranny which he had used towards the Hebrews, and the unjust shedding of their blood. 

12 In fear therefore he turned himself to God, and, confessing his sins, said; “O God, in truth I deserve the things which Thou hast sent upon me: and indeed just art Thou in thy judgments; 

13 Thou humblest him who is exalted, and bringest down him who is puffed up: but thine is greatness, and magnificence, and majesty, and prowess. 

14 Truly, I own, I have oppressed the people, and have both acted and decreed tyrannically against them. 

15 Forgive, I pray Thee, O God, this my error; and wipe out my sin, and bestow on me my health: and my care shall be to fill the treasury of thy house with gold and silver: 

16 and to strew the floor of the house of thy sanctuary with purple vestments; and to be circumcised; and to proclaim throughout all my kingdom, that Thou only art the true God, without any partner, and that there is no God besides thee.” 

17 But God did not hear his prayers, nor accept his supplication: but his troubles so increased on him that he voided his bowels: and his ulcers increased to that degree, that his flesh dropt off from his body. 

18 Then he died, and was buried in his place. And his son reigned in his stead, whose name was Eupator.

\chapter{9}

\par \textit{The history of the eight days of dedication}

1 When Judas had put to flight Ptolemy, and Nicanor, and Gorgias, and had slain their men; he himself and his troops returned into the country” of the holy house. 

2 And he commanded all the altars to be destroyed which Antiochus had ordered to be built: 

3 and he removed all the idols which were in the sanctuary: and they built up a new altar, and he commanded sacrifices to be offered upon that. 

4 They prayed also 4 to the great and good God, that He would bring forth the holy fire which might remain upon the altar: 

5 and fire came out from some stones of the altar, and burned up the wood and the sacrifices; and from it fire continued on the altar until the third carrying into captivity’. 

6 And then they kept the festival of the new altar for eight days, beginning on the twenty-fifth day of the month Casleu. 

7 And then they placed bread on the table of the house of God, and lighted the lamps of the candlestick. 

8 And on each of these eight days they assembled together for prayer and praise: and moreover they appointed it an ordinance for every year to come. 

\chapter{10}

\par \textit{The history of Judas’ battles with Gorgias and Ptolemy}

1 Now after the days of dedication, Judas marched into the country of the Idumzans, to the mountain Sarah for Gorgias was staying there. 

2 And Gorgias went out against him with a great army, and there were sore battles betwixt them; and there fell of Gorgias’ men twenty thou sand. 

3 And Gorgias fled to Ptolemy into the land of the west, (for Antiochus had made him governor of that country, and there he was staying,) and told him what had befallen him. 

4 Whereupon Ptolemy went forth with an army, in which were a hundred and twenty thousand men of Macedonia and the east. 

5 And he went on until he came to the country of Giares, (that is to say, Gilead,) and the parts adjacent; and he slew great numbers of the Jews. 

6 So they wrote to Judas, telling him what had happened to them, begging him to come and defeat Ptolemy and drive him away from them. 

7 And their ‘letter reached him at the same time that a letter came to him from the inhabitants of the mountain of Galilee I likewise, informing him how the Macedonians; who were at Tyre and Sidon had now united against them, and had attacked them, killing several. 

8 Now when Judas had read both the letters, he 8 called together his men, and shewed them the contents of the letters, and appointed a fast and supplication. 

9 After this, he ordered his brother Simeon to take with him three thousand men of the Jews, and to march with all speed to the mountain of Galilee, «and to quell the Macedonians who were there. 

10 And Simeon went. But Judas 10 hastened to encounter Ptolemy. 

11 And Simeon attacked the Macedonians unexpectedly, and slew of them eight thousand men, and gave rest to the Galilzans. 

12 But Judas marched on until he came up with Gorgias and Ptolemy; pressing them and besieging them: and the two armies encountered, and very fierce battles took place betwixt them. 

13 For Ptolemy headed a numerous, stout, and brave body of men. But Judas was accompanied by a very small band: 

14 yet, as the people who were with him consisted of the bravest and strongest troops, he steadily resisted, and the battle between them lasted long, and grew very sore. 

15 Wherefore Judas called out to the great and good God, and invoked his aid. 

16 And he related that he had seen five youthful horsemen, three of whom fought against Ptolemy’s army, and two stood near him self. 

17 Whom when he viewed attentively, they 18 seemed to him to be angels of God. 

18 Wherefore his heart was comforted, and the hearts of his companions; and making frequent assaults upon the enemy, they put them to flight, and slew great multitudes of them. 

19 And the number of those who were slain of Ptolemy’s army, from the beginning of this battle until the end, was twenty thousand and five hundred. 

20 After these things, Ptolemy and his men fled to the sea-coast; while Judas pursued them, and slew as many of them as he caught. 

21 But Ptolemy fled to Gaza, and remained there; and the men of Chalisam came to him. 

22 And Judas marched against them; and when he found them, he defeated them: and Ptolemy’s men were dispersed, but he himself fled to Gaza, and there fortified 93 himself. 

23 And Judas’ men pursued the flying body, and slew great numbers of them. And Judas and the men who were with him marched straight to Gaza, and he pitched his camp and besieged it. 

24 And Judas’ men returned to him; and they who were left of Ptolemy’s forces went up upon the fortification, and abused Judas with much reviling. 

25 And the fighting between them and Judas’ troops lasted for five days. But when the fifth day was come, the people continued to cast reproaches upon Judas, and to revile his religion: 

26 whereupon twenty of Judas’ men grew angry; who taking shields on their left hands, and swords in their right, and having with them a man bearing a ladder which they had made, marched until they came to the wall: 

27 and eighteen of them stood and threw darts at those who were on the wall; and two, hastening to the wall, raised up the ladder, and by it mounted. 

28 But certain of those who were there, perceiving that they had ascended, and that their companions had followed, and also had gone down from the wall into the city, descended from the wall after them: whom Judas’ men defeated, slaying great numbers of their enemies. 

29 But the army of Judas pressed forward to the gate of the city; and the twenty began to run toward the gate that they might open it: but they were driven thence most fiercely; wherefore they called out with loud cries. 

30 Judas therefore and his men knew that they had come near to the gate: and the battle grew sore both without the gate and within. 

31 And Judas and his men attacked the gate with fire, and it fell down; and the people perished, and the men who had reviled Judas, were taken, and he commanded them to be brought out and burned. 

32 Moreover he commanded the city to be utterly smitten with the sword; and the slaughter continued in it for two days, and then it was wasted with fire. 

33 But Ptolemy fled; nor were tidings of him heard at that time; because that he had changed his clothes, and concealed himself in one of the pits , and no account of him was had. 

34 But his two brothers were taken, and brought to Judas; and he ordered them to be beheaded. 

35 After this he went into the land of the sanctuary, with abundance of spoil; and both he and his company offered prayers therein, giving thanks to God for the benefits which they had received. 


\chapter{11}

\par \textit{The relation of the batile between Judas and Lysias the general of Eupator, after the death of king Antiochus}

1 The name of Antiochus, of whom mention has been made above, was Epiphanius: but the name of his son who reigned after him was Eupator, who also was named Antiochus. 

2 And when the battles of Judas with these generals had taken place, they» wrote on the subject to Eupator; who sent with Lysias, his cousin’s son, a large army, in which were eighty thousand horsemen and eighty elephants. 

3 Who coming to a city which is called Bethner, pitched their camp around it, and besieged it, because it was a large city, and much people was therein. 

4 And Lysias raised engines of war around it, and began to besiege the inhabitants: 

5 which being told to Judas, he himself and his company went out to some fortified mountains; 

6 and there they abode; lest if they remained in any city, Lysias should come and besiege it, and should overpower them. 

7 Judas therefore collected his company, and resolved to march with them to Lysias’ camp, after they should have gone to the house of God and offered sacrifices in it; 

8 beseeching the great and good God to turn away from them the malice of their enemies, and to grant them victory over them: which thing they did. 

9 After this, they marched from the region of the holy house to Bethner. For they had designed to come upon the army suddenly, and to defeat it without a struggle. 

10 Now men say, that there appeared to Judas a certain personage between heaven and earth, riding on a fiery-horse, and holding in his hand a large spear, with which he smote the army of the Gentiles. 

11 So that what they had seen gave them additional courage and spirits. And they made haste and charged the army, and slew great numbers of its men. 

12 Wherefore the enemy’s army was troubled and thrown into the greatest confusion, and the whole of it betook itself to a confused flight. 

13 And the sword of Judas and. his company pressed sore upon them; and he slew of them eleven thousand footmen, and sixteen hundred horsemen. 

14 Liysias also was chased, with his company, to a distant place, in which he remained in safety.

15 And he sent to Judas, desiring him to be subject to the king, retaining his own and his people’s religion: 

16 to whom Judas consented in this matter, until word could be written to the king, and an answer of his agreeing thereto could be received. 

17 And Judas wrote concerning this business: Lysias also wrote to the king, informing him of what had happened, and what proof he had had of the strength and bravery of the Jewish nation; 

18 and that a continuation of wars with them would exterminate his men, as these beforementioned had been exterminated: he told him also their agreement, and his own waiting until he should receive a letter to say what he must do. 

19 To whom the king replied, that it seemed right to him to make peace with the nation of the Jews, taking away that stumblingblock concerning the exercise of their religion: for that this very thing had incited them to the revolts, and to the attacks made on his predecessors. 

20 He also commanded him to make with them a treaty of peace and obedience; so that no obstacles should be thrown in their way in the matter of religion. 

21 He wrote also to Judas, and to all the Jews who were in the land of Judah, according to this effect: and this peace continued between them for some space of time. 


\chapter{12}

\par \textit{An account of the beginning of the power of the Romans, and of the enlargement of their empire.}

1 At this same time, of which we have been speaking, the affairs of the Romans began to be exalted: that the great and good God might fulfil that which Daniel the prophet (to whom be peace) had foretold> concerning the fourth empire. 

2 There was also at this time a certain most munificent king in Africa, whose name was Annibal. And the royal seat of his empire was Carthage. He determined to take possession of the kingdom of the Romans: 

3 wherefore they united to oppose him; and wars were multiplied between them, so that they fought eighteen? battles in the space of ten years; and they were not able to drive him out of their country, by reason of his innumerable army and people. 

4 They determined therefore to raise a large force selected from their bravest troops and armies, and to attack Annibal in war, and to persevere until they should turn away his forces from them. 

5 Which thing truly they did: and they placed at the head of their armies two most renowned men; the name of one was Aimilius, and of the other Varro. 

6 Who meeting Annibal engaged with him; and there were slain of their army ninety thousand men; and of Annibal’s army forty thousand men were slain. 

7 But Varro fled into a certain very large and strong city called Venusia: him Annibal did not pursue; but he marched to Rome, to take it, and there to remain. 

8 So he lay before it for eight days, and began to build houses opposite to it; 

9 which when the citizens saw, they deliberated on entering into a peace and treaty with him, and on surrendering the country. 

10 But there was among them a certain young man named Scipio, (for the Romans at that time were without a king, and the entire administration of their affairs was committed to three hundred and twenty men, over whom presided a person who was called senior or elder.) 

11 Scipio therefore comes to these, and persuaded them not to trust to Annibal nor to submission to him. To whom they answered, that they did not trust him, but that they were unable to resist him. 

12 To whom he said; the country of Africa is wholly destitute of soldiers, because that they are all here with Annibal: give me therefore a troop of chosen men, that I may go into Africa: 

13 and I will perform such feats in it, that when tidings of them shall reach him, perhaps he will quit you, and you will be freed from him, and will be in peace: and having retrieved and strengthened your resources, if he should prepare to return, you will be able to oppose him. 

14 And the advice of Scipio appeared to them to be right; and they committed to him thirty thousand of their bravest men. 

15 And he proceeded into Africa. And Asdrubal the brother of Annibal met him, and fought with him; whom Scipio defeated, and cut off his head, and took it, with the rest of the prey, and returned to Rome. 

16 And mounting upon the rampart, he called to Annibal, and said: How will you be able to prevail against this our country, when you are not able to expel me from your own land, to which I have gone: I have destroyed it, and have killed your brother, and have brought away his head. 

17 Then he threw the head to him. Which being brought to Annibal and recognised by him, he was increased in fury and anger against the people, and sware that he would not depart till he had taken Rome. 

18 But the citizens, to withdraw him from them, and keep him in check, took counsel to send back Scipio to besiege and attack Carthage. 

19 And Scipio returned with his army into Africa: and they pitched their camp around Carthage, and besieged it with a most active siege. 

20 Wherefore the inhabitants wrote to Annibal, saying, You are coveting a foreign country, which you know not whether you will be able to win or not: but there has come to your own country one who is endeavouring to gain possession of it. 

21 Wherefore, if you delay coming, we will surrender to him the country, and will give up your family and all your substance and your treasures; that we and our property may go unhurt. 

22 Now when this letter was brought to him, he departed from Rome; and hastened till he came into Africa: 

23 and Scipio went forward and met him, and fought a most fierce battle with him three times, and there were slain fifty thousand of his men. 

24 But Annibal, being put to flight, retired into the land of Egypt; whom Scipio pursued, and took him prisoner, and returned to Africa. 

25 And when he was there, Annibal disdained to be seen by the Africans; wherefore he took poison and died. 

26 And Scipio won the country of Africa, and possessed himself of all the goods, and servants, and treasures of Annibal. 

27 By which means the fame of the Romans was magnified, and their power from that time began to receive increase. 

\chapter{13}

\par \textit{An account of the letter of the Romans to Judas, and of the treaty which took place between them.}

1 “From the elder and three hundred and twenty governors, unto Judas, general of the army, and to the Jews. 

2 Health be to you. We have already heard of your victories, and courage, and endurance in war; whereof we rejoice. We have also understood that you have entered into an agreement with Antiochus. 

3 We write to you to this effect, that you should be friends to us, and not to the Greeks who have done you harm: moreover we intend to go to Antioch, and to make war upon its inhabitants: 

4 wherefore make haste to acquaint us with whom you are at enmity, and with whom you have a league of friendship; that we may act accordingly.” 

5 THE COPY OF THE TREATY. “This is the treaty made by the elder and three hundred and twenty governors with Judas, general of the army, and the Jews; that they should be joined to the Romans, and that the Romans and Jews may be of one mind in wars and victories for ever.

6 Now if war should come upon the Romans, Judas and his people shall help them, giving no aid to the enemies of the Romans, by provisions or by any kind of weapons. 

7 And when war shall come upon the Jews, the Romans shall help them to the utmost of their power, giving no aid to their enemies by assistance of any kind. 

8 And as the Jews are bound to the Romans, so likewise are the Romans to the Jews, without any increase or decrease.” 

9 And Judas and his people accepted this; and the treaty stood, and continued between them and the Romans for a long time. 

\chapter{14}

\par \textit{An account of the batile which took place between Judas, Ptolemy, and Gorgias.}

1 After this, Ptolemy collected an hundred and twenty thousand men, and a thousand horsemen, and they went after Judas. And Judas met him with ten thousand men, and routed him, and many of Ptolemy’s men were slain. 

2 And he besought Judas, and humbly entreated him to let him escape with his life; and swore that he would never more make war against him, and that he would shew kindness to the Jews who were in all his countries. 

3 And Judas had compassion on him, and let him go; and Ptolemy adhered to his oath. 

4 But Gorgias having collected three thousand men from mount Sarah’, (that 1s, of Idumea,) and four hundred horsemen, met Judas, and slew the captain of his army and certain of his men. 

5 Then Judas and his men advanced towards them; and Gorgias was put to flight, and the greater part of his army was killed or fled: and he was sought for, and no tidings were heard of him; but it is reported that he fell in the battle. 

\chapter{15}

\par \textit{An account of the dissolution of the treaty which Antiochus had made with Judas, and of his march (together with Lysias his cousin's son) with a great army, and of his wars.}

1 But when word was brought to Antiochus Eupator that Judas’ affairs had gained strength, and what victories he had gained, he was very angry; 

2 and broke the treaty which he had made with Judas, and collected a large army, in which were twenty-two elephants: 

3 and he marched with Lysias his cousin’s son into the country of Judah, directing his course to the city Beth-ner?, before which he pitched his camp, and besieged it. 

4 Now when this was reported to Judas, he and all the elders of the children of Israel met together, and prayed to the great and good God, offering many sacrifices; 

5 which being finished, Judas proceeded with the leaders of his forces, and came into the camp by night, and made a sudden attack upon it, and slew of the enemy four thousand men and one of the elephants: and he returned to his own camp until the dawn of day should begin to break. 

6 Then each army was drawn out, and the battle grew fierce between them. 

7 And Judas perceived one of the elephants with golden trappings, and he supposed that the king was sitting upon him: so he called his men, and said to them, Which of you will go out and kill this elephant? 

8 And a young man, one of his servants, who was called Eleazar, went out and rushed upon the enemy’s line, slaying on the right and left, so that the men turned aside out of his view; 

9 and he went forward until he came even to the elephant; and creeping under him, he cut open his belly; and the elephant fell down upon him, and he died. So the king perceiving this, commanded to sound a retreat; and it was done. 

10 And the amount of men of the higher rank slain that day in the battle was eight hundred men, besides those of the common men who were slain, and those who had been killed during the night.

11 Then it was told the king, that a certain man of his friends named Philip had revolted from him: and that Demetrius the son of Seleucus had gone forth from Rome with a great army of Romans, intending to take the kingdom out of his hand. 

12 At which being much affrighted, he sent to Judas concerning making peace between them: to which Judas assented; and Antiochus and Lysias his cousin’s son sware to him, that they would never more make war upon him. 

13 And the king displayed a large sum of money, and gave it to Judas for a present to the house of God. 

14 The king also commanded Menelaus to be seized, one of the threef wicked men who had brought evil on the Jews in the days of Antiochus his father; and he ordered him to be carried up to a lofty tower, and to be thrown headlong thence; which was done. 

15 For by this the king designed to gratify the Jews, since this man was one of their chief enemies, and had slain great numbers of them. 

\chapter{16}

\par \textit{The history of the arrival at Antioch of Demetrius the son of Seleucus, and of his defeating Eupator.}

1 After these things, king Eupator marched into the country of Macedonia, and then returned to Antioch. 

2 Whom Demetrius attacked with an army of Romans, and defeated, and slew, together with Lysias his cousin’s son; and he reigned at Antioch. 

3 But to him went Alcimus, the leader of those three? wicked men; who, coming into his presence, prostrated himself before him, and wept most vehemently, and said; 

4 “O king, Judas and his company have been slaying great numbers of us; because, having desert ed their religion, we have embraced the religion of the king. Wherefore, O king, assist us against them, and avenge us on them.”

5 Then he made the Jews go to him, and incensed him; suggesting to them such things as might provoke Demetrius, and irritate him to fit out an army to vanquish Judas. 

6 To whom the king giving heed, sent a general named Nicanor, with a great army and an abundant supply of weapons of war. 

7 And when Nicanor had come into the Holy Land, he sent messengers to Judas to come to him; and did not disclose that he had come to conquer the nation, 

8 but stated that he came only on account of the peace which was made between him and the nation, and that they also were under obedience to the Romans. 

9 And Judas went out to him with a certain number of his men, who were endued with strength and courage: and he commanded them not to go far from him, lest Demetrius might lay a snare for him.

10 When therefore he had met Demetrius, he saluted him; and, a seat being placed for each of them, they sat down, and Demetrius conversed with him as he pleased: afterwards each of them went into a tent which the troops had erected for him. 

11 And Nicanor and Judas departed into the Holy City, and there dwelt together: and a firm friend ship grew up between them: 

12 which being made known to Alcimus, he went to Demetrius and incensed him against Judas, and persuaded him to write and command Nicanor to send Judas to him bound in chains. 

13 But tidings of this came yo to Judas, and he went out from the city by night, and departed to Sebaste, and sent to his companions to come to him. 

14 And when they were come, he sounded the trumpet, and commanded them to prepare themselves to attack Nicanor. 

15 But Nicanor sought Judas with great diligence, and could learn no tidings of him. 

16 Wherefore he went to the house of God, requiring of the priests to give him up to him, that he might send him bound in chains to the king: but they sware that he had not come into the house of God. 

17 Whereupon he abused both them and the house of God, and spake insolently of the temple, and threatened that he would demolish it from the very foundations; and departed in a rage. He also took care to search all the houses of the Holy City. 

18 Likewise he sent his men to the 18 house of a certain excellent mans’, who had been seized in the time of Antiochus, and put to extreme torture; but after the death of Antiochus the Jews increased his authority and greatly honoured him. 

19 And when the messengers of 19 Nicanor came to him, he feared lest he should meet with the same treatment which he had received from Antiochus; wherefore he laid hands on himself. 

20 When this was told to Judas, he was very sorry and much afflicted: and he sent to Nicanor, saying; “Do not seek me in the city, for I am not there: therefore come forth to me, that we may meet each other, either in the plains or in the mountains, as you chuse.” 

21 And Nicanor went forth to him, and Judas met him with these words: “O God, it was Thou who didst exterminate the army of king Sennacherib; and he indeed was greater than this man, in fame, in empire, and in the multitude of his host: 

22 and Thou didst deliver Ezechiah king of Judah from him, when he had trusted in Thee and prayed to Thee: deliver us, I pray thee, O God, from his malice, and make us victorious over him.” 

23 Then he made ready himself for battle, and advanced to Nicanor, saying, “'Take care of your self, it is to you I come.” 

24 And Nicanor turned his back and fled: and Judas pursuing smote him on the shoulders, which he divided; and his men were put to flight. 

25 And there fell of them on that day thirty thousand: and the inhabitants of the cities went out and slew them, so that they left not one of them.

26 And they decreed that that day should be every year a day of thanksgiving to the great and good God, and a day of gladness, and of feasting, and of drinking. [Thus far is finished the Second Book from the translation of the Hebrews.] 

\chapter{17}

\par \textit{An account of the death of Judas}

1 But when nearly the same season of the year came round, Bacchides went forth with thirty thousand of the bravest of the Macedonians; 

2 and came upon Judas without any tidings thereof coming to him, when he was in a certain city called Lalis, with three thousand men: 

3 wherefore most of those who were with him fled; and there remained with him eight hundred men, and his brothers Simeon and Jonathan. 

4 But those who remained with Judas were the strongest and bravest, and who had already endured much in the several battles which he had fought. 

5 And Judas and his company went out to meet Bacchides and his army. 

6 And Bacchides divided his army, placing fifteen 6 thousand on the right hand of Judas and his company, and fifteen thousand on their left. 

7 Then each part shouted against Judas and his company. Who attentively regarding each, perceived that the enemy’s strongest and bravest troops were on the right, and found out that Bacchides himself was there among them. 

8 Judas likewise divided his company, and took the bravest of them with him, and gave the rest to his brothers. Then he made a charge upon those on the right, and he with his company slew about two thousand men. 

9 Then perceiving Bacchides, he directed his eyes and steps towards him, and slew all the bravest men who were about him. 

10 And he in person with his company sustained the multitudes which pressed upon him, felling to the ground the greater part of them, and he came near to Bacchides. 

11 Whom when Bacchides saw coming towards him like a lion, brandishing in his hand a large sword stained with blood, he was excessively afraid of him, and trembled, and fled out of his sight. 

12 And Judas with his company pursued him, and they slew his people with the sword, so that they put to death the greater part of those fifteen thousand: and Bacchides fled even to Ashdod. 

13 And the fifteen thousand which were on Judas’ left, followed him, and attacked Judas, to whom by this time were come his brothers and those who were with them, greatly fatigued. 

14 And those fifteen thousand rushed upon them, and a very great battle took place between them and Judas; and there fell on both sides a certain number of slain, in which number was Judas. 

15 Whom his brothers carried and buried beside the sepulchre of Mattathias his father, [God be merciful to them]; and the children of Israel bewailed him many years. 

16 Now the time of his governing was seven years, and Jonathan his brother succeeded him in the government. 

\chapter{18}

\par \textit{The history of Jonathan the son of Mattathias}

1 And Jonathan succeeded his brother, and he went to Jordan with a small number of men; which when Bacchides heard of, he marched to him with a large army. 

2 And when Jonathan saw him, his men swam over Jordan; and Bacchides and his army followed them, and surrounded them. 

3 But Jonathan rushed on Bacchides; and as the men gave way to Jonathan, he and his company went out from the midst of them, and departed to Beersheba: 

4 and his bro4 ther Simeon joined him, and they abode there; and they repaired whatever of the fortifications had fallen down, and they fortified themselves there.

5 But Bacchides marched to them, and besieged them: and Jonathan and his brother, and they which were with them, went out to him by night, and slew great numbers of his army, and burned the battering rams.and engines of war; 

6 and his army was dispersed, and Bacchides fled into the desert. And Jonathan and Simeon, and the men who were with him, pursued and took him. 

7 Who, when he saw Jonathan, knew that his death was near: wherefore he proclaimed peace with Jonathan, and sware that he would never more make war upon him, and moreover, that he would restore the whole of the captives which he had taken of the army of Judas. 

8 And Jonathan gave him his hand, and departed from him: nor after this was there any more war between them. And not long after this, Jonathan died, and his brother Simeon succeeded him. 


\chapter{19}

\par \textit{The history of Simeon the son of Mattathias}

1 Then Simeon the son of Mattathias succeeded to the government; and he gathered together all those who remained of the army of Judas: 

2 and his affairs prospered, and he subdued all those who had exercised hostility against the Jews after the death of his brother Judas; and he behaved well towards his people, and the matters of his country were rightly ordered. 

3 Wherefore Antiochus? attacked him, and also Demetrius the son of Seleucus; and sent a great army against him: 

4 to meet which, Simeon and his two sons went out; and he divided his army into two parts, one of which he kept with himself, and gave the other to his sons. 

5 Then he and they which were with him went to the army; and he sent his two sons and their followers by another way, and appointed with them to attack the army at a given time. 

6 After this, he met the army of Antiochus, and attacked it, and began to prevail against it: and his two sons came when the battle had now begun, and the fight grew fierce, and they came round the rear of the army. 

7 And Antiochus’ army, being placed between two armies, was cut to pieces, nor did a single man of them escape: nor did Antiochus return any more to fight with Simeon. 

8 And peace and quietness continued among the Jews all the days of Simeon. And the time of his government was two years. 

9 Then Ptolemy his son-in-law rushed on him, and slew him, at a certain feast where he was present. And he seized his wife and his two sons. And Simeon’s son, whose name was Hyrcanus, was set in his father’s place. 

\par [Here ends the history as given in the two books usually attached to our Bibles.]

\chapter{20}

\par \textit{The history of Hyrcanus the son of Simeon}

1 Now Simeon, while he was yet alive, had appointed Jochanan his son to be captain; and having gathered to him very many troops, he sent him to vanquish a certain man who had come out against him, and was called Hyrcanus. 

2 Now he was a man of great fame, powerful in strength, and of an ancient sovereignty. 

3 Whom Jonathan encountered, and defeated: wherefore Simeon named his son Jochanan Hyrcanus; on account of his slaying Hyrcanus, and gaining a victory over him. 

4 But when this Hyrcanus had heard that Ptolemy had killed his father, he was afraid of Ptolemy, and fled to Gaza: and Ptolemy pursued him with many followers. 

5 But the citizens of Gaza helped Hyrcanus, and shut the gates of their city, and hindered Ptolemy from reaching Hyrcanus. 

6 And Ptolemy returned, and departed to Dagon‘, having with him the mother of Hyrcanus and his two brothers. strongly fortified castle. Now Dagon had at that time a strongly fortified castle.

7 But Hyrcanus went to the Holy House, and offered sacrifices, and succeeded his father: and he collected a large army and went to attack Ptolemy. Wherefore Ptolemy shut the gate of Dagon upon himself and his company, and fortified himself therein. 

8 And Hyrcanus besieged him, and made an iron ram to batter the wall, and to open it: and the battle between them lasted long, 

9 and Hyrcanus prevailed against Ptolemy, and went up close to the castle, and almost took it. 

10 When Ptolemy therefore saw this, he commanded the mother of Hyrcanus and his two brothers to be brought out upon the wall, and to be tortured most severely; which was done to them. 

11 But Hyrcanus, seeing this, stood still; and fearing that they would be put to death, desisted from fighting. 

12 To whom his mother called out, and said; “My son, do not be moved by love and filial piety towards me and your brethren, in preference to your father: 

13 nor on account of our captivity be weakened in your desire of avenging him; but demand satisfaction for the rights of your father and mine, to the utmost of your power. 

14 But that which you fear for us from that tyrant, he will necessarily do to us at all events: wherefore press forward your siege without any intermission.” 

15 When therefore Hyrcanus had heard the words of his mother, he urged on the siege: wherefore Ptolemy increased the tortures of his mother and his brothers; and sware that he would throw them headlong from the castle, as often as Hyrcanus came near to the wall. 

16 Therefore Hyrcanus feared, lest he should be the cause of their death; and he returned to his camp, still continuing “the siege of Ptolemy. 

17 Now it happened, that the feast of tabernacles was at hand’; wherefore Hyrcanus went into the city of the Holy House, that he might be present at the feast and the solemnity and the sacrifices. 

18 And when Ptolemy knew that he had departed to the Holy City, and was detained there, he seized upon the mother of Hyrcanus and his brothers, and slew them; and he fled into a place whither Hyrcanus could not come. 



\chapter{21}

\par \textit{The history of the going up of Antiochus to the city of the Holy House, to fight with Hyrcanus.}

1 Now when Antiochus had heard that Simeon was dead, he collected an army, and marched until he came to the city of the Holy House: 

2 and he encamped around it, and besieged it, designing to take it by force: but he could not, by reason of the height and strength of the walls, and the multitude of warriors who were in it. 

3 But by God’s will he was restrained from winning it: for le had betaken himself to the northern side of the city, and had built there an hundred and thirty towers opposite to the wall; 

4 and had caused men to mount them, to fight with those who should endeavour to go up upon the walls of the city. 

5 He also appointed men to dig up the earth in a certain spot, till they came to the foundation of the wall: which finding to be of wood, they burned it with fire, and a very large portion of the wall fell down. 

6 And Hyrcanus’ men opposed them, and prevented them from entering, keeping guard over the ruined portion; 

7 and Hyrcanus went out with the better part of his fighting men against the army of Antiochus, and defeated them with great 8 slaughter. 

8 And Antiochus and his men were routed; whom Hyrcanus with his troops pursued, till they had driven them away from the city. 

9 Then returning to the towers which Antiochus had built, they destroyed them; and abode in the city, and around it. 

10 But Antiochus encamped in a certain place, which was distant from the city of the house of God about two furlongs. 

11 And at the approach of the feast of tabernacles Hyrcanus sent ambassadors to him, to treat for a truce until the solemnity should be passed; which he granted him; and sent victims, and gold and silver”, to the house of God. 

12 And Hyrcanus commanded the priests toFeceive what Antiochus had sent; and they did so. 

13 Now when Hyrcanus and the priests saw the reverence of Antiochus towards the temple of God, he sent ambassadors to him, to treat for peace. 

14 To which Antiochus agreed; and he went to Jerusalem: and Hyrcanus meeting him, they entered the city together. 

15 And Hyrecanus made a feast for Antiochus and his princes; and they did eat and drink together and he made him a present of three hundred talents of gold: 

16 and each of them agreed with his companion about peace and rendering assistance, and Antiochus departed into his own country. 

17 But it is related, that Hyrcanus opened the treasury, which had been made by some kings of the sons of David, [to whom be peace,] and he brought out thence a great sum of money, and left as much in it, consigning it to its former state of secrecy. 

18 Then he built up and repaired that part of the wall which had fallen down; and he provided carefully for the convenience and advantage of his flock, and behaved himself uprightly towards them. 

19 Now when Antiochus had come into his own country, he determined to go and fight with the king of Persia, for he had revolted from the time of the first Antiochus: 

20 and he sent ambassadors to Hyrcanus, that he should go to him; and Hyrcanus went with him, and departed into the country of Persia. 

21 And an army of the Persians met him, arid fought with him; whom Antiochus putting to flight defeated and put to the sword. 

22 Then he stayed in the place where he was, and erected a wonderful building, that it might be a memorial of him in their country. 

23 And after some time he went forward to meet the king of the Persians; and Hyrcanus remained behind, by reason of the sabbath, which Pentecost immediately followed. 

24 And the king of Persia and Antiochus met; and very great battles took place between them, in which Antiochus and many of his army were slain. 

25 And when news of this was brought to Hyricanus, he marched into the country of Syria, and on his journey besieged Halepus: 

26 and the citizens surrendered to him, paying him tribute; and he departed from them, and returned into the Holy City, and remained there for some days.

27 Then he departed into the country of Samaria, and fought against Neapolis; but the citizens hindered him from entering into it. 

28 And he destroyed whatever buildings they had on mount Jezabel, and the temple; which was done two hundred years after that Sanballat the Samaritan had built it. He also slew the priests who were in Sebaste. 

29 And he marched into the country of Idumeza, that is, the mountains Sarah, and they surrendered to him: with whom he stipulated that they should be circumcised and adopt the religion of Torah (or the Mosaic law). 

30 And they agreed with him, and were circumcised, and became Jews, and were confirmed in this practice even till the destruction of the second house.

31 And Hyreanus | went on to all the surrounding nations; and hey all submitted to him, and at the same time entered into an agreement of peace and obedience. 

32 He also sent ambassadors to the Romans, sahite ing to them concerning the renewal of the leaguewhich was between them. 

33 When therefore his ambassadors had come to the Romans, they honoured them; and appointed them a seat of dignity; and gave attention to the embassy on account of which they had come; and despatched their business, and replied to his letter. 

\chapter{22}

\textit{The copy of the Romans’ letter to Hyrcanus}

1 “From the elder, and his three hundred_and twenty governors, to Hyrcanus the king of Judah, health. 

2 Your letter has even now reached us, on reading which we rejoiced; and we have questioned your ambassadors concerning the state of your affairs. 

3 Also we have acknowledged their place of dignity in science, moral discipline, and the virtues; and we have honoured them, and made them sit in the presence of our elder: 

4 who has been careful to transact all their business, giving command that all the cities which Antiochus had taken away by force should be restored to you; 

5 and that every obstacle to the exercise of your religion should be removed; and that all should be made void which Antiochus had decreed against you. 

6 He has also commanded thatall the cities which he had taken should continue faithful for you; he has likewise given orders by letter to all his provinces, that your ambassadors should be treated with respect and honour. 

7 Moreover he has sent with them an ambassador to you named Cynzeus, bearing a letter; to whom also he has entrusted an embassy, that he might treat with you in person.” 

8 Therefore when this epistle of the Romans had reached Hyrcanus, he began to be styled king, being formerly called high priest: and thus the royal and sacerdotal dignities were united in him. 

9 And he was the first who was called king among the chiefs of the Jews in the time of the second house.


\chapter{23}

\par \textit{The history of the wars of Hyrcanus with the Samaritans}

1 Now Hyrcanus marched to Sebaste, and besieged the Samaritans who were therein, for a long time; till he reduced them to such straits, that they were compelled to feed upon every kind of dead carcass. 

2 Nevertheless they bore this patiently, fearing his sword, and trusting to the Macedonians and Egyptians, whose aid they had implored. 

3 In the mean time comes on the great fast, at which Hyrcanus must be present in the Holy House, to offer sacrifices on that day. 

4 Wherefore he substituted his two sons, Antigonus and Aristobulus, as commanders of the army; leaving them orders to besiege the Samaritans, and reduce them to extremities. 

5 Likewise he commanded the army to obey his sons, and to execute their or ders: and he departed to the city of the Holy House.

6 Moreover Antiochus the Macedonian marched to help the inhabitants of Sebaste; and tidings of it were brought to the two sons of Hyrcanus; 

7 who, having substituted a general to conduct the siege of Sebaste, went to meet Antiochus; whom they encountered and routed, and returned to Sebaste. 

8 There came likewise out of Egypt Lythras, the son of queen Cleopatra to help the Samaritans. 

9 When news of this was brought to Hyrcanus, he went to meet him, the solemnity being now past: whom when he met, he encountered most fiercely, and slew very many of his men:

10 and Lythras was put to flight; nor did the Egyptians any more after this return to give assistance to the Samaritans. 

11 And king Hyrcanys returned to Sebaste, and pressed sore on it, till he took it with the sword, and slew those of its citizens who were remaining, and utterly destroyed it, and pulled down its walls. 

\chapter{24}

\par \textit{The history of Lythras the son of Cleopatra, and of his marching out against his mother in Egypt.}

1 Lythras the son of Clegpatra, having become strong in goods and in men, revolted from Cleopatra his mother; the chief men of the kingdom being his abettors. 

2 Therefore Cleopatra, having sent for two Jews, one of whom was called Chelcias, and the other Hananias, placed them at the head of those princes of Egypt who remained on her side, and made them both generals of the Egyptian army. 

3 Now they managed all matters well with the common people, and conducted the affairs of the empire with wisdom. Them Cleopatra sent to fight with Lythras; 

4 who going to him nrade war, and routed him, putting his men to flight: and he fled to Cyprus, and there remained, with a few who adhered to him. 

\chapter{25}

\par \textit{An account of the Jewish sects at this time.}

1 At that time there were three sects among the Jews. One, of the Pharisees, that is, the “separated,” or religious; 

2 whose rule it was, to maintain whatever was contained in the law, according to the expositions of their forefathers. 

3 The second, that of the Sadducees; and these are followers of a certain man of the doctors, by name Sadoc; 

4 whose rule it was, to maintain according to the things found in the text of the law, and of which there is demonstration in the Scripture itself; but not that which is not extant in the text, nor is proved from it. 

5 The third sect was that of the Hasdanim, or those who studied the virtues: but the author of this book did not make mention of their rule, nor do we know at except im so far as it is discovered by their name: 

6 for they applied themselves to such practices as came near to the more eminent virtues; namely, to select from those two other rules whatever was most safe in belief, most sure and guarded. 

7 Hyrcanus at first was one of the Pharisees; afterwards he went over to the Sadducees; 

8 because that one of the Pharisees had said to him, it is not lawful for you to be high priest, because your mother was a captive before she bare you, in the days of Antiochus; but it becometh not that the son of a captive should be high priest.

9 And this conversation took place in the presence of the chief men of the Pharisees; which was the cause of his going over to the rule of the Sadducees. .

10 Now the Sadducees were at enmity with the Pharisees; wherefore they kept up differences betwixt one another, and they prevailed on him so far, as to slay great numbers of the Pharisees. 

11 And the trouble came to such a height, that wars and many evils continued among them for a great length of time. 

\chapter{26}

\par \textit{The account of Hyrcanus’ death, and of the time of his reign}

1 Hyrcanus had three sons, namely, Antigonus, Aristobulus, and Alexander. 

2 And Hyrcanus loved Antigonus and Aristobulus; but Alexander was odious to him. 

3 And on a time he saw in a dream, that of his sons, Alexander would reign after his death; and this gave him uneasiness. 

4 And he did not think fit, while he lived, to set up either of the sons whom he loved, on account of his vision; 

5 nor to appoint Alexander king, because he was disliked by him.. Wherefore he deferred the business; that after his death it might take that turn which should please the great and good God. 

6 Now the Jews had been, in the time of his father and uncles, united in affection towards them; and prompt to obey them, on account of their subduing of their enemies, and the excellent feats which they performed. 

7 They also continued united in affection to Hyrcanus; until the slaughter of the Pharisees was committed by him, and the rooting out of the Jews, and the civil wars on account of religion. 

8 From hence sprung perpetual enmities, and ceaseless evils, and many murders. Which was the reason why many detested Hyrcanus. 

9 Now the time of his reign was thirty-one? years, and he died. 

\chapter{27}

\par \textit{The history of Aristobulus the son of Hyrcanus}

1 HyrcaNnus being dead, his son Aristobulus succeeded him on the throne; who displayed haughtiness, pride, and power; and placed on his head a large crown, in contempt of the crown of the sacred priesthood. 

2 Now he was affectionately inclined towards his brother Antigonus, whom he preferred to all his friends: but his brother sess he kept in prison, as also his mother, by reason of her love for Alexander. 

3 And he sent his brother Antigonus, who fought against him, and conquered him, with all his abettors and troops, which he put to flight, and returned into the city of the Holy House: This happened while Aristobulus lay sick.” 

4 When therefore Antigonus was on his way to the city, the sickness of his brother was_reported to him; who, entering the city, went to the house of God, to give thanks for the mercy shewn in his deliverance from the enemy, and to beseech the great and good God to restore health to his brother. 

5 Therefore certain of those who were adversaries and haters of Antigonus go to.Aristobulus and say; 

6 In sooth the news of your sickness was carried to your brother, and behold he is coming with his partisans, armed; and is now gone into the sanctuary to make to himself friends, that he may come suddenly upon you and slay you. 

7 And king Aristobulus was afraid to take any hasty step against his brother respecting that which had been told him, till he should know the correctness of the intelligence. 

8 Wherefore he commanded all his attendants to post themselves armed in a certain place, from which whoever came to his palace could not turn aside. 

9 He likewise ordered it to be publicly proclaimed, that no one wearing arms of any kind should come to the king into the court, without being hidden. 

10 After this, he sent to Antigonus, ordering him to come to him: whereupon Antigonus took off his arms in obedience to the king. 

11 In the mean time there comes to him a messéiiger from the wife of his brother Aristobulus, (who hated him,) saying to him; 

12 The king says to you, “I have now heard “of the beauty of your dress when you entered the city, and am desirous of beholding you thus habited; wherefore come to me in that form, that I may be gratified in seeing you.” 

13 And Antigonus doubted not that this message was _ from the king, as the messenger had reported; 

14 and that he did not wish to put him on the same footing with others as to the laying aside their arms: and he went to him in that manner and dress. 

15 And when he had come to that place in which king Aristobulus had commanded his men to post themselves, with orders to kill any person who should come thither armed; 

16 and when the men saw him wearing his arms;—they rushed on him, and instantly slew him; and his blood flowed over the marble pavement on that spot. 

17 And the cry of men grew loud, and their weeping and lamentation was magnified, grieving over the death of Antigonus, for his beauty, and the elegance of his discourse, and his exploits. 

18 So the king, hearing the noise of the men, enquired concerning it; and found that Antigonus had been slain; 

19 which caused him the greatest sorrow, both for the affection which he bare towards him, and because he did not deserve this fate: and he perceived that a snare had been laid for his brother: 

20 and he cried aloud, and wept exceedingly; and smote his breast unceasingly; so that some blood-vessels of his breast were burst, and the blood flowed out of his mouth. 

21 But his attendants and the chief of his friends came to him, consoling him, and appeasing and soothing him, so as to restrain him from this action; 

22 being apprehensive that he would die, as he was weak, and was almost expiring under that which he had already done. 

23 And they took a golden basin, to receive the blood which gushed forth at his mouth; 

24 and they sent the basin, with the blood with: was in it, by one of the attendants to a physician, that he might see it, and advise what was to be done for him. 

25 And the page went with the basin: and when he came to the place where Antigonus had been slain, and his blood had flowed about, the page slipped, and fell; and spilled the king’s blood which was in the basin over the blood of his murdered brother. 

26 And the page returned 26 with the basin, and told the courtiers what had happened; who abused and reviled him; while he justified himself, and sware that he had not designedly or voluntarily done this. 

27 But when the king heard them quarrelling, he asked to. be told what they were saying: and they held their tongues: but when he threatened them, they told him. 

28 Who then said, “Praise be to the Just Judge, who hath shed the blood of the oppressor over the blood of the oppressed.” 

29 Then he groaned, and forthwith expired. And the time of his reign was one full year. 

30 And all his flock lamented him; for he was noble-minded, victorious, and liberal: and his brother Alexander we reigned in his stead. 

\chapter{28}

\par \textit{The account of Alexander the son of Hyrcanus}

1 AFTER that Aristobulus was dead, his brother Alexander was released from his fetters; and being brought out of prison, succeeded to the throne. 

2 Now the governor of the city Acche (which is Ptolemais) had rebelled; and had sent messengers to Lythras the son of Cleopatra, requesting that he would aid him, and take him under his protection; 

3 but he for a long time refused, fearing a recurrence of the things which he had before suffered from Hyrcanus. 

4 But the messenger gave him courage by means of the succours promised by the lord of Tyre, of Sidon, and others. And Lythras marched with thirty thousand men: 

5 and the report of it was brought to Alexander, who anticipated him at Ptolemais, and attacked it; and the citizens of Ptolemais shut the gate in his face, and endeavoured to keep him out. 

6 Wherefore Alexander straitened them, and continued to besiege them; until he was informed of the marching of Lythras: then he retired from before them, Lythras with his troops being at hand. 

7 Now there was among the citizens of Ptolemais an old man‘ of acknowledged authority, who persuaded the citizens not to permit Lythras to enter their city, nor to take on themselves obedience to him, since he was of a different religion. 

8 He also said to them, Far more advantageous to 8 you in every way will be submission to Alexander, who is of the same religion, than submission to Lythras: nor did he cease, until they agreed to his sentiments. 

9 And they prevented Lythras from entering Ptolemais, refusing submission to him. And Lythras was perplexed in his affairs, nor did he take counsel what was best for him to do. 

10 And this was told to the king of Sidon’, and he sent messengers to him, that he should help him in the war against Alexander; that either they might defeat him, or fake some of his cities, and thus punish him; 

11 and thus Lythras might return into his own country, after performing deeds which might render him formidable; which in truth would be more to his advantage than to return without having effected his purpose. 

12 And this was told to Alexander; who sent to Lythras an honourable embassy with a very valuable present, and proposed to him not to aid the king of Sidon. 

13 And Lythras accepted Alexander’s present, agreeing to his request. 

14 But Alexander marched to Sidon, and fought against its sovereign; and God made him victorious over him, and he slew great numbers of his men; and having put him to flight, gained possession of his country. 

15 After this, Alexander sent messengers to Cleopatra, that she should come with an army pc. against Lythras her son; and that he also would march with his army against him, and would deliver him a prisoner to her. 

16 Which when Lythras found out, he departed into the mountain of Galileef, and slew great numbers of the inhabitants, and carried away ten thousand captives: a great number of his own men also were slain. 

17 From thence he marched even till he came to Jordan, and there encamped; that his men and horses might rest themselves, and afterwards he might march to Jerusalem to fight with Alexander. 

18 This was told to Alexander; who went against him with fifty thousand men, of whom six thousand had shields of brass: and it is said that each of those could resist any number of men. 

19 And he attacked him at the Jordan, and engaged with him there; but did not obtain the victory, because he trusted in his men, and had placed his confidence in their number. 

20 But with Lythras there were men very skilful in battles and in drawing up armies; who advised him to divide his forces into two parts, so that one might be with Lythras and his company prepared for battle, and the other part might be with another captain of their company. 

21 And he fought even until noon, and great numbers of his men were slain. 

22 And his friend advanced, with the remainder of the army which was with him, whose strength was yet entire, against Alexander and his men, who were by this time overcome with fatigue: 

23 and he dealt with them as he pleased, and slew great multitudes of them; and Alexander and the men who had remained with him fled into the city of the Holy House. 

24 Lythras also departed towards evening into a 24 certain town near at hand; and by chance some Jewish women with their children met him;

25 and he commanded some of the children to be killed, and their flesh to be dressed, pretending that there were some in his army who fed on human flesh; designing by these acts to strike the inhabitants of the country with a dread of his troops. 

26 After this came Cleopatra; whom Alexander met, and told her what Lythras had done to his army, and appointed to go with her in search of him. 

27 Which being told to Lythras, he fled to a place where was a station of his ships; going on board which, he returned to Cyprus; and Cleopatra returned into Egypt.

28 But at the end of the year Alexander marched against Gaza; because its chief had revolted from him, and had sent to a certain king of the Arabians named Hartasi to assist him; who consented to do so, and marched towards Gaza: 

29 this was told to Alexander; who leaving some of his men before Gaza, marched against Hartas, and engaged him, and put him to flight. 

30 Then he returned to Gaza, and lying sore upon it, took it at the end of a year. 

31 But the cause of his taking it was the brother of that chief; who coming suddenly on him, slew him. 

32 When the citizens sought to kill him, he collected his friends, and went to the gate of the city, and addressed Alexander, begging that on giving security for his life and the lives of his friends, he would enter the city; 

33 which Alexander promising, entered Gaza, and slew its inhabitants, and overthrew the temple which was in it, and burned the gilded idol which was in the temple. 

34 After which he departed to the city of the Holy House, and there celebrated the feast of tabernacles. 

35 And when the feast was past, he made himself ready against Hartas, whom he encountered, and slew a great number of his men: 

36 and Hartas’ affairs were much straitened and crippled, and he feared his own utter extinction. Wherefore suing to Alexander for his life, he yielded him obedience, and paid him tributes. 

37 And Alexander departed from him, and marched against Hemath and Tyre, and took them; and having received tribute from the inhabitants, he returned into the city of the Holy House. 

\chapter{29}

\par \textit{An account of the battles which took place between the Pharisees and Sadducees.}

1 Afterwards evils arose between the Pharisees and Sadducees, and continued by the space of six years. 

2 And Alexander helped the Sadducees against the Pharisees, of whom there were slain within six years fifty thousand. 

3 Wherefore between these two sects the state of 3 things was reduced to utter destruction, and their enmity was completely confirmed. 

4 So Alexander, having sent for the elder men of each sect, spake kindly to them, and advised a reconciliation. 

5 But they answered him, “In truth you, in our opinion, are worthy of death, for the abundance of innocent blood which you have shed: wherefore let there be nothing between us but the sword.” 

6 Then after this, they began to shew their enmity openly, sending messengers to Demetrius the king of Macedon, that he should come to them with an army; 

7 promising that they would assist him against Alexander and his party, and would reduce the Hebrews to submission to the Macedonians. And Demetrius marched to them with a large army.

8 Which also was told to Alexander; who sent a person to hire six thousand Macedonians, whom joining to his own forces he advanced against Demetrius. 

9 Many also of the Jews, Pharisees, went over to Demetrius. 

10 And Demetrius sent secretly persons to those Macedonians who were with Alexander, to seduce them from him; but they hearkened not unto him. 

11 Alexander also sent secretly men to the Jews who were with Demetrius, to turn them to his side; but neither did these do as he would have them. 

12 And Alexander and Demetrius met, and fought a battle; in which all Alexander’s men fell, and he escaped alone into the land of Judah. 

13 But when his men heard it whispered that he had escaped in safety, and found out the place where he was; 

14 there assembled unto him about six thousand men of the bravest of the sons of Israel; and many of those, who had revolted to Demetrius, joined themselves to him. 

15 Afterwards men flocked to him from every side; and he returned to give battle to Demetrius with a numerous force, and put him to flight: and Demetrius returned into his own country. 

16 And Alexander marched against him to Antioch, and besieged it three years: and po when Demetrius came out to fight, Alexander conquered him and slew him: 

17 and he departed from the city, and returned to Jerusalem to his citizens; who magnified him, honouring and praising him for having defeated his enemies. 

18 And the Jews agreed to submit to him, and his heart was at rest: and he sent his armies against all his enemies, whom he put to flight, and gained the victory over them. 

19 He also gained possession of the mountains of Sarah, and the country of Ammon, and Moab, and the country of the Philistines, and all the parts which were in the hands of the Arabians who fought with him, even to the bounds of the desert. 

20 And the affairs of his kingdom were ruled aright; and he placed his people and his country in a state of safety. 

\chapter{30}

\par \textit{The account of the death of Alexander the son of Hyrcanus}

1 Afterwards king Alexander fell sick with a quartan fever, for three whole years. 

2 But when the governor of a city named Ragaba revolted from him, he led thither a powerful army, taking with him his wife and family, and besieged the city. 

3 But when it was on the point of being taken, his disease increased and his strength declined; and his wife, who was named Alexandra, lost all hope of his recovery: 

4 who going up to him said; “You know now what differences there are between you and the Pharisees: and your two sons are little boys, and I am a woman, and altogether we shall not be able to resist them: what advice therefore do you give to me and them?” 

5 He said to her, “My advice is, that you persevere against the city till it be taken, which will be shortly. 

6 And when it shall have been won, establish its government according as the other cities have been established. 

7 But towards all these people, pretend that I am sick; and whatever you do, pretend that you do it at my suggestion; and reveal my death to those servants on whom you can depend. 

8 And when you shall have finished these matters, go into the city of the Holy House, having previously dried and embalmed my body with spices; and fill the place where I lie with many perfumes, that no unpleasant smell may proceed from me. 

9 And when the affairs of the country are settled, go thence, and roll me up in abundance of perfumes, and carry me into the palace, as if sick: 

10 and when I am there, send for the principal men of the Pharisees; and when they come, honour them, and speak good words to them: 

11 then say, Alexander is already dead, and behold I give him up to you, do with him whatever seems good to you: and from henceforth will behave to you as you shall please. 

12 For if you do this, I know very well that they will do nothing to me and you, except that which is good; and the people will follow them, and your affairs will be ordered aright after my death, and you will reign securely until your two sons be grown up.” 

13 After this, Alexander died; and his wife concealed his death; and when the city was taken, she returned to Jerusalem; and having sent for the chief men of the Pharisees, she addressed them as Alexander had advised her. 

14 To whom they replied, that Alexander had been their king, and they had been his people; and they spoke to her with all affection, and promised to place her at the head of their government.

15 Then they went out and collected men; and taking Alexander’s body, they carried it forth magnificently to its burial: and they sent for men to appoint Alexandra queen; with whose concurrence she was so appointed. 

16 And the years of Alexander’s reign were twenty-seven. 

\chapter{31}

\par \textit{The history of queen Alexandra}

1 Now while Alexandra reigned, she called to her the chief men of the Pharisees, and commanded them to write to all those of their sect who had fled into Egypt and other parts, in the days of Hyrcanus and of Alexander, that they should return into the land of Judah. 

2 And she shewed them her favourable inclination towards them, and did not oppose herself to their rites, nor forbid their ceremonies, as Alexander and Hyrcanus had forbidden them. 

3 She also released all of them who were detained in prison. 

4 And they came together from every quarter; and the Sadducees forbore offering them any violence. 

5 And their affairs were well ordered, and their condition became improved by the removal of quarrels. 

6 But when Hyrcanus and Aristobulus the two sons of Alexander grew up, the queen made Hyrcanus high priest, for he was meek, mild, and honest: 

7 but Aristobulus she made general of the army, for he was stout, brave, and high-spirited; and she also gave to him the army of the Sadducees: but she did not think it meet to appoint him king, as he was still a boy. 

8 Moreover she sent to all those who paid tribute to Alexander, and took their kings’ sons, whom she detained near her as hostages; and they continued uninterruptedly in their obedience to her, paying tribute every year. 

9 And she walked uprightly with her people, distributing justice, and commanding her people to do the same. Wherefore there was a lasting peace between the parties, and she gained their good-will. 

\chapter{32}

\par \textit{An account of the things which were done to the Sadducees by the Pharisees in the time of Alexandra.}

1 There was among the Sadducees a chief man, who had been promoted by Alexander, named Diogenes, who formerly had induced him to slay eight hundred men of the Pharisees. 

2 Therefore the leaders of the Pharisees come to Alexandra, and remind her of what Diogenes had done, asking her leave to slay him; which she gave: and they, having it, slew many Sadducees together with him. 

3 Which the Sadducees taking very much to heart, went to Aristobulus; and, taking him with them, went to the queen, and said to her: 

4 “You are aware what terrible and heavy things we have undergone, and the many wars and battles which we have fought, in aid of Alexander and his father Hyrcanus. 

5 Wherefore it was not meet to trample on our rights, and to lift up the hand of our enemies over us, and to lower our dignities; 

6 for a matter of this kind will not be hidden from Hartas and others of your enemies; who have experienced our bravery, and have not been able to resist us, and their hearts have been filled with the fear of us. 

7 When therefore they shall perceive what you have done to us, they will imagine that our hearts are devising plans against you; of which when they shall be certified, trust that they will play false towards you. 

8 Nor will we endure to be killed by the Pharisees, like sheep. 

9 Therefore, either restrain their malice from us, or allow us to go out from the city into some of the towns of Judah.” 

10 And she said to them, “Do this, that their annoyance to you may be prevented.” 

11 And the Sadducees went forth of the city; and their chiefs departed with the men of war who adhered to them; and went with their cattle to those of the towns of Judah which they had selected, and dwelt in them; 

12 and there were joined to them those who were devoted to virtue, (i.e. the Hasdanim.) 

\chapter{33}

\par \textit{The account of the death of Alexandra}

1 After these things, Alexandra fell into a disease, of which she died. 

2 And when her recovery was almost despaired of, her son Aristobulus went out from Jerusalem by night, attended by his servant: 

3 and he departed to Gabatha?, to a certain chief man among the Sadducees, one of his friends; 

4 and taking him with him, he proceeded to the cities where the Sadducees dwelt; and opened to them his purpose, and exhorted them to go out with him, and to be his allies in war against his brother and the Pharisees, and to appoint him king. 

5 To whom they assenting, openly played false with Alexandra, collecting men from feet quarter to join Aristobulus. 

6 When the fame of these things reached Hyrcanus the son of Alexandra, the high priest, and the elders of the Pharisees, they went to Alexandra, sick as she was, and related the matter to her; 

7 pressing on her the great fear which they had for her and her son Hyrcanus, from Aristobulus and those who were with him. 

8 To whom she answered; “I truly am near death, so that it is more proper and profitable for me to attend to my own affairs; what therefore can I do, being situated thus? 

9 But my men, and my goods, and my arms, are with you and in your hands; therefore order the business as it seemeth to you right, imploring the aid of God upon your matters, and asking deliverance from Him.” Then she died. 

10 The amount of her age was seventy-three years; and the time of her reign nine years,

\chapter{34}

\par \textit{The account of Aristobulus’ attack on his brother Hyrcanus, after Alexandra’s death.}

1 When Aristobulus departed from Jerusalem in the days of Alexandra, he left his wife and children in Jerusalem. 

2 But when the news of his departure reached Alexandra, she confined them in a certain house, setting a guard over them. 

3 But when Alexandra was dead, Hyrcanus called them to him, and behaved kindly to them, and took care of them; that they might deliver him from his brother, if haply he should conquer him. 

4 Then Aristobulus led out a great army as far as to Jordan; and Hyrcanus went out against him with an army of Pharisees. 

5 And when the two armies had encountered, great numbers of Hyrcanus’ army were slain; and Hyrcanus, and the remainder of his army, took to flight. 

6 Whom Aristobulus and his troops pursuing, slew every one whom they caught, excepting those who surrendered themselves. 

7 Then Hyrcanus retreated into the Holy City; whither also arrived Aristobulus and his army; and he surrounded it on every side with his tents, and attempted by stratagem to destroy the fortification. 

8 And the elders of Judah, and the elders of the priests, went out to him, and forbade his doing what he had designed; requesting him to dismiss from his mind whatever hostile feeling he had towards his brother: to which proposal he assented. 

9 Then it was agreed between them that Aristobulus should be king over Judah, and Hyrcanus should be high priest” in the house of God, and next to the king dignity. 

10 And Aristobulus assented to these terms, and entered the city, and had an interview with his brother in the house of God; and they took an oath together to ratify those terms which the elders had mutually agreed on. 

11 So Aristobulus was made king, and Hyrcanus was ranked next unto him. 

12 And men were at peace, and the affairs of these two brothers were rightly ordered, and the state of their people and of their country became one of tranquillity. 

\chapter{35}

\par \textit{The account of Antipater, (that is, Herod the king,) and of the seditions and battles which he kindled between Hyrcanus and Aristobulus.}

1 There was a man of the Jews, of the sons of certain of those who went up out of Babylon with Ezra the priest, xamed Antipater. 

2 And he was wise, prudent, acute, brave, and high-minded, of a good disposition, kind, and courteous; also rich, and possessing many houses, goods, and flocks. 

3 This man king Alexander had made governor of the country of the Idumzans, from whence he had taken a wife; by whom he had four sons, namely, Phaselus, Herod, who reigned over Judah, Pheroras, and Josephus. 

4 Afterwards, being removed from the mountains of Sarah‘, that is, the country of the Idumezans, in the days of Alexander, he dwelt in the city of the Holy House: 

5 and Hyrcanus loved him, and was much inclined towards him: wherefore Aristobulus sought to kill him; which, however, he did not accomplish. 

6 So Antipater was excessively afraid of Aristo6 bulus, and for that reason began secretly to plot against Aristobulus’ kingdom. 

7 He went therefore to the principal men of the kingdom, and having gotten from them a pledge of secrecy respecting the matters which he was about to communicate, 

8 he began to talk to them of the infamous life of Aristobulus, his tyranny, his impiety, and the bloodshed which he had caused, and his usurpation of the throne; of which his elder brother was more worthy. 

9 Then he bade them beware of the great and good God, unless they took away the tyrant’s ruling hand, and restored what was due to their rightful sovereign. 

10 Nor was there left a single one of the chief men, whom he did not overreach, and incline to submit to Hyrcanus, seducing them from their obedience to Aristobulus, Hyrcanus knowing nothing of the matter: 

11 but Antipater ascribed? all this to him, being unwilling to tell him before he had established the thing. 

12 Therefore, when he had fully settled this business with the people, he went to Hyrcanus, and said to him; 

13 Truly your brother is greatly afraid of you, because he sees that his estate will be-nowise secure while you are alive; on which account he is seeking about for an opportunity to slay you, and will not suffer you to live. 

14 But Hyrcanus did not give credence to him, because of the goodness and sincerity of his heart. Wherefore Antipater repeated this discourse to him again and again. 

15 Also he gave large sums of money to the persons in whom Hyrcanus placed confidence, and agreed with them that they should tell him similar things to what Antipater had mentioned; 

16 only taking care that he should not imagine that they knew that Antipater had been speaking to him on the subject. 

17 So Hyrcanus believed their words; and was induced to devise a plan by which he might be delivered from his brother. 

18 When therefore Antipater spoke again to him of the matter, he informed him that the gc. truth of his words was now manifest to a him, and that he knew that he had advised him well; and he asked his counsel in this affair. 

19 And Antipater advised him to go out of the city to some one in whom he could confide, and who might be able to aid and assist him. 

20 And Antipater went to Hartam, and agreed with him that he should receive Hyrcanus as a guest when he came, since he was rather afraid of dwelling with his brother. 

21 At which Hartam rejoiced, and came into the plan, and agreed with Antipater that in no case would he deliver up Hyrcanus and Antipater to their enemies, and that he would assist and protect them. 

22 And he returned to Jerusalem, and made known to Hyrcanus what he had done, and how he had agreed with Hartam concerning their going to him. 

23 Wherefore both of them went out of the city by night, and went to Hartam, and remained with him for some time. 

24 Then Antipater began to persuade Hartam to lead forth an army with Hyrcanus, to reduce and capture his brother Aristobulus. 

25 But Hartam declined prosecuting this plan, fearing that he had not strength to resist Aristobulus. 

26 But Antipater ceased not to shew him that the business with Aristobulus was easy, and to urge him to it by arguments of the treasure to be gained, and by the greatness of glory which he would acquire, and the memory which he would leave behind him: 

27 until he consented to march; yet upon condition that Hyrcanus would restore to him whatever cities and towns‘ belonging to him his father Alexander had taken away. 

28 To which Hyrcanus agreeing and completing the treaty, Hartam marched (and Hyrcanus with him) with fifty thousand horse and foot soldiers, bending his course to the country of Judah: against whom Aristobulus went forth and engaged them. 

29 And when the fight had become fierce, many of Aristobulus’ army went over to Hyrcanus. 

30 Which Aristobulus perceiving, sounded a retreat, and returned to his camp, fearing lest his whole army should gradually slip away fo the enemy, and thus he himself should be taken prisoner. 

31 But when night was coming on, Aristobulus departed from the camp alone, and went to the Holy City. 

32 And when on break of day his departure became known to the army, the greater part of them joined themselves to Hyrcanus, and the rest dispersed and went their ways. 

33 But Hyrcanus, Hartam, and Antipater, went straight to the city of the Holy House, carrying with them a large army; 

34 and they found Aristobulus already prepared for a siege; for he had closed the gates of the city, and had placed men on the ramparts to defend them. 

35 And Hyrcanus and Hartam encamped with their forces against the city, and besieged it. 


\chapter{36}

\par \textit{The history of Gneus, general of the army of the Romans.}

Now it happened, that Gneus, general of the army of the Romans, went forth to fight with Tyrcanes the Armenian: 

2 for the citizens of Damascus, and Hames and Halepum, and the rest of them of Syria who are belonging to the Armenians, had lately rebelled against the Romans: 

3 and on that account Gneus had sent Scaurus to Damascus and to its territories, to take possession of them; which thing was told to Aristobulus and Hyrcanus. 

4 Therefore Aristobulus sent ambassadors to Scaurus, and much money, requesting him to come to him with an army, and assist him against Hyrcanus. 

5 Hyrcanus also sent ambassadors to him, requesting his aid against Aristobulus; but he did not send him a present. 

But Scaurus refused to go to either of them: but he wrote to Hartam, ordering him to retire with his army from the city of the Holy House, and forbade him to give help to Hyrcanus against his brother; 

7 and threatened that he would come into his country with an army of Romans and Syrians, unless he obeyed. 

8 Now when this letter had reached Hartam, he immediately retired from the city: 

9 Hyrcanus also retreated; whom Aristobulus pursued with a certain number of his troops, and overtook them, and engaged them‘: and a great number of the Arabians were slain in that battle, and very many of the Jews: and Aristobulus returned into the Holy City.

10 In the mean time, Gneus reached Damascus; to whom Aristobulus sent, by the hand of a man named Nicomedes, a garden and vineyard: of gold, altogether weighing five hundred talents, with a most rich present; and besought him to assist him against Hyrcanus. 

11 Hyrcanus also sent Antipater to Pompey, with the like request. 

12 And Pompey (who is Gneus) was inclined to help Aristobulus. 

13 Which when Antipater saw, he watched an opportunity that he might speak with Pompey alone, and said to him: 

14 “In truth, that present which you have received from Aristobulus needs not be restored to him, even though you should not assist him; 

15 yet Hyrcanus offers you twice so much: and Aristobulus will not be able to bring the Jews into subjection to you, but this Hyrcanus will do.” 

16 And Pompey supposed the matter to be so as Antipater had said; and rejoiced to think that he could bring the Jews under his dominion. 

17 Wherefore he said to Antipater, will assist your friend against Aristobulus; although I may pretend to help him against you, that he may entrust himself to me. 

18 For I am sure, that as soon as he shall find out that I am giving aid to his brother against him, he will play false with all his men, and will take care of himself, and his business will be much longer delayed. 

19 But I will send for him, and will go with him into the Holy City, and then will so act that your friend shall obtain his right; but with this condition, that he shall pay us an annual tribute.” 

20 THE MESSENGER OF ARISTOBULUS. After this, having sent for Nicomedes, he said to him; “Go to your master, and tell him, that I have consented to his request; and carry him my letter, and say to him, that he must come to me in haste without delay, for I am waiting for him.” 

21 And he wrote a letter to Aristobulus, of which this is a copy: 

22 “From Gneus, general of the army of the Romans, to king Aristobulus, heir to the throne and high-priesthood, health be to you. 

23 Your garden and vine of gold have arrived; and I have received them, and have sent them to the “elder and governors; which they have accepted “and have placed in the temple! at Rome, return“ing you thanks.

24 They have written, moreover, that I should assist you, and appoint you king over the Jews. 

25 If therefore you think fit to “come to me with all speed, that I may go up with you to the Holy City, and fulfil your wishes, I will do so.” 

26 And Nicomedes departed to Aristobulus with the letter of Gneus. And Antipater, returning to Hyrcanus, told him of the promise of Gneus, advising him to go to Damascus. 

27 So Hyrcanus went to Damascus: Aristobulus went also: and they met at Damascus in the audience-room of Pompey, (that is, Gneus;) and Antipater and the elders of the Jews said to Gneus; 

28 “Know, most “illustrious general, that this. Aristobulus has been dealing falsely towards us, and has usurped by the sword the kingdom of his brother Hyrcanus, who is more worthy of it than he, seeing that he is the elder brother, and of a better and more correct way of life. 

29 And it was not enough for him to oppress his brother, but he has oppressed all the nations which are round about us; shedding their blood and pillaging their goods unjustly, and keeping up enmities between us and them, a thing which we abhor.” 

30 Then stood up a thousand aged men, attesting the truth of his words. 

31 And Aristobulus said, “Truly this my brother is a better man than I; but I did not seek for the throne, until I saw that all those who had been subject to our father Alexander were dealing falsely with us after his death, knowing the inability of my brother. 

32 Which when I looked into, I perceived that it was my duty to undertake the sovereignty, in that I was better than he in matters of war, and by that was better suited for preserving the monarchy: 

33 and I went to war with all those who dealt falsely with us, and reduced them to obedience: and this was the command of our father before his death.” 

34 And he brought forward witnesses who attested the truth of his words. 

35 After these things Pompey departed from the city Damascus, journeying to the Holy House. 

36 But Antipater sent privately to the inhabitants of the cities which Aristobulus had won, exciting them to complain to Gneus, setting forth the tyranny which he had exercised over them; which thing they did. 

37 And Gneus ordered him to write them a testimonial of their freedom, and to say that he would in no wise trouble them more; which truly he did, and the nations were released from their obedience to the Jews. 

38 But when Aristobulus saw what Gneus had done to him, he and his men departed by night from Gneus’ army without acquainting him with it, and went on to the city of the Holy House: 

39 and Gneus followed him till he came to the city of the Holy House, around which he encamped. 

40 But when he beheld the height of the walls, and the strength of its buildings, and the multitude of men who were in it, and the mountains which encircled it, he perceived that flattery and cunning would be more serviceable against Aristobulus than acts of provocation: 

41 wherefore he sent ambassadors to him, that he should come out to him, promising him safe conduct: and Aristobulus went out to him; whom Gneus received kindly, not saying a word about his former doings. After this Aristobulus said to Gneus, 

42 “I wish that you would aid me against my brother, giving my enemies no power over me; and for this you shall have whatsoever you wish.” 

43 Gneus replied, “If you wish this, bring to me whatever money and precious stones are in the “temple, and I will put you in possession of what you wish.” And Aristobulus said to him, 

44 “Undoubtedly this I will do.” And Gneus sent a captain named Gabinius with a great number of men, to receive whatever of gold and jewels there was in the temple. 

45 But the citizens and the priests refused to permit this: wherefore they resisted Gabinius, killing many of his men and of his friends, and drove him out of the city. 

46 Upon which, Gneus, being wrath with Aristobulus, threw him into prison. ”

47 Then he marched with his army, to force his way into the city and enter it. But a great body of the citizens going forth, hindered him from doing this, by slaying great numbers of his men. 

48 And in truth, the numbers, the spirit, and the bravery of the nation, which he had seen, frightened him; so that, being alarmed at these, he had resolved to retire from them, had not mischievous quarrels arisen in the city between the friends of Aristobulus and the friends of Hyrcanus. 

49 For some of them wished to open the gates to Pompey, but others were averse to this. Wherefore they came to blows on this account; and as this state of things increased rather than diminished, the war continued. 

50 Which Pompey noticing, beset with his army the gate of the city: and as some of the people opened a wicket to him, he entered, and took possession of the king’s palace; but could not gain the temple, because the priests had closed the doors, and had secured the approaches by armed men. 

51 Against these he sent men to attack them from every side, and they put them to flight. And his friends coming to the temple, mounted the wall and descended into it, and opened its gates, after slaying a multitude of priests. 

52 Then Gneus came, and entered into it, and greatly admired its beauty and magnificence which he beheld, and was astonished when he saw its riches and the precious stones which were in it: 

53 and he forebore to take any thing out of it; and he commanded the priests to cleanse the house from the slain, and to offer sacrifices according to the ceremonies of their country. 

\chapter{37}

\par \textit{The account of the appointment of Hyrcanus the son of Alexander to be king of the Jews, and of the return to Rome of the general of the Roman army.}

1 Having arranged these matters, Pompey appointed Hyrcanus to be king; and carried away his brother Aristobulus in chains:

2 he also ordered that the Jews should have no dominion over those nations who had been subdued by their kings before his arrival; 

3 and he exacted a tribute from the city of the Holy House; and covenanted with Hyrcanus, that he should receive inauguration from the Romans every year. 

4 And he departed, taking with him Aristobulus, and two of his sons, and his daughters: and he had a son remaining, named Alexander, whom Pompey could not seize, because he had fled. 

5 So Pompey placed in hisroom in the city of the Holy House, Hyrcanus, and Antipater, and with them his own colleague Scaurus.

\chapter{38}

\par \textit{The history of Alexander the son of Aristobulus}

1 When Pompey had set out for Rome, Hyrcanus and Antipater marched against the Arabians, to bring them under the dominion of the Romans. 

2 To which the Arabians submitted, trusting to their intimacy with Antipater, and paying great regard to his advice; by which acts Antipater designed to reconcile the Romans to him. 

3 Therefore when Alexander the son of Aristoulus perceived the expedition of Hyrcanus, Antipater, and Scaurus, against the Arabians, and that they had departed to a great distance from the Holy City; 

4 he journeyed till he arrived there; and entering into the palace, he brought out thence money for the expence of repairing the city-wall which Pompey had broken down. 

5 And he raised for himself an army, and arranged all those matters which he wished, before Hyrcanus and his party should return to the city of the Holy House: and when they returned, 

6 he went out to meet them, and engaged them, and put them to flight. 


\chapter{39}

\par \textit{The history of Gabinius and of Alexander the son of Aristobulus.}

1 Now Gabinius had gone out from Rome, to dwell in the land of Syria, to take care of it; 

2 and it was told him what Alexander the son of Aristobulus had done, by building up that which Pompey had pulled down, and by opposing his successor, and slaying his friends. 

3 Wherefore he went straight until he came to Jerusalem; and Hyrcanus and his party joined him. 

4 Against whom Alexander went out with ten thousand foot and fifteen hundred horse, and encountered them: 

5 and they routed him, and slew a certain number of his friends; and he fled into a certain city in the land of Judah, called Alexandrium , in which he fortified himself with his 6 company. 

6 And Hyrcanus, and Gabinius, and their forces, marched against him and_ besieged him. 

7 And Alexander went out against them, and engaged them, and slew great numbers of their men. 

8 And Marcus, who is called Antonius, marched against him, and forced him to flee again into Alexandrium.

9 And Alexander’s mother went out to Gabinius, deprecating his anger, and imploring him to grant her son Alexander his life: 

10 to whom Gabinius assented in this point; and Alexander went out to him; and Gabinius put him to death; and thought proper to divide the territories of Judah into five portions. 

11 One is, the country of Jerusalem and the parts adjacent; and over this part Hyrcanus was made governor. Another portion is Gadira, and the places about it. 

12 The third is, Jericho and the plains. The fourth is, Hamath in the land of Judah. And the fifth is, Sephoris. 

13 By these means he intended to remove wars and seditions out of the land of Judah; but they were by no means removed. 

\chapter{40}

\par \textit2{The history of the fight of Aristobulus and his son Antigonus from Rome, and their return into the land of Judah: also, an account of the death of Aristobulus}

1 Then Aristobulus devised plans, till he had succeeded in escaping from Rome with his son Antigonus, and had arrived in the city of Judah. 

2 And when Aristobulus shewed himself in public, a great multitude of men flocked round him; out of whom he selected eight thousand, and marched against Gabinius, and engaged him; and there were slain of the Roman army a very great number: 

3 there fell also of his own men seven thousand, but one thousand escaped; and the enemy’s army pursued him; but he and they who were left to him ceased not to resist even till the total destruction of his men; 

4 nor was there one left but he alone; and he fought most furiously until he fell overpowered by wounds, and was taken and led to Gabinius; who ordered him to be taken care of until he was healed. 

5 Then he sent him in chains to Rome. 

\par [And he remained shut up in prison until the reign of Caesar; who brought him out of prison, and loaded him with gifts and favours; 

6 and giving to him two generals and twelve thousand men, sent him into the land of Judah, [B.C. 49.] to detach the Jews from Pompey’s party, and bring them over to obey Cesar: for Pompey at that time was governor of the land of Egypt. 

7 And the report of Aristobulus and his party reached Hyrcanus; who was greatly afraid, and wrote to Antipater to avert his power from him by his customary devices. 

8 So Antipater sent some of the chief men of Jerusalem, giving to one of them poison, charging him to administer it craftily to Aristobulus. 

9 And they met him in the land of Syria, as though they were ambassadors to him from the Holy City: and he received them joyfully, and they did eat and drink with him. 

10 And those men laid plots till they gave him the poison; and he died, and was buried in the land of Syria. 

11 Now the time of his reign’, until he was taken prisoner the first time, was three years and a half; and he was a man of courage, weight, and excellent disposition. ] 

12 Now Gabinius had written to the senate, to send away his two sons to their mother, since she had requested it; which they did. 

13 But it came to pass, that when Pompey had departed to a great distance from Jerusalem, they broke their engagement of obedience to the Romans: 

14 wherefore Gabinius went against them, encountered them, and conquered them, and reduced them again to submission to the Romans. 

15 In the mean time the land of Egypt rebelled against Ptolemy, and expelled him from his royal city, refusing to pay tribute to the Romans. 

16 Whereupon Ptolemy wrote to Gabi16 nius that he should come and help him against the Egyptians, that he might bring them again into subjection to the Romans. 

17 And Gabinius marched out of the country of Syria, and wrote to Hyrcanus to meet him with an army, that they might go to Ptolemy. 

18 And Antipater went with a large army to Gabinius, and met him at Damascus, congratulating with him on the victory which he had gained over the Persians: 

19 and Gabinius ordered him to hasten to Ptolemy, which he did, and fought against the Egyptians, and slew of them a very great number. 

20 Afterwards Gabinius com20 ing up, replaced Ptolemy on his throne, and went back to the Holy City, and renewed Hyrcanus’ sovereignty, and returned to Rome. 


\chapter{41}

\par \text{The history of Crassus}

1 When Gabinius had returned to Rome, the Persians played false to the Romans; 

2 and Crassus marched with a large army into Syria, and came to Jerusalem, requiring of the priests that they should deliver to him whatever money there was in the house of God. 

3 To whom they made answer, how will this be lawful for you, when Pompey, Gabinius, and others have deemed it unlawful? But he answered, I must do it at all events. 

4 And Eleazar the priest said to him, Swear to me that you will not lay your hand upon any thing which belongs to it, and I 5 will give you three hundred mine of gold. 

5 And he sware to him that he would take nothing from the treasure of the house of God, if he would deliver to him what he had mentioned. 

6 And Eleazar gave him a bar of wrought gold, the upper part of which had been inserted into the wall of the treasury of the temple, upon which were placed every year the old veils of the house, new ones being substituted for them. 

7 And the bar weighed three hundred minae of gold, and it was covered with the veils which were accumulated during a long course of years, being known to no one besides Eleazar. 

8 Crassus then, having received this bar, broke his word, going back from the agreement made with Eleazar; and he took all the treasures of the temple, and plundered whatever money was therein, to the amount of two thousand talents: 

9 for this money had been accumulating from the building of the temple until that time, out of the spoils of the kings of Judah and their offerings, and also from the presents which the kings of the Gentiles had sent; 

10 and they were multiplied and increased in the lapse of years; all which he took. 

11 Then that vile Crassus went off with the money and his army into the country of the Persians; and they defeated him and his army in battle, slaying them in a single day: 

12 and the Persian army took as spoil every thing which was in the camp of Crassus. 

13 After this feat, they marched into the country of Syria, which they won, and detached from its submission to the Romans. 

14 Which the Romans learning, sent a renowned general named Cassius with a great army: who, arriving in the country of Syria, drove out those of the Persians who were in it. 

15 Then proceeding to the Holy City, he delivered Hyrcanus from the war which the Jews were waging against him, reconciling the parties. 

16 Afterwards, passing the Euphrates», he fought with the Persians, and brought them back to their subjection to the Romans: 

17 he also reduced to submission the two and twenty kings! whom Pompey had subdued; and reduced under obedience to the Romans every thing in the countries of the east. 

\chapter{42}

\par \textit{The history of Cesar, king of the Romans}

1 It is reported that there was at Rome a certain woman who was pregnant, who, being near to her delivery, and racked with most violent pains of childbirth, died: 

2 but as the child was in motion, the belly of the mother was opened, and it was brought forth thence and lived, and grew, and was named Julius, because he was born in the fifth month; and was called Cesar, 

3 because the belly of his mother, from whence he was extracted, was ripped open. (Lat. caesa.) 

4 But when the elder of Rome sent Pompey into the east, he likewise sent Cesar into the west, to subdue certain nations which had revolted from the Romans. 

5 And Ceesar went, and conquered them, and reduced them to obedience to the Romans, and returned to Rome with great glory: 

6 and his fame increased, and his affairs became much renowned, and excessive pride took hold on him; wherefore he requested the Romans to name him king. 

7 But the elder and governors answered him, “Truly our fathers took an oath in the days of Tarquin the king,—who had taken by force another man’s wife, who laid hands on herself that he might not enjoy her,

8 —that they would not give the title of king to any of those who should be placed at the head of their affairs; on account of which oath (said they) we are not able to gratify you in this particular.” 

9 Wherefore he stirred up seditions, and waged furious battles at Rome, slaying many people, until he seized on the throne of the Romans, and entitled himself king, putting a diadem on his head. 

10 From thenceforth they were called kings of the Romans, from their kingdom: they were also called Ceesars.

11 When therefore Pompey heard this news of Caesar, and that he had slain the three hundred and twenty governors, he collected his armies and marched into Cappadocia: 

12 and Cesar going to meet him engaged him, conquered and slew him, and gained possession of the whole territory of the Romans. 

13 After this, Caesar went into the province of Syria; whom Mithridates the Armenian met with his army, assuring him that he was come with peaceful designs, and was ready to attack whatsoever enemies he should command. 

14 Cesar ordered him to depart into Egypt; and Mithridates marched till he came to Ascalon. 

15 Now Hyrcanus feared Caesar very much, because his submission to Pompey, whom Cesar had slain, was known. 

16 Wherefore he despatched hastily Antipater with a brave army to assist Mithridates: and Antipater marched to him, and aided him against a certain one of the cities of Egypt, and they took it. 

17 But as they departed thence, they found an army of the Jews who dwelled in Egypt, making a stand at the entrance, to prevent Mithridates from entering Egypt. 

18 And Antipater produced to them a letter from Hyrcanus, commanding them to desist, and not oppose Mithridates, the friend of Cesar. And they forbore. 

19 But the others marched till they came to the city of the then reigning king; who came out to them with all the armies of the Egyptians, and when they engaged with him, he conquered and routed them; 

20 and Mithridates turned his back and fled; whom, when “he was surrounded by the Egyptian troops, Antipater saved from death: 

21 and Antipater and his men ceased not to resist the Egyptians in battle, whom he routed and conquered, and won the whole country of Egypt. 

22 And Mithridates wrote to Cesar, shewing him what Antipater had done, and what battles he had endured, and what wounds he had received; 

23 and that the winning of the country was to be ascribed not to him but to Antipater, and. that he had reduced the Egyptians to obedience unto Cesar. 

24 And when Cesar had read the letter of Mithridates, he commended Antipater for his exploits,-and resolved to advance and exalt him. 

25 After these acts, Mithridates and Antipater went to Cesar, who then was at Damascus; and he obtained from Caesar whatsoever he liked, and he promised him whatever he wished for. 

\chapter{43}

\par \textit{The account of the coming of Antigonus the son of Aristobulus unto Cesar, complaining of Antipater who had caused his father’s death.}

1 But Antigonus the sonsef Aristobulus came to Cesar, and related to him the expedition of Aristobulus his father to attack Pompey, and how obedient and obsequious he was to him. 

2 Then he told him that Hyrcanus and Antipater had secretly sent a man to his father to destroy him by poison, intending (sazd he) to assist Pompey against your friends. 

3 Cesar therefore sent to Antipater, and questioned him on this matter; to whom Antipater replied; 

4 “Certainly I did obey Pompey, because then he was the ruling person, and conferred benefits on me; but I did not now fight with the Egyptians for the sake of Pompey, who is already dead; 

5 nor did I go through difficulties in defeating them and reducing them to obedience to Pompey; but I did this out of duty to Cesar, and that I might reduce Lins to obedience to him.” 

6 Then Antipater uncovered his head and his hands, and said; “These wounds, which are on my head and body, testify that my affection and obedience to Cesar are greater than my affection and obedience to Pompey; 

7 for I did not expose myself in the days of Pompey, to the things to which I have exposed my self in the days of king Cesar.” 

8 And Cesar said to him, “Peace be to thee, and to all thy friends, O bravest of the Jews: for thou hast truly shewn this fortitude, magnanimity, obedience, and affection towards us.” 

9 And from that time Cesar increased in affection towards Antipater, and advanced him above all his friends, and promoted him to be general of his armies, and took him with him into the country of the Persians: 

10 and he saw from his bravery and his successful exploits, that he more and more excited in him a longing and affection for him: 

11 at length he brought him back into the land of Judah, covered with honours and crowned with a post of authority . 

12 And Caesar marched to Rome, having settled the affairs of Hyrcanus; who built the walls of the Holy City, and conducted himself towards the people in a most excellent manner: 

13 for he was a good man, endued with virtues, of irreproachable life, but his inability in wars was notorious to all men. 


\chapter{44}

\par \textit{The account of the embassy of Hyrcanus to Cesar, asking for a renewal of the treaty between them; and of the copy of the treaty which Hyrcanus sent to him.}

1 Therefore Hyrcanus sent ambassadors to Cesar, with a letter concerning a renewal of the treaty which was between him and the Romans. 

2 And when Hyrcanus’ ambassadors came to Caesar, he ordered them to be seated in his presence; an honour which he had not conferred on any one of the ambassadors of the kings who used to come to him. 

3 Moreover he acted kindly to them, by expediting their business, and ordered an answer to be given to Hyrcanus’ letter; to whom also he wrote the treaty, of which the following is a copy. 

4 “From Cesar, king of kings, to the princes of the Romans who are at Tyre and Sidon, peace be with you. 

5 I give you to know, that a letter of Hyrcanus the son of Alexander, both kings of the Jews, has been brought to me; 

6 at the arrival of which I rejoiced, by reason of the continued good-will which both he and his people declare that they have towards me and the Roman nation. 

7 And verily the truth of his words I have proved by this; that he formerly sent Antipater a captain of the Jews, and their cavalry, with Mithridates my friend, whom the troops of Egypt attacked; 

8 and he saved Mithridates from death, having won for us the country of Egypt, and reduced the Egyptians to obedience to the Romans: he also marched with me into the country of the Persians, serving as a volunteer. 

9 And therefore I order that all the inhabitants of the sea-coast, from Gaza as far as Sidon, shall pay all the tributes which they owe us, every year, to the house of the great God which is in Jerusalem; 

10 except the citizens of Sidon; and let these pay to it, according to the appointment of their tribute, twenty thousand five hundred and fifty vibe of wheat every year. 

11 I also order, that Laodicea and its possessions, and all things which were in the hand of the kings of Judah, even to the bank of the Euphrates; 

12 with all those places which the Asmoneans won from the passing over Jordan,— be restored to Hyrcanus the son of Alexander king of Judah. 

13 For all these things his fathers had won by their sword, but Pompey had unjustly taken them away in the time of Aristobulus: 

14 and from this time and for the future let them belong to Hyrcanus, and to the succeeding kings of Judah. 

15 And this treaty is for me, and for every one of the kings of Rome my successors: whoever therefore shall break it or any part of it, may God destroy him by the sword, and may his house and his government be made desolate and be cut down! 

16 And when you shall read this my epistle, write it in letters engraved on tables of brass, in the language of the Romans and in their characters, and in the language of the Greeks and in their characters: 

17 and place the tables in conspicuous parts of the temples which are at Tyre and_Sidon; that every person may be able to see them, and may understand what I have appointed for “Hyrcanus and the Jews.” 

\chapter{45}

\par \textit{The history of Cesar’s death}

1 There were with Cesar two of Pompey’s friends; of whom the one was called Cassius, and the other Brutus; who laid a plot to kill Cesar. 

2 For which purpose they concealed themselves in the temple? at Rome which he had set apart for himself to pray in. 

3 To which therefore when he came, careless, safe, and taking no sort of heed to himself, they rushed upon him, and killed him. 

4 And Cassius got possession of the throne, and gathered a large army, and transported it beyond the sea; fearing Caesar’s party if he should continue to reside at Rome. 

5 And he marched into the land of Asia, and wasted it: from thence he went into the country of Judah: 

6 and Antipater wished to attack him; but seeing that his strength was not equal to the task, he made peace with him. 

7 And Cassius laid a tribute of seven hundred talents of gold4 on the land of Judah; and Antipater bound himself as surety for the money; 

8 and charged his son Herod to raise it on the country of Judah, and to carry it to Cassius: who receiving it marched into the country of Macedonia, and there remained through fear of the Romans. 

\chapter{46}

\par \textit{The history of the death of Antipater}

1 Now the princes of Judah had taken counsel to slay Antipater; and for that purpose had se| cretly set upon him a man who was called Malchiah. 

2 And Malchiah made the attempt, but its execution was delayed for a long time. 

3 And the report of it reached Antipater, who sought out Malchiah to kill him: 

4 but Malchiah cleared himself in the sight of Antipater of the things whereof he had been accused to him; and sware to him that the report was groundless: and Antipater believed him, putting aside all suspicion from him. 

5 But Malchiah, having given a large sum of money to Hyrcanus’ cup-bearer, agreed with him to give Antipater poison, while he was on the banqueting couch in the king’s presence. 

6 And the cup-bearer did this, and king Antipater> died on that same day: and the thing was not by the design, nor with the knowledge, of the king. And when An7 tipater was dead, Hyrcanus substituted Malchiah in his place. 

\chapter{47}

\par \textit{The history of the death of Malchiah}

1 Now when Herod the son of Antipater was informed that Malchiah had caused his father’s death, he thought to rush openly upon Malchiah; but his brother prevented him from doing this, advising that he should be taken off by stratagem. 

2 And Herod went to Cassius, and told him what Malchiah had done: to whom the other replied, when I am gone to Tyre, and Hyrcanus is with me, and with him Malchiah, then rush on him and kill him. 

3 When therefore Cassius had gone to Tyre, and Hyrcanus had gone to join him, taking Malchiah with him; and they were standing together in Cassius’ presence, at a certain feast to which Cassius had invited them with all his friends: 

4 (now Cassius had given orders to his servants to do whatever Herod should order them:) 

5 Herod also was standing with his brother amongst the companions of Hyrcanus, and Herod agreed with some of the servants to kill Malchiah, when a signal should be given by a wink of the eye. 

6 When therefore Hyrcanus had eaten and drunken with his friends, they went to sleep in the afternoon: 

7 and when they had awaked from sleep, Hyrcanus ordered one to prepare a couch for him in the open air, before the entrance of the banqueting room in which they had slept: 

8 and he himself sat down, and commanded Malchiah to sit with him: he also ordered Herod and his brother to be seated: 

9 and Cassius’ servants stood near Hyrcanus; to whom Herod winked against Malchiah, and they immediately rushed on him and slew him: 

10 and Hyrcanus was greatly frightened, and fell into a fit of fainting. 

11 But when Cassius’ attendants had retired, and the slain Malchiah was carried out, Hyrcanus came to himself again, and asked of Herod the cause of Malchiah’s death. 

12 And Herod answered; “I am wholly ignorant, nor do I know the “cause of the thing.” And Hyrcanus held his peace, and never again asked more of the matter. 

13 And Cassius marched into Macedonia, to meet Octavian¢ the son of Cesar’s brother, and Antony the general of his army: for they had set out from Rome with a great army in search of Cassius. 

\chapter{48}

\par \textit{The history of Octavian, (the same is Augustus the son of Cesar’s brother,) and of Antony, general of his army, and of Cassius’ death.}

1 When Octavian had marched into Macedonia, Cassius went out to meet him, and engaged with him; and Cassius was put to flight; 

2 whom Octavian pursuing, entirely defeated and killed: and Octavian won the kingdom in place of his uncle Cesar; and he also was surnamed Ceesar, after the name of his uncle. 

3 Now when the death of Cassius became known to Hyrcanus, he sent ambassadors with presents, money, and jewels, to Augustus and Antony: 

4 and he wrote to him, asking for a renewal of the treaty which had been entered anto with Cesar; 

5 and that he would order all the captives of Judah who were in his kingdom, and those who had been made captives in the days of Cassius, to be set free; 

6 and that he would permit all the Jews who were in the country of the Greeks, and in the land of Asia, to return into the country of Judah, 

7 without requiring any ransom, or redemption, or any obstacle being thrown in the way by any one. 

8 So when the ambassadors of Hyrcanus came to Augustus, with their letter and presents, he honoured the ambassadors, 

9 and accepted the presents, and acceded to all things which Hyrcanus had asked; writing to him a letter, of which this is the copy. 

10 “From Augustus, king of kings, and Antony his colleague, to Hyrcanus king of Judah; Health be to you. 

11 Your letter has even now reached us, at which we rejoiced; and we have sent that which you wished, respecting the renewal of the treaty, and the writing, to all our provinces, which extend from the country of the Indias even to the western ocean. 

12 But that which delayed us from sooner writing to you concerning the renewal of the treaty was, our occupation in subduing Cassius, that filthy tyrant; 

13 who, acting wickedly towards Cesar, 

14 Wherefore we have contended with him with our whole strength, until the great and good God rendered us victorious, and caused him to fall into our hands; 

15 whom we have put to death. We have also slain Brutus his colleague; and we have delivered the country of Asia out of his hand, after he had laid it waste, and had exterminated its inhabitants. 

16 Nor did he adhere to any engagement; nor honour any temple; nor do justice to the oppressed; nor pity a Jew, or any other of our subjects: 

17 but with his followers he wickedly did many evils to all men through oppression and tyranny: 

18 wherefore God hath turned their malice back on their own heads, delivering them up, with those who were confederate with them. 

19 Rejoice now therefore, O king Hyrcanus, and other Jews, and inhabitants of the Holy Region, and priests who are in the temple of Jerusalem: 

20 and let them accept the present which we have sent to the most glorious temple, and pray for Augustus ever. 

21 We have written also to all our provinces, that there remain in none of them any one of the Jews, be it servant or maid, but that all should be let go, without price and without ransom: 

22 and that they should be hindered by no person from returning into the land of Judah; and this by command of Augustus, and likewise of Antony his colleague.” 

23 Moreover, he wrote’ to his friends, who are at Tyre and Sidon, and in other places, to restore whatever they had taken out of the land of Judah in the days of that filthy Cassius: 

24 and to treat the Jews peaceably, and not to oppose them in any thing, and to do for them whatever Cesar had decreed in his treaty with them. 

25 Now Antony remained in the country of Syria; and Cleopatra queen of Egypt came to him, whom he took for his wife. 

26 She was a wise woman, skilled in magical arts and properties of things: so that she enticed him, and got possession of his heart to that degree that he could deny her nothing. 

27 At this same time, a hundred men of the chief of the Jews went to Antony‘, and complained of Herod and his brother Phaselus the sons of Antipater, saying; 

28 They have now gotten every thing belonging to Hyrcanus, and there remains to him nothing of the kingdom except the name; and the concealment of this matter is a proof of the captivity of their lord. 

29 But when Antony had inquired of Hyrcanus the truth of the things which they had mentioned to him, Hyrcanus declared that they spoke falsely; clearing Herod and his brother from that which they had laid to their charge. 

30 And Antony rejoiced at this; for he was greatly inclined towards them, and loved them. 

31 Moreover, other persons at another time complained to him of Herod and his brother, when he was at Tyre: 

32 but he not only refused to entertain their words, but put to death some of them, and cast the rest into prison; 

33 and he advanced the dignity of Herod and his brother, doing them services, and sent them back to Jerusalem with great honour. But Antony himself; going into the country of the Persians‘, defeated them, and subdued them, and returned to Rome. 

\chapter{49}

\par \textit{The history of Antigonus the son of Aristobulus, and of his capedition against his uncle Hyrcanus: and of the succour which was obtained from the king of the Persians.}

1 When Augustus and Antony had arrived at Rome, Antigonus went to the king of the Persians, and promised him a thousand talents of coined gold, and eight hundred virgins of the daughters of Judah and of its princes, beautiful and wise; 

2 if he would send with him a general leading a great army against Jerusalem, and would order him to make him king over Judah, and would take prisoner his uncle Hyrcanus, and kill Herod and his brother. 

3 To whom the king assenting, sent with him a general with a great army: 

4 and they marched until they came into the land of Syria; and they slew a friend of Antony and certain Romans who were dwelling there. 

5 From thence they marched against the Holy 5 City; professing security and peace, and that Antigonus had only come to pray in the sanctuary, and then would return to his own friends. 

6 And they entered the city; into which when they had gotten, they played foul, and began to kill men, and to plunder the city, according to the orders of the king of Persia to them. 

7 And Herod and his men ran forward to defend the palace of Hyrcanus: but he sent his brother, and commanded him to guard the way which leads from the walls to the palace. 

8 And when he had possessed himself of each position, he chose out some of his men, and marched against the Persians who were in the city; 

9 and his brother followed with a certain number of his men; and they slew the greater part of the Persians who were in the city, but the rest fled out of the city. 

10 And when the general of the Persians saw that things had not gone to his mind, he despatched messengers to Herod and his brother, to treat for peace; 

11 informing them, that now he was satisfied of their valour and bravery, that they ought to be preferred to Antigonus; and that for that reason he would persuade his troops to aid Hyrcanus and them rather than Antigonus: 

12 and this his wish he confirmed by the most solemn oaths, so that Hyrcanus and Phaselus believed him, but not Herod. 

13 So Hyrcanus and Phaselus, going out to the general of the Persians, signified to him their reliance on him; and he advised them to go to his colleague who was at Damascus; and they went. 

14 And when they were come to him, he received them honourably, and made a display of holding them in high esteem, and treated them courteously; although he had secretly given orders that they should be made prisoners. 

15 And some of the principal men of the land coming to them, told them of this very design; advising them to flee, with a promise of aiding their escape. 

16 But they did not trust these men, fearing lest it were some plot against them; wherefore they stayed. 

17 And when night came on, they were seized: Phaselus indeed laid hands on himself; but Hyrcanus was bound in chains, and by order of the general of the Persians his ear was cut off, that he might never be high priest again; 

18 and he sent him to Herak, to the king of the Persians; to whom when he came, the king ordered his chains to be struck off, and shewed him kindness; 

19 and he remained in Herak loaded with honours, until Herod demanded him from the king of the Persians: and when he was sent back to Herod, those things befell him which did befall him. 

20 After this, the general went up with Antigonus into the Holy City: and it was told Herod what had been done to. Hyrcanus and Phaselus: 

21 wherefore taking his mother Cypris, and his wife Mariamne the daughter of Aristobulus, and her mother Alexandra, he sent them with horses and much baggage to Joseph his brother to mount Sarah: 

22 but himself with an army of a thousand men marched slowly, and waited for those of the Persians who might attempt to pursue him. 

23 And the general of the Persians pursued him with his army; whom Herod attacked, and conquered, and put to flight. 

24 After this, Antigonus’ troops also pursued him, and fought with him most fiercely: and these he smote, and slew great numbers of them. 

25 Then he marched to the mountains of Sarah; and found his brother Josephus, whom he ordered to secure the families in a safe place, and to provide all things which were necessary for them: 

26 and he gave them abundance of money, that if need were, they might buy themselves provisions. 

27 And having left his men with his brother Josephus, himself with a few companions went into Egypt, that he might take ship and proceed to the country of the Romans. 

28 Cleopatra entertained him courteously, and requested him to take the command of her armies and the management of all her affairs; to whom he notified that it was quite necessary for him to go to Rome. 

29 And she gave him money and ships: and he went till he reached Rome, and abode with Antony, and told him what Antigonus had done, and what he had committed against Hyrcanus and his brother, by help of the king of the Persians: 

30 and Antony rode with him to Augustus and to the senate, and told them the selfsame thing. 

\chapter{50}

\par \textit{The history of Herod when the Romans appointed him king over the Jews, and his departure from Rome with an army to fight against the Holy House.}

AvuGUSTUS and the senate, informed of what 1 Antigonus had done, with one consent appointed Herod king over the Jews; commanding him to 2 put a golden diadem on his head, and to mount a horse, and that it should be proclaimed by trumpets preceding him, “Herod is king over the Jews “and the holy city Jerusalem:” which was done. And returning to Augustus, he rode, and Augus3 tus, and Antony; and they went to Antony’s house, who had invited the senate and all the citizens of Rome to a banquet which he had prepared. And they did eat and drink, and rejoiced over 4 

a Compare Josephus, Ant. XIV. 26, 27. Bell. I. 11, 12. 

B.C. 40. CHAP. L. 399 

Herod with great joy, making with him a treaty engraven in tables of brass; and it was placed in 
5 the temples. And they inscribed that day as the first of Herod’s reign, and from that time it was taken for an zera, by which times are counted. 
6 After these things, Antony and Herod departed by sea with a great and abundant army: and when they came to Antioch, they divided their 
7 forces: and Antony took a part, and led it into the country of the Persians which is Herak” and the parts adjacent: and Herod, taking another 
8 part, went straight till he came to Ptolemais. So Antigonus, hearing that Antony had made an expedition into the country of the Persians, and that Herod had reached Ptolemais, marched out from the Holy House to the mountain Sarah‘, to take Josephus, Herod’s brother, and those who were 
9 with him. Whom he assaulted, and besieged; and having cut off a canal, intercepted the water which flowed down to them: so that thirst prevailed among them, and their affairs were reduced 
10 to great straits. Wherefore Josephus determined to flee; and the families had deliberated upon surrendering themselves to Antigonus, if Josephus 
11 should flee. But God sent to them an abundant rain, which filled all their cisterns and vessels: wherefore their hearts were encouraged, and their 
12 condition was improved; and Josephus continued to repulse Antony‘ and his men from the strong 

b See the note on ch. liv.l. take for Antigonus: Antony, ¢ See above, at ch. xlix. as we have read at ver. 7, had 21. 24, and the note. Joseled his troops into Babylonia phus states the place to be at this time, where we find Massada. him employed at ch, li. 1—3. 

d This obviously is a mis& 

400 BOOK V. B.C. 39. 

hold, nor could the latter gain any advantage over him. But Herod marched straight to the moun13 tain Sarah, to bring back his brother, and the families, and the men who were with him, to Jeru. salem. And he found Antigonus besieging his 14 brother; upon whom he made a sudden attack; and Josephus and his men came out to them, and the greater part of Antigonus’ army was destroyed, and he fled into Jerusalem. 
Whom Herod pursued with a great army of 15 Jews, who had come to him from every quarter, when they found that he had returned; and he was well supplied with assistance, so that he stood in less need of the army of the Romans. When 16 therefore Herod had reached the Holy City, An_ tigonus shut the gates in his face; and fought against him; and sent much money to the chiefs of the army of the Romans, requesting them not to assist Herod: which they did for him. Wherefore the war lasted a long time between 17 Antigonus and Herod, neither of them prevailing over his fellow [i.e. antagonist ]. 

CHAPTER LI.# 

The history of the magnanimity of certain of Herod's men, and of their bravery. 

Now thieves, and they who were longing for 1 
p.c. the property of others, had multiplied 
during the time of Antigonus; betaking 2 themselves to some caves in the mountains, to which there was no approach except for one man 

e It appears that Silo, a fence of Antigonus’ interests. Roman general, was bribed, a Compare. Josephus, Ant. and exerted himself in deXIV. 27. Bell, I. 12. 

B.C. 39. CHAP. LI. 401 

at a time, through certain places fitted for the 
3 purpose by them, and known to them alone: and even though others should know them, they could not go up to the cave; because that a man was ever ready at the mouth, who, with a very little trouble, could easily repel a person who was 
4 climbing up. And now some of these men had gotten to themselves in that cave abundance of arms, provisions, and drink, and all those things 
5 which they needed; together with all the spoils which they had gained by attacking those whom they met, and that which they had taken by right or wrong. 
6 When therefore Herod had learnt their proceedings, and found that their matters were likely to cause delay”; also that men could not at present mount up to them by ladders, nor in fact 
7 climb up in any way: he made use of great wooden chests fitted and joined together, and filled them with men, (adding food and water,) bearing 
8 very long hooked spears: and those chests he ordered to be let down from the summit of the mountains, at the middle of which the caves were, until they were placed opposite to their mouths: 
9 and when they were opposite to these, he desired that his men should attack them in close fight with swords, and from a distance should drag 
10 them out with those spears. And the chests were 11 made, and filled with men. And when some of them were let down, and were opposite to the mouths of those caves, no information having 

b In other words, that in be put a stop to, from the all probability their marauddificulty of coming at their ing system would not easily retreats. 

pd 

402 BOOK V. B.C. 39. 

been given to the persons living there; one of the 
men who were in the chests rushed into the caves, 
followed by his companions; and they killed the robbers who were in them, together with their followers, and threw them down into the valleys below; all the men whom Herod had sent, emulating these jist. And in this exploit, their courage, bravery, and boldness was so conspicuous, that the like of it was never seen: and they wholly rooted out the robbers from all those parts. 
CHAPTER LII.@ 
An account of Antony's return from the country of the Persians after killing the king of the Persians, and his meeting with Herod. 
THEN Antony, after leaving Herod», marched from Antioch into the country of the Persians, and fought with the king of the Persians, over

13 

came, slew him, and won hes land; and having 2 

reduced the Persians to obedience to the Romans, he turned aside to the Euphrates‘. 

And when his fame was told to Herod, he set 3 

out to congratulate with him on his victory; and to request him to come with him into the Holy 

Country. And he found a very large multitude 4 

collected, wishing to approach Antony; to which many bodies of Arabians had opposed themselves, preventing it from coming to Antony’s presence. 

And Herod marched against the Arabians, and 5 

a Compare Joseph. Antiq. ¢ On the banks of which KIV. 27. Bell. 113. river. he laid siege to, and b See above, ch.1.7. The subsequently took, the imgreat defeat, however, was portant city of Samosata. given to the Parthians, not 4 Jn the neighbourhood of by Antony, but by Ventidius Antioch. (Josephus.) his lieutenant. 

B.C. 38. CHAP. LII. 403 

slew them, opening a passage for all who wished 
6 to approach Antony. And this was reported to Antony, before that Herod arrived: whereupon he sent him a golden diadem, and a great many horses. 
7 But when Herod came, Antony received him courteously, praising him for his exploits against the Arabians: and he attached to him Sosius the general of his army, with a large force, ordering him to go with him to the city of the Holy House: 
8 giving him also letters to all the country of Syria, which is from Damascus even to the Euphrates, and from the Euphrates to the country of Ar
9 menia; saying to them, “Augustus, king of kings, “and Antony his colleague, and the Roman se“nate, have now appointed Herod king over the “Jews; and they desire you to lead forth all “your men of war with Herod to assist him: if “therefore you act contrary to this, you must go to war with us.” 
10 Then Antony marched to the sea-coast, and thence into Egypt: but Herod, and Sosius with his 1] army, commanded the forces of Syria. But when Herod drew nigh to Damascus, he found that » « his brother Josephus had gone out from the Holy House with an army of Romans, to besiege 12 Jericho and to cut down its corn: against whom came forth Pappus the general of Antigonus’ forces, and slew of them thirty thousand, having 13 also slain Josephus Herod’s brother: and when his head was presented to Antigonus, Pheroras 

€ Probably there isanerror about three thousand men, in the number. Josephus rewere slain, lates that six cohorts, that is, 

dd 2 

404 BOOK V. B.C. 38. 

his brother bought it for five hundred talents‘, and buried it in the sepulchre of his fathers: and 14 he heard also that Antigonus and Pappus were advancing against him with a large army. Which 15 Herod having fully ascertained, determined to make an onset on Antigonus, and to crush him unexpectedly: and he agreed with Sosius that he 16 should take twelve thousand Romans and twenty thousand Jews, and march against Antigonus, but that the other should slowly follow his footsteps with the remainder of the army. 
And Herod marched with his troops in a body, 17 and met with Antigonus in the mountainous parts of Galilee: and they fought with him from midday even until night. Then the army was dis18 persed; and Herod with some of his men passed the night in a certain house, and the house fell upon them; but they all escaped’ from the ruin with their lives, without a bone of any one of them being broken. 
Shortly afterwards Herod hastened to fight 19 with Antigonus, and there was a very great battle between them, and Antigonus fled into the Holy House; Pappus meanwhile resisting bravely, and continuing the fight, for he was high-spirited and very brave. And the greater part of Anti20 gonus’ army was slain on that day; Pappus also was killed, whose head Pheroras cut off, and they 

f Truly a large sum to be partiality for Herod, adduces given for such an object. Jothis occurrence for a proof sephus, with greater probabihow much he was beloved by lity of being right, states God, whose providence preSifty: served his life in so extraor
& Josephus, who frequently dinary a manner. in his works manifests a strong 

B.C. 87. CHAP. LII. 405 

carried it to Herod, who ordered it to be buried. 
When therefore none remained of Antigonus’ army, except prisoners or runaways, Herod gave orders to his men to take rest, and to eat and 22 drink. But he himself went to a certain bath 
which was in the next town, and went into the 23 bath unarmed. Now there lay hidden: in: the bath three strong and brave men, holding in their hands drawn swords: who, when they saw him come into the bath, and unarmed, made all haste to go out one after the other, being afraid of him; and so he escaped. 
After this came Sosius; and they marched together to the city of the Holy House, which they surrounded with a trench; and fierce battles took 25 place between them and Antigonus: and great 
numbers of Sosius’ men were slain, Antigonus 
frequently overcoming them; but he could not 
put them to flight, by reason of their firmness 26 and endurance in bearing fis assaults. Then Herod prevailed against Antigonus; and Antigonus fled, and entering the city shut the gates against Herod, and Herod besieged him a long time. 
But on a certain night the guards of the gate fell asleep: which some of Herod’s men x ¢. discovering, twenty of them ran, and taking %7ladders placed them against the wall, and climbing 

21 

24 

27 

h Or rather, Herod cut off _ the head, and sent it to Pheroras. ne 
~~ 1 It appears from Josephus that they had not gone thither for the purpose of attacking Herod; but that they had 

chanced to resort to the bath as a place of concealment; and upon the unexpected appearance of Herod with his attendant, were too happy to escape with their lives. 

pd3 

406 BOOK V. B.C. 87. 

up killed the guards. And Herod with his men 28 hastened to the gate of the city which was opposite to them, and burst it in, and entered the city. Which the Romans taking, began to slaughter 29 the citizens; at which Herod being troubled said to Sosius, “If you shall destroy all my people, over whom will you appoint me king?” and So30 sius ordered proclamation to be made that the sword should be. stayed; nor was any person slain after the proclamation. But Sosius’ cap31 tains, eager for prey, ran to plunder the house of God: but Herod standing at the gate, holding a drawn sword in his hand, prevented them; and sent to Sosius to restrain his men, promising them money. And Sosius ordered proclamation 32 to be made to his men to abstain from plunder, and they abstained. And they sought Antigonus and found him, and Antigonus was taken prisoner. After these things, Sosius betook himself into 33 Egypt to his colleague Antony, carrying with him Antigonus in chains. But Herod sent to 34 Antony a very great and fair present, requesting him to slay Antigonus; and Antony slew him: and this was in the third year of the reign of Herod, which also was the third year of Antigonus. 

\chapter{LIIL? The history of Herod after the death of Antigonus}WHEN Herod was certified of the death of An} 

k Thus terminated the goremarks on Antony’s putting vernment of the Asmonean to death the king, given by princes, in the hundred and Josephus out of Strabo. twenty-sixth year from its 4 Compare Joseph. Antiq. first establishment under JuXV.1. Bell. I. 13. das Maccabeus. See some 

B.C.37. CHAP. LIV. 407 

tigonus, he considered himself secure that no one of the royal Asmonzan family would contend with 2 him: wherefore he employed himself in ‘advancing the dignities, in kindnesses and promotions, of those who were well inclined to him and 3 obeyed his will. He also exerted himself in destroying those persons, together with their families, and in plundering their cattle and their goods, who had opposed him, furnishing aid against him. 4 And he oppressed persons, taking away their property, and despoiling all those who had shaken off obedience to the Jews; and slew those who re5 sisted him, and plundered their goods. Also he made an agreement with all who were obedient to 6 him, that they should pay him money. He also stationed guards at the gates of the Holy House, who might search those who went out, and take whatever gold or silver they should find on any 7 one, and bring it to him. He also ordered the coffins of the dead to be searched; and whatever money any person might endeavour to carry out 8 by stratagem, the same to be taken. And he heaped together so much money as none of the kings of the second house had amassed. 

CHAPTER LIV.? 

The history of Hyrcanus the son of Alexander, the uncle of Antigonus, and of his return into Jerusalem at the request of Herod, and of the death to which he put 

him. 1 Hyrcanus, after that the king of the Persians had set him at liberty, remained in Herakin, in 

@ Compare Joseph. Antiq. this narrative above, at ch. XV. 1, 2, 9. xix Wee: b See the preceding part of “Josephus im loco reads 

pd4 

408 BOOK V. B.C. 37. 

a most respectable condition and great honour: wherefore Herod was afraid lest any thing might 2 induce the king of the Persians to appoint him king’, and send him into the land of Judah. Wherefore wishing to set his mind at rest, he laid 3 plots for this business; and sent to the king of the Persians a very great present, and a letter; in which he made mention of Hyrcanus’ deserts 4 and kind deeds towards him; and how he had gone to Rome on account of what Antigonus his brother’s son had done to him; and that having 5 now attained the throne, and his affairs being in order, he wished to reward him in a proper manner for the benefits which he had conferred. 
So the king of the Persians sent a messenger to 6 Hyrcanus, saying; “If you wish to return into “the land of Judah, return: but I warn you to 7 “beware of Herod; and I distinctly inform you, “that he does not seek for you to do you any “good, but his design is to render himself secure, “as there is none remaining whom he fears, ex“cept you: wherefore take heed of him most dili“gently, and be not led into a snare.” The Jews 8 of Babylon also came to him, and said to him the like words. Again they say to him, “You now 9 “are an old man, and not fit to discharge the “office of high priest, because of the stain which “your nephew inflicted on you: but Herod is a 10 

Babylon. In fact Yerak, or from the office of high priest, Irak, the Arabian name for yet the crafty Herod knew the district or country of Bathat this was no obstacle to bylonia, is retained to the his reappearing among his present day. See above, ch. countrymen in the capacity of shi 7elB. 1.7. ~, their monarch. 
~~ d Although the loss of his See the account of this ear disqualified Hyrcanus transaction at ch. xlix. 16, 

B.C. 37. CHAP. LIV. 409 

 bad man, and a shedder of blood; and he re
“calls you only because he fears you; and you 
“do not want for any thing among us, and you 
“are with us in that station in which you ought 11 to be. And your family there is in the best 
 condition; wherefore remain with us, and do not 
 aid your enemy against yourself.” 12 But Hyrcanus acceded not to their words; nor 
listened to the advice of one who advised him 13 well. And he set out and journeyed till he came into the Holy City, for the very great longing which he had towards the house of God, his family, and his country. 
And when he had come near to the city, Herod met him, shewing such honour and magnificence, that Hyrcanus was deceived, and trusted in him. 15 And Herod in the public assembly, and before his 
own friends, used to call him “Father:” but ne
vertheless he ceased not to devise plots in his heart, only so that they should not be imputed to 16 him. Wherefore Alexandra and Mariamne her daughter go to Hyrcanus, putting him in fear of 
Herod, and counselling him to take care of him17 self; but neither to them did he attend, although 
they repeated this to him again and again, ad
vising him to flee to some one of the kings of the 18 Arabians: yet he attended not to all these things, until they drove him to it by repeated warnings and alarmings. 19 Then therefore-he wrote to that king’ of Ara

14 

of this book, and read the note there. 
f Namely, Malchus. See above, ch. xlixs-20, and the corresponding part in Jose

phus. It is to be observed, that Josephus places the present transaction somewhat later in the history, viz. after Herod had heard of the de

410 BOOK V. B.C. 37. 

bia; and having sent for a certain man, (whose: 

brother Herod had slain, and had confiscated his goods, and had visited him with many evils,) he told him that he wished to impart to him a certain secret, adjuring him not to tell it to any one; and giving him money and the letter to the king of the Arabians, communicated to him what he 

20 

requested in the letter. So the messenger, having 21 

received the letter, thought that he should obtain a high post with Herod, and should remove from himself the evil which he was continually fearing at his hands, if he communicated the matter to Herod; and that this would be more profitable to him than the keeping of Hyrcanus’ secret: since in the other case he was not safe, and sure that the thing would not be told to Herod at some time or other, and thus would be the cause of his destruction. He therefore carried the letter to Herod, and unfolded to him the whole business: who said to him, Carry the letter, as it is, to the king of the Arabians, and bring me back his answer, that I may know it: tell me also the place where the men will be, whom the king of the Arabians will send, that Hyrcanus may go back with them. 
So the messenger went, and carried Hyrcanus’ letter to the king of the Arabians; who rejoiced, and sent some of his men; ordering them to go 

feat of Antony at Actium, and had become apprehensive of the reception which he might meet with from Au. gustus; i.e. in the year B.C. 30, which indeed appears nearly to agree with the state

ment made in the last verse of this chapter, that Hyrcanus had reigned forty years. & Josephus names him Dositheus, and his brother Josephus. 

22 

23 

24 

25 

26 

B.C. 37. CHAP. LIV. 411 

to a certain place near to the Holy City, and there to wait until Hyrcanus should come to them; and then to attend Hyrcanus till they brought him to 27 his presence. He wrote likewise to Hyrcanus an answer to his letter, and sent it by the messenger. 28 So the men proceeded with the messenger to the appointed place, and there waited: but the messenger carried the letter to Herod, who learned its contents: he told him also the place of the men, to whom Herod sent persons to take them. Afterwards, having sent for seventy old men of the elders of the Jews, and having sent also for Hyrcanus; when he came, he said to him, Is there any interchange of letters between you and 30 the king of the Arabians? and Hyrcanus said, No. Then he said to him, Did you send that you 31 might flee to him? and he said, No. And Herod ordered his messenger to come forward, and the Arabians, and the horses; he also brought out 32 the answer to his letter, and it was read. Then he commanded Hyrcanus’ head to be stricken off'; and his head was stricken off, and no one dared to utter a word for him. 

29 

which remain to us, it appears 

h Josephus, agreeing in these particulars, informs us that he took his account from the “Commentaries, or Acts, “of Herod himself,” other authors relating them in a different manner. There is reason to believe that these Acts were written by a personal friend of Herod, Nicolaus Damascenus, who is menticned by Josephus, Ant. XIV. 2; XVI. 15, 16,17; XVII. 7. From the accounts of him 

that Nicolaus was intimate with Augustus; and in fact that it was he who succeeded in procuring for Herod a favourable reception by the Roman court, at a most critical juncture. He wrote several works; as, “A History of « Augustus.” “A History “of the World:” a large volume of “Assyrian History:” « A Collection of strange « Customs,” &c. See Photii 

412 

Now Hyrcanus had delivered Herodi from the 33 

BOOK V. 

B.C. 36. 

death which was justly awarded him in the assembly of judgment, commanding the assembly to be deferred till the morrow, and sending away 

Herod that same night. Whence he was destined 34 

to become his murderer, regardless of his services 

to him and to his father. 

Hyrcanus was put to 35 

death when he was eighty years old, and he reigned forty years: nor was there any one of the 

kings of the Asmonzan 

race of a more praise

worthy conduct, or more honourable way of life. 

\chapter{LV.3 The history of Aristobulus the son of Hyrcanus}

ARISTOBULUS the son of Hyrcanus? was of such 1 

p.c, beauty of form, and 

exquisite figure and un

36. derstanding, that his equal was not known. His sister Mariamne also, the wife of Herod, was 2 like to him in beauty; and Herod was wonder

fully attached to her. 

But Herod was averse to 3 

appoint Aristobulus high priest in the place of his 

Bibliothec. cod. CLXXXIX. Montacutii Apparatum 5 ad Origines Ecclesiast. p. 169. (ed. 1635), Valesii Excerpta Peiresc. 4to. 1634, where are considerable fragments of his writings: Fabricii Biblioth. Grec. edit. Harles. III. p: 500: and especially, Grotii Epistol. ad Gallos, p. 249— 320. edit. 1648. 12mo. 
i See Joseph. Ant. XIV. 9. 
k Josephus, although in general a favourer of Herod, cannot here refrain from bearing testimony to the respectable character of Hyrcanus, 

and to the shameful usage which he met with at the hands of Herod, whose very best friend and benefactor he in truth had been. 
a Compare Josephus, Ant. DOV ah 
b This is an error: Aristobulus and Mariamne were the children, not of Hyrcanus, but of Alexander son of Aristobulus the brother of Hyrcanus, by Alexandra daughter of Hyrcanus. See the Genealogical Table subjoined to the introduction to this volume. 

B.C. 36. CHAP. LV. 413 

father; lest the Jews, being attached to him through their affection to his father, should at 
4 some future time make him king. Wherefore he appointed some one of the number of common priests, who was not of the family of the Asmoneeans, to be high priest. 
5 At which Alexandra the mother of Aristobulus being vexed, wrote to Cleopatra; requesting to have a letter from Antony to Herod, that he should remove the priest whom he had elevated, and appoint her son Aristobulus high priest in 
6 his stead. And Cleopatra granted this; and requested Antony to write a letter to Herod on this subject, and to send it by some chief man of his 
7 servants. So Antony wrote a letter, and sent. it by his servant Gellius: and Gellius coming to 
8 Herod, delivered to him Antony’s letter. But Herod fforbore to do that which Antony had written ¢o order, asserting that it was not the custom among the Jews to depose any priest from his station. 
9 Now it happened that Gellius saw Aristobulus, and was greatly struck with the beauty of his form and the perfection of his carriage, which he 
10 saw. Wherefore he painted a picture of his likeness, and sent it to Antony, writing beneath the picture to this effect; that no man had begotten Aristobulus, but that an angel cohabiting with 
11 Alexandra, begat him on her. Therefore when the picture reached Antony, he was seized with a 

¢ Josephus states, that he nelus. Ant. XV.2. Yet afsent to Babylon for one of terwards he describes him as the obscure Jews residing being rod dpxtepatixod yévovs. there, whose name was Ana(See XV. 3.) 

414 BOOK V. B.C. 36. 

most vehement desire to see Aristobulus. And he wrote a letter to Herod, reminding him how he had appointed him king, and had assisted him against his enemies, recounting his kindnesses towards him: adding a request, that he would send Aristobulus to him; and he threatened him in this business for the words‘ which he had sent back. 
But when Antony’s epistle was brought to Herod, he refused to send Aristobulus, knowing what Antony designed; and on that account he disdained to do it: and he hastily deposed 4 the high priest whom he had appointed, establishing Aristobulus in his place. 
And then he wrote to Antony, informing him that he had already executed that which he had formerly written to him, about the placing Aristobulus in his father’s post, before his dast letter arrived: which business he had fo that time delayed, because it was necessary to debate the matter with the priests and Jews, after an interval of some days, as the thing was unusual; but it having passed according to his wish, he had immediately appointed him. But now that he was appointed, it was not lawful for him to go out of Jerusalem; as he was not king, but a priest at

¢ Or the sense may be, “he threatened him repeat edly if he should not com“ply with his desires in this “matter.” 
d Josephus complains, that in this Herod acted contrary to the law; which declared that a person once appointed high priest could not be re

moved from his office. He states that the first instance of this being done was that of Antiochus Epiphanes, who through bribery consented to depose Jason, and substituted his brother Menelaus (or Onias). See the account of that transaction above, at 3 Mace. iv. 23, 24. 

seca, 

12 

16 

17 

B. C. 35. CHAP. LV. 415 

18 tached to the service of the temple: and as often as he wished to compel him to go out, the Jews refused, and would not allow him, even if he 
19 should slay the greater part of them. Therefore 
when Herod’s letter reached Antony, he desisted 
from asking for Aristobulus; and Aristobulus was made high priest. 
Then came on the feast of tabernacles; and men, assembled before the house of God, beheld Aristobulus clothed in the sacerdotal robes standing at the altar, and they heard him blessing 21 them: and he pleased men so much, that they 
exhibited their affection towards him in a very 
22 marked manner. Which Herod being fully in
formed of, was much grieved; and feared lest, when 
Aristobulus’ party gained strength, he should de
mand from him the kingdom, if his life should 
be prolonged: wherefore he began to plot his death. 
Now it was customary for the kings to go out, after the feast of tabernacles, to some pleapo sure-residences at Jericho which former 24 kings had made: and there are many gardens adjoining each other, in which were wide and deep fish-ponds, to which they had conducted streams of water, and had erected fair buildings in those gardens: they also had built in Jericho fair palaces and handsome edifices. ee 
Now the author of the book ¢ relates, that bal

20 

23 

25 

e Who is the author alluded to? [See the same expression occurring before, at ch. xxv. 5, and below, at ch. lix. 96.] Josephus in various passages mentions the 

balsam-trees; as at Antiq. IV. 

5; VIII. 6: where he states that they were first brought into Judea by the queen of Saba, who presented them to Solomon: and IX. 1. Again 

416 

BOOK V. 

B.C. 35. 

sam-trees grew abundantly in Jericho; and that they were found no where else but there; and that many kings had carried them thence into 

their own country, but 

at XV. 4, he says, dpe 3 7 xepa TO Badoapov, Tiswdrarov Tov eet, Kal Tapa pdvors pverat. But Josephus is not “the “author” of this book. Several heathen authors relate the fact of the balsam being thought peculiar to Judea; but I have not discovered what early writer it is that has recorded the experiment of transplanting, and the failure of the original trees, mentioned in the text. 
Diodorus Siculus (speaking of the balsam) says, oddapod pev tHs GdAns oikoupevns etpirkopevov Tod duTod rovrov. Biblioth. II. 48. and XIX. 9. (ed. Wesseling.) But we must remember that Diodorus Siculus assigns the lake Asphaltites and its coasts, not to the Jews, but to the Nabathzan Arabians; which circumstance perhaps may help to reconcile the seemingly conflicting accounts of different historians; some of whom confine the plant to Judea, while others assert its native place to be Arabia. 
Pliny states, “Omnibus odoribus prefertur balsam“um, uni terrarum Jude “concessum.—Quondam in “duobus tantum hortis, utro“que regio, altero jugerum “xx non amplius, altero pau“ciorum. Ostendere arbusculam hane urbi Impera“tores Vespasiani; clarumque 

none grew, except those 

 dictu, a Pompeio magno in “triumpho arbores quoque “duximus. Servit nunc hec “et tributa pendit cum sua gente.” Hist. Nat. XII. 54. Again; “Fastidit balsamum alibi nasci.” Id. XVI. 39. Justin reports to the same effect.—It is proved, that the Romans, after their conquest of Judea, enlarged the plantations of balsam at Jericho; so that the produce became greater, and the article itself less costly. It is perhaps almost needless to observe, that the “vineyards of Engaddi,” mentioned in the Song of Solomon, are the groves of balsam, which were in that neighbourhood. Historians have said, that some of the trees were carried from Judea into Egypt by Cleopatra, in the days of Herod: but this is contested in a note to Haverkamp’s Josephus, vol. II. p.66. See the fullest details on this subject in Salmasii Exercitationes Pliniane in Solini Polyhistorem, cap. 35. p. 418— 
30. edit. Traject. 1689: also in the notes on Theophrasti Hist. Plant. IX. 6. edit. Stapel, fol. 1644. 
Some Arabian authors relate, that the balsam-tree had been carried by the Saracens to Matarea, (the ancient Heliopolis,) but that the plants had continually failed there. 

B. C. 35. CHAP. LY. 417 

26 which were carried into Egypt; and that they did not fail in Jericho until after the destruction of the second House; but then they withered away, and never sprouted forth again. 
27 So Herod went out to Jericho in quest of plea
28 sure, and Aristobulus followed him. And when they came to Jericho, Herod commanded some of his servants to go down into the fish-ponds, and 
play as was customary: and that if Aristobulus should come down to them, they should play with 
29 him for some time, and then drown him. But Herod sat in a banqueting-room which he had prepared for himself to sit in: and Herod sent for Aristobulus, and made him sit by his side: also the chief of his attendants and of his friends sat 
30 in his presence: and he commanded eatables and drink to be brought; and they did eat and drink: and the attendants hastened down to the waters 
31 according to custom, and sported. And Aristobulus greatly wished to go down with them into the water, the wine now mastering them, and 
32 asked leave of Herod to do so: who replied, This neither befits you nor any one like you: and when he was urgent, he admonished him and forbade him: but when Aristobulus repeated his request to him, he said to him, Do as you please. 
33 And then Herod, rising up went to a certain pa
34 lace that he might go to sleep there. And Aristobulus went down to the waters, and played for a long time with the attendants: who, when they perceived that being now weary and tired out he wished to go up, held him under water, killed him, and carried him out dead. 
Ee 

418 BOOK V. B.C. 35. 

And there was a great tumult of the people, 35 and cry, and a lamentation was set up. And He36 rod running up, came out to see what had happened: who, when he saw Aristobulus dead, bewailed him, and wept over him very tenderly with a most vehement flood of tears. Then he ordered 37 him to be borne into the Holy City, and accompanied him until he came into the city, and compelled the people to attend his funeral, and there was no point of the very highest honour which he omitted to pay him. And he died when a 38 youth of sixteen years of age, and his high-priesthood continued only fora few days. 
On which account enmity grew up between his 39 mother Alexandra and her daughter Mariamne Herod’s wife, and the mother and sister of Herod?. And the execrations and revilings which Mari40 amne heaped upon them were known; and although these reached Herod, yet he did not forbid her nor reprove her, through his great affection for her: he feared also, lest she should ima41 gine in her mind that he was well inclined towards the others: from hence these doings lasted long between these women. And Herod’s sister, 42 who was endued with the greatest malice, and consummate artifice, began to plot against Mariamne: but Mariamne was religious, upright, 43 modest, and virtuous: but she was a little tinged with haughtiness, pride, and hatred towards her husband. 

f Namely, Cypris his moabove, at ch. xlix. 20 3 and ther, who was mentioned Salome his sister. 

B.C, 34. CHAP. LVI. 419 

CHAPTER LVI.@ 

The history of Antony, and of his expedition against Augustus, and of the aid which he asked from Herod. And an account of the earthquake which occurred in the land of Judah, and of the battle which took place between them and the Arabians. 

i CLEOPATRA, the queen of Egypt, was the wife of Antony: and she discovered such methods » ¢. of adorning and painting herself, by which women are wont to allure men, as no other woman 
2in the world had found out: so that, while she was a woman advanced in age, she seemed as a little unmarried girl, and even more delicate and 
3 more fair. Antony also found in her those methods of beauty, and those means of creating pleasure, which he had never found in the vast number of women whom he had enjoyed. Wherefore she so completely gained possession of Antony’s heart, that no room was left in it for affec
4 tion to any other person. She therefore persuaded him to discomfit certain kings who were subject to the Romans, from her own private considerations; and he obeyed her in this, putting to death certain kings at her instance; and some he left alive by her orders, making them servants and slaves to her. / 
5 And this was told to Augustus; who wrote to him, abominating such conduct, and desiring him 
6 not to be guilty of the like again. And Antony told Cleopatra what Augustus had written_to 

a Compare Joseph. Antig. bazes, &c.; the account of XV. 6, 7, 8. Bell. I. 14. which proceedings may be b As Lysanias, and Artaseen in Josephus. 

EeQ 

420 BOOK V. B.C, 32. 

to him; and she advised him to revolt from Augustus, and shewed him that the thing was very easy.; To whose opinion he assenting, openly played 7 pc, false with Augustus; and gathered an 33. army and supplies, that he might go by sea to Antioch, and thence might march by land to meet Augustus wheresoever he might chance to find him. He sent also for Herod, that he 8 might accompany him. And Herod went to him with a most powerful army and most complete n.c, supplies. And when he had come to him, 9 32. Antony said to him; Right reason advises us to make an expedition against the Arabians, and to engage with them: for we are by no means secure that they may not make an incursion upon the Jews and the land of Egypt, so soon as we shall have turned our backs. ~~ And Antony departed by sea: but Herod 10 made an inroad upon the Arabians: and Cleopatra sent a general named Athenio with a great army, to assist Herod in subduing the Arabians: and she commanded him to place Herod and his 11 men in the first rank’, and to make agreement with the king of the Arabians, that they together should enclose Herod and cut his men to pieces. To this she was led by a desire of obtaining pos12 session of all which Herod was worth: Alex13 andra also some time previously had requested her to induce Antony to put Herod to death; 




¢ Even thus, at an earlier “in the forefront of the hotperiod of Jewish history, had “test battle, and retire ye the same iniquitous command “from him, that he may be been given: “Set ye Uriah “smitten and die !” 

B.C. 82. CHAP. LVI. 421 

which indeed she had done, but Antony refused 
14 to commit this act. To this was added the circumstance, that Cleopatra had formerly longed for Herod, and had at some time desired intercourse with him; but he restrained himself, for he was chaste. And these were the causes which 
15 had induced her to this line of conduct. So Athenio coming to Herod, according to the command of Cleopatra, sent to make agreement with the king of the Arabians, that he might surround 
16 him. And when Herod and his Arabians met and encountered, Athenio and his men attacked Herod, who was intercepted between the two armies, and the battle grew fierce against him 
17 both before and behind. But Herod seeing what had happened, collected his men, and fought most vigorously until they were beyond the reach of both armies, after the greatest exertion; and he returned into the Holy House. 
18 And there happened a great earthquake in the land of Judah, such as had not occurred since the time of king Harbah4, in which a great number 
19 of men and of animals was destroyed. And this alarmed Herod much, and caused him great fear, and broke down his spirit. He therefore took counsel with the elders of Judah about making an agreement with all nations round about them; designing peace, and tranquillity, and the removal 
20 of wars and bloodshed. He sent also ambassadors on these matters to the surrounding nations, all of whom embraced the peace to which he had 

@ Probably by this name is Scripture that a violent earthmeant Uzziah king of Judah, quake took place. See Amos, in whose days we learn from i. 1. and Zechariah, xiv. 5. 

Ee3 




422 BOOK V. B.C. 32. 

invited them, except the king of the Arabians; who ordered the ambassadors whom Herod had 21 sent to him to be put to death; for he supposed that Herod had done this because his men had been destroyed in the earthquake, and therefore, being weakened, he had turned himself to making peace. Wherefore he resolved to go to war with 22 Herod; and having collected a large and wellprovided army, he marched against him. 
And this was told to Herod; and he was much 23 vexed, for two reasons: one, on account of the slaughter of his ambassadors, an act which none of the kings had hitherto committed; another, because he had dared to attack him, imagining in his mind his weakness and want of troops. But 24 he wished to shew him that the matter was otherwise: that all, to whom he had sent ambassadors, to treat of peace, might know that he had not done this through any fear or weakness, but from a wish of that which was kind and good; that no one might dare make attempts against the Jews, or imagine in his mind that they were weak. Besides, he wished to take vengeance on 25 the king of the Arabians for his ambassadors: on these accounts he determined in all haste to march , against him. / | 
Therefore he collected troops from the land of 26 Judah, and said to them: “You are aware of the “slaughter of our ambassadors perpetrated by “that Arab; an act which no king hitherto has “committed: for he thinks that we have been 27 “weakened and have become powerless; and he “has dared to provoke us, and thinks that he “shall obtain all his desires over us: nor will he 

B. C. 32. CHAP. LVI. 423 

28 “cease from warring on us continually. Where“fore you must struggle against difficulties, that “you may shew forth your bravery, and may “subdue your enemies, and bear off their spoils: 
29 “although fortune may at one time shew herself “favourable, at another time adverse to us, ac“cording to the custom and usual vicissitudes of 
30 “this world. In truth, you must immediately “undertake an expedition, to take vengeance on “those oppressors, and to curb the audacity of 
31 “all who hold you in little esteem. But if you “shall say, this earthquake has disheartened us, “and has destroyed great numbers of us; you “know full well, that it has destroyed none of 
32 “the fighting men, but certain others. Nor ought we to think it at all unreasonable, that it “has destroyed the worst among our nation, but “has left the best to survive. It is also un doubted, that this has improved your spirits 
33 “and your inward feelings. But the duty of him, “whom God has saved from destruction, and has preserved from ruin, requires that he should obey Him, and should do what is good and 
34 “right. And truly no obedience is more honour“able or glorious, than to seek redress for the “‘ oppressed on the oppressor; and to subdue the “enemies of God and his religion and nation, by “aiding those who shew obedience and attention 
35 “to Him. Nor is it unknown to you, what befell “us lately with those Arabs, when they had sur
e Josephus remarks, that houses having fallen upon about ten (in another place them; but that the soldiers, he says thirty) thousand perbeing abroad and under tents, sons perished in this earthescaped free from every 

quake, principally from the harm. Ee 4 

424 BOOK V. B.C. 32. 

“rounded us with Athenio'; and how the great “and good God helped us against them, and 
“delivered us from them. Therefore fear God, 36 following your ancient custom, and the laudable 
 custom of your forefathers; and.prepare your
“selves against this enemy before he makes ready 
 against you, and be beforehand with him before “he anticipates you: and God will supply you 
“with aid and succour against your enemy.” 
So when the men had heard the address of He37 rod, they replied, that they were ready to undertake the expedition, and would make no delay. And he returned thanks to God and to them for it, 38 and ordered many sacrifices to be offered: he also ordered an army to be raised; and a great multitude was gathered from the tribe of Judah and Benjamin. And Herod marching against the king of 39 the Arabians, encountered him;; and the battle grew fierce between them, five thousand of the Arabians being slain. There was again a battle, 40 and four thousand of the Arabians were killed: wherefore the Arabians returned to their camp, and remained there; and Herod could do nothing against them, for the place was fortified; but he remained with his army, besieging them in the same place, and not allowing them to go out. And they remained five days in this condition; 41 and a most violent thirst came upon them; they sent therefore ambassadors to Herod with a most valuable present, asking for a truce, and liberty to draw water to drink: but he did not listen to them, but continued in the same furious hostility. 

f See above, verses 10, 15, 16. 

B.C. 31. CHAP. LVII. 425 

42 The Arabians then said therefore, Let us go out against this nation; for it is better for us to con43 quer or die, than to perish from thirst. And they went out against them; and Herod’s party overcame them, and slew nine thousand of them; and Herod with his men pursued the Arabians as they fled, slaying great numbers of them; and he 44 besieged their cities and took them. Wherefore they sued for their lives, promising obedience; to which he agreeing, retired from them, and returned into the Holy House. 
Now the Arabians mentioned in this book are the Arabians who dwelled from the country of Sarah as far as to Hegiaz and the adjacent parts; 5 and they were of great renown and large numbers. 

45 

CHAPTER LVII.@ 

The history of Antony's battle with Augustus, and of the death of Antony, and of Herod’s going to Augustus. 

— 

WHEN Antony had marched out of Egypt into the country of the Romans, and had encountered Augustus>, most severe battles 

B.C. 31. 

& This name is still preserved in Arabia; a large and important district, extending down the shore of the Red sea, and embracing the cities of Mecca and Medina, still bears the appellation of Hedjaz. It is likely that formerly there was a chief town, bearing nearly the same name. 
a Compare Joseph. Antiq. XV, 9, 10. Bell. I. 15. 
b Namely at Actium, a town on the sea-coast of Epirus, which was at this time 

part of the Roman dominions, where his fortunes were fatally shattered: but it is not true that he fell in battle, or at Actium. Antony livedtill the next year, and had retired into Egypt: when, after fruitless attempts at a reconciliation with his rival, he once more resolved to try the chances of war, and made ready for battle at Alexan_dria: but fortune again proved adverse to him; and on hearing a (false) report of 

426 BOOK V. B.C. 30. 

took place between them, in which victory sided with Augustus, and Antony fell in battle; and 2 Augustus got possession of his camp and all which was in it. After this done, he proceeded to Rhodes, that taking ship there he might pass into Egypt. 
And tidings were brought to Herod, and he was 3 
pc. very much concerned at the death of An
30. tony; and he feared Augustus most exceedingly; and he resolved to go to him, to salute him and congratulate with him. Wherefore he 4 
sent_his mother and sister with his brother, to a strong hold¢ which he had in mount Sarah: he sent also his wife Mariamne and her mother Alexandra to Alexandrium‘, under the care of Josephus a Tyrian; adjuring him to kill his wife and her mother, so soon as his death should be reported to him. 
After this, he went to Augustus with a very 5 valuable present. Now Augustus had already de6 termined to put Herod to death; because he had been the friend and supporter ‘of Antony, and because he had formerly deliberated‘ upon marching 

Cleopatra’s death, in despair he fell on his own sword. 
¢ Namely Massada, mentioned above, in the notes on ch. xlix. 20; ‘and 1/8. “as a place made use of by Herod for the same purpose on an occasion somewhat similar. 
d Which see described above, at ch. xxxix. 5. 
e Josephus reads, “to Jo“seph his steward, and Soe“mus an Iturean.” (Compare ch. lviii. 1.) In another place he calls him “the 

“husband of his sister Sa“lome,” agreeably to our author: see below, ch. lviii. 1. 
f See the preceding chapter, ver. 8,9. Herod, however, appears even in this instance not to have forgotten his usual crafty foresight: and in dividing the duties of the campaign with Antony, managed so as to avoid coming into direct collision with Augustus, and to employ himself rather in subduing the Arabians; that if at a future 

B.C. 30. CHAP. LVII. 427 

7 with Antony to attack him. When therefore Herod’s arrival was notified to Augustus, he ordered him into his presence, in his royal habit which he had on; except the diadem, for this he had 
8 ordered to be laid aside from his head. Who, when he was in his presence, having laid aside his diadem as Augustus had commanded, said: 
9 “O king, perhaps on account of my love towards « Antony you have been thus violently angry with “me, that you have put off the diadem from my 
10 “head; or was it from some other cause ? Since, “if you are wroth with me by reason of my ad“herence to Antony, truly, I say, I adhered to « him because he deserved well of me, and placed “upon my head that diadem which you have 11 “taken off. And indeed he had requested my as“sistance against you, which I gave him; even “as he also many times gave his assistance to me: 12 “but it was not my lot to be present at the battle “which he fought with you, nor have I drawn 
' “my sword’ against you, nor fought; the cause 
“of which was, my being engaged in subduing 13 “the Arabians. But I never failed supplying “him with aid of men and arms and provisions, “as his friendship and his good deeds to me re“quired. And in truth I am sorry that I left “him; lest men should conceive that I deserted “my friend when he was in need~of my help. 14 “Certainly, if I had been with him, I would have 

time the sun of Antony should & See the preceding note. set before the power and inh According to an arrangefluence of his aspiring rival, ment made with Antony, as the door of reconciliation related above, at ch. lvi. 9— might not be irretrievably 1. 
closed against him. 

428 BOOK V. B.C. 30. 

“helped him with all my might; and would have “encouraged him if he had been fearful, and “would have strengthened him if he had been ‘“‘ weakened, and would have lifted him up if he “had fallen, until God should have ruled matters “as He pleased. And this truly would have been 15 “less grievous to me, than that it should be ima“gined that I had failed a man who had implored “my aid, and thus it should come to pass that “‘ my friendship should be little esteemed. In-my 16 “opinion indeed he fell through his own bad po“licy, by yielding to that enchantress Cleopatra; “whom I had advised him to slay, and thus to “remove her malice from him; but he did not “assent. But now, if you have removed from 17 “my head the diadem, certainly you shall not re‘‘ move from me my understanding and my cou“rage; and whatever I.am, I will be a friend to “my friends and an enemy to my enemies.’ uae Augustus replied to him, “Antony indeed we 18 “have overcome by our troops; but you we will “master by alluring you to us; and will take ‘ care, by our good offices towards you, that your affection to us shall be doubled, because you are worthy of this. And as Antony played false by “the advice of Cleopatra, by the same reason he “behaved ungratefully towards us; returning for our kindnesses evils, and for our favours rebel“lion. But we are glad of the war which you 20 “have waged with the Arabians, who are our “enemies: for whoever is your enemy, is ours “also; and whoever pays you obedience, pays It “to us likewise.” Then Augustus ordered the golden diadem to 9] 

_ 

9 

B.C. 29. CHAP. LVIII. 429 

be placed on Herod’s head, and as many provinces to be added to him as he already had. 
22 And Herod accompanied Augustus into Egypt; and all the things which Antony had destined for Cleopatra were surrendered to him. And Augustus departed to Rome: but Herod returned into the Holy City. | 

CHAPTER LVIII.2 

The history of the murder which Herod committed on his wife Mariamne. 
1 Now Josephus?, the husband of Herod’s sister, had revealed to Mariamne that Herod had 3.¢. ordered him to put her and her mother to ” death, as soon as he himself should perish in his 
2 going up to Augustus. And she already had a dislike of Herod, since the time when he killed her father and brother; and to this no little addition of hatred was made, when she was informed of the orders which he had given against her. 
3 Therefore when Herod arrived out of Egypt, he found her totally overcome by hatred towards him: at which being greatly troubled, he tried to 
4 reconcile her to him by all possible methods. But his sister came on a certain day, after some quarrels which had taken place between her and Ma

h Augustus not only restored those portions of Judea which Antony had taken away and given to Cleopatra, but likewise enlarged Herod’s dominion by the gift of many other towns and _ districts, which the reader may see 

enumerated in Josephus. 
@ Compare Josephus, Ant. XV. 11. Bell. I. 17. 
b The person who was left in charge of Mariamne, and her mother, as related above, at ch. lvii. 4; and see the note there. 

430 BOOK V. B.C. 29. 

riamne, and said to him, Certainly Joseph my husband has gone aside with Mariamne. But 5 Herod paid no attention to her words, knowing how pure and chaste Mariamne was. After this, 6 Herod went to see Mariamne on the night which followed that day, and behaved kindly and affectionately towards her, recounting his love for her, saying much upon this head: to whom she said, 7 «“Did you ever see a man love another, and order “him to be put to death? and is he a hater un“less he shews such proofs?” Then Herod per8 ceived that Josephus had discovered to Mariamne the secret which he had entrusted to him; and believed that he would not have done that, unless she had given herself up to him: and he believed 9 that which his sister had told him on this subject; and immediately departing from Mariamne, he hated and detested her. 

Which his sister learning, went to the cup10 

bearer, and giving him money, delivered to him some poison, and said; Carry this to the king, and say to him, Mariamne the king’s wife gave me this poison, and this money, commanding that it might be mixed in the king’s drink. This the 1 cupbearer did. And the king seeing the poison, doubted not of the truth of the thing: whereupon he gives orders to behead Josephus his brother-inlaw immediately; and also orders Mariamne to be put in chains, until the_seventy elders should be present, and should pass a due sentence upon her. 
So Herod’s sister feared’, lest what she had 

— 

¢ It is observable, that this in some few circumstances account of Mariamne’s confrom that which is given by demnation and death differs Josephus. 

2 

B. C. 29. CHAP. LVIII. 431 

done should be discovered, and she herself should perish, if Mariamne were set free: so she said to him, O king, if you put off Mariamne’s death till to-morrow, you will not be at all able to effect it: 13 for as soon as it shall become known that you wish to kill her, the whole house of her father will come, and all their servants and neighbours, and will interpose; and you will not be able. to 14 obtain her death until after great tumults. And 15 Herod said, Do as it seems best to you. And Herod’s sister sent in all haste a man to bring out Mariamne to the place of slaughter, setting upon her her maids, and other women, to insult her4, and upbraid her with all manner of indecency: 16 but she answered nothing to any of them, nor even moved her head in the least: nor was her colour changed by all this treatment, nor did any fear or confusion appear in her, nor was her gait 17 altered; but with her wonted manner she proceeded to the place whither she was led to be slain; and bending her knees, she held out her 18 neck voluntarily: and departed this life, renowned for religion and chastity, marked by ‘no crime, branded with no guilt; howbeit she was not wholly free from haughtiness, according to the 

d It deserves remark, that the author of this book takes no notice whatsoever of the story which appears in Josephus, of Alexandra joining with the wicked Salome and her creatures, in their indecent revilings of her own daughter,;Mariamne. In truth, such an act would have been not only most unworthy of a mother, but also unlike the 

straight-forward conduct of that spirited, but unfortunate queen. 
e IT am not satisfied as to the correctness of this rendering: Gabriel’s Latin has ““apicem protulit; ” and the French version is here too loose ‘to afford any certain in

formation. Perhaps the sense 

may be, “she did not utter a “single syllable, or letter.” 

432 

habit of her family. And of this not the least cause 19 

BOOK V. 

BiCols. 

was the obsequious attention and affection of Herod towards her, by reason of the elegance of her form; from whence she suspected no change in 

him towards her. 

Now Herod had begotten_of her two sons‘, namely, Alexander and Aristobulus; who, when their mother was slain, were living at Rome; for he had sent them thither, to learn the litera

ture and language of the Romans. 

Herod repented that he had killed his wife; and he was affected with grief to that degree on account of her death, that by it he contracted a disease, of which he had nearly died. 

_Mariamne being dead, her mother Alexandra 22 laid plans to put Herod to death; which 

B.C. 

28. coming to his knowledge, he made away 

with her. 

\chapter{LIX}

The history of the coming 

of the two sons of Herod, 

Alexander and Ari era as soon as they heard that their mother had been put to death by Herod. 

WHEN news was brought to Alexander and 1] 

B.C. 16. 

come by excessive grief; 

Aristobulus of the murder committed on 

‘ad departing from 2 

Rome? they came into the Holy City, paying no 

f Josephus informs us that she bare him three sons, but that the youngest of them died while Uae Bo his studies at Rome,/ 
a ise se pie Antiq. AVES D2 Gee ee 

16, 17, Bell. I..17. 
b From the text of this verse it would appear that the brothers quitted Rome immediately after hearing of their mother’s death: but from Josephus we collect that 

Afterwards, 21 

B.C. 15. CHAP. LIX. 433 

respect to their father Herod as they had formerly been wont to do, through the hatred of him which they felt in their minds on account of their mo
3 ther’s death. Now Alexander had married the daughter of king Archelaus¢: and Aristobulus had married the daughter of Herod’s sister‘. 
4 Therefore when Herod perceived that they paid him no respect, he saw that he was hated by them, and he avoided them: and this did not escape the observation of the young men, and of his family. 
5 Now king Herod had married a wife before Mariamne, by name Dosithea, by whom » ¢ 
6 he had a son named Antipater. When }3: therefore Herod was assured respecting his two sons, as was observed above, he brought his wife Dosithea to his palace, and attached to himself his son Antipater, committing to him all his business; and he appointed him by will his successor. 
7 And that Antipater persecuted his brothers Alexander and Aristobulus, designing to procure peace to himself while his father lived, that after his 
8 death he might have no rival. Wherefore he said to his father, “In truth my brothers are “‘ seeking an inheritance’ because of the family of “their mother, because it is more noble than the “family of my mother; and. therefore they have “a better right than I have to the fortune of 

they did not return until see Josephus calls her Doris. veral years after; namely, in f That is, they are devising the year B.C. 16. means of securing to them. 
¢ He was king of Cappaselves the succession to the docia. (J osephus. ) throne, which they know you 

d Namely, Bernice, daughhave destined for me. ter of Joseph and Salome. an EF f 

43.4 BOOK V. B.C. 12. 

“‘ which the king has judged me worthy: for this 9 “cause they are striving to put you to death, and “me also they will slay soon after.” And this 10 he frequently repeated to Herod, sending also secretly to him persons to insinuate to him things which might produce in him a greater hatred towards them. In the mean time Herod goes to Rome to 11 z.c, Augustus, taking with him his son Alex12. ander. “And when he had come into Augustus’ presence, Herod complained to him of his son, requesting that he would reprove him. But Alexander said; “Indeed I do not deny my 12 ‘ anguish on account of the murder of my mother “without any fault; for even brute beasts them“selves shew affection to their mothers much “better than men, and love them more: but any 13 “design of parricide I utterly deny, and I clear “‘ myself of it before God: for I am possessed of “the same feelings toward my father as toward “my mother: nor am I that sort of man as to 14 “bring upon me guilt for crime towards my parent, and more especially eternal torments.” Alexander then wept with bitter and most vehe15 ment weeping; and Augustus pitied him, and all the chiefs of the Romans, who were standing near, wept also. Then Augustus asked Herod to 16 take back his sons into his former kindness and intimacy: and he desired Alexander to kiss his father’s feet, who did so. He also ordered Herod to embrace and kiss him, and Herod obeyed him. Afterwards Augustus ordered a magnificent 17 present for Herod, and it was carried to him: and after passing some days with him, Herod 

18 

19 

20 

21 

22 

23 

BCI, CHAP. LIX. 435 

returned to the Holy House; and calling to him the elders of Judah, he said: “Know ye. that “Antipater is my eldest son and firstborn, but “his mother is of an ignoble family: but the “mother of Alexander and Aristobulus my sons “is of the family of the high priests and kings. “Moreover, God hath enlarged my kingdom, and “hath extended my power; and therefore: it “seems good to me to appoint these my three “sons to equal authority; so that Antipater shall “have no command over his brothers, nor shall “his brothers have command over him. Obey “therefore all three, O ye assembly of men, nor “interfere in any thing which their minds may “be able to agree on; nor propose any thing “which may produce misleadings and disagree“ment among them. And do not drink with “them, nor talk too much with them. For from “thence it will come to pass, that some one of “them may unguardedly utter to you the designs “which he has against his brother: upon which, “that you may conciliate them to you, will follow “your agreement’ with every one of them, accord“ing to what seems good to him; and you will “bring them to destruction, and yourselves will “be destroyed also. It is your parts indeed, my “sons, to be obedient to God, and to me; that “you may live long, and that your affairs may prosper.” Soon afterwards he embraced and kissed them, and commanded the people to retire. 

& That is, by which means personal interest, instead of you will be led to become remaining faithful counsellors partisans of one individual or and supporters of their united the other, from motives of authority. 

Ff 2 

436 BOOK V. B.C. 9. 

But that which Herod did came to no happy result, nor were the hearts of his sons united in agreement. For Antipater wanted every thing to be put into his hands, as his father had formerly appointed: and to his brothers it did not seem at all fair that he should be thought equal to them. Now Antipater was endued with perseverance, and all bad and feigned friendship; but not so his two brothers: Antipater therefore set spies on his brothers, who should bring him tidings of them: he also planted others who should carry false reports of them to Pilate”. But when Antipater was in presence of the king, and heard any one relating such things of his brothers, he repelled the charge from them, declaring that the authors were unworthy of credit, and entreating the king not to believe the reports. Which Antipater did, that he might not inspire the king with any doubt or suspicion: of himself. From hence the king entertained no doubt that he was well-inclined towards his. brothers, and wished them no harm. 
Which when Antipater found out, he bent to B.c. his purpose Pheroras his uncle, and his aunt, 
(for these were at enmity with his brothers on 

24 

25 

26 

27 

28 

29 

their mother’s account,) offering Pheroras a most 

valuable present, requesting him to inform the king 

h This manifestly is a mistake of the author or copier, for Herod. Pilate does not appear in Jewish history till more than thirty years had elapsed after this transaction; namely, i in the reign of Tiberius Cesar. See Luke iii. 1A 
i The reader will have re

marked the just retaliation upon Herod by his crafty son; who now practises against his own father, and with equal success, that system of duplicity and false accusation which Herod and his father Antipater had ever employed for their own advancement. 

B.C. 9. CHAP. LIX. 437 

that Alexander and_Aristobulus had laid a plan 
30 to murder the king. (Now Herod was well in
clined towards Pheroras his brother, and at
tended to whatever he said; inasmuch as he paid every year to him a large sum out of the provinces which he governed on the bank of the 
Euphrates.) And this Pheroras did. Afterwards 
Antipater went to Herod, and said to him; “O 
“king, in ss my brothers have laid a plot to 
“destroy me.” Antipater moreover gave money 
to the king's s three eunuchs, that they should say, 
“Alexander has given us money, that he might 
“make a wicked use of us, and that we might 
 slay thee: and when we shrank from it, he 
“threatened us with death.” 
33 And the king was wroth with Alexander, and ordered him to be put in chains: and he seized and put to the torture all the servants of Alexander, till they should confess what they knew about 
34 Alexander’s plot for murdering | him. And many of these, though they died under the torture, never told a falsehood respecting Alexander: but some of them, being unable to endure the violence of the torment, devised falsehoods through a desire 
35 of liberating themselves; asserting that Alexander and Aristobulus had planned to attack the king, and slay him, and flee to Rome; and having received an army from Augustus, to march against the Holy House, to slay their brother Antipater, and to seize on the throne of Judza. 
36 And the king commanded Aristobulus to be seized and put in chains: and he was bound, and was placed with his brother. 
Ff3 

3 

— 

3 

bo 

438 BOOK V. B.C. 8. 

But when news of Alexander was brought to 37 pc, his father-in-law Archelaus, he went to 8... Herod, pretending to be in a great fury against Alexander: as if, on hearing a report of the 38 intended parricide, he had come on purpose to see whether his daughter, the wife of Alexander, was privy to the business, and had not revealed it to him, that he might put her to death: but that, if she was not privy to any thing of the kind, he might separate her from Alexander, and take her to his own home. 
Now this Archelaus was a prudent, wise, and 39 eloquent man. And when Herod had heard his words, and was satisfied of his prudence and honesty, he wonderfully got possession of his heart; and he trusted himself to him, and relied on him without the slightest hesitation. Archelaus there40 fore, finding Herod’s inclination towards him, after a long intimacy, said to him one day when they had retired together; “Truly, O king, by reflect41 “‘ ing on your affairs I have found, that you being now in advanced age are much in want of re
“‘ pose of mind, and to have solace in your sons; ‘ whereas on the contrary you have derived from “them grief and anxiety. Moreover I have 42 “thought respecting these your two sons, and I “do not find that you have been deficient in de“serving well of them; for you have promoted “them, and made them kings, and have left un“done nothing, which might drive them wickedly “to contrive your death, nor have they any cause “for entering on this business. But perhaps this 43 

k The king of Cappadocia, as mentioned above, at verse 3. 

B.C. 8. CHAP. LIX. 439 

“has come from some malicious person, who is desiring evil against you and them, or who “through envy or enmity has induced you to ab44“hor them. If therefore he has gained influence “over you, who are an old man, endued with “knowledge, information, and experience, chang“ing you from paternal mildness to cruelty and 45 “fury against your children; how much easier could he have wrought on them, who are young, “inexperienced, and unguarded, and with no “knowledge of men and their guiles, so that he “has gained from them that which he wished in 46 “this matter. Consider therefore your affairs, O king; and do not give ear to the words of “informers, nor do any thing hastily against your “children; and enquire who that is who has been “contriving evil against you and them.” 47. And the king replied to him; “Indeed the “thing is as you have mentioned: I wish that I “knew who has induced them to do this.” Archelaus answered, “This is your brother Phero“yas.” The king replied, “It may be so.” ny 4g After this, the king became greatly changed in his behaviour towards Pheroras: which Pheroras perceiving, was afraid of him; and coming to 49 Archelaus, said to him; “I perceive how that “the king is changed towards me; wherefore I “intreat you to reconcile his mind to me, remov«“ing the feelings which he cherishes in his heart 50 “against me.” To whom Archelaus replied; “I « will do it indeed, if you will promise to disclose “to the king the truth concerning the plots which « you have laid against Alexander and Aristobu“lus.” And to this he assented. Ff 4 

440 BOOK V. B.C. 8. 

And after a few days, Archelaus said to the 51 king; “O king, truly a man’s relatives are to “him as his own limbs: and as it is good for a “man, if any one of his limbs becomes affected by “some disease which befalls it, to restore it by “medicines, even although it may cause him re pain; and it is not good to cut it off, lest the 52 pain should be increased, the body be weak
“ened, and the limbs should fail; and thus from “the loss of that limb, he should feel the want of “many conveniences: but let him endure the 53 “pains of the medical treatment, that the limb “may become better, and may be healed, and his “body may return to its former perfectness and “strength. So is it meet for a man, so often as 54 “any one of his relatives is altered towards him, “from any abominable cause whatsoever, to re“concile him to himself; alluring him to civility 55 “and friendship, admitting his excuses, and dis“missing the charges against him: and that he “do not put him hastily to death, nor remove “him too long away from his presence. For the 56 relatives of a man are his supporters and assist“ants, and in them consists his honour and glory; ‘ and through them he obtains that which other““wise he would not be able to obtain. Pheroras 57 truly is the king’s brother, and the son of ‘his “father and of his mother: and he confesses his “fault, entreating the king to spare him, and to “dismiss from his mind his error.” And the king replied, “This I will do.” And he ordered Phe58 roras to come before him; who, when he was in the presence, said to him; “I have sinned now “in the sight of the great and good God, and to 

a 

5 

ie) 

60 

61 

62 

6 

Oo 

B.C. 8. CHAP. LIX. 44) 

“the king, devising mischiefs, and plans which “might injure the affairs of the king and his ‘ sons, by lying falsehoods. But that which in“duced me to act thus was, that the king took “away from me a certain woman, my concubine, “and separated her and me.” The king said to Archelaus, “I have now pardoned Pheroras, as “you requested me: for I find that you have “cured the disease which was in our affairs by “your soothing methods, even as an ingenious “physician heals the corruptions of a sick body. “Wherefore I entreat you to pardon Alexander, reconciling your daughter to her husband; for ‘ | regard her as my daughter, since I know that “‘ she is more prudent than he, and that she turns “him aside from many things by her prudence “and her exhortations. Wherefore I pray you “not to separate them and destroy him: for he “agrees with her, and obtains many advantages ‘‘ from her guidance.” But Archelaus answered, “My daughter is the king’s handmaid: but him “my soul hath lately detested, by reason of his “evil design. Let the king therefore permit me “to separate him from my daughter, whom the “king may unite to whomsoever of his servants 

64 “he pleases.” ‘Tio whom the king replied; “Do 

65 

“not go beyond my request; and let your daugh“tey remain with him, and do not contradict me.” And Archelaus said; “Surely I will do it; and “will not contradict the king in any thing which “he shall enjoin me.” 
Soon afterwards, Herod orders Alexander and Aristobulus to be loosed from their chains, and to come before him: who, when they were in his 

442 BOOK V. B.C. 6. 

presence, prostrated themselves before him, confessing their faults, excusing themselves, and begging for pardon and forgiveness. And he com66 manded them to stand up, and causing them ‘to come near him, he kissed_them, and ordered them to depart to their own homes, and to return the next day. And they came to eating and drinking with him, and he reinstated them in a place of greater honour. And to Archelaus he gave se67 venty talents and a golden couch, enjoining likewise all the chief men of his friends to offer valuable presents to Archelaus: and they did so. This being accomplished, Archelaus departed from 68 the city of the Holy House to his own country; whom Herod accompanied, and at length, having taken leave of him, returned to the Holy House. Nevertheless, Antipater did not leave off his 69 p.c. plots against his brothers, that he might 6make them odious. Now it happened that 70 a certain man” came to Herod, having some valuable and handsome articles, with which kings are usually won; these he presented to the king, who, 71 taking them from him, repaid him for them; and the man obtained a very high place in his affections, and having been taken into his retinue, enjoyed his confidence: this man was named_Eurycles, When therefore Antipater perceived that 72 this man had wholly engrossed his father’s favour, he offered him money, requesting that he would 

b Josephus informs us that turned to his own country, he was a Lacedemonian, by after having kindled fatal disname Eurycles, of a sordid cord in Herod’s family, the and treacherous disposition; Lacedemonians banished him so much so, that when he refrom the realm. 

BUCA. CHAP. LIX. 443 

dexterously insinuate to Herod, and maintain that his two sons Alexander and Aristobulus were planning to murder him; which the man _pro73 mised him to do. He soon afterwards went to Alexander, and became intimate and familiar with him to that degree, that he was known to be in his friendship, and it was made known to the 74 king that he was intimate with him. After this, he went aside with the king, and said to him; “Certainly you have this right over me, O king, “that nothing ought to prevent me from giving you good advice: and in truth I have a matter “which the king ought to know, and which I 75 “ought to unfold to you.” The king said to him, “What have you?” The man answered him, “I heard Alexander saying, ‘ Truly God hath “deferred vengeance on my father for the death “of my mother, of my grandfather, and of my ‘ relatives, without any crime, that it may take “place by my hand: and I hope that I shall take 76 “vengeance for them upon him.’ And now he “has agreed with some chiefs to attack you, and he wished to implicate me in the plans which he had formed: but I held it to be a crime, on “account of the king’s acts of kindness towards 77 “me, and his liberality. But my intention is to “admonish him well, and to report this to him, “for he has both eyes and understanding.” 78 And when the king had heard these words, he by no means set them at nought, but speedily be79 gan to make enquiry as to their truth: but he found out nothing on which he could rely, except a letter forged in the name of Alexander and Ari

444 BOOK V. B.C. 6. 

stobulus to the governor of a certain town. And 80 there was in the letter, “We wish to kill our fa
“ther, and to flee to you; wherefore prepare us “a place wherein we may remain until the people 
 assemble round us, and our affairs are settled.” And this indeed was confirmed to the king, and 81 appeared probable: wherefore he seized the governor of that city and put him to the torture, that he might confess what was inserted into that letter. Which this man denied, clearing himself 82 from the charge: nor was any thing proved against them in this matter, or in any thing else which the informer had charged upon them. But Herod 83 ordered them to be seized and bound with chains and fetters. Then he went to Tyre a, and from Tyre to Cesarea, carrying them with him in chains. And all the captains and all the soldiers 84 pitied them: but no one interceded for them with the king, lest he should admit that to be true of himself which the informer had asserted. 
Now there was in the army a certain old war85 rior who had a son in the service of Alexander. When therefore the old man saw the wretched condition of Herod’s two sons, he pitied their change of fortune marvellously, and cried out with as loud a voice as he was able, “Pity is “gone; goodness and piety have vanished away; 

¢ Of Alexandrium. (Joseing no defence to be made. phus.) Of course the unhappy youths 4 Josephus relates that Hewere condemned. He then rod brought his sons toa pubcarried them off to Tyre and lic mock trial at Berytus, himCaesarea. self accusing them in most ¢ Josephus recordshis name, violent language, and allowwhich was Tero (or Tiro). 

86 

B.C. 6. CHAP. LIX. 445 

“truth is removed out of the world.” Then he said to the king, “O thou merciless to thy chil“dren, enemy of thy friends, and friend to thy 

“enemies, receiving the words of informers and 

87 

88 

8 

No) 

90 

9 

— 

92 

93 

“of persons who wish no good to thee!” And the enemies of Alexander and Aristobulus ran up to him, and reproved him, and said to the king 5 “O king, it is not love towards you and towards “your sons which has induced this man to speak thus; but he has wished to babble out the hatred which he bare in his heart towards you, and to “speak ill of your counsel and administration, as “being a faithful adviser. And indeed some ob“servers have informed us of him, that he had “already covenanted with the king’s barber, to “slay him with the razor while he was shaving “him.” And the king ordered the old man, and his son, and the barber, to be seized; and the old man and the barber to be scourged with rods till they should confess. And they were beaten with rods most cruelly, and were subjected to various kinds of tortures; but they confessed nothing of those things which they had not done. When therefore the son of the old man saw the sad condition of his father, and the state to which he had come, he pitied him, and thought that he would be liberated, if he himself should confess that which was laid to his father, after receiving from the king a promise for his life. Wherefore he said to the king; “O king, give me security for my father and my“self, that I may tell you that which you are “seeking.” And the king said, “You may have “this.” To whom he said; “Alexander had al

“ready agreed with my father that he should kill 

446 BOOK V. B.C. 6. 

“you: but my father agreed with the barber, as “has been told you.” 
Then the king commanded that old man and 94 his son to be slain, and the barber. He likewise ordered both his sons Alexander and Aristobulus to be taken to Sebaste, and there to be slain and fixed on a eibbet: and they were taken, killed, and fixed on a gibbet. 
Now Alexander left two sons who survived 95 him, namely, Tyrcanes and Alexander, by the daughter of king Archelaus: and Aristobulus left three sons, namely, Aristobulus, Agrippa, and Herod. But the history of Herod’s son Antipater has already been described‘ in our former accounts. 

96 

f What are the “former “accounts” here spoken of, it is not easy to determine. The subsequent history of Antipater must be sought in 

the 17th book of the Antiqui-_ 

ties of Josephus... In truth, the conclusion of this tragedy, is quite in keeping with the former melancholy scenes of it: Antipater becoming at last afraid of his father, whose ferocious and indiscriminate massacres he had not only 

END OF 

witnessed, but had abetted for some time past, seeks means to destroy him by poison: but Herod detecting the plot, although tormented by a complication of diseases, and almost at death’s door, summons his last energies to order Antipater to be slain, which is instantly done. He himself follows his son to the grave within five days, one year after§ the {birth} of our Saviour Jesus Christ. 

BOOK V. 

INDE X. 


\end{document}