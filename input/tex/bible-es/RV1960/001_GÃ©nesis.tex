\begin{document}

\title{Génesis}

\chapter{1}

\section*{La creación}

\par 1 En el principio creó Dios los cielos y la tierra.
\par 2 Y la tierra estaba desordenada y vacía, y las tinieblas estaban sobre la faz del abismo, y el Espíritu de Dios se movía sobre la faz de las aguas.
\par 3 Y dijo Dios: Sea la luz; y fue la luz.
\par 4 Y vio Dios que la luz era buena; y separó Dios la luz de las tinieblas.
\par 5 Y llamó Dios a la luz Día, y a las tinieblas llamó Noche. Y fue la tarde y la mañana un día.
\par 6 Luego dijo Dios: Haya expansión en medio de las aguas, y separe las aguas de las aguas.
\par 7 E hizo Dios la expansión, y separó las aguas que estaban debajo de la expansión, de las aguas que estaban sobre la expansión. Y fue así.
\par 8 Y llamó Dios a la expansión Cielos. Y fue la tarde y la mañana el día segundo.
\par 9 Dijo también Dios: Júntense las aguas que están debajo de los cielos en un lugar, y descúbrase lo seco. Y fue así.
\par 10 Y llamó Dios a lo seco Tierra, y a la reunión de las aguas llamó Mares. Y vio Dios que era bueno.
\par 11 Después dijo Dios: Produzca la tierra hierba verde, hierba que dé semilla; árbol de fruto que dé fruto según su género, que su semilla esté en él, sobre la tierra. Y fue así.
\par 12 Produjo, pues, la tierra hierba verde, hierba que da semilla según su naturaleza, y árbol que da fruto, cuya semilla está en él, según su género. Y vio Dios que era bueno.
\par 13 Y fue la tarde y la mañana el día tercero.
\par 14 Dijo luego Dios: Haya lumbreras en la expansión de los cielos para separar el día de la noche; y sirvan de señales para las estaciones, para días y años,
\par 15 y sean por lumbreras en la expansión de los cielos para alumbrar sobre la tierra. Y fue así.
\par 16 E hizo Dios las dos grandes lumbreras; la lumbrera mayor para que señorease en el día, y la lumbrera menor para que señorease en la noche; hizo también las estrellas.
\par 17 Y las puso Dios en la expansión de los cielos para alumbrar sobre la tierra,
\par 18 y para señorear en el día y en la noche, y para separar la luz de las tinieblas. Y vio Dios que era bueno.
\par 19 Y fue la tarde y la mañana el día cuarto.
\par 20 Dijo Dios: Produzcan las aguas seres vivientes, y aves que vuelen sobre la tierra, en la abierta expansión de los cielos.
\par 21 Y creó Dios los grandes monstruos marinos, y todo ser viviente que se mueve, que las aguas produjeron según su género, y toda ave alada según su especie. Y vio Dios que era bueno.
\par 22 Y Dios los bendijo, diciendo: Fructificad y multiplicaos, y llenad las aguas en los mares, y multiplíquense las aves en la tierra.
\par 23 Y fue la tarde y la mañana el día quinto.
\par 24 Luego dijo Dios: Produzca la tierra seres vivientes según su género, bestias y serpientes y animales de la tierra según su especie. Y fue así.
\par 25 E hizo Dios animales de la tierra según su género, y ganado según su género, y todo animal que se arrastra sobre la tierra según su especie. Y vio Dios que era bueno.
\par 26 Entonces dijo Dios: Hagamos al hombre a nuestra imagen, conforme a nuestra semejanza; y señoree en los peces del mar, en las aves de los cielos, en las bestias, en toda la tierra, y en todo animal que se arrastra sobre la tierra.
\par 27 Y creó Dios al hombre a su imagen, a imagen de Dios lo creó; varón y hembra los creó.
\par 28 Y los bendijo Dios, y les dijo: Fructificad y multiplicaos; llenad la tierra, y sojuzgadla, y señoread en los peces del mar, en las aves de los cielos, y en todas las bestias que se mueven sobre la tierra.
\par 29 Y dijo Dios: He aquí que os he dado toda planta que da semilla, que está sobre toda la tierra, y todo árbol en que hay fruto y que da semilla; os serán para comer.
\par 30 Y a toda bestia de la tierra, y a todas las aves de los cielos, y a todo lo que se arrastra sobre la tierra, en que hay vida, toda planta verde les será para comer. Y fue así.
\par 31 Y vio Dios todo lo que había hecho, y he aquí que era bueno en gran manera. Y fue la tarde y la mañana el día sexto.

\chapter{2}

\par 1 Fueron, pues, acabados los cielos y la tierra, y todo el ejército de ellos.
\par 2 Y acabó Dios en el día séptimo la obra que hizo; y reposó el día séptimo de toda la obra que hizo.
\par 3 Y bendijo Dios al día séptimo, y lo santificó, porque en él reposó de toda la obra que había hecho en la creación.

\section*{El hombre en el huerto del Edén}

\par 4 Estos son los orígenes de los cielos y de la tierra cuando fueron creados, el día que Jehová Dios hizo la tierra y los cielos,
\par 5 y toda planta del campo antes que fuese en la tierra, y toda hierba del campo antes que naciese; porque Jehová Dios aún no había hecho llover sobre la tierra, ni había hombre para que labrase la tierra,
\par 6 sino que subía de la tierra un vapor, el cual regaba toda la faz de la tierra.
\par 7 Entonces Jehová Dios formó al hombre del polvo de la tierra, y sopló en su nariz aliento de vida, y fue el hombre un ser viviente.
\par 8 Y Jehová Dios plantó un huerto en Edén, al oriente; y puso allí al hombre que había formado.
\par 9 Y Jehová Dios hizo nacer de la tierra todo árbol delicioso a la vista, y bueno para comer; también el árbol de vida en medio del huerto, y el árbol de la ciencia del bien y del mal.
\par 10 Y salía de Edén un río para regar el huerto, y de allí se repartía en cuatro brazos.
\par 11 El nombre del uno era Pisón; éste es el que rodea toda la tierra de Havila, donde hay oro;
\par 12 y el oro de aquella tierra es bueno; hay allí también bedelio y ónice.
\par 13 El nombre del segundo río es Gihón; éste es el que rodea toda la tierra de Cus.
\par 14 Y el nombre del tercer río es Hidekel; éste es el que va al oriente de Asiria. Y el cuarto río es el Eufrates.
\par 15 Tomó, pues, Jehová Dios al hombre, y lo puso en el huerto de Edén, para que lo labrara y lo guardase.
\par 16 Y mandó Jehová Dios al hombre, diciendo: De todo árbol del huerto podrás comer;
\par 17 mas del árbol de la ciencia del bien y del mal no comerás; porque el día que de él comieres, ciertamente morirás.
\par 18 Y dijo Jehová Dios: No es bueno que el hombre esté solo; le haré ayuda idónea para él.
\par 19 Jehová Dios formó, pues, de la tierra toda bestia del campo, y toda ave de los cielos, y las trajo a Adán para que viese cómo las había de llamar; y todo lo que Adán llamó a los animales vivientes, ese es su nombre.
\par 20 Y puso Adán nombre a toda bestia y ave de los cielos y a todo ganado del campo; mas para Adán no se halló ayuda idónea para él.
\par 21 Entonces Jehová Dios hizo caer sueño profundo sobre Adán, y mientras éste dormía, tomó una de sus costillas, y cerró la carne en su lugar.
\par 22 Y de la costilla que Jehová Dios tomó del hombre, hizo una mujer, y la trajo al hombre.
\par 23 Dijo entonces Adán: Esto es ahora hueso de mis huesos y carne de mi carne; ésta será llamada Varona, porque del varón fue tomada.
\par 24 Por tanto, dejará el hombre a su padre y a su madre, y se unirá a su mujer, y serán una sola carne.
\par 25 Y estaban ambos desnudos, Adán y su mujer, y no se avergonzaban.

\chapter{3}

\section*{Desobediencia del hombre}

\par 1 Pero la serpiente era astuta, más que todos los animales del campo que Jehová Dios había hecho; la cual dijo a la mujer: ¿Conque Dios os ha dicho: No comáis de todo árbol del huerto?
\par 2 Y la mujer respondió a la serpiente: Del fruto de los árboles del huerto podemos comer;
\par 3 pero del fruto del árbol que está en medio del huerto dijo Dios: No comeréis de él, ni le tocaréis, para que no muráis.
\par 4 Entonces la serpiente dijo a la mujer: No moriréis;
\par 5 sino que sabe Dios que el día que comáis de él, serán abiertos vuestros ojos, y seréis como Dios, sabiendo el bien y el mal.
\par 6 Y vio la mujer que el árbol era bueno para comer, y que era agradable a los ojos, y árbol codiciable para alcanzar la sabiduría; y tomó de su fruto, y comió; y dio también a su marido, el cual comió así como ella.
\par 7 Entonces fueron abiertos los ojos de ambos, y conocieron que estaban desnudos; entonces cosieron hojas de higuera, y se hicieron delantales.
\par 8 Y oyeron la voz de Jehová Dios que se paseaba en el huerto, al aire del día; y el hombre y su mujer se escondieron de la presencia de Jehová Dios entre los árboles del huerto.
\par 9 Mas Jehová Dios llamó al hombre, y le dijo: ¿Dónde estás tú?
\par 10 Y él respondió: Oí tu voz en el huerto, y tuve miedo, porque estaba desnudo; y me escondí.
\par 11 Y Dios le dijo: ¿Quién te enseñó que estabas desnudo? ¿Has comido del árbol de que yo te mandé no comieses?
\par 12 Y el hombre respondió: La mujer que me diste por compañera me dio del árbol, y yo comí.
\par 13 Entonces Jehová Dios dijo a la mujer: ¿Qué es lo que has hecho? Y dijo la mujer: La serpiente me engañó, y comí.
\par 14 Y Jehová Dios dijo a la serpiente: Por cuanto esto hiciste, maldita serás entre todas las bestias y entre todos los animales del campo; sobre tu pecho andarás, y polvo comerás todos los días de tu vida.
\par 15 Y pondré enemistad entre ti y la mujer, y entre tu simiente y la simiente suya; ésta te herirá en la cabeza, y tú le herirás en el calcañar.
\par 16 A la mujer dijo: Multiplicaré en gran manera los dolores en tus preñeces; con dolor darás a luz los hijos; y tu deseo será para tu marido, y él se enseñoreará de ti.
\par 17 Y al hombre dijo: Por cuanto obedeciste a la voz de tu mujer, y comiste del árbol de que te mandé diciendo: No comerás de él; maldita será la tierra por tu causa; con dolor comerás de ella todos los días de tu vida.
\par 18 Espinos y cardos te producirá, y comerás plantas del campo.
\par 19 Con el sudor de tu rostro comerás el pan hasta que vuelvas a la tierra, porque de ella fuiste tomado; pues polvo eres, y al polvo volverás.
\par 20 Y llamó Adán el nombre de su mujer, Eva, por cuanto ella era madre de todos los vivientes.
\par 21 Y Jehová Dios hizo al hombre y a su mujer túnicas de pieles, y los vistió.
\par 22 Y dijo Jehová Dios: He aquí el hombre es como uno de nosotros, sabiendo el bien y el mal; ahora, pues, que no alargue su mano, y tome también del árbol de la vida, y coma, y viva para siempre.
\par 23 Y lo sacó Jehová del huerto del Edén, para que labrase la tierra de que fue tomado.
\par 24 Echó, pues, fuera al hombre, y puso al oriente del huerto de Edén querubines, y una espada encendida que se revolvía por todos lados, para guardar el camino del árbol de la vida.

\chapter{4}

\section*{Caín y Abel}

\par 1 Conoció Adán a su mujer Eva, la cual concibió y dio a luz a Caín, y dijo: Por voluntad de Jehová he adquirido varón.
\par 2 Después dio a luz a su hermano Abel. Y Abel fue pastor de ovejas, y Caín fue labrador de la tierra.
\par 3 Y aconteció andando el tiempo, que Caín trajo del fruto de la tierra una ofrenda a Jehová.
\par 4 Y Abel trajo también de los primogénitos de sus ovejas, de lo más gordo de ellas. Y miró Jehová con agrado a Abel y a su ofrenda;
\par 5 pero no miró con agrado a Caín y a la ofrenda suya. Y se ensañó Caín en gran manera, y decayó su semblante.
\par 6 Entonces Jehová dijo a Caín: ¿Por qué te has ensañado, y por qué ha decaído tu semblante?
\par 7 Si bien hicieres, ¿no serás enaltecido? y si no hicieres bien, el pecado está a la puerta; con todo esto, a ti será su deseo, y tú te enseñorearás de él.
\par 8 Y dijo Caín a su hermano Abel: Salgamos al campo. Y aconteció que estando ellos en el campo, Caín se levantó contra su hermano Abel, y lo mató.
\par 9 Y Jehová dijo a Caín: ¿Dónde está Abel tu hermano? Y él respondió: No sé. ¿Soy yo acaso guarda de mi hermano?
\par 10 Y él le dijo: ¿Qué has hecho? La voz de la sangre de tu hermano clama a mí desde la tierra.
\par 11 Ahora, pues, maldito seas tú de la tierra, que abrió su boca para recibir de tu mano la sangre de tu hermano.
\par 12 Cuando labres la tierra, no te volverá a dar su fuerza; errante y extranjero serás en la tierra.
\par 13 Y dijo Caín a Jehová: Grande es mi castigo para ser soportado.
\par 14 He aquí me echas hoy de la tierra, y de tu presencia me esconderé, y seré errante y extranjero en la tierra; y sucederá que cualquiera que me hallare, me matará.
\par 15 Y le respondió Jehová: Ciertamente cualquiera que matare a Caín, siete veces será castigado. Entonces Jehová puso señal en Caín, para que no lo matase cualquiera que le hallara.
\par 16 Salió, pues, Caín de delante de Jehová, y habitó en tierra de Nod, al oriente de Edén.
\par 17 Y conoció Caín a su mujer, la cual concibió y dio a luz a Enoc; y edificó una ciudad, y llamó el nombre de la ciudad del nombre de su hijo, Enoc.
\par 18 Y a Enoc le nació Irad, e Irad engendró a Mehujael, y Mehujael engendró a Metusael, y Metusael engendró a Lamec.
\par 19 Y Lamec tomó para sí dos mujeres; el nombre de la una fue Ada, y el nombre de la otra, Zila.
\par 20 Y Ada dio a luz a Jabal, el cual fue padre de los que habitan en tiendas y crían ganados.
\par 21 Y el nombre de su hermano fue Jubal, el cual fue padre de todos los que tocan arpa y flauta.
\par 22 Y Zila también dio a luz a Tubal-caín, artífice de toda obra de bronce y de hierro; y la hermana de Tubal-caín fue Naama.
\par 23 Y dijo Lamec a sus mujeres:
\par Ada y Zila, oíd mi voz;
\par Mujeres de Lamec, escuchad mi dicho:
\par Que un varón mataré por mi herida,
\par Y un joven por mi golpe.
\par 24 Si siete veces será vengado Caín,
\par Lamec en verdad setenta veces siete lo será.
\par 25 Y conoció de nuevo Adán a su mujer, la cual dio a luz un hijo, y llamó su nombre Set: Porque Dios (dijo ella) me ha sustituido otro hijo en lugar de Abel, a quien mató Caín.
\par 26 Y a Set también le nació un hijo, y llamó su nombre Enós. Entonces los hombres comenzaron a invocar el nombre de Jehová.

\chapter{5}

\section*{Los descendientes de Adán}

\par 1 Este es el libro de las generaciones de Adán. El día en que creó Dios al hombre, a semejanza de Dios lo hizo.
\par 2 Varón y hembra los creó; y los bendijo, y llamó el nombre de ellos Adán, el día en que fueron creados.
\par 3 Y vivió Adán ciento treinta años, y engendró un hijo a su semejanza, conforme a su imagen, y llamó su nombre Set.
\par 4 Y fueron los días de Adán después que engendró a Set, ochocientos años, y engendró hijos e hijas.
\par 5 Y fueron todos los días que vivió Adán novecientos treinta años; y murió.
\par 6 Vivió Set ciento cinco años, y engendró a Enós.
\par 7 Y vivió Set, después que engendró a Enós, ochocientos siete años, y engendró hijos e hijas.
\par 8 Y fueron todos los días de Set novecientos doce años; y murió.
\par 9 Vivió Enós noventa años, y engendró a Cainán.
\par 10 Y vivió Enós, después que engendró a Cainán, ochocientos quince años, y engendró hijos e hijas.
\par 11 Y fueron todos los días de Enós novecientos cinco años; y murió.
\par 12 Vivió Cainán setenta años, y engendró a Mahalaleel.
\par 13 Y vivió Cainán, después que engendró a Mahalaleel, ochocientos cuarenta años, y engendró hijos e hijas.
\par 14 Y fueron todos los días de Cainán novecientos diez años; y murió.
\par 15 Vivió Mahalaleel sesenta y cinco años, y engendró a Jared.
\par 16 Y vivió Mahalaleel, después que engendró a Jared, ochocientos treinta años, y engendró hijos e hijas.
\par 17 Y fueron todos los días de Mahalaleel ochocientos noventa y cinco años; y murió.
\par 18 Vivió Jared ciento sesenta y dos años, y engendró a Enoc.
\par 19 Y vivió Jared, después que engendró a Enoc, ochocientos años, y engendró hijos e hijas.
\par 20 Y fueron todos los días de Jared novecientos sesenta y dos años; y murió.
\par 21 Vivió Enoc sesenta y cinco años, y engendró a Matusalén.
\par 22 Y caminó Enoc con Dios, después que engendró a Matusalén, trescientos años, y engendró hijos e hijas.
\par 23 Y fueron todos los días de Enoc trescientos sesenta y cinco años.
\par 24 Caminó, pues, Enoc con Dios, y desapareció, porque le llevó Dios.
\par 25 Vivió Matusalén ciento ochenta y siete años, y engendró a Lamec.
\par 26 Y vivió Matusalén, después que engendró a Lamec, setecientos ochenta y dos años, y engendró hijos e hijas.
\par 27 Fueron, pues, todos los días de Matusalén novecientos sesenta y nueve años; y murió.
\par 28 Vivió Lamec ciento ochenta y dos años, y engendró un hijo;
\par 29 y llamó su nombre Noé, diciendo: Este nos aliviará de nuestras obras y del trabajo de nuestras manos, a causa de la tierra que Jehová maldijo.
\par 30 Y vivió Lamec, después que engendró a Noé, quinientos noventa y cinco años, y engendró hijos e hijas.
\par 31 Y fueron todos los días de Lamec setecientos setenta y siete años; y murió.
\par 32 Y siendo Noé de quinientos años, engendró a Sem, a Cam y a Jafet.

\chapter{6}

\section*{La maldad de los hombres}

\par 1 Aconteció que cuando comenzaron los hombres a multiplicarse sobre la faz de la tierra, y les nacieron hijas,
\par 2 que viendo los hijos de Dios que las hijas de los hombres eran hermosas, tomaron para sí mujeres, escogiendo entre todas.
\par 3 Y dijo Jehová: No contenderá mi espíritu con el hombre para siempre, porque ciertamente él es carne; mas serán sus días ciento veinte años.
\par 4 Había gigantes en la tierra en aquellos días, y también después que se llegaron los hijos de Dios a las hijas de los hombres, y les engendraron hijos. Estos fueron los valientes que desde la antigüedad fueron varones de renombre.
\par 5 Y vio Jehová que la maldad de los hombres era mucha en la tierra, y que todo designio de los pensamientos del corazón de ellos era de continuo solamente el mal.
\par 6 Y se arrepintió Jehová de haber hecho hombre en la tierra, y le dolió en su corazón.
\par 7 Y dijo Jehová: Raeré de sobre la faz de la tierra a los hombres que he creado, desde el hombre hasta la bestia, y hasta el reptil y las aves del cielo; pues me arrepiento de haberlos hecho.
\par 8 Pero Noé halló gracia ante los ojos de Jehová.

\section*{Noé construye el arca}

\par 9 Estas son las generaciones de Noé: Noé, varón justo, era perfecto en sus generaciones; con Dios caminó Noé.
\par 10 Y engendró Noé tres hijos: a Sem, a Cam y a Jafet.
\par 11 Y se corrompió la tierra delante de Dios, y estaba la tierra llena de violencia.
\par 12 Y miró Dios la tierra, y he aquí que estaba corrompida; porque toda carne había corrompido su camino sobre la tierra.
\par 13 Dijo, pues, Dios a Noé: He decidido el fin de todo ser, porque la tierra está llena de violencia a causa de ellos; y he aquí que yo los destruiré con la tierra.
\par 14 Hazte un arca de madera de gofer; harás aposentos en el arca, y la calafatearás con brea por dentro y por fuera.
\par 15 Y de esta manera la harás: de trescientos codos la longitud del arca, de cincuenta codos su anchura, y de treinta codos su altura.
\par 16 Una ventana harás al arca, y la acabarás a un codo de elevación por la parte de arriba; y pondrás la puerta del arca a su lado; y le harás piso bajo, segundo y tercero.
\par 17 Y he aquí que yo traigo un diluvio de aguas sobre la tierra, para destruir toda carne en que haya espíritu de vida debajo del cielo; todo lo que hay en la tierra morirá.
\par 18 Mas estableceré mi pacto contigo, y entrarás en el arca tú, tus hijos, tu mujer, y las mujeres de tus hijos contigo.
\par 19 Y de todo lo que vive, de toda carne, dos de cada especie meterás en el arca, para que tengan vida contigo; macho y hembra serán.
\par 20 De las aves según su especie, y de las bestias según su especie, de todo reptil de la tierra según su especie, dos de cada especie entrarán contigo, para que tengan vida.
\par 21 Y toma contigo de todo alimento que se come, y almacénalo, y servirá de sustento para ti y para ellos.
\par 22 Y lo hizo así Noé; hizo conforme a todo lo que Dios le mandó.

\chapter{7}

\section*{El diluvio}

\par 1 Dijo luego Jehová a Noé: Entra tú y toda tu casa en el arca; porque a ti he visto justo delante de mí en esta generación.
\par 2 De todo animal limpio tomarás siete parejas, macho y su hembra; mas de los animales que no son limpios, una pareja, el macho y su hembra.
\par 3 También de las aves de los cielos, siete parejas, macho y hembra, para conservar viva la especie sobre la faz de la tierra.
\par 4 Porque pasados aún siete días, yo haré llover sobre la tierra cuarenta días y cuarenta noches; y raeré de sobre la faz de la tierra a todo ser viviente que hice.
\par 5 E hizo Noé conforme a todo lo que le mandó Jehová.
\par 6 Era Noé de seiscientos años cuando el diluvio de las aguas vino sobre la tierra.
\par 7 Y por causa de las aguas del diluvio entró Noé al arca, y con él sus hijos, su mujer, y las mujeres de sus hijos.
\par 8 De los animales limpios, y de los animales que no eran limpios, y de las aves, y de todo lo que se arrastra sobre la tierra,
\par 9 de dos en dos entraron con Noé en el arca; macho y hembra, como mandó Dios a Noé.
\par 10 Y sucedió que al séptimo día las aguas del diluvio vinieron sobre la tierra.
\par 11 El año seiscientos de la vida de Noé, en el mes segundo, a los diecisiete días del mes, aquel día fueron rotas todas las fuentes del grande abismo, y las cataratas de los cielos fueron abiertas,
\par 12 y hubo lluvia sobre la tierra cuarenta días y cuarenta noches.
\par 13 En este mismo día entraron Noé, y Sem, Cam y Jafet hijos de Noé, la mujer de Noé, y las tres mujeres de sus hijos, con él en el arca;
\par 14 ellos, y todos los animales silvestres según sus especies, y todos los animales domesticados según sus especies, y todo reptil que se arrastra sobre la tierra según su especie, y toda ave según su especie, y todo pájaro de toda especie.
\par 15 Vinieron, pues, con Noé al arca, de dos en dos de toda carne en que había espíritu de vida.
\par 16 Y los que vinieron, macho y hembra de toda carne vinieron, como le había mandado Dios; y Jehová le cerró la puerta.
\par 17 Y fue el diluvio cuarenta días sobre la tierra; y las aguas crecieron, y alzaron el arca, y se elevó sobre la tierra.
\par 18 Y subieron las aguas y crecieron en gran manera sobre la tierra; y flotaba el arca sobre la superficie de las aguas.
\par 19 Y las aguas subieron mucho sobre la tierra; y todos los montes altos que había debajo de todos los cielos, fueron cubiertos.
\par 20 Quince codos más alto subieron las aguas, después que fueron cubiertos los montes.
\par 21 Y murió toda carne que se mueve sobre la tierra, así de aves como de ganado y de bestias, y de todo reptil que se arrastra sobre la tierra, y todo hombre.
\par 22 Todo lo que tenía aliento de espíritu de vida en sus narices, todo lo que había en la tierra, murió.
\par 23 Así fue destruido todo ser que vivía sobre la faz de la tierra, desde el hombre hasta la bestia, los reptiles, y las aves del cielo; y fueron raídos de la tierra, y quedó solamente Noé, y los que con él estaban en el arca.
\par 24 Y prevalecieron las aguas sobre la tierra ciento cincuenta días.

\chapter{8}

\par 1 Y se acordó Dios de Noé, y de todos los animales, y de todas las bestias que estaban con él en el arca; e hizo pasar Dios un viento sobre la tierra, y disminuyeron las aguas.
\par 2 Y se cerraron las fuentes del abismo y las cataratas de los cielos; y la lluvia de los cielos fue detenida.
\par 3 Y las aguas decrecían gradualmente de sobre la tierra; y se retiraron las aguas al cabo de ciento cincuenta días.
\par 4 Y reposó el arca en el mes séptimo, a los diecisiete días del mes, sobre los montes de Ararat.
\par 5 Y las aguas fueron decreciendo hasta el mes décimo; en el décimo, al primero del mes, se descubrieron las cimas de los montes.
\par 6 Sucedió que al cabo de cuarenta días abrió Noé la ventana del arca que había hecho,
\par 7 y envió un cuervo, el cual salió, y estuvo yendo y volviendo hasta que las aguas se secaron sobre la tierra.
\par 8 Envió también de sí una paloma, para ver si las aguas se habían retirado de sobre la faz de la tierra.
\par 9 Y no halló la paloma donde sentar la planta de su pie, y volvió a él al arca, porque las aguas estaban aún sobre la faz de toda la tierra. Entonces él extendió su mano, y tomándola, la hizo entrar consigo en el arca.
\par 10 Esperó aún otros siete días, y volvió a enviar la paloma fuera del arca.
\par 11 Y la paloma volvió a él a la hora de la tarde; y he aquí que traía una hoja de olivo en el pico; y entendió Noé que las aguas se habían retirado de sobre la tierra.
\par 12 Y esperó aún otros siete días, y envió la paloma, la cual no volvió ya más a él.
\par 13 Y sucedió que en el año seiscientos uno de Noé, en el mes primero, el día primero del mes, las aguas se secaron sobre la tierra; y quitó Noé la cubierta del arca, y miró, y he aquí que la faz de la tierra estaba seca.
\par 14 Y en el mes segundo, a los veintisiete días del mes, se secó la tierra.
\par 15 Entonces habló Dios a Noé, diciendo:
\par 16 Sal del arca tú, y tu mujer, y tus hijos, y las mujeres de tus hijos contigo.
\par 17 Todos los animales que están contigo de toda carne, de aves y de bestias y de todo reptil que se arrastra sobre la tierra, sacarás contigo; y vayan por la tierra, y fructifiquen y multiplíquense sobre la tierra.
\par 18 Entonces salió Noé, y sus hijos, su mujer, y las mujeres de sus hijos con él.
\par 19 Todos los animales, y todo reptil y toda ave, todo lo que se mueve sobre la tierra según sus especies, salieron del arca.
\par 20 Y edificó Noé un altar a Jehová, y tomó de todo animal limpio y de toda ave limpia, y ofreció holocausto en el altar.
\par 21 Y percibió Jehová olor grato; y dijo Jehová en su corazón: No volveré más a maldecir la tierra por causa del hombre; porque el intento del corazón del hombre es malo desde su juventud; ni volveré más a destruir todo ser viviente, como he hecho.
\par 22 Mientras la tierra permanezca, no cesarán la sementera y la siega, el frío y el calor, el verano y el invierno, y el día y la noche.

\chapter{9}

\section*{Pacto de Dios con Noé}

\par 1 Bendijo Dios a Noé y a sus hijos, y les dijo: Fructificad y multiplicaos, y llenad la tierra.
\par 2 El temor y el miedo de vosotros estarán sobre todo animal de la tierra, y sobre toda ave de los cielos, en todo lo que se mueva sobre la tierra, y en todos los peces del mar; en vuestra mano son entregados.
\par 3 Todo lo que se mueve y vive, os será para mantenimiento: así como las legumbres y plantas verdes, os lo he dado todo.
\par 4 Pero carne con su vida, que es su sangre, no comeréis.
\par 5 Porque ciertamente demandaré la sangre de vuestras vidas; de mano de todo animal la demandaré, y de mano del hombre; de mano del varón su hermano demandaré la vida del hombre.
\par 6 El que derramare sangre de hombre, por el hombre su sangre será derramada; porque a imagen de Dios es hecho el hombre.
\par 7 Mas vosotros fructificad y multiplicaos; procread abundantemente en la tierra, y multiplicaos en ella.
\par 8 Y habló Dios a Noé y a sus hijos con él, diciendo:
\par 9 He aquí que yo establezco mi pacto con vosotros, y con vuestros descendientes después de vosotros;
\par 10 y con todo ser viviente que está con vosotros; aves, animales y toda bestia de la tierra que está con vosotros, desde todos los que salieron del arca hasta todo animal de la tierra.
\par 11 Estableceré mi pacto con vosotros, y no exterminaré ya más toda carne con aguas de diluvio, ni habrá más diluvio para destruir la tierra.
\par 12 Y dijo Dios: Esta es la señal del pacto que yo establezco entre mí y vosotros y todo ser viviente que está con vosotros, por siglos perpetuos:
\par 13 Mi arco he puesto en las nubes, el cual será por señal del pacto entre mí y la tierra.
\par 14 Y sucederá que cuando haga venir nubes sobre la tierra, se dejará ver entonces mi arco en las nubes.
\par 15 Y me acordaré del pacto mío, que hay entre mí y vosotros y todo ser viviente de toda carne; y no habrá más diluvio de aguas para destruir toda carne.
\par 16 Estará el arco en las nubes, y lo veré, y me acordaré del pacto perpetuo entre Dios y todo ser viviente, con toda carne que hay sobre la tierra.
\par 17 Dijo, pues, Dios a Noé: Esta es la señal del pacto que he establecido entre mí y toda carne que está sobre la tierra.

\section*{Embriaguez de Noé}

\par 18 Los hijos de Noé que salieron del arca fueron Sem, Cam y Jafet. Cam es el padre de Canaán. 19 Estos tres fueron los hijos de Noé, y de ellos se pobló toda la tierra.
\par 20 Después comenzó Noé a labrar la tierra y plantó una viña.
\par 21 Bebió el vino, se embriagó y se desnudó en medio de su tienda.
\par 22 Cam, padre de Canaán, vio la desnudez de su padre y lo dijo a sus dos hermanos que estaban fuera.
\par 23 Entonces Sem y Jafet tomaron la ropa, la pusieron sobre sus propios hombros, y andando hacia atrás cubrieron la desnudez de su padre. Al tener vueltos sus rostros, no vieron la desnudez de su padre.
\par 24 Cuando despertó Noé de su embriaguez y supo lo que le había hecho su hijo más joven,
\par 25 dijo:
\par «¡Maldito sea Canaán!
\par ¡Siervo de siervos será a sus hermanos!»
\par 26 Y añadió:
\par «¡Bendiga Jehová, mi Dios, a Sem
\par y sea Canaán su siervo!
\par 27 ¡Engrandezca Dios a Jafet,
\par que habite en las tiendas de Sem
\par y sea Canaán su siervo!»
\par 28 Después del diluvio, Noé vivió trescientos cincuenta años. 29 Todos los días de Noé fueron novecientos cincuenta años, y murió.

\chapter{10}

\section*{Los descendientes de los hijos de Noé}

\par 1 Estas son las generaciones de los hijos de Noé: Sem, Cam y Jafet, a quienes nacieron hijos después del diluvio.
\par 2 Los hijos de Jafet: Gomer, Magog, Madai, Javán, Tubal, Mesec y Tiras.
\par 3 Los hijos de Gomer: Askenaz, Rifat y Togarma.
\par 4 Los hijos de Javán: Elisa, Tarsis, Quitim y Dodanim.
\par 5 De éstos se poblaron las costas, cada cual según su lengua, conforme a sus familias en sus naciones.
\par 6 Los hijos de Cam: Cus, Mizraim, Fut y Canaán.
\par 7 Y los hijos de Cus: Seba, Havila, Sabta, Raama y Sabteca. Y los hijos de Raama: Seba y Dedán.
\par 8 Y Cus engendró a Nimrod, quien llegó a ser el primer poderoso en la tierra.
\par 9 Este fue vigoroso cazador delante de Jehová; por lo cual se dice: Así como Nimrod, vigoroso cazador delante de Jehová.
\par 10 Y fue el comienzo de su reino Babel, Erec, Acad y Calne, en la tierra de Sinar.
\par 11 De esta tierra salió para Asiria, y edificó Nínive, Rehobot, Cala,
\par 12 y Resén entre Nínive y Cala, la cual es ciudad grande.
\par 13 Mizraim engendró a Ludim, a Anamim, a Lehabim, a Naftuhim,
\par 14 a Patrusim, a Casluhim, de donde salieron los filisteos, y a Caftorim.
\par 15 Y Canaán engendró a Sidón su primogénito, a Het,
\par 16 al jebuseo, al amorreo, al gergeseo,
\par 17 al heveo, al araceo, al sineo,
\par 18 al arvadeo, al zemareo y al hamateo; y después se dispersaron las familias de los cananeos.
\par 19 Y fue el territorio de los cananeos desde Sidón, en dirección a Gerar, hasta Gaza; y en dirección de Sodoma, Gomorra, Adma y Zeboim, hasta Lasa.
\par 20 Estos son los hijos de Cam por sus familias, por sus lenguas, en sus tierras, en sus naciones.
\par 21 También le nacieron hijos a Sem, padre de todos los hijos de Heber, y hermano mayor de Jafet.
\par 22 Los hijos de Sem fueron Elam, Asur, Arfaxad, Lud y Aram.
\par 23 Y los hijos de Aram: Uz, Hul, Geter y Mas.
\par 24 Arfaxad engendró a Sala, y Sala engendró a Heber.
\par 25 Y a Heber nacieron dos hijos: el nombre del uno fue Peleg, porque en sus días fue repartida la tierra; y el nombre de su hermano, Joctán.
\par 26 Y Joctán engendró a Almodad, Selef, Hazar-mavet, Jera,
\par 27 Adoram, Uzal, Dicla,
\par 28 Obal, Abimael, Seba,
\par 29 Ofir, Havila y Jobab; todos estos fueron hijos de Joctán.
\par 30 Y la tierra en que habitaron fue desde Mesa en dirección de Sefar, hasta la región montañosa del oriente.
\par 31 Estos fueron los hijos de Sem por sus familias, por sus lenguas, en sus tierras, en sus naciones.
\par 32 Estas son las familias de los hijos de Noé por sus descendencias, en sus naciones; y de éstos se esparcieron las naciones en la tierra después del diluvio.

\chapter{11}

\section*{La torre de Babel}

\par 1 Tenía entonces toda la tierra una sola lengua y unas mismas palabras.
\par 2 Y aconteció que cuando salieron de oriente, hallaron una llanura en la tierra de Sinar, y se establecieron allí.
\par 3 Y se dijeron unos a otros: Vamos, hagamos ladrillo y cozámoslo con fuego. Y les sirvió el ladrillo en lugar de piedra, y el asfalto en lugar de mezcla.
\par 4 Y dijeron: Vamos, edifiquémonos una ciudad y una torre, cuya cúspide llegue al cielo; y hagámonos un nombre, por si fuéremos esparcidos sobre la faz de toda la tierra.
\par 5 Y descendió Jehová para ver la ciudad y la torre que edificaban los hijos de los hombres.
\par 6 Y dijo Jehová: He aquí el pueblo es uno, y todos éstos tienen un solo lenguaje; y han comenzado la obra, y nada les hará desistir ahora de lo que han pensado hacer.
\par 7 Ahora, pues, descendamos, y confundamos allí su lengua, para que ninguno entienda el habla de su compañero.
\par 8 Así los esparció Jehová desde allí sobre la faz de toda la tierra, y dejaron de edificar la ciudad.
\par 9 Por esto fue llamado el nombre de ella Babel, porque allí confundió Jehová el lenguaje de toda la tierra, y desde allí los esparció sobre la faz de toda la tierra.

\section*{Descendientes de Sem}

\par 10 Estas son las generaciones de Sem: Sem, de edad de cien años, engendró a Arfaxad, dos años después del diluvio.
\par 11 Y vivió Sem, después que engendró a Arfaxad, quinientos años, y engendró hijos e hijas.
\par 12 Arfaxad vivió treinta y cinco años, y engendró a Sala.
\par 13 Y vivió Arfaxad, después que engendró a Sala, cuatrocientos tres años, y engendró hijos e hijas.
\par 14 Sala vivió treinta años, y engendró a Heber.
\par 15 Y vivió Sala, después que engendró a Heber, cuatrocientos tres años, y engendró hijos e hijas.
\par 16 Heber vivió treinta y cuatro años, y engendró a Peleg.
\par 17 Y vivió Heber, después que engendró a Peleg, cuatrocientos treinta años, y engendró hijos e hijas.
\par 18 Peleg vivió treinta años, y engendró a Reu.
\par 19 Y vivió Peleg, después que engendró a Reu, doscientos nueve años, y engendró hijos e hijas.
\par 20 Reu vivió treinta y dos años, y engendró a Serug.
\par 21 Y vivió Reu, después que engendró a Serug, doscientos siete años, y engendró hijos e hijas.
\par 22 Serug vivió treinta años, y engendró a Nacor.
\par 23 Y vivió Serug, después que engendró a Nacor, doscientos años, y engendró hijos e hijas.
\par 24 Nacor vivió veintinueve años, y engendró a Taré.
\par 25 Y vivió Nacor, después que engendró a Taré, ciento diecinueve años, y engendró hijos e hijas.
\par 26 Taré vivió setenta años, y engendró a Abram, a Nacor y a Harán.

\section*{Los descendientes de Taré}

\par 27 Estas son las generaciones de Taré: Taré engendró a Abram, a Nacor y a Harán; y Harán engendró a Lot.
\par 28 Y murió Harán antes que su padre Taré en la tierra de su nacimiento, en Ur de los caldeos.
\par 29 Y tomaron Abram y Nacor para sí mujeres; el nombre de la mujer de Abram era Sarai, y el nombre de la mujer de Nacor, Milca, hija de Harán, padre de Milca y de Isca.
\par 30 Mas Sarai era estéril, y no tenía hijo.
\par 31 Y tomó Taré a Abram su hijo, y a Lot hijo de Harán, hijo de su hijo, y a Sarai su nuera, mujer de Abram su hijo, y salió con ellos de Ur de los caldeos, para ir a la tierra de Canaán; y vinieron hasta Harán, y se quedaron allí.
\par 32 Y fueron los días de Taré doscientos cinco años; y murió Taré en Harán.

\chapter{12}

\section*{Dios llama a Abram}

\par 1 Pero Jehová había dicho a Abram: Vete de tu tierra y de tu parentela, y de la casa de tu padre, a la tierra que te mostraré.
\par 2 Y haré de ti una nación grande, y te bendeciré, y engrandeceré tu nombre, y serás bendición.
\par 3 Bendeciré a los que te bendijeren, y a los que te maldijeren maldeciré; y serán benditas en ti todas las familias de la tierra.
\par 4 Y se fue Abram, como Jehová le dijo; y Lot fue con él. Y era Abram de edad de setenta y cinco años cuando salió de Harán.
\par 5 Tomó, pues, Abram a Sarai su mujer, y a Lot hijo de su hermano, y todos sus bienes que habían ganado y las personas que habían adquirido en Harán, y salieron para ir a tierra de Canaán; y a tierra de Canaán llegaron.
\par 6 Y pasó Abram por aquella tierra hasta el lugar de Siquem, hasta el encino de More; y el cananeo estaba entonces en la tierra.
\par 7 Y apareció Jehová a Abram, y le dijo: A tu descendencia daré esta tierra. Y edificó allí un altar a Jehová, quien le había aparecido.
\par 8 Luego se pasó de allí a un monte al oriente de Bet-el, y plantó su tienda, teniendo a Bet-el al occidente y Hai al oriente; y edificó allí altar a Jehová, e invocó el nombre de Jehová.
\par 9 Y Abram partió de allí, caminando y yendo hacia el Neguev.

\section*{Abram en Egipto}

\par 10 Hubo entonces hambre en la tierra, y descendió Abram a Egipto para morar allá; porque era grande el hambre en la tierra.
\par 11 Y aconteció que cuando estaba para entrar en Egipto, dijo a Sarai su mujer: He aquí, ahora conozco que eres mujer de hermoso aspecto;
\par 12 y cuando te vean los egipcios, dirán: Su mujer es; y me matarán a mí, y a ti te reservarán la vida.
\par 13 Ahora, pues, di que eres mi hermana, para que me vaya bien por causa tuya, y viva mi alma por causa de ti.
\par 14 Y aconteció que cuando entró Abram en Egipto, los egipcios vieron que la mujer era hermosa en gran manera.
\par 15 También la vieron los príncipes de Faraón, y la alabaron delante de él; y fue llevada la mujer a casa de Faraón.
\par 16 E hizo bien a Abram por causa de ella; y él tuvo ovejas, vacas, asnos, siervos, criadas, asnas y camellos.
\par 17 Mas Jehová hirió a Faraón y a su casa con grandes plagas, por causa de Sarai mujer de Abram.
\par 18 Entonces Faraón llamó a Abram, y le dijo: ¿Qué es esto que has hecho conmigo? ¿Por qué no me declaraste que era tu mujer?
\par 19 ¿Por qué dijiste: Es mi hermana, poniéndome en ocasión de tomarla para mí por mujer? Ahora, pues, he aquí tu mujer; tómala, y vete.
\par 20 Entonces Faraón dio orden a su gente acerca de Abram; y le acompañaron, y a su mujer, con todo lo que tenía.

\chapter{13}

\section*{Abram y Lot se separan}

\par 1 Subió, pues, Abram de Egipto hacia el Neguev, él y su mujer, con todo lo que tenía, y con él Lot.
\par 2 Y Abram era riquísimo en ganado, en plata y en oro.
\par 3 Y volvió por sus jornadas desde el Neguev hacia Bet-el, hasta el lugar donde había estado antes su tienda entre Bet-el y Hai,
\par 4 al lugar del altar que había hecho allí antes; e invocó allí Abram el nombre de Jehová.
\par 5 También Lot, que andaba con Abram, tenía ovejas, vacas y tiendas.
\par 6 Y la tierra no era suficiente para que habitasen juntos, pues sus posesiones eran muchas, y no podían morar en un mismo lugar.
\par 7 Y hubo contienda entre los pastores del ganado de Abram y los pastores del ganado de Lot; y el cananeo y el ferezeo habitaban entonces en la tierra.
\par 8 Entonces Abram dijo a Lot: No haya ahora altercado entre nosotros dos, entre mis pastores y los tuyos, porque somos hermanos.
\par 9 ¿No está toda la tierra delante de ti? Yo te ruego que te apartes de mí. Si fueres a la mano izquierda, yo iré a la derecha; y si tú a la derecha, yo iré a la izquierda.
\par 10 Y alzó Lot sus ojos, y vio toda la llanura del Jordán, que toda ella era de riego, como el huerto de Jehová, como la tierra de Egipto en la dirección de Zoar, antes que destruyese Jehová a Sodoma y a Gomorra.
\par 11 Entonces Lot escogió para sí toda la llanura del Jordán; y se fue Lot hacia el oriente, y se apartaron el uno del otro.
\par 12 Abram acampó en la tierra de Canaán, en tanto que Lot habitó en las ciudades de la llanura, y fue poniendo sus tiendas hasta Sodoma.
\par 13 Mas los hombres de Sodoma eran malos y pecadores contra Jehová en gran manera.
\par 14 Y Jehová dijo a Abram, después que Lot se apartó de él: Alza ahora tus ojos, y mira desde el lugar donde estás hacia el norte y el sur, y al oriente y al occidente.
\par 15 Porque toda la tierra que ves, la daré a ti y a tu descendencia para siempre.
\par 16 Y haré tu descendencia como el polvo de la tierra; que si alguno puede contar el polvo de la tierra, también tu descendencia será contada.
\par 17 Levántate, ve por la tierra a lo largo de ella y a su ancho; porque a ti la daré.
\par 18 Abram, pues, removiendo su tienda, vino y moró en el encinar de Mamre, que está en Hebrón, y edificó allí altar a Jehová.

\chapter{14}

\section*{Abram liberta a Lot}

\par 1 Aconteció en los días de Amrafel rey de Sinar, Arioc rey de Elasar, Quedorlaomer rey de Elam, y Tidal rey de Goim,
\par 2 que éstos hicieron guerra contra Bera rey de Sodoma, contra Birsa rey de Gomorra, contra Sinab rey de Adma, contra Semeber rey de Zeboim, y contra el rey de Bela, la cual es Zoar.
\par 3 Todos éstos se juntaron en el valle de Sidim, que es el Mar Salado.
\par 4 Doce años habían servido a Quedorlaomer, y en el decimotercero se rebelaron.
\par 5 Y en el año decimocuarto vino Quedorlaomer, y los reyes que estaban de su parte, y derrotaron a los refaítas en Astarot Karnaim, a los zuzitas en Ham, a los emitas en Save-quiriataim,
\par 6 y a los horeos en el monte de Seir, hasta la llanura de Parán, que está junto al desierto.
\par 7 Y volvieron y vinieron a En-mispat, que es Cades, y devastaron todo el país de los amalecitas, y también al amorreo que habitaba en Hazezontamar.
\par 8 Y salieron el rey de Sodoma, el rey de Gomorra, el rey de Adma, el rey de Zeboim y el rey de Bela, que es Zoar, y ordenaron contra ellos batalla en el valle de Sidim;
\par 9 esto es, contra Quedorlaomer rey de Elam, Tidal rey de Goim, Amrafel rey de Sinar, y Arioc rey de Elasar; cuatro reyes contra cinco.
\par 10 Y el valle de Sidim estaba lleno de pozos de asfalto; y cuando huyeron el rey de Sodoma y el de Gomorra, algunos cayeron allí; y los demás huyeron al monte.
\par 11 Y tomaron toda la riqueza de Sodoma y de Gomorra, y todas sus provisiones, y se fueron.
\par 12 Tomaron también a Lot, hijo del hermano de Abram, que moraba en Sodoma, y sus bienes, y se fueron.
\par 13 Y vino uno de los que escaparon, y lo anunció a Abram el hebreo, que habitaba en el encinar de Mamre el amorreo, hermano de Escol y hermano de Aner, los cuales eran aliados de Abram.
\par 14 Oyó Abram que su pariente estaba prisionero, y armó a sus criados, los nacidos en su casa, trescientos dieciocho, y los siguió hasta Dan.
\par 15 Y cayó sobre ellos de noche, él y sus siervos, y les atacó, y les fue siguiendo hasta Hoba al norte de Damasco.
\par 16 Y recobró todos los bienes, y también a Lot su pariente y sus bienes, y a las mujeres y demás gente.

\section*{Melquisedec bendice a Abram}

\par 17 Cuando volvía de la derrota de Quedorlaomer y de los reyes que con él estaban, salió el rey de Sodoma a recibirlo al valle de Save, que es el Valle del Rey.
\par 18 Entonces Melquisedec, rey de Salem y sacerdote del Dios Altísimo, sacó pan y vino;
\par 19 y le bendijo, diciendo: Bendito sea Abram del Dios Altísimo, creador de los cielos y de la tierra;
\par 20 y bendito sea el Dios Altísimo, que entregó tus enemigos en tu mano. Y le dio Abram los diezmos de todo.
\par 21 Entonces el rey de Sodoma dijo a Abram: Dame las personas, y toma para ti los bienes.
\par 22 Y respondió Abram al rey de Sodoma: He alzado mi mano a Jehová Dios Altísimo, creador de los cielos y de la tierra,
\par 23 que desde un hilo hasta una correa de calzado, nada tomaré de todo lo que es tuyo, para que no digas: Yo enriquecí a Abram;
\par 24 excepto solamente lo que comieron los jóvenes, y la parte de los varones que fueron conmigo, Aner, Escol y Mamre, los cuales tomarán su parte.

\chapter{15}

\section*{Dios promete a Abram un hijo}

\par 1 Después de estas cosas vino la palabra de Jehová a Abram en visión, diciendo: No temas, Abram; yo soy tu escudo, y tu galardón será sobremanera grande.
\par 2 Y respondió Abram: Señor Jehová, ¿qué me darás, siendo así que ando sin hijo, y el mayordomo de mi casa es ese damasceno Eliezer?
\par 3 Dijo también Abram: Mira que no me has dado prole, y he aquí que será mi heredero un esclavo nacido en mi casa.
\par 4 Luego vino a él palabra de Jehová, diciendo: No te heredará éste, sino un hijo tuyo será el que te heredará.
\par 5 Y lo llevó fuera, y le dijo: Mira ahora los cielos, y cuenta las estrellas, si las puedes contar. Y le dijo: Así será tu descendencia.
\par 6 Y creyó a Jehová, y le fue contado por justicia.
\par 7 Y le dijo: Yo soy Jehová, que te saqué de Ur de los caldeos, para darte a heredar esta tierra.
\par 8 Y él respondió: Señor Jehová, ¿en qué conoceré que la he de heredar?
\par 9 Y le dijo: Tráeme una becerra de tres años, y una cabra de tres años, y un carnero de tres años, una tórtola también, y un palomino.
\par 10 Y tomó él todo esto, y los partió por la mitad, y puso cada mitad una enfrente de la otra; mas no partió las aves.
\par 11 Y descendían aves de rapiña sobre los cuerpos muertos, y Abram las ahuyentaba.
\par 12 Mas a la caída del sol sobrecogió el sueño a Abram, y he aquí que el temor de una grande oscuridad cayó sobre él.
\par 13 Entonces Jehová dijo a Abram: Ten por cierto que tu descendencia morará en tierra ajena, y será esclava allí, y será oprimida cuatrocientos años.
\par 14 Mas también a la nación a la cual servirán, juzgaré yo; y después de esto saldrán con gran riqueza.
\par 15 Y tú vendrás a tus padres en paz, y serás sepultado en buena vejez.
\par 16 Y en la cuarta generación volverán acá; porque aún no ha llegado a su colmo la maldad del amorreo hasta aquí.
\par 17 Y sucedió que puesto el sol, y ya oscurecido, se veía un horno humeando, y una antorcha de fuego que pasaba por entre los animales divididos.
\par 18 En aquel día hizo Jehová un pacto con Abram, diciendo: A tu descendencia daré esta tierra, desde el río de Egipto hasta el río grande, el río Eufrates;
\par 19 la tierra de los ceneos, los cenezeos, los cadmoneos,
\par 20 los heteos, los ferezeos, los refaítas,
\par 21 los amorreos, los cananeos, los gergeseos y los jebuseos.

\chapter{16}

\section*{Agar e Ismael}

\par 1 Sarai mujer de Abram no le daba hijos; y ella tenía una sierva egipcia, que se llamaba Agar.
\par 2 Dijo entonces Sarai a Abram: Ya ves que Jehová me ha hecho estéril; te ruego, pues, que te llegues a mi sierva; quizá tendré hijos de ella. Y atendió Abram al ruego de Sarai.
\par 3 Y Sarai mujer de Abram tomó a Agar su sierva egipcia, al cabo de diez años que había habitado Abram en la tierra de Canaán, y la dio por mujer a Abram su marido.
\par 4 Y él se llegó a Agar, la cual concibió; y cuando vio que había concebido, miraba con desprecio a su señora.
\par 5 Entonces Sarai dijo a Abram: Mi afrenta sea sobre ti; yo te di mi sierva por mujer, y viéndose encinta, me mira con desprecio; juzgue Jehová entre tú y yo.
\par 6 Y respondió Abram a Sarai: He aquí, tu sierva está en tu mano; haz con ella lo que bien te parezca. Y como Sarai la afligía, ella huyó de su presencia.
\par 7 Y la halló el ángel de Jehová junto a una fuente de agua en el desierto, junto a la fuente que está en el camino de Shur.
\par 8 Y le dijo: Agar, sierva de Sarai, ¿de dónde vienes tú, y a dónde vas? Y ella respondió: Huyo de delante de Sarai mi señora.
\par 9 Y le dijo el ángel de Jehová: Vuélvete a tu señora, y ponte sumisa bajo su mano.
\par 10 Le dijo también el ángel de Jehová: Multiplicaré tanto tu descendencia, que no podrá ser contada a causa de la multitud.
\par 11 Además le dijo el ángel de Jehová: He aquí que has concebido, y darás a luz un hijo, y llamarás su nombre Ismael, porque Jehová ha oído tu aflicción.
\par 12 Y él será hombre fiero; su mano será contra todos, y la mano de todos contra él, y delante de todos sus hermanos habitará.
\par 13 Entonces llamó el nombre de Jehová que con ella hablaba: Tú eres Dios que ve; porque dijo: ¿No he visto también aquí al que me ve?
\par 14 Por lo cual llamó al pozo: Pozo del Viviente-que-me-ve. He aquí está entre Cades y Bered.
\par 15 Y Agar dio a luz un hijo a Abram, y llamó Abram el nombre del hijo que le dio Agar, Ismael.
\par 16 Era Abram de edad de ochenta y seis años, cuando Agar dio a luz a Ismael.

\chapter{17}

\section*{La circuncisión, señal del pacto}

\par 1 Era Abram de edad de noventa y nueve años, cuando le apareció Jehová y le dijo: Yo soy el Dios Todopoderoso; anda delante de mí y sé perfecto.
\par 2 Y pondré mi pacto entre mí y ti, y te multiplicaré en gran manera.
\par 3 Entonces Abram se postró sobre su rostro, y Dios habló con él, diciendo:
\par 4 He aquí mi pacto es contigo, y serás padre de muchedumbre de gentes.
\par 5 Y no se llamará más tu nombre Abram, sino que será tu nombre Abraham, porque te he puesto por padre de muchedumbre de gentes.
\par 6 Y te multiplicaré en gran manera, y haré naciones de ti, y reyes saldrán de ti.
\par 7 Y estableceré mi pacto entre mí y ti, y tu descendencia después de ti en sus generaciones, por pacto perpetuo, para ser tu Dios, y el de tu descendencia después de ti.
\par 8 Y te daré a ti, y a tu descendencia después de ti, la tierra en que moras, toda la tierra de Canaán en heredad perpetua; y seré el Dios de ellos.
\par 9 Dijo de nuevo Dios a Abraham: En cuanto a ti, guardarás mi pacto, tú y tu descendencia después de ti por sus generaciones.
\par 10 Este es mi pacto, que guardaréis entre mí y vosotros y tu descendencia después de ti: Será circuncidado todo varón de entre vosotros.
\par 11 Circuncidaréis, pues, la carne de vuestro prepucio, y será por señal del pacto entre mí y vosotros.
\par 12 Y de edad de ocho días será circuncidado todo varón entre vosotros por vuestras generaciones; el nacido en casa, y el comprado por dinero a cualquier extranjero, que no fuere de tu linaje.
\par 13 Debe ser circuncidado el nacido en tu casa, y el comprado por tu dinero; y estará mi pacto en vuestra carne por pacto perpetuo.
\par 14 Y el varón incircunciso, el que no hubiere circuncidado la carne de su prepucio, aquella persona será cortada de su pueblo; ha violado mi pacto.
\par 15 Dijo también Dios a Abraham: A Sarai tu mujer no la llamarás Sarai, mas Sara será su nombre.
\par 16 Y la bendeciré, y también te daré de ella hijo; sí, la bendeciré, y vendrá a ser madre de naciones; reyes de pueblos vendrán de ella.
\par 17 Entonces Abraham se postró sobre su rostro, y se rió, y dijo en su corazón: ¿A hombre de cien años ha de nacer hijo? ¿Y Sara, ya de noventa años, ha de concebir?
\par 18 Y dijo Abraham a Dios: Ojalá Ismael viva delante de ti.
\par 19 Respondió Dios: Ciertamente Sara tu mujer te dará a luz un hijo, y llamarás su nombre Isaac; y confirmaré mi pacto con él como pacto perpetuo para sus descendientes después de él.
\par 20 Y en cuanto a Ismael, también te he oído; he aquí que le bendeciré, y le haré fructificar y multiplicar mucho en gran manera; doce príncipes engendrará, y haré de él una gran nación.
\par 21 Mas yo estableceré mi pacto con Isaac, el que Sara te dará a luz por este tiempo el año que viene.
\par 22 Y acabó de hablar con él, y subió Dios de estar con Abraham.
\par 23 Entonces tomó Abraham a Ismael su hijo, y a todos los siervos nacidos en su casa, y a todos los comprados por su dinero, a todo varón entre los domésticos de la casa de Abraham, y circuncidó la carne del prepucio de ellos en aquel mismo día, como Dios le había dicho.
\par 24 Era Abraham de edad de noventa y nueve años cuando circuncidó la carne de su prepucio.
\par 25 E Ismael su hijo era de trece años, cuando fue circuncidada la carne de su prepucio.
\par 26 En el mismo día fueron circuncidados Abraham e Ismael su hijo.
\par 27 Y todos los varones de su casa, el siervo nacido en casa, y el comprado del extranjero por dinero, fueron circuncidados con él.

\chapter{18}

\section*{Promesa del nacimiento de Isaac}

\par 1 Después le apareció Jehová en el encinar de Mamre, estando él sentado a la puerta de su tienda en el calor del día.
\par 2 Y alzó sus ojos y miró, y he aquí tres varones que estaban junto a él; y cuando los vio, salió corriendo de la puerta de su tienda a recibirlos, y se postró en tierra,
\par 3 y dijo: Señor, si ahora he hallado gracia en tus ojos, te ruego que no pases de tu siervo.
\par 4 Que se traiga ahora un poco de agua, y lavad vuestros pies; y recostaos debajo de un árbol,
\par 5 y traeré un bocado de pan, y sustentad vuestro corazón, y después pasaréis; pues por eso habéis pasado cerca de vuestro siervo. Y ellos dijeron: Haz así como has dicho.
\par 6 Entonces Abraham fue de prisa a la tienda a Sara, y le dijo: Toma pronto tres medidas de flor de harina, y amasa y haz panes cocidos debajo del rescoldo.
\par 7 Y corrió Abraham a las vacas, y tomó un becerro tierno y bueno, y lo dio al criado, y éste se dio prisa a prepararlo.
\par 8 Tomó también mantequilla y leche, y el becerro que había preparado, y lo puso delante de ellos; y él se estuvo con ellos debajo del árbol, y comieron.
\par 9 Y le dijeron: ¿Dónde está Sara tu mujer? Y él respondió: Aquí en la tienda.
\par 10 Entonces dijo: De cierto volveré a ti; y según el tiempo de la vida, he aquí que Sara tu mujer tendrá un hijo. Y Sara escuchaba a la puerta de la tienda, que estaba detrás de él.
\par 11 Y Abraham y Sara eran viejos, de edad avanzada; y a Sara le había cesado ya la costumbre de las mujeres.
\par 12 Se rió, pues, Sara entre sí, diciendo: ¿Después que he envejecido tendré deleite, siendo también mi señor ya viejo?
\par 13 Entonces Jehová dijo a Abraham: ¿Por qué se ha reído Sara diciendo: ¿Será cierto que he de dar a luz siendo ya vieja?
\par 14 ¿Hay para Dios alguna cosa difícil? Al tiempo señalado volveré a ti, y según el tiempo de la vida, Sara tendrá un hijo.
\par 15 Entonces Sara negó, diciendo: No me reí; porque tuvo miedo. Y él dijo: No es así, sino que te has reído.

\section*{Abraham intercede por Sodoma}

\par 16 Y los varones se levantaron de allí, y miraron hacia Sodoma; y Abraham iba con ellos acompañándolos.
\par 17 Y Jehová dijo: ¿Encubriré yo a Abraham lo que voy a hacer,
\par 18 habiendo de ser Abraham una nación grande y fuerte, y habiendo de ser benditas en él todas las naciones de la tierra?
\par 19 Porque yo sé que mandará a sus hijos y a su casa después de sí, que guarden el camino de Jehová, haciendo justicia y juicio, para que haga venir Jehová sobre Abraham lo que ha hablado acerca de él.
\par 20 Entonces Jehová le dijo: Por cuanto el clamor contra Sodoma y Gomorra se aumenta más y más, y el pecado de ellos se ha agravado en extremo,
\par 21 descenderé ahora, y veré si han consumado su obra según el clamor que ha venido hasta mí; y si no, lo sabré.
\par 22 Y se apartaron de allí los varones, y fueron hacia Sodoma; pero Abraham estaba aún delante de Jehová.
\par 23 Y se acercó Abraham y dijo: ¿Destruirás también al justo con el impío?
\par 24 Quizá haya cincuenta justos dentro de la ciudad: ¿destruirás también y no perdonarás al lugar por amor a los cincuenta justos que estén dentro de él?
\par 25 Lejos de ti el hacer tal, que hagas morir al justo con el impío, y que sea el justo tratado como el impío; nunca tal hagas. El Juez de toda la tierra, ¿no ha de hacer lo que es justo?
\par 26 Entonces respondió Jehová: Si hallare en Sodoma cincuenta justos dentro de la ciudad, perdonaré a todo este lugar por amor a ellos.
\par 27 Y Abraham replicó y dijo: He aquí ahora que he comenzado a hablar a mi Señor, aunque soy polvo y ceniza.
\par 28 Quizá faltarán de cincuenta justos cinco; ¿destruirás por aquellos cinco toda la ciudad? Y dijo: No la destruiré, si hallare allí cuarenta y cinco.
\par 29 Y volvió a hablarle, y dijo: Quizá se hallarán allí cuarenta. Y respondió: No lo haré por amor a los cuarenta.
\par 30 Y dijo: No se enoje ahora mi Señor, si hablare: quizá se hallarán allí treinta. Y respondió: No lo haré si hallare allí treinta.
\par 31 Y dijo: He aquí ahora que he emprendido el hablar a mi Señor: quizá se hallarán allí veinte. No la destruiré, respondió, por amor a los veinte.
\par 32 Y volvió a decir: No se enoje ahora mi Señor, si hablare solamente una vez: quizá se hallarán allí diez. No la destruiré, respondió, por amor a los diez.
\par 33 Y Jehová se fue, luego que acabó de hablar a Abraham; y Abraham volvió a su lugar.

\chapter{19}

\section*{Destrucción de Sodoma y Gomorra}

\par 1 Llegaron, pues, los dos ángeles a Sodoma a la caída de la tarde; y Lot estaba sentado a la puerta de Sodoma. Y viéndolos Lot, se levantó a recibirlos, y se inclinó hacia el suelo,
\par 2 y dijo: Ahora, mis señores, os ruego que vengáis a casa de vuestro siervo y os hospedéis, y lavaréis vuestros pies; y por la mañana os levantaréis, y seguiréis vuestro camino. Y ellos respondieron: No, que en la calle nos quedaremos esta noche.
\par 3 Mas él porfió con ellos mucho, y fueron con él, y entraron en su casa; y les hizo banquete, y coció panes sin levadura, y comieron.
\par 4 Pero antes que se acostasen, rodearon la casa los hombres de la ciudad, los varones de Sodoma, todo el pueblo junto, desde el más joven hasta el más viejo.
\par 5 Y llamaron a Lot, y le dijeron: ¿Dónde están los varones que vinieron a ti esta noche? Sácalos, para que los conozcamos.
\par 6 Entonces Lot salió a ellos a la puerta, y cerró la puerta tras sí,
\par 7 y dijo: Os ruego, hermanos míos, que no hagáis tal maldad.
\par 8 He aquí ahora yo tengo dos hijas que no han conocido varón; os las sacaré fuera, y haced de ellas como bien os pareciere; solamente que a estos varones no hagáis nada, pues que vinieron a la sombra de mi tejado.
\par 9 Y ellos respondieron: Quita allá; y añadieron: Vino este extraño para habitar entre nosotros, ¿y habrá de erigirse en juez? Ahora te haremos más mal que a ellos. Y hacían gran violencia al varón, a Lot, y se acercaron para romper la puerta.
\par 10 Entonces los varones alargaron la mano, y metieron a Lot en casa con ellos, y cerraron la puerta.
\par 11 Y a los hombres que estaban a la puerta de la casa hirieron con ceguera desde el menor hasta el mayor, de manera que se fatigaban buscando la puerta.
\par 12 Y dijeron los varones a Lot: ¿Tienes aquí alguno más? Yernos, y tus hijos y tus hijas, y todo lo que tienes en la ciudad, sácalo de este lugar;
\par 13 porque vamos a destruir este lugar, por cuanto el clamor contra ellos ha subido de punto delante de Jehová; por tanto, Jehová nos ha enviado para destruirlo.
\par 14 Entonces salió Lot y habló a sus yernos, los que habían de tomar sus hijas, y les dijo: Levantaos, salid de este lugar; porque Jehová va a destruir esta ciudad. Mas pareció a sus yernos como que se burlaba.
\par 15 Y al rayar el alba, los ángeles daban prisa a Lot, diciendo: Levántate, toma tu mujer, y tus dos hijas que se hallan aquí, para que no perezcas en el castigo de la ciudad.
\par 16 Y deteniéndose él, los varones asieron de su mano, y de la mano de su mujer y de las manos de sus dos hijas, según la misericordia de Jehová para con él; y lo sacaron y lo pusieron fuera de la ciudad.
\par 17 Y cuando los hubieron llevado fuera, dijeron: Escapa por tu vida; no mires tras ti, ni pares en toda esta llanura; escapa al monte, no sea que perezcas.
\par 18 Pero Lot les dijo: No, yo os ruego, señores míos.
\par 19 He aquí ahora ha hallado vuestro siervo gracia en vuestros ojos, y habéis engrandecido vuestra misericordia que habéis hecho conmigo dándome la vida; mas yo no podré escapar al monte, no sea que me alcance el mal, y muera.
\par 20 He aquí ahora esta ciudad está cerca para huir allá, la cual es pequeña; dejadme escapar ahora allá (¿no es ella pequeña?), y salvaré mi vida.
\par 21 Y le respondió: He aquí he recibido también tu súplica sobre esto, y no destruiré la ciudad de que has hablado.
\par 22 Date prisa, escápate allá; porque nada podré hacer hasta que hayas llegado allí. Por eso fue llamado el nombre de la ciudad, Zoar.
\par 23 El sol salía sobre la tierra, cuando Lot llegó a Zoar.
\par 24 Entonces Jehová hizo llover sobre Sodoma y sobre Gomorra azufre y fuego de parte de Jehová desde los cielos;
\par 25 y destruyó las ciudades, y toda aquella llanura, con todos los moradores de aquellas ciudades, y el fruto de la tierra.
\par 26 Entonces la mujer de Lot miró atrás, a espaldas de él, y se volvió estatua de sal.
\par 27 Y subió Abraham por la mañana al lugar donde había estado delante de Jehová.
\par 28 Y miró hacia Sodoma y Gomorra, y hacia toda la tierra de aquella llanura miró; y he aquí que el humo subía de la tierra como el humo de un horno.
\par 29 Así, cuando destruyó Dios las ciudades de la llanura, Dios se acordó de Abraham, y envió fuera a Lot de en medio de la destrucción, al asolar las ciudades donde Lot estaba.
\par 30 Pero Lot subió de Zoar y moró en el monte, y sus dos hijas con él; porque tuvo miedo de quedarse en Zoar, y habitó en una cueva él y sus dos hijas.
\par 31 Entonces la mayor dijo a la menor: Nuestro padre es viejo, y no queda varón en la tierra que entre a nosotras conforme a la costumbre de toda la tierra.
\par 32 Ven, demos a beber vino a nuestro padre, y durmamos con él, y conservaremos de nuestro padre descendencia.
\par 33 Y dieron a beber vino a su padre aquella noche, y entró la mayor, y durmió con su padre; mas él no sintió cuándo se acostó ella, ni cuándo se levantó.
\par 34 El día siguiente, dijo la mayor a la menor: He aquí, yo dormí la noche pasada con mi padre; démosle a beber vino también esta noche, y entra y duerme con él, para que conservemos de nuestro padre descendencia.
\par 35 Y dieron a beber vino a su padre también aquella noche, y se levantó la menor, y durmió con él; pero él no echó de ver cuándo se acostó ella, ni cuándo se levantó.
\par 36 Y las dos hijas de Lot concibieron de su padre.
\par 37 Y dio a luz la mayor un hijo, y llamó su nombre Moab, el cual es padre de los moabitas hasta hoy.
\par 38 La menor también dio a luz un hijo, y llamó su nombre Ben-ammi, el cual es padre de los amonitas hasta hoy.

\chapter{20}

\section*{Abraham y Abimelec}

\par 1 De allí partió Abraham a la tierra del Neguev, y acampó entre Cades y Shur, y habitó como forastero en Gerar.
\par 2 Y dijo Abraham de Sara su mujer: Es mi hermana. Y Abimelec rey de Gerar envió y tomó a Sara.
\par 3 Pero Dios vino a Abimelec en sueños de noche, y le dijo: He aquí, muerto eres, a causa de la mujer que has tomado, la cual es casada con marido.
\par 4 Mas Abimelec no se había llegado a ella, y dijo: Señor, ¿matarás también al inocente?
\par 5 ¿No me dijo él: Mi hermana es; y ella también dijo: Es mi hermano? Con sencillez de mi corazón y con limpieza de mis manos he hecho esto.
\par 6 Y le dijo Dios en sueños: Yo también sé que con integridad de tu corazón has hecho esto; y yo también te detuve de pecar contra mí, y así no te permití que la tocases.
\par 7 Ahora, pues, devuelve la mujer a su marido; porque es profeta, y orará por ti, y vivirás. Y si no la devolvieres, sabe que de cierto morirás tú, y todos los tuyos.
\par 8 Entonces Abimelec se levantó de mañana y llamó a todos sus siervos, y dijo todas estas palabras en los oídos de ellos; y temieron los hombres en gran manera.
\par 9 Después llamó Abimelec a Abraham, y le dijo: ¿Qué nos has hecho? ¿En qué pequé yo contra ti, que has atraído sobre mí y sobre mi reino tan grande pecado? Lo que no debiste hacer has hecho conmigo.
\par 10 Dijo también Abimelec a Abraham: ¿Qué pensabas, para que hicieses esto?
\par 11 Y Abraham respondió: Porque dije para mí: Ciertamente no hay temor de Dios en este lugar, y me matarán por causa de mi mujer.
\par 12 Y a la verdad también es mi hermana, hija de mi padre, mas no hija de mi madre, y la tomé por mujer.
\par 13 Y cuando Dios me hizo salir errante de la casa de mi padre, yo le dije: Esta es la merced que tú harás conmigo, que en todos los lugares adonde lleguemos, digas de mí: Mi hermano es.
\par 14 Entonces Abimelec tomó ovejas y vacas, y siervos y siervas, y se los dio a Abraham, y le devolvió a Sara su mujer.
\par 15 Y dijo Abimelec: He aquí mi tierra está delante de ti; habita donde bien te parezca.
\par 16 Y a Sara dijo: He aquí he dado mil monedas de plata a tu hermano; mira que él te es como un velo para los ojos de todos los que están contigo, y para con todos; así fue vindicada.
\par 17 Entonces Abraham oró a Dios; y Dios sanó a Abimelec y a su mujer, y a sus siervas, y tuvieron hijos.
\par 18 Porque Jehová había cerrado completamente toda matriz de la casa de Abimelec, a causa de Sara mujer de Abraham.

\chapter{21}

\section*{Nacimiento de Isaac}

\par 1 Visitó Jehová a Sara, como había dicho, e hizo Jehová con Sara como había hablado.
\par 2 Y Sara concibió y dio a Abraham un hijo en su vejez, en el tiempo que Dios le había dicho.
\par 3 Y llamó Abraham el nombre de su hijo que le nació, que le dio a luz Sara, Isaac.
\par 4 Y circuncidó Abraham a su hijo Isaac de ocho días, como Dios le había mandado.
\par 5 Y era Abraham de cien años cuando nació Isaac su hijo.
\par 6 Entonces dijo Sara: Dios me ha hecho reír, y cualquiera que lo oyere, se reirá conmigo.
\par 7 Y añadió: ¿Quién dijera a Abraham que Sara habría de dar de mamar a hijos? Pues le he dado un hijo en su vejez.

\section*{Agar e Ismael son echados de la casa de Abraham}

\par 8 Y creció el niño, y fue destetado; e hizo Abraham gran banquete el día que fue destetado Isaac.
\par 9 Y vio Sara que el hijo de Agar la egipcia, el cual ésta le había dado a luz a Abraham, se burlaba de su hijo Isaac.
\par 10 Por tanto, dijo a Abraham: Echa a esta sierva y a su hijo, porque el hijo de esta sierva no ha de heredar con Isaac mi hijo.
\par 11 Este dicho pareció grave en gran manera a Abraham a causa de su hijo.
\par 12 Entonces dijo Dios a Abraham: No te parezca grave a causa del muchacho y de tu sierva; en todo lo que te dijere Sara, oye su voz, porque en Isaac te será llamada descendencia.
\par 13 Y también del hijo de la sierva haré una nación, porque es tu descendiente.
\par 14 Entonces Abraham se levantó muy de mañana, y tomó pan, y un odre de agua, y lo dio a Agar, poniéndolo sobre su hombro, y le entregó el muchacho, y la despidió. Y ella salió y anduvo errante por el desierto de Beerseba.
\par 15 Y le faltó el agua del odre, y echó al muchacho debajo de un arbusto,
\par 16 y se fue y se sentó enfrente, a distancia de un tiro de arco; porque decía: No veré cuando el muchacho muera. Y cuando ella se sentó enfrente, el muchacho alzó su voz y lloró.
\par 17 Y oyó Dios la voz del muchacho; y el ángel de Dios llamó a Agar desde el cielo, y le dijo: ¿Qué tienes, Agar? No temas; porque Dios ha oído la voz del muchacho en donde está.
\par 18 Levántate, alza al muchacho, y sostenlo con tu mano, porque yo haré de él una gran nación.
\par 19 Entonces Dios le abrió los ojos, y vio una fuente de agua; y fue y llenó el odre de agua, y dio de beber al muchacho.
\par 20 Y Dios estaba con el muchacho; y creció, y habitó en el desierto, y fue tirador de arco.
\par 21 Y habitó en el desierto de Parán; y su madre le tomó mujer de la tierra de Egipto.

\section*{Pacto entre Abraham y Abimelec}

\par 22 Aconteció en aquel mismo tiempo que habló Abimelec, y Ficol príncipe de su ejército, a Abraham, diciendo: Dios está contigo en todo cuanto haces.
\par 23 Ahora, pues, júrame aquí por Dios, que no faltarás a mí, ni a mi hijo ni a mi nieto, sino que conforme a la bondad que yo hice contigo, harás tú conmigo, y con la tierra en donde has morado.
\par 24 Y respondió Abraham: Yo juraré.
\par 25 Y Abraham reconvino a Abimelec a causa de un pozo de agua, que los siervos de Abimelec le habían quitado.
\par 26 Y respondió Abimelec: No sé quién haya hecho esto, ni tampoco tú me lo hiciste saber, ni yo lo he oído hasta hoy.
\par 27 Y tomó Abraham ovejas y vacas, y dio a Abimelec; e hicieron ambos pacto.
\par 28 Entonces puso Abraham siete corderas del rebaño aparte.
\par 29 Y dijo Abimelec a Abraham: ¿Qué significan esas siete corderas que has puesto aparte?
\par 30 Y él respondió: Que estas siete corderas tomarás de mi mano, para que me sirvan de testimonio de que yo cavé este pozo.
\par 31 Por esto llamó a aquel lugar Beerseba; porque allí juraron ambos.
\par 32 Así hicieron pacto en Beerseba; y se levantó Abimelec, y Ficol príncipe de su ejército, y volvieron a tierra de los filisteos.
\par 33 Y plantó Abraham un árbol tamarisco en Beerseba, e invocó allí el nombre de Jehová Dios eterno.
\par 34 Y moró Abraham en tierra de los filisteos muchos días.

\chapter{22}

\section*{Dios ordena a Abraham que sacrifique a Isaac}

\par 1 Aconteció después de estas cosas, que probó Dios a Abraham, y le dijo: Abraham. Y él respondió: Heme aquí.
\par 2 Y dijo: Toma ahora tu hijo, tu único, Isaac, a quien amas, y vete a tierra de Moriah, y ofrécelo allí en holocausto sobre uno de los montes que yo te diré.
\par 3 Y Abraham se levantó muy de mañana, y enalbardó su asno, y tomó consigo dos siervos suyos, y a Isaac su hijo; y cortó leña para el holocausto, y se levantó, y fue al lugar que Dios le dijo.
\par 4 Al tercer día alzó Abraham sus ojos, y vio el lugar de lejos.
\par 5 Entonces dijo Abraham a sus siervos: Esperad aquí con el asno, y yo y el muchacho iremos hasta allí y adoraremos, y volveremos a vosotros.
\par 6 Y tomó Abraham la leña del holocausto, y la puso sobre Isaac su hijo, y él tomó en su mano el fuego y el cuchillo; y fueron ambos juntos.
\par 7 Entonces habló Isaac a Abraham su padre, y dijo: Padre mío. Y él respondió: Heme aquí, mi hijo. Y él dijo: He aquí el fuego y la leña; mas ¿dónde está el cordero para el holocausto?
\par 8 Y respondió Abraham: Dios se proveerá de cordero para el holocausto, hijo mío. E iban juntos.
\par 9 Y cuando llegaron al lugar que Dios le había dicho, edificó allí Abraham un altar, y compuso la leña, y ató a Isaac su hijo, y lo puso en el altar sobre la leña.
\par 10 Y extendió Abraham su mano y tomó el cuchillo para degollar a su hijo.
\par 11 Entonces el ángel de Jehová le dio voces desde el cielo, y dijo: Abraham, Abraham. Y él respondió: Heme aquí.
\par 12 Y dijo: No extiendas tu mano sobre el muchacho, ni le hagas nada; porque ya conozco que temes a Dios, por cuanto no me rehusaste tu hijo, tu único.
\par 13 Entonces alzó Abraham sus ojos y miró, y he aquí a sus espaldas un carnero trabado en un zarzal por sus cuernos; y fue Abraham y tomó el carnero, y lo ofreció en holocausto en lugar de su hijo.
\par 14 Y llamó Abraham el nombre de aquel lugar, Jehová proveerá. Por tanto se dice hoy: En el monte de Jehová será provisto.
\par 15 Y llamó el ángel de Jehová a Abraham por segunda vez desde el cielo,
\par 16 y dijo: Por mí mismo he jurado, dice Jehová, que por cuanto has hecho esto, y no me has rehusado tu hijo, tu único hijo;
\par 17 de cierto te bendeciré, y multiplicaré tu descendencia como las estrellas del cielo y como la arena que está a la orilla del mar; y tu descendencia poseerá las puertas de sus enemigos.
\par 18 En tu simiente serán benditas todas las naciones de la tierra, por cuanto obedeciste a mi voz.
\par 19 Y volvió Abraham a sus siervos, y se levantaron y se fueron juntos a Beerseba; y habitó Abraham en Beerseba.
\par 20 Aconteció después de estas cosas, que fue dada noticia a Abraham, diciendo: He aquí que también Milca ha dado a luz hijos a Nacor tu hermano:
\par 21 Uz su primogénito, Buz su hermano, Kemuel padre de Aram,
\par 22 Quesed, Hazo, Pildas, Jidlaf y Betuel.
\par 23 Y Betuel fue el padre de Rebeca. Estos son los ocho hijos que dio a luz Milca, de Nacor hermano de Abraham.
\par 24 Y su concubina, que se llamaba Reúma, dio a luz también a Teba, a Gaham, a Tahas y a Maaca.


\chapter{23}

\section*{Muerte y sepultura de Sara}

\par 1 Fue la vida de Sara ciento veintisiete años; tantos fueron los años de la vida de Sara.
\par 2 Y murió Sara en Quiriat-arba, que es Hebrón, en la tierra de Canaán; y vino Abraham a hacer duelo por Sara, y a llorarla.
\par 3 Y se levantó Abraham de delante de su muerta, y habló a los hijos de Het, diciendo:
\par 4 Extranjero y forastero soy entre vosotros; dadme propiedad para sepultura entre vosotros, y sepultaré mi muerta de delante de mí.
\par 5 Y respondieron los hijos de Het a Abraham, y le dijeron:
\par 6 Oyenos, señor nuestro; eres un príncipe de Dios entre nosotros; en lo mejor de nuestros sepulcros sepulta a tu muerta; ninguno de nosotros te negará su sepulcro, ni te impedirá que entierres tu muerta.
\par 7 Y Abraham se levantó, y se inclinó al pueblo de aquella tierra, a los hijos de Het,
\par 8 y habló con ellos, diciendo: Si tenéis voluntad de que yo sepulte mi muerta de delante de mí, oídme, e interceded por mí con Efrón hijo de Zohar,
\par 9 para que me dé la cueva de Macpela, que tiene al extremo de su heredad; que por su justo precio me la dé, para posesión de sepultura en medio de vosotros.
\par 10 Este Efrón estaba entre los hijos de Het; y respondió Efrón heteo a Abraham, en presencia de los hijos de Het, de todos los que entraban por la puerta de su ciudad, diciendo:
\par 11 No, señor mío, óyeme: te doy la heredad, y te doy también la cueva que está en ella; en presencia de los hijos de mi pueblo te la doy; sepulta tu muerta.
\par 12 Entonces Abraham se inclinó delante del pueblo de la tierra,
\par 13 y respondió a Efrón en presencia del pueblo de la tierra, diciendo: Antes, si te place, te ruego que me oigas. Yo daré el precio de la heredad; tómalo de mí, y sepultaré en ella mi muerta.
\par 14 Respondió Efrón a Abraham, diciéndole:
\par 15 Señor mío, escúchame: la tierra vale cuatrocientos siclos de plata; ¿qué es esto entre tú y yo? Entierra, pues, tu muerta.
\par 16 Entonces Abraham se convino con Efrón, y pesó Abraham a Efrón el dinero que dijo, en presencia de los hijos de Het, cuatrocientos siclos de plata, de buena ley entre mercaderes.
\par 17 Y quedó la heredad de Efrón que estaba en Macpela al oriente de Mamre, la heredad con la cueva que estaba en ella, y todos los árboles que había en la heredad, y en todos sus contornos,
\par 18 como propiedad de Abraham, en presencia de los hijos de Het y de todos los que entraban por la puerta de la ciudad.
\par 19 Después de esto sepultó Abraham a Sara su mujer en la cueva de la heredad de Macpela al oriente de Mamre, que es Hebrón, en la tierra de Canaán.
\par 20 Y quedó la heredad y la cueva que en ella había, de Abraham, como una posesión para sepultura, recibida de los hijos de Het.

\chapter{24}

\section*{Abraham busca esposa para Isaac}

\par 1 Era Abraham ya viejo, y bien avanzado en años; y Jehová había bendecido a Abraham en todo.
\par 2 Y dijo Abraham a un criado suyo, el más viejo de su casa, que era el que gobernaba en todo lo que tenía: Pon ahora tu mano debajo de mi muslo,
\par 3 y te juramentaré por Jehová, Dios de los cielos y Dios de la tierra, que no tomarás para mi hijo mujer de las hijas de los cananeos, entre los cuales yo habito;
\par 4 sino que irás a mi tierra y a mi parentela, y tomarás mujer para mi hijo Isaac.
\par 5 El criado le respondió: Quizá la mujer no querrá venir en pos de mí a esta tierra. ¿Volveré, pues, tu hijo a la tierra de donde saliste?
\par 6 Y Abraham le dijo: Guárdate que no vuelvas a mi hijo allá.
\par 7 Jehová, Dios de los cielos, que me tomó de la casa de mi padre y de la tierra de mi parentela, y me habló y me juró, diciendo: A tu descendencia daré esta tierra; él enviará su ángel delante de ti, y tú traerás de allá mujer para mi hijo.
\par 8 Y si la mujer no quisiere venir en pos de ti, serás libre de este mi juramento; solamente que no vuelvas allá a mi hijo.
\par 9 Entonces el criado puso su mano debajo del muslo de Abraham su señor, y le juró sobre este negocio.
\par 10 Y el criado tomó diez camellos de los camellos de su señor, y se fue, tomando toda clase de regalos escogidos de su señor; y puesto en camino, llegó a Mesopotamia, a la ciudad de Nacor.
\par 11 E hizo arrodillar los camellos fuera de la ciudad, junto a un pozo de agua, a la hora de la tarde, la hora en que salen las doncellas por agua.
\par 12 Y dijo: Oh Jehová, Dios de mi señor Abraham, dame, te ruego, el tener hoy buen encuentro, y haz misericordia con mi señor Abraham.
\par 13 He aquí yo estoy junto a la fuente de agua, y las hijas de los varones de esta ciudad salen por agua.
\par 14 Sea, pues, que la doncella a quien yo dijere: Baja tu cántaro, te ruego, para que yo beba, y ella respondiere: Bebe, y también daré de beber a tus camellos; que sea ésta la que tú has destinado para tu siervo Isaac; y en esto conoceré que habrás hecho misericordia con mi señor.
\par 15 Y aconteció que antes que él acabase de hablar, he aquí Rebeca, que había nacido a Betuel, hijo de Milca mujer de Nacor hermano de Abraham, la cual salía con su cántaro sobre su hombro.
\par 16 Y la doncella era de aspecto muy hermoso, virgen, a la que varón no había conocido; la cual descendió a la fuente, y llenó su cántaro, y se volvía.
\par 17 Entonces el criado corrió hacia ella, y dijo: Te ruego que me des a beber un poco de agua de tu cántaro.
\par 18 Ella respondió: Bebe, señor mío; y se dio prisa a bajar su cántaro sobre su mano, y le dio a beber.
\par 19 Y cuando acabó de darle de beber, dijo: También para tus camellos sacaré agua, hasta que acaben de beber.
\par 20 Y se dio prisa, y vació su cántaro en la pila, y corrió otra vez al pozo para sacar agua, y sacó para todos sus camellos.
\par 21 Y el hombre estaba maravillado de ella, callando, para saber si Jehová había prosperado su viaje, o no.
\par 22 Y cuando los camellos acabaron de beber, le dio el hombre un pendiente de oro que pesaba medio siclo, y dos brazaletes que pesaban diez,
\par 23 y dijo: ¿De quién eres hija? Te ruego que me digas: ¿hay en casa de tu padre lugar donde posemos?
\par 24 Y ella respondió: Soy hija de Betuel hijo de Milca, el cual ella dio a luz a Nacor.
\par 25 Y añadió: También hay en nuestra casa paja y mucho forraje, y lugar para posar.
\par 26 El hombre entonces se inclinó, y adoró a Jehová,
\par 27 y dijo: Bendito sea Jehová, Dios de mi amo Abraham, que no apartó de mi amo su misericordia y su verdad, guiándome Jehová en el camino a casa de los hermanos de mi amo.
\par 28 Y la doncella corrió, e hizo saber en casa de su madre estas cosas.
\par 29 Y Rebeca tenía un hermano que se llamaba Labán, el cual corrió afuera hacia el hombre, a la fuente.
\par 30 Y cuando vio el pendiente y los brazaletes en las manos de su hermana, que decía: Así me habló aquel hombre, vino a él; y he aquí que estaba con los camellos junto a la fuente.
\par 31 Y le dijo: Ven, bendito de Jehová; ¿por qué estás fuera? He preparado la casa, y el lugar para los camellos.
\par 32 Entonces el hombre vino a casa, y Labán desató los camellos; y les dio paja y forraje, y agua para lavar los pies de él, y los pies de los hombres que con él venían.
\par 33 Y le pusieron delante qué comer; mas él dijo: No comeré hasta que haya dicho mi mensaje. Y él le dijo: Habla.
\par 34 Entonces dijo: Yo soy criado de Abraham.
\par 35 Y Jehová ha bendecido mucho a mi amo, y él se ha engrandecido; y le ha dado ovejas y vacas, plata y oro, siervos y siervas, camellos y asnos.
\par 36 Y Sara, mujer de mi amo, dio a luz en su vejez un hijo a mi señor, quien le ha dado a él todo cuanto tiene.
\par 37 Y mi amo me hizo jurar, diciendo: No tomarás para mi hijo mujer de las hijas de los cananeos, en cuya tierra habito;
\par 38 sino que irás a la casa de mi padre y a mi parentela, y tomarás mujer para mi hijo.
\par 39 Y yo dije: Quizá la mujer no querrá seguirme.
\par 40 Entonces él me respondió: Jehová, en cuya presencia he andado, enviará su ángel contigo, y prosperará tu camino; y tomarás para mi hijo mujer de mi familia y de la casa de mi padre.
\par 41 Entonces serás libre de mi juramento, cuando hayas llegado a mi familia; y si no te la dieren, serás libre de mi juramento.
\par 42 Llegué, pues, hoy a la fuente, y dije: Jehová, Dios de mi señor Abraham, si tú prosperas ahora mi camino por el cual ando,
\par 43 he aquí yo estoy junto a la fuente de agua; sea, pues, que la doncella que saliere por agua, a la cual dijere: Dame de beber, te ruego, un poco de agua de tu cántaro,
\par 44 y ella me respondiere: Bebe tú, y también para tus camellos sacaré agua; sea ésta la mujer que destinó Jehová para el hijo de mi señor.
\par 45 Antes que acabase de hablar en mi corazón, he aquí Rebeca, que salía con su cántaro sobre su hombro; y descendió a la fuente, y sacó agua; y le dije: Te ruego que me des de beber.
\par 46 Y bajó prontamente su cántaro de encima de sí, y dijo: Bebe, y también a tus camellos daré de beber. Y bebí, y dio también de beber a mis camellos.
\par 47 Entonces le pregunté, y dije: ¿De quién eres hija? Y ella respondió: Hija de Betuel hijo de Nacor, que le dio a luz Milca. Entonces le puse un pendiente en su nariz, y brazaletes en sus brazos;
\par 48 y me incliné y adoré a Jehová, y bendije a Jehová Dios de mi señor Abraham, que me había guiado por camino de verdad para tomar la hija del hermano de mi señor para su hijo.
\par 49 Ahora, pues, si vosotros hacéis misericordia y verdad con mi señor, declarádmelo; y si no, declarádmelo; y me iré a la diestra o a la siniestra.
\par 50 Entonces Labán y Betuel respondieron y dijeron: De Jehová ha salido esto; no podemos hablarte malo ni bueno.
\par 51 He ahí Rebeca delante de ti; tómala y vete, y sea mujer del hijo de tu señor, como lo ha dicho Jehová.
\par 52 Cuando el criado de Abraham oyó sus palabras, se inclinó en tierra ante Jehová.
\par 53 Y sacó el criado alhajas de plata y alhajas de oro, y vestidos, y dio a Rebeca; también dio cosas preciosas a su hermano y a su madre.
\par 54 Y comieron y bebieron él y los varones que venían con él, y durmieron; y levantándose de mañana, dijo: Enviadme a mi señor.
\par 55 Entonces respondieron su hermano y su madre: Espere la doncella con nosotros a lo menos diez días, y después irá.
\par 56 Y él les dijo: No me detengáis, ya que Jehová ha prosperado mi camino; despachadme para que me vaya a mi señor.
\par 57 Ellos respondieron entonces: Llamemos a la doncella y preguntémosle.
\par 58 Y llamaron a Rebeca, y le dijeron: ¿Irás tú con este varón? Y ella respondió: Sí, iré.
\par 59 Entonces dejaron ir a Rebeca su hermana, y a su nodriza, y al criado de Abraham y a sus hombres.
\par 60 Y bendijeron a Rebeca, y le dijeron: Hermana nuestra, sé madre de millares de millares, y posean tus descendientes la puerta de sus enemigos.
\par 61 Entonces se levantó Rebeca y sus doncellas, y montaron en los camellos, y siguieron al hombre; y el criado tomó a Rebeca, y se fue.
\par 62 Y venía Isaac del pozo del Viviente-que-me-ve; porque él habitaba en el Neguev.
\par 63 Y había salido Isaac a meditar al campo, a la hora de la tarde; y alzando sus ojos miró, y he aquí los camellos que venían.
\par 64 Rebeca también alzó sus ojos, y vio a Isaac, y descendió del camello;
\par 65 porque había preguntado al criado: ¿Quién es este varón que viene por el campo hacia nosotros? Y el criado había respondido: Este es mi señor. Ella entonces tomó el velo, y se cubrió.
\par 66 Entonces el criado contó a Isaac todo lo que había hecho.
\par 67 Y la trajo Isaac a la tienda de su madre Sara, y tomó a Rebeca por mujer, y la amó; y se consoló Isaac después de la muerte de su madre.

\chapter{25}

\section*{Los descendientes de Abraham y Cetura}

\par 1 Abraham tomó otra mujer, cuyo nombre era Cetura,
\par 2 la cual le dio a luz a Zimram, Jocsán, Medán, Madián, Isbac y Súa.
\par 3 Y Jocsán engendró a Seba y a Dedán; e hijos de Dedán fueron Asurim, Letusim y Leumim.
\par 4 E hijos de Madián: Efa, Efer, Hanoc, Abida y Elda. Todos estos fueron hijos de Cetura.
\par 5 Y Abraham dio todo cuanto tenía a Isaac.
\par 6 Pero a los hijos de sus concubinas dio Abraham dones, y los envió lejos de Isaac su hijo, mientras él vivía, hacia el oriente, a la tierra oriental.

\section*{Muerte y sepultura de Abraham}

\par 7 Y estos fueron los días que vivió Abraham: ciento setenta y cinco años.
\par 8 Y exhaló el espíritu, y murió Abraham en buena vejez, anciano y lleno de años, y fue unido a su pueblo.
\par 9 Y lo sepultaron Isaac e Ismael sus hijos en la cueva de Macpela, en la heredad de Efrón hijo de Zohar heteo, que está enfrente de Mamre,
\par 10 heredad que compró Abraham de los hijos de Het; allí fue sepultado Abraham, y Sara su mujer.
\par 11 Y sucedió, después de muerto Abraham, que Dios bendijo a Isaac su hijo; y habitó Isaac junto al pozo del Viviente-que-me-ve.

\section*{Los descendientes de Ismael}

\par 12 Estos son los descendientes de Ismael hijo de Abraham, a quien le dio a luz Agar egipcia, sierva de Sara;
\par 13 estos, pues, son los nombres de los hijos de Ismael, nombrados en el orden de su nacimiento: El primogénito de Ismael, Nebaiot; luego Cedar, Adbeel, Mibsam,
\par 14 Misma, Duma, Massa,
\par 15 Hadar, Tema, Jetur, Nafis y Cedema.
\par 16 Estos son los hijos de Ismael, y estos sus nombres, por sus villas y por sus campamentos; doce príncipes por sus familias.
\par 17 Y estos fueron los años de la vida de Ismael, ciento treinta y siete años; y exhaló el espíritu Ismael, y murió, y fue unido a su pueblo.
\par 18 Y habitaron desde Havila hasta Shur, que está enfrente de Egipto viniendo a Asiria; y murió en presencia de todos sus hermanos.

\section*{Nacimiento de Jacob y Esaú}

\par 19 Estos son los descendientes de Isaac hijo de Abraham: Abraham engendró a Isaac,
\par 20 y era Isaac de cuarenta años cuando tomó por mujer a Rebeca, hija de Betuel arameo de Padan-aram, hermana de Labán arameo.
\par 21 Y oró Isaac a Jehová por su mujer, que era estéril; y lo aceptó Jehová, y concibió Rebeca su mujer.
\par 22 Y los hijos luchaban dentro de ella; y dijo: Si es así, ¿para qué vivo yo? Y fue a consultar a Jehová;
\par 23 y le respondió Jehová:
\par Dos naciones hay en tu seno,
\par Y dos pueblos serán divididos desde tus entrañas;
\par El un pueblo será más fuerte que el otro pueblo,
\par Y el mayor servirá al menor.
\par 24 Cuando se cumplieron sus días para dar a luz, he aquí había gemelos en su vientre.
\par 25 Y salió el primero rubio, y era todo velludo como una pelliza; y llamaron su nombre Esaú.
\par 26 Después salió su hermano, trabada su mano al calcañar de Esaú; y fue llamado su nombre Jacob. Y era Isaac de edad de sesenta años cuando ella los dio a luz.

\section*{Esaú vende su primogenitura}

\par 27 Y crecieron los niños, y Esaú fue diestro en la caza, hombre del campo; pero Jacob era varón quieto, que habitaba en tiendas.
\par 28 Y amó Isaac a Esaú, porque comía de su caza; mas Rebeca amaba a Jacob.
\par 29 Y guisó Jacob un potaje; y volviendo Esaú del campo, cansado,
\par 30 dijo a Jacob: Te ruego que me des a comer de ese guiso rojo, pues estoy muy cansado. Por tanto fue llamado su nombre Edom.
\par 31 Y Jacob respondió: Véndeme en este día tu primogenitura.
\par 32 Entonces dijo Esaú: He aquí yo me voy a morir; ¿para qué, pues, me servirá la primogenitura?
\par 33 Y dijo Jacob: Júramelo en este día. Y él le juró, y vendió a Jacob su primogenitura.
\par 34 Entonces Jacob dio a Esaú pan y del guisado de las lentejas; y él comió y bebió, y se levantó y se fue. Así menospreció Esaú la primogenitura.

\chapter{26}

\section*{Isaac en Gerar}

\par 1 Después hubo hambre en la tierra, además de la primera hambre que hubo en los días de Abraham; y se fue Isaac a Abimelec rey de los filisteos, en Gerar.
\par 2 Y se le apareció Jehová, y le dijo: No desciendas a Egipto; habita en la tierra que yo te diré.
\par 3 Habita como forastero en esta tierra, y estaré contigo, y te bendeciré; porque a ti y a tu descendencia daré todas estas tierras, y confirmaré el juramento que hice a Abraham tu padre.
\par 4 Multiplicaré tu descendencia como las estrellas del cielo, y daré a tu descendencia todas estas tierras; y todas las naciones de la tierra serán benditas en tu simiente,
\par 5 por cuanto oyó Abraham mi voz, y guardó mi precepto, mis mandamientos, mis estatutos y mis leyes.
\par 6 Habitó, pues, Isaac en Gerar.
\par 7 Y los hombres de aquel lugar le preguntaron acerca de su mujer; y él respondió: Es mi hermana; porque tuvo miedo de decir: Es mi mujer; pensando que tal vez los hombres del lugar lo matarían por causa de Rebeca, pues ella era de hermoso aspecto.
\par 8 Sucedió que después que él estuvo allí muchos días, Abimelec, rey de los filisteos, mirando por una ventana, vio a Isaac que acariciaba a Rebeca su mujer.
\par 9 Y llamó Abimelec a Isaac, y dijo: He aquí ella es de cierto tu mujer. ¿Cómo, pues, dijiste: Es mi hermana? E Isaac le respondió: Porque dije: Quizá moriré por causa de ella.
\par 10 Y Abimelec dijo: ¿Por qué nos has hecho esto? Por poco hubiera dormido alguno del pueblo con tu mujer, y hubieras traído sobre nosotros el pecado.
\par 11 Entonces Abimelec mandó a todo el pueblo, diciendo: El que tocare a este hombre o a su mujer, de cierto morirá.
\par 12 Y sembró Isaac en aquella tierra, y cosechó aquel año ciento por uno; y le bendijo Jehová.
\par 13 El varón se enriqueció, y fue prosperado, y se engrandeció hasta hacerse muy poderoso.
\par 14 Y tuvo hato de ovejas, y hato de vacas, y mucha labranza; y los filisteos le tuvieron envidia.
\par 15 Y todos los pozos que habían abierto los criados de Abraham su padre en sus días, los filisteos los habían cegado y llenado de tierra.
\par 16 Entonces dijo Abimelec a Isaac: Apártate de nosotros, porque mucho más poderoso que nosotros te has hecho.
\par 17 E Isaac se fue de allí, y acampó en el valle de Gerar, y habitó allí.
\par 18 Y volvió a abrir Isaac los pozos de agua que habían abierto en los días de Abraham su padre, y que los filisteos habían cegado después de la muerte de Abraham; y los llamó por los nombres que su padre los había llamado.
\par 19 Pero cuando los siervos de Isaac cavaron en el valle, y hallaron allí un pozo de aguas vivas,
\par 20 los pastores de Gerar riñeron con los pastores de Isaac, diciendo: El agua es nuestra. Por eso llamó el nombre del pozo Esek, porque habían altercado con él.
\par 21 Y abrieron otro pozo, y también riñeron sobre él; y llamó su nombre Sitna.
\par 22 Y se apartó de allí, y abrió otro pozo, y no riñeron sobre él; y llamó su nombre Rehobot, y dijo: Porque ahora Jehová nos ha prosperado, y fructificaremos en la tierra.
\par 23 Y de allí subió a Beerseba.
\par 24 Y se le apareció Jehová aquella noche, y le dijo: Yo soy el Dios de Abraham tu padre; no temas, porque yo estoy contigo, y te bendeciré, y multiplicaré tu descendencia por amor de Abraham mi siervo.
\par 25 Y edificó allí un altar, e invocó el nombre de Jehová, y plantó allí su tienda; y abrieron allí los siervos de Isaac un pozo.
\par 26 Y Abimelec vino a él desde Gerar, y Ahuzat, amigo suyo, y Ficol, capitán de su ejército.
\par 27 Y les dijo Isaac: ¿Por qué venís a mí, pues que me habéis aborrecido, y me echasteis de entre vosotros?
\par 28 Y ellos respondieron: Hemos visto que Jehová está contigo; y dijimos: Haya ahora juramento entre nosotros, entre tú y nosotros, y haremos pacto contigo,
\par 29 que no nos hagas mal, como nosotros no te hemos tocado, y como solamente te hemos hecho bien, y te enviamos en paz; tú eres ahora bendito de Jehová.
\par 30 Entonces él les hizo banquete, y comieron y bebieron.
\par 31 Y se levantaron de madrugada, y juraron el uno al otro; e Isaac los despidió, y ellos se despidieron de él en paz.
\par 32 En aquel día sucedió que vinieron los criados de Isaac, y le dieron nuevas acerca del pozo que habían abierto, y le dijeron: Hemos hallado agua.
\par 33 Y lo llamó Seba; por esta causa el nombre de aquella ciudad es Beerseba hasta este día.
\par 34 Y cuando Esaú era de cuarenta años, tomó por mujer a Judit hija de Beeri heteo, y a Basemat hija de Elón heteo;
\par 35 y fueron amargura de espíritu para Isaac y para Rebeca.

\chapter{27}

\section*{Jacob obtiene la bendición de Isaac}

\par 1 Aconteció que cuando Isaac envejeció, y sus ojos se oscurecieron quedando sin vista, llamó a Esaú su hijo mayor, y le dijo: Hijo mío. Y él respondió: Heme aquí.
\par 2 Y él dijo: He aquí ya soy viejo, no sé el día de mi muerte.
\par 3 Toma, pues, ahora tus armas, tu aljaba y tu arco, y sal al campo y tráeme caza;
\par 4 y hazme un guisado como a mí me gusta, y tráemelo, y comeré, para que yo te bendiga antes que muera.
\par 5 Y Rebeca estaba oyendo, cuando hablaba Isaac a Esaú su hijo; y se fue Esaú al campo para buscar la caza que había de traer.
\par 6 Entonces Rebeca habló a Jacob su hijo, diciendo: He aquí yo he oído a tu padre que hablaba con Esaú tu hermano, diciendo:
\par 7 Tráeme caza y hazme un guisado, para que coma, y te bendiga en presencia de Jehová antes que yo muera.
\par 8 Ahora, pues, hijo mío, obedece a mi voz en lo que te mando.
\par 9 Ve ahora al ganado, y tráeme de allí dos buenos cabritos de las cabras, y haré de ellos viandas para tu padre, como a él le gusta;
\par 10 y tú las llevarás a tu padre, y comerá, para que él te bendiga antes de su muerte.
\par 11 Y Jacob dijo a Rebeca su madre: He aquí, Esaú mi hermano es hombre velloso, y yo lampiño.
\par 12 Quizá me palpará mi padre, y me tendrá por burlador, y traeré sobre mí maldición y no bendición.
\par 13 Y su madre respondió: Hijo mío, sea sobre mí tu maldición; solamente obedece a mi voz y ve y tráemelos.
\par 14 Entonces él fue y los tomó, y los trajo a su madre; y su madre hizo guisados, como a su padre le gustaba.
\par 15 Y tomó Rebeca los vestidos de Esaú su hijo mayor, los preciosos, que ella tenía en casa, y vistió a Jacob su hijo menor;
\par 16 y cubrió sus manos y la parte de su cuello donde no tenía vello, con las pieles de los cabritos;
\par 17 y entregó los guisados y el pan que había preparado, en manos de Jacob su hijo.
\par 18 Entonces éste fue a su padre y dijo: Padre mío. E Isaac respondió: Heme aquí; ¿quién eres, hijo mío?
\par 19 Y Jacob dijo a su padre: Yo soy Esaú tu primogénito; he hecho como me dijiste: levántate ahora, y siéntate, y come de mi caza, para que me bendigas.
\par 20 Entonces Isaac dijo a su hijo: ¿Cómo es que la hallaste tan pronto, hijo mío? Y él respondió: Porque Jehová tu Dios hizo que la encontrase delante de mí.
\par 21 E Isaac dijo a Jacob: Acércate ahora, y te palparé, hijo mío, por si eres mi hijo Esaú o no.
\par 22 Y se acercó Jacob a su padre Isaac, quien le palpó, y dijo: La voz es la voz de Jacob, pero las manos, las manos de Esaú.
\par 23 Y no le conoció, porque sus manos eran vellosas como las manos de Esaú; y le bendijo.
\par 24 Y dijo: ¿Eres tú mi hijo Esaú? Y Jacob respondió: Yo soy.
\par 25 Dijo también: Acércamela, y comeré de la caza de mi hijo, para que yo te bendiga; y Jacob se la acercó, e Isaac comió; le trajo también vino, y bebió.
\par 26 Y le dijo Isaac su padre: Acércate ahora, y bésame, hijo mío.
\par 27 Y Jacob se acercó, y le besó; y olió Isaac el olor de sus vestidos, y le bendijo, diciendo:
\par Mira, el olor de mi hijo,
\par Como el olor del campo que Jehová ha bendecido;
\par 28 Dios, pues, te dé del rocío del cielo,
\par Y de las grosuras de la tierra,
\par Y abundancia de trigo y de mosto.
\par 29 Sírvante pueblos,
\par Y naciones se inclinen a ti;
\par Sé señor de tus hermanos,
\par Y se inclinen ante ti los hijos de tu madre.
\par Malditos los que te maldijeren,
\par Y benditos los que te bendijeren. m
\par 30 Y aconteció, luego que Isaac acabó de bendecir a Jacob, y apenas había salido Jacob de delante de Isaac su padre, que Esaú su hermano volvió de cazar.
\par 31 E hizo él también guisados, y trajo a su padre, y le dijo: Levántese mi padre, y coma de la caza de su hijo, para que me bendiga.
\par 32 Entonces Isaac su padre le dijo: ¿Quién eres tú? Y él le dijo: Yo soy tu hijo, tu primogénito, Esaú.
\par 33 Y se estremeció Isaac grandemente, y dijo: ¿Quién es el que vino aquí, que trajo caza, y me dio, y comí de todo antes que tú vinieses? Yo le bendije, y será bendito.
\par 34 Cuando Esaú oyó las palabras de su padre, clamó con una muy grande y muy amarga exclamación, y le dijo: Bendíceme también a mí, padre mío.
\par 35 Y él dijo: Vino tu hermano con engaño, y tomó tu bendición.
\par 36 Y Esaú respondió: Bien llamaron su nombre Jacob, pues ya me ha suplantado dos veces: se apoderó de mi primogenitura, y he aquí ahora ha tomado mi bendición. Y dijo: ¿No has guardado bendición para mí?
\par 37 Isaac respondió y dijo a Esaú: He aquí yo le he puesto por señor tuyo, y le he dado por siervos a todos sus hermanos; de trigo y de vino le he provisto; ¿qué, pues, te haré a ti ahora, hijo mío?
\par 38 Y Esaú respondió a su padre: ¿No tienes más que una sola bendición, padre mío? Bendíceme también a mí, padre mío. Y alzó Esaú su voz, y lloró.
\par 39 Entonces Isaac su padre habló y le dijo:
\par He aquí, será tu habitación en grosuras de la tierra,
\par Y del rocío de los cielos de arriba;
\par 40 Y por tu espada vivirás, y a tu hermano servirás;
\par Y sucederá cuando te fortalezcas,
\par Que descargarás su yugo de tu cerviz.

\section*{Jacob huye de Esaú}

\par 41 Y aborreció Esaú a Jacob por la bendición con que su padre le había bendecido, y dijo en su corazón: Llegarán los días del luto de mi padre, y yo mataré a mi hermano Jacob.
\par 42 Y fueron dichas a Rebeca las palabras de Esaú su hijo mayor; y ella envió y llamó a Jacob su hijo menor, y le dijo: He aquí, Esaú tu hermano se consuela acerca de ti con la idea de matarte.
\par 43 Ahora pues, hijo mío, obedece a mi voz; levántate y huye a casa de Labán mi hermano en Harán,
\par 44 y mora con él algunos días, hasta que el enojo de tu hermano se mitigue;
\par 45 hasta que se aplaque la ira de tu hermano contra ti, y olvide lo que le has hecho; yo enviaré entonces, y te traeré de allá. ¿Por qué seré privada de vosotros ambos en un día?
\par 46 Y dijo Rebeca a Isaac: Fastidio tengo de mi vida, a causa de las hijas de Het. Si Jacob toma mujer de las hijas de Het, como éstas, de las hijas de esta tierra, ¿para qué quiero la vida?

\chapter{28}

\par 1 Entonces Isaac llamó a Jacob, y lo bendijo, y le mandó diciendo: No tomes mujer de las hijas de Canaán.
\par 2 Levántate, ve a Padan-aram, a casa de Betuel, padre de tu madre, y toma allí mujer de las hijas de Labán, hermano de tu madre.
\par 3 Y el Dios omnipotente te bendiga, y te haga fructificar y te multiplique, hasta llegar a ser multitud de pueblos;
\par 4 y te dé la bendición de Abraham, y a tu descendencia contigo, para que heredes la tierra en que moras, que Dios dio a Abraham.
\par 5 Así envió Isaac a Jacob, el cual fue a Padan-aram, a Labán hijo de Betuel arameo, hermano de Rebeca madre de Jacob y de Esaú.
\par 6 Y vio Esaú cómo Isaac había bendecido a Jacob, y le había enviado a Padan-aram, para tomar para sí mujer de allí; y que cuando le bendijo, le había mandado diciendo: No tomarás mujer de las hijas de Canaán;
\par 7 y que Jacob había obedecido a su padre y a su madre, y se había ido a Padan-aram.
\par 8 Vio asimismo Esaú que las hijas de Canaán parecían mal a Isaac su padre;
\par 9 y se fue Esaú a Ismael, y tomó para sí por mujer a Mahalat, hija de Ismael hijo de Abraham, hermana de Nebaiot, además de sus otras mujeres.

\section*{Dios se aparece a Jacob en Bet-el}

\par 10 Salió, pues, Jacob de Beerseba, y fue a Harán.
\par 11 Y llegó a un cierto lugar, y durmió allí, porque ya el sol se había puesto; y tomó de las piedras de aquel paraje y puso a su cabecera, y se acostó en aquel lugar.
\par 12 Y soñó: y he aquí una escalera que estaba apoyada en tierra, y su extremo tocaba en el cielo; y he aquí ángeles de Dios que subían y descendían por ella.
\par 13 Y he aquí, Jehová estaba en lo alto de ella, el cual dijo: Yo soy Jehová, el Dios de Abraham tu padre, y el Dios de Isaac; la tierra en que estás acostado te la daré a ti y a tu descendencia.
\par 14 Será tu descendencia como el polvo de la tierra, y te extenderás al occidente, al oriente, al norte y al sur; y todas las familias de la tierra serán benditas en ti y en tu simiente.
\par 15 He aquí, yo estoy contigo, y te guardaré por dondequiera que fueres, y volveré a traerte a esta tierra; porque no te dejaré hasta que haya hecho lo que te he dicho.
\par 16 Y despertó Jacob de su sueño, y dijo: Ciertamente Jehová está en este lugar, y yo no lo sabía.
\par 17 Y tuvo miedo, y dijo: !!Cuán terrible es este lugar! No es otra cosa que casa de Dios, y puerta del cielo.
\par 18 Y se levantó Jacob de mañana, y tomó la piedra que había puesto de cabecera, y la alzó por señal, y derramó aceite encima de ella.
\par 19 Y llamó el nombre de aquel lugar Bet-el, aunque Luz era el nombre de la ciudad primero.
\par 20 E hizo Jacob voto, diciendo: Si fuere Dios conmigo, y me guardare en este viaje en que voy, y me diere pan para comer y vestido para vestir,
\par 21 y si volviere en paz a casa de mi padre, Jehová será mi Dios.
\par 22 Y esta piedra que he puesto por señal, será casa de Dios; y de todo lo que me dieres, el diezmo apartaré para ti.

\chapter{29}

\section*{Jacob sirve a Labán por Raquel y Lea}

\par 1 Siguió luego Jacob su camino, y fue a la tierra de los orientales.
\par 2 Y miró, y vio un pozo en el campo; y he aquí tres rebaños de ovejas que yacían cerca de él, porque de aquel pozo abrevaban los ganados; y había una gran piedra sobre la boca del pozo.
\par 3 Y juntaban allí todos los rebaños; y revolvían la piedra de la boca del pozo, y abrevaban las ovejas, y volvían la piedra sobre la boca del pozo a su lugar.
\par 4 Y les dijo Jacob: Hermanos míos, ¿de dónde sois? Y ellos respondieron: De Harán somos.
\par 5 El les dijo: ¿Conocéis a Labán hijo de Nacor? Y ellos dijeron: Sí, le conocemos.
\par 6 Y él les dijo: ¿Está bien? Y ellos dijeron: Bien, y he aquí Raquel su hija viene con las ovejas.
\par 7 Y él dijo: He aquí es aún muy de día; no es tiempo todavía de recoger el ganado; abrevad las ovejas, e id a apacentarlas.
\par 8 Y ellos respondieron: No podemos, hasta que se junten todos los rebaños, y remuevan la piedra de la boca del pozo, para que abrevemos las ovejas.
\par 9 Mientras él aún hablaba con ellos, Raquel vino con el rebaño de su padre, porque ella era la pastora.
\par 10 Y sucedió que cuando Jacob vio a Raquel, hija de Labán hermano de su madre, y las ovejas de Labán el hermano de su madre, se acercó Jacob y removió la piedra de la boca del pozo, y abrevó el rebaño de Labán hermano de su madre.
\par 11 Y Jacob besó a Raquel, y alzó su voz y lloró.
\par 12 Y Jacob dijo a Raquel que él era hermano de su padre, y que era hijo de Rebeca; y ella corrió, y dio las nuevas a su padre.
\par 13 Así que oyó Labán las nuevas de Jacob, hijo de su hermana, corrió a recibirlo, y lo abrazó, lo besó, y lo trajo a su casa; y él contó a Labán todas estas cosas.
\par 14 Y Labán le dijo: Ciertamente hueso mío y carne mía eres. Y estuvo con él durante un mes.
\par 15 Entonces dijo Labán a Jacob: ¿Por ser tú mi hermano, me servirás de balde? Dime cuál será tu salario.
\par 16 Y Labán tenía dos hijas: el nombre de la mayor era Lea, y el nombre de la menor, Raquel.
\par 17 Y los ojos de Lea eran delicados, pero Raquel era de lindo semblante y de hermoso parecer.
\par 18 Y Jacob amó a Raquel, y dijo: Yo te serviré siete años por Raquel tu hija menor.
\par 19 Y Labán respondió: Mejor es que te la dé a ti, y no que la dé a otro hombre; quédate conmigo.
\par 20 Así sirvió Jacob por Raquel siete años; y le parecieron como pocos días, porque la amaba.
\par 21 Entonces dijo Jacob a Labán: Dame mi mujer, porque mi tiempo se ha cumplido, para unirme a ella.
\par 22 Entonces Labán juntó a todos los varones de aquel lugar, e hizo banquete.
\par 23 Y sucedió que a la noche tomó a Lea su hija, y se la trajo; y él se llegó a ella.
\par 24 Y dio Labán su sierva Zilpa a su hija Lea por criada.
\par 25 Venida la mañana, he aquí que era Lea; y Jacob dijo a Labán: ¿Qué es esto que me has hecho? ¿No te he servido por Raquel? ¿Por qué, pues, me has engañado?
\par 26 Y Labán respondió: No se hace así en nuestro lugar, que se dé la menor antes de la mayor.
\par 27 Cumple la semana de ésta, y se te dará también la otra, por el servicio que hagas conmigo otros siete años.
\par 28 E hizo Jacob así, y cumplió la semana de aquélla; y él le dio a Raquel su hija por mujer.
\par 29 Y dio Labán a Raquel su hija su sierva Bilha por criada.
\par 30 Y se llegó también a Raquel, y la amó también más que a Lea; y sirvió a Labán aún otros siete años.

\section*{Los hijos de Jacob}

\par 31 Y vio Jehová que Lea era menospreciada, y le dio hijos; pero Raquel era estéril.
\par 32 Y concibió Lea, y dio a luz un hijo, y llamó su nombre Rubén, porque dijo: Ha mirado Jehová mi aflicción; ahora, por tanto, me amará mi marido.
\par 33 Concibió otra vez, y dio a luz un hijo, y dijo: Por cuanto oyó Jehová que yo era menospreciada, me ha dado también éste. Y llamó su nombre Simeón.
\par 34 Y concibió otra vez, y dio a luz un hijo, y dijo: Ahora esta vez se unirá mi marido conmigo, porque le he dado a luz tres hijos; por tanto, llamó su nombre Leví.
\par 35 Concibió otra vez, y dio a luz un hijo, y dijo: Esta vez alabaré a Jehová; por esto llamó su nombre Judá; y dejó de dar a luz.

\chapter{30}

\par 1 Viendo Raquel que no daba hijos a Jacob, tuvo envidia de su hermana, y decía a Jacob: Dame hijos, o si no, me muero.
\par 2 Y Jacob se enojó contra Raquel, y dijo: ¿Soy yo acaso Dios, que te impidió el fruto de tu vientre?
\par 3 Y ella dijo: He aquí mi sierva Bilha; llégate a ella, y dará a luz sobre mis rodillas, y yo también tendré hijos de ella.
\par 4 Así le dio a Bilha su sierva por mujer; y Jacob se llegó a ella.
\par 5 Y concibió Bilha, y dio a luz un hijo a Jacob.
\par 6 Dijo entonces Raquel: Me juzgó Dios, y también oyó mi voz, y me dio un hijo. Por tanto llamó su nombre Dan.
\par 7 Concibió otra vez Bilha la sierva de Raquel, y dio a luz un segundo hijo a Jacob.
\par 8 Y dijo Raquel: Con luchas de Dios he contendido con mi hermana, y he vencido. Y llamó su nombre Neftalí.
\par 9 Viendo, pues, Lea, que había dejado de dar a luz, tomó a Zilpa su sierva, y la dio a Jacob por mujer.
\par 10 Y Zilpa sierva de Lea dio a luz un hijo a Jacob.
\par 11 Y dijo Lea: Vino la ventura; y llamó su nombre Gad.
\par 12 Luego Zilpa la sierva de Lea dio a luz otro hijo a Jacob.
\par 13 Y dijo Lea: Para dicha mía; porque las mujeres me dirán dichosa; y llamó su nombre Aser.
\par 14 Fue Rubén en tiempo de la siega de los trigos, y halló mandrágoras en el campo, y las trajo a Lea su madre; y dijo Raquel a Lea: Te ruego que me des de las mandrágoras de tu hijo.
\par 15 Y ella respondió: ¿Es poco que hayas tomado mi marido, sino que también te has de llevar las mandrágoras de mi hijo? Y dijo Raquel: Pues dormirá contigo esta noche por las mandrágoras de tu hijo.
\par 16 Cuando, pues, Jacob volvía del campo a la tarde, salió Lea a él, y le dijo: Llégate a mí, porque a la verdad te he alquilado por las mandrágoras de mi hijo. Y durmió con ella aquella noche.
\par 17 Y oyó Dios a Lea; y concibió, y dio a luz el quinto hijo a Jacob.
\par 18 Y dijo Lea: Dios me ha dado mi recompensa, por cuanto di mi sierva a mi marido; por eso llamó su nombre Isacar.
\par 19 Después concibió Lea otra vez, y dio a luz el sexto hijo a Jacob.
\par 20 Y dijo Lea: Dios me ha dado una buena dote; ahora morará conmigo mi marido, porque le he dado a luz seis hijos; y llamó su nombre Zabulón.
\par 21 Después dio a luz una hija, y llamó su nombre Dina.
\par 22 Y se acordó Dios de Raquel, y la oyó Dios, y le concedió hijos.
\par 23 Y concibió, y dio a luz un hijo, y dijo: Dios ha quitado mi afrenta;
\par 24 y llamó su nombre José, diciendo: Añádame Jehová otro hijo.

\section*{Tretas de Jacob y de Labán}

\par 25 Aconteció cuando Raquel hubo dado a luz a José, que Jacob dijo a Labán: Envíame, e iré a mi lugar, y a mi tierra.
\par 26 Dame mis mujeres y mis hijos, por las cuales he servido contigo, y déjame ir; pues tú sabes los servicios que te he hecho.
\par 27 Y Labán le respondió: Halle yo ahora gracia en tus ojos, y quédate; he experimentado que Jehová me ha bendecido por tu causa.
\par 28 Y dijo: Señálame tu salario, y yo lo daré.
\par 29 Y él respondió: Tú sabes cómo te he servido, y cómo ha estado tu ganado conmigo.
\par 30 Porque poco tenías antes de mi venida, y ha crecido en gran número, y Jehová te ha bendecido con mi llegada; y ahora, ¿cuándo trabajaré también por mi propia casa?
\par 31 Y él dijo: ¿Qué te daré? Y respondió Jacob: No me des nada; si hicieres por mí esto, volveré a apacentar tus ovejas.
\par 32 Yo pasaré hoy por todo tu rebaño, poniendo aparte todas las ovejas manchadas y salpicadas de color, y todas las ovejas de color oscuro, y las manchadas y salpicadas de color entre las cabras; y esto será mi salario.
\par 33 Así responderá por mí mi honradez mañana, cuando vengas a reconocer mi salario; toda la que no fuere pintada ni manchada en las cabras, y de color oscuro entre mis ovejas, se me ha de tener como de hurto.
\par 34 Dijo entonces Labán: Mira, sea como tú dices.
\par 35 Y Labán apartó aquel día los machos cabríos manchados y rayados, y todas las cabras manchadas y salpicadas de color, y toda aquella que tenía en sí algo de blanco, y todas las de color oscuro entre las ovejas, y las puso en mano de sus hijos.
\par 36 Y puso tres días de camino entre sí y Jacob; y Jacob apacentaba las otras ovejas de Labán.
\par 37 Tomó luego Jacob varas verdes de álamo, de avellano y de castaño, y descortezó en ellas mondaduras blancas, descubriendo así lo blanco de las varas.
\par 38 Y puso las varas que había mondado delante del ganado, en los canales de los abrevaderos del agua donde venían a beber las ovejas, las cuales procreaban cuando venían a beber.
\par 39 Así concebían las ovejas delante de las varas; y parían borregos listados, pintados y salpicados de diversos colores.
\par 40 Y apartaba Jacob los corderos, y ponía con su propio rebaño los listados y todo lo que era oscuro del hato de Labán. Y ponía su hato aparte, y no lo ponía con las ovejas de Labán.
\par 41 Y sucedía que cuantas veces se hallaban en celo las ovejas más fuertes, Jacob ponía las varas delante de las ovejas en los abrevaderos, para que concibiesen a la vista de las varas.
\par 42 Pero cuando venían las ovejas más débiles, no las ponía; así eran las más débiles para Labán, y las más fuertes para Jacob.
\par 43 Y se enriqueció el varón muchísimo, y tuvo muchas ovejas, y siervas y siervos, y camellos y asnos.

\chapter{31}

\par 1 Y oía Jacob las palabras de los hijos de Labán, que decían: Jacob ha tomado todo lo que era de nuestro padre, y de lo que era de nuestro padre ha adquirido toda esta riqueza.
\par 2 Miraba también Jacob el semblante de Labán, y veía que no era para con él como había sido antes.
\par 3 También Jehová dijo a Jacob: Vuélvete a la tierra de tus padres, y a tu parentela, y yo estaré contigo.
\par 4 Envió, pues, Jacob, y llamó a Raquel y a Lea al campo donde estaban sus ovejas,
\par 5 y les dijo: Veo que el semblante de vuestro padre no es para conmigo como era antes; mas el Dios de mi padre ha estado conmigo.
\par 6 Vosotras sabéis que con todas mis fuerzas he servido a vuestro padre;
\par 7 y vuestro padre me ha engañado, y me ha cambiado el salario diez veces; pero Dios no le ha permitido que me hiciese mal.
\par 8 Si él decía así: Los pintados serán tu salario, entonces todas las ovejas parían pintados; y si decía así: Los listados serán tu salario; entonces todas las ovejas parían listados.
\par 9 Así quitó Dios el ganado de vuestro padre, y me lo dio a mí.
\par 10 Y sucedió que al tiempo que las ovejas estaban en celo, alcé yo mis ojos y vi en sueños, y he aquí los machos que cubrían a las hembras eran listados, pintados y abigarrados.
\par 11 Y me dijo el ángel de Dios en sueños: Jacob. Y yo dije: Heme aquí.
\par 12 Y él dijo: Alza ahora tus ojos, y verás que todos los machos que cubren a las hembras son listados, pintados y abigarrados; porque yo he visto todo lo que Labán te ha hecho.
\par 13 Yo soy el Dios de Bet-el, donde tú ungiste la piedra, y donde me hiciste un voto. Levántate ahora y sal de esta tierra, y vuélvete a la tierra de tu nacimiento.
\par 14 Respondieron Raquel y Lea, y le dijeron: ¿Tenemos acaso parte o heredad en la casa de nuestro padre?
\par 15 ¿No nos tiene ya como por extrañas, pues que nos vendió, y aun se ha comido del todo nuestro precio?
\par 16 Porque toda la riqueza que Dios ha quitado a nuestro padre, nuestra es y de nuestros hijos; ahora, pues, haz todo lo que Dios te ha dicho.

\section*{Jacob huye de Labán}

\par 17 Entonces se levantó Jacob, y subió sus hijos y sus mujeres sobre los camellos,
\par 18 y puso en camino todo su ganado, y todo cuanto había adquirido, el ganado de su ganancia que había obtenido en Padan-aram, para volverse a Isaac su padre en la tierra de Canaán.
\par 19 Pero Labán había ido a trasquilar sus ovejas; y Raquel hurtó los ídolos de su padre.
\par 20 Y Jacob engañó a Labán arameo, no haciéndole saber que se iba.
\par 21 Huyó, pues, con todo lo que tenía; y se levantó y pasó el Eufrates, y se dirigió al monte de Galaad.
\par 22 Y al tercer día fue dicho a Labán que Jacob había huido.
\par 23 Entonces Labán tomó a sus parientes consigo, y fue tras Jacob camino de siete días, y le alcanzó en el monte de Galaad.
\par 24 Y vino Dios a Labán arameo en sueños aquella noche, y le dijo: Guárdate que no hables a Jacob descomedidamente.
\par 25 Alcanzó, pues, Labán a Jacob; y éste había fijado su tienda en el monte; y Labán acampó con sus parientes en el monte de Galaad.
\par 26 Y dijo Labán a Jacob: ¿Qué has hecho, que me engañaste, y has traído a mis hijas como prisioneras de guerra?
\par 27 ¿Por qué te escondiste para huir, y me engañaste, y no me lo hiciste saber para que yo te despidiera con alegría y con cantares, con tamborín y arpa?
\par 28 Pues ni aun me dejaste besar a mis hijos y mis hijas. Ahora, locamente has hecho.
\par 29 Poder hay en mi mano para haceros mal; mas el Dios de tu padre me habló anoche diciendo: Guárdate que no hables a Jacob descomedidamente.
\par 30 Y ya que te ibas, porque tenías deseo de la casa de tu padre, ¿por qué me hurtaste mis dioses?
\par 31 Respondió Jacob y dijo a Labán: Porque tuve miedo; pues pensé que quizá me quitarías por fuerza tus hijas.
\par 32 Aquel en cuyo poder hallares tus dioses, no viva; delante de nuestros hermanos reconoce lo que yo tenga tuyo, y llévatelo. Jacob no sabía que Raquel los había hurtado.
\par 33 Entró Labán en la tienda de Jacob, en la tienda de Lea, y en la tienda de las dos siervas, y no los halló; y salió de la tienda de Lea, y entró en la tienda de Raquel.
\par 34 Pero tomó Raquel los ídolos y los puso en una albarda de un camello, y se sentó sobre ellos; y buscó Labán en toda la tienda, y no los halló.
\par 35 Y ella dijo a su padre: No se enoje mi señor, porque no me puedo levantar delante de ti; pues estoy con la costumbre de las mujeres. Y él buscó, pero no halló los ídolos.
\par 36 Entonces Jacob se enojó, y riñó con Labán; y respondió Jacob y dijo a Labán: ¿Qué transgresión es la mía? ¿Cuál es mi pecado, para que con tanto ardor hayas venido en mi persecución?
\par 37 Pues que has buscado en todas mis cosas, ¿qué has hallado de todos los enseres de tu casa? Ponlo aquí delante de mis hermanos y de los tuyos, y juzguen entre nosotros.
\par 38 Estos veinte años he estado contigo; tus ovejas y tus cabras nunca abortaron, ni yo comí carnero de tus ovejas.
\par 39 Nunca te traje lo arrebatado por las fieras: yo pagaba el daño; lo hurtado así de día como de noche, a mí me lo cobrabas.
\par 40 De día me consumía el calor, y de noche la helada, y el sueño huía de mis ojos.
\par 41 Así he estado veinte años en tu casa; catorce años te serví por tus dos hijas, y seis años por tu ganado, y has cambiado mi salario diez veces.
\par 42 Si el Dios de mi padre, Dios de Abraham y temor de Isaac, no estuviera conmigo, de cierto me enviarías ahora con las manos vacías; pero Dios vio mi aflicción y el trabajo de mis manos, y te reprendió anoche.
\par 43 Respondió Labán y dijo a Jacob: Las hijas son hijas mías, y los hijos, hijos míos son, y las ovejas son mis ovejas, y todo lo que tú ves es mío: ¿y qué puedo yo hacer hoy a estas mis hijas, o a sus hijos que ellas han dado a luz?
\par 44 Ven, pues, ahora, y hagamos pacto tú y yo, y sea por testimonio entre nosotros dos.
\par 45 Entonces Jacob tomó una piedra, y la levantó por señal.
\par 46 Y dijo Jacob a sus hermanos: Recoged piedras. Y tomaron piedras e hicieron un majano, y comieron allí sobre aquel majano.
\par 47 Y lo llamó Labán, Jegar Sahaduta; y lo llamó Jacob, Galaad.
\par 48 Porque Labán dijo: Este majano es testigo hoy entre nosotros dos; por eso fue llamado su nombre Galaad;
\par 49 y Mizpa, por cuanto dijo: Atalaye Jehová entre tú y yo, cuando nos apartemos el uno del otro.
\par 50 Si afligieres a mis hijas, o si tomares otras mujeres además de mis hijas, nadie está con nosotros; mira, Dios es testigo entre nosotros dos.
\par 51 Dijo más Labán a Jacob: He aquí este majano, y he aquí esta señal, que he erigido entre tú y yo.
\par 52 Testigo sea este majano, y testigo sea esta señal, que ni yo pasaré de este majano contra ti, ni tú pasarás de este majano ni de esta señal contra mí, para mal.
\par 53 El Dios de Abraham y el Dios de Nacor juzgue entre nosotros, el Dios de sus padres. Y Jacob juró por aquel a quien temía Isaac su padre.
\par 54 Entonces Jacob inmoló víctimas en el monte, y llamó a sus hermanos a comer pan; y comieron pan, y durmieron aquella noche en el monte.
\par 55 Y se levantó Labán de mañana, y besó sus hijos y sus hijas, y los bendijo; y regresó y se volvió a su lugar.

\chapter{32}

\section*{Jacob se prepara para el encuentro con Esaú}

\par 1 Jacob siguió su camino, y le salieron al encuentro ángeles de Dios.
\par 2 Y dijo Jacob cuando los vio: Campamento de Dios es este; y llamó el nombre de aquel lugar Mahanaim.
\par 3 Y envió Jacob mensajeros delante de sí a Esaú su hermano, a la tierra de Seir, campo de Edom.
\par 4 Y les mandó diciendo: Así diréis a mi señor Esaú: Así dice tu siervo Jacob: Con Labán he morado, y me he detenido hasta ahora;
\par 5 y tengo vacas, asnos, ovejas, y siervos y siervas; y envío a decirlo a mi señor, para hallar gracia en tus ojos.
\par 6 Y los mensajeros volvieron a Jacob, diciendo: Vinimos a tu hermano Esaú, y él también viene a recibirte, y cuatrocientos hombres con él.
\par 7 Entonces Jacob tuvo gran temor, y se angustió; y distribuyó el pueblo que tenía consigo, y las ovejas y las vacas y los camellos, en dos campamentos.
\par 8 Y dijo: Si viene Esaú contra un campamento y lo ataca, el otro campamento escapará.
\par 9 Y dijo Jacob: Dios de mi padre Abraham, y Dios de mi padre Isaac, Jehová, que me dijiste: Vuélvete a tu tierra y a tu parentela, y yo te haré bien;
\par 10 menor soy que todas las misericordias y que toda la verdad que has usado para con tu siervo; pues con mi cayado pasé este Jordán, y ahora estoy sobre dos campamentos.
\par 11 Líbrame ahora de la mano de mi hermano, de la mano de Esaú, porque le temo; no venga acaso y me hiera la madre con los hijos.
\par 12 Y tú has dicho: Yo te haré bien, y tu descendencia será como la arena del mar, que no se puede contar por la multitud.
\par 13 Y durmió allí aquella noche, y tomó de lo que le vino a la mano un presente para su hermano Esaú:
\par 14 doscientas cabras y veinte machos cabríos, doscientas ovejas y veinte carneros,
\par 15 treinta camellas paridas con sus crías, cuarenta vacas y diez novillos, veinte asnas y diez borricos.
\par 16 Y lo entregó a sus siervos, cada manada de por sí; y dijo a sus siervos: Pasad delante de mí, y poned espacio entre manada y manada.
\par 17 Y mandó al primero, diciendo: Si Esaú mi hermano te encontrare, y te preguntare, diciendo: ¿De quién eres? ¿y adónde vas? ¿y para quién es esto que llevas delante de ti?
\par 18 entonces dirás: Es un presente de tu siervo Jacob, que envía a mi señor Esaú; y he aquí también él viene tras nosotros.
\par 19 Mandó también al segundo, y al tercero, y a todos los que iban tras aquellas manadas, diciendo: Conforme a esto hablaréis a Esaú, cuando le hallareis.
\par 20 Y diréis también: He aquí tu siervo Jacob viene tras nosotros. Porque dijo: Apaciguaré su ira con el presente que va delante de mí, y después veré su rostro; quizá le seré acepto.
\par 21 Pasó, pues, el presente delante de él; y él durmió aquella noche en el campamento.

\section*{Jacob lucha con el ángel en Peniel}

\par 22 Y se levantó aquella noche, y tomó sus dos mujeres, y sus dos siervas, y sus once hijos, y pasó el vado de Jaboc.
\par 23 Los tomó, pues, e hizo pasar el arroyo a ellos y a todo lo que tenía.
\par 24 Así se quedó Jacob solo; y luchó con él un varón hasta que rayaba el alba.
\par 25 Y cuando el varón vio que no podía con él, tocó en el sitio del encaje de su muslo, y se descoyuntó el muslo de Jacob mientras con él luchaba.
\par 26 Y dijo: Déjame, porque raya el alba. Y Jacob le respondió: No te dejaré, si no me bendices.
\par 27 Y el varón le dijo: ¿Cuál es tu nombre? Y él respondió: Jacob.
\par 28 Y el varón le dijo: No se dirá más tu nombre Jacob, sino Israel; porque has luchado con Dios y con los hombres, y has vencido.
\par 29 Entonces Jacob le preguntó, y dijo: Declárame ahora tu nombre. Y el varón respondió: ¿Por qué me preguntas por mi nombre? Y lo bendijo allí.
\par 30 Y llamó Jacob el nombre de aquel lugar, Peniel; porque dijo: Vi a Dios cara a cara, y fue librada mi alma.
\par 31 Y cuando había pasado Peniel, le salió el sol; y cojeaba de su cadera.
\par 32 Por esto no comen los hijos de Israel, hasta hoy día, del tendón que se contrajo, el cual está en el encaje del muslo; porque tocó a Jacob este sitio de su muslo en el tendón que se contrajo.

\chapter{33}

\section*{Reconciliación entre Jacob y Esaú}

\par 1 Alzando Jacob sus ojos, miró, y he aquí venía Esaú, y los cuatrocientos hombres con él; entonces repartió él los niños entre Lea y Raquel y las dos siervas.
\par 2 Y puso las siervas y sus niños delante, luego a Lea y sus niños, y a Raquel y a José los últimos.
\par 3 Y él pasó delante de ellos y se inclinó a tierra siete veces, hasta que llegó a su hermano.
\par 4 Pero Esaú corrió a su encuentro y le abrazó, y se echó sobre su cuello, y le besó; y lloraron.
\par 5 Y alzó sus ojos y vio a las mujeres y los niños, y dijo: ¿Quiénes son éstos? Y él respondió: Son los niños que Dios ha dado a tu siervo.
\par 6 Luego vinieron las siervas, ellas y sus niños, y se inclinaron.
\par 7 Y vino Lea con sus niños, y se inclinaron; y después llegó José y Raquel, y también se inclinaron.
\par 8 Y Esaú dijo: ¿Qué te propones con todos estos grupos que he encontrado? Y Jacob respondió: El hallar gracia en los ojos de mi señor.
\par 9 Y dijo Esaú: Suficiente tengo yo, hermano mío; sea para ti lo que es tuyo.
\par 10 Y dijo Jacob: No, yo te ruego; si he hallado ahora gracia en tus ojos, acepta mi presente, porque he visto tu rostro, como si hubiera visto el rostro de Dios, pues que con tanto favor me has recibido.
\par 11 Acepta, te ruego, mi presente que te he traído, porque Dios me ha hecho merced, y todo lo que hay aquí es mío. E insistió con él, y Esaú lo tomó.
\par 12 Y Esaú dijo: Anda, vamos; y yo iré delante de ti.
\par 13 Y Jacob le dijo: Mi señor sabe que los niños son tiernos, y que tengo ovejas y vacas paridas; y si las fatigan, en un día morirán todas las ovejas.
\par 14 Pase ahora mi señor delante de su siervo, y yo me iré poco a poco al paso del ganado que va delante de mí, y al paso de los niños, hasta que llegue a mi señor a Seir.
\par 15 Y Esaú dijo: Dejaré ahora contigo de la gente que viene conmigo. Y Jacob dijo: ¿Para qué esto? Halle yo gracia en los ojos de mi señor.
\par 16 Así volvió Esaú aquel día por su camino a Seir.
\par 17 Y Jacob fue a Sucot, y edificó allí casa para sí, e hizo cabañas para su ganado; por tanto, llamó el nombre de aquel lugar Sucot.
\par 18 Después Jacob llegó sano y salvo a la ciudad de Siquem, que está en la tierra de Canaán, cuando venía de Padan-aram; y acampó delante de la ciudad.
\par 19 Y compró una parte del campo, donde plantó su tienda, de mano de los hijos de Hamor padre de Siquem, por cien monedas.
\par 20 Y erigió allí un altar, y lo llamó El-Elohe-Israel.

\chapter{34}

\section*{La deshonra de Dina vengada}

\par 1 Salió Dina la hija de Lea, la cual ésta había dado a luz a Jacob, a ver a las hijas del país.
\par 2 Y la vio Siquem hijo de Hamor heveo, príncipe de aquella tierra, y la tomó, y se acostó con ella, y la deshonró.
\par 3 Pero su alma se apegó a Dina la hija de Lea, y se enamoró de la joven, y habló al corazón de ella.
\par 4 Y habló Siquem a Hamor su padre, diciendo: Tómame por mujer a esta joven.
\par 5 Pero oyó Jacob que Siquem había amancillado a Dina su hija; y estando sus hijos con su ganado en el campo, calló Jacob hasta que ellos viniesen.
\par 6 Y se dirigió Hamor padre de Siquem a Jacob, para hablar con él.
\par 7 Y los hijos de Jacob vinieron del campo cuando lo supieron; y se entristecieron los varones, y se enojaron mucho, porque hizo vileza en Israel acostándose con la hija de Jacob, lo que no se debía haber hecho.
\par 8 Y Hamor habló con ellos, diciendo: El alma de mi hijo Siquem se ha apegado a vuestra hija; os ruego que se la deis por mujer.
\par 9 Y emparentad con nosotros; dadnos vuestras hijas, y tomad vosotros las nuestras.
\par 10 Y habitad con nosotros, porque la tierra estará delante de vosotros; morad y negociad en ella, y tomad en ella posesión.
\par 11 Siquem también dijo al padre de Dina y a los hermanos de ella: Halle yo gracia en vuestros ojos, y daré lo que me dijereis.
\par 12 Aumentad a cargo mío mucha dote y dones, y yo daré cuanto me dijereis; y dadme la joven por mujer.
\par 13 Pero respondieron los hijos de Jacob a Siquem y a Hamor su padre con palabras engañosas, por cuanto había amancillado a Dina su hermana.
\par 14 Y les dijeron: No podemos hacer esto de dar nuestra hermana a hombre incircunciso, porque entre nosotros es abominación.
\par 15 Mas con esta condición os complaceremos: si habéis de ser como nosotros, que se circuncide entre vosotros todo varón.
\par 16 Entonces os daremos nuestras hijas, y tomaremos nosotros las vuestras; y habitaremos con vosotros, y seremos un pueblo.
\par 17 Mas si no nos prestareis oído para circuncidaros, tomaremos nuestra hija y nos iremos.
\par 18 Y parecieron bien sus palabras a Hamor, y a Siquem hijo de Hamor.
\par 19 Y no tardó el joven en hacer aquello, porque la hija de Jacob le había agradado; y él era el más distinguido de toda la casa de su padre.
\par 20 Entonces Hamor y Siquem su hijo vinieron a la puerta de su ciudad, y hablaron a los varones de su ciudad, diciendo:
\par 21 Estos varones son pacíficos con nosotros, y habitarán en el país, y traficarán en él; pues he aquí la tierra es bastante ancha para ellos; nosotros tomaremos sus hijas por mujeres, y les daremos las nuestras.
\par 22 Mas con esta condición consentirán estos hombres en habitar con nosotros, para que seamos un pueblo: que se circuncide todo varón entre nosotros, así como ellos son circuncidados.
\par 23 Su ganado, sus bienes y todas sus bestias serán nuestros; solamente convengamos con ellos, y habitarán con nosotros.
\par 24 Y obedecieron a Hamor y a Siquem su hijo todos los que salían por la puerta de la ciudad, y circuncidaron a todo varón, a cuantos salían por la puerta de su ciudad.
\par 25 Pero sucedió que al tercer día, cuando sentían ellos el mayor dolor, dos de los hijos de Jacob, Simeón y Leví, hermanos de Dina, tomaron cada uno su espada, y vinieron contra la ciudad, que estaba desprevenida, y mataron a todo varón.
\par 26 Y a Hamor y a Siquem su hijo los mataron a filo de espada; y tomaron a Dina de casa de Siquem, y se fueron.
\par 27 Y los hijos de Jacob vinieron a los muertos, y saquearon la ciudad, por cuanto habían amancillado a su hermana.
\par 28 Tomaron sus ovejas y vacas y sus asnos, y lo que había en la ciudad y en el campo,
\par 29 y todos sus bienes; llevaron cautivos a todos sus niños y sus mujeres, y robaron todo lo que había en casa.
\par 30 Entonces dijo Jacob a Simeón y a Leví: Me habéis turbado con hacerme abominable a los moradores de esta tierra, el cananeo y el ferezeo; y teniendo yo pocos hombres, se juntarán contra mí y me atacarán, y seré destruido yo y mi casa.
\par 31 Pero ellos respondieron: ¿Había él de tratar a nuestra hermana como a una ramera?

\chapter{35}

\section*{Dios bendice a Jacob en Bet-el}

\par 1 Dijo Dios a Jacob: Levántate y sube a Bet-el, y quédate allí; y haz allí un altar al Dios que te apareció cuando huías de tu hermano Esaú.
\par 2 Entonces Jacob dijo a su familia y a todos los que con él estaban: Quitad los dioses ajenos que hay entre vosotros, y limpiaos, y mudad vuestros vestidos.
\par 3 Y levantémonos, y subamos a Bet-el; y haré allí altar al Dios que me respondió en el día de mi angustia, y ha estado conmigo en el camino que he andado.
\par 4 Así dieron a Jacob todos los dioses ajenos que había en poder de ellos, y los zarcillos que estaban en sus orejas; y Jacob los escondió debajo de una encina que estaba junto a Siquem.
\par 5 Y salieron, y el terror de Dios estuvo sobre las ciudades que había en sus alrededores, y no persiguieron a los hijos de Jacob.
\par 6 Y llegó Jacob a Luz, que está en tierra de Canaán (esta es Bet-el), él y todo el pueblo que con él estaba.
\par 7 Y edificó allí un altar, y llamó al lugar El-bet-el, porque allí le había aparecido Dios, cuando huía de su hermano.
\par 8 Entonces murió Débora, ama de Rebeca, y fue sepultada al pie de Bet-el, debajo de una encina, la cual fue llamada Alón-bacut.
\par 9 Apareció otra vez Dios a Jacob, cuando había vuelto de Padan-aram, y le bendijo.
\par 10 Y le dijo Dios: Tu nombre es Jacob; no se llamará más tu nombre Jacob, sino Israel será tu nombre; y llamó su nombre Israel.
\par 11 También le dijo Dios: Yo soy el Dios omnipotente: crece y multiplícate; una nación y conjunto de naciones procederán de ti, y reyes saldrán de tus lomos.
\par 12 La tierra que he dado a Abraham y a Isaac, la daré a ti, y a tu descendencia después de ti daré la tierra.
\par 13 Y se fue de él Dios, del lugar en donde había hablado con él.
\par 14 Y Jacob erigió una señal en el lugar donde había hablado con él, una señal de piedra, y derramó sobre ella libación, y echó sobre ella aceite.
\par 15 Y llamó Jacob el nombre de aquel lugar donde Dios había hablado con él, Bet-el.

\section*{Muerte de Raquel}

\par 16 Después partieron de Bet-el; y había aún como media legua de tierra para llegar a Efrata, cuando dio a luz Raquel, y hubo trabajo en su parto.
\par 17 Y aconteció, como había trabajo en su parto, que le dijo la partera: No temas, que también tendrás este hijo.
\par 18 Y aconteció que al salírsele el alma (pues murió), llamó su nombre Benoni; mas su padre lo llamó Benjamín.
\par 19 Así murió Raquel, y fue sepultada en el camino de Efrata, la cual es Belén.
\par 20 Y levantó Jacob un pilar sobre su sepultura; esta es la señal de la sepultura de Raquel hasta hoy.
\par 21 Y salió Israel, y plantó su tienda más allá de Migdal-edar.

\section*{Los hijos de Jacob}

\par 22 Aconteció que cuando moraba Israel en aquella tierra, fue Rubén y durmió con Bilha la concubina de su padre; lo cual llegó a saber Israel. Ahora bien, los hijos de Israel fueron doce:
\par 23 los hijos de Lea: Rubén el primogénito de Jacob; Simeón, Leví, Judá, Isacar y Zabulón.
\par 24 Los hijos de Raquel: José y Benjamín.
\par 25 Los hijos de Bilha, sierva de Raquel: Dan y Neftalí.
\par 26 Y los hijos de Zilpa, sierva de Lea: Gad y Aser. Estos fueron los hijos de Jacob, que le nacieron en Padan-aram.

\section*{Muerte de Isaac}

\par 27 Después vino Jacob a Isaac su padre a Mamre, a la ciudad de Arba, que es Hebrón, donde habitaron Abraham e Isaac.
\par 28 Y fueron los días de Isaac ciento ochenta años.
\par 29 Y exhaló Isaac el espíritu, y murió, y fue recogido a su pueblo, viejo y lleno de días; y lo sepultaron Esaú y Jacob sus hijos.

\chapter{36}

\section*{Los descendientes de Esaú}

\par 1 Estas son las generaciones de Esaú, el cual es Edom:
\par 2 Esaú tomó sus mujeres de las hijas de Canaán: a Ada, hija de Elón heteo, a Aholibama, hija de Aná, hijo de Zibeón heveo,
\par 3 y a Basemat hija de Ismael, hermana de Nebaiot.
\par 4 Ada dio a luz a Esaú a Elifaz; y Basemat dio a luz a Reuel.
\par 5 Y Aholibama dio a luz a Jeús, a Jaalam y a Coré; estos son los hijos de Esaú, que le nacieron en la tierra de Canaán.
\par 6 Y Esaú tomó sus mujeres, sus hijos y sus hijas, y todas las personas de su casa, y sus ganados, y todas sus bestias, y todo cuanto había adquirido en la tierra de Canaán, y se fue a otra tierra, separándose de Jacob su hermano.
\par 7 Porque los bienes de ellos eran muchos; y no podían habitar juntos, ni la tierra en donde moraban los podía sostener a causa de sus ganados.
\par 8 Y Esaú habitó en el monte de Seir; Esaú es Edom.
\par 9 Estos son los linajes de Esaú, padre de Edom, en el monte de Seir.
\par 10 Estos son los nombres de los hijos de Esaú: Elifaz, hijo de Ada mujer de Esaú; Reuel, hijo de Basemat mujer de Esaú.
\par 11 Y los hijos de Elifaz fueron Temán, Omar, Zefo, Gatam y Cenaz.
\par 12 Y Timna fue concubina de Elifaz hijo de Esaú, y ella le dio a luz a Amalec; estos son los hijos de Ada, mujer de Esaú.
\par 13 Los hijos de Reuel fueron Nahat, Zera, Sama y Miza; estos son los hijos de Basemat mujer de Esaú.
\par 14 Estos fueron los hijos de Aholibama mujer de Esaú, hija de Aná, que fue hijo de Zibeón: ella dio a luz a Jeús, Jaalam y Coré, hijos de Esaú.
\par 15 Estos son los jefes de entre los hijos de Esaú: hijos de Elifaz, primogénito de Esaú: los jefes Temán, Omar, Zefo, Cenaz,
\par 16 Coré, Gatam y Amalec; estos son los jefes de Elifaz en la tierra de Edom; estos fueron los hijos de Ada.
\par 17 Y estos son los hijos de Reuel, hijo de Esaú: los jefes Nahat, Zera, Sama y Miza; estos son los jefes de la línea de Reuel en la tierra de Edom; estos hijos vienen de Basemat mujer de Esaú.
\par 18 Y estos son los hijos de Aholibama mujer de Esaú: los jefes Jeús, Jaalam y Coré; estos fueron los jefes que salieron de Aholibama mujer de Esaú, hija de Aná.
\par 19 Estos, pues, son los hijos de Esaú, y sus jefes; él es Edom.
\par 20 Estos son los hijos de Seir horeo, moradores de aquella tierra: Lotán, Sobal, Zibeón, Aná,
\par 21 Disón, Ezer y Disán; estos son los jefes de los horeos, hijos de Seir, en la tierra de Edom.
\par 22 Los hijos de Lotán fueron Hori y Hemam; y Timna fue hermana de Lotán.
\par 23 Los hijos de Sobal fueron Alván, Manahat, Ebal, Sefo y Onam.
\par 24 Y los hijos de Zibeón fueron Aja y Aná. Este Aná es el que descubrió manantiales en el desierto, cuando apacentaba los asnos de Zibeón su padre.
\par 25 Los hijos de Aná fueron Disón, y Aholibama hija de Aná.
\par 26 Estos fueron los hijos de Disón: Hemdán, Esbán, Itrán y Querán.
\par 27 Y estos fueron los hijos de Ezer: Bilhán, Zaaván y Acán.
\par 28 Estos fueron los hijos de Disán: Uz y Arán.
\par 29 Y estos fueron los jefes de los horeos: los jefes Lotán, Sobal, Zibeón, Aná,
\par 30 Disón, Ezer y Disán; estos fueron los jefes de los horeos, por sus mandos en la tierra de Seir.
\par 31 Y los reyes que reinaron en la tierra de Edom, antes que reinase rey sobre los hijos de Israel, fueron estos:
\par 32 Bela hijo de Beor reinó en Edom; y el nombre de su ciudad fue Dinaba.
\par 33 Murió Bela, y reinó en su lugar Jobab hijo de Zera, de Bosra.
\par 34 Murió Jobab, y en su lugar reinó Husam, de tierra de Temán.
\par 35 Murió Husam, y reinó en su lugar Hadad hijo de Bedad, el que derrotó a Madián en el campo de Moab; y el nombre de su ciudad fue Avit.
\par 36 Murió Hadad, y en su lugar reinó Samla de Masreca.
\par 37 Murió Samla, y reinó en su lugar Saúl de Rehobot junto al Eufrates.
\par 38 Murió Saúl, y en lugar suyo reinó Baal-hanán hijo de Acbor.
\par 39 Y murió Baal-hanán hijo de Acbor, y reinó Hadar en lugar suyo; y el nombre de su ciudad fue Pau; y el nombre de su mujer, Mehetabel hija de Matred, hija de Mezaab.
\par 40 Estos, pues, son los nombres de los jefes de Esaú por sus linajes, por sus lugares, y sus nombres: Timna, Alva, Jetet,
\par 41 Aholibama, Ela, Pinón,
\par 42 Cenaz, Temán, Mibzar,
\par 43 Magdiel e Iram. Estos fueron los jefes de Edom según sus moradas en la tierra de su posesión. Edom es el mismo Esaú, padre de los edomitas.

\chapter{37}

\section*{José es vendido por sus hermanos}

\par 1 Habitó Jacob en la tierra donde había morado su padre, en la tierra de Canaán.
\par 2 Esta es la historia de la familia de Jacob: José, siendo de edad de diecisiete años, apacentaba las ovejas con sus hermanos; y el joven estaba con los hijos de Bilha y con los hijos de Zilpa, mujeres de su padre; e informaba José a su padre la mala fama de ellos.
\par 3 Y amaba Israel a José más que a todos sus hijos, porque lo había tenido en su vejez; y le hizo una túnica de diversos colores.
\par 4 Y viendo sus hermanos que su padre lo amaba más que a todos sus hermanos, le aborrecían, y no podían hablarle pacíficamente.
\par 5 Y soñó José un sueño, y lo contó a sus hermanos; y ellos llegaron a aborrecerle más todavía.
\par 6 Y él les dijo: Oíd ahora este sueño que he soñado:
\par 7 He aquí que atábamos manojos en medio del campo, y he aquí que mi manojo se levantaba y estaba derecho, y que vuestros manojos estaban alrededor y se inclinaban al mío.
\par 8 Le respondieron sus hermanos: ¿Reinarás tú sobre nosotros, o señorearás sobre nosotros? Y le aborrecieron aun más a causa de sus sueños y sus palabras.
\par 9 Soñó aun otro sueño, y lo contó a sus hermanos, diciendo: He aquí que he soñado otro sueño, y he aquí que el sol y la luna y once estrellas se inclinaban a mí.
\par 10 Y lo contó a su padre y a sus hermanos; y su padre le reprendió, y le dijo: ¿Qué sueño es este que soñaste? ¿Acaso vendremos yo y tu madre y tus hermanos a postrarnos en tierra ante ti?
\par 11 Y sus hermanos le tenían envidia, mas su padre meditaba en esto.
\par 12 Después fueron sus hermanos a apacentar las ovejas de su padre en Siquem.
\par 13 Y dijo Israel a José: Tus hermanos apacientan las ovejas en Siquem: ven, y te enviaré a ellos. Y él respondió: Heme aquí.
\par 14 E Israel le dijo: Ve ahora, mira cómo están tus hermanos y cómo están las ovejas, y tráeme la respuesta. Y lo envió del valle de Hebrón, y llegó a Siquem.
\par 15 Y lo halló un hombre, andando él errante por el campo, y le preguntó aquel hombre, diciendo: ¿Qué buscas?
\par 16 José respondió: Busco a mis hermanos; te ruego que me muestres dónde están apacentando.
\par 17 Aquel hombre respondió: Ya se han ido de aquí; y yo les oí decir: Vamos a Dotán. Entonces José fue tras de sus hermanos, y los halló en Dotán.
\par 18 Cuando ellos lo vieron de lejos, antes que llegara cerca de ellos, conspiraron contra él para matarle.
\par 19 Y dijeron el uno al otro: He aquí viene el soñador.
\par 20 Ahora pues, venid, y matémosle y echémosle en una cisterna, y diremos: Alguna mala bestia lo devoró; y veremos qué será de sus sueños.
\par 21 Cuando Rubén oyó esto, lo libró de sus manos, y dijo: No lo matemos.
\par 22 Y les dijo Rubén: No derraméis sangre; echadlo en esta cisterna que está en el desierto, y no pongáis mano en él; por librarlo así de sus manos, para hacerlo volver a su padre.
\par 23 Sucedió, pues, que cuando llegó José a sus hermanos, ellos quitaron a José su túnica, la túnica de colores que tenía sobre sí;
\par 24 y le tomaron y le echaron en la cisterna; pero la cisterna estaba vacía, no había en ella agua.
\par 25 Y se sentaron a comer pan; y alzando los ojos miraron, y he aquí una compañía de ismaelitas que venía de Galaad, y sus camellos traían aromas, bálsamo y mirra, e iban a llevarlo a Egipto.
\par 26 Entonces Judá dijo a sus hermanos: ¿Qué provecho hay en que matemos a nuestro hermano y encubramos su muerte?
\par 27 Venid, y vendámosle a los ismaelitas, y no sea nuestra mano sobre él; porque él es nuestro hermano, nuestra propia carne. Y sus hermanos convinieron con él.
\par 28 Y cuando pasaban los madianitas mercaderes, sacaron ellos a José de la cisterna, y le trajeron arriba, y le vendieron a los ismaelitas por veinte piezas de plata. Y llevaron a José a Egipto.
\par 29 Después Rubén volvió a la cisterna, y no halló a José dentro, y rasgó sus vestidos.
\par 30 Y volvió a sus hermanos, y dijo: El joven no parece; y yo, ¿adónde iré yo?
\par 31 Entonces tomaron ellos la túnica de José, y degollaron un cabrito de las cabras, y tiñeron la túnica con la sangre;
\par 32 y enviaron la túnica de colores y la trajeron a su padre, y dijeron: Esto hemos hallado; reconoce ahora si es la túnica de tu hijo, o no.
\par 33 Y él la reconoció, y dijo: La túnica de mi hijo es; alguna mala bestia lo devoró; José ha sido despedazado.
\par 34 Entonces Jacob rasgó sus vestidos, y puso cilicio sobre sus lomos, y guardó luto por su hijo muchos días.
\par 35 Y se levantaron todos sus hijos y todas sus hijas para consolarlo; mas él no quiso recibir consuelo, y dijo: Descenderé enlutado a mi hijo hasta el Seol. Y lo lloró su padre.
\par 36 Y los madianitas lo vendieron en Egipto a Potifar, oficial de Faraón, capitán de la guardia.

\chapter{38}

\section*{Judá y Tamar}

\par 1 Aconteció en aquel tiempo, que Judá se apartó de sus hermanos, y se fue a un varón adulamita que se llamaba Hira.
\par 2 Y vio allí Judá la hija de un hombre cananeo, el cual se llamaba Súa; y la tomó, y se llegó a ella.
\par 3 Y ella concibió, y dio a luz un hijo, y llamó su nombre Er.
\par 4 Concibió otra vez, y dio a luz un hijo, y llamó su nombre Onán.
\par 5 Y volvió a concebir, y dio a luz un hijo, y llamó su nombre Sela. Y estaba en Quezib cuando lo dio a luz.
\par 6 Después Judá tomó mujer para su primogénito Er, la cual se llamaba Tamar.
\par 7 Y Er, el primogénito de Judá, fue malo ante los ojos de Jehová, y le quitó Jehová la vida.
\par 8 Entonces Judá dijo a Onán: Llégate a la mujer de tu hermano, y despósate con ella, y levanta descendencia a tu hermano.
\par 9 Y sabiendo Onán que la descendencia no había de ser suya, sucedía que cuando se llegaba a la mujer de su hermano, vertía en tierra, por no dar descendencia a su hermano.
\par 10 Y desagradó en ojos de Jehová lo que hacía, y a él también le quitó la vida.
\par 11 Y Judá dijo a Tamar su nuera: Quédate viuda en casa de tu padre, hasta que crezca Sela mi hijo; porque dijo: No sea que muera él también como sus hermanos. Y se fue Tamar, y estuvo en casa de su padre.
\par 12 Pasaron muchos días, y murió la hija de Súa, mujer de Judá. Después Judá se consoló, y subía a los trasquiladores de sus ovejas a Timnat, él y su amigo Hira el adulamita.
\par 13 Y fue dado aviso a Tamar, diciendo: He aquí tu suegro sube a Timnat a trasquilar sus ovejas.
\par 14 Entonces se quitó ella los vestidos de su viudez, y se cubrió con un velo, y se arrebozó, y se puso a la entrada de Enaim junto al camino de Timnat; porque veía que había crecido Sela, y ella no era dada a él por mujer.
\par 15 Y la vio Judá, y la tuvo por ramera, porque ella había cubierto su rostro.
\par 16 Y se apartó del camino hacia ella, y le dijo: Déjame ahora llegarme a ti: pues no sabía que era su nuera; y ella dijo: ¿Qué me darás por llegarte a mí?
\par 17 El respondió: Yo te enviaré del ganado un cabrito de las cabras. Y ella dijo: Dame una prenda hasta que lo envíes.
\par 18 Entonces Judá dijo: ¿Qué prenda te daré? Ella respondió: Tu sello, tu cordón, y tu báculo que tienes en tu mano. Y él se los dio, y se llegó a ella, y ella concibió de él.
\par 19 Luego se levantó y se fue, y se quitó el velo de sobre sí, y se vistió las ropas de su viudez.
\par 20 Y Judá envió el cabrito de las cabras por medio de su amigo el adulamita, para que éste recibiese la prenda de la mujer; pero no la halló.
\par 21 Y preguntó a los hombres de aquel lugar, diciendo: ¿Dónde está la ramera de Enaim junto al camino? Y ellos le dijeron: No ha estado aquí ramera alguna.
\par 22 Entonces él se volvió a Judá, y dijo: No la he hallado; y también los hombres del lugar dijeron: Aquí no ha estado ramera.
\par 23 Y Judá dijo: Tómeselo para sí, para que no seamos menospreciados; he aquí yo he enviado este cabrito, y tú no la hallaste.
\par 24 Sucedió que al cabo de unos tres meses fue dado aviso a Judá, diciendo: Tamar tu nuera ha fornicado, y ciertamente está encinta a causa de las fornicaciones. Y Judá dijo: Sacadla, y sea quemada.
\par 25 Pero ella, cuando la sacaban, envió a decir a su suegro: Del varón cuyas son estas cosas, estoy encinta. También dijo: Mira ahora de quién son estas cosas, el sello, el cordón y el báculo.
\par 26 Entonces Judá los reconoció, y dijo: Más justa es ella que yo, por cuanto no la he dado a Sela mi hijo. Y nunca más la conoció.
\par 27 Y aconteció que al tiempo de dar a luz, he aquí había gemelos en su seno.
\par 28 Sucedió cuando daba a luz, que sacó la mano el uno, y la partera tomó y ató a su mano un hilo de grana, diciendo: Este salió primero.
\par 29 Pero volviendo él a meter la mano, he aquí salió su hermano; y ella dijo: !!Qué brecha te has abierto! Y llamó su nombre Fares.
\par 30 Después salió su hermano, el que tenía en su mano el hilo de grana, y llamó su nombre Zara.

\chapter{39}

\section*{José y la esposa de Potifar}

\par 1 Llevado, pues, José a Egipto, Potifar oficial de Faraón, capitán de la guardia, varón egipcio, lo compró de los ismaelitas que lo habían llevado allá.
\par 2 Mas Jehová estaba con José, y fue varón próspero; y estaba en la casa de su amo el egipcio.
\par 3 Y vio su amo que Jehová estaba con él, y que todo lo que él hacía, Jehová lo hacía prosperar en su mano.
\par 4 Así halló José gracia en sus ojos, y le servía; y él le hizo mayordomo de su casa y entregó en su poder todo lo que tenía.
\par 5 Y aconteció que desde cuando le dio el encargo de su casa y de todo lo que tenía, Jehová bendijo la casa del egipcio a causa de José, y la bendición de Jehová estaba sobre todo lo que tenía, así en casa como en el campo.
\par 6 Y dejó todo lo que tenía en mano de José, y con él no se preocupaba de cosa alguna sino del pan que comía. Y era José de hermoso semblante y bella presencia.
\par 7 Aconteció después de esto, que la mujer de su amo puso sus ojos en José, y dijo: Duerme conmigo.
\par 8 Y él no quiso, y dijo a la mujer de su amo: He aquí que mi señor no se preocupa conmigo de lo que hay en casa, y ha puesto en mi mano todo lo que tiene.
\par 9 No hay otro mayor que yo en esta casa, y ninguna cosa me ha reservado sino a ti, por cuanto tú eres su mujer; ¿cómo, pues, haría yo este grande mal, y pecaría contra Dios?
\par 10 Hablando ella a José cada día, y no escuchándola él para acostarse al lado de ella, para estar con ella,
\par 11 aconteció que entró él un día en casa para hacer su oficio, y no había nadie de los de casa allí.
\par 12 Y ella lo asió por su ropa, diciendo: Duerme conmigo. Entonces él dejó su ropa en las manos de ella, y huyó y salió.
\par 13 Cuando vio ella que le había dejado su ropa en sus manos, y había huido fuera,
\par 14 llamó a los de casa, y les habló diciendo: Mirad, nos ha traído un hebreo para que hiciese burla de nosotros. Vino él a mí para dormir conmigo, y yo di grandes voces;
\par 15 y viendo que yo alzaba la voz y gritaba, dejó junto a mí su ropa, y huyó y salió.
\par 16 Y ella puso junto a sí la ropa de José, hasta que vino su señor a su casa.
\par 17 Entonces le habló ella las mismas palabras, diciendo: El siervo hebreo que nos trajiste, vino a mí para deshonrarme.
\par 18 Y cuando yo alcé mi voz y grité, él dejó su ropa junto a mí y huyó fuera.
\par 19 Y sucedió que cuando oyó el amo de José las palabras que su mujer le hablaba, diciendo: Así me ha tratado tu siervo, se encendió su furor.
\par 20 Y tomó su amo a José, y lo puso en la cárcel, donde estaban los presos del rey, y estuvo allí en la cárcel.
\par 21 Pero Jehová estaba con José y le extendió su misericordia, y le dio gracia en los ojos del jefe de la cárcel.
\par 22 Y el jefe de la cárcel entregó en mano de José el cuidado de todos los presos que había en aquella prisión; todo lo que se hacía allí, él lo hacía.
\par 23 No necesitaba atender el jefe de la cárcel cosa alguna de las que estaban al cuidado de José, porque Jehová estaba con José, y lo que él hacía, Jehová lo prosperaba.

\chapter{40}

\section*{José interpreta dos sueños}

\par 1 Aconteció después de estas cosas, que el copero del rey de Egipto y el panadero delinquieron contra su señor el rey de Egipto.
\par 2 Y se enojó Faraón contra sus dos oficiales, contra el jefe de los coperos y contra el jefe de los panaderos,
\par 3 y los puso en prisión en la casa del capitán de la guardia, en la cárcel donde José estaba preso.
\par 4 Y el capitán de la guardia encargó de ellos a José, y él les servía; y estuvieron días en la prisión.
\par 5 Y ambos, el copero y el panadero del rey de Egipto, que estaban arrestados en la prisión, tuvieron un sueño, cada uno su propio sueño en una misma noche, cada uno con su propio significado.
\par 6 Vino a ellos José por la mañana, y los miró, y he aquí que estaban tristes.
\par 7 Y él preguntó a aquellos oficiales de Faraón, que estaban con él en la prisión de la casa de su señor, diciendo: ¿Por qué parecen hoy mal vuestros semblantes?
\par 8 Ellos le dijeron: Hemos tenido un sueño, y no hay quien lo interprete. Entonces les dijo José: ¿No son de Dios las interpretaciones? Contádmelo ahora.
\par 9 Entonces el jefe de los coperos contó su sueño a José, y le dijo: Yo soñaba que veía una vid delante de mí,
\par 10 y en la vid tres sarmientos; y ella como que brotaba, y arrojaba su flor, viniendo a madurar sus racimos de uvas.
\par 11 Y que la copa de Faraón estaba en mi mano, y tomaba yo las uvas y las exprimía en la copa de Faraón, y daba yo la copa en mano de Faraón.
\par 12 Y le dijo José: Esta es su interpretación: los tres sarmientos son tres días.
\par 13 Al cabo de tres días levantará Faraón tu cabeza, y te restituirá a tu puesto, y darás la copa a Faraón en su mano, como solías hacerlo cuando eras su copero.
\par 14 Acuérdate, pues, de mí cuando tengas ese bien, y te ruego que uses conmigo de misericordia, y hagas mención de mí a Faraón, y me saques de esta casa.
\par 15 Porque fui hurtado de la tierra de los hebreos; y tampoco he hecho aquí por qué me pusiesen en la cárcel.
\par 16 Viendo el jefe de los panaderos que había interpretado para bien, dijo a José: También yo soñé que veía tres canastillos blancos sobre mi cabeza.
\par 17 En el canastillo más alto había de toda clase de manjares de pastelería para Faraón; y las aves las comían del canastillo de sobre mi cabeza.
\par 18 Entonces respondió José, y dijo: Esta es su interpretación: Los tres canastillos tres días son.
\par 19 Al cabo de tres días quitará Faraón tu cabeza de sobre ti, y te hará colgar en la horca, y las aves comerán tu carne de sobre ti.
\par 20 Al tercer día, que era el día del cumpleaños de Faraón, el rey hizo banquete a todos sus sirvientes; y alzó la cabeza del jefe de los coperos, y la cabeza del jefe de los panaderos, entre sus servidores.
\par 21 E hizo volver a su oficio al jefe de los coperos, y dio éste la copa en mano de Faraón.
\par 22 Mas hizo ahorcar al jefe de los panaderos, como lo había interpretado José.
\par 23 Y el jefe de los coperos no se acordó de José, sino que le olvidó.

\chapter{41}

\section*{José interpreta el sueño de Faraón}

\par 1 Aconteció que pasados dos años tuvo Faraón un sueño. Le parecía que estaba junto al río;
\par 2 y que del río subían siete vacas, hermosas a la vista, y muy gordas, y pacían en el prado.
\par 3 Y que tras ellas subían del río otras siete vacas de feo aspecto y enjutas de carne, y se pararon cerca de las vacas hermosas a la orilla del río;
\par 4 y que las vacas de feo aspecto y enjutas de carne devoraban a las siete vacas hermosas y muy gordas. Y despertó Faraón.
\par 5 Se durmió de nuevo, y soñó la segunda vez: Que siete espigas llenas y hermosas crecían de una sola caña,
\par 6 y que después de ellas salían otras siete espigas menudas y abatidas del viento solano;
\par 7 y las siete espigas menudas devoraban a las siete espigas gruesas y llenas. Y despertó Faraón, y he aquí que era sueño.
\par 8 Sucedió que por la mañana estaba agitado su espíritu, y envió e hizo llamar a todos los magos de Egipto, y a todos sus sabios; y les contó Faraón sus sueños, mas no había quien los pudiese interpretar a Faraón.
\par 9 Entonces el jefe de los coperos habló a Faraón, diciendo: Me acuerdo hoy de mis faltas.
\par 10 Cuando Faraón se enojó contra sus siervos, nos echó a la prisión de la casa del capitán de la guardia a mí y al jefe de los panaderos.
\par 11 Y él y yo tuvimos un sueño en la misma noche, y cada sueño tenía su propio significado.
\par 12 Estaba allí con nosotros un joven hebreo, siervo del capitán de la guardia; y se lo contamos, y él nos interpretó nuestros sueños, y declaró a cada uno conforme a su sueño.
\par 13 Y aconteció que como él nos los interpretó, así fue: yo fui restablecido en mi puesto, y el otro fue colgado.
\par 14 Entonces Faraón envió y llamó a José. Y lo sacaron apresuradamente de la cárcel, y se afeitó, y mudó sus vestidos, y vino a Faraón.
\par 15 Y dijo Faraón a José: Yo he tenido un sueño, y no hay quien lo interprete; mas he oído decir de ti, que oyes sueños para interpretarlos.
\par 16 Respondió José a Faraón, diciendo: No está en mí; Dios será el que dé respuesta propicia a Faraón.
\par 17 Entonces Faraón dijo a José: En mi sueño me parecía que estaba a la orilla del río;
\par 18 y que del río subían siete vacas de gruesas carnes y hermosa apariencia, que pacían en el prado.
\par 19 Y que otras siete vacas subían después de ellas, flacas y de muy feo aspecto; tan extenuadas, que no he visto otras semejantes en fealdad en toda la tierra de Egipto.
\par 20 Y las vacas flacas y feas devoraban a las siete primeras vacas gordas;
\par 21 y éstas entraban en sus entrañas, mas no se conocía que hubiesen entrado, porque la apariencia de las flacas era aún mala, como al principio. Y yo desperté.
\par 22 Vi también soñando, que siete espigas crecían en una misma caña, llenas y hermosas.
\par 23 Y que otras siete espigas menudas, marchitas, abatidas del viento solano, crecían después de ellas;
\par 24 y las espigas menudas devoraban a las siete espigas hermosas; y lo he dicho a los magos, mas no hay quien me lo interprete.
\par 25 Entonces respondió José a Faraón: El sueño de Faraón es uno mismo; Dios ha mostrado a Faraón lo que va a hacer.
\par 26 Las siete vacas hermosas siete años son; y las espigas hermosas son siete años: el sueño es uno mismo.
\par 27 También las siete vacas flacas y feas que subían tras ellas, son siete años; y las siete espigas menudas y marchitas del viento solano, siete años serán de hambre.
\par 28 Esto es lo que respondo a Faraón. Lo que Dios va a hacer, lo ha mostrado a Faraón.
\par 29 He aquí vienen siete años de gran abundancia en toda la tierra de Egipto.
\par 30 Y tras ellos seguirán siete años de hambre; y toda la abundancia será olvidada en la tierra de Egipto, y el hambre consumirá la tierra.
\par 31 Y aquella abundancia no se echará de ver, a causa del hambre siguiente la cual será gravísima.
\par 32 Y el suceder el sueño a Faraón dos veces, significa que la cosa es firme de parte de Dios, y que Dios se apresura a hacerla.
\par 33 Por tanto, provéase ahora Faraón de un varón prudente y sabio, y póngalo sobre la tierra de Egipto.
\par 34 Haga esto Faraón, y ponga gobernadores sobre el país, y quinte la tierra de Egipto en los siete años de la abundancia.
\par 35 Y junten toda la provisión de estos buenos años que vienen, y recojan el trigo bajo la mano de Faraón para mantenimiento de las ciudades; y guárdenlo.
\par 36 Y esté aquella provisión en depósito para el país, para los siete años de hambre que habrá en la tierra de Egipto; y el país no perecerá de hambre.

\section*{José, gobernador de Egipto}

\par 37 El asunto pareció bien a Faraón y a sus siervos,
\par 38 y dijo Faraón a sus siervos: ¿Acaso hallaremos a otro hombre como éste, en quien esté el espíritu de Dios?
\par 39 Y dijo Faraón a José: Pues que Dios te ha hecho saber todo esto, no hay entendido ni sabio como tú.
\par 40 Tú estarás sobre mi casa, y por tu palabra se gobernará todo mi pueblo; solamente en el trono seré yo mayor que tú.
\par 41 Dijo además Faraón a José: He aquí yo te he puesto sobre toda la tierra de Egipto.
\par 42 Entonces Faraón quitó su anillo de su mano, y lo puso en la mano de José, y lo hizo vestir de ropas de lino finísimo, y puso un collar de oro en su cuello;
\par 43 y lo hizo subir en su segundo carro, y pregonaron delante de él: !!Doblad la rodilla!; y lo puso sobre toda la tierra de Egipto.
\par 44 Y dijo Faraón a José: Yo soy Faraón; y sin ti ninguno alzará su mano ni su pie en toda la tierra de Egipto.
\par 45 Y llamó Faraón el nombre de José, Zafnat-panea; y le dio por mujer a Asenat, hija de Potifera sacerdote de On. Y salió José por toda la tierra de Egipto.
\par 46 Era José de edad de treinta años cuando fue presentado delante de Faraón rey de Egipto; y salió José de delante de Faraón, y recorrió toda la tierra de Egipto.
\par 47 En aquellos siete años de abundancia la tierra produjo a montones.
\par 48 Y él reunió todo el alimento de los siete años de abundancia que hubo en la tierra de Egipto, y guardó alimento en las ciudades, poniendo en cada ciudad el alimento del campo de sus alrededores.
\par 49 Recogió José trigo como arena del mar, mucho en extremo, hasta no poderse contar, porque no tenía número.
\par 50 Y nacieron a José dos hijos antes que viniese el primer año del hambre, los cuales le dio a luz Asenat, hija de Potifera sacerdote de On.
\par 51 Y llamó José el nombre del primogénito, Manasés; porque dijo: Dios me hizo olvidar todo mi trabajo, y toda la casa de mi padre.
\par 52 Y llamó el nombre del segundo, Efraín; porque dijo: Dios me hizo fructificar en la tierra de mi aflicción.
\par 53 Así se cumplieron los siete años de abundancia que hubo en la tierra de Egipto.
\par 54 Y comenzaron a venir los siete años del hambre, como José había dicho; y hubo hambre en todos los países, mas en toda la tierra de Egipto había pan.
\par 55 Cuando se sintió el hambre en toda la tierra de Egipto, el pueblo clamó a Faraón por pan. Y dijo Faraón a todos los egipcios: Id a José, y haced lo que él os dijere.
\par 56 Y el hambre estaba por toda la extensión del país. Entonces abrió José todo granero donde había, y vendía a los egipcios; porque había crecido el hambre en la tierra de Egipto.
\par 57 Y de toda la tierra venían a Egipto para comprar de José, porque por toda la tierra había crecido el hambre.

\chapter{42}

\section*{Los hermanos de José vienen por alimentos}

\par 1 Viendo Jacob que en Egipto había alimentos, dijo a sus hijos: ¿Por qué os estáis mirando?
\par 2 Y dijo: He aquí, yo he oído que hay víveres en Egipto; descended allá, y comprad de allí para nosotros, para que podamos vivir, y no muramos.
\par 3 Y descendieron los diez hermanos de José a comprar trigo en Egipto.
\par 4 Mas Jacob no envió a Benjamín, hermano de José, con sus hermanos; porque dijo: No sea que le acontezca algún desastre.
\par 5 Vinieron los hijos de Israel a comprar entre los que venían; porque había hambre en la tierra de Canaán.
\par 6 Y José era el señor de la tierra, quien le vendía a todo el pueblo de la tierra; y llegaron los hermanos de José, y se inclinaron a él rostro a tierra.
\par 7 Y José, cuando vio a sus hermanos, los conoció; mas hizo como que no los conocía, y les habló ásperamente, y les dijo: ¿De dónde habéis venido? Ellos respondieron: De la tierra de Canaán, para comprar alimentos.
\par 8 José, pues, conoció a sus hermanos; pero ellos no le conocieron.
\par 9 Entonces se acordó José de los sueños que había tenido acerca de ellos, y les dijo: Espías sois; por ver lo descubierto del país habéis venido.
\par 10 Ellos le respondieron: No, señor nuestro, sino que tus siervos han venido a comprar alimentos.
\par 11 Todos nosotros somos hijos de un varón; somos hombres honrados; tus siervos nunca fueron espías.
\par 12 Pero José les dijo: No; para ver lo descubierto del país habéis venido.
\par 13 Y ellos respondieron: Tus siervos somos doce hermanos, hijos de un varón en la tierra de Canaán; y he aquí el menor está hoy con nuestro padre, y otro no parece.
\par 14 Y José les dijo: Eso es lo que os he dicho, afirmando que sois espías.
\par 15 En esto seréis probados: Vive Faraón, que no saldréis de aquí, sino cuando vuestro hermano menor viniere aquí.
\par 16 Enviad a uno de vosotros y traiga a vuestro hermano, y vosotros quedad presos, y vuestras palabras serán probadas, si hay verdad en vosotros; y si no, vive Faraón, que sois espías.
\par 17 Entonces los puso juntos en la cárcel por tres días.
\par 18 Y al tercer día les dijo José: Haced esto, y vivid: Yo temo a Dios.
\par 19 Si sois hombres honrados, quede preso en la casa de vuestra cárcel uno de vuestros hermanos, y vosotros id y llevad el alimento para el hambre de vuestra casa.
\par 20 Pero traeréis a vuestro hermano menor, y serán verificadas vuestras palabras, y no moriréis. Y ellos lo hicieron así.
\par 21 Y decían el uno al otro: Verdaderamente hemos pecado contra nuestro hermano, pues vimos la angustia de su alma cuando nos rogaba, y no le escuchamos; por eso ha venido sobre nosotros esta angustia.
\par 22 Entonces Rubén les respondió, diciendo: ¿No os hablé yo y dije: No pequéis contra el joven, y no escuchasteis? He aquí también se nos demanda su sangre.
\par 23 Pero ellos no sabían que los entendía José, porque había intérprete entre ellos.
\par 24 Y se apartó José de ellos, y lloró; después volvió a ellos, y les habló, y tomó de entre ellos a Simeón, y lo aprisionó a vista de ellos.
\par 25 Después mandó José que llenaran sus sacos de trigo, y devolviesen el dinero de cada uno de ellos, poniéndolo en su saco, y les diesen comida para el camino; y así se hizo con ellos.
\par 26 Y ellos pusieron su trigo sobre sus asnos, y se fueron de allí.
\par 27 Pero abriendo uno de ellos su saco para dar de comer a su asno en el mesón, vio su dinero que estaba en la boca de su costal.
\par 28 Y dijo a sus hermanos: Mi dinero se me ha devuelto, y helo aquí en mi saco. Entonces se les sobresaltó el corazón, y espantados dijeron el uno al otro: ¿Qué es esto que nos ha hecho Dios?
\par 29 Y venidos a Jacob su padre en tierra de Canaán, le contaron todo lo que les había acontecido, diciendo:
\par 30 Aquel varón, el señor de la tierra, nos habló ásperamente, y nos trató como a espías de la tierra.
\par 31 Y nosotros le dijimos: Somos hombres honrados, nunca fuimos espías.
\par 32 Somos doce hermanos, hijos de nuestro padre; uno no parece, y el menor está hoy con nuestro padre en la tierra de Canaán.
\par 33 Entonces aquel varón, el señor de la tierra, nos dijo: En esto conoceré que sois hombres honrados: dejad conmigo uno de vuestros hermanos, y tomad para el hambre de vuestras casas, y andad,
\par 34 y traedme a vuestro hermano el menor, para que yo sepa que no sois espías, sino hombres honrados; así os daré a vuestro hermano, y negociaréis en la tierra.
\par 35 Y aconteció que vaciando ellos sus sacos, he aquí que en el saco de cada uno estaba el atado de su dinero; y viendo ellos y su padre los atados de su dinero, tuvieron temor.
\par 36 Entonces su padre Jacob les dijo: Me habéis privado de mis hijos; José no parece, ni Simeón tampoco, y a Benjamín le llevaréis; contra mí son todas estas cosas.
\par 37 Y Rubén habló a su padre, diciendo: Harás morir a mis dos hijos, si no te lo devuelvo; entrégalo en mi mano, que yo lo devolveré a ti.
\par 38 Y él dijo: No descenderá mi hijo con vosotros, pues su hermano ha muerto, y él solo ha quedado; y si le aconteciere algún desastre en el camino por donde vais, haréis descender mis canas con dolor al Seol.

\chapter{43}

\section*{Los hermanos de José regresan con Benjamín}

\par 1 El hambre era grande en la tierra;
\par 2 y aconteció que cuando acabaron de comer el trigo que trajeron de Egipto, les dijo su padre: Volved, y comprad para nosotros un poco de alimento.
\par 3 Respondió Judá, diciendo: Aquel varón nos protestó con ánimo resuelto, diciendo: No veréis mi rostro si no traéis a vuestro hermano con vosotros.
\par 4 Si enviares a nuestro hermano con nosotros, descenderemos y te compraremos alimento.
\par 5 Pero si no le enviares, no descenderemos; porque aquel varón nos dijo: No veréis mi rostro si no traéis a vuestro hermano con vosotros.
\par 6 Dijo entonces Israel: ¿Por qué me hicisteis tanto mal, declarando al varón que teníais otro hermano?
\par 7 Y ellos respondieron: Aquel varón nos preguntó expresamente por nosotros, y por nuestra familia, diciendo: ¿Vive aún vuestro padre? ¿Tenéis otro hermano? Y le declaramos conforme a estas palabras. ¿Acaso podíamos saber que él nos diría: Haced venir a vuestro hermano?
\par 8 Entonces Judá dijo a Israel su padre: Envía al joven conmigo, y nos levantaremos e iremos, a fin de que vivamos y no muramos nosotros, y tú, y nuestros niños.
\par 9 Yo te respondo por él; a mí me pedirás cuenta. Si yo no te lo vuelvo a traer, y si no lo pongo delante de ti, seré para ti el culpable para siempre;
\par 10 pues si no nos hubiéramos detenido, ciertamente hubiéramos ya vuelto dos veces.
\par 11 Entonces Israel su padre les respondió: Pues que así es, hacedlo; tomad de lo mejor de la tierra en vuestros sacos, y llevad a aquel varón un presente, un poco de bálsamo, un poco de miel, aromas y mirra, nueces y almendras.
\par 12 Y tomad en vuestras manos doble cantidad de dinero, y llevad en vuestra mano el dinero vuelto en las bocas de vuestros costales; quizá fue equivocación.
\par 13 Tomad también a vuestro hermano, y levantaos, y volved a aquel varón.
\par 14 Y el Dios Omnipotente os dé misericordia delante de aquel varón, y os suelte al otro vuestro hermano, y a este Benjamín. Y si he de ser privado de mis hijos, séalo.
\par 15 Entonces tomaron aquellos varones el presente, y tomaron en su mano doble cantidad de dinero, y a Benjamín; y se levantaron y descendieron a Egipto, y se presentaron delante de José.
\par 16 Y vio José a Benjamín con ellos, y dijo al mayordomo de su casa: Lleva a casa a esos hombres, y degüella una res y prepárala, pues estos hombres comerán conmigo al mediodía.
\par 17 E hizo el hombre como José dijo, y llevó a los hombres a casa de José.
\par 18 Entonces aquellos hombres tuvieron temor, cuando fueron llevados a casa de José, y decían: Por el dinero que fue devuelto en nuestros costales la primera vez nos han traído aquí, para tendernos lazo, y atacarnos, y tomarnos por siervos a nosotros, y a nuestros asnos.
\par 19 Y se acercaron al mayordomo de la casa de José, y le hablaron a la entrada de la casa.
\par 20 Y dijeron: Ay, señor nuestro, nosotros en realidad de verdad descendimos al principio a comprar alimentos.
\par 21 Y aconteció que cuando llegamos al mesón y abrimos nuestros costales, he aquí el dinero de cada uno estaba en la boca de su costal, nuestro dinero en su justo peso; y lo hemos vuelto a traer con nosotros.
\par 22 Hemos también traído en nuestras manos otro dinero para comprar alimentos; nosotros no sabemos quién haya puesto nuestro dinero en nuestros costales.
\par 23 El les respondió: Paz a vosotros, no temáis; vuestro Dios y el Dios de vuestro padre os dio el tesoro en vuestros costales; yo recibí vuestro dinero. Y sacó a Simeón a ellos.
\par 24 Y llevó aquel varón a los hombres a casa de José; y les dio agua, y lavaron sus pies, y dio de comer a sus asnos.
\par 25 Y ellos prepararon el presente entretanto que venía José a mediodía, porque habían oído que allí habrían de comer pan.
\par 26 Y vino José a casa, y ellos le trajeron el presente que tenían en su mano dentro de la casa, y se inclinaron ante él hasta la tierra.
\par 27 Entonces les preguntó José cómo estaban, y dijo: ¿Vuestro padre, el anciano que dijisteis, lo pasa bien? ¿Vive todavía?
\par 28 Y ellos respondieron: Bien va a tu siervo nuestro padre; aún vive. Y se inclinaron, e hicieron reverencia.
\par 29 Y alzando José sus ojos vio a Benjamín su hermano, hijo de su madre, y dijo: ¿Es éste vuestro hermano menor, de quien me hablasteis? Y dijo: Dios tenga misericordia de ti, hijo mío.
\par 30 Entonces José se apresuró, porque se conmovieron sus entrañas a causa de su hermano, y buscó dónde llorar; y entró en su cámara, y lloró allí.
\par 31 Y lavó su rostro y salió, y se contuvo, y dijo: Poned pan.
\par 32 Y pusieron para él aparte, y separadamente para ellos, y aparte para los egipcios que con él comían; porque los egipcios no pueden comer pan con los hebreos, lo cual es abominación a los egipcios.
\par 33 Y se sentaron delante de él, el mayor conforme a su primogenitura, y el menor conforme a su menor edad; y estaban aquellos hombres atónitos mirándose el uno al otro.
\par 34 Y José tomó viandas de delante de sí para ellos; mas la porción de Benjamín era cinco veces mayor que cualquiera de las de ellos. Y bebieron, y se alegraron con él.

\chapter{44}

\section*{La copa de José}

\par 1 Mandó José al mayordomo de su casa, diciendo: Llena de alimento los costales de estos varones, cuanto puedan llevar, y pon el dinero de cada uno en la boca de su costal.
\par 2 Y pondrás mi copa, la copa de plata, en la boca del costal del menor, con el dinero de su trigo. Y él hizo como dijo José.
\par 3 Venida la mañana, los hombres fueron despedidos con sus asnos.
\par 4 Habiendo ellos salido de la ciudad, de la que aún no se habían alejado, dijo José a su mayordomo: Levántate y sigue a esos hombres; y cuando los alcances, diles: ¿Por qué habéis vuelto mal por bien? ¿Por qué habéis robado mi copa de plata?
\par 5 ¿No es ésta en la que bebe mi señor, y por la que suele adivinar? Habéis hecho mal en lo que hicisteis.
\par 6 Cuando él los alcanzó, les dijo estas palabras.
\par 7 Y ellos le respondieron: ¿Por qué dice nuestro señor tales cosas? Nunca tal hagan tus siervos.
\par 8 He aquí, el dinero que hallamos en la boca de nuestros costales, te lo volvimos a traer desde la tierra de Canaán; ¿cómo, pues, habíamos de hurtar de casa de tu señor plata ni oro?
\par 9 Aquel de tus siervos en quien fuere hallada la copa, que muera, y aun nosotros seremos siervos de mi señor.
\par 10 Y él dijo: También ahora sea conforme a vuestras palabras; aquel en quien se hallare será mi siervo, y vosotros seréis sin culpa.
\par 11 Ellos entonces se dieron prisa, y derribando cada uno su costal en tierra, abrió cada cual el costal suyo.
\par 12 Y buscó; desde el mayor comenzó, y acabó en el menor; y la copa fue hallada en el costal de Benjamín.
\par 13 Entonces ellos rasgaron sus vestidos, y cargó cada uno su asno y volvieron a la ciudad.
\par 14 Vino Judá con sus hermanos a casa de José, que aún estaba allí, y se postraron delante de él en tierra.
\par 15 Y les dijo José: ¿Qué acción es esta que habéis hecho? ¿No sabéis que un hombre como yo sabe adivinar?
\par 16 Entonces dijo Judá: ¿Qué diremos a mi señor? ¿Qué hablaremos, o con qué nos justificaremos? Dios ha hallado la maldad de tus siervos; he aquí, nosotros somos siervos de mi señor, nosotros, y también aquel en cuyo poder fue hallada la copa.
\par 17 José respondió: Nunca yo tal haga. El varón en cuyo poder fue hallada la copa, él será mi siervo; vosotros id en paz a vuestro padre.

\section*{Judá intercede por Benjamín}

\par 18 Entonces Judá se acercó a él, y dijo: Ay, señor mío, te ruego que permitas que hable tu siervo una palabra en oídos de mi señor, y no se encienda tu enojo contra tu siervo, pues tú eres como Faraón.
\par 19 Mi señor preguntó a sus siervos, diciendo: ¿Tenéis padre o hermano?
\par 20 Y nosotros respondimos a mi señor: Tenemos un padre anciano, y un hermano joven, pequeño aún, que le nació en su vejez; y un hermano suyo murió, y él solo quedó de los hijos de su madre; y su padre lo ama.
\par 21 Y tú dijiste a tus siervos: Traédmelo, y pondré mis ojos sobre él.
\par 22 Y nosotros dijimos a mi señor: El joven no puede dejar a su padre, porque si lo dejare, su padre morirá.
\par 23 Y dijiste a tus siervos: Si vuestro hermano menor no desciende con vosotros, no veréis más mi rostro.
\par 24 Aconteció, pues, que cuando llegamos a mi padre tu siervo, le contamos las palabras de mi señor.
\par 25 Y dijo nuestro padre: Volved a comprarnos un poco de alimento.
\par 26 Y nosotros respondimos: No podemos ir; si nuestro hermano va con nosotros, iremos; porque no podremos ver el rostro del varón, si no está con nosotros nuestro hermano el menor.
\par 27 Entonces tu siervo mi padre nos dijo: Vosotros sabéis que dos hijos me dio a luz mi mujer;
\par 28 y el uno salió de mi presencia, y pienso de cierto que fue despedazado, y hasta ahora no lo he visto.
\par 29 Y si tomáis también a éste de delante de mí, y le acontece algún desastre, haréis descender mis canas con dolor al Seol.
\par 30 Ahora, pues, cuando vuelva yo a tu siervo mi padre, si el joven no va conmigo, como su vida está ligada a la vida de él,
\par 31 sucederá que cuando no vea al joven, morirá; y tus siervos harán descender las canas de tu siervo nuestro padre con dolor al Seol.
\par 32 Como tu siervo salió por fiador del joven con mi padre, diciendo: Si no te lo vuelvo a traer, entonces yo seré culpable ante mi padre para siempre;
\par 33 te ruego, por tanto, que quede ahora tu siervo en lugar del joven por siervo de mi señor, y que el joven vaya con sus hermanos.
\par 34 Porque ¿cómo volveré yo a mi padre sin el joven? No podré, por no ver el mal que sobrevendrá a mi padre.

\chapter{45}

\section*{José se da a conocer a sus hermanos}

\par 1 No podía ya José contenerse delante de todos los que estaban al lado suyo, y clamó: Haced salir de mi presencia a todos. Y no quedó nadie con él, al darse a conocer José a sus hermanos.
\par 2 Entonces se dio a llorar a gritos; y oyeron los egipcios, y oyó también la casa de Faraón.
\par 3 Y dijo José a sus hermanos: Yo soy José; ¿vive aún mi padre? Y sus hermanos no pudieron responderle, porque estaban turbados delante de él.
\par 4 Entonces dijo José a sus hermanos: Acercaos ahora a mí. Y ellos se acercaron. Y él dijo: Yo soy José vuestro hermano, el que vendisteis para Egipto.
\par 5 Ahora, pues, no os entristezcáis, ni os pese de haberme vendido acá; porque para preservación de vida me envió Dios delante de vosotros.
\par 6 Pues ya ha habido dos años de hambre en medio de la tierra, y aún quedan cinco años en los cuales ni habrá arada ni siega.
\par 7 Y Dios me envió delante de vosotros, para preservaros posteridad sobre la tierra, y para daros vida por medio de gran liberación.
\par 8 Así, pues, no me enviasteis acá vosotros, sino Dios, que me ha puesto por padre de Faraón y por señor de toda su casa, y por gobernador en toda la tierra de Egipto.
\par 9 Daos prisa, id a mi padre y decidle: Así dice tu hijo José: Dios me ha puesto por señor de todo Egipto; ven a mí, no te detengas.
\par 10 Habitarás en la tierra de Gosén, y estarás cerca de mí, tú y tus hijos, y los hijos de tus hijos, tus ganados y tus vacas, y todo lo que tienes.
\par 11 Y allí te alimentaré, pues aún quedan cinco años de hambre, para que no perezcas de pobreza tú y tu casa, y todo lo que tienes.
\par 12 He aquí, vuestros ojos ven, y los ojos de mi hermano Benjamín, que mi boca os habla.
\par 13 Haréis, pues, saber a mi padre toda mi gloria en Egipto, y todo lo que habéis visto; y daos prisa, y traed a mi padre acá.
\par 14 Y se echó sobre el cuello de Benjamín su hermano, y lloró; y también Benjamín lloró sobre su cuello.
\par 15 Y besó a todos sus hermanos, y lloró sobre ellos; y después sus hermanos hablaron con él.
\par 16 Y se oyó la noticia en la casa de Faraón, diciendo: Los hermanos de José han venido. Y esto agradó en los ojos de Faraón y de sus siervos.
\par 17 Y dijo Faraón a José: Di a tus hermanos: Haced esto: cargad vuestras bestias, e id, volved a la tierra de Canaán;
\par 18 y tomad a vuestro padre y a vuestras familias y venid a mí, porque yo os daré lo bueno de la tierra de Egipto, y comeréis de la abundancia de la tierra.
\par 19 Y tú manda: Haced esto: tomaos de la tierra de Egipto carros para vuestros niños y vuestras mujeres, y traed a vuestro padre, y venid.
\par 20 Y no os preocupéis por vuestros enseres, porque la riqueza de la tierra de Egipto será vuestra.
\par 21 Y lo hicieron así los hijos de Israel; y les dio José carros conforme a la orden de Faraón, y les suministró víveres para el camino.
\par 22 A cada uno de todos ellos dio mudas de vestidos, y a Benjamín dio trescientas piezas de plata, y cinco mudas de vestidos.
\par 23 Y a su padre envió esto: diez asnos cargados de lo mejor de Egipto, y diez asnas cargadas de trigo, y pan y comida, para su padre en el camino.
\par 24 Y despidió a sus hermanos, y ellos se fueron. Y él les dijo: No riñáis por el camino.
\par 25 Y subieron de Egipto, y llegaron a la tierra de Canaán a Jacob su padre.
\par 26 Y le dieron las nuevas, diciendo: José vive aún; y él es señor en toda la tierra de Egipto. Y el corazón de Jacob se afligió, porque no los creía.
\par 27 Y ellos le contaron todas las palabras de José, que él les había hablado; y viendo Jacob los carros que José enviaba para llevarlo, su espíritu revivió.
\par 28 Entonces dijo Israel: Basta; José mi hijo vive todavía; iré, y le veré antes que yo muera.

\chapter{46}

\section*{Jacob y su familia en Egipto}

\par 1 Salió Israel con todo lo que tenía, y vino a Beerseba, y ofreció sacrificios al Dios de su padre Isaac.
\par 2 Y habló Dios a Israel en visiones de noche, y dijo: Jacob, Jacob. Y él respondió: Heme aquí.
\par 3 Y dijo: Yo soy Dios, el Dios de tu padre; no temas de descender a Egipto, porque allí yo haré de ti una gran nación.
\par 4 Yo descenderé contigo a Egipto, y yo también te haré volver; y la mano de José cerrará tus ojos.
\par 5 Y se levantó Jacob de Beerseba; y tomaron los hijos de Israel a su padre Jacob, y a sus niños, y a sus mujeres, en los carros que Faraón había enviado para llevarlo.
\par 6 Y tomaron sus ganados, y sus bienes que habían adquirido en la tierra de Canaán, y vinieron a Egipto, Jacob y toda su descendencia consigo;
\par 7 sus hijos, y los hijos de sus hijos consigo; sus hijas, y las hijas de sus hijos, y a toda su descendencia trajo consigo a Egipto.
\par 8 Y estos son los nombres de los hijos de Israel, que entraron en Egipto, Jacob y sus hijos: Rubén, el primogénito de Jacob.
\par 9 Y los hijos de Rubén: Hanoc, Falú, Hezrón y Carmi.
\par 10 Los hijos de Simeón: Jemuel, Jamín, Ohad, Jaquín, Zohar, y Saúl hijo de la cananea.
\par 11 Los hijos de Leví: Gersón, Coat y Merari.
\par 12 Los hijos de Judá: Er, Onán, Sela, Fares y Zara; mas Er y Onán murieron en la tierra de Canaán. Y los hijos de Fares fueron Hezrón y Hamul.
\par 13 Los hijos de Isacar: Tola, Fúa, Job y Simrón.
\par 14 Los hijos de Zabulón: Sered, Elón y Jahleel.
\par 15 Estos fueron los hijos de Lea, los que dio a luz a Jacob en Padan-aram, y además su hija Dina; treinta y tres las personas todas de sus hijos e hijas.
\par 16 Los hijos de Gad: Zifión, Hagui, Ezbón, Suni, Eri, Arodi y Areli.
\par 17 Y los hijos de Aser: Imna, Isúa, Isúi, Bería, y Sera hermana de ellos. Los hijos de Bería: Heber y Malquiel.
\par 18 Estos fueron los hijos de Zilpa, la que Labán dio a su hija Lea, y dio a luz éstos a Jacob; por todas dieciséis personas.
\par 19 Los hijos de Raquel, mujer de Jacob: José y Benjamín.
\par 20 Y nacieron a José en la tierra de Egipto Manasés y Efraín, los que le dio a luz Asenat, hija de Potifera sacerdote de On.
\par 21 Los hijos de Benjamín fueron Bela, Bequer, Asbel, Gera, Naamán, Ehi, Ros, Mupim, Hupim y Ard.
\par 22 Estos fueron los hijos de Raquel, que nacieron a Jacob; por todas catorce personas.
\par 23 Los hijos de Dan: Husim.
\par 24 Los hijos de Neftalí: Jahzeel, Guni, Jezer y Silem.
\par 25 Estos fueron los hijos de Bilha, la que dio Labán a Raquel su hija, y dio a luz éstos a Jacob; por todas siete personas.
\par 26 Todas las personas que vinieron con Jacob a Egipto, procedentes de sus lomos, sin las mujeres de los hijos de Jacob, todas las personas fueron sesenta y seis.
\par 27 Y los hijos de José, que le nacieron en Egipto, dos personas. Todas las personas de la casa de Jacob, que entraron en Egipto, fueron setenta.
\par 28 Y envió Jacob a Judá delante de sí a José, para que le viniese a ver en Gosén; y llegaron a la tierra de Gosén.
\par 29 Y José unció su carro y vino a recibir a Israel su padre en Gosén; y se manifestó a él, y se echó sobre su cuello, y lloró sobre su cuello largamente.
\par 30 Entonces Israel dijo a José: Muera yo ahora, ya que he visto tu rostro, y sé que aún vives.
\par 31 Y José dijo a sus hermanos, y a la casa de su padre: Subiré y lo haré saber a Faraón, y le diré: Mis hermanos y la casa de mi padre, que estaban en la tierra de Canaán, han venido a mí.
\par 32 Y los hombres son pastores de ovejas, porque son hombres ganaderos; y han traído sus ovejas y sus vacas, y todo lo que tenían.
\par 33 Y cuando Faraón os llamare y dijere: ¿Cuál es vuestro oficio?
\par 34 entonces diréis: Hombres de ganadería han sido tus siervos desde nuestra juventud hasta ahora, nosotros y nuestros padres; a fin de que moréis en la tierra de Gosén, porque para los egipcios es abominación todo pastor de ovejas.

\chapter{47}

\par 1 Vino José y lo hizo saber a Faraón, y dijo: Mi padre y mis hermanos, y sus ovejas y sus vacas, con todo lo que tienen, han venido de la tierra de Canaán, y he aquí están en la tierra de Gosén.
\par 2 Y de los postreros de sus hermanos tomó cinco varones, y los presentó delante de Faraón.
\par 3 Y Faraón dijo a sus hermanos: ¿Cuál es vuestro oficio? Y ellos respondieron a Faraón: Pastores de ovejas son tus siervos, así nosotros como nuestros padres.
\par 4 Dijeron además a Faraón: Para morar en esta tierra hemos venido; porque no hay pasto para las ovejas de tus siervos, pues el hambre es grave en la tierra de Canaán; por tanto, te rogamos ahora que permitas que habiten tus siervos en la tierra de Gosén.
\par 5 Entonces Faraón habló a José, diciendo: Tu padre y tus hermanos han venido a ti.
\par 6 La tierra de Egipto delante de ti está; en lo mejor de la tierra haz habitar a tu padre y a tus hermanos; habiten en la tierra de Gosén; y si entiendes que hay entre ellos hombres capaces, ponlos por mayorales del ganado mío.
\par 7 También José introdujo a Jacob su padre, y lo presentó delante de Faraón; y Jacob bendijo a Faraón.
\par 8 Y dijo Faraón a Jacob: ¿Cuántos son los días de los años de tu vida?
\par 9 Y Jacob respondió a Faraón: Los días de los años de mi peregrinación son ciento treinta años; pocos y malos han sido los días de los años de mi vida, y no han llegado a los días de los años de la vida de mis padres en los días de su peregrinación.
\par 10 Y Jacob bendijo a Faraón, y salió de la presencia de Faraón.
\par 11 Así José hizo habitar a su padre y a sus hermanos, y les dio posesión en la tierra de Egipto, en lo mejor de la tierra, en la tierra de Ramesés, como mandó Faraón.
\par 12 Y alimentaba José a su padre y a sus hermanos, y a toda la casa de su padre, con pan, según el número de los hijos.
\par 13 No había pan en toda la tierra, y el hambre era muy grave, por lo que desfalleció de hambre la tierra de Egipto y la tierra de Canaán.
\par 14 Y recogió José todo el dinero que había en la tierra de Egipto y en la tierra de Canaán, por los alimentos que de él compraban; y metió José el dinero en casa de Faraón.
\par 15 Acabado el dinero de la tierra de Egipto y de la tierra de Canaán, vino todo Egipto a José, diciendo: Danos pan; ¿por qué moriremos delante de ti, por haberse acabado el dinero?
\par 16 Y José dijo: Dad vuestros ganados y yo os daré por vuestros ganados, si se ha acabado el dinero.
\par 17 Y ellos trajeron sus ganados a José, y José les dio alimentos por caballos, y por el ganado de las ovejas, y por el ganado de las vacas, y por asnos; y les sustentó de pan por todos sus ganados aquel año.
\par 18 Acabado aquel año, vinieron a él el segundo año, y le dijeron: No encubrimos a nuestro señor que el dinero ciertamente se ha acabado; también el ganado es ya de nuestro señor; nada ha quedado delante de nuestro señor sino nuestros cuerpos y nuestra tierra.
\par 19 ¿Por qué moriremos delante de tus ojos, así nosotros como nuestra tierra? Cómpranos a nosotros y a nuestra tierra por pan, y seremos nosotros y nuestra tierra siervos de Faraón; y danos semilla para que vivamos y no muramos, y no sea asolada la tierra.
\par 20 Entonces compró José toda la tierra de Egipto para Faraón; pues los egipcios vendieron cada uno sus tierras, porque se agravó el hambre sobre ellos; y la tierra vino a ser de Faraón.
\par 21 Y al pueblo lo hizo pasar a las ciudades, desde un extremo al otro del territorio de Egipto.
\par 22 Solamente la tierra de los sacerdotes no compró, por cuanto los sacerdotes tenían ración de Faraón, y ellos comían la ración que Faraón les daba; por eso no vendieron su tierra.
\par 23 Y José dijo al pueblo: He aquí os he comprado hoy, a vosotros y a vuestra tierra, para Faraón; ved aquí semilla, y sembraréis la tierra.
\par 24 De los frutos daréis el quinto a Faraón, y las cuatro partes serán vuestras para sembrar las tierras, y para vuestro mantenimiento, y de los que están en vuestras casas, y para que coman vuestros niños.
\par 25 Y ellos respondieron: La vida nos has dado; hallemos gracia en ojos de nuestro señor, y seamos siervos de Faraón.
\par 26 Entonces José lo puso por ley hasta hoy sobre la tierra de Egipto, señalando para Faraón el quinto, excepto sólo la tierra de los sacerdotes, que no fue de Faraón.
\par 27 Así habitó Israel en la tierra de Egipto, en la tierra de Gosén; y tomaron posesión de ella, y se aumentaron, y se multiplicaron en gran manera.
\par 28 Y vivió Jacob en la tierra de Egipto diecisiete años; y fueron los días de Jacob, los años de su vida, ciento cuarenta y siete años.
\par 29 Y llegaron los días de Israel para morir, y llamó a José su hijo, y le dijo: Si he hallado ahora gracia en tus ojos, te ruego que pongas tu mano debajo de mi muslo, y harás conmigo misericordia y verdad. Te ruego que no me entierres en Egipto.
\par 30 Mas cuando duerma con mis padres, me llevarás de Egipto y me sepultarás en el sepulcro de ellos. Y José respondió: Haré como tú dices.
\par 31 E Israel dijo: Júramelo. Y José le juró. Entonces Israel se inclinó sobre la cabecera de la cama.

\chapter{48}

\section*{Jacob bendice a Efraín y a Manasés}

\par 1  Sucedió después de estas cosas que dijeron a José: He aquí tu padre está enfermo. Y él tomó consigo a sus dos hijos, Manasés y Efraín.
\par 2 Y se le hizo saber a Jacob, diciendo: He aquí tu hijo José viene a ti. Entonces se esforzó Israel, y se sentó sobre la cama,
\par 3 y dijo a José: El Dios Omnipotente me apareció en Luz en la tierra de Canaán, y me bendijo,
\par 4 y me dijo: He aquí yo te haré crecer, y te multiplicaré, y te pondré por estirpe de naciones; y daré esta tierra a tu descendencia después de ti por heredad perpetua.
\par 5 Y ahora tus dos hijos Efraín y Manasés, que te nacieron en la tierra de Egipto, antes que viniese a ti a la tierra de Egipto, míos son; como Rubén y Simeón, serán míos.
\par 6 Y los que después de ellos has engendrado, serán tuyos; por el nombre de sus hermanos serán llamados en sus heredades.
\par 7 Porque cuando yo venía de Padan-aram, se me murió Raquel en la tierra de Canaán, en el camino, como media legua de tierra viniendo a Efrata; y la sepulté allí en el camino de Efrata, que es Belén.
\par 8 Y vio Israel los hijos de José, y dijo: ¿Quiénes son éstos?
\par 9 Y respondió José a su padre: Son mis hijos, que Dios me ha dado aquí. Y él dijo: Acércalos ahora a mí, y los bendeciré.
\par 10 Y los ojos de Israel estaban tan agravados por la vejez, que no podía ver. Les hizo, pues, acercarse a él, y él les besó y les abrazó.
\par 11 Y dijo Israel a José: No pensaba yo ver tu rostro, y he aquí Dios me ha hecho ver también a tu descendencia.
\par 12 Entonces José los sacó de entre sus rodillas, y se inclinó a tierra.
\par 13 Y los tomó José a ambos, Efraín a su derecha, a la izquierda de Israel, y Manasés a su izquierda, a la derecha de Israel; y los acercó a él.
\par 14 Entonces Israel extendió su mano derecha, y la puso sobre la cabeza de Efraín, que era el menor, y su mano izquierda sobre la cabeza de Manasés, colocando así sus manos adrede, aunque Manasés era el primogénito.
\par 15 Y bendijo a José, diciendo: El Dios en cuya presencia anduvieron mis padres Abraham e Isaac, el Dios que me mantiene desde que yo soy hasta este día,
\par 16 el Angel que me liberta de todo mal, bendiga a estos jóvenes; y sea perpetuado en ellos mi nombre, y el nombre de mis padres Abraham e Isaac, y multiplíquense en gran manera en medio de la tierra.
\par 17 Pero viendo José que su padre ponía la mano derecha sobre la cabeza de Efraín, le causó esto disgusto; y asió la mano de su padre, para cambiarla de la cabeza de Efraín a la cabeza de Manasés.
\par 18 Y dijo José a su padre: No así, padre mío, porque éste es el primogénito; pon tu mano derecha sobre su cabeza.
\par 19 Mas su padre no quiso, y dijo: Lo sé, hijo mío, lo sé; también él vendrá a ser un pueblo, y será también engrandecido; pero su hermano menor será más grande que él, y su descendencia formará multitud de naciones.
\par 20 Y los bendijo aquel día, diciendo: En ti bendecirá Israel, diciendo: Hágate Dios como a Efraín y como a Manasés. Y puso a Efraín antes de Manasés.
\par 21 Y dijo Israel a José: He aquí yo muero; pero Dios estará con vosotros, y os hará volver a la tierra de vuestros padres.
\par 22 Y yo te he dado a ti una parte más que a tus hermanos, la cual tomé yo de mano del amorreo con mi espada y con mi arco.

\chapter{49}

\section*{Profecía de Jacob acerca de sus hijos}

\par 1 Y llamó Jacob a sus hijos, y dijo: Juntaos, y os declararé lo que os ha de acontecer en los días venideros.
\par 2 Juntaos y oíd, hijos de Jacob,
\par Y escuchad a vuestro padre Israel.
\par 3 Rubén, tú eres mi primogénito, mi fortaleza, y el principio de mi vigor;
\par Principal en dignidad, principal en poder.
\par 4 Impetuoso como las aguas, no serás el principal,
\par Por cuanto subiste al lecho de tu padre;
\par Entonces te envileciste, subiendo a mi estrado.
\par 5 Simeón y Leví son hermanos;
\par Armas de iniquidad sus armas.
\par 6 En su consejo no entre mi alma,
\par Ni mi espíritu se junte en su compañía.
\par Porque en su furor mataron hombres,
\par Y en su temeridad desjarretaron toros.
\par 7 Maldito su furor, que fue fiero;
\par Y su ira, que fue dura.
\par Yo los apartaré en Jacob,
\par Y los esparciré en Israel.
\par 8 Judá, te alabarán tus hermanos;
\par Tu mano en la cerviz de tus enemigos;
\par Los hijos de tu padre se inclinarán a ti.
\par 9 Cachorro de león, Judá;
\par De la presa subiste, hijo mío.
\par Se encorvó, se echó como león,
\par Así como león viejo: ¿quién lo despertará?
\par 10 No será quitado el cetro de Judá,
\par Ni el legislador de entre sus pies,
\par Hasta que venga Siloh;
\par Y a él se congregarán los pueblos.
\par 11 Atando a la vid su pollino,
\par Y a la cepa el hijo de su asna,
\par Lavó en el vino su vestido,
\par Y en la sangre de uvas su manto.
\par 12 Sus ojos, rojos del vino,
\par Y sus dientes blancos de la leche.
\par 13 Zabulón en puertos de mar habitará;
\par Será para puerto de naves,
\par Y su límite hasta Sidón.
\par 14 Isacar, asno fuerte
\par Que se recuesta entre los apriscos;
\par 15 Y vio que el descanso era bueno, y que la tierra era deleitosa;
\par Y bajó su hombro para llevar,
\par Y sirvió en tributo.
\par 16 Dan juzgará a su pueblo,
\par Como una de las tribus de Israel.
\par 17 Será Dan serpiente junto al camino,
\par Víbora junto a la senda,
\par Que muerde los talones del caballo,
\par Y hace caer hacia atrás al jinete.
\par 18 Tu salvación esperé, oh Jehová.
\par 19 Gad, ejército lo acometerá;
\par Mas él acometerá al fin.
\par 20 El pan de Aser será substancioso,
\par Y él dará deleites al rey.
\par 21 Neftalí, cierva suelta,
\par Que pronunciará dichos hermosos.
\par 22 Rama fructífera es José,
\par Rama fructífera junto a una fuente,
\par Cuyos vástagos se extienden sobre el muro.
\par 23 Le causaron amargura,
\par Le asaetearon,
\par Y le aborrecieron los arqueros;
\par 24 Mas su arco se mantuvo poderoso,
\par Y los brazos de sus manos se fortalecieron
\par Por las manos del Fuerte de Jacob
\par (Por el nombre del Pastor, la Roca de Israel),
\par 25 Por el Dios de tu padre, el cual te ayudará,
\par Por el Dios Omnipotente, el cual te bendecirá
\par Con bendiciones de los cielos de arriba,
\par Con bendiciones del abismo que está abajo,
\par Con bendiciones de los pechos y del vientre.
\par 26 Las bendiciones de tu padre
\par Fueron mayores que las bendiciones de mis progenitores;
\par Hasta el término de los collados eternos
\par Serán sobre la cabeza de José,
\par Y sobre la frente del que fue apartado de entre sus hermanos.
\par 27 Benjamín es lobo arrebatador;
\par A la mañana comerá la presa,
\par Y a la tarde repartirá los despojos.

\section*{Muerte y sepelio de Jacob}

\par 28 Todos éstos fueron las doce tribus de Israel, y esto fue lo que su padre les dijo, al bendecirlos; a cada uno por su bendición los bendijo.
\par 29 Les mandó luego, y les dijo: Yo voy a ser reunido con mi pueblo. Sepultadme con mis padres en la cueva que está en el campo de Efrón el heteo,
\par 30 en la cueva que está en el campo de Macpela, al oriente de Mamre en la tierra de Canaán, la cual compró Abraham con el mismo campo de Efrón el heteo, para heredad de sepultura.
\par 31 Allí sepultaron a Abraham y a Sara su mujer; allí sepultaron a Isaac y a Rebeca su mujer; allí también sepulté yo a Lea.
\par 32 La compra del campo y de la cueva que está en él, fue de los hijos de Het.
\par 33 Y cuando acabó Jacob de dar mandamientos a sus hijos, encogió sus pies en la cama, y expiró, y fue reunido con sus padres.

\chapter{50}

\par 1 Entonces se echó José sobre el rostro de su padre, y lloró sobre él, y lo besó.
\par 2 Y mandó José a sus siervos los médicos que embalsamasen a su padre; y los médicos embalsamaron a Israel.
\par 3 Y le cumplieron cuarenta días, porque así cumplían los días de los embalsamados, y lo lloraron los egipcios setenta días.
\par 4 Y pasados los días de su luto, habló José a los de la casa de Faraón, diciendo: Si he hallado ahora gracia en vuestros ojos, os ruego que habléis en oídos de Faraón, diciendo:
\par 5 Mi padre me hizo jurar, diciendo: He aquí que voy a morir; en el sepulcro que cavé para mí en la tierra de Canaán, allí me sepultarás; ruego, pues, que vaya yo ahora y sepulte a mi padre, y volveré.
\par 6 Y Faraón dijo: Ve, y sepulta a tu padre, como él te hizo jurar.
\par 7 Entonces José subió para sepultar a su padre; y subieron con él todos los siervos de Faraón, los ancianos de su casa, y todos los ancianos de la tierra de Egipto,
\par 8 y toda la casa de José, y sus hermanos, y la casa de su padre; solamente dejaron en la tierra de Gosén sus niños, y sus ovejas y sus vacas.
\par 9 Subieron también con él carros y gente de a caballo, y se hizo un escuadrón muy grande.
\par 10 Y llegaron hasta la era de Atad, que está al otro lado del Jordán, y endecharon allí con grande y muy triste lamentación; y José hizo a su padre duelo por siete días.
\par 11 Y viendo los moradores de la tierra, los cananeos, el llanto en la era de Atad, dijeron: Llanto grande es este de los egipcios; por eso fue llamado su nombre Abel-mizraim, que está al otro lado del Jordán.
\par 12 Hicieron, pues, sus hijos con él según les había mandado;
\par 13 pues lo llevaron sus hijos a la tierra de Canaán, y lo sepultaron en la cueva del campo de Macpela, la que había comprado Abraham con el mismo campo, para heredad de sepultura, de Efrón el heteo, al oriente de Mamre.
\par 14 Y volvió José a Egipto, él y sus hermanos, y todos los que subieron con él a sepultar a su padre, después que lo hubo sepultado.

\section*{Muerte de José}

\par 15 Viendo los hermanos de José que su padre era muerto, dijeron: Quizá nos aborrecerá José, y nos dará el pago de todo el mal que le hicimos.
\par 16 Y enviaron a decir a José: Tu padre mandó antes de su muerte, diciendo:
\par 17 Así diréis a José: Te ruego que perdones ahora la maldad de tus hermanos y su pecado, porque mal te trataron; por tanto, ahora te rogamos que perdones la maldad de los siervos del Dios de tu padre. Y José lloró mientras hablaban.
\par 18 Vinieron también sus hermanos y se postraron delante de él, y dijeron: Henos aquí por siervos tuyos.
\par 19 Y les respondió José: No temáis; ¿acaso estoy yo en lugar de Dios?
\par 20 Vosotros pensasteis mal contra mí, mas Dios lo encaminó a bien, para hacer lo que vemos hoy, para mantener en vida a mucho pueblo.
\par 21 Ahora, pues, no tengáis miedo; yo os sustentaré a vosotros y a vuestros hijos. Así los consoló, y les habló al corazón.
\par 22 Y habitó José en Egipto, él y la casa de su padre; y vivió José ciento diez años.
\par 23 Y vio José los hijos de Efraín hasta la tercera generación; también los hijos de Maquir hijo de Manasés fueron criados sobre las rodillas de José.
\par 24 Y José dijo a sus hermanos: Yo voy a morir; mas Dios ciertamente os visitará, y os hará subir de esta tierra a la tierra que juró a Abraham, a Isaac y a Jacob.
\par 25 E hizo jurar José a los hijos de Israel, diciendo: Dios ciertamente os visitará, y haréis llevar de aquí mis huesos.
\par 26 Y murió José a la edad de ciento diez años; y lo embalsamaron, y fue puesto en un ataúd en Egipto.

\end{document}
