\begin{document}

\title{Éxodo}

\chapter{1}

\section*{Aflicción de los israelitas en Egipto}

\par 1 Estos son los nombres de los hijos de Israel que entraron en Egipto con Jacob; cada uno entró con su familia:
\par 2 Rubén, Simeón, Leví, Judá,
\par 3 Isacar, Zabulón, Benjamín,
\par 4 Dan, Neftalí, Gad y Aser.
\par 5 Todas las personas que le nacieron a Jacob fueron setenta. Y José estaba en Egipto.
\par 6 Y murió José, y todos sus hermanos, y toda aquella generación.
\par 7 Y los hijos de Israel fructificaron y se multiplicaron, y fueron aumentados y fortalecidos en extremo, y se llenó de ellos la tierra.
\par 8 Entretanto, se levantó sobre Egipto un nuevo rey que no conocía a José; y dijo a su pueblo:
\par 9 He aquí, el pueblo de los hijos de Israel es mayor y más fuerte que nosotros.
\par 10 Ahora, pues, seamos sabios para con él, para que no se multiplique, y acontezca que viniendo guerra, él también se una a nuestros enemigos y pelee contra nosotros, y se vaya de la tierra.
\par 11 Entonces pusieron sobre ellos comisarios de tributos que los molestasen con sus cargas; y edificaron para Faraón las ciudades de almacenaje, Pitón y Ramesés.
\par 12 Pero cuanto más los oprimían, tanto más se multiplicaban y crecían, de manera que los egipcios temían a los hijos de Israel.
\par 13 Y los egipcios hicieron servir a los hijos de Israel con dureza,
\par 14 y amargaron su vida con dura servidumbre, en hacer barro y ladrillo, y en toda labor del campo y en todo su servicio, al cual los obligaban con rigor.
\par 15 Y habló el rey de Egipto a las parteras de las hebreas, una de las cuales se llamaba Sifra, y otra Fúa, y les dijo:
\par 16 Cuando asistáis a las hebreas en sus partos, y veáis el sexo, si es hijo, matadlo; y si es hija, entonces viva.
\par 17 Pero las parteras temieron a Dios, y no hicieron como les mandó el rey de Egipto, sino que preservaron la vida a los niños.
\par 18 Y el rey de Egipto hizo llamar a las parteras y les dijo: ¿Por qué habéis hecho esto, que habéis preservado la vida a los niños?
\par 19 Y las parteras respondieron a Faraón: Porque las mujeres hebreas no son como las egipcias; pues son robustas, y dan a luz antes que la partera venga a ellas.
\par 20 Y Dios hizo bien a las parteras; y el pueblo se multiplicó y se fortaleció en gran manera.
\par 21 Y por haber las parteras temido a Dios, él prosperó sus familias.
\par 22 Entonces Faraón mandó a todo su pueblo, diciendo: Echad al río a todo hijo que nazca, y a toda hija preservad la vida.

\chapter{2}

\section*{Nacimiento de Moisés}

\par 1 Un varón de la familia de Leví fue y tomó por mujer a una hija de Leví,
\par 2 la que concibió, y dio a luz un hijo; y viéndole que era hermoso, le tuvo escondido tres meses.
\par 3 Pero no pudiendo ocultarle más tiempo, tomó una arquilla de juncos y la calafateó con asfalto y brea, y colocó en ella al niño y lo puso en un carrizal a la orilla del río.
\par 4 Y una hermana suya se puso a lo lejos, para ver lo que le acontecería.
\par 5 Y la hija de Faraón descendió a lavarse al río, y paseándose sus doncellas por la ribera del río, vio ella la arquilla en el carrizal, y envió una criada suya a que la tomase.
\par 6 Y cuando la abrió, vio al niño; y he aquí que el niño lloraba. Y teniendo compasión de él, dijo: De los niños de los hebreos es éste.
\par 7 Entonces su hermana dijo a la hija de Faraón: ¿Iré a llamarte una nodriza de las hebreas, para que te críe este niño?
\par 8 Y la hija de Faraón respondió: Ve. Entonces fue la doncella, y llamó a la madre del niño,
\par 9 a la cual dijo la hija de Faraón: Lleva a este niño y críamelo, y yo te lo pagaré. Y la mujer tomó al niño y lo crió.
\par 10 Y cuando el niño creció, ella lo trajo a la hija de Faraón, la cual lo prohijó, y le puso por nombre Moisés, diciendo: Porque de las aguas lo saqué.

\section*{Moisés huye de Egipto}

\par 11 En aquellos días sucedió que crecido ya Moisés, salió a sus hermanos, y los vio en sus duras tareas, y observó a un egipcio que golpeaba a uno de los hebreos, sus hermanos.
\par 12 Entonces miró a todas partes, y viendo que no parecía nadie, mató al egipcio y lo escondió en la arena.
\par 13 Al día siguiente salió y vio a dos hebreos que reñían; entonces dijo al que maltrataba al otro: ¿Por qué golpeas a tu prójimo?
\par 14 Y él respondió: ¿Quién te ha puesto a ti por príncipe y juez sobre nosotros? ¿Piensas matarme como mataste al egipcio? Entonces Moisés tuvo miedo, y dijo: Ciertamente esto ha sido descubierto.
\par 15 Oyendo Faraón acerca de este hecho, procuró matar a Moisés; pero Moisés huyó de delante de Faraón, y habitó en la tierra de Madián.
\par 16 Y estando sentado junto al pozo, siete hijas que tenía el sacerdote de Madián vinieron a sacar agua para llenar las pilas y dar de beber a las ovejas de su padre.
\par 17 Mas los pastores vinieron y las echaron de allí; entonces Moisés se levantó y las defendió, y dio de beber a sus ovejas.
\par 18 Y volviendo ellas a Reuel su padre, él les dijo: ¿Por qué habéis venido hoy tan pronto?
\par 19 Ellas respondieron: Un varón egipcio nos defendió de mano de los pastores, y también nos sacó el agua, y dio de beber a las ovejas.
\par 20 Y dijo a sus hijas: ¿Dónde está? ¿Por qué habéis dejado a ese hombre? Llamadle para que coma.
\par 21 Y Moisés convino en morar con aquel varón; y él dio su hija Séfora por mujer a Moisés.
\par 22 Y ella le dio a luz un hijo; y él le puso por nombre Gersón, porque dijo: Forastero soy en tierra ajena.
\par 23 Aconteció que después de muchos días murió el rey de Egipto, y los hijos de Israel gemían a causa de la servidumbre, y clamaron; y subió a Dios el clamor de ellos con motivo de su servidumbre.
\par 24 Y oyó Dios el gemido de ellos, y se acordó de su pacto con Abraham, Isaac y Jacob.
\par 25 Y miró Dios a los hijos de Israel, y los reconoció Dios.

\chapter{3}

\section*{Llamamiento de Moisés}
\par 1 Apacentando Moisés las ovejas de Jetro su suegro, sacerdote de Madián, llevó las ovejas a través del desierto, y llegó hasta Horeb, monte de Dios.
\par 2 Y se le apareció el Angel de Jehová en una llama de fuego en medio de una zarza; y él miró, y vio que la zarza ardía en fuego, y la zarza no se consumía.
\par 3 Entonces Moisés dijo: Iré yo ahora y veré esta grande visión, por qué causa la zarza no se quema.
\par 4 Viendo Jehová que él iba a ver, lo llamó Dios de en medio de la zarza, y dijo: !!Moisés, Moisés! Y él respondió: Heme aquí.
\par 5 Y dijo: No te acerques; quita tu calzado de tus pies, porque el lugar en que tú estás, tierra santa es.
\par 6 Y dijo: Yo soy el Dios de tu padre, Dios de Abraham, Dios de Isaac, y Dios de Jacob. Entonces Moisés cubrió su rostro, porque tuvo miedo de mirar a Dios.
\par 7 Dijo luego Jehová: Bien he visto la aflicción de mi pueblo que está en Egipto, y he oído su clamor a causa de sus exactores; pues he conocido sus angustias,
\par 8 y he descendido para librarlos de mano de los egipcios, y sacarlos de aquella tierra a una tierra buena y ancha, a tierra que fluye leche y miel, a los lugares del cananeo, del heteo, del amorreo, del ferezeo, del heveo y del jebuseo.
\par 9 El clamor, pues, de los hijos de Israel ha venido delante de mí, y también he visto la opresión con que los egipcios los oprimen.
\par 10 Ven, por tanto, ahora, y te enviaré a Faraón, para que saques de Egipto a mi pueblo, los hijos de Israel.
\par 11 Entonces Moisés respondió a Dios: ¿Quién soy yo para que vaya a Faraón, y saque de Egipto a los hijos de Israel?
\par 12 Y él respondió: Ve, porque yo estaré contigo; y esto te será por señal de que yo te he enviado: cuando hayas sacado de Egipto al pueblo, serviréis a Dios sobre este monte.
\par 13 Dijo Moisés a Dios: He aquí que llego yo a los hijos de Israel, y les digo: El Dios de vuestros padres me ha enviado a vosotros. Si ellos me preguntaren: ¿Cuál es su nombre?, ¿qué les responderé?
\par 14 Y respondió Dios a Moisés: YO SOY EL QUE SOY. Y dijo: Así dirás a los hijos de Israel: YO SOY me envió a vosotros.
\par 15 Además dijo Dios a Moisés: Así dirás a los hijos de Israel: Jehová, el Dios de vuestros padres, el Dios de Abraham, Dios de Isaac y Dios de Jacob, me ha enviado a vosotros. Este es mi nombre para siempre; con él se me recordará por todos los siglos.
\par 16 Ve, y reúne a los ancianos de Israel, y diles: Jehová, el Dios de vuestros padres, el Dios de Abraham, de Isaac y de Jacob, me apareció diciendo: En verdad os he visitado, y he visto lo que se os hace en Egipto;
\par 17 y he dicho: Yo os sacaré de la aflicción de Egipto a la tierra del cananeo, del heteo, del amorreo, del ferezeo, del heveo y del jebuseo, a una tierra que fluye leche y miel.
\par 18 Y oirán tu voz; e irás tú, y los ancianos de Israel, al rey de Egipto, y le diréis: Jehová el Dios de los hebreos nos ha encontrado; por tanto, nosotros iremos ahora camino de tres días por el desierto, para que ofrezcamos sacrificios a Jehová nuestro Dios.
\par 19 Mas yo sé que el rey de Egipto no os dejará ir sino por mano fuerte.
\par 20 Pero yo extenderé mi mano, y heriré a Egipto con todas mis maravillas que haré en él, y entonces os dejará ir.
\par 21 Y yo daré a este pueblo gracia en los ojos de los egipcios, para que cuando salgáis, no vayáis con las manos vacías;
\par 22 sino que pedirá cada mujer a su vecina y a su huéspeda alhajas de plata, alhajas de oro, y vestidos, los cuales pondréis sobre vuestros hijos y vuestras hijas; y despojaréis a Egipto.

\chapter{4}

\par 1 Entonces Moisés respondió diciendo: He aquí que ellos no me creerán, ni oirán mi voz; porque dirán: No te ha aparecido Jehová.
\par 2 Y Jehová dijo: ¿Qué es eso que tienes en tu mano? Y él respondió: Una vara.
\par 3 El le dijo: Echala en tierra. Y él la echó en tierra, y se hizo una culebra; y Moisés huía de ella.
\par 4 Entonces dijo Jehová a Moisés: Extiende tu mano, y tómala por la cola. Y él extendió su mano, y la tomó, y se volvió vara en su mano.
\par 5 Por esto creerán que se te ha aparecido Jehová, el Dios de tus padres, el Dios de Abraham, Dios de Isaac y Dios de Jacob.
\par 6 Le dijo además Jehová: Mete ahora tu mano en tu seno. Y él metió la mano en su seno; y cuando la sacó, he aquí que su mano estaba leprosa como la nieve.
\par 7 Y dijo: Vuelve a meter tu mano en tu seno. Y él volvió a meter su mano en su seno; y al sacarla de nuevo del seno, he aquí que se había vuelto como la otra carne.
\par 8 Si aconteciere que no te creyeren ni obedecieren a la voz de la primera señal, creerán a la voz de la postrera.
\par 9 Y si aún no creyeren a estas dos señales, ni oyeren tu voz, tomarás de las aguas del río y las derramarás en tierra; y se cambiarán aquellas aguas que tomarás del río y se harán sangre en la tierra.
\par 10 Entonces dijo Moisés a Jehová: !!Ay, Señor! nunca he sido hombre de fácil palabra, ni antes, ni desde que tú hablas a tu siervo; porque soy tardo en el habla y torpe de lengua.
\par 11 Y Jehová le respondió: ¿Quién dio la boca al hombre? ¿o quién hizo al mudo y al sordo, al que ve y al ciego? ¿No soy yo Jehová?
\par 12 Ahora pues, ve, y yo estaré con tu boca, y te enseñaré lo que hayas de hablar.
\par 13 Y él dijo: !!Ay, Señor! envía, te ruego, por medio del que debes enviar.
\par 14 Entonces Jehová se enojó contra Moisés, y dijo: ¿No conozco yo a tu hermano Aarón, levita, y que él habla bien? Y he aquí que él saldrá a recibirte, y al verte se alegrará en su corazón.
\par 15 Tú hablarás a él, y pondrás en su boca las palabras, y yo estaré con tu boca y con la suya, y os enseñaré lo que hayáis de hacer.
\par 16 Y él hablará por ti al pueblo; él te será a ti en lugar de boca, y tú serás para él en lugar de Dios.
\par 17 Y tomarás en tu mano esta vara, con la cual harás las señales.

\section*{Moisés vuelve a Egipto}

\par 18 Así se fue Moisés, y volviendo a su suegro Jetro, le dijo: Iré ahora, y volveré a mis hermanos que están en Egipto, para ver si aún viven. Y Jetro dijo a Moisés: Ve en paz.
\par 19 Dijo también Jehová a Moisés en Madián: Ve y vuélvete a Egipto, porque han muerto todos los que procuraban tu muerte.
\par 20 Entonces Moisés tomó su mujer y sus hijos, y los puso sobre un asno, y volvió a tierra de Egipto. Tomó también Moisés la vara de Dios en su mano.
\par 21 Y dijo Jehová a Moisés: Cuando hayas vuelto a Egipto, mira que hagas delante de Faraón todas las maravillas que he puesto en tu mano; pero yo endureceré su corazón, de modo que no dejará ir al pueblo.
\par 22 Y dirás a Faraón: Jehová ha dicho así: Israel es mi hijo, mi primogénito.
\par 23 Ya te he dicho que dejes ir a mi hijo, para que me sirva, mas no has querido dejarlo ir; he aquí yo voy a matar a tu hijo, tu primogénito.
\par 24 Y aconteció en el camino, que en una posada Jehová le salió al encuentro, y quiso matarlo.
\par 25 Entonces Séfora tomó un pedernal afilado y cortó el prepucio de su hijo, y lo echó a sus pies, diciendo: A la verdad tú me eres un esposo de sangre.
\par 26 Así le dejó luego ir. Y ella dijo: Esposo de sangre, a causa de la circuncisión.
\par 27 Y Jehová dijo a Aarón: Ve a recibir a Moisés al desierto. Y él fue, y lo encontró en el monte de Dios, y le besó.
\par 28 Entonces contó Moisés a Aarón todas las palabras de Jehová que le enviaba, y todas las señales que le había dado.
\par 29 Y fueron Moisés y Aarón, y reunieron a todos los ancianos de los hijos de Israel.
\par 30 Y habló Aarón acerca de todas las cosas que Jehová había dicho a Moisés, e hizo las señales delante de los ojos del pueblo.
\par 31 Y el pueblo creyó; y oyendo que Jehová había visitado a los hijos de Israel, y que había visto su aflicción, se inclinaron y adoraron.

\chapter{5}

\section*{Moisés y Aarón ante Faraón}

\par 1 Después Moisés y Aarón entraron a la presencia de Faraón y le dijeron: Jehová el Dios de Israel dice así: Deja ir a mi pueblo a celebrarme fiesta en el desierto.
\par 2 Y Faraón respondió: ¿Quién es Jehová, para que yo oiga su voz y deje ir a Israel? Yo no conozco a Jehová, ni tampoco dejaré ir a Israel.
\par 3 Y ellos dijeron: El Dios de los hebreos nos ha encontrado; iremos, pues, ahora, camino de tres días por el desierto, y ofreceremos sacrificios a Jehová nuestro Dios, para que no venga sobre nosotros con peste o con espada.
\par 4 Entonces el rey de Egipto les dijo: Moisés y Aarón, ¿por qué hacéis cesar al pueblo de su trabajo? Volved a vuestras tareas.
\par 5 Dijo también Faraón: He aquí el pueblo de la tierra es ahora mucho, y vosotros les hacéis cesar de sus tareas.
\par 6 Y mandó Faraón aquel mismo día a los cuadrilleros del pueblo que lo tenían a su cargo, y a sus capataces, diciendo:
\par 7 De aquí en adelante no daréis paja al pueblo para hacer ladrillo, como hasta ahora; vayan ellos y recojan por sí mismos la paja.
\par 8 Y les impondréis la misma tarea de ladrillo que hacían antes, y no les disminuiréis nada; porque están ociosos, por eso levantan la voz diciendo: Vamos y ofrezcamos sacrificios a nuestro Dios.
\par 9 Agrávese la servidumbre sobre ellos, para que se ocupen en ella, y no atiendan a palabras mentirosas.
\par 10 Y saliendo los cuadrilleros del pueblo y sus capataces, hablaron al pueblo, diciendo: Así ha dicho Faraón: Yo no os doy paja.
\par 11 Id vosotros y recoged la paja donde la halléis; pero nada se disminuirá de vuestra tarea.
\par 12 Entonces el pueblo se esparció por toda la tierra de Egipto para recoger rastrojo en lugar de paja.
\par 13 Y los cuadrilleros los apremiaban, diciendo: Acabad vuestra obra, la tarea de cada día en su día, como cuando se os daba paja.
\par 14 Y azotaban a los capataces de los hijos de Israel que los cuadrilleros de Faraón habían puesto sobre ellos, diciendo: ¿Por qué no habéis cumplido vuestra tarea de ladrillo ni ayer ni hoy, como antes?
\par 15 Y los capataces de los hijos de Israel vinieron a Faraón y se quejaron a él, diciendo: ¿Por qué lo haces así con tus siervos?
\par 16 No se da paja a tus siervos, y con todo nos dicen: Haced el ladrillo. Y he aquí tus siervos son azotados, y el pueblo tuyo es el culpable.
\par 17 Y él respondió: Estáis ociosos, sí, ociosos, y por eso decís: Vamos y ofrezcamos sacrificios a Jehová.
\par 18 Id pues, ahora, y trabajad. No se os dará paja, y habéis de entregar la misma tarea de ladrillo.
\par 19 Entonces los capataces de los hijos de Israel se vieron en aflicción, al decírseles: No se disminuirá nada de vuestro ladrillo, de la tarea de cada día.
\par 20 Y encontrando a Moisés y a Aarón, que estaban a la vista de ellos cuando salían de la presencia de Faraón,
\par 21 les dijeron: Mire Jehová sobre vosotros, y juzgue; pues nos habéis hecho abominables delante de Faraón y de sus siervos, poniéndoles la espada en la mano para que nos maten.

\section*{Jehová comisiona a Moisés y a Aarón}

\par 22 Entonces Moisés se volvió a Jehová, y dijo: Señor, ¿por qué afliges a este pueblo? ¿Para qué me enviaste?
\par 23 Porque desde que yo vine a Faraón para hablarle en tu nombre, ha afligido a este pueblo; y tú no has librado a tu pueblo.

\chapter{6}

\par 1 Jehová respondió a Moisés: Ahora verás lo que yo haré a Faraón; porque con mano fuerte los dejará ir, y con mano fuerte los echará de su tierra.
\par 2 Habló todavía Dios a Moisés, y le dijo: Yo soy JEHOVÁ.
\par 3 Y aparecí a Abraham, a Isaac y a Jacob como Dios Omnipotente, mas en mi nombre JEHOVÁ no me di a conocer a ellos.
\par 4 También establecí mi pacto con ellos, de darles la tierra de Canaán, la tierra en que fueron forasteros, y en la cual habitaron.
\par 5 Asimismo yo he oído el gemido de los hijos de Israel, a quienes hacen servir los egipcios, y me he acordado de mi pacto.
\par 6 Por tanto, dirás a los hijos de Israel: Yo soy JEHOVÁ; y yo os sacaré de debajo de las tareas pesadas de Egipto, y os libraré de su servidumbre, y os redimiré con brazo extendido, y con juicios grandes;
\par 7 y os tomaré por mi pueblo y seré vuestro Dios; y vosotros sabréis que yo soy Jehová vuestro Dios, que os sacó de debajo de las tareas pesadas de Egipto.
\par 8 Y os meteré en la tierra por la cual alcé mi mano jurando que la daría a Abraham, a Isaac y a Jacob; y yo os la daré por heredad. Yo JEHOVÁ.
\par 9 De esta manera habló Moisés a los hijos de Israel; pero ellos no escuchaban a Moisés a causa de la congoja de espíritu, y de la dura servidumbre.
\par 10 Y habló Jehová a Moisés, diciendo:
\par 11 Entra y habla a Faraón rey de Egipto, que deje ir de su tierra a los hijos de Israel.
\par 12 Y respondió Moisés delante de Jehová: He aquí, los hijos de Israel no me escuchan; ¿cómo, pues, me escuchará Faraón, siendo yo torpe de labios?
\par 13 Entonces Jehová habló a Moisés y a Aarón y les dio mandamiento para los hijos de Israel, y para Faraón rey de Egipto, para que sacasen a los hijos de Israel de la tierra de Egipto.
\par 14 Estos son los jefes de las familias de sus padres: Los hijos de Rubén, el primogénito de Israel: Hanoc, Falú, Hezrón y Carmi; estas son las familias de Rubén.
\par 15 Los hijos de Simeón: Jemuel, Jamín, Ohad, Jaquín, Zohar, y Saúl hijo de una cananea. Estas son las familias de Simeón.
\par 16 Estos son los nombres de los hijos de Leví por sus linajes: Gersón, Coat y Merari. Y los años de la vida de Leví fueron ciento treinta y siete años.
\par 17 Los hijos de Gersón: Libni y Simei, por sus familias.
\par 18 Y los hijos de Coat: Amram, Izhar, Hebrón y Uziel. Y los años de la vida de Coat fueron ciento treinta y tres años.
\par 19 Y los hijos de Merari: Mahli y Musi. Estas son las familias de Leví por sus linajes.
\par 20 Y Amram tomó por mujer a Jocabed su tía, la cual dio a luz a Aarón y a Moisés. Y los años de la vida de Amram fueron ciento treinta y siete años.
\par 21 Los hijos de Izhar: Coré, Nefeg y Zicri.
\par 22 Y los hijos de Uziel: Misael, Elzafán y Sitri.
\par 23 Y tomó Aarón por mujer a Elisabet hija de Aminadab, hermana de Naasón; la cual dio a luz a Nadab, Abiú, Eleazar e Itamar.
\par 24 Los hijos de Coré: Asir, Elcana y Abiasaf. Estas son las familias de los coreítas.
\par 25 Y Eleazar hijo de Aarón tomó para sí mujer de las hijas de Futiel, la cual dio a luz a Finees. Y estos son los jefes de los padres de los levitas por sus familias.
\par 26 Este es aquel Aarón y aquel Moisés, a los cuales Jehová dijo: Sacad a los hijos de Israel de la tierra de Egipto por sus ejércitos.
\par 27 Estos son los que hablaron a Faraón rey de Egipto, para sacar de Egipto a los hijos de Israel. Moisés y Aarón fueron éstos.
\par 28 Cuando Jehová habló a Moisés en la tierra de Egipto,
\par 29 entonces Jehová habló a Moisés, diciendo: Yo soy JEHOVÁ; di a Faraón rey de Egipto todas las cosas que yo te digo a ti.
\par 30 Y Moisés respondió delante de Jehová: He aquí, yo soy torpe de labios; ¿cómo, pues, me ha de oír Faraón?

\chapter{7}

\par 1 Jehová dijo a Moisés: Mira, yo te he constituido dios para Faraón, y tu hermano Aarón será tu profeta.
\par 2 Tú dirás todas las cosas que yo te mande, y Aarón tu hermano hablará a Faraón, para que deje ir de su tierra a los hijos de Israel.
\par 3 Y yo endureceré el corazón de Faraón, y multiplicaré en la tierra de Egipto mis señales y mis maravillas.
\par 4 Y Faraón no os oirá; mas yo pondré mi mano sobre Egipto, y sacaré a mis ejércitos, mi pueblo, los hijos de Israel, de la tierra de Egipto, con grandes juicios.
\par 5 Y sabrán los egipcios que yo soy Jehová, cuando extienda mi mano sobre Egipto, y saque a los hijos de Israel de en medio de ellos.
\par 6 E hizo Moisés y Aarón como Jehová les mandó; así lo hicieron.
\par 7 Era Moisés de edad de ochenta años, y Aarón de edad de ochenta y tres, cuando hablaron a Faraón.

\section*{La vara de Aarón}

\par 8 Habló Jehová a Moisés y a Aarón, diciendo:
\par 9 Si Faraón os respondiere diciendo: Mostrad milagro; dirás a Aarón: Toma tu vara, y échala delante de Faraón, para que se haga culebra.
\par 10 Vinieron, pues, Moisés y Aarón a Faraón, e hicieron como Jehová lo había mandado. Y echó Aarón su vara delante de Faraón y de sus siervos, y se hizo culebra.
\par 11 Entonces llamó también Faraón sabios y hechiceros, e hicieron también lo mismo los hechiceros de Egipto con sus encantamientos;
\par 12 pues echó cada uno su vara, las cuales se volvieron culebras; mas la vara de Aarón devoró las varas de ellos.
\par 13 Y el corazón de Faraón se endureció, y no los escuchó, como Jehová lo había dicho.

\section*{La plaga de sangre}

\par 14 Entonces Jehová dijo a Moisés: El corazón de Faraón está endurecido, y no quiere dejar ir al pueblo.
\par 15 Ve por la mañana a Faraón, he aquí que él sale al río; y tú ponte a la ribera delante de él, y toma en tu mano la vara que se volvió culebra,
\par 16 y dile: Jehová el Dios de los hebreos me ha enviado a ti, diciendo: Deja ir a mi pueblo, para que me sirva en el desierto; y he aquí que hasta ahora no has querido oír.
\par 17 Así ha dicho Jehová: En esto conocerás que yo soy Jehová: he aquí, yo golpearé con la vara que tengo en mi mano el agua que está en el río, y se convertirá en sangre.
\par 18 Y los peces que hay en el río morirán, y hederá el río, y los egipcios tendrán asco de beber el agua del río.
\par 19 Y Jehová dijo a Moisés: Di a Aarón: Toma tu vara, y extiende tu mano sobre las aguas de Egipto, sobre sus ríos, sobre sus arroyos y sobre sus estanques, y sobre todos sus depósitos de aguas, para que se conviertan en sangre, y haya sangre por toda la región de Egipto, así en los vasos de madera como en los de piedra.
\par 20 Y Moisés y Aarón hicieron como Jehová lo mandó; y alzando la vara golpeó las aguas que había en el río, en presencia de Faraón y de sus siervos; y todas las aguas que había en el río se convirtieron en sangre.
\par 21 Asimismo los peces que había en el río murieron; y el río se corrompió, tanto que los egipcios no podían beber de él. Y hubo sangre por toda la tierra de Egipto.
\par 22 Y los hechiceros de Egipto hicieron lo mismo con sus encantamientos; y el corazón de Faraón se endureció, y no los escuchó; como Jehová lo había dicho.
\par 23 Y Faraón se volvió y fue a su casa, y no dio atención tampoco a esto.
\par 24 Y en todo Egipto hicieron pozos alrededor del río para beber, porque no podían beber de las aguas del río.
\par 25 Y se cumplieron siete días después que Jehová hirió el río.

\chapter{8}

\section*{La plaga de ranas}

\par 1 Entonces Jehová dijo a Moisés: Entra a la presencia de Faraón y dile: Jehová ha dicho así: Deja ir a mi pueblo, para que me sirva.
\par 2 Y si no lo quisieres dejar ir, he aquí yo castigaré con ranas todos tus territorios.
\par 3 Y el río criará ranas, las cuales subirán y entrarán en tu casa, en la cámara donde duermes, y sobre tu cama, y en las casas de tus siervos, en tu pueblo, en tus hornos y en tus artesas.
\par 4 Y las ranas subirán sobre ti, sobre tu pueblo, y sobre todos tus siervos.
\par 5 Y Jehová dijo a Moisés: Di a Aarón: Extiende tu mano con tu vara sobre los ríos, arroyos y estanques, para que haga subir ranas sobre la tierra de Egipto.
\par 6 Entonces Aarón extendió su mano sobre las aguas de Egipto, y subieron ranas que cubrieron la tierra de Egipto.
\par 7 Y los hechiceros hicieron lo mismo con sus encantamientos, e hicieron venir ranas sobre la tierra de Egipto.
\par 8 Entonces Faraón llamó a Moisés y a Aarón, y les dijo: Orad a Jehová para que quite las ranas de mí y de mi pueblo, y dejaré ir a tu pueblo para que ofrezca sacrificios a Jehová.
\par 9 Y dijo Moisés a Faraón: Dígnate indicarme cuándo debo orar por ti, por tus siervos y por tu pueblo, para que las ranas sean quitadas de ti y de tus casas, y que solamente queden en el río.
\par 10 Y él dijo: Mañana. Y Moisés respondió: Se hará conforme a tu palabra, para que conozcas que no hay como Jehová nuestro Dios.
\par 11 Y las ranas se irán de ti, y de tus casas, de tus siervos y de tu pueblo, y solamente quedarán en el río.
\par 12 Entonces salieron Moisés y Aarón de la presencia de Faraón. Y clamó Moisés a Jehová tocante a las ranas que había mandado a Faraón.
\par 13 E hizo Jehová conforme a la palabra de Moisés, y murieron las ranas de las casas, de los cortijos y de los campos.
\par 14 Y las juntaron en montones, y apestaba la tierra.
\par 15 Pero viendo Faraón que le habían dado reposo, endureció su corazón y no los escuchó, como Jehová lo había dicho.

\section*{La plaga de piojos}

\par 16 Entonces Jehová dijo a Moisés: Di a Aarón: Extiende tu vara y golpea el polvo de la tierra, para que se vuelva piojos por todo el país de Egipto.
\par 17 Y ellos lo hicieron así; y Aarón extendió su mano con su vara, y golpeó el polvo de la tierra, el cual se volvió piojos, así en los hombres como en las bestias; todo el polvo de la tierra se volvió piojos en todo el país de Egipto.
\par 18 Y los hechiceros hicieron así también, para sacar piojos con sus encantamientos; pero no pudieron. Y hubo piojos tanto en los hombres como en las bestias.
\par 19 Entonces los hechiceros dijeron a Faraón: Dedo de Dios es éste. Mas el corazón de Faraón se endureció, y no los escuchó, como Jehová lo había dicho.

\section*{La plaga de moscas}

\par 20 Jehová dijo a Moisés: Levántate de mañana y ponte delante de Faraón, he aquí él sale al río; y dile: Jehová ha dicho así: Deja ir a mi pueblo, para que me sirva.
\par 21 Porque si no dejas ir a mi pueblo, he aquí yo enviaré sobre ti, sobre tus siervos, sobre tu pueblo y sobre tus casas toda clase de moscas; y las casas de los egipcios se llenarán de toda clase de moscas, y asimismo la tierra donde ellos estén.
\par 22 Y aquel día yo apartaré la tierra de Gosén, en la cual habita mi pueblo, para que ninguna clase de moscas haya en ella, a fin de que sepas que yo soy Jehová en medio de la tierra.
\par 23 Y yo pondré redención entre mi pueblo y el tuyo. Mañana será esta señal.
\par 24 Y Jehová lo hizo así, y vino toda clase de moscas molestísimas sobre la casa de Faraón, sobre las casas de sus siervos, y sobre todo el país de Egipto; y la tierra fue corrompida a causa de ellas.
\par 25 Entonces Faraón llamó a Moisés y a Aarón, y les dijo: Andad, ofreced sacrificio a vuestro Dios en la tierra.
\par 26 Y Moisés respondió: No conviene que hagamos así, porque ofreceríamos a Jehová nuestro Dios la abominación de los egipcios. He aquí, si sacrificáramos la abominación de los egipcios delante de ellos, ¿no nos apedrearían?
\par 27 Camino de tres días iremos por el desierto, y ofreceremos sacrificios a Jehová nuestro Dios, como él nos dirá.
\par 28 Dijo Faraón: Yo os dejaré ir para que ofrezcáis sacrificios a Jehová vuestro Dios en el desierto, con tal que no vayáis más lejos; orad por mí.
\par 29 Y respondió Moisés: He aquí, al salir yo de tu presencia, rogaré a Jehová que las diversas clases de moscas se vayan de Faraón, y de sus siervos, y de su pueblo mañana; con tal que Faraón no falte más, no dejando ir al pueblo a dar sacrificio a Jehová.
\par 30 Entonces Moisés salió de la presencia de Faraón, y oró a Jehová.
\par 31 Y Jehová hizo conforme a la palabra de Moisés, y quitó todas aquellas moscas de Faraón, de sus siervos y de su pueblo, sin que quedara una.
\par 32 Mas Faraón endureció aun esta vez su corazón, y no dejó ir al pueblo.

\chapter{9}

\section*{La plaga en el ganado}

\par 1 Entonces Jehová dijo a Moisés: Entra a la presencia de Faraón, y dile: Jehová, el Dios de los hebreos, dice así: Deja ir a mi pueblo, para que me sirva.
\par 2 Porque si no lo quieres dejar ir, y lo detienes aún,
\par 3 he aquí la mano de Jehová estará sobre tus ganados que están en el campo, caballos, asnos, camellos, vacas y ovejas, con plaga gravísima.
\par 4 Y Jehová hará separación entre los ganados de Israel y los de Egipto, de modo que nada muera de todo lo de los hijos de Israel.
\par 5 Y Jehová fijó plazo, diciendo: Mañana hará Jehová esta cosa en la tierra.
\par 6 Al día siguiente Jehová hizo aquello, y murió todo el ganado de Egipto; mas del ganado de los hijos de Israel no murió uno.
\par 7 Entonces Faraón envió, y he aquí que del ganado de los hijos de Israel no había muerto uno. Mas el corazón de Faraón se endureció, y no dejó ir al pueblo.

\section*{La plaga de úlceras}

\par 8 Y Jehová dijo a Moisés y a Aarón: Tomad puñados de ceniza de un horno, y la esparcirá Moisés hacia el cielo delante de Faraón;
\par 9 y vendrá a ser polvo sobre toda la tierra de Egipto, y producirá sarpullido con úlceras en los hombres y en las bestias, por todo el país de Egipto.
\par 10 Y tomaron ceniza del horno, y se pusieron delante de Faraón, y la esparció Moisés hacia el cielo; y hubo sarpullido que produjo úlceras tanto en los hombres como en las bestias.
\par 11 Y los hechiceros no podían estar delante de Moisés a causa del sarpullido, porque hubo sarpullido en los hechiceros y en todos los egipcios.
\par 12 Pero Jehová endureció el corazón de Faraón, y no los oyó, como Jehová lo había dicho a Moisés.

\section*{La plaga de granizo}

\par 13 Entonces Jehová dijo a Moisés: Levántate de mañana, y ponte delante de Faraón, y dile: Jehová, el Dios de los hebreos, dice así: Deja ir a mi pueblo, para que me sirva.
\par 14 Porque yo enviaré esta vez todas mis plagas a tu corazón, sobre tus siervos y sobre tu pueblo, para que entiendas que no hay otro como yo en toda la tierra.
\par 15 Porque ahora yo extenderé mi mano para herirte a ti y a tu pueblo de plaga, y serás quitado de la tierra.
\par 16 Y a la verdad yo te he puesto para mostrar en ti mi poder, y para que mi nombre sea anunciado en toda la tierra.
\par 17 ¿Todavía te ensoberbeces contra mi pueblo, para no dejarlos ir?
\par 18 He aquí que mañana a estas horas yo haré llover granizo muy pesado, cual nunca hubo en Egipto, desde el día que se fundó hasta ahora.
\par 19 Envía, pues, a recoger tu ganado, y todo lo que tienes en el campo; porque todo hombre o animal que se halle en el campo, y no sea recogido a casa, el granizo caerá sobre él, y morirá.
\par 20 De los siervos de Faraón, el que tuvo temor de la palabra de Jehová hizo huir sus criados y su ganado a casa;
\par 21 mas el que no puso en su corazón la palabra de Jehová, dejó sus criados y sus ganados en el campo.
\par 22 Y Jehová dijo a Moisés: Extiende tu mano hacia el cielo, para que venga granizo en toda la tierra de Egipto sobre los hombres, y sobre las bestias, y sobre toda la hierba del campo en el país de Egipto.
\par 23 Y Moisés extendió su vara hacia el cielo, y Jehová hizo tronar y granizar, y el fuego se descargó sobre la tierra; y Jehová hizo llover granizo sobre la tierra de Egipto.
\par 24 Hubo, pues, granizo, y fuego mezclado con el granizo, tan grande, cual nunca hubo en toda la tierra de Egipto desde que fue habitada.
\par 25 Y aquel granizo hirió en toda la tierra de Egipto todo lo que estaba en el campo, así hombres como bestias; asimismo destrozó el granizo toda la hierba del campo, y desgajó todos los árboles del país.
\par 26 Solamente en la tierra de Gosén, donde estaban los hijos de Israel, no hubo granizo.
\par 27 Entonces Faraón envió a llamar a Moisés y a Aarón, y les dijo: He pecado esta vez; Jehová es justo, y yo y mi pueblo impíos.
\par 28 Orad a Jehová para que cesen los truenos de Dios y el granizo, y yo os dejaré ir, y no os detendréis más.
\par 29 Y le respondió Moisés: Tan pronto salga yo de la ciudad, extenderé mis manos a Jehová, y los truenos cesarán, y no habrá más granizo; para que sepas que de Jehová es la tierra.
\par 30 Pero yo sé que ni tú ni tus siervos temeréis todavía la presencia de Jehová Dios.
\par 31 El lino, pues, y la cebada fueron destrozados, porque la cebada estaba ya espigada, y el lino en caña.
\par 32 Mas el trigo y el centeno no fueron destrozados, porque eran tardíos.
\par 33 Y salido Moisés de la presencia de Faraón, fuera de la ciudad, extendió sus manos a Jehová, y cesaron los truenos y el granizo, y la lluvia no cayó más sobre la tierra.
\par 34 Y viendo Faraón que la lluvia había cesado, y el granizo y los truenos, se obstinó en pecar, y endurecieron su corazón él y sus siervos.
\par 35 Y el corazón de Faraón se endureció, y no dejó ir a los hijos de Israel, como Jehová lo había dicho por medio de Moisés.

\chapter{10}

\section*{La plaga de langostas}

\par 1 Jehová dijo a Moisés: Entra a la presencia de Faraón; porque yo he endurecido su corazón, y el corazón de sus siervos, para mostrar entre ellos estas mis señales,
\par 2 y para que cuentes a tus hijos y a tus nietos las cosas que yo hice en Egipto, y mis señales que hice entre ellos; para que sepáis que yo soy Jehová.
\par 3 Entonces vinieron Moisés y Aarón a Faraón, y le dijeron: Jehová el Dios de los hebreos ha dicho así: ¿Hasta cuándo no querrás humillarte delante de mí? Deja ir a mi pueblo, para que me sirva.
\par 4 Y si aún rehúsas dejarlo ir, he aquí que mañana yo traeré sobre tu territorio la langosta,
\par 5 la cual cubrirá la faz de la tierra, de modo que no pueda verse la tierra; y ella comerá lo que escapó, lo que os quedó del granizo; comerá asimismo todo árbol que os fructifica en el campo.
\par 6 Y llenará tus casas, y las casas de todos tus siervos, y las casas de todos los egipcios, cual nunca vieron tus padres ni tus abuelos, desde que ellos fueron sobre la tierra hasta hoy. Y se volvió y salió de delante de Faraón.
\par 7 Entonces los siervos de Faraón le dijeron: ¿Hasta cuándo será este hombre un lazo para nosotros? Deja ir a estos hombres, para que sirvan a Jehová su Dios. ¿Acaso no sabes todavía que Egipto está ya destruido?
\par 8 Y Moisés y Aarón volvieron a ser llamados ante Faraón, el cual les dijo: Andad, servid a Jehová vuestro Dios. ¿Quiénes son los que han de ir?
\par 9 Moisés respondió: Hemos de ir con nuestros niños y con nuestros viejos, con nuestros hijos y con nuestras hijas; con nuestras ovejas y con nuestras vacas hemos de ir; porque es nuestra fiesta solemne para Jehová.
\par 10 Y él les dijo: !!Así sea Jehová con vosotros! ¿Cómo os voy a dejar ir a vosotros y a vuestros niños? !!Mirad cómo el mal está delante de vuestro rostro!
\par 11 No será así; id ahora vosotros los varones, y servid a Jehová, pues esto es lo que vosotros pedisteis. Y los echaron de la presencia de Faraón.
\par 12 Entonces Jehová dijo a Moisés: Extiende tu mano sobre la tierra de Egipto para traer la langosta, a fin de que suba sobre el país de Egipto, y consuma todo lo que el granizo dejó.
\par 13 Y extendió Moisés su vara sobre la tierra de Egipto, y Jehová trajo un viento oriental sobre el país todo aquel día y toda aquella noche; y al venir la mañana el viento oriental trajo la langosta.
\par 14 Y subió la langosta sobre toda la tierra de Egipto, y se asentó en todo el país de Egipto en tan gran cantidad como no la hubo antes ni la habrá después;
\par 15 y cubrió la faz de todo el país, y oscureció la tierra; y consumió toda la hierba de la tierra, y todo el fruto de los árboles que había dejado el granizo; no quedó cosa verde en árboles ni en hierba del campo, en toda la tierra de Egipto.
\par 16 Entonces Faraón se apresuró a llamar a Moisés y a Aarón, y dijo: He pecado contra Jehová vuestro Dios, y contra vosotros.
\par 17 Mas os ruego ahora que perdonéis mi pecado solamente esta vez, y que oréis a Jehová vuestro Dios que quite de mí al menos esta plaga mortal.
\par 18 Y salió Moisés de delante de Faraón, y oró a Jehová.
\par 19 Entonces Jehová trajo un fortísimo viento occidental, y quitó la langosta y la arrojó en el Mar Rojo; ni una langosta quedó en todo el país de Egipto.
\par 20 Pero Jehová endureció el corazón de Faraón, y éste no dejó ir a los hijos de Israel.

\section*{La plaga de tinieblas}

\par 21 Jehová dijo a Moisés: Extiende tu mano hacia el cielo, para que haya tinieblas sobre la tierra de Egipto, tanto que cualquiera las palpe.
\par 22 Y extendió Moisés su mano hacia el cielo, y hubo densas tinieblas sobre toda la tierra de Egipto, por tres días.
\par 23 Ninguno vio a su prójimo, ni nadie se levantó de su lugar en tres días; mas todos los hijos de Israel tenían luz en sus habitaciones.
\par 24 Entonces Faraón hizo llamar a Moisés, y dijo: Id, servid a Jehová; solamente queden vuestras ovejas y vuestras vacas; vayan también vuestros niños con vosotros.
\par 25 Y Moisés respondió: Tú también nos darás sacrificios y holocaustos que sacrifiquemos para Jehová nuestro Dios.
\par 26 Nuestros ganados irán también con nosotros; no quedará ni una pezuña; porque de ellos hemos de tomar para servir a Jehová nuestro Dios, y no sabemos con qué hemos de servir a Jehová hasta que lleguemos allá.
\par 27 Pero Jehová endureció el corazón de Faraón, y no quiso dejarlos ir.
\par 28 Y le dijo Faraón: Retírate de mí; guárdate que no veas más mi rostro, porque en cualquier día que vieres mi rostro, morirás.
\par 29 Y Moisés respondió: Bien has dicho; no veré más tu rostro.

\chapter{11}

\section*{Anunciada la muerte de los primogénitos}

\par 1 Jehová dijo a Moisés: Una plaga traeré aún sobre Faraón y sobre Egipto, después de la cual él os dejará ir de aquí; y seguramente os echará de aquí del todo.
\par 2 Habla ahora al pueblo, y que cada uno pida a su vecino, y cada una a su vecina, alhajas de plata y de oro.
\par 3 Y Jehová dio gracia al pueblo en los ojos de los egipcios. También Moisés era tenido por gran varón en la tierra de Egipto, a los ojos de los siervos de Faraón, y a los ojos del pueblo.
\par 4 Dijo, pues, Moisés: Jehová ha dicho así: A la medianoche yo saldré por en medio de Egipto,
\par 5 y morirá todo primogénito en tierra de Egipto, desde el primogénito de Faraón que se sienta en su trono, hasta el primogénito de la sierva que está tras el molino, y todo primogénito de las bestias.
\par 6 Y habrá gran clamor por toda la tierra de Egipto, cual nunca hubo, ni jamás habrá.
\par 7 Pero contra todos los hijos de Israel, desde el hombre hasta la bestia, ni un perro moverá su lengua, para que sepáis que Jehová hace diferencia entre los egipcios y los israelitas.
\par 8 Y descenderán a mí todos estos tus siervos, e inclinados delante de mí dirán: Vete, tú y todo el pueblo que está debajo de ti; y después de esto yo saldré. Y salió muy enojado de la presencia de Faraón.
\par 9 Y Jehová dijo a Moisés: Faraón no os oirá, para que mis maravillas se multipliquen en la tierra de Egipto.
\par 10 Y Moisés y Aarón hicieron todos estos prodigios delante de Faraón; pues Jehová había endurecido el corazón de Faraón, y no envió a los hijos de Israel fuera de su país.

\chapter{12}

\section*{La Pascua}

\par 1 Habló Jehová a Moisés y a Aarón en la tierra de Egipto, diciendo:
\par 2 Este mes os será principio de los meses; para vosotros será éste el primero en los meses del año.
\par 3 Hablad a toda la congregación de Israel, diciendo: En el diez de este mes tómese cada uno un cordero según las familias de los padres, un cordero por familia.
\par 4 Mas si la familia fuere tan pequeña que no baste para comer el cordero, entonces él y su vecino inmediato a su casa tomarán uno según el número de las personas; conforme al comer de cada hombre, haréis la cuenta sobre el cordero.
\par 5 El animal será sin defecto, macho de un año; lo tomaréis de las ovejas o de las cabras.
\par 6 Y lo guardaréis hasta el día catorce de este mes, y lo inmolará toda la congregación del pueblo de Israel entre las dos tardes.
\par 7 Y tomarán de la sangre, y la pondrán en los dos postes y en el dintel de las casas en que lo han de comer.
\par 8 Y aquella noche comerán la carne asada al fuego, y panes sin levadura; con hierbas amargas lo comerán.
\par 9 Ninguna cosa comeréis de él cruda, ni cocida en agua, sino asada al fuego; su cabeza con sus pies y sus entrañas.
\par 10 Ninguna cosa dejaréis de él hasta la mañana; y lo que quedare hasta la mañana, lo quemaréis en el fuego.
\par 11 Y lo comeréis así: ceñidos vuestros lomos, vuestro calzado en vuestros pies, y vuestro bordón en vuestra mano; y lo comeréis apresuradamente; es la Pascua de Jehová.
\par 12 Pues yo pasaré aquella noche por la tierra de Egipto, y heriré a todo primogénito en la tierra de Egipto, así de los hombres como de las bestias; y ejecutaré mis juicios en todos los dioses de Egipto. Yo Jehová.
\par 13 Y la sangre os será por señal en las casas donde vosotros estéis; y veré la sangre y pasaré de vosotros, y no habrá en vosotros plaga de mortandad cuando hiera la tierra de Egipto.
\par 14 Y este día os será en memoria, y lo celebraréis como fiesta solemne para Jehová durante vuestras generaciones; por estatuto perpetuo lo celebraréis.
\par 15 Siete días comeréis panes sin levadura; y así el primer día haréis que no haya levadura en vuestras casas; porque cualquiera que comiere leudado desde el primer día hasta el séptimo, será cortado de Israel.
\par 16 El primer día habrá santa convocación, y asimismo en el séptimo día tendréis una santa convocación; ninguna obra se hará en ellos, excepto solamente que preparéis lo que cada cual haya de comer.
\par 17 Y guardaréis la fiesta de los panes sin levadura, porque en este mismo día saqué vuestras huestes de la tierra de Egipto; por tanto, guardaréis este mandamiento en vuestras generaciones por costumbre perpetua.
\par 18 En el mes primero comeréis los panes sin levadura, desde el día catorce del mes por la tarde hasta el veintiuno del mes por la tarde.
\par 19 Por siete días no se hallará levadura en vuestras casas; porque cualquiera que comiere leudado, así extranjero como natural del país, será cortado de la congregación de Israel.
\par 20 Ninguna cosa leudada comeréis; en todas vuestras habitaciones comeréis panes sin levadura.
\par 21 Y Moisés convocó a todos los ancianos de Israel, y les dijo: Sacad y tomaos corderos por vuestras familias, y sacrificad la pascua.
\par 22 Y tomad un manojo de hisopo, y mojadlo en la sangre que estará en un lebrillo, y untad el dintel y los dos postes con la sangre que estará en el lebrillo; y ninguno de vosotros salga de las puertas de su casa hasta la mañana.
\par 23 Porque Jehová pasará hiriendo a los egipcios; y cuando vea la sangre en el dintel y en los dos postes, pasará Jehová aquella puerta, y no dejará entrar al heridor en vuestras casas para herir.
\par 24 Guardaréis esto por estatuto para vosotros y para vuestros hijos para siempre.
\par 25 Y cuando entréis en la tierra que Jehová os dará, como prometió, guardaréis este rito.
\par 26 Y cuando os dijeren vuestros hijos: ¿Qué es este rito vuestro?,
\par 27 vosotros responderéis: Es la víctima de la pascua de Jehová, el cual pasó por encima de las casas de los hijos de Israel en Egipto, cuando hirió a los egipcios, y libró nuestras casas. Entonces el pueblo se inclinó y adoró.
\par 28 Y los hijos de Israel fueron e hicieron puntualmente así, como Jehová había mandado a Moisés y a Aarón.

\section*{Muerte de los primogénitos}

\par 29 Y aconteció que a la medianoche Jehová hirió a todo primogénito en la tierra de Egipto, desde el primogénito de Faraón que se sentaba sobre su trono hasta el primogénito del cautivo que estaba en la cárcel, y todo primogénito de los animales.
\par 30 Y se levantó aquella noche Faraón, él y todos sus siervos, y todos los egipcios; y hubo un gran clamor en Egipto, porque no había casa donde no hubiese un muerto.
\par 31 E hizo llamar a Moisés y a Aarón de noche, y les dijo: Salid de en medio de mi pueblo vosotros y los hijos de Israel, e id, servid a Jehová, como habéis dicho.
\par 32 Tomad también vuestras ovejas y vuestras vacas, como habéis dicho, e idos; y bendecidme también a mí.
\par 33 Y los egipcios apremiaban al pueblo, dándose prisa a echarlos de la tierra; porque decían: Todos somos muertos.
\par 34 Y llevó el pueblo su masa antes que se leudase, sus masas envueltas en sus sábanas sobre sus hombros.
\par 35 E hicieron los hijos de Israel conforme al mandamiento de Moisés, pidiendo de los egipcios alhajas de plata, y de oro, y vestidos.
\par 36 Y Jehová dio gracia al pueblo delante de los egipcios, y les dieron cuanto pedían; así despojaron a los egipcios.

\section*{Los israelitas salen de Egipto}

\par 37 Partieron los hijos de Israel de Ramesés a Sucot, como seiscientos mil hombres de a pie, sin contar los niños.
\par 38 También subió con ellos grande multitud de toda clase de gentes, y ovejas, y muchísimo ganado.
\par 39 Y cocieron tortas sin levadura de la masa que habían sacado de Egipto, pues no había leudado, porque al echarlos fuera los egipcios, no habían tenido tiempo ni para prepararse comida.
\par 40 El tiempo que los hijos de Israel habitaron en Egipto fue cuatrocientos treinta años.
\par 41 Y pasados los cuatrocientos treinta años, en el mismo día todas las huestes de Jehová salieron de la tierra de Egipto.
\par 42 Es noche de guardar para Jehová, por haberlos sacado en ella de la tierra de Egipto. Esta noche deben guardarla para Jehová todos los hijos de Israel en sus generaciones.
\par 43 Y Jehová dijo a Moisés y a Aarón: Esta es la ordenanza de la pascua; ningún extraño comerá de ella.
\par 44 Mas todo siervo humano comprado por dinero comerá de ella, después que lo hubieres circuncidado.
\par 45 El extranjero y el jornalero no comerán de ella.
\par 46 Se comerá en una casa, y no llevarás de aquella carne fuera de ella, ni quebraréis hueso suyo.
\par 47 Toda la congregación de Israel lo hará.
\par 48 Mas si algún extranjero morare contigo, y quisiere celebrar la pascua para Jehová, séale circuncidado todo varón, y entonces la celebrará, y será como uno de vuestra nación; pero ningún incircunciso comerá de ella.
\par 49 La misma ley será para el natural, y para el extranjero que habitare entre vosotros.
\par 50 Así lo hicieron todos los hijos de Israel; como mandó Jehová a Moisés y a Aarón, así lo hicieron.
\par 51 Y en aquel mismo día sacó Jehová a los hijos de Israel de la tierra de Egipto por sus ejércitos.

\chapter{13}

\section*{Consagración de los primogénitos}

\par 1 Jehová habló a Moisés, diciendo:
\par 2 Conságrame todo primogénito. Cualquiera que abre matriz entre los hijos de Israel, así de los hombres como de los animales, mío es.
\par 3 Y Moisés dijo al pueblo: Tened memoria de este día, en el cual habéis salido de Egipto, de la casa de servidumbre, pues Jehová os ha sacado de aquí con mano fuerte; por tanto, no comeréis leudado.
\par 4 Vosotros salís hoy en el mes de Abib.
\par 5 Y cuando Jehová te hubiere metido en la tierra del cananeo, del heteo, del amorreo, del heveo y del jebuseo, la cual juró a tus padres que te daría, tierra que destila leche y miel, harás esta celebración en este mes.
\par 6 Siete días comerás pan sin leudar, y el séptimo día será fiesta para Jehová.
\par 7 Por los siete días se comerán los panes sin levadura, y no se verá contigo nada leudado, ni levadura, en todo tu territorio.
\par 8 Y lo contarás en aquel día a tu hijo, diciendo: Se hace esto con motivo de lo que Jehová hizo conmigo cuando me sacó de Egipto.
\par 9 Y te será como una señal sobre tu mano, y como un memorial delante de tus ojos, para que la ley de Jehová esté en tu boca; por cuanto con mano fuerte te sacó Jehová de Egipto.
\par 10 Por tanto, tú guardarás este rito en su tiempo de año en año.
\par 11 Y cuando Jehová te haya metido en la tierra del cananeo, como te ha jurado a ti y a tus padres, y cuando te la hubiere dado,
\par 12 dedicarás a Jehová todo aquel que abriere matriz, y asimismo todo primer nacido de tus animales; los machos serán de Jehová.
\par 13 Mas todo primogénito de asno redimirás con un cordero; y si no lo redimieres, quebrarás su cerviz. También redimirás al primogénito de tus hijos.
\par 14 Y cuando mañana te pregunte tu hijo, diciendo: ¿Qué es esto?, le dirás: Jehová nos sacó con mano fuerte de Egipto, de casa de servidumbre;
\par 15 y endureciéndose Faraón para no dejarnos ir, Jehová hizo morir en la tierra de Egipto a todo primogénito, desde el primogénito humano hasta el primogénito de la bestia; y por esta causa yo sacrifico para Jehová todo primogénito macho, y redimo al primogénito de mis hijos.
\par 16 Te será, pues, como una señal sobre tu mano, y por un memorial delante de tus ojos, por cuanto Jehová nos sacó de Egipto con mano fuerte.

\section*{La columna de nube y de fuego}

\par 17 Y luego que Faraón dejó ir al pueblo, Dios no los llevó por el camino de la tierra de los filisteos, que estaba cerca; porque dijo Dios: Para que no se arrepienta el pueblo cuando vea la guerra, y se vuelva a Egipto.
\par 18 Mas hizo Dios que el pueblo rodease por el camino del desierto del Mar Rojo. Y subieron los hijos de Israel de Egipto armados.
\par 19 Tomó también consigo Moisés los huesos de José, el cual había juramentado a los hijos de Israel, diciendo: Dios ciertamente os visitará, y haréis subir mis huesos de aquí con vosotros.
\par 20 Y partieron de Sucot y acamparon en Etam, a la entrada del desierto.
\par 21 Y Jehová iba delante de ellos de día en una columna de nube para guiarlos por el camino, y de noche en una columna de fuego para alumbrarles, a fin de que anduviesen de día y de noche.
\par 22 Nunca se apartó de delante del pueblo la columna de nube de día, ni de noche la columna de fuego.

\chapter{14}

\section*{Los israelitas cruzan el Mar Rojo}

\par 1 Habló Jehová a Moisés, diciendo:
\par 2 Di a los hijos de Israel que den la vuelta y acampen delante de Pi-hahirot, entre Migdol y el mar hacia Baal-zefón; delante de él acamparéis junto al mar.
\par 3 Porque Faraón dirá de los hijos de Israel: Encerrados están en la tierra, el desierto los ha encerrado.
\par 4 Y yo endureceré el corazón de Faraón para que los siga; y seré glorificado en Faraón y en todo su ejército, y sabrán los egipcios que yo soy Jehová. Y ellos lo hicieron así.
\par 5 Y fue dado aviso al rey de Egipto, que el pueblo huía; y el corazón de Faraón y de sus siervos se volvió contra el pueblo, y dijeron: ¿Cómo hemos hecho esto de haber dejado ir a Israel, para que no nos sirva?
\par 6 Y unció su carro, y tomó consigo su pueblo;
\par 7 y tomó seiscientos carros escogidos, y todos los carros de Egipto, y los capitanes sobre ellos.
\par 8 Y endureció Jehová el corazón de Faraón rey de Egipto, y él siguió a los hijos de Israel; pero los hijos de Israel habían salido con mano poderosa.
\par 9 Siguiéndolos, pues, los egipcios, con toda la caballería y carros de Faraón, su gente de a caballo, y todo su ejército, los alcanzaron acampados junto al mar, al lado de Pi-hahirot, delante de Baal-zefón.
\par 10 Y cuando Faraón se hubo acercado, los hijos de Israel alzaron sus ojos, y he aquí que los egipcios venían tras ellos; por lo que los hijos de Israel temieron en gran manera, y clamaron a Jehová.
\par 11 Y dijeron a Moisés: ¿No había sepulcros en Egipto, que nos has sacado para que muramos en el desierto? ¿Por qué has hecho así con nosotros, que nos has sacado de Egipto?
\par 12 ¿No es esto lo que te hablamos en Egipto, diciendo: Déjanos servir a los egipcios? Porque mejor nos fuera servir a los egipcios, que morir nosotros en el desierto.
\par 13 Y Moisés dijo al pueblo: No temáis; estad firmes, y ved la salvación que Jehová hará hoy con vosotros; porque los egipcios que hoy habéis visto, nunca más para siempre los veréis.
\par 14 Jehová peleará por vosotros, y vosotros estaréis tranquilos.
\par 15 Entonces Jehová dijo a Moisés: ¿Por qué clamas a mí? Di a los hijos de Israel que marchen.
\par 16 Y tú alza tu vara, y extiende tu mano sobre el mar, y divídelo, y entren los hijos de Israel por en medio del mar, en seco.
\par 17 Y he aquí, yo endureceré el corazón de los egipcios para que los sigan; y yo me glorificaré en Faraón y en todo su ejército, en sus carros y en su caballería;
\par 18 y sabrán los egipcios que yo soy Jehová, cuando me glorifique en Faraón, en sus carros y en su gente de a caballo.
\par 19 Y el ángel de Dios que iba delante del campamento de Israel, se apartó e iba en pos de ellos; y asimismo la columna de nube que iba delante de ellos se apartó y se puso a sus espaldas,
\par 20 e iba entre el campamento de los egipcios y el campamento de Israel; y era nube y tinieblas para aquéllos, y alumbraba a Israel de noche, y en toda aquella noche nunca se acercaron los unos a los otros.
\par 21 Y extendió Moisés su mano sobre el mar, e hizo Jehová que el mar se retirase por recio viento oriental toda aquella noche; y volvió el mar en seco, y las aguas quedaron divididas.
\par 22 Entonces los hijos de Israel entraron por en medio del mar, en seco, teniendo las aguas como muro a su derecha y a su izquierda.
\par 23 Y siguiéndolos los egipcios, entraron tras ellos hasta la mitad del mar, toda la caballería de Faraón, sus carros y su gente de a caballo.
\par 24 Aconteció a la vigilia de la mañana, que Jehová miró el campamento de los egipcios desde la columna de fuego y nube, y trastornó el campamento de los egipcios,
\par 25 y quitó las ruedas de sus carros, y los trastornó gravemente. Entonces los egipcios dijeron: Huyamos de delante de Israel, porque Jehová pelea por ellos contra los egipcios.
\par 26 Y Jehová dijo a Moisés: Extiende tu mano sobre el mar, para que las aguas vuelvan sobre los egipcios, sobre sus carros, y sobre su caballería.
\par 27 Entonces Moisés extendió su mano sobre el mar, y cuando amanecía, el mar se volvió en toda su fuerza, y los egipcios al huir se encontraban con el mar; y Jehová derribó a los egipcios en medio del mar.
\par 28 Y volvieron las aguas, y cubrieron los carros y la caballería, y todo el ejército de Faraón que había entrado tras ellos en el mar; no quedó de ellos ni uno.
\par 29 Y los hijos de Israel fueron por en medio del mar, en seco, teniendo las aguas por muro a su derecha y a su izquierda.
\par 30 Así salvó Jehová aquel día a Israel de mano de los egipcios; e Israel vio a los egipcios muertos a la orilla del mar.
\par 31 Y vio Israel aquel grande hecho que Jehová ejecutó contra los egipcios; y el pueblo temió a Jehová, y creyeron a Jehová y a Moisés su siervo.

\chapter{15}

\section*{Cántico de Moisés y de María}

\par 1 Entonces cantó Moisés y los hijos de Israel este cántico a Jehová, y dijeron:
\par Cantaré yo a Jehová, porque se ha magnificado grandemente;
\par Ha echado en el mar al caballo y al jinete.
\par 2 Jehová es mi fortaleza y mi cántico,
\par Y ha sido mi salvación.
\par Este es mi Dios, y lo alabaré;
\par Dios de mi padre, y lo enalteceré.
\par 3 Jehová es varón de guerra;
\par Jehová es su nombre.
\par 4 Echó en el mar los carros de Faraón y su ejército;
\par Y sus capitanes escogidos fueron hundidos en el Mar Rojo.
\par 5 Los abismos los cubrieron;
\par Descendieron a las profundidades como piedra.
\par 6 Tu diestra, oh Jehová, ha sido magnificada en poder;
\par Tu diestra, oh Jehová, ha quebrantado al enemigo.
\par 7 Y con la grandeza de tu poder has derribado a los que se levantaron contra ti.
\par Enviaste tu ira; los consumió como a hojarasca.
\par 8 Al soplo de tu aliento se amontonaron las aguas;
\par Se juntaron las corrientes como en un montón;
\par Los abismos se cuajaron en medio del mar.
\par 9 El enemigo dijo:
\par Perseguiré, apresaré, repartiré despojos;
\par Mi alma se saciará de ellos;
\par Sacaré mi espada, los destruirá mi mano.
\par 10 Soplaste con tu viento; los cubrió el mar;
\par Se hundieron como plomo en las impetuosas aguas.
\par 11 ¿Quién como tú, oh Jehová, entre los dioses?
\par ¿Quién como tú, magnífico en santidad,
\par Terrible en maravillosas hazañas, hacedor de prodigios?
\par 12 Extendiste tu diestra;
\par La tierra los tragó.
\par 13 Condujiste en tu misericordia a este pueblo que redimiste;
\par Lo llevaste con tu poder a tu santa morada.
\par 14 Lo oirán los pueblos, y temblarán;
\par Se apoderará dolor de la tierra de los filisteos.
\par 15 Entonces los caudillos de Edom se turbarán;
\par A los valientes de Moab les sobrecogerá temblor;
\par Se acobardarán todos los moradores de Canaán.
\par 16 Caiga sobre ellos temblor y espanto;
\par A la grandeza de tu brazo enmudezcan como una piedra;
\par Hasta que haya pasado tu pueblo, oh Jehová,
\par Hasta que haya pasado este pueblo que tú rescataste.
\par 17 Tú los introducirás y los plantarás en el monte de tu heredad,
\par En el lugar de tu morada, que tú has preparado, oh Jehová,
\par En el santuario que tus manos, oh Jehová, han afirmado.
\par 18 Jehová reinará eternamente y para siempre.
\par 19 Porque Faraón entró cabalgando con sus carros y su gente de a caballo en el mar, y Jehová hizo volver las aguas del mar sobre ellos; mas los hijos de Israel pasaron en seco por en medio del mar.
\par 20 Y María la profetisa, hermana de Aarón, tomó un pandero en su mano, y todas las mujeres salieron en pos de ella con panderos y danzas.
\par 21 Y María les respondía:
\par Cantad a Jehová, porque en extremo se ha engrandecido;
\par Ha echado en el mar al caballo y al jinete.

\section*{El agua amarga de Mara}

\par 22 E hizo Moisés que partiese Israel del Mar Rojo, y salieron al desierto de Shur; y anduvieron tres días por el desierto sin hallar agua.
\par 23 Y llegaron a Mara, y no pudieron beber las aguas de Mara, porque eran amargas; por eso le pusieron el nombre de Mara.
\par 24 Entonces el pueblo murmuró contra Moisés, y dijo: ¿Qué hemos de beber?
\par 25 Y Moisés clamó a Jehová, y Jehová le mostró un árbol; y lo echó en las aguas, y las aguas se endulzaron. Allí les dio estatutos y ordenanzas, y allí los probó;
\par 26 y dijo: Si oyeres atentamente la voz de Jehová tu Dios, e hicieres lo recto delante de sus ojos, y dieres oído a sus mandamientos, y guardares todos sus estatutos, ninguna enfermedad de las que envié a los egipcios te enviaré a ti; porque yo soy Jehová tu sanador.
\par 27 Y llegaron a Elim, donde había doce fuentes de aguas, y setenta palmeras; y acamparon allí junto a las aguas.

\chapter{16}

\section*{Dios da el maná}

\par 1 Partió luego de Elim toda la congregación de los hijos de Israel, y vino al desierto de Sin, que está entre Elim y Sinaí, a los quince días del segundo mes después que salieron de la tierra de Egipto.
\par 2 Y toda la congregación de los hijos de Israel murmuró contra Moisés y Aarón en el desierto;
\par 3 y les decían los hijos de Israel: Ojalá hubiéramos muerto por mano de Jehová en la tierra de Egipto, cuando nos sentábamos a las ollas de carne, cuando comíamos pan hasta saciarnos; pues nos habéis sacado a este desierto para matar de hambre a toda esta multitud.
\par 4 Y Jehová dijo a Moisés: He aquí yo os haré llover pan del cielo; y el pueblo saldrá, y recogerá diariamente la porción de un día, para que yo lo pruebe si anda en mi ley, o no.
\par 5 Mas en el sexto día prepararán para guardar el doble de lo que suelen recoger cada día.
\par 6 Entonces dijeron Moisés y Aarón a todos los hijos de Israel: En la tarde sabréis que Jehová os ha sacado de la tierra de Egipto,
\par 7 y a la mañana veréis la gloria de Jehová; porque él ha oído vuestras murmuraciones contra Jehová; porque nosotros, ¿qué somos, para que vosotros murmuréis contra nosotros?
\par 8 Dijo también Moisés: Jehová os dará en la tarde carne para comer, y en la mañana pan hasta saciaros; porque Jehová ha oído vuestras murmuraciones con que habéis murmurado contra él; porque nosotros, ¿qué somos? Vuestras murmuraciones no son contra nosotros, sino contra Jehová.
\par 9 Y dijo Moisés a Aarón: Di a toda la congregación de los hijos de Israel: Acercaos a la presencia de Jehová, porque él ha oído vuestras murmuraciones.
\par 10 Y hablando Aarón a toda la congregación de los hijos de Israel, miraron hacia el desierto, y he aquí la gloria de Jehová apareció en la nube.
\par 11 Y Jehová habló a Moisés, diciendo:
\par 12 Yo he oído las murmuraciones de los hijos de Israel; háblales, diciendo: Al caer la tarde comeréis carne, y por la mañana os saciaréis de pan, y sabréis que yo soy Jehová vuestro Dios.
\par 13 Y venida la tarde, subieron codornices que cubrieron el campamento; y por la mañana descendió rocío en derredor del campamento.
\par 14 Y cuando el rocío cesó de descender, he aquí sobre la faz del desierto una cosa menuda, redonda, menuda como una escarcha sobre la tierra.
\par 15 Y viéndolo los hijos de Israel, se dijeron unos a otros: ¿Qué es esto? porque no sabían qué era. Entonces Moisés les dijo: Es el pan que Jehová os da para comer.
\par 16 Esto es lo que Jehová ha mandado: Recoged de él cada uno según lo que pudiere comer; un gomer por cabeza, conforme al número de vuestras personas, tomaréis cada uno para los que están en su tienda.
\par 17 Y los hijos de Israel lo hicieron así; y recogieron unos más, otros menos;
\par 18 y lo medían por gomer, y no sobró al que había recogido mucho, ni faltó al que había recogido poco; cada uno recogió conforme a lo que había de comer.
\par 19 Y les dijo Moisés: Ninguno deje nada de ello para mañana.
\par 20 Mas ellos no obedecieron a Moisés, sino que algunos dejaron de ello para otro día, y crió gusanos, y hedió; y se enojó contra ellos Moisés.
\par 21 Y lo recogían cada mañana, cada uno según lo que había de comer; y luego que el sol calentaba, se derretía.
\par 22 En el sexto día recogieron doble porción de comida, dos gomeres para cada uno; y todos los príncipes de la congregación vinieron y se lo hicieron saber a Moisés.
\par 23 Y él les dijo: Esto es lo que ha dicho Jehová: Mañana es el santo día de reposo, el reposo consagrado a Jehová; lo que habéis de cocer, cocedlo hoy, y lo que habéis de cocinar, cocinadlo; y todo lo que os sobrare, guardadlo para mañana.
\par 24 Y ellos lo guardaron hasta la mañana, según lo que Moisés había mandado, y no se agusanó, ni hedió.
\par 25 Y dijo Moisés: Comedlo hoy, porque hoy es día de reposo para Jehová; hoy no hallaréis en el campo.
\par 26 Seis días lo recogeréis; mas el séptimo día es día de reposo; en él no se hallará.
\par 27 Y aconteció que algunos del pueblo salieron en el séptimo día a recoger, y no hallaron.
\par 28 Y Jehová dijo a Moisés: ¿Hasta cuándo no querréis guardar mis mandamientos y mis leyes?
\par 29 Mirad que Jehová os dio el día de reposo, y por eso en el sexto día os da pan para dos días. Estése, pues, cada uno en su lugar, y nadie salga de él en el séptimo día.
\par 30 Así el pueblo reposó el séptimo día.
\par 31 Y la casa de Israel lo llamó Maná; y era como semilla de culantro, blanco, y su sabor como de hojuelas con miel.
\par 32 Y dijo Moisés: Esto es lo que Jehová ha mandado: Llenad un gomer de él, y guardadlo para vuestros descendientes, a fin de que vean el pan que yo os di a comer en el desierto, cuando yo os saqué de la tierra de Egipto.
\par 33 Y dijo Moisés a Aarón: Toma una vasija y pon en ella un gomer de maná, y ponlo delante de Jehová, para que sea guardado para vuestros descendientes.
\par 34 Y Aarón lo puso delante del Testimonio para guardarlo, como Jehová lo mandó a Moisés.
\par 35 Así comieron los hijos de Israel maná cuarenta años, hasta que llegaron a tierra habitada; maná comieron hasta que llegaron a los límites de la tierra de Canaán.
\par 36 Y un gomer es la décima parte de un efa.

\chapter{17}

\section*{Agua de la roca}

\par 1 Toda la congregación de los hijos de Israel partió del desierto de Sin por sus jornadas, conforme al mandamiento de Jehová, y acamparon en Refidim; y no había agua para que el pueblo bebiese.
\par 2 Y altercó el pueblo con Moisés, y dijeron: Danos agua para que bebamos. Y Moisés les dijo: ¿Por qué altercáis conmigo? ¿Por qué tentáis a Jehová?
\par 3 Así que el pueblo tuvo allí sed, y murmuró contra Moisés, y dijo: ¿Por qué nos hiciste subir de Egipto para matarnos de sed a nosotros, a nuestros hijos y a nuestros ganados?
\par 4 Entonces clamó Moisés a Jehová, diciendo: ¿Qué haré con este pueblo? De aquí a un poco me apedrearán.
\par 5 Y Jehová dijo a Moisés: Pasa delante del pueblo, y toma contigo de los ancianos de Israel; y toma también en tu mano tu vara con que golpeaste el río, y ve.
\par 6 He aquí que yo estaré delante de ti allí sobre la peña en Horeb; y golpearás la peña, y saldrán de ella aguas, y beberá el pueblo. Y Moisés lo hizo así en presencia de los ancianos de Israel.
\par 7 Y llamó el nombre de aquel lugar Masah y Meriba, por la rencilla de los hijos de Israel, y porque tentaron a Jehová, diciendo: ¿Está, pues, Jehová entre nosotros, o no?

\section*{Guerra con Amalec}

\par 8 Entonces vino Amalec y peleó contra Israel en Refidim.
\par 9 Y dijo Moisés a Josué: Escógenos varones, y sal a pelear contra Amalec; mañana yo estaré sobre la cumbre del collado, y la vara de Dios en mi mano.
\par 10 E hizo Josué como le dijo Moisés, peleando contra Amalec; y Moisés y Aarón y Hur subieron a la cumbre del collado.
\par 11 Y sucedía que cuando alzaba Moisés su mano, Israel prevalecía; mas cuando él bajaba su mano, prevalecía Amalec.
\par 12 Y las manos de Moisés se cansaban; por lo que tomaron una piedra, y la pusieron debajo de él, y se sentó sobre ella; y Aarón y Hur sostenían sus manos, el uno de un lado y el otro de otro; así hubo en sus manos firmeza hasta que se puso el sol.
\par 13 Y Josué deshizo a Amalec y a su pueblo a filo de espada.
\par 14 Y Jehová dijo a Moisés: Escribe esto para memoria en un libro, y di a Josué que raeré del todo la memoria de Amalec de debajo del cielo.
\par 15 Y Moisés edificó un altar, y llamó su nombre Jehová-nisi;
\par 16 y dijo: Por cuanto la mano de Amalec se levantó contra el trono de Jehová, Jehová tendrá guerra con Amalec de generación en generación.

\chapter{18}

\section*{Jetro visita a Moisés}

\par 1 Oyó Jetro sacerdote de Madián, suegro de Moisés, todas las cosas que Dios había hecho con Moisés, y con Israel su pueblo, y cómo Jehová había sacado a Israel de Egipto.
\par 2 Y tomó Jetro suegro de Moisés a Séfora la mujer de Moisés, después que él la envió,
\par 3 y a sus dos hijos; el uno se llamaba Gersón, porque dijo: Forastero he sido en tierra ajena;
\par 4 y el otro se llamaba Eliezer, porque dijo: El Dios de mi padre me ayudó, y me libró de la espada de Faraón.
\par 5 Y Jetro el suegro de Moisés, con los hijos y la mujer de éste, vino a Moisés en el desierto, donde estaba acampado junto al monte de Dios;
\par 6 y dijo a Moisés: Yo tu suegro Jetro vengo a ti, con tu mujer, y sus dos hijos con ella.
\par 7 Y Moisés salió a recibir a su suegro, y se inclinó, y lo besó; y se preguntaron el uno al otro cómo estaban, y vinieron a la tienda.
\par 8 Y Moisés contó a su suegro todas las cosas que Jehová había hecho a Faraón y a los egipcios por amor de Israel, y todo el trabajo que habían pasado en el camino, y cómo los había librado Jehová.
\par 9 Y se alegró Jetro de todo el bien que Jehová había hecho a Israel, al haberlo librado de mano de los egipcios.
\par 10 Y Jetro dijo: Bendito sea Jehová, que os libró de mano de los egipcios, y de la mano de Faraón, y que libró al pueblo de la mano de los egipcios.
\par 11 Ahora conozco que Jehová es más grande que todos los dioses; porque en lo que se ensoberbecieron prevaleció contra ellos.
\par 12 Y tomó Jetro, suegro de Moisés, holocaustos y sacrificios para Dios; y vino Aarón y todos los ancianos de Israel para comer con el suegro de Moisés delante de Dios.

\section*{Nombramiento de jueces}

\par 13 Aconteció que al día siguiente se sentó Moisés a juzgar al pueblo; y el pueblo estuvo delante de Moisés desde la mañana hasta la tarde.
\par 14 Viendo el suegro de Moisés todo lo que él hacía con el pueblo, dijo: ¿Qué es esto que haces tú con el pueblo? ¿Por qué te sientas tú solo, y todo el pueblo está delante de ti desde la mañana hasta la tarde?
\par 15 Y Moisés respondió a su suegro: Porque el pueblo viene a mí para consultar a Dios.
\par 16 Cuando tienen asuntos, vienen a mí; y yo juzgo entre el uno y el otro, y declaro las ordenanzas de Dios y sus leyes.
\par 17 Entonces el suegro de Moisés le dijo: No está bien lo que haces.
\par 18 Desfallecerás del todo, tú, y también este pueblo que está contigo; porque el trabajo es demasiado pesado para ti; no podrás hacerlo tú solo.
\par 19 Oye ahora mi voz; yo te aconsejaré, y Dios estará contigo. Está tú por el pueblo delante de Dios, y somete tú los asuntos a Dios.
\par 20 Y enseña a ellos las ordenanzas y las leyes, y muéstrales el camino por donde deben andar, y lo que han de hacer.
\par 21 Además escoge tú de entre todo el pueblo varones de virtud, temerosos de Dios, varones de verdad, que aborrezcan la avaricia; y ponlos sobre el pueblo por jefes de millares, de centenas, de cincuenta y de diez.
\par 22 Ellos juzgarán al pueblo en todo tiempo; y todo asunto grave lo traerán a ti, y ellos juzgarán todo asunto pequeño. Así aliviarás la carga de sobre ti, y la llevarán ellos contigo.
\par 23 Si esto hicieres, y Dios te lo mandare, tú podrás sostenerte, y también todo este pueblo irá en paz a su lugar.
\par 24 Y oyó Moisés la voz de su suegro, e hizo todo lo que dijo.
\par 25 Escogió Moisés varones de virtud de entre todo Israel, y los puso por jefes sobre el pueblo, sobre mil, sobre ciento, sobre cincuenta, y sobre diez.
\par 26 Y juzgaban al pueblo en todo tiempo; el asunto difícil lo traían a Moisés, y ellos juzgaban todo asunto pequeño.
\par 27 Y despidió Moisés a su suegro, y éste se fue a su tierra.

\chapter{19}

\section*{Israel en Sinaí}

\par 1 En el mes tercero de la salida de los hijos de Israel de la tierra de Egipto, en el mismo día llegaron al desierto de Sinaí.
\par 2 Habían salido de Refidim, y llegaron al desierto de Sinaí, y acamparon en el desierto; y acampó allí Israel delante del monte.
\par 3 Y Moisés subió a Dios; y Jehová lo llamó desde el monte, diciendo: Así dirás a la casa de Jacob, y anunciarás a los hijos de Israel:
\par 4 Vosotros visteis lo que hice a los egipcios, y cómo os tomé sobre alas de águilas, y os he traído a mí.
\par 5 Ahora, pues, si diereis oído a mi voz, y guardareis mi pacto, vosotros seréis mi especial tesoro sobre todos los pueblos; porque mía es toda la tierra.
\par 6 Y vosotros me seréis un reino de sacerdotes, y gente santa. Estas son las palabras que dirás a los hijos de Israel.
\par 7 Entonces vino Moisés, y llamó a los ancianos del pueblo, y expuso en presencia de ellos todas estas palabras que Jehová le había mandado.
\par 8 Y todo el pueblo respondió a una, y dijeron: Todo lo que Jehová ha dicho, haremos. Y Moisés refirió a Jehová las palabras del pueblo.
\par 9 Entonces Jehová dijo a Moisés: He aquí, yo vengo a ti en una nube espesa, para que el pueblo oiga mientras yo hablo contigo, y también para que te crean para siempre.
\par Y Moisés refirió las palabras del pueblo a Jehová.
\par 10 Y Jehová dijo a Moisés: Ve al pueblo, y santifícalos hoy y mañana; y laven sus vestidos,
\par 11 y estén preparados para el día tercero, porque al tercer día Jehová descenderá a ojos de todo el pueblo sobre el monte de Sinaí.
\par 12 Y señalarás término al pueblo en derredor, diciendo: Guardaos, no subáis al monte, ni toquéis sus límites; cualquiera que tocare el monte, de seguro morirá.
\par 13 No lo tocará mano, porque será apedreado o asaeteado; sea animal o sea hombre, no vivirá. Cuando suene largamente la bocina, subirán al monte.
\par 14 Y descendió Moisés del monte al pueblo, y santificó al pueblo; y lavaron sus vestidos.
\par 15 Y dijo al pueblo: Estad preparados para el tercer día; no toquéis mujer.
\par 16 Aconteció que al tercer día, cuando vino la mañana, vinieron truenos y relámpagos, y espesa nube sobre el monte, y sonido de bocina muy fuerte; y se estremeció todo el pueblo que estaba en el campamento.
\par 17 Y Moisés sacó del campamento al pueblo para recibir a Dios; y se detuvieron al pie del monte.
\par 18 Todo el monte Sinaí humeaba, porque Jehová había descendido sobre él en fuego; y el humo subía como el humo de un horno, y todo el monte se estremecía en gran manera.
\par 19 El sonido de la bocina iba aumentando en extremo; Moisés hablaba, y Dios le respondía con voz tronante.
\par 20 Y descendió Jehová sobre el monte Sinaí, sobre la cumbre del monte; y llamó Jehová a Moisés a la cumbre del monte, y Moisés subió.
\par 21 Y Jehová dijo a Moisés: Desciende, ordena al pueblo que no traspase los límites para ver a Jehová, porque caerá multitud de ellos.
\par 22 Y también que se santifiquen los sacerdotes que se acercan a Jehová, para que Jehová no haga en ellos estrago.
\par 23 Moisés dijo a Jehová: El pueblo no podrá subir al monte Sinaí, porque tú nos has mandado diciendo: Señala límites al monte, y santifícalo.
\par 24 Y Jehová le dijo: Ve, desciende, y subirás tú, y Aarón contigo; mas los sacerdotes y el pueblo no traspasen el límite para subir a Jehová, no sea que haga en ellos estrago.
\par 25 Entonces Moisés descendió y se lo dijo al pueblo.

\chapter{20}

\section*{Los Diez Mandamientos}

\par 1 Y habló Dios todas estas palabras, diciendo:
\par 2 Yo soy Jehová tu Dios, que te saqué de la tierra de Egipto, de casa de servidumbre.
\par 3 No tendrás dioses ajenos delante de mí.
\par 4 No te harás imagen, ni ninguna semejanza de lo que esté arriba en el cielo, ni abajo en la tierra, ni en las aguas debajo de la tierra.
\par 5 No te inclinarás a ellas, ni las honrarás; porque yo soy Jehová tu Dios, fuerte, celoso, que visito la maldad de los padres sobre los hijos hasta la tercera y cuarta generación de los que me aborrecen,
\par 6 y hago misericordia a millares, a los que me aman y guardan mis mandamientos.
\par 7 No tomarás el nombre de Jehová tu Dios en vano; porque no dará por inocente Jehová al que tomare su nombre en vano.
\par 8 Acuérdate del día de reposo[a] para santificarlo.
\par 9 Seis días trabajarás, y harás toda tu obra;
\par 10 mas el séptimo día es reposo[b] para Jehová tu Dios; no hagas en él obra alguna, tú, ni tu hijo, ni tu hija, ni tu siervo, ni tu criada, ni tu bestia, ni tu extranjero que está dentro de tus puertas.
\par 11 Porque en seis días hizo Jehová los cielos y la tierra, el mar, y todas las cosas que en ellos hay, y reposó en el séptimo día; por tanto, Jehová bendijo el día de reposo[c] y lo santificó.
\par 12 Honra a tu padre y a tu madre, para que tus días se alarguen en la tierra que Jehová tu Dios te da.
\par 13 No matarás.
\par 14 No cometerás adulterio.
\par 15 No hurtarás.
\par 16 No hablarás contra tu prójimo falso testimonio.
\par 17 No codiciarás la casa de tu prójimo, no codiciarás la mujer de tu prójimo, ni su siervo, ni su criada, ni su buey, ni su asno, ni cosa alguna de tu prójimo.

\section*{El terror del pueblo}

\par 18 Todo el pueblo observaba el estruendo y los relámpagos, y el sonido de la bocina, y el monte que humeaba; y viéndolo el pueblo, temblaron, y se pusieron de lejos.
\par 19 Y dijeron a Moisés: Habla tú con nosotros, y nosotros oiremos; pero no hable Dios con nosotros, para que no muramos.
\par 20 Y Moisés respondió al pueblo: No temáis; porque para probaros vino Dios, y para que su temor esté delante de vosotros, para que no pequéis.
\par 21 Entonces el pueblo estuvo a lo lejos, y Moisés se acercó a la oscuridad en la cual estaba Dios.
\par 22 Y Jehová dijo a Moisés: Así dirás a los hijos de Israel: Vosotros habéis visto que he hablado desde el cielo con vosotros.
\par 23 No hagáis conmigo dioses de plata, ni dioses de oro os haréis.
\par 24 Altar de tierra harás para mí, y sacrificarás sobre él tus holocaustos y tus ofrendas de paz, tus ovejas y tus vacas; en todo lugar donde yo hiciere que esté la memoria de mi nombre, vendré a ti y te bendeciré.
\par 25 Y si me hicieres altar de piedras, no las labres de cantería; porque si alzares herramienta sobre él, lo profanarás.
\par 26 No subirás por gradas a mi altar, para que tu desnudez no se descubra junto a él.

\chapter{21}

\section*{Leyes sobre los esclavos}

\par 1 Estas son las leyes que les propondrás.
\par 2 Si comprares siervo hebreo, seis años servirá; mas al séptimo saldrá libre, de balde.
\par 3 Si entró solo, solo saldrá; si tenía mujer, saldrá él y su mujer con él.
\par 4 Si su amo le hubiere dado mujer, y ella le diere hijos o hijas, la mujer y sus hijos serán de su amo, y él saldrá solo.
\par 5 Y si el siervo dijere: Yo amo a mi señor, a mi mujer y a mis hijos, no saldré libre;
\par 6 entonces su amo lo llevará ante los jueces, y le hará estar junto a la puerta o al poste; y su amo le horadará la oreja con lesna, y será su siervo para siempre.
\par 7 Y cuando alguno vendiere su hija por sierva, no saldrá ella como suelen salir los siervos.
\par 8 Si no agradare a su señor, por lo cual no la tomó por esposa, se le permitirá que se rescate, y no la podrá vender a pueblo extraño cuando la desechare.
\par 9 Mas si la hubiere desposado con su hijo, hará con ella según la costumbre de las hijas.
\par 10 Si tomare para él otra mujer, no disminuirá su alimento, ni su vestido, ni el deber conyugal.
\par 11 Y si ninguna de estas tres cosas hiciere, ella saldrá de gracia, sin dinero.

\section*{Leyes sobre actos de violencia}

\par 12 El que hiriere a alguno, haciéndole así morir, él morirá.
\par 13 Mas el que no pretendía herirlo, sino que Dios lo puso en sus manos, entonces yo te señalaré lugar al cual ha de huir.
\par 14 Pero si alguno se ensoberbeciere contra su prójimo y lo matare con alevosía, de mi altar lo quitarás para que muera.
\par 15 El que hiriere a su padre o a su madre, morirá.
\par 16 Asimismo el que robare una persona y la vendiere, o si fuere hallada en sus manos, morirá.
\par 17 Igualmente el que maldijere a su padre o a su madre, morirá.
\par 18 Además, si algunos riñeren, y uno hiriere a su prójimo con piedra o con el puño, y éste no muriere, pero cayere en cama;
\par 19 si se levantare y anduviere fuera sobre su báculo, entonces será absuelto el que lo hirió; solamente le satisfará por lo que estuvo sin trabajar, y hará que le curen.
\par 20 Y si alguno hiriere a su siervo o a su sierva con palo, y muriere bajo su mano, será castigado;
\par 21 mas si sobreviviere por un día o dos, no será castigado, porque es de su propiedad.
\par 22 Si algunos riñeren, e hirieren a mujer embarazada, y ésta abortare, pero sin haber muerte, serán penados conforme a lo que les impusiere el marido de la mujer y juzgaren los jueces.
\par 23 Mas si hubiere muerte, entonces pagarás vida por vida,
\par 24 ojo por ojo, diente por diente, mano por mano, pie por pie,
\par 25 quemadura por quemadura, herida por herida, golpe por golpe.

\section*{Leyes sobre responsabilidades de amos y dueños}

\par 26 Si alguno hiriere el ojo de su siervo, o el ojo de su sierva, y lo dañare, le dará libertad por razón de su ojo.
\par 27 Y si hiciere saltar un diente de su siervo, o un diente de su sierva, por su diente le dejará ir libre.
\par 28 Si un buey acorneare a hombre o a mujer, y a causa de ello muriere, el buey será apedreado, y no será comida su carne; mas el dueño del buey será absuelto.
\par 29 Pero si el buey fuere acorneador desde tiempo atrás, y a su dueño se le hubiere notificado, y no lo hubiere guardado, y matare a hombre o mujer, el buey será apedreado, y también morirá su dueño.
\par 30 Si le fuere impuesto precio de rescate, entonces dará por el rescate de su persona cuanto le fuere impuesto.
\par 31 Haya acorneado a hijo, o haya acorneado a hija, conforme a este juicio se hará con él.
\par 32 Si el buey acorneare a un siervo o a una sierva, pagará su dueño treinta siclos de plata,  y el buey será apedreado.
\par 33 Y si alguno abriere un pozo, o cavare cisterna, y no la cubriere, y cayere allí buey o asno,
\par 34 el dueño de la cisterna pagará el daño, resarciendo a su dueño, y lo que fue muerto será suyo.
\par 35 Y si el buey de alguno hiriere al buey de su prójimo de modo que muriere, entonces venderán el buey vivo y partirán el dinero de él, y también partirán el buey muerto.
\par 36 Mas si era notorio que el buey era acorneador desde tiempo atrás, y su dueño no lo hubiere guardado, pagará buey por buey, y el buey muerto será suyo.

\chapter{22}

\section*{Leyes sobre la restitución}

\par 1 Cuando alguno hurtare buey u oveja, y lo degollare o vendiere, por aquel buey pagará cinco bueyes, y por aquella oveja cuatro ovejas.
\par 2 Si el ladrón fuere hallado forzando una casa, y fuere herido y muriere, el que lo hirió no será culpado de su muerte.
\par 3 Pero si fuere de día, el autor de la muerte será reo de homicidio. El ladrón hará completa restitución; si no tuviere con qué, será vendido por su hurto.
\par 4 Si fuere hallado con el hurto en la mano, vivo, sea buey o asno u oveja, pagará el doble.
\par 5 Si alguno hiciere pastar en campo o viña, y metiere su bestia en campo de otro, de lo mejor de su campo y de lo mejor de su viña pagará.
\par 6 Cuando se prendiere fuego, y al quemar espinos quemare mieses amontonadas o en pie, o campo, el que encendió el fuego pagará lo quemado.
\par 7 Cuando alguno diere a su prójimo plata o alhajas a guardar, y fuere hurtado de la casa de aquel hombre, si el ladrón fuere hallado, pagará el doble.
\par 8 Si el ladrón no fuere hallado, entonces el dueño de la casa será presentado a los jueces, para que se vea si ha metido su mano en los bienes de su prójimo.
\par 9 En toda clase de fraude, sobre buey, sobre asno, sobre oveja, sobre vestido, sobre toda cosa perdida, cuando alguno dijere: Esto es mío, la causa de ambos vendrá delante de los jueces; y el que los jueces condenaren, pagará el doble a su prójimo.
\par 10 Si alguno hubiere dado a su prójimo asno, o buey, u oveja, o cualquier otro animal a guardar, y éste muriere o fuere estropeado, o fuere llevado sin verlo nadie;
\par 11 juramento de Jehová habrá entre ambos, de que no metió su mano a los bienes de su prójimo; y su dueño lo aceptará, y el otro no pagará.
\par 12 Mas si le hubiere sido hurtado, resarcirá a su dueño.
\par 13 Y si le hubiere sido arrebatado por fiera, le traerá testimonio, y no pagará lo arrebatado.
\par 14 Pero si alguno hubiere tomado prestada bestia de su prójimo, y fuere estropeada o muerta, estando ausente su dueño, deberá pagarla.
\par 15 Si el dueño estaba presente no la pagará. Si era alquilada, reciba el dueño el alquiler.

\section*{Leyes humanitarias}

\par 16 Si alguno engañare a una doncella que no fuere desposada, y durmiere con ella, deberá dotarla y tomarla por mujer.
\par 17 Si su padre no quisiere dársela, él le pesará plata conforme a la dote de las vírgenes.
\par 18 A la hechicera no dejarás que viva.
\par 19 Cualquiera que cohabitare con bestia, morirá.
\par 20 El que ofreciere sacrificio a dioses excepto solamente a Jehová, será muerto.
\par 21 Y al extranjero no engañarás ni angustiarás, porque extranjeros fuisteis vosotros en la tierra de Egipto.
\par 22 A ninguna viuda ni huérfano afligiréis.
\par 23 Porque si tú llegas a afligirles, y ellos clamaren a mí, ciertamente oiré yo su clamor;
\par 24 y mi furor se encenderá, y os mataré a espada, y vuestras mujeres serán viudas, y huérfanos vuestros hijos.
\par 25 Cuando prestares dinero a uno de mi pueblo, al pobre que está contigo, no te portarás con él como logrero, ni le impondrás usura.
\par 26 Si tomares en prenda el vestido de tu prójimo, a la puesta del sol se lo devolverás.
\par 27 Porque sólo eso es su cubierta, es su vestido para cubrir su cuerpo. ¿En qué dormirá? Y cuando él clamare a mí, yo le oiré, porque soy misericordioso.
\par 28 No injuriarás a los jueces, ni maldecirás al príncipe de tu pueblo.
\par 29 No demorarás la primicia de tu cosecha ni de tu lagar. Me darás el primogénito de tus hijos.
\par 30 Lo mismo harás con el de tu buey y de tu oveja; siete días estará con su madre, y al octavo día me lo darás.
\par 31 Y me seréis varones santos. No comeréis carne destrozada por las fieras en el campo; a los perros la echaréis.

\chapter{23}

\par 1 No admitirás falso rumor. No te concertarás con el impío para ser testigo falso.
\par 2 No seguirás a los muchos para hacer mal, ni responderás en litigio inclinándote a los más para hacer agravios;
\par 3 ni al pobre distinguirás en su causa.
\par 4 Si encontrares el buey de tu enemigo o su asno extraviado, vuelve a llevárselo.
\par 5 Si vieres el asno del que te aborrece caído debajo de su carga, ¿le dejarás sin ayuda? Antes bien le ayudarás a levantarlo.
\par 6 No pervertirás el derecho de tu mendigo en su pleito.
\par 7 De palabra de mentira te alejarás, y no matarás al inocente y justo; porque yo no justificaré al impío.
\par 8 No recibirás presente; porque el presente ciega a los que ven, y pervierte las palabras de los justos.
\par 9 Y no angustiarás al extranjero; porque vosotros sabéis cómo es el alma del extranjero, ya que extranjeros fuisteis en la tierra de Egipto.
\par 10 Seis años sembrarás tu tierra, y recogerás su cosecha;
\par 11 mas el séptimo año la dejarás libre, para que coman los pobres de tu pueblo; y de lo que quedare comerán las bestias del campo; así harás con tu viña y con tu olivar.
\par 12 Seis días trabajarás, y al séptimo día reposarás,  para que descanse tu buey y tu asno, y tome refrigerio el hijo de tu sierva, y el extranjero.
\par 13 Y todo lo que os he dicho, guardadlo. Y nombre de otros dioses no mentaréis, ni se oirá de vuestra boca.

\section*{Las tres fiestas anuales}

\par 14 Tres veces en el año me celebraréis fiesta.
\par 15 La fiesta de los panes sin levadura guardarás. Siete días comerás los panes sin levadura, como yo te mandé, en el tiempo del mes de Abib, porque en él saliste de Egipto; y ninguno se presentará delante de mí con las manos vacías.
\par 16 También la fiesta de la siega, los primeros frutos de tus labores, que hubieres sembrado en el campo, y la fiesta de la cosecha a la salida del año, cuando hayas recogido los frutos de tus labores del campo.
\par 17 Tres veces en el año se presentará todo varón delante de Jehová el Señor.
\par 18 No ofrecerás con pan leudo la sangre de mi sacrificio, ni la grosura de mi víctima quedará de la noche hasta la mañana.
\par 19 Las primicias de los primeros frutos de tu tierra traerás a la casa de Jehová tu Dios. No guisarás el cabrito en la leche de su madre.

\section*{El Angel de Jehová enviado para guiar a Israel}

\par 20 He aquí yo envío mi Angel delante de ti para que te guarde en el camino, y te introduzca en el lugar que yo he preparado.
\par 21 Guárdate delante de él, y oye su voz; no le seas rebelde; porque él no perdonará vuestra rebelión, porque mi nombre está en él.
\par 22 Pero si en verdad oyeres su voz e hicieres todo lo que yo te dijere, seré enemigo de tus enemigos, y afligiré a los que te afligieren.
\par 23 Porque mi Angel irá delante de ti, y te llevará a la tierra del amorreo, del heteo, del ferezeo, del cananeo, del heveo y del jebuseo, a los cuales yo haré destruir.
\par 24 No te inclinarás a sus dioses, ni los servirás, ni harás como ellos hacen; antes los destruirás del todo, y quebrarás totalmente sus estatuas.
\par 25 Mas a Jehová vuestro Dios serviréis, y él bendecirá tu pan y tus aguas; y yo quitaré toda enfermedad de en medio de ti.
\par 26 No habrá mujer que aborte, ni estéril en tu tierra; y yo completaré el número de tus días.
\par 27 Yo enviaré mi terror delante de ti, y consternaré a todo pueblo donde entres, y te daré la cerviz de todos tus enemigos.
\par 28 Enviaré delante de ti la avispa, que eche fuera al heveo, al cananeo y al heteo, de delante de ti.
\par 29 No los echaré de delante de ti en un año, para que no quede la tierra desierta, y se aumenten contra ti las fieras del campo.
\par 30 Poco a poco los echaré de delante de ti, hasta que te multipliques y tomes posesión de la tierra.
\par 31 Y fijaré tus límites desde el Mar Rojo hasta el mar de los filisteos, y desde el desierto hasta el Eufrates; porque pondré en tus manos a los moradores de la tierra, y tú los echarás de delante de ti.
\par 32 No harás alianza con ellos, ni con sus dioses.
\par 33 En tu tierra no habitarán, no sea que te hagan pecar contra mí sirviendo a sus dioses, porque te será tropiezo.

\chapter{24}

\section*{Moisés y los ancianos en el Monte Sinaí}

\par 1 Dijo Jehová a Moisés: Sube ante Jehová, tú, y Aarón, Nadab, y Abiú, y setenta de los ancianos de Israel; y os inclinaréis desde lejos.
\par 2 Pero Moisés solo se acercará a Jehová; y ellos no se acerquen, ni suba el pueblo con él.
\par 3 Y Moisés vino y contó al pueblo todas las palabras de Jehová, y todas las leyes; y todo el pueblo respondió a una voz, y dijo: Haremos todas las palabras que Jehová ha dicho.
\par 4 Y Moisés escribió todas las palabras de Jehová, y levantándose de mañana edificó un altar al pie del monte, y doce columnas, según las doce tribus de Israel.
\par 5 Y envió jóvenes de los hijos de Israel, los cuales ofrecieron holocaustos y becerros como sacrificios de paz a Jehová.
\par 6 Y Moisés tomó la mitad de la sangre, y la puso en tazones, y esparció la otra mitad de la sangre sobre el altar.
\par 7 Y tomó el libro del pacto y lo leyó a oídos del pueblo, el cual dijo: Haremos todas las cosas que Jehová ha dicho, y obedeceremos.
\par 8 Entonces Moisés tomó la sangre y roció sobre el pueblo, y dijo: He aquí la sangre del pacto que Jehová ha hecho con vosotros sobre todas estas cosas.
\par 9 Y subieron Moisés y Aarón, Nadab y Abiú, y setenta de los ancianos de Israel;
\par 10 y vieron al Dios de Israel; y había debajo de sus pies como un embaldosado de zafiro, semejante al cielo cuando está sereno.
\par 11 Mas no extendió su mano sobre los príncipes de los hijos de Israel; y vieron a Dios, y comieron y bebieron.
\par 12 Entonces Jehová dijo a Moisés: Sube a mí al monte, y espera allá, y te daré tablas de piedra, y la ley, y mandamientos que he escrito para enseñarles.
\par 13 Y se levantó Moisés con Josué su servidor, y Moisés subió al monte de Dios.
\par 14 Y dijo a los ancianos: Esperadnos aquí hasta que volvamos a vosotros; y he aquí Aarón y Hur están con vosotros; el que tuviere asuntos, acuda a ellos.
\par 15 Entonces Moisés subió al monte, y una nube cubrió el monte.
\par 16 Y la gloria de Jehová reposó sobre el monte Sinaí, y la nube lo cubrió por seis días; y al séptimo día llamó a Moisés de en medio de la nube.
\par 17 Y la apariencia de la gloria de Jehová era como un fuego abrasador en la cumbre del monte, a los ojos de los hijos de Israel.
\par 18 Y entró Moisés en medio de la nube, y subió al monte; y estuvo Moisés en el monte cuarenta días y cuarenta noches.

\chapter{25}

\section*{La ofrenda para el tabernáculo}

\par 1 Jehová habló a Moisés, diciendo:
\par 2 Di a los hijos de Israel que tomen para mí ofrenda; de todo varón que la diere de su voluntad, de corazón, tomaréis mi ofrenda.
\par 3 Esta es la ofrenda que tomaréis de ellos: oro, plata, cobre,
\par 4 azul, púrpura, carmesí, lino fino, pelo de cabras,
\par 5 pieles de carneros teñidas de rojo, pieles de tejones, madera de acacia,
\par 6 aceite para el alumbrado, especias para el aceite de la unción y para el incienso aromático,
\par 7 piedras de ónice, y piedras de engaste para el efod y para el pectoral.
\par 8 Y harán un santuario para mí, y habitaré en medio de ellos.
\par 9 Conforme a todo lo que yo te muestre, el diseño del tabernáculo, y el diseño de todos sus utensilios, así lo haréis.

\section*{El arca del testimonio}

\par 10 Harán también un arca de madera de acacia, cuya longitud será de dos codos   y medio, su anchura de codo y medio, y su altura de codo y medio.
\par 11 Y la cubrirás de oro puro por dentro y por fuera, y harás sobre ella una cornisa de oro alrededor.
\par 12 Fundirás para ella cuatro anillos de oro, que pondrás en sus cuatro esquinas; dos anillos a un lado de ella, y dos anillos al otro lado.
\par 13 Harás unas varas de madera de acacia, las cuales cubrirás de oro.
\par 14 Y meterás las varas por los anillos a los lados del arca, para llevar el arca con ellas.
\par 15 Las varas quedarán en los anillos del arca; no se quitarán de ella.
\par 16 Y pondrás en el arca el testimonio que yo te daré.
\par 17 Y harás un propiciatorio de oro fino, cuya longitud será de dos codos   y medio, y su anchura de codo y medio.
\par 18 Harás también dos querubines de oro; labrados a martillo los harás en los dos extremos del propiciatorio.
\par 19 Harás, pues, un querubín en un extremo, y un querubín en el otro extremo; de una pieza con el propiciatorio harás los querubines en sus dos extremos.
\par 20 Y los querubines extenderán por encima las alas, cubriendo con sus alas el propiciatorio; sus rostros el uno enfrente del otro, mirando al propiciatorio los rostros de los querubines.
\par 21 Y pondrás el propiciatorio encima del arca, y en el arca pondrás el testimonio que yo te daré.
\par 22 Y de allí me declararé a ti, y hablaré contigo de sobre el propiciatorio, de entre los dos querubines que están sobre el arca del testimonio, todo lo que yo te mandare para los hijos de Israel.

\section*{La mesa para el pan de la proposición}

\par 23 Harás asimismo una mesa de madera de acacia; su longitud será de dos codos,   y de un codo su anchura, y su altura de codo y medio.
\par 24 Y la cubrirás de oro puro, y le harás una cornisa de oro alrededor.
\par 25 Le harás también una moldura alrededor, de un palmo menor   de anchura, y harás a la moldura una cornisa de oro alrededor.
\par 26 Y le harás cuatro anillos de oro, los cuales pondrás en las cuatro esquinas que corresponden a sus cuatro patas.
\par 27 Los anillos estarán debajo de la moldura, para lugares de las varas para llevar la mesa.
\par 28 Harás las varas de madera de acacia, y las cubrirás de oro, y con ellas será llevada la mesa.
\par 29 Harás también sus platos, sus cucharas, sus cubiertas y sus tazones, con que se libará; de oro fino los harás.
\par 30 Y pondrás sobre la mesa el pan de la proposición delante de mí continuamente.

\section*{El candelero de oro}

\par 31 Harás además un candelero de oro puro; labrado a martillo se hará el candelero; su pie, su caña, sus copas, sus manzanas y sus flores, serán de lo mismo.
\par 32 Y saldrán seis brazos de sus lados; tres brazos del candelero a un lado, y tres brazos al otro lado.
\par 33 Tres copas en forma de flor de almendro en un brazo, una manzana y una flor; y tres copas en forma de flor de almendro en otro brazo, una manzana y una flor; así en los seis brazos que salen del candelero;
\par 34 y en la caña central del candelero cuatro copas en forma de flor de almendro, sus manzanas y sus flores.
\par 35 Habrá una manzana debajo de dos brazos del mismo, otra manzana debajo de otros dos brazos del mismo, y otra manzana debajo de los otros dos brazos del mismo, así para los seis brazos que salen del candelero.
\par 36 Sus manzanas y sus brazos serán de una pieza, todo ello una pieza labrada a martillo, de oro puro.
\par 37 Y le harás siete lamparillas, las cuales encenderás para que alumbren hacia adelante.
\par 38 También sus despabiladeras y sus platillos, de oro puro.
\par 39 De un talento de oro   fino lo harás, con todos estos utensilios.
\par 40 Mira y hazlos conforme al modelo que te ha sido mostrado en el monte.

\chapter{26}

\section*{El tabernáculo}

\par 1 Harás el tabernáculo de diez cortinas de lino torcido, azul, púrpura y carmesí; y lo harás con querubines de obra primorosa.
\par 2 La longitud de una cortina de veintiocho codos,  y la anchura de la misma cortina de cuatro codos; todas las cortinas tendrán una misma medida.
\par 3 Cinco cortinas estarán unidas una con la otra, y las otras cinco cortinas unidas una con la otra.
\par 4 Y harás lazadas de azul en la orilla de la última cortina de la primera unión; lo mismo harás en la orilla de la cortina de la segunda unión.
\par 5 Cincuenta lazadas harás en la primera cortina, y cincuenta lazadas harás en la orilla de la cortina que está en la segunda unión; las lazadas estarán contrapuestas la una a la otra.
\par 6 Harás también cincuenta corchetes de oro, con los cuales enlazarás las cortinas la una con la otra, y se formará un tabernáculo.
\par 7 Harás asimismo cortinas de pelo de cabra para una cubierta sobre el tabernáculo; once cortinas harás.
\par 8 La longitud de cada cortina será de treinta codos,  y la anchura de cada cortina de cuatro codos; una misma medida tendrán las once cortinas.
\par 9 Y unirás cinco cortinas aparte y las otras seis cortinas aparte; y doblarás la sexta cortina en el frente del tabernáculo.
\par 10 Y harás cincuenta lazadas en la orilla de la cortina, al borde en la unión, y cincuenta lazadas en la orilla de la cortina de la segunda unión.
\par 11 Harás asimismo cincuenta corchetes de bronce, los cuales meterás por las lazadas; y enlazarás las uniones para que se haga una sola cubierta.
\par 12 Y la parte que sobra en las cortinas de la tienda, la mitad de la cortina que sobra, colgará a espaldas del tabernáculo.
\par 13 Y un codo   de un lado, y otro codo del otro lado, que sobra a lo largo de las cortinas de la tienda, colgará sobre los lados del tabernáculo a un lado y al otro, para cubrirlo.
\par 14 Harás también a la tienda una cubierta de pieles de carneros teñidas de rojo, y una cubierta de pieles de tejones encima.
\par 15 Y harás para el tabernáculo tablas de madera de acacia, que estén derechas.
\par 16 La longitud de cada tabla será de diez codos,  y de codo y medio la anchura.
\par 17 Dos espigas tendrá cada tabla, para unirlas una con otra; así harás todas las tablas del tabernáculo.
\par 18 Harás, pues, las tablas del tabernáculo; veinte tablas al lado del mediodía, al sur.
\par 19 Y harás cuarenta basas de plata debajo de las veinte tablas; dos basas debajo de una tabla para sus dos espigas, y dos basas debajo de otra tabla para sus dos espigas.
\par 20 Y al otro lado del tabernáculo, al lado del norte, veinte tablas;
\par 21 y sus cuarenta basas de plata; dos basas debajo de una tabla, y dos basas debajo de otra tabla.
\par 22 Y para el lado posterior del tabernáculo, al occidente, harás seis tablas.
\par 23 Harás además dos tablas para las esquinas del tabernáculo en los dos ángulos posteriores;
\par 24 las cuales se unirán desde abajo, y asimismo se juntarán por su alto con un gozne; así será con las otras dos; serán para las dos esquinas.
\par 25 De suerte que serán ocho tablas, con sus basas de plata, dieciséis basas; dos basas debajo de una tabla, y dos basas debajo de otra tabla.
\par 26 Harás también cinco barras de madera de acacia, para las tablas de un lado del tabernáculo,
\par 27 y cinco barras para las tablas del otro lado del tabernáculo, y cinco barras para las tablas del lado posterior del tabernáculo, al occidente.
\par 28 Y la barra de en medio pasará por en medio de las tablas, de un extremo al otro.
\par 29 Y cubrirás de oro las tablas, y harás sus anillos de oro para meter por ellos las barras; también cubrirás de oro las barras.
\par 30 Y alzarás el tabernáculo conforme al modelo que te fue mostrado en el monte.
\par 31 También harás un velo de azul, púrpura, carmesí y lino torcido; será hecho de obra primorosa, con querubines;
\par 32 y lo pondrás sobre cuatro columnas de madera de acacia cubiertas de oro; sus capiteles de oro, sobre basas de plata.
\par 33 Y pondrás el velo debajo de los corchetes, y meterás allí, del velo adentro, el arca del testimonio; y aquel velo os hará separación entre el lugar santo y el santísimo.
\par 34 Pondrás el propiciatorio sobre el arca del testimonio en el lugar santísimo.
\par 35 Y pondrás la mesa fuera del velo, y el candelero enfrente de la mesa al lado sur del tabernáculo; y pondrás la mesa al lado del norte.
\par 36 Harás para la puerta del tabernáculo una cortina de azul, púrpura, carmesí y lino torcido, obra de recamador.
\par 37 Y harás para la cortina cinco columnas de madera de acacia, las cuales cubrirás de oro, con sus capiteles de oro; y fundirás cinco basas de bronce para ellas.

\chapter{27}

\section*{El altar de bronce}

\par 1 Harás también un altar de madera de acacia de cinco codos   de longitud, y de cinco codos de anchura; será cuadrado el altar, y su altura de tres codos.
\par 2 Y le harás cuernos en sus cuatro esquinas; los cuernos serán parte del mismo; y lo cubrirás de bronce.
\par 3 Harás también sus calderos para recoger la ceniza, y sus paletas, sus tazones, sus garfios y sus braseros; harás todos sus utensilios de bronce.
\par 4 Y le harás un enrejado de bronce de obra de rejilla, y sobre la rejilla harás cuatro anillos de bronce a sus cuatro esquinas.
\par 5 Y la pondrás dentro del cerco del altar abajo; y llegará la rejilla hasta la mitad del altar.
\par 6 Harás también varas para el altar, varas de madera de acacia, las cuales cubrirás de bronce.
\par 7 Y las varas se meterán por los anillos, y estarán aquellas varas a ambos lados del altar cuando sea llevado.
\par 8 Lo harás hueco, de tablas; de la manera que te fue mostrado en el monte, así lo harás.

\section*{El atrio del tabernáculo}

\par 9 Asimismo harás el atrio del tabernáculo. Al lado meridional, al sur, tendrá el atrio cortinas de lino torcido, de cien codos   de longitud para un lado.
\par 10 Sus veinte columnas y sus veinte basas serán de bronce; los capiteles de las columnas y sus molduras, de plata.
\par 11 De la misma manera al lado del norte habrá a lo largo cortinas de cien codos   de longitud, y sus veinte columnas con sus veinte basas de bronce; los capiteles de sus columnas y sus molduras, de plata.
\par 12 El ancho del atrio, del lado occidental, tendrá cortinas de cincuenta codos; sus columnas diez, con sus diez basas.
\par 13 Y en el ancho del atrio por el lado del oriente, al este, habrá cincuenta codos.
\par 14 Las cortinas a un lado de la entrada serán de quince codos;  sus columnas tres, con sus tres basas.
\par 15 Y al otro lado, quince codos   de cortinas; sus columnas tres, con sus tres basas.
\par 16 Y para la puerta del atrio habrá una cortina de veinte codos, de azul, púrpura y carmesí, y lino torcido, de obra de recamador; sus columnas cuatro, con sus cuatro basas.
\par 17 Todas las columnas alrededor del atrio estarán ceñidas de plata; sus capiteles de plata, y sus basas de bronce.
\par 18 La longitud del atrio será de cien codos,  y la anchura cincuenta por un lado y cincuenta por el otro, y la altura de cinco codos; sus cortinas de lino torcido, y sus basas de bronce.
\par 19 Todos los utensilios del tabernáculo en todo su servicio, y todas sus estacas, y todas las estacas del atrio, serán de bronce.

\section*{Aceite para las lámparas}

\par 20 Y mandarás a los hijos de Israel que te traigan aceite puro de olivas machacadas, para el alumbrado, para hacer arder continuamente las lámparas.
\par 21 En el tabernáculo de reunión, afuera del velo que está delante del testimonio, las pondrá en orden Aarón y sus hijos para que ardan delante de Jehová desde la tarde hasta la mañana, como estatuto perpetuo de los hijos de Israel por sus generaciones.

\chapter{28}

\section*{Las vestiduras de los sacerdotes}

\par 1 Harás llegar delante de ti a Aarón tu hermano, y a sus hijos consigo, de entre los hijos de Israel, para que sean mis sacerdotes; a Aarón y a Nadab, Abiú, Eleazar e Itamar hijos de Aarón.
\par 2 Y harás vestiduras sagradas a Aarón tu hermano, para honra y hermosura.
\par 3 Y tú hablarás a todos los sabios de corazón, a quienes yo he llenado de espíritu de sabiduría, para que hagan las vestiduras de Aarón, para consagrarle para que sea mi sacerdote.
\par 4 Las vestiduras que harán son estas: el pectoral, el efod, el manto, la túnica bordada, la mitra y el cinturón. Hagan, pues, las vestiduras sagradas para Aarón tu hermano, y para sus hijos, para que sean mis sacerdotes.
\par 5 Tomarán oro, azul, púrpura, carmesí y lino torcido,
\par 6 y harán el efod de oro, azul, púrpura, carmesí y lino torcido, de obra primorosa.
\par 7 Tendrá dos hombreras que se junten a sus dos extremos, y así se juntará.
\par 8 Y su cinto de obra primorosa que estará sobre él, será de la misma obra, parte del mismo; de oro, azul, púrpura, carmesí y lino torcido.
\par 9 Y tomarás dos piedras de ónice, y grabarás en ellas los nombres de los hijos de Israel;
\par 10 seis de sus nombres en una piedra, y los otros seis nombres en la otra piedra, conforme al orden de nacimiento de ellos.
\par 11 De obra de grabador en piedra, como grabaduras de sello, harás grabar las dos piedras con los nombres de los hijos de Israel; les harás alrededor engastes de oro.
\par 12 Y pondrás las dos piedras sobre las hombreras del efod, para piedras memoriales a los hijos de Israel; y Aarón llevará los nombres de ellos delante de Jehová sobre sus dos hombros por memorial.
\par 13 Harás, pues, los engastes de oro,
\par 14 y dos cordones de oro fino, los cuales harás en forma de trenza; y fijarás los cordones de forma de trenza en los engastes.
\par 15 Harás asimismo el pectoral del juicio de obra primorosa, lo harás conforme a la obra del efod, de oro, azul, púrpura, carmesí y lino torcido.
\par 16 Será cuadrado y doble, de un palmo   de largo y un palmo de ancho;
\par 17 y lo llenarás de pedrería en cuatro hileras de piedras; una hilera de una piedra sárdica, un topacio y un carbunclo;
\par 18 la segunda hilera, una esmeralda, un zafiro y un diamante;
\par 19 la tercera hilera, un jacinto, una ágata y una amatista;
\par 20 la cuarta hilera, un berilo, un ónice y un jaspe. Todas estarán montadas en engastes de oro.
\par 21 Y las piedras serán según los nombres de los hijos de Israel, doce según sus nombres; como grabaduras de sello cada una con su nombre, serán según las doce tribus.
\par 22 Harás también en el pectoral cordones de hechura de trenzas de oro fino.
\par 23 Y harás en el pectoral dos anillos de oro, los cuales pondrás a los dos extremos del pectoral.
\par 24 Y fijarás los dos cordones de oro en los dos anillos a los dos extremos del pectoral;
\par 25 y pondrás los dos extremos de los dos cordones sobre los dos engastes, y los fijarás a las hombreras del efod en su parte delantera.
\par 26 Harás también dos anillos de oro, los cuales pondrás a los dos extremos del pectoral, en su orilla que está al lado del efod hacia adentro.
\par 27 Harás asimismo los dos anillos de oro, los cuales fijarás en la parte delantera de las dos hombreras del efod, hacia abajo, delante de su juntura sobre el cinto del efod.
\par 28 Y juntarán el pectoral por sus anillos a los dos anillos del efod con un cordón de azul, para que esté sobre el cinto del efod, y no se separe el pectoral del efod.
\par 29 Y llevará Aarón los nombres de los hijos de Israel en el pectoral del juicio sobre su corazón, cuando entre en el santuario, por memorial delante de Jehová continuamente.
\par 30 Y pondrás en el pectoral del juicio Urim y Tumim,  para que estén sobre el corazón de Aarón cuando entre delante de Jehová; y llevará siempre Aarón el juicio de los hijos de Israel sobre su corazón delante de Jehová.
\par 31 Harás el manto del efod todo de azul;
\par 32 y en medio de él por arriba habrá una abertura, la cual tendrá un borde alrededor de obra tejida, como el cuello de un coselete, para que no se rompa.
\par 33 Y en sus orlas harás granadas de azul, púrpura y carmesí alrededor, y entre ellas campanillas de oro alrededor.
\par 34 Una campanilla de oro y una granada, otra campanilla de oro y otra granada, en toda la orla del manto alrededor.
\par 35 Y estará sobre Aarón cuando ministre; y se oirá su sonido cuando él entre en el santuario delante de Jehová y cuando salga, para que no muera.
\par 36 Harás además una lámina de oro fino, y grabarás en ella como grabadura de sello, SANTIDAD A JEHOVÁ.
\par 37 Y la pondrás con un cordón de azul, y estará sobre la mitra; por la parte delantera de la mitra estará.
\par 38 Y estará sobre la frente de Aarón, y llevará Aarón las faltas cometidas en todas las cosas santas, que los hijos de Israel hubieren consagrado en todas sus santas ofrendas; y sobre su frente estará continuamente, para que obtengan gracia delante de Jehová.
\par 39 Y bordarás una túnica de lino, y harás una mitra de lino; harás también un cinto de obra de recamador.
\par 40 Y para los hijos de Aarón harás túnicas; también les harás cintos, y les harás tiaras para honra y hermosura.
\par 41 Y con ellos vestirás a Aarón tu hermano, y a sus hijos con él; y los ungirás, y los consagrarás y santificarás, para que sean mis sacerdotes.
\par 42 Y les harás calzoncillos de lino para cubrir su desnudez; serán desde los lomos hasta los muslos.
\par 43 Y estarán sobre Aarón y sobre sus hijos cuando entren en el tabernáculo de reunión, o cuando se acerquen al altar para servir en el santuario, para que no lleven pecado y mueran. Es estatuto perpetuo para él, y para su descendencia después de él.

\chapter{29}

\section*{Consagración de Aarón y de sus hijos}

\par 1 Esto es lo que les harás para consagrarlos, para que sean mis sacerdotes: Toma un becerro de la vacada, y dos carneros sin defecto;
\par 2 y panes sin levadura, y tortas sin levadura amasadas con aceite, y hojaldres sin levadura untadas con aceite; las harás de flor de harina de trigo.
\par 3 Y las pondrás en un canastillo, y en el canastillo las ofrecerás, con el becerro y los dos carneros.
\par 4 Y llevarás a Aarón y a sus hijos a la puerta del tabernáculo de reunión, y los lavarás con agua.
\par 5 Y tomarás las vestiduras, y vestirás a Aarón la túnica, el manto del efod, el efod y el pectoral, y le ceñirás con el cinto del efod;
\par 6 y pondrás la mitra sobre su cabeza, y sobre la mitra pondrás la diadema santa.
\par 7 Luego tomarás el aceite de la unción, y lo derramarás sobre su cabeza, y le ungirás.
\par 8 Y harás que se acerquen sus hijos, y les vestirás las túnicas.
\par 9 Les ceñirás el cinto a Aarón y a sus hijos, y les atarás las tiaras, y tendrán el sacerdocio por derecho perpetuo. Así consagrarás a Aarón y a sus hijos.
\par 10 Después llevarás el becerro delante del tabernáculo de reunión, y Aarón y sus hijos pondrán sus manos sobre la cabeza del becerro.
\par 11 Y matarás el becerro delante de Jehová, a la puerta del tabernáculo de reunión.
\par 12 Y de la sangre del becerro tomarás y pondrás sobre los cuernos del altar con tu dedo, y derramarás toda la demás sangre al pie del altar.
\par 13 Tomarás también toda la grosura que cubre los intestinos, la grosura de sobre el hígado, los dos riñones, y la grosura que está sobre ellos, y lo quemarás sobre el altar.
\par 14 Pero la carne del becerro, y su piel y su estiércol, los quemarás a fuego fuera del campamento; es ofrenda por el pecado.
\par 15 Asimismo tomarás uno de los carneros, y Aarón y sus hijos pondrán sus manos sobre la cabeza del carnero.
\par 16 Y matarás el carnero, y con su sangre rociarás sobre el altar alrededor.
\par 17 Cortarás el carnero en pedazos, y lavarás sus intestinos y sus piernas, y las pondrás sobre sus trozos y sobre su cabeza.
\par 18 Y quemarás todo el carnero sobre el altar; es holocausto de olor grato  para Jehová, es ofrenda quemada a Jehová.
\par 19 Tomarás luego el otro carnero, y Aarón y sus hijos pondrán sus manos sobre la cabeza del carnero.
\par 20 Y matarás el carnero, y tomarás de su sangre y la pondrás sobre el lóbulo de la oreja derecha de Aarón, sobre el lóbulo de la oreja de sus hijos, sobre el dedo pulgar de las manos derechas de ellos, y sobre el dedo pulgar de los pies derechos de ellos, y rociarás la sangre sobre el altar alrededor.
\par 21 Y con la sangre que estará sobre el altar, y el aceite de la unción, rociarás sobre Aarón, sobre sus vestiduras, sobre sus hijos, y sobre las vestiduras de éstos; y él será santificado, y sus vestiduras, y sus hijos, y las vestiduras de sus hijos con él.
\par 22 Luego tomarás del carnero la grosura, y la cola, y la grosura que cubre los intestinos, y la grosura del hígado, y los dos riñones, y la grosura que está sobre ellos, y la espaldilla derecha; porque es carnero de consagración.
\par 23 También una torta grande de pan, y una torta de pan de aceite, y una hojaldre del canastillo de los panes sin levadura presentado a Jehová,
\par 24 y lo pondrás todo en las manos de Aarón, y en las manos de sus hijos; y lo mecerás como ofrenda mecida delante de Jehová.
\par 25 Después lo tomarás de sus manos y lo harás arder en el altar, sobre el holocausto, por olor grato delante de Jehová. Es ofrenda encendida a Jehová.
\par 26 Y tomarás el pecho del carnero de las consagraciones, que es de Aarón, y lo mecerás por ofrenda mecida delante de Jehová; y será porción tuya.
\par 27 Y apartarás el pecho de la ofrenda mecida, y la espaldilla de la ofrenda elevada, lo que fue mecido y lo que fue elevado del carnero de las consagraciones de Aarón y de sus hijos,
\par 28 y será para Aarón y para sus hijos como estatuto perpetuo para los hijos de Israel, porque es ofrenda elevada; y será una ofrenda elevada de los hijos de Israel, de sus sacrificios de paz, porción de ellos elevada en ofrenda a Jehová.
\par 29 Y las vestiduras santas, que son de Aarón, serán de sus hijos después de él, para ser ungidos en ellas, y para ser en ellas consagrados.
\par 30 Por siete días las vestirá el que de sus hijos tome su lugar como sacerdote, cuando venga al tabernáculo de reunión para servir en el santuario.
\par 31 Y tomarás el carnero de las consagraciones, y cocerás su carne en lugar santo.
\par 32 Y Aarón y sus hijos comerán la carne del carnero, y el pan que estará en el canastillo, a la puerta del tabernáculo de reunión.
\par 33 Y comerán aquellas cosas con las cuales se hizo expiación, para llenar sus manos para consagrarlos; mas el extraño no las comerá, porque son santas.
\par 34 Y si sobrare hasta la mañana algo de la carne de las consagraciones y del pan, quemarás al fuego lo que hubiere sobrado; no se comerá, porque es cosa santa.
\par 35 Así, pues, harás a Aarón y a sus hijos, conforme a todo lo que yo te he mandado; por siete días los consagrarás.
\par 36 Cada día ofrecerás el becerro del sacrificio por el pecado, para las expiaciones; y purificarás el altar cuando hagas expiación por él, y lo ungirás para santificarlo.
\par 37 Por siete días harás expiación por el altar, y lo santificarás, y será un altar santísimo: cualquiera cosa que tocare el altar, será santificada.

\section*{Las ofrendas diarias}

\par 38 Esto es lo que ofrecerás sobre el altar: dos corderos de un año cada día, continuamente.
\par 39 Ofrecerás uno de los corderos por la mañana, y el otro cordero ofrecerás a la caída de la tarde.
\par 40 Además, con cada cordero una décima parte de un efa   de flor de harina amasada con la cuarta parte de un hin de aceite de olivas machacadas; y para la libación, la cuarta parte de un hin de vino.
\par 41 Y ofrecerás el otro cordero a la caída de la tarde, haciendo conforme a la ofrenda de la mañana, y conforme a su libación, en olor grato; ofrenda encendida a Jehová.
\par 42 Esto será el holocausto continuo por vuestras generaciones, a la puerta del tabernáculo de reunión, delante de Jehová, en el cual me reuniré con vosotros, para hablaros allí.
\par 43 Allí me reuniré con los hijos de Israel; y el lugar será santificado con mi gloria.
\par 44 Y santificaré el tabernáculo de reunión y el altar; santificaré asimismo a Aarón y a sus hijos, para que sean mis sacerdotes.
\par 45 Y habitaré entre los hijos de Israel, y seré su Dios.
\par 46 Y conocerán que yo soy Jehová su Dios, que los saqué de la tierra de Egipto, para habitar en medio de ellos. Yo Jehová su Dios.

\chapter{30}

\section*{El altar del incienso}

\par 1 Harás asimismo un altar para quemar el incienso; de madera de acacia lo harás.
\par 2 Su longitud será de un codo,  y su anchura de un codo; será cuadrado, y su altura de dos codos; y sus cuernos serán parte del mismo.
\par 3 Y lo cubrirás de oro puro, su cubierta, sus paredes en derredor y sus cuernos; y le harás en derredor una cornisa de oro.
\par 4 Le harás también dos anillos de oro debajo de su cornisa, a sus dos esquinas a ambos lados suyos, para meter las varas con que será llevado.
\par 5 Harás las varas de madera de acacia, y las cubrirás de oro.
\par 6 Y lo pondrás delante del velo que está junto al arca del testimonio, delante del propiciatorio que está sobre el testimonio, donde me encontraré contigo.
\par 7 Y Aarón quemará incienso aromático sobre él; cada mañana cuando aliste las lámparas lo quemará.
\par 8 Y cuando Aarón encienda las lámparas al anochecer, quemará el incienso; rito perpetuo delante de Jehová por vuestras generaciones.
\par 9 No ofreceréis sobre él incienso extraño, ni holocausto, ni ofrenda; ni tampoco derramaréis sobre él libación.
\par 10 Y sobre sus cuernos hará Aarón expiación una vez en el año con la sangre del sacrificio por el pecado para expiación; una vez en el año hará expiación sobre él por vuestras generaciones; será muy santo a Jehová.

\section*{El dinero del rescate}

\par 11 Habló también Jehová a Moisés, diciendo:
\par 12 Cuando tomes el número de los hijos de Israel conforme a la cuenta de ellos, cada uno dará a Jehová el rescate de su persona, cuando los cuentes, para que no haya en ellos mortandad cuando los hayas contado.
\par 13 Esto dará todo aquel que sea contado; medio siclo,  conforme al siclo del santuario. El siclo es de veinte geras. La mitad de un siclo será la ofrenda a Jehová.
\par 14 Todo el que sea contado, de veinte años arriba, dará la ofrenda a Jehová.
\par 15 Ni el rico aumentará, ni el pobre disminuirá del medio siclo,  cuando dieren la ofrenda a Jehová para hacer expiación por vuestras personas.
\par 16 Y tomarás de los hijos de Israel el dinero de las expiaciones, y lo darás para el servicio del tabernáculo de reunión; y será por memorial a los hijos de Israel delante de Jehová, para hacer expiación por vuestras personas.
\par 17 Habló más Jehová a Moisés, diciendo:
\par 18 Harás también una fuente de bronce, con su base de bronce, para lavar; y la colocarás entre el tabernáculo de reunión y el altar, y pondrás en ella agua.
\par 19 Y de ella se lavarán Aarón y sus hijos las manos y los pies.
\par 20 Cuando entren en el tabernáculo de reunión, se lavarán con agua, para que no mueran; y cuando se acerquen al altar para ministrar, para quemar la ofrenda encendida para Jehová,
\par 21 se lavarán las manos y los pies, para que no mueran. Y lo tendrán por estatuto perpetuo él y su descendencia por sus generaciones.

\section*{El aceite de la unción, y el incienso}

\par 22 Habló más Jehová a Moisés, diciendo:
\par 23 Tomarás especias finas: de mirra excelente quinientos siclos,  y de canela aromática la mitad, esto es, doscientos cincuenta, de cálamo aromático doscientos cincuenta,
\par 24 de casia quinientos, según el siclo del santuario, y de aceite de olivas un hin.
\par 25 Y harás de ello el aceite de la santa unción; superior ungüento, según el arte del perfumador, será el aceite de la unción santa.
\par 26 Con él ungirás el tabernáculo de reunión, el arca del testimonio,
\par 27 la mesa con todos sus utensilios, el candelero con todos sus utensilios, el altar del incienso,
\par 28 el altar del holocausto con todos sus utensilios, y la fuente y su base.
\par 29 Así los consagrarás, y serán cosas santísimas; todo lo que tocare en ellos, será santificado.
\par 30 Ungirás también a Aarón y a sus hijos, y los consagrarás para que sean mis sacerdotes.
\par 31 Y hablarás a los hijos de Israel, diciendo: Este será mi aceite de la santa unción por vuestras generaciones.
\par 32 Sobre carne de hombre no será derramado, ni haréis otro semejante, conforme a su composición; santo es, y por santo lo tendréis vosotros.
\par 33 Cualquiera que compusiere ungüento semejante, y que pusiere de él sobre extraño, será cortado de entre su pueblo.
\par 34 Dijo además Jehová a Moisés: Toma especias aromáticas, estacte y uña aromática y gálbano aromático e incienso puro; de todo en igual peso,
\par 35 y harás de ello el incienso, un perfume según el arte del perfumador, bien mezclado, puro y santo.
\par 36 Y molerás parte de él en polvo fino, y lo pondrás delante del testimonio en el tabernáculo de reunión, donde yo me mostraré a ti. Os será cosa santísima.
\par 37 Como este incienso que harás, no os haréis otro según su composición; te será cosa sagrada para Jehová.
\par 38 Cualquiera que hiciere otro como este para olerlo, será cortado de entre su pueblo.

\chapter{31}

\section*{Llamamiento de Bezaleel y de Aholiab}

\par 1 Habló Jehová a Moisés, diciendo:
\par 2 Mira, yo he llamado por nombre a Bezaleel hijo de Uri, hijo de Hur, de la tribu de Judá;
\par 3 y lo he llenado del Espíritu de Dios, en sabiduría y en inteligencia, en ciencia y en todo arte,
\par 4 para inventar diseños, para trabajar en oro, en plata y en bronce,
\par 5 y en artificio de piedras para engastarlas, y en artificio de madera; para trabajar en toda clase de labor.
\par 6 Y he aquí que yo he puesto con él a Aholiab hijo de Ahisamac, de la tribu de Dan; y he puesto sabiduría en el ánimo de todo sabio de corazón, para que hagan todo lo que te he mandado;
\par 7 el tabernáculo de reunión, el arca del testimonio, el propiciatorio que está sobre ella, y todos los utensilios del tabernáculo,
\par 8 la mesa y sus utensilios, el candelero limpio y todos sus utensilios, el altar del incienso,
\par 9 el altar del holocausto y todos sus utensilios, la fuente y su base,
\par 10 los vestidos del servicio, las vestiduras santas para Aarón el sacerdote, las vestiduras de sus hijos para que ejerzan el sacerdocio,
\par 11 el aceite de la unción, y el incienso aromático para el santuario; harán conforme a todo lo que te he mandado.

\section*{El día de reposo como señal}

\par 12 Habló además Jehová a Moisés, diciendo:
\par 13 Tú hablarás a los hijos de Israel, diciendo: En verdad vosotros guardaréis mis días de reposo; porque es señal entre mí y vosotros por vuestras generaciones, para que sepáis que yo soy Jehová que os santifico.
\par 14 Así que guardaréis el día de reposo, porque santo es a vosotros; el que lo profanare, de cierto morirá; porque cualquiera que hiciere obra alguna en él, aquella persona será cortada de en medio de su pueblo.
\par 15 Seis días se trabajará, mas el día séptimo es día de reposo consagrado a Jehová; cualquiera que trabaje en el día de reposo, ciertamente morirá.
\par 16 Guardarán, pues, el día de reposo los hijos de Israel, celebrándolo por sus generaciones por pacto perpetuo.
\par 17 Señal es para siempre entre mí y los hijos de Israel; porque en seis días hizo Jehová los cielos y la tierra, y en el séptimo día cesó y reposó.

\section*{El becerro de oro}

\par 18 Y dio a Moisés, cuando acabó de hablar con él en el monte de Sinaí, dos tablas del testimonio, tablas de piedra escritas con el dedo de Dios.

\chapter{32}

\par 1 Viendo el pueblo que Moisés tardaba en descender del monte, se acercaron entonces a Aarón, y le dijeron: Levántate, haznos dioses que vayan delante de nosotros; porque a este Moisés, el varón que nos sacó de la tierra de Egipto, no sabemos qué le haya acontecido.
\par 2 Y Aarón les dijo: Apartad los zarcillos de oro que están en las orejas de vuestras mujeres, de vuestros hijos y de vuestras hijas, y traédmelos.
\par 3 Entonces todo el pueblo apartó los zarcillos de oro que tenían en sus orejas, y los trajeron a Aarón;
\par 4 y él los tomó de las manos de ellos, y le dio forma con buril, e hizo de ello un becerro de fundición. Entonces dijeron: Israel, estos son tus dioses, que te sacaron de la tierra de Egipto.
\par 5 Y viendo esto Aarón, edificó un altar delante del becerro; y pregonó Aarón, y dijo: Mañana será fiesta para Jehová.
\par 6 Y al día siguiente madrugaron, y ofrecieron holocaustos, y presentaron ofrendas de paz; y se sentó el pueblo a comer y a beber, y se levantó a regocijarse.
\par 7 Entonces Jehová dijo a Moisés: Anda, desciende, porque tu pueblo que sacaste de la tierra de Egipto se ha corrompido.
\par 8 Pronto se han apartado del camino que yo les mandé; se han hecho un becerro de fundición, y lo han adorado, y le han ofrecido sacrificios, y han dicho: Israel, estos son tus dioses, que te sacaron de la tierra de Egipto.
\par 9 Dijo más Jehová a Moisés: Yo he visto a este pueblo, que por cierto es pueblo de dura cerviz.
\par 10 Ahora, pues, déjame que se encienda mi ira en ellos, y los consuma; y de ti yo haré una nación grande.
\par 11 Entonces Moisés oró en presencia de Jehová su Dios, y dijo: Oh Jehová, ¿por qué se encenderá tu furor contra tu pueblo, que tú sacaste de la tierra de Egipto con gran poder y con mano fuerte?
\par 12 ¿Por qué han de hablar los egipcios, diciendo: Para mal los sacó, para matarlos en los montes, y para raerlos de sobre la faz de la tierra? Vuélvete del ardor de tu ira, y arrepiéntete de este mal contra tu pueblo.
\par 13 Acuérdate de Abraham, de Isaac y de Israel tus siervos, a los cuales has jurado por ti mismo, y les has dicho: Yo multiplicaré vuestra descendencia como las estrellas del cielo; y daré a vuestra descendencia toda esta tierra de que he hablado, y la tomarán por heredad para siempre.
\par 14 Entonces Jehová se arrepintió del mal que dijo que había de hacer a su pueblo.
\par 15 Y volvió Moisés y descendió del monte, trayendo en su mano las dos tablas del testimonio, las tablas escritas por ambos lados; de uno y otro lado estaban escritas.
\par 16 Y las tablas eran obra de Dios, y la escritura era escritura de Dios grabada sobre las tablas.
\par 17 Cuando oyó Josué el clamor del pueblo que gritaba, dijo a Moisés: Alarido de pelea hay en el campamento.
\par 18 Y él respondió: No es voz de alaridos de fuertes, ni voz de alaridos de débiles; voz de cantar oigo yo.
\par 19 Y aconteció que cuando él llegó al campamento, y vio el becerro y las danzas, ardió la ira de Moisés, y arrojó las tablas de sus manos, y las quebró al pie del monte.
\par 20 Y tomó el becerro que habían hecho, y lo quemó en el fuego, y lo molió hasta reducirlo a polvo, que esparció sobre las aguas, y lo dio a beber a los hijos de Israel.
\par 21 Y dijo Moisés a Aarón: ¿Qué te ha hecho este pueblo, que has traído sobre él tan gran pecado?
\par 22 Y respondió Aarón: No se enoje mi señor; tú conoces al pueblo, que es inclinado a mal.
\par 23 Porque me dijeron: Haznos dioses que vayan delante de nosotros; porque a este Moisés, el varón que nos sacó de la tierra de Egipto, no sabemos qué le haya acontecido.
\par 24 Y yo les respondí: ¿Quién tiene oro? Apartadlo. Y me lo dieron, y lo eché en el fuego, y salió este becerro.
\par 25 Y viendo Moisés que el pueblo estaba desenfrenado, porque Aarón lo había permitido, para vergüenza entre sus enemigos,
\par 26 se puso Moisés a la puerta del campamento, y dijo: ¿Quién está por Jehová? Júntese conmigo. Y se juntaron con él todos los hijos de Leví.
\par 27 Y él les dijo: Así ha dicho Jehová, el Dios de Israel: Poned cada uno su espada sobre su muslo; pasad y volved de puerta a puerta por el campamento, y matad cada uno a su hermano, y a su amigo, y a su pariente.
\par 28 Y los hijos de Leví lo hicieron conforme al dicho de Moisés; y cayeron del pueblo en aquel día como tres mil hombres.
\par 29 Entonces Moisés dijo: Hoy os habéis consagrado a Jehová, pues cada uno se ha consagrado en su hijo y en su hermano, para que él dé bendición hoy sobre vosotros.
\par 30 Y aconteció que al día siguiente dijo Moisés al pueblo: Vosotros habéis cometido un gran pecado, pero yo subiré ahora a Jehová; quizá le aplacaré acerca de vuestro pecado.
\par 31 Entonces volvió Moisés a Jehová, y dijo: Te ruego, pues este pueblo ha cometido un gran pecado, porque se hicieron dioses de oro,
\par 32 que perdones ahora su pecado, y si no, ráeme ahora de tu libro que has escrito.
\par 33 Y Jehová respondió a Moisés: Al que pecare contra mí, a éste raeré yo de mi libro.
\par 34 Ve, pues, ahora, lleva a este pueblo a donde te he dicho; he aquí mi ángel irá delante de ti; pero en el día del castigo, yo castigaré en ellos su pecado.
\par 35 Y Jehová hirió al pueblo, porque habían hecho el becerro que formó Aarón.

\chapter{33}

\section*{La presencia de Dios prometida}

\par 1 Jehová dijo a Moisés: Anda, sube de aquí, tú y el pueblo que sacaste de la tierra de Egipto, a la tierra de la cual juré a Abraham, Isaac y Jacob, diciendo: A tu descendencia la daré;
\par 2 y yo enviaré delante de ti el ángel, y echaré fuera al cananeo y al amorreo, al heteo, al ferezeo, al heveo y al jebuseo
\par 3 (a la tierra que fluye leche y miel); pero yo no subiré en medio de ti, porque eres pueblo de dura cerviz, no sea que te consuma en el camino.
\par 4 Y oyendo el pueblo esta mala noticia, vistieron luto, y ninguno se puso sus atavíos.
\par 5 Porque Jehová había dicho a Moisés: Di a los hijos de Israel: Vosotros sois pueblo de dura cerviz; en un momento subiré en medio de ti, y te consumiré. Quítate, pues, ahora tus atavíos, para que yo sepa lo que te he de hacer.
\par 6 Entonces los hijos de Israel se despojaron de sus atavíos desde el monte Horeb.
\par 7 Y Moisés tomó el tabernáculo, y lo levantó lejos, fuera del campamento, y lo llamó el Tabernáculo de Reunión. Y cualquiera que buscaba a Jehová, salía al tabernáculo de reunión que estaba fuera del campamento.
\par 8 Y sucedía que cuando salía Moisés al tabernáculo, todo el pueblo se levantaba, y cada cual estaba en pie a la puerta de su tienda, y miraban en pos de Moisés, hasta que él entraba en el tabernáculo.
\par 9 Cuando Moisés entraba en el tabernáculo, la columna de nube descendía y se ponía a la puerta del tabernáculo, y Jehová hablaba con Moisés.
\par 10 Y viendo todo el pueblo la columna de nube que estaba a la puerta del tabernáculo, se levantaba cada uno a la puerta de su tienda y adoraba.
\par 11 Y hablaba Jehová a Moisés cara a cara, como habla cualquiera a su compañero. Y él volvía al campamento; pero el joven Josué hijo de Nun, su servidor, nunca se apartaba de en medio del tabernáculo.
\par 12 Y dijo Moisés a Jehová: Mira, tú me dices a mí: Saca este pueblo; y tú no me has declarado a quién enviarás conmigo. Sin embargo, tú dices: Yo te he conocido por tu nombre, y has hallado también gracia en mis ojos.
\par 13 Ahora, pues, si he hallado gracia en tus ojos, te ruego que me muestres ahora tu camino, para que te conozca, y halle gracia en tus ojos; y mira que esta gente es pueblo tuyo.
\par 14 Y él dijo: Mi presencia irá contigo, y te daré descanso.
\par 15 Y Moisés respondió: Si tu presencia no ha de ir conmigo, no nos saques de aquí.
\par 16 ¿Y en qué se conocerá aquí que he hallado gracia en tus ojos, yo y tu pueblo, sino en que tú andes con nosotros, y que yo y tu pueblo seamos apartados de todos los pueblos que están sobre la faz de la tierra?
\par 17 Y Jehová dijo a Moisés: También haré esto que has dicho, por cuanto has hallado gracia en mis ojos, y te he conocido por tu nombre.
\par 18 El entonces dijo: Te ruego que me muestres tu gloria.
\par 19 Y le respondió: Yo haré pasar todo mi bien delante de tu rostro, y proclamaré el nombre de Jehová delante de ti; y tendré misericordia del que tendré misericordia, y seré clemente para con el que seré clemente.
\par 20 Dijo más: No podrás ver mi rostro; porque no me verá hombre, y vivirá.
\par 21 Y dijo aún Jehová: He aquí un lugar junto a mí, y tú estarás sobre la peña;
\par 22 y cuando pase mi gloria, yo te pondré en una hendidura de la peña, y te cubriré con mi mano hasta que haya pasado.
\par 23 Después apartaré mi mano, y verás mis espaldas; mas no se verá mi rostro.

\chapter{34}

\section*{El pacto renovado}

\par 1 Y Jehová dijo a Moisés: Alísate dos tablas de piedra como las primeras, y escribiré sobre esas tablas las palabras que estaban en las tablas primeras que quebraste.
\par 2 Prepárate, pues, para mañana, y sube de mañana al monte de Sinaí, y preséntate ante mí sobre la cumbre del monte.
\par 3 Y no suba hombre contigo, ni parezca alguno en todo el monte; ni ovejas ni bueyes pazcan delante del monte.
\par 4 Y Moisés alisó dos tablas de piedra como las primeras; y se levantó de mañana y subió al monte Sinaí, como le mandó Jehová, y llevó en su mano las dos tablas de piedra.
\par 5 Y Jehová descendió en la nube, y estuvo allí con él, proclamando el nombre de Jehová.
\par 6 Y pasando Jehová por delante de él, proclamó: ¡Jehová! ¡Jehová! fuerte, misericordioso y piadoso; tardo para la ira, y grande en misericordia y verdad;
\par 7 que guarda misericordia a millares, que perdona la iniquidad, la rebelión y el pecado, y que de ningún modo tendrá por inocente al malvado; que visita la iniquidad de los padres sobre los hijos y sobre los hijos de los hijos, hasta la tercera y cuarta generación.
\par 8 Entonces Moisés, apresurándose, bajó la cabeza hacia el suelo y adoró.
\par 9 Y dijo: Si ahora, Señor, he hallado gracia en tus ojos, vaya ahora el Señor en medio de nosotros; porque es un pueblo de dura cerviz; y perdona nuestra iniquidad y nuestro pecado, y tómanos por tu heredad.
\par 10 Y él contestó: He aquí, yo hago pacto delante de todo tu pueblo; haré maravillas que no han sido hechas en toda la tierra, ni en nación alguna, y verá todo el pueblo en medio del cual estás tú, la obra de Jehová; porque será cosa tremenda la que yo haré contigo.

\section*{Advertencia contra la idolatría de Canaán}

\par 11 Guarda lo que yo te mando hoy; he aquí que yo echo de delante de tu presencia al amorreo, al cananeo, al heteo, al ferezeo, al heveo y al jebuseo.
\par 12 Guárdate de hacer alianza con los moradores de la tierra donde has de entrar, para que no sean tropezadero en medio de ti.
\par 13 Derribaréis sus altares, y quebraréis sus estatuas, y cortaréis sus imágenes de Asera.
\par 14 Porque no te has de inclinar a ningún otro dios, pues Jehová, cuyo nombre es Celoso, Dios celoso es.
\par 15 Por tanto, no harás alianza con los moradores de aquella tierra; porque fornicarán en pos de sus dioses, y ofrecerán sacrificios a sus dioses, y te invitarán, y comerás de sus sacrificios;
\par 16 o tomando de sus hijas para tus hijos, y fornicando sus hijas en pos de sus dioses, harán fornicar también a tus hijos en pos de los dioses de ellas.
\par 17 No te harás dioses de fundición.

\section*{Fiestas anuales}

\par 18 La fiesta de los panes sin levadura guardarás; siete días comerás pan sin levadura, según te he mandado, en el tiempo señalado del mes de Abib; porque en el mes de Abib saliste de Egipto.
\par 19 Todo primer nacido, mío es; y de tu ganado todo primogénito de vaca o de oveja, que sea macho.
\par 20 Pero redimirás con cordero el primogénito del asno; y si no lo redimieres, quebrarás su cerviz. Redimirás todo primogénito de tus hijos; y ninguno se presentará delante de mí con las manos vacías.
\par 21 Seis días trabajarás, mas en el séptimo día descansarás; aun en la arada y en la siega, descansarás.
\par 22 También celebrarás la fiesta de las semanas, la de las primicias de la siega del trigo, y la fiesta de la cosecha a la salida del año.
\par 23 Tres veces en el año se presentará todo varón tuyo delante de Jehová el Señor, Dios de Israel.
\par 24 Porque yo arrojaré a las naciones de tu presencia, y ensancharé tu territorio; y ninguno codiciará tu tierra, cuando subas para presentarte delante de Jehová tu Dios tres veces en el año.
\par 25 No ofrecerás cosa leudada junto con la sangre de mi sacrificio, ni se dejará hasta la mañana nada del sacrificio de la fiesta de la pascua.
\par 26 Las primicias de los primeros frutos de tu tierra llevarás a la casa de Jehová tu Dios. No cocerás el cabrito en la leche de su madre.

\section*{Moisés y las tablas de la ley}

\par 27 Y Jehová dijo a Moisés: Escribe tú estas palabras; porque conforme a estas palabras he hecho pacto contigo y con Israel.
\par 28 Y él estuvo allí con Jehová cuarenta días y cuarenta noches; no comió pan, ni bebió agua; y escribió en tablas las palabras del pacto, los diez mandamientos.
\par 29 Y aconteció que descendiendo Moisés del monte Sinaí con las dos tablas del testimonio en su mano, al descender del monte, no sabía Moisés que la piel de su rostro resplandecía, después que hubo hablado con Dios.
\par 30 Y Aarón y todos los hijos de Israel miraron a Moisés, y he aquí la piel de su rostro era resplandeciente; y tuvieron miedo de acercarse a él.
\par 31 Entonces Moisés los llamó; y Aarón y todos los príncipes de la congregación volvieron a él, y Moisés les habló.
\par 32 Después se acercaron todos los hijos de Israel, a los cuales mandó todo lo que Jehová le había dicho en el monte Sinaí.
\par 33 Y cuando acabó Moisés de hablar con ellos, puso un velo sobre su rostro.
\par 34 Cuando venía Moisés delante de Jehová para hablar con él, se quitaba el velo hasta que salía; y saliendo, decía a los hijos de Israel lo que le era mandado.
\par 35 Y al mirar los hijos de Israel el rostro de Moisés, veían que la piel de su rostro era resplandeciente; y volvía Moisés a poner el velo sobre su rostro, hasta que entraba a hablar con Dios.

\chapter{35}

\section*{Reglamento del día de reposo}

\par 1 Moisés convocó a toda la congregación de los hijos de Israel y les dijo: Estas son las cosas que Jehová ha mandado que sean hechas:
\par 2 Seis días se trabajará, mas el día séptimo os será santo, día de reposo para Jehová; cualquiera que en él hiciere trabajo alguno, morirá.
\par 3 No encenderéis fuego en ninguna de vuestras moradas en el día de reposo.

\section*{La ofrenda para el tabernáculo}

\par 4 Y habló Moisés a toda la congregación de los hijos de Israel, diciendo: Esto es lo que Jehová ha mandado:
\par 5 Tomad de entre vosotros ofrenda para Jehová; todo generoso de corazón la traerá a Jehová; oro, plata, bronce,
\par 6 azul, púrpura, carmesí, lino fino, pelo de cabras,
\par 7 pieles de carneros teñidas de rojo, pieles de tejones, madera de acacia,
\par 8 aceite para el alumbrado, especias para el aceite de la unción y para el incienso aromático,
\par 9 y piedras de ónice y piedras de engaste para el efod y para el pectoral.

\section*{La obra del tabernáculo}

\par 10 Todo sabio de corazón de entre vosotros vendrá y hará todas las cosas que Jehová ha mandado:
\par 11 el tabernáculo, su tienda, su cubierta, sus corchetes, sus tablas, sus barras, sus columnas y sus basas;
\par 12 el arca y sus varas, el propiciatorio, el velo de la tienda;
\par 13 la mesa y sus varas, y todos sus utensilios, y el pan de la proposición;
\par 14 el candelero del alumbrado y sus utensilios, sus lámparas, y el aceite para el alumbrado;
\par 15 el altar del incienso y sus varas, el aceite de la unción, el incienso aromático, la cortina de la puerta para la entrada del tabernáculo;
\par 16 el altar del holocausto, su enrejado de bronce y sus varas, y todos sus utensilios, y la fuente con su base;
\par 17 las cortinas del atrio, sus columnas y sus basas, la cortina de la puerta del atrio;
\par 18 las estacas del tabernáculo, y las estacas del atrio y sus cuerdas;
\par 19 las vestiduras del servicio para ministrar en el santuario, las sagradas vestiduras de Aarón el sacerdote, y las vestiduras de sus hijos para servir en el sacerdocio.

\section*{El pueblo trae la ofrenda}

\par 20 Y salió toda la congregación de los hijos de Israel de delante de Moisés.
\par 21 Y vino todo varón a quien su corazón estimuló, y todo aquel a quien su espíritu le dio voluntad, con ofrenda a Jehová para la obra del tabernáculo de reunión y para toda su obra, y para las sagradas vestiduras.
\par 22 Vinieron así hombres como mujeres, todos los voluntarios de corazón, y trajeron cadenas y zarcillos, anillos y brazaletes y toda clase de joyas de oro; y todos presentaban ofrenda de oro a Jehová.
\par 23 Todo hombre que tenía azul, púrpura, carmesí, lino fino, pelo de cabras, pieles de carneros teñidas de rojo, o pieles de tejones, lo traía.
\par 24 Todo el que ofrecía ofrenda de plata o de bronce traía a Jehová la ofrenda; y todo el que tenía madera de acacia la traía para toda la obra del servicio.
\par 25 Además todas las mujeres sabias de corazón hilaban con sus manos, y traían lo que habían hilado: azul, púrpura, carmesí o lino fino.
\par 26 Y todas las mujeres cuyo corazón las impulsó en sabiduría hilaron pelo de cabra.
\par 27 Los príncipes trajeron piedras de ónice, y las piedras de los engastes para el efod y el pectoral,
\par 28 y las especias aromáticas, y el aceite para el alumbrado, y para el aceite de la unción, y para el incienso aromático.
\par 29 De los hijos de Israel, así hombres como mujeres, todos los que tuvieron corazón voluntario para traer para toda la obra, que Jehová había mandado por medio de Moisés que hiciesen, trajeron ofrenda voluntaria a Jehová.

\section*{Llamamiento de Bezaleel y de Aholiab}

\par 30 Y dijo Moisés a los hijos de Israel: Mirad, Jehová ha nombrado a Bezaleel hijo de Uri, hijo de Hur, de la tribu de Judá;
\par 31 y lo ha llenado del Espíritu de Dios, en sabiduría, en inteligencia, en ciencia y en todo arte,
\par 32 para proyectar diseños, para trabajar en oro, en plata y en bronce,
\par 33 y en la talla de piedras de engaste, y en obra de madera, para trabajar en toda labor ingeniosa.
\par 34 Y ha puesto en su corazón el que pueda enseñar, así él como Aholiab hijo de Ahisamac, de la tribu de Dan;
\par 35 y los ha llenado de sabiduría de corazón, para que hagan toda obra de arte y de invención, y de bordado en azul, en púrpura, en carmesí, en lino fino y en telar, para que hagan toda labor, e inventen todo diseño.

\chapter{36}

\par 1 Así, pues, Bezaleel y Aholiab, y todo hombre sabio de corazón a quien Jehová dio sabiduría e inteligencia para saber hacer toda la obra del servicio del santuario, harán todas las cosas que ha mandado Jehová.

\section*{Moisés suspende la ofrenda del pueblo}

\par 2 Y Moisés llamó a Bezaleel y a Aholiab y a todo varón sabio de corazón, en cuyo corazón había puesto Jehová sabiduría, todo hombre a quien su corazón le movió a venir a la obra para trabajar en ella.
\par 3 Y tomaron de delante de Moisés toda la ofrenda que los hijos de Israel habían traído para la obra del servicio del santuario, a fin de hacerla. Y ellos seguían trayéndole ofrenda voluntaria cada mañana.
\par 4 Tanto, que vinieron todos los maestros que hacían toda la obra del santuario, cada uno de la obra que hacía,
\par 5 y hablaron a Moisés, diciendo: El pueblo trae mucho más de lo que se necesita para la obra que Jehová ha mandado que se haga.
\par 6 Entonces Moisés mandó pregonar por el campamento, diciendo: Ningún hombre ni mujer haga más para la ofrenda del santuario. Así se le impidió al pueblo ofrecer más;
\par 7 pues tenían material abundante para hacer toda la obra, y sobraba.

\section*{Construcción del tabernáculo}

\par 8 Todos los sabios de corazón de entre los que hacían la obra, hicieron el tabernáculo de diez cortinas de lino torcido, azul, púrpura y carmesí; las hicieron con querubines de obra primorosa.
\par 9 La longitud de una cortina era de veintiocho codos, y la anchura de cuatro codos; todas las cortinas eran de igual medida.
\par 10 Cinco de las cortinas las unió entre sí, y asimismo unió las otras cinco cortinas entre sí.
\par 11 E hizo lazadas de azul en la orilla de la cortina que estaba al extremo de la primera serie; e hizo lo mismo en la orilla de la cortina final de la segunda serie.
\par 12 Cincuenta lazadas hizo en la primera cortina, y otras cincuenta en la orilla de la cortina de la segunda serie; las lazadas de la una correspondían a las de la otra.
\par 13 Hizo también cincuenta corchetes de oro, con los cuales enlazó las cortinas una con otra, y así quedó formado un tabernáculo.
\par 14 Hizo asimismo cortinas de pelo de cabra para una tienda sobre el tabernáculo; once cortinas hizo.
\par 15 La longitud de una cortina era de treinta codos, y la anchura de cuatro codos; las once cortinas tenían una misma medida.
\par 16 Y unió cinco de las cortinas aparte, y las otras seis cortinas aparte.
\par 17 Hizo además cincuenta lazadas en la orilla de la cortina que estaba al extremo de la primera serie, y otras cincuenta lazadas en la orilla de la cortina final de la segunda serie.
\par 18 Hizo también cincuenta corchetes de bronce para enlazar la tienda, de modo que fuese una.
\par 19 E hizo para la tienda una cubierta de pieles de carneros teñidas de rojo, y otra cubierta de pieles de tejones encima.
\par 20 Además hizo para el tabernáculo las tablas de madera de acacia, derechas.
\par 21 La longitud de cada tabla era de diez codos, y de codo y medio la anchura.
\par 22 Cada tabla tenía dos espigas, para unirlas una con otra; así hizo todas las tablas del tabernáculo.
\par 23 Hizo, pues, las tablas para el tabernáculo; veinte tablas al lado del sur, al mediodía.
\par 24 Hizo también cuarenta basas de plata debajo de las veinte tablas: dos basas debajo de una tabla, para sus dos espigas, y dos basas debajo de otra tabla para sus dos espigas.
\par 25 Y para el otro lado del tabernáculo, al lado norte, hizo otras veinte tablas,
\par 26 con sus cuarenta basas de plata; dos basas debajo de una tabla, y dos basas debajo de otra tabla.
\par 27 Y para el lado occidental del tabernáculo hizo seis tablas.
\par 28 Para las esquinas del tabernáculo en los dos lados hizo dos tablas,
\par 29 las cuales se unían desde abajo, y por arriba se ajustaban con un gozne; así hizo a la una y a la otra en las dos esquinas.
\par 30 Eran, pues, ocho tablas, y sus basas de plata dieciséis; dos basas debajo de cada tabla.
\par 31 Hizo también las barras de madera de acacia; cinco para las tablas de un lado del tabernáculo,
\par 32 cinco barras para las tablas del otro lado del tabernáculo, y cinco barras para las tablas del lado posterior del tabernáculo hacia el occidente.
\par 33 E hizo que la barra de en medio pasase por en medio de las tablas de un extremo al otro.
\par 34 Y cubrió de oro las tablas, e hizo de oro los anillos de ellas, por donde pasasen las barras; cubrió también de oro las barras.
\par 35 Hizo asimismo el velo de azul, púrpura, carmesí y lino torcido; lo hizo con querubines de obra primorosa.
\par 36 Y para él hizo cuatro columnas de madera de acacia, y las cubrió de oro, y sus capiteles eran de oro; y fundió para ellas cuatro basas de plata.
\par 37 Hizo también el velo para la puerta del tabernáculo, de azul, púrpura, carmesí y lino torcido, obra de recamador;
\par 38 y sus cinco columnas con sus capiteles; y cubrió de oro los capiteles y las molduras, e hizo de bronce sus cinco basas.

\chapter{37}

\section*{Mobiliario del tabernáculo}

\par 1 Hizo también Bezaleel el arca de madera de acacia; su longitud era de dos codos y medio, su anchura de codo y medio, y su altura de codo y medio.
\par 2 Y la cubrió de oro puro por dentro y por fuera, y le hizo una cornisa de oro en derredor.
\par 3 Además fundió para ella cuatro anillos de oro a sus cuatro esquinas; en un lado dos anillos y en el otro lado dos anillos.
\par 4 Hizo también varas de madera de acacia, y las cubrió de oro.
\par 5 Y metió las varas por los anillos a los lados del arca, para llevar el arca.
\par 6 Hizo asimismo el propiciatorio de oro puro; su longitud de dos codos y medio, y su anchura de codo y medio.
\par 7 Hizo también los dos querubines de oro, labrados a martillo, en los dos extremos del propiciatorio.
\par 8 Un querubín a un extremo, y otro querubín al otro extremo; de una pieza con el propiciatorio hizo los querubines a sus dos extremos.
\par 9 Y los querubines extendían sus alas por encima, cubriendo con sus alas el propiciatorio; y sus rostros el uno enfrente del otro miraban hacia el propiciatorio.
\par 10 Hizo también la mesa de madera de acacia; su longitud de dos codos, su anchura de un codo, y de codo y medio su altura;
\par 11 y la cubrió de oro puro, y le hizo una cornisa de oro alrededor.
\par 12 Le hizo también una moldura de un palmo menor   de anchura alrededor, e hizo en derredor de la moldura una cornisa de oro.
\par 13 Le hizo asimismo de fundición cuatro anillos de oro, y los puso a las cuatro esquinas que correspondían a las cuatro patas de ella.
\par 14 Debajo de la moldura estaban los anillos, por los cuales se metían las varas para llevar la mesa.
\par 15 E hizo las varas de madera de acacia para llevar la mesa, y las cubrió de oro.
\par 16 También hizo los utensilios que habían de estar sobre la mesa, sus platos, sus cucharas, sus cubiertos y sus tazones con que se había de libar, de oro fino.
\par 17 Hizo asimismo el candelero de oro puro, labrado a martillo; su pie, su caña, sus copas, sus manzanas y sus flores eran de lo mismo.
\par 18 De sus lados salían seis brazos; tres brazos de un lado del candelero, y otros tres brazos del otro lado del candelero.
\par 19 En un brazo, tres copas en forma de flor de almendro, una manzana y una flor, y en otro brazo tres copas en figura de flor de almendro, una manzana y una flor; así en los seis brazos que salían del candelero.
\par 20 Y en la caña del candelero había cuatro copas en figura de flor de almendro, sus manzanas y sus flores,
\par 21 y una manzana debajo de dos brazos del mismo, y otra manzana debajo de otros dos brazos del mismo, y otra manzana debajo de los otros dos brazos del mismo, conforme a los seis brazos que salían de él.
\par 22 Sus manzanas y sus brazos eran de lo mismo; todo era una pieza labrada a martillo, de oro puro.
\par 23 Hizo asimismo sus siete lamparillas, sus despabiladeras y sus platillos, de oro puro.
\par 24 De un talento de oro   puro lo hizo, con todos sus utensilios.
\par 25 Hizo también el altar del incienso, de madera de acacia; de un codo su longitud, y de otro codo su anchura; era cuadrado, y su altura de dos codos; y sus cuernos de la misma pieza.
\par 26 Y lo cubrió de oro puro, su cubierta y sus paredes alrededor, y sus cuernos, y le hizo una cornisa de oro alrededor.
\par 27 Le hizo también dos anillos de oro debajo de la cornisa en las dos esquinas a los dos lados, para meter por ellos las varas con que había de ser conducido.
\par 28 E hizo las varas de madera de acacia, y las cubrió de oro.
\par 29 Hizo asimismo el aceite santo de la unción, y el incienso puro, aromático, según el arte del perfumador.

\chapter{38}

\par 1 Igualmente hizo de madera de acacia el altar del holocausto; su longitud de cinco codos, y su anchura de otros cinco codos, cuadrado, y de tres codos de altura.
\par 2 E hizo sus cuernos a sus cuatro esquinas, los cuales eran de la misma pieza, y lo cubrió de bronce.
\par 3 Hizo asimismo todos los utensilios del altar; calderos, tenazas, tazones, garfios y palas; todos sus utensilios los hizo de bronce.
\par 4 E hizo para el altar un enrejado de bronce de obra de rejilla, que puso por debajo de su cerco hasta la mitad del altar.
\par 5 También fundió cuatro anillos a los cuatro extremos del enrejado de bronce, para meter las varas.
\par 6 E hizo las varas de madera de acacia, y las cubrió de bronce.
\par 7 Y metió las varas por los anillos a los lados del altar, para llevarlo con ellas; hueco lo hizo, de tablas.
\par 8 También hizo la fuente de bronce y su base de bronce, de los espejos de las mujeres que velaban a la puerta del tabernáculo de reunión.
\par El atrio del tabernáculo
\par 9 Hizo asimismo el atrio; del lado sur, al mediodía, las cortinas del atrio eran de cien codos, de lino torcido.
\par 10 Sus columnas eran veinte, con sus veinte basas de bronce; los capiteles de las columnas y sus molduras, de plata.
\par 11 Y del lado norte cortinas de cien codos;  sus columnas, veinte, con sus veinte basas de bronce; los capiteles de las columnas y sus molduras, de plata.
\par 12 Del lado del occidente, cortinas de cincuenta codos;  sus columnas diez, y sus diez basas; los capiteles de las columnas y sus molduras, de plata.
\par 13 Del lado oriental, al este, cortinas de cincuenta codos;
\par 14 a un lado cortinas de quince codos,  sus tres columnas y sus tres basas;
\par 15 al otro lado, de uno y otro lado de la puerta del atrio, cortinas de quince codos,  con sus tres columnas y sus tres basas.
\par 16 Todas las cortinas del atrio alrededor eran de lino torcido.
\par 17 Las basas de las columnas eran de bronce; los capiteles de las columnas y sus molduras, de plata; asimismo las cubiertas de las cabezas de ellas, de plata; y todas las columnas del atrio tenían molduras de plata.
\par 18 La cortina de la entrada del atrio era de obra de recamador, de azul, púrpura, carmesí y lino torcido; era de veinte codos   de longitud, y su anchura, o sea su altura, era de cinco codos, lo mismo que las cortinas del atrio.
\par 19 Sus columnas eran cuatro, con sus cuatro basas de bronce y sus capiteles de plata; y las cubiertas de los capiteles de ellas, y sus molduras, de plata.
\par 20 Todas las estacas del tabernáculo y del atrio alrededor eran de bronce.

\section*{Dirección de la obra}

\par 21 Estas son las cuentas del tabernáculo, del tabernáculo del testimonio, las que se hicieron por orden de Moisés por obra de los levitas bajo la dirección de Itamar hijo del sacerdote Aarón.
\par 22 Y Bezaleel hijo de Uri, hijo de Hur, de la tribu de Judá, hizo todas las cosas que Jehová mandó a Moisés.
\par 23 Y con él estaba Aholiab hijo de Ahisamac, de la tribu de Dan, artífice, diseñador y recamador en azul, púrpura, carmesí y lino fino.

\section*{Metales usados en el santuario}

\par 24 Todo el oro empleado en la obra, en toda la obra del santuario, el cual fue oro de la ofrenda, fue veintinueve talentos   y setecientos treinta siclos, según el siclo del santuario.
\par 25 Y la plata de los empadronados de la congregación fue cien talentos   y mil setecientos setenta y cinco siclos, según el siclo del santuario;
\par 26 medio siclo por cabeza, según el siclo del santuario; a todos los que pasaron por el censo, de edad de veinte años arriba, que fueron seiscientos tres mil quinientos cincuenta.
\par 27 Hubo además cien talentos de plata   para fundir las basas del santuario y las basas del velo; en cien basas, cien talentos, a talento por basa.
\par 28 Y de los mil setecientos setenta y cinco siclos   hizo los capiteles de las columnas, y cubrió los capiteles de ellas, y las ciñó.
\par 29 El bronce ofrendado fue setenta talentos   y dos mil cuatrocientos siclos,
\par 30 del cual fueron hechas las basas de la puerta del tabernáculo de reunión, y el altar de bronce y su enrejado de bronce, y todos los utensilios del altar,
\par 31 las basas del atrio alrededor, las basas de la puerta del atrio, y todas las estacas del tabernáculo y todas las estacas del atrio alrededor.

\chapter{39}

\section*{Hechura de las vestiduras de los sacerdotes}

\par 1 Del azul, púrpura y carmesí hicieron las vestiduras del ministerio para ministrar en el santuario, y asimismo hicieron las vestiduras sagradas para Aarón, como Jehová lo había mandado a Moisés.
\par 2 Hizo también el efod de oro, de azul, púrpura, carmesí y lino torcido.
\par 3 Y batieron láminas de oro, y cortaron hilos para tejerlos entre el azul, la púrpura, el carmesí y el lino, con labor primorosa.
\par 4 Hicieron las hombreras para que se juntasen, y se unían en sus dos extremos.
\par 5 Y el cinto del efod que estaba sobre él era de lo mismo, de igual labor; de oro, azul, púrpura, carmesí y lino torcido, como Jehová lo había mandado a Moisés.
\par 6 Y labraron las piedras de ónice montadas en engastes de oro, con grabaduras de sello con los nombres de los hijos de Israel,
\par 7 y las puso sobre las hombreras del efod, por piedras memoriales para los hijos de Israel, como Jehová lo había mandado a Moisés.
\par 8 Hizo también el pectoral de obra primorosa como la obra del efod, de oro, azul, púrpura, carmesí y lino torcido.
\par 9 Era cuadrado; doble hicieron el pectoral; su longitud era de un palmo,  y de un palmo su anchura, cuando era doblado.
\par 10 Y engastaron en él cuatro hileras de piedras. La primera hilera era un sardio, un topacio y un carbunclo; esta era la primera hilera.
\par 11 La segunda hilera, una esmeralda, un zafiro y un diamante.
\par 12 La tercera hilera, un jacinto, una ágata y una amatista.
\par 13 Y la cuarta hilera, un berilo, un ónice y un jaspe, todas montadas y encajadas en engastes de oro.
\par 14 Y las piedras eran conforme a los nombres de los hijos de Israel, doce según los nombres de ellos; como grabaduras de sello, cada una con su nombre, según las doce tribus.
\par 15 Hicieron también sobre el pectoral los cordones de forma de trenza, de oro puro.
\par 16 Hicieron asimismo dos engastes y dos anillos de oro, y pusieron dos anillos de oro en los dos extremos del pectoral,
\par 17 y fijaron los dos cordones de oro en aquellos dos anillos a los extremos del pectoral.
\par 18 Fijaron también los otros dos extremos de los dos cordones de oro en los dos engastes que pusieron sobre las hombreras del efod por delante.
\par 19 E hicieron otros dos anillos de oro que pusieron en los dos extremos del pectoral, en su orilla, frente a la parte baja del efod.
\par 20 Hicieron además dos anillos de oro que pusieron en la parte delantera de las dos hombreras del efod, hacia abajo, cerca de su juntura, sobre el cinto del efod.
\par 21 Y ataron el pectoral por sus anillos a los anillos del efod con un cordón de azul, para que estuviese sobre el cinto del mismo efod y no se separase el pectoral del efod, como Jehová lo había mandado a Moisés.
\par 22 Hizo también el manto del efod de obra de tejedor, todo de azul,
\par 23 con su abertura en medio de él, como el cuello de un coselete, con un borde alrededor de la abertura, para que no se rompiese.
\par 24 E hicieron en las orillas del manto granadas de azul, púrpura, carmesí y lino torcido.
\par 25 Hicieron también campanillas de oro puro, y pusieron campanillas entre las granadas en las orillas del manto, alrededor, entre las granadas;
\par 26 una campanilla y una granada, otra campanilla y otra granada alrededor, en las orillas del manto, para ministrar, como Jehová lo mandó a Moisés.
\par 27 Igualmente hicieron las túnicas de lino fino de obra de tejedor, para Aarón y para sus hijos.
\par 28 Asimismo la mitra de lino fino, y los adornos de las tiaras de lino fino, y los calzoncillos de lino, de lino torcido.
\par 29 También el cinto de lino torcido, de azul, púrpura y carmesí, de obra de recamador, como Jehová lo mandó a Moisés.
\par 30 Hicieron asimismo la lámina de la diadema santa de oro puro, y escribieron en ella como grabado de sello: SANTIDAD A JEHOVÁ.
\par 31 Y pusieron en ella un cordón de azul para colocarla sobre la mitra por arriba, como Jehová lo había mandado a Moisés.

\section*{La obra del tabernáculo terminada}

\par 32 Así fue acabada toda la obra del tabernáculo, del tabernáculo de reunión; e hicieron los hijos de Israel como Jehová lo había mandado a Moisés; así lo hicieron.
\par 33 Y trajeron el tabernáculo a Moisés, el tabernáculo y todos sus utensilios; sus corchetes, sus tablas, sus barras, sus columnas, sus basas;
\par 34 la cubierta de pieles de carnero teñidas de rojo, la cubierta de pieles de tejones, el velo del frente;
\par 35 el arca del testimonio y sus varas, el propiciatorio;
\par 36 la mesa, todos sus vasos, el pan de la proposición;
\par 37 el candelero puro, sus lamparillas, las lamparillas que debían mantenerse en orden, y todos sus utensilios, el aceite para el alumbrado;
\par 38 el altar de oro, el aceite de la unción, el incienso aromático, la cortina para la entrada del tabernáculo;
\par 39 el altar de bronce con su enrejado de bronce, sus varas y todos sus utensilios, la fuente y su base;
\par 40 las cortinas del atrio, sus columnas y sus basas, la cortina para la entrada del atrio, sus cuerdas y sus estacas, y todos los utensilios del servicio del tabernáculo, del tabernáculo de reunión;
\par 41 las vestiduras del servicio para ministrar en el santuario, las sagradas vestiduras para Aarón el sacerdote, y las vestiduras de sus hijos, para ministrar en el sacerdocio.
\par 42 En conformidad a todas las cosas que Jehová había mandado a Moisés, así hicieron los hijos de Israel toda la obra.
\par 43 Y vio Moisés toda la obra, y he aquí que la habían hecho como Jehová había mandado; y los bendijo.

\chapter{40}

\section*{Moisés erige el tabernáculo}

\par 1 Luego Jehová habló a Moisés, diciendo:
\par 2 En el primer día del mes primero harás levantar el tabernáculo, el tabernáculo de reunión;
\par 3 y pondrás en él el arca del testimonio, y la cubrirás con el velo.
\par 4 Meterás la mesa y la pondrás en orden; meterás también el candelero y encenderás sus lámparas,
\par 5 y pondrás el altar de oro para el incienso delante del arca del testimonio, y pondrás la cortina delante a la entrada del tabernáculo.
\par 6 Después pondrás el altar del holocausto delante de la entrada del tabernáculo, del tabernáculo de reunión.
\par 7 Luego pondrás la fuente entre el tabernáculo de reunión y el altar, y pondrás agua en ella.
\par 8 Finalmente pondrás el atrio alrededor, y la cortina a la entrada del atrio.
\par 9 Y tomarás el aceite de la unción y ungirás el tabernáculo, y todo lo que está en él; y lo santificarás con todos sus utensilios, y será santo.
\par 10 Ungirás también el altar del holocausto y todos sus utensilios; y santificarás el altar, y será un altar santísimo.
\par 11 Asimismo ungirás la fuente y su base, y la santificarás.
\par 12 Y llevarás a Aarón y a sus hijos a la puerta del tabernáculo de reunión, y los lavarás con agua.
\par 13 Y harás vestir a Aarón las vestiduras sagradas, y lo ungirás, y lo consagrarás, para que sea mi sacerdote.
\par 14 Después harás que se acerquen sus hijos, y les vestirás las túnicas;
\par 15 y los ungirás, como ungiste a su padre, y serán mis sacerdotes, y su unción les servirá por sacerdocio perpetuo, por sus generaciones.
\par 16 Y Moisés hizo conforme a todo lo que Jehová le mandó; así lo hizo.
\par 17 Así, en el día primero del primer mes, en el segundo año, el tabernáculo fue erigido.
\par 18 Moisés hizo levantar el tabernáculo, y asentó sus basas, y colocó sus tablas, y puso sus barras, e hizo alzar sus columnas.
\par 19 Levantó la tienda sobre el tabernáculo, y puso la sobrecubierta encima del mismo, como Jehová había mandado a Moisés.
\par 20 Y tomó el testimonio y lo puso dentro del arca, y colocó las varas en el arca, y encima el propiciatorio sobre el arca.
\par 21 Luego metió el arca en el tabernáculo, y puso el velo extendido, y ocultó el arca del testimonio, como Jehová había mandado a Moisés.
\par 22 Puso la mesa en el tabernáculo de reunión, al lado norte de la cortina, fuera del velo,
\par 23 y sobre ella puso por orden los panes delante de Jehová, como Jehová había mandado a Moisés.
\par 24 Puso el candelero en el tabernáculo de reunión, enfrente de la mesa, al lado sur de la cortina,
\par 25 y encendió las lámparas delante de Jehová, como Jehová había mandado a Moisés.
\par 26 Puso también el altar de oro en el tabernáculo de reunión, delante del velo,
\par 27 y quemó sobre él incienso aromático, como Jehová había mandado a Moisés.
\par 28 Puso asimismo la cortina a la entrada del tabernáculo.
\par 29 Y colocó el altar del holocausto a la entrada del tabernáculo, del tabernáculo de reunión, y sacrificó sobre él holocausto y ofrenda, como Jehová había mandado a Moisés.
\par 30 Y puso la fuente entre el tabernáculo de reunión y el altar, y puso en ella agua para lavar.
\par 31 Y Moisés y Aarón y sus hijos lavaban en ella sus manos y sus pies.
\par 32 Cuando entraban en el tabernáculo de reunión, y cuando se acercaban al altar, se lavaban, como Jehová había mandado a Moisés.
\par 33 Finalmente erigió el atrio alrededor del tabernáculo y del altar, y puso la cortina a la entrada del atrio. Así acabó Moisés la obra.

\section*{La nube sobre el tabernáculo}

\par 34 Entonces una nube cubrió el tabernáculo de reunión, y la gloria de Jehová llenó el tabernáculo.
\par 35 Y no podía Moisés entrar en el tabernáculo de reunión, porque la nube estaba sobre él, y la gloria de Jehová lo llenaba.
\par 36 Y cuando la nube se alzaba del tabernáculo, los hijos de Israel se movían en todas sus jornadas;
\par 37 pero si la nube no se alzaba, no se movían hasta el día en que ella se alzaba.
\par 38 Porque la nube de Jehová estaba de día sobre el tabernáculo, y el fuego estaba de noche sobre él, a vista de toda la casa de Israel, en todas sus jornadas.

\end{document}
