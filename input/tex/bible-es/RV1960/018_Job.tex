\begin{document}

\title{Job}

\chapter{1}

\section*{Las calamidades de Job}

\par 1 Hubo en tierra de Uz un varón llamado Job; y era este hombre perfecto y recto, temeroso de Dios y apartado del mal.
\par 2 Y le nacieron siete hijos y tres hijas.
\par 3 Su hacienda era siete mil ovejas, tres mil camellos, quinientas yuntas de bueyes, quinientas asnas, y muchísimos criados; y era aquel varón más grande que todos los orientales.
\par 4 E iban sus hijos y hacían banquetes en sus casas, cada uno en su día; y enviaban a llamar a sus tres hermanas para que comiesen y bebiesen con ellos.
\par 5 Y acontecía que habiendo pasado en turno los días del convite, Job enviaba y los santificaba, y se levantaba de mañana y ofrecía holocaustos conforme al número de todos ellos. Porque decía Job: Quizá habrán pecado mis hijos, y habrán blasfemado contra Dios en sus corazones. De esta manera hacía todos los días.
\par 6 Un día vinieron a presentarse delante de Jehová los hijos de Dios, entre los cuales vino también Satanás.
\par 7 Y dijo Jehová a Satanás: ¿De dónde vienes? Respondiendo Satanás a Jehová, dijo: De rodear la tierra y de andar por ella. 
\par 8 Y Jehová dijo a Satanás: ¿No has considerado a mi siervo Job, que no hay otro como él en la tierra, varón perfecto y recto, temeroso de Dios y apartado del mal?
\par 9 Respondiendo Satanás a Jehová, dijo: ¿Acaso teme Job a Dios de balde?
\par 10 ¿No le has cercado alrededor a él y a su casa y a todo lo que tiene? Al trabajo de sus manos has dado bendición; por tanto, sus bienes han aumentado sobre la tierra.
\par 11 Pero extiende ahora tu mano y toca todo lo que tiene, y verás si no blasfema contra ti en tu misma presencia. 
\par 12 Dijo Jehová a Satanás: He aquí, todo lo que tiene está en tu mano; solamente no pongas tu mano sobre él. Y salió Satanás de delante de Jehová.
\par 13 Y un día aconteció que sus hijos e hijas comían y bebían vino en casa de su hermano el primogénito,
\par 14 y vino un mensajero a Job, y le dijo: Estaban arando los bueyes, y las asnas paciendo cerca de ellos,
\par 15 y acometieron los sabeos y los tomaron, y mataron a los criados a filo de espada; solamente escapé yo para darte la noticia.
\par 16 Aún estaba éste hablando, cuando vino otro que dijo: Fuego de Dios cayó del cielo, que quemó las ovejas y a los pastores, y los consumió; solamente escapé yo para darte la noticia.
\par 17 Todavía estaba éste hablando, y vino otro que dijo: Los caldeos hicieron tres escuadrones, y arremetieron contra los camellos y se los llevaron, y mataron a los criados a filo de espada; y solamente escapé yo para darte la noticia.
\par 18 Entre tanto que éste hablaba, vino otro que dijo: Tus hijos y tus hijas estaban comiendo y bebiendo vino en casa de su hermano el primogénito;
\par 19 y un gran viento vino del lado del desierto y azotó las cuatro esquinas de la casa, la cual cayó sobre los jóvenes, y murieron; y solamente escapé yo para darte la noticia.
\par 20 Entonces Job se levantó, y rasgó su manto, y rasuró su cabeza, y se postró en tierra y adoró,
\par 21 y dijo: Desnudo salí del vientre de mi madre, y desnudo volveré allá. Jehová dio, y Jehová quitó; sea el nombre de Jehová bendito.
\par 22 En todo esto no pecó Job, ni atribuyó a Dios despropósito alguno.

\chapter{2}

\par 1 Aconteció que otro día vinieron los hijos de Dios para presentarse delante de Jehová, y Satanás vino también entre ellos presentándose delante de Jehová.
\par 2 Y dijo Jehová a Satanás: ¿De dónde vienes? Respondió Satanás a Jehová, y dijo: De rodear la tierra, y de andar por ella.
\par 3 Y Jehová dijo a Satanás: ¿No has considerado a mi siervo Job, que no hay otro como él en la tierra, varón perfecto y recto, temeroso de Dios y apartado del mal, y que todavía retiene su integridad, aun cuando tú me incitaste contra él para que lo arruinara sin causa?
\par 4 Respondiendo Satanás, dijo a Jehová: Piel por piel, todo lo que el hombre tiene dará por su vida.
\par 5 Pero extiende ahora tu mano, y toca su hueso y su carne, y verás si no blasfema contra ti en tu misma presencia.
\par 6 Y Jehová dijo a Satanás: He aquí, él está en tu mano; mas guarda su vida.
\par 7 Entonces salió Satanás de la presencia de Jehová, e hirió a Job con una sarna maligna desde la planta del pie hasta la coronilla de la cabeza.
\par 8 Y tomaba Job un tiesto para rascarse con él, y estaba sentado en medio de ceniza.
\par 9 Entonces le dijo su mujer: ¿Aún retienes tu integridad? Maldice a Dios, y muérete.
\par 10 Y él le dijo: Como suele hablar cualquiera de las mujeres fatuas, has hablado. ¿Qué? ¿Recibiremos de Dios el bien, y el mal no lo recibiremos? En todo esto no pecó Job con sus labios.
\par 11 Y tres amigos de Job, Elifaz temanita, Bildad suhita, y Zofar naamatita, luego que oyeron todo este mal que le había sobrevenido, vinieron cada uno de su lugar; porque habían convenido en venir juntos para condolerse de él y para consolarle.
\par 12 Los cuales, alzando los ojos desde lejos, no lo conocieron, y lloraron a gritos; y cada uno de ellos rasgó su manto, y los tres esparcieron polvo sobre sus cabezas hacia el cielo.
\par 13 Así se sentaron con él en tierra por siete días y siete noches, y ninguno le hablaba palabra, porque veían que su dolor era muy grande.

\chapter{3}

\section*{Job maldice el día en que nació}

\par 1 Después de esto abrió Job su boca, y maldijo su día. 
\par 2 Y exclamó Job, y dijo:
\par 3 Perezca el día en que yo nací,
\par Y la noche en que se dijo: Varón es concebido.
\par 4 Sea aquel día sombrío,
\par Y no cuide de él Dios desde arriba,
\par Ni claridad sobre él resplandezca.
\par 5 Aféenlo tinieblas y sombra de muerte;
\par Repose sobre él nublado
\par Que lo haga horrible como día caliginoso.
\par 6 Ocupe aquella noche la oscuridad;
\par No sea contada entre los días del año,
\par Ni venga en el número de los meses.
\par 7 ¡Oh, que fuera aquella noche solitaria,
\par Que no viniera canción alguna en ella!
\par 8 Maldíganla los que maldicen el día,
\par Los que se aprestan para despertar a Leviatán.
\par 9 Oscurézcanse las estrellas de su alba;
\par Espere la luz, y no venga,
\par Ni vea los párpados de la mañana;
\par 10 Por cuanto no cerró las puertas del vientre donde yo estaba,
\par Ni escondió de mis ojos la miseria.
\par 11 ¿Por qué no morí yo en la matriz,
\par O expiré al salir del vientre?
\par 12 ¿Por qué me recibieron las rodillas?
\par ¿Y a qué los pechos para que mamase?
\par 13 Pues ahora estaría yo muerto, y reposaría;
\par Dormiría, y entonces tendría descanso,
\par 14 Con los reyes y con los consejeros de la tierra,
\par Que reedifican para sí ruinas;
\par 15 O con los príncipes que poseían el oro,
\par Que llenaban de plata sus casas.
\par 16 ¿Por qué no fui escondido como abortivo,
\par Como los pequeñitos que nunca vieron la luz?
\par 17 Allí los impíos dejan de perturbar,
\par Y allí descansan los de agotadas fuerzas.
\par 18 Allí también reposan los cautivos;
\par No oyen la voz del capataz.
\par 19 Allí están el chico y el grande,
\par Y el siervo libre de su señor. 
\par 20 ¿Por qué se da luz al trabajado,
\par Y vida a los de ánimo amargado,
\par 21 Que esperan la muerte, y ella no llega, 
\par Aunque la buscan más que tesoros; 
\par 22 Que se alegran sobremanera,
\par Y se gozan cuando hallan el sepulcro?
\par 23 ¿Por qué se da vida al hombre que no sabe por donde ha de ir,
\par Y a quien Dios ha encerrado?
\par 24 Pues antes que mi pan viene mi suspiro,
\par Y mis gemidos corren como aguas.
\par 25 Porque el temor que me espantaba me ha venido,
\par Y me ha acontecido lo que yo temía.
\par 26 No he tenido paz, no me aseguré, ni estuve reposado;
\par No obstante, me vino turbación.

\chapter{4}

\section*{Elifaz reprende a Job}

\par 1 Entonces respondió Elifaz temanita, y dijo:
\par 2 Si probáremos a hablarte, te será molesto;
\par Pero ¿quién podrá detener las palabras?
\par 3 He aquí, tú enseñabas a muchos,
\par Y fortalecías las manos débiles;
\par 4 Al que tropezaba enderezaban tus palabras,
\par Y esforzabas las rodillas que decaían. 
\par 5 Mas ahora que el mal ha venido sobre ti, te desalientas;
\par Y cuando ha llegado hasta ti, te turbas.
\par 6 ¿No es tu temor a Dios tu confianza?
\par ¿No es tu esperanza la integridad de tus caminos? 
\par 7 Recapacita ahora; ¿qué inocente se ha perdido?
\par Y ¿en dónde han sido destruidos los rectos?
\par 8 Como yo he visto, los que aran iniquidad
\par Y siembran injuria, la siegan.
\par 9 Perecen por el aliento de Dios,
\par Y por el soplo de su ira son consumidos.
\par 10 Los rugidos del león, y los bramidos del rugiente,
\par Y los dientes de los leoncillos son quebrantados. 
\par 11 El león viejo perece por falta de presa,
\par Y los hijos de la leona se dispersan.
\par 12 El asunto también me era a mí oculto;
\par Mas mi oído ha percibido algo de ello.
\par 13 En imaginaciones de visiones nocturnas,
\par Cuando el sueño cae sobre los hombres,
\par 14 Me sobrevino un espanto y un temblor,
\par Que estremeció todos mis huesos;
\par 15 Y al pasar un espíritu por delante de mí,
\par Hizo que se erizara el pelo de mi cuerpo.
\par 16 Paróse delante de mis ojos un fantasma,
\par Cuyo rostro yo no conocí,
\par Y quedo, oí que decía:
\par 17 ¿Será el hombre más justo que Dios?
\par ¿Será el varón más limpio que el que lo hizo?
\par 18 He aquí, en sus siervos no confía,
\par Y notó necedad en sus ángeles;
\par 19 ¡Cuánto más en los que habitan en casas de barro,
\par Cuyos cimientos están en el polvo,
\par Y que serán quebrantados por la polilla!
\par 20 De la mañana a la tarde son destruidos,
\par Y se pierden para siempre, sin haber quien repare en ello. 
\par 21 Su hermosura, ¿no se pierde con ellos mismos?
\par Y mueren sin haber adquirido sabiduría.

\chapter{5}

\par 1 Ahora, pues, da voces; ¿habrá quien te responda?
\par ¿Y a cuál de los santos te volverás?
\par 2 Es cierto que al necio lo mata la ira,
\par Y al codicioso lo consume la envidia.
\par 3 Yo he visto al necio que echaba raíces,
\par Y en la misma hora maldije su habitación.
\par 4 Sus hijos estarán lejos de la seguridad;
\par En la puerta serán quebrantados,
\par Y no habrá quien los libre.
\par 5 Su mies comerán los hambrientos,
\par Y la sacarán de entre los espinos,
\par Y los sedientos beberán su hacienda.
\par 6 Porque la aflicción no sale del polvo,
\par Ni la molestia brota de la tierra.
\par 7 Pero como las chispas se levantan para volar por el aire,
\par Así el hombre nace para la aflicción.
\par 8 Ciertamente yo buscaría a Dios,
\par Y encomendaría a él mi causa;
\par 9 El cual hace cosas grandes e inescrutables,
\par Y maravillas sin número;
\par 10 Que da la lluvia sobre la faz de la tierra,
\par Y envía las aguas sobre los campos;
\par 11 Que pone a los humildes en altura,
\par Y a los enlutados levanta a seguridad;
\par 12 Que frustra los pensamientos de los astutos,
\par Para que sus manos no hagan nada;
\par 13 Que prende a los sabios en la astucia de ellos, 
\par Y frustra los designios de los perversos.
\par 14 De día tropiezan con tinieblas,
\par Y a mediodía andan a tientas como de noche.
\par 15 Así libra de la espada al pobre, de la boca de los impíos,
\par Y de la mano violenta;
\par 16 Pues es esperanza al menesteroso,
\par Y la iniquidad cerrará su boca.
\par 17 He aquí, bienaventurado es el hombre a quien Dios castiga;
\par Por tanto, no menosprecies la corrección del Todopoderoso. 
\par 18 Porque él es quien hace la llaga, y él la vendará;
\par El hiere, y sus manos curan.
\par 19 En seis tribulaciones te librará,
\par Y en la séptima no te tocará el mal.
\par 20 En el hambre te salvará de la muerte,
\par Y del poder de la espada en la guerra.
\par 21 Del azote de la lengua serás encubierto;
\par No temerás la destrucción cuando viniere.
\par 22 De la destrucción y del hambre te reirás,
\par Y no temerás de las fieras del campo;
\par 23 Pues aun con las piedras del campo tendrás tu pacto,
\par Y las fieras del campo estarán en paz contigo.
\par 24 Sabrás que hay paz en tu tienda;
\par Visitarás tu morada, y nada te faltará.
\par 25 Asimismo echarás de ver que tu descendencia es mucha,
\par Y tu prole como la hierba de la tierra.
\par 26 Vendrás en la vejez a la sepultura,
\par Como la gavilla de trigo que se recoge a su tiempo.
\par 27 He aquí lo que hemos inquirido, lo cual es así;
\par Oyelo, y conócelo tú para tu provecho.

\chapter{6}

\section*{Job reprocha la actitud de sus amigos}

\par 1 Respondió entonces Job, y dijo:
\par 2 ¡Oh, que pesasen justamente mi queja y mi tormento,
\par Y se alzasen igualmente en balanza!
\par 3 Porque pesarían ahora más que la arena del mar;
\par Por eso mis palabras han sido precipitadas.
\par 4 Porque las saetas del Todopoderoso están en mí,
\par Cuyo veneno bebe mi espíritu;
\par Y terrores de Dios me combaten.
\par 5 ¿Acaso gime el asno montés junto a la hierba?
\par ¿Muge el buey junto a su pasto?
\par 6 ¿Se comerá lo desabrido sin sal?
\par ¿Habrá gusto en la clara del huevo?
\par 7 Las cosas que mi alma no quería tocar,
\par Son ahora mi alimento.
\par 8 ¡Quién me diera que viniese mi petición,
\par Y que me otorgase Dios lo que anhelo,
\par 9 Y que agradara a Dios quebrantarme;
\par Que soltara su mano, y acabara conmigo! 
\par 10 Sería aún mi consuelo,
\par Si me asaltase con dolor sin dar más tregua,
\par Que yo no he escondido las palabras del Santo.
\par 11 ¿Cuál es mi fuerza para esperar aún?
\par ¿Y cuál mi fin para que tenga aún paciencia? 
\par 12 ¿Es mi fuerza la de las piedras,
\par O es mi carne de bronce?
\par 13 ¿No es así que ni aun a mí mismo me puedo valer,
\par Y que todo auxilio me ha faltado?
\par 14 El atribulado es consolado por su compañero;
\par Aun aquel que abandona el temor del Omnipotente. 
\par 15 Pero mis hermanos me traicionaron como un torrente;
\par Pasan como corrientes impetuosas
\par 16 Que están escondidas por la helada,
\par Y encubiertas por la nieve;
\par 17 Que al tiempo del calor son deshechas,
\par Y al calentarse, desaparecen de su lugar;
\par 18 Se apartan de la senda de su rumbo,
\par Van menguando, y se pierden.
\par 19 Miraron los caminantes de Temán,
\par Los caminantes de Sabá esperaron en ellas;
\par 20 Pero fueron avergonzados por su esperanza;
\par Porque vinieron hasta ellas, y se hallaron confusos.
\par 21 Ahora ciertamente como ellas sois vosotros;
\par Pues habéis visto el tormento, y teméis.
\par 22 ¿Os he dicho yo: Traedme,
\par Y pagad por mí de vuestra hacienda;
\par 23 Libradme de la mano del opresor,
\par Y redimidme del poder de los violentos? 
\par 24 Enseñadme, y yo callaré;
\par Hacedme entender en qué he errado.
\par 25 ¡Cuán eficaces son las palabras rectas!
\par Pero ¿qué reprende la censura vuestra?
\par 26 ¿Pensáis censurar palabras,
\par Y los discursos de un desesperado, que son como el viento?
\par 27 También os arrojáis sobre el huérfano,
\par Y caváis un hoyo para vuestro amigo.
\par 28 Ahora, pues, si queréis, miradme,
\par Y ved si digo mentira delante de vosotros.
\par 29 Volved ahora, y no haya iniquidad;
\par Volved aún a considerar mi justicia en esto.
\par 30 ¿Hay iniquidad en mi lengua?
\par ¿Acaso no puede mi paladar discernir las cosas inicuas?

\chapter{7}

\section*{Job argumenta contra Dios}

\par 1 ¿No es acaso brega la vida del hombre sobre la tierra,
\par Y sus días como los días del jornalero?
\par 2 Como el siervo suspira por la sombra,
\par Y como el jornalero espera el reposo de su trabajo, 
\par 3 Así he recibido meses de calamidad,
\par Y noches de trabajo me dieron por cuenta.
\par 4 Cuando estoy acostado, digo: ¿Cuándo me levantaré?
\par Mas la noche es larga, y estoy lleno de inquietudes hasta el alba. 
\par 5 Mi carne está vestida de gusanos, y de costras de polvo;
\par Mi piel hendida y abominable.
\par 6 Y mis días fueron más veloces que la lanzadera del tejedor,
\par Y fenecieron sin esperanza.
\par 7 Acuérdate que mi vida es un soplo,
\par Y que mis ojos no volverán a ver el bien.
\par 8 Los ojos de los que me ven, no me verán más;
\par Fijarás en mí tus ojos, y dejaré de ser.
\par 9 Como la nube se desvanece y se va,
\par Así el que desciende al Seol no subirá;
\par 10 No volverá más a su casa,
\par Ni su lugar le conocerá más.
\par 11 Por tanto, no refrenaré mi boca;
\par Hablaré en la angustia de mi espíritu,
\par Y me quejaré con la amargura de mi alma.
\par 12 ¿Soy yo el mar, o un monstruo marino,
\par Para que me pongas guarda?
\par 13 Cuando digo: Me consolará mi lecho,
\par Mi cama atenuará mis quejas; 
\par 14 Entonces me asustas con sueños,
\par Y me aterras con visiones.
\par 15 Y así mi alma tuvo por mejor la estrangulación,
\par Y quiso la muerte más que mis huesos.
\par 16 Abomino de mi vida; no he de vivir para siempre;
\par Déjame, pues, porque mis días son vanidad.
\par 17 ¿Qué es el hombre, para que lo engrandezcas,
\par Y para que pongas sobre él tu corazón, 
\par 18 Y lo visites todas las mañanas,
\par Y todos los momentos lo pruebes?
\par 19 ¿Hasta cuándo no apartarás de mí tu mirada,
\par Y no me soltarás siquiera hasta que trague mi saliva?
\par 20 Si he pecado, ¿qué puedo hacerte a ti, oh Guarda de los hombres?
\par ¿Por qué me pones por blanco tuyo,
\par Hasta convertirme en una carga para mí mismo?
\par 21 ¿Y por qué no quitas mi rebelión, y perdonas mi iniquidad?
\par Porque ahora dormiré en el polvo,
\par Y si me buscares de mañana, ya no existiré.

\chapter{8}

\section*{Bildad proclama la justicia de Dios}

\par 1 Respondió Bildad suhita, y dijo:
\par 2 ¿Hasta cuándo hablarás tales cosas,
\par Y las palabras de tu boca serán como viento impetuoso?
\par 3 ¿Acaso torcerá Dios el derecho,
\par O pervertirá el Todopoderoso la justicia?
\par 4 Si tus hijos pecaron contra él,
\par El los echó en el lugar de su pecado.
\par 5 Si tú de mañana buscares a Dios,
\par Y rogares al Todopoderoso;
\par 6 Si fueres limpio y recto,
\par Ciertamente luego se despertará por ti,
\par Y hará próspera la morada de tu justicia.
\par 7 Y aunque tu principio haya sido pequeño,
\par Tu postrer estado será muy grande.
\par 8 Porque pregunta ahora a las generaciones pasadas,
\par Y disponte para inquirir a los padres de ellas;
\par 9 Pues nosotros somos de ayer, y nada sabemos,
\par Siendo nuestros días sobre la tierra como sombra.
\par 10 ¿No te enseñarán ellos, te hablarán,
\par Y de su corazón sacarán palabras?
\par 11 ¿Crece el junco sin lodo?
\par ¿Crece el prado sin agua?
\par 12 Aun en su verdor, y sin haber sido cortado,
\par Con todo, se seca primero que toda hierba.
\par 13 Tales son los caminos de todos los que olvidan a Dios;
\par Y la esperanza del impío perecerá;
\par 14 Porque su esperanza será cortada,
\par Y su confianza es tela de araña.
\par 15 Se apoyará él en su casa, mas no permanecerá ella en pie;
\par Se asirá de ella, mas no resistirá.
\par 16 A manera de un árbol está verde delante del sol,
\par Y sus renuevos salen sobre su huerto;
\par 17 Se van entretejiendo sus raíces junto a una fuente,
\par Y enlazándose hasta un lugar pedregoso.
\par 18 Si le arrancaren de su lugar,
\par Este le negará entonces, diciendo: Nunca te vi.
\par 19 Ciertamente este será el gozo de su camino;
\par Y del polvo mismo nacerán otros.
\par 20 He aquí, Dios no aborrece al perfecto,
\par Ni apoya la mano de los malignos.
\par 21 Aún llenará tu boca de risa,
\par Y tus labios de júbilo.
\par 22 Los que te aborrecen serán vestidos de confusión;
\par Y la habitación de los impíos perecerá.

\chapter{9}

\section*{Incapacidad de Job para responder a Dios}

\par 1 Respondió Job, y dijo:
\par 2 Ciertamente yo sé que es así;
\par ¿Y cómo se justificará el hombre con Dios?
\par 3 Si quisiere contender con él,
\par No le podrá responder a una cosa entre mil.
\par 4 El es sabio de corazón, y poderoso en fuerzas;
\par ¿Quién se endureció contra él, y le fue bien?
\par 5 El arranca los montes con su furor,
\par Y no saben quién los trastornó;
\par 6 El remueve la tierra de su lugar,
\par Y hace temblar sus columnas;
\par 7 El manda al sol, y no sale;
\par Y sella las estrellas;
\par 8 El solo extendió los cielos,
\par Y anda sobre las olas del mar;
\par 9 El hizo la Osa, el Orión y las Pléyades, 
\par Y los lugares secretos del sur; 
\par 10 El hace cosas grandes e incomprensibles,
\par Y maravillosas, sin número.
\par 11 He aquí que él pasará delante de mí, y yo no lo veré;
\par Pasará, y no lo entenderé.
\par 12 He aquí, arrebatará; ¿quién le hará restituir?
\par ¿Quién le dirá: ¿Qué haces?
\par 13 Dios no volverá atrás su ira,
\par Y debajo de él se abaten los que ayudan a los soberbios.
\par 14 ¿Cuánto menos le responderé yo,
\par Y hablaré con él palabras escogidas?
\par 15 Aunque fuese yo justo, no respondería;
\par Antes habría de rogar a mi juez.
\par 16 Si yo le invocara, y él me respondiese,
\par Aún no creeré que haya escuchado mi voz.
\par 17 Porque me ha quebrantado con tempestad,
\par Y ha aumentado mis heridas sin causa. 
\par 18 No me ha concedido que tome aliento,
\par Sino que me ha llenado de amarguras.
\par 19 Si habláremos de su potencia, por cierto es fuerte;
\par Si de juicio, ¿quién me emplazará?
\par 20 Si yo me justificare, me condenaría mi boca;
\par Si me dijere perfecto, esto me haría inicuo.
\par 21 Si fuese íntegro, no haría caso de mí mismo;
\par Despreciaría mi vida.
\par 22 Una cosa resta que yo diga: 
\par Al perfecto y al impío él los consume.
\par 23 Si azote mata de repente,
\par Se ríe del sufrimiento de los inocentes.
\par 24 La tierra es entregada en manos de los impíos,
\par Y él cubre el rostro de sus jueces.
\par Si no es él, ¿quién es? ¿Dónde está?
\par 25 Mis días han sido más ligeros que un correo;
\par Huyeron, y no vieron el bien.
\par 26 Pasaron cual naves veloces;
\par Como el águila que se arroja sobre la presa.
\par 27 Si yo dijere: Olvidaré mi queja,
\par Dejaré mi triste semblante, y me esforzaré, 
\par 28 Me turban todos mis dolores;
\par Sé que no me tendrás por inocente.
\par 29 Yo soy impío;
\par ¿Para qué trabajaré en vano?
\par 30 Aunque me lave con aguas de nieve,
\par Y limpie mis manos con la limpieza misma,
\par 31 Aún me hundirás en el hoyo,
\par Y mis propios vestidos me abominarán.
\par 32 Porque no es hombre como yo, para que yo le responda,
\par Y vengamos juntamente a juicio.
\par 33 No hay entre nosotros árbitro
\par Que ponga su mano sobre nosotros dos. 
\par 34 Quite de sobre mí su vara,
\par Y su terror no me espante.
\par 35 Entonces hablaré, y no le temeré;
\par Porque en este estado no estoy en mí.

\chapter{10}

\section*{Job lamenta su condición}

\par 1 Está mi alma hastiada de mi vida;
\par Daré libre curso a mi queja,
\par Hablaré con amargura de mi alma.
\par 2 Diré a Dios: No me condenes;
\par Hazme entender por qué contiendes conmigo.
\par 3 ¿Te parece bien que oprimas,
\par Que deseches la obra de tus manos,
\par Y que favorezcas los designios de los impíos?
\par 4 ¿Tienes tú acaso ojos de carne?
\par ¿Ves tú como ve el hombre?
\par 5 ¿Son tus días como los días del hombre,
\par O tus años como los tiempos humanos,
\par 6 Para que inquieras mi iniquidad,
\par Y busques mi pecado,
\par 7 Aunque tú sabes que no soy impío,
\par Y que no hay quien de tu mano me libre?
\par 8 Tus manos me hicieron y me formaron;
\par ¿Y luego te vuelves y me deshaces?
\par 9 Acuérdate que como a barro me diste forma;
\par ¿Y en polvo me has de volver?
\par 10 ¿No me vaciaste como leche,
\par Y como queso me cuajaste?
\par 11 Me vestiste de piel y carne,
\par Y me tejiste con huesos y nervios. 
\par 12 Vida y misericordia me concediste,
\par Y tu cuidado guardó mi espíritu.
\par 13 Estas cosas tienes guardadas en tu corazón;
\par Yo sé que están cerca de ti.
\par 14 Si pequé, tú me has observado,
\par Y no me tendrás por limpio de mi iniquidad. 
\par 15 Si fuere malo, ¡ay de mí!
\par Y si fuere justo, no levantaré mi cabeza,
\par Estando hastiado de deshonra, y de verme afligido.
\par 16 Si mi cabeza se alzare, cual león tú me cazas;
\par Y vuelves a hacer en mí maravillas.
\par 17 Renuevas contra mí tus pruebas,
\par Y aumentas conmigo tu furor como tropas de relevo.
\par 18 ¿Por qué me sacaste de la matriz?
\par Hubiera yo expirado, y ningún ojo me habría visto.
\par 19 Fuera como si nunca hubiera existido,
\par Llevado del vientre a la sepultura.
\par 20 ¿No son pocos mis días?
\par Cesa, pues, y déjame, para que me consuele un poco,
\par 21 Antes que vaya para no volver,
\par A la tierra de tinieblas y de sombra de muerte;
\par 22 Tierra de oscuridad, lóbrega,
\par Como sombra de muerte y sin orden,
\par Y cuya luz es como densas tinieblas.

\chapter{11}

\section*{Zofar acusa de maldad a Job}

\par 1 Respondió Zofar naamatita, y dijo:
\par 2 ¿Las muchas palabras no han de tener respuesta?
\par ¿Y el hombre que habla mucho será justificado?
\par 3 ¿Harán tus falacias callar a los hombres?
\par ¿Harás escarnio y no habrá quien te avergüence?
\par 4 Tú dices: Mi doctrina es pura,
\par Y yo soy limpio delante de tus ojos.
\par 5 Mas ¡oh, quién diera que Dios hablara,
\par Y abriera sus labios contigo,
\par 6 Y te declarara los secretos de la sabiduría,
\par Que son de doble valor que las riquezas!
\par Conocerías entonces que Dios te ha castigado menos de lo que tu iniquidad merece.
\par 7 ¿Descubrirás tú los secretos de Dios?
\par ¿Llegarás tú a la perfección del Todopoderoso?
\par 8 Es más alta que los cielos; ¿qué harás?
\par Es más profunda que el Seol; ¿cómo la conocerás?
\par 9 Su dimensión es más extensa que la tierra,
\par Y más ancha que el mar.
\par 10 Si él pasa, y aprisiona, y llama a juicio,
\par ¿Quién podrá contrarrestarle?
\par 11 Porque él conoce a los hombres vanos;
\par Ve asimismo la iniquidad, ¿y no hará caso?
\par 12 El hombre vano se hará entendido,
\par Cuando un pollino de asno montés nazca hombre.
\par 13 Si tú dispusieres tu corazón,
\par Y extendieres a él tus manos;
\par 14 Si alguna iniquidad hubiere en tu mano, y la echares de ti,
\par Y no consintieres que more en tu casa la injusticia,
\par 15 Entonces levantarás tu rostro limpio de mancha,
\par Y serás fuerte, y nada temerás;
\par 16 Y olvidarás tu miseria,
\par O te acordarás de ella como de aguas que pasaron.
\par 17 La vida te será más clara que el mediodía;
\par Aunque oscureciere, será como la mañana.
\par 18 Tendrás confianza, porque hay esperanza;
\par Mirarás alrededor, y dormirás seguro.
\par 19 Te acostarás, y no habrá quien te espante;
\par Y muchos suplicarán tu favor.
\par 20 Pero los ojos de los malos se consumirán,
\par Y no tendrán refugio; 
\par Y su esperanza será dar su último suspiro.

\chapter{12}

\section*{Job proclama el poder y la sabiduría de Dios}

\par 1 Respondió entonces Job, diciendo:
\par 2 Ciertamente vosotros sois el pueblo,
\par Y con vosotros morirá la sabiduría.
\par 3 También tengo yo entendimiento como vosotros;
\par No soy yo menos que vosotros;
\par ¿Y quién habrá que no pueda decir otro tanto? 
\par 4 Yo soy uno de quien su amigo se mofa,
\par Que invoca a Dios, y él le responde;
\par Con todo, el justo y perfecto es escarnecido.
\par 5 Aquel cuyos pies van a resbalar
\par Es como una lámpara despreciada de aquel que está a sus anchas.
\par 6 Prosperan las tiendas de los ladrones,
\par Y los que provocan a Dios viven seguros,
\par En cuyas manos él ha puesto cuanto tienen.
\par 7 Y en efecto, pregunta ahora a las bestias, y ellas te enseñarán;
\par A las aves de los cielos, y ellas te lo mostrarán;
\par 8 O habla a la tierra, y ella te enseñará;
\par Los peces del mar te lo declararán también.
\par 9 ¿Qué cosa de todas estas no entiende
\par Que la mano de Jehová la hizo?
\par 10 En su mano está el alma de todo viviente,
\par Y el hálito de todo el género humano.
\par 11 Ciertamente el oído distingue las palabras,
\par Y el paladar gusta las viandas.
\par 12 En los ancianos está la ciencia,
\par Y en la larga edad la inteligencia.
\par 13 Con Dios está la sabiduría y el poder;
\par Suyo es el consejo y la inteligencia.
\par 14 Si él derriba, no hay quien edifique;
\par Encerrará al hombre, y no habrá quien le abra.
\par 15 Si él detiene las aguas, todo se seca;
\par Si las envía, destruyen la tierra.
\par 16 Con él está el poder y la sabiduría;
\par Suyo es el que yerra, y el que hace errar. 
\par 17 El hace andar despojados de consejo a los consejeros,
\par Y entontece a los jueces.
\par 18 El rompe las cadenas de los tiranos,
\par Y les ata una soga a sus lomos.
\par 19 El lleva despojados a los príncipes,
\par Y trastorna a los poderosos.
\par 20 Priva del habla a los que dicen verdad,
\par Y quita a los ancianos el consejo. 
\par 21 El derrama menosprecio sobre los príncipes,
\par Y desata el cinto de los fuertes.
\par 22 El descubre las profundidades de las tinieblas,
\par Y saca a luz la sombra de muerte.
\par 23 El multiplica las naciones, y él las destruye;
\par Esparce a las naciones, y las vuelve a reunir.
\par 24 El quita el entendimiento a los jefes del pueblo de la tierra,
\par Y los hace vagar como por un yermo sin camino.
\par 25 Van a tientas, como en tinieblas y sin luz,
\par Y los hace errar como borrachos.

\chapter{13}

\section*{Job defiende su integridad}

\par 1 He aquí que todas estas cosas han visto mis ojos,
\par Y oído y entendido mis oídos.
\par 2 Como vosotros lo sabéis, lo sé yo;
\par No soy menos que vosotros.
\par 3 Mas yo hablaría con el Todopoderoso,
\par Y querría razonar con Dios.
\par 4 Porque ciertamente vosotros sois fraguadores de mentira;
\par Sois todos vosotros médicos nulos.
\par 5 Ojalá callarais por completo,
\par Porque esto os fuera sabiduría.
\par 6 Oíd ahora mi razonamiento,
\par Y estad atentos a los argumentos de mis labios. 
\par 7 ¿Hablaréis iniquidad por Dios?
\par ¿Hablaréis por él engaño?
\par 8 ¿Haréis acepción de personas a su favor?
\par ¿Contenderéis vosotros por Dios?
\par 9 ¿Sería bueno que él os escudriñase?
\par ¿Os burlaréis de él como quien se burla de algún hombre?
\par 10 El os reprochará de seguro,
\par Si solapadamente hacéis acepción de personas.
\par 11 De cierto su alteza os habría de espantar,
\par Y su pavor habría de caer sobre vosotros.
\par 12 Vuestras máximas son refranes de ceniza,
\par Y vuestros baluartes son baluartes de lodo.
\par 13 Escuchadme, y hablaré yo,
\par Y que me venga después lo que viniere.
\par 14 ¿Por qué quitaré yo mi carne con mis dientes,
\par Y tomaré mi vida en mi mano?
\par 15 He aquí, aunque él me matare, en él esperaré;
\par No obstante, defenderé delante de él mis caminos,
\par 16 Y él mismo será mi salvación,
\par Porque no entrará en su presencia el impío.
\par 17 Oíd con atención mi razonamiento,
\par Y mi declaración entre en vuestros oídos.
\par 18 He aquí ahora, si yo expusiere mi causa,
\par Sé que seré justificado.
\par 19 ¿Quién es el que contenderá conmigo?
\par Porque si ahora yo callara, moriría.
\par 20 A lo menos dos cosas no hagas conmigo;
\par Entonces no me esconderé de tu rostro:
\par 21 Aparta de mí tu mano,
\par Y no me asombre tu terror.
\par 22 Llama luego, y yo responderé;
\par O yo hablaré, y respóndeme tú.
\par 23 ¿Cuántas iniquidades y pecados tengo yo?
\par Hazme entender mi transgresión y mi pecado.
\par 24 ¿Por qué escondes tu rostro,
\par Y me cuentas por tu enemigo?
\par 25 ¿A la hoja arrebatada has de quebrantar,
\par Y a una paja seca has de perseguir?
\par 26 ¿Por qué escribes contra mí amarguras,
\par Y me haces cargo de los pecados de mi juventud?
\par 27 Pones además mis pies en el cepo, y observas todos mis caminos,
\par Trazando un límite para las plantas de mis pies. 
\par 28 Y mi cuerpo se va gastando como de carcoma,
\par Como vestido que roe la polilla. 

\chapter{14}

\section*{Job discurre sobre la brevedad de la vida}

\par 1 El hombre nacido de mujer,
\par Corto de días, y hastiado de sinsabores,
\par 2 Sale como una flor y es cortado,
\par Y huye como la sombra y no permanece.
\par 3 ¿Sobre éste abres tus ojos,
\par Y me traes a juicio contigo?
\par 4 ¿Quién hará limpio a lo inmundo?
\par Nadie.
\par 5 Ciertamente sus días están determinados,
\par Y el número de sus meses está cerca de ti;
\par Le pusiste límites, de los cuales no pasará.
\par 6 Si tú lo abandonares, él dejará de ser;
\par Entre tanto deseará, como el jornalero, su día.
\par 7 Porque si el árbol fuere cortado, aún queda de él esperanza;
\par Retoñará aún, y sus renuevos no faltarán.
\par 8 Si se envejeciere en la tierra su raíz,
\par Y su tronco fuere muerto en el polvo,
\par 9 Al percibir el agua reverdecerá,
\par Y hará copa como planta nueva.
\par 10 Mas el hombre morirá, y será cortado;
\par Perecerá el hombre, ¿y dónde estará él?
\par 11 Como las aguas se van del mar,
\par Y el río se agota y se seca,
\par 12 Así el hombre yace y no vuelve a levantarse;
\par Hasta que no haya cielo, no despertarán,
\par Ni se levantarán de su sueño.
\par 13 ¡Oh, quién me diera que me escondieses en el Seol,
\par Que me encubrieses hasta apaciguarse tu ira,
\par Que me pusieses plazo, y de mí te acordaras!
\par 14 Si el hombre muriere, ¿volverá a vivir?
\par Todos los días de mi edad esperaré,
\par Hasta que venga mi liberación.
\par 15 Entonces llamarás, y yo te responderé;
\par Tendrás afecto a la hechura de tus manos.
\par 16 Pero ahora me cuentas los pasos,
\par Y no das tregua a mi pecado;
\par 17 Tienes sellada en saco mi prevaricación,
\par Y tienes cosida mi iniquidad.
\par 18 Ciertamente el monte que cae se deshace,
\par Y las peñas son removidas de su lugar;
\par 19 Las piedras se desgastan con el agua impetuosa, que se lleva el polvo de la tierra;
\par De igual manera haces tú perecer la esperanza del hombre.
\par 20 Para siempre serás más fuerte que él, y él se va;
\par Demudarás su rostro, y le despedirás. 
\par 21 Sus hijos tendrán honores, pero él no lo sabrá;
\par O serán humillados, y no entenderá de ello.
\par 22 Mas su carne sobre él se dolerá,
\par Y se entristecerá en él su alma. 

\chapter{15}

\section*{Elifaz reprende a Job}

\par 1 Respondió Elifaz temanita, y dijo:
\par 2 ¿Proferirá el sabio vana sabiduría,
\par Y llenará su vientre de viento solano?
\par 3 ¿Disputará con palabras inútiles,
\par Y con razones sin provecho?
\par 4 Tú también disipas el temor,
\par Y menoscabas la oración delante de Dios.
\par 5 Porque tu boca declaró tu iniquidad,
\par Pues has escogido el hablar de los astutos.
\par 6 Tu boca te condenará, y no yo;
\par Y tus labios testificarán contra ti.
\par 7 ¿Naciste tú primero que Adán?
\par ¿O fuiste formado antes que los collados?
\par 8 ¿Oíste tú el secreto de Dios,
\par Y está limitada a ti la sabiduría?
\par 9 ¿Qué sabes tú que no sepamos?
\par ¿Qué entiendes tú que no se halle en nosotros? 
\par 10 Cabezas canas y hombres muy ancianos hay entre nosotros,
\par Mucho más avanzados en días que tu padre.
\par 11 ¿En tan poco tienes las consolaciones de Dios,
\par Y las palabras que con dulzura se te dicen?
\par 12 ¿Por qué tu corazón te aleja,
\par Y por qué guiñan tus ojos,
\par 13 Para que contra Dios vuelvas tu espíritu,
\par Y saques tales palabras de tu boca?
\par 14 ¿Qué cosa es el hombre para que sea limpio,
\par Y para que se justifique el nacido de mujer?
\par 15 He aquí, en sus santos no confía,
\par Y ni aun los cielos son limpios delante de sus ojos;
\par 16 ¿Cuánto menos el hombre abominable y vil,
\par Que bebe la iniquidad como agua?
\par 17 Escúchame; yo te mostraré,
\par Y te contaré lo que he visto;
\par 18 Lo que los sabios nos contaron
\par De sus padres, y no lo encubrieron;
\par 19 A quienes únicamente fue dada la tierra,
\par Y no pasó extraño por en medio de ellos.
\par 20 Todos sus días, el impío es atormentado de dolor,
\par Y el número de sus años está escondido para el violento. 
\par 21 Estruendos espantosos hay en sus oídos;
\par En la prosperidad el asolador vendrá sobre él.
\par 22 El no cree que volverá de las tinieblas,
\par Y descubierto está para la espada.
\par 23 Vaga alrededor tras el pan, diciendo: ¿En dónde está?
\par Sabe que le está preparado día de tinieblas.
\par 24 Tribulación y angustia le turbarán,
\par Y se esforzarán contra él como un rey dispuesto para la batalla,
\par 25 Por cuanto él extendió su mano contra Dios,
\par Y se portó con soberbia contra el Todopoderoso.
\par 26 Corrió contra él con cuello erguido, 
\par Con la espesa barrera de sus escudos.
\par 27 Porque la gordura cubrió su rostro,
\par E hizo pliegues sobre sus ijares;
\par 28 Y habitó las ciudades asoladas,
\par Las casas inhabitadas,
\par Que estaban en ruinas.
\par 29 No prosperará, ni durarán sus riquezas,
\par Ni extenderá por la tierra su hermosura.
\par 30 No escapará de las tinieblas;
\par La llama secará sus ramas,
\par Y con el aliento de su boca perecerá.
\par 31 No confíe el iluso en la vanidad,
\par Porque ella será su recompensa.
\par 32 El será cortado antes de su tiempo,
\par Y sus renuevos no reverdecerán.
\par 33 Perderá su agraz como la vid,
\par Y derramará su flor como el olivo.
\par 34 Porque la congregación de los impíos será asolada,
\par Y fuego consumirá las tiendas de soborno.
\par 35 Concibieron dolor, dieron a luz iniquidad,
\par Y en sus entrañas traman engaño.

\chapter{16}

\section*{Job se queja contra Dios}

\par 1 Respondió Job, y dijo:
\par 2 Muchas veces he oído cosas como estas;
\par Consoladores molestos sois todos vosotros.
\par 3 ¿Tendrán fin las palabras vacías?
\par ¿O qué te anima a responder?
\par 4 También yo podría hablar como vosotros,
\par Si vuestra alma estuviera en lugar de la mía;
\par Yo podría hilvanar contra vosotros palabras,
\par Y sobre vosotros mover mi cabeza.
\par 5 Pero yo os alentaría con mis palabras,
\par Y la consolación de mis labios apaciguaría vuestro dolor. 
\par 6 Si hablo, mi dolor no cesa;
\par Y si dejo de hablar, no se aparta de mí.
\par 7 Pero ahora tú me has fatigado;
\par Has asolado toda mi compañía.
\par 8 Tú me has llenado de arrugas; testigo es mi flacura,
\par Que se levanta contra mí para testificar en mi rostro.
\par 9 Su furor me despedazó, y me ha sido contrario;
\par Crujió sus dientes contra mí;
\par Contra mí aguzó sus ojos mi enemigo.
\par 10 Abrieron contra mí su boca;
\par Hirieron mis mejillas con afrenta;
\par Contra mí se juntaron todos.
\par 11 Me ha entregado Dios al mentiroso,
\par Y en las manos de los impíos me hizo caer.
\par 12 Próspero estaba, y me desmenuzó;
\par Me arrebató por la cerviz y me despedazó, 
\par Y me puso por blanco suyo.
\par 13 Me rodearon sus flecheros,
\par Partió mis riñones, y no perdonó;
\par Mi hiel derramó por tierra.
\par 14 Me quebrantó de quebranto en quebranto;
\par Corrió contra mí como un gigante.
\par 15 Cosí cilicio sobre mi piel,
\par Y puse mi cabeza en el polvo.
\par 16 Mi rostro está inflamado con el lloro,
\par Y mis párpados entenebrecidos,
\par 17 A pesar de no haber iniquidad en mis manos,
\par Y de haber sido mi oración pura.
\par 18 ¡Oh tierra! no cubras mi sangre,
\par Y no haya lugar para mi clamor.
\par 19 Mas he aquí que en los cielos está mi testigo,
\par Y mi testimonio en las alturas.
\par 20 Disputadores son mis amigos;
\par Mas ante Dios derramaré mis lágrimas.
\par 21 ¡Ojalá pudiese disputar el hombre con Dios,
\par Como con su prójimo!
\par 22 Mas los años contados vendrán,
\par Y yo iré por el camino de donde no volveré.

\chapter{17}

\par 1 Mi aliento se agota, se acortan mis días,
\par Y me está preparado el sepulcro.
\par 2 No hay conmigo sino escarnecedores,
\par En cuya amargura se detienen mis ojos.
\par 3 Dame fianza, oh Dios; sea mi protección cerca de ti.
\par Porque ¿quién querría responder por mí?
\par 4 Porque a éstos has escondido de su corazón la inteligencia;
\par Por tanto, no los exaltarás.
\par 5 Al que denuncia a sus amigos como presa,
\par Los ojos de sus hijos desfallecerán.
\par 6 El me ha puesto por refrán de pueblos,
\par Y delante de ellos he sido como tamboril.
\par 7 Mis ojos se oscurecieron por el dolor,
\par Y mis pensamientos todos son como sombra.
\par 8 Los rectos se maravillarán de esto,
\par Y el inocente se levantará contra el impío.
\par 9 No obstante, proseguirá el justo su camino,
\par Y el limpio de manos aumentará la fuerza.
\par 10 Pero volved todos vosotros, y venid ahora,
\par Y no hallaré entre vosotros sabio.
\par 11 Pasaron mis días, fueron arrancados mis pensamientos,
\par Los designios de mi corazón.
\par 12 Pusieron la noche por día,
\par Y la luz se acorta delante de las tinieblas.
\par 13 Si yo espero, el Seol es mi casa;
\par Haré mi cama en las tinieblas.
\par 14 A la corrupción he dicho: Mi padre eres tú;
\par A los gusanos: Mi madre y mi hermana.
\par 15 ¿Dónde, pues, estará ahora mi esperanza?
\par Y mi esperanza, ¿quién la verá?
\par 16 A la profundidad del Seol descenderán,
\par Y juntamente descansarán en el polvo. 

\chapter{18}

\section*{Bildad describe la suerte de los malos}

\par 1 Respondió Bildad suhita, y dijo:
\par 2 ¿Cuándo pondréis fin a las palabras?
\par Entended, y después hablemos.
\par 3 ¿Por qué somos tenidos por bestias,
\par Y a vuestros ojos somos viles?
\par 4 Oh tú, que te despedazas en tu furor,
\par ¿Será abandonada la tierra por tu causa,
\par Y serán removidas de su lugar las peñas? 
\par 5 Ciertamente la luz de los impíos será apagada,
\par Y no resplandecerá la centella de su fuego.
\par 6 La luz se oscurecerá en su tienda,
\par Y se apagará sobre él su lámpara.
\par 7 Sus pasos vigorosos serán acortados,
\par Y su mismo consejo lo precipitará.
\par 8 Porque red será echada a sus pies,
\par Y sobre mallas andará.
\par 9 Lazo prenderá su calcañar;
\par Se afirmará la trampa contra él.
\par 10 Su cuerda está escondida en la tierra,
\par Y una trampa le aguarda en la senda.
\par 11 De todas partes lo asombrarán temores,
\par Y le harán huir desconcertado.
\par 12 Serán gastadas de hambre sus fuerzas,
\par Y a su lado estará preparado quebrantamiento.
\par 13 La enfermedad roerá su piel,
\par Y a sus miembros devorará el primogénito de la muerte.
\par 14 Su confianza será arrancada de su tienda,
\par Y al rey de los espantos será conducido.
\par 15 En su tienda morará como si no fuese suya;
\par Piedra de azufre será esparcida sobre su morada.
\par 16 Abajo se secarán sus raíces,
\par Y arriba serán cortadas sus ramas.
\par 17 Su memoria perecerá de la tierra,
\par Y no tendrá nombre por las calles.
\par 18 De la luz será lanzado a las tinieblas,
\par Y echado fuera del mundo.
\par 19 No tendrá hijo ni nieto en su pueblo,
\par Ni quien le suceda en sus moradas.
\par 20 Sobre su día se espantarán los de occidente,
\par Y pavor caerá sobre los de oriente.
\par 21 Ciertamente tales son las moradas del impío,
\par Y este será el lugar del que no conoció a Dios.

\chapter{19}

\section*{Job confía en que Dios lo justificará}

\par 1 Respondió entonces Job, y dijo:
\par 2 ¿Hasta cuándo angustiaréis mi alma,
\par Y me moleréis con palabras?
\par 3 Ya me habéis vituperado diez veces;
\par ¿No os avergonzáis de injuriarme?
\par 4 Aun siendo verdad que yo haya errado,
\par Sobre mí recaería mi error.
\par 5 Pero si vosotros os engrandecéis contra mí,
\par Y contra mí alegáis mi oprobio,
\par 6 Sabed ahora que Dios me ha derribado,
\par Y me ha envuelto en su red.
\par 7 He aquí, yo clamaré agravio, y no seré oído;
\par Daré voces, y no habrá juicio.
\par 8 Cercó de vallado mi camino, y no pasaré;
\par Y sobre mis veredas puso tinieblas.
\par 9 Me ha despojado de mi gloria,
\par Y quitado la corona de mi cabeza.
\par 10 Me arruinó por todos lados, y perezco;
\par Y ha hecho pasar mi esperanza como árbol arrancado.
\par 11 Hizo arder contra mí su furor,
\par Y me contó para sí entre sus enemigos.
\par 12 Vinieron sus ejércitos a una, y se atrincheraron en mí,
\par Y acamparon en derredor de mi tienda.
\par 13 Hizo alejar de mí a mis hermanos,
\par Y mis conocidos como extraños se apartaron de mí.
\par 14 Mis parientes se detuvieron,
\par Y mis conocidos se olvidaron de mí.
\par 15 Los moradores de mi casa y mis criadas me tuvieron por extraño;
\par Forastero fui yo a sus ojos.
\par 16 Llamé a mi siervo, y no respondió;
\par De mi propia boca le suplicaba.
\par 17 Mi aliento vino a ser extraño a mi mujer,
\par Aunque por los hijos de mis entrañas le rogaba.
\par 18 Aun los muchachos me menospreciaron;
\par Al levantarme, hablaban contra mí.
\par 19 Todos mis íntimos amigos me aborrecieron,
\par Y los que yo amaba se volvieron contra mí.
\par 20 Mi piel y mi carne se pegaron a mis huesos,
\par Y he escapado con sólo la piel de mis dientes.
\par 21 ¡Oh, vosotros mis amigos, tened compasión de mí, tened compasión de mí!
\par Porque la mano de Dios me ha tocado.
\par 22 ¿Por qué me perseguís como Dios,
\par Y ni aun de mi carne os saciáis?
\par 23 ¡Quién diese ahora que mis palabras fuesen escritas!
\par ¡Quién diese que se escribiesen en un libro;
\par 24 Que con cincel de hierro y con plomo
\par Fuesen esculpidas en piedra para siempre!
\par 25 Yo sé que mi Redentor vive,
\par Y al fin se levantará sobre el polvo;
\par 26 Y después de deshecha esta mi piel,
\par En mi carne he de ver a Dios;
\par 27 Al cual veré por mí mismo,
\par Y mis ojos lo verán, y no otro,
\par Aunque mi corazón desfallece dentro de mí.
\par 28 Mas debierais decir: ¿Por qué le perseguimos?
\par Ya que la raíz del asunto se halla en mí.
\par 29 Temed vosotros delante de la espada;
\par Porque sobreviene el furor de la espada a causa de las injusticias,
\par Para que sepáis que hay un juicio.

\chapter{20}

\section*{Zofar describe las calamidades de los malos}

\par 1 Respondió Zofar naamatita, y dijo:
\par 2 Por cierto mis pensamientos me hacen responder,
\par Y por tanto me apresuro.
\par 3 La reprensión de mi censura he oído,
\par Y me hace responder el espíritu de mi inteligencia.
\par 4 ¿No sabes esto, que así fue siempre,
\par Desde el tiempo que fue puesto el hombre sobre la tierra,
\par 5 Que la alegría de los malos es breve,
\par Y el gozo del impío por un momento?
\par 6 Aunque subiere su altivez hasta el cielo,
\par Y su cabeza tocare en las nubes,
\par 7 Como su estiércol, perecerá para siempre;
\par Los que le hubieren visto dirán: ¿Qué hay de él?
\par 8 Como sueño volará, y no será hallado,
\par Y se disipará como visión nocturna.
\par 9 El ojo que le veía, nunca más le verá,
\par Ni su lugar le conocerá más.
\par 10 Sus hijos solicitarán el favor de los pobres,
\par Y sus manos devolverán lo que él robó. 
\par 11 Sus huesos están llenos de su juventud,
\par Mas con él en el polvo yacerán.
\par 12 Si el mal se endulzó en su boca,
\par Si lo ocultaba debajo de su lengua,
\par 13 Si le parecía bien, y no lo dejaba,
\par Sino que lo detenía en su paladar;
\par 14 Su comida se mudará en sus entrañas;
\par Hiel de áspides será dentro de él.
\par 15 Devoró riquezas, pero las vomitará;
\par De su vientre las sacará Dios.
\par 16 Veneno de áspides chupará;
\par Lo matará lengua de víbora.
\par 17 No verá los arroyos, los ríos, 
\par Los torrentes de miel y de leche.
\par 18 Restituirá el trabajo conforme a los bienes que tomó,
\par Y no los tragará ni gozará.
\par 19 Por cuanto quebrantó y desamparó a los pobres,
\par Robó casas, y no las edificó;
\par 20 Por tanto, no tendrá sosiego en su vientre,
\par Ni salvará nada de lo que codiciaba. 
\par 21 No quedó nada que no comiese;
\par Por tanto, su bienestar no será duradero. 
\par 22 En el colmo de su abundancia padecerá estrechez;
\par La mano de todos los malvados vendrá sobre él.
\par 23 Cuando se pusiere a llenar su vientre,
\par Dios enviará sobre él el ardor de su ira,
\par Y la hará llover sobre él y sobre su comida.
\par 24 Huirá de las armas de hierro,
\par Y el arco de bronce le atravesará.
\par 25 La saeta le traspasará y saldrá de su cuerpo,
\par Y la punta relumbrante saldrá por su hiel;
\par Sobre él vendrán terrores.
\par 26 Todas las tinieblas están reservadas para sus tesoros;
\par Fuego no atizado los consumirá;
\par Devorará lo que quede en su tienda.
\par 27 Los cielos descubrirán su iniquidad,
\par Y la tierra se levantará contra él.
\par 28 Los renuevos de su casa serán transportados;
\par Serán esparcidos en el día de su furor.
\par 29 Esta es la porción que Dios prepara al hombre impío,
\par Y la heredad que Dios le señala por su palabra.

\chapter{21}

\section*{Job afirma que los malos prosperan}

\par 1 Entonces respondió Job, y dijo:
\par 2 Oíd atentamente mi palabra,
\par Y sea esto el consuelo que me deis.
\par 3 Toleradme, y yo hablaré;
\par Y después que haya hablado, escarneced.
\par 4 ¿Acaso me quejo yo de algún hombre?
\par ¿Y por qué no se ha de angustiar mi espíritu?
\par 5 Miradme, y espantaos,
\par Y poned la mano sobre la boca.
\par 6 Aun yo mismo, cuando me acuerdo, me asombro,
\par Y el temblor estremece mi carne.
\par 7 ¿Por qué viven los impíos,
\par Y se envejecen, y aun crecen en riquezas?
\par 8 Su descendencia se robustece a su vista,
\par Y sus renuevos están delante de sus ojos.
\par 9 Sus casas están a salvo de temor,
\par Ni viene azote de Dios sobre ellos.
\par 10 Sus toros engendran, y no fallan;
\par Paren sus vacas, y no malogran su cría.
\par 11 Salen sus pequeñuelos como manada,
\par Y sus hijos andan saltando.
\par 12 Al son de tamboril y de cítara saltan,
\par Y se regocijan al son de la flauta.
\par 13 Pasan sus días en prosperidad,
\par Y en paz descienden al Seol.
\par 14 Dicen, pues, a Dios: Apártate de nosotros,
\par Porque no queremos el conocimiento de tus caminos.
\par 15 ¿Quién es el Todopoderoso, para que le sirvamos?
\par ¿Y de qué nos aprovechará que oremos a él?
\par 16 He aquí que su bien no está en mano de ellos;
\par El consejo de los impíos lejos esté de mí.
\par 17 ¡Oh, cuántas veces la lámpara de los impíos es apagada,
\par Y viene sobre ellos su quebranto,
\par Y Dios en su ira les reparte dolores!
\par 18 Serán como la paja delante del viento,
\par Y como el tamo que arrebata el torbellino.
\par 19 Dios guardará para los hijos de ellos su violencia;
\par Le dará su pago, para que conozca.
\par 20 Verán sus ojos su quebranto,
\par Y beberá de la ira del Todopoderoso.
\par 21 Porque ¿qué deleite tendrá él de su casa después de sí,
\par Siendo cortado el número de sus meses?
\par 22 ¿Enseñará alguien a Dios sabiduría,
\par Juzgando él a los que están elevados?
\par 23 Este morirá en el vigor de su hermosura, todo quieto y pacífico;
\par 24 Sus vasijas estarán llenas de leche,
\par Y sus huesos serán regados de tuétano.
\par 25 Y este otro morirá en amargura de ánimo,
\par Y sin haber comido jamás con gusto.
\par 26 Igualmente yacerán ellos en el polvo,
\par Y gusanos los cubrirán.
\par 27 He aquí, yo conozco vuestros pensamientos,
\par Y las imaginaciones que contra mí forjáis.
\par 28 Porque decís: ¿Qué hay de la casa del príncipe,
\par Y qué de la tienda de las moradas de los impíos?
\par 29 ¿No habéis preguntado a los que pasan por los caminos,
\par Y no habéis conocido su respuesta,
\par 30 Que el malo es preservado en el día de la destrucción?
\par Guardado será en el día de la ira.
\par 31 ¿Quién le denunciará en su cara su camino?
\par Y de lo que él hizo, ¿quién le dará el pago?
\par 32 Porque llevado será a los sepulcros,
\par Y sobre su túmulo estarán velando.
\par 33 Los terrones del valle le serán dulces;
\par Tras de él será llevado todo hombre,
\par Y antes de él han ido innumerables.
\par 34 ¿Cómo, pues, me consoláis en vano,
\par Viniendo a parar vuestras respuestas en falacia? 


\chapter{22}

\section*{Elifaz acusa a Job de gran maldad}

\par 1 Respondió Elifaz temanita, y dijo:
\par 2 ¿Traerá el hombre provecho a Dios?
\par Al contrario, para sí mismo es provechoso el hombre sabio.
\par 3 ¿Tiene contentamiento el Omnipotente en que tú seas justificado,
\par O provecho de que tú hagas perfectos tus caminos? 
\par 4 ¿Acaso te castiga,
\par O viene a juicio contigo, a causa de tu piedad?
\par 5 Por cierto tu malicia es grande,
\par Y tus maldades no tienen fin.
\par 6 Porque sacaste prenda a tus hermanos sin causa,
\par Y despojaste de sus ropas a los desnudos.
\par 7 No diste de beber agua al cansado,
\par Y detuviste el pan al hambriento.
\par 8 Pero el hombre pudiente tuvo la tierra,
\par Y habitó en ella el distinguido.
\par 9 A las viudas enviaste vacías,
\par Y los brazos de los huérfanos fueron quebrados.
\par 10 Por tanto, hay lazos alrededor de ti,
\par Y te turba espanto repentino;
\par 11 O tinieblas, para que no veas,
\par Y abundancia de agua te cubre.
\par 12 ¿No está Dios en la altura de los cielos?
\par Mira lo encumbrado de las estrellas, cuán elevadas están.
\par 13 ¿Y dirás tú: ¿Qué sabe Dios?
\par ¿Cómo juzgará a través de la oscuridad?
\par 14 Las nubes le rodearon, y no ve;
\par Y por el circuito del cielo se pasea.
\par 15 ¿Quieres tú seguir la senda antigua
\par Que pisaron los hombres perversos,
\par 16 Los cuales fueron cortados antes de tiempo,
\par Cuyo fundamento fue como un río derramado?
\par 17 Decían a Dios: Apártate de nosotros.
\par ¿Y qué les había hecho el Omnipotente?
\par 18 Les había colmado de bienes sus casas.
\par Pero sea el consejo de ellos lejos de mí.
\par 19 Verán los justos y se gozarán;
\par Y el inocente los escarnecerá, diciendo:
\par 20 Fueron destruidos nuestros adversarios,
\par Y el fuego consumió lo que de ellos quedó.
\par 21 Vuelve ahora en amistad con él, y tendrás paz;
\par Y por ello te vendrá bien.
\par 22 Toma ahora la ley de su boca,
\par Y pon sus palabras en tu corazón.
\par 23 Si te volvieres al Omnipotente, serás edificado;
\par Alejarás de tu tienda la aflicción;
\par 24 Tendrás más oro que tierra,
\par Y como piedras de arroyos oro de Ofir;
\par 25 El Todopoderoso será tu defensa,
\par Y tendrás plata en abundancia.
\par 26 Porque entonces te deleitarás en el Omnipotente,
\par Y alzarás a Dios tu rostro.
\par 27 Orarás a él, y él te oirá;
\par Y tú pagarás tus votos.
\par 28 Determinarás asimismo una cosa, y te será firme,
\par Y sobre tus caminos resplandecerá luz.
\par 29 Cuando fueren abatidos, dirás tú: Enaltecimiento habrá;
\par Y Dios salvará al humilde de ojos.
\par 30 El libertará al inocente,
\par Y por la limpieza de tus manos éste será librado.

\chapter{23}

\section*{Job desea abogar su causa delante de Dios}

\par 1 Respondió Job, y dijo:
\par 2 Hoy también hablaré con amargura;
\par Porque es más grave mi llaga que mi gemido.
\par 3 ¡Quién me diera el saber dónde hallar a Dios!
\par Yo iría hasta su silla.
\par 4 Expondría mi causa delante de él,
\par Y llenaría mi boca de argumentos.
\par 5 Yo sabría lo que él me respondiese,
\par Y entendería lo que me dijera.
\par 6 ¿Contendería conmigo con grandeza de fuerza? 
\par No; antes él me atendería.
\par 7 Allí el justo razonaría con él;
\par Y yo escaparía para siempre de mi juez.
\par 8 He aquí yo iré al oriente, y no lo hallaré;
\par Y al occidente, y no lo percibiré;
\par 9 Si muestra su poder al norte, yo no lo veré;
\par Al sur se esconderá, y no lo veré.
\par 10 Mas él conoce mi camino;
\par Me probará, y saldré como oro.
\par 11 Mis pies han seguido sus pisadas;
\par Guardé su camino, y no me aparté.
\par 12 Del mandamiento de sus labios nunca me separé;
\par Guardé las palabras de su boca más que mi comida.
\par 13 Pero si él determina una cosa, ¿quién lo hará cambiar?
\par Su alma deseó, e hizo.
\par 14 El, pues, acabará lo que ha determinado de mí;
\par Y muchas cosas como estas hay en él.
\par 15 Por lo cual yo me espanto en su presencia;
\par Cuando lo considero, tiemblo a causa de él.
\par 16 Dios ha enervado mi corazón, 
\par Y me ha turbado el Omnipotente.
\par 17 ¿Por qué no fui yo cortado delante de las tinieblas,
\par Ni fue cubierto con oscuridad mi rostro?

\chapter{24}

\section*{Job se queja de que Dios es indiferente ante la maldad}

\par 1 Puesto que no son ocultos los tiempos al Todopoderoso,
\par ¿Por qué los que le conocen no ven sus días?
\par 2 Traspasan los linderos,
\par Roban los ganados, y los apacientan.
\par 3 Se llevan el asno de los huérfanos,
\par Y toman en prenda el buey de la viuda.
\par 4 Hacen apartar del camino a los menesterosos,
\par Y todos los pobres de la tierra se esconden.
\par 5 He aquí, como asnos monteses en el desierto,
\par Salen a su obra madrugando para robar;
\par El desierto es mantenimiento de sus hijos.
\par 6 En el campo siegan su pasto,
\par Y los impíos vendimian la viña ajena.
\par 7 Al desnudo hacen dormir sin ropa,
\par Sin tener cobertura contra el frío.
\par 8 Con las lluvias de los montes se mojan,
\par Y abrazan las peñas por falta de abrigo.
\par 9 Quitan el pecho a los huérfanos,
\par Y de sobre el pobre toman la prenda.
\par 10 Al desnudo hacen andar sin vestido,
\par Y a los hambrientos quitan las gavillas.
\par 11 Dentro de sus paredes exprimen el aceite,
\par Pisan los lagares, y mueren de sed.
\par 12 Desde la ciudad gimen los moribundos,
\par Y claman las almas de los heridos de muerte,
\par Pero Dios no atiende su oración.
\par 13 Ellos son los que, rebeldes a la luz,
\par Nunca conocieron sus caminos,
\par Ni estuvieron en sus veredas.
\par 14 A la luz se levanta el matador; mata al pobre y al necesitado, 
\par Y de noche es como ladrón.
\par 15 El ojo del adúltero está aguardando la noche,
\par Diciendo: No me verá nadie;
\par Y esconde su rostro.
\par 16 En las tinieblas minan las casas
\par Que de día para sí señalaron;
\par No conocen la luz.
\par 17 Porque la mañana es para todos ellos como sombra de muerte;
\par Si son conocidos, terrores de sombra de muerte los toman.
\par 18 Huyen ligeros como corriente de aguas;
\par Su porción es maldita en la tierra;
\par No andarán por el camino de las viñas.
\par 19 La sequía y el calor arrebatan las aguas de la nieve; 
\par Así también el Seol a los pecadores.
\par 20 Los olvidará el seno materno; de ellos sentirán los gusanos dulzura;
\par Nunca más habrá de ellos memoria,
\par Y como un árbol los impíos serán quebrantados.
\par 21 A la mujer estéril, que no concebía, afligió,
\par Y a la viuda nunca hizo bien.
\par 22 Pero a los fuertes adelantó con su poder;
\par Una vez que se levante, ninguno está seguro de la vida.
\par 23 El les da seguridad y confianza;
\par Sus ojos están sobre los caminos de ellos.
\par 24 Fueron exaltados un poco, mas desaparecen,
\par Y son abatidos como todos los demás;
\par Serán encerrados, y cortados como cabezas de espigas.
\par 25 Y si no, ¿quién me desmentirá ahora,
\par O reducirá a nada mis palabras?

\chapter{25}

\section*{Bildad niega que el hombre pueda ser justificado delante de Dios}

\par 1 Respondió Bildad suhita, y dijo:
\par 2 El señorío y el temor están con él;
\par El hace paz en sus alturas.
\par 3 ¿Tienen sus ejércitos número?
\par ¿Sobre quién no está su luz?
\par 4 ¿Cómo, pues, se justificará el hombre para con Dios?
\par ¿Y cómo será limpio el que nace de mujer?
\par 5 He aquí que ni aun la misma luna será resplandeciente,
\par Ni las estrellas son limpias delante de sus ojos; 
\par 6 ¿Cuánto menos el hombre, que es un gusano,
\par Y el hijo de hombre, también gusano?

\chapter{26}

\section*{Job proclama la soberanía de Dios}

\par 1 Respondió Job, y dijo:
\par 2 ¿En qué ayudaste al que no tiene poder?
\par ¿Cómo has amparado al brazo sin fuerza?
\par 3 ¿En qué aconsejaste al que no tiene ciencia,
\par Y qué plenitud de inteligencia has dado a conocer?
\par 4 ¿A quién has anunciado palabras,
\par Y de quién es el espíritu que de ti procede?
\par 5 Las sombras tiemblan en lo profundo,
\par Los mares y cuanto en ellos mora.
\par 6 El Seol está descubierto delante de él, y el Abadón no tiene cobertura.
\par 7 El extiende el norte sobre vacío,
\par Cuelga la tierra sobre nada.
\par 8 Ata las aguas en sus nubes,
\par Y las nubes no se rompen debajo de ellas.
\par 9 El encubre la faz de su trono,
\par Y sobre él extiende su nube.
\par 10 Puso límite a la superficie de las aguas,
\par Hasta el fin de la luz y las tinieblas.
\par 11 Las columnas del cielo tiemblan,
\par Y se espantan a su reprensión.
\par 12 El agita el mar con su poder,
\par Y con su entendimiento hiere la arrogancia suya.
\par 13 Su espíritu adornó los cielos;
\par Su mano creó la serpiente tortuosa.
\par 14 He aquí, estas cosas son sólo los bordes de sus caminos;
\par ¡Y cuán leve es el susurro que hemos oído de él!
\par Pero el trueno de su poder, ¿quién lo puede comprender?

\chapter{27}

\section*{Job describe el castigo de los malos}

\par 1 Reasumió Job su discurso, y dijo:
\par 2 Vive Dios, que ha quitado mi derecho,
\par Y el Omnipotente, que amargó el alma mía,
\par 3 Que todo el tiempo que mi alma esté en mí,
\par Y haya hálito de Dios en mis narices,
\par 4 Mis labios no hablarán iniquidad,
\par Ni mi lengua pronunciará engaño.
\par 5 Nunca tal acontezca que yo os justifique;
\par Hasta que muera, no quitaré de mí mi integridad.
\par 6 Mi justicia tengo asida, y no la cederé;
\par No me reprochará mi corazón en todos mis días.
\par 7 Sea como el impío mi enemigo,
\par Y como el inicuo mi adversario.
\par 8 Porque ¿cuál es la esperanza del impío, por mucho que hubiere robado,
\par Cuando Dios le quitare la vida?
\par 9 ¿Oirá Dios su clamor
\par Cuando la tribulación viniere sobre él?
\par 10 ¿Se deleitará en el Omnipotente?
\par ¿Invocará a Dios en todo tiempo?
\par 11 Yo os enseñaré en cuanto a la mano de Dios;
\par No esconderé lo que hay para con el Omnipotente.
\par 12 He aquí que todos vosotros lo habéis visto;
\par ¿Por qué, pues, os habéis hecho tan enteramente vanos?
\par 13 Esta es para con Dios la porción del hombre impío,
\par Y la herencia que los violentos han de recibir del Omnipotente:
\par 14 Si sus hijos fueren multiplicados, serán para la espada;
\par Y sus pequeños no se saciarán de pan.
\par 15 Los que de él quedaren, en muerte serán sepultados,
\par Y no los llorarán sus viudas.
\par 16 Aunque amontone plata como polvo,
\par Y prepare ropa como lodo;
\par 17 La habrá preparado él, mas el justo se vestirá,
\par Y el inocente repartirá la plata.
\par 18 Edificó su casa como la polilla,
\par Y como enramada que hizo el guarda.
\par 19 Rico se acuesta, pero por última vez;
\par Abrirá sus ojos, y nada tendrá.
\par 20 Se apoderarán de él terrores como aguas;
\par Torbellino lo arrebatará de noche.
\par 21 Le eleva el solano, y se va;
\par Y tempestad lo arrebatará de su lugar.
\par 22 Dios, pues, descargará sobre él, y no perdonará;
\par Hará él por huir de su mano.
\par 23 Batirán las manos sobre él,
\par Y desde su lugar le silbarán.

\chapter{28}

\section*{El hombre en busca de la sabiduría}

\par 1 Ciertamente la plata tiene sus veneros,
\par Y el oro lugar donde se refina.
\par 2 El hierro se saca del polvo,
\par Y de la piedra se funde el cobre.
\par 3 A las tinieblas ponen término,
\par Y examinan todo a la perfección,
\par Las piedras que hay en oscuridad y en sombra de muerte. 
\par 4 Abren minas lejos de lo habitado,
\par En lugares olvidados, donde el pie no pasa.
\par Son suspendidos y balanceados, lejos de los demás hombres.
\par 5 De la tierra nace el pan,
\par Y debajo de ella está como convertida en fuego.
\par 6 Lugar hay cuyas piedras son zafiro,
\par Y sus polvos de oro.
\par 7 Senda que nunca la conoció ave,
\par Ni ojo de buitre la vio;
\par 8 Nunca la pisaron animales fieros,
\par Ni león pasó por ella.
\par 9 En el pedernal puso su mano,
\par Y trastornó de raíz los montes.
\par 10 De los peñascos cortó ríos,
\par Y sus ojos vieron todo lo preciado.
\par 11 Detuvo los ríos en su nacimiento,
\par E hizo salir a luz lo escondido.
\par 12 Mas ¿dónde se hallará la sabiduría?
\par ¿Dónde está el lugar de la inteligencia?
\par 13 No conoce su valor el hombre,
\par Ni se halla en la tierra de los vivientes.
\par 14 El abismo dice: No está en mí;
\par Y el mar dijo: Ni conmigo.
\par 15 No se dará por oro,
\par Ni su precio será a peso de plata.
\par 16 No puede ser apreciada con oro de Ofir,
\par Ni con ónice precioso, ni con zafiro.
\par 17 El oro no se le igualará, ni el diamante,
\par Ni se cambiará por alhajas de oro fino.
\par 18 No se hará mención de coral ni de perlas;
\par La sabiduría es mejor que las piedras preciosas.
\par 19 No se igualará con ella topacio de Etiopía;
\par No se podrá apreciar con oro fino.
\par 20 ¿De dónde, pues, vendrá la sabiduría?
\par ¿Y dónde está el lugar de la inteligencia?
\par 21 Porque encubierta está a los ojos de todo viviente,
\par Y a toda ave del cielo es oculta.
\par 22 El Abadón y la muerte dijeron:
\par Su fama hemos oído con nuestros oídos.
\par 23 Dios entiende el camino de ella,
\par Y conoce su lugar.
\par 24 Porque él mira hasta los fines de la tierra,
\par Y ve cuanto hay bajo los cielos.
\par 25 Al dar peso al viento,
\par Y poner las aguas por medida;
\par 26 Cuando él dio ley a la lluvia,
\par Y camino al relámpago de los truenos,
\par 27 Entonces la veía él, y la manifestaba;
\par La preparó y la descubrió también.
\par 28 Y dijo al hombre:
\par He aquí que el temor del Señor es la sabiduría, 
\par Y el apartarse del mal, la inteligencia.

\chapter{29}

\section*{Job recuerda su felicidad anterior}

\par 1 Volvió Job a reanudar su discurso, y dijo:
\par 2 ¡Quién me volviese como en los meses pasados,
\par Como en los días en que Dios me guardaba,
\par 3 Cuando hacía resplandecer sobre mi cabeza su lámpara,
\par A cuya luz yo caminaba en la oscuridad;
\par 4 Como fui en los días de mi juventud,
\par Cuando el favor de Dios velaba sobre mi tienda; 
\par 5 Cuando aún estaba conmigo el Omnipotente,
\par Y mis hijos alrededor de mí;
\par 6 Cuando lavaba yo mis pasos con leche,
\par Y la piedra me derramaba ríos de aceite!
\par 7 Cuando yo salía a la puerta a juicio,
\par Y en la plaza hacía preparar mi asiento,
\par 8 Los jóvenes me veían, y se escondían;
\par Y los ancianos se levantaban, y estaban de pie.
\par 9 Los príncipes detenían sus palabras;
\par Ponían la mano sobre su boca. 
\par 10 La voz de los principales se apagaba,
\par Y su lengua se pegaba a su paladar.
\par 11 Los oídos que me oían me llamaban bienaventurado,
\par Y los ojos que me veían me daban testimonio,
\par 12 Porque yo libraba al pobre que clamaba,
\par Y al huérfano que carecía de ayudador.
\par 13 La bendición del que se iba a perder venía sobre mí,
\par Y al corazón de la viuda yo daba alegría.
\par 14 Me vestía de justicia, y ella me cubría;
\par Como manto y diadema era mi rectitud. 
\par 15 Yo era ojos al ciego,
\par Y pies al cojo.
\par 16 A los menesterosos era padre,
\par Y de la causa que no entendía, me informaba con diligencia;
\par 17 Y quebrantaba los colmillos del inicuo,
\par Y de sus dientes hacía soltar la presa.
\par 18 Decía yo: En mi nido moriré,
\par Y como arena multiplicaré mis días.
\par 19 Mi raíz estaba abierta junto a las aguas,
\par Y en mis ramas permanecía el rocío.
\par 20 Mi honra se renovaba en mí,
\par Y mi arco se fortalecía en mi mano.
\par 21 Me oían, y esperaban,
\par Y callaban a mi consejo.
\par 22 Tras mi palabra no replicaban, 
\par Y mi razón destilaba sobre ellos.
\par 23 Me esperaban como a la lluvia,
\par Y abrían su boca como a la lluvia tardía.
\par 24 Si me reía con ellos, no lo creían;
\par Y no abatían la luz de mi rostro.
\par 25 Calificaba yo el camino de ellos, y me sentaba entre ellos como el jefe;
\par Y moraba como rey en el ejército,
\par Como el que consuela a los que lloran.

\chapter{30}

\section*{Job lamenta su desdicha actual}

\par 1 Pero ahora se ríen de mí los más jóvenes que yo,
\par A cuyos padres yo desdeñara poner con los perros de mi ganado.
\par 2 ¿Y de qué me serviría ni aun la fuerza de sus manos?
\par No tienen fuerza alguna.
\par 3 Por causa de la pobreza y del hambre andaban solos;
\par Huían a la soledad, a lugar tenebroso, asolado y desierto.
\par 4 Recogían malvas entre los arbustos,
\par Y raíces de enebro para calentarse.
\par 5 Eran arrojados de entre las gentes,
\par Y todos les daban grita como tras el ladrón.
\par 6 Habitaban en las barrancas de los arroyos,
\par En las cavernas de la tierra, y en las rocas.
\par 7 Bramaban entre las matas,
\par Y se reunían debajo de los espinos.
\par 8 Hijos de viles, y hombres sin nombre,
\par Más bajos que la misma tierra.
\par 9 Y ahora yo soy objeto de su burla,
\par Y les sirvo de refrán.
\par 10 Me abominan, se alejan de mí,
\par Y aun de mi rostro no detuvieron su saliva.
\par 11 Porque Dios desató su cuerda, y me afligió, 
\par Por eso se desenfrenaron delante de mi rostro.
\par 12 A la mano derecha se levantó el populacho;
\par Empujaron mis pies,
\par Y prepararon contra mí caminos de perdición.
\par 13 Mi senda desbarataron,
\par Se aprovecharon de mi quebrantamiento,
\par Y contra ellos no hubo ayudador.
\par 14 Vinieron como por portillo ancho,
\par Se revolvieron sobre mi calamidad.
\par 15 Se han revuelto turbaciones sobre mí;
\par Combatieron como viento mi honor,
\par Y mi prosperidad pasó como nube.
\par 16 Y ahora mi alma está derramada en mí;
\par Días de aflicción se apoderan de mí.
\par 17 La noche taladra mis huesos,
\par Y los dolores que me roen no reposan.
\par 18 La violencia deforma mi vestidura; me ciñe como el cuello de mi túnica.
\par 19 El me derribó en el lodo,
\par Y soy semejante al polvo y a la ceniza.
\par 20 Clamo a ti, y no me oyes;
\par Me presento, y no me atiendes.
\par 21 Te has vuelto cruel para mí;
\par Con el poder de tu mano me persigues.
\par 22 Me alzaste sobre el viento, me hiciste cabalgar en él,
\par Y disolviste mi sustancia.
\par 23 Porque yo sé que me conduces a la muerte,
\par Y a la casa determinada a todo viviente.
\par 24 Mas él no extenderá la mano contra el sepulcro;
\par ¿Clamarán los sepultados cuando él los quebrantare?
\par 25 ¿No lloré yo al afligido?
\par Y mi alma, ¿no se entristeció sobre el menesteroso?
\par 26 Cuando esperaba yo el bien, entonces vino el mal;
\par Y cuando esperaba luz, vino la oscuridad.
\par 27 Mis entrañas se agitan, y no reposan;
\par Días de aflicción me han sobrecogido.
\par 28 Ando ennegrecido, y no por el sol;
\par Me he levantado en la congregación, y clamado.
\par 29 He venido a ser hermano de chacales,
\par Y compañero de avestruces.
\par 30 Mi piel se ha ennegrecido y se me cae,
\par Y mis huesos arden de calor.
\par 31 Se ha cambiado mi arpa en luto,
\par Y mi flauta en voz de lamentadores.

\chapter{31}

\section*{Job afirma su integridad}

\par 1 Hice pacto con mis ojos;
\par ¿Cómo, pues, había yo de mirar a una virgen?
\par 2 Porque ¿qué galardón me daría de arriba Dios,
\par Y qué heredad el Omnipotente desde las alturas?
\par 3 ¿No hay quebrantamiento para el impío,
\par Y extrañamiento para los que hacen iniquidad?
\par 4 ¿No ve él mis caminos,
\par Y cuenta todos mis pasos?
\par 5 Si anduve con mentira,
\par Y si mi pie se apresuró a engaño,
\par 6 Péseme Dios en balanzas de justicia,
\par Y conocerá mi integridad.
\par 7 Si mis pasos se apartaron del camino,
\par Si mi corazón se fue tras mis ojos,
\par Y si algo se pegó a mis manos,
\par 8 Siembre yo, y otro coma,
\par Y sea arrancada mi siembra.
\par 9 Si fue mi corazón engañado acerca de mujer,
\par Y si estuve acechando a la puerta de mi prójimo, 
\par 10 Muela para otro mi mujer,
\par Y sobre ella otros se encorven.
\par 11 Porque es maldad e iniquidad
\par Que han de castigar los jueces.
\par 12 Porque es fuego que devoraría hasta el Abadón,
\par Y consumiría toda mi hacienda.
\par 13 Si hubiera tenido en poco el derecho de mi siervo y de mi sierva,
\par Cuando ellos contendían conmigo,
\par 14 ¿Qué haría yo cuando Dios se levantase?
\par Y cuando él preguntara, ¿qué le respondería yo?
\par 15 El que en el vientre me hizo a mí, ¿no lo hizo a él?
\par ¿Y no nos dispuso uno mismo en la matriz?
\par 16 Si estorbé el contento de los pobres,
\par E hice desfallecer los ojos de la viuda;
\par 17 Si comí mi bocado solo,
\par Y no comió de él el huérfano
\par 18 (Porque desde mi juventud creció conmigo como con un padre,
\par Y desde el vientre de mi madre fui guía de la viuda);
\par 19 Si he visto que pereciera alguno sin vestido,
\par Y al menesteroso sin abrigo;
\par 20 Si no me bendijeron sus lomos,
\par Y del vellón de mis ovejas se calentaron;
\par 21 Si alcé contra el huérfano mi mano,
\par Aunque viese que me ayudaran en la puerta;
\par 22 Mi espalda se caiga de mi hombro,
\par Y el hueso de mi brazo sea quebrado.
\par 23 Porque temí el castigo de Dios,
\par Contra cuya majestad yo no tendría poder.
\par 24 Si puse en el oro mi esperanza,
\par Y dije al oro: Mi confianza eres tú;
\par 25 Si me alegré de que mis riquezas se multiplicasen,
\par Y de que mi mano hallase mucho;
\par 26 Si he mirado al sol cuando resplandecía,
\par O a la luna cuando iba hermosa,
\par 27 Y mi corazón se engañó en secreto,
\par Y mi boca besó mi mano;
\par 28 Esto también sería maldad juzgada;
\par Porque habría negado al Dios soberano.
\par 29 Si me alegré en el quebrantamiento del que me aborrecía,
\par Y me regocijé cuando le halló el mal
\par 30 (Ni aun entregué al pecado mi lengua,
\par Pidiendo maldición para su alma);
\par 31 Si mis siervos no decían:
\par ¿Quién no se ha saciado de su carne?
\par 32 (El forastero no pasaba fuera la noche;
\par Mis puertas abría al caminante);
\par 33 Si encubrí como hombre mis transgresiones,
\par Escondiendo en mi seno mi iniquidad,
\par 34 Porque tuve temor de la gran multitud,
\par Y el menosprecio de las familias me atemorizó,
\par Y callé, y no salí de mi puerta;
\par 35 ¡Quién me diera quien me oyese!
\par He aquí mi confianza es que el Omnipotente testificará por mí,
\par Aunque mi adversario me forme proceso.
\par 36 Ciertamente yo lo llevaría sobre mi hombro,
\par Y me lo ceñiría como una corona.
\par 37 Yo le contaría el número de mis pasos,
\par Y como príncipe me presentaría ante él.
\par 38 Si mi tierra clama contra mí,
\par Y lloran todos sus surcos;
\par 39 Si comí su sustancia sin dinero,
\par O afligí el alma de sus dueños,
\par 40 En lugar de trigo me nazcan abrojos,
\par Y espinos en lugar de cebada.
\par Aquí terminan las palabras de Job.

\chapter{32}

\section*{Eliú justifica su derecho de contestar a Job}

\par 1 Cesaron estos tres varones de responder a Job, por cuanto él era justo a sus propios ojos.
\par 2 Entonces Eliú hijo de Baraquel buzita, de la familia de Ram, se encendió en ira contra Job; se encendió en ira, por cuanto se justificaba a sí mismo más que a Dios.
\par 3 Asimismo se encendió en ira contra sus tres amigos, porque no hallaban qué responder, aunque habían condenado a Job.
\par 4 Y Eliú había esperado a Job en la disputa, porque los otros eran más viejos que él.
\par 5 Pero viendo Eliú que no había respuesta en la boca de aquellos tres varones, se encendió en ira.
\par 6 Y respondió Eliú hijo de Baraquel buzita, y dijo:
\par Yo soy joven, y vosotros ancianos;
\par Por tanto, he tenido miedo, y he temido declararos mi opinión.
\par 7 Yo decía: Los días hablarán,
\par Y la muchedumbre de años declarará sabiduría.
\par 8 Ciertamente espíritu hay en el hombre,
\par Y el soplo del Omnipotente le hace que entienda.
\par 9 No son los sabios los de mucha edad,
\par Ni los ancianos entienden el derecho.
\par 10 Por tanto, yo dije: Escuchadme;
\par Declararé yo también mi sabiduría.
\par 11 He aquí yo he esperado a vuestras razones,
\par He escuchado vuestros argumentos,
\par En tanto que buscabais palabras.
\par 12 Os he prestado atención,
\par Y he aquí que no hay de vosotros quien redarguya a Job,
\par Y responda a sus razones.
\par 13 Para que no digáis: Nosotros hemos hallado sabiduría;
\par Lo vence Dios, no el hombre.
\par 14 Ahora bien, Job no dirigió contra mí sus palabras,
\par Ni yo le responderé con vuestras razones.
\par 15 Se espantaron, no respondieron más;
\par Se les fueron los razonamientos.
\par 16 Yo, pues, he esperado, pero no hablaban;
\par Más bien callaron y no respondieron más.
\par 17 Por eso yo también responderé mi parte;
\par También yo declararé mi juicio. 
\par 18 Porque lleno estoy de palabras, 
\par Y me apremia el espíritu dentro de mí.
\par 19 De cierto mi corazón está como el vino que no tiene respiradero,
\par Y se rompe como odres nuevos.
\par 20 Hablaré, pues, y respiraré;
\par Abriré mis labios, y responderé.
\par 21 No haré ahora acepción de personas,
\par Ni usaré con nadie de títulos lisonjeros.
\par 22 Porque no sé hablar lisonjas;
\par De otra manera, en breve mi Hacedor me consumiría.

\chapter{33}

\section*{Eliú censura a Job}

\par 1 Por tanto, Job, oye ahora mis razones,
\par Y escucha todas mis palabras.
\par 2 He aquí yo abriré ahora mi boca,
\par Y mi lengua hablará en mi garganta.
\par 3 Mis razones declararán la rectitud de mi corazón,
\par Y lo que saben mis labios, lo hablarán con sinceridad.
\par 4 El espíritu de Dios me hizo,
\par Y el soplo del Omnipotente me dio vida.
\par 5 Respóndeme si puedes;
\par Ordena tus palabras, ponte en pie.
\par 6 Heme aquí a mí en lugar de Dios, conforme a tu dicho;
\par De barro fui yo también formado.
\par 7 He aquí, mi terror no te espantará,
\par Ni mi mano se agravará sobre ti.
\par 8 De cierto tú dijiste a oídos míos,
\par Y yo oí la voz de tus palabras que decían: 
\par 9 Yo soy limpio y sin defecto;
\par Soy inocente, y no hay maldad en mí. 
\par 10 He aquí que él buscó reproches contra mí,
\par Y me tiene por su enemigo;
\par 11 Puso mis pies en el cepo,
\par Y vigiló todas mis sendas.
\par 12 He aquí, en esto no has hablado justamente;
\par Yo te responderé que mayor es Dios que el hombre.
\par 13 ¿Por qué contiendes contra él?
\par Porque él no da cuenta de ninguna de sus razones.
\par 14 Sin embargo, en una o en dos maneras habla Dios;
\par Pero el hombre no entiende.
\par 15 Por sueño, en visión nocturna,
\par Cuando el sueño cae sobre los hombres, 
\par Cuando se adormecen sobre el lecho, 
\par 16 Entonces revela al oído de los hombres,
\par Y les señala su consejo,
\par 17 Para quitar al hombre de su obra,
\par Y apartar del varón la soberbia.
\par 18 Detendrá su alma del sepulcro,
\par Y su vida de que perezca a espada.
\par 19 También sobre su cama es castigado
\par Con dolor fuerte en todos sus huesos,
\par 20 Que le hace que su vida aborrezca el pan,
\par Y su alma la comida suave.
\par 21 Su carne desfallece, de manera que no se ve,
\par Y sus huesos, que antes no se veían, aparecen.
\par 22 Su alma se acerca al sepulcro,
\par Y su vida a los que causan la muerte.
\par 23 Si tuviese cerca de él
\par Algún elocuente mediador muy escogido,
\par Que anuncie al hombre su deber;
\par 24 Que le diga que Dios tuvo de él misericordia,
\par Que lo libró de descender al sepulcro,
\par Que halló redención;
\par 25 Su carne será más tierna que la del niño,
\par Volverá a los días de su juventud.
\par 26 Orará a Dios, y éste le amará,
\par Y verá su faz con júbilo;
\par Y restaurará al hombre su justicia.
\par 27 El mira sobre los hombres; y al que dijere:
\par Pequé, y pervertí lo recto,
\par Y no me ha aprovechado,
\par 28 Dios redimirá su alma para que no pase al sepulcro,
\par Y su vida se verá en luz.
\par 29 He aquí, todas estas cosas hace Dios
\par Dos y tres veces con el hombre,
\par 30 Para apartar su alma del sepulcro,
\par Y para iluminarlo con la luz de los vivientes.
\par 31 Escucha, Job, y óyeme;
\par Calla, y yo hablaré.
\par 32 Si tienes razones, respóndeme;
\par Habla, porque yo te quiero justificar.
\par 33 Y si no, óyeme tú a mí;
\par Calla, y te enseñaré sabiduría. 

\chapter{34}

\section*{Eliú justifica a Dios}

\par 1 Además Eliú dijo:
\par 2 Oíd, sabios, mis palabras;
\par Y vosotros, doctos, estadme atentos.
\par 3 Porque el oído prueba las palabras,
\par Como el paladar gusta lo que uno come.
\par 4 Escojamos para nosotros el juicio,
\par Conozcamos entre nosotros cuál sea lo bueno.
\par 5 Porque Job ha dicho: Yo soy justo,
\par Y Dios me ha quitado mi derecho.
\par 6 ¿He de mentir yo contra mi razón?
\par Dolorosa es mi herida sin haber hecho yo transgresión.
\par 7 ¿Qué hombre hay como Job,
\par Que bebe el escarnio como agua,
\par 8 Y va en compañía con los que hacen iniquidad,
\par Y anda con los hombres malos?
\par 9 Porque ha dicho: De nada servirá al hombre
\par El conformar su voluntad a Dios.
\par 10 Por tanto, varones de inteligencia, oídme:
\par Lejos esté de Dios la impiedad,
\par Y del Omnipotente la iniquidad.
\par 11 Porque él pagará al hombre según su obra,
\par Y le retribuirá conforme a su camino. 
\par 12 Sí, por cierto, Dios no hará injusticia,
\par Y el Omnipotente no pervertirá el derecho.
\par 13 ¿Quién visitó por él la tierra?
\par ¿Y quién puso en orden todo el mundo?
\par 14 Si él pusiese sobre el hombre su corazón,
\par Y recogiese así su espíritu y su aliento,
\par 15 Toda carne perecería juntamente,
\par Y el hombre volvería al polvo.
\par 16 Si, pues, hay en ti entendimiento, oye esto;
\par Escucha la voz de mis palabras.
\par 17 ¿Gobernará el que aborrece juicio?
\par ¿Y condenarás tú al que es tan justo?
\par 18 ¿Se dirá al rey: Perverso;
\par Y a los príncipes: Impíos?
\par 19 ¿Cuánto menos a aquel que no hace acepción de personas de príncipes.
\par Ni respeta más al rico que al pobre,
\par Porque todos son obra de sus manos?
\par 20 En un momento morirán,
\par Y a medianoche se alborotarán los pueblos, y pasarán,
\par Y sin mano será quitado el poderoso.
\par 21 Porque sus ojos están sobre los caminos del hombre,
\par Y ve todos sus pasos.
\par 22 No hay tinieblas ni sombra de muerte
\par Donde se escondan los que hacen maldad.
\par 23 No carga, pues, él al hombre más de lo justo,
\par Para que vaya con Dios a juicio. 
\par 24 El quebrantará a los fuertes sin indagación,
\par Y hará estar a otros en su lugar.
\par 25 Por tanto, él hará notorias las obras de ellos,
\par Cuando los trastorne en la noche, y sean quebrantados.
\par 26 Como a malos los herirá
\par En lugar donde sean vistos;
\par 27 Por cuanto así se apartaron de él,
\par Y no consideraron ninguno de sus caminos,
\par 28 Haciendo venir delante de él el clamor del pobre,
\par Y que oiga el clamor de los necesitados.
\par 29 Si él diere reposo, ¿quién inquietará?
\par Si escondiere el rostro, ¿quién lo mirará?
\par Esto sobre una nación, y lo mismo sobre un hombre;
\par 30 Haciendo que no reine el hombre impío
\par Para vejaciones del pueblo.
\par 31 De seguro conviene que se diga a Dios:
\par He llevado ya castigo, no ofenderé ya más;
\par 32 Enséñame tú lo que yo no veo;
\par Si hice mal, no lo haré más.
\par 33 ¿Ha de ser eso según tu parecer?
\par El te retribuirá, ora rehúses, ora aceptes, y no yo;
\par Di, si no, lo que tú sabes.
\par 34 Los hombres inteligentes dirán conmigo,
\par Y el hombre sabio que me oiga: 
\par 35 Que Job no habla con sabiduría,
\par Y que sus palabras no son con entendimiento.
\par 36 Deseo yo que Job sea probado ampliamente,
\par A causa de sus respuestas semejantes a las de los hombres inicuos.
\par 37 Porque a su pecado añadió rebeldía;
\par Bate palmas contra nosotros,
\par Y contra Dios multiplica sus palabras.

\chapter{35}

\par 1 Prosiguió Eliú en su razonamiento, y dijo:
\par 2 ¿Piensas que es cosa recta lo que has dicho:
\par Más justo soy yo que Dios?
\par 3 Porque dijiste: ¿Qué ventaja sacaré de ello?
\par ¿O qué provecho tendré de no haber pecado?
\par 4 Yo te responderé razones, 
\par Y a tus compañeros contigo.
\par 5 Mira a los cielos, y ve, 
\par Y considera que las nubes son más altas que tú. 
\par 6 Si pecares, ¿qué habrás logrado contra él?
\par Y si tus rebeliones se multiplicaren, ¿qué le harás tú?
\par 7 Si fueres justo, ¿qué le darás a él?
\par ¿O qué recibirá de tu mano?
\par 8 Al hombre como tú dañará tu impiedad,
\par Y al hijo de hombre aprovechará tu justicia. 
\par 9 A causa de la multitud de las violencias claman,
\par Y se lamentan por el poderío de los grandes.
\par 10 Y ninguno dice: ¿Dónde está Dios mi Hacedor,
\par Que da cánticos en la noche,
\par 11 Que nos enseña más que a las bestias de la tierra,
\par Y nos hace sabios más que a las aves del cielo?
\par 12 Allí clamarán, y él no oirá,
\par Por la soberbia de los malos.
\par 13 Ciertamente Dios no oirá la vanidad,
\par Ni la mirará el Omnipotente.
\par 14 ¿Cuánto menos cuando dices que no haces caso de él?
\par La causa está delante de él; por tanto, aguárdale.
\par 15 Mas ahora, porque en su ira no castiga,
\par Ni inquiere con rigor,
\par 16 Por eso Job abre su boca vanamente,
\par Y multiplica palabras sin sabiduría.

\chapter{36}

\section*{Eliú exalta la grandeza de Dios}

\par 1 Añadió Eliú y dijo:
\par 2 Espérame un poco, y te enseñaré;
\par Porque todavía tengo razones en defensa de Dios. 
\par 3 Tomaré mi saber desde lejos,
\par Y atribuiré justicia a mi Hacedor.
\par 4 Porque de cierto no son mentira mis palabras;
\par Contigo está el que es íntegro en sus conceptos.
\par 5 He aquí que Dios es grande, pero no desestima a nadie;
\par Es poderoso en fuerza de sabiduría.
\par 6 No otorgará vida al impío,
\par Pero a los afligidos dará su derecho.
\par 7 No apartará de los justos sus ojos;
\par Antes bien con los reyes los pondrá en trono para siempre,
\par Y serán exaltados.
\par 8 Y si estuvieren prendidos en grillos,
\par Y aprisionados en las cuerdas de aflicción,
\par 9 El les dará a conocer la obra de ellos,
\par Y que prevalecieron sus rebeliones. 
\par 10 Despierta además el oído de ellos para la corrección,
\par Y les dice que se conviertan de la iniquidad.
\par 11 Si oyeren, y le sirvieren,
\par Acabarán sus días en bienestar,
\par Y sus años en dicha.
\par 12 Pero si no oyeren, serán pasados a espada,
\par Y perecerán sin sabiduría.
\par 13 Mas los hipócritas de corazón atesoran para sí la ira,
\par Y no clamarán cuando él los atare.
\par 14 Fallecerá el alma de ellos en su juventud,
\par Y su vida entre los sodomitas.
\par 15 Al pobre librará de su pobreza,
\par Y en la aflicción despertará su oído.
\par 16 Asimismo te apartará de la boca de la angustia
\par A lugar espacioso, libre de todo apuro,
\par Y te preparará mesa llena de grosura.
\par 17 Mas tú has llenado el juicio del impío,
\par En vez de sustentar el juicio y la justicia.
\par 18 Por lo cual teme, no sea que en su ira te quite con golpe,
\par El cual no puedas apartar de ti con gran rescate. 
\par 19 ¿Hará él estima de tus riquezas, del oro,
\par O de todas las fuerzas del poder?
\par 20 No anheles la noche,
\par En que los pueblos desaparecen de su lugar.
\par 21 Guárdate, no te vuelvas a la iniquidad;
\par Pues ésta escogiste más bien que la aflicción.
\par 22 He aquí que Dios es excelso en su poder;
\par ¿Qué enseñador semejante a él?
\par 23 ¿Quién le ha prescrito su camino?
\par ¿Y quién le dirá: Has hecho mal?
\par 24 Acuérdate de engrandecer su obra,
\par La cual contemplan los hombres.
\par 25 Los hombres todos la ven;
\par La mira el hombre de lejos.
\par 26 He aquí, Dios es grande, y nosotros no le conocemos,
\par Ni se puede seguir la huella de sus años.
\par 27 El atrae las gotas de las aguas,
\par Al transformarse el vapor en lluvia,
\par 28 La cual destilan las nubes,
\par Goteando en abundancia sobre los hombres.
\par 29 ¿Quién podrá comprender la extensión de las nubes,
\par Y el sonido estrepitoso de su morada?
\par 30 He aquí que sobre él extiende su luz,
\par Y cobija con ella las profundidades del mar.
\par 31 Bien que por esos medios castiga a los pueblos,
\par A la multitud él da sustento.
\par 32 Con las nubes encubre la luz,
\par Y le manda no brillar, interponiendo aquéllas.
\par 33 El trueno declara su indignación,
\par Y la tempestad proclama su ira contra la iniquidad.

\chapter{37}

\par 1 Por eso también se estremece mi corazón,
\par Y salta de su lugar.
\par 2 Oíd atentamente el estrépito de su voz,
\par Y el sonido que sale de su boca.
\par 3 Debajo de todos los cielos lo dirige,
\par Y su luz hasta los fines de la tierra.
\par 4 Después de ella brama el sonido,
\par Truena él con voz majestuosa;
\par Y aunque sea oída su voz, no los detiene.
\par 5 Truena Dios maravillosamente con su voz;
\par El hace grandes cosas, que nosotros no entendemos.
\par 6 Porque a la nieve dice: Desciende a la tierra;
\par También a la llovizna, y a los aguaceros torrenciales.
\par 7 Así hace retirarse a todo hombre,
\par Para que los hombres todos reconozcan su obra.
\par 8 Las bestias entran en su escondrijo,
\par Y se están en sus moradas.
\par 9 Del sur viene el torbellino,
\par Y el frío de los vientos del norte.
\par 10 Por el soplo de Dios se da el hielo,
\par Y las anchas aguas se congelan.
\par 11 Regando también llega a disipar la densa nube,
\par Y con su luz esparce la niebla.
\par 12 Asimismo por sus designios se revuelven las nubes en derredor,
\par Para hacer sobre la faz del mundo,
\par En la tierra, lo que él les mande.
\par 13 Unas veces por azote, otras por causa de su tierra,
\par Otras por misericordia las hará venir.
\par 14 Escucha esto, Job;
\par Detente, y considera las maravillas de Dios.
\par 15 ¿Sabes tú cómo Dios las pone en concierto,
\par Y hace resplandecer la luz de su nube?
\par 16 ¿Has conocido tú las diferencias de las nubes,
\par Las maravillas del Perfecto en sabiduría?
\par 17 ¿Por qué están calientes tus vestidos
\par Cuando él sosiega la tierra con el viento del sur?
\par 18 ¿Extendiste tú con él los cielos,
\par Firmes como un espejo fundido? 
\par 19 Muéstranos qué le hemos de decir; 
\par Porque nosotros no podemos ordenar las ideas a causa de las tinieblas.
\par 20 ¿Será preciso contarle cuando yo hablare?
\par Por más que el hombre razone, quedará como abismado.
\par 21 Mas ahora ya no se puede mirar la luz esplendente en los cielos,
\par Luego que pasa el viento y los limpia,
\par 22 Viniendo de la parte del norte la dorada claridad.
\par En Dios hay una majestad terrible.
\par 23 El es Todopoderoso, al cual no alcanzamos, grande en poder;
\par Y en juicio y en multitud de justicia no afligirá.
\par 24 Lo temerán por tanto los hombres;
\par El no estima a ninguno que cree en su propio corazón ser sabio.

\chapter{38}

\section*{Jehová convence a Job de su ignorancia}

\par 1 Entonces respondió Jehová a Job desde un torbellino, y dijo:
\par 2 ¿Quién es ése que oscurece el consejo
\par Con palabras sin sabiduría?
\par 3 Ahora ciñe como varón tus lomos;
\par Yo te preguntaré, y tú me contestarás.
\par 4 ¿Dónde estabas tú cuando yo fundaba la tierra?
\par Házmelo saber, si tienes inteligencia.
\par 5 ¿Quién ordenó sus medidas, si lo sabes?
\par ¿O quién extendió sobre ella cordel?
\par 6 ¿Sobre qué están fundadas sus bases?
\par ¿O quién puso su piedra angular,
\par 7 Cuando alababan todas las estrellas del alba,
\par Y se regocijaban todos los hijos de Dios?
\par 8 ¿Quién encerró con puertas el mar,
\par Cuando se derramaba saliéndose de su seno, 
\par 9 Cuando puse yo nubes por vestidura suya,
\par Y por su faja oscuridad,
\par 10 Y establecí sobre él mi decreto,
\par Le puse puertas y cerrojo,
\par 11 Y dije: Hasta aquí llegarás, y no pasarás adelante,
\par Y ahí parará el orgullo de tus olas? 
\par 12 ¿Has mandado tú a la mañana en tus días?
\par ¿Has mostrado al alba su lugar,
\par 13 Para que ocupe los fines de la tierra,
\par Y para que sean sacudidos de ella los impíos?
\par 14 Ella muda luego de aspecto como barro bajo el sello,
\par Y viene a estar como con vestidura;
\par 15 Mas la luz de los impíos es quitada de ellos,
\par Y el brazo enaltecido es quebrantado.
\par 16 ¿Has entrado tú hasta las fuentes del mar,
\par Y has andado escudriñando el abismo?
\par 17 ¿Te han sido descubiertas las puertas de la muerte,
\par Y has visto las puertas de la sombra de muerte?
\par 18 ¿Has considerado tú hasta las anchuras de la tierra?
\par Declara si sabes todo esto.
\par 19 ¿Por dónde va el camino a la habitación de la luz,
\par Y dónde está el lugar de las tinieblas,
\par 20 Para que las lleves a sus límites,
\par Y entiendas las sendas de su casa?
\par 21 ¡Tú lo sabes! Pues entonces ya habías nacido,
\par Y es grande el número de tus días.
\par 22 ¿Has entrado tú en los tesoros de la nieve,
\par O has visto los tesoros del granizo,
\par 23 Que tengo reservados para el tiempo de angustia,
\par Para el día de la guerra y de la batalla?
\par 24 ¿Por qué camino se reparte la luz,
\par Y se esparce el viento solano sobre la tierra? 
\par 25 ¿Quién repartió conducto al turbión,
\par Y camino a los relámpagos y truenos,
\par 26 Haciendo llover sobre la tierra deshabitada, 
\par Sobre el desierto, donde no hay hombre,
\par 27 Para saciar la tierra desierta e inculta,
\par Y para hacer brotar la tierna hierba?
\par 28 ¿Tiene la lluvia padre?
\par ¿O quién engendró las gotas del rocío?
\par 29 ¿De qué vientre salió el hielo?
\par Y la escarcha del cielo, ¿quién la engendró?
\par 30 Las aguas se endurecen a manera de piedra,
\par Y se congela la faz del abismo.
\par 31 ¿Podrás tú atar los lazos de las Pléyades,
\par O desatarás las ligaduras de Orión? 
\par 32 ¿Sacarás tú a su tiempo las constelaciones de los cielos,
\par O guiarás a la Osa Mayor con sus hijos?
\par 33 ¿Supiste tú las ordenanzas de los cielos?
\par ¿Dispondrás tú de su potestad en la tierra?
\par 34 ¿Alzarás tú a las nubes tu voz,
\par Para que te cubra muchedumbre de aguas?
\par 35 ¿Enviarás tú los relámpagos, para que ellos vayan?
\par ¿Y te dirán ellos: Henos aquí?
\par 36 ¿Quién puso la sabiduría en el corazón?
\par ¿O quién dio al espíritu inteligencia?
\par 37 ¿Quién puso por cuenta los cielos con sabiduría?
\par Y los odres de los cielos, ¿quién los hace inclinar,
\par 38 Cuando el polvo se ha convertido en dureza,
\par Y los terrones se han pegado unos con otros?
\par 39 ¿Cazarás tú la presa para el león?
\par ¿Saciarás el hambre de los leoncillos,
\par 40 Cuando están echados en las cuevas,
\par O se están en sus guaridas para acechar?
\par 41 ¿Quién prepara al cuervo su alimento,
\par Cuando sus polluelos claman a Dios,
\par Y andan errantes por falta de comida?

\chapter{39}

\par 1 ¿Sabes tú el tiempo en que paren las cabras monteses?
\par ¿O miraste tú las ciervas cuando están pariendo? 
\par 2 ¿Contaste tú los meses de su preñez,
\par Y sabes el tiempo cuando han de parir?
\par 3 Se encorvan, hacen salir sus hijos,
\par Pasan sus dolores.
\par 4 Sus hijos se fortalecen, crecen con el pasto;
\par Salen, y no vuelven a ellas. 
\par 5 ¿Quién echó libre al asno montés,
\par Y quién soltó sus ataduras?
\par 6 Al cual yo puse casa en la soledad,
\par Y sus moradas en lugares estériles.
\par 7 Se burla de la multitud de la ciudad;
\par No oye las voces del arriero.
\par 8 Lo oculto de los montes es su pasto,
\par Y anda buscando toda cosa verde.
\par 9 ¿Querrá el búfalo servirte a ti,
\par O quedar en tu pesebre?
\par 10 ¿Atarás tú al búfalo con coyunda para el surco?
\par ¿Labrará los valles en pos de ti?
\par 11 ¿Confiarás tú en él, por ser grande su fuerza,
\par Y le fiarás tu labor? 
\par 12 ¿Fiarás de él para que recoja tu semilla,
\par Y la junte en tu era?
\par 13 ¿Diste tú hermosas alas al pavo real,
\par o alas y plumas al avestruz?
\par 14 El cual desampara en la tierra sus huevos,
\par Y sobre el polvo los calienta,
\par 15 Y olvida que el pie los puede pisar,
\par Y que puede quebrarlos la bestia del campo.
\par 16 Se endurece para con sus hijos, como si no fuesen suyos,
\par No temiendo que su trabajo haya sido en vano;
\par 17 Porque le privó Dios de sabiduría,
\par Y no le dio inteligencia.
\par 18 Luego que se levanta en alto, 
\par Se burla del caballo y de su jinete.
\par 19 ¿Diste tú al caballo la fuerza?
\par ¿Vestiste tú su cuello de crines ondulantes? 
\par 20 ¿Le intimidarás tú como a langosta?
\par El resoplido de su nariz es formidable.
\par 21 Escarba la tierra, se alegra en su fuerza,
\par Sale al encuentro de las armas;
\par 22 Hace burla del espanto, y no teme,
\par Ni vuelve el rostro delante de la espada.
\par 23 Contra él suenan la aljaba,
\par El hierro de la lanza y de la jabalina;
\par 24 Y él con ímpetu y furor escarba la tierra,
\par Sin importarle el sonido de la trompeta;
\par 25 Antes como que dice entre los clarines: ¡Ea!
\par Y desde lejos huele la batalla,
\par El grito de los capitanes, y el vocerío.
\par 26 ¿Vuela el gavilán por tu sabiduría,
\par Y extiende hacia el sur sus alas?
\par 27 ¿Se remonta el águila por tu mandamiento,
\par Y pone en alto su nido?
\par 28 Ella habita y mora en la peña,
\par En la cumbre del peñasco y de la roca.
\par 29 Desde allí acecha la presa;
\par Sus ojos observan de muy lejos.
\par 30 Sus polluelos chupan la sangre;
\par Y donde hubiere cadáveres, allí está ella.

\chapter{40}

\par 1 Además respondió Jehová a Job, y dijo:
\par 2 ¿Es sabiduría contender con el Omnipotente? 
\par El que disputa con Dios, responda a esto.
\par 3 Entonces respondió Job a Jehová, y dijo:
\par 4 He aquí que yo soy vil; ¿qué te responderé? 
\par Mi mano pongo sobre mi boca. 
\par 5 Una vez hablé, mas no responderé; Aun dos veces, mas no volveré a hablar.
\par Manifestaciones del poder de Dios 
\par 6 Respondió Jehová a Job desde el torbellino, y dijo:
\par 7 Cíñete ahora como varón tus lomos;
\par Yo te preguntaré, y tú me responderás.
\par 8 ¿Invalidarás tú también mi juicio?
\par ¿Me condenarás a mí, para justificarte tú?
\par 9 ¿Tienes tú un brazo como el de Dios?
\par ¿Y truenas con voz como la suya?
\par 10 Adórnate ahora de majestad y de alteza,
\par Y vístete de honra y de hermosura.
\par 11 Derrama el ardor de tu ira;
\par Mira a todo altivo, y abátelo.
\par 12 Mira a todo soberbio, y humíllalo,
\par Y quebranta a los impíos en su sitio.
\par 13 Encúbrelos a todos en el polvo,
\par Encierra sus rostros en la oscuridad;
\par 14 Y yo también te confesaré
\par Que podrá salvarte tu diestra.
\par 15 He aquí ahora behemot, el cual hice como a ti;
\par Hierba come como buey.
\par 16 He aquí ahora que su fuerza está en sus lomos,
\par Y su vigor en los músculos de su vientre.
\par 17 Su cola mueve como un cedro,
\par Y los nervios de sus muslos están entretejidos.
\par 18 Sus huesos son fuertes como bronce,
\par Y sus miembros como barras de hierro. 
\par 19 El es el principio de los caminos de Dios;
\par El que lo hizo, puede hacer que su espada a él se acerque. 
\par 20 Ciertamente los montes producen hierba para él;
\par Y toda bestia del campo retoza allá.
\par 21 Se echará debajo de las sombras,
\par En lo oculto de las cañas y de los lugares húmedos.
\par 22 Los árboles sombríos lo cubren con su sombra;
\par Los sauces del arroyo lo rodean. 
\par 23 He aquí, sale de madre el río, pero él no se inmuta;
\par Tranquilo está, aunque todo un Jordán se estrelle contra su boca. 
\par 24 ¿Lo tomará alguno cuando está vigilante,
\par Y horadará su nariz?

\chapter{41}

\par 1 ¿Sacarás tú al leviatán con anzuelo,
\par O con cuerda que le eches en su lengua?
\par 2 ¿Pondrás tú soga en sus narices,
\par Y horadarás con garfio su quijada?
\par 3 ¿Multiplicará él ruegos para contigo?
\par ¿Te hablará él lisonjas?
\par 4 ¿Hará pacto contigo
\par Para que lo tomes por siervo perpetuo?
\par 5 ¿Jugarás con él como con pájaro,
\par O lo atarás para tus niñas?
\par 6 ¿Harán de él banquete los compañeros?
\par ¿Lo repartirán entre los mercaderes?
\par 7 ¿Cortarás tú con cuchillo su piel,
\par O con arpón de pescadores su cabeza?
\par 8 Pon tu mano sobre él;
\par Te acordarás de la batalla, y nunca más volverás.
\par 9 He aquí que la esperanza acerca de él será burlada,
\par Porque aun a su sola vista se desmayarán.
\par 10 Nadie hay tan osado que lo despierte;
\par ¿Quién, pues, podrá estar delante de mí?
\par 11 ¿Quién me ha dado a mí primero, para que yo restituya? 
\par Todo lo que hay debajo del cielo es mío.
\par 12 No guardaré silencio sobre sus miembros,
\par Ni sobre sus fuerzas y la gracia de su disposición.
\par 13 ¿Quién descubrirá la delantera de su vestidura?
\par ¿Quién se acercará a él con su freno doble?
\par 14 ¿Quién abrirá las puertas de su rostro?
\par Las hileras de sus dientes espantan.
\par 15 La gloria de su vestido son escudos fuertes,
\par Cerrados entre sí estrechamente. 
\par 16 El uno se junta con el otro,
\par Que viento no entra entre ellos.
\par 17 Pegado está el uno con el otro;
\par Están trabados entre sí, que no se pueden apartar.
\par 18 Con sus estornudos enciende lumbre,
\par Y sus ojos son como los párpados del alba.
\par 19 De su boca salen hachones de fuego;
\par Centellas de fuego proceden.
\par 20 De sus narices sale humo,
\par Como de una olla o caldero que hierve. 
\par 21 Su aliento enciende los carbones,
\par Y de su boca sale llama.
\par 22 En su cerviz está la fuerza,
\par Y delante de él se esparce el desaliento.
\par 23 Las partes más flojas de su carne están endurecidas;
\par Están en él firmes, y no se mueven.
\par 24 Su corazón es firme como una piedra,
\par Y fuerte como la muela de abajo.
\par 25 De su grandeza tienen temor los fuertes,
\par Y a causa de su desfallecimiento hacen por purificarse. 
\par 26 Cuando alguno lo alcanzare,
\par Ni espada, ni lanza, ni dardo, ni coselete durará.
\par 27 Estima como paja el hierro,
\par Y el bronce como leño podrido.
\par 28 Saeta no le hace huir;
\par Las piedras de honda le son como paja.
\par 29 Tiene toda arma por hojarasca,
\par Y del blandir de la jabalina se burla.
\par 30 Por debajo tiene agudas conchas;
\par Imprime su agudez en el suelo.
\par 31 Hace hervir como una olla el mar profundo,
\par Y lo vuelve como una olla de ungüento.
\par 32 En pos de sí hace resplandecer la senda, 
\par Que parece que el abismo es cano.
\par 33 No hay sobre la tierra quien se le parezca;
\par Animal hecho exento de temor.
\par 34 Menosprecia toda cosa alta;
\par Es rey sobre todos los soberbios.

\chapter{42}

\section*{Confesión y justificación de Job}

\par 1 Respondió Job a Jehová, y dijo:
\par 2 Yo conozco que todo lo puedes,
\par Y que no hay pensamiento que se esconda de ti.
\par 3 ¿Quién es el que oscurece el consejo sin entendimiento?
\par Por tanto, yo hablaba lo que no entendía;
\par Cosas demasiado maravillosas para mí, que yo no comprendía.
\par 4 Oye, te ruego, y hablaré;
\par Te preguntaré, y tú me enseñarás. 
\par 5 De oídas te había oído;
\par Mas ahora mis ojos te ven.
\par 6 Por tanto me aborrezco,
\par Y me arrepiento en polvo y ceniza.
\par 7 Y aconteció que después que habló Jehová estas palabras a Job, Jehová dijo a Elifaz temanita: Mi ira se encendió contra ti y tus dos compañeros; porque no habéis hablado de mí lo recto, como mi siervo Job.
\par 8 Ahora, pues, tomaos siete becerros y siete carneros, e id a mi siervo Job, y ofreced holocausto por vosotros, y mi siervo Job orará por vosotros; porque de cierto a él atenderé para no trataros afrentosamente, por cuanto no habéis hablado de mí con rectitud, como mi siervo Job.
\par 9 Fueron, pues, Elifaz temanita, Bildad suhita y Zofar naamatita, e hicieron como Jehová les dijo; y Jehová aceptó la oración de Job.

\section*{Restauración de la prosperidad de Job}

\par 10 Y quitó Jehová la aflicción de Job, cuando él hubo orado por sus amigos; y aumentó al doble todas las cosas que habían sido de Job. 
\par 11 Y vinieron a él todos sus hermanos y todas sus hermanas, y todos los que antes le habían conocido, y comieron con él pan en su casa, y se condolieron de él, y le consolaron de todo aquel mal que Jehová había traído sobre él; y cada uno de ellos le dio una pieza de dinero y un anillo de oro.
\par 12 Y bendijo Jehová el postrer estado de Job más que el primero; porque tuvo catorce mil ovejas, seis mil camellos, mil yuntas de bueyes y mil asnas,
\par 13 y tuvo siete hijos y tres hijas.
\par 14 Llamó el nombre de la primera, Jemima, el de la segunda, Cesia, y el de la tercera, Keren-hapuc.
\par 15 Y no había mujeres tan hermosas como las hijas de Job en toda la tierra; y les dio su padre herencia entre sus hermanos.
\par 16 Después de esto vivió Job ciento cuarenta años, y vio a sus hijos, y a los hijos de sus hijos, hasta la cuarta generación.
\par 17 Y murió Job viejo y lleno de días.

\end{document}