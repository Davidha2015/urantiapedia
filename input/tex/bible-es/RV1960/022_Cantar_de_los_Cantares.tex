\begin{document}

\title{Cantar de los Cantares}

\chapter{1}

\section*{La esposa y las hijas de Jerusalén}

\par 1 Cantar de los cantares, el cual es de Salomón. 
\par 2 ¡Oh, si él me besara con besos de su boca!
\par Porque mejores son tus amores que el vino.
\par 3 A más del olor de tus suaves ungüentos,
\par Tu nombre es como ungüento derramado;
\par Por eso las doncellas te aman.
\par 4 Atráeme; en pos de ti correremos.
\par El rey me ha metido en sus cámaras;
\par Nos gozaremos y alegraremos en ti;
\par Nos acordaremos de tus amores más que del vino;
\par Con razón te aman.
\par 5 Morena soy, oh hijas de Jerusalén, pero codiciable
\par Como las tiendas de Cedar,
\par Como las cortinas de Salomón.
\par 6 No reparéis en que soy morena,
\par Porque el sol me miró.
\par Los hijos de mi madre se airaron contra mí;
\par Me pusieron a guardar las viñas;
\par Y mi viña, que era mía, no guardé.
\par 7 Hazme saber, oh tú a quien ama mi alma,
\par Dónde apacientas, dónde sesteas al mediodía;
\par Pues ¿por qué había de estar yo como errante
\par Junto a los rebaños de tus compañeros?
\par 8 Si tú no lo sabes, oh hermosa entre las mujeres,
\par Ve, sigue las huellas del rebaño,
\par Y apacienta tus cabritas junto a las cabañas de los pastores. 
\par La esposa y el esposo
\par 9 A yegua de los carros de Faraón
\par Te he comparado, amiga mía.
\par 10 Hermosas son tus mejillas entre los pendientes,
\par Tu cuello entre los collares.
\par 11 Zarcillos de oro te haremos,
\par Tachonados de plata.
\par 12 Mientras el rey estaba en su reclinatorio,
\par Mi nardo dio su olor.
\par 13 Mi amado es para mí un manojito de mirra,
\par Que reposa entre mis pechos.
\par 14 Racimo de flores de alheña en las viñas de En-gadi
\par Es para mí mi amado.
\par 15 He aquí que tú eres hermosa, amiga mía; 
\par He aquí eres bella; tus ojos son como palomas.
\par 16 He aquí que tú eres hermoso, amado mío, y dulce;
\par Nuestro lecho es de flores.
\par 17 Las vigas de nuestra casa son de cedro,
\par Y de ciprés los artesonados.

\chapter{2}

\par 1 Yo soy la rosa de Sarón,
\par Y el lirio de los valles.
\par 2 Como el lirio entre los espinos,
\par Así es mi amiga entre las doncellas.
\par 3 Como el manzano entre los árboles silvestres,
\par Así es mi amado entre los jóvenes;
\par Bajo la sombra del deseado me senté,
\par Y su fruto fue dulce a mi paladar.
\par 4 Me llevó a la casa del banquete,
\par Y su bandera sobre mí fue amor.
\par 5 Sustentadme con pasas, confortadme con manzanas;
\par Porque estoy enferma de amor.
\par 6 Su izquierda esté debajo de mi cabeza,
\par Y su derecha me abrace.
\par 7 Yo os conjuro, oh doncellas de Jerusalén,
\par Por los corzos y por las ciervas del campo,
\par Que no despertéis ni hagáis velar al amor,
\par Hasta que quiera.
\par 8 ¡La voz de mi amado! He aquí él viene
\par Saltando sobre los montes,
\par Brincando sobre los collados.
\par 9 Mi amado es semejante al corzo,
\par O al cervatillo.
\par Helo aquí, está tras nuestra pared,
\par Mirando por las ventanas,
\par Atisbando por las celosías.
\par 10 Mi amado habló, y me dijo:
\par Levántate, oh amiga mía, hermosa mía, y ven.
\par 11 Porque he aquí ha pasado el invierno,
\par Se ha mudado, la lluvia se fue;
\par 12 Se han mostrado las flores en la tierra,
\par El tiempo de la canción ha venido,
\par Y en nuestro país se ha oído la voz de la tórtola.
\par 13 La higuera ha echado sus higos,
\par Y las vides en cierne dieron olor;
\par Levántate, oh amiga mía, hermosa mía, y ven.
\par 14 Paloma mía, que estás en los agujeros de la peña, en lo escondido de escarpados parajes,
\par Muéstrame tu rostro, hazme oír tu voz;
\par Porque dulce es la voz tuya, y hermoso tu aspecto.
\par 15 Cazadnos las zorras, las zorras pequeñas, que echan a perder las viñas;
\par Porque nuestras viñas están en cierne.
\par 16 Mi amado es mío, y yo suya;
\par El apacienta entre lirios.
\par 17 Hasta que apunte el día, y huyan las sombras,
\par Vuélvete, amado mío; sé semejante al corzo, o como el cervatillo
\par Sobre los montes de Beter. 

\chapter{3}

\section*{El ensueño de la esposa}

\par 1 Por las noches busqué en mi lecho al que ama mi alma;
\par Lo busqué, y no lo hallé.
\par 2 Y dije: Me levantaré ahora, y rodearé por la ciudad;
\par Por las calles y por las plazas
\par Buscaré al que ama mi alma; 
\par Lo busqué, y no lo hallé.
\par 3 Me hallaron los guardas que rondan la ciudad,
\par Y les dije: ¿Habéis visto al que ama mi alma?
\par 4 Apenas hube pasado de ellos un poco,
\par Hallé luego al que ama mi alma;
\par Lo así, y no lo dejé,
\par Hasta que lo metí en casa de mi madre,
\par Y en la cámara de la que me dio a luz.
\par 5 Yo os conjuro, oh doncellas de Jerusalén,
\par Por los corzos y por las ciervas del campo,
\par Que no despertéis ni hagáis velar al amor,
\par Hasta que quiera.
\par El cortejo de bodas 
\par 6 ¿Quién es ésta que sube del desierto como columna de humo,
\par Sahumada de mirra y de incienso
\par Y de todo polvo aromático?
\par 7 He aquí es la litera de Salomón;
\par Sesenta valientes la rodean,
\par De los fuertes de Israel.
\par 8 Todos ellos tienen espadas, diestros en la guerra; 
\par Cada uno su espada sobre su muslo,
\par Por los temores de la noche.
\par 9 El rey Salomón se hizo una carroza
\par De madera del Líbano.
\par 10 Hizo sus columnas de plata,
\par Su respaldo de oro,
\par Su asiento de grana,
\par Su interior recamado de amor
\par Por las doncellas de Jerusalén.
\par 11 Salid, oh doncellas de Sion, y ved al rey Salomón
\par Con la corona con que le coronó su madre en el día de su desposorio,
\par Y el día del gozo de su corazón. 

\chapter{4}

\section*{El esposo alaba a la esposa}

\par 1 He aquí que tú eres hermosa, amiga mía; he aquí que tú eres hermosa;
\par Tus ojos entre tus guedejas como de paloma;
\par Tus cabellos como manada de cabras
\par Que se recuestan en las laderas de Galaad.
\par 2 Tus dientes como manadas de ovejas trasquiladas,
\par Que suben del lavadero,
\par Todas con crías gemelas,
\par Y ninguna entre ellas estéril.
\par 3 Tus labios como hilo de grana,
\par Y tu habla hermosa;
\par Tus mejillas, como cachos de granada detrás de tu velo.
\par 4 Tu cuello, como la torre de David, edificada para armería;
\par Mil escudos están colgados en ella,
\par Todos escudos de valientes.
\par 5 Tus dos pechos, como gemelos de gacela,
\par Que se apacientan entre lirios.
\par 6 Hasta que apunte el día y huyan las sombras,
\par Me iré al monte de la mirra,
\par Y al collado del incienso.
\par 7 Toda tú eres hermosa, amiga mía,
\par Y en ti no hay mancha.
\par 8 Ven conmigo desde el Líbano, oh esposa mía;
\par Ven conmigo desde el Líbano.
\par Mira desde la cumbre de Amana,
\par Desde la cumbre de Senir y de Hermón,
\par Desde las guaridas de los leones,
\par Desde los montes de los leopardos.
\par 9 Prendiste mi corazón, hermana, esposa mía;
\par Has apresado mi corazón con uno de tus ojos,
\par Con una gargantilla de tu cuello.
\par 10 ¡Cuán hermosos son tus amores, hermana, esposa mía!
\par ¡Cuánto mejores que el vino tus amores,
\par Y el olor de tus ungüentos que todas las especias aromáticas! 
\par 11 Como panal de miel destilan tus labios, oh esposa;
\par Miel y leche hay debajo de tu lengua;
\par Y el olor de tus vestidos como el olor del Líbano.
\par 12 Huerto cerrado eres, hermana mía, esposa mía;
\par Fuente cerrada, fuente sellada.
\par 13 Tus renuevos son paraíso de granados, con frutos suaves,
\par De flores de alheña y nardos;
\par 14 Nardo y azafrán, caña aromática y canela,
\par Con todos los árboles de incienso;
\par Mirra y áloes, con todas las principales especias aromáticas.
\par 15 Fuente de huertos,
\par Pozo de aguas vivas,
\par Que corren del Líbano.
\par 16 Levántate, Aquilón, y ven, Austro;
\par Soplad en mi huerto, despréndanse sus aromas.
\par Venga mi amado a su huerto,
\par Y coma de su dulce fruta.

\chapter{5}

\par 1 Yo vine a mi huerto, oh hermana, esposa mía;
\par He recogido mi mirra y mis aromas;
\par He comido mi panal y mi miel,
\par Mi vino y mi leche he bebido.
\par Comed, amigos; bebed en abundancia, oh amados.
\par El tormento de la separación
\par 2 Yo dormía, pero mi corazón velaba.
\par Es la voz de mi amado que llama:
\par Abreme, hermana mía, amiga mía, paloma mía, perfecta mía,
\par Porque mi cabeza está llena de rocío,
\par Mis cabellos de las gotas de la noche.
\par 3 Me he desnudado de mi ropa; ¿cómo me he de vestir?
\par He lavado mis pies; ¿cómo los he de ensuciar?
\par 4 Mi amado metió su mano por la ventanilla,
\par Y mi corazón se conmovió dentro de mí.
\par 5 Yo me levanté para abrir a mi amado,
\par Y mis manos gotearon mirra,
\par Y mis dedos mirra, que corría
\par Sobre la manecilla del cerrojo.
\par 6 Abrí yo a mi amado;
\par Pero mi amado se había ido, había ya pasado;
\par Y tras su hablar salió mi alma.
\par Lo busqué, y no lo hallé;
\par Lo llamé, y no me respondió.
\par 7 Me hallaron los guardas que rondan la ciudad;
\par Me golpearon, me hirieron;
\par Me quitaron mi manto de encima los guardas de los muros.
\par 8 Yo os conjuro, oh doncellas de Jerusalén, si halláis a mi amado,
\par Que le hagáis saber que estoy enferma de amor.
\par La esposa alaba al esposo
\par 9 ¿Qué es tu amado más que otro amado,
\par Oh la más hermosa de todas las mujeres?
\par ¿Qué es tu amado más que otro amado,
\par Que así nos conjuras?
\par 10 Mi amado es blanco y rubio,
\par Señalado entre diez mil.
\par 11 Su cabeza como oro finísimo;
\par Sus cabellos crespos, negros como el cuervo.
\par 12 Sus ojos, como palomas junto a los arroyos de las aguas,
\par Que se lavan con leche, y a la perfección colocados.
\par 13 Sus mejillas, como una era de especias aromáticas, como fragantes flores;
\par Sus labios, como lirios que destilan mirra fragante.
\par 14 Sus manos, como anillos de oro engastados de jacintos;
\par Su cuerpo, como claro marfil cubierto de zafiros.
\par 15 Sus piernas, como columnas de mármol fundadas sobre basas de oro fino;
\par Su aspecto como el Líbano, escogido como los cedros.
\par 16 Su paladar, dulcísimo, y todo él codiciable.
\par Tal es mi amado, tal es mi amigo,
\par Oh doncellas de Jerusalén.

\chapter{6}

\section*{Mutuo encanto del esposo y de la esposa}

\par 1 ¿A dónde se ha ido tu amado, oh la más hermosa de todas las mujeres?
\par ¿A dónde se apartó tu amado,
\par Y lo buscaremos contigo?
\par 2 Mi amado descendió a su huerto, a las eras de las especias,
\par Para apacentar en los huertos, y para recoger los lirios.
\par 3 Yo soy de mi amado, y mi amado es mío;
\par El apacienta entre los lirios.
\par 4 Hermosa eres tú, oh amiga mía, como Tirsa;
\par De desear, como Jerusalén;
\par Imponente como ejércitos en orden.
\par 5 Aparta tus ojos de delante de mí,
\par Porque ellos me vencieron.
\par Tu cabello es como manada de cabras
\par Que se recuestan en las laderas de Galaad.
\par 6 Tus dientes, como manadas de ovejas que suben del lavadero,
\par Todas con crías gemelas,
\par Y estéril no hay entre ellas.
\par 7 Como cachos de granada son tus mejillas
\par Detrás de tu velo.
\par 8 Sesenta son las reinas, y ochenta las concubinas,
\par Y las doncellas sin número;
\par 9 Mas una es la paloma mía, la perfecta mía;
\par Es la única de su madre,
\par La escogida de la que la dio a luz.
\par La vieron las doncellas, y la llamaron bienaventurada;
\par Las reinas y las concubinas, y la alabaron.
\par 10 ¿Quién es ésta que se muestra como el alba,
\par Hermosa como la luna,
\par Esclarecida como el sol,
\par Imponente como ejércitos en orden?
\par 11 Al huerto de los nogales descendí
\par A ver los frutos del valle,
\par Y para ver si brotaban las vides,
\par Si florecían los granados.
\par 12 Antes que lo supiera, mi alma me puso
\par Entre los carros de Aminadab.
\par 13 Vuélvete, vuélvete, oh sulamita;
\par Vuélvete, vuélvete, y te miraremos.
\par ¿Qué veréis en la sulamita?
\par Algo como la reunión de dos campamentos.

\chapter{7}

\par 1 ¡Cuán hermosos son tus pies en las sandalias,
\par Oh hija de príncipe!
\par Los contornos de tus muslos son como joyas,
\par Obra de mano de excelente maestro.
\par 2 Tu ombligo como una taza redonda
\par Que no le falta bebida.
\par Tu vientre como montón de trigo
\par Cercado de lirios.
\par 3 Tus dos pechos, como gemelos de gacela.
\par 4 Tu cuello, como torre de marfil;
\par Tus ojos, como los estanques de Hesbón junto a la puerta de Bat-rabim;
\par Tu nariz, como la torre del Líbano,
\par Que mira hacia Damasco.
\par 5 Tu cabeza encima de ti, como el Carmelo;
\par Y el cabello de tu cabeza, como la púrpura del rey
\par Suspendida en los corredores. 
\par 6 ¡Qué hermosa eres, y cuán suave,
\par Oh amor deleitoso!
\par 7 Tu estatura es semejante a la palmera,
\par Y tus pechos a los racimos.
\par 8 Yo dije: Subiré a la palmera,
\par Asiré sus ramas.
\par Deja que tus pechos sean como racimos de vid,
\par Y el olor de tu boca como de manzanas,
\par 9 Y tu paladar como el buen vino,
\par Que se entra a mi amado suavemente,
\par Y hace hablar los labios de los viejos.
\par 10 Yo soy de mi amado,
\par Y conmigo tiene su contentamiento.
\par 11 Ven, oh amado mío, salgamos al campo,
\par Moremos en las aldeas.
\par 12 Levantémonos de mañana a las viñas;
\par Veamos si brotan las vides, si están en cierne,
\par Si han florecido los granados;
\par Allí te daré mis amores.
\par 13 Las mandrágoras han dado olor,
\par Y a nuestras puertas hay toda suerte de dulces frutas,
\par Nuevas y añejas, que para ti, oh amado mío, he guardado. 

\chapter{8}

\par 1 ¡Oh, si tú fueras como un hermano mío
\par Que mamó los pechos de mi madre!
\par Entonces, hallándote fuera, te besaría,
\par Y no me menospreciarían.
\par 2 Yo te llevaría, te metería en casa de mi madre;
\par Tú me enseñarías,
\par Y yo te haría beber vino
\par Adobado del mosto de mis granadas.
\par 3 Su izquierda esté debajo de mi cabeza,
\par Y su derecha me abrace.
\par 4 Os conjuro, oh doncellas de Jerusalén,
\par Que no despertéis ni hagáis velar al amor,
\par Hasta que quiera.
\par El poder del amor
\par 5 ¿Quién es ésta que sube del desierto,
\par Recostada sobre su amado?
\par Debajo de un manzano te desperté;
\par Allí tuvo tu madre dolores,
\par Allí tuvo dolores la que te dio a luz.
\par 6 Ponme como un sello sobre tu corazón, como una marca sobre tu brazo;
\par Porque fuerte es como la muerte el amor;
\par Duros como el Seol los celos;
\par Sus brasas, brasas de fuego, fuerte llama.
\par 7 Las muchas aguas no podrán apagar el amor,
\par Ni lo ahogarán los ríos.
\par Si diese el hombre todos los bienes de su casa por este amor,
\par De cierto lo menospreciarían.
\par 8 Tenemos una pequeña hermana,
\par Que no tiene pechos;
\par ¿Qué haremos a nuestra hermana
\par Cuando de ella se hablare?
\par 9 Si ella es muro,
\par Edificaremos sobre él un palacio de plata;
\par Si fuere puerta,
\par La guarneceremos con tablas de cedro.
\par 10 Yo soy muro, y mis pechos como torres,
\par Desde que fui en sus ojos como la que halla paz.
\par 11 Salomón tuvo una viña en Baal-hamón,
\par La cual entregó a guardas,
\par Cada uno de los cuales debía traer mil monedas de plata por su fruto.
\par 12 Mi viña, que es mía, está delante de mí;
\par Las mil serán tuyas, oh Salomón,
\par Y doscientas para los que guardan su fruto.
\par 13 Oh, tú que habitas en los huertos,
\par Los compañeros escuchan tu voz;
\par Házmela oír.
\par 14 Apresúrate, amado mío,
\par Y sé semejante al corzo, o al cervatillo,
\par Sobre las montañas de los aromas.

\end{document}