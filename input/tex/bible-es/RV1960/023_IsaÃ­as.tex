\begin{document}
\chapter{1}



Una nación pecadora  
1:1 Visión de Isaías hijo de Amoz, la cual vio acerca de Judá y Jerusalén en días de Uzías, Jotam, Acaz y Ezequías, reyes de Judá.  
1:2 Oíd, cielos, y escucha tú, tierra; porque habla Jehová: Crié hijos, y los engrandecí, y ellos se rebelaron contra mí.  
1:3 El buey conoce a su dueño, y el asno el pesebre de su señor; Israel no entiende, mi pueblo no tiene conocimiento.  
1:4 ¡Oh gente pecadora, pueblo cargado de maldad, generación de malignos, hijos depravados! Dejaron a Jehová, provocaron a ira al Santo de Israel, se volvieron atrás.  
1:5 ¿Por qué querréis ser castigados aún? ¿Todavía os rebelaréis? Toda cabeza está enferma, y todo corazón doliente.  
1:6 Desde la planta del pie hasta la cabeza no hay en él cosa sana, sino herida, hinchazón y podrida llaga; no están curadas, ni vendadas, ni suavizadas con aceite.  
1:7 Vuestra tierra está destruida, vuestras ciudades puestas a fuego, vuestra tierra delante de vosotros comida por extranjeros, y asolada como asolamiento de extraños.  
1:8 Y queda la hija de Sion como enramada en viña, y como cabaña en melonar, como ciudad asolada.  
1:9 Si Jehová de los ejércitos no nos hubiese dejado un resto pequeño, como Sodoma fuéramos, y semejantes a Gomorra. 
Llamamiento al arrepentimiento verdadero  
1:10 Príncipes de Sodoma, oíd la palabra de Jehová; escuchad la ley de nuestro Dios, pueblo de Gomorra.  
1:11 ¿Para qué me sirve, dice Jehová, la multitud de vuestros sacrificios? Hastiado estoy de holocaustos de carneros y de sebo de animales gordos; no quiero sangre de bueyes, ni de ovejas, ni de machos cabríos.  
1:12 ¿Quién demanda esto de vuestras manos, cuando venís a presentaros delante de mí para hollar mis atrios?  
1:13 No me traigáis más vana ofrenda; el incienso me es abominación; luna nueva y día de reposo, el convocar asambleas, no lo puedo sufrir; son iniquidad vuestras fiestas solemnes.  
1:14 Vuestras lunas nuevas y vuestras fiestas solemnes las tiene aborrecidas mi alma; me son gravosas; cansado estoy de soportarlas.  
1:15 Cuando extendáis vuestras manos, yo esconderé de vosotros mis ojos; asimismo cuando multipliquéis la oración, yo no oiré; llenas están de sangre vuestras manos.  
1:16 Lavaos y limpiaos; quitad la iniquidad de vuestras obras de delante de mis ojos; dejad de hacer lo malo;  
1:17 aprended a hacer el bien; buscad el juicio, restituid al agraviado, haced justicia al huérfano, amparad a la viuda.  
1:18 Venid luego, dice Jehová, y estemos a cuenta: si vuestros pecados fueren como la grana, como la nieve serán emblanquecidos; si fueren rojos como el carmesí, vendrán a ser como blanca lana.  
1:19 Si quisiereis y oyereis, comeréis el bien de la tierra;  
1:20 si no quisiereis y fuereis rebeldes, seréis consumidos a espada; porque la boca de Jehová lo ha dicho.  
Juicio y redención de Jerusalén  
1:21 ¿Cómo te has convertido en ramera, oh ciudad fiel? Llena estuvo de justicia, en ella habitó la equidad; pero ahora, los homicidas.  
1:22 Tu plata se ha convertido en escorias, tu vino está mezclado con agua.  
1:23 Tus príncipes, prevaricadores y compañeros de ladrones; todos aman el soborno, y van tras las recompensas; no hacen justicia al huérfano, ni llega a ellos la causa de la viuda.  
1:24 Por tanto, dice el Señor, Jehová de los ejércitos, el Fuerte de Israel: Ea, tomaré satisfacción de mis enemigos, me vengaré de mis adversarios;  
1:25 y volveré mi mano contra ti, y limpiaré hasta lo más puro tus escorias, y quitaré toda tu impureza.  
1:26 Restauraré tus jueces como al principio, y tus consejeros como eran antes; entonces te llamarán Ciudad de justicia, Ciudad fiel.  
1:27 Sion será rescatada con juicio, y los convertidos de ella con justicia.  
1:28 Pero los rebeldes y pecadores a una serán quebrantados, y los que dejan a Jehová serán consumidos.  
1:29 Entonces os avergonzarán las encinas que amasteis, y os afrentarán los huertos que escogisteis.  
1:30 Porque seréis como encina a la que se le cae la hoja, y como huerto al que le faltan las aguas.  
1:31 Y el fuerte será como estopa, y lo que hizo como centella; y ambos serán encendidos juntamente, y no habrá quien apague.  

\chapter{2}

Reinado universal de Jehová  
 

2:1 Lo que vio Isaías hijo de Amoz acerca de Judá y de Jerusalén.  
2:2 Acontecerá en lo postrero de los tiempos, que será confirmado el monte de la casa de Jehová como cabeza de los montes, y será exaltado sobre los collados, y correrán a él todas las naciones.  
2:3 Y vendrán muchos pueblos, y dirán: Venid, y subamos al monte de Jehová, a la casa del Dios de Jacob; y nos enseñará sus caminos, y caminaremos por sus sendas. Porque de Sion saldrá la ley, y de Jerusalén la palabra de Jehová.  
2:4 Y juzgará entre las naciones, y reprenderá a muchos pueblos; y volverán sus espadas en rejas de arado, y sus lanzas en hoces; no alzará espada nación contra nación, ni se adiestrarán más para la guerra.  
Juicio de Jehová contra los soberbios  
2:5 Venid, oh casa de Jacob, y caminaremos a la luz de Jehová.  
2:6 Ciertamente tú has dejado tu pueblo, la casa de Jacob, porque están llenos de costumbres traídas del oriente, y de agoreros, como los filisteos; y pactan con hijos de extranjeros.  
2:7 Su tierra está llena de plata y oro, sus tesoros no tienen fin. También está su tierra llena de caballos, y sus carros son innumerables.  
2:8 Además su tierra está llena de ídolos, y se han arrodillado ante la obra de sus manos y ante lo que fabricaron sus dedos.  
2:9 Y se ha inclinado el hombre, y el varón se ha humillado; por tanto, no los perdones.  
2:10 Métete en la peña, escóndete en el polvo, de la presencia temible de Jehová, y del resplandor de su majestad.  
2:11 La altivez de los ojos del hombre será abatida, y la soberbia de los hombres será humillada; y Jehová solo será exaltado en aquel día.  
2:12 Porque día de Jehová de los ejércitos vendrá sobre todo soberbio y altivo, sobre todo enaltecido, y será abatido;  
2:13 sobre todos los cedros del Líbano altos y erguidos, y sobre todas las encinas de Basán;  
2:14 sobre todos los montes altos, y sobre todos los collados elevados;  
2:15 sobre toda torre alta, y sobre todo muro fuerte;  
2:16 sobre todas las naves de Tarsis, y sobre todas las pinturas preciadas.  
2:17 La altivez del hombre será abatida, y la soberbia de los hombres será humillada; y solo Jehová será exaltado en aquel día.  
2:18 Y quitará totalmente los ídolos.  
2:19 Y se meterán en las cavernas de las peñas y en las aberturas de la tierra, por la presencia temible de Jehová, y por el resplandor de su majestad, cuando él se levante para castigar la tierra.  
2:20 Aquel día arrojará el hombre a los topos y murciélagos sus ídolos de plata y sus ídolos de oro, que le hicieron para que adorase,  
2:21 y se meterá en las hendiduras de las rocas y en las cavernas de las peñas, por la presencia formidable de Jehová, y por el resplandor de su majestad, cuando se levante para castigar la tierra.  
2:22 Dejaos del hombre, cuyo aliento está en su nariz; porque ¿de qué es él estimado?  

\chapter{3}

Juicio de Jehová contra Judá y Jerusalén  

3:1 Porque he aquí que el Señor Jehová de los ejércitos quita de Jerusalén y de Judá al sustentador y al fuerte, todo sustento de pan y todo socorro de agua;  
3:2 el valiente y el hombre de guerra, el juez y el profeta, el adivino y el anciano;  
3:3 el capitán de cincuenta y el hombre de respeto, el consejero, el artífice excelente y el hábil orador.  
3:4 Y les pondré jóvenes por príncipes, y muchachos serán sus señores.  
3:5 Y el pueblo se hará violencia unos a otros, cada cual contra su vecino; el joven se levantará contra el anciano, y el villano contra el noble.  
3:6 Cuando alguno tomare de la mano a su hermano, de la familia de su padre, y le dijere: Tú tienes vestido, tú serás nuestro príncipe, y toma en tus manos esta ruina;  
3:7 él jurará aquel día, diciendo: No tomaré ese cuidado; porque en mi casa ni hay pan, ni qué vestir; no me hagáis príncipe del pueblo.  
3:8 Pues arruinada está Jerusalén, y Judá ha caído; porque la lengua de ellos y sus obras han sido contra Jehová para irritar los ojos de su majestad.  
3:9 La apariencia de sus rostros testifica contra ellos; porque como Sodoma publican su pecado, no lo disimulan. ¡Ay del alma de ellos! porque amontonaron mal para sí.  
3:10 Decid al justo que le irá bien, porque comerá de los frutos de sus manos.  
3:11 ¡Ay del impío! Mal le irá, porque según las obras de sus manos le será pagado.  
3:12 Los opresores de mi pueblo son muchachos, y mujeres se enseñorearon de él. Pueblo mío, los que te guían te engañan, y tuercen el curso de tus caminos.  
3:13 Jehová está en pie para litigar, y está para juzgar a los pueblos.  
3:14 Jehová vendrá a juicio contra los ancianos de su pueblo y contra sus príncipes; porque vosotros habéis devorado la viña, y el despojo del pobre está en vuestras casas.  
3:15 ¿Qué pensáis vosotros que majáis mi pueblo y moléis las caras de los pobres? dice el Señor, Jehová de los ejércitos.  
Juicio contra las hijas de Sion  
3:16 Asimismo dice Jehová: Por cuanto las hijas de Sion se ensoberbecen, y andan con cuello erguido y con ojos desvergonzados; cuando andan van danzando, y haciendo son con los pies;  
3:17 por tanto, el Señor raerá la cabeza de las hijas de Sion, y Jehová descubrirá sus vergüenzas.  
3:18 Aquel día quitará el Señor el atavío del calzado, las redecillas, las lunetas,  
3:19 los collares, los pendientes y los brazaletes,  
3:20 las cofias, los atavíos de las piernas, los partidores del pelo, los pomitos de olor y los zarcillos,  
3:21 los anillos, y los joyeles de las narices,  
3:22 las ropas de gala, los mantoncillos, los velos, las bolsas,  
3:23 los espejos, el lino fino, las gasas y los tocados.  
3:24 Y en lugar de los perfumes aromáticos vendrá hediondez; y cuerda en lugar de cinturón, y cabeza rapada en lugar de la compostura del cabello; en lugar de ropa de gala ceñimiento de cilicio, y quemadura en vez de hermosura.  
3:25 Tus varones caerán a espada, y tu fuerza en la guerra.  
3:26 Sus puertas se entristecerán y enlutarán, y ella, desamparada, se sentará en tierra.  

\chapter{4}


4:1 Echarán mano de un hombre siete mujeres en aquel tiempo, diciendo: Nosotras comeremos de nuestro pan, y nos vestiremos de nuestras ropas; solamente permítenos llevar tu nombre, quita nuestro oprobio.  
Futuro glorioso de Jerusalén  
4:2 En aquel tiempo el renuevo de Jehová será para hermosura y gloria, y el fruto de la tierra para grandeza y honra, a los sobrevivientes de Israel.  
4:3 Y acontecerá que el que quedare en Sion, y el que fuere dejado en Jerusalén, será llamado santo; todos los que en Jerusalén estén registrados entre los vivientes,  
4:4 cuando el Señor lave las inmundicias de las hijas de Sion, y limpie la sangre de Jerusalén de en medio de ella, con espíritu de juicio y con espíritu de devastación.  
4:5 Y creará Jehová sobre toda la morada del monte de Sion, y sobre los lugares de sus convocaciones, nube y oscuridad de día, y de noche resplandor de fuego que eche llamas; porque sobre toda gloria habrá un dosel,  
4:6 y habrá un abrigo para sombra contra el calor del día, para refugio y escondedero contra el turbión y contra el aguacero.  

\chapter{5}

Parábola de la viña  

5:1 Ahora cantaré por mi amado el cantar de mi amado a su viña. Tenía mi amado una viña en una ladera fértil.  
5:2 La había cercado y despedregado y plantado de vides escogidas; había edificado en medio de ella una torre, y hecho también en ella un lagar; y esperaba que diese uvas, y dio uvas silvestres.  
5:3 Ahora, pues, vecinos de Jerusalén y varones de Judá, juzgad ahora entre mí y mi viña.  
5:4 ¿Qué más se podía hacer a mi viña, que yo no haya hecho en ella? ¿Cómo, esperando yo que diese uvas, ha dado uvas silvestres?  
5:5 Os mostraré, pues, ahora lo que haré yo a mi viña: Le quitaré su vallado, y será consumida; aportillaré su cerca, y será hollada.  
5:6 Haré que quede desierta; no será podada ni cavada, y crecerán el cardo y los espinos; y aun a las nubes mandaré que no derramen lluvia sobre ella.  
5:7 Ciertamente la viña de Jehová de los ejércitos es la casa de Israel, y los hombres de Judá planta deliciosa suya. Esperaba juicio, y he aquí vileza; justicia, y he aquí clamor.  
Ayes sobre los malvados 
5:8 ¡Ay de los que juntan casa a casa, y añaden heredad a heredad hasta ocuparlo todo! ¿Habitaréis vosotros solos en medio de la tierra?  
5:9 Ha llegado a mis oídos de parte de Jehová de los ejércitos, que las muchas casas han de quedar asoladas, sin morador las grandes y hermosas.  
5:10 Y diez yugadas de viña producirán un bato,  y un homer de semilla producirá un efa.  
5:11 ¡Ay de los que se levantan de mañana para seguir la embriaguez; que se están hasta la noche, hasta que el vino los enciende!  
5:12 Y en sus banquetes hay arpas, vihuelas, tamboriles, flautas y vino, y no miran la obra de Jehová, ni consideran la obra de sus manos.  
5:13 Por tanto, mi pueblo fue llevado cautivo, porque no tuvo conocimiento; y su gloria pereció de hambre, y su multitud se secó de sed.  
5:14 Por eso ensanchó su interior el Seol, y sin medida extendió su boca; y allá descenderá la gloria de ellos, y su multitud, y su fausto, y el que en él se regocijaba.  
5:15 Y el hombre será humillado, y el varón será abatido, y serán bajados los ojos de los altivos.  
5:16 Pero Jehová de los ejércitos será exaltado en juicio, y el Dios Santo será santificado con justicia.  
5:17 Y los corderos serán apacentados según su costumbre; y extraños devorarán los campos desolados de los ricos.  
5:18 ¡Ay de los que traen la iniquidad con cuerdas de vanidad, y el pecado como con coyundas de carreta,  
5:19 los cuales dicen: Venga ya, apresúrese su obra, y veamos; acérquese, y venga el consejo del Santo de Israel, para que lo sepamos!  
5:20 ¡Ay de los que a lo malo dicen bueno, y a lo bueno malo; que hacen de la luz tinieblas, y de las tinieblas luz; que ponen lo amargo por dulce, y lo dulce por amargo!  
5:21 ¡Ay de los sabios en sus propios ojos, y de los que son prudentes delante de sí mismos!  
5:22 ¡Ay de los que son valientes para beber vino, y hombres fuertes para mezclar bebida;  
5:23 los que justifican al impío mediante cohecho, y al justo quitan su derecho!  
5:24 Por tanto, como la lengua del fuego consume el rastrojo, y la llama devora la paja, así será su raíz como podredumbre, y su flor se desvanecerá como polvo; porque desecharon la ley de Jehová de los ejércitos, y abominaron la palabra del Santo de Israel.  
5:25 Por esta causa se encendió el furor de Jehová contra su pueblo, y extendió contra él su mano, y le hirió; y se estremecieron los montes, y sus cadáveres fueron arrojados en medio de las calles. Con todo esto no ha cesado su furor, sino que todavía su mano está extendida.  
5:26 Alzará pendón a naciones lejanas, y silbará al que está en el extremo de la tierra; y he aquí que vendrá pronto y velozmente.  
5:27 No habrá entre ellos cansado, ni quien tropiece; ninguno se dormirá, ni le tomará sueño; a ninguno se le desatará el cinto de los lomos, ni se le romperá la correa de sus sandalias.  
5:28 Sus saetas estarán afiladas, y todos sus arcos entesados; los cascos de sus caballos parecerán como de pedernal, y las ruedas de sus carros como torbellino.  
5:29 Su rugido será como de león; rugirá a manera de leoncillo, crujirá los dientes, y arrebatará la presa; se la llevará con seguridad, y nadie se la quitará.  
5:30 Y bramará sobre él en aquel día como bramido del mar; entonces mirará hacia la tierra, y he aquí tinieblas de tribulación, y en sus cielos se oscurecerá la luz.  

\chapter{6}

Visión y llamamiento de Isaías  

6:1 En el año que murió el rey Uzías vi yo al Señor sentado sobre un trono alto y sublime, y sus faldas llenaban el templo.  
6:2 Por encima de él había serafines; cada uno tenía seis alas; con dos cubrían sus rostros, con dos cubrían sus pies, y con dos volaban.  
6:3 Y el uno al otro daba voces, diciendo: Santo, santo, santo, Jehová de los ejércitos; toda la tierra está llena de su gloria.  
6:4 Y los quiciales de las puertas se estremecieron con la voz del que clamaba, y la casa se llenó de humo. 
6:5 Entonces dije: ¡Ay de mí! que soy muerto; porque siendo hombre inmundo de labios, y habitando en medio de pueblo que tiene labios inmundos, han visto mis ojos al Rey, Jehová de los ejércitos.  
6:6 Y voló hacia mí uno de los serafines, teniendo en su mano un carbón encendido, tomado del altar con unas tenazas;  
6:7 y tocando con él sobre mi boca, dijo: He aquí que esto tocó tus labios, y es quitada tu culpa, y limpio tu pecado.  
6:8 Después oí la voz del Señor, que decía: ¿A quién enviaré, y quién irá por nosotros? Entonces respondí yo: Heme aquí, envíame a mí.  
6:9 Y dijo: Anda, y di a este pueblo: Oíd bien, y no entendáis; ved por cierto, mas no comprendáis.  
6:10 Engruesa el corazón de este pueblo, y agrava sus oídos, y ciega sus ojos, para que no vea con sus ojos, ni oiga con sus oídos, ni su corazón entienda, ni se convierta, y haya para él sanidad. 
6:11 Y yo dije: ¿Hasta cuándo, Señor? Y respondió él: Hasta que las ciudades estén asoladas y sin morador, y no haya hombre en las casas, y la tierra esté hecha un desierto;  
6:12 hasta que Jehová haya echado lejos a los hombres, y multiplicado los lugares abandonados en medio de la tierra.  
6:13 Y si quedare aún en ella la décima parte, ésta volverá a ser destruida; pero como el roble y la encina, que al ser cortados aún queda el tronco, así será el tronco, la simiente santa.  

\chapter{7}

Mensaje de Isaías a Acaz  

7:1 Aconteció en los días de Acaz hijo de Jotam, hijo de Uzías, rey de Judá, que Rezín rey de Siria y Peka hijo de Remalías, rey de Israel, subieron contra Jerusalén para combatirla; pero no la pudieron tomar. 
7:2 Y vino la nueva a la casa de David, diciendo: Siria se ha confederado con Efraín. Y se le estremeció el corazón, y el corazón de su pueblo, como se estremecen los árboles del monte a causa del viento.  
7:3 Entonces dijo Jehová a Isaías: Sal ahora al encuentro de Acaz, tú, y Sear-jasub tu hijo, al extremo del acueducto del estanque de arriba, en el camino de la heredad del Lavador,  
7:4 y dile: Guarda, y repósate; no temas, ni se turbe tu corazón a causa de estos dos cabos de tizón que humean, por el ardor de la ira de Rezín y de Siria, y del hijo de Remalías.  
7:5 Ha acordado maligno consejo contra ti el sirio, con Efraín y con el hijo de Remalías, diciendo:  
7:6 Vamos contra Judá y aterroricémosla, y repartámosla entre nosotros, y pongamos en medio de ella por rey al hijo de Tabeel.  
7:7 Por tanto, Jehová el Señor dice así: No subsistirá, ni será.  
7:8 Porque la cabeza de Siria es Damasco, y la cabeza de Damasco, Rezín; y dentro de sesenta y cinco años Efraín será quebrantado hasta dejar de ser pueblo.  
7:9 Y la cabeza de Efraín es Samaria, y la cabeza de Samaria el hijo de Remalías. Si vosotros no creyereis, de cierto no permaneceréis.  
7:10 Habló también Jehová a Acaz, diciendo:  
7:11 Pide para ti señal de Jehová tu Dios, demandándola ya sea de abajo en lo profundo, o de arriba en lo alto.  
7:12 Y respondió Acaz: No pediré, y no tentaré a Jehová.  
7:13 Dijo entonces Isaías: Oíd ahora, casa de David. ¿Os es poco el ser molestos a los hombres, sino que también lo seáis a mi Dios?  
7:14 Por tanto, el Señor mismo os dará señal: He aquí que la virgen concebirá, y dará a luz un hijo, y llamará su nombre Emanuel.  
7:15 Comerá mantequilla y miel, hasta que sepa desechar lo malo y escoger lo bueno.  
7:16 Porque antes que el niño sepa desechar lo malo y escoger lo bueno, la tierra de los dos reyes que tú temes será abandonada.  
7:17 Jehová hará venir sobre ti, sobre tu pueblo y sobre la casa de tu padre, días cuales nunca vinieron desde el día que Efraín se apartó de Judá, esto es, al rey de Asiria.  
7:18 Y acontecerá que aquel día silbará Jehová a la mosca que está en el fin de los ríos de Egipto, y a la abeja que está en la tierra de Asiria;  
7:19 y vendrán y acamparán todos en los valles desiertos, y en las cavernas de las piedras, y en todos los zarzales, y en todas las matas.  
7:20 En aquel día el Señor raerá con navaja alquilada, con los que habitan al otro lado del río, esto es, con el rey de Asiria, cabeza y pelo de los pies, y aun la barba también quitará.  
7:21 Acontecerá en aquel tiempo, que criará un hombre una vaca y dos ovejas;  
7:22 y a causa de la abundancia de leche que darán, comerá mantequilla; ciertamente mantequilla y miel comerá el que quede en medio de la tierra.  
7:23 Acontecerá también en aquel tiempo, que el lugar donde había mil vides que valían mil siclos de plata,  será para espinos y cardos.  
7:24 Con saetas y arco irán allá, porque toda la tierra será espinos y cardos.  
7:25 Y a todos los montes que se cavaban con azada, no llegarán allá por el temor de los espinos y de los cardos, sino que serán para pasto de bueyes y para ser hollados de los ganados.  

\chapter{8}

Sea Jehová vuestro temor  

8:1 Me dijo Jehová: Toma una tabla grande, y escribe en ella con caracteres legibles tocante a Maher-salal-hasbaz.  
8:2 Y junté conmigo por testigos fieles al sacerdote Urías y a Zacarías hijo de Jeberequías.  
8:3 Y me llegué a la profetisa, la cual concibió, y dio a luz un hijo. Y me dijo Jehová: Ponle por nombre Maher-salal-hasbaz.  
8:4 Porque antes que el niño sepa decir: Padre mío, y Madre mía, será quitada la riqueza de Damasco y los despojos de Samaria delante del rey de Asiria.  
8:5 Otra vez volvió Jehová a hablarme, diciendo:  
8:6 Por cuanto desechó este pueblo las aguas de Siloé, que corren mansamente, y se regocijó con Rezín y con el hijo de Remalías;  
8:7 he aquí, por tanto, que el Señor hace subir sobre ellos aguas de ríos, impetuosas y muchas, esto es, al rey de Asiria con todo su poder; el cual subirá sobre todos sus ríos, y pasará sobre todas sus riberas;  
8:8 y pasando hasta Judá, inundará y pasará adelante, y llegará hasta la garganta; y extendiendo sus alas, llenará la anchura de tu tierra, oh Emanuel.  
8:9 Reuníos, pueblos, y seréis quebrantados; oíd, todos los que sois de lejanas tierras; ceñíos, y seréis quebrantados; disponeos, y seréis quebrantados.  
8:10 Tomad consejo, y será anulado; proferid palabra, y no será firme, porque Dios está con nosotros.  
8:11 Porque Jehová me dijo de esta manera con mano fuerte, y me enseñó que no caminase por el camino de este pueblo, diciendo:  
8:12 No llaméis conspiración a todas las cosas que este pueblo llama conspiración; ni temáis lo que ellos temen, ni tengáis miedo.  
8:13 A Jehová de los ejércitos, a él santificad; sea él vuestro temor, y él sea vuestro miedo.  
8:14 Entonces él será por santuario; pero a las dos casas de Israel, por piedra para tropezar, y por tropezadero para caer, y por lazo y por red al morador de Jerusalén.  
8:15 Y muchos tropezarán entre ellos, y caerán, y serán quebrantados; y se enredarán y serán apresados.  
8:16 Ata el testimonio, sella la ley entre mis discípulos.  
8:17 Esperaré, pues, a Jehová, el cual escondió su rostro de la casa de Jacob, y en él confiaré. 
8:18 He aquí, yo y los hijos que me dio Jehová somos por señales y presagios en Israel, de parte de Jehová de los ejércitos, que mora en el monte de Sion.  
8:19 Y si os dijeren: Preguntad a los encantadores y a los adivinos, que susurran hablando, responded: ¿No consultará el pueblo a su Dios? ¿Consultará a los muertos por los vivos?  
8:20 ¡A la ley y al testimonio! Si no dijeren conforme a esto, es porque no les ha amanecido.  
8:21 Y pasarán por la tierra fatigados y hambrientos, y acontecerá que teniendo hambre, se enojarán y maldecirán a su rey y a su Dios, levantando el rostro en alto.  
8:22 Y mirarán a la tierra, y he aquí tribulación y tinieblas, oscuridad y angustia; y serán sumidos en las tinieblas.  

\chapter{9}

Nacimiento y reinado del Mesías  

9:1 Mas no habrá siempre oscuridad para la que está ahora en angustia, tal como la aflicción que le vino en el tiempo que livianamente tocaron la primera vez a la tierra de Zabulón y a la tierra de Neftalí; pues al fin llenará de gloria el camino del mar, de aquel lado del Jordán, en Galilea de los gentiles. 
9:2 El pueblo que andaba en tinieblas vio gran luz; los que moraban en tierra de sombra de muerte, luz resplandeció sobre ellos. 
9:3 Multiplicaste la gente, y aumentaste la alegría. Se alegrarán delante de ti como se alegran en la siega, como se gozan cuando reparten despojos.  
9:4 Porque tú quebraste su pesado yugo, y la vara de su hombro, y el cetro de su opresor, como en el día de Madián.  
9:5 Porque todo calzado que lleva el guerrero en el tumulto de la batalla, y todo manto revolcado en sangre, serán quemados, pasto del fuego.  
9:6 Porque un niño nos es nacido, hijo nos es dado, y el principado sobre su hombro; y se llamará su nombre Admirable, Consejero, Dios Fuerte, Padre Eterno, Príncipe de Paz.  
9:7 Lo dilatado de su imperio y la paz no tendrán límite, sobre el trono de David y sobre su reino, disponiéndolo y confirmándolo en juicio y en justicia desde ahora y para siempre. El celo de Jehová de los ejércitos hará esto.  
La ira de Jehová contra Israel  
9:8 El Señor envió palabra a Jacob, y cayó en Israel.  
9:9 Y la sabrá todo el pueblo, Efraín y los moradores de Samaria, que con soberbia y con altivez de corazón dicen:  
9:10 Los ladrillos cayeron, pero edificaremos de cantería; cortaron los cabrahigos, pero en su lugar pondremos cedros.  
9:11 Pero Jehová levantará los enemigos de Rezín contra él, y juntará a sus enemigos;  
9:12 del oriente los sirios, y los filisteos del poniente; y a boca llena devorarán a Israel. Ni con todo eso ha cesado su furor, sino que todavía su mano está extendida.  
9:13 Pero el pueblo no se convirtió al que lo castigaba, ni buscó a Jehová de los ejércitos.  
9:14 Y Jehová cortará de Israel cabeza y cola, rama y caña en un mismo día.  
9:15 El anciano y venerable de rostro es la cabeza; el profeta que enseña mentira, es la cola.  
9:16 Porque los gobernadores de este pueblo son engañadores, y sus gobernados se pierden.  
9:17 Por tanto, el Señor no tomará contentamiento en sus jóvenes, ni de sus huérfanos y viudas tendrá misericordia; porque todos son falsos y malignos, y toda boca habla despropósitos. Ni con todo esto ha cesado su furor, sino que todavía su mano está extendida.  
9:18 Porque la maldad se encendió como fuego, cardos y espinos devorará; y se encenderá en lo espeso del bosque, y serán alzados como remolinos de humo.  
9:19 Por la ira de Jehová de los ejércitos se oscureció la tierra, y será el pueblo como pasto del fuego; el hombre no tendrá piedad de su hermano.  
9:20 Cada uno hurtará a la mano derecha, y tendrá hambre, y comerá a la izquierda, y no se saciará; cada cual comerá la carne de su brazo;  
9:21 Manasés a Efraín, y Efraín a Manasés, y ambos contra Judá. Ni con todo esto ha cesado su furor, sino que todavía su mano está extendida.  

\chapter{10}


10:1 ¡Ay de los que dictan leyes injustas, y prescriben tiranía,  
10:2 para apartar del juicio a los pobres, y para quitar el derecho a los afligidos de mi pueblo; para despojar a las viudas, y robar a los huérfanos!  
10:3 ¿Y qué haréis en el día del castigo? ¿A quién os acogeréis para que os ayude, cuando venga de lejos el asolamiento? ¿En dónde dejaréis vuestra gloria?  
10:4 Sin mí se inclinarán entre los presos, y entre los muertos caerán. Ni con todo esto ha cesado su furor, sino que todavía su mano está extendida.  
Asiria, instrumento de Dios  
10:5 Oh Asiria, vara y báculo de mi furor, en su mano he puesto mi ira.  
10:6 Le mandaré contra una nación pérfida, y sobre el pueblo de mi ira le enviaré, para que quite despojos, y arrebate presa, y lo ponga para ser hollado como lodo de las calles. 
10:7 Aunque él no lo pensará así, ni su corazón lo imaginará de esta manera, sino que su pensamiento será desarraigar y cortar naciones no pocas.  
10:8 Porque él dice: Mis príncipes, ¿no son todos reyes?  
10:9 ¿No es Calno como Carquemis, Hamat como Arfad, y Samaria como Damasco?  
10:10 Como halló mi mano los reinos de los ídolos, siendo sus imágenes más que las de Jerusalén y de Samaria;  
10:11 como hice a Samaria y a sus ídolos, ¿no haré también así a Jerusalén y a sus ídolos?  
10:12 Pero acontecerá que después que el Señor haya acabado toda su obra en el monte de Sion y en Jerusalén, castigará el fruto de la soberbia del corazón del rey de Asiria, y la gloria de la altivez de sus ojos.  
10:13 Porque dijo: Con el poder de mi mano lo he hecho, y con mi sabiduría, porque he sido prudente; quité los territorios de los pueblos, y saqueé sus tesoros, y derribé como valientes a los que estaban sentados;  
10:14 y halló mi mano como nido las riquezas de los pueblos; y como se recogen los huevos abandonados, así me apoderé yo de toda la tierra; y no hubo quien moviese ala, ni abriese boca y graznase.  
10:15 ¿Se gloriará el hacha contra el que con ella corta? ¿Se ensoberbecerá la sierra contra el que la mueve? ¡Como si el báculo levantase al que lo levanta; como si levantase la vara al que no es leño!  
10:16 Por esto el Señor, Jehová de los ejércitos, enviará debilidad sobre sus robustos, y debajo de su gloria encenderá una hoguera como ardor de fuego.  
10:17 Y la luz de Israel será por fuego, y su Santo por llama, que abrase y consuma en un día sus cardos y sus espinos.  
10:18 La gloria de su bosque y de su campo fértil consumirá totalmente, alma y cuerpo, y vendrá a ser como abanderado en derrota.  
10:19 Y los árboles que queden en su bosque serán en número que un niño los pueda contar.  
10:20 Acontecerá en aquel tiempo, que los que hayan quedado de Israel y los que hayan quedado de la casa de Jacob, nunca más se apoyarán en el que los hirió, sino que se apoyarán con verdad en Jehová, el Santo de Israel.  
10:21 El remanente volverá, el remanente de Jacob volverá al Dios fuerte.  
10:22 Porque si tu pueblo, oh Israel, fuere como las arenas del mar, el remanente de él volverá; la destrucción acordada rebosará justicia.  
10:23 Pues el Señor, Jehová de los ejércitos, hará consumación ya determinada en medio de la tierra. 
10:24 Por tanto el Señor, Jehová de los ejércitos, dice así: Pueblo mío, morador de Sion, no temas de Asiria. Con vara te herirá, y contra ti alzará su palo, a la manera de Egipto;  
10:25 mas de aquí a muy poco tiempo se acabará mi furor y mi enojo, para destrucción de ellos.  
10:26 Y levantará Jehová de los ejércitos azote contra él como la matanza de Madián en la peña de Oreb, y alzará su vara sobre el mar como hizo por la vía de Egipto.  
10:27 Acontecerá en aquel tiempo que su carga será quitada de tu hombro, y su yugo de tu cerviz, y el yugo se pudrirá a causa de la unción.  
10:28 Vino hasta Ajat, pasó hasta Migrón; en Micmas contará su ejército.  
10:29 Pasaron el vado; se alojaron en Geba; Ramá tembló; Gabaa de Saúl huyó.  
10:30 Grita en alta voz, hija de Galim; haz que se oiga hacia Lais, pobrecilla Anatot.  
10:31 Madmena se alborotó; los moradores de Gebim huyen.  
10:32 Aún vendrá día cuando reposará en Nob; alzará su mano al monte de la hija de Sion, al collado de Jerusalén.  
10:33 He aquí el Señor, Jehová de los ejércitos, desgajará el ramaje con violencia, y los árboles de gran altura serán cortados, y los altos serán humillados.  
10:34 Y cortará con hierro la espesura del bosque, y el Líbano caerá con estruendo.  

\chapter{11}

Reinado justo del Mesías  

11:1 Saldrá una vara del tronco de Isaí, y un vástago retoñará de sus raíces. 
11:2 Y reposará sobre él el Espíritu de Jehová; espíritu de sabiduría y de inteligencia, espíritu de consejo y de poder, espíritu de conocimiento y de temor de Jehová.  
11:3 Y le hará entender diligente en el temor de Jehová. No juzgará según la vista de sus ojos, ni argüirá por lo que oigan sus oídos;  
11:4 sino que juzgará con justicia a los pobres, y argüirá con equidad por los mansos de la tierra; y herirá la tierra con la vara de su boca, y con el espíritu de sus labios matará al impío. 
11:5 Y será la justicia cinto de sus lomos, y la fidelidad ceñidor de su cintura.  
11:6 Morará el lobo con el cordero, y el leopardo con el cabrito se acostará; el becerro y el león y la bestia doméstica andarán juntos, y un niño los pastoreará.  
11:7 La vaca y la osa pacerán, sus crías se echarán juntas; y el león como el buey comerá paja.  
11:8 Y el niño de pecho jugará sobre la cueva del áspid, y el recién destetado extenderá su mano sobre la caverna de la víbora.  
11:9 No harán mal ni dañarán en todo mi santo monte; porque la tierra será llena del conocimiento de Jehová, como las aguas cubren el mar. 
11:10 Acontecerá en aquel tiempo que la raíz de Isaí, la cual estará puesta por pendón a los pueblos, será buscada por las gentes; y su habitación será gloriosa.  
11:11 Asimismo acontecerá en aquel tiempo, que Jehová alzará otra vez su mano para recobrar el remanente de su pueblo que aún quede en Asiria, Egipto, Patros, Etiopía, Elam, Sinar y Hamat, y en las costas del mar.  
11:12 Y levantará pendón a las naciones, y juntará los desterrados de Israel, y reunirá los esparcidos de Judá de los cuatro confines de la tierra.  
11:13 Y se disipará la envidia de Efraín, y los enemigos de Judá serán destruidos. Efraín no tendrá envidia de Judá, ni Judá afligirá a Efraín;  
11:14 sino que volarán sobre los hombros de los filisteos al occidente, saquearán también a los de oriente; Edom y Moab les servirán, y los hijos de Amón los obedecerán.  
11:15 Y secará Jehová la lengua del mar de Egipto; y levantará su mano con el poder de su espíritu sobre el río, y lo herirá en sus siete brazos, y hará que pasen por él con sandalias. 
11:16 Y habrá camino para el remanente de su pueblo, el que quedó de Asiria, de la manera que lo hubo para Israel el día que subió de la tierra de Egipto.  

\chapter{12}

Cántico de acción de gracias  

12:1 En aquel día dirás: Cantaré a ti, oh Jehová; pues aunque te enojaste contra mí, tu indignación se apartó, y me has consolado.  
12:2 He aquí Dios es salvación mía; me aseguraré y no temeré; porque mi fortaleza y mi canción es JAH Jehová, quien ha sido salvación para mí.  
12:3 Sacaréis con gozo aguas de las fuentes de la salvación.  
12:4 Y diréis en aquel día: Cantad a Jehová, aclamad su nombre, haced célebres en los pueblos sus obras, recordad que su nombre es engrandecido.  
12:5 Cantad salmos a Jehová, porque ha hecho cosas magníficas; sea sabido esto por toda la tierra.  
12:6 Regocíjate y canta, oh moradora de Sion; porque grande es en medio de ti el Santo de Israel.  

\chapter{13}

Profecía sobre Babilonia  

13:1 Profecía sobre Babilonia, revelada a Isaías hijo de Amoz.  
13:2 Levantad bandera sobre un alto monte; alzad la voz a ellos, alzad la mano, para que entren por puertas de príncipes. Yo mandé a mis consagrados, asimismo llamé a mis valientes para mi ira, a los que se alegran con mi gloria.  
13:4 Estruendo de multitud en los montes, como de mucho pueblo; estruendo de ruido de reinos, de naciones reunidas; Jehová de los ejércitos pasa revista a las tropas para la batalla.  
13:5 Vienen de lejana tierra, de lo postrero de los cielos, Jehová y los instrumentos de su ira, para destruir toda la tierra.  
13:6 Aullad, porque cerca está el día de Jehová; vendrá como asolamiento del Todopoderoso. 
13:7 Por tanto, toda mano se debilitará, y desfallecerá todo corazón de hombre,  
13:8 y se llenarán de terror; angustias y dolores se apoderarán de ellos; tendrán dolores como mujer de parto; se asombrará cada cual al mirar a su compañero; sus rostros, rostros de llamas.  
13:9 He aquí el día de Jehová viene, terrible, y de indignación y ardor de ira, para convertir la tierra en soledad, y raer de ella a sus pecadores.  
13:10 Por lo cual las estrellas de los cielos y sus luceros no darán su luz; y el sol se oscurecerá al nacer, y la luna no dará su resplandor. 
13:11 Y castigaré al mundo por su maldad, y a los impíos por su iniquidad; y haré que cese la arrogancia de los soberbios, y abatiré la altivez de los fuertes.  
13:12 Haré más precioso que el oro fino al varón, y más que el oro de Ofir al hombre.  
13:13 Porque haré estremecer los cielos, y la tierra se moverá de su lugar, en la indignación de Jehová de los ejércitos, y en el día del ardor de su ira.  
13:14 Y como gacela perseguida, y como oveja sin pastor, cada cual mirará hacia su pueblo, y cada uno huirá a su tierra.  
13:15 Cualquiera que sea hallado será alanceado; y cualquiera que por ellos sea tomado, caerá a espada.  
13:16 Sus niños serán estrellados delante de ellos; sus casas serán saqueadas, y violadas sus mujeres.  
13:17 He aquí que yo despierto contra ellos a los medos, que no se ocuparán de la plata, ni codiciarán oro.  
13:18 Con arco tirarán a los niños, y no tendrán misericordia del fruto del vientre, ni su ojo perdonará a los hijos.  
13:19 Y Babilonia, hermosura de reinos y ornamento de la grandeza de los caldeos, será como Sodoma y Gomorra, a las que trastornó Dios 
13:20 Nunca más será habitada, ni se morará en ella de generación en generación; ni levantará allí tienda el árabe, ni pastores tendrán allí majada;  
13:21 sino que dormirán allí las fieras del desierto, y sus casas se llenarán de hurones; allí habitarán avestruces, y allí saltarán las cabras salvajes.  
13:22 En sus palacios aullarán hienas, y chacales en sus casas de deleite; y cercano a llegar está su tiempo, y sus días no se alargarán.  

\chapter{14}

Escarnio contra el rey de Babilonia  

14:1 Porque Jehová tendrá piedad de Jacob, y todavía escogerá a Israel, y lo hará reposar en su tierra; y a ellos se unirán extranjeros, y se juntarán a la familia de Jacob.  
14:2 Y los tomarán los pueblos, y los traerán a su lugar; y la casa de Israel los poseerá por siervos y criadas en la tierra de Jehová; y cautivarán a los que los cautivaron, y señorearán sobre los que los oprimieron.  
14:3 Y en el día que Jehová te dé reposo de tu trabajo y de tu temor, y de la dura servidumbre en que te hicieron servir,  
14:4 pronunciarás este proverbio contra el rey de Babilonia, y dirás: ¡Cómo paró el opresor, cómo acabó la ciudad codiciosa de oro!  
14:5 Quebrantó Jehová el báculo de los impíos, el cetro de los señores;  
14:6 el que hería a los pueblos con furor, con llaga permanente, el que se enseñoreaba de las naciones con ira, y las perseguía con crueldad.  
14:7 Toda la tierra está en reposo y en paz; se cantaron alabanzas.  
14:8 Aun los cipreses se regocijaron a causa de ti, y los cedros del Líbano, diciendo: Desde que tú pereciste, no ha subido cortador contra nosotros.  
14:9 El Seol abajo se espantó de ti; despertó muertos que en tu venida saliesen a recibirte, hizo levantar de sus sillas a todos los príncipes de la tierra, a todos los reyes de las naciones.  
14:10 Todos ellos darán voces, y te dirán: ¿Tú también te debilitaste como nosotros, y llegaste a ser como nosotros?  
14:11 Descendió al Seol tu soberbia, y el sonido de tus arpas; gusanos serán tu cama, y gusanos te cubrirán.  
14:12 ¡Cómo caíste del cielo, oh Lucero, hijo de la mañana! Cortado fuiste por tierra, tú que debilitabas a las naciones.  
14:13 Tú que decías en tu corazón: Subiré al cielo; en lo alto, junto a las estrellas de Dios, levantaré mi trono, y en el monte del testimonio me sentaré, a los lados del norte;  
14:14 sobre las alturas de las nubes subiré, y seré semejante al Altísimo.  
14:15 Mas tú derribado eres hasta el Seol, a los lados del abismo.  
14:16 Se inclinarán hacia ti los que te vean, te contemplarán, diciendo: ¿Es éste aquel varón que hacía temblar la tierra, que trastornaba los reinos;  
14:17 que puso el mundo como un desierto, que asoló sus ciudades, que a sus presos nunca abrió la cárcel?  
14:18 Todos los reyes de las naciones, todos ellos yacen con honra cada uno en su morada;  
14:19 pero tú echado eres de tu sepulcro como vástago abominable, como vestido de muertos pasados a espada, que descendieron al fondo de la sepultura; como cuerpo muerto hollado.  
14:20 No serás contado con ellos en la sepultura; porque tú destruiste tu tierra, mataste a tu pueblo. No será nombrada para siempre la descendencia de los malignos.  
14:21 Preparad sus hijos para el matadero, por la maldad de sus padres; no se levanten, ni posean la tierra, ni llenen de ciudades la faz del mundo.  
14:22 Porque yo me levantaré contra ellos, dice Jehová de los ejércitos, y raeré de Babilonia el nombre y el remanente, hijo y nieto, dice Jehová.  
14:23 Y la convertiré en posesión de erizos, y en lagunas de agua; y la barreré con escobas de destrucción, dice Jehová de los ejércitos. 
Asiria será destruida  
14:24 Jehová de los ejércitos juró diciendo: Ciertamente se hará de la manera que lo he pensado, y será confirmado como lo he determinado;  
14:25 que quebrantaré al asirio en mi tierra, y en mis montes lo hollaré; y su yugo será apartado de ellos, y su carga será quitada de su hombro.  
14:26 Este es el consejo que está acordado sobre toda la tierra, y esta, la mano extendida sobre todas las naciones.  
14:27 Porque Jehová de los ejércitos lo ha determinado, ¿y quién lo impedirá? Y su mano extendida, ¿quién la hará retroceder?  
Profecía sobre Filistea  
14:28 En el año que murió el rey Acaz fue esta profecía:  
14:29 No te alegres tú, Filistea toda, por haberse quebrado la vara del que te hería; porque de la raíz de la culebra saldrá áspid, y su fruto, serpiente voladora.  
14:30 Y los primogénitos de los pobres serán apacentados, y los menesterosos se acostarán confiados; mas yo haré morir de hambre tu raíz, y destruiré lo que de ti quedare.  
14:31 Aúlla, oh puerta; clama, oh ciudad; disuelta estás toda tú, Filistea; porque humo vendrá del norte, no quedará uno solo en sus asambleas.  
14:32 ¿Y qué se responderá a los mensajeros de las naciones? Que Jehová fundó a Sion, y que a ella se acogerán los afligidos de su pueblo.  

\chapter{15}

Profecía sobre Moab  

15:1 Profecía sobre Moab. Cierto, de noche fue destruida Ar de Moab, puesta en silencio. Cierto, de noche fue destruida Kir de Moab, reducida a silencio.  
15:2 Subió a Bayit y a Dibón, lugares altos, a llorar; sobre Nebo y sobre Medeba aullará Moab; toda cabeza de ella será rapada, y toda barba rasurada.  
15:3 Se ceñirán de cilicio en sus calles; en sus terrados y en sus plazas aullarán todos, deshaciéndose en llanto.  
15:4 Hesbón y Eleale gritarán, hasta Jahaza se oirá su voz; por lo que aullarán los guerreros de Moab, se lamentará el alma de cada uno dentro de él.  
15:5 Mi corazón dará gritos por Moab; sus fugitivos huirán hasta Zoar, como novilla de tres años. Por la cuesta de Luhit subirán llorando, y levantarán grito de quebrantamiento por el camino de Horonaim.  
15:6 Las aguas de Nimrim serán consumidas, y se secará la hierba, se marchitarán los retoños, todo verdor perecerá.  
15:7 Por tanto, las riquezas que habrán adquirido, y las que habrán reservado, las llevarán al torrente de los sauces.  
15:8 Porque el llanto rodeó los límites de Moab; hasta Eglaim llegó su alarido, y hasta Beer-elim su clamor.  
15:9 Y las aguas de Dimón se llenarán de sangre; porque yo traeré sobre Dimón males mayores, leones a los que escaparen de Moab, y a los sobrevivientes de la tierra.  

\chapter{16}


16:1 Enviad cordero al señor de la tierra, desde Sela del desierto al monte de la hija de Sion.  
16:2 Y cual ave espantada que huye de su nido, así serán las hijas de Moab en los vados de Arnón.  
16:3 Reúne consejo, haz juicio; pon tu sombra en medio del día como la noche; esconde a los desterrados, no entregues a los que andan errantes.  
16:4 Moren contigo mis desterrados, oh Moab; sé para ellos escondedero de la presencia del devastador; porque el atormentador fenecerá, el devastador tendrá fin, el pisoteador será consumido de sobre la tierra.  
16:5 Y se dispondrá el trono en misericordia; y sobre él se sentará firmemente, en el tabernáculo de David, quien juzgue y busque el juicio, y apresure la justicia.  
16:6 Hemos oído la soberbia de Moab; muy grandes son su soberbia, su arrogancia y su altivez; pero sus mentiras no serán firmes.  
16:7 Por tanto, aullará Moab, todo él aullará; gemiréis en gran manera abatidos, por las tortas de uvas de Kir-hareset.  
16:8 Porque los campos de Hesbón fueron talados, y las vides de Sibma; señores de naciones pisotearon sus generosos sarmientos; habían llegado hasta Jazer, y se habían extendido por el desierto; se extendieron sus plantas, pasaron el mar.  
16:9 Por lo cual lamentaré con lloro de Jazer por la viña de Sibma; te regaré con mis lágrimas, oh Hesbón y Eleale; porque sobre tus cosechas y sobre tu siega caerá el grito de guerra.  
16:10 Quitado es el gozo y la alegría del campo fértil; en las viñas no cantarán, ni se regocijarán; no pisará vino en los lagares el pisador; he hecho cesar el grito del lagarero.  
16:11 Por tanto, mis entrañas vibrarán como arpa por Moab, y mi corazón por Kir-hareset.  
16:12 Y cuando apareciere Moab cansado sobre los lugares altos, cuando venga a su santuario a orar, no le valdrá.  
16:13 Esta es la palabra que pronunció Jehová sobre Moab desde aquel tiempo;  
16:14 pero ahora Jehová ha hablado, diciendo: Dentro de tres años, como los años de un jornalero, será abatida la gloria de Moab, con toda su gran multitud; y los sobrevivientes serán pocos, pequeños y débiles. 

\chapter{17}

Profecía sobre Damasco  

17:1 Profecía sobre Damasco. He aquí que Damasco dejará de ser ciudad, y será montón de ruinas.  
17:2 Las ciudades de Aroer están desamparadas, en majadas se convertirán; dormirán allí, y no habrá quien los espante.  
17:3 Y cesará el socorro de Efraín, y el reino de Damasco; y lo que quede de Siria será como la gloria de los hijos de Israel, dice Jehová de los ejércitos.  
Juicio sobre Israel  
17:4 En aquel tiempo la gloria de Jacob se atenuará, y se enflaquecerá la grosura de su carne.  
17:5 Y será como cuando el segador recoge la mies, y con su brazo siega las espigas; será también como el que recoge espigas en el valle de Refaim.  
17:6 Y quedarán en él rebuscos, como cuando sacuden el olivo; dos o tres frutos en la punta de la rama, cuatro o cinco en sus ramas más fructíferas, dice Jehová Dios de Israel.  
17:7 En aquel día mirará el hombre a su Hacedor, y sus ojos contemplarán al Santo de Israel.  
17:8 Y no mirará a los altares que hicieron sus manos, ni mirará a lo que hicieron sus dedos, ni a los símbolos de Asera, ni a las imágenes del sol.  
17:9 En aquel día sus ciudades fortificadas serán como los frutos que quedan en los renuevos y en las ramas, los cuales fueron dejados a causa de los hijos de Israel; y habrá desolación.  
17:10 Porque te olvidaste del Dios de tu salvación, y no te acordaste de la roca de tu refugio; por tanto, sembrarás plantas hermosas, y plantarás sarmiento extraño.  
17:11 El día que las plantes, las harás crecer, y harás que su simiente brote de mañana; pero la cosecha será arrebatada en el día de la angustia, y del dolor desesperado.  
17:12 ¡Ay! multitud de muchos pueblos que harán ruido como estruendo del mar, y murmullo de naciones que harán alboroto como bramido de muchas aguas.  
17:13 Los pueblos harán estrépito como de ruido de muchas aguas; pero Dios los reprenderá, y huirán lejos; serán ahuyentados como el tamo de los montes delante del viento, y como el polvo delante del torbellino.  
17:14 Al tiempo de la tarde, he aquí la turbación, pero antes de la mañana el enemigo ya no existe. Esta es la parte de los que nos aplastan, y la suerte de los que nos saquean.  

\chapter{18}

Profecía sobre Etiopía  

18:1 ¡Ay de la tierra que hace sombra con las alas, que está tras los ríos de Etiopía; 
18:2 que envía mensajeros por el mar, y en naves de junco sobre las aguas! Andad, mensajeros veloces, a la nación de elevada estatura y tez brillante, al pueblo temible desde su principio y después, gente fuerte y conquistadora, cuya tierra es surcada por ríos.  
18:3 Vosotros, todos los moradores del mundo y habitantes de la tierra, cuando se levante bandera en los montes, mirad; y cuando se toque trompeta, escuchad.  
18:4 Porque Jehová me dijo así: Me estaré quieto, y los miraré desde mi morada, como sol claro después de la lluvia, como nube de rocío en el calor de la siega.  
18:5 Porque antes de la siega, cuando el fruto sea perfecto, y pasada la flor se maduren los frutos, entonces podará con podaderas las ramitas, y cortará y quitará las ramas.  
18:6 Y serán dejados todos para las aves de los montes y para las bestias de la tierra; sobre ellos tendrán el verano las aves, e invernarán todas las bestias de la tierra.  
18:7 En aquel tiempo será traída ofrenda a Jehová de los ejércitos, del pueblo de elevada estatura y tez brillante, del pueblo temible desde su principio y después, gente fuerte y conquistadora, cuya tierra es surcada por ríos, al lugar del nombre de Jehová de los ejércitos, al monte de Sion.  

\chapter{19}

Profecía sobre Egipto  
é
19:1 Profecía sobre Egipto. He aquí que Jehová monta sobre una ligera nube, y entrará en Egipto; y los ídolos de Egipto temblarán delante de él, y desfallecerá el corazón de los egipcios dentro de ellos.  
19:2 Levantaré egipcios contra egipcios, y cada uno peleará contra su hermano, cada uno contra su prójimo; ciudad contra ciudad, y reino contra reino.  
19:3 Y el espíritu de Egipto se desvanecerá en medio de él, y destruiré su consejo; y preguntarán a sus imágenes, a sus hechiceros, a sus evocadores y a sus adivinos.  
19:4 Y entregaré a Egipto en manos de señor duro, y rey violento se enseñoreará de ellos, dice el Señor, Jehová de los ejércitos.  
19:5 Y las aguas del mar faltarán, y el río se agotará y secará.  
19:6 Y se alejarán los ríos, se agotarán y secarán las corrientes de los fosos; la caña y el carrizo serán cortados.  
19:7 La pradera de junto al río, de junto a la ribera del río, y toda sementera del río, se secarán, se perderán, y no serán más.  
19:8 Los pescadores también se entristecerán; harán duelo todos los que echan anzuelo en el río, y desfallecerán los que extienden red sobre las aguas.  
19:9 Los que labran lino fino y los que tejen redes serán confundidos,  
19:10 porque todas sus redes serán rotas; y se entristecerán todos los que hacen viveros para peces.  
19:11 Ciertamente son necios los príncipes de Zoán; el consejo de los prudentes consejeros de Faraón se ha desvanecido. ¿Cómo diréis a Faraón: Yo soy hijo de los sabios, e hijo de los reyes antiguos?  
19:12 ¿Dónde están ahora aquellos tus sabios? Que te digan ahora, o te hagan saber qué es lo que Jehová de los ejércitos ha determinado sobre Egipto.  
19:13 Se han desvanecido los príncipes de Zoán, se han engañado los príncipes de Menfis; engañaron a Egipto los que son la piedra angular de sus familias.  
19:14 Jehová mezcló espíritu de vértigo en medio de él; e hicieron errar a Egipto en toda su obra, como tambalea el ebrio en su vómito.  
19:15 Y no aprovechará a Egipto cosa que haga la cabeza o la cola, la rama o el junco.  
19:16 En aquel día los egipcios serán como mujeres; porque se asombrarán y temerán en la presencia de la mano alta de Jehová de los ejércitos, que él levantará contra ellos.  
19:17 Y la tierra de Judá será de espanto a Egipto; todo hombre que de ella se acordare temerá por causa del consejo que Jehová de los ejércitos acordó sobre aquél.  
19:18 En aquel tiempo habrá cinco ciudades en la tierra de Egipto que hablen la lengua de Canaán, y que juren por Jehová de los ejércitos; una será llamada la ciudad de Herez.  
19:19 En aquel tiempo habrá altar para Jehová en medio de la tierra de Egipto, y monumento a Jehová junto a su frontera.  
19:20 Y será por señal y por testimonio a Jehová de los ejércitos en la tierra de Egipto; porque clamarán a Jehová a causa de sus opresores, y él les enviará salvador y príncipe que los libre.  
19:21 Y Jehová será conocido de Egipto, y los de Egipto conocerán a Jehová en aquel día, y harán sacrificio y oblación; y harán votos a Jehová, y los cumplirán.  
19:22 Y herirá Jehová a Egipto; herirá y sanará, y se convertirán a Jehová, y les será clemente y los sanará.  
19:23 En aquel tiempo habrá una calzada de Egipto a Asiria, y asirios entrarán en Egipto, y egipcios en Asiria; y los egipcios servirán con los asirios a Jehová.  
19:24 En aquel tiempo Israel será tercero con Egipto y con Asiria para bendición en medio de la tierra;  
19:25 porque Jehová de los ejércitos los bendecirá diciendo: Bendito el pueblo mío Egipto, y el asirio obra de mis manos, e Israel mi heredad. 
  

\chapter{20}

Predicción de la conquista de Egipto y de Etiopía por Asiria  

20:1 En el año que vino el Tartán a Asdod, cuando lo envió Sargón rey de Asiria, y peleó contra Asdod y la tomó;  
20:2 en aquel tiempo habló Jehová por medio de Isaías hijo de Amoz, diciendo: Ve y quita el cilicio de tus lomos, y descalza las sandalias de tus pies. Y lo hizo así, andando desnudo y descalzo.  
20:3 Y dijo Jehová: De la manera que anduvo mi siervo Isaías desnudo y descalzo tres años, por señal y pronóstico sobre Egipto y sobre Etiopía,  
20:4 así llevará el rey de Asiria a los cautivos de Egipto y los deportados de Etiopía, a jóvenes y a ancianos, desnudos y descalzos, y descubiertas las nalgas para vergüenza de Egipto.  
20:5 Y se turbarán y avergonzarán de Etiopía su esperanza, y de Egipto su gloria.  
20:6 Y dirá en aquel día el morador de esta costa: Mirad qué tal fue nuestra esperanza, a donde nos acogimos por socorro para ser libres de la presencia del rey de Asiria; ¿y cómo escaparemos nosotros?  

\chapter{21}

Profecía sobre el desierto del mar  

21:1 Profecía sobre el desierto del mar. Como torbellino del Neguev, así viene del desierto, de la tierra horrenda.  
21:2 Visión dura me ha sido mostrada. El prevaricador prevarica, y el destructor destruye. Sube, oh Elam; sitia, oh Media. Todo su gemido hice cesar.  
21:3 Por tanto, mis lomos se han llenado de dolor; angustias se apoderaron de mí, como angustias de mujer de parto; me agobié oyendo, y al ver me he espantado.  
21:4 Se pasmó mi corazón, el horror me ha intimidado; la noche de mi deseo se me volvió en espanto.  
21:5 Ponen la mesa, extienden tapices; comen, beben. ¡Levantaos, oh príncipes, ungid el escudo!  
21:6 Porque el Señor me dijo así: Ve, pon centinela que haga saber lo que vea.  
21:7 Y vio hombres montados, jinetes de dos en dos, montados sobre asnos, montados sobre camellos; y miró más atentamente,  
21:8 y gritó como un león: Señor, sobre la atalaya estoy yo continuamente de día, y las noches enteras sobre mi guarda;  
21:9 y he aquí vienen hombres montados, jinetes de dos en dos. Después habló y dijo: Cayó, cayó Babilonia; y todos los ídolos de sus dioses quebrantó en tierra.  
21:10 Oh pueblo mío, trillado y aventado, os he dicho lo que oí de Jehová de los ejércitos, Dios de Israel.  
Profecía sobre Duma  
21:11 Profecía sobre Duma. Me dan voces de Seir: Guarda, ¿qué de la noche? Guarda, ¿qué de la noche?  
21:12 El guarda respondió: La mañana viene, y después la noche; preguntad si queréis, preguntad; volved, venid.  
Profecía sobre Arabia  
21:13 Profecía sobre Arabia. En el bosque pasaréis la noche en Arabia, oh caminantes de Dedán.  
21:14 Salid a encontrar al sediento; llevadle agua, moradores de tierra de Tema, socorred con pan al que huye.  
21:15 Porque ante la espada huye, ante la espada desnuda, ante el arco entesado, ante el peso de la batalla.  
21:16 Porque así me ha dicho Jehová: De aquí a un año, semejante a años de jornalero, toda la gloria de Cedar será deshecha;  
21:17 y los sobrevivientes del número de los valientes flecheros, hijos de Cedar, serán reducidos; porque Jehová Dios de Israel lo ha dicho.  

\chapter{22}

Profecía sobre el valle de la visión  

22:1 Profecía sobre el valle de la visión. ¿Qué tienes ahora, que con todos los tuyos has subido sobre los terrados?  
22:2 Tú, llena de alborotos, ciudad turbulenta, ciudad alegre; tus muertos no son muertos a espada, ni muertos en guerra.  
22:3 Todos tus príncipes juntos huyeron del arco, fueron atados; todos los que en ti se hallaron, fueron atados juntamente, aunque habían huido lejos.  
22:4 Por esto dije: Dejadme, lloraré amargamente; no os afanéis por consolarme de la destrucción de la hija de mi pueblo.  
22:5 Porque día es de alboroto, de angustia y de confusión, de parte del Señor, Jehová de los ejércitos, en el valle de la visión, para derribar el muro, y clamar al monte.  
22:6 Y Elam tomó aljaba, con carros y con jinetes, y Kir sacó el escudo.  
22:7 Tus hermosos valles fueron llenos de carros, y los de a caballo acamparon a la puerta.  
22:8 Y desnudó la cubierta de Judá; y miraste en aquel día hacia la casa de armas del bosque.  
22:9 Visteis las brechas de la ciudad de David, que se multiplicaron; y recogisteis las aguas del estanque de abajo. 
22:10 Y contasteis las casas de Jerusalén, y derribasteis casas para fortificar el muro.  
22:11 Hicisteis foso entre los dos muros para las aguas del estanque viejo; y no tuvisteis respeto al que lo hizo, ni mirasteis de lejos al que lo labró.  
22:12 Por tanto, el Señor, Jehová de los ejércitos, llamó en este día a llanto y a endechas, a raparse el cabello y a vestir cilicio;  
22:13 y he aquí gozo y alegría, matando vacas y degollando ovejas, comiendo carne y bebiendo vino, diciendo: Comamos y bebamos, porque mañana moriremos. 
22:14 Esto fue revelado a mis oídos de parte de Jehová de los ejércitos: Que este pecado no os será perdonado hasta que muráis, dice el Señor, Jehová de los ejércitos.  
Sebna será sustituido por Eliaquim  
22:15 Jehová de los ejércitos dice así: Ve, entra a este tesorero, a Sebna el mayordomo, y dile:  
22:16 ¿Qué tienes tú aquí, o a quién tienes aquí, que labraste aquí sepulcro para ti, como el que en lugar alto labra su sepultura, o el que esculpe para sí morada en una peña?  
22:17 He aquí que Jehová te transportará en duro cautiverio, y de cierto te cubrirá el rostro.  
22:18 Te echará a rodar con ímpetu, como a bola por tierra extensa; allá morirás, y allá estarán los carros de tu gloria, oh vergüenza de la casa de tu señor.  
22:19 Y te arrojaré de tu lugar, y de tu puesto te empujaré.  
22:20 En aquel día llamaré a mi siervo Eliaquim hijo de Hilcías,  
22:21 y lo vestiré de tus vestiduras, y lo ceñiré de tu talabarte, y entregaré en sus manos tu potestad; y será padre al morador de Jerusalén, y a la casa de Judá.  
22:22 Y pondré la llave de la casa de David sobre su hombro; y abrirá, y nadie cerrará; cerrará, y nadie abrirá. 
22:23 Y lo hincaré como clavo en lugar firme; y será por asiento de honra a la casa de su padre.  
22:24 Colgarán de él toda la honra de la casa de su padre, los hijos y los nietos, todos los vasos menores, desde las tazas hasta toda clase de jarros.  
22:25 En aquel día, dice Jehová de los ejércitos, el clavo hincado en lugar firme será quitado; será quebrado y caerá, y la carga que sobre él se puso se echará a perder; porque Jehová habló. 

\chapter{23}

Profecía sobre Tiro  

23:1 Profecía sobre Tiro. Aullad, naves de Tarsis, porque destruida es Tiro hasta no quedar casa, ni a donde entrar; desde la tierra de Quitim les es revelado.  
23:2 Callad, moradores de la costa, mercaderes de Sidón, que pasando el mar te abastecían.  
23:3 Su provisión procedía de las sementeras que crecen con las muchas aguas del Nilo, de la mies del río. Fue también emporio de las naciones.  
23:4 Avergüénzate, Sidón, porque el mar, la fortaleza del mar habló, diciendo: Nunca estuve de parto, ni di a luz, ni crié jóvenes, ni levanté vírgenes.  
23:5 Cuando llegue la noticia a Egipto, tendrán dolor de las nuevas de Tiro.  
23:6 Pasaos a Tarsis; aullad, moradores de la costa.  
23:7 ¿No era ésta vuestra ciudad alegre, con muchos días de antigüedad? Sus pies la llevarán a morar lejos.  
23:8 ¿Quién decretó esto sobre Tiro, la que repartía coronas, cuyos negociantes eran príncipes, cuyos mercaderes eran los nobles de la tierra?  
23:9 Jehová de los ejércitos lo decretó, para envilecer la soberbia de toda gloria, y para abatir a todos los ilustres de la tierra.  
23:10 Pasa cual río de tu tierra, oh hija de Tarsis, porque no tendrás ya más poder.  
23:11 Extendió su mano sobre el mar, hizo temblar los reinos; Jehová mandó respecto a Canaán, que sus fortalezas sean destruidas.  
23:12 Y dijo: No te alegrarás más, oh oprimida virgen hija de Sidón. Levántate para pasar a Quitim, y aun allí no tendrás reposo.  
23:13 Mira la tierra de los caldeos. Este pueblo no existía; Asiria la fundó para los moradores del desierto. Levantaron sus fortalezas, edificaron sus palacios; él la convirtió en ruinas.  
23:14 Aullad, naves de Tarsis, porque destruida es vuestra fortaleza.  
23:15 Acontecerá en aquel día, que Tiro será puesta en olvido por setenta años, como días de un rey. Después de los setenta años, cantará Tiro canción como de ramera.  
23:16 Toma arpa, y rodea la ciudad, oh ramera olvidada; haz buena melodía, reitera la canción, para que seas recordada.  
23:17 Y acontecerá que al fin de los setenta años visitará Jehová a Tiro; y volverá a comerciar, y otra vez fornicará con todos los reinos del mundo sobre la faz de la tierra.  
23:18 Pero sus negocios y ganancias serán consagrados a Jehová; no se guardarán ni se atesorarán, porque sus ganancias serán para los que estuvieren delante de Jehová, para que coman hasta saciarse, y vistan espléndidamente. 

\chapter{24}

El juicio de Jehová sobre la tierra  

24:1 He aquí que Jehová vacía la tierra y la desnuda, y trastorna su faz, y hace esparcir a sus moradores.  
24:2 Y sucederá así como al pueblo, también al sacerdote; como al siervo, así a su amo; como a la criada, a su ama; como al que compra, al que vende; como al que presta, al que toma prestado; como al que da a logro, así al que lo recibe.  
24:3 La tierra será enteramente vaciada, y completamente saqueada; porque Jehová ha pronunciado esta palabra.  
24:4 Se destruyó, cayó la tierra; enfermó, cayó el mundo; enfermaron los altos pueblos de la tierra.  
24:5 Y la tierra se contaminó bajo sus moradores; porque traspasaron las leyes, falsearon el derecho, quebrantaron el pacto sempiterno.  
24:6 Por esta causa la maldición consumió la tierra, y sus moradores fueron asolados; por esta causa fueron consumidos los habitantes de la tierra, y disminuyeron los hombres.  
24:7 Se perdió el vino, enfermó la vid, gimieron todos los que eran alegres de corazón.  
24:8 Cesó el regocijo de los panderos, se acabó el estruendo de los que se alegran, cesó la alegría del arpa.  
24:9 No beberán vino con cantar; la sidra les será amarga a los que la bebieren.  
24:10 Quebrantada está la ciudad por la vanidad; toda casa se ha cerrado, para que no entre nadie.  
24:11 Hay clamores por falta de vino en las calles; todo gozo se oscureció, se desterró la alegría de la tierra.  
24:12 La ciudad quedó desolada, y con ruina fue derribada la puerta.  
24:13 Porque así será en medio de la tierra, en medio de los pueblos, como olivo sacudido, como rebuscos después de la vendimia.  
24:14 Estos alzarán su voz, cantarán gozosos por la grandeza de Jehová; desde el mar darán voces.  
24:15 Glorificad por esto a Jehová en los valles; en las orillas del mar sea nombrado Jehová Dios de Israel.  
24:16 De lo postrero de la tierra oímos cánticos: Gloria al justo. Y yo dije: ¡Mi desdicha, mi desdicha, ay de mí! Prevaricadores han prevaricado; y han prevaricado con prevaricación de desleales.  
24:17 Terror, foso y red sobre ti, oh morador de la tierra.  
24:18 Y acontecerá que el que huyere de la voz del terror caerá en el foso; y el que saliere de en medio del foso será preso en la red; porque de lo alto se abrirán ventanas, y temblarán los cimientos de la tierra.  
24:19 Será quebrantada del todo la tierra, enteramente desmenuzada será la tierra, en gran manera será la tierra conmovida.  
24:20 Temblará la tierra como un ebrio, y será removida como una choza; y se agravará sobre ella su pecado, y caerá, y nunca más se levantará.  
24:21 Acontecerá en aquel día, que Jehová castigará al ejército de los cielos en lo alto, y a los reyes de la tierra sobre la tierra.  
24:22 Y serán amontonados como se amontona a los encarcelados en mazmorra, y en prisión quedarán encerrados, y serán castigados después de muchos días.  
24:23 La luna se avergonzará, y el sol se confundirá, cuando Jehová de los ejércitos reine en el monte de Sion y en Jerusalén, y delante de sus ancianos sea glorioso.  

\chapter{25}

Cántico de alabanza por el favor de Jehová  

25:1 Jehová, tú eres mi Dios; te exaltaré, alabaré tu nombre, porque has hecho maravillas; tus consejos antiguos son verdad y firmeza.  
25:2 Porque convertiste la ciudad en montón, la ciudad fortificada en ruina; el alcázar de los extraños para que no sea ciudad, ni nunca jamás sea reedificado.  
25:3 Por esto te dará gloria el pueblo fuerte, te temerá la ciudad de gentes robustas. 
25:4 Porque fuiste fortaleza al pobre, fortaleza al menesteroso en su aflicción, refugio contra el turbión, sombra contra el calor; porque el ímpetu de los violentos es como turbión contra el muro.  
25:5 Como el calor en lugar seco, así humillarás el orgullo de los extraños; y como calor debajo de nube harás marchitar el renuevo de los robustos.  
25:6 Y Jehová de los ejércitos hará en este monte a todos los pueblos banquete de manjares suculentos, banquete de vinos refinados, de gruesos tuétanos y de vinos purificados.  
25:7 Y destruirá en este monte la cubierta con que están cubiertos todos los pueblos, y el velo que envuelve a todas las naciones.  
25:8 Destruirá a la muerte para siempre; y enjugará Jehová el Señor toda lágrima de todos los rostros; y quitará la afrenta de su pueblo de toda la tierra; porque Jehová lo ha dicho.  
25:9 Y se dirá en aquel día: He aquí, éste es nuestro Dios, le hemos esperado, y nos salvará; éste es Jehová a quien hemos esperado, nos gozaremos y nos alegraremos en su salvación.  
25:10 Porque la mano de Jehová reposará en este monte; pero Moab será hollado en su mismo sitio, como es hollada la paja en el muladar.  
25:11 Y extenderá su mano por en medio de él, como la extiende el nadador para nadar; y abatirá su soberbia y la destreza de sus manos;  
25:12 Y abatirá la fortaleza de tus altos muros; la humillará y la echará a tierra, hasta el polvo.  

\chapter{26}

Cántico de confianza en la protección de Jehová  

26:1 En aquel día cantarán este cántico en tierra de Judá: Fuerte ciudad tenemos; salvación puso Dios por muros y antemuro.  
26:2 Abrid las puertas, y entrará la gente justa, guardadora de verdades.  
26:3 Tú guardarás en completa paz a aquel cuyo pensamiento en ti persevera; porque en ti ha confiado.  
26:4 Confiad en Jehová perpetuamente, porque en Jehová el Señor está la fortaleza de los siglos.  
26:5 Porque derribó a los que moraban en lugar sublime; humilló a la ciudad exaltada, la humilló hasta la tierra, la derribó hasta el polvo.  
26:6 La hollará pie, los pies del afligido, los pasos de los menesterosos.  
26:7 El camino del justo es rectitud; tú, que eres recto, pesas el camino del justo.  
26:8 También en el camino de tus juicios, oh Jehová, te hemos esperado; tu nombre y tu memoria son el deseo de nuestra alma.  
26:9 Con mi alma te he deseado en la noche, y en tanto que me dure el espíritu dentro de mí, madrugaré a buscarte; porque luego que hay juicios tuyos en la tierra, los moradores del mundo aprenden justicia.  
26:10 Se mostrará piedad al malvado, y no aprenderá justicia; en tierra de rectitud hará iniquidad, y no mirará a la majestad de Jehová.  
26:11 Jehová, tu mano está alzada, pero ellos no ven; verán al fin, y se avergonzarán los que envidian a tu pueblo; y a tus enemigos fuego los consumirá. 
26:12 Jehová, tú nos darás paz, porque también hiciste en nosotros todas nuestras obras.  
26:13 Jehová Dios nuestro, otros señores fuera de ti se han enseñoreado de nosotros; pero en ti solamente nos acordaremos de tu nombre.  
26:14 Muertos son, no vivirán; han fallecido, no resucitarán; porque los castigaste, y destruiste y deshiciste todo su recuerdo.  
26:15 Aumentaste el pueblo, oh Jehová, aumentaste el pueblo; te hiciste glorioso; ensanchaste todos los confines de la tierra.  
26:16 Jehová, en la tribulación te buscaron; derramaron oración cuando los castigaste.  
26:17 Como la mujer encinta cuando se acerca el alumbramiento gime y da gritos en sus dolores, así hemos sido delante de ti, oh Jehová.  
26:18 Concebimos, tuvimos dolores de parto, dimos a luz viento; ninguna liberación hicimos en la tierra, ni cayeron los moradores del mundo.  
26:19 Tus muertos vivirán; sus cadáveres resucitarán. ¡Despertad y cantad, moradores del polvo! porque tu rocío es cual rocío de hortalizas, y la tierra dará sus muertos.  
26:20 Anda, pueblo mío, entra en tus aposentos, cierra tras ti tus puertas; escóndete un poquito, por un momento, en tanto que pasa la indignación.  
26:21 Porque he aquí que Jehová sale de su lugar para castigar al morador de la tierra por su maldad contra él; y la tierra descubrirá la sangre derramada sobre ella, y no encubrirá ya más a sus muertos.  

\chapter{27}

Liberación y regreso de Israel  

27:1 En aquel día Jehová castigará con su espada dura, grande y fuerte al leviatán serpiente veloz, y al leviatán serpiente tortuosa; y matará al dragón que está en el mar.  
27:2 En aquel día cantad acerca de la viña del vino rojo.  
27:3 Yo Jehová la guardo, cada momento la regaré; la guardaré de noche y de día, para que nadie la dañe.  
27:4 No hay enojo en mí. ¿Quién pondrá contra mí en batalla espinos y cardos? Yo los hollaré, los quemaré a una.  
27:5 ¿O forzará alguien mi fortaleza? Haga conmigo paz; sí, haga paz conmigo.  
27:6 Días vendrán cuando Jacob echará raíces, florecerá y echará renuevos Israel, y la faz del mundo llenará de fruto.  
27:7 ¿Acaso ha sido herido como quien lo hirió, o ha sido muerto como los que lo mataron?  
27:8 Con medida lo castigarás en sus vástagos. El los remueve con su recio viento en el día del aire solano.  
27:9 De esta manera, pues, será perdonada la iniquidad de Jacob, y este será todo el fruto, la remoción de su pecado; cuando haga todas las piedras del altar como piedras de cal desmenuzadas, y no se levanten los símbolos de Asera ni las imágenes del sol.  
27:10 Porque la ciudad fortificada será desolada, la ciudad habitada será abandonada y dejada como un desierto; allí pastará el becerro, allí tendrá su majada, y acabará sus ramas.  
27:11 Cuando sus ramas se sequen, serán quebradas; mujeres vendrán a encenderlas; porque aquel no es pueblo de entendimiento; por tanto, su Hacedor no tendrá de él misericordia, ni se compadecerá de él el que lo formó.  
27:12 Acontecerá en aquel día, que trillará Jehová desde el río Eufrates hasta el torrente de Egipto, y vosotros, hijos de Israel, seréis reunidos uno a uno.  
27:13 Acontecerá también en aquel día, que se tocará con gran trompeta, y vendrán los que habían sido esparcidos en la tierra de Asiria, y los que habían sido desterrados a Egipto, y adorarán a Jehová en el monte santo, en Jerusalén.  

\chapter{28}

Condenación de Efraín  

28:1 ¡Ay de la corona de soberbia de los ebrios de Efraín, y de la flor caduca de la hermosura de su gloria, que está sobre la cabeza del valle fértil de los aturdidos del vino!  
28:2 He aquí, Jehová tiene uno que es fuerte y poderoso; como turbión de granizo y como torbellino trastornador, como ímpetu de recias aguas que inundan, con fuerza derriba a tierra.  
28:3 Con los pies será pisoteada la corona de soberbia de los ebrios de Efraín.  
28:4 Y será la flor caduca de la hermosura de su gloria que está sobre la cabeza del valle fértil, como la fruta temprana, la primera del verano, la cual, apenas la ve el que la mira, se la traga tan luego como la tiene a mano.  
28:5 En aquel día Jehová de los ejércitos será por corona de gloria y diadema de hermosura al remanente de su pueblo;  
28:6 y por espíritu de juicio al que se sienta en juicio, y por fuerzas a los que rechacen la batalla en la puerta.  
28:7 Pero también éstos erraron con el vino, y con sidra se entontecieron; el sacerdote y el profeta erraron con sidra, fueron trastornados por el vino; se aturdieron con la sidra, erraron en la visión, tropezaron en el juicio.  
28:8 Porque toda mesa está llena de vómito y suciedad, hasta no haber lugar limpio.  
28:9 ¿A quién se enseñará ciencia, o a quién se hará entender doctrina? ¿A los destetados? ¿a los arrancados de los pechos?  
28:10 Porque mandamiento tras mandamiento, mandato sobre mandato, renglón tras renglón, línea sobre línea, un poquito allí, otro poquito allá;  
28:11 porque en lengua de tartamudos, y en extraña lengua hablará a este pueblo,  
28:12 a los cuales él dijo: Este es el reposo; dad reposo al cansado; y este es el refrigerio; mas no quisieron oir. 
28:13 La palabra, pues, de Jehová les será mandamiento tras mandamiento, mandato sobre mandato, renglón tras renglón, línea sobre línea, un poquito allí, otro poquito allá; hasta que vayan y caigan de espaldas, y sean quebrantados, enlazados y presos.  
Amonestación a Jerusalén 
28:14 Por tanto, varones burladores que gobernáis a este pueblo que está en Jerusalén, oíd la palabra de Jehová.  
28:15 Por cuanto habéis dicho: Pacto tenemos hecho con la muerte, e hicimos convenio con el Seol; cuando pase el turbión del azote, no llegará a nosotros, porque hemos puesto nuestro refugio en la mentira, y en la falsedad nos esconderemos;  
28:16 por tanto, Jehová el Señor dice así: He aquí que yo he puesto en Sion por fundamento una piedra, piedra probada, angular, preciosa, de cimiento estable; el que creyere, no se apresure.  
28:17 Y ajustaré el juicio a cordel, y a nivel la justicia; y granizo barrerá el refugio de la mentira, y aguas arrollarán el escondrijo.  
28:18 Y será anulado vuestro pacto con la muerte, y vuestro convenio con el Seol no será firme; cuando pase el turbión del azote, seréis de él pisoteados.  
28:19 Luego que comience a pasar, él os arrebatará; porque de mañana en mañana pasará, de día y de noche; y será ciertamente espanto el entender lo oído.  
28:20 La cama será corta para poder estirarse, y la manta estrecha para poder envolverse.  
28:21 Porque Jehová se levantará como en el monte Perazim, como en el valle de Gabaón se enojará; para hacer su obra, su extraña obra, y para hacer su operación, su extraña operación.  
28:22 Ahora, pues, no os burléis, para que no se aprieten más vuestras ataduras; porque destrucción ya determinada sobre toda la tierra he oído del Señor, Jehová de los ejércitos.  
28:23 Estad atentos, y oíd mi voz; atended, y oíd mi dicho.  
28:24 El que ara para sembrar, ¿arará todo el día? ¿Romperá y quebrará los terrones de la tierra?  
28:25 Cuando ha igualado su superficie, ¿no derrama el eneldo, siembra el comino, pone el trigo en hileras, y la cebada en el lugar señalado, y la avena en su borde apropiado?  
28:26 Porque su Dios le instruye, y le enseña lo recto;  
28:27 que el eneldo no se trilla con trillo, ni sobre el comino se pasa rueda de carreta; sino que con un palo se sacude el eneldo, y el comino con una vara.  
28:28 El grano se trilla; pero no lo trillará para siempre, ni lo comprime con la rueda de su carreta, ni lo quebranta con los dientes de su trillo.  
28:29 También esto salió de Jehová de los ejércitos, para hacer maravilloso el consejo y engrandecer la sabiduría.  

\chapter{29}

Ariel y sus enemigos  

29:1 ¡Ay de Ariel, de Ariel, ciudad donde habitó David! Añadid un año a otro, las fiestas sigan su curso.  
29:2 Mas yo pondré a Ariel en apretura, y será desconsolada y triste; y será a mí como Ariel.  
29:3 Porque acamparé contra ti alrededor, y te sitiaré con campamentos, y levantaré contra ti baluartes.  
29:4 Entonces serás humillada, hablarás desde la tierra, y tu habla saldrá del polvo; y será tu voz de la tierra como la de un fantasma, y tu habla susurrará desde el polvo.  
29:5 Y la muchedumbre de tus enemigos será como polvo menudo, y la multitud de los fuertes como tamo que pasa; y será repentinamente, en un momento.  
29:6 Por Jehová de los ejércitos serás visitada con truenos, con terremotos y con gran ruido, con torbellino y tempestad, y llama de fuego consumidor.  
29:7 Y será como sueño de visión nocturna la multitud de todas las naciones que pelean contra Ariel, y todos los que pelean contra ella y su fortaleza, y los que la ponen en apretura.  
29:8 Y les sucederá como el que tiene hambre y sueña, y le parece que come, pero cuando despierta, su estómago está vacío; o como el que tiene sed y sueña, y le parece que bebe, pero cuando despierta, se halla cansado y sediento; así será la multitud de todas las naciones que pelearán contra el monte de Sion.  
Ceguera e hipocresía de Israel  
29:9 Deteneos y maravillaos; ofuscaos y cegaos; embriagaos, y no de vino; tambalead, y no de sidra.  
29:10 Porque Jehová derramó sobre vosotros espíritu de sueño, y cerró los ojos de vuestros profetas, y puso velo sobre las cabezas de vuestros videntes.  
29:11 Y os será toda visión como palabras de libro sellado, el cual si dieren al que sabe leer, y le dijeren: Lee ahora esto; él dirá: No puedo, porque está sellado.  
29:12 Y si se diere el libro al que no sabe leer, diciéndole: Lee ahora esto; él dirá: No sé leer.  
29:13 Dice, pues, el Señor: Porque este pueblo se acerca a mí con su boca, y con sus labios me honra, pero su corazón está lejos de mí, y su temor de mí no es más que un mandamiento de hombres que les ha sido enseñado;  
29:14 por tanto, he aquí que nuevamente excitaré yo la admiración de este pueblo con un prodigio grande y espantoso; porque perecerá la sabiduría de sus sabios, y se desvanecerá la inteligencia de sus entendidos. 
29:15 ¡Ay de los que se esconden de Jehová, encubriendo el consejo, y sus obras están en tinieblas, y dicen: ¿Quién nos ve, y quién nos conoce?  
29:16 Vuestra perversidad ciertamente será reputada como el barro del alfarero. ¿Acaso la obra dirá de su hacedor: No me hizo? ¿Dirá la vasija de aquel que la ha formado: No entendió?  
Redención de Israel  
29:17 ¿No se convertirá de aquí a muy poco tiempo el Líbano en campo fructífero, y el campo fértil será estimado por bosque?  
29:18 En aquel tiempo los sordos oirán las palabras del libro, y los ojos de los ciegos verán en medio de la oscuridad y de las tinieblas.  
29:19 Entonces los humildes crecerán en alegría en Jehová, y aun los más pobres de los hombres se gozarán en el Santo de Israel.  
29:20 Porque el violento será acabado, y el escarnecedor será consumido; serán destruidos todos los que se desvelan para hacer iniquidad,  
29:21 los que hacen pecar al hombre en palabra; los que arman lazo al que reprendía en la puerta, y pervierten la causa del justo con vanidad.  
29:22 Por tanto, Jehová, que redimió a Abraham, dice así a la casa de Jacob: No será ahora avergonzado Jacob, ni su rostro se pondrá pálido;  
29:23 porque verá a sus hijos, obra de mis manos en medio de ellos, que santificarán mi nombre; y santificarán al Santo de Jacob, y temerán al Dios de Israel.  
29:24 Y los extraviados de espíritu aprenderán inteligencia, y los murmuradores aprenderán doctrina.  

\chapter{30}

La futilidad de confiar en Egipto  
é
30:1 ¡Ay de los hijos que se apartan, dice Jehová, para tomar consejo, y no de mí; para cobijarse con cubierta, y no de mi espíritu, añadiendo pecado a pecado!  
30:2 Que se apartan para descender a Egipto, y no han preguntado de mi boca; para fortalecerse con la fuerza de Faraón, y poner su esperanza en la sombra de Egipto.  
30:3 Pero la fuerza de Faraón se os cambiará en vergüenza, y el amparo en la sombra de Egipto en confusión.  
30:4 Cuando estén sus príncipes en Zoán, y sus embajadores lleguen a Hanes,  
30:5 todos se avergonzarán del pueblo que no les aprovecha, ni los socorre, ni les trae provecho; antes les será para vergüenza y aun para oprobio.  
30:6 Profecía sobre las bestias del Neguev: Por tierra de tribulación y de angustia, de donde salen la leona y el león, la víbora y la serpiente que vuela, llevan sobre lomos de asnos sus riquezas, y sus tesoros sobre jorobas de camellos, a un pueblo que no les será de provecho.  
30:7 Ciertamente Egipto en vano e inútilmente dará ayuda; por tanto yo le di voces, que su fortaleza sería estarse quietos.  
30:8 Ve, pues, ahora, y escribe esta visión en una tabla delante de ellos, y regístrala en un libro, para que quede hasta el día postrero, eternamente y para siempre.  
30:9 Porque este pueblo es rebelde, hijos mentirosos, hijos que no quisieron oír la ley de Jehová;  
30:10 que dicen a los videntes: No veáis; y a los profetas: No nos profeticéis lo recto, decidnos cosas halagüeñas, profetizad mentiras;  
30:11 dejad el camino, apartaos de la senda, quitad de nuestra presencia al Santo de Israel.  
30:12 Por tanto, el Santo de Israel dice así: Porque desechasteis esta palabra, y confiasteis en violencia y en iniquidad, y en ello os habéis apoyado;  
30:13 por tanto, os será este pecado como grieta que amenaza ruina, extendiéndose en una pared elevada, cuya caída viene súbita y repentinamente.  
30:14 Y se quebrará como se quiebra un vaso de alfarero, que sin misericordia lo hacen pedazos; tanto, que entre los pedazos no se halla tiesto para traer fuego del hogar, o para sacar agua del pozo.  
30:15 Porque así dijo Jehová el Señor, el Santo de Israel: En descanso y en reposo seréis salvos; en quietud y en confianza será vuestra fortaleza. Y no quisisteis,  
30:16 sino que dijisteis: No, antes huiremos en caballos; por tanto, vosotros huiréis. Sobre corceles veloces cabalgaremos; por tanto, serán veloces vuestros perseguidores.  
30:17 Un millar huirá a la amenaza de uno; a la amenaza de cinco huiréis vosotros todos, hasta que quedéis como mástil en la cumbre de un monte, y como bandera sobre una colina. 
Promesa de la gracia de Dios a Israel  
30:18 Por tanto, Jehová esperará para tener piedad de vosotros, y por tanto, será exaltado teniendo de vosotros misericordia; porque Jehová es Dios justo; bienaventurados todos los que confían en él.  
30:19 Ciertamente el pueblo morará en Sion, en Jerusalén; nunca más llorarás; el que tiene misericordia se apiadará de ti; al oír la voz de tu clamor te responderá.  
30:20 Bien que os dará el Señor pan de congoja y agua de angustia, con todo, tus maestros nunca más te serán quitados, sino que tus ojos verán a tus maestros.  
30:21 Entonces tus oídos oirán a tus espaldas palabra que diga: Este es el camino, andad por él; y no echéis a la mano derecha, ni tampoco torzáis a la mano izquierda.  
30:22 Entonces profanarás la cubierta de tus esculturas de plata, y la vestidura de tus imágenes fundidas de oro; las apartarás como trapo asqueroso; ¡Sal fuera! les dirás.  
30:23 Entonces dará el Señor lluvia a tu sementera, cuando siembres la tierra, y dará pan del fruto de la tierra, y será abundante y pingüe; tus ganados en aquel tiempo serán apacentados en espaciosas dehesas.  
30:24 Tus bueyes y tus asnos que labran la tierra comerán grano limpio, aventado con pala y criba.  
30:25 Y sobre todo monte alto, y sobre todo collado elevado, habrá ríos y corrientes de aguas el día de la gran matanza, cuando caerán las torres.  
30:26 Y la luz de la luna será como la luz del sol, y la luz del sol siete veces mayor, como la luz de siete días, el día que vendare Jehová la herida de su pueblo, y curare la llaga que él causó.  
El juicio de Jehová sobre Asiria  
30:27 He aquí que el nombre de Jehová viene de lejos; su rostro encendido, y con llamas de fuego devorador; sus labios llenos de ira, y su lengua como fuego que consume.  
30:28 Su aliento, cual torrente que inunda; llegará hasta el cuello, para zarandear a las naciones con criba de destrucción; y el freno estará en las quijadas de los pueblos, haciéndoles errar.  
30:29 Vosotros tendréis cántico como de noche en que se celebra pascua, y alegría de corazón, como el que va con flauta para venir al monte de Jehová, al Fuerte de Israel.  
30:30 Y Jehová hará oír su potente voz, y hará ver el descenso de su brazo, con furor de rostro y llama de fuego consumidor, con torbellino, tempestad y piedra de granizo.  
30:31 Porque Asiria que hirió con vara, con la voz de Jehová será quebrantada.  
30:32 Y cada golpe de la vara justiciera que asiente Jehová sobre él, será con panderos y con arpas; y en batalla tumultuosa peleará contra ellos.  
30:33 Porque Tofet ya de tiempo está dispuesto y preparado para el rey, profundo y ancho, cuya pira es de fuego, y mucha leña; el soplo de Jehová, como torrente de azufre, lo enciende.  

\chapter{31}

Los egipcios son hombres y no dioses  

31:1 ¡Ay de los que descienden a Egipto por ayuda, y confían en caballos; y su esperanza ponen en carros, porque son muchos, y en jinetes, porque son valientes; y no miran al Santo de Israel, ni buscan a Jehová!  
31:2 Pero él también es sabio, y traerá el mal, y no retirará sus palabras. Se levantará, pues, contra la casa de los malignos, y contra el auxilio de los que hacen iniquidad.  
31:3 Y los egipcios hombres son, y no Dios; y sus caballos carne, y no espíritu; de manera que al extender Jehová su mano, caerá el ayudador y caerá el ayudado, y todos ellos desfallecerán a una.  
31:4 Porque Jehová me dijo a mí de esta manera: Como el león y el cachorro de león ruge sobre la presa, y si se reúne cuadrilla de pastores contra él, no lo espantarán sus voces, ni se acobardará por el tropel de ellos; así Jehová de los ejércitos descenderá a pelear sobre el monte de Sion, y sobre su collado.  
31:5 Como las aves que vuelan, así amparará Jehová de los ejércitos a Jerusalén, amparando, librando, preservando y salvando.  
31:6 Volved a aquel contra quien se rebelaron profundamente los hijos de Israel.  
31:7 Porque en aquel día arrojará el hombre sus ídolos de plata y sus ídolos de oro, que para vosotros han hecho vuestras manos pecadoras. 
31:8 Entonces caerá Asiria por espada no de varón, y la consumirá espada no de hombre; y huirá de la presencia de la espada, y sus jóvenes serán tributarios.  
31:9 Y de miedo pasará su fortaleza, y sus príncipes, con pavor, dejarán sus banderas, dice Jehová, cuyo fuego está en Sion, y su horno en Jerusalén.  

\chapter{32}

El Rey justo  

32:1 He aquí que para justicia reinará un rey, y príncipes presidirán en juicio.  
32:2 Y será aquel varón como escondedero contra el viento, y como refugio contra el turbión; como arroyos de aguas en tierra de sequedad, como sombra de gran peñasco en tierra calurosa.  
32:3 No se ofuscarán entonces los ojos de los que ven, y los oídos de los oyentes oirán atentos.  
32:4 Y el corazón de los necios entenderá para saber, y la lengua de los tartamudos hablará rápida y claramente.  
32:5 El ruin nunca más será llamado generoso, ni el tramposo será llamado espléndido.  
32:6 Porque el ruin hablará ruindades, y su corazón fabricará iniquidad, para cometer impiedad y para hablar escarnio contra Jehová, dejando vacía el alma hambrienta, y quitando la bebida al sediento.  
32:7 Las armas del tramposo son malas; trama intrigas inicuas para enredar a los simples con palabras mentirosas, y para hablar en juicio contra el pobre.  
32:8 Pero el generoso pensará generosidades, y por generosidades será exaltado.  
Advertencia a las mujeres de Jerusalén  
32:9 Mujeres indolentes, levantaos, oíd mi voz; hijas confiadas, escuchad mi razón.  
32:10 De aquí a algo más de un año tendréis espanto, oh confiadas; porque la vendimia faltará, y la cosecha no vendrá.  
32:11 Temblad, oh indolentes; turbaos, oh confiadas; despojaos, desnudaos, ceñid los lomos con cilicio.  
32:12 Golpeándose el pecho lamentarán por los campos deleitosos, por la vid fértil.  
32:13 Sobre la tierra de mi pueblo subirán espinos y cardos, y aun sobre todas las casas en que hay alegría en la ciudad de alegría.  
32:14 Porque los palacios quedarán desiertos, la multitud de la ciudad cesará; las torres y fortalezas se volverán cuevas para siempre, donde descansen asnos monteses, y ganados hagan majada;  
32:15 hasta que sobre nosotros sea derramado el Espíritu de lo alto, y el desierto se convierta en campo fértil, y el campo fértil sea estimado por bosque.  
32:16 Y habitará el juicio en el desierto, y en el campo fértil morará la justicia.  
32:17 Y el efecto de la justicia será paz; y la labor de la justicia, reposo y seguridad para siempre.  
32:18 Y mi pueblo habitará en morada de paz, en habitaciones seguras, y en recreos de reposo.  
32:19 Y cuando caiga granizo, caerá en los montes; y la ciudad será del todo abatida.  
32:20 Dichosos vosotros los que sembráis junto a todas las aguas, y dejáis libres al buey y al asno.  

\chapter{33}

Jehová traerá salvación  

33:1 ¡Ay de ti, que saqueas, y nunca fuiste saqueado; que haces deslealtad, bien que nadie contra ti la hizo! Cuando acabes de saquear, serás tú saqueado; y cuando acabes de hacer deslealtad, se hará contra ti.  
33:2 Oh Jehová, ten misericordia de nosotros, a ti hemos esperado; tú, brazo de ellos en la mañana, sé también nuestra salvación en tiempo de la tribulación.  
33:3 Los pueblos huyeron a la voz del estruendo; las naciones fueron esparcidas al levantarte tú.  
33:4 Sus despojos serán recogidos como cuando recogen orugas; correrán sobre ellos como de una a otra parte corren las langostas.  
33:5 Será exaltado Jehová, el cual mora en las alturas; llenó a Sion de juicio y de justicia.  
33:6 Y reinarán en tus tiempos la sabiduría y la ciencia, y abundancia de salvación; el temor de Jehová será su tesoro.  
33:7 He aquí que sus embajadores darán voces afuera; los mensajeros de paz llorarán amargamente.  
33:8 Las calzadas están deshechas, cesaron los caminantes; ha anulado el pacto, aborreció las ciudades, tuvo en nada a los hombres.  
33:9 Se enlutó, enfermó la tierra; el Líbano se avergonzó, y fue cortado; Sarón se ha vuelto como desierto, y Basán y el Carmelo fueron sacudidos.  
33:10 Ahora me levantaré, dice Jehová; ahora seré exaltado, ahora seré engrandecido.  
33:11 Concebisteis hojarascas, rastrojo daréis a luz; el soplo de vuestro fuego os consumirá.  
33:12 Y los pueblos serán como cal quemada; como espinos cortados serán quemados con fuego.  
33:13 Oíd, los que estáis lejos, lo que he hecho; y vosotros los que estáis cerca, conoced mi poder.  
33:14 Los pecadores se asombraron en Sion, espanto sobrecogió a los hipócritas. ¿Quién de nosotros morará con el fuego consumidor? ¿Quién de nosotros habitará con las llamas eternas?  
33:15 El que camina en justicia y habla lo recto; el que aborrece la ganancia de violencias, el que sacude sus manos para no recibir cohecho, el que tapa sus oídos para no oír propuestas sanguinarias; el que cierra sus ojos para no ver cosa mala;  
33:16 éste habitará en las alturas; fortaleza de rocas será su lugar de refugio; se le dará su pan, y sus aguas serán seguras.  
33:17 Tus ojos verán al Rey en su hermosura; verán la tierra que está lejos.  
33:18 Tu corazón imaginará el espanto, y dirá: ¿Qué es del escriba? ¿qué del pesador del tributo? ¿qué del que pone en lista las casas más insignes?  
33:19 No verás a aquel pueblo orgulloso, pueblo de lengua difícil de entender, de lengua tartamuda que no comprendas.  
33:20 Mira a Sion, ciudad de nuestras fiestas solemnes; tus ojos verán a Jerusalén, morada de quietud, tienda que no será desarmada, ni serán arrancadas sus estacas, ni ninguna de sus cuerdas será rota. 
33:21 Porque ciertamente allí será Jehová para con nosotros fuerte, lugar de ríos, de arroyos muy anchos, por el cual no andará galera de remos, ni por él pasará gran nave.  
33:22 Porque Jehová es nuestro juez, Jehová es nuestro legislador, Jehová es nuestro Rey; él mismo nos salvará.  
33:23 Tus cuerdas se aflojaron; no afirmaron su mástil, ni entesaron la vela; se repartirá entonces botín de muchos despojos; los cojos arrebatarán el botín.  
33:24 No dirá el morador: Estoy enfermo; al pueblo que more en ella le será perdonada la iniquidad.  
  

\chapter{34}

La ira de Jehová contra las naciones  

34:1 Acercaos, naciones, juntaos para oír; y vosotros, pueblos, escuchad. Oiga la tierra y cuanto hay en ella, el mundo y todo lo que produce.  
34:2 Porque Jehová está airado contra todas las naciones, e indignado contra todo el ejército de ellas; las destruirá y las entregará al matadero.  
34:3 Y los muertos de ellas serán arrojados, y de sus cadáveres se levantará hedor; y los montes se disolverán por la sangre de ellos.  
34:4 Y todo el ejército de los cielos se disolverá, y se enrollarán los cielos como un libro; y caerá todo su ejército, como se cae la hoja de la parra, y como se cae la de la higuera. 
34:5 Porque en los cielos se embriagará mi espada; he aquí que descenderá sobre Edom en juicio, y sobre el pueblo de mi anatema.  
34:6 Llena está de sangre la espada de Jehová, engrasada está de grosura, de sangre de corderos y de machos cabríos, de grosura de riñones de carneros; porque Jehová tiene sacrificios en Bosra, y grande matanza en tierra de Edom. 
34:7 Y con ellos caerán búfalos, y toros con becerros; y su tierra se embriagará de sangre, y su polvo se engrasará de grosura.  
34:8 Porque es día de venganza de Jehová, año de retribuciones en el pleito de Sion.  
34:9 Y sus arroyos se convertirán en brea, y su polvo en azufre, y su tierra en brea ardiente.  
34:10 No se apagará de noche ni de día, perpetuamente subirá su humo; de generación en generación será asolada, nunca jamás pasará nadie por ella.  
34:11 Se adueñarán de ella el pelícano y el erizo, la lechuza y el cuervo morarán en ella; y se extenderá sobre ella cordel de destrucción, y niveles de asolamiento. 
34:12 Llamarán a sus príncipes, príncipes sin reino; y todos sus grandes serán nada.  
34:13 En sus alcázares crecerán espinos, y ortigas y cardos en sus fortalezas; y serán morada de chacales, y patio para los pollos de los avestruces.  
34:14 Las fieras del desierto se encontrarán con las hienas, y la cabra salvaje gritará a su compañero; la lechuza también tendrá allí morada, y hallará para sí reposo.  
34:15 Allí anidará el buho, pondrá sus huevos, y sacará sus pollos, y los juntará debajo de sus alas; también se juntarán allí buitres, cada uno con su compañera.  
34:16 Inquirid en el libro de Jehová, y leed si faltó alguno de ellos; ninguno faltó con su compañera; porque su boca mandó, y los reunió su mismo Espíritu.  
34:17 Y él les echó suertes, y su mano les repartió con cordel; para siempre la tendrán por heredad; de generación en generación morarán allí.  

\chapter{35}

Futuro glorioso de Sion  

35:1 Se alegrarán el desierto y la soledad; el yermo se gozará y florecerá como la rosa.  
35:2 Florecerá profusamente, y también se alegrará y cantará con júbilo; la gloria del Líbano le será dada, la hermosura del Carmelo y de Sarón. Ellos verán la gloria de Jehová, la hermosura del Dios nuestro.  
35:3 Fortaleced las manos cansadas, afirmad las rodillas endebles. 
35:4 Decid a los de corazón apocado: Esforzaos, no temáis; he aquí que vuestro Dios viene con retribución, con pago; Dios mismo vendrá, y os salvará.  
35:5 Entonces los ojos de los ciegos serán abiertos, y los oídos de los sordos se abrirán.  
35:6 Entonces el cojo saltará como un ciervo, y cantará la lengua del mudo; porque aguas serán cavadas en el desierto, y torrentes en la soledad.  
35:7 El lugar seco se convertirá en estanque, y el sequedal en manaderos de aguas; en la morada de chacales, en su guarida, será lugar de cañas y juncos.  
35:8 Y habrá allí calzada y camino, y será llamado Camino de Santidad; no pasará inmundo por él, sino que él mismo estará con ellos; el que anduviere en este camino, por torpe que sea, no se extraviará.  
35:9 No habrá allí león, ni fiera subirá por él, ni allí se hallará, para que caminen los redimidos.  
35:10 Y los redimidos de Jehová volverán, y vendrán a Sion con alegría; y gozo perpetuo será sobre sus cabezas; y tendrán gozo y alegría, y huirán la tristeza y el gemido. 

\chapter{36}

La invasión de Senaquerib  

36:1 Aconteció en el año catorce del rey Ezequías, que Senaquerib rey de Asiria subió contra todas las ciudades fortificadas de Judá, y las tomó.  
36:2 Y el rey de Asiria envió al Rabsaces con un gran ejército desde Laquis a Jerusalén contra el rey Ezequías; y acampó junto al acueducto del estanque de arriba, en el camino de la heredad del Lavador.  
36:3 Y salió a él Eliaquim hijo de Hilcías, mayordomo, y Sebna, escriba, y Joa hijo de Asaf, canciller,  
36:4 a los cuales dijo el Rabsaces: Decid ahora a Ezequías: El gran rey, el rey de Asiria, dice así: ¿Qué confianza es esta en que te apoyas?  
36:5 Yo digo que el consejo y poderío para la guerra, de que tú hablas, no son más que palabras vacías. Ahora bien, ¿en quién confías para que te rebeles contra mí?  
36:6 He aquí que confías en este báculo de caña frágil, en Egipto, en el cual si alguien se apoyare, se le entrará por la mano, y la atravesará. Tal es Faraón rey de Egipto para con todos los que en él confían.  
36:7 Y si me decís: En Jehová nuestro Dios confiamos; ¿no es éste aquel cuyos lugares altos y cuyos altares hizo quitar Ezequías, y dijo a Judá y a Jerusalén: Delante de este altar adoraréis?  
36:8 Ahora, pues, yo te ruego que des rehenes al rey de Asiria mi señor, y yo te daré dos mil caballos, si tú puedes dar jinetes que cabalguen sobre ellos.  
36:9 ¿Cómo, pues, podrás resistir a un capitán, al menor de los siervos de mi señor, aunque estés confiado en Egipto con sus carros y su gente de a caballo?  
36:10 ¿Acaso vine yo ahora a esta tierra para destruirla sin Jehová? Jehová me dijo: Sube a esta tierra y destrúyela.  
36:11 Entonces dijeron Eliaquim, Sebna y Joa al Rabsaces: Te rogamos que hables a tus siervos en arameo, porque nosotros lo entendemos; y no hables con nosotros en lengua de Judá, porque lo oye el pueblo que está sobre el muro.  
36:12 Y dijo el Rabsaces: ¿Acaso me envió mi señor a que dijese estas palabras a ti y a tu señor, y no a los hombres que están sobre el muro, expuestos a comer su estiércol y beber su orina con vosotros?  
36:13 Entonces el Rabsaces se puso en pie y gritó a gran voz en lengua de Judá, diciendo: Oíd las palabras del gran rey, el rey de Asiria.  
36:14 El rey dice así: No os engañe Ezequías, porque no os podrá librar.  
36:15 Ni os haga Ezequías confiar en Jehová, diciendo: Ciertamente Jehová nos librará; no será entregada esta ciudad en manos del rey de Asiria.  
36:16 No escuchéis a Ezequías, porque así dice el rey de Asiria: Haced conmigo paz, y salid a mí; y coma cada uno de su viña, y cada uno de su higuera, y beba cada cual las aguas de su pozo,  
36:17 hasta que yo venga y os lleve a una tierra como la vuestra, tierra de grano y de vino, tierra de pan y de viñas.  
36:18 Mirad que no os engañe Ezequías diciendo: Jehová nos librará. ¿Acaso libraron los dioses de las naciones cada uno su tierra de la mano del rey de Asiria?  
36:19 ¿Dónde está el dios de Hamat y de Arfad? ¿Dónde está el dios de Sefarvaim? ¿Libraron a Samaria de mi mano?  
36:20 ¿Qué dios hay entre los dioses de estas tierras que haya librado su tierra de mi mano, para que Jehová libre de mi mano a Jerusalén?  
36:21 Pero ellos callaron, y no le respondieron palabra; porque el rey así lo había mandado, diciendo: No le respondáis.  
36:22 Entonces Eliaquim hijo de Hilcías, mayordomo, y Sebna escriba, y Joa hijo de Asaf, canciller, vinieron a Ezequías, rasgados sus vestidos, y le contaron las palabras del Rabsaces.  

\chapter{37}

Judá es librado de Senaquerib  

37:1 Aconteció, pues, que cuando el rey Ezequías oyó esto, rasgó sus vestidos, y cubierto de cilicio vino a la casa de Jehová.  
37:2 Y envió a Eliaquim mayordomo, a Sebna escriba y a los ancianos de los sacerdotes, cubiertos de cilicio, al profeta Isaías hijo de Amoz.  
37:3 Los cuales le dijeron: Así ha dicho Ezequías: Día de angustia, de reprensión y de blasfemia es este día; porque los hijos han llegado hasta el punto de nacer, y la que da a luz no tiene fuerzas.  
37:4 Quizá oirá Jehová tu Dios las palabras del Rabsaces, al cual el rey de Asiria su señor envió para blasfemar al Dios vivo, y para vituperar con las palabras que oyó Jehová tu Dios; eleva, pues, oración tú por el remanente que aún ha quedado.  
37:5 Vinieron, pues, los siervos de Ezequías a Isaías.  
37:6 Y les dijo Isaías: Diréis así a vuestro señor: Así ha dicho Jehová: No temas por las palabras que has oído, con las cuales me han blasfemado los siervos del rey de Asiria.  
37:7 He aquí que yo pondré en él un espíritu, y oirá un rumor, y volverá a su tierra; y haré que en su tierra perezca a espada.  
37:8 Vuelto, pues, el Rabsaces, halló al rey de Asiria que combatía contra Libna; porque ya había oído que se había apartado de Laquis.  
37:9 Mas oyendo decir de Tirhaca rey de Etiopía: He aquí que ha salido para hacerte guerra; al oírlo, envió embajadores a Ezequías, diciendo:  
37:10 Así diréis a Ezequías rey de Judá: No te engañe tu Dios en quien tú confías, diciendo: Jerusalén no será entregada en mano del rey de Asiria.  
37:11 He aquí que tú oíste lo que han hecho los reyes de Asiria a todas las tierras, que las destruyeron; ¿y escaparás tú?  
37:12 ¿Acaso libraron sus dioses a las naciones que destruyeron mis antepasados, a Gozán, Harán, Resef y a los hijos de Edén que moraban en Telasar?  
37:13 ¿Dónde está el rey de Hamat, el rey de Arfad, y el rey de la ciudad de Sefarvaim, de Hena y de Iva?  
37:14 Y tomó Ezequías las cartas de mano de los embajadores, y las leyó; y subió a la casa de Jehová, y las extendió delante de Jehová. 
37:15 Entonces Ezequías oró a Jehová, diciendo:  
37:16 Jehová de los ejércitos, Dios de Israel, que moras entre los querubines, sólo tú eres Dios de todos los reinos de la tierra; tú hiciste los cielos y la tierra.  
37:17 Inclina, oh Jehová, tu oído, y oye; abre, oh Jehová, tus ojos, y mira; y oye todas las palabras de Senaquerib, que ha enviado a blasfemar al Dios viviente.  
37:18 Ciertamente, oh Jehová, los reyes de Asiria destruyeron todas las tierras y sus comarcas,  
37:19 y entregaron los dioses de ellos al fuego; porque no eran dioses, sino obra de manos de hombre, madera y piedra; por eso los destruyeron. 
37:20 Ahora pues, Jehová Dios nuestro, líbranos de su mano, para que todos los reinos de la tierra conozcan que sólo tú eres Jehová.  
37:21 Entonces Isaías hijo de Amoz envió a decir a Ezequías: Así ha dicho Jehová Dios de Israel: Acerca de lo que me rogaste sobre Senaquerib rey de Asiria,  
37:22 estas son las palabras que Jehová habló contra él: La virgen hija de Sion te menosprecia, te escarnece; detrás de ti mueve su cabeza la hija de Jerusalén.  
37:23 ¿A quién vituperaste, y a quién blasfemaste? ¿Contra quién has alzado tu voz, y levantado tus ojos en alto? Contra el Santo de Israel.  
37:24 Por mano de tus siervos has vituperado al Señor, y dijiste: Con la multitud de mis carros subiré a las alturas de los montes, a las laderas del Líbano; cortaré sus altos cedros, sus cipreses escogidos; llegaré hasta sus más elevadas cumbres, al bosque de sus feraces campos.  
37:25 Yo cavé, y bebí las aguas, y con las pisadas de mis pies secaré todos los ríos de Egipto.  
37:26 ¿No has oído decir que desde tiempos antiguos yo lo hice, que desde los días de la antigüedad lo tengo ideado? Y ahora lo he hecho venir, y tú serás para reducir las ciudades fortificadas a montones de escombros.  
37:27 Sus moradores fueron de corto poder; fueron acobardados y confusos, fueron como hierba del campo y hortaliza verde, como heno de los terrados, que antes de sazón se seca.  
37:28 He conocido tu condición, tu salida y tu entrada, y tu furor contra mí.  
37:29 Porque contra mí te airaste, y tu arrogancia ha subido a mis oídos; pondré, pues, mi garfio en tu nariz, y mi freno en tus labios, y te haré volver por el camino por donde viniste.  
37:30 Y esto te será por señal: Comeréis este año lo que nace de suyo, y el año segundo lo que nace de suyo; y el año tercero sembraréis y segaréis, y plantaréis viñas, y comeréis su fruto.  
37:31 Y lo que hubiere quedado de la casa de Judá y lo que hubiere escapado, volverá a echar raíz abajo, y dará fruto arriba.  
37:32 Porque de Jerusalén saldrá un remanente, y del monte de Sion los que se salven. El celo de Jehová de los ejércitos hará esto.  
37:33 Por tanto, así dice Jehová acerca del rey de Asiria: No entrará en esta ciudad, ni arrojará saeta en ella; no vendrá delante de ella con escudo, ni levantará contra ella baluarte.  
37:34 Por el camino que vino, volverá, y no entrará en esta ciudad, dice Jehová.  
37:35 Porque yo ampararé a esta ciudad para salvarla, por amor de mí mismo, y por amor de David mi siervo.  
37:36 Y salió el ángel de Jehová y mató a ciento ochenta y cinco mil en el campamento de los asirios; y cuando se levantaron por la mañana, he aquí que todo era cuerpos de muertos.  
37:37 Entonces Senaquerib rey de Asiria se fue, e hizo su morada en Nínive.  
37:38 Y aconteció que mientras adoraba en el templo de Nisroc su dios, sus hijos Adramelec y Sarezer le mataron a espada, y huyeron a la tierra de Ararat; y reinó en su lugar Esarhadón su hijo. 

\chapter{38}

Enfermedad de Ezequías  
 
38:1 En aquellos días Ezequías enfermó de muerte. Y vino a él el profeta Isaías hijo de Amoz, y le dijo: Jehová dice así: Ordena tu casa, porque morirás, y no vivirás.  
38:2 Entonces volvió Ezequías su rostro a la pared, e hizo oración a Jehová,  
38:3 y dijo: Oh Jehová, te ruego que te acuerdes ahora que he andado delante de ti en verdad y con íntegro corazón, y que he hecho lo que ha sido agradable delante de tus ojos. Y lloró Ezequías con gran lloro.  
38:4 Entonces vino palabra de Jehová a Isaías, diciendo:  
38:5 Ve y di a Ezequías: Jehová Dios de David tu padre dice así: He oído tu oración, y visto tus lágrimas; he aquí que yo añado a tus días quince años.  
38:6 Y te libraré a ti y a esta ciudad, de mano del rey de Asiria; y a esta ciudad ampararé.  
38:7 Y esto te será señal de parte de Jehová, que Jehová hará esto que ha dicho:  
38:8 He aquí yo haré volver la sombra por los grados que ha descendido con el sol, en el reloj de Acaz, diez grados atrás. Y volvió el sol diez grados atrás, por los cuales había ya descendido.  
38:9 Escritura de Ezequías rey de Judá, de cuando enfermó y sanó de su enfermedad:  
38:10 Yo dije: A la mitad de mis días iré a las puertas del Seol; privado soy del resto de mis años.  
38:11 Dije: No veré a JAH, a JAH en la tierra de los vivientes; ya no veré más hombre con los moradores del mundo.  
38:12 Mi morada ha sido movida y traspasada de mí, como tienda de pastor. Como tejedor corté mi vida; me cortará con la enfermedad; me consumirás entre el día y la noche.  
38:13 Contaba yo hasta la mañana. Como un león molió todos mis huesos; de la mañana a la noche me acabarás.  
38:14 Como la grulla y como la golondrina me quejaba; gemía como la paloma; alzaba en alto mis ojos. Jehová, violencia padezco; fortaléceme.  
38:15 ¿Qué diré? El que me lo dijo, él mismo lo ha hecho. Andaré humildemente todos mis años, a causa de aquella amargura de mi alma.  
38:16 Oh Señor, por todas estas cosas los hombres vivirán, y en todas ellas está la vida de mi espíritu; pues tú me restablecerás, y harás que viva.  
38:17 He aquí, amargura grande me sobrevino en la paz, mas a ti agradó librar mi vida del hoyo de corrupción; porque echaste tras tus espaldas todos mis pecados.  
38:18 Porque el Seol no te exaltará, ni te alabará la muerte; ni los que descienden al sepulcro esperarán tu verdad.  
38:19 El que vive, el que vive, éste te dará alabanza, como yo hoy; el padre hará notoria tu verdad a los hijos.  
38:20 Jehová me salvará; por tanto cantaremos nuestros cánticos en la casa de Jehová todos los días de nuestra vida.  
38:21 Y había dicho Isaías: Tomen masa de higos, y pónganla en la llaga, y sanará.  
38:22 Había asimismo dicho Ezequías: ¿Qué señal tendré de que subiré a la casa de Jehová?  

\chapter{39}

Ezequías recibe a los enviados de Babilonia  


39:1 En aquel tiempo Merodac-baladán hijo de Baladán, rey de Babilonia, envió cartas y presentes a Ezequías; porque supo que había estado enfermo, y que había convalecido.  
39:2 Y se regocijó con ellos Ezequías, y les mostró la casa de su tesoro, plata y oro, especias, ungüentos preciosos, toda su casa de armas, y todo lo que se hallaba en sus tesoros; no hubo cosa en su casa y en todos sus dominios, que Ezequías no les mostrase.  
39:3 Entonces el profeta Isaías vino al rey Ezequías, y le dijo: ¿Qué dicen estos hombres, y de dónde han venido a ti? Y Ezequías respondió: De tierra muy lejana han venido a mí, de Babilonia.  
39:4 Dijo entonces: ¿Qué han visto en tu casa? Y dijo Ezequías: Todo lo que hay en mi casa han visto, y ninguna cosa hay en mis tesoros que no les haya mostrado.  
39:5 Entonces dijo Isaías a Ezequías: Oye palabra de Jehová de los ejércitos:  
39:6 He aquí vienen días en que será llevado a Babilonia todo lo que hay en tu casa, y lo que tus padres han atesorado hasta hoy; ninguna cosa quedará, dice Jehová.  
39:7 De tus hijos que saldrán de ti, y que habrás engendrado, tomarán, y serán eunucos en el palacio del rey de Babilonia. 
39:8 Y dijo Ezequías a Isaías: La palabra de Jehová que has hablado es buena. Y añadió: A lo menos, haya paz y seguridad en mis días.  

\chapter{40}

Jehová consuela a Sion  

40:1 Consolaos, consolaos, pueblo mío, dice vuestro Dios.  
40:2 Hablad al corazón de Jerusalén; decidle a voces que su tiempo es ya cumplido, que su pecado es perdonado; que doble ha recibido de la mano de Jehová por todos sus pecados.  
40:3 Voz que clama en el desierto: Preparad camino a Jehová; enderezad calzada en la soledad a nuestro Dios. 
40:4 Todo valle sea alzado, y bájese todo monte y collado; y lo torcido se enderece, y lo áspero se allane.  
40:5 Y se manifestará la gloria de Jehová, y toda carne juntamente la verá; porque la boca de Jehová ha hablado. 
40:6 Voz que decía: Da voces. Y yo respondí: ¿Qué tengo que decir a voces? Que toda carne es hierba, y toda su gloria como flor del campo.  
40:7 La hierba se seca, y la flor se marchita, porque el viento de Jehová sopló en ella; ciertamente como hierba es el pueblo.  
40:8 Sécase la hierba, marchítase la flor; mas la palabra del Dios nuestro permanece para siempre. 
40:9 Súbete sobre un monte alto, anunciadora de Sion; levanta fuertemente tu voz, anunciadora de Jerusalén; levántala, no temas; di a las ciudades de Judá: ¡Ved aquí al Dios vuestro!  
40:10 He aquí que Jehová el Señor vendrá con poder, y su brazo señoreará; he aquí que su recompensa viene con él, y su paga delante de su rostro. 
40:11 Como pastor apacentará su rebaño; en su brazo llevará los corderos, y en su seno los llevará; pastoreará suavemente a las recién paridas.  
El incomparable Dios de Israel  
40:12 ¿Quién midió las aguas con el hueco de su mano y los cielos con su palmo, con tres dedos juntó el polvo de la tierra, y pesó los montes con balanza y con pesas los collados?  
40:13 ¿Quién enseñó al Espíritu de Jehová, o le aconsejó enseñándole? 
40:14 ¿A quién pidió consejo para ser avisado? ¿Quién le enseñó el camino del juicio, o le enseñó ciencia, o le mostró la senda de la prudencia?  
40:15 He aquí que las naciones le son como la gota de agua que cae del cubo, y como menudo polvo en las balanzas le son estimadas; he aquí que hace desaparecer las islas como polvo.  
40:16 Ni el Líbano bastará para el fuego, ni todos sus animales para el sacrificio.  
40:17 Como nada son todas las naciones delante de él; y en su comparación serán estimadas en menos que nada, y que lo que no es.  
40:18 ¿A qué, pues, haréis semejante a Dios, o qué imagen le compondréis?  
40:19 El artífice prepara la imagen de talla, el platero le extiende el oro y le funde cadenas de plata.  
40:20 El pobre escoge, para ofrecerle, madera que no se apolille; se busca un maestro sabio, que le haga una imagen de talla que no se mueva.  
40:21 ¿No sabéis? ¿No habéis oído? ¿Nunca os lo han dicho desde el principio? ¿No habéis sido enseñados desde que la tierra se fundó?  
40:22 El está sentado sobre el círculo de la tierra, cuyos moradores son como langostas; él extiende los cielos como una cortina, los despliega como una tienda para morar.  
40:23 El convierte en nada a los poderosos, y a los que gobiernan la tierra hace como cosa vana.  
40:24 Como si nunca hubieran sido plantados, como si nunca hubieran sido sembrados, como si nunca su tronco hubiera tenido raíz en la tierra; tan pronto como sopla en ellos se secan, y el torbellino los lleva como hojarasca.  
40:25 ¿A qué, pues, me haréis semejante o me compararéis? dice el Santo.  
40:26 Levantad en alto vuestros ojos, y mirad quién creó estas cosas; él saca y cuenta su ejército; a todas llama por sus nombres; ninguna faltará; tal es la grandeza de su fuerza, y el poder de su dominio.  
40:27 ¿Por qué dices, oh Jacob, y hablas tú, Israel: Mi camino está escondido de Jehová, y de mi Dios pasó mi juicio?  
40:28 ¿No has sabido, no has oído que el Dios eterno es Jehová, el cual creó los confines de la tierra? No desfallece, ni se fatiga con cansancio, y su entendimiento no hay quien lo alcance.  
40:29 El da esfuerzo al cansado, y multiplica las fuerzas al que no tiene ningunas.  
40:30 Los muchachos se fatigan y se cansan, los jóvenes flaquean y caen;  
40:31 pero los que esperan a Jehová tendrán nuevas fuerzas; levantarán alas como las águilas; correrán, y no se cansarán; caminarán, y no se fatigarán.  

\chapter{41}

Seguridad de Dios para Israel  

41:1 Escuchadme, costas, y esfuércense los pueblos; acérquense, y entonces hablen; estemos juntamente a juicio.  
41:2 ¿Quién despertó del oriente al justo, lo llamó para que le siguiese, entregó delante de él naciones, y le hizo enseñorear de reyes; los entregó a su espada como polvo, como hojarasca que su arco arrebata?  
41:3 Los siguió, pasó en paz por camino por donde sus pies nunca habían entrado.  
41:4 ¿Quién hizo y realizó esto? ¿Quién llama las generaciones desde el principio? Yo Jehová, el primero, y yo mismo con los postreros.  
41:5 Las costas vieron, y tuvieron temor; los confines de la tierra se espantaron; se congregaron, y vinieron.  
41:6 Cada cual ayudó a su vecino, y a su hermano dijo: Esfuérzate.  
41:7 El carpintero animó al platero, y el que alisaba con martillo al que batía en el yunque, diciendo: Buena está la soldadura; y lo afirmó con clavos, para que no se moviese.  
41:8 Pero tú, Israel, siervo mío eres; tú, Jacob, a quien yo escogí, descendencia de Abraham mi amigo. 
41:9 Porque te tomé de los confines de la tierra, y de tierras lejanas te llamé, y te dije: Mi siervo eres tú; te escogí, y no te deseché.  
41:10 No temas, porque yo estoy contigo; no desmayes, porque yo soy tu Dios que te esfuerzo; siempre te ayudaré, siempre te sustentaré con la diestra de mi justicia.  
41:11 He aquí que todos los que se enojan contra ti serán avergonzados y confundidos; serán como nada y perecerán los que contienden contigo.  
41:12 Buscarás a los que tienen contienda contigo, y no los hallarás; serán como nada, y como cosa que no es, aquellos que te hacen la guerra.  
41:13 Porque yo Jehová soy tu Dios, quien te sostiene de tu mano derecha, y te dice: No temas, yo te ayudo.  
41:14 No temas, gusano de Jacob, oh vosotros los pocos de Israel; yo soy tu socorro, dice Jehová; el Santo de Israel es tu Redentor.  
41:15 He aquí que yo te he puesto por trillo, trillo nuevo, lleno de dientes; trillarás montes y los molerás, y collados reducirás a tamo.  
41:16 Los aventarás, y los llevará el viento, y los esparcirá el torbellino; pero tú te regocijarás en Jehová, te gloriarás en el Santo de Israel.  
41:17 Los afligidos y menesterosos buscan las aguas, y no las hay; seca está de sed su lengua; yo Jehová los oiré, yo el Dios de Israel no los desampararé.  
41:18 En las alturas abriré ríos, y fuentes en medio de los valles; abriré en el desierto estanques de aguas, y manantiales de aguas en la tierra seca.  
41:19 Daré en el desierto cedros, acacias, arrayanes y olivos; pondré en la soledad cipreses, pinos y bojes juntamente,  
41:20 para que vean y conozcan, y adviertan y entiendan todos, que la mano de Jehová hace esto, y que el Santo de Israel lo creó.  
Dios reta a los falsos dioses  
41:21 Alegad por vuestra causa, dice Jehová; presentad vuestras pruebas, dice el Rey de Jacob.  
41:22 Traigan, anúnciennos lo que ha de venir; dígannos lo que ha pasado desde el principio, y pondremos nuestro corazón en ello; sepamos también su postrimería, y hacednos entender lo que ha de venir.  
41:23 Dadnos nuevas de lo que ha de ser después, para que sepamos que vosotros sois dioses; o a lo menos haced bien, o mal, para que tengamos qué contar, y juntamente nos maravillemos.  
41:24 He aquí que vosotros sois nada, y vuestras obras vanidad; abominación es el que os escogió.  
41:25 Del norte levanté a uno, y vendrá; del nacimiento del sol invocará mi nombre; y pisoteará príncipes como lodo, y como pisa el barro el alfarero.  
41:26 ¿Quién lo anunció desde el principio, para que sepamos; o de tiempo atrás, y diremos: Es justo? Cierto, no hay quien anuncie; sí, no hay quien enseñe; ciertamente no hay quien oiga vuestras palabras.  
41:27 Yo soy el primero que he enseñado estas cosas a Sion, y a Jerusalén daré un mensajero de alegres nuevas.  
41:28 Miré, y no había ninguno; y pregunté de estas cosas, y ningún consejero hubo; les pregunté, y no respondieron palabra.  
41:29 He aquí, todos son vanidad, y las obras de ellos nada; viento y vanidad son sus imágenes fundidas.  

\chapter{42}

El Siervo de Jehová  

42:1 He aquí mi siervo, yo le sostendré; mi escogido, en quien mi alma tiene contentamiento; he puesto sobre él mi Espíritu; él traerá justicia a las naciones. 
42:2 No gritará, ni alzará su voz, ni la hará oír en las calles.  
42:3 No quebrará la caña cascada, ni apagará el pábilo que humeare; por medio de la verdad traerá justicia.  
42:4 No se cansará ni desmayará, hasta que establezca en la tierra justicia; y las costas esperarán su ley.  
42:5 Así dice Jehová Dios, Creador de los cielos, y el que los despliega; el que extiende la tierra y sus productos; el que da aliento al pueblo que mora sobre ella, y espíritu a los que por ella andan:  
42:6 Yo Jehová te he llamado en justicia, y te sostendré por la mano; te guardaré y te pondré por pacto al pueblo, por luz de las naciones, 
42:7 para que abras los ojos de los ciegos, para que saques de la cárcel a los presos, y de casas de prisión a los que moran en tinieblas.  
42:8 Yo Jehová; este es mi nombre; y a otro no daré mi gloria, ni mi alabanza a esculturas.  
42:9 He aquí se cumplieron las cosas primeras, y yo anuncio cosas nuevas; antes que salgan a luz, yo os las haré notorias.  
Alabanza por la liberación poderosa de Jehová  
42:10 Cantad a Jehová un nuevo cántico, su alabanza desde el fin de la tierra; los que descendéis al mar, y cuanto hay en él, las costas y los moradores de ellas.  
42:11 Alcen la voz el desierto y sus ciudades, las aldeas donde habita Cedar; canten los moradores de Sela, y desde la cumbre de los montes den voces de júbilo.  
42:12 Den gloria a Jehová, y anuncien sus loores en las costas.  
42:13 Jehová saldrá como gigante, y como hombre de guerra despertará celo; gritará, voceará, se esforzará sobre sus enemigos.  
42:14 Desde el siglo he callado, he guardado silencio, y me he detenido; daré voces como la que está de parto; asolaré y devoraré juntamente.  
42:15 Convertiré en soledad montes y collados, haré secar toda su hierba; los ríos tornaré en islas, y secaré los estanques.  
42:16 Y guiaré a los ciegos por camino que no sabían, les haré andar por sendas que no habían conocido; delante de ellos cambiaré las tinieblas en luz, y lo escabroso en llanura. Estas cosas les haré, y no los desampararé.  
42:17 Serán vueltos atrás y en extremo confundidos los que confían en ídolos, y dicen a las imágenes de fundición: Vosotros sois nuestros dioses.  
Israel no aprende de la disciplina  
42:18 Sordos, oíd, y vosotros, ciegos, mirad para ver.  
42:19 ¿Quién es ciego, sino mi siervo? ¿Quién es sordo, como mi mensajero que envié? ¿Quién es ciego como mi escogido, y ciego como el siervo de Jehová,  
42:20 que ve muchas cosas y no advierte, que abre los oídos y no oye?  
42:21 Jehová se complació por amor de su justicia en magnificar la ley y engrandecerla.  
42:22 Mas este es pueblo saqueado y pisoteado, todos ellos atrapados en cavernas y escondidos en cárceles; son puestos para despojo, y no hay quien libre; despojados, y no hay quien diga: Restituid.  
42:23 ¿Quién de vosotros oirá esto? ¿Quién atenderá y escuchará respecto al porvenir?  
42:24 ¿Quién dio a Jacob en botín, y entregó a Israel a saqueadores? ¿No fue Jehová, contra quien pecamos? No quisieron andar en sus caminos, ni oyeron su ley.  
42:25 Por tanto, derramó sobre él el ardor de su ira, y fuerza de guerra; le puso fuego por todas partes, pero no entendió; y le consumió, mas no hizo caso.  

\chapter{43}

Jehová es el único Redentor  

43:1 Ahora, así dice Jehová, Creador tuyo, oh Jacob, y Formador tuyo, oh Israel: No temas, porque yo te redimí; te puse nombre, mío eres tú.  
43:2 Cuando pases por las aguas, yo estaré contigo; y si por los ríos, no te anegarán. Cuando pases por el fuego, no te quemarás, ni la llama arderá en ti.  
43:3 Porque yo Jehová, Dios tuyo, el Santo de Israel, soy tu Salvador; a Egipto he dado por tu rescate, a Etiopía y a Seba por ti.  
43:4 Porque a mis ojos fuiste de gran estima, fuiste honorable, y yo te amé; daré, pues, hombres por ti, y naciones por tu vida.  
43:5 No temas, porque yo estoy contigo; del oriente traeré tu generación, y del occidente te recogeré.  
43:6 Diré al norte: Da acá; y al sur: No detengas; trae de lejos mis hijos, y mis hijas de los confines de la tierra,  
43:7 todos los llamados de mi nombre; para gloria mía los he creado, los formé y los hice.  
43:8 Sacad al pueblo ciego que tiene ojos, y a los sordos que tienen oídos.  
43:9 Congréguense a una todas las naciones, y júntense todos los pueblos. ¿Quién de ellos hay que nos dé nuevas de esto, y que nos haga oír las cosas primeras? Presenten sus testigos, y justifíquense; oigan, y digan: Verdad es.  
43:10 Vosotros sois mis testigos, dice Jehová, y mi siervo que yo escogí, para que me conozcáis y creáis, y entendáis que yo mismo soy; antes de mí no fue formado dios, ni lo será después de mí.  
43:11 Yo, yo Jehová, y fuera de mí no hay quien salve.  
43:12 Yo anuncié, y salvé, e hice oír, y no hubo entre vosotros dios ajeno. Vosotros, pues, sois mis testigos, dice Jehová, que yo soy Dios.  
43:13 Aun antes que hubiera día, yo era; y no hay quien de mi mano libre. Lo que hago yo, ¿quién lo estorbará?  
43:14 Así dice Jehová, Redentor vuestro, el Santo de Israel: Por vosotros envié a Babilonia, e hice descender como fugitivos a todos ellos, aun a los caldeos en las naves de que se gloriaban.  
43:15 Yo Jehová, Santo vuestro, Creador de Israel, vuestro Rey.  
43:16 Así dice Jehová, el que abre camino en el mar, y senda en las aguas impetuosas;  
43:17 el que saca carro y caballo, ejército y fuerza; caen juntamente para no levantarse; fenecen, como pábilo quedan apagados.  
43:18 No os acordéis de las cosas pasadas, ni traigáis a memoria las cosas antiguas.  
43:19 He aquí que yo hago cosa nueva; pronto saldrá a luz; ¿no la conoceréis? Otra vez abriré camino en el desierto, y ríos en la soledad.  
43:20 Las fieras del campo me honrarán, los chacales y los pollos del avestruz; porque daré aguas en el desierto, ríos en la soledad, para que beba mi pueblo, mi escogido.  
43:21 Este pueblo he creado para mí; mis alabanzas publicará.  
43:22 Y no me invocaste a mí, oh Jacob, sino que de mí te cansaste, oh Israel.  
43:23 No me trajiste a mí los animales de tus holocaustos, ni a mí me honraste con tus sacrificios; no te hice servir con ofrenda, ni te hice fatigar con incienso.  
43:24 No compraste para mí caña aromática por dinero, ni me saciaste con la grosura de tus sacrificios, sino pusiste sobre mí la carga de tus pecados, me fatigaste con tus maldades.  
43:25 Yo, yo soy el que borro tus rebeliones por amor de mí mismo, y no me acordaré de tus pecados.  
43:26 Hazme recordar, entremos en juicio juntamente; habla tú para justificarte.  
43:27 Tu primer padre pecó, y tus enseñadores prevaricaron contra mí.  
43:28 Por tanto, yo profané los príncipes del santuario, y puse por anatema a Jacob y por oprobio a Israel.  

\chapter{44}

Jehová es el único Dios  

44:1 Ahora pues, oye, Jacob, siervo mío, y tú, Israel, a quien yo escogí.  
44:2 Así dice Jehová, Hacedor tuyo, y el que te formó desde el vientre, el cual te ayudará: No temas, siervo mío Jacob, y tú, Jesurún, a quien yo escogí.  
44:3 Porque yo derramaré aguas sobre el sequedal, y ríos sobre la tierra árida; mi Espíritu derramaré sobre tu generación, y mi bendición sobre tus renuevos;  
44:4 y brotarán entre hierba, como sauces junto a las riberas de las aguas.  
44:5 Este dirá: Yo soy de Jehová; el otro se llamará del nombre de Jacob, y otro escribirá con su mano: A Jehová, y se apellidará con el nombre de Israel.  
44:6 Así dice Jehová Rey de Israel, y su Redentor, Jehová de los ejércitos: Yo soy el primero, y yo soy el postrero, y fuera de mí no hay Dios.  
44:7 ¿Y quién proclamará lo venidero, lo declarará, y lo pondrá en orden delante de mí, como hago yo desde que establecí el pueblo antiguo? Anúncienles lo que viene, y lo que está por venir.  
44:8 No temáis, ni os amedrentéis; ¿no te lo hice oír desde la antigüedad, y te lo dije? Luego vosotros sois mis testigos. No hay Dios sino yo. No hay Fuerte; no conozco ninguno.  
La insensatez de la idolatría  
44:9 Los formadores de imágenes de talla, todos ellos son vanidad, y lo más precioso de ellos para nada es útil; y ellos mismos son testigos para su confusión, de que los ídolos no ven ni entienden.  
44:10 ¿Quién formó un dios, o quién fundió una imagen que para nada es de provecho?  
44:11 He aquí que todos los suyos serán avergonzados, porque los artífices mismos son hombres. Todos ellos se juntarán, se presentarán, se asombrarán, y serán avergonzados a una.  
44:12 El herrero toma la tenaza, trabaja en las ascuas, le da forma con los martillos, y trabaja en ello con la fuerza de su brazo; luego tiene hambre, y le faltan las fuerzas; no bebe agua, y se desmaya.  
44:13 El carpintero tiende la regla, lo señala con almagre, lo labra con los cepillos, le da figura con el compás, lo hace en forma de varón, a semejanza de hombre hermoso, para tenerlo en casa.  
44:14 Corta cedros, y toma ciprés y encina, que crecen entre los árboles del bosque; planta pino, que se críe con la lluvia.  
44:15 De él se sirve luego el hombre para quemar, y toma de ellos para calentarse; enciende también el horno, y cuece panes; hace además un dios, y lo adora; fabrica un ídolo, y se arrodilla delante de él.  
44:16 Parte del leño quema en el fuego; con parte de él come carne, prepara un asado, y se sacia; después se calienta, y dice: ¡Oh! me he calentado, he visto el fuego;  
44:17 y hace del sobrante un dios, un ídolo suyo; se postra delante de él, lo adora, y le ruega diciendo: Líbrame, porque mi Dios eres tú.  
44:18 No saben ni entienden; porque cerrados están sus ojos para no ver, y su corazón para no entender.  
44:19 No discurre para consigo, no tiene sentido ni entendimiento para decir: Parte de esto quemé en el fuego, y sobre sus brasas cocí pan, asé carne, y la comí. ¿Haré del resto de él una abominación? ¿Me postraré delante de un tronco de árbol?  
44:20 De ceniza se alimenta; su corazón engañado le desvía, para que no libre su alma, ni diga: ¿No es pura mentira lo que tengo en mi mano derecha?  
Jehová es el Redentor de Israel  
44:21 Acuérdate de estas cosas, oh Jacob, e Israel, porque mi siervo eres. Yo te formé, siervo mío eres tú; Israel, no me olvides.  
44:22 Yo deshice como una nube tus rebeliones, y como niebla tus pecados; vuélvete a mí, porque yo te redimí.  
44:23 Cantad loores, oh cielos, porque Jehová lo hizo; gritad con júbilo, profundidades de la tierra; prorrumpid, montes, en alabanza; bosque, y todo árbol que en él está; porque Jehová redimió a Jacob, y en Israel será glorificado.  
44:24 Así dice Jehová, tu Redentor, que te formó desde el vientre: Yo Jehová, que lo hago todo, que extiendo solo los cielos, que extiendo la tierra por mí mismo;  
44:25 que deshago las señales de los adivinos, y enloquezco a los agoreros; que hago volver atrás a los sabios, y desvanezco su sabiduría. 
44:26 Yo, el que despierta la palabra de su siervo, y cumple el consejo de sus mensajeros; que dice a Jerusalén: Serás habitada; y a las ciudades de Judá: Reconstruidas serán, y sus ruinas reedificaré;  
44:27 que dice a las profundidades: Secaos, y tus ríos haré secar;  
44:28 que dice de Ciro: Es mi pastor, y cumplirá todo lo que yo quiero, al decir a Jerusalén: Serás edificada; y al templo: Serás fundado.  

\chapter{45}

Encargo de Dios para Ciro  

45:1 Así dice Jehová a su ungido, a Ciro, al cual tomé yo por su mano derecha, para sujetar naciones delante de él y desatar lomos de reyes; para abrir delante de él puertas, y las puertas no se cerrarán:  
45:2 Yo iré delante de ti, y enderezaré los lugares torcidos; quebrantaré puertas de bronce, y cerrojos de hierro haré pedazos;  
45:3 y te daré los tesoros escondidos, y los secretos muy guardados, para que sepas que yo soy Jehová, el Dios de Israel, que te pongo nombre.  
45:4 Por amor de mi siervo Jacob, y de Israel mi escogido, te llamé por tu nombre; te puse sobrenombre, aunque no me conociste.  
45:5 Yo soy Jehová, y ninguno más hay; no hay Dios fuera de mí. Yo te ceñiré, aunque tú no me conociste,  
45:6 para que se sepa desde el nacimiento del sol, y hasta donde se pone, que no hay más que yo; yo Jehová, y ninguno más que yo,  
45:7 que formo la luz y creo las tinieblas, que hago la paz y creo la adversidad. Yo Jehová soy el que hago todo esto.  
Jehová el Creador  
45:8 Rociad, cielos, de arriba, y las nubes destilen la justicia; ábrase la tierra, y prodúzcanse la salvación y la justicia; háganse brotar juntamente. Yo Jehová lo he creado.  
45:9 ¡Ay del que pleitea con su Hacedor! ¡el tiesto con los tiestos de la tierra! ¿Dirá el barro al que lo labra: ¿Qué haces? o tu obra: No tiene manos?  
45:10 ¡Ay del que dice al padre: ¿Por qué engendraste? y a la mujer: ¿Por qué diste a luz?!  
45:11 Así dice Jehová, el Santo de Israel, y su Formador: Preguntadme de las cosas por venir; mandadme acerca de mis hijos, y acerca de la obra de mis manos.  
45:12 Yo hice la tierra, y creé sobre ella al hombre. Yo, mis manos, extendieron los cielos, y a todo su ejército mandé.  
45:13 Yo lo desperté en justicia, y enderezaré todos sus caminos; él edificará mi ciudad, y soltará mis cautivos, no por precio ni por dones, dice Jehová de los ejércitos.  
45:14 Así dice Jehová: El trabajo de Egipto, las mercaderías de Etiopía, y los sabeos, hombres de elevada estatura, se pasarán a ti y serán tuyos; irán en pos de ti, pasarán con grillos; te harán reverencia y te suplicarán diciendo: Ciertamente en ti está Dios, y no hay otro fuera de Dios.  
45:15 Verdaderamente tú eres Dios que te encubres, Dios de Israel, que salvas.  
45:16 Confusos y avergonzados serán todos ellos; irán con afrenta todos los fabricadores de imágenes.  
45:17 Israel será salvo en Jehová con salvación eterna; no os avergonzaréis ni os afrentaréis, por todos los siglos.  
45:18 Porque así dijo Jehová, que creó los cielos; él es Dios, el que formó la tierra, el que la hizo y la compuso; no la creó en vano, para que fuese habitada la creó: Yo soy Jehová, y no hay otro.  
45:19 No hablé en secreto, en un lugar oscuro de la tierra; no dije a la descendencia de Jacob: En vano me buscáis. Yo soy Jehová que hablo justicia, que anuncio rectitud.  
Jehová y los ídolos de Babilonia  
45:20 Reuníos, y venid; juntaos todos los sobrevivientes de entre las naciones. No tienen conocimiento aquellos que erigen el madero de su ídolo, y los que ruegan a un dios que no salva.  
45:21 Proclamad, y hacedlos acercarse, y entren todos en consulta; ¿quién hizo oír esto desde el principio, y lo tiene dicho desde entonces, sino yo Jehová? Y no hay más Dios que yo; Dios justo y Salvador; ningún otro fuera de mí.  
45:22 Mirad a mí, y sed salvos, todos los términos de la tierra, porque yo soy Dios, y no hay más.  
45:23 Por mí mismo hice juramento, de mi boca salió palabra en justicia, y no será revocada: Que a mí se doblará toda rodilla, y jurará toda lengua. 
45:24 Y se dirá de mí: Ciertamente en Jehová está la justicia y la fuerza; a él vendrán, y todos los que contra él se enardecen serán avergonzados.  
45:25 En Jehová será justificada y se gloriará toda la descendencia de Israel.  

\chapter{46}


46:1 Se postró Bel, se abatió Nebo; sus imágenes fueron puestas sobre bestias, sobre animales de carga; esas cosas que vosotros solíais llevar son alzadas cual carga, sobre las bestias cansadas.  
46:2 Fueron humillados, fueron abatidos juntamente; no pudieron escaparse de la carga, sino que tuvieron ellos mismos que ir en cautiverio.  
46:3 Oídme, oh casa de Jacob, y todo el resto de la casa de Israel, los que sois traídos por mí desde el vientre, los que sois llevados desde la matriz.  
46:4 Y hasta la vejez yo mismo, y hasta las canas os soportaré yo; yo hice, yo llevaré, yo soportaré y guardaré.  
46:5 ¿A quién me asemejáis, y me igualáis, y me comparáis, para que seamos semejantes?  
46:6 Sacan oro de la bolsa, y pesan plata con balanzas, alquilan un platero para hacer un dios de ello; se postran y adoran.  
46:7 Se lo echan sobre los hombros, lo llevan, y lo colocan en su lugar; allí se está, y no se mueve de su sitio. Le gritan, y tampoco responde, ni libra de la tribulación.  
46:8 Acordaos de esto, y tened vergüenza; volved en vosotros, prevaricadores.  
46:9 Acordaos de las cosas pasadas desde los tiempos antiguos; porque yo soy Dios, y no hay otro Dios, y nada hay semejante a mí,  
46:10 que anuncio lo por venir desde el principio, y desde la antigüedad lo que aún no era hecho; que digo: Mi consejo permanecerá, y haré todo lo que quiero;  
46:11 que llamo desde el oriente al ave, y de tierra lejana al varón de mi consejo. Yo hablé, y lo haré venir; lo he pensado, y también lo haré.  
46:12 Oídme, duros de corazón, que estáis lejos de la justicia:  
46:13 Haré que se acerque mi justicia; no se alejará, y mi salvación no se detendrá. Y pondré salvación en Sion, y mi gloria en Israel.  

\chapter{47}

Juicio sobre Babilonia  

47:1 Desciende y siéntate en el polvo, virgen hija de Babilonia. Siéntate en la tierra, sin trono, hija de los caldeos; porque nunca más te llamarán tierna y delicada.  
47:2 Toma el molino y muele harina; descubre tus guedejas, descalza los pies, descubre las piernas, pasa los ríos.  
47:3 Será tu vergüenza descubierta, y tu deshonra será vista; haré retribución, y no se librará hombre alguno.  
47:4 Nuestro Redentor, Jehová de los ejércitos es su nombre, el Santo de Israel.  
47:5 Siéntate, calla, y entra en tinieblas, hija de los caldeos; porque nunca más te llamarán señora de reinos.  
47:6 Me enojé contra mi pueblo, profané mi heredad, y los entregué en tu mano; no les tuviste compasión; sobre el anciano agravaste mucho tu yugo.  
47:7 Dijiste: Para siempre seré señora; y no has pensado en esto, ni te acordaste de tu postrimería.  
47:8 Oye, pues, ahora esto, mujer voluptuosa, tú que estás sentada confiadamente, tú que dices en tu corazón: Yo soy, y fuera de mí no hay más; no quedaré viuda, ni conoceré orfandad.  
47:9 Estas dos cosas te vendrán de repente en un mismo día, orfandad y viudez; en toda su fuerza vendrán sobre ti, a pesar de la multitud de tus hechizos y de tus muchos encantamientos.  
47:10 Porque te confiaste en tu maldad, diciendo: Nadie me ve. Tu sabiduría y tu misma ciencia te engañaron, y dijiste en tu corazón: Yo, y nadie más.  
47:11 Vendrá, pues, sobre ti mal, cuyo nacimiento no sabrás; caerá sobre ti quebrantamiento, el cual no podrás remediar; y destrucción que no sepas vendrá de repente sobre ti.  
47:12 Estate ahora en tus encantamientos y en la multitud de tus hechizos, en los cuales te fatigaste desde tu juventud; quizá podrás mejorarte, quizá te fortalecerás.  
47:13 Te has fatigado en tus muchos consejos. Comparezcan ahora y te defiendan los contempladores de los cielos, los que observan las estrellas, los que cuentan los meses, para pronosticar lo que vendrá sobre ti.  
47:14 He aquí que serán como tamo; fuego los quemará, no salvarán sus vidas del poder de la llama; no quedará brasa para calentarse, ni lumbre a la cual se sienten.  
47:15 Así te serán aquellos con quienes te fatigaste, los que traficaron contigo desde tu juventud; cada uno irá por su camino, no habrá quien te salve. 

\chapter{48}

Dios reprende la infidelidad de Israel  

48:1 Oíd esto, casa de Jacob, que os llamáis del nombre de Israel, los que salieron de las aguas de Judá, los que juran en el nombre de Jehová, y hacen memoria del Dios de Israel, mas no en verdad ni en justicia;  
48:2 porque de la santa ciudad se nombran, y en el Dios de Israel confían; su nombre es Jehová de los ejércitos.  
48:3 Lo que pasó, ya antes lo dije, y de mi boca salió; lo publiqué, lo hice pronto, y fue realidad.  
48:4 Por cuanto conozco que eres duro, y barra de hierro tu cerviz, y tu frente de bronce,  
48:5 te lo dije ya hace tiempo; antes que sucediera te lo advertí, para que no dijeras: Mi ídolo lo hizo, mis imágenes de escultura y de fundición mandaron estas cosas.  
48:6 Lo oíste, y lo viste todo; ¿y no lo anunciaréis vosotros? Ahora, pues, te he hecho oír cosas nuevas y ocultas que tú no sabías.  
48:7 Ahora han sido creadas, no en días pasados, ni antes de este día las habías oído, para que no digas: He aquí que yo lo sabía.  
48:8 Sí, nunca lo habías oído, ni nunca lo habías conocido; ciertamente no se abrió antes tu oído; porque sabía que siendo desleal habías de desobedecer, por tanto te llamé rebelde desde el vientre.  
48:9 Por amor de mi nombre diferiré mi ira, y para alabanza mía la reprimiré para no destruirte.  
48:10 He aquí te he purificado, y no como a plata; te he escogido en horno de aflicción.  
48:11 Por mí, por amor de mí mismo lo haré, para que no sea amancillado mi nombre, y mi honra no la daré a otro. 
48:12 Oyeme, Jacob, y tú, Israel, a quien llamé: Yo mismo, yo el primero, yo también el postrero. 
48:13 Mi mano fundó también la tierra, y mi mano derecha midió los cielos con el palmo; al llamarlos yo, comparecieron juntamente.  
48:14 Juntaos todos vosotros, y oíd. ¿Quién hay entre ellos que anuncie estas cosas? Aquel a quien Jehová amó ejecutará su voluntad en Babilonia, y su brazo estará sobre los caldeos.  
48:15 Yo, yo hablé, y le llamé y le traje; por tanto, será prosperado su camino.  
48:16 Acercaos a mí, oíd esto: desde el principio no hablé en secreto; desde que eso se hizo, allí estaba yo; y ahora me envió Jehová el Señor, y su Espíritu.  
48:17 Así ha dicho Jehová, Redentor tuyo, el Santo de Israel: Yo soy Jehová Dios tuyo, que te enseña provechosamente, que te encamina por el camino que debes seguir.  
48:18 ¡Oh, si hubieras atendido a mis mandamientos! Fuera entonces tu paz como un río, y tu justicia como las ondas del mar.  
48:19 Fuera como la arena tu descendencia, y los renuevos de tus entrañas como los granos de arena; nunca su nombre sería cortado, ni raído de mi presencia.  
48:20 Salid de Babilonia, huid de entre los caldeos; dad nuevas de esto con voz de alegría, publicadlo, llevadlo hasta lo postrero de la tierra; decid: Redimió Jehová a Jacob su siervo.  
48:21 No tuvieron sed cuando los llevó por los desiertos; les hizo brotar agua de la piedra; abrió la peña, y corrieron las aguas.  
48:22 No hay paz para los malos, dijo Jehová. 

\chapter{49}

Israel, siervo de Jehová  

49:1 Oídme, costas, y escuchad, pueblos lejanos. Jehová me llamó desde el vientre, desde las entrañas de mi madre tuvo mi nombre en memoria.  
49:2 Y puso mi boca como espada aguda, me cubrió con la sombra de su mano; y me puso por saeta bruñida, me guardó en su aljaba;  
49:3 y me dijo: Mi siervo eres, oh Israel, porque en ti me gloriaré.  
49:4 Pero yo dije: Por demás he trabajado, en vano y sin provecho he consumido mis fuerzas; pero mi causa está delante de Jehová, y mi recompensa con mi Dios.  
49:5 Ahora pues, dice Jehová, el que me formó desde el vientre para ser su siervo, para hacer volver a él a Jacob y para congregarle a Israel (porque estimado seré en los ojos de Jehová, y el Dios mío será mi fuerza);  
49:6 dice: Poco es para mí que tú seas mi siervo para levantar las tribus de Jacob, y para que restaures el remanente de Israel; también te di por luz de las naciones, para que seas mi salvación hasta lo postrero de la tierra. 
49:7 Así ha dicho Jehová, Redentor de Israel, el Santo suyo, al menospreciado de alma, al abominado de las naciones, al siervo de los tiranos: Verán reyes, y se levantarán príncipes, y adorarán por Jehová; porque fiel es el Santo de Israel, el cual te escogió.  
Dios promete restaurar a Sion  
49:8 Así dijo Jehová: En tiempo aceptable te oí, y en el día de salvación te ayudé; y te guardaré, y te daré por pacto al pueblo, para que restaures la tierra, para que heredes asoladas heredades;  
49:9 para que digas a los presos: Salid; y a los que están en tinieblas: Mostraos. En los caminos serán apacentados, y en todas las alturas tendrán sus pastos.  
49:10 No tendrán hambre ni sed, ni el calor ni el sol los afligirá; porque el que tiene de ellos misericordia los guiará, y los conducirá a manantiales de aguas. 
49:11 Y convertiré en camino todos mis montes, y mis calzadas serán levantadas.  
49:12 He aquí éstos vendrán de lejos; y he aquí éstos del norte y del occidente, y éstos de la tierra de Sinim.  
49:13 Cantad alabanzas, oh cielos, y alégrate, tierra; y prorrumpid en alabanzas, oh montes; porque Jehová ha consolado a su pueblo, y de sus pobres tendrá misericordia.  
49:14 Pero Sion dijo: Me dejó Jehová, y el Señor se olvidó de mí.  
49:15 ¿Se olvidará la mujer de lo que dio a luz, para dejar de compadecerse del hijo de su vientre? Aunque olvide ella, yo nunca me olvidaré de ti.  
49:16 He aquí que en las palmas de las manos te tengo esculpida; delante de mí están siempre tus muros.  
49:17 Tus edificadores vendrán aprisa; tus destruidores y tus asoladores saldrán de ti.  
49:18 Alza tus ojos alrededor, y mira: todos éstos se han reunido, han venido a ti. Vivo yo, dice Jehová, que de todos, como de vestidura de honra, serás vestida; y de ellos serás ceñida como novia.  
49:19 Porque tu tierra devastada, arruinada y desierta, ahora será estrecha por la multitud de los moradores, y tus destruidores serán apartados lejos.  
49:20 Aun los hijos de tu orfandad dirán a tus oídos: Estrecho es para mí este lugar; apártate, para que yo more.  
49:21 Y dirás en tu corazón: ¿Quién me engendró éstos? Porque yo había sido privada de hijos y estaba sola, peregrina y desterrada; ¿quién, pues, crió éstos? He aquí yo había sido dejada sola; ¿dónde estaban éstos?  
49:22 Así dijo Jehová el Señor: He aquí, yo tenderé mi mano a las naciones, y a los pueblos levantaré mi bandera; y traerán en brazos a tus hijos, y tus hijas serán traídas en hombros.  
49:23 Reyes serán tus ayos, y sus reinas tus nodrizas; con el rostro inclinado a tierra te adorarán, y lamerán el polvo de tus pies; y conocerás que yo soy Jehová, que no se avergonzarán los que esperan en mí.  
49:24 ¿Será quitado el botín al valiente? ¿Será rescatado el cautivo de un tirano?  
49:25 Pero así dice Jehová: Ciertamente el cautivo será rescatado del valiente, y el botín será arrebatado al tirano; y tu pleito yo lo defenderé, y yo salvaré a tus hijos.  
49:26 Y a los que te despojaron haré comer sus propias carnes, y con su sangre serán embriagados como con vino; y conocerá todo hombre que yo Jehová soy Salvador tuyo y Redentor tuyo, el Fuerte de Jacob.  

\chapter{50}

Jehová ayuda a quienes confían en él  

50:1 Así dijo Jehová: ¿Qué es de la carta de repudio de vuestra madre, con la cual yo la repudié? ¿O quiénes son mis acreedores, a quienes yo os he vendido? He aquí que por vuestras maldades sois vendidos, y por vuestras rebeliones fue repudiada vuestra madre.  
50:2 ¿Por qué cuando vine, no hallé a nadie, y cuando llamé, nadie respondió? ¿Acaso se ha acortado mi mano para no redimir? ¿No hay en mí poder para librar? He aquí que con mi reprensión hago secar el mar; convierto los ríos en desierto; sus peces se pudren por falta de agua, y mueren de sed. 
50:3 Visto de oscuridad los cielos, y hago como cilicio su cubierta.  
50:4 Jehová el Señor me dio lengua de sabios, para saber hablar palabras al cansado; despertará mañana tras mañana, despertará mi oído para que oiga como los sabios.  
50:5 Jehová el Señor me abrió el oído, y yo no fui rebelde, ni me volví atrás.  
50:6 Di mi cuerpo a los heridores, y mis mejillas a los que me mesaban la barba; no escondí mi rostro de injurias y de esputos. 
50:7 Porque Jehová el Señor me ayudará, por tanto no me avergoncé; por eso puse mi rostro como un pedernal, y sé que no seré avergonzado.  
50:8 Cercano está de mí el que me salva; ¿quién contenderá conmigo? Juntémonos. ¿Quién es el adversario de mi causa? Acérquese a mí.  
50:9 He aquí que Jehová el Señor me ayudará; ¿quién hay que me condene? He aquí que todos ellos se envejecerán como ropa de vestir, serán comidos por la polilla.  
50:10 ¿Quién hay entre vosotros que teme a Jehová, y oye la voz de su siervo? El que anda en tinieblas y carece de luz, confíe en el nombre de Jehová, y apóyese en su Dios.  
50:11 He aquí que todos vosotros encendéis fuego, y os rodeáis de teas; andad a la luz de vuestro fuego, y de las teas que encendisteis. De mi mano os vendrá esto; en dolor seréis sepultados.  

\chapter{51}

Palabras de consuelo para Sion  

51:1 Oídme, los que seguís la justicia, los que buscáis a Jehová. Mirad a la piedra de donde fuisteis cortados, y al hueco de la cantera de donde fuisteis arrancados.  
51:2 Mirad a Abraham vuestro padre, y a Sara que os dio a luz; porque cuando no era más que uno solo lo llamé, y lo bendije y lo multipliqué.  
51:3 Ciertamente consolará Jehová a Sion; consolará todas sus soledades, y cambiará su desierto en paraíso, y su soledad en huerto de Jehová; se hallará en ella alegría y gozo, alabanza y voces de canto.  
51:4 Estad atentos a mí, pueblo mío, y oídme, nación mía; porque de mí saldrá la ley, y mi justicia para luz de los pueblos.  
51:5 Cercana está mi justicia, ha salido mi salvación, y mis brazos juzgarán a los pueblos; a mí me esperan los de la costa, y en mi brazo ponen su esperanza.  
51:6 Alzad a los cielos vuestros ojos, y mirad abajo a la tierra; porque los cielos serán deshechos como humo, y la tierra se envejecerá como ropa de vestir, y de la misma manera perecerán sus moradores; pero mi salvación será para siempre, mi justicia no perecerá.  
51:7 Oídme, los que conocéis justicia, pueblo en cuyo corazón está mi ley. No temáis afrenta de hombre, ni desmayéis por sus ultrajes.  
51:8 Porque como a vestidura los comerá polilla, como a lana los comerá gusano; pero mi justicia permanecerá perpetuamente, y mi salvación por siglos de siglos.  
51:9 Despiértate, despiértate, vístete de poder, oh brazo de Jehová; despiértate como en el tiempo antiguo, en los siglos pasados. ¿No eres tú el que cortó a Rahab, y el que hirió al dragón?  
51:10 ¿No eres tú el que secó el mar, las aguas del gran abismo; el que transformó en camino las profundidades del mar para que pasaran los redimidos?  
51:11 Ciertamente volverán los redimidos de Jehová; volverán a Sion cantando, y gozo perpetuo habrá sobre sus cabezas; tendrán gozo y alegría, y el dolor y el gemido huirán.  
51:12 Yo, yo soy vuestro consolador. ¿Quién eres tú para que tengas temor del hombre, que es mortal, y del hijo de hombre, que es como heno?  
51:13 Y ya te has olvidado de Jehová tu Hacedor, que extendió los cielos y fundó la tierra; y todo el día temiste continuamente del furor del que aflige, cuando se disponía para destruir. ¿Pero en dónde está el furor del que aflige?  
51:14 El preso agobiado será libertado pronto; no morirá en la mazmorra, ni le faltará su pan.  
51:15 Porque yo Jehová, que agito el mar y hago rugir sus ondas, soy tu Dios, cuyo nombre es Jehová de los ejércitos.  
51:16 Y en tu boca he puesto mis palabras, y con la sombra de mi mano te cubrí, extendiendo los cielos y echando los cimientos de la tierra, y diciendo a Sion: Pueblo mío eres tú.  
51:17 Despierta, despierta, levántate, oh Jerusalén, que bebiste de la mano de Jehová el cáliz de su ira; porque el cáliz de aturdimiento bebiste hasta los sedimentos.  
51:18 De todos los hijos que dio a luz, no hay quien la guíe; ni quien la tome de la mano, de todos los hijos que crió.  
51:19 Estas dos cosas te han acontecido: asolamiento y quebrantamiento, hambre y espada. ¿Quién se dolerá de ti? ¿Quién te consolará?  
51:20 Tus hijos desmayaron, estuvieron tendidos en las encrucijadas de todos los caminos, como antílope en la red, llenos de la indignación de Jehová, de la ira del Dios tuyo.  
51:21 Oye, pues, ahora esto, afligida, ebria, y no de vino:  
51:22 Así dijo Jehová tu Señor, y tu Dios, el cual aboga por su pueblo: He aquí he quitado de tu mano el cáliz de aturdimiento, los sedimentos del cáliz de mi ira; nunca más lo beberás.  
51:23 Y lo pondré en mano de tus angustiadores, que dijeron a tu alma: Inclínate, y pasaremos por encima de ti. Y tú pusiste tu cuerpo como tierra, y como camino, para que pasaran.  

\chapter{52}

Dios librará del cautiverio a Sion  

52:1 Despierta, despierta, vístete de poder, oh Sion; vístete tu ropa hermosa, oh Jerusalén, ciudad santa; porque nunca más vendrá a ti incircunciso ni inmundo.  
52:2 Sacúdete del polvo; levántate y siéntate, Jerusalén; suelta las ataduras de tu cuello, cautiva hija de Sion.  
52:3 Porque así dice Jehová: De balde fuisteis vendidos; por tanto, sin dinero seréis rescatados.  
52:4 Porque así dijo Jehová el Señor: Mi pueblo descendió a Egipto en tiempo pasado, para morar allá, y el asirio lo cautivó sin razón.  
52:5 Y ahora ¿qué hago aquí, dice Jehová, ya que mi pueblo es llevado injustamente? Y los que en él se enseñorean, lo hacen aullar, dice Jehová, y continuamente es blasfemado mi nombre todo el día.  
52:6 Por tanto, mi pueblo sabrá mi nombre por esta causa en aquel día; porque yo mismo que hablo, he aquí estaré presente.  
52:7 ¡Cuán hermosos son sobre los montes los pies del que trae alegres nuevas, del que anuncia la paz, del que trae nuevas del bien, del que publica salvación, del que dice a Sion: ¡Tu Dios reina!  
52:8 ¡Voz de tus atalayas! Alzarán la voz, juntamente darán voces de júbilo; porque ojo a ojo verán que Jehová vuelve a traer a Sion.  
52:9 Cantad alabanzas, alegraos juntamente, soledades de Jerusalén; porque Jehová ha consolado a su pueblo, a Jerusalén ha redimido.  
52:10 Jehová desnudó su santo brazo ante los ojos de todas las naciones, y todos los confines de la tierra verán la salvación del Dios nuestro.  
52:11 Apartaos, apartaos, salid de ahí, no toquéis cosa inmunda, salid de en medio de ella; purificaos los que lleváislos utensilios de Jehová.  
52:12 Porque no saldréis apresurados, ni iréis huyendo; porque Jehová irá delante de vosotros, y os congregará el Dios de Israel.  
Sufrimientos del Siervo de Jehová  
52:13 He aquí que mi siervo será prosperado, será engrandecido y exaltado, y será puesto muy en alto.  
52:14 Como se asombraron de ti muchos, de tal manera fue desfigurado de los hombres su parecer, y su hermosura más que la de los hijos de los hombres,  
52:15 así asombrará él a muchas naciones; los reyes cerrarán ante él la boca, porque verán lo que nunca les fue contado, y entenderán lo que jamás habían oído. 

\chapter{53}


53:1 ¿Quién ha creído a nuestro anuncio? ¿y sobre quién se ha manifestado el brazo de Jehová?  
53:2 Subirá cual renuevo delante de él, y como raíz de tierra seca; no hay parecer en él, ni hermosura; le veremos, mas sin atractivo para que le deseemos.  
53:3 Despreciado y desechado entre los hombres, varón de dolores, experimentado en quebranto; y como que escondimos de él el rostro, fue menospreciado, y no lo estimamos.  
53:4 Ciertamente llevó él nuestras enfermedades, y sufrió nuestros dolores; y nosotros le tuvimos por azotado, por herido de Dios y abatido.  
53:5 Mas él herido fue por nuestras rebeliones, molido por nuestros pecados; el castigo de nuestra paz fue sobre él, y por su llaga fuimos nosotros curados. 
53:6 Todos nosotros nos descarriamos como ovejas, cada cual se apartó por su camino; mas Jehová cargó en él el pecado de todos nosotros.  
53:7 Angustiado él, y afligido, no abrió su boca; como cordero fue llevado al matadero; y como oveja delante de sus trasquiladores, enmudeció, y no abrió su boca.  
53:8 Por cárcel y por juicio fue quitado; y su generación, ¿quién la contará? Porque fue cortado de la tierra de los vivientes, y por la rebelión de mi pueblo fue herido.  
53:9 Y se dispuso con los impíos su sepultura, mas con los ricos fue en su muerte; aunque nunca hizo maldad, ni hubo engaño en su boca. 
53:10 Con todo eso, Jehová quiso quebrantarlo, sujetándole a padecimiento. Cuando haya puesto su vida en expiación por el pecado, verá linaje, vivirá por largos días, y la voluntad de Jehová será en su mano prosperada.  
53:11 Verá el fruto de la aflicción de su alma, y quedará satisfecho; por su conocimiento justificará mi siervo justo a muchos, y llevará las iniquidades de ellos.  
53:12 Por tanto, yo le daré parte con los grandes, y con los fuertes repartirá despojos; por cuanto derramó su vida hasta la muerte, y fue contado con los pecadores, habiendo él llevado el pecado de muchos, y orado por los transgresores.  

\chapter{54}

El amor eterno de Jehová hacia Israel  

54:1 Regocíjate, oh estéril, la que no daba a luz; levanta canción y da voces de júbilo, la que nunca estuvo de parto; porque más son los hijos de la desamparada que los de la casada, ha dicho Jehová.  
54:2 Ensancha el sitio de tu tienda, y las cortinas de tus habitaciones sean extendidas; no seas escasa; alarga tus cuerdas, y refuerza tus estacas.  
54:3 Porque te extenderás a la mano derecha y a la mano izquierda; y tu descendencia heredará naciones, y habitará las ciudades asoladas.  
54:4 No temas, pues no serás confundida; y no te avergüences, porque no serás afrentada, sino que te olvidarás de la vergüenza de tu juventud, y de la afrenta de tu viudez no tendrás más memoria.  
54:5 Porque tu marido es tu Hacedor; Jehová de los ejércitos es su nombre; y tu Redentor, el Santo de Israel; Dios de toda la tierra será llamado.  
54:6 Porque como a mujer abandonada y triste de espíritu te llamó Jehová, y como a la esposa de la juventud que es repudiada, dijo el Dios tuyo.  
54:7 Por un breve momento te abandoné, pero te recogeré con grandes misericordias.  
54:8 Con un poco de ira escondí mi rostro de ti por un momento; pero con misericordia eterna tendré compasión de ti, dijo Jehová tu Redentor.  
54:9 Porque esto me será como en los días de Noé, cuando juré que nunca más las aguas de Noé pasarían sobre la tierra; así he jurado que no me enojaré contra ti, ni te reñiré.  
54:10 Porque los montes se moverán, y los collados temblarán, pero no se apartará de ti mi misericordia, ni el pacto de mi paz se quebrantará, dijo Jehová, el que tiene misericordia de ti.  
54:11 Pobrecita, fatigada con tempestad, sin consuelo; he aquí que yo cimentaré tus piedras sobre carbunclo, y sobre zafiros te fundaré.  
54:12 Tus ventanas pondré de piedras preciosas, tus puertas de piedras de carbunclo, y toda tu muralla de piedras preciosas. 
54:13 Y todos tus hijos serán enseñados por Jehová; y se multiplicará la paz de tus hijos.  
54:14 Con justicia serás adornada; estarás lejos de opresión, porque no temerás, y de temor, porque no se acercará a ti.  
54:15 Si alguno conspirare contra ti, lo hará sin mí; el que contra ti conspirare, delante de ti caerá.  
54:16 He aquí que yo hice al herrero que sopla las ascuas en el fuego, y que saca la herramienta para su obra; y yo he creado al destruidor para destruir.  
54:17 Ninguna arma forjada contra ti prosperará, y condenarás toda lengua que se levante contra ti en juicio. Esta es la herencia de los siervos de Jehová, y su salvación de mí vendrá, dijo Jehová.  

\chapter{55}

Misericordia gratuita para todos  

55:1 A todos los sedientos: Venid a las aguas; y los que no tienen dinero, venid, comprad y comed. Venid, comprad sin dinero y sin precio, vino y leche.  
55:2 ¿Por qué gastáis el dinero en lo que no es pan, y vuestro trabajo en lo que no sacia? Oídme atentamente, y comed del bien, y se deleitará vuestra alma con grosura.  
55:3 Inclinad vuestro oído, y venid a mí; oíd, y vivirá vuestra alma; y haré con vosotros pacto eterno, las misericordias firmes a David. 
55:4 He aquí que yo lo di por testigo a los pueblos, por jefe y por maestro a las naciones.  
55:5 He aquí, llamarás a gente que no conociste, y gentes que no te conocieron correrán a ti, por causa de Jehová tu Dios, y del Santo de Israel que te ha honrado.  
55:6 Buscad a Jehová mientras puede ser hallado, llamadle en tanto que está cercano.  
55:7 Deje el impío su camino, y el hombre inicuo sus pensamientos, y vuélvase a Jehová, el cual tendrá de él misericordia, y al Dios nuestro, el cual será amplio en perdonar.  
55:8 Porque mis pensamientos no son vuestros pensamientos, ni vuestros caminos mis caminos, dijo Jehová.  
55:9 Como son más altos los cielos que la tierra, así son mis caminos más altos que vuestros caminos, y mis pensamientos más que vuestros pensamientos.  
55:10 Porque como desciende de los cielos la lluvia y la nieve, y no vuelve allá, sino que riega la tierra, y la hace germinar y producir, y da semilla al que siembra, y pan al que come,  
55:11 así será mi palabra que sale de mi boca; no volverá a mí vacía, sino que hará lo que yo quiero, y será prosperada en aquello para que la envié.  
55:12 Porque con alegría saldréis, y con paz seréis vueltos; los montes y los collados levantarán canción delante de vosotros, y todos los árboles del campo darán palmadas de aplauso.  
55:13 En lugar de la zarza crecerá ciprés, y en lugar de la ortiga crecerá arrayán; y será a Jehová por nombre, por señal eterna que nunca será raída.  

\chapter{56}

Recompensa de los que guardan el pacto de Dios  

56:1 Así dijo Jehová: Guardad derecho, y haced justicia; porque cercana está mi salvación para venir, y mi justicia para manifestarse.  
56:2 Bienaventurado el hombre que hace esto, y el hijo de hombre que lo abraza; que guarda el día de reposo para no profanarlo, y que guarda su mano de hacer todo mal.  
56:3 Y el extranjero que sigue a Jehová no hable diciendo: Me apartará totalmente Jehová de su pueblo. Ni diga el eunuco: He aquí yo soy árbol seco.  
56:4 Porque así dijo Jehová: A los eunucos que guarden mis días de reposo, y escojan lo que yo quiero, y abracen mi pacto,  
56:5 yo les daré lugar en mi casa y dentro de mis muros, y nombre mejor que el de hijos e hijas; nombre perpetuo les daré, que nunca perecerá.  
56:6 Y a los hijos de los extranjeros que sigan a Jehová para servirle, y que amen el nombre de Jehová para ser sus siervos; a todos los que guarden el día de reposo para no profanarlo, y abracen mi pacto,  
56:7 yo los llevaré a mi santo monte, y los recrearé en mi casa de oración; sus holocaustos y sus sacrificios serán aceptos sobre mi altar; porque mi casa será llamada casa de oración para todos los pueblos. 
56:8 Dice Jehová el Señor, el que reúne a los dispersos de Israel: Aún juntaré sobre él a sus congregados.  
56:9 Todas las bestias del campo, todas las fieras del bosque, venid a devorar.  
56:10 Sus atalayas son ciegos, todos ellos ignorantes; todos ellos perros mudos, no pueden ladrar; soñolientos, echados, aman el dormir.  
56:11 Y esos perros comilones son insaciables; y los pastores mismos no saben entender; todos ellos siguen sus propios caminos, cada uno busca su propio provecho, cada uno por su lado.  
56:12 Venid, dicen, tomemos vino, embriaguémonos de sidra; y será el día de mañana como este, o mucho más excelente.  

\chapter{57}

Condenación de la idolatría de Israel  

57:1 Perece el justo, y no hay quien piense en ello; y los piadosos mueren, y no hay quien entienda que de delante de la aflicción es quitado el justo.  
57:2 Entrará en la paz; descansarán en sus lechos todos los que andan delante de Dios.  
57:3 Mas vosotros llegaos acá, hijos de la hechicera, generación del adúltero y de la fornicaria.  
57:4 ¿De quién os habéis burlado? ¿Contra quién ensanchasteis la boca, y alargasteis la lengua? ¿No sois vosotros hijos rebeldes, generación mentirosa,  
57:5 que os enfervorizáis con los ídolos debajo de todo árbol frondoso, que sacrificáis los hijos en los valles, debajo de los peñascos?  
57:6 En las piedras lisas del valle está tu parte; ellas, ellas son tu suerte; y a ellas derramaste libación, y ofreciste presente. ¿No habré de castigar estas cosas?  
57:7 Sobre el monte alto y empinado pusiste tu cama; allí también subiste a hacer sacrificio.  
57:8 Y tras la puerta y el umbral pusiste tu recuerdo; porque a otro, y no a mí, te descubriste, y subiste, y ensanchaste tu cama, e hiciste con ellos pacto; amaste su cama dondequiera que la veías.  
57:9 Y fuiste al rey con ungüento, y multiplicaste tus perfumes, y enviaste tus embajadores lejos, y te abatiste hasta la profundidad del Seol.  
57:10 En la multitud de tus caminos te cansaste, pero no dijiste: No hay remedio; hallaste nuevo vigor en tu mano, por tanto, no te desalentaste.  
57:11 ¿Y de quién te asustaste y temiste, que has faltado a la fe, y no te has acordado de mí, ni te vino al pensamiento? ¿No he guardado silencio desde tiempos antiguos, y nunca me has temido?  
57:12 Yo publicaré tu justicia y tus obras, que no te aprovecharán.  
57:13 Cuando clames, que te libren tus ídolos; pero a todos ellos llevará el viento, un soplo los arrebatará; mas el que en mí confía tendrá la tierra por heredad, y poseerá mi santo monte.  
57:14 Y dirá: Allanad, allanad; barred el camino, quitad los tropiezos del camino de mi pueblo.  
57:15 Porque así dijo el Alto y Sublime, el que habita la eternidad, y cuyo nombre es el Santo: Yo habito en la altura y la santidad, y con el quebrantado y humilde de espíritu, para hacer vivir el espíritu de los humildes, y para vivificar el corazón de los quebrantados.  
57:16 Porque no contenderé para siempre, ni para siempre me enojaré; pues decaería ante mí el espíritu, y las almas que yo he creado.  
57:17 Por la iniquidad de su codicia me enojé, y le herí, escondí mi rostro y me indigné; y él siguió rebelde por el camino de su corazón.  
57:18 He visto sus caminos; pero le sanaré, y le pastorearé, y le daré consuelo a él y a sus enlutados;  
57:19 produciré fruto de labios: Paz, paz al que está lejos y al cercano, dijo Jehová; y lo sanaré.  
57:20 Pero los impíos son como el mar en tempestad, que no puede estarse quieto, y sus aguas arrojan cieno y lodo.  
57:21 No hay paz, dijo mi Dios, para los impíos. 

\chapter{58}

El verdadero ayuno  

58:1 Clama a voz en cuello, no te detengas; alza tu voz como trompeta, y anuncia a mi pueblo su rebelión, y a la casa de Jacob su pecado.  
58:2 Que me buscan cada día, y quieren saber mis caminos, como gente que hubiese hecho justicia, y que no hubiese dejado la ley de su Dios; me piden justos juicios, y quieren acercarse a Dios.  
58:3 ¿Por qué, dicen, ayunamos, y no hiciste caso; humillamos nuestras almas, y no te diste por entendido? He aquí que en el día de vuestro ayuno buscáis vuestro propio gusto, y oprimís a todos vuestros trabajadores.  
58:4 He aquí que para contiendas y debates ayunáis y para herir con el puño inicuamente; no ayunéis como hoy, para que vuestra voz sea oída en lo alto.  
58:5 ¿Es tal el ayuno que yo escogí, que de día aflija el hombre su alma, que incline su cabeza como junco, y haga cama de cilicio y de ceniza? ¿Llamaréis esto ayuno, y día agradable a Jehová?  
58:6 ¿No es más bien el ayuno que yo escogí, desatar las ligaduras de impiedad, soltar las cargas de opresión, y dejar ir libres a los quebrantados, y que rompáis todo yugo?  
58:7 ¿No es que partas tu pan con el hambriento, y a los pobres errantes albergues en casa; que cuando veas al desnudo, lo cubras, y no te escondas de tu hermano?  
58:8 Entonces nacerá tu luz como el alba, y tu salvación se dejará ver pronto; e irá tu justicia delante de ti, y la gloria de Jehová será tu retaguardia.  
58:9 Entonces invocarás, y te oirá Jehová; clamarás, y dirá él: Heme aquí. Si quitares de en medio de ti el yugo, el dedo amenazador, y el hablar vanidad;  
58:10 y si dieres tu pan al hambriento, y saciares al alma afligida, en las tinieblas nacerá tu luz, y tu oscuridad será como el mediodía.  
58:11 Jehová te pastoreará siempre, y en las sequías saciará tu alma, y dará vigor a tus huesos; y serás como huerto de riego, y como manantial de aguas, cuyas aguas nunca faltan.  
58:12 Y los tuyos edificarán las ruinas antiguas; los cimientos de generación y generación levantarás, y serás llamado reparador de portillos, restaurador de calzadas para habitar.  
La observancia del día de reposo  
58:13 Si retrajeres del día de reposo tu pie, de hacer tu voluntad en mi día santo, y lo llamares delicia, santo, glorioso de Jehová; y lo venerares, no andando en tus propios caminos, ni buscando tu voluntad, ni hablando tus propias palabras,  
58:14 entonces te deleitarás en Jehová; y yo te haré subir sobre las alturas de la tierra, y te daré a comer la heredad de Jacob tu padre; porque la boca de Jehová lo ha hablado.  

\chapter{59}

Confesión del pecado de Israel  

59:1 He aquí que no se ha acortado la mano de Jehová para salvar, ni se ha agravado su oído para oír;  
59:2 pero vuestras iniquidades han hecho división entre vosotros y vuestro Dios, y vuestros pecados han hecho ocultar de vosotros su rostro para no oír.  
59:3 Porque vuestras manos están contaminadas de sangre, y vuestros dedos de iniquidad; vuestros labios pronuncian mentira, habla maldad vuestra lengua.  
59:4 No hay quien clame por la justicia, ni quien juzgue por la verdad; confían en vanidad, y hablan vanidades; conciben maldades, y dan a luz iniquidad.  
59:5 Incuban huevos de áspides, y tejen telas de arañas; el que comiere de sus huevos, morirá; y si los apretaren, saldrán víboras.  
59:6 Sus telas no servirán para vestir, ni de sus obras serán cubiertos; sus obras son obras de iniquidad, y obra de rapiña está en sus manos.  
59:7 Sus pies corren al mal, se apresuran para derramar la sangre inocente; sus pensamientos, pensamientos de iniquidad; destrucción y quebrantamiento hay en sus caminos.  
59:8 No conocieron camino de paz, ni hay justicia en sus caminos; sus veredas son torcidas; cualquiera que por ellas fuere, no conocerá paz.  
59:9 Por esto se alejó de nosotros la justicia, y no nos alcanzó la rectitud; esperamos luz, y he aquí tinieblas; resplandores, y andamos en oscuridad.  
59:10 Palpamos la pared como ciegos, y andamos a tientas como sin ojos; tropezamos a mediodía como de noche; estamos en lugares oscuros como muertos.  
59:11 Gruñimos como osos todos nosotros, y gemimos lastimeramente como palomas; esperamos justicia, y no la hay; salvación, y se alejó de nosotros.  
59:12 Porque nuestras rebeliones se han multiplicado delante de ti, y nuestros pecados han atestiguado contra nosotros; porque con nosotros están nuestras iniquidades, y conocemos nuestros pecados:  
59:13 el prevaricar y mentir contra Jehová, y el apartarse de en pos de nuestro Dios; el hablar calumnia y rebelión, concebir y proferir de corazón palabras de mentira.  
59:14 Y el derecho se retiró, y la justicia se puso lejos; porque la verdad tropezó en la plaza, y la equidad no pudo venir.  
59:15 Y la verdad fue detenida, y el que se apartó del mal fue puesto en prisión; y lo vio Jehová, y desagradó a sus ojos, porque pereció el derecho.  
59:16 Y vio que no había hombre, y se maravilló que no hubiera quien se interpusiese; y lo salvó su brazo, y le afirmó su misma justicia. 
59:17 Pues de justicia se vistió como de una coraza, con yelmo de salvación en su cabeza; tomó ropas de venganza por vestidura, y se cubrió de celo como de manto,  
59:18 como para vindicación, como para retribuir con ira a sus enemigos, y dar el pago a sus adversarios; el pago dará a los de la costa.  
59:19 Y temerán desde el occidente el nombre de Jehová, y desde el nacimiento del sol su gloria; porque vendrá el enemigo como río, mas el Espíritu de Jehová levantará bandera contra él.  
59:20 Y vendrá el Redentor a Sion, y a los que se volvieren de la iniquidad en Jacob, dice Jehová.  
59:21 Y este será mi pacto con ellos, dijo Jehová: El Espíritu mío que está sobre ti, y mis palabras que puse en tu boca, no faltarán de tu boca, ni de la boca de tus hijos, ni de la boca de los hijos de tus hijos, dijo Jehová, desde ahora y para siempre.  

\chapter{60}

La futura gloria de Sion  

60:1 Levántate, resplandece; porque ha venido tu luz, y la gloria de Jehová ha nacido sobre ti.  
60:2 Porque he aquí que tinieblas cubrirán la tierra, y oscuridad las naciones; mas sobre ti amanecerá Jehová, y sobre ti será vista su gloria.  
60:3 Y andarán las naciones a tu luz, y los reyes al resplandor de tu nacimiento.  
60:4 Alza tus ojos alrededor y mira, todos éstos se han juntado, vinieron a ti; tus hijos vendrán de lejos, y tus hijas serán llevadas en brazos.  
60:5 Entonces verás, y resplandecerás; se maravillará y ensanchará tu corazón, porque se haya vuelto a ti la multitud del mar, y las riquezas de las naciones hayan venido a ti.  
60:6 Multitud de camellos te cubrirá; dromedarios de Madián y de Efa; vendrán todos los de Sabá; traerán oro e incienso, y publicarán alabanzas de Jehová.  
60:7 Todo el ganado de Cedar será juntado para ti; carneros de Nebaiot te serán servidos; serán ofrecidos con agrado sobre mi altar, y glorificaré la casa de mi gloria.  
60:8 ¿Quiénes son éstos que vuelan como nubes, y como palomas a sus ventanas?  
60:9 Ciertamente a mí esperarán los de la costa, y las naves de Tarsis desde el principio, para traer tus hijos de lejos, su plata y su oro con ellos, al nombre de Jehová tu Dios, y al Santo de Israel, que te ha glorificado.  
60:10 Y extranjeros edificarán tus muros, y sus reyes te servirán; porque en mi ira te castigué, mas en mi buena voluntad tendré de ti misericordia.  
60:11 Tus puertas estarán de continuo abiertas; no se cerrarán de día ni de noche, para que a ti sean traídas las riquezas de las naciones, y conducidos a ti sus reyes.  
60:12 Porque la nación o el reino que no te sirviere perecerá, y del todo será asolado. 
60:13 La gloria del Líbano vendrá a ti, cipreses, pinos y bojes juntamente, para decorar el lugar de mi santuario; y yo honraré el lugar de mis pies.  
60:14 Y vendrán a ti humillados los hijos de los que te afligieron, y a las pisadas de tus pies se encorvarán todos los que te escarnecían, y te llamarán Ciudad de Jehová, Sion del Santo de Israel.  
60:15 En vez de estar abandonada y aborrecida, tanto que nadie pasaba por ti, haré que seas una gloria eterna, el gozo de todos los siglos.  
60:16 Y mamarás la leche de las naciones, el pecho de los reyes mamarás; y conocerás que yo Jehová soy el Salvador tuyo y Redentor tuyo, el Fuerte de Jacob.  
60:17 En vez de bronce traeré oro, y por hierro plata, y por madera bronce, y en lugar de piedras hierro; y pondré paz por tu tributo, y justicia por tus opresores.  
60:18 Nunca más se oirá en tu tierra violencia, destrucción ni quebrantamiento en tu territorio, sino que a tus muros llamarás Salvación, y a tus puertas Alabanza.  
60:19 El sol nunca más te servirá de luz para el día, ni el resplandor de la luna te alumbrará, sino que Jehová te será por luz perpetua, y el Dios tuyo por tu gloria. 
60:20 No se pondrá jamás tu sol, ni menguará tu luna; porque Jehová te será por luz perpetua, y los días de tu luto serán acabados.  
60:21 Y tu pueblo, todos ellos serán justos, para siempre heredarán la tierra; renuevos de mi plantío, obra de mis manos, para glorificarme.  
60:22 El pequeño vendrá a ser mil, el menor, un pueblo fuerte. Yo Jehová, a su tiempo haré que esto sea cumplido pronto.  

\chapter{61}

Buenas nuevas de salvación para Sion  

61:1 El Espíritu de Jehová el Señor está sobre mí, porque me ungió Jehová; me ha enviado a predicar buenas nuevas a los abatidos, a vendar a los quebrantados de corazón, a publicar libertad a los cautivos, y a los presos apertura de la cárcel;  
61:2 a proclamar el año de la buena voluntad de Jehová, y el día de venganza del Dios nuestro; a consolar a todos los enlutados; 
61:3 a ordenar que a los afligidos de Sion se les dé gloria en lugar de ceniza, óleo de gozo en lugar de luto, manto de alegría en lugar del espíritu angustiado; y serán llamados árboles de justicia, plantío de Jehová, para gloria suya.  
61:4 Reedificarán las ruinas antiguas, y levantarán los asolamientos primeros, y restaurarán las ciudades arruinadas, los escombros de muchas generaciones.  
61:5 Y extranjeros apacentarán vuestras ovejas, y los extraños serán vuestros labradores y vuestros viñadores.  
61:6 Y vosotros seréis llamados sacerdotes de Jehová, ministros de nuestro Dios seréis llamados; comeréis las riquezas de las naciones, y con su gloria seréis sublimes.  
61:7 En lugar de vuestra doble confusión y de vuestra deshonra, os alabarán en sus heredades; por lo cual en sus tierras poseerán doble honra, y tendrán perpetuo gozo.  
61:8 Porque yo Jehová soy amante del derecho, aborrecedor del latrocinio para holocausto; por tanto, afirmaré en verdad su obra, y haré con ellos pacto perpetuo.  
61:9 Y la descendencia de ellos será conocida entre las naciones, y sus renuevos en medio de los pueblos; todos los que los vieren, reconocerán que son linaje bendito de Jehová. 
61:10 En gran manera me gozaré en Jehová, mi alma se alegrará en mi Dios; porque me vistió con vestiduras de salvación, me rodeó de manto de justicia, como a novio me atavió, y como a novia adornada con sus joyas. 
61:11 Porque como la tierra produce su renuevo, y como el huerto hace brotar su semilla, así Jehová el Señor hará brotar justicia y alabanza delante de todas las naciones.  

\chapter{62}


62:1 Por amor de Sion no callaré, y por amor de Jerusalén no descansaré, hasta que salga como resplandor su justicia, y su salvación se encienda como una antorcha.  
62:2 Entonces verán las gentes tu justicia, y todos los reyes tu gloria; y te será puesto un nombre nuevo, que la boca de Jehová nombrará.  
62:3 Y serás corona de gloria en la mano de Jehová, y diadema de reino en la mano del Dios tuyo.  
62:4 Nunca más te llamarán Desamparada, ni tu tierra se dirá más Desolada; sino que serás llamada Hefzi-bá, y tu tierra, Beula; porque el amor de Jehová estará en ti, y tu tierra será desposada.  
62:5 Pues como el joven se desposa con la virgen, se desposarán contigo tus hijos; y como el gozo del esposo con la esposa, así se gozará contigo el Dios tuyo.  
62:6 Sobre tus muros, oh Jerusalén, he puesto guardas; todo el día y toda la noche no callarán jamás. Los que os acordáis de Jehová, no reposéis,  
62:7 ni le deis tregua, hasta que restablezca a Jerusalén, y la ponga por alabanza en la tierra.  
62:8 Juró Jehová por su mano derecha, y por su poderoso brazo: Que jamás daré tu trigo por comida a tus enemigos, ni beberán los extraños el vino que es fruto de tu trabajo;  
62:9 sino que los que lo cosechan lo comerán, y alabarán a Jehová; y los que lo vendimian, lo beberán en los atrios de mi santuario.  
62:10 Pasad, pasad por las puertas; barred el camino al pueblo; allanad, allanad la calzada, quitad las piedras, alzad pendón a los pueblos.  
62:11 He aquí que Jehová hizo oír hasta lo último de la tierra: Decid a la hija de Sion: He aquí viene tu Salvador; he aquí su recompensa con él, y delante de él su obra. 
62:12 Y les llamarán Pueblo Santo, Redimidos de Jehová; y a ti te llamarán Ciudad Deseada, no desamparada.  

\chapter{63}

El día de la venganza de Jehová  

63:1 ¿Quién es éste que viene de Edom, de Bosra, con vestidos rojos? ¿éste hermoso en su vestido, que marcha en la grandeza de su poder? Yo, el que hablo en justicia, grande para salvar.  
63:2 ¿Por qué es rojo tu vestido, y tus ropas como del que ha pisado en lagar?  
63:3 He pisado yo solo el lagar, y de los pueblos nadie había conmigo; los pisé con mi ira, y los hollé con mi furor; y su sangre salpicó mis vestidos, y manché todas mis ropas. 
63:4 Porque el día de la venganza está en mi corazón, y el año de mis redimidos ha llegado.  
63:5 Miré, y no había quien ayudara, y me maravillé que no hubiera quien sustentase; y me salvó mi brazo, y me sostuvo mi ira. 
63:6 Y con mi ira hollé los pueblos, y los embriagué en mi furor, y derramé en tierra su sangre.  
Bondad de Jehová hacia Israel  
63:7 De las misericordias de Jehová haré memoria, de las alabanzas de Jehová, conforme a todo lo que Jehová nos ha dado, y de la grandeza de sus beneficios hacia la casa de Israel, que les ha hecho según sus misericordias, y según la multitud de sus piedades.  
63:8 Porque dijo: Ciertamente mi pueblo son, hijos que no mienten; y fue su Salvador.  
63:9 En toda angustia de ellos él fue angustiado, y el ángel de su faz los salvó; en su amor y en su clemencia los redimió, y los trajo, y los levantó todos los días de la antigüedad.  
63:10 Mas ellos fueron rebeldes, e hicieron enojar su santo espíritu; por lo cual se les volvió enemigo, y él mismo peleó contra ellos.  
63:11 Pero se acordó de los días antiguos, de Moisés y de su pueblo, diciendo: ¿Dónde está el que les hizo subir del mar con el pastor de su rebaño? ¿dónde el que puso en medio de él su santo espíritu,  
63:12 el que los guió por la diestra de Moisés con el brazo de su gloria; el que dividió las aguas delante de ellos, haciéndose así nombre perpetuo,  
63:13 el que los condujo por los abismos, como un caballo por el desierto, sin que tropezaran?  
63:14 El Espíritu de Jehová los pastoreó, como a una bestia que desciende al valle; así pastoreaste a tu pueblo, para hacerte nombre glorioso.  
Plegaria pidiendo misericordia y ayuda  
63:15 Mira desde el cielo, y contempla desde tu santa y gloriosa morada. ¿Dónde está tu celo, y tu poder, la conmoción de tus entrañas y tus piedades para conmigo? ¿Se han estrechado?  
63:16 Pero tú eres nuestro padre, si bien Abraham nos ignora, e Israel no nos conoce; tú, oh Jehová, eres nuestro padre; nuestro Redentor perpetuo es tu nombre.  
63:17 ¿Por qué, oh Jehová, nos has hecho errar de tus caminos, y endureciste nuestro corazón a tu temor? Vuélvete por amor de tus siervos, por las tribus de tu heredad.  
63:18 Por poco tiempo lo poseyó tu santo pueblo; nuestros enemigos han hollado tu santuario.  
63:19 Hemos venido a ser como aquellos de quienes nunca te enseñoreaste, sobre los cuales nunca fue llamado tu nombre.  

\chapter{64}


64:1 ¡Oh, si rompieses los cielos, y descendieras, y a tu presencia se escurriesen los montes,  
64:2 como fuego abrasador de fundiciones, fuego que hace hervir las aguas, para que hicieras notorio tu nombre a tus enemigos, y las naciones temblasen a tu presencia!  
64:3 Cuando, haciendo cosas terribles cuales nunca esperábamos, descendiste, fluyeron los montes delante de ti.  
64:4 Ni nunca oyeron, ni oídos percibieron, ni ojo ha visto a Dios fuera de ti, que hiciese por el que en él espera. 
64:5 Saliste al encuentro del que con alegría hacía justicia, de los que se acordaban de ti en tus caminos; he aquí, tú te enojaste porque pecamos; en los pecados hemos perseverado por largo tiempo; ¿podremos acaso ser salvos?  
64:6 Si bien todos nosotros somos como suciedad, y todas nuestras justicias como trapo de inmundicia; y caímos todos nosotros como la hoja, y nuestras maldades nos llevaron como viento.  
64:7 Nadie hay que invoque tu nombre, que se despierte para apoyarse en ti; por lo cual escondiste de nosotros tu rostro, y nos dejaste marchitar en poder de nuestras maldades.  
64:8 Ahora pues, Jehová, tú eres nuestro padre; nosotros barro, y tú el que nos formaste; así que obra de tus manos somos todos nosotros.  
64:9 No te enojes sobremanera, Jehová, ni tengas perpetua memoria de la iniquidad; he aquí, mira ahora, pueblo tuyo somos todos nosotros.  
64:10 Tus santas ciudades están desiertas, Sion es un desierto, Jerusalén una soledad.  
64:11 La casa de nuestro santuario y de nuestra gloria, en la cual te alabaron nuestros padres, fue consumida al fuego; y todas nuestras cosas preciosas han sido destruidas.  
64:12 ¿Te estarás quieto, oh Jehová, sobre estas cosas? ¿Callarás, y nos afligirás sobremanera?  

\chapter{65}

Castigo de los rebeldes  

65:1 Fui buscado por los que no preguntaban por mí; fui hallado por los que no me buscaban. Dije a gente que no invocaba mi nombre: Heme aquí, heme aquí.  
65:2 Extendí mis manos todo el día a pueblo rebelde, el cual anda por camino no bueno, en pos de sus pensamientos;  
65:3 pueblo que en mi rostro me provoca de continuo a ira, sacrificando en huertos, y quemando incienso sobre ladrillos;  
65:4 que se quedan en los sepulcros, y en lugares escondidos pasan la noche; que comen carne de cerdo, y en sus ollas hay caldo de cosas inmundas;  
65:5 que dicen: Estate en tu lugar, no te acerques a mí, porque soy más santo que tú; éstos son humo en mi furor, fuego que arde todo el día.  
65:6 He aquí que escrito está delante de mí; no callaré, sino que recompensaré, y daré el pago en su seno  
65:7 por vuestras iniquidades, dice Jehová, y por las iniquidades de vuestros padres juntamente, los cuales quemaron incienso sobre los montes, y sobre los collados me afrentaron; por tanto, yo les mediré su obra antigua en su seno.  
65:8 Así ha dicho Jehová: Como si alguno hallase mosto en un racimo, y dijese: No lo desperdicies, porque bendición hay en él; así haré yo por mis siervos, que no lo destruiré todo.  
65:9 Sacaré descendencia de Jacob, y de Judá heredero de mis montes; y mis escogidos poseerán por heredad la tierra, y mis siervos habitarán allí.  
65:10 Y será Sarón para habitación de ovejas, y el valle de Acor para majada de vacas, para mi pueblo que me buscó.  
65:11 Pero vosotros los que dejáis a Jehová, que olvidáis mi santo monte, que ponéis mesa para la Fortuna, y suministráis libaciones para el Destino;  
65:12 yo también os destinaré a la espada, y todos vosotros os arrodillaréis al degolladero, por cuanto llamé, y no respondisteis; hablé, y no oísteis, sino que hicisteis lo malo delante de mis ojos, y escogisteis lo que me desagrada.  
65:13 Por tanto, así dijo Jehová el Señor: He aquí que mis siervos comerán, y vosotros tendréis hambre; he aquí que mis siervos beberán, y vosotros tendréis sed; he aquí que mis siervos se alegrarán, y vosotros seréis avergonzados;  
65:14 he aquí que mis siervos cantarán por júbilo del corazón, y vosotros clamaréis por el dolor del corazón, y por el quebrantamiento de espíritu aullaréis.  
65:15 Y dejaréis vuestro nombre por maldición a mis escogidos, y Jehová el Señor te matará, y a sus siervos llamará por otro nombre.  
65:16 El que se bendijere en la tierra, en el Dios de verdad se bendecirá; y el que jurare en la tierra, por el Dios de verdad jurará; porque las angustias primeras serán olvidadas, y serán cubiertas de mis ojos.  
Cielos nuevos y tierra nueva  
65:17 Porque he aquí que yo crearé nuevos cielos y nueva tierra; y de lo primero no habrá memoria, ni más vendrá al pensamiento.  
65:18 Mas os gozaréis y os alegraréis para siempre en las cosas que yo he creado; porque he aquí que yo traigo a Jerusalén alegría, y a su pueblo gozo.  
65:19 Y me alegraré con Jerusalén, y me gozaré con mi pueblo; y nunca más se oirán en ella voz de lloro, ni voz de clamor.  
65:20 No habrá más allí niño que muera de pocos días, ni viejo que sus días no cumpla; porque el niño morirá de cien años, y el pecador de cien años será maldito.  
65:21 Edificarán casas, y morarán en ellas; plantarán viñas, y comerán el fruto de ellas.  
65:22 No edificarán para que otro habite, ni plantarán para que otro coma; porque según los días de los árboles serán los días de mi pueblo, y mis escogidos disfrutarán la obra de sus manos.  
65:23 No trabajarán en vano, ni darán a luz para maldición; porque son linaje de los benditos de Jehová, y sus descendientes con ellos.  
65:24 Y antes que clamen, responderé yo; mientras aún hablan, yo habré oído.  
65:25 El lobo y el cordero serán apacentados juntos, y el león comerá paja como el buey; y el polvo será el alimento de la serpiente. No afligirán, ni harán mal en todo mi santo monte, dijo Jehová. 

\chapter{66}

Los juicios de Jehová y la futura prosperidad de Sion  

66:1 Jehová dijo así: El cielo es mi trono, y la tierra estrado de mis pies; ¿dónde está la casa que me habréis de edificar, y dónde el lugar de mi reposo? 
66:2 Mi mano hizo todas estas cosas, y así todas estas cosas fueron, dice Jehová; pero miraré a aquel que es pobre y humilde de espíritu, y que tiembla a mi palabra.  
66:3 El que sacrifica buey es como si matase a un hombre; el que sacrifica oveja, como si degollase un perro; el que hace ofrenda, como si ofreciese sangre de cerdo; el que quema incienso, como si bendijese a un ídolo. Y porque escogieron sus propios caminos, y su alma amó sus abominaciones,  
66:4 también yo escogeré para ellos escarnios, y traeré sobre ellos lo que temieron; porque llamé, y nadie respondió; hablé, y no oyeron, sino que hicieron lo malo delante de mis ojos, y escogieron lo que me desagrada.  
66:5 Oíd palabra de Jehová, vosotros los que tembláis a su palabra: Vuestros hermanos que os aborrecen, y os echan fuera por causa de mi nombre, dijeron: Jehová sea glorificado. Pero él se mostrará para alegría vuestra, y ellos serán confundidos.  
66:6 Voz de alboroto de la ciudad, voz del templo, voz de Jehová que da el pago a sus enemigos.  
66:7 Antes que estuviese de parto, dio a luz; antes que le viniesen dolores, dio a luz hijo. 
66:8 ¿Quién oyó cosa semejante? ¿quién vio tal cosa? ¿Concebirá la tierra en un día? ¿Nacerá una nación de una vez? Pues en cuanto Sion estuvo de parto, dio a luz sus hijos.  
66:9 Yo que hago dar a luz, ¿no haré nacer? dijo Jehová. Yo que hago engendrar, ¿impediré el nacimiento? dice tu Dios.  
66:10 Alegraos con Jerusalén, y gozaos con ella, todos los que la amáis; llenaos con ella de gozo, todos los que os enlutáis por ella;  
66:11 para que maméis y os saciéis de los pechos de sus consolaciones; para que bebáis, y os deleitéis con el resplandor de su gloria.  
66:12 Porque así dice Jehová: He aquí que yo extiendo sobre ella paz como un río, y la gloria de las naciones como torrente que se desborda; y mamaréis, y en los brazos seréis traídos, y sobre las rodillas seréis mimados.  
66:13 Como aquel a quien consuela su madre, así os consolaré yo a vosotros, y en Jerusalén tomaréis consuelo.  
66:14 Y veréis, y se alegrará vuestro corazón, y vuestros huesos reverdecerán como la hierba; y la mano de Jehová para con sus siervos será conocida, y se enojará contra sus enemigos.  
66:15 Porque he aquí que Jehová vendrá con fuego, y sus carros como torbellino, para descargar su ira con furor, y su reprensión con llama de fuego.  
66:16 Porque Jehová juzgará con fuego y con su espada a todo hombre; y los muertos de Jehová serán multiplicados.  
66:17 Los que se santifican y los que se purifican en los huertos, unos tras otros, los que comen carne de cerdo y abominación y ratón, juntamente serán talados, dice Jehová.  
66:18 Porque yo conozco sus obras y sus pensamientos; tiempo vendrá para juntar a todas las naciones y lenguas; y vendrán, y verán mi gloria.  
66:19 Y pondré entre ellos señal, y enviaré de los escapados de ellos a las naciones, a Tarsis, a Fut y Lud que disparan arco, a Tubal y a Javán, a las costas lejanas que no oyeron de mí, ni vieron mi gloria; y publicarán mi gloria entre las naciones.  
66:20 Y traerán a todos vuestros hermanos de entre todas las naciones, por ofrenda a Jehová, en caballos, en carros, en literas, en mulos y en camellos, a mi santo monte de Jerusalén, dice Jehová, al modo que los hijos de Israel traen la ofrenda en utensilios limpios a la casa de Jehová.  
66:21 Y tomaré también de ellos para sacerdotes y levitas, dice Jehová.  
66:22 Porque como los cielos nuevos y la nueva tierra que yo hago permanecerán delante de mí, dice Jehová, así permanecerá vuestra descendencia y vuestro nombre.  
66:23 Y de mes en mes, y de día de reposo en día de reposo, vendrán todos a adorar delante de mí, dijo Jehová.  
66:24 Y saldrán, y verán los cadáveres de los hombres que se rebelaron contra mí; porque su gusano nunca morirá, ni su fuego se apagará, y serán abominables a todo hombre. 

\end{document}