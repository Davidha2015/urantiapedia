\begin{document}

\title{Jonás}

\chapter{1}

\section*{Jonás huye de Jehová}

\par 1 Vino palabra de Jehová a Jonás hijo de Amitai, diciendo:
\par 2 Levántate y ve a Nínive, aquella gran ciudad, y pregona contra ella; porque ha subido su maldad delante de mí.
\par 3 Y Jonás se levantó para huir de la presencia de Jehová a Tarsis, y descendió a Jope, y halló una nave que partía para Tarsis; y pagando su pasaje, entró en ella para irse con ellos a Tarsis, lejos de la presencia de Jehová.
\par 4 Pero Jehová hizo levantar un gran viento en el mar, y hubo en el mar una tempestad tan grande que se pensó que se partiría la nave.
\par 5 Y los marineros tuvieron miedo, y cada uno clamaba a su dios; y echaron al mar los enseres que había en la nave, para descargarla de ellos. Pero Jonás había bajado al interior de la nave, y se había echado a dormir.
\par 6 Y el patrón de la nave se le acercó y le dijo: ¿Qué tienes, dormilón? Levántate, y clama a tu Dios; quizá él tendrá compasión de nosotros, y no pereceremos.
\par 7 Y dijeron cada uno a su compañero: Venid y echemos suertes, para que sepamos por causa de quién nos ha venido este mal. Y echaron suertes, y la suerte cayó sobre Jonás.
\par 8 Entonces le dijeron ellos: Decláranos ahora por qué nos ha venido este mal. ¿Qué oficio tienes, y de dónde vienes? ¿Cuál es tu tierra, y de qué pueblo eres?
\par 9 Y él les respondió: Soy hebreo, y temo a Jehová, Dios de los cielos, que hizo el mar y la tierra.
\par 10 Y aquellos hombres temieron sobremanera, y le dijeron: ¿Por qué has hecho esto? Porque ellos sabían que huía de la presencia de Jehová, pues él se lo había declarado.
\par 11 Y le dijeron: ¿Qué haremos contigo para que el mar se nos aquiete? Porque el mar se iba embraveciendo más y más.
\par 12 El les respondió: Tomadme y echadme al mar, y el mar se os aquietará; porque yo sé que por mi causa ha venido esta gran tempestad sobre vosotros. 
\par 13 Y aquellos hombres trabajaron para hacer volver la nave a tierra; mas no pudieron, porque el mar se iba embraveciendo más y más contra ellos.
\par 14 Entonces clamaron a Jehová y dijeron: Te rogamos ahora, Jehová, que no perezcamos nosotros por la vida de este hombre, ni pongas sobre nosotros la sangre inocente; porque tú, Jehová, has hecho como has querido.
\par 15 Y tomaron a Jonás, y lo echaron al mar; y el mar se aquietó de su furor.
\par 16 Y temieron aquellos hombres a Jehová con gran temor, y ofrecieron sacrificio a Jehová, e hicieron votos.
\par 17 Pero Jehová tenía preparado un gran pez que tragase a Jonás; y estuvo Jonás en el vientre del pez tres días y tres noches. 

\chapter{2}

\section*{Oración de Jonás}

\par 1 Entonces oró Jonás a Jehová su Dios desde el vientre del pez,
\par 2 y dijo:
\par Invoqué en mi angustia a Jehová, y él me oyó;
\par Desde el seno del Seol clamé,
\par Y mi voz oíste.
\par 3 Me echaste a lo profundo, en medio de los mares,
\par Y me rodeó la corriente;
\par Todas tus ondas y tus olas pasaron sobre mí.
\par 4 Entonces dije: Desechado soy de delante de tus ojos;
\par Mas aún veré tu santo templo.
\par 5 Las aguas me rodearon hasta el alma,
\par Rodeóme el abismo;
\par El alga se enredó a mi cabeza.
\par 6 Descendí a los cimientos de los montes;
\par La tierra echó sus cerrojos sobre mí para siempre;
\par Mas tú sacaste mi vida de la sepultura, oh Jehová Dios mío.
\par 7 Cuando mi alma desfallecía en mí, me acordé de Jehová,
\par Y mi oración llegó hasta ti en tu santo templo.
\par 8 Los que siguen vanidades ilusorias,
\par Su misericordia abandonan.
\par 9 Mas yo con voz de alabanza te ofreceré sacrificios;
\par Pagaré lo que prometí.
\par La salvación es de Jehová.
\par 10 Y mandó Jehová al pez, y vomitó a Jonás en tierra.

\chapter{3}

\section*{Nínive se arrepiente}

\par 1 Vino palabra de Jehová por segunda vez a Jonás, diciendo:
\par 2 Levántate y ve a Nínive, aquella gran ciudad, y proclama en ella el mensaje que yo te diré.
\par 3 Y se levantó Jonás, y fue a Nínive conforme a la palabra de Jehová. Y era Nínive ciudad grande en extremo, de tres días de camino.
\par 4 Y comenzó Jonás a entrar por la ciudad, camino de un día, y predicaba diciendo: De aquí a cuarenta días Nínive será destruida.
\par 5 Y los hombres de Nínive creyeron a Dios, y proclamaron ayuno, y se vistieron de cilicio desde el mayor hasta el menor de ellos. 
\par 6 Y llegó la noticia hasta el rey de Nínive, y se levantó de su silla, se despojó de su vestido, y se cubrió de cilicio y se sentó sobre ceniza.
\par 7 E hizo proclamar y anunciar en Nínive, por mandato del rey y de sus grandes, diciendo: Hombres y animales, bueyes y ovejas, no gusten cosa alguna; no se les dé alimento, ni beban agua;
\par 8 sino cúbranse de cilicio hombres y animales, y clamen a Dios fuertemente; y conviértase cada uno de su mal camino, de la rapiña que hay en sus manos.
\par 9 ¿Quién sabe si se volverá y se arrepentirá Dios, y se apartará del ardor de su ira, y no pereceremos?
\par 10 Y vio Dios lo que hicieron, que se convirtieron de su mal camino; y se arrepintió del mal que había dicho que les haría, y no lo hizo.

\chapter{4}

\section*{El enojo de Jonás}

\par 1 Pero Jonás se apesadumbró en extremo, y se enojó.
\par 2 Y oró a Jehová y dijo: Ahora, oh Jehová, ¿no es esto lo que yo decía estando aún en mi tierra? Por eso me apresuré a huir a Tarsis; porque sabía yo que tú eres Dios clemente y piadoso, tardo en enojarte, y de grande misericordia, y que te arrepientes del mal.
\par 3 Ahora pues, oh Jehová, te ruego que me quites la vida; porque mejor me es la muerte que la vida.
\par 4 Y Jehová le dijo: ¿Haces tú bien en enojarte tanto?
\par 5 Y salió Jonás de la ciudad, y acampó hacia el oriente de la ciudad, y se hizo allí una enramada, y se sentó debajo de ella a la sombra, hasta ver qué acontecería en la ciudad.
\par 6 Y preparó Jehová Dios una calabacera, la cual creció sobre Jonás para que hiciese sombra sobre su cabeza, y le librase de su malestar; y Jonás se alegró grandemente por la calabacera.
\par 7 Pero al venir el alba del día siguiente, Dios preparó un gusano, el cual hirió la calabacera, y se secó.
\par 8 Y aconteció que al salir el sol, preparó Dios un recio viento solano, y el sol hirió a Jonás en la cabeza, y se desmayaba, y deseaba la muerte, diciendo: Mejor sería para mí la muerte que la vida.
\par 9 Entonces dijo Dios a Jonás: ¿Tanto te enojas por la calabacera? Y él respondió: Mucho me enojo, hasta la muerte.
\par 10 Y dijo Jehová: Tuviste tú lástima de la calabacera, en la cual no trabajaste, ni tú la hiciste crecer; que en espacio de una noche nació, y en espacio de otra noche pereció.
\par 11 ¿Y no tendré yo piedad de Nínive, aquella gran ciudad donde hay más de ciento veinte mil personas que no saben discernir entre su mano derecha y su mano izquierda, y muchos animales?

\end{document}