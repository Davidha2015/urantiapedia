\begin{document}
%\title{EL EVANGELIO SEGÚN SAN MATEO}
\title{Evangelio de Mateo}

\chapter{1}

\section*{Genealogía de Jesucristo}

\par 1 Libro de la genealogía de Jesucristo, hijo de David, hijo de Abraham.
\par 2 Abraham engendró a Isaac, Isaac a Jacob, y Jacob a Judá y a sus hermanos.
\par 3 Judá engendró de Tamar a Fares y a Zara, Fares a Esrom, y Esrom a Aram.
\par 4 Aram engendró a Aminadab, Aminadab a Naasón, y Naasón a Salmón.
\par 5 Salmón engendró de Rahab a Booz, Booz engendró de Rut a Obed, y Obed a Isa.
\par 6 Isaí engendró al rey David, y el rey David engendró a Salomón de la que fue mujer de Urías.
\par 7 Salomón engendró a Roboam, Roboam a Abías, y Abías a Asa.
\par 8 Asa engendró a Josafat, Josafat a Joram, y Joram a Uzías.
\par 9 Uzías engendró a Jotam, Jotam a Acaz, y Acaz a Ezequías.
\par 10 Ezequías engendró a Manasés, Manasés a Amón, y Amón a Josías.
\par 11 Josías engendró a Jeconías y a sus hermanos, en el tiempo de la deportación a Babilonia.
\par 12 Después de la deportación a Babilonia, Jeconías engendró a Salatiel, y Salatiel a Zorobabel.
\par 13 Zorobabel engendró a Abiud, Abiud a Eliaquim, y Eliaquim a Azor.
\par 14 Azor engendró a Sadoc, Sadoc a Aquim, y Aquim a Eliud.
\par 15 Eliud engendró a Eleazar, Eleazar a Matán, Matán a Jacob;
\par 16 y Jacob engendró a José, marido de María, de la cual nació Jesús, llamado el Cristo.
\par 17 De manera que todas las generaciones desde Abraham hasta David son catorce; desde David hasta la deportación a Babilonia, catorce; y desde la deportación a Babilonia hasta Cristo, catorce.

\section*{Nacimiento de Jesucristo}

\par 18 El nacimiento de Jesucristo fue así: Estando desposada María su madre con José, antes que se juntasen, se halló que había concebido del Espíritu Santo.
\par 19 José su marido, como era justo, y no quería infamarla, quiso dejarla secretamente.
\par 20 Y pensando él en esto, he aquí un ángel del Señor le apareció en sueños y le dijo: José, hijo de David, no temas recibir a María tu mujer, porque lo que en ella es engendrado, del Espíritu Santo es.
\par 21 Y dará a luz un hijo, y llamarás su nombre JESÚS, porque él salvará a su pueblo de sus pecados.
\par 22 Todo esto aconteció para que se cumpliese lo dicho por el Señor por medio del profeta, cuando dijo:
\par 23 He aquí, una virgen concebirá y dará a luz un hijo,
\par Y llamarás su nombre Emanuel,
\par que traducido es: Dios con nosotros.
\par 24 Y despertando José del sueño, hizo como el ángel del Señor le había mandado, y recibió a su mujer.
\par 25 Pero no la conoció hasta que dio a luz a su hijo primogénito; y le puso por nombre JESÚS.

\chapter{2}

\section*{La visita de los magos}

\par 1 Cuando Jesús nació en Belén de Judea en días del rey Herodes, vinieron del oriente a Jerusalén unos magos,
\par 2 diciendo: ¿Dónde está el rey de los judíos, que ha nacido? Porque su estrella hemos visto en el oriente, y venimos a adorarle.
\par 3 Oyendo esto, el rey Herodes se turbó, y toda Jerusalén con él.
\par 4 Y convocados todos los principales sacerdotes, y los escribas del pueblo, les preguntó dónde había de nacer el Cristo.
\par 5 Ellos le dijeron: En Belén de Judea; porque así está escrito por el profeta:
\par 6 Y tú, Belén, de la tierra de Judá,
\par No eres la más pequeña entre los príncipes de Judá;
\par Porque de ti saldrá un guiador,
\par Que apacentará a mi pueblo Israel.
\par 7 Entonces Herodes, llamando en secreto a los magos, indagó de ellos diligentemente el tiempo de la aparición de la estrella;
\par 8 y enviándolos a Belén, dijo: Id allá y averiguad con diligencia acerca del niño; y cuando le halléis, hacédmelo saber, para que yo también vaya y le adore.
\par 9 Ellos, habiendo oído al rey, se fueron; y he aquí la estrella que habían visto en el oriente iba delante de ellos, hasta que llegando, se detuvo sobre donde estaba el niño.
\par 10 Y al ver la estrella, se regocijaron con muy grande gozo.
\par 11 Y al entrar en la casa, vieron al niño con su madre María, y postrándose, lo adoraron; y abriendo sus tesoros, le ofrecieron presentes: oro, incienso y mirra.
\par 12 Pero siendo avisados por revelación en sueños que no volviesen a Herodes, regresaron a su tierra por otro camino.

\section*{Matanza de los niños}

\par 13 Después que partieron ellos, he aquí un ángel del Señor apareció en sueños a José y dijo: Levántate y toma al niño y a su madre, y huye a Egipto, y permanece allá hasta que yo te diga; porque acontecerá que Herodes buscará al niño para matarlo.
\par 14 Y él, despertando, tomó de noche al niño y a su madre, y se fue a Egipto,
\par 15 y estuvo allá hasta la muerte de Herodes; para que se cumpliese lo que dijo el Señor por medio del profeta, cuando dijo: De Egipto llamé a mi Hijo.
\par 16 Herodes entonces, cuando se vio burlado por los magos, se enojó mucho, y mandó matar a todos los niños menores de dos años que había en Belén y en todos sus alrededores, conforme al tiempo que había inquirido de los magos.
\par 17 Entonces se cumplió lo que fue dicho por el profeta Jeremías, cuando dijo:
\par 18 Voz fue oída en Ramá,
\par Grande lamentación, lloro y gemido;
\par Raquel que llora a sus hijos,
\par Y no quiso ser consolada, porque perecieron.
\par 19 Pero después de muerto Herodes, he aquí un ángel del Señor apareció en sueños a José en Egipto,
\par 20 diciendo: Levántate, toma al niño y a su madre, y vete a tierra de Israel, porque han muerto los que procuraban la muerte del niño.
\par 21 Entonces él se levantó, y tomó al niño y a su madre, y vino a tierra de Israel.
\par 22 Pero oyendo que Arquelao reinaba en Judea en lugar de Herodes su padre, tuvo temor de ir allá; pero avisado por revelación en sueños, se fue a la región de Galilea,
\par 23 y vino y habitó en la ciudad que se llama Nazaret, para que se cumpliese lo que fue dicho por los profetas, que habría de ser llamado nazareno.

\chapter{3}

\section*{Predicación de Juan el Bautista}

\par 1 En aquellos días vino Juan el Bautista predicando en el desierto de Judea,
\par 2 y diciendo: Arrepentíos, porque el reino de los cielos se ha acercado.
\par 3 Pues éste es aquel de quien habló el profeta Isaías, cuando dijo:
\par Voz del que clama en el desierto:
\par Preparad el camino del Señor,
\par Enderezad sus sendas.
\par 4 Y Juan estaba vestido de pelo de camello, y tenía un cinto de cuero alrededor de sus lomos; y su comida era langostas y miel silvestre.
\par 5 Y salía a él Jerusalén, y toda Judea, y toda la provincia de alrededor del Jordán,
\par 6 y eran bautizados por él en el Jordán, confesando sus pecados.
\par 7 Al ver él que muchos de los fariseos y de los saduceos venían a su bautismo, les decía: ¡Generación de víboras! ¿Quién os enseñó a huir de la ira venidera?
\par 8 Haced, pues, frutos dignos de arrepentimiento,
\par 9 y no penséis decir dentro de vosotros mismos: A Abraham tenemos por padre; porque yo os digo que Dios puede levantar hijos a Abraham aun de estas piedras.
\par 10 Y ya también el hacha está puesta a la raíz de los árboles; por tanto, todo árbol que no da buen fruto es cortado y echado en el fuego.
\par 11 Yo a la verdad os bautizo en agua para arrepentimiento; pero el que viene tras mí, cuyo calzado yo no soy digno de llevar, es más poderoso que yo; él os bautizará en Espíritu Santo y fuego.
\par 12 Su aventador está en su mano, y limpiará su era; y recogerá su trigo en el granero, y quemará la paja en fuego que nunca se apagará.

\section*{El bautismo de Jesús}

\par 13 Entonces Jesús vino de Galilea a Juan al Jordán, para ser bautizado por él.
\par 14 Mas Juan se le oponía, diciendo: Yo necesito ser bautizado por ti, ¿y tú vienes a mí?
\par 15 Pero Jesús le respondió: Deja ahora, porque así conviene que cumplamos toda justicia. Entonces le dejó.
\par 16 Y Jesús, después que fue bautizado, subió luego del agua; y he aquí cielos le fueron abiertos, y vio al Espíritu de Dios que descendía como paloma, y venía sobre él.
\par 17 Y hubo una voz de los cielos, que decía: Este es mi Hijo amado, en quien tengo complacencia.

\chapter{4}

\section*{Tentación de Jesús}

\par 1 Entonces Jesús fue llevado por el Espíritu al desierto, para ser tentado por el diablo.
\par 2 Y después de haber ayunado cuarenta días y cuarenta noches, tuvo hambre.
\par 3 Y vino a él el tentador, y le dijo: Si eres Hijo de Dios, di que estas piedras se conviertan en pan.
\par 4 Él respondió y dijo: Escrito está: No sólo de pan vivirá el hombre, sino de toda palabra que sale de la boca de Dios.
\par 5 Entonces el diablo le llevó a la santa ciudad, y le puso sobre el pináculo del templo,
\par 6 y le dijo: Si eres Hijo de Dios, échate abajo; porque escrito está:
\par A sus ángeles mandará acerca de ti, y,
\par En sus manos te sostendrán,
\par Para que no tropieces con tu pie en piedra.
\par 7 Jesús le dijo: Escrito está también: No tentarás al Señor tu Dios.
\par 8 Otra vez le llevó el diablo a un monte muy alto, y le mostró todos los reinos del mundo y la gloria de ellos,
\par 9 y le dijo: Todo esto te daré, si postrado me adorares.
\par 10 Entonces Jesús le dijo: Vete, Satanás, porque escrito está: Al Señor tu Dios adorarás, y a él sólo servirás.
\par 11 El diablo entonces le dejó; y he aquí vinieron ángeles y le servían.

\section*{Jesús principia su ministerio}

\par 12 Cuando Jesús oyó que Juan estaba preso, volvió a Galilea;
\par 13 y dejando a Nazaret, vino y habitó en Capernaum, ciudad marítima, en la región de Zabulón y de Neftalí,
\par 14 para que se cumpliese lo dicho por el profeta Isaías, cuando dijo:
\par 15 Tierra de Zabulón y tierra de Neftalí,
\par Camino del mar, al otro lado del Jordán,
\par Galilea de los gentiles;
\par 16 El pueblo asentado en tinieblas vio gran luz;
\par Y a los asentados en región de sombra de muerte,
\par Luz les resplandeció.
\par 17 Desde entonces comenzó Jesús a predicar, y a decir: Arrepentíos, porque el reino de los cielos se ha acercado.
\par 18 Andando Jesús junto al mar de Galilea, vio a dos hermanos, Simón, llamado Pedro, y Andrés su hermano, que echaban la red en el mar; porque eran pescadores.
\par 19 Y les dijo: Venid en pos de mí, y os haré pescadores de hombres.
\par 20 Ellos entonces, dejando al instante las redes, le siguieron.
\par 21 Pasando de allí, vio a otros dos hermanos, Jacobo hijo de Zebedeo, y Juan su hermano, en la barca con Zebedeo su padre, que remendaban sus redes; y los llamó.
\par 22 Y ellos, dejando al instante la barca y a su padre, le siguieron.
\par 23 Y recorrió Jesús toda Galilea, enseñando en las sinagogas de ellos, y predicando el evangelio del reino, y sanando toda enfermedad y toda dolencia en el pueblo.
\par 24 Y se difundió su fama por toda Siria; y le trajeron todos los que tenían dolencias, los afligidos por diversas enfermedades y tormentos, los endemoniados, lunáticos y paralíticos; y los sanó.
\par 25 Y le siguió mucha gente de Galilea, de Decápolis, de Jerusalén, de Judea y del otro lado del Jordán.

\chapter{5}

\section*{El Sermón del monte: Las bienaventuranzas}

\par 1 Viendo la multitud, subió al monte; y sentándose, vinieron a él sus discípulos.
\par 2 Y abriendo su boca les enseñaba, diciendo:
\par 3 Bienaventurados los pobres en espíritu, porque de ellos es el reino de los cielos.
\par 4 Bienaventurados los que lloran, porque ellos recibirán consolación.
\par 5 Bienaventurados los mansos, porque ellos recibirán la tierra por heredad.
\par 6 Bienaventurados los que tienen hambre y sed de justicia, porque ellos serán saciados.
\par 7 Bienaventurados los misericordiosos, porque ellos alcanzarán misericordia.
\par 8 Bienaventurados los de limpio corazón, porque ellos verán a Dios.
\par 9 Bienaventurados los pacificadores, porque ellos serán llamados hijos de Dios.
\par 10 Bienaventurados los que padecen persecución por causa de la justicia, porque de ellos es el reino de los cielos.
\par 11 Bienaventurados sois cuando por mi causa os vituperen y os persigan, y digan toda clase de mal contra vosotros, mintiendo.
\par 12 Gozaos y alegraos, porque vuestro galardón es grande en los cielos; porque así persiguieron a los profetas que fueron antes de vosotros.

\section*{La sal de la tierra}

\par 13 Vosotros sois la sal de la tierra; pero si la sal se desvaneciere, ¿con qué será salada? No sirve más para nada, sino para ser echada fuera y hollada por los hombres.

\section*{La luz del mundo}

\par 14 Vosotros sois la luz del mundo; una ciudad asentada sobre un monte no se puede esconder.
\par 15 Ni se enciende una luz y se pone debajo de un almud, sino sobre el candelero, y alumbra a todos los que están en casa.
\par 16 Así alumbre vuestra luz delante de los hombres, para que vean vuestras buenas obras, y glorifiquen a vuestro Padre que están los cielos.

\section*{Jesús y la ley}

\par 17 No penséis que he venido para abrogar la ley o los profetas; no he venido para abrogar, sino para cumplir.
\par 18 Porque de cierto os digo que hasta que pasen el cielo y la tierra, ni una jota ni una tilde pasará
\par de la ley, hasta que todo se haya cumplido.
\par 19 De manera que cualquiera que quebrante uno de estos mandamientos muy pequeños, y así enseñe a los hombres, muy pequeño será llamado en el reino de los cielos; mas cualquiera que los haga y los enseñe, éste será llamado grande en el reino de los cielos.
\par 20 Porque os digo que si vuestra justicia no fuere mayor que la de los escribas y fariseos, no entraréis en el reino de los cielos.

\section*{Jesús y la ira}

\par 21 Oísteis que fue dicho a los antiguos: No matarás; y cualquiera que matare será culpable de juicio.
\par 22 Pero yo os digo que cualquiera que se enoje contra su hermano, será culpable de juicio; y cualquiera que diga: Necio, a su hermano, será culpable ante el concilio; y cualquiera que le diga: Fatuo, quedará expuesto al infierno de fuego.
\par 23 Por tanto, si traes tu ofrenda al altar, y allí te acuerdas de que tu hermano tiene algo contra ti,
\par 24 deja allí tu ofrenda delante del altar, y anda, reconcíliate primero con tu hermano, y entonces ven y presenta tu ofrenda.
\par 25 Ponte de acuerdo con tu adversario pronto, entre tanto que estás con él en el camino, no sea que el adversario te entregue al juez, y el juez al alguacil, y seas echado en la cárcel.
\par 26 De cierto te digo que no saldrás de allí, hasta que pagues el último cuadrante.

\section*{Jesús y el adulterio}

\par 27 Oísteis que fue dicho: No cometerás adulterio.
\par 28 Pero yo os digo que cualquiera que mira a una mujer para codiciarla, ya adulteró con ella en su corazón.
\par 29 Por tanto, si tu ojo derecho te es ocasión de caer, sácalo, y échalo de ti; pues mejor te es que se pierda uno de tus miembros, y no que todo tu cuerpo sea echado al infierno.
\par 30 Y si tu mano derecha te es ocasión de caer, córtala, y échala de ti; pues mejor te es que se pierda uno de tus miembros, y no que todo tu cuerpo sea echado al infierno.

\section*{Jesús y el divorcio}

\par 31 También fue dicho: Cualquiera que repudie a su mujer, dele carta de divorcio.
\par 32 Pero yo os digo que el que repudia a su mujer, a no ser por causa de fornicación, hace que ella adultere; y el que se casa con la repudiada, comete adulterio.

\section*{Jesús y los juramentos}

\par 33 Además habéis oído que fue dicho a los antiguos: No perjurarás, sino cumplirás al Señor tus juramentos.
\par 34 Pero yo os digo: No juréis en ninguna manera; ni por el cielo, porque es el trono de Dios;
\par 35 ni por la tierra, porque es el estrado de sus pies; ni por Jerusalén, porque es la ciudad del gran Rey.
\par 36 Ni por tu cabeza jurarás, porque no puedes hacer blanco o negro un solo cabello.
\par 37 Pero sea vuestro hablar: Sí, sí; no, no; porque lo que es más de esto, de mal procede.

\section*{El amor hacia los enemigos}

\par 38 Oísteis que fue dicho: Ojo por ojo, y diente por diente.
\par 39 Pero yo os digo: No resistáis al que es malo; antes, a cualquiera que te hiera en la mejilla derecha, vuélvele también la otra;
\par 40 y al que quiera ponerte a pleito y quitarte la túnica, déjale también la capa;
\par 41 y a cualquiera que te obligue a llevar carga por una milla, vecon él dos.
\par 42 Al que te pida, dale; y al que quiera tomar de ti prestado, no se lo rehúses.
\par 43 Oísteis que fue dicho: Amarás a tu prójimo, y aborrecerás a tu enemigo.
\par 44 Pero yo os digo: Amad a vuestros enemigos, bendecid a los que os maldicen, haced bien a los que os aborrecen, y orad por los que os ultrajan y os persiguen;
\par 45 para que seáis hijos de vuestro Padre que está en los cielos, que hace salir su sol sobre malos y buenos, y que hace llover sobre justos e injustos.
\par 46 Porque si amáis a los que os aman, ¿qué recompensa tendréis? ¿No hacen también lo mismo los publicanos?
\par 47 Y si saludáis a vuestros hermanos solamente, ¿qué hacéis de más? ¿No hacen también así los gentiles?
\par 48 Sed, pues, vosotros perfectos, como vuestro Padre que está en los cielos es perfecto.

\chapter{6}

\section*{Jesús y la limosna}

\par 1 Guardaos de hacer vuestra justicia delante de los hombres, para ser vistos de ellos; de otra manera no tendréis recompensa de vuestro Padre que está en los cielos.
\par 2 Cuando, pues, des limosna, no hagas tocar trompeta delante de ti, como hacen los hipócritas en las sinagogas y en las calles, para ser alabados por los hombres; de cierto os digo que ya tienen su recompensa.
\par 3 Mas cuando tú des limosna, no sepa tu izquierda lo que hace tu derecha,
\par 4 para que sea tu limosna en secreto; y tu Padre que ve en lo secreto te recompensará en público.

\section*{Jesús y la oración}

\par 5 Y cuando ores, no seas como los hipócritas; porque ellos aman el orar en pie en las sinagogas y en las esquinas de las calles, para ser vistos de los hombres; de cierto os digo que ya tienen su recompensa.
\par 6 Mas tú, cuando ores, entra en tu aposento, y cerrada la puerta, ora a tu Padre que está en secreto; y tu Padre que ve en lo secreto te recompensará en público.
\par 7 Y orando, no uséis vanas repeticiones, como los gentiles, que piensan que por su palabrería serán oídos.
\par 8 No os hagáis, pues, semejantes a ellos; porque vuestro Padre sabe de qué cosas tenéis necesidad, antes que vosotros le pidáis.
\par 9 Vosotros, pues, oraréis así: Padre nuestro que estás en los cielos, santificado sea tu nombre.
\par 10 Venga tu reino. Hágase tu voluntad, como en el cielo, así también en la tierra.
\par 11 El pan nuestro de cada día, dánoslo hoy.
\par 12 Y perdónanos nuestras deudas, como también nosotros perdonamos a nuestros deudores.
\par 13 Y no nos metas en tentación, mas líbranos del mal; porque tuyo es el reino, y el poder, y la gloria, por todos los siglos. Amén.
\par 14 Porque si perdonáis a los hombres sus ofensas, os perdonará también a vosotros vuestro Padre celestial;
\par 15 mas si no perdonáis a los hombres sus ofensas, tampoco vuestro Padre os perdonará vuestras ofensas.

\section*{Jesús y el ayuno}

\par 16 Cuando ayunéis, no seáis austeros, como los hipócritas; porque ellos demudan sus rostros para mostrar a los hombres que ayunan; de cierto os digo que ya tienen su recompensa.
\par 17 Pero tú, cuando ayunes, unge tu cabeza y lava tu rostro,
\par 18 para no mostrar a los hombres que ayunas, sino a tu Padre que está en secreto; y tu Padre que ve en lo secreto te recompensará en público.

\section*{Tesoros en el cielo}

\par 19 No os hagáis tesoros en la tierra, donde la polilla y el orín corrompen, y donde ladrones minan y hurtan;
\par 20 sino haceos tesoros en el cielo, donde ni la polilla ni el orín corrompen, y donde ladrones no minan ni hurtan.
\par 21 Porque donde esté vuestro tesoro, allí estará también vuestro corazón.

\section*{La lámpara del cuerpo}

\par 22 La lámpara del cuerpo es el ojo; así que, si tu ojo es bueno, todo tu cuerpo estará lleno de luz;
\par 23 pero si tu ojo es maligno, todo tu cuerpo estaráen tinieblas. Así que, si la luz que en ti hay es tinieblas, ¿cuántas no serán las mismas tinieblas?

\section*{Dios y las riquezas}

\par 24 Ninguno puede servir a dos señores; porque o aborrecerá al uno y amará al otro, o estimará al uno y menospreciará al otro. No podéis servir a Dios y a las riquezas.

\section*{El afán y la ansiedad}

\par 25 Por tanto os digo: No os afanéis por vuestra vida, qué habéis de comer o qué habéis de beber; ni por vuestro cuerpo, qué habéis de vestir. ¿No es la vida más que el alimento, y el cuerpo más que el vestido?
\par 26 Mirad las aves del cielo, que no siembran, ni siegan, ni recogen en graneros; y vuestro Padre celestial las alimenta. ¿No valéis vosotros mucho más que ellas?
\par 27 ¿Y quién de vosotros podrá, por mucho que se afane, añadir a su estatura un codo?
\par 28 Y por el vestido, ¿por qué os afanáis? Considerad los lirios del campo, cómo crecen: no trabajan ni hilan;
\par 29 pero os digo, que ni aun Salomón con toda su gloria se vistió así como uno de ellos.
\par 30 Y si la hierba del campo que hoy es, y mañana se echa en el horno, Dios la viste así, ¿no hará mucho más a vosotros, hombres de poca fe?
\par 31 No os afanéis, pues, diciendo: ¿Qué comeremos, o qué beberemos, o qué vestiremos?
\par 32 Porque los gentiles buscan todas estas cosas; pero vuestro Padre celestial sabe que tenéis necesidad de todas estas cosas.
\par 33 Mas buscad primeramente el reino de Dios y su justicia, y todas estas cosas os serán añadidas.
\par 34 Así que, no os afanéis por el día de mañana, porque el día de mañana traerá su afán. Basta a cada día su propio mal.

\chapter{7}

\section*{El juzgar a los demás}

\par 1 No juzguéis, para que no seáis juzgados.
\par 2 Porque con el juicio con que juzgáis, seréis juzgados, y con la medida con que medís, os será medido.
\par 3 ¿Y por qué miras la paja que está en el ojo de tu hermano, y no echas de ver la viga que está en tu propio ojo?
\par 4 ¿O cómo dirás a tu hermano: Déjame sacar la paja de tu ojo, y he aquí la viga en el ojo tuyo?
\par 5 ¡Hipócrita! saca primero la viga de tu propio ojo, y entonces verás bien para sacar la paja del ojo de tu hermano.
\par 6 No deis lo santo a los perros, ni echéis vuestras perlas delante de los cerdos, no sea que las pisoteen, y se vuelvan y os despedacen.

\section*{La oración, y la regla de oro}

\par 7 Pedid, y se os dará; buscad, y hallaréis; llamad, y se os abrirá.
\par 8 Porque todo aquel que pide, recibe; y el que busca, halla; y al que llama, se le abrirá.
\par 9 ¿Qué hombre hay de vosotros, que si su hijo le pide pan, le dará una piedra?
\par 10 ¿O si le pide un pescado, le dará una serpiente?
\par 11 Pues si vosotros, siendo malos, sabéis dar buenas dádivas a vuestros hijos, ¿cuánto más vuestro Padre que está en los cielos dará buenas cosas a los que le pidan?
\par 12 Así que, todas las cosas que queráis que los hombres hagan con vosotros, así también haced vosotros con ellos; porque esto es la ley y los profetas.

\section*{La puerta estrecha}

\par 13 Entrad por la puerta estrecha; porque ancha es la puerta, y espacioso el camino que lleva a la perdición, y muchos son los que entran por ella;
\par 14 porque estrecha es la puerta, y angosto el camino que lleva a la vida, y pocos son los que la hallan.

\section*{Por sus frutos los conoceréis}

\par 15 Guardaos de los falsos profetas, que vienen a vosotros con vestidos de ovejas, pero por dentro son lobos rapaces.
\par 16 Por sus frutos los conoceréis. ¿Acaso se recogen uvas de los espinos, o higos de los abrojos?
\par 17 Así, todo buen árbol da buenos frutos, pero el árbol malo da frutos malos.
\par 18 No puede el buen árbol dar malos frutos, ni el árbol malo dar frutos buenos.
\par 19 Todo árbol que no da buen fruto, es cortado y echado en el fuego.
\par 20 Así que, por sus frutos los conoceréis.

\section*{Nunca os conocí}

\par 21 No todo el que me dice: Señor, Señor, entrará en el reino de los cielos, sino el que hace la voluntad de mi Padre que está en los cielos.
\par 22 Muchos me dirán en aquel día: Señor, Señor, ¿no profetizamos en tu nombre, y en tu nombre echamos fuera demonios, y en tu nombre hicimos muchos milagros?
\par 23 Y entonces les declararé: Nunca os conocí; apartaos de mí, hacedores de maldad.

\section*{Los dos cimientos}

\par 24 Cualquiera, pues, que me oye estas palabras, y las hace, le compararé a un hombre prudente, que edificó su casa sobre la roca.
\par 25 Descendió lluvia, y vinieron ríos, y soplaron vientos, y golpearon contra aquella casa; y no cayó, porque estaba fundada sobre la roca.
\par 26 Pero cualquiera que me oye estas palabras y no las hace, le compararé a un hombre insensato, que edificó su casa sobre la arena;
\par 27 y descendió lluvia, y vinieron ríos, y soplaron vientos, y dieron con ímpetu contra aquella casa; y cayó, y fue grande su ruina.
\par 28 Y cuando terminó Jesús estas palabras, la gente se admiraba de su doctrina;
\par 29 porque les enseñaba como quien tiene autoridad, y no como los escribas.

\chapter{8}

\section*{Jesús sana a un leproso}

\par 1 Cuando descendió Jesús del monte, le seguía mucha gente.
\par 2 Y he aquí vino un leproso y se postró ante él, diciendo: Señor, si quieres, puedes limpiarme.
\par 3 Jesús extendió la mano y le tocó, diciendo: Quiero; sé limpio. Y al instante su lepra desapareció.
\par 4 Entonces Jesús le dijo: Mira, no lo digas a nadie; sino ve, muéstrate al sacerdote, y presenta la ofrenda que ordenó Moisés, para testimonio a ellos.

\section*{Jesús sana al siervo de un centurión}

\par 5 Entrando Jesús en Capernaum, vino a él un centurión, rogándole,
\par 6 y diciendo: Señor, mi criado está postrado en casa, paralítico, gravemente atormentado.
\par 7 Y Jesús le dijo: Yo iré y le sanaré.
\par 8 Respondió el centurión y dijo: Señor, no soy digno de que entres bajo mi techo; solamente dí la palabra, y mi criado sanará.
\par 9 Porque también yo soy hombre bajo autoridad, y tengo bajo mis órdenes soldados; y digo a éste: Ve, y va; y al otro: Ven, y viene; y a mi siervo: Haz esto, y lo hace.
\par 10 Al oírlo Jesús, se maravilló, y dijo a los que le seguían: De cierto os digo, que ni aun en Israel he hallado tanta fe.
\par 11 Y os digo que vendrán muchos del oriente y del occidente, y se sentarán con Abraham e Isaac y Jacob en el reino de los cielos;
\par 12 mas los hijos del reino serán echados a las tinieblas de afuera; allí será el lloro y el crujir de dientes.
\par 13 Entonces Jesús dijo al centurión: Ve, y como creíste, te sea hecho. Y su criado fue sanado en aquella misma hora.

\section*{Jesús sana a la suegra de Pedro}

\par 14 Vino Jesús a casa de Pedro, y vio a la suegra de éste postrada en cama, con fiebre.
\par 15 Y tocó su mano, y la fiebre la dejó; y ella se levantó, y les servía.
\par 16 Y cuando llegó la noche, trajeron a él muchos endemoniados; y con la palabra echó fuera a los demonios, y sanó a todos los enfermos;
\par 17 para que se cumpliese lo dicho por el profeta Isaías, cuando dijo: El mismo tomó nuestras enfermedades, y llevó nuestras dolencias.

\section*{Los que querían seguir a Jesús}

\par 18 Viéndose Jesús rodeado de mucha gente, mandó pasar al otro lado.
\par 19 Y vino un escriba y le dijo: Maestro, te seguiré adondequiera que vayas.
\par 20 Jesús le dijo: Las zorras tienen guaridas, y las aves del cielo nidos; mas el Hijo del Hombre no tiene dónde recostar su cabeza.
\par 21 Otro de sus discípulos le dijo: Señor, permíteme que vaya primero y entierre a mi padre.
\par 22 Jesús le dijo: Sígueme; deja que los muertos entierren a sus muertos.

\section*{Jesús calma la tempestad}

\par 23 Y entrando él en la barca, sus discípulos le siguieron.
\par 24 Y he aquí que se levantó en el mar una tempestad tan grande que las olas cubrían la barca; pero él dormía.
\par 25 Y vinieron sus discípulos y le despertaron, diciendo: ¡Señor, sálvanos, que perecemos!
\par 26 El les dijo: ¿Por qué teméis, hombres de poca fe? Entonces, levantándose, reprendió a los vientos y al mar; y se hizo grande bonanza.
\par 27 Y los hombres se maravillaron, diciendo: ¿Qué hombre es éste, que aun los vientos y el mar le obedecen?

\section*{Los endemoniados gadarenos}

\par 28 Cuando llegó a la otra orilla, a la tierra de los gadarenos, vinieron a su encuentro dos endemoniados que salían de los sepulcros, feroces en gran manera, tanto que nadie podía pasar por aquel camino.
\par 29 Y clamaron diciendo: ¿Qué tienes con nosotros, Jesús, Hijo de Dios? ¿Has venido acá para atormentarnos antes de tiempo?
\par 30 Estaba paciendo lejos de ellos un hato de muchos cerdos.
\par 31 Y los demonios le rogaron diciendo: Si nos echas fuera, permítenos ir a aquel hato de cerdos.
\par 32 El les dijo: Id. Y ellos salieron, y se fueron a aquel hato de cerdos; y he aquí, todo el hato de cerdos se precipitó en el mar por un despeñadero, y perecieron en las aguas.
\par 33 Y los que los apacentaban huyeron, y viniendo a la ciudad, contaron todas las cosas, y lo que había pasado con los endemoniados.
\par 34 Y toda la ciudad salió al encuentro de Jesús; y cuando le vieron, le rogaron que se fuera de sus contornos.

\chapter{9}

\section*{Jesús sana a un paralítico}

\par 1 Entonces, entrando Jesús en la barca, pasó al otro lado y vino a su ciudad.
\par 2 Y sucedió que le trajeron un paralítico, tendido sobre una cama; y al ver Jesús la fe de ellos, dijo al paralítico: Ten ánimo, hijo; tus pecados te son perdonados.
\par 3 Entonces algunos de los escribas decían dentro de sí: Este blasfema.
\par 4 Y conociendo Jesús los pensamientos de ellos, dijo: ¿Por qué pensáis mal en vuestros corazones?
\par 5 Porque, ¿qué es más fácil, decir: Los pecados te son perdonados, o decir: Levántate y anda?
\par 6 Pues para que sepáis que el Hijo del Hombre tiene potestad en la tierra para perdonar pecados (dice entonces al paralítico): Levántate, toma tu cama, y vete a tu casa.
\par 7 Entonces él se levantó y se fue a su casa.
\par 8 Y la gente, al verlo, se maravilló y glorificó a Dios, que había dado tal potestad a los hombres.

\section*{Llamamiento de Mateo}

\par 9 Pasando Jesús de allí, vio a un hombre llamado Mateo, que estaba sentado al banco de los tributos públicos, y le dijo: Sígueme. Y se levantó y le siguió.
\par 10 Y aconteció que estando él sentado a la mesa en la casa, he aquí que muchos publicanos y pecadores, que habían venido, se sentaron juntamente a la mesa con Jesús y sus discípulos.
\par 11 Cuando vieron esto los fariseos, dijeron a los discípulos: ¿Porqué come vuestro Maestro con los publicanos y pecadores?
\par 12 Al oír esto Jesús, les dijo: Los sanos no tienen necesidad de médico, sino los enfermos.
\par 13 Id, pues, y aprended lo que significa: Misericordia quiero, y no sacrificio. Porque no he venido a llamar a justos, sino a pecadores, al arrepentimiento.

\section*{La pregunta sobre el ayuno}

\par 14 Entonces vinieron a él los discípulos de Juan, diciendo: ¿Por qué nosotros y los fariseos ayunamos muchas veces, y tus discípulos no ayunan?
\par 15 Jesús les dijo: ¿Acaso pueden los que están de bodas tener luto entre tanto que el esposo está con ellos? Pero vendrán días cuando el esposo les será quitado, y entonces ayunarán.
\par 16 Nadie pone remiendo de paño nuevo en vestido viejo; porque tal remiendo tira del vestido, y se hace peor la rotura.
\par 17 Ni echan vino nuevo en odres viejos; de otra manera los odres se rompen, y el vino se derrama, y los odres se pierden; pero echan el vino nuevo en odres nuevos, y lo uno y lo otro se conservan juntamente.

\section*{La hija de Jairo, y la mujer que tocó el manto de Jesús}

\par 18 Mientras él les decía estas cosas, vino un hombre principal y se postró ante él, diciendo: Mi hija acaba de morir; mas ven y pon tu mano sobre ella, y vivirá.
\par 19 Y se levantó Jesús, y le siguió con sus discípulos.
\par 20 Y he aquí una mujer enferma de flujo de sangre desde hacía doce años, se le acercó por detrás y tocó el borde de su manto;
\par 21 porque decía dentro de sí: Si tocare solamente su manto, seré salva.
\par 22 Pero Jesús, volviéndose y mirándola, dijo: Ten ánimo, hija; tu fe te ha salvado. Y la mujer fue salva desde aquella hora.
\par 23 Al entrar Jesús en la casa del principal, viendo a los que tocaban flautas, y la gente que hacía alboroto,
\par 24 les dijo: Apartaos, porque la niña no está muerta, sino duerme. Y se burlaban de él.
\par 25 Pero cuando la gente había sido echada fuera, entró, y tomó de la mano a la niña, y ella se levantó.
\par 26 Y se difundió la fama de esto por toda aquella tierra.

\section*{Dos ciegos reciben la vista}

\par 27 Pasando Jesús de allí, le siguieron dos ciegos, dando voces y diciendo: ¡Ten misericordia de nosotros, Hijo de David!
\par 28 Y llegado a la casa, vinieron a él los ciegos; y Jesús les dijo: ¿Creéis que puedo hacer esto? Ellos dijeron: Sí, Señor.
\par 29 Entonces les tocó los ojos, diciendo: Conforme a vuestra fe os sea hecho.
\par 30 Y los ojos de ellos fueron abiertos. Y Jesús les encargó rigurosamente, diciendo: Mirad que nadie lo sepa.
\par 31 Pero salidos ellos, divulgaron la fama de él por toda aquella tierra.

\section*{Un mudo habla}

\par 32 Mientras salían ellos, he aquí, le trajeron un mudo, endemoniado.
\par 33 Y echado fuera el demonio, el mudo habló; y la gente se maravillaba, y decía: Nunca se ha visto cosa semejante en Israel.
\par 34 Pero los fariseos decían: Por el príncipe de los demonios echa fuera los demonios.

\section*{La mies es mucha}

\par 35 Recorría Jesús todas las ciudades y aldeas, enseñando en las sinagogas de ellos, y predicando el evangelio del reino, y sanando toda enfermedad y toda dolencia en el pueblo.
\par 36 Y al ver las multitudes, tuvo compasión de ellas; porque estaban desamparadas y dispersas como ovejas que no tienen pastor.
\par 37 Entonces dijo a sus discípulos: A la verdad la mies es mucha, mas los obreros pocos.
\par 38 Rogad, pues, al Señor de la mies, que envíe obreros a su mies.

\chapter{10}

\section*{Elección de los doce apóstoles}

\par 1 Entonces llamando a sus doce discípulos, les dio autoridad sobre los espíritus inmundos, para que los echasen fuera, y para sanar toda enfermedad y toda dolencia.
\par 2 Los nombres de los doce apóstoles son estos: primero Simón, llamado Pedro, y Andrés su hermano; Jacobo hijo de Zebedeo, y Juan su hermano;
\par 3 Felipe, Bartolomé, Tomás, Mateo el publicano, Jacobo hijo de Alfeo, Lebeo, por sobrenombre Tadeo,
\par 4 Simón el cananista, y Judas Iscariote, el que también le entregó.

\section*{Misión de los doce}

\par 5 A estos doce envió Jesús, y les dio instrucciones, diciendo: Por camino de gentiles no vayáis, y en ciudad de samaritanos no entréis,
\par 6 sino id antes a las ovejas perdidas de la casa de Israel.
\par 7 Y yendo, predicad, diciendo: El reino de los cielos se ha acercado.
\par 8 Sanad enfermos, limpiad leprosos, resucitad muertos, echad fuera demonios; de gracia recibisteis, dad de gracia.
\par 9 No os proveáis de oro, ni plata, ni cobre en vuestros cintos;
\par 10 ni de alforja para el camino, ni de dos túnicas, ni de calzado, ni de bordón; porque el obrero es digno de su alimento.
\par 11 Mas en cualquier ciudad o aldea donde entréis, informaos quién en ella sea digno, y posad allí hasta que salgáis.
\par 12 Y al entrar en la casa, saludadla.
\par 13 Y si la casa fuere digna, vuestra paz vendrá sobre ella; mas si no fuere digna, vuestra paz se volverá a vosotros.
\par 14 Y si alguno no os recibiere, ni oyere vuestras palabras, salid de aquella casa o ciudad, y sacudid el polvo de vuestros pies.
\par 15 De cierto os digo que en el día del juicio, será más tolerable el castigo para la tierra de Sodoma y de Gomorra, que para aquella ciudad.

\section*{Persecuciones venideras}

\par 16 He aquí, yo os envío como a ovejas en medio de lobos; sed, pues, prudentes como serpientes, y sencillos como palomas.
\par 17 Y guardaos de los hombres, porque os entregarán a los concilios, y en sus sinagogas os azotarán;
\par 18 y aun ante gobernadores y reyes seréis llevados por causa de mí, para testimonio a ellos y a los gentiles.
\par 19 Mas cuando os entreguen, no os preocupéis por cómo o qué hablaréis; porque en aquella hora os será dado lo que habéis de hablar.
\par 20 Porque no sois vosotros los que habláis, sino el Espíritu de vuestro Padre que habla en vosotros.
\par 21 El hermano entregará a la muerte al hermano, y el padre al hijo; y los hijos se levantarán contra los padres, y los harán morir.
\par 22 Y seréis aborrecidos de todos por causa de mi nombre; mas el que persevere hasta el fin, éste será salvo.
\par 23 Cuando os persigan en esta ciudad, huid a la otra; porque de cierto os digo, que no acabaréis de recorrer todas las ciudades de Israel, antes que venga el Hijo de Hombre.
\par 24 El discípulo no es más que su maestro, ni el siervo más que su señor.
\par 25 Bástale al discípulo ser como su maestro, y al siervo como su señor. Si al padre de familia llamaron Beelzeb, ¿cuánto más a los de su casa?

\section*{A quién se debe temer}

\par 26 Así que, no los temáis; porque nada hay encubierto, que no haya de ser manifestado; ni oculto, que no haya de saberse.
\par 27 Lo que os digo en tinieblas, decidlo en la luz; y lo que oís al oído, proclamadlo desde las azoteas.
\par 28 Y no temáis a los que matan el cuerpo, mas el alma no pueden matar; temed más bien a aquel que puede destruir el alma y el cuerpo en el infierno.
\par 29 ¿No se venden dos pajarillos por un cuarto? Con todo, ni uno de ellos cae a tierra sin vuestro Padre.
\par 30 Pues aun vuestros cabellos están todos contados.
\par 31 Así que, no temáis; más valéis vosotros que muchos pajarillos.
\par 32 A cualquiera, pues, que me confiese delante de los hombres, yo también le confesaré delante de mi Padre que está en los cielos.
\par 33 Y a cualquiera que me niegue delante de los hombres, yo también le negaré delante de mi Padre que está en los cielos.

\section*{Jesús, causa de división}

\par 34 No penséis que he venido para traer paz a la tierra; no he venido para traer paz, sino espada.
\par 35 Porque he venido para poner en disensión al hombre contra su padre, a la hija contra su madre, y a la nuera contra su suegra;
\par 36 y los enemigos del hombre serán los de su casa.
\par 37 El que ama a padre o madre más que a mí, no es digno de mí; el que ama a hijo o hija más que a mí, no es digno de mí;
\par 38 y el que no toma su cruz y sigue en pos de mí, no es digno de mí.
\par 39 El que halla su vida, la perderá; y el que pierde su vida por causa de mí, la hallará.

\section*{Recompensas}

\par 40 El que a vosotros recibe, a mí me recibe; y el que me recibe a mí, recibe al que me envió.
\par 41 El que recibe a un profeta por cuanto es profeta, recompensa de profeta recibirá; y el que recibe a un justo por cuanto es justo, recompensa de justo recibirá.
\par 42 Y cualquiera que dé a uno de estos pequeñitos un vaso de agua fría solamente, por cuanto es discípulo, de cierto os digo que no perderá su recompensa.
\chapter{11}

\section*{Los mensajeros de Juan el Bautista}

\par 1 Cuando Jesús terminó de dar instrucciones a sus doce discípulos, se fue de allí a enseñar y a predicar en las ciudades de ellos.
\par 2 Y al oír Juan, en la cárcel, los hechos de Cristo, le envió dos de sus discípulos,
\par 3 para preguntarle: ¿Eres tú aquel que había de venir, o esperaremos a otro?
\par 4 Respondiendo Jesús, les dijo: Id, y haced saber a Juan las cosas que oís y veis.
\par 5 Los ciegos ven, los cojos andan, los leprosos son limpiados, los sordos oyen, los muertos son resucitados, y a los pobres es anunciado el evangelio;
\par 6 y bienaventurado es el que no halle tropiezo en mí.
\par 7 Mientras ellos se iban, comenzó Jesús a decir de Juan a la gente: ¿Qué salisteis a ver al desierto? ¿Una caña sacudida por el viento?
\par 8 ¿O qué salisteis a ver? ¿A un hombre cubierto de vestiduras delicadas? He aquí, los que llevan vestiduras delicadas, en las casas de los reyes están.
\par 9 Pero ¿qué salisteis a ver? ¿A un profeta? Sí, os digo, y más que profeta.
\par 10 Porque éste es de quien está escrito:
\par He aquí, yo envío mi mensajero delante de tu faz,
\par El cual preparará tu camino delante de ti.
\par 11 De cierto os digo: Entre los que nacen de mujer no se ha levantado otro mayor que Juan el Bautista; pero el más pequeño en el reino de los cielos, mayor es que él.
\par 12 Desde los días de Juan el Bautista hasta ahora, el reino de los cielos sufre violencia, y los violentos lo arrebatan.
\par 13 Porque todos los profetas y la ley profetizaron hasta Juan.
\par 14 Y si queréis recibirlo, él es aquel Elías que había de venir.
\par 15 El que tiene oídos para oír, oiga.
\par 16 Mas ¿a qué compararé esta generación? Es semejante a los muchachos que se sientan en las plazas, y dan voces a sus compañeros,
\par 17 diciendo: Os tocamos flauta, y no bailasteis; os endechamos, y no lamentasteis.
\par 18 Porque vino Juan, que ni comía ni bebía, y dicen: Demonio tiene.
\par 19 Vino el Hijo del Hombre, que come y bebe, y dicen: He aquí un hombre comilón, y bebedor de vino, amigo de publicanos y de pecadores. Pero la sabiduría es justificada por sus hijos.

\section*{Ayes sobre las ciudades impenitentes}

\par 20 Entonces comenzó a reconvenir a las ciudades en las cuales había hecho muchos de sus milagros, porque no se habían arrepentido, diciendo:
\par 21 Ay de ti, Corazín! Ay de ti, Betsaida! Porque si en Tiro y en Sidón se hubieran hecho los milagros que han sido hechos en vosotras, tiempo ha que se hubieran arrepentido en cilicio y en ceniza.
\par 22 Por tanto os digo que en el día del juicio, será más tolerable el castigo para Tiro y para Sidón, que para vosotras.
\par 23 Y tú, Capernaum, que eres levantada hasta el cielo, hasta el Hades serás abatida; porque si en Sodoma se hubieran hecho los milagros que han sido hechos en ti, habría permanecido hasta el día de hoy.
\par 24 Por tanto os digo que en el día del juicio, será más tolerable el castigo para la tierra de Sodoma, que para ti.

\section*{Venid a mí y descansad}

\par 25 En aquel tiempo, respondiendo Jesús, dijo: Te alabo, Padre, Señor del cielo y de la tierra, porque escondiste estas cosas de los sabios y de los entendidos, y las revelaste a los niños.
\par 26 Sí, Padre, porque así te agradó.
\par 27 Todas las cosas me fueron entregadas por mi Padre; y nadie conoce al Hijo, sino el Padre, ni al Padre conoce alguno, sino el Hijo, y aquel a quien el Hijo lo quiera revelar.
\par 28 Venid a mí todos los que estáis trabajados y cargados, y yo os haré descansar.
\par 29 Llevad mi yugo sobre vosotros, y aprended de mí, que soy manso y humilde de corazón; y hallaréis descanso para vuestras almas;
\par 30 porque mi yugo es fácil, y ligera mi carga.

\chapter{12}

\section*{Los discípulos recogen espigas en el día de reposo}

\par 1 En aquel tiempo iba Jesús por los sembrados en un día de reposo; y sus discípulos tuvieron hambre, y comenzaron a arrancar espigas y a comer.
\par 2 Viéndolo los fariseos, le dijeron: He aquí tus discípulos hacen lo que no es lícito hacer en el día de reposo.
\par 3 Pero él les dijo: ¿No habéis leído lo que hizo David, cuando él y los que con él estaban tuvieron hambre;
\par 4 cómo entró en la casa de Dios, y comió los panes de la proposición, que no les era lícito comer ni a él ni a los que con él estaban, sino solamente a los sacerdotes?
\par 5 ¿O no habéis leído en la ley, cómo en el día de reposo los sacerdotes en el templo profanan el día de reposo, y son sin culpa?
\par 6 Pues os digo que uno mayor que el templo está aquí.
\par 7 Y si supieseis qué significa: Misericordia quiero, y no sacrificio, no condenaríais a los inocentes;
\par 8 porque el Hijo del Hombre es Señor del día de reposo.

\section*{El hombre de la mano seca}

\par 9 Pasando de allí, vino a la sinagoga de ellos.
\par 10 Y he aquí había allí uno que tenía seca una mano; y preguntaron a Jesús, para poder acusarle:
\par ¿Es lícito sanar en el día de reposo?
\par 11 El les dijo: ¿Qué hombre habrá de vosotros, que tenga una oveja, y si ésta cayere en un hoyo en día de reposo, no le eche mano, y la levante?
\par 12 Pues ¿cuánto más vale un hombre que una oveja? Por consiguiente, es lícito hacer el bien en los días de reposo.
\par 13 Entonces dijo a aquel hombre: Extiende tu mano. Y él la extendió, y le fue restaurada sana como la otra.
\par 14 Y salidos los fariseos, tuvieron consejo contra Jesús para destruirle.

\section*{El siervo escogido}

\par 15 Sabiendo esto Jesús, se apartó de allí; y le siguió mucha gente, y sanaba a todos,
\par 16 y les encargaba rigurosamente que no le descubriesen;
\par 17 para que se cumpliese lo dicho por el profeta Isaías, cuando dijo:
\par 18 He aquí mi siervo, a quien he escogido;
\par Mi Amado, en quien se agrada mi alma;
\par Pondré mi Espíritu sobre él,
\par Y a los gentiles anunciará juicio.
\par 19 No contenderá, ni voceará,
\par Ni nadie oirá en las calles su voz.
\par 20 La caña cascada no quebrará,
\par Y el pábilo que humea no apagará,
\par Hasta que saque a victoria el juicio.
\par 21 Y en su nombre esperarán los gentiles.

\section*{La blasfemia contra el Espíritu Santo}

\par 22 Entonces fue traído a él un endemoniado, ciego y mudo; y le sanó, de tal manera que el ciego y mudo veía y hablaba.
\par 23 Y toda la gente estaba atónita, y decía: ¿Será éste aquel Hijo de David?
\par 24 Mas los fariseos, al oírlo, decían: Este no echa fuera los demonios sino por Beelzebú, príncipe de los demonios.
\par 25 Sabiendo Jesús los pensamientos de ellos, les dijo: Todo reino dividido contra sí mismo, es asolado, y toda ciudad o casa dividida contra sí misma, no permanecerá.
\par 26 Y si Satanás echa fuera a Satanás, contra sí mismo está dividido; ¿cómo, pues, permanecerá su reino?
\par 27 Y si yo echo fuera los demonios por Beelzebú, ¿por quién los echan vuestros hijos? Por tanto, ellos serán vuestros jueces.
\par 28 Pero si yo por el Espíritu de Dios echo fuera los demonios, ciertamente ha llegado a vosotros el reino de Dios.
\par 29 Porque ¿cómo puede alguno entrar en la casa del hombre fuerte, y saquear sus bienes, si primero no le ata? Y entonces podrá saquear su casa.
\par 30 El que no es conmigo, contra mí es; y el que conmigo no recoge, desparrama.
\par 31 Por tanto os digo: Todo pecado y blasfemia será perdonado a los hombres; mas la blasfemia contra el Espíritu no les será perdonada.
\par 32 A cualquiera que dijere alguna palabra contra el Hijo del Hombre, le será perdonado; pero al que hable contra el Espíritu Santo, no le será perdonado, ni en este siglo ni en el venidero.
\par 33 O haced el árbol bueno, y su fruto bueno, o haced el árbol malo, y su fruto malo; porque por el fruto se conoce el árbol.
\par 34 ¡Generación de víboras! ¿Cómo podéis hablar lo bueno, siendo malos? Porque de la abundancia del corazón habla la boca.
\par 35 El hombre bueno, del buen tesoro del corazón saca buenas cosas; y el hombre malo, del mal tesoro saca malas cosas.
\par 36 Mas yo os digo que de toda palabra ociosa que hablen los hombres, de ella darán cuenta en el día del juicio.
\par 37 Porque por tus palabras serás justificado, y por tus palabras serás condenado.

\section*{La generación perversa demanda señal}

\par 38 Entonces respondieron algunos de los escribas y de los fariseos, diciendo: Maestro, deseamos ver de ti señal.
\par 39 El respondió y les dijo: La generación mala y adúltera demanda señal; pero señal no le será dada, sino la señal del profeta Jonás.
\par 40 Porque como estuvo Jonás en el vientre del gran pez tres días y tres noches, así estará el Hijo del Hombre en el corazón de la tierra tres días y tres noches.
\par 41 Los hombres de Nínive se levantarán en el juicio con esta generación, y la condenarán; porque ellos se arrepintieron a la predicación de Jonás, y he aquí más que Jonás en este lugar.
\par 42 La reina del Sur se levantará en el juicio con esta generación, y la condenará; porque ella vino de los fines de la tierra para oír la sabiduría de Salomón, y he aquí más que Salomón en este lugar.

\section*{El espíritu inmundo que vuelve}

\par 43 Cuando el espíritu inmundo sale del hombre, anda por lugares secos, buscando reposo, y no lo halla.
\par 44 Entonces dice: Volveré a mi casa de donde salí; y cuando llega, la halla desocupada, barrida y adornada.
\par 45 Entonces va, y toma consigo otros siete espíritus peores que él, y entrados, moran allí; y el postrer estado de aquel hombre viene a ser peor que el primero. Así también acontecerá a esta mala generación.

\section*{La madre y los hermanos de Jesús}

\par 46 Mientras él aún hablaba a la gente, he aquí su madre y sus hermanos estaban afuera, y le querían hablar.
\par 47 Y le dijo uno: He aquí tu madre y tus hermanos están afuera, y te quieren hablar.
\par 48 Respondiendo él al que le decía esto, dijo: ¿Quién es mi madre, y quiénes son mis hermanos?
\par 49 Y extendiendo su mano hacia sus discípulos, dijo: He aquí mi madre y mis hermanos.
\par 50 Porque todo aquel que hace la voluntad de mi Padre que los cielos, ése es mi hermano, y hermana, y madre.

\chapter{13}

\section*{Parábola del sembrador}

\par 1 Aquel día salió Jesús de la casa y se sentó unto al mar.
\par 2 Y se le juntó mucha gente; y entrando él en la barca, se sentó, y toda la gente estaba en la playa.
\par 3 Y les habló muchas cosas por parábolas, diciendo: He aquí, el sembrador salió a sembrar.
\par 4 Y mientras sembraba, parte de la semilla cayó junto al camino; y vinieron las aves y la comieron.
\par 5 Parte cayó en pedregales, donde no había mucha tierra; y brotó pronto, porque no tenía profundidad de tierra;
\par 6 pero salido el sol, se quemó; y porque no tenía raíz, se secó.
\par 7 Y parte cayó entre espinos; y los espinos crecieron, y la ahogaron.
\par 8 Pero parte cayó en buena tierra, y dio fruto, cuál a ciento, cuál a sesenta, y cuál a treinta por uno.
\par 9 El que tiene oídos para oír, oiga.

\section*{Propósito de las parábolas}

\par 10 Entonces, acercándose los discípulos, le dijeron: ¿Por qué les hablas por parábolas?
\par 11 El respondiendo, les dijo: Porque a vosotros os es dado saber los misterios del reino de los cielos; mas a ellos no les es dado.
\par 12 Porque a cualquiera que tiene, se le dará, y tendrá más; pero al que no tiene, aun lo que tiene le será quitado.
\par 13 Por eso les hablo por parábolas: porque viendo no ven, y oyendo no oyen, ni entienden.
\par 14 De manera que se cumple en ellos la profecía de Isaías, que dijo:
\par De oído oiréis, y no entenderéis;
\par Y viendo veréis, y no percibiréis.
\par 15 Porque el corazón de este pueblo se ha engrosado,
\par Y con los oídos oyen pesadamente,
\par Y han cerrado sus ojos;
\par Para que no vean con los ojos,
\par Y oigan con los oídos,
\par Y con el corazón entiendan,
\par Y se conviertan,
\par Y yo los sane.
\par 16 Pero bienaventurados vuestros ojos, porque ven; y vuestros oídos, porque oyen.
\par 17 Porque de cierto os digo, que muchos profetas y justos desearon ver lo que veis, y no lo vieron; y oír lo que oís, y no lo oyeron.

\section*{Jesús explica la parábola del sembrador}

\par 18 Oíd, pues, vosotros la parábola del sembrador:
\par 19 Cuando alguno oye la palabra del reino y no la entiende, viene el malo, y arrebata lo que fue sembrado en su corazón. Este es el que fue sembrado junto al camino.
\par 20 Y el que fue sembrado en pedregales, éste es el que oye la palabra, y al momento la recibe con gozo;
\par 21 pero no tiene raíz en sí, sino que es de corta duración, pues al venir la aflicción o la persecución por causa de la palabra, luego tropieza.
\par 22 El que fue sembrado entre espinos, éste es el que oye la palabra, pero el afán de este siglo y el engaño de las riquezas ahogan la palabra, y se hace infructuosa.
\par 23 Mas el que fue sembrado en buena tierra, éste es el que oye y entiende la palabra, y da fruto; y produce a ciento, a sesenta, y a treinta por uno.

\section*{Parábola del trigo y la cizaña}

\par 24 Les refirió otra parábola, diciendo: El reino de los cielos es semejante a un hombre que sembró buena semilla en su campo;
\par 25 pero mientras dormían los hombres, vino su enemigo y sembró cizaña entre el trigo, y se fue.
\par 26 Y cuando salió la hierba y dio fruto, entonces apareció también la cizaña.
\par 27 Vinieron entonces los siervos del padre de familia y le dijeron: Señor, ¿no sembraste buena semilla en tu campo? ¿De dónde, pues, tiene cizaña?
\par 28 El les dijo: Un enemigo ha hecho esto. Y los siervos le dijeron: ¿Quieres, pues, que vayamos y la arranquemos?
\par 29 El les dijo: No, no sea que al arrancar la cizaña, arranquéis también con ella el trigo.
\par 30 Dejad crecer juntamente lo uno y lo otro hasta la siega; y al tiempo de la siega yo diré a los segadores: Recoged primero la cizaña, y atadla en manojos para quemarla; pero recoged el trigo en mi granero.

\section*{Parábola de la semilla de mostaza}

\par 31 Otra parábola les refirió, diciendo: El reino de los cielos es semejante al grano de mostaza, que un hombre tomó y sembró en su campo;
\par 32 el cual a la verdad es la más pequeña de todas las semillas; pero cuando ha crecido, es la mayor de las hortalizas, y se hace árbol, de tal manera que vienen las aves del cielo y hacen nidos en sus ramas.

\section*{Parábola de la levadura}

\par 33 Otra parábola les dijo: El reino de los cielos es semejante a la levadura que tomó una mujer, y escondió en tres medidas de harina, hasta que todo fue leudado.

\section*{El uso que Jesús hace de las parábolas}

\par 34 Todo esto habló Jesús por parábolas a la gente, y sin parábolas no les hablaba;
\par 35 para que se cumpliese lo dicho por el profeta, cuando dijo:
\par Abriré en parábolas mi boca;
\par Declararé cosas escondidas desde la fundación del mundo.

\section*{Jesús explica la parábola de la cizaña}

\par 36 Entonces, despedida la gente, entró Jesús en la casa; y acercándose a él sus discípulos, le dijeron: Explícanos la parábola de la cizaña del campo.
\par 37 Respondiendo él, les dijo: El que siembra la buena semilla es el Hijo del Hombre.
\par 38 El campo es el mundo; la buena semilla son los hijos del reino, y la cizaña son los hijos del malo.
\par 39 El enemigo que la sembróes el diablo; la siega es el fin del siglo; y los segadores son los ángeles.
\par 40 De manera que como se arranca la cizaña, y se quema en el fuego, asíseráen el fin de este siglo.
\par 41 Enviará el Hijo del Hombre a sus ángeles, y recogerán de su reino a todos los que sirven de tropiezo, y a los que hacen iniquidad,
\par 42 y los echarán en el horno de fuego; allí será el lloro y el crujir de dientes.
\par 43 Entonces los justos resplandecerán como el sol en el reino de su Padre. El que tiene oídos para oír, oiga.

\section*{El tesoro escondido}

\par 44 Además, el reino de los cielos es semejante a un tesoro escondido en un campo, el cual un hombre halla, y lo esconde de nuevo; y gozoso por ello va y vende todo lo que tiene, y compra aquel campo.

\section*{La perla de gran precio}

\par 45 También el reino de los cielos es semejante a un mercader que busca buenas perlas,
\par 46 que habiendo hallado una perla preciosa, fue y vendió todo lo que tenía, y la compró.

\section*{La red}

\par 47 Asimismo el reino de los cielos es semejante a una red, que echada en el mar, recoge de toda clase de peces;
\par 48 y una vez llena, la sacan a la orilla; y sentados, recogen lo bueno en cestas, y lo malo echan fuera.
\par 49 Asíserá al fin del siglo: saldrán los ángeles, y apartarán a los malos de entre los justos,
\par 50 y los echarán en el horno de fuego; allí será el lloro y el crujir de dientes.

\section*{Tesoros nuevos y viejos}

\par 51 Jesús les dijo: ¿Habéis entendido todas estas cosas? Ellos respondieron: Sí, Señor.
\par 52 El les dijo: Por eso todo escriba docto en el reino de los cielos es semejante a un padre de familia, que saca de su tesoro cosas nuevas y cosas viejas.

\section*{Jesús en Nazaret}

\par 53 Aconteció que cuando terminó Jesús estas parábolas, se fue de allí.
\par 54 Y venido a su tierra, les enseñaba en la sinagoga de ellos, de tal manera que se maravillaban, y decían: ¿De dónde tiene éste esta sabiduría y estos milagros?
\par 55 ¿No es éste el hijo del carpintero? ¿No se llama su madre María, y sus hermanos, Jacobo, José, Simón y Judas?
\par 56 ¿No están todas sus hermanas con nosotros? ¿De dónde, pues, tiene éste todas estas cosas?
\par 57 Y se escandalizaban de él. Pero Jesús les dijo: No hay profeta sin honra, sino en su propia tierra y en su casa.
\par 58 Y no hizo allí muchos milagros, a causa de la incredulidad de ellos.

\chapter{14}

\section*{Muerte de Juan el Bautista}

\par 1 En aquel tiempo Herodes el tetrarca oyó la fama de Jesús,
\par 2 y dijo a sus criados: Este es Juan el Bautista; ha resucitado de los muertos, y por eso actúan en él estos poderes.
\par 3 Porque Herodes había prendido a Juan, y le había encadenado y metido en la cárcel, por causa de Herodías, mujer de Felipe su hermano;
\par 4 porque Juan le decía: No te es lícito tenerla.
\par 5 Y Herodes quería matarle, pero temía al pueblo; porque tenían a Juan por profeta.
\par 6 Pero cuando se celebraba el cumpleaños de Herodes, la hija de Herodías danzó en medio, y agradó a Herodes,
\par 7 por lo cual éste le prometió con juramento darle todo lo que pidiese.
\par 8 Ella, instruida primero por su madre, dijo: Dame aquí en un plato la cabeza de Juan el Bautista.
\par 9 Entonces el rey se entristeció; pero a causa del juramento, y de los que estaban con él a la mesa, mandó que se la diesen,
\par 10 y ordenó decapitar a Juan en la cárcel.
\par 11 Y fue traída su cabeza en un plato, y dada a la muchacha; y ella la presentó a su madre.
\par 12 Entonces llegaron sus discípulos, y tomaron el cuerpo y lo enterraron; y fueron y dieron las nuevas a Jesús.

\section*{Alimentación de los cinco mil}

\par 13 Oyéndolo Jesús, se apartó de allí en una barca a un lugar desierto y apartado; y cuando la gente lo oyó, le siguió a pie desde las ciudades.
\par 14 Y saliendo Jesús, vio una gran multitud, y tuvo compasión de ellos, y sanó a los que de ellos estaban enfermos.
\par 15 Cuando anochecía, se acercaron a él sus discípulos, diciendo: El lugar es desierto, y la hora ya pasada; despide a la multitud, para que vayan por las aldeas y compren de comer.
\par 16 Jesús les dijo: No tienen necesidad de irse; dadles vosotros de comer.
\par 17 Y ellos dijeron: No tenemos aquí sino cinco panes y dos peces.
\par 18 El les dijo: Traédmelos acá.
\par 19 Entonces mandóa la gente recostarse sobre la hierba; y tomando los cinco panes y los dos peces, y levantando los ojos al cielo, bendijo, y partió y dio los panes a los discípulos, y los discípulos a la multitud.
\par 20 Y comieron todos, y se saciaron; y recogieron lo que sobró de los pedazos, doce cestas llenas.
\par 21 Y los que comieron fueron como cinco mil hombres, sin contar las mujeres y los niños.

\section*{Jesús anda sobre el mar}

\par 22 En seguida Jesús hizo a sus discípulos entrar en la barca e ir delante de él a la otra ribera, entre tanto que él despedía a la multitud.
\par 23 Despedida la multitud, subió al monte a orar aparte; y cuando llegó la noche, estaba allí solo.
\par 24 Y ya la barca estaba en medio del mar, azotada por las olas; porque el viento era contrario.
\par 25 Mas a la cuarta vigilia de la noche, Jesús vino a ellos andando sobre el mar.
\par 26 Y los discípulos, viéndole andar sobre el mar, se turbaron, diciendo: ¡Un fantasma! Y dieron voces de miedo.
\par 27 Pero en seguida Jesús les habló, diciendo: ¡Tened ánimo; yo soy, no temáis!
\par 28 Entonces le respondió Pedro, y dijo: Señor, si eres tú, manda que yo vaya a ti sobre las aguas.
\par 29 Y él dijo: Ven. Y descendiendo Pedro de la barca, andaba sobre las aguas para ir a Jesús.
\par 30 Pero al ver el fuerte viento, tuvo miedo; y comenzando a hundirse, dio voces, diciendo: ¡Señor, sálvame!
\par 31 Al momento Jesús, extendiendo la mano, asió de él, y le dijo: ¡Hombre de poca fe! ¿Por qué dudaste?
\par 32 Y cuando ellos subieron en la barca, se calmó el viento.
\par 33 Entonces los que estaban en la barca vinieron y le adoraron, diciendo: Verdaderamente eres Hijo de Dios.

\section*{Jesús sana a los enfermos en Genesaret}

\par 34 Y terminada la travesía, vinieron a tierra de Genesaret.
\par 35 Cuando le conocieron los hombres de aquel lugar, enviaron noticia por toda aquella tierra alrededor, y trajeron a él todos los enfermos;
\par 36 y le rogaban que les dejase tocar solamente el borde de su manto; y todos los que lo tocaron, quedaron sanos.

\chapter{15}

\section*{Lo que contamina al hombre}

\par 1 Entonces se acercaron a Jesús ciertos escribas y fariseos de Jerusalén, diciendo:
\par 2 ¿Por qué tus discípulos quebrantan la tradición de los ancianos? Porque no se lavan las manos cuando comen pan.
\par 3 Respondiendo él, les dijo: ¿Por qué también vosotros quebrantáis el mandamiento de Dios por vuestra tradición?
\par 4 Porque Dios mandó diciendo: Honra a tu padre y a tu madre; y: El que maldiga al padre o a la madre, muera irremisiblemente.
\par 5 Pero vosotros decís: Cualquiera que diga a su padre o a su madre: Es mi ofrenda a Dios todo aquello con que pudiera ayudarte,
\par 6 ya no ha de honrar a su padre o a su madre. Así habéis invalidado el mandamiento de Dios por vuestra tradición.
\par 7 Hipócritas, bien profetizó de vosotros Isaías, cuando dijo:
\par 8 Este pueblo de labios me honra;
\par Mas su corazón está lejos de mí.
\par 9 Pues en vano me honran,

\section*{Enseñando como doctrinas, mandamientos de hombres}

\par 10 Y llamando a sí a la multitud, les dijo: Oíd, y entended:
\par 11 No lo que entra en la boca contamina al hombre; mas lo que sale de la boca, esto contamina al hombre.
\par 12 Entonces acercándose sus discípulos, le dijeron: ¿Sabes que los fariseos se ofendieron cuando oyeron esta palabra?
\par 13 Pero respondiendo él, dijo: Toda planta que no plantó mi Padre celestial, será desarraigada.
\par 14 Dejadlos; son ciegos guías de ciegos; y si el ciego guiare al ciego, ambos caerán en el hoyo.
\par 15 Respondiendo Pedro, le dijo: Explícanos esta parábola.
\par 16 Jesús dijo: ¿También vosotros sois aún sin entendimiento?
\par 17 ¿No entendéis que todo lo que entra en la boca va al vientre, y es echado en la letrina?
\par 18 Pero lo que sale de la boca, del corazón sale; y esto contamina al hombre.
\par 19 Porque del corazón salen los malos pensamientos, los homicidios, los adulterios, las fornicaciones, los hurtos, los falsos testimonios, las blasfemias.
\par 20 Estas cosas son las que contaminan al hombre; pero el comer con las manos sin lavar no contamina al hombre.

\section*{La fe de la mujer cananea}

\par 21 Saliendo Jesús de allí, se fue a la región de Tiro y de Sidón.
\par 22 Y he aquí una mujer cananea que había salido de aquella región clamaba, diciéndole: ¡Señor, Hijo de David, ten misericordia de mí! Mi hija es gravemente atormentada por un demonio.
\par 23 Pero Jesús no le respondió palabra. Entonces acercándose sus discípulos, le rogaron, diciendo: Despídela, pues da voces tras nosotros.
\par 24 El respondiendo, dijo: No soy enviado sino a las ovejas perdidas de la casa de Israel.
\par 25 Entonces ella vino y se postró ante él, diciendo: ¡Señor, socórreme!
\par 26 Respondiendo él, dijo: No está bien tomar el pan de los hijos, y echarlo a los perrillos.
\par 27 Y ella dijo: Sí, Señor; pero aun los perrillos comen de las migajas que caen de la mesa de sus amos.
\par 28 Entonces respondiendo Jesús, dijo: Oh mujer, grande es tu fe; hágase contigo como quieres. Y su hija fue sanada desde aquella hora.

\section*{Jesús sana a muchos}

\par 29 Pasó Jesús de allí y vino junto al mar de Galilea; y subiendo al monte, se sentó allí.
\par 30 Y se le acercó mucha gente que traía consigo a cojos, ciegos, mudos, mancos, y otros muchos enfermos; y los pusieron a los pies de Jesús, y los sanó;
\par 31 de manera que la multitud se maravillaba, viendo a los mudos hablar, a los mancos sanados, a los cojos andar, y a los ciegos ver; y glorificaban al Dios de Israel.

\section*{Alimentación de los cuatro mil}

\par 32 Y Jesús, llamando a sus discípulos, dijo: Tengo compasión de la gente, porque ya hace tres días que están conmigo, y no tienen qué comer; y enviarlos en ayunas no quiero, no sea que desmayen en el camino.
\par 33 Entonces sus discípulos le dijeron: ¿De dónde tenemos nosotros tantos panes en el desierto, para saciar a una multitud tan grande?
\par 34 Jesús les dijo: ¿Cuántos panes tenéis? Y ellos dijeron: Siete, y unos pocos pececillos.
\par 35 Y mandó a la multitud que se recostase en tierra.
\par 36 Y tomando los siete panes y los peces, dio gracias, los partió y dio a sus discípulos, y los discípulos a la multitud.
\par 37 Y comieron todos, y se saciaron; y recogieron lo que sobró de los pedazos, siete canastas llenas.
\par 38 Y eran los que habían comido, cuatro mil hombres, sin contar las mujeres y los niños.
\par 39 Entonces, despedida la gente, entró en la barca, y vino a la región de Magdala.

\chapter{16}

\section*{La demanda de una señal}

\par 1 Vinieron los fariseos y los saduceos para tentarle, y le pidieron que les mostrase señal del cielo.
\par 2 Mas él respondiendo, les dijo: Cuando anochece, decís: Buen tiempo; porque el cielo tiene arreboles.
\par 3 Y por la mañana: Hoy habrá tempestad; porque tiene arreboles el cielo nublado. ¡Hipócritas! que sabéis distinguir el aspecto del cielo, ¡mas las señales de los tiempos no podéis!
\par 4 La generación mala y adúltera demanda señal; pero señal no le será dada, sino la señal del profeta Jonás. Y dejándolos, se fue.

\section*{La levadura de los fariseos}

\par 5 Llegando sus discípulos al otro lado, se habían olvidado de traer pan.
\par 6 Y Jesús les dijo: Mirad, guardaos de la levadura de los fariseos y de los saduceos.
\par 7 Ellos pensaban dentro de sí, diciendo: Esto dice porque no trajimos pan.
\par 8 Y entendiéndolo Jesús, les dijo: ¿Por qué pensáis dentro de vosotros, hombres de poca fe, que no tenéis pan?
\par 9 ¿No entendéis aún, ni os acordáis de los cinco panes entre cinco mil hombres, y cuántas cestas recogisteis?
\par 10 ¿Ni de los siete panes entre cuatro mil, y cuántas canastas recogisteis?
\par 11 ¿Cómo es que no entendéis que no fue por el pan que os dije que os guardaseis de la levadura de los fariseos y de los saduceos?
\par 12 Entonces entendieron que no les había dicho que se guardasen de la levadura del pan, sino de la doctrina de los fariseos y de los saduceos.

\section*{La confesión de Pedro}

\par 13 Viniendo Jesús a la región de Cesarea de Filipo, preguntó a sus discípulos, diciendo: ¿Quién dicen los hombres que es el Hijo del Hombre?
\par 14 Ellos dijeron: Unos, Juan el Bautista; otros, Elías; y otros, Jeremías, o alguno de los profetas.
\par 15 El les dijo: Y vosotros, ¿quién decís que soy yo?
\par 16 Respondiendo Simón Pedro, dijo: Tú eres el Cristo, el Hijo del Dios viviente.
\par 17 Entonces le respondió Jesús: Bienaventurado eres, Simón, hijo de Jonás, porque no te lo reveló carne ni sangre, sino mi Padre que está en los cielos.
\par 18 Y yo también te digo, que tú eres Pedro, y sobre esta roca edificaré mi iglesia; y las puertas del Hades no prevalecerán contra ella.
\par 19 Y a ti te daré las llaves del reino de los cielos; y todo lo que atares en la tierra será atado en los cielos; y todo lo que desatares en la tierra será desatado en los cielos.
\par 20 Entonces mandó a sus discípulos que a nadie dijesen que él era Jesús el Cristo.

\section*{Jesús anuncia su muerte}

\par 21 Desde entonces comenzó Jesús a declarar a sus discípulos que le era necesario ir a Jerusalén y padecer mucho de los ancianos, de los principales sacerdotes y de los escribas; y ser muerto, y resucitar al tercer día.
\par 22 Entonces Pedro, tomándolo aparte, comenzó a reconvenirle, diciendo: Señor, ten compasión de ti; en ninguna manera esto te acontezca.
\par 23 Pero él, volviéndose, dijo a Pedro: ¡Quítate de delante de mí, Satanás!; me eres tropiezo, porque no pones la mira en las cosas de Dios, sino en las de los hombres.
\par 24 Entonces Jesús dijo a sus discípulos: Si alguno quiere venir en pos de mí, niéguese a sí mismo, y tome su cruz, y sígame.
\par 25 Porque todo el que quiera salvar su vida, la perderá; y todo el que pierda su vida por causa de mí, la hallará.
\par 26 Porque ¿qué aprovechará al hombre, si ganare todo el mundo, y perdiere su alma? ¿O qué recompensa dará el hombre por su alma?
\par 27 Porque el Hijo del Hombre vendrá en la gloria de su Padre con sus ángeles, y entonces pagará a cada uno conforme a sus obras.
\par 28 De cierto os digo que hay algunos de los que están aquí, que no gustarán la muerte, hasta que hayan visto al Hijo del Hombre viniendo en su reino.

\chapter{17}

\section*{La transfiguración}

\par 1 Seis días después, Jesús tomó a Pedro, a Jacobo y a Juan su hermano, y los llevó aparte a un monte alto;
\par 2 y se transfiguró delante de ellos, y resplandeció su rostro como el sol, y sus vestidos se hicieron blancos como la luz.
\par 3 Y he aquí les aparecieron Moisés y Elías, hablando con él.
\par 4 Entonces Pedro dijo a Jesús: Señor, bueno es para nosotros que estemos aquí; si quieres, hagamos aquí tres enramadas: una para ti, otra para Moisés, y otra para Elías.
\par 5 Mientras él aún hablaba, una nube de luz los cubrió; y he aquí una voz desde la nube, que decía: Este es mi Hijo amado, en quien tengo complacencia; a él oíd.
\par 6 Al oír esto los discípulos, se postraron sobre sus rostros, y tuvieron gran temor.
\par 7 Entonces Jesús se acercó y los tocó, y dijo: Levantaos, y no temáis.
\par 8 Y alzando ellos los ojos, a nadie vieron sino a Jesús solo.
\par 9 Cuando descendieron del monte, Jesús les mandó, diciendo: No digáis a nadie la visión, hasta que el Hijo del Hombre resucite de los muertos.
\par 10 Entonces sus discípulos le preguntaron, diciendo: ¿Por qué, pues, dicen los escribas que es necesario que Elías venga primero?
\par 11 Respondiendo Jesús, les dijo: A la verdad, Elías viene primero, y restaurará todas las cosas.
\par 12 Mas os digo que Elías ya vino, y no le conocieron, sino que hicieron con él todo lo que quisieron; así también el Hijo del Hombre padecerá de ellos.
\par 13 Entonces los discípulos comprendieron que les había hablado de Juan el Bautista.

\section*{Jesús sana a un muchacho lunático}

\par 14 Cuando llegaron al gentío, vino a él un hombre que se arrodilló delante de él, diciendo:
\par 15 Señor, ten misericordia de mi hijo, que es lunático, y padece muchísimo; porque muchas veces cae en el fuego, y muchas en el agua.
\par 16 Y lo he traído a tus discípulos, pero no le han podido sanar.
\par 17 Respondiendo Jesús, dijo: ¡Oh generación incrédula y perversa! ¿Hasta cuándo he de estar con vosotros? ¿Hasta cuándo os he de soportar? Traédmelo acá.
\par 18 Y reprendió Jesús al demonio, el cual salió del muchacho, y éste quedó sano desde aquella hora.
\par 19 Viniendo entonces los discípulos a Jesús, aparte, dijeron: ¿Por qué nosotros no pudimos echarlo fuera?
\par 20 Jesús les dijo: Por vuestra poca fe; porque de cierto os digo, que si tuviereis fe como un grano de mostaza, diréis a este monte: Pásate de aquí allá, y se pasará; y nada os será imposible.
\par 21 Pero este género no sale sino con oración y ayuno.

\section*{Jesús anuncia otra vez su muerte}

\par 22 Estando ellos en Galilea, Jesús les dijo: El Hijo del Hombre será entregado en manos de hombres,
\par 23 y le matarán; mas al tercer día resucitará. Y ellos se entristecieron en gran manera.

\section*{Pago del impuesto del templo}

\par 24 Cuando llegaron a Capernaum, vinieron a Pedro los que cobraban las dos dracmas, y le dijeron: ¿Vuestro Maestro no paga las dos dracmas?
\par 25 El dijo: Sí. Y al entrar él en casa, Jesús le habló primero, diciendo: ¿Qué te parece, Simón? Los reyes de la tierra, ¿de quiénes cobran los tributos o los impuestos? ¿De sus hijos, o de los extraños?
\par 26 Pedro le respondió: De los extraños. Jesús le dijo: Luego los hijos están exentos.
\par 27 Sin embargo, para no ofenderles, ve al mar, y echa el anzuelo, y el primer pez que saques, tómalo, y al abrirle la boca, hallarás un estatero; tómalo, y dáselo por mí y por ti.

\chapter{18}

\section*{¿Quién es el mayor?}

\par 1 En aquel tiempo los discípulos vinieron a Jesús, diciendo: ¿Quién es el mayor en el reino de los cielos?
\par 2 Y llamando Jesús a un niño, lo puso en medio de ellos,
\par 3 y dijo: De cierto os digo, que si no os volvéis y os hacéis como niños, no entraréis en el reino de los cielos.
\par 4 Así que, cualquiera que se humille como este niño, ése es el mayor en el reino de los cielos.
\par 5 Y cualquiera que reciba en mi nombre a un niño como este, a mí me recibe.

\section*{Ocasiones de caer}

\par 6 Y cualquiera que haga tropezar a alguno de estos pequeños que creen en mí, mejor le fuera que se le colgase al cuello una piedra de molino de asno, y que se le hundiese en lo profundo del mar.
\par 7 ¡Ay del mundo por los tropiezos! porque es necesario que vengan tropiezos, pero ¡ay de aquel hombre por quien viene el tropiezo!
\par 8 Por tanto, si tu mano o tu pie te es ocasión de caer, córtalo y échalo de ti; mejor te es entrar en la vida cojo o manco, que teniendo dos manos o dos pies ser echado en el fuego eterno.
\par 9 Y si tu ojo te es ocasión de caer, sácalo y échalo de ti; mejor te es entrar con un solo ojo en la vida, que teniendo dos ojos ser echado en el infierno de fuego.

\section*{Parábola de la oveja perdida}

\par 10 Mirad que no menospreciéis a uno de estos pequeños; porque os digo que sus ángeles en los cielos ven siempre el rostro de mi Padre que está en los cielos.
\par 11 Porque el Hijo del Hombre ha venido para salvar lo que se había perdido.
\par 12 ¿Qué os parece? Si un hombre tiene cien ovejas, y se descarría una de ellas, ¿no deja las noventa y nueve y va por los montes a buscar la que se había descarriado?
\par 13 Y si acontece que la encuentra, de cierto os digo que se regocija más por aquélla, que por las noventa y nueve que no se descarriaron.
\par 14 Así, no es la voluntad de vuestro Padre que está en los cielos, que se pierda uno de estos pequeños.

\section*{Cómo se debe perdonar al hermano}

\par 15 Por tanto, si tu hermano peca contra ti, ve y repréndele estando tú y él solos; si te oyere, has ganado a tu hermano.
\par 16 Mas si no te oyere, toma aún contigo a uno o dos, para que en boca de dos o tres testigos conste toda palabra.
\par 17 Si no los oyere a ellos, dilo a la iglesia; y si no oyere a la iglesia, tenle por gentil y publicano.
\par 18 De cierto os digo que todo lo que atéis en la tierra, será atado en el cielo; y todo lo que desatéis en la tierra, será desatado en el cielo.
\par 19 Otra vez os digo, que si dos de vosotros se pusieren de acuerdo en la tierra acerca de cualquiera cosa que pidieren, les será hecho por mi Padre que está en los cielos.
\par 20 Porque donde están dos o tres congregados en mi nombre, allí estoy yo en medio de ellos.
\par 21 Entonces se le acercó Pedro y le dijo: Señor, ¿cuántas veces perdonaré a mi hermano que peque contra mí? ¿Hasta siete?
\par 22 Jesús le dijo: No te digo hasta siete, sino aun hasta setenta veces siete.

\section*{Los dos deudores}

\par 23 Por lo cual el reino de los cielos es semejante a un rey que quiso hacer cuentas con sus siervos.
\par 24 Y comenzando a hacer cuentas, le fue presentado uno que le debía diez mil talentos
\par 25 A éste, como no pudo pagar, ordenó su señor venderle, y a su mujer e hijos, y todo lo que tenía, para que se le pagase la deuda.
\par 26 Entonces aquel siervo, postrado, le suplicaba, diciendo: Señor, ten paciencia conmigo, y yo te lo pagaré todo.
\par 27 El señor de aquel siervo, movido a misericordia, le soltó y le perdonó la deuda.
\par 28 Pero saliendo aquel siervo, halló a uno de sus consiervos, que le debía cien denarios; y asiendo de él, le ahogaba, diciendo: Págame lo que me debes.
\par 29 Entonces su consiervo, postrándose a sus pies, le rogaba diciendo: Ten paciencia conmigo, y yo te lo pagaré todo.
\par 30 Mas él no quiso, sino fue y le echó en la cárcel, hasta que pagase la deuda.
\par 31 Viendo sus consiervos lo que pasaba, se entristecieron mucho, y fueron y refirieron a su señor todo lo que había pasado.
\par 32 Entonces, llamándole su señor, le dijo: Siervo malvado, toda aquella deuda te perdoné, porque me rogaste.
\par 33 ¿No debías tú también tener misericordia de tu consiervo, como yo tuve misericordia de ti?
\par 34 Entonces su señor, enojado, le entregó a los verdugos, hasta que pagase todo lo que le debía. 18:35 Así también mi Padre celestial hará con vosotros si no perdonáis de todo corazón cada uno a su hermano sus ofensas.

\chapter{19}

\section*{Jesús enseña sobre el divorcio}

\par 1 Aconteció que cuando Jesús terminó estas palabras, se alejó de Galilea, y fue a las regiones de Judea al otro lado del Jordán.
\par 2 Y le siguieron grandes multitudes, y los sanó allí.
\par 3 Entonces vinieron a él los fariseos, tentándole y diciéndole: ¿Es lícito al hombre repudiar a su mujer por cualquier causa?
\par 4 El, respondiendo, les dijo: ¿No habéis leído que el que los hizo al principio, varón y hembra los hizo,
\par 5 y dijo: Por esto el hombre dejará padre y madre, y se unirá a su mujer, y los dos serán una sola carne?
\par 6 Así que no son ya más dos, sino una sola carne; por tanto, lo que Dios juntó, no lo separe el hombre.
\par 7 Le dijeron: ¿Por qué, pues, mandó Moisés dar carta de divorcio, y repudiarla?
\par 8 El les dijo: Por la dureza de vuestro corazón Moisés os permitió repudiar a vuestras mujeres; mas al principio no fue así.
\par 9 Y yo os digo que cualquiera que repudia a su mujer, salvo por causa de fornicación, y se casa con otra, adultera; y el que se casa con la repudiada, adultera.
\par 10 Le dijeron sus discípulos: Si así es la condición del hombre con su mujer, no conviene casarse.
\par 11 Entonces él les dijo: No todos son capaces de recibir esto, sino aquellos a quienes es dado.
\par 12 Pues hay eunucos que nacieron así del vientre de su madre, y hay eunucos que son hechos eunucos por los hombres, y hay eunucos que a sí mismos se hicieron eunucos por causa del reino de los cielos. El que sea capaz de recibir esto, que lo reciba.

\section*{Jesús bendice a los niños}

\par 13 Entonces le fueron presentados unos niños, para que pusiese las manos sobre ellos, y orase; y los discípulos les reprendieron.
\par 14 Pero Jesús dijo: Dejad a los niños venir a mí, y no se lo impidáis; porque de los tales es el reino de los cielos.
\par 15 Y habiendo puesto sobre ellos las manos, se fue de allí.

\section*{El joven rico}

\par 16 Entonces vino uno y le dijo: Maestro bueno, ¿qué bien haré para tener la vida eterna?
\par 17 El le dijo: ¿Por qué me llamas bueno? Ninguno hay bueno sino uno: Dios. Mas si quieres entrar en la vida, guarda los mandamientos.
\par 18 Le dijo: ¿Cuáles? Y Jesús dijo: No matarás. No adulterarás. No hurtarás. No dirás falso testimonio.
\par 19 Honra a tu padre y a tu madre; y, Amarás a tu prójimo como a ti mismo.
\par 20 El joven le dijo: Todo esto lo he guardado desde mi juventud. ¿Qué más me falta?
\par 21 Jesús le dijo: Si quieres ser perfecto, anda, vende lo que tienes, y dalo a los pobres, y tendrás tesoro en el cielo; y ven y sígueme.
\par 22 Oyendo el joven esta palabra, se fue triste, porque tenía muchas posesiones.
\par 23 Entonces Jesús dijo a sus discípulos: De cierto os digo, que difícilmente entrará un rico en el reino de los cielos.
\par 24 Otra vez os digo, que es más fácil pasar un camello por el ojo de una aguja, que entrar un rico en el reino de Dios.
\par 25 Sus discípulos, oyendo esto, se asombraron en gran manera, diciendo: ¿Quién, pues, podrá ser salvo?
\par 26 Y mirándolos Jesús, les dijo: Para los hombres esto es imposible; mas para Dios todo es posible.
\par 27 Entonces respondiendo Pedro, le dijo: He aquí, nosotros lo hemos dejado todo, y te hemos seguido; ¿qué, pues, tendremos?
\par 28 Y Jesús les dijo: De cierto os digo que en la regeneración, cuando el Hijo del Hombre se siente en el trono de su gloria, vosotros que me habéis seguido también os sentaréis sobre doce tronos, para juzgar a las doce tribus de Israel.
\par 29 Y cualquiera que haya dejado casas, o hermanos, o hermanas, o padre, o madre, o mujer, o hijos, o tierras, por mi nombre, recibirá cien veces más, y heredará la vida eterna.
\par 30 Pero muchos primeros serán postreros, y postreros, primeros.

\chapter{20}

\section*{Los obreros de la viña}

\par 1 Porque el reino de los cielos es semejante a un hombre, padre de familia, que salió por la mañana a contratar obreros para su viña.
\par 2 Y habiendo convenido con los obreros en un denario al día, los envió a su viña.
\par 3 Saliendo cerca de la hora tercera del día, vio a otros que estaban en la plaza desocupados;
\par 4 y les dijo: Id también vosotros a mi viña, y os daré lo que sea justo. Y ellos fueron.
\par 5 Salió otra vez cerca de las horas sexta y novena, e hizo lo mismo.
\par 6 Y saliendo cerca de la hora undécima, halló a otros que estaban desocupados; y les dijo: ¿Por qué estáis aquí todo el día desocupados?
\par 7 Le dijeron: Porque nadie nos ha contratado. El les dijo: Id también vosotros a la viña, y recibiréis lo que sea justo.
\par 8 Cuando llegó la noche, el señor de la viña dijo a su mayordomo: Llama a los obreros y págales el jornal, comenzando desde los postreros hasta los primeros.
\par 9 Y al venir los que habían ido cerca de la hora undécima, recibieron cada uno un denario.
\par 10 Al venir también los primeros, pensaron que habían de recibir más; pero también ellos recibieron cada uno un denario.
\par 11 Y al recibirlo, murmuraban contra el padre de familia,
\par 12 diciendo: Estos postreros han trabajado una sola hora, y los has hecho iguales a nosotros, que hemos soportado la carga y el calor del día.
\par 13 El, respondiendo, dijo a uno de ellos: Amigo, no te hago agravio; ¿no conviniste conmigo en un denario?
\par 14 Toma lo que es tuyo, y vete; pero quiero dar a este postrero, como a ti.
\par 15 ¿No me es lícito hacer lo que quiero con lo mío? ¿O tienes tú envidia, porque yo soy bueno?
\par 16 Así, los primeros serán postreros, y los postreros, primeros; porque muchos son llamados, mas pocos escogidos.

\section*{Nuevamente Jesús anuncia su muerte}

\par 17 Subiendo Jesús a Jerusalén, tomó a sus doce discípulos aparte en el camino, y les dijo:
\par 18 He aquí subimos a Jerusalén, y el Hijo del Hombre será entregado a los principales sacerdotes y a los escribas, y le condenarán a muerte;
\par 19 y le entregarán a los gentiles para que le escarnezcan, le azoten, y le crucifiquen; mas al tercer día resucitará.

\section*{Petición de Santiago y de Juan}

\par 20 Entonces se le acercó la madre de los hijos de Zebedeo con sus hijos, postrándose ante él y pidiéndole algo.
\par 21 El le dijo: ¿Qué quieres? Ella le dijo: Ordena que en tu reino se sienten estos dos hijos míos, el uno a tu derecha, y el otro a tu izquierda.
\par 22 Entonces Jesús respondiendo, dijo: No sabéis lo que pedís. ¿Podéis beber del vaso que yo he de beber, y ser bautizados con el bautismo con que yo soy bautizado? Y ellos le dijeron: Podemos.
\par 23 El les dijo: A la verdad, de mi vaso beberéis, y con el bautismo con que yo soy bautizado, seréis bautizados; pero el sentaros a mi derecha y a mi izquierda, no es mío darlo, sino a aquellos para quienes está preparado por mi Padre.
\par 24 Cuando los diez oyeron esto, se enojaron contra los dos hermanos.
\par 25 Entonces Jesús, llamándolos, dijo: Sabéis que los gobernantes de las naciones se enseñorean de ellas, y los que son grandes ejercen sobre ellas potestad.
\par 26 Mas entre vosotros no será así, sino que el que quiera hacerse grande entre vosotros será vuestro servidor,
\par 27 y el que quiera ser el primero entre vosotros será vuestro siervo;
\par 28 como el Hijo del Hombre no vino para ser servido, sino para servir, y para dar su vida en rescate por muchos.

\section*{Dos ciegos reciben la vista}

\par 29 Al salir ellos de Jericó, le seguía una gran multitud.
\par 30 Y dos ciegos que estaban sentados junto al camino, cuando oyeron que Jesús pasaba, clamaron, diciendo: ¡Señor, Hijo de David, ten misericordia de nosotros!
\par 31 Y la gente les reprendió para que callasen; pero ellos clamaban más, diciendo: ¡Señor, Hijo de David, ten misericordia de nosotros!
\par 32 Y deteniéndose Jesús, los llamó, y les dijo: ¿Qué queréis que os haga?
\par 33 Ellos le dijeron: Señor, que sean abiertos nuestros ojos.
\par 34 Entonces Jesús, compadecido, les tocó los ojos, y en seguida recibieron la vista; y le siguieron.

\chapter{21}

\section*{La entrada triunfal en Jerusalén}

\par 1 Cuando se acercaron a Jerusalén, y vinieron a Betfagé, al monte de los Olivos, Jesús envió dos discípulos,
\par 2 diciéndoles: Id a la aldea que está enfrente de vosotros, y luego hallaréis una asna atada, y un pollino con ella; desatadla, y traédmelos.
\par 3 Y si alguien os dijere algo, decid: El Señor los necesita; y luego los enviará.
\par 4 Todo esto aconteció para que se cumpliese lo dicho por el profeta, cuando dijo:
\par 5 Decid a la hija de Sion:
\par He aquí, tu Rey viene a ti,
\par Manso, y sentado sobre una asna,
\par Sobre un pollino, hijo de animal de carga.
\par 6 Y los discípulos fueron, e hicieron como Jesús les mandó;
\par 7 y trajeron el asna y el pollino, y pusieron sobre ellos sus mantos; y él se sentó encima.
\par 8 Y la multitud, que era muy numerosa, tendía sus mantos en el camino; y otros cortaban ramas de los árboles, y las tendían en el camino.
\par 9 Y la gente que iba delante y la que iba detrás aclamaba, diciendo: ¡Hosanna al Hijo de David! ¡Bendito el que viene en el nombre del Señor! ¡Hosanna en las alturas!
\par 10 Cuando entró él en Jerusalén, toda la ciudad se conmovió, diciendo: ¿Quién es éste?
\par 11 Y la gente decía: Este es Jesús el profeta, de Nazaret de Galilea.

\section*{Purificación del templo}

\par 12 Y entró Jesús en el templo de Dios, y echó fuera a todos los que vendían y compraban en el templo, y volcó las mesas de los cambistas, y las sillas de los que vendían palomas;
\par 13 y les dijo: Escrito está: Mi casa, casa de oración será llamada; mas vosotros la habéis hecho cueva de ladrones.
\par 14 Y vinieron a él en el templo ciegos y cojos, y los sanó.
\par 15 Pero los principales sacerdotes y los escribas, viendo las maravillas que hacía, y a los muchachos aclamando en el templo y diciendo: ¡Hosanna al Hijo de David! se indignaron,
\par 16 y le dijeron: ¿Oyes lo que éstos dicen? Y Jesús les dijo: Sí; ¿nunca leísteis:
\par De la boca de los niños y de los que maman
\par Perfeccionaste la alabanza?
\par 17 Y dejándolos, salió fuera de la ciudad a Betania, y posó allí.

\section*{Maldición de la higuera estéril}

\par 18 Por la mañana, volviendo a la ciudad, tuvo hambre.
\par 19 Y viendo una higuera cerca del camino, vino a ella, y no halló nada en ella, sino hojas solamente; y le dijo: Nunca jamás nazca de ti fruto. Y luego se secó la higuera.
\par 20 Viendo esto los discípulos, decían maravillados: ¿Cómo es que se secó en seguida la higuera?
\par 21 Respondiendo Jesús, les dijo: De cierto os digo, que si tuviereis fe, y no dudareis, no sólo haréis esto de la higuera, sino que si a este monte dijereis: Quítate y échate en el mar, será hecho.
\par 22 Y todo lo que pidiereis en oración, creyendo, lo recibiréis.

\section*{La autoridad de Jesús}

\par 23 Cuando vino al templo, los principales sacerdotes y los ancianos del pueblo se acercaron a él mientras enseñaba, y le dijeron: ¿Con qué autoridad haces estas cosas? ¿y quién te dio esta autoridad?
\par 24 Respondiendo Jesús, les dijo: Yo también os haré una pregunta, y si me la contestáis, también yo os diré con qué autoridad hago estas cosas.
\par 25 El bautismo de Juan, ¿de dónde era? ¿Del cielo, o de los hombres? Ellos entonces discutían entre sí, diciendo: Si decimos, del cielo, nos dirá: ¿Por qué, pues, no le creísteis?
\par 26 Y si decimos, de los hombres, tememos al pueblo; porque todos tienen a Juan por profeta.
\par 27 Y respondiendo a Jesús, dijeron: No sabemos. Y él también les dijo: Tampoco yo os digo con qué autoridad hago estas cosas.

\section*{Parábola de los dos hijos}

\par 28 Pero ¿qué os parece? Un hombre tenía dos hijos, y acercándose al primero, le dijo: Hijo, vé hoy a trabajar en mi viña.
\par 29 Respondiendo él, dijo: No quiero; pero después, arrepentido, fue.
\par 30 Y acercándose al otro, le dijo de la misma manera; y respondiendo él, dijo: Sí, señor, voy. Y no fue.
\par 31 ¿Cuál de los dos hizo la voluntad de su padre? Dijeron ellos: El primero. Jesús les dijo: De cierto os digo, que los publicanos y las rameras van delante de vosotros al reino de Dios.
\par 32 Porque vino a vosotros Juan en camino de justicia, y no le creísteis; pero los publicanos y las rameras le creyeron; y vosotros, viendo esto, no os arrepentisteis después para creerle.

\section*{Los labradores malvados}

\par 33 Oíd otra parábola: Hubo un hombre, padre de familia, el cual plantó una viña, la cercó de vallado, cavó en ella un lagar, edificó una torre, y la arrendó a unos labradores, y se fue lejos.
\par 34 Y cuando se acercó el tiempo de los frutos, envió sus siervos a los labradores, para que recibiesen sus frutos.
\par 35 Mas los labradores, tomando a los siervos, a uno golpearon, a otro mataron, y a otro apedrearon.
\par 36 Envió de nuevo otros siervos, más que los primeros; e hicieron con ellos de la misma manera.
\par 37 Finalmente les envió su hijo, diciendo: Tendrán respeto a mi hijo.
\par 38 Mas los labradores, cuando vieron al hijo, dijeron entre sí: Este es el heredero; venid, matémosle, y apoderémonos de su heredad.
\par 39 Y tomándole, le echaron fuera de la viña, y le mataron.
\par 40 Cuando venga, pues, el señor de la viña, ¿qué hará a aquellos labradores?
\par 41 Le dijeron: A los malos destruirá sin misericordia, y arrendará su viña a otros labradores, que le paguen el fruto a su tiempo.
\par 42 Jesús les dijo: ¿Nunca leísteis en las Escrituras:
\par La piedra que desecharon los edificadores,
\par Ha venido a ser cabeza del ángulo.
\par El Señor ha hecho esto,
\par Y es cosa maravillosa a nuestros ojos?
\par 43 Por tanto os digo, que el reino de Dios será quitado de vosotros, y será dado a gente que produzca los frutos de él.
\par 44 Y el que cayere sobre esta piedra será quebrantado; y sobre quien ella cayere, le desmenuzará.
\par 45 Y oyendo sus parábolas los principales sacerdotes y los fariseos, entendieron que hablaba de ellos.
\par 46 Pero al buscar cómo echarle mano, temían al pueblo, porque éste le tenía por profeta.

\chapter{22}

\section*{Parábola de la fiesta de bodas}

\par 1 Respondiendo Jesús, les volvió a hablar en parábolas, diciendo:
\par 2 El reino de los cielos es semejante a un rey que hizo fiesta de bodas a su hijo;
\par 3 y envió a sus siervos a llamar a los convidados a las bodas; mas éstos no quisieron venir.
\par 4 Volvió a enviar otros siervos, diciendo: Decid a los convidados: He aquí, he preparado mi comida; mis toros y animales engordados han sido muertos, y todo está dispuesto; venid a las bodas.
\par 5 Mas ellos, sin hacer caso, se fueron, uno a su labranza, y otro a sus negocios;
\par 6 y otros, tomando a los siervos, los afrentaron y los mataron.
\par 7 Al oírlo el rey, se enojó; y enviando sus ejércitos, destruyó a aquellos homicidas, y quemó su ciudad.
\par 8 Entonces dijo a sus siervos: Las bodas a la verdad están preparadas; mas los que fueron convidados no eran dignos.
\par 9 Id, pues, a las salidas de los caminos, y llamad a las bodas a cuantos halléis.
\par 10 Y saliendo los siervos por los caminos, juntaron a todos los que hallaron, juntamente malos y buenos; y las bodas fueron llenas de convidados.
\par 11 Y entró el rey para ver a los convidados, y vio allí a un hombre que no estaba vestido de boda.
\par 12 Y le dijo: Amigo, ¿cómo entraste aquí, sin estar vestido de boda? Mas él enmudeció.
\par 13 Entonces el rey dijo a los que servían: Atadle de pies y manos, y echadle en las tinieblas de afuera; allí será el lloro y el crujir de dientes.
\par 14 Porque muchos son llamados, y pocos escogidos.

\section*{La cuestión del tributo}

\par 15 Entonces se fueron los fariseos y consultaron cómo sorprenderle en alguna palabra.
\par 16 Y le enviaron los discípulos de ellos con los herodianos, diciendo: Maestro, sabemos que eres amante de la verdad, y que enseñas con verdad el camino de Dios, y que no te cuidas de nadie, porque no miras la apariencia de los hombres.
\par 17 Dinos, pues, qué te parece: ¿Es lícito dar tributo a César, o no?
\par 18 Pero Jesús, conociendo la malicia de ellos, les dijo: ¿Por qué me tentáis, hipócritas?
\par 19 Mostradme la moneda del tributo. Y ellos le presentaron un denario.
\par 20 Entonces les dijo:¿De quién es esta imagen, y la inscripción?
\par 21 Le dijeron: De César. Y les dijo: Dad, pues, a César lo que es de César, y a Dios lo que es de Dios.
\par 22 Oyendo esto, se maravillaron, y dejándole, se fueron.

\section*{La pregunta sobre la resurrección}

\par 23 Aquel día vinieron a él los saduceos, que dicen que no hay resurrección, y le preguntaron,
\par 24 diciendo: Maestro, Moisés dijo: Si alguno muriere sin hijos, su hermano se casará con su mujer, y levantará descendencia a su hermano.
\par 25 Hubo, pues, entre nosotros siete hermanos; el primero se casó, y murió; y no teniendo descendencia, dejó su mujer a su hermano.
\par 26 De la misma manera también el segundo, y el tercero, hasta el séptimo.
\par 27 Y después de todos murió también la mujer.
\par 28 En la resurrección, pues, ¿de cuál de los siete será ella mujer, ya que todos la tuvieron?
\par 29 Entonces respondiendo Jesús, les dijo: Erráis, ignorando las Escrituras y el poder de Dios.
\par 30 Porque en la resurrección ni se casarán ni se darán en casamiento, sino serán como los ángeles de Dios en el cielo.
\par 31 Pero respecto a la resurrección de los muertos, ¿no habéis leído lo que os fue dicho por Dios, cuando dijo:
\par 32 Yo soy el Dios de Abraham, el Dios de Isaac y el Dios de Jacob? Dios no es Dios de muertos, sino de vivos.
\par 33 Oyendo esto la gente, se admiraba de su doctrina.

\section*{El gran mandamiento}

\par 34 Entonces los fariseos, oyendo que había hecho callar a los saduceos, se juntaron a una.
\par 35 Y uno de ellos, intérprete de la ley, preguntó por tentarle, diciendo:
\par 36 Maestro, ¿cuál es el gran mandamiento en la ley?
\par 37 Jesús le dijo: Amarás al Señor tu Dios con todo tu corazón, y con toda tu alma, y con toda tu mente.
\par 38 Este es el primero y grande mandamiento.
\par 39 Y el segundo es semejante: Amarás a tu prójimo como a ti mismo.
\par 40 De estos dos mandamientos depende toda la ley y los profetas.

\section*{¿De quién es hijo el Cristo?}

\par 41 Y estando juntos los fariseos, Jesús les preguntó,
\par 42 diciendo: ¿Qué pensáis del Cristo? ¿De quién es hijo? Le dijeron: De David.
\par 43 El les dijo: ¿Pues cómo David en el Espíritu le llama Señor, diciendo:
\par 44 Dijo el Señor a mi Señor:
\par Siéntate a mi derecha,
\par Hasta que ponga a tus enemigos por estrado de tus pies?
\par 45 Pues si David le llama Señor, ¿cómo es su hijo?
\par 46 Y nadie le podía responder palabra; ni osó alguno desde aquel día preguntarle más.

\chapter{23}

\section*{Jesús acusa a escribas y fariseos}

\par 1 Entonces habló Jesús a la gente y a sus discípulos, diciendo:
\par 2 En la cátedra de Moisés se sientan los escribas y los fariseos.
\par 3 Así que, todo lo que os digan que guardéis, guardadlo y hacedlo; mas no hagáis conforme a sus obras, porque dicen, y no hacen.
\par 4 Porque atan cargas pesadas y difíciles de llevar, y las ponen sobre los hombros de los hombres; pero ellos ni con un dedo quieren moverlas.
\par 5 Antes, hacen todas sus obras para ser vistos por los hombres. Pues ensanchan sus filacterias, y extienden los flecos de sus mantos;
\par 6 y aman los primeros asientos en las cenas, y las primeras sillas en las sinagogas,
\par 7 y las salutaciones en las plazas, y que los hombres los llamen: Rabí, Rabí.
\par 8 Pero vosotros no queráis que os llamen Rabí; porque uno es vuestro Maestro, el Cristo, y todos vosotros sois hermanos.
\par 9 Y no llaméis padre vuestro a nadie en la tierra; porque uno es vuestro Padre, el que está en los cielos.
\par 10 Ni seáis llamados maestros; porque uno es vuestro Maestro, el Cristo.
\par 11 El que es el mayor de vosotros, sea vuestro siervo.
\par 12 Porque el que se enaltece será humillado, y el que se humilla será enaltecido.
\par 13 Mas ¡ay de vosotros, escribas y fariseos, hipócritas! porque cerráis el reino de los cielos delante de los hombres; pues ni entráis vosotros, ni dejáis entrar a los que están entrando.
\par 14 ¡Ay de vosotros, escribas y fariseos, hipócritas! porque devoráis las casas de las viudas, y como pretexto hacéis largas oraciones; por esto recibiréis mayor condenación.
\par 15 ¡Ay de vosotros, escribas y fariseos, hipócritas! porque recorréis mar y tierra para hacer un prosélito, y una vez hecho, le hacéis dos veces más hijo del infierno que vosotros.
\par 16 ¡Ay de vosotros, guías ciegos! que decís: Si alguno jura por el templo, no es nada; pero si alguno jura por el oro del templo, es deudor.
\par 17 ¡Insensatos y ciegos! porque ¿cuál es mayor, el oro, o el templo que santifica al oro?
\par 18 También decís: Si alguno jura por el altar, no es nada; pero si alguno jura por la ofrenda que está sobre él, es deudor.
\par 19 ¡Necios y ciegos! porque ¿cuál es mayor, la ofrenda, o el altar que santifica la ofrenda?
\par 20 Pues el que jura por el altar, jura por él, y por todo lo que está sobre él;
\par 21 y el que jura por el templo, jura por él, y por el que lo habita;
\par 22 y el que jura por el cielo, jura por el trono de Dios, y por aquel que está sentado en él.
\par 23 ¡Ay de vosotros, escribas y fariseos, hipócritas! porque diezmáis la menta y el eneldo y el comino, y dejáis lo más importante de la ley: la justicia, la misericordia y la fe. Esto era necesario hacer, sin dejar de hacer aquello.
\par 24 ¡Guías ciegos, que coláis el mosquito, y tragáis el camello!
\par 25 ¡Ay de vosotros, escribas y fariseos, hipócritas! porque limpiáis lo de fuera del vaso y del plato, pero por dentro estáis llenos de robo y de injusticia.
\par 26 ¡Fariseo ciego! Limpia primero lo de dentro del vaso y del plato, para que también lo de fuera sea limpio.
\par 27 ¡Ay de vosotros, escribas y fariseos, hipócritas! porque sois semejantes a sepulcros blanqueados, que por fuera, a la verdad, se muestran hermosos, mas por dentro están llenos de huesos de muertos y de toda inmundicia.
\par 28 Así también vosotros por fuera, a la verdad, os mostráis justos a los hombres, pero por dentro estáis llenos de hipocresía e iniquidad.
\par 29 ¡Ay de vosotros, escribas y fariseos, hipócritas! porque edificáis los sepulcros de los profetas, y adornáis los monumentos de los justos,
\par 30 y decís: Si hubiésemos vivido en los días de nuestros padres, no hubiéramos sido sus cómplices en la sangre de los profetas.
\par 31 Así que dais testimonio contra vosotros mismos, de que sois hijos de aquellos que mataron a los profetas.
\par 32 ¡Vosotros también llenad la medida de vuestros padres!
\par 33 ¡Serpientes, generación de víboras! ¿Cómo escaparéis de la condenación del infierno?
\par 34 Por tanto, he aquí yo os envío profetas y sabios y escribas; y de ellos, a unos mataréis y crucificaréis, y a otros azotaréis en vuestras sinagogas, y perseguiréis de ciudad en ciudad;
\par 35 para que venga sobre vosotros toda la sangre justa que se ha derramado sobre la tierra, desde la sangre de Abel el justo hasta la sangre de Zacarías hijo de Berequías, a quien matasteis entre el templo y el altar.
\par 36 De cierto os digo que todo esto vendrá sobre esta generación.

\section*{Lamento de Jesús sobre Jerusalén}

\par 37 ¡Jerusalén, Jerusalén, que matas a los profetas, y apedreas a los que te son enviados! ¡Cuántas veces quise juntar a tus hijos, como la gallina junta sus polluelos debajo de las alas, y no quisiste!
\par 38 He aquí vuestra casa os es dejada desierta.
\par 39 Porque os digo que desde ahora no me veréis, hasta que digáis: Bendito el que viene en el nombre del Señor.

\chapter{24}

\section*{Jesús predice la destrucción del templo}

\par 1 Cuando Jesús salió del templo y se iba, se acercaron sus discípulos para mostrarle los edificios del templo.
\par 2 Respondiendo él, les dijo: ¿Veis todo esto? De cierto os digo, que no quedará aquí piedra sobre piedra, que no sea derribada.

\section*{Señales antes del fin}

\par 3 Y estando él sentado en el monte de los Olivos, los discípulos se le acercaron aparte,
\par diciendo: Dinos, ¿cuándo serán estas cosas, y qué señal habrá de tu venida, y del fin del siglo?
\par 4 Respondiendo Jesús, les dijo: Mirad que nadie os engañe.
\par 5 Porque vendrán muchos en mi nombre, diciendo: Yo soy el Cristo; y a muchos engañarán.
\par 6 Y oiréis de guerras y rumores de guerras; mirad que no os turbéis, porque es necesario que todo esto acontezca; pero aún no es el fin.
\par 7 Porque se levantará nación contra nación, y reino contra reino; y habrá pestes, y hambres, y terremotos en diferentes lugares.
\par 8 Y todo esto será principio de dolores.
\par 9 Entonces os entregarán a tribulación, y os matarán, y seréis aborrecidos de todas las gentes por causa de mi nombre.
\par 10 Muchos tropezarán entonces, y se entregarán unos a otros, y unos a otros se aborrecerán.
\par 11 Y muchos falsos profetas se levantarán, y engañarán a muchos;
\par 12 y por haberse multiplicado la maldad, el amor de muchos se enfriará.
\par 13 Mas el que persevere hasta el fin, éste será salvo.
\par 14 Y será predicado este evangelio del reino en todo el mundo, para testimonio a todas las naciones; y entonces vendrá el fin.
\par 15 Por tanto, cuando veáis en el lugar santo la abominación desoladora de que habló el profeta Daniel (el que lee, entienda),
\par 16 entonces los que estén en Judea, huyan a los montes.
\par 17 El que esté en la azotea, no descienda para tomar algo de su casa;
\par 18 y el que esté en el campo, no vuelva atrás para tomar su capa.
\par 19 Mas ¡ay de las que estén encintas, y de las que críen en aquellos días!
\par 20 Orad, pues, que vuestra huida no sea en invierno ni en día de reposo;
\par 21 porque habrá entonces gran tribulación, cual no la ha habido desde el principio del mundo hasta ahora, ni la habrá.
\par 22 Y si aquellos días no fuesen acortados, nadie sería salvo; mas por causa de los escogidos, aquellos días serán acortados.
\par 23 Entonces, si alguno os dijere: Mirad, aquí está el Cristo, o mirad, allí está, no lo creáis.
\par 24 Porque se levantarán falsos Cristos, y falsos profetas, y harán grandes señales y prodigios, de tal manera que engañarán, si fuere posible, aun a los escogidos.
\par 25 Ya os lo he dicho antes.
\par 26 Así que, si os dijeren: Mirad, está en el desierto, no salgáis; o mirad, está en los aposentos, no lo creáis.
\par 27 Porque como el relámpago que sale del oriente y se muestra hasta el occidente, así será también la venida del Hijo del Hombre.
\par 28 Porque dondequiera que estuviere el cuerpo muerto, allí se juntarán las águilas.

\section*{La venida del Hijo del Hombre}

\par 29 E inmediatamente después de la tribulación de aquellos días, el sol se oscurecerá, y la luna no dará su resplandor, y las estrellas caerán del cielo, y las potencias de los cielos serán conmovidas.
\par 30 Entonces aparecerá la señal del Hijo del Hombre en el cielo; y entonces lamentarán todas las tribus de la tierra, y verán al Hijo del Hombre viniendo sobre las nubes del cielo, con poder y gran gloria.
\par 31 Y enviará sus ángeles con gran voz de trompeta, y juntarán a sus escogidos, de los cuatro vientos, desde un extremo del cielo hasta el otro.
\par 32 De la higuera aprended la parábola: Cuando ya su rama está tierna, y brotan las hojas, sabéis que el verano está cerca.
\par 33 Así también vosotros, cuando veáis todas estas cosas, conoced que está cerca, a las puertas.
\par 34 De cierto os digo, que no pasará esta generación hasta que todo esto acontezca.
\par 35 El cielo y la tierra pasarán, pero mis palabras no pasarán.
\par 36 Pero del día y la hora nadie sabe, ni aun los ángeles de los cielos, sino sólo mi Padre.
\par 37 Mas como en los días de Noé, así será la venida del Hijo del Hombre.
\par 38 Porque como en los días antes del diluvio estaban comiendo y bebiendo, casándose y dando en casamiento, hasta el día en que Noé entró en el arca,
\par 39 y no entendieron hasta que vino el diluvio y se los llevó a todos, así será también la venida del Hijo del Hombre.
\par 40 Entonces estarán dos en el campo; el uno será tomado, y el otro será dejado.
\par 41 Dos mujeres estarán moliendo en un molino; la una será tomada, y la otra será dejada.
\par 42 Velad, pues, porque no sabéis a qué hora ha de venir vuestro Señor.
\par 43 Pero sabed esto, que si el padre de familia supiese a qué hora el ladrón habría de venir, velaría, y no dejaría minar su casa.
\par 44 Por tanto, también vosotros estad preparados; porque el Hijo del Hombre vendrá a la hora que no pensáis.
\par 45 ¿Quién es, pues, el siervo fiel y prudente, al cual puso su señor sobre su casa para que les dé el alimento a tiempo?
\par 46 Bienaventurado aquel siervo al cual, cuando su señor venga, le halle haciendo así.
\par 47 De cierto os digo que sobre todos sus bienes le pondrá.
\par 48 Pero si aquel siervo malo dijere en su corazón: Mi señor tarda en venir;
\par 49 y comenzare a golpear a sus consiervos, y aun a comer y a beber con los borrachos,
\par 50 vendrá el señor de aquel siervo en día que éste no espera, y a la hora que no sabe,
\par 51 y lo castigará duramente, y pondrá su parte con los hipócritas; allí será el lloro y el crujir de dientes.

\chapter{25}

\section*{Parábola de las diez vírgenes}

\par 1 Entonces el reino de los cielos será semejante a diez vírgenes que tomando sus lámparas, salieron a recibir al esposo.
\par 2 Cinco de ellas eran prudentes y cinco insensatas.
\par 3 Las insensatas, tomando sus lámparas, no tomaron consigo aceite;
\par 4 mas las prudentes tomaron aceite en sus vasijas, juntamente con sus lámparas.
\par 5 Y tardándose el esposo, cabecearon todas y se durmieron.
\par 6 Y a la medianoche se oyó un clamor: ¡Aquí viene el esposo; salid a recibirle!
\par 7 Entonces todas aquellas vírgenes se levantaron, y arreglaron sus lámparas.
\par 8 Y las insensatas dijeron a las prudentes: Dadnos de vuestro aceite; porque nuestras lámparas se apagan.
\par 9 Mas las prudentes respondieron diciendo: Para que no nos falte a nosotras y a vosotras, id más bien a los que venden, y comprad para vosotras mismas.
\par 10 Pero mientras ellas iban a comprar, vino el esposo; y las que estaban preparadas entraron con él a las bodas; y se cerró la puerta.
\par 11 Después vinieron también las otras vírgenes, diciendo: ¡Señor, señor, ábrenos!
\par 12 Mas él, respondiendo, dijo: De cierto os digo, que no os conozco.
\par 13 Velad, pues, porque no sabéis el día ni la hora en que el Hijo del Hombre ha de venir.

\section*{Parábola de los talentos}

\par 14 Porque el reino de los cielos es como un hombre que yéndose lejos, llamó a sus siervos y les entregó sus bienes.
\par 15 A uno dio cinco talentos, y a otro dos, y a otro uno, a cada uno conforme a su capacidad; y luego se fue lejos.
\par 16 Y el que había recibido cinco talentos fue y negoció con ellos, y ganó otros cinco talentos
\par 17 Asimismo el que había recibido dos, ganó también otros dos.
\par 18 Pero el que había recibido uno fue y cavó en la tierra, y escondió el dinero de su señor.
\par 19 Después de mucho tiempo vino el señor de aquellos siervos, y arregló cuentas con ellos.
\par 20 Y llegando el que había recibido cinco talentos, trajo otros cinco talentos, diciendo: Señor, cinco talentos me entregaste; aquí tienes, he ganado otros cinco talentos sobre ellos.
\par 21 Y su señor le dijo: Bien, buen siervo y fiel; sobre poco has sido fiel, sobre mucho te pondré; entra en el gozo de tu señor.
\par 22 Llegando también el que había recibido dos talentos, dijo: Señor, dos talentos me entregaste; aquí tienes, he ganado otros dos talentos sobre ellos.
\par 23 Su señor le dijo: Bien, buen siervo y fiel; sobre poco has sido fiel, sobre mucho te pondré; entra en el gozo de tu señor.
\par 24 Pero llegando también el que había recibido un talento, dijo: Señor, te conocía que eres hombre duro, que siegas donde no sembraste y recoges donde no esparciste;
\par 25 por lo cual tuve miedo, y fui y escondí tu talento en la tierra; aquí tienes lo que es tuyo.
\par 26 Respondiendo su señor, le dijo: Siervo malo y negligente, sabías que siego donde no sembré, y que recojo donde no esparcí.
\par 27 Por tanto, debías haber dado mi dinero a los banqueros, y al venir yo, hubiera recibido lo que es mío con los intereses.
\par 28 Quitadle, pues, el talento, y dadlo al que tiene diez talentos.
\par 29 Porque al que tiene, le será dado, y tendrá más; y al que no tiene, aun lo que tiene le será quitado.
\par 30 Y al siervo inútil echadle en las tinieblas de afuera; allí será el lloro y el crujir de dientes.

\section*{El juicio de las naciones}

\par 31 Cuando el Hijo del Hombre venga en su gloria, y todos los santos ángeles con él, entonces se sentará en su trono de gloria,
\par 32 y serán reunidas delante de él todas las naciones; y apartarálos unos de los otros, como aparta el pastor las ovejas de los cabritos.
\par 33 Y pondrá las ovejas a su derecha, y los cabritos a su izquierda.
\par 34 Entonces el Rey dirá a los de su derecha: Venid, benditos de mi Padre, heredad el reino preparado para vosotros desde la fundación del mundo.
\par 35 Porque tuve hambre, y me disteis de comer; tuve sed, y me disteis de beber; fui forastero, y me recogisteis;
\par 36 estuve desnudo, y me cubristeis; enfermo, y me visitasteis; en la cárcel, y vinisteis a mí.
\par 37 Entonces los justos le responderán diciendo: Señor, ¿cuándo te vimos hambriento, y te sustentamos, o sediento, y te dimos de beber?
\par 38 ¿Y cuándo te vimos forastero, y te recogimos, o desnudo, y te cubrimos?
\par 39 ¿O cuándo te vimos enfermo, o en la cárcel, y vinimos a ti?
\par 40 Y respondiendo el Rey, les dirá: De cierto os digo que en cuanto lo hicisteis a uno de estos mis hermanos más pequeños, a mí lo hicisteis.
\par 41 Entonces dirá también a los de la izquierda: Apartaos de mí, malditos, al fuego eterno preparado para el diablo y sus ángeles.
\par 42 Porque tuve hambre, y no me disteis de comer; tuve sed, y no me disteis de beber;
\par 43 fui forastero, y no me recogisteis; estuve desnudo, y no me cubristeis; enfermo, y en la cárcel, y no me visitasteis.
\par 44 Entonces también ellos le responderán diciendo: Señor, ¿cuándo te vimos hambriento, sediento, forastero, desnudo, enfermo, o en la cárcel, y no te servimos?
\par 45 Entonces les responderá diciendo: De cierto os digo que en cuanto no lo hicisteis a uno de estos más pequeños, tampoco a mí lo hicisteis.
\par 46 E irán éstos al castigo eterno, y los justos a la vida eterna.

\chapter{26}

\section*{El complot para prender a Jesús}

\par 1 Cuando hubo acabado Jesús todas estas palabras, dijo a sus discípulos:
\par 2 Sabéis que dentro de dos días se celebra la pascua, y el Hijo del Hombre será entregado para ser crucificado.
\par 3 Entonces los principales sacerdotes, los escribas, y los ancianos del pueblo se reunieron en el patio del sumo sacerdote llamado Caifás,
\par 4 y tuvieron consejo para prender con engaño a Jesús, y matarle.
\par 5 Pero decían: No durante la fiesta, para que no se haga alboroto en el pueblo.

\section*{Jesús es ungido en Betania}

\par 6 Y estando Jesús en Betania, en casa de Simón el leproso,
\par 7 vino a él una mujer, con un vaso de alabastro de perfume de gran precio, y lo derramó sobre la cabeza de él, estando sentado a la mesa.
\par 8 Al ver esto, los discípulos se enojaron, diciendo: ¿Para qué este desperdicio?
\par 9 Porque esto podía haberse vendido a gran precio, y haberse dado a los pobres.
\par 10 Y entendiéndolo Jesús, les dijo: ¿Por qué molestáis a esta mujer? pues ha hecho conmigo una buena obra.
\par 11 Porque siempre tendréis pobres con vosotros, pero a mí no siempre me tendréis.
\par 12 Porque al derramar este perfume sobre mi cuerpo, lo ha hecho a fin de prepararme para la sepultura.
\par 13 De cierto os digo que dondequiera que se predique este evangelio, en todo el mundo, también se contará lo que ésta ha hecho, para memoria de ella.

\section*{Judas ofrece entregar a Jesús}

\par 14 Entonces uno de los doce, que se llamaba Judas Iscariote, fue a los principales sacerdotes,
\par 15 y les dijo: ¿Qué me queréis dar, y yo os lo entregaré? Y ellos le asignaron treinta piezas de plata.
\par 16 Y desde entonces buscaba oportunidad para entregarle.

\section*{Institución de la Cena del Señor}

\par 17 El primer día de la fiesta de los panes sin levadura, vinieron los discípulos a Jesús, diciéndole: ¿Dónde quieres que preparemos para que comas la pascua?
\par 18 Y él dijo: Id a la ciudad a cierto hombre, y decidle: El Maestro dice: Mi tiempo está cerca; en tu casa celebraré la pascua con mis discípulos.
\par 19 Y los discípulos hicieron como Jesús les mandó, y prepararon la pascua.
\par 20 Cuando llegó la noche, se sentó a la mesa con los doce.
\par 21 Y mientras comían, dijo: De cierto os digo, que uno de vosotros me va a entregar.
\par 22 Y entristecidos en gran manera, comenzó cada uno de ellos a decirle: ¿Soy yo, Señor?
\par 23 Entonces él respondiendo, dijo: El que mete la mano conmigo en el plato, ése me va a entregar.
\par 24 A la verdad el Hijo del Hombre va, según está escrito de él, mas ¡ay de aquel hombre por quien el Hijo del Hombre es entregado! Bueno le fuera a ese hombre no haber nacido.
\par 25 Entonces respondiendo Judas, el que le entregaba, dijo: ¿Soy yo, Maestro? Le dijo: Tú lo has dicho.
\par 26 Y mientras comían, tomó Jesús el pan, y bendijo, y lo partió, y dio a sus discípulos, y dijo: Tomad, comed; esto es mi cuerpo.
\par 27 Y tomando la copa, y habiendo dado gracias, les dio, diciendo: Bebed de ella todos;
\par 28 porque esto es mi sangre del nuevo pacto, que por muchos es derramada para remisión de los pecados.
\par 29 Y os digo que desde ahora no beberé más de este fruto de la vid, hasta aquel día en que lo beba nuevo con vosotros en el reino de mi Padre.

\section*{Jesús anuncia la negación de Pedro}

\par 30 Y cuando hubieron cantado el himno, salieron al monte de los Olivos.
\par 31 Entonces Jesús les dijo: Todos vosotros os escandalizaréis de mí esta noche; porque escrito está: Heriré al pastor, y las ovejas del rebaño serán dispersadas.
\par 32 Pero después que haya resucitado, iré delante de vosotros a Galilea.
\par 33 Respondiendo Pedro, le dijo: Aunque todos se escandalicen de ti, yo nunca me escandalizaré.
\par 34 Jesús le dijo: De cierto te digo que esta noche, antes que el gallo cante, me negarás tres veces.
\par 35 Pedro le dijo: Aunque me sea necesario morir contigo, no te negaré. Y todos los discípulos dijeron lo mismo.

\section*{Jesús ora en Getsemaní}

\par 36 Entonces llegó Jesús con ellos a un lugar que se llama Getsemaní, y dijo a sus discípulos: Sentaos aquí, entre tanto que voy allí y oro.
\par 37 Y tomando a Pedro, y a los dos hijos de Zebedeo, comenzó a entristecerse y a angustiarse en gran manera.
\par 38 Entonces Jesús les dijo: Mi alma está muy triste, hasta la muerte; quedaos aquí, y velad conmigo.
\par 39 Yendo un poco adelante, se postró sobre su rostro, orando y diciendo: Padre mío, si es posible, pase de mí esta copa; pero no sea como yo quiero, sino como tú.
\par 40 Vino luego a sus discípulos, y los halló durmiendo, y dijo a Pedro: ¿Así que no habéis podido velar conmigo una hora?
\par 41 Velad y orad, para que no entréis en tentación; el espíritu a la verdad está dispuesto, pero la carne es débil.
\par 42 Otra vez fue, y oró por segunda vez, diciendo: Padre mío, si no puede pasar de mí esta copa sin que yo la beba, hágase tu voluntad.
\par 43 Vino otra vez y los halló durmiendo, porque los ojos de ellos estaban cargados de sueño.
\par 44 Y dejándolos, se fue de nuevo, y oró por tercera vez, diciendo las mismas palabras.
\par 45 Entonces vino a sus discípulos y les dijo: Dormid ya, y descansad. He aquí ha llegado la hora, y el Hijo del Hombre es entregado en manos de pecadores.
\par 46 Levantaos, vamos; ved, se acerca el que me entrega.

\section*{Arresto de Jesús}

\par 47 Mientras todavía hablaba, vino Judas, uno de los doce, y con él mucha gente con espadas y palos, de parte de los principales sacerdotes y de los ancianos del pueblo.
\par 48 Y el que le entregaba les había dado señal, diciendo: Al que yo besare, ése es; prendedle.
\par 49 Y en seguida se acercó a Jesús y dijo: ¡Salve, Maestro! Y le besó.
\par 50 Y Jesús le dijo: Amigo, ¿a qué vienes? Entonces se acercaron y echaron mano a Jesús, y le prendieron.
\par 51 Pero uno de los que estaban con Jesús, extendiendo la mano, sacó su espada, e hiriendo a un siervo del sumo sacerdote, le quitó la oreja.
\par 52 Entonces Jesús le dijo: Vuelve tu espada a su lugar; porque todos los que tomen espada, a espada perecerán.
\par 53 ¿Acaso piensas que no puedo ahora orar a mi Padre, y que él no me daría más de doce legiones de ángeles?
\par 54 ¿Pero cómo entonces se cumplirían las Escrituras, de que es necesario que así se haga?
\par 55 En aquella hora dijo Jesús a la gente:¿Como contra un ladrón habéis salido con espadas y con palos para prenderme? Cada día me sentaba con vosotros enseñando en el templo, y no me prendisteis.
\par 56 Mas todo esto sucede, para que se cumplan las Escrituras de los profetas.Entonces todos los discípulos, dejándole, huyeron.

\section*{Jesús ante el concilio}

\par 57 Los que prendieron a Jesús le llevaron al sumo sacerdote Caifás, adonde estaban reunidos los escribas y los ancianos.
\par 58 Mas Pedro le seguía de lejos hasta el patio del sumo sacerdote; y entrando, se sentó con los alguaciles, para ver el fin.
\par 59 Y los principales sacerdotes y los ancianos y todo el concilio, buscaban falso testimonio contra Jesús, para entregarle a la muerte,
\par 60 y no lo hallaron, aunque muchos testigos falsos se presentaban. Pero al fin vinieron dos testigos falsos,
\par 61 que dijeron: Este dijo: Puedo derribar el templo de Dios, y en tres días reedificarlo.
\par 62 Y levantándose el sumo sacerdote, le dijo: ¿No respondes nada? ¿Qué testifican éstos contra ti?
\par 63 Mas Jesús callaba. Entonces el sumo sacerdote le dijo: Te conjuro por el Dios viviente, que nos digas si eres tú el Cristo, el Hijo de Dios.
\par 64 Jesús le dijo: Tú lo has dicho; y además os digo, que desde ahora veréis al Hijo del Hombre sentado a la diestra del poder de Dios, y viniendo en las nubes del cielo.
\par 65 Entonces el sumo sacerdote rasgó sus vestiduras, diciendo: ¡Ha blasfemado! ¿Qué más necesidad tenemos de testigos? He aquí, ahora mismo habéis oído su blasfemia.
\par 66 ¿Qué os parece? Y respondiendo ellos, dijeron: ¡Es reo de muerte!
\par 67 Entonces le escupieron en el rostro, y le dieron de puñetazos, y otros le abofeteaban,
\par 68 diciendo: Profetízanos, Cristo, quién es el que te golpeó.

\section*{Pedro niega a Jesús}

\par 69 Pedro estaba sentado fuera en el patio; y se le acercó una criada, diciendo: Tú también estabas con Jesús el galileo.
\par 70 Mas él negó delante de todos, diciendo: No sé lo que dices.
\par 71 Saliendo él a la puerta, le vio otra, y dijo a los que estaban allí: También éste estaba con Jesús el nazareno.
\par 72 Pero él negó otra vez con juramento: No conozco al hombre.
\par 73 Un poco después, acercándose los que por allí estaban, dijeron a Pedro: Verdaderamente también tú eres de ellos, porque aun tu manera de hablar te descubre.
\par 74 Entonces él comenzó a maldecir, y a jurar: No conozco al hombre. Y en seguida cantó el gallo.
\par 75 Entonces Pedro se acordó de las palabras de Jesús, que le había dicho:Antes que cante el gallo, me negarás tres veces. Y saliendo fuera, lloró amargamente.

\chapter{27}

\section*{Jesús ante Pilato}

\par 1 Venida la mañana, todos los principales sacerdotes y los ancianos del pueblo entraron en consejo contra Jesús, para entregarle a muerte.
\par 2 Y le llevaron atado, y le entregaron a Poncio Pilato, el gobernador.

\section*{Muerte de Judas}

\par 3 Entonces Judas, el que le había entregado, viendo que era condenado, devolvió arrepentido las treinta piezas de plata a los principales sacerdotes y a los ancianos,
\par 4 diciendo: Yo he pecado entregando sangre inocente. Mas ellos dijeron: ¿Qué nos importa a nosotros? ¡Allá tú!
\par 5 Y arrojando las piezas de plata en el templo, salió, y fue y se ahorcó.
\par 6 Los principales sacerdotes, tomando las piezas de plata, dijeron: No es lícito echarlas en el tesoro de las ofrendas, porque es precio de sangre.
\par 7 Y después de consultar, compraron con ellas el campo del alfarero, para sepultura de los extranjeros.
\par 8 Por lo cual aquel campo se llama hasta el día de hoy: Campo de sangre.
\par 9 Así se cumplió lo dicho por el profeta Jeremías, cuando dijo: Y tomaron las treinta piezas de plata, precio del apreciado, según precio puesto por los hijos de Israel;
\par 10 y las dieron para el campo del alfarero, como me ordenó el Señor.

\section*{Pilato interroga a Jesús}

\par 11 Jesús, pues, estaba en pie delante del gobernador; y éste le preguntó, diciendo: ¿Eres tú el Rey de los judíos? Y Jesús le dijo: Tú lo dices.
\par 12 Y siendo acusado por los principales sacerdotes y por los ancianos, nada respondió.
\par 13 Pilato entonces le dijo: ¿No oyes cuántas cosas testifican contra ti?
\par 14 Pero Jesús no le respondió ni una palabra; de tal manera que el gobernador se maravillaba mucho.

\section*{Jesús sentenciado a muerte}

\par 15 Ahora bien, en el día de la fiesta acostumbraba el gobernador soltar al pueblo un preso, el que quisiesen.
\par 16 Y tenían entonces un preso famoso llamado Barrabás.
\par 17 Reunidos, pues, ellos, les dijo Pilato: ¿A quién queréis que os suelte: a Barrabás, o a Jesús, llamado el Cristo?
\par 18 Porque sabía que por envidia le habían entregado.
\par 19 Y estando él sentado en el tribunal, su mujer le mandó decir: No tengas nada que ver con ese justo; porque hoy he padecido mucho en sueños por causa de él.
\par 20 Pero los principales sacerdotes y los ancianos persuadieron a la multitud que pidiese a Barrabás, y que Jesús fuese muerto.
\par 21 Y respondiendo el gobernador, les dijo: ¿A cuál de los dos queréis que os suelte? Y ellos dijeron: A Barrabás.
\par 22 Pilato les dijo: ¿Qué, pues, haré de Jesús, llamado el Cristo? Todos le dijeron: ¡Sea crucificado!
\par 23 Y el gobernador les dijo: Pues ¿qué mal ha hecho? Pero ellos gritaban aún más, diciendo: ¡Sea crucificado!
\par 24 Viendo Pilato que nada adelantaba, sino que se hacía más alboroto, tomó agua y se lavó las manos delante del pueblo, diciendo: Inocente soy yo de la sangre de este justo; allá vosotros.
\par 25 Y respondiendo todo el pueblo, dijo: Su sangre sea sobre nosotros, y sobre nuestros hijos.
\par 26 Entonces les soltó a Barrabás; y habiendo azotado a Jesús, le entregó para ser crucificado.
\par 27 Entonces los soldados del gobernador llevaron a Jesús al pretorio, y reunieron alrededor de él a toda la compañía;
\par 28 y desnudándole, le echaron encima un manto de escarlata,
\par 29 y pusieron sobre su cabeza una corona tejida de espinas, y una caña en su mano derecha; e hincando la rodilla delante de él, le escarnecían, diciendo: ¡Salve, Rey de los judíos!
\par 30 Y escupiéndole, tomaban la caña y le golpeaban en la cabeza.
\par 31 Después de haberle escarnecido, le quitaron el manto, le pusieron sus vestidos, y le llevaron para crucificarle.

\section*{Crucifixión y muerte de Jesús}

\par 32 Cuando salían, hallaron a un hombre de Cirene que se llamaba Simón; a éste obligaron a que llevase la cruz.
\par 33 Y cuando llegaron a un lugar llamado Gólgota, que significa: Lugar de la Calavera,
\par 34 le dieron a beber vinagre mezclado con hiel; pero después de haberlo probado, no quiso beberlo.
\par 35 Cuando le hubieron crucificado, repartieron entre sí sus vestidos, echando suertes, para que se cumpliese lo dicho por el profeta: Partieron entre sí mis vestidos, y sobre mi ropa echaron suertes.
\par 36 Y sentados le guardaban allí.
\par 37 Y pusieron sobre su cabeza su causa escrita: ESTE ES JESÚS, EL REY DE LOS JUDÍOS.
\par 38 Entonces crucificaron con él a dos ladrones, uno a la derecha, y otro a la izquierda.
\par 39 Y los que pasaban le injuriaban, meneando la cabeza,
\par 40 y diciendo: Tú que derribas el templo, y en tres días lo reedificas, sálvate a ti mismo; si eres Hijo de Dios, desciende de la cruz.
\par 41 De esta manera también los principales sacerdotes, escarneciéndole con los escribas y los fariseos y los ancianos, decían:
\par 42 A otros salvó, a sí mismo no se puede salvar; si es el Rey de Israel, descienda ahora de la cruz, y creeremos en él.
\par 43 Confió en Dios; líbrele ahora si le quiere; porque ha dicho: Soy Hijo de Dios.
\par 44 Lo mismo le injuriaban también los ladrones que estaban crucificados con él.
\par 45 Y desde la hora sexta hubo tinieblas sobre toda la tierra hasta la hora novena.
\par 46 Cerca de la hora novena, Jesús clamó a gran voz, diciendo: Elí, Elí, ¿lama sabactani? Esto es: Dios mío, Dios mío, ¿por qué me has desamparado?
\par 47 Algunos de los que estaban allí decían, al oírlo: A Elías llama éste.
\par 48 Y al instante, corriendo uno de ellos, tomó una esponja, y la empapó de vinagre, y poniéndola en una caña, le dio a beber.
\par 49 Pero los otros decían: Deja, veamos si viene Elías a librarle.
\par 50 Mas Jesús, habiendo otra vez clamado a gran voz, entregó el espíritu.
\par 51 Y he aquí, el velo del templo se rasgó en dos, de arriba abajo; y la tierra tembló, y las rocas se partieron;
\par 52 y se abrieron los sepulcros, y muchos cuerpos de santos que habían dormido, se levantaron;
\par 53 y saliendo de los sepulcros, después de la resurrección de él, vinieron a la santa ciudad, y aparecieron a muchos.
\par 54 El centurión, y los que estaban con él guardando a Jesús, visto el terremoto, y las cosas que habían sido hechas, temieron en gran manera, y dijeron: Verdaderamente éste era Hijo de Dios.
\par 55 Estaban allí muchas mujeres mirando de lejos, las cuales habían seguido a Jesús desde Galilea, sirviéndole,
\par 56 entre las cuales estaban María Magdalena, María la madre de Jacobo y de José, y la madre de los hijos de Zebedeo.

\section*{Jesús es sepultado}

\par 57 Cuando llegó la noche, vino un hombre rico de Arimatea, llamado José, que también había sido discípulo de Jesús.
\par 58 Este fue a Pilato y pidió el cuerpo de Jesús. Entonces Pilato mandó que se le diese el cuerpo.
\par 59 Y tomando José el cuerpo, lo envolvió en una sábana limpia,
\par 60 y lo puso en su sepulcro nuevo, que había labrado en la peña; y después de hacer rodar una gran piedra a la entrada del sepulcro, se fue.
\par 61 Y estaban allí María Magdalena, y la otra María, sentadas delante del sepulcro.

\section*{La guardia ante la tumba}

\par 62 Al día siguiente, que es después de la preparación, se reunieron los principales sacerdotes y los fariseos ante Pilato,
\par 63 diciendo: Señor, nos acordamos que aquel engañador dijo, viviendo aún: Después de tres días resucitaré.
\par 64 Manda, pues, que se asegure el sepulcro hasta el tercer día, no sea que vengan sus discípulos de noche, y lo hurten, y digan al pueblo: Resucitó de entre los muertos. Y será el postrer error peor que el primero.
\par 65 Y Pilato les dijo: Ahí tenéis una guardia; id, aseguradlo como sabéis.
\par 66 Entonces ellos fueron y aseguraron el sepulcro, sellando la piedra y poniendo la guardia.

\chapter{28}

\section*{La resurrección}

\par 1 Pasado el día de reposo, al amanecer del primer día de la semana, vinieron María Magdalena y la otra María, a ver el sepulcro.
\par 2 Y hubo un gran terremoto; porque un ángel del Señor, descendiendo del cielo y llegando, removió la piedra, y se sentó sobre ella.
\par 3 Su aspecto era como un relámpago, y su vestido blanco como la nieve.
\par 4 Y de miedo de él los guardas temblaron y se quedaron como muertos.
\par 5 Mas el ángel, respondiendo, dijo a las mujeres: No temáis vosotras; porque yo sé que buscáis a Jesús, el que fue crucificado.
\par 6 No está aquí, pues ha resucitado, como dijo. Venid, ved el lugar donde fue puesto el Señor.
\par 7 E id pronto y decid a sus discípulos que ha resucitado de los muertos, y he aquí va delante de vosotros a Galilea; allí le veréis. He aquí, os lo he dicho.
\par 8 Entonces ellas, saliendo del sepulcro con temor y gran gozo, fueron corriendo a dar las nuevas a sus discípulos. Y mientras iban a dar las nuevas a los discípulos,
\par 9 he aquí, Jesús les salió al encuentro, diciendo: ¡Salve! Y ellas, acercándose, abrazaron sus pies, y le adoraron.
\par 10 Entonces Jesús les dijo: No temáis; id, dad las nuevas a mis hermanos, para que vayan a Galilea, y allí me verán.

\section*{El informe de la guardia}

\par 11 Mientras ellas iban, he aquí unos de la guardia fueron a la ciudad, y dieron aviso a los principales sacerdotes de todas las cosas que habían acontecido.
\par 12 Y reunidos con los ancianos, y habido consejo, dieron mucho dinero a los soldados,
\par 13 diciendo: Decid vosotros: Sus discípulos vinieron de noche, y lo hurtaron, estando nosotros dormidos.
\par 14 Y si esto lo oyere el gobernador, nosotros le persuadiremos, y os pondremos a salvo.
\par 15 Y ellos, tomando el dinero, hicieron como se les había instruido. Este dicho se ha divulgado entre los judíos hasta el día de hoy.

\section*{La gran comisión}

\par 16 Pero los once discípulos se fueron a Galilea, al monte donde Jesús les había ordenado.
\par 17 Y cuando le vieron, le adoraron; pero algunos dudaban.
\par 18 Y Jesús se acercó y les habló diciendo: Toda potestad me es dada en el cielo y en la tierra.
\par 19 Por tanto, id, y haced discípulos a todas las naciones, bautizándolos en el nombre del Padre, y del Hijo, y del Espíritu Santo;
\par 20 enseñándoles que guarden todas las cosas que os he mandado; y he aquí yo estoy con vosotros todos los días, hasta el fin del mundo. Amén.

\end{document}
