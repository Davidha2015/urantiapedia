\begin{document}
%\title{La Epístola del Apóstol San Pablo a los GÁLATAS}
\title{Epístola a los Gálatas}

\chapter{1}

\section*{Salutación}

\par 1 Pablo, apóstol (no de hombres ni por hombre, sino por Jesucristo y por Dios el Padre que lo resucitó de los muertos),
\par 2 y todos los hermanos que están conmigo, a las iglesias de Galacia:
\par 3 Gracia y paz sean a vosotros, de Dios el Padre y de nuestro Señor Jesucristo,
\par 4 el cual se dio a sí mismo por nuestros pecados para librarnos del presente siglo malo, conforme a la voluntad de nuestro Dios y Padre,
\par 5 a quien sea la gloria por los siglos de los siglos. Amén.

\section*{No hay otro evangelio}

\par 6 Estoy maravillado de que tan pronto os hayáis alejado del que os llamó por la gracia de Cristo, para seguir un evangelio diferente.
\par 7 No que haya otro, sino que hay algunos que os perturban y quieren pervertir el evangelio de Cristo.
\par 8 Mas si aun nosotros, o un ángel del cielo, os anunciare otro evangelio diferente del que os hemos anunciado, sea anatema.
\par 9 Como antes hemos dicho, también ahora lo repito: Si alguno os predica diferente evangelio del que habéis recibido, sea anatema.
\par 10 Pues, ¿busco ahora el favor de los hombres, o el de Dios? ¿O trato de agradar a los hombres? Pues si todavía agradara a los hombres, no sería siervo de Cristo.

\section*{El ministerio de Pablo}

\par 11 Mas os hago saber, hermanos, que el evangelio anunciado por mí, no es según hombre;
\par 12 pues yo ni lo recibí ni lo aprendí de hombre alguno, sino por revelación de Jesucristo.
\par 13 Porque ya habéis oído acerca de mi conducta en otro tiempo en el judaísmo, que perseguía sobremanera a la iglesia de Dios, y la asolaba;
\par 14 y en el judaísmo aventajaba a muchos de mis contemporáneos en mi nación, siendo mucho más celoso de las tradiciones de mis padres.
\par 15 Pero cuando agradó a Dios, que me apartó desde el vientre de mi madre, y me llamó por su gracia,
\par 16 revelar a su Hijo en mí, para que yo le predicase entre los gentiles, no consulté en seguida con carne y sangre,
\par 17 ni subí a Jerusalén a los que eran apóstoles antes que yo; sino que fui a Arabia, y volví de nuevo a Damasco.
\par 18 Después, pasados tres años, subí a Jerusalén para ver a Pedro, y permanecí con él quince días;
\par 19 pero no vi a ningún otro de los apóstoles, sino a Jacobo el hermano del Señor.
\par 20 En esto que os escribo, he aquí delante de Dios que no miento.
\par 21 Después fui a las regiones de Siria y de Cilicia,
\par 22 y no era conocido de vista a las iglesias de Judea, que eran en Cristo;
\par 23 solamente oían decir: Aquel que en otro tiempo nos perseguía, ahora predica la fe que en otro tiempo asolaba.
\par 24 Y glorificaban a Dios en mí.

\chapter{2}

\par 1 Después, pasados catorce años, subí otra vez a Jerusalén con Bernabé, llevando también conmigo a Tito.
\par 2 Pero subí según una revelación, y para no correr o haber corrido en vano, expuse en privado a los que tenían cierta reputación el evangelio que predico entre los gentiles.
\par 3 Mas ni aun Tito, que estaba conmigo, con todo y ser griego, fue obligado a circuncidarse;
\par 4 y esto a pesar de los falsos hermanos introducidos a escondidas, que entraban para espiar nuestra libertad que tenemos en Cristo Jesús, para reducirnos a esclavitud,
\par 5 a los cuales ni por un momento accedimos a someternos, para que la verdad del evangelio permaneciese con vosotros.
\par 6 Pero de los que tenían reputación de ser algo (lo que hayan sido en otro tiempo nada me importa; Dios no hace acepción de personas), a mí, pues, los de reputación nada nuevo me comunicaron.
\par 7 Antes por el contrario, como vieron que me había sido encomendado el evangelio de la incircuncisión, como a Pedro el de la circuncisión
\par 8 (pues el que actuó en Pedro para el apostolado de la circuncisión, actuó también en mí para con los gentiles),
\par 9 y reconociendo la gracia que me había sido dada, Jacobo, Cefas y Juan, que eran considerados como columnas, nos dieron a mí y a Bernabé la diestra en señal de compañerismo, para que nosotros fuésemos a los gentiles, y ellos a la circuncisión.
\par 10 Solamente nos pidieron que nos acordásemos de los pobres; lo cual también procuré con diligencia hacer.

\section*{Pablo reprende a Pedro en Antioquía}

\par 11 Pero cuando Pedro vino a Antioquía, le resistí cara a cara, porque era de condenar.
\par 12 Pues antes que viniesen algunos de parte de Jacobo, comía con los gentiles; pero después que vinieron, se retraía y se apartaba, porque tenía miedo de los de la circuncisión.
\par 13 Y en su simulación participaban también los otros judíos, de tal manera que aun Bernabé fue también arrastrado por la hipocresía de ellos.
\par 14 Pero cuando vi que no andaban rectamente conforme a la verdad del evangelio, dije a Pedro delante de todos: Si tú, siendo judío, vives como los gentiles y no como judío, ¿por qué obligas a los gentiles a judaizar?
\par 15 Nosotros, judíos de nacimiento, y no pecadores de entre los gentiles,
\par 16 sabiendo que el hombre no es justificado por las obras de la ley, sino por la fe de Jesucristo, nosotros también hemos creído en Jesucristo, para ser justificados por la fe de Cristo y no por las obras de la ley, por cuanto por las obras de la ley nadie será justificado.
\par 17 Y si buscando ser justificados en Cristo, también nosotros somos hallados pecadores, ¿es por eso Cristo ministro de pecado? En ninguna manera.
\par 18 Porque si las cosas que destruí, las mismas vuelvo a edificar, transgresor me hago.
\par 19 Porque yo por la ley soy muerto para la ley, a fin de vivir para Dios.
\par 20 Con Cristo estoy juntamente crucificado, y ya no vivo yo, mas vive Cristo en mí; y lo que ahora vivo en la carne, lo vivo en la fe del Hijo de Dios, el cual me amó y se entregó a sí mismo por mí.
\par 21 No desecho la gracia de Dios; pues si por la ley fuese la justicia, entonces por demás murió Cristo.

\chapter{3}

\section*{El Espíritu se recibe por la fe}

\par 1 ¡Oh gálatas insensatos! ¿quién os fascinó para no obedecer a la verdad, a vosotros ante cuyos ojos Jesucristo fue ya presentado claramente entre vosotros como crucificado?
\par 2 Esto solo quiero saber de vosotros: ¿Recibisteis el Espíritu por las obras de la ley, o por el oír con fe?
\par 3 ¿Tan necios sois? ¿Habiendo comenzado por el Espíritu, ahora vais a acabar por la carne?
\par 4 ¿Tantas cosas habéis padecido en vano? si es que realmente fue en vano.
\par 5 Aquel, pues, que os suministra el Espíritu, y hace maravillas entre vosotros, ¿lo hace por las obras de la ley, o por el oír con fe?

\section*{El pacto de Dios con Abraham}

\par 6 Así Abraham creyó a Dios, y le fue contado por justicia.
\par 7 Sabed, por tanto, que los que son de fe, éstos son hijos de Abraham.
\par 8 Y la Escritura, previendo que Dios había de justificar por la fe a los gentiles, dio de antemano la buena nueva a Abraham, diciendo: En ti serán benditas todas las naciones.
\par 9 De modo que los de la fe son bendecidos con el creyente Abraham.
\par 10 Porque todos los que dependen de las obras de la ley están bajo maldición, pues escrito está: Maldito todo aquel que no permaneciere en todas las cosas escritas en el libro de la ley, para hacerlas.
\par 11 Y que por la ley ninguno se justifica para con Dios, es evidente, porque: El justo por la fe vivirá;
\par 12 y la ley no es de fe, sino que dice: El que hiciere estas cosas vivirá por ellas.
\par 13 Cristo nos redimió de la maldición de la ley, hecho por nosotros maldición (porque está escrito: Maldito todo el que es colgado en un madero),
\par 14 para que en Cristo Jesús la bendición de Abraham alcanzase a los gentiles, a fin de que por la fe recibiésemos la promesa del Espíritu.
\par 15 Hermanos, hablo en términos humanos: Un pacto, aunque sea de hombre, una vez ratificado, nadie lo invalida, ni le añade.
\par 16 Ahora bien, a Abraham fueron hechas las promesas, y a su simiente. No dice: Y a las simientes, como si hablase de muchos, sino como de uno: Y a tu simiente, la cual es Cristo.
\par 17 Esto, pues, digo: El pacto previamente ratificado por Dios para con Cristo, la ley que vino cuatrocientos treinta años después, no lo abroga, para invalidar la promesa.
\par 18 Porque si la herencia es por la ley, ya no es por la promesa; pero Dios la concedió a Abraham mediante la promesa.

\section*{El propósito de la ley}

\par 19 Entonces, ¿para qué sirve la ley? Fue añadida a causa de las transgresiones, hasta que viniese la simiente a quien fue hecha la promesa; y fue ordenada por medio de ángeles en mano de un mediador.
\par 20 Y el mediador no lo es de uno solo; pero Dios es uno.
\par 21 ¿Luego la ley es contraria a las promesas de Dios? En ninguna manera; porque si la ley dada pudiera vivificar, la justicia fuera verdaderamente por la ley.
\par 22 Mas la Escritura lo encerró todo bajo pecado, para que la promesa que es por la fe en Jesucristo fuese dada a los creyentes.
\par 23 Pero antes que viniese la fe, estábamos confinados bajo la ley, encerrados para aquella fe que iba a ser revelada.
\par 24 De manera que la ley ha sido nuestro ayo, para llevarnos a Cristo, a fin de que fuésemos justificados por la fe.
\par 25 Pero venida la fe, ya no estamos bajo ayo,
\par 26 pues todos sois hijos de Dios por la fe en Cristo Jesús;
\par 27 porque todos los que habéis sido bautizados en Cristo, de Cristo estáis revestidos.
\par 28 Ya no hay judío ni griego; no hay esclavo ni libre; no hay varón ni mujer; porque todos vosotros sois uno en Cristo Jesús.
\par 29 Y si vosotros sois de Cristo, ciertamente linaje de Abraham sois, y herederos según la promesa.

\chapter{4}

\par 1 Pero también digo: Entre tanto que el heredero es niño, en nada difiere del esclavo, aunque es señor de todo;
\par 2 sino que está bajo tutores y curadores hasta el tiempo señalado por el padre.
\par 3 Así también nosotros, cuando éramos niños, estábamos en esclavitud bajo los rudimentos del mundo.
\par 4 Pero cuando vino el cumplimiento del tiempo, Dios envió a su Hijo, nacido de mujer y nacido bajo la ley,
\par 5 para que redimiese a los que estaban bajo la ley, a fin de que recibiésemos la adopción de hijos.
\par 6 Y por cuanto sois hijos, Dios envió a vuestros corazones el Espíritu de su Hijo, el cual clama: ¡Abba, Padre!
\par 7 Así que ya no eres esclavo, sino hijo; y si hijo, también heredero de Dios por medio de Cristo.

\section*{Exhortación contra el volver a la esclavitud}

\par 8 Ciertamente, en otro tiempo, no conociendo a Dios, servíais a los que por naturaleza no son dioses;
\par 9 mas ahora, conociendo a Dios, o más bien, siendo conocidos por Dios, ¿cómo es que os volvéis de nuevo a los débiles y pobres rudimentos, a los cuales os queréis volver a esclavizar?
\par 10 Guardáis los días, los meses, los tiempos y los años.
\par 11 Me temo de vosotros, que haya trabajado en vano con vosotros.
\par 12 Os ruego, hermanos, que os hagáis como yo, porque yo también me hice como vosotros. Ningún agravio me habéis hecho.
\par 13 Pues vosotros sabéis que a causa de una enfermedad del cuerpo os anuncié el evangelio al principio;
\par 14 y no me despreciasteis ni desechasteis por la prueba que tenía en mi cuerpo, antes bien me recibisteis como a un ángel de Dios, como a Cristo Jesús.
\par 15 ¿Dónde, pues, está esa satisfacción que experimentabais? Porque os doy testimonio de que si hubieseis podido, os hubierais sacado vuestros propios ojos para dármelos.
\par 16 ¿Me he hecho, pues, vuestro enemigo, por deciros la verdad?
\par 17 Tienen celo por vosotros, pero no para bien, sino que quieren apartaros de nosotros para que vosotros tengáis celo por ellos.
\par 18 Bueno es mostrar celo en lo bueno siempre, y no solamente cuando estoy presente con vosotros.
\par 19 Hijitos míos, por quienes vuelvo a sufrir dolores de parto, hasta que Cristo sea formado en vosotros,
\par 20 quisiera estar con vosotros ahora mismo y cambiar de tono, pues estoy perplejo en cuanto a vosotros.

\section*{Alegoría de Sara y Agar}

\par 21 Decidme, los que queréis estar bajo la ley: ¿no habéis oído la ley?
\par 22 Porque está escrito que Abraham tuvo dos hijos; uno de la esclava, el otro de la libre.
\par 23 Pero el de la esclava nació según la carne; mas el de la libre, por la promesa.
\par 24 Lo cual es una alegoría, pues estas mujeres son los dos pactos; el uno proviene del monte Sinaí, el cual da hijos para esclavitud; éste es Agar.
\par 25 Porque Agar es el monte Sinaí en Arabia, y corresponde a la Jerusalén actual, pues ésta, junto con sus hijos, está en esclavitud.
\par 26 Mas la Jerusalén de arriba, la cual es madre de todos nosotros, es libre.
\par 27 Porque está escrito:
\par Regocíjate, oh estéril, tú que no das a luz;
\par Prorrumpe en júbilo y clama, tú que no tienes dolores de parto;
\par Porque más son los hijos de las desolada, que de la que tiene marido.
\par 28 Así que, hermanos, nosotros, como Isaac, somos hijos de la promesa.
\par 29 Pero como entonces el que había nacido según la carne perseguía al que había nacido según el Espíritu, así también ahora.
\par 30 Mas ¿qué dice la Escritura? Echa fuera a la esclava y a su hijo, porque no heredará el hijo de la esclava con el hijo de la libre.
\par 31 De manera, hermanos, que no somos hijos de la esclava, sino de la libre.

\chapter{5}

\section*{Estad firmes en la libertad}

\par 1 Estad, pues, firmes en la libertad con que Cristo nos hizo libres, y no estéis otra vez sujetos al yugo de esclavitud.
\par 2 He aquí, yo Pablo os digo que si os circuncidáis, de nada os aprovechará Cristo.
\par 3 Y otra vez testifico a todo hombre que se circuncida, que está obligado a guardar toda la ley.
\par 4 De Cristo os desligasteis, los que por la ley os justificáis; de la gracia habéis caído.
\par 5 Pues nosotros por el Espíritu aguardamos por fe la esperanza de la justicia;
\par 6 porque en Cristo Jesús ni la circuncisión vale algo, ni la incircuncisión, sino la fe que obra por el amor.
\par 7 Vosotros corríais bien; ¿quién os estorbó para no obedecer a la verdad?
\par 8 Esta persuasión no procede de aquel que os llama.
\par 9 Un poco de levadura leuda toda la masa.
\par 10 Yo confío respecto de vosotros en el Señor, que no pensaréis de otro modo; mas el que os perturba llevará la sentencia, quienquiera que sea.
\par 11 Y yo, hermanos, si aún predico la circuncisión, ¿por qué padezco persecución todavía? En tal caso se ha quitado el tropiezo de la cruz.
\par 12 ¡Ojalá se mutilasen los que os perturban!
\par 13 Porque vosotros, hermanos, a libertad fuisteis llamados; solamente que no uséis la libertad como ocasión para la carne, sino servíos por amor los unos a los otros.
\par 14 Porque toda la ley en esta sola palabra se cumple: Amarás a tu prójimo como a ti mismo.
\par 15 Pero si os mordéis y os coméis unos a otros, mirad que también no os consumáis unos a otros.

\section*{Las obras de la carne y el fruto del Espíritu}

\par 16 Digo, pues: Andad en el Espíritu, y no satisfagáis los deseos de la carne.
\par 17 Porque el deseo de la carne es contra el Espíritu, y el del Espíritu es contra la carne; y éstos se oponen entre sí, para que no hagáis lo que quisiereis.
\par 18 Pero si sois guiados por el Espíritu, no estáis bajo la ley.
\par 19 Y manifiestas son las obras de la carne, que son: adulterio, fornicación, inmundicia, lascivia,
\par 20 idolatría, hechicerías, enemistades, pleitos, celos, iras, contiendas, disensiones, herejías,
\par 21 envidias, homicidios, borracheras, orgías, y cosas semejantes a estas; acerca de las cuales os amonesto, como ya os lo he dicho antes, que los que practican tales cosas no heredarán el reino de Dios.
\par 22 Mas el fruto del Espíritu es amor, gozo, paz, paciencia, benignidad, bondad, fe,
\par 23 mansedumbre, templanza; contra tales cosas no hay ley.
\par 24 Pero los que son de Cristo han crucificado la carne con sus pasiones y deseos.
\par 25 Si vivimos por el Espíritu, andemos también por el Espíritu.
\par 26 No nos hagamos vanagloriosos, irritándonos unos a otros, envidiándonos unos a otros.

\chapter{6}

\par 1 Hermanos, si alguno fuere sorprendido en alguna falta, vosotros que sois espirituales, restauradle con espíritu de mansedumbre, considerándote a ti mismo, no sea que tú también seas tentado.
\par 2 Sobrellevad los unos las cargas de los otros, y cumplid así la ley de Cristo.
\par 3 Porque el que se cree ser algo, no siendo nada, a sí mismo se engaña.
\par 4 Así que, cada uno someta a prueba su propia obra, y entonces tendrá motivo de gloriarse sólo respecto de sí mismo, y no en otro;
\par 5 porque cada uno llevará su propia carga.
\par 6 El que es enseñado en la palabra, haga partícipe de toda cosa buena al que lo instruye.
\par 7 No os engañéis; Dios no puede ser burlado: pues todo lo que el hombre sembrare, eso también segará.
\par 8 Porque el que siembra para su carne, de la carne segará corrupción; mas el que siembra para el Espíritu, del Espíritu segará vida eterna.
\par 9 No nos cansemos, pues, de hacer bien; porque a su tiempo segaremos, si no desmayamos.
\par 10 Así que, según tengamos oportunidad, hagamos bien a todos, y mayormente a los de la familia de la fe.

\section*{Pablo se gloría en la cruz de Cristo}

\par 11 Mirad con cuán grandes letras os escribo de mi propia mano.
\par 12 Todos los que quieren agradar en la carne, éstos os obligan a que os circuncidéis, solamente para no padecer persecución a causa de la cruz de Cristo.
\par 13 Porque ni aun los mismos que se circuncidan guardan la ley; pero quieren que vosotros os circuncidéis, para gloriarse en vuestra carne.
\par 14 Pero lejos esté de mí gloriarme, sino en la cruz de nuestro Señor Jesucristo, por quien el mundo me es crucificado a mí, y yo al mundo.
\par 15 Porque en Cristo Jesús ni la circuncisión vale nada, ni la incircuncisión, sino una nueva creación.
\par 16 Y a todos los que anden conforme a esta regla, paz y misericordia sea a ellos, y al Israel de Dios.
\par 17 De aquí en adelante nadie me cause molestias; porque yo traigo en mi cuerpo las marcas del Señor Jesús.

\section*{Bendición final}

\par 18 Hermanos, la gracia de nuestro Señor Jesucristo sea con vuestro espíritu. Amén.

\end{document}