\begin{document}
\title{Epístola a los filipenses}

\chapter{1}

\section*{Salutación}

\par 1 Pablo y Timoteo, siervos de Jesucristo, a todos los santos en Cristo Jesús que están en Filipos, con los obispos y diáconos:
\par 2 Gracia y paz a vosotros, de Dios nuestro Padre y del Señor Jesucristo.

\section*{Oración de Pablo por los creyentes}

\par 3 Doy gracias a mi Dios siempre que me acuerdo de vosotros,
\par 4 siempre en todas mis oraciones rogando con gozo por todos vosotros,
\par 5 por vuestra comunión en el evangelio, desde el primer día hasta ahora;
\par 6 estando persuadido de esto, que el que comenzó en vosotros la buena obra, la perfeccionará hasta el día de Jesucristo;
\par 7 como me es justo sentir esto de todos vosotros, por cuanto os tengo en el corazón; y en mis prisiones, y en la defensa y confirmación del evangelio, todos vosotros sois participantes conmigo de la gracia.
\par 8 Porque Dios me es testigo de cómo os amo a todos vosotros con el entrañable amor de Jesucristo.
\par 9 Y esto pido en oración, que vuestro amor abunde aun más y más en ciencia y en todo conocimiento,
\par 10 para que aprobéis lo mejor, a fin de que seáis sinceros e irreprensibles para el día de Cristo,
\par 11 llenos de frutos de justicia que son por medio de Jesucristo, para gloria y alabanza de Dios.

\section*{Para mí el vivir es Cristo}

\par 12 Quiero que sepáis, hermanos, que las cosas que me han sucedido, han redundado más bien para el progreso del evangelio,
\par 13 de tal manera que mis prisiones se han hecho patentes en Cristo en todo el pretorio, y a todos los demás.
\par 14 Y la mayoría de los hermanos, cobrando ánimo en el Señor con mis prisiones, se atreven mucho más a hablar la palabra sin temor.
\par 15 Algunos, a la verdad, predican a Cristo por envidia y contienda; pero otros de buena voluntad.
\par 16 Los unos anuncian a Cristo por contención, no sinceramente, pensando añadir aflicción a mis prisiones;
\par 17 pero los otros por amor, sabiendo que estoy puesto para la defensa del evangelio.
\par 18 ¿Qué, pues? Que no obstante, de todas maneras, o por pretexto o por verdad, Cristo es anunciado; y en esto me gozo, y me gozaré aún.
\par 19 Porque sé que por vuestra oración y la suministración del Espíritu de Jesucristo, esto resultará en mi liberación,
\par 20 conforme a mi anhelo y esperanza de que en nada seré avergonzado; antes bien con toda confianza, como siempre, ahora también será magnificado Cristo en mi cuerpo, o por vida o por muerte.
\par 21 Porque para mí el vivir es Cristo, y el morir es ganancia.
\par 22 Mas si el vivir en la carne resulta para mí en beneficio de la obra, no sé entonces qué escoger.
\par 23 Porque de ambas cosas estoy puesto en estrecho, teniendo deseo de partir y estar con Cristo, lo cual es muchísimo mejor;
\par 24 pero quedar en la carne es más necesario por causa de vosotros.
\par 25 Y confiado en esto, sé que quedaré, que aún permaneceré con todos vosotros, para vuestro provecho y gozo de la fe,
\par 26 para que abunde vuestra gloria de mí en Cristo Jesús por mi presencia otra vez entre vosotros.
\par 27 Solamente que os comportéis como es digno del evangelio de Cristo, para que o sea que vaya a veros, o que esté ausente, oiga de vosotros que estáis firmes en un mismo espíritu, combatiendo unánimes por la fe del evangelio,
\par 28 y en nada intimidados por los que se oponen, que para ellos ciertamente es indicio de perdición, mas para vosotros de salvación; y esto de Dios.
\par 29 Porque a vosotros os es concedido a causa de Cristo, no sólo que creáis en él, sino también que padezcáis por él,
\par 30 teniendo el mismo conflicto que habéis visto en mí, y ahora oís que hay en mí.

\chapter{2}

\section*{Humillación y exaltación de Cristo}

\par 1 Por tanto, si hay alguna consolación en Cristo, si algún consuelo de amor, si alguna comunión del Espíritu, si algún afecto entrañable, si alguna misericordia,
\par 2 completad mi gozo, sintiendo lo mismo, teniendo el mismo amor, unánimes, sintiendo una misma cosa.
\par 3 Nada hagáis por contienda o por vanagloria; antes bien con humildad, estimando cada uno a los demás como superiores a él mismo;
\par 4 no mirando cada uno por lo suyo propio, sino cada cual también por lo de los otros.
\par 5 Haya, pues, en vosotros este sentir que hubo también en Cristo Jesús,
\par 6 el cual, siendo en forma de Dios, no estimó el ser igual a Dios como cosa a que aferrarse,
\par 7 sino que se despojó a sí mismo, tomando forma de siervo, hecho semejante a los hombres;
\par 8 y estando en la condición de hombre, se humilló a sí mismo, haciéndose obediente hasta la muerte, y muerte de cruz.
\par 9 Por lo cual Dios también le exaltó hasta lo sumo, y le dio un nombre que es sobre todo nombre,
\par 10 para que en el nombre de Jesús se doble toda rodilla de los que están en los cielos, y en la tierra, y debajo de la tierra;
\par 11 y toda lengua confiese que Jesucristo es el Señor, para gloria de Dios Padre.

\section*{Luminares en el mundo}

\par 12 Por tanto, amados míos, como siempre habéis obedecido, no como en mi presencia solamente, sino mucho más ahora en mi ausencia, ocupaos en vuestra salvación con temor y temblor,
\par 13 porque Dios es el que en vosotros produce así el querer como el hacer, por su buena voluntad.
\par 14 Haced todo sin murmuraciones y contiendas,
\par 15 para que seáis irreprensibles y sencillos, hijos de Dios sin mancha en medio de una generación maligna y perversa, en medio de la cual resplandecéis como luminares en el mundo;
\par 16 asidos de la palabra de vida, para que en el día de Cristo yo pueda gloriarme de que no he corrido en vano, ni en vano he trabajado.
\par 17 Y aunque sea derramado en libación sobre el sacrificio y servicio de vuestra fe, me gozo y regocijo con todos vosotros.
\par 18 Y asimismo gozaos y regocijaos también vosotros conmigo.

\section*{Timoteo y Epafrodito}

\par 19 Espero en el Señor Jesús enviaros pronto a Timoteo, para que yo también esté de buen ánimo al saber de vuestro estado;
\par 20 pues a ninguno tengo del mismo ánimo, y que tan sinceramente se interese por vosotros.
\par 21 Porque todos buscan lo suyo propio, no lo que es de Cristo Jesús.
\par 22 Pero ya conocéis los méritos de él, que como hijo a padre ha servido conmigo en el evangelio.
\par 23 Así que a éste espero enviaros, luego que yo vea cómo van mis asuntos;
\par 24 y confío en el Señor que yo también iré pronto a vosotros.
\par 25 Mas tuve por necesario enviaros a Epafrodito, mi hermano y colaborador y compañero de milicia, vuestro mensajero, y ministrador de mis necesidades;
\par 26 porque él tenía gran deseo de veros a todos vosotros, y gravemente se angustió porque habíais oído que había enfermado.
\par 27 Pues en verdad estuvo enfermo, a punto de morir; pero Dios tuvo misericordia de él, y no solamente de él, sino también de mí, para que yo no tuviese tristeza sobre tristeza.
\par 28 Así que le envío con mayor solicitud, para que al verle de nuevo, os gocéis, y yo esté con menos tristeza.
\par 29 Recibidle, pues, en el Señor, con todo gozo, y tened en estima a los que son como él;
\par 30 porque por la obra de Cristo estuvo próximo a la muerte, exponiendo su vida para suplir lo que faltaba en vuestro servicio por mí.

\chapter{3}

\section*{Prosigo al blanco}

\par 1 Por lo demás, hermanos, gozaos en el Señor. A mí no me es molesto el escribiros las mismas cosas, y para vosotros es seguro.
\par 2 Guardaos de los perros, guardaos de los malos obreros, guardaos de los mutiladores del cuerpo.
\par 3 Porque nosotros somos la circuncisión, los que en espíritu servimos a Dios y nos gloriamos en Cristo Jesús, no teniendo confianza en la carne.
\par 4 Aunque yo tengo también de qué confiar en la carne. Si alguno piensa que tiene de qué confiar en la carne, yo más:
\par 5 circuncidado al octavo día, del linaje de Israel, de la tribu de Benjamín, hebreo de hebreos; en cuanto a la ley, fariseo;
\par 6 en cuanto a celo, perseguidor de la iglesia; en cuanto a la justicia que es en la ley, irreprensible.
\par 7 Pero cuantas cosas eran para mí ganancia, las he estimado como pérdida por amor de Cristo.
\par 8 Y ciertamente, aun estimo todas las cosas como pérdida por la excelencia del conocimiento de Cristo Jesús, mi Señor, por amor del cual lo he perdido todo, y lo tengo por basura, para ganar a Cristo,
\par 9 y ser hallado en él, no teniendo mi propia justicia, que es por la ley, sino la que es por la fe de Cristo, la justicia que es de Dios por la fe;
\par 10 a fin de conocerle, y el poder de su resurrección, y la participación de sus padecimientos, llegando a ser semejante a él en su muerte,
\par 11 si en alguna manera llegase a la resurrección de entre los muertos.
\par 12 No que lo haya alcanzado ya, ni que ya sea perfecto; sino que prosigo, por ver si logro asir aquello para lo cual fui también asido por Cristo Jesús.
\par 13 Hermanos, yo mismo no pretendo haberlo ya alcanzado; pero una cosa hago: olvidando ciertamente lo que queda atrás, y extendiéndome a lo que está delante,
\par 14 prosigo a la meta, al premio del supremo llamamiento de Dios en Cristo Jesús.
\par 15 Así que, todos los que somos perfectos, esto mismo sintamos; y si otra cosa sentís, esto también os lo revelará Dios.
\par 16 Pero en aquello a que hemos llegado, sigamos una misma regla, sintamos una misma cosa.
\par 17 Hermanos, sed imitadores de mí, y mirad a los que así se conducen según el ejemplo que tenéis en nosotros.
\par 18 Porque por ahí andan muchos, de los cuales os dije muchas veces, y aun ahora lo digo llorando, que son enemigos de la cruz de Cristo;
\par 19 el fin de los cuales será perdición, cuyo dios es el vientre, y cuya gloria es su vergüenza; que sólo piensan en lo terrenal.
\par 20 Mas nuestra ciudadanía está en los cielos, de donde también esperamos al Salvador, al Señor Jesucristo;
\par 21 el cual transformará el cuerpo de la humillación nuestra, para que sea semejante al cuerpo de la gloria suya, por el poder con el cual puede también sujetar a sí mismo todas las cosas.

\chapter{4}

\section*{Regocijaos en el Señor siempre}

\par 1 Así que, hermanos míos amados y deseados, gozo y corona mía, estad así firmes en el Señor, amados.
\par 2 Ruego a Evodia y a Síntique, que sean de un mismo sentir en el Señor.
\par 3 Asimismo te ruego también a ti, compañero fiel, que ayudes a éstas que combatieron juntamente conmigo en el evangelio, con Clemente también y los demás colaboradores míos, cuyos nombres están en el libro de la vida.
\par 4 Regocijaos en el Señor siempre. Otra vez digo: ¡Regocijaos!
\par 5 Vuestra gentileza sea conocida de todos los hombres. El Señor está cerca.
\par 6 Por nada estéis afanosos, sino sean conocidas vuestras peticiones delante de Dios en toda oración y ruego, con acción de gracias.
\par 7 Y la paz de Dios, que sobrepasa todo entendimiento, guardará vuestros corazones y vuestros pensamientos en Cristo Jesús.

\section*{En esto pensad}

\par 8 Por lo demás, hermanos, todo lo que es verdadero, todo lo honesto, todo lo justo, todo lo puro, todo lo amable, todo lo que es de buen nombre; si hay virtud alguna, si algo digno de alabanza, en esto pensad.
\par 9 Lo que aprendisteis y recibisteis y oísteis y visteis en mí, esto haced; y el Dios de paz estará con vosotros.

\section*{Dádivas de los filipenses}

\par 10 En gran manera me gocé en el Señor de que ya al fin habéis revivido vuestro cuidado de mí; de lo cual también estabais solícitos, pero os faltaba la oportunidad.
\par 11 No lo digo porque tenga escasez, pues he aprendido a contentarme, cualquiera que sea mi situación.
\par 12 Sé vivir humildemente, y sé tener abundancia; en todo y por todo estoy enseñado, así para estar saciado como para tener hambre, así para tener abundancia como para padecer necesidad.
\par 13 Todo lo puedo en Cristo que me fortalece.
\par 14 Sin embargo, bien hicisteis en participar conmigo en mi tribulación.
\par 15 Y sabéis también vosotros, oh filipenses, que al principio de la predicación del evangelio, cuando partí de Macedonia, ninguna iglesia participó conmigo en razón de dar y recibir, sino vosotros solos;
\par 16 pues aun a Tesalónica me enviasteis una y otra vez para mis necesidades.
\par 17 No es que busque dádivas, sino que busco fruto que abunde en vuestra cuenta.
\par 18 Pero todo lo he recibido, y tengo abundancia; estoy lleno, habiendo recibido de Epafrodito lo que enviasteis; olor fragante, sacrificio acepto, agradable a Dios.
\par 19 Mi Dios, pues, suplirá todo lo que os falta conforme a sus riquezas en gloria en Cristo Jesús.
\par 20 Al Dios y Padre nuestro sea gloria por los siglos de los siglos. Amén.

\section*{Salutaciones finales}

\par 21 Saludad a todos los santos en Cristo Jesús. Los hermanos que están conmigo os saludan.
\par 22 Todos los santos os saludan, y especialmente los de la casa de César.
\par 23 La gracia de nuestro Señor Jesucristo sea con todos vosotros. Amén.

\end{document}