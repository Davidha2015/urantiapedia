\begin{document}
\title{La Epístola del Apóstol San Pablo a TITO}

\chapter{1}

\section*{Salutación}

\par 1 Pablo, siervo de Dios y apóstol de Jesucristo, conforme a la fe de los escogidos de Dios y el conocimiento de la verdad que es según la piedad,
\par 2 en la esperanza de la vida eterna, la cual Dios, que no miente, prometió desde antes del principio de los siglos,
\par 3 y a su debido tiempo manifestó su palabra por medio de la predicación que me fue encomendada por mandato de Dios nuestro Salvador,
\par 4 a Tito, verdadero hijo en la común fe: Gracia, misericordia y paz, de Dios Padre y del Señor Jesucristo nuestro Salvador.

\section*{Requisitos de ancianos y obispos}

\par 5 Por esta causa te dejé en Creta, para que corrigieses lo deficiente, y establecieses ancianos en cada ciudad, así como yo te mandé;
\par 6 el que fuere irreprensible, marido de una sola mujer, y tenga hijos creyentes que no estén acusados de disolución ni de rebeldía.
\par 7 Porque es necesario que el obispo sea irreprensible, como administrador de Dios; no soberbio, no iracundo, no dado al vino, no pendenciero, no codicioso de ganancias deshonestas,
\par 8 sino hospedador, amante de lo bueno, sobrio, justo, santo, dueño de sí mismo,
\par 9 retenedor de la palabra fiel tal como ha sido enseñada, para que también pueda exhortar con sana enseñanza y convencer a los que contradicen.
\par 10 Porque hay aún muchos contumaces, habladores de vanidades y engañadores, mayormente los de la circuncisión,
\par 11 a los cuales es preciso tapar la boca; que trastornan casas enteras, enseñando por ganancia deshonesta lo que no conviene.
\par 12 Uno de ellos, su propio profeta, dijo: Los cretenses, siempre mentirosos, malas bestias, glotones ociosos.
\par 13 Este testimonio es verdadero; por tanto, repréndelos duramente, para que sean sanos en la fe,
\par 14 no atendiendo a fábulas judaicas, ni a mandamientos de hombres que se apartan de la verdad.
\par 15 Todas las cosas son puras para los puros, mas para los corrompidos e incrédulos nada les es puro; pues hasta su mente y su conciencia están corrompidas.
\par 16 Profesan conocer a Dios, pero con los hechos lo niegan, siendo abominables y rebeldes, reprobados en cuanto a toda buena obra.

\chapter{2}

\section*{Enseñanza de la sana doctrina}

\par 1 Pero tú habla lo que está de acuerdo con la sana doctrina.
\par 2 Que los ancianos sean sobrios, serios, prudentes, sanos en la fe, en el amor, en la paciencia.
\par 3 Las ancianas asimismo sean reverentes en su porte; no calumniadoras, no esclavas del vino, maestras del bien;
\par 4 que enseñen a las mujeres jóvenes a amar a sus maridos y a sus hijos,
\par 5 a ser prudentes, castas, cuidadosas de su casa, buenas, sujetas a sus maridos, para que la palabra de Dios no sea blasfemada.
\par 6 Exhorta asimismo a los jóvenes a que sean prudentes;
\par 7 presentándote tú en todo como ejemplo de buenas obras; en la enseñanza mostrando integridad, seriedad,
\par 8 palabra sana e irreprochable, de modo que el adversario se avergüence, y no tenga nada malo que decir de vosotros.
\par 9 Exhorta a los siervos a que se sujeten a sus amos, que agraden en todo, que no sean respondones;
\par 10 no defraudando, sino mostrándose fieles en todo, para que en todo adornen la doctrina de Dios nuestro Salvador.
\par 11 Porque la gracia de Dios se ha manifestado para salvación a todos los hombres,
\par 12 enseñándonos que, renunciando a la impiedad y a los deseos mundanos, vivamos en este siglo sobria, justa y piadosamente,
\par 13 aguardando la esperanza bienaventurada y la manifestación gloriosa de nuestro gran Dios y Salvador Jesucristo,
\par 14 quien se dio a sí mismo por nosotros para redimirnos de toda iniquidad y purificar para sí un pueblo propio, celoso de buenas obras.
\par 15 Esto habla, y exhorta y reprende con toda autoridad. Nadie te menosprecie.

\chapter{3}

\section*{Justificados por gracia}

\par 1 Recuérdales que se sujeten a los gobernantes y autoridades, que obedezcan, que estén dispuestos a toda buena obra.
\par 2 Que a nadie difamen, que no sean pendencieros, sino amables, mostrando toda mansedumbre para con todos los hombres.
\par 3 Porque nosotros también éramos en otro tiempo insensatos, rebeldes, extraviados, esclavos de concupiscencias y deleites diversos, viviendo en malicia y envidia, aborrecibles, y aborreciéndonos unos a otros.
\par 4 Pero cuando se manifestó la bondad de Dios nuestro Salvador, y su amor para con los hombres,
\par 5 nos salvó, no por obras de justicia que nosotros hubiéramos hecho, sino por su misericordia, por el lavamiento de la regeneración y por la renovación en el Espíritu Santo,
\par 6 el cual derramó en nosotros abundantemente por Jesucristo nuestro Salvador,
\par 7 para que justificados por su gracia, viniésemos a ser herederos conforme a la esperanza de la vida eterna.
\par 8 Palabra fiel es esta, y en estas cosas quiero que insistas con firmeza, para que los que creen en Dios procuren ocuparse en buenas obras. Estas cosas son buenas y útiles a los hombres.
\par 9 Pero evita las cuestiones necias, y genealogías, y contenciones, y discusiones acerca de la ley; porque son vanas y sin provecho.
\par 10 Al hombre que cause divisiones, después de una y otra amonestación deséchalo,
\par 11 sabiendo que el tal se ha pervertido, y peca y está condenado por su propio juicio.

\section*{Instrucciones personales}

\par 12 Cuando envíe a ti a Artemas o a Tíquico, apresúrate a venir a mí en Nicópolis, porque allí he determinado pasar el invierno.
\par 13 A Zenas intérprete de la ley, y a Apolos, encamínales con solicitud, de modo que nada les falte.
\par 14 Y aprendan también los nuestros a ocuparse en buenas obras para los casos de necesidad, para que no sean sin fruto.

\section*{Salutaciones y bendición final}

\par 15 Todos los que están conmigo te saludan. Saluda a los que nos aman en la fe. La gracia sea con todos vosotros. Amén.

\end{document}