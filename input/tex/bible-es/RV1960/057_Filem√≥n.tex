\begin{document}
%\title{La Epístola del Apóstol San Pablo a FILEMÓN}
\title{Epístola a Filemón}

\chapter{1}

\section*{Salutación}

\par 1 Pablo, prisionero de Jesucristo, y el hermano Timoteo, al amado Filemón, colaborador nuestro,
\par 2 y a la amada hermana Apia, y a Arquipo nuestro compañero de milicia, y a la iglesia que está en tu casa:
\par 3 Gracia y paz a vosotros, de Dios nuestro Padre y del Señor Jesucristo.

\section*{El amor y la fe de Filemón}

\par 4 Doy gracias a mi Dios, haciendo siempre memoria de tí en mis oraciones,
\par 5 porque oigo del amor y de la fe que tienes hacia el Señor Jesús, y para con todos los santos;
\par 6 para que la participación de tu fe sea eficaz en el conocimiento de todo el bien que está en vosotros por Cristo Jesús.
\par 7 Pues tenemos gran gozo y consolación en tu amor, porque por ti, oh hermano, han sido confortados los corazones de los santos.

\section*{Pablo intercede por Onésimo}

\par 8 Por lo cual, aunque tengo mucha libertad en Cristo para mandarte lo que conviene,
\par 9 más bien te ruego por amor, siendo como soy, Pablo ya anciano, y ahora, además, prisionero de Jesucristo;
\par 10 te ruego por mi hijo Onésimo, a quien engendré en mis prisiones,
\par 11 el cual en otro tiempo te fue inútil, pero ahora a ti y a mí nos es útil,
\par 12 el cual vuelvo a enviarte; tú, pues, recíbele como a mí mismo.
\par 13 Yo quisiera retenerle conmigo, para que en lugar tuyo me sirviese en mis prisiones por el evangelio;
\par 14 pero nada quise hacer sin tu consentimiento, para que tu favor no fuese como de necesidad, sino voluntario.
\par 15 Porque quizás para esto se apartó de ti por algún tiempo, para que le recibieses para siempre;
\par 16 no ya como esclavo, sino como más que esclavo, como hermano amado, mayormente para mí, pero cuánto más para ti, tanto en la carne como en el Señor.
\par 17 Así que, si me tienes por compañero, recíbele como a mí mismo.
\par 18 Y si en algo te dañó, o te debe, ponlo a mi cuenta.
\par 19 Yo Pablo lo escribo de mi mano, yo lo pagaré; por no decirte que aun tú mismo te me debes también.
\par 20 Sí, hermano, tenga yo algún provecho de ti en el Señor; conforta mi corazón en el Señor.
\par 21 Te he escrito confiando en tu obediencia, sabiendo que harás aun más de lo que te digo.
\par 22 Prepárame también alojamiento; porque espero que por vuestras oraciones os seré concedido.

\section*{Salutaciones y bendición final}

\par 23 Te saludan Epafras, mi compañero de prisiones por Cristo Jesús,
\par 24 Marcos, Aristarco, Demas y Lucas, mis colaboradores.
\par 25 La gracia de nuestro Señor Jesucristo sea con vuestro espíritu. Amén.

\end{document}