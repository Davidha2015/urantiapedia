\begin{document}
\title{La Epístola Universal de SANTIAGO}

\chapter{1}

\section*{Salutación}

\par 1 Santiago, siervo de Dios y del Señor Jesucristo, a las doce tribus que están en la dispersión: Salud.

\section*{La sabiduría que viene de Dios}

\par 2 Hermanos míos, tened por sumo gozo cuando os halléis en diversas pruebas,
\par 3 sabiendo que la prueba de vuestra fe produce paciencia.
\par 4 Mas tenga la paciencia su obra completa, para que seáis perfectos y cabales, sin que os falte cosa alguna.
\par 5 Y si alguno de vosotros tiene falta de sabiduría, pídala a Dios, el cual da a todos abundantemente y sin reproche, y le será dada.
\par 6 Pero pida con fe, no dudando nada; porque el que duda es semejante a la onda del mar, que es arrastrada por el viento y echada de una parte a otra.
\par 7 No piense, pues, quien tal haga, que recibirá cosa alguna del Señor.
\par 8 El hombre de doble ánimo es inconstante en todos sus caminos.
\par 9 El hermano que es de humilde condición, gloríese en su exaltación;
\par 10 pero el que es rico, en su humillación; porque él pasará como la flor de la hierba.
\par 11 Porque cuando sale el sol con calor abrasador, la hierba se seca, su flor se cae, y perece su hermosa apariencia; así también se marchitará el rico en todas sus empresas.

\section*{Soportando las pruebas}

\par 12 Bienaventurado el varón que soporta la tentación; porque cuando haya resistido la prueba, recibirá la corona de vida, que Dios ha prometido a los que le aman.
\par 13 Cuando alguno es tentado, no diga que es tentado de parte de Dios; porque Dios no puede ser tentado por el mal, ni él tienta a nadie;
\par 14 sino que cada uno es tentado, cuando de su propia concupiscencia es atraído y seducido.
\par 15 Entonces la concupiscencia, después que ha concebido, da a luz el pecado; y el pecado, siendo consumado, da a luz la muerte.
\par 16 Amados hermanos míos, no erréis.
\par 17 Toda buena dádiva y todo don perfecto desciende de lo alto, del Padre de las luces, en el cual no hay mudanza, ni sombra de variación.
\par 18 El, de su voluntad, nos hizo nacer por la palabra de verdad, para que seamos primicias de sus criaturas.

\section*{Hacedores de la palabra}

\par 19 Por esto, mis amados hermanos, todo hombre sea pronto para oír, tardo para hablar, tardo para airarse;
\par 20 porque la ira del hombre no obra la justicia de Dios.
\par 21 Por lo cual, desechando toda inmundicia y abundancia de malicia, recibid con mansedumbre la palabra implantada, la cual puede salvar vuestras almas.
\par 22 Pero sed hacedores de la palabra, y no tan solamente oidores, engañándoos a vosotros mismos.
\par 23 Porque si alguno es oidor de la palabra pero no hacedor de ella, éste es semejante al hombre que considera en un espejo su rostro natural.
\par 24 Porque él se considera a sí mismo, y se va, y luego olvida cómo era.
\par 25 Mas el que mira atentamente en la perfecta ley, la de la libertad, y persevera en ella, no siendo oidor olvidadizo, sino hacedor de la obra, éste será bienaventurado en lo que hace.
\par 26 Si alguno se cree religioso entre vosotros, y no refrena su lengua, sino que engaña su corazón, la religión del tal es vana.
\par 27 La religión pura y sin mácula delante de Dios el Padre es esta: Visitar a los huérfanos y a las viudas en sus tribulaciones, y guardarse sin mancha del mundo.

\chapter{2}

\section*{Amonestación contra la parcialidad}

\par 1 Hermanos míos, que vuestra fe en nuestro glorioso Señor Jesucristo sea sin acepción de personas.
\par 2 Porque si en vuestra congregación entra un hombre con anillo de oro y con ropa espléndida, y también entra un pobre con vestido andrajoso,
\par 3 y miráis con agrado al que trae la ropa espléndida y le decís: Siéntate tú aquí en buen lugar; y decís al pobre: Estate tú allí en pie, o siéntate aquí bajo mi estrado;
\par 4 ¿no hacéis distinciones entre vosotros mismos, y venís a ser jueces con malos pensamientos?
\par 5 Hermanos míos amados, oíd: ¿No ha elegido Dios a los pobres de este mundo, para que sean ricos en fe y herederos del reino que ha prometido a los que le aman?
\par 6 Pero vosotros habéis afrentado al pobre. ¿No os oprimen los ricos, y no son ellos los mismos que os arrastran a los tribunales?
\par 7 ¿No blasfeman ellos el buen nombre que fue invocado sobre vosotros?
\par 8 Si en verdad cumplís la ley real, conforme a la Escritura: Amarás a tu prójimo como a ti mismo, bien hacéis;
\par 9 pero si hacéis acepción de personas, cometéis pecado, y quedáis convictos por la ley como transgresores.
\par 10 Porque cualquiera que guardare toda la ley, pero ofendiere en un punto, se hace culpable de todos.
\par 11 Porque el que dijo: No cometerás adulterio, también ha dicho: No matarás. Ahora bien, si no cometes adulterio, pero matas, ya te has hecho transgresor de la ley.
\par 12 Así hablad, y así haced, como los que habéis de ser juzgados por la ley de la libertad.
\par 13 Porque juicio sin misericordia se hará con aquel que no hiciere misericordia; y la misericordia triunfa sobre el juicio.

\section*{La fe sin obras es muerta}

\par 14 Hermanos míos, ¿de qué aprovechará si alguno dice que tiene fe, y no tiene obras? ¿Podrá la fe salvarle?
\par 15 Y si un hermano o una hermana están desnudos, y tienen necesidad del mantenimiento de cada día,
\par 16 y alguno de vosotros les dice: Id en paz, calentaos y saciaos, pero no les dais las cosas que son necesarias para el cuerpo, ¿de qué aprovecha?
\par 17 Así también la fe, si no tiene obras, es muerta en sí misma.
\par 18 Pero alguno dirá: Tú tienes fe, y yo tengo obras. Muéstrame tu fe sin tus obras, y yo te mostraré mi fe por mis obras.
\par 19 Tú crees que Dios es uno; bien haces. También los demonios creen, y tiemblan.
\par 20 ¿Mas quieres saber, hombre vano, que la fe sin obras es muerta?
\par 21 ¿No fue justificado por las obras Abraham nuestro padre, cuando ofreció a su hijo Isaac sobre el altar?
\par 22 ¿No ves que la fe actuó juntamente con sus obras, y que la fe se perfeccionó por las obras?
\par 23 Y se cumplió la Escritura que dice: Abraham creyó a Dios, y le fue contado por justicia, y fue llamado amigo de Dios.
\par 24 Vosotros veis, pues, que el hombre es justificado por las obras, y no solamente por la fe.
\par 25 Asimismo también Rahab la ramera, ¿no fue justificada por obras, cuando recibió a los mensajeros y los envió por otro camino?
\par 26 Porque como el cuerpo sin espíritu está muerto, así también la fe sin obras está muerta.

\chapter{3}

\section*{La lengua}

\par 1 Hermanos míos, no os hagáis maestros muchos de vosotros, sabiendo que recibiremos mayor condenación.
\par 2 Porque todos ofendemos muchas veces. Si alguno no ofende en palabra, éste es varón perfecto, capaz también de refrenar todo el cuerpo.
\par 3 He aquí nosotros ponemos freno en la boca de los caballos para que nos obedezcan, y dirigimos así todo su cuerpo.
\par 4 Mirad también las naves; aunque tan grandes, y llevadas de impetuosos vientos, son gobernadas con un muy pequeño timón por donde el que las gobierna quiere.
\par 5 Así también la lengua es un miembro pequeño, pero se jacta de grandes cosas. He aquí, ¡cuán grande bosque enciende un pequeño fuego!
\par 6 Y la lengua es un fuego, un mundo de maldad. La lengua está puesta entre nuestros miembros, y contamina todo el cuerpo, e inflama la rueda de la creación, y ella misma es inflamada por el infierno.
\par 7 Porque toda naturaleza de bestias, y de aves, y de serpientes, y de seres del mar, se doma y ha sido domada por la naturaleza humana;
\par 8 pero ningún hombre puede domar la lengua, que es un mal que no puede ser refrenado, llena de veneno mortal.
\par 9 Con ella bendecimos al Dios y Padre, y con ella maldecimos a los hombres, que están hechos a la semejanza de Dios.
\par 10 De una misma boca proceden bendición y maldición. Hermanos míos, esto no debe ser así.
\par 11 ¿Acaso alguna fuente echa por una misma abertura agua dulce y amarga?
\par 12 Hermanos míos, ¿puede acaso la higuera producir aceitunas, o la vid higos? Así también ninguna fuente puede dar agua salada y dulce.

\section*{La sabiduría de lo alto}

\par 13 ¿Quién es sabio y entendido entre vosotros? Muestre por la buena conducta sus obras en sabia mansedumbre.
\par 14 Pero si tenéis celos amargos y contención en vuestro corazón, no os jactéis, ni mintáis contra la verdad;
\par 15 porque esta sabiduría no es la que desciende de lo alto, sino terrenal, animal, diabólica.
\par 16 Porque donde hay celos y contención, allí hay perturbación y toda obra perversa.
\par 17 Pero la sabiduría que es de lo alto es primeramente pura, después pacífica, amable, benigna, llena de misericordia y de buenos frutos, sin incertidumbre ni hipocresía.
\par 18 Y el fruto de justicia se siembra en paz para aquellos que hacen la paz.

\chapter{4}

\section*{La amistad con el mundo}

\par 1 ¿De dónde vienen las guerras y los pleitos entre vosotros? ¿No es de vuestras pasiones, las cuales combaten en vuestros miembros?
\par 2 Codiciáis, y no tenéis; matáis y ardéis de envidia, y no podéis alcanzar; combatís y lucháis, pero no tenéis lo que deseáis, porque no pedís.
\par 3 Pedís, y no recibís, porque pedís mal, para gastar en vuestros deleites.
\par 4 ¡Oh almas adúlteras! ¿No sabéis que la amistad del mundo es enemistad contra Dios? Cualquiera, pues, que quiera ser amigo del mundo, se constituye enemigo de Dios.
\par 5 ¿O pensáis que la Escritura dice en vano: El Espíritu que él ha hecho morar en nosotros nos anhela celosamente?
\par 6 Pero él da mayor gracia. Por esto dice: Dios resiste a los soberbios, y da gracia a los humildes.
\par 7 Someteos, pues, a Dios; resistid al diablo, y huirá de vosotros.
\par 8 Acercaos a Dios, y él se acercará a vosotros. Pecadores, limpiad las manos; y vosotros los de doble ánimo, purificad vuestros corazones.
\par 9 Afligíos, y lamentad, y llorad. Vuestra risa se convierta en lloro, y vuestro gozo en tristeza.
\par 10 Humillaos delante del Señor, y él os exaltará.

\section*{Juzgando al hermano}

\par 11 Hermanos, no murmuréis los unos de los otros. El que murmura del hermano y juzga a su hermano, murmura de la ley y juzga a la ley; pero si tú juzgas a la ley, no eres hacedor de la ley, sino juez.
\par 12 Uno solo es el dador de la ley, que puede salvar y perder; pero tú, ¿quién eres para que juzgues a otro?

\section*{No os gloriéis del día de mañana}

\par 13 ¡Vamos ahora! los que decís: Hoy y mañana iremos a tal ciudad, y estaremos allá un año, y traficaremos, y ganaremos;
\par 14 cuando no sabéis lo que será mañana. Porque ¿qué es vuestra vida? Ciertamente es neblina que se aparece por un poco de tiempo, y luego se desvanece.
\par 15 En lugar de lo cual deberíais decir: Si el Señor quiere, viviremos y haremos esto o aquello.
\par 16 Pero ahora os jactáis en vuestras soberbias. Toda jactancia semejante es mala;
\par 17 y al que sabe hacer lo bueno, y no lo hace, le es pecado.

\chapter{5}

\section*{Contra los ricos opresores}

\par 1 ¡Vamos ahora, ricos! Llorad y aullad por las miserias que os vendrán.
\par 2 Vuestras riquezas están podridas, y vuestras ropas están comidas de polilla.
\par 3 Vuestro oro y plata están enmohecidos; y su moho testificará contra vosotros, y devorará del todo vuestras carnes como fuego. Habéis acumulado tesoros para los días postreros.
\par 4 He aquí, clama el jornal de los obreros que han cosechado vuestras tierras, el cual por engaño no les ha sido pagado por vosotros; y los clamores de los que habían segado han entrado en los oídos del Señor de los ejércitos.
\par 5 Habéis vivido en deleites sobre la tierra, y sido disolutos; habéis engordado vuestros corazones como en día de matanza.
\par 6 Habéis condenado y dado muerte al justo, y él no os hace resistencia.

\section*{Sed pacientes y orad}

\par 7 Por tanto, hermanos, tened paciencia hasta la venida del Señor. Mirad cómo el labrador espera el precioso fruto de la tierra, aguardando con paciencia hasta que reciba la lluvia temprana y la tardía.
\par 8 Tened también vosotros paciencia, y afirmad vuestros corazones; porque la venida del Señor se acerca.
\par 9 Hermanos, no os quejéis unos contra otros, para que no seáis condenados; he aquí, el juez está delante de la puerta.
\par 10 Hermanos míos, tomad como ejemplo de aflicción y de paciencia a los profetas que hablaron en nombre del Señor.
\par 11 He aquí, tenemos por bienaventurados a los que sufren. Habéis oído de la paciencia de Job, y habéis visto el fin del Señor, que el Señor es muy misericordioso y compasivo.
\par 12 Pero sobre todo, hermanos míos, no juréis, ni por el cielo, ni por la tierra, ni por ningún otro juramento; sino que vuestro sí sea sí, y vuestro no sea no, para que no caigáis en condenación.
\par 13 ¿Está alguno entre vosotros afligido? Haga oración. ¿Está alguno alegre? Cante alabanzas.
\par 14 ¿Está alguno enfermo entre vosotros? Llame a los ancianos de la iglesia, y oren por él, ungiéndole con aceite en el nombre del Señor.
\par 15 Y la oración de fe salvará al enfermo, y el Señor lo levantará; y si hubiere cometido pecados, le serán perdonados.
\par 16 Confesaos vuestras ofensas unos a otros, y orad unos por otros, para que seáis sanados. La oración eficaz del justo puede mucho.
\par 17 Elías era hombre sujeto a pasiones semejantes a las nuestras, y oró fervientemente para que no lloviese, y no llovió sobre la tierra por tres años y seis meses.
\par 18 Y otra vez oró, y el cielo dio lluvia, y la tierra produjo su fruto.
\par 19 Hermanos, si alguno de entre vosotros se ha extraviado de la verdad, y alguno le hace volver,
\par 20 sepa que el que haga volver al pecador del error de su camino, salvará de muerte un alma, y cubrirá multitud de pecados.

\end{document}