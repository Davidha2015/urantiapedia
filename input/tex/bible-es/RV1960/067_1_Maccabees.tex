\begin{document}

\title{1 Macabeos}

\chapter{1}

\par 1 Y aconteció que después que Alejandro, hijo de Filipo, el macedonio, que había venido de la tierra de Jettim, derrotó a Darío, rey de los persas y de los medos, reinó en su lugar, el primero sobre Grecia,
\par 2 E hicieron muchas guerras, conquistaron muchas fortalezas y mataron a los reyes de la tierra.
\par 3 Y atravesó los confines de la tierra y tomó despojos de muchas naciones, de modo que la tierra quedó en calma ante él; con lo cual fue exaltado y su corazón se enalteció.
\par 4 Y reunió un ejército muy fuerte y gobernó países, naciones y reyes, que le fueron tributarios.
\par 5 Después de esto cayó enfermo y se dio cuenta de que iba a morir.
\par 6 Por eso llamó a sus siervos, los que eran honorables y habían sido criados con él desde su juventud, y repartió entre ellos su reino mientras él aún vivía.
\par 7 Alejandro reinó doce años y luego murió.
\par 8 Y sus siervos gobiernan cada uno en su lugar.
\par 9 Y después de su muerte, todos se pusieron coronas; Así hicieron sus hijos después de ellos muchos años: y los males se multiplicaron en la tierra.
\par 10 Y de ellos surgió una raíz malvada, Antíoco, llamado Epífanes, hijo del rey Antíoco, que había estado como rehén en Roma y reinó en el año ciento treinta y siete del reino de los griegos.
\par 11 En aquellos días salieron de Israel hombres malvados que persuadieron a muchos, diciendo: Vayamos y hagamos un pacto con las naciones que nos rodean; porque desde que nos separamos de ellos hemos tenido mucho dolor.
\par 12 Así que este dispositivo les gustó mucho.
\par 13 Entonces algunos del pueblo fueron tan atrevidos con esto, que acudieron al rey, quien les dio licencia para seguir las ordenanzas de las naciones:
\par 14 Después de lo cual construyeron en Jerusalén un lugar de ejercicio según las costumbres de los paganos.
\par 15 Se hicieron incircuncisos, abandonaron el santo pacto, se unieron a las naciones y fueron vendidos para hacer maldad.
\par 16 Cuando Antíoco estableció el reino, pensó en reinar sobre Egipto para dominar dos reinos.
\par 17 Por lo cual entró en Egipto con una gran multitud, con carros, elefantes, gente de a caballo y una gran armada,
\par 18 E hizo la guerra a Tolomeo, rey de Egipto, pero Tolomeo tuvo miedo de él y huyó. y muchos fueron heridos de muerte.
\par 19 Así tomaron las ciudades fuertes en la tierra de Egipto y él tomó el botín de ellas.
\par 20 Y después que Antíoco derrotó a Egipto, regresó de nuevo en el año ciento cuarenta y tres y subió contra Israel y Jerusalén con una gran multitud,
\par 21 Y entró orgullosamente en el santuario y se llevó el altar de oro, el candelero de luz y todos sus utensilios.
\par 22 Y la mesa de los panes de la proposición, los vasos para servir y las copas. y los incensarios de oro, y el velo, y la corona, y los adornos de oro que estaban delante del templo, todo lo cual se quitó.
\par 23 Tomó también la plata, el oro y los objetos preciosos, y también los tesoros escondidos que encontró.
\par 24 Y cuando se lo llevó todo, se fue a su tierra, hizo una gran matanza y habló con gran altivez.
\par 25 Por eso hubo un gran luto en Israel, en todos los lugares donde estaban;
\par 26 De modo que los príncipes y los ancianos lloraron, las vírgenes y los jóvenes se debilitaron y la belleza de las mujeres cambió.
\par 27 Todos los novios se lamentaban, y la que estaba sentada en la cámara nupcial estaba triste,
\par 28 También la tierra fue conmovida para sus habitantes, y toda la casa de Jacob quedó cubierta de confusión.
\par 29 Y transcurridos dos años, el rey envió a las ciudades de Judá a su principal recaudador de tributos, el cual llegó a Jerusalén con una gran multitud,
\par 30 Y les habló palabras de paz, pero todo fue engaño; porque cuando le habían creído, cayó repentinamente sobre la ciudad, la hirió con gran dolor y destruyó a gran parte del pueblo de Israel.
\par 31 Y cuando tomó el botín de la ciudad, le prendió fuego y derribó las casas y los muros que la rodeaban.
\par 32 Pero las mujeres y los niños se llevaron cautivos y se apoderaron del ganado.
\par 33 Entonces edificaron la ciudad de David con una muralla grande y fuerte y con torres poderosas, y la convirtieron en un bastión fuerte para ellos.
\par 34 Y pusieron allí una nación pecadora, hombres malvados, y se fortificaron allí.
\par 35 También lo almacenaron con armas y víveres, y cuando reunieron el botín de Jerusalén, lo pusieron allí, y se convirtió en una dolorosa trampa.
\par 36 Porque era un lugar de acecho contra el santuario y un malvado adversario para Israel.
\par 37 Así derramaron sangre inocente por todos lados del santuario y lo profanaron.
\par 38 De tal manera que los habitantes de Jerusalén huyeron a causa de ellos; por lo cual la ciudad se convirtió en habitación de extraños, y llegó a ser extraña para los que habían nacido en ella; y sus propios hijos la abandonaron.
\par 39 Su santuario fue arrasado como un desierto, sus fiestas se convirtieron en luto, sus sábados en vituperio, su honor en desprecio.
\par 40 Como había sido su gloria, así aumentó su deshonra, y su gloria se convirtió en luto.
\par 41 Además, el rey Antíoco escribió a todo su reino que todos deberían ser un solo pueblo,
\par 42 Y cada uno debía abandonar sus leyes: así todos los paganos se pusieron de acuerdo según el mandamiento del rey.
\par 43 Incluso muchos de los israelitas aceptaron su religión, sacrificaron a los ídolos y profanaron el sábado.
\par 44 Porque el rey había enviado cartas por medio de mensajeros a Jerusalén y a las ciudades de Judá para que siguieran las leyes extrañas del país,
\par 45 Y prohibir los holocaustos, los sacrificios y las libaciones en el templo; y que profanen los sábados y los días festivos:
\par 46 y contaminarán el santuario y el pueblo santo:
\par 47 Levantad altares, bosques y capillas para ídolos, y sacrificad carne de cerdos y animales inmundos.
\par 48 Que también dejen a sus hijos incircuncisos y hagan sus almas abominables con toda clase de inmundicia y profanación.
\par 49 Hasta el fin podrían olvidarse de la ley y cambiar todas las ordenanzas.
\par 50 Y a todo aquel que no obedezca el mandato del rey, le dijo que morirá.
\par 51 De la misma manera escribió a todo su reino, y nombró jefes sobre todo el pueblo, ordenando a las ciudades de Judá que hicieran sacrificios, ciudad por ciudad.
\par 52 Entonces se les reunió mucha gente del pueblo, es decir, todos los que habían abandonado la ley; y así cometieron males en la tierra;
\par 53 Y expulsaron a los israelitas a lugares secretos, incluso a cualquier lugar donde pudieran huir en busca de ayuda.
\par 54 El día quince del mes Casleu, en el año ciento cuarenta y cinco, pusieron sobre el altar la abominación desoladora, y edificaron altares de ídolos por todas las ciudades de Judá, alrededor;
\par 55 Y quemaban incienso a las puertas de sus casas y en las calles.
\par 56 Y cuando rompieron en pedazos los libros de la ley que encontraron, los quemaron al fuego.
\par 57 Y a cualquiera que fuese hallado en posesión del libro del testamento, o cualquiera que estuviera sometido a la ley, el mandamiento del rey era que lo mataran.
\par 58 Así lo hacían, según su autoridad, con los israelitas cada mes, con todos los que se encontraban en las ciudades.
\par 59 El día veinticinco del mes sacrificaron sobre el altar de los ídolos que estaba sobre el altar de Dios.
\par 60 Entonces, según el mandamiento, mataron a algunas mujeres que habían hecho circuncidar a sus hijos.
\par 61 Colgaron a los niños al cuello, saquearon sus casas y mataron a los que los habían circuncidado.
\par 62 Sin embargo, muchos en Israel estaban decididos y decididos a no comer nada inmundo.
\par 63 Por eso prefirieron morir para no contaminarse con alimentos y no profanar la santa alianza, y entonces murieron.
\par 64 Y hubo una ira muy grande sobre Israel.

\chapter{2}

\par 1 En aquellos días se levantó Matatías, hijo de Juan, hijo de Simeón, sacerdote de los hijos de Joarib, de Jerusalén, y habitó en Modín.
\par 2 Y tuvo cinco hijos, Juanán, llamado Caddis:
\par 3 Simón; llamado Thassi:
\par 4 Judas, llamado Macabeo:
\par 5 Eleazar, llamado Avarán, y Jonatán, cuyo sobrenombre era Afo.
\par 6 Y cuando vio las blasfemias que se cometían en Judá y en Jerusalén,
\par 7 Él dijo: ¡Ay de mí! ¿Por qué nací yo para ver esta miseria de mi pueblo y de la ciudad santa, y para habitar allí, cuando fue entregada en manos del enemigo, y el santuario en manos de extraños?
\par 8 Su templo se ha vuelto como un hombre sin gloria.
\par 9 Sus gloriosos vasos son llevados en cautiverio, sus niños son asesinados en las calles, sus jóvenes a espada del enemigo.
\par 10 ¿Qué nación no ha tenido parte en su reino y no ha recibido parte de su botín?
\par 11 Le han quitado todos sus adornos; de una mujer libre se ha convertido en esclava.
\par 12 Y he aquí, nuestro santuario, nuestra hermosura y nuestra gloria, han sido arrasados ​​y los gentiles lo han profanado.
\par 13 ¿Para qué, pues, viviremos más?
\par 14 Entonces Matatías y sus hijos rasgaron sus vestidos, se vistieron de cilicio y lloraron profundamente.
\par 15 Mientras tanto, los oficiales del rey, que habían obligado al pueblo a rebelarse, llegaron a la ciudad de Modin para ofrecerles sacrificios.
\par 16 Y cuando muchos de Israel vinieron a ellos, también se reunieron Matatías y sus hijos.
\par 17 Entonces respondieron los oficiales del rey y dijeron a Matatías: Tú eres un gobernante, un hombre honorable y grande en esta ciudad, y estás fortalecido con hijos y hermanos.
\par 18 Ahora, pues, ven tú primero y cumple el mandamiento del rey, como lo han hecho todas las naciones, y también los hombres de Judá y los que quedan en Jerusalén; así estarás tú y tu casa en el número de los amigos del rey, y tú y tus hijos seréis honrados con plata y oro, y muchas recompensas.
\par 19 Entonces Matatías respondió y habló en alta voz: Aunque todas las naciones que están bajo el dominio del rey le obedezcan y se aparten cada una de la religión de sus padres y acepten sus mandamientos,
\par 20 Aún así, yo, mis hijos y mis hermanos andaremos en el pacto de nuestros padres.
\par 21 No permita Dios que abandonemos la ley y las ordenanzas.
\par 22 No escucharemos las palabras del rey para alejarnos de nuestra religión, ni a la derecha ni a la izquierda.
\par 23 Cuando terminó de decir estas palabras, vino uno de los judíos, en presencia de todos, para ofrecer sacrificios en el altar que estaba en Modin, según la orden del rey.
\par 24 Cuando Matatías vio esto, se encendió de celo y le temblaron las riendas, y no pudo evitar mostrar su ira según el juicio; por lo que corrió y lo mató sobre el altar.
\par 25 También mató en aquel momento al comisario del rey, que obligaba a los hombres a sacrificar, y derribó el altar.
\par 26 Así procedió con celo por la ley de Dios, como lo hizo Finees con Zambri, hijo de Salom.
\par 27 Y Matatías gritó a gran voz por toda la ciudad, diciendo: Cualquiera que sea celoso de la ley y guarde el pacto, sígame.
\par 28 Entonces él y sus hijos huyeron a las montañas y dejaron todo lo que tenían en la ciudad.
\par 29 Entonces muchos que buscaban justicia y juicio descendieron al desierto para habitar allí.
\par 30 Tanto ellos como sus hijos y sus mujeres; y su ganado; porque las aflicciones aumentaron sobre ellos.
\par 31 Cuando se informó a los servidores del rey y al ejército que estaba en Jerusalén, en la ciudad de David, que ciertos hombres que habían violado el mandamiento del rey habían descendido a lugares secretos en el desierto,
\par 32 Los persiguieron muchos y, al alcanzarlos, acamparon contra ellos y les hicieron la guerra en sábado.
\par 33 Y ellos les dijeron: Basta lo que habéis hecho hasta ahora; Salid y haced conforme al mandamiento del rey, y viviréis.
\par 34 Pero ellos dijeron: No saldremos ni cumpliremos el mandamiento del rey para profanar el día de reposo.
\par 35 Entonces les dieron la batalla a toda velocidad.
\par 36 Pero ellos no les respondieron, ni les arrojaron piedra, ni detuvieron los lugares donde se escondían;
\par 37 Pero dijo: Muramos todos en nuestra inocencia: el cielo y la tierra darán testimonio de nosotros de que nos habéis matado injustamente.
\par 38 Entonces se levantaron contra ellos en sábado y los mataron junto con sus mujeres, sus hijos y su ganado, en número de mil personas.
\par 39 Cuando Matatías y sus amigos se enteraron de esto, se lamentaron profundamente por ellos.
\par 40 Y uno de ellos dijo a otro: Si todos hacemos lo que han hecho nuestros hermanos y no luchamos por nuestras vidas y leyes contra los paganos, pronto nos desarraigarán de la tierra.
\par 41 Entonces decretaron en aquel tiempo, diciendo: Cualquiera que venga a pelear contra nosotros en día de sábado, pelearemos contra él; ni moriremos todos, como nuestros hermanos que fueron asesinados en los lugares secretos.
\par 42 Entonces se le acercó un grupo de asideos, los valientes de Israel, todos los que voluntariamente observaban la ley.
\par 43 También se unieron a ellos todos los que huían por la persecución, y les sirvieron de apoyo.
\par 44 Entonces unieron sus fuerzas y derrotaron a los pecadores en su ira y a los impíos en su ira; pero los demás huyeron a las naciones en busca de ayuda.
\par 45 Entonces Matatías y sus amigos rodearon y derribaron los altares.
\par 46 Y a los niños incircuncisos que encontraron en la costa de Israel, los circuncidaron valientemente.
\par 47 También persiguieron a los soberbios, y la obra prosperó en sus manos.
\par 48 Así recuperaron la ley de manos de los gentiles y de manos de los reyes, y no permitieron que triunfara el pecador.
\par 49 Cuando se acercaba el tiempo de la muerte de Matatías, dijo a sus hijos: Ahora se han fortalecido el orgullo y la afrenta, y el tiempo de la destrucción y la ira de la ira.
\par 50 Ahora pues, hijos míos, sed celosos de la ley y dad vuestras vidas por el pacto de vuestros padres.
\par 51 Llama a la memoria lo que hicieron nuestros padres en su tiempo; así recibiréis gran honra y un nombre eterno.
\par 52 ¿No fue Abraham fiel en la tentación y le fue imputado como justicia?
\par 53 José, en el momento de su angustia, guardó el mandamiento y fue hecho señor de Egipto.
\par 54 Nuestro padre Finees, siendo celoso y ferviente, obtuvo el pacto del sacerdocio eterno.
\par 55 Jesús, por cumplir la palabra, fue nombrado juez en Israel.
\par 56 Caleb, por dar testimonio ante la congregación, recibió la herencia de la tierra.
\par 57 David, por ser misericordioso, poseyó el trono de un reino eterno.
\par 58 Elías, por ser celoso y ferviente de la ley, fue llevado al cielo.
\par 59 Ananías, Azarías y Misael, al creer, se salvaron de la llama.
\par 60 Daniel, por su inocencia, fue librado de la boca de los leones.
\par 61 Y así, durante todos los siglos, considerad que ninguno de los que confían en él será vencido.
\par 62 No temáis, pues, las palabras del pecador: porque su gloria será estiércol y gusanos.
\par 63 Hoy será exaltado y mañana no será encontrado, porque ha vuelto a su polvo y su pensamiento se ha desvanecido.
\par 64 Por tanto, hijos míos, sed valientes y sed hombres en defensa de la ley; porque por él obtendréis gloria.
\par 65 Y he aquí, yo sé que tu hermano Simón es un hombre de consejo; escúchalo siempre: él será para ti un padre.
\par 66 En cuanto a Judas Macabeo, ha sido valiente y fuerte desde su juventud: que sea tu capitán y pelee la batalla del pueblo.
\par 67 Tomad también con vosotros a todos los que observan la ley y vengad el mal de vuestro pueblo.
\par 68 Recompensad plenamente a los paganos y guardad los mandamientos de la ley.
\par 69 Entonces él los bendijo y se reunió con sus padres.
\par 70 Y murió en el año ciento cuarenta y seis, y sus hijos lo sepultaron en los sepulcros de sus padres en Modín, y todo Israel hizo gran lamentación por él.

\chapter{3}

\par 1 Entonces se levantó en su lugar su hijo Judas, llamado Macabeo.
\par 2 Y todos sus hermanos lo ayudaron, y también todos los que estaban con su padre, y pelearon con alegría la batalla de Israel.
\par 3 Entonces hizo grandes honores para su pueblo, se vistió con una coraza como si fuera un gigante, se ciñó sus arneses de guerra y combatió, protegiendo al ejército con su espada.
\par 4 En sus actos era como un león, y como un cachorro de león que ruge por su presa.
\par 5 Porque persiguió a los malvados, los buscó y quemó a los que afligían a su pueblo.
\par 6 Por eso los impíos retrocedieron por temor a él, y todos los obradores de iniquidad se turbaron, porque la salvación prosperaba en su mano.
\par 7 También entristeció a muchos reyes, y alegró a Jacob con sus actos, y su memoria será bendita para siempre.
\par 8 Además recorrió las ciudades de Judá, destruyendo entre ellas a los impíos y apartando la ira de Israel.
\par 9 De modo que fue famoso hasta lo último de la tierra y recibió a los que estaban a punto de perecer.
\par 10 Entonces Apolonio reunió a los gentiles y a un gran ejército de Samaria para luchar contra Israel.
\par 11 Cuando Judas se dio cuenta de esto, salió a su encuentro, lo hirió y lo mató; también muchos cayeron muertos, pero los demás huyeron.
\par 12 Entonces Judas tomó el botín y también la espada de Apolonio, y con ella peleó toda su vida.
\par 13 Cuando Serón, príncipe del ejército de Siria, oyó decir que Judas había reunido consigo una multitud y un grupo de fieles para salir con él a la guerra,
\par 14 Él dijo: Me conseguiré nombre y honor en el reino; porque iré a pelear contra Judas y los que con él están, los que desprecian el mandamiento del rey.
\par 15 Entonces lo preparó para subir, y con él iba un poderoso ejército de impíos para ayudarlo y vengarse de los hijos de Israel.
\par 16 Cuando llegó cerca de la subida a Bet-horón, Judas salió a su encuentro con un pequeño grupo.
\par 17 Los cuales, al ver el ejército que venía a su encuentro, dijeron a Judas: ¿Cómo podremos, siendo tan pocos, luchar contra una multitud tan grande y tan fuerte, si estamos dispuestos a desmayar de ayuno durante todo este tiempo? ¿día?
\par 18 A lo cual Judas respondió: No es difícil para muchos estar encerrados en manos de unos pocos; y con el Dios del cielo todo es uno, para entregar con una gran multitud, o una pequeña compañía:
\par 19 Porque la victoria en la batalla no está en la multitud del ejército; pero la fuerza viene del cielo.
\par 20 Vienen contra nosotros con mucho orgullo e iniquidad para destruirnos a nosotros, a nuestras mujeres y a nuestros hijos, y despojarnos.
\par 21 Pero luchamos por nuestras vidas y nuestras leyes.
\par 22 Por tanto, el Señor mismo los derribará delante de nosotros; y vosotros, no tengáis miedo de ellos.
\par 23 Tan pronto como terminó de hablar, saltó repentinamente sobre ellos, y Serón y su ejército fueron derribados ante él.
\par 24 Y los persiguieron desde la bajada de Bet-horón hasta la llanura, donde fueron asesinados unos ochocientos hombres de ellos; y el resto huyó a la tierra de los filisteos.
\par 25 Entonces comenzó el temor de Judas y sus hermanos, y un temor muy grande, a caer sobre las naciones circundantes.
\par 26 Cuando su fama llegó hasta el rey, y todas las naciones hablaban de las batallas de Judas.
\par 27 Cuando el rey Antíoco oyó estas cosas, se llenó de indignación, por lo que envió y reunió todas las fuerzas de su reino, un ejército muy fuerte.
\par 28 Abrió también su tesoro y pagó a sus soldados por un año, mandándoles que estuvieran preparados para cuando los necesitara.
\par 29 Sin embargo, cuando vio que el dinero de sus tesoros se había acabado y que los tributos en el país eran pequeños, a causa de las discordias y la peste que había traído sobre la tierra al quitar las leyes antiguas, ;
\par 30 Temía no poder soportar más las cargas ni tener regalos tan generosos como antes, porque había abundado más que los reyes que le precedieron.
\par 31 Por lo tanto, estando muy perplejo en su mente, decidió ir a Persia, para tomar allí los tributos de los países y reunir mucho dinero.
\par 32 Entonces dejó a Lisias, un hombre noble y de sangre real, para que supervisara los asuntos del rey desde el río Éufrates hasta las fronteras de Egipto.
\par 33 Y para criar a su hijo Antíoco hasta que volviera.
\par 34 Además, le entregó la mitad de su ejército y los elefantes, y le encargó de todo lo que quería hacer, así como de los que habitaban en Judá y en Jerusalén.
\par 35 Es decir, que enviaría un ejército contra ellos para destruir y desarraigar la fuerza de Israel y el remanente de Jerusalén, y quitar su memoria de aquel lugar;
\par 36 Y que colocaría extranjeros en todas sus viviendas y dividiría sus tierras por suertes.
\par 37 Entonces el rey tomó la mitad de las tropas que quedaban y partió de Antioquía, su ciudad real, el año ciento cuarenta y siete; y habiendo pasado el río Éufrates, atravesó las tierras altas.
\par 38 Entonces Lisias eligió a Ptolomeo, hijo de Dorimenes, a Nicanor y a Gorgias, hombres valientes de los amigos del rey.
\par 39 Y con ellos envió cuarenta mil hombres de a pie y siete mil de a caballo para ir a la tierra de Judá y destruirla, tal como el rey había ordenado.
\par 40 Salieron con todas sus fuerzas y acamparon junto a Emaús, en la llanura.
\par 41 Y los mercaderes del país, al enterarse de su fama, tomaron mucha plata y oro con sus sirvientes, y vinieron al campamento para comprar esclavos a los hijos de Israel: una potencia también de Siria y de la tierra de los filisteos se unieron a ellos.
\par 42 Cuando Judas y sus hermanos vieron que las miserias se multiplicaban y que las tropas acampaban en sus fronteras, pues sabían que el rey había ordenado destruir al pueblo y abolirlo por completo,
\par 43 Se dijeron unos a otros: Restablezcamos la fortuna de nuestro pueblo y luchemos por nuestro pueblo y por el santuario.
\par 44 Entonces se reunió la congregación para estar preparados para la batalla y orar y pedir misericordia y compasión.
\par 45 Ahora Jerusalén estaba desierta como un desierto, y ninguno de sus hijos entraba ni salía; también el santuario fue hollado, y los extranjeros ocupaban la fortaleza; los paganos tenían su habitación en ese lugar; y le fue quitado el gozo a Jacob, y cesó la flauta con el arpa.
\par 46 Entonces los israelitas se reunieron y llegaron a Masfa, frente a Jerusalén; porque en Maspha era el lugar donde oraban antiguamente en Israel.
\par 47 Aquel día ayunaron, se vistieron de cilicio, se echaron ceniza sobre la cabeza y rasgaron sus vestidos.
\par 48 Y abrió el libro de la ley en el que los paganos habían tratado de pintar la imagen de sus imágenes.
\par 49 Trajeron también las vestiduras de los sacerdotes, las primicias y los diezmos, y excitaron a los nazareos que habían cumplido sus días.
\par 50 Entonces clamaron a gran voz hacia el cielo, diciendo: ¿Qué haremos con estos y adónde los llevaremos?
\par 51 Porque tu santuario está hollado y profanado, y tus sacerdotes están afligidos y abatidos.
\par 52 Y he aquí, los paganos se han reunido contra nosotros para destruirnos: lo que imaginan contra nosotros, tú lo sabes.
\par 53 ¿Cómo podremos hacerles frente, si tú, oh Dios, no eres nuestra ayuda?
\par 54 Entonces tocaron las trompetas y gritaron a gran voz.
\par 55 Después de esto, Judas nombró capitanes sobre el pueblo, capitanes de mil, de cien, de cincuenta y de diez.
\par 56 A los que construían casas, a los que estaban comprometidos, a los que plantaban viñas o a los que tenían miedo, les ordenó que volvieran cada uno a su casa, conforme a la ley.
\par 57 Entonces el campamento se levantó y asentó al sur de Emaús.
\par 58 Entonces Judas dijo: Armaos y sed hombres valientes, y estad preparados para la mañana, a fin de luchar contra estas naciones que se han reunido contra nosotros para destruirnos a nosotros y a nuestro santuario.
\par 59 Porque es mejor para nosotros morir en la batalla que contemplar las calamidades de nuestro pueblo y de nuestro santuario.
\par 60 Sin embargo, como la voluntad de Dios está en los cielos, así haga.

\chapter{4}

\par 1 Entonces Gorgias tomó cinco mil hombres de a pie y mil de los mejores jinetes y salió del campamento de noche;
\par 2 Con el fin de atacar el campamento de los judíos y derrotarlos de repente. Y los hombres de la fortaleza fueron sus guías.
\par 3 Cuando Judas se enteró de esto, se desplazó él mismo, y con él a los hombres valientes, para derrotar al ejército del rey que estaba en Emaús.
\par 4 Mientras aún las fuerzas estaban dispersadas del campamento.
\par 5 Mientras tanto, Gorgias llegó de noche al campamento de Judas y, como no encontró allí a nadie, los buscó en las montañas, diciendo: Estos hombres huyen de nosotros.
\par 6 Pero cuando se hizo de día, Judas apareció en la llanura con tres mil hombres, los cuales, sin embargo, no tenían armas ni espadas en sus manos.
\par 7 Y vieron que el campamento de las naciones era fuerte y bien armado, y rodeado de gente de a caballo; y éstos eran expertos en la guerra.
\par 8 Entonces Judas dijo a los hombres que estaban con él: No temáis a su multitud ni temáis sus ataques.
\par 9 Acordaos de cómo nuestros padres fueron liberados en el Mar Rojo, cuando Faraón los persiguió con un ejército.
\par 10 Ahora, pues, clamemos al cielo, por si acaso el Señor tenga misericordia de nosotros y se acuerde del pacto de nuestros padres y destruya hoy este ejército delante de nosotros.
\par 11 Para que todas las naciones sepan que hay uno que libra y salva a Israel.
\par 12 Entonces los extraños alzaron los ojos y los vieron acercarse a ellos.
\par 13 Entonces salieron del campamento a la batalla; pero los que estaban con Judas tocaron las trompetas.
\par 14 Entonces trabaron batalla, y los paganos, desconcertados, huyeron a la llanura.
\par 15 Pero los últimos de ellos fueron muertos a espada; porque los persiguieron hasta Gazera y hasta las llanuras de Idumea, Azoto y Jamnia, de modo que murieron más de tres mil hombres.
\par 16 Hecho esto, Judas volvió con su ejército de perseguirlos,
\par 17 Y dijo al pueblo: No codiciéis el botín, ya que tenemos una batalla por delante.
\par 18 Y Gorgias y su ejército están aquí junto a nosotros en la montaña; pero ahora enfréntate a nuestros enemigos y vencelos, y después de esto podrás tomar con valentía el botín.
\par 19 Mientras Judas aún estaba pronunciando estas palabras, apareció una parte de ellos mirando desde el monte:
\par 20 Los cuales, cuando vieron que los judíos habían puesto en fuga a su ejército y quemaban las tiendas, porque el humo que se vio declaró lo hecho:
\par 21 Cuando vieron estas cosas, tuvieron mucho miedo, y al ver también al ejército de Judas en la llanura dispuesto a pelear,
\par 22 Cada uno de ellos huyó a tierra de extraños.
\par 23 Entonces Judas volvió a saquear las tiendas, y allí consiguió mucho oro, plata, seda azul, púrpura del mar y grandes riquezas.
\par 24 Después de esto, regresaron a sus casas, cantaron un cántico de acción de gracias y alabaron al Señor en el cielo, porque es bueno, porque para siempre es su misericordia.
\par 25 Y aquel día Israel tuvo una gran liberación.
\par 26 Todos los extranjeros que habían escapado vinieron y contaron a Lisias lo sucedido:
\par 27 El cual, cuando oyó esto, se confundió y desanimó, porque ni se habían hecho a Israel lo que él quería, ni se habían cumplido las cosas que el rey le había ordenado.
\par 28 El año siguiente, después de Lisias, reunió sesenta mil hombres escogidos de a pie y cinco mil jinetes para someterlos.
\par 29 Llegaron a Idumea y plantaron sus tiendas en Betsur, y Judas les salió al encuentro con diez mil hombres.
\par 30 Y cuando vio aquel poderoso ejército, oró y dijo: Bendito eres, oh Salvador de Israel, que sofocaste la violencia del valiente por mano de tu siervo David, y entregaste al ejército de extraños a la tierra. manos de Jonatán hijo de Saúl, y de su escudero;
\par 31 Encierra este ejército en manos de tu pueblo Israel, y queden avergonzados en su poder y en su gente de a caballo.
\par 32 Haz que pierdan su valor y desvanezcan la valentía de su fuerza, y que tiemblen ante su destrucción.
\par 33 Derribalos con la espada de los que te aman, y todos los que conocen tu nombre te alaban con acción de gracias.
\par 34 Entonces se unieron a la batalla; Y del ejército de Lisias fueron muertos unos cinco mil hombres, incluso antes de que ellos fueran asesinados.
\par 35 Cuando Lisias vio su ejército en fuga y la virilidad de los soldados de Judas, y cómo estaban dispuestos a vivir o morir valientemente, fue a Antioquía, reunió un grupo de extranjeros y, habiendo hecho su ejército mayor que el que era, se propuso volver a Judea.
\par 36 Entonces Judas y sus hermanos dijeron: He aquí, nuestros enemigos están derrotados: subamos a limpiar y dedicar el santuario.
\par 37 Entonces se reunió todo el ejército y subieron al monte de Sión.
\par 38 Y cuando vieron el santuario desolado, y el altar profanado, y las puertas quemadas, y los arbustos creciendo en los atrios como en un bosque o en una de las montañas, y las cámaras de los sacerdotes derribadas,
\par 39 Se rasgaron las vestiduras, hicieron grandes lamentaciones y se echaron ceniza sobre la cabeza.
\par 40 Y cayeron rostro en tierra, tocaron la alarma con las trompetas y gritaron al cielo.
\par 41 Entonces Judas designó a algunos hombres para que pelearan contra los que estaban en la fortaleza, hasta que hubo limpiado el santuario.
\par 42 Entonces escogió sacerdotes de conducta irreprochable, que se deleitaban en la ley.
\par 43 Quien limpió el santuario y sacó las piedras contaminadas en un lugar inmundo.
\par 44 Y mientras consultaban qué hacer con el altar de los holocaustos, que estaba profanado,
\par 45 Pensaron que sería mejor derribarlo, para que no fuera un reproche para ellos, porque los paganos lo habían contaminado. Por eso lo derribaron,
\par 46 Y puso las piedras en el monte del templo en un lugar conveniente, hasta que viniera un profeta para mostrar lo que se debía hacer con ellas.
\par 47 Luego tomaron piedras enteras según la ley y construyeron un nuevo altar según el anterior;
\par 48 Y edificó el santuario y los objetos que había dentro del templo, y santificó los atrios.
\par 49 Hicieron también vasos santos nuevos y llevaron al templo el candelero, el altar de los holocaustos, el incienso y la mesa.
\par 50 Y quemaban incienso sobre el altar y encendían las lámparas que estaban sobre el candelero, para alumbrar el templo.
\par 51 Además pusieron los panes sobre la mesa, extendieron los velos y terminaron todos los trabajos que habían comenzado a hacer.
\par 52 El día veinticinco del mes noveno, llamado Casleu, del año ciento cuarenta y ocho, se levantaron muy de mañana,
\par 53 Y ofrecieron sacrificios según la ley sobre el nuevo altar de holocaustos que habían hecho.
\par 54 Mirad a qué hora y en qué día la habían profanado los paganos, incluso en aquel momento fue consagrada con cánticos, cítaras, arpas y címbalos.
\par 55 Entonces todo el pueblo se postró sobre sus rostros, adorando y alabando al Dios del cielo, que les había dado el éxito.
\par 56 Y así celebraron la dedicación del altar durante ocho días y con alegría ofrecieron holocaustos y sacrificaron sacrificios de liberación y de alabanza.
\par 57 También adornaron la parte delantera del templo con coronas de oro y escudos; y renovaron las puertas y las cámaras, y colgaron sobre ellas puertas.
\par 58 Entonces hubo una gran alegría entre el pueblo, porque el oprobio de las naciones había sido quitado.
\par 59 Además, Judas y sus hermanos, con toda la congregación de Israel, dispusieron que los días de la dedicación del altar se guardaran en su tiempo de año en año durante ocho días, a partir del día veinticinco del día siguiente. mes Casleu, con regocijo y alegría.
\par 60 También en aquel tiempo edificaron el monte Sión con muros altos y torres fuertes alrededor, para que los gentiles no vinieran y lo pisotearan como lo habían hecho antes.
\par 61 Y pusieron allí una guarnición para defenderla, y fortificaron Betsur para protegerla; para que el pueblo tuviera defensa contra Idumea.

\chapter{5}

\par 1 Cuando las naciones de alrededor oyeron que el altar había sido construido y el santuario renovado como antes, les desagradó mucho.
\par 2 Por lo tanto, pensaron en destruir a la generación de Jacob que estaba entre ellos, y entonces comenzaron a matar y destruir al pueblo.
\par 3 Entonces Judas peleó contra los hijos de Esaú en Idumea, en Arabattine, porque tenían sitiados a Gael; y los derrotó grandemente, les quitó el valor y se apoderó de sus despojos.
\par 4 También se acordó de la injuria de los hijos de Bean, que habían sido una trampa y un escándalo para el pueblo, acechándolos en los caminos.
\par 5 Los encerró en las torres, acampó contra ellos, los destruyó por completo y quemó a fuego las torres de aquel lugar y todo lo que en ellas había.
\par 6 Después pasó a los hijos de Amón, donde encontró un gran poder y un gran pueblo, con Timoteo como capitán.
\par 7 Y luchó contra ellos muchas batallas, hasta que al final quedaron derrotados ante él; y él los hirió.
\par 8 Y cuando tomó Jazar y sus ciudades, regresó a Judea.
\par 9 Entonces los paganos que estaban en Galaad se reunieron contra los israelitas que estaban en sus campamentos para destruirlos; pero ellos huyeron a la fortaleza de Dathema.
\par 10 Y envió cartas a Judas y a sus hermanos: Las naciones que nos rodean se han reunido contra nosotros para destruirnos.
\par 11 Y se preparan para venir y tomar la fortaleza a la que hemos huido, siendo Timoteo el capitán de su ejército.
\par 12 Venid, pues, ahora y líbranos de sus manos, porque muchos de nosotros hemos sido asesinados.
\par 13 Sí, todos nuestros hermanos que estaban en el lugar de Tobie fueron ejecutados; también a sus mujeres y a sus hijos se llevaron cautivos y se llevaron sus bienes; y allí destrozaron como mil hombres.
\par 14 Mientras aún se leían estas cartas, he aquí, llegaron otros mensajeros de Galilea con sus vestidos rasgados, y contaron lo siguiente:
\par 15 Y dijo: Los de Tolemaida, los de Tiro, los de Sidón y toda Galilea de los gentiles se han reunido contra nosotros para consumirnos.
\par 16 Cuando Judas y el pueblo oyeron estas palabras, se reunió una gran multitud para consultar qué debían hacer por sus hermanos que estaban en apuros y agredidos por ellos.
\par 17 Entonces Judas dijo a su hermano Simón: Escoge hombres y ve y libera a tus hermanos que están en Galilea, porque mi hermano Jonatán y yo iremos a la tierra de Galaad.
\par 18 Entonces dejó a José, hijo de Zacarías, y a Azarías, capitanes del pueblo, con el resto del ejército en Judea para guardarlo.
\par 19 A ellos les dio esta orden: Tomad el mando de este pueblo y cuidad de no hacer guerra contra las naciones hasta el momento en que regresemos.
\par 20 A Simón le fueron dados tres mil hombres para ir a Galilea, y a Judas ocho mil hombres para el país de Galaad.
\par 21 Entonces Simón fue a Galilea, donde peleó muchas batallas contra los paganos, de modo que los paganos quedaron derrotados por él.
\par 22 Y los persiguió hasta la puerta de Tolemaida; y fueron asesinados entre las naciones unos tres mil hombres, cuyo botín tomó.
\par 23 Y a los que estaban en Galilea y en Arbattis, con sus mujeres y sus hijos y todo lo que tenían, se los llevó consigo y los trajo a Judea con gran alegría.
\par 24 También Judas Macabeo y su hermano Jonatán cruzaron el Jordán y viajaron tres días de camino por el desierto.
\par 25 Allí se encontraron con los nabatitas, quienes se acercaron pacíficamente a ellos y les contaron todo lo que les había sucedido a sus hermanos en la tierra de Galaad:
\par 26 Y cómo muchos de ellos fueron encerrados en Bosora, Bosor, Alema, Casfor, Maked y Carnaim; todas estas ciudades son fuertes y grandes:
\par 27 Y que estaban encerrados en el resto de las ciudades del país de Galaad, y que habían designado para mañana traer su ejército contra las fortalezas, tomarlas y destruirlas a todas en un día. .
\par 28 Entonces Judas y su ejército se dirigieron repentinamente por el camino del desierto hacia Bosora; y cuando tomó la ciudad, mató a todos los varones a filo de espada, tomó todo el botín y quemó la ciudad a fuego.
\par 29 De donde salió de noche y caminó hasta llegar a la fortaleza.
\par 30 Al amanecer, alzaron la vista y vieron que había un pueblo innumerable que llevaba escaleras y otras armas de guerra para tomar la fortaleza, porque los asaltaron.
\par 31 Cuando Judas vio que la batalla había comenzado y que el clamor de la ciudad subía al cielo con trompetas y gran sonido,
\par 32 Dijo a su ejército: Lucha hoy por tus hermanos.
\par 33 Entonces él salió detrás de ellos en tres compañías, que tocaron las trompetas y clamaron en oración.
\par 34 Entonces el ejército de Timoteo, sabiendo que era Macabeo, huyó de él; por lo que los hirió con una gran matanza; De modo que aquel día murieron de ellos unos ocho mil hombres.
\par 35 Hecho esto, Judas se volvió hacia Masfa; y después de haberla asaltado, tomó y mató a todos los varones que había en ella, tomó el botín y la quemó al fuego.
\par 36 De allí partió y tomó Casfón, Maged, Bosor y las demás ciudades del país de Galaad.
\par 37 Después de esto, Timoteo reunió otro ejército y acampó contra Rafón, al otro lado del arroyo.
\par 38 Entonces Judas envió hombres a reconocer al ejército, quienes le avisaron, diciendo: Todas las naciones que están alrededor de nosotros se han reunido con ellos, un ejército muy grande.
\par 39 También ha contratado a los árabes para que los ayuden y ellos han plantado sus tiendas más allá del arroyo, listos para venir y pelear contra ti. Entonces Judas salió a su encuentro.
\par 40 Entonces Timoteo dijo a los capitanes de su ejército: Cuando Judas y su ejército se acerquen al arroyo, si él pasa primero hacia nosotros, no podremos resistirle; porque él prevalecerá poderosamente contra nosotros:
\par 41 Pero si tiene miedo y acampa al otro lado del río, pasaremos a él y prevaleceremos contra él.
\par 42 Cuando Judas llegó cerca del arroyo, hizo que los escribas del pueblo se quedaran junto al arroyo, a quienes dio orden, diciendo: Que nadie se quede en el campamento, sino que todos vengan a la batalla.
\par 43 Entonces él pasó primero hacia ellos, y todo el pueblo después de él; entonces todas las naciones, desconcertadas ante él, arrojaron sus armas y huyeron al templo que estaba en Carnaim.
\par 44 Pero tomaron la ciudad y quemaron el templo con todo lo que había en él. Así fue sometida Carnaim, y ya no pudieron resistir ante Judas.
\par 45 Entonces Judas reunió a todos los israelitas que estaban en el país de Galaad, desde el menor hasta el mayor, sus mujeres, sus hijos y sus pertenencias, en un ejército muy grande, para poder llegar al país. tierra de Judea.
\par 46 Cuando llegaron a Efrón (ésta era una ciudad grande en el camino por el que debían ir, muy bien fortificada), no podían desviarse de ella ni a la derecha ni a la izquierda, sino que tenían que pasar por el medio de ello.
\par 47 Entonces los de la ciudad los cerraron y bloquearon las puertas con piedras.
\par 48 Entonces Judas les envió un mensaje pacífico, diciendo: Pasemos por vuestra tierra para ir a nuestra tierra, y nadie os hará ningún daño. Sólo pasaremos a pie, pero no le quisieron abrir.
\par 49 Entonces Judas ordenó que se hiciera proclamar en todo el ejército que cada uno plantara su tienda en el lugar donde se encontraba.
\par 50 Los soldados acamparon y asaltaron la ciudad todo el día y toda la noche, hasta que al fin la ciudad fue entregada en sus manos.
\par 51 Los cuales mataron a todos los varones a filo de espada, arrasaron la ciudad, tomaron su botín y atravesaron la ciudad sobre los muertos.
\par 52 Después de esto cruzaron el Jordán y llegaron a la gran llanura que está frente a Betsán.
\par 53 Entonces Judas reunió a los que venían detrás y exhortó al pueblo durante todo el camino hasta llegar a la tierra de Judea.
\par 54 Entonces subieron con gozo y alegría al monte Sión, donde ofrecieron holocaustos, porque ninguno de ellos había sido asesinado hasta que regresaron en paz.
\par 55 Cuando estaban Judas y Jonatán en la tierra de Galaad, y Simón su hermano en Galilea delante de Tolemaida,
\par 56 José, hijo de Zacarías, y Azarías, capitanes de las guarniciones, se enteraron de las hazañas y las hazañas de guerra que habían realizado.
\par 57 Entonces dijeron: Consigamos también nosotros un nombre y vayamos a luchar contra los paganos que nos rodean.
\par 58 Entonces, cuando dieron el mando a la guarnición que estaba con ellos, se dirigieron hacia Jamnia.
\par 59 Entonces Gorgias y sus hombres salieron de la ciudad para luchar contra ellos.
\par 60 Y sucedió que José y Azaras fueron puestos en fuga y perseguidos hasta las fronteras de Judea; y aquel día fueron asesinados del pueblo de Israel unos dos mil hombres.
\par 61 Se produjo así una gran destrucción entre los hijos de Israel, porque no obedecieron a Judas y a sus hermanos, sino que pensaron en realizar alguna acción valiente.
\par 62 Además, estos hombres no procedían de la descendencia de aquellos por cuya mano se dio la salvación a Israel.
\par 63 Sin embargo, Judas y sus hermanos eran muy famosos ante los ojos de todo Israel y de todas las naciones dondequiera que se oyera su nombre;
\par 64 El pueblo se reunió con ellos con aclamaciones de júbilo.
\par 65 Después salió Judas con sus hermanos y peleó contra los hijos de Esaú en la tierra del sur, donde derrotó a Hebrón y a sus ciudades, derribó su fortaleza y quemó sus torres alrededor. .
\par 66 De allí partió para ir a la tierra de los filisteos y pasó por Samaria.
\par 67 En aquel tiempo, algunos sacerdotes, deseosos de demostrar su valor, fueron muertos en la batalla, por lo que salieron a pelear imprudentemente.
\par 68 Entonces Judas se volvió hacia Azoto en la tierra de los filisteos, y después de derribar sus altares, quemar al fuego sus imágenes talladas y saquear sus ciudades, regresó a la tierra de Judea.

\chapter{6}

\par 1 Por aquel tiempo, el rey Antíoco, que viajaba por las tierras altas, oyó decir que Elimais, en el país de Persia, era una ciudad muy famosa por sus riquezas, plata y oro;
\par 2 Y que había en él un templo muy rico, en el que estaban las cubiertas de oro, las corazas y los escudos que Alejandro, hijo de Filipo, el rey macedonio, el primero en reinar entre los griegos, había dejado allí.
\par 3 Por eso vino y trató de tomar la ciudad y saquearla; pero no pudo, porque los de la ciudad, advertidos de ello,
\par 4 Se levantó contra él en batalla, y él huyó, partió de allí con gran angustia y regresó a Babilonia.
\par 5 Llegó además uno que le trajo la noticia a Persia de que los ejércitos que atacaban la tierra de Judea habían sido puestos en fuga.
\par 6 Y que Lisias, que había salido primero con gran poder, fue expulsado de los judíos; y que se fortalecieron con la armadura, el poder y el botín que habían obtenido de los ejércitos que habían destruido:
\par 7 También que habían derribado la abominación que él había puesto sobre el altar en Jerusalén, y que habían rodeado el santuario con altos muros, como antes, y su ciudad, Betsur.
\par 8 Cuando el rey oyó estas palabras, quedó estupefacto y conmovido; entonces lo acostó en su cama y cayó enfermo de tristeza, porque no le había sucedido lo que esperaba.
\par 9 Y permaneció allí muchos días, porque su dolor era cada vez mayor y pensaba que iba a morir.
\par 10 Entonces llamó a todos sus amigos y les dijo: El sueño se ha ido de mis ojos y mi corazón desfallece por muchas preocupaciones.
\par 11 Y pensé dentro de mí mismo: ¿En qué tribulación he llegado, y cuán grande es la inundación de miseria en la que me encuentro ahora? porque fui generoso y amado en mi poder.
\par 12 Pero ahora me acuerdo de los males que hice en Jerusalén, y que tomé todos los objetos de oro y plata que había en ella, y envié a destruir a los habitantes de Judea sin causa.
\par 13 Por tanto, comprendo que por esta causa me sobrevienen estas dificultades, y he aquí que perezco de gran dolor en tierra extraña.
\par 14 Entonces llamó a Felipe, uno de sus amigos, a quien nombró gobernante de todo su reino,
\par 15 Y le dio la corona, el manto y el sello para que criara a su hijo Antíoco y lo criara para el reino.
\par 16 Y murió allí el rey Antíoco en el año ciento cuarenta y nueve.
\par 17 Cuando Lisias supo que el rey había muerto, puso en su lugar a su hijo Antíoco, a quien había criado siendo joven, para que reinara en su lugar, y llamó su nombre Eupátor.
\par 18 Por aquel tiempo los que estaban en la torre encerraron a los israelitas alrededor del santuario, y buscaban siempre su mal y el fortalecimiento de los paganos.
\par 19 Entonces Judas, queriendo destruirlos, convocó a todo el pueblo para sitiarlos.
\par 20 Entonces se reunieron y los sitiaron en el año ciento cincuenta, y él hizo contra ellos monturas para tiro y otras armas.
\par 21 Sin embargo, algunos de los sitiados salieron, a quienes se unieron algunos hombres impíos de Israel:
\par 22 Y fueron al rey y le dijeron: ¿Hasta cuándo ejecutarás tu juicio y vengarás a nuestros hermanos?
\par 23 Hemos querido servir a tu padre, hacer lo que él quisiera y obedecer sus mandamientos;
\par 24 Por esta razón, los de nuestra nación sitiaron la torre y se alejaron de nosotros; además, mataron a cuantos de nosotros pudieron encontrar y saquearon nuestra herencia.
\par 25 No sólo han extendido su mano contra nosotros, sino también contra sus fronteras.
\par 26 Y he aquí, hoy están sitiando la torre de Jerusalén para tomarla; también han fortificado el santuario y Betsur.
\par 27 Por lo tanto, si no los previenes rápidamente, harán cosas mayores que éstas, y tú no podrás gobernarlos.
\par 28 Cuando el rey oyó esto, se enojó y reunió a todos sus amigos, a los capitanes de su ejército y a los que estaban a cargo de la caballería.
\par 29 También vinieron a él bandas de soldados a sueldo de otros reinos y de las islas del mar.
\par 30 De modo que el número de su ejército era cien mil hombres de a pie, veinte mil jinetes y treinta y dos elefantes ejercitados en la batalla.
\par 31 Estos atravesaron Idumea y acamparon contra Betsur, y la asaltaron durante muchos días, haciendo máquinas de guerra; pero los de Betsur salieron, los quemaron al fuego y pelearon valientemente.
\par 32 Entonces Judas abandonó la torre y acampó en Batzacarías, frente al campamento del rey.
\par 33 Entonces el rey, madrugando mucho, marchó ferozmente con su ejército hacia Batzacarías, donde sus ejércitos los prepararon para la batalla y tocaron las trompetas.
\par 34 Y para provocar a los elefantes a pelear, les mostraron sangre de uvas y moras.
\par 35 Además, dividieron las bestias entre los ejércitos, y por cada elefante designaron mil hombres, armados con cotas de malla y con cascos de bronce en la cabeza; y además de esto, por cada bestia se ordenaron quinientos jinetes de los mejores.
\par 36 Éstos estaban preparados para todo momento: dondequiera que estuviese la bestia y a dondequiera que fuese, ellos también iban y no se apartaban de ella.
\par 37 Y sobre las bestias había fuertes torres de madera, que las cubrían a cada una de ellas, y estaban ceñidas con armas; también había sobre cada una treinta y dos hombres fuertes que luchaban contra ellas, además de los indios. que lo gobernaba.
\par 38 En cuanto al resto de los jinetes, los colocaron de un lado y de otro en las dos partes del ejército, dándoles señales de lo que debían hacer, y enjaezándolos por todas partes en medio de las filas.
\par 39 Cuando el sol brillaba sobre los escudos de oro y bronce, las montañas resplandecían con ellos y brillaban como lámparas de fuego.
\par 40 Así que una parte del ejército del rey se dispersó en las altas montañas y otra parte en los valles de abajo, y marcharon con seguridad y orden.
\par 41 Por lo cual todos los que oyeron el ruido de su multitud, el paso de la compañía y el ruido de los arneses, se conmovieron; porque el ejército era muy grande y poderoso.
\par 42 Entonces Judas y su ejército se acercaron y entraron en batalla, y murieron seiscientos hombres del ejército del rey.
\par 43 También Eleazar, de sobrenombre Savaran, vio que uno de los animales, armado con arneses reales, era más alto que todos los demás, y supuso que el rey estaba sobre él,
\par 44 Se puso en peligro para liberar a su pueblo y conseguirle un nombre perpetuo.
\par 45 Entonces corrió valientemente hacia él en medio de la batalla, matando a diestra y siniestra, de modo que quedaron separados de él por ambos bandos.
\par 46 Hecho esto, se deslizó debajo del elefante, lo empujó debajo y lo mató; entonces el elefante cayó sobre él y allí murió.
\par 47 Pero los demás judíos, viendo la fuerza del rey y la violencia de sus tropas, se apartaron de ellos.
\par 48 Entonces el ejército del rey subió a Jerusalén para recibirlos, y el rey plantó sus tiendas frente a Judea y frente al monte Sión.
\par 49 Pero hizo las paces con los que estaban en Betsur, porque salieron de la ciudad porque no tenían allí víveres para soportar el asedio, siendo este un año de descanso para la tierra.
\par 50 Entonces el rey tomó Betsur y puso allí guarnición para protegerla.
\par 51 En cuanto al santuario, lo sitió durante muchos días y colocó allí artillería con máquinas e instrumentos para arrojar fuego y piedras, y piezas para lanzar dardos y hondas.
\par 52 Entonces ellos también construyeron máquinas contra sus máquinas y las mantuvieron en batalla durante una larga temporada.
\par 53 Pero al final, estando sus vasijas sin víveres (pues era el año séptimo, y los que en Judea habían sido librados de los gentiles, habían comido el resto del almacén);
\par 54 Quedaban pocos en el santuario, porque el hambre era tal que los atacaba, que de buena gana se dispersaban, cada uno a su lugar.
\par 55 En aquel tiempo oyó decir Lisias que Felipe, a quien el rey Antíoco, mientras vivía, había encargado de criar a su hijo Antíoco para que fuera rey,
\par 56 Había regresado de Persia y de Media, y también el ejército del rey que iba con él, y que buscaba tomar para él el gobierno de los asuntos.
\par 57 Entonces fue a toda prisa y dijo al rey y a los capitanes del ejército y de la compañía: Cada día nos descomponemos, y nuestras provisiones son escasas, y el lugar que sitiamos es fuerte, y los asuntos de el reino recaiga sobre nosotros:
\par 58 Ahora, pues, seamos amigos de estos hombres y hagamos la paz con ellos y con toda su nación;
\par 59 Y pacta con ellos que vivirán según sus leyes como antes; porque por eso están disgustados y han hecho todas estas cosas porque abolimos sus leyes.
\par 60 Entonces el rey y los príncipes estaban contentos, por lo que les envió a hacer la paz; y ellos lo aceptaron.
\par 61 También el rey y los príncipes les hicieron un juramento y salieron de la fortaleza.
\par 62 Entonces el rey entró en el monte Sión; pero cuando vio la fortaleza del lugar, rompió el juramento que había hecho y mandó derribar el muro alrededor.
\par 63 Después partió a toda prisa y regresó a Antioquía, donde encontró que Felipe era dueño de la ciudad; entonces peleó contra él y tomó la ciudad por la fuerza.

\chapter{7}

\par 1 En el año ciento cincuenta y uno, Demetrio, hijo de Seleuco, salió de Roma y subió con unos pocos hombres a una ciudad de la costa del mar, y reinó allí.
\par 2 Y cuando entró en el palacio de sus antepasados, sus tropas habían tomado a Antíoco y a Lisias para llevárselos.
\par 3 Entonces, cuando lo supo, dijo: No me dejes ver sus rostros.
\par 4 Entonces su ejército los mató. Ahora bien, cuando Demetrio fue puesto en el trono de su reino,
\par 5 Vinieron a él todos los hombres malvados e impíos de Israel, teniendo por capitán a Alcimo, que deseaba ser sumo sacerdote:
\par 6 Y acusaron al pueblo ante el rey, diciendo: Judas y sus hermanos han matado a todos tus amigos y nos han expulsado de nuestra tierra.
\par 7 Ahora, pues, envía a algún hombre en quien confíes y que vaya a ver los estragos que ha causado entre nosotros y en la tierra del rey, y que los castigue con todos los que los ayudan.
\par 8 Entonces el rey eligió a Báquides, amigo del rey, que reinó más allá del diluvio, y que era un hombre importante en el reino y fiel al rey,
\par 9 Y lo envió con el malvado Alcimo, a quien nombró sumo sacerdote, y le ordenó vengarse de los hijos de Israel.
\par 10 Partieron, pues, y llegaron con gran poder a la tierra de Judea, donde enviaron engañosamente mensajeros a Judas y a sus hermanos con palabras pacíficas.
\par 11 Pero ellos no hicieron caso de sus palabras; porque vieron que habían venido con un gran poder.
\par 12 Entonces se reunió con Alcimo y Báquides un grupo de escribas para exigir justicia.
\par 13 Los asideos fueron los primeros entre los hijos de Israel en buscar la paz de ellos.
\par 14 Porque decían: Uno de los descendientes de Aarón ha venido con este ejército y no nos hará ningún mal.
\par 15 Entonces él les habló pacíficamente y les juró, diciendo: Ni a vosotros ni a vuestros amigos os haremos daño ni a vosotros ni a vuestros amigos.
\par 16 Entonces ellos le creyeron, pero él tomó de ellos sesenta hombres y los mató en un día, conforme a las palabras que había escrito:
\par 17 Echaron fuera la carne de tus santos y derramaron su sangre alrededor de Jerusalén, y no hubo quien los sepultara.
\par 18 Por eso el temor y el temor de ellos cayó sobre todo el pueblo, que decía: No hay verdad ni justicia en ellos; porque han roto el pacto y el juramento que hicieron.
\par 19 Después de esto, Báquides salió de Jerusalén y plantó sus tiendas en Bezeth, donde envió y tomó a muchos de los hombres que lo habían abandonado, y también a algunos del pueblo, y después de matarlos, los arrojó en el gran pozo.
\par 20 Entonces entregó el país a Alcimo y le dejó un poder que le ayudaría; así Báquides acudió al rey.
\par 21 Pero Alcimo contendió por el sumo sacerdocio.
\par 22 Y acudieron a él todos los que alborotaban al pueblo, los cuales, después de haber tomado la tierra de Judá en su poder, causaron mucho daño a Israel.
\par 23 Cuando Judas vio todos los males que Alcimo y su compañía habían hecho entre los israelitas, incluso entre los paganos,
\par 24 Salió por todos los territorios de Judea y se vengó de los que se habían rebelado contra él, de modo que no se atrevieron a salir más al país.
\par 25 Por otro lado, cuando Alcimo vio que Judas y su compañía habían tomado la delantera, y supo que no podía soportar su fuerza, fue otra vez al rey y les dijo lo peor de ellos, que él podría.
\par 26 Entonces el rey envió a Nicanor, uno de sus honorables príncipes, un hombre que odiaba mortalmente a Israel, con la orden de destruir al pueblo.
\par 27 Entonces Nicanor llegó a Jerusalén con gran fuerza; y envió engañosamente palabras amistosas a Judas y a sus hermanos, diciendo:
\par 28 No haya batalla entre tú y yo; Vendré con algunos hombres para poder veros en paz.
\par 29 Llegó entonces a Judas y se saludaron pacíficamente. Sin embargo, los enemigos estaban dispuestos a llevarse a Judas por la violencia.
\par 30 Lo cual, cuando Judas supo que había venido a él con engaño, tuvo mucho miedo de él y no quiso ver más su rostro.
\par 31 También Nicanor, cuando vio que su consejo era descubierto, salió a pelear contra Judas junto a Cafarsalama:
\par 32 Allí fueron asesinados unos cinco mil hombres del bando de Nicanor, y el resto huyó a la ciudad de David.
\par 33 Después de esto, Nicanor subió al monte Sión, y salieron del santuario algunos de los sacerdotes y algunos de los ancianos del pueblo para saludarlo pacíficamente y mostrarle el holocausto que se ofrecía por el rey. .
\par 34 Pero él se burlaba de ellos, se reía de ellos, los insultaba vergonzosamente y hablaba con altivez:
\par 35 Y juró en su ira, diciendo: Si Judas y su ejército no son entregados ahora en mis manos, si alguna vez vuelvo sano y salvo, quemaré esta casa. Y con esto salió furioso.
\par 36 Entonces los sacerdotes entraron y se pararon delante del altar y del templo, llorando y diciendo:
\par 37 Tú, oh Señor, escogiste esta casa para que llevara tu nombre y fuera casa de oración y petición para tu pueblo.
\par 38 Véngate de este hombre y de su ejército, y déjalos caer a espada; recuerda sus blasfemias y no permitas que continúen más.
\par 39 Entonces Nicanor salió de Jerusalén y plantó sus tiendas en Bethorón, donde lo encontró un ejército de Siria.
\par 40 Pero Judas acampó en Adasa con tres mil hombres, y allí oró, diciendo:
\par 41 Oh Señor, cuando los enviados por el rey de Asiria blasfemaron, tu ángel salió e hirió a ciento ochenta y cinco mil de ellos.
\par 42 Destruye hoy este ejército delante de nosotros, para que los demás sepan que ha blasfemado contra tu santuario, y júzgalo según su maldad.
\par 43 Así que el día trece del mes de Adar los ejércitos entraron en batalla, pero el ejército de Nicanor quedó derrotado y él mismo fue el primero en morir en la batalla.
\par 44 Cuando el ejército de Nicanor vio que había sido asesinado, arrojaron sus armas y huyeron.
\par 45 Entonces los persiguieron durante un día de camino, desde Adasa hasta Gazera, tocando tras ellos la alarma con sus trompetas.
\par 46 Entonces salieron de todas las ciudades de Judea de los alrededores y las cercaron; de modo que ellos, volviéndose contra los que los perseguían, fueron todos muertos a espada, y no quedó ninguno de ellos.
\par 47 Después tomaron el botín y el botín, cortaron a Nicanor la cabeza y la mano derecha que con tanto orgullo extendía, y se los llevaron y los colgaron hacia Jerusalén.
\par 48 Por esto el pueblo se alegró mucho y celebraron aquel día como un día de gran alegría.
\par 49 Además, ordenaron que se celebrara anualmente este día, que es el trece de Adar.
\par 50 Así la tierra de Judá estuvo en reposo por un poco de tiempo.

\chapter{8}

\par 1 Judas había oído hablar de los romanos, que eran hombres fuertes y valientes, que aceptaban con amor a todos los que se unían a ellos y hacían una alianza de amistad con todos los que venían a ellos;
\par 2 Y que eran hombres de gran valor. Le contaron también las guerras y los nobles actos que habían realizado entre los gálatas, y cómo los habían conquistado y sometido a tributo;
\par 3 Y lo que habían hecho en el país de España, para la explotación de las minas de plata y oro que allí hay;
\par 4 Y que con su política y paciencia habían conquistado todo el lugar, aunque estaba muy lejos de ellos; y también a los reyes que vinieron contra ellos desde lo último de la tierra, hasta que los desconcertaron y les dieron gran destrucción, de modo que los demás les daban tributo cada año:
\par 5 Además, cómo habían derrotado en la batalla a Filipo y a Perseo, rey de los ciudadanos, y a otros que se habían levantado contra ellos y los habían vencido:
\par 6 Cómo también Antíoco, el gran rey de Asia, que venía contra ellos en batalla, teniendo ciento veinte elefantes, gente de a caballo, carros y un ejército muy grande, fue derrotado por ellos;
\par 7 Y cómo lo capturaron vivo y acordaron que él y los que reinarían después de él pagarían un gran tributo y entregarían rehenes, y lo acordado,
\par 8 Y las tierras de la India, Media, Lidia y las mejores tierras que le arrebataron y se las dieron al rey Eumenes:
\par 9 Además, cómo los griegos habían decidido venir y destruirlos;
\par 10 Y que ellos, sabiendo esto, enviaron contra ellos un capitán que, peleando con ellos, mató a muchos de ellos, y se llevó cautivos a sus mujeres y a sus hijos, los despojó, se apoderó de sus tierras y los destruyó. sus fortalezas, y los trajo para que fueran sus siervos hasta el día de hoy:
\par 11 Le contaron además cómo destruyeron y sometieron a su dominio a todos los demás reinos e islas que en algún momento se les resistieron;
\par 12 Pero mantenían amistad con sus amigos y con quienes confiaban en ellos, y que habían conquistado reinos lejanos y cercanos, de modo que todos los que oían su nombre les tenían miedo.
\par 13 Además, aquellos a quienes quieren ayudar a tener un reino, esos reinan; y a quienes nuevamente quisieran, los desplazan: finalmente, que fueron muy exaltados:
\par 14 Sin embargo, a pesar de todo esto, ninguno de ellos llevaba corona ni se vestía de púrpura para ser engrandecido con ello.
\par 15 Además, se habían construido un Senado en el que trescientos veinte hombres se reunían cada día en consejo, consultando siempre al pueblo, para que estuvieran bien ordenados.
\par 16 Y que confiaban su gobierno a un hombre cada año, que gobernaba todo su país, y que todos eran obedientes a él, y que no había envidia ni emulación entre ellos.
\par 17 Teniendo en cuenta estas cosas, Judas escogió a Eupólemo, hijo de Juan, hijo de Accos, y a Jasón, hijo de Eleazar, y los envió a Roma para hacer con ellos una alianza y una confederación.
\par 18 Y para rogarles que les quitaran el yugo; porque vieron que el reino de los griegos oprimió a Israel con servidumbre.
\par 19 Fueron entonces a Roma, que era un viaje muy largo, y llegaron al Senado, donde hablaron y dijeron.
\par 20 Judas Macabeo, con sus hermanos y el pueblo judío, nos ha enviado a vosotros para hacer confederación y paz con vosotros, y para que seamos registrados como vuestros aliados y amigos.
\par 21 De modo que esto agradó mucho a los romanos.
\par 22 Y ésta es la copia de la epístola que el Senado redactó nuevamente en tablas de bronce y envió a Jerusalén para tener allí un memorial de paz y confederación:
\par 23 Buena suerte para los romanos y para el pueblo judío, por mar y por tierra para siempre; la espada y el enemigo estén lejos de ellos,
\par 24 Si primero se produce una guerra contra los romanos o contra alguno de sus aliados en todo su dominio,
\par 25 El pueblo de los judíos los ayudará, en el momento señalado, con todo su corazón.
\par 26 Tampoco darán nada a los que les hacen la guerra, ni les ayudarán con víveres, armas, dinero o barcos, como a los romanos les pareció bien; pero guardarán sus pactos sin tomar nada por ello.
\par 27 De la misma manera, si primero llega la guerra a la nación de los judíos, los romanos la ayudarán de todo corazón, según el tiempo que se les haya señalado.
\par 28 A los que se oponen a ellos no se les dará víveres, ni armas, ni dinero, ni barcos, como a los romanos les pareció bien; pero guardarán sus pactos, y eso sin engaño.
\par 29 Según estos artículos los romanos hicieron un pacto con el pueblo judío.
\par 30 Sin embargo, si en adelante una u otra parte piensan reunirse para agregar o disminuir algo, podrán hacerlo a su gusto, y todo lo que agreguen o quiten será ratificado.
\par 31 Y en cuanto a los males que Demetrio hace a los judíos, le hemos escrito, diciendo: ¿Por qué endureciste tu yugo sobre nuestros amigos y aliados de los judíos?
\par 32 Si, pues, se quejan más contra ti, les haremos justicia y pelearemos contigo por mar y por tierra.

\chapter{9}

\par 1 Además, cuando Demetrio se enteró de que Nicanor y su ejército habían sido muertos en la batalla, envió por segunda vez a Báquides y a Alcimo a la tierra de Judea, y con ellos a la fuerza principal de su ejército:
\par 2 Los cuales fueron por el camino que lleva a Galgala y plantaron sus tiendas delante de Masaloth, que está en Arbela, y después de conquistarla mataron a mucha gente.
\par 3 También en el primer mes del año ciento cincuenta y dos acamparon delante de Jerusalén.
\par 4 De allí partieron y se dirigieron a Berea con veinte mil hombres de a pie y dos mil de a caballo.
\par 5 Judas había plantado sus tiendas en Eleasa, y con él tres mil hombres escogidos:
\par 6 Los cuales, al ver la multitud del otro ejército tan grande, tuvieron mucho miedo; Entonces muchos salieron del ejército, de modo que no se quedaron entre ellos más que ochocientos hombres.
\par 7 Entonces Judas, cuando vio que su ejército se escapaba y que la batalla lo apremiaba, se turbó mucho y se afligió mucho, porque no tenía tiempo de reunirlos.
\par 8 Pero a los que quedaron les dijo: Levantémonos y subamos contra nuestros enemigos, por si acaso podemos luchar contra ellos.
\par 9 Pero ellos lo desanimaron, diciendo: Nunca podremos; mejor salvemos ahora nuestras vidas, y en el futuro regresaremos con nuestros hermanos y pelearemos contra ellos, porque somos pocos.
\par 10 Entonces Judas dijo: ¡Dios me libre de hacer esto y huir de ellos! Si llega nuestro momento, muramos valientemente por nuestros hermanos y no manchemos nuestro honor.
\par 11 Dicho esto, el ejército de Báquides salió de sus tiendas y se puso frente a ellos; sus jinetes estaban divididos en dos tropas, y sus honderos y arqueros iban delante del ejército y los que marchaban al frente eran todos hombres valientes.
\par 12 Báquides estaba en el ala derecha, y el ejército se acercó por ambas partes y tocaron las trompetas.
\par 13 También los del bando de Judas tocaron sus trompetas, de modo que la tierra tembló ante el estruendo de los ejércitos, y la batalla continuó desde la mañana hasta la noche.
\par 14 Cuando Judas vio que Báquides y el ejército de su ejército estaban a favor, tomó consigo a todos los hombres valientes.
\par 15 Los cuales derrotaron al ala derecha y los persiguieron hasta el monte Azoto.
\par 16 Pero cuando los del ala izquierda vieron que los del ala derecha estaban derrotados, siguieron a Judas y a los que le seguían por detrás:
\par 17 Entonces se produjo una encarnizada batalla, en la que murieron muchos de ambas partes.
\par 18 También Judas fue asesinado, y el resto huyó.
\par 19 Entonces Jonatán y Simón tomaron a Judas, su hermano, y lo sepultaron en el sepulcro de sus padres en Modín.
\par 20 Además lo lloraron, y todo Israel hizo gran lamentación por él y se lamentó durante muchos días, diciendo:
\par 21 ¡Cómo ha caído el hombre valiente que libró a Israel!
\par 22 En cuanto a las demás cosas acerca de Judas y sus guerras, y las nobles acciones que realizó, y sus grandezas, no están escritas, porque eran muchísimas.
\par 23 Después de la muerte de Judas, los impíos comenzaron a extender sus cabezas por todos los territorios de Israel, y se levantaron todos los que hacían iniquidad.
\par 24 También en aquellos días hubo una gran hambre, por lo que el país se rebeló y se fue con ellos.
\par 25 Entonces Báquides escogió a los malvados y los hizo señores del país.
\par 26 Investigaron y buscaron a los amigos de Judas y se los llevaron a Báquides, quien se vengó de ellos y los ultrajó.
\par 27 Hubo entonces en Israel una gran aflicción, como no se había visto desde el tiempo que no se había visto entre ellos profeta.
\par 28 Por esto se reunieron todos los amigos de Judas y dijeron a Jonatán:
\par 29 Desde que murió tu hermano Judas, no tenemos nadie como él para salir contra nuestros enemigos, contra Báquides y contra los de nuestra nación que son nuestros adversarios.
\par 30 Ahora, pues, te hemos elegido hoy para que seas nuestro príncipe y capitán en su lugar, para que pelees nuestras batallas.
\par 31 Entonces Jonatán asumió el poder en aquel momento y se levantó en lugar de su hermano Judas.
\par 32 Pero cuando Báquides se enteró de esto, intentó matarlo.
\par 33 Al enterarse Jonatán, su hermano Simón y todos los que estaban con él, huyeron al desierto de Tecoe y plantaron sus tiendas junto al agua del estanque de Asfar.
\par 34 Cuando Báquides entendió lo cual, se acercó al Jordán con todo su ejército un día de reposo.
\par 35 Jonatán había enviado a su hermano Juan, capitán del pueblo, a rogar a sus amigos los nabatitas que les dejaran su carruaje, que era mucho.
\par 36 Pero los hijos de Jambri salieron de Medaba, tomaron a Juan y todo lo que tenía y se fueron con ello.
\par 37 Después de esto, llegó la noticia a Jonatán y a su hermano Simón de que los hijos de Jambri habían hecho una gran boda y traían a la novia de Nadabata con una gran comitiva, como si fuera hija de uno de los grandes príncipes de Canaán.
\par 38 Entonces se acordaron de Juan, su hermano, y subieron y se escondieron al amparo del monte.
\par 39 Cuando alzaron los ojos y miraron, he aquí, había mucho alboroto y gran alboroto: y el novio, y sus amigos y hermanos, salieron a recibirlos con tambores, instrumentos de música y muchas canciones. armas.
\par 40 Entonces Jonatán y los que estaban con él se levantaron contra ellos desde el lugar donde estaban emboscados y los mataron de tal manera que muchos cayeron muertos, y el resto huyó a la montaña, y tomó todo su botín.
\par 41 Así las bodas se convirtieron en luto, y el ruido de su melodía en lamentación.
\par 42 Así que, cuando hubieron vengado plenamente la sangre de su hermano, regresaron al pantano del Jordán.
\par 43 Cuando Báquides se enteró de esto, llegó un día de reposo a las orillas del Jordán con gran poder.
\par 44 Entonces Jonatán dijo a su compañía: Subamos ahora y luchemos por nuestras vidas, porque ya no nos aguanta hoy como antes.
\par 45 Porque he aquí, la batalla está delante y detrás de nosotros, y el agua del Jordán de un lado y de otro, el pantano también y el bosque, y no hay lugar para que nos desviemos.
\par 46 Por tanto, clamad ahora al cielo para que seáis librados de la mano de vuestros enemigos.
\par 47 Entonces trabaron batalla, y Jonatán extendió su mano para herir a Báquides, pero él se apartó de él.
\par 48 Entonces Jonatán y los que estaban con él saltaron al Jordán y nadaron hasta la otra orilla, pero la otra orilla no pasó hasta ellos.
\par 49 Así que aquel día murieron unos mil hombres del bando de Báquides.
\par 50 Después Báquides regresó a Jerusalén y reparó las ciudades fuertes de Judea; Fortaleció las fortalezas de Jericó, Emaús, Bethorón, Betel, Tamnata, Faratoni y Tafón, con muros altos, puertas y cerrojos.
\par 51 Y puso en ellos una guarnición para hacer maldad contra Israel.
\par 52 Fortificó también la ciudad de Betsur, Gazera y la torre, y puso en ellas tropas y provisiones.
\par 53 Además, tomó como rehenes a los hijos de los principales hombres del país y los metió en la torre de Jerusalén para que los custodiaran.
\par 54 Además, en el año ciento cincuenta y tres, en el mes segundo, Alcimo ordenó que se derribara el muro del atrio interior del santuario; derribó también las obras de los profetas
\par 55 Y cuando empezaba a derribar, en aquel mismo momento Alcimo sufrió una plaga y sus empresas se vieron obstaculizadas: porque se le cerró la boca y quedó paralizado, de modo que ya no podía hablar ni dar nada. orden relativa a su casa.
\par 56 Entonces Alcimo murió en gran tormento.
\par 57 Cuando Báquides vio que Alcimo había muerto, volvió al rey, y la tierra de Judea estuvo en reposo durante dos años.
\par 58 Entonces todos los impíos se reunieron en consejo y dijeron: He aquí que Jonatán y su compañía están tranquilos y viven tranquilos. Ahora, pues, traeremos aquí a Báquides, que los capturará a todos en una noche.
\par 59 Entonces fueron y consultaron con él.
\par 60 Entonces se fue, vino con un gran ejército y envió cartas en secreto a sus seguidores en Judea, para que capturaran a Jonatán y a los que estaban con él; pero no pudieron, porque conocían su consejo.
\par 61 Entonces tomaron de los hombres del país que habían hecho aquel mal, unas cincuenta personas, y las mataron.
\par 62 Después Jonatán, Simón y los que estaban con él se llevaron a Bet-basi, que está en el desierto, y repararon sus deterioros y la reforzaron.
\par 63 Lo supo Báquides, reunió a todo su ejército y envió un mensaje a los de Judea.
\par 64 Entonces fue y puso cerco a Bet-basi; y lucharon contra él durante mucho tiempo y fabricaron máquinas de guerra.
\par 65 Pero Jonatán dejó a su hermano Simón en la ciudad y salió al campo, y salió con un cierto número.
\par 66 Y derrotó a Odonarkes, a sus hermanos y a los hijos de Fasirón en su tienda.
\par 67 Y cuando comenzó a atacarlos y subió con sus fuerzas, Simón y su compañía salieron de la ciudad y quemaron las armas de guerra.
\par 68 Y luchó contra Báquides, quien fue derrotado por ellos y lo afligieron gravemente, porque sus consejos y sus esfuerzos fueron en vano.
\par 69 Por lo tanto, se enojó mucho con los hombres malvados que le aconsejaron venir al país, ya que mató a muchos de ellos y se propuso regresar a su propio país.
\par 70 Cuando Jonatán se enteró de esto, le envió embajadores para hacer la paz con él y entregarles los prisioneros.
\par 71 Lo cual aceptó, e hizo según sus demandas, y le juró que nunca le haría daño en todos los días de su vida.
\par 72 Entonces, cuando le devolvió a los prisioneros que había tomado antes en la tierra de Judea, regresó y se fue a su propia tierra, y nunca más volvió a sus fronteras.
\par 73 Así cesó la espada en Israel; pero Jonatán se quedó en Macmas y comenzó a gobernar al pueblo; y destruyó a los hombres impíos de Israel.

\chapter{10}

\par 1 En el año ciento sesenta, Alejandro, hijo de Antíoco, llamado Epífanes, subió y tomó a Tolemaida, porque el pueblo lo había recibido y por medio de él reinaba allí.
\par 2 Cuando el rey Demetrio se enteró de esto, reunió un ejército muy grande y salió contra él para pelear.
\par 3 Además, Demetrio envió cartas a Jonatán con palabras amorosas, para engrandecerlo.
\par 4 Porque dijo: Primero hagamos la paz con él, antes de que se una a Alejandro contra nosotros.
\par 5 De lo contrario, se acordará de todos los males que hemos hecho contra él, sus hermanos y su pueblo.
\par 6 Por lo cual le dio autoridad para reunir un ejército y proporcionarle armas para ayudarlo en la batalla; también ordenó que le entregaran los rehenes que estaban en la torre.
\par 7 Entonces Jonatán llegó a Jerusalén y leyó las cartas en presencia de todo el pueblo y de los que estaban en la torre:
\par 8 Los cuales tuvieron mucho miedo al oír que el rey le había dado autoridad para reunir un ejército.
\par 9 Entonces los de la torre entregaron sus rehenes a Jonatán, y él los entregó a sus padres.
\par 10 Hecho esto, Jonatán se instaló en Jerusalén y comenzó a edificar y reparar la ciudad.
\par 11 Y ordenó a los obreros que construyeran las murallas y el monte Sión y sus alrededores con piedras cuadradas para fortificación; y así lo hicieron.
\par 12 Entonces los extranjeros que estaban en las fortalezas que Báquides había construido huyeron;
\par 13 Cada uno dejó su lugar y se fue a su propia tierra.
\par 14 Sólo en Betsur permanecieron algunos de los que habían abandonado la ley y los mandamientos, porque era su lugar de refugio.
\par 15 Cuando el rey Alejandro oyó las promesas que Demetrio había hecho a Jonatán, cuando también le contó las batallas y las nobles hazañas que él y sus hermanos habían hecho, y los dolores que habían soportado,
\par 16 Él dijo: ¿Encontraremos otro hombre así? Ahora pues, haremos de él nuestro amigo y cómplice.
\par 17 Sobre esto escribió una carta y se la envió con estas palabras, diciendo:
\par 18 El rey Alejandro envía un saludo a su hermano Jonatán:
\par 19 Hemos oído hablar de ti que eres un hombre poderoso y digno de ser nuestro amigo.
\par 20 Por eso hoy te constituimos sumo sacerdote de tu nación y te llamamos amigo del rey; (y con ello le envió un manto de púrpura y una corona de oro:) y pedirte que estés de nuestra parte y mantengas amistad con nosotros.
\par 21 Así que en el séptimo mes del año ciento sesenta, en la fiesta de las Tiendas, Jonatán se vistió el manto sagrado, reunió fuerzas y se proveyó de muchas armas.
\par 22 Cuando Demetrio se enteró de esto, se entristeció mucho y dijo:
\par 23 ¿Qué hemos hecho nosotros para que Alejandro nos haya impedido hacer amistad con los judíos para fortalecerse?
\par 24 También les escribiré palabras de aliento y les prometeré dignidades y regalos para poder contar con su ayuda.
\par 25 Entonces les envió este mensaje: El rey Demetrio envía saludos al pueblo de los judíos:
\par 26 Puesto que habéis guardado vuestros pactos con nosotros y permanecéis en nuestra amistad, sin uniros a nuestros enemigos, nosotros lo hemos oído y nos alegramos.
\par 27 Por lo tanto, continuad siendo fieles a nosotros, y os recompensaremos bien por las cosas que hacéis en nuestro favor.
\par 28 Y os concederá muchas inmunidades y os dará recompensas.
\par 29 Y ahora os libero, y por vosotros libero a todos los judíos, de los tributos, de las aduanas de la sal y de los impuestos de la corona,
\par 30 Y de lo que me corresponde recibir la tercera parte de la semilla y la mitad del fruto de los árboles, lo libero desde hoy en adelante, para que no sean quitados de la tierra de Judea. , ni de los tres gobiernos que se le añaden desde el país de Samaria y Galilea, desde este día en adelante para siempre.
\par 31 Sea también Jerusalén santa y libre con sus límites, tanto de décimos como de tributos.
\par 32 Y en cuanto a la torre que está en Jerusalén, le cedo autoridad y le doy al sumo sacerdote el poder de poner en ella a los hombres que él elija para guardarla.
\par 33 Además, pondré en libertad a todos los judíos que fueron llevados cautivos de la tierra de Judea a cualquier parte de mi reino, y haré que todos mis oficiales remitan los tributos incluso de su ganado.
\par 34 Además, quiero que todas las fiestas, los sábados, las lunas nuevas, los días solemnes, los tres días anteriores a la fiesta y los tres días después de la fiesta sean todos de inmunidad y libertad para todos los judíos en mi reino.
\par 35 Tampoco nadie tendrá autoridad para entrometerse o molestar a ninguno de ellos en ningún asunto.
\par 36 Quiero además que se alistan entre las fuerzas del rey unos treinta mil hombres judíos, a quienes se les pagará la misma remuneración que a todas las fuerzas del rey.
\par 37 Y de ellos algunos serán colocados en las fortalezas del rey, de los cuales también algunos serán puestos sobre los asuntos del reino, que son de confianza; y quiero que sus supervisores y gobernadores sean de ellos mismos, y que vivirán según sus propias leyes, tal como el rey ha mandado en la tierra de Judea.
\par 38 Y en cuanto a los tres gobiernos que se añaden a Judea desde el país de Samaria, que se unan a Judea, para que sean considerados bajo uno solo, y no estén obligados a obedecer a otra autoridad que la del sumo sacerdote.
\par 39 En cuanto a Tolemaida y sus tierras, las doy como regalo al santuario de Jerusalén para los gastos necesarios del santuario.
\par 40 Además, doy cada año quince mil siclos de plata de las cuentas del rey de los lugares correspondientes.
\par 41 Y todo el sobrante que los oficiales no pagaron como antes, de ahora en adelante se donará a las obras del templo.
\par 42 Además de esto, los cinco mil siclos de plata que cada año tomaban de las cuentas del uso del templo, serán liberados, porque pertenecen a los sacerdotes que ministran.
\par 43 Y quienesquiera que huyan al templo de Jerusalén, o se encuentren dentro de sus libertades, estando en deuda con el rey o por cualquier otro asunto, queden en libertad, y todo lo que tengan en mi reino.
\par 44 También los gastos de construcción y reparación del santuario se incluirán en las cuentas del rey.
\par 45 Además, los gastos para la construcción de los muros de Jerusalén y su fortificación alrededor se pagarán de las cuentas del rey, como también para la construcción de los muros en Judea.
\par 46 Cuando Jonatán y el pueblo oyeron estas palabras, no les dieron crédito ni las recibieron, porque se acordaron del gran mal que había hecho en Israel; porque los había afligido mucho.
\par 47 Pero estaban muy contentos con Alejandro, porque era el primero que les pedía una verdadera paz y siempre estuvieron con él.
\par 48 Entonces el rey Alejandro reunió grandes fuerzas y acampó frente a Demetrio.
\par 49 Cuando los dos reyes se enfrentaron, el ejército de Demetrio huyó, pero Alejandro lo siguió y los venció.
\par 50 Y continuó la batalla encarnizadamente hasta que se puso el sol; y aquel día fue asesinado Demetrio.
\par 51 Después Alejandro envió embajadores a Ptolomeo, rey de Egipto, con este mensaje:
\par 52 Por cuanto he vuelto a mi reino y me he sentado en el trono de mis antepasados, y he obtenido el dominio, y derribado a Demetrio, y recuperado nuestro país;
\par 53 Porque después de haber entrado en batalla con él, tanto él como su ejército fueron derrotados por nosotros, de modo que nos sentamos en el trono de su reino.
\par 54 Ahora, pues, hagamos una alianza de amistad y dame ahora a tu hija por esposa, y yo seré tu yerno y te daré a ti y a ella según tu dignidad.
\par 55 Entonces el rey Tolomeo respondió diciendo: Feliz sea el día en que regreses a la tierra de tus padres y te sientes en el trono de su reino.
\par 56 Ahora haré contigo lo que has escrito: encuéntrame, pues, en Tolemaida, para que nos veamos; porque te casaré con mi hija según tu deseo.
\par 57 Entonces Tolomeo salió de Egipto con su hija Cleopatra y llegaron a Tolemaida en el año ciento setenta y dos.
\par 58 Cuando el rey Alejandro lo encontró, le entregó a su hija Cleopatra y celebró sus bodas en Tolemaida con gran gloria, como es costumbre en los reyes.
\par 59 El rey Alejandro había escrito a Jonatán diciéndole que fuera a encontrarse con él.
\par 60 Éste se dirigió honorablemente a Tolemaida, donde se encontró con los dos reyes, les dio a ellos y a sus amigos plata y oro y muchos regalos, y halló gracia ante sus ojos.
\par 61 En aquel tiempo, algunos israelitas pestilentes y de mala vida se reunieron contra él para acusarlo, pero el rey no los escuchó.
\par 62 Aún más, el rey ordenó que le quitaran sus vestiduras y lo vistieran de púrpura, y así lo hicieron.
\par 63 Y le hizo sentarse solo, y dijo a sus príncipes: Id con él por el centro de la ciudad y proclamad que nadie se queje contra él por ningún asunto, y que nadie le moleste de ninguna manera. De causa.
\par 64 Cuando sus acusadores vieron que era honrado según la proclamación y vestido de púrpura, todos huyeron.
\par 65 Entonces el rey lo honró, lo inscribió entre sus principales amigos y lo nombró duque y partícipe de su dominio.
\par 66 Después Jonatán regresó a Jerusalén en paz y alegría.
\par 67 Además en el; El año ciento sesenta y cinco vino Demetrio, hijo de Demetrio, de Creta a la tierra de sus padres.
\par 68 Cuando el rey Alejandro se enteró de esto, se arrepintió y regresó a Antioquía.
\par 69 Entonces Demetrio nombró su general a Apolonio, gobernador de Celosiria, quien reunió un gran ejército, acampó en Jamnia y envió a decir al sumo sacerdote Jonatán:
\par 70 Sólo tú te alzas contra nosotros, y yo soy objeto de burla y desprecio por ti. ¿Y por qué alardeas de tu poder contra nosotros en las montañas?
\par 71 Ahora pues, si confías en tus propias fuerzas, baja a nosotros al campo y allí probaremos juntos el asunto; porque conmigo está el poder de las ciudades.
\par 72 Pregunta y descubre quién soy yo y los demás que están de nuestra parte, y te dirán que tu pie no puede volar en su propia tierra.
\par 73 Por lo tanto, ahora no podrás soportar a los jinetes y a un poder tan grande en la llanura, donde no hay piedra ni pedernal, ni lugar a donde huir.
\par 74 Cuando Jonatán oyó estas palabras de Apolonio, se conmovió y, escogiendo diez mil hombres, salió de Jerusalén, donde su hermano Simón lo encontró para ayudarlo.
\par 75 Y plantó sus tiendas frente a Jope; pero; Los de Jope le expulsaron de la ciudad, porque Apolonio tenía allí una guarnición.
\par 76 Entonces Jonatán la sitió, pero los de la ciudad le dejaron entrar por miedo, y así Jonatán ganó Jope.
\par 77 Cuando Apolonio se enteró de esto, tomó tres mil jinetes y un gran ejército de a pie, y como un viajero se dirigió a Azoto y lo llevó a la llanura. porque tenía gran número de jinetes, en quienes depositaba su confianza.
\par 78 Entonces Jonatán lo siguió hasta Azoto, donde los ejércitos se enfrentaron.
\par 79 Ahora bien, Apolonio había dejado mil jinetes en una emboscada.
\par 80 Y Jonatán se dio cuenta de que detrás de él había una emboscada; porque habían rodeado a su ejército y lanzaban dardos contra el pueblo, desde la mañana hasta la tarde.
\par 81 Pero el pueblo se detuvo, tal como Jonatán les había ordenado, y los caballos de los enemigos estaban cansados.
\par 82 Entonces Simón sacó a su ejército y los puso contra los de a pie (porque los de a caballo estaban agotados), que estaban derrotados por él, y huyeron.
\par 83 También los jinetes, dispersos por el campo, huyeron a Azoto y, en busca de seguridad, entraron en Bet-dagón, el templo de su ídolo.
\par 84 Pero Jonatán prendió fuego a Azoto y a las ciudades circundantes y se llevó el botín. y quemó al fuego el templo de Dagón, y con los que en él habían huido.
\par 85 Así fueron quemados y muertos a espada casi ocho mil hombres.
\par 86 Y de allí Jonatán retiró su ejército y acampó frente a Ascalón, de donde salieron los hombres de la ciudad y lo recibieron con gran pompa.
\par 87 Después de esto, Jonatán y su ejército regresaron a Jerusalén con todo el botín.
\par 88 Cuando el rey Alejandro oyó estas cosas, honró aún más a Jonatán.
\par 89 Y le envió un broche de oro, como es el que se da a los que son de la sangre del rey; y también le dio Accaron con sus bordes en posesión.

\chapter{11}

\par 1 Y el rey de Egipto reunió un gran ejército, como la arena que se encuentra a la orilla del mar, y muchas naves, y con engaños intentó apoderarse del reino de Alejandro y unirlo al suyo.
\par 2 Después de lo cual emprendió su viaje a España en paz, de modo que los de las ciudades se abrieron a él y le salieron al encuentro, porque el rey Alejandro les había ordenado que lo hicieran, porque era su cuñado.
\par 3 Cuando Tolomeo entró en las ciudades, puso en cada una de ellas una guarnición de soldados para protegerlas.
\par 4 Y cuando llegó cerca de Azoto, le mostraron el templo de Dagón que había sido quemado, y Azoto y sus ejidos destruidos, y los cuerpos que habían sido arrojados al extranjero y los que había quemado en la batalla; porque los habían amontonado en el camino por donde él debía pasar.
\par 5 También contaron al rey todo lo que Jonatán había hecho, para que pudiera culparlo; pero el rey guardó silencio.
\par 6 Entonces Jonatán se reunió con el rey con gran pompa en Jope, donde se saludaron y se alojaron.
\par 7 Después de que Jonatán había ido con el rey al río llamado Eleutero, regresó otra vez a Jerusalén.
\par 8 Entonces el rey Ptolomeo, habiendo conseguido el dominio de las ciudades junto al mar hasta Seleucia, en la costa del mar, ideó malos planes contra Alejandro.
\par 9 Entonces envió embajadores al rey Demetrio, diciendo: Ven, hagamos una alianza entre nosotros y te daré mi hija que tiene Alejandro, y tú reinarás en el reino de tu padre.
\par 10 Porque me arrepiento de haberle entregado a mi hija, porque él quería matarme.
\par 11 Así lo calumnió porque deseaba su reino.
\par 12 Por lo tanto, le quitó a su hija, se la dio a Demetrio y abandonó a Alejandro, de modo que su odio se hizo público.
\par 13 Entonces Tolomeo entró en Antioquía, donde puso sobre su cabeza dos coronas, la de Asia y la de Egipto.
\par 14 Mientras tanto, Alejandro estaba en Cilicia como rey, porque los habitantes de aquella región se habían rebelado contra él.
\par 15 Pero cuando Alejandro se enteró de esto, vino a la guerra contra él; entonces el rey Ptolomeo sacó su ejército, lo atacó con un gran poder y lo puso en fuga.
\par 16 Entonces Alejandro huyó a Arabia para defenderse allí; pero el rey Ptolomeo fue exaltado:
\par 17 Porque Zabdiel el árabe le cortó la cabeza a Alejandro y se la envió a Ptolomeo.
\par 18 Al tercer día murió también el rey Tolomeo, y los que estaban en las fortalezas fueron asesinados unos a otros.
\par 19 De esta manera Demetrio reinó en el año ciento sesenta y siete.
\par 20 Al mismo tiempo Jonatán reunió a los que estaban en Judea para tomar la torre que estaba en Jerusalén y armó contra ella muchas armas de guerra.
\par 21 Entonces vinieron unos impíos que odiaban a su propio pueblo, fueron al rey y le dijeron que Jonatán había sitiado la torre.
\par 22 Al enterarse de esto, se enojó y, al instante, se fue a Tolemaida y escribió a Jonatán que no sitiara la torre, sino que viniera a hablar con él a toda prisa a Tolemaida.
\par 23 Sin embargo, cuando Jonatán oyó esto, mandó sitiarla todavía; y escogió a algunos de los ancianos de Israel y a los sacerdotes, y se puso en peligro;
\par 24 Y tomando plata y oro, y vestidos, y además diversos regalos, fue a Tolemaida, al rey, donde halló favor ante sus ojos.
\par 25 Y aunque algunos hombres impíos del pueblo se quejaron contra él,
\par 26 Sin embargo, el rey le suplicó como lo habían hecho antes sus predecesores, y le hizo ascender ante todos sus amigos.
\par 27 Y lo confirmó en el sumo sacerdocio y en todos los honores que antes había tenido, y le dio la preeminencia entre sus principales amigos.
\par 28 Entonces Jonatán pidió al rey que liberara de tributos a Judea y a los tres gobiernos, junto con el país de Samaria; y le prometió trescientos talentos.
\par 29 Entonces el rey aceptó y escribió cartas a Jonatán contándole todas estas cosas de la siguiente manera:
\par 30 El rey Demetrio envía saludos a su hermano Jonatán y a la nación de los judíos:
\par 31 Os enviamos aquí una copia de la carta que escribimos a nuestro primo Lástenes acerca de vosotros, para que la podáis ver.
\par 32 El rey Demetrio envía un saludo a su padre Lástenes:
\par 33 Estamos decididos a hacer el bien al pueblo judío, que es nuestro amigo, y a cumplir nuestros pactos con nosotros, por su buena voluntad hacia nosotros.
\par 34 Por lo tanto, les hemos ratificado las fronteras de Judea, con los tres gobiernos de Aferema, Lida y Ramathem, que se añaden a Judea desde el país de Samaria, y todo lo que les pertenece, para todos los que sacrifican en Jerusalén, en lugar de los pagos que el rey recibía anualmente de ellos de los frutos de la tierra y de los árboles.
\par 35 Y en cuanto a otras cosas que nos pertenecen, los diezmos y las costumbres que nos pertenecen, así como también las salinas y los impuestos de la corona que nos deben, los liberamos de todos ellos para su alivio.
\par 36 Y nada de esto será revocado desde ahora y para siempre.
\par 37 Ahora, pues, procura hacer una copia de estas cosas y entregársela a Jonatán y colocarla en un lugar visible sobre el monte santo.
\par 38 Después de esto, cuando el rey Demetrio vio que el país estaba tranquilo ante él y que no se le oponía resistencia, envió todas sus tropas, cada uno a su lugar, excepto algunas bandas de extraños, a quienes había reunidos de las islas de las naciones; por eso todos los ejércitos de sus padres lo odiaban.
\par 39 Además, hubo un tal Trifón, que había sido del lado de Alejandro antes, el cual, al ver que todo el ejército murmuraba contra Demetrio, fue a Simalcue el árabe, quien crió a Antíoco, el joven hijo de Alejandro,
\par 40 Y le rogaron que le entregara al joven Antíoco para que pudiera reinar en lugar de su padre. Le contó, pues, todo lo que Demetrio había hecho y cómo sus hombres de guerra estaban enemistados con él, y allí permaneció. una larga temporada.
\par 41 Mientras tanto, Jonatán envió al rey Demetrio que echara de Jerusalén a los de la torre y también a los de las fortalezas, porque peleaban contra Israel.
\par 42 Entonces Demetrio envió a decir a Jonatán: No sólo haré esto por ti y tu pueblo, sino que, si se presenta la oportunidad, te honraré mucho a ti y a tu nación.
\par 43 Ahora pues, bien harás si me envías hombres que me ayuden; porque todas mis fuerzas se han ido de mí.
\par 44 Entonces Jonatán envió tres mil hombres fuertes a Antioquía, y cuando llegaron al rey, el rey se alegró mucho de su llegada.
\par 45 Pero los habitantes de la ciudad se reunieron en medio de la ciudad, ciento veinte mil hombres, y querían matar al rey.
\par 46 Entonces el rey huyó al patio, pero los de la ciudad guardaron los pasos de la ciudad y comenzaron a pelear.
\par 47 Entonces el rey pidió ayuda a los judíos, quienes acudieron a él todos a la vez y, dispersándose por la ciudad, mataron aquel día en la ciudad a cien mil personas.
\par 48 También prendieron fuego a la ciudad, tomaron mucho botín ese día y liberaron al rey.
\par 49 Cuando los habitantes de la ciudad vieron que los judíos habían tomado la ciudad como querían, su ánimo se apagó; por lo que rogaron al rey y gritaron, diciendo:
\par 50 Concédenos la paz y que los judíos dejen de atacarnos a nosotros y a la ciudad.
\par 51 Dicho esto, arrojaron sus armas e hicieron la paz; y los judíos fueron honrados ante los ojos del rey, y ante los ojos de todos los que estaban en su reino; y regresaron a Jerusalén, teniendo grandes despojos.
\par 52 Entonces el rey Demetrio se sentó en el trono de su reino y el país quedó en paz ante él.
\par 53 Sin embargo, él disimulaba todo lo que decía y se alejaba de Jonatán, ni le recompensaba según los beneficios que había recibido de él, sino que lo afligía mucho.
\par 54 Después de esto regresó Trifón, y con él el niño Antíoco, que reinó y fue coronado.
\par 55 Entonces se reunieron junto a él todos los hombres de guerra que Demetrio había expulsado, y pelearon contra Demetrio, quien se volvió y huyó.
\par 56 Además, Trifón tomó los elefantes y conquistó Antioquía.
\par 57 En aquel tiempo el joven Antíoco escribió a Jonatán, diciéndole: Te confirmo en el sumo sacerdocio y te nombro gobernador de los cuatro gobiernos, y uno de los amigos del rey.
\par 58 Entonces le envió vasos de oro para que los sirvieran, le dio permiso para beber en oro, vestirse de púrpura y llevar un broche de oro.
\par 59 También nombró capitán a su hermano Simón desde el lugar llamado La Escala de Tiro hasta las fronteras de Egipto.
\par 60 Entonces Jonatán salió y atravesó las ciudades al otro lado del agua, y todas las fuerzas de Siria se reunieron con él para ayudarlo; y cuando llegó a Ascalón, los de la ciudad lo recibieron con honores.
\par 61 De allí salió a Gaza, pero los de Gaza le cerraron la puerta; Por lo cual la sitió, y quemó a fuego sus ejidos, y los despojó.
\par 62 Después, cuando los de Gaza rogaron a Jonatán, él hizo la paz con ellos, tomó como rehenes a los hijos de sus jefes, los envió a Jerusalén y atravesó el país hasta Damasco.
\par 63 Cuando Jonatán se enteró de que los príncipes de Demetrio habían llegado a Cades, que está en Galilea, con gran poder, con el propósito de expulsarlo del país,
\par 64 Él fue a su encuentro y dejó a su hermano Simón en el campo.
\par 65 Entonces Simón acampó contra Betsur y luchó contra ella durante mucho tiempo, y la encerró.
\par 66 Pero ellos querían tener paz con él, y él se los concedió, y luego los expulsó de allí, tomó la ciudad y puso en ella una guarnición.
\par 67 En cuanto a Jonatán y su ejército, acamparon junto al agua de Genesar, desde donde temprano en la mañana los llevaron a la llanura de Nasor.
\par 68 Y he aquí, en la llanura les salió al encuentro un ejército de extranjeros que, habiéndole puesto emboscadas en las montañas, se acercaron a él.
\par 69 Entonces, cuando los que estaban emboscados se levantaron de sus lugares y trabaron batalla, todos los que estaban del lado de Jonatán huyeron;
\par 70 De modo que no quedó ninguno de ellos, excepto Matatías hijo de Absalón y Judas hijo de Calfi, capitanes del ejército.
\par 71 Entonces Jonatán rasgó sus vestidos, se echó tierra sobre la cabeza y oró.
\par 72 Después, volviendo a la batalla, los hizo huir, y ellos huyeron.
\par 73 Al ver esto sus propios hombres que habían huido, se volvieron hacia él y con él los persiguieron hasta Cades, hasta sus tiendas, y allí acamparon.
\par 74 Aquel día fueron asesinados entre las naciones unos tres mil hombres; pero Jonatán volvió a Jerusalén.

\chapter{12}

\par 1 Jonatán, viendo que el tiempo le había llegado, escogió a algunos hombres y los envió a Roma para confirmar y renovar la amistad que tenían con ellos.
\par 2 También envió cartas a los lacedemonios y a otros lugares con el mismo propósito.
\par 3 Ellos fueron a Roma, entraron en el Senado y dijeron: El sumo sacerdote Jonatán y el pueblo de los judíos nos enviaron a vosotros para renovar la amistad que teníais con ellos. y liga, como en tiempos pasados.
\par 4 Entonces los romanos les dieron cartas a los gobernadores de cada lugar para que los llevaran pacíficamente a la tierra de Judea.
\par 5 Y ésta es la copia de las cartas que Jonatán escribió a los lacedemonios:
\par 6 El sumo sacerdote Jonatán, los ancianos de la nación, los sacerdotes y los demás judíos, envían saludos a sus hermanos lacedemonios:
\par 7 En tiempos pasados, Darío, que entonces reinaba entre vosotros, envió cartas al sumo sacerdote Onías para indicarles que sois nuestros hermanos, como lo indica la copia aquí suscrita.
\par 8 Entonces Onías rogó al embajador que había sido enviado honorablemente y recibió las cartas en las que se declaraba la alianza y la amistad.
\par 9 Por eso también nosotros, aunque no necesitemos nada de esto, teniendo en nuestras manos los santos libros de las Escrituras para consolarnos,
\par 10 Sin embargo, he intentado enviaros un mensaje para renovar la hermandad y la amistad, para que no seamos completamente extraños para vosotros; porque hace mucho tiempo que nos enviasteis un mensaje.
\par 11 Por lo tanto, nosotros, en todo momento y sin cesar, tanto en nuestras fiestas como en otros días convenientes, nos acordamos de ti en los sacrificios que ofrecemos y en nuestras oraciones, según la razón y la conveniencia de pensar en nuestros hermanos. :
\par 12 Y nos alegramos mucho de tu honor.
\par 13 En cuanto a nosotros, hemos tenido grandes problemas y guerras por todos lados, tanto que los reyes que están alrededor de nosotros han peleado contra nosotros.
\par 14 Sin embargo, no queremos ser molestos para vosotros ni para otros de nuestros aliados y amigos en estas guerras.
\par 15 Porque tenemos la ayuda del cielo que nos socorre, así como somos librados de nuestros enemigos, y nuestros enemigos son puestos bajo nuestros pies.
\par 16 Por esta razón elegimos a Numenio, hijo de Antíoco, y a Antípatro, hijo de Jasón, y los enviamos a los romanos para renovar la amistad que teníamos con ellos y la alianza anterior.
\par 17 También les ordenamos que fueran a vosotros, y os saludaran y os entregaran nuestras cartas sobre la renovación de nuestra hermandad.
\par 18 Por tanto, ahora haréis bien en darnos una respuesta.
\par 19 Y esta es la copia de las cartas que envió Oniares.
\par 20 Areo, rey de los Lacedemonios, saludó al sumo sacerdote Onías:
\par 21 Está escrito que los lacedemonios y los judíos son hermanos y que son del linaje de Abraham:
\par 22 Ahora pues, ahora que esto ha llegado a nuestro conocimiento, haréis bien en escribirnos acerca de vuestra prosperidad.
\par 23 Os escribimos de nuevo, que vuestro ganado y vuestros bienes son nuestros, y los nuestros son vuestros. Por tanto, ordenamos a nuestros embajadores que os informen de este modo.
\par 24 Cuando Jonatán se enteró de que los príncipes de Demebio habían venido a luchar contra él con un ejército mayor que antes,
\par 25 Salió de Jerusalén y los encontró en la tierra de Amathis, porque no les dio tregua para entrar en su país.
\par 26 También envió espías a sus tiendas, los cuales regresaron y le dijeron que estaban designados para atacarlos durante la noche.
\par 27 Por lo tanto, tan pronto como se puso el sol, Jonatán ordenó a sus hombres que velaran y estuvieran en armas para que estuvieran listos para luchar durante toda la noche; también envió centinelas alrededor del ejército.
\par 28 Pero cuando los enemigos oyeron que Jonatán y sus hombres estaban preparados para la batalla, temieron y temblaron en su corazón, y encendieron hogueras en su campamento.
\par 29 Pero Jonatán y su compañía no lo supieron hasta la mañana siguiente, porque vieron las luces encendidas.
\par 30 Entonces Jonatán los persiguió, pero no los alcanzó, porque habían cruzado el río Eleutero.
\par 31 Entonces Jonatán se volvió hacia los árabes, llamados zabadeos, los derrotó y se llevó el botín.
\par 32 Y partiendo de allí, llegó a Damasco, y así recorrió toda la tierra,
\par 33 Simón también salió y atravesó la región hasta Ascalón y sus fortalezas vecinas, de donde se desvió hacia Jope y la conquistó.
\par 34 Porque había oído que entregarían la fortaleza a los que estaban del lado de Demetrio; por lo que puso allí una guarnición para custodiarla.
\par 35 Después de esto, Jonatán volvió a su casa y, reuniendo a los ancianos del pueblo, consultó con ellos sobre la construcción de fortalezas en Judea.
\par 36 y alzaron los muros de Jerusalén y levantaron un gran monte entre la torre y la ciudad, para separarla de la ciudad, para que estuviera sola y no se pudiera vender ni comprar en ella.
\par 37 Entonces se reunieron para reconstruir la ciudad, ya que parte del muro que daba al arroyo en el lado oriental se había derrumbado, y repararon lo que se llamaba Cafenata.
\par 38 Simón también levantó a Adida en Sefela, y la fortificó con puertas y cerrojos.
\par 39 Trifón se dispuso a apoderarse del reino de Asia y a matar al rey Antíoco para poder ponerse la corona sobre su propia cabeza.
\par 40 Sin embargo, temía que Jonatán no lo tolerara y peleara contra él; por lo que buscó la manera de prender a Jonatán, para poder matarlo. Así que partió y llegó a Betsán.
\par 41 Entonces Jonatán salió a su encuentro con cuarenta mil hombres escogidos para la batalla y llegó a Betsan.
\par 42 Cuando Trifón vio que Jonatán venía con tanta fuerza, no se atrevió a extender su mano contra él;
\par 43 Pero lo recibió honorablemente, lo encomendó a todos sus amigos, le hizo regalos y ordenó a sus hombres de guerra que le fueran tan obedientes como a él mismo.
\par 44 También a Jonatán le dijo: ¿Por qué has causado tantos problemas a todo este pueblo, si no hay guerra entre nosotros?
\par 45 Por lo tanto, envíalos ahora de nuevo a casa, escoge algunos hombres que te sirvan y ven conmigo a Tolemaida, porque te la daré a ti y al resto de las fortalezas y fuerzas, y a todos los que tienen algún poder. cargo: en cuanto a mí, volveré y partiré: porque esta es la causa de mi venida.
\par 46 Entonces Jonatán, creyendo en él, hizo lo que le había dicho y despidió a su ejército, que se fue a la tierra de Judea.
\par 47 Y sólo se quedó con tres mil hombres, de los cuales envió dos mil a Galilea, y mil fueron con él.
\par 48 Tan pronto como Jonatán entró en Tolemaida, los de Tolemaida cerraron las puertas y lo apresaron, y mataron a espada a todos los que venían con él.
\par 49 Entonces Trifón envió un ejército de a pie y de caballería a Galilea y a la gran llanura, para destruir toda la compañía de Jonatán.
\par 50 Pero cuando supieron que Jonatán y los que estaban con él habían sido capturados y asesinados, se animaron unos a otros; y se acercaron, dispuestos a pelear.
\par 51 Entonces los que los seguían, sabiendo que estaban dispuestos a luchar por sus vidas, se dieron la vuelta.
\par 52 Entonces todos llegaron pacíficamente a la tierra de Judea, y allí lloraron a Jonatán y a los que estaban con él, y tuvieron mucho miedo; Por lo cual todo Israel hizo gran lamentación.
\par 53 Entonces todas las naciones de los alrededores intentaron destruirlos, porque decían: No tienen capitán ni nadie que los ayude. Ahora, pues, hagamos la guerra contra ellos y quitemos su recuerdo de entre los hombres.

\chapter{13}

\par 1 Cuando Simón oyó que Trifón había reunido un gran ejército para invadir la tierra de Judea y destruirla,
\par 2 Y viendo que el pueblo estaba temblando y atemorizado, subió a Jerusalén y reunió al pueblo,
\par 3 Y les exhortó, diciendo: Vosotros sabéis las grandes cosas que yo, mis hermanos y la casa de mi padre hemos hecho por las leyes y el santuario, y las batallas y tribulaciones que hemos visto.
\par 4 Por lo cual todos mis hermanos son asesinados por causa de Israel, y yo quedo solo.
\par 5 Ahora, pues, esté lejos de mí el tener que perdonar mi propia vida en cualquier momento de angustia, porque no soy mejor que mis hermanos.
\par 6 Sin duda vengaré a mi nación, a mi santuario, a nuestras mujeres y a nuestros hijos, porque todas las naciones se han reunido para destruirnos con toda malicia.
\par 7 Cuando el pueblo escuchó estas palabras, su espíritu revivió.
\par 8 Y ellos respondieron en alta voz, diciendo: Tú serás nuestro líder en lugar de Judas y Jonatán tu hermano.
\par 9 Lucha tú nuestras batallas, y todo lo que nos mandes, eso haremos.
\par 10 Entonces reunió a todos los hombres de guerra y se apresuró a terminar las murallas de Jerusalén y la fortificó alrededor.
\par 11 También envió a Jonatán, hijo de Absalón, y con él a un gran poder, a Jope, quienes, expulsando a los que allí estaban, se quedaron allí.
\par 12 Entonces Trifón partió de Ptolemao con gran poder para invadir la tierra de Judea, y Jonatán estaba con él en su guardia.
\par 13 Pero Simón instaló sus tiendas en Adida, frente a la llanura.
\par 14 Cuando Trifón supo que Simón se había levantado en lugar de su hermano Jonatán y quería luchar contra él, le envió mensajeros, diciéndole:
\par 15 Mientras que tenemos retenido a tu hermano Jonatán, es por dinero que él debe al tesoro del rey, por el negocio que le fue encomendado.
\par 16 Por tanto, envía ahora cien talentos de plata y dos de sus hijos como rehenes, para que cuando esté en libertad no se rebele contra nosotros y lo dejaremos ir.
\par 17 Entonces Simón, aunque se dio cuenta de que le hablaban con engaño, envió el dinero y los niños, no fuera a provocarse un gran odio hacia el pueblo.
\par 18 ¿Quién podría haber dicho: Porque no le envié el dinero ni los niños, por eso Jonatán está muerto?
\par 19 Entonces les envió los niños y los cien talentos; pero Trifón fingió y no dejó ir a Jonatán.
\par 20 Después de esto vino Trifón para invadir la tierra y destruirla, rodeando el camino que conduce a Adora; pero Simón y su ejército marchaban contra él en todos los lugares a dondequiera que iba.
\par 21 Los que estaban en la torre enviaron mensajeros a Trifón para que acelerara su llegada a ellos por el desierto y les enviara provisiones.
\par 22 Por eso Trifón preparó a todos sus jinetes para venir esa noche, pero cayó una nieve muy grande, por lo que no pudo venir. Partió, pues, y llegó al país de Galaad.
\par 23 Y cuando llegó cerca de Bascama, mató a Jonatán, que estaba allí enterrado.
\par 24 Después regresó Trifón y se fue a su tierra.
\par 25 Entonces Simón envió a tomar los huesos de su hermano Jonatán y los enterró en Modín, la ciudad de sus padres.
\par 26 Y todo Israel hizo gran lamentación por él y lo lloró durante muchos días.
\par 27 Simón también edificó un monumento sobre el sepulcro de su padre y de sus hermanos, y lo levantó a la vista, con piedras labradas detrás y delante.
\par 28 Además, erigió siete pirámides, una frente a otra, para su padre, su madre y sus cuatro hermanos.
\par 29 Y en estos hizo ingeniosos diseños, alrededor de los cuales colocó grandes columnas, y sobre las columnas hizo todas sus armas para memoria perpetua, y junto a las armas esculpió barcos para que pudieran ser vistos por todos los que navegaban. el mar.
\par 30 Éste es el sepulcro que hizo en Modín, y que aún permanece en pie hasta el día de hoy.
\par 31 Trifón engañó al joven rey Antíoco y lo mató.
\par 32 Y reinó en su lugar, se coronó rey de Asia y trajo una gran calamidad sobre la tierra.
\par 33 Entonces Simón edificó fortalezas en Judea, las rodeó con altas torres, grandes murallas, puertas y cerrojos, y puso en ellas provisiones.
\par 34 Además, Simón escogió hombres y envió al rey Demetrio para que concediera inmunidad al país, porque Trifón lo único que había hecho era saquearlo.
\par 35 A quien respondió el rey Demetrio y escribió así:
\par 36 El rey Demetrio envía saludos a Simón, sumo sacerdote y amigo de los reyes, así como a los ancianos y a la nación de los judíos:
\par 37 Hemos recibido la corona de oro y el manto escarlata que nos enviasteis, y estamos dispuestos a hacer una paz firme con vosotros, sí, y a escribir a nuestros oficiales para confirmar las inmunidades que tenemos. otorgada.
\par 38 Y todos los pactos que hemos hecho con vosotros se mantendrán; y las fortalezas que habéis construido, serán vuestras.
\par 39 En cuanto a cualquier descuido o falta cometido hasta el día de hoy, os lo perdonamos, y también el impuesto de la corona que vosotros nos debéis; y si se hubiera pagado algún otro tributo en Jerusalén, no se pagará más.
\par 40 Y mirad quiénes de vosotros son los adecuados para estar en nuestra corte, que sean empadronados y que haya paz entre nosotros.
\par 41 Así fue quitado a Israel el yugo de las naciones en el año ciento setenta.
\par 42 Entonces los hijos de Israel comenzaron a escribir en sus instrumentos y contratos, En el primer año del sumo sacerdote Simón, gobernador y líder de los judíos.
\par 43 En aquellos días Simón acampó frente a Gaza y la sitió alrededor; Hizo también una máquina de guerra, la colocó junto a la ciudad, derribó una torre y la tomó.
\par 44 Y los que estaban en la locomotora saltaron a la ciudad; Entonces hubo un gran alboroto en la ciudad:
\par 45 Los habitantes de la ciudad rasgaron sus vestidos, subieron a las murallas con sus mujeres y sus hijos y clamaron a gran voz, rogando a Simón que les concediera la paz.
\par 46 Y ellos dijeron: No trates con nosotros según nuestra maldad, sino según tu misericordia.
\par 47 Entonces Simón se apaciguó con ellos y ya no peleó más contra ellos, sino que los echó de la ciudad, limpió las casas donde estaban los ídolos y entró en ella con cánticos y acciones de gracias.
\par 48 Además, eliminó de allí toda impureza, colocó allí a hombres capaces de guardar la ley, la hizo más fuerte que antes y edificó en ella una morada para él.
\par 49 También los de la torre de Jerusalén estaban tan apurados que no podían salir ni entrar en el campo, ni comprar ni vender; por lo que estaban en gran angustia por falta de víveres y un gran número de ellos perecieron a causa del hambre.
\par 50 Entonces clamaron a Simón, rogándole que fuera uno con ellos, lo cual él les concedió; Y cuando los hubo echado de allí, limpió la torre de toda contaminación.
\par 51 Y entró en él el día veintitrés del mes segundo del año ciento setenta y uno, con acción de gracias, ramas de palmeras, arpas, címbalos, violas, himnos y cánticos. porque fue destruido un gran enemigo fuera de Israel.
\par 52 También ordenó que ese día se celebrara cada año con alegría. Además, fortaleció aún más el monte del templo que estaba junto a la torre, y allí habitó con su compañía.
\par 53 Y cuando Simón vio que su hijo Juan era un hombre valiente, lo nombró capitán de todos los ejércitos; y habitó en Gazera.

\chapter{14}

\par 1 En el año ciento setenta y doce, el rey Demetrio reunió sus fuerzas y fue a Media en busca de ayuda para luchar contra Trifone.
\par 2 Pero cuando Arsaces, rey de Persia y Media, se enteró de que Demetrio había entrado en su territorio, envió a uno de sus príncipes para que lo capturara vivo:
\par 3 El cual fue y derrotó al ejército de Demetrio, lo tomó y lo llevó a Arsaces, donde lo encerró.
\par 4 En cuanto a la tierra de Judea, estuvo tranquila todos los días de Simón; porque buscó el bien de su nación de tal manera que cada vez más su autoridad y honor les agradaran.
\par 5 Y así como fue honorable en todos sus actos, así también en esto de tomar a Jope por puerto y hacer entrada a las islas del mar,
\par 6 Y amplió los límites de su nación y recuperó el país,
\par 7 Reunió un gran número de cautivos y dominó Gazera, Betsur y la torre, de donde sacó toda inmundicia, y no hubo quien se le resistiera.
\par 8 Entonces labraron su tierra en paz, y la tierra dio su fruto y los árboles del campo su fruto.
\par 9 Los ancianos se sentaban todos en las calles, hablando de cosas buenas, y los jóvenes se vestían con ropas gloriosas y guerreras.
\par 10 Proporcionó víveres a las ciudades y puso en ellas todo tipo de municiones, para que su honorable nombre fuera renombrado hasta el fin del mundo.
\par 11 Hizo la paz en la tierra, e Israel se regocijó con gran alegría.
\par 12 Porque cada uno se sentaba debajo de su vid y de su higuera, y no había nadie que las asustara.
\par 13 Tampoco quedó nadie en la tierra que pudiera luchar contra ellos; incluso los reyes mismos fueron derrocados en aquellos días.
\par 14 Además, fortaleció a todos los de su pueblo que estaban abatidos: buscó la ley; y a todo impío y despreciador de la ley lo quitó.
\par 15 Embelleció el santuario y multiplicó los utensilios del templo.
\par 16 Cuando se supo en Roma y en Esparta que Jonatán había muerto, se entristecieron mucho.
\par 17 Pero cuando supieron que su hermano Simón había sido nombrado sumo sacerdote en su lugar y gobernaba el país y sus ciudades,
\par 18 Le escribieron en tablas de bronce para renovar la amistad y la alianza que habían hecho con Judas y Jonatán, sus hermanos:
\par 19 Los cuales fueron leídos ante la congregación en Jerusalén.
\par 20 Y esta es la copia de las cartas que enviaron los lacedemonios; Los príncipes de los lacedemonios con la ciudad, el sumo sacerdote Simón, los ancianos y los sacerdotes y el resto del pueblo judío, nuestros hermanos, envían saludos:
\par 21 Los embajadores que fueron enviados a nuestro pueblo nos certificaron tu gloria y honor; por eso nos alegramos de su llegada,
\par 22 Y registraron así lo que hablaban en el consejo del pueblo; Numenio, hijo de Antíoco, y Antípatro, hijo de Jasón, embajadores de los judíos, vinieron a nosotros para renovar la amistad que tenían con nosotros.
\par 23 Y al pueblo le pareció bien agasajar a aquellos hombres con honores y poner la copia de su embajada en registros públicos, para que el pueblo de los lacedemonios tuviera un recuerdo de ello; además, hemos escrito una copia de ella a Simón el gran sacerdote.
\par 24 Después de esto, Simón envió a Numenio a Roma con un gran escudo de oro que pesaba mil libras para confirmar la alianza con ellos.
\par 25 Cuando el pueblo se enteró, dijo: ¿Qué gracias daremos a Simón y a sus hijos?
\par 26 Porque él, sus hermanos y la casa de su padre afirmaron a Israel, expulsaron de allí a sus enemigos en la batalla y confirmaron su libertad.
\par 27 Entonces lo escribieron en tablas de bronce, que pusieron sobre columnas en el monte Sión. Y esta es la copia de la escritura; El día dieciocho del mes Elul, en el año ciento sesenta y doce, siendo el año tercero del sumo sacerdote Simón,
\par 28 En Saramel, en la gran congregación de los sacerdotes, el pueblo, los gobernantes de la nación y los ancianos del país, nos fueron notificadas estas cosas.
\par 29 Por cuanto muchas veces hubo guerras en el país, en las cuales, por el mantenimiento de su santuario y de la ley, Simón, hijo de Matatías, de la posteridad de Jarib, junto con sus hermanos, se pusieron en peligro y resistieron. Los enemigos de su nación hicieron gran honor a su nación:
\par 30 (Porque después de esto, Jonatán, habiendo reunido a su nación y siendo sumo sacerdote, fue añadido a su pueblo,
\par 31 Sus enemigos se dispusieron a invadir su país para destruirlo y apoderarse del santuario.
\par 32 Entonces Simón se levantó y peleó por su nación, gastó mucho de sus bienes, armó a los valientes de su nación y les dio salarios.
\par 33 Y fortificó las ciudades de Judea y Betsur, que está en los límites de Judea, donde antes estaban las armas de los enemigos; pero puso allí una guarnición de judíos:
\par 34 Además, fortificó Jope, que está junto al mar, y Gazera, que limita con Azoto, donde antes habitaban los enemigos; pero colocó allí a judíos y les proporcionó todo lo necesario para su reparación.)
\par 35 Entonces el pueblo cantó las hazañas de Simón, y la gloria que pensaba llevar a su nación, lo nombraron gobernador y sumo sacerdote, por haber hecho todas estas cosas y por la justicia y la fe que guardaba en sus nación, y para ello buscó por todos los medios enaltecer a su pueblo.
\par 36 Porque en su tiempo las cosas prosperaron en sus manos, de modo que las naciones fueron expulsadas de su país, y también los que estaban en la ciudad de David en Jerusalén, que se habían hecho una torre de donde salían. y contaminaron todo lo que rodea el santuario, e hicieron mucho daño en el lugar santo.
\par 37 Pero puso allí a judíos. y la fortificó para seguridad del país y de la ciudad, y levantó los muros de Jerusalén.
\par 38 También el rey Demetrio le confirmó en el sumo sacerdocio conforme a estas cosas,
\par 39 Y lo hizo uno de sus amigos y lo honró con grandes honores.
\par 40 Porque había oído decir que los romanos llamaban a los judíos sus amigos, aliados y hermanos; y que habían agasajado honorablemente a los embajadores de Simón;
\par 41 También que los judíos y los sacerdotes estaban muy contentos de que Simón fuera su gobernador y sumo sacerdote para siempre, hasta que surgiera un profeta fiel;
\par 42 Además, para que fuera su capitán y se encargara del santuario, para ponerlos sobre sus obras, sobre el país, sobre las armas y sobre las fortalezas, para que, digo, se hiciera cargo de ellos. del santuario;
\par 43 Además, que todos le obedezcan, que todas las escrituras del país se escriban en su nombre, que se vista de púrpura y se vista de oro.
\par 44 Además, a ninguno del pueblo ni a los sacerdotes les sería lícito violar cualquiera de estas cosas, o contradecir sus palabras, o reunir una asamblea en el campo sin él, o vestirse de púrpura o llevar un manto. hebilla de oro;
\par 45 Y quienquiera que haga lo contrario o rompa cualquiera de estas cosas, deberá ser castigado.
\par 46 Así quiso que todo el pueblo tratara con Simón y hiciera lo que se le había dicho.
\par 47 Entonces Simón aceptó esto y se alegró de ser sumo sacerdote, capitán y gobernador de los judíos y de los sacerdotes, y defenderlos a todos.
\par 48 Entonces ordenaron que esta escritura se pusiera en tablas de bronce y que se colocaran dentro del recinto del santuario en un lugar visible;
\par 49 También que sus copias se guardaran en el tesoro, para que las tuvieran Simón y sus hijos.

\chapter{15}

\par 1 Además, el rey Antíoco, hijo del rey Demetrio, envió cartas desde las islas del mar a Simón, sacerdote y príncipe de los judíos, y a todo el pueblo;
\par 2 Su contenido era el siguiente: El rey Antíoco saludó a Simón, sumo sacerdote y príncipe de su nación, y al pueblo de los judíos:
\par 3 Puesto que ciertos hombres pestilentes han usurpado el reino de nuestros padres y mi propósito es desafiarlo nuevamente para restaurarlo a su antiguo estado, y para ello he reunido una multitud de soldados extranjeros y preparado barcos de guerra;
\par 4 Mi intención también es atravesar el país para vengarme de los que lo han destruido y han asolado muchas ciudades del reino.
\par 5 Ahora, pues, te confirmo todas las ofrendas que te concedieron los reyes que me precedieron, y todos los regalos adicionales que te concedieron.
\par 6 Te doy permiso también para acuñar moneda para tu país con tu propio sello.
\par 7 Y en cuanto a Jerusalén y el santuario, sean libres; y todas las armas que has hecho y las fortalezas que has construido y tienes en tus manos, queden en ti.
\par 8 Y si algo se debe o se debe al rey, te será perdonado desde ahora y para siempre.
\par 9 Además, cuando hayamos obtenido nuestro reino, te honraremos a ti, a tu nación y a tu templo con gran honor, para que tu honor sea conocido en todo el mundo.
\par 10 En el año ciento sesenta y catorce, Antíoco llegó a la tierra de sus padres; en ese momento se reunieron todas las fuerzas contra él, de modo que pocos quedaron con Trifón.
\par 11 Por lo que, perseguido por el rey Antíoco, huyó a Dora, que está junto al mar.
\par 12 Porque vio que de repente le sobrevinieron problemas y que sus fuerzas lo habían abandonado.
\par 13 Entonces Antíoco acampó frente a Dora, llevando consigo ciento veinte mil hombres de guerra y ocho mil jinetes.
\par 14 Y después de rodear la ciudad y juntar las naves cerca de la ciudad, por el lado del mar, atacó la ciudad por tierra y por mar, sin permitir que nadie entrara ni saliera.
\par 15 Mientras tanto, Numenio y su compañía llegaron de Roma con cartas para los reyes y los países; donde estaban escritas estas cosas:
\par 16 Lucio, cónsul de los romanos ante el rey Ptolomeo, saluda:
\par 17 Los embajadores de los judíos, nuestros amigos y aliados, vinieron a nosotros para renovar la antigua amistad y alianza, enviados por el sumo sacerdote Simón y por el pueblo de los judíos:
\par 18 Y trajeron un escudo de oro de mil libras.
\par 19 Por eso nos pareció bueno escribir a los reyes y a los países para que no les hicieran daño, ni pelearan contra ellos, sus ciudades o países, ni ayudaran a sus enemigos contra ellos.
\par 20 También a nosotros nos pareció bien recibir el escudo de ellos.
\par 21 Por tanto, si hay algún pestilente que haya huido de su tierra hacia vosotros, entrégaselo al sumo sacerdote Simón, para que los castigue según su propia ley.
\par 22 Lo mismo escribió también al rey Demetrio, a Atalo, a Ariarates y a Arsaces,
\par 23 Y a todos los países, a Samsames, a los Lacedemonios, a Delus, a Mindus, a Sición, a Caria, a Samos, a Panfilia, a Licia, a Halicarnaso, a Rodus, a Aradus y a Cos, y Side, Aradus, Gortina, Cnido, Chipre y Cirene.
\par 24 Y escribieron esta copia al sumo sacerdote Simón.
\par 25 El segundo día, el rey Antíoco acampó contra Dora, atacándola continuamente y fabricando máquinas, con lo que encerró a Trifón para que no pudiera salir ni entrar.
\par 26 En aquel tiempo Simón le envió dos mil hombres escogidos para ayudarle; también plata, oro y muchas armas.
\par 27 Sin embargo, él no quiso recibirlos, sino que rompió todos los pactos que había hecho con él antes y se volvió extraño para él.
\par 28 Además, le envió a Atenobio, uno de sus amigos, para que hablara con él y le dijera: Vosotros retenéis a Jope y Gazera; con la torre que está en Jerusalén, que son ciudades de mi reino.
\par 29 Habéis arrasado sus fronteras, hecho muchos daños a la tierra y dominado muchos lugares dentro de mi reino.
\par 30 Entregad, pues, ahora las ciudades que habéis tomado y los tributos de los lugares que habéis dominado fuera de las fronteras de Judea.
\par 31 O si no, dame quinientos talentos de plata por ellos; y por el daño que habéis hecho, y los tributos de las ciudades, otros quinientos talentos; si no, vendremos y pelearemos contra vosotros.
\par 32 Entonces Atenobio, el amigo del rey, llegó a Jerusalén y, al ver la gloria de Simón, el aparador de oro y plata y su gran asistencia, quedó asombrado y le comunicó el mensaje del rey.
\par 33 Entonces respondió Simón y le dijo: No hemos tomado tierra ajena, ni hemos poseído lo ajeno, sino la herencia de nuestros padres, que nuestros enemigos tuvieron injustamente en posesión durante algún tiempo.
\par 34 Por lo cual, si tenemos oportunidad, poseemos la herencia de nuestros padres.
\par 35 Y aunque exiges a Jope y Gazera, aunque hicieron un gran daño a la gente de nuestro país, te daremos cien talentos por ellos. A esto Atenobius no le respondió una palabra;
\par 36 Pero volvió enojado al rey y le contó estas palabras, la gloria de Simón y todo lo que había visto; por lo que el rey se enojó mucho.
\par 37 Mientras tanto, Trifón huyó en barco hacia Ortosias.
\par 38 Entonces el rey nombró a Cendebeo capitán de la costa del mar y le dio un ejército de a pie y de a caballo,
\par 39 Y le ordenó que llevara su ejército hacia Judea; también le mandó edificar Cedrón, y fortificar las puertas, y hacer guerra contra el pueblo; pero el rey mismo persiguió a Trifón.
\par 40 Entonces Cendebeo llegó a Jamnia y comenzó a provocar al pueblo, a invadir Judea, a tomar prisioneros al pueblo y a matarlo.
\par 41 Y cuando hubo reconstruido Cedrou, puso allí gente de a caballo y un ejército de a pie, para que, saliendo, pudieran abrir caminos por los caminos de Judea, tal como el rey le había ordenado.

\chapter{16}

\par 1 Entonces Juan subió desde Gazera y contó a Simón, su padre, lo que había hecho Cendebeo.
\par 2 Entonces Simón llamó a sus dos hijos mayores, Judas y Juan, y les dijo: Yo, mis hermanos y la casa de mi padre hemos luchado desde mi juventud hasta el día de hoy contra los enemigos de Israel; y las cosas han prosperado tan bien en nuestras manos, que muchas veces hemos librado a Israel.
\par 3 Pero ahora yo soy viejo, y vosotros, por la misericordia de Dios, sois mayores de edad: sed en lugar de mí y de mi hermano, y id a luchar por nuestra nación, y la ayuda del cielo esté con vosotros.
\par 4 Entonces escogió del país veinte mil hombres de guerra y de a caballo, que salieron contra Cendebeo y descansaron esa noche en Modín.
\par 5 Cuando se levantaron por la mañana y entraron en la llanura, he aquí que un gran ejército, tanto de a pie como de jinete, vino contra ellos; pero había un arroyo de agua entre ellos.
\par 6 Entonces él y su pueblo acamparon frente a ellos; y cuando vio que el pueblo tenía miedo de cruzar el arroyo, pasó primero por encima de él, y luego los hombres que lo vieron pasaron detrás de él.
\par 7 Hecho esto, dividió a sus hombres y puso a los de a caballo en medio de los de a pie, porque los de a caballo de los enemigos eran muchos.
\par 8 Entonces tocaron las trompetas sagradas, y Cendebeo y su ejército fueron puestos en fuga, de modo que muchos de ellos murieron, y el resto los llevó a la fortaleza.
\par 9 En aquel tiempo fue herido el hermano de Judas Juan; Pero Juan los siguió hasta llegar a Cedrón, que Cendebeo había construido.
\par 10 Entonces huyeron hasta las torres en los campos de Azoto; y lo quemó al fuego, de modo que murieron entre ellos unos dos mil hombres. Después regresó en paz a la tierra de Judea.
\par 11 Además, en la llanura de Jericó fue nombrado capitán Ptolomeo, hijo de Abubo, y tenía plata y oro en abundancia.
\par 12 Porque él era yerno del sumo sacerdote.
\par 13 Por lo tanto, ensoberbecido su corazón, pensó en apoderarse del país, y entonces conspiró engañosamente contra Simón y sus hijos para destruirlos.
\par 14 Simón visitaba las ciudades del campo y se ocupaba de su buen orden; En aquel tiempo descendió él mismo a Jericó con sus hijos Matatías y Judas, en el año ciento sesenta y siete, en el mes undécimo, llamado Sabat:
\par 15 Donde el hijo de Abubus, recibiéndolos engañosamente en una pequeña fortaleza que él había construido, llamada Docus, les hizo un gran banquete, aunque había escondido hombres allí.
\par 16 Entonces, cuando Simón y sus hijos habían bebido mucho, Tolomeo y sus hombres se levantaron, tomaron sus armas, se encontraron con Simón en el lugar del banquete y lo mataron a él, a sus dos hijos y a algunos de sus sirvientes.
\par 17 Con esto cometió una gran traición y devolvió mal por bien.
\par 18 Entonces Tolomeo escribió estas cosas y envió al rey que le enviara un ejército para ayudarle y que le entregara la tierra y las ciudades.
\par 19 También envió otros a Gazera para matar a Juan, y a los tribunos envió cartas para que vinieran a él, para darles plata, oro y recompensas.
\par 20 Y envió a otros para tomar Jerusalén y el monte del templo.
\par 21 Uno de ellos había ido corriendo a Gazera y le había dicho a Juan que su padre y sus hermanos habían sido asesinados, y dijo: Tolomeo ha enviado a matarte a ti también.
\par 22 Cuando oyó esto, quedó muy asombrado, así que echó mano a los que habían venido a destruirlo y los mató; porque sabía que querían alejarlo.
\par 23 En cuanto a los demás hechos de Juan, sus guerras, las hazañas que realizó, la construcción de las murallas que hizo y sus obras,
\par 24 He aquí, esto está escrito en las crónicas de su sacerdocio, desde el momento en que fue nombrado sumo sacerdote después de su padre.

\end{document}