\begin{document}

\title{2 Macabeos}


\chapter{1}

\par 1 Los hermanos judíos que están en Jerusalén y en la tierra de Judea, desean a los hermanos judíos que están en todo Egipto salud y paz:
\par 2 Dios tenga misericordia de vosotros, y acordaos del pacto que hizo con Abraham, Isaac y Jacob, sus fieles servidores;
\par 3 y dad a todos un corazón para servirle y hacer su voluntad con buen ánimo y voluntad;
\par 4 y abran vuestros corazones en su ley y sus mandamientos, y os envíen la paz,
\par 5 Y escucharán vuestras oraciones, serán uno con vosotros y nunca os abandonarán en el momento de la angustia.
\par 6 Y ahora estamos aquí orando por vosotros.
\par 7 Cuando reinó Demetrio, en el año ciento sesenta y nueve, nosotros los judíos os escribimos en medio de la gran angustia que nos sobrevino en aquellos años, desde el momento en que Jasón y su compañía se rebelaron contra la Tierra Santa y Reino,
\par 8 Y quemamos el pórtico y derramamos sangre inocente. Entonces oramos al Señor y fuimos escuchados. Ofrecimos también sacrificios y flor de harina, encendimos las lámparas y dispusimos los panes.
\par 9 Y ahora cuidad de celebrar la fiesta de los Tabernáculos en el mes Casleu.
\par 10 En el año ciento ochenta y ocho, el pueblo que estaba en Jerusalén y en Judea, el concilio y Judas enviaron saludos y salud a Aristóbulo, maestro del rey Ptolomeo, que era del linaje de los sacerdotes ungidos, y a los judíos que estaban en Egipto:
\par 11 Por cuanto Dios nos ha librado de grandes peligros, le damos muchas gracias por haber estado en la batalla contra un rey.
\par 12 Porque expulsó a los que peleaban dentro de la ciudad santa.
\par 13 Porque cuando el líder llegó a Persia, y con él el ejército que parecía invencible, fueron asesinados en el templo de Nanea por el engaño de los sacerdotes de Nanea.
\par 14 Porque Antíoco, como si quisiera casarse con ella, vino al lugar con sus amigos que estaban con él para recibir dinero en concepto de dote.
\par 15 Cuando los sacerdotes de Nanea se pusieron en marcha y él y un pequeño grupo entraron en el recinto del templo, cerraron el templo tan pronto como Antíoco entró.
\par 16 Y abriendo una puerta del tejado, arrojaron piedras como rayos, hirieron al capitán, los despedazaron, les cortaron la cabeza y se las arrojaron a los que estaban fuera.
\par 17 Bendito sea en todo nuestro Dios, que entregó a los impíos.
\par 18 Por lo tanto, mientras que ahora estamos decididos a celebrar la purificación del templo el día veinticinco del mes Casleu, hemos considerado necesario haceros certificar esto, para que también vosotros la guardéis como la fiesta de los tabernáculos, y del fuego que nos fue dado cuando Neemías ofreció sacrificios, después de haber edificado el templo y el altar.
\par 19 Porque cuando nuestros padres fueron llevados a Persia, los sacerdotes entonces devotos tomaron en secreto el fuego del altar y lo escondieron en el lugar hueco de un hoyo sin agua, donde lo guardaron con seguridad, de modo que el lugar fuera desconocido para todos los hombres.
\par 20 Después de muchos años, cuando agradó a Dios, Neemías, enviado por el rey de Persia, envió al fuego a la descendencia de los sacerdotes que lo habían escondido; pero cuando nos lo dijeron, no encontraron fuego, sino agua espesa;
\par 21 Entonces les ordenó que lo sacaran y lo trajeran; y cuando se colocaron los sacrificios, Neemías ordenó a los sacerdotes que rociaran con agua la leña y las cosas puestas encima.
\par 22 Cuando esto sucedió, y cuando llegó el momento en que brillaba el sol, que antes estaba escondido en las nubes, se encendió un gran fuego, de modo que todos se maravillaron.
\par 23 Y mientras se consumía el sacrificio, los sacerdotes hicieron una oración, digo, tanto los sacerdotes como todos los demás, comenzando Jonatán, y los demás respondiendo a ella, como lo hizo Neemías.
\par 24 Y la oración fue de esta manera; Oh Señor, Señor Dios, Creador de todas las cosas, que eres temible y fuerte, justo y misericordioso, y Rey único y misericordioso,
\par 25 Tú, el único dador de todas las cosas, el único justo, todopoderoso y eterno, que libras a Israel de toda angustia, que escogiste a los padres y los santificaste.
\par 26 Recibe el sacrificio por todo tu pueblo Israel, guarda tu porción y santifícala.
\par 27 Reúne a los que están dispersos entre nosotros, libra a los que sirven entre las naciones, mira a los despreciados y aborrecidos, y hazles saber a las naciones que tú eres nuestro Dios.
\par 28 Castigad a los que nos oprimen y con soberbia nos hacen mal.
\par 29 Vuelve a plantar a tu pueblo en tu lugar santo, como dijo Moisés.
\par 30 Y los sacerdotes cantaban salmos de acción de gracias.
\par 31 Una vez consumido el sacrificio, Neemías ordenó que se derramara el agua que quedaba sobre las grandes piedras.
\par 32 Cuando esto sucedió, se encendió una llama, pero la luz que brillaba desde el altar la consumió.
\par 33 Cuando se supo esto, le dijeron al rey de Persia que en el lugar donde los sacerdotes habían sido llevados habían escondido el fuego, apareció agua y que Neemías había purificado con ella los sacrificios.
\par 34 Entonces el rey, cercando el lugar, lo santificó después de haber examinado el asunto.
\par 35 Y el rey tomó muchos regalos y los repartió entre aquellos a quienes quería complacer.
\par 36 Y Neemías llamó a esto Naftar, que es como decir limpieza; pero muchos lo llaman Nefi.

\chapter{2}

\par 1 También se encuentra en los anales que el profeta Jeremías ordenó a los llevados que tomaran del fuego, como se ha indicado:
\par 2 Y cómo el profeta, después de haberles dado la ley, les encargó que no olvidaran los mandamientos del Señor y que no se equivocaran en su mente al ver imágenes de plata y de oro con sus adornos.
\par 3 Y con otras palabras similares les exhortaba a que la ley no se apartara de sus corazones.
\par 4 También está escrito en el mismo escrito que el profeta, advertido por Dios, ordenó que el tabernáculo y el arca fueran con él, mientras subía al monte donde Moisés subió y vio la herencia de Dios.
\par 5 Y cuando Jeremías llegó allí, encontró una cueva hueca, donde puso el tabernáculo, el arca y el altar del incienso, y cerró la puerta.
\par 6 Y algunos de los que lo seguían vinieron a marcar el camino, pero no pudieron encontrarlo.
\par 7 Al darse cuenta de esto, Jeremías los reprendió, diciendo: En cuanto a ese lugar, será desconocido hasta el día en que Dios reúna a su pueblo y los reciba en misericordia.
\par 8 Entonces el Señor les mostrará estas cosas, y aparecerá la gloria del Señor, y también la nube, como se mostró en tiempos de Moisés, y como cuando Salomón deseó que el lugar fuera santificado con honor.
\par 9 También se declaró que él, siendo sabio, ofreció el sacrificio de la dedicación y de la terminación del templo.
\par 10 Y como cuando Moisés oró al Señor, descendió fuego del cielo y consumió los sacrificios, así también oró Salomón, y descendió fuego del cielo y consumió los holocaustos.
\par 11 Y Moisés dijo: Como la ofrenda por el pecado no se podía comer, se consumió.
\par 12 Así que Salomón guardó aquellos ocho días.
\par 13 Lo mismo se cuenta en los escritos y comentarios de Neemías; y cómo fundó una biblioteca y reunió las actas de los reyes, y de los profetas, y de David, y las epístolas de los reyes acerca de los santos dones.
\par 14 De la misma manera también Judas reunió todas las cosas que se habían perdido a causa de la guerra que tuvimos, y quedaron con nosotros,
\par 15 Por tanto, si lo necesitáis, enviad a algunos para que os lo traigan.
\par 16 Ya que estamos a punto de celebrar la purificación, os hemos escrito: y os irá bien si guardáis los mismos días.
\par 17 Esperamos también que el Dios que libró a todo su pueblo y les dio a todos la herencia, el reino, el sacerdocio y el santuario,
\par 18 Pronto tendrá misericordia de nosotros, como lo prometió en la ley, y nos reunirá de todas las tierras bajo el cielo en el lugar santo, porque él nos libró de grandes angustias y purificó el lugar.
\par 19 En cuanto a Judas Macabeo y a sus hermanos, y a la purificación del gran templo y a la dedicación del altar,
\par 20 Y las guerras contra Antíoco Epífanes y Eupátor su hijo,
\par 21 Y las señales manifiestas que vinieron del cielo a aquellos que se comportaron virilmente en su honor por el judaísmo: de modo que, siendo sólo unos pocos, dominaron todo el país y persiguieron multitudes bárbaras,
\par 22 Y reconstruyeron el templo famoso en todo el mundo, liberaron la ciudad y mantuvieron las leyes que estaban en vigor, siendo el Señor misericordioso con ellos.
\par 23 Todas estas cosas, digo, habiendo sido declaradas por Jasón de Cirene en cinco libros, intentaremos resumirlas en un solo volumen.
\par 24 Por considerar el número infinito, y la dificultad que encuentran en el deseo de examinar las narraciones de la historia, por la variedad del asunto,
\par 25 Hemos tenido cuidado de que los que lean se deleiten, de que los que desean memorizarlo se sientan cómodos y de que todos aquellos a quienes llegue la lectura obtengan beneficio.
\par 26 Por lo tanto, para nosotros, que hemos asumido esta dolorosa tarea de reducir, no fue fácil, sino una cuestión de sudor y vigilancia;
\par 27 Así como no es fácil para quien prepara un banquete y busca el beneficio de otros, sin embargo, para complacer a muchos, emprenderemos con gusto este gran esfuerzo;
\par 28 Dejando al autor el manejo exacto de cada detalle, y trabajando para seguir las reglas de un resumen.
\par 29 Porque como el arquitecto de una casa nueva debe cuidar de todo el edificio; pero el que se propone colocarlo y pintarlo, debe buscar cosas adecuadas para adornarlo: así creo que nos pasa a nosotros.
\par 30 Estar en cada punto, examinar las cosas en general y sentir curiosidad por los detalles, pertenece al primer autor de la historia:
\par 31 Pero al que haga un compendio se le concederá ser breve y evitar mucho trabajo.
\par 32 Aquí, pues, comenzaremos la historia, añadiendo sólo esto a lo dicho: que es una tontería hacer un prólogo largo y ser breve en la historia misma.

\chapter{3}

\par 1 Cuando la ciudad santa estaba habitada en plena paz y las leyes se guardaban muy bien, a causa de la piedad del sumo sacerdote Onías y de su odio a la maldad,
\par 2 Y sucedió que incluso los reyes mismos honraron el lugar y engrandecieron el templo con sus mejores regalos;
\par 3 De modo que Seleuco de Asia se hizo cargo de sus propios ingresos con todos los gastos correspondientes al servicio de los sacrificios.
\par 4 Pero un tal Simón de la tribu de Benjamín, que era nombrado gobernador del templo, se peleó con el sumo sacerdote por el desorden en la ciudad.
\par 5 Y como no pudo vencer a Onías, lo llevó a Apolonio, hijo de Traseas, que entonces era gobernador de Celosiria y Fenicia,
\par 6 Y le dijo que el tesoro de Jerusalén estaba lleno de sumas infinitas de dinero, de modo que la multitud de sus riquezas, que no pertenecían a la cuenta de los sacrificios, era innumerable, y que era posible reunirlas todas en la mano del rey.
\par 7 Cuando Apolonio llegó al rey y le mostró el dinero que le había dicho, el rey eligió a su tesorero Heliodoro y le envió con la orden de que le trajera el dinero mencionado.
\par 8 Inmediatamente Heliodoro emprendió su viaje; bajo el pretexto de visitar las ciudades de Celosyria y Fenice, pero de hecho para cumplir el propósito del rey.
\par 9 Cuando llegó a Jerusalén y fue recibido cortésmente por el sumo sacerdote de la ciudad, éste le contó qué información le habían dado sobre el dinero, le explicó el motivo de su visita y le preguntó si en verdad era así.
\par 10 Entonces el sumo sacerdote le dijo que había dinero reservado para socorrer a las viudas y a los huérfanos:
\par 11 Y que parte de ello pertenecía a Hircano hijo de Tobías, un hombre de gran dignidad, y no como el malvado Simón había dicho mal: la suma total era cuatrocientos talentos de plata y doscientos de oro.
\par 12 Y que era completamente imposible que se les hiciera tal daño a aquellos que lo habían comprometido a la santidad del lugar y a la majestuosidad e inviolable santidad del templo, venerado en todo el mundo.
\par 13 Pero Heliodoro, a causa de la orden que le había dado el rey, dijo: «De cualquier manera debe ser llevado al tesoro del rey».
\par 14 Así que el día que había fijado entró para ordenar este asunto; por lo que hubo no poca agonía en toda la ciudad.
\par 15 Pero los sacerdotes, postrándose ante el altar con sus vestiduras sacerdotales, invocaron al cielo al que había dictado la ley sobre las cosas que él debía guardar, para que fueran preservadas con seguridad para quienes las habían encomendado.
\par 16 Entonces, cualquiera que hubiera mirado al sumo sacerdote cara a cara, habría herido su corazón: porque su semblante y el cambio de su color declaraban la agonía interior de su mente.
\par 17 Porque el hombre estaba tan abrumado por el miedo y el horror del cuerpo, que era evidente para los que lo miraban el dolor que tenía ahora en su corazón.
\par 18 Otros salieron corriendo de sus casas para acudir a la súplica general, porque el lugar estaba a punto de ser despreciado.
\par 19 Y las mujeres, ceñidas con cilicio debajo del pecho, abundaban en las calles, y las vírgenes que estaban encerradas corrían, unas hacia las puertas, otras hacia las murallas, y otras miraban por las ventanas.
\par 20 Y todos, con las manos hacia el cielo, rogaban.
\par 21 Entonces a un hombre le daría lástima ver cómo la multitud de todas clases se desplomaba y el temor de que el sumo sacerdote estuviera en tal agonía.
\par 22 Entonces pidieron al Señor Todopoderoso que mantuviera las cosas confiadas seguras y seguras para quienes las habían cometido.
\par 23 Sin embargo, Heliodoro ejecutó lo que estaba decretado.
\par 24 Mientras se presentaba con su guardia cerca del tesoro, el Señor de los espíritus y Príncipe de todo poder hizo una gran aparición, de modo que todos los que se atrevían a entrar con él quedaron asombrados del poder de Dios, y desmayaron, y tuvieron mucho miedo.
\par 25 Porque se les apareció un caballo con un jinete terrible encima, y ​​adornado con un manto muy hermoso, y corrió ferozmente, y golpeó a Heliodoro con sus patas delanteras, y parecía que el que estaba sentado en el caballo había completado arnés de oro.
\par 26 Además aparecieron ante él otros dos jóvenes, notables en fuerza, excelentes en belleza y hermosos vestidos, que estaban junto a él a cada lado; y lo azotaron continuamente y le dieron muchos azotes dolorosos.
\par 27 Y Heliodoro cayó repentinamente al suelo y se vio rodeado de una gran oscuridad; pero los que estaban con él lo levantaron y lo pusieron en una litera.
\par 28 Así, al que acababa de entrar con un gran séquito y con toda su guardia en dicho tesoro, lo sacaron, no pudiendo ayudarse con sus armas, y claramente reconocieron el poder de Dios.
\par 29 Porque él, por mano de Dios, fue abatido y quedó mudo, sin toda esperanza de vida.
\par 30 Pero ellos alabaron al Señor, que milagrosamente había honrado su propio lugar: por el templo; que un poco antes estaba llena de temor y angustia, cuando apareció el Señor Todopoderoso, se llenó de gozo y alegría.
\par 31 Inmediatamente algunos amigos de Heliodoro rogaron a Onías que invocara al Altísimo para que le concediera la vida, quien estaba dispuesto a entregar el espíritu.
\par 32 Entonces el sumo sacerdote, sospechando que el rey no creyera que los judíos habían cometido alguna traición contra Heliodoro, ofreció un sacrificio por la salud del hombre.
\par 33 Mientras el sumo sacerdote estaba haciendo la expiación, aparecieron aquellos jóvenes vestidos con las mismas vestiduras y se pusieron junto a Heliodoro, diciendo: Dad muchas gracias al sumo sacerdote Onías, por cuanto por él el Señor te ha concedido la vida.
\par 34 Y, puesto que has sido azotado desde el cielo, declara a todos los hombres el gran poder de Dios. Y cuando hubieron dicho estas palabras, ya no aparecieron.
\par 35 Entonces Heliodoro, después de ofrecer sacrificios al Señor, hacer grandes votos al que le había salvado la vida y saludar a Onías, regresó con su ejército al rey.
\par 36 Entonces dio testimonio a todos de las obras del gran Dios, que había visto con sus ojos.
\par 37 Y cuando el rey Heliodoro, que podría ser un hombre apto para ser enviado una vez más a Jerusalén, dijo:
\par 38 Si tienes algún enemigo o traidor, envíalo allí y lo recibirás bien azotado si escapa con vida; porque en ese lugar, sin duda; hay un poder especial de Dios.
\par 39 Porque el que habita en el cielo tiene sus ojos puestos en ese lugar y lo defiende; y golpea y destruye a los que vienen a dañarla.
\par 40 Y de esta manera se desarrollaron las cosas relativas a Heliodoro y la administración del tesoro.

\chapter{4}

\par 1 Este Simón, de quien antes hemos hablado, habiendo sido un traidor al dinero y a su patria, calumnió a Onías, como si hubiera aterrorizado a Heliodoro y hubiera sido el autor de estos males.
\par 2 Por eso se atrevió a llamar traidor a aquel que merecía el bien de la ciudad, cuidaba a su nación y era tan celoso de las leyes.
\par 3 Pero cuando su odio llegó a tal punto que uno de los secuaces de Simón cometió asesinatos,
\par 4 Onías, viendo el peligro de esta contienda y que Apolonio, como gobernador de Celosiria y Fenicia, se enfurecía y aumentaba la malicia de Simón,
\par 5 Vino al rey, no para acusar a sus compatriotas, sino para buscar el bien de todos, tanto público como privado:
\par 6 Porque vio que era imposible que el estado permaneciera tranquilo y Simón abandonara su locura, a menos que el rey se ocupara de ello.
\par 7 Pero después de la muerte de Seleuco, cuando Antíoco, llamado Epífanes, tomó el reino, Jasón, hermano de Onías, trabajó en secreto para ser sumo sacerdote,
\par 8 Prometiendo al rey por intercesión trescientos sesenta talentos de plata, y de otra renta ochenta talentos:
\par 9 Además de esto, prometió asignar ciento cincuenta más, si tenía licencia para establecerle un lugar para el ejercicio y la formación de los jóvenes en las modas de los paganos, y para escribirles en Jerusalén con el nombre de antioqueños.
\par 10 Lo cual, cuando el rey lo concedió y tomó el poder en sus manos, inmediatamente puso a su nación a la manera griega.
\par 11 Y los privilegios reales concedidos a los judíos por medio de Juan, padre de Eupólemo, que fue embajador en Roma en busca de amistad y ayuda, los quitó; y derribando los gobiernos que estaban conforme a la ley, hizo surgir nuevas costumbres contra la ley:
\par 12 Porque con mucho gusto construyó un lugar de ejercicio debajo de la torre, y sometió a los jóvenes principales a su sujeción, y les obligó a llevar sombrero.
\par 13 Tal fue el auge de las modas griegas y el aumento de las costumbres paganas a causa de la excesiva profanidad de Jasón, aquel malvado impío que no era sumo sacerdote;
\par 14 Que los sacerdotes no tuvieron valor para seguir sirviendo en el altar, sino que despreciando el templo y descuidando los sacrificios, se apresuraron a participar de la ración ilegal en el lugar del ejercicio, después de que el juego del disco los convocó;
\par 15 No se dejan llevar por los honores de sus padres, sino que prefieren la gloria de los griegos.
\par 16 Por lo cual les sobrevino una terrible calamidad, porque tenían por enemigos y vengadores a aquellos cuya costumbre seguían con tanto fervor y a quienes deseaban ser semejantes en todo.
\par 17 Porque no es fácil hacer lo malo contra las leyes de Dios; pero el tiempo siguiente declarará estas cosas.
\par 18 Cuando se celebró en Tiro el juego que se usaba cada año religioso, estando presente el rey,
\par 19 Este descortés Jasón envió mensajeros especiales desde Jerusalén, que eran antioqueños, para llevar trescientas dracmas de plata para el sacrificio de Hércules, las cuales ni siquiera sus portadores consideraron oportuno donar para el sacrificio, porque no era conveniente, pero reservarse para otros cargos.
\par 20 Este dinero, pues, fue destinado al sacrificio de Hércules en relación con el remitente; pero a causa de sus portadores, se empleó en hacer galeras.
\par 21 Cuando Apolonio, hijo de Menesteo, fue enviado a Egipto para la coronación del rey Ptolomeo Filometor, Antíoco, al ver que no estaba bien afectado en sus asuntos, se preocupó por su propia seguridad, después de lo cual llegó a Jope, y de allí a Jerusalén:
\par 22 Allí fue recibido con honores por Jasón y por la ciudad, y lo llevaron con antorchas encendidas y con grandes gritos, y después se dirigió con su ejército a Fenicia.
\par 23 Tres años después, Jasón envió a Menelans, el hermano de Simón, para que llevara el dinero al rey y le informara de ciertos asuntos necesarios.
\par 24 Pero él, llevado ante el rey, después de haberlo engrandecido por la gloriosa apariencia de su poder, obtuvo para sí el sacerdocio, ofreciendo trescientos talentos de plata más que Jasón.
\par 25 Llegó, pues, con el mandato del rey, sin traer nada digno de sumo sacerdocio, sino con la ira de un tirano cruel y la furia de una bestia salvaje.
\par 26 Entonces Jasón, que había minado a su propio hermano, siendo debilitado por otro, se vio obligado a huir al país de los amonitas.
\par 27 Menelano obtuvo el principado, pero no aceptó el dinero que había prometido al rey, aunque Sóstratis, el gobernador del castillo, lo pidió.
\par 28 Porque a él le pertenecía la recopilación de las costumbres. Por lo que ambos fueron llamados ante el rey.
\par 29 Menelano dejó en su lugar a su hermano Lisímaco en el sacerdocio; y Sóstrato abandonó a Crates, que era gobernador de los chipriotas.
\par 30 Mientras se hacían estas cosas, los de Tarso y Mallo se rebelaron, porque habían sido entregados a la concubina del rey, llamada Antíoco.
\par 31 Entonces el rey vino a toda prisa para apaciguar las cosas, dejando a Andrónico, un hombre con autoridad, como su sustituto.
\par 32 Menelan, creyendo que había llegado el momento oportuno, robó del templo algunos vasos de oro, y dio algunos a Andrónico, y otros los vendió a Tiro y a las ciudades de los alrededores.
\par 33 Lo cual, sabiendo Onías la certeza, lo reprendió y se retiró a un santuario en Dafne, cerca de Antioquía.
\par 34 Menelan, desarmando a Andrónico, le rogó que pusiera a Onías en sus manos; El cual, persuadido de ello, vino a Onías con engaño y le dio su mano derecha con juramento; y aunque sospechaba de él, lo persuadió a salir del santuario, a quien inmediatamente encerró sin tener en cuenta la justicia.
\par 35 Por esta causa, no sólo los judíos, sino también muchos de otras naciones se indignaron mucho y se entristecieron mucho por el injusto asesinato de aquel hombre.
\par 36 Y cuando el rey volvió de los alrededores de Cilicia, los judíos que estaban en la ciudad, y algunos de los griegos que también aborrecían este hecho, se quejaron de que Onías había sido asesinado sin causa.
\par 37 Entonces Antíoco se arrepintió de todo corazón, se compadeció y lloró por la conducta sobria y modesta del que había muerto.
\par 38 Entonces, enfurecido, le quitó a Andrónico su púrpura, le rasgó las vestiduras y, llevándolo por toda la ciudad hasta el mismo lugar donde había cometido impiedad contra Onías, mató al maldito asesino. Así el Señor le recompensó el castigo que merecía.
\par 39 Cuando Lisímaco había cometido muchos sacrilegios en la ciudad con el consentimiento de Menelanos, y sus frutos se habían difundido, la multitud se reunió contra Lisímaco, llevándose ya muchos vasos de oro.
\par 40 Entonces el pueblo, levantándose y lleno de ira, Lisímaco armó unos tres mil hombres y comenzó a hacer violencia; Un tal Auranus era el líder, un hombre avanzado en años, y no menos en locura.
\par 41 Viendo entonces la tentativa de Lisímaco, unos cogieron piedras, otros palos, otros cogieron puñados de polvo que tenían a mano y los arrojaron todos juntos sobre Lisímaco y los que se lanzaban sobre ellos.
\par 42 Así hirieron a muchos de ellos, a algunos los derribaron y a todos los obligaron a huir; pero al ladrón de la iglesia mismo lo mataron junto al tesoro.
\par 43 Por estas cosas, pues, se acusó a Menelanos.
\par 44 Cuando el rey llegó a Tiro, tres hombres enviados por el Senado defendieron la causa ante él:
\par 45 Pero Menelano, ya convencido, prometió a Ptolomeo, hijo de Dorímenes, que le daría mucho dinero si conseguía apaciguar al rey con él.
\par 46 Entonces Tolomeo, llevando aparte al rey a cierta galería, como para tomar el aire, le hizo cambiar de opinión:
\par 47 De modo que absolvió de las acusaciones a Menelanos, que sin embargo era el causante de todos los males; y a aquellos pobres hombres que, si hubieran contado su causa, incluso ante los escitas, habrían sido juzgados inocentes, los condenó a muerte.
\par 48 Así, los que seguían el asunto de la ciudad, del pueblo y de los objetos sagrados pronto sufrieron un castigo injusto.
\par 49 Por lo que incluso los de Tiro, movidos por el odio por aquella mala acción, hicieron que los enterraran honorablemente.
\par 50 Y así, por la codicia de los poderosos, Menelanos permaneció todavía en el poder, aumentando su malicia y siendo un gran traidor a los ciudadanos.

\chapter{5}

\par 1 Casi al mismo tiempo Antíoco preparaba su segundo viaje a Egipto:
\par 2 Y aconteció que por toda la ciudad, durante casi cuarenta días, se vieron corriendo por el aire jinetes vestidos con vestidos de oro y armados de lanzas, como una tropa de soldados,
\par 3 Y tropas de jinetes en formación, enfrentándose y corriendo unos contra otros, agitando escudos, y multitud de picas, desenvainando espadas, lanzando dardos, y luciendo adornos de oro y arneses de toda clase.
\par 4 Por lo tanto, todos oraron para que esa aparición se convirtiera en bien.
\par 5 Cuando se corrió el falso rumor de que Antíoco había muerto, Jasón tomó al menos mil hombres y de repente atacó la ciudad; y siendo devueltos los que estaban en las murallas, y finalmente tomada la ciudad, Menelans huyó al castillo:
\par 6 Pero Jasón mató a sus propios ciudadanos sin piedad, sin considerar que acabar con los de su propia nación sería un día muy infeliz para él; pero pensando que habían sido sus enemigos, y no sus compatriotas, a quienes venció.
\par 7 Pero a pesar de todo esto no obtuvo el principado, sino que al final recibió la vergüenza por la recompensa de su traición y huyó de nuevo al país de los amonitas.
\par 8 Al final tuvo un regreso desgraciado: fue acusado ante Aretas, rey de los árabes, huyendo de ciudad en ciudad, perseguido por todos, odiado como un transgresor de las leyes y abominado como un abierto enemigo de su país y de sus compatriotas, fue expulsado a Egipto.
\par 9 Así, el que había expulsado a muchos de su país, pereció en tierra extraña, retirándose a los Lacedemonios, pensando encontrar allí ayuda por parte de sus parientes.
\par 10 Y el que había expulsado a muchos insepultos no tenía nadie con quien llorar por él, ni funerales solemnes, ni sepulcro con sus padres.
\par 11 Cuando esto sucedió llegó al carro del rey, pensó que Judea se había rebelado, y saliendo de Egipto enfurecido, tomó la ciudad por la fuerza.
\par 12 Y ordenó a sus hombres de guerra que no perdonaran a los que encontraran y que mataran a los que subieran a las casas.
\par 13 Así se mataba a jóvenes y viejos, se despojaba a hombres, mujeres y niños, se mataba a vírgenes y a niños.
\par 14 Y en el espacio de tres días enteros fueron destruidos ochenta mil, de los cuales cuarenta mil murieron en el combate; y no menos vendidos que asesinados.
\par 15 Pero no se contentó con esto, sino que se atrevió a entrar en el templo más santo de todo el mundo; Menelans, ese traidor a las leyes y a su propia patria, siendo su guía:
\par 16 Y tomando con manos impuras los vasos sagrados, y con manos profanas derribando los objetos que otros reyes habían dedicado para el engrandecimiento, la gloria y el honor del lugar, los entregó.
\par 17 Y Antíoco tenía una mente tan altiva que no pensó que el Señor estaba enojado por un tiempo por los pecados de los habitantes de la ciudad, y por eso sus ojos no estaban puestos en ese lugar.
\par 18 Porque, si antes no hubieran estado envueltos en muchos pecados, éste, nada más llegar, habría sido azotado y despojado de su soberbia, como lo fue Heliodoro, a quien el rey Seleuco envió a ver el tesoro.
\par 19 Sin embargo, Dios no escogió al pueblo por el lugar, sino el lugar por el pueblo.
\par 20 Y por eso el lugar mismo, que participó con ellos en la adversidad que aconteció a la nación, comunicó después los beneficios enviados por el Señor; y así como fue abandonado en la ira del Todopoderoso, así también el gran Señor reconciliado, fue establecido con toda gloria.
\par 21 Entonces Antíoco, habiendo sacado del templo mil ochocientos talentos, partió a toda prisa hacia Antioquía, orgulloso de hacer navegable la tierra y transitable a pie el mar. Tal era la altivez de su mente.
\par 22 Y dejó gobernadores para que molestaran a la nación: en Jerusalén, a Felipe, por su país frigio y por sus costumbres más bárbaras que las del que lo puso allí;
\par 23 Y en Garizim, Andrónico; y además Menelanos, que peor que todos los demás, ejercía mano dura sobre los ciudadanos, teniendo una mente maliciosa contra sus compatriotas los judíos.
\par 24 Envió también al detestable cabecilla Apolonio con un ejército de veintidós mil personas, ordenándole matar a todos los que estaban en su mejor edad y vender a las mujeres y a los más jóvenes.
\par 25 Este, viniendo a Jerusalén y fingiendo estar en paz, se detuvo hasta el día santo del sábado, cuando tomó a los judíos que celebraban el día santo y ordenó a sus hombres que se armaran.
\par 26 Y así mató a todos los que habían ido a celebrar el sábado, y corriendo por la ciudad con armas mató a grandes multitudes.
\par 27 Pero Judas Macabeo, con otros nueve aproximadamente, se retiró al desierto y vivió en las montañas como las bestias, con su compañía, que se alimentaba continuamente de hierbas, para no ser partícipes de la contaminación.

\chapter{6}

\par 1 Poco después el rey envió a un anciano de Atenas para obligar a los judíos a apartarse de las leyes de sus padres y a no vivir según las leyes de Dios:
\par 2 Y también contaminar el templo de Jerusalén, llamándolo templo de Júpiter Olimpio; y el de Garizim, de Júpiter, el Defensor de los extraños, como deseaban los que habitaban en el lugar.
\par 3 La llegada de este mal fue dolorosa y dolorosa para el pueblo:
\par 4 Porque el templo estaba lleno de alboroto y juerga de los gentiles, que se juntaban con rameras y se relacionaban con mujeres dentro del recinto de los lugares santos, y además traían cosas que no eran lícitas.
\par 5 También el altar estaba lleno de cosas profanas que la ley prohíbe.
\par 6 Tampoco era lícito al hombre guardar días de reposo ni ayunos antiguos, ni declararse judío en absoluto.
\par 7 Y cada mes, el día del nacimiento del rey, los obligaban a comer de los sacrificios con amarga obligación; y cuando se guardaba el ayuno de Baco, los judíos se veían obligados a ir en procesión hacia Baco, llevando hiedra.
\par 8 Además, por sugerencia de Tolomeo, se emitió un decreto en las ciudades vecinas de los paganos contra los judíos, ordenándoles que observaran las mismas costumbres y participaran de sus sacrificios:
\par 9 Y quien no se conforme con las costumbres de los gentiles, será condenado a muerte. Entonces un hombre podría haber visto la miseria actual.
\par 10 Porque trajeron dos mujeres que habían circuncidado a sus hijos; a quienes, habiendo conducido abiertamente alrededor de la ciudad, con los niños agarrados al pecho, los arrojaron de cabeza desde la muralla.
\par 11 Otros, que se habían reunido en cuevas cercanas para guardar en secreto el día de reposo, fueron descubiertos por Felipe y fueron quemados todos juntos, porque tenían la conciencia de ayudarse a sí mismos para honrar el día santísimo.
\par 12 Ahora ruego a los que lean este libro que no se desanimen por estas calamidades, sino que consideren que esos castigos no son para destrucción, sino para un castigo de nuestra nación.
\par 13 Porque es una señal de su gran bondad el que los malhechores no sean tolerados por mucho tiempo, sino castigados inmediatamente.
\par 14 Porque no como con otras naciones, a quienes el Señor pacientemente se abstiene de castigar hasta que lleguen a la plenitud de sus pecados, así nos trata a nosotros,
\par 15 No sea que, habiendo llegado al colmo del pecado, después se vengue de nosotros.
\par 16 Por eso nunca retira de nosotros su misericordia, y aunque castiga con adversidades, nunca abandona a su pueblo.
\par 17 Pero que esto que hemos dicho nos sirva de advertencia. Y ahora pasaremos a exponer el asunto en pocas palabras.
\par 18 Eleazar, uno de los escribas principales, un hombre anciano y de hermoso rostro, fue obligado a abrir la boca y comer carne de cerdo.
\par 19 Pero él, prefiriendo morir gloriosamente a vivir manchado con tal abominación, la escupió y vino por su propia voluntad al tormento.
\par 20 Como les convenía venir, que están decididos a oponerse a cosas que no son lícitas para gustar el amor a la vida.
\par 21 Pero los encargados de aquel festín malvado, a causa de la antigua relación que tenían con aquel hombre, lo llevaron aparte y le rogaron que trajera carne de su propia provisión, que le fuera lícita usar, y que hiciera como si comiera de la carne extraída del sacrificio ordenado por el rey;
\par 22 Para que, al hacerlo, pudiera librarse de la muerte y recuperar su antigua amistad con ellos.
\par 23 Pero él comenzó a considerar discretamente lo que correspondía a su edad, a la excelencia de su vejez, al honor de su cabello gris, de donde procedía, y a su educación más honesta desde niño, o mejor dicho, la santa ley hecha y dada por Dios: por lo tanto, respondió en consecuencia, y ordenó que lo enviaran inmediatamente a la tumba.
\par 24 Porque no es propio de nuestra época, dijo, fingir en modo alguno que muchos jóvenes pudieran pensar que Eleazar, que tenía ochenta años y diez años, había adoptado una religión extraña;
\par 25 Y así ellos, por mi hipocresía y mi deseo de vivir un poco de tiempo y un momento más, serán engañados por mí, y mancharé mi vejez y la haré abominable.
\par 26 Aunque por el momento sea librado del castigo de los hombres, no escaparé de la mano del Todopoderoso, ni vivo ni muerto.
\par 27 Por lo tanto, ahora, cambiando valientemente esta vida, me mostraré como lo requiere mi edad,
\par 28 Y deja un ejemplo notable a los jóvenes que mueren voluntaria y valientemente por las leyes santas y honorables. Y habiendo dicho estas palabras, inmediatamente fue al tormento:
\par 29 Los que lo guiaron cambiaron la buena voluntad que un poco antes le habían mostrado en odio, porque tales discursos procedían, según pensaban, de una mente desesperada.
\par 30 Pero cuando estaba a punto de morir azotado, gimió y dijo: Es manifiesto al Señor, que tiene el santo conocimiento, que mientras podría haber sido librado de la muerte, ahora sufro dolorosos dolores en el cuerpo por siendo azotado; pero en el alma estoy contento de sufrir estas cosas, porque le temo.
\par 31 Y así murió este hombre, dejando su muerte como ejemplo de noble valentía y memorial de virtud, no sólo para los jóvenes, sino para toda su nación.

\chapter{7}

\par 1 Aconteció también que siete hermanos y su madre fueron apresados ​​y obligados por el rey, contra la ley, a probar carne de cerdo, y fueron azotados con azotes y látigos.
\par 2 Pero uno de los que hablaron primero dijo así: ¿Qué quieres preguntar o aprender de nosotros? estamos dispuestos a morir antes que a transgredir las leyes de nuestros padres.
\par 3 Entonces el rey, enojado, mandó que se calentaran sartenes y calderos.
\par 4 Entonces, enardecido, ordenó que al que hablaba primero le cortaran la lengua y le cortaran todo el cuerpo, ante la vista de los demás hermanos y de su madre.
\par 5 Ahora bien, cuando quedó mutilado en todos sus miembros, mandó que, aún vivo, lo llevaran al fuego y lo frieran en la sartén; y como el vapor de la sartén se había dispersado durante un buen espacio, ellos Se exhortaban unos a otros con la madre a morir valientemente, diciendo así:
\par 6 El Señor Dios nos mira y en verdad tiene consuelo en nosotros, como lo declaró Moisés en su cántico, que atestiguó frente a ellos, diciendo: Y será consolado en sus siervos.
\par 7 Cuando el primero murió después de este número, trajeron al segundo para burlarse de él; y después de arrancarle la piel de la cabeza con el pelo, le preguntaron: ¿Comerás antes de morir? castigado en cada miembro de tu cuerpo?
\par 8 Pero él respondió en su propia lengua y dijo: No. Por lo que también recibió el siguiente tormento, como el primero.
\par 9 Y cuando estaba en el último suspiro, dijo: Tú como un furor nos sacas de esta vida presente, pero el Rey del mundo nos resucitará a la vida eterna, a los que hemos muerto por sus leyes.
\par 10 Después de él, el tercero fue puesto en ridículo; y cuando lo reclamaron, sacó la lengua, y pronto extendió las manos valientemente.
\par 11 Y dijo con valentía: Estos los recibí del cielo; y por sus leyes las desprecio; y de él espero volver a recibirlos.
\par 12 De tal manera que el rey y los que estaban con él se maravillaron del coraje del joven, pues no se preocupaba de los dolores.
\par 13 Cuando también éste murió, al cuarto lo atormentaron y lo destrozaron de la misma manera.
\par 14 Entonces, cuando estaba a punto de morir, dijo así: Bueno es, habiendo sido muerto por los hombres, esperar de Dios la esperanza de ser resucitado por él; en cuanto a ti, no tendrás resurrección a la vida.
\par 15 Después trajeron también al quinto y lo destrozaron.
\par 16 Entonces miró al rey y dijo: Tú tienes poder sobre los hombres, eres corruptible, haces lo que quieres; sin embargo, no penséis que nuestra nación está abandonada por Dios;
\par 17 Pero espera un poco y contempla su gran poder, cómo te atormentará a ti y a tu descendencia.
\par 18 Después de él trajeron también al sexto, el cual, estando a punto de morir, dijo: No os engañéis sin causa, porque nosotros mismos sufrimos estas cosas, habiendo pecado contra nuestro Dios; por eso nos hacen cosas maravillosas.
\par 19 Pero tú, que estás decidido a luchar contra Dios, no pienses que quedarás impune.
\par 20 Pero la madre era sobre todo maravillosa y digna de honorable memoria: porque cuando vio a sus siete hijos asesinados en el espacio de un día, lo sobrellevó con buen ánimo, por la esperanza que tenía en el Señor.
\par 21 Y ella, llena de ánimo valiente, exhortó a cada uno de ellos en su propia lengua; y excitando sus pensamientos femeninos con un estómago varonil, les dijo:
\par 22 No puedo decir cómo entrasteis en mi vientre; porque ni os di aliento ni vida, ni fui yo quien formé los miembros de cada uno de vosotros;
\par 23 Pero sin duda el Creador del mundo, que formó la generación del hombre y descubrió el principio de todas las cosas, también por su misericordia os dará de nuevo aliento y vida, ya que ahora no os consideráis a vosotros mismos como suyos por el bien de las leyes.
\par 24 Antíoco, creyéndose despreciado y sospechando que se trataba de unas palabras de reproche, mientras el menor aún vivía, no sólo le exhortó con palabras, sino que también le aseguró con juramentos que le haría rico y un hombre feliz, si se apartara de las leyes de sus padres; y que también lo tomaría por amigo y le confiaría sus asuntos.
\par 25 Pero como el joven no quiso escucharlo, el rey llamó a su madre y la exhortó a que aconsejara al joven que le salvara la vida.
\par 26 Y él, después de haberla exhortado con muchas palabras, ella le prometió que aconsejaría a su hijo.
\par 27 Pero ella, inclinándose ante él, burlándose del cruel tirano, habló de esta manera en la lengua de su país; Oh hijo mío, ten piedad de mí, que te llevé nueve meses en mi vientre, y te di tres años, y te alimenté, y te crié hasta esta edad, y soporté las penas de la educación.
\par 28 Te ruego, hijo mío, que mires el cielo y la tierra y todo lo que en ellos hay, y consideres que Dios los hizo de cosas que no eran; y así fue hecha la humanidad de la misma manera.
\par 29 No temas a este verdugo, sino que, siendo digno de tus hermanos, acepta tu muerte para que yo pueda volver a recibirte en misericordia con tus hermanos.
\par 30 Mientras ella aún hablaba estas palabras, el joven dijo: ¿A quién esperáis? No obedeceré el mandamiento del rey, sino que obedeceré el mandamiento de la ley que fue dada a nuestros padres por medio de Moisés.
\par 31 Y tú, que has sido el autor de todos los males contra los hebreos, no escaparás de las manos de Dios.
\par 32 Porque sufrimos a causa de nuestros pecados.
\par 33 Y aunque el Señor viviente se enoje un poco con nosotros por nuestro castigo y corrección, volverá a ser uno con sus siervos.
\par 34 Pero tú, impío y entre todos los más malvados, no te envanezcas sin causa, ni te envanezcas con esperanzas inciertas, alzando tu mano contra los siervos de Dios.
\par 35 Porque todavía no has escapado del juicio del Dios Todopoderoso, que todo lo ve.
\par 36 Porque nuestros hermanos, que ahora han sufrido un breve dolor, están muertos bajo el pacto de vida eterna de Dios; pero tú, por el juicio de Dios, recibirás el justo castigo por tu orgullo.
\par 37 Pero yo, como hermanos, ofrezco mi cuerpo y mi vida por las leyes de nuestros padres, rogando a Dios que pronto tenga misericordia de nuestra nación; y para que mediante tormentos y plagas confieses que sólo él es Dios;
\par 38 Y para que en mí y en mis hermanos cese la ira del Todopoderoso que con justicia cae sobre nuestra nación.
\par 39 Entonces el rey, enojado, le entregó algo peor que a todos los demás y se entristeció de que se burlaran de él.
\par 40 Así que éste murió sin mancha y puso toda su confianza en el Señor.
\par 41 Finalmente, después de que murieron los hijos, murió la madre.
\par 42 Basta ahora hablar de las fiestas idólatras y de los tormentos extremos.

\chapter{8}

\par 1 Entonces Judas Macabeo y los que estaban con él fueron a las ciudades en secreto, reunieron a sus parientes y tomaron con ellos a todos los que profesaban la religión judía, y reunieron alrededor de seis mil hombres.
\par 2 Y rogaron al Señor que mirara al pueblo pisoteado por todos; y también compadecerse del templo profanado por hombres impíos;
\par 3 Y que tuviera compasión de la ciudad, muy desfigurada y lista para ser nivelada con la tierra; y oye la sangre que clamaba a él,
\par 4 Y acordaos de la perversa matanza de niños inocentes y de las blasfemias cometidas contra su nombre; y que mostraría su odio contra los malvados.
\par 5 Ahora bien, cuando Macabeo estaba rodeado de su compañía, los paganos no pudieron resistirle, porque la ira del Señor se transformó en misericordia.
\par 6 Por lo tanto, atacó sin darse cuenta, quemó ciudades y pueblos, se apoderó de los lugares más privilegiados y venció y puso en fuga a no pocos de sus enemigos.
\par 7 Pero aprovechó especialmente la noche para tales intentos privados, de modo que el fruto de su santidad se esparció por todas partes.
\par 8 Así que, cuando Filipo vio que este hombre crecía poco a poco y que las cosas le iban cada vez más bien, escribió a Ptolomeo, gobernador de Celosiria y Fenicia, para que prestara más ayuda a los asuntos del rey.
\par 9 Luego, escogiendo a Nicanor, hijo de Patroclo, uno de sus mejores amigos, lo envió con no menos de veinte mil de todas las naciones bajo su mando, para extirpar a toda la generación de los judíos; y con él se unió también el capitán Gorgias, que en materia de guerra tenía gran experiencia.
\par 10 Entonces Nicanor se propuso sacar de los judíos cautivos tanto dinero como para sufragar el tributo de dos mil talentos que el rey debía pagar a los romanos.
\par 11 Por lo tanto, inmediatamente envió a las ciudades de la costa del mar, proclamando una venta de los judíos cautivos, y prometiendo que recibirían ochenta y diez cuerpos por un talento, sin esperar la venganza que iba a seguir sobre él por parte del Todopoderoso. Dios.
\par 12 Cuando Judas fue avisado de la llegada de Nicanor, y éste informó a los que estaban con él que el ejército estaba cerca,
\par 13 Los que tenían miedo y desconfiaban de la justicia de Dios, huyeron y se alejaron.
\par 14 Otros vendieron todo lo que les quedaba, y además rogaron al Señor que los liberara, vendidos por el malvado Nicanor antes de reunirse.
\par 15 Y si no por ellos mismos, al menos por los pactos que había hecho con sus padres, y por causa de su santo y glorioso nombre, por el cual fueron llamados.
\par 16 Entonces Macabeo reunió a sus hombres en número de seis mil y los exhortó a no dejarse llevar por el terror del enemigo, ni a temer a la gran multitud de paganos que venían injustamente contra ellos; pero luchar valientemente,
\par 17 Y para poner ante sus ojos el daño que habían hecho injustamente al lugar santo, y el trato cruel de la ciudad, del cual se burlaron, y también el despojo del gobierno de sus antepasados:
\par 18 Porque ellos, dijo, confían en sus armas y en su audacia; pero nuestra confianza está en el Todopoderoso, que a su entera disposición puede derribar a los que vienen contra nosotros, y también al mundo entero.
\par 19 Además, les contó qué ayudas habían encontrado sus padres y cómo fueron librados cuando, bajo Senaquerib, perecieron ciento ochenta y cinco mil.
\par 20 Y les contó la batalla que tuvieron en Babilonia contra los gálatas, cómo sólo ocho mil en total se unieron a la batalla, con cuatro mil macedonios, y que los macedonios, desconcertados, los ocho mil destruyeron a ciento veinte mil a causa de la ayuda que tuvieron del cielo, y así recibieron un gran botín.
\par 21 Así que, cuando los hubo atrevido con estas palabras y dispuestos a morir por la ley y la patria, dividió su ejército en cuatro partes;
\par 22 Y reunió consigo a sus propios hermanos, jefes de cada grupo, es decir, Simón, José y Jonatán, dando cada uno mil quinientos hombres.
\par 23 También encargó a Eleazar que leyera el libro sagrado; y cuando les hubo dado este lema: La ayuda de Dios; él mismo liderando la primera banda,
\par 24 Y con la ayuda del Todopoderoso mataron a más de nueve mil de sus enemigos, e hirieron y mutilaron a la mayor parte del ejército de Nicanor, y así hicieron huir a todos.
\par 25 Y tomando el dinero que había venido a comprarlos, los persiguieron hasta muy lejos, pero, por falta de tiempo, regresaron.
\par 26 Porque era la víspera del sábado y ya no los perseguían.
\par 27 Entonces, cuando reunieron sus armas y despojaron a sus enemigos, se ocuparon del sábado, rindiendo grandes alabanzas y gracias al Señor, que los había preservado hasta ese día, que fue el comienzo de la misericordia que destilaba sobre ellos.
\par 28 Y después del sábado, cuando dieron parte del botín a los mancos, a las viudas y a los huérfanos, el resto lo repartieron entre ellos y sus siervos.
\par 29 Hecho esto, y habiendo hecho una súplica común, rogaron al Señor misericordioso que se reconciliara con sus siervos para siempre.
\par 30 Además, de los que estaban con Timoteo y Báquides, que luchaban contra ellos, mataron a más de veinte mil, y con gran facilidad conquistaron posiciones altas y fuertes, y se repartieron entre sí muchos más despojos, y dejaron mutilados, huérfanos y viudas, sí, y también los ancianos, iguales en botín que ellos mismos.
\par 31 Y cuando reunieron sus armas, las colocaron todas cuidadosamente en lugares convenientes, y el resto del botín lo llevaron a Jerusalén.
\par 32 También mataron a Filarcas, el malvado que estaba con Timoteo y que había molestado mucho a los judíos.
\par 33 Además, mientras celebraban la fiesta por la victoria en su país, quemaron a Calístenes, que había prendido fuego a las puertas santas, y que había huido a una pequeña casa; y así recibió una recompensa adecuada por su maldad.
\par 34 En cuanto al despiadado Nicanor, que había traído mil mercaderes para comprar a los judíos,
\par 35 Con la ayuda del Señor, fue derrotado por aquellos a quienes menos tenía en cuenta; y despojándose de su ropa gloriosa y despidiendo su compañía, vino como un siervo fugitivo por el centro del país hasta Antioquía, teniendo muy grande deshonra, porque su ejército había sido destruido.
\par 36 Así, el que se encargó de pagar el tributo a los romanos mediante cautivos en Jerusalén, dijo en el exterior que los judíos tenían a Dios para luchar por ellos y que, por tanto, no podían sufrir ningún daño, porque seguían el leyes que él les dio.

\chapter{9}

\par 1 Por aquel tiempo Antíoco salió con deshonra del país de Persia.
\par 2 Porque había entrado en la ciudad llamada Persépolis y se propuso saquear el templo y apoderarse de la ciudad; Entonces la multitud que corrió a defenderse con sus armas los hizo huir; Y sucedió que Antíoco, puesto en fuga por los habitantes, volvió avergonzado.
\par 3 Cuando llegó a Ecbatane, le trajeron noticias de lo que les había sucedido a Nicanor y Timoteo.
\par 4 Luego se hincha de ira. Pensó vengar de los judíos la deshonra que le habían hecho aquellos que lo hicieron huir. Por lo tanto, ordenó a su auriga que condujera sin cesar y que despachara el viaje, siguiendo ahora el juicio de Dios. Porque había hablado con orgullo de esta manera: Iría a Jerusalén y la convertiría en lugar de sepultura común para los judíos.
\par 5 Pero el Señor Todopoderoso, el Dios de Israel, lo hirió con una plaga incurable e invisible: o, tan pronto como hubo pronunciado estas palabras, le sobrevino un dolor de entrañas que no tenía remedio, y dolorosos tormentos en las partes interiores;
\par 6 Y esto con mucha razón, porque había atormentado las entrañas de otros hombres con muchos y extraños tormentos.
\par 7 Sin embargo, no dejó de fanfarronear, sino que, lleno de orgullo, exhaló fuego en su ira contra los judíos y ordenó que se apresuraran el camino; pero aconteció que cayó del carro, llevado violentamente; de modo que al sufrir una dura caída, todos los miembros de su cuerpo sufrieron mucho dolor.
\par 8 Y así, el que un poco antes pensó que podría dominar las olas del mar (tan orgulloso era más allá de la condición de un hombre) y pesar las altas montañas en una balanza, ahora fue arrojado al suelo y llevado en una litera de caballos, mostrando a todos el poder manifiesto de Dios.
\par 9 De modo que los gusanos surgieron del cuerpo de este hombre malvado, y mientras vivía en tristeza y dolor, su carne se desprendió, y la inmundicia de su olor era repugnante para todo su ejército.
\par 10 Y el hombre que pensó un poco antes de poder alcanzar las estrellas del cielo, ningún hombre pudo soportar llevarlo a causa de su intolerable hedor.
\par 11 Aquí, pues, afligido, comenzó a dejar su gran orgullo y a conocerse a sí mismo por el azote de Dios, y su dolor aumentaba a cada momento.
\par 12 Y como él mismo no podía soportar su propio olor, dijo estas palabras: Es conveniente estar sujeto a Dios, y que un hombre que es mortal no se enorgullezca de sí mismo si fuera Dios.
\par 13 Este malvado también hizo un voto al Señor, quien ya no tendría más misericordia de él, diciendo así:
\par 14 Que dejaría en libertad la ciudad santa, adonde se dirigía apresuradamente para nivelarla y convertirla en lugar de sepultura común.
\par 15 Y en cuanto a los judíos, a quienes no había juzgado dignos de ser enterrados, sino de ser expulsados ​​con sus hijos para ser devorados por las aves y las fieras, los igualaría a todos con los ciudadanos de Atenas:
\par 16 Y el templo santo, que antes había saqueado, lo adornaría con regalos maravillosos, y restauraría todos los utensilios sagrados con muchos más, y de sus propios ingresos sufragaría los gastos correspondientes a los sacrificios.
\par 17 Sí, y que también él mismo se haría judío, recorrería todo el mundo habitado y declararía el poder de Dios.
\par 18 Pero a pesar de todo esto sus dolores no cesaban, porque el justo juicio de Dios había caído sobre él; por lo tanto, desesperando de su salud, escribió a los judíos la carta suscrita, que contenía la forma de una súplica, de la siguiente manera:
\par 19 Antíoco, rey y gobernador, desea a sus buenos ciudadanos judíos mucha alegría, salud y prosperidad:
\par 20 Si a vosotros y a vuestros hijos os va bien y todo os satisface, doy muchas gracias a Dios, teniendo mi esperanza en el cielo.
\par 21 En cuanto a mí, era débil; de lo contrario, habría recordado con bondad tu honor y tu buena voluntad cuando regresaste de Persia y, al caer enfermo con una grave enfermedad, pensé que era necesario cuidar de la salud común de todos.
\par 22 No desconfiando de mi salud, sino teniendo gran esperanza de escapar de esta enfermedad.
\par 23 Pero considerando que incluso mi padre, en aquel momento dirigió un ejército a las tierras altas nombró un sucesor,
\par 24 Para que, si algo sucediera contra lo esperado, o si llegara alguna noticia que fuera perjudicial, los habitantes del país, sabiendo a quién había quedado el estado, no se turbaran.
\par 25 Considerando también que los príncipes que son fronterizos y vecinos de mi reino esperan oportunidades y esperan lo que será el acontecimiento. He nombrado rey a mi hijo Antíoco, a quien muchas veces encomendé y encomendé a muchos de vosotros cuando subí a las provincias altas; a quien le he escrito lo siguiente:
\par 26 Por eso os ruego y os pido que recordéis los beneficios que os he hecho en general y en particular, y que todo hombre siga siendo fiel a mí y a mi hijo.
\par 27 Porque estoy seguro de que el que comprenda mi mente cederá favorablemente y con gracia a tus deseos.
\par 28 Así, el asesino y blasfemo, habiendo sufrido mucho mientras suplicaba a otros hombres, murió miserablemente en un país extraño, en las montañas.
\par 29 Y Felipe, que se había criado con él, se llevó su cuerpo, y éste, temiendo también al hijo de Antíoco, fue a Egipto a ver a Ptolomeo Filometor.

\chapter{10}

\par 1 Macabeo y su compañía, guiados por el Señor, recuperaron el templo y la ciudad.
\par 2 Pero los altares que los paganos habían construido en la calle, y también las capillas, los derribaron.
\par 3 Y después de limpiar el templo, hicieron otro altar, y al golpear las piedras, sacaron de ellas fuego, y al cabo de dos años ofrecieron sacrificios, pusieron incienso, lámparas y panes de la proposición.
\par 4 Cuando lo hicieron, cayeron de bruces y rogaron al Señor que no volvieran a sufrir tales problemas; pero si pecaban más contra él, él mismo los castigaría con misericordia y no serían entregados a las naciones blasfemas y bárbaras.
\par 5 Y el mismo día que los extraños profanaron el templo, el mismo día fue nuevamente limpiado, es decir, el día veinticinco del mismo mes, que es Casleu.
\par 6 Y celebraron los ocho días con alegría, como en la fiesta de las Tiendas, recordando que no mucho antes habían celebrado la fiesta de las Tiendas, cuando vagaban por las montañas y las cuevas como las bestias.
\par 7 Por lo tanto, desnudaron ramas, hermosos ramos y también palmeras, y cantaron salmos al que les había dado buen éxito en la limpieza de su lugar.
\par 8 También dispusieron por estatuto y decreto común que cada año se guardaran aquellos días para toda la nación de los judíos.
\par 9 Y así fue el fin de Antíoco, llamado Epífanes.
\par 10 Ahora contaremos los hechos de Antíoco Eupátor, que era hijo de este malvado, resumiendo brevemente las calamidades de las guerras.
\par 11 Así que, cuando llegó al trono, puso a un tal Lisias a cargo de los asuntos de su reino y lo nombró gobernador en jefe de Celosiria y Fenicia.
\par 12 Porque Ptolomeo, llamado Macron, prefirió hacer justicia a los judíos por el mal que les habían hecho y se esforzó por continuar la paz con ellos.
\par 13 Entonces, acusado ante Eupátor de los amigos del rey y llamado traidor por cada palabra que Filometor le había encomendado por haber salido de Chipre, se fue a Antíoco Epífanes, y viendo que no se encontraba en ningún lugar honorable, fue tan desanimado, que se envenenó y murió.
\par 14 Pero cuando Gorgias era gobernador de las fortalezas, contrató soldados y mantuvo continuamente la guerra contra los judíos.
\par 15 Y todos los idumeos, habiendo tomado en sus manos las mejores fortalezas, mantuvieron ocupados a los judíos y, recibiendo a los desterrados de Jerusalén, se dispusieron a alimentar la guerra.
\par 16 Entonces los que estaban con Macabeo rogaron a Dios que les ayudara; y así corrieron con violencia contra las fortalezas de los idumeos,
\par 17 Y atacándolos con fuerza, conquistaron las fortalezas, rechazaron a todos los que luchaban en la muralla y mataron a todos los que cayeron en sus manos, y mataron no menos de veinte mil.
\par 18 Y como algunos, que eran no menos de nueve mil, huyeron juntos a dos castillos muy fuertes, teniendo todo lo necesario para resistir el asedio,
\par 19 Macabeo dejó a Simón y a José, y también a Zaqueo y a los que estaban con él, que eran suficientes para sitiarlos, y se dirigió a aquellos lugares que más necesitaban su ayuda.
\par 20 Los que estaban con Simón, llevados por la codicia, fueron persuadidos por algunos de los que estaban en el castillo para pedir dinero, y tomaron setenta mil dracmas, y dejaron escapar a algunos de ellos.
\par 21 Pero cuando Macabeo supo lo sucedido, convocó a los gobernadores del pueblo y los acusó de haber vendido a sus hermanos por dinero y haber dejado libres a sus enemigos para luchar contra ellos.
\par 22 Entonces mató a los que consideraban traidores y al instante tomó los dos castillos.
\par 23 Y teniendo buen éxito con sus armas en todo lo que tomó, mató en las dos bodegas a más de veinte mil.
\par 24 Ahora bien, Timoteo, a quien los judíos habían vencido antes, cuando había reunido una gran multitud de fuerzas extranjeras y no pocos caballos de Asia, llegó como si quisiera tomar a los judíos por la fuerza.
\par 25 Pero cuando él se acercó, los que estaban con Macabeo se volvieron para orar a Dios, se rociaron la cabeza con tierra y se ciñeron los lomos con cilicio.
\par 26 Y postrándose al pie del altar, le suplicó que fuera misericordioso con ellos, y que fuera enemigo de sus enemigos y adversario de sus adversarios, como lo declara la ley.
\par 27 Después de la oración, tomaron sus armas y se alejaron de la ciudad; y cuando se acercaron a sus enemigos, se quedaron solos.
\par 28 Cuando ya había salido el sol, unieron a ambos; una parte tiene junto con su virtud su refugio también en el Señor como garantía de su éxito y victoria; la otra parte hace de su ira el líder de su batalla.
\par 29 Pero cuando la batalla se intensificaba, aparecieron desde el cielo a los enemigos cinco hombres hermosos a caballo, con frenos de oro, y dos de ellos guiaban a los judíos.
\par 30 Y tomó a Macabeo entre ellos, lo cubrió con armas por todos lados y lo mantuvo a salvo, pero disparó flechas y relámpagos contra los enemigos, de modo que, confundidos por la ceguera y llenos de angustia, fueron asesinados.
\par 31 Y murieron veinte mil quinientos hombres de a pie y seiscientos jinetes.
\par 32 En cuanto al propio Timoteo, huyó a una fortaleza muy fuerte llamada Gawra, donde gobernaba Quereas.
\par 33 Pero los que estaban con Macabeo sitiaron valientemente la fortaleza durante cuatro días.
\par 34 Y los que estaban dentro, confiando en la fortaleza del lugar, blasfemaron mucho y pronunciaron malas palabras.
\par 35 Sin embargo, al quinto día, veinte jóvenes de la compañía de Macabeo, enardecidos de ira por las blasfemias, atacaron varonilmente la muralla y con gran valentía mataron a todos los que encontraron.
\par 36 Otros, subiendo tras ellos, mientras estaban ocupados con los que estaban dentro, quemaron las torres y encendieron hogueras quemaron vivos a los blasfemadores; y otros rompieron las puertas y, habiendo recibido al resto del ejército, tomaron la ciudad.
\par 37 Y mató a Timoteo, que estaba escondido en una fosa, y a su hermano Quereas, y a Apolofanes.
\par 38 Una vez hecho esto, alabaron con salmos y acciones de gracias al Señor, que había hecho tantas cosas por Israel y les había dado la victoria.

\chapter{11}

\par 1 Poco después, el protector y primo del rey Lisias, que también administraba los asuntos, se disgustó mucho por lo que se había hecho.
\par 2 Y cuando reunió unos ochenta mil con toda la gente de a caballo, vino contra los judíos, pensando en hacer de la ciudad una habitación de los gentiles.
\par 3 y para hacer ganancia del templo, como de las demás capillas de los paganos, y poner en venta el sumo sacerdocio cada año:
\par 4 Sin considerar en absoluto el poder de Dios, sino engreídos con sus diez mil hombres de a pie, sus miles de jinetes y sus ochenta elefantes.
\par 5 Llegó a Judea y se acercó a Betsur, que era una ciudad fuerte, pero que estaba a unos cinco estadios de Jerusalén, y la sitió duramente.
\par 6 Cuando los que estaban con Macabeo oyeron que él había sitiado las fortalezas, ellos y todo el pueblo, con lamentos y lágrimas, rogaron al Señor que enviara un ángel bueno para liberar a Israel.
\par 7 Entonces Macabeo fue el primero en tomar las armas, exhortando al otro a que se arriesgaran junto con él para ayudar a sus hermanos; así que salieron juntos con buena voluntad.
\par 8 Y mientras estaban en Jerusalén, apareció ante ellos uno vestido de blanco, a caballo, sacudiendo su armadura de oro.
\par 9 Entonces todos juntos alabaron al Dios misericordioso y se animaron, de modo que estaban dispuestos no sólo a luchar contra los hombres, sino también contra las bestias más crueles y a atravesar muros de hierro.
\par 10 Avanzaron así con sus armas, teniendo un ayudante del cielo, porque el Señor fue misericordioso con ellos.
\par 11 Y atacando a sus enemigos como leones, mataron a once mil hombres de a pie y a mil seiscientos jinetes, y pusieron en fuga a todos los demás.
\par 12 Muchos de ellos, también heridos, escaparon desnudos; y el propio Lisias huyó avergonzado y así escapó.
\par 13 El cual, siendo hombre inteligente, echando en cara la pérdida que había sufrido, y considerando que los hebreos no podían ser vencidos, porque el Dios Todopoderoso los ayudaba, les envió:
\par 14 Y los persuadió para que aceptaran todas las condiciones razonables y les prometió que convencería al rey de que debía ser su amigo.
\par 15 Entonces Macabeo accedió a todo lo que Lisias deseaba, cuidando del bien común; y todo lo que Macabeo escribió a Lisias acerca de los judíos, el rey se lo concedió.
\par 16 Porque Lisias había escrito a los judíos cartas en las que decía: Lisias envía saludos al pueblo de los judíos:
\par 17 Juan y Absalón, enviados por vosotros, me entregaron la petición suscrita y me pidieron que se cumpliera su contenido.
\par 18 Por lo tanto, todo lo que era necesario informar al rey, lo he declarado, y él ha concedido todo lo que era posible.
\par 19 Y si entonces os mantenéis leales al Estado, en lo sucesivo también me esforzaré por ser un medio para vuestro bien.
\par 20 Pero en cuanto a los particulares, he ordenado a éstos y a los demás que vinieron de mí, que comulguen con vosotros.
\par 21 Que os vaya bien. El año ciento ocho y cuarenta, el día veinticuatro del mes de Dioscorinto.
\par 22 La carta del rey contenía estas palabras: El rey Antíoco envía un saludo a su hermano Lisias:
\par 23 Dado que nuestro padre es trasladado a los dioses, nuestra voluntad es que los que están en nuestro reino vivan tranquilamente, para que cada uno pueda ocuparse de sus propios asuntos.
\par 24 También entendemos que los judíos no quisieron que nuestro padre fuera llevado a la costumbre de los gentiles, sino que prefirieron mantener su propia manera de vivir, por lo que nos exigen que suframos viviendo según sus propias leyes.
\par 25 Por eso pensamos que esta nación descansará y hemos decidido restaurarles su templo para que vivan según las costumbres de sus antepasados.
\par 26 Por tanto, harás bien en enviarles paz y concederles paz, para que, cuando estén seguros de nuestra voluntad, se sientan cómodos y se ocupen siempre alegremente de sus propios asuntos.
\par 27 Y la carta del rey a la nación de los judíos era la siguiente: El rey Antíoco envía saludos al consejo y al resto de los judíos:
\par 28 Si a vosotros os va bien, tendremos nuestro deseo; nosotros también gozamos de buena salud.
\par 29 Menelanos nos declaró que vuestro deseo era volver a casa y dedicaros a vuestros propios asuntos:
\par 30 Por tanto, los que partan tendrán salvoconducto hasta el día treinta de Jántico con seguridad.
\par 31 Y los judíos usarán sus propios alimentos y leyes, como antes; y ninguno de ellos será molestado de ninguna manera por cosas hechas por ignorancia.
\par 32 También he enviado a Menelanos para que os consuele.
\par 33 Que te vaya bien. En el año ciento cuarenta y ocho, a los quince días del mes Xántico.
\par 34 Los romanos también les enviaron una carta que contenía estas palabras: Quinto Memio y Tito Manlio, embajadores de los romanos, saludan al pueblo judío.
\par 35 Todo lo que Lisias, primo del rey, ha concedido, también nosotros nos complacemos.
\par 36 Pero en cuanto a las cosas que él juzgó que deben ser remitidas al rey, después de haberlo informado, envía uno inmediatamente, para que podamos declarar lo que más te convenga, porque ahora nos vamos a Antioquía.
\par 37 Envía, pues, pronto algunos, para que sepamos lo que piensas.
\par 38 Adiós. Este año ciento ocho y cuarenta, el día quince del mes Xántico.

\chapter{12}

\par 1 Cuando se hicieron estos pactos, Lisias fue al rey y los judíos se ocuparon de sus tareas agrícolas.
\par 2 Pero de los gobernadores de varios lugares, Timoteo y Apolonio, hijo de Geneo, también Jerónimo y Demofón, y junto a ellos Nicanor, gobernador de Chipre, no les permitieron estar tranquilos y vivir en paz.
\par 3 Los hombres de Jope también cometieron una acción tan impía: rogaron a los judíos que vivían entre ellos que subieran con sus esposas e hijos a las barcas que habían preparado, como si no quisieran hacerles ningún daño.
\par 4 Los cuales lo aceptaron según el decreto común de la ciudad, deseando vivir en paz y sin sospechar nada; pero cuando se internaron en lo profundo, ahogaron no menos de doscientos de ellos.
\par 5 Cuando Judas se enteró de esta crueldad cometida contra sus compatriotas, ordenó a los que estaban con él que los prepararan.
\par 6 E invocando a Dios, el juez justo, se lanzó contra los asesinos de sus hermanos, y de noche quemó el puerto, prendió fuego a las barcas y mató a los que huían allí.
\par 7 Y cuando la ciudad fue cerrada, él retrocedió, como si fuera a regresar para exterminar a todos los de la ciudad de Jope.
\par 8 Pero cuando oyó que los jamnitas querían hacer lo mismo con los judíos que habitaban entre ellos,
\par 9 También de noche atacó a los jamnitas y prendió fuego al puerto y a la marina, de modo que la luz del fuego se vio en Jerusalén a doscientos cuarenta estadios.
\par 10 Cuando habían recorrido nueve estadios de allí hacia Timoteo, se lanzaron contra él no menos de cinco mil hombres árabes a pie y quinientos jinetes.
\par 11 Entonces hubo una batalla muy encarnizada; pero el lado de Judas con la ayuda de Dios consiguió la victoria; De modo que los nómadas de Arabia, vencidos, rogaron a Judas la paz, prometiéndole darle ganado y complacerlo de otra manera.
\par 12 Entonces Judas, pensando que en muchas cosas les sería útil, les concedió la paz; después de lo cual se dieron la mano y se fueron a sus tiendas.
\par 13 También se dispuso a construir un puente hacia una ciudad fuerte, cercada con murallas y habitada por gente de diferentes países; y su nombre era Caspis.
\par 14 Pero los que estaban dentro de ella confiaban tanto en la solidez de las murallas y en el suministro de víveres, que se comportaron con rudeza con los que estaban con Judas, injuriando y blasfemando, y pronunciando palabras que no debían pronunciarse.
\par 15 Por lo tanto, Judas y su compañía, invocando al gran Señor del mundo, que sin arietes ni máquinas de guerra derribó Jericó en tiempos de Josué, lanzaron un feroz asalto contra las murallas,
\par 16 Y tomaron la ciudad por la voluntad de Dios, e hicieron matanzas indecibles, de modo que cerca de allí se vio un lago de dos estadios de ancho, lleno de sangre.
\par 17 Partieron de allí setecientos cincuenta estadios y llegaron a Characa, a los judíos llamados Tubieni.
\par 18 Pero a Timoteo no lo encontraron en el lugar; antes de que hubiera enviado nada, partió de allí, dejando una guarnición muy fuerte en cierta fortaleza.
\par 19 Pero Dositeo y Sosípatro, que eran capitanes de Macabeo, salieron y mataron a los que Timoteo había dejado en la fortaleza, más de diez mil hombres.
\par 20 Macabeo organizó su ejército en grupos, los puso al frente de ellos y fue contra Timoteo, que tenía cerca de él ciento veinte mil hombres de a pie y dos mil quinientos de a caballo.
\par 21 Cuando Timoteo se enteró de la llegada de Judas, envió a las mujeres, a los niños y al resto del equipaje a una fortaleza llamada Carnión, porque la ciudad era difícil de asediar y de difícil acceso a causa de la estrechez de todos los lugares.
\par 22 Pero cuando Judas, su primer grupo, apareció a la vista, los enemigos, invadidos de miedo y terror por la aparición de Aquel que todo lo ve, huyeron, uno corriendo por un lado y otro por otro, de modo que se encontraban a menudo heridos por sus propios hombres y heridos con las puntas de sus propias espadas.
\par 23 También Judas los persiguió con gran diligencia y mató a aquellos malvados, de los cuales mató a unos treinta mil hombres.
\par 24 Además, el propio Timoteo cayó en manos de Dositeo y de Sosípatro, a quienes rogó con gran astucia que le dejaran ir con vida, porque tenía muchos padres judíos y hermanos de algunos de ellos que, si lo mataron, no debe ser tenido en cuenta.
\par 25 Entonces, cuando les aseguró con muchas palabras que los restauraría sin daño, según el acuerdo, lo dejaron ir para salvar a sus hermanos.
\par 26 Entonces Macabeo se dirigió a Carnión y al templo de Atargatis, y allí mató a veinticinco mil personas.
\par 27 Y después de haberlos puesto en fuga y destruidos, Judas dirigió el ejército hacia Efrón, una ciudad fuerte en la que moraban Lisias y una gran multitud de diversas naciones, y los jóvenes fuertes guardaban las murallas y las defendían poderosamente. : donde también había gran provisión de máquinas y dardos.
\par 28 Pero Judas y su compañía invocaron al Dios Todopoderoso, que con su poder quebranta las fuerzas de sus enemigos, tomaron la ciudad y mataron a veinticinco mil de los que estaban dentro.
\par 29 De allí partieron hacia Escitópolis, que está a seiscientos estadios de Jerusalén,
\par 30 Pero cuando los judíos que habitaban allí dieron testimonio de que los escitopolitas los trataban con amor y les rogaban bondadosamente en el momento de su adversidad,
\par 31 Les dieron gracias, pidiéndoles que siguieran siendo amigables con ellos, y así llegaron a Jerusalén, acercándose la fiesta de las semanas.
\par 32 Y después de la fiesta llamada Pentecostés, salieron contra Gorgias, gobernador de Idumea,
\par 33 Los cuales salieron con tres mil hombres de a pie y cuatrocientos de a caballo.
\par 34 Y sucedió que en la pelea entre ellos murieron algunos de los judíos.
\par 35 En ese momento, Dositeo, uno de los hombres de Bacenor, que iba a caballo y era un hombre fuerte, estaba todavía sobre Gorgias y, agarrándolo de su abrigo, lo atrajo por la fuerza; y cuando quería capturar vivo al maldito, un jinete de Tracia se le acercó y le cortó el hombro, de modo que Gorgias huyó hacia Marisa.
\par 36 Cuando los que estaban con Gorgias habían luchado mucho y estaban cansados, Judas invocó al Señor para que se mostrara como su ayudante y líder de la batalla.
\par 37 Y diciendo esto en su propia lengua, cantó salmos en alta voz y, arremetiendo desprevenido contra los hombres de Gorgias, los hizo huir.
\par 38 Entonces Judas reunió a su ejército y llegó a la ciudad de Odollam. Cuando llegó el séptimo día, se purificaron como era costumbre y guardaron el sábado en el mismo lugar.
\par 39 Al día siguiente, como era costumbre, Judas y su compañía fueron a recoger los cuerpos de los muertos y a sepultarlos con sus parientes en las tumbas de sus padres.
\par 40 Bajo las túnicas de todos los asesinados se encontraron cosas consagradas a los ídolos de los jamnitas, lo cual la ley prohíbe a los judíos. Entonces todos vieron que ésta era la causa por la que habían sido asesinados.
\par 41 Todos, pues, alababan al Señor, Juez justo, que había abierto lo que estaba escondido,
\par 42 Se pusieron a orar y le rogaron que el pecado cometido fuera completamente borrado de la memoria. Además, aquel noble Judas exhortaba al pueblo a guardarse del pecado, por cuanto veían ante sus ojos las cosas que acontecían por los pecados de los que eran asesinados.
\par 43 Y habiendo reunido entre toda la congregación la suma de dos mil dracmas de plata, los envió a Jerusalén para ofrecer una ofrenda por el pecado, obrando muy bien y honradamente, teniendo presente la resurrección.
\par 44 Porque si no hubiera esperado que los muertos resucitarían, habría sido superfluo y vano orar por los muertos.
\par 45 Y también el hecho de que percibiera que había un gran favor reservado para aquellos que morían piadosamente, era un pensamiento santo y bueno. Después de lo cual hizo la reconciliación por los muertos, para que fueran librados del pecado.

\chapter{13}

\par 1 En el año ciento cuarenta y nueve, le avisaron a Judas que Antíoco Eupátor llegaba a Judea con gran poder,
\par 2 Y con él Lisias, su protector y gobernante de sus asuntos, teniendo cada uno de ellos un ejército griego de infantes, ciento diez mil, cinco mil trescientos jinetes, veintidós elefantes y trescientos carros armados con ganchos.
\par 3 También se unió a ellos Menelano, y con gran disimulo animó a Antíoco, no por la salvaguardia del país, sino porque creía haber sido nombrado gobernador.
\par 4 Pero el Rey de reyes incitó a Antíoco contra este malvado, y Lisias le informó al rey que este hombre era el causante de todos los males, por lo que el rey ordenó llevarlo a Berea y darle muerte, como es la costumbre en ese lugar.
\par 5 Había en aquel lugar una torre de cincuenta codos de altura, llena de cenizas, y tenía un instrumento redondo que colgaba por todos lados dentro de las cenizas.
\par 6 Y a cualquiera que fuera condenado por sacrilegio o por cualquier otro delito grave, allí todos lo arrojaban a muerte.
\par 7 Tal muerte le ocurrió al impío, sin tener siquiera sepultura en la tierra; y eso muy justamente:
\par 8 Porque habiendo cometido muchos pecados en relación con el altar, cuyo fuego y cenizas eran santos, recibió la muerte en cenizas.
\par 9 Ahora bien, el rey vino con ánimo bárbaro y altivo para hacer a los judíos mucho peor que lo que se había hecho en tiempos de su padre.
\par 10 Cuando Judas se dio cuenta de esto, ordenó a la multitud que invocara al Señor de noche y de día, para que, si alguna vez en otro momento, él también los ayudara a ellos, estando a punto de ser puestos de su ley, de su país, y del santo templo:
\par 11 Y que no permitiría que el pueblo, que hasta entonces había descansado un poco, fuera sometido a las naciones blasfemas.
\par 12 Cuando todos juntos hicieron esto y rogaron al Señor misericordioso con llanto y ayuno y tendido en el suelo durante tres días, Judas los exhortó y les ordenó que estuvieran preparados.
\par 13 Y Judas, estando aparte con los ancianos, decidió, antes de que el ejército del rey entrara en Judea y tomara la ciudad, salir y probar el asunto en la batalla con la ayuda del Señor.
\par 14 Así que, cuando lo hubo confiado todo al Creador del mundo y exhortó a sus soldados a luchar valientemente, incluso hasta la muerte, por las leyes, el templo, la ciudad, el país y la república, acampó junto a Modin:
\par 15 Y habiendo dado la consigna a los que estaban a su alrededor: La victoria es de Dios; Con los jóvenes más valientes y escogidos entró de noche en la tienda del rey y mató en el campamento a unos cuatro mil hombres, y al jefe de los elefantes, con todo lo que llevaba encima.
\par 16 Finalmente llenaron el campamento de miedo y alboroto y se marcharon con éxito.
\par 17 Esto lo hizo al amanecer, porque la protección del Señor le ayudó.
\par 18 Cuando el rey hubo probado la virilidad de los judíos, se dispuso a tomar el control por medio de la política,
\par 19 Y se dirigió hacia Betsur, que era un bastión de los judíos, pero huyó, fracasó y perdió a sus hombres.
\par 20 Porque Judas había entregado a los que estaban en ella lo necesario.
\par 21 Pero Rodoco, que estaba en el ejército de los judíos, reveló los secretos a los enemigos; Por eso lo buscaron, y cuando lo capturaron, lo metieron en la cárcel.
\par 22 El rey trató con ellos por segunda vez en Betsum, les dio la mano, les tomó la de ellos, se fue, peleó con Judas y fue vencido;
\par 23 Oí que Felipe, que había quedado a cargo de los asuntos en Antioquía, se encorvó desesperadamente, se confundió, injurió a los judíos, se sometió y juró igualdad de condiciones para todos, estuvo de acuerdo con ellos, ofreció sacrificios, honró el templo y trató amablemente con el lugar,
\par 24 Y agradó a Macabeo y lo nombró gobernador principal desde Tolemaida hasta los gerrenios;
\par 25 Llegaron a Tolemaida; el pueblo estaba afligido por los pactos; porque asaltaron, porque querían anular sus pactos:
\par 26 Lisias subió al tribunal, habló todo lo que pudo en defensa de la causa, los persuadió, los tranquilizó, los conmovió y regresó a Antioquía. Así fue la venida y la partida del rey.

\chapter{14}

\par 1 Después de tres años, Judas supo que Demetrio, hijo de Seleuco, había entrado por el puerto de Trípolis con gran poder y armada,
\par 2 Había tomado el país y había matado a Antíoco y a Lisias, su protector.
\par 3 Un tal Alcimo, que había sido sumo sacerdote y se había contaminado voluntariamente cuando se mezclaban con los gentiles, viendo que de ningún modo podía salvarse ni tener más acceso al altar santo,
\par 4 El año ciento cincuenta y uno vino al rey Demetrio y le presentó una corona de oro, una palma y las ramas que se usaban solemnemente en el templo; y aquel día guardó silencio.
\par 5 Sin embargo, cuando tuvo la oportunidad de continuar con su necia empresa, y cuando Demetrio lo llamó a consejo y le preguntó cómo se encontraban los judíos y qué se proponían, respondió:
\par 6 Los judíos a los que llamó asideos, cuyo capitán es Judas Macabeo, fomentan la guerra y son sediciosos, y no dejan que los demás estén en paz.
\par 7 Por eso yo, privado del honor de mis antepasados, es decir, del sumo sacerdocio, he venido ahora acá.
\par 8 En primer lugar, por el cuidado sincero que tengo de las cosas del rey; y en segundo lugar, incluso con eso pretendo el bien de mis propios compatriotas: porque toda nuestra nación se encuentra en una miseria no pequeña debido al trato imprudente de ellos antes mencionado.
\par 9 Por lo tanto, oh rey, sabiendo todas estas cosas, ten cuidado con el país y con nuestra nación, que está oprimida por todas partes, según la clemencia que muestras fácilmente a todos.
\par 10 Mientras viva Judas, no es posible que el Estado esté tranquilo.
\par 11 Apenas se habló de él, otros amigos del rey, enfrentándose maliciosamente contra Judas, incendiaron aún más a Demetrio.
\par 12 Y llamando luego a Nicanor, que había sido dueño de los elefantes, y nombrándolo gobernador de Judea, lo envió,
\par 13 Le ordenó que matara a Judas, dispersara a los que estaban con él y nombrara a Alcimo sumo sacerdote del gran templo.
\par 14 Entonces los paganos que habían huido de Judea de Judea, llegaron en rebaños a Nicanor, pensando que el daño y las calamidades de los judíos serían su bienestar.
\par 15 Cuando los judíos oyeron la llegada de Nicanor y que los paganos se habían levantado contra ellos, echaron tierra sobre sus cabezas y rogaron al que había establecido a su pueblo para siempre y que siempre ayuda a su porción con la manifestación de su presencia.
\par 16 Entonces, por orden del capitán, partieron inmediatamente de allí y se acercaron a ellos en la ciudad de Dessau.
\par 17 Simón, el hermano de Judas, había entrado en batalla con Nicanor, pero quedó un poco desconcertado por el repentino silencio de sus enemigos.
\par 18 Sin embargo, Nicanor, oyendo la virilidad de los que estaban con Judas y el valor que tenían para luchar por su patria, no se atrevió a probar el asunto con la espada.
\par 19 Por lo que envió a Posidonio, a Teodoto y a Matatías a hacer la paz.
\par 20 Después de haber deliberado largamente sobre esto, y el capitán lo había informado a la multitud, y parecía que todos estaban de acuerdo, aceptaron los pactos.
\par 21 Y fijaron un día para reunirse aparte; y cuando llegó el día y se dispusieron taburetes para cada uno de ellos,
\par 22 Ludas colocó hombres armados en lugares convenientes, para que los enemigos no cometieran alguna traición de repente, y concertaron una conferencia pacífica.
\par 23 Nicanor se quedó en Jerusalén y no hizo ningún daño, sino que despidió al pueblo que acudía a él.
\par 24 Y no quiso perder de vista a Judas, porque lo ama de todo corazón.
\par 25 También le rogó que tomara esposa y engendrara hijos; así que se casó, estuvo tranquilo y participó de esta vida.
\par 26 Pero Alcimo, viendo el amor que había entre ellos y considerando los pactos que habían hecho, vino a Demetrio y le dijo que Nicanor no estaba bien afectado por el estado; por eso había ordenado a Judas, un traidor a su reino, como sucesor del rey.
\par 27 Entonces el rey, enojado y irritado por las acusaciones del hombre más malvado, escribió a Nicanor, dándole a entender que estaba muy disgustado con los pactos, y ordenándole que enviara a Macabeo prisionero a toda prisa a Antioquía.
\par 28 Cuando esto llegó a oídos de Nicanor, se sintió muy confundido en sí mismo, y le pesó mucho tener que anular los acuerdos convenidos, sin que aquel hombre tuviera culpa alguna.
\par 29 Pero como no había ningún trato contra el rey, esperó el momento oportuno para lograrlo mediante una política.
\par 30 Sin embargo, cuando Macabeo vio que Nicanor empezaba a ser grosero con él y que le suplicaba con más rudeza de lo habitual, comprendiendo que tal conducta agria no era buena, reunió a no pocos de sus hombres y se retiró de Nicanor.
\par 31 Pero el otro, sabiendo que la política de Judas se lo impedía notablemente, entró en el gran y santo templo y ordenó a los sacerdotes que ofrecían sus habituales sacrificios que le entregaran al hombre.
\par 32 Y cuando juraron que no podían saber dónde estaba el hombre que buscaba,
\par 33 Extendió su mano derecha hacia el templo e hizo este juramento: Si no me liberáis a Judas preso, pondré este templo de Dios a ras de suelo y derribaré el altar y erigir un templo notable a Baco.
\par 34 Después de estas palabras se fue. Entonces los sacerdotes alzaron sus manos hacia el cielo y rogaron al que siempre fue defensor de su nación, diciendo de esta manera;
\par 35 Tú, Señor de todas las cosas, que de nada tienes necesidad, te agradó que el templo de tu habitación estuviera entre nosotros:
\par 36 Por tanto, ahora, oh Santo Señor de toda santidad, mantén siempre incontaminada esta casa que recientemente fue limpiada, y tapa toda boca injusta.
\par 37 Entonces fue acusado ante Nicanor un tal Razis, uno de los ancianos de Jerusalén, amante de sus compatriotas y hombre de muy buena reputación, quien por su bondad era llamado padre de los judíos.
\par 38 Porque en tiempos pasados, cuando no se mezclaban con los gentiles, él había sido acusado de judaísmo y arriesgaba audazmente su cuerpo y su vida con toda vehemencia por la religión de los judíos.
\par 39 Entonces Nicanor, queriendo declarar el odio que sentía hacia los judíos, envió más de quinientos hombres de guerra para capturarlo:
\par 40 Porque pensaba que al llevarlo consigo haría mucho daño a los judíos.
\par 41 Cuando la multitud quería tomar la torre, irrumpir violentamente en la puerta exterior y ordenar que trajeran fuego para quemarla, él, dispuesto a ser capturado por todos lados, cayó sobre su espada;
\par 42 Prefiriendo morir virilmente antes que caer en manos de los malvados y ser abusado de otra manera que lo que correspondía a su noble nacimiento:
\par 43 Pero, por la prisa, falló su golpe y la multitud se precipitó también hacia las puertas, él corrió valientemente hacia la pared y se arrojó valientemente entre los más gruesos.
\par 44 Pero ellos rápidamente retrocedieron, y habiendo hecho un hueco, él cayó en medio del vacío.
\par 45 Sin embargo, cuando todavía había aliento dentro de él, inflamado de ira, se levantó; y aunque su sangre brotaba como chorros de agua, y sus heridas eran graves, corrió en medio de la multitud; y de pie sobre una roca escarpada,
\par 46 Cuando ya se le había acabado la sangre, le sacó las entrañas y, tomándolas con ambas manos, las arrojó sobre la multitud e invocó al Señor de la vida y del espíritu para que se las devolviera, entonces falleció.

\chapter{15}

\par 1 Pero Nicanor, al enterarse de que Judas y su compañía estaban en las fortalezas alrededor de Samaria, decidió atacarlos sin ningún peligro en el día del sábado.
\par 2 Sin embargo, los judíos que se vieron obligados a ir con él dijeron: ¡No destruyáis de forma tan cruel y bárbara, sino honrad ese día que el que ve todas las cosas ha honrado con santidad más que todos los demás días!
\par 3 Entonces el más despiadado preguntó si había un Poderoso en el cielo que hubiera ordenado que se guardara el día de reposo.
\par 4 Y cuando dijeron: Hay en el cielo un Señor vivo y poderoso, que ordenó que se guardara el séptimo día,
\par 5 Entonces el otro dijo: Yo también soy poderoso en la tierra y mando tomar armas y hacer los negocios del rey. Sin embargo, logró que no se hiciera su mala voluntad.
\par 6 Entonces Nicanor, lleno de orgullo y altivez, decidió erigir un monumento público de su victoria sobre Judas y los que estaban con él.
\par 7 Pero Macabeo siempre estuvo seguro de que el Señor le ayudaría:
\par 8 Por lo tanto, exhortó a su pueblo a no temer la llegada de los paganos contra ellos, sino a recordar la ayuda que en tiempos anteriores habían recibido del cielo, y a esperar ahora la victoria y la ayuda que les llegaría del cielo. Todopoderoso.
\par 9 Y así, animándolos con la ley y los profetas, y recordándoles las batallas que habían ganado antes, los animó más.
\par 10 Y cuando los hubo despertado, les encargó, mostrándoles con ello toda la falsedad de los paganos y el incumplimiento de los juramentos.
\par 11 Así armó a cada uno de ellos, no tanto con defensas de escudos y lanzas, sino con palabras consoladoras y buenas; y además les contó un sueño digno de ser creído, como si realmente hubiera sido así, lo cual los alegró no poco.
\par 12 Y esta fue su visión: que Onías, que había sido sumo sacerdote, hombre virtuoso y bueno, de trato respetuoso, de trato amable, también bien hablado, y ejercitado desde niño en todas las virtudes, sosteniendo sus manos oraron por todo el cuerpo de los judíos.
\par 13 Hecho esto, de la misma manera apareció un hombre canoso, muy glorioso, de una majestad admirable y excelsa.
\par 14 Entonces Onías respondió y dijo: Este es un amante de los hermanos, que ora mucho por el pueblo y por la ciudad santa, es decir, Jeremías, el profeta de Dios.
\par 15 Entonces Jeremías, extendiendo su mano derecha, le dio a Judas una espada de oro, y al entregársela habló así:
\par 16 Toma esta espada santa, regalo de Dios, con la que herirás a los adversarios.
\par 17 Así, consolados por las palabras de Judas, que eran muy buenas y capaces de infundirles valor y animar el corazón de los jóvenes, decidieron no montar el campamento, sino atacarlos valientemente. , y valientemente tratar el asunto mediante conflicto, porque la ciudad y el santuario y el templo estaban en peligro.
\par 18 Porque el cuidado que tenían de sus mujeres, de sus hijos, de sus hermanos y de su familia era el que menos les importaba, pero el mayor y principal temor era el del santo templo.
\par 19 Tampoco los que estaban en la ciudad se preocuparon en lo más mínimo, preocupados por el conflicto en el exterior.
\par 20 Y cuando todos miraban lo que iba a ser la prueba, y los enemigos ya estaban cerca, y el ejército estaba dispuesto, y las bestias colocadas convenientemente, y los jinetes dispuestos en alas,
\par 21 Macabeo, al ver la llegada de la multitud, los diversos preparativos de las armas y la ferocidad de las bestias, extendió sus manos hacia el cielo e invocó al Señor que hace maravillas, sabiendo que la victoria no se logra con las armas, sino con las armas. Incluso si le parece bien, se lo da a los que son dignos:
\par 22 Por eso en su oración dijo de esta manera: Oh Señor, enviaste tu ángel en tiempos de Ezequías, rey de Judea, y mataste en el ejército de Senaquerib a ciento ochenta y cinco mil.
\par 23 Por tanto, ahora también, oh Señor del cielo, envía delante de nosotros un ángel bueno para que les tema y espante;
\par 24 Y por el poder de tu brazo, sean aterrados los que vienen contra tu pueblo santo para blasfemar. Y terminó así.
\par 25 Entonces Nicanor y los que estaban con él se acercaron con trompetas y cánticos.
\par 26 Pero Judas y su compañía se enfrentaron a los enemigos con invocación y oración.
\par 27 De modo que, luchando con sus manos y orando a Dios con su corazón, mataron a no menos de treinta y cinco mil hombres; porque la aparición de Dios los animó mucho.
\par 28 Terminada la batalla, volvieron con alegría y supieron que Nicanor yacía muerto en sus arneses.
\par 29 Entonces lanzaron un gran grito y ruido, alabando al Todopoderoso en su propia lengua.
\par 30 Y Judas, que siempre fue el principal defensor de los ciudadanos en cuerpo y alma, y ​​que continuó su amor hacia sus compatriotas durante toda su vida, ordenó cortar a Nicanor la cabeza y la mano con el hombro, y traerlos a Jerusalén.
\par 31 Cuando estuvo allí, convocó a los de su nación y puso a los sacerdotes ante el altar, y envió por los de la torre,
\par 32 Y les mostró la cabeza del vil Nicanor y la mano de aquel blasfemo que con orgullosa jactancia había extendido contra el santo templo del Todopoderoso.
\par 33 Y cuando le cortó la lengua al impío Nicanor, ordenó que se la dieran en trozos a las aves y que colgaran delante del templo el premio por su locura.
\par 34 Entonces cada uno alabó hacia el cielo al glorioso Señor, diciendo: Bendito el que ha conservado su lugar sin mancha.
\par 35 Colgó también la cabeza de Nicanor en la torre, señal evidente y manifiesta para todos del auxilio del Señor.
\par 36 Y ordenaron todos con un decreto común que en ningún caso dejaran pasar ese día sin solemnidad, sino que celebraran el día treinta del mes duodécimo, que en lengua siria se llama Adar, la víspera del día de Mardoqueo.
\par 37 Así le sucedió a Nicanor, y desde entonces los hebreos tuvieron la ciudad en su poder. Y aquí pondré fin.
\par 38 Y si lo he hecho bien, y como corresponde a la historia, es lo que deseaba; pero si lo he hecho con modestia y mezquindad, es lo que podía lograr.
\par 39 Porque así como es perjudicial beber vino o agua solos, y como el vino mezclado con agua es agradable y agradable al paladar, así también la palabra finamente estructurada deleita los oídos de los que leen la historia. Y aquí habrá un final.

\end{document}