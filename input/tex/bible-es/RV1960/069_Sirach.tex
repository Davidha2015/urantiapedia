\begin{document}

\title{Eclesiástico}


\chapter{1}

\par 1 Toda sabiduría viene del Señor y está con él para siempre.
\par 2 ¿Quién podrá contar la arena del mar, las gotas de lluvia y los días de la eternidad?
\par 3 ¿Quién podrá conocer la altura del cielo, la anchura de la tierra, el abismo y la sabiduría?
\par 4 La sabiduría fue creada antes de todas las cosas, y la inteligencia y la prudencia desde la eternidad.
\par 5 La palabra del Dios Altísimo es fuente de sabiduría; y sus caminos son mandamientos eternos.
\par 6 [¿A quién se le ha revelado la raíz de la sabiduría? ¿O quién conoció sus sabios consejos?
\par 7 ¿A quién se le ha manifestado el conocimiento de la sabiduría? ¿Y quién ha comprendido su gran experiencia?]
\par 8 Hay uno sabio y muy temible: el Señor sentado en su trono.
\par 9 Él la creó, la vio, la contó y la derramó sobre todas sus obras.
\par 10 Ella está con toda carne según su don, y él la ha dado a los que lo aman.
\par 11 El temor del Señor es honra, gloria, alegría y corona de alegría.
\par 12 El temor del Señor alegra el corazón, da gozo, alegría y larga vida.
\par 13 A quien teme al Señor le irá bien al final y encontrará favor en el día de su muerte.
\par 14 Temer al Señor es el principio de la sabiduría, que fue creada con los fieles en el vientre.
\par 15 Ella ha construido con los hombres un fundamento eterno, y permanecerá con su descendencia.
\par 16 Temer al Señor es plenitud de sabiduría y colma a los hombres de sus frutos.
\par 17 Ella llena toda su casa de bienes deseables, y los graneros con sus frutos.
\par 18 El temor del Señor es corona de sabiduría, que hace florecer la paz y la perfecta salud; Ambos son dones de Dios: y aumenta el regocijo de los que le aman.
\par 19 La sabiduría hace llover la habilidad y el conocimiento de la comprensión, y exalta a los que la sostienen con honor.
\par 20 La raíz de la sabiduría es el temor del Señor, y sus ramas, larga vida.
\par 21 El temor del Señor aleja los pecados, y donde está, aleja la ira.
\par 22 El hombre furioso no puede justificarse; porque el dominio de su furor será su destrucción.
\par 23 El hombre paciente se desgarrará por un tiempo, pero después le brotará la alegría.
\par 24 Él ocultará sus palabras por un tiempo, y los labios de muchos declararán su sabiduría.
\par 25 Las parábolas de la ciencia están en los tesoros de la sabiduría, pero la piedad es abominación para el pecador.
\par 26 Si deseas sabiduría, guarda los mandamientos y el Señor te la dará.
\par 27 Porque el temor del Señor es sabiduría e instrucción, y la fe y la mansedumbre son su deleite.
\par 28 No desconfíes del temor del Señor cuando seas pobre, ni vengas a él con corazón doblegado.
\par 29 No seas hipócrita ante los hombres y presta atención a lo que dices.
\par 30 No te enaltezcas, no sea que caigas y traigas deshonra a tu alma, y ​​Dios descubra tus secretos y te arroje en medio de la congregación, porque no viniste en verdad al temor del Señor, sino que Tu corazón está lleno de engaño.

\chapter{2}

\par 1 Hijo mío, si vienes a servir al Señor, prepara tu alma para la tentación.
\par 2 Endereza tu corazón, y persevera constantemente, y no te apresures en tiempos de angustia.
\par 3 Adhiérete a él y no te apartes, para que seas multiplicado en tu último fin.
\par 4 Todo lo que te sobrevenga, tómalo con alegría y ten paciencia cuando caigas en una condición humilde.
\par 5 Porque el oro se prueba en el fuego, y los hombres aceptables en el horno de la adversidad.
\par 6 Cree en él y él te ayudará; Ordena bien tu camino y confía en él.
\par 7 Los que teméis al Señor, esperad su misericordia; y no os apartéis, para que no caigáis.
\par 8 Los que teméis al Señor, creedle; y tu recompensa no faltará.
\par 9 Los que teméis al Señor, esperad el bien, el gozo y la misericordia eterna.
\par 10 Mirad las generaciones pasadas y ved; ¿Confió alguna vez alguno en el Señor y quedó confundido? ¿O alguno permaneció en su temor y fue abandonado? ¿O a quién despreció alguna vez el que lo invocó?
\par 11 Porque el Señor es compasivo y misericordioso, paciente y muy misericordioso, y perdona los pecados y salva en tiempos de aflicción.
\par 12 ¡Ay del corazón temeroso, de las manos débiles y del pecador que va por dos caminos!
\par 13 ¡Ay del cobarde! porque no cree; por tanto, no será defendido.
\par 14 ¡Ay de vosotros los que habéis perdido la paciencia! ¿Y qué haréis cuando el Señor os visite?
\par 15 Los que temen al Señor no desobedecerán su Palabra; y los que le aman guardarán sus caminos.
\par 16 Los que temen al Señor buscarán lo bueno que le agrada; y los que le aman serán llenos de la ley.
\par 17 Los que temen al Señor prepararán su corazón y humillarán su alma ante él,
\par 18 Diciendo: Caeremos en manos del Señor, y no en manos de los hombres; porque como es su majestad, así es su misericordia.

\chapter{3}

\par 1 Oídme, oh hijos, vuestro padre, y obrad después para que estéis a salvo.
\par 2 Porque el Señor ha honrado al padre sobre los hijos y confirmado la autoridad de la madre sobre los hijos.
\par 3 Quien honra a su padre, hace expiación por sus pecados.
\par 4 Y el que honra a su madre es como quien atesora.
\par 5 El que honra a su padre se regocijará con sus propios hijos; y cuando haga su oración, será oído.
\par 6 El que honra a su padre tendrá una larga vida; y el que obedece al Señor será consuelo para su madre.
\par 7 El que teme al Señor honrará a su padre y servirá a sus padres como a sus amos.
\par 8 Honra a tu padre y a tu madre, tanto de palabra como de obra, para que de ellos te llegue una bendición.
\par 9 Porque la bendición del padre afirma las casas de los hijos; pero la maldición de la madre desarraiga los cimientos.
\par 10 No te gloríes en la deshonra de tu padre; porque la deshonra de tu padre no te es gloria.
\par 11 Porque la gloria del hombre proviene del honor de su padre; y una madre deshonrada es oprobio para los hijos.
\par 12 Hijo mío, ayuda a tu padre en su edad y no le entristezcas mientras viva.
\par 13 Y si su entendimiento falla, tened paciencia con él; y no lo desprecies cuando estés en todas tus fuerzas.
\par 14 Porque el alivio de tu padre no será olvidado, y en lugar de los pecados se añadirá para edificarte.
\par 15 En el día de tu aflicción será recordado; Tus pecados también se derretirán como el hielo en el buen tiempo cálido.
\par 16 El que abandona a su padre es como un blasfemo; y el que enoja a su madre, maldito es: de Dios.
\par 17 Hijo mío, continúa con tus asuntos con mansedumbre; así serás amado por el que es aprobado.
\par 18 Cuanto más grande seas, más humilde serás y hallarás gracia delante del Señor.
\par 19 Muchos están en lugares altos y famosos, pero a los humildes se revelan los misterios.
\par 20 Porque el poder del Señor es grande y es honrado por los humildes.
\par 21 No busques lo que te resulta demasiado difícil, ni busques lo que está por encima de tus fuerzas.
\par 22 Pero piensa en lo que te ha sido mandado con reverencia, porque no es necesario que veas con tus ojos las cosas que están en secreto.
\par 23 No seas curioso en cosas innecesarias, porque se te muestran más cosas de las que los hombres entienden.
\par 24 Porque muchos se dejan engañar por sus vanas opiniones; y una mala sospecha ha trastornado su juicio.
\par 25 Sin ojos te faltará la luz: no confieses, pues, el conocimiento que no tienes.
\par 26 Al corazón obstinado le irá mal al final; y el que ama el peligro, en él perecerá.
\par 27 El corazón obstinado se cargará de dolores; y el impío acumulará pecado sobre pecado.
\par 28 Para el castigo de los soberbios no hay remedio; porque en él ha echado raíces la planta de la maldad.
\par 29 El corazón del prudente entenderá una parábola; y el oído atento es el deseo del sabio.
\par 30 El agua apagará el fuego llameante; y la limosna hace expiación por los pecados.
\par 31 Y el que hace buenas obras piensa en lo que vendrá después; y cuando caiga, encontrará un soporte.

\chapter{4}

\par 1 Hijo mío, no defraudes a los pobres en su sustento, ni hagas esperar mucho a los ojos necesitados.
\par 2 No entristezcas al alma hambrienta; ni provoques al hombre en su angustia.
\par 3 No añadas más problemas a un corazón afligido; y no postergues el dar al necesitado.
\par 4 No rechaces la súplica del afligido; ni apartes tu rostro del pobre.
\par 5 No apartes tu mirada del necesitado, ni le des ocasión de maldecirte.
\par 6 Porque si te maldice con la amargura de su alma, su oración será oída por el que lo hizo.
\par 7 Gánate el amor de la congregación e inclina tu cabeza ante un gran hombre.
\par 8 No te entristezca inclinar tu oído al pobre y darle una respuesta amistosa con mansedumbre.
\par 9 Libra al que sufre injusticia de mano del opresor; y no desmayes cuando te sientes a juzgar.
\par 10 Sé como padre para el huérfano, y en lugar de marido para su madre: así serás como hijo del Altísimo, y él te amará más que tu madre.
\par 11 La sabiduría exalta a sus hijos y atrapa a los que la buscan.
\par 12 El que la ama, ama la vida; y los que desde temprano la buscan se llenarán de alegría.
\par 13 El que la retenga heredará la gloria; y donde ella entre, el Señor la bendecirá.
\par 14 Los que la sirven servirán al Santo; y a los que la aman, el Señor los ama.
\par 15 El que la escucha juzgará a las naciones, y el que la escucha vivirá seguro.
\par 16 Si alguno se entrega a ella, la heredará; y su generación la tendrá en posesión.
\par 17 Porque al principio ella caminará con él por caminos torcidos, le traerá temor y pavor y lo atormentará con su disciplina, hasta que pueda confiar en su alma y probarlo con sus leyes.
\par 18 Entonces ella volverá a él por el camino recto, lo consolará y le mostrará sus secretos.
\par 19 Pero si él se equivoca, ella lo abandonará y lo entregará a su propia ruina.
\par 20 Guardad la oportunidad y guardaos del mal; y no te avergüences en lo que toca a tu alma.
\par 21 Porque hay vergüenza que trae el pecado; y hay una vergüenza que es gloria y gracia.
\par 22 No aceptes a nadie contra tu alma, ni permitas que la reverencia de nadie te haga caer.
\par 23 Y no te abstengas de hablar cuando haya ocasión de hacer el bien, y no ocultes tu sabiduría en su belleza.
\par 24 Porque con la palabra se conocerá la sabiduría, y con la palabra de la lengua se conocerá el saber.
\par 25 No hables nunca contra la verdad; pero avergüénzate del error de tu ignorancia.
\par 26 No te avergüences de confesar tus pecados; y no fuerces el curso del río.
\par 27 No te hagas subordinado de un hombre necio; ni aceptes la persona de los poderosos.
\par 28 Lucha por la verdad hasta la muerte, y el Señor peleará por ti.
\par 29 No seas apresurado en tu lengua, ni negligente y negligente en tus obras.
\par 30 No seas como un león en tu casa, ni te alborotes entre tus siervos.
\par 31 No extiendas tu mano para recibir, ni la cierres para pagar.

\chapter{5}

\par 1 Pon tu corazón en tus bienes; y digas que no, que tengo suficiente para mi vida.
\par 2 No sigas tu propia mente ni tus fuerzas para andar en los caminos de tu corazón.
\par 3 Y no digáis: ¿Quién controlará mis obras? porque el Señor seguramente vengará tu soberbia.
\par 4 No digas: He pecado, ¿y qué daño me ha sucedido? porque el Señor es paciente, no te dejará ir.
\par 5 En cuanto a la propiciación, no tengáis miedo de añadir pecado sobre pecado.
\par 6 Y no digáis que su misericordia es grande; él será apaciguado por la multitud de mis pecados: porque de él proceden la misericordia y la ira, y su ira reposará sobre los pecadores.
\par 7 No tardes en volverte al Señor, ni pospongas el día en el día; porque de repente vendrá la ira del Señor, y en tu seguridad serás destruido, y perecerás en el día de la venganza.
\par 8 No pongas tu corazón en los bienes obtenidos injustamente, porque de nada te aprovecharán en el día de la calamidad.
\par 9 No vendes con todo viento, ni vayas por todos los caminos; porque así hace el pecador que tiene doble lengua.
\par 10 Sé firme en tu entendimiento; y sea tu palabra la misma.
\par 11 Sed prontos para oír; y deja que tu vida sea sincera; y con paciencia responde.
\par 12 Si tienes entendimiento, responde a tu prójimo; si no, pon tu mano sobre tu boca.
\par 13 Honra y vergüenza están en las palabras, y la lengua del hombre es su caída.
\par 14 No te dejes llamar chismoso, ni pongas asechanzas con tu lengua; porque vergüenza atroz caerá sobre el ladrón, y mala condenación sobre el que habla con doble lengua.
\par 15 No ignoréis nada, ni en lo grande ni en lo pequeño.

\chapter{6}

\par 1 En lugar de ser un amigo, no te conviertas en un enemigo; porque [con ello] heredarás mal nombre, vergüenza y oprobio; así también el pecador que tiene doble lengua.
\par 2 No te ensalces en el consejo de tu corazón; que tu alma no sea despedazada como un toro [extraviado solo].
\par 3 Comerás tus hojas, perderás tu fruto y quedarás como árbol seco.
\par 4 El alma malvada destruirá a quien la posee y hará que sus enemigos se burlen de él.
\par 5 La lengua dulce multiplicará los amigos, y la lengua amable aumentará los saludos amables.
\par 6 Estad en paz con muchos; sin embargo, tened un solo consejero entre mil.
\par 7 Si quieres conseguir un amigo, pruébalo primero y no te apresures a darle crédito.
\par 8 Porque alguno es amigo de su propia situación, y no aguantará en el día de tu angustia.
\par 9 Y hay un amigo que, convertido en enemistad y contienda, descubrirá tu oprobio.
\par 10 Además, algún amigo es compañero de mesa, y no permanecerá en el día de tu aflicción.
\par 11 Pero en tu prosperidad será como tú y será valiente con tus siervos.
\par 12 Si estás humillado, él estará contra ti y se esconderá de tu presencia.
\par 13 Apártate de tus enemigos y ten cuidado de tus amigos.
\par 14 Un amigo fiel es una defensa fuerte; y quien lo ha encontrado, ha encontrado un tesoro.
\par 15 Nada puede igualar a un amigo fiel, y su excelencia es invaluable.
\par 16 Un amigo fiel es la medicina de la vida; y los que temen al Señor lo encontrarán.
\par 17 Quien teme al Señor, debe dirigir rectamente su amistad: porque como él es, así será también su prójimo.
\par 18 Hijo mío, recoge la instrucción desde tu juventud, y así hallarás sabiduría hasta tu vejez.
\par 19 Acércate a ella como quien ara y siembra, y espera sus buenos frutos; porque no te fatigarás mucho trabajando en ella, sino que pronto comerás de sus frutos.
\par 20 Ella es muy desagradable para los ignorantes: el que no tiene entendimiento no permanecerá con ella.
\par 21 Ella caerá sobre él como una poderosa piedra de prueba; y él la echará de sí antes de que pase mucho tiempo.
\par 22 Porque la sabiduría es según su nombre, y no se manifiesta a muchos.
\par 23 Escucha, hijo mío, recibe mis consejos y no los rechaces,
\par 24 Y mete tus pies en sus cadenas, y tu cuello en sus cadenas.
\par 25 Inclina tu hombro y sostenla, y no te aflijas por sus prisiones.
\par 26 Ven a ella con todo tu corazón y guarda sus caminos con todas tus fuerzas.
\par 27 Busca y busca, y ella te será conocida; y cuando la agarres, no la dejes ir.
\par 28 Porque al final encontrarás su descanso, y eso se convertirá en tu alegría.
\par 29 Entonces sus cadenas serán para ti una fuerte defensa, y sus cadenas un manto de gloria.
\par 30 Porque lleva sobre ella un adorno de oro, y sus brazaletes son de encaje púrpura.
\par 31 La vestirás como un vestido de honor y la pondrás a tu alrededor como una corona de alegría.
\par 32 Hijo mío, si quieres, serás instruido; y si aplicas tu mente, serás prudente.
\par 33 Si amas oír, adquirirás entendimiento; y si inclinas el oído, serás sabio,
\par 34 Ponte en pie entre la multitud de los ancianos; y adhiérete al sabio.
\par 35 Estén dispuestos a escuchar todo discurso santo; y no dejes que se te escapen las parábolas del entendimiento.
\par 36 Y si ves a un hombre inteligente, ve a él a tiempo y deja que tu pie roce los escalones de su puerta.
\par 37 Deja que tu mente esté en las ordenanzas del Señor y medita continuamente en sus mandamientos: él afirmará tu corazón y te dará la sabiduría que deseas.

\chapter{7}

\par 1 No hagas ningún mal, para que no te suceda ningún daño.
\par 2 Apártate de los injustos, y la iniquidad se apartará de ti.
\par 3 Hijo mío, no siembres en los surcos de la injusticia, y no los cosecharás siete veces.
\par 4 No busques del Señor la preeminencia, ni del rey el lugar de honor.
\par 5 No te justifiques ante el Señor; y no te jactes de tu sabiduría delante del rey.
\par 6 No busques ser juez, no pudiendo quitar la iniquidad; no sea que temas la persona del poderoso, tropezadero en el camino de tu rectitud.
\par 7 No ofendas a la multitud de una ciudad, y entonces no te arrojarás entre el pueblo.
\par 8 No vinculéis un pecado con otro; porque en uno no quedarás impune.
\par 9 No digas: Dios mirará la multitud de mis ofrendas, y cuando la ofrezca al Dios Altísimo, él la aceptará.
\par 10 No desmayes cuando hagas tu oración, ni dejes de dar limosna.
\par 11 Nadie se ría de nadie con desprecio en la amargura de su alma, porque hay quien humilla y quien exalta.
\par 12 No inventes mentira contra tu hermano; Tampoco haz lo mismo con tu amigo.
\par 13 No hagáis mentira alguna, porque su costumbre no es buena.
\par 14 No uses muchas palabras ante una multitud de ancianos, ni hables mucho cuando ores.
\par 15 No odiéis el trabajo duro ni la labranza que el Altísimo ha ordenado.
\par 16 No te cuentes entre la multitud de pecadores, pero recuerda que la ira no tardará.
\par 17 Humíllate mucho, porque la venganza de los impíos es fuego y gusanos.
\par 18 No cambies a un amigo por ningún bien de ningún modo; ni hermano fiel por el oro de Ofir.
\par 19 No desprecies a la mujer sabia y buena, porque su gracia es superior al oro.
\par 20 Mientras que tu siervo obra con verdad, no le supliques el mal ni el asalariado que se entrega enteramente por ti.
\par 21 Que ame tu alma al buen siervo y no le prive de su libertad.
\par 22 ¿Tienes ganado? mantenlos atentos; y si son para tu provecho, guárdalos contigo.
\par 23 ¿Tienes hijos? instrúyelos, e inclina su cerviz desde su juventud.
\par 24 ¿Tienes hijas? Cuida su cuerpo y no te muestres alegre para con ellos.
\par 25 Cásate con tu hija y así habrás cumplido un asunto importante; pero entrégasela a un hombre inteligente.
\par 26 ¿Tienes esposa según tus deseos? No la abandones, pero no te entregues a una mujer liviana.
\par 27 Honra a tu padre con todo tu corazón y no olvides los dolores de tu madre.
\par 28 Recuerda que fuiste engendrado de ellos; ¿Y cómo les podrás recompensar por las cosas que han hecho por ti?
\par 29 Teme al Señor con toda tu alma y reverencia a sus sacerdotes.
\par 30 Ama al que te hizo con todas tus fuerzas y no abandones a sus ministros.
\par 31 Teme al Señor y honra al sacerdote; y dale su porción, como te es mandado; las primicias, la ofrenda por la culpa, la ofrenda de hombros, el sacrificio de santificación y las primicias de las cosas santas.
\par 32 Y extiende tu mano hacia los pobres, para que tu bendición sea perfecta.
\par 33 Un don tiene gracia a los ojos de todo hombre viviente; y por los muertos no lo detengas.
\par 34 No dejes de estar con los que lloran, y llora con los que lloran.
\par 35 No tardes en visitar a los enfermos: eso te hará ser amado.
\par 36 Todo lo que emprendas, recuerda el fin y nunca harás nada malo.

\chapter{8}

\par 1 No luches con un hombre fuerte, no sea que caigas en sus manos.
\par 2 No te enemistades con el rico, no sea que te supere; porque el oro ha destruido a muchos y pervertido el corazón de los reyes.
\par 3 No riñas con el hombre de lengua llena, ni amontones leña sobre su fuego.
\par 4 No bromees con el hombre grosero, para que tus antepasados ​​no caigan en desgracia.
\par 5 No reproches al hombre que se aparta del pecado, sino recuerda que todos somos dignos de castigo.
\par 6 No deshonres a nadie en su vejez, porque incluso algunos de nosotros envejecemos.
\par 7 No te regocijes por la muerte de tu mayor enemigo, sino recuerda que todos morimos.
\par 8 No desprecies las palabras de los sabios, sino familiarízate con sus proverbios, porque de ellos aprenderás la instrucción y cómo servir con facilidad a los grandes.
\par 9 No te pierdas el discurso de los ancianos, porque ellos también aprendieron de sus padres, y de ellos aprenderás a entender y a responder según sea necesario.
\par 10 No enciendas las brasas del pecador, no sea que te quemes con la llama de su fuego.
\par 11 No te levantes [enfadado] ante la presencia de una persona injuriosa, no sea que esté al acecho para atraparte en tus palabras.
\par 12 No prestes a quien es más fuerte que tú; porque si lo prestas, tenlo por perdido.
\par 13 No fies más que tu poder; porque si eres fiador, cuídate de pagarlo.
\par 14 No vayas a pleito con un juez; porque juzgarán por él según su honor.
\par 15 No vayas por el camino con un hombre valiente, no sea que te resulte gravoso; porque él hará según su voluntad, y tú perecerás con él por su necedad.
\par 16 No pelees con un hombre enojado, ni vayas con él a un lugar solitario; porque la sangre es nada ante sus ojos, y donde no hay ayuda, te derribará.
\par 17 No consultes con un necio; porque no puede mantener un consejo.
\par 18 No hagas nada secreto delante de un extraño; porque no sabes lo que dará a luz.
\par 19 No abras tu corazón a nadie, no sea que te pague con astucia.

\chapter{9}

\par 1 No tengas celos de la esposa de tu seno, ni le enseñes malas lecciones contra ti mismo.
\par 2 No entregues tu alma a una mujer para que ponga su pie sobre tus bienes.
\par 3 No te encuentres con una ramera, no sea que caigas en sus trampas.
\par 4 No uses mucho la compañía de una mujer que canta, no sea que te sorprendan sus intentos.
\par 5 No mires a una doncella, para no caer en lo que es precioso en ella.
\par 6 No entregues tu alma a las rameras, para que no pierdas tu herencia.
\par 7 No mires a tu alrededor por las calles de la ciudad, ni deambules por sus lugares solitarios.
\par 8 Aparta tu mirada de la mujer hermosa, y no mires la belleza ajena; porque muchos han sido engañados por la belleza de una mujer; porque aquí el amor se enciende como un fuego.
\par 9 No te sientes con la mujer de otro, ni te sientes con ella en tus brazos, ni gastes tu dinero con ella en el vino; no sea que tu corazón se incline hacia ella, y por tu deseo caigas en destrucción.
\par 10 No abandones a un viejo amigo; porque lo nuevo no es comparable a él: un amigo nuevo es como vino nuevo; cuando esté viejo, lo beberás con gusto.
\par 11 No envidies la gloria del pecador, porque no sabes cuál será su fin.
\par 12 No te deleites en lo que agrada a los impíos; pero recuerda que no quedarán impunes hasta la tumba.
\par 13 Manténte alejado del hombre que tiene poder para matar; así no dudarás del temor de la muerte: y si vienes a él, no cometas falta, no sea que te quite la vida al instante: recuerda que vas en medio de trampas, y que caminas sobre las almenas de la ciudad.
\par 14 Adivina lo más que puedas a tu prójimo y consulta con los sabios.
\par 15 Sea tu conversación con los sabios, y toda tu comunicación en la ley del Altísimo.
\par 16 Y que los justos coman y beban contigo; y que tu gloria sea en el temor del Señor.
\par 17 Por la mano del artífice será alabada la obra, y por el sabio gobernante del pueblo por su palabra.
\par 18 El hombre de mala lengua es peligroso en su ciudad; y el que habla imprudentemente será aborrecido.

\chapter{10}

\par 1 Un juez sabio instruirá a su pueblo; y el gobierno del hombre prudente es bien ordenado.
\par 2 Así como él mismo es el juez del pueblo, así también lo son sus oficiales; y cuál sea la clase de hombre que sea el gobernante de la ciudad, tales son todos los que en ella habitan.
\par 3 El rey imprudente destruye a su pueblo; pero por la prudencia de los que están en autoridad la ciudad será habitada.
\par 4 El poder de la tierra está en la mano del Señor, y a su debido tiempo él pondrá sobre ella uno que sea provechoso.
\par 5 En la mano de Dios está la prosperidad del hombre, y sobre la persona del escriba pondrá su honor.
\par 6 No odies a tu prójimo por ningún mal; y no hacer nada en absoluto mediante prácticas nocivas.
\par 7 La soberbia es aborrecible delante de Dios y de los hombres, y con ambos se comete iniquidad.
\par 8 A causa de las injusticias, de las injurias y de las riquezas obtenidas con engaño, el reino pasa de un pueblo a otro.
\par 9 ¿Por qué se enorgullecen la tierra y las cenizas? No hay cosa más malvada que el hombre codicioso: porque tal persona pone en venta su propia alma; porque mientras vive desecha sus entrañas.
\par 10 El médico corta la enfermedad larga; y el que hoy es rey mañana morirá.
\par 11 Porque cuando un hombre muera, heredará reptiles, bestias y gusanos.
\par 12 El comienzo del orgullo es cuando uno se aparta de Dios y su corazón se aparta de su Hacedor.
\par 13 Porque el principio del pecado es la soberbia, y quien la tiene derramará abominación. Por eso el Señor trajo sobre ellos calamidades extrañas y los destruyó por completo.
\par 14 El Señor derribó los tronos de los príncipes orgullosos y puso en su lugar a los humildes.
\par 15 El Señor arrancó las raíces de las naciones soberbias y plantó a los humildes en su lugar.
\par 16 El Señor derribó las tierras de los paganos y las destruyó hasta los cimientos de la tierra.
\par 17 A algunos de ellos los quitó, los destruyó y hizo desaparecer de la tierra su memoria.
\par 18 No fue hecho para los hombres el orgullo, ni la ira furiosa para los nacidos de mujer.
\par 19 Los que temen al Señor son una simiente segura, y los que lo aman, una planta hermosa; los que no respetan la ley, una simiente deshonrosa; los que transgreden los mandamientos son simiente engañosa.
\par 20 Entre los hermanos, el que es jefe es honorable; Así son los que temen al Señor ante sus ojos.
\par 21 El temor del Señor es anterior a la obtención de la autoridad, pero la rudeza y la soberbia son la pérdida de la misma.
\par 22 Ya sea rico, noble o pobre, su gloria es el temor del Señor.
\par 23 No está bien despreciar al pobre que tiene entendimiento; tampoco conviene engrandecer a un hombre pecador.
\par 24 Los grandes, los jueces y los potentados serán honrados; sin embargo, ninguno de ellos es mayor que el que teme al Señor.
\par 25 Al siervo sabio le servirán los libres; y el que tiene conocimiento no guardará rencor cuando se corrija.
\par 26 No seas demasiado imprudente en tus negocios; y no te jactes en el tiempo de tu angustia.
\par 27 Mejor es el que trabaja y abunda en todo, que el que se jacta y carece de pan.
\par 28 Hijo mío, glorifica tu alma con mansedumbre y dale honor según su dignidad.
\par 29 ¿Quién justificará al que peca contra su propia alma? ¿Y quién honrará al que deshonra su propia vida?
\par 30 El pobre es honrado por su habilidad, y el rico es honrado por sus riquezas.
\par 31 El que es honrado en la pobreza, ¿cuánto más en las riquezas? y el que es deshonroso en las riquezas, ¿cuánto más en la pobreza?

\chapter{11}

\par 1 La sabiduría levanta la cabeza del humilde y lo hace sentarse entre los grandes.
\par 2 No elogies a ningún hombre por su belleza; ni aborrezcáis al hombre por su apariencia exterior.
\par 3 La abeja es pequeña entre las moscas; pero su fruto es el principal de los dulces.
\par 4 No te jactes de tu ropa ni de tu vestido, ni te exaltes en el día de la honra; porque las obras del Señor son maravillosas, y sus obras están ocultas entre los hombres.
\par 5 Muchos reyes se han sentado en el suelo; y alguien en quien nunca se pensó ha llevado la corona.
\par 6 Muchos valientes han sido grandemente deshonrados; y los honorables entregados en manos ajenas.
\par 7 No culpes antes de haber examinado la verdad: comprende primero y luego reprende.
\par 8 No respondas antes de haber oído la causa, ni interrumpas a los hombres en medio de su conversación.
\par 9 No te esfuerces en asuntos que no te conciernen; y no juzguéis a los pecadores.
\par 10 Hijo mío, no te metas en muchos asuntos; porque si te metes mucho, no serás inocente; y si sigues, no obtendrás, ni escaparás huyendo.
\par 11 Hay quien se esfuerza, se esfuerza y ​​se apresura, y tanto más se queda atrás.
\par 12 También hay otro que es lento y necesitado de ayuda, falto de capacidad y lleno de pobreza; sin embargo, los ojos del Señor lo miraron para bien y lo levantaron de su humillación.
\par 13 Y alzó su cabeza en medio de la miseria; para que muchos que vieron de él haya paz sobre todos los
\par 14 Bienestar y adversidad, vida y muerte, pobreza y riqueza, vienen del Señor.
\par 15 La sabiduría, el conocimiento y la comprensión de la ley son del Señor; de él son el amor y el camino de las buenas obras.
\par 16 El error y las tinieblas comenzaron con los pecadores, y el mal envejecerá con los que se glorían en ellos.
\par 17 El don del Señor permanece con los impíos, y su favor trae prosperidad para siempre.
\par 18 Hay quien se enriquece con su cautela y su austeridad, y esta es su parte de su recompensa:
\par 19 Mientras que él dice: He encontrado descanso y ahora comeré continuamente de mis bienes; y, sin embargo, no sabe qué tiempo le sobrevendrá y que deberá dejar esas cosas a otros y morir.
\par 20 Mantente firme en tu pacto, conviértete en él y envejece en tu trabajo.
\par 21 No os maravilléis de las obras de los pecadores; sino confía en el Señor, y permanece en tu trabajo; porque es cosa fácil a los ojos del Señor enriquecer de repente a un pobre.
\par 22 La bendición del Señor está en la recompensa de los piadosos, y de repente hace florecer su bendición.
\par 23 No digáis: ¿Qué beneficio obtendré de mi servicio? ¿Y qué bienes tendré en el futuro?
\par 24 No digáis tampoco: Tengo suficiente y poseo muchas cosas, ¿y qué mal tendré en el futuro?
\par 25 En el día de la prosperidad hay olvido de la aflicción, y en el día de la aflicción ya no hay recuerdo de la prosperidad.
\par 26 Porque al Señor le es fácil recompensar al hombre según sus caminos en el día de su muerte.
\par 27 La aflicción de una hora hace que el hombre olvide el placer, y al final sus obras serán descubiertas.
\par 28 No juzguéis a ningún bienaventurado antes de su muerte: porque el hombre será conocido en sus hijos.
\par 29 No metas a nadie en tu casa, porque el hombre engañador tiene muchas familias.
\par 30 Como una perdiz encerrada en una jaula, así es el corazón de los soberbios; y como un espía, vigila tu caída:
\par 31 Porque él acecha y convierte el bien en mal, y en las cosas dignas de alabanza te reprochará.
\par 32 De una chispa de fuego se enciende un montón de brasas, y el pecador acecha la sangre.
\par 33 ¡Cuidado con el hombre malvado, porque hace maldades! no sea que te traiga mancha perpetua.
\par 34 Recibe a un extraño en tu casa, y él te perturbará y te echará de la tuya.

\chapter{12}

\par 1 Cuando quieras hacer el bien, conoce a quién se lo haces; así te agradecerán tus beneficios.
\par 2 Haz el bien al justo y obtendrás su recompensa; y si no de él, al menos del Altísimo.
\par 3 No puede venir ningún bien al que siempre está ocupado en el mal, ni al que no da limosna.
\par 4 Da al justo y no ayudes al pecador.
\par 5 Haz bien al humilde, pero no le des al impío; retiene tu pan y no se lo des, no sea que te domine; porque de lo contrario recibirás el doble de mal por todos los bien le habrás hecho.
\par 6 Porque el Altísimo aborrece a los pecadores, quiere vengarse de los impíos y los protege del gran día de su castigo.
\par 7 Da al bueno y no ayudes al pecador.
\par 8 En la prosperidad no se puede conocer al amigo, ni en la adversidad se puede ocultar al enemigo.
\par 9 En la prosperidad de un hombre los enemigos serán afligidos, pero en su adversidad incluso el amigo se alejará.
\par 10 Nunca confíes en tu enemigo, porque como el hierro se oxida, así es su maldad.
\par 11 Aunque se humille y ande agachado, ten cuidado y ten cuidado con él, y serás para él como si hubieras limpiado un espejo, y sabrás que su óxido no ha sido limpiado del todo.
\par 12 No lo pongas junto a ti, no sea que, cuando te derribe, se levante en tu lugar; ni dejes que se siente a tu diestra, no sea que intente ocupar tu asiento, y al final te acuerdes de mis palabras y seas pinchado con ellas.
\par 13 ¿Quién se compadecerá del encantador mordido por una serpiente o del que se acerca a las fieras?
\par 14 Así que el que acude a un pecador y se contamina con él en sus pecados, ¿quién se apiadará?
\par 15 Él permanecerá contigo por un tiempo, pero si comienzas a caer, no se quedará.
\par 16 El enemigo habla dulcemente con sus labios, pero en su corazón imagina cómo arrojarte a un hoyo; llorará con sus ojos, pero si encuentra la oportunidad, no se saciará de sangre.
\par 17 Si te sobreviene alguna desgracia, allí lo encontrarás primero; y aunque pretenda ayudarte, te socavará.
\par 18 Meneará la cabeza, batirá palmas, susurrará mucho y cambiará de semblante.

\chapter{13}

\par 1 El que toque brea quedará contaminado con ella; y el que tiene compañerismo con el orgulloso será semejante a él.
\par 2 No te excedas de tus fuerzas mientras vivas; y no tengas comunión con nadie que sea más poderoso y más rico que tú: porque ¿cómo concuerdan la marmita y la vasija de barro? porque si uno es herido contra el otro, será quebrantado.
\par 3 El rico ha hecho mal y, sin embargo, amenaza con ello; el pobre es agraviado y él también debe suplicar.
\par 4 Si buscas su beneficio, él te usará; pero si no tienes nada, te abandonará.
\par 5 Si tienes algo, él vivirá contigo; incluso te desnudará y no se arrepentirá de ello.
\par 6 Si tiene necesidad de ti, te engañará, te sonreirá y te dará esperanza; Él te hablará con dulzura y te dirá: ¿Qué quieres?
\par 7 Y te avergonzará con sus comidas, hasta dejarte seco dos o tres veces, y al final se reirá de ti hasta despreciarte, y cuando te vea, te abandonará y meneará la cabeza ante ti.
\par 8 Cuídate de no dejarte engañar ni desanimarte en tu alegría.
\par 9 Si eres invitado por un valiente, retírate, y tanto más te invitará.
\par 10 No lo presiones, para que no te devuelvan; No te quedes lejos, para que no caigas en el olvido.
\par 11 No finjas ser igual a él en palabras, ni creas en sus muchas palabras: porque con mucha comunicación te tentará, y sonriéndote revelará tus secretos.
\par 12 Pero él guardará tus palabras con crueldad y no dejará de hacerte daño y encarcelarte.
\par 13 Observa y ten cuidado, porque caminas en peligro de ser destruido. Cuando oigas estas cosas, despierta en tu sueño.
\par 14 Ama al Señor toda tu vida e invócalo para tu salvación.
\par 15 Cada animal ama a su semejante, y cada hombre ama a su prójimo.
\par 16 Toda carne se asocia según su especie, y el hombre se unirá a su semejante.
\par 17 ¿Qué compañerismo tiene el lobo con el cordero? así el pecador con el piadoso.
\par 18 ¿Qué acuerdo hay entre la hiena y un perro? ¿Y qué paz entre ricos y pobres?
\par 19 Como el asno montés es presa del león en el desierto, así el rico se come al pobre.
\par 20 Así como los soberbios odian la humildad, así el rico aborrece al pobre.
\par 21 El rico que comienza a caer es sostenido por sus amigos, pero el pobre que está caído es rechazado por sus amigos.
\par 22 Cuando el rico cae, tiene muchos ayudantes: habla cosas que no se deben decir, y sin embargo los hombres lo justifican; el pobre resbaló, y sin embargo también lo reprendieron; habló sabiamente y no pudo tener lugar.
\par 23 Cuando un rico habla, todos se callan y, mira, lo que dice, lo exaltan hasta las nubes; pero si el pobre habla, dicen: ¿Quién es éste? y si tropieza, le ayudarán a derribarlo.
\par 24 Las riquezas son buenas para el que no tiene pecado, y la pobreza es mala en boca de los impíos.
\par 25 El corazón del hombre cambia su rostro, ya sea para bien o para mal; y un corazón alegre alegra el rostro.
\par 26 El rostro alegre es señal de un corazón que está en prosperidad; y descubrir parábolas es un trabajo mental tedioso.

\chapter{14}

\par 1 Bienaventurado el hombre que no resbaló con su boca, ni se dejó picar por la multitud de pecados.
\par 2 Bienaventurado aquel cuya conciencia no lo ha condenado, y que no ha perdido su esperanza en el Señor.
\par 3 Las riquezas no son agradables al avaro: ¿y qué hará el envidioso con el dinero?
\par 4 El que recolecta defraudando su propia alma, recolecta para otros, que gastarán sus bienes desenfrenadamente.
\par 5 El que es malo consigo mismo, ¿con quién será bueno? no se complacerá en sus bienes.
\par 6 No hay nadie peor que el que se envidia a sí mismo; y esta es la recompensa de su maldad.
\par 7 Y si hace el bien, lo hace de mala gana; y al final declarará su maldad.
\par 8 El envidioso tiene malos ojos; Vuelve su rostro y menosprecia a los hombres.
\par 9 Los ojos del codicioso no se sacian de su porción; y la iniquidad del impío seca su alma.
\par 10 El ojo malvado tiene envidia de su pan, y es tacaño en su mesa.
\par 11 Hijo mío, hazte bien a ti mismo según tus posibilidades y entrega al Señor su ofrenda debida.
\par 12 Recuerda que la muerte no tardará en llegar, y que el pacto del sepulcro no te será revelado.
\par 13 Haz el bien a tu amigo antes de morir y, según tus posibilidades, extiende tu mano y dáselo.
\par 14 No te defraudes del buen día, ni dejes que la parte del buen deseo te supere.
\par 15 ¿No dejarás a otro tus trabajos? y tus trabajos serán repartidos por suerte?
\par 16 Da y recibe, y santifica tu alma; porque no se buscan delicias en la tumba.
\par 17 Toda carne se envejece como un vestido; porque el pacto desde el principio es: Morirás de muerte.
\par 18 Como de las hojas verdes del árbol frondoso, unas caen y otras crecen; así es la generación de carne y sangre: una termina y otra nace.
\par 19 Toda obra se pudre y se consume, y su trabajador se marchará.
\par 20 Bienaventurado el hombre que medita las cosas buenas con sabiduría, y que razona las cosas santas con su inteligencia. En g.
\par 21 El que considera sus caminos en su corazón, también tendrá entendimiento en sus secretos.
\par 22 Id tras ella como quien rastrea, y acechad en sus caminos.
\par 23 El que espía en sus ventanas, también escuchará en sus puertas.
\par 24 El que habite cerca de su casa, también clavará estacas en sus paredes.
\par 25 Cerca de ella plantará su tienda y se alojará en una posada donde haya cosas buenas.
\par 26 A sus hijos pondrá bajo su abrigo, y bajo sus ramas se alojará.
\par 27 Ella lo protegerá del calor y habitará en su gloria.

\chapter{15}

\par 1 El que teme al Señor hará el bien, y el que conoce la ley la obtendrá.
\par 2 Y como una madre lo encontrará y lo recibirá como a una esposa desposada con una virgen.
\par 3 Ella lo alimentará con el pan de la inteligencia y le dará a beber el agua de la sabiduría.
\par 4 Sobre ella permanecerá, y no será conmovido; y confiarán en ella, y no serán confundidos.
\par 5 Ella lo exaltará sobre sus vecinos, y en medio de la congregación abrirá su boca.
\par 6 Hallará gozo y una corona de alegría, y ella le hará heredar un nombre eterno.
\par 7 Pero los necios no la alcanzarán, ni los pecadores la verán.
\par 8 Porque ella está lejos del orgullo, y los hombres mentirosos no pueden recordarla.
\par 9 La alabanza no es propia de la boca del pecador, porque no le ha sido enviada por el Señor.
\par 10 Porque la alabanza se pronunciará con sabiduría y el Señor la prosperará.
\par 11 No digas: «Por el Señor he caído», porque no debes hacer lo que él aborrece.
\par 12 No digas: Él me ha hecho errar, porque no tiene necesidad del hombre pecador.
\par 13 El Señor aborrece toda abominación; y los que temen a Dios no lo aman.
\par 14 Él mismo hizo al hombre desde el principio y lo dejó en manos de su consejo;
\par 15 Si quieres, guarda los mandamientos y practica una fidelidad aceptable.
\par 16 Fuego y agua ha puesto delante de ti; extiende tu mano hacia lo que quieras.
\par 17 Ante el hombre está la vida y la muerte; y se le dará lo que quiera.
\par 18 Porque la sabiduría del Señor es grande, y poderoso en poder, y contempla todas las cosas.
\par 19 Y sus ojos están sobre los que le temen, y conoce todas las obras del hombre.
\par 20 A nadie ha ordenado hacer el mal, ni ha dado a nadie licencia para pecar.

\chapter{16}

\par 1 No desees tener muchos hijos inútiles, ni te deleites con los hijos impíos.
\par 2 Aunque se multipliquen, no os regocijéis en ellos, a menos que el temor del Señor esté con ellos.
\par 3 No te fíes de su vida, ni respetes su multitud: porque mejor es uno que es justo que mil; y mejor es morir sin hijos, que tenerlos impíos.
\par 4 Porque con el que tiene entendimiento la ciudad se reabastecerá, pero las familias de los impíos pronto quedarán asoladas.
\par 5 Muchas cosas así he visto con mis ojos, y mi oído ha oído cosas mayores que éstas.
\par 6 En la congregación de los impíos se encenderá un fuego; y en una nación rebelde se enciende la ira.
\par 7 No se tranquilizó con los viejos gigantes, que cayeron en el poder de su necedad.
\par 8 Tampoco perdonó el lugar donde residió Lot, sino que los aborreció por su orgullo.
\par 9 No se compadeció del pueblo de perdición, que fue llevado por sus pecados.
\par 10 Ni los seiscientos mil soldados de a pie, que estaban reunidos en la dureza de sus corazones.
\par 11 Y si hay alguno duro de cerviz entre el pueblo, es maravilla que quede sin castigo, porque la misericordia y la ira están con él; él es poderoso para perdonar y derramar el descontento.
\par 12 Como es grande su misericordia, así también lo es su corrección: juzga al hombre según sus obras.
\par 13 El pecador no escapará con su botín, ni la paciencia de los piadosos será frustrada.
\par 14 Abrid paso a toda obra de misericordia, porque cada uno hallará según sus obras.
\par 15 El Señor endureció a Faraón para que no lo conociera, para que el mundo conociera sus poderosas obras.
\par 16 Su misericordia es manifiesta para toda criatura; y ha separado su luz de las tinieblas con un diamante.
\par 17 No digas: Me esconderé del Señor. ¿Se acordará de mí alguno desde arriba? No seré recordado entre tanta gente: porque ¿qué es mi alma entre un número tan infinito de criaturas?
\par 18 He aquí, los cielos y los cielos de los cielos, el abismo, la tierra y todo lo que hay en ellos se estremecerán cuando él los visite.
\par 19 También las montañas y los cimientos de la tierra se estremecerán cuando el Señor los mire.
\par 20 Ningún corazón puede pensar en estas cosas dignamente; ¿y quién podrá concebir sus caminos?
\par 21 Es una tempestad que nadie puede ver: la mayor parte de sus obras están ocultas.
\par 22 ¿Quién podrá declarar las obras de su justicia? ¿O quién podrá soportarlos? porque lejos está su pacto, y la prueba de todas las cosas está al final.
\par 23 El que carece de entendimiento pensará en cosas vanas, y el necio yerra imagina locuras.
\par 24 Hijo, escúchame, aprende conocimiento y recuerda mis palabras en tu corazón.
\par 25 Mostraré la doctrina con peso y declararé su conocimiento exactamente.
\par 26 Las obras del Señor son hechas en juicio desde el principio; y desde el momento en que las hizo, dispuso de sus partes.
\par 27 Él adornó sus obras para siempre, y en su mano están las principales de ellas para todas las generaciones: no trabajan, ni se cansan, ni cesan en sus obras.
\par 28 Ninguno de ellos obstaculizará a otro, y nunca desobedecerán su palabra.
\par 29 Después de esto, el Señor miró la tierra y la llenó de sus bendiciones.
\par 30 Con toda clase de seres vivientes cubrió su faz; y volverán a ella otra vez.

\chapter{17}

\par 1 El Señor creó al hombre de la tierra y lo transformó nuevamente en ella.
\par 2 Les dio pocos días y poco tiempo, y poder sobre todo lo que había en ellos.
\par 3 Él los fortaleció por sí mismos y los hizo conforme a su imagen.
\par 4 Y puso el temor del hombre sobre toda carne, y le dio dominio sobre las bestias y las aves.
\par 5 Recibieron el uso de las cinco operaciones del Señor, y en el sexto lugar les impartió entendimiento, y en el séptimo discurso, un intérprete de sus reflexiones.
\par 6 Les dio consejo, lengua, ojos, oídos y corazón para entender.
\par 7 Y además los llenó de conocimiento y entendimiento, y les mostró el bien y el mal.
\par 8 Él puso sus ojos en sus corazones para mostrarles la grandeza de sus obras.
\par 9 Les dio la gloria eterna de sus maravillas, para que contasen con inteligencia sus obras.
\par 10 Y los elegidos alabarán su santo nombre.
\par 11 Además de esto, les dio en herencia la ciencia y la ley de vida.
\par 12 Hizo con ellos un pacto eterno y les mostró sus juicios.
\par 13 Sus ojos vieron la majestad de su gloria, y sus oídos oyeron su gloriosa voz.
\par 14 Y él les dijo: Guardaos de toda injusticia; y dio a cada uno mandamientos acerca de su prójimo.
\par 15 Sus caminos están siempre delante de él, y no serán ocultos a sus ojos.
\par 16 Todo hombre desde su juventud es dado al mal; Tampoco podrían transformarse en corazones de piedra por corazones de carne.
\par 17 Porque al dividir las naciones de toda la tierra, puso un gobernante sobre cada pueblo; pero Israel es la porción del Señor:
\par 18 A quien, siendo su primogénito, lo alimenta con disciplina y, dándole la luz de su amor, no lo abandona.
\par 19 Por eso todas sus obras son como el sol delante de él, y sus ojos están continuamente sobre sus caminos.
\par 20 Ninguna de sus injusticias le es ocultada, sino que todos sus pecados están delante del Señor.
\par 21 Pero el Señor, siendo misericordioso y conociendo su obra, no los dejó ni los abandonó, sino que los perdonó.
\par 22 La limosna del hombre es como un sello para él, y guardará las buenas obras del hombre como a la niña de sus ojos, y dará arrepentimiento a sus hijos e hijas.
\par 23 Después se levantará y los recompensará, y les pagará la recompensa sobre sus cabezas.
\par 24 Pero a los que se arrepintieron les concedió el retorno y consoló a los que perdieron la paciencia.
\par 25 Vuélvete al Señor y abandona tus pecados, haz tu oración delante de él y ofendes menos.
\par 26 Vuélvete al Altísimo y apártate de la iniquidad, porque él te sacará de las tinieblas a la luz de la salud y aborrecerá con vehemencia tu abominación.
\par 27 ¿Quién alabará al Altísimo en el Sepulcro, en lugar de los que viven y dan gracias?
\par 28 La acción de gracias desaparece entre los muertos, como entre los que no existen; los vivos y los sanos de corazón alabarán al Señor.
\par 29 ¡Cuán grande es la misericordia del Señor nuestro Dios y su compasión hacia los que se vuelven a él en santidad!
\par 30 Porque no todo puede estar en los hombres, porque el hijo del hombre no es inmortal.
\par 31 ¿Qué es más brillante que el sol? sin embargo, su luz falla; y la carne y la sangre imaginarán el mal.
\par 32 Él ve el poder de las alturas del cielo; y todos los hombres no son más que tierra y cenizas.

\chapter{18}

\par 1 El que vive para siempre ha creado todas las cosas en general.
\par 2 Sólo el Señor es justo, y no hay otro fuera de él,
\par 3 Quien gobierna el mundo con la palma de su mano y todas las cosas obedecen a su voluntad; porque él es el Rey de todo, y con su poder separa entre sí las cosas santas de las profanas.
\par 4 ¿A quién le ha dado poder para declarar sus obras? ¿Y quién se enterará de sus nobles actos?
\par 5 ¿Quién podrá contar el poder de su majestad? ¿Y quién anunciará también sus misericordias?
\par 6 En cuanto a las maravillas del Señor, no se les puede quitar nada, ni se les puede poner nada, ni se puede descubrir su fundamento.
\par 7 Cuando el hombre termina, comienza; y cuando lo deje, entonces dudará.
\par 8 ¿Qué es el hombre y para qué sirve? ¿Cuál es su bien y cuál su mal?
\par 9 Los días de un hombre son como máximo cien años.
\par 10 Como gota de agua en el mar, y como grava en comparación con la arena; así son mil años hasta los días de la eternidad.
\par 11 Por eso Dios es paciente con ellos y derrama sobre ellos su misericordia.
\par 12 Vio y percibió que su fin era malo; por eso multiplicó su compasión.
\par 13 La misericordia del hombre es para con su prójimo; pero la misericordia del Señor es sobre toda carne: él reprende, sustenta, enseña y hace volver, como pastor a su rebaño.
\par 14 Él tiene misericordia de aquellos que reciben disciplina y que diligentemente buscan sus juicios.
\par 15 Hijo mío, no manches tus buenas obras ni uses palabras desagradables cuando des algo.
\par 16 ¿No calmará el rocío el calor? así es mejor una palabra que un regalo.
\par 17 He aquí, ¿no es mejor una palabra que un regalo? pero ambos están con un hombre amable.
\par 18 El necio reprende groseramente, y el regalo del envidioso consume los ojos.
\par 19 Aprende antes de hablar y usa la medicina o enfermarás.
\par 20 Antes del juicio, examínate a ti mismo, y en el día de la visita encontrarás misericordia.
\par 21 Humíllate antes de enfermarte y, en el momento de tus pecados, muestra arrepentimiento.
\par 22 Que nada te impida cumplir tu promesa a su debido tiempo, y no tardes hasta la muerte para ser justificado.
\par 23 Antes de orar, prepárate; y no seas como el que tienta al Señor.
\par 24 Piensa en la ira que vendrá al final, y en el tiempo de la venganza, cuando él vuelva su rostro.
\par 25 Cuando tengas suficiente, recuerda el tiempo del hambre; y cuando seas rico, piensa en la pobreza y la necesidad.
\par 26 Desde la mañana hasta la tarde el tiempo cambia y pronto todo sucede ante el Señor.
\par 27 El sabio temerá todo y en el día de su pecado se guardará de la ofensa; pero el necio no observará el tiempo.
\par 28 Todo hombre inteligente conoce la sabiduría y alabará a quien la encontró.
\par 29 Los que eran entendidos en dichos, también ellos mismos se volvieron sabios y contaron hermosas parábolas.
\par 30 No sigas tus concupiscencias, sino absténte de tus apetitos.
\par 31 Si concedes a tu alma los deseos que la agradan, ella te convertirá en el hazmerreír de tus enemigos que te calumnian.
\par 32 No te deleites con el buen humor, ni te atasques a su gasto.
\par 33 No te conviertas en mendigo al dar un banquete con préstamos cuando no tienes nada en tu bolsa; porque estarás acechando tu propia vida y serás engañado.

\chapter{19}

\par 1 El hombre que trabaja y se emborracha, no se hará rico; y el que menosprecia las cosas pequeñas, poco a poco caerá.
\par 2 El vino y las mujeres harán apostatar a los hombres prudentes, y el que se une a rameras se volverá insolente.
\par 3 La polilla y los gusanos se lo apoderarán, y el hombre valiente será arrebatado.
\par 4 El que se apresura a dar crédito es liviano; y el que pecare, ofenderá contra su propia alma.
\par 5 Quien se deleita en la maldad será condenado; pero el que resiste los deleites coronará su vida.
\par 6 El que sabe dominar su lengua vivirá sin contiendas; y el que aborrece la palabrería, tendrá menos mal.
\par 7 No cuentes a nadie lo que te dicen, y nunca te irá peor.
\par 8 Ya sea con amigos o enemigos, no hables de la vida de otros hombres; y si puedes sin ofender, no los reveles.
\par 9 Porque él te escuchó y te observó, y cuando llegue el momento te odiará.
\par 10 Si has oído una palabra, que muera contigo; y sé valiente, no te reventará.
\par 11 El necio da a luz con una palabra, como la mujer que da a luz.
\par 12 Como flecha que se clava en el muslo del hombre, así es la palabra en el vientre del necio.
\par 13 Amonesta a un amigo, quizá no lo haya hecho; y si lo ha hecho, que no lo haga más.
\par 14 Amonesta a tu amigo, quizá no lo haya dicho; y si lo ha hecho, que no lo vuelva a decir.
\par 15 Amonestar a un amigo: porque muchas veces es calumnia, y no creen todas las historias.
\par 16 Hay quien se equivoca en su palabra, pero no en su corazón; ¿Y quién es aquel que no ha ofendido con su lengua?
\par 17 Amonesta a tu prójimo antes de amenazarlo; y sin enojaros, ceded a la ley del Altísimo.
\par 18 El temor del Señor es el primer paso para ser aceptado [de él] y la sabiduría obtiene su amor.
\par 19 El conocimiento de los mandamientos del Señor es doctrina de vida; y los que hacen lo que le agrada recibirán el fruto del árbol de la inmortalidad.
\par 20 El temor del Señor es toda sabiduría; y en toda sabiduría está el cumplimiento de la ley y el conocimiento de su omnipotencia.
\par 21 Si un siervo dice a su señor: No haré lo que te place; aunque después lo hace, enoja al que le sustenta.
\par 22 El conocimiento de la maldad no es sabiduría, ni el consejo de los pecadores prudencia.
\par 23 Hay maldad y abominación; y hay un necio falto de sabiduría.
\par 24 Mejor es el que tiene poco entendimiento y teme a Dios que el que tiene mucha sabiduría y transgrede la ley del Altísimo.
\par 25 Hay una sutileza exquisita, y la misma es injusta; y hay quien se desvía para hacer aparecer el juicio; y hay sabio que en el juicio justifica.
\par 26 Hay un hombre malvado que agacha la cabeza con tristeza; pero por dentro está lleno de engaño,
\par 27 Bajando su rostro y haciendo como si no oyera: donde no es conocido, te hará daño sin que te des cuenta.
\par 28 Y si por falta de poder se le impide pecar, cuando encuentre la oportunidad, hará el mal.
\par 29 El hombre puede ser conocido por su mirada, y el inteligente por su rostro, cuando lo encuentras.
\par 30 La vestimenta del hombre, la risa excesiva y el andar muestran lo que es.

\chapter{20}

\par 1 Hay reprensión que no es agradable; además, alguno calla y es sabio.
\par 2 Es mucho mejor reprender que enojarse en secreto; y el que confiesa su falta será preservado del daño.
\par 3 ¡Qué bueno es mostrar arrepentimiento cuando eres reprendido! porque así escaparás del pecado voluntario.
\par 4 Como es el deseo de un eunuco de desvirgar a una virgen; así es el que ejecuta juicio con violencia.
\par 5 Hay uno que calla y se hace sabio, y otro que, con muchas palabrerías, se vuelve odioso.
\par 6 Algunos callan porque no tienen que responder, y otros callan sabiendo que ha llegado el momento.
\par 7 El hombre sabio se calla hasta que ve la oportunidad, pero el charlatán y el necio no hacen caso del tiempo.
\par 8 El que usa muchas palabras será aborrecido; y el que toma para sí autoridad en ello será odiado.
\par 9 Hay pecador que logra el bien en las cosas malas; y hay ganancia que se convierte en pérdida.
\par 10 Hay un regalo que no te aprovechará; y hay un don cuya recompensa es doble.
\par 11 Hay humillación a causa de la gloria; y hay quien levanta la cabeza desde una posición abatida.
\par 12 Hay quien compra mucho por poco y paga siete veces.
\par 13 El sabio se hace amado con sus palabras, pero las gracias de los necios serán derramadas.
\par 14 El regalo de un necio no te servirá de nada si lo tienes; Tampoco el envidioso por su necesidad, porque espera recibir muchas cosas por una.
\par 15 Da poco y reprende mucho; abre su boca como un pregonero; hoy presta, y mañana volverá a pedirlo: tal persona debe ser odiada por Dios y por los hombres.
\par 16 El necio dice: No tengo amigos, no tengo agradecimiento por todas mis buenas obras, y los que comen mi pan hablan mal de mí.
\par 17 ¡Cuántas veces y de cuántos serán objeto de burla! porque no sabe bien lo que es tener; y todo es uno para él como si no lo tuviera.
\par 18 Mejor es resbalar en la acera que resbalar con la lengua: así la caída de los impíos llegará pronto.
\par 19 Siempre habrá un cuento fuera de temporada en boca de los imprudentes.
\par 20 La sentencia sabia que sale de la boca de un necio será rechazada; porque no lo hablará a su debido tiempo.
\par 21 Hay quien por necesidad le impide pecar; y cuando descanse, no se perturbará.
\par 22 Hay quien, por vergüenza, destruye su propia alma, y ​​aceptando a las personas, se trastorna a sí mismo.
\par 23 Hay quien por vergüenza promete a su amigo y lo convierte en enemigo sin motivo alguno.
\par 24 La mentira es una mancha inmunda en el hombre, pero está continuamente en boca de los ignorantes.
\par 25 Mejor es un ladrón que un hombre acostumbrado a mentir; pero ambos tendrán destrucción de su herencia.
\par 26 El carácter del mentiroso es deshonroso, y su vergüenza siempre lo acompaña.
\par 27 El hombre sabio se honrará con sus palabras, y el inteligente agradará a los grandes.
\par 28 El que cultiva su tierra aumentará su montón, y el que agrada a los grandes obtendrá el perdón de su iniquidad.
\par 29 Los presentes y los regalos ciegan los ojos del sabio y tapan su boca para no poder reprender.
\par 30 La sabiduría escondida y el tesoro atesorado, ¿qué provecho hay en ambos?
\par 31 Mejor es el que oculta su necedad que el hombre que oculta su sabiduría.
\par 32 Mejor es la paciencia necesaria en la búsqueda del Señor, que el que vive sin guía.

\chapter{21}

\par 1 Hijo mío, ¿has pecado? No lo hagas más, sino pide perdón por tus pecados anteriores.
\par 2 Huye del pecado como de la cara de una serpiente: porque si te acercas demasiado, te morderá; sus dientes son como dientes de león, y matan las almas de los hombres.
\par 3 Toda iniquidad es como espada de dos filos, cuyas heridas no se pueden curar.
\par 4 El terror y la iniquidad desperdician las riquezas: así la casa de los soberbios quedará desolada.
\par 5 La oración que sale de la boca del pobre llega a los oídos de Dios, y su juicio llega pronto.
\par 6 El que aborrece la reprensión anda en el camino del pecador; pero el que teme al Señor se arrepentirá de corazón.
\par 7 El hombre elocuente es conocido de lejos y de cerca; pero el hombre prudente sabe cuándo resbala.
\par 8 El que construye su casa con dinero ajeno es como el que junta piedras para el sepulcro de su sepultura.
\par 9 La congregación de los impíos es como estopa enrollada, y su fin es una llama de fuego para destruirlos.
\par 10 El camino de los pecadores está allanado con piedras, pero al final está el abismo del infierno.
\par 11 El que guarda la ley del Señor adquiere su entendimiento; y la perfección del temor del Señor es sabiduría.
\par 12 El que no es sabio no será enseñado, pero hay sabiduría que multiplica la amargura.
\par 13 El conocimiento del sabio abundará como una inundación, y su consejo será como una fuente pura de vida.
\par 14 Las entrañas del necio son como un vaso roto, y no retendrá ningún conocimiento mientras viva.
\par 15 Si un hombre hábil oye una palabra sabia, la elogiará y añadirá algo a ella; pero al que no tiene entendimiento la oye, le desagrada y la echa a la espalda.
\par 16 Las palabras del necio son como una carga en el camino, pero la gracia se halla en los labios de los sabios.
\par 17 Consultan la boca del sabio en la congregación y meditan sus palabras en su corazón.
\par 18 Como casa destruida, así es la sabiduría para el necio; y el conocimiento del necio es como palabras sin sentido.
\par 19 La doctrina es para los necios como grilletes en los pies y como esposas en la mano derecha.
\par 20 El necio alza su voz en risa; pero un hombre sabio apenas sonríe un poco.
\par 21 La ciencia es para el sabio como un adorno de oro y como un brazalete en su brazo derecho.
\par 22 El pie del necio pronto llega a la casa de su prójimo, pero el hombre experimentado se avergüenza de él.
\par 23 El necio se asomará a la puerta de la casa, pero el bien educado se quedará afuera.
\par 24 Es grosería del hombre escuchar a la puerta, pero el sabio se entristecerá con la afrenta.
\par 25 Los labios de los que hablan dicen cosas que no les conciernen, pero las palabras de los que tienen entendimiento se pesan en la balanza.
\par 26 El corazón de los necios está en su boca, pero la boca de los sabios está en su corazón.
\par 27 Cuando el impío maldice a Satanás, maldice su propia alma.
\par 28 El chismoso contamina su propia alma y es odiado dondequiera que habita.

\chapter{22}

\par 1 El hombre perezoso es comparado con una piedra inmunda, y todos lo silban para avergonzarlo.
\par 2 El hombre perezoso es como la inmundicia del muladar: cualquiera que la levante le estrechará la mano.
\par 3 El hombre mal criado es la deshonra de su padre que lo engendró, y una hija [insensata] le nace para su pérdida.
\par 4 La hija sabia traerá herencia a su marido, pero la que vive deshonestamente es pesadez de su padre.
\par 5 La atrevida deshonra a su padre y a su marido, pero ambos la desprecian.
\par 6 La historia fuera de tiempo es como música de duelo, pero los azotes y la corrección de la sabiduría nunca son fuera de tiempo.
\par 7 Quien enseña a un necio es como quien pega un tiesto con pegamento, y como quien despierta de un sueño profundo.
\par 8 El que le cuenta una historia a un necio, le habla a uno que está dormido; cuando la haya contado, dirá: ¿Qué pasa?
\par 9 Si los hijos viven honestamente y tienen medios, cubrirán la bajeza de sus padres.
\par 10 Pero los niños, al ser altivos, por desprecio y falta de educación manchan la nobleza de sus parientes.
\par 11 Llorad por el muerto, porque ha perdido la luz; y llorad por el necio, porque le falta entendimiento; llorad poco por el muerto, porque está en reposo; pero la vida del necio es peor que la muerte.
\par 12 Siete días hacen duelo los hombres por el muerto; sino para el necio y el impío todos los días de su vida.
\par 13 No hables mucho con el necio, ni vayas con el que no tiene entendimiento; guárdate de él, no sea que tengas problemas, y nunca te contamines con sus tonterías; apártate de él, y encontrarás descanso, y Nunca te inquietes con la locura.
\par 14 ¿Qué es más pesado que el plomo? ¿Y cuál es su nombre sino tonto?
\par 15 La arena, la sal y una masa de hierro son más fáciles de soportar que un hombre sin entendimiento.
\par 16 Como la madera ceñida y atada en un edificio no se puede soltar con un temblor, así el corazón que está firme en el consejo aconsejado no temerá en ningún momento.
\par 17 Como un hermoso enlucido en la pared de una galería, es un corazón centrado en el pensamiento de la comprensión.
\par 18 Las colinas colocadas en un lugar alto nunca resistirán el viento: así el corazón temeroso en la imaginación de un necio no podrá resistir ningún temor.
\par 19 El que se pincha el ojo hará caer lágrimas, y el que se pincha el corazón, hará que se muestre su conocimiento.
\par 20 El que arroja una piedra a los pájaros los espanta, y el que reprende a su amigo rompe la amistad.
\par 21 Aunque desenvaines la espada contra tu amigo, no desesperes, porque puede haber una devolución .
\par 22 Si has abierto tu boca contra tu amigo, no temas; porque puede haber reconciliación, excepto por reproche, orgullo, revelación de secretos o herida traicionera; porque por estas cosas todo amigo se apartará.
\par 23 Sé fiel a tu prójimo en su pobreza, para que puedas regocijarte en su prosperidad; permanece firme con él en el tiempo de su angustia, para que puedas ser heredero con él de su herencia; porque una propiedad miserable no siempre es buena: ni al rico que es tonto para ser admirado.
\par 24 Como el vapor y el humo del horno van delante del fuego; tan injurioso ante la sangre.
\par 25 No me avergonzaré de defender a un amigo; ni me esconderé de él.
\par 26 Y si me sucede algún mal a causa de él, todo el que lo oiga se guardará de él.
\par 27 ¿Quién pondrá guarda delante de mi boca y sello de sabiduría sobre mis labios, para que no caiga repentinamente en ellos y mi lengua no me destruya?

\chapter{23}

\par 1 Oh Señor, Padre y Gobernador de toda mi vida, no me dejes abandonado a sus consejos, ni me dejes caer en ellos.
\par 2 ¿Quién pondrá azotes sobre mis pensamientos y la disciplina de la sabiduría sobre mi corazón? que no me perdonen mis ignorancias, y que no pasen por alto mis pecados:
\par 3 No sea que mis ignorancias crezcan y mis pecados abunden para mi destrucción, y caiga ante mis adversarios, y mi enemigo se regocije sobre mí, cuya esperanza está lejos de tu misericordia.
\par 4 Oh Señor, Padre y Dios de mi vida, no me mires con altivez, sino aleja de tus siervos la mente siempre altiva.
\par 5 Aparta de mí las vanas esperanzas y las concupiscencias, y sostendrás a quien siempre desea servirte.
\par 6 No dejes que la avidez del vientre ni la concupiscencia de la carne se apoderen de mí; y no entregues a tu siervo a una mente insolente.
\par 7 Oíd, hijos, la disciplina de la boca: el que la guarda, nunca será tomado en sus labios.
\par 8 El pecador quedará en su necedad; tanto el que habla mal como el soberbio caerán en ella.
\par 9 No acostumbres tu boca a jurar; ni te acostumbres a nombrar al Santo.
\par 10 Porque como un siervo que es continuamente golpeado no quedará sin una marca azul, así el que jura y nombra a Dios continuamente no quedará sin culpa.
\par 11 El hombre que jura mucho quedará lleno de iniquidad, y la plaga nunca se apartará de su casa; si peca, su pecado recaerá sobre él; y si no reconoce su pecado, comete una doble ofensa. : y si jura en vano, no será inocente, sino que su casa estará llena de calamidades.
\par 12 Hay una palabra que está vestida de muerte: Quiera Dios que no se encuentre en la herencia de Jacob; porque todas estas cosas estarán lejos de los piadosos, y no se revolcarán en sus pecados.
\par 13 No uses tu boca para jurar intemperantemente, porque en ello está la palabra de pecado.
\par 14 Acuérdate de tu padre y de tu madre cuando te sientes entre los grandes. No te olvides de ellos, y así te volverás loco por tu costumbre, y desearás no haber nacido, y maldecirás el día de tu nacimiento.
\par 15 El hombre que está acostumbrado a palabras oprobiosas, nunca será reformado en todos los días de su vida.
\par 16 Dos clases de hombres multiplican el pecado, y la tercera traerá ira: una mente ardiente es como un fuego ardiente, que nunca se apagará hasta que se consuma; un fornicario en el cuerpo de su carne nunca cesará hasta que haya encendió un fuego.
\par 17 Todo pan es dulce para el fornicario; no lo dejará hasta la muerte.
\par 18 El hombre que rompe el matrimonio y dice en su corazón: ¿Quién me ve? Estoy rodeado de tinieblas, los muros me cubren y nadie me ve; ¿Qué tengo que temer? el Altísimo no se acordará de mis pecados:
\par 19 Tal hombre sólo teme los ojos de los hombres, y no sabe que los ojos del Señor son diez mil veces más brillantes que el sol, contemplando todos los caminos de los hombres y considerando las partes más secretas.
\par 20 Él conocía todas las cosas antes de que fueran creadas; así también después que fueron perfeccionados los miró a todos.
\par 21 Este hombre será castigado en las calles de la ciudad, y donde no lo sepa será llevado.
\par 22 Así le sucederá a la mujer que deja a su marido y da a luz un heredero a otro.
\par 23 Porque primero desobedeció la ley del Altísimo; y en segundo lugar, ha prevaricado contra su propio marido; y en tercer lugar, se prostituyó en adulterio y tuvo hijos de otro hombre.
\par 24 La sacarán a la congregación y se interrogará a sus hijos.
\par 25 Sus hijos no echarán raíces, y sus ramas no darán fruto.
\par 26 Su memoria será maldecida y su afrenta no será borrada.
\par 27 Y los que queden sabrán que no hay nada mejor que el temor del Señor, y que no hay nada más dulce que prestar atención a los mandamientos del Señor.
\par 28 Gran gloria es seguir al Señor, y ser recibido de él es una larga vida.

\chapter{24}

\par 1 La sabiduría se alabará y se gloriará en medio de su pueblo.
\par 2 En la congregación del Altísimo abrirá su boca y triunfará ante su poder.
\par 3 Salí de la boca del Altísimo y cubrí como una nube la tierra.
\par 4 Habité en las alturas, y mi trono está en una columna de nube.
\par 5 Yo solo rodeé el circuito del cielo y caminé en el fondo del abismo.
\par 6 En las olas del mar y en toda la tierra, y en cada pueblo y nación, obtuve posesión.
\par 7 Con todos ellos busqué descanso. ¿Y en la herencia de quién permaneceré?
\par 8 Entonces el Creador de todas las cosas me dio un mandamiento, y el que me hizo hizo descansar mi tabernáculo, y dijo: Sea tu morada en Jacob, y tu herencia en Israel.
\par 9 Él me creó desde el principio, antes del mundo, y nunca fallaré.
\par 10 En el tabernáculo santo serví delante de él; y así me establecí en Sión.
\par 11 Asimismo, en la ciudad amada me dio descanso, y en Jerusalén estaba mi poder.
\par 12 Y eché raíces en un pueblo honorable, en la porción de la herencia del Señor.
\par 13 Fui exaltado como un cedro en el Líbano y como un ciprés en los montes de Hermón.
\par 14 Fui exaltado como palmera en En-gaddi, y como rosa en Jericó, como hermoso olivo en un campo agradable, y crecí como plátano junto al agua.
\par 15 Emití un olor dulce como de canela y de espárrago, y despedí un olor agradable como el de la mejor mirra, como el de gálbano, el de ónice y el de estoraque aromático, y como el humo del incienso en el tabernáculo.
\par 16 Como árbol de trementina extendí mis ramas, y mis ramas son ramas de honor y de gracia.
\par 17 Como la vid, yo produzco olor agradable, y mis flores son frutos de honor y riquezas.
\par 18 Yo soy la madre del hermoso amor, del temor, de la ciencia y de la santa esperanza; por eso, siendo eterna, soy dada a todos mis hijos que llevan su nombre.
\par 19 Venid a mí todos los que me deseáis y saciaos de mis frutos.
\par 20 Porque mi memoria es más dulce que la miel, y mi herencia más que el panal.
\par 21 Los que me coman tendrán todavía hambre, y los que me beban tendrán todavía sed.
\par 22 El que me obedece nunca será confundido, y los que trabajan para mí no harán nada malo.
\par 23 Todas estas cosas son el libro de la alianza del Dios Altísimo, la ley que Moisés ordenó como herencia a las congregaciones de Jacob.
\par 24 No desmayes para ser fuerte en el Señor; para que él os confirme, uníos a él: porque el Señor Todopoderoso es sólo Dios, y fuera de él no hay otro Salvador.
\par 25 Él llena todas las cosas con su sabiduría, como Fisón y como Tigris en el tiempo de los nuevos frutos.
\par 26 Él hace que el entendimiento crezca como el Éufrates y como el Jordán en el tiempo de la siega.
\par 27 Él hace aparecer la doctrina del conocimiento como la luz, y como Geon en el tiempo de la vendimia.
\par 28 El primer hombre no la conoció perfectamente, ni el último la descubrirá.
\par 29 Porque sus pensamientos son más que el mar, y sus consejos más profundos que el gran abismo.
\par 30 Yo también salí como arroyo de un río, y como conducto a un huerto.
\par 31 Dije: Regaré mi mejor jardín y regaré abundantemente el lecho de mi jardín. Y he aquí, mi arroyo se convirtió en río, y mi río en mar.
\par 32 Todavía haré que la doctrina brille como la mañana y enviaré su luz a lo lejos.
\par 33 Todavía derramaré la doctrina como profecía y la dejaré para todos los siglos para siempre.
\par 34 He aquí que no he trabajado sólo para mí, sino para todos los que buscan la sabiduría.

\chapter{25}

\par 1 En tres cosas fui embellecido y me puse hermoso delante de Dios y de los hombres: la unidad de los hermanos, el amor al prójimo y la unión de un hombre y una mujer.
\par 2 Tres clases de hombres aborrece mi alma y me ofende mucho su vida: el pobre soberbio, el rico mentiroso y el viejo adúltero y codicioso.
\par 3 Si nada acumulaste en tu juventud, ¿cómo podrás encontrar algo en tu edad?
\par 4 ¡Oh, qué hermoso es el juicio para las canas, y para los ancianos saber aconsejar!
\par 5 ¡Cuán hermosa es la sabiduría de los ancianos, y la inteligencia y el consejo de los hombres honrados!
\par 6 La gran experiencia es la corona de los ancianos, y el temor de Dios es su gloria.
\par 7 Nueve cosas he juzgado felices en mi corazón, y la décima la pronunciaré con mi lengua: El hombre que se alegra de sus hijos; y el que vive para ver la caída de su enemigo:
\par 8 Bienaventurado el que vive con una mujer prudente, que no se ha desviado de su lengua y que no ha servido a un hombre más indigno que él.
\par 9 Bienaventurado el que ha encontrado la prudencia y el que habla a los oídos de los que quieren oír.
\par 10 ¡Oh, cuán grande es el que encuentra sabiduría! sin embargo, no hay nadie mejor que el que teme al Señor.
\par 11 Pero el amor del Señor supera todas las cosas en iluminación: el que lo posee, ¿a quién será semejante?
\par 12 El temor del Señor es el principio de su amor, y la fe es el principio de la adhesión a él.
\par 13 Cualquier plaga, excepto la del corazón, y cualquier maldad, excepto la maldad de una mujer.
\par 14 Y cualquier aflicción, excepto la aflicción de los que me odian; y toda venganza, excepto la venganza de los enemigos.
\par 15 No hay más cabeza que la cabeza de una serpiente; y no hay ira mayor que la ira del enemigo.
\par 16 Preferiría vivir con el león y el dragón, que vivir en casa con una mujer malvada.
\par 17 La maldad de la mujer cambia su rostro y oscurece su rostro como cilicio.
\par 18 Su marido se sentará entre sus vecinos; y cuando lo oiga, suspirará amargamente.
\par 19 Toda maldad es pequeña comparada con la maldad de una mujer; que la porción del pecador caiga sobre ella.
\par 20 Como es para los pies del anciano la subida por un camino arenoso, así es la esposa llena de palabras para el hombre tranquilo.
\par 21 No tropezéis ante la belleza de una mujer, ni la deseéis por placer.
\par 22 La mujer, si mantiene a su marido, se llena de ira, de insolencia y de muchos reproches.
\par 23 La mujer malvada debilita el ánimo, se entristece el rostro y el corazón herido; la mujer que no consuela a su marido en la angustia, debilita las manos y las rodillas.
\par 24 De la mujer vino el principio del pecado, y por ella todos morimos.
\par 25 No dejéis pasar el agua; ni una mujer malvada libertad para vagabundear.
\par 26 Si ella no se va como tú quieres, córtala de tu carne, dale carta de divorcio y déjala ir.

\chapter{26}

\par 1 Bienaventurado el hombre que tiene una esposa virtuosa, porque el número de sus días será doble.
\par 2 La mujer virtuosa alegra a su marido, y él cumplirá en paz los años de su vida.
\par 3 Una buena esposa es una buena porción que les corresponderá a los que temen al Señor.
\par 4 Sea un hombre rico o pobre, si tiene un buen corazón para con el Señor, en todo momento se regocijará con un semblante alegre.
\par 5 Tres son las cosas que teme mi corazón; y por el cuarto tuve mucho miedo: la calumnia de una ciudad, la reunión de una multitud rebelde y una acusación falsa: todo esto es peor que la muerte.
\par 6 Pero tristeza y dolor de corazón es la mujer que tiene celos de otra mujer, y azote de la lengua que se comunica con todos.
\par 7 La esposa mala es un yugo que se mueve de un lado a otro; el que la agarra es como si tuviera un escorpión.
\par 8 La mujer ebria y tramposa en el extranjero provoca gran ira, y no encubre su propia vergüenza.
\par 9 La fornicación de una mujer puede verse en su mirada altiva y en sus párpados.
\par 10 Si tu hija es desvergonzada, mantenla en apuros, no sea que se abuse de sí misma con demasiada libertad.
\par 11 Vigila el ojo insolente, y no te maravilles si peca contra ti.
\par 12 Ella abrirá su boca, como el viajero sediento que encuentra una fuente, y beberá de toda agua que esté cerca de ella; se sentará junto a cada vallado y abrirá su aljaba contra cada flecha.
\par 13 La gracia de una esposa deleita a su marido, y su discreción engordará sus huesos.
\par 14 Una mujer silenciosa y amorosa es un regalo del Señor; y no hay nada que valga tanto como una mente bien instruida.
\par 15 Una mujer fiel y avergonzada es una doble gracia, y su mente continente no puede ser valorada.
\par 16 Como el sol cuando sale en lo alto del cielo; así es la belleza de una buena esposa en el orden de su casa.
\par 17 Como la luz clara sobre el candelero santo; así es la belleza del rostro en la edad madura.
\par 18 Como las columnas de oro sobre basas de plata; así son los pies bellos con un corazón constante.
\par 19 Hijo mío, conserva sana la flor de tu edad; y no des tu fuerza a los extraños.
\par 20 Cuando hayas obtenido una posesión fructífera en todo el campo, siémbrala con tu propia semilla, confiando en la bondad de tu ganado.
\par 21 Así, tu raza que dejes será engrandecida, teniendo la confianza de su buena descendencia.
\par 22 La ramera será contada como escupitajo; pero la mujer casada es torre contra la muerte de su marido.
\par 23 La mujer mala es dada en herencia al hombre malvado, pero la mujer piadosa es dada al que teme al Señor.
\par 24 La mujer deshonesta desprecia la vergüenza, pero la mujer honesta reverenciará a su marido.
\par 25 La mujer desvergonzada será contada como un perro; pero la avergonzada temerá al Señor.
\par 26 La mujer que honra a su marido será considerada sabia por todos; pero la que lo deshonra con su orgullo, será considerada impía por todos.
\par 27 Se buscará una mujer que grite y un regaño para ahuyentar a los enemigos.
\par 28 Hay dos cosas que entristecen mi corazón; y el tercero me enoja: un hombre de guerra que sufre pobreza; y hombres de entendimiento que no se dejan fijar por; y uno que vuelve de la justicia al pecado; El Señor prepara a éste para la espada.
\par 29 Difícilmente el mercader se abstendrá de hacer el mal; y el vendedor ambulante no quedará libre del pecado.

\chapter{27}

\par 1 Muchos han pecado por algo pequeño; y el que busca abundancia apartará sus ojos.
\par 2 Como un clavo que se clava entre las uniones de las piedras; así el pecado permanece cerca entre la compra y la venta.
\par 3 Si el hombre no se mantiene diligentemente en el temor del Señor, su casa pronto será derribada.
\par 4 Como cuando se tamiza con un tamiz, quedan los desechos; así la inmundicia del hombre en su charla.
\par 5 El horno prueba las vasijas del alfarero; así la prueba del hombre está en su razonamiento.
\par 6 El fruto declara si el árbol ha sido labrado; también lo es la expresión de una vanidad en el corazón del hombre.
\par 7 No alabes a nadie antes de oírle hablar; porque ésta es la prueba de los hombres.
\par 8 Si sigues la justicia, la obtendrás y la vestirás como un manto largo y glorioso.
\par 9 Las aves recurrirán a sus semejantes; así la verdad volverá a los que practican en ella.
\par 10 Como el león acecha la presa; Así pecarán los que hacen iniquidad.
\par 11 El discurso del hombre piadoso es siempre con sabiduría; pero el necio cambia como la luna.
\par 12 Si eres de los indiscretos, observa el tiempo; sino estad continuamente entre hombres de entendimiento.
\par 13 Las palabras de los necios son fastidiosas, y su diversión es la desenfrenada del pecado.
\par 14 Al que dice muchas malas palabras se le erizan los cabellos; y sus riñas hacen que uno se tape los oídos.
\par 15 La contienda de los soberbios es derramamiento de sangre, y sus injurias son dolorosas al oído.
\par 16 Quien descubre secretos pierde su crédito; y nunca encontrará un amigo para su mente.
\par 17 Ama a tu amigo y sé fiel a él; pero si traicionas sus secretos, no lo sigas más.
\par 18 Porque como un hombre destruye a su enemigo, así has ​​perdido el amor de tu prójimo.
\par 19 Como el que deja escapar un pájaro de su mano, así dejas ir a tu prójimo y no lo recuperas más.
\par 20 No lo sigáis más, porque está demasiado lejos; es como corzo escapado del lazo.
\par 21 En cuanto a la herida, se puede vendar; y después de la injuria puede haber reconciliación; pero el que traiciona los secretos no tiene esperanza.
\par 22 El que guiña los ojos hace el mal, y el que lo conoce se apartará de él.
\par 23 Cuando estés presente, hablará dulcemente y admirará tus palabras, pero al final retorcerá la boca y calumniará tus dichos.
\par 24 He odiado muchas cosas, pero ninguna como él; porque el Señor lo aborrecerá.
\par 25 Quien arroja una piedra a lo alto, la arroja sobre su propia cabeza; y el golpe engañoso causará heridas.
\par 26 El que cava un hoyo caerá en él, y el que pone una trampa será atrapado en él.
\par 27 Al que hace el mal, éste le caerá encima y no sabrá de dónde viene.
\par 28 La burla y el oprobio son de los soberbios; pero la venganza, como un león, los acechará.
\par 29 Los que se alegran de la caída del justo caerán en la trampa; y la angustia los consumirá antes de morir.
\par 30 La malicia y la ira también son abominaciones; y el hombre pecador tendrá ambas cosas.

\chapter{28}

\par 1 El que se venga, encontrará venganza del Señor y seguramente guardará sus pecados.
\par 2 Perdona a tu prójimo el daño que te ha hecho, así también te serán perdonados tus pecados cuando ores.
\par 3 ¿Un hombre odia a otro y busca el perdón del Señor?
\par 4 No tiene misericordia del hombre que es como él, ¿y pide perdón por sus propios pecados?
\par 5 Si el que es carne alberga odio, ¿quién pedirá perdón por sus pecados?
\par 6 Acuérdate de tu fin y cese la enemistad; [acordaos] de la corrupción y de la muerte, y guardad los mandamientos.
\par 7 Recuerda los mandamientos y no guardes rencor hacia tu prójimo; recuerda el pacto del Altísimo y haz un guiño a la ignorancia.
\par 8 Abstente de contiendas y disminuirás tus pecados; porque el hombre furioso encenderá contiendas,
\par 9 El hombre pecador inquieta a sus amigos y discute entre los que están en paz.
\par 10 Como es la materia del fuego, así arde; y según es la fuerza del hombre, así es su ira; y conforme a sus riquezas aumenta su ira; y cuanto más fuertes sean los que contiendan, más se inflamarán.
\par 11 La contienda apresurada enciende el fuego, y la pelea apresurada derrama sangre.
\par 12 Si soplas la chispa, arderá; si escupes sobre ella, se apagará; y ambas cosas saldrán de tu boca.
\par 13 Maldecid al murmurador y al de doble lengua, porque tales han destruido a muchos que estaban en paz.
\par 14 La lengua calumniosa inquietó a muchos y los expulsó de nación en nación; derribó ciudades fuertes y derribó las casas de los grandes.
\par 15 La lengua calumniosa expulsó a las mujeres virtuosas y las privó de sus trabajos.
\par 16 Quien la escucha nunca encontrará descanso ni habitará en paz.
\par 17 El golpe del látigo deja marcas en la carne, pero el golpe de la lengua quebranta los huesos.
\par 18 Muchos han caído a filo de espada, pero no tantos como los que han caído por la lengua.
\par 19 Bienaventurado el que se defiende de su veneno; que no ha arrastrado su yugo, ni ha estado atado con sus ataduras.
\par 20 Porque su yugo es yugo de hierro, y sus ataduras, ataduras de bronce.
\par 21 Su muerte es una muerte mala; mejor sería el sepulcro.
\par 22 No se enseñoreará de los que temen a Dios, ni serán quemados con su llama.
\par 23 Los que abandonan al Señor caerán en él; y arderá en ellos, y no se apagará; será enviado sobre ellos como un león y como un leopardo los devorará.
\par 24 Mira, cerca tu posesión con espinos y ata tu plata y tu oro,
\par 25 Y pesa tus palabras en una balanza, y haz una puerta y un cerrojo para tu boca.
\par 26 Cuídate de no deslizarte por ella, no sea que caigas delante del que acecha.

\chapter{29}

\par 1 El que es misericordioso prestará a su prójimo; y el que fortalece su mano guarda los mandamientos.
\par 2 Presta a tu prójimo cuando lo necesite, y págale a tu prójimo a su debido tiempo.
\par 3 Mantén tu palabra y sé fiel a él, y siempre encontrarás lo que necesitas.
\par 4 Muchos, cuando se les prestaba algo, lo consideraban encontrado y molestaban a los que les ayudaban.
\par 5 Hasta que haya recibido, besará la mano del hombre; y por el dinero de su prójimo hablará sumisamente; pero cuando deba pagar, prolongará el tiempo, devolverá palabras de dolor y se quejará del tiempo.
\par 6 Si gana, apenas recibirá la mitad y contará como si la hubiera encontrado; si no, le habrá privado de su dinero y le habrá ganado un enemigo sin motivo: le pagará con maldiciones y rejas; y por honor le pagará deshonra.
\par 7 Por eso muchos se negaron a prestar dinero para las malas acciones de otros, por temor a ser defraudados.
\par 8 Sin embargo, ten paciencia con el hombre pobre y no tardes en mostrarle misericordia.
\par 9 Ayudad al pobre por causa del mandamiento, y no lo rechacéis a causa de su pobreza.
\par 10 Pierde tu dinero por tu hermano y por tu amigo, y no dejes que se oxide debajo de una piedra para perderse.
\par 11 Haz tu tesoro según los mandamientos del Altísimo, y te reportará más beneficios que el oro.
\par 12 Guarda la limosna en tus almacenes, y te librará de toda aflicción.
\par 13 Mejor luchará a tu favor contra tus enemigos que un escudo poderoso y una lanza fuerte.
\par 14 El hombre honesto es fiador de su prójimo, pero el imprudente lo abandonará.
\par 15 No olvides la amistad de tu fiador, porque él ha dado su vida por ti.
\par 16 El pecador arruinará el buen estado de su fianza:
\par 17 Y el de mente ingrata dejará al que lo libró.
\par 18 La fianza destruyó a muchos de buena posición y los sacudió como las olas del mar; a los valientes los arrojó de sus casas, de modo que vagaron entre naciones extrañas.
\par 19 El hombre malvado que transgrede los mandamientos del Señor caerá en fianza; y el que emprende y sigue negocios ajenos para obtener ganancias, caerá en juicio.
\par 20 Ayuda a tu prójimo según tus fuerzas, y ten cuidado de no caer tú mismo en lo mismo.
\par 21 Lo principal para la vida es agua, pan, vestido y una casa para cubrir la vergüenza.
\par 22 Mejor es la vida de un hombre pobre en una humilde cabaña, que una comida delicada en la casa de otro.
\par 23 Sea poco o mucho, mantente contento y no oirás el oprobio de tu casa.
\par 24 Porque es una vida miserable andar de casa en casa; porque donde eres forastero, no te atreves a abrir la boca.
\par 25 Celebrarás y festejarás, y no darás gracias; además, oirás palabras amargas:
\par 26 Ven, extranjero, y prepárame una mesa, y dame de comer lo que tienes preparado.
\par 27 Da lugar, extraño, a un hombre honorable; Mi hermano viene a dormir, y tengo necesidad de mi casa.
\par 28 Estas cosas son dolorosas para el hombre inteligente; la reprensión del cuarto de la casa y el reproche del prestamista.

\chapter{30}

\par 1 El que ama a su hijo, le hace sentir la vara muchas veces, para que al final se regocije con él.
\par 2 El que castiga a su hijo se alegrará con él y se regocijará con él entre sus conocidos.
\par 3 El que enseña a su hijo entristece al enemigo, y delante de sus amigos se alegrará de él.
\par 4 Aunque su padre muera, él es como si no estuviera muerto, porque ha dejado detrás de sí a uno que es igual a él.
\par 5 Mientras vivió, lo vio y se regocijó en él; y cuando murió, no se entristeció.
\par 6 Dejó detrás de sí un vengador contra sus enemigos, y uno que recompensará con bondad a sus amigos.
\par 7 El que menosprecia a su hijo vendará sus heridas; y sus entrañas se estremecerán con cada llanto.
\par 8 El caballo que no se doma se vuelve testarudo, y el niño abandonado se vuelve obstinado.
\par 9 Encorva a tu hijo y te asustará; juega con él y te entristecerá.
\par 10 No te rías con él, no sea que tengas tristeza con él, y no sea que al final rechinarás tus dientes.
\par 11 No le des libertad en su juventud ni le hagas un guiño a sus locuras.
\par 12 Inclina su cuello mientras es joven y golpéalo en los costados cuando es niño, para que no se obstine y te desobedezca y traiga así tristeza a tu corazón.
\par 13 Castiga a tu hijo y haz que trabaje, para que su conducta lasciva no te sea una ofensa.
\par 14 Mejor es el pobre, sano y fuerte de constitución, que el rico, afligido en su cuerpo.
\par 15 La salud y el buen estado del cuerpo están por encima de todo el oro, y un cuerpo fuerte por encima de las riquezas infinitas.
\par 16 No hay riqueza más que un cuerpo sano, ni alegría más que la alegría del corazón.
\par 17 Mejor es la muerte que una vida amarga o una enfermedad continua.
\par 18 Las delicias derramadas sobre la boca cerrada son como platos de carne puestos sobre una tumba.
\par 19 ¿De qué sirve la ofrenda a un ídolo? porque no puede comer ni oler: así es el que es perseguido por el Señor.
\par 20 Él ve con sus ojos y gime, como un eunuco que abraza a una virgen y suspira.
\par 21 No entregues tu mente a la tristeza, ni te aflijas en tus propios consejos.
\par 22 La alegría del corazón es la vida del hombre, y la alegría del hombre prolonga sus días.
\par 23 Ama tu alma y consuela tu corazón, aleja de ti la tristeza, porque la tristeza ha matado a muchos, y no hay provecho en ella.
\par 24 La envidia y la ira acortan la vida, y el cuidado hace envejecer antes de tiempo.
\par 25 El corazón alegre y bueno cuidará de su comida y de su alimentación.

\chapter{31}

\par 1 La búsqueda de las riquezas consume la carne, y el afán por ellas ahuyenta el sueño.
\par 2 La vigilancia no deja dormir al hombre, como una enfermedad que perturba el sueño.
\par 3 El rico tiene mucho trabajo para reunir riquezas; y cuando descansa, se llena de sus delicadezas.
\par 4 El pobre trabaja en su pobreza; y cuando termina, todavía está necesitado.
\par 5 El que ama el oro no será justificado, y el que sigue la corrupción se saciará de él.
\par 6 El oro fue la ruina de muchos, y su destrucción estaba presente.
\par 7 Es una piedra de tropiezo para los que le ofrecen sacrificios, y todo necio quedará atrapado en ella.
\par 8 Bienaventurado el rico que se encuentra sin defecto y no va tras el oro.
\par 9 ¿Quién es él? y lo llamaremos bienaventurado, porque ha hecho maravillas en medio de su pueblo.
\par 10 ¿Quién fue probado por ello y hallado perfecto? entonces que se gloríe. ¿Quién pudo ofender y no ha ofendido? ¿O has hecho el mal y no lo has hecho?
\par 11 Sus bienes serán establecidos y la congregación declarará sus limosnas.
\par 12 Si te sientas a una mesa generosa, no seas codicioso ni digas: Hay mucha comida en ella.
\par 13 Acordaos de que el ojo malvado es algo malo: ¿y qué hay creado más malvado que el ojo? por eso llora en cada ocasión.
\par 14 No extiendas tu mano hacia donde mire, ni la metas con ella en el plato.
\par 15 No juzgues a tu prójimo por ti mismo, y sé discreto en todo.
\par 16 Come como es propio de un hombre lo que te ponen delante; y devora la nota, para que no seas odiado.
\par 17 Déjalo primero por cortesía; y no seas insaciable, para que no ofendas.
\par 18 Cuando te sientes entre muchos, no extiendas la mano el primero de todos.
\par 19 Al hombre bien alimentado le basta muy poco y no se queda sin aliento en la cama.
\par 20 El sueño profundo viene de una comida moderada; se levanta temprano y su ingenio está con él; pero el dolor de la vigilia, la cólera y los dolores de estómago son del hombre insaciable.
\par 21 Y si te han obligado a comer, levántate, sal, vomita y descansarás.
\par 22 Hijo mío, escúchame y no me desprecies, y al final encontrarás lo que te dije: sé rápido en todas tus obras, para que no te sobrevenga ninguna enfermedad.
\par 23 El que es generoso con su comida, hablarán bien de él; y se creerá el informe de su buen gobierno.
\par 24 Pero contra el que escatima en su comida, toda la ciudad murmurará; y los testimonios de su avaricia no serán dudosos.
\par 25 No muestres tu valentía en el vino; porque el vino ha destruido a muchos.
\par 26 El horno demuestra su filo con la inmersión, y el vino con la embriaguez el corazón de los soberbios.
\par 27 El vino es para el hombre como la vida, si se bebe con moderación. ¿Qué es, pues, la vida para el hombre que no tiene vino? porque fue hecho para alegrar a los hombres.
\par 28 El vino, bebido considerablemente y a su sazón, alegra el corazón y alegra la mente.
\par 29 Pero el vino borracho en exceso produce amargura del ánimo, con riñas y riñas.
\par 30 La ebriedad aumenta la ira del necio hasta el punto de ofender; disminuye las fuerzas y produce heridas.
\par 31 No reprendas a tu prójimo a la hora del vino, ni lo desprecies en su alegría; no le des palabras despectivas ni lo presiones instándolo [a beber].

\chapter{32}

\par 1 Si te hacen maestro [de un banquete], no te enaltezcas, sino sé entre ellos como uno de los demás; Cuídalos con diligencia y siéntate.
\par 2 Y cuando hayas cumplido con todo tu oficio, toma tu lugar para divertirte con ellos y recibir una corona por haber organizado bien la fiesta.
\par 3 Habla, tú que eres mayor, porque te conviene, pero con buen juicio; y no estorbes la música.
\par 4 No pronuncies palabras donde hay músico, ni muestres sabiduría fuera de tiempo.
\par 5 Un concierto de música en un banquete de vino es como un sello de carbunclo engastado en oro.
\par 6 Como el sello de la esmeralda engastada en una pieza de oro, así es la melodía de la música con el vino agradable.
\par 7 Habla, joven, si es necesario, pero apenas cuando te lo pidan dos veces.
\par 8 Que tu discurso sea breve, comprendiendo mucho en pocas palabras; sé como el que sabe y, sin embargo, se calla.
\par 9 Si estás entre los grandes, no te hagas igual a ellos; y cuando los hombres antiguos estén en su lugar, no uses muchas palabras.
\par 10 Antes del trueno sale el relámpago; y delante del hombre avergonzado irá el favor.
\par 11 Levántate temprano y no seas el último; pero vete a casa sin demora.
\par 12 Allí disfruta y haz lo que quieras, pero no peques con palabras soberbias.
\par 13 Y por estas cosas, bendice al que te hizo y te colmó con sus bienes.
\par 14 Quien teme al Señor recibirá su disciplina; y los que lo buscan temprano encontrarán favor.
\par 15 El que busca la ley se saciará de ella, pero el hipócrita se escandalizará con ella.
\par 16 Los que temen al Señor encontrarán el juicio y encenderán la justicia como una luz.
\par 17 El hombre pecador no será reprendido, sino que encontrará una excusa según su voluntad.
\par 18 El hombre de consejo será considerado; pero el hombre extraño y orgulloso no se deja intimidar por el temor, incluso cuando de sí mismo ha obrado sin consejo.
\par 19 No hagas nada sin consejo; y una vez que lo hayas hecho, no te arrepientas.
\par 20 No vayas por camino en el que puedas caer, ni tropieces entre las piedras.
\par 21 No os confiéis abiertamente.
\par 22 Y ten cuidado con tus propios hijos.
\par 23 En toda buena obra confía en tu propia alma; porque esto es guardar los mandamientos.
\par 24 El que cree en el Señor guarda el mandamiento; y al que en él confía, nunca le irá peor.

\chapter{33}

\par 1 Al que teme al Señor no le sucederá ningún mal; pero aún en la tentación lo librará.
\par 2 El sabio no aborrece la ley; pero el que en esto es hipócrita es como un barco en la tormenta.
\par 3 El hombre inteligente confía en la ley; y la ley le es fiel, como un oráculo.
\par 4 Prepara lo que vas a decir y así serás escuchado; encuaderna la instrucción y luego responde.
\par 5 El corazón de los necios es como una rueda de carreta; y sus pensamientos son como eje rodante.
\par 6 El caballo semental es como un amigo burlón; relincha debajo de todo el que lo monta.
\par 7 ¿Por qué un día es mayor que otro, cuando toda la luz de cada día del año proviene del sol?
\par 8 Por el conocimiento del Señor se distinguieron, y él alteró las estaciones y las fiestas.
\par 9 A algunos de ellos les hizo días solemnes y los santificó, ya otros les hizo días ordinarios.
\par 10 Y todos los hombres proceden de la tierra, y Adán fue creado de la tierra.
\par 11 Con mucha ciencia, el Señor los dividió y diversificó sus caminos.
\par 12 A algunos de ellos los bendijo y los exaltó, a otros los santificó y los puso cerca de sí, pero a otros los maldijo, los humilló y los desplazó de sus lugares.
\par 13 Como el barro está en la mano del alfarero para moldearlo a su gusto, así el hombre está en la mano de quien lo hizo, para dárselo como mejor le parezca.
\par 14 El bien se opone al mal, y la vida a la muerte; así el piadoso al pecador, y el pecador al piadoso.
\par 15 Mirad, pues, todas las obras del Altísimo; y son dos y dos, uno contra otro.
\par 16 Me desperté el último de todos, como el que vendimia tras los vendimiadores; con la bendición del Señor aproveché y pisé mi lagar como un vendimiador.
\par 17 Considera que no trabajé sólo por mí, sino por todos los que buscan conocimiento.
\par 18 Oídme, oh grandes hombres del pueblo, y escuchad con vuestros oídos, gobernantes de la congregación.
\par 19 No le des poder sobre ti a tu hijo y a tu mujer, a tu hermano y a tu amigo mientras vivas, ni des tus bienes a otro, no sea que te arrepientas y vuelvas a implorar lo mismo.
\par 20 Mientras vivas y tengas aliento en ti, no te entregues a nadie.
\par 21 Porque mejor es que tus hijos te busquen a ti, que que tú obedezcas a su cortesía.
\par 22 En todas tus obras guarda para ti la preeminencia; No dejes ni una mancha en tu honor.
\par 23 En el momento en que termines tus días y termines tu vida, reparte tu herencia.
\par 24 El forraje, la vara y las cargas son para el asno; y pan, corrección y trabajo, para un siervo.
\par 25 Si pones a tu siervo a trabajar, encontrarás descanso; pero si lo dejas ocioso, buscará la libertad.
\par 26 El yugo y el collar doblan el cuello, así son los tormentos y los tormentos para el siervo malo.
\par 27 Envíalo a trabajar para que no esté ocioso; porque la ociosidad enseña muchos males.
\par 28 Ponlo a trabajar como le conviene; si no es obediente, ponle más grillos pesados.
\par 29 Pero no seáis excesivos con nadie; y sin discreción no hagáis nada.
\par 30 Si tienes un siervo, déjalo ser para ti como a ti mismo, porque lo has comprado por precio.
\par 31 Si tienes un siervo, trátalo como a un hermano, porque lo necesitas como a tu propia alma. Si le infliges mal y él huye de ti, ¿por qué camino irás a buscarlo?

\chapter{34}

\par 1 Las esperanzas del hombre falto de entendimiento son vanas y falsas; y los sueños enaltecen a los necios.
\par 2 Quien se fija en los sueños es como el que se fija en una sombra y sigue el viento.
\par 3 La visión de los sueños es la semejanza de una cosa con otra, como la semejanza de un rostro con un rostro.
\par 4 ¿De lo inmundo, qué se puede limpiar? ¿Y de lo que es falso, qué verdad puede venir?
\par 5 Las adivinaciones, las adivinaciones y los sueños son vanos, y el corazón imagina como el corazón de una mujer que está de parto.
\par 6 Si no son enviados por el Altísimo en tu visita, no pongas tu corazón en ellos.
\par 7 Porque los sueños engañaron a muchos, y fracasaron los que en ellos confiaban.
\par 8 La ley será perfecta y sin mentira, y la sabiduría es perfección para la boca fiel.
\par 9 El hombre que ha viajado sabe muchas cosas; y el que tiene mucha experiencia declarará sabiduría.
\par 10 El que no tiene experiencia sabe poco, pero el que ha viajado está lleno de prudencia.
\par 11 Cuando viajaba, vi muchas cosas; y entiendo más de lo que puedo expresar.
\par 12 Muchas veces estuve en peligro de muerte, pero por estas cosas fui librado.
\par 13 El espíritu de los que temen al Señor vivirá; porque su esperanza está en aquel que los salva.
\par 14 Quien teme al Señor, no temerá ni tendrá miedo; porque él es su esperanza.
\par 15 Bienaventurada el alma del que teme al Señor: ¿a quién mirará? ¿Y quién es su fuerza?
\par 16 Porque los ojos del Señor están sobre los que lo aman; él es su poderosa protección y su fuerte apoyo, una defensa contra el calor y un amparo contra el sol al mediodía, una protección contra los tropiezos y un apoyo contra las caídas.
\par 17 Él levanta el alma e ilumina los ojos; da salud, vida y bendición.
\par 18 El que sacrifica algo mal adquirido, su ofrenda es ridícula; y no se aceptan dádivas de hombres injustos.
\par 19 Al Altísimo no le agradan las ofrendas de los impíos; ni es apaciguado por el pecado con la multitud de sacrificios.
\par 20 Cualquiera que presente una ofrenda de los bienes de los pobres es como quien mata a su hijo ante los ojos de su padre.
\par 21 El pan del pobre es su vida; el que lo defrauda es un hombre de sangre.
\par 22 El que quita el sustento a su prójimo, lo mata; y el que defrauda al trabajador en su salario es un derramador de sangre.
\par 23 Cuando uno construye y otro derriba, ¿qué provecho obtienen sino con el trabajo?
\par 24 Cuando uno ora y otro maldice, ¿la voz de quién escuchará el Señor?
\par 25 El que se lava después de tocar un cadáver, si lo vuelve a tocar, ¿de qué le sirve lavarse?
\par 26 Así sucede con el hombre que ayuna por sus pecados y vuelve a hacer lo mismo: ¿quién escuchará su oración? ¿O de qué le aprovecha su humillación?

\chapter{35}

\par 1 El que guarda la ley trae ofrendas suficientes; el que observa el mandamiento ofrece ofrendas de paz.
\par 2 El que da una buena recompensa ofrece flor de harina; y el que da limosna sacrifica alabanza.
\par 3 Apartarse de la maldad es algo que agrada al Señor; y abandonar la injusticia es una propiciación.
\par 4 No aparecerás vacío ante el Señor.
\par 5 Porque todas estas cosas [deben hacerse] por causa del mandamiento.
\par 6 La ofrenda de los justos engorda el altar, y su olor grato está delante del Altísimo.
\par 7 El sacrificio del justo es aceptable y su memoria nunca será olvidada.
\par 8 Honra al Señor con buenos ojos y no menosprecies las primicias de tus manos.
\par 9 En todas tus ofrendas muestra un rostro alegre y dedica tus diezmos con alegría.
\par 10 Da al Altísimo según él te haya enriquecido; y lo que hayas recibido, dalo con ojos alegres.
\par 11 Porque el Señor te recompensará y te dará siete veces más.
\par 12 No penséis corromper con regalos; porque tales no los recibirá; y no confíe en sacrificios injustos; porque el Señor es juez, y en él no hay acepción de personas.
\par 13 No aceptará a nadie contra el pobre, sino que escuchará la oración de los oprimidos.
\par 14 No despreciará las súplicas del huérfano; ni la viuda, cuando derrama su queja.
\par 15 ¿No corren las lágrimas por las mejillas de la viuda? ¿Y no es su grito contra él el que los hace caer?
\par 16 El que sirve al Señor será aceptado con favor, y su oración llegará hasta las nubes.
\par 17 La oración del humilde atraviesa las nubes, y hasta que llegue, no encontrará consuelo; y no se apartará, hasta que el Altísimo mire para juzgar con justicia y ejecutar juicio.
\par 18 Porque el Señor no se aflojará, ni el Poderoso será paciente con ellos, hasta que haya destrozado los lomos de los inmisericordes y haya pagado venganza a las naciones; hasta que quite la multitud de los soberbios y rompa el cetro de los injustos;
\par 19 Hasta que haya pagado a cada uno según sus obras, y a las obras de los hombres según sus designios; hasta que juzgue la causa de su pueblo y los haga regocijarse en su misericordia.
\par 20 La misericordia es oportuna en tiempo de aflicción, como la nube de lluvia en tiempo de sequía.

\chapter{36}

\par 1 Ten piedad de nosotros, Señor Dios de todos, y míranos:
\par 2 Y envía tu temor sobre todas las naciones que no te buscan.
\par 3 Levanta tu mano contra las naciones extrañas, y déjales ver tu poder.
\par 4 Como fuiste santificado en nosotros delante de ellos, así sé engrandecido entre ellos delante de nosotros.
\par 5 Y que te conozcan, como nosotros te conocemos, que no hay más Dios que tú, oh Dios.
\par 6 Haz nuevos signos y haz otras maravillas extrañas: glorifica tu mano y tu brazo derecho, para que expongan tus maravillas.
\par 7 Levanta ira y derrama ira; elimina al adversario y destruye al enemigo.
\par 8 Aprovecha el tiempo, recuerda el pacto y anuncia tus maravillas.
\par 9 El que escape será consumido por la furia del fuego; y perezcan los que oprimen al pueblo.
\par 10 Destroza la cabeza de los gobernantes de las naciones, que dicen: No hay otro fuera de nosotros.
\par 11 Reúne a todas las tribus de Jacob y heredalas como al principio.
\par 12 Oh Señor, ten piedad del pueblo que lleva tu nombre y de Israel, a quien pusiste el nombre de tu primogénito.
\par 13 Ten misericordia de Jerusalén, tu ciudad santa, el lugar de tu descanso.
\par 14 Llena a Sión con tus inefables palabras y a tu pueblo con tu gloria.
\par 15 Da testimonio a los que has poseído desde el principio y suscita profetas que han estado en tu nombre.
\par 16 Recompensa a los que esperan en ti, y que tus profetas sean hallados fieles.
\par 17 Oye, Señor, la oración de tus siervos, conforme a la bendición de Aarón sobre tu pueblo, para que todos los habitantes de la tierra sepan que tú eres el Señor, el Dios eterno.
\par 18 El vientre devora todos los alimentos, pero un alimento es mejor que otro.
\par 19 Como el paladar saborea diversas clases de carne de venado, así también el corazón entendido en palabras falsas.
\par 20 El corazón perverso entristece, pero el hombre experimentado le recompensará.
\par 21 Una mujer recibe a todo hombre, pero una hija es mejor que otra.
\par 22 La belleza de la mujer alegra el rostro, y el hombre no ama nada mejor.
\par 23 Si en su lengua hay bondad, mansedumbre y consuelo, entonces su marido no es como los demás hombres.
\par 24 El que toma esposa comienza una posesión, una ayuda como él mismo y una columna de descanso.
\par 25 Donde no hay cerco, allí se pierde la posesión; y el que no tiene esposa, vagará de un lado a otro enlutado.
\par 26 ¿Quién confiará en un ladrón bien armado que va de ciudad en ciudad? Así, ¿quién creerá al hombre que no tiene casa y se aloja dondequiera que le lleve la noche?

\chapter{37}

\par 1 Todo amigo dice: Yo también soy su amigo; pero hay un amigo que sólo lo es de nombre.
\par 2 ¿No es un dolor de muerte cuando un compañero y amigo se convierte en enemigo?
\par 3 ¡Oh imaginación malvada! ¿De dónde vienes para cubrir la tierra con engaño?
\par 4 Hay un compañero que se alegra de la prosperidad de su amigo, pero en el momento de la angustia se opone a él.
\par 5 Hay un compañero que ayuda a su amigo en el vientre y toma el escudo contra el enemigo.
\par 6 No olvides a tu amigo en tu mente, ni te olvides de él en tus riquezas.
\par 7 Todo consejero ensalza el consejo; pero hay algunos que se aconsejan a sí mismos.
\par 8 Cuídate del consejero y averigua de antemano qué necesidad tiene; porque él mismo se aconsejará; para que no te eche la suerte,
\par 9 Y te dirá: Tu camino es bueno; y después se pondrá del otro lado para ver qué te acontece.
\par 10 No consultes con quien sospecha de ti, y oculta tu consejo a los que te envidian.
\par 11 Ni consultes con una mujer que la toca y tiene celos; ni con cobarde en materia de guerra; ni con un comerciante en materia de cambio; ni con comprador de venta; ni con un hombre envidioso y agradecido; ni con un hombre despiadado que toca la bondad; ni con los perezosos para ningún trabajo; ni con un asalariado por un año de trabajo acabado; ni con un criado ocioso de muchos negocios: no escuches a éstos en ningún asunto de consejo.
\par 12 Más bien, sé siempre con un hombre piadoso, de quien sabes que guarda los mandamientos del Señor, cuyo pensamiento es conforme al tuyo, y que se entristecerá contigo si abortas.
\par 13 Y mantén firme el consejo de tu corazón, porque no hay nadie más fiel a ti que él.
\par 14 Porque la mente de un hombre a veces suele decirle más que siete centinelas sentados arriba en una torre alta.
\par 15 Y sobre todo rogad al Altísimo, para que él guíe vuestro camino con verdad.
\par 16 Dejemos que la razón preceda a toda empresa y el consejo a toda acción.
\par 17 El semblante es señal de cambio del corazón.
\par 18 Hay cuatro tipos de cosas que aparecen: el bien y el mal, la vida y la muerte, pero la lengua las domina continuamente.
\par 19 Hay alguien que es sabio y enseña a muchos, pero a sí mismo no le resulta provechoso.
\par 20 Hay quien hace gala de sabiduría con palabras, y es odiado: se quedará sin todo alimento.
\par 21 Porque la gracia del Señor no le es concedida, porque está privado de toda sabiduría.
\par 22 Otro es sabio consigo mismo; y los frutos del entendimiento son loables en su boca.
\par 23 El hombre sabio instruye a su pueblo; y los frutos de su entendimiento no faltan.
\par 24 El hombre sabio será colmado de bendiciones; y todos los que le vean le tendrán por feliz.
\par 25 Los días de la vida del hombre pueden estar contados, pero los días de Israel son innumerables.
\par 26 El hombre sabio heredará gloria entre su pueblo y su nombre será perpetuo.
\par 27 Hijo mío, prueba tu alma en tu vida, y ve lo que le es malo, y no se lo des.
\par 28 Porque no todas las cosas son provechosas para todos, ni cada alma se complace en todo.
\par 29 No seas insaciable con los manjares, ni codiciosos de manjares.
\par 30 Porque el exceso de alimentos trae enfermedades, y el exceso se convertirá en cólera.
\par 31 Por la hartazgo muchos perecieron; pero el que tiene cuidado prolongará su vida.

\chapter{38}

\par 1 Honrad al médico con el honor que le corresponde por los usos que de él podáis tener, porque el Señor lo ha creado.
\par 2 Porque del Altísimo viene la curación y recibirá la honra del rey.
\par 3 La habilidad del médico levantará su cabeza, y ante los grandes hombres quedará admirado.
\par 4 El Señor ha creado medicinas de la tierra; y el sabio no los aborrecerá.
\par 5 ¿No se endulzó el agua con leña, para que se conociera su virtud?
\par 6 Y ha dado a los hombres habilidad para que sean honrados por sus maravillas.
\par 7 Con tales sana [a los hombres] y les quita sus dolores.
\par 8 De ellos hace el boticario un dulce; y sus obras no tienen fin; y de él viene la paz sobre toda la tierra,
\par 9 Hijo mío, en tu enfermedad no seas negligente, sino ora al Señor y él te sanará.
\par 10 Deja el pecado, ordena tus manos y limpia tu corazón de toda maldad.
\par 11 Dad olor grato y recuerdo de flor de harina; y haréis ofrenda engordada, como si no existiera.
\par 12 Entonces deja lugar al médico, porque el Señor lo ha creado; no dejes que se aleje de ti, porque lo necesitas.
\par 13 Hay un tiempo en que en sus manos está el buen éxito.
\par 14 Porque también orarán al Señor para que prospere lo que dan como alivio y remedio para prolongar la vida.
\par 15 El que peque delante de su Hacedor, caiga en manos del médico.
\par 16 Hijo mío, deja que las lágrimas caigan sobre los muertos y comienza a lamentarte, como si tú mismo hubieras sufrido un gran daño; y luego cubre su cuerpo según la costumbre, y no descuidas su entierro.
\par 17 Llora amargamente, y haz grandes gemidos y lamenta, como es digno, y eso por un día o dos, para que no hablen mal de ti; y luego consuélate de tu tristeza.
\par 18 Porque de la tristeza viene la muerte, y la tristeza del corazón quebranta las fuerzas.
\par 19 También en la aflicción permanece el dolor, y la vida de los pobres es la maldición del corazón.
\par 20 No te preocupes por la tristeza: ahuyentala y recuerda el último fin.
\par 21 No lo olvides, porque ya no hay vuelta atrás: no le harás ningún bien, sino que te hará daño a ti mismo.
\par 22 Acuérdate de mi juicio, porque el tuyo también será así; ayer por mí, y hoy por ti.
\par 23 Cuando el muerto esté en reposo, que descanse su memoria; y sed consolados por él, cuando su Espíritu se aparte de él.
\par 24 La sabiduría del hombre sabio llega con la oportunidad del tiempo libre; y el que tiene pocos negocios se volverá sabio.
\par 25 ¿Cómo puede adquirir sabiduría el que empuña el arado, el que se regocija en el aguijón, el que conduce bueyes y se ocupa en sus trabajos, y cuyo habla de bueyes?
\par 26 Se propone hacer surcos; y es diligente en dar forraje a las vacas.
\par 27 Así, todo carpintero y maestro de obra que trabaja de día y de noche, y los que tallan y graban sellos, y los que se esmeran en hacer gran variedad, y se dedican a la falsificación de imágenes, y velan para terminar la obra,
\par 28 También el herrero sentado junto al yunque, y considerando los trabajos de hierro, el vapor del fuego consume su carne, y pelea con el calor del horno; el ruido del martillo y del yunque está siempre en sus oídos, y sus ojos todavía miran el modelo de lo que hace; se propone terminar su trabajo y vela para pulirlo perfectamente:
\par 29 Así hace el alfarero sentado en su trabajo y haciendo girar la rueda con los pies, el que está siempre cuidadosamente ocupado en su trabajo y hace todo su trabajo por número;
\par 30 Con su brazo moldea el barro, y ante sus pies inclina su fuerza; se aplica a guiarlo; y es diligente en limpiar el horno:
\par 31 Todos ellos confían en sus manos, y cada uno es sabio en su trabajo.
\par 32 Sin ellos no puede habitarse una ciudad, y no habitarán donde quieran, ni andarán de arriba a abajo.
\par 33 No serán buscados en el consejo público, ni ocuparán puestos altos en la congregación; no se sentarán en el tribunal, ni entenderán la sentencia del juicio; no podrán declarar justicia y juicio; y no se encontrarán donde se hablan parábolas.
\par 34 Pero ellos mantendrán el estado del mundo y [todo] su deseo estará en el trabajo de su oficio.

\chapter{39}

\par 1 Pero aquel que se concentra en la ley del Altísimo y se ocupa en meditarla, buscará la sabiduría de todos los antiguos y se ocupará en profecías.
\par 2 Él guardará las palabras de los hombres famosos; y donde haya parábolas sutiles, allí también estará.
\par 3 Buscará los secretos de las sentencias graves y se familiarizará con parábolas oscuras.
\par 4 Servirá entre los grandes y se presentará ante los príncipes; viajará por países extraños; porque él ha probado el bien y el mal entre los hombres.
\par 5 Dará su corazón para acudir temprano al Señor que lo creó, orará ante el Altísimo, abrirá su boca en oración y rogará por sus pecados.
\par 6 Cuando el gran Señor quiera, se llenará de espíritu de inteligencia; pronunciará sabias sentencias y dará gracias al Señor en su oración.
\par 7 Él dirigirá su consejo y su conocimiento, y meditará en sus secretos.
\par 8 Manifestará lo que ha aprendido y se gloriará en la ley de la alianza del Señor.
\par 9 Muchos alabarán su inteligencia; y mientras el mundo dure, no será borrado; su memoria no desaparecerá, y su nombre vivirá de generación en generación.
\par 10 Diez naciones proclamarán su sabiduría y la congregación proclamará sus alabanzas.
\par 11 Si muere, dejará un nombre mayor que mil; y si vive, lo aumentará.
\par 12 Aún tengo más que decir sobre lo que he pensado; porque estoy lleno como la luna llena.
\par 13 Oídme, hijos santos, y brotad como una rosa que crece junto al arroyo del campo.
\par 14 Y dad un olor suave como el incienso, y floreced como un lirio, exhalad un olor y cantad un cántico de alabanza, bendecid al Señor en todas sus obras.
\par 15 Engrandeced su nombre y proclamad sus alabanzas con los cánticos de vuestros labios y con arpas, y al alabarle diréis de esta manera:
\par 16 Todas las obras del Señor son sumamente buenas, y todo lo que él ordena se cumplirá a su debido tiempo.
\par 17 Y nadie podrá decir: ¿Qué es esto? ¿por qué es eso? porque en el momento oportuno todos serán buscados: a su orden las aguas se amontonaban, y a las palabras de su boca los receptáculos de aguas.
\par 18 Por orden suya se hace todo lo que le place; y nadie puede impedir, cuando él salvará.
\par 19 Las obras de toda carne están ante él, y nada se puede ocultar a sus ojos.
\par 20 Él ve desde la eternidad hasta la eternidad; y no hay nada maravilloso delante de él.
\par 21 El hombre no necesita decir: ¿Qué es esto? ¿por qué es eso? porque él hizo todas las cosas para su uso.
\par 22 Su bendición cubrió la tierra seca como un río y la regó como una inundación.
\par 23 Así como él transformó las aguas en sal, así heredarán las naciones su ira.
\par 24 Como sus caminos son claros para los santos; así son tropiezos para los impíos.
\par 25 Porque los buenos son cosas buenas creadas desde el principio, y las cosas malas para los pecadores.
\par 26 Las cosas principales para el uso total de la vida del hombre son el agua, el fuego, el hierro, la sal, la harina de trigo, la miel, la leche, la sangre de las uvas, el aceite y los vestidos.
\par 27 Todas estas cosas son para bien de los piadosos, pero para los pecadores se convierten en mal.
\par 28 Hay espíritus creados para la venganza, que en su furia dan golpes dolorosos; en el tiempo de la destrucción derraman su fuerza y ​​apaciguan la ira del que los hizo.
\par 29 Fuego, granizo, hambre y muerte, todo esto fue creado para venganza;
\par 30 Dientes de fieras, escorpiones, serpientes y espadas que castigan a los impíos hasta la destrucción.
\par 31 Se regocijarán en su mandamiento y estarán listos en la tierra cuando sea necesario; y cuando llegue su hora, no transgredirán su palabra.
\par 32 Por eso desde el principio estuve decidido y pensé en estas cosas, y las dejé por escrito.
\par 33 Todas las obras del Señor son buenas, y él dará todo lo necesario a su debido tiempo.
\par 34 De modo que nadie puede decir: Esto es peor que aquello, porque con el tiempo todos serán bien aprobados.
\par 35 Por tanto, alabad al Señor con todo el corazón y con la boca, y bendecid el nombre del Señor.

\chapter{40}

\par 1 Grandes dolores de parto son creados para cada hombre, y un yugo pesado cae sobre los hijos de Adán, desde el día que salen del vientre de su madre hasta el día en que regresan a la madre de todas las cosas.
\par 2 Su imaginación sobre lo venidero y el día de la muerte, [turba] sus pensamientos y [causa] temor en el corazón;
\par 3 Desde el que está sentado en un trono de gloria, hasta el que es humillado en la tierra y en la ceniza;
\par 4 Desde el que viste púrpura y corona, hasta el que viste manto de lino.
\par 5 La ira y la envidia, la angustia y la inquietud, el miedo a la muerte, la ira y la contienda, y durante el tiempo de descanso en su cama, su sueño nocturno, cambian su conocimiento.
\par 6 Un poco o nada es su descanso, y después duerme, como en un día de vigilia, turbado en la visión de su corazón, como si hubiera escapado de una batalla.
\par 7 Cuando todo está a salvo, se despierta y se maravilla de que el miedo no haya sido nada.
\par 8 [Esto sucede] a toda carne, tanto a los hombres como a los animales, y esto es siete veces más para los pecadores.
\par 9 Muerte y sangre, contiendas y espadas, calamidades, hambre, tribulaciones y azotes;
\par 10 Estas cosas fueron creadas para los malvados, y por ellos vino el diluvio.
\par 11 Todo lo que es de la tierra volverá a la tierra, y lo que es de las aguas volverá al mar.
\par 12 Todo soborno y toda injusticia serán borrados, pero la honestidad perdurará para siempre.
\par 13 Los bienes de los injustos se secarán como un río y se desvanecerán con ruido, como un gran trueno bajo la lluvia.
\par 14 Cuando abra su mano se regocijará; así los transgresores quedarán destruidos.
\par 15 Los hijos de los impíos no producirán muchas ramas, sino que serán como raíces inmundas sobre una roca dura.
\par 16 La maleza que crece en todas las aguas y orillas de un río será arrancada antes que toda hierba.
\par 17 Como jardín fructífero es la generosidad, y para siempre es la misericordia.
\par 18 Trabajar y contentarse con lo que uno tiene es una vida dulce, pero el que encuentra un tesoro está por encima de ambas cosas.
\par 19 Los hijos y la edificación de una ciudad son el nombre del hombre, pero la esposa irreprensible es considerada por encima de ambos.
\par 20 El vino y la música alegran el corazón, pero el amor a la sabiduría está por encima de ambos.
\par 21 La flauta y el salterio entonan dulce melodía, pero sobre ambos está la lengua agradable.
\par 22 Tus ojos desean favor y hermosura, pero más que ambos granos mientras están verdes.
\par 23 Amigo y compañero nunca están de más, pero por encima de ambos está la esposa con su marido.
\par 24 En tiempos de angustia están los hermanos y la ayuda, pero la limosna salvará más que ambos.
\par 25 El oro y la plata afirman el pie, pero el consejo es más estimado que ambos.
\par 26 Las riquezas y la fuerza elevan el corazón, pero el temor del Señor está por encima de ambos: el temor del Señor no falta, y no es necesario buscar ayuda.
\par 27 El temor del Señor es un jardín fructífero que lo cubre sobre toda gloria.
\par 28 Hijo mío, no seas mendigo; pues mejor es morir que mendigar.
\par 29 La vida del que depende de la mesa de otro no se cuenta como vida; porque se contamina con la comida ajena; pero el sabio y bien alimentado se guardará de ello.
\par 30 La mendicidad es dulce en la boca del desvergonzado, pero en su vientre arderá un fuego.

\chapter{41}

\par 1 ¡Oh muerte, cuán amargo es el recuerdo de ti para el hombre que vive tranquilo en sus bienes, para el hombre que no tiene nada que lo aflija, y que tiene prosperidad en todas las cosas; sí, para el que aún puede! para recibir carne!
\par 2 ¡Oh muerte, aceptable es tu sentencia para el necesitado, y para aquel cuyas fuerzas fallan, que ya está en el último siglo y está atormentado por todas las cosas, y para aquel que desespera y ha perdido la paciencia!
\par 3 No temas la sentencia de muerte, recuerda a los que fueron antes de ti y a los que vendrán después; porque esta es la sentencia del Señor sobre toda carne.
\par 4 ¿Y por qué estás contra la voluntad del Altísimo? No hay inquisición en la tumba, ya sea que hayas vivido diez, cien o mil años.
\par 5 Los hijos de los pecadores son hijos abominables, y los que frecuentan la morada de los impíos.
\par 6 La herencia de los hijos de los pecadores perecerá, y su posteridad tendrá un oprobio perpetuo.
\par 7 Los hijos se quejarán del padre impío, porque serán vituperados por su causa.
\par 8 ¡Ay de vosotros, hombres impíos, que habéis abandonado la ley del Dios Altísimo! porque si crecéis, será para vuestra destrucción:
\par 9 Y si nacéis, naceréis para maldición; y si muréis, maldición será vuestra porción.
\par 10 Todos los habitantes de la tierra volverán a la tierra; así los impíos pasarán de la maldición a la destrucción.
\par 11 El luto de los hombres será por sus cuerpos, pero el mal nombre de los pecadores será borrado.
\par 12 Mira tu nombre; porque eso continuará contigo más que mil grandes tesoros de oro.
\par 13 La buena vida dura pocos días, pero el buen nombre permanece para siempre.
\par 14 Hijos míos, guardad en paz la disciplina; porque la sabiduría escondida y el tesoro que no se ve, ¿de qué sirven ambas cosas?
\par 15 Mejor es el hombre que oculta su necedad que el que oculta su sabiduría.
\par 16 Por tanto, avergonzaos según mi palabra; porque no es bueno conservar toda vergüenza; ni es del todo aprobado en todo.
\par 17 Avergonzaos de la fornicación delante del padre y de la madre, y de la mentira delante del príncipe y del poderoso;
\par 18 De una ofensa ante un juez y gobernante; de iniquidad ante una congregación y un pueblo; de trato injusto ante tu socio y amigo;
\par 19 Y de robo en relación con el lugar donde moras y en relación con la verdad de Dios y su alianza; y apoyarte con el codo sobre la carne; y de menospreciar el dar y el recibir;
\par 20 Y de silencio delante de los que te saludan; y mirar a una ramera;
\par 21 Y para apartar tu rostro de tu pariente; o para llevarse una porción o un regalo; o mirar a la esposa de otro hombre.
\par 22 O estar demasiado ocupado con su sierva y no acercarse a su cama; o de discursos de reproche ante amigos; y después de haber dado, no reprendas;
\par 23 O de repetir y repetir lo que has oído; y de revelación de secretos.
\par 24 Así te avergonzarás verdaderamente y hallarás gracia ante todos los hombres.

\chapter{42}

\par 1 No te avergüences de estas cosas, ni permitas que nadie peque por ellas:
\par 2 De la ley del Altísimo y de su pacto; y de juicio para justificar a los impíos;
\par 3 De hacer cuentas con tus compañeros y viajeros; o de la donación de la herencia de amigos;
\par 4 De la exactitud de la balanza y de las pesas; o de conseguir mucho o poco;
\par 5 Y de la venta indiferente de los comerciantes; de mucha corrección de los niños; y hacer sangrar el costado del siervo malo.
\par 6 La seguridad es buena donde está la esposa mala; y cállate, donde hay muchas manos.
\par 7 Entregad todas las cosas en número y peso; y pon por escrito todo lo que das o recibes.
\par 8 No te avergüences de informar a los insensatos y a los necios, ni al anciano que contiende con los más jóvenes: así serás verdaderamente instruido y aprobado por todos los vivientes.
\par 9 El padre despierta por la hija, sin que nadie lo sepa; y el cuidado de ella le quita el sueño: cuando es joven, no sea que se le pase la flor de su edad; y estando casada, para que no sea odiada:
\par 10 en su virginidad, para que no se contamine y conciba en la casa de su padre; y tener marido, para que no se porte mal; y cuando esté casada, para que no quede estéril.
\par 11 Vigila a una hija desvergonzada, no sea que te convierta en motivo de burla para tus enemigos, en burla en la ciudad y en oprobio entre el pueblo, y te avergüence ante la multitud.
\par 12 No miréis la belleza de cada uno, ni os sentéis en medio de mujeres.
\par 13 Porque de la ropa sale la polilla, y de las mujeres la maldad.
\par 14 Mejor es la grosería de un hombre que una mujer cortés, una mujer, digo, que trae vergüenza y oprobio.
\par 15 Ahora me acordaré de las obras del Señor y contaré lo que he visto: En las palabras del Señor están sus obras.
\par 16 El sol que alumbra mira todas las cosas, y su obra está llena de la gloria del Señor.
\par 17 El Señor no ha dado a los santos poder para declarar todas sus maravillas, que el Señor Todopoderoso estableció firmemente, para que todo fuera establecido para su gloria.
\par 18 Él escudriña el abismo y el corazón, y examina sus astucias; porque el Señor sabe todo lo que se puede conocer y contempla los signos del mundo.
\par 19 Él declara las cosas pasadas y las futuras, y revela los pasos de las cosas ocultas.
\par 20 Ningún pensamiento se le escapa, ninguna palabra se le oculta.
\par 21 Él ha adornado las excelentes obras de su sabiduría, y existe desde la eternidad hasta la eternidad: nada se le puede añadir ni disminuir, y no tiene necesidad de ningún consejero.
\par 22 ¡Oh cuán deseables son todas sus obras! y que un hombre pueda ver hasta una chispa.
\par 23 Todas estas cosas viven y permanecen para siempre para todos los usos, y todas son obedientes.
\par 24 Todas las cosas son dobles unas contra otras, y él nada ha hecho imperfecto.
\par 25 Una cosa determina el bien o la otra: ¿y quién se llenará de contemplar su gloria?

\chapter{43}

\par 1 La soberbia de las alturas, el claro firmamento, la hermosura del cielo, con su gloria;
\par 2 El sol, cuando aparece, anuncia en su salida un instrumento maravilloso, obra del Altísimo:
\par 3 Al mediodía el país se reseca, ¿y quién puede soportar su calor abrasador?
\par 4 El hombre que sopla un horno hace calor, pero el sol quema las montañas tres veces más; exhalando vapores de fuego y lanzando rayos brillantes, oscurece los ojos.
\par 5 Grande es el Señor que lo hizo; y a su orden corre apresuradamente.
\par 6 Hizo también la luna para que en su tiempo sirviera de declaración de los tiempos y de señal del mundo.
\par 7 De la luna proviene el signo de las fiestas, una luz que disminuye en su perfección.
\par 8 El mes lleva su nombre y aumenta maravillosamente en sus cambios, siendo un instrumento de los ejércitos de lo alto, brillando en el firmamento del cielo;
\par 9 La hermosura del cielo, la gloria de las estrellas, adorno que alumbra en las alturas del Señor.
\par 10 Por orden del Santo permanecerán en su orden y nunca desmayarán en sus guardias.
\par 11 Mira el arco iris y alaba a quien lo hizo; muy hermoso es en su brillo.
\par 12 Rodea el cielo con un círculo glorioso, y las manos del Altísimo lo doblan.
\par 13 Con su orden hace caer la nieve en un lugar y envía rápidamente los relámpagos de su juicio.
\par 14 Por medio de esto se abren los tesoros, y las nubes vuelan como aves.
\par 15 Con su gran poder fortalece las nubes y desmenuza el granizo.
\par 16 A su vista se estremecen las montañas, y a su voluntad sopla el viento del sur.
\par 17 El ruido del trueno hace temblar la tierra, así como la tempestad del norte y el torbellino; como pájaros que vuelan esparce la nieve, y su caída es como el relámpago de langostas.
\par 18 Los ojos se maravillan ante la belleza de su blancura, y el corazón se asombra ante su lluvia.
\par 19 Derrama la escarcha como sal sobre la tierra, y cuando se congela, reposa sobre puntas de estacas afiladas.
\par 20 Cuando sopla el frío viento del norte y el agua se congela en hielo, permanece sobre toda reunión de agua y la cubre como una coraza.
\par 21 Devora las montañas, quema el desierto y consume la hierba como fuego.
\par 22 El remedio presente para todos es la niebla que llega rápidamente, el rocío que llega después de que el calor refresca.
\par 23 Con sus consejos apacigua el abismo y planta en él islas.
\par 24 Los que navegan por el mar cuentan su peligro; y cuando lo oímos con nuestros oídos, nos maravillamos.
\par 25 Porque en él se crearán obras extrañas y maravillosas, variedad de toda clase de bestias y ballenas.
\par 26 Por él, sus fines prosperan y por su palabra todas las cosas subsisten.
\par 27 Podemos hablar mucho y, sin embargo, quedarnos cortos: por lo tanto, en resumen, él es todo.
\par 28 ¿Cómo podremos engrandecerlo? porque él es grande sobre todas sus obras.
\par 29 El Señor es terrible y muy grande, y maravilloso es su poder.
\par 30 Cuando glorificéis al Señor, exaltadlo tanto como podáis; porque aún se excederá con creces; y cuando lo exaltéis, emplead todas vuestras fuerzas, y no os canséis; porque nunca podréis llegar lo suficientemente lejos.
\par 31 ¿Quién le vio para que nos lo diga? ¿Y quién podrá engrandecerlo tal como es?
\par 32 Aún hay cosas más ocultas que éstas, pues de sus obras sólo hemos visto algunas.
\par 33 Porque el Señor ha hecho todas las cosas; y a los piadosos les ha dado sabiduría.

\chapter{44}

\par 1 Alabemos ahora a los hombres ilustres y a nuestros padres que nos engendraron.
\par 2 El Señor les ha hecho gran gloria con su gran poder desde el principio.
\par 3 Los que gobernaban en sus reinos, hombres famosos por su poder, que aconsejaban con su inteligencia y proclamaban profecías:
\par 4 Los líderes del pueblo, con sus consejos y su conocimiento de la ciencia, son buenos para el pueblo, sabios y elocuentes son sus instrucciones:
\par 5 Los que aprenden melodías musicales y recitan versos por escrito:
\par 6 Hombres ricos y dotados de capacidad, que habitan pacíficamente en sus habitaciones:
\par 7 Todos estos fueron honrados en sus generaciones y fueron la gloria de sus tiempos.
\par 8 Hay algunos que han dejado un nombre tras de sí, para que sus alabanzas sean contadas.
\par 9 Y hay algunos que no tienen memoria; que han perecido, como si nunca hubieran existido; y se vuelven como si nunca hubieran nacido; y sus hijos después de ellos.
\par 10 Pero estos eran hombres misericordiosos, cuya justicia no ha sido olvidada.
\par 11 Su descendencia tendrá siempre una buena herencia y sus hijos estarán dentro del pacto.
\par 12 Su descendencia permanecerá firme, y sus hijos por amor de ellos.
\par 13 Su descendencia permanecerá para siempre, y su gloria no será borrada.
\par 14 En paz serán sepultados sus cuerpos; pero su nombre vive para siempre.
\par 15 El pueblo hablará de su sabiduría y la congregación proclamará sus alabanzas.
\par 16 Enoc agradó al Señor y fue trasladado, siendo ejemplo de arrepentimiento para todas las generaciones.
\par 17 Noé fue hallado perfecto y justo; en el tiempo de la ira fue tomado a cambio [del mundo]; por tanto, cuando vino el diluvio, quedó como remanente en la tierra.
\par 18 Se hizo con él un pacto eterno: que ninguna carne perecería más a causa del diluvio.
\par 19 Abraham fue un gran padre de muchos pueblos; en gloria no hubo nadie como él;
\par 20 Que guardó la ley del Altísimo y firmó un pacto con él: estableció el pacto en su carne; y cuando fue probado, fue hallado fiel.
\par 21 Por eso le aseguró con juramento que bendeciría a las naciones en su descendencia, que lo multiplicaría como el polvo de la tierra, que exaltaría a su descendencia como las estrellas y que les haría heredar de mar a mar, y desde el río hasta lo último de la tierra.
\par 22 También estableció con Isaac la bendición de todos los hombres y el pacto, y lo hizo reposar sobre la cabeza de Jacob. Le reconoció en su bendición, le dio herencia y repartió sus porciones; entre las doce tribus los dividió.

\chapter{45}

\par 1 Y sacó de él un hombre misericordioso, que halló gracia ante los ojos de toda carne: Moisés, amado de Dios y de los hombres, cuyo memorial es bendito.
\par 2 Lo hizo semejante a los santos gloriosos y lo engrandeció, de modo que sus enemigos le temieron.
\par 3 Con sus palabras hizo cesar los milagros, lo hizo glorioso ante los reyes, le dio mandamientos para su pueblo y le mostró parte de su gloria.
\par 4 Lo santificó en su fidelidad y mansedumbre, y lo escogió entre todos los hombres.
\par 5 Le hizo oír su voz, lo metió en la nube oscura y le dio mandamientos delante de él, la ley de la vida y del conocimiento, para que enseñara a Jacob sus pactos y a Israel sus juicios.
\par 6 Y ensalzó a Aarón, un varón santo como él, su hermano, de la tribu de Leví.
\par 7 Hizo con él un pacto eterno y le dio el sacerdocio entre el pueblo; lo embelleció con hermosos adornos y lo vistió con un manto de gloria.
\par 8 Él vistió sobre él perfecta gloria; y lo fortaleció con ricos vestidos, con calzones, con un manto largo y el efod.
\par 9 Y lo rodeó de granadas y de muchas campanillas de oro alrededor, para que a su paso se oyera un sonido y un ruido que se oyera en el templo, en memoria de los hijos de su pueblo.
\par 10 Con un manto sagrado, con oro, seda azul y púrpura, obra de bordado, con un pectoral de juicio, y con Urim y Tumim;
\par 11 Con escarlata torcida, obra de hábil artífice, con piedras preciosas grabadas como sellos y engastadas en oro, obra de orfebre, con una escritura grabada para memoria, según el número de las tribus de Israel.
\par 12 Puso sobre la mitra una corona de oro, en la que estaba grabada la Santidad, adorno de honor, obra costosa, los deseos de los ojos, buenos y bellos.
\par 13 Antes de él no hubo tales, ni ningún extraño se los puso, sino sólo sus hijos y los hijos de sus hijos para siempre.
\par 14 Sus sacrificios se consumirán íntegramente cada día dos veces seguidas.
\par 15 Moisés lo santificó y lo ungió con óleo santo: esto fue establecido para él por pacto eterno, y para su descendencia, mientras existieran los cielos, para que le sirvieran y ejecutaran el oficio del sacerdocio, y bendecir al pueblo en su nombre.
\par 16 Lo escogió entre todos los vivientes para ofrecer sacrificios al Señor, incienso y olor grato, en memoria, para reconciliar a su pueblo.
\par 17 Le dio sus mandamientos y autoridad en los estatutos de los juicios, para que enseñara a Jacob los testimonios e informara a Israel en sus leyes.
\par 18 Los extraños conspiraron contra él y lo calumniaron en el desierto, los hombres de Datán y de Abirón, y la congregación de Coré, con furor e ira.
\par 19 El Señor vio esto y no le agradó, y en su ira se consumieron; hizo maravillas con ellos, consumiéndolos con llama de fuego.
\par 20 Pero hizo más honorable a Aarón, le dio una herencia y le repartió las primicias de lo cultivado; especialmente preparó pan en abundancia:
\par 21 Porque comen de los sacrificios que el Señor le dio a él y a su descendencia.
\par 22 Sin embargo, él no tuvo herencia en la tierra del pueblo, ni tuvo parte entre el pueblo; porque el Señor mismo es su porción y herencia.
\par 23 El tercero en gloria es Finees, hijo de Eleazar, porque tuvo celo en el temor del Señor y se mantuvo firme con buen corazón, cuando el pueblo se volvió atrás e hizo la reconciliación con Israel.
\par 24 Por eso se hizo con él un pacto de paz, para que él fuera el jefe del santuario y de su pueblo, y que él y su posteridad tuvieran la dignidad del sacerdocio para siempre.
\par 25 Según el pacto hecho con David, hijo de Isaí, de la tribu de Judá, de que la herencia del rey sería sólo para su posteridad, así la herencia de Aarón sería también para su descendencia.
\par 26 Dios te dé sabiduría en tu corazón para juzgar a su pueblo con justicia, para que sus bienes no sean abolidos y su gloria perdure para siempre.

\chapter{46}

\par 1 Jesús, el hijo de Nave, fue valiente en las guerras y fue el sucesor de Moisés en las profecías, quien según su nombre se hizo grande para salvar a los elegidos de Dios y vengarse de los enemigos que se levantaron contra ellos, para establecer a Israel en su herencia.
\par 2 ¡Cuánta gloria obtuvo cuando alzó sus manos y extendió su espada contra las ciudades!
\par 3 ¿Quién antes que él se mantuvo así? porque el Señor mismo trajo a él sus enemigos.
\par 4 ¿No volvió el sol por su medio? ¿Y no dura un día como dos?
\par 5 Invocó al Señor Altísimo cuando los enemigos lo acosaban por todos lados; y el gran Señor lo escuchó.
\par 6 Y con granizo de gran poder hizo caer violentamente la batalla sobre las naciones, y en el descenso [de Bet-horón] destruyó a los que resistieron, para que las naciones conocieran todas sus fuerzas, porque él peleó en el vista del Señor, y siguió al Poderoso.
\par 7 También en tiempos de Moisés hizo una obra de misericordia, él y Caleb, hijo de Jefone, resistieron a la congregación, impidieron al pueblo pecar y apaciguaron las murmuraciones de los malvados.
\par 8 Y de seiscientos mil hombres de a pie, dos fueron preservados para llevarlos a la heredad, a la tierra que mana leche y miel.
\par 9 El Señor también dio a Caleb fuerza que permaneció con él hasta su vejez, de modo que entró en las alturas de la tierra y su descendencia la obtuvo como heredad.
\par 10 Para que todos los hijos de Israel vean que es bueno seguir al Señor.
\par 11 Y en cuanto a los jueces, todos por nombre, cuyo corazón no se prostituyó ni se apartó del Señor, sea bendita su memoria.
\par 12 Que sus huesos florezcan en su lugar, y que el nombre de los honrados permanezca sobre sus hijos.
\par 13 Samuel, el profeta del Señor, amado de su Señor, estableció un reino y ungió príncipes sobre su pueblo.
\par 14 Él juzgó a la congregación según la ley del Señor, y el Señor tuvo respeto por Jacob.
\par 15 Por su fidelidad fue hallado un verdadero profeta, y por su palabra fue conocido como fiel en visión.
\par 16 Invocó al poderoso Señor cuando sus enemigos lo apretujaron por todos lados, cuando ofreció el cordero de leche.
\par 17 Y el Señor tronó desde el cielo y con gran ruido hizo oír su voz.
\par 18 Y destruyó a los gobernantes de Tiro y a todos los príncipes de los filisteos.
\par 19 Y antes de su largo sueño, protestó delante del Señor y de su ungido: No he tomado bienes de nadie, ni siquiera un zapato, y nadie lo acusó.
\par 20 Y después de su muerte profetizó, y mostró al rey su fin, y alzó su voz desde la tierra en profecía, para borrar la maldad del pueblo.

\chapter{47}

\par 1 Después de él se levantó Natán para profetizar en tiempos de David.
\par 2 Como se quita la grasa de la ofrenda de paz, así fue elegido David entre los hijos de Israel.
\par 3 Jugaba con leones como con cabritos, y con osos como con corderos.
\par 4 ¿No mató a un gigante cuando aún era joven? ¿Y no quitó el oprobio del pueblo, cuando alzó su mano con la piedra en la honda, y derribó la jactancia de Goliat?
\par 5 Porque invocó al Señor Altísimo; y le dio fuerza en su mano derecha para matar a aquel valiente guerrero, y levantar el cuerno de su pueblo.
\par 6 Entonces el pueblo lo honró con diez mil y lo alabó con las bendiciones del Señor, dándole una corona de gloria.
\par 7 Porque destruyó a los enemigos de todas partes, y destruyó a los filisteos, sus adversarios, y les quebró el cuerno hasta el día de hoy.
\par 8 En todas sus obras alababa al Santo Altísimo con palabras de gloria; con todo su corazón cantaba canciones, y amaba al que lo hizo.
\par 9 Puso también cantores delante del altar, para que con sus voces cantaran dulces melodías y cantaran cada día alabanzas con sus cánticos.
\par 10 Él embelleció sus fiestas y dispuso los tiempos solemnes hasta el fin, para que alabaran su santo nombre y para que el templo resonara desde la mañana.
\par 11 El Señor quitó sus pecados y exaltó su poder para siempre; le dio un pacto de reyes y un trono de gloria en Israel.
\par 12 Después de él surgió un hijo sabio, y por él vivió en libertad.
\par 13 Salomón reinó en tiempos de paz y fue honrado; porque Dios hizo que todo a su alrededor se tranquilizara, para edificar una casa a su nombre y preparar su santuario para siempre.
\par 14 ¡Cuán sabio eras en tu juventud y, como una inundación, lleno de inteligencia!
\par 15 Tu alma cubrió toda la tierra y la llenaste de oscuras parábolas.
\par 16 Tu nombre llegó hasta las islas; y por tu paz fuiste amado.
\par 17 Las naciones se maravillaron de ti por tus cánticos, proverbios, parábolas e interpretaciones.
\par 18 En el nombre del Señor Dios, que se llama el Señor Dios de Israel, reuniste oro como estaño y multiplicaste plata como plomo.
\par 19 Doblaste tus lomos ante las mujeres, y por tu cuerpo fuiste sujetada.
\par 20 Manchaste tu honor y contaminaste tu descendencia, de modo que enojaste a tus hijos y te entristeciste por tu necedad.
\par 21 Entonces el reino se dividió y en Efraín reinó un reino rebelde.
\par 22 Pero el Señor nunca abandonará su misericordia, ni ninguna de sus obras perecerá, ni abolirá la posteridad de sus elegidos, ni quitará la descendencia del que lo ama; por eso dio un remanente a Jacob, y de él una raíz a David.
\par 23 Así descansó Salomón con sus padres, y de su descendencia dejó tras de sí a Roboam, el necio del pueblo, el insensato, que con sus consejos desvió al pueblo. También estuvo Jeroboam hijo de Nabat, el que hizo pecar a Israel y mostró a Efraín el camino del pecado:
\par 24 Y sus pecados se multiplicaron tanto que fueron expulsados ​​de la tierra.
\par 25 Porque buscaban toda maldad, hasta que vino sobre ellos la venganza.

\chapter{48}

\par 1 Entonces el profeta Elías se levantó como fuego, y su palabra ardía como una lámpara.
\par 2 Les trajo un hambre terrible y con su celo disminuyó su número.
\par 3 Por palabra del Señor cerró el cielo y también hizo descender fuego tres veces.
\par 4 ¡Oh Elías, cómo fuiste honrado por tus maravillas! ¿Y quién puede gloriarse como tú?
\par 5 ¿Quién resucitaste de la muerte a un muerto y su alma del lugar de los muertos, por la palabra del Altísimo?
\par 6 Quien hizo perecer a reyes y a hombres ilustres de sus lechos;
\par 7 ¿Quién oyó en el Sinaí la reprensión del Señor, y en Horeb el juicio de venganza?
\par 8 Quien ungió reyes para vengarse y profetas para sucederlo:
\par 9 Quien fue arrebatado en un torbellino de fuego y en un carro de caballos de fuego;
\par 10 Quien fue ordenado para reprender en sus tiempos, para apaciguar la ira del juicio del Señor, antes de que estalle en furor, y para volver el corazón del padre hacia el hijo, y para restaurar las tribus de Jacob.
\par 11 Bienaventurados los que te vieron y durmieron en amor; porque ciertamente viviremos.
\par 12 Era Elías, que estaba cubierto por un torbellino, y Eliseo estaba lleno de su espíritu; mientras vivió, no se conmovió ante la presencia de ningún príncipe, ni nadie pudo someterlo.
\par 13 Ninguna palabra pudo vencerlo; y después de su muerte su cuerpo profetizó.
\par 14 Hizo maravillas en su vida, y en su muerte sus obras fueron maravillosas.
\par 15 Por todo esto el pueblo no se arrepintió ni se apartó de sus pecados, hasta que fueron despojados y expulsados ​​de su tierra y esparcidos por toda la tierra; sin embargo, quedó un pequeño pueblo y un gobernante en la casa de David:
\par 16 De los cuales algunos hacían lo que agrada a Dios, y otros multiplicaban los pecados.
\par 17 Ezequías fortificó su ciudad y metió agua en ella; cavó con hierro la dura roca y hizo pozos para las aguas.
\par 18 En su tiempo, Senaquerib subió y envió a Rabsaces, y alzó su mano contra Sión y se jactó con orgullo.
\par 19 Entonces les tembló el corazón y las manos, y sintieron dolor como mujeres de parto.
\par 20 Pero ellos invocaron al Señor, que es misericordioso, y extendieron sus manos hacia él; e inmediatamente el Santo los escuchó desde el cielo y los libró por el ministerio de Esay.
\par 21 Derrotó al ejército de los asirios, y su ángel los destruyó.
\par 22 Porque Ezequías había hecho lo que agradó al Señor y se fortaleció en los caminos de su padre David, tal como le había ordenado el profeta Esay, que era grande y fiel en su visión.
\par 23 En su tiempo el sol se puso hacia atrás, y él prolongó la vida del rey.
\par 24 Vio con gran espíritu lo que sucedería al final, y consoló a los que estaban de duelo en Sión.
\par 25 Él mostró lo que sucedería para siempre, y las cosas secretas que alguna vez sucedieron.

\chapter{49}

\par 1 El recuerdo de Josías es como la composición del perfume elaborado por el arte del boticario: es dulce como la miel en todos los paladares, y como la música en un banquete de vino.
\par 2 Se comportó con rectitud al convertir al pueblo y quitó las abominaciones de la iniquidad.
\par 3 Dirigió su corazón al Señor y en tiempos de los impíos estableció el culto a Dios.
\par 4 Todos, excepto David, Ezequías y Josías, fracasaron: porque abandonaron la ley del Altísimo, incluso los reyes de Judá fracasaron.
\par 5 Por eso entregó su poder a otros y su gloria a una nación extraña.
\par 6 Incendiaron la ciudad escogida del santuario y asolaron sus calles, conforme a la profecía de Jeremías.
\par 7 Porque le rogaban mal a él, que sin embargo era profeta, santificado en el vientre de su madre, para desarraigar, afligir y destruir; y para que también edifique y plante.
\par 8 Fue Ezequiel quien tuvo la gloriosa visión que le fue mostrada en el carro de los querubines.
\par 9 Porque él hizo mención de los enemigos bajo la figura de la lluvia y dirigió a los que iban bien.
\par 10 Y de los doce profetas, que se bendiga el recuerdo y que sus huesos vuelvan a florecer en su lugar, porque consolaron a Jacob y los libraron con una esperanza segura.
\par 11 ¿Cómo magnificaremos a Zorobabel? Incluso él era como un sello en la mano derecha:
\par 12 Así fue Jesús, hijo de Josedec, quien en su tiempo edificó la casa y levantó un templo santo al Señor, preparado para la gloria eterna.
\par 13 Y entre los elegidos estaba Neemías, cuyo renombre es grande, quien levantó para nosotros los muros caídos, levantó las puertas y los cerrojos y levantó de nuevo nuestras ruinas.
\par 14 Pero ningún hombre fue creado sobre la tierra como Enoc; porque fue quitado de la tierra.
\par 15 No hubo ningún joven nacido como José, gobernador de sus hermanos, sostén del pueblo, cuyos huesos fueran considerados por el Señor.
\par 16 Sem y Set gozaban de gran honor entre los hombres, y Adán también estaba por encima de todos los seres vivientes de la creación.

\chapter{50}

\par 1 Simón el sumo sacerdote, hijo de Onías, quien en su vida reparó la casa y en sus días fortificó el templo:
\par 2 Y él construyó desde los cimientos una doble altura, la alta fortaleza del muro que rodea el templo:
\par 3 En su tiempo, la cisterna para recibir agua, que estaba en su extensión como el mar, estaba cubierta con placas de bronce.
\par 4 Cuidó el templo para que no cayera y fortificó la ciudad contra los asedios.
\par 5 ¡Cómo fue honrado en medio del pueblo cuando salió del santuario!
\par 6 Era como la estrella de la mañana en medio de una nube y como la luna llena:
\par 7 Como el sol que brilla sobre el templo del Altísimo, y como el arco iris que ilumina las nubes brillantes:
\par 8 Y como la flor de las rosas en la primavera del año, como los lirios junto a las corrientes de las aguas, y como las ramas del árbol del incienso en la época del verano:
\par 9 Como fuego e incienso en el incensario, y como vaso de oro batido, engastado con toda clase de piedras preciosas:
\par 10 Y como hermoso olivo que da frutos, y como ciprés que crece hasta las nubes.
\par 11 Cuando se vistió con el manto de honor y se vistió con la perfección de la gloria, cuando subió al altar santo, hizo honorable el manto de la santidad.
\par 12 Cuando tomó las porciones de las manos de los sacerdotes, él mismo se paró junto al hogar del altar, rodeado como un cedro joven en el Líbano; y como palmeras lo rodearon.
\par 13 Así estaban todos los hijos de Aarón en su gloria, y las ofrendas del Señor en sus manos, delante de toda la congregación de Israel.
\par 14 Y terminando el servicio ante el altar, para adornar la ofrenda del Altísimo Todopoderoso,
\par 15 Extendió su mano hacia la copa, derramó la sangre de la uva y derramó al pie del altar olor fragante para el Rey Altísimo de todos.
\par 16 Entonces los hijos de Aarón gritaron, tocaron las trompetas de plata e hicieron un gran estruendo para ser oído, en memoria ante el Altísimo.
\par 17 Entonces todo el pueblo se apresuró a unirse y se postró en tierra sobre sus rostros para adorar a su Señor Dios Todopoderoso, el Altísimo.
\par 18 Los cantores también cantaban alabanzas con sus voces, y con gran variedad de sonidos se hacía dulce melodía.
\par 19 Y el pueblo rogaba al Señor Altísimo en oración ante el misericordioso, hasta que terminó la solemnidad del Señor y terminaron su servicio.
\par 20 Entonces descendió y alzó sus manos sobre toda la congregación de los hijos de Israel, para bendecir al Señor con sus labios y regocijarse en su nombre.
\par 21 Y se inclinaron para adorar por segunda vez, para recibir la bendición del Altísimo.
\par 22 Ahora pues, bendecid al Dios de todo, que en todas partes hace maravillas, que exalta nuestros días desde el vientre materno y nos trata según su misericordia.
\par 23 Que nos conceda alegría de corazón y que la paz esté en nuestros días en Israel para siempre.
\par 24 ¡Para que confirme su misericordia con nosotros y nos libre a su tiempo!
\par 25 Hay dos clases de naciones que mi corazón aborrece, y la tercera no es nación:
\par 26 Los que habitan en el monte de Samaria, los que habitan entre los filisteos y el pueblo insensato que habita en Siquem.
\par 27 Jesús, hijo de Sirac, de Jerusalén, escribió en este libro la instrucción del entendimiento y del conocimiento, quien de su corazón derramó sabiduría.
\par 28 Bienaventurado el que se ejercita en estas cosas; y el que las atesora en su corazón se hará sabio.
\par 29 Porque si las hace, será fuerte para todo; porque lo guía la luz del Señor, que da sabiduría a los piadosos. Bendito sea el nombre del Señor por los siglos. Amén, amén.

\chapter{51}

\par Oración de Jesús hijo de Sirác.

\par 1 Te daré gracias, oh Señor y Rey, y te alabaré, oh Dios mi Salvador: alabo tu nombre:
\par 2 Porque tú eres mi defensor y mi ayuda, y has preservado mi cuerpo de la destrucción, del lazo de la lengua calumniosa y de los labios que forjan mentiras, y has sido mi ayuda contra mis adversarios.
\par 3 Y me has librado, conforme a la multitud de las misericordias y la grandeza de tu nombre, de los dientes de los que estaban listos para devorarme, y de las manos de los que buscaban mi vida, y de los múltiples aflicciones que tuve;
\par 4 Del fuego que lo ahoga por todas partes, y de en medio del fuego que no encendí;
\par 5 Del fondo del infierno, de la lengua inmunda y de las palabras mentirosas.
\par 6 Por la acusación hecha al rey por una lengua injusta, mi alma se acercó hasta la muerte, mi vida estuvo cerca del infierno.
\par 7 Me rodearon por todas partes, y no había nadie que me ayudara. Busqué el auxilio de los hombres, pero no lo encontré.
\par 8 Entonces pensé en tu misericordia, oh Señor, y en tus actos de antaño, cómo liberas a los que esperan en ti y los salvas de las manos de los enemigos.
\par 9 Entonces levanté mis súplicas de la tierra y oré por la liberación de la muerte.
\par 10 Invoqué al Señor, Padre de mi Señor, para que no me dejara en los días de mi angustia, y en el tiempo de los soberbios, cuando no había ayuda.
\par 11 Alabaré tu nombre continuamente y cantaré alabanzas con acción de gracias; y así fue escuchada mi oración:
\par 12 Porque tú me salvaste de la destrucción y me libraste del mal tiempo; por eso te daré gracias, te alabaré y bendeciré su nombre, oh Señor.
\par 13 Cuando era todavía joven, o cuando viajaba al extranjero, deseaba abiertamente sabiduría en mi oración.
\par 14 Oré por ella ante el templo y la buscaré hasta el fin.
\par 15 Desde la flor hasta que la uva estuvo madura, mi corazón se deleitó en ella; mi pie fue por el camino recto, desde mi juventud la busqué.
\par 16 Incliné un poco mi oído, la recibí y obtuve mucha ciencia.
\par 17 Yo aproveché esto; por tanto, daré gloria al que me da la sabiduría.
\par 18 Porque me propuse hacer lo que ella es, y seguí con empeño lo bueno; así no seré avergonzado.
\par 19 Mi alma luchó con ella, y fui exacto en mis acciones: extendí mis manos hacia el cielo y lamenté mi ignorancia sobre ella.
\par 20 A ella dirigí mi alma y la encontré en pureza; mi corazón está unido a ella desde el principio, por eso no seré abandonado.
\par 21 Mi corazón se turbó buscándola; por eso obtuve una buena posesión.
\par 22 El Señor me ha dado una lengua como recompensa, y con ella lo alabaré.
\par 23 Acercaos a mí, ignorantes, y habitad en casa de la ciencia.
\par 24 ¿Por qué sois lentos y qué decís a estas cosas, teniendo vuestras almas mucha sed?
\par 25 Abrí la boca y dije: Compradla vosotros sin dinero.
\par 26 Pon tu cuello bajo el yugo y deja que tu alma reciba instrucción: es difícil de encontrar.
\par 27 Mirad con vuestros ojos que tengo poco trabajo y mucho descanso.
\par 28 Consigue aprender con una gran suma de dinero y obtén mucho oro con ella.
\par 29 Alégrese tu alma de su misericordia y no te avergüences de sus alabanzas.
\par 30 Trabaja tu trabajo temprano, y a su tiempo él te dará tu recompensa.

\end{document}