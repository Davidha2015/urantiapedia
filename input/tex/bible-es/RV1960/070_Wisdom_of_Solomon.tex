\begin{document}

\title{Sabiduría de Salomón}


\chapter{1}

\par 1 Amad la justicia, jueces de la tierra; pensad en el Señor con buen corazón y buscadle con sencillez de corazón.
\par 2 Porque será hallado por aquellos que no lo tientan; y se muestra a los que no desconfían de él.
\par 3 Porque los pensamientos perversos separan a Dios, y su poder, cuando se prueba, reprende a los imprudentes.
\par 4 Porque en el alma maliciosa no entrará la sabiduría; ni habitar en el cuerpo que está sujeto al pecado.
\par 5 Porque el espíritu santo de disciplina huirá del engaño, se alejará de los pensamientos sin entendimiento y no permanecerá cuando entre la injusticia.
\par 6 Porque la sabiduría es un espíritu amoroso; y no absolverá al blasfemo de sus palabras; porque Dios es testigo de sus riñones, y fiel ve su corazón, y oyente de su lengua.
\par 7 Porque el Espíritu del Señor llena el mundo, y lo que contiene todas las cosas tiene conocimiento de la voz.
\par 8 Por lo tanto, el que habla cosas injustas no puede ser encubierto, ni la venganza, cuando castigue, pasará de largo.
\par 9 Porque se investigarán los designios de los impíos, y el sonido de sus palabras llegará al Señor para la manifestación de sus malas acciones.
\par 10 Porque el oído de los celos todo lo oye, y el ruido de las murmuraciones no se oculta.
\par 11 Por tanto, guardaos de la murmuración, que no es provechosa; y refrena tu lengua de calumniar; porque no hay palabra tan secreta que quede en vano; y la boca que descree, mata el alma.
\par 12 No busquéis la muerte en el error de vuestra vida, ni atraigéis sobre vosotros la destrucción con las obras de vuestras manos.
\par 13 Porque Dios no hizo la muerte, ni le agrada la destrucción de los vivientes.
\par 14 Porque él creó todas las cosas para que existieran; y las generaciones del mundo fueron saludables; y no hay en ellos veneno de destrucción, ni reino de muerte sobre la tierra:
\par 15 (Porque la justicia es inmortal:)
\par 16 Pero los hombres impíos con sus obras y palabras la llamaron; porque, pensando tenerla como amiga, la consumieron e hicieron con ella un pacto, porque eran dignos de participar en ella.

\chapter{2}

\par 1 Porque los impíos decían, razonando entre sí, pero no correctamente: Nuestra vida es corta y tediosa, y para la muerte de un hombre no hay remedio; ni se sabe de ningún hombre que haya regresado de la tumba.
\par 2 Porque nacemos en toda aventura, y seremos en el futuro como si nunca hubiéramos existido: porque el aliento en nuestras narices es como humo, y una pequeña chispa en el movimiento de nuestro corazón.
\par 3 El cual, extinguido, nuestro cuerpo se convertirá en cenizas, y nuestro espíritu se desvanecerá como el aire suave,
\par 4 Y nuestro nombre será olvidado con el tiempo, y nadie se acordará de nuestras obras, y nuestra vida pasará como la huella de una nube, y se dispersará como la niebla que se disipa con los rayos. del sol, y vencidos por su calor.
\par 5 Porque nuestro tiempo es una sombra que pasa; y después de nuestro fin no hay retorno: porque está firmemente sellado, para que nadie vuelva a venir.
\par 6 Vamos, pues, disfrutemos de los bienes que están presentes y usemos rápidamente a las criaturas como en la juventud.
\par 7 Llenémonos de vino y de ungüentos costosos, y no dejemos pasar ninguna flor de la primavera.
\par 8 Coronémonos de capullos de rosas antes de que se sequen:
\par 9 Que ninguno de nosotros se vaya sin su parte de nuestra voluptuosidad: dejemos muestras de nuestra alegría en todo lugar: porque ésta es nuestra porción, y ésta es nuestra suerte.
\par 10 Oprimamos al justo pobre, no perdonemos a la viuda, ni reverenciamos las antiguas canas del anciano.
\par 11 Que nuestra fuerza sea la ley de la justicia: porque lo débil resulta despreciable.
\par 12 Por tanto, acechemos a los justos; porque no es para nosotros, y es claramente contrario a nuestras acciones: nos reprende por nuestras ofensas a la ley, y objeta a nuestra infamia las transgresiones de nuestra educación.
\par 13 Profesa tener conocimiento de Dios y se llama a sí mismo hijo del Señor.
\par 14 Él fue creado para reprender nuestros pensamientos.
\par 15 Nos resulta doloroso incluso verlo, porque su vida no es como la de los demás, sus caminos son diferentes.
\par 16 Nos tiene por falsos; se abstiene de nuestros caminos como de inmundicias, declara bienaventurado el fin del justo y se jacta de que Dios es su padre.
\par 17 Veamos si sus palabras son ciertas y comprobemos lo que le sucederá al final.
\par 18 Porque si el justo es hijo de Dios, él lo ayudará y lo librará de manos de sus enemigos.
\par 19 Examinémoslo con desprecio y tortura, para que conozcamos su mansedumbre y probemos su paciencia.
\par 20 Condenémoslo con una muerte vergonzosa, porque según sus propias palabras será respetado.
\par 21 Se imaginaban tales cosas y se engañaban, porque su propia maldad los había cegado.
\par 22 En cuanto a los misterios de Dios, no los conocían, ni esperaban recompensa de justicia, ni discernían recompensa para las almas inocentes.
\par 23 Porque Dios creó al hombre para que fuera inmortal y lo hizo imagen de su propia eternidad.
\par 24 Sin embargo, por envidia del diablo entró la muerte en el mundo, y los que están de su lado la encuentran.

\chapter{3}

\par 1 Pero las almas de los justos están en manos de Dios y ningún tormento las alcanzará.
\par 2 A los ojos de los imprudentes parecían morir, y su partida es tomada por miseria,
\par 3 Y su partida de nosotros será destrucción total, pero ellos están en paz.
\par 4 Porque aunque sean castigados ante los ojos de los hombres, su esperanza está llena de inmortalidad.
\par 5 Y después de haber sido castigados un poco, recibirán una gran recompensa: porque Dios los probó y los encontró dignos de sí.
\par 6 Los probó como oro en el horno y los recibió en holocausto.
\par 7 Y en el tiempo de su visita brillarán y correrán de un lado a otro como chispas entre el rastrojo.
\par 8 Juzgarán a las naciones y dominarán a los pueblos, y su Señor reinará para siempre.
\par 9 Los que confían en él comprenderán la verdad, y los fieles en el amor permanecerán con él; porque la gracia y la misericordia son para sus santos, y él tiene cuidado de sus elegidos.
\par 10 Pero los impíos serán castigados según sus propias imaginaciones, que han descuidado a los justos y abandonado al Señor.
\par 11 Porque quien desprecia la sabiduría y la disciplina es miserable, y su esperanza es vana, sus esfuerzos infructuosos y sus obras inútiles.
\par 12 Sus mujeres son necias y sus hijos malvados.
\par 13 Su descendencia está maldita. Por tanto, bienaventurada la estéril que es inmaculada, que no conoció el lecho del pecado: ella tendrá frutos en la visita de las almas.
\par 14 Y bienaventurado el eunuco que con sus manos no ha hecho iniquidad ni ha imaginado cosas malas contra Dios; porque a él le será dado el don especial de la fe, y una herencia en el templo del Señor más agradable a sus mente.
\par 15 Porque glorioso es el fruto del buen trabajo, y la raíz de la sabiduría nunca fallará.
\par 16 En cuanto a los hijos de adúlteros, no llegarán a su perfección, y la semilla de un lecho injusto será desarraigada.
\par 17 Aunque vivan muchos años, no serán considerados en nada, y su última edad será sin honor.
\par 18 O, si mueren rápidamente, no tendrán esperanza ni consuelo en el día de la prueba.
\par 19 Porque horrible será el fin de la generación injusta.

\chapter{4}

\par 1 Mejor es no tener hijos y tener virtud, porque su recuerdo es inmortal, porque es conocido por Dios y por los hombres.
\par 2 Cuando está presente, los hombres toman ejemplo de ello; y cuando se va, lo desean: lleva una corona y triunfa para siempre, habiendo obtenido la victoria, luchando por recompensas inmaculadas.
\par 3 Pero la multiplicación de los impíos no prosperará, ni echará raíces profundas en deslices bastardos, ni echará cimientos sólidos.
\par 4 Porque aunque florecen en las ramas por un tiempo; pero no permaneciendo firmes, serán sacudidos por el viento, y por la fuerza de los vientos serán desarraigados.
\par 5 Las ramas imperfectas serán rotas, su fruto no será provechoso, no estará maduro para comer, y no servirá para nada.
\par 6 Porque los hijos nacidos de lechos ilícitos son testigos de la maldad contra sus padres en el juicio.
\par 7 Pero aunque el justo sea impedido por la muerte, aún descansará.
\par 8 Porque edad honorable no es la que dura el tiempo, ni la que se mide por el número de años.
\par 9 Pero la sabiduría es para los hombres las canas, y la vida sin mancha es la vejez.
\par 10 Agradó a Dios y fue amado por él, de modo que viviendo entre pecadores fue trasladado.
\par 11 Y rápidamente fue quitado, para que la maldad no alterara su entendimiento, o el engaño engañara su alma.
\par 12 Porque el encanto de la maldad oscurece las cosas honestas; y el extravío de la concupiscencia socava la mente simple.
\par 13 Él, perfeccionado en poco tiempo, cumplió mucho tiempo:
\par 14 Porque su alma agradó al Señor; por eso se apresuró a apartarlo de entre los malvados.
\par 15 Esto lo vio el pueblo, pero no lo entendió, ni pensó en esto: que su gracia y misericordia están con sus santos, y que tiene respeto hacia sus escogidos.
\par 16 Así, el justo muerto condenará a los impíos que viven; y juventud que pronto se perfecciona en los muchos años y vejez de los injustos.
\par 17 Porque verán el fin del sabio y no entenderán lo que Dios en su consejo ha decretado para él, ni con qué fin el Señor lo ha puesto a salvo.
\par 18 Lo verán y lo despreciarán; pero Dios se burlará de ellos, y en lo sucesivo serán un cadáver vil y un oprobio entre los muertos para siempre.
\par 19 Porque él los desgarrará y los derribará hasta dejarlos mudos; y los sacudirá desde los cimientos; y serán completamente devastados y entristecidos; y su memoria perecerá.
\par 20 Y cuando rindan cuentas de sus pecados, vendrán con miedo, y sus propias iniquidades los convencerán en su propia cara.

\chapter{5}

\par 1 Entonces el justo se presentará con gran confianza ante los que lo afligieron y no tomaron en cuenta sus trabajos.
\par 2 Cuando lo vean, se llenarán de miedo y se asombrarán de lo extraño de su salvación, mucho más allá de lo que esperaban.
\par 3 Y ellos, arrepintiéndose y gimiendo de angustia de espíritu, dirán dentro de sí: Éste era aquel de quien algunas veces teníamos en burla y en proverbio de oprobio:
\par 4 Nosotros, los necios, consideramos su vida una locura y su fin sin honor.
\par 5 ¡Cómo es contado entre los hijos de Dios, y su suerte entre los santos!
\par 6 Por eso nos hemos desviado del camino de la verdad, y la luz de la justicia no ha brillado para nosotros, y el sol de la justicia no ha salido para nosotros.
\par 7 Nos cansamos en el camino de la maldad y la destrucción; incluso pasamos por desiertos donde no había camino, pero el camino del Señor no lo conocíamos.
\par 8 ¿De qué nos ha aprovechado el orgullo? ¿O qué bien nos han traído las riquezas con nuestra jactancia?
\par 9 Todas esas cosas pasaron como una sombra y como un poste que pasa deprisa;
\par 10 Y como un barco que pasa sobre las olas del agua, y cuando pasa no se encuentra en las olas su huella, ni el camino de la quilla;
\par 11 O como cuando un pájaro ha volado por el aire, y no se encuentra ninguna señal de su camino, sino que el ligero aire, golpeado con el golpe de sus alas y dividido con el violento ruido y movimiento de ellas, pasa. a través, y allí después no se encuentra ninguna señal de dónde fue;
\par 12 O como cuando una flecha, lanzada al blanco, divide el aire, y al momento se vuelve a juntar, de modo que el hombre no puede saber por dónde atravesó.
\par 13 Así también nosotros, tan pronto como nacimos, comenzamos a llegar a nuestro fin y no teníamos ningún signo de virtud que mostrar; pero fuimos consumidos en nuestra propia maldad.
\par 14 Porque la esperanza de los santos es como polvo que se lleva el viento; como una fina espuma que se lleva la tormenta; como el humo que se dispersa aquí y allá con la tempestad, y se desvanece como el recuerdo de un huésped que se queda sólo un día.
\par 15 Pero los justos vivirán para siempre; Su recompensa también está con el Señor, y el cuidado de ellos está con el Altísimo.
\par 16 Por tanto, recibirán de la mano del Señor un reino glorioso y una hermosa corona: porque con su diestra los cubrirá y con su brazo los protegerá.
\par 17 Tomará sus celos como armadura completa y hará de la criatura su arma para vengarse de sus enemigos.
\par 18 Se vestirá de justicia como coraza, y de juicio verdadero en lugar de yelmo.
\par 19 Tomará la santidad como escudo invencible.
\par 20 Afilará su severa ira como espada, y el mundo peleará con él contra los imprudentes.
\par 21 Entonces los rayos que apuntan correctamente saldrán disparados; y desde las nubes, como desde un arco bien tensado, volarán hacia el blanco.
\par 22 Y granizo lleno de ira será lanzado como desde un arco de piedra, y el agua del mar se enfurecerá contra ellos, y las corrientes los ahogarán cruelmente.
\par 23 Sí, un viento fuerte se levantará contra ellos, y como una tormenta los arrastrará: así la iniquidad devastará toda la tierra, y las malas acciones derribarán los tronos de los poderosos.

\chapter{6}

\par 1 Oíd, pues, reyes, y entended; Aprended, vosotros que sois jueces de los confines de la tierra.
\par 2 ¡Oíd, los que gobernáis al pueblo, y gloriaos en la multitud de las naciones!
\par 3 Porque el poder os ha sido dado por el Señor, y el poder del Altísimo, que examinará vuestras obras y explorará vuestros consejos.
\par 4 Porque siendo ministros de su reino, no habéis juzgado rectamente, ni guardado la ley, ni andado según el consejo de Dios;
\par 5 Vendrá sobre vosotros de manera terrible y rápida, porque los que están en las altas esferas recibirán un juicio severo.
\par 6 Porque la misericordia pronto perdonará al más humilde, pero los valientes serán terriblemente atormentados.
\par 7 Porque el que es Señor de todo no temerá a la persona de nadie, ni temerá la grandeza de nadie; porque él hizo al pequeño y al grande, y cuida de todos por igual.
\par 8 Pero a los poderosos les sobrevendrá una dura prueba.
\par 9 A vosotros, pues, os hablo, oh reyes, para que adquiráis sabiduría y no os desviéis.
\par 10 Porque los que guardan la santidad santamente serán juzgados santos; y los que han aprendido tales cosas encontrarán qué responder.
\par 11 Por tanto, pon tu afecto en mis palabras; deseadlos, y seréis instruidos.
\par 12 La sabiduría es gloriosa y nunca se desvanece; sí, fácilmente la ven los que la aman, y la encuentran los que la buscan.
\par 13 Ella previene a los que la desean, dándose a conocer primero a ellos.
\par 14 Quien la busque temprano no tendrá grandes dolores de parto, porque la encontrará sentada a su puerta.
\par 15 Por lo tanto, pensar en ella es la perfección de la sabiduría, y quien la vigila pronto quedará despreocupado.
\par 16 Porque ella anda buscando a los que son dignos de ella, se muestra favorable a ellos en los caminos y está a su altura en todos los pensamientos.
\par 17 Porque su verdadero principio es el deseo de disciplina; y el cuidado de la disciplina es amor;
\par 18 Y el amor es la observancia de sus leyes; y la atención a sus leyes es la seguridad de la incorrupción;
\par 19 Y la incorrupción nos acerca a Dios:
\par 20 Por eso el deseo de la sabiduría conduce al reino.
\par 21 Si, pues, os deleitáis en tronos y cetros, vosotros, reyes del pueblo, honrad la sabiduría, para reinar por los siglos de los siglos.
\par 22 En cuanto a la sabiduría, qué es y cómo surgió, os lo diré, y no os ocultaré misterios, sino que la buscaré desde el principio de su nacimiento y sacaré a la luz su conocimiento. , y no pasará por alto la verdad.
\par 23 Tampoco iré con envidia devoradora; porque tal hombre no tendrá comunión con la sabiduría.
\par 24 Pero la multitud de sabios es el bienestar del mundo, y un rey sabio es el sostén del pueblo.
\par 25 Recibe, pues, instrucción mediante mis palabras, y te hará bien.

\chapter{7}

\par 1 Yo también soy un hombre mortal, como todos, y descendiente del primero que fue creado de la tierra,
\par 2 Y en el vientre de mi madre fue formado carne en el tiempo de diez meses, siendo compactado en sangre, de la simiente del hombre, y el placer que viene con el sueño.
\par 3 Y cuando nací, aspiré el aire común y caí sobre la tierra, que es de naturaleza similar, y la primera voz que pronuncié fue un clamor como todos los demás.
\par 4 Me criaron en pañales y con cuidados.
\par 5 Porque no hay rey ​​que haya nacido de otro modo.
\par 6 Porque todos los hombres tienen una misma entrada a la vida y una misma salida.
\par 7 Por lo cual oré, y me fue dado entendimiento: invoqué a Dios, y el espíritu de sabiduría vino a mí.
\par 8 La preferí a los cetros y a los tronos, y no estimé las riquezas en comparación con ella.
\par 9 No comparé con ella ninguna piedra preciosa, porque para ella todo el oro es como un poco de arena, y la plata se cuenta delante de ella como barro.
\par 10 La amé más que a la salud y a la belleza, y la escogí en lugar de la luz, porque la luz que de ella sale nunca se apaga.
\par 11 Con ella vinieron a mí todos los bienes juntos, y en sus manos innumerables riquezas.
\par 12 Y me alegré de todos ellos, porque la sabiduría va delante de ellos, y no sabía que ella era su madre.
\par 13 He aprendido con diligencia y la comunico generosamente; no oculto sus riquezas.
\par 14 Porque ella es para los hombres un tesoro que nunca falla: quienes lo utilizan se convierten en amigos de Dios, siendo alabados por los dones que provienen del aprendizaje.
\par 15 Dios me ha concedido hablar como quisiera y concebir como conviene a lo que me es dado, porque él es el que guía a la sabiduría y dirige a los sabios.
\par 16 Porque en su mano estamos nosotros y nuestras palabras; toda sabiduría también, y conocimiento de obra.
\par 17 Porque él me ha dado cierto conocimiento de las cosas que existen, es decir, saber cómo fue hecho el mundo y cómo funcionan los elementos.
\par 18 El principio, el fin y la mitad de los tiempos, los cambios de la rotación del sol y el cambio de las estaciones.
\par 19 Los circuitos de los años y las posiciones de las estrellas:
\par 20 La naturaleza de los seres vivientes y los furores de las fieras, la violencia de los vientos y los razonamientos de los hombres: la diversidad de las plantas y las virtudes de las raíces:
\par 21 Y todo lo que es secreto o manifiesto, lo sé.
\par 22 Porque me enseñó la sabiduría, que es la creadora de todas las cosas; porque en ella hay un espíritu inteligente, santo, uno solo, múltiple, astuto, vivaz, claro, inmaculado, sencillo, no susceptible a daño, que ama lo que es bueno rápido, que no se puede dejar, dispuesto a hacer el bien,
\par 23 Bondadoso para con los hombres, firme, seguro, libre de preocupaciones, omnipotente, omnipotente y omnisciente, de espíritu puro y sutilísimo.
\par 24 Porque la sabiduría es más conmovedora que cualquier movimiento: por su pureza pasa y atraviesa todas las cosas.
\par 25 Porque ella es el aliento del poder de Dios y una influencia pura que fluye de la gloria del Todopoderoso; por eso, nada contaminado puede caer en ella.
\par 26 Porque ella es el resplandor de la luz eterna, el espejo sin mancha del poder de Dios y la imagen de su bondad.
\par 27 Y siendo una sola, puede hacer todas las cosas; y permaneciendo en sí misma, hace nuevas todas las cosas; y entrando en todas las edades en las almas santas, las hace amigas de Dios y profetas.
\par 28 Porque Dios no ama a nadie más que a aquel que habita en la sabiduría.
\par 29 Porque ella es más hermosa que el sol y, sobre todo, que el orden de las estrellas: comparada con la luz, se encuentra antes que ella.
\par 30 Porque después de esto vendrá la noche, pero el vicio no prevalecerá contra la sabiduría.

\chapter{8}

\par 1 La sabiduría se extiende poderosamente de un extremo al otro y todo lo ordena dulcemente.
\par 2 La amé y la busqué desde mi juventud, quise hacerla mi esposa y fui amante de su belleza.
\par 3 En su relación con Dios, engrandece su nobleza; incluso el Señor de todas las cosas la amó.
\par 4 Porque ella conoce los misterios del conocimiento de Dios y ama sus obras.
\par 5 Si las riquezas son una posesión deseable en esta vida; ¿Qué hay más rico que la sabiduría, que obra todas las cosas?
\par 6 Y si obra la prudencia; ¿Quién de todos los que existen es más hábil que ella?
\par 7 Y si un hombre ama la justicia, sus trabajos son virtudes: porque ella enseña la templanza y la prudencia, la justicia y la fortaleza, que son cosas que no pueden tener nada más provechoso en su vida.
\par 8 Si un hombre desea mucha experiencia, sabe cosas antiguas y conjetura correctamente lo que está por venir; conoce las sutilezas de los discursos y puede exponer frases oscuras; prevé señales y prodigios, y los acontecimientos de las estaciones y los tiempos. .
\par 9 Por eso decidí tomarla para que viviera conmigo, sabiendo que sería mi consejera en los bienes y mi consuelo en las preocupaciones y en los dolores.
\par 10 Por ella tendré estima entre la multitud y honra entre los mayores, aunque sea joven.
\par 11 Seré hallado de rápido engreimiento en el juicio, y seré admirado ante los ojos de los grandes.
\par 12 Cuando callo, esperarán mi tiempo, y cuando hable, me escucharán bien; si hablo mucho, se pondrán las manos en la boca.
\par 13 Además, por medio de ella alcanzaré la inmortalidad y dejaré tras de mí un recuerdo eterno para los que vengan detrás de mí.
\par 14 Pondré en orden a los pueblos y las naciones se sujetarán a mí.
\par 15 Los tiranos terribles temerán cuando oigan hablar de mí; Seré bueno entre la multitud y valiente en la guerra.
\par 16 Cuando llegue a mi casa, descansaré con ella, porque su conversación no tiene amargura; y vivir con ella no tiene tristeza, sino alegría y alegría.
\par 17 Ahora bien, cuando consideraba estas cosas en mí mismo y las meditaba en mi corazón, cómo la unión de la sabiduría es la inmortalidad;
\par 18 Y es un gran placer tener su amistad; y en las obras de sus manos hay infinitas riquezas; y en el ejercicio de la conferencia con ella, la prudencia; y al hablar con ella, buen informe; Anduve buscando cómo llevármela.
\par 19 Porque yo era un niño ingenioso y de buen espíritu.
\par 20 Más bien, siendo bueno, entré en un cuerpo sin mancha.
\par 21 Sin embargo, cuando entendí que de otra manera no podría obtenerla, a menos que Dios me la diera; y ese también era un punto de sabiduría para saber de quién era el regalo; Oré al Señor y le rogué, y con todo mi corazón dije:

\chapter{9}

\par 1 Oh Dios de mis padres, y Señor de misericordia, que hiciste todas las cosas con tu palabra,
\par 2 Y con tu sabiduría ordenaste al hombre para que tuviera dominio sobre las criaturas que tú creaste,
\par 3 y ordenar el mundo según la equidad y la rectitud, y ejecutar el juicio con rectitud de corazón.
\par 4 Dame sabiduría, que estoy sentado junto a tu trono; y no me rechaces de entre tus hijos:
\par 5 Porque yo, tu siervo y el hijo de tu sierva, soy una persona débil, de corta edad y demasiado joven para entender el derecho y las leyes.
\par 6 Porque aunque un hombre nunca sea tan perfecto entre los hijos de los hombres, si tu sabiduría no está con él, no será tenido en cuenta.
\par 7 Tú me has elegido para ser rey de tu pueblo y juez de tus hijos y de tus hijas.
\par 8 Tú me has ordenado que edifique un templo en tu santo monte y un altar en la ciudad donde habitas, a semejanza del santo tabernáculo que tú preparaste desde el principio.
\par 9 Y contigo estaba la sabiduría, que conoce tus obras, que estaba presente cuando creaste el mundo y que sabía lo que era agradable ante tus ojos y lo recto en tus mandamientos.
\par 10 Envíala desde tus santos cielos y desde el trono de tu gloria, para que, estando presente, trabaje conmigo y pueda saber lo que te agrada.
\par 11 Porque ella sabe y comprende todas las cosas, y me guiará con seriedad en mis acciones y me preservará en su poder.
\par 12 Entonces mis obras serán aceptables, y entonces juzgaré a tu pueblo con justicia y seré digno de sentarme en el trono de mi padre.
\par 13 Porque ¿quién es aquel que puede conocer el consejo de Dios? ¿O quién puede pensar cuál es la voluntad del Señor?
\par 14 Porque los pensamientos de los mortales son miserables, y nuestras intenciones, inciertas.
\par 15 Porque el cuerpo corruptible oprime el alma, y ​​el tabernáculo terrenal oprime la mente que reflexiona sobre muchas cosas.
\par 16 Y difícilmente podemos adivinar las cosas que están en la tierra, y con trabajo encontramos las cosas que están delante de nosotros; pero las cosas que están en el cielo, ¿quién las ha investigado?
\par 17 ¿Y quién conocerá tus consejos si no les das sabiduría y envías tu Espíritu Santo desde lo alto?
\par 18 Así se cambiaron las costumbres de los habitantes de la tierra, se enseñó a los hombres lo que te agrada y se salvaron mediante la sabiduría.

\chapter{10}

\par 1 Ella preservó al primer padre formado del mundo, que fue creado solo, y lo sacó de su caída,
\par 2 Y le dio poder para gobernar todas las cosas.
\par 3 Pero cuando el injusto se alejó de ella en su ira, él también pereció en la ira con que mató a su hermano.
\par 4 Por cuya causa la tierra, sumergida en el diluvio, la sabiduría la preservó y dirigió el camino de los justos en un trozo de madera de poco valor.
\par 5 Además, estando avergonzadas las naciones en su malvada conspiración, ella descubrió al justo y lo preservó irreprochable ante Dios y lo mantuvo fuerte contra su tierna compasión hacia su hijo.
\par 6 Cuando perecieron los impíos, ella libró al justo, que huyó del fuego que cayó sobre las cinco ciudades.
\par 7 De cuya maldad, hasta el día de hoy, son testimonio la tierra desolada que humea, las plantas que dan frutos que nunca llegan a madurar, y la estatua de sal en pie, monumento del alma incrédula.
\par 8 Porque por no tener en cuenta la sabiduría, no sólo sufrieron el daño de no conocer las cosas buenas, sino que también sufrieron el daño de no saber lo que era bueno; pero también dejaron detrás de ellos para el mundo un recuerdo de su necedad: de modo que en las cosas en las que ofendieron no pudieron ni siquiera ocultarse.
\par 9 Rut la sabiduría libró del dolor a quienes la asistían.
\par 10 Cuando el justo huía de la ira de su hermano, ella lo guió por caminos rectos, le mostró el reino de Dios, le dio conocimiento de las cosas santas, lo enriqueció en sus viajes y multiplicó el fruto de sus trabajos.
\par 11 En la codicia de los que lo oprimían, ella lo apoyó y lo enriqueció.
\par 12 Ella lo defendió de sus enemigos, lo mantuvo a salvo de los que acechaban y en una dura lucha le dio la victoria; para que sepa que el bien es más fuerte que todo.
\par 13 Cuando el justo fue vendido, ella no lo abandonó, sino que lo libró del pecado: descendió con él a la fosa,
\par 14 Y no lo dejó en prisiones hasta que ella le trajo el cetro del reino y el poder contra los que lo oprimían; en cuanto a los que lo habían acusado, los mostró como mentirosos y le dio gloria perpetua.
\par 15 Ella libró al pueblo justo y a la descendencia irreprochable de la nación que los oprimía.
\par 16 Ella entró en el alma del siervo del Señor y resistió a reyes terribles con prodigios y señales;
\par 17 Dio a los justos la recompensa por sus trabajos, los guió por caminos maravillosos y fue para ellos un refugio durante el día y una luz de estrellas durante la noche;
\par 18 Los hizo atravesar el mar Rojo y los condujo a través de muchas aguas.
\par 19 Pero ella ahogó a sus enemigos y los arrojó del fondo del abismo.
\par 20 Por eso los justos despojaron a los impíos, y alabaron tu santo nombre, oh Señor, y unánimemente engrandecieron tu mano que luchaba por ellos.
\par 21 Porque la sabiduría abrió la boca de los mudos y ensanchó la lengua de los que no pueden hablar.

\chapter{11}

\par 1 Ella hizo prosperar sus obras en manos del santo profeta.
\par 2 Atravesaron un desierto inhabitado y levantaron tiendas en lugares donde no había camino.
\par 3 Se enfrentaron a sus enemigos y se vengaron de sus adversarios.
\par 4 Cuando tuvieron sed, te invocaron, y del pedernal se les dio agua, y de la piedra dura se les apagó la sed.
\par 5 Porque con lo que fueron castigados sus enemigos, con lo mismo se beneficiaron ellos en su necesidad.
\par 6 Porque en lugar de un río continuo y lleno de sangre inmunda,
\par 7 Para reprender claramente el mandamiento por el cual los niños fueron asesinados, les diste agua en abundancia por un medio que no esperaban.
\par 8 Con esa sed declaras cómo castigaste a sus adversarios.
\par 9 Pues cuando fueron probados, aunque castigados con misericordia, sabían que los impíos eran juzgados con ira y atormentados, teniendo sed diferente que los justos.
\par 10 A estos los amonestaste y probaste como a un padre, pero al otro, como a un rey severo, los condenaste y castigaste.
\par 11 Tanto si estaban ausentes como si estaban presentes, se enfadaron por igual.
\par 12 Porque les sobrevino un doble dolor y un gemido por el recuerdo de lo pasado.
\par 13 Porque cuando oyeron que sus propios castigos beneficiaban a otros, sintieron algo del Señor.
\par 14 Porque a quien respetaban con desprecio, cuando mucho antes había sido echado fuera al dar a luz a los niños, al final, cuando vieron lo que había sucedido, lo admiraron.
\par 15 Pero por las necias maquinaciones de su maldad, con las que, engañados, adoraban serpientes irracionales y bestias viles, tú enviaste sobre ellos multitud de bestias irracionales para vengarse;
\par 16 Para que supieran que todo lo que el hombre peca, también por eso será castigado.
\par 17 Porque tu mano omnipotente, que hizo el mundo de la materia sin forma, no faltó medios para enviar entre ellos una multitud de osos o leones feroces,
\par 18 O bestias salvajes desconocidas, llenas de ira, recién creadas, que exhalan vapor de fuego, o olores inmundos de humo esparcido, o lanzan horribles destellos de sus ojos:
\par 19 Por lo cual no sólo el daño podría matarlos de inmediato, sino que también el terrible espectáculo los destruiría por completo.
\par 20 Sí, y sin estas fuerzas habrían caído de un solo golpe, perseguidos por venganza, y esparcidos por el soplo de tu poder; pero tú ordenaste todas las cosas en medida, número y peso.
\par 21 Porque puedes mostrar tu gran fuerza en todo momento cuando quieras; ¿Y quién podrá resistir el poder de tu brazo?
\par 22 Porque el mundo entero ante ti es como un pequeño grano de la balanza, sí, como una gota del rocío de la mañana que cae sobre la tierra.
\par 23 Pero tú tienes misericordia de todos; porque tú puedes hacer todas las cosas, y hacer un guiño a los pecados de los hombres, porque deberían enmendarse.
\par 24 Porque amas todas las cosas que existen y no aborreces nada de lo que has hecho; porque nunca habrías hecho nada si lo hubieras odiado.
\par 25 ¿Y cómo podría haber perdurado algo si no hubiera sido tu voluntad? ¿O ha sido preservado, si no es llamado por ti?
\par 26 Pero tú los perdonas a todos, porque tuyos son, oh Señor, amante de las almas.

\chapter{12}

\par 1 Porque tu Espíritu incorruptible está en todas las cosas.
\par 2 Por eso, castigas poco a poco a los que ofenden y les adviertes recordándoles en qué han ofendido, para que, dejando su maldad, crean en ti, oh Señor.
\par 3 Porque fuiste tu voluntad destruir por mano de nuestros padres a los antiguos habitantes de tu tierra santa,
\par 4 A quien odiabas por hacer las más odiosas obras de brujería y malvados sacrificios;
\par 5 Y también esos despiadados asesinos de niños, y devoradores de carne humana, y los banquetes de sangre,
\par 6 Con sus sacerdotes de en medio de su grupo idólatra y con los padres que mataban con sus propias manos a las almas necesitadas de ayuda:
\par 7 Para que la tierra que tú estimabas por encima de todas las demás pudiera recibir una colonia digna de los hijos de Dios.
\par 8 Sin embargo, aun a aquellos que eran hombres, los perdonaste y enviaste avispas, precursoras de tu ejército, para destruirlos poco a poco.
\par 9 No es que no hayas podido poner a los impíos bajo la mano de los justos en la batalla, ni destruirlos de inmediato con bestias crueles o con una sola palabra dura:
\par 10 Pero, ejecutando poco a poco tus juicios sobre ellos, les diste lugar de arrepentimiento, sin ignorar que eran una generación traviesa, que su malicia se había engendrado en ellos y que su pensamiento nunca cambiaría.
\par 11 Porque fue simiente maldita desde el principio; ni por temor a nadie les diste perdón de aquellas cosas en que pecaron.
\par 12 Porque ¿quién dirá: ¿Qué has hecho? ¿O quién resistirá tu juicio? ¿O quién te acusará de las naciones que tú creaste, que perecen? ¿O quién vendrá a oponerse a ti para vengarse de los hombres injustos?
\par 13 Porque no hay más Dios que tú, que se preocupa por todos y a quien puedes mostrar que tu juicio no es injusto.
\par 14 Ni el rey ni el tirano podrán volverse contra ti por cualquiera a quien hayas castigado.
\par 15 Por tanto, puesto que tú mismo eres justo, ordenas todas las cosas con rectitud, pensando que no te conviene condenar a quien no merece ser castigado.
\par 16 Porque tu poder es el principio de la justicia, y porque eres el Señor de todos, te hace ser misericordioso con todos.
\par 17 Porque cuando los hombres no creen que tienes todo el poder, muestras tu fuerza, y entre los que lo saben manifiestas su audacia.
\par 18 Pero tú, dominando tu poder, juzgas con equidad y nos ordenas con gran favor, porque puedes usar el poder cuando quieras.
\par 19 Pero con tales obras has enseñado a tu pueblo que el justo debe ser misericordioso, y has hecho que tus hijos tengan buena esperanza de que les des arrepentimiento de los pecados.
\par 20 Porque si castigaras a los enemigos de tus hijos y a los condenados a muerte con tanta deliberación, dándoles el tiempo y el lugar para librarlos de su malicia,
\par 21 ¿Con cuánta prudencia juzgaste a tus propios hijos, a cuyos padres juraste e hiciste pactos con buenas promesas?
\par 22 Por eso, mientras nos castigas, azotas mil veces más a nuestros enemigos, para que, cuando juzguemos, pensemos cuidadosamente en tu bondad, y cuando nosotros mismos seamos juzgados, busquemos misericordia.
\par 23 Por eso, mientras los hombres vivían de manera disoluta e injusta, tú los atormentaste con sus propias abominaciones.
\par 24 Porque se extraviaron mucho en el camino del error, y los tuvieron por dioses, que incluso entre las bestias de sus enemigos eran despreciados, siendo engañados como hijos sin entendimiento.
\par 25 Por eso, a ellos, como a niños sin uso de razón, les enviaste un juicio para burlarte de ellos.
\par 26 Pero aquellos que no quieran ser reformados por la corrección con la que él se entretuvo con ellos, sentirán un juicio digno de Dios.
\par 27 Porque, mirad, qué cosas guardaron rencor cuando fueron castigados, es decir, por aquellos que tenían por dioses; [ahora] siendo castigados en ellos, cuando lo vieron, reconocieron que era el Dios verdadero, a quien antes negaban conocer: y por eso vino sobre ellos la condenación extrema.

\chapter{13}

\par 1 Ciertamente vanos son por naturaleza todos los hombres que ignoran a Dios y no pueden conocer al que es por las cosas buenas que ven; ni por las obras reconocieron al maestro;
\par 2 Pero considera que el fuego, o el viento, o el aire veloz, o el círculo de las estrellas, o las aguas violentas, o las luces del cielo, son los dioses que gobiernan el mundo.
\par 3 cuya belleza, si estaban encantados, los tomaban por dioses; que sepan cuánto mejor es el Señor de ellos: porque los creó el primer autor de la belleza.
\par 4 Pero si se asombraron de su poder y de su virtud, entiendan por ellos cuánto más poderoso es aquel que los hizo.
\par 5 Porque en la grandeza y belleza de las criaturas se ve en proporción a quién las hizo.
\par 6 Pero, sin embargo, por esto son menos culpables: porque tal vez se equivocan, buscando a Dios y deseando encontrarlo.
\par 7 Porque, conocedores de sus obras, lo examinan con diligencia y creen en lo que ven, porque las cosas que se ven son hermosas.
\par 8 Pero tampoco deben ser perdonados.
\par 9 Porque si pudieran saber tanto, podrían apuntar al mundo; ¿Cómo no descubrieron antes a su Señor?
\par 10 Pero son miserables y esperan en las cosas muertas los que llaman dioses a las obras de manos de los hombres, al oro y a la plata, para mostrar el arte y la semejanza de las bestias, o una piedra que no sirve para nada. obra de una mano antigua.
\par 11 Ahora bien, el carpintero que tala madera, después de haber aserrado un árbol apto para ese fin, quitado hábilmente toda la corteza alrededor, labrado bellamente y hecho de él un recipiente apto para el servicio de la vida del hombre. ;
\par 12 Y después de gastar los restos de su trabajo en preparar su comida, se ha saciado;
\par 13 Y tomando de entre los desechos que no servían para nada, un trozo de madera torcido y lleno de nudos, lo talló con diligencia, cuando no tenía otra cosa que hacer, y lo formó con la habilidad de su entendimiento. , y lo modeló a imagen de hombre;
\par 14 O lo hizo como una bestia vil, cubriéndolo de bermellón y pintándolo de rojo, y cubriendo todas sus manchas;
\par 15 Y cuando le hizo un lugar conveniente, lo colocó en la pared y lo aseguró con hierro.
\par 16 Porque él hizo lo necesario para que no cayera, sabiendo que no podía ayudarse a sí mismo; porque es una imagen y necesita ayuda:
\par 17 Entonces ora por sus bienes, por su mujer y por sus hijos, y no se avergüenza de hablar con quien no tiene vida.
\par 18 Al que está débil le pide salud; porque a lo que está muerto le pide vida; porque ayuda humildemente pide al que menos medios tiene para ayudar: y para un buen viaje pide al que no puede poner un pie adelante:
\par 19 Y para ganar y conseguir, y para el buen éxito de sus manos, al que es más incapaz de hacer nada, le pide capacidad para hacer.

\chapter{14}

\par 1 Otra vez, el que se prepara para navegar y está a punto de atravesar las olas furiosas, invoca un trozo de madera más podrido que el barco que lo transporta.
\par 2 Porque, en verdad, el deseo de ganancias lo inventó, y el obrero lo construyó con su habilidad.
\par 3 Pero tu providencia, oh Padre, la gobierna: porque tú has abierto camino en el mar y senda segura entre las olas;
\par 4 Mostrando que puedes salvarte de todo peligro, incluso aunque un hombre se hiciera a la mar sin arte.
\par 5 Sin embargo, no quieres que las obras de tu sabiduría queden en vano, y por eso los hombres entregan sus vidas a un pequeño trozo de madera, y se salvan atravesando el mar embravecido en un barco débil.
\par 6 Porque también en los tiempos antiguos, cuando perecieron los orgullosos gigantes, la esperanza del mundo gobernado por tu mano se escapó en un recipiente débil, y dejó para todas las edades una semilla de generación.
\par 7 Porque bendita es la madera de donde viene la justicia.
\par 8 Pero lo hecho con manos es maldito tanto él como el que lo hizo: él, porque lo hizo; y esto, porque siendo corruptible, se llamaba dios.
\par 9 Porque tanto el impío como su impiedad son aborrecibles ante Dios.
\par 10 Porque lo que se hace será castigado junto con el que lo hizo.
\par 11 Por tanto, también los ídolos de las naciones serán castigados, porque en la criatura de Dios son abominación y escándalo para las almas de los hombres, y lazo para los pies de los insensatos.
\par 12 Porque la invención de ídolos fue el comienzo de la fornicación espiritual, y su invención, la corrupción de la vida.
\par 13 Porque ni existieron desde el principio, ni existirán para siempre.
\par 14 Porque por la vana gloria de los hombres entraron en el mundo, y por eso pronto llegarán a su fin.
\par 15 Porque un padre afligido por un duelo prematuro, cuando hizo una imagen de su hijo que pronto se le quitaría, ahora lo honraba como a un dios, que entonces era un hombre muerto, y entregaba a sus subordinados ceremonias y sacrificios.
\par 16 Así, con el tiempo, una costumbre impía que se había afianzado se convirtió en ley, y las imágenes talladas fueron adoradas según los mandamientos de los reyes.
\par 17 A quien los hombres no podían honrar en presencia porque vivía lejos, tomaron de lejos la imitación de su rostro y le hicieron una imagen expresa de un rey a quien honraban, para adular con su atrevimiento. al que estaba ausente, como si estuviera presente.
\par 18 También la singular diligencia del artífice ayudó a empujar a los ignorantes a más superstición.
\par 19 Porque él, tal vez queriendo agradar a alguien con autoridad, puso toda su habilidad en hacer la semejanza de la mejor moda.
\par 20 Y la multitud, seducida por la gracia de la obra, lo tomó por un dios que poco antes era venerado.
\par 21 Y ésta fue una ocasión para engañar al mundo: porque los hombres, sirviendo a la calamidad o a la tiranía, atribuían a las piedras y a los leños el nombre incomunicable.
\par 22 Además, no les bastó con extraviarse en el conocimiento de Dios; pero mientras vivían en la gran guerra de la ignorancia, aquellas plagas tan grandes las llamaban paz.
\par 23 Porque mientras mataban a sus hijos en sacrificios, o practicaban ceremonias secretas, o celebraban ritos extraños;
\par 24 Ya no mantuvieron intactas sus vidas ni sus matrimonios, sino que o uno mató a otro por traición, o lo entristeció con adulterio.
\par 25 De modo que en todos los hombres reinaba, sin excepción, la sangre, el homicidio, el robo, el engaño, la corrupción, la infidelidad, los tumultos, el perjurio,
\par 26 Inquietud de los buenos, olvido de las buenas obras, contaminación de las almas, cambio de género, desorden en el matrimonio, adulterio e impureza descarada.
\par 27 Porque el culto a ídolos sin nombre es el principio, la causa y el fin de todo mal.
\par 28 Porque o se vuelven locos cuando están alegres, o profetizan mentiras, o viven injustamente, o perjuran a la ligera.
\par 29 Porque si confían en ídolos que no tienen vida, aunque juren en falso, no parecen sufrir daño.
\par 30 Sin embargo, serán castigados con justicia por ambas causas: por no tener buena opinión de Dios, obedeciendo a los ídolos, y también por jurar con engaño injustamente, despreciando la santidad.
\par 31 Porque no es el poder de aquellos por quienes juran, sino la justa venganza de los pecadores, que castiga siempre la ofensa de los impíos.

\chapter{15}

\par 1 Pero tú, oh Dios, eres clemente y verdadero, paciente y con misericordia que ordenas todas las cosas,
\par 2 Porque si pecamos, tuyos somos, conociendo tu poder; pero no pecaremos, sabiendo que somos contados por tuyos.
\par 3 Porque conocerte es justicia perfecta; sí, conocer tu poder es la raíz de la inmortalidad.
\par 4 Porque ni nos engañó la invención maliciosa de los hombres, ni una imagen manchada de diversos colores, el trabajo infructuoso del pintor;
\par 5 Su visión incita a los necios a codiciarla y desean la forma de una imagen muerta que no tiene aliento.
\par 6 Tanto los que los hacen, como los que los desean y los que los adoran, son amantes de las cosas malas y dignos de tenerlas en qué confiar.
\par 7 Porque el alfarero, templando la tierra blanda, con mucho trabajo elabora cada vaso para nuestro servicio; y de la misma arcilla hace tanto los vasos que sirven para usos limpios como también todos los que sirven para lo contrario. ¿De qué sirve cualquiera de las dos cosas? El alfarero mismo es el juez.
\par 8 Y empleando sus trabajos lascivamente, hace un dios vano de la misma arcilla, incluso el que un poco antes era él mismo de tierra, y al poco tiempo vuelve a lo mismo, cuando su vida que le fue prestada. será exigido.
\par 9 Sin embargo, su preocupación no es que tenga que trabajar mucho ni que su vida sea corta, sino que se esfuerza por superar a los orfebres y plateros, y se esfuerza por hacer como los trabajadores del bronce, y considera su gloria hacer cosas falsificadas. .
\par 10 Su corazón es ceniza, su esperanza más vil que la tierra y su vida menos valiosa que el barro.
\par 11 Por cuanto no conoció a su Hacedor, ni a aquel que le inspiró un alma activa y respiró un espíritu viviente.
\par 12 Pero consideraban nuestra vida como un pasatiempo, y nuestro tiempo aquí como un mercado para ganar dinero; porque, dicen, debemos conseguirlo en todos los sentidos, aunque sea por malos medios.
\par 13 Porque este hombre, que hace vasos frágiles e imágenes talladas con materia terrenal, sabe que es más ofensivo que todos los demás.
\par 14 Y todos los enemigos de tu pueblo que lo tienen en sujeción son muy necios y más miserables que los niños.
\par 15 Porque tenían por dioses a todos los ídolos de las naciones, que no tienen ojos para ver, ni narices para respirar, ni oídos para oír, ni dedos de las manos para tocar; y sus pies son lentos para andar.
\par 16 Porque los hizo el hombre, y el que tomó prestado su propio espíritu los formó; pero ningún hombre puede hacer un dios semejante a él.
\par 17 Porque, siendo mortal, hace algo muerto con manos malvadas; porque él mismo es mejor que las cosas que adora; mientras que él vivió una vez, pero ellas nunca.
\par 18 Y adoraron también a las bestias más aborrecibles: porque comparadas entre sí, unas son peores que otras.
\par 19 Tampoco son tan hermosos como para ser deseados en comparación con las bestias, sino que anduvieron sin la alabanza de Dios y su bendición.

\chapter{16}

\par 1 Por eso fueron castigados dignamente con semejantes animales y atormentados con multitud de bestias.
\par 2 En lugar de este castigo, tratando con bondad a los tuyos, les preparas carne de sabor extraño, incluso codornices, para estimular su apetito.
\par 3 Con el fin de que, deseando comida, a causa del feo espectáculo de las bestias enviadas entre ellos, aborrezcan incluso aquello que necesitan desear; pero éstos, al sufrir penuria durante un breve espacio de tiempo, podrían convertirse en partícipes de un gusto extraño.
\par 4 Porque era necesario que a los que ejercían la tiranía les sobreviniera una miseria que no podían evitar; pero a éstos sólo se les debía mostrar cómo eran atormentados sus enemigos.
\par 5 Porque cuando la terrible fiereza de las bestias cayó sobre ellos y perecieron bajo las picaduras de serpientes tortuosas, tu ira no duró para siempre.
\par 6 Pero fueron perturbados por un corto tiempo, para ser amonestados, teniendo señal de salvación, para recordarles el mandamiento de tu ley.
\par 7 Porque el que se volvió hacia ella no fue salvo por lo que vio, sino por ti, que eres el Salvador de todos.
\par 8 Y con esto haces confesar a tus enemigos que eres tú quien libra de todo mal:
\par 9 A ellos los mataron las picaduras de langostas y moscas, y no se encontró remedio para sus vidas, porque eran dignos de ser castigados por tales.
\par 10 Pero a tus hijos ni siquiera los dientes de dragones venenosos vencieron: porque tu misericordia estuvo siempre sobre ellos y los sanó.
\par 11 Porque fueron aguijoneados para recordar tus palabras; y fueron pronto salvos, para que, sin caer en profundo olvido, tuvieran siempre presente tu bondad.
\par 12 Porque no fue ni la hierba ni el apósito suavizante lo que les devolvió la salud, sino tu palabra, oh Señor, que todo lo cura.
\par 13 Porque tienes poder sobre la vida y la muerte: conduces a las puertas del infierno y las haces subir.
\par 14 A la verdad el hombre mata con su malicia, pero el espíritu, cuando sale, no regresa; Ni el alma recibida vuelve a subir.
\par 15 Pero no es posible escapar de tu mano.
\par 16 Porque los impíos que negaban conocerte, fueron azotados con la fuerza de tu brazo; con lluvias extrañas, granizos y aguaceros fueron perseguidos, que no pudieron evitar, y en el fuego fueron consumidos.
\par 17 Porque, lo que es más sorprendente, es que el fuego tenía más fuerza que el agua, que lo apaga todo: porque el mundo lucha por los justos.
\par 18 Durante algún tiempo la llama se atenuó para que no quemara a las bestias enviadas contra los impíos; pero ellos mismos podían ver y percibir que eran perseguidos por el juicio de Dios.
\par 19 Y en otra ocasión arde incluso en medio del agua por encima del poder del fuego, para destruir los frutos de una tierra injusta.
\par 20 En lugar de eso, alimentaste a tu pueblo con comida de ángeles y les enviaste desde el cielo pan preparado sin su trabajo, capaz de satisfacer el deleite de todos y del gusto de todos.
\par 21 Porque tu sustento declaraba tu dulzura a tus hijos, y sirviendo al apetito del que come, se templaba al gusto de cada uno.
\par 22 Pero la nieve y el hielo resistieron el fuego y no se derritieron, para que supieran que el fuego, ardiendo en el granizo y brillando en la lluvia, destruía los frutos de los enemigos.
\par 23 Pero éste también olvidó incluso sus propias fuerzas, para que los justos pudieran ser alimentados.
\par 24 Porque la criatura que te sirve, que eres el Hacedor, aumenta sus fuerzas contra los injustos para su castigo, y las disminuye en beneficio de los que confían en ti.
\par 25 Por eso ya entonces se transformó en todas las formas y obedeció a tu gracia, que todo lo sustenta, según el deseo de los que tenían necesidad.
\par 26 Para que tus hijos, oh Señor, a quienes amas, sepan que no es el crecimiento de los frutos lo que nutre al hombre, sino que es tu palabra la que preserva a los que en ti confían.
\par 27 Porque lo que no fue destruido por el fuego, calentado por un pequeño rayo de sol, pronto se derritió.
\par 28 Para que se sepa que debemos impedir que el sol te dé gracias y orar a ti en la aurora.
\par 29 Porque la esperanza de los ingratos se derretirá como la escarcha del invierno, y se escurrirá como agua inútil.

\chapter{17}

\par 1 Porque tus juicios son grandes y no se pueden expresar: por eso las almas incultas se han equivocado.
\par 2 Porque cuando los injustos pensaban oprimir a la nación santa, ellos, encerrados en sus casas, prisioneros de las tinieblas y encadenados con las ataduras de una larga noche, yacían [allí] exiliados de la eterna providencia.
\par 3 Pues, mientras creían que yacían escondidos en sus pecados secretos, fueron esparcidos bajo un velo oscuro de olvido, quedando terriblemente asombrados y perturbados por [extrañas] apariciones.
\par 4 Porque ni el ángulo que los retenía pudo impedirles el miedo; sino que se oían a su alrededor ruidos [como de agua] que caían, y se les aparecían visiones tristes con semblantes tristes.
\par 5 Ningún poder del fuego podría iluminarlos, ni las brillantes llamas de las estrellas podrían resistir para iluminar aquella horrible noche.
\par 6 Pero se les apareció un fuego encendido por sí solo, muy espantoso; porque, muy aterrorizados, pensaban que lo que veían era peor que lo que no veían.
\par 7 En cuanto a las ilusiones del arte mágico, fueron humilladas, y su alarde de sabiduría fue reprendida con vergüenza.
\par 8 Porque aquellos que prometieron ahuyentar los terrores y las angustias de un alma enferma, ellos mismos estaban hartos de miedo y eran dignos de burla.
\par 9 Pues aunque no temían nada terrible; sin embargo, asustados por las fieras que pasaban y por el silbido de las serpientes,
\par 10 Murieron de miedo, negando haber visto el aire, que de ningún modo podía evitarse.
\par 11 Porque la maldad, condenada por su propio testimonio, es muy temerosa y, apremiada por la conciencia, siempre presagia cosas dolorosas.
\par 12 Porque el miedo no es más que una traición a los auxilios que ofrece la razón.
\par 13 Y la esperanza interior, siendo menor, considera más la ignorancia que la causa que trae el tormento.
\par 14 Pero aquella noche durmieron el mismo sueño, que en verdad era insoportable y que les sobrevino desde el fondo del infierno inevitable,
\par 15 En parte estaban atormentados por apariciones monstruosas, y en parte desmayados, desfalleciéndoles el corazón; porque un miedo repentino e inesperado les sobrevino.
\par 16 Entonces, cualquiera que cayera allí sería retenido, encerrado en una prisión sin barrotes de hierro,
\par 17 Porque ya fuera labrador, o pastor, o trabajador del campo, se vio sorprendido y soportó una necesidad que no podía evitar: porque todos estaban atados por una misma cadena de oscuridad.
\par 18 Ya sea un silbido del viento, o un melodioso canto de pájaros entre las ramas extendidas, o una agradable caída de agua que corre violentamente,
\par 19 Ni el ruido terrible de las piedras arrojadas, ni la carrera invisible de las bestias que saltan, ni el rugido de las fieras más salvajes, ni el eco que resuena desde las montañas huecas; estas cosas les hicieron desmayarse de miedo.
\par 20 Porque el mundo entero resplandecía con una luz clara y nadie encontraba obstáculo en su trabajo.
\par 21 Sólo sobre ellos se extendía una noche pesada, imagen de las tinieblas que más tarde los recibirían; pero, sin embargo, eran para sí mismos más dolorosos que las tinieblas.

\chapter{18}

\par 1 Sin embargo, tus santos tuvieron una luz muy grande; oyendo su voz, y no viendo su forma, porque tampoco ellos habían sufrido lo mismo, los tuvieron por felices.
\par 2 Pero ahora no les hicieron daño a aquellos de quienes antes habían sido agraviados, les dieron las gracias y les pidieron perdón por haber sido enemigos.
\par 3 En lugar de eso, les diste una columna de fuego ardiente, que les serviría de guía en el viaje desconocido y un sol inofensivo para entretenerlos honorablemente.
\par 4 Porque eran dignos de ser privados de la luz y encarcelados en las tinieblas los que habían encerrado a tus hijos, por quienes la luz incorrupta de la ley debía ser dada al mundo.
\par 5 Y cuando decidieron matar a los niños de los santos, siendo arrojado y salvado un niño para reprenderlos, les quitaste la multitud de sus hijos y los destruiste por completo en una gran agua.
\par 6 Nuestros padres fueron informados de aquella noche, para que, sabiendo en qué juramentos habían creído, pudieran tener buen ánimo después.
\par 7 Así fue aceptada por tu pueblo tanto la salvación de los justos como la destrucción de los enemigos.
\par 8 Porque con lo que castigaste a nuestros adversarios, con ello glorificaste a nosotros, a quienes habías llamado.
\par 9 Porque los hijos justos de los hombres buenos sacrificaban en secreto y de común acuerdo dictaban una ley santa para que los santos fueran como partícipes del mismo bien y del mismo mal, mientras los padres cantaban ahora cánticos de alabanza.
\par 10 Pero del otro lado se oyó un grito desagradable de los enemigos, y un ruido lamentable se extendió por los niños que lloraban.
\par 11 El amo y el siervo fueron castigados de la misma manera; y como el rey, así padecía el hombre común.
\par 12 Así que todos juntos tuvieron innumerables muertos con una misma clase de muerte; Tampoco los vivos eran suficientes para enterrarlos: porque en un momento el descendiente más noble de ellos fue destruido.
\par 13 Porque ellos no quisieron creer nada a causa de los encantamientos; tras la destrucción de los primogénitos, reconocieron a este pueblo como hijos de Dios.
\par 14 Porque mientras todo estaba en silencio y la noche avanzaba velozmente,
\par 15 Tu palabra omnipotente saltó del cielo, desde tu trono real, como un guerrero feroz en medio de una tierra de destrucción,
\par 16 Y trajiste tu mandamiento sincero como una espada aguda, y estando en pie llenó todas las cosas de muerte; y tocó el cielo, pero permaneció sobre la tierra.
\par 17 De repente, visiones de sueños horribles los perturbaron dolorosamente y les sobrevinieron terrores inesperados.
\par 18 Y uno arrojado aquí y otro allá, medio muertos, indicaron la causa de su muerte.
\par 19 Porque los sueños que los atormentaban anunciaban esto, para que no perecieran y no supieran por qué estaban afligidos.
\par 20 Incluso el gusto de la muerte tocó también a los justos, y hubo destrucción de la multitud en el desierto; pero la ira no duró mucho.
\par 21 Entonces el hombre inocente se apresuró a defenderlos; y trayendo el escudo de su propio ministerio, es decir, la oración y la propiciación del incienso, se enfrentó a la ira y así puso fin a la calamidad, declarando que era tu siervo.
\par 22 Así venció al destructor, no con la fuerza del cuerpo ni con la fuerza de las armas, sino con una palabra sometió al que castigaba, alegando los juramentos y pactos hechos con los padres.
\par 23 Porque cuando los muertos ya estaban caídos unos sobre otros, interponiéndose entre ellos, él detuvo la ira y separó el camino hacia los vivos.
\par 24 Porque en el vestido largo estaba el mundo entero, y en las cuatro hileras de piedras estaba esculpida la gloria de los padres, y tu Majestad sobre la diadema de su cabeza.
\par 25 El destructor cedió su lugar a estos y tuvo miedo de ellos, porque les bastaba con que probaran la ira.

\chapter{19}

\par 1 En cuanto a los impíos, la ira vino sobre ellos sin piedad hasta el fin, porque él sabía de antemano lo que harían;
\par 2 Cómo es que, habiéndoles dado permiso para partir y despidiéndolos apresuradamente, se arrepintieran y los persiguieran.
\par 3 Porque, mientras aún estaban llorando y lamentándose ante las tumbas de los muertos, añadieron otra locura y los persiguieron como a fugitivos, a quienes habían suplicado que se fueran.
\par 4 Porque el destino del que eran dignos los atrajo a este fin y les hizo olvidar las cosas que ya habían sucedido, para poder cumplir el castigo que faltaba a sus tormentos:
\par 5 Y para que tu pueblo pase por un camino maravilloso, pero encuentre una muerte extraña.
\par 6 Porque toda la criatura en su propia especie fue formada de nuevo, cumpliendo los preceptos peculiares que les fueron dados, para que tus hijos pudieran ser guardados sin daño.
\par 7 Es decir, una nube que ensombrece el campamento; y donde antes había agua, apareció tierra seca; y del mar Rojo un camino sin impedimento; y de la corriente violenta un campo verde:
\par 8 Por donde pasó todo el pueblo que estaba defendido por tu mano, viendo tus maravillosas y extrañas maravillas.
\par 9 Porque andaban sueltos como caballos y saltaban como corderos, alabandote, oh Señor, que los has librado.
\par 10 Porque aún recordaban lo que habían hecho mientras peregrinaban en tierra extraña: cómo la tierra producía moscas en lugar de ganado, y cómo el río producía multitud de ranas en lugar de peces.
\par 11 Pero después vieron una nueva generación de aves y, llevados por su apetito, pidieron manjares delicados.
\par 12 Porque subieron hasta ellos codornices del mar para su contentamiento.
\par 13 Y los pecadores recibieron el castigo no sin signos previos, con la fuerza de los truenos: porque sufrieron justamente según su propia maldad, en la medida en que se comportaban con mayor dureza y odio hacia los extraños.
\par 14 Porque los sodomitas no recibieron a aquellos a quienes no conocían cuando llegaron, sino que estos llevaron a servidumbre a amigos que bien los merecían.
\par 15 Y no sólo esto, sino que tal vez se les tenga algo de respeto, porque trataron a extraños y no de manera amistosa:
\par 16 Pero estos, a quienes habían recibido con festines, los afligieron muy gravemente y ya se les hizo partícipes de las mismas leyes.
\par 17 Por eso, también quedaron ciegos estos, como los que estaban a las puertas del justo, cuando, rodeados por una gran oscuridad y espantosa, cada uno buscaba el paso por sus propias puertas.
\par 18 Porque los elementos fueron transformados en sí mismos por una especie de armonía, como en un salterio las notas cambian el nombre de la melodía, y sin embargo son siempre sonidos; que bien puede percibirse por la vista de las cosas que se han hecho.
\par 19 Porque las cosas terrenas se volvieron acuosas, y las cosas que antes nadaban en el agua, ahora iban por la tierra.
\par 20 El fuego tuvo poder en el agua, olvidándose de su propia virtud, y el agua olvidó su propia naturaleza de apagar.
\par 21 Por otro lado, las llamas no consumieron la carne de los seres vivientes corruptibles, aunque caminaban por ella; ni derritieron el tipo helado de carne celestial que por naturaleza era apto para derretirse.
\par 22 Porque, oh Señor, en todo engrandeciste a tu pueblo y lo glorificaste, sin menospreciarlo, sino que lo ayudaste en todo momento y lugar.


\end{document}