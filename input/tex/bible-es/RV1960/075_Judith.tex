\begin{document}

\title{Judit}


\chapter{1}

\par 1 En el año duodécimo del reinado de Nabucodonosor, que reinó en Nínive, la gran ciudad; en los días de Arfaxad, que reinó sobre los medos en Ecbatane,
\par 2 Y edificó en Ecbatane muros alrededor de piedras labradas de tres codos de ancho y seis codos de largo, y la altura del muro de setenta codos, y su anchura de cincuenta codos.
\par 3 Y pondrá sus torres sobre sus puertas, de cien codos de alto, y de sesenta codos de ancho en sus cimientos.
\par 4 E hizo sus puertas, puertas que se elevaban hasta una altura de setenta codos, y de cuarenta codos de ancho, para la salida de sus poderosos ejércitos y para la formación de sus soldados de a pie.
\par 5 En aquellos días, el rey Nabucodonosor hizo la guerra al rey Arfaxad en la gran llanura, que es la llanura en los límites de Ragau.
\par 6 Y vinieron a él todos los que habitaban en la región montañosa, y todos los que habitaban junto al Éufrates, el Tigris, el Hidaspes y la llanura de Arioc, rey de los Elimeos, y muchísimas naciones de los hijos de Quelod, se reunieron para la batalla.
\par 7 Entonces Nabucodonosor, rey de los asirios, envió un mensaje a todos los que habitaban en Persia, a todos los que habitaban al oeste, a los que habitaban en Cilicia, a Damasco, al Líbano y a Antilíbano, y a todos los que habitaban en la costa del mar. ,
\par 8 Y a los de las naciones del Carmelo, de Galaad, de la Alta Galilea y de la gran llanura de Esdrelom,
\par 9 Y a todos los que estaban en Samaria y sus ciudades, y al otro lado del Jordán hasta Jerusalén, Betane, Quelus, Cades, el río de Egipto, Tafnes, Ramesés y toda la tierra de Gesem,
\par 10 Hasta llegar más allá de Tanis y Menfis, y de todos los habitantes de Egipto, hasta llegar a las fronteras de Etiopía.
\par 11 Pero todos los habitantes de la tierra despreciaron la orden de Nabucodonosor, rey de los asirios, y no fueron con él a la batalla; porque no le tenían miedo; sí, él estaba delante de ellos como un solo hombre, y despidieron de ellos a sus embajadores sin efecto y con deshonra.
\par 12 Por lo tanto, Nabucodonosor se enojó mucho contra todo este país, y juró por su trono y reino, que ciertamente se vengaría de todas aquellas costas de Cilicia, Damasco y Siria, y que mataría a espada a todos los habitantes de la tierra de Moab, y los hijos de Amón, y toda Judea, y todos los que estaban en Egipto, hasta llegar a la frontera de los dos mares.
\par 13 En el año diecisiete, con su poder, marchó en orden de batalla contra el rey Arfaxad, y venció en su batalla, pues derrotó todo el poder de Arfaxad, toda su gente de a caballo y todos sus carros.
\par 14 Y se hizo señor de sus ciudades, y vino a Ecbatane, y tomó las torres, despojó sus calles y convirtió en vergüenza su belleza.
\par 15 También tomó a Arfaxad en las montañas de Ragau, lo hirió con sus dardos y lo destruyó por completo ese día.
\par 16 Entonces regresó después a Nínive, siendo él y toda su compañía de diversas naciones una multitud muy grande de hombres de guerra, y allí se relajó y celebró un banquete, tanto él como su ejército, durante ciento veinte días.

\chapter{2}

\par 1 Y en el año dieciocho, el día veintidós del mes primero, se habló en casa de Nabucodonosor, rey de los asirios, de que, como él había dicho, se vengaría en toda la tierra.
\par 2 Entonces llamó a todos sus oficiales y a todos sus nobles, les comunicó su consejo secreto y de su propia boca sacó la aflicción de toda la tierra.
\par 3 Entonces decretaron destruir toda carne que no obedeciera el mandamiento de su boca.
\par 4 Y cuando terminó su consejo, Nabucodonosor, rey de los asirios, llamó a Holofernes, el jefe de su ejército, que estaba a su lado, y le dijo.
\par 5 Así dice el gran rey, señor de toda la tierra: He aquí, tú saldrás de mi presencia y tomarás contigo hombres que confían en sus propias fuerzas, ciento veinte mil de a pie; y el número de los caballos con sus jinetes, doce mil.
\par 6 E irás contra todo el país del oeste, porque desobedecieron mi mandamiento.
\par 7 Y declararás que me preparan tierra y agua; porque saldré en mi ira contra ellos y cubriré toda la faz de la tierra con los pies de mi ejército, y los daré por estropearles:
\par 8 De modo que sus muertos llenarán sus valles y arroyos y el río se llenará de sus muertos hasta desbordarse.
\par 9 Y los llevaré cautivos hasta los confines de toda la tierra.
\par 10 Tú, pues, saldrás tomarás de antemano para mí todos sus términos; y si se entregan a ti, me los reservarás hasta el día de su castigo.
\par 11 Pero no dejes que tu ojo perdone a los que se rebelan; sino llévalos al matadero y destrózalos dondequiera que vayas.
\par 12 Porque vivo yo y por el poder de mi reino, todo lo que he dicho, lo haré por mi mano.
\par 13 Y ten cuidado de no transgredir ninguno de los mandamientos de tu señor, sino cumplidlos plenamente, tal como te he ordenado, y no tardes en cumplirlos.
\par 14 Entonces Holofernes salió de delante de su señor y llamó a todos los gobernadores y capitanes y a los oficiales del ejército de Asur;
\par 15 Y reunió a los hombres escogidos para la batalla, tal como su señor le había ordenado, ciento veinte mil y doce mil arqueros a caballo;
\par 16 Y los puso en orden, como se ordena un gran ejército para la guerra.
\par 17 Y tomó camellos y asnos para sus carros, en gran número; y ovejas, bueyes y cabras sin número para su provisión:
\par 18 Y víveres en abundancia para cada hombre del ejército, y mucho oro y plata de la casa del rey.
\par 19 Luego salió con todo su poder para ir delante del rey Nabucodonosor en el viaje y cubrir toda la faz de la tierra hacia el oeste con sus carros, su gente de a caballo y su gente de a pie escogida.
\par 20 Y vinieron con ellos gran número de países diversos, como langostas y como arena de la tierra, porque la multitud era incontable.
\par 21 Y partieron desde Nínive, camino de tres días, hacia la llanura de Bectileth, y acamparon desde Bectileth, cerca de la montaña que está a la izquierda de la alta Cilicia.
\par 22 Entonces tomó todo su ejército, su gente de a pie, su gente de a caballo y sus carros, y de allí se fue a la región montañosa;
\par 23 Y destruyó a Fud y a Lud, y despojó a todos los hijos de Rasés y a los hijos de Israel que estaban hacia el desierto, al sur de la tierra de los quelianos.
\par 24 Luego cruzó el Éufrates, atravesó Mesopotamia y destruyó todas las ciudades altas que estaban junto al río Arbonai, hasta llegar al mar.
\par 25 Y tomó las fronteras de Cilicia, y mató a todos los que se le resistían, y llegó a las fronteras de Jafet, que estaban hacia el sur, frente a Arabia.
\par 26 También rodeó a todos los hijos de Madián, quemó sus tiendas y saqueó sus majadas.
\par 27 Luego descendió a la llanura de Damasco en el tiempo de la cosecha del trigo, y quemó todos sus campos, destruyó sus ovejas y sus vacas, también despojó sus ciudades, devastó sus tierras y mató a todos sus jóvenes hombres a filo de espada.
\par 28 Por lo tanto, el temor y el temor de él cayeron sobre todos los habitantes de las costas del mar, que estaban en Sidón y Tiro, y sobre los que habitaban en Sur y Ocina, y sobre todos los que habitaban en Jemnaan; y los que habitaban en Azoto y Ascalón le temían mucho.

\chapter{3}

\par 1 Entonces le enviaron embajadores para tratar de paz, diciendo:
\par 2 He aquí, nosotros, los servidores del gran rey Nabucodonosor, yacemos ante ti; Úsanos como sea bueno delante de tus ojos.
\par 3 He aquí, nuestras casas y todos nuestros lugares, y todos nuestros campos de trigo, y ovejas y vacas, y todas las cabañas de nuestras tiendas yacen delante de ti; úsalos como te plazca.
\par 4 He aquí, incluso nuestras ciudades y sus habitantes son tus siervos; ven y trata con ellos como mejor te parezca.
\par 5 Entonces los hombres fueron a Holofernes y le dijeron lo siguiente.
\par 6 Luego descendió hacia la costa del mar, él y su ejército, y puso guarniciones en las ciudades altas, y de ellas tomó hombres escogidos para ayudar.
\par 7 Entonces ellos y toda la región de alrededor los recibieron con guirnaldas, con danzas y con panderos.
\par 8 Sin embargo, derribó sus fronteras y taló sus bosques, porque había decretado destruir todos los dioses de la tierra, que todas las naciones adorarían únicamente a Nabucodonosor y que todas las lenguas y tribus lo invocarían como dios.
\par 9 También pasó contra Esdrelón, cerca de Judea, frente al gran estrecho de Judea.
\par 10 Acampó entre Geba y Escitópolis y permaneció allí un mes entero para reunir todos los carros de su ejército.

\chapter{4}

\par 1 Los hijos de Israel, que habitaban en Judea, oyeron todo lo que Holofernes, el capitán de Nabucodonosor, rey de los asirios, había hecho a las naciones, y cómo había saqueado y destruido todos sus templos.
\par 2 Por eso temieron mucho de él y se preocuparon por Jerusalén y por el templo del Señor su Dios.
\par 3 Porque recién habían regresado del cautiverio, y todo el pueblo de Judea se había reunido recientemente; y los vasos, el altar y la casa fueron santificados después de la profanación.
\par 4 Enviaron, pues, a todo el término de Samaria, a las aldeas, a Bet-orón, a Belmen, a Jericó, a Choba, a Esora y al valle de Salem:
\par 5 Y se apoderaron de antemano de todas las cimas de las altas montañas, fortificaron las aldeas que había en ellas y almacenaron provisiones para la guerra, porque sus campos ya habían sido segados recientemente.
\par 6 También el sumo sacerdote Joaquín, que estaba en aquellos días en Jerusalén, escribió a los que habitaban en Betulia y en Betomestham, que está frente a Esdrelón, hacia la campiña, cerca de Dotaim,
\par 7 Les ordenó que guardaran los pasos de la región montañosa, porque por ellos había una entrada a Judea, y era fácil detener a los que subían, porque el paso era recto, para dos hombres como máximo.
\par 8 Y los hijos de Israel, junto con los ancianos de todo el pueblo de Israel que habitaban en Jerusalén, hicieron como les había ordenado el sumo sacerdote Joaquín.
\par 9 Entonces todos los israelitas clamaron a Dios con gran fervor y con gran vehemencia humillaron sus almas:
\par 10 Tanto ellos como sus mujeres, sus hijos, sus ganados, todos los extranjeros y asalariados y sus sirvientes comprados con dinero se visten de cilicio sobre sus lomos.
\par 11 Así, todos los hombres y mujeres, los niños y los habitantes de Jerusalén se postraron ante el templo, se echaron ceniza sobre la cabeza y extendieron su cilicio delante del Señor; también se pusieron cilicio alrededor del altar,
\par 12 Y clamaron todos de común acuerdo al Dios de Israel, para que no entregara a sus hijos por presa, ni a sus mujeres por botín, ni las ciudades de su herencia por destrucción, ni el santuario por profanación y oprobio. , y para que las naciones se regocijen.
\par 13 Entonces Dios escuchó sus oraciones y miró sus aflicciones: porque el pueblo ayunó muchos días en toda Judea y Jerusalén ante el santuario del Señor Todopoderoso.
\par 14 Y Joaquín, el sumo sacerdote y todos los sacerdotes que estaban delante del Señor y los que servían al Señor, estaban ceñidos sus lomos con cilicio y ofrecían los holocaustos diarios, con los votos y las ofrendas del pueblo.
\par 15 Tenían cenizas en sus mitras y clamaban con todas sus fuerzas al Señor para que mirara con bondad a toda la casa de Israel.

\chapter{5}

\par 1 Entonces fue informado a Holofernes, capitán en jefe del ejército de Asur, que los hijos de Israel se habían preparado para la guerra, habían cerrado los pasos de la región montañosa y habían fortificado todas las cimas de las colinas altas y habían puesto impedimentos en los países de campaña:
\par 2 Por lo cual se enojó mucho y llamó a todos los príncipes de Moab, a los capitanes de Amón y a todos los gobernadores de la costa del mar,
\par 3 Y él les dijo: Decidme ahora, hijos de Canaán, quién es este pueblo que habita en la región montañosa, y cuáles son las ciudades que habitan, y cuál es la multitud de su ejército, y en qué es su poder y fuerza, y qué rey está sobre ellos, o capitán de su ejército;
\par 4 ¿Y por qué han decidido no venir a mi encuentro, más que todos los habitantes del oeste?
\par 5 Entonces dijo Achior, capitán de todos los hijos de Amón: Oiga ahora mi señor una palabra de boca de tu siervo, y te declararé la verdad acerca de este pueblo que habita cerca de ti y habita en el tierras montañosas, y ninguna mentira saldrá de la boca de tu siervo.
\par 6 Este pueblo es descendiente de los caldeos:
\par 7 Y hasta entonces residían en Mesopotamia, porque no querían seguir a los dioses de sus padres, que estaban en la tierra de Caldea.
\par 8 Porque abandonaron el camino de sus antepasados ​​y adoraron al Dios del cielo, al Dios que conocían; por eso los expulsaron de la presencia de sus dioses y huyeron a Mesopotamia, donde permanecieron muchos días.
\par 9 Entonces su Dios les ordenó que abandonaran el lugar donde moraban y se dirigieran a la tierra de Canaán, donde habitaron y se enriquecieron con oro y plata y con mucho ganado.
\par 10 Pero cuando el hambre cubrió toda la tierra de Canaán, descendieron a Egipto y permanecieron allí mientras se alimentaban, y se convirtieron allí en una gran multitud, de modo que no se podía contar su nación.
\par 11 Por eso el rey de Egipto se levantó contra ellos, los trató con astucia, los sometió a trabajar en ladrillos y los hizo esclavos.
\par 12 Entonces clamaron a su Dios, y él hirió toda la tierra de Egipto con plagas incurables; por eso los egipcios los echaron de su vista.
\par 13 Y Dios secó el mar Rojo delante de ellos,
\par 14 Y los llevó al monte Sina y a Cades-Barne, y expulsó a todos los que habitaban en el desierto.
\par 15 Y habitaron en la tierra de los amorreos, y con su fuerza destruyeron a todos los de Esebón, y pasando el Jordán se apoderaron de toda la región montañosa.
\par 16 Y echaron delante de ellos a los cananeos, a los ferezeos, a los jebuseos, a los siquemitas y a todos los gergeseos, y vivieron en aquella tierra muchos días.
\par 17 Y aunque no pecaron ante su Dios, prosperaron, porque el Dios que aborrece la iniquidad estaba con ellos.
\par 18 Pero cuando se apartaron del camino que él les había señalado, fueron destruidos en muchas batallas muy duras, y fueron llevados cautivos a una tierra que no era la suya, y el templo de su Dios fue derribado por tierra, y sus Las ciudades fueron tomadas por los enemigos.
\par 19 Pero ahora han vuelto a su Dios, han subido de los lugares donde estaban dispersos, han poseído Jerusalén, donde está su santuario, y se han asentado en la región montañosa; porque estaba desolado.
\par 20 Ahora pues, señor y gobernador mío, si hay algún error contra este pueblo y pecan contra su Dios, consideremos que esto será su ruina, y subamos y los venceremos.
\par 21 Pero si no hay iniquidad en su nación, pase ahora mi señor, no sea que su Señor los defienda, y su Dios esté con ellos, y seamos oprobio ante todo el mundo.
\par 22 Cuando Achior terminó estas palabras, todo el pueblo que estaba alrededor de la tienda murmuró, y los principales de Holofernes y todos los que habitaban junto al mar y en Moab dijeron que lo matara.
\par 23 Porque, dicen, no temeremos delante de los hijos de Israel; porque he aquí, es un pueblo que no tiene fuerza ni fuerza para una batalla fuerte.
\par 24 Ahora pues, señor Holofernes, subiremos y serán presa para devorar a todo tu ejército.

\chapter{6}

\par 1 Y cuando cesó el tumulto de los hombres que estaban alrededor del consejo, Holofernes, el capitán principal del ejército de Asur, dijo a Achior y a todos los moabitas delante de toda la compañía de otras naciones:
\par 2 ¿Y quién eres tú, Ajior y los asalariados de Efraín, que has profetizado contra nosotros como hoy, y has dicho que no debemos hacer guerra contra los hijos de Israel, porque su Dios los defenderá? ¿Y quién es Dios sino Nabucodonosor?
\par 3 Él enviará su poder y los destruirá de la faz de la tierra, y su Dios no los librará; pero nosotros, sus siervos, los destruiremos como a un solo hombre; porque no pueden sostener la potencia de nuestros caballos.
\par 4 Porque con ellos los hollaremos, y sus montañas se embriagarán con su sangre, y sus campos se llenarán de sus cadáveres, y sus pasos no podrán sostenerse delante de nosotros, porque completamente serán destruidos, dice el rey Nabucodonosor, señor de toda la tierra; porque dijo: Ninguna de mis palabras serán en vano.
\par 5 Y tú, Achior, asalariado de Amón, que pronunciaste estas palabras el día de tu iniquidad, no volverás a ver mi rostro desde hoy, hasta que tome venganza de esta nación que salió de Egipto.
\par 6 Y entonces la espada de mi ejército y la multitud de mis servidores pasarán por tus costados, y tú caerás entre sus muertos cuando yo regrese.
\par 7 Ahora pues, mis siervos te llevarán de vuelta a la región montañosa y te instalarán en una de las ciudades de los pasos.
\par 8 Y no perecerás hasta que seas destruido con ellos.
\par 9 Y si estás convencido de que serán apresados, no desmayes tu rostro: lo he dicho, y ninguna de mis palabras será en vano.
\par 10 Entonces Holofernes ordenó a sus servidores que esperaban en su tienda que tomaran a Achior y lo llevaran a Betulia y lo entregaran en manos de los hijos de Israel.
\par 11 Entonces sus siervos lo tomaron y lo sacaron del campamento a la llanura, y desde en medio de la llanura pasaron a la montaña, y llegaron a las fuentes que estaban debajo de Betulia.
\par 12 Y cuando los hombres de la ciudad los vieron, tomaron sus armas y salieron de la ciudad a la cima de la colina; y todos los que usaban una honda les impedían acercarse arrojando piedras contra ellos.
\par 13 Sin embargo, habiendo llegado en secreto debajo de la colina, ataron a Achior, lo arrojaron abajo, lo dejaron al pie de la colina y regresaron a su señor.
\par 14 Pero los israelitas descendieron de su ciudad, vinieron a él, lo soltaron, lo llevaron a Betulia y lo presentaron a los gobernadores de la ciudad.
\par 15 Los cuales eran en aquellos días Ozías hijo de Micaía, de la tribu de Simeón, Cabris hijo de Gotoniel y Carmis hijo de Melquiel.
\par 16 Y reunieron a todos los ancianos de la ciudad, y todos sus jóvenes y sus mujeres corrieron juntos a la asamblea, y pusieron a Ajior en medio de todo su pueblo. Entonces Ozías le preguntó qué había hecho.
\par 17 Él respondió y les contó las palabras del concilio de Holofernes, y todas las palabras que había hablado en medio de los príncipes de Asur, y todo lo que Holofernes había hablado con orgullo contra la casa de Israel.
\par 18 Entonces el pueblo se postró, adoró a Dios y clamó a Dios diciendo,
\par 19 Oh Señor, Dios del cielo, mira su orgullo, y compadécete de la humillación de nuestra nación, y mira el rostro de aquellos que hoy son santificados para ti.
\par 20 Entonces consolaron a Ajior y lo alabaron mucho.
\par 21 Y Ozías lo sacó de la asamblea y lo llevó a su casa, e hizo un banquete a los ancianos; y toda aquella noche pidieron ayuda al Dios de Israel.

\chapter{7}

\par 1 Al día siguiente, Holofernes ordenó a todo su ejército y a todo su pueblo que habían venido a unirse a él, que trasladaran su campamento frente a Betulia, para tomar de antemano las subidas de la región montañosa y hacer la guerra contra los hijos de Israel.
\par 2 Aquel día sus hombres fuertes levantaron sus campamentos, y el ejército de los hombres de guerra era ciento setenta mil hombres de a pie y doce mil de a caballo, sin contar los bagajes, y otros hombres que iban entre ellos a pie, una gran multitud.
\par 3 Y acamparon en el valle cerca de Betulia, junto a la fuente, y se extendieron a lo ancho desde Dotaim hasta Belmaim, y a lo largo desde Betulia hasta Cinamo, que está frente a Esdrelón.
\par 4 Los hijos de Israel, al ver su multitud, se turbaron mucho y decían cada uno a su prójimo: Ahora estos hombres lamerán la faz de la tierra; porque ni las altas montañas, ni los valles, ni las colinas, pueden soportar su peso.
\par 5 Entonces cada uno tomó sus armas de guerra y, después de haber encendido fuego en sus torres, se quedaron velando toda esa noche.
\par 6 Pero el segundo día, Holofernes sacó toda su gente de a caballo ante los ojos de los hijos de Israel que estaban en Betulia,
\par 7 Y observó los pasajes que conducían a la ciudad, y llegó a las fuentes de sus aguas, las tomó y puso guarniciones de hombres de guerra sobre ellas, y él mismo se dirigió hacia su pueblo.
\par 8 Entonces vinieron a él todos los jefes de los hijos de Esaú, y todos los gobernadores del pueblo de Moab, y los capitanes de la costa del mar, y le dijeron:
\par 9 Que nuestro señor escuche ahora una palabra para que no haya derrota en tu ejército.
\par 10 Porque este pueblo de los hijos de Israel no confía en sus lanzas, sino en la altura de las montañas en las que habita, porque no es fácil subir a las cimas de sus montañas.
\par 11 Ahora pues, señor mío, no pelees contra ellos en orden de batalla, y no perecerá ni un solo hombre de tu pueblo.
\par 12 Quédate en tu campamento y guarda a todos los hombres de tu ejército, y deja que tus siervos tomen en sus manos la fuente de agua que brota al pie de la montaña.
\par 13 Porque allí tienen agua todos los habitantes de Betulia; Entonces la sed los matará, y abandonarán su ciudad, y nosotros y nuestro pueblo subiremos a las cumbres de los montes que están cerca, y acamparemos sobre ellos, para vigilar que nadie salga de la ciudad.
\par 14 Así, ellos, sus mujeres y sus hijos serán consumidos por el fuego, y antes de que la espada venga contra ellos, serán derribados en las calles donde habitan.
\par 15 Así les darás una mala recompensa; porque se rebelaron y no encontraron paz contigo.
\par 16 Y estas palabras agradaron a Holofernes y a todos sus servidores, y ordenó hacer lo que habían dicho.
\par 17 Entonces partió el campamento de los hijos de Amón, y con ellos cinco mil de los asirios, y acamparon en el valle y tomaron las aguas y las fuentes de agua de los hijos de Israel.
\par 18 Entonces los hijos de Esaú subieron con los hijos de Amón y acamparon en la región montañosa frente a Dotaim; y enviaron algunos de ellos hacia el sur y hacia el este, frente a Ekrebel, que está cerca de Chusi, eso está sobre el arroyo Mochmur; y el resto del ejército de los asirios acampó en la llanura, y cubrió la faz de toda la tierra; y se levantaron sus tiendas y carruajes ante una gran multitud.
\par 19 Entonces los hijos de Israel clamaron al Señor su Dios, porque su corazón desfalleció, porque todos sus enemigos los habían rodeado y no había manera de escapar de en medio de ellos.
\par 20 Así permaneció alrededor de ellos toda la compañía de Asur, tanto su infantería como sus carros y su gente de a caballo, durante treinta y cuatro días, de modo que todos sus vasos de agua acabaron con todos los inhibidores de Betulia.
\par 21 Y las cisternas se vaciaron, y durante un día no tuvieron agua para beber hasta saciarse; porque les dieron de beber por medida.
\par 22 Por eso sus niños se desanimaron, y sus mujeres y sus jóvenes desmayaron de sed, y cayeron en las calles de la ciudad y junto a las puertas, y ya no tenían fuerzas.
\par 23 Entonces todo el pueblo se reunió con Ozías y con el jefe de la ciudad, tanto jóvenes como mujeres y niños, y clamaron a gran voz y dijeron delante de todos los ancianos:
\par 24 Dios sea juez entre nosotros y vosotros, porque nos habéis hecho un gran daño al no exigir la paz de los hijos de Asur.
\par 25 Porque ahora no tenemos quien nos ayude; pero Dios nos ha vendido en sus manos, para que seamos arrojados delante de ellos con sed y gran destrucción.
\par 26 Ahora pues, llámalos y entrega toda la ciudad como botín a los habitantes de Holofernes y a todo su ejército.
\par 27 Porque mejor nos es ser despojado para ellos que morir de sed; porque seremos sus siervos, para que nuestras almas vivan y no veamos ante nuestros ojos la muerte de nuestros niños, ni nuestras esposas ni nuestros hijos muriendo.
\par 28 Tomamos por testigos contra vosotros al cielo y a la tierra, y a nuestro Dios y Señor de nuestros padres, que nos castiga según nuestros pecados y los pecados de nuestros padres, para no hacer como hemos dicho hoy.
\par 29 Entonces hubo un gran llanto unánime en medio de la asamblea; y clamaron a Jehová Dios en alta voz.
\par 30 Entonces Ozías les dijo: Hermanos, confiad, aguantaremos aún cinco días, en los cuales el Señor nuestro Dios volverá a nosotros su misericordia; porque él no nos desamparará del todo.
\par 31 Y si pasan estos días y no nos llega ninguna ayuda, haré según tu palabra.
\par 32 Y dispersó al pueblo, cada uno por su cuenta; y fueron a los muros y torres de su ciudad, y enviaron a las mujeres y a los niños a sus casas; y fueron muy humillados en la ciudad.

\chapter{8}

\par 1 En aquel tiempo lo oyó Judit, hija de Merari, hijo de Buey, hijo de José, hijo de Ozel, hijo de Elcia, hijo de Ananías, hijo de Gedeón, hijo de de Rafaim, hijo de Acitho, hijo de Eliu, hijo de Eliab, hijo de Natanael, hijo de Samael, hijo de Salasadal, hijo de Israel.
\par 2 Y Manasés era su marido, de su tribu y de su parentela, que murió en la cosecha de la cebada.
\par 3 Mientras estaba vigilando a los que ataban gavillas en el campo, el calor le afectó la cabeza, cayó en su cama y murió en la ciudad de Betulia; y lo sepultaron con sus padres en el campo entre Dotaim y Balamo.
\par 4 Judit quedó viuda en su casa durante tres años y cuatro meses.
\par 5 Y le hizo una tienda en el terrado de su casa, y se vistió de cilicio sobre sus lomos y vistió su ropa de viuda.
\par 6 Y ayunó todos los días de su viudez, excepto las vísperas de los sábados, y de los sábados, y las vísperas de luna nueva, y de luna nueva, y de las fiestas y días solemnes de la casa de Israel.
\par 7 Ella también era de hermoso semblante y muy hermosa a la vista; y su marido Manasés le había dejado oro y plata, siervos y siervas, ganado y tierras; y ella permaneció sobre ellos.
\par 8 Y no hubo nadie que le hablara mal; y ella temía mucho a Dios.
\par 9 Cuando oyó las malas palabras del pueblo contra el gobernador, que desmayaban por falta de agua, porque Judith había oído todas las palabras que Ozías les había hablado, y que había jurado entregar la ciudad a los asirios después de cinco días;
\par 10 Entonces envió a su criada, que gobernaba todo lo que ella tenía, a llamar a Ozías, a Chabris y a Carmis, los ancianos de la ciudad.
\par 11 Y vinieron a ella, y ella les dijo: Oídme ahora, gobernadores de los habitantes de Betulia; porque las palabras que habéis hablado hoy delante del pueblo no son correctas acerca de este juramento que habéis hecho y pronunciado entre Dios y vosotros, y hemos prometido entregar la ciudad a nuestros enemigos, a menos que dentro de estos días el Señor vuelva en vuestro socorro.
\par 12 ¿Y ahora quiénes sois vosotros, que hoy tentáis a Dios y estáis en lugar de Dios entre los hijos de los hombres?
\par 13 Ahora, pues, probad al Señor Todopoderoso, pero nunca sabréis nada.
\par 14 Porque no podéis encontrar la profundidad del corazón del hombre, ni podéis percibir las cosas que piensa. Entonces, ¿cómo podréis buscar a Dios, que ha hecho todas estas cosas, y conocer su mente, o comprender su propósito? Es más, hermanos míos, no provoquéis a ira al Señor nuestro Dios.
\par 15 Porque si no nos ayuda en estos cinco días, tiene poder para defendernos cuando quiera, incluso todos los días, o para destruirnos delante de nuestros enemigos.
\par 16 No obligues a los consejos del Señor nuestro Dios, porque Dios no es como un hombre para ser amenazado; ni es como hijo de hombre, para que vacile.
\par 17 Por tanto, esperemos su salvación e invoquémosle para que nos ayude, y él escuchará nuestra voz, si le place.
\par 18 Porque ni ha surgido en nuestra época, ni hay hoy entre nosotros tribu, ni familia, ni pueblo, ni ciudad, que adore a dioses hechos con manos, como antes.
\par 19 Por esta causa nuestros padres fueron entregados a la espada y al botín, y sufrieron una gran caída ante nuestros enemigos.
\par 20 Pero como no conocemos ningún otro dios, confiamos en que él no nos despreciará ni a nosotros ni a nadie de nuestra nación.
\par 21 Porque si así nos toman, toda Judea quedará desierta y nuestro santuario será saqueado; y exigirá su profanación de nuestra boca.
\par 22 Y la matanza de nuestros hermanos, el cautiverio de la tierra y la devastación de nuestra herencia, volverá sobre nuestras cabezas entre los gentiles, dondequiera que estemos en servidumbre; y seremos escandaloso y oprobio a todos los que nos poseen.
\par 23 Porque nuestra servidumbre no será para favor, sino que el Señor nuestro Dios la convertirá en deshonra.
\par 24 Ahora pues, hermanos, demos ejemplo a nuestros hermanos, porque sus corazones dependen de nosotros, y el santuario, la casa y el altar dependen de nosotros.
\par 25 Además, demos gracias al Señor nuestro Dios, que nos prueba como lo hizo con nuestros padres.
\par 26 Acordaos de lo que le hizo a Abraham, de cómo probó a Isaac y de lo que le sucedió a Jacob en Mesopotamia de Siria, cuando cuidaba las ovejas de Labán, el hermano de su madre.
\par 27 Porque no nos probó en el fuego como a ellos, para examinar sus corazones, ni se vengó de nosotros, sino que el Señor azota a los que a él se acercan para amonestarlos.
\par 28 Entonces Ozías le dijo: Todo lo que has dicho, lo has dicho con buen corazón, y no hay nadie que pueda contradecir tus palabras.
\par 29 Porque éste no es el primer día en que se manifiesta tu sabiduría; pero desde el principio de tus días todo el pueblo ha conocido tu entendimiento, porque la disposición de tu corazón es buena.
\par 30 Pero el pueblo tenía mucha sed y nos obligó a hacerles lo que habíamos dicho y a hacernos un juramento que no romperemos.
\par 31 Por tanto, ahora ora por nosotros, porque eres una mujer piadosa, y el Señor nos enviará lluvia para llenar nuestras cisternas, y no desmayaremos más.
\par 32 Entonces Judit les dijo: Oídme, y haré algo que se repetirá por todas las generaciones de los hijos de nuestra nación.
\par 33 Estaréis esta noche a la puerta, y yo saldré con mi criada; y dentro de los días que habéis prometido entregar la ciudad a nuestros enemigos, el Señor visitará a Israel por mi mano.
\par 34 Pero no preguntéis por mis actos, porque no os lo declararé hasta que haya terminado lo que hago.
\par 35 Entonces Ozías y los príncipes le dijeron: Ve en paz, y el Señor Dios esté delante de ti para vengarse de nuestros enemigos.
\par 36 Entonces ellos regresaron de la tienda y se dirigieron a sus guardias.

\chapter{9}

\par 1 Judit se postró sobre su rostro, se puso ceniza en la cabeza y descubrió el cilicio con que estaba vestida. Y a la hora en que se ofrecía el incienso de aquella tarde en Jerusalén, en la casa del Señor, Judit clamó a gran voz y dijo:
\par 2 Oh Señor, Dios de mi padre Simeón, a quien diste la espada para vengarse de los extranjeros, que desató el cinto de una doncella para contaminarla, y descubrió el muslo para su vergüenza, y profanó su virginidad para su oprobio. ; porque dijiste: No será así; y sin embargo lo hicieron:
\par 3 Por eso entregaste a sus gobernantes para que los mataran, de modo que teñieran su cama en sangre, siendo engañados, e hirieran a los siervos con sus señores, y a los señores en sus tronos;
\par 4 Y has entregado a sus mujeres por botín, y a sus hijas en cautiverio, y todo su botín para repartir entre tus queridos hijos; que se conmovieron con tu celo, y aborrecieron la contaminación de su sangre, y te pidieron ayuda: Oh Dios, oh Dios mío, escúchame también a mí, viuda.
\par 5 Porque no sólo hiciste esas cosas, sino también las que sucedieron antes y las que siguieron después; has pensado en las cosas que son ahora y en las que están por venir.
\par 6 Sí, lo que habías determinado estaba a la mano, y dijiste: He aquí, estamos aquí; porque todos tus caminos están preparados, y tus juicios están en tu conocimiento previo.
\par 7 Porque he aquí que los asirios se han multiplicado en su poder; son exaltados con el caballo y el hombre; se jactan de la fuerza de sus lacayos; confían en el escudo, la lanza, el arco y la honda; y no sabes que tú eres el Señor que rompe las batallas: el Señor es tu nombre.
\par 8 Derriba sus fuerzas con tu poder, y derriba sus fuerzas con tu ira; porque se han propuesto profanar tu santuario, y contaminar el tabernáculo donde reposa tu glorioso nombre, y derribar con espada el cuerno de tu altar.
\par 9 Mira su soberbia y envía tu ira sobre sus cabezas; entrega en mis manos, que soy viuda, el poder que he concebido.
\par 10 Hiere con el engaño de mis labios al siervo con el príncipe, y al príncipe con el siervo; derriba su majestuosidad con mano de mujer.
\par 11 Porque tu poder no está en la multitud, ni tu poder en los hombres fuertes; porque tú eres un Dios de los afligidos, un ayudante de los oprimidos, un sostenedor de los débiles, un protector de los desamparados, un salvador de los que están Sin esperanza.
\par 12 Te ruego, te ruego, Dios de mi padre y Dios de la herencia de Israel, Señor de los cielos y de la tierra, Creador de las aguas, Rey de toda criatura, escucha mi oración:
\par 13 Y haz que mi palabra y mi engaño sean tu herida y tu azote, los que han planeado cosas crueles contra tu alianza y tu casa santa, contra la cumbre de Sión y contra la casa de posesión de tus hijos.
\par 14 Y haz que cada nación y tribu reconozca que tú eres el Dios de todo poder y poder, y que no hay otro que proteja al pueblo de Israel excepto tú.

\chapter{10}

\par 1 Después de esto, ella dejó de invocar al Dios de Israel y puso fin a todas estas palabras.
\par 2 Se levantó del lugar donde había caído, llamó a su sierva y descendió a la casa donde moraba en los días de reposo y en los días de fiesta.
\par 3 Y se quitó el cilicio que vestía, se despojó de los vestidos de su viudez, lavó todo su cuerpo con agua, se ungió con ungüento precioso, se trenzó el cabello y se puso una tiara sobre él, y se puso las vestiduras de alegría con que estuvo vestida durante la vida de Manasés su marido.
\par 4 Y se calzó sandalias, se puso sus brazaletes, sus cadenas, sus anillos, sus aretes y todos sus adornos, y se atavió valientemente para atraer los ojos de todos los hombres que la vieran.
\par 5 Luego dio a su criada un odre de vino y una vasija de aceite, y llenó un costal con trigo tostado, trozos de higos y pan fino; Entonces ella dobló todas estas cosas y se las puso encima.
\par 6 Así que salieron a la puerta de la ciudad de Betulia, y encontraron allí a Ocías y a los ancianos de la ciudad, Cabris y Carmis.
\par 7 Y cuando vieron que su rostro había cambiado y su ropa cambiada, se maravillaron mucho de su belleza y le dijeron.
\par 8 Que el Dios, el Dios de nuestros padres, te conceda tu favor y lleve a cabo tus empresas para gloria de los hijos de Israel y para la exaltación de Jerusalén. Luego adoraron a Dios.
\par 9 Y ella les dijo: Mandad que me abran las puertas de la ciudad, para que pueda salir a cumplir lo que habéis hablado conmigo. Entonces ordenaron a los jóvenes que le abrieran, tal como ella había dicho.
\par 10 Y cuando hubieron terminado, salió Judit, ella y su criada con ella; y los hombres de la ciudad la cuidaron, hasta que bajó de la montaña, y hasta que pasó el valle, y no pudo verla más.
\par 11 Y así avanzaron directamente hacia el valle, cuando la primera guardia de los asirios salió a su encuentro,
\par 12 Y tomándola, le preguntó: ¿De qué pueblo eres? ¿Y de dónde vienes? ¿Y adónde vas? Y ella dijo: Soy mujer hebrea, y de ellos huí, porque os serán entregados para que los consumáis.
\par 13 Y voy a presentarme ante Holofernes, capitán en jefe de tu ejército, para declarar palabras de verdad; y le mostraré el camino por el cual irá y conquistará toda la región montañosa, sin perder el cuerpo ni la vida de ninguno de sus hombres.
\par 14 Cuando los hombres oyeron sus palabras y vieron su rostro, se maravillaron mucho de su belleza y le dijeron:
\par 15 Has salvado tu vida porque te has apresurado a descender ante la presencia de nuestro señor. Ahora pues, ven a su tienda, y algunos de nosotros te guiarán hasta que te entreguen en sus manos.
\par 16 Y cuando estés delante de él, no temas en tu corazón, sino muéstrale conforme a tu palabra; y él te rogará bien.
\par 17 Entonces escogieron entre ellos cien hombres para que la acompañaran a ella y a su sierva; y la llevaron a la tienda de Holofernes.
\par 18 Entonces se produjo una conmoción en todo el campamento; porque entre las tiendas se oía su llegada, y la rodeaban mientras estaba fuera de la tienda de Holofernes, hasta que le avisaron de ella.
\par 19 Y se maravillaron de su belleza y admiraron a los hijos de Israel a causa de ella, y cada uno decía a su vecino: ¿Quién despreciará a este pueblo que tiene mujeres así? Ciertamente no es bueno que quede entre ellos un solo hombre que, al ser dejado ir, pueda engañar a toda la tierra.
\par 20 Y salieron los que estaban cerca de Holofernes, y todos sus servidores, y la metieron en la tienda.
\par 21 Holofernes yacía en su cama, bajo un dosel tejido de púrpura, oro, esmeraldas y piedras preciosas.
\par 22 Entonces le mostraron de ella; y salió delante de su tienda, con lámparas de plata delante de él.
\par 23 Y cuando Judit llegó ante él y sus sirvientes, todos se maravillaron de la belleza de su rostro; y ella se postró sobre su rostro y le hizo reverencia; y sus siervos la levantaron.

\chapter{11}

\par 1 Entonces Holofernes le dijo: Mujer, consuélate, no temas en tu corazón, porque nunca hice daño a nadie que quisiera servir a Nabucodonosor, rey de toda la tierra.
\par 2 Ahora pues, si tu pueblo que habita en las montañas no me hubiera despreciado, yo no habría alzado mi lanza contra ellos, sino que ellos mismos se han hecho estas cosas.
\par 3 Pero ahora dime por qué has huido de ellos y has venido a nosotros: porque has venido para salvaguardarnos; Ten buen consuelo, vivirás esta noche y la siguiente:
\par 4 Porque nadie te hará daño, sino que te suplicará bien, como lo hacen los siervos del rey Nabucodonosor mi señor.
\par 5 Entonces Judit le dijo: Recibe las palabras de tu siervo y deja que tu sierva hable en tu presencia, y no declararé mentira a mi señor esta noche.
\par 6 Y si sigues las palabras de tu sierva, Dios hará que todo suceda perfectamente por ti; y mi señor no fallará en sus propósitos.
\par 7 Vive Nabucodonosor, rey de toda la tierra, y vive su poder, que te envió para sustentar a todo ser viviente: porque junto a ti no sólo los hombres le servirán, sino también las bestias del campo y los animales. El ganado y las aves del cielo vivirán bajo tu poder bajo Nabucodonosor y toda su casa.
\par 8 Porque hemos oído hablar de tu sabiduría y de tus políticas, y se dice en toda la tierra que tú eres excelente en todo el reino, poderoso en conocimiento y maravilloso en hazañas de guerra.
\par 9 En cuanto a lo que Ajior habló en tu consejo, hemos oído sus palabras; porque los hombres de Betulia lo salvaron, y él les contó todo lo que te había dicho.
\par 10 Por tanto, oh señor y gobernador, no respetes su palabra; pero guárdalo en tu corazón, porque es verdad: porque nuestra nación no será castigada, ni la espada prevalecerá contra ellos, a menos que pequen contra su Dios.
\par 11 Y ahora, para que mi señor no sea derrotado y frustrado en su propósito, incluso la muerte ha caído sobre ellos, y su pecado los ha alcanzado, con el cual provocarán a ira a su Dios cada vez que hagan lo que no conviene hacer:
\par 12 Porque les faltan víveres y toda el agua escasea, y han decidido apoderarse de su ganado y se han propuesto consumir todas aquellas cosas que Dios les ha prohibido comer según sus leyes.
\par 13 Y decidieron gastar las primicias de los diezmos del vino y del aceite que habían santificado y reservado para los sacerdotes que sirven en Jerusalén delante de nuestro Dios; cosas que a ninguno del pueblo le es lícito tocar con las manos.
\par 14 Porque han enviado algunos a Jerusalén, porque también los que allí habitan han hecho lo mismo, para traerles una licencia del Senado.
\par 15 Ahora bien, cuando les avisen, lo harán inmediatamente y te los entregarán para que los destruyas ese mismo día.
\par 16 Por eso yo, tu sierva, sabiendo todo esto, huí de su presencia; y Dios me ha enviado a hacer cosas contigo, de las cuales quedará estupefacta toda la tierra, y cuantos lo oigan.
\par 17 Porque tu siervo es religioso y sirve al Dios del cielo de día y de noche. Ahora pues, señor mío, me quedaré contigo, y tu siervo saldrá de noche al valle, y oraré a Dios, y él me dirá cuando hayan cometido sus pecados:
\par 18 Yo vendré y te lo mostraré: entonces saldrás con todo tu ejército y no habrá ninguno de ellos que te resista.
\par 19 Y te guiaré por en medio de Judea, hasta llegar delante de Jerusalén; y pondré tu trono en medio de él; y las ahuyentarás como a ovejas que no tienen pastor, y un perro ni siquiera abrirá ante ti su boca; porque estas cosas me fueron dichas según mi previo conocimiento, y me fueron declaradas, y soy enviado a contarlas. El e.
\par 20 Entonces sus palabras agradaron a Holofernes y a todos sus servidores; y se maravillaron de su sabiduría, y dijeron:
\par 21 No hay mujer así desde un extremo de la tierra hasta el otro, ni en belleza de rostro ni en sabiduría de palabras.
\par 22 Lo mismo le dijo Holofernes. Dios ha hecho bien en enviarte delante del pueblo, para que la fuerza esté en nuestras manos y la destrucción sobre los que menosprecian a mi señor.
\par 23 Y ahora eres hermoso en tu rostro e ingenioso en tus palabras. Ciertamente, si haces lo que has dicho, tu Dios será mi Dios, y habitarás en la casa del rey Nabucodonosor, y serás famoso por toda la tierra.

\chapter{12}

\par 1 Entonces mandó que la trajeran a donde estaba puesto su plato; y ordenó que le prepararan de sus propias comidas y que ella bebiera de su propio vino.
\par 2 Y Judit dijo: No comeré de ello, para que no haya escándalo; pero me proveerán de lo que he traído.
\par 3 Entonces Holofernes le dijo: Si falta tu provisión, ¿cómo te daremos otra igual? porque no habrá nadie con nosotros de tu nación.
\par 4 Entonces Judit le dijo: Vive tu alma, señor mío, que tu sierva no gastará lo que yo tengo, antes de que el Señor haga con mi mano lo que él ha determinado.
\par 5 Entonces los sirvientes de Holofernes la llevaron a la tienda, y ella durmió hasta medianoche, y se levantó cuando ya era la vigilia de la mañana.
\par 6 Y envió a Holofernes a decir: «Que mi señor ordene ahora que tu sierva salga a orar».
\par 7 Entonces Holofernes ordenó a su guardia que no la detuvieran, y ella permaneció tres días en el campamento y salió de noche al valle de Betulia y se lavó en una fuente de agua junto al campamento.
\par 8 Y cuando salió, rogó al Señor Dios de Israel que le dirigiera el camino para levantar a los hijos de su pueblo.
\par 9 Ella entró limpia y se quedó en la tienda hasta que comió su carne por la tarde.
\par 10 Y al cuarto día Holofernes hizo un banquete sólo para sus servidores, y no llamó a ninguno de los oficiales al banquete.
\par 11 Entonces dijo al eunuco Bagoas, que estaba a cargo de todo lo que tenía: Ve ahora y convence a esta mujer hebrea que está contigo para que venga a nosotros, y coma y beba con nosotros.
\par 12 Porque, he aquí, sería una vergüenza para nuestra persona si dejáramos ir a tal mujer sin haberla acompañado; porque si no la atraemos hacia nosotros, se burlará de nosotros.
\par 13 Entonces Bagoas salió de la presencia de Holofernes y se acercó a ella y le dijo: No tema esta hermosa joven venir a mi señor y ser honrada en su presencia, beber vino y divertirse con nosotros y serás hecha hoy como una de las hijas de los asirios, que sirven en la casa de Nabucodonosor.
\par 14 Entonces Judit le dijo: ¿Quién soy yo ahora para contradecir a mi señor? ciertamente todo lo que le agrada lo haré presto, y será mi gozo hasta el día de mi muerte.
\par 15 Entonces ella se levantó, se vistió con sus ropas y todos sus atavíos de mujer, y su doncella fue y le tendió en el suelo, frente a Holofernes, las pieles suaves que había recibido de Bagoas para su uso diario, para poder siéntate y come sobre ellos.
\par 16 Cuando Judit entró y se sentó, Holofernes se embelesó con su corazón y se conmovió su mente, y deseó mucho su compañía; porque esperó un tiempo para engañarla, desde el día en que la vio.
\par 17 Entonces Holofernes le dijo: Bebe ahora y alégrate con nosotros.
\par 18 Entonces Judit dijo: Beberé ahora, señor mío, porque mi vida se ha engrandecido en mí hoy más que todos los días desde que nací.
\par 19 Entonces ella tomó, comió y bebió delante de él lo que su criada había preparado.
\par 20 Y Holofernes se deleitó mucho con ella y bebió más vino del que había bebido en un solo día desde que nació.

\chapter{13}

\par 1 Cuando llegó la noche, sus sirvientes se apresuraron a partir, y Bagoas cerró su tienda afuera y despidió a los camareros de la presencia de su señor; y se acostaron, porque todos estaban cansados, porque la fiesta se había prolongado.
\par 2 Y Judit se quedó en la tienda, y Holofernes acostado en su cama, porque estaba harto de vino.
\par 3 Judit había ordenado a su doncella que se quedara fuera de su alcoba y la esperara salir, como lo hacía diariamente: porque dijo que iría a sus oraciones, y habló a Bagoas de acuerdo con el mismo propósito.
\par 4 Así que todos salieron y no quedó nadie en el dormitorio, ni pequeño ni grande. Entonces Judit, poniéndose en pie junto a su cama, dijo en su corazón: Oh Señor Dios todopoderoso, mira este presente en las obras de mis manos para la exaltación de Jerusalén.
\par 5 Porque ahora es el tiempo de ayudar a tu herencia y de ejecutar tus empresas para destruir a los enemigos que se han levantado contra nosotros.
\par 6 Entonces se acercó a la columna de la cama que estaba a la cabecera de Holofernes y bajó de allí su fajín.
\par 7 Y acercándose a su cama, tomándole el pelo de la cabeza, dijo: Fortaléceme hoy, Señor Dios de Israel.
\par 8 Y ella le golpeó dos veces en el cuello con todas sus fuerzas y le arrancó la cabeza.
\par 9 Y derribando su cuerpo de la cama, derribó el dosel de las columnas; Y poco después ella salió y entregó la cabeza de Holofernes a su doncella;
\par 10 Y ella lo puso en su bolsa de carne; y los dos fueron juntos según su costumbre a orar; y cuando pasaron el campamento, rodearon el valle, subieron a la montaña de Betulia y llegaron a las puertas del mismo.
\par 11 Entonces Judit dijo desde lejos a los centinelas de la puerta: Abrid, abrid ahora la puerta. Dios, nuestro Dios, está con nosotros para mostrar su poder todavía en Jerusalén y sus fuerzas contra el enemigo, como él incluso ha hecho este día.
\par 12 Cuando los hombres de su ciudad oyeron su voz, se apresuraron a descender a la puerta de su ciudad y llamaron a los ancianos de la ciudad.
\par 13 Y entonces todos corrieron juntos, tanto pequeños como grandes, porque les parecía extraño que ella hubiera venido; entonces abrieron la puerta, los recibieron, encendieron fuego para alumbrarse y se pararon alrededor de ellos.
\par 14 Entonces ella les dijo en alta voz: Alabad, alabad a Dios, alabad a Dios, os digo, porque él no ha quitado su misericordia a la casa de Israel, sino que esta noche ha destruido a nuestros enemigos con mis manos.
\par 15 Entonces ella sacó la cabeza de la bolsa, la mostró y les dijo: He aquí la cabeza de Holofernes, capitán en jefe del ejército de Asur, y he aquí el dosel donde yacía en su borrachera; y el Señor lo hirió por mano de mujer.
\par 16 Vive el Señor, que me ha guardado en el camino que anduve, que mi rostro lo ha engañado para su destrucción, y sin embargo él no ha cometido pecado conmigo para contaminarme y avergonzarme.
\par 17 Entonces todo el pueblo quedó estupefacto, se postraron y adoraron a Dios, y dijeron unánimes: Bendito seas, oh Dios nuestro, que hoy has destruido a los enemigos de tu pueblo.
\par 18 Entonces Ozías le dijo: Oh hija, bendita eres del Dios Altísimo sobre todas las mujeres de la tierra; y bendito sea el Señor Dios, que creó los cielos y la tierra, que te ordenó cortar la cabeza del jefe de nuestros enemigos.
\par 19 Por eso tu confianza no se apartará del corazón de los hombres, que recuerdan para siempre el poder de Dios.
\par 20 Y Dios te convierta estas cosas en alabanza perpetua, para visitarte con bienes, porque no has perdonado tu vida por la aflicción de nuestra nación, sino que has vengado nuestra ruina, andando por camino recto delante de nuestro Dios. Y todo el pueblo dijo; Que así sea, que así sea.

\chapter{14}

\par 1 Entonces Judit les dijo: Escuchen ahora, hermanos míos, y tomen esta cabeza y cuélguenla en lo más alto de sus muros.
\par 2 Y tan pronto como amanezca y salga el sol sobre la tierra, tomad cada uno sus armas y salid de la ciudad, cada uno de los valientes, y nombrad un capitán sobre ellos, como si descenderíais al campo hacia la guardia de los asirios; pero no bajes.
\par 3 Entonces tomarán sus armas y entrarán en su campamento, levantarán a los capitanes del ejército de Asur y correrán a la tienda de Holofernes, pero no lo encontrarán; entonces el miedo caerá sobre ellos, y huirán delante de ti.
\par 4 Así que vosotros y todos los habitantes de la costa de Israel los perseguiréis y los derrotaréis a su paso.
\par 5 Pero antes de hacer estas cosas, llámame Ajior el amonita, para que pueda ver y conocer al que despreciaba a la casa de Israel y lo envió a nosotros como si fuera a su muerte.
\par 6 Entonces llamaron a Achior de la casa de Ozías; y cuando llegó, y vio la cabeza de Holofernes en la mano de un hombre en la asamblea del pueblo, cayó de bruces y desfalleció su espíritu.
\par 7 Pero cuando lo recuperaron, cayó a los pies de Judit, la reverenció y dijo: Bendita eres en todos los tabernáculos de Judá y en todas las naciones que al oír tu nombre se asombrarán.
\par 8 Ahora, pues, cuéntame todas las cosas que has hecho en estos días. Entonces Judit le contó en medio del pueblo todo lo que había hecho, desde el día que salió hasta la hora en que les habló.
\par 9 Cuando ella terminó de hablar, el pueblo gritó a gran voz y alborotó en su ciudad.
\par 10 Y cuando Ajior vio todo lo que el Dios de Israel había hecho, creyó mucho en Dios y circuncidó la carne de su prepucio, y se unió a la casa de Israel hasta el día de hoy.
\par 11 Y tan pronto como amaneció, colgaron la cabeza de Holofernes en la pared, y cada uno tomó sus armas, y en grupos partieron hacia el estrecho de la montaña.
\par 12 Pero cuando los asirios los vieron, enviaron a sus jefes, quienes fueron a sus capitanes y tribunos, y a cada uno de sus gobernantes.
\par 13 Entonces llegaron a la tienda de Holofernes y dijeron al que estaba a cargo de todas sus cosas: Despierta ahora a nuestro señor, porque los esclavos se han atrevido a descender contra nosotros a la batalla para ser destruidos por completo.
\par 14 Entonces entró Bagoas y llamó a la puerta de la tienda; porque pensó que se había acostado con Judith.
\par 15 Pero como nadie respondía, la abrió, entró en el dormitorio y lo encontró muerto, tirado en el suelo, y le habían quitado la cabeza.
\par 16 Entonces gritó a gran voz, con llanto, con gemidos y con gran clamor, y rasgó sus vestiduras.
\par 17 Después entró en la tienda donde dormía Judit y, al no encontrarla, saltó hacia el pueblo y gritó:
\par 18 Estos esclavos han sido traicioneros; Una mujer hebrea ha avergonzado la casa del rey Nabucodonosor: porque he aquí, Holofernes yace en el suelo sin cabeza.
\par 19 Cuando los capitanes del ejército asirio oyeron estas palabras, rasgaron sus túnicas y sus mentes se turbaron maravillosamente, y hubo un grito y un gran ruido en todo el campamento.

\chapter{15}

\par 1 Y cuando lo oyeron los que estaban en las tiendas, quedaron asombrados de lo que había sucedido.
\par 2 Y el miedo y el temblor cayeron sobre ellos, de modo que nadie se atrevía a permanecer a la vista de su vecino, sino que, corriendo todos a una, huyeron por todos los caminos de la llanura y de la región montañosa.
\par 3 También los que habían acampado en las montañas alrededor de Betulia huyeron. Entonces los hijos de Israel, todos los que entre ellos eran guerreros, se abalanzaron sobre ellos.
\par 4 Entonces envió a Ozías a Betomasthem, a Bebai, a Chobai, a Cola y a todas las costas de Israel, para que contaran lo que había sucedido y que todos se lanzaran sobre sus enemigos para destruirlos.
\par 5 Cuando los hijos de Israel oyeron esto, todos a una se lanzaron sobre ellos y los mataron hasta Chobai; lo mismo hicieron también los que venían de Jerusalén y de toda la región montañosa (pues les habían dicho lo que había hecho en el campamento de sus enemigos) y los que estaban en Galaad y en Galilea, los persiguieron con gran matanza, hasta que pasaron Damasco y sus límites.
\par 6 Y los restantes que habitaban en Betulia cayeron sobre el campamento de Asur, los saquearon y se enriquecieron enormemente.
\par 7 Y los hijos de Israel que regresaron de la matanza se quedaron con lo que quedó; y las aldeas y las ciudades que estaban en la montaña y en la llanura tomaron mucho botín, porque la multitud era muy grande.
\par 8 Entonces vinieron el sumo sacerdote Joaquín y los ancianos de los hijos de Israel que habitaban en Jerusalén, para ver los bienes que Dios había mostrado a Israel, y para ver a Judit y saludarla.
\par 9 Y cuando llegaron a ella, la bendijeron unánimes y le dijeron: Tú eres la exaltación de Jerusalén, tú eres la gran gloria de Israel, tú eres el gran regocijo de nuestra nación.
\par 10 Has hecho todas estas cosas con tu mano; has hecho mucho bien a Israel, y Dios está complacido con ello: bendito seas del Señor Todopoderoso por los siglos. Y todo el pueblo dijo: Que así sea.
\par 11 Y el pueblo saqueó el campamento durante treinta días; y entregaron a Judit Holofernes su tienda, y toda su vajilla, sus camas, sus utensilios y todas sus pertenencias; y ella lo tomó y lo puso sobre su mula. ; y preparó sus carros y los puso sobre ellos.
\par 12 Entonces todas las mujeres de Israel corrieron juntas a verla, la bendijeron y bailaron entre ellas para ella; y ella tomó ramas en su mano y se las dio también a las mujeres que estaban con ella.
\par 13 Y le pusieron una guirnalda de olivo a ella y a su sierva que estaba con ella, y ella iba delante de todo el pueblo en la danza, guiando a todas las mujeres; y todos los hombres de Israel la seguían con sus armaduras y guirnaldas, y con canciones en la boca.

\chapter{16}

\par 1 Entonces Judit comenzó a cantar esta acción de gracias en todo Israel, y todo el pueblo cantaba tras ella este cántico de alabanza.
\par 2 Y Judit dijo: Comenzad a cantar a mi Dios con panderos, cantad a mi Señor con címbalos, cantadle un salmo nuevo, ensalzadle e invocad su nombre.
\par 3 Porque Dios rompe las batallas; porque en los campamentos, en medio del pueblo, me ha librado de las manos de los que me perseguían.
\par 4 Asur salió de las montañas del norte, con diez mil hombres de su ejército, cuya multitud detuvo los torrentes y sus jinetes cubrieron las colinas.
\par 5 Se jactaba de que quemaría mis fronteras, mataría a espada a mis jóvenes, estrellaría contra el suelo a los niños de pecho y convertiría a mis niños en presa y a mis vírgenes en botín.
\par 6 Pero el Señor Todopoderoso los ha decepcionado por mano de una mujer.
\par 7 Porque el valiente no cayó junto a los jóvenes, ni los hijos de los titanes lo derrotaron, ni los grandes gigantes lo atacaron, sino que Judit, la hija de Merari, lo debilitó con la belleza de su rostro.
\par 8 Porque se quitó el manto de su viudez para enaltecimiento de los oprimidos en Israel, se ungió el rostro con ungüento, se ató el cabello con una cinta y tomó un vestido de lino para engañarlo.
\par 9 Sus sandalias cautivaron sus ojos, su belleza cautivó su mente y el fauchion le atravesó el cuello.
\par 10 Los persas temblaron ante su audacia, y los medos se amedrentaron ante su valentía.
\par 11 Entonces mis afligidos gritaron de alegría, y mis débiles gritaron a gran voz; pero ellos quedaron atónitos: alzaron su voz, pero fueron derribados.
\par 12 Los hijos de las jóvenes los traspasaron y los hirieron como a hijos de fugitivos: perecieron en la batalla del Señor.
\par 13 Cantaré al Señor un cántico nuevo: Oh Señor, tú eres grande y glorioso, maravilloso en fuerza e invencible.
\par 14 Deja que todas las criaturas te sirvan: porque tú hablaste y fueron hechas, enviaste tu espíritu y él las creó, y no hay nadie que pueda resistir tu voz.
\par 15 Porque las montañas serán movidas por las aguas desde sus cimientos, las rocas se derretirán como cera ante tu presencia; pero tú eres misericordioso con los que te temen.
\par 16 Porque todo sacrificio es muy poco para ti como olor grato, y toda grasa no basta para tu holocausto; pero el que teme al Señor es grande en todo tiempo.
\par 17 ¡Ay de las naciones que se levantan contra mis hermanos! el Señor Todopoderoso se vengará de ellos en el día del juicio, poniendo fuego y gusanos en sus carnes; y los sentirán y llorarán por siempre.
\par 18 Tan pronto como entraron en Jerusalén, adoraron al Señor; y tan pronto como el pueblo fue purificado, ofrecieron sus holocaustos, sus ofrendas gratuitas y sus presentes.
\par 19 Judit también dedicó todos los objetos que el pueblo le había dado a Holofernes, y entregó el dosel que había sacado de su dormitorio como regalo al Señor.
\par 20 Así que el pueblo continuó festejando en Jerusalén delante del santuario durante tres meses y Judit se quedó con ellos.
\par 21 Después de este tiempo, cada uno volvió a su propia herencia, y Judit fue a Betulia, y permaneció en su posesión, y en su tiempo fue honorable en todo el país.
\par 22 Y muchos la deseaban, pero ninguno la conoció en todos los días de su vida; después que murió Manasés, su marido, y se reunió con su pueblo.
\par 23 Pero ella creció cada vez más en honor, y envejeció en la casa de su marido, cuando tenía ciento cinco años, y dejó libre a su sierva; Y murió en Betulia; y la sepultaron en la cueva de Manasés su marido.
\par 24 Y la casa de Israel la lamentó durante siete días; y antes de morir, distribuyó sus bienes a todos los parientes más cercanos de Manasés su marido, y a los más cercanos de su parentesco.
\par 25 Y ya no hubo nadie que aterrorizara a los hijos de Israel en los días de Judit, ni mucho tiempo después de su muerte.


\end{document}