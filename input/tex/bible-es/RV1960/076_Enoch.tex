\begin{document}

\title{Libro de Enoc}

\part{I. INTRODUCCIÓN}

\chapter{1}

\par 1 Las palabras de la bendición de Enoc, con la que bendijo a los elegidos y justos, que vivirán en el día de la tribulación, cuando todos los malvados e impíos serán eliminados.
\par 2 Y retomó su parábola y dijo: «Enoc, un hombre justo, cuyos ojos fueron abiertos por Dios, vio la visión del Santo en el cielo, que los ángeles me mostraron, y de ellos oí todo, y de ellos entendí lo que vi, pero no para esta generación, sino para la remota que está por venir».
\par 3 «Dije acerca de los elegidos y retomé mi parábola acerca de ellos: El Santo Grande saldrá de su morada».
\par 4 «Y el Dios eterno pisará la tierra, (incluso) el monte Sinaí, [y aparecerá de su campamento] y aparecerá con la fuerza de su poder desde el cielo de los cielos».
\par 5 «Y todos serán azotados por el miedo y los Vigilantes temblarán, y un gran temor y temblor se apoderarán de ellos hasta los confines de la tierra».
\par 6 «Y las altas montañas serán tembladas, y las altas colinas se abatirán, y se derretirán como cera ante la llama».
\par 7 «Y la tierra será completamente dividida, y todo lo que hay sobre la tierra perecerá, y habrá juicio sobre todos (los hombres)».
\par 8 Pero con los justos hará la paz. Y protegerá a los elegidos, y la misericordia será con ellos. Y todos serán de Dios, y serán prosperados, y todos serán benditos. Y Él los ayudará a todos, y se les aparecerá la luz, y Él hará las paces con ellos».
\par 9 «¡Y he aquí! Él viene con diez mil de sus santos para ejecutar juicio sobre todos, y para destruir a todos los impíos y para convencer a toda carne de todas las obras de su impiedad que han cometido impíamente, y de todas las cosas duras que los pecadores impíos han dicho. En su contra.»

\chapter{2}

\par 1 Observad todo lo que sucede en el cielo, cómo no cambian sus órbitas y las luminarias que están en el cielo, cómo todos se levantan y se ponen en orden cada uno en su tiempo, y no transgreden el orden establecido.
\par 2 Mirad la tierra y prestad atención a las cosas que en ella suceden desde el principio hasta el fin, cuán firmes son, cómo nada de lo que hay en la tierra cambia, sino que todas las obras de Dios se os aparecen a vosotros.
\par 3 Mirad el verano y el invierno, cómo toda la tierra está llena de agua, y sobre ella hay nubes, rocío y lluvia.

\chapter{3}

\par 1 Observe y vea cómo (en invierno) todos los árboles parecen como si se hubieran marchitado y perdido todas sus hojas, excepto catorce árboles, que no pierden su follaje pero conservan el follaje viejo de dos a tres años hasta el nuevo llega.

\chapter{4}

\par 1 Y observad también en los días de verano cómo el sol está sobre la tierra, frente a ella. Y buscáis sombra y refugio a causa del calor del sol, y también la tierra arde con creciente calor, y por eso no podéis pisar la tierra, ni sobre una roca a causa de su calor.

\chapter{5}

\par 1 Observad cómo los árboles se cubren de hojas verdes y dan fruto; por tanto, prestad atención y conoced todas sus obras, y reconoced cómo las ha hecho así el que vive para siempre.
\par 2 Y todas Sus obras continúan así de año en año para siempre, y todas las tareas que realizan para Él, y sus tareas no cambian, sino que se hacen según lo que Dios ha ordenado.
\par 3 Y mira cómo el mar y los ríos igualmente cumplen sus tareas y no se apartan de sus mandamientos.
\par 4 Pero vosotros no habéis sido firmes ni habéis cumplido los mandamientos del Señor, sino que os habéis desviado y hablado palabras soberbias y duras con vuestra boca inmunda contra su grandeza. Oh, duros de corazón, no encontraréis paz.
\par 5 Por tanto, execraréis vuestros días, y los años de vuestra vida perecerán, y los años de vuestra destrucción se multiplicarán en eterna execración, y no encontraréis misericordia.
\par 6 [a] En aquellos días haréis de vuestros nombres una execración eterna para todos los justos. [b] Y por ti pasarán todos los que maldicen, y todos los pecadores e impíos imprecarán por ti. Y todos los . . . se regocijarán, y habrá perdón de pecados, y toda misericordia, paz y paciencia; habrá para ellos salvación, una hermosa luz. Y para todos vosotros pecadores no habrá salvación, sino que sobre todos vosotros reposará la maldición.
\par 7 Pero para los elegidos habrá luz, alegría y paz, y heredarán la tierra. Y para vosotros, los impíos, habrá maldición.
\par 8 Y entonces a los elegidos se les concederá sabiduría, y todos vivirán y nunca más pecarán, ni por impiedad ni por orgullo. Pero los sabios serán humildes.
\par 9 Y no volverán a transgredir, ni pecarán en todos los días de su vida, ni morirán de ira o ira (la divina), sino que completarán el número de los días de su vida. Y sus vidas serán aumentadas en paz, y los años de su gozo se multiplicarán, en eterna alegría y paz, todos los días de su vida.

\chapter{6}

\par 1 Y aconteció que cuando los hijos de los hombres se multiplicaron, en aquellos días les nacieron hijas hermosas y hermosas.
\par 2 Y los ángeles, los hijos del cielo, los vieron y los codiciaron, y se dijeron unos a otros: «Venid, escojamos esposas de entre los hijos de los hombres y engendremos hijos».
\par 3 Y Semjaza, que era su líder, les dijo: «Me temo que no aceptaréis hacer este acto, y yo solo tendré que pagar la pena de un gran pecado».
\par 4 Y todos le respondieron y dijeron: «Hagamos todos un juramento y comprometámonos todos mediante imprecaciones mutuas a no abandonar este plan sino a hacer esto».
\par 5 Entonces juraron todos juntos y se comprometieron con imprecaciones mutuas.
\par 6 Y eran en total doscientos; quienes descendieron en los días de Jared a la cumbre del Monte Hermón, y lo llamaron Monte Hermón, porque habían jurado y se habían obligado mediante mutuas imprecaciones sobre él.
\par 7 Y estos son los nombres de sus líderes: Semiazaz, su líder, Arakiba, Rameel, Kokabiel, Tamiel, Ramiel, Danel, Ezeqeel, Baraqijal, Asael, Armaros, Batarel, Ananel, Zaqiel, Samsapeel, Satarel, Turel, Jomjael. , Sariel.
\par 8 Estos son sus jefes de decenas.

\chapter{7}

\par 1 Y todos los que estaban con ellos tomaron para sí mujeres, y cada uno escogió una para sí, y comenzaron a acercarse a ellas y a contaminarse con ellas, y les enseñaban hechizos y encantamientos, y a cortar raíces, y les familiarizó con las plantas.
\par 2 Y quedaron embarazadas y dieron a luz a grandes gigantes, cuya altura era de tres mil vanas,
\par 3 que consumió todas las posesiones de los hombres. Y cuando los hombres ya no pudieron sostenerlos,
\par 4 los gigantes se volvieron contra ellos y devoraron a la humanidad.
\par 5 Y comenzaron a pecar contra las aves, las bestias, los reptiles y los peces, y a devorar la carne de otros y a beber su sangre.
\par 6 Entonces la tierra acusó a los impíos.

\chapter{8}

\par 1 Y Azazel enseñó a los hombres a hacer espadas, cuchillos, escudos y corazas, y les dio a conocer los metales de la tierra, el arte de trabajarlos, los brazaletes, los adornos, el uso del antimonio y el embellecimiento de los párpados, y toda clase de piedras preciosas, y toda tintura colorante.
\par 2 Y surgió mucha impiedad y fornicaron, se extraviaron y se corrompieron en todos sus caminos.
\par 3 Semjaza enseñó encantamientos y cortes de raíces, Armaros la resolución de encantamientos, Baraqijal (enseñó) la astrología, Kokabel las constelaciones, Ezeqeel el conocimiento de las nubes, Araqiel los signos de la tierra, Shamsiel los signos del sol y Sariel el curso de la luna.
\par 4 Y cuando los hombres perecieron, lloraron y su clamor llegó hasta el cielo. . .

\chapter{9}

\par 1 Y entonces Miguel, Uriel, Rafael y Gabriel miraron desde el cielo y vieron mucha sangre derramada sobre la tierra y toda la maldad que se cometía sobre la tierra.
\par 2 Y se dijeron unos a otros: «La tierra sin habitante grita la voz de su llanto hasta las puertas del cielo».
\par 3 «Y ahora a vosotros, santos del cielo, las almas de los hombres presentan su pleito, diciendo: «Llevad nuestra causa ante el Altísimo»».
\par 4 Y dijeron al Señor de los siglos: «Señor de señores, Dios de los dioses, Rey de reyes y Dios de los siglos».
\par 5 «¡El trono de tu gloria (está) por todas las generaciones de los siglos, y tu nombre santo, glorioso y bendito por todos los siglos! Tú hiciste todas las cosas, y tienes poder sobre todas las cosas, y todas las cosas están desnudas y abiertas a tus ojos, y tú ves todas las cosas, y nada puede esconderse de ti».
\par 6 «Tú ves lo que ha hecho Azazel, quien enseñó toda la injusticia en la tierra y reveló los secretos eternos que estaban (conservados) en el cielo, que los hombres se esforzaban por aprender».
\par 7 «Y Semjaza, a quien le has dado autoridad para gobernar a sus asociados».
\par 8 «Y se fueron a las hijas de los hombres sobre la tierra, se acostaron con las mujeres, se contaminaron y les revelaron toda clase de pecados».
\par 9 «Y las mujeres dieron a luz gigantes, y por eso toda la tierra se llenó de sangre e injusticia».
\par 10 «Y ahora, he aquí, las almas de los que han muerto lloran y se dirigen a las puertas del cielo, y sus lamentos han ascendido y no pueden cesar a causa de las iniquidades que se cometen en la tierra».
\par 11 «Y Tú sabes todas las cosas antes de que sucedan, y las ves y las sufres, y no nos dices qué debemos hacer con ellas con respecto a ellas».

\chapter{10}

\par 1 Entonces habló el Altísimo, el Santo y Grande, y envió a Uriel al hijo de Lamec, y le dijo:
\par 2 «Ve a Noé y dile en mi nombre: «¡Escóndete!» y reveladle el fin que se acerca, que toda la tierra será destruida, y que un diluvio está a punto de venir sobre toda la tierra, y destruirá todo lo que hay en ella».
\par 3 «Y ahora instrúyelo para que pueda escapar y su descendencia sea preservada para todas las generaciones del mundo».
\par 4 Y nuevamente el Señor dijo a Rafael: «Ata a Azazel de pies y manos, y échalo en las tinieblas, y haz una abertura en el desierto que está en Dudael, y échalo allí».
\par 5 «Y colocad sobre él piedras ásperas y dentadas, y cúbrelo de oscuridad, y que permanezca allí para siempre, y cúbrele el rostro para que 6,7 no vea la luz».
\par 6 «Y el día del gran juicio será arrojado al fuego».
\par 7 «Y sanad la tierra que los ángeles han corrompido, y proclamad la curación de la tierra, para que puedan curar la plaga, y para que todos los hijos de los hombres no perezcan a causa de todas las cosas secretas que los Vigilantes han revelado y han enseñado a sus hijos».
\par 8 «Y toda la tierra se ha corrompido por las obras que fueron enseñadas por Azazel: a él atribuyan todo pecado».
\par 9 Y a Gabriel le dijo el Señor: «Procede contra los bastardos y los reprobados, y contra los hijos de la fornicación, y destruye [a los hijos de la fornicación y] a los hijos de los Vigilantes de entre los hombres [y haz que salgan ], envíalos unos contra otros para que se destruyan unos a otros en la batalla, porque no tendrán días largos».
\par 10 «Y ninguna petición que ellos (es decir, sus padres) te hagan, será concedida a sus padres en su nombre; porque esperan vivir vida eterna, y que cada uno de ellos vivirá quinientos años».
\par 11 Y el Señor dijo a Miguel: «Ve, ata a Semjaza y a sus asociados que se han unido con mujeres para contaminarse con ellas en toda su impureza».
\par 12 Y cuando sus hijos se hayan matado unos a otros y hayan visto la destrucción de sus amados, atadlos durante setenta generaciones en los valles de la tierra, hasta el día de su juicio y de su consumación, hasta el día de su muerte. El juicio que es por los siglos de los siglos está consumado».
\par 13 «En aquellos días serán llevados al abismo del fuego, y al tormento y a la prisión en la que serán confinados para siempre».
\par 14 «Y cualquiera que sea condenado y destruido, desde entonces quedará ligado con ellos hasta el fin de todas las generaciones».
\par 15 «Y destruir todos los espíritus de los réprobos y de los hijos de los Vigilantes, porque han agraviado a la humanidad».
\par 16 «Destruid todo mal de la faz de la tierra y deja que toda obra mala llegue a su fin, y que aparezca la planta de justicia y verdad, y resultará una bendición; las obras de justicia y de verdad serán plantadas en verdad y gozo para siempre».
\par 17 Entonces todos los justos escaparán y vivirán hasta engendrar miles de hijos, y todos los días de su juventud y su vejez los completarán en paz.
\par 18 Y entonces toda la tierra será labrada con justicia, y toda ella será plantada de árboles y será llena de bendiciones.
\par 19 Y se plantarán en él todos los árboles deseables, y plantarán vides en él, y la vid que planten en él producirá vino en abundancia, y en cuanto a toda la semilla que se siembre en él, cada medida (de ella) dará mil, y cada medida de aceitunas dará diez lagares de aceite.
\par 20 Y limpia la tierra de toda opresión, de toda injusticia, de todo pecado, de toda impiedad, y de toda impureza que se hace sobre la tierra, destrúyela de sobre la tierra.
\par 21 Y todos los hijos de los hombres serán justos, y todas las naciones me adorarán y me alabarán, y todos me adorarán.
\par 22 Y la tierra será limpiada de toda contaminación, de todo pecado, de todo castigo y de todo tormento, y nunca más los enviaré sobre ella de generación en generación y para siempre.

\chapter{11}

\par 1 Y en aquellos días abriré los depósitos de bendición que están en el cielo, para enviar
\par 2 ellos sobre la tierra sobre el trabajo y el trabajo de los hijos de los hombres. Y la verdad y la paz estarán asociadas durante todos los días del mundo y en todas las generaciones de los hombres.

\chapter{12}

\par 1 Antes de esto Enoc estaba escondido, y ninguno de los hijos de los hombres sabía dónde estaba escondido, dónde moraba y qué había sido de él.
\par 2 Y sus actividades tenían que ver con los Vigilantes, y sus días eran con los santos.
\par 3 Y yo, Enoc, estaba bendiciendo al Señor de la majestad y al Rey de los siglos, y ¡he aquí! Los Vigilantes me llamaron —Enoc el escriba— y me dijeron:
\par 4 «Enoc, escriba de justicia, ve y declara a los Vigilantes del cielo que han abandonado el cielo alto, el lugar santo y eterno, y se han contaminado con mujeres, y han hecho lo que hacen los hijos de la tierra, y han tomado para sí mujeres.
\par 5 Habéis causado gran destrucción en la tierra. Y no tendréis paz ni perdón de pecados,
\par 6 y mientras se deleitan en sus hijos, verán el asesinato de sus amados, y por la destrucción de sus hijos se lamentarán y suplicarán por la eternidad, pero ni la misericordia ni la paz alcanzaréis. »

\chapter{13}

\par 1 Y Enoc fue y dijo: «Azazel, no tendrás paz: se ha pronunciado contra ti una sentencia severa para encarcelarte».
\par 2 «Y no se te concederá tolerancia ni petición por la injusticia que has enseñado, y por todas las obras de impiedad, injusticia y pecado que has mostrado a los hombres».
\par 3 Entonces fui y les hablé a todos juntos, y todos tuvieron miedo, y el miedo y el temblor se apoderaron de ellos.
\par 4 Y me rogaron que les redactara una petición para que pudieran encontrar perdón y que leyera su petición en presencia del Señor del cielo.
\par 5 Porque desde entonces no pudieron hablar (con Él) ni alzar los ojos al cielo avergonzados de los pecados por los que habían sido condenados.
\par 6 Luego escribí su petición y la oración en relación con sus espíritus y sus obras individualmente y en relación con sus peticiones para que tuvieran perdón y duración.
\par 7 Y fui y me senté junto a las aguas de Dan, en la tierra de Dan, al sur del oeste de Hermón; leí su petición hasta que me quedé dormido.
\par 8 Y he aquí, tuve un sueño, y cayeron sobre mí visiones, y vi visiones de castigo, y vino una voz que me ordenaba que lo contara a los hijos del cielo y los reprendiera.
\par 9 Y cuando desperté, me acerqué a ellos, y estaban todos sentados juntos, llorando en Abelsjail, que está entre el Líbano y Seneser, con sus rostros cubiertos.
\par 10 Y les conté todas las visiones que había tenido en sueños, y comencé a pronunciar palabras de justicia y a reprender a los Vigilantes celestiales.

\chapter{14}

\par 1 El libro de las palabras de justicia y de la reprimenda de los Vigilantes eternos de acuerdo con el mandato del Santo Grande en esa visión.
\par 2 Vi en sueños lo que ahora diré con la lengua de carne y con el aliento de mi boca: lo que el Grande ha dado a los hombres para conversar con él y entender con el corazón.
\par 3 Así como Él creó y dio al hombre el poder de comprender la palabra de sabiduría, así también me creó a mí y me dio el poder de reprender a los Vigilantes, los hijos del cielo.
\par 4 «Escribí tu petición, y en mi visión apareció así, que tu petición no te será concedida durante todos los días de la eternidad, y que finalmente se ha dictado sentencia sobre ti: sí (tu petición) será no os será concedido».
\par 5 «Y desde ahora en adelante no subiréis al cielo por toda la eternidad, y en las ataduras de la tierra ha salido el decreto para ataros por todos los días del mundo».
\par 6 «Y (que) antes habéis visto la destrucción de vuestros amados hijos y no os complaceréis en ellos, sino que caerán ante vosotros a espada».
\par 7 «Y tu petición en favor de ellos no será concedida, ni tampoco la tuya propia, aunque llores y ores y pronuncies todas las palabras contenidas en la escritura que he escrito».
\par 8 Y así se me mostró la visión. He aquí, en la visión las nubes me invitaron y una niebla me convocó, y el curso de las estrellas y los relámpagos me aceleraron y apresuraron, y los vientos en la visión me hicieron volar y me levantaron hacia arriba y me llevaron al cielo.
\par 9 Y entré hasta que llegué cerca de una pared hecha de cristales y rodeada de lenguas de fuego, y comencé a asustarme.
\par 10 Y entré en las lenguas de fuego y me acerqué a una casa grande que estaba construida de cristales, y las paredes de la casa eran como un piso de mosaico (hecho) de cristales, y su base era de cristal.
\par 11 Su techo era como la trayectoria de las estrellas y de los relámpagos, y entre ellos había querubines de fuego, y su cielo era (claro como) agua.
\par 12 Un fuego llameante rodeaba los muros, y sus portales ardían de fuego.
\par 13 Y entré en aquella casa, y hacía calor como el fuego y frío como el hielo: no había en ella delicias de la vida; El miedo me cubrió y el temblor se apoderó de mí.
\par 14 Y mientras temblaba y temblaba, caí sobre mi rostro. Y tuve una visión,
\par 15 ¡Y he aquí! Había una segunda casa, más grande que la anterior, y todo el portal estaba abierto ante mí, y estaba construido con llamas de fuego.
\par 16 Y en todos los aspectos sobresalía tanto en esplendor, magnificencia y extensión, que no puedo describiros su esplendor y su extensión.
\par 17 Y su suelo era de fuego, y encima de él había relámpagos y la trayectoria de las estrellas, y también su techo era de fuego llameante.
\par 18 Miré y vi en él un trono elevado: su aspecto era como el cristal, y sus ruedas como el sol resplandeciente, y había visión de querubines.
\par 19 Y de debajo del trono salían corrientes de fuego que no podía mirar.
\par 20 Y la Gran Gloria se sentaba sobre él, y Su vestido brillaba más que el sol y era más blanco que cualquier nieve.
\par 21 Ninguno de los ángeles podía entrar y contemplar su rostro a causa de la magnificencia y la gloria, y ninguna carne podía contemplarlo.
\par 22 Una llama de fuego lo rodeaba, y un gran fuego estaba delante de él, y nadie alrededor podía acercarse a él: diez mil veces diez mil (estuvieron) delante de él, pero no necesitaba consejero.
\par 23 Y los santos que estaban cerca de Él no se alejaron de noche ni se apartaron de Él.
\par 24 Y hasta entonces yo estaba postrado sobre mi rostro, temblando, y el Señor me llamó con su propia boca, y me dijo: «Ven acá, Enoc, y escucha mi palabra».
\par 25 Y uno de los santos vino a mí y me despertó, y me hizo levantarme y acercarme a la puerta, e incliné mi rostro hacia abajo.

\chapter{15}

\par 1 Y Él respondió y me dijo, y oí su voz: «No temas, Enoc, hombre justo y escriba de justicia: acércate acá y escucha mi voz».
\par 2 «Y ve, di a los Vigilantes del cielo, que te han enviado a interceder por ellos: «Debéis interceder» por los hombres, y no los hombres por vosotros:»
\par 3 «¿Por qué habéis dejado el cielo alto, santo y eterno, y os acostasteis con mujeres, y os contaminasteis con las hijas de los hombres, y tomasteis mujeres, y habéis hecho como los hijos de la tierra, y engendrasteis gigantes (como vuestros) hijos.»
\par 4 «Y aunque erais santos, espirituales y vivíais la vida eterna, os habéis contaminado con sangre de mujeres, y habéis engendrado (hijos) con sangre de carne, y, como hijos de hombres, habéis codiciado carne y sangre, como también hacen los que mueren y perecen».
\par 5 Por eso les he dado también esposas para que las conciban y engendren hijos de ellas, para que así nada les falte en la tierra.
\par 6 «Pero vosotros antes erais espirituales, vivíais la vida eterna y erais inmortales para todas las generaciones del mundo».
\par 7 Por eso no os he designado esposas; porque en cuanto a los espirituales del cielo, en el cielo está su morada».
\par 8 «Y ahora, los gigantes, que son producidos a partir de espíritus y carne, serán llamados espíritus malignos sobre la tierra, y en la tierra será su morada».
\par 9 «De sus cuerpos han salido espíritus malignos; porque nacen de los hombres y de los santos Vigilantes es su principio y origen primordial; serán espíritus malignos en la tierra, y espíritus malignos serán llamados».
\par 10 «[En cuanto a los espíritus del cielo, en el cielo serán su morada, pero en cuanto a los espíritus de la tierra que nacieron sobre la tierra, en la tierra será su morada.]»
\par 11 «Y los espíritus de los gigantes afligen, oprimen, destruyen, atacan, luchan, destruyen la tierra y causan disturbios: no comen, pero sin embargo tienen hambre y sed y causan escándalo».
\par 12 «Y estos espíritus se levantarán contra los hijos de los hombres y contra las mujeres, porque de ellos proceden».

\chapter{16}

\par 1 Desde los días de la matanza, destrucción y muerte de los gigantes, de cuyas almas los espíritus, habiendo salido, destruirán sin incurrir en juicio, así destruirán hasta el día de la consumación, el gran juicio en el cual se consumará la era, sobre los Vigilantes y los impíos, sí, se consumará por completo.
\par 2 Y ahora, en cuanto a los centinelas que te enviaron a interceder por ellos, que estuvieron antes en el cielo, diles:
\par 3 «Estuvisteis en el cielo, pero aún no os habían sido revelados todos los misterios, y conocisteis los que eran inútiles, y éstos, con la dureza de vuestro corazón, los diste a conocer a las mujeres, y a través de estos misterios las mujeres y los hombres hacen mucho mal en la tierra».
\par 4 Diles, pues: «No tenéis paz».

\chapter{17}

\par 1 Y me tomaron y me llevaron a un lugar donde los que estaban allí eran como llamas de fuego y, cuando querían, aparecían como hombres.
\par 2 Y me llevaron a un lugar oscuro y a una montaña cuya cima llegaba al cielo.
\par 3 Y vi los lugares de las luminarias y los tesoros de las estrellas y de los truenos y en las profundidades más profundas, donde había un arco de fuego y flechas y su aljaba, y una espada de fuego y todos los relámpagos.
\par 4 Y me llevaron a las aguas vivas y al fuego del occidente, que recibe cada puesta del sol.
\par 5 Y llegué a un río de fuego en el que el fuego fluye como agua y desemboca en el gran mar hacia el oeste.
\par 6 Vi los grandes ríos y llegué al gran río y a la gran oscuridad, y fui al lugar donde nadie camina.
\par 7 Vi las montañas de la oscuridad del invierno y el lugar de donde brotan todas las aguas del abismo.
\par 8 Vi las desembocaduras de todos los ríos de la tierra y las desembocaduras del abismo.

\chapter{18}

\par 1 Vi los tesoros de todos los vientos. Vi cómo había provisto con ellos toda la creación y los firmes cimientos de la tierra.
\par 2 Y vi la piedra angular de la tierra. Vi los cuatro vientos que sostienen [la tierra y] el firmamento de los cielos.
\par 3 Y vi cómo los vientos extienden las bóvedas de los cielos y tienen su lugar entre el cielo y la tierra: éstas son las columnas del cielo.
\par 4 Vi los vientos del cielo que giran y traen la circunferencia del sol y todas las estrellas a su ocaso.
\par 5 Vi los vientos en la tierra llevando las nubes. Vi los caminos de los ángeles. Vi en el fin de la tierra el firmamento de los cielos arriba.
\par 6 Y continué y vi un lugar que arde día y noche, donde hay siete montañas de piedras magníficas, tres hacia el este y tres hacia el sur.
\par 7 Los que estaban hacia el oriente eran de piedra coloreada, uno de perla, otro de jacinto, y los del sur, de piedra roja.
\par 8 Pero el del medio, como el trono de Dios, llegaba al cielo, de alabastro, y la cima del trono era de zafiro.
\par 9 Y vi un fuego llameante. Y más allá de estas montañas
\par 10 Es una región el fin de la gran tierra: allí se completaron los cielos.
\par 11 Y vi un abismo profundo, con columnas de fuego celestial, y entre ellas vi caer columnas de fuego que eran incomparables tanto hacia lo alto como hacia lo profundo.
\par 12 Y más allá de ese abismo vi un lugar que no tenía firmamento del cielo arriba, ni tierra firme debajo; no había agua ni pájaros, sino que era un lugar desierto y horrible.
\par 13 Vi allí siete estrellas como grandes montañas ardientes, y cuando pregunté por ellas,
\par 14 El ángel dijo: «Este lugar es el fin del cielo y de la tierra; esto se ha convertido en prisión para las estrellas y el ejército del cielo».
\par 15 «Y las estrellas que ruedan sobre el fuego son aquellas que transgredieron el mandamiento del Señor al principio de su ascenso, porque no salieron en el tiempo señalado».
\par 16 «Y se enojó contra ellos y los ató hasta el momento en que su culpa fuera consumada durante diez mil años».

\chapter{19}

\par 1 Y Uriel me dijo: «Aquí estarán los ángeles que se han relacionado con las mujeres, y sus espíritus, que adoptan muchas formas diferentes, contaminan a la humanidad y los llevarán por mal camino para que sacrifiquen a los demonios como dioses (aquí estarán, ) hasta el día del gran juicio en el cual serán juzgados hasta que sean acabados».
\par 2 «Y también las mujeres de los ángeles que se extraviaron se convertirán en sirenas».
\par 3 «Y solo yo, Enoc, vi la visión, el fin de todas las cosas; y nadie verá como yo he visto».

\chapter{20}

\par 1 Y estos son los nombres de los santos ángeles que velan.
\par 2 Uriel, uno de los santos ángeles, que está sobre el mundo y sobre el Tártaro.
\par 3 Rafael, uno de los santos ángeles, que está sobre los espíritus de los hombres.
\par 4 Raguel, uno de los santos ángeles que se venga del mundo de las luminarias.
\par 5 Miguel, uno de los santos ángeles, es decir, el que está encargado de lo mejor de la humanidad y del caos.
\par 6 Saraqael, uno de los santos ángeles, que está sobre los espíritus que pecan en el espíritu.
\par 7 Gabriel, uno de los santos ángeles, que está sobre el paraíso, las serpientes y los querubines.
\par 8 Remiel, uno de los santos ángeles a quienes Dios puso sobre los que resucitarán.

\chapter{21}

\par 1 Y procedí a donde las cosas eran caóticas.
\par 2 Y vi allí algo horrible: no vi ni un cielo arriba ni una tierra firmemente fundada, sino un lugar caótico y horrible.
\par 3 Y allí vi siete estrellas del cielo unidas en él, como grandes montañas y ardiendo en fuego.
\par 4 Entonces dije: «¿Por qué pecado están atados y por qué han sido arrojados aquí?»
\par 5 Entonces Uriel, uno de los santos ángeles que estaba conmigo y era el jefe de ellos, dijo: «Enoc, ¿por qué preguntas y por qué anhelas la verdad?
\par 6 Éstas son del número de estrellas del cielo que han transgredido el mandamiento del Señor y están atadas aquí hasta que se consuman diez mil años, el tiempo de sus pecados.»
\par 7 Y de allí fui a otro lugar, que era aún más horrible que el primero, y vi una cosa horrible: allí un gran fuego que ardía y ardía, y el lugar estaba hendido hasta el abismo, estando lleno de grandes columnas de fuego descendentes: ni su extensión ni magnitud podía ver, ni podía conjeturar.
\par 8 Entonces dije: «¡Qué espantoso es el lugar y qué terrible contemplarlo!»
\par 9 Entonces Uriel, uno de los santos ángeles que estaba conmigo, me respondió y me dijo: «Enoc, ¿por qué tienes tanto miedo y espanto?» Y yo respondí: «A causa de este lugar espantoso y a causa del espectáculo del dolor».
\par 10 Y me dijo: «Este lugar es la prisión de los ángeles, y aquí estarán encarcelados para siempre».

\chapter{22}

\par 1 De allí me fui a otro lugar, a un monte de dura roca.
\par 2 Y había en él cuatro huecos, profundos, anchos y muy lisos. «Cuán suaves son los lugares huecos y cuán profundos y oscuros son a la vista».
\par 3 Entonces respondió Rafael, uno de los santos ángeles que estaba conmigo, y me dijo: «Estos lugares huecos han sido creados con este mismo propósito, para que los espíritus de las almas de los muertos se reúnan en ellos, sí, para que todos las almas de los hijos de los hombres deberían reunirse aquí».
\par 4 «Y estos lugares han sido creados para recibirlos hasta el día de su juicio y hasta su período señalado [hasta el período señalado], hasta que (venga) sobre ellos el gran juicio».
\par 5 Vi (el espíritu de) un hombre muerto haciendo un traje, y su voz se elevó al cielo y hizo un traje.
\par 6 Y pregunté a Rafael, el ángel que estaba conmigo, y le dije: «Este espíritu que hace la demanda, ¿de quién es, cuya voz sale y hace la demanda al cielo?»
\par 7 Y él me respondió diciendo: «Este es el espíritu que salió de Abel, a quien su hermano Caín mató, y él presenta su demanda contra él hasta que su descendencia sea destruida de la faz de la tierra, y su descendencia sea aniquilada de entre la simiente de los hombres».
\par 8 Entonces pregunté respecto a esto y a todos los lugares huecos: «¿Por qué están separados unos de otros?»
\par 9 Y él me respondió y me dijo: «Estos tres fueron creados para separar los espíritus de los muertos. Y tal división se ha hecho (para) los espíritus de los justos, en los cuales hay un manantial de agua brillante».
\par 10 «Y esto se ha hecho para los pecadores cuando mueren y son sepultados en la tierra y no se les ha ejecutado juicio durante su vida».
\par 11 «Aquí sus espíritus serán apartados en este gran dolor hasta el gran día del juicio, el castigo y el tormento de aquellos que maldicen para siempre y la retribución de sus espíritus. Allí los atará para siempre».
\par 12 «Y tal división se ha hecho para los espíritus de aquellos que presentan su demanda, que revelan acerca de su destrucción, cuando fueron asesinados en los días de los pecadores».
\par 13 «Esto ha sido hecho para los espíritus de los hombres que no eran justos sino pecadores, que fueron completos en transgresión, y de los transgresores serán compañeros; pero sus espíritus no serán asesinados en el día del juicio ni serán resucitará de allí».
\par 14 Entonces bendije al Señor de la gloria y dije: «Bendito sea mi Señor, el Señor de la justicia, que gobierna por los siglos».

\chapter{23}

\par 1 De allí me fui a otro lugar, al occidente de los confines de la tierra.
\par 2 Y vi un fuego ardiendo que corría sin descanso, y no se detenía ni de día ni de noche, sino que corría regularmente.
\par 3 Y pregunté diciendo: «¿Qué es esto que no descansa?»
\par 4 Entonces Ragüel, uno de los santos ángeles que estaba conmigo, me respondió y me dijo: «Este curso de fuego que has visto es el fuego del occidente que persigue a todas las lumbreras del cielo».

\chapter{24}

\par 1 Y de allí fui a otro lugar de la tierra, y él me mostró una montaña de fuego que ardía día y noche.
\par 2 Y fui más allá y vi siete montañas magníficas, todas diferentes entre sí, y las piedras (de ellas) eran magníficas y hermosas, magníficas en conjunto, de apariencia gloriosa y hermosa exterior: tres hacia el este, una fundada del otro, y tres hacia el Sur, uno sobre el otro, y profundas quebradas abruptas, sin que ninguna se juntara con otra.
\par 3 Y en medio de ellos estaba el séptimo monte, que los superaba en altura, semejante al asiento de un trono; y árboles aromáticos rodeaban el trono.
\par 4 Y entre ellos había un árbol como nunca antes había olido, ni había ninguno entre ellos ni había otros como él: tenía una fragancia más allá de toda fragancia, y sus hojas, flores y madera no se marchitaban para siempre; y su El fruto es hermoso y su fruto se parece a los dátiles de una palmera.
\par 5 Entonces dije: «¡Qué hermoso y fragante es este árbol, y sus hojas son hermosas, y sus flores, de apariencia muy deliciosa!».
\par 6 Entonces respondió Miguel, uno de los ángeles santos y honrados que estaba conmigo y era su líder.

\chapter{25}

\par 1 Y me dijo: «Enoc, ¿por qué me preguntas sobre la fragancia del árbol y por qué deseas aprender la verdad?»
\par 2 Entonces le respondí diciendo: «Quiero saber de todo, pero especialmente de este árbol».
\par 3 Y él respondió diciendo: «Este monte alto que has visto, cuya cumbre es como el trono de Dios, es Su trono, donde se sentará el Santo Grande, el Señor de la Gloria, el Rey Eterno, cuando Él descenderá a visitar la tierra con bondad».
\par 4 «Y en cuanto a este árbol fragante, a ningún mortal le está permitido tocarlo hasta el gran juicio, cuando Él se vengará de todos y llevará (todo) a su consumación para siempre. Entonces será dada a los justos y santos».
\par 5 «Su fruto será alimento para los elegidos: será trasplantado al lugar santo, al templo del Señor, el Rey Eterno».
\par 6 «Entonces se regocijarán y se alegrarán, y entrarán en el lugar santo; y su fragancia estará en sus huesos, y vivirán una larga vida en la tierra, como vivieron tus padres; y en sus días no los tocará tristeza, ni plaga, ni tormento, ni calamidad».
\par 7 Entonces bendije al Dios de la Gloria, al Rey Eterno, que preparó tales cosas para los justos, las creó y prometió dárselas.

\chapter{26}

\par 1 Y de allí fui al centro de la tierra, y vi un lugar bendito en el que había árboles con ramas que permanecían y florecían [de un árbol desmembrado].
\par 2 Y allí vi una montaña santa, y debajo de la montaña hacia el este había un arroyo que fluía hacia el sur.
\par 3 Y vi hacia el oriente otro monte más alto que este, y entre ellos un barranco profundo y estrecho; por él también corría un arroyo debajo del monte.
\par 4 Y al occidente de él había otro monte, más bajo que el anterior y de poca elevación, y entre ellos un barranco profundo y seco; y otro barranco profundo y seco había en los extremos de los tres montes.
\par 5 Y todos los barrancos eran profundos y angostos, de dura roca, y no se plantaban árboles sobre ellos.
\par 6 Y me maravillé de las rocas, y me maravillé del barranco, y me maravillé mucho.

\chapter{27}

\par 1 Entonces dije: «¿Para qué sirve esta tierra bendita, que está enteramente llena de árboles, y este valle maldito en medio?»
\par 2 Entonces Uriel, uno de los santos ángeles que estaba conmigo, respondió y dijo: «Este valle maldito es para los que son malditos para siempre: aquí se reunirán todos los malditos que con sus labios hablan indecorosamente contra el Señor. Las palabras y de su gloria hablan cosas duras. Aquí serán reunidos y aquí será su lugar de juicio».
\par 3 «En los últimos días habrá sobre ellos el espectáculo del juicio justo, en presencia de los justos para siempre: aquí los misericordiosos bendecirán al Señor de la gloria, al Rey Eterno».
\par 4 «En los días del juicio sobre los primeros, lo bendecirán por la misericordia que les ha asignado (su suerte)».
\par 5 Entonces bendije al Señor de la gloria, expuse su gloria y lo alabé gloriosamente.

\chapter{28}

\par 1 Y de allí me dirigí hacia el este, en medio de la cadena montañosa del desierto, y vi un desierto y era solitario,
\par 2 llenos de árboles y plantas.
\par 3 Y el agua brotó de arriba. Corriendo como un caudaloso curso de agua [que fluía] hacia el noroeste, hizo subir por todos lados nubes y rocío.

\chapter{29}

\par 1 De allí me dirigí a otro lugar en el desierto y me acerqué al este de esta montaña.
\par 2 Y allí vi árboles aromáticos que exhalaban fragancia de incienso y mirra, y también los árboles eran parecidos al almendro.

\chapter{30}

\par 1 Y más allá de estos, me alejé hacia el este y vi otro lugar, un valle (lleno) de agua.
\par 2 Y dentro había un árbol, del color () de árboles fragantes como el lentisco.
\par 3 Y en las laderas de aquellos valles vi canela fragante. Y más allá de estos seguí hacia el este.

\chapter{31}

\par 1 Y vi otras montañas, y entre ellas había arboledas, de las cuales brotaba néctar, llamado sarara y gálbano.
\par 2 Y más allá de estos montes vi otro monte al oriente de los confines de la tierra, en el cual había áloes, y todos los árboles estaban llenos de estacto, como almendros.
\par 3 Y cuando se quemaba, olía más dulce que cualquier olor fragante.

\chapter{32}

\par 1 Y después de estos olores fragantes, mientras miraba hacia el norte sobre las montañas, vi siete montañas llenas de nardos selectos, árboles aromáticos, canela y pimienta.
\par 2 Y desde allí crucé las cumbres de todas estas montañas, muy hacia el este de la tierra, y pasé sobre el mar Eritrea y me alejé de él, y pasé sobre el ángel Zotiel.
\par 3 Y llegué al Jardín de la Justicia, yo y de lejos árboles más numerosos que yo, estos árboles y dos grandes árboles allí, muy grandes, hermosos, gloriosos y magníficos, y el árbol del conocimiento, cuyo santo fruta que comen y conocen gran sabiduría.
\par 4 Ese árbol es alto como el abeto, y sus hojas como el algarrobo, y su fruto como los racimos de la vid, muy hermosos; y el olor del árbol llega hasta lejos.
\par 5 Entonces dije: «¡Qué hermoso es el árbol y qué atractivo es su aspecto!»
\par 6 Entonces el santo ángel Rafael, que estaba conmigo, me respondió y dijo: «Éste es el árbol de la sabiduría, del cual comieron tu padre anciano y tu madre anciana, que fueron antes que tú, y aprendieron sabiduría y se les abrieron los ojos, y supieron que estaban desnudos y fueron expulsados ​​del jardín».

\chapter{33}

\par 1 Y desde allí fui hasta los confines de la tierra y vi allí grandes bestias, cada una diferente de la otra; y (vi) también aves diferentes en apariencia, belleza y voz, diferentes unas de otras.
\par 2 Y al oriente de aquellas bestias vi los confines de la tierra sobre los cuales reposa el cielo, y las puertas del cielo abiertas.
\par 3 Y vi cómo salían las estrellas del cielo, y conté las puertas por donde salen, y anoté todas sus salidas, de cada estrella en particular, según su número y sus nombres, sus cursos y sus posiciones, y sus tiempos y sus meses, como me lo mostró el santo ángel Uriel que estaba conmigo.
\par 4 Él me mostró todas las cosas y me las escribió; también me escribió sus nombres, sus leyes y sus compañías.

\chapter{34}

\par 1 Y desde allí me dirigí hacia el norte hasta los confines de la tierra, y allí vi un gran y glorioso monumento en los confines de toda la tierra.
\par 2 Y aquí vi tres puertas del cielo abiertas en el cielo: por cada una de ellas pasan los vientos del norte; cuando soplan, hay frío, granizo, escarcha, nieve, rocío y lluvia.
\par 3 Y por una puerta soplan para bien; pero cuando soplan por las otras dos puertas, es con violencia y aflicción en la tierra, y soplan con violencia.

\chapter{35}

\par 1 Desde allí me dirigí hacia el occidente, hasta los confines de la tierra, y vi allí tres puertas del cielo abiertas, como las que había visto en el oriente, tantas puertas y tantas salidas.

\chapter{36}

\par 1 Desde allí fui hacia el sur, hasta los confines de la tierra, y vi allí tres puertas del cielo abiertas, y de allí venían el rocío, la lluvia y el viento.
\par 2 Y desde allí fui hacia el este hasta los confines del cielo, y vi aquí las tres puertas orientales del cielo abiertas y pequeñas puertas sobre ellas.
\par 3 A través de cada uno de estos pequeños portales pasan las estrellas del cielo y siguen su curso hacia el oeste por el camino que se les muestra.
\par 4 Y todas las veces que lo veía, bendije siempre al Señor de la Gloria, y seguí bendiciendo al Señor de la Gloria que ha hecho grandes y gloriosas maravillas, para mostrar la grandeza de Su obra a los ángeles, a los espíritus y a los hombres. , para que puedan alabar Su obra y toda Su creación: para que puedan ver la obra de Su poder y alabar la gran obra de Sus manos y bendecirlo por siempre.

\part{Sección II. Capítulos XXXVII—LXXI. Las parábolas}

\chapter{37}

\par 1 La segunda visión que tuvo, la visión de sabiduría, que vio Enoc, hijo de Jared, hijo de Mahalaleel, hijo de Cainán, hijo de Enós, hijo de Set, hijo de Adán.
\par 2 Y este es el comienzo de las palabras de sabiduría que alcé mi voz para hablar y decir a los que habitan en la tierra:
\par 3 «Oíd, hombres antiguos, y ved, los que venís después, las palabras del Santo que yo pronunciaré delante del Señor de los espíritus. Sería mejor declararlas (sólo) a los hombres de antaño, pero ni siquiera a los que vendrán después les negaremos el principio de la sabiduría».
\par 4 «Hasta el día de hoy, el Señor de los espíritus nunca me ha dado tal sabiduría como la que he recibido según mi intuición, según la buena voluntad del Señor de los espíritus, por quien me ha sido dada la suerte de la vida eterna.»
\par 5 «Y me fueron contadas tres parábolas, y alcé mi voz y las conté a los moradores de la tierra».

\chapter{38}
\par 1 La primera parábola. Cuando aparezca la congregación de los justos, y los pecadores sean juzgados por sus pecados y expulsados ​​de la faz de la tierra:
\par 2 Y cuando el Justo aparezca ante los ojos de los justos, cuyas obras elegidas dependen del Señor de los espíritus, y la luz aparecerá para los justos y los elegidos que habitan en la tierra, donde entonces estará la morada de los pecadores, y dónde descansan los que han negado al Señor de los espíritus? Bueno les hubiera sido no haber nacido.
\par 3 Cuando los secretos de los justos sean revelados y los pecadores juzgados, y los impíos expulsados ​​de la presencia de los justos y elegidos,
\par 4 A partir de entonces los poseedores de la tierra ya no serán poderosos ni exaltados, ni podrán contemplar el rostro del santo, porque el Señor de los espíritus ha hecho aparecer su luz sobre el rostro del santo. , justos y elegidos.
\par 5 Entonces los reyes y los poderosos perecerán y serán entregados en manos de los justos y santos.
\par 6 Y de ahora en adelante nadie buscará para sí misericordia del Señor de los espíritus porque su vida ha llegado a su fin.

\chapter{39}

\par 1 [Y sucederá en aquellos días que los niños elegidos y santos descenderán del alto cielo, y su descendencia será una con los hijos de los hombres.
\par 2 Y en aquellos días Enoc recibió libros de celo y de ira, y libros de inquietud y expulsión.] Y no se les concederá misericordia, dice el Señor de los Espíritus.
\par 3 En aquellos días, un torbellino me arrastró de la tierra y me puso al final de los cielos.
\par 4 Y allí tuve otra visión: las moradas de los santos y las moradas de los justos.
\par 5 Aquí mis ojos vieron sus moradas con sus ángeles justos, y sus lugares de descanso con los santos. Y rogaron e intercedieron y oraron por los hijos de los hombres, y la justicia fluyó ante ellos como agua, y la misericordia como rocío sobre la tierra: así será entre ellos por los siglos de los siglos.
\par 6 [a] Y en aquel lugar mis ojos vieron al Elegido de justicia y de fe, [b] Y la justicia prevalecerá en sus días, y los justos y elegidos serán innumerables delante de Él por los siglos de los siglos.
\par 7 [a] Y vi su morada bajo las alas del Señor de los espíritus. [b] Y todos los justos y elegidos delante de Él serán fuertes como lumbreras de fuego, y su boca estará llena de bendición, y sus labios ensalzarán el nombre del Señor de los Espíritus, y la justicia delante de Él nunca faltará, [Y la rectitud nunca fallará ante Él.]
\par 8 Allí deseaba morar, y mi espíritu anhelaba esa morada; y allí ha sido hasta ahora mi porción, porque así ha sido establecido para mí ante el Señor de los espíritus.
\par 9 En aquellos días alabé y ensalcé el nombre del Señor de los espíritus con bendiciones y alabanzas, porque Él me ha destinado para bendición y gloria según la buena voluntad del Señor de los espíritus.
\par 10 Durante mucho tiempo mis ojos contemplaron aquel lugar, y lo bendije y lo alabé, diciendo: «Bendito sea Él, y sea bendito desde el principio y por los siglos».
\par 11 «Y delante de Él no hay cesar. Él sabe, antes de que el mundo fuera creado, lo que es para siempre y lo que será de generación en generación».
\par 12 «Los que no duermen te bendicen: están delante de tu gloria y bendicen, alaban y ensalzan, diciendo: Santo, santo, santo, es el Señor de los espíritus: él llena la tierra de espíritus».
\par 13 Y aquí mis ojos vieron a todos los que no duermen: se presentan ante Él y bendicen y dicen: «Bendito seas, y bendito sea el nombre del Señor por los siglos de los siglos».
\par 14 Y mi rostro cambió; porque ya no podía contemplar.

\chapter{40}

\par 1 Y después de eso vi miles de miles y diez mil veces diez mil, vi una multitud incalculable y sin número, que estaba de pie ante el Señor de los espíritus.
\par 2 Y en los cuatro lados del Señor de los espíritus vi cuatro presencias, diferentes de las que no duermen, y aprendí sus nombres; porque el ángel que iba conmigo me hizo saber sus nombres y me mostró todos los cosas ocultas.
\par 3 Y oí las voces de aquellas cuatro presencias que pronunciaban alabanzas ante el Señor de la gloria.
\par 4 La primera voz bendice al Señor de los Espíritus por los siglos de los siglos.
\par 5 Y oí la segunda voz bendiciendo al Elegido y a los elegidos que penden del Señor de los Espíritus.
\par 6 Y la tercera voz que oí orar e interceder por los que habitan la tierra y suplicar en el nombre del Señor de los espíritus.
\par 7 Y oí la cuarta voz que rechazaba a los demonios y les prohibía presentarse ante el Señor de los espíritus para acusar a los que moran en la tierra.
\par 8 Después le pregunté al ángel de paz que iba conmigo y que me mostró todo lo que está oculto: «¿Quiénes son estas cuatro presencias que he visto y cuyas palabras he oído y escrito?»
\par 9 Y me dijo: Este primero es Miguel, el misericordioso y sufrido; el segundo, encargado de todas las enfermedades y todas las heridas de los hijos de los hombres, es Rafael; y el tercero, El que está sobre todos los poderes, es Gabriel; y el cuarto, que está sobre el arrepentimiento para la esperanza de los que heredan la vida eterna, se llama Fanuel».
\par 10 Y estos son los cuatro ángeles del Señor de los espíritus y las cuatro voces que oí en aquellos días.

\chapter{41}

\par 1 Y después vi todos los secretos de los cielos, cómo se divide el reino y cómo se pesan en la balanza las acciones de los hombres.
\par 2 Y allí vi las mansiones de los elegidos y las mansiones de los santos, y mis ojos vieron allí a todos los pecadores que eran expulsados ​​de allí y que negaban el nombre del Señor de los espíritus, y eran arrastrados; y no podían permanecéis a causa del castigo que procede del Señor de los Espíritus.
\par 3 Y allí mis ojos vieron los secretos de los relámpagos y del trueno, y los secretos de los vientos, cómo se dividen para soplar sobre la tierra, y los secretos de las nubes y del rocío, y allí vi de dónde proceden de aquel lugar y de donde saturan la tierra polvorienta.
\par 4 Y allí vi las cámaras cerradas de donde se dividen los vientos, la cámara del granizo y de los vientos, la cámara de la niebla y de las nubes, y su nube se cierne sobre la tierra desde el principio del mundo.
\par 5 Y vi las cámaras del sol y de la luna, de donde salen y adónde regresan, y su glorioso regreso, y cómo uno es superior al otro, y su majestuosa órbita, y cómo no salen de su órbita. , y no añaden nada a su órbita ni quitan nada de ella, y se mantienen fieles unos a otros, de acuerdo con el juramento por el que están unidos.
\par 6 Y primero sale el sol y recorre su camino según el mandamiento del Señor de los espíritus, y poderoso es su nombre por los siglos de los siglos.
\par 7 Y después vi el camino oculto y visible de la luna, y ella sigue su camino en ese lugar de día y de noche, uno frente al otro ante el Señor de los espíritus. Y dan gracias y alabanzas y no descansan; porque para ellos es su reposo de acción de gracias.
\par 8 Porque el sol muchas veces se convierte en bendición o en maldición, y el camino de la luna es luz para los justos y oscuridad para los pecadores, en el nombre del Señor, que hizo separación entre la luz y las tinieblas, y dividió los espíritus de los hombres, y fortaleció los espíritus de los justos, en el nombre de su justicia.
\par 9 Porque ningún ángel lo impide, ni ningún poder puede impedirlo; porque Él nombra juez para todos ellos y los juzga a todos delante de Él.

\chapter{42}

\par 1 La sabiduría no encontró lugar donde habitar; luego se le asignó una morada en los cielos.
\par 2 La Sabiduría salió a habitar entre los hijos de los hombres, pero no encontró morada; la Sabiduría volvió a su lugar y se sentó entre los ángeles.
\par 3 Y la injusticia salió de sus aposentos; a quienes no buscaba, los encontraba y moraba con ellos, como la lluvia en el desierto y el rocío en una tierra sedienta.

\chapter{43}

\par 1 Y vi otros relámpagos y las estrellas del cielo, y vi cómo él los llamaba a todos por sus nombres y le escuchaban.
\par 2 Y vi cómo son pesados ​​en una balanza justa según sus proporciones de luz: (vi) la anchura de sus espacios y el día de su aparición, y cómo su revolución produce relámpagos: y (vi) sus revolución según el número de los ángeles, y (cómo) se mantienen fieles unos a otros.
\par 3 Y le pregunté al ángel que iba conmigo y que me mostró lo que estaba escondido: «¿Qué es esto?»
\par 4 Y me dijo: «El Señor de los Espíritus te ha mostrado su significado parabólico (lit. su parábola): estos son los nombres de los santos que habitan en la tierra y creen en el nombre del Señor de los Espíritus por siglos de los siglos.»

\chapter{44}

\par 1 También vi otro fenómeno con respecto a los relámpagos: cómo algunas estrellas surgen y se convierten en relámpagos y no pueden separarse de su nueva forma.

\chapter{45}

\par 1 Esta es la segunda parábola de aquellos que niegan el nombre de la morada de los santos y del Señor de los espíritus.
\par 2 Y no subirán al cielo, ni vendrán a la tierra: tal será la suerte de los pecadores que han negado el nombre del Señor de los espíritus, que así serán preservados para el día del sufrimiento y tribulación.
\par 3 En aquel día, mi Elegido se sentará en el trono de gloria y probará sus obras, y sus lugares de descanso serán innumerables. Y sus almas se fortalecerán dentro de ellos cuando vean a Mis Elegidos y a los que han invocado Mi glorioso nombre:
\par 4 Entonces haré que mi Elegido habite entre ellos. Y transformaré el cielo y lo convertiré en bendición y luz eterna.
\par 5 Y transformaré la tierra y la convertiré en bendición, y haré que mis elegidos habiten en ella, pero los pecadores y los malhechores no pondrán un pie en ella.
\par 6 Porque he provisto y satisfecho de paz a mis justos y los he hecho habitar delante de mí; pero para los pecadores hay un juicio inminente ante mí, de modo que los destruiré de la faz de la tierra.

\chapter{46}

\par 1 Y allí vi a Uno que tenía «Cabeza de Días», y Su cabeza era blanca como lana, y con Él estaba otro ser cuyo rostro tenía la apariencia de un hombre, y su rostro estaba lleno de gracia, como uno de los santos ángeles.
\par 2 Y le pregunté al ángel que iba conmigo y me mostró todas las cosas ocultas acerca de ese Hijo del Hombre, quién era y de dónde era, (y) por qué iba con la «Cabeza de los Días».
\par 3 Y él respondió y me dijo: «Este es el hijo del Hombre que tiene justicia, en quien mora la justicia, y que revela todos los tesoros de lo que está escondido, porque el Señor de los espíritus lo ha elegido, y cuyo Lot tiene la preeminencia ante el Señor de los Espíritus en rectitud para siempre».
\par 4 «Y este Hijo del Hombre que has visto levantará de sus tronos a los reyes y a los poderosos, y soltará las riendas de los fuertes y quebrará los dientes de los pecadores. »
\par 5 «[Y derribará a los reyes de sus tronos y reinos] porque no lo ensalzan ni lo alaban, ni reconocen humildemente de dónde les fue otorgado el reino».
\par 6 «Y humillará el rostro de los fuertes y los llenará de vergüenza. Y las tinieblas serán su morada, y los gusanos serán su lecho, y no tendrán esperanza de levantarse de su lecho, porque no ensalzan el nombre del Señor de los Espíritus».
\par 7 «Y éstos son los que juzgan las estrellas del cielo, [y levantan sus manos contra el Altísimo], pisan la tierra y habitan en ella. Y todas sus obras manifiestan injusticia, y su poder descansa en sus riquezas, y su fe está en los dioses que han hecho con sus manos, y niegan el nombre del Señor de los espíritus».
\par 8 «Y persiguen las casas de sus congregaciones y a los fieles que se aferran al nombre del Señor de los espíritus».

\chapter{47}

\par 1 Y en aquellos días habrá ascendido de la tierra la oración de los justos y la sangre de los justos ante el Señor de los espíritus.
\par 2 En aquellos días, los santos que habitan arriba en los cielos se unirán con una sola voz y suplicarán, orarán y alabarán, y darán gracias y bendecirán el nombre del Señor de los espíritus por la sangre de los justos que ha sido derramada, y que la oración de los justos no sea en vano ante el Señor de los Espíritus, para que se les haga juicio y no tengan que sufrir para siempre.
\par 3 En aquellos días vi la «cabeza de los días», cuando se sentó en el trono de su gloria, y los libros de los vivientes fueron abiertos delante de él; y todo su ejército que está arriba en el cielo y sus consejeros estaban delante de él. A él,
\par 4 Y el corazón de los santos se llenó de alegría; porque se había ofrecido el número de los justos, y se había escuchado la oración de los justos, y se había requerido la sangre de los justos ante el Señor de los espíritus.

\chapter{48}

\par 1 Y en aquel lugar vi la fuente de justicia, que era inagotable; y alrededor de ella había muchas fuentes de sabiduría; y todos los sedientos bebían de ellas y se llenaban de sabiduría, y sus moradas estaban con los justos y los santos y electo.
\par 2 Y en aquella hora fue nombrado aquel Hijo del Hombre delante del Señor de los espíritus, y su nombre delante de la Cabeza de los Días.
\par 3 Sí, antes que fueran creados el sol y los signos, antes que fueran creadas las estrellas del cielo, su nombre fue nombrado delante del Señor de los espíritus.
\par 4 Él será para los justos un bastón en el que apoyarse y no caer, y será la luz de los gentiles y la esperanza de los de corazón turbado.
\par 5 Todos los habitantes de la tierra se postrarán y adorarán ante él, y alabarán, bendecirán y celebrarán con cánticos al Señor de los espíritus.
\par 6 Y por eso fue elegido y escondido delante de Él, antes de la creación del mundo y para siempre.
\par 7 Y la sabiduría del Señor de los espíritus lo ha revelado a los santos y justos; porque él ha preservado la suerte de los justos, porque han odiado y despreciado este mundo de injusticia, y han odiado todas sus obras y caminos en el nombre del Señor de los espíritus: porque en su nombre son salvos, y según su ha sido un buen placer con respecto a su vida.
\par 8 En estos días, los reyes de la tierra y los fuertes que poseen la tierra estarán abatidos de rostro a causa de las obras de sus manos, porque en el día de su angustia y aflicción no podrán salvarse a sí mismos.
\par 9 Y los entregaré en manos de mis elegidos: como paja en el fuego, así arderán ante el rostro del santo; como plomo en el agua se hundirán ante el rostro del justo, y ningún rastro de ellos se encontrará más.
\par 10 Y en el día de su aflicción habrá descanso en la tierra, y delante de ellos caerán y no se levantarán de nuevo, y no habrá nadie que los tome con sus manos y los levante, porque han negado el Señor de los Espíritus y Su Ungido. Bendito sea el nombre del Señor de los Espíritus.

\chapter{49}

\par 1 Porque la sabiduría se derrama como agua y la gloria nunca falta delante de él.
\par 2 Porque él es poderoso en todos los secretos de la justicia, y la injusticia desaparecerá como una sombra y no tendrá continuidad; porque el Elegido está delante del Señor de los Espíritus, y su gloria es por los siglos de los siglos, y su poder por todas las generaciones.
\par 3 Y en él habita el espíritu de sabiduría, el espíritu de discernimiento, el espíritu de inteligencia y de poder, y el espíritu de los que durmieron en justicia.
\par 4 Y él juzgará las cosas secretas, y nadie podrá pronunciar palabra mentirosa delante de él; porque él es el Elegido delante del Señor de los Espíritus según Su beneplácito.

\chapter{50}

\par 1 Y en aquellos días se producirá un cambio para los santos y los elegidos, y la luz de los días permanecerá sobre ellos, y la gloria y el honor se volverán para los santos,
\par 2 En el día de la aflicción en que se atesorará el mal contra los pecadores. Y los justos saldrán victoriosos en el nombre del Señor de los Espíritus: Y Él hará que los demás sean testigos (de esto) para que se arrepientan y renuncien a las obras de sus manos.
\par 3 No tendrán honor en el nombre del Señor de los espíritus, pero por su nombre serán salvos, y el Señor de los espíritus tendrá compasión de ellos, porque su compasión es grande.
\par 4 Y él también es justo en su juicio, y en presencia de su gloria tampoco se mantendrá la injusticia: en su juicio, los impenitentes perecerán delante de él.
\par 5 Y desde ahora no tendré misericordia de ellos, dice el Señor de los espíritus.

\chapter{51}

\par 1 Y en aquellos días también la tierra devolverá lo que le ha sido confiado, y el Seol también devolverá lo que recibió, y el infierno devolverá lo que debe.
\par 2 Y elegirá de entre ellos a los justos y santos, porque está cerca el día en que serán salvos.
\par 3 Y en aquellos días el Elegido se sentará en Mi trono, y su boca derramará todos los secretos de la sabiduría y del consejo, porque el Señor de los Espíritus se los ha dado y lo ha glorificado.
\par 4 Y en aquellos días las montañas saltarán como carneros, y también los collados saltarán como corderos saciados de leche, y los rostros de [todos] los ángeles en el cielo se iluminarán de alegría.
\par 5 [a] Porque en aquellos días se levantará el Elegido, [b] y la tierra se alegrará, [c] y los justos habitarán en ella, [d] y los elegidos caminarán sobre ella.

\chapter{52}

\par 1 Y después de aquellos días en aquel lugar donde había visto todas las visiones de lo oculto —pues había sido arrastrado por un torbellino y me habían llevado hacia el oeste—,
\par 2 Allí mis ojos vieron todas las cosas secretas del cielo que serán: una montaña de hierro, una montaña de cobre, una montaña de plata, una montaña de oro, una montaña de metal blando y una montaña de plomo.
\par 3 Y pregunté al ángel que iba conmigo, diciendo: ¿Qué es esto que he visto en secreto?
\par 4 Y me dijo: «Todas estas cosas que has visto servirán al dominio de Su Ungido para que sea poderoso y poderoso en la tierra».
\par 5 Y aquel ángel de paz respondió, diciéndome: «Espera un poco, y se te revelarán todas las cosas secretas que rodean al Señor de los Espíritus».
\par 6 Y estos montes que han visto tus ojos, el monte de hierro, el monte de cobre, el monte de plata, el monte de oro, el monte de metal blando y el monte de plomo, todos éstos estarán en presencia del Elegido como cera: delante del fuego, y como el agua que corre desde arriba [sobre esas montañas], y quedarán impotentes ante sus pies».
\par 7 «Y sucederá en aquellos días que nadie se salvará, ni con oro ni con plata, y nadie podrá escapar».
\par 8 «Y no habrá hierro para la guerra, ni nadie se vestirá con coraza. El bronce no servirá para nada, y el estaño [no servirá para nada y] no será estimado, y el plomo no será deseable».
\par 9 «Y todas estas cosas serán [negadas y] destruidas de la faz de la tierra, cuando el Elegido aparezca ante la faz del Señor de los Espíritus».

\chapter{53}

\par 1 Allí mis ojos vieron un valle profundo con bocas abiertas, y todos los habitantes de la tierra, del mar y de las islas le traerán regalos, presentes y muestras de homenaje, pero ese valle profundo no se llenará.
\par 2 Y sus manos cometen actos ilegales, y los pecadores devoran a todos los que oprimen ilegalmente; sin embargo, los pecadores serán destruidos ante la faz del Señor de los espíritus, y serán desterrados de la faz de su tierra, y perecerá por los siglos de los siglos.
\par 3 Porque vi a todos los ángeles del castigo morando (allí) y preparando todos los instrumentos de Satanás.
\par 4 Y le pregunté al ángel de paz que iba conmigo: «¿Para quién están preparando estos Instrumentos?»
\par 5 Y me dijo: «Estos los preparan para los reyes y los poderosos de esta tierra, para que así sean destruidos».
\par 6 «Y después de esto, el Justo y Elegido hará aparecer la casa de su congregación: de ahora en adelante ya no serán obstaculizados en el nombre del Señor de los Espíritus».
\par 7 «Y estos montes no serán como la tierra ante su justicia, sino que los collados serán como fuentes de agua, y los justos descansarán de la opresión de los pecadores».

\chapter{54}

\par 1 Miré y me volví hacia otra parte de la tierra, y vi allí un valle profundo con fuego ardiente.
\par 2 Y trajeron a los reyes y a los poderosos y comenzaron a arrojarlos a este valle profundo.
\par 3 Y allí mis ojos vieron cómo hacían de estos sus instrumentos, cadenas de hierro de peso inconmensurable.
\par 4 Y pregunté al ángel de paz que iba conmigo, diciendo: «¿Para quién están preparadas estas cadenas?»
\par 5 Y me dijo: «Estos están siendo preparados para las huestes de Azazel, para que los tomen y los arrojen al abismo de la completa condenación, y cubrirán sus mandíbulas con piedras ásperas como el Señor de los espíritus ordenó.»
\par 6 «Y Miguel, Gabriel, Rafael y Fanuel los agarrarán en ese gran día y los arrojarán ese día al horno ardiendo, para que el Señor de los espíritus se vengue de ellos por su injusticia en sometiéndose a Satanás y extraviando a los moradores de la tierra».
\par 7 Y en aquellos días vendrá el castigo del Señor de los espíritus, y abrirá todas las cámaras de aguas que están sobre los cielos y de las fuentes que están debajo de la tierra.
\par 8 Y todas las aguas se unirán con las aguas: lo que está sobre los cielos es lo masculino, y el agua que está debajo de la tierra es lo femenino.
\par 9 Y destruirán a todos los que habitan en la tierra y a los que habitan bajo los confines del cielo.
\par 10 Y cuando reconozcan las injusticias que han cometido en la tierra, por ellas perecerán.

\chapter{55}

\par 1 Y después el Jefe de los Días se arrepintió y dijo: «En vano he destruido a todos los habitantes de la tierra».
\par 2 Y juró por su gran nombre: «De ahora en adelante no haré así a todos los habitantes de la tierra, y pondré una señal en el cielo, y esto será prenda de buena fe entre Yo y ellos para siempre, mientras el cielo esté sobre la tierra».
\par 3 «Y esto está de acuerdo con Mi mandato. Cuando yo haya querido agarrarlos de la mano de los ángeles en el día de la tribulación y del dolor por esto, haré que mi castigo y mi ira repose sobre ellos, dice Dios, Señor de los espíritus».
\par 4 «Ustedes, reyes poderosos que habitan sobre la tierra, tendréis que contemplar a Mi Elegido, cómo se sienta en el trono de gloria y juzga a Azazel, y a todos sus asociados, y a todos sus ejércitos en el nombre del Señor de Espíritu.»

\chapter{56}

\par 1 Y vi allí las huestes de los ángeles del castigo que iban, trayendo azotes y cadenas de hierro y bronce.
\par 2 Y pregunté al ángel de paz que iba conmigo, diciendo: «¿A quién van estos que llevan el azote?».
\par 3 Y me dijo: «A sus elegidos y amados, para que sean arrojados al abismo del valle».
\par 4 «Y entonces ese valle se llenará de sus elegidos y amados, y los días de sus vidas llegarán a su fin, y los días en que se extraviaron no serán contados en adelante».
\par 5 Y en aquellos días los ángeles volverán y se lanzarán hacia el este sobre los partos y los medos. Incitarán a los reyes, de modo que un espíritu de inquietud vendrá sobre ellos y los levantará de sus tronos. , para que surjan como leones de sus guaridas, y como lobos hambrientos entre sus rebaños».
\par 6 «Y subirán y hollarán la tierra de sus elegidos [y la tierra de sus elegidos será delante de ellos una era y una calzada:]»
\par 7 «Pero la ciudad de mis justos será un obstáculo para sus caballos. Y comenzarán a pelear entre sí, y su diestra se fortalecerá contra ellos mismos, y el hombre no conocerá a su hermano, ni el hijo a su padre ni a su madre, hasta que no se cuenten los cadáveres de su matanza, y su castigo no sea en vano».
\par 8 «En aquellos días el Seol abrirá sus fauces, y serán tragados en él y su destrucción llegará a su fin; El Seol devorará a los pecadores en presencia de los elegidos».

\chapter{57}

\par 1 Y aconteció después de esto que vi otra multitud de carros y hombres montados en ellos, y que venían impulsados ​​por los vientos del este y del oeste hacia el sur.
\par 2 Y se oyó el ruido de sus carros, y cuando se produjo este tumulto, los santos del cielo lo notaron, y las columnas de la tierra fueron movidas de su lugar, y su sonido se escuchó desde un extremo del cielo al otro, en un día.
\par 3 Y todos se postrarán y adorarán al Señor de los espíritus. Y este es el final de la segunda parábola.

\chapter{58}

\par 1 Y comencé a hablar la tercera parábola sobre los justos y los elegidos.
\par 2 Bienaventurados vosotros, justos y elegidos, porque gloriosa será vuestra suerte.
\par 3 Y los justos estarán a la luz del sol. Y los elegidos a la luz de la vida eterna: Los días de su vida serán interminables, y los días de los santos innumerables.
\par 4 Y buscarán la luz y encontrarán justicia en el Señor de los espíritus: habrá paz para los justos en el nombre del Señor Eterno.
\par 5 Y después de esto se dirá a los santos en el cielo que busquen los secretos de la justicia, la herencia de la fe: porque se ha vuelto brillante como el sol sobre la tierra, y las tinieblas han pasado.
\par 6 Y habrá una luz que nunca se acaba, y hasta un límite (lit. número) de días no vendrán, porque las tinieblas primero habrán sido destruidas, [y la luz establecida ante el Señor de los espíritus] y la luz de la rectitud establecida para siempre ante el Señor de los espíritus.

\chapter{59}

\par 1 En aquellos días mis ojos vieron los secretos de los relámpagos y de las luces, y los juicios que ejecutan (lit. su juicio): y alumbran para bendición o maldición, según la voluntad del Señor de los espíritus.
\par 2 Y allí vi los secretos del trueno, y cómo cuando resuena arriba en el cielo, se oye su sonido, y él me hizo ver los juicios ejecutados en la tierra, ya sean para el bienestar y bendición, o para maldición según la palabra del Señor de los Espíritus.
\par 3 Y después de eso me fueron mostrados todos los secretos de las luces y los relámpagos, y ellos iluminaron para bendecir y satisfacer.

\chapter{60} Un fragmento del libro de Noé

\par 1 En el año 500, en el séptimo mes, el día catorce del mes en la vida de Enoc. En aquella parábola vi cómo un gran temblor hizo temblar los cielos de los cielos, y el ejército del Altísimo, y los ángeles, mil miles y diez mil veces diez mil, se turbaron con una gran inquietud.
\par 2 Y el «Cabeza de los Días» se sentó en el trono de su gloria, y los ángeles y los justos estaban alrededor de él.
\par 3 Y un gran temblor se apoderó de mí, y el miedo se apoderó de mí, y mis lomos se flaquearon, y mis riendas se disolvieron, y caí de bruces.
\par 4 Y Miguel envió otro ángel de entre los santos y él me levantó, y cuando me levantó mi espíritu volvió; porque no había podido soportar la mirada de este ejército, ni la conmoción y el temblor del cielo.
\par 5 Y Miguel me dijo: «¿Por qué te inquietas con tal visión hasta que duró el día de Su misericordia? y ha sido misericordioso y paciente para con los moradores de la tierra».
\par 6 «Y cuando llegue el día, el poder, el castigo y el juicio que el Señor de los espíritus ha preparado para los que no adoran la ley justa, para los que niegan el justo juicio y para los que que toman su nombre en vano, ese día está preparado, para los elegidos un pacto, pero para los pecadores una inquisición».
\par 7 Y aquel día se separaron dos monstruos, un monstruo femenino llamado Leviatán, para habitar en los abismos del océano, sobre las fuentes de las aguas.
\par 8 Pero el hombre se llama Behemoth, que ocupaba con su pecho un desierto desierto llamado Duidain, al este del jardín donde habitan los elegidos y los justos, donde fue acogido mi abuelo, el séptimo desde Adán, el primer hombre que el Señor de los Espíritus creó.
\par 9 Y rogué al otro ángel que me mostrara el poder de aquellos monstruos, cómo fueron divididos un día y arrojados, uno a los abismos del mar y el otro a la tierra seca del desierto.
\par 10 Y él me dijo: «Hijo de hombre, aquí buscas saber lo que está oculto». Y el ángel de paz que estaba conmigo me dijo: «Estos dos monstruos, preparados conforme a la grandeza de Dios, se alimentarán. . . Cuando el castigo del Señor de los Espíritus recaiga sobre ellos, reposará para que el castigo del Señor de los Espíritus no venga en vano, y matará a los hijos con sus madres y a los hijos con sus padres. Después el juicio se llevará a cabo según Su misericordia y Su paciencia».
\par 11 Y el otro ángel que fue conmigo y me mostró lo que estaba escondido me dijo lo primero y lo último en el cielo en las alturas, y debajo de la tierra en las profundidades, y en los confines del cielo, y en el fundamento del cielo.
\par 12 Y las cámaras de los vientos, y cómo se dividen los vientos, y cómo se pesan, y (cómo) se cuentan las puertas de los vientos, cada una según la fuerza del viento y la fuerza de las luces de la luna, y según la potencia que conviene; y las divisiones de las estrellas según sus nombres, y cómo están divididas todas las divisiones.
\par 13 Y los truenos según el lugar donde caen, y todas las divisiones que se hacen entre los relámpagos para que alumbren, y su ejército para que inmediatamente obedezcan.
\par 14 Porque el trueno tiene lugares de reposo (que) le son asignados mientras espera su repique; y el trueno y el relámpago son inseparables, y aunque no son uno ni indivisibles, ambos van juntos a través del espíritu y no se separan.
\par 15 Porque cuando el relámpago brilla, el trueno emite su voz, y el espíritu impone una pausa durante el repique, y se divide en partes iguales; porque el tesoro de sus repiques es como la arena, y cada uno de ellos, mientras repica, es sostenido con un freno, y por el poder del espíritu lo hace retroceder y lo empuja hacia adelante según los muchos rincones de la tierra.
\par 16 Y el espíritu del mar es masculino y fuerte, y según el poder de su fuerza lo tira hacia atrás con una rienda, y de la misma manera es impulsado hacia adelante y se dispersa entre todas las montañas de la tierra.
\par 17 Y el espíritu de la escarcha es su ángel, y el espíritu del granizo es un ángel bueno.
\par 18 Y el espíritu de la nieve ha abandonado sus aposentos a causa de su fuerza. Hay en él un espíritu especial, y lo que sube de él es como humo, y su nombre es escarcha.
\par 19 Y el espíritu de la niebla no está unido con ellos en sus aposentos, sino que tiene un aposento especial; porque su curso es glorioso tanto en la luz como en las tinieblas, en el invierno y en el verano, y en su cámara hay un ángel.
\par 20 Y el espíritu del rocío tiene su morada en los confines del cielo, y está conectado con las cámaras de la lluvia, y su curso es en invierno y en verano; y sus nubes y las nubes de la niebla están conectadas, y el uno le da al otro.
\par 21 Y cuando el espíritu de la lluvia sale de su cámara, los ángeles vienen, abren la cámara y la sacan, y cuando se difunde por toda la tierra, se une con el agua de la tierra. Y siempre que se una con el agua en la tierra. . .
\par 22 Porque las aguas son para los habitantes de la tierra; porque son alimento para la tierra del Altísimo que está en los cielos: por eso hay una medida para la lluvia, 22, y los ángeles se encargan de ella.
\par 23 Y estas cosas vi hacia el Jardín de los Justos.

\chapter{61}

\par 1 Y vi en aquellos días cómo les dieron largas cuerdas a aquellos ángeles, y tomaron alas y volaron, y se dirigieron hacia el norte.
\par 2 Y pregunté al ángel, diciéndole: «¿Por qué esos (ángeles) tomaron estas cuerdas y se fueron?»
\par 3 Y él me dijo: «Han ido a la medida». Y el ángel que iba conmigo me dijo: «Estos traerán las medidas de los justos, y las cuerdas de los justos a los justos, para que se apoyen en el nombre del Señor de los espíritus por los siglos de los siglos».
\par 4 «Los elegidos comenzarán a morar con los elegidos, y esas son las medidas que se darán a la fe y que fortalecerán la justicia».
\par 5 «Y estas medidas revelarán todos los secretos de las profundidades de la tierra, y los que han sido destruidos por el desierto, y los que han sido devorados por las bestias, y los que han sido devorados por los peces del mar, para que puedan regresar y permanecer ellos mismos en el día del Elegido; porque nadie será destruido ante el Señor de los Espíritus, y nadie podrá ser destruido».
\par 6 «Y todos los que habitan arriba en el cielo recibieron un mando y un poder y una sola voz y una luz como el fuego».
\par 7 «Y a Aquél (con) sus primeras palabras lo bendijeron, lo ensalzaron y lo alabaron con sabiduría, y fueron sabios en expresión y en espíritu de vida».
\par 8 «Y el Señor de los Espíritus colocó al Elegido en el trono de gloria. Y él juzgará todas las obras de los santos arriba en el cielo, y en la balanza serán pesadas sus obras».
\par 9 «Y cuando él alce su rostro para juzgar sus caminos secretos según la palabra del nombre del Señor de los espíritus, y su camino según el camino del justo juicio del Señor de los espíritus, entonces todos a una sola voz hablan y bendicen, glorifican, ensalzan y santifican el nombre del Señor de los Espíritus».
\par 10 «Y Él convocará a todo el ejército de los cielos, y a todos los santos de arriba, y al ejército de Dios, a los Querubines, a los Serafines y a los Ofaninos, y a todos los ángeles poderosos, y a todos los ángeles de los principados, y el Elegido y los demás poderes sobre la tierra (y) sobre el agua».
\par 11 «Ese día alzarán una voz, y bendecirán, glorificarán y exaltarán con espíritu de fe, con espíritu de sabiduría, con espíritu de paciencia, con espíritu de misericordia y con espíritu de de juicio y de paz, y con espíritu de bondad, y todos dirán a una sola voz: Bendito sea, y sea bendito el nombre del Señor de los Espíritus por los siglos de los siglos».
\par 12 Le bendecirán todos los que no duermen arriba en el cielo; le bendecirán todos los santos que están en el cielo, y todos los elegidos que habitan en el jardín de la vida; y todo espíritu de luz que pueda bendecir y glorifica, ensalza y santifica tu bendito nombre, y toda carne sin medida glorificará y bendecirá tu nombre por los siglos de los siglos.
\par 13 Porque grande es la misericordia del Señor de los espíritus, y es paciente, y todas sus obras y todo lo que ha creado lo ha revelado a los justos y elegidos en el nombre del Señor de los espíritus.

\chapter{62}

\par 1 Y así el Señor ordenó a los reyes, a los poderosos, a los exaltados y a los habitantes de la tierra, y dijo: «Abrid vuestros ojos y alzad vuestros cuernos si podéis reconocer al Elegido».
\par 2 Y el Señor de los espíritus lo sentó en su trono de gloria, y el espíritu de justicia se derramó sobre él, y la palabra de su boca mata a todos los pecadores, y todos los injustos son destruidos delante de él.
\par 3 Y en aquel día se levantarán todos los reyes, los poderosos, los excelsos y los que poseen la tierra, y verán y reconocerán cómo él se sienta en el trono de su gloria, y la justicia es juzgada delante de él. , y ante él no se pronuncia palabra mentirosa.
\par 4 Entonces les sobrevendrá dolor como a la mujer que está de parto, [y tiene dolor al dar a luz] cuando su hijo entra en la boca del vientre, y ella tiene dolor al dar a luz.
\par 5 Y una parte de ellos mirará a la otra, y se aterrorizarán, y se abatirán de rostro, y el dolor se apoderará de ellos, cuando vean al Hijo del Hombre sentado en el trono de su gloria.
\par 6 Y los reyes y los poderosos y todos los poseedores de la tierra bendecirán, glorificarán y ensalzarán al que domina sobre todo, el que estaba escondido.
\par 7 Porque desde el principio el Hijo del Hombre estaba escondido, y el Altísimo lo preservó en presencia de su poder y lo reveló a los elegidos.
\par 8 Y será sembrada la congregación de los elegidos y santos, y todos los elegidos estarán delante de él en aquel día.
\par 9 Y todos los reyes, los poderosos, los exaltados y los que gobiernan la tierra se postrarán ante él sobre sus rostros, y adorarán y pondrán su esperanza en ese Hijo del Hombre, y le suplicarán y suplicarán misericordia de sus manos.
\par 10 Sin embargo, el Señor de los espíritus los presionará de tal manera que se apresurarán a salir de su presencia, y sus rostros se llenarán de vergüenza y la oscuridad se hará más profunda en sus rostros.
\par 11 Y los entregará a los ángeles para que los castiguen y se venguen de ellos por haber oprimido a sus hijos y a sus elegidos.
\par 12 Y serán un espectáculo para los justos y para sus elegidos: se alegrarán con ellos, porque la ira del Señor de los espíritus reposa sobre ellos, y su espada está ebria con su sangre.
\par 13 Y los justos y los elegidos serán salvos en ese día, y nunca más volverán a ver el rostro de los pecadores e injustos.
\par 14 Y el Señor de los espíritus morará sobre ellos, y con ese Hijo del Hombre comerán, se acostarán y se levantarán por los siglos de los siglos.
\par 15 Y los justos y los elegidos se levantarán de la tierra y dejarán de estar abatidos. Y serán vestidos con vestiduras de gloria,
\par 16 Y éstas serán las vestiduras de vida del Señor de los espíritus: y vuestras vestiduras no envejecerán, ni vuestra gloria pasará ante el Señor de los espíritus.

\chapter{63}

\par 1 En aquellos días, los poderosos y los reyes que poseen la tierra le implorarán que les conceda un pequeño respiro de sus ángeles de castigo a quienes fueron entregados, para que puedan postrarse y adorar ante el Señor de los espíritus, y confesar sus pecados delante de Él.
\par 2 Y bendecirán y glorificarán al Señor de los espíritus, y dirán: 'Bendito el Señor de los espíritus y el Señor de los reyes, el Señor de los poderosos, el Señor de los ricos, el Señor de la gloria y de los Señor de la sabiduría,
\par 3 Y espléndido en cada cosa secreta es Tu poder de generación en generación, y Tu gloria por los siglos de los siglos: Profundos son todos Tus secretos e innumerables, y Tu justicia es incalculable.
\par 4 Ahora hemos aprendido que debemos glorificar y bendecir al Señor de los reyes y al que es Rey sobre todos los reyes.
\par 5 Y dirán: «¡Ojalá tuviéramos descanso para glorificarnos y dar gracias y confesar nuestra fe delante de Su gloria!»
\par 6 «Y ahora anhelamos un poco de descanso, pero no lo encontramos: lo seguimos con todas nuestras fuerzas y no lo obtenemos: y la luz se ha desvanecido delante de nosotros, y las tinieblas son nuestra morada por los siglos de los siglos:»
\par 7 «Porque no creímos delante de Él ni glorificamos el nombre del Señor de los espíritus, [ni glorificamos a nuestro Señor] sino que nuestra esperanza estaba en el cetro de nuestro reino y en nuestra gloria».
\par 8 «Y en el día de nuestro sufrimiento y tribulación Él no nos salva, y no encontramos respiro para confesar que nuestro Señor es veraz en todas Sus obras, y en Sus juicios y Su justicia, y Sus juicios no tienen respeto de personas».
\par 9 «Y nosotros pasamos de delante de Su faz a causa de nuestras obras, y todos nuestros pecados son contados en justicia».
\par 10 Ahora se dirán a sí mismos: «Nuestras almas están llenas de ganancias injustas, pero eso no nos impide descender de en medio de ellas a la carga del Seol».
\par 11 Y después sus rostros se llenarán de oscuridad y de vergüenza ante el Hijo del Hombre, y serán expulsados ​​de su presencia, y la espada permanecerá delante de su rostro en medio de ellos.
\par 12 Así habló el Señor de los espíritus: «Esta es la ordenanza y el juicio con respecto a los poderosos, los reyes, los exaltados y los que poseen la tierra delante del Señor de los espíritus».

\chapter{64}

\par 1 Y otras formas vi escondidas en ese lugar.
\par 2 Oí la voz del ángel que decía: «Estos son los ángeles que descendieron a la tierra y revelaron lo que estaba oculto a los hijos de los hombres y los indujeron a cometer pecado».

\chapter{65}

\par 1 En aquellos días Noé vio que la tierra se había hundido y que su destrucción estaba cerca.
\par 2 Y se levantó de allí y fue hasta los confines de la tierra, y clamó en voz alta a su abuelo Enoc: y Noé dijo tres veces con voz amarga: «Escúchame, escúchame, escúchame».
\par 3 Y le dije: «Dime qué es lo que está cayendo sobre la tierra, que la tierra está en tan mala situación y sacudida, no sea que yo perezca con ella».
\par 4 Entonces hubo un gran alboroto en la tierra, y se oyó una voz del cielo, y caí de bruces.
\par 5 Y mi abuelo Enoc vino y se paró a mi lado, y me dijo: «¿Por qué has clamado a mí con un llanto y un llanto amargos?»
\par 6 «Y de la presencia del Señor ha salido una orden para los que habitan en la tierra, de que su ruina se ha cumplido porque han aprendido todos los secretos de los ángeles, y toda la violencia de los satanás, y todos sus poderes —los más secretos— y todo el poder de los que practican hechicerías, y el poder de las hechicerías, y el poder de los que hacen imágenes de fundición para toda la tierra»;
\par 7 «Y cómo se produce la plata del polvo de la tierra, y cómo el metal blando se origina en la tierra».
\par 8 «Porque el plomo y el estaño no surgen de la tierra como los primeros: es una fuente que los produce, y en ella hay un ángel, y ese ángel es preeminente».
\par 9 Y después de eso mi abuelo Enoc me tomó de la mano y me levantó y me dijo: «Ve, porque le he preguntado al Señor de los Espíritus en cuanto a esta conmoción en la tierra».
\par 10 «Y Él me dijo: «A causa de su injusticia, su juicio ha sido determinado y Yo no lo retendré para siempre. A causa de las hechicerías que han investigado y aprendido, la tierra y los que habitan en ella serán destruidos».
\par 11 «Y éstos no tienen lugar de arrepentimiento para siempre, porque les han mostrado lo que estaba oculto, y son los condenados; pero en cuanto a ti, hijo mío, el Señor de los espíritus sabe que eres puro, y inocente de este reproche referente a los secretos».
\par 12 «Y Él ha destinado tu nombre a estar entre los santos, y te preservará entre los que habitan la tierra, y ha destinado tu descendencia justa tanto para realeza como para grandes honores, y de tu descendencia brotará una fuente de los justos y santos sin número para siempre».

\chapter{66}

\par 1 Y después me mostró los ángeles del castigo que están preparados para venir y desatar todos los poderes de las aguas que están debajo de la tierra para traer juicio y destrucción sobre todos los que [permanecen y] habitan en el tierra.
\par 2 Y el Señor de los espíritus dio orden a los ángeles que iban saliendo, de que no hicieran subir las aguas, sino que las retuvieran; porque aquellos ángeles estaban sobre las potestades de las aguas.
\par 3 Y me alejé de la presencia de Enoc.

\chapter{67}

\par 1 Y en aquellos días vino a mí la palabra de Dios, y me dijo: «Noé, tu suerte ha llegado delante de mí, mucha sin culpa, mucho amor y rectitud».
\par 2 «Y ahora los ángeles están construyendo un (edificio) de madera, y cuando hayan completado esa tarea, pondré Mi mano sobre él y lo preservaré, y de él brotará la semilla de vida, y se producirá un cambio para que la tierra no quede sin habitante».
\par 3 «Y afirmaré tu semilla delante de mí por los siglos de los siglos, y esparciré a los que habitan contigo: no quedará infructuosa sobre la faz de la tierra, sino que será bendita y se multiplicará sobre la tierra en el nombre del Señor».
\par 4 Y encarcelará a esos ángeles que han mostrado injusticia, en ese valle ardiente que mi abuelo Enoc me había mostrado anteriormente en el oeste, entre las montañas de oro, plata, hierro, metal blando y estaño.
\par 5 Y vi aquel valle en el que había una gran convulsión y una convulsión de las aguas.
\par 6 Y cuando todo esto sucedió, del metal fundido en llamas y de su convulsión en ese lugar, se produjo un olor a azufre, que estaba relacionado con aquellas aguas y con aquel valle de los ángeles que habían extraviado (La humanidad) ardió debajo de esa tierra.
\par 7 Y por sus valles corren corrientes de fuego, donde son castigados estos ángeles que han extraviado a los habitantes de la tierra.
\par 8 Pero aquellas aguas en aquellos días servirán a los reyes, a los poderosos, a los excelsos y a los habitantes de la tierra, para la curación del cuerpo, pero para el castigo del espíritu; ahora su espíritu está lleno de lujuria, para que sean castigados en su cuerpo, porque han negado al Señor de los espíritus y ven su castigo diariamente, y sin embargo no creen en Su nombre.
\par 9 Y a medida que la quema de sus cuerpos se vuelve más intensa, también se producirá un cambio correspondiente en su espíritu por los siglos de los siglos; porque ante el Señor de los Espíritus nadie pronunciará una palabra ociosa.
\par 10 Porque les sobrevendrá el juicio, porque creen en los deseos de su cuerpo y niegan el Espíritu del Señor.
\par 11 Y esas mismas aguas sufrirán un cambio en aquellos días; porque cuando esos ángeles sean castigados en estas aguas, estos manantiales cambiarán de temperatura, y cuando los ángeles asciendan, esta agua de los manantiales cambiará y se enfriará.
\par 12 Y oí a Miguel responder y decir: «Este juicio con el que se juzga a los ángeles es un testimonio para los reyes y los poderosos que poseen la tierra».
\par 13 «Porque estas aguas del juicio ministran la curación del cuerpo de los reyes y la concupiscencia de su cuerpo; por eso no verán ni creerán que esas aguas cambiarán y se convertirán en fuego que arderá para siempre».

\chapter{68}

\par 1 Y después de eso, mi abuelo Enoc me dio la enseñanza de todos los secretos en el libro de las Parábolas que le habían sido dados, y él los reunió para mí en las palabras del libro de las Parábolas.
\par 2 Y ese día Miguel respondió a Rafael y dijo: «El poder del espíritu me transporta y me hace temblar a causa de la severidad del juicio de los secretos, el juicio de los ángeles: ¿quién podrá soportar el severo juicio que ha ¿Han sido ejecutados y ante el cual se derriten?
\par 3 Y Miguel respondió de nuevo y dijo a Rafael: «¿Quién es aquel cuyo corazón no se ablanda con respecto a esto, y cuyos riñones no se perturban por esta palabra de juicio (que) ha salido sobre ellos a causa de aquellos que así lo han hecho? ¿Los sacó?
\par 4 Y aconteció que cuando se presentó ante el Señor de los espíritus, Miguel le dijo a Rafael lo siguiente: «No tomaré parte de ellos bajo la mirada del Señor; porque el Señor de los espíritus se ha enojado con ellos porque hacen como si fueran el Señor».
\par 5 Por tanto, todo lo oculto vendrá sobre ellos por los siglos de los siglos; porque ni el ángel ni el hombre tendrán su parte (en ello), sino que solos han recibido su juicio por los siglos de los siglos».

\chapter{69}

\par 1 Y después de este juicio se sentirán aterrorizados y temblados, porque así se lo han mostrado a los habitantes de la tierra.
\par 2 Y he aquí los nombres de esos ángeles [y estos son sus nombres: el primero de ellos es Samjaza, el segundo Artaqifa, y el tercero Armen, el cuarto Kokabel, el quinto Turael, el sexto Rumjal, el séptimo Danjal, el el octavo Neqael, el noveno Baraqel, el décimo Azazel, el undécimo Armaros, el duodécimo Batarjal, el decimotercero Busasejal, el decimocuarto Hananel, el decimoquinto Turel, el decimosexto Simapesiel, el decimoséptimo Jetrel, el decimoctavo Tumael, el decimonoveno Turel, el vigésimo Rumael, el vigésimo primer Azazel.
\par 3 Y estos son los jefes de sus ángeles y sus nombres, y sus jefes sobre cientos, sobre cincuenta y sobre decenas].
\par 4 El nombre del primer Jeqon, es decir, el que extravió [a todos] los hijos de Dios, y los hizo descender a la tierra, y los extravió a través de las hijas de los hombres.
\par 5 Y el segundo se llamaba Asbeel; impartió malos consejos a los santos hijos de Dios y los extravió hasta el punto de que contaminaron sus cuerpos con las hijas de los hombres.
\par 6 Y el tercero se llamaba Gadreel: él es quien mostró a los hijos de los hombres todos los golpes de la muerte, y engañó a Eva, y mostró [las armas de la muerte a los hijos de los hombres] el escudo y la túnica de cota de malla, y espada para la batalla, y todas las armas de muerte para los hijos de los hombres.
\par 7 Y de su mano han salido contra los que habitan la tierra desde aquel día y para siempre.
\par 8 Y el cuarto se llamaba Penemue: enseñó a los hijos de los hombres lo amargo y lo dulce, y les enseñó todos los secretos de su sabiduría.
\par 9 Y enseñó a los hombres a escribir con tinta y papel, y por eso muchos pecaron de eternidad en eternidad y hasta el día de hoy.
\par 10 Porque los hombres no fueron creados para tal fin, para confirmar con pluma y tinta su buena fe.
\par 11 Porque los hombres fueron creados exactamente como los ángeles, para que permanecieran puros y justos, y la muerte, que todo lo destruye, no podría haberlos alcanzado, pero por este conocimiento perecen, y por esto poder me esta consumiendo.
\par 12 Y el quinto se llamó Kasdeja: éste es el que mostró a los hijos de los hombres todos los malos azotes de los espíritus y demonios, y los azotes del embrión en el útero, para que muera, y [los azotes del alma] las mordeduras de la serpiente y los azotes que sobrevienen por el calor del mediodía, el hijo de la serpiente llamado Taba'et.
\par 13 Y esta es la tarea de Kasbeel, el jefe del juramento que hizo a los santos cuando habitó en lo alto en gloria, y su nombre es Biqa.
\par 14 Este (ángel) pidió a Miguel que le mostrara el nombre oculto, para poder pronunciarlo en el juramento, para que temblaran ante ese nombre y juramento los que revelaban todo lo secreto a los hijos de los hombres.
\par 15 Y este es el poder de este juramento, porque es poderoso y fuerte, y él puso este juramento Akae en la mano de Michael.
\par 16 Y estos son los secretos de este juramento. . . Y son fuertes por su juramento: Y los cielos fueron suspendidos antes de que el mundo fuera creado, y para siempre.
\par 17 Y por él la tierra fue fundada sobre el agua, y de los escondites secretos de las montañas brotan hermosas aguas, desde la creación del mundo y hasta la eternidad.
\par 18 Y por este juramento fue creado el mar, y le puso como fundamento la arena contra el tiempo de su ira, y no se atreverá a pasar más allá de él desde la creación del mundo hasta la eternidad.
\par 19 Y mediante ese juramento se afirman las profundidades, y permanecen y no se mueven de su lugar de eternidad en eternidad.
\par 20 Y mediante ese juramento el sol y la luna completan su curso y no se desvían de su ordenanza de eternidad en eternidad.
\par 21 Y por ese juramento las estrellas completan su carrera, y Él las llama por sus nombres, y ellas le responden de eternidad en eternidad.
\par 22 [Y de la misma manera los espíritus del agua y de los vientos, y de todos los céfiros, y (sus) caminos de todos los rincones de los vientos.
\par 23 Y allí están preservadas las voces del trueno y la luz de los relámpagos; y allí están preservadas las cámaras del granizo y las cámaras de la escarcha, y las cámaras de la niebla, y las cámaras de la lluvia y el Rocío.
\par 24 Y todos estos creen y dan gracias delante del Señor de los espíritus, y lo glorifican con todo su poder, y su alimento está en cada acto de acción de gracias: agradecen, glorifican y ensalzan el nombre del Señor de los espíritus por siglos de los siglos.]
\par 25 Y este juramento es poderoso sobre ellos y por él [son preservados y] sus caminos son preservados, y su carrera no es destruida.
\par 26 Y hubo gran alegría entre ellos, y bendijeron, glorificaron y ensalzaron porque el nombre del Hijo del Hombre les había sido revelado.
\par 27 Y se sentó en el trono de su gloria, y la suma del juicio fue dada al Hijo del Hombre, e hizo desaparecer y destruir a los pecadores de la faz de la tierra, y a los que habían guiado el mundo descarriado.
\par 28 Serán atados con cadenas, encarcelados en su lugar de reunión de destrucción, y todas sus obras desaparecerán de la faz de la tierra.
\par 29 Y desde ahora nada será corruptible; porque aquel Hijo del Hombre ha aparecido, y se ha sentado en el trono de su gloria, y todo mal pasará delante de él, y la palabra de aquel Hijo del Hombre saldrá. Y sed fuertes ante el Señor de los Espíritus.

\chapter{70}

\par 1 Y aconteció después de esto que su nombre, durante su vida, fue elevado a aquel Hijo del Hombre y al Señor de los espíritus de entre los que habitan la tierra.
\par 2 Y fue elevado a lo alto en los carros del espíritu y su nombre desapareció entre ellos.
\par 3 Y desde aquel día ya no fui contado entre ellos: y me puso entre los dos vientos, entre el Norte y el Oeste, donde los ángeles tomaron las cuerdas para medirme el lugar de los elegidos y los justos.
\par 4 Y allí vi a los primeros padres y a los justos que desde el principio habitaron en ese lugar.

\chapter{71}

\par 1 Y aconteció después de esto que mi espíritu fue trasladado y ascendió a los cielos, y vi a los santos hijos de Dios. Estaban pisando llamas de fuego: sus vestidos eran blancos [y sus vestiduras], y sus rostros resplandecían como la nieve.
\par 2 Y vi dos corrientes de fuego, y la luz de ese fuego brilló como un jacinto, y caí de bruces ante el Señor de los espíritus.
\par 3 Y el ángel Miguel [uno de los arcángeles] me tomó de mi mano derecha, me levantó y me llevó a todos los secretos, y me mostró todos los secretos de la justicia.
\par 4 Y me mostró todos los secretos de los confines del cielo, y todas las cámaras de todas las estrellas y todas las luminarias, de donde proceden ante la faz de los santos.
\par 5 Y trasladó mi espíritu al cielo de los cielos, y vi allí como una estructura construida de cristales, y entre esos cristales lenguas de fuego vivo.
\par 6 Y mi espíritu vio el cinturón que rodeaba aquella casa de fuego, y en sus cuatro lados había arroyos llenos de fuego vivo, que rodeaban aquella casa.
\par 7 Y alrededor estaban Serafín, Querúbico y Ofanín: estos son los que no duermen y guardan el trono de su gloria.
\par 8 Y vi ángeles incontables, mil miles, diez mil veces diez mil, que rodeaban aquella casa. Y Miguel, Rafael, Gabriel, Fanuel y los santos ángeles que están sobre los cielos entran y salen de esa casa.
\par 9 Y de aquella casa salieron Miguel, Gabriel, Rafael, Fanuel y muchos santos ángeles sin número.
\par 10 Y con ellos la «Cabeza de los Días», su cabeza blanca y pura como lana, y Su vestido indescriptible.
\par 11 Y caí de bruces, y todo mi cuerpo se relajó y mi espíritu se transfiguró; y lloré a gran voz, . . . con espíritu de poder, y bendecido, glorificado y ensalzado.
\par 12 Y estas bendiciones que salían de mi boca eran muy agradables ante aquella «Cabeza de los Días». Y ese «Cabeza de Días» vino con Miguel y Gabriel, Rafael y Fanuel, miles y decenas de miles de ángeles sin número.
\par 13 [Pasaje perdido en el que se describe al Hijo del Hombre acompañando a la «Cabeza de los Días», y Enoc le preguntó a uno de los ángeles (como en XLVI. 3) acerca del Hijo del Hombre quién era.]
\par 14 Y él (es decir, el ángel) vino a mí y me saludó con su voz, y me dijo: «Éste es el Hijo del Hombre que ha nacido para justicia, y la justicia permanece sobre él, y la justicia del El Jefe de los Días «no lo abandona».
\par 15 Y me dijo: «Él te proclama la paz en nombre del mundo venidero; porque de aquí procede la paz desde la creación del mundo, y así será contigo por los siglos de los siglos de los siglos».
\par 16 «Y todos caminarán en sus caminos, ya que la justicia nunca lo abandonará: con él estarán sus moradas, y con él su herencia, y no serán separados de él por los siglos de los siglos».
\par 17 «Y así se prolongarán los días para aquel Hijo del Hombre, y los justos tendrán paz y camino recto en el nombre del Señor de los espíritus por los siglos de los siglos».

\part{Sección III. Capítulos LXXII—LXXXII. El Libro de las Luminarias Celestiales.}

\chapter{72}

\par 1 El libro de los cursos de los astros del cielo, las relaciones de cada uno, según sus clases, su dominio y sus estaciones, según sus nombres y lugares de origen, y según sus meses, que Uriel, el me mostró el santo ángel que estaba conmigo, que es su guía; y me mostró todas sus leyes exactamente como son, y cómo son con respecto a todos los años del mundo y hasta la eternidad, hasta que se cumpla la nueva creación que dura hasta la eternidad.
\par 2 Y esta es la primera ley de las luminarias: la luminaria, el Sol, sale por las puertas orientales del cielo y se pone por las puertas occidentales del cielo.
\par 3 Y vi seis portales por donde sale el sol, y seis portales por donde el sol se pone y la luna sale y se pone en estos portales, y a los líderes de las estrellas y a aquellos a quienes guían: seis en el oriente y seis en el oeste, y todos seguidos en orden exactamente correspondiente: también muchas ventanas a la derecha e izquierda de estos portales.
\par 4 Y primero sale la gran luminaria, llamada Sol, y su circunferencia es como la circunferencia del cielo, y está completamente lleno de fuego que ilumina y calienta.
\par 5 El carro en el que sube, lo conduce el viento, y el sol se pone del cielo y regresa por el norte para llegar al este, y es guiado de tal manera que llega al portal apropiado (lit. ese) y brilla ante el cielo.
\par 6 De esta manera se eleva en el primer mes en el gran portal, que es el cuarto [esos seis portales en el reparto].
\par 7 Y en la cuarta puerta, por donde sale el sol en el primer mes, hay doce ventanas, de las que sale una llama cuando se abren en su tiempo.
\par 8 Cuando el sol sale en el cielo, sale por esa cuarta puerta treinta mañanas seguidas y se pone exactamente en la cuarta puerta al oeste del cielo.
\par 9 Y durante este período el día se hace cada día más largo y la noche cada noche más corta hasta la trigésima mañana.
\par 10 En aquel día el día fue más largo que la noche en una novena parte, y el día equivale exactamente a diez partes y la noche a ocho partes.
\par 11 Y el sol sale por la cuarta puerta, se pone por la cuarta y regresa a la quinta puerta del este treinta mañanas, y sale por allí y se pone por la quinta puerta.
\par 12 Y entonces el día se alarga dos partes y llega a once partes, y la noche se acorta y llega a siete partes.
\par 13 Y vuelve hacia el este y entra por la sexta puerta, y sale y se pone en la sexta puerta treinta y una mañanas a causa de su señal.
\par 14 En aquel día, el día se hizo más largo que la noche, y el día se hizo el doble que la noche, y el día se hizo doce partes, y la noche se acortó y llegó a ser seis partes.
\par 15 Y el sol sale para acortar el día y alargar la noche, y el sol vuelve al oriente y entra por la sexta puerta, sale por ella y se pone treinta mañanas.
\par 16 Y cuando se cumplen treinta mañanas, el día disminuye exactamente en una parte y se convierte en once partes, y la noche en siete.
\par 17 Y el sol sale por la sexta puerta en el oeste, y va hacia el este y sale por la quinta puerta durante treinta mañanas, y se pone nuevamente por el oeste en la quinta puerta occidental.
\par 18 En aquel día el día disminuirá en dos partes y será de diez partes, y la noche de ocho partes.
\par 19 Y el sol sale por esa quinta puerta y se pone por la quinta puerta del oeste, y sale por la cuarta puerta durante treinta y una mañanas a causa de su signo, y se pone por el oeste.
\par 20 En aquel día el día se iguala con la noche, y la noche tiene nueve partes y el día nueve partes.
\par 21 Y el sol sale por esa puerta y se pone por el oeste, y vuelve al este y sale treinta mañanas por la tercera puerta y se pone por el oeste por la tercera puerta.
\par 22 Y en ese día la noche se hizo más larga que el día, y la noche se hizo más larga que la noche, y el día más corto que el día hasta la mañana treinta, y la noche equivale exactamente a diez partes y el día a ocho partes.
\par 23 Y el sol sale por esa tercera puerta y se pone por la tercera puerta en el oeste y regresa al este, y durante treinta mañanas sale por la segunda puerta en el este, y de la misma manera se pone en la segunda puerta en el al oeste del cielo.
\par 24 Y en aquel día la noche fue de once partes y el día de siete partes.
\par 25 Y el sol sale ese día por esa segunda puerta y se pone por el oeste por la segunda puerta, y regresa al este por la primera puerta durante treinta y una mañanas, y se pone por la primera puerta por el oeste del cielo.
\par 26 Y en aquel día la noche se hizo más larga y llegó a ser el doble del día; y la noche llegó a ser exactamente doce partes y el día a seis.
\par 27 Y el sol ha recorrido las divisiones de su órbita y gira nuevamente sobre esas divisiones de su órbita, y entra por ese portal treinta mañanas y se pone también en el oeste frente a él.
\par 28 Y aquella noche la duración de la noche disminuyó en una novena parte, y la noche se convirtió en once partes, y el día en siete partes.
\par 29 Y el sol volvió y entró por la segunda puerta en el oriente, y volvió a esas divisiones de su órbita durante treinta mañanas, saliendo y poniéndose.
\par 30 Y en ese día la noche disminuye en duración, y la noche llega a diez partes y el día a ocho.
\par 31 Y ese día el sol sale por esa puerta, se pone por el oeste, vuelve al este, sale por la tercera puerta durante treinta y una mañanas y se pone por el oeste del cielo.
\par 32 En aquel día la noche disminuye y asciende a nueve partes, y el día a nueve partes, y la noche es igual al día y el año es exactamente igual a sus días trescientos sesenta y cuatro.
\par 33 Y la duración del día y de la noche, y la brevedad del día y de la noche surgen: a través del curso del sol se hacen estas distinciones (literalmente, 'están separadas').
\par 34 De modo que su curso se hace cada día más largo y su curso cada noche más corto.
\par 35 Y ésta es la ley y el curso del sol, y su regreso tantas veces como regresa sesenta veces y sale, es decir, la gran lumbrera que se llama sol, por los siglos de los siglos.
\par 36 Y lo que (así) se eleva es la gran luminaria, y recibe ese nombre según su apariencia, tal como el Señor ordenó.
\par 37 Así como sube, así se pone, y no disminuye, ni descansa, sino que corre día y noche, y su luz es siete veces más brillante que la de la luna; pero en cuanto a tamaño ambos son iguales.

\chapter{73}

\par 1 Y después de esta ley vi otra ley que se refería a la luminaria más pequeña, que se llama Luna.
\par 2 Y su circunferencia es como la circunferencia del cielo, y su carro en el que viaja es impulsado por el viento, y la luz le es dada en medida (definida).
\par 3 Y su salida y su puesta cambian cada mes; y sus días son como los días del sol, y cuando su luz es uniforme (es decir, plena), equivale a la séptima parte de la luz del sol.
\par 4 Y así ella se levanta. Y su primera fase en el este sale en la trigésima mañana: y en ese día ella se hace visible, y constituye para vosotros la primera fase de la luna en el día trigésimo junto con el sol en el portal por donde sale el sol.
\par 5 Y la mitad de ella avanza en una séptima parte, y toda su circunferencia está vacía, sin luz, a excepción de una séptima parte de ella, (y) la decimocuarta parte de su luz.
\par 6 Y cuando ella recibe la séptima parte de la mitad de su luz, su luz equivale a la séptima parte y la mitad de ella.
\par 7 Y ella se pone con el sol, y cuando el sol sale, la luna sale con él y recibe la mitad de una parte de luz, y en esa noche, al comienzo de su mañana [al comienzo del día lunar] la La luna se pone con el sol, y se hace invisible aquella noche con las catorce partes y la mitad de uno de ellos.
\par 8 Y ella sale ese día exactamente con una séptima parte, y sale y retrocede desde la salida del sol, y en sus días restantes se vuelve brillante en las trece partes (restantes).

\chapter{74}

\par 1 Y vi otro curso, una ley para ella, (y) cómo según esa ley ella realiza su revolución mensual.
\par 2 Y todos estos Uriel, el santo ángel que es el líder de todos ellos, me mostró y sus posiciones, y yo anoté sus posiciones tal como él me las mostró, y anoté sus meses tal como eran, y se cumplió la aparición de sus lumbreras hasta quince días.
\par 3 En una séptima parte realiza toda su luz en el oriente, y en una séptima parte completa toda su oscuridad en el oeste.
\par 4 Y en ciertos meses ella altera su configuración, y en ciertos meses sigue su propio curso peculiar.
\par 5 En dos meses la luna se pone con el sol: en esos dos portales del medio el tercero y el cuarto.
\par 6 Sale durante siete días, da la vuelta y regresa por la puerta por donde sale el sol, y cumple toda su luz; y se aleja del sol, y en ocho días entra por la sexta puerta por donde sale el sol adelante.
\par 7 Y cuando el sol sale por la cuarta puerta, sale siete días, hasta que sale por la quinta y en siete días vuelve a la cuarta puerta y cumple toda su luz; y retrocede y entra en la primer portal en ocho días.
\par 8 Y ella regresa nuevamente al cabo de siete días a la cuarta puerta por donde sale el sol.
\par 9 Así vi su posición, cómo salían las lunas y se ponía el sol en aquellos días.
\par 10 Y si se suman cinco años, al sol le sobran treinta días, y todos los días que le corresponden durante uno de esos cinco años, cuando están llenos, suman 364 días.
\par 11 Y el excedente del sol y de las estrellas asciende a seis días: en 5 años, 6 días cada año suman 30 días; y la 12 luna queda detrás del sol y de las estrellas en número de 30 días.
\par 12 Y el sol y las estrellas introducen todos los años exactamente, de modo que no avanzan ni retrasan su posición ni un solo día hasta la eternidad; pero completa los años con perfecta justicia en 364 días.
\par 13 En 3 años hay 1.092 días, y en 5 años 1.820 días, de modo que en 8 años hay 2.912 días.
\par 14 Sólo para la luna los días ascienden en 3 años a 1.062 días,
\par 15 y en 5 años se retrasa 50 días: [es decir, a la suma (de 1.770) hay que sumarle 5 (1.000 y) 62 días.] Y en 5 años hay 1.770 días, de modo que para la luna los días 6 en 8 años suman 21.832 días.
\par 16 [Porque en 8 años se retrasa hasta la cantidad de 80 días], todos los 17 días que se retrasa en 8 años son 80. Y el año se completa exactamente de conformidad con sus estaciones mundiales y las estaciones del sol, que salen por los portales por donde sale y se pone 30 días.

\chapter{75}

\par 1 Y los jefes de los jefes de miles, que están sobre toda la creación y sobre todas las estrellas, también tienen que ver con los cuatro días intercalados, siendo inseparables de su cargo, según el cómputo del año, y éstos prestan servicio los cuatro días que no se cuentan en el cómputo del año.
\par 2 Y debido a ellos los hombres se equivocan en eso, porque esas luminarias verdaderamente prestan servicio en las estaciones del mundo, una en el primer portal, otra en el tercer portal del cielo, otra en el cuarto portal y otra en el sexto portal, y la exactitud del año se logra a través de sus trescientas sesenta y cuatro estaciones separadas.
\par 3 Porque las señales, los tiempos, los años y los días me mostró el ángel Uriel, a quien el Señor de la gloria ha puesto para siempre sobre todas las lumbreras del cielo, en el cielo y en el mundo, para que gobernar sobre la faz del cielo y ser vistos en la tierra, y ser líderes del día y de la noche, es decir, el sol, la luna y las estrellas, y todas las criaturas ministrantes que hacen su revolución en todos los carros del cielo.
\par 4 De la misma manera Uriel me mostró doce puertas, abiertas en la circunferencia del carro del sol en el cielo, a través de las cuales salen los rayos del sol; y de ellas se difunde el calor sobre la tierra, cuando se abren por su parte a sus estaciones designadas.
\par 5 [Y para los vientos y el espíritu del rocío cuando están abiertos, estando abiertos en los cielos en los extremos.]
\par 6 En cuanto a las doce puertas del cielo, en los confines de la tierra, de donde salen el sol, la luna, las estrellas y todas las obras del cielo en oriente y occidente,
\par 7 Hay muchas ventanas abiertas a izquierda y derecha de ellas, y una ventana en su estación produce calor, correspondiente (como éstas) a aquellas puertas por donde salen las estrellas según Él les ha ordenado, y donde fijan correspondiente a su número.
\par 8 Y vi carros en el cielo, corriendo por el mundo, sobre aquellas puertas en las que giran las estrellas que nunca se ponen.
\par 9 Y uno es más grande que todos los demás, y es el que recorre el mundo entero.

\chapter{76}

\par 1 Y en los confines de la tierra vi doce puertas abiertas a todos los confines (del cielo), de donde salen los vientos y soplan sobre la tierra.
\par 2 Tres de ellos están abiertos en la cara (es decir, el este) de los cielos, y tres en el oeste, y tres a la derecha (es decir, el sur) del cielo, y tres a la izquierda (es decir, el norte).
\par 3 Los tres primeros son los del este, tres del norte, tres del sur y tres del oeste.
\par 4 A través de cuatro de ellos vienen vientos de bendición y prosperidad, y de esos ocho vienen vientos dañinos: cuando son enviados, traen destrucción sobre toda la tierra y sobre el agua sobre ella, y sobre todos los que habitan en ella y sobre todo lo que hay en el agua y en la tierra.
\par 5 Y el primer viento de aquellas puertas, llamado viento del este, sale por la primera puerta que está en el este, inclinándose hacia el sur: de allí salen desolación, sequía, calor y destrucción.
\par 6 Y por la segunda puerta que está en el medio sale lo que conviene, y de ella sale la lluvia, la fertilidad, la prosperidad y el rocío; y por la tercera puerta que está hacia el norte entran el frío y la sequía.
\par 7 Y después de estos salen los vientos del sur por tres puertas: por la primera de ellas, que se inclina hacia el este, sale un viento caliente.
\par 8 Y por la puerta del medio, junto a ella, salen olores fragantes, rocío, lluvia, prosperidad y salud.
\par 9 Y por la tercera puerta que está hacia el oeste salen rocío y lluvia, langostas y desolación.
\par 10 Y después de estos, los vientos del norte: desde la séptima puerta, en el oriente, vienen el rocío y la lluvia, las langostas y la desolación.
\par 11 Y desde el portal del medio vienen en dirección directa la salud, la lluvia, el rocío y la prosperidad; y por la tercera puerta, en el oeste, entran nubes y escarcha, nieve, lluvia, rocío y langostas.
\par 12 Y después de estos [cuatro] vienen los vientos del oeste: por la primera puerta que da al norte salen el rocío y la escarcha, el frío, la nieve y la escarcha.
\par 13 Y del portal del medio sale rocío y lluvia, prosperidad y bendición; y por el último portal que linda con el sur salen la sequía y la desolación, el incendio y la destrucción.
\par 14 Y con ello quedan completadas las doce puertas de las cuatro partes del cielo, y todas sus leyes, todas sus plagas y todos sus beneficios te he mostrado a ti, hijo mío Matusalén.

\chapter{77}

\par 1 Y el primer barrio se llama oriente, porque es el primero, y el segundo, el sur, porque allí descenderá el Altísimo y, en un sentido muy especial, descenderá allí el bendito por los siglos.
\par 2 Y el sector occidental se llama disminuido, porque allí todos los astros del cielo menguan y disminuyen.
\par 3 Y el cuarto sector, llamado Norte, se divide en tres partes: la primera es para la morada de los hombres; y la segunda contiene mares de agua, y abismos, bosques, ríos, tinieblas y nubes; y la tercera parte contiene el jardín de la justicia.
\par 4 Vi siete montes muy altos, más altos que todos los montes que hay sobre la tierra; de allí sale la escarcha, y pasan los días, las estaciones y los años.
\par 5 Vi en la tierra siete ríos más grandes que todos los ríos: uno de ellos, que viene del oeste, vierte sus aguas en el Gran Mar.
\par 6 Y estos dos vienen desde el norte hacia el mar y vierten sus aguas en el mar Eritrea, en el este.
\par 7 Y los restantes, cuatro salen por el lado del norte hacia su propio mar, dos de ellos hacia el mar Eritreo y dos hacia el mar Grande y desembarcan allí [y algunos dicen: en el desierto].
\par 8 Siete grandes islas vi en el mar y en el continente: dos en el continente y cinco en el Gran Mar.

\chapter{78}

\par 1 Y los nombres del sol son los siguientes: el primero Orjares, y el segundo Tomás.
\par 2 Y la luna tiene cuatro nombres: el primer nombre es Asonja, el segundo Ebla, el tercero Benase y el cuarto Erae.
\par 3 Estos son los dos grandes astros: su circunferencia es como la circunferencia del cielo, y el tamaño de la circunferencia de ambos es igual.
\par 4 En la circunferencia del sol hay siete porciones de luz que se le añaden más que a la luna, y en medidas definidas se transfiere hasta que se agota la séptima porción del sol.
\par 5 Y se ponen y entran por las puertas del oeste, hacen su revolución por el norte y salen por las puertas del este sobre la faz del cielo.
\par 6 Y cuando sale la luna, aparece en el cielo la catorceava parte: [la luz se llena en ella]: al decimocuarto día ella cumple su luz.
\par 7 Y se le transfieren quince partes de luz hasta el día quince (cuando) su luz se cumple, según el signo del año, y ella se convierte en quince partes, y la luna crece en (la suma de) catorce partes .
\par 8 Y en su menguante (la luna) disminuye el primer día a catorce partes de su luz, el segundo a trece partes de luz, el tercero a doce, el cuarto a once, el quinto a diez, el sexto a las nueve, el séptimo a las ocho, el octavo a las siete, el noveno a las seis, el décimo a las cinco, el undécimo a las cuatro, el duodécimo a las tres, el decimotercero a las dos, el decimocuarto a la mitad de un séptimo, y toda la luz restante desaparece por completo el día quince.
\par 9 Y en ciertos meses el mes tiene veintinueve días y una vez veintiocho.
\par 10 Y Uriel me mostró otra ley: cuándo se transfiere la luz a la luna y de qué lado se transfiere hacia ella el sol.
\par 11 Durante todo el período en que la luna crece en su luz, ella se la transfiere a sí misma cuando frente al sol durante catorce días su luz se cumple en el cielo, y cuando está completamente iluminada, su luz se cumple llena en el cielo.
\par 12 Y el primer día la llaman luna nueva, porque ese día la luz brilla sobre ella.
\par 13 Ella se convierte en luna llena exactamente el día en que el sol se pone por el oeste, y desde el este sale por la noche, y la luna brilla toda la noche hasta que el sol sale frente a ella y la luna se ve frente al sol.
\par 14 Del lado de donde sale la luz de la luna, allí nuevamente mengua hasta que toda la luz se desvanece y todos los días del mes llegan a su fin, y su circunferencia queda vacía, sin luz.
\par 15 Y hace tres meses de treinta días, y a su vez hace tres meses de veintinueve días cada uno, en los cuales cumple su mengua en el primer período de tiempo, y en el primer portal durante ciento setenta. -siete días.
\par 16 Y en el tiempo de su salida aparece durante tres meses de treinta días cada uno, y durante tres meses aparece de veintinueve días cada uno.
\par 17 De noche parece un hombre durante veinte días cada vez, y de día parece el cielo, y no hay nada más en ella que su luz.

\chapter{79}

\par 1 Y ahora, hijo mío, te lo he mostrado todo y se ha cumplido la ley de todas las estrellas del cielo.
\par 2 Y me mostró todas las leyes de estos para cada día, y para cada estación del gobierno, y para cada año, y para su salida, y para el orden prescrito para cada mes y cada semana:
\par 3 Y la luna menguante que tiene lugar en el sexto portal: porque en este sexto portal se cumple su luz, y después comienza el menguante:
\par 4 (Y el menguante) que tiene lugar en la primera puerta de su estación, hasta que se cumplan ciento setenta y siete días: contados según semanas, veinticinco (semanas) y dos días.
\par 5 Ella se queda detrás del sol y del orden de las estrellas exactamente cinco días en el transcurso de un período, y cuando el lugar que ves ha sido atravesado.
\par 6 Tal es el cuadro y el boceto de cada luminaria que me mostró el arcángel Uriel, que es su líder.

\chapter{80}

\par 1 Y en aquellos días el ángel Uriel respondió y me dijo: «He aquí, te he mostrado todo, Enoc, y te he revelado todo para que veas este sol y esta luna, y los líderes de las estrellas del cielo y de todos los que los giran, sus tareas y tiempos y salidas.
\par 2 Y en los días de los pecadores los años se acortarán, y su semilla se retrasará en sus tierras y campos, y todas las cosas sobre la tierra cambiarán y no aparecerán a su tiempo; y la lluvia será retenido Y el cielo lo retendrá.
\par 3 Y en aquellos tiempos los frutos de la tierra se retrasarán y no crecerán en su tiempo, y los frutos de los árboles serán retenidos en su tiempo.
\par 4 Y la luna cambiará su orden y no aparecerá a su tiempo.
\par 5 [Y en aquellos días se verá el sol y él viajará al atardecer en el extremo del gran carro en el oeste] y brillará más intensamente de lo que concuerda con el orden de la luz.
\par 6 Y muchos jefes de las estrellas transgredirán el orden (prescrito). Y estos alterarán sus órbitas y tareas, y no aparecerán en las estaciones que les sean prescritas.
\par 7 Y todo el orden de las estrellas quedará oculto a los pecadores, y los pensamientos de los que están en la tierra se equivocarán respecto a ellas, [y serán alterados de todos sus caminos], sí, se equivocarán y los tomarán por dioses.
\par 8 Y el mal se multiplicará sobre ellos, y el castigo les sobrevendrá hasta el punto de destruirlo todo.»

\chapter{81}

\par 1 Y me dijo: «Observa, Enoc, estas tablas celestiales, lee lo que en ellas está escrito y marca cada hecho individual».
\par 2 Y observé las tablas celestiales, leí todo lo que estaba escrito (en ellas), y entendí todo, y leí el libro de todos los hechos de la humanidad y de todos los hijos de la carne que habrán sobre la tierra hasta el más remoto generaciones.
\par 3 Y en seguida bendije al gran Señor, Rey de gloria para siempre, porque hizo todas las obras del mundo, y ensalcé al Señor por su paciencia, y lo bendije por los hijos de los hombres.
\par 4 Y después dije: «Bienaventurado el hombre que muere en justicia y bondad, sobre quien no hay libro escrito de injusticia, y contra quien no se encontrará día de juicio».
\par 5 Y aquellos siete santos me trajeron y me colocaron en el suelo ante la puerta de mi casa, y me dijeron: «Declara todo a tu hijo Matusalén, y muestra a todos tus hijos que ninguna carne es justa ante los ojos del Señor, porque él es su Creador».
\par 6 «Un año te dejaremos con tu hijo, hasta que des tus (últimas) órdenes, para que puedas enseñar a tus hijos y registrarlas para ellos, y testificar a todos tus hijos; y en el segundo año te quitarán de en medio de ellos».
\par 7 «Esté fuerte tu corazón, porque los buenos anunciarán la justicia a los buenos; los justos con los justos se alegrarán y se felicitarán unos a otros».
\par 8 «Pero los pecadores morirán con los pecadores, y el apóstata descenderá con el apóstata».
\par 9 «Y los que practican la justicia morirán por las obras de los hombres, y serán quitados por las obras de los impíos».
\par 10 En aquellos días dejaron de hablarme y fui a mi pueblo bendiciendo al Señor del mundo.

\chapter{82}

\par 1 ¡Y ahora, hijo mío Matusalén, te cuento todas estas cosas y te las escribo! y te lo he revelado todo, y te he dado libros sobre todo esto; así que conserva, hijo mío Matusalén, los libros de la mano de tu padre, y (mira) que los entregues a las generaciones del mundo.
\par 2 Te he dado Sabiduría a ti y a tus hijos, para que la den a sus hijos por generaciones, esta sabiduría que va más allá de sus pensamientos.
\par 3 Y los que la entiendan no dormirán, sino que escucharán con el oído para aprender esta sabiduría, y agradará más a los que la comen que la buena comida.
\par 4 Bienaventurados todos los justos, bienaventurados todos los que caminan por el camino de la justicia y no pecan como los pecadores, en el cómputo de todos sus días, en los que el sol recorre el cielo, entrando y saliendo de las puertas de treinta días con las cabezas de miles del orden de las estrellas, junto con las cuatro que están intercaladas que dividen las cuatro porciones del año, que las conducen y entran con ellas cuatro días.
\par 5 Debido a ellos, los hombres cometerán falta y no los contarán en todo el cómputo del año; incluso los hombres cometerán falta y no los reconocerán con precisión.
\par 6 Porque pertenecen al cómputo del año y están verdaderamente registrados para siempre, uno en el primer portal, otro en el tercero, uno en el cuarto y otro en el sexto, y el año se completa en trescientos sesenta y cuatro días.
\par 7 Y su cuenta es exacta y su cómputo registrado exacto; porque las luminarias, y los meses, y las fiestas, y los años y los días, me ha mostrado y revelado Uriel, a quien el Señor de toda la creación del mundo ha sometido el ejército del cielo.
\par 8 Y tiene poder sobre la noche y el día en el cielo para hacer que la luz alumbre a los hombres: el sol, la luna y las estrellas, y todos los poderes del cielo que giran en sus carros circulares.
\par 9 Y éste es el orden de las estrellas, que se ponen en sus lugares, en sus estaciones, en sus fiestas y en sus meses.
\par 10 Y estos son los nombres de los que los dirigen, los que vigilan que entren en sus tiempos, en sus órdenes, en sus estaciones, en sus meses, en sus períodos de dominio y en sus posiciones.
\par 11 Sus cuatro jefes, que dividen las cuatro partes del año, entran primero; y después de ellos los doce jefes de las órdenes que dividen los meses; y para los trescientos sesenta (días) hay cabezas de millares que dividen los días; y durante los cuatro días intercalares están los líderes que dividen las cuatro partes del año.
\par 12 Y estas cabezas de miles están intercaladas entre líder y líder, cada uno detrás de una estación, pero sus líderes hacen la división.
\par 13 Y estos son los nombres de los líderes que dividen las cuatro partes del año que están ordenadas: Milki'el, Hel'emmelek, Melejal y Narel.
\par 14 Y los nombres de los que los dirigen: Adnar'el, Ijasusa'el y 'Elome'el; estos tres siguen a los líderes de las órdenes, y hay uno que sigue a los tres líderes de las órdenes que siguen a aquellos líderes de estaciones que dividen las cuatro partes del año.
\par 15 Al comienzo del año, Melkejal se levanta primero y gobierna, quien se llama Tamaini y sol, y todos los días de su dominio mientras gobierna son noventa y un días.
\par 16 Y estas son las señales de los días que se verán en la tierra en los días de su dominio: sudor, calor y calma; y todos los árboles dan fruto, y se producen hojas en todos los árboles, y la cosecha de trigo, y las rosas, y todas las flores que brotan en el campo, pero los árboles de la estación invernal se marchitan.
\par 17 Y estos son los nombres de los jefes que están bajo sus mandos: Berka'el, Zelebs'el, y otro al que se añade un jefe de mil, llamado Hilujaseph. Y los días del dominio de este (líder) son al final.
\par 18 El siguiente líder después de él es Hel'emmelek, a quien se llama sol brillante, y todos los días de su luz son noventa y un días.
\par 19 Y estas son las señales de (sus) días en la tierra: calor abrasador y sequedad, y los árboles maduran sus frutos y producen todos sus frutos maduros y listos, y las ovejas se aparean y quedan preñadas, y todos los frutos de se recoge la tierra, y todo lo que hay en el campo, y el lagar: estas cosas acontecen en los días de su señorío.
\par 20 Estos son los nombres, las órdenes y los líderes de esos jefes de mil: Gida'ljal, Ke'el y He'el, y el nombre del jefe de mil que se les añade, Asfa. 'el: y los días de su dominio han llegado a su fin.

\part{Sección IV. Capítulos LXXXIII—XC. Las visiones oníricas.}

\chapter{83}

\par 1 Ahora, pues, Matusalén, hijo mío, te mostraré todas las visiones que he tenido y las contaré delante de ti.
\par 2 Dos visiones tuve antes de tomar esposa, y una era muy diferente de la otra: la primera cuando estaba aprendiendo a escribir; la segunda antes de tomar a tu madre, (cuando) tuve una visión terrible. Y por ellos oré al Señor.
\par 3 Me había acostado en casa de mi abuelo Mahalalel, (cuando) vi en una visión cómo el cielo se derrumbó y fue arrastrado y cayó a la tierra.
\par 4 Y cuando cayó a la tierra, vi cómo la tierra era tragada en un gran abismo, y las montañas quedaban suspendidas sobre las montañas, y las colinas se hundían sobre las colinas, y los árboles altos se arrancaban de sus troncos y eran arrojados hacia abajo y hundido en el abismo.
\par 5 Y entonces una palabra cayó en mi boca, y alcé (mi voz) para gritar en voz alta, y dije: «La tierra está destruida».
\par 6 Y mi abuelo Mahalaleel me despertó mientras yacía cerca de él y me dijo:
\par 7 «¿Por qué lloras así, hijo mío, y por qué te lamentas así?» Y le conté toda la visión que había visto, y él me dijo: «Una cosa terrible has visto, hijo mío, y de grave momento es tu visión onírica en cuanto a los secretos de todos los pecados de la tierra. : debe hundirse en el abismo y ser destruido con gran destrucción».
\par 8 «Y ahora, hijo mío, levántate y suplica al Señor de la gloria, ya que eres creyente, que quede un remanente sobre la tierra y que Él no destruya toda la tierra».
\par 9 «Hijo mío, desde el cielo todo esto vendrá sobre la tierra, y sobre la tierra habrá una gran destrucción».
\par 10 Después me levanté y oré, imploré y supliqué, y escribí mi oración para las generaciones del mundo, y todo te lo mostraré a ti, mi hijo Matusalén.
\par 11 Y cuando bajé y vi el cielo, y el sol saliendo por el este, y la luna poniéndose por el oeste, y algunas estrellas, y toda la tierra, y todo como Él lo había conocido en el principio, entonces bendije al Señor del juicio y lo ensalcé porque había hecho salir el sol por las ventanas del oriente, y ascendió y se levantó sobre la faz del cielo, y se puso en camino y siguió recorriendo el camino mostrado a a él.

\chapter{84}

\par 1 Y alcé mis manos en justicia y bendije al Santo y Grande, y hablé con el aliento de mi boca y con la lengua de carne que Dios hizo para los hijos de la carne de los hombres, para que debía hablar con él, y les dio aliento, lengua y boca para que hablaran con él:
\par 2 «Bendito seas, oh Señor, Rey, grande y poderoso en tu grandeza, Señor de toda la creación del cielo, Rey de reyes y Dios del mundo entero. Y tu poder, tu realeza y tu grandeza permanecen por los siglos de los siglos, y por todas las generaciones tu dominio y todos los cielos son tu trono para siempre, y toda la tierra el estrado de tus pies por los siglos de los siglos».
\par 3 «Porque Tú hiciste y gobiernas todas las cosas, y nada es demasiado difícil para Ti, la Sabiduría no se aparta del lugar de Tu trono, Ni se aparta de Tu presencia. Y Tú lo sabes, lo ves y lo oyes todo, y no hay nada oculto para Ti [porque Tú lo ves todo]».
\par 4 «Y ahora los ángeles de tus cielos son culpables de transgresión, y tu ira está sobre la carne de los hombres hasta el gran día del juicio».
\par 5 «Y ahora, oh Dios, Señor y Gran Rey, te imploro y suplico que cumplas mi oración, que me dejes una posteridad en la tierra, y que no destruyas toda la carne del hombre, y dejes la tierra sin habitantes, para que debería haber una destrucción eterna».
\par 6 Ahora pues, Señor mío, destruye de la tierra la carne que ha despertado tu ira, pero la carne de justicia y rectitud establece como planta de semilla eterna, y no escondas tu rostro de la oración de tu siervo. Oh Señor.»

\chapter{85}

\par 1 Y después de esto tuve otro sueño, y te mostraré todo el sueño, hijo mío. Y Enoc levantó (su voz) y habló a su hijo Matusalén:
\par 2 «A ti, hijo mío, hablaré: escucha mis palabras; inclina tu oído a los sueños de tu padre».
\par 3 «Antes de tomar a tu madre Edna, tuve una visión en mi cama, y ​​he aquí un toro que salía de la tierra, y ese toro era blanco; y después salió una novilla, y junto con ésta salieron dos toros, uno de ellos negro y el otro rojo».
\par 4 «Y ese toro negro corneó al rojo y lo persiguió por la tierra, y entonces ya no pude ver ese toro rojo».
\par 5 «Pero aquel toro negro creció y la novilla iba con él, y vi que de él salían muchos bueyes que se le parecían y lo seguían».
\par 6 «Y esa vaca, la primera, se fue de la presencia del primer toro para buscar al rojo, pero no lo encontró, y se lamentó con gran lamento por él y lo buscó».
\par 7 «Y miré hasta que el primer toro vino hacia ella y la calmó, y desde ese momento en adelante no lloró más».
\par 8 «Y después de esto dio a luz otro toro blanco, y después de él dio a luz muchos toros y vacas negras».
\par 9 «Y vi en sueños que el toro blanco también crecía y se convertía en un gran toro blanco, y de él procedían muchos toros blancos, y se parecían a él».
\par 10 «Y comenzaron a engendrar muchos toros blancos, que se parecían a ellos, uno tras otro, (incluso) muchos».

\chapter{86}

\par 1 Y otra vez vi con mis ojos mientras dormía, y miré el cielo arriba, y he aquí una estrella que cayó del cielo, y se levantó y comió y pastaba entre aquellos bueyes.
\par 2 Después vi los bueyes grandes y los negros, y he aquí que todos cambiaron de establos, de pastos y de ganado, y empezaron a vivir unos con otros.
\par 3 Y de nuevo miré en la visión y miré hacia el cielo, y he aquí vi muchas estrellas descender y arrojarse del cielo hacia la primera estrella, y se convirtieron en toros entre aquellos ganados y pastaron con ellos [entre ellos].
\par 4 Y los miré y vi, y he aquí que todos dejaban salir sus miembros privados, como caballos, y comenzaron a cubrir las vacas de los bueyes, y todos quedaron preñados y dieron a luz elefantes, camellos y asnos.
\par 5 Y todos los bueyes los temieron y tuvieron miedo de ellos, y comenzaron a morder con los dientes, a devorar y a acornear con los cuernos.
\par 6 Y además comenzaron a devorar aquellos bueyes; y he aquí todos los hijos de la tierra comenzaron a temblar y a temblar delante de ellos y a huir de ellos.

\chapter{87}

\par 1 Y de nuevo vi cómo comenzaron a cornearse unos a otros y a devorarse unos a otros, y la tierra comenzó a gritar con fuerza.
\par 2 Y alcé de nuevo mis ojos al cielo y vi en la visión, y he aquí que del cielo salían unos seres parecidos a hombres blancos; y de aquel lugar salían cuatro y tres con ellos.
\par 3 Y los tres últimos que habían salido me tomaron de la mano y me alzaron lejos de las generaciones de la tierra, y me elevaron a un lugar elevado, y me mostraron una torre elevada por encima de la tierra, y todas las colinas eran más bajas.
\par 4 Y uno me dijo: «Quédate aquí hasta que veas todo lo que les sucede a esos elefantes, camellos y asnos, y las estrellas y los bueyes, y a todos ellos».

\chapter{88}

\par 1 Y vi a uno de los cuatro que habían salido primero, y agarró la primera estrella que había caído del cielo, la ató de pies y manos y la arrojó al abismo, que ahora era estrecho y profundo, y horrible y oscuro.
\par 2 Y uno de ellos sacó una espada y se la dio a aquellos elefantes, camellos y asnos; entonces comenzaron a golpearse unos a otros, y toda la tierra tembló a causa de ellos.
\par 3 Y mientras yo contemplaba la visión, uno de los cuatro que habían salido (los) apedreó desde el cielo, y juntó y tomó todas las grandes estrellas cuyos miembros eran como los de los caballos, y los ató a todos de manos y pies, y los arrojó al abismo de la tierra.

\chapter{89}

\par 1 Y uno de aquellos cuatro se acercó al toro blanco y le instruyó en secreto, sin que éste se asustara: nació toro y se hizo hombre, y se construyó un gran barco y habitó en él; y tres toros moraban con él en aquella vasija y estaban cubiertos.
\par 2 Y otra vez levanté mis ojos hacia el cielo y vi un techo alto, sobre el cual siete torrentes de agua corrían con mucha agua hacia un recinto.
\par 3 Y volví a mirar, y he aquí que se abrieron fuentes en la superficie de aquel gran recinto, y el agua empezó a hincharse y a subir sobre la superficie, y vi aquel recinto hasta que toda su superficie se cubrió de agua.
\par 4 Y el agua, la oscuridad y la niebla aumentaron sobre él; y mientras miraba la altura de esa agua, esa agua se había elevado por encima de la altura de ese recinto, y fluía sobre ese recinto, y estaba sobre la tierra.
\par 5 Y todo el ganado de aquel recinto se reunió hasta que vi cómo se hundían y eran tragados y perecían en aquella agua.
\par 6 Pero aquel barco flotaba en el agua, mientras todos los bueyes, los elefantes, los camellos y los asnos se hundían hasta el fondo con todos los animales, de modo que ya no podía verlos y no podían escapar, (pero) pereció y se hundió en las profundidades.
\par 7 Y volví a ver en la visión hasta que aquellos torrentes de agua fueron quitados de aquel alto techo, y los abismos de la tierra fueron nivelados y otros abismos se abrieron.
\par 8 Entonces el agua empezó a correr dentro de ellos, hasta que la tierra se hizo visible; pero aquella vasija se posó en la tierra, y las tinieblas se retiraron y apareció la luz.
\par 9 Pero de aquella vasija salió aquel toro blanco que se había hecho hombre, y los tres toros con él, y uno de esos tres era blanco como aquel toro, y uno de ellos era rojo como la sangre, y el otro negro: y aquel toro blanco se alejó de ellos.
\par 10 Y comenzaron a criar animales del campo y aves, de modo que surgieron diferentes géneros: leones, tigres, lobos, perros, hienas, jabalíes, zorros, ardillas, cerdos, halcones, buitres, milanos, águilas, y cuervos; y entre ellos nació un toro blanco.
\par 11 Y comenzaron a morderse unos a otros; pero aquel toro blanco que nació entre ellos engendró un asno montés y con él un toro blanco, y los asnos montés se multiplicaron.
\par 12 Pero el toro que nació de él engendró un jabalí negro y una oveja blanca; y el primero engendró muchos jabalíes, pero la oveja engendró doce ovejas.
\par 13 Y cuando aquellas doce ovejas crecieron, entregaron una de ellas a los asnos, y esos asnos volvieron a entregar esa oveja a los lobos, y esa oveja creció entre los lobos.
\par 14 Y el Señor trajo las once ovejas para que vivieran con ella y pastaran con ella entre los lobos; y se multiplicaron y se convirtieron en muchos rebaños de ovejas.
\par 15 Y los lobos empezaron a temerlas y las oprimieron hasta matar a sus pequeños y arrojaron a sus crías a un río lleno de agua; pero aquellas ovejas comenzaron a gritar a causa de sus pequeños, y a quejarse ante su Señor.
\par 16 Y una oveja que había sido salvada de los lobos huyó y escapó a los asnos salvajes; y vi a las ovejas cómo se lamentaban y lloraban, y rogaban a su Señor con todas sus fuerzas, hasta que aquel Señor de las ovejas descendió a la voz de las ovejas desde una morada elevada, y vino a ellas y las apacentó.
\par 17 Y llamó a la oveja que había escapado de los lobos y le habló acerca de los lobos para que les advirtiera que no tocaran a las ovejas.
\par 18 Y las ovejas fueron donde los lobos conforme a la palabra del Señor, y otra oveja la encontró y fue con ella, y las dos fueron y entraron juntas en la reunión de aquellos lobos, y hablaron con ellos y les amonestaron con no tocar a las ovejas de ahora en adelante.
\par 19 Entonces vi a los lobos y cómo oprimieron a las ovejas con todo su poder; y las ovejas gritaron en voz alta.
\par 20 Y el Señor se acercó a las ovejas y ellas comenzaron a herir a los lobos, y los lobos comenzaron a lamentarse; pero las ovejas se callaron y al instante dejaron de gritar.
\par 21 Y vi las ovejas hasta que se apartaron de entre los lobos; pero los ojos de los lobos quedaron cegados, y aquellos lobos partieron persiguiendo a las ovejas con todas sus fuerzas.
\par 22 Y el Señor de las ovejas iba con ellas como líder, y todas sus ovejas le seguían; y su rostro era resplandeciente, glorioso y terrible de contemplar.
\par 23 Pero los lobos comenzaron a perseguir a aquellas ovejas hasta que llegaron a un mar de agua.
\par 24 Y el mar se dividió, y el agua se paró de un lado y de otro delante de ellos, y su Señor los guió y se puso entre ellos y los lobos.
\par 25 Y como aquellos lobos aún no habían visto a las ovejas, se adentraron en medio de ese mar, y los lobos siguieron a las ovejas, y [esos lobos] corrieron tras ellas hacia ese mar.
\par 26 Y cuando vieron al Señor de las ovejas, se volvieron para huir delante de Él, pero el mar se juntó y volvió a ser como había sido creado, y el agua se hinchó y subió hasta cubrir a esos lobos.
\par 27 Y vi hasta que todos los lobos que perseguían a aquellas ovejas perecieron y se ahogaron.
\par 28 Pero las ovejas escaparon de aquella agua y se fueron al desierto, donde no había agua ni hierba; y comenzaron a abrir los ojos y a ver; y vi al Señor de las ovejas apacentándolas y dándoles agua y pasto, y aquella oveja yendo y guiándolas.
\par 29 Y la oveja subió a la cima de aquella roca elevada, y el Señor de las ovejas se la envió.
\par 30 Y después vi al Señor de las ovejas que estaba delante de ellas, y su apariencia era grande, terrible y majestuosa, y todas aquellas ovejas lo vieron y tuvieron miedo ante su presencia.
\par 31 Y todos temieron y temblaron ante Él, y clamaron a la oveja que estaba con ellos [que estaba entre ellas]: «No podemos estar delante de nuestro Señor ni contemplarlo».
\par 32 Y la oveja que los guiaba subió de nuevo a la cima de aquella roca, pero las ovejas comenzaron a quedar cegadas y a desviarse del camino que él les había mostrado, pero la oveja no lo sabía.
\par 33 Y el Señor de las ovejas se enojó muchísimo contra ellas, y las ovejas lo descubrieron, descendieron de la cima de la roca, llegaron hasta las ovejas y encontraron a la mayor parte de ellas ciegas y caídas.
\par 34 Y cuando lo vieron, temieron y temblaron ante su presencia, y desearon volver a sus rebaños.
\par 35 Y aquella oveja tomó consigo otras ovejas, y vino hacia las que se habían descarriado y comenzó a matarlas; y las ovejas temieron su presencia, y así aquella oveja trajo de vuelta a las ovejas que se habían apartado, y ellas regresaron a sus rediles.
\par 36 Y vi en esta visión hasta que la oveja se hizo hombre y edificó una casa para el Señor de las ovejas, y puso a todas las ovejas en esa casa.
\par 37 Y vi hasta que esta oveja que se había encontrado con la oveja que las guiaba se durmió; y vi hasta que todas las ovejas grandes perecieron y las pequeñas se levantaron en su lugar, y llegaron a un pasto, y se acercaron a un arroyo de agua.
\par 38 Entonces la oveja, su líder, que se había hecho hombre, se apartó de ellos y se durmió, y todas las ovejas la buscaron y lloraron con gran llanto.
\par 39 Y vi hasta que dejaron de llorar por esa oveja y cruzaron aquella corriente de agua, y allí las dos ovejas se levantaron como líderes en lugar de aquellas que las habían conducido y se habían quedado dormidas (literalmente, «se habían quedado dormidas y las habían conducido»).
\par 40 Y vi hasta que las ovejas llegaron a un lugar bueno, a una tierra placentera y gloriosa, y vi hasta que aquellas ovejas quedaron saciadas; y esa casa estaba entre ellos en la tierra agradable.
\par 41 Y unas veces se les abrían los ojos, y otras se les cegaba, hasta que se levantaba otra oveja, las guiaba y las hacía volver a todas, y se les abrían los ojos.
\par 42 Y los perros, las zorras y los jabalíes comenzaron a devorar aquellas ovejas, hasta que el Señor de las ovejas levantó [otra oveja] de en medio de ellas un carnero que las guiaba.
\par 43 Y el carnero empezó a golpear a los perros, a las zorras y a los jabalíes por ambos lados hasta destruirlos a todos.
\par 44 Y la oveja que tenía los ojos abiertos vio el carnero que estaba entre las ovejas, hasta que abandonó su gloria y comenzó a golpearlas, a pisotearlas y a comportarse de manera indecorosa.
\par 45 Y el Señor de las ovejas envió el cordero a otro cordero y lo levantó para que fuera carnero y líder de las ovejas en lugar del carnero que había abandonado su gloria.
\par 46 Y fue hacia él y le habló a solas, y lo levantó hasta convertirlo en carnero, y lo hizo príncipe y guía de las ovejas; pero durante todo esto aquellos perros oprimían a las ovejas.
\par 47 Y el primer carnero persiguió al segundo carnero, y éste se levantó y huyó delante de él; y vi hasta que esos perros derribaron al primer carnero.
\par 48 Y el segundo carnero se levantó y guió a las ovejitas. Y aquel carnero engendró muchas ovejas y se durmió; y una ovejita se convirtió en carnero en su lugar, y llegó a ser príncipe y líder de aquellas ovejas.
\par 49 Y aquellas ovejas crecieron y se multiplicaron; pero todos los perros, las zorras y los jabalíes temieron y huyeron ante él, y aquel carnero embistió y mató a las fieras, y aquellas fieras ya no tenían ningún poder entre las ovejas y ya no les robaban nada.
\par 50 Y esa casa se hizo grande y ancha, y fue construida para aquellas ovejas; (y) sobre la casa se construyó una torre alta y grande para el Señor de las ovejas, y esa casa era baja, pero la torre era elevada y altivo, y el Señor de las ovejas se paró en esa torre y ofrecieron una mesa llena delante de Él.
\par 51 Y de nuevo vi que aquellas ovejas se desviaban de nuevo y se alejaban por muchos caminos, y abandonaban su casa, y el Señor de las ovejas llamó a algunas de entre las ovejas y las envió a las ovejas, pero las ovejas comenzaron a matarlas.
\par 52 Y uno de ellos se salvó y no fue asesinado, y salió corriendo y gritó a gran voz sobre las ovejas; y quisieron matarla, pero el Señor de las ovejas la salvó de las ovejas, la trajo a mí y la hizo habitar allí.
\par 53 Y envió muchas otras ovejas a aquellas ovejas para que les testificaran y se lamentaran por ellas.
\par 54 Y después vi que cuando abandonaron la casa del Señor y su torre, cayeron por completo y sus ojos quedaron cegados; y vi al Señor de las ovejas cómo provocó mucha matanza entre ellas en sus rebaños hasta que esas ovejas invitaron a esa matanza y traicionaron Su lugar.
\par 55 Y los entregó en manos de los leones y de los tigres, de los lobos y de las hienas, en manos de las zorras y de todas las fieras, y aquellas fieras comenzaron a despedazar a aquellas ovejas.
\par 56 Y vi que abandonó su casa y su torre y los entregó a todos en manos de los leones para desgarrarlos y devorarlos, en manos de todas las fieras.
\par 57 Y comencé a gritar con todas mis fuerzas y a apelar al Señor de las ovejas y a declararle que las ovejas eran devoradas por todas las fieras.
\par 58 Pero Él, aunque lo vio, permaneció impasible y se alegró de que fueran devorados, tragados y robados, y los dejó para que fueran devorados en manos de todas las bestias.
\par 59 Y llamó a setenta pastores, y les arrojó aquellas ovejas para que las apacentaran, y habló a los pastores y a sus compañeros: «De ahora en adelante, cada uno de vosotros apacentará las ovejas y todo lo que os ordenaré que hazlo».
\par 60 «Y os los entregaré debidamente contados, y os diré cuáles de ellos serán destruidos, y ellos los destruiréis vosotros». Y les entregó aquellas ovejas.
\par 61 Y llamó a otro y le habló: «Observa y observa todo lo que los pastores harán con esas ovejas; porque destruirán a más de ellos de los que les he ordenado».
\par 62 «Y todo exceso y destrucción que se produzca a través de los pastores, registra (es decir) cuántos destruyen según mi orden, y cuántos según su propio capricho: registra contra cada pastor individual toda la destrucción que efectúa.»
\par 63 «Y lee delante de mí por número cuántos destruyen y cuántos entregan para la destrucción, para que tenga esto como testimonio contra ellos, y sepa cada obra de los pastores, para que pueda comprender y ver lo que hacen, cumplan o no mis órdenes que les he ordenado».
\par 64 «Pero ellos no lo sabrán, y no se lo declararás ni los amonestarás, sino que sólo registrarás contra cada individuo toda la destrucción que los pastores hacen cada uno en su tiempo y lo contarás todo ante mí».
\par 65 Y vi que aquellos pastores pastaban a su debido tiempo y comenzaron a matar y destruir más de las que se les había ordenado, y entregaron aquellas ovejas en manos de los leones.
\par 66 Y los leones y los tigres comieron y devoraron a la mayor parte de aquellas ovejas, y los jabalíes comieron junto con ellas; y quemaron esa torre y demolieron esa casa.
\par 67 Y me entristecí mucho por esa torre porque la casa de las ovejas fue demolida, y después no pude ver si esas ovejas entraban en esa casa.
\par 68 Y los pastores y sus compañeros entregaron aquellas ovejas a todas las fieras para que las devoraran, y cada uno de ellos recibió en su tiempo un número determinado: el otro escribía en un libro cuántas de cada una de ellas destruyeron.
\par 69 Y cada uno mató y destruyó muchos más de los prescritos; y comencé a llorar y a lamentarme a causa de aquellas ovejas.
\par 70 Y así, en la visión vi al que escribía, cómo anotaba cada uno de los que aquellos pastores destruían, día tras día, y los llevaba, los depositaba y mostraba en realidad todo el libro al Señor de las ovejas. —(incluso) todo lo que habían hecho, y todo lo que cada uno de ellos había destruido, y todo lo que habían entregado a la destrucción.
\par 71 Y fue leído el libro delante del Señor de las ovejas, y Él tomó el libro de su mano, lo leyó, lo selló y lo dejó.
\par 72 Y en seguida vi cómo los pastores pastaban durante doce horas, y he aquí tres de aquellas ovejas regresaron y entraron y comenzaron a reconstruir todo lo que se había caído de aquella casa; pero los jabalíes intentaron impedírselo, pero no pudieron.
\par 73 Y comenzaron de nuevo a construir como antes, y levantaron esa torre, y la llamaron torre alta; Y comenzaron de nuevo a poner una mesa delante de la torre, pero todo el pan que había sobre ella estaba contaminado y no puro.
\par 74 Y en cuanto a todo esto, los ojos de aquellas ovejas quedaron cegados para no ver, y lo mismo (los ojos de) sus pastores; y las entregaron en gran número a sus pastores para destrucción, y pisotearon las ovejas con sus pies y las devoraron.
\par 75 Y el Señor de las ovejas permaneció impasible hasta que todas las ovejas se dispersaron por el campo y se mezclaron con ellas (es decir, las bestias), y ellos (es decir, los pastores) no las salvaron de las manos de las bestias.
\par 76 Y el que escribió el libro lo llevó, lo mostró y lo leyó delante del Señor de las ovejas, y le imploró por ellas, y le suplicó por ellas mientras le mostraba todas las obras de los pastores. , y dio testimonio delante de él contra todos los pastores.
\par 77 Entonces tomó el libro, lo dejó a su lado y se fue.

\chapter{90}

\par 1 Y vi hasta que de esta manera treinta y cinco pastores se encargaron del pastoreo (de las ovejas), y cada uno de ellos cumplió sus períodos como el primero; y otros los recibieron en sus manos, para apacentarlos durante su tiempo, cada pastor en su propio tiempo.
\par 2 Y después vi en mi visión venir todas las aves del cielo: las águilas, los buitres, los milanos y los cuervos; pero las águilas guiaban a todas las aves; y comenzaron a devorar aquellas ovejas, y a sacarles los ojos y a devorar su carne.
\par 3 Y las ovejas gritaban porque los pájaros devoraban su carne, y yo miraba y me lamentaba en sueños por aquel pastor que apacentaba las ovejas.
\par 4 Y vi hasta que aquellas ovejas fueron devoradas por los perros, las águilas y los milanos, y no les quedó ni carne, ni piel, ni tendones, hasta que sólo quedaron allí sus huesos; y también sus huesos cayeron a la tierra, y las ovejas se convirtieron en pocos.
\par 5 Y vi que veintitrés habían emprendido el pastoreo y habían completado en sus distintos períodos cincuenta y ocho veces.
\par 6 Pero he aquí que aquellas ovejas blancas llevaban corderos, y ellas comenzaron a abrir los ojos y a ver, y a gritar a las ovejas.
\par 7 Incluso les clamaron, pero no escucharon lo que les decían, sino que estaban sumamente sordos y sus ojos estaban sumamente cegados.
\par 8 Y vi en la visión cómo los cuervos volaban sobre aquellos corderos, tomaban uno de ellos, despedazaban a las ovejas y las devoraban.
\par 9 Y vi hasta que a aquellos corderos les crecieron cuernos y los cuervos arrojaron sus cuernos; y vi hasta que a una de aquellas ovejas le salió un gran cuerno, y se les abrieron los ojos.
\par 10 Y el animal los miró y gritó a las ovejas, y los carneros lo vieron y todos corrieron hacia él.
\par 11 Y a pesar de todo esto, aquellas águilas, y buitres, cuervos y milanos seguían despedazando a las ovejas, lanzándose sobre ellas y devorándolas; las ovejas seguían en silencio, pero los carneros se lamentaban y gritaban.
\par 12 Y aquellos cuervos lucharon y lucharon contra él y trataron de derribarle el cuerno, pero no tenían poder sobre él.
\par 13 Se reunieron todas las águilas, los buitres, los cuervos y los milanos, y con ellos vinieron todas las ovejas del campo, y todos se juntaron y se ayudaron unos a otros a quebrar el cuerno del carnero.
\par 14 [versículo 17 repetido]
\par 15 [versículo 18 repetido]
\par 16 [versículo 13 repetido]
\par 17 Y vi a aquel hombre que escribió el libro según el mandato del Señor, hasta que abrió el libro sobre la destrucción que habían causado aquellos doce últimos pastores, y mostró que habían destruido mucho más que sus predecesores, ante el Señor de las ovejas.
\par 18 Y vi hasta que el Señor de las ovejas vino a ellos y tomó en su mano el bastón de su ira, e hirió la tierra, y la tierra se partió en pedazos, y todas las bestias y todas las aves del cielo cayeron entre aquellas ovejas, y fueron tragados por la tierra y ella los cubrió.
\par 19 Y miré hasta que una gran espada fue dada a las ovejas, y las ovejas procedieron contra todas las bestias del campo para matarlas, y todas las bestias y las aves del cielo huyeron delante de ellas.
\par 20 Y vi hasta que se erigió un trono en la tierra agradable, y el Señor de las ovejas se sentó allí, y el otro tomó los libros sellados y los abrió delante del Señor de las ovejas.
\par 21 Y el Señor llamó a aquellos hombres los siete primeros blancos, y les ordenó que trajeran delante de Él, comenzando por la primera estrella que guiaba el camino, todas las estrellas cuyos miembros eran como los de los caballos, y los trajeron todos delante de Él.
\par 22 Y le dijo al hombre que escribía delante de Él, siendo uno de aquellos siete blancos, y le dijo: «Toma esos setenta pastores a quienes entregué las ovejas, y que tomándolas por su propia autoridad mataron a más de Yo les ordené».
\par 23 Y he aquí que todos estaban atados, y todos estaban delante de Él.
\par 24 Y el juicio se celebró primero sobre las estrellas, y fueron juzgados y declarados culpables, y fueron al lugar de condenación, y fueron arrojados a un abismo, lleno de fuego y llamas, y lleno de columnas de fuego.
\par 25 Y aquellos setenta pastores fueron juzgados y declarados culpables, y arrojados a aquel abismo de fuego.
\par 26 Y vi en aquel momento cómo se abrió un abismo semejante en medio de la tierra, lleno de fuego, y trajeron aquellas ovejas ciegas, y todas fueron juzgadas y declaradas culpables y arrojadas a este abismo de fuego, y quemado; ahora este abismo estaba a la derecha de aquella casa.
\par 27 Y vi aquellas ovejas ardiendo y sus huesos ardiendo.
\par 28 Y me levanté para ver hasta que cerraron esa vieja casa; y se llevaron todas las columnas, y todas las vigas y adornos de la casa fueron al mismo tiempo dobladas con él, y lo llevaron y lo pusieron en un lugar al sur de la tierra.
\par 29 Y vi que el Señor de las ovejas trajo una casa nueva, más grande y más alta que la primera, y la levantó en lugar de la primera, que estaba doblada; todas sus columnas eran nuevas y sus adornos eran nuevos y más grandes que los del primero, el viejo que había quitado, y todas las ovejas estaban dentro de él.
\par 30 Y vi todas las ovejas que habían quedado, y todos los animales de la tierra, y todas las aves del cielo, postrándose y rindiendo homenaje a aquellas ovejas, pidiéndoles y obedeciéndoles en todo.
\par 31 Y entonces aquellos tres que estaban vestidos de blanco y me habían agarrado de la mano, y la mano del carnero también me agarró, me levantaron y me pusieron en el suelo en medio de aquellas ovejas antes de que se llevara a cabo el juicio.
\par 32 Y aquellas ovejas eran todas blancas, y su lana era abundante y limpia.
\par 33 Y todos los que habían sido destruidos y dispersos, y todas las bestias del campo y todas las aves del cielo, se reunieron en esa casa, y el Señor de las ovejas se regocijó con gran alegría porque todos eran buenos y habían Regresó a su casa.
\par 34 Y vi hasta que dejaron la espada que les habían dado a las ovejas, la llevaron de regreso a la casa, y fue sellada delante de la presencia del Señor, y todas las ovejas fueron invitadas a esa casa. , pero no los retuvo.
\par 35 Y a todos se les abrieron los ojos y vieron el bien, y no hubo entre ellos uno que no viera.
\par 36 Y vi que aquella casa era grande, espaciosa y muy llena.
\par 37 Y vi que nació un toro blanco, con grandes cuernos, y todas las bestias del campo y todas las aves del cielo le temían y le rogaban todo el tiempo.
\par 38 Y vi hasta que todas sus generaciones fueron transformadas, y todos se convirtieron en toros blancos; y el primero entre ellos se convirtió en un cordero, y ese cordero se convirtió en un animal grande y tenía grandes cuernos negros en su cabeza; y el Señor de las ovejas se regocijó por ella y por todos los bueyes.
\par 39 Y dormí en medio de ellos; y desperté y vi todo.
\par 40 Esta es la visión que tuve mientras dormía, cuando desperté y bendije al Señor de la justicia y le di gloria.
\par 41 Entonces lloré con gran llanto y mis lágrimas no cesaron hasta que ya no pude soportarlo más: cuando vi, brotaron a causa de lo que había visto; porque todo vendrá y se cumplirá, y me fueron mostradas todas las obras de los hombres en su orden.
\par 42 Aquella noche me acordé del primer sueño, y a causa de él lloré y me turbé, porque había visto esa visión.

\part {Sección V. XCI—CIV (es decir, XCII, XCI. 1-10, 18-19, XCIII. 1-10, XCI. 12-17, XCIV—CIV.). Un libro de exhortación y bendición prometida para los justos y de maldición y ay para los pecadores.}

\chapter{91}

\par 1 «Y ahora, hijo mío Matusalén, llama a todos tus hermanos y reúneme a todos los hijos de tu madre; porque la palabra me llama, y ​​el espíritu se derrama sobre mí, para mostraros todo lo que os acontecerá para siempre».
\par 2 Y allí fue a Matusalén y llamó a todos sus hermanos y reunió a sus parientes.
\par 3 Y habló a todos los hijos de justicia y dijo: «Oíd, hijos de Enoc, todas las palabras de vuestro padre, y escuchad atentamente la voz de mi boca; porque os exhorto y os digo, amados: Amad la rectitud y andad en ella».
\par 4 «Y no os acerquéis a la rectitud con doble corazón, ni os asociéis con personas de doble corazón, sino caminad en rectitud, hijos míos. Y os guiará por buenos senderos, y la justicia será vuestra compañera».
\par 5 «Porque sé que la violencia aumentará en la tierra, y un gran castigo será ejecutado en la tierra, y toda injusticia llegará a su fin: sí, será cortada de sus raíces, y toda su estructura será destruida.»
\par 6 «Y la injusticia volverá a consumarse en la tierra, y todos los actos de injusticia, violencia y transgresión prevalecerán en doble grado».
\par 7 «Y cuando aumenten el pecado, la injusticia, la blasfemia y la violencia en toda clase de actos, y aumenten la apostasía, la transgresión y la inmundicia, un gran castigo vendrá del cielo sobre todos ellos, y el Santo Señor saldrá con ira y castigo. Para ejecutar juicio en la tierra».
\par 8 En aquellos días la violencia será cortada de raíz, y las raíces de la injusticia junto con el engaño, y serán destruidas de debajo del cielo.
\par 9 «Y todos los ídolos de las naciones serán abandonados, y los templos quemados con fuego, y los quitarán de toda la tierra, y ellos (es decir, los paganos) serán arrojados al juicio del fuego, y serán perezcan en ira y en juicio doloroso para siempre».
\par 10 «Y los justos se levantarán de su sueño, y la sabiduría surgirá y les será dada».
\par 11 «[Y después de eso las raíces de la injusticia serán cortadas, y los pecadores serán destruidos por la espada. . . serán cortados de los blasfemos en todo lugar, y los que planean violencia y los que cometen blasfemia perecerán a espada.]»
\par 12 [este versículo debe leerse después del capítulo 93] 12 Y después habrá otra, la octava semana, la de la justicia, y se le dará una espada para que se ejecute un juicio justo sobre los opresores, y Los pecadores serán entregados en manos de los justos.
\par 13 [este versículo debe leerse después del capítulo 93] 13 Y al final adquirirán casas mediante su justicia, y se construirá una casa para el Gran Rey en gloria para siempre, 14d Y toda la humanidad mirará hacia el camino de rectitud.
\par 14 [este versículo debe leerse después del capítulo 93] [a] Y después de eso, en la novena semana, el juicio justo será revelado al mundo entero, [b] y todas las obras de los impíos desaparecerán de todos la tierra, [c] Y el mundo será escrito para destrucción.
\par 15 [este versículo debe leerse después del capítulo 93] Y después de esto, en la décima semana de la séptima parte, tendrá lugar el gran juicio eterno, en el cual Él ejecutará venganza entre los ángeles.
\par 16 [este versículo debe leerse después del capítulo 93] Y el primer cielo pasará y pasará, y aparecerá un cielo nuevo, y todos los poderes de los cielos darán siete luces.
\par 17 [este versículo debe leerse después del capítulo 93] Y después de eso habrá muchas semanas sin número para siempre, y todo será en bondad y justicia, y el pecado nunca más será mencionado.
\par 18 Y ahora os lo digo, hijos míos, y os muestro los caminos de la justicia y los caminos de la violencia. Sí, os las mostraré otra vez para que sepáis lo que sucederá.
\par 19 Ahora pues, escúchenme, hijos míos, y caminen por las sendas de la justicia, y no anden por las sendas de la violencia; porque todos los que andan por sendas de injusticia perecerán para siempre.

\chapter{92}

\par 1 El libro escrito por Enoc: Enoc ciertamente escribió esta doctrina completa de sabiduría, (que es) alabada por todos los hombres y juez de toda la tierra, para todos mis hijos que habitarán la tierra. Y para las generaciones futuras que observarán la rectitud y la paz.
\par 2 No se turbe vuestro espíritu a causa de los tiempos; porque el Santo y Grande ha señalado días para todas las cosas.
\par 3 Y el justo se levantará del sueño y caminará por los senderos de la justicia, y todo su camino y su conducta serán en bondad y gracia eternas.
\par 4 Él será misericordioso con el justo y le dará rectitud eterna, y le dará poder para que sea (dotado) de bondad y justicia. Y caminará en la luz eterna.
\par 5 Y el pecado perecerá en las tinieblas para siempre, y desde aquel día nunca más será visto.

\chapter{93}

\par 1 Y después de eso, Enoc dio y comenzó a contar los libros.
\par 2 Y Enoc dijo: «Con respecto a los hijos de justicia, a los elegidos del mundo y a la planta de rectitud, hablaré estas cosas. Sí, yo Enoc os las declararé, hijos míos: a lo que me apareció en la visión celestial, y que he conocido por la palabra de los santos ángeles, y he aprendido en las tablas celestiales».
\par 3 Y Enoc comenzó a contar los libros y dijo: «Yo nací el séptimo en la primera semana, mientras aún permanecían el juicio y la justicia».
\par 4 «Y después de mí surgirá en la segunda semana gran maldad y habrá surgido el engaño; y en él estará el primer fin. Y en él el hombre será salvo; y después que termine, crecerá la injusticia y se dictará una ley para los pecadores».
\par 5 «Y después de eso, al terminar la tercera semana, un hombre será elegido como planta de justo juicio, y su posteridad se convertirá en planta de justicia para siempre».
\par 6 «Y después de eso, en la cuarta semana, al terminar, se verán visiones de los santos y de los justos, y se les dictará una ley para todas las generaciones y un cercado».
\par 7 «Y después de eso, en la quinta semana, cuando termine, la casa de gloria y dominio será edificada para siempre».
\par 8 «Y después de eso, en la sexta semana, todos los que habitan en ella serán cegados, y el corazón de todos ellos abandonará impíamente la sabiduría. Y en él ascenderá un hombre; y a su fin la casa de dominio será quemada al fuego, y toda la raza de la raíz escogida será dispersada».
\par 9 «Y después de eso, en la séptima semana, se levantará una generación apóstata, y sus obras serán muchas, y todas sus obras serán apóstatas».
\par 10 «Y al final serán elegidos Los justos elegidos de la eterna planta de justicia, para recibir siete instrucciones sobre toda Su creación».
\par 11 Porque ¿quién entre todos los hijos de los hombres puede oír la voz del Santo sin perturbarse? ¿Y quién puede pensar Sus pensamientos y quién hay que pueda contemplar todas las obras del cielo?»
\par 12 «¿Y cómo podría haber alguien que pueda contemplar el cielo, y que haya allí que pueda entender las cosas del cielo y ver un alma o un espíritu y poder decirlo, o ascender y ver todos sus fines y pensarlos o ¿Te gustan?
\par 13 ¿Y quién entre todos los hombres puede saber cuál es la anchura y la longitud de la tierra y a quién se le ha mostrado la medida de todas ellas?
\par 14 ¿O hay alguien que pueda discernir la longitud del cielo, su altura, su fundamento, el número de las estrellas y dónde descansan todas las lumbreras?



\chapter{94}

\par 1 Y ahora os digo, hijos míos, que améis la justicia y andéis por ella; porque los caminos de la justicia son dignos de ser aceptados, pero los caminos de la injusticia serán repentinamente destruidos y desaparecerán.
\par 2 Y a ciertos hombres de una generación se les revelarán los caminos de la violencia y de la muerte, y se mantendrán alejados de ellos y no los seguirán.
\par 3 Y ahora os digo a vosotros, los justos: No andéis por los caminos de la maldad, ni por los caminos de la muerte, ni os acerquéis a ellos, para que no seáis destruidos.
\par 4 Pero buscad y escoged para vosotros la justicia y una vida elegida, y andad por los caminos de la paz, y viviréis y prosperaréis.
\par 5 Y retened mis palabras en los pensamientos de vuestros corazones, y no permitáis que se borren de vuestros corazones; porque sé que los pecadores tentarán a los hombres para que imploren mal la sabiduría, de modo que no se encuentre lugar para ella y ninguna tentación minimice.
\par 6 ¡Ay de los que construyen injusticia y opresión y ponen como fundamento el engaño! porque de repente serán destruidos y no tendrán paz.
\par 7 ¡Ay de los que construyen sus casas con pecado! porque de todos sus cimientos serán derribados, y a espada caerán. [Y los que adquieran oro y plata en el juicio, de repente perecerán.]
\par 8 ¡Ay de vosotros, ricos!, porque habéis confiado en vuestras riquezas, y de ellas os apartaréis, porque no os habéis acordado del Altísimo en los días de vuestras riquezas.
\par 9 Habéis cometido blasfemia e injusticia, y os habéis preparado para el día de la matanza, el día de las tinieblas y el día del gran juicio.
\par 10 Así os hablo y os declaro: El que os creó os derribará, y no habrá compasión por vuestra caída, y vuestro Creador se alegrará de vuestra destrucción.
\par 11 Y en aquellos días vuestros justos serán afrenta de los pecadores y de los impíos.

\chapter{95}

\par 1 ¡Oh, si mis ojos fueran [una nube de] agua para llorar sobre ti y derramar mis lágrimas como una nube de agua, para descansar de la angustia de mi corazón!
\par 2 ¿Quién os ha permitido cometer afrentas y maldades? ¡Y así os alcanzará el juicio, pecadores!
\par 3 No temáis, oh justos, a los pecadores; porque otra vez el Señor los entregará en vuestras manos, para que ejecutéis en ellos juicio según vuestras peticiones.
\par 4 ¡Ay de vosotros, que fulmináis anatemas que no se pueden revertir! ¡Por tanto, la curación estará lejos de vosotros a causa de vuestros pecados!
\par 5 ¡Ay de ti, que pagas mal a tu prójimo! ¡Porque seréis recompensados ​​según vuestras obras!
\par 6 ¡Ay de vosotros, testigos mentirosos y de los que sopesan la injusticia, porque de repente pereceréis!
\par 7 ¡Ay de vosotros, pecadores, que perseguís a los justos! porque seréis entregados y perseguidos a causa de la injusticia, y su yugo será pesado sobre vosotros.

\chapter{96}

\par 1 Tened esperanza, justos; porque repentinamente los pecadores perecerán delante de vosotros, y tendréis señorío sobre ellos según vuestras concupiscencias.
\par 2 [Y en el día de la tribulación de los pecadores, vuestros hijos subirán y se elevarán como las águilas, y vuestro nido será más alto que los buitres, y ascenderéis y entraréis en las grietas de la tierra y en las hendiduras de la roca para siempre como conejos ante los injustos, y las sirenas suspirarán por ti y llorarán.]
\par 3 Por tanto, no temáis, los que habéis sufrido; porque la curación será vuestra porción, y una luz brillante os iluminará, y la voz del descanso oiréis desde el cielo.
\par 4 ¡Ay de vosotros, pecadores, porque vuestras riquezas os hacen parecer justos, pero vuestro corazón os convence de ser pecadores, y este hecho será un testimonio contra vosotros como recordatorio de (vuestras) malas acciones!
\par 5 ¡Ay de vosotros, que devoráis lo mejor del trigo, bebéis vino en grandes copas y pisoteáis a los humildes con vuestra fuerza!
\par 6 ¡Ay de vosotros, los que bebéis agua de toda fuente, porque de repente os consumiréis y os marchitaréis, porque habéis abandonado la fuente de la vida!
\par 7 ¡Ay de vosotros, los que hacéis injusticia, engaño y blasfemia! ¡Será contra vosotros un memorial del mal!
\par 8 ¡Ay de vosotros, valientes, que oprimís con fuerza al justo! ¡Porque llega el día de tu destrucción! En aquellos días vendrán muchos y buenos días para los justos, en el día de vuestro juicio.

\chapter{97}

\par 1 Creed, justos, que los pecadores serán avergonzados y perecerán en el día de la injusticia.
\par 2 Sepan vosotros (pecadores) que el Altísimo se acuerda de vuestra destrucción, y los ángeles del cielo se alegran por vuestra destrucción.
\par 3 ¿Qué haréis, pecadores, y adónde huiréis en el día del juicio, cuando oigáis la voz de la oración de los justos?
\par 4 Sí, os irá como aquellos contra quienes será testimonio esta palabra: «Habéis sido compañeros de pecadores».
\par 5 Y en aquellos días la oración de los justos llegará al Señor, y para vosotros vendrán los días de vuestro juicio.
\par 6 Y todas las palabras de vuestra injusticia serán leídas delante del Gran Santo, y vuestros rostros se cubrirán de vergüenza, y Él rechazará toda obra que esté basada en la injusticia.
\par 7 ¡Ay de vosotros, pecadores, que habitáis en medio del océano y en la tierra seca, cuyo recuerdo es malo contra vosotros!
\par 8 ¡Ay de vosotros, que adquirís plata y oro con injusticia y decís: «Nos hemos enriquecido con riquezas y tenemos posesiones; ¡Y hemos adquirido todo lo que hemos deseado!
\par 9 «Y ahora hagamos lo que nos propusimos: ¡porque hemos reunido plata!» [c] «¡Y muchos son los labradores en nuestras casas!» [d] «¡Y nuestros graneros están (hasta el tope) llenos como de agua!»
\par 10 Y como agua correrán vuestras mentiras; porque vuestras riquezas no permanecerán sino que rápidamente ascenderán de vosotros; porque todo lo habéis adquirido con injusticia, y seréis entregados a gran maldición.

\chapter{98}

\par 1 Y ahora os lo juro, a los sabios y a los necios, que tendréis muchas experiencias en la tierra.
\par 2 Porque los hombres se pondrán más adornos que una mujer, y vestidos de colores más que una virgen: en realeza, en grandeza y en poder, en plata, en oro, en púrpura, en esplendor y en comida fluirán como agua.
\par 3 Por tanto, les faltará doctrina y sabiduría, y por ello perecerán junto con sus bienes; ¡Y con toda su gloria y su esplendor, y en vergüenza y en matanza y en gran miseria, sus espíritus serán arrojados en el horno de fuego!
\par 4 Os he jurado, pecadores, que como una montaña no se ha convertido en esclava, ni una colina se ha convertido en esclava de una mujer, así tampoco el pecado ha sido enviado a la tierra, sino que el hombre mismo lo ha creado, y bajo gran maldición caerán los que lo cometan.
\par 5 Y a la mujer no le ha sido dado esterilidad, sino que a causa de las obras de sus propias manos muere sin hijos.
\par 6 Os he jurado, pecadores, por el Santo Grande, que todas vuestras malas acciones están reveladas en los cielos, y que ninguna de vuestras obras de opresión está cubierta ni oculta.
\par 7 Y no penséis en vuestro espíritu ni digáis en vuestro corazón que no sabéis y que no veis que cada pecado queda registrado cada día en el cielo delante del Altísimo.
\par 8 Desde ahora sabéis que toda vuestra opresión con que oprimís queda escrita todos los días hasta el día de vuestro juicio.
\par 9 ¡Ay de vosotros, necios, porque a causa de vuestra necedad pereceréis, y transgrediéis contra los sabios, y la buena suerte no será vuestra parte!
\par 10 Ahora pues, sabed que estáis preparados para el día de la destrucción. Por tanto, pecadores, no esperéis vivir, sino que partiréis y moriréis; porque no sabéis rescate; porque estáis preparados para el día del gran juicio, para el día de la tribulación y de gran vergüenza para vuestro espíritu.
\par 11 ¡Ay de vosotros, obstinados de corazón, que hacéis maldades y coméis sangre! ¿De dónde tenéis cosas buenas para comer y beber y saciaros de todos los bienes que el Señor Altísimo ha puesto en abundancia sobre la tierra? Por tanto, no tendréis paz.
\par 12 ¡Ay de los que aman las obras de injusticia! Por tanto, esperáis buena suerte para vosotros mismos, sabiendo que seréis entregados en manos de los justos, y os cortarán el cuello y os matarán, y no tendrán misericordia de vosotros.
\par 13 ¡Ay de vosotros, que os regocijáis en la tribulación de los justos! ¡Porque no se cavará ninguna tumba para vosotros!
\par 14 ¡Ay de vosotros, que despreciáis las palabras del justo! porque no tendréis esperanza de vida.
\par 15 ¡Ay de vosotros, los que escribís palabras mentirosas y impías! porque escriben sus mentiras para que los hombres las escuchen y actúen impíamente hacia (su) prójimo.
\par 16 Por tanto, no tendrán paz sino que morirán de muerte repentina.

\chapter{99}

\par 1 ¡Ay de vosotros, los que cometéis impiedad y os enorgullezcéis de mentir y los ensalzáis! Pereceréis y no tendréis una vida feliz.
\par 2 ¡Ay de aquellos que pervierten las palabras de rectitud, transgreden la ley eterna y se transforman en lo que no eran! ¡Serán pisoteados en la tierra!
\par 3 En aquellos días, justos, preparaos para elevar vuestras oraciones como memorial y ponerlas como testimonio ante los ángeles, para que ellos pongan el pecado de los pecadores en memorial ante el Altísimo.
\par 4 En aquellos días se agitarán las naciones, y las familias de las naciones se levantarán en el día de la destrucción.
\par 5 Y en aquellos días los indigentes saldrán y se llevarán a sus hijos, y los abandonarán, de modo que sus hijos perecerán a causa de ellos; y abandonarán a sus hijos (que aún son) de pecho, y no volverán a ellos, y no tendrán piedad de sus amados.
\par 6 Y de nuevo os juro, pecadores, que el pecado está preparado para un día de incesante derramamiento de sangre.
\par 7 Y los que adoran piedras y ídolos de oro, plata, madera (y piedra) y barro, y los que adoran espíritus impuros y demonios, y toda clase de ídolos no según el conocimiento, no recibirán ayuda alguna de ellos.
\par 8 Y se volverán impíos a causa de la locura de su corazón, y sus ojos quedarán cegados por el miedo de su corazón y por las visiones de sus sueños.
\par 9 Por estas cosas se volverán impíos y temerosos; porque en mentira habrán hecho toda su obra, y a una piedra habrán adorado: ¡Por tanto, en un instante perecerán!
\par 10 Pero en aquellos días, bienaventurados todos los que aceptan las palabras de la sabiduría, las entienden, siguen los caminos del Altísimo, caminan por la senda de su justicia y no se vuelven impíos con los impíos porque serán salvos.
\par 11 ¡Ay de los que hacéis el mal a vuestros vecinos! ¡Porque seréis asesinados en el Seol!
\par 12 ¡Ay de vosotros, los que tomáis medidas engañosas y falsas, y los que causáis amargura en la tierra! ¡Porque así serán completamente consumidos!
\par 13 ¡Ay de vosotros, que construís vuestras casas con el duro trabajo de otros, y todos los materiales de construcción son ladrillos y piedras del pecado! ¡Os digo que no tendréis paz!
\par 14 ¡Ay de aquellos que rechazan la medida y la herencia eterna de sus padres y cuyas almas siguen a los ídolos! ¡Porque no tendrán descanso!
\par 15 ¡Ay de los que hacen injusticia, ayudan a la opresión y matan a su prójimo hasta el día del gran juicio!
\par 16 Porque él derribará vuestra gloria y traerá aflicción a vuestros corazones, despertará su furia y os destruirá a todos con la espada; y todos los santos y justos se acordarán de tus pecados.

\chapter{100}

\par 1 Y en aquellos días, en un mismo lugar, los padres junto con sus hijos serán heridos y los hermanos morirán unos con otros, hasta que los arroyos corran con su sangre.
\par 2 Porque el hombre no detendrá su mano para matar a sus hijos y a los hijos de sus hijos, ni el pecador detendrá su mano de su honrado hermano: desde el alba hasta el ocaso se matarán unos a otros.
\par 3 Y el caballo subirá hasta el pecho en la sangre de los pecadores, y el carro será hundido hasta su altura.
\par 4 En aquellos días los ángeles descenderán a los lugares secretos y reunirán en un solo lugar a todos los que hicieron caer el pecado. Y el Altísimo se levantará en ese día del juicio para ejecutar un gran juicio entre los pecadores.
\par 5 Y sobre todos los justos y santos nombrará guardianes de entre los santos ángeles, para que los guarden como a la niña de sus ojos, hasta que acabe con toda maldad y todo pecado, y aunque los justos duerman un largo sueño, no tienen nada que temer.
\par 6 Y (entonces) los hijos de la tierra verán a los sabios con seguridad, entenderán todas las palabras de este libro y reconocerán que sus riquezas no podrán salvarlos en la destrucción de sus pecados.
\par 7 ¡Ay de vosotros, pecadores, en el día de la gran angustia, los que afligís a los justos y los quemáis con fuego! ¡Os retribuirán según vuestras obras!
\par 8 ¡Ay de vosotros, obstinados de corazón, que acecháis para idear el mal! Por eso os sobrevendrá el miedo y no habrá quien os ayude.
\par 9 ¡Ay de vosotros, pecadores, por las palabras de vuestra boca y por las obras de vuestras manos que vuestra impiedad realizó, en llamas ardientes, que arden peor que el fuego!
\par 10 Ahora bien, sabed que Él preguntará a los ángeles sobre vuestras obras en el cielo, al sol, a la luna y a las estrellas, acerca de vuestros pecados, porque sobre la tierra ejecutáis juicio sobre los justos.
\par 11 Y Él convocará para testificar contra vosotros toda nube, niebla, rocío y lluvia; porque a todos ellos se les impedirá por tu culpa descender sobre ti, y se acordarán de tus pecados.
\par 12 Y ahora dad regalos a la lluvia, para que no le impida descender sobre vosotros, ni al rocío, cuando ha recibido de vosotros oro y plata para descender.
\par 13 Cuando caigan sobre vosotros la escarcha y la nieve con su frío, y todas las tormentas de nieve con todas sus plagas, en aquellos días no podréis hacerles frente.

\chapter{101}

\par 1 Observad el cielo, hijos del cielo, y toda obra del Altísimo, temedle y no hagáis mal en su presencia.
\par 2 Si Él cierra las ventanas de los cielos e impide que la lluvia y el rocío desciendan sobre la tierra por vuestra causa, ¿qué haréis entonces?
\par 3 Y si Él envía su ira contra vosotros a causa de vuestras obras, no podéis pedirle nada; porque habéis hablado palabras soberbias e insolentes contra su justicia; por tanto, no tendréis paz.
\par 4 ¿Y no veis a los marineros de las naves, cómo sus naves son sacudidas por las olas, y sacudidas por los vientos, y en duras dificultades?
\par 5 Por eso temen que todas sus riquezas se vayan con ellos al mar y tienen malos presentimientos de que el mar los tragará y perecerán en él.
\par 6 ¿No es todo el mar, todas sus aguas y todos sus movimientos obra del Altísimo, y él no ha puesto límites a sus obras y las ha limitado por completo a la arena?
\par 7 Y ante su reprensión, tiene miedo y se seca, y mueren todos sus peces y todo lo que hay en él; pero vosotros, pecadores que estáis en la tierra, no le teméis.
\par 8 ¿No hizo Él los cielos y la tierra y todo lo que en ellos hay, Quien dio entendimiento y sabiduría a todo lo que se mueve sobre la tierra y el mar?
\par 9 ¿No temen al mar los marineros de las naves, pero los pecadores no temen al Altísimo?

\chapter{102}

\par 1 En aquellos días en que Él haya traído sobre vosotros un fuego terrible, ¿adónde huiréis y dónde encontraréis salvación? Y cuando Él lance Su Palabra contra vosotros, ¿no os asustaréis y temeréis?
\par 2 Y todas las luminarias se espantarán de gran temor, y toda la tierra se espantará, temblará y se alarmará.
\par 3 Y todos los ángeles ejecutarán sus órdenes. Y buscarán esconderse de la presencia de la Gran Gloria, y los hijos de la tierra temblarán y temblarán; y vosotros, pecadores, seréis malditos para siempre, y no tendréis paz.
\par 4 No temáis, almas de justos, y tened esperanza, los que habéis muerto en justicia.
\par 5 Y no te entristezcas si tu alma ha descendido con tristeza al Seol, y si en tu vida tu cuerpo no ha ido según tu bondad, sino espera el día del juicio de los pecadores y el día de la maldición y del castigo.
\par 6 Y, sin embargo, cuando morís, los pecadores os hablan: «Como morimos nosotros, así mueren los justos, y ¿qué beneficio obtienen ellos de sus obras?»
\par 7 «He aquí, como nosotros, así ellos mueren en pena y oscuridad, y ¿qué tienen ellos más que nosotros? Desde ahora seremos iguales».
\par 8 «¿Y qué recibirán y qué verán para siempre? He aquí que también ellos han muerto, y de ahora en adelante no verán la luz para siempre».
\par 9 Os digo, pecadores, que os contentáis con comer y beber, robar, pecar, desnudar a los hombres, adquirir riquezas y ver días buenos.
\par 10 ¿Habéis visto a los justos cómo les cae el fin, sin que se halle en ellos ningún tipo de violencia hasta su muerte?
\par 11 «Sin embargo, perecieron y quedaron como si no hubieran existido, y sus espíritus descendieron al Seol en tribulación».

\chapter{103}

\par 1 Ahora, pues, os juro a vosotros, los justos, por la gloria del Grande, Honrado y Poderoso en dominio, y por su grandeza os lo juro.
\par 2 Conozco un misterio y he leído las tablas celestiales, he visto los libros sagrados y he encontrado en ellos escritos y escritos acerca de ellos:
\par 3 Que todo el bien, la alegría y la gloria estén preparados para ellos y escritos para los espíritus de aquellos que han muerto en justicia, y que a vosotros se os darán muchos bienes en recompensa por vuestros trabajos, y que vuestra suerte será abundante más allá de la suerte de los vivos.
\par 4 Y vuestros espíritus, que habéis muerto en justicia, vivirán y se regocijarán, y sus espíritus no perecerán, ni su recuerdo delante de la faz del Grande para todas las generaciones del mundo: por tanto, ya no tengáis miedo de su afrenta.
\par 5 ¡Ay de vosotros, pecadores, cuando habéis muerto, si muréis en la riqueza de vuestros pecados, y los que son como vosotros dicen de vosotros: «Bienaventurados los pecadores: han visto todos sus días!»
\par 6 «Y cómo han muerto en prosperidad y riqueza, y no han visto tribulación ni asesinato en su vida; y han muerto con honor, y no se ha ejecutado juicio sobre ellos durante su vida».
\par 7 Sabed que sus almas descenderán al Seol y serán desdichados en su gran tribulación.
\par 8 Y en las tinieblas, en las cadenas y en la llama ardiente, donde hay un juicio doloroso, entrarán vuestros espíritus; y el gran juicio será para todas las generaciones del mundo. ¡Ay de vosotros, porque no tendréis paz!
\par 9 No digas acerca de los justos y buenos que hay en la vida: «En nuestros días de angustia hemos trabajado laboriosamente y hemos experimentado todas las dificultades, y hemos sufrido muchos males y hemos sido consumidos, y nos hemos vuelto pocos y nuestro espíritu pequeño».
\par 10 «Y fuimos destruidos y no encontramos a nadie que nos ayudara ni siquiera con una palabra: fuimos torturados [y destruidos], y no esperábamos ver vida día tras día».
\par 11 «Esperábamos ser cabeza y hemos llegado a ser cola: nos hemos afanado y no hemos tenido satisfacción en nuestro trabajo; y nos hemos convertido en alimento de pecadores e injustos, y ellos han puesto pesadamente sobre nosotros su yugo».
\par 12 «Se han enseñoreado de nosotros, los que nos odiaban y nos golpeaban; y ante los que nos odiaban, hemos inclinado el cuello, pero no se han compadecido de nosotros».
\par 13 «Queríamos alejarnos de ellos para escapar y descansar, pero no encontramos un lugar al que huir y estar a salvo de ellos».
\par 14 «Y en nuestra tribulación nos quejamos ante los gobernantes y clamamos contra los que nos devoraron, pero ellos no escucharon nuestros gritos ni escucharon nuestra voz».
\par 15 «Y ayudaron a los que nos robaron y nos devoraron, y a los que nos hicieron pocos; y ocultaron su opresión, y no quitaron de nosotros el yugo de los que nos devoraron, nos dispersaron y nos asesinaron, y ocultaron su asesinato, y no se acordaron de que habían levantado sus manos contra nosotros».

\chapter{104}

\par 1 Os juro que en el cielo los ángeles se acordarán de vosotros para siempre ante la gloria del Grande, y vuestros nombres están escritos delante de la gloria del Grande.
\par 2 Ten esperanza; porque antes fuisteis avergonzados por el mal y la aflicción; pero ahora brillaréis como las luces del cielo, brillaréis y seréis vistos, y se os abrirán las puertas del cielo.
\par 3 Y en tu clamor, clama por juicio, y se te aparecerá; porque toda vuestra tribulación recaerá sobre los gobernantes y sobre todos los que ayudaron a los que os saquearon.
\par 4 Tened esperanza y no desechéis vuestras esperanzas, porque gozaréis de gran alegría como los ángeles del cielo.
\par 5 ¿Qué estaréis obligados a hacer? No tendréis que esconderos en el día del gran juicio y no seréis hallados pecadores, y el juicio eterno estará lejos de vosotros por todas las generaciones del mundo.
\par 6 Y ahora, justos, no temáis cuando veáis a los pecadores fortalecerse y prosperar en sus caminos; no seáis compañeros de ellos, sino manteneos alejados de su violencia; porque seréis compañeros de las huestes del cielo.
\par 7 Y aunque vosotros, pecadores, decís: «Todos nuestros pecados no serán investigados ni escritos», sin embargo escribirán todos vuestros pecados cada día.
\par 8 Y ahora os muestro que la luz y las tinieblas, de día y de noche, ven todos vuestros pecados.
\par 9 No seáis impíos en vuestros corazones, ni mientas, ni alteréis las palabras de rectitud, ni acuséis de mentir las palabras del Santo Grande, ni toméis en cuenta a vuestros ídolos; porque todas vuestras mentiras y toda vuestra impiedad no resultan en justicia sino en gran pecado.
\par 10 Y ahora conozco este misterio: que los pecadores alterarán y pervertirán de muchas maneras las palabras de justicia, hablarán malas palabras, mentirán, practicarán grandes engaños y escribirán libros sobre sus palabras.
\par 11 Pero cuando escriban fielmente todas mis palabras en sus idiomas, y no cambien ni minimicen mis palabras, sino que las escriban todas fielmente, todo lo que primero testifiqué acerca de ellos.
\par 12 Entonces conozco otro misterio: que los libros serán dados a los justos y a los sabios para que sean motivo de alegría, de rectitud y de mucha sabiduría.
\par 13 Y a ellos se les darán los libros, creerán en ellos y se alegrarán con ellos, y entonces todos los justos que hayan aprendido en ellos todos los caminos de la rectitud serán recompensados.

\chapter{105}

\par 1 En aquellos días, el Señor les mandó que convocaran y testificaran a los hijos de la tierra acerca de su sabiduría: Muéstrensela; porque vosotros sois sus guías y una recompensa sobre toda la tierra.
\par 2 Porque Mi hijo y yo estaremos unidos con ellos para siempre en los caminos de rectitud de sus vidas; y tendréis paz: alegraos, hijos de rectitud. Amén.

\part{Fragmento del Libro de Noé}

\chapter{106}

\par 1 Y después de algunos días, mi hijo Matusalén tomó esposa para su hijo Lamec, y ella quedó embarazada de él y dio a luz un hijo.
\par 2 Y su cuerpo era blanco como la nieve y rojo como la flor de una rosa, y el cabello de su cabeza y sus largos mechones eran blancos como la lana, y sus ojos hermosos. Y cuando abrió los ojos, iluminó toda la casa como el sol, y toda la casa quedó muy luminosa.
\par 3 Entonces se levantó en manos de la partera, abrió la boca y conversó con el Señor de la justicia.
\par 4 Y su padre Lamec tuvo miedo de él y huyó y vino a su padre Matusalén.
\par 5 Y él le dijo: «He engendrado un hijo extraño, diferente y diferente del hombre, y parecido a los hijos del Dios del cielo; y su naturaleza es diferente y no es como nosotros, y sus ojos son como los rayos del sol, y su rostro es glorioso».
\par 6 «Y me parece que no ha nacido de mí, sino de los ángeles, y temo que en sus días se haga un milagro en la tierra».
\par 7 «Y ahora, padre mío, estoy aquí para pedirte e implorarte que vayas a Enoc, nuestro padre, y aprendas de él la verdad, porque su morada está entre los ángeles».
\par 8 Y cuando Matusalén oyó las palabras de su hijo, vino a mí hasta los confines de la tierra; porque había oído lo que había allí, y gritó con fuerza, y yo oí su voz y me acerqué a él. Y le dijo: «Heme aquí, hijo mío, ¿por qué has venido a mí?».
\par 9 Y él respondió y dijo: «A causa de una gran preocupación he venido a ti, y a causa de una visión perturbadora me he acercado».
\par 10 «Y ahora, padre mío, escúchame: a mi hijo Lamec le ha nacido un hijo, como no hay ninguno, y su naturaleza no es como la naturaleza del hombre, y el color de su cuerpo es más blanco que el nieve y más roja que la flor de una rosa, y el cabello de su cabeza es más blanco que la lana blanca, y sus ojos son como los rayos del sol, y abrió los ojos y en eso iluminó toda la casa».
\par 11 «Y él se levantó en manos de la partera, abrió su boca y bendijo al Señor del cielo».
\par 12 «Y su padre Lamec tuvo miedo y huyó hacia mí, y no creía que hubiera nacido de él, sino que era semejante a los ángeles del cielo; y he aquí, he venido a ti para que me hagas saber la verdad».
\par 13 Y yo, Enoc, respondí y le dije: «El Señor hará algo nuevo en la tierra, y esto ya lo he visto en una visión, y te haré saber que en la generación de mi padre Jared algunos de los ángeles del cielo transgredieron la palabra del Señor».
\par 14 «Y he aquí que cometen pecado y transgreden la ley, se unen con mujeres y cometen pecado con ellas, se casan con algunas de ellas y engendran hijos de ellas».
\par 15 «Sí, vendrá una gran destrucción sobre toda la tierra, y habrá un diluvio y una gran destrucción durante un año».
\par 16 «Y este hijo que os ha nacido quedará en la tierra, y sus tres hijos se salvarán con él, cuando toda la humanidad que está sobre la tierra muera [él y sus hijos se salvarán]. »
\par 17 «Y producirán en la tierra gigantes no según el espíritu, sino según la carne, y habrá un gran castigo sobre la tierra, y la tierra será limpiada de toda impureza».
\par 18 Y ahora haz saber a tu hijo Lamec que el que ha nacido es en verdad su hijo, y llama su nombre Noé; porque él os será dejado, y él y sus hijos serán salvos de la destrucción que vendrá sobre la tierra a causa de todo el pecado y de toda la injusticia, que se consumará sobre la tierra en sus días».
\par 19 Y después de eso habrá aún más injusticia que la que primero se cumplió en la tierra; porque conozco los misterios de los santos; porque Él, el Señor, me lo ha mostrado y me ha informado, y lo he leído en las tablas celestiales».

\chapter{107}

\par 1 «Y vi escrito en ellos que transgredirán de generación en generación, hasta que surja una generación de justicia, y la transgresión sea destruida y el pecado desaparezca de la tierra, y venga sobre ella todo bien».
\par 2 «Y ahora, hijo mío, ve y haz saber a tu hijo Lamec que este hijo que ha nacido es en verdad su hijo, y que (esto) no es mentira».
\par 3 Y cuando Matusalén escuchó las palabras de su padre Enoc, porque le había mostrado todo en secreto, regresó y se las mostró y llamó el nombre de ese hijo Noé; porque él consolará a la tierra después de toda la destrucción.

\chapter{108}

\par 1 Otro libro que Enoc escribió para su hijo Matusalén y para aquellos que vendrán después de él y guardarán la ley en los últimos días.
\par 2 Vosotros, los que habéis hecho el bien, esperaréis esos días hasta que se acabe con los que hacen el mal; y el fin del poder de los transgresores.
\par 3 Y esperad verdaderamente hasta que el pecado haya pasado, porque sus nombres serán borrados del libro de la vida y de los libros sagrados, y su descendencia será destruida para siempre, y sus espíritus serán muertos, y ellos llorarán y harán lamentación en un lugar que es un desierto caótico, y en el fuego arderán; porque allí no hay tierra.
\par 4 Y vi allí algo como una nube invisible; porque a causa de su profundidad no podía mirar hacia arriba, y vi una llama de fuego ardiendo intensamente, y cosas como montañas brillantes dando vueltas y barriendo de un lado a otro.
\par 5 Y pregunté a uno de los santos ángeles que estaba conmigo y le dije: «¿Qué es esta cosa que brilla? Porque no es un cielo sino sólo la llama de un fuego abrasador y la voz del llanto, del clamor y de la lamentación. ¿Y dolor fuerte?
\par 6 Y me dijo: «Este lugar que ves, aquí están arrojados los espíritus de los pecadores y blasfemos, y de los que hacen el mal, y de los que pervierten todo lo que el Señor ha hablado por boca de los profetas. —(incluso) las cosas que serán».
\par 7 Porque algunas de ellas están escritas e inscritas arriba en el cielo, para que los ángeles las lean y sepan lo que les acontecerá a los pecadores, a los espíritus de los humildes y a los que han afligido sus cuerpos, y sido recompensado por Dios; y de los avergonzados por los impíos.
\par 8 «Que aman a Dios y no amaron ni el oro ni la plata ni ninguno de los bienes que hay en el mundo, sino que entregaron sus cuerpos a la tortura».
\par 9 «Quienes, desde que nacieron, no añoraban el alimento terrenal, sino que consideraban todo como un aliento pasajero, y vivían en consecuencia, y el Señor los probó mucho, y sus espíritus se encontraron puros para que bendijeran Su nombre.»
\par 10 «Y todas las bendiciones destinadas a ellos las he contado en los libros. Y les ha asignado su recompensa, porque se ha descubierto que amaban el cielo más que su vida en el mundo, y aunque fueron pisoteados por hombres malvados, y sufrieron abusos e injurias por parte de ellos y fueron avergonzados. , sin embargo, me bendijeron».
\par 11 «Y ahora convocaré a los espíritus de los buenos que pertenecen a la generación de la luz, y transformaré a los que nacieron en las tinieblas, que en la carne no fueron recompensados ​​con el honor que merecía su fidelidad».
\par 12 «Y sacaré a luz resplandeciente a los que han amado mi santo nombre, y sentaré a cada uno en su trono de honor».
\par 13 «Y resplandecerán infinitas veces; porque la justicia es el juicio de Dios; porque a los fieles les dará fidelidad en la morada de caminos rectos».
\par 14 «Y verán a los que nacieron en las tinieblas conducidos a las tinieblas, mientras que los justos resplandecerán».
\par 15 «Y los pecadores llorarán en voz alta y los verán resplandecientes, y de hecho irán adonde les han sido prescritos los días y las estaciones».

\end{document}