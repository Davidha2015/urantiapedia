\begin{document}

\title{2 Esdras}

\chapter{1}

\par 1 El segundo libro del profeta Esdras, hijo de Saraías, hijo de Azarías, hijo de Helquías, hijo de Sadamías, alma de Sadoc, hijo de Ajitob,
\par 2 El hijo de Acías, el hijo de Finees, el hijo de Heli, el hijo de Amarías, el hijo de Aziei, el hijo de Marimot, el hijo de Y habló con los de Borit, el hijo de Abisei, los hijo de Finees, hijo de Eleazar,
\par 3 Hijo de Aarón, de la tribu de Leví; que estuvo cautivo en la tierra de los medos, durante el reinado de Artejerjes, rey de los persas.
\par 4 Y vino a mí palabra del Señor, diciendo:
\par 5 Ve y muestra a mi pueblo sus obras pecaminosas, y a sus hijos las maldades que han cometido contra mí; para que puedan decirle a los hijos de sus hijos:
\par 6 Porque los pecados de sus padres se han multiplicado en ellos, porque se han olvidado de mí y han ofrecido sacrificios a dioses extraños.
\par 7 ¿No soy yo también el que los sacó de la tierra de Egipto, de la casa de servidumbre? pero me provocaron a ira y despreciaron mis consejos.
\par 8 Quítate entonces el cabello de tu cabeza y echa sobre ellos todo mal, porque no han obedecido mi ley, sino que son un pueblo rebelde.
\par 9 ¿Hasta cuándo tendré que soportar a aquellos a quienes he hecho tanto bien?
\par 10 Por causa de ellos destruí a muchos reyes; Faraón con sus siervos y todo su poder he derribado.
\par 11 Destruí a todas las naciones delante de ellos, y en el oriente dispersé a los habitantes de dos provincias, de Tiro y de Sidón, y maté a todos sus enemigos.
\par 12 Háblales, pues, y diles: Así dice el Señor:
\par 13 Yo os guié a través del mar y al principio os di un paso amplio y seguro; Os di a Moisés por líder y a Aarón por sacerdote.
\par 14 Yo os encendí en una columna de fuego, y grandes maravillas hice entre vosotros; Aún os habéis olvidado de mí, dice el Señor.
\par 15 Así dice el Señor Todopoderoso: Las codornices os fueron por señal; Os di tiendas para vuestra protección; sin embargo, murmurasteis allí,
\par 16 Y no triunfasteis en mi nombre para destruir a vuestros enemigos, sino que hasta el día de hoy todavía murmuráis.
\par 17 ¿Dónde están los beneficios que os he hecho? Cuando tuvisteis hambre y sed en el desierto, ¿no clamasteis a mí?
\par 18 Diciendo: ¿Por qué nos has traído a este desierto para matarnos? Más nos hubiera valido haber servido a los egipcios, que morir en este desierto.
\par 19 Entonces me compadecí de vuestros duelos y os di maná para comer; Así comisteis pan de los ángeles.
\par 20 Cuando tuvisteis sed, ¿no partí yo la roca y brotaron aguas hasta saciaros? para el calor te cubrí con las hojas de los árboles.
\par 21 Yo repartí entre vosotros una tierra fértil, expulsé de delante de vosotros a los cananeos, a los ferezeos y a los filisteos. ¿Qué más haré por vosotros? dice el Señor.
\par 22 Así dice el Señor Todopoderoso: Cuando estabais en el desierto, junto al río de los amorreos, teniendo sed y blasfemando mi nombre,
\par 23 No os di fuego por vuestras blasfemias, sino que eché un árbol al agua y endulcé el río.
\par 24 ¿Qué haré contigo, oh Jacob? Tú, Judá, no quisiste obedecerme; me convertiré en otras naciones, y a ellas daré mi nombre, para que guarden mis estatutos.
\par 25 Puesto que me habéis abandonado, yo también os abandonaré a vosotros; cuando deseéis que tenga misericordia de vosotros, no tendré misericordia de vosotros.
\par 26 Cuando me invoquéis, no os escucharé, porque habéis contaminado vuestras manos con sangre y vuestros pies se apresuran para cometer homicidio.
\par 27 No me habéis abandonado a mí, sino a vosotros mismos, dice el Señor.
\par 28 Así dice el Señor Todopoderoso: ¿No te he orado como un padre a sus hijos, como una madre a sus hijas y como una nodriza a sus pequeños,
\par 29 Para que vosotros seáis mi pueblo y yo sea vuestro Dios; ¿Que vosotros seáis mis hijos y yo sea vuestro padre?
\par 30 Yo os reuní como la gallina junta a sus polluelos debajo de las alas; pero ahora, ¿qué haré con vosotros? Te echaré de mi presencia.
\par 31 Cuando me ofrecáis, apartaré de vosotros mi rostro; porque he abandonado vuestras fiestas solemnes, vuestras lunas nuevas y vuestras circuncisiones.
\par 32 Yo os envié a mis siervos los profetas, a quienes vosotros habéis tomado, matado y despedazado sus cuerpos, cuya sangre demandaré de vuestras manos, dice el Señor.
\par 33 Así dice el Señor Todopoderoso: Tu casa está desolada; te arrojaré fuera como el viento a la hojarasca.
\par 34 Y tus hijos no darán fruto; porque han menospreciado mi mandamiento, y han hecho lo malo delante de mí.
\par 35 Daré vuestras casas al pueblo que vendrá; los cuales, sin haber oído aún de mí, me creerán; a quienes no les he mostrado señales, pero harán lo que les he mandado.
\par 36 No han visto profetas, pero recordarán sus pecados y los reconocerán.
\par 37 Doy por testigo de la gracia del pueblo venidero, cuyos pequeños se alegran de alegría; y aunque no me han visto con los ojos corporales, en espíritu creen lo que digo.
\par 38 Y ahora, hermano, mira qué gloria; y mira la gente que viene del oriente:
\par 39 A quienes pondré por líderes: Abraham, Isaac y Jacob, Oseas, Amós, Miqueas, Joel, Abdías y Jonás,
\par 40 Nahum, Abacuc, Sofonías, Aggeo, Zacarías y Malaquías, que también es llamado ángel del Señor.

\chapter{2}

\par 1 Así dice el Señor: Yo saqué a este pueblo de la servidumbre y les di mis mandamientos por medio de siervos los profetas; a quienes no quisieron escuchar, sino que despreciaron mis consejos.
\par 2 La madre que los dio a luz les dijo: Hijitos, id; porque soy viuda y desamparada.
\par 3 Con alegría os crié; pero con tristeza y tristeza os he perdido, porque habéis pecado delante de Jehová vuestro Dios, y habéis hecho lo malo delante de él.
\par 4 Pero ¿qué haré ahora con vosotros? Soy viuda y desamparada: id, hijos míos, y pedid misericordia al Señor.
\par 5 En cuanto a mí, oh padre, te invoco como testigo sobre la madre de estos niños, que no guardaron mi pacto,
\par 6 para que los avergüences y despojes a su madre, para que no quede descendencia de ellos.
\par 7 Que sean esparcidos entre las naciones, que sus nombres sean borrados de la tierra, porque han despreciado mi pacto.
\par 8 ¡Ay de ti, Assur, que escondes en ti a los injustos! Oh gente malvada, recuerda lo que hice con Sodoma y Gomorra;
\par 9 cuya tierra está cubierta de terrones de brea y montones de ceniza: así también haré con los que no me escuchan, dice el Señor Todopoderoso.
\par 10 Así dice el Señor a Esdras: Di a mi pueblo que les daré el reino de Jerusalén que hubiera dado a Israel.
\par 11 Tomaré también su gloria y les daré las moradas eternas que les había preparado.
\par 12 Tendrán el árbol de la vida como ungüento de olor agradable; No trabajarán ni se cansarán.
\par 13 Id y recibiréis; orad por algunos días para que sean acortados: el reino ya está preparado para vosotros: velad.
\par 14 Tomad por testigos al cielo y a la tierra; porque yo desmenuzé el mal y creé el bien; porque vivo yo, dice el Señor.
\par 15 Madre, abraza a tus hijos y críalos con alegría; afirma sus pies como una columna, porque yo te he escogido, dice el Señor.
\par 16 Y a los muertos los resucitaré de sus lugares y los sacaré de sus sepulcros, porque yo he conocido mi nombre en Israel.
\par 17 No temas, madre de los niños, porque yo te he escogido, dice el Señor.
\par 18 En tu ayuda enviaré a mis siervos Esaú y a Jeremías, tras cuyo consejo he santificado y preparado para ti doce árboles cargados de diversos frutos,
\par 19 Y otras tantas fuentes de las que fluyen leche y miel, y siete montes imponentes, en los que crecen rosas y lirios, con los que colmaré de alegría a tus hijos.
\par 20 Haz justicia a la viuda, juzga al huérfano, da al pobre, defiende al huérfano, viste al desnudo,
\par 21 Sana al quebrantado y al débil, no te burles del cojo, defiende al manco y deja que el ciego entre a la vista de mi claridad.
\par 22 Mantén a los viejos y a los jóvenes dentro de tus muros.
\par 23 Dondequiera que encuentres muertos, tómalos y entiérralos, y yo te daré el primer lugar en mi resurrección.
\par 24 Quédate tranquilo, pueblo mío, y descansa, porque aún llega tu tranquilidad.
\par 25 Nutre a tus hijos, oh buena nodriza; estabilizar sus pies.
\par 26 En cuanto a los siervos que te he dado, ninguno de ellos perecerá; porque los demandaré de en medio de tu número.
\par 27 No te canses, porque cuando llegue el día de la angustia y de la tristeza, otros llorarán y se entristecerán, pero tú estarás alegre y tendrás abundancia.
\par 28 Las naciones te envidiarán, pero nada podrán hacer contra ti, dice el Señor.
\par 29 Mis manos te cubrirán, para que tus hijos no vean el infierno.
\par 30 Alégrate, oh madre, con tus hijos; porque yo te libraré, dice el Señor.
\par 31 Acuérdate de tus hijos que duermen, porque los sacaré de los confines de la tierra y les mostraré misericordia; porque yo soy misericordioso, dice el Señor Todopoderoso.
\par 32 Abraza a tus hijos hasta que yo venga y les tenga misericordia; porque mis fuentes rebosan y mi gracia no falta.
\par 33 Yo, Esdras, recibí el encargo del Señor en el monte Oreb de ir a Israel; pero cuando llegué a ellos, me despreciaron y despreciaron el mandamiento del Señor.
\par 34 Por eso os digo, oh naciones que oís y entendéis, buscad a vuestro Pastor, él os dará descanso eterno; porque cerca está el que vendrá en el fin del mundo.
\par 35 Estad preparados para la recompensa del reino, porque la luz eterna brillará sobre vosotros para siempre.
\par 36 Huye de la sombra de este mundo, recibe el gozo de tu gloria: testifico abiertamente a mi Salvador.
\par 37 Recibe el don que se te ha concedido y alégrate dando gracias a aquel que te ha guiado al reino de los cielos.
\par 38 Levántate y ponte en pie, he aquí el número de los sellados en la fiesta del Señor;
\par 39 Los que se apartaron de la sombra del mundo y recibieron las vestiduras gloriosas del Señor.
\par 40 Toma tu número, oh Sión, y encierra a tus vestidos blancos que han cumplido la ley del Señor.
\par 41 Se ha cumplido el número de tus hijos que tanto anhelabas. Ruega al poder del Señor, para que tu pueblo, que ha sido llamado desde el principio, sea santificado.
\par 42 Yo, Esdras, vi en el monte Sión un pueblo grande, que no podía contar, y todos alababan al Señor con cánticos.
\par 43 Y en medio de ellos había un joven de gran estatura, más alto que todos los demás, y sobre cada una de sus cabezas puso coronas, y era más exaltado; lo cual me maravilló enormemente.
\par 44 Entonces pregunté al ángel y le dije: Señor, ¿qué es esto?
\par 45 Él respondió y me dijo: Estos son los que se han quitado el vestido mortal y se han vestido con el inmortal, y han confesado el nombre de Dios: ahora están coronados y reciben palmas.
\par 46 Entonces dije al ángel: ¿Quién es el joven que los corona y les pone las palmas en las manos?
\par 47 Entonces él respondió y me dijo: Este es el Hijo de Dios, a quien han confesado en el mundo. Entonces comencé a elogiar mucho a los que tan firmemente defendían el nombre del Señor.
\par 48 Entonces el ángel me dijo: Ve y cuenta a mi pueblo qué cosas y cuán grandes maravillas del Señor tu Dios has visto.

\chapter{3}

\par 1 El año treinta después de la destrucción de la ciudad, yo estaba en Babilonia, y yacía angustiado en mi cama, y ​​mis pensamientos subían a mi corazón:
\par 2 Porque vi la desolación de Sión y las riquezas de los que habitaban en Babilonia.
\par 3 Y mi espíritu se conmovió tanto que comencé a hablar palabras llenas de temor al Altísimo, y dije:
\par 4 Oh Señor, que gobiernas, tú dijiste al principio, cuando plantaste la tierra, y sólo tú mismo, y ordenaste al pueblo,
\par 5 Y diste a Adán un cuerpo sin alma, obra de tus manos, y le diste aliento de vida, y volvió a vivir delante de ti.
\par 6 Y lo llevas al paraíso que tu diestra había plantado antes de que la tierra surgiera.
\par 7 Y le diste mandamiento de amar tu camino, el cual transgredió, e inmediatamente designaste la muerte en él y en sus generaciones, de las cuales surgieron naciones, tribus, pueblos y familias, sin número.
\par 8 Y cada pueblo anduvo según su voluntad, hizo maravillas delante de ti y despreció tus mandamientos.
\par 9 Y de nuevo, con el tiempo, trajiste el diluvio sobre los que habitaban en el mundo y los destruiste.
\par 10 Y aconteció en cada uno de ellos que, como para Adán fue la muerte, así fue para éstos el diluvio.
\par 11 Sin embargo, de ellos dejaste a uno, a saber, a Noé con su casa, de donde procedieron todos los hombres justos.
\par 12 Y sucedió que cuando los habitantes de la tierra comenzaron a multiplicarse y tuvieron muchos hijos y eran un pueblo grande, comenzaron de nuevo a ser más impíos que los primeros.
\par 13 Cuando ellos hacían tanta maldad delante de ti, escogiste entre ellos a un hombre que se llamaba Abraham.
\par 14 A él amaste y sólo a él le mostraste tu voluntad.
\par 15 E hiciste con él un pacto eterno, prometiéndole que nunca abandonarías a su descendencia.
\par 16 Y a él le diste a Isaac, y a Isaac también le diste a Jacob y a Esaú. En cuanto a Jacob, tú lo escogiste para ti y lo pusiste junto a Esaú; y así Jacob se convirtió en una gran multitud.
\par 17 Y aconteció que cuando sacaste a su descendencia de Egipto, los llevaste al monte Sinaí.
\par 18 E inclinando los cielos, afirmaste la tierra, conmoviste el mundo entero, hiciste temblar los abismos y perturbaste a los hombres de esa época.
\par 19 Y tu gloria atravesó las cuatro puertas: la del fuego, la del terremoto, la del viento y la del frío; para que dieras la ley a la descendencia de Jacob, y la diligencia a la generación de Israel.
\par 20 Y, sin embargo, no les quitaste el corazón malvado, para que tu ley pudiera dar fruto en ellos.
\par 21 Porque el primer Adán, que tenía un corazón malvado, transgredió y fue vencido; y así sean todos los que de él nacen.
\par 22 Así la enfermedad se hizo permanente; y la ley (también) en el corazón del pueblo con la malignidad de la raíz; de modo que los buenos se fueron y los malos permanecieron.
\par 23 Así pasaron los tiempos y se acabaron los años. Entonces te levantaste un siervo llamado David:
\par 24 A quien le ordenaste que edificara una ciudad a tu nombre y que te ofreciera en ella incienso y ofrendas.
\par 25 Cuando esto sucedió muchos años, entonces los habitantes de la ciudad te abandonaron,
\par 26 E hicieron en todo lo mismo que Adán y todas sus generaciones, pues también ellos tenían un corazón malvado.
\par 27 Y así entregaste tu ciudad en manos de tus enemigos.
\par 28 ¿Acaso son mejores las obras de los que habitan en Babilonia, para dominar a Sión?
\par 29 Porque cuando llegué allí y vi innumerables impiedades, mi alma vio muchos malhechores en este año treinta, de modo que mi corazón desfalleció.
\par 30 Porque he visto cómo los toleraste pecar, y perdonaste a los malhechores, y destruiste a tu pueblo, y preservaste a tus enemigos, y no lo dijiste.
\par 31 No recuerdo cómo se puede dejar este camino: ¿Son, pues, mejores los de Babilonia que los de Sión?
\par 32 ¿O hay algún otro pueblo que te conozca además de Israel? ¿O qué generación ha creído tanto en tus pactos como Jacob?
\par 33 Y, sin embargo, su recompensa no aparece, y su trabajo no da fruto: porque he ido de aquí para allá entre las naciones, y veo que fluyen en riquezas, y no piensan en tus mandamientos.
\par 34 Pesa, pues, ahora en la balanza nuestra maldad, y también la de los que habitan el mundo; y así tu nombre no será hallado en ninguna parte sino en Israel.
\par 35 ¿O cuándo fue que los habitantes de la tierra no pecaron ante ti? ¿O qué pueblo ha guardado así tus mandamientos?
\par 36 Descubrirás que el nombre de Israel ha guardado tus preceptos; pero no los paganos.

\chapter{4}

\par 1 Y el ángel que me fue enviado, cuyo nombre era Uriel, me respondió:
\par 2 Y dijo: Tu corazón ha ido demasiado lejos en este mundo, ¿y crees que podrás comprender el camino del Altísimo?
\par 3 Entonces dije: Sí, mi señor. Y él me respondió y dijo: Soy enviado para mostrarte tres caminos y exponerte tres similitudes:
\par 4 De lo cual, si puedes declararme uno, te mostraré también el camino que deseas ver, y te mostraré de dónde viene el corazón malvado.
\par 5 Y yo dije: Habla, señor mío. Entonces me dijo: Ve, pésame el peso del fuego, o mídeme el soplo del viento, o llámame de nuevo el día que pasó.
\par 6 Entonces respondí y dije: ¿Quién puede hacer eso para que me pidas tales cosas?
\par 7 Y él me dijo: Si te preguntara cuántas moradas hay en medio del mar, o cuántas fuentes hay en el principio del abismo, o cuántas fuentes hay sobre el firmamento, o cuáles son las salidas del paraíso:
\par 8 Quizás me dirías: Nunca bajé al abismo, ni todavía al infierno, ni subí al cielo.
\par 9 Sin embargo, ahora sólo te he preguntado sobre el fuego y el viento, y sobre el día por el que has pasado, y sobre cosas de las que no puedes separarte, y sin embargo no puedes darme respuesta sobre ellas.
\par 10 Y me dijo además: Lo tuyo y lo que crece contigo, ¿no lo puedes saber?
\par 11 ¿Cómo podrá entonces tu vasija comprender el camino del Altísimo, y, estando ahora el mundo corrompido exteriormente, comprender la corrupción que se manifiesta ante mis ojos?
\par 12 Entonces le dije: Más vale no existir, que vivir todavía en la maldad y sufrir y no saber por qué.
\par 13 Él me respondió y dijo: Fui a un bosque a una llanura y los árboles consultaron.
\par 14 Y dijo: Venid, vayamos y peleemos contra el mar, para que se aleje de nosotros y hagamos más bosques para nosotros.
\par 15 También las corrientes del mar tomaron consejo de la misma manera, y dijeron: Venid, subamos y sometamos los bosques de la llanura, para que también allí hagamos otra tierra.
\par 16 Fue en vano pensar en la leña, porque vino el fuego y la consumió.
\par 17 La idea de las inundaciones del mar quedó en nada, porque la arena se levantó y las detuvo.
\par 18 Si ahora fueras juez entre estos dos, ¿a quién comenzarías a justificar? ¿O a quién condenarías?
\par 19 Respondí y dije: En verdad, es una idea tonta la que ambos han ideado, porque la tierra está dada al bosque, y también el mar tiene su lugar para soportar sus inundaciones.
\par 20 Entonces él me respondió y dijo: Tú has juzgado correctamente, pero ¿por qué no te juzgas también a ti mismo?
\par 21 Porque como la tierra es entregada al bosque y el mar a sus inundaciones, así los que habitan en la tierra no pueden entender nada más que lo que hay sobre la tierra; y el que habita sobre los cielos sólo puede entender las cosas que están por encima de la altura de los cielos.
\par 22 Entonces respondí y dije: Te ruego, oh Señor, que me des entendimiento.
\par 23 Porque no era mi intención sentir curiosidad por las cosas elevadas, sino por las que pasan cada día junto a nosotros, es decir, por qué Israel es entregado como oprobio a las naciones, y por qué motivo el pueblo que tú amas. es entregado a naciones impías, y por qué la ley de nuestros antepasados ​​es anulada, y los pactos escritos quedan sin efecto,
\par 24 Y pasamos del mundo como saltamontes, y nuestra vida es asombro y temor, y no somos dignos de alcanzar misericordia.
\par 25 ¿Qué hará entonces con el nombre con que somos llamados? De estas cosas he preguntado.
\par 26 Entonces él me respondió y dijo: Cuanto más busques, más te maravillarás; porque el mundo se apresura a pasar,
\par 27 Y no pueden comprender las cosas que se prometen a los justos en el futuro, porque este mundo está lleno de injusticia y debilidades.
\par 28 Pero lo que me preguntas, te lo diré; porque el mal está sembrado, pero su destrucción aún no ha llegado.
\par 29 Por lo tanto, si lo sembrado no se invierte y el lugar donde se sembró el mal no pasa, entonces lo que se sembró con bien no puede venir.
\par 30 Porque el grano de mala semilla fue sembrado en el corazón de Adán desde el principio, ¿y cuánta impiedad ha producido hasta ahora? ¿Y cuánto producirá todavía hasta que llegue el tiempo de la trilla?
\par 31 Considera ahora por ti mismo cuán grandes frutos de maldad ha producido el grano de la mala semilla.
\par 32 Y cuando sean cortadas las espigas que son innumerables, ¿cuánto espacio llenarán?
\par 33 Entonces respondí y dije: ¿Cómo y cuándo sucederán estas cosas? ¿Por qué nuestros años son pocos y malos?
\par 34 Y él me respondió diciendo: No te apresures por encima del Altísimo; porque en vano es tu prisa por estar por encima de él, porque te has excedido mucho.
\par 35 ¿No preguntaban también las almas de los justos sobre estas cosas en sus aposentos, diciendo: ¿Hasta cuándo estaré esperando así? ¿Cuándo vendrá el fruto de la planta de nuestra recompensa?
\par 36 Y a estas cosas les respondió el arcángel Uriel, y dijo: Incluso cuando en vosotros se llene el número de semillas, porque él ha pesado el mundo en la balanza.
\par 37 Con medida midió los tiempos; y con números ha contado los tiempos; y no los mueve ni los remueve, hasta que se cumpla dicha medida.
\par 38 Entonces respondí y dije: ¡Oh Señor, que eres el que manda! Incluso nosotros todos estamos llenos de impiedad.
\par 39 Y quizá por nosotros los pisos de los justos no se llenen, a causa de los pecados de los habitantes de la tierra.
\par 40 Entonces él me respondió y dijo: Ve a una mujer que está encinta y pregúntale cuando haya cumplido sus nueve meses, si su vientre puede retener por más tiempo el nacimiento dentro de ella.
\par 41 Entonces dije: No, Señor, eso no puede. Y me dijo: En el sepulcro las cámaras de las almas son como el vientre de una mujer:
\par 42 Porque como la mujer que está de parto se apresura a escapar de la necesidad del parto, así también estos lugares se apresuran a entregar lo que les ha sido encomendado.
\par 43 Mira desde el principio, lo que deseas ver, te será mostrado.
\par 44 Entonces respondí y dije: Si he hallado gracia ante tus ojos, y si es posible, y si soy digno de ello,
\par 45 Muéstrame, pues, si hay más por venir que lo pasado, o más pasado de lo que está por venir.
\par 46 Lo que pasó lo sé, pero lo que está por venir no lo sé.
\par 47 Y me dijo: Ponte a la derecha y te explicaré la semejanza.
\par 48 Entonces me puse en pie y miré, y he aquí un horno ardiendo pasó delante de mí; y sucedió que cuando la llama se apagó, miré y he aquí, el humo permaneció quieto.
\par 49 Después de esto pasó delante de mí una nube de agua que hizo caer mucha lluvia con una tormenta; y cuando pasó la tormenta, las gotas quedaron quietas.
\par 50 Entonces me dijo: Considera dentro de ti mismo; como es más la lluvia que las gotas, y como es más el fuego que el humo; pero las gotas y el humo quedan atrás: así la cantidad que pasó fue más sobrante.
\par 51 Entonces oré y dije: ¿Crees que podré vivir hasta entonces? ¿O qué sucederá en aquellos días?
\par 52 Él me respondió y dijo: En cuanto a las señales que me pides, puedo decírtelas en parte; pero en cuanto a tu vida, no soy enviado a mostrártelas; porque no lo sé.

\chapter{5}

\par 1 Sin embargo, cuando lleguen las señales, he aquí, vendrán días en que los habitantes de la tierra serán apresados ​​en gran número, y el camino de la verdad se ocultará y la tierra quedará estéril de la fe.
\par 2 Pero la iniquidad aumentará más que lo que ahora ves o lo que has oído hace mucho tiempo.
\par 3 Y la tierra que ahora ves echada raíces, de repente la verás asolada.
\par 4 Pero si el Altísimo te concede vivir, verás después de la tercera trompeta que el sol volverá a brillar de repente durante la noche y la luna tres veces durante el día.
\par 5 Y de la madera brotará sangre, y la piedra dará su voz, y el pueblo se turbará.
\par 6 Y gobernará aquel a quien no esperan los que habitan en la tierra, y las aves volarán juntas.
\par 7 Y el mar de Sodoma arrojará peces y hará por la noche un ruido que muchos no han conocido, pero todos oirán su voz.
\par 8 También habrá confusión en muchos lugares, y el fuego se encenderá de nuevo, y las fieras cambiarán de lugar, y las mujeres menstruantes darán a luz monstruos.
\par 9 Y las aguas saladas se encontrarán en las dulces, y todos los amigos se destruirán unos a otros; entonces el ingenio se esconderá y el entendimiento se retirará a su cámara secreta,
\par 10 Y muchos serán buscados y no serán encontrados; entonces la injusticia y la incontinencia se multiplicarán sobre la tierra.
\par 11 También un país preguntará a otro, y dirá: ¿Ha pasado por ti la justicia que hace justo al hombre? Y dirá: No.
\par 12 Al mismo tiempo los hombres esperarán, pero nada obtendrán; trabajarán, pero sus caminos no prosperarán.
\par 13 Tengo permiso para mostrarte tales señales; y si oras de nuevo, y lloras como ahora, y ayunas incluso días, oirás cosas aún mayores.
\par 14 Entonces me desperté, y un gran temor recorrió todo mi cuerpo, y mi mente se turbó tanto que desmayó.
\par 15 Entonces el ángel que había venido a hablar conmigo me abrazó, me consoló y me puso de pie.
\par 16 Y aconteció que la segunda noche vino a mí Salatiel, capitán del pueblo, y me dijo: ¿Dónde has estado? ¿Y por qué está tan pesado tu rostro?
\par 17 ¿No sabes que Israel te está encomendado en la tierra de su cautiverio?
\par 18 Levántate, pues, come pan y no nos abandones, como el pastor que deja su rebaño en manos de lobos crueles.
\par 19 Entonces le dije: Aléjate de mí y no te acerques a mí. Y él escuchó lo que yo dije y se alejó de mí.
\par 20 Y así ayuné siete días, lamentándome y llorando, tal como me ordenó el ángel Uriel.
\par 21 Y después de siete días aconteció que los pensamientos de mi corazón volvieron a ser muy pesados ​​para mí,
\par 22 Y mi alma recobró el espíritu de entendimiento, y comencé a hablar de nuevo con el Altísimo,
\par 23 Y dijo: Oh Señor, que gobiernas, de todos los bosques de la tierra y de todos sus árboles, tú has escogido una sola vid:
\par 24 Y de todas las tierras del mundo entero te escogiste un hoyo, y de todas sus flores un lirio:
\par 25 Y de todas las profundidades del mar te llenaste un solo río, y de todas las ciudades edificadas, santificaste a Sión para ti.
\par 26 Y de todas las aves creadas te pusiste por nombre una paloma, y ​​de todo el ganado creado te diste una oveja.
\par 27 Y entre todas las multitudes de pueblos, formaste un solo pueblo; y a este pueblo que amas le diste una ley aprobada por todos.
\par 28 Ahora bien, Señor, ¿por qué has entregado este pueblo a muchos? y sobre una sola raíz has preparado otros, y ¿por qué has esparcido entre muchos a tu único pueblo?
\par 29 Y los que contradijeron tus promesas y no creyeron en tus pactos, los pisotearon.
\par 30 Si tanto odias a tu pueblo, deberías castigarlo con tus propias manos.
\par 31 Cuando hube dicho estas palabras, me fue enviado el ángel que había venido a mí la noche anterior,
\par 32 Y me dijo: Escúchame y te enseñaré; escucha lo que te digo, y te diré más.
\par 33 Y yo dije: Habla, Señor mío. Entonces me dijo: Estás muy turbado por causa de Israel. ¿Amas a ese pueblo más que al que lo hizo?
\par 34 Y dije: No, Señor; pero de gran dolor he hablado: porque mis riñones me duelen a cada hora, mientras me esfuerzo por comprender el camino del Altísimo y por buscar parte de su juicio.
\par 35 Y él me dijo: No puedes. Y dije: ¿Por qué, Señor? ¿Dónde nací entonces? ¿O por qué el vientre de mi madre no fue entonces mi sepultura, para no haber visto los dolores de parto de Jacob, y el trabajo agotador del linaje de Israel?
\par 36 Y él me dijo: Cuéntame las cosas que aún no han llegado, recogeme la escoria que está esparcida, hazme reverdecer las flores que están secas,
\par 37 Ábreme los lugares cerrados, y sácame los vientos que en ellos están encerrados, muéstrame la imagen de una voz, y entonces te declararé lo que te esfuerzas por saber.
\par 38 Y dije: Oh Señor, que gobiernas, ¿quién puede saber estas cosas sino aquel que no tiene su morada con los hombres?
\par 39 En cuanto a mí, soy imprudente: ¿cómo podría, pues, hablar de estas cosas que me preguntas?
\par 40 Entonces me dijo: Así como tú no puedes hacer ninguna de estas cosas de las que te he hablado, así tampoco podrás conocer mi juicio ni, en definitiva, el amor que he prometido a mi pueblo.
\par 41 Y dije: He aquí, Señor, todavía estás cerca de los que están reservados hasta el fin. ¿Y qué harán los que fueron antes de mí, o los que estamos ahora, o los que vendrán después de nosotros?
\par 42 Y me dijo: Compararé mi juicio con un anillo: así como no hay lentitud en los últimos, así tampoco hay rapidez en los primeros.
\par 43 Entonces respondí y dije: ¿No podrías hacer al mismo tiempo lo que ha sido creado y lo que es ahora, y lo que está por venir? ¿Para que puedas mostrar tu juicio más pronto?
\par 44 Entonces él me respondió y dijo: La criatura no puede apresurarse más que el creador; ni el mundo podrá retener de una vez los que en él serán creados.
\par 45 Y dije: Como dijiste a tu siervo que tú, que das vida a todos, al mismo tiempo diste vida a la criatura que creaste, y la criatura la dio a luz, así ahora también podría dar a luz. los que ahora están presentes a la vez.
\par 46 Y él me dijo: Pregunta en el vientre de una mujer, y dile: Si tienes hijos, ¿por qué no los tienes juntos, sino uno tras otro? rogadle, pues, que dé a luz diez hijos a la vez.
\par 47 Y dije: Ella no puede, pero debe hacerlo por tiempo.
\par 48 Entonces me dijo: Así también he dado el vientre de la tierra a los que en ella serán sembrados en su tiempo.
\par 49 Porque como un niño no puede dar a luz lo que es propio de un anciano, así he dispuesto yo del mundo que he creado.
\par 50 Y pregunté y dije: Ya que me has dado el camino, procederé a hablar delante de ti; porque nuestra madre, de quien me has dicho que es joven, ya está cerca de la edad.
\par 51 Él me respondió y dijo: Pregunta a una mujer que esté pariendo hijos, y ella te lo dirá.
\par 52 Dile: ¿Por qué los que ahora has engendrado son como los de antes, pero de menor estatura?
\par 53 Y ella te responderá: Los que nacen en la fuerza de la juventud son de una manera, y los que nacen en la vejez, cuando el útero falla, son de otra manera.
\par 54 Considerad, pues, también vosotros que sois de menor estatura que los que fueron antes de vosotros.
\par 55 Y los que vienen detrás de vosotros son menos que vosotros, como las criaturas que ahora comienzan a envejecer y han superado la fuerza de la juventud.
\par 56 Entonces dije: Señor, te ruego que, si he hallado gracia ante tus ojos, muestres a tu siervo por quién visitas a tu criatura.

\chapter{6}

\par 1 Y me dijo: En el principio, cuando la tierra fue hecha, antes de que existieran los confines del mundo, antes de que soplaran los vientos,
\par 2 Antes de que tronara y relámpago, o antes de que se pusieran los cimientos del paraíso,
\par 3 Antes de que aparecieran las hermosas flores, antes de que se establecieran las fuerzas móviles, antes de que se reuniera la innumerable multitud de ángeles,
\par 4 O se elevaron las alturas del aire antes de que se nombraran las medidas del firmamento, o se calentaron las chimeneas en Sión,
\par 5 Y antes de que se buscaran los años presentes, y de que se convirtieran las invenciones de los que ahora pecan, antes de que fueran sellados los que han acumulado la fe como tesoro,
\par 6 Entonces pensé en estas cosas, y todas fueron hechas sólo por mí y por ningún otro; por mí también terminarán, y por ningún otro.
\par 7 Entonces respondí y dije: ¿Cuál será la división de los tiempos? ¿O cuándo será el fin del primero y el principio del siguiente?
\par 8 Y me dijo: Desde Abraham hasta Isaac, cuando Jacob y Esaú nacieron de él, la mano de Jacob tomó primero el calcañar de Esaú.
\par 9 Porque Esaú es el fin del mundo, y Jacob es el principio del que sigue.
\par 10 La mano del hombre está entre el talón y la mano; otra pregunta, Esdras, no la hagas tú.
\par 11 Entonces respondí y dije: Oh Señor, que eres el gobernante, si he hallado gracia ante tus ojos,
\par 12 Te ruego que muestres a tu siervo el resto de tus prendas, de las cuales me mostraste parte la última noche.
\par 13 Entonces él respondió y me dijo: Levántate sobre tus pies y oye una voz poderosa.
\par 14 Y será como un gran movimiento; pero el lugar donde estás no será conmovido.
\par 15 Por tanto, cuando ella hable, no temáis, porque la palabra es del fin y el fundamento de la tierra es comprensible.
\par 16 ¿Y por qué? porque la palabra de estas cosas tiembla y se conmueve, sabiendo que el fin de estas cosas es necesario cambiar.
\par 17 Y sucedió que cuando lo oí, me levanté sobre mis pies y escuché, y he aquí, hubo una voz que hablaba, y su sonido era como el estruendo de muchas aguas.
\par 18 Y dijo: He aquí vienen días en que comenzaré a acercarme y a visitar a los moradores de la tierra,
\par 19 Y comenzará a preguntarles quiénes son los que han dañado injustamente con su injusticia, y cuándo se cumplirá la aflicción de Sión;
\par 20 Y cuando el mundo que comienza a desaparecer se acabe, entonces mostraré estas señales: los libros se abrirán delante del firmamento y lo verán todo junto:
\par 21 Y los niños de un año hablarán con sus voces, las mujeres encintas darán a luz niños prematuros de tres o cuatro meses, y vivirán y resucitarán.
\par 22 Y de repente los lugares sembrados aparecerán sin sembrar, los graneros llenos de repente se encontrarán vacíos.
\par 23 Y la trompeta dará un sonido que, cuando todos lo oigan, de repente tendrán miedo.
\par 24 En aquel tiempo, los amigos lucharán unos contra otros como enemigos, y la tierra estará aterrorizada con sus habitantes, las fuentes de las fuentes se detendrán y en tres horas no correrán.
\par 25 Quien quede de todo lo que te he dicho, escapará y verá mi salvación y el fin de vuestro mundo.
\par 26 Y lo verán los hombres que sean recibidos, los que no han gustado la muerte desde su nacimiento; y el corazón de los habitantes se transformará y se transformará en otro significado.
\par 27 Porque el mal será extinguido y el engaño extinguido.
\par 28 En cuanto a la fe, florecerá, la corrupción será vencida y la verdad, que durante tanto tiempo estuvo sin fruto, será declarada.
\par 29 Y mientras él hablaba conmigo, he aquí, poco a poco iba mirando a aquel ante quien estaba.
\par 30 Y me dijo estas palabras; He venido para mostrarte la hora de la noche venidera.
\par 31 Si oras aún más y ayunas siete días más, te contaré cada día cosas más grandes de las que he oído.
\par 32 Porque tu voz se oye delante del Altísimo; porque el Poderoso ha visto tu justicia, ha visto también tu castidad, que has tenido desde tu juventud.
\par 33 Por eso me ha enviado a mostrarte todas estas cosas y a decirte: Confórtate y no temas.
\par 34 Y no te apresures con los tiempos pasados, pensando cosas vanas, para no apresurarte desde los últimos tiempos.
\par 35 Y aconteció después de esto que lloré otra vez y ayuné de la misma manera siete días, para poder cumplir las tres semanas que él me había dicho.
\par 36 Y en la octava noche mi corazón volvió a angustiarse dentro de mí, y comencé a hablar ante el Altísimo.
\par 37 Porque mi espíritu se encendió en gran manera y mi alma estaba angustiada.
\par 38 Y dije: Oh Señor, tú hablaste desde el principio de la creación, es decir, el primer día, y dijiste así; Que se hagan el cielo y la tierra; y tu palabra fue obra perfecta.
\par 39 Y entonces apareció el espíritu, y por todas partes hubo oscuridad y silencio; El sonido de la voz del hombre aún no se había formado.
\par 40 Entonces ordenaste que de tus tesoros saliera una hermosa luz, para que tu obra pudiera aparecer.
\par 41 El segundo día creaste el espíritu del firmamento y le ordenaste que se separara y dividiera las aguas, de modo que una parte subiera y la otra quedara abajo.
\par 42 Al tercer día ordenaste que se recogieran las aguas en la séptima parte de la tierra; seis palmas las secaste y las guardaste, para que de ellas, unas de las plantadas por Dios y cultivadas, sirvieran. El e.
\par 43 Porque tan pronto como salió tu palabra, la obra fue hecha.
\par 44 Porque en seguida hubo grandes e innumerables frutos, y muchos y diversos placeres para el paladar, y flores de inmutable color, y olores de maravilloso olfato: y esto sucedió al tercer día.
\par 45 Al cuarto día ordenaste que el sol brillara, que la luna alumbrara y que las estrellas estuvieran en orden.
\par 46 Y les diste el encargo de prestar el servicio al hombre que debía realizarse.
\par 47 El día quinto le dijiste a la séptima parte, donde se reunían las aguas, que produjera seres vivientes, aves y peces: y así sucedió.
\par 48 Porque el agua muda y sin vida, por mandato de Dios, produjo seres vivientes, para que todos los pueblos pudieran alabar tus maravillas.
\par 49 Entonces ordenaste dos seres vivientes, a uno lo llamaste Enoc y al otro Leviatán;
\par 50 Y separaste el uno del otro, porque la séptima parte, es decir, donde se juntaba el agua, no podía contener a ambos.
\par 51 A Enoc le diste una parte que se secó al tercer día, para que habitara en la misma parte donde hay mil colinas.
\par 52 Pero a Leviatán le diste la séptima parte, es decir, lo húmedo; y lo has guardado para que sea devorado por quien tú quieras y cuando.
\par 53 El sexto día ordenaste a la tierra que produjera delante de ti bestias, ganado y reptiles.
\par 54 Y después de estos, también Adán, a quien hiciste señor de todas tus criaturas: de él procedemos todos nosotros, y también el pueblo que tú escogiste.
\par 55 Todo esto he hablado delante de ti, oh Señor, porque tú hiciste el mundo por nosotros.
\par 56 En cuanto a los demás pueblos, que también proceden de Adán, dijiste que no son nada, sino que son como saliva, y comparaste su abundancia con una gota que cae de un vaso.
\par 57 Y ahora, Señor, he aquí que estos paganos, que siempre habían sido tenidos por nada, han comenzado a dominarnos y a devorarnos.
\par 58 Pero nosotros, tu pueblo, a quien llamaste tu primogénito, tu unigénito y tu ferviente amante, somos entregados en sus manos.
\par 59 Si el mundo fue hecho ahora para nosotros, ¿por qué no poseemos una herencia con el mundo? ¿Cuánto tiempo durará esto?

\chapter{7}

\par 1 Y cuando terminé de decir estas palabras, me fue enviado el ángel que me había sido enviado las noches anteriores:
\par 2 Y me dijo: Levántate, Esdras, y escucha las palabras que he venido a decirte.
\par 3 Y yo dije: Habla, Dios mío. Entonces me dijo: El mar está puesto en un lugar ancho, para que sea profundo y grande.
\par 4 Pero la entrada era estrecha, como un río;
\par 5 ¿Quién, pues, podría entrar en el mar para contemplarlo y gobernarlo? Si no pasó por lo angosto, ¿cómo podría llegar a lo ancho?
\par 6 También hay otra cosa; Una ciudad está edificada y asentada sobre un campo amplio, y está llena de todo bien:
\par 7 Su entrada es estrecha y está situada en un lugar peligroso para caer, como si hubiera fuego a la derecha y agua profunda a la izquierda.
\par 8 Y entre ambos había un solo camino, incluso entre el fuego y el agua, tan pequeño que sólo un hombre podía recorrerlo a la vez.
\par 9 Si ahora esta ciudad fuera dada a un hombre en herencia, si nunca pasa el peligro que se le presenta, ¿cómo recibirá esta herencia?
\par 10 Y dije: Así es, Señor. Entonces me dijo: Así también es la porción de Israel.
\par 11 Porque por amor a ellos hice el mundo; y cuando Adán transgredió mis estatutos, entonces se decretó lo que ahora se cumple.
\par 12 Entonces las entradas de este mundo se hicieron estrechas, llenas de dolor y de dolores: son pocas y malas, llenas de peligros y muy dolorosas.
\par 13 Porque las entradas al mundo antiguo eran amplias y seguras, y traían frutos inmortales.
\par 14 Si, pues, los que viven se esfuerzan por no entrar en estas cosas estrechas y vanas, nunca podrán recibir lo que les está reservado.
\par 15 Ahora pues, ¿por qué te inquietas tú mismo, siendo que eres un hombre corruptible? ¿Y por qué te conmueves, si no eres más que un mortal?
\par 16 ¿Por qué no has pensado en lo que está por venir, en lugar de lo que está presente?
\par 17 Entonces respondí y dije: Señor, gobernante, tú has ordenado en tu ley que los justos hereden estas cosas, pero los impíos perezcan.
\par 18 Sin embargo, los justos sufrirán dificultades y esperarán lo ancho; porque los que hicieron lo malo sufrieron las dificultades y no verán lo ancho.
\par 19 Y él me dijo. No hay juez superior a Dios, ni nadie que tenga entendimiento superior al Altísimo.
\par 20 Porque hay muchos que perecen en esta vida porque desprecian la ley de Dios que se les presenta.
\par 21 Porque Dios ha dado a los que vienen mandamientos estrictos sobre lo que deben hacer para vivir tal como vinieron y lo que deben observar para evitar el castigo.
\par 22 Pero ellos no le obedecieron; pero habló contra él e imaginó cosas vanas;
\par 23 Y se engañaron a sí mismos con sus malas acciones; y dijo del Altísimo, que no es; y no conoció sus caminos:
\par 24 Pero despreciaron su ley y negaron sus pactos; En sus estatutos no han sido fieles, ni han cumplido sus obras.
\par 25 Por eso, Esdras, lo vacío es lo vacío y lo lleno es lo lleno.
\par 26 He aquí, llegará el momento en que se cumplirán estas señales que te he dicho, y aparecerá la novia, y se verá saliendo la que ahora ha sido retirada de la tierra.
\par 27 Y quien se libre de los males mencionados verá mis maravillas.
\par 28 Porque mi hijo Jesús se manifestará con los que estén con él, y los que queden se alegrarán dentro de cuatrocientos años.
\par 29 Después de estos años morirá mi hijo Cristo, y todos los hombres que tengan vida.
\par 30 Y por siete días el mundo volverá al antiguo silencio, como en los juicios anteriores, de modo que nadie quedará.
\par 31 Y después de siete días, el mundo que aún no ha despierto se levantará y morirá el corrupto.
\par 32 Y la tierra restaurará a los que duermen en ella, y también el polvo a los que habitan en silencio, y los lugares secretos librarán a las almas que les fueron encomendadas.
\par 33 Y el Altísimo aparecerá en el tribunal, y la miseria pasará, y el largo sufrimiento tendrá fin.
\par 34 Pero sólo el juicio permanecerá, la verdad permanecerá y la fe se fortalecerá:
\par 35 Y seguirá el trabajo, y se mostrará la recompensa, y las buenas obras tendrán fuerza, y las malas acciones no tendrán regla.
\par 36 Entonces dije: Abraham oró primero por los sodomitas, y Moisés por los padres que pecaron en el desierto.
\par 37 Y Jesús después de él por Israel en tiempos de Acán:
\par 38 Y Samuel y David por la destrucción, y Salomón por los que debían venir al santuario.
\par 39 Y Helias para los que recibieron la lluvia; y por los muertos, para que viva:
\par 40 Y Ezequías por el pueblo en tiempos de Senaquerib, y muchos por muchos.
\par 41 Así también ahora, cuando la corrupción ha crecido y la maldad ha aumentado, y los justos han orado por los impíos, ¿por qué no será así también ahora?
\par 42 Él me respondió y dijo: Esta vida presente no es el fin en el que permanece mucha gloria; por eso han orado por los débiles.
\par 43 Pero el día del juicio será el fin de este tiempo y el comienzo de la inmortalidad venidera, en la que la corrupción ya pasó,
\par 44 La intemperancia ha llegado a su fin, la infidelidad ha sido eliminada, la justicia ha crecido y la verdad ha brotado.
\par 45 Entonces nadie podrá salvar al destruido ni oprimir al vencedor.
\par 46 Respondí entonces y dije: Esta es mi primera y última palabra: que hubiera sido mejor no haberle dado la tierra a Adán, o, cuando se la hubiera dado, haberle impedido pecar.
\par 47 ¿De qué le sirve a los hombres, ahora en este tiempo, vivir con tristeza y después de la muerte esperar el castigo?
\par 48 Oh tú, Adán, ¿qué has hecho? porque aunque fuiste tú quien pecó, no solo tú has caído, sino todos los que de ti venimos.
\par 49 ¿De qué nos sirve si se nos promete un tiempo inmortal, si hemos hecho obras que traen la muerte?
\par 50 ¿Y que se nos ha prometido una esperanza eterna, mientras que nosotros, siendo los más malvados, somos vanidosos?
\par 51 ¿Y que tenemos guardadas viviendas seguras y saludables, mientras que hemos vivido en la maldad?
\par 52 ¿Y que la gloria del Altísimo se guarda para defender a aquellos que han llevado una vida cautelosa, mientras que nosotros hemos andado por los caminos más malvados de todos?
\par 53 ¿Y que se nos muestre un paraíso cuyo fruto perdure para siempre, en el que haya seguridad y medicina, ya que no entraremos en él?
\par 54 (Porque hemos caminado por lugares desagradables.)
\par 55 ¿Y que los rostros de los que han practicado la abstinencia brillarán sobre las estrellas, mientras que nuestros rostros serán más negros que las tinieblas?
\par 56 Porque mientras vivíamos y cometíamos iniquidad, no pensábamos que íbamos a sufrir por ella después de la muerte.
\par 57 Entonces él me respondió y dijo: Ésta es la condición de la batalla que peleará el hombre nacido en la tierra;
\par 58 Que, si es vencido, sufrirá como tú has dicho; pero si obtiene la victoria, recibirá lo que yo digo.
\par 59 Porque esta es la vida de la que Moisés habló al pueblo mientras vivía, diciendo: Escoge la vida para que vivas.
\par 60 Sin embargo, no le creyeron a él, ni tampoco los profetas posteriores a él, ni yo, que les hablamos,
\par 61 Para que no haya tanta tristeza en su destrucción como alegría sobre los que están persuadidos a la salvación.
\par 62 Entonces respondí y dije: Sé, Señor, que el Altísimo se llama misericordioso, porque tiene misericordia de los que aún no han venido al mundo.
\par 63 Y también sobre los que se vuelven a su ley;
\par 64 Y que es paciente y soporta largamente a los que han pecado como a sus criaturas;
\par 65 Y que es generoso, porque está dispuesto a dar donde sea necesario;
\par 66 Y que es muy misericordioso, pues multiplica cada vez más sus misericordias hacia los presentes y pasados, y también hacia los venideros.
\par 67 Porque si él no multiplica sus misericordias, el mundo no permanecerá con sus herederos.
\par 68 Y él perdona; porque si no hiciera así por su bondad, para que los que han cometido iniquidades fueran librados de ellas, la diezmilésima parte de los hombres no quedaría con vida.
\par 69 Y siendo juez, si no perdona a los que son curados con su palabra y elimina multitud de contiendas,
\par 70 Quizás queden muy pocos en una multitud innumerable.

\chapter{8}

\par 1 Y él me respondió diciendo: El Altísimo ha hecho este mundo para muchos, pero el mundo venidero para pocos.
\par 2 Te diré un ejemplo, Esdras; Como cuando preguntas a la tierra, te dirá que da mucho molde con que se hacen los vasos de barro, pero poco polvo del que sale el oro: así es el curso de este mundo presente.
\par 3 Muchos serán los creados, pero pocos serán los salvos.
\par 4 Entonces respondí y dije: ¡Traga, alma mía, la inteligencia y devora la sabiduría!
\par 5 Porque has aceptado escuchar y estás dispuesto a profetizar, porque ya no te queda más espacio que el de vivir.
\par 6 Oh Señor, si no permites a tu siervo que oremos delante de ti y nos des semilla en nuestro corazón y cultura en nuestro entendimiento, para que de ello venga fruto; ¿Cómo vivirá cada uno de los corruptos que ocupan el lugar de un hombre?
\par 7 Porque tú estás solo y todos nosotros somos una obra de tus manos, tal como has dicho.
\par 8 Porque cuando el cuerpo se forma ahora en el vientre de la madre y tú le das miembros, tu criatura se conserva en el fuego y en el agua, y nueve meses dura tu hechura tu criatura creada en ella.
\par 9 Pero lo que guarda y lo que es guardado, será preservado; y cuando llegue el momento, el útero preservado entregará lo que en él creció.
\par 10 Porque tú has ordenado que de las partes del cuerpo, es decir, de los pechos, se dé leche, que es el fruto de los pechos,
\par 11 Para que lo que está formado pueda ser nutrido por un tiempo, hasta que lo dispongas a tu misericordia.
\par 12 Lo educaste con tu justicia, lo criaste en tu ley y lo reformaste con tu juicio.
\par 13 Y la mortificarás como a tu criatura y la vivificarás como a tu obra.
\par 14 Por lo tanto, si destruyes lo que con tanto trabajo fue creado, es fácil que sea ordenado por tu mandamiento, para que lo que fue hecho se conserve.
\par 15 Ahora pues, Señor, hablaré; En cuanto al hombre en general, tú lo sabes mejor; pero toca a tu pueblo, por cuyo amor me arrepiento;
\par 16 Y por tu herencia, por cuya causa me lamento; y por Israel, por quien estoy pesado; y por Jacob, por quien estoy turbado;
\par 17 Por tanto, comenzaré a orar delante de ti por mí y por ellos, porque veo nuestras caídas, los que habitamos en esta tierra.
\par 18 Pero he oído la rapidez del juez que ha de venir.
\par 19 Por tanto, escucha mi voz y entiende mis palabras, y hablaré delante de ti. Este es el comienzo de las palabras de Esdras, antes de que fuera alzado: y dije:
\par 20 Oh Señor, tú que habitas en la eternidad, que miras desde arriba las cosas en el cielo y en el aire;
\par 21 Cuyo trono es inestimable; cuya gloria no puede ser comprendida; Ante quien están temblando las huestes de los ángeles,
\par 22 Cuyo servicio es experto en el viento y el fuego; cuya palabra es verdadera y dichos constantes; cuyo mandamiento es fuerte y ordenanza temible;
\par 23 Cuya mirada seca los abismos, y la ira hace derretirse las montañas; que la verdad atestigua:
\par 24 Oye la oración de tu siervo y presta atención a la petición de tu criatura.
\par 25 Porque mientras viva hablaré, y mientras tenga entendimiento responderé.
\par 26 No mires los pecados de tu pueblo; sino sobre los que te sirven en verdad.
\par 27 No te preocupes por las malas invenciones de las naciones, sino por el deseo de los que guardan tus testimonios en las aflicciones.
\par 28 No pienses en los que anduvieron fingiendo delante de ti, sino acuérdate de aquellos que, según tu voluntad, conocieron tu temor.
\par 29 Que no sea tu voluntad destruir a los que han vivido como bestias; sino mirar a los que claramente han enseñado tu ley.
\par 30 No te enojes con aquellos que son considerados peores que las bestias; pero ama a los que siempre ponen su confianza en tu justicia y gloria.
\par 31 Porque nosotros y nuestros padres languidecemos a causa de tales enfermedades; pero a causa de nosotros, pecadores, tú serás llamado misericordioso.
\par 32 Porque si quieres tener misericordia de nosotros, serás llamado misericordioso para con nosotros, los que no tenemos obras de justicia.
\par 33 Porque los justos que tienen muchas buenas obras reservadas para ti recibirán recompensa de sus propias obras.
\par 34 ¿Qué es el hombre para que te enfades con él? ¿O qué es una generación corruptible, para que seas tan amargo con ella?
\par 35 Porque en verdad no hay nadie entre los nacidos que no haya hecho maldad; y entre los fieles no hay ninguno que no haya hecho mal.
\par 36 Porque en esto, oh Señor, se declararán tu justicia y tu bondad, si tienes misericordia de los que no confían en las buenas obras.
\par 37 Entonces él me respondió y dijo: Algunas cosas has dicho bien, y según tus palabras será.
\par 38 Porque ciertamente no pensaré en la disposición de los que pecaron antes de la muerte, antes del juicio y antes de la destrucción:
\par 39 Pero me regocijaré por el carácter de los justos, y también me acordaré de su peregrinación, de la salvación y de la recompensa que recibirán.
\par 40 Como he dicho ahora, así sucederá.
\par 41 Porque como el labrador siembra mucha semilla en la tierra y planta muchos árboles, pero lo que se siembra bien en su tiempo no crece, ni todo lo plantado echa raíces, así también ocurre con los que son sembrados en el mundo; no todos serán salvos.
\par 42 Respondí entonces y dije: Si he hallado gracia, déjame hablar.
\par 43 Como perece la simiente del labrador, si no brota y no recibe la lluvia a su debido tiempo; o si llueve demasiado y la corrompe:
\par 44 Así también perece el hombre que fue formado con tus manos y fue llamado tu imagen, porque eres semejante a aquel por quien hiciste todas las cosas, y lo asemejaste a la descendencia del labrador.
\par 45 No te enojes con nosotros, sino perdona a tu pueblo y ten misericordia de tu propia herencia, porque tú eres misericordioso con tu criatura.
\par 46 Entonces él me respondió y dijo: Lo presente es para lo presente, y lo por venir, para lo que será.
\par 47 Porque te falta mucho para poder amar a mi criatura más que a mí; pero muchas veces me he acercado a ti y a ella, pero nunca a los injustos.
\par 48 También en esto eres admirable ante el Altísimo:
\par 49 Por haberte humillado, como te conviene, y no haberte juzgado digno de ser muy glorificado entre los justos.
\par 50 Porque a los que en el último tiempo habitarán en el mundo les sufrirán muchas y grandes miserias, porque han caminado con gran soberbia.
\par 51 Pero tú, compréndelo a ti mismo y busca la gloria para los que son como tú.
\par 52 Porque a vosotros se os abre el paraíso, se planta el árbol de la vida, se prepara el tiempo venidero, se prepara la abundancia, se construye una ciudad y se permite el descanso, sí, la bondad y la sabiduría perfectas.
\par 53 La raíz del mal está sellada para ti, la debilidad y la polilla están ocultas para ti, y la corrupción ha huido al infierno para ser olvidada.
\par 54 Los dolores pasan, y al final se muestra el tesoro de la inmortalidad.
\par 55 Por tanto, no hagas más preguntas sobre la multitud de los que perecen.
\par 56 Porque cuando se tomaron la libertad, despreciaron al Altísimo, despreciaron su ley y abandonaron sus caminos.
\par 57 Además, pisotearon a sus justos,
\par 58 Y decían en su corazón que no hay Dios; sí, y sabiendo que deben morir.
\par 59 Porque así como las cosas antes mencionadas os recibirán a vosotros, así les están preparadas la sed y el dolor; porque no era su voluntad que los hombres perdieran la vida.
\par 60 Pero los que fueron creados profanaron el nombre de quien los hizo y fueron desagradecidos con quien les preparó la vida.
\par 61 Por eso mi juicio está ahora a la mano.
\par 62 Estas cosas no las he mostrado a todos, sino a ti y a unos pocos como tú. Entonces respondí y dije:
\par 63 He aquí, Señor, ahora me has mostrado la multitud de maravillas que comenzarás a realizar en los últimos tiempos, pero no me has mostrado cuándo.

\chapter{9}

\par 1 Entonces él me respondió y me dijo: Mide cuidadosamente el tiempo en sí mismo; y cuando veas parte de las señales pasadas que te he dicho antes,
\par 2 Entonces entenderás que es el mismo momento en que el Altísimo comenzará a visitar el mundo que él creó.
\par 3 Por tanto, cuando se vean terremotos y alborotos de los pueblos en el mundo,
\par 4 Entonces entenderás bien que el Altísimo habló de estas cosas desde los días que fueron antes de ti, desde el principio.
\par 5 Porque así como todo lo que existe en el mundo tiene un principio y un fin, y el fin es manifiesto:
\par 6 Así también los tiempos del Altísimo tienen un comienzo claro con maravillas y obras poderosas, y un final con efectos y señales.
\par 7 Y todo el que se salve y pueda escapar por sus obras y por la fe en la que creísteis,
\par 8 Serán preservados de dichos peligros y verán mi salvación en mi tierra y dentro de mis términos, porque yo los he santificado para mí desde el principio.
\par 9 Entonces sufrirán lástima los que ahora han abusado de mis caminos; y los que los han desechado con desprecio vivirán en tormentos.
\par 10 Porque los que en su vida recibieron beneficios y no me conocieron;
\par 11 Y los que aborrecieron mi ley, cuando todavía tenían libertad, y cuando todavía se les abrió el lugar del arrepentimiento, no la entendieron, sino que la despreciaron;
\par 12 Lo mismo debe saberlo después de la muerte por el dolor.
\par 13 Por tanto, no te preguntes cómo y cuándo serán castigados los impíos, sino cómo se salvarán los justos, de quién es el mundo y para quién fue creado el mundo.
\par 14 Entonces respondí y dije:
\par 15 He dicho antes, y lo hablo ahora, y lo diré también en lo sucesivo, que son muchos más los que perecen que los que se salvan.
\par 16 Como una ola es mayor que una gota.
\par 17 Y él me respondió diciendo: Como es el campo, así es la semilla; como son las flores, así son también los colores; Tal como es el obrero, así también es el trabajo; y como es el labrador, así es también su labranza; porque era el tiempo del mundo.
\par 18 Y cuando preparé el mundo, que aún no había sido creado, para que habitaran en él los que ahora viven, nadie habló contra mí.
\par 19 Porque entonces todos obedecieron; pero ahora las costumbres de los que son creados en este mundo creado, son corrompidas por una semilla perpetua, y por una ley inescrutable se libran ellos mismos.
\par 20 Entonces miré el mundo, y he aquí que había peligro a causa de los artefactos que habían entrado en él.
\par 21 Y vi y la perdoné mucho, y me guardé una uva del racimo y una planta de un gran pueblo.
\par 22 Perezca pues la multitud que nació en vano; y que se guarden mi uva y mi planta; porque con mucho trabajo lo he perfeccionado.
\par 23 Sin embargo, si cesas aún siete días más (pero no ayunas en ellos),
\par 24 Pero id a un campo florido, donde no hay casa construida, y comed sólo las flores del campo; No pruebes carne, no bebas vino, solo come flores;)
\par 25 Y ora continuamente al Altísimo, y entonces vendré y hablaré contigo.
\par 26 Así que me fui al campo llamado Ardat, tal como él me había ordenado; y allí me senté entre las flores, y comí de las hierbas del campo, y la carne de ellas me saciaba.
\par 27 Después de siete días, me senté sobre la hierba y mi corazón se afligió dentro de mí como antes.
\par 28 Y abrí mi boca y comencé a hablar delante del Altísimo, y dije:
\par 29 Oh Señor, tú que te muestras a nosotros, fuiste manifestado a nuestros padres en el desierto, en un lugar donde nadie pisa, en un lugar árido, cuando salieron de Egipto.
\par 30 Y tú hablaste diciendo: Escúchame, oh Israel; y recuerda mis palabras, descendencia de Jacob.
\par 31 Porque he aquí, yo siembro en vosotros mi ley, y ella dará fruto en vosotros, y en ella seréis honrados para siempre.
\par 32 Pero nuestros padres, que recibieron la ley, no la guardaron, ni observaron tus ordenanzas; y aunque el fruto de tu ley no pereció, tampoco pereció, porque era tuyo;
\par 33 Pero los que lo recibieron perecieron, porque no guardaron lo que en ellos fue sembrado.
\par 34 Y he aquí, es costumbre que, cuando la tierra ha recibido semilla, o el mar un barco, o cualquier vaso, comida o bebida, perezca aquello en que fue sembrado o arrojado,
\par 35 También lo que fue sembrado, o arrojado en él, o recibido, perece y no permanece con nosotros, pero entre nosotros no ha sucedido así.
\par 36 Porque nosotros que hemos recibido la ley perecemos por el pecado, y también nuestro corazón que la recibió.
\par 37 Sin embargo, la ley no perece, sino que permanece en su vigencia.
\par 38 Y mientras hablaba estas cosas en mi corazón, miré hacia atrás con mis ojos, y a mi derecha vi a una mujer, la cual se lamentó y lloró a gran voz, y se entristeció mucho en su corazón. y sus vestidos estaban rasgados, y tenía ceniza sobre su cabeza.
\par 39 Entonces dejé ir mis pensamientos en los que estaba y me volví hacia ella,
\par 40 Y le dijo: ¿Por qué lloras? ¿Por qué estás tan afligido en tu mente?
\par 41 Y ella me dijo: Señor, déjame, para que me lamente y aumente mi tristeza, porque estoy muy afligida y abatida.
\par 42 Y le dije: ¿Qué te pasa? dime.
\par 43 Ella me dijo: Yo, tu sierva, he sido estéril y no he tenido hijos, aunque tuve marido durante treinta años.
\par 44 Y durante esos treinta años no hice otra cosa, día y noche y a cada hora, sino elevar mi oración al Altísimo.
\par 45 Después de treinta años, Dios me escuchó, tu sierva, vio mi miseria, consideró mi aflicción y me dio un hijo; y yo me alegré mucho de él, lo mismo que mi marido y todos mis vecinos; y le dimos mucho. honra al Todopoderoso.
\par 46 Y lo alimenté con grandes trabajos.
\par 47 Cuando él creció y llegó el momento en que debía tener esposa, hice un banquete.

\chapter{10}

\par 1 Y aconteció que cuando mi hijo entró en su cámara nupcial, cayó y murió.
\par 2 Entonces todos apagamos las luces, y todos mis vecinos se levantaron para consolarme: así descansé hasta el segundo día por la noche.
\par 3 Y aconteció que cuando todos dejaron de consolarme, pude quedarme tranquilo hasta el fin; Entonces me levanté de noche y huí, y vine a este campo, como ves.
\par 4 Y ahora tengo la intención de no volver a la ciudad, sino quedarme aquí, y no comer ni beber, sino llorar y ayunar continuamente hasta que muera.
\par 5 Entonces salí de mis meditaciones en las que estaba y le hablé enojado, diciendo:
\par 6 Mujer más necia que todas las demás, ¿no ves nuestro duelo y lo que nos sucede?
\par 7 ¿Cómo puede ser que nuestra madre Sión esté llena de toda tristeza y muy humillada, y de luto muy doloroso?
\par 8 Y ahora, viendo que todos lloramos y estamos tristes, porque todos estamos afligidos, ¿estás afligido por un solo hijo?
\par 9 Pregúntale a la tierra, y ella te dirá, que ella es la que debe llorar la caída de tantos que crecen en ella.
\par 10 Porque de ella salieron todos al principio, y de ella saldrán todos los demás, y he aquí, casi todos caminan hacia la destrucción, y una multitud de ellos es completamente desarraigada.
\par 11 ¿Quién, pues, podría hacer más luto que ella, que ha perdido tanta multitud? ¿Y no tú, que te arrepientes sino de uno solo?
\par 12 Pero si me dices: Mi lamento no es como el de la tierra, porque he perdido el fruto de mi vientre, que con dolores di a luz y que con dolores di a luz;
\par 13 Pero no así la tierra, porque la multitud que estaba en ella según el curso de la tierra, se fue como vino.
\par 14 Entonces te digo que como tú has parido con trabajo; Así también la tierra ha dado su fruto, es decir, el hombre, desde el principio, hasta el que la hizo.
\par 15 Ahora pues, guarda para ti tu dolor y soporta con buen ánimo lo que te ha sucedido.
\par 16 Porque si reconoces que la determinación de Dios es justa, recibirás a tu hijo a tiempo y serás alabada entre las mujeres.
\par 17 Vé, pues, a la ciudad, donde tu marido.
\par 18 Y ella me dijo: No haré eso; no iré a la ciudad, sino que aquí moriré.
\par 19 Entonces empecé a hablarle más y le dije:
\par 20 No hagáis eso, sino aconsejaos. por mí: ¿para cuántas son las adversidades de Sión? consolaos por el dolor de Jerusalén.
\par 21 Porque ves que nuestro santuario está arrasado, nuestro altar derribado, nuestro templo destruido;
\par 22 Nuestro salterio fue puesto en tierra, nuestro cántico fue acallado, nuestro regocijo llegó a su fin, la luz de nuestro candelero fue apagada, el arca de nuestra alianza fue echada a perder, nuestras cosas sagradas fueron contaminadas y el El nombre que se nos invoca es casi profanado: nuestros hijos son avergonzados, nuestros sacerdotes son quemados, nuestros levitas son llevados al cautiverio, nuestras vírgenes son contaminadas y nuestras mujeres violadas; nuestros justos fueron llevados, nuestros pequeños destruidos, nuestros jóvenes fueron reducidos a servidumbre y nuestros hombres fuertes se debilitaron;
\par 23 Y, lo que es más grande que todo, el sello de Sión ahora ha perdido su honor; porque ella está entregada en manos de los que nos odian.
\par 24 Y, por tanto, deshazte de tu gran pesadez y deshazte de la multitud de dolores, para que el Poderoso vuelva a tener misericordia de ti y el Altísimo te dé descanso y tranquilidad en tu trabajo.
\par 25 Y aconteció que mientras hablaba con ella, de repente su rostro se iluminó en gran manera y su semblante resplandeció, de modo que tuve miedo de ella y pensé qué podría ser.
\par 26 Y he aquí, de repente ella lanzó un gran grito, muy espantoso, de modo que la tierra tembló ante el ruido de la mujer.
\par 27 Y miré, y he aquí, la mujer ya no se me apareció más, sino que había una ciudad edificada, y un lugar grande se asomaba desde los cimientos. Entonces tuve miedo, y clamé a gran voz, y dije ,
\par 28 ¿Dónde está el ángel Uriel, que vino a mí la primera vez? porque él me ha hecho caer en muchos trances, y mi fin se ha vuelto en corrupción, y mi oración en reprensión.
\par 29 Y mientras yo hablaba estas palabras, he aquí, él vino a mí y me miró.
\par 30 Y he aquí, yacía como un muerto y me fue quitado el entendimiento; y él me tomó de la mano derecha, me consoló, me puso sobre mis pies y me dijo:
\par 31 ¿Qué te pasa? ¿Y por qué estás tan inquieto? ¿Y por qué se turba tu entendimiento y los pensamientos de tu corazón?
\par 32 Y dije: Porque me has abandonado, y sin embargo hice según tus palabras, y fui al campo, y he aquí, he visto, y todavía veo, que no puedo expresar.
\par 33 Y él me dijo: Levántate valientemente y yo te aconsejaré.
\par 34 Entonces dije: Habla, señor mío, en mí; sólo que no me abandones, no sea que muera frustrado mi esperanza.
\par 35 Porque he visto lo que no sabía, y oigo que no sé.
\par 36 ¿O está engañado mi sentido, o mi alma en un sueño?
\par 37 Ahora pues, te ruego que le muestres a tu siervo esta visión.
\par 38 Entonces él me respondió y dijo: Escúchame y te informaré y te diré por qué tienes miedo, porque el Altísimo te revelará muchos secretos.
\par 39 Él ha visto que tu camino es recto, porque continuamente te afliges por tu pueblo y haces grandes lamentaciones por Sión.
\par 40 Éste, pues, es el significado de la visión que acabas de ver:
\par 41 Viste a una mujer enlutada y comenzaste a consolarla.
\par 42 Pero ahora ya no ves la figura de la mujer, sino que se te apareció una ciudad edificada.
\par 43 Y mientras ella te anunció la muerte de su hijo, esta es la solución:
\par 44 Esta mujer que viste es Sión; y ella te dijo: La que ves como una ciudad edificada:
\par 45 Mientras que ella te dijo que ha sido estéril durante treinta años: esos son los treinta años en los que no se hizo ninguna ofrenda en ella.
\par 46 Pero después de treinta años, Salomón edificó la ciudad y ofreció ofrendas, y luego dio a luz un hijo a la estéril.
\par 47 Y mientras ella te decía que lo alimentaba con trabajo, esa era la morada en Jerusalén.
\par 48 Pero ella te dijo: «Mi hijo, al entrar en su alcoba, tuvo una enfermedad y murió», ésta fue la destrucción que sobrevino a Jerusalén.
\par 49 Y he aquí, viste su semejanza y, como ella lloraba por su hijo, comenzaste a consolarla; y de estas cosas que te han sucedido, éstas te serán reveladas.
\par 50 Porque ahora el Altísimo ve que estás sinceramente afligido y que sufres de todo corazón por ella, y te ha mostrado el resplandor de su gloria y la hermosura de su hermosura.
\par 51 Por eso te ordené que te quedaras en el campo donde no se había construido ninguna casa.
\par 52 Porque sabía que el Altísimo te mostraría esto.
\par 53 Por eso te ordené que fueras al campo, donde no había cimientos de ningún edificio.
\par 54 Porque en el lugar donde el Altísimo comienza a mostrar su ciudad, ningún edificio de hombre podrá mantenerse en pie.
\par 55 Por tanto, no temas, no se espante tu corazón, sino entra y contempla la hermosura y la grandeza del edificio, tanto como tus ojos puedan ver.
\par 56 Y entonces oirás todo lo que tus oídos puedan comprender.
\par 57 Porque tú eres bendito más que muchos, y eres llamado con el Altísimo; y también lo son unos pocos.
\par 58 Pero mañana por la noche permanecerás aquí;
\par 59 Y así el Altísimo te mostrará visiones de las cosas elevadas que el Altísimo hará a los que habitan la tierra en los últimos días. Así que dormí esa noche y otra, como él me ordenó.

\chapter{11}

\par 1 Entonces tuve un sueño y he aquí que subía del mar un águila que tenía doce alas emplumadas y tres cabezas.
\par 2 Y miré, y he aquí, ella extendió sus alas sobre toda la tierra, y todos los vientos del aire soplaron sobre ella y se juntaron.
\par 3 Y miré, y de sus plumas nacieron otras plumas contrarias; y se convirtieron en plumas pequeñas y pequeñas.
\par 4 Pero sus cabezas estaban en reposo: la cabeza en medio era más grande que la otra, pero reposaba con el resto.
\par 5 Y miré, y he aquí, el águila volaba con sus plumas y reinaba sobre la tierra y sobre los que en ella habitaban.
\par 6 Y vi que todas las cosas bajo el cielo estaban sujetas a ella, y ningún hombre hablaba contra ella, ni siquiera una sola criatura sobre la tierra.
\par 7 Y miré, y he aquí, el águila se levantó sobre sus garras y habló a sus plumas, diciendo:
\par 8 No veléis todos a la vez: duerme cada uno en su lugar y velad por turno:
\par 9 Pero que las cabezas se conserven para el final.
\par 10 Y miré, y he aquí, la voz no salía de su cabeza, sino de en medio de su cuerpo.
\par 11 Y conté sus plumas contrarias, y he aquí, eran ocho.
\par 12 Y miré, y he aquí, a la derecha se levantaba una pluma y reinaba sobre toda la tierra;
\par 13 Y sucedió que cuando reinó, llegó su fin y su lugar desapareció: así se levantaron los siguientes. y reinó, y la pasó muy bien;
\par 14 Y aconteció que cuando reinó, vino también su fin, como el primero, de modo que no apareció más.
\par 15 Entonces llegó una voz que decía:
\par 16 Oye tú, que has gobernado la tierra durante tanto tiempo: esto te digo, antes de que comiences a no aparecer más:
\par 17 Nadie después de ti alcanzará tu tiempo, ni la mitad de él.
\par 18 Entonces se levantó el tercero y reinó como el otro antes, y tampoco apareció más.
\par 19 Así sucedió con todos los demás, uno tras otro, de modo que cada uno reinó y luego no apareció más.
\par 20 Entonces miré, y he aquí, con el tiempo las plumas que seguían se levantaron hacia el lado derecho, para poder también gobernar; y algunos de ellos gobernaron, pero al cabo de un tiempo ya no aparecieron:
\par 21 Porque algunos de ellos fueron establecidos, pero no gobernaron.
\par 22 Después de esto miré, y he aquí, ya no aparecieron las doce plumas, ni las dos plumitas:
\par 23 Y sobre el cuerpo del águila ya no había más que tres cabezas que descansaban y seis alitas.
\par 24 Entonces vi también que dos plumitas se separaban de las seis y quedaban debajo de la cabeza que estaba al lado derecho; porque las cuatro continuaban en su lugar.
\par 25 Y miré, y he aquí, las plumas que estaban debajo del ala pensaron levantarse y dominar.
\par 26 Y miré, y he aquí que había uno colocado, pero al poco tiempo ya no apareció.
\par 27 Y el segundo estaba más cerca que el primero.
\par 28 Y miré, y he aquí, los dos que quedaban pensaban también reinar en sí mismos.
\par 29 Y cuando pensaban así, he aquí, una de las cabezas que estaban en reposo, es decir, la que estaba en medio, se despertó; porque ésta era mayor que las otras dos cabezas.
\par 30 Y entonces vi que las otras dos cabezas estaban unidas a él.
\par 31 Y he aquí, la cabeza se volvió con los que estaban con ella y se comió las dos plumas debajo del ala que habría de reinar.
\par 32 Pero este jefe infundió temor a toda la tierra y ejerció dominio sobre todos los que habitaban la tierra con mucha opresión; y tenía el gobierno del mundo más que todas las alas que habían existido.
\par 33 Y después de esto miré, y he aquí, la cabeza que estaba en medio de repente dejó de aparecer como las alas.
\par 34 Pero quedaron las dos cabezas, que también gobernaban sobre la tierra y sobre los que en ella habitaban.
\par 35 Y miré, y he aquí, la cabeza del lado derecho devoró a la que estaba del lado izquierdo.
\par 36 Entonces oí una voz que me dijo: Mira delante de ti y considera lo que ves.
\par 37 Y miré, y he aquí, como un león rugiente perseguido fuera del bosque; y vi que envió una voz de hombre al águila, y dijo:
\par 38 Oye, hablaré contigo y el Altísimo te dirá:
\par 39 ¿No eres tú el que queda de las cuatro bestias, a quienes hice reinar en mi mundo, para que por medio de ellas llegue el fin de sus tiempos?
\par 40 Y vino el cuarto, y venció a todas las bestias que habían pasado, y tuvo poder sobre el mundo con gran pavor, y sobre toda la extensión de la tierra con gran opresión perversa; y por tanto tiempo habitó sobre la tierra con engaño.
\par 41 Porque no has juzgado la tierra con verdad.
\par 42 Porque afligiste a los humildes, heriste a los pacíficos, amaste a los mentirosos, destruiste las moradas de los que daban fruto y derribaste los muros de los que no te hacían daño.
\par 43 Por eso tu maldad ha subido al Altísimo, y tu orgullo al Poderoso.
\par 44 El Altísimo también miró los tiempos de soberbia, y he aquí, han llegado a su fin y sus abominaciones se han cumplido.
\par 45 Por tanto, no aparezcas más, águila, ni tus horribles alas, ni tus malvadas plumas, ni tus maliciosas cabezas, ni tus dañinas garras, ni todo tu vano cuerpo.
\par 46 Para que toda la tierra se renueve y vuelva, libre de tu violencia, y pueda esperar el juicio y la misericordia de aquel que la hizo.

\chapter{12}

\par 1 Y aconteció que mientras el león hablaba estas palabras al águila, vi:
\par 2 Y he aquí, la cabeza que había quedado y las cuatro alas desaparecieron, y los dos fueron hacia ella y se pusieron a reinar, y su reino era pequeño y estaba lleno de alboroto.
\par 3 Y miré, y he aquí, ya no aparecían, y todo el cuerpo del águila estaba quemado, de modo que la tierra estaba en gran temor; entonces desperté de la angustia y del trance de mi mente, y de gran miedo, y dijo a mi espíritu:
\par 4 He aquí, esto me has hecho, al buscar los caminos del Altísimo.
\par 5 He aquí, todavía estoy cansado de mente y muy débil de espíritu; y poca fuerza hay en mí, por el gran temor que me afligió esta noche.
\par 6 Por eso ahora rogaré al Altísimo que me consuele hasta el fin.
\par 7 Y dije: Señor, que gobiernas, si he hallado gracia ante tus ojos, y si soy justificado contigo ante muchos otros, y si mi oración verdaderamente llega ante tu faz;
\par 8 Consuélame, pues, y muéstrame, tu siervo, la interpretación y la clara diferencia de esta terrible visión, para que puedas consolar perfectamente mi alma.
\par 9 Porque me has juzgado digno de mostrármelo los últimos tiempos.
\par 10 Y me dijo: Ésta es la interpretación de la visión:
\par 11 El águila que viste subir del mar es el reino que apareció en la visión de tu hermano Daniel.
\par 12 Pero a él no le fue explicado, por eso ahora te lo declaro a ti.
\par 13 He aquí, vendrán días en que se levantará un reino sobre la tierra, y será más temido que todos los reinos que existieron antes de él.
\par 14 En ella reinarán doce reyes, uno tras otro:
\par 15 De los cuales el segundo comenzará a reinar y tendrá más tiempo que cualquiera de los doce.
\par 16 Y esto significan las doce alas que viste.
\par 17 En cuanto a la voz que oíste hablar, y que viste que no salía de sus cabezas, sino de en medio de su cuerpo, esta es la interpretación:
\par 18 Que después del tiempo de ese reino surgirán grandes luchas y estará en peligro de fracasar; sin embargo, no caerá, sino que será restaurado nuevamente a su comienzo.
\par 19 Y si viste las ocho pequeñas plumas inferiores que se pegaban a sus alas, esta es la interpretación:
\par 20 Que en él se levantarán ocho reyes, cuyos tiempos serán cortos y sus años veloces.
\par 21 Y dos de ellos perecerán, al acercarse el tiempo medio; cuatro serán guardados hasta que su fin comience a acercarse; pero dos serán guardados hasta el fin.
\par 22 Y lo que viste tres cabezas reposando, esta es la interpretación:
\par 23 En sus últimos días, el Altísimo levantará tres reinos, y en ellos renovará muchas cosas, y tendrán dominio sobre la tierra.
\par 24 Y a los que allí habitan, con mucha opresión, más que a todos los que fueron antes de ellos: por eso se les llama cabezas de águila.
\par 25 Porque éstos son los que cometerán su maldad y cumplirán su último fin.
\par 26 Y si viste que la gran cabeza ya no aparecía, significa que uno de ellos morirá en su cama, pero con dolor.
\par 27 Porque los dos que queden serán muertos a espada.
\par 28 Porque la espada de uno devorará al otro, pero al final él mismo caerá por la espada.
\par 29 Y mientras viste dos plumas debajo de las alas que pasaban sobre la cabeza que está al lado derecho;
\par 30 Esto significa que estos son aquellos a quienes el Altísimo ha mantenido hasta su fin: este es el reino pequeño y lleno de problemas, como viste.
\par 31 Y el león que viste levantándose del bosque, rugiendo y hablando con el águila, y reprendiéndola por su injusticia con todas las palabras que has oído;
\par 32 Este es el ungido que el Altísimo ha guardado para ellos y para su maldad hasta el fin: él los reprenderá y los reprenderá por su crueldad.
\par 33 Porque él los presentará vivos ante él en el juicio, y los reprenderá y corregirá.
\par 34 Porque al resto de mi pueblo él librará con misericordia a los que están presionados en mis fronteras, y los alegrará hasta que llegue el día del juicio, del cual te he hablado desde el principio.
\par 35 Este es el sueño que viste, y éstas son las interpretaciones.
\par 36 Sólo tú has sido apto para conocer este secreto del Altísimo.
\par 37 Escribe, pues, en un libro todas estas cosas que has visto, y escóndelas.
\par 38 Y enséñaselos a los sabios del pueblo, cuyos corazones conoces, pueden comprender y guardar estos secretos.
\par 39 Pero tú, espera aquí todavía siete días más, para que se te muestre todo lo que el Altísimo quiera declararte. Y con eso siguió su camino.
\par 40 Y aconteció que cuando todo el pueblo vio que habían pasado los siete días y que yo no volvía a la ciudad, se reunieron todos, desde el más pequeño hasta el mayor, y vinieron a mí y me dijeron ,
\par 41 ¿En qué te hemos ofendido? ¿Y qué mal hemos hecho contra ti, para que nos abandones y te sientes aquí en este lugar?
\par 42 Porque de todos los profetas sólo tú has quedado de nosotros, como racimo de vendimia, como candela en lugar oscuro, y como puerto o barco preservado de la tempestad.
\par 43 ¿No son suficientes los males que nos sobrevienen?
\par 44 Si nos abandonaras, ¿cuánto mejor sería para nosotros si también nosotros fuéramos quemados en medio de Sión?
\par 45 Porque no somos mejores que los que murieron allí. Y lloraron a gran voz. Entonces les respondí y dije:
\par 46 Ten consuelo, oh Israel; y no os canséis, casa de Jacob:
\par 47 Porque el Altísimo te recuerda, y el Poderoso no te olvida en la tentación.
\par 48 En cuanto a mí, no os he abandonado ni me he apartado de vosotros, sino que he venido a este lugar para orar por la desolación de Sión y para pedir misericordia para la humillación de vuestro santuario.
\par 49 Ahora volved cada uno a casa, y después de estos días volveré a vosotros.
\par 50 Entonces el pueblo entró en la ciudad, tal como yo les había ordenado:
\par 51 Pero me quedé en el campo siete días, tal como el ángel me ordenó; y en aquellos días sólo comía flores del campo, y mi carne consistía en hierbas

\chapter{13}

\par 1 Y aconteció que después de siete días tuve un sueño de noche:
\par 2 Y he aquí, se levantó del mar un viento que movía todas sus olas.
\par 3 Y miré, y he aquí, aquel hombre se hacía fuerte con los millares del cielo; y cuando volvía su rostro para mirar, temblaban todas las cosas que se veían debajo de él.
\par 4 Y cada vez que la voz salía de su boca, todos los que oían su voz se quemaban, como desfallece la tierra cuando siente el fuego.
\par 5 Y después de esto miré, y he aquí, se había reunido una multitud de hombres, sin número, de los cuatro vientos del cielo, para dominar al hombre que había salido del mar.
\par 6 Pero yo miré y he aquí que él había esculpido una gran montaña y voló sobre ella.
\par 7 Pero quise ver la región o el lugar donde estaba esculpida la colina, pero no pude.
\par 8 Y después de esto miré y he aquí que todos los que se habían reunido para someterlo tuvieron mucho miedo y, sin embargo, se atrevieron a luchar.
\par 9 Y he aquí, al ver la violencia de la multitud que venía, no alzó la mano, ni empuñó espada, ni instrumento alguno de guerra.
\par 10 Pero sólo yo vi que de su boca lanzaba como una ráfaga de fuego, y de sus labios un aliento de fuego, y de su lengua expulsaba chispas y tempestades.
\par 11 Y estaban todos mezclados; la explosión de fuego, el aliento de fuego y la gran tempestad; y cayó con violencia sobre la multitud que estaba preparada para pelear, y los quemó a todos, de modo que de repente de una multitud innumerable no se percibió nada, sino sólo polvo y olor a humo: cuando vi esto tuve miedo. .
\par 12 Después vi al mismo hombre descender del monte y llamar a otra multitud pacífica.
\par 13 Y vino a él mucha gente, de la cual algunos se alegraron, otros se entristecieron, algunos fueron atados y otros trajeron de los que eran ofrecidos. Entonces enfermé de gran miedo, y desperté, y dicho,
\par 14 Tú has mostrado a tu siervo estas maravillas desde el principio y me has tenido por digno de recibir mi oración.
\par 15 Muéstrame ahora todavía la interpretación de este sueño.
\par 16 Pues, tal como lo concibo en mi entendimiento, ¡ay de los que quedarán en aquellos días, y mucho más ay de los que no queden atrás!
\par 17 Porque los que quedaron estaban afligidos.
\par 18 Ahora entiendo lo que está guardado en los últimos días, lo que les sucederá a ellos y a los que quedarán atrás.
\par 19 Por eso se ven en grandes peligros y en muchas necesidades, como lo anuncian estos sueños.
\par 20 Pero es más fácil para el que está en peligro llegar a estas cosas, que pasar como una nube fuera del mundo y no ver las cosas que sucederán en los últimos días. Y él me respondió y dijo:
\par 21 Yo te mostraré la interpretación de la visión y te abriré lo que me pides.
\par 22 Lo que dijiste de los que quedaron atrás, esta es la interpretación:
\par 23 El que soporta el peligro en aquel tiempo se conserva a sí mismo; los que caen en peligro son los que tienen obras y fe en el Todopoderoso.
\par 24 Sepan, pues, que los que quedan atrás son más bienaventurados que los que mueren.
\par 25 Este es el significado de la visión: Mientras viste a un hombre que subía de en medio del mar,
\par 26 Éste es aquel a quien el Dios Altísimo ha reservado un gran tiempo, el cual por sí mismo librará a su criatura y ordenará a las que quedan atrás.
\par 27 Y mientras viste que de su boca salía una ráfaga de viento, fuego y tormenta;
\par 28 Y que no empuñaba espada ni ningún instrumento de guerra, sino que su irrupción destruyó a toda la multitud que venía a someterlo; esta es la interpretación:
\par 29 He aquí, vienen días en que el Altísimo comenzará a librar a los que están sobre la tierra.
\par 30 Y vendrá para asombro de los habitantes de la tierra.
\par 31 Y uno se comprometerá a luchar contra otro, una ciudad contra otra, un lugar contra otro, un pueblo contra otro, y un reino contra otro.
\par 32 Y llegará el tiempo en que sucederán estas cosas, y se realizarán las señales que antes te he mostrado, y entonces será declarado mi Hijo, a quien viste como un hombre ascendiendo.
\par 33 Y cuando todo el pueblo oiga su voz, cada uno en su propia tierra abandonará la batalla que tienen unos contra otros.
\par 34 Y se reunirá una multitud innumerable, como tú los viste, dispuesta a venir y vencerlo peleando.
\par 35 Pero él estará sobre la cima del monte Sión.
\par 36 Y vendrá Sión, y será mostrada a todos los hombres, preparada y edificada, como viste la colina esculpida sin manos.
\par 37 Y este mi Hijo reprenderá los malvados inventos de aquellas naciones, que por su mala vida cayeron en la tempestad;
\par 38 Y les expondrá sus malos pensamientos y los tormentos con que comenzarán a ser atormentados, que son como una llama, y ​​los destruirá sin trabajo según la ley que es como yo.
\par 39 Y mientras viste que reunió consigo otra multitud pacífica;
\par 40 Estas son las diez tribus que fueron llevadas prisioneras de su propia tierra en tiempos del rey Osea, a quienes Salmanasar, rey de Asiria, llevó cautivos, y los llevó a través de las aguas, y así llegaron a otra tierra.
\par 41 Pero ellos decidieron entre ellos dejar la multitud de los paganos e ir a un país más lejano, donde nunca habitó la humanidad.
\par 42 Para que allí pudieran guardar sus estatutos, que nunca guardaron en su propia tierra.
\par 43 Y entraron en el Éufrates por las estrechas del río.
\par 44 Porque el Altísimo les hizo señales y detuvo el diluvio hasta que pasaron.
\par 45 Porque a través de aquella tierra había un largo camino por recorrer, es decir, de año y medio; y la misma región se llama Arsareth.
\par 46 Entonces habitaron allí hasta el fin; y ahora cuando comiencen a venir,
\par 47 El Altísimo detendrá de nuevo las fuentes del río para que puedan pasar. Por eso viste a la multitud en paz.
\par 48 Pero los que quedarán de tu pueblo serán los que se encuentren dentro de mis fronteras.
\par 49 Ahora, cuando destruya la multitud de las naciones reunidas, defenderá a su pueblo que quede.
\par 50 Y entonces les mostrará grandes maravillas.
\par 51 Entonces dije: Oh Señor, que gobiernas, muéstrame esto: ¿Por qué he visto al hombre que sube de en medio del mar?
\par 52 Y me dijo: Así como tú no puedes buscar ni conocer las cosas que están en las profundidades del mar, así tampoco ningún hombre en la tierra puede ver a mi Hijo, ni a los que están con él, sino en el hora del día.
\par 53 Esta es la interpretación del sueño que viste y por el cual aquí sólo te iluminas.
\par 54 Porque abandonaste tu camino y aplicaste tu diligencia a mi ley y la buscaste.
\par 55 Has ordenado tu vida con sabiduría y has llamado a la inteligencia tu madre.
\par 56 Por eso te he mostrado los tesoros del Altísimo: después de otros tres días te hablaré otras cosas y te declararé cosas poderosas y maravillosas.
\par 57 Entonces salí al campo, alabando y dando gracias al Altísimo por las maravillas que había hecho a tiempo;
\par 58 Y porque él gobierna lo mismo y las cosas que caen en su tiempo, y estuve allí sentado tres días.

\chapter{14}

\par 1 Y aconteció que al tercer día, estaba sentado debajo de una encina, y he aquí, de un arbusto que estaba frente a mí salió una voz que decía: Esdras, Esdras.
\par 2 Y dije: Aquí estoy, Señor, y me levanté sobre mis pies.
\par 3 Entonces me dijo: En la zarza me revelé claramente a Moisés y hablé con él, cuando mi pueblo servía en Egipto.
\par 4 Lo envié y saqué a mi pueblo de Egipto y lo llevé al monte, donde lo retuve conmigo por un largo tiempo.
\par 5 Y le contó muchas maravillas y le mostró los secretos de los tiempos y del fin; y le ordenó, diciendo:
\par 6 Estas palabras declararás y ocultarás.
\par 7 Y ahora te digo:
\par 8 Para que guardes en tu corazón las señales que te he mostrado, los sueños que has visto y las interpretaciones que has oído.
\par 9 Porque serás quitado de todos, y desde ahora permanecerás con mi Hijo y con los que son como tú, hasta el fin de los tiempos.
\par 10 Porque el mundo ha perdido su juventud y los tiempos comienzan a envejecer.
\par 11 Porque el mundo está dividido en doce partes, y las diez partes ya han desaparecido, y la mitad de la décima parte.
\par 12 Y queda lo que está después de la mitad de la décima parte.
\par 13 Ahora, pues, ordena tu casa y reprende a tu pueblo, consola a los que están en problemas y renuncia ahora a la corrupción.
\par 14 Deja ir de ti los pensamientos mortales, desecha las cargas del hombre, despojate ahora de la naturaleza débil,
\par 15 Y deja a un lado los pensamientos que te resultan más pesados ​​y apresúrate a huir de estos tiempos.
\par 16 Porque en el futuro se producirán males aún mayores que los que has visto suceder.
\par 17 Porque mirad, cuánto más débil será el mundo con el tiempo, tanto más aumentarán los males sobre los que en él habitan.
\par 18 Porque el tiempo ha pasado muy lejos y el tiempo es difícil de alcanzar; porque ahora se apresura la visión que has visto.
\par 19 Entonces respondí delante de ti y dije:
\par 20 He aquí, Señor, iré como me has mandado y reprenderé al pueblo que está presente; pero a los que nacerán después, ¿quién los amonestará? Así el mundo está en tinieblas, y los que en él habitan están sin luz.
\par 21 Porque tu ley ha sido quemada, por eso nadie sabe lo que has hecho ni la obra que ha de comenzar.
\par 22 Pero si he hallado gracia delante de ti, envía dentro de mí el Espíritu Santo, y escribiré todo lo que se ha hecho en el mundo desde el principio, que fue escrito en tu ley, para que los hombres encuentren tu camino y para que vivan los que vivirán en los últimos días.
\par 23 Y él me respondió diciendo: Ve, reúne al pueblo y diles que no te buscarán durante cuarenta días.
\par 24 Pero mira, prepara muchos bojes y lleva contigo a Sarea, Dabria, Selemia, Ecano y Asiel, estos cinco que están listos para escribir rápidamente;
\par 25 Y ven acá y encenderé una vela de entendimiento en tu corazón, que no se apagará hasta que se cumplan las cosas que comienzas a escribir.
\par 26 Y cuando hayas terminado, algunas cosas publicarás y otras las mostrarás en secreto a los sabios; mañana a esta hora comenzarás a escribir.
\par 27 Entonces salí tal como él me había ordenado, reuní a todo el pueblo y dije:
\par 28 Oye estas palabras, oh Israel.
\par 29 Nuestros padres al principio fueron extranjeros en Egipto, de donde fueron liberados.
\par 30 Y recibieron la ley de la vida, que ellos no guardaron, y que también vosotros habéis transgredido después de ellos.
\par 31 Entonces la tierra, la tierra de Sión, fue repartida entre vosotros por suertes; pero vuestros padres y vosotros mismos habéis hecho injusticia y no habéis guardado los caminos que el Altísimo os había ordenado.
\par 32 Y como es un juez justo, te quitó a tiempo lo que te había dado.
\par 33 Y ahora estáis aquí vosotros y vuestros hermanos entre vosotros.
\par 34 Por lo tanto, si dominan su propio entendimiento y reforman sus corazones, serán mantenidos con vida y después de la muerte obtendrán misericordia.
\par 35 Porque después de la muerte vendrá el juicio, cuando volveremos a vivir; y entonces serán manifiestos los nombres de los justos y se declararán las obras de los impíos.
\par 36 Por tanto, nadie venga a mí ahora ni me busque durante estos cuarenta días.
\par 37 Entonces tomé a los cinco hombres, tal como él me había ordenado, y salimos al campo y nos quedamos allí.
\par 38 Y al día siguiente, he aquí, una voz me llamó, diciendo: Esdras, abre tu boca y bebe lo que yo te doy de beber.
\par 39 Entonces abrí la boca y he aquí que me alcanzó una copa llena, que estaba como llena de agua, pero de color como de fuego.
\par 40 Y lo tomé y bebí; y cuando hube bebido de él, mi corazón expresó entendimiento, y la sabiduría creció en mi pecho, porque mi espíritu fortaleció mi memoria.
\par 41 Y mi boca se abrió y nunca más se cerró.
\par 42 El Altísimo dio entendimiento a los cinco hombres, y escribieron las maravillosas visiones de la noche que les eran contadas y que no conocían; y estuvieron sentados cuarenta días, y escribieron de día, y de noche comieron pan.
\par 43 En cuanto a mí. Hablé de día, y de noche no mordí la lengua.
\par 44 En cuarenta días escribieron doscientos cuatro libros.
\par 45 Y aconteció que cuando se cumplieron los cuarenta días, el Altísimo habló, diciendo: Lo primero que hayas escrito, publícalo abiertamente, para que lo lean los dignos y los indignos.
\par 46 Pero guarda los setenta últimos, para entregárselos sólo a los sabios del pueblo.
\par 47 Porque en ellos está la fuente del entendimiento, la fuente de la sabiduría y el torrente del conocimiento.
\par 48 Y así lo hice.

\chapter{15}

\par 1 He aquí, habla a oídos de mi pueblo las palabras de profecía que yo pondré en tu boca, dice el Señor:
\par 2 Y haz que se escriban en papel, porque son fieles y verdaderos.
\par 3 No temas las imaginaciones contra ti, no dejes que la incredulidad de aquellos que hablan contra ti te turben.
\par 4 Porque todos los infieles morirán en su infidelidad.
\par 5 He aquí, dice el Señor, traeré plagas sobre el mundo; la espada, el hambre, la muerte y la destrucción.
\par 6 Porque la maldad ha contaminado enormemente toda la tierra, y sus obras dañinas se han cumplido.
\par 7 Por tanto, dice el Señor:
\par 8 Ya no me callaré en cuanto a sus maldades que cometen profanamente, ni los toleraré en aquellas cosas en las que se ejercitan malvadamente. He aquí, la sangre inocente y justa clama a mí, y las almas de los justos se quejan continuamente.
\par 9 Por eso, dice el Señor, ciertamente los vengaré y recibiré toda la sangre inocente de entre ellos.
\par 10 He aquí, mi pueblo es llevado como rebaño al matadero; no permitiré que habiten ahora en la tierra de Egipto.
\par 11 Pero los derrotaré con mano fuerte y brazo extendido, y heriré a Egipto con plagas como antes, y destruiré toda su tierra.
\par 12 Egipto se lamentará y sus cimientos serán heridos con la plaga y el castigo que Dios traerá sobre él.
\par 13 Los que labran la tierra se lamentarán, porque sus semillas se perderán a causa del estallido y del granizo, y con una constelación espantosa.
\par 14 ¡Ay del mundo y de sus habitantes!
\par 15 Porque la espada y su destrucción están cerca, y un pueblo se levantará y peleará contra otro, con espada en la mano.
\par 16 Porque habrá rebeliones entre los hombres y se invadirán unos a otros; No tendrán en cuenta a sus reyes ni a sus príncipes, y el curso de sus acciones quedará en su poder.
\par 17 Si alguno desea entrar en una ciudad, no podrá.
\par 18 Porque a causa de su orgullo las ciudades se estremecerán, las casas serán destruidas y los hombres tendrán miedo.
\par 19 El hombre no tendrá compasión de su prójimo, sino que destruirá sus casas a espada y saqueará sus bienes, por falta de pan y por gran tribulación.
\par 20 He aquí, dice Dios, yo convocaré a todos los reyes de la tierra para que me reverencian, desde donde nace el sol, desde el sur, desde el oriente y desde el Líbano; para volverse unos contra otros, y pagar lo que les han hecho.
\par 21 Como ellos hacen hoy con mis escogidos, así también lo haré yo y les daré la recompensa en su seno. Así dice el Señor Dios;
\par 22 Mi diestra no perdonará a los pecadores, y mi espada no cesará sobre los que derraman sangre inocente sobre la tierra.
\par 23 De su ira salió fuego que consumió los cimientos de la tierra y a los pecadores como paja que se enciende.
\par 24 ¡Ay de los que pecan y no guardan mis mandamientos! dice el Señor.
\par 25 No los perdonaré: idos, hijos, lejos del poder, no profanéis mi santuario.
\par 26 Porque el Señor conoce a todos los que pecan contra Él y, por eso, los entrega a la muerte y a la destrucción.
\par 27 Porque ahora han llegado plagas sobre toda la tierra y vosotros permaneceréis en ellas; porque Dios no os librará, porque habéis pecado contra él.
\par 28 He aquí una visión horrible y su aspecto desde el oriente:
\par 29 De donde saldrán las naciones de los dragones de Arabia con muchos carros, y su multitud será arrastrada como el viento sobre la tierra, para que todos los que los escuchen teman y tiemblen.
\par 30 También los carmanianos, enfurecidos, saldrán como jabalíes del bosque, y vendrán con gran poder, y se enfrentarán a ellos, y devastarán una parte de la tierra de los asirios.
\par 31 Y entonces los dragones tomarán la delantera, recordando su naturaleza; y si se vuelven, conspirando juntos con gran poder para perseguirlos,
\par 32 Entonces éstos se espantarán, callarán por su poder y huirán.
\par 33 Y desde la tierra de Asiria el enemigo los asediará y consumirá a algunos de ellos, y en su ejército habrá temor y pavor, y contienda entre sus reyes.
\par 34 He aquí las nubes desde el oriente y desde el norte hasta el sur, y son muy horribles de ver, llenas de ira y tormenta.
\par 35 Se herirán unos a otros y derribarán sobre la tierra una gran multitud de estrellas, incluso su propia estrella; y la sangre correrá desde la espada hasta el vientre,
\par 36 y estiércol de hombre en la cueva del camello.
\par 37 Y habrá gran temor y temblor sobre la tierra; y los que vean la ira tendrán miedo, y el temblor vendrá sobre ellos.
\par 38 Y entonces vendrán grandes tormentas del sur, y del norte, y otra parte del oeste.
\par 39 Y se levantarán fuertes vientos del oriente y la abrirán; y la nube que levantó con ira, y la estrella que se agitó para atemorizar al viento del este y del oeste, serán destruidas.
\par 40 Las nubes grandes y poderosas se hincharán llenas de ira, y la estrella, para atemorizar a toda la tierra y a sus habitantes; y derramarán sobre todo lugar alto y eminente una estrella horrible,
\par 41 Fuego, granizo, espadas voladoras y muchas aguas, para que todos los campos y todos los ríos se llenen de grandes aguas.
\par 42 Y derribarán las ciudades y las murallas, los montes y las colinas, los árboles del bosque, la hierba de los prados y su trigo.
\par 43 E irán firmemente a Babilonia y la aterrorizarán.
\par 44 Vendrán a ella y la asediarán, derramarán sobre ella la estrella y toda la ira; entonces el polvo y el humo subirán al cielo, y todos los que la rodean lamentarán su presencia.
\par 45 Y los que permanezcan bajo ella servirán a los que la aterrorizaron.
\par 46 Y tú, Asia, que participas de la esperanza de Babilonia y eres la gloria de su persona:
\par 47 ¡Ay de ti, desgraciada, porque te has hecho semejante a ella! y has adornado a tus hijas con fornicación, para agradar y gloriarse de tus amantes, que siempre han deseado fornicar contigo.
\par 48 Tú has seguido a la odiada en todas sus obras e invenciones. Por eso dice Dios:
\par 49 Enviaré sobre ti plagas; viudez, pobreza, hambre, espada y pestilencia, para devastar tus casas con destrucción y muerte.
\par 50 Y la gloria de tu poder se secará como una flor, surgirá el calor que se envía sobre ti.
\par 51 Como pobre mujer serás debilitada por los azotes y como castigada por las heridas, de modo que los valientes y los amantes no podrán recibirte.
\par 52 ¿Ojalá con celos hubiera procedido así contra ti?, dice el Señor,
\par 53 Si no hubieras matado siempre a mis elegidos, exaltando el golpe de tus manos y diciendo sobre sus muertos, cuando estabas ebrio:
\par 54 ¿Exaltas la hermosura de tu rostro?
\par 55 La recompensa de tu fornicación estará en tu seno, por eso recibirás recompensa.
\par 56 Como tú has hecho con mis escogidos, dice el Señor, así hará Dios contigo y te entregará al mal.
\par 57 Tus hijos morirán de hambre, y tú caerás a espada; tus ciudades serán derribadas, y todos los tuyos perecerán a espada en el campo.
\par 58 Los que están en las montañas morirán de hambre, comerán su propia carne y beberán su propia sangre, por hambre de pan y sed de agua.
\par 59 Tú, infeliz, atravesarás el mar y volverás a recibir plagas.
\par 60 Y en el camino se precipitarán sobre la ciudad ociosa, destruirán una parte de tu tierra y consumirán parte de tu gloria, y regresarán a la Babilonia que fue destruida.
\par 61 Y serás derribado por ellos como hojarasca, y ellos te serán como fuego;
\par 62 Y te consumirá a ti y a tus ciudades, tu tierra y tus montañas; todos tus bosques y tus árboles fructíferos quemarán al fuego.
\par 63 Se llevarán cautivos a tus hijos y, mira, lo que tienes lo estropearán y estropearán la belleza de tu rostro.

\chapter{16}

\par 1 ¡Ay de ti, Babilonia y Asia! ¡Ay de vosotros, Egipto y Siria!
\par 2 Cíñete de telas de cilicio y de pelo, llora y llora a tus hijos; porque tu destrucción está cerca.
\par 3 Una espada ha sido enviada sobre ti, ¿y quién podrá hacerla retroceder?
\par 4 Se ha enviado un fuego entre vosotros, ¿y quién podrá apagarlo?
\par 5 Os han sido enviadas plagas, ¿y quién es el que puede ahuyentarlas?
\par 6 ¿Puede alguno ahuyentar al león hambriento en el bosque? ¿O podrá alguno apagar el fuego con hojarasca, cuando ha comenzado a arder?
\par 7 ¿Se puede volver a lanzar la flecha lanzada por un arquero fuerte?
\par 8 El Señor poderoso envía las plagas, ¿y quién puede expulsarlas?
\par 9 Un fuego brotará de su ira, ¿y quién podrá apagarlo?
\par 10 Él arrojará relámpagos, ¿y quién no temerá? Él tronará, ¿y quién no tendrá miedo?
\par 11 El Señor amenazará, ¿y quién no será reducido a polvo en su presencia?
\par 12 La tierra y sus cimientos tiemblan; El mar se levanta con olas de lo profundo, y sus olas se agitan, y también sus peces, delante del Señor y delante de la gloria de su poder.
\par 13 Porque fuerte es su mano derecha que tensa el arco, sus flechas que dispara son agudas y no fallarán cuando comiencen a ser lanzadas hasta los confines del mundo.
\par 14 He aquí, las plagas son enviadas y no volverán más hasta que vengan sobre la tierra.
\par 15 El fuego se enciende y no se apagará hasta que consuma los cimientos de la tierra.
\par 16 Como la flecha lanzada por un poderoso arquero que no regresa hacia atrás, así las plagas que serán enviadas sobre la tierra no volverán más.
\par 17 ¡Ay de mí! ¡Ay de mí! ¿Quién me librará en aquellos días?
\par 18 El principio de dolores y de grandes lutos; el comienzo del hambre y la gran muerte; el comienzo de las guerras, y las potencias tendrán miedo; ¡El comienzo de los males! ¿Qué haré cuando vengan estos males?
\par 19 He aquí, el hambre y la peste, la tribulación y la angustia, son enviados como azotes para la enmienda.
\par 20 Pero a pesar de todo esto no se apartarán de su maldad, ni se acordarán siempre de los azotes.
\par 21 He aquí, las provisiones serán tan baratas en la tierra, que se considerarán en buenas condiciones, y aun así crecerán sobre la tierra los males, la espada, el hambre y una gran confusión.
\par 22 Porque muchos de los habitantes de la tierra perecerán de hambre; y a los otros que escapen del hambre, la espada los destruirá.
\par 23 Y los muertos serán arrojados como estiércol, y no habrá quien los consuele; porque la tierra será asolada y las ciudades serán arrasadas.
\par 24 No quedará ningún hombre que labrará la tierra y la sembrará.
\par 25 Los árboles darán frutos, ¿y quién los recogerá?
\par 26 Las uvas madurarán, ¿y quién las pisará? porque todos los lugares serán desolados de los hombres:
\par 27 De modo que uno deseará ver a otro y oír su voz.
\par 28 Porque de una ciudad quedarán diez y dos del campo, que se esconderán en los espesos bosques y en las hendiduras de las peñas.
\par 29 Como en un huerto de olivos, de cada árbol quedan tres o cuatro aceitunas;
\par 30 O como cuando se recoge una viña, quedan algunos racimos de los que buscan diligentemente en la viña:
\par 31 Así también en aquellos días quedarán tres o cuatro de los que registran sus casas a espada.
\par 32 Y la tierra será devastada, y sus campos envejecerán, y sus caminos y todos sus senderos se llenarán de espinas, porque nadie pasará por ellos.
\par 33 Las vírgenes se lamentarán por no tener novio; las mujeres harán duelo por no tener marido; sus hijas harán duelo, sin tener ayuda.
\par 34 En las guerras sus novios serán destruidos y sus maridos perecerán de hambre.
\par 35 Oíd ahora estas cosas y entendedlas, siervos del Señor.
\par 36 He aquí la palabra del Señor, recíbanla; no crean en los dioses de quienes habló el Señor.
\par 37 He aquí que las plagas se acercan y no ceden.
\par 38 Como cuando una mujer encinta en el noveno mes da a luz a su hijo, a las dos o tres horas de su nacimiento, grandes dolores rodean su vientre, los cuales, cuando nace el niño, no amainan ni un momento.
\par 39 Así también las plagas no tardarán en venir sobre la tierra, y el mundo se lamentará y vendrán dolores sobre él por todas partes.
\par 40 Oh pueblo mío, escucha mi palabra: prepárate para la batalla y en esos males sé como peregrinos sobre la tierra.
\par 41 El que vende, sea como el que huye, y el que compra, como el que va a perder.
\par 42 El que se ocupa de la mercancía, como el que no obtiene provecho de ella, y el que construye, como el que no habita en ella.
\par 43 El que siembra, como si no fuera a cosechar, así también el que planta la viña, como el que no recoge las uvas.
\par 44 Los que se casan, como los que no tendrán hijos; y los que no se casan, como los viudos.
\par 45 Por eso los que trabajan, trabajan en vano.
\par 46 Porque los extranjeros cosecharán sus frutos, saquearán sus bienes, derribarán sus casas y tomarán cautivos a sus hijos, porque en el cautiverio y el hambre tendrán hijos.
\par 47 Y los que ocupan sus mercancías con el robo, más embellecen sus ciudades, sus casas, sus posesiones y sus propias personas.
\par 48 Más me enojaré con ellos por su pecado, dice el Señor.
\par 49 Como la ramera envidia a la mujer honesta y virtuosa:
\par 50 Así aborrecerá la justicia la iniquidad, cuando se engañe, y la acusará en su cara, cuando venga aquel que defienda a aquel que diligentemente investiga todos los pecados de la tierra.
\par 51 Por tanto, no seáis como él ni como sus obras.
\par 52 Porque dentro de poco la iniquidad será quitada de la tierra y la justicia reinará entre vosotros.
\par 53 No diga el pecador que no ha pecado, porque Dios quemará carbones encendidos sobre su cabeza, diciendo delante del Señor Dios y de su gloria: No he pecado.
\par 54 He aquí, el Señor conoce todas las obras de los hombres, sus imaginaciones, sus pensamientos y sus corazones.
\par 55 Que sólo dijo la palabra: Hágase la tierra; y fue hecho: Háganse los cielos; y fue creado.
\par 56 En su palabra fueron hechas las estrellas, y él sabe su número.
\par 57 Él escudriña las profundidades y sus tesoros; Midió el mar y lo que contiene.
\par 58 Él cerró el mar en medio de las aguas, y con su palabra hizo colgar la tierra sobre las aguas.
\par 59 Él extiende los cielos como una bóveda; sobre las aguas la fundó.
\par 60 En el desierto hizo manantiales de agua y estanques en las cimas de los montes, para que de las altas peñas brotaran torrentes que riegan la tierra.
\par 61 Hizo al hombre, puso su corazón en medio del cuerpo y le dio aliento, vida y entendimiento.
\par 62 Sí, y el Espíritu de Dios Todopoderoso, que hizo todas las cosas y escudriña todas las cosas ocultas en los secretos de la tierra,
\par 63 Ciertamente él conoce vuestras invenciones y lo que pensáis en vuestro corazón, incluso aquellos que pecan y quieren ocultar su pecado.
\par 64 Por eso el Señor ha examinado exactamente todas vuestras obras y os avergonzará a todos.
\par 65 Y cuando vuestros pecados sean revelados, seréis avergonzados delante de los hombres, y vuestros propios pecados serán vuestros acusadores en aquel día.
\par 66 ¿Qué haréis? ¿O cómo esconderéis vuestros pecados delante de Dios y sus ángeles?
\par 67 He aquí, Dios mismo es el juez; temedle; dejad vuestros pecados y olvidad vuestras iniquidades, para no entrometeros más en ellos para siempre; así Dios os guiará y os librará de toda angustia.
\par 68 Porque he aquí, la ira ardiente de una gran multitud se ha encendido contra vosotros, y os quitarán a algunos de vosotros, y os alimentarán, estando ociosos, con cosas sacrificadas a los ídolos.
\par 69 Y los que consientan en ello serán objeto de escarnio y de reproche, y pisoteados.
\par 70 Porque habrá en todos los lugares y en las ciudades vecinas una gran rebelión contra los que temen al Señor.
\par 71 Serán como locos, no perdonarán a nadie, pero despojarán y destruirán a los que temen al Señor.
\par 72 Porque desperdiciarán y quitarán sus bienes y los echarán de sus casas.
\par 73 Entonces serán conocidos quiénes son mis elegidos; y serán probados como oro en el fuego.
\par 74 Oíd, amados míos, dice el Señor: He aquí que los días de angustia están cerca, pero yo os libraré de ellos.
\par 75 No temáis ni dudéis; porque Dios es tu guía,
\par 76 Y el guía de los que guardan mis mandamientos y preceptos, dice el Señor Dios: No dejéis que vuestros pecados os pesen, ni vuestras iniquidades se enaltezcan.
\par 77 ¡Ay de aquellos que están atados por sus pecados y cubiertos de iniquidades, como un campo cubierto de arbustos y su camino cubierto de espinas, para que nadie pueda pasar!
\par 78 Lo dejan desnudo y lo arrojan al fuego para que se consuma en él.

\end{document}