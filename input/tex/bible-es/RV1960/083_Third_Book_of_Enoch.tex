\begin{document}

\title{Tercer Libro de Enoc}

\chapter{1}

\par \textit{INTRODUCCIÓN: R. Ismael asciende al cielo para contemplar la visión del Merkaba y es confiado a Metatrón}

\par \textit{Y ENOC CAMINÓ CON DIOS: Y NO FUE ; PORQUE DIOS SE LO LLEVÓ (Gen. v.4)}

\par 1 Rabí Ismael dijo: Cuando subí a lo alto para contemplar la visión del Merkaba y entré en las seis salas, una dentro de la otra:

\par 2 tan pronto como llegué a la puerta del séptimo Salón me quedé quieto en oración ante el Santo, bendito sea, y, levantando mis ojos en lo alto (es decir, hacia la Divina Majestad), dije:

\par 3 «Señor del universo, te ruego que el mérito de Aarón, el hijo de Amram, el amante de la paz y el buscador de la paz, que recibió la corona del sacerdocio de tu gloria en el monte del Sinaí, sea válido. por mí en esta hora, para que Qafsiel, el príncipe, y los ángeles con él no tengan poder sobre mí ni me arrojen del cielo».

\par 4 Inmediatamente el Santo, bendito sea, envió a mí Metatrón, su Siervo ('Ebed), el ángel, el Príncipe de la Presencia, y él, extendiendo sus alas, con gran alegría vino a mi encuentro para salvarme. yo de su mano.

\par 5 Y me tomó de la mano a la vista de ellos, diciéndome: «Entra en paz ante el alto y exaltado Rey y contempla la imagen del Merkaba».

\par 6 Luego entré al séptimo Salón, y él me llevó al campamento(s) de Shekina y me colocó ante el Santo, bendito sea Él, para contemplar el Merkaba.

\par 7 Tan pronto como los príncipes de Merkaba y los serafines llameantes me vieron, fijaron sus ojos en mí. Al instante me invadió un temblor y un estremecimiento y caí al suelo y quedé paralizado por la imagen radiante de sus ojos y el aspecto espléndido de sus rostros; hasta que el Santo, bendito sea, los reprendió, diciendo:

\par 8 «¡Mis siervos, mis Serafines, mis Querubines y mis Ofannimos! ¡Cúbrete los ojos delante de Ismael, hijo mío, amigo mío, amado mío y gloria mía, para que no tiemble ni se estremezca!

\par 9 Inmediatamente vino Metatrón, el Príncipe de la Presencia, y restauró mi espíritu y me puso sobre mis pies.

\par 10 Después de ese (momento) no hubo en mí fuerzas suficientes para decir un cántico ante el Trono de Gloria del Rey glorioso, el más poderoso de todos los reyes, el más excelso de todos los príncipes, hasta después de que pasó la hora.

\par 11 Después de una hora (había pasado), el Santo, bendito sea, me abrió las puertas de Shekina, las puertas de la Paz, las puertas de la Sabiduría, las puertas de la Fuerza, las puertas del Poder, las puertas del Habla. (Dibbur)> las puertas de Song, las puertas de Qedushsha, las puertas de Chant.

\par 12 E iluminó mis ojos y mi corazón con palabras de salmos, cánticos, alabanza, exaltación, acción de gracias, exaltación, glorificación, himno y panegírico. Y cuando abrí mi boca, pronunciando una canción ante el Santo, bendito sea Él, el Santo Chayyoth debajo y sobre el Trono de Gloria respondió y dijo: «santo» y «¡bendita sea la gloria de yhwh desde su lugar!» (es decir, cantó el Qedushsha).

\chapter{2}

\par \textit{Las clases más altas de ángeles hacen preguntas sobre R. Ismael, las cuales son respondidas por Metatrón}

\par 1 R. Ismael dijo: En aquella hora las águilas de la Merkaba, los 'Ophannim flamígeros y los Serafines del fuego consumidor preguntaron a Metatrón, diciéndole:

\par 2 «¡Juventud! ¿Por qué permites que alguien nacido de mujer entre y contemple el Merkaba? ¿De qué nación, de qué tribu es éste? ¿Cuál es su carácter?

\par 3 Metatrón respondió y les dijo: «De la nación de Israel, a quien el Santo, bendito sea, escogió para su pueblo, de entre setenta lenguas (naciones), de la tribu de Leví, a quien apartó como contribución a su nombre y de la simiente de Aarón a quien el Santo, bendito sea, eligió para su siervo y le puso la corona del sacerdocio en el Sinaí».

\par 4 Inmediatamente hablaron y dijeron: «En verdad, éste es digno de contemplar el Merkaba». Y dijeron: «¡Feliz el pueblo que se encuentra en tal caso!»

\chapter{3}

\par \textit{Metatrón tiene 70 nombres, pero Dios lo llama 'Juventud'}

\par 1 R. Ismael dijo: En aquella hora le pregunté a Metatrón, el ángel, el Príncipe de la Presencia: «¿Cuál es tu nombre?»

\par 2 Él me respondió: «Tengo setenta nombres, correspondientes a las setenta lenguas del mundo y todos ellos están basados ​​en el nombre Metatrón, ángel de la Presencia; pero mi Rey me llama 'Joven' (Na'ar)».

\chapter{4}

\par \textit{Metatrón es idéntico a Enoc, quien fue trasladado al cielo en el momento del Diluvio}

\par 1 R. Ismael dijo: Le pregunté a Metatrón y le dije: «¿Por qué te llaman con el nombre de tu Creador, con setenta nombres? Tú eres mayor que todos los príncipes, más alto que todos los ángeles, más amado que todos los siervos, más honrado que todos los poderosos en realeza, grandeza y gloria: ¿por qué te llaman 'Joven' en las alturas?

\par 2 Él respondió y me dijo; «Porque yo soy Enoc, el hijo de Jared.

\par 3 Porque cuando la generación del diluvio pecó y se confundió en sus obras, diciendo a Dios: «Apártate de nosotros, porque no deseamos el conocimiento de tus caminos», entonces el Santo, bendito sea, me sacó de su medio para ser testigo contra ellos en las alturas de los cielos, a todos los habitantes del mundo, para que no digan: 'El Misericordioso es cruel'.

\par 4 ADEL: ¿Qué pecaron todas esas multitudes, sus esposas, sus hijos y sus hijas, sus caballos, sus mulas y su ganado y sus propiedades, y todas las aves del mundo, todas las cuales el Santo, bendito sea? , destruidos del mundo junto con ellos en las aguas del diluvio? Tampoco puede decir: '¿Y si la generación del diluvio pecó? las bestias y las aves, ¿qué habían pecado para perecer con ellas? BC: '¿Qué pecados habían cometido todas esas multitudes? O si pecaron, ¿qué pecaron sus hijos y sus hijas, sus mulas y sus ganados? ¿Y también todos los animales, domésticos y salvajes, y las aves del mundo que Dios destruyó del mundo?

\par 5 Por lo tanto, el Santo, bendito sea, me levantó durante su vida ante sus ojos para ser un testigo contra ellos en el mundo futuro. Y el Santo, bendito sea, me asignó como príncipe y gobernante entre los ángeles ministradores.

\par 6 En esa hora, tres de los ángeles ministradores, 'uzza, 'azza y 'azzael, vinieron y presentaron acusaciones contra mí en las alturas de los cielos, diciendo ante el Santo, bendito sea: «¿No dijeron los Ancianos (primero Unos) justamente ante Ti: '¡No creéis al hombre!', el Santo, bendito sea, respondió y les dijo: «Yo he hecho y llevaré, sí, llevaré y entregaré».

\par 7 Tan pronto como me vieron, dijeron delante de Él: «¡Señor del universo! ¿Qué es éste que debe ascender a lo alto de las alturas? ¿No es él uno de entre los hijos de [los hijos de] los que perecieron en los días del Diluvio? '¿Qué hace en la Raqia'?

\par 8 Nuevamente el Santo, bendito sea, respondió y les dijo: «¿Qué sois para que entréis y habléis en mi presencia? Me deleito en éste más que en todos vosotros, y por eso será príncipe y gobernante sobre vosotros en las alturas.

\par 9 Inmediatamente todos se levantaron y salieron a mi encuentro, se postraron ante mí y dijeron: «Feliz tú y feliz tu padre porque tu Creador te favorece».

\par 10 Y como soy pequeño y joven entre ellos de días, meses y años, por eso me llaman «Joven» (Na'ar).

\chapter{5}

\par \textit{La idolatría de la generación de Enósh hace que Dios elimine la Shekina de la tierra. La idolatría inspirada por 'Azza, 'Uzza y 'Azziel}

\par 1 R. Ismael dijo: Metatrón, el Príncipe de la Presencia, me dijo: Desde el día en que el Santo, bendito sea, expulsó al primer Adán del Jardín del Edén (en adelante), Shekina habitaba en un Kerub bajo el Árbol de la Vida.

\par 2 Y los ángeles ministradores se reunían y descendían del cielo en grupos, de la Raqia en compañías y del cielo en campamentos, para hacer Su voluntad en todo el mundo.

\par 3 Y el primer hombre y su generación estaban sentados fuera de la puerta del Jardín para contemplar la apariencia radiante de la Shekina.

\par 4 Porque el esplendor de la Shekina atravesó el mundo de un extremo al otro (con un esplendor) 365.000 veces (el) del globo solar. Y todo aquel que hacía uso del esplendor de la Shekina, sobre él no descansaban moscas ni mosquitos, ni enfermaba ni padecía dolor alguno. Ningún demonio tuvo poder sobre él ni pudo dañarlo.

\par 5 Cuando el Santo, bendito sea, salió y entró: del Jardín al Edén, del Edén al Jardín, del Jardín a Raqia y de Raqia al Jardín del Edén, entonces todos contemplaron el esplendor. de Su Shekina y no resultaron heridos;

\par 6 hasta el tiempo de la generación de Enós, que era el jefe de todos los idólatras del mundo.

\par 7 ¿Y qué hizo la generación de Enós? Iban de un extremo al otro del mundo, y cada uno traía plata, oro, piedras preciosas y perlas en montones como montañas y colinas, haciendo con ellas ídolos por todo el mundo. Y erigieron ídolos en todos los rincones del mundo: el tamaño de cada ídolo era parasangs.

\par 8 Y bajaron el sol, la luna, los planetas y las constelaciones, y los colocaron delante de los ídolos a su derecha y a su izquierda, para que los atendieran como asisten al Santo, bendito sea, como es. escrito: «Y todo el ejército del cielo estaba junto a él, a su derecha y a su izquierda».

\par 9 ¿Qué poder había en ellos para poder derribarlos? No habrían podido derribarlos si no fuera por 'uzza, 'azza y 'azziel, quienes les enseñaron hechicerías mediante las cuales los derribaron y utilizaron.

\par 10 En aquel tiempo los ángeles ministradores presentaron acusaciones (contra ellos) ante el Santo, bendito sea, diciendo delante de él: «¡Amo del mundo! ¿Qué tienes que ver con los hijos de los hombres? Como está escrito '¿Qué es el hombre (Enós) para que te acuerdes de él?' Aquí no está escrito 'Mah Adam', sino 'Mah Enosh', porque él (Enosh) es el jefe de los adoradores de ídolos.

\par 11 ¿Por qué has dejado [ADE: lo más alto de los cielos, la morada de tu glorioso Nombre, y el alto y exaltado Trono en 'Arabotk en lo alto] [B: los 'Araboth Raqia que están llenos de tu gloria, poderoso y alto por igual, y el Trono alto y exaltado en el 'Araboth Raqia en lo más alto] [CL: el más alto de los cielos altos que están llenos de la majestad de tu gloria y son altos, elevados y exaltados, y los altos y Trono exaltado en el Raqia 'Araboth en lo alto] y te has ido y moras con los hijos de los hombres que adoran ídolos y te igualan a los ídolos.

\par 12 Ahora tú estás en la tierra y los ídolos también. ¿Qué tienes que ver con los habitantes de la tierra que adoran ídolos?

\par 13 Inmediatamente el Santo, bendito sea, levantó Su Shekina de la tierra, de en medio de ellos.

\par 14 En ese momento llegaron los ángeles ministradores, las tropas de ejércitos y los ejércitos de 'Araboth en mil campamentos y diez mil ejércitos: trajeron trompetas y tomaron los cuernos en sus manos y rodearon a la Shekina con toda clase de canciones. Y ascendió a los altos cielos, como está escrito: «Dios ha subido con voz de mando, el Señor con sonido de trompeta».

\chapter{6}

\par \textit{Enoc fue elevado al cielo junto con la Shekina. Las protestas de los ángeles respondidas por Dios}

\par 1 R. Ismael dijo: Metatrón, el Ángel, el Príncipe de la Presencia, me dijo: Cuando el Santo, bendito sea, deseó elevarme a lo alto, primero envió a 'Anafid H (H= Tetrqgrgmmatfm ), el Príncipe, y me tomó de en medio de ellos a la vista de ellos y me llevó con gran gloria sobre un carro de fuego con caballos de fuego, servidores de la gloria. Y él me elevó a los altos cielos junto con la Shekina.

\par 2 Tan pronto como llegué a los altos cielos, los Santos Chayyoth, los 'Ophannim, los Serafines, los Kerubim, las Ruedas de los Merkaba (los Galgallim) y los ministros del fuego consumidor, percibieron mi olor desde la distancia. de 365.000 miríadas de parasangas, dijo: [R: «¿Qué olor de uno nacido de mujer y qué sabor de una gota blanca (es ésta) que asciende a lo alto, y (he aquí, él es simplemente) un mosquito entre aquellos que 'dividen' ¿llamas' (de fuego)?»] [B: «¿Qué es uno nacido de mujer entre (entre) nosotros? El sabor de una gota blanca que sube a los altos cielos para ministrar entre los que 'dividen llamas de fuego'».] [CDEL: «¿Qué olor de mujer es éste y qué sabor de una gota blanca que sube a lo alto?» cielos para ministrar entre divisores de llamas.]

\par 3 El Santo, bendito sea, respondió y les habló: «¡Mis siervos, mis ejércitos, mis Kerubim, mis 'Ophannim, mis Serafines! ¡No os enojéis por esto! Puesto que todos los hijos de los hombres me han negado a mí y a mi gran Reino y se han ido adorando ídolos, he quitado mi Shekina de entre ellos y la he elevado a lo alto. Pero éste a quien he tomado de entre ellos es un elegido entre (los habitantes de) el mundo y es igual a todos ellos en fe, rectitud y perfección de obra y lo he tomado como tributo de mi mundo. bajo todos los cielos».

\chapter{7}

\par \textit{Enoc elevado sobre las alas de la Shekina al lugar del Trono, el Merkaba y las huestes angelicales}

\par 1 R. Ismael dijo: Metatrón, el Ángel, el Príncipe de la Presencia, me dijo: Cuando el Santo, bendito sea, me sacó de la generación del Diluvio, me levantó en las alas del viento de Shekina al cielo más alto y me llevó a los grandes palacios del 'Araboth Raqia en lo alto, donde están el glorioso Trono de Shekina, el Merkaba, las tropas de la ira, los ejércitos de la vehemencia, los Shin'anim ardientes, los Querubines llameantes y los 'Ophannim ardientes, los sirvientes llameantes, los Jashmallim resplandecientes y los Serafines relámpagos. Y me colocó (allí) para asistir al Trono de Gloria día tras día.

\chapter{8}

\par \textit{Las puertas (de los tesoros del cielo) se abrieron a Metatrón}

\par 1 R. Ismael dijo: Metatrón, el Príncipe de la Presencia, me dijo: Antes de designarme para asistir al Trono de Gloria, el Santo, bendito sea, me abrió

\par trescientas mil puertas de la comprensión

\par trescientas mil puertas de la Sutileza

\par trescientas mil puertas de la Vida

\par trescientas mil puertas de 'gracia y bondad amorosa'

\par trescientas mil puertas del amor

\par trescientas mil puertas de Tora

\par trescientas mil puertas de mansedumbre

\par trescientas mil puertas de mantenimiento

\par trescientas mil puertas de la misericordia

\par trescientas mil puertas del miedo al cielo.

\par 2 En aquella hora el Santo, bendito sea, añadió en mí sabiduría a sabiduría, entendimiento a entendimiento, sutileza a sutileza, conocimiento a conocimiento, misericordia a misericordia, instrucción a instrucción, amor a amor, bondad a amor. -bondad, bondad con bondad, mansedumbre con mansedumbre, poder con poder, fuerza con fuerza, poder con poder, brillo con brillo, belleza con belleza, esplendor con esplendor, y fui honrado y adornado con todas estas cosas buenas y dignas de alabanza más que todos los hijos del cielo.


\chapter{9}

\par \textit{Enoc recibe bendiciones del Altísimo y es adornado con atributos angelicales}

\par 1 R. Ismael dijo: Metatrón, el Príncipe de la Presencia, me dijo: Después de todas estas cosas, el Santo, bendito sea, puso Su mano sobre mí y me bendijo con1 bendiciones.

\par 2 Y fui resucitado y engrandecido al tamaño del largo y ancho del mundo.

\par 3 Y me hizo crecer alas a cada lado. Y cada ala era como el mundo entero.

\par 4 Y fijó en mí 365 ojos: cada ojo era como una gran lumbrera.

\par 5 Y no dejó ningún tipo de esplendor, brillo, resplandor, belleza en todas las luces del universo que no haya fijado en mí.

\chapter{10}

\par \textit{Dios coloca a Metatrón en un trono a la puerta del séptimo Salón y anuncia a través del Heraldo, que Metatrón de ahora en adelante es el representante de Dios y gobernante sobre todos los príncipes de los reinos y todos los hijos del cielo, excepto los ocho altos príncipes. llamado YHWH por el nombre de su Rey}

\par 1 R. Ismael dijo: Metatrón, el Príncipe de la Presencia, me dijo: Todas estas cosas el Santo, bendito sea, hizo para mí: Él me hizo un Trono, similar al Trono de Gloria. Y extendió sobre mí una cortina de esplendor y apariencia brillante, de hermosura, gracia y misericordia, semejante a la cortina del Trono de Gloria; y en él estaban fijadas todas las clases de luces del universo.

\par 2 Y lo colocó a la puerta del séptimo salón y me sentó en él.

\par 3 Y el heraldo salió por todos los cielos, diciendo: Éste es Metatrón, mi siervo. Lo he constituido príncipe y gobernante sobre todos los príncipes de mis reinos y sobre todos los hijos del cielo, excepto los ocho grandes príncipes, los honrados y reverenciados que se llaman YHWH, por el nombre de su Rey.

\par 4 Y todo ángel y todo príncipe que tenga una palabra que decir en mi presencia (ante mí), irá a su presencia (delante de él) y le hablará (en su lugar).

\par 5 Y guardad y cumplid cada mandamiento que os diga en mi nombre. Porque al Príncipe de la Sabiduría y al Príncipe del Entendimiento le he encomendado instruirlo en la sabiduría de las cosas celestiales y de las cosas terrenas, en la sabiduría de este mundo y del mundo venidero.

\par 6 Además, lo he puesto sobre todos los tesoros de los palapes de Araboth y sobre todas las reservas de vida que hay en las alturas.

\chapter{11}

\par \textit{Dios revela todos los misterios y secretos a Metatrón}

\par 1 R. Ismael dijo: Metatrón, el ángel, el Príncipe de la Presencia, me dijo: De ahora en adelante el Santo, bendito sea, me reveló todos los misterios de la Torá y todos los secretos de la sabiduría y todas las profundidades. de la Ley Perfecta; y los pensamientos del corazón de todos los seres vivientes y todos los secretos del universo y todos los secretos de la Creación me fueron revelados tal como se revelan al Hacedor de la Creación.

\par 2 Y miré atentamente para contemplar los secretos de la profundidad y el maravilloso misterio. [ABL: Antes de que un hombre pensara en secreto, lo vi y antes de que un hombre hiciera algo, lo contemplé.] [C: Antes de que un hombre pensara, sabía lo que había en su pensamiento.]

\par 3 [ABL: Y no había nada en lo alto ni en lo profundo del mundo escondido de mí.] [C: Y no había nada arriba en lo alto ni abajo en lo profundo escondido de mí.]

\chapter{12}

\par \textit{Dios viste a Metatrón con un manto de gloria, pone una corona real en su cabeza y lo llama «el YHWH Menor»}

\par 1 R. Ismael dijo: Metatrón, el Príncipe de la Presencia, me dijo: A causa del amor con el que el Santo, bendito sea, me amó más que a todos los hijos del cielo, Él me hizo un vestido. de gloria sobre la cual estaban fijadas toda clase de luces, y Él me vistió de ella.

\par 2 Y me hizo un manto de honor en el que estaban fijados toda clase de belleza, esplendor, brillo y majestad.

\par 3 Y me hizo una corona real en la que estaban fijadas cuarenta y nueve piedras preciosas, semejantes a la luz del globo solar.

\par 4 Porque su esplendor se extendió por las cuatro partes del 'Araboth Raqia, y en (a través de) los siete cielos, y en las cuatro partes del mundo. Y me lo puso en la cabeza.

\par 5 Y me llamó Yhwh menor delante de toda su casa celestial; como está escrito: «Porque mi nombre está en él».

\chapter{13}

\par \textit{Dios escribe con un estilo llameante en la corona de Metatrón las letras cósmicas mediante las cuales se crearon el cielo y la tierra}

\par 1 R. Ismael dijo: Metatrón, el ángel, el Príncipe de la Presencia, la Gloria de todos los cielos, me dijo: Por el gran amor y misericordia con que el Santo, bendito sea, me amó y me estimó. Más que todos los hijos del cielo, escribió con su dedo con estilo llameante sobre la corona de mi cabeza las letras con las que fueron creados el cielo y la tierra, los mares y los ríos, las montañas y los cerros, los planetas y las constelaciones, los relámpagos. , los vientos, los terremotos y las voces (truenos), la nieve y el granizo, la tempestad y la tempestad; las letras mediante las cuales fueron creadas todas las necesidades del mundo y todos los órdenes de la Creación.

\par 2 Y cada carta enviada una y otra vez como relámpagos, una y otra vez como antorchas, una y otra vez como llamas de fuego, una y otra vez (rayos) como [como] la salida del sol y la luna y los planetas.

\chapter{14}

\par \textit{Todos los príncipes más altos, los ángeles elementales y los ángeles planetarios y sidéricos temen y tiemblan al ver a Metatrón coronado}

\par 1 R. Ismael dijo: Metatrón, el Ángel, el Príncipe de la Presencia, me dijo: Cuando el Santo, bendito sea, puso esta corona en mi cabeza, (entonces) temblaron ante mí todos los Príncipes de los Reinos. que están en la altura de Araboth Raqia y todas las huestes de cada cielo; e incluso los príncipes (de) los 'Elim, los príncipes (de) los 'Er'ellim y los príncipes (de) los Tafsarim, que son mayores que todos los ángeles ministradores que ministran ante el Trono de Gloria, temblaron, temieron y temblaron delante de mí cuando me vieron.

\par 2 Incluso Sammael, el Príncipe de los Acusadores, que es más grande que todos los príncipes de los reinos en lo alto, temió y tembló ante mí.

\par 3 Y también el ángel del fuego, el ángel del granizo, el ángel del viento, el ángel del relámpago, el ángel de la ira, el ángel del trueno y el ángel de la nieve, y el ángel de la lluvia; y el ángel del día, y el ángel de la noche, y el ángel del sol y el ángel de la luna y el ángel de los planetas y el ángel de las constelaciones que gobiernan el mundo bajo sus manos, temieron y temblaron y se asustaron delante de mí al verme.

\par 4 Estos son los nombres de los gobernantes del mundo: Gabriel, el ángel del fuego, Baradiel, el ángel del granizo, Ruchiel, quien está designado sobre el viento, Baraqiel, quien está designado sobre los relámpagos, Zaamiel, quien Zi'iel que está designado sobre la conmoción, Za'aphiel que está designado sobre el viento de tormenta, Ra'amiel que está designado sobre los truenos, Ra'ashiel que Shalgiel está designado sobre la nieve, Matariel está designado sobre la lluvia, Simshiel está designado sobre el día, Lailiel está designado sobre la noche, Galgalliel está designado sobre el globo del sol, ' Ophanniel, quien está designado sobre el globo de la luna, Kokbiel, quien está designado sobre los planetas, Rahatiel, quien está designado sobre las constelaciones.

\par 5 Y todos, al verme, se postraron. Y no podían contemplarme a causa de la majestuosa gloria y la hermosura de la aparición de la luz resplandeciente de la corona de gloria sobre mi cabeza.

\chapter{15}

\par \textit{Metatrón transformado en fuego}

\par 1 R. Ismael dijo: Metatrón, el ángel, el Príncipe de la Presencia, la Gloria de todos los cielos, me dijo: Tan pronto como el Santo, bendito sea, tomó mi (Su) servicio para asistir al Trono de la Gloria y las Ruedas (Galgallim) del Merkaba y las necesidades de Shekina, inmediatamente mi carne se transformó en llamas, mis tendones en fuego llameante, mis huesos en brasas de enebro ardiente, la luz de mis párpados en esplendor de relámpagos. , mis globos oculares en tizones, el cabello de mi cabeza en llamas ardientes, todos mis miembros en alas de fuego ardiente y todo mi cuerpo en fuego resplandeciente.

\par 2 Y a mi derecha había divisiones de llamas de fuego, a mi izquierda ardían tizones, a mi alrededor soplaban vientos tormentosos y tempestades y delante y detrás de mí rugían truenos y terremotos.

\chapter{15b}

\par \textit{Suma que ocurre en B y L}

\par 1 [B: R. Ismael dijo: Me dijo Metatrón, el Príncipe de la Presencia y el príncipe sobre todos los príncipes, y está delante] [L: Metatrón, él es el príncipe sobre todos los príncipes, y está ante ] Aquel que es mayor que todos los Elohim. Y entra bajo el Trono de Gloria. Y tiene en lo alto un gran tabernáculo de luz. Y él saca el fuego de la sordera y lo pone en los oídos del Santo Chayyoth, para que no escuchen la voz de la Palabra (Dibbur) que sale de la boca de la Divina Majestad.

\par 2 Y cuando Moisés ascendió a lo alto, ayunó hasta que le fueron abiertas las habitaciones del Jashmal; y él [B: vio el corazón dentro del corazón del León] [L: vio que era blanco como el corazón del León] y vio las innumerables compañías de las huestes a su alrededor. Y quisieron quemarlo. Pero Moisés oró pidiendo misericordia, primero para Israel y después para sí mismo; y el que está sentado en el Merkaba abrió las ventanas que están sobre las cabezas de los querubines. Y una hueste de abogados, y con ellos el Príncipe de la Presencia, Metatrón, salió al encuentro de Moisés. Y tomaron las oraciones de Israel y las pusieron como una corona sobre la cabeza del Santo, bendito sea.

\par 3 Y ellos dijeron: «Escucha, oh Israel; el Señor nuestro Dios es un solo Señor» [B: y su rostro brilló y se regocijó sobre Shekina] [L: y el rostro de Shekina brilló y se regocijó] y le dijeron a Metatrón: «¿Qué son estos? ¿Y a quién le dan todo este honor y gloria?» Y ellos respondieron: «Al Glorioso Señor de Israel». Y hablaron: [B: «Oye, Israel: el Señor, nuestro Dios, el Señor uno es. A quien le será dada abundancia de honor y majestad sino a Ti YHWH, la Divina Majestad, el Rey, vivo y eterno». ] [L: «YHWH el Vivo y Eterno». ]

\par 4 En ese momento habló Akatriel Yah Yehod Sebaoth y le dijo a Metatrón, el Príncipe de la Presencia: «Que ninguna oración que él haga delante de mí regrese (a él) vacía. Escucha su oración y cumple su deseo, ya sea grande o pequeño».

\par 5 Inmediatamente Metatrón, el Príncipe de la Presencia, dijo a Moisés: «¡Hijo de Amram! No temas, porque ahora Dios se deleita en ti. Y pide tu deseo de Gloria y Majestad. Porque tu rostro brilla de un extremo al otro del mundo». Pero Moisés le respondió: «(Temo) traer sobre mí mismo culpa». Metatrón le dijo: «Recibe las cartas del juramento, en (por) el cual no se rompe el pacto» (lo que excluye cualquier incumplimiento del pacto),

\chapter{16}

\par \textit{Metatrón se despojó de su privilegio de presidir su propio Trono debido al malentendido de Acher al tomarlo por un segundo Poder Divino}

\par 1 R. Ismael dijo: Metatrón, el Ángel, el Príncipe de la Presencia, la Gloria de todo el cielo, me dijo: Al principio estaba sentado en un gran Trono a la puerta del Séptimo Salón; y yo estaba juzgando a los hijos del cielo, a la familia en lo alto, por la autoridad del Santo, bendito sea. Y dividí la Grandeza, la Realeza, la Dignidad, el Gobierno, el Honor y la Alabanza, y la Diadema y la Corona de Gloria entre todos los príncipes de los reinos, mientras yo presidía (lit. sentado) en la Corte Celestial (Yeshiba), y los príncipes de los reinos estaban parados delante de mí, a mi derecha y a mi izquierda, por la autoridad del Santo, bendito sea.

\par 2 Pero cuando Acher llegó a contemplar la visión del Merkaba y fijó sus ojos en mí, tuvo miedo y tembló ante mí y su alma tuvo miedo incluso de apartarse de él, a causa del miedo, el horror y el temor hacia mí, cuando él me vio sentado en un trono como un rey con todos los ángeles ministradores de pie a mi lado como mis siervos y todos los príncipes de los reinos adornados con coronas rodeándome:

\par 3 en ese momento abrió la boca y dijo: «¡En efecto, hay dos Poderes Divinos en el cielo!»

\par 4 Inmediatamente Bath Qol (la Voz Divina) salió del cielo delante de la Shekina y dijo: «¡Regresad, hijos descarriados, excepto Acher!»

\par 5 Entonces vino 'Aniyel, el Príncipe, el honrado, glorificado, amado, maravilloso, reverenciado y temible, por encargo del Santo, bendito sea Él, y me dio sesenta golpes con latigazos de fuego y me hizo pararme sobre mi pies.

\chapter{17}

\par \textit{Los príncipes de los siete cielos, del sol y la luna, los planetas y las constelaciones y sus séquitos de ángeles}

\par 1 R. Ismael dijo: Metatrón, el ángel, el Príncipe de la Presencia, la gloria de todos los cielos, me dijo: Siete (son los) príncipes, los grandes, hermosos, reverenciados, maravillosos y honrados que son designados sobre los siete cielos. Y estos son ellos: [A: MIKAEL, GABRIEL, SHATQIEL, SHACHAQIEL, BAKARIEL, BADARIEL, PACHRIEL.] [D: MIKAEL y GABRIEL, SHATQIEL y BARADIEL y SHACHAQIEL y BARAQIEL y SIDRIEL. ]

\par 2 Y cada uno de ellos es el príncipe del ejército del cielo. Y cada uno de ellos está acompañado por 496.000 miríadas de ángeles ministradores.

\par 3 Miguel, el gran príncipe, está designado sobre el séptimo cielo, el más alto, que está en el Araboth.

Gabriel, el príncipe del ejército, es designado sobre el sexto cielo que está en Makón.

Shataqiel, príncipe del ejército, es designado sobre el quinto cielo que está en Ma'on.

Shahaqi'el, príncipe del ejército, es designado sobre el cuarto cielo que está en Zebul.

Badariel, príncipe del ejército, es designado sobre el tercer cielo que está en Shehaqim.

Barakiel, príncipe del ejército, es designado sobre el segundo cielo que está en la altura de (Merom) Raqia.

Pazriel, príncipe de los ejércitos, es nombrado sobre el primer cielo que está en Wilon, que está en Shamayim.

\par 4 Debajo de ellos está Galgalliel, el príncipe que está designado sobre el globo (galgal) del sol, y con él están los ángeles grandes y honorables que mueven el sol en Raqia.

\par 5 Debajo de ellos está 'Ophanniel, el príncipe que está situado sobre el globo ('ophan) de la luna. Y con él están los ángeles que mueven el globo lunar mil parasangs cada noche en el momento en que la luna se encuentra en el Este en su punto de inflexión. ¿Y cuándo está la luna en el Este en su punto de inflexión? Respuesta: el día quince de cada mes.

\par 6 Bajo ellos está Rahatiel, el príncipe encargado de las constelaciones. Y lo acompañan 72 grandes y honrados ángeles. ¿Y por qué se llama Rahatiel? Porque él hace que las estrellas corran (marhit) en sus órbitas y recorra 339 mil parasangas cada noche de Este a Oeste y de Oeste a Este. Porque el Santo, bendito sea, ha hecho una tienda para todos ellos, para el sol, la luna, los planetas y las estrellas en las que viajan de noche de Occidente a Oriente.

\par 7 Bajo ellos está Kokbiel, el príncipe que está designado sobre todos los planetas. Y con él están 365.000 miríadas de ángeles ministradores, grandes y honorables que mueven los planetas de ciudad en ciudad y de provincia en provincia en la Raqia de los cielos.

\par 8 Y sobre ellos están setenta y dos príncipes de reinos en las alturas que corresponden a las lenguas del mundo. Y todos ellos están coronados con coronas reales y vestidos con vestiduras reales y envueltos en mantos reales. Y todos ellos van montados en caballos reales y llevan cetros reales en sus manos. Y delante de cada uno de ellos cuando viaja en Raqia, los sirvientes reales corren con gran gloria y majestad [A: así como en la tierra ellos (príncipes) viajan en carros con jinetes y grandes ejércitos y en gloria y grandeza con alabanza, canto y honor. ] [D: y delante de cada uno de ellos, cuando viajan en Raqia', corren grandes ejércitos, tal como (la costumbre es) en la tierra, con carro(s), en gloria y grandeza, alabanza, canción y honor. ]


\chapter{18}

\par \textit{El orden de los rangos de los ángeles y el homenaje que reciben los rangos superiores de los inferiores}

\par 1 R. Ismael dijo: Metatrón, el Ángel, el Príncipe de la Presencia, la gloria de todo el cielo, me dijo: Los ángeles del primer cielo, cuando ven a su príncipe, desmontan de sus caballos. y caen de bruces.

Y el príncipe del primer cielo, cuando ve al príncipe del segundo cielo, desmonta, se quita la corona de gloria de su cabeza y cae de bruces.

Y el príncipe del segundo cielo, cuando ve al príncipe del tercer cielo, se quita la corona de gloria de su cabeza y cae de bruces.

Y el príncipe del tercer cielo, cuando ve al príncipe del cuarto cielo, se quita la corona de gloria de su cabeza y cae de bruces.

Y el príncipe del cuarto cielo, cuando ve al príncipe del quinto cielo, se quita la corona de gloria de su cabeza y cae de bruces.

Y el príncipe del quinto cielo, cuando ve al príncipe del sexto cielo, se quita la corona de gloria de su cabeza y cae de bruces.

Y el príncipe del sexto cielo, cuando ve al príncipe del séptimo cielo, se quita la corona de gloria de su cabeza y cae de bruces.

\par 2 Y el príncipe del séptimo cielo, cuando ve a los setenta y dos príncipes de los reinos, se quita la corona de gloria de su cabeza y cae de bruces.

\par Y los setenta y dos príncipes de reinos, cuando ven a los porteros del primer salón en el 'Araboth Raqia en lo más alto, se quitan la corona real de sus cabezas y caen de bruces.

\par 3 Y los porteros de la primera sala, cuando ven a los porteros de la segunda sala, se quitan la corona de gloria de sus cabezas y caen de bruces.

Y los porteros de la segunda sala, cuando ven a los porteros de la tercera sala, se quitan la corona de gloria de sus cabezas y caen de bruces.

Y los porteros de la tercera sala, cuando ven a los porteros de la cuarta sala, se quitan la corona de gloria de sus cabezas y caen de bruces.

Y los porteros de la cuarta sala, cuando ven a los porteros de la quinta sala, se quitan la corona de gloria de sus cabezas y caen de bruces.

Y los porteros de la quinta sala, cuando ven a los porteros de la sexta sala, se quitan la corona de gloria de sus cabezas y caen de bruces.

Y los porteros de la sexta sala, cuando ven a los porteros de la séptima sala, se quitan la corona de gloria de sus cabezas y caen de bruces.

\par 4 Y los porteros del séptimo salón, cuando ven a los cuatro grandes príncipes, los honorables, que están designados sobre los cuatro campamentos de shekina, se quitan la(s) corona(s) de gloria de sus cabezas y caen sobre sus cabezas. caras.

\par 5 Y los cuatro grandes príncipes, cuando ven a tag' as, el príncipe, grande y honrado con cánticos (y) alabanza, a la cabeza de todos los hijos del cielo, se quitan la corona de gloria de sus cabezas y caen sobre sus caras.

\par 6 Y Tag'as, el gran y honorable príncipe, cuando ve a Barattiel, el gran príncipe de tres dedos en la altura de 'Araboth, el cielo más alto, se quita la corona de gloria de su cabeza y cae sobre su rostro. .

\par 7 Y Barattiel, el gran príncipe, cuando ve a Hamon, el gran príncipe, el temible y honrado, agradable y terrible, que hace temblar a todos los hijos del cielo, cuando se acerca el tiempo (que está fijado) para el dicho del (Tres) Santo como está escrito: «Al ruido del tumulto (hamon) los pueblos huyen; en tu elevación las naciones serán esparcidas» – se quita la corona de gloria de su cabeza y cae de bruces.

\par 8 Y Hamon, el gran príncipe, cuando ve a Tutresiel, el gran príncipe, se quita la corona de gloria de su cabeza y cae de bruces.

\par 9 Y TutresielIT, el gran príncipe, cuando ve a Atrugiel, el gran príncipe, se quita la corona de gloria de su cabeza y cae de bruces.

\par 10 Y Atrugiel el gran príncipe, cuando ve a Na'aririel H', el gran príncipe, se quita la corona de gloria de su cabeza y cae de bruces.

\par 11 Y Na'aririel H', el gran príncipe, cuando ve a Sasnigiel, el gran príncipe, se quita la corona de gloria de su cabeza y cae de bruces.

\par 12 Y Sasnigiel H', cuando ve a Zazriel H', el gran príncipe, se quita la corona de gloria de su cabeza y cae de bruces.

\par 13 Y el príncipe Zazriel H', cuando ve a Geburatiel H', el príncipe, se quita la corona de gloria de su cabeza y cae de bruces.

\par 14 Y Geburatiel H', el príncipe, cuando ve a 'Araphiel H', el príncipe, se quita la corona de gloria de su cabeza y cae de bruces.

\par 15 Y 'Araphiel H', el príncipe, cuando ve a 'Ashruylu, el príncipe, que preside todas las sesiones de los hijos del cielo, se quita la corona de gloria de su cabeza y cae de bruces.

\par 16 Y Ashruylu H', el príncipe, cuando ve a Gallisur H', el Príncipe, que revela todos los secretos de la ley (Tora), se quita la corona de gloria de su cabeza y cae de bruces.

\par 17 Y Gallisur H', el príncipe, cuando ve a Zakzakiel H', el príncipe designado para escribir los méritos de Israel en el Trono de Gloria, se quita la corona de gloria de su cabeza y cae de bruces. .

\par 18 Y Zakzakiel H', el gran príncipe, cuando ve a 'Anaph(i)el H', el príncipe que guarda las llaves de los Salones celestiales, se quita la corona de gloria de su cabeza y cae de bruces. ¿Por qué se le llama con el nombre de Anafiel? Porque la rama de su honor y majestad y su corona y su esplendor y su brillo cubre (eclipsa) todas las cámaras de 'Araboth Raqia en lo alto, así como el Creador del mundo (las eclipsa). Así como está escrito con respecto al Hacedor del mundo: «Su gloria cubrió los cielos, y la tierra se llenó de su alabanza», así también el honor y la majestad de 'Anaphiel cubren todas las glorias de 'Araboth el más alto. .

\par 19 Y cuando ve a Sother 'Ashiel H', el príncipe, el grande, temible y honrado, se quita la corona de gloria de su cabeza y cae de bruces. ¿Por qué se llama Sother Ashiel? Porque él está designado sobre las cuatro cabeceras del río de fuego frente al Trono de Gloria; y todo príncipe que sale o entra ante la Shekina, sale o entra sólo con su permiso. Porque a él le están confiadas las focas del río de fuego. Y además, su altura es de 7000 miríadas de parasangas. Y aviva el fuego del río; y sale y entra ante la Shekina para exponer lo que está escrito (registrado) acerca de los habitantes del mundo. Según está escrito: «se fijó el juicio y se abrieron los libros».

\par 20 Y el príncipe Sother 'Ashiel, cuando ve a Shoqed Chozi, el gran príncipe, el poderoso, terrible y honorable, se quita la corona de gloria de su cabeza y cae sobre su rostro. ¿Y por qué se llama Shoqed Chozi? Porque él pesa todos los méritos (del hombre) en una balanza en presencia del Santo, bendito sea.

\par 21 Y cuando ve a Zehanpuryu H', el gran príncipe, el poderoso y terrible, honrado, glorificado y temido en toda la casa celestial, se quita la corona de gloria de su cabeza y cae de bruces. ¿Por qué se llama Zehanpuryu? Porque reprende al río de fuego y lo empuja de regreso a su lugar.

\par 22 Y cuando ve a 'Azbuga H', el gran príncipe, glorificado, venerado, honrado, adornado, maravilloso, exaltado, amado y temido entre todos los grandes príncipes que conocen el misterio del Trono de Gloria, se quita la corona de la gloria de su cabeza y cae sobre su rostro. ¿Por qué se llama 'Azbuga? Porque en el futuro ceñirá (vestirá) a los justos y piadosos del mundo con los vestidos de la vida y los envolverá en el manto de la vida, para que vivan en ellos una vida eterna.

\par 23 Y cuando ve a los dos grandes príncipes, los fuertes y glorificados, que están de pie sobre él, se quita la corona de gloria de su cabeza y cae de bruces. Y estos son los nombres de los dos príncipes:

\par Sopheriel H' (quien) mata, (Sopheriel H' el Asesino), el gran príncipe, el honrado, glorificado, irreprochable, venerable, antiguo y poderoso; (y) Sopheriel H' (quien) da vida (Sopheriel H' el Dador de vida), el gran príncipe, el honrado, glorificado, irreprochable, antiguo y poderoso.

\par 24 ¿Por qué se llama Sopheriel H' el que mata? Porque está encargado de los libros de los muertos: [de modo que] cada uno, cuando se acerca el día de su muerte, lo escribe en los libros de los muertos.

\par ¿Por qué se le llama Sopheriel H' el que da vida (Sopheriel H' the Lifegiver)? Debido a que él está designado sobre los libros de los vivos (de la vida), para que todo aquel a quien el Santo, bendito sea Él, traiga a la vida, lo escribe en el libro de los vivos (de la vida), por autoridad de MAQOM. Tal vez podrías decir: «Como el Santo, bendito sea, está sentado en un trono, ellos también están sentados cuando escriben». (Respuesta): La Escritura nos enseña: «Y todo el ejército del cielo está junto a él». «El ejército del cielo» (se dice) para mostrarnos que incluso los Grandes Príncipes, ninguno como los que hay en los altos cielos, no cumplen los pedidos de la Shekina más que de pie. Pero ¿cómo es posible que escriban estando de pie? Es así:

\par 25 Uno está sobre las ruedas de la tempestad, y el otro está sobre las ruedas del viento tormentoso.

\par Uno está vestido con vestiduras reales, el otro está vestido con vestiduras reales.

\par Uno está envuelto en un manto de majestad y el otro está envuelto en un manto de majestad.

\par Uno está coronado con una corona real y el otro está coronado con una corona real.

\par El cuerpo de uno está lleno de ojos y el cuerpo del otro está lleno de ojos.

La apariencia de uno es como la apariencia de un relámpago, y la apariencia del otro es como la apariencia de un relámpago.

Los ojos de uno son como el sol en su fuerza, y los ojos del otro son como el sol en su fuerza.

\par La altura de uno es como la altura de los siete cielos, y la altura del otro es como la altura de los siete cielos.

\par Las alas de uno son tantos como los días del año, y las alas del otro son tantos como los días del año.

\par ¡Las alas de uno se extienden a lo ancho de Raqict, y las alas del otro se extienden a lo ancho de Raqia!.

Los labios de uno son como las puertas del Oriente, y los labios del otro son como las puertas del Oriente.

La lengua de uno es tan alta como las olas del mar, y la lengua del otro es tan alta como las olas del mar.

De la boca de uno sale una llama, y ​​de la boca del otro sale una llama.

De la boca de uno salen relámpagos y de la boca del otro salen relámpagos.

Del sudor de uno se enciende el fuego, y del sudor del otro se enciende el fuego.

\par De la lengua de uno arde una antorcha, y de la lengua del otro arde una antorcha.

\par En la cabeza de uno hay una piedra de zafiro, y en la cabeza del otro hay una piedra de zafiro.

\par Sobre los hombros de uno hay una rueda de querubín veloz, y sobre los hombros del otro hay una rueda de querubín veloz.

\par Uno tiene en su mano un pergamino ardiendo, el otro tiene en su mano un pergamino ardiendo.

\par Uno tiene en su mano un estilo llameante, el otro tiene en su mano un estilo llameante.

\par La longitud del rollo es miríadas de parasangas; el tamaño del estilo es de 3000 miríadas de parasangas; el tamaño de cada letra que escriben es de 365 parasangas.

\chapter{19}

\par \textit{Rikbiel, el príncipe de las ruedas del Merkaba. Los alrededores de la Merkaba. La conmoción entre las huestes angelicales en el momento del Qedushsha}

\par 1 R. Ismael dijo: Metatrón, el Ángel, el Príncipe de la Presencia, me dijo: Por encima de estos tres ángeles, estos grandes príncipes hay un Príncipe, distinguido, honrado, noble, glorificado, adornado, temible, valiente, fuerte, grande, magnificado, glorioso, coronado, maravilloso, exaltado, irreprochable, amado, señorial, alto y altivo, antiguo y poderoso, como nadie entre los príncipes. Su nombre es Rikbiel H', el gran y venerado príncipe que está junto al Merkaba.

\par 2 ¿Y por qué se llama Rikbiel? Porque él es designado sobre las ruedas del Merkaba, y ellas están a su cargo.

\par 3 ¿Y cuantas son las ruedas? Ocho; dos en cada dirección. Y cuatro vientos los rodean. Y estos son sus nombres: «el Viento-Tormenta», «la Tempestad», «el Viento Fuerte» y «el Viento del Terremoto».

\par 4 Y debajo de ellos corren continuamente cuatro ríos de fuego, un río de fuego a cada lado. Y alrededor de ellos, entre los ríos, están plantadas (colocadas) cuatro nubes, y estas son: «nubes de fuego», «nubes de lámparas», «nubes de carbón», «nubes de azufre» y están de pie sobre contra [sus] ruedas.

\par 5 Y los pies de los Chayyoth descansan sobre las ruedas. Y entre una rueda y otra rueda ruge el terremoto y truena el trueno.

\par 6 Y cuando se acerca el momento de recitar la Canción, (entonces) las multitudes de ruedas se mueven, la multitud de nubes tiemblan, todos los jefes (shallishim) tienen miedo, todos los jinetes (parashim) se enfurecen. , todos los poderosos (gtbborim) están excitados, todos los ejércitos (seba'im) están asustados, todas las tropas (gedudim) tienen miedo, todos los designados (memunnim) se apresuran, todos los príncipes (sarim) y ejércitos. (chayyelim) están consternados, todos los siervos (mesharetim) desmayan y todos los ángeles (maVakim) y divisiones (degalim) sufren dolores de parto.

\par 7 Y una rueda hace un sonido que se oye a la otra y un Kerub a otro, un Chayya a otro, un Serafín a otro (diciendo): «Ensalzad al que cabalga en Araboth, por su nombre Jah y regocijaos delante de ¡a él!»

\chapter{20}

\par \textit{Chayyliel, el príncipe de los Chayyoth}

\par 1 R. Ismael dijo: Metatrón, el ángel, el Príncipe de la Presencia, me dijo: Por encima de estos hay un príncipe grande y poderoso. Su nombre es Chayyliel H', un príncipe noble y venerado, un príncipe glorioso y poderoso, un príncipe grande y venerado, un príncipe ante quien tiemblan todos los hijos del cielo, un príncipe que puede tragarse toda la tierra en un momento ( de un bocado).

\par 2 ¿Y por qué se llama Chayyliel H'? Porque él es designado sobre los Santos Chayyoth y golpea a los Chayyoth con latigazos de fuego: y los glorifica, cuando dan alabanza, gloria y regocijo y les hace apresurarse a decir «Santo» y "Bendita sea la Gloria de H' ¡Desde su lugar! (es decir, el Qedushsha).

\chapter{21}

\par \textit{El Chayyoth}

\par 1 R. Ismael dijo: Metatrón, el ángel, el Príncipe de la Presencia, me dijo: Cuatro (son) los Chayyoth correspondientes a los cuatro vientos. Cada Chayya es como el espacio del mundo entero. Y cada uno tiene cuatro caras; y cada rostro es como el rostro del Oriente.

\par 2 Cada uno tiene cuatro alas y cada ala es como la cubierta (techo) del universo.

\par 3 Y cada uno tiene caras en medio de caras y alas en medio de alas. El tamaño de las caras es (como el tamaño de) las caras, y el tamaño de las alas es (como el tamaño de) las alas.

\par 4 Y cada uno será coronado con coronas en su cabeza. Y cada corona es como el arco en la nube. Y su esplendor es semejante al esplendor del globo solar. Y las chispas que brotan de cada uno son como el esplendor del astro de la mañana (planeta Venus) en el Este.

\chapter{22}

\par \textit{Kerubiel, el Príncipe de los Querubines. Descripción de los Kerubim}

\par 1 R. Ismael dijo: Metatrón, el ángel, el Príncipe de la Presencia, me dijo: Por encima de estos hay un príncipe, noble, maravilloso, fuerte y alabado con toda clase de alabanzas. Su nombre es Kerubiel H', un príncipe poderoso, lleno de poder y fuerza.

\par [AD: un príncipe de alteza, y la Alteza (está) con él, un príncipe justo, y la justicia (está) con él, un príncipe santo, y la santidad (está) con él, un príncipe] [B: un príncipe de alteza, y con él (hay) un príncipe justo, de justicia, y con él un príncipe santo, de santidad, y con él (hay) un príncipe ] glorificado en (por) mil ejércitos, exaltado por diez mil ejércitos .

\par 2 Ante su ira la tierra tiembla, ante su ira se conmueven los campamentos, de miedo a él se estremecen los cimientos, ante su reprensión tiemblan los árabes.

\par 3 Su estatura está llena de brasas. La altura de su estatura es como la altura de los siete cielos, la anchura de su estatura es como la anchura de los siete cielos y el espesor de su estatura es como los siete cielos.

\par 4 La apertura de su boca es como una lámpara de fuego. Su lengua es un fuego consumidor. Sus cejas son como el esplendor del relámpago. Sus ojos son como chispas de brillo. Su rostro es como un fuego ardiente.

\par 5 Y sobre su cabeza hay una corona de santidad en la que está grabado el Nombre explícito, y de ella salen relámpagos. Y el arco de Shekina está entre sus hombros.

\par 6 [AD: Y su espada está sobre sus lomos y sus flechas son como relámpagos en su cinto. Y sobre su cuello hay un escudo de fuego consumidor, y alrededor de él hay carbones de enebro.] [B: Y su espada es como un relámpago; y sobre sus lomos hay flechas como llamas, y sobre su armadura y escudo hay un fuego consumidor, y sobre su cuello hay carbones de enebro ardiendo y (también) alrededor de él (hay carbones de enebro ardiendo). ]

\par 7 Y el esplendor de Shekina está en su rostro; y los cuernos de majestad en sus ruedas; y una diadema real sobre su cráneo.

\par 8 Y su cuerpo está lleno de ojos. Y las alas cubren toda su alta estatura (lit. la altura de su estatura son todas las alas).

\par 9 A su derecha arde una llama, y ​​a su izquierda arde un fuego; y en él arden brasas. Y de su cuerpo salen tizones. Y de su rostro salen relámpagos. Con él siempre hay trueno sobre (en) trueno, a su lado siempre hay terremoto sobre (en) terremoto.

\par 10 Y los dos príncipes de Merkaba están junto a él.

\par 11 ¿Por qué se llama Kerubiel H', el Príncipe? Porque está encargado del carro de los querubines. Y los poderosos Kerubim son entregados a su cargo. Y adorna las coronas de sus cabezas y pule la diadema de sus cráneos.

\par 12 Él magnifica la gloria de su apariencia. Y glorifica la belleza de su majestad. Y aumenta la grandeza de su honor. Él hace que se cante el cántico de alabanza. Él intensifica su hermosa fuerza. Él hace brillar el brillo de su gloria. Él embellece su buena misericordia y bondad amorosa. Él enmarca la justicia de su resplandor. Él embellece aún más su belleza misericordiosa. Él glorifica su recta majestad. Exalta el orden de sus alabanzas, para establecer la morada de aquel «que habita sobre los querubines».

\par 13 Y los Kerubim están de pie junto al Santo Chayyoth, y sus alas están levantadas hasta sus cabezas (lit. son como la altura de sus cabezas).
\par y Shekina está (descansando) sobre ellos
\par y el resplandor de la Gloria está sobre sus rostros
\par y canto y alabanza en su boca
\par y sus manos están debajo de sus alas
\par y sus pies están cubiertos por sus alas
\par y cuernos de gloria sobre sus cabezas
\par y el esplendor de Shekina en su rostro
\par y Shekina está (descansando) sobre ellos
\par y piedras de zafiro alrededor de ellos
\par y columnas de fuego en sus cuatro lados
\par y columnas de tizones al lado de ellos.

\par 14 Hay un zafiro a un lado y otro zafiro al otro lado, y debajo de los zafiros hay brasas de enebro ardiendo.

\par 15 Y un Kerub está de pie en cada dirección, pero las alas de los Kerubim se rodean en gloria sobre sus cráneos; y los extendieron para cantar con ellos un cántico al que habita las nubes y para alabar con ellos la temible majestad del rey de reyes.

\par 16 Y Kerubiel H', el príncipe que está designado sobre ellos, los viste con trajes hermosos, hermosos y agradables y los exalta con toda clase de exaltación, dignidad y gloria. Y los apresura, en gloria y poder, a hacer la voluntad de su Creador en todo momento. Porque sobre sus elevadas cabezas permanece continuamente la gloria del gran rey «que habita sobre los querubines».

\chapter{22b}

\par 1 [L(mr), siguiendo después de la rec. del cap. XXII c. vss.1-3 (medio): Y hay un atrio delante del Trono de Gloria,] [B: R. Ismael me dijo: Metatrón, el ángel, el Príncipe de la Presencia, me dijo: ¿Cómo están los ángeles? parado en lo alto? Él dijo: Como un puente que se pone sobre un río para que todos puedan pasar por él, así también se pone un puente desde el principio de la entrada hasta el final.]

\par 2 [Lmr en el que ningún serafín ni ángel puede entrar, y son 6.000 miríadas de parasangs, como está escrito: «y los Serafines están sobre él» (la última palabra del pasaje de las Escrituras es [valor numérico: 36] ).] [B: Y tres ángeles ministradores la rodean y entonan un cántico delante de YHWH, el Dios de Israel. Y delante de él están los señores del terror y los capitanes del miedo, mil veces mil y diez mil veces diez mil en número y cantan alabanzas e himnos delante de YHWH, el Dios de Israel. ]

\par 3 [Lmr: Como el valor numérico de (36) es el número de puentes allí. ] [B: Allí hay numerosos puentes: puentes de fuego y numerosos puentes de granizo. También numerosos ríos de granizo, numerosos tesoros de nieve y numerosas ruedas de fuego. ]

\par 4 [Lmr: Y hay miríadas de ruedas de fuego. Y los ángeles ministradores son 12.000 miríadas. Y hay 12.000 ríos de granizo y 12.000 tesoros de nieve. Y en las siete Salas hay carros de fuego y llamas, sin cálculo, ni fin, ni búsqueda. (Lmr. termina aquí.) ] [B: ¿Y cuántos son los ángeles ministradores? 12.000 miríadas: seis (mil miríadas) arriba y seis (mil miríadas) abajo. Y 12.000 son los tesoros de nieve, seis arriba y seis abajo. Y miríadas de ruedas de fuego, 12 (miríadas) arriba y 12 (miríadas) abajo. Y rodean los puentes y los ríos de fuego y los ríos de granizo. Y hay numerosos ángeles ministradores, formando entradas para todas las criaturas que están en medio de ellas, correspondientes a (frente a) los senderos de RaqtaShamayim. ]

\par 5 ¿Qué hace YHWH, Dios de Israel, Rey de gloria? El Dios grande y temible, poderoso en fuerza, cubre su rostro.

\par 6 En Araboth hay 660.000 miríadas de ángeles de gloria de pie frente al Trono de Gloria y las divisiones de fuego llameante. Y el Rey de Gloria cubrirá Su rostro; porque de lo contrario, el Araboth Raqia' se desgarraría en medio de él debido a la majestad, el esplendor, la belleza, el resplandor, la hermosura, la brillantez, el brillo y la excelencia de la apariencia de (el Santo), bendito sea.

\par 7 Hay numerosos ángeles ministradores que realizan su voluntad, numerosos reyes, numerosos príncipes en el Araboth de su deleite, ángeles que son reverenciados entre los gobernantes del cielo, distinguidos, adornados con canciones y trayendo amor a la memoria: (que) están asustados por el esplendor de la Shekina, y sus ojos quedan deslumbrados por la brillante belleza de su Rey, sus rostros se vuelven negros y sus fuerzas fallan.

\par 8 De allí brotan ríos de alegría, corrientes de alegría, ríos de regocijo, corrientes de triunfo, ríos de amor, corrientes de amistad (otra lectura:) de conmoción, y fluyen y avanzan ante el Trono de Gloria. y crece y atraviesa las puertas de los senderos de 'Araboth Raqia a la voz de los gritos y la música de los Chayyoth, a la voz del regocijo de los panderos de sus 'Ophannim y a la melodía de los címbalos de Sus Kerubim. . Y se engrandecen y salen conmocionados al son del himno: «Santo, santo, santo, el Señor de los ejércitos; ¡Toda la tierra está llena de su gloria!

\chapter{22c}

\par \textit{(en B, Lo y Lmr)}

\par 1 Ismael dijo: Metatrón y el Príncipe de la Presencia me dijo: ¿Cuál es la distancia entre un puente y otro? 12 miríadas de parasangs. Su ascenso es de miríadas de parasangs y su descenso de 12 miríadas de parasangs.

\par 2 (La distancia) entre los ríos del pavor y los ríos del miedo es de 22 miríadas de parasangas; entre los ríos de granizo y los ríos de oscuridad 36 miríadas de parasangs; entre las cámaras de los relámpagos y las nubes de la compasión 42 miríadas de parasangs; entre las nubes de la compasión y el Merkaba 148 miríadas de parasangs; entre los Merkaba y los Kerubim 148 miríadas de parasangs; entre los Kerubim y los 'Ophannim 24 miríadas de parasangs; entre los Ophannim y las cámaras de las cámaras 24 miríadas de parasangs; entre las cámaras de las cámaras y el Santo Chayyoth 40.000 miríadas de parasangs; entre un whig (de los Chayyoth) y otros 12 miríadas de parasangs; y el ancho de cada ala es de esa misma medida; y la distancia entre el Santo Chayyoth y el Trono de Gloria es de 30.000 miríadas de parasangs.

\par 3 Y desde el pie del Trono hasta el asiento hay cuarenta mil miríadas de parasangas. Y el nombre del que está sentado sobre él: ¡santificado sea el nombre!

\par 4 [Y los arcos del Arco están colocados sobre el 'Araboth, y tienen 1.000 mil y 10.000 veces diez mil (de parasangs) de altura. Su medida es según la medida del Irin y Qaddishin (Vigilantes y Santos). Como está escrito «Mi arco lo he puesto en la nube». No está escrito aquí «pondré» sino «he puesto», (es decir) ya; nubes que rodean el Trono de Gloria. A medida que pasan Sus nubes, los ángeles del granizo (se convierten en) carbón ardiente.

\par 5 Y un fuego de voz desciende del Santo Chayyoth. Y por el aliento de aquella voz «corren» a otro lugar, temiendo que les mande ir; y «regresan» para que no los dañe desde el otro lado. Por eso «corren y regresan».

\par 6 Y estos arcos del Arco son más hermosos y radiantes que el resplandor del sol durante el solsticio de verano. Y son más blancos que una llama de fuego y son grandes y hermosos.

\par 7 Sobre los arcos del Arco están las ruedas de los 'Ophannim. Su altura es 1000 mil y 10,000 veces 10,000 unidades de medida según la medida de los Serafines y las Tropas (Gedudim).]

\chapter{23}

\par \textit{Los vientos que soplan 'bajo las alas de los Kerubim'}

\par 1 R. Ismael dijo: Metatrón, el Ángel, el Príncipe de la Presencia, me dijo: Hay numerosos vientos que soplan bajo las alas de los Kerubim. Sopla «el viento melancólico», como está escrito: «y el viento de Dios se movía sobre la faz de las aguas».

\par 2 Sopla «el viento fuerte», como se dice: «y el Señor hizo retroceder el mar con un fuerte viento del este toda aquella noche».

\par 3 Sopla «el viento del este», como está escrito: «el viento del este trajo las langostas».

\par 4 Sopla «el viento de las codornices», como está escrito: «Y salió un viento de parte del Señor y trajo codornices».

\par 5 Sopla «el viento de los celos», como está escrito: «Y el viento de los celos vino sobre él».

\par 6 Allí sopla el «Viento del Terremoto», como está escrito: «y después el viento del terremoto; pero el Señor no estaba en el terremoto».

\par 7 Sopla el «Viento de H'» como está escrito: «y él me sacó con el viento de H' y me depositó».

\par 8 Sopla el «Viento Maligno», como está escrito: «y el viento maligno se alejó de él».

\par 9 Allí sopla el «Viento de la Sabiduría» y el «Viento del Entendimiento» y el «Viento del Conocimiento» y el «Viento del Temor de H'» como está escrito: «Y el viento de H' reposará sobre él ;el viento de sabiduría y de entendimiento, el viento de consejo y de poder, el viento de conocimiento y de temorP».

\par 10 Sopla el «Viento de Lluvia», como está escrito: «el viento del norte trae lluvia».

\par 11 Sopla el «Viento de los relámpagos», como está escrito: «Él hace los relámpagos para la lluvia y saca el viento de sus tesoros».

\par 12 Sopla el «Viento que rompe las rocas», como está escrito: «Pasó el Señor y un viento grande y fuerte (desgarró las montañas y desmenuzó las rocas delante del Señor)».

\par 13 Sopla el «Viento de calma del mar», como está escrito: «Y Dios hizo pasar un viento sobre la tierra, y las aguas se calmaron».

\par 14 Sopla el «Viento de ira», como está escrito: «y he aquí vino un gran viento del desierto y golpeó las cuatro esquinas de la casa y cayó».

\par 15 Sopla el «Viento de Tormenta», como está escrito: «Viento de Tormenta, cumpliendo su palabra».

\par 16 Y Satanás está parado entre estos vientos, porque «viento de tormenta» no es otra cosa que «Satanás», y todos estos vientos no soplan sino bajo las alas de los Kerubim^ como está escrito: «y cabalgó sobre un querubín y voló, sí, y voló velozmente sobre las alas del viento».

\par 17 ¿Y adónde van todos estos vientos? La Escritura nos enseña que salen de debajo de las alas de los Querubines y descienden sobre el globo solar, como está escrito: «El viento va hacia el sur y gira hacia el norte; gira continuamente en su curso y el viento vuelve nuevamente a sus circuitos». Y del globo solar regresan y descienden sobre [los ríos y los mares, sobre] las montañas y sobre las colinas, como está escrito: «Porque he aquí el que forma las montañas y crea el viento».

\par 18 Y de las montañas y de las colinas regresan y descienden a los mares y a los ríos; y de los mares y de los ríos regresan y descienden sobre (las) ciudades y provincias; y de las ciudades y provincias regresan y descienden al Huerto, y del Huerto regresan y descienden al Edén, como está escrito: «caminando en el Huerto al viento del día». Y en medio del Jardín se juntan y soplan de un lado a otro y se perfuman con las especias del Jardín hasta en sus partes más remotas, hasta que se separan unas de otras, y, llenas del olor de las especias puras. , traen el olor de las partes más remotas del Edén y las especias del Jardín a los justos y piadosos que en el futuro heredarán el Jardín del Edén y el Árbol de la Vida, como está escrito: «Despierta, oh norte». viento; y ven al sur; sopla sobre mi jardín, y fluirán sus especias. Que mi amado entre en su huerto y coma de sus preciosos frutos».

\chapter{24}

\par \textit{Los diferentes carros del Santo, bendito sea}

\par 1 R. Ismael dijo: Metatrón, el Ángel, el Príncipe de la Presencia, la gloria de todo el cielo, me dijo: Numerosos carros tiene el Santo, bendito sea: Él tiene los «Carros de (los) Querubines », como está escrito: «Y montó sobre un querubín y voló».

\par 2 Él tiene los «Carros del Viento», como está escrito (ib.): «y voló velozmente sobre las alas del viento

\par 3 Él tiene los «Carros de (la) Nube Veloz», como está escrito: «He aquí, el Señor cabalga sobre una nube veloz».

\par 4 Él tiene «los Carros de las Nubes», como está escrito: «He aquí, vengo a ti en una nube».

\par 5 Él tiene los «Carros del Altar», como está escrito: «Vi al Señor de pie sobre el Altar».

\par 6 Él tiene los «Carros de Ribbotaim», como está escrito: «Los carros de Dios son Ribbotaim; miles de ángeles».

\par 7 Tiene los «Carros de la Tienda», como está escrito: «Y el Señor apareció en la Tienda en una columna de nube».

\par 8 Tiene los «Carros del Tabernáculo», como está escrito: «Y el Señor le habló desde el tabernáculo».

\par 9 Él tiene los «Carros del Propiciatorio», como está escrito: «entonces oyó la Voz que le hablaba desde sobre el propiciatorio».

\par 10 Él tiene los «Carros de Piedra de Zafiro», como está escrito: «y había debajo de sus pies como un pavimento de piedra de zafiro».

\par 11 Él tiene los «Carros de Águilas», como está escrito: «Te llevo sobre alas de águila». Aquí no nos referimos literalmente a las águilas, sino a «los que vuelan velozmente como las águilas».

\par 12Tiene los «carros de Grito», como está escrito: «Dios ha subido con grito».

\par 13 Él tiene los «Carros de 'Araboth», como está escrito: «Ensalza a Aquel que cabalga sobre el 'Araboth».

\par 14 Él tiene los «Carros de Nubes Espesadas», como está escrito: «quien hace de las nubes espesas Su carro».

\par 15 Él tiene los «Carros de los Chayyoth» como está escrito: «y los Chayyoth corrieron y regresaron». Corren con permiso y regresan con permiso, porque Shekina está por encima de sus cabezas.

\par 16 Él tiene los «Carros de Ruedas (Galgallim)», como está escrito: «Y dijo: Métete entre las ruedas que giran».

\par 17 Tiene los «Carros de un Querubín veloz», como está escrito (?): «montado sobre un querubín veloz».

Y cuando cabalga sobre un veloz kerub, cuando pone uno de sus pies sobre él, antes de poner el otro sobre su espalda, mira dieciocho mil mundos de una sola mirada. Y él discierne y ve dentro de todos ellos y sabe lo que hay en cada uno de ellos, y luego pone el otro pie sobre él, según está escrito: «Alrededor de dieciocho mil».

¿De dónde sabemos que Él examina cada uno de ellos todos los días? Está escrito: «Miró desde el cielo a los hijos de los hombres para ver si había alguno que entendiera y buscara a Dios».

\par 18 Él tiene los «Carros de los y Ophannim», como está escrito: «y los y Ophannim estaban llenos de ojos alrededor».

\par 19 Él tiene los «Carros de Su Santo Trono», como está escrito: «Dios se sienta en su santo trono».

\par 20 Él tiene los «carros del Trono de Yah», como está escrito: «Porque una mano está levantada sobre el Trono de Jah».

\par 21 Él tiene los «Carros del Trono del Juicio», como está escrito: «pero el Señor de los ejércitos será exaltado en el juicio».

\par 22 Él tiene los «Carros del Trono de Gloria», como está escrito: «El Trono de Gloria, enaltecido desde el principio, es el lugar de nuestro santuario».

\par 23 Él tiene los «Carros del Trono Alto y Exaltado», como está escrito: «Vi al Señor sentado en el trono alto y exaltado».

\chapter{25}

\par \textit{'Ophphanniel y el príncipe de los 'Ophannirn. Descripción de los 'Ophannim}

\par 1 R. Ismael dijo: Metatrón, el Ángel, el Príncipe de la Presencia, me dijo: Por encima de estos hay un gran príncipe, venerado, alto, señorial, temible, antiguo y fuerte, 'ofphanniel H' es su nombre .

\par 2 Tiene dieciséis caras, cuatro caras a cada lado, y cien alas a cada lado. Y tiene ojos, correspondientes a los días del año. [A: 2190 — y algunos dicen 2116 — en cada lado.] [DE: 2191 (E:196) y dieciséis en cada lado. ]

\par 3 Y en esos dos ojos de su rostro, en cada uno de ellos brillan relámpagos, y de cada uno de ellos arden tizones; y ninguna criatura puede contemplarlos; porque cualquiera que los mira, al instante se quema.

\par 4 Su altura es (como) la distancia de un viaje de años. Ningún ojo puede contemplar y ninguna boca puede expresar el gran poder de su fuerza, excepto el Rey de reyes, el Santo, bendito sea Él, solo.

\par 5 ¿Por qué se llama Offanniel?

\par Porque él es designado sobre los 'Ophannim y los 'Ophannim están a su cargo. Él está todos los días y los atiende y los embellece. Y él exalta y ordena sus habitaciones (DE: carreras), y pule sus lugares de pie, y hace brillantes sus moradas, nivela sus rincones y limpia sus asientos. Y él los atiende temprano y tarde, de día y de noche, para aumentar su belleza, engrandecer su dignidad y hacerlos diligentes en la alabanza de su Creador.

\par 6 Y todos los 'Ophannim están llenos de ojos, y todos están llenos de brillo; setenta y dos piedras de zafiro están fijadas en sus vestiduras en su lado derecho y setenta y dos piedras de zafiro están fijadas en sus vestiduras en su lado izquierdo.

\par 7 Y en la corona de cada uno de ellos están fijadas cuatro piedras de carbunco, cuyo esplendor se extiende en las cuatro direcciones de 'Araboth, así como el esplendor del globo solar se extiende en todas las direcciones del universo. ¿Y por qué? ¿Se llama Carbuncle (Bareqet)? Porque su esplendor es como la apariencia de un relámpago (Baraq). Y tiendas de esplendor, tiendas de esplendor, tiendas de resplandor como de zafiro y de carbunclo los encierran a causa del aspecto resplandeciente de sus ojos.

\chapter{26}

\par \textit{Seraphiel, el Príncipe de los Serafines. Descripción de los Serafines}

\par 1 R. Ismael dijo: Metatrón, el Ángel, el Príncipe de la Presencia, me dijo: Por encima de estos hay un príncipe, maravilloso, noble, grande, honorable, poderoso, terrible, un jefe y líder x y un veloz escriba, glorificado, honrado y amado.

\par 2 Él está todo lleno de esplendor, lleno de alabanza y resplandor; y está totalmente lleno de brillo, de luz y de belleza; y todo él está lleno de bondad y grandeza.

\par 3 Su rostro es totalmente semejante al de los ángeles, pero su cuerpo es como el cuerpo de un águila.

\par 4 Su esplendor es como un relámpago, su apariencia como tizones de fuego, su belleza como chispas, su honor como carbones encendidos, su majestad como chaskmals, su resplandor como la luz del planeta Venus. Su imagen es semejante a la Luz Mayor. Su altura es como los siete cielos. La luz de sus cejas es como la luz séptuple.

\par 5 La piedra de zafiro que tiene sobre su cabeza es tan grande como el universo entero y semejante al esplendor de los mismos cielos en su resplandor.

\par 6 Su ​​cuerpo está lleno de ojos como las estrellas del cielo, innumerables e inescrutables. Cada ojo es como el planeta Venus. Sin embargo, hay algunos de ellos como la Luz Menor y otros como la Luz Mayor. Desde sus tobillos hasta sus rodillas (son) como estrellas de relámpago, desde sus rodillas hasta sus muslos como el planeta Venus, desde sus muslos hasta sus lomos como la luna, desde sus lomos hasta su cuello como el sol, desde su cuello hasta su cráneo como a la Luz Imperecedera.

\par 7 La corona sobre su cabeza es semejante al esplendor del Trono de Gloria. La medida de la corona es la distancia de años de viaje. No hay ningún tipo de esplendor, ningún tipo de brillo, ningún tipo de resplandor, ningún tipo de luz en el universo que no esté fijado en esa corona.

\par 8 El nombre de ese príncipe es Seraphiel H'. Y la corona en su cabeza, su nombre es «el Príncipe de la Paz». ¿Y por qué se le llama con el nombre de Seraphiel H'? Porque está designado sobre los Serafines. Y los Serafines flamígeros quedan a su cargo. Y él los preside de día y de noche y les enseña canto, alabanza, proclamación de hermosura, poder y majestad; para que puedan proclamar la belleza de su Rey en toda forma de Alabanza y Santificación (Qedushsha).

\par 9 ¿Cuántos son los Serafines? Cuatro, correspondiente a los cuatro vientos del mundo. ¿Y cuántas alas tiene cada uno? Seis, correspondientes a los seis días de la Creación. ¿Y cuántas caras tienen? Cada uno de ellos, cuatro caras.

\par 10 La medida de los Serafines y la altura de cada uno de ellos corresponden a la altura de los siete cielos. El tamaño de cada ala es como la medida de toda Raqia. El tamaño de cada cara es como el de la cara de Oriente.

\par 11 Y cada uno de ellos emite luz como el esplendor del Trono de Gloria: de modo que ni siquiera los Santos Chayyoth, los honorables 'Ophannim, ni los majestuosos Kerubim pueden contemplarlo. Todo aquel que la contempla, sus ojos se oscurecen a causa de su gran esplendor.

\par 12 ¿Por qué se les llama Serafines? Porque queman (saraph) las tablas de escribir de Satanás: Todos los días Satanás se sienta junto con Sammael, el Príncipe de Roma, y ​​con dubbiel, el Príncipe de Persia, y escriben las iniquidades de Israel en tablas de escribir que entregan. a los Serafines, para que los presenten ante el Santo, bendito sea, para que destruya a Israel del mundo. Pero los Serafines saben por los secretos del Santo, bendito sea, que él no desea que este pueblo de Israel perezca. ¿Qué hacen los Serafines? Cada día los reciben (aceptan) de la mano de Satanás y los queman en el fuego ardiente frente al alto y exaltado Trono para que no puedan presentarse ante el Santo, bendito sea Él, en el momento en que él esté. sentado en el Trono del Juicio, juzgando al mundo entero en verdad.



\chapter{27}

\par \textit{Radweriel, el guardián del Libro de los Registros}

\par 1 R. Ismael dijo: Metatrón, el Ángel x de H', el Príncipe de la Presencia, me dijo: Por encima de los Serafines hay un príncipe, exaltado sobre todos los príncipes, maravilloso más que todos los sirvientes. Su nombre es Radweriel H' y está encargado de la tesorería de los libros.

\par 2 Él trae el Estuche de Escritos (con) el Libro de los Registros en él, y lo lleva ante el Santo, bendito sea. Y rompe los sellos del estuche, lo abre, saca los libros y los entrega ante el Santo, bendito sea. Y el Santo, bendito sea, los recibe de su mano y los entrega ante sus ojos a los escribas, para que puedan leerlos en el Gran Beth Dinin en la altura de 'Araboth Raqia', ante la casa celestial.

\par 3 ¿Y por qué se llama radweriel? Porque de cada palabra que sale de su boca se crea un ángel: y él está en las canciones (en la compañía de cantos) de los ángeles ministradores y pronuncia una canción ante el Santo, bendito sea Él cuando se acerca el tiempo para la recitación del (Tres veces) Santo.

\chapter{28}

\par \textit{El 'Irin y Qaddishin}

\par 1 R. Ismael dijo: Metatrón, el Ángel, el Príncipe de la Presencia, me dijo: Por encima de todos estos hay cuatro grandes príncipes, llamados Irin y Qaddishin: altos, honrados, reverenciados, amados, maravillosos y gloriosos. , mayor que todos los hijos del cielo. No hay nadie como ellos entre todos los príncipes celestiales y ninguno igual entre todos los Siervos. Porque cada uno de ellos es igual a todos los demás juntos.

\par 2 Y su morada está frente al Trono de Gloria, y su posición frente al Santo, bendito sea, de modo que el brillo de su morada es un reflejo del brillo del Trono de Gloria. Y el esplendor de su rostro es un reflejo del esplendor de Shekina.

\par 3 Y son glorificados por la gloria de la Divina Majestad (Gebura) y alabados por (a través de) la alabanza de Shekina.

\par 4 Y no sólo eso, sino que el Santo, bendito sea, no hace nada en su mundo sin consultarles primero, sino que después lo hace. Como está escrito: «La sentencia es por decreto del 'Irin y la demanda por la palabra del Qaddishin»

\par 5 Los 'Irin son dos y los Qaddishin son dos. ¿Y cómo están ellos ante el Santo, bendito sea? Debe entenderse que un 'Ir está parado a un lado y el otro 'Ir al otro lado, y un Qaddish está parado a un lado y el otro al otro lado.

\par 6 Y siempre exaltan a los humildes, humillan a los soberbios y humillan a los humildes.

\par 7 Y cada día, mientras el Santo, bendito sea, se sienta en el Trono del Juicio y juzga al mundo entero, y los Libros de los Vivos y los Libros de los Muertos se abren ante Él, entonces todos los niños del cielo están ante él con miedo, pavor, temor y temblor. En ese momento, (cuando) el Santo, bendito sea Él, esté sentado en el Trono del Juicio para ejecutar el juicio, su vestimenta es blanca como la nieve, el cabello de su cabeza como lana pura y todo su manto es como el brillo. luz. Y está cubierto de justicia por todas partes como de una cota de malla.

\par 8 Y aquellos 'Irin y Qaddishin están de pie ante él como funcionarios del tribunal ante el juez. Y plantean y argumentan cada caso y cierran el caso que se presenta ante el Santo, bendito sea Él, en el juicio, tal como está escrito: «La sentencia es por el decreto del 'Irin y la demanda por la palabra del 'Irin'. Qaddishin»

\par 9 Algunos de ellos discuten y otros dictan la sentencia en el Gran Bet Din en 'Araboth. Algunos de ellos hacen las peticiones ante la Divina Majestad y otros cierran los casos ante el Altísimo. Otros terminan bajando y (confirmando =) ejecutando las sentencias en la tierra a continuación. Según está escrito: «He aquí, un 'Ir y un Qaddish descendieron del cielo y clamaron en voz alta y dijeron así: Corta el árbol, corta sus ramas, sacude sus hojas y esparce su fruto: deja que las bestias se vayan. de debajo de él, y las aves de sus ramas».

\par 10 ¿Por qué se llaman 'Irin y Qaddishin? Por eso santifican el cuerpo y el espíritu con azotes de fuego en el tercer día del juicio, como está escrito: «Después de dos días nos revivirá; al tercero nos resucitará, y viviremos antes. a él.»

\chapter{29}

\par \textit{Descripción de una clase de ángeles}

\par 1 R. Ismael dijo: Metatrón, el Ángel, el Príncipe de la Presencia, me dijo: Cada uno de thetaa nas setenta nombres correspondientes a las setenta lenguas del mundo. Y todos ellos están (basados) en el nombre del Santo, bendito sea. Y cada nombre está escrito con un estilo llameante sobre la Corona Temerosa (Kether Nora) que está sobre la cabeza del alto y exaltado Rey.

\par 2 Y de cada uno de ellos salen chispas y relámpagos. Y cada uno de ellos está rodeado de cuernos de esplendor alrededor. De cada uno brillan luces, y cada uno está rodeado de tiendas de fulgor, de modo que ni siquiera los Serafines y los Chayyoth que son mayores que todos los hijos del cielo pueden contemplarlos.

\chapter{30}

\par \textit{Los Príncipes de Reinos y el Príncipe del Mundo oficiando en el Gran Sanedrín en el cielo}

\par 1 R. Ismael dijo: Metatrón, el Ángel, el Príncipe de la Presencia, me dijo: Siempre que el Gran Beth Din está sentado en el 'Araboth Raqia en lo alto, nadie en el mundo puede abrir la boca excepto esos grandes príncipes que son llamados H'f por el nombre del Santo, bendito sea.

\par 2 ¿Cuántos son esos príncipes? Setenta y dos príncipes de los reinos del mundo además del Príncipe del Mundo que habla (suplica) a favor del mundo ante el Santo, bendito sea Él, todos los días, a la hora en que se abre el libro en el que están registrados. todos los hechos del mundo, según está escrito: «Se fijó el juicio y se abrieron los libros».

\chapter{31}

\par \textit{(Los atributos de) Justicia, Misericordia y Verdad ante el Trono del Juicio}

\par 1 R. Ismael dijo: Metatrón, el Ángel, el Príncipe de la Presencia, me dijo: En el momento en que el Santo, bendito sea, esté sentado en el Trono del Juicio, (entonces) la Justicia estará de pie. a Su derecha y Misericordia a Su izquierda y Verdad ante Su rostro.

\par 2 Y cuando el hombre entra ante Él para ser juzgado, (entonces) surge hacia él del resplandor de la Misericordia como (por así decirlo) un bastón y se para frente a él. Inmediatamente el hombre cae sobre su rostro, (y) todos los ángeles de la destrucción temen y tiemblan ante él, según está escrito: «Y con misericordia se establecerá el trono, y él se sentará sobre él en verdad».

\chapter{32}

\par \textit{La ejecución del juicio sobre los impíos. la espada de dios}

\par 1 R. Ismael dijo: Metatrón, el Ángel, el Príncipe de la Presencia, me dijo: Cuando el Santo, bendito sea, abre el Libro, la mitad del cual es fuego y mitad llama, (entonces) salen de delante de Él en todo momento para ejecutar el juicio sobre los impíos mediante Su espada (es decir) desenvainada y cuyo esplendor brilla como un relámpago e impregna el mundo de un extremo al otro, como está escrito: «Porque con fuego litigará el Señor (y con su espada con toda carne)».

\par 2 Y todos los habitantes del mundo (es decir, los que vienen al mundo) temen y tiemblan ante Él, cuando ven su espada afilada, como un relámpago, de un extremo del mundo al otro, y las chispas y destellos del tamaño de las estrellas de Raqict que salen de él; según está escrito: «Si afilo el relámpago de mi espada».


\chapter{33}

\par \textit{Los ángeles de la Misericordia, de la Paz y de la Destrucción junto al Trono del Juicio. Los escribas, (vss. i,) Los ángeles junto al Trono de Gloria y los ríos de fuego debajo de él. (vss.-5)}

\par 1 R. Ismael dijo: Metatrón, el Ángel, el Príncipe de la Presencia, me dijo: En el momento en que el Santo, bendito sea, esté sentado en el Trono del Juicio, (entonces) los ángeles de la Misericordia están parados a Su derecha, los ángeles de la Paz están parados a Su izquierda y los ángeles de la Destrucción están parados frente a Él.

\par 2 Y un escriba estaba debajo de él, y otro escriba encima de él.

\par 3 Y los gloriosos Serafines [A: los rodean como tizones alrededor del Trono de Gloria.] [E: rodean el Trono por sus cuatro lados con muros de relámpagos, y los 'Ophannim los rodean con tizones alrededor del Trono de Gloria.] Trono de Gloria. ] Y nubes de fuego y nubes de llamas los rodean a derecha e izquierda; y los Santos Chayyoth llevan el Trono de Gloria desde abajo: cada uno con tres dedos. La medida de los dedos de cada uno es 800.000 y multiplicado por cien, (y) 66.000 parasangas.

\par 4 Y bajo los pies del Chayyoth corren y fluyen siete ríos de fuego. Y la anchura de cada río es de mil parasangs y su profundidad es de mil miríadas de parasangs. Su longitud es inescrutable e inmensurable.

\par 5 Y cada río gira en un arco en las cuatro direcciones de 'Araboth Raqia, y (desde allí) cae hasta Mot on y se detiene (?), y de Ma'on a Zebul y de Zebul a Shechaqim. , desde Shechaqim hasta Raqia, desde Raqia hasta Shamayim y desde Shamayim sobre las cabezas de los impíos que están en la Gehena, como está escrito: «He aquí, un torbellino de Jehová, incluso su furia, se ha ido, sí, una tempestad giratoria; estallará sobre la cabeza de los impíos».

\chapter{34}

\par \textit{Los diferentes círculos concéntricos alrededor del Chayyoth, que consisten en fuego, agua, granizo, etc. y de los ángeles pronunciando el responsorium Qedushsha}

\par 1 R. Ismael dijo: Metatrón; El Ángel, el Príncipe de la Presencia, me dijo: Los cascos de los Chayyoth están rodeados por siete nubes de carbones encendidos. Las nubes de brasas están rodeadas por fuera por siete paredes de llamas. Los siete muros de llamas están rodeados en el exterior por siete muros de granizo (piedras de 'El-gabishy). Los granizos están rodeados por fuera por piedras de granizo (piedra de Bar ad). Las piedras de granizo están rodeadas por fuera por piedras de «las alas de la tempestad». Las piedras de «las alas de la tempestad» están rodeadas por fuera por llamas de fuego. Las llamas del fuego están rodeadas por las cámaras del torbellino. Las cámaras del torbellino están rodeadas por fuera por el fuego y el agua.

\par 2 Alrededor del fuego y del agua están los que pronuncian lo «Santo». Alrededor de los que pronuncian lo «Santo» están los que pronuncian lo «Bendito». Alrededor de quienes pronuncian el «Bienaventurado» están las nubes brillantes. Las nubes brillantes están rodeadas por fuera por brasas de enebro ardiente; y afuera, alrededor de las cales de enebro ardiente, hay mil campamentos de fuego y diez mil huestes de llama(s). Y entre cada campamento y cada ejército hay una nube, para que no sean quemados por el fuego.

\chapter{35}

\par \textit{Los campamentos de ángeles en 'Araboth Raqia': ángeles realizando el Qedushsha}

\par 1 R. Ismael dijo: Metatrón, el Ángel, el Príncipe de la Presencia, me dijo: 506 mil miríadas de campamentos tiene el Santo, bendito sea, en la altura de 'Araboth Raqia. Y cada campamento está (compuesto por) 496 mil ángeles.

\par 2 Y cada ángel, la altura de su estatura es como el gran mar; y el aspecto de su rostro como la apariencia de un relámpago, y sus ojos como lámparas de fuego, y sus brazos y sus pies como de color como el del bronce bruñido, y el rugido de sus palabras como la voz de una multitud.

\par 3 Y todos están de pie delante del Trono de Gloria en cuatro filas. Y los príncipes del ejército están a la cabeza de cada fila.

\par 4 Y algunos de ellos pronuncian el «Santo» y otros pronuncian el «Bendito», algunos de ellos corren como mensajeros, otros están presentes, tal como está escrito: «Mil miles le sirvieron, y diez mil veces diez mil se presentaron ante él: se fijó el juicio y se abrieron los libros».

\par 5 Y en la hora, cuando se acerca el tiempo para decir el «Santo», (entonces) primero surge un torbellino delante del Santo, bendito sea, y estalla sobre el campamento de Shekina y se levanta gran conmoción entre ellos, como está escrito: «He aquí, el torbellino del Señor sale con furor, con conmoción continua».

\par 6 En ese momento, miles de miles de ellos se transforman en chispas, miles de miles de ellos en tizones, miles de miles en destellos, miles de miles en llamas, miles de miles en machos, miles de miles en hembras, miles de miles en vientos, miles de miles en fuegos ardientes, miles de miles en llamas, miles de miles en chispas, miles de miles en jashmals de luz; hasta que tomen sobre sí el yugo del reino de los cielos, el alto y sublime, del Creador de todos ellos, con temor, pavor, temor y temblor, con conmoción, angustia, terror y trepidación. Luego son transformados nuevamente a su forma anterior para tener el temor de su Rey siempre delante de ellos, ya que han puesto sus corazones en decir la Canción continuamente, como está escrito: «Y uno clamaba a otro y decía (Santo, Santo, Santo, etc.)».


\chapter{36}

\par \textit{Los ángeles se bañan en el río de fuego antes de recitar la 'Canción'}

\par 1 R. Ismael dijo: Metatrón, el Ángel, el Príncipe de la Presencia, me dijo: En el momento en que los ángeles ministradores deseen decir (la) Canción, (entonces) Nehar di-Nur (la corriente de fuego) Se eleva con muchos «miles de miles y miríadas de miríadas» (de ángeles) de poder y fuerza de fuego y corre y pasa bajo el Trono de Gloria, entre los campamentos de los ángeles ministradores y las tropas de 'Araboth.

\par 2 Y todos los ángeles ministradores primero bajan a Nehar di-Nur, y se sumergen en el fuego y mojan su lengua y su boca siete veces; y después de eso suben y se ponen la vestidura de 'Machaqe Samal' y se cubren con mantos de jashmal y se paran en cuatro filas frente al Trono de Gloria, en todos los cielos.


\chapter{37}

\par \textit{Los cuatro campamentos de Shekina y sus alrededores}

\par 1 R. Ismael dijo: Metatrón, el Ángel, el Príncipe de la Presencia, me dijo: En las siete Salas hay cuatro carros de Shekina, y delante de cada uno están los cuatro campamentos de Shekina. Entre cada campamento fluye continuamente un río de fuego.

\par 2 Entre cada río hay nubes brillantes [que los rodean], y entre cada nube hay columnas de azufre. Entre un pilar y otro hay ruedas en llamas que los rodean. Y entre una rueda y otra hay llamas de fuego en derredor. Entre una llama y otra hay tesoros de relámpagos; detrás de los tesoros de los relámpagos están las alas del viento de tormenta. Detrás de las alas del viento tormentoso están las cámaras de la tempestad; detrás de las cámaras de la tempestad hay vientos, voces, truenos, chispas [sobre] chispas y terremotos [sobre] terremotos.


\chapter{38}

\par \textit{El miedo que sobreviene a todos los cielos ante el sonido del 'Santo', esp. los cuerpos celestes. Estos apaciguados por el Príncipe del Mundo}

\par 1 R. Ismael dijo: Metatrón, el Ángel, el Príncipe de la Presencia, me dijo: En el momento en que los ángeles ministradores pronuncien (el Tres Veces) Santo, entonces todas las columnas de los cielos y sus órbitas tiemblarán. , y las puertas de los Salones de 'Araboth Raqia se sacuden y los cimientos de Shechaqim y el Universo (Tebel) se mueven, y las órdenes de Ma'on y las cámaras de Makon tiemblan, y todas las órdenes de Raqia y las constelaciones y los planetas están consternados, y los globos del sol y de la luna se apresuran y huyen de sus cursos y corren 12.000 parasangs y tratan de arrojarse desde el cielo,

\par 2 por el estruendo de sus cánticos, por el estruendo de sus alabanzas y por las chispas y relámpagos que brotan de sus rostros; como está escrito: «La voz de tu trueno estaba en el cielo (los relámpagos iluminaron el mundo, la tierra tembló y se estremeció)».

\par 3 Hasta que el príncipe del mundo los llame y les diga: «¡Callad en vuestro lugar! No temáis por los ángeles ministradores que cantan la Canción ante el Santo, bendito sea». Como está escrito: «Cuando las estrellas de la mañana cantaban juntas y todos los hijos del cielo gritaban de alegría».



\chapter{39}

\par \textit{Los nombres explícitos salen volando del Trono y todas las huestes angelicales se postran ante él en el momento del Qedushsha}

\par 1 R. Ismael dijo: Metatrón, el Ángel, el Príncipe de la Presencia, me dijo: Cuando los ángeles ministrantes pronuncian el «Santo» entonces vuelan todos los nombres explícitos que están grabados con un estilo llameante en el Trono de Gloria. como águilas, con dieciséis alas. Y rodean y rodean al Santo, bendito sea, en los cuatro lados del lugar de Su Shekina.

\par 2 Y los ángeles del ejército, y los Siervos llameantes, y los poderosos 'Ophanmm, y los Kerubim de la Shekina, y los Santos Chayyoth, y los Serafines, y los 'Er'ellim, y los Taphsarim y las tropas de el fuego consumidor, y los ejércitos de fuego, y las huestes llameantes, y los santos príncipes, adornados con coronas, vestidos de majestad real, envueltos en gloria, ceñidos de altivez, caen sobre sus rostros tres veces, diciendo: «Bendito sea el nombre de Su glorioso reino por los siglos de los siglos».


\chapter{40}

\par \textit{Los ángeles ministradores recompensaban con coronas, cuando pronunciaban el ii Holyy en su orden correcto, y castigaban con fuego consumidor si no. Nuevos creados en lugar de los ángeles consumidos}

\par 1 R. Ismael dijo: Metatrón, el Ángel, el Príncipe de la Presencia, me dijo: Cuando los ángeles ministradores dicen «Santo» ante el Santo, bendito sea Él, de la manera apropiada, entonces los siervos de Su Trono, los servidores de Su Gloria, salen con gran alegría de debajo del Trono de Gloria.

\par 2 Y todos llevan en sus manos, cada uno de ellos, mil mil diez mil veces diez mil coronas de estrellas, similares en apariencia al planeta Venus, y se las ponen a los ángeles ministradores y a los grandes príncipes que pronuncian la « Santo». Tres coronas le pusieron a cada uno de ellos: una corona porque dicen «Santo», otra corona, porque dicen «Santo, Santo», y una tercera corona porque dicen «Santo, Santo, Santo, es el Señor de los Ejércitos». »,

\par 3 Y en el momento en que no pronuncian el «Santo» en el orden correcto, un fuego consumidor sale del dedo meñique del Santo, bendito sea Él, y cae en medio de sus filas y se divide en mil partes correspondientes a los cuatro campamentos de los ángeles ministradores, y las consume en un momento, como está escrito: «Un fuego va delante de él y quema a sus adversarios en derredor».

\par 4 Después de eso, el Santo, bendito sea, abre Su boca y pronuncia una palabra y crea otras en su lugar, otras nuevas como ellas. Y cada uno está ante Su Trono de Gloria, pronunciando el «Santo», como está escrito (Lam. iii.3): «Son nuevos cada mañana; grande es tu fidelidad».


\chapter{41}

\par \textit{Metatrón muestra a R. Ismael las letras grabadas en el Trono de Gloria mediante las cuales ha sido creado todo lo que hay en el cielo y en la tierra}

\par 1 R. Ismael dijo: Metatrón, el Ángel, el Príncipe de la Presencia, me dijo: Ven y contempla las letras con las que fueron creados el cielo y la tierra,
\par las letras con las que se crearon las montañas y las colinas,
\par las letras con las que se crearon los mares y los ríos,
\par las letras con las que se crearon los árboles y las hierbas,
\par las letras con las que se crearon los planetas y las constelaciones,
\par las letras mediante las cuales fueron creados el globo de la luna y el globo del sol, Orión, las Pléyades y todas las diferentes luminarias de Raqia.

\par 2 las letras mediante las cuales se crearon el Trono de Gloria y las Ruedas del Merkaba, las letras mediante las cuales se crearon las necesidades de los mundos,

\par 3 las letras por las cuales fueron creadas la sabiduría, la inteligencia, la ciencia, la prudencia, la mansedumbre y la justicia, con las que se sustenta el mundo entero.

\par 4 Y caminé a su lado y él me tomó de su mano y me levantó sobre sus alas y me mostró aquellas letras, todas ellas, que están grabadas con un estilo llameante en el Trono de Gloria: y de ellas salen chispas. y cubrir todas las cámaras de Araboth.


\chapter{42}

\par \textit{Ejemplos de polos opuestos mantenidos en equilibrio por varios Nombres Divinos y otras maravillas similares}

\par 1 R. Ismael dijo: Metatrón, el Ángel, el Príncipe de la Presencia, me dijo: Ven y te mostraré dónde están suspendidas las aguas en las alturas, dónde arde el fuego en medio del granizo, dónde Los relámpagos brillan en medio de las montañas nevadas, donde los truenos rugen en las alturas celestiales, donde una llama arde en medio del fuego ardiente y donde las voces se hacen oír en medio de los truenos y los terremotos.

\par 2 Entonces pasé a su lado y él me tomó de la mano, me levantó sobre sus alas y me mostró todas esas cosas. Vi las aguas suspendidas en lo alto en 'Araboth Raqia por (fuerza de) el nombre YAH 'EH YE 'ASHER 'EH YE (Jah, yo soy el que soy), y sus frutos descendiendo del cielo y regando la faz del mundo, como está escrito: «(Él riega los montes desde sus aposentos:) la tierra está saciada del fruto de tu trabajo».

\par 3 Y vi fuego, nieve y granizo que estaban mezclados entre sí y sin embargo no habían sido dañados, por (la fuerza de) el nombre 'ESH 'OKELA (fuego consumidor), como está escrito: «Porque el Señor, tu Dios, es fuego consumidor».

\par 4 Y vi relámpagos que caían de las montañas de nieve y, sin embargo, no fueron dañados (apagados), por (la fuerza de) el nombre YAH SUR 'OLAMIM (Jah, la roca eterna), como está escrito: «Porque en Jah, YHWH, la roca eterna».

\par 5 Y vi truenos y voces que rugían en medio de llamas de fuego y no fueron dañadas (silenciadas), por (la fuerza de) el nombre'EL-SHADDAI RABBA (el Gran Dios Todopoderoso) como está escrito: « Yo soy Dios Todopoderoso».

\par 6 Y vi una llama (y) un resplandor (llamas resplandecientes) que ardía y brillaba en medio del fuego ardiente, y sin embargo no fue dañado (devorado), por (la fuerza de) el nombre YAD AL KES YAH ( la mano sobre el Trono del Señor) como está escrito: «Y dijo: porque la mano está sobre el Trono del Señor».

\par 7 Y vi ríos de fuego en medio de ríos de agua y no fueron dañados (apagados) por (la fuerza de) el nombre 'OSE SHALOM (Hacedor de Paz) como está escrito: «Él hace la paz en su lugares altos». Porque él hace las paces entre el fuego y el agua, entre el granizo y el fuego, entre el viento y la nube, entre el terremoto y las chispas.

\chapter{43}

\par \textit{Metatron muestra a R. Ismael la morada de los espíritus no nacidos y de los espíritus de los justos muertos}

\par 1 R. Ismael dijo: Metatrón me dijo: Ven y te mostraré dónde están los espíritus de los justos que han sido creados y han regresado, y los espíritus de los justos que aún no han sido creados.

\par 2 Y me levantó a su lado, me tomó de su mano y me levantó cerca del Trono de Gloria junto al lugar de la Shekina; y me reveló el Trono de Gloria, y me mostró los espíritus que habían sido creados y habían regresado: y volaban sobre el Trono de Gloria delante del Santo, bendito sea.

\par 3 Después de eso fui a interpretar el siguiente versículo de la Escritura y encontré en lo que está escrito: «porque el espíritu se vistió delante de mí, y las almas que yo hice» que («porque el espíritu se vistió delante de mí») significa los espíritus que han sido creados en la cámara de creación de los justos y que han regresado ante el Santo, bendito sea; (y las palabras:) y las almas que he hecho se refieren a los espíritus de los justos que aún no han sido creados en la cámara (GUPH).


\chapter{44}

\par \textit{Metatron muestra a R. Ismael la morada de los malvados y los intermedios en el Seol (vss.-6) Los Patriarcas oran por la liberación de Israel (vss. 7-10)}

\par 1 R. Ismael dijo: Metatrón, el Ángel, el Príncipe de la Presencia, me dijo: Ven y te mostraré los espíritus de los malvados y los espíritus de los intermedios donde están, y los espíritus de los intermedios, adónde descienden, y los espíritus de los impíos, adónde descienden.

\par 2 Y me dijo: Los espíritus de los malvados descienden al Seol por manos de dos ángeles de destrucción: zaaphiel y simkiel son sus nombres.

\par 3 Simkiel es designado sobre el intermediario para apoyarlos y purificarlos debido a la gran misericordia del Príncipe del Lugar (Maqom). Za'aphiel es designado sobre los espíritus de los malvados para expulsarlos de la presencia del Santo, bendito sea Él, y del esplendor de la Shekina al She'ol, para ser castigados en el fuego de la Gehenna con duelas de carbón ardiendo.

\par 4 Y yo fui a su lado, y él me tomó de la mano y me las mostró todas con sus dedos.

\par 5 Y vi el aspecto de sus rostros (y he aquí que era) como el aspecto de hijos de hombres, y sus cuerpos como de águilas. Y no sólo eso, sino que (además) el color del semblante de los intermedios era como gris pálido a causa de sus obras, porque hay manchas sobre ellos hasta que hayan sido limpiados de su iniquidad en el fuego.

\par 6 Y el color de los malvados era como el fondo de una olla a causa de la maldad de sus acciones.

\par 7 Y vi los espíritus de los Patriarcas Abraham Isaac y Jacob y el resto de los justos a quienes sacaron de sus tumbas y ascendieron al Cielo (Raqia). Y estaban orando ante el Santo, bendito sea, diciendo en su oración: «¡Señor del Universo! ¿Hasta cuándo te sentarás en (tu) trono como un doliente en los días de su duelo con tu mano derecha detrás de ti y no entregarás a tus hijos y revelarás tu Reino en el mundo? ¿Y por cuánto tiempo no tendrás piedad de tus hijos que están ¿Hizo esclavos entre las naciones del mundo? ¿Ni sobre tu mano derecha que está detrás de ti, con la cual extendiste los cielos y la tierra y los cielos de los cielos? ¿Cuándo tendrás compasión?

\par 8 Entonces el Santo, bendito sea, respondió a cada uno de ellos, diciendo: «Ya que estos malvados pecan tal y cual cosa, y transgreden con tales y tales transgresiones contra mí, ¿cómo podría yo entregar mi gran Mano Derecha en el caída por sus manos (causada por ellos).

\par 9 En ese momento Metatrón me llamó y me habló: «¡Mi siervo! ¡Toma los libros y lee sus malas acciones! Inmediatamente tomé los libros y leí sus acciones y se encontraron 36 transgresiones (anotadas) con respecto a cada malvado y además, que habían transgredido todas las letras de la Torá, como está escrito: «Sí, todo Israel». he transgredido tu Ley». No está escrito 'al torateka sino 'et torateka, porque han transgredido desde y Aleph hasta Taw, han transgredido 40 estatutos por cada letra.

\par 10 Inmediatamente Abraham, Isaac y Jacob lloraron. Entonces el Santo, bendito sea, les dijo: «¡Abraham, mi amado, Isaac, mi Elegido, Jacob, mi primogénito! ¿Cómo puedo ahora librarlos de entre las naciones de ¿el mundo?« Y en seguida Miguel, el Príncipe de Israel, clamó y lloró a gran voz y dijo: «¿Por qué estás lejos, oh Señor? »


\chapter{45}

\par \textit{Metatron muestra a R. Ismael eventos pasados ​​y futuros registrados en la Cortina del Trono}

\par 1 R. Ismael dijo: Metatrón me dijo: Ven y te mostraré la Cortina de MAQOM (la Divina Majestad) que se extiende ante el Santo, bendito sea, (y) en la que están grabadas todas las generaciones. del mundo y todas sus obras, tanto las que han hecho como las que harán hasta el fin de todas las generaciones.

\par 2 Y fui, y él me lo mostró señalándolo con sus dedos, como un padre que enseña a sus hijos las letras de la Torá. Y vi cada generación, los gobernantes de cada generación,
\par y los jefes de cada generación,
\par los pastores de cada generación,
\par los opresores (impulsores) de cada generación,
\par los guardianes de cada generación,
\par los azotadores de cada generación,
\par los supervisores de cada generación,
\par los jueces de cada generación,
\par los funcionarios judiciales de cada generación,
\par los maestros de cada generación,
\par los partidarios de cada generación,
\par los jefes de cada generación,
\par los presidentes de academias de cada generación,
\par los magistrados de cada generación,
\par los príncipes de cada generación,
\par los consejeros de cada generación,
\par los nobles de cada generación,
\par y los hombres valientes de cada generación,
\par los mayores de cada generación,
\par y los guías de cada generación.

\par 3 Y vi a Adán, a su generación, sus obras y sus pensamientos,
\par Noé y su generación, sus obras y sus pensamientos,
\par y la generación del diluvio, sus obras y sus pensamientos,
\par Sem y su generación, sus obras y sus pensamientos,
\par Nimrod y la generación de la confusión de lenguas, y su
\par generación, sus obras y sus pensamientos,
\par Abraham y su generación, sus obras y sus pensamientos,
Isaac y su generación, sus obras y sus pensamientos,
Ismael y su generación, sus obras y sus pensamientos,
\par Jacob y su generación, sus obras y sus pensamientos,
\par José y su generación, sus obras y sus pensamientos,
\par las tribus y su generación, sus obras y sus pensamientos,
Amram y su generación, sus obras y sus pensamientos,
Moisés y su generación, sus obras y sus pensamientos,

\par 4 Aarón y Miriam sus obras y sus hechos,
\par los príncipes y los ancianos, sus obras y hechos,
\par Josué y su generación, sus obras y sus hechos,
\par los jueces y su generación, sus obras y hechos,
\par Elí y su generación, sus obras y sus hechos,
\par Finees, sus (?) obras y hechos,
\par Elcana y su generación, sus obras y sus hechos,
\par Samuel y su generación, sus obras y sus hechos,
los reyes de Judá con sus generaciones, sus obras y sus hechos,
los reyes de Israel y sus generaciones, sus obras y sus hechos,
Los príncipes de Israel, sus obras y sus hechos; los príncipes de las naciones del mundo, sus obras y sus hechos,
los jefes de los concilios de Israel, sus obras y sus hechos;
\par los jefes de (los concilios en) las naciones del mundo, sus generaciones, sus obras y sus hechos;
los gobernantes de Israel y su generación, sus obras y sus hechos;
los nobles de Israel y su generación, sus obras y sus hechos; los nobles de las naciones del mundo y su(s) generación(es), sus obras y sus hechos;
Los hombres famosos en Israel, su generación, sus obras y sus hechos;
los jueces de Israel, su generación, sus obras y sus hechos;
\par los jueces de las naciones del mundo y su generación, sus obras y sus hechos;
los maestros de los niños de Israel, sus generaciones, sus obras y sus hechos; los maestros de niños en las naciones del mundo, sus generaciones, sus obras y sus hechos;
\par los consejeros (intérpretes) de Israel, su generación, sus obras y sus hechos; los consejeros (intérpretes) de las naciones del mundo, su generación, sus obras y sus hechos;
todos los profetas de Israel, su generación, sus obras y sus hechos; todos los profetas de las naciones del mundo, su generación, sus obras y sus hechos;

\par 5 y todas las peleas y guerras que las naciones del mundo libraron contra el pueblo de Israel durante su reinado.

Y vi al Mesías, hijo de José, y a su generación, y las obras y hechos que harán contra las naciones del mundo. Y vi al Mesías, hijo de David, y a su generación, y todas las peleas y guerras, y sus obras y sus hechos que harán con Israel para bien y para mal. Y vi todas las peleas y guerras que Gog y Magog pelearán en los días del Mesías, y todo lo que el Santo, bendito sea, hará con ellos en el tiempo venidero.

\par 6 Y todo el resto de todos los jefes de las generaciones y todas las obras de las generaciones tanto en Israel como en las naciones del mundo, tanto lo que se ha hecho como lo que se hará en el futuro por todas las generaciones hasta el fin de los tiempos, (todos) fueron grabados en la Cortina de MAQOM. Y vi todas estas cosas con mis ojos; y después de haberlo visto, abrí la boca en alabanza a MAQOM (la Divina Majestad) (diciendo así): «Porque la palabra del Rey tiene poder (y ¿quién puede decirle: ¿Qué haces?) Quien guarda los mandamientos deberá no conozcas el mal». Y dije: «¡Oh Señor, cuán múltiples son tus obras!»



\chapter{46}

\par \textit{El lugar de las estrellas mostrado a R. Ismael}

\par 1 R. Ismael dijo: Metatrón me dijo: (Ven y te mostraré) el espacio de las estrellas que están paradas en Raqta' noche tras noche por temor al Todopoderoso (MAQOM) y (te mostraré) adónde van y dónde se encuentran.

\par 2 Caminé a su lado, y él me tomó de la mano y me señaló todo con sus dedos. Y estaban de pie sobre chispas de llamas alrededor del Merkaba del Todopoderoso (MAQOM). ¿Qué hizo Metatrón? En ese momento dio una palmada y los ahuyentó de su lugar. Inmediatamente volaron con alas llameantes, se elevaron y huyeron de los cuatro lados del Trono del Merkaba, y (mientras volaban) él me dijo los nombres de cada uno de ellos. Como está escrito: «Él cuenta el número de las estrellas; él les da a todos sus nombres», enseñando que el Santo, bendito sea, ha dado un nombre a cada uno de ellos.

\par 3 Y todos entran en orden contado bajo la guía de (lit. a través de, por las manos de) Rahatiel a Raqia ha-shSHamayim para servir al mundo. Y salen en orden contado a alabar al Santo, bendito sea, con cánticos e himnos, según está escrito: «Los cielos declaran la gloria de Dios».

\par 4 Pero en el futuro el Santo, bendito sea, los creará de nuevo 9, como está escrito: «Son nuevos cada mañana». Y abren la boca y pronuncian una canción. ¿Cuál es el cántico que pronuncian?: «Cuando considero tus cielos».

\chapter{47}

\par \textit{Metatron muestra a R. Ismael los espíritus de los ángeles castigados}

\par 1 R. Ismael dijo: Metatrón me dijo: Ven y te mostraré las almas de los ángeles y los espíritus de los siervos ministrantes cuyos cuerpos han sido quemados en el fuego de MAQOM (el Todopoderoso) que sale de su dedo meñique. Y han sido convertidos en carbones encendidos en medio del río de fuego (Nehar di-Nur). Pero sus espíritus y sus almas están detrás de la Shekina.

\par 2 Cada vez que los ángeles ministrantes pronuncian una canción en el momento equivocado o no designado para ser cantada 5 son quemados y consumidos por el fuego de su Creador y por una llama de su Hacedor, [A: en los lugares (cámaras) del torbellino , porque sopla sobre ellos y los empuja] [E: en su lugar (= en el lugar); y un torbellino sopla sobre ellos y los arroja ] al Nehar di-Nur; y allí se convierten en numerosas montañas de carbón ardiendo. Pero su espíritu y su alma regresan a su Creador, y todos están detrás de su Maestro.

\par 3 Y yo fui a su lado y él me tomó de su mano; y me mostró todas las almas de los ángeles y los espíritus de los sirvientes que estaban parados detrás de la Shekina sobre las alas del torbellino y los muros de fuego que los rodeaban.

\par 4 En ese momento Metatrón me abrió las puertas de los muros dentro de los cuales estaban parados detrás de la Shekina, y levanté mis ojos y los vi, y he aquí, la semejanza de cada uno era como (la de) ángeles y sus alas como de pájaro, hechas de llamas, obra de fuego ardiente. En ese momento abrí mi boca en alabanza a MAQOM y dije: «Cuán grandes son tus obras, oh Señor».

\chapter{48a}

\par \textit{Metatron muestra a R. Ismael la Mano Derecha del Altísimo, ahora inactiva detrás de Él, pero en el futuro destinada a obrar la liberación de Israel}

\par 1 R. Ismael dijo: Metatrón me dijo: Ven y te mostraré la Mano Derecha de MAQOM, puesta detrás (de Él) debido a la destrucción del Templo Sagrado; del cual brillan toda clase de esplendor y luz, y por el cual se crearon los cielos; y a quienes ni siquiera a los Serafines y los 'Ophannim se les permite (contemplar), hasta que llegue el día de la salvación.

\par 2 Y fui a su lado y él me tomó de su mano y me mostró (la Mano Derecha de MAQOM) y con toda clase de alabanza, regocijo y canto: y ninguna boca puede decir su alabanza, y ningún ojo puede contemplarla , por su grandeza, dignidad, majestad, gloria y belleza.

\par 3 Y no sólo eso 4, sino que todas las almas de los justos que son considerados dignos de contemplar 4a la alegría de Jerusalén, están junto a ella, alabando y orando delante de ella tres veces al día, diciendo: «Despierta, despierta». vístete de fuerza, oh brazo de Jehová» según está escrito: «Hizo ir su brazo glorioso a la diestra de Moisés».

\par 4 En ese momento la Mano Derecha de MAQOM estaba llorando. Y de sus cinco dedos salieron cinco ríos de lágrimas y cayeron en el gran mar y sacudieron al mundo entero, como está escrito: «La tierra completamente quebrantada, la tierra limpiamente disuelta, la tierra se conmueve en gran manera, la tierra tambaleará como un ebrio y será movida de un lado a otro como una choza», cinco veces correspondientes a los dedos de su Gran Mano Derecha.

\par 5 Pero cuando el Santo, bendito sea, vea que no hay ningún hombre justo en la generación, ni ningún hombre piadoso (jasid) en la tierra, ni justicia en manos de los hombres; y (que no hay) ningún hombre como Moisés, ni ningún intercesor como Samuel, que pudiera orar ante MAQOM por la salvación y por la liberación, y por Su Reino, para que sea revelado en el mundo entero; y por su gran diestra, que la volvió a poner delante de sí para obrar por ella gran salvación a Israel,

\par 6 Entonces, inmediatamente el Santo, bendito sea, recordará su propia justicia, favor, misericordia y gracia, y librará por sí mismo su gran brazo, y su justicia lo sostendrá. Según está escrito: «Y vio que no había ningún hombre» – (es decir:) como Moisés, que oró innumerables veces por Israel en el desierto y desvió los decretos (Divinos) de ellos – «y se maravilló, que no había intercesor» – como Samuel que suplicó al Santo, bendito sea, y lo llamó y él le respondió y cumplió su deseo, incluso si no era adecuado (de acuerdo con el plan Divino), según como está escrito: «¿No es hoy la cosecha del trigo? Clamaré al Señor».

\par 7 Y no sólo eso, sino que se unió a Moisés en todo lugar, como está escrito: «Moisés y Aarón entre sus sacerdotes». Y nuevamente está escrito: «Aunque Moisés y Samuel estuvieron delante de mí»: «Mío propio brazo me trajo la salvación».

\par 8 Dijo el Santo, bendito sea He en aquella hora: «¿Hasta cuándo esperaré a que los hijos de Mento obren la salvación según su justicia en mi brazo? Por mi propio bien y por mi mérito y mi justicia entregaré mi brazo y por él redimiré a mis hijos de entre las naciones del mundo». Como está escrito: «Por amor a mí lo haré. ¿Cómo debería ser profanado mi nombre?».

\par 9 En ese momento el Santo, bendito sea, revelará Su Gran Brazo y lo mostrará a las naciones del mundo: porque su longitud es como la longitud del mundo y su anchura es como la anchura del mundo. Y la apariencia de su esplendor es semejante al esplendor del sol en su poder, en el solsticio de verano.

\par 10 Entonces Israel será salvo de entre las naciones del mundo. Y se les aparecerá el Mesías y los hará subir a Jerusalén con gran alegría. Y no sólo eso sino que [R: comerán y beberán porque glorificarán el Reino del Mesías, de la casa de David, en las cuatro partes del mundo. Y las naciones del mundo no prevalecerán contra ellos, ] [E: Israel vendrá de las cuatro partes del mundo y comerá con el Mesías. Pero las naciones del mundo no comerán con ellos, ] como está escrito: «Jehová ha desnudo su santo brazo ante los ojos de todas las naciones; y todos los confines de la tierra verán la salvación de nuestro Dios». Y nuevamente: «Sólo el Señor lo guió, y no había ningún dios extraño con él».: «Y el Señor será rey sobre toda la tierra».



\chapter{48b}

\par \textit{Los Nombres Divinos que salen del Trono de Gloria, coronados y escoltados por numerosas huestes angelicales a través de los cielos y de regreso al Trono: los ángeles cantan lo 'Santo' y lo 'Bendito'}

\par 1 [AEFGH: Estos son los nombres del Santo, bendito sea] [K: Estos son los setenta y dos nombres escritos en el corazón del Santo, bendito sea: SS, SeDeQ (rectitud), SaHI 'eL SUR, SBI, SaDdlQ {justo}, S'Ph, SHN, SeBa'oTh (Señor de los Ejércitos), ShaDdaY (Dios Todopoderoso), 'eLoHIM (Dios), YHWH, SH, DGUL, WDOM, SSS'', 'YW, y T, 'HW, HB, YaH, HW, WWW, SSS, PPP, NN, HH, HaY (vivo), HaY, ROKeB 'aRaBOTh (cabalgando sobre el Araboth), YH, HH, WH, MMM, NNN, HWW, YH, YHH, HPhS, H'S, W, S'Z' QQQ (Santo, Santo, Santo), QShR, BW, ZK, GINUR, GINURYa', T, YOD, 'aLePh, H'N, P 'P, R'W, YYW, YYW, BBB, DDD, TTT, KKK, KLL, SYS, TT BShKMLW (= bendito sea el Nombre de Su glorioso reino por los siglos de los siglos), completado para MeLeK Ha'OLaM (el Rey del Universo), BRH LB' (el comienzo de la Sabiduría para los hijos de los hombres), BNLK W''Y (bendito sea Aquel que da fuerzas a los cansados ​​y aumenta las fuerzas a los que no tienen fuerzas.) ]que salen (adornado) con numerosas coronas de fuego, con numerosas coronas de llama, con numerosas coronas de jashmal, con numerosas coronas de relámpagos delante del Trono de Gloria. Y con ellos (hay) milcientos de poder (es decir, ángeles poderosos) que los escoltan como un rey [AE: con honor y columnas de fuego y nube(s), y columnas de fuego, y con relámpagos de resplandor y con semejanza de (el) jashmal.] [FG: con temblor y pavor, con temor y escalofrío, con honor y majestad y temor, con terror, con grandeza y dignidad, con gloria y fuerza, con entendimiento y conocimiento y con una columna de fuego y una columna de llama y de relámpagos, y su luz es como relámpagos de luz, y con semejanza de jashmal. ]

\par 2 Y les dan gloria y responden y claman ante ellos: Santo, Santo, Santo. Y los hacen rodar por todos los cielos como príncipes poderosos y honrados. Y cuando los traen a todos de regreso al lugar 0/el Trono de Gloria, entonces todos los Chayyoth junto al Merkaba abren su boca en alabanza de Su glorioso nombre, diciendo: «Bendito sea el nombre de Su glorioso reino por los siglos de los siglos. ».



\chapter{48c}

\par \textit{Una pieza de Enoc-Metatrón}

\par 1 [AEFGH: Aleph lo hizo fuerte, lo tomé, lo nombré: (es decir) Metatrón, mi siervo que es uno (único) entre todos los hijos del cielo. Lo hice fuerte en la generación del primer Adán. Pero cuando vi que los hombres de la generación del diluvio eran corruptos, entonces fui y quité mi Shekina de entre ellos. Y lo levantó en alto con sonido de trompeta y con voz de trompeta, como está escrito: «Dios ha subido con voz de trompeta, el Señor con voz de trompeta». ] [K: «Lo agarré, lo tomé y lo nombré» - es decir, Enoc, el ]


\par 2 [AEFGH: «Y lo tomé»: (es decir) Enoc, el hijo de Jared, de entre ellos. Y lo levanté con sonido de trompeta y con teru'a (grito) a los altos cielos, para que fuera mi testigo junto con los Chayyoth, por la Merkaba en el mundo venidero. ] [K: hijo de Jared, cuyo nombre es Metatrón (2) y lo tomé de entre los hijos de los hombres (5) y le hice un Trono frente a mi Trono. ¿Cuál es el tamaño de ese Trono? Setenta mil parasangas (todos) de fuego. 9 Le encomendé 70 ángeles correspondientes a las naciones (del mundo) y le entregué a su cargo toda la casa de arriba y de abajo. (7) Y le encomendé Sabiduría e Inteligencia más que (a) todos los ángeles. Y llamé su nombre «el yah menor», ​​cuyo nombre es por Gematria 71. Y dispuse para él todas las obras de la Creación. E hice que su poder trascendiera (es decir, le hice poder más que) todos los ángeles ministradores. (Termina K).]

\par 3 [AEFGH: Lo nombré sobre todos los tesoros y almacenes que tengo en cada cielo. Y entregué en su mano las llaves de cada uno. ] [Lm (comienza aquí): Encomendó a Metatrón, es decir, Enoc, el hijo de Jared, todos los tesoros. Y lo puse sobre todos los almacenes que tengo en cada cielo. Y encomendé en sus manos las llaves de cada almacén celestial. ]

\par 4 [AEFGH: Hice (de) príncipe sobre todos los príncipes y ministro del Trono de Gloria (y) los Salones de 'Araboth: para abrirme sus puertas, y (de) el Trono de Gloria, para exaltarlo y arreglarlo; (y lo designé sobre) el Santo Chayyoth para que les coronara la cabeza; los majestuosos Ophannim para coronarlos de fuerza y ​​gloria; los honrados Querubines, para vestirlos de majestad; sobre las chispas radiantes, para hacerlas brillar con esplendor y brillo; sobre los Serafines llameantes, para cubrirlos de alteza; los Jashmallim de luz, para hacerlos radiantes con luz y preparar el asiento para cada mañana] [Lm: Lo hice (de) él príncipe sobre todos los príncipes, y lo hice (de) ministro de mi Trono de Gloria, para proveer y organizar el Santo Chayyoth, para coronarles coronas (coronarlos con coronas), vestirlos con honor y majestad y prepararles un asiento ] [A: mientras me siento en el Trono de Gloria. Y para ensalzar y magnificar mi gloria en la altura de mi poder; (y le he encomendado) los secretos de arriba y los secretos de abajo (secretos celestiales y secretos terrenales). ] [FGH: cuando esté sentado en mi Trono en gloria y dignidad para que él vea mi gloria en la altura de mi poder, en los secretos de arriba y en los secretos de abajo. ] [Lm: cuando está sentado en su trono para magnificar su gloria en las alturas]

\par 5 [AFGH: Lo hice más alto que todos. La altura de su estatura, en medio de todos (los que son) de alta estatura (hice) setenta mil parasangas. Hice grande su Trono con la majestad de mi Trono. Y aumenté su gloria con el honor de mi gloria.] [Lm: La altura de su estatura entre todos los (que son) de alta estatura (es) setenta mil parasangs. Y hice grande su gloria como la majestad de mi gloria]

\par 6 [AFGH: Transformé su carne en antorchas de fuego, y todos los huesos de su cuerpo en carbones encendidos; e hice aparecer sus ojos como el relámpago, y la luz de sus cejas como la luz incorruptible. Hice su rostro resplandeciente como el esplendor del sol, y sus ojos como el esplendor del Trono de Gloria. ] [Lm: y el brillo de sus ojos como el esplendor del Trono de Gloria]

\par 7 [AFGH: Hice honor y majestad su vestimenta, belleza y alteza su manto que lo cubre y una corona real de 500 por (veces) parasangs (su) diadema. ] [Lm: su vestidura de honor y majestad, su corona real de 500 por 500 parasanga.] [AFGHLm: Y lo puse de mi honor, de mi majestad y del esplendor, de mi gloria que está sobre mi Trono de Gloria. Lo llamé el menor yhwh, el Príncipe de la Presencia, el Conocedor de los Secretos: porque cada secreto5a le revelé como a un padre y todos los misterios le declaré con rectitud. ]

\par 8 Coloqué su trono a la puerta de mi salón para que se siente y juzgue a la casa celestial en las alturas. Y puse a cada príncipe delante de él, para recibir de él autoridad, para realizar su voluntad.

\par 9 Setenta nombres tomé de (mis) nombres y con ellos lo llamé para realzar su gloria.

\par 10 Setenta príncipes se entregaron en su mano para mandarles mis preceptos y mis palabras en todos los idiomas; [AFGH: humillar con su palabra a los soberbios hasta el suelo, y exaltar con la expresión de sus labios a los humildes hasta lo alto; para herir a los reyes con su discurso, para desviar a los reyes de sus caminos, para establecer (los) gobernantes sobre su dominio como está escrito: «y él cambia los tiempos y las estaciones, y da sabiduría a todos los sabios del mundo y entendimiento (y) conocimiento a todos los que entienden conocimiento», como está escrito: «y conocimiento a los que saben entendimiento», para revelarles los secretos de mis palabras y enseñarles el decreto de mi justo juicio, tal como es. escrito: ] [Lm: y para humillar a los soberbios hasta el suelo y exaltar a los humildes a la altura y herir a los reyes y humillar a los gobernantes y establecer reyes y gobernantes y él cambia los tiempos y las estaciones, destituye a los reyes y establece reyes, da sabiduría a los sabios y conocimiento a los que saben entender y lo nombré para revelar secretos y enseñar juicio y justicia, ] «así será mi palabra que sale de mi boca; no volverá a mí vacía sino que cumplirá (lo que yo quiera), "E'ie'seh y (lo lograré) no está escrito aquí, sino ll asah y (él lo logrará), es decir, que cualquier palabra y cualquier cosa La expresión surge ante el Santo, bendito sea, Metatrón se levanta y la lleva a cabo. Y establece los decretos del Santo, bendito sea. (Aquí termina la versión Lm del fragmento c.)

\par 11 [«Y él hará prosperar lo que yo envié». 'Asli'a'h (haré prosperar) no está escrito aquí, sino w'e'hisli'a'h (y él hará prosperar), enseñando que cualquier decreto que salga de delante del Santo, Bendito sea Él, respecto a un hombre, tan pronto como se arrepiente, no lo ejecutan (sobre él), sino sobre otro, malvado, como está escrito: «El justo es librado de la angustia, y el impío viene». en su lugar».]

\par 12 Y no sólo eso, sino que Metatrón se sienta tres horas cada día en los altos cielos, y reúne todas las almas de aquellos muertos que murieron en el vientre de su madre, y de los lactantes que murieron en los pechos de su madre, y de los eruditos que murieron en los cinco libros de la Ley. Y los trae bajo el Trono de Gloria y los coloca en compañías, divisiones y clases alrededor de la Presencia: y les enseña la Ley, y (los libros) de la Sabiduría, la Hagadá y la Tradición y termina (completa) su instrucción (educación). [para ellos]. Como está escrito: «¿A quién enseñará conocimiento? ¿Y a quién hará entender la tradición? los que son destetados de la leche y sacados del pecho».


\chapter{48d}

\par \textit{Los nombres de Metatrón. Los tesoros de la Sabiduría se abrieron a Moisés en el monte Sinaí. Los ángeles protestan contra Metatrón por revelar los secretos a Moisés y son respondidos y reprendidos por Dios. La cadena de la tradición y el poder de los misterios transmitidos para curar enfermedades}

\par 1 Setenta nombres tiene Metatrón que el Santo, bendito sea, tomó de su propio nombre y se lo puso. Y estos son:

\par YeHOEL YaH, YeHOEL, YOPHIEL y Yophphiel, y 5 'APHPHIEL y MaRGeZIEL,GIPpUYEL, Pa'aZIEL, A'aH, PeRIEL, TaTRIEL, TaBKIEL, 'W, YHWH, DH WHYH,'eBeD, DiBbURIEL, 'aPH' aPIEL, SPPIEL, PaSPaSIEL, SeNeGRON, MeTaTRON, SOGDIN,'A- DRIGON, 'ASUM, SaQPaM, SaQTaM, MIGON, MITTON, MOTTRON, ROSPHIM, QINOTh,ChaTaTYaH, DeGaZYaH, PSPYaH, BSKNYH, MZRG, BaRaD, MKRKK, MSPRD, ChShG, ChShB, MNRTTT, BSYRYM, MITMON, TITMON, PiSQON, SaPhSaPhYaH, ZRCh, ZRChYaH, BeYaH, HBH BeYaH, PeLeT, PLTYaH, RaBRaBYaH, ChaS, ChaSYaH, TaPhTaPhYaH,TaMTaMYaH, SeHaSYaH, IR'URYaH, L'aLYaH, BaZRIDYa h , SaTSaTKYaH, SaSDYaH, RaZRaZYAH, BaZRaZYaH, aRIMYaH, SBHYaH, SBIBKHYH, StMKaM, YaHSeYaH,SSBIBYaH, SaBKaSBeYaH, QeLILQaLYaH, KIHHH, HHYH, ZoWH, WHYH, ZaKklKYaH, TUTRISYaH, SURYaH, ZeH, PeNIRHYaH , 'ZrH, GaL RaZaYYa, MaMLIKYaH , TTYaH, eMeQ, QaMYaH, MeKaPpeR YaH, PeRISHYaH, SePhaM, GBIR, GiBbORYaH, GOR, GOR YaH, ZIW, OKBaR, el YHWH MENOR, tras el nombre de su Maestro, «porque mi nombre está en él», RaBIBIEL, TUMIEL , Segansakkiel, el Príncipe de la Sabiduría.

\par 2 ¿Y por qué se le llama Sagnesakiel? Porque en su mano están confiados todos los tesoros de la sabiduría.

\par 3 Y todas ellas fueron abiertas a Moisés en el Sinaí, de modo que las aprendió durante los cuarenta días, mientras estuvo de pie (permaneciendo): la Torá en los setenta aspectos de las setenta lenguas, los Profetas en los setenta aspectos de las setenta lenguas, las Escrituras en los setenta aspectos de las setenta lenguas, las Halakas en los setenta aspectos de las setenta lenguas, las Tradiciones en los setenta aspectos de las setenta lenguas, las Haggadas en los setenta aspectos de las setenta lenguas y las Toseftas en los setenta aspectos de las setenta lenguas.

\par 4 Pero cuando pasaron los cuarenta días, en un momento se olvidó de todos ellos. Entonces el Santo, bendito sea, llamó a Yephiphyah, el Príncipe de la Ley, y (a través de él) fueron entregados a Moisés como un regalo. Como está escrito: «y el Señor me los dio». Y después de eso se quedó con él. ¿Y de dónde sabemos que permaneció (en su memoria)? Porque está escrito: «Acordaos de la ley de Moisés mi siervo,12 que le encargué en Horeb mis estatutos y mis juicios para todo Israel». 'La Ley de Moisés': es decir, la Torá, los Profetas y los Escritos, 'estatutos'; es decir, las Halakas y Tradiciones, 'juicios'; Esos son los Haggadas y los Toseftas. Y todos ellos fueron entregados a Moisés en lo alto del Sinaí,

\par 5 Estos setenta nombres (son) un reflejo de los Nombres Explícitos en el Merkaba que están grabados en el Trono de Gloria. Porque el Santo, bendito sea Él, tomó de Su Nombre(s) Explícito(s) y puso el nombre de Metatrón: Setenta Nombres Suyos por los cuales los ángeles ministradores llaman al Rey de reyes de reyes, bendito sea Él, en las alturas. cielos, y veintidós letras que están en el anillo de su dedo con las que están sellados los destinos de los príncipes de los reinos en lo alto en grandeza y poder y con las que están sellados las suertes del Ángel de la Muerte, y los destinos de todos nación y lengua.

\par 6 SaidMetatron, el Ángel, el Príncipe de la Presencia; el Ángel, el Príncipe de la Sabiduría; el Ángel, el Príncipe del Entendimiento; el Ángel, el Príncipe de los Reyes; el Ángel, el Príncipe de los Gobernantes; el ángel, el Príncipe de la Gloria; el ángel, el Príncipe de los altísimos, y de los príncipes, los exaltados, grandes y honrados, en el cielo y en la tierra:

\par 7 H, Dios de Israel, es testigo de esto: cuando revelé este secreto a Moisés, entonces todos los ejércitos de todos los cielos en lo alto se enfurecieron contra mí y me dijeron:

\par 8 ¿Por qué revelas a un hijo de hombre, nacido de mujer, corrupto e inmundo, un hombre de una gota putrefacta, este secreto, el secreto por el cual fueron creados el cielo y la tierra, el mar y la tierra seca, las montañas y las colinas, los ríos y manantiales, la Gehena de fuego y granizo, el Jardín del Edén y el Árbol de la Vida; y por el cual fueron formados Adán y Eva, y las bestias, y las bestias salvajes, y las aves del cielo, y los peces del mar, y Behemot y Leviatán, y los reptiles, los gusanos, los dragones del mar, y los reptiles de los desiertos; y la Tora y la Sabiduría y el Conocimiento y el Pensamiento y la Gnosis de las cosas de arriba y el temor del cielo. ¿Por qué revelas esto a carne y sangre? [Respuesta: ¿Has obtenido autoridad de MAQOM? Y nuevamente: ¿Has recibido permiso? Los Nombres Explícitos surgieron delante de mí] [FG: Les respondí: Porque el Santo, bendito sea, me ha dado autoridad, y además, he obtenido permiso del alto y exaltado Trono, desde el cual todos los Los Nombres Explícitos salen ] con relámpagos de fuego y chashmallim llameantes.

\par 9 Pero no se apaciguaron hasta que el Santo, bendito sea, los reprendió y los ahuyentó de delante de él con reprensión, diciéndoles: «Me deleito y he puesto mi amor en él, y les he confiado y confiado únicamente a Metatrón, mi Siervo, porque él es Uno (único) entre todos los hijos del cielo.

\par 10 Metatrón los sacó de su casa de tesoros y se los entregó a Moisés, y Moisés a Josué, y Josué a los ancianos, y los ancianos a los profetas, y los profetas a los hombres de la Gran Sinagoga, y a los hombres. de la Gran Sinagoga a Ezra y Ezra el Escriba a Hillel el mayor, y Hillel el mayor a R. Abbahu y R. Abbahu a R. Zera, y R. Zera a los hombres de fe, y los hombres de fe (los encomendó ) para advertir y sanar por medio de ellos todas las enfermedades que azotan el mundo, como está escrito: «Si escuchas atentamente la voz de Jehová tu Dios, y haces lo recto delante de sus ojos, y Si prestas atención a sus mandamientos y guardas todos sus estatutos, ninguna enfermedad de las que envié a los egipcios te enviaré a ti, porque yo soy el Señor que te sana». (Terminó y terminó. Alabado sea el Creador del mundo).




\end{document}