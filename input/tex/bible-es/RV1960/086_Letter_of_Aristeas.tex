\begin{document}

\title{Carta de Aristeas}

\chapter{1}

\par \textit{En la época del cautiverio judío en Egipto, Ptolomeo Filadelfo se revela como el primer gran bibliófilo. Desea tener todos los libros del mundo en su biblioteca; para obtener las Leyes de Moisés, ofrece intercambiar 100.000 cautivos por ese trabajo y exclama: «¡Es una pequeña bendición en verdad!»}

\par 1 YA que he reunido material para una historia memorable de mi visita a Eleazar, el sumo sacerdote de los judíos, y porque tú, Filócrates, como no pierdes la oportunidad de recordármelo, has dado gran importancia a recibir una relación de los motivos y objeto de mi misión, he intentado exponerle claramente el asunto, porque percibo que usted posee un amor natural por el aprendizaje, cualidad que es la posesión más elevada del hombre: estar constantemente intentando «añadir a su acervo de conocimientos y adquisiciones», ya sea a través del estudio de la historia o participando realmente en los acontecimientos mismos.

\par 2 Por este medio, tomando en sí los elementos más nobles, el alma se establece en la pureza, y fijando su objetivo en la piedad, la meta más noble de todas, toma de ésta su guía infalible y así adquiere un propósito definido.

\par 3 Fue mi devoción a la búsqueda de conocimientos religiosos lo que me llevó a emprender la embajada ante el hombre que he mencionado, que era tenido en la más alta estima por sus propios ciudadanos y por los demás, tanto por su virtud como por su majestad, y que tenía en su poder documentos del más alto valor para los judíos en su propio país y en tierras extranjeras para la interpretación de la ley divina, porque sus leyes están escritas en pergaminos de cuero con caracteres judíos.

\par 4 Entonces emprendí esta embajada con entusiasmo, habiendo encontrado primero la oportunidad de interceder ante el rey en favor de los cautivos judíos que habían sido transportados de Judea a Egipto por el padre del rey, cuando tomó posesión de esta ciudad por primera vez y conquistó la tierra de Egipto.

\par 5 Vale la pena que te cuente también esta historia, ya que estoy convencido de que tú, con tu disposición a la santidad y tu simpatía hacia los hombres que viven según la santa ley, escucharás con mayor facilidad el relato que me propongo exponer, ya que usted mismo ha venido últimamente a nosotros desde la isla y está ansioso de oír todo lo que tiende a edificar el alma.

\par 6 También en una ocasión anterior os envié un registro de los hechos que pensé que valía la pena contar sobre la raza judía, el registro que había obtenido de los sumos sacerdotes más eruditos de la tierra más erudita de Egipto.

\par 7 Como estás tan ansioso por adquirir el conocimiento de aquellas cosas que pueden beneficiar a la mente, siento que me corresponde impartirte toda la información que esté en mi poder.

\par 8 El mismo deber sentiría hacia todos los que tuvieran la misma disposición, pero lo siento especialmente hacia ti, que tienes aspiraciones tan nobles y que no sólo eres mi hermano en carácter, no menos que en sangre, sino también son uno conmigo también en la búsqueda del bien.

\par 9 Porque ni el placer que se obtiene del oro ni de ninguna otra posesión apreciada por las mentes superficiales confiere el mismo beneficio que la búsqueda de la cultura y el estudio que dedicamos a conseguirla.

\par 10 Pero para no cansarlos con una introducción demasiado larga, procederé inmediatamente al fondo de mi narración.

\par 11 Demetrio de Falero, presidente de la biblioteca del rey, recibió una gran suma de dinero con el fin de reunir, en la medida de lo posible, todos los libros del mundo.

\par 12 Por medio de la compra y la transcripción, cumplió, lo mejor que pudo, el propósito del rey.

\par 13 En una ocasión, cuando yo estaba presente, le preguntaron: ¿Cuántos miles de libros hay en la biblioteca? y él respondió: «Más de doscientos mil, oh rey, y en el futuro inmediato me esforzaré en reunir también el resto, para que se pueda llegar al total de quinientos mil. ¡Me han dicho que vale la pena transcribir las leyes de los judíos y merecen un lugar en su biblioteca!»

\par 14 «¿Qué te impide hacer esto?» respondió el rey. ¡Todo lo necesario se ha puesto a vuestra disposición!

\par 15 «Es necesario traducirlos», respondió Demetrio, «porque en el país de los judíos usan un alfabeto peculiar (así como los egipcios también tienen una forma especial de letras) y hablan un dialecto peculiar».

\par 16 «Se supone que deben usar la lengua siríaca, pero no es así; su idioma es bastante diferente.»

\par 17 Y cuando el rey comprendió todos los hechos del caso, ordenó que se escribiera una carta al sumo sacerdote judío para que se cumpliera su propósito (que ya se ha descrito).

\par 18 Pensando que había llegado el momento de insistir en la exigencia que muchas veces había planteado a Sosibio de Tarento y a Andrés, el jefe de la guardia personal, para la emancipación de los judíos que habían sido transportados de Judea por el padre del rey, para cuando, gracias a una combinación de buena fortuna y coraje, logró llevar a buen término su ataque contra todo el distrito de Celesiria y Fenicia, en el proceso de aterrorizar al país hasta someterlo, transportó a algunos de sus enemigos y redujo a otros al cautiverio.

\par 19 El número de los que transportó desde el país de los judíos a Egipto fue de no menos de cien mil.

\par 20 De ellos armó a treinta mil hombres escogidos y los instaló en guarniciones en las regiones rurales.

\par 21 (Y antes de ese tiempo, muchos judíos habían llegado a Egipto con los persas, y en tiempos anteriores otros habían sido enviados a Egipto para ayudar a Psamético en su campaña contra el rey de los etíopes. Pero estos no eran tan numerosos como los cautivos que llevó Ptolomeo, hijo de Lagus.)

\par 22 Como ya he dicho, Ptolomeo escogió a los mejores de ellos, a los hombres que estaban en la flor de la vida y se distinguían por su coraje, y los armó, pero la gran masa de los demás, los que eran demasiado viejos o demasiado viejos. A los jóvenes para este fin, y a las mujeres también, las redujo a esclavitud, no porque quisiera hacerlo por su propia voluntad, sino que fue obligado por sus soldados, que las reclamaban como recompensa por los servicios que habían prestado en la guerra.

\par 23 Habiendo obtenido, como ya se ha dicho, la oportunidad de asegurar su emancipación, me dirigí al rey con los siguientes argumentos. «No seamos tan irracionales como para permitir que nuestros hechos desmientan nuestras palabras.»

\par 24 «Si la ley que queremos no sólo transcribir sino también traducir pertenece a toda la raza judía, ¿qué justificación podremos encontrar para nuestra embajada mientras tantos de ellos permanecen en estado de esclavitud en vuestro país? ¿Reino?»

\par 25 «En la perfección y riqueza de tu clemencia libera a aquellos que están sujetos a una esclavitud tan miserable, ya que, como me he esforzado en descubrir, el Dios que les dio su ley es el Dios que mantiene tu reino.»

\par 26 «Adoran al mismo Dios, Señor y Creador del universo, como todos los demás hombres, como nosotros mismos, oh rey, aunque le llamemos con diferentes nombres, como Zeus 1 o Dis».

\par 27 «Este nombre le fue otorgado muy apropiadamente por nuestros primeros antepasados, para significar que Él, a través de quien todas las cosas están dotadas de vida y vienen a la existencia, es necesariamente el Jinete y Señor del Universo».

\par 28 «Dad a toda la humanidad un ejemplo de magnanimidad liberando a los que están en cautiverio.»

\par 29 Después de un breve intervalo, mientras yo ofrecía ferviente oración a Dios para que dispusiera la voluntad del rey de tal manera que todos los cautivos pudieran ser puestos en libertad (porque la raza humana, siendo creación de Dios, es influenciado e influenciado por Él.

\par 30 Por lo tanto, con muchas y diversas oraciones invoqué a Aquel que gobierna el corazón, para que el rey se sintiera obligado a conceder mi petición.

\par 31 Porque tenía grandes esperanzas en la salvación de aquellos hombres, ya que estaba seguro de que Dios concedería el cumplimiento de mi oración.

\par 32 Porque cuando los hombres por motivos puros planean alguna acción en interés de la justicia y la realización de actos nobles, Dios Todopoderoso lleva sus esfuerzos y propósitos a un resultado exitoso)—el rey levantó la cabeza y me miró con una sonrisa alegre. Su semblante preguntó: «¿Cuántos miles crees que serán?»

\par 33 Andrés, que estaba cerca, respondió: «Un poco más de cien mil».

\par 34 «¡Es un pequeño favor, en verdad», dijo el rey, «lo que Aristeas nos pide!»

\par 35 Entonces Sosibio y algunos otros que estaban presentes dijeron: «Sí, pero será un digno tributo a tu magnanimidad que ofrezcas la emancipación de estos hombres como un acto de devoción al Dios supremo».

\par 36 «Has sido muy honrado por Dios Todopoderoso y exaltado en gloria por encima de todos tus antepasados, y es justo que le rindas la mayor ofrenda de agradecimiento que puedas».

\par 37 Muy satisfecho con estos argumentos, ordenó que se aumentara el salario de los soldados con el importe del dinero de redención, que se pagaran a los propietarios veinte dracmas por cada esclavo, que se estableciera un orden público y que se le adjunten registros de los cautivos.

\par 38 Mostró el mayor entusiasmo en el negocio, porque fue Dios quien hizo que nuestro propósito se cumpliera en su totalidad y lo obligó a redimir no sólo a los que habían entrado en Egipto con el ejército de su padre, sino también a todos los que habían venido antes de ese momento o había sido introducido posteriormente en el reino.

\par 39 Le dijeron que el dinero del rescate excedería los cuatrocientos talentos.

\par 40 Creo que será útil insertar una copia del decreto, porque de esta manera se hará más clara y manifiesta la magnanimidad del rey, a quien Dios le dio poder para salvar a tan grandes multitudes.

\par 41 El decreto del rey decía lo siguiente: «Todos los que sirvieron en el ejército de nuestro padre en la campaña contra Siria y Fenicia y en el ataque al país de los judíos y se apoderaron de los cautivos judíos y los trajeron de vuelta a la ciudad de Alejandría y la tierra de Egipto o las vendieron a otros, y de la misma manera todos los cautivos que estuvieron en nuestra tierra antes de ese tiempo o fueron traídos aquí después, todos los que posean tales cautivos están obligados a ponerlos en libertad inmediatamente. , recibiendo veinte dracmas por cabeza como dinero de rescate.

\par 42 «Los soldados recibirán este dinero como regalo añadido a su salario, el resto del tesoro del rey».

\par 43 «Pensamos que fue contra la voluntad de nuestro padre y contra toda conveniencia que los hubieran hecho cautivos y que la devastación de su tierra y el transporte de los judíos a Egipto fue un acto de desenfreno militar».

\par 44 «El botín que cayeron en manos de los soldados en el campo de batalla fue todo el botín que deberían haber reclamado».

\par 45 «Reducir además al pueblo a la esclavitud fue un acto de absoluta injusticia.»

\par 46 «Por lo tanto, ya que se reconoce que estamos acostumbrados a hacer justicia a todos los hombres y especialmente a aquellos que están injustamente en una condición de servidumbre, y ya que nos esforzamos por tratar equitativamente a todos los hombres de acuerdo con las exigencias de la justicia y piedad, hemos decretado, en referencia a las personas de los judíos que se encuentran en cualquier condición de esclavitud en cualquier parte de nuestro dominio, que quienes los posean recibirán la suma de dinero estipulada y los pondrán en libertad y que ningún hombre podrá mostrar alguna tardanza en el cumplimiento de sus obligaciones.»

\par 47 «Dentro de los tres días siguientes a la publicación de este decreto, deberán hacer listas de esclavos para los oficiales designados para cumplir nuestra voluntad, y presentar inmediatamente a las personas de los cautivos».

\par 48 «Porque consideramos que será ventajoso para nosotros y para nuestros asuntos que el asunto llegue a su fin».

\par 49 «Cualquiera que quiera puede dar información sobre cualquiera que desobedezca el decreto, con la condición de que si se demuestra que el hombre es culpable se convertirá en su esclavo; sus bienes, sin embargo, serán entregados al tesoro real.»

\par 50 Cuando el decreto fue llevado para ser leído ante el rey para su aprobación, contenía todas las demás disposiciones excepto la frase «todos los cautivos que estaban en la tierra antes de ese momento o fueron traídos aquí después», y en su magnanimidad y en la grandeza de su corazón, el rey insertó esta cláusula y dio órdenes de que la concesión de dinero requerida para la redención se depositara íntegramente en manos de los pagadores de las fuerzas y de los banqueros reales, y así se decidió el asunto y se ratificó el decreto dentro de siete días.

\par 51 La subvención para la redención ascendió a más de seiscientos sesenta talentos; porque muchos niños amamantados se emancipaban junto con sus madres.

\par 52 Cuando se preguntó si se debía pagar por ellos la suma de veinte talentos, el rey ordenó que así se hiciera, y así cumplió su decisión de la manera más completa.

\par \textit{Notas al pie}
\par \textit{143:1 Una comparación importante entre Dios y Zeus.}

\chapter{2}

\par \textit{Mostrando cómo se llevaban los registros más cuidadosos de los asuntos de estado. La burocracia gubernamental. Se nombra un comité de seis personas para ir a ver al Sumo Sacerdote en Jerusalén y organizar el intercambio. Aristeas queda a cargo de la delegación.}

\par 1 Una vez hecho esto, ordenó a Demetrio que redactara un memorial sobre la transcripción de los libros judíos.

\par 2 Porque todos los asuntos del Estado se llevaban a cabo por medio de decretos y con la mayor exactitud por parte de estos reyes egipcios, y nada se hacía de manera descuidada o al azar.

\par 3 Por eso he insertado copias del memorial y de las cartas, el número de los regalos enviados y la naturaleza de cada uno, ya que cada uno de ellos sobresalía en magnificencia y habilidad técnica.

\par 4 La siguiente es una copia del memorial. La Memoria de Demetrio al gran rey. «Desde que me has dado instrucciones, oh rey, de que se reúnan los libros necesarios para completar tu biblioteca y se reparen los que están defectuosos, me he dedicado con el mayor cuidado al cumplimiento de tus deseos, y ahora tengo la siguiente propuesta para presentarles.»

\par 5 «Los libros de la ley de los judíos (y algunos otros más) están ausentes de la biblioteca.»

\par 6 «Están escritos en caracteres y idioma hebreos y han sido interpretados descuidadamente, y no representan el texto original, como me informan quienes saben; porque nunca han tenido el cuidado de un rey para protegerlos.»

\par 7 «Es necesario que estos sean precisos para vuestra biblioteca, ya que la ley que contienen, por ser de origen divino, está llena de sabiduría y libre de toda mancha».

\par 8 «Por esta razón los literatos y poetas y la masa de escritores históricos se han abstenido de referirse a estos libros y a los hombres que han vivido y viven de acuerdo con ellos, porque su concepción de la vida es tan sagrada y religiosa, como dice Hecateo de Abdera.»

\par 9 «Si te place, oh rey, se escribirá una carta al Sumo Sacerdote de Jerusalén, pidiéndole que envíe seis ancianos de cada tribu, hombres que hayan vivido la vida más noble y sean los más hábiles en su ley, para que podamos averiguar los puntos en los que la mayoría de ellos están de acuerdo, y así, habiendo obtenido una traducción precisa, podamos colocarla en un lugar visible de una manera digna del trabajo en sí y de su propósito.»

\par 10 «¡Que vuestra prosperidad sea continua!»

\par 11 Cuando se presentó este memorial, el rey ordenó que se escribiera una carta a Eleazar sobre el asunto, dándole también cuenta de la emancipación de los judíos cautivos.

\par 12 Y dio cincuenta talentos de oro, setenta talentos de plata y una gran cantidad de piedras preciosas para hacer cuencos, copas, una mesa y copas para libaciones.

\par 13 También ordenó a los que custodiaban sus arcas que permitieran a los artífices seleccionar los materiales que necesitaran para ello, y que se enviaran cien talentos en dinero para los sacrificios para el templo y para otras necesidades.

\par 14 Os daré cuenta completa de la obra después de haberos entregado copias de las cartas. La carta del rey decía lo siguiente:

\par 15 «El rey Ptolomeo envía saludos y saludos al sumo sacerdote Eleazar.»

\par 16 «Como hay muchos judíos asentados en nuestro reino, que fueron llevados de Jerusalén por los persas durante su poder, y muchos más que vinieron con mi padre a Egipto como cautivos, a muchos de ellos los puso en el ejército y les pagó salarios más altos de lo habitual, y cuando demostró la lealtad de sus líderes, construyó fortalezas y las puso a su cargo para que los egipcios nativos pudieran ser intimidados por ellos.

\par 17 «Y yo, cuando subí al trono, adopté una actitud bondadosa hacia todos mis súbditos, y más particularmente hacia aquellos que eran ciudadanos tuyos: he puesto en libertad a más de cien mil cautivos, pagando a sus dueños el precio correspondiente de mercado para ellos, y si alguna vez se ha hecho mal a tu pueblo a través de las pasiones de la multitud, les he hecho reparación.»

\par 18 «El motivo que impulsó mi acción ha sido el deseo de actuar piadosamente y rendir al Dios supremo una ofrenda de agradecimiento por mantener mi reino en paz y gran gloria en todo el mundo.»

\par 19 «Además, he reclutado en mi ejército a aquellos de tu pueblo que estaban en la flor de la vida, y a aquellos que eran aptos para estar unidos a mi persona y dignos de la confianza de la corte, los he colocado en puestos oficiales.»

\par 20 «Ahora bien, como estoy ansioso por mostrar mi gratitud a estos hombres y a los judíos en todo el mundo y a las generaciones venideras, he decidido que vuestra ley se traduzca de la lengua hebrea que se usa entre vosotros al idioma griego, para que estos libros se agreguen a los demás libros reales de mi biblioteca.»

\par 21 «Será una bondad de tu parte y una recompensa por mi celo si eliges a seis ancianos de cada una de tus tribus, hombres de vida noble, expertos en tu ley y capaces de interpretarla, que en cuestiones de disputa tal vez podamos descubrir el veredicto en el que la mayoría está de acuerdo, ya que la investigación es de la mayor importancia posible.»

\par 22 «Espero ganar gran renombre con la realización de este trabajo.»

\par 23 «He enviado a Andrés, el jefe de mi guardia personal, y a Aristeas, hombres a quienes tengo en gran estima, para exponerte el asunto y presentarte cien talentos de plata, las primicias de mi ofrenda para el templo y los sacrificios y demás ritos religiosos.»

\par 24 «Si me escribes acerca de tus deseos en estos asuntos, me harás un gran favor y me brindarás una nueva promesa de amistad, porque todos tus deseos se cumplirán lo más rápidamente posible. ¡Despedida!»

\par 25 A esta carta, Eleazar respondió apropiadamente lo siguiente: «Eleazar, el Sumo Sacerdote, envía saludos al rey Ptolomeo, su verdadero amigo».

\par 26 «Mis mayores deseos son para tu bienestar y el de la reina Arsínoe, tu hermana y tus hijos».

\par 27 «Yo también estoy bien. He recibido su carta y me alegro mucho de su propósito y de su noble consejo.»

\par 28 «Reuní a todo el pueblo y se lo leí para que supieran de vuestra devoción a nuestro Dios».

\par 29 «Les mostré también las copas que enviasteis, veinte de oro y treinta de plata, los cinco tazones y la mesa de la dedicación, y los cien talentos de plata para ofrecer los sacrificios y proveer las cosas de las cuales el El templo está en necesidad.'

\par 30 «Estos regalos me fueron traídos por Andrés, uno de tus más honrados servidores, y por Aristeas, ambos hombres buenos y leales, distinguidos por su erudición y dignos en todo de ser representantes de tus elevados principios y justos propósitos.»

\par 31 «Estos hombres me comunicaron tu mensaje y recibieron de mí una respuesta conforme a tu carta. Aceptaré todo lo que sea ventajoso para usted aunque su petición sea muy inusual.»

\par 32 «Porque habéis otorgado a nuestros ciudadanos grandes beneficios que nunca olvidaremos en muchos sentidos.
»
\par 33 «Inmediatamente ofrecí sacrificios por ti, por tu hermana, por tus hijos y por tus amigos, y todo el pueblo oró para que tus planes prosperaran continuamente y para que Dios Todopoderoso preservara tu reino en paz y con honor, y para que la traducción de la santa ley os resulte ventajosa y se lleve a cabo con éxito.»

\par 34 «En presencia de todo el pueblo elegí a seis ancianos de cada tribu, hombres buenos y leales, y te los envié con una copia de nuestra ley.»

\par 35 «Será una bondad, oh rey justo, si nos ordenas que tan pronto como se complete la traducción de la ley, esos hombres nos sean devueltos sanos y salvos. ¡Despedida!'

\par 36 Estos son los nombres de los ancianos: De la primera tribu, José, Ezequías, Zacarías, Juan, Ezequías y Eliseo.

\par 37 De la segunda tribu, Judas, Simón, Samuel, Adeo, Matatías y Esclemias.

\par 38 De la tercera tribu, Nehemías, José, Teodosio, Baseas, Ornias y Dakis.

\par 39 De la cuarta tribu, Jonatán, Abraeus, Eliseo, Ananías y Cabrías. . . .

\par 40 De la quinta tribu, Isaac, Jacob, Jesús, Sabbateo, Simón y Leví.

\par 41 De la sexta tribu, Judas, José, Simón, Zacarías, Samuel y Selemas.

\par 42 De la séptima tribu, Sabbateo, Sedequías, Jacob, Isaac, Jesías y Nateo.

\par 43 De la octava tribu, Teodosio, Jasón, Jesús, Teodoto, Juan y Jonatán.

\par 44 De la novena tribu, Teófilo, Abraham, Arsamos, Jasón, Endemias y Daniel.

\par 45 De la décima tribu, Jeremías, Eleazar, Zacarías, Baneas, Eliseo y Dateo.

\par 46 De la undécima tribu: Samuel, José, Judas, Jonathés, Chabú y Dositeo.

\par 47 De la duodécima tribu, Isaelo, Juan, Teodosio, Arsamos, Abietes y Ezequiel.

\par 48 Eran en total setenta y dos. Ésta fue la respuesta que dieron Eleazar y sus amigos a la carta del rey.

\chapter{3}

\par \textit{En el que se describe la mesa más exquisita y hermosa jamás producida. También otros ricos regalos, Interesantes a la luz de las recientes excavaciones en Egipto.}

\par 1 Ahora procederé a canjear mi promesa y daré una descripción de las obras de arte.

\par 2 Fueron trabajados con una habilidad excepcional, porque el rey no escatimó en gastos y supervisó personalmente a los trabajadores individualmente.

\par 3 Por lo tanto, no podían escabullirse de ninguna parte de la obra ni terminarla negligentemente.

\par 4 Primero que nada te daré una descripción de la tabla.

\par 5 Al rey le preocupaba que esta obra fuera de dimensiones excepcionales, e hizo preguntar a los judíos de la localidad sobre el tamaño de la mesa que ya se encontraba en el templo de Jerusalén.

\par 6 Y cuando le describieron las medidas, procedió a preguntar si podría hacer una estructura más grande.

\par 7 Y algunos de los sacerdotes y de los demás judíos respondieron que nada se lo impedía.

\par 8 Y dijo que quería hacerlo cinco veces más grande, pero dudaba por miedo a que resultara inútil para los servicios del templo.

\par 9 Deseaba que su ofrenda no se limitara a colocarla en el templo, porque le proporcionaría un placer mucho mayor si los hombres encargados de ofrecer los sacrificios apropiados pudieran hacerlo apropiadamente en la mesa que él había preparado.

\par 10 No supuso que fuera por falta de oro que la mesa anterior hubiera sido hecha de tamaño pequeño, pero parece haber, dijo, alguna razón por la cual estaba hecha de esta dimensión.

\par 11 Porque si se hubiera dado la orden, no habrían faltado medios.

\par 12 Por lo tanto, no debemos transgredir ni ir más allá de la medida apropiada.

\par 13 Al mismo tiempo les ordenó poner al servicio todas las múltiples formas del arte, porque era un hombre de las más elevadas concepciones y la naturaleza le había dotado de una aguda imaginación que le permitía imaginar la apariencia que sería presentado por la obra terminada.

\par 14 También ordenó que donde no había instrucciones escritas en las Escrituras judías, todo se hiciera lo más hermoso posible.

\par 15 Cuando se establecieran tales instrucciones, debían cumplirse al pie de la letra.

\par 16 Hicieron la mesa de dos codos de largo, un codo de ancho y un codo y medio de alto, y la hicieron de oro macizo puro.

\par 17 Lo que estoy describiendo no era oro fino colocado sobre otro fundamento, sino que toda la estructura era de oro macizo soldado entre sí.

\par 18 Y le hicieron un borde de un palmo de ancho alrededor.

\par 19 Y había una corona de ondas, grabada en relieve en forma de cuerdas maravillosamente labradas en sus tres lados.

\par 20 Porque era de forma triangular y el estilo de la obra era exactamente el mismo en cada uno de los lados, de modo que cualquiera que fuera el lado en que se voltearan, presentaban la misma apariencia.

\par 21 De los dos lados bajo el borde, el que descendía hacia la mesa era una obra muy hermosa, pero era el lado exterior el que atraía la mirada del espectador.

\par 22 Ahora bien, el borde superior de los dos lados, al estar elevado, era agudo, ya que, como hemos dicho, el borde tenía tres lados, desde cualquier punto de vista que se mirara.

\par 23 Y en medio del cordón repujado había sobre él capas de piedras preciosas, y estaban entrelazadas unas con otras con un arte inimitable.

\par 24 Por razones de seguridad, todos fueron fijados con agujas de oro que se insertaron en perforaciones en las piedras.

\par 25 A los lados estaban sujetos entre sí mediante sujetadores para mantenerlos firmes.

\par 26 En la parte del borde que rodeaba la mesa y que se inclinaba hacia arriba y daba a los ojos, había un patrón de huevos labrados en piedras preciosas, minuciosamente grabados mediante una pieza continua de relieve acanalado, estrechamente unidos entre sí alrededor de toda la mesa.

\par 27 Y debajo de las piedras que habían sido dispuestas para representar los huevos, los artistas hicieron una corona que contenía toda clase de frutas, teniendo en su parte superior racimos de uvas y espigas de maíz, también dátiles, manzanas, granadas y cosas similares, bien dispuestas.

\par 28 Estos frutos estaban hechos de piedras preciosas, del mismo color que los frutos mismos, y los sujetaban alrededor de todos los lados de la mesa con una cinta de oro.

\par 29 Y después de poner la corona de frutas, debajo se puso otro modelo de huevos en piedras preciosas, y otras obras acanaladas y en relieve, para que ambos lados de la mesa pudieran usarse según los deseos de los dueños y por esta razón el trabajo ondulado y el borde se extendieron hasta los pies de la mesa.

\par 30 Hicieron y sujetaron debajo de toda la anchura de la mesa una placa maciza de cuatro dedos de espesor, para que se pudieran introducir los pies en ella, y la sujetaron con pasadores que encajaban en encajes debajo del borde, de modo que cualquier lado de la tabla que la gente prefiriera.

\par 31 Así quedó manifiestamente claro que la obra estaba destinada a ser utilizada de cualquier manera.

\par 32 En la misma mesa grabaron un meandro, en el que sobresalían piedras preciosas en el centro: rubíes, esmeraldas, ónices y muchas otras clases de piedras de gran belleza.

\par 33 Y al lado del meandro había un maravilloso trozo de red, que hacía que el centro de la mesa pareciera un romboide, y sobre él había sido labrado un cristal y un llamado ámbar, lo que produjo una impresión incomparable en los espectadores.

\par 34 Hicieron los pies de la mesa con cabezas como lirios, de modo que parecían lirios inclinados debajo de la mesa, y las partes visibles representaban hojas que estaban erguidas.

\par 35 La base del pie en el suelo era un rubí y su altura era de un palmo en todo su contorno.

\par 36 Tenía la apariencia de un zapato y medía ocho dedos de ancho.

\par 37 Sobre él se apoyaba toda la extensión del pie.

\par 38 E hicieron que el pie pareciera como hiedra que crecía de la piedra, entretejido con akanthus y rodeado por una vid que lo rodeaba con racimos de uvas, que estaban trabajados en piedras, hasta la parte superior del pie.

\par 39 Los cuatro pies estaban hechos del mismo estilo, y todo estaba trabajado y ajustado con tanta habilidad, y se invirtió tanta habilidad y conocimiento en hacerlo fiel a la naturaleza, que cuando el aire era agitado por un soplo de viento, Se impartió movimiento a las hojas y todo se diseñó para corresponder con la realidad real que representaba.

\par 40 Y hicieron la parte superior de la mesa en tres partes a modo de tríptico, y las encajaron y encajaron con espigas a lo largo de toda la obra, de modo que la unión de las uniones no se pudiera ver ni descubrir.

\par 41 El grosor de la mesa no era menor de medio codo, por lo que toda la obra debió costar muchos talentos.

\par 42 Porque como el rey no quería aumentar su tamaño, gastó en los detalles la misma suma de dinero que se habría necesitado si la mesa hubiera sido de mayores dimensiones.

\par 43 Y todo se completó de acuerdo con su plan, de la manera más maravillosa y notable, con arte inimitable y belleza incomparable.

\par 44 Dos de los cuencos estaban labrados en oro, y desde la base hasta el centro estaban grabados en relieve en forma de escamas, y entre las escamas se insertaban piedras preciosas con gran habilidad artística.

\par 45 Había entonces un meandro de un codo de altura y su superficie labrada con piedras preciosas de muchos colores, de gran esfuerzo artístico y belleza.

\par 46 Sobre éste había un mosaico trabajado en forma de rombo, que tenía forma de red y llegaba hasta el borde.

\par 47 En el medio, pequeños escudos hechos de diferentes piedras preciosas, colocados alternativamente y de diferentes tipos, no menos de cuatro dedos de ancho, realzaban la belleza de su apariencia.

\par 48 En lo alto del ala había un adorno de lirios en flor, y alrededor estaban grabados racimos de uvas entrelazados.

\par 49 Así estaban hechas las copas de oro, y en cada una de ellas cabían más de dos cántaros.

\par 50 Los cuencos de plata tenían una superficie lisa y estaban maravillosamente hechos, como si estuvieran destinados a espejos, de modo que todo lo que se acercaba a ellos se reflejaba aún más claramente que en los espejos.

\par 51 Pero es imposible describir la impresión real que estas obras de arte produjeron en la mente cuando fueron terminadas.

\par 52 Porque cuando estos vasos estuvieron terminados y colocados uno al lado del otro, primero un cuenco de plata y luego uno de oro, luego otro de plata y luego otro de oro, el aspecto que presentaban era totalmente indescriptible, y quienes iban a verlos No pudieron apartarse de la brillante vista y del fascinante espectáculo.

\par 53 Las impresiones producidas por el espectáculo fueron de diversa índole.

\par 54 Cuando los hombres contemplaban las vasijas de oro y sus mentes examinaban por completo cada detalle de la obra, sus almas se estremecían de asombro.

\par 55 Cuando un hombre quería dirigir su mirada hacia los vasos de plata que estaban ante él, todo parecía brillar con luz alrededor del lugar donde se encontraba, y causaba un deleite aún mayor a los espectadores.

\par 56 De modo que es realmente imposible describir la belleza artística de las obras.

\par 57 Las copas de oro las grabaron en el centro con coronas de vid.

\par 58 Y alrededor de los bordes tejieron una corona de hiedra, mirto y olivo en relieve, y le insertaron piedras preciosas.

\par 59 Las demás partes de los relieves las hicieron de diferentes formas, ya que tenían por cuestión de honor terminar todo de una manera digna de la majestad del rey.

\par 60 En una palabra, se puede decir que ni en el tesoro del rey ni en ningún otro había obras que las igualaran en valor o en habilidad artística.

\par 61 Porque el rey no se preocupaba poco por ellos, pues le encantaba obtener gloria por la excelencia de sus designios.

\par 62 Porque muchas veces descuidaba sus asuntos oficiales y pasaba su tiempo con los artistas en su ansiedad de que completaran todo de una manera digna del lugar al que debían enviarse los regalos.

\par 63 Así que todo se llevó a cabo a gran escala, de manera digna del rey que envió los regalos y del sumo sacerdote que era gobernante del país.

\par 64 No escasearon las piedras preciosas, pues se utilizaron no menos de cinco mil, todas ellas de gran tamaño.

\par 65 Se empleó la más excepcional habilidad artística, de modo que el costo de las piedras y la mano de obra fue cinco veces mayor que el del oro.

\par \textit{Notas al pie}

\par \textit{148:1 Un codo son 18 pulgadas.}

\chapter{4}

\par \textit{Detalles vívidos del sacrificio. Es notable la precisión infalible de los sacerdotes. Una orgía salvaje. Una descripción del templo y sus instalaciones hidráulicas.}

\par 1 Os HE dado esta descripción de los regalos porque pensé que era necesario.

\par 2 El siguiente punto de la narración es un relato de nuestro viaje a Eleazar, pero primero les daré una descripción de todo el país.

\par 3 Cuando llegamos a la tierra de los judíos, vimos la ciudad situada en medio de toda Judea, en la cima de una montaña de considerable altura.

\par 4 En la cima se había construido el templo en todo su esplendor.

\par 5 Estaba rodeado por tres muros de más de setenta codos de alto y de largo y ancho, correspondientes a la estructura del edificio.

\par 6 Todos los edificios se caracterizaban por una magnificencia y un coste sin precedentes.

\par 7 Era evidente que no se había escatimado en gastos en la puerta y en los cierres que la unían a los postes y en la estabilidad del dintel.

\par 8 También el estilo de la cortina era completamente proporcional al de la entrada.

\par 9 Su tejido, debido a la corriente de aire, estaba en perpetuo movimiento, y como este movimiento se comunicaba desde abajo y la cortina se hinchaba hasta su punto más alto, ofrecía un espectáculo agradable del que un hombre apenas podía separarse.

\par 10 La construcción del altar estaba en consonancia con el lugar mismo y con los holocaustos que se quemaban en él, y el acceso hasta él era de la misma escala.

\par 11 Había hasta él una pendiente gradual, convenientemente dispuesta por motivos de decencia, y los sacerdotes ministrantes estaban vestidos con vestiduras de lino hasta los tobillos.

\par 12 El templo mira hacia el oriente y su espalda hacia el occidente.

\par 13 Todo el suelo está pavimentado con piedras y desciende hasta los lugares designados para llevar agua y lavar la sangre de los sacrificios, ya que allí se sacrifican muchos miles de animales en los días festivos.

\par 14 Y hay un suministro inagotable de agua, porque del interior del templo brota un abundante manantial natural.

\par 15 Hay además cisternas maravillosas e indescriptibles bajo tierra, como me señalaron, a una distancia de cinco estadios alrededor del sitio del templo, y cada una de ellas tiene innumerables caños para que converjan los diferentes arroyos.

\par 16 Y todo esto estaba fijado con plomo en el fondo y en las paredes laterales, y sobre ellos se había extendido una gran cantidad de yeso, y cada parte del trabajo se había hecho con el mayor cuidado.

\par 17 Al pie del altar hay muchas aberturas para el agua, que son invisibles para todos excepto para los que están ocupados en el ministerio, de modo que toda la sangre de los sacrificios, que se recoge en grandes cantidades, se lava en un abrir y cerrar de ojos.

\par 18 Tal es mi opinión con respecto al carácter de los embalses y ahora les mostraré cómo se confirmó.

\par 19 Me llevaron fuera de la ciudad más de cuatro estadios y me dijeron que mirara hacia un lugar determinado y escuchara el ruido que hacía el encuentro de las aguas, de modo que se me hizo evidente el gran tamaño de los depósitos, como ya se ha señalado.

\par 20 El ministerio de los sacerdotes es insuperable en todos los sentidos, tanto por su resistencia física como por su servicio ordenado y silencioso.

\par 21 Porque todos trabajan espontáneamente, aunque esto implica un esfuerzo muy doloroso, y cada uno tiene una tarea especial asignada.

\par 22 El servicio se realiza sin interrupción: unos proporcionan la leña, otros el aceite, otros la fina harina de trigo, otros las especias; otros traen nuevamente los trozos de carne para el holocausto, mostrando un grado de fuerza maravilloso.

\par 23 Porque toman con ambas manos los miembros de un becerro, cada uno de los cuales pesa más de dos talentos, y con cada mano los arrojan de manera maravillosa sobre el lugar alto del altar y nunca dejan de colocarlos sobre el lugar adecuado.

\par 24 De la misma manera los pedazos de las ovejas y también de las cabras son maravillosos tanto por su peso como por su gordura.

\par 25 Aquellos a quienes corresponde el oficio, escojan siempre animales sin defecto y especialmente gordos, y así se realizará el sacrificio que he descrito.

\par 26 Hay un lugar especial para que descansen, donde se sientan los que están relevados de su deber.

\par 27 Cuando esto sucede, los que ya han descansado y están listos para reanudar sus deberes se levantan espontáneamente, ya que no hay nadie que dé órdenes con respecto a la disposición de los sacrificios.

\par 28 Reina el más completo silencio, de modo que se podría imaginar que no había ni una sola persona presente, aunque en realidad hay setecientos hombres ocupados en el trabajo, además de la gran cantidad de los que se ocupan de preparar los sacrificios.

\par 29 Todo se hace con reverencia y de manera digna del gran Dios.

\par 30 Nos quedamos muy asombrados cuando vimos a Eleazar ocupado en el ministerio, por la forma de su vestimenta y la majestuosidad de su apariencia, que se revelaba en el manto que vestía y en las piedras preciosas que llevaba.

\par 31 Sobre el manto que llegaba hasta sus pies había campanillas de oro que emitían una especie de melodía peculiar, y a ambos lados de ellas había granadas con flores multicolores de un maravilloso color.

\par 32 Estaba ceñido con un cinturón de notable belleza, tejido con los más bellos colores.

\par 33 Sobre su pecho llevaba el llamado oráculo de Dios, en el que estaban incrustadas doce piedras de diferentes tipos, unidas con oro, que contenían los nombres de los líderes de las tribus, según su orden original. , cada uno brillando de forma indescriptible con su color particular.

\par 34 En la cabeza llevaba una tiara, como se llama, y ​​sobre ella, en medio de la frente, un turbante inimitable, la diadema real llena de gloria con el nombre de Dios escrito en letras sagradas sobre una placa de oro. . . habiendo sido juzgado digno de llevar estos emblemas en los ministerios.

\par 35 Su aparición creaba tal asombro y confusión mental que uno sentía que había llegado a la presencia de un hombre que pertenecía a un mundo diferente.

\par 36 Estoy seguro de que cualquiera que participe en el espectáculo que he descrito se llenará de asombro y asombro indescriptible y quedará profundamente conmovido en su mente al pensar en la santidad que se atribuye a cada detalle del servicio.

\par 37 Pero para obtener información completa, subimos a la cima de la ciudadela vecina y miramos a nuestro alrededor.

\par 38 Está situado en un lugar muy elevado y está fortificado con muchas torres, que han sido construidas hasta lo más alto, de piedras inmensas, con el fin, según nos informaron, de custodiar el recinto del templo, para que si hubiera un ataque, o una insurrección o un ataque del enemigo, nadie podría forzar una entrada dentro de los muros que rodean el templo.

\par 39 En las torres de la ciudadela estaban colocadas máquinas de guerra y diferentes tipos de máquinas, y la posición era mucho más alta que el círculo de murallas que he mencionado.

\par 40 Las torres también estaban custodiadas por hombres de confianza que habían dado la máxima prueba de su lealtad a su país.

\par 41 A estos hombres nunca se les permitió salir de la ciudadela, excepto los días de fiesta y sólo en destacamentos, ni permitieron la entrada a ningún extraño.

\par 42 También fueron muy cuidadosos cuando el oficial superior les dio la orden de permitir que los visitantes inspeccionaran el lugar, como nos enseñó nuestra propia experiencia.

\par 43 Ellos eran muy reacios a permitirnos, aunque éramos sólo dos hombres desarmados, para ver la ofrenda de los sacrificios.

\par 44 Y afirmaron que estaban obligados por un juramento cuando se les confió el encargo, porque todos habían jurado y estaban obligados a cumplir el juramento sagradamente al pie de la letra, que aunque eran quinientos en número, no Permitir la entrada de más de cinco hombres a la vez.

\par 45 La ciudadela era la protección especial del templo y su fundador la había fortificado con tanta fuerza que podría protegerla eficientemente.

\chapter{5}

\par \textit{Una descripción de la ciudad y el campo. Compare el versículo 11 con las condiciones de hoy. Los versículos 89-41 revelan cómo los antiguos estimaban a un erudito y a un caballero.}

\par 1 EL tamaño de la ciudad es de dimensiones moderadas.

\par 2 Tiene unos cuarenta estadios 1 de circunferencia, por lo que uno podría conjeturar.

\par 3 Tiene sus torres dispuestas en forma de teatro, con calles que conducen entre ellas ahora son visibles los cruces de las torres inferiores pero los de las torres superiores son más frecuentados.

\par 4 Porque la tierra asciende, pues la ciudad está edificada sobre un monte.

\par 5 También hay escalones que conducen al cruce de caminos, y algunas personas siempre suben y otras bajan y se mantienen lo más separados posible unos de otros en el camino debido a aquellos que están sujetos a las reglas de pureza. , para que no toquen nada que sea ilícito.

\par 6 No en vano los primeros fundadores de la ciudad la construyeron en las proporciones adecuadas, ya que tenían una idea clara de lo que se necesitaba.

\par 7 Porque el país es extenso y hermoso.

\par 8 Algunas partes de ella son llanas, especialmente las regiones que pertenecen a la llamada Samaria, y que limitan con la tierra de los idumeos, otras partes son montañosas, especialmente las que son contiguas a la tierra de Judea.

\par 9 Por lo tanto, el pueblo está obligado a dedicarse a la agricultura y al cultivo de la tierra, para que así pueda tener abundantes provisiones de cosechas.

\par 10 De esta manera se realizan cultivos de toda clase y se obtiene una cosecha abundante en toda la tierra mencionada.

\par 11 Las ciudades grandes y que gozan de la correspondiente prosperidad están bien pobladas, pero descuidan las zonas rurales, ya que todos los hombres se inclinan a una vida de placer, ya que cada uno tiene una tendencia natural hacia la búsqueda del placer.

\par 12 Lo mismo ocurrió en Alejandría, que supera a todas las ciudades en tamaño y prosperidad.

\par 13 Los campesinos, al emigrar de las zonas rurales y establecerse en la ciudad, desacreditaron la agricultura; por eso, para evitar que se establecieran en la ciudad, el rey ordenó que no permanecieran en ella más de veinte días. 2

\par 14 Y de la misma manera dio instrucciones escritas a los jueces, que si era necesario citar a alguien que vivía en el campo, el caso debía resolverse dentro de cinco días.

\par 15 Y como consideraba que el asunto era de gran importancia, nombró también abogados para cada distrito con sus asistentes, para que los agricultores y sus abogados no vaciaran, en interés del negocio, los graneros de la ciudad, es decir, de el producto de la ganadería.

\par 16 He permitido esta digresión porque fue Eleazar quien señaló con gran claridad los puntos que se han mencionado.

\par 17 Porque mucha es la energía que gastan en labrar la tierra.

\par 18 Porque la tierra está repleta de olivos, de cereales y de legumbres, y también de vides, y abunda la miel.

\par 19 Otras clases de árboles frutales y dátiles no cuentan comparados con estos.

\par 20 Hay ganado de todas clases en gran cantidad y abundantes pastos para él.

\par 21 Por eso reconocen con razón que las regiones rurales necesitan una gran población y que las relaciones entre la ciudad y las aldeas están debidamente reguladas.

\par 22 Los árabes traen al país una gran cantidad de especias, piedras preciosas y oro.

\par 23 Porque el país está bien adaptado no sólo para la agricultura sino también para el comercio, y la ciudad es rica en artes y no carece de ninguna de las mercancías que se traen a través del mar.

\par 24 Posee puertos muy adecuados y cómodos en Askalon, Jope y Gaza, así como en Ptolemaida, que fue fundada por el rey y ocupa una posición central en comparación con los otros lugares mencionados, no estando muy lejos de ninguno de ellos.

\par 25 El país produce de todo en abundancia, ya que está bien regado en todas direcciones y bien protegido de las tormentas.

\par 26 El río Jordán, como lo llaman, que nunca se seca, atraviesa la tierra.

\par 27 Originalmente el país contenía no menos de 60 millones de acres (aunque después los pueblos vecinos hicieron incursiones contra él) y 600.000 hombres se establecieron en él en granjas de cien acres cada una.

\par 28 El río, como el Nilo, nace en el tiempo de la cosecha e riega una gran porción de la tierra.

\par 29 Cerca de la región de los Ptolomeos desemboca en otro río que desemboca en el mar.

\par 30 Otros torrentes montañosos, como se les llama, desembocan en la llanura y abarcan las partes alrededor de Gaza y el distrito de Asdod.

\par 31 El país está rodeado por una valla natural y es muy difícil de atacar y no puede ser asaltado por grandes fuerzas, debido a los pasos estrechos, con acantilados que sobresalen y profundos barrancos, y el carácter accidentado de las regiones montañosas que rodean todo la tierra.

\par 32 Nos dijeron que antiguamente se obtenía cobre y hierro de las montañas vecinas de Arabia.

\par 33 Sin embargo, esto se detuvo durante el dominio persa, ya que las autoridades de la época difundieron en el extranjero una noticia falsa de que la explotación de las minas era inútil y costosa para evitar que su país fuera destruido por la minería en estos distritos y posiblemente se los quitaron debido al dominio persa, ya que con la ayuda de este informe falso encontraron una excusa para ingresar al distrito.

\par 34 Ya te he dado, mi querido hermano Filócrates, toda la información esencial sobre este tema en forma breve.

\par 35 Describiré el trabajo de traducción en la continuación.

\par 36 El Sumo Sacerdote seleccionó a hombres del mejor carácter y la más alta cultura, como uno esperaría de su noble ascendencia.

\par 37 Eran hombres que no sólo habían aprendido la literatura judía, sino que también habían estudiado con mucho cuidado la literatura griega.

\par 38 Por lo tanto, estaban especialmente calificados para servir en las embajadas y asumían esta tarea siempre que era necesario.

\par 39 Tenían gran facilidad para celebrar conferencias y discutir problemas relacionados con el derecho.

\par 40 Adoptaron el camino intermedio, y éste es siempre el mejor camino a seguir.

\par 41 Rechazaban los modales rudos y groseros, pero estaban completamente por encima del orgullo y nunca asumían un aire de superioridad sobre los demás, y en la conversación estaban dispuestos a escuchar y dar una respuesta apropiada a cada pregunta.

\par 42 Y todos observaron cuidadosamente esta regla y se preocuparon por encima de todo de superarse unos a otros en su observancia y todos fueron dignos de su líder y de su virtud.

\par 43 Y se podía observar cómo amaban a Eleazar por no querer separarse de él y cómo él los amaba.

\par 44 Además de la carta que escribió al rey acerca de su regreso sano y salvo, también rogó encarecidamente a Andrés que trabajara por el mismo fin y me instó a mí también a ayudar lo mejor que pudiera.

\par 45 Y aunque le prometimos prestar nuestra mejor atención al asunto, él dijo que todavía estaba muy angustiado, porque sabía que el rey, por la bondad de su naturaleza, consideraba que era su mayor privilegio cada vez que oía hablar de un hombre que era superior a sus compañeros en cultura y sabiduría, para convocarlo a su corte.

\par 46 Porque he oído una hermosa frase suya según la cual, reuniendo a su alrededor a hombres justos y prudentes, conseguiría la mayor protección para su reino, ya que tales amigos le darían sin reservas los mejores consejos.

\par 47 Y los hombres que ahora le enviaba Eleazar poseían sin duda estas cualidades.

\par 48 Y con frecuencia afirmó bajo juramento que nunca dejaría ir a los hombres si fuera simplemente algún interés privado el que constituía el motivo impulsor, pero era para el beneficio común de todos los ciudadanos que los enviaba.

\par 49 Porque, explicó, la buena vida consiste en observar las disposiciones de la ley, y este fin se logra mucho más escuchando que leyendo.

\par 50 De esta y otras declaraciones similares quedó claro cuáles eran sus sentimientos hacia ellos.

\par \textit{Notas al pie}

\par \textit{154:1 Un furlong es 1/8 de milla (es decir, 220 yardas).}

\par \textit{154:2 Este relato de las medidas adoptadas en Alejandría para evitar la despoblación del campo a través de migraciones a la ciudad es una revelación interesante de que la cuestión era tan aguda hace 2000 años como lo es hoy.}

\chapter{6}

\par \textit{Explicaciones de las costumbres del pueblo que muestran lo que se entiende por la palabra «Inmundo». La esencia y origen de la «Creencia en Dios». Los versículos 48-44 dan una descripción pintoresca de la Divinidad de la fisiología.}

\par 1 Vale la pena mencionar brevemente la información que dio en respuesta a nuestras preguntas.

\par 2 Porque supongo que la mayoría de la gente siente curiosidad por algunas disposiciones de la ley, especialmente las relativas a las carnes y bebidas y a los animales declarados inmundos.

\par 3 Cuando preguntamos por qué, siendo sólo una forma de creación, algunos animales son considerados impuros para el consumo, y otros impuros incluso para el tacto (pues si bien la ley es escrupulosa en la mayoría de los puntos, lo es especialmente en tales cuestiones como estas) comenzó su respuesta de la siguiente manera:

\par 4 «Observas», dijo, «qué efecto producen en nosotros nuestros modos de vida y nuestras asociaciones; al asociarse con los malos, los hombres captan sus depravaciones y se vuelven miserables a lo largo de su vida; pero si viven con los sabios y prudentes, encuentran los medios para escapar de la ignorancia y enmendar sus vidas.»

\par 5 Nuestro legislador, ante todo, estableció los principios de la piedad y de la justicia y los inculcó punto por punto, no sólo con prohibiciones, sino también con el uso de ejemplos, demostrando los efectos nocivos del pecado y los castigos infligidos por Dios a los culpable.

\par 6 Porque demostró ante todo que hay un solo Dios y que su poder se manifiesta en todo el universo, ya que cada lugar está lleno de su soberanía y ninguna de las cosas que los hombres hacen en secreto sobre la tierra escapa a Su conocimiento.

\par 7 Porque todo lo que el hombre hace y todo lo que sucederá en el futuro le son manifiestos.

\par 8 Trabajando cuidadosamente estas verdades y habiéndolas aclarado, demostró que incluso si un hombre pensara en hacer el mal, por no decir nada en realizarlo, no escaparía a la detección, porque dejó claro que el poder de Dios impregnó toda la ley.

\par 9 Partiendo de su punto de partida, continuó demostrando que toda la humanidad, excepto nosotros, cree en la existencia de muchos dioses, aunque ellos mismos son mucho más poderosos que los seres a quienes adoran en vano.

\par 10 Porque cuando hacen estatuas de piedra y de madera, dicen que son imágenes de quienes han inventado algo útil para la vida y las adoran, aunque tienen pruebas claras de que no tienen ningún sentimiento.

\par 11 Pues sería completamente absurdo suponer que alguien se convirtió en dios en virtud de sus inventos.

\par 12 Porque los inventores simplemente tomaron ciertos objetos ya creados y, combinándolos, demostraron que poseían una nueva utilidad: ellos mismos no crearon la sustancia de la cosa, por lo que es vano y tonto que la gente haga dioses de hombres como ellos.

\par 13 Porque en nuestros tiempos hay muchos que son mucho más inventivos y más eruditos que los hombres de tiempos pasados ​​que fueron divinizados, y sin embargo nunca vendrían a adorarlos.

\par 14 Los creadores y autores de estos mitos piensan que son los más sabios de los griegos.

\par 15 ¿Por qué hablar de otros pueblos encaprichados, egipcios y similares, que confían en las fieras y en la mayoría de los reptiles y en el ganado, y los adoran y les ofrecen sacrificios tanto vivos como muertos?

\par 16 Nuestro Legislador, como hombre sabio y especialmente dotado por Dios para entender todas las cosas, miró detalladamente cada detalle y nos rodeó con murallas inexpugnables y muros de hierro, para que no nos mezcláramos en absoluto con nadie de las demás naciones, sino que permanezcan puros en cuerpo y alma, libres de toda imaginación vana, adorando al único Dios Todopoderoso sobre toda la creación.

\par 17 Por eso los principales sacerdotes egipcios, habiendo examinado atentamente muchos asuntos y siendo conscientes de nuestros asuntos, nos llaman «hombres de Dios».

\par 18 Este es un título que no pertenece al resto de la humanidad, sino sólo a aquellos que adoran al Dios verdadero.

\par 19 Los demás no son hombres de Dios, sino de comida, bebida y vestido.

\par 20 Porque todo su carácter los lleva a encontrar consuelo en estas cosas que no tienen en cuenta, sino en todas sus cosas.

\par 21 Entre nuestro pueblo toda esa vida, su consideración principal es la soberanía de Dios.

\par 22 Por eso, para que ninguna abominación nos corrompa o nuestra vida se pervierta con malas comunicaciones, Él nos rodeó por todos lados con reglas de pureza, afectando por igual lo que comemos, bebemos, tocamos, oímos o ver.

\par 23 Porque, aunque en general todas las cosas son iguales en su constitución natural, ya que todas están gobernadas por una y la misma potencia, sin embargo, hay una razón profunda en cada caso particular por la cual nos abstenemos del uso de ciertas cosas y disfrutar del uso común de los demás.

\par 24 A modo de ilustración, repasaré uno o dos puntos y se los explicaré.

\par 25 Porque no debéis caer en la degradante idea de que Moisés redactó sus leyes con tanto cuidado por respeto a los ratones, las comadrejas y otras cosas similares. 1

\par 26 Todas estas ordenanzas se hicieron por causa de la justicia para ayudar en la búsqueda de la virtud y el perfeccionamiento del carácter.

\par 27 Porque todas las aves que utilizamos son mansas y se distinguen por su limpieza, y se alimentan de diversas clases de cereales y legumbres, como las palomas, las tórtolas, las langostas, las perdices, también los gansos y todas las demás aves de esta clase.

\par 28 Pero descubriréis que las aves prohibidas son salvajes y carnívoras, que tiranizan a las demás con la fuerza que poseen y se alimentan cruelmente de las aves domesticadas antes enumeradas.

\par 29 Y no sólo esto, sino que se apoderan de los corderos y de los cabritos, y también hieren a los hombres, ya sean vivos o muertos, y por eso, llamándolos inmundos, dio por medio de ellos una señal de que aquellos a quienes le fue ordenada la legislación, deben practicar la justicia en sus corazones y no tiranizar a nadie confiando en sus propias fuerzas ni robarle nada, sino dirigir su vida de acuerdo con la justicia, tal como los pájaros mansos, ya mencionados, consumen el diferentes tipos de pulsos que crecen sobre la tierra y no tiranizan hasta la destrucción de sus propios parientes.

\par 30 Nuestro legislador nos enseñó, por tanto, que con métodos como éstos se dan indicaciones a los sabios para que sean justos y no hagan nada con violencia y se abstengan de tiranizar a los demás confiando en sus propias fuerzas.

\par 31 Puesto que, debido a sus costumbres, se considera indecoroso incluso tocar animales tan inmundos como los que hemos mencionado, ¿no deberíamos tomar todas las precauciones necesarias para que nuestro propio carácter no se destruya en la misma medida?

\par 32 Por lo tanto, todas las reglas que él ha establecido con respecto a lo que está permitido en el caso de estas aves y otros animales, las ha promulgado con el objeto de enseñarnos una lección moral.

\par 33 Porque la división de la pezuña y la separación de las garras tienen como objetivo enseñarnos que debemos discriminar entre nuestras acciones individuales con miras a la práctica de la virtud.

\par 34 Porque la fuerza de todo nuestro cuerpo y su actividad dependen de nuestros hombros y miembros.

\par 35 Por lo tanto, nos obliga a reconocer que debemos realizar todas nuestras acciones con discriminación de acuerdo con el estándar de justicia, especialmente porque hemos sido claramente separados del resto de la humanidad.

\par 36 Porque la mayoría de los hombres se contaminan con relaciones sexuales promiscuas, cometiendo así grandes iniquidades, y países y ciudades enteras se enorgullecen de tales vicios.

\par 37 Porque no sólo tienen relaciones sexuales con hombres, sino que contaminan a sus propias madres y hasta a sus hijas.

\par 38 Pero a nosotros se nos ha mantenido apartados de tales pecados.

\par 39 Y a las personas que han sido separadas de la manera antes mencionada también el Legislador las caracteriza como poseedoras del don de la memoria.

\par 40 Porque todos los animales «que tienen patas hendidas y rumian» representan para los iniciados el símbolo de la memoria.

\par 41 Porque el acto de rumiar no es más que la reminiscencia de la vida y la existencia.

\par 42 Porque la vida suele sustentarse con alimentos, por lo que también nos exhorta en la Escritura con estas palabras: «Seguramente te acordarás del Señor que hizo en ti aquellas cosas grandes y maravillosas».

\par 43 Porque cuando son concebidos correctamente, son manifiestamente grandes y gloriosos; primero la construcción del cuerpo y la disposición de los alimentos y la separación de cada miembro individual y, más aún, la organización de los sentidos, el funcionamiento y movimiento invisible de la mente, la rapidez de sus acciones particulares y su descubrimiento del artes, muestran un ingenio infinito.

\par 44 Por lo tanto, nos exhorta a recordar que las partes antes mencionadas se mantienen juntas por el poder divino con consumada habilidad.

\par 45 Porque él ha señalado cada momento y lugar para que podamos recordar continuamente al Dios que nos gobierna y nos preserva.

\par 46 En cuanto a las comidas y bebidas, nos ordena ofrecer primero una parte como sacrificio y luego disfrutar de nuestra comida.

\par 47 Además, nos ha dado sobre nuestras vestiduras un símbolo de recuerdo, y de la misma manera nos ha ordenado que pongamos los oráculos divinos en nuestras puertas y puertas como recuerdo de Dios.

\par 48 Y también ordena expresamente que se coloque en nuestras manos el símbolo, mostrando claramente que debemos realizar cada acto con rectitud, recordando nuestra propia creación y, sobre todo, el temor de Dios.

\par 49 También invita a los hombres, cuando se acuestan a dormir y se levantan nuevamente, a meditar en las obras de Dios, no sólo de palabra, sino observando claramente el cambio y la impresión que se produce en ellos cuando se van a dormir y también su despertar, cuán divino e incomprensible es el cambio de uno de estos estados al otro.

\par 50 Ahora se les ha señalado la excelencia de la analogía con respecto a la discriminación y la memoria, según nuestra interpretación de «la pezuña hendida y la rumia».

\par 51 Porque nuestras leyes no fueron formuladas al azar ni según el primer pensamiento casual que se nos ocurrió, sino con vistas a la verdad y a la indicación de la recta razón.

\par 52 Pues mediante las instrucciones que da sobre las comidas y bebidas y sobre casos particulares de contacto, nos ordena que no hagamos ni escuchemos nada irreflexivamente ni que recurramos a la injusticia mediante el abuso del poder de la razón.

\par 53 También en el caso de los animales salvajes se puede descubrir el mismo principio.

\par 54 Porque el carácter de la comadreja, de los ratones y de animales como éstos, que expresamente se mencionan, es destructivo.

\par 55 Los ratones lo contaminan y dañan todo, no sólo para su propia comida, sino incluso hasta el punto de hacer absolutamente inútil para el hombre todo lo que se les cruza en el camino para dañar.

\par 56 También la especie de las comadrejas es peculiar: además de lo dicho, tiene una característica contaminante: concibe por los oídos y da a luz por la boca.

\par 57 Por eso esta práctica es declarada inmunda en los hombres.

\par 58 Pues, al plasmar en la palabra todo lo que reciben a través de los oídos, involucran a los demás en males y cometen impurezas no ordinarias, estando ellos mismos completamente contaminados por la contaminación de la impiedad.

\par 59 Y según sabemos, vuestro rey hace muy bien en destruir a tales hombres.

\par 60 Entonces dije: «Supongo que te refieres a los informantes, porque él constantemente los expone a torturas y a formas dolorosas de muerte».

\par 61 «Sí», respondió, 'estos son los hombres a los que me refiero; porque velar por la destrucción de los hombres es algo impío».

\par 62 Y nuestra ley nos prohíbe dañar a nadie, ya sea de palabra o de hecho.

\par 63 Mi breve relato de estos asuntos debería haberos convencido de que todas nuestras regulaciones han sido redactadas con miras a la justicia, y que nada ha sido promulgado en las Escrituras sin pensar o sin la debida razón, sino que su propósito es permitir a lo largo de toda nuestra vida y en todas nuestras acciones a practicar la justicia ante todos los hombres, teniendo presente a Dios Todopoderoso.

\par 64 Por lo tanto, en lo que respecta a los alimentos y a las cosas inmundas, a los reptiles y a las bestias salvajes, todo el sistema apunta a la justicia y a las relaciones justas entre el hombre y el hombre.

\par 65 Me pareció que había hecho una buena defensa en todos los puntos; porque también en referencia a los terneros, carneros y cabras que se ofrecen, dijo que era necesario tomarlos de las vacas y de las ovejas, y sacrificar animales mansos y no ofrecer nada salvaje, para que los oferentes de los sacrificios entendieran el significado simbólico del legislador y no estar bajo la influencia de una autoconciencia arrogante.

\par 66 Porque quien ofrece un sacrificio, también ofrece su propia alma en todos sus estados de ánimo.

\par 67 Creo que vale la pena narrar estos detalles relacionados con nuestra discusión, y debido a la santidad y el significado natural de la ley, me he visto obligado a explicártelos claramente, Filócrates, debido a tu propia devoción a la ciencia.

\par \textit{Notas al pie}

\par \textit{158:1 Compare esta curiosa idea con 1 Corintios, IX, 9.}

\chapter{7}

\par \textit{La llegada de los enviados con el manuscrito del precioso libro y regalos. Preparativos para un banquete real. El anfitrión, inmediatamente después de sentarse a la mesa, entretiene a sus invitados con preguntas y respuestas. Algunos comentarios sabios sobre sociología.}

\par 1 Y Eleazar, después de ofrecer el sacrificio, seleccionar a los enviados y preparar muchos regalos para el rey, nos despachó con gran seguridad para nuestro viaje.

\par 2 Y cuando llegamos a Alejandría, el rey fue inmediatamente informado de nuestra llegada.

\par 3 Al entrar en palacio, Andrés y yo saludamos calurosamente al rey y le entregamos la carta escrita por Eleazar.

\par 4 El rey estaba muy ansioso por recibir a los enviados y ordenó que todos los demás funcionarios fueran despedidos y convocados a los enviados a su presencia inmediatamente.

\par 5 Ahora bien, esta sorpresa general provocó que los que vienen a pedir audiencia al rey para asuntos importantes sean admitidos a su presencia al quinto día, mientras que los enviados de los reyes o de ciudades muy importantes difícilmente la consiguen pero a estos hombres los tuvo por dignos de mayor honor, ya que tenía en tan alta estima a su señor, y por eso inmediatamente despidió a aquellos cuya presencia consideraba superflua y siguió caminando hasta que entraron y él pudo darles la bienvenida.

\par 6 Cuando entraron con los regalos que les habían enviado y los valiosos pergaminos en los que estaba escrita la ley en oro con caracteres judíos, porque el pergamino estaba maravillosamente preparado y la conexión entre las páginas se había efectuado de tal manera que ser invisibles, el rey en cuanto los vio comenzó a preguntarles por los libros.

\par 7 Y cuando sacaron los rollos de sus envolturas y desdoblaron las páginas, el rey se quedó quieto por un largo tiempo y luego, haciendo reverencias unas siete veces, dijo:

\par 8 «Os doy gracias, amigos míos, y agradezco a quien os envió aún más, y sobre todo a Dios, de quien son estos oráculos.»

\par 9 Y cuando todos, los enviados y también los demás presentes, gritaron al unísono y con una sola voz: «¡Dios salve al Rey!» rompió a llorar de alegría.

\par 10 Porque la exaltación de su alma y el sentimiento del abrumador honor que se le había concedido le obligaron a llorar por su buena suerte.

\par 11 Les ordenó que volvieran a colocar los rollos en su lugar y luego, después de saludar a los hombres, dijo: «Era justo, hombres de Dios, que primero presentara mi reverencia a los libros por los cuales Te convoqué aquí y entonces cuando lo hice, para extenderte la mano derecha de la amistad.

\par 12 «Fue por esta razón que hice esto primero.»

\par 13 «He decretado que este día en el que llegaste se conserve como un gran día y se celebrará anualmente durante toda mi vida.»

\par 14 'Sucede también que es el aniversario de mi victoria naval sobre Antígono. Por lo tanto, estaré encantado de festejar contigo hoy.

\par 15 «Todo lo que puedas usar», dijo, «será preparado para ti como es debido, y también para mí contigo».

\par 16 Después de que expresaron su alegría, ordenó que se les asignara el mejor alojamiento cerca de la ciudadela y que se hicieran los preparativos para el banquete.

\par 17 Y Nicanor llamó al mayordomo Doroteo, que era el oficial especial designado para cuidar de los judíos, y le ordenó que hiciera los preparativos necesarios para cada uno.

\par 18 Porque este acuerdo lo había hecho el rey y es un acuerdo que veis que se mantiene hasta el día de hoy.

\par 19 Porque todas las ciudades que tienen costumbres especiales en materia de bebida, comida y descanso, tienen funcionarios especiales que se ocupan de sus necesidades.

\par 20 Y cuando vienen a visitar a los reyes, se hacen los preparativos según sus propias costumbres, para que ninguna molestia pueda perturbar el disfrute de su visita.

\par 21 El. Se tomó la misma precaución en el caso de los enviados judíos.

\par 22 Doroteo, el patrón designado para atender a los invitados judíos, era un hombre muy concienzudo.

\par 23 Sacó para el banquete todos los provisiones que estaban bajo su control y reservados para la recepción de tales invitados.

\par 24 Dispuso los asientos en dos filas según las instrucciones del rey.

\par 25 Porque le había ordenado que hiciera sentar a la mitad de los hombres a su derecha y al resto detrás de él, para no negarles el mayor honor posible.

\par 26 Cuando se sentaron, ordenó a Doroteo que hiciera todo según las costumbres vigentes entre sus invitados judíos.

\par 27 Por lo tanto, prescindió de los servicios de los heraldos sagrados, de los sacerdotes sacrificadores y de los demás que solían ofrecer las oraciones, y llamó a uno de nosotros, Eleazar, el mayor de los sacerdotes judíos, para que ofreciera oración en su lugar.

\par 28 Entonces él se levantó e hizo una oración extraordinaria. «¡Que Dios Todopoderoso te enriquezca, oh rey, con todas las cosas buenas que ha hecho y te conceda a ti, a tu esposa, a tus hijos y a tus camaradas la posesión continua de ellos mientras vivas!»

\par 29 Al oír estas palabras estalló un fuerte y alegre aplauso que duró mucho tiempo, y luego se dedicaron a disfrutar del banquete que habían preparado.

\par 30 Todos los arreglos para el servicio de mesa se llevaron a cabo de acuerdo con el mandato de Doroteo.

\par 31 Entre los asistentes estaban los pajes reales y otras personas que ocupaban lugares de honor en la corte del rey.

\par 32 Aprovechando una pausa en el banquete, el rey preguntó al enviado que estaba sentado en el asiento de honor (porque estaban ordenados según la antigüedad), ¿cómo podría mantener su reino intacto hasta el fin?

\par 33 Después de reflexionar un momento, respondió: «La mejor manera de establecer su seguridad sería imitar la incesante benignidad de Dios. Porque si mostráis clemencia e infligís castigos leves a quienes los merecen según sus méritos, los alejaréis del mal y los conduciréis al arrepentimiento.»

\par 34 El rey elogió la respuesta y luego preguntó al siguiente hombre: ¿Cómo podría hacer todo lo mejor en todas sus acciones?

\par 35 Y él respondió: «Si un hombre se comporta con justicia hacia todos, siempre actuará rectamente en cada ocasión, recordando que cada pensamiento es conocido de Dios. Si tomas el temor de Dios como punto de partida, nunca perderás la meta.»

\par 36 El rey también felicitó a este hombre por su respuesta y preguntó a otro: ¿Cómo podría tener amigos que compartieran sus mismos pensamientos?

\par 37 Él respondió: «Si te ven estudiando los intereses de las multitudes sobre las que gobiernas, Harás bien en observar cómo Dios otorga sus beneficios a la raza humana, proporcionándoles salud, alimento y... todas las demás cosas a su debido tiempo.»

\par 38 Después de expresar su acuerdo con la respuesta, el rey preguntó al siguiente invitado: ¿Cómo podría al dar audiencias y emitir juicios ganarse los elogios incluso de aquellos que no lograron ganar su pleito?

\par 39 Y él dijo: «Si sois justos en vuestras palabras con todos y nunca actuéis con insolencia ni tiranía en vuestro trato a los ofensores. Y lo harás si observas el método por el cual Dios actúa. Las peticiones de los dignos siempre son cumplidas, mientras que aquellos que no obtienen respuesta a sus oraciones son informados por medio de sueños o acontecimientos de lo que fue perjudicial en sus peticiones y que Dios no los castiga según sus pecados o la grandeza de sus pecados. Su fuerza, pero actúa con paciencia hacia ellos.»

\par 40 El rey elogió calurosamente al hombre por su respuesta y preguntó al siguiente en orden: ¿Cómo podía ser invencible en los asuntos militares?

\par 41 Y él respondió: «Si no confiara enteramente en sus multitudes ni en sus fuerzas guerreras, sino que invocara continuamente a Dios para que llevara sus empresas a buen término, mientras él mismo cumplió con todos sus deberes con espíritu de justicia.»

\par 42 Agradeciendo esta respuesta, le preguntó a otro cómo podría llegar a ser objeto de temor para sus enemigos.

\par 43 Y él respondió: 'Si, mientras mantuviera un gran suministro de armas y fuerzas, recordara que estas cosas no podían lograr un resultado duradero y concluyente. Porque incluso Dios infunde miedo en las mentes de los hombres concediendo indultos y haciendo simplemente una demostración de la grandeza de su poder.

\par 44 El rey alabó a este hombre y luego dijo al siguiente: «¿Cuál es el mayor bien en la vida?»

\par 45 Y él respondió: «Saber que Dios es el Señor del universo, y que en nuestros mejores logros no somos nosotros los que logramos el éxito, sino Dios, quien con su poder lleva todas las cosas a su plenitud y nos conduce a la meta».

\par 46 El rey exclamó que el hombre había respondido bien y luego preguntó al siguiente cómo podía conservar todas sus posesiones intactas y finalmente transmitirlas a sus sucesores en las mismas condiciones.

\par 47 Y él respondió: «Rogando constantemente a Dios para que puedas inspirarte con motivos elevados en todas tus empresas y advirtiendo a tus descendientes que no se dejen deslumbrar por la fama o la riqueza, porque es Dios quien otorga todos estos dones y hombres nunca por sí solos ganan la supremacía.»

\par 48 El rey se mostró conforme con la respuesta y preguntó al siguiente invitado: ¿Cómo podría soportar con ecuanimidad lo que le sucediera?

\par 49 Y él dijo: «Si comprendes bien la idea de que todos los hombres son designados por Dios para compartir tanto el mayor mal como el mayor bien, ya que es imposible que un hombre esté exento de estos. Pero Dios, a quien siempre debemos orar, nos inspira valor para perseverar.'

\par 50 Encantado con la respuesta del hombre, el rey dijo que todas sus respuestas habían sido buenas. «Me haré una pregunta», añadió, «y luego me detendré por el momento: para que podamos centrar nuestra atención en disfrutar de la fiesta y pasar un rato agradable».

\par 51 Entonces preguntó al hombre: «¿Cuál es el verdadero objetivo del coraje?»

\par 52 Y él respondió: «Si en la hora del peligro se lleva a cabo un plan correcto de acuerdo con la intención original. Porque todo es hecho por Dios para tu beneficio, oh rey, ya que tu propósito es bueno.»

\par 53 Cuando todos expresaron con aplausos su acuerdo con la respuesta, el rey dijo a los filósofos (pues no eran pocos los que estaban presentes): «En mi opinión, estos hombres sobresalen en virtud y poseen conocimientos extraordinarios, ya que de improviso han dado respuestas adecuadas a estas preguntas que les he planteado, y todos han hecho de Dios el punto de partida de sus palabras.'

\par 54 Y Menedemo, el filósofo de Eretria, dijo: «Es cierto, oh Rey, porque dado que el universo está gobernado por la providencia y dado que percibimos correctamente que el hombre es la creación de Dios, se sigue que todo el poder y la belleza de la palabra proceden de Dios.'

\par 55 Cuando el rey hubo asentido a este sentimiento, cesaron las palabras y procedieron a divertirse. Cuando llegó la noche, el banquete terminó.

\par \textit{Notas al pie}

\par \textit{162:1 Compare esta actitud hacia los criminales con la de la llamada visión humanitaria moderna. También Abeja Capítulo VIII. 11.}

\chapter{8}

\par \textit{Más preguntas y respuestas. Note el versículo 20 con su referencia a volar por el aire escrito en 150 a.C.}

\par 1 Al día siguiente se sentaron nuevamente a la mesa y continuaron el banquete según las mismas disposiciones.

\par 2 Cuando el rey pensó que había llegado la oportunidad adecuada para hacer preguntas a sus invitados, procedió a hacer más preguntas a los hombres que estaban sentados a continuación, en orden a las que habían respondido el día anterior.

\par 3 Comenzó la conversación con el undécimo hombre, porque en la primera ocasión habían hecho preguntas a diez.

\par 4 Cuando se estableció el silencio, preguntó cómo podía seguir siendo rico.

\par 5 Después de una breve reflexión, el hombre a quien se le había hecho la pregunta respondió: «Si no hizo nada indigno de su posición, nunca actuó licenciosamente, nunca prodigó gastos en ocupaciones vacías y vanas, sino que mediante actos de benevolencia hizo que todos sus súbditos bien dispuesto consigo mismo. Porque Dios es el autor de todas las cosas buenas y a Él el hombre debe obedecer.'

\par 6 El rey lo elogió y luego preguntó a otro cómo podía mantener la verdad.

\par 7 Respondiendo a la pregunta, dijo: «Reconociendo que la mentira trae gran deshonra a todos los hombres, y más especialmente a los reyes. Puesto que tienen el poder de hacer lo que quieran, ¿por qué habrían de recurrir a la mentira? Además de esto, siempre debes recordar, oh Rey, que Dios ama la verdad.»

\par 8 El rey recibió la respuesta con gran alegría y, mirando a otro, dijo: «¿Cuál es la enseñanza de la sabiduría?»

\par 9 Y el otro respondió: «Así como deseas que no te suceda ningún mal, sino ser partícipe de todas las cosas buenas, así debes actuar con el mismo principio hacia tus súbditos y ofensores, y debes amonestar suavemente a los nobles y bueno. Porque Dios atrae a todos los hombres hacia sí por su benignidad.»

\par 10 El rey lo elogió y le preguntó al siguiente cómo podía ser amigo de los hombres.

\par 11 Y él respondió: «Al observar que el género humano crece y nace con muchos problemas y grandes sufrimientos, por lo que no debéis castigarlos a la ligera ni infligirles tormentos, ya que sabéis que la vida de los hombres se compone de dolores y sanciones. Porque si entendieras todo te llenarías de lástima, ¡para Dios también es lamentable!'

\par 12 El rey recibió la respuesta con aprobación y preguntó al siguiente: «¿Cuál es el requisito más esencial para gobernar?»

\par 13 «Para mantenerse», respondió, «libre de soborno y practicar la sobriedad durante la mayor parte de la vida, honrar la justicia sobre todas las cosas y hacer amigos con hombres de este tipo». ¡Porque también Dios es amante de la justicia! Habiendo manifestado su aprobación, el rey dijo a otro: «¿Cuál es la verdadera señal de piedad?»

\par 14 Y él respondió: «Para percibir que Dios obra constantemente en el Universo y sabe todas las cosas, y ningún hombre que actúa injustamente y hace maldad puede escapar de Su atención. Así como Dios es el benefactor del mundo entero, ¡tú también debes imitarlo y no ofenderte!»

\par 15 El rey dio su consentimiento y dijo a otro: «¿Cuál es la esencia del reinado?»

\par 16 Y él respondió: «Gobernarse bien y no dejarse llevar por la riqueza o la fama hacia deseos inmoderados o indecorosos, ésta es la verdadera manera de gobernar, si se razona bien el asunto. Porque todo lo que realmente necesitas es tuyo, y Dios está libre de necesidades y además es benigno. ¡Que tus pensamientos sean propios de un hombre y no desees muchas cosas, sino sólo las necesarias para gobernar!»

\par 17 El rey lo elogió y preguntó a otro hombre: ¿Cómo podrían ser mejores sus deliberaciones?

\par 18 Y él respondió: «Si constantemente ponía ante sí la justicia en todo y pensaba que la injusticia equivalía a la privación de la vida. ¡Porque Dios siempre promete las mayores bendiciones a los justos!»

\par 19 Después de haberlo elogiado, el rey preguntó al siguiente: ¿Cómo podría estar libre de pensamientos perturbadores mientras dormía?

\par 20 Y él respondió: «Me has pedido una. Pregunta que es muy difícil de responder, porque no podemos poner en juego nuestro verdadero yo durante las horas de sueño, sino que estamos retenidos en ellas por imaginaciones que no pueden ser controladas por la razón. Porque nuestras almas tienen la sensación de que realmente ven las cosas que entran en nuestra conciencia durante el sueño. Pero nos equivocamos si suponemos que en realidad estamos navegando por el mar en barcos o volando por el aire 1 o viajando a otras regiones o cualquier otra cosa por el estilo. Y, sin embargo, realmente imaginamos que tales cosas están ocurriendo.»

\par 21 En la medida en que me es posible decidir, he llegado a la siguiente conclusión. De todas las maneras posibles, oh Rey, debes regir tus palabras y acciones por la regla de la piedad, para que tengas la conciencia de que estás manteniendo la virtud y que nunca elijas gratificarte a expensas de la razón y nunca abusando de tu poder haz desprecio a la justicia.

\par 22 Porque el alma se ocupa principalmente durante el sueño en las mismas cosas de las que se ocupa en la vigilia. Y aquel que tiene todos sus pensamientos y acciones orientados hacia los fines más nobles, se establece en la justicia tanto cuando está despierto como cuando duerme. Porque debes ser firme en la constante disciplina de ti mismo.

\par 23 El rey elogió a aquel hombre y dijo a otro: «Ya que eres el décimo en responder, cuando hayas hablado, nos dedicaremos al banquete». Y luego me planteó la pregunta: ¿cómo puedo evitar hacer algo indigno de mí mismo?

\par 24 Y él respondió: «Mira siempre tu propia fama y tu propia posición suprema, para que puedas hablar y pensar sólo lo que sea coherente con ella, sabiendo que todos tus súbditos piensan y hablan de ti. Porque no debes parecer peor que los actores, que estudian cuidadosamente el papel que les corresponde desempeñar y configuran todas sus acciones de acuerdo con él. No actúas como tal, sino que eres realmente un rey, ya que Dios te ha concedido una autoridad real acorde con tu carácter.»

\par 25 Cuando el rey hubo aplaudido fuerte y largamente de la manera más amable, se instó a los invitados a buscar reposo. Así que cuando cesó la conversación, se dedicaron al siguiente plato del banquete.

\par 26 Al día siguiente se hizo lo mismo, y cuando el rey tuvo oportunidad de hacer preguntas a los hombres, interrogó al primero de los que habían quedado para el siguiente interrogatorio: ¿Cuál es la forma más elevada de ¿gobierno?

\par 27 Y él respondió: «Para gobernarse a uno mismo y no dejarse llevar por los impulsos. Porque todos los hombres poseen una cierta inclinación natural de la mente. Es probable que la mayoría de los hombres tengan inclinación por la comida, la bebida y los placeres, y los reyes por la adquisición de territorios y gran renombre. Pero es bueno que haya moderación en todo.»

\par 28 «Lo que Dios te da, eso debes tomarlo y conservarlo, pero nunca anhelar cosas que están fuera de tu alcance.»

\par 29 Complacido con estas palabras, el rey preguntó al siguiente: ¿Cómo podría estar libre de envidia?

\par 30 Y él, después de una breve pausa, respondió: «Si consideras en primer lugar que es Dios quien otorga a todos los reyes gloria y grandes riquezas y que nadie es rey por su propio poder. ¡Todos los hombres desean compartir esta gloria pero no pueden, ya que es don de Dios!»

\par 31 «El rey elogió al hombre en un largo discurso y luego preguntó a otro: ¿Cómo podía despreciar a sus enemigos?»

\par 32 Y él respondió: «Si eres bondadoso con todos los hombres y te ganas su amistad, no tendrás por qué temer a nadie. ¡Ser popular entre todos los hombres es el mejor de los buenos regalos que podemos recibir de Dios!»

\par 33 Habiendo elogiado esta respuesta, el rey ordenó al siguiente hombre que respondiera a la pregunta: ¿Cómo podría mantener su gran fama?

\par 34 Y él respondió: «Si eres generoso y de gran corazón al ofrecer bondad y actos de gracia a los demás, nunca perderás tu renombre, pero si deseas que las gracias antes mencionadas continúen siendo tuyas, debes invocar a Dios continuamente».'

\par 35 El rey expresó su aprobación y preguntó al siguiente: ¿Con quién debe el hombre ser liberal?

\par 36 Y él respondió: «Todos reconocen que debemos mostrarnos liberales con aquellos que están bien dispuestos hacia nosotros, pero creo que debemos mostrar el mismo espíritu agudo de generosidad hacia aquellos que se oponen a nosotros, que con esto significa que podemos ganárselos hacia la derecha y hacia lo que es ventajoso para nosotros. Pero debemos orar a Dios para que esto se logre, porque él gobierna las mentes de todos los hombres.'

\par 37 Habiendo expresado su acuerdo con la respuesta, el rey pidió al sexto que respondiera a la pregunta: ¿A quién debemos mostrar gratitud?

\par 38 Y él respondió: 'A nuestros padres continuamente, porque Dios nos ha dado un mandamiento muy importante con respecto al honor debido a los padres. A continuación, considera la actitud de un amigo hacia un amigo, porque habla de «un amigo que es como tu propia alma». Haces bien en intentar que todos los hombres se hagan amigos de ti mismo.

\par 39 El rey le habló amablemente y luego le preguntó al siguiente: ¿Qué es lo que se parece a la belleza en valor?

\par 40 Y él dijo: «La piedad, porque es la forma suprema de la belleza, y su poder reside en el amor, que es don de Dios. Esto ya lo has adquirido y con él todas las bendiciones de la vida.»

\par 41 El rey aplaudió muy amablemente la respuesta y preguntó a otro: ¿Cómo, si fracasaba, podría recuperar su reputación en el mismo grado?

\par 42 Y él dijo: «No es posible que fracases, porque has sembrado en todos los hombres las semillas de la gratitud que producen una cosecha de buena voluntad, y esto es más poderoso que las armas más poderosas y garantiza la mayor seguridad. Pero si alguien falla, nunca más debe hacer las cosas que causaron su fracaso, sino que debe formar amistades y actuar con justicia. Porque es don de Dios poder hacer buenas acciones y no al contrario.»

\par 43 Encantado con estas palabras, el rey preguntó a otro: ¿Cómo podría librarse del dolor?

\par 44 Y él respondió: «Si nunca hizo daño a nadie, sino que hizo el bien a todos y siguió el camino de la justicia, porque sus frutos liberan del dolor». Pero debemos orar a Dios para que males inesperados como la muerte, la enfermedad, el dolor o cualquier cosa por el estilo no nos sobrevengan y nos dañen. Pero como eres devoto de la piedad, jamás te sobrevendrá tal desgracia.

\par 45 El rey lo elogió mucho y preguntó al décimo: ¿Cuál es la forma más elevada de gloria?

\par 46 Y él dijo: «Para honrar a Dios, y esto no se hace con regalos y sacrificios, sino con pureza de alma y santa convicción, ya que todas las cosas están formadas y gobernadas por Dios de acuerdo con su voluntad». De este propósito estás en posesión constante, como todos los hombres, de tus logros en el pasado y en el presente.

\par 47 El rey los saludó a todos en alta voz y les habló amablemente, y todos los presentes expresaron su aprobación, especialmente los filósofos. Porque eran muy superiores a ellos [es decir, a los filósofos] tanto en conducta como en argumentos, ya que siempre tomaban de Dios su punto de partida.

\par 48 Después de esto, el rey, para mostrar su buen sentimiento, procedió a beber por la salud de sus invitados.

\par \textit{Notas al pie}

\par \textit{165:1 ¡Escrito alrededor del 150 a.C.!}

\chapter{9}

\par \textit{El versículo 8 personifica el valor del conocimiento. Versículo 28, cariño paternal. Note especialmente la pregunta del versículo 26 y la respuesta. Note también la pregunta en el versículo 47 y la respuesta. Este es un sabio consejo para hombres de negocios.}

\par 1 Al día siguiente se hicieron los mismos arreglos para el banquete, y el rey, tan pronto como se presentó la oportunidad, comenzó a hacer preguntas a los hombres que estaban sentados junto a los que ya habían respondido, y dijo al primero «¿Se puede enseñar la sabiduría?»

\par 2 Y él dijo: «El alma está constituida de tal manera que, por el poder divino, puede recibir todo el bien y rechazar lo contrario».

\par 3 El rey expresó su aprobación y preguntó al siguiente: ¿Qué es lo más beneficioso para la salud?

\par 4 Y él dijo: «La templanza, y no es posible adquirirla a menos que Dios cree una disposición hacia ella».

\par 5 El rey habló amablemente con el hombre y le dijo a otro: «¿Cómo puede un hombre pagar dignamente la deuda de gratitud a sus padres?»

\par 6 Y él dijo: «Sin causarles nunca dolor, y esto no es posible a menos que Dios disponga la mente para perseguir los fines más nobles».

\par 7 El rey expresó su acuerdo y le preguntó al siguiente: ¿Cómo podría convertirse en un oyente entusiasta?

\par 8 Y él dijo: «Recordando que todo conocimiento es útil, porque te permite, con la ayuda de Dios, en un momento de emergencia, seleccionar algunas de las cosas que has aprendido y aplicarlas a la crisis que enfrentas. Y así los esfuerzos de los hombres se cumplen con la ayuda de Dios.»

\par 9 El rey lo elogió y le preguntó al siguiente: ¿Cómo podría evitar hacer algo contrario a la ley?

\par 10 Y él dijo: «Si reconocéis que es Dios quien ha puesto en el corazón de los legisladores el pensamiento de preservar la vida de los hombres, los seguiréis».

\par 11 El rey escuchó la respuesta del hombre y dijo a otro: «¿Cuál es la ventaja del parentesco?»

\par 12 Y él respondió: «Si consideramos que nosotros mismos somos afligidos por las desgracias que caen sobre nuestros parientes y si sus sufrimientos se convierten en los nuestros, entonces la fuerza del parentesco se manifiesta de inmediato, porque sólo cuando tal sentimiento se expresa demostrado que ganaremos honor y estima ante sus ojos. Porque la ayuda, cuando está unida a la bondad, es en sí misma un vínculo totalmente indisoluble. Y en el día de su prosperidad no debemos anhelar sus posesiones, sino que debemos orar a Dios para que les conceda todo tipo de bien.'

\par 13 Y después de haberle concedido los mismos elogios que a los demás, el rey preguntó a otro: ¿Cómo podría liberarse del miedo?

\par 14 Y él dijo: «Cuando la mente es consciente de que no ha hecho ningún mal y cuando Dios la orienta hacia todos los buenos consejos».

\par 15 El rey expresó su aprobación y preguntó a otro: ¿Cómo podría mantener siempre un juicio correcto?

\par 16 Y él respondió: «Si constantemente pone ante sus ojos las desgracias que acontecen a los hombres y reconoce que es Dios quien quita la prosperidad a unos y lleva a otros a gran honor y gloria.»

\par 17 El rey lo recibió amablemente y le pidió al siguiente que respondiera la pregunta: ¿Cómo podría evitar una vida tranquila y placentera?

\par 18 Y él respondió: «Si recordaba continuamente que era el gobernante de un gran imperio y el señor de grandes multitudes, y que su mente no debería estar ocupada con otras cosas, sino que debería estar siempre considerando cómo él podría promover mejor su bienestar. También debe orar a Dios para que no se descuide ningún deber.

\par 19 Después de haberlo elogiado, el rey preguntó al décimo: ¿Cómo podía reconocer a los que lo traicionaban?

\par 20 Y él respondió a la pregunta: «Si observara si el porte de los que lo rodeaban era natural y si mantenían la debida regla de precedencia en las recepciones y concilios, y en sus relaciones generales, sin excederse nunca de los límites del decoro en felicitaciones o en otros asuntos de comportamiento. Pero Dios inclinará tu mente, oh Rey, hacia todo lo que es noble.'

\par 21 Cuando el rey expresó su aprobación en voz alta y los alabó a todos individualmente (en medio de los aplausos de todos los presentes), se dedicaron a disfrutar de la fiesta.

\par 22 Y al día siguiente, cuando se presentó la oportunidad, el rey preguntó al siguiente: ¿Cuál es la forma más grave de negligencia?

\par 23 Y él respondió: «Si un hombre no se preocupa por sus hijos y no dedica todo su esfuerzo a educarlos. Porque siempre oramos a Dios no tanto por nosotros sino por nuestros hijos, para que toda bendición sea para ellos. Nuestro deseo de que nuestros hijos posean dominio propio sólo se realiza mediante el poder de Dios.»

\par 24 El rey dijo que había hablado bien y luego preguntó a otro: ¿Cómo podía ser patriótico?

\par 25 «Teniendo presente», respondió él, «el pensamiento de que es bueno vivir y morir en el propio país». La residencia en el extranjero 1 acarrea desprecio a los pobres y vergüenza a los ricos, como si hubieran sido desterrados por un delito. Si concedes beneficios a todos, como lo haces continuamente, Dios te concederá el favor de todos y serás considerado patriótico.'

\par 26 Después de escuchar a este hombre, el rey preguntó al siguiente: ¿Cómo podría vivir amigablemente con su esposa?

\par 27 Y él respondió: «Al reconocer que las mujeres son por naturaleza testarudas y enérgicas en la búsqueda de sus propios deseos, y sujetas a cambios repentinos de opinión a través de razonamientos falaces, y su naturaleza es esencialmente débil. Es necesario tratarlos sabiamente y no provocar conflictos. Para la conducta exitosa de la vida el timonel debe conocer la meta hacia la cual debe dirigir su rumbo. Sólo invocando la ayuda de Dios pueden los hombres seguir un verdadero curso de vida en todo momento.»

\par 28 El rey expresó su acuerdo y preguntó al siguiente: ¿Cómo podría estar libre de error?

\par 29 Y él respondió: «Si actúas siempre con prudencia y nunca das crédito a las calumnias, sino que compruebas por ti mismo lo que te dicen y resuelves con tu propio juicio las peticiones que te hacen y cumples todo en a la luz de tu juicio, estarás libre de error, oh Rey. Pero el conocimiento y la práctica de estas cosas es obra del poder divino.'

\par 30 Encantado con estas palabras, el rey preguntó a otro: ¿Cómo podría librarse de la ira?

\par 31 Y respondió a la pregunta: «Si reconociera que tiene poder sobre todos, incluso para infligirles la muerte, si cediera a la ira, y que sería inútil y lamentable si él, sólo porque Fue señor, privó a muchos de la vida.'

\par 32 '¿Qué necesidad había de ira, cuando todos los hombres estaban sometidos y nadie le era hostil? Es necesario reconocer que Dios gobierna el mundo entero con espíritu de bondad y sin ira alguna, y tú«, dijo, »oh Rey, debes necesariamente copiar su ejemplo».

\par 33 El rey dijo que había respondido bien y luego preguntó al siguiente: ¿Qué es un buen consejo?

\par 34 «Actuar bien en todo momento y con la debida reflexión», explicó, "comparando lo que es ventajoso para nuestra propia política con los efectos perjudiciales que resultarían de la adopción del punto de vista opuesto, para que, sopesando cada punto podremos estar bien asesorados y nuestro propósito podrá lograrse. Y lo más importante de todo es que, por el poder de Dios, cada plan tuyo se cumplirá porque practicas la piedad.'

\par 35 El rey dijo que éste había respondido bien y preguntó a otro: ¿Qué es la filosofía?

\par 36 Y explicó: 'Deliberar bien sobre cualquier cuestión que surja y nunca dejarse llevar por los impulsos, sino reflexionar sobre los daños que resultan de las pasiones, y actuar rectamente según lo exijan las circunstancias, practicando la moderación. Pero debemos orar a Dios para que inculque en nuestra mente el respeto por estas cosas.

\par 37 El rey dio su consentimiento y preguntó a otro: ¿Cómo podría obtener reconocimiento cuando viajaba al extranjero?

\par 38 «Siendo justo con todos los hombres», respondió, «y pareciendo inferior en lugar de superior a aquellos entre quienes viajaba». Porque es un principio reconocido que Dios por su propia naturaleza acepta a los humildes. Y el género humano ama a quienes están dispuestos a someterse a ellos.'

\par 39 Habiendo expresado su aprobación por esta respuesta, el rey preguntó a otro: ¿Cómo podría construir de tal manera que sus estructuras perduraran después de él?

\par 40 Y él respondió a la pregunta: «Si sus creaciones fueran grandes y nobles, de modo que los espectadores las respetaran por su belleza, y si nunca despidiera a ninguno de los que realizaron tales obras y nunca obligara a otros a hacerlo ministre sus necesidades sin salario.'

\par 41 Por observar cómo Dios provee a la raza humana, otorgándoles salud y capacidad mental y todos los demás dones, él mismo debe seguir su ejemplo dando a los hombres una recompensa por su arduo trabajo. 1 ¡Porque son las obras realizadas con justicia las que permanecen continuamente!

\par 42 El rey dijo que también éste había respondido bien y preguntó al décimo: ¿Cuál es el fruto de la sabiduría?

\par 43 Y él respondió: «Para que el hombre tenga conciencia de que no ha hecho ningún mal y que viva su vida en la verdad». Puesto que es de ellos, oh Rey poderoso, que te corresponde la mayor alegría y la firmeza del alma y la fuerte fe en Dios si gobiernas tu reino con piedad.'

\par 44 Y cuando oyeron la respuesta, todos gritaron con gran aclamación, y después el rey, en la plenitud de su alegría, comenzó a beber por su salud.

\par 45 Y al día siguiente el banquete siguió el mismo curso que las ocasiones anteriores, y cuando se presentó la oportunidad, el rey procedió a hacer preguntas a los demás invitados, y dijo al primero: «¿Cómo puede un hombre mantenerse a sí mismo?» ¿Por orgullo?

\par 46 Y él respondió: «Si mantiene la igualdad y recuerda en todo momento que es un hombre que domina a los hombres. ¡Y Dios anula a los soberbios y exalta a los mansos y humildes!»

\par 47 El rey le habló amablemente y le preguntó: ¿A quién debe elegir un hombre como consejero?

\par 48 Y él respondió: «Aquellos que han sido probados en muchos asuntos y mantienen una absoluta buena voluntad hacia él y participan de su propio carácter». Y Dios se manifiesta a aquellos que son dignos de que se alcancen estos fines.'

\par 49 El rey lo alabó y preguntó a otro: ¿Cuál es el bien más necesario para un rey?

\par 50 «La amistad y el amor de sus súbditos», respondió, «porque es por esto que el vínculo de buena voluntad se vuelve indisoluble». Y es Dios quien garantiza que esto suceda según tu deseo.'

\par 51 El rey lo alabó y preguntó a otro: ¿Cuál es el objetivo de la palabra? Y él respondió: «Convencer a tu oponente mostrándole sus errores en un ejército de argumentos bien ordenado».

\par 52 «Porque así conquistarás a tu oyente, no oponiéndote a él, sino alabándolo con el fin de persuadirlo. Y es por el poder de Dios que se logra la persuasión.»

\par 53 El rey dijo que había dado una buena respuesta y preguntó a otro: ¿Cómo podría vivir amigablemente con las diferentes razas que formaban la población de su reino?

\par 54 «Actuando como corresponde hacia cada uno», respondió, «y tomando la justicia como guía, como lo estás haciendo ahora con la ayuda de la percepción que Dios te concede».

\par 55 El rey quedó encantado con esta respuesta y preguntó a otro: «¿En qué circunstancias debe un hombre sufrir dolor?»

\par 56 «En las desgracias que acontecen a nuestros amigos», respondió, «cuando vemos que son prolongadas e irremediables. La razón no nos permite lamentarnos por los que están muertos y liberados del mal, pero todos los hombres sí se lamentan por ellos porque sólo piensan en sí mismos y en su propio beneficio. ¡Sólo por el poder de Dios podemos escapar de todo mal!»

\par 57 El rey dijo que había dado una respuesta adecuada y preguntó a otro: ¿Cómo se pierde la reputación?

\par 58 Y él respondió: «Cuando dominan el orgullo y la confianza ilimitada en uno mismo, se engendra deshonra y pérdida de reputación. Porque Dios es el Señor de toda reputación y la otorga donde quiere.»

\par 59 El rey confirmó la respuesta y preguntó al siguiente: ¿A quién deben confiar los hombres?

\par 60 «A aquellos», respondió, «que os sirven por buena voluntad y no por miedo o interés propio, pensando sólo en su propio beneficio. Porque uno es el signo del amor, el otro la señal de la mala voluntad y del desinterés.»

\par 61 'Porque el hombre que siempre está buscando su propio beneficio es un traidor de corazón. Pero posees el afecto de todos tus súbditos con la ayuda del buen consejo que Dios te da.

\par 62 El rey dijo que había respondido sabiamente y preguntó a otro: ¿Qué es lo que mantiene seguro un reino?

\par 63 Y él respondió a la pregunta: «Cuidado y previsión para que los que están en posición de autoridad sobre el pueblo no hagan ningún mal, y esto lo hacéis siempre con la ayuda de Dios, que os inspira un juicio serio».'

\par 64 El rey le habló palabras de aliento y preguntó a otro: ¿Qué es lo que mantiene la gratitud y el honor?

\par 65 Y él respondió: «La virtud, porque es la creadora de las buenas obras, y por ella se destruye el mal, así como tú muestras nobleza de carácter hacia todos por el don que Dios te concede.»

\par 66 El rey agradeció la respuesta y preguntó al undécimo (ya que eran dos más de setenta), ¿cómo podría en tiempo de guerra mantener la tranquilidad del alma?

\par 67 Y él respondió: 'Recordando que no había hecho ningún mal a ninguno de sus súbditos, y que todos lucharían por él a cambio de los beneficios que habían recibido, sabiendo que incluso si pierden la vida, tú cuidar a quienes dependen de ellos. Porque nunca dejáis de reparar a nadie; tal es la bondad que Dios os ha inspirado.

\par 68 El rey los aplaudió a todos con gran fuerza, les habló muy amablemente y luego bebió un largo trago a la salud de cada uno, entregándose al disfrute y prodigando a sus invitados la amistad más generosa y alegre.

\par \textit{Notas al pie}

\par \textit{169:1 También había residentes extranjeros en aquellos días.}

\par \textit{170:1 Aquí se considera que la política de un salario justo por un día de trabajo justo no es tan moderna como a veces pensamos en lo que nos complace llamar esta era ilustrada.}

\chapter{10}

\par \textit{Las preguntas y respuestas continúan. Mostrando cómo se deben seleccionar los oficiales del ejército. Lo que el hombre es digno de admiración y otros problemas de la vida diaria son tan ciertos hoy como hace 2000 años. Los versículos 15-17 se destacan por recomendar el teatro. Los versículos 2i-22 describen la sabiduría de elegir un presidente o tener un rey.}

\par 1 El séptimo día se hicieron muchos más preparativos y muchos otros estaban presentes de diferentes ciudades (entre ellos un gran número de embajadores).

\par 2 Cuando se presentó una oportunidad, el rey preguntó al primero de los que aún no habían sido interrogados, ¿cómo podría evitar ser engañado por razonamientos falaces?

\par 3 Y él respondió: «Observando atentamente al que habla, lo que se habla y el tema en discusión, y volviendo a plantear las mismas preguntas después de un intervalo en diferentes formas. Pero poseer una mente alerta y poder formarse un buen juicio en cada caso es uno de los buenos dones de Dios, y tú lo posees, oh Rey.»

\par 4 El rey aplaudió ruidosamente la respuesta y preguntó a otro: ¿Por qué la mayoría de los hombres nunca llegan a ser virtuosos?

\par 5 «Porque», respondió, «todos los hombres son por naturaleza intemperantes e inclinados al placer». De ahí surge la injusticia y un torrente de avaricia. El hábito de la virtud es un obstáculo para quienes se dedican a una vida de placer porque les impone la preferencia de la templanza y la rectitud. Porque Dios es el dueño de estas cosas.'

\par 6 El rey dijo que había respondido bien y preguntó: ¿Qué deben obedecer los reyes? Y él dijo: 'Las leyes, para que mediante promulgaciones justas puedan restaurar la vida de los hombres. Así como tú, con tal conducta, en obediencia al mandato divino, te has reservado un memorial perpetuo.

\par 7 El rey dijo que también éste había hablado bien y preguntó al siguiente: ¿A quién debemos nombrar gobernadores?

\par 8 Y él respondió: 'Todos los que odian la maldad e imitan su propia conducta, actúan con rectitud para poder mantener una buena reputación constantemente. Porque esto es lo que haces, oh Rey poderoso«, dijo, »y es Dios quien te ha concedido la corona de justicia».

\par 9 El rey aclamó en voz alta la respuesta y luego, mirando al siguiente hombre, dijo: «¿A quién debemos nombrar como oficiales sobre las fuerzas?»

\par 10 Y él explicó: 'Aquellos que sobresalen en coraje y rectitud y aquellos que están más preocupados por la seguridad de sus hombres que por obtener una victoria arriesgando sus vidas por temeridad. Porque así como Dios actúa bien con todos los hombres, así también tú, a imitación de Él, eres el benefactor de todos tus súbditos.

\par 11 El rey dijo que había dado una buena respuesta y preguntó a otro: ¿Qué hombre es digno de admiración?

\par 12 Y él respondió: «El hombre que está dotado de reputación, riqueza y poder y posee un alma igual a todo eso». Tú misma demuestras con tus acciones que eres muy digna de admiración gracias a la ayuda de Dios que te hace cuidar de estas cosas.'

\par 13 El rey expresó su aprobación y dijo a otro: «¿A qué asuntos deben dedicar los reyes más tiempo?»

\par 14 Y él respondió: «A la lectura y estudio de las actas de los viajes oficiales, que se escriben en referencia a los distintos reinos, con miras a la reforma y preservación de los súbditos. Y es mediante tal actividad que habéis alcanzado una gloria que otros nunca han alcanzado, con la ayuda de Dios que cumple todos vuestros deseos.»

\par 15 El rey habló con entusiasmo al hombre y le preguntó a otro: ¿En qué debe ocuparse un hombre durante sus horas de descanso y recreación?

\par 16 Y él respondió: «Ver aquellas obras que se pueden representar con propiedad y presentar ante los ojos escenas tomadas de la vida y representadas con dignidad y decencia es provechoso y apropiado».

\par 17 «Pues incluso en estas diversiones se puede encontrar algo de edificación, pues a menudo las cosas más insignificantes de la vida enseñan alguna lección deseable. Pero al practicar la máxima corrección en todas tus acciones, has demostrado que eres un filósofo y que Dios te honra a causa de tu virtud.»

\par 18 El rey, contento con las palabras que acababan de pronunciar, dijo al noveno hombre: ¿Cómo debe comportarse un hombre en los banquetes?

\par 19 Y él respondió: «Deberías convocar a tu lado a hombres eruditos y a aquellos que puedan darte consejos útiles con respecto a los asuntos de tu reino y la vida de tus súbditos (porque no pudiste encontrar ningún tema más adecuado o más educativo que esto) ya que tales hombres son queridos por Dios porque han entrenado sus mentes para contemplar los temas más nobles, como de hecho lo estás haciendo tú mismo, ya que todas tus acciones están dirigidas por Dios.»

\par 20 Encantado con la respuesta, el rey preguntó al siguiente: ¿Qué es lo mejor para el pueblo? ¿Que un ciudadano privado debería ser nombrado rey sobre ellos o miembro de la familia real?

\par 21 Y él respondió: «El que es mejor por naturaleza». Porque los reyes que provienen de linaje real son a menudo duros y severos con sus súbditos. Y aún más es el caso de algunos de los que han surgido de las filas de los ciudadanos privados, que después de haber experimentado el mal y haber soportado su parte de pobreza, cuando gobiernan a multitudes resultan ser más crueles que los tiranos impíos.

\par 22 «Pero, como ya he dicho, una buena naturaleza bien educada es capaz de gobernar, y tú eres un gran rey, no tanto porque sobresalgas en la gloria de tu gobierno y tus riquezas, sino más bien porque Habéis superado a todos los hombres en clemencia y filantropía, gracias a Dios que os ha dotado de estas cualidades.'

\par 23 El rey pasó algún tiempo alabando a este hombre y luego preguntó al último de todos: ¿Cuál es el mayor logro al gobernar un imperio?

\par 24 Y él respondió: «Que los súbditos vivan siempre en paz y que se haga justicia rápidamente en los casos de disputa».

\par 25 «Estos resultados se logran mediante la influencia del gobernante, cuando es un hombre que odia el mal y ama el bien y dedica sus energías a salvar las vidas de los hombres, así como consideras la injusticia como la peor forma del mal y por tu justa administración te ha forjado una reputación eterna, ya que Dios te concede una mente pura y sin mancha de ningún mal.»

\par 26 Y cuando terminó, estallaron grandes y alegres aplausos que duraron mucho tiempo. Cuando se detuvo, el rey tomó una copa y brindó en honor de todos sus invitados y de las palabras que habían pronunciado.

\par 27 «Y para concluir dijo: He obtenido el mayor beneficio de tu presencia. Me he beneficiado mucho del sabio almacenamiento en caché que me habéis dado en referencia al arte de gobernar.»

\par 28 Luego ordenó que se les presentaran tres talentos de plata a cada uno de ellos y nombró a uno de sus esclavos para que entregara el dinero.

\par 29 Todos aclamaron a gritos y el banquete se convirtió en un escenario de alegría, mientras el rey se entregaba a una continua fiesta.

\chapter{11}

\par \textit{Para un comentario sobre estenografía antigua, vea el versículo 7. La traducción se presenta para su aprobación y se acepta como leída, y (versículo 23) se toma un voto creciente de aprobación y se aprueba por unanimidad.}

\par 1 HE escrito extensamente y debo pedir tu perdón, Filócrates.

\par 2 Me quedé extraordinariamente asombrado por los hombres y por la forma en que, espontáneamente, dieron respuestas que realmente necesitaron mucho tiempo para idear.

\par 3 Pues, aunque el que preguntaba había pensado mucho en cada pregunta concreta, los que respondieron uno tras otro tenían sus respuestas preparadas de inmediato y así me parecieron a mí y a todos los presentes, y especialmente a los filósofos, ser digno de admiración.

\par 4 Y supongo que la cosa les parecerá increíble a quienes lean mi narración en el futuro.

\par 5 Pero es indecoroso tergiversar hechos que están registrados en los archivos públicos.

\par 6 Y no sería justo que yo transgrediera en tal asunto. Cuento la historia tal como sucedió, evitando concienzudamente cualquier error.

\par 7 Quedé tan impresionado por la fuerza de sus declaraciones, que hice un esfuerzo por consultar a aquellos cuya misión era registrar todo lo que sucedía en las audiencias y banquetes reales.

\par 8 Porque, como sabéis, es costumbre, desde el momento en que el rey comienza a hacer sus negocios hasta el momento en que se retira a descansar, que se lleve un registro de todos sus dichos y acciones, lo cual es un arreglo excelente y útil.

\par 9 Porque al día siguiente se lee el acta de los hechos y dichos del día anterior antes de comenzar los negocios, y si ha habido alguna irregularidad, el asunto se arregla inmediatamente.

\par 10 Por lo tanto, obtuve, como se ha dicho, información precisa de los registros públicos, y he expuesto los hechos en el orden correcto, ya que sé cuán ansioso está usted por obtener información útil.

\par 11 Tres días después, Demetrio tomó a los hombres y, pasando por el malecón de siete estadios de largo, llegó a la isla, cruzó el puente y se dirigió a los distritos del norte de Faro.

\par 12 Allí los reunió en una casa muy hermosa y apartada, construida a la orilla del mar, y los invitó a realizar el trabajo de traducción, ya que todo lo que necesitaban para ello estaba disponible puesto a su disposición.

\par 13 Entonces se pusieron a comparar los diferentes resultados y a ponerlos de acuerdo, y todo lo que habían acordado fue copiado convenientemente bajo la dirección de Demetrio.

\par 14 Y la sesión duró hasta la hora novena; después de esto fueron liberados para atender sus necesidades físicas.

\par 15 Todo lo que querían les fue proporcionado en gran escala. Además de esto, Doroteo hacía diariamente para ellos los mismos preparativos que para el propio rey, pues así se lo había ordenado el rey.

\par 16 Temprano en la mañana se presentaban cada día en la corte y, después de saludar al rey, regresaban a su lugar.

\par 17 Y como es costumbre de todos los judíos, se lavaron las manos en el mar y oraron a Dios y luego se dedicaron a leer y traducir el pasaje en particular en el que estaban ocupados, y yo les pregunté: ¿Por qué? ¿Fue que se lavaron las manos antes de orar?

\par 18 Y explicaron que era una señal de que no habían hecho ningún mal (porque toda actividad se realiza con las manos), ya que en su noble y santa manera consideran todo como un símbolo de justicia y verdad.

\par 19 Como ya he dicho, se reunían diariamente en el lugar que era encantador por su tranquilidad y su claridad y se aplicaban a su tarea.

\par 20 Y sucedió que la obra de traducción se completó en setenta y dos días, como si hubiera sido dispuesta con un propósito determinado.

\par 21 Cuando terminó la obra, Demetrio reunió a la población judía en el lugar donde se había hecho la traducción y la leyó a todos, en presencia de los traductores, quienes fueron muy bien recibidos también por parte del pueblo, por los grandes beneficios que les habían conferido.

\par 22 También elogiaron calurosamente a Demetrio y le instaron a que hiciera transcribir toda la ley y la presentara a sus jefes.

\par 23 Después de leer los libros, los sacerdotes, los ancianos de los traductores, la comunidad judía y los líderes del pueblo se levantaron y dijeron que, puesto que se había hecho una traducción tan excelente, sagrada y exacta, solo era posible derecho a que permanezca como estaba y no se haga ninguna alteración en él.

\par 24 Y cuando todo el grupo expresó su aprobación, les ordenaron que, según su costumbre, pronunciaran una maldición contra cualquiera que hiciera alguna alteración, ya sea agregando algo o cambiando de cualquier manera alguna de las palabras que habían sido escritas o haciendo cualquier omisión.

\par 25 Esta fue una precaución muy sabia para garantizar que el libro pudiera conservarse sin cambios en el futuro.

\par 26 Cuando el rey conoció el asunto, se alegró mucho, porque sabía que el plan que había trazado se había llevado a cabo con seguridad.

\par 27 Le leyeron todo el libro y quedó muy asombrado del espíritu del legislador.

\par 28 Y dijo a Demetrio: «¿Cómo es posible que ninguno de los historiadores ni de los poetas haya considerado jamás que valiera la pena aludir a tan maravilloso logro?»

\par 29 Y él respondió: Porque la ley es sagrada y de origen divino. Y algunos de los que tenían la intención de ocuparse de ello han sido heridos por Dios y por lo tanto desistieron de su propósito.'

\par 30 Dijo que había oído de Teopompo que había estado enloquecido durante más de treinta días porque tenía la intención de incluir en su historia algunos de los incidentes de las traducciones anteriores y poco fiables de la ley.

\par 31 Cuando se recuperó un poco, rogó a Dios que le aclarara por qué le había sucedido aquella desgracia.

\par 32 Y en un sueño se le reveló que por vana curiosidad deseaba comunicar verdades sagradas a los hombres comunes, y que si desistía, recuperaría su salud.

\par 33 También he oído de labios de Teodectes, uno de los poetas trágicos, que cuando estaba a punto de adaptar algunos de los incidentes registrados en el libro para una de sus obras, sufrió cataratas en ambos ojos.

\par 34 Y cuando comprendió la causa de su desgracia, oró a Dios durante muchos días y después fue restablecido.

\par 35 Y cuando el rey, como ya he dicho, recibió la explicación de Demetrio sobre este punto, rindió homenaje y ordenó que se tuvieran mucho cuidado con los libros y que se guardaran sagradamente.

\par 36 Y exhortó a los traductores a que lo visitaran con frecuencia después de su regreso a Judea, porque era justo, dijo, que ahora los enviara a casa.

\par 37 Pero cuando regresaran, él los trataría como a amigos, como era correcto, y recibirían ricos regalos de su parte.

\par 38 Ordenó que se hicieran preparativos para su regreso a casa y los trató con gran generosidad.

\par 39 Y les regaló a cada uno tres vestidos de primera calidad, dos talentos de oro, un aparador que pesaba un talento y todos los muebles para tres divanes.

\par 40 Y con la escolta envió a Eleazar diez camillas con patas de plata y todo el equipamiento necesario, un aparador por valor de treinta talentos, diez túnicas, púrpura y una corona magnífica, y cien piezas de lino fino tejido, además de cuencos y platos y dos vasos de oro para dedicarlos a Dios.

\par 41 También le instó en una carta a que si alguno de los hombres prefería volver a él, no se lo impidiera.

\par 42 Porque consideraba un gran privilegio disfrutar de la compañía de hombres tan eruditos, y prefería prodigar su riqueza en ellos antes que en vanidades.

\par 43 Y ahora Filócrates, tienes la historia completa de acuerdo con mi promesa.

\par 44 Creo que estas cosas te complacen más que los escritos de los mitólogos.

\par 45 Porque te dedicas al estudio de aquellas cosas que pueden beneficiar al alma, y ​​dedicas mucho tiempo a ello. Intentaré narrar cualquier otro acontecimiento que valga la pena registrar, para que al leerlo puedas obtener la mayor recompensa por tu celo.


\end{document}