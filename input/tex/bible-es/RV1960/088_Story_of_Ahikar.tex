\begin{document}

\title{Historia de Ahikar}

\chapter{1}

\par \textit{Ahikar, Gran Visir de Asiria, tiene 60 esposas pero está destinado a no tener ningún hijo. Por eso adopta a su sobrino. Lo llena de sabiduría y conocimiento más que de pan y agua.}

\par 1 LA historia de Haiqâr el Sabio, Visir del Rey Senaquerib, y de Nadan, hijo de la hermana de Haiqâr el Sabio.

\par 2 Había un visir en los días del rey Senaquerib, hijo de Sarhadum, rey de Asiria y de Nínive, un hombre sabio llamado Haiqâr, y era visir del rey Senaquerib.

\par 3 Tenía riqueza, fortuna y muchos bienes, y era hábil, sabio, filósofo, en conocimiento, en opinión y en gobernar, y se había casado con sesenta mujeres, y había construido un castillo para cada una de ellas.

\par 4 Pero con todo esto no tuvo hijo de nadie. de estas mujeres, quien podría ser su heredera.

\par 5 Y él estaba muy triste por esto, y un día reunió a los astrólogos, a los eruditos y a los magos, y les explicó su condición y el motivo de su esterilidad.

\par 6 Y ellos le dijeron: «Ve, sacrifica a los dioses y suplicales que tal vez te proporcionen un niño».

\par 7 E hizo lo que le dijeron y ofreció sacrificios a los ídolos, y les suplicó y les imploró con ruegos y súplicas.

\par 8 Y ellos no le respondieron ni una palabra. Y él se fue triste y abatido, con dolor en el corazón.

\par 9 Y volviendo, imploró al Dios Altísimo, y creyó, rogándole con un ardor en su corazón, diciendo: '¡Oh Dios Altísimo, oh Creador de los cielos y de la tierra, oh Creador de todas las cosas creadas! !'

\par 10 «Te ruego que me des un niño para que él me consuele y pueda estar presente en mi salud, para que me cierre los ojos y me entierre».

\par 11 Entonces vino a él una voz que le decía: «Por cuanto te has apoyado primero en imágenes talladas y les has ofrecido sacrificios, por eso no tendrás hijos durante toda tu vida».

\par 12 «Pero toma a Nadan, el hijo de tu hermana, y hazlo tuyo, y enséñale tus conocimientos y tu buena educación, y cuando mueras, él te enterrará».

\par 13 Entonces tomó a Nadan, el hijo de su hermana, que estaba de pecho. Y lo entregó a ocho nodrizas, para que lo amamantaran y lo criaran.

\par 14 Y lo criaron con buena comida, educación suave y vestidos de seda, púrpura y carmesí. Y estaba sentado en lechos de seda.

\par 15 Y cuando Nadan creció y caminaba como un cedro alto, le enseñó buenos modales, escritura, ciencia y filosofía.

\par 16 Y después de muchos días, el rey Senaquerib miró a Haiqâr y vio que había envejecido mucho, y además le dijo.

\par 17 '¡Oh, mi honorable amigo, el hábil, el confiable, el sabio, el gobernador, mi secretario, mi visir, mi canciller y director! en verdad estás muy viejo y cargado de años; y tu partida de este mundo debe estar cerca.'

\par 18 «Dime quién tendrá un lugar en mi servicio después de ti». Y Haiqâr le dijo: '¡Oh mi señor, que tu cabeza viva para siempre! Allí está Nadan, el hijo de mi hermana; lo he hecho mi hijo.'

\par 19 «Y yo lo crié y le enseñé mi sabiduría y mi conocimiento».

\par 20 Y el rey le dijo: '¡Oh Haiqâr! tráelo a mi presencia, para que pueda verlo, y si lo encuentro adecuado, ponlo en tu lugar; y te irás para descansar y vivir el resto de tu vida en dulce reposo.

\par 21 Entonces Haiqâr fue y le presentó a Nadan el hijo de su hermana. Y le rindió homenaje y le deseó poder y honor.

\par 22 Y él lo miró y lo admiró y se regocijó en él y dijo a Haiqâr: '¿Es este tu hijo, oh Haiqâr? Ruego que Dios lo preserve. Y así como tú nos has servido a mí y a mi padre Sarhadum, que este hijo tuyo me sirva y cumpla con mis empresas, mis necesidades y mis negocios, para que pueda honrarlo y hacerlo poderoso por tu causa.'

\par 23 Y Haiqâr rindió homenaje al rey y le dijo: '¡Viva tu cabeza, oh mi señor el rey, para siempre! Te pido que puedas ser paciente con mi muchacho Nadan y perdonar sus errores para que pueda servirte como corresponde.'

\par 24 Entonces el rey le juró que lo convertiría en el mayor de sus favoritos y en el más poderoso de sus amigos, y que estaría con él en todo honor y respeto. Y le besó las manos y se despidió.

\par 25 Y tomó a Nadan. El hijo de su hermana estaba con él, lo sentó en una sala y se puso a enseñarle día y noche, hasta llenarlo de sabiduría y conocimiento más que de pan y agua.



\chapter{2}

\par \textit{Un «Almanaque del pobre Ricardo» de la antigüedad. Preceptos inmortales de conducta humana respecto al dinero, las mujeres, la vestimenta, los negocios, los amigos. Se encuentran proverbios especialmente interesantes en los versículos 12, 17, 23, 37, 45, 47. Compare el versículo 63 con algo del cinismo de hoy.}

\par 1 ASÍ le enseñó, diciendo: '¡Oh hijo mío! Escucha mi discurso, sigue mis consejos y recuerda lo que digo.'

\par 2 '¡Oh hijo mío! Si oyes una palabra, déjala morir en tu corazón, y no la reveles a otro, no sea que se convierta en un carbón encendido, y queme tu lengua, y te cause dolor en el cuerpo, y gane reproche, y seas avergonzado delante de Dios y hombre.'

\par 3 '¡Oh hijo mío! Si has oído una noticia, no la difundas; y si has visto algo, no lo digas.

\par 4 '¡Oh hijo mío! haz que tu elocuencia sea fácil para el oyente y no te apresures a dar una respuesta.

\par 5 '¡Oh hijo mío! Cuando hayas oído algo, no lo ocultes.

\par 6 '¡Oh hijo mío! No desatéis un nudo sellado, ni lo desatéis, ni selléis un nudo aflojado.'

\par 7 '¡Oh hijo mío! No codiciéis la belleza exterior, porque decae y pasa, pero un recuerdo honorable dura para siempre.'

\par 8 '¡Oh hijo mío! No dejes que una mujer tonta te engañe con su discurso, no sea que mueras de la manera más miserable y ella te enrede en la red hasta que quedes atrapado.'

\par 9 '¡Oh hijo mío! No desees a una mujer adornada con vestidos y ungüentos, que es despreciable y necia de alma. ¡Ay de ti si le das algo que es tuyo, o le encomiendas lo que tienes en la mano y ella te induce a pecar, y Dios se enoja contigo!

\par 10 '¡Oh hijo mío! No seáis como el almendro, que produce hojas antes que todos los árboles, y frutos comestibles después de todos, sino sed como el morero, que produce frutos comestibles antes que todos los árboles, y hojas después de todos.'

\par 11 '¡Oh hijo mío! Inclina tu cabeza hacia abajo, suaviza tu voz, sé cortés, camina por el camino recto y no seas tonto. Y no levantes la voz cuando rías, porque si a gran voz se construyera una casa, el asno construiría muchas casas cada día; y si fuera a fuerza de fuerza que se impulsara el arado, el arado nunca sería quitado de debajo de los hombros de los camellos.'

\par 12 '¡Oh hijo mío! Mejor es quitar piedras con un hombre sabio que beber vino con un hombre afligido.'

\par 13 '¡Oh hijo mío! Derrama tu vino sobre las tumbas de los justos y no bebas con gente ignorante y despreciable.'

\par 14 '¡Oh hijo mío! Adhiérete a los sabios que temen a Dios y son como ellos, y no te acerques a los ignorantes, para que no seas como él y aprendas sus caminos.'

\par 15 '¡Oh hijo mío! cuando tengas un camarada o un amigo, pruébalo y luego hazlo camarada y amigo; y no lo alabéis sin prueba; y no estropees tu palabra con un hombre falto de sabiduría.'

\par 16 '¡Oh hijo mío! Mientras un zapato permanezca en tu pie, camina con él sobre los espinos, y haz un camino para tu hijo, y para tu casa y tus hijos, y tensa tu barco antes de que se adentre en el mar y sus olas y se hunda y no pueda. salvado.'

\par 17 '¡Oh hijo mío! si el rico come una serpiente, dicen: «Es por su sabiduría», y si un pobre la come, la gente dice: «Por su hambre».

\par 18 '¡Oh hijo mío! se contenta con tu pan de cada día y con tus bienes, y no codicia lo ajeno.'

\par 19 '¡Oh hijo mío! No seas prójimo del necio, ni comas pan con él, ni te alegres de las calamidades de tu prójimo. 1 Si tu enemigo te hace daño, muéstrale bondad.'

\par 20 '¡Oh hijo mío! Al hombre que teme a Dios, le temes y le honras.'

\par 21 ¡Oh hijo mío! el ignorante cae y tropieza, y el sabio, aunque tropiece, no se estremece, y aunque caiga se levanta rápidamente, y si está enfermo, puede cuidar de su vida. Pero en cuanto al hombre ignorante y estúpido, no existe ningún medicamento para su enfermedad.

\par 22 '¡Oh hijo mío! Si se acerca a ti un hombre que es inferior a ti, ve hacia él y quédate de pie, y si no puede recompensarte, su Señor te lo recompensará por él.'

\par 23 '¡Oh hijo mío! No dudes en golpear a tu hijo, porque la paliza de tu hijo es como estiércol para el huerto, y como atar la boca de una bolsa, y como atar a las bestias, y como el cerrojo de una puerta.'

\par 24 '¡Oh hijo mío! refrena a tu hijo de la maldad y enséñale modales antes de que se rebele contra ti y te desprecie entre el pueblo y agaches tu cabeza en las calles y en las asambleas y seas castigado por la maldad de sus malas acciones.'

\par 25 '¡Oh hijo mío! Consíguete un buey gordo con prepucio, y un asno grande de pezuñas, y no te hagas amigo de un hombre astuto, ni te hagas amigo de un esclavo pendenciero, ni de una sierva ladrona, para todo lo que cometes. a ellos les arruinarán.'

\par 26 '¡Oh hijo mío! No dejes que tus padres te maldigan, y el Señor se complazca con ellos; porque se ha dicho: «El que menosprecia a su padre o a su madre, que muera de muerte (me refiero a la muerte del pecado); y el que honra a sus padres prolongará sus días y su vida y verá todo lo bueno.»'

\par 27 '¡Oh hijo mío! No camines por el camino sin armas, porque no sabes cuándo te encontrará el enemigo, para que puedas estar preparado para él.

\par 28 '¡Oh hijo mío! No seas como un árbol desnudo y sin hojas que no crece, sino sé como un árbol cubierto de sus hojas y de sus ramas; porque el hombre que no tiene esposa ni hijos es deshonrado en el mundo y es aborrecido por ellos, como un árbol sin hojas y sin fruto.'

\par 29 '¡Oh hijo mío! sé como un árbol fructífero al borde del camino, cuyo fruto comen todos los que pasan, y las bestias del desierto descansan bajo su sombra y comen de sus hojas.'

\par 30 '¡Oh hijo mío! toda oveja que se desvía de su camino y sus compañeras se convierten en alimento para el lobo.'

\par 31 '¡Oh hijo mío! No digas: «Mi señor es un tonto y yo soy sabio», y no cuentes palabras de ignorancia y locura, para que no te desprecie.

\par 32 '¡Oh hijo mío! No seas uno de esos siervos a quienes sus señores dicen: «Aléjate de nosotros», sino sé uno de aquellos a quienes dicen: «Acércate y acércate a nosotros».

\par 33 '¡Oh hijo mío! No acaricies a tu esclavo en presencia de su compañero, porque no sabes cuál de ellos será de mayor valor para ti al final.

\par 34 '¡Oh hijo mío! No temas a tu Señor, que te creó, para que no te guarde silencio.'

\par 35 '¡Oh hijo mío! haz que tu habla sea justa y endulza tu lengua; y no permitas que tu compañero te pise el pie, no sea que en otra ocasión te pise el pecho.'

\par 36 '¡Oh hijo mío! si golpeas a un sabio con una palabra de sabiduría, ésta se esconderá en su pecho como un sutil sentimiento de vergüenza; pero si golpeas al ignorante con un palo, no entenderá ni oirá.'

\par 37 '¡Oh hijo mío! si envías a un hombre sabio para tus necesidades, no le des muchas órdenes, porque él hará tus negocios como deseas; y si envías a un necio, no le des órdenes, sino ve tú mismo y haz tus negocios, porque si Ordenalo, no hará lo que deseas. Si te envían por negocios, apresúrate a cumplirlo rápidamente.'

\par 38 '¡Oh hijo mío! No hagas enemigo de un hombre más fuerte que tú, porque él tomará tu medida y se vengará de ti.'

\par 39 '¡Oh hijo mío! prueba a tu hijo y a tu siervo antes de entregarles tus bienes, para que no los acaben con ellos; porque el que tiene la mano llena se llama sabio, aunque sea estúpido e ignorante, y el que tiene la mano vacía se llama pobre, ignorante, aunque sea príncipe de los sabios.'

\par 40 '¡Oh hijo mío! He comido un colocinto y tragado áloes, y no he encontrado nada más amargo que la pobreza y la escasez.

\par 41 '¡Oh hijo mío! enseña a tu hijo la frugalidad y el hambre, para que le vaya bien en el gobierno de su casa.'

\par 42 '¡Oh hijo mío! No enseñéis al ignorante la lengua de los sabios, porque le resultará gravosa.'

\par 43 '¡Oh hijo mío! No muestres tu condición a tu amigo, para que no seas despreciado por él.'

\par 44 '¡Oh hijo mío! La ceguera del corazón es más grave que la ceguera de los ojos, porque la ceguera de los ojos puede ser guiada poco a poco, pero la ceguera del corazón no se guía, y se sale del camino recto, y va por el torcido. forma.'

\par 45 '¡Oh hijo mío! Mejor es el tropiezo de un hombre con su pie que el tropiezo de su lengua.'

\par 46 '¡Oh hijo mío! Mejor es un amigo cercano que un hermano más excelente que está lejos.'

\par 47 '¡Oh hijo mío! la belleza se desvanece pero el aprendizaje perdura, y el mundo decae y se vuelve vano, pero un buen nombre no se vuelve vano ni decae.'

\par 48 '¡Oh hijo mío! el hombre que no tiene descanso, mejor sería su muerte que su vida; y mejor es el sonido del llanto que el sonido del canto; porque la tristeza y el llanto, si el temor de Dios está en ellos, son mejores que el sonido del canto y el regocijo.'

\par 49 '¡Oh hija mía! Mejor es un muslo de rana en tu mano que un ganso en la olla de tu prójimo; y mejor es una oveja cerca de ti que un buey lejos; y mejor es un gorrión en tu mano que mil gorriones volando; 1 y mejor es la pobreza que acumula, que la dispersión de mucha provisión; y mejor es un zorro vivo que un león muerto; y mejor es una libra de lana que una libra de riqueza, quiero decir de oro y de plata; porque el oro y la plata están escondidos y cubiertos en la tierra, y no se ven; pero la lana se queda en los mercados y se ve, y es una belleza para quien la usa.'

\par 50 '¡Oh hijo mío! una pequeña fortuna es mejor que una fortuna dispersa.

\par 51 '¡Oh hijo mío! Un perro vivo es mejor que un pobre muerto.

\par 52 '¡Oh hijo mío! Mejor es un pobre que hace el bien que un rico muerto en pecados.'

\par 53 '¡Oh hijo mío! Guarda una palabra en tu corazón, y será mucho para ti, y ten cuidado si revelas el secreto de tu amigo.

\par 54 '¡Oh hijo mío! No salga palabra de tu boca hasta que hayas consultado con tu corazón. Y no estés entre personas que riñen, porque de una mala palabra surge riña, y de una riña viene guerra, y de la guerra viene riña, y te verás obligado a dar testimonio; pero huye de allí y descansa.'

\par 55 '¡Oh hijo mío! No resistas a un hombre más fuerte que tú, sino consigue un espíritu paciente, paciencia y una conducta recta, porque no hay nada más excelente que eso.'

\par 56 '¡Oh hijo mío! No odies a tu primer amigo, porque el segundo puede no durar.

\par 57 '¡Oh hijo mío! Visita al pobre en su aflicción, habla de él en presencia del sultán y haz tu diligencia para salvarlo de la boca del león. 2

\par 58 '¡Oh hijo mío! No te regocijes por la muerte de tu enemigo, porque dentro de poco serás su prójimo, y al que se burla de ti respétalo y honralo y prepárate para saludarlo.'

\par 59 '¡Oh hijo mío! Si el agua se detuviera en el cielo, y un cuervo negro se volviera blanco, y la mirra se volviera dulce como la miel, entonces los hombres ignorantes y tontos podrían comprender y volverse sabios.'

\par 60 '¡Oh hijo mío! si quieres ser sabio, refrena tu lengua de mentir, y tu mano de hurtar, y tus ojos de ver el mal; entonces te llamarán sabio.'

\par 61 '¡Oh hijo mío! Que el sabio te golpee con vara, pero que el necio no te unja con ungüento dulce. Sé humilde en tu juventud y serás honrado en tu vejez.'

\par 62 '¡Oh hijo mío! No resistas a un hombre en los días de su poder, ni a un río en los días de su crecida.'

\par 63 '¡Oh hijo mío! No te apresures en la boda de una esposa, porque si sale bien, ella dirá: «Señor mío, haz provisiones para mí»; y si resulta malo, denunciará a quien fue la causa.'

\par 64 '¡Oh hijo mío! quien es elegante en su vestido, lo es en su discurso; y el que tiene apariencia humilde en su forma de vestir, también lo es en su forma de hablar.'

\par 65 '¡Oh hijo mío! Si has cometido un robo, hazlo saber al Sultán y dale una parte del mismo, para que puedas ser librado de él, porque de lo contrario soportarás amargura.'

\par 66 '¡Oh hijo mío! hazte amigo del hombre cuya mano está saciada y llena, y no te hagas amigo del hombre cuya mano está cerrada y hambrienta.'

\par 67 «Hay cuatro cosas en las que ni el rey ni su ejército pueden estar seguros: la opresión del visir, el mal gobierno, la perversión de la voluntad y la tiranía sobre los súbditos; y cuatro cosas que no se pueden ocultar: los prudentes, los necios, los ricos y los pobres.'

\par \textit{Notas al pie}

\par \textit{201:1 Cfr. Salmos CXLI. 4.}

\par \textit{203:1 Cfr. «Más vale pájaro en mano que ciento volando.»}

\par \textit{203:2 Cfr. 2 Timoteo, IV, 17.}

\chapter{3}

\par \textit{Ahikar se retira de la participación activa en los asuntos de estado. Entrega sus posesiones a su traidor sobrino. Aquí está la asombrosa historia de cómo un libertino ingrato se convierte en falsificador. Un complot inteligente para enredar a Ahikar resulta en su condena a muerte. Aparentemente el fin de Ahikar.}

\par 1 ASÍ habló Haiqâr, y cuando hubo terminado estos mandatos y proverbios a Nadan, el hijo de su hermana, imaginó que los cumpliría todos, y no sabía que en lugar de eso le estaba mostrando cansancio, desprecio y burla. .

\par 2 Después de eso, Haiqâr se quedó quieto en su casa y entregó a Nadan todos sus bienes, los esclavos, las sirvientas, los caballos, el ganado y todo lo demás que había poseído y ganado; y el poder de ordenar y prohibir permaneció en manos de Nadan.

\par 3 Y Haiqâr se sentó a descansar en su casa, y de vez en cuando Haiqâr iba y presentaba sus respetos al rey, y regresaba a casa.

\par 4 Ahora bien, cuando Nadan se dio cuenta de que el poder de ordenar y de prohibir estaba en su propia mano, despreció la posición de Haiqâr y se burló de él, y comenzó a culparlo cada vez que aparecía, diciendo: 'Mi tío Haiqâr está en su es una tontería y ahora no sabe nada.

\par 5 Y comenzó a golpear a los esclavos y a las sirvientas, y a vender los caballos y los camellos y a despilfarrar todo lo que su tío Haiqâr había poseído.

\par 6 Y cuando Haiqâr vio que no tenía compasión de sus sirvientes ni de su casa, se levantó y lo echó de su casa, y envió a informar al rey que había esparcido sus posesiones y sus provisiones.

\par 7 Y el rey se levantó y llamó a Nadan y le dijo: 'Mientras Haiqâr esté sano, nadie podrá gobernar sobre sus bienes, ni sobre su casa, ni sobre sus posesiones.'

\par 8 Y la mano de Nadan se separó de su tío Haiqâr y de todos sus bienes, y mientras tanto no entró ni salió, ni lo saludó.

\par 9 Entonces Haiqâr se arrepintió de su trabajo con Nadan, el hijo de su hermana, y continuó estando muy triste.

\par 10 Y Nadan tenía un hermano menor llamado Benuzârdân, por lo que Haiqâr lo tomó para sí en lugar de Nadan, y lo crió y honró con el mayor honor. Y le entregó todo lo que poseía, y le nombró gobernador de su casa.

\par 11 Ahora bien, cuando Nadan se dio cuenta de lo que había sucedido, se apoderó de la envidia y los celos, y comenzó a quejarse con todos los que lo interrogaban y a burlarse de su tío Haiqâr, diciendo: 'Mi tío me ha echado de su casa, y ha preferido a mi hermano antes que a mí, pero si el Dios Altísimo me da el poder, le traeré la desgracia de que me maten.'

\par 12 Y Nadan continuó meditando sobre el obstáculo que podría ponerle en peligro. Y después de un tiempo Nadan le dio vueltas a esto y escribió una carta a Aquis, hijo de Shah el Sabio, rey de Persia, diciendo lo siguiente:

\par 13 '¡Paz, salud, poder y honor de Senaquerib, rey de Asiria y de Nínive, y de su visir y su secretario Haiqâr, para ti, oh gran rey! Que haya peniques entre tú y yo.

\par 14 «Y cuando te llegue esta carta, si te levantas y vas rápidamente a la llanura de Nisrîn, a Asiria y a Nínive, te entregaré el reino sin guerra y sin orden de batalla».

\par 15 Y escribió también otra carta en nombre de Haiqâr al faraón rey de Egipto. '¡Que haya paz entre tú y yo, oh rey poderoso!'

\par 16 'Si en el momento en que te llegue esta carta te levantas y vas a Asiria y Nínive a la llanura de Nisrîn, te entregaré el reino sin guerra y sin peleas.'

\par 17 Y la escritura de Nadan era como la escritura de su tío Haiqâr.

\par 18 Luego dobló las dos cartas y las selló con el sello de su tío Haiqâr; sin embargo, estaban en el palacio del rey.

\par 19 Luego fue y escribió una carta del rey a su tío Haiqâr: «Paz y salud para mi Visir, mi Secretario, mi Canciller, Haiqâr».

\par 20 'Oh Haiqâr, cuando te llegue esta carta, reúne a todos los soldados que están contigo, y que sean perfectos en vestimenta y en número, y tráelos a mí el quinto día en la llanura de Nisrîn.'

\par 21 «Y cuando me veas allí viniendo hacia ti, apresúrate y haz que el ejército se mueva contra mí como un enemigo que quisiera pelear conmigo, porque tengo conmigo a los embajadores de Faraón rey de Egipto, para que puedan ver el fuerza de nuestro ejército y pueden temernos, porque son nuestros enemigos y nos odian.'

\par 22 Luego selló la carta y se la envió a Haiqâr por medio de uno de los sirvientes del rey. Y tomó la otra carta que había escrito y la extendió delante del rey y se la leyó y le mostró el sello.

\par 23 Y cuando el rey oyó lo que había en la carta, quedó perplejo con gran perplejidad y se enojó con una ira grande y feroz, y dijo: '¡Ah, he mostrado mi sabiduría! ¿Qué le he hecho a Haiqâr para que haya escrito estas cartas a mis enemigos? ¿Es ésta mi recompensa por los beneficios que le he brindado?

\par 24 Y Nadan le dijo: '¡No te entristezcas, oh rey! ni nos enojemos, sino que vayamos a la llanura de Nisrîn y veamos si la historia es cierta o no.'

\par 25 Entonces Nadan se levantó al quinto día y tomó al rey, a los soldados y al visir, y se dirigieron al desierto, a la llanura de Nisrîn. Y el rey miró, y ¡he aquí! Haiqâr y el ejército estaban dispuestos.

\par 26 Y cuando Haiqâr vio que el rey estaba allí, se acercó e hizo una señal al ejército para que se moviera como en guerra y lucharan en orden contra el rey como se había encontrado en la carta, sin saber qué pozo había hecho Nadan. cavado para él.

\par 27 Y cuando el rey vio el acto de Haiqâr, se sintió invadido por la ansiedad, el terror y la perplejidad, y se enojó con una gran ira.

\par 28 Y Nadan le dijo: '¿Has visto, oh mi señor el rey? ¿Qué ha hecho este desgraciado? pero no te enojes ni te entristezcas ni te aflijas, sino ve a tu casa y siéntate en tu trono, y te traeré a Haiqâr atado y encadenado con cadenas, y ahuyentaré a tu enemigo de ti sin esfuerzo.'

\par 29 Y el rey volvió a su trono, irritado por Haiqâr, y no hizo nada respecto a él. Y Nadan fue a Haiqâr y le dijo: '¡W'allah, oh tío mío! En verdad, el rey se regocija en ti con gran alegría y te agradece por haber hecho lo que te ordenó.'

\par 30 «Y ahora me ha enviado a ti para que despidas a los soldados a sus deberes y vengas tú mismo a él con las manos atadas a la espalda y los pies encadenados, para que los embajadores de Faraón puedan ver esto y que los que el rey sea temido por ellos y por su rey.'

\par 31 Entonces Haiqâr respondió y dijo: «Oír es obedecer». Y él se levantó luego, le ató las manos a la espalda y le encadenó los pies.

\par 32 Nadan lo tomó y fue con él al rey. Y cuando Haiqâr entró en presencia del rey, le rindió reverencia en el suelo y deseó poder y vida perpetua para el rey.

\par 33 Entonces dijo el rey: 'Oh Haiqâr, mi secretario, el gobernador de mis asuntos, mi canciller, el gobernante de mi estado, dime qué mal te he hecho para que me hayas recompensado con este feo acto.'

\par 34 Entonces le mostraron las cartas escritas y selladas. Y cuando Haiqâr vio esto, sus miembros temblaron y su lengua se trabó de inmediato, y no pudo pronunciar una palabra por miedo; pero él agachó la cabeza hacia la tierra y quedó mudo.

\par 35 Y cuando el rey vio esto, estuvo seguro de que la cosa provenía de él, y de inmediato se levantó y les ordenó matar a Haiqâr y herirle el cuello con la espada fuera de la ciudad.

\par 36 Entonces Nadan gritó y dijo: '¡Oh Haiqâr, oh cara negra! ¿De qué te sirve tu meditación o tu poder para realizar este acto ante el rey?

\par 37 Así dice el narrador. Y el nombre del espadachín era Abu Samîk. Y el rey le dijo: '¡Oh espadachín! levántate, ve, corta el cuello de Haiqâr a la puerta de su casa y separa su cabeza de su cuerpo cien codos.'

\par 38 Entonces Haiqâr se arrodilló ante el rey y dijo: '¡Viva mi señor el rey para siempre! y si deseas matarme, que se cumpla tu deseo; y sé que no soy culpable, pero el impío debe dar cuenta de su maldad; sin embargo, ¡oh mi señor el rey! Te ruego a ti y a tu amistad que permitas que el espadachín entregue mi cuerpo a mis esclavos para que me entierren y que tu esclavo sea tu sacrificio.

\par 39 El rey se levantó y ordenó al espadachín que hiciera con él lo que él deseaba.

\par 40 E inmediatamente ordenó a sus sirvientes que tomaran a Haiqâr y al espadachín y fueran con él desnudos para matarlo.

\par 41 Y cuando Haiqâr supo con certeza que lo iban a matar, envió a su esposa y le dijo: 'Sal a recibirme, y que estén contigo mil jóvenes vírgenes, y vístelas con vestidos de púrpura y seda para que lloren por mí antes de mi muerte.

\par 42 'Y prepara una mesa para el espadachín y sus sirvientes. Y mezclad mucho vino para que beban.

\par 43 Y ella hizo todo lo que él le ordenó. Y ella era muy sabia, inteligente y prudente. Y ella unió toda la cortesía y el aprendizaje posibles.

\par 44 Y cuando llegaron el ejército del rey y el espadachín, encontraron la mesa puesta en orden, el vino y las viandas suntuosas, y comenzaron a comer y beber hasta atiborrarse y emborracharse.

\par 45 Entonces Haiqâr llevó al espadachín aparte del grupo y le dijo: 'Oh Abu Samîk, ¿no sabes que cuando Sarhadum el rey, el padre de Senaquerib, quiso matarte, te tomé y te escondí en un lugar determinado? lugar hasta que la ira del rey se calmó y preguntó por ti?

\par 46 «Y cuando te llevé ante él, se regocijó en ti; y ahora recuerda el bien que te hice».

\par 47 «Y sé que el rey se arrepentirá de mí y se enojará con gran ira por mi ejecución».

\par 48 «Porque no soy culpable, y sucederá que cuando me presentes ante él en su palacio, tendrás gran suerte y sabrás que Nadan, el hijo de mi hermana, me ha engañado y ha hecho esta mala acción para yo, y el rey se arrepentirá de haberme matado; y ahora tengo un sótano en el jardín de mi casa, y nadie lo sabe.'

\par 49 'Escóndeme en él con el conocimiento de mi esposa. Y tengo un esclavo en prisión que merece ser asesinado.'

\par 50 'Sáquenlo, vístanlo con mis ropas y ordenen a los sirvientes que, cuando estén borrachos, lo maten. No sabrán a quién están matando.

\par 51 'Y aparta su cabeza cien codos de su cuerpo y entrega su cuerpo a mis siervos para que lo entierren. Y habrás guardado conmigo un gran tesoro.'

\par 52 'Y entonces el espadachín hizo lo que Haiqâr le había ordenado, y fue al rey y le dijo: «¡Que tu cabeza viva para siempre!»'

\par 53 'Entonces la esposa de Haiqâr le dejaba en el escondite cada semana lo que era suficiente para él, y nadie lo sabía excepto ella misma.'

\par 54 'Y se contó, repitió y difundió en todos los lugares la historia de cómo Haiqâr el Sabio había sido asesinado y estaba muerto, y toda la gente de esa ciudad lloraba por él.'

\par 55 'Y lloraron y dijeron: «¡Ay de ti, oh Haiqâr! ¡Y por tu aprendizaje y tu cortesía! ¡Qué triste por ti y por tu conocimiento! ¿Dónde se puede encontrar otro como tú? ¿Y dónde puede haber un hombre tan inteligente, tan erudito, tan hábil en gobernar como para parecerse a ti y poder ocupar tu lugar?

\par 56 'Pero el rey se estaba arrepintiendo de Haiqâr, y su arrepentimiento no le sirvió de nada.'

\par 57 'Entonces llamó a Nadan y le dijo: «Ve y toma a tus amigos contigo y haz duelo y llanto por tu tío Haiqâr, y lamenta por él como es costumbre, honrando su memoria». '

\par 58 «Pero cuando Nadan, el insensato, el ignorante y el duro de corazón, fue a la casa de su tío, no lloró ni se entristeció ni se lamentó, sino que reunió a gente cruel y disoluta y se puso a comer y beber». 1

\par 59 'Y Nadan comenzó a apoderarse de las sirvientas y los esclavos pertenecientes a Haiqâr, y los ató, los torturó y los azotó con una fuerte paliza.'

\par 60 «Y no respetó a la esposa de su tío, quien lo había criado como a su propio hijo, sino que quería que ella cayera en pecado con él».

\par 61 'Pero Haiqâr había sido encerrado en el escondite, y escuchó el llanto de sus esclavos y de sus vecinos, y alabó al Dios Altísimo, el Misericordioso, y dio gracias, y siempre oró y suplicó al Dios Altísimo.'

\par 62 'Y el espadachín venía de vez en cuando a Haiqâr mientras él estaba en medio del escondite: y Haiqâr venía y le suplicaba. Y lo consoló y le deseó liberación.'

\par 63 'Y cuando se difundió en otros países la historia de que Haiqâr el Sabio había sido asesinado, todos los reyes se entristecieron y despreciaron al rey Senaquerib, y se lamentaron por Haiqâr, el solucionador de enigmas.'

\par \textit{Notas al pie}

\par \textit{207:1 Compare este relato de la juerga de Nadan y su paliza a los sirvientes con Mateo XXIV. 48-51 y Lucas XII. 43-46. Verás que el lenguaje de Ahikar ha coloreado una de las parábolas de nuestro Señor.}

\chapter{4}

\par \textit{«Los enigmas de la Esfinge». Lo que realmente le pasó a Ahikar. Su regreso.}

\par 1 Y cuando el rey de Egipto se hubo asegurado de que Haiqâr fuera asesinado, se levantó inmediatamente y escribió una carta al rey Senaquerib, recordándole en ella 'la paz, la salud, el poder y el honor que deseamos especialmente para a ti, mi amado hermano, rey Senaquerib.'

\par 2 «He estado deseando construir un castillo entre el cielo y la tierra, y quiero que me envíes un hombre sabio e inteligente de tu parte para que me lo construya, y que me responda a todas mis preguntas, y que yo podrá tener los impuestos y derechos de aduana de Asiria durante tres años».

\par 3 Luego selló la carta y la envió a Senaquerib.

\par 4 Él lo tomó, lo leyó y se lo dio a sus visires y a los nobles de su reino, y ellos estaban perplejos y avergonzados, y él se enojó con una gran ira y se preguntaba cómo debía actuar.

\par 5 Entonces reunió a los ancianos, a los eruditos, a los sabios, a los filósofos, a los adivinos, a los astrólogos y a todos los que estaban en su país, les leyó la carta y les dijo: «¿Quién de entre ¿Irás a Faraón rey de Egipto y le responderás sus preguntas?

\par 6 Y ellos le dijeron: '¡Oh rey nuestro señor! Debes saber que no hay nadie en tu reino que esté familiarizado con estas preguntas excepto Haiqâr, tu visir y secretario.'

\par 7 «Pero nosotros no tenemos habilidad para esto, a menos que sea Nadan, el hijo de su hermana, porque él le enseñó toda su sabiduría, ciencia y conocimiento. Llámalo, tal vez pueda desatar este duro nudo.

\par 8 Entonces el rey llamó a Nadan y le dijo: «Mira esta carta y comprende lo que contiene». Y cuando Nadan lo leyó, dijo: '¡Oh mi señor! ¿Quién podrá construir un castillo entre el cielo y la tierra?

\par 9 Y cuando el rey escuchó las palabras de Nadan, se entristeció con un dolor grande y doloroso, descendió de su trono y se sentó sobre las cenizas, y comenzó a llorar y lamentarse por Haiqâr.

\par 10 Diciendo: '¡'Oh pena mía! ¡Oh Haiqâr, que conocías los secretos y los enigmas! ¡Ay de mí por ti, oh Haiqâr! Oh maestro de mi país y gobernante de mi reino, ¿dónde encontraré a alguien como tú? Oh Haiqâr, oh maestro de mi país, ¿adónde acudiré por ti? ¡Ay de mí por ti! ¡Cómo te destruí! y escuché la charla de un niño estúpido e ignorante, sin conocimientos, sin religión, sin virilidad.'

\par 11 '¡Ah! y otra vez ¡Ah por mí! ¿Quién puede entregárteme aunque sea por una vez o traerme noticias de que Haiqâr está vivo? y yo le daría la mitad de mi reino.'

\par 12 '¿De dónde viene esto para mí? ¡Ah, Haiqâr! para poder verte sólo por una vez, para poder hartarme de mirarte y de deleitarme en ti.'

\par 13 '¡Ah! ¡Oh mi dolor por ti para siempre! ¡Oh Haiqâr, cómo te he matado! y no me detuve en tu caso hasta que vi el final del asunto.

\par 14 Y el rey lloraba noche y día. Ahora bien, cuando el espadachín vio la ira del rey y su dolor por Haiqâr, su corazón se ablandó hacia él, se acercó a su presencia y le dijo:

\par 15 '¡Oh mi señor! Ordena a tus siervos que me corten la cabeza. Entonces el rey le dijo: '¡Ay de ti, Abu Samîk! ¿Cuál es tu culpa?'

\par 16 Y el espadachín le dijo: '¡Oh, maestro mío! todo esclavo que actúa en contra de la palabra de su amo es asesinado, y yo he actuado en contra de tu orden.'

\par 17 Entonces el rey le dijo. '¡Ay de ti, oh Abu Samîk, en qué has actuado en contra de mis órdenes?'

\par 18 Y el espadachín le dijo: '¡Oh, señor mío! Me ordenaste que matara a Haiqâr, y supe que te arrepentirías de él, y que había sido agraviado, y lo escondí en cierto lugar, y maté a uno de sus esclavos, y ahora está a salvo en el cisterna, y si me lo ordenas, te lo traeré.

\par 19 Y el rey le dijo. '¡Ay de ti, oh Abu Samîk! Te has burlado de mí y yo soy tu señor.

\par 20 Y el espadachín le dijo: '¡No, pero por la vida de tu cabeza, oh mi señor! Haiqâr está sano y salvo.'

\par 21 Y cuando el rey escuchó esas palabras, se sintió seguro del asunto, y su cabeza daba vueltas y se desmayó de alegría, y les ordenó que trajeran a Haiqâr.

\par 22 Y dijo al espadachín: '¡Oh siervo fiel! Si lo que dices es cierto, quisiera enriquecerte y exaltar tu dignidad por encima de la de todos tus amigos.

\par 23 Y el espadachín siguió alegremente hasta llegar a la casa de Haiqâr. Y abrió la puerta del escondite, bajó y encontró a Haiqâr sentado, alabando a Dios y dándole gracias.

\par 24 Y él le gritó, diciendo: '¡Oh Haiqâr, te traigo la mayor alegría, felicidad y deleite!'

\par 25 Y Haiqâr le dijo: '¿Cuáles son las noticias, oh Abu Samîk?' Y le contó todo acerca de Faraón, desde el principio hasta el fin. Luego lo tomó y fue al rey.

\par 26 Y cuando el rey lo miró, vio que estaba en un estado de miseria, que tenía el pelo largo como el de las fieras, y las uñas como garras de águila, y que su cuerpo estaba sucio de polvo. , y el color de su rostro había cambiado y se había desvanecido y ahora era como cenizas.

\par 27 Y cuando el rey lo vio, se entristeció por él y se levantó inmediatamente, lo abrazó, lo besó, lloró sobre él y dijo: «¡Alabado sea Dios!» ¿Quién te ha hecho volver a mí?

\par 28 Entonces lo consoló y lo consoló. Y se quitó la túnica y se la puso al espadachín, y fue muy misericordioso con él, le dio grandes riquezas y hizo descansar a Haiqâr.

\par 29 Entonces Haiqâr dijo al rey: '¡Viva mi señor el rey para siempre! Éstas sean las obras de los hijos del mundo. Levanté una palmera para apoyarme en ella, pero se dobló y me derribó.'

\par 30 ¡Pero, Señor mío! Desde que me he aparecido ante ti, ¡no dejes que el cuidado te oprima! Y el rey le dijo: 'Bendito sea Dios, que te mostró misericordia y supo que habías sido agraviado, y te salvó y te libró de la muerte'.

\par 31 'Pero ve a un baño caliente, aféitate la cabeza, córtate las uñas, cámbiate de ropa y diviértete durante cuarenta días, para que puedas hacerte bien y mejorar tu condición y tu color. de tu rostro puede volver a ti.'

\par 32 Entonces el rey se despojó de su costoso manto y se lo puso a Haiqâr, y Haiqâr dio gracias a Dios, rindió reverencia al rey y se fue a su morada contento y feliz, alabando al Dios Altísimo.

\par 33 Y la gente de su casa se regocijó con él, y también se regocijaron sus amigos y todos los que oyeron que estaba vivo.

\chapter{5}

\par \textit{La letra de los «acertijos» se muestra a Ahikar. Los chicos de las águilas. El primer viaje en «avión». Hacia Egipto. Ahikar, al ser un hombre de sabiduría, también tiene sentido del humor. (Verso 27).}

\par 1 E hizo lo que el rey le había ordenado y descansó cuarenta días.

\par 2 Luego se vistió con su ropa más alegre y cabalgó hacia el rey, con sus esclavos detrás y delante de él, regocijándose y regocijándose.

\par 3 Pero cuando Nadan, el hijo de su hermana, se dio cuenta de lo que estaba sucediendo, el miedo y el terror se apoderaron de él, y quedó perplejo, sin saber qué hacer.

\par 4 Y cuando Haiqâr lo vio, entró en presencia del rey y lo saludó, y él le devolvió el saludo y lo hizo sentarse a su lado, diciéndole: '¡Oh, mi querido Haiqâr! Mira estas cartas que nos envió el rey de Egipto, después de enterarse de que habías sido asesinado.

\par 5 «Nos han provocado y vencido, y muchos habitantes de nuestro país han huido a Egipto por miedo a los impuestos que el rey de Egipto nos ha enviado para exigirnos».

\par 6 Entonces Haiqâr tomó la carta, la leyó y comprendió su contenido.

\par 7 Entonces dijo al rey. '¡No te enojes, oh mi señor! Iré a Egipto, y devolveré las respuestas a Faraón, y le mostraré esta carta, y le responderé sobre los impuestos, y enviaré de vuelta a todos los que han huido; y avergonzaré a tus enemigos con la ayuda del Dios Altísimo, y para la felicidad de tu reino.'

\par 8 Y cuando el rey escuchó estas palabras de Haiqâr, se regocijó con gran alegría, y su corazón se ensanchó y le mostró favor.

\par 9 Y Haiqâr dijo al rey: 'Concédeme un plazo de cuarenta días para que pueda considerar esta cuestión y solucionarla.' Y el rey lo permitió.

\par 10 Y Haiqâr fue a su morada, y ordenó a los cazadores que capturaran dos aguiluchos jóvenes para él, y ellos los capturaron y se los trajeron; y ordenó a los tejedores de cuerdas que le tejieran dos cables de algodón, cada uno de ellos dos mil codos de largo, e hizo traer carpinteros y les mandó que hicieran dos cajas grandes, y así lo hicieron.

\par 11 Entonces tomó dos muchachos y pasó cada día sacrificando corderos y alimentando a las águilas y a los niños, y haciendo montar a los niños sobre los lomos de las águilas, y los ató con un nudo fuerte y ató el cable a las patas de las águilas, y las dejó elevarse poco a poco cada día, hasta una distancia de diez codos, hasta que se acostumbraran y se educaran a ello; y subieron a lo largo de toda la cuerda hasta llegar al cielo; los chicos estando boca arriba. Luego los atrajo hacia sí.

\par 12 Y cuando Haiqâr vio que su deseo se había cumplido, encargó a los niños que cuando los llevaran al cielo debían gritar, diciendo:

\par 13 «Tráenos barro y piedra para que podamos construir un castillo para el rey Faraón, porque somos ociosos».

\par 14 Y Haiqâr nunca terminó de entrenarlos y ejercitarlos hasta que alcanzaron el punto (de habilidad) máximo posible.

\par 15 Entonces, dejándolos, se dirigió al rey y le dijo: «¡Señor mío! la obra está terminada según tu deseo. Levántate conmigo para que pueda mostrarte la maravilla.'

\par 16 Entonces el rey saltó y se sentó con Haiqâr y fue a un lugar amplio y envió a traer las águilas y los niños, y Haiqâr los ató y los dejó volar en el aire a lo largo de las cuerdas, y comenzaron a griten como él les había enseñado. Luego los atrajo hacia sí y los puso en sus lugares.

\par 17 Y el rey y los que estaban con él quedaron maravillados con gran asombro; y el rey besó a Haiqâr entre los ojos y le dijo: '¡Ve en paz, oh amado mío! ¡Oh orgullo de mi reino! a Egipto y responder las preguntas de Faraón y vencerlo con la fuerza del Dios Altísimo.'

\par 18 Entonces se despidió de él, tomó sus tropas y su ejército, los jóvenes y las águilas, y se dirigió hacia las moradas de Egipto; y cuando llegó, se volvió hacia el país del rey.

\par 19 Y cuando el pueblo de Egipto supo que Senaquerib había enviado a un hombre de su consejo privado para hablar con Faraón y responder a sus preguntas, llevaron la noticia al rey Faraón, y él envió un grupo de sus consejeros privados para traerlo. Antes que él.

\par 20 Y vino y entró en presencia de Faraón y le rindió homenaje como corresponde a los reyes.

\par 21 Y él le dijo: '¡Oh mi señor el rey! El rey Senaquerib te saluda con abundancia de paz, poder y honor.

\par 22 «Y él me ha enviado a mí, que soy uno de sus esclavos, para que pueda responderte a tus preguntas y cumplir todos tus deseos; porque has enviado a buscar de mi señor el rey un hombre que te edifique una castillo entre el cielo y la tierra.'

\par 23 «Y yo, con la ayuda del Dios Altísimo, tu noble favor y el poder de mi señor el rey, te la construiré como deseas».

\par 24 'Pero, ¡oh mi señor el rey! lo que has dicho en él sobre los impuestos de Egipto durante tres años: ahora la estabilidad de un reino es estricta justicia, y si ganas y mi mano no tiene habilidad para responderte, entonces mi señor el rey te enviará los impuestos. que has mencionado.'

\par 25 «Y si te respondo a tus preguntas, te quedará enviar todo lo que has mencionado a mi señor el rey».

\par 26 Y cuando Faraón escuchó estas palabras, se maravilló y quedó perplejo por la libertad de su lengua y lo agradable de su habla.

\par 27 Y el rey Faraón le dijo: «¡Hombre! ¿Cuál es tu nombre? Y él dijo: 'Tu siervo es Abiqâm, y yo una hormiguita de las hormigas del rey Senaquerib.'

\par 28 Y Faraón le dijo: «¿No tenía tu señor nadie de mayor dignidad que tú, para que me haya enviado una hormiguita para responderme y conversar conmigo?»

\par 29 Y Haiqâr le dijo: '¡Oh mi señor el rey! Quisiera al Dios Altísimo que pueda cumplir lo que tienes en mente, porque Dios está con los débiles para confundir a los fuertes.'

\par 30 Entonces Faraón ordenó que prepararan una morada para Abiqâm y le proporcionaran forraje, carne y bebida, y todo lo que necesitara.

\par 31 Y cuando todo estuvo terminado, tres días después Faraón se vistió de púrpura y rojo y se sentó en su trono, y todos sus visires y los magnates de su reino estaban de pie con las manos cruzadas, los pies juntos y las cabezas. encorvado.

\par 32 Entonces Faraón envió a buscar a Abiqâm, y cuando se lo presentaron, se postró ante él y besó el suelo delante de él.

\par 33 Y el rey Faraón le dijo: 'Oh Abiqâm, ¿a quién me parezco? y los nobles de mi reino, ¿a quiénes se parecen?

\par 34 Y Haiqâr le dijo: 'Oh mi señor, pariente, tú eres como el ídolo Bel, y los nobles de tu reino son como sus sirvientes.'

\par 35 Él le dijo: «Ve y vuelve mañana acá». Entonces Haiqâr fue como el rey Faraón le había ordenado.

\par 36 Y al día siguiente, Haiqâr fue a la presencia de Faraón, hizo reverencias y se presentó ante el rey. Y el faraón estaba vestido de rojo, y los nobles estaban vestidos de blanco.

\par 37 Y Faraón le dijo: 'Oh Abiqâm, ¿a quién me parezco? y los nobles de mi reino, ¿a quiénes se parecen?

\par 38 Y Abiqâm le dijo: '¡Oh mi señor! tú eres como el sol, y tus siervos como sus rayos.' Y Faraón le dijo: 'Ve a tu morada, y ven acá mañana'.

\par 39 Entonces Faraón ordenó a su corte que se vistiera de blanco puro, y Faraón se vistió como ellos y se sentó en su trono, y les ordenó que trajeran a Haiqâr. Y él entró y se sentó delante de él.

\par 40 Y Faraón le dijo: 'Oh Abiqâm, ¿a quién me parezco? Y mis nobles, ¿a quién se parecen?

\par 41 Y Abiqâm le dijo: '¡Oh mi señor! eres como la luna, y tus nobles son como los planetas y las estrellas.' Y Faraón le dijo: «Ve y mañana estarás aquí».

\par 42 Entonces Faraón ordenó a sus sirvientes que se vistieran túnicas de varios colores, y Faraón se puso un vestido de terciopelo rojo, se sentó en su trono y les ordenó que trajeran a Abiqâm. Y él entró y se postró ante él.

\par 43 Y él dijo: 'Oh Abiqâm, ¿a quién me parezco? y mis ejércitos, ¿a quién se parecen? Y él dijo: '¡Oh mi señor! eres como el mes de abril, y tus ejércitos como sus flores.'

\par 44 Y cuando el rey lo oyó, se regocijó con gran alegría y dijo: «¡Oh Abiqâm! la primera vez me comparaste con el ídolo Bel, y a mis nobles con sus sirvientes.'

\par 45 «Y la segunda vez me comparaste con el sol, y a mis nobles con los rayos del sol».

\par 46 «Y la tercera vez me comparaste con la luna, y a mis nobles con los planetas y las estrellas».

\par 47 'Y la cuarta vez me comparaste con el mes de abril, y a mis nobles con sus flores. ¡Pero ahora, oh Abiqâm! Dime, tu señor el rey Senaquerib, ¿a quién se parece? y sus nobles, ¿a quiénes se parecen?

\par 48 Y Haiqâr gritó en voz alta y dijo: '¡Lejos de mí hacer mención de mi señor el rey y de ti sentado en tu trono! Pero levántate para que yo te diga quién es mi señor el rey y quiénes son sus nobles.

\par 49 Y Faraón quedó perplejo por la libertad de su lengua y su audacia al responder. Entonces Faraón se levantó de su trono, se paró delante de Haiqâr y le dijo: 'Dime ahora, para que pueda ver a quién se parece tu señor el rey, y sus nobles, a quién se parecen.'

\par 50 Y Haiqâr le dijo: 'Mi señor es el Dios del cielo, y sus nobles son los relámpagos y los truenos, y cuando él quiere, soplan los vientos y cae la lluvia.'

\par 51 «Y él ordena el trueno, y los relámpagos y la lluvia, y retiene el sol, y no da su luz, y la luna y las estrellas, y no giran».

\par 52 «Y él ordena la tempestad, y ella sopla y cae la lluvia y pisotea abril y destruye sus flores y sus casas».

\par 53 Y cuando Faraón escuchó estas palabras, quedó muy perplejo y se enojó con una gran ira, y le dijo: '¡Oh hombre! Dime la verdad y déjame saber quién eres realmente.

\par 54 Y él le dijo la verdad. 'Soy Haiqâr el escriba, el más grande de los Consejeros Privados del rey Senaquerib, y soy su visir y el Gobernador de su reino, y su Canciller.'

\par 55 Y él le dijo: «Has dicho la verdad en estas palabras». Pero hemos oído hablar de Haiqâr, que el rey Senaquerib lo ha matado, pero pareces estar vivo y bien.'

\par 56 Y Haiqâr le dijo: 'Sí, así fue, pero alabado sea Dios, que sabe lo que está oculto, porque mi señor el rey ordenó que me mataran, y él creyó la palabra de hombres libertinos, pero el Señor me libró, y bienaventurado el que en Él confía.'

\par 57 Y Faraón dijo a Haiqâr: 'Ve, y mañana estarás aquí y dime una palabra que nunca haya oído de mis nobles ni de la gente de mi reino y de mi país.'

\chapter{6}

\par \textit{La artimaña tiene éxito. Ahikar responde a todas las preguntas del faraón. Los chicos de las águilas son el clímax del día. El ingenio, que rara vez se encuentra en las Escrituras antiguas, se revela en los versículos 34-45.}

\par 1 Y Haiqâr fue a su morada y escribió una carta, diciendo en ella lo siguiente:

\par 2 De Senaquerib rey de Asiria. y Nínive a Faraón rey de Egipto.

\par 3 '¡Paz a ti, hermano mío! y lo que te hacemos saber con esto es que un hermano tiene necesidad de su hermano, y reyes unos de otros, y mi esperanza de ti es que me prestes novecientos talentos de oro, porque lo necesito para el avituallamiento de algunos de los soldados, para que pueda gastarlo en ellos. Y dentro de poco te lo enviaré.

\par 4 Luego dobló la carta y al día siguiente se la presentó al Faraón.

\par 5 Y cuando lo vio, quedó perplejo y le dijo: «En verdad, nunca he oído de nadie nada parecido a esta lengua».

\par 6 Entonces Haiqâr le dijo: 'En verdad, ésta es una deuda que tienes con mi señor el rey.'

\par 7 Y Faraón aceptó esto, diciendo: 'Oh Haiqâr, eres como tú, que eres honesto en el servicio de los reyes'.

\par 8 «Bendito sea Dios, que te ha perfeccionado en sabiduría y te ha adornado con filosofía y conocimiento».

\par 9 'Y ahora, oh Haiqâr, queda lo que deseamos de ti: que construyas un castillo entre el cielo y la tierra.'

\par 10 Entonces dijo Haiqâr: 'Oír es obedecer. Te construiré un castillo según tu deseo y elección; pero, oh mi señor, yo preparo cal, piedra, barro y obreros, y tengo hábiles albañiles que te edificarán como tú deseas.'

\par 11 Y el rey preparó todo para él, y fueron a un lugar espacioso; y Haiqâr y sus muchachos llegaron allí, y tomó consigo las águilas y los jóvenes; y el rey y todos sus nobles fueron y toda la ciudad se reunió, para ver qué haría Haiqâr.

\par 12 Entonces Haiqâr sacó a las águilas de las cajas, ató a los jóvenes a sus espaldas, ató las cuerdas a las patas de las águilas y las dejó volar en el aire. Y se elevaron hacia arriba, hasta quedar entre el cielo y la tierra.

\par 13 Y los muchachos comenzaron a gritar, diciendo: '¡Traed ladrillos, traed arcilla para que podamos construir el castillo del rey, porque estamos sin hacer nada!'

\par 14 Y la multitud estaba asombrada y perpleja y maravillada. Y el rey y sus nobles estaban maravillados.

\par 15 Y Haiqâr y sus sirvientes comenzaron a golpear a los trabajadores, y llamaron a gritos a las tropas del rey, diciéndoles: 'Traed a los trabajadores expertos lo que quieran y no les impidáis realizar su trabajo.'

\par 16 Y el rey le dijo: 'Estás loco; ¿Quién puede acercar algo a esa distancia?

\par 17 Y Haiqâr le dijo: '¡Oh mi señor! ¿Cómo construiremos un castillo en el aire? y si mi señor el rey estuviera aquí, habría construido varios castillos en un solo día.

\par 18 Y Faraón le dijo: 'Ve, oh Haiqâr, a tu morada y descansa, porque hemos dejado de construir el castillo, y mañana ven a verme'.

\par 19 Entonces Haiqâr fue a su morada y al día siguiente se presentó ante el Faraón. Y Faraón dijo: 'Oh Haiqâr, ¿qué noticias hay del caballo de tu señor? porque cuando relincha en el país de Asiria y de Nínive, y nuestras yeguas oyen su voz, arrojan a sus crías.'

\par 20 Y cuando Haiqâr escuchó estas palabras, fue y tomó un gato, la ató y comenzó a azotarla con fuertes azotes hasta que los egipcios lo oyeron, y fueron y se lo contaron al rey.

\par 21 Y Faraón envió a buscar a Haiqâr y le dijo: 'Oh Haiqâr, ¿por qué azotas así y golpeas a esa bestia muda?'

\par 22 Y Haiqâr le dijo: ¡Mi señor el rey! En verdad ella me ha hecho un acto feo, y ha merecido esta paliza y azotes, porque mi señor el rey Senaquerib me había dado un gallo hermoso, y tenía una voz fuerte y verdadera y conocía las horas del día y de la noche.

\par 23 Y el gato se levantó esta misma noche, se cortó la cabeza y se fue, y por este hecho le he dado esta paliza.

\par 24 Y Faraón le dijo: 'Oh Haiqâr, por todo esto veo que estás envejeciendo y que estás en tu vejez, porque entre Egipto y Nínive hay sesenta y ocho parasangs, ¿y cómo fue ella esta misma noche y ¿Cortarle la cabeza a tu polla y volver?

\par 25 Y Haiqâr le dijo: '¡Oh mi señor! Si hubiera tal distancia entre Egipto y Nínive, ¿cómo podrían tus yeguas oír el relincho del caballo de mi señor el rey y arrojar a sus crías? ¿Y cómo podría llegar hasta Egipto la voz del caballo?

\par 26 Y cuando Faraón escuchó esto, supo que Haiqâr había respondido a sus preguntas.

\par 27 Y Faraón dijo: 'Oh Haiqâr, quiero que me hagas cuerdas de arena del mar'.

\par 28 Y Haiqâr le dijo: '¡Oh mi señor el rey! Ordenad que me traigan una cuerda del tesoro para hacer una igual.

\par 29 Entonces Haiqâr fue a la parte trasera de la casa y hizo agujeros en la dura orilla del mar, y tomó un puñado de arena en su mano, arena de mar, y cuando salió el sol, y penetró en los agujeros, extendió la arena al sol hasta que quedó como tejida como cuerdas.

\par 30 Y Haiqâr dijo: 'Ordena a tus siervos que tomen estas cuerdas, y cuando lo desees, te tejeré algunas como ellas'.

\par 31 Y Faraón dijo: 'Oh Haiqâr, tenemos aquí una piedra de molino y se ha roto y quiero que la cosas'.

\par 32 Entonces Haiqâr la miró y encontró otra piedra.

\par 33 Y dijo a Faraón: '¡Oh, señor mío! Soy extranjera y no tengo herramienta para coser.

\par 34 «Pero quiero que ordenes a tus fieles zapateros que corten punzones en esta piedra, para que yo pueda coser esa piedra de molino».

\par 35 Entonces el faraón y todos sus nobles se rieron. Y él dijo: 'Bendito sea el Dios Altísimo, que te dio este ingenio y este conocimiento.'

\par 36 Y cuando Faraón vio que Haiqâr lo había vencido y le devolvió sus respuestas, inmediatamente se enojó y les ordenó que le cobraran los impuestos de tres años y los trajeran a Haiqâr.

\par 37 Y se despojó de sus túnicas y se las vistió a Haiqâr, a sus soldados y a sus sirvientes, y le pagó los gastos de su viaje.

\par 38 Y él le dijo: '¡Vete en paz, fortaleza de su señor y orgullo de sus médicos! ¿Tiene alguno de los sultanes como usted? Saluda de mi parte a tu señor el rey Senaquerib y dile cómo le hemos enviado regalos, porque los reyes se contentan con poco.

\par 39 Entonces Haiqâr se levantó y besó las manos del rey Faraón y besó el suelo delante de él, y le deseó fuerza y ​​continuidad y abundancia en su tesoro, y le dijo: '¡Oh mi señor! Deseo de ti que ninguno de nuestros compatriotas permanezca en Egipto.

\par 40 Entonces Faraón se levantó y envió heraldos para proclamar en las calles de Egipto que ni uno solo del pueblo de Asiria o de Nínive debía permanecer en la tierra de Egipto, sino que debía ir con Haiqâr.

\par 41 Entonces Haiqâr fue, se despidió del rey Faraón y partió en busca de la tierra de Asiria y de Nínive; y tenía algunos tesoros y muchas riquezas.

\par 42 Y cuando al rey Senaquerib le llegó la noticia de que Haiqâr iba a venir, salió a recibirlo y se regocijó enormemente por él con gran alegría y lo abrazó, lo besó y le dijo: '¡Bienvenido a casa! ¡Oh pariente! mi hermano Haiqâr, la fuerza de mi reino y el orgullo de mi reino.'

\par 43 «Pide lo que quieras de mí, aunque desees la mitad de mi reino y de mis bienes».

\par 44 Entonces Haiqâr le dijo: '¡Oh mi señor el rey, vive para siempre! ¡Muestra tu favor, oh mi señor el rey! a Abu Samîk en mi lugar, porque mi vida estaba en manos de Dios y en las suyas.'

\par 45 Entonces el rey Senaquerib dijo: '¡Honra a ti, oh mi amado Haiqâr! Haré que la posición de Abu Samîk, el espadachín, sea más alta que la de todos mis Consejeros Privados y mis favoritos.'

\par 46 Entonces el rey comenzó a preguntarle cómo le había ido con Faraón desde que llegó hasta que se fue de su presencia, y cómo había respondido a todas sus preguntas, y cómo había recibido los impuestos de él, y los cambios de vestimenta y los regalos.

\par 47 Y el rey Senaquerib se regocijó con gran alegría y dijo a Haiqâr: «Toma lo que quieras de este tributo, porque está todo al alcance de tu mano».

\par 48 Y Haiqâr mid: '¡Que el rey viva para siempre! No deseo nada más que la seguridad de mi señor el rey y la continuación de su grandeza.

\par 49 '¡Oh mi señor! ¿Qué puedo hacer con la riqueza y cosas similares? pero si quieres mostrarme tu favor, dame a Nadan, el hijo de mi hermana, para que le pague por lo que me ha hecho, y concédeme su sangre y me tendrá por inocente.'

\par 50 Y el rey Senaquerib dijo: «Tómalo, te lo he dado». Y Haiqâr tomó a Nadan, el hijo de su hermana, y le ató las manos con cadenas de hierro, y lo llevó a su morada, y le puso un grillete pesado en los pies, y lo ató con un nudo apretado, y después de atarlo así lo arrojó. a una habitación oscura, junto al lugar de descanso, y nombró a Nabu-hal como centinela sobre él para darle cada día un pan y un poco de agua.

\chapter{7}

\par \textit{Las parábolas de Ahikar en las que completa la educación de su sobrino. Símiles sorprendentes. Ahikar llama al niño nombres pintorescos. Aquí termina la historia de Ahikar.}

\par 1 Y cada vez que Haiqâr entraba o salía, reprendía a Nadan, el hijo de su hermana, diciéndole sabiamente:

\par 2 '¡Oh Nadan, muchacho! Te he hecho todo lo bueno y amable y tú me has recompensado con lo feo y lo malo y con matar.'

\par 3 '¡Oh hijo mío! Dicen los proverbios: Al que no escucha con el oído, con la nuca le harán escuchar.'

\par 4 Y Nadan dijo: «¿Por qué estás enojado conmigo?»

\par 5 Y Haiqâr le dijo: 'Porque te crié, te enseñé, te di honor y respeto, te hice grande, te crié con la mejor crianza y te senté en mi lugar para que pudieras ser mi heredero en el mundo, y me trataste con la muerte y me pagaste con mi ruina.'

\par 6 Pero el Señor sabía que yo había sido agraviado y me salvó de la amenaza que me habías preparado, porque el Señor sana los corazones quebrantados y obstaculiza a los envidiosos y a los altivos.

\par 7 ¡Oh muchacho! Has sido para mí como el escorpión que, al golpear el bronce, lo traspasa.

\par 8 ¡Oh muchacho! Eres como la gacela que comía las raíces de la rubia, y me añade hoy y mañana se broncearán se esconden en mis raíces.

\par 9 ¡Oh muchacho! has sido tú quien vio a su camarada desnudo en la fría época del invierno; y tomó agua fría y se la derramó.

\par 10 ¡Oh muchacho! Has sido para mí como un hombre que tomó una piedra y la arrojó al cielo para apedrear con ella a su Señor. Y la piedra no golpeó, y no llegó lo suficientemente alto, sino que se convirtió en causa de culpa y pecado.

\par 11 ¡Oh muchacho! Si me hubieras honrado y respetado y hubieras escuchado mis palabras, habrías sido mi heredero y habrías reinado sobre mis dominios.

\par 12 ¡Oh hijo mío! Sabes que si la cola del perro o del cerdo tuviera diez codos de largo, no se acercaría al valor de la del caballo, aunque fuera como la seda.

\par 13 ¡Oh muchacho! Pensé que habrías sido mi heredero a mi muerte; y tú por tu envidia y tu insolencia quisiste matarme. Pero el Señor me libró de tu astucia.

\par 14 ¡Oh hijo mío! Has sido para mí como una trampa tendida en el muladar, y vino un gorrión y encontró la trampa tendida. Y el gorrión dijo a la trampa: «¿Qué haces aquí?» Dijo la trampa: «Estoy orando aquí a Dios».

\par 15 Y la alondra le preguntó también: «¿Cuál es el trozo de madera que tienes en la mano?» Dijo la trampa: «Ese es un roble joven en el que me apoyo en el momento de la oración».

\par 16 Dijo la alondra: «¿Y qué es eso que tienes en la boca?» Dijo la trampa: 'Esto es pan y víveres que llevo para todos los hambrientos y pobres que se acercan a mí.'

\par 17 Dijo la alondra: '¿Puedo ahora acercarme y comer, porque tengo hambre?' Y la trampa le dijo: «Adelante». Y la alondra se acercó para comer.

\par 18 Pero la trampa se levantó y agarró a la alondra por el cuello.

\par 19 Y la alondra respondió y dijo a la trampa: «Si ese es tu pan para el hambriento, Dios no acepta tus limosnas y tus buenas obras.

\par 20 Y si esto es tu ayuno y tus oraciones, Dios no acepta de ti ni tu ayuno ni tu oración, y Dios no perfeccionará lo que es bueno para ti.

\par 21 ¡Oh muchacho! has sido para mí (como) un león que se hizo amigo de un asno, y el asno estuvo caminando delante del león por un tiempo; y un día el león saltó sobre el asno y se lo comió.

\par 22 ¡Oh muchacho! Has sido para mí como gorgojo del trigo, que no hace ningún bien a nada, sino que estropea el trigo y lo roe.

\par 23 ¡Oh muchacho! has sido como un hombre que sembró diez medidas de trigo, y cuando llegó el tiempo de la cosecha, se levantó y lo segó, lo recogió, lo trilló y trabajó en ello al máximo, y resultó que eran diez medidas. medidas, y su amo le dijo: '¡Oh, holgazán! No has crecido ni te has encogido.'

\par 24 ¡Oh muchacho! Has sido para mí como la perdiz que habían sido echadas en la red, y no pudo salvarse, pero llamó a las perdices para echarlas consigo en la red.

\par 25 ¡Oh hijo mío! has sido para mí como el perro que tuvo frío y entró en casa del alfarero para calentarse.

\par 26 Y cuando se calentó, empezó a ladrarles, y ellos lo ahuyentaron y lo golpearon para que no los mordiera.

\par 27 ¡Oh hijo mío! Has sido para mí como el cerdo que se metió en el baño caliente con gente de calidad, y cuando salió del baño caliente, vio un agujero inmundo y se hundió y se revolcó en él.

\par 28 ¡Oh hijo mío! Has sido para mí como la cabra que se unió a sus compañeros en el camino al sacrificio y no pudo salvarse.

\par 29 ¡Oh muchacho! el perro que no se alimenta de su caza se convierte en alimento para las moscas.

\par 30 ¡Oh hijo mío! la mano que no trabaja ni ara y es codiciosa y astuta será cortada de su hombro.

\par 31 ¡Oh hijo mío! el ojo en el que no se ve la luz, los cuervos lo picarán y se lo arrancarán.

\par 32 ¡Oh muchacho! Tú has sido para mí como un árbol cuyas ramas estaban cortando, y les decía: 'Si algo de mí no estuviera en vuestras manos, en verdad no podríais cortarme'.

\par 33 ¡Oh muchacho! Eres como el gato al que le dijeron: «Deja de robar hasta que te hagamos una cadena de oro y te alimentemos con azúcar y almendras».

\par 34 Y ella dijo: «No me olvido del oficio de mi padre y de mi madre».

\par 35 ¡Oh hijo mío! has sido como la serpiente cabalgando sobre un espino cuando estaba en medio de un río, y un lobo los vio y dijo: 'Daño tras daño, y que el que sea más travieso que ellos dirija ambos.'

\par 36 Y la serpiente dijo al lobo: 'Los corderos, las cabras y las ovejas que has comido toda tu vida, ¿los devolverás a sus padres y a sus padres o no?'

\par 37 Dijo el lobo: 'No.' Y la serpiente le dijo: «Creo que después de mí eres el peor de nosotros».

\par 38 '¡Oh muchacho! Yo te alimenté con buena comida y tú no me alimentaste con pan seco.'

\par 39 '¡Oh muchacho! Te di agua azucarada. bebida y buen almíbar, y no me diste a beber agua del pozo.'

\par 40 '¡Oh muchacho! Yo te enseñé y te crié, y tú cavaste un escondite para mí y me ocultaste.'

\par 41 '¡Oh muchacho! Te crié con la mejor educación y te formé como un alto cedro; y me has torcido y doblado.'

\par 42 '¡Oh muchacho! Era mi esperanza acerca de ti que me construyeras un castillo fortificado, para que pudiera estar escondido de mis enemigos en él, y tú fuiste para mí como un sepultado en lo profundo de la tierra; pero el Señor se compadeció de mí y me libró de tu astucia.'

\par 43 '¡Oh muchacho! Te deseé bien, y me recompensaste con maldad y odio, y ahora quisiera arrancarte los ojos, y hacerte comida para los perros, y cortarte la lengua, y cortarte la cabeza a filo de espada, y te recompensaré por tus abominables actos.'

\par 44 Y cuando Nadan escuchó este discurso de su tío Haiqâr, dijo: '¡Oh, tío mío! trátame según tu conocimiento y perdóname mis pecados; porque ¿quién hay que haya pecado como yo, o quién haya que perdone como tú?'

\par 45 '¡Acéptame, oh tío mío! Ahora serviré en tu casa, cuidaré tus caballos, barreré el estiércol de tus ganados y apacentaré tus ovejas, porque yo soy el malvado y tú el justo: yo el culpable y tú el perdonador.' 1

\par 46 Y Haiqâr le dijo: '¡Oh, muchacho! Eres como el árbol que estaba sin fruto junto al agua, y su dueño quiso cortarlo, y le dijo: 'Llévame a otro lugar, y si no doy fruto, córtame.'

\par 47 Y su señor le dijo: «Estando junto al agua no has dado fruto, ¿cómo darás fruto si estás en otro lugar?»

\par 48 '¡Oh muchacho! Mejor es la vejez del águila que la juventud del cuervo.

\par 49 '¡Oh muchacho! Le dijeron al lobo: «Aléjate de las ovejas, no sea que su polvo te haga daño». Y el lobo dijo: 'Los restos de la leche de la oveja son buenos para mis ojos'.

\par 50' ¡Oh muchacho! Hicieron que el lobo fuera a la escuela para que aprendiera a leer y le dijeron: «Di A, B». Él dijo: «Cordero y cabrito en mi campana»'

\par 51 '¡Oh muchacho! Pusieron el asno en la mesa y cayó, y empezó a revolcarse en el polvo y uno decía: «Que se ruede, que es su naturaleza, no cambiará».

\par 52 '¡Oh muchacho! Se ha confirmado el dicho que dice: «Si engendras un niño, llámalo tu hijo, y si crías un niño, llámalo tu esclavo».

\par 53 '¡Oh muchacho! el que hace el bien encontrará el bien; y el que hace el mal, se encontrará con el mal, porque el Señor paga al hombre según la medida de su trabajo.'

\par 54 '¡Oh muchacho! ¿Qué más te diré que estos dichos? porque el Señor sabe lo que está oculto, y conoce los misterios y los secretos.'

\par 55 «Y Él te recompensará y juzgará entre ti y yo, y te recompensará según tu merecimiento.»,

\par 56 Y cuando Nadan escuchó ese discurso de su tío Haiqâr, se hinchó inmediatamente y se volvió como una vejiga reventada.

\par 57 Y sus miembros se hincharon, sus piernas, sus pies y su costado, y se desgarró, y su vientre estalló en dos, y sus entrañas se esparcieron, y pereció y murió.

\par 58 Su último fin fue la perdición y fue al infierno. Porque el que cava un hoyo para su hermano, caerá en él; y el que pone trampas quedará atrapado en ellas.

\par 59 Esto es lo que sucedió y (lo que) encontramos sobre la historia de Haiqâr, y alabado sea Dios por siempre. Amén y paz.

\par 60 ¡Esta crónica se termina con la ayuda de Dios, exaltado sea! Amén, Amén, Amén.

\par \textit{Notas al pie}

\par \textit{218:1 Compárese con la parábola del hijo pródigo en Lucas XV. 19.}


\end{document}