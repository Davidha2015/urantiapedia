\begin{document}

\title{Testamento de Rubén}

\chapter{1}

\par \textit{Rubén, el hijo primogénito de Jacob y Lea. El hombre de experiencia aconseja contra la fornicación y señala las formas en que los hombres son más propensos a caer en el error.}

\par 1 LA Copia del Testamento de Rubén, las órdenes que dio a sus hijos antes de morir en el año ciento veinticinco de su vida.

\par 2 Dos años después de la muerte de su hermano José, cuando Rubén enfermó, sus hijos y los hijos de sus hijos se reunieron para visitarlo.

\par 3 Y él les dijo: Hijitos míos, he aquí que me muero y sigo el camino de mis padres.

\par 4 Y viendo allí a Judá, a Gad y a Aser, sus hermanos, les dijo: Levantadme para que pueda contar a mis hermanos y a mis hijos lo que tengo escondido en mi corazón, porque he aquí finalmente Estoy falleciendo.

\par 5 Y se levantó, los besó y les dijo: Oíd, hermanos míos, y vosotros, hijos míos, escuchad a Rubén vuestro padre los mandamientos que os doy.

\par 6 Y he aquí que hoy pongo por testigo contra vosotros al Dios del cielo, de que no andéis en los pecados de la juventud y de la fornicación, con los cuales fui derramado y profané el lecho de mi padre Jacob.

\par 7 Y os digo que me hirió con una plaga dolorosa en mis lomos durante siete meses; y si mi padre Jamb no hubiera orado por mí al Señor, el Señor me habría destruido.

\par 8 Porque yo tenía treinta años cuando hice el mal ante el Señor, y durante siete meses estuve enfermo de muerte.

\par 9 Y después de esto me arrepentí con el propósito de mi alma durante siete años delante del Señor.

\par 10 No bebí vino ni sidra, ni entró carne en mi boca, ni comí manjares deliciosos; pero me lamenté de mi pecado, porque era tan grande, cual no lo hubo en Israel.

\par 11 Y ahora, hijos míos, oídme lo que vi acerca de los siete espíritus del engaño cuando me arrepentí.

\par 12 Por tanto, siete espíritus están designados contra el hombre, y son los líderes en las obras de la juventud.

\par 13 Y en el momento de su creación se le dan otros siete espíritus para que por medio de ellos se realice toda obra del hombre.

\par 14 El primero es el espíritu de vida, con el cual se crea la constitución del hombre.

\par 15 El segundo es el sentido de la vista, con el que surge el deseo.

\par 16 El tercero es el oído, con el que viene la enseñanza.

\par 17 El cuarto es el sentido del olfato, con el que se dan los gustos para aspirar el aire y respirar.

\par 18 El quinto es el poder del habla, con el cual viene el conocimiento.

\par 19 El sexto es el gusto, que acompaña al consumo de alimentos y bebidas; y por él se produce la fuerza, porque en el alimento está el fundamento. de fuerza.

\par 20 La séptima es la facultad de procreación y de las relaciones sexuales, con la que entran los pecados por el amor al deleite.

\par 21 Por eso es el último en el orden de la creación y el primero en el de la juventud, porque está lleno de ignorancia y conduce al joven como a un ciego a un hoyo y como a una bestia a un precipicio.

\par 22 Además de todos estos, hay un octavo espíritu del sueño, con el que se produce el trance de la naturaleza y el de la muerte.

\par 23 Con estos espíritus se mezclan los espíritus del error.

\par 24 Primero, el espíritu de fornicación reside en la naturaleza y en los sentidos;

\par 25 El segundo, el espíritu de insaciabilidad en el vientre;

\par 26 El tercero, el espíritu de lucha, en el hígado y en la hiel.

\par 27 El cuarto es el espíritu de servilismo y astucia, para que mediante la atención oficiosa uno pueda parecer justo.

\par 28 El quinto es el espíritu de soberbia, para ser jactanciosos y arrogantes.

\par 29 El sexto es el espíritu de mentira, de perdición y celos, para practicar engaños y ocultaciones a parientes y amigos.

\par 30 El séptimo es el espíritu de injusticia, con el que se cometen robos y actos de rapacidad, para que el hombre pueda cumplir el deseo de su corazón; porque la injusticia colabora con los otros espíritus al recibir regalos.

\par 31 Y a todo esto se une el espíritu del sueño, que es el del error y la fantasía.

\par 32 Y así perece todo joven, que oscurece su mente de la verdad y no comprende la ley de Dios ni obedece las amonestaciones de sus padres, como también me sucedió a mí en mi juventud.

\par 33 Ahora pues, hijos míos, amad la verdad y ella os preservará: oíd las palabras de Rubén vuestro padre.

\par 34 No prestes atención al rostro de una mujer,

\par 35 Ni te juntarás con la mujer de otro,

\par 36 Ni te metas en asuntos de mujeres.

\par 37 Porque si no hubiera visto a Bilha bañándose en un lugar cubierto, no habría caído en esta gran iniquidad.

\par 38 Porque mi mente, al pensar en la desnudez de la mujer, no me permitió dormir hasta haber cometido aquella abominable cosa.

\par 39 Porque mientras nuestro padre Jacob había ido a visitar a su padre Isaac, cuando estábamos en Eder, cerca de Efrata en Belén, Bilha se emborrachó y dormía descubierta en su cámara.

\par 40 Entré y vi su desnudez, cometí la impiedad sin que ella se diera cuenta y, dejándola dormida, me fui.

\par 41 Entonces un ángel de Dios reveló a mi padre mi impiedad, y vino y se lamentó por mí y no la tocó más.

\chapter{2}

\par \textit{Rubén continúa con sus experiencias y sus buenos consejos.}

\par 1 Por tanto, hijos míos, no prestéis atención a la belleza de las mujeres ni os fijéis en sus asuntos; sino caminad con sencillez de corazón en el temor del Señor, y esforzaos en buenas obras, en el estudio y en vuestros rebaños, hasta que el Señor os dé la mujer que Él quiera, para que no sufráis como yo.

\par 2 Porque hasta la muerte de mi padre no tuve valor para mirarlo a la cara ni para hablar con ninguno de mis hermanos a causa del oprobio.

\par 3 Hasta ahora mi conciencia me aflige por mi impiedad.

\par 4 Sin embargo, mi padre me consoló mucho y oró por mí al Señor para que la ira del Señor pasara de mí, tal como el Señor me había mostrado.

\par 5 Desde entonces hasta ahora he estado en guardia y no he pecado.

\par 6 Por eso, hijos míos, os digo que guardéis todo lo que os mando y no pecaréis.

\par 7 Porque un hoyo para el alma es el pecado de la fornicación, que la separa de Dios y la acerca a los ídolos, porque engaña la mente y el entendimiento y hace descender a los jóvenes al Hades antes de tiempo.

\par 8 La fornicación destruyó a muchos; porque aunque un hombre sea viejo o noble, o rico o pobre, se acarrea afrenta con los hijos de los hombres y escarnio con Beliar.

\par 9 Porque habéis oído acerca de José cómo se protegió de una mujer, y limpió sus pensamientos de toda fornicación, y halló favor ante los ojos de Dios y de los hombres.

\par 10 Porque la mujer egipcia le hizo muchas cosas, llamó a magos y le ofreció pociones de amor, pero el propósito de su alma no admitía ningún mal deseo.

\par 11 Por eso el Dios de vuestros padres lo libró de todo mal y muerte oculta.

\par 12 Porque si la fornicación no vence en tu mente, tampoco Beliar podrá vencerte.

\par 13 Porque malas son las mujeres, hijos míos; y como no tienen poder ni fuerza sobre el hombre, utilizan artimañas mediante atracciones externas para atraerlo hacia sí.

\par 14 Y a quien no pueden hechizar con atracciones externas, lo vencen con astucia.

\par 15 Además, acerca de ellos, el ángel del Señor me dijo y me enseñó que las mujeres son dominadas por el espíritu de fornicación más que los hombres, y en su corazón conspiran contra los hombres; y por medio de sus adornos engañan primero sus mentes, y con la mirada les infunden el veneno, y luego mediante el acto consumado los llevan cautivos.

\par 16 Porque la mujer no puede forzar abiertamente al hombre, sino que lo engaña con un porte de ramera.

\par 17 Huid, pues, de la fornicación, hijos míos, y mandad a vuestras mujeres y a vuestras hijas que no se adornen la cabeza ni el rostro para engañar la mente; porque toda mujer que usa estas artimañas está reservada para el castigo eterno.

\par 18 Porque así atrajeron a los Vigilantes 1 que estaban antes del diluvio; porque como éstos los contemplaban continuamente, los codiciaban y concebían el acto en su mente; porque se transformaron en forma de hombres y se les aparecieron cuando estaban con sus maridos.

\par 19 Y las mujeres, que codiciaban mentalmente sus formas, dieron a luz gigantes, pues les parecía que los Vigilantes llegaban hasta el cielo.

\par 20 Guardaos, pues, de la fornicación; y si deseas ser puro de mente, guarda tus sentidos de toda mujer.

\par 21 Y ordena también a las mujeres que no se relacionen con los hombres, para que también ellas sean puras de mente.

\par 22 Porque las reuniones constantes, aunque no se cometa el acto impío, son para ellos una enfermedad irremediable, y para nosotros una destrucción de Beliar y un reproche eterno.

\par 23 Porque en la fornicación no hay ni entendimiento ni piedad, y todo celo reside en su concupiscencia.

\par 24 Por tanto, os digo que tendréis celos de los hijos de Leví y procuraréis ser exaltados sobre ellos; pero no podréis.

\par 25 Porque Dios los vengará y vosotros moriréis de mala muerte. Porque a Leví Dios le dio el poder, y a Judá con él, y a mí también, y a Dan y a José, para que fuéramos por gobernantes.

\par 26 Por eso te ordeno que escuches a Leví, porque él conocerá la ley del Señor, dará ordenanzas para el juicio y sacrificará por todo Israel hasta la consumación de los tiempos, como el Sumo Sacerdote ungido, de quien el Señor habló.

\par 27 Os conjuro por el Dios del cielo que cada uno practique la verdad con su prójimo y que cada uno tenga amor por su hermano.

\par 28 Y acercaos a Leví con humildad y corazón, para recibir bendición de su boca.

\par 29 Porque él bendecirá a Israel y a Judá, porque a él el Señor ha elegido por rey sobre toda la nación.

\par 30 Y postraos ante su descendencia, porque por nosotros morirá en guerras visibles e invisibles, y será entre vosotros un rey eterno.

\par 31 Y Rubén murió después de haber dado estas órdenes a sus hijos. Y lo pusieron en un ataúd hasta que lo sacaron de Egipto y lo sepultaron en Hebrón, en la cueva donde estaba su padre.

\par \textit{Notas al pie}

\par \textit{223:1 Ver El Segundo Libro de Adán y Eva, Capítulo XX}


\end{document}