\begin{document}

\title{Testamento de Judá}

\chapter{1}

\par \textit{Judá, el cuarto hijo de Jacob y Lea. Él es el gigante, el atleta, el guerrero; relata hazañas heroicas. Corre tan rápido que puede adelantar a una cierva.}

\par 1 LA copia de las palabras de Judá, lo que habló a sus hijos antes de morir.

\par 2 Entonces se reunieron y vinieron a él, y él les dijo: Escuchen, hijos míos, a Judá, su padre.

\par 3 Yo fui el cuarto hijo de mi padre Jacob; y mi madre Lea me llamó Judá, diciendo: Doy gracias a Jehová, porque también me ha dado un cuarto hijo.

\par 4 Fui ágil en mi juventud y obediente a mi padre en todo.

\par 5 Y honré a mi madre y a la hermana de mi madre.

\par 6 Y cuando me hice hombre, mi padre me bendijo, diciendo: Serás un rey prosperado en todo.

\par 7 Y el Señor me mostró favor en todas mis obras, tanto en el campo como en la casa.

\par 8 Sé que corrí con una cierva, la cacé, preparé carne para mi padre y él comió.

\par 9 Y yo dominaba las corvas en la caza y alcanzaba a todos los que había en la llanura.

\par 10 Una yegua salvaje la alcancé, la atrapé y la domé.

\par 11 Maté un león y le saqué un cabrito de la boca.

\par 12 Tomé un oso por la pata y lo arrojé por un acantilado, y quedó aplastado.

\par 13 Corrí más rápido que el jabalí y, agarrándolo mientras corría, lo partí en pedazos.

\par 14 Un leopardo en Hebrón saltó sobre mi perro, lo agarré por la cola, lo arrojé contra las rocas y se partió en dos.

\par 15 Encontré un buey salvaje pastando en el campo, y agarrándolo por los cuernos, haciéndolo girar y aturdiéndolo, lo arrojé lejos de mí y lo maté.

\par 16 Y cuando los dos reyes de los cananeos vinieron enfundados y armados contra nuestros rebaños, y mucha gente con ellos, yo solo me abalancé sobre el rey de Hazor, lo golpeé en las grevas y lo arrastré hacia abajo, y así lo mató.

\par 17 Y al otro, el rey de Tappuah, mientras estaba sentado en su caballo, lo maté, y así dispersé a todo su pueblo.

\par 18 Encontré al rey Acor, un hombre de estatura gigante, que lanzaba jabalinas por delante y por detrás mientras montaba a caballo; y tomé una piedra que pesaba sesenta libras, la arrojé, golpeé su caballo y lo maté.

\par 19 Y luché con este otro durante dos horas; y partí su escudo en dos, le corté los pies y lo maté.

\par 20 Y mientras yo le quitaba la coraza, he aquí nueve hombres suyos comenzaron a pelear conmigo,

\par 21 Y me enrollé mi manto en la mano; y les lancé piedras, y maté a cuatro de ellos, y los demás huyeron.

\par 22 Y mi padre Jacob mató a Beelesat, rey de todos los reyes, un gigante de fuerza de doce codos de altura.

\par 23 Y el miedo se apoderó de ellos y dejaron de luchar contra nosotros.

\par 24 Por eso mi padre estaba libre de preocupaciones en las guerras cuando yo estaba con mis hermanos.

\par 25 Porque vio en una visión acerca de mí que un ángel poderoso me seguía a todas partes para que no fuera vencido.

\par 26 Y en el sur nos sobrevino una guerra mayor que la de Siquem; y me puse en orden de batalla con mis hermanos, y perseguí a mil hombres, y de ellos maté a doscientos hombres y a cuatro reyes.

\par 27 Subí a la muralla y maté a cuatro hombres valientes.

\par 28 Entonces capturamos a Hazor y nos llevamos todo el botín.

\par 29 Y al día siguiente partimos hacia Aretán, una ciudad fuerte, amurallada e inaccesible, que nos amenazaba de muerte.

\par 30 Pero Gad y yo nos acercamos por el lado oriental de la ciudad, y Rubén y Leví por el oeste.

\par 31 Y los que estaban en el muro, pensando que estábamos solos, se abalanzaron contra nosotros.

\par 32 Entonces mis hermanos treparon en secreto por ambos lados de la muralla con estacas y entraron en la ciudad, sin que los hombres lo supieran.

\par 33 Y la tomamos a filo de espada.

\par 34 Y a los que se habían refugiado en la torre, les prendimos fuego y nos apoderamos de ella y de ellos.

\par 35 Y mientras nos íbamos, los hombres de Tappuah se apoderaron de nuestro botín y, al ver esto, peleamos con ellos.

\par 36 Y los matamos a todos y recuperamos nuestro botín.

\par 37 Y estando yo junto a las aguas de Kozeba, los hombres de Jobel vinieron contra nosotros para pelear.

\par 38 Y peleamos con ellos y los derrotamos; y a sus aliados los matamos en Silo, y no les dejamos poder para venir contra nosotros.

\par 39 Y al quinto día los hombres de Maquir vinieron sobre nosotros para apoderarse de nuestro botín; y los atacamos y los vencimos en feroz batalla; porque había entre ellos un ejército de hombres valientes, y los matamos antes de que hubieran subido la cuesta.

\par 40 Y cuando llegamos a su ciudad, sus mujeres hicieron rodar sobre nosotros piedras desde la cima de la colina sobre la que se encontraba la ciudad.

\par 41 Y Simeón y yo nos pusimos detrás de la ciudad, tomamos las alturas y destruimos también esta ciudad.

\par 42 Y al día siguiente nos dijeron que el rey de la ciudad de Gaash con un ejército poderoso venía contra nosotros.

\par 43 Dan y yo nos hicimos pasar por amorreos y entramos aliados en su ciudad.

\par 44 Y en plena noche vinieron nuestros hermanos y les abrimos las puertas; y destruimos a todos los hombres y sus bienes, y tomamos por presa todo lo que era suyo, y derribamos sus tres muros.

\par 45 Y nos acercamos a Thamna, donde estaban todos los bienes de los reyes enemigos.

\par 46 Entonces, insultado por ellos, me enojé y me lancé contra ellos hasta la cima; y seguían lanzando contra mí piedras y dardos.

\par 47 Y si mi hermano Dan no me hubiera ayudado, me habrían matado.

\par 48 Entonces nos abalanzamos sobre ellos con ira y todos huyeron; y pasando por otro camino, pelearon contra mi padre, y él hizo la paz con ellos.

\par 49 Y no les hicimos ningún daño, sino que se hicieron tributarios de nosotros y les devolvimos su botín.

\par 50 Y yo edifiqué Tamna, y mi padre edificó a Pabael.

\par 51 Yo tenía veinte años cuando estalló esta guerra. Y los cananeos temían a mí y a mis hermanos.

\par 52 Tenía mucho ganado y tenía como jefe de pastores a Iram el adullamita.

\par 53 Y cuando fui a él, vi a Parsaba, rey de Adulam; y nos habló, y nos hizo un banquete; y cuando me enojé, me dio a su hija Bathshua por esposa.

\par 54 Ella me dio a luz a Er, a Onán y a Sela; y a dos de ellos hirió Jehová; porque Sela vivió, y vosotros sois sus hijos.

\chapter{2}

\par \textit{Judá describe algunos hallazgos arqueológicos, una ciudad con murallas de hierro y puertas de bronce. Tiene un encuentro con una aventurera.}

\par 1 Y dieciocho años mi padre estuvo en paz con su hermano Esaú, y sus hijos con nosotros, después que vinimos de Mesopotamia, de Labán.

\par 2 Y cuando se cumplieron dieciocho años, en el año cuarenta de mi vida, Esaú, el hermano de mi padre, vino sobre nosotros con un pueblo poderoso y fuerte.

\par 3 Y Jacob hirió a Esaú con una flecha, y fue llevado herido al monte de Seir, y yendo, murió en Anoniram.

\par 4 Y perseguimos a los hijos de Esaú.

\par 5 Tenían una ciudad con muros de hierro y puertas de bronce; y no pudimos entrar en ella, y acampamos alrededor, y la sitiamos.

\par 6 Y como no nos abrieron después de veinte días, levanté una escalera a la vista de todos y con mi escudo sobre mi cabeza subí, soportando el asalto de piedras de más de tres talentos de peso; y maté a cuatro de sus valientes.

\par 7 Rubén y Gad mataron a otros seis.

\par 8 Entonces nos pidieron condiciones de paz; y habiendo consultado con nuestro padre, los recibimos como tributarios.

\par 9 Y nos dieron quinientos cors de trigo, quinientos batos de aceite y quinientas medidas de vino, hasta que llegó el hambre y bajamos a Egipto.

\par 10 Y después de estas cosas, mi hijo Er tomó por mujer a Tamar, de Mesopotamia, hija de Aram.

\par 11 Er era malvado y tenía necesidad de Tamar, porque ella no era de la tierra de Canaán.

\par 12 Y la tercera noche, un ángel del Señor lo hirió.

\par 13 Y él no la había conocido según la malvada astucia de su madre, porque no quería tener hijos de ella.

\par 14 En los días de las bodas le di a Onán en matrimonio; y él también en maldad no la conoció, aunque estuvo con ella un año.

\par 15 Y cuando lo amenacé, él entró a ella, pero derramó la semilla en el suelo, según la orden de su madre, y él también murió a causa de la maldad.

\par 16 Y quise darle también a Sela, pero su madre no lo permitió; porque hizo mal contra Tamar, por no ser hija de Canaán, como también lo era ella misma.

\par 17 Y sabía que la raza de los cananeos era mala, pero el impulso de la juventud cegó mi mente.

\par 18 Y cuando la vi derramando vino, a causa de la embriaguez del vino, me engañé y la tomé sin que mi padre me lo hubiera aconsejado.

\par 19 Y mientras yo estaba fuera, ella fue y tomó para Sela una esposa de Canaán.

\par 20 Y cuando supe lo que había hecho, la maldije en la angustia de mi alma.

\par 21 Y ella también murió por su maldad junto con sus hijos.

\par 22 Después de estas cosas, siendo Tamar viuda, al cabo de dos años oyó que yo iba a esquilar mis ovejas, se atavió con ropas nupciales y se sentó en la ciudad de Enaim, junto a la puerta.

\par 23 Porque era ley de los amorreos que la que iba a casarse debía permanecer siete días en la puerta fornicando.

\par 24 Entonces, ebrio de vino, no la reconocí; y su hermosura me engañó por la moda de sus adornos.

\par 25 Entonces me volví hacia ella y le dije: Déjame entrar a ti.

\par 26 Y ella dijo: ¿Qué me darás? Y le di en prenda mi cayado, mi cinto y la diadema de mi reino.

\par 27 Y entré a ella, y ella concibió.

\par 28 Y sin saber lo que había hecho, quise matarla; pero ella en secreto envió mis promesas y me avergonzó.

\par 29 Y cuando la llamé, oí también las palabras secretas que hablé mientras yacía con ella en mi borrachera; y no pude matarla, porque era del Señor.

\par 30 Porque dije: No sea que lo haya hecho con astucia, habiendo recibido la promesa de otra mujer.

\par 31 Pero no volví a acercarme a ella mientras vivía, porque había cometido esta abominación en todo Israel.

\par 32 Además, los que estaban en la ciudad dijeron que no había ninguna ramera en la puerta, porque venía de otro lugar, y se quedó un rato sentada a la puerta.

\par 33 Y pensé que nadie sabía que me había unido a ella.

\par 34 Después de esto, a causa del hambre, llegamos a Egipto donde José.

\par 35 Yo tenía cuarenta y seis años y setenta y tres años viví en Egipto.

\chapter{3}

\par \textit{Él desaconseja el vino y la lujuria como males gemelos. «Porque el que está ebrio no respeta a nadie». (Verso 13).}

\par 1 Y ahora os ordeno, hijos míos, que oigáis a vuestro padre Judá y guardéis mis palabras para cumplir todas las ordenanzas del Señor y obedecer los mandamientos de Dios.

\par 2 Y no andéis tras vuestras concupiscencias, ni en la imaginación de vuestros pensamientos con altivez de corazón; y no te gloríes en las obras y en las fuerzas de tu juventud, porque también esto es malo ante los ojos del Señor.

\par 3 Puesto que también me gloriaba en que en las guerras nunca me sedujo el rostro de mujer hermosa, y reprendí a mi hermano Rubén acerca de Bilha, la esposa de mi padre, los espíritus de celos y de fornicación se alinearon contra mí, hasta que me acosté con Betsúa, la cananea y Tamar, la desposada de mis hijos.

\par 4 Porque dije a mi suegro: Consultaré con mi padre y también tomaré a tu hija.

\par 5 Pero él no quiso, pero me mostró una cantidad ilimitada de oro en nombre de su hija; porque era rey.

\par 6 Y la adornó con oro y perlas, y la hizo servir vino para nosotros en la fiesta con la belleza de las mujeres.

\par 7 Y el vino desvió mis ojos, y el placer cegó mi corazón.

\par 8 Y me enamoré de ella y me acosté con ella, y transgredí el mandamiento del Señor y el mandamiento de mis padres, y la tomé por esposa.

\par 9 Y el Señor me recompensó según la imaginación de mi corazón, ya que no tenía alegría en sus hijos.

\par 10 Ahora bien, hijos míos, os digo que no os embriaguéis con vino; porque el vino desvía la mente de la verdad, inspira la pasión de la concupiscencia y lleva los ojos al error.

\par 11 Porque el espíritu de fornicación tiene el vino como ministro para complacer la mente; porque estos dos también quitan la mente del hombre.

\par 12 Porque si un hombre bebe vino hasta emborracharse, esto perturba la mente con pensamientos inmundos que conducen a la fornicación, y calienta el cuerpo para la unión carnal; y si se presenta la ocasión de la concupiscencia, comete el pecado y no se avergüenza.

\par 13 Así es el ebrio, hijos míos; porque el que está ebrio no respeta a nadie.

\par 14 Porque he aquí, también a mí me hizo errar, para no avergonzarme de la multitud de la ciudad, al volverme ante los ojos de todos hacia Tamar, y cometí un gran pecado, y descubrí el velo de la vergüenza de mis hijos.

\par 15 Después de haber bebido vino, no respeté el mandamiento de Dios y tomé por esposa a una mujer de Canaán.

\par 16 Porque mucha discreción necesita el hombre que bebe vino, hijos míos; y en esto está la discreción al beber vino, un hombre puede beberlo siempre que conserve la modestia.

\par 17 Pero si va más allá de este límite, el espíritu de engaño ataca su mente, y hace que el borracho hable obscenamente y transgreda, y no se avergüence, sino que incluso se gloríe en su vergüenza y se considere honorable.

\par 18 El que comete fornicación no se da cuenta cuando sufre pérdida, ni se avergüenza cuando es deshonrado.

\par 19 Porque aunque un hombre sea rey y cometa fornicación, quedará despojado de su realeza al convertirse en esclavo de la fornicación, como también yo sufrí.

\par 20 Porque di mi cayado, es decir, el sustento de mi tribu; y mi cinto, es decir, mi poder; y mi diadema, es decir, la gloria de mi reino.

\par 21 Y ciertamente me arrepentí de estas cosas; No comí vino ni carne hasta mi vejez, ni vi alegría alguna.

\par 22 Y el ángel de Dios me mostró que para siempre las mujeres dominan tanto al rey como al mendigo.

\par 23 Y al rey le quitan su gloria, al valiente su poder, y al mendigo incluso lo poco que es el sustento de su pobreza.

\par 24 Por tanto, hijos míos, observad el límite justo en el vino; porque hay en él cuatro espíritus malignos: de la concupiscencia, del deseo ardiente, del libertinaje y de las ganancias deshonestas.

\par 25 Si bebéis vino con alegría, sed modestos en el temor de Dios.

\par 26 Porque si en vuestra alegría desaparece el temor de Dios, entonces surge la embriaguez y se infiltra la desvergüenza.

\par 27 Pero si queréis vivir sobriamente, no toquéis vino en absoluto, no sea que pequéis con palabras ultrajes, con peleas, con calumnias y con transgresiones de los mandamientos de Dios, y perezcáis antes de tiempo.

\par 28 Además, el vino revela los misterios de Dios y de los hombres, así como también yo revelé a la cananea Betsúa los mandamientos de Dios y los misterios de mi padre Jacob, los cuales Dios me había ordenado que no revelara.

\par 29 Y el vino es causa de guerra y de confusión.

\par 30 Ahora pues, os mando, hijos míos, que no améis el dinero ni miréis la belleza de las mujeres; porque por causa del dinero y de la belleza fui llevado por el mal camino a Betsúa la cananea.

\par 31 Porque sé que por estas dos cosas mi raza caerá en la maldad.

\par 32 Porque incluso los sabios de mis hijos arruinarán y reducirán el reino de Judá que el Señor me dio por mi obediencia a mi padre.

\par 33 Porque nunca causé tristeza a mi padre Jacob; porque todo lo que él me ordenó, lo hice.

\par 34 Isaac, el padre de mi padre, me bendijo para que fuera rey de Israel, y Jacob me bendijo de la misma manera.

\par 35 Y sé que de mí se establecerá el reino.

\par 36 Y sé los males que haréis en los últimos días.

\par 37 Guardaos, pues, hijos míos, de la fornicación y del amor al dinero, y escuchad a vuestro padre Judá.

\par 38 Porque estas cosas apartan de la ley de Dios, ciegan la inclinación del alma, enseñan la soberbia y no permiten que nadie tenga compasión de su prójimo.

\par 39 Privan su alma de todo bien, la oprimen con trabajos y angustias, le quitan el sueño y devoran su carne.

\par 40 Y obstaculiza los sacrificios de Dios; y no se acuerda de la bendición de Dios, no escucha al profeta cuando habla, y se resiente de las palabras de piedad.

\par 41 Porque es esclavo de dos pasiones contrarias y no puede obedecer a Dios, porque han cegado su alma, y ​​camina de día como de noche.

\par 42 Hijitos míos, el amor al dinero lleva a la idolatría; porque, descarriados por el dinero, los hombres nombran dioses a los que no lo son, y eso hace caer en la locura a quien lo tiene.

\par 43 Por causa del dinero perdí a mis hijos, y si mi arrepentimiento, mi humillación y las oraciones de mi padre no hubieran sido aceptados, habría muerto sin hijos.

\par 44 Pero el Dios de mis padres tuvo misericordia de mí, porque lo hice en ignorancia.

\par 45 Y el príncipe del engaño me cegó, y pequé como hombre y como carne, corrompiéndome por los pecados; y aprendí mi propia debilidad creyéndome invencible.

\par 46 Sepan, pues, hijos míos, que dos espíritus acechan al hombre: el espíritu de verdad y el espíritu de engaño.

\par 47 Y en medio está el espíritu de entendimiento de la mente, al cual pertenece para volverse a donde quiera.

\par 48 Y las obras de la verdad y las obras del engaño están escritas en el corazón de los hombres, y el Señor conoce cada una de ellas.

\par 49 Y no hay tiempo en que las obras de los hombres puedan ocultarse; porque en el corazón mismo están escritas delante del Señor.

\par 50 Y el espíritu de verdad testifica todas las cosas y todo lo acusa; y el pecador se quema en su propio corazón y no puede levantar su rostro ante el juez.

\chapter{4}

\par \textit{Judá hace un vívido símil sobre la tiranía y una terrible profecía sobre la moral de sus oyentes.}

\par 1 Y ahora, hijos míos, os mando que améis a Leví para que perseveréis y no os ensoberbéis contra él, para que no seáis destruidos por completo.

\par 2 Porque a mí el Señor me dio el reino, y a él el sacerdocio, y puso el reino bajo el sacerdocio.

\par 3 A mí me dio las cosas que hay en la tierra; a él las cosas que están en los cielos.

\par 4 Como el cielo es más alto que la tierra, así el sacerdocio de Dios es más alto que el reino terrenal, a menos que por el pecado se aparte del Señor y sea dominado por el reino terrenal.

\par 5 Porque el ángel del Señor me dijo: El Señor lo escogió a él antes que a ti, para acercarse a Él, comer de su mesa y ofrecerle las primicias de las cosas escogidas de los hijos de Israel. ; pero tú serás rey de Jacob.

\par 6 Y serás entre ellos como el mar.

\par 7 Porque así como en el mar son arrojados justos e injustos, unos llevados cautivos y otros enriquecidos, así también estará en ti toda raza humana: algunos serán empobrecidos siendo llevados cautivos, y otros se enriquecerán saqueando las posesiones ajenas.

\par 8 Porque los reyes serán como monstruos marinos.

\par 9 Se tragarán a los hombres como peces; esclavizarán a los hijos y a las hijas de los libres; Casas, tierras, rebaños, dinero saquearán:

\par 10 Y con la carne de muchos alimentarán injustamente a los cuervos y a las grullas; y avanzarán en el mal en avaricia enaltecida, y habrá falsos profetas como tempestad, y perseguirán a todos los justos.

\par 11 Y el Señor traerá sobre ellos divisiones unos contra otros.

\par 12 Y habrá guerras continuas en Israel; y entre los hombres de otra raza será puesto fin a mi reino, hasta que venga la salvación de Israel.

\par 13 Hasta la aparición del Dios de justicia, para que Jacob y todos los gentiles descansen en paz.

\par 14 Y Él guardará el poder de mi reino para siempre; porque el Señor me hizo saber con juramento que no destruiría el reino de mi descendencia para siempre.

\par 15 Ahora tengo mucho dolor, hijos míos, por vuestras lascivias y hechicerías e idolatrías que practicaréis contra el reino, siguiendo a espíritus familiares, adivinos y demonios del error.

\par 16 Haréis que vuestras hijas sean cantoras y rameras, y os mezclaréis con las abominaciones de las naciones.

\par 17 Por estas cosas el Señor traerá sobre vosotros hambre y pestilencia, muerte y espada, asedios de enemigos e injurias a amigos, matanza de niños, violación de esposas, saqueo de bienes, incendio de el templo de Dios, la devastación de la tierra, la servidumbre de vosotros mismos entre los gentiles.

\par 18 Y algunas de vosotras tomarán eunucos para sus esposas.

\par 19 Hasta que el Señor os visite, cuando con corazón perfecto os arrepintáis y andéis en todos sus mandamientos, y Él os saque del cautiverio entre los gentiles.

\par 20 Y después de estas cosas os surgirá una estrella de paz de parte de Jacob,

\par 21 Y de mi descendencia surgirá un hombre como el sol de justicia,

\par 22 Andando con los hijos de los hombres en mansedumbre y rectitud;

\par 23 Y no se hallará en él ningún pecado.

\par 24 Y los cielos se le abrirán para derramar el espíritu, la bendición del Santo Padre; y Él derramará sobre vosotros el espíritu de gracia;

\par 25 Y seréis para Él hijos en verdad, y andaréis en Sus mandamientos, primero y último.

\par 26 Entonces resplandecerá el cetro de mi reino; y de tu raíz surgirá un tallo; y de él crecerá una vara de justicia para los gentiles, para juzgar y salvar a todos los que invocan al Señor.

\par 27 Y después de esto resucitarán Abraham, Isaac y Jacob; y yo y mis hermanos seremos jefes de las tribus de Israel:

\par 28 Leví primero, yo el segundo, José el tercero, Benjamín el cuarto, Simeón el quinto, Isacar el sexto, y así todos en orden.

\par 29 Y el Señor me bendijo a Leví y al ángel de la Presencia; los poderes de la gloria, Simeón; el cielo, Rubén; la tierra, Isacar; el mar, Zabulón; las montañas, José; el tabernáculo, Benjamín; las luminarias, Dan; Edén, Neftalí; el sol, Gad; la luna, Aser.

\par 30 Y seréis el pueblo del Señor y tendréis una sola lengua; y no habrá espíritu de engaño en Beliar, porque será arrojado al fuego para siempre.

\par 31 Y los que murieron en tristeza se levantarán con alegría, y los que eran pobres por causa del Señor se enriquecerán, y los que son condenados a muerte por causa del Señor despertarán a la vida.

\par 32 Y los ciervos de Jacob correrán de alegría, y las águilas de Israel volarán de alegría; y todo el pueblo glorificará al Señor para siempre.

\par 33 Por tanto, hijos míos, observad toda la ley del Señor, porque hay esperanza para todos los que se aferran a sus caminos.

\par 34 Y él les dijo: He aquí, hoy muero ante vuestros ojos, a la edad de ciento diecinueve años.

\par 35 Que nadie me entierre con ropas costosas ni me abra las entrañas, porque esto harán los reyes; y llévame contigo a Hebrón.

\par 36 Y Judá, habiendo dicho estas cosas, se durmió; E hicieron sus hijos conforme a todo lo que él les mandó, y lo sepultaron en Hebrón con sus padres.


\end{document}