\begin{document}

\title{Testamento de Neftalí}

\chapter{1}

\par \textit{Neftalí, el octavo hijo de Jacob y Bilha. El corredor. Una lección de fisiología.}

\par 1 LA copia del testamento de Neftalí, que ordenó en el momento de su muerte en el año ciento treinta de su vida.

\par 2 Cuando sus hijos se reunieron en el mes séptimo, el primer día del mes, cuando todavía estaban sanos, les preparó un banquete de comida y vino.

\par 3 Y cuando se despertó por la mañana, les dijo: Me muero; y no le creyeron.

\par 4 Y mientras glorificaba al Señor, se fortaleció y dijo que después de la fiesta de ayer moriría.

\par 5 Y comenzó entonces a decir: Oíd, hijos míos, hijos de Neftalí, oíd las palabras de vuestro padre.

\par 6 Yo nací de Bilha, y como Raquel actuó con astucia y entregó a Bilha en lugar de ella a Jacob, y ella concibió y me dio a luz sobre las rodillas de Raquel, por eso llamó mi nombre Neftalí.

\par 7 Porque Raquel me amó mucho porque nací en su regazo; y cuando yo era aún joven solía besarme y decirme: Que tenga yo desde mi propio vientre un hermano tuyo como tú.

\par 8 Por lo cual también José fue semejante a mí en todo, según las oraciones de Raquel.

\par 9 Mi madre era Bilhá, hija de Roteo, hermano de Débora, nodriza de Rebeca, que nació el mismo día que Raquel.

\par 10 Y Roteo era de la familia de Abraham, caldeo, temeroso de Dios, libre y noble.

\par 11 Y fue llevado cautivo y comprado por Labán; y le dio por mujer a Euna su sierva, la cual dio a luz una hija, y llamó su nombre Zilpá, según el nombre de la aldea en la que había sido llevado cautivo.

\par 12 Y luego dio a luz a Bilha, diciendo: Mi hija se apresura tras lo nuevo, porque apenas nació, tomó el pecho y se apresuró a mamarlo.

\par 13 Y yo era ligero de pies como el ciervo, y mi padre Jacob me designó para todos los mensajes, y como un ciervo me dio su bendición.

\par 14 Porque así como el alfarero sabe cuánto ha de contener la vasija y trae el barro correspondiente, así también el Señor hace el cuerpo a semejanza del espíritu, y según la capacidad del cuerpo implanta el espíritu.

\par 15 Y el uno no es inferior al otro ni en la tercera parte de un cabello; porque con peso, medida y regla fue hecha toda la creación.

\par 16 Y como el alfarero sabe el uso de cada vaso y para qué sirve, así también el Señor conoce el cuerpo, hasta qué punto persistirá en el bien y cuándo comenzará en el mal.

\par 17 Porque no hay inclinación ni pensamiento que el Señor no conozca, pues Él creó a cada hombre a su imagen.

\par 18 Porque como la fuerza del hombre, así también en su trabajo; como su ojo, así también en su sueño; como su alma, así también en su palabra ya sea en la ley del Señor o en la ley de Beliar.

\par 19 Y así como hay división entre la luz y las tinieblas, entre ver y oír, así también hay división entre hombre y hombre, y entre mujer y mujer; y no se puede decir que uno sea igual al otro ni en el rostro ni en la mente.

\par 20 Porque Dios hizo todas las cosas en su orden, los cinco sentidos en la cabeza, y unió el cuello a la cabeza, añadiendo a ella también el cabello para la hermosura y la gloria, luego el corazón para el entendimiento, el vientre para la los excrementos, y el estómago para moler, la tráquea para respirar, el hígado para la ira, la hiel para la amargura, el bazo para la risa, las riendas para la prudencia, los músculos de los lomos para la fuerza, los pulmones para la aspiración, los lomos para fortalecerse, etc.

\par 21 Así que, hijos míos, todas vuestras obras háganse con orden, con buena intención y en el temor de Dios, y no hagan nada desordenadamente, con desprecio o fuera de tiempo.

\par 22 Porque si pides al ojo que oiga, no puede; Así que ni estando en tinieblas podréis hacer las obras de la luz.

\par 23 Por tanto, no os afanéis por corromper vuestras obras con avaricia o con palabras vanas para engañar vuestras almas; porque si calláis con pureza de corazón, sabréis retener la voluntad de Dios y desechar la voluntad de Beliar.

\par 24 Sol, luna y estrellas, no cambien su orden; Así también vosotros no cambiéis la ley de Dios en el desorden de vuestras obras.

\par 25 Los gentiles se extraviaron y abandonaron al Señor, impusieron sus órdenes y obedecieron a los leones y a las piedras, espíritus de engaño.

\par 26 Pero vosotros, hijos míos, no seréis así, reconociendo en el firmamento, en la tierra, en el mar y en todas las cosas creadas, al Señor que hizo todas las cosas, para que no seáis como Sodoma, que cambió el mundo. orden de la naturaleza.

\par 27 De la misma manera cambiaron el orden de su naturaleza los Vigilantes, a quienes el Señor maldijo en el diluvio, por cuya causa dejó la tierra sin habitantes y sin fruto.

\par 28 Estas cosas os digo, hijos míos, porque he leído en los escritos de Enoc que vosotros también os apartaréis del Señor, andando según toda la iniquidad de los gentiles, y haréis según todas las leyes. maldad de Sodoma.

\par 29 Y el Señor traerá cautiverio sobre vosotros, y allí serviréis a vuestros enemigos, y seréis abatidos por toda aflicción y tribulación, hasta que el Señor os consuma a todos.

\par 30 Y cuando os hayáis reducido y reducido, volvéis y reconocéis al Señor vuestro Dios; y él os hará volver a vuestra tierra, según su gran misericordia.

\par 31 Y sucederá que después que lleguen a la tierra de sus padres, nuevamente se olvidarán del Señor y se volverán impíos.

\par 32 Y el Señor los esparcirá sobre la faz de toda la tierra, hasta que venga la compasión del Señor, un hombre que haga justicia y haga misericordia con todos los que están lejos y con los que están cerca.



\chapter{2}

\par \textit{Él hace un llamado a vivir en orden. Destacados por su eterna sabiduría son los versículos 27-40.}

\par 1 Porque en el año cuarenta de mi vida, tuve una visión en el monte de los Olivos, al oriente de Jerusalén, en la que el sol y la luna se detenían.

\par 2 Y he aquí Isaac, el padre de mi padre, nos dijo: Corred y tomadlos, cada uno según sus fuerzas; y al que los apodere le pertenecerán el sol y la luna.

\par 3 Y todos corrimos juntos, y Leví se apoderó del sol, y Judá adelantó a los demás y se apoderó de la luna, y ambos fueron levantados con ellos.

\par 4 Y cuando Leví se volvió como un sol, he aquí un joven le dio doce ramos de palma; y Judá era resplandeciente como la luna, y bajo sus pies había doce rayos.

\par 5 Entonces los dos, Leví y Judá, corrieron y los agarraron.

\par 6 Y he aquí un toro sobre la tierra, con dos grandes cuernos y alas de águila en su lomo; y quisimos prenderle, pero no pudimos.

\par 7 Pero vino José, lo agarró y subió con él a lo alto.

\par 8 Y miré, porque estaba allí, y he aquí se nos apareció una escritura santa que decía: Asirios, medos, persas, caldeos y sirios poseerán en cautiverio a las doce tribus de Israel.

\par 9 Y después de siete días, vi a nuestro padre Jacob de pie junto al mar de Jamnia, y estábamos con él.

\par 10 Y he aquí, llegó un barco que pasaba sin marineros ni piloto; y en la nave estaba escrito: La nave de Jacob.

\par 11 Y nuestro padre nos dijo: Venid, embarquemos en nuestro barco.

\par 12 Y cuando subió a bordo, se levantó una fuerte tempestad y una fuerte tempestad de viento; y nuestro padre, que llevaba el timón, se apartó de nosotros.

\par 13 Y nosotros, azotados por la tempestad, fuimos arrastrados por el mar; y la nave se llenó de agua, y era azotada por grandes olas, hasta romperla.

\par 14 Entonces José huyó en una pequeña barca, y todos estábamos divididos sobre nueve tablas, y Leví y Judá estaban juntos.

\par 15 Y todos fuimos esparcidos hasta los confines de la tierra.

\par 16 Entonces Leví, vestido de cilicio, oró por todos nosotros al Señor.

\par 17 Y cuando la tormenta cesó, el barco llegó a tierra como en paz.

\par 18 Y he aquí que vino nuestro padre y todos nos alegramos unánimemente.

\par 19 Estos dos sueños se los conté a mi padre; y me dijo: Es necesario que estas cosas se cumplan a su tiempo, después que Israel ha soportado muchas cosas.

\par 20 Entonces mi padre me dijo: Creo en Dios que José vive, porque siempre veo que el Señor lo cuenta contigo.

\par 21 Y él dijo llorando: Ah, yo, hijo mío José, tú vives, aunque yo no te miro ni tú ves a Jacob, tu engendrador.

\par 22 Por eso también a mí me hizo llorar con estas palabras, y ardía en mi corazón al declarar que José había sido vendido, pero tenía miedo de mis hermanos.

\par 23 ¡Y he aquí! Hijos míos, os he mostrado en los últimos tiempos cómo sucederá todo en Israel.

\par 24 Por tanto, encargad también vosotros a vuestros hijos que se unan a Leví y a Judá; porque por medio de ellos surgirá la salvación para Israel, y en ellos será bendito Jacob.

\par 25 Porque Dios aparecerá a través de sus tribus, habitando entre los hombres en la tierra, para salvar al linaje de Israel y reunir a los justos de entre las naciones.

\par 26 Hijitos, si hacéis el bien, tanto los hombres como los ángeles os bendecirán; y Dios será glorificado entre los gentiles por medio de vosotros, y el diablo huirá de vosotros, y las fieras os temerán, y el Señor os amará, y los ángeles se unirán a vosotros.

\par 27 Como se recuerda con bondad a un hombre que ha educado bien a un niño; así también de una buena obra hay un buen recuerdo delante de Dios.

\par 28 Pero al que no hace lo bueno, tanto los ángeles como los hombres lo maldecirán, y Dios será deshonrado entre los gentiles por él, y el diablo lo convertirá en su propio instrumento, y toda bestia salvaje lo dominará. , y el Señor lo aborrecerá.

\par 29 Porque los mandamientos de la ley son dobles y es necesario cumplirlos con prudencia.

\par 30 Porque hay un tiempo para que el hombre abrace a su mujer, y un tiempo para abstenerse de ella para orar.

\par 31 Así que hay dos mandamientos; y, a menos que se hagan en el debido orden, acarrean un pecado muy grande sobre los hombres.

\par 32 Lo mismo ocurre con los demás mandamientos.

\par 33 Hijos míos, sed, pues, sabios en Dios y prudentes, comprendiendo el orden de sus mandamientos y las leyes de cada palabra, para que el Señor os ame.

\par 34 Y después de haberles encargado muchas palabras similares, los exhortó a que llevaran sus huesos a Hebrón y lo enterraran con sus padres.

\par 35 Y cuando hubo comido y bebido con corazón alegre, se cubrió el rostro y murió.

\par 36 Y sus hijos hicieron conforme a todo lo que Neftalí su padre les había mandado.



\end{document}