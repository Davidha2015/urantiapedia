\begin{document}

\title{Testamento de Gad}

\chapter{1}

\par \textit{Gad, el noveno hijo de Jacob y Zilpa. Pastor y hombre fuerte pero asesino de corazón. El versículo 25 es una definición notable de odio.}

\par 1 Copia del testamento de Gad, lo que habló a sus hijos en el año ciento veinticinco de su vida, diciéndoles:

\par 2 Escuchen, hijos míos, yo era el noveno hijo de Jacob y era valiente cuidando el rebaño.

\par 3 Por eso yo guardaba el rebaño por la noche; y cuando el león, o el lobo, o cualquier fiera salvaje venía al redil, lo perseguía, y alcanzándolo, le agarré el pie con la mano y lo arrojé a un tiro de piedra, y así lo maté.

\par 4 Mi hermano José estuvo con nosotros apacentando el rebaño durante más de treinta días y, siendo joven, enfermó a causa del calor.

\par 5 Y volvió a Hebrón, donde nuestro padre, quien lo hizo acostarse junto a él, porque lo amaba mucho.

\par 6 Y José le dijo a nuestro padre que los hijos de Zilpa y Bilha estaban matando lo mejor del rebaño y comiéndoselo en contra del juicio de Rubén y Judá.

\par 7 Porque vio que yo había sacado un cordero de la boca de un oso y había matado al oso; sino que había matado el cordero, entristecido porque no podía vivir y porque lo habíamos comido.

\par 8 Y por este asunto estuve enojado con José hasta el día en que fue vendido.

\par 9 Y el espíritu de odio estaba dentro de mí, y no quería oír de José con los oídos ni verlo con los ojos, porque nos reprendía en la cara diciendo que comíamos del rebaño sin Judá.

\par 10 Todo lo que le decía a nuestro padre, él le creía.

\par 11 Confieso ahora mi ginebra, hijos míos, que muchas veces quise matarlo, porque lo odiaba de corazón.

\par 12 Además, lo odié aún más por sus sueños; y quise lamerlo para quitarlo de la tierra de los vivientes, como lame el buey la hierba del campo.

\par 13 Y Judá lo vendió en secreto a los ismaelitas.

\par 14 Así el Dios de nuestros padres lo libró de nuestras manos, para que no cometiéramos grandes desafueros en Israel.

\par 15 Ahora pues, hijos míos, escuchad las palabras de verdad para obrar la justicia y toda la ley del Altísimo, y no os extraviéis por el espíritu de odio, porque es malo en todas las acciones de los hombres.

\par 16 Todo lo que el hombre hace, el que lo odia lo abomina; y el que obra la ley del Señor, no lo alaba; Aunque un hombre teme al Señor y se complace en lo justo, no lo ama.

\par 17 Desprecia la verdad, envidia a los que prosperan, acoge con agrado las malas palabras, ama la soberbia, porque el odio ciega su alma; como yo también entonces miré a José.

\par 18 Por tanto, hijos míos, estad atentos al odio, que produce iniquidad incluso contra el Señor mismo.

\par 19 Porque no escucha las palabras de sus mandamientos acerca del amor al prójimo, y peca contra Dios.

\par 20 Porque si un hermano tropieza, inmediatamente se deleita en anunciarlo a todos y urge que por ello sea juzgado, castigado y condenado a muerte.

\par 21 Y si es un siervo, lo incita contra su amo, y con cada aflicción que planea contra él, si es posible, puede ser condenado a muerte.

\par 22 Porque el odio actúa junto con la envidia también contra los que prosperan: mientras oye o ve su éxito, siempre languidece.

\par 23 Porque así como el amor reviviría incluso a los muertos y llamaría a los condenados a morir, así el odio mataría a los vivos, y a los que habían pecado venialmente no les permitiría vivir.

\par 24 Porque el espíritu de odio coopera con Satanás, por la precipitación de los espíritus, en todo para la muerte de los hombres; pero el espíritu de amor obra junto con la ley de Dios en la paciencia para la salvación de los hombres.

\par 25 Por lo tanto, el odio es malo, porque siempre va de la mano de la mentira y de hablar contra la verdad; y hace grandes las cosas pequeñas, y convierte la luz en tinieblas, y llama amargo a lo dulce, y enseña la calumnia, y enciende la ira, y suscita la guerra, la violencia y toda codicia; llena el corazón de males y veneno diabólico.

\par 26 Por eso, hijos míos, os digo estas cosas por experiencia, para que expulséis el odio que viene del diablo y os acerquéis al amor de Dios.

\par 27 La justicia echa fuera el odio, la humildad destruye la envidia.

\par 28 Porque el justo y humilde se avergüenza de hacer lo injusto, y no es reprendido por otro, sino por su propio corazón, porque el Señor mira su inclinación.

\par 29 No habla contra un hombre santo, porque el temor de Dios vence al odio.

\par 30 Porque, temiendo ofender al Señor, no hará mal a nadie, ni siquiera en el pensamiento.

\par 31 Estas cosas finalmente las aprendí, después de arrepentirme de José.

\par 32 Porque el verdadero arrepentimiento según Dios destruye la ignorancia, disipa las tinieblas, ilumina los ojos, da conocimiento al alma y guía la mente a la salvación.

\par 33 Y lo que no ha aprendido del hombre, lo sabe mediante el arrepentimiento.

\par 34 Porque Dios me trajo una enfermedad del hígado; y si las oraciones de mi padre Jacob no me hubieran socorrido, apenas hubieran fallado, pero mi espíritu se habría apartado.

\par 35 Porque lo que el hombre transgrede también es castigado.

\par 36 Por tanto, puesto que mi hígado fue atacado sin piedad contra José, también en mi hígado sufrí sin piedad y fui juzgado durante once meses, durante todo el tiempo que estuve enojado contra José.

\par \textit{Notas al pie}

\par \textit{254:1 Incluso nuestra jerga actual tiene siglos de antigüedad.}

\chapter{2}

\par \textit{Gad exhorta a sus oyentes contra el odio mostrando cómo éste le ha traído a tantos problemas. Los versículos 8-11 son memorables.}

\par 1 Y ahora, hijos míos, os exhorto a que améis cada uno a su hermano, y apartad de vuestro corazón el odio, que os améis unos a otros de obra, de palabra y de inclinación del alma.

\par 2 Porque en presencia de mi padre hablé pacíficamente a José; y cuando salí, el espíritu de odio oscureció mi mente y agitó mi alma para matarlo.

\par 3 Amaos unos a otros de corazón; y si alguno peca contra ti, háblale pacíficamente, y no guardes engaño en tu alma; y si se arrepiente y confiesa, perdónalo.

\par 4 Pero si él lo niega, no te enojes con él, no sea que, tomando el veneno de ti, se ponga a jurar y peques doblemente.

\par 5 No permitas que otro hombre escuche tus secretos cuando esté involucrado en una disputa legal, no sea que llegue a odiarte y se convierta en tu enemigo y cometa un gran pecado contra ti; porque muchas veces se dirige a ti con astucia o se ocupa de ti con malas intenciones.

\par 6 Y aunque lo niegue y se avergüence al ser reprendido, deja de reprenderlo.

\par 7 Quien niegue, puede arrepentirse para no volver a ofenderte; sí, también podrá honrarte, temerte y estar en paz contigo.

\par 8 Y si es desvergonzado y persiste en su maldad, perdónalo de corazón y deja a Dios la venganza.

\par 9 Si alguien prospera más que tú, no te enojes, sino ora también por él, para que tenga perfecta prosperidad.

\par 10 por lo que te convenga.

\par 11 Y si se enaltece aún más, no le envidiéis, recordando que toda carne morirá; y alabad a Dios, que da cosas buenas y provechosas a todos los hombres.

\par 12 Busca los juicios del Señor y tu mente descansará y estará en paz.

\par 13 Y aunque un hombre se enriquezca por medios malvados, como Esaú, el hermano de mi padre, no tengas celos; pero esperad el fin del Señor.

\par 14 Porque si le quita a un hombre riquezas obtenidas por medios malos, si se arrepiente le perdona, pero al que no se arrepiente le reserva el castigo eterno.

\par 15 Porque el pobre, si libre de envidia agrada al Señor en todo, es bendito más que todos los hombres, porque no sufre las fatigas de los hombres vanos.

\par 16 Apartad, pues, de vuestras almas los celos, y amaos unos a otros con rectitud de corazón.

\par 17 Hablad, pues, también vosotros estas cosas a vuestros hijos, para que honren a Judá y a Leví, porque de ellos el Señor levantará la salvación a Israel.

\par 18 Porque sé que al final tus hijos se apartarán de Él y andarán en maldad, aflicción y corrupción delante del Señor.

\par 19 Y cuando hubo descansado un poco, volvió a decir: Hijos míos, obedeced a vuestro padre y sepultadme junto a mis padres.

\par 20 Entonces levantó los pies y se durmió en paz.

\par 21 Cinco años después lo llevaron a Hebrón y lo pusieron con sus padres.



\end{document}