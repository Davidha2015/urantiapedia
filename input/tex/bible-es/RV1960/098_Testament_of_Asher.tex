\begin{document}

\title{Testamento de Aser}

\chapter{1}

\par \textit{Aser, el décimo hijo de Jacob y Zilpa. Una explicación de la doble personalidad. La primera historia de Jekyll y Hyde. Para una declaración de la Ley de Compensación que Emerson habría disfrutado, vea el versículo 27.}

\par 1 LA copia del Testamento A Aser, lo que habló a sus hijos en el año ciento veinticinco de su vida.

\par 2 Porque, cuando aún estaba sano, les dijo: Hijos de Aser, escuchad a vuestro padre, y os declararé todo lo recto ante los ojos del Señor.

\par 3 Dos caminos ha dado Dios a los hijos de los hombres, dos inclinaciones, dos modos de actuar, dos modos de actuar y dos resultados.

\par 4 Por tanto, todas las cosas son de dos en dos, uno frente al otro.

\par 5 Porque hay dos caminos, el del bien y el del mal, y con éstos están las dos inclinaciones en nuestro pecho que los distinguen.

\par 6 Por lo tanto, si el alma se complace en la buena inclinación, todas sus acciones serán rectas; y si peca, al instante se arrepiente.

\par 7 Porque, puesto sus pensamientos en la justicia y desechando la maldad, inmediatamente derriba el mal y desarraiga el pecado.

\par 8 Pero si se inclina hacia el mal, todas sus acciones serán malas, y rechazará el bien, se unirá al mal y se regirá por Beliar; aunque hace el bien, lo pervierte para el mal.

\par 9 Porque cuando comienza a hacer el bien, fuerza el resultado de la acción hacia el mal, ya que el tesoro de la inclinación está lleno de un espíritu maligno.

\par 10 Uno puede, pues, ayudar con palabras al bien a cambio del mal, pero el resultado de la acción conduce al mal.

\par 11 Hay hombre que no tiene compasión del que le sirve en el mal; y esto tiene dos aspectos, pero el conjunto es malo.

\par 12 Y hay quien ama al que hace el mal, porque preferiría incluso morir en el mal por él; y respecto a esto es claro que tiene dos aspectos, pero el conjunto es una obra mala.

\par 13 Aunque tenga amor, es malvado el que oculta el mal por causa del buen nombre, pero el fin de la acción tiende al mal.

\par 14 Otro roba, hace injusticia, saquea, defrauda y además se compadece de los pobres: esto también tiene dos aspectos, pero el conjunto es malo.

\par 15 El que defrauda a su prójimo irrita a Dios, jura falsamente contra el Altísimo y, sin embargo, se compadece de los pobres; el Señor, que impuso la ley, desprecia y irrita, y sin embargo refresca a los pobres.

\par 16 Él contamina el alma y alegra el cuerpo; Mata a muchos y se compadece de unos pocos: esto también tiene un doble aspecto, pero el conjunto es malo.

\par 17 Otro comete adulterio y fornicación, se abstiene de comer alimentos, y cuando ayuna hace el mal, y con el poder de sus riquezas abruma a muchos; y a pesar de su excesiva maldad, cumple los mandamientos: esto también tiene un doble aspecto, pero el conjunto es malo.

\par 18 Estos hombres son liebres; limpios, como los que tienen la pezuña dividida, pero en realidad son inmundos.

\par 19 Porque así lo ha declarado Dios en las tablas de los mandamientos.

\par 20 Pero vosotros, hijos míos, no llevéis dos rostros como ellos, el de bondad y el de maldad; pero aférrate únicamente al bien, porque Dios tiene allí su morada y los hombres la desean.

\par 21 Pero huye de la maldad, destruyendo con tus buenas obras la inclinación al mal; porque los de doble cara no sirven a Dios, sino a sus propias concupiscencias, para agradar a Beliar y a hombres como ellos.

\par 22 Porque los hombres buenos, incluso los de rostro sencillo, aunque sean pensados ​​por los que tienen doble rostro para pecar, son justos ante Dios.

\par 23 Porque muchos, al matar a los malvados, hacen dos obras, una buena y otra mala; pero el conjunto es bueno, porque él desarraigó y destruyó lo malo.

\par 24 Un hombre odia al misericordioso e injusto, y al que comete adulterio y ayuna; esto también tiene un doble aspecto, pero toda la obra es buena, porque sigue el ejemplo del Señor, al no aceptar el parecer bueno como el bien genuino.

\par 25 Otro no desea ver el buen día con los que no, para no contaminar su cuerpo y contaminar su alma; Esto también tiene dos caras, pero el conjunto es bueno.

\par 26 Porque tales hombres son semejantes a los ciervos y a las ciervas, porque a la manera de los animales salvajes parecen inmundos, pero todos están limpios; porque caminan con celo por el Señor y se abstienen de lo que Dios también aborrece y prohíbe por sus mandamientos, alejando el mal del bien.

\par 27 Ya veis, hijos míos, que en todas las cosas hay dos, uno contra otro, y el uno escondido detrás del otro: en la riqueza se esconde la codicia, en la convivencia la embriaguez, en la risa la tristeza, en el libertinaje matrimonial.

\par 28 La muerte sucede a la vida, la deshonra a la gloria, la noche al día y las tinieblas a la luz; y todas las cosas son bajo el día, lo justo bajo la vida, lo injusto bajo la muerte; Por eso también la vida eterna espera la muerte.

\par 29 Ni se puede decir que la verdad sea mentira, ni que lo correcto sea incorrecto; porque toda verdad está bajo la luz, así como todas las cosas están bajo Dios.

\par 30 Por lo tanto, probé todas estas cosas en mi vida, y no me desvié de la verdad del Señor, y busqué los mandamientos del Altísimo, caminando con todas mis fuerzas y con sencillez de rostro hacia aquello que es bueno.

\par 31 Por tanto, hijos míos, guardad también vosotros los mandamientos del Señor, siguiendo la verdad con sencillez de rostro.

\par 32 Porque los que tienen doble cara cometen doble pecado; porque ambos hacen el mal y se complacen en los que lo hacen, siguiendo el ejemplo de los espíritus del engaño y luchando contra la humanidad.

\par 33 Por tanto, hijos míos, guardad la ley del Señor y no prestéis atención al mal como al bien; sino mira lo que es realmente bueno, y guárdalo en todos los mandamientos del Señor, teniendo en ello tu conversación y descansando en ello.

\par 34 Porque los últimos fines de los hombres muestran su justicia o su injusticia cuando se encuentran con los ángeles del Señor y de Satanás.

\par 35 Porque cuando el alma parte atribulada, es atormentada por el espíritu maligno, al cual también sirvió en las concupiscencias y en las malas obras.

\par 36 Pero si está en paz y gozo, encontrará al ángel de la paz y lo conducirá a la vida eterna.

\par 37 No os volváis, hijos míos, como Sodoma, que pecó contra los ángeles del Señor y pereció para siempre.

\par 38 Porque sé que pecaréis y seréis entregados en manos de vuestros enemigos; y vuestra tierra será desolada, y vuestros lugares santos destruidos, y seréis esparcidos por los cuatro confines de la tierra.

\par 39 Y seréis destruidos en la dispersión que se desvanecerá como el agua.

\par 40 Hasta que el Altísimo visite la tierra, viniendo en forma de hombre, con los hombres comiendo y bebiendo, y rompiendo la cabeza del dragón en el agua.

\par 41 Él salvará a Israel y a todos los gentiles, hablando Dios en persona del hombre.

\par 42 Por tanto, hijos míos, decid también vosotros estas cosas a vuestros hijos, para que no le desobedezcan.

\par 43 Porque sé que ciertamente seréis desobedientes y ciertamente actuaréis impíamente, no atendiendo a la ley de Dios, sino a los mandamientos de los hombres, corrompiéndoos por la maldad.

\par 44 Por eso seréis dispersos como mis hermanos Gad y Dan, y no conoceréis vuestras tierras, vuestra tribu y vuestra lengua.

\par 45 Pero el Señor os reunirá en la fe por su tierna misericordia y por amor a Abraham, Isaac y Jacob.

\par 46 Y cuando les dijo estas cosas, les mandó diciendo: Entiérrenme en Hebrón.

\par 47 Y durmió y murió en buena vejez.

\par 48 Sus hijos hicieron lo que él les había ordenado y lo llevaron a Hebrón y lo sepultaron con sus padres.

\end{document}