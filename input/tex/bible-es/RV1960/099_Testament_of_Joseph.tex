\begin{document}

\title{Testamento de José}

\chapter{1}

\par \textit{José, el undécimo hijo de Jacob y Raquel, la hermosa y amada. Su lucha contra la tentadora egipcia.}

\par 1 LA copia del Testamento de José.

\par 2 Cuando estaba a punto de morir, reunió a sus hijos y a sus hermanos y les dijo:

\par 3 Hermanos míos e hijos míos, escuchad a José, el amado de Israel; escuchad, hijos míos, a vuestro padre.

\par 4 He visto en mi vida envidia y muerte, pero no me extravié, sino que perseveré en la verdad del Señor.

\par 5 Estos mis hermanos me odiaban, pero el Señor me amaba:

\par 6 Querían matarme, pero el Dios de mis padres me protegió.

\par 7 Me arrojaron a un hoyo, y el Altísimo me hizo subir.

\par 8 Fui vendido como esclavo, y el Señor de todos me liberó:

\par 9 Fui llevado cautivo, y su mano fuerte me socorrió.

\par 10 Tuve hambre y el Señor mismo me alimentó.

\par 11 Estaba solo y Dios me consoló:

\par 12 Estuve enfermo y el Señor me visitó.

\par 13 Yo estaba en prisión, y mi Dios me mostró favor;

\par 14 En cadenas, y Él me liberó;

\par 15 Calumniado, y defendió mi causa;

\par 16 Los egipcios me criticaron duramente y él me libró;

\par 17 Envidiado por mis compañeros de esclavitud, Él me exaltó.

\par 18 Y este capitán de Faraón me confió su casa.

\par 19 Y luché contra una mujer desvergonzada, instándome a transgredir con ella; pero el Dios de Israel mi padre me libró de la llama ardiente.

\par 20 Fui encarcelado, golpeado y escarnecido; pero el Señor me concedió encontrar misericordia, ante los ojos del guardián de la prisión.

\par 21 Porque el Señor no abandona a los que le temen, ni en tinieblas, ni en prisiones, ni en tribulaciones, ni en necesidades.

\par 22 Porque Dios no se avergüenza como un hombre, ni tiene miedo como un hijo de hombre, ni se debilita ni se asusta como un terrenal.

\par 23 Pero en todo protege y de diversas maneras consuela, aunque por un breve espacio de tiempo se aleja para probar las inclinaciones del alma.

\par 24 En diez tentaciones me mostró aprobado, y en todas las soporté; porque la paciencia es un gran encanto, y la paciencia produce muchos bienes.

\par 25 ¡Cuántas veces me amenazó de muerte la egipcia!

\par 26 ¡Cuántas veces me entregó al castigo, luego me llamó y me amenazó, y cuando yo no quise acompañarla, me dijo:

\par 27 Tú serás señor de mí y de todo lo que hay en mi casa, si te entregas a mí y eres como nuestro señor.

\par 28 Pero me acordé de las palabras de mi padre y, entrando en mi aposento, lloré y oré al Señor.

\par 29 Y ayuné durante esos siete años, y a los egipcios me apareció como alguien que vive con delicadeza, porque aquellos que ayunan por amor de Dios obtienen belleza de rostro.

\par 30 Y cuando mi señor estaba fuera de casa, no bebía vino; ni durante tres días tomé mi comida, sino que la di a los pobres y a los enfermos.

\par 31 Desde temprano busqué al Señor y lloré por la egipcia de Menfis, que continuamente me molestaba y que incluso de noche venía a verme con el pretexto de visitarme.

\par 32 Y como no tenía ningún hijo varón, fingió considerarme como a un hijo.

\par 33 Y por un tiempo ella me abrazó como a un hijo, y yo no lo supe; pero más tarde trató de arrastrarme a la fornicación.

\par 34 Y cuando lo supe, me entristecí hasta la muerte; y cuando ella salió, volví en mí, y lamenté por ella muchos días, porque reconocía su astucia y su engaño.

\par 35 Y le comuniqué las palabras del Altísimo, por si acaso se apartaba de su mala concupiscencia.

\par 36 Por eso, muchas veces me halagó con palabras como a un hombre santo, y con engaños en sus palabras elogió mi castidad ante su marido, mientras deseaba atraparme cuando estábamos solos.

\par 37 Porque ella abiertamente me alababa como casta, y en secreto me decía: No temas a mi marido; porque está convencido acerca de tu castidad; porque incluso si alguien le dijera acerca de nosotros, no creería.

\par 38 A causa de todo esto, me tumbé en el suelo y rogué a Dios que el Señor me librara de su engaño.

\par 39 Y como no pudo lograr nada con ello, volvió a mí con el ruego de que la instruyera para aprender la palabra de Dios.

\par 40 Y ella me dijo: Si quieres que deje mis ídolos, acuéstate conmigo, y persuadiré a mi marido a que se aparte de sus ídolos, y caminaremos en la ley junto a tu Señor.

\par 41 Y yo le dije: El Señor no quiere. para que los que le temen estén en inmundicia, ni se complace en los que cometen adulterio, sino en los que se acercan a él con corazón puro y labios sin mancha.

\par 42 Pero ella presta atención a su paz, anhelando cumplir su mal deseo.

\par 43 Y me entregué aún más al ayuno y a la oración, para que el Señor me librara de ella.

\par 44 Y otra vez me dijo otra vez: Si no cometes adulterio, mataré a mi marido con veneno; y tomarte por marido.

\par 45 Entonces yo, cuando oí esto, rasgué mis vestidos y le dije:

\par 46 Mujer, reverencia a Dios y no hagas esta mala acción, no sea que seas destruida; porque sabed en verdad que declararé este vuestro designio a todos los hombres.

\par 47 Entonces ella, teniendo miedo, me rogó que no declarara este dispositivo.

\par 48 Y ella se fue, alentándome con regalos y enviándome todos los deleites de los hijos de los hombres.

\par 49 Y después me envió comida mezclada con encantamientos.

\par 50 Y cuando llegó el eunuco que lo traía, miré hacia arriba y vi a un hombre terrible que me daba una espada con el plato, y comprendí que su plan era engañarme.

\par 51 Y cuando él salió, lloré y no probé ni aquel ni ningún otro alimento suyo.

\par 52 Un día después, ella vino a mí, observó la comida y me dijo: ¿Por qué no has comido de la comida?

\par 53 Y le dije: Es porque la has llenado de encantamientos mortales; ¿Y cómo dijiste: No me acerco a los ídolos sino sólo al Señor?

\par 54 Ahora pues, debes saber que el Dios de mi padre me ha revelado por su ángel tu maldad, y yo la he guardado para convencerte, si es que puedes ver y arrepentirte.

\par 55 Pero para que aprendas que la maldad de los impíos no tiene poder sobre aquellos que adoran a Dios con castidad, he aquí, yo tomaré de ello y comeré delante de ti.

\par 56 Y habiendo dicho esto, oré así: El Dios de mis padres y el ángel de Abraham, estén conmigo; Y comí.

\par 57 Y cuando vio esto, cayó de bruces a mis pies, llorando; y la levanté y la amonesté.

\par 58 Y ella prometió no hacer más esta iniquidad.

\par 59 Pero su corazón todavía estaba obsesionado con el mal, y buscaba a su alrededor cómo tenderme una trampa, y suspirando profundamente se abatió, aunque no estaba enferma.

\par 60 Y cuando su marido la vio, le dijo: ¿Por qué está decaído tu rostro?

\par 61 Y ella le dijo: Tengo dolor en el corazón, y los gemidos de mi espíritu me oprimen; y así consoló a la que no estaba enferma.

\par 62 Entonces, aprovechando la oportunidad, corrió hacia mí mientras su marido aún estaba fuera y me dijo: Me ahorcaré o me arrojaré por un acantilado si tú no te acuestas conmigo.

\par 63 Y cuando vi que el espíritu de Beliar la atormentaba, oré al Señor y le dije:

\par 64 ¿Por qué, desgraciada, estás turbada y perturbada, ciega por los pecados?

\par 65 Recuerda que si te matas, Asteho, la concubina de tu marido, tu rival, golpeará a tus hijos y destruirás tu monumento de la tierra.

\par 66 Y ella me dijo: He aquí, entonces me amas; que esto me baste: sólo esforzarme por mi vida y por mis hijos, y espero disfrutar también de mi deseo.

\par 67 Pero ella no sabía que yo hablaba así por mi señor, y no por ella.

\par 68 Porque si un hombre ha caído ante la pasión de un deseo perverso y se ha vuelto esclavo de él, al igual que ella, cualquier cosa buena que oiga acerca de esa pasión, la recibirá con miras a su deseo perverso.

\par 69 Por tanto, os declaro, hijos míos, que era alrededor de la hora sexta cuando ella se apartó de mí; y estuve arrodillado delante del Señor todo el día y toda la noche; y al amanecer me levanté, llorando mientras oraba por su liberación.

\par 70 Finalmente, ella se apoderó de mis vestidos y me arrastró por la fuerza para tener conexión con ella.

\par 71 Cuando vi que en su locura se aferraba a mi vestido, lo dejé atrás y huí desnudo.

\par 72 Y ella, aferrándose al manto, me acusó falsamente, y cuando llegó su marido, me metió en la cárcel de su casa; y al día siguiente me azotó y me envió a la cárcel de Faraón.

\par 73 Y cuando yo estaba encarcelado, la mujer egipcia se sintió oprimida por el dolor, y vino y escuchó cómo yo daba gracias al Señor y cantaba alabanzas en la morada de las tinieblas, y con voz alegre se regocijaba, glorificando a mi Dios por haberme fue librado del deseo lujurioso de la mujer egipcia.

\par 74 Y muchas veces me ha enviado diciendo: Consiente en cumplir mi deseo, y te liberaré de tus ataduras, y te liberaré de las tinieblas.

\par 75 Y ni siquiera en pensamiento me incliné hacia ella.

\par 76 Porque Dios ama más a aquel que en un antro de maldad combina el ayuno con la castidad, que al hombre que en los aposentos de los reyes combina el lujo con la licencia.

\par 77 Y si un hombre vive en castidad y desea también la gloria, y el Altísimo sabe que le conviene, también a mí me la concede.

\par 78 ¡Cuántas veces, estando enferma, venía a mí en momentos inesperados y escuchaba mi voz mientras yo oraba!

\par 79 Y cuando oí sus gemidos, callé.

\par 80 Porque cuando yo estaba en su casa, ella solía desnudarse los brazos, los pechos y las piernas para que yo pudiera acostarme con ella; porque era muy hermosa, espléndidamente adornada para seducirme.

\par 81 Y el Señor me guardó de sus maquinaciones.

\chapter{2}

\par \textit{José es víctima de muchos complots del malvado ingenio de la mujer memphiana. Para una parábola profética interesante, vea los versículos 73-74.}

\par 1 Ved, pues, hijos míos, cuán grandes cosas obra la paciencia, y la oración con el ayuno.

\par 2 Así también vosotros, si seguís la castidad y la pureza con paciencia y oración, con ayuno y humildad de corazón, el Señor habitará entre vosotros porque ama la castidad.

\par 3 Y dondequiera que habita el Altísimo, aunque sobrevenga a un hombre la envidia, la esclavitud o la calumnia, el Señor que habita en él, por su castidad, no sólo lo libra del mal, sino que también lo exalta como a mí.

\par 4 Porque en todo se eleva el hombre, ya sea en las obras, en las palabras o en los pensamientos.

\par 5 Mis hermanos sabían cuánto me amaba mi padre, pero yo no me exaltaba en mi mente: aunque era niño, tenía el temor de Dios en mi corazón; porque sabía que todas las cosas pasarían.

\par 6 Y no me levanté contra ellos con malas intenciones, sino que honré a mis hermanos; y por respeto a ellos, incluso cuando me vendieron, me abstuve de decirles a los ismaelitas que yo era hijo de Jacob, un gran hombre y valiente.

\par 7 Hijos míos, vosotros también tened delante de vuestros ojos el temor de Dios en todas vuestras obras y honrad a vuestros hermanos.

\par 8 Porque todo el que cumple la ley del Señor será amado por Él.

\par 9 Y cuando llegué a Indocolpitae con los ismaelitas, me preguntaron, diciendo:

\par 10 ¿Eres tú un esclavo? Y dije que era un esclavo nativo, para no avergonzar a mis hermanos.

\par 11 Y el mayor de ellos me dijo: No eres un esclavo, porque incluso tu apariencia lo hace manifiesto.

\par 12 Pero yo dije que era su esclavo.

\par 13 Cuando llegamos a Egipto, discutieron acerca de mí, quién de ellos debería comprarme y tomarme.

\par 14 Por lo tanto, a todos les pareció bien que yo me quedara en Egipto con los mercaderes de su comercio, hasta que regresaran trayendo mercancías.

\par 15 Y el Señor me dio gracia ante los ojos del mercader, y me confió su casa.

\par 16 Y Dios lo bendijo por medio de mí y le aumentó en oro y plata y en sirvientes.

\par 17 Y estuve con él tres meses y cinco días.

\par 18 En aquel tiempo la mujer de Menfis, esposa de Pentefris, descendió en un carro con gran pompa, porque había oído hablar de mí por parte de sus eunucos.

\par 19 Y ella le dijo a su marido que el mercader se había enriquecido gracias a un joven hebreo, y dicen que seguramente había sido robado de la tierra de Canaán.

\par 20 Ahora, pues, hazle justicia y lleva al joven a tu casa; Así te bendecirá el Dios de los hebreos, porque la gracia del cielo está sobre él.

\par 21 Pentefris, persuadida por sus palabras, mandó traer al mercader y le dijo:

\par 22 ¿Qué es esto que oigo acerca de ti, que robas personas de la tierra de Canaán y las vendes como esclavos?

\par 23 Pero el mercader se postró a sus pies y le suplicó, diciendo: Te lo ruego, señor mío, no sé lo que dices.

\par 24 Y Pentefris le dijo: ¿De dónde, pues, es el esclavo hebreo?

\par 25 Y él dijo: Los ismaelitas me lo confiaron hasta que regresaran.

\par 26 Pero él no le creyó, sino que mandó que lo desnudaran y lo golpearan.

\par 27 Y como él insistía en esta afirmación, Pentefris dijo: Que traigan al joven.

\par 28 Y cuando me trajeron, rindí homenaje a Pentefris, porque él era el tercero en rango entre los oficiales de Faraón.

\par 29 Entonces me separó de él y me dijo: ¿Eres esclavo o eres libre?

\par 30 Y yo dije: Un esclavo.

\par 31 Y él dijo: ¿De quién?

\par 32 Y dije: Los ismaelitas.

\par 33 Y él dijo: ¿Cómo llegaste a ser su esclavo?

\par 34 Y dije: Me compraron de la tierra de Canaán.

\par 35 Y él me dijo: En verdad mientes; y luego mandó que me desnudaran y me golpearan.

\par 36 Mientras me golpeaban, la mujer de Memphis estaba mirándome por una ventana, porque su casa estaba cerca, y le envió a decir:

\par 37 Tu juicio es injusto; porque castigas al hombre libre que ha sido robado, como si fuera un transgresor.

\par 38 Y como no hice ningún cambio en mi declaración, a pesar de que me golpearon, ordenó que me encarcelaran, hasta que, dijo, vinieran los dueños del niño.

\par 39 Y la mujer dijo a su marido: ¿Por qué detienes en prisiones a este joven cautivo y de buena familia, que más bien debería ser puesto en libertad y atendido?

\par 40 Porque ella quería verme por deseo de pecado, pero yo ignoraba todas estas cosas.

\par 41 Y él le dijo: No es costumbre de los egipcios tomar lo que es ajeno antes de haber sido probado.

\par 42 Esto, pues, dijo del mercader: pero el muchacho debe ser encarcelado.

\par 43 Veinticuatro días después llegaron los ismaelitas; porque habían oído que mi padre Jacob estaba muy triste por mí.

\par 44 Y vinieron y me dijeron: ¿Cómo es que dices que eres esclavo? y he aquí, hemos sabido que eres hijo de un valiente en la tierra de Canaán, y que tu padre todavía llora por ti en cilicio y ceniza.

\par 45 Cuando oí esto, mis entrañas se deshicieron y mi corazón se derritió, y tuve grandes deseos de llorar, pero me contuve para no avergonzar a mis hermanos.

\par 46 Y les dije: No lo sé, soy un esclavo.

\par 47 Entonces decidieron venderme para que no me encontraran en sus manos.

\par 48 Porque temían que mi padre viniera y ejecutara sobre ellos una venganza terrible.

\par 49 Porque habían oído que él era poderoso ante Dios y ante los hombres.

\par 50 Entonces el mercader les dijo: Libérame del juicio de Pentifri.

\par 51 Y vinieron y me rogaron, diciendo: Di que fuiste comprado por nosotros con dinero, y él nos liberará.

\par 52 La mujer de Memphis dijo a su marido: Compra al joven; porque he oído, dijo ella, que lo están vendiendo.

\par 53 E inmediatamente envió un eunuco a los ismaelitas y les pidió que me vendieran.

\par 54 Pero como el eunuco no quiso comprarme a su precio, volvió, después de probarlas, y le hizo saber a su señora que pedían un alto precio por su esclavo.

\par 55 Y ella envió a otro eunuco, diciendo: Aunque te pidan dos minas, dales, no ahorres el oro; Sólo compra al niño y tráemelo.

\par 56 Entonces el eunuco fue y les dio ochenta piezas de oro, y él me recibió; pero a la egipcia dijo: Le he dado cien.

\par 57 Y aunque lo sabía, me callé, para que el eunuco no quedara avergonzado.

\par 58 Ya veis, hijos míos, cuán grandes cosas soporté para no avergonzar a mis hermanos.

\par 59 Por tanto, también vosotros amaos unos a otros y ocultad con paciencia los unos de los otros las faltas.

\par 60 Porque Dios se deleita en la unidad de los hermanos y en la determinación de un corazón que se deleita en el amor.

\par 61 Y cuando mis hermanos llegaron a Egipto, se enteraron de que yo les había devuelto su dinero, y no los reprendí, sino que los consolé.

\par 62 Y después de la muerte de mi padre Jacob, los amé más y todo lo que él me ordenó lo hice con creces.

\par 63 Y no permití que fueran afligidos en lo más mínimo; y todo lo que estaba en mi mano se lo di.

\par 64 Y sus hijos fueron mis hijos, y mis hijos como sus siervos; y su vida era mi vida, y todo su sufrimiento era mi sufrimiento, y toda su enfermedad era mi debilidad.

\par 65 Mi tierra era la tierra de ellos, y sus consejos mi consejo.

\par 66 Y no me exalté entre ellos con arrogancia a causa de mi gloria mundana, sino que fui entre ellos como uno de los más pequeños.

\par 67 Por lo tanto, si también vosotros andáis en los mandamientos del Señor, hijos míos, Él os exaltará allí y os bendecirá con bienes por los siglos de los siglos.

\par 68 Y si alguno procura haceros mal, hacedle bien y orad por él, y seréis redimidos por el Señor de todo mal.

\par 69 Porque he aquí, veis que por mi humildad y paciencia tomé por esposa a la hija del sacerdote de Heliópolis.

\par 70 Y con ella me dieron cien talentos de oro, y el Señor los puso a mi servicio.

\par 71 Y también me dio una belleza como una flor mayor que las hermosas de Israel; y me preservó hasta la vejez en fuerza y ​​hermosura, porque era en todo semejante a Jacob.

\par 72 Y oíd, hijos míos, también la visión que tuve.

\par 73 Había doce ciervos paciendo; y los nueve se dispersaron primero por toda la tierra, y también los tres.

\par 74 Y vi que de Judá nació una virgen vestida de lino, y de ella nació un cordero sin mancha; ya su mano izquierda había como un león; y todos los animales se abalanzaron contra él, y el cordero los venció, los destruyó y los pisoteó.

\par 75 Y por él se regocijaron los ángeles y los hombres, y toda la tierra.

\par 76 Y estas cosas sucederán en su tiempo, en los últimos días.

\par 77 Por tanto, hijos míos, guardad los mandamientos del Señor y honrad a Leví y a Judá; porque de ellos os surgirá el Cordero de Dios, que quita el pecado del mundo, el que salva a todas las naciones y a Israel.

\par 78 Porque su reino es un reino eterno, que no pasará; pero mi reino entre vosotros llegará a su fin como la hamaca de un vigilante, que después del verano desaparece.

\par 79 Porque sé que después de mi muerte los egipcios os afligirán, pero Dios os vengará y os hará cumplir lo que prometió a vuestros padres.

\par 80 Pero llevaréis mis huesos con vosotros; porque cuando mis huesos sean llevados allí, el Señor estará con vosotros en la luz, y Beliar estará en tinieblas con los egipcios.

\par 81 Y llevad a vuestra madre Asenat al hipódromo, y enterradla cerca de Raquel, vuestra madre.

\par 82 Y habiendo dicho estas cosas, extendió los pies y murió en buena vejez.

\par 83 Y todo Israel y todo Egipto hicieron duelo por él con gran duelo.

\par 84 Cuando los hijos de Israel salieron de Egipto, se llevaron los huesos de José y lo sepultaron en Hebrón con sus padres. Los años de su vida fueron ciento diez años.





\end{document}