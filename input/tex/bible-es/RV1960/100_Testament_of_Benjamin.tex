\begin{document}

\title{Testamento de Benjamín}

\chapter{1}

\par \textit{Benjamín, el duodécimo hijo de Jacob y Raquel, la bebé de la familia, se convierte en filósofo y filántropo.}

\par 1 Copia de las palabras que Benjamín ordenó que cumplieran sus hijos, después de haber vivido ciento veinticinco años.

\par 2 Y los besó y dijo: Como Isaac le nació a Abraham en su vejez, así también yo a Jacob.

\par 3 Y desde que mi madre Raquel murió al darme a luz, no tuve leche; Por eso fui amamantado por Bilha su sierva.

\par 4 Porque Raquel permaneció estéril doce años después de dar a luz a José; y oró al Señor con ayuno de doce días, y concibió y me dio a luz.

\par 5 Porque mi padre amaba mucho a Raquel y oraba para poder ver nacer de ella dos hijos.

\par 6 Por eso me llamaron Benjamín, es decir, hijo de días.

\par 7 Y cuando fui a Egipto, donde José, y mi hermano me reconoció, me dijo: ¿Qué le dijeron a mi padre cuando me vendieron?

\par 8 Y yo le dije: Te mancharon la túnica con sangre y la enviaron, diciendo: «Sabe si ésta es la túnica de tu hijo».

\par 9 Y él me dijo: Así también, hermano, cuando me quitaron la túnica me entregaron a los ismaelitas, y me dieron un taparrabos, me azotaron y me hicieron correr.

\par 10 Y a uno de los que me habían golpeado con una vara, un león le salió al encuentro y lo mató.

\par 11 Y sus compañeros se asustaron.

\par 12 Por tanto, hijos míos, también vosotros amad al Señor, Dios del cielo y de la tierra, y guardad sus mandamientos, siguiendo el ejemplo del bueno y santo varón José.

\par 13 Y vuestra mente sea buena, tal como me conocéis; porque el que baña bien su mente ve todas las cosas correctamente.

\par 14 Temed al Señor y amad a vuestro prójimo; y aunque los espíritus de Beliar pretendan que os aflijan con todo mal, no tendrán dominio sobre vosotros, como no lo tuvieron sobre José mi hermano.

\par 15 ¡Cuántos hombres quisieron matarlo y Dios lo protegió!

\par 16 Porque el que teme a Dios y ama a su prójimo no puede ser herido por el espíritu de Beliar, estando protegido por el temor de Dios.

\par 17 Tampoco puede ser dominado por las artimañas de los hombres ni de las bestias, sino que el Señor le ayuda mediante el amor que tiene hacia su prójimo.

\par 18 Porque José también rogó a nuestro padre que orara por sus hermanos, para que el Señor no les imputara como pecado el mal que le habían hecho.

\par 19 Entonces Jacob gritó: «Hija mía, has vencido las entrañas de tu padre Jacob».

\par 20 Y él lo abrazó y lo besó durante dos horas, diciendo:

\par 21 En ti se cumplirá la profecía del cielo acerca del Cordero de Dios, y Salvador del mundo, de que uno sin mancha será entregado por los impíos, y uno sin pecado morirá por los impíos en la sangre de el pacto, para la salvación de los gentiles y de Israel, y destruirá a Beliar y a sus siervos.

\par 22 ¿Veis, pues, hijos míos, el fin del hombre bueno?

\par 23 Por tanto, sed seguidores de su compasión con buena intención, para que también vosotros podáis llevar coronas de gloria.

\par 24 Porque el hombre bueno no tiene ojos oscuros; porque muestra misericordia a todos los hombres, aunque sean pecadores.

\par 25 Y aunque inventen malas intenciones en cuanto a él, haciendo el bien vence el mal, estando protegido por Dios; y ama al justo como a su propia alma.

\par 26 Si alguien es glorificado, no le tiene envidia; si alguno se enriquece, no tiene celos; si alguno es valiente, lo alaba; al hombre virtuoso lo alaba; del pobre tiene misericordia; de los débiles tiene compasión; a Dios canta alabanzas.

\par 27 Y el que tiene la gracia de un buen espíritu, lo ama como a su propia alma.

\par 28 Si, pues, también vosotros tenéis buena intención, ambos impíos estarán en paz con vosotros, y los libertinos os reverenciarán y se volverán al bien; y los codiciosos no sólo cesarán de su deseo desmesurado, sino que incluso darán los objetos de su codicia a los afligidos.

\par 29 Si hacéis el bien, hasta los espíritus inmundos huirán de vosotros; y las bestias te temerán.

\par 30 Porque donde hay reverencia por las buenas obras y luz en la mente, incluso las tinieblas huyen de él.

\par 31 Porque si alguno hace violencia a un hombre santo, se arrepiente; porque el santo varón es misericordioso con el que lo maldice, y calla.

\par 32 Y si alguno traiciona a un justo, el justo ora; aunque por un momento sea humillado, no mucho después aparecerá mucho más glorioso, como lo fue José mi hermano.

\par 33 La inclinación del hombre bueno no está en poder del engaño del espíritu de Beliar, porque el ángel de la paz guía su alma.

\par 34 Y no mira apasionadamente las cosas corruptibles, ni acumula riquezas por el deseo de deleite.

\par 35 No se deleita en los placeres, no entristece a su prójimo, no se sacia de lujos, no se equivoca al alzar los ojos, porque el Señor es su porción.

\par 36 La buena inclinación no recibe gloria ni deshonra de los hombres, y no conoce engaños, ni mentiras, ni peleas, ni injurias; porque el Señor habita en él e ilumina su alma, y ​​se regocija para con todos los hombres siempre.

\par 37 La mente buena no tiene dos lenguas, de bendición y de maldición, de vergüenza y de honor, de tristeza y de alegría, de quietud y de confusión, de hipocresía y de verdad, de pobreza y de riqueza; pero tiene una sola disposición, incorrupta y pura, respecto de todos los hombres.

\par 38 No tiene doble vista ni doble oído; porque en todo lo que hace, habla o ve, sabe que el Señor mira su alma.

\par 39 Y limpia su mente para no ser condenado tanto por los hombres como por Dios.

\par 40 Y de la misma manera las obras de Beliar son dobles y no hay unidad en ellas.

\par 41 Por tanto, hijos míos, os digo que huyáis de la malicia de Beliar; porque da espada a los que le obedecen.

\par 42 Y la espada es la madre de siete males. Primero la mente concibe a través de Beliar, y primero ocurre el derramamiento de sangre; en segundo lugar ruina; tercero, tribulación; cuarto, el exilio; quinto, escasez; sexto, pánico; séptimo, destrucción.

\par 43 Por eso también Caín fue entregado por Dios a siete venganzas, pues cada cien años el Señor le traía una plaga.

\par 44 Cuando tenía doscientos años comenzó a sufrir, y al año novecientos fue destruido.

\par 45 Porque a causa de Abel su hermano fue juzgado con todos los males, pero Lamec con setenta veces siete.

\par 46 Porque los que son como Caín en envidia y odio a los hermanos, serán castigados con el mismo juicio para siempre.

\chapter{2}

\par \textit{El versículo 3 contiene un ejemplo sorprendente de la sencillez, aunque vivaz, de las figuras retóricas de estos antiguos patriarcas.}

\par 1 Y vosotros, hijos míos, huyed de las malas acciones, de la envidia y del odio hacia los hermanos, y aferraos a la bondad y al amor.

\par 2 El que tiene una mente pura en el amor, no busca mujer para fornicar; porque no tiene contaminación en su corazón, porque el Espíritu de Dios reposa sobre él.

\par 3 Porque así como el sol no se contamina al brillar sobre el estiércol y el lodo, sino que más bien seca ambos y ahuyenta el mal olor, así también la mente pura, aunque rodeada por las impurezas de la tierra, más bien las limpia y no se contamina ella misma.

\par 4 Y creo que también habrá entre vosotros malas acciones, a partir de las palabras de Enoc el justo: que cometeréis fornicación con la fornicación de Sodoma, y ​​todos perecerán, excepto unos pocos, y renovaréis las obras desenfrenadas con mujeres; y el reino del Señor no estará entre vosotros, porque luego él lo quitará.

\par 5 Sin embargo, el templo de Dios estará en vuestra porción, y el último templo será más glorioso que el primero.

\par 6 Y se reunirán allí las doce tribus y todas las naciones, hasta que el Altísimo envíe su salvación en la visita de un unigénito profeta.

\par 7 Y entrará en el primer templo, y allí el Señor será ultrajado y enaltecido sobre un madero.

\par 8 Y el velo del templo se rasgará, y el Espíritu de Dios pasará a los gentiles como fuego derramado.

\par 9 Y ascenderá del Hades y pasará de la tierra al cielo.

\par 10 Y sé cuán humilde será en la tierra y cuán glorioso en el cielo.

\par 11 Cuando José estaba en Egipto, deseaba ver su figura y la forma de su rostro; y por las oraciones de mi padre Jacob lo vi, estando despierto durante el día, incluso su figura entera exactamente como era.

\par 12 Y habiendo dicho estas cosas, les dijo: Sabed, pues, hijos míos, que me muero.

\par 13 Por tanto, sed sinceros cada uno con su prójimo y guardad la ley del Señor y sus mandamientos.

\par 14 Por estas cosas os dejo en lugar de herencia.

\par 15 Por tanto, vosotros también dadlas a vuestros hijos en posesión eterna; porque lo mismo hicieron Abraham, Isaac y Jacob.

\par 16 Por todo esto nos dieron por herencia, diciendo: Guardad los mandamientos de Dios, hasta que el Señor revele su salvación a todos los gentiles.

\par 17 Y entonces veréis a Enoc, a Noé, a Sem, a Abraham, a Isaac y a Jacob, levantándose a la derecha con alegría,

\par 18 Entonces también nosotros nos levantaremos, cada uno sobre nuestra tribu, y adoraremos al Rey del cielo, que apareció en la tierra en forma de hombre con humildad.

\par 19 Y todos los que creen en Él en la tierra se alegrarán con Él.

\par 20 Entonces también todos los hombres se levantarán, algunos para gloria y otros para vergüenza.

\par 21 Y el Señor juzgará primero a Israel por su injusticia; porque cuando apareció como Dios en carne para librarlos, no le creyeron.

\par 22 Y entonces juzgará a todos los gentiles, a cuantos no le creyeron cuando apareció sobre la tierra.

\par 23 Y convencerá a Israel por medio de los escogidos de los gentiles, como reprendió a Esaú por medio de los madianitas, quienes engañaron a sus hermanos, haciéndolos caer en la fornicación y la idolatría; y fueron alejados de Dios, llegando a ser, por tanto, hijos en la porción de los que temen al Señor.

\par 24 Por tanto, hijos míos, si camináis en santidad según los mandamientos del Señor, volveréis a vivir seguros conmigo y todo Israel será reunido en el Señor.

\par 25 Y ya no seré llamado lobo rapaz a causa de vuestros estragos, sino trabajador del Señor que distribuye alimento a los que hacen el bien.

\par 26 Y en los postreros días se levantará un amado del Señor, de la tribu de Judá y de Leví, hacedor de su buena voluntad en su boca, con conocimiento nuevo que iluminará a los gentiles.

\par 27 Hasta el fin del mundo estará en las sinagogas de las naciones y entre sus gobernantes, como un son de música en boca de todos.

\par 28 Y sus obras y sus palabras serán inscritas en los libros sagrados, y será un elegido de Dios para siempre.

\par 29 Y por medio de ellos irá de un lado a otro, como mi padre Jacob, diciendo: Él llenará lo que falta a tu tribu.

\par 30 Y habiendo dicho estas cosas, extendió los pies.

\par 31 Y murió en un hermoso y agradable sueño.

\par 32 Sus hijos hicieron lo que él les había ordenado y recogieron su cuerpo y lo sepultaron en Hebrón con sus padres.

\par 33 Y el número de los días de su vida fue ciento veinticinco años.


\end{document}