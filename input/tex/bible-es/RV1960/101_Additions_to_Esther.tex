\begin{document}

\title{Adiciones a Ester}

\chapter{1}

\par 1 Entonces Mardoqueo dijo: Dios ha hecho estas cosas.
\par 2 Porque recuerdo un sueño que tuve acerca de estas cosas, y nada de ello ha fallado.
\par 3 Una pequeña fuente se convirtió en un río, y había luz, sol y mucha agua. Este río es Ester, con quien el rey se casó y nombró reina.
\par 4 Y los dos dragones somos yo y Aman.
\par 5 Y fueron las naciones las que se reunieron para destruir el nombre de los judíos.
\par 6 Y mi nación es este Israel, que clamó a Dios y fue salvo; porque el Señor ha salvado a su pueblo, y el Señor nos ha librado de todos esos males, y Dios ha hecho señales y grandes prodigios, que no han ocurrido. hecho entre los gentiles.
\par 7 Por eso echó dos suertes: una para el pueblo de Dios y otra para todos los gentiles.
\par 8 Y estas dos suertes llegaron en la hora, el tiempo y el día del juicio delante de Dios entre todas las naciones.
\par 9 Entonces Dios se acordó de su pueblo y justificó su herencia.
\par 10 Por tanto, aquellos días serán para ellos en el mes de Adar, los días catorce y quince del mismo mes, de asamblea, alegría y alegría delante de Dios, según las generaciones para siempre entre su pueblo.

\chapter{2}

\par 1 En el año cuarto del reinado de Ptolomeo y Cleopatra, Dositeo, que decía ser sacerdote y levita, y Ptolomeo su hijo, trajeron esta epístola de Purim, que decían que era la misma, y ​​que Lisímaco hijo de Ptolomeo, que estaba en Jerusalén, lo había interpretado.
\par 2 En el año segundo del reinado de Artejerjes el grande, el primer día del mes de Nisán, Mardoqueo, hijo de Jairo, hijo de Semei, hijo de Cisai, de la tribu de Benjamín, tuvo un sueño;
\par 3 El cual era judío y habitaba en la ciudad de Susa, un hombre importante, que servía en la corte del rey.
\par 4 También fue uno de los cautivos que Nabucodonosor, rey de Babilonia, llevó de Jerusalén con Jeconías, rey de Judea; y este era su sueño:
\par 5 He aquí un ruido de alboroto, con truenos, terremotos y alboroto en la tierra.
\par 6 Y he aquí, dos grandes dragones salieron listos para pelear, y su grito fue grande.
\par 7 Y ante su grito, todas las naciones se prepararon para la batalla, para luchar contra el pueblo justo.
\par 8 Y he aquí un día de oscuridad y oscuridad, de tribulación y angustia, de aflicción y de gran alboroto sobre la tierra.
\par 9 Y toda la nación justa se turbó, temiendo sus propios males, y estaba a punto de perecer.
\par 10 Entonces clamaron a Dios, y de su clamor, como si brotara de una pequeña fuente, surgió una gran inundación, mucha agua.
\par 11 Salieron la luz y el sol, y los humildes fueron exaltados y devoraron a los gloriosos.
\par 12 Cuando Mardoqueo, que había visto este sueño y lo que Dios había decidido hacer, estaba despierto, lo tuvo presente en su mente, y hasta la noche estuvo deseoso de saberlo.

\chapter{3}

\par 1 Y Mardoqueo descansó en el patio con Gabatha y Tarra, los dos eunucos del rey y guardianes del palacio.
\par 2 Y él escuchó sus maquinaciones, investigó sus propósitos y supo que estaban a punto de echar mano al rey Artejerjes; y así certificó al rey de ellos.
\par 3 Entonces el rey examinó a los dos eunucos y, después de confesarlo, los estrangularon.
\par 4 Y el rey hizo acta de estas cosas, y Mardoqueo también las escribió.
\par 5 Entonces el rey ordenó a Mardoqueo que sirviera en la corte y le recompensó por ello.
\par 6 Pero Amán, hijo de Amadato agagueo, que gozaba de gran honor ante el rey, intentó molestar a Mardoqueo y a su pueblo a causa de los dos eunucos del rey.

\chapter{4}

\par 1 La copia de las cartas era ésta: El gran rey Artejerjes escribe estas cosas a los príncipes y gobernadores que están bajo su mando desde la India hasta Etiopía en ciento veinte provincias.
\par 2 Después de eso me convertí en señor de muchas naciones y tuve dominio sobre el mundo entero, sin envanecerme con presunción de mi autoridad, sino comportandome siempre con equidad y apacibilidad, me propuse mantener a mis súbditos continuamente en una vida tranquila, y haciendo mi reino pacífico y abierto al paso hasta las costas más lejanas, para renovar la paz, que es deseada por todos los hombres.
\par 3 Cuando pregunté a mis consejeros cómo podía suceder esto, Amán, que sobresalía entre nosotros en sabiduría, era aprobado por su constante buena voluntad y su inquebrantable fidelidad, y tenía el honor del segundo lugar en el reino,
\par 4 Nos ha declarado que en todas las naciones del mundo había un pueblo maligno disperso, que tenía leyes contrarias a todas las naciones y continuamente despreciaba los mandamientos de los reyes, a fin de que la unificación de nuestros reinos, que nosotros pretendíamos honorablemente. no puede seguir adelante.
\par 5 Entendemos, pues, que sólo este pueblo se opone continuamente a todos los hombres, diferenciándose en la extraña manera de sus leyes y perjudicando a nuestro estado, haciendo todos los males que pueden para que nuestro reino no esté firmemente establecido.
\par 6 Por lo tanto, hemos ordenado que todos los que os han sido enviados por escrito por Aman, quien es el encargado de los asuntos y es nuestro siguiente, sean completamente destruidos, junto con sus esposas e hijos, por la espada de sus enemigos, sin toda misericordia y piedad, el día catorce del mes duodécimo de Adar de este año presente:
\par 7 Para que ellos, que en el pasado y ahora también son maliciosos, en un día vayan con violencia a la tumba, y así en el futuro hagan que nuestros asuntos estén bien arreglados y sin problemas.
\par 8 Entonces Mardoqueo pensó en todas las obras del Señor y le dirigió su oración:
\par 9 Diciendo: Señor, Señor, Rey Todopoderoso: porque el mundo entero está en tu poder, y si tú has designado salvar a Israel, no hay nadie que pueda contradecirte:
\par 10 Porque tú hiciste los cielos y la tierra y todas las maravillas que hay debajo del cielo.
\par 11 Tú eres Señor de todas las cosas, y no hay hombre que pueda resistirte, que eres el Señor.
\par 12 Tú lo sabes todo y sabes, Señor, que no fue por desprecio ni por orgullo ni por ningún deseo de gloria que no me postré ante el orgulloso Amán.
\par 13 Porque yo podría haberme contentado con besar las plantas de sus pies por la salvación de Israel.
\par 14 Pero esto lo hice para no anteponer la gloria del hombre a la gloria de Dios; ni a nadie adoraré sino a ti, oh Dios, ni lo haré con soberbia.
\par 15 Ahora pues, Señor Dios y Rey, perdona a tu pueblo, porque sus ojos están sobre nosotros para destruirnos; sí, desean destruir la herencia que ha sido tuya desde el principio.
\par 16 No menosprecies la porción que has librado de Egipto para ti.
\par 17 Oye mi oración y ten misericordia de tu herencia; convierte nuestra tristeza en alegría, para que vivamos, oh Señor, y alabemos tu nombre; y no destruyas la boca de los que te alaban, oh Señor.
\par 18 De la misma manera, todo Israel clamaba intensamente al Señor, porque su muerte estaba ante sus ojos.

\chapter{5}

\par 1 También la reina Ester, temiendo la muerte, acudió al Señor:
\par 2 Y se quitó su ropa gloriosa y se vistió con ropas de angustia y luto; y en lugar de ungüentos preciosos, se cubrió la cabeza con ceniza y estiércol, y humilló mucho su cuerpo, y todos los lugares de su alegría se lleno de su cabello desgarrado.
\par 3 Y oró al Señor, Dios de Israel, diciendo: Señor mío, sólo tú eres nuestro Rey: ayúdame, mujer desolada, que no tiene más ayuda que tú.
\par 4 Porque mi peligro está en mi mano.
\par 5 Desde mi juventud he oído en la tribu de mi familia que tú, oh Señor, tomaste a Israel de entre todos los pueblos, y a nuestros padres de todos sus predecesores, para heredad perpetua, y cumpliste todo lo que prometiste. a ellos.
\par 6 Y ahora hemos pecado delante de ti; por eso nos has entregado en manos de nuestros enemigos,
\par 7 Porque adoramos a sus dioses: Señor, tú eres justo.
\par 8 Sin embargo, no les satisface que estemos en amargo cautiverio, sino que han golpeado sus manos con sus ídolos,
\par 9 para abolir lo que tú has ordenado con tu boca, y destruir tu herencia, y tapar la boca de los que te alaban, y apagar la gloria de tu casa y de tu altar,
\par 10 Y abren la boca de las naciones para proclamar las alabanzas de los ídolos y para ensalzar a un rey carnal para siempre.
\par 11 Oh Señor, no entregues tu cetro a los que son nada, ni permitas que se rían de nuestra caída; sino que vuelvan su estrategia contra ellos mismos y hagan un ejemplo de aquel que ha comenzado esto contra nosotros.
\par 12 Acuérdate, oh Señor, de darte a conocer en el momento de nuestra aflicción, y dame confianza, oh Rey de las naciones y Señor de todo poder.
\par 13 Dame palabras elocuentes en mi boca delante del león: haz que su corazón odie a quien pelea contra nosotros, para que haya fin para él y para todos los que piensan como él.
\par 14 Pero líbranos con tu mano y ayúdame a mí, que estoy desolado y que no tiene otra ayuda que tú.
\par 15 Tú lo sabes todo, oh Señor; tú sabes que aborrezco la gloria de los injustos, y aborrezco el lecho de los incircuncisos y de todas las naciones.
\par 16 Tú sabes mi necesidad: porque aborrezco el signo de mi alta posición, que está sobre mi cabeza en los días en que me muestro, y que lo aborrezco como un trapo menstrual, y que no lo uso cuando estoy privado por mi cuenta.
\par 17 Y que tu sierva no ha comido a la mesa de Amán, ni he apreciado mucho el banquete del rey, ni he bebido el vino de las libaciones.
\par 18 Tampoco tu sierva tuvo alegría alguna desde el día en que fui traído hasta aquí, sino en ti, oh Señor Dios de Abraham.
\par 19 Oh Dios fuerte sobre todo, escucha la voz de los desamparados y líbranos de las manos de los malvados, y líbrame de mi miedo.

\chapter{6}

\par 1 Y al tercer día, cuando terminó sus oraciones, se quitó sus vestidos de luto y se vistió con su ropa gloriosa.
\par 2 Y, espléndidamente ataviada, después de haber invocado a Dios, que es quien contempla y salva todas las cosas, tomó consigo a dos doncellas:
\par 3 Y sobre uno se apoyaba, como si se llevara con delicadeza;
\par 4 Y la otra la siguió, llevando su séquito.
\par 5 Ella era rubicunda por la perfección de su belleza, y su rostro era alegre y muy amable; pero su corazón estaba angustiado por el miedo.
\par 6 Luego, tras atravesar todas las puertas, se presentó ante el rey, que estaba sentado en su trono real y estaba vestido con todas sus vestiduras majestuosas, todas resplandecientes de oro y piedras preciosas; y era muy espantoso.
\par 7 Entonces, alzando su rostro resplandeciente de majestad, la miró fijamente; y la reina cayó al suelo, pálida y desmayada, y se inclinó sobre la cabeza de la doncella que iba delante de ella.
\par 8 Entonces Dios transformó en apacibilidad el espíritu del rey, quien aterrorizado saltó de su trono y la tomó en sus brazos, hasta que ella volvió en sí, y la consoló con palabras amorosas y le dijo:
\par 9 Ester, ¿qué te pasa? Soy tu hermano, ten buen ánimo:
\par 10 Aunque nuestro mandamiento sea general, no morirás: acércate.
\par 11 Entonces levantó su cetro de oro y lo puso sobre su cuello,
\par 12 Y abrazándola, le dijo: Háblame.
\par 13 Entonces ella le dijo: Te vi, señor mío, como un ángel de Dios, y mi corazón se turbó por temor a tu majestad.
\par 14 Porque eres maravilloso, Señor, y tu rostro está lleno de gracia.
\par 15 Y mientras ella hablaba, cayó desmayada.
\par 16 Entonces el rey se turbó y todos sus servidores la consolaron.

\chapter{7}

\par 1 El gran rey Artejerjes, a los príncipes y gobernadores de ciento veinte provincias desde la India hasta Etiopía, y a todos nuestros súbditos fieles, saludos.
\par 2 Muchos, cuanto más a menudo son honrados con la gran generosidad de sus graciosos príncipes, más orgullosos se vuelven,
\par 3 Y no sólo trates de dañar a nuestros súbditos, sino que, como no puedes soportar la abundancia, trata de actuar también contra aquellos que les hacen bien:
\par 4 Y quitad no sólo la gratitud de entre los hombres, sino también los enaltecidos con gloriosas palabras de los malvados, que nunca fueron buenos, y piensan escapar de la justicia de Dios, que todo lo ve y aborrece el mal.
\par 5 Muchas veces también las palabras justas de aquellos a quienes se les ha confiado la gestión de los asuntos de sus amigos han hecho que muchos de los que tienen autoridad sean partícipes de sangre inocente y los han envuelto en calamidades sin remedio.
\par 6 Seduciendo con la falsedad y el engaño de su carácter lascivo la inocencia y la bondad de los príncipes.
\par 7 Ahora bien, tal como hemos declarado, podréis ver esto, no tanto por las historias antiguas, como si escudriñáis lo que se ha hecho mal últimamente a través del comportamiento pestilente de aquellos que están indignamente colocados en autoridad.
\par 8 Y debemos cuidarnos en el futuro, para que nuestro reino sea tranquilo y pacífico para todos los hombres,
\par 9 Tanto cambiando nuestros propósitos, como juzgando siempre las cosas que son evidentes con un procedimiento más igualitario.
\par 10 Porque Amán, macedonio, hijo de Amadata, era en verdad un extraño de sangre persa y muy alejado de nuestra bondad, y como un extraño nos recibió,
\par 11 Había obtenido hasta ahora el favor que mostramos a todas las naciones, hasta el punto de que lo llamaban nuestro padre, y era continuamente honrado por todos los próximos al rey.
\par 12 Pero él, no teniendo su gran dignidad, quiso privarnos de nuestro reino y de nuestra vida:
\par 13 Habiendo buscado con múltiples y astutos engaños la destrucción de nosotros, tanto de Mardoqueo, que nos salvó la vida y continuamente procuraba nuestro bien, como también de la inocente Ester, participante de nuestro reino, con toda su nación.
\par 14 Porque de esta manera pensó, al encontrarnos sin amigos, trasladar el reino de los persas a los macedonios.
\par 15 Pero vemos que los judíos a quienes este malvado ha entregado a la destrucción total no son malhechores, sino que viven según leyes muy justas.
\par 16 Y que sean hijos del Dios vivo, altísimo y poderoso, que ha ordenado el reino tanto para nosotros como para nuestros progenitores de la manera más excelente.
\par 17 Por lo tanto, haréis bien en no ejecutar las cartas que os envió Aman, hijo de Amadatha.
\par 18 Porque el que hizo estas cosas, será colgado a las puertas de Susa con toda su familia: Dios, que gobierna todas las cosas, prontamente le dará venganza según sus merecimientos.
\par 19 Por tanto, publicaréis la copia de esta carta en todas partes, para que los judíos puedan vivir libremente según sus propias leyes.
\par 20 Y los ayudaréis, para que en el mismo día, que es el día trece del mes duodécimo de Adar, se venguen de aquellos que en el momento de su aflicción los atacarán.
\par 21 Porque Dios Todopoderoso les ha convertido en alegría el día en que el pueblo elegido perecería.
\par 22 Por tanto, entre vuestras fiestas solemnes celebraréis un día solemne con todos los banquetes.
\par 23 Para que tanto ahora como en el futuro haya seguridad para nosotros y para los bien afectados persas; pero para aquellos que conspiran contra nosotros, un memorial de destrucción.
\par 24 Por lo tanto, toda ciudad y país que no haga esto será destruido sin piedad, con fuego y espada, y será hecho no sólo intransitable para los hombres, sino también muy aborrecible para las fieras y las aves para siempre. .

\end{document}