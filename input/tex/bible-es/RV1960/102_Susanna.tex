\begin{document}

\title{Susana}


\chapter{1}

\par 1 Apartado del principio de Daniel, porque no está en hebreo, como tampoco la Narración de Bel y el Dragón.[1] Habitaba en Babilonia un hombre llamado Joacim:
\par 2 Y tomó por esposa a Susana, hija de Chelcias, mujer muy hermosa y temerosa del Señor.
\par 3 También sus padres eran justos y educaban a su hija según la ley de Moisés.
\par 4 Joaquín era un hombre muy rico y tenía un hermoso jardín junto a su casa; y los judíos acudieron a él; porque era más honorable que todos los demás.
\par 5 Aquel mismo año fueron nombrados dos de los ancianos del pueblo para que fueran jueces, tal como el Señor había dicho; la maldad venía de Babilonia por parte de los jueces antiguos, que parecían gobernar al pueblo.
\par 6 Éstos se quedaban mucho en casa de Joaquín, y acudían a ellos todos los que tenían algún pleito.
\par 7 Cuando la gente se fue al mediodía, Susana fue al jardín de su marido a caminar.
\par 8 Y los dos ancianos la veían entrar y caminar todos los días; de modo que su lujuria se encendió hacia ella.
\par 9 Y pervirtieron su propia mente y apartaron sus ojos para no mirar al cielo ni recordar los juicios justos.
\par 10 Y aunque ambos estaban heridos por su amor, ninguno se atrevió a mostrarle al otro su dolor.
\par 11 Porque les daba vergüenza declarar su lujuria, que deseaban tener relaciones con ella.
\par 12 Sin embargo, ellos velaban día tras día para verla.
\par 13 Y el uno dijo al otro: Vámonos ahora a casa, que es hora de cenar.
\par 14 Cuando salieron, se separaron el uno del otro y, volviendo atrás, llegaron al mismo lugar; y después de preguntarse mutuamente la causa, reconocieron su lujuria; luego fijaron un tiempo juntos para encontrarla sola.
\par 15 Y aconteció que, como esperaban el momento oportuno, ella entró como antes con sólo dos criadas, y quiso lavarse en el jardín, porque hacía calor.
\par 16 Y no había nadie allí excepto los dos ancianos que se habían escondido y la vigilaban.
\par 17 Entonces dijo a sus criadas: Traedme aceite y bolas para lavar, y cerrad las puertas del jardín para que me lave.
\par 18 Ellos hicieron lo que ella les había ordenado, cerraron las puertas del jardín y salieron por las puertas privadas a buscar las cosas que ella les había ordenado; pero no vieron a los ancianos, porque estaban escondidos.
\par 19 Cuando las criadas salieron, los dos mayores se levantaron y corrieron hacia ella, diciendo:
\par 20 He aquí, las puertas del jardín están cerradas, para que nadie pueda vernos, y estamos enamorados de ti; Consiente, pues, en nosotros, y acuéstate con nosotros.
\par 21 Si no quieres, daremos testimonio contra ti de que un joven estaba contigo y por eso despediste a tus siervas.
\par 22 Entonces Susana suspiró y dijo: Estoy angustiada por todos lados, porque si hago esto, me mataré; y si no lo hago, no puedo escapar de tus manos.
\par 23 Es mejor para mí caer en vuestras manos y no hacerlo, que pecar ante los ojos del Señor.
\par 24 Entonces Susana gritó a gran voz, y los dos ancianos clamaron contra ella.
\par 25 Entonces uno corrió y abrió la puerta del jardín.
\par 26 Cuando los sirvientes de la casa oyeron el grito en el jardín, se apresuraron a entrar por la puerta del retrete para ver qué le habían hecho.
\par 27 Pero cuando los ancianos expusieron su asunto, los sirvientes se avergonzaron mucho, porque nunca se había hecho tal informe sobre Susana.
\par 28 Y aconteció que al día siguiente, cuando el pueblo se había reunido junto a su marido Joaquín, vinieron también los dos ancianos llenos de ideas maliciosas contra Susana para matarla;
\par 29 Y dijo delante del pueblo: Envíad traer a Susana, hija de Chelcías, mujer de Joacim. Y así lo enviaron.
\par 30 Y ella vino con su padre y su madre, sus hijos y todos sus parientes.
\par 31 Susana era una mujer muy delicada y hermosa a la vista.
\par 32 Y estos malvados ordenaron que le descubrieran el rostro (porque estaba cubierta) para poder admirar su belleza.
\par 33 Por eso lloraron sus amigos y todos los que la vieron.
\par 34 Entonces los dos ancianos se levantaron en medio del pueblo y le pusieron las manos sobre la cabeza.
\par 35 Y ella, llorando, miró al cielo, porque su corazón confiaba en el Señor.
\par 36 Y los ancianos dijeron: Mientras caminábamos solos por el jardín, entró esta mujer con dos criadas, cerró las puertas del jardín y despidió a las criadas.
\par 37 Entonces se le acercó un joven que estaba escondido y se acostó con ella.
\par 38 Entonces nosotros, que estábamos en un rincón del jardín, al ver esta maldad, corrimos hacia ellos.
\par 39 Y cuando los vimos juntos, no pudimos retener al hombre, porque era más fuerte que nosotros, abrió la puerta y saltó fuera.
\par 40 Pero tomando a esta mujer, preguntamos quién era el joven, pero ella no nos dijo: estas cosas damos testimonio.
\par 41 Entonces la asamblea les creyó como a los ancianos y jueces del pueblo, y la condenaron a muerte.
\par 42 Entonces Susana gritó a gran voz y dijo: ¡Oh Dios eterno, que conoces los secretos y conoces todas las cosas antes de que existan!
\par 43 Tú sabes que han dado falso testimonio contra mí y he aquí que debo morir; mientras que nunca hice cosas como las que estos hombres han inventado maliciosamente contra mí.
\par 44 Y el Señor escuchó su voz.
\par 45 Por eso, cuando ella era llevada a la muerte, el Señor levantó el espíritu santo de un joven llamado Daniel:
\par 46 Quien clamaba a gran voz, Yo soy limpio de la sangre de esta mujer.
\par 47 Entonces todo el pueblo se volvió hacia él y le dijeron: ¿Qué significan estas palabras que has dicho?
\par 48 Entonces él, que estaba en medio de ellos, dijo: ¿Sois tan necios, hijos de Israel, que sin examinar ni conocer la verdad habéis condenado a una hija de Israel?
\par 49 Vuelve otra vez al lugar del juicio, porque han dado falso testimonio contra ella.
\par 50 Entonces todo el pueblo se volvió apresuradamente y los ancianos le dijeron: Ven, siéntate entre nosotros y muéstranoslo, ya que Dios te ha concedido el honor de anciano.
\par 51 Entonces Daniel les dijo: Apartad a estos dos, uno lejos del otro, y yo los examinaré.
\par 52 Entonces, cuando fueron separados uno del otro, llamó a uno de ellos y le dijo: Oh tú, que te has envejecido en la maldad, ahora tus pecados que cometiste antes han salido a la luz.
\par 53 Porque has pronunciado juicio falso, has condenado al inocente y has dejado en libertad al culpable; aunque el Señor diga: No matarás al inocente y al justo.
\par 54 Ahora bien, si la has visto, dime: ¿Bajo qué árbol los viste juntos? Quien respondió: Debajo de un lentisco.
\par 55 Y Daniel dijo: Muy bien; has mentido contra tu propia cabeza; porque incluso ahora el ángel de Dios ha recibido la sentencia de Dios de cortarte en dos.
\par 56 Entonces él, apartándolo, mandó traer al otro y le dijo: descendiente de Canaán, y no de Judá, la belleza te ha engañado y la concupiscencia ha pervertido tu corazón.
\par 57 Así habéis hecho con las hijas de Israel, y ellas por miedo os acompañaron, pero la hija de Judá no soportó vuestra maldad.
\par 58 Ahora pues, dime: ¿Bajo qué árbol los llevaste juntos? Quien respondió: Bajo una encina.
\par 59 Entonces Daniel le dijo: Bien; También has mentido contra tu propia cabeza; porque el ángel de Dios espera con la espada para cortarte en dos y destruirte.
\par 60 Entonces toda la asamblea gritó a gran voz y alababa a Dios, que salva a los que en él confían.
\par 61 Y se levantaron contra los dos ancianos, porque Daniel los había convencido de falso testimonio por su propia boca.
\par 62 Y conforme a la ley de Moisés, hicieron con ellos lo que maliciosamente pretendían hacer con su prójimo, y los mataron. Así la sangre inocente fue salvada el mismo día.
\par 63 Por eso Chelcias y su mujer alabaron a Dios por su hija Susana, por su marido Joaquín y por todos sus parientes, porque no se había encontrado en ella ninguna deshonestidad.
\par 64 Desde aquel día Daniel tuvo gran reputación ante el pueblo.

\end{document}