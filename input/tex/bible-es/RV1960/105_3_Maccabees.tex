\begin{document}

\title{3 Macabeos}


\chapter{1}

\par 1 Cuando Filopátor se enteró por los que habían regresado de que las regiones que había controlado habían sido tomadas por Antíoco, dio órdenes a todas sus fuerzas, tanto de infantería como de caballería, tomó consigo a su hermana Arsínoe y marchó hacia la región cercana. Rafia, donde estaban acampados los partidarios de Antíoco.
\par 2 Pero un tal Teodoto, decidido a llevar a cabo el complot que había ideado, tomó consigo lo mejor de las armas ptolemaicas que le habían sido entregadas previamente y cruzó de noche a la tienda de Ptolomeo, con la intención de matarlo con sus manos y así poner fin a la guerra.
\par 3 Pero Dositeo, conocido como hijo de Drimilo, un judío de nacimiento que más tarde cambió de religión y apostató de las tradiciones ancestrales, se llevó al rey y dispuso que cierto hombre insignificante durmiera en la tienda; y resultó que este hombre incurrió en la venganza destinada al rey.
\par 4 Cuando se produjo una encarnizada lucha y las cosas se inclinaban más bien a favor de Antíoco, Arsínoe se dirigió a las tropas llorando y llorando, con los cabellos despeinados, y los exhortó a defenderse valientemente a sí mismos, a sus hijos y a sus mujeres, prometiéndoles defenderse con valentía. Dales a cada uno dos minas de oro si ganan la batalla.
\par 5 Y aconteció que el enemigo fue derrotado en la batalla, y también muchos cautivos fueron hechos.
\par 6 Ahora que había frustrado el complot, Ptolomeo decidió visitar las ciudades vecinas y animarlas.
\par 7 Con esto y dotando de regalos sus recintos sagrados, fortaleció la moral de sus súbditos.
\par 8 Como los judíos habían enviado a algunos de su consejo y a algunos ancianos para saludarlo, traerle regalos de bienvenida y felicitarlo por lo sucedido, él estaba aún más ansioso por visitarlos lo antes posible.
\par 9 Cuando llegó a Jerusalén, ofreció sacrificios al Dios supremo, hizo ofrendas de gracias e hizo lo que convenía al lugar santo. Luego, al entrar al lugar y quedar impresionado por su excelencia y su belleza,
\par 10 se maravilló del buen orden del templo y concibió el deseo de entrar en el lugar santísimo.
\par 11 Cuando dijeron que esto no estaba permitido, porque ni siquiera los miembros de su propia nación podían entrar, ni siquiera todos los sacerdotes, sino sólo el sumo sacerdote, que era preeminente sobre todos, y sólo una vez al año, el rey no se dejó convencer en modo alguno.
\par 12 Incluso después de que le leyeron la ley, no dejó de afirmar que debía entrar, diciendo: «Aunque esos hombres sean privados de este honor, yo no debería serlo».
\par 13 Y preguntó por qué, cuando entraba en todos los templos, nadie le había detenido.
\par 14 Y alguien, sin darse cuenta, dijo que estaba mal tomar esto como un signo en sí mismo.
\par 15 «Pero puesto que esto ha sucedido», dijo el rey, «¿por qué no debería yo al menos entrar, lo quieran o no?».
\par 16 Entonces los sacerdotes, vestidos con todas sus vestiduras, se postraron y rogaron al Dios supremo que ayudara en la presente situación y evitara la violencia de este malvado designio, y llenaron el templo de gritos y lágrimas;
\par 17 y los que se quedaron en la ciudad se agitaron y se apresuraron a salir, pensando que algo misterioso estaba sucediendo.
\par 18 Las vírgenes que estaban encerradas en sus aposentos salieron corriendo con sus madres, rociaron sus cabellos con polvo y llenaron las calles de gemidos y lamentos.
\par 19 Las mujeres que acababan de vestirse para el matrimonio abandonaron los aposentos nupciales preparados para la unión matrimonial y, descuidando la debida modestia, se congregaron desordenadamente en la ciudad.
\par 20 Las madres y las nodrizas abandonaban aquí y allá hasta a los niños recién nacidos, unos en las casas y otros en las calles, y sin mirar atrás se agolpaban en el templo más alto.
\par 21 Varias fueron las súplicas de los allí reunidos a causa de los planes profanos del rey.
\par 22 Además, el más audaz de los ciudadanos no toleraría la realización de sus planes ni el cumplimiento de su propósito previsto.
\par 23 Gritaron a sus compañeros que tomaran las armas y murieran valientemente por la ley ancestral, y provocaron un gran alboroto en el lugar santo; y siendo apenas frenados por los ancianos y los ancianos, recurrieron a la misma postura de súplica que los demás.
\par 24 Mientras tanto, la multitud, como antes, estaba ocupada en oración,
\par 25 Mientras los ancianos que estaban cerca del rey intentaban de diversas maneras desviar su arrogante mente del plan que había concebido.
\par 26 Pero él, en su arrogancia, no se dio cuenta de nada y comenzó a acercarse, decidido a llevar a cabo el plan mencionado.
\par 27 Cuando los que estaban alrededor de él vieron esto, se volvieron, junto con nuestro pueblo, para invocar a aquel que tiene todo el poder para defenderlos en el presente problema y no pasar por alto este acto ilegal y altivo.
\par 28 El grito continuo, vehemente y concertado de la multitud resultó en un inmenso alboroto;
\par 29 porque parecía que no sólo los hombres sino también los muros y toda la tierra alrededor resonaban, pues en verdad todos preferían entonces la muerte a la profanación del lugar.

\chapter{2}

\par 1 Entonces el sumo sacerdote Simón, de cara al santuario, doblando las rodillas y extendiendo las manos con serena dignidad, oró así:
\par 2 «Señor, Señor, Rey de los cielos y soberano de toda la creación, santo entre los santos, único gobernante, todopoderoso, presta atención a nosotros, que sufrimos gravemente por culpa de un hombre impío y profano, envanecido en su audacia y poder».
\par 3 «Porque tú, el creador de todas las cosas y el gobernador de todo, eres un gobernante justo, y juzgas a los que han hecho algo con insolencia y arrogancia.»
\par 4 «Tú destruiste a los que en el pasado cometieron injusticias, entre los cuales había incluso gigantes que confiaban en su fuerza y ​​audacia, a quienes destruiste trayendo sobre ellos un diluvio sin límites».
\par 5 «Consumiste con fuego y azufre a los hombres de Sodoma que actuaron con arrogancia y que eran famosos por sus vicios; y los hiciste ejemplo para los que vendrían después».
\par 6 «Hiciste notorio tu gran poder al infligir muchos y variados castigos al audaz Faraón que había esclavizado a tu santo pueblo Israel».
\par 7 «Y cuando él los perseguía con carros y muchas tropas, lo derrotaste en las profundidades del mar, pero sacaste sanos y salvos a los que habían confiado en ti, el Soberano de toda la creación».
\par 8 «Y cuando vieron las obras de tus manos, te alabaron, el Todopoderoso».
\par 9 «Tú, oh Rey, cuando creaste la tierra ilimitada e inconmensurable, elegiste esta ciudad y santificaste este lugar para tu nombre, aunque no necesitas nada; y cuando la hubiste glorificado con tu magnífica manifestación, le hiciste un fundamento firme para la gloria de tu grande y honrado nombre».
\par 10 «Y como amas a la casa de Israel, prometiste que si tuviéramos problemas y la tribulación nos sobreviniera, escucharías nuestra petición cuando viniéramos a este lugar y oráramos».
\par 11 «Y ciertamente tú eres fiel y verdadero».
\par 12 Y porque muchas veces, cuando nuestros padres estaban oprimidos, tú los ayudaste en su humillación y los libraste de grandes males.
\par 13 «Mira ahora, oh Santo Rey, que a causa de nuestros muchos y grandes pecados estamos aplastados por el sufrimiento, sometidos a nuestros enemigos y abrumados por la impotencia».
\par 14 «En nuestra caída, este hombre audaz y profano se propone violar el lugar santo en la tierra dedicado a tu glorioso nombre».
\par 15 Porque vuestra morada, el cielo de los cielos, es inaccesible al hombre.
\par 16 «Pero porque generosamente concediste tu gloria a tu pueblo Israel, santificaste este lugar».
\par 17 No nos castigues por la inmundicia de estos hombres ni nos pidas cuentas por esta profanación, no sea que los transgresores se jacten de su ira o se regocijen con la arrogancia de su lengua, diciendo:
\par 18 ««Hemos pisoteado la casa del santuario como son pisoteadas las casas ofensivas»».
\par 19 «Limpia nuestros pecados y dispersa nuestros errores, y revela tu misericordia en esta hora».
\par 20 «Que pronto nos alcancen tus misericordias, y pon alabanzas en la boca de los abatidos y quebrantados de espíritu, y danos paz».
\par 21 Entonces Dios, que todo lo supervisa, el primer Padre de todos, el Santo entre los santos, habiendo oído la súplica legítima, azotó al que se había exaltado con insolencia y audacia.
\par 22 Lo sacudió de un lado y del otro, como la caña es sacudida por el viento, de modo que quedó indefenso en el suelo y, además de quedar paralizado de sus miembros, no podía ni siquiera hablar, ya que había sido herido por un justo juicio.
\par 23 Entonces, tanto los amigos como los guardaespaldas, al ver el severo castigo que le había tocado, y temiendo que perdiera la vida, rápidamente lo sacaron a rastras, presas del pánico y del miedo tan grande que sentían.
\par 24 Al cabo de un tiempo se recuperó y, aunque había sido castigado, no se arrepintió, sino que se fue proferiendo amargas amenazas.
\par 25 Cuando llegó a Egipto, incrementó sus actos de malicia, instigado por los compañeros de bebida y camaradas antes mencionados, que eran ajenos a todo lo justo.
\par 26 No se contentó con sus innumerables actos libertinos, sino que continuó con tal audacia que forjó malas noticias en las distintas localidades; y muchos de sus amigos, observando atentamente el propósito del rey, también siguieron su voluntad.
\par 27 Se propuso infligir vergüenza pública a la comunidad judía y erigió una piedra en la torre del atrio con esta inscripción:
\par 28 «Ninguno de los que no hagan sacrificios entrará en sus santuarios, y todos los judíos estarán sujetos a un registro que implicará un impuesto de capitación y al estatus de esclavos. Quienes se opongan a esto serán apresados ​​por la fuerza y ​​ejecutados».
\par 29 «Aquellos que estén registrados también serán marcados a fuego en sus cuerpos con el símbolo de la hoja de hiedra de Dioniso, y también serán reducidos a su estado limitado anterior».
\par 30 Para que no parezca un enemigo para todos, escribió a continuación: «Pero si alguno de ellos prefiere unirse a los que han sido iniciados en los misterios, tendrá la misma ciudadanía que los alejandrinos».
\par 31 Ahora bien, algunos, sin embargo, con evidente aborrecimiento por el precio que debía exigirse por mantener la religión de su ciudad, se entregaron rápidamente, ya que esperaban mejorar su reputación con su futura asociación con el rey.
\par 32 Pero la mayoría actuó con firmeza y espíritu valiente y no se apartó de su religión; y pagando dinero a cambio de la vida intentaron confiadamente salvarse del registro.
\par 33 Tenían la firme esperanza de obtener ayuda y aborrecían a quienes se separaban de ellos, considerándolos enemigos de la nación judía y privándolos de la comunión común y de la ayuda mutua.

\chapter{3}

\par 1 Cuando el rey impío comprendió esta situación, se enfureció tanto que no sólo se enfureció contra los judíos que vivían en Alejandría, sino que se mostró aún más amargamente hostil hacia los del campo; y ordenó que inmediatamente todos fueran reunidos en un solo lugar y ejecutados por los medios más crueles.
\par 2 Mientras se preparaban estos asuntos, circuló un rumor hostil contra la nación judía, por parte de hombres que conspiraban para hacerles mal, con el pretexto de que habían impedido a otros observar sus costumbres.
\par 3 Los judíos, sin embargo, continuaron manteniendo buena voluntad y lealtad inquebrantable hacia la dinastía;
\par 4 pero como adoraban a Dios y se comportaban según su ley, se mantenían separados en cuanto a los alimentos. Por eso a algunos les parecieron odiosos;
\par 5 pero adornando su estilo de vida con las buenas obras de personas rectas, adquirieron buena reputación entre todos los hombres.
\par 6 Sin embargo, los de otras razas no prestaron atención al buen servicio a su nación, lo cual era común entre todos;
\par 7 en cambio, chismeaban sobre las diferencias en el culto y la comida, alegando que esta gente no era leal ni al rey ni a sus autoridades, sino que era hostil y muy opuesta a su gobierno. Así que no les hicieron ningún reproche ordinario.
\par 8 Los griegos de la ciudad, aunque no sufrieron ningún daño, cuando vieron un tumulto inesperado en torno a esta gente y las multitudes que de repente se estaban formando, no fueron lo suficientemente fuertes para ayudarlos, porque vivían bajo tiranía. Intentaron consolarlos, afligidos por la situación, y esperaban que las cosas cambiaran;
\par 9 porque una comunidad tan grande no debería ser abandonada a su suerte si no ha cometido ningún delito.
\par 10 Y algunos de sus vecinos, amigos y socios de negocios ya se habían llevado a algunos de ellos en privado y se comprometían a protegerlos y a hacer esfuerzos más serios para ayudarlos.
\par 11 Entonces el rey, jactancioso de su actual buena fortuna, y sin considerar el poder del Dios supremo, sino pensando que perseveraría constantemente en su mismo propósito, escribió esta carta contra ellos:
\par 12 «El rey Ptolomeo Filopátor a sus generales y soldados en Egipto y en todas sus regiones, saludos y buena salud».
\par 13 «A mí y a nuestro gobierno nos va bien».
\par 14 «Cuando nuestra expedición tuvo lugar en Asia, como ustedes mismos saben, concluyó, según lo planeado, por la alianza deliberada de los dioses con nosotros en la batalla.»
\par 15 «Y consideramos que no debíamos gobernar a las naciones que habitaban en Celesiria y Fenicia con el poder de la lanza, sino que debíamos tratarlas con clemencia y gran benevolencia, tratándolas bien con gusto».
\par 16 «Y después de haber concedido grandes ingresos a los templos de las ciudades, llegamos también a Jerusalén y subimos a honrar el templo de esos malvados que nunca cesan en su necedad».
\par 17 «Aceptaron nuestra presencia de palabra, pero de manera poco sincera de hecho, porque cuando les propusimos entrar en su templo interior y honrarlo con magníficas y bellísimas ofrendas»,
\par 18 «Se dejaron llevar por su tradicional vanidad y nos excluyeron de la entrada; pero se les evitó el ejercicio de nuestro poder debido a la benevolencia que tenemos hacia todos».
\par 19 «Al mantener su manifiesta mala voluntad hacia nosotros, se convierten en el único pueblo entre todas las naciones que mantienen la cabeza en alto desafiando a los reyes y a sus propios benefactores, y no están dispuestos a considerar sincera ninguna acción».
\par 20 «Pero nosotros, cuando llegamos victoriosos a Egipto, nos acomodamos a su locura e hicimos lo correcto, ya que tratamos a todas las naciones con benevolencia».
\par 21 «Entre otras cosas, les dimos a conocer a todos nuestra amnistía hacia sus compatriotas aquí, tanto por su alianza con nosotros como por los innumerables asuntos que les habían confiado generosamente desde el principio; y nos aventuramos a hacer un cambio, al decidir considerarlos dignos de la ciudadanía alejandrina y hacerlos partícipes de nuestros ritos religiosos regulares».
\par 22 «Pero, en su maldad innata, tomaron esto con espíritu contrario y desdeñaron lo que es bueno. Puesto que se inclinan constantemente al mal»,
\par 23 «No sólo desprecian la valiosa ciudadanía, sino que también abominan con la palabra y con el silencio a los pocos que entre ellos están sinceramente dispuestos hacia nosotros; En cada situación, de acuerdo con su infame forma de vida, sospechan en secreto que pronto podremos cambiar nuestra política».
\par 24 «Por lo tanto, plenamente convencidos por estos indicios de que están mal dispuestos hacia nosotros en todos los sentidos, hemos tomado precauciones para que, si luego surgiera contra nosotros un desorden repentino, tuviéramos a estas personas impías detrás de nuestras espaldas como traidores y enemigos bárbaros».
\par 25 «Por lo tanto, hemos dado orden de que, tan pronto como llegue esta carta, nos envíen a los que viven entre ustedes, junto con sus esposas e hijos, con insultos y malos tratos, y atados con grillos de hierro. , sufrir la muerte segura y vergonzosa que corresponde a los enemigos».
\par 26 «Porque cuando todos estos hayan sido castigados, estamos seguros de que durante el tiempo restante el gobierno se establecerá para nosotros en buen orden y en el mejor estado».
\par 27 «Pero quien albergue a algún judío, ya sea anciano, niño o incluso bebé, será torturado hasta la muerte con los tormentos más odiosos, junto con su familia».
\par 28 «Quien quiera dar información recibirá los bienes del castigado, además de dos mil dracmas del tesoro real, y se le concederá la libertad».
\par 29 «Todo lugar que se descubra albergando a un judío será hecho inaccesible y quemado con fuego, y quedará inútil para siempre para cualquier criatura mortal».
\par 30 La carta fue escrita en la forma anterior.

\chapter{4}

\par 1 Así pues, en cada lugar donde llegó este decreto, se organizó una fiesta pública para los gentiles con gritos y alegría, porque la enemistad inveterada que durante mucho tiempo había estado en sus mentes ahora se hacía evidente y abierta.
\par 2 Pero entre los judíos había continuo llanto, lamentos y llantos; por todas partes ardía su corazón y gemían a causa de la destrucción inesperada que de repente les había sido decretada.
\par 3 ¿Qué distrito, qué ciudad, qué lugar habitable o qué calles no se llenaron de luto y llanto por ellos?
\par 4 Porque los generales de las distintas ciudades los expulsaban con tal dureza y crueldad que, al ver sus extraordinarios castigos, incluso algunos de sus enemigos, al ver el objeto común de compasión ante ellos, sus ojos, reflexionaban sobre la incertidumbre de la vida y derramaban lágrimas ante la más miserable expulsión de estas personas.
\par 5 Porque se llevaban a una multitud de ancianos de cabellos grises, perezosos y encorvados por la edad, obligados a marchar a paso rápido por la violencia con que los empujaban de manera tan vergonzosa.
\par 6 Y las jóvenes que acababan de entrar en la cámara nupcial para compartir la vida matrimonial cambiaron la alegría por el llanto, sus cabellos perfumados de mirra se rociaron con cenizas, y se llevaron sin velo, todas juntas lanzando un lamento en lugar de un cántico nupcial, mientras estaban desgarrados por el duro trato de los paganos.
\par 7 Atados y a la vista del público fueron arrastrados violentamente hasta el lugar de embarque.
\par 8 Sus maridos, en la flor de la juventud, con el cuello ceñido con cuerdas en lugar de guirnaldas, pasaron los días restantes de su fiesta nupcial en lamentaciones en lugar de alegría y juerga juvenil, viendo la muerte inmediatamente ante ellos.
\par 9 Los subieron a bordo como animales salvajes, conducidos bajo ataduras de hierro; unos estaban atados por el cuello a los bancos de las barcas, otros tenían los pies asegurados con grilletes irrompibles,
\par 10 y además fueron confinados bajo una cubierta sólida, para que, con los ojos en completa oscuridad, sufrieran durante todo el viaje un tratamiento propio de los traidores.
\par 11 Cuando estos hombres fueron llevados al lugar llamado Schedia, y el viaje concluyó como lo había decretado el rey, ordenó que los encerraran en el hipódromo que había sido construido con un monstruoso muro perimetral frente a la ciudad. , y que era muy adecuado para convertirlos en un espectáculo evidente para todos los que regresaban a la ciudad y para los de la ciudad que salían al campo, de modo que no podían comunicarse con las fuerzas del rey ni de ninguna manera pretender estar dentro del circuito de la ciudad.
\par 12 Cuando esto sucedió, el rey, al enterarse de que los compatriotas judíos de la ciudad salían con frecuencia en secreto a lamentar amargamente la innoble desgracia de sus hermanos,
\par 13 ordenó en su ira que estos hombres fueran tratados exactamente de la misma manera que los demás, sin omitir ningún detalle de su castigo.
\par 14 Toda la raza debía ser registrada individualmente, no para los duros trabajos que antes hemos mencionado brevemente, sino para ser torturada con los ultrajes que él había ordenado, y al final ser destruida en el espacio de un solo día.
\par 15 Por lo tanto, el registro de estas personas se llevó a cabo con amarga prisa y celoso celo desde la salida del sol hasta su puesta, y aunque no se completó, se detuvo después de cuarenta días.
\par 16 El rey estaba siempre lleno de alegría, organizando fiestas en honor de todos sus ídolos, con una mente alejada de la verdad y con una boca profana, alabando a cosas mudas que ni siquiera pueden comunicarse ni ayudar. , y pronunciar palabras inapropiadas contra el Dios supremo.
\par 17 Pero después del intervalo de tiempo antes mencionado, los escribas declararon al rey que ya no podían hacer el censo de los judíos a causa de su innumerable multitud,
\par 18 aunque la mayoría de ellos todavía estaban en el campo, algunos todavía residiendo en sus casas, y otros en el lugar; la tarea era imposible para todos los generales de Egipto.
\par 19 Después de haberlos amenazado severamente, acusándolos de haber sido sobornados para idear una forma de escapar, quedó claramente convencido del asunto.
\par 20 cuando dijeron y demostraron que tanto el papel como los bolígrafos que usaban para escribir ya se habían agotado.
\par 21 Pero esto fue un acto de la providencia invencible de aquel que ayudaba a los judíos desde el cielo.

\chapter{5}

\par 1 Entonces el rey, totalmente inflexible, se llenó de ira y de ira abrumadoras; Entonces llamó a Hermón, guardián de los elefantes,
\par 2 y le ordenó que al día siguiente drogara a todos los elefantes, quinientos en total, con grandes puñados de incienso y mucho vino sin mezclar, y los arrojara dentro, enloquecidos por la abundancia de licor, para que los judíos podrían encontrarse con su perdición.
\par 3 Después de haber dado estas órdenes, volvió a su banquete, junto con sus amigos y los del ejército que eran especialmente hostiles hacia los judíos.
\par 4 Y Hermón, guardián de los elefantes, procedió fielmente a cumplir las órdenes.
\par 5 Los criados que estaban a cargo de los judíos salieron por la tarde, ataron las manos del desdichado pueblo y dispusieron que continuaran bajo custodia durante la noche, convencidos de que toda la nación experimentaría su destrucción final.
\par 6 Porque a los gentiles les pareció que los judíos se habían quedado sin ayuda alguna,
\par 7 porque en sus prisiones estaban confinados por la fuerza por todas partes. Pero con lágrimas y con una voz difícil de silenciar, todos invocaron al Señor Todopoderoso y Gobernante de todo poder, su Dios y Padre misericordioso, orando
\par 8 para que desvíe con venganza el malvado complot contra ellos y en una manifestación gloriosa los rescate del destino que ahora les espera.
\par 9 Y su súplica ascendió con fervor al cielo.
\par 10 Hermón, sin embargo, después de drogar a los despiadados elefantes hasta llenarlos con una gran cantidad de vino y saciarlos de incienso, se presentó en el patio temprano en la mañana para informar al rey sobre estos preparativos.
\par 11 Pero el Señor envió al rey una porción de sueño, ese beneficio que desde el principio, de noche y de día, concede quien lo concede a quien quiere.
\par 12 Y por la acción del Señor fue vencido por un sueño tan placentero y profundo que fracasó por completo en su propósito inicuo y quedó completamente frustrado en su inflexible plan.
\par 13 Entonces los judíos, habiendo escapado de la hora señalada, alabaron a su santo Dios y nuevamente rogaron al que se reconcilia fácilmente que mostrara el poder de su mano omnipotente a los arrogantes gentiles.
\par 14 Pero como ya era casi la mitad de la hora décima, el encargado de las invitaciones, al ver que los invitados estaban reunidos, se acercó al rey y le dio un codazo.
\par 15 Y cuando con dificultad lo despertó, le señaló que ya se estaba pasando la hora del banquete, y le contó lo sucedido.
\par 16 El rey, considerando esto, volvió a beber y ordenó a los presentes en el banquete que se sentaran frente a él.
\par 17 Una vez hecho esto, los instó a entregarse a la juerga y a alegrar aún más la porción presente del banquete celebrando aún más.
\par 18 Después de algún tiempo de la fiesta, el rey llamó a Hermón y, con fuertes amenazas, le preguntó por qué se había permitido que los judíos siguieran con vida hasta el día de hoy.
\par 19 Pero cuando él, con la confirmación de sus amigos, señaló que cuando aún era de noche había cumplido plenamente la orden que le habían dado,
\par 20 El rey, poseído por un salvajismo peor que el de Falaris, dijo que los judíos se habían beneficiado del sueño de hoy, «pero», añadió, «mañana sin demora prepara a los elefantes de la misma manera para la destrucción de los malvados. ¡Judíos!
\par 21 Cuando el rey hubo hablado, todos los presentes, unánimes y alegres, dieron su aprobación y se fueron cada uno a su casa.
\par 22 Pero no aprovechaban tanto la duración de la noche durmiendo como ideando toda clase de insultos para aquellos que creían condenados.
\par 23 Entonces, tan pronto como cantó el gallo, de madrugada, Hermón, después de equipar a las bestias, comenzó a hacerlas pasar por la gran columnata.
\par 24 La multitud de la ciudad se había reunido para este espectáculo tan lamentable y esperaban ansiosamente el amanecer.
\par 25 Pero los judíos, en su último suspiro, ya que se les había acabado el tiempo, extendieron sus manos hacia el cielo y con súplicas llenas de lágrimas y tristes endechas imploraron al Dios supremo que los ayudara nuevamente de inmediato.
\par 26 Los rayos del sol aún no habían salido, y mientras el rey recibía a sus amigos, llegó Hermón y lo invitó a salir, indicando que lo que el rey deseaba estaba listo para actuar.
\par 27 Pero él, al recibir el informe y sorprendido por la inusual invitación a salir, ya que estaba completamente incomprendido por la incomprensión, preguntó cuál era el motivo por el cual se había hecho esto con tanto celo para él.
\par 28 Este fue el acto de Dios, que gobierna todas las cosas, porque había implantado en la mente del rey el olvido de las cosas que antes había ideado.
\par 29 Entonces Hermón y todos los amigos del rey le dijeron que las bestias y las fuerzas armadas estaban preparadas: «Oh rey, conforme a tu anhelante propósito».
\par 30 Pero al oír estas palabras se llenó de una ira abrumadora, porque por la providencia de Dios toda su mente se había trastornado con respecto a estos asuntos; y con una mirada amenazadora dijo:
\par 31 «Si tus padres o tus hijos estuvieran presentes, yo los habría preparado como un rico banquete para las bestias salvajes en lugar de para los judíos, quienes no me dan motivo de queja y han demostrado en un grado extraordinario una lealtad plena y firme a mis ancestros.»
\par 32 «De hecho, te habrían privado de la vida en lugar de éstas, si no fuera por un afecto que surge de nuestra educación en común y de tu utilidad».
\par 33 Entonces Hermón sufrió una amenaza inesperada y peligrosa, y sus ojos vacilaron y su rostro decayó.
\par 34 Los amigos del rey, uno por uno, se fueron alejando malhumorados y despidieron al pueblo reunido, cada uno a su propia ocupación.
\par 35 Entonces los judíos, al oír lo que el rey había dicho, alabaron al Señor Dios manifiesto, Rey de reyes, ya que también ésta era la ayuda que habían recibido.
\par 36 Pero el rey convocó de nuevo a la fiesta de la misma manera e instó a los invitados a que volvieran a sus celebraciones.
\par 37 Después de llamar a Hermón, le dijo en tono amenazador: «¿Cuántas veces, pobre desgraciado, tendré que darte órdenes sobre estas cosas?»
\par 38 «¡Equipa a los elefantes ahora una vez más para la destrucción de los judíos mañana!»
\par 39 Pero los funcionarios que estaban a la mesa con él, maravillados de su inestabilidad mental, le protestaron lo siguiente:
\par 40 «Oh rey, ¿hasta cuándo nos juzgarás como si fuéramos idiotas, ordenando ahora por tercera vez que sean destruidos y revocando de nuevo tu decreto al respecto?»
\par 41 «Por lo tanto, la ciudad está alborotada a causa de su expectación; está repleta de masas de gente y también en constante peligro de ser saqueada».
\par 42 Ante esto, el rey, un Phalaris en todo y lleno de locura, no tuvo en cuenta los cambios de opinión que se habían producido en él para la protección de los judíos, y juró firmemente que los enviaría a morir sin demora, destrozado por las rodillas y los pies de las bestias,
\par 43 y también marcharía contra Judea y rápidamente la arrasaría con fuego y lanza, y quemando hasta los cimientos el templo inaccesible para él, rápidamente lo dejaría vacío para siempre de quienes allí ofrecían sacrificios.
\par 44 Entonces los amigos y los oficiales partieron con gran alegría y apostaron confiadamente las fuerzas armadas en los lugares de la ciudad más favorables para hacer guardia.
\par 45 Cuando los animales habían sido llevados prácticamente a un estado de locura, por así decirlo, por los tragos muy fragantes de vino mezclado con incienso y habían sido equipados con dispositivos espantosos, el cuidador de elefantes
\par 46 entró en el patio casi al amanecer (la ciudad estaba ahora llena de innumerables masas de gente que se apiñaban en el hipódromo) y apremió al rey sobre el asunto en cuestión.
\par 47 Entonces él, cuando su mente impía se llenó de una profunda ira, salió corriendo con todas sus fuerzas junto con las bestias, deseando presenciar con corazón invulnerable y con sus propios ojos la destrucción dolorosa y lamentable de aquel pueblo.
\par 48 Y cuando los judíos vieron el polvo que levantaban los elefantes que salían por la puerta y las tropas que los seguían, así como el pisoteo de la multitud, y oyeron el ruido fuerte y tumultuoso,
\par 49 Pensaron que éste era el último momento de su vida, el fin de su más miserable suspenso, y entre lamentos y gemidos se besaron, abrazaron a los parientes y se abrazaron unos a otros, padres e hijos, madres e hijas y otros con bebés al pecho que estaban extrayendo su última leche.
\par 50 No sólo esto, sino que considerando la ayuda que antes habían recibido del cielo, se postraron unánimes en tierra, sacando a los niños de sus pechos,
\par 51 y gritó a gran voz, implorando al Gobernante de todos los poderes que se manifestara y tuviera misericordia de ellos, que ahora se encontraban a las puertas de la muerte.

\chapter{6}

\par 1 Entonces un tal Eleazar, famoso entre los sacerdotes del país, que había llegado a una edad avanzada y durante toda su vida había sido adornado con todas las virtudes, ordenó a los ancianos que lo rodeaban que dejaran de invocar al Dios santo y oró de la siguiente manera :
\par 2 «Rey de gran poder, Dios Todopoderoso Altísimo, que gobierna toda la creación con misericordia»,
\par 3 «Mira, oh Padre, a la descendencia de Abraham, a los hijos del santo Jacob, pueblo de tu porción consagrada que perece como extranjeros en tierra extraña».
\par 4 «A Faraón, con su abundancia de carros, al antiguo gobernante de este Egipto, exaltado con insolencia sin ley y lengua jactanciosa, tú destruiste junto con su arrogante ejército ahogándolos en el mar, manifestando la luz de tu misericordia sobre la nación de Israel.»
\par 5 Senaquerib, exultante con sus innumerables fuerzas, rey opresor de los asirios, que ya había dominado el mundo entero con la lanza y se alzó contra tu santa ciudad, hablando palabras dolorosas con jactancia e insolencia, tú, oh Señor. , desmenuzado, mostrando tu poder a muchas naciones».
\par 6 «A los tres compañeros en Babilonia que voluntariamente entregaron sus vidas a las llamas para no servir a cosas vanas, los rescataste ilesos hasta un cabello, humedeciendo el horno de fuego con rocío y dirigiendo la llama contra todos sus enemigos. »
\par 7 «Daniel, a quien por calumnias envidiosas fue arrojado a la tierra para los leones como alimento para las fieras, tú lo sacaste ileso a la luz».
\par 8 «Y Jonás, consumido en el vientre de un enorme monstruo nacido en el mar, tú, Padre, cuidaste y devolviste ilesa a toda su familia».
\par 9 «Y ahora, tú que odias la insolencia, todo misericordioso y protector de todos, revélate rápidamente a los de la nación de Israel, que están siendo tratados atrozmente por los gentiles abominables y sin ley».
\par 10 «Aunque nuestras vidas se hayan enredado en impiedades en nuestro exilio, líbranos de la mano del enemigo y destrúyenos, Señor, por el destino que elijas».
\par 11 «No dejes que los vanidosos alabe su vanidad ante la destrucción de tu amado pueblo, diciendo: «Ni siquiera su dios los ha librado»».
\par 12 «Pero tú, oh Eterno, que tienes todo poder y todo poder, cuídanos ahora y ten misericordia de nosotros, que por la insensata insolencia de los malvados estamos siendo privados de la vida a la manera de los traidores».
\par 13 «Y que hoy los gentiles se encojan ante tu poder invencible, oh honrado, que tienes poder para salvar a la nación de Jacob».
\par 14 «Toda la multitud de niños y sus padres os suplican con lágrimas».
\par 15 «Que se manifieste a todos los gentiles que tú, oh Señor, estás con nosotros y que no has vuelto de nosotros tu rostro; pero tal como has dicho: «Ni siquiera cuando estaban en la tierra de sus enemigos los descuidé», así cumplelo, oh Señor».
\par 16 Justo cuando Eleazar estaba terminando su oración, el rey llegó al hipódromo con las bestias y toda la arrogancia de sus fuerzas.
\par 17 Y cuando los judíos vieron esto, lanzaron grandes gritos al cielo, de modo que incluso los valles cercanos resonaron con ellos y trajeron un terror incontrolable sobre el ejército.
\par 18 Entonces el Dios más glorioso, todopoderoso y verdadero reveló su santo rostro y abrió las puertas del cielo, de las cuales descendieron dos ángeles gloriosos de aspecto temible, visibles para todos, excepto para los judíos.
\par 19 Se enfrentaron a las fuerzas del enemigo y las llenaron de confusión y terror, atadas con grilletes inamovibles.
\par 20 Incluso el rey comenzó a estremecerse y se olvidó de su sombría insolencia.
\par 21 Las bestias se volvieron hacia las fuerzas armadas que las seguían y comenzaron a pisotearlas y destruirlas.
\par 22 Entonces la ira del rey se transformó en compasión y lágrimas por las cosas que había ideado de antemano.
\par 23 Porque cuando escuchó los gritos y vio que todos caían de cabeza hacia la destrucción, lloró y amenazó enojado a sus amigos, diciendo:
\par 24 «Estáis cometiendo traición y superando en crueldad a los tiranos; e incluso a mí, vuestro benefactor, ahora intentáis privarme del dominio y de la vida, ideando en secreto actos que no benefician al reino.
\par 25 «¿Quién es el que ha sacado a cada hombre de su casa y ha reunido aquí sin sentido a los que fielmente han defendido las fortalezas de nuestro país?»
\par 26 «¿Quién es el que ha tratado tan ilegalmente con escandalosos tratos a aquellos que desde el principio diferían de todas las naciones en su buena voluntad hacia nosotros y que a menudo aceptaron voluntariamente los peores peligros humanos?»
\par 27 «¡Desatad y desatad sus ataduras injustas! ¡Envíalos de regreso a sus hogares en paz, pidiendo perdón por tus acciones anteriores!
\par 28 «Liberen a los hijos del Dios todopoderoso y viviente del cielo, que desde el tiempo de nuestros antepasados ​​hasta ahora ha otorgado una estabilidad notable y sin obstáculos a nuestro gobierno».
\par 29 Estas fueron, pues, las cosas que dijo; y los judíos, inmediatamente liberados, alabaron a su santo Dios y Salvador, ya que ahora habían escapado de la muerte.
\par 30 Entonces el rey, cuando regresó a la ciudad, llamó al funcionario encargado de las rentas y le ordenó que proporcionara a los judíos vinos y todo lo necesario para una fiesta de siete días, y decidió que celebrarían su rescatar con toda alegría en ese mismo lugar en el que esperaban encontrar su destrucción.
\par 31 Por lo tanto, los que estaban a sus puertas, maltratados y próximos a la muerte, o mejor dicho, que estaban a sus puertas, organizaron un banquete de liberación en lugar de una muerte amarga y lamentable, y llenos de alegría repartieron entre los celebrantes el lugar que había sido preparado para su destrucción y entierro.
\par 32 Dejaron de cantar endechas y retomaron el cántico de sus padres, alabando a Dios, su Salvador y hacedor de maravillas. Poniendo fin a todo luto y llanto, formaron coros en señal de alegría pacífica.
\par 33 Asimismo también el rey, después de convocar un gran banquete para celebrar estos acontecimientos, dio gracias al cielo sin cesar y abundantemente por el inesperado rescate que había experimentado.
\par 34 Y aquellos que antes habían creído que los judíos serían destruidos y se convertirían en pasto de pájaros, y lo habían registrado con alegría, gemían al verse vencidos por la desgracia, y su audacia que escupía fuego fue apagada ignominiosamente.
\par 35 Pero los judíos, cuando ya habían organizado el grupo coral antes mencionado, como ya hemos dicho, se entretuvieron en un banquete al son de alegres acciones de gracias y salmos.
\par 36 Y después de haber ordenado un rito público para estas cosas en toda su comunidad y para sus descendientes, instituyeron la observancia de los días antes mencionados como una fiesta, no para la bebida y la glotonería, sino a causa de la liberación que había llegado a ellos a través de Dios.
\par 37 Entonces rogaron al rey que les permitiera regresar a sus casas.
\par 38 Su registro se hizo desde el día veinticinco de Pachón hasta el cuarto de Epeif, durante cuarenta días; y su destrucción fue fijada del quinto al séptimo de Epeiph, los tres días
\par 39 En el cual el Señor de todos reveló gloriosamente su misericordia y los rescató a todos juntos y sanos y salvos.
\par 40 Luego comieron, provistos de todo por el rey, hasta el día catorce, en el que también pidieron su despido.
\par 41 El rey accedió inmediatamente a su petición y les escribió la siguiente carta a los generales de las ciudades, expresando magnánimamente su preocupación:

\chapter{7}

\par 1 «El rey Ptolomeo Filopátor a los generales en Egipto y a todos los que tienen autoridad en su gobierno, saludos y buena salud».
\par 2 «Nosotros y nuestros hijos estamos bien, el gran Dios guía nuestros asuntos según nuestro deseo».
\par 3 «Algunos de nuestros amigos, instándonos frecuentemente con intenciones maliciosas, nos persuadieron a reunir a los judíos del reino en un cuerpo y castigarlos con penas bárbaras como traidores»;
\par 4 «porque declararon que nuestro gobierno nunca estaría firmemente establecido hasta que esto se lograra, debido a la mala voluntad que este pueblo tenía hacia todas las naciones».
\par 5 «También los sacaron con duros tratos como esclavos, o más bien como traidores, y, ceñiéndose con una crueldad más salvaje que la costumbre escita, intentaron matarlos sin ninguna investigación ni examen».
\par 6 «Pero los amenazamos severamente por estos actos y, de acuerdo con la clemencia que tenemos para con todos los hombres, apenas les perdonamos la vida. Desde que nos hemos dado cuenta de que el Dios del cielo seguramente defiende a los judíos, poniéndose siempre de su parte como lo hace un padre con sus hijos»,
\par 7 «y como hemos tenido en cuenta la amistosa y firme buena voluntad que tenían hacia nosotros y nuestros antepasados, con justicia los hemos absuelto de todo cargo de cualquier tipo».
\par 8 «También hemos ordenado a cada uno que regrese a su casa, sin que nadie en ningún lugar les haga daño alguno ni les reproche las cosas irracionales que han sucedido».
\par 9 «Porque debéis saber que si ideamos algún mal contra ellos o les causamos algún dolor, siempre tendremos no al hombre, sino al Soberano de todos los poderes, al Dios Altísimo, en todo e ineludiblemente como antagonista de vengar tales actos. Despedida.»
\par 10 Al recibir esta carta, los judíos no se apresuraron a partir inmediatamente, sino que pidieron al rey que por sus propias manos aquellos de la nación judía que habían transgredido voluntariamente contra el Dios santo y la ley de Dios recibieran la castigo que merecían.
\par 11 Porque declararon que aquellos que por causa del vientre habían transgredido los mandamientos divinos nunca serían favorables al gobierno del rey.
\par 12 Entonces el rey, admitiendo y aprobando la verdad de lo que decían, les concedió licencia general para que libremente y sin autoridad ni supervisión real pudieran destruir a los que en todo su reino habían transgredido la ley de Dios.
\par 13 Cuando lo aplaudieron como era debido, sus sacerdotes y toda la multitud gritaron aleluya y se marcharon gozosos.
\par 14 Y en el camino castigaban y condenaban a muerte pública y vergonzosa a todos los compatriotas que encontraban que se habían contaminado.
\par 15 Ese día mataron a más de trescientos hombres; y celebraron el día como una fiesta alegre, ya que habían destruido a los profanadores.
\par 16 Pero los que se habían aferrado a Dios hasta la muerte y habían recibido el pleno disfrute de la liberación, comenzaron a salir de la ciudad, coronados con toda clase de flores muy fragantes, dando gracias con alegría y en voz alta al único Dios de sus padres. , el eterno Salvador de Israel, con palabras de alabanza y toda clase de cánticos melodiosos.
\par 17 Cuando llegaron a Tolemaida, llamada «la de las rosas» por una característica del lugar, la flota los esperó, de acuerdo con el deseo común, durante siete días.
\par 18 Allí celebraron su liberación, porque el rey les había proporcionado generosamente todo lo necesario para el viaje, a cada uno hasta su casa.
\par 19 Y cuando desembarcaron en paz con la debida acción de gracias, también allí decidieron celebrar estos días como una fiesta alegre durante el tiempo de su estancia.
\par 20 Luego, después de inscribirlos como santos en una columna y dedicar un lugar de oración en el lugar de la fiesta, partieron ilesos, libres y llenos de alegría, ya que por orden del rey habían sido llevados sanos y salvos por tierra y mar y río cada uno a su lugar.
\par 21 También tenían mayor prestigio entre sus enemigos, siendo respetados y reverenciados; y no estaban sujetos en absoluto a la confiscación de sus pertenencias por parte de nadie.
\par 22 Además, todos recuperaron todos sus bienes, según el registro, de modo que quienes los poseían se los devolvieron con gran temor. De modo que el Dios supremo realizó perfectamente grandes obras para su liberación.
\par 23 ¡Bendito sea el Libertador de Israel en todos los tiempos! Amén.

\end{document}