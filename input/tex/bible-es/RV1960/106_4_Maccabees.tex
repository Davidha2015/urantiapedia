\begin{document}

\title{4 Macabeos}

\chapter{1}

\par \textit{Un resumen de la filosofía de la antigüedad sobre la Razón Inspirada. La civilización nunca ha alcanzado un pensamiento superior. Una discusión sobre «Represiones». El versículo 48 resume toda la Filosofía de la humanidad.}

\par 1 FILOSÓFICO en el más alto grado es la cuestión que me propongo discutir, a saber, si la Razón Inspirada es gobernante suprema sobre las pasiones; y en cuanto a su filosofía, le ruego seriamente que preste mucha atención.

\par 2 Porque la materia no sólo es generalmente necesaria como rama del conocimiento, sino que también incluye la alabanza de las mayores virtudes, es decir, el dominio de uno mismo.

\par 3 Es decir, si se demuestra que la razón controla las pasiones adversas a la templanza, la glotonería y la lujuria, también se muestra claramente que es dueña de las pasiones, como la malevolencia, opuestas a la justicia, y de las que se oponen a la virilidad, es decir, rabia, dolor y miedo.

\par 4 Pero algunos se preguntarán: si la razón es dueña de las pasiones, ¿por qué no controla el olvido y la ignorancia? su objetivo es ridiculizar.

\par 5 La respuesta es que la razón no es dueña de los defectos inherentes a la mente misma, sino de las pasiones o defectos morales que son adversos a la justicia, la virilidad, la templanza y el juicio; y su acción en su caso no es extirpar las pasiones, sino permitirnos resistirlas con éxito.

\par 6 Podría presentarles muchos ejemplos, tomados de diversas fuentes, en los que la razón ha demostrado ser dueña de las pasiones, pero el mejor ejemplo, con diferencia, que puedo dar es la noble conducta de aquellos que murieron por causa de la virtud. Eleazar, los Siete Hermanos y la Madre.

\par 7 Porque todos ellos, con su desprecio de los dolores, incluso hasta la muerte, demostraron que la razón se eleva por encima de las pasiones.

\par 8 Podría extenderme aquí en alabanza de sus virtudes, ellos, los hombres con la Madre, que mueren en este día que celebramos por amor a la belleza y la bondad moral, pero más bien quisiera felicitarlos por los honores que han alcanzado.

\par 9 Porque la admiración que sentían por su coraje y resistencia, no sólo el mundo en general sino también sus mismos verdugos, los convirtió en los autores de la caída de la tiranía bajo la cual se encontraba nuestra nación, derrotando al tirano con su resistencia. para que a través de ellos fuera purificada su patria.

\par 10 Pero ahora aprovecharé la oportunidad para discutir esto, después de haber comenzado con la teoría general, como tengo por costumbre hacerlo, y luego procederé a su historia, dando gloria al Dios omnisapiente.

\par 11 Nuestra pregunta, entonces, es si la razón es dueña suprema de las pasiones.

\par 12 Pero debemos definir exactamente qué es la razón y qué es la pasión, y cuántas formas de pasión hay, y si la razón es suprema sobre todas ellas.

\par 13 Considero que la razón es la mente que prefiere con clara deliberación la vida de sabiduría.

\par 14 Considero que la sabiduría es el conocimiento de las cosas divinas y humanas y de sus causas.

\par 15 Considero que esta es la cultura adquirida bajo la Ley, mediante la cual aprendemos con la debida reverencia las cosas de Dios y para nuestro beneficio mundano las cosas de los hombres.

\par 16 Ahora bien, la sabiduría se manifiesta en forma de juicio, de justicia, de valentía y de templanza.

\par 17 Pero el juicio o el dominio de sí es el que domina a todos, porque a través de él, en verdad, la razón afirma su autoridad sobre las pasiones.

\par 18 Pero entre las pasiones hay dos fuentes generales, a saber, el placer y el dolor, y ambas pertenecen esencialmente tanto al alma como al cuerpo.

\par 19 Y tanto en el placer como en el dolor hay muchos casos en que las pasiones tienen ciertas secuencias.

\par 20 Así, mientras que el deseo precede al placer, le sigue la satisfacción, y mientras el miedo precede al dolor, después del dolor viene la tristeza.

\par 21 La ira, además, si un hombre quiere seguir el curso de sus sentimientos, es una pasión en la que se mezclan placer y dolor.

\par 22 Bajo el placer se produce también la degradación moral que manifiesta la más amplia variedad de pasiones.

\par 23 Se manifiesta en el alma como ostentación, avaricia, vanagloria, contienda y calumnia, y en el cuerpo como comida extraña, glotonería y glotonería en secreto.

\par 24 Ahora bien, siendo el placer y el dolor como dos árboles que crecen del cuerpo y del alma, de estas pasiones brotan muchos retoños; y la Razón de cada hombre como maestro jardinero, que desyerba, poda, venda, abre el agua y la dirige de aquí para allá, domestica la maraña de disposiciones y pasiones.

\par 25 Porque, si bien la razón es guía de las virtudes, es dueña de las pasiones.

\par 26 Observemos ahora, en primer lugar, que la razón llega a ser suprema sobre las pasiones en virtud de la acción inhibidora de la templanza.

\par 27 La templanza, según creo, es la represión de los deseos; pero de los deseos algunos son mentales y otros físicos, y ambos tipos están claramente controlados por la Razón; Cuando nos sentimos tentados por las carnes prohibidas, ¿cómo llegamos a renunciar a los placeres que se derivan de ellas?

\par 28 ¿No es que la razón tiene poder para reprimir los apetitos? En mi opinión es así.

\par 29 Por lo tanto, cuando sentimos el deseo de comer animales acuáticos, aves, bestias y carnes de cualquier tipo que la Ley nos prohíbe, nos abstenemos por predominio de la razón.

\par 30 Porque la mente templada controla e inhibe las propensiones de nuestros apetitos, y todos los movimientos del cuerpo obedecen al freno de la razón.

\par 31 ¿Y de qué hay que sorprenderse si se apaga el deseo natural del alma de disfrutar del fruto de la belleza?

\par 32 Por eso ciertamente alabamos al virtuoso José, porque con su razón, con un esfuerzo mental, refrenó el impulso carnal. 1 Porque él, joven en la edad en que el deseo físico es fuerte, con su razón apagó el impulso de sus pasiones.

\par 33 Y está demostrado que la razón domina no sólo el impulso del deseo sexual, sino también toda clase de codicia.

\par 34 Porque la ley dice: «No codiciarás la mujer de tu prójimo ni nada que sea de tu prójimo».

\par 35 En verdad, cuando la Ley nos ordena no codiciar, creo que debería confirmar firmemente el argumento de que la Razón es capaz de controlar los deseos codiciosos, así como lo hace con las pasiones que militan contra la justicia.

\par 36 ¿De qué otra manera se puede enseñar a un hombre, naturalmente glotón, codicioso y borracho, a cambiar de naturaleza, si la razón no es manifiestamente dueña de las pasiones?

\par 37 Ciertamente, tan pronto como un hombre ordena su vida según la Ley, si es avaro actúa contra su naturaleza y presta dinero a los necesitados sin interés, y al séptimo año cancela la deuda.

\par 38 Y si es parsimonioso, la Ley lo domina por la acción de la razón y se abstiene de recoger sus rastrojos o de recoger las últimas uvas de sus viñedos.

\par 39 Y respecto de todo lo demás podemos reconocer que la razón está en posición de dueña de las pasiones o afecciones.

\par 40 Porque la Ley está por encima del amor a los padres, para que el hombre no pueda renunciar a su virtud por ellos, y prevalece sobre el amor a la esposa, de modo que ella transgrede al hombre para reprenderla, y rige el amor a los hijos. de modo que si son malos, un hombre debería castigarlos, y controla los derechos de la amistad, de modo que un hombre debería reprender a sus amigos si hacen el mal.

\par 41 Y no penséis que es paradójico que la razón, mediante la ley, sea capaz de vencer incluso el odio, de modo que un hombre se abstenga de talar los huertos del enemigo, proteja la propiedad del enemigo de los saqueadores y recoja sus bienes. bienes que han sido dispersos.

\par 42 Y también se ha demostrado que el dominio de la razón se extiende a través de las pasiones o vicios más agresivos: la ambición, la vanidad, la ostentación, el orgullo y la calumnia.

\par 43 Porque la mente templada repele todas estas pasiones degradadas, al igual que la ira, pues incluso ésta vence.

\par 44 Y Moisés, cuando se enojó contra Datán y Abiram, no dio rienda suelta a su ira, sino que controló su ira con su razón.

\par 45 Porque la mente templada puede, como dije, vencer las pasiones, modificando unas y aplastando absolutamente otras.

\par 46 ¿Por qué, si no, nuestro sabio padre Jacob culpó a las casas de Simeón y Leví por su irracional matanza de la tribu de los siquemitas, diciendo: «¡Maldita sea su ira!»

\par 47 Pues si la razón no hubiera tenido el poder de contener su ira, no habría hablado así.

\par 48 Porque el día en que Dios creó al hombre, implantó en él sus pasiones e inclinaciones, y también, al mismo tiempo, puso la mente en un trono en medio de los sentidos para que fuera su guía sagrada en todas las cosas; y a la mente le dio la Ley, por la cual si un hombre se ordena, reinará sobre un reino templado, justo, virtuoso y valiente.

\par \textit{Notas al pie}

\par \textit{179:1 Véase El Testamento de José, página 260.}

\chapter{2}

\par \textit{El dominio del Deseo y la Ira. La historia de la sed de David. Capítulos conmovedores de la historia antigua. Intentos salvajes de hacer que los judíos coman cerdos. Referencias interesantes a un banco antiguo (Versículo 21.)}

\par 1 BIEN entonces, alguien se preguntará, si la Razón es dueña de las pasiones ¿por qué no lo es del olvido y de la ignorancia?

\par 2 Pero el argumento es sumamente ridículo. Porque la razón no se muestra dueña de las pasiones o defectos en sí misma, sino de las del cuerpo.

\par 3 Por ejemplo, ninguno de vosotros puede extirpar nuestro deseo natural, pero la razón puede permitirle escapar de ser esclavo del deseo.

\par 4 Ninguno de vosotros puede extirpar la ira del alma, pero es posible que la razón acuda en su ayuda contra la ira.

\par 5 Ninguno de vosotros puede extirpar una disposición malévola, pero la Razón puede ser su poderoso aliado contra la influencia de la malevolencia.

\par 6 La razón no es la extirpación de las pasiones, sino su antagonista.

\par 7 El caso de la sed del rey David puede servir al menos para aclarar esto.

\par 8 Porque David, después de haber luchado durante todo el día contra los filisteos y haber matado a muchos de ellos con la ayuda de los guerreros de nuestro país, llegó al anochecer, fatigado de sudor y de trabajo, a la tienda real, alrededor de la cual Estaba acampado todo el ejército de nuestros antepasados.

\par 9 Entonces todo el ejército se dispuso a cenar; pero el rey, consumido por una intensa sed, aunque tenía mucha agua, no pudo saciarla.

\par 10 En cambio, un deseo irracional por el agua que estaba en posesión del enemigo con creciente intensidad lo quemó, lo desmanteló y lo consumió.

\par 11 Entonces, cuando su guardia personal murmuró contra el deseo del rey, dos jóvenes, valientes guerreros, avergonzados de que su rey no pudiera satisfacer sus deseos, se vistieron con todas sus armas, tomaron un recipiente con agua y escalaron las murallas enemigas. ; y, pasando desapercibidos por los guardias de la puerta, registraron todo el campamento enemigo.

\par 12 Y valientemente encontraron el manantial y sacaron de él agua potable para el rey.

\par 13 Pero David, aunque todavía ardía de sed, consideró que semejante bebida, considerada como equivalente a sangre, era un grave peligro para su alma.

\par 14 Por eso, oponiendo su razón a su deseo, derramó el agua como ofrenda a Dios.

\par 15 Porque la mente templada es capaz de vencer los dictados de las pasiones, apagar el fuego del deseo y luchar victoriosamente con los dolores de nuestro cuerpo, aunque sean extremadamente fuertes, y por la belleza moral y la bondad de la razón. desafiar con desprecio todo dominio de las pasiones.

\par 16 Y ahora la ocasión nos llama a exponer la historia de la Razón autocontrolada.

\par 17 En un tiempo en que nuestros padres gozaban de gran paz gracias a la debida observancia de la ley y estaban en buena situación, de modo que Seleuco Nicanor, rey de Asia, sancionó el impuesto para el servicio del templo y reconoció nuestra política, Precisamente entonces, ciertos hombres, actuando en facciones contra la concordia general, nos involucraron en muchas y diversas calamidades.

\par 18 Onías, hombre de gran carácter, siendo entonces sumo sacerdote y teniendo el oficio de su vida, un tal Simón levantó una facción contra él, pero como a pesar de toda clase de calumnias no logró perjudicarlo a causa del pueblo , huyó al extranjero con la intención de traicionar a su país.

\par 19 Entonces vino a Apolonio, gobernador de Siria, Fenicia y Cilicia, y le dijo: «Siendo leal al rey, estoy aquí para informarte que en los tesoros de Jerusalén se guardan muchos miles de depósitos privados, que no pertenecen a la cuenta del templo, y por derecho propiedad del rey Seleuco.

\par 20 Apolonio, tras investigar los detalles del asunto, elogió a Simón por su leal servicio al rey y, corriendo a la corte de Seleuco, le reveló el valioso tesoro; luego, después de recibir autoridad para tratar el asunto, rápidamente marchó a nuestro país, acompañado por el maldito Simón y un ejército muy poderoso, y anunció que estaba allí por orden del rey para tomar posesión de los depósitos privados en el tesoro.

\par 21 Nuestro pueblo se enojó profundamente con este anuncio y protestó enérgicamente, considerando que era un ultraje que aquellos que habían confiado sus depósitos al tesoro del templo fueran despojados de ellos, y pusieron todos los obstáculos posibles en su camino.

\par 22 Pero Apolonio, con amenazas, entró en el templo.

\par 23 Entonces los sacerdotes en el templo, las mujeres y los niños rogaron a Dios que viniera en ayuda de su Lugar Santo que estaba siendo violado; y cuando Apolonio con su hueste armada entró para apoderarse de los dineros, aparecieron del cielo ángeles montados a caballo, con relámpagos brillando en sus brazos, y les infundieron gran temor y temblor.

\par 24 Y Apolonio cayó medio muerto en el atrio de los gentiles, extendió sus manos al cielo y, entre lágrimas, suplicó a los hebreos que intercedieran por él y calmaran la ira de las huestes celestiales.

\par 25 Porque decía que había pecado y que era digno incluso de la muerte, y que si le dieran la vida alabaría ante todos los hombres la bienaventuranza del Lugar Santo.

\par 26 Conmovido por estas palabras, el sumo sacerdote Onías, aunque muy escrupuloso en otros casos, intercedió por él para que el rey Seleuco no pensara que Apolonio había sido derrocado por un designio humano y no por la justicia divina.

\par 27 Por lo tanto, Apolonio, después de su sorprendente liberación, partió para contar al rey lo que le había sucedido.

\par 28 Pero al morir Seleuco, le sucedió en el trono su hijo Antíoco Epífanes, un hombre arrogante y terrible; quien despidió a Onías de su sagrado oficio y en su lugar nombró sumo sacerdote a su hermano Jasón, con la condición de que, a cambio del nombramiento, Jasón le pagara tres mil seiscientos sesenta talentos al año.

\par 29 Entonces nombró a Jasón sumo sacerdote y lo hizo jefe del pueblo.

\par 30 Y él (Jasón) introdujo a nuestro pueblo una nueva forma de vida y una nueva constitución en total desafío a la Ley; de modo que no sólo instaló un gimnasio en el Monte de nuestros padres, sino que incluso abolió el servicio del templo.

\par 31 Por eso la justicia divina se encendió en ira y puso al mismo Antíoco como enemigo contra nosotros.

\par 32 Para cuando. Estaba librando una guerra contra Ptolomeo en Egipto y oyó que los habitantes de Jerusalén se habían alegrado mucho por la noticia de su muerte, y de inmediato volvió contra ellos.

\par 33 Y después de saquear la ciudad, promulgó un decreto condenando a muerte a cualquiera que pareciera vivir según la ley de nuestros padres.

\par 34 Pero encontró que todos sus decretos eran inútiles para quebrantar la fidelidad de nuestro pueblo a la Ley, y vio todas sus amenazas y castigos completamente despreciados, de modo que incluso las mujeres por circuncidar a sus hijos, aunque sabían de antemano lo que les sucedería, sea ​​su destino, fueron arrojados, junto con su descendencia, de cabeza desde las rocas.

\par 35 Por lo tanto, cuando sus decretos continuaron siendo despreciados por la masa del pueblo, él personalmente trató de obligar mediante torturas a cada hombre por separado a comer carnes inmundas y así abjurar de la religión judía.

\par 36 Entonces el tirano Antíoco, acompañado de sus consejeros, se sentó a juzgar en un lugar alto, con sus tropas dispuestas a su alrededor y con armadura completa, y ordenó a sus guardias que arrastraran allí a todos los hebreos y los obligaran. comer carne de cerdo y cosas sacrificadas a los ídolos; pero si alguno se negaba a contaminarse con cosas inmundas, debía ser torturado y ejecutado.

\par 37 Y cuando muchos fueron tomados por la fuerza, uno de los primeros fue llevado ante Antíoco, un hebreo llamado Eleazar, sacerdote de nacimiento, versado en el conocimiento de la ley, un hombre de avanzada edad y de buena salud. Conocido por muchos miembros de la corte del tirano por su filosofía.

\par 38 Y Antíoco, mirándolo, dijo: 'Antes de que permita que comiencen las torturas para ti, oh venerable hombre, te daría este consejo: que comas carne de cerdo y salves tu vida; porque respeto tu edad y tus canas, aunque haberlas llevado durante tanto tiempo y seguir aferrándote a la religión judía me hace pensar que no eres un filósofo.

\par 39 'Porque la carne de este animal que la naturaleza nos ha concedido es excelente, ¿y por qué deberíais abominarla? En verdad, es una locura no disfrutar de placeres inocentes y es un error rechazar los favores de la Naturaleza.

\par 40 «Pero creo que sería una locura aún mayor por tu parte si, con vanas vaporizaciones sobre la verdad, procedieras a desafiar incluso a mí a tu propio castigo».

\par 41 '¿No despertarás de tu absurda filosofía? ¿No dejarás a un lado las tonterías de tus cálculos y, adoptando otro estado de ánimo acorde con tu edad madura, aprenderás la verdadera filosofía de la conveniencia y cómo seguir mis caritativos consejos, y tendrás piedad de tu venerable edad?

\par 42 «Porque considerad esto también: incluso si hay algún Poder que tenga los ojos puestos en esta religión vuestra, siempre os perdonará por una transgresión cometida bajo coacción».

\par 43 autobús instado por el tirano a comer ilegalmente carne inmunda, Eleazar pidió permiso para hablar; y al recibirlo, comenzó su discurso ante el tribunal de la siguiente manera:

\par 44 'Nosotros, oh Antíoco, habiendo aceptado la Ley Divina como la Ley de nuestro país, no creemos que se nos imponga ninguna necesidad más fuerte que la de nuestra obediencia a la Ley.'

\par 45 'Por lo tanto, ciertamente no lo consideramos correcto. de cualquier manera para transgredir la Ley.'

\par 46 «Sin embargo, si nuestra Ley, como sugieres, no fuera verdaderamente divina, mientras que en vano creyéramos que es divina, ni siquiera así sería correcto que destruyéramos nuestra reputación de piedad».

\par 47 'No penséis, pues, que es un pecado pequeño el que comamos algo inmundo, porque la transgresión de la Ley, ya sea en cosas pequeñas o en grandes, es igualmente atroz; porque en cualquier caso igualmente se desprecia la Ley.'

\par 48 «Y os mofáis de nuestra filosofía, como si según ella viviéramos de forma contraria a la razón».

\par 49 'No es así, porque la Ley nos enseña dominio propio, de modo que seamos dueños de todos nuestros placeres y deseos y estemos entrenados completamente en la virilidad para soportar todo dolor con prontitud; y enseña justicia, de modo que con todas nuestras diversas disposiciones actuemos justamente, y enseña rectitud, de modo que con la debida reverencia adoremos sólo al Dios que es.'

\par 50 'Por tanto, no comeremos carne inmunda; por creer que nuestra Ley es dada por Dios, sabemos también que el Creador del mundo, como Legislador, siente por nosotros según nuestra naturaleza.'

\par 51 «Él nos ha ordenado comer lo que sea conveniente para nuestra alma, y ​​nos ha prohibido comer carnes que serían lo contrario».

\par 52 «Pero es un acto de tirano que nos obligues no sólo a transgredir la Ley, sino que también nos hagas comer de tal manera que puedas burlarte de esta contaminación que es tan abominable para nosotros».

\par 53 «Pero no os burlaréis de mí de esta manera, ni romperé los juramentos sagrados de mis antepasados ​​de guardar la Ley, ni aunque me arranques los ojos y me quemes las entrañas».

\par 54 'No soy tan débil por la vejez que, cuando la justicia está en juego, la fuerza de la juventud vuelve a mi Razón.'

\par 55 'Así que gira con fuerza tus parrillas y calienta más tu horno. No me compadezco tanto de mi vejez como para violar la Ley de mis padres en mi propia persona.'

\par 56 'No te desmentiré, oh Ley que fuiste mi maestra; No te abandonaré, oh amado dominio propio; No te avergonzaré, oh Razón amante de la sabiduría, ni te negaré, oh venerado sacerdocio y conocimiento de la Ley.'

\par 57 'Tampoco mancillarás la boca pura de mi vejez y mi constancia de toda la vida a la Ley. Limpio me recibirán mis padres, sin temer tus tormentos hasta la muerte.'

\par 58 «Porque ciertamente puedes ser un tirano sobre los hombres injustos, pero no dominarás mi resolución en materia de justicia, ni con tus palabras ni con tus obras».

\chapter{3}

\par \textit{Eleazar, el anciano de espíritu amable, muestra tal fortaleza que incluso cuando leemos estas palabras 2000 años después, parecen un fuego inextinguible.}

\par 1 PERO cuando Eleazar respondió tan elocuentemente a las exhortaciones de los tiranos, los guardias que lo rodeaban lo arrastraron bruscamente al lugar de tortura.

\par 2 Y primero desnudaron al anciano, que estaba adornado con la belleza de la santidad.

\par 3 Luego, atándole los brazos a ambos lados, lo azotaron, mientras un heraldo estaba de pie frente a él y gritaba: «¡Obedece las órdenes del rey!».

\par 4 Pero el hombre noble y de gran alma, en verdad Eleazar, no se conmovió más en su mente que si estuviera siendo atormentado en un sueño; sí, el anciano, manteniendo sus ojos firmemente elevados al cielo, sufrió que los azotes desgarraran su carne hasta que quedó bañado en sangre y sus costados se convirtieron en una masa de heridas; e incluso cuando cayó al suelo porque su cuerpo ya no podía soportar el dolor, mantuvo su Razón erguida e inflexible.

\par 5 Con el pie, uno de los guardias de las vinagreras, mientras caía, le dio una patada salvaje en el costado para obligarlo a levantarse.

\par 6 Pero él soportó la angustia, despreció la coacción, soportó los tormentos y, como un valiente atleta que soporta el castigo, el anciano superó a sus verdugos.

\par 7 El sudor le cubría la frente y respiraba entrecortadamente, hasta que su nobleza de alma arrancó la admiración de sus propios verdugos.

\par 8 Entonces, en parte compadecidos por su vejez, en parte por simpatía hacia su amigo, en parte admirados por su valentía, algunos de los cortesanos del rey se acercaron a él y le dijeron:

\par 9 '¿Por qué, oh Eleazar, te destruyes locamente en esta miseria? Te traeremos carne cocida, pero finge sólo comer carne de cerdo y así sálvate.

\par 10 Y Eleazar, como si su consejo no hiciera más que aumentar sus tormentos, gritó en voz alta: 'No. Que nosotros, los hijos de Abraham, nunca tengamos un pensamiento tan malo como el de falsificar, con un corazón descorazonado, una parte indecorosa para nosotros.'

\par 11 'Contrariamente a la razón, en verdad, si fuera para nosotros, después de vivir según la verdad hasta la vejez y guardar bajo apariencia legítima la reputación de vivir así, cambiar ahora y convertirnos en nuestras propias personas en un modelo para los jóvenes de la humanidad. impiedad, hasta el punto de incitarlos a comer carne inmunda.'

\par 12 «Sería vergonzoso que viviéramos un poco más, mientras todos los hombres se burlaban de ellos por cobardía y, mientras el tirano los despreciaba como poco varoniles, no defendiéramos la Ley Divina hasta la muerte».

\par 13 'Por tanto, hijos de Abraham, morid con nobleza por causa de la justicia; pero en cuanto a vosotros, oh servidores del tirano, ¿por qué detenéis vuestro trabajo?

\par 14 Entonces ellos, al verlo triunfante sobre las torturas e impasible incluso ante la compasión de sus verdugos, lo arrastraron al fuego.

\par 15 Allí lo arrojaron sobre él, lo quemaron con artimañas crueles y le echaron en la nariz un caldo maloliente.

\par 16 Pero cuando el fuego ya le había llegado a los huesos y estaba a punto de extinguir, alzó los ojos a Dios y dijo:

\par 17 'Tú, oh Dios, sabes que, aunque pueda salvarme, moriré en tormentos de fuego por tu ley. Ten misericordia de tu pueblo y deja que nuestro castigo sea una satisfacción para ellos. Haz de mi sangre su purificación, y toma mi alma para rescatar sus almas'.

\par 18 «Y con estas palabras el santo varón entregó noblemente su espíritu bajo el tormento que yo y por el bien de la Ley impuesta por su razón incluso contra los tormentos hasta la muerte».

\par 19 Sin lugar a dudas, entonces, la Razón Inspirada domina las pasiones; porque si sus pasiones o sufrimientos hubieran prevalecido sobre su razón, les habríamos atribuido esta evidencia de su poder superior.

\par 20 Pero ahora que su razón ha vencido sus pasiones, con razón le atribuimos el poder de dominarlas.

\par 21 Y es justo que admitamos que el dominio recae en la razón, al menos en los casos en que vence dolores que vienen de fuera de nosotros mismos; porque sería ridículo negarlo.

\par 22 Y mi prueba no sólo abarca la superioridad de la razón sobre los dolores, sino también su superioridad sobre los placeres; tampoco se entrega a ellos.

\chapter{4}

\par \textit{Esta llamada «Era de la Razón» puede leerse en este capítulo que la Filosofía de la Razón tiene 2000 años. La historia de siete hijos y su madre.}

\par 1 POR LA Razón de nuestro padre Eleazar, como un excelente timonel que dirige el barco de la santidad en el mar de las pasiones, aunque azotado por las amenazas del tirano y arrastrado por las crecientes olas de las torturas, nunca se movió ni un solo momento. el timón de la santidad hasta navegar hacia el puerto de la victoria sobre la muerte.

\par 2 Ninguna ciudad asediada con muchas y astutas máquinas jamás se defendió tan bien como aquel santo varón cuando su alma sagrada fue atacada con azotes, tormentos y llamas, y conmovió a los que asediaban su alma mediante su Razón que era el escudo de la santidad.

\par 3 Porque nuestro padre Eleazar, fijando su película mental como un escarpado acantilado, rompió el loco inicio de las oleadas de las pasiones.

\par 4 Oh sacerdote digno de tu sacerdocio, no contaminaste tus santos dientes, ni contaminaste con carne inmunda tu vientre, en el que sólo había lugar para la piedad y la pureza.

\par 5 ¡Oh confesor de la Ley y filósofo de la vida divina! Tales deben ser aquellos cuyo oficio es servir la Ley y defenderla con su propia sangre y sudor honorable ante los sufrimientos hasta la muerte.

\par 6 Tú, oh padre, fortaleciste nuestra fidelidad a la Ley con tu firmeza hasta la gloria; y habiendo hablado en honor de la santidad no desmentiste tu discurso, y confirmaste las palabras de la filosofía divina con tus hechos, oh anciano que fuiste más contundente que las torturas.

\par 7 ¡Oh reverendo anciano que eras más tenso que la llama, tú, gran rey sobre las pasiones, Eleazar!

\par 8 Porque así como nuestro padre Aarón, armado con el incensario, corrió entre la multitud contra el ángel de fuego y lo venció, así el hijo de Aarón, Eleazar, consumido por el calor del fuego, permaneció inquebrantable en su razón. .

\par 9 Y, sin embargo, lo más maravilloso de todo es que él, ya viejo, con los tendones de su cuerpo debilitados, sus músculos relajados y sus nervios debilitados, volvió a crecer como un hombre joven en el espíritu de su razón y con la razón de Isaac. convirtió la tortura de la cabeza de hidra en impotencia.

\par 10 ¡Oh edad bendita, oh reverenda cabeza gris, oh vida fiel a la Ley y perfeccionada por el sello de la muerte!

\par 11 Ciertamente, pues, si un anciano despreciaba los tormentos hasta la muerte por causa de la justicia, hay que admitir que la Razón Inspirada es capaz de guiar las pasiones.

\par 12 Pero quizá algunos respondan que no todos los hombres son dueños de las pasiones, porque no todos tienen la razón iluminada.

\par 13 Pero todos los que de todo corazón ponen como primer pensamiento la justicia, sólo éstos pueden dominar la debilidad de la carne, creyendo que para Dios no mueren, como no murieron nuestros patriarcas Abraham, Isaac y Jacob. sino que vivan para Dios.

\par 14 Por lo tanto, no hay nada contradictorio en que algunas personas parezcan esclavas de la pasión debido a la debilidad de su razón.

\par 15 Porque, ¿quién hay que, siendo un filósofo que sigue rectamente todas las reglas de la filosofía, y habiendo puesto su confianza en Dios, y sabiendo que es una cosa bienaventurada soportar todas las durezas por causa de la virtud, no venza sus pasiones? ¿Por causa de la justicia?

\par 16 Porque sólo el hombre sabio y sobrio es el valiente gobernante de las pasiones.

\par 17 De esta manera, incluso los jóvenes, siendo filósofos en virtud de la razón que es conforme a la justicia, han triunfado sobre tormentos aún más dolorosos.

\par 18 Porque cuando el tirano se vio notablemente derrotado en su primer intento y incapaz de obligar a un anciano a comer carne inmunda, entonces, verdaderamente enfurecido, ordenó a los guardias que trajeran a otros de los jóvenes hebreos, y si comían carne inmunda para liberarlos después de comerla, pero si se negaban, los torturaban aún más salvajemente.

\par 19 Y bajo estas órdenes del tirano fueron llevados prisioneros ante él siete hermanos junto con su anciana madre, todos hermosos, modestos, bien nacidos y, en general, atractivos.

\par 20 Y cuando el tirano los vio allí, de pie como si fueran un coro de fiesta, con su madre en medio, se fijó en ellos y, impresionado por su porte noble y distinguido, les sonrió y, llamándolos más cerca, les dijo: :

\par 21 'Oh jóvenes, deseo el bien a cada uno de vosotros, y admiro vuestra belleza, y honro mucho a tan numeroso grupo de hermanos; Así que no sólo os aconsejo que no persistáis en la locura de ese viejo que ya ha sufrido, sino que incluso os ruego que os entreguéis a mí y os hagáis partícipes de mi amistad.'

\par 22 'Porque así como puedo castigar a los que desobedecen mis órdenes, así puedo hacer avanzar a los que me obedecen.'

\par 23 'Tengan la seguridad, entonces, de que se les darán puestos de importancia y autoridad a mi servicio si rechazan la ley ancestral de su gobierno.'

\par 24 'Participa en la vida helénica, camina por un nuevo camino y disfruta de tu juventud; porque si me enojáis con vuestra desobediencia, me obligaréis a recurrir a penas terribles y a matar a cada uno de vosotros mediante tortura.'

\par 25 'Tened, pues, compasión de vosotros mismos, a quienes incluso yo, vuestro adversario, me compadezco de vuestra juventud y de vuestra belleza.'

\par 26 «¿No pensaréis en esto de que si me desobedecéis, no os espera más que muerte en tormentos?»

\par 27 Dicho esto, ordenó que les acercaran los instrumentos de tortura para inducirles por miedo a comer carne inmunda.

\par 28 Pero cuando los guardias hubieron sacado ruedas, dislocadores, potros, trituradoras de huesos, catapultas, calderos, braseros, empulgueras, garras de hierro, cuñas y hierros para marcar, los El tirano volvió a hablar y dijo:

\par 29 «Será mejor que sintáis miedo, muchachos, y la justicia que adoráis perdonará vuestra transgresión involuntaria».

\par 30 Pero ellos, al oír sus persuasiones y ver sus terribles máquinas, no sólo no mostraron miedo, sino que incluso desplegaron su filosofía contra el tirano, y con su derecho la razón rebajó su tiranía.

\par 31 Y, sin embargo, considera; Suponiendo que algunos de ellos hubieran sido pusilánimes y cobardes, ¿qué tipo de lenguaje habrían usado? ¿(ella) no habría sido en este sentido?

\par 32 '¡Ay! ¡Miserables criaturas que somos y necios sin medida! Cuando el rey nos invite y nos solicite un trato amable, ¿no le obedeceremos?

\par 33 '¿Por qué nos animamos con deseos vanos y nos atrevemos a una desobediencia que nos costará la vida? ¿No deberíamos, oh hermanos míos, temer los terribles instrumentos, sopesar bien sus amenazas de tortura y abandonar estas alardes vacías y esta fatal jactancia?

\par 34 'Tengamos compasión de nuestra juventud y tengamos compasión de la edad de nuestra madre; y tomemos en serio que si desobedecemos, moriremos.'

\par 35 'Y también la justicia divina tendrá misericordia de nosotros, si por necesidad nos sometemos al rey con miedo. ¿Por qué deberíamos desechar esta querida vida y privarnos de este dulce mundo?'

\par 36 'No luchemos contra la necesidad ni invitemos con vana confianza a nuestra tortura.'

\par 37 «Ni siquiera la ley misma nos condena voluntariamente a muerte, pues tenemos miedo de los instrumentos de tortura».

\par 38 «¿Por qué nos inflama tanta discordia y una terquedad fatal encuentra favor entre nosotros, cuando podríamos tener una vida pacífica obedeciendo al rey?»

\par 39 Pero a estos jóvenes no se les escaparon tales palabras ante la perspectiva de la tortura, ni tales pensamientos entraron en sus mentes.

\par 40 Porque despreciaban las pasiones y dominaban el dolor.

\chapter{5}

\par \textit{Un capítulo de horror y tortura que revela la antigua tiranía en su máximo salvajismo. El versículo 26 es una verdad profunda.}

\par 1 Y tan pronto como el tirano terminó de instarlos a comer carne inmunda, todos a una voz y como un solo alma le dijeron:

\par 2 '¿Por qué te demoras, oh tirano? Estamos dispuestos a morir antes que transgredir los mandamientos de nuestros padres.'

\par 3 «Porque también avergonzaríamos a nuestros antepasados, si no andáramos en obediencia a la Ley y no tomáramos a Moisés como nuestro consejero».

\par 4 'Oh tirano que nos aconsejas transgredir la Ley, no nos odies, odiándonos, no te compadezcas más allá de nosotros mismos.'

\par 5 'Porque estimamos tu misericordia, que es generosa. darnos la vida a cambio de una infracción de la Ley, algo más difícil de soportar que la muerte misma.'

\par 6 «Nos aterrorizarías con tus amenazas de muerte bajo tortura, como si hace poco no hubieras aprendido nada de Eleazar».

\par 7 «Pero si los ancianos de los hebreos soportaron los tormentos por causa de la justicia, sí, hasta morir, más dignamente moriremos nosotros, los jóvenes, despreciando los tormentos de tu compulsión, sobre la cual también triunfó nuestro anciano maestro».

\par 8 'Por tanto, prueba, oh tirano. Y si nos quitas la vida por causa de la justicia, no pienses que nos haces daño con tus tormentos.'

\par 9 'Porque a través de este mal trato y de nuestra resistencia ganaremos el premio de la virtud; pero tú, por nuestro cruel asesinato, sufrirás a manos de la justicia divina suficiente tormento de fuego para siempre.

\par 10 Estas palabras de los jóvenes redoblaron la ira del tirano, no sólo por su desobediencia sino también por lo que él consideraba su ingratitud.

\par 11 Entonces, siguiendo sus órdenes, los azotadores trajeron al mayor de ellos, lo despojaron de su manto y le ataron las manos y los brazos a ambos lados con correas.

\par 12 Pero después de azotarle hasta cansarlo, sin conseguir nada con ello, le echaron sobre la rueda.

\par 13 Y en él el noble joven fue torturado hasta que se le descoyuntaron los huesos. Y mientras cedían una y otra vez, denunció al tirano con estas palabras:

\par 14 «Oh, abominable tirano, enemigo de la justicia del cielo y sanguinario, así me atormentas, no por homicidio ni por impiedad, sino por defender la ley de Dios».

\par 15 Y cuando los guardias le dijeron: «Consiente en comer, para que así puedas ser liberado de tus tormentos», él les dijo: «Vuestro método, oh miserables servidores, no es lo suficientemente fuerte como para cautivar a mi Razón. Cortad mis miembros, quemad mis carnes y retorcid mis coyunturas; A través de todos los tormentos os mostraré que en favor de la virtud sólo los hijos de los hebreos son invencibles.'

\par 16 Mientras hablaba así, le pusieron encima carbones encendidos y, intensificando el tormento, lo apretaron aún más sobre la rueda.

\par 17 Y toda la rueda quedó manchada con su sangre, y las brasas amontonadas fueron apagadas por los humores de su cuerpo que caían, y la carne desgarrada corrió alrededor de los ejes de la máquina.

\par 18 Y con su cuerpo ya disuelto, este joven de gran alma, como un verdadero hijo de Abraham, no gimió en absoluto; pero como si sufriera un cambio por fuego a la incorrupción, soportó noblemente el tormento, diciendo:

\par 19 'Sed mi ejemplo, oh hermanos. No me abandonéis para siempre y no renunciéis a nuestra hermandad por nobleza de alma.

\par 20 'La guerra, una guerra santa y honorable en nombre de la justicia, mediante la cual la justa Providencia que cuidó a nuestros padres se vuelva misericordiosa con su pueblo y se vengue del maldito tirano.'

\par 21 Y con estas palabras el santo joven entregó el espíritu.

\par 22 Pero mientras todos se maravillaban de su constancia de alma, los guardias trajeron al segundo en edad del. hijos, y agarrándolo con manos de hierro con garras afiladas, lo sujetaron a los motores y a la catapulta.

\par 23 Pero cuando escucharon su noble decisión en respuesta a su pregunta: «¿Preferiría comer antes que torturar?» Estas bestias parecidas a panteras desgarraron sus tendones con garras de hierro, le arrancaron toda la carne de las mejillas y le arrancaron la piel de la cabeza.

\par 24 Pero él, soportando con firmeza esta agonía, dijo: «¡Cuán dulce es toda forma de muerte por la justicia de nuestros padres!».

\par 25 Y al tirano le dijo: «Oh, el más despiadado de los tiranos, ¿no te parece que en este momento tú mismo sufres tormentos peores que los míos al ver que la intención arrogante de tu tiranía es superada por mi resistencia por causa de la justicia?»

\par 26 'Porque yo soy sostenido en el dolor por las alegrías que vienen por la virtud, mientras que tú estás atormentado mientras te glorias en tu impiedad; Tampoco escaparás, oh abominable tirano, de las penas de la ira divina.

\par 27 Y cuando hubo afrontado valientemente su muerte gloriosa, apareció el tercer hijo, y muchos le rogaron encarecidamente que lo probara y así se salvara.

\par 28 Pero él respondió en voz alta: «¿Ignoráis que un mismo padre nos engendró a mí y a mis hermanos muertos, y que una misma madre nos dio a luz, y que yo fui criado en las mismas doctrinas?»

\par 29 'No renuncio al noble vínculo de la hermandad.'

\par 30 'Por tanto, si tenéis algún instrumento de tormento, aplícalo a este cuerpo mío; porque mi alma no podéis alcanzarla, aunque quisierais.'

\par 31 Pero ellos se enojaron mucho por la audacia del hombre y le dislocaron las manos y los pies con sus máquinas de dislocación, le arrancaron los miembros de sus órbitas y los desataron; y se retorcieron alrededor de sus dedos, brazos, piernas y codos.

\par 32 Y como no podían ahogar su espíritu, le arrancaron la piel, tomando consigo las puntas de los dedos, le arrancaron el cuero cabelludo a la manera escita y en seguida lo llevaron a la rueda.

\par 33 Y entonces le torcieron la columna hasta que vio su propia carne colgando en tiras y grandes chorros de sangre brotando de sus entrañas.

\par 34 Y a punto de morir dijo: 'Nosotros, oh abominable tirano, sufrimos así por nuestra educación y nuestra virtud que son de Dios; pero tú, por tu impiedad y tu crueldad, sufrirás tormentos sin fin.'

\par 35 Y cuando éste murió dignamente de sus hermanos, trajeron al cuarto y le dijeron: «No te vuelvas loco con la misma locura que tus hermanos, sino obedece al rey y sálvate a ti mismo».

\par 36 Pero él les dijo: «Para mí no tenéis ningún fuego tan ardiente que me haga cobarde».

\par 37 «Por la bendita muerte de mis hermanos, por la perdición eterna del tirano y por la vida gloriosa de los justos, no negaré mi noble hermandad».

\par 38 «Inventa torturas, oh tirano, para que así sepas que soy hermano de los que ya han sido torturados».

\par 39 Al oír esto, Antíoco, el sanguinario, asesino y absolutamente abominable, les ordenó que le cortaran la lengua.

\par 40 Pero él dijo: «Aunque me quites el órgano del habla, Dios también oye a los mudos».

\par 41 «He aquí, saqué mi lengua lista: córtala, porque con ello no silenciarás mi razón».

\par 42 Con gusto entregamos nuestros miembros corporales para que sean mutilados por la causa de Dios.

\par 43 'Pero Dios pronto te perseguirá; porque cortaste la lengua que le cantaba canciones de alabanza.'

\par 44 Pero cuando también este hombre fue condenado a muerte agonizante con los tormentos, el quinto se adelantó diciendo: «No tengo miedo, oh tirano, de exigir el tormento por amor a la virtud».

\par 45 «Sí, por mí mismo me acerco, para que, matándome también a mí, puedas aumentar con aún más delitos la pena que debes a la justicia del Cielo».

\par 46 'Oh enemigo de la virtud y enemigo del hombre, ¿por qué crimen nos destruyes de esta manera?'

\par 47 '¿Te parece mal que adoremos al Creador de todo y vivamos según su virtuosa Ley?'

\par 48 «Pero estas cosas son dignas de honores, no de tormentos, si comprendes las aspiraciones humanas y tienes esperanza de salvación ante Dios».

\par 49 'He aquí, ahora eres enemigo de Dios y haces la guerra a los que adoran a Dios.'

\par 50 Mientras hablaba así, los guardias lo ataron y lo llevaron ante la catapulta; y lo ataron a él sobre sus rodillas, y, sujetándolos allí con abrazaderas de hierro, le retorcieron los lomos sobre la 'cuña' rodante de modo que quedó completamente enroscado hacia atrás como un escorpión y todas las articulaciones quedaron descoyuntadas.

\par 51 Y así, con grave falta de aliento y angustia corporal, exclamó: 'Gloriosos, oh tirano, gloriosos contra tu voluntad son los beneficios que me otorgas, permitiéndome mostrar mi fidelidad a la Ley a través de torturas aún más honorables. .'

\par 52 Y cuando también éste estaba muerto, trajeron al sexto, un simple niño, quien, respondiendo a la pregunta del tirano si quería comer y ser liberado, dijo:

\par 53 'No soy tan viejo en años como mis hermanos, pero soy tan viejo en mente. Porque nacimos y crecimos con el mismo propósito y estamos igualmente obligados a morir por la misma causa; Así que si eliges torturarnos por no comer carne inmunda, tortura.'

\par 54 Mientras hablaba estas palabras, lo llevaron a la rueda, lo estiraron con cuidado, le dislocaron los huesos de la espalda y le prendieron fuego debajo.

\par 55 Le hicieron pinchos afilados al rojo vivo, se los clavaron en la espalda y, atravesándole los costados, le quemaron también las entrañas.

\par 56 Pero él, en medio de sus tormentos, exclamó: «¡Oh lucha digna de santos, en la que tantos de nosotros, hermanos, por la causa de la justicia, hemos sido presentados a una competencia de tormentos y no hemos sido vencidos!»

\par 57 Porque el justo entendimiento, oh tirano, es invencible.

\par 58 Con la armadura de la virtud voy a unirme a mis hermanos en la muerte y a añadir en mí un fuerte vengador más para castigarte, oh autor de las torturas y enemigo de los verdaderamente justos.

\par 59 Nosotros, seis jóvenes, hemos derrocado tu tiranía. 'Porque tu impotencia para alterar nuestra Razón u obligarnos a comer carne inmunda, ¿no es un derrocamiento para ti?'

\par 60 'Tu fuego es frío para nosotros, tus máquinas de tortura no atormentan y tu violencia es impotente.'

\par 61 'Porque los guardias han sido para nosotros oficiales, no de un tirano, sino de la Ley Divina; y por lo tanto tenemos nuestra Razón aún invicta.'

\chapter{6}

\par \textit{Lazos fraternales y amor de madre.}

\par 1 Y cuando éste también murió benditamente, siendo arrojado en el caldero, se adelantó el séptimo hijo, el menor de todos.

\par 2 Pero el tirano, aunque furiosamente irritado por sus hermanos, sintió lástima del muchacho y, al verlo ya atado, hizo que lo acercaran y trató de persuadirlo, diciéndole:

\par 3 'Tú ves el fin de la necedad de tus hermanos; porque por su desobediencia han sido atormentados hasta la muerte. Tú también, si no obedeces, también serás miserablemente torturado y ejecutado antes de tiempo; pero si obedeces, serás mi amigo y ascenderás a un alto cargo en los negocios del reino.

\par 4 Y mientras le pedía esto, mandó llamar a la madre del niño, para que en su dolor por la pérdida de tantos hijos pudiera instar al superviviente a obedecer y salvarse.

\par 5 Pero la madre, hablando en lengua hebrea, como diré más adelante, animó al niño, y éste dijo a los guardias: 'Desatadme, para que pueda hablar con el rey y con todos sus amigos. '

\par 6 Y ellos, regocijados por la petición del muchacho, se apresuraron a soltarlo.

\par 7 Y corriendo hacia el brasero al rojo vivo, exclamó: «¡Oh tirano impío!», y el más impío de todos los pecadores, ¿no te avergüenzas de tomar tus bendiciones y tu realeza en manos de Dios y matar? sus siervos y torturar a los seguidores de la justicia?'

\par 8 «Por lo cual la justicia divina te entrega a un fuego más rápido y eterno y a tormentos que no te dejarán presa por toda la eternidad».

\par 9 «¿No te avergüenza, siendo hombre, oh desgraciado con corazón de fiera, tomar contigo hombres de sentimientos similares, hechos de los mismos elementos, y arrancarles la lengua, y azotarlos y torturarlos en ¿de esta manera?'

\par 10 'Pero mientras ellos hayan cumplido su justicia hacia Dios en sus nobles muertes, tú llorarás miserablemente: «¡Ay!» por tu injusto asesinato de los campeones de la virtud.'

\par 11 Y estando al borde de la muerte, dijo: «No soy un renegado del testimonio de mis hermanos».

\par 12 «Y yo invoco al Dios de mis padres para que tenga misericordia de mi nación».

\par 13 «Y él te castigará tanto en esta vida presente como después de que estés muerto».

\par 14 Y con esta oración se arrojó en el brasero al rojo vivo, y así entregó el espíritu.

\par 15 Si, pues, los siete hermanos despreciaron los tormentos hasta la muerte, está universalmente demostrado que la razón inspirada es señor supremo sobre las pasiones.

\par 16 Porque si hubieran cedido a sus pasiones o sufrimientos y hubieran comido carne inmunda, habríamos dicho que por eso habían sido vencidos.

\par 17 Pero en esta cámara no fue así; al contrario, por su razón, alabada ante los ojos de Dios, se elevaron por encima de sus pasiones.

\par 18 Y es imposible negar la supremacía de la mente; porque obtuvieron la victoria sobre sus pasiones y sus dolores.

\par 19 ¿Cómo podemos hacer otra cosa que admitir el dominio de la razón recta sobre la pasión en estos hombres que no retrocedieron ante las agonías de la quema?

\par 20 Porque así como las torres de los muelles del puerto rechazan los embates de las olas y ofrecen una entrada tranquila a quienes entran en el puerto, así la recta Razón de los jóvenes, de siete torres, defendió el puerto de la justicia y rechazó la tempestad de las pasiones. .

\par 21 Formaron un coro santo de justicia y se animaban unos a otros, diciendo:

\par 22 «Muramos como hermanos, oh hermanos, por la ley».

\par 23 'Imitemos a los Tres Niños de la corte asiria que despreciaban esta misma prueba del horno.'

\par 24 «No nos acobardemos ante la prueba de la justicia».

\par 25 Y uno dijo: «Hermano, ten buen ánimo», y otro: «Compórtalo con nobleza»; y otro recordando el pasado: 'Recordad de qué estirpe sois, y a cuyas manos paternales se entregó Isaac por causa de la justicia para ser un sacrificio'.

\par 26 Y todos juntos, mirándose unos a otros con gran valentía y valentía, dijeron: «Con todo el corazón nos consagraremos a Dios, que nos dio nuestras almas, y prestaremos nuestros cuerpos para guardar la salvación». Ley.'

\par 27 'No temamos al que piensa matar; porque una gran lucha y peligro del alma aguarda en tormento eterno a quienes transgreden la ordenanza de Dios.'

\par 28 «Armémonos, pues, del dominio de las pasiones por parte de la Razón divina».

\par 29 «Después de esta nuestra pasión, nos recibirán Abraham, Isaac y Jacob, y todos nuestros antepasados ​​nos alabarán».

\par 30 Y a cada uno de los hermanos, mientras eran arrastrados, los que aún estaban por llegar, dijeron: «No nos deshonres, hermano, ni seas mentiroso con nuestros hermanos ya muertos».

\par 31 No ignoráis el amor fraternal, que la divina y sabia Providencia ha dado en herencia a los que son engendrados por sus padres, implantándolos desde el vientre de su madre; en el que los hermanos habitan en el mismo período, y toman su forma durante el mismo tiempo, y se alimentan de la misma sangre, y son vivificados con la misma alma, y ​​son traídos al mundo después del mismo espacio, y extraen leche del las mismas fuentes, por las cuales sus almas fraternas son alimentadas juntas en brazos junto al pecho; y están aún más unidos a través de una crianza común, un compañerismo diario y otra educación, y a través de nuestra disciplina bajo la Ley de Dios.

\par 32 Como el sentimiento de amor fraternal era así naturalmente fuerte, la concordia mutua de los siete hermanos se fortaleció aún más. Porque educados en la misma ley, disciplinados en las mismas virtudes y educados juntos en la vida recta, se amaban más abundantemente unos a otros. Su celo común por la belleza y la bondad morales aumentó su concordia mutua, porque junto con su piedad hizo que su amor fraternal fuera más ferviente.

\par 33 Pero aunque la naturaleza, el compañerismo y su carácter virtuoso aumentaron el ardor de su amor fraternal, los hijos supervivientes, a través de su religión, soportaron la visión de sus hermanos, que estaban en el tormento, siendo torturados hasta la muerte; es más, incluso los animaron a afrontar la agonía, para no sólo despreciar sus propias torturas, sino también conquistar su pasión de afecto fraternal por sus hermanos.

\par 34 ¡Oh mentes racionales, más reales que los reyes, más libres que los hombres libres, de la armonía de los siete hermanos, santos y bien sintonizados con la nota clave de la piedad!

\par 35 Ninguno de los siete jóvenes se volvió cobarde, ninguno se encogió ante la muerte, sino que todos se apresuraron a la muerte mediante la tortura como si recorrieran el camino hacia la inmortalidad.

\par 36 Porque así como las manos y los pies se mueven en armonía con los impulsos del alma, así aquellos santos jóvenes, como impulsados ​​por el alma inmortal de la religión, fueron en armonía a la muerte por ella.

\par 37 ¡Oh santísima séptuple compañía de hermanos en armonía!

\par 38 Pues, como los siete días de la creación del mundo destruyeron la religión, así los jóvenes, como un coro, interrumpieron su séptuple compañía, y dejaron sin importancia el terror de las torturas.

\par 39 Ahora nos estremecemos cuando escuchamos el sufrimiento de aquellos jóvenes; pero ellos, no sólo viéndolo con sus ojos, ni simplemente escuchando la amenaza inminente, sino que sintiendo realmente el dolor, lo soportaron; y en el tormento del fuego, ¿qué mayor agonía se puede encontrar?

\par 40 Porque agudo y riguroso es el poder del fuego, y rápidamente disolvió sus cuerpos.

\par 41 Y no os sorprenda que en aquellos hombres la razón triunfara sobre los tormentos, cuando incluso el alma de una mujer despreciaba una diversidad aún mayor de dolores; porque la madre de los siete jóvenes soportó los tormentos infligidos a cada uno de sus hijos.

\par 42 Pero considera cuán múltiples son los anhelos del corazón de una madre, de modo que su sentimiento por su descendencia se convierte en el centro de todo su mundo; y de hecho,

Aquí, incluso los animales irracionales sienten por sus crías un afecto y un amor similar al de los hombres.

\par 43 Por ejemplo, entre los pájaros, los mansos que se refugian bajo nuestros techos defienden a sus polluelos; y los que anidan en las cimas de las montañas, en las grietas de las rocas, en los agujeros de los árboles y en las ramas, y allí crían a sus crías, también ahuyentan al intruso.

\par 44 Y luego, si no pueden ahuyentarlo, revolotean apasionadamente alrededor de los polluelos, llamándolos con sus propias palabras, y socorren a sus polluelos en todo lo que pueden.

\par 45 ¿Y qué necesidad tenemos de ejemplos del amor a la descendencia entre los animales irracionales, cuando incluso las abejas, en la época en que se hace el panal, rechazan a los intrusos y los apuñalan con su aguijón, como con una espada? ¿Quiénes se acercan a sus descendientes y luchan contra ellos hasta la muerte?

\par 46 Pero ella, la madre de aquellos jóvenes, con un alma como Abraham, no se desvió de su propósito por el amor que sentía por sus hijos.

\chapter{7}

\par \textit{Una comparación de los afectos de una madre y un padre, en este capítulo hay algunos picos de elocuencia.}

\par 1 ¡RAZÓN de los hijos, señor de las pasiones! ¡Oh religión, esa era más querida para la madre que sus hijos!

\par 2 La madre, teniendo ante sí dos opciones, la religión y la salvación presente de sus siete hijos, según la promesa del tirano, amó más bien la religión, que salva para vida eterna según Dios.

\par 3 ¿Cómo puedo expresar el amor apasionado de los padres por los hijos? Imprimimos una semejanza maravillosa de nuestra alma y de nuestra forma en la tierna naturaleza del niño y, sobre todo, en la simpatía de la madre por sus hijos, que es más profunda que la del padre.

\par 4 Porque las mujeres son más blandas de alma que los hombres, y cuanto más hijos tienen, más abundan en su amor por ellos.

\par 5 Pero de todas las madres, ella de los siete hijos abundaba en amor más que las demás, ya que, habiendo sentido en siete partos ternura maternal por el fruto de su vientre, y habiendo sido constreñida por los muchos dolores en a los que tenía a cada uno con un gran afecto, sin embargo, por temor de Dios rechazó la seguridad presente de sus hijos.

\par 6 Sí, y más que eso, a través de la belleza moral y la bondad de sus hijos y su obediencia a la Ley, su amor maternal por ellos se hizo más fuerte.

\par 7 Porque eran justos, sobrios, valientes y de gran alma, y ​​amaban tanto el uno al otro como a su madre, que la obedecieron en la observancia de la Ley hasta la muerte.

\par 8 Sin embargo, aunque tuvo tantas tentaciones de ceder a sus instintos maternales, en ningún caso la terrible variedad de torturas tuvo poder para alterar su Razón; pero la madre instó a cada hijo por separado y a todos juntos a morir por su religión.

\par 9 ¡Oh naturaleza santa, y amor paternal, y anhelo de los padres por tener descendencia, y salario de la crianza, y cariño invencible de las madres!

\par 10 La madre, al verlos uno por uno atormentados y quemados, permaneció imperturbable en su alma por causa de la religión.

\par 11 Vio la carne de sus hijos consumida en el fuego, y las extremidades de sus manos y pies esparcidas por el suelo, y la cubierta de carne arrancada de sus cabezas hasta sus mejillas, esparcida como máscaras.

\par 12 ¡Oh madre, que ahora conocía dolores más agudos que los dolores del parto! ¡Oh mujer, única entre las mujeres, cuyo fruto fue la religión perfecta!

\par 13 Tu primogénito, al entregar el espíritu, no alteró tu resolución, ni el segundo, que te miró con ojos compasivos bajo sus tormentos, ni el tercero, que exhaló su espíritu.

\par 14 Tampoco lloraste cuando contemplaste los ojos de cada uno en medio de los tormentos mirando con valentía la misma angustia, y viste en sus temblorosas narices los signos de la muerte próxima.

\par 15 Cuando viste la carne de un hijo cortada tras la carne de otro, y mano tras mano cortada, y cabeza tras cabeza desollada, y cadáver arrojado sobre cadáver, y el lugar lleno de espectadores a causa de la tormentos de tus hijos, no derramarás ni una lágrima.

\par 16 Ni las melodías de las sirenas ni los cantos de los cisnes con dulce sonido encantan tanto los oídos del oyente como sonaban las voces de los hijos, hablando a la madre en medio de los tormentos.

\par 17 ¡Cuántos y cuán grandes fueron los tormentos con que fue atormentada la madre mientras sus hijos eran torturados con tormentos de tormento y fuego!

\par 18 Pero la Razón Inspirada prestó a su corazón la fuerza de un hombre bajo su pasión de sufrimiento, y la exaltó para no tener en cuenta los actuales anhelos del amor maternal.

\par 19 Y aunque vio la destrucción de sus siete hijos y las muchas y variadas formas de sus tormentos, la noble madre los entregó voluntariamente mediante la fe en Dios.

\par 20 Porque en su interior contemplaba, como si fueran astutos abogados en una sala del consejo, la naturaleza, la paternidad, el amor maternal y a sus hijos en el tormento, y era como si ella, la madre , pudiendo elegir entre dos votos en el caso de sus hijos, uno para su muerte y otro para salvarlos con vida, entonces no consideró por un corto tiempo la salvación de sus siete hijos, sino que, como verdadera hija de Abraham, fue llamada a Tenga en cuenta su coraje temeroso de Dios.

\par 21 ¡Oh madre de la raza, vindicadora de nuestra Ley, defensora de nuestra religión y ganadora del premio en la lucha interior!

\par 22 ¡Oh mujer, más noble para resistir que los hombres y más valiente que los guerreros para resistir!

\par 23 Porque así como el Arca de Noé, con todo el mundo viviente como carga en el Diluvio que arrasó al mundo, resistió las poderosas olas, así tú, el guardián de la Ley, golpeado por todos lados por las olas del mar. pasiones, y tenso como con fuertes explosiones por las torturas de tus hijos, resististe noblemente las tormentas que te asaltaron por causa de la religión.

\par 24 Así, pues, si una mujer, de edad avanzada y madre de siete hijos, soportaba ver a sus hijos siendo torturados hasta la muerte, la Razón Inspirada debe confesar ser gobernante suprema sobre las pasiones.

\par 25 He demostrado, pues, que no sólo los hombres han triunfado sobre sus sufrimientos, sino que también la mujer ha despreciado los más terribles tormentos.

\par 26 Y no eran tan feroces los leones que rodeaban a Daniel, ni tan ardiente el horno de fuego de Misael, como para quemar en ella el instinto de maternidad al ver a sus siete hijos siendo torturados.

\par 27 Pero por su Razón guiada por la religión la madre apagó sus pasiones, por muchas y fuertes que fueran.

\par 28 Porque también hay que tener en cuenta que, si la mujer hubiera sido débil de espíritu a pesar de su maternidad, habría llorado sobre ellos y tal vez habría hablado así:

\par 29 '¡Ah, tres veces desdichado soy, y más de tres veces desdichado! ¡He tenido siete hijos y me he quedado sin hijos!'

\par 30 «En vano estuve embarazada siete veces, y en vano tuve que soportar siete veces la carga de diez meses, y mis cuidados fueron infructuosos, y mis lactantes fueron dolorosos».

\par 31 'En vano para vosotros, hijos míos, soporté los muchos dolores del trabajo y los cuidados más difíciles de vuestra educación.'

\par 32 '¡Ay de mis hijos, que algunos aún no estaban casados ​​y los que estaban casados ​​no habían engendrado hijos! Nunca veré hijos tuyos, ni me llamarán por el nombre de abuelo.'

\par 33 '¡Ay de mí, que tuve muchos hijos hermosos, y estoy viuda y desolada en mi aflicción! ¡Ni habrá ningún hijo que me entierre cuando esté muerto! ''

\par 34 Pero la madre santa y temerosa de Dios no gimió con esta lamentación por ninguno de ellos, ni rogó a ninguno que escapara de la muerte, ni se lamentó por ellos como si fueran moribundos; pero, como si tuviera un alma de diamante y estuviera dando a luz a muchos de sus hijos, por segunda vez, a la vida inmortal, más bien les rogó y les rogó que murieran por causa de la religión.

\par 35 Oh madre, guerrera de Dios en la causa de la religión, anciana y mujer, derrotaste al tirano con tu resistencia y fuiste encontrada más fuerte que un hombre, tanto en hechos como en palabras.

\par 36 Porque, en verdad, cuando estabas encarcelado con tus hijos, estuviste allí, viendo cómo torturaban a Eleazar, y les hablaste a tus hijos en lengua hebrea:

\par 37 'Hijos míos, noble es la lucha; y vosotros, siendo llamados a dar testimonio de nuestra nación, luchad allí con celo en nombre de la Ley de nuestros padres.'

\par 38 «Porque sería vergonzoso si, mientras este anciano soportó la agonía por causa de la religión, ustedes, los jóvenes, se encogieran ante el dolor».

\par 39 'Recordad que por amor de Dios habéis venido al mundo y habéis disfrutado de la vida, y que, por tanto, le debéis a Dios soportar todo dolor por su causa; por quien también nuestro padre Abraham se apresuró a sacrificar a su hijo Isaac, padre de nuestra nación; e Isaac, al ver la mano de su padre alzando el cuchillo contra él, no retrocedió.

\par 40 «Y Daniel, el justo, fue arrojado a los leones, y Ananías, Azarías y Misael fueron arrojados al horno de fuego, y resistieron por amor de Dios».

\par 41 'Y vosotros también, teniendo la misma fe en Dios, no os turbéis; porque sería contrario a la razón que vosotros, conociendo la justicia, no soportéis los dolores.'

\par 42 Con estas palabras la madre de los siete animó a cada uno de sus hijos a morir antes que transgredir la ordenanza de Dios; ellos mismos también saben bien que los hombres que mueren por Dios viven para Dios, como viven Abraham, Isaac, Jacob y todos los patriarcas.



\chapter{8}

\par \textit{Los famosos «Atletas de la Justicia». Aquí termina la historia de valentía llamada Cuarto Libro de los Macabeos.}

\par 1 Algunos de los guardias declararon que cuando ella también estaba a punto de ser apresada y ejecutada, se arrojó en la pira para que ningún hombre tocara su cuerpo.

\par 2 ¡Oh madre, que junto con tus siete hijos derrotaste la fuerza del tirano y desbarataste sus malvadas maquinaciones y diste ejemplo de la nobleza de la fe!

\par 3 Tú fuiste puesto como un techo noble sobre tus hijos como columnas, y el terremoto de los tormentos no te sacudió en absoluto.

\par 4 Alégrate, pues, madre de alma pura, teniendo segura la esperanza de tu paciencia de la mano de Dios.

\par 5 No es tan majestuosa la luna entre las estrellas del cielo como tú, habiendo iluminado el camino de tus siete hijos estrellados hacia la justicia, estás en honor de Dios; y tú estás sentado en el cielo con ellos.

\par 6 Porque tu concepción vino del hijo de Abraham.

\par 7 Y si nos hubiera sido lícito pintar, como podría hacerlo cualquier artista, la historia de tu piedad, ¿no se habrían estremecido los espectadores al ver a la madre de siete hijos sufriendo por causa de la justicia multitud de tormentos hasta la muerte?

\par 8 Y, en verdad, sería apropiado escribir sobre su lugar de descanso estas palabras, en memoria de las generaciones futuras de nuestro pueblo:

\par AQUÍ MIENTE UN SACERDOTE ENVEJECIDO
\par Y UNA MUJER LLENA DE AÑOS
\par Y SUS SIETE HIJOS
\par A TRAVÉS DE LA VIOLENCIA DE UN TIRANO
\par DESEANDO DESTRUIR LA NACIÓN HEBREA.
\par VINDICARON LOS DERECHOS DE NUESTRO PUEBLO
\par MIRANDO A DIOS Y PERSISTIENDO
\par LOS TORMENTOS INCLUSO HASTA
\par MUERTE.

\par 9 Porque verdaderamente fue una guerra santa la que libraron ellos. Porque aquel día la virtud, probándolos con la paciencia, les presentó el premio de la victoria en la incorrupción en la vida eterna.

\par 10 Pero el primero en la pelea fue Eleazar, y la madre de los siete hijos participó en la batalla, y los hermanos pelearon.

\par 11 El tirano era su adversario y el mundo y la vida del hombre eran los espectadores.

\par 12 Y la justicia ganó al vencedor y dio la corona a sus atletas. ¿Quién sino se preguntaba por los atletas de la verdadera Ley?

\par 13 ¿Quiénes no se asombraron de ellos? El propio tirano y todo su consejo admiraron su resistencia, por la cual ahora están junto al trono de Dios y viven la era bendita.

\par 14 Porque Moisés dice: «Todos los que se han santificado están bajo tus manos».

\par 15 Y estos hombres, santificados por amor de Dios, no sólo recibieron este honor, sino también el honor de que a través de ellos el enemigo ya no tenía poder sobre nuestro pueblo, y el tirano sufrió el castigo, y nuestra patria fue purificados, habiéndose convertido, por así decirlo, en rescate por el pecado de nuestra nación; y por la sangre de estos justos y la propiciación de su muerte, la divina Providencia libró al Israel que antes era maltratado.

\par 16 Pues, cuando el tirano Antíoco vio el heroísmo de su virtud y su resistencia a las torturas, públicamente presentó su resistencia a sus soldados como ejemplo; y así inspiró a sus hombres un sentido de honor y heroísmo en el campo de batalla y en las labores de asedio, de modo que saqueó y derrocó a todos sus enemigos.

\par 17 Oh israelitas, hijos nacidos de la simiente de Abraham, obedeced esta Ley y sed justos en todo, reconociendo que la Razón Inspirada es señor sobre las pasiones y sobre los dolores, no sólo desde dentro, sino desde fuera de nosotros mismos; por lo cual aquellos hombres, entregando sus cuerpos al tormento por causa de la justicia, no sólo ganaron la admiración de la humanidad, sino que fueron considerados dignos de una herencia divina.

\par 18 Y gracias a ellos la nación obtuvo la paz y, restableciendo la observancia de la ley en nuestro país, arrebató la ciudad al enemigo.

\par 19 Y la venganza ha perseguido al tirano Antíoco en la tierra, y en la muerte sufre el castigo.

\par 20 Porque, como no logró obligar a los habitantes de Jerusalén a vivir como gentiles y a abandonar las costumbres de nuestros padres, abandonó Jerusalén y marchó contra los persas.

\par 21 Estas son las palabras que la madre de los siete hijos, la mujer justa, habló a sus hijos:

\par 22 «Yo era una doncella pura, y no me desvié de la casa de mi padre, y cuidé la costilla con la que se construyó Eva».

\par 23 'Ningún seductor del desierto, ningún engañador en el campo me corrompió; ni la Serpiente falsa y seductora manchó la pureza de mi doncellez; Viví con mi marido todos los días de mi juventud; pero cuando estos mis hijos crecieron, su padre murió.'

\par 24 'Feliz era él; porque vivió una vida bendecida con hijos y nunca conoció el dolor de su pérdida.'

\par 25 'Quien, estando aún con nosotros, os enseñó la ley y los profetas. Nos leyó sobre Abel, que fue asesinado por Caín, y sobre Isaac, que fue ofrecido en holocausto, y sobre José en la prisión.'

\par 26 «Y nos habló de Finees, el celoso sacerdote, y os enseñó el cántico de Ananías, Azarías y Misael en el fuego».

\par 27 Y glorificó también a Daniel en el foso de los leones y lo bendijo; y os recordó la palabra de Isaías:

\par 28 «Aunque pases por el fuego, la llama no te hará daño».

\par 29 «Nos cantó las palabras del salmista David: »Muchas son las aflicciones del justo».

\par 30 «Nos citó el proverbio de Salomón: »Él es árbol de vida para todos los que hacen su voluntad».

\par 31 'Él confirmó las palabras de Ezequiel: «¿Vivirán estos huesos secos?» Porque no olvidó el cántico que enseñó Moisés, que dice: Mataré y daré vida. Esta es tu vida y la bienaventuranza de tus días».

\par 32 ¡Ah, cruel fue el día, y sin embargo no cruel, en que el cruel tirano de los griegos encendió el fuego de sus bárbaros braseros y con sus pasiones hirviendo llevó a la catapulta y de nuevo a sus tormentos a los siete hijos de hija de Abraham, y cegó los globos de sus ojos, y les cortó la lengua, y los mató con muchas clases de tormento.

\par 33 Por lo cual el juicio de Dios persiguió y perseguirá al desgraciado.

\par 34 Pero los hijos de Abraham, con su madre victoriosa, se han reunido en el lugar de sus antepasados, habiendo recibido de Dios almas puras e inmortales, a quien sea la gloria por los siglos de los siglos.


\end{document}