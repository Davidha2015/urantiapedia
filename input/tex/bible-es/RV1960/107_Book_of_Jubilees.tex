\begin{document}

\title{Libro de los Jubileos}

\chapter{1}

\par \textit{Moisés recibe las tablas de la ley y las instrucciones sobre la historia pasada y futura que debe inscribir en un libro, 1-4. Apostasía de Israel, 5-9. Cautiverio de Israel y Judá, 10-13. El regreso de Judá y la reconstrucción del templo, 15-18. La oración de Moisés por Israel, 19-21. La promesa de Dios de redimirlos y morar con ellos, 22-5, 28. Moisés ordenó escribir la historia futura del mundo (¿el Libro de los Jubileos?), 26. Y un ángel para escribir la ley, 27. Este ángel toma las tablas cronológicas celestiales para dictarlas a Moisés, 29.}

\par ESTA es la historia de la división de los días de la ley y del testimonio, de los acontecimientos de los años, de sus semanas (años), de sus jubileos en todos los años del mundo, como el Señor les habló. Moisés en el monte Sinaí cuando subió a recibir las tablas de la ley y del mandamiento, según la voz de Dios que le decía: 'Sube a la cima del monte'.

\par 1 Y aconteció que en el primer año de la salida de los hijos de Israel de Egipto, en el mes tercero, a los dieciséis días del mes, [2450 Anno Mundi], Dios habló a Moisés, diciendo: 'Sube a mí al monte, y te daré dos tablas de piedra de la ley y de los mandamientos que he escrito, para que las enseñes.'
\par 2 Y Moisés subió al monte de Dios, y la gloria del Señor reposó en el monte Sinaí, y una nube lo cubrió con su sombra durante seis días.
\par 3 Y al séptimo día llamó a Moisés desde en medio de la nube, y la apariencia de la gloria del Señor era como una llama de fuego en la cima del monte.
\par 4 Y estuvo Moisés en el monte cuarenta días y cuarenta noches, y Dios le enseñó la historia anterior y posterior de la división de todos los días de la ley y del testimonio.
\par 5 Y Él dijo: 'Inclina tu corazón a cada palabra que te diré en este monte, y escríbelas en un libro para que sus generaciones vean que no los he abandonado a pesar de todo el mal que han hecho. obraron al transgredir el pacto que establezco entre mí y ti para sus generaciones hoy en el monte Sinaí.'
\par 6 «Y así sucederá que cuando les sobrevengan todas estas cosas, reconocerán que yo soy más justo que ellos en todos sus juicios y en todas sus acciones, y reconocerán que verdaderamente he estado con ellos. a ellos.'
\par 7 «Y escribe para ti todas estas palabras que te declaro hoy, porque conozco su rebelión y su dureza de cerviz, antes de introducirlos en la tierra que juré a sus padres, a Abraham y a a Isaac y a Jacob, diciendo: A vuestra descendencia daré una tierra que mana leche y miel.
\par 8 «Y comerán y se saciarán, y se volverán a dioses extraños, que no pueden librarlos de nada de su tribulación; y este testimonio será oído como testigo contra ellos. Porque se olvidarán de todos Mis mandamientos, de todo lo que Yo les mando, y andarán en pos de los gentiles, en pos de su inmundicia y en pos de su vergüenza, y servirán a sus dioses, y esto les resultará escandaloso y una tribulación y una aflicción y una trampa.'
\par 9 «Y muchos perecerán, serán llevados cautivos y caerán en manos del enemigo, porque han abandonado mis ordenanzas y mis mandamientos, las fiestas de mi pacto, mis días de reposo y mi lugar santo. que me he santificado en medio de ellos, y mi tabernáculo, y mi santuario, que me he santificado en medio de la tierra, para que ponga mi nombre sobre ella, y habite (allí).'
\par 10 «Y se harán lugares altos, bosques e imágenes talladas, y adorarán cada uno su propia imagen, hasta extraviarse, y sacrificarán a sus hijos a los demonios y a todos los obras del error de sus corazones.'
\par 11 «Y les enviaré testigos para que testifiquen contra ellos, pero no escucharán, y matarán también a los testigos, perseguirán a los que buscan la ley, y derogarán y cambiarán todo para como para hacer el mal ante Mis ojos.'
\par 12 «Y esconderé de ellos mi rostro, y los entregaré en manos de las naciones en cautiverio, en presa y devoradores, y los quitaré de en medio de la tierra, y los esparcirán entre los gentiles.'
\par 13 «Y se olvidarán de toda Mi ley, de todos Mis mandamientos y de todos Mis juicios, y se extraviarán como en lunas nuevas, sábados, fiestas, jubileos y ordenanzas».
\par 14 «Y después de esto, de entre los gentiles se volverán a Mí con todo su corazón, con toda su alma y con todas sus fuerzas, y yo los reuniré de entre todos los gentiles, y me buscarán, para que Seré hallado por ellos cuando me busquen con todo su corazón y con toda su alma.'
\par 15 «Y les revelaré paz abundante con justicia, y les quitaré la planta de la rectitud, con todo Mi corazón y con toda Mi alma, y ​​serán para bendición y no para maldición, y serán será la cabeza y no la cola.'
\par 16 'Y edificaré mi santuario en medio de ellos, y habitaré con ellos, y seré su Dios y ellos serán mi pueblo en verdad y justicia.'
\par 17 'Y no los desampararé ni les fallaré; porque yo soy el Señor su Dios.'
\par 18 Entonces Moisés cayó rostro en tierra y oró y dijo: «Oh Señor, Dios mío, no abandones a tu pueblo y a tu herencia, para que se extravíen en el error de sus corazones, y no los entregues en manos de sus enemigos, los gentiles, para que no se enseñoreen de ellos y les hagan pecar contra ti.'
\par 19 «Que tu misericordia, oh Señor, se alce sobre tu pueblo, y crea en ellos un espíritu recto, y no permitas que el espíritu de Beliar los domine para acusarlos delante de ti y para atraparlos en todos los caminos. de justicia, para que perezcan delante de Tu faz.'
\par 20 «Pero ellos son tu pueblo y tu herencia, que con tu gran poder has librado de las manos de los egipcios: crea en ellos un corazón limpio y un espíritu santo, y que de ahora en adelante no queden atrapados en sus pecados. hasta la eternidad.'
\par 21 Y el Señor dijo a Moisés: «Conozco sus rebeliones, sus pensamientos y su dureza de cerviz, y no serán obedientes hasta que confiesen su propio pecado y el pecado de sus padres».
\par 22 «Y después de esto se volverán a Mí con toda rectitud, con todo su corazón y con toda su alma, y ​​yo circuncidaré el prepucio de su corazón y el prepucio del corazón de su descendencia, y Crearé en ellos espíritu santo y los limpiaré para que no se aparten de mí desde aquel día hasta la eternidad.'
\par 23 «Y sus almas se unirán a Mí y a todos Mis mandamientos, y cumplirán Mis mandamientos, y Yo seré su Padre y ellos serán Mis hijos».
\par 24 'Y todos ellos serán llamados hijos del Dios viviente, y todo ángel y todo espíritu sabrán, sí, sabrán que estos son Mis hijos, y que Yo soy su Padre en rectitud y justicia, y que Yo soy amarlos.'
\par 25 'Y escribe para ti todas estas palabras que te anuncio en este monte, las primeras y las últimas, que sucederán en todas las divisiones de los días en la ley, en el testimonio y en las semanas y los jubileos hasta la eternidad, hasta que descienda y habite con ellos por toda la eternidad.'
\par 26 Y dijo al ángel de la presencia: «Escribe para Moisés desde el principio de la creación hasta que mi santuario sea construido entre ellos para toda la eternidad».
\par 27 'Y el Señor aparecerá a los ojos de todos, y todos sabrán que yo soy el Dios de Israel y el Padre de todos los hijos de Jacob, y Rey en el monte Sión por toda la eternidad. Y Sión y Jerusalén serán santas.'
\par 28 Y el ángel de la presencia que iba delante del campamento de Israel tomó las tablas de las divisiones de los años desde el momento de la creación de la ley y del testimonio de las semanas de los jubileos, según el años individuales, conforme a todo el número de los jubileos [según los años individuales], desde el día de la [nueva] creación cuando los cielos y la tierra serán renovados y toda su creación conforme a las potencias de los cielos, y conforme a toda la creación de la tierra, hasta que sea hecho el santuario del Señor en Jerusalén en el monte Sión, y sean renovadas todas las luminarias para sanidad y para paz y para bendición para todos los escogidos de Israel, y que así sea será desde aquel día y hasta todos los días de la tierra.

\chapter{2}

\par \textit{La historia de los veintidós actos distintos de la creación en los seis días, 1-16. Institución del sábado: su observancia por los ángeles más elevados, con quienes luego se asociará Israel, 17-32. (cf. Génesis i-ii. 3.)}

\par 1 Y el ángel de la presencia habló a Moisés conforme a la palabra del Señor, diciendo: Escribe la historia completa de la creación, cómo en seis días el Señor Dios terminó todas sus obras y todo lo que creó, y guardó el sábado. en el séptimo día y lo santificó por todos los siglos, y lo puso como señal para todas sus obras.
\par 2 Porque en el primer día creó los cielos que están arriba, la tierra, las aguas y todos los espíritus que sirven delante de él: los ángeles de la presencia, los ángeles de la santificación y los ángeles [del espíritu de fuego y los ángeles] del espíritu de los vientos, y los ángeles del espíritu de las nubes, y de las tinieblas, y de la nieve y del granizo y de la escarcha, y los ángeles de las voces y del trueno y del relámpagos, y los ángeles de los espíritus del frío y del calor, y del invierno y de la primavera y del otoño y del verano y de todos los espíritus de sus criaturas que están en los cielos y en la tierra, (Creó) los abismos y las tinieblas, crepúsculo (y noche), y la luz, aurora y día, que Él ha preparado en el conocimiento de su corazón.
\par 3 Entonces vimos sus obras y le alabamos y alabamos delante de él por todas sus obras. porque siete grandes obras creó en el primer día.
\par 4 Y en el segundo día creó el firmamento en medio de las aguas, y las aguas se dividieron ese día: la mitad de ellas subieron arriba y la otra mitad descendieron debajo del firmamento (que estaba) en medio sobre la faz de toda la tierra. Y esta fue la única obra (Dios) creada en el segundo día.
\par 5 Y al tercer día ordenó que las aguas pasaran de sobre la faz de la tierra a un solo lugar, y que apareciera la tierra seca.
\par 6 Y las aguas hicieron tal como Él les había ordenado, y se retiraron de la faz de la tierra a un lugar fuera de este firmamento, y apareció la tierra seca.
\par 7 Y en aquel día creó para ellos todos los mares según sus respectivos lugares de reunión, y todos los ríos y las concentraciones de las aguas en las montañas y en toda la tierra, y todos los lagos y todos los el rocío de la tierra, y la semilla que se siembra, y todo lo que brota, y los árboles frutales, y los árboles del bosque, y el jardín del Edén, en el Edén y todas las \textit{plantas según su especie}.
\par 8 Estas cuatro grandes obras Dios creó en el tercer día. Y al cuarto día creó el sol, la luna y las estrellas, y los puso en la expansión de los cielos, para alumbrar toda la tierra, y para señorear en el día y en la noche, y separar la luz del cielo. oscuridad.
\par 9 Y Dios puso el sol como gran señal sobre la tierra durante los días, los sábados, los meses, las fiestas, los años, los sábados de los años, los jubileos y todas las estaciones de los años.
\par 10 Y separa la luz de las tinieblas para la prosperidad, para que prosperen todas las cosas que brotan y crecen en la tierra.
\par 11 Estas tres clases las hizo el cuarto día. Y al quinto día creó grandes monstruos marinos en las profundidades de las aguas, porque estas fueron las primeras cosas de carne que fueron creadas por sus manos, los peces y todo lo que se mueve en las aguas, y todo lo que vuela, las aves y todos los de su especie.
\par 12 Y el sol salió sobre ellos para prosperar, y sobre todo lo que hay sobre la tierra, sobre todo lo que brota de la tierra, sobre todos los árboles frutales y sobre toda carne.
\par 13 Estos tres géneros los creó el quinto día. Y en el sexto día creó todos los animales de la tierra, y todos los ganados, y todo lo que se mueve sobre la tierra.
\par 14 Y después de todo esto creó al hombre, los creó al hombre y a la mujer, y le dio dominio sobre todo lo que hay en la tierra y en los mares, y sobre todo lo que vuela, y sobre las bestias y el ganado, y sobre todo lo que se mueve sobre la tierra, y sobre toda la tierra, y sobre todo esto le dio señorío.
\par 15 Y creó estos cuatro géneros en el sexto día. Y había en total veintidós clases.
\par 16 Y acabó toda su obra en el sexto día, todo lo que hay en los cielos y en la tierra, en los mares y en los abismos, en la luz y en las tinieblas, y en todo.
\par 17 Y nos dio una gran señal: el día de reposo, para que trabajemos seis días, pero guardemos el sábado de todo trabajo el séptimo día.
\par 18 Y a todos los ángeles de la presencia y a todos los ángeles de la santificación, estas dos grandes clases, Él nos ha ordenado que guardemos el sábado con Él en el cielo y en la tierra.
\par 19 Y Él nos dijo: 'He aquí, yo separaré para mí un pueblo de entre todos los pueblos, y éstos guardarán el día de reposo, y los santificaré para mí como mi pueblo, y los bendeciré; Como he santificado el día de reposo y lo santifico para mí, así los bendeciré, y ellos serán mi pueblo y yo seré su Dios.'
\par 20 'Y yo escogí la descendencia de Jacob de entre todo lo que he visto, y lo he escrito como Mi hijo primogénito, y lo he santificado para Mí por los siglos de los siglos; y les enseñaré el día del sábado, para que guarden el sábado en él en todo trabajo.'
\par 21 Y así creó en él una señal según la cual debían guardar el sábado entre nosotros el séptimo día, para comer y beber, y para bendecir a Aquel que creó todas las cosas, así como Él bendijo y santificó para sí mismo una peculiar pueblo sobre todos los pueblos, y que guarden el sábado junto con nosotros.
\par 22 E hizo ascender sus mandamientos como un olor grato, agradable delante de Él todos los días. . .
\par 23 Hubo veintidós cabezas de hombres desde Adán hasta Jacob, y veintidós clases de trabajo se hicieron hasta el séptimo día; esto es bendito y santo; y el primero también es bendito y santo; y éste sirve con aquél para santificación y bendición.
\par 24 Y a éste (Jacob y su descendencia) les fue concedido ser siempre los bienaventurados y santos del primer testimonio y de la ley, así como Él había santificado y bendecido el día del sábado en el séptimo día.
\par 25 En seis días creó los cielos y la tierra y todo lo que creó, y Dios santificó el séptimo día para todas sus obras; por lo tanto, ordenó en su nombre que cualquiera que haga cualquier trabajo en él morirá, y que el que lo contamine ciertamente morirá.
\par 26 ¿Por qué ordenas a los hijos de Israel que observen este día para santificarlo y no hacer en él ningún trabajo y no contaminarlo, ya que es más santo que todos los demás días?
\par 27 Y cualquiera que lo profane ciertamente morirá, y cualquiera que haga algún trabajo en él ciertamente morirá eternamente, para que los hijos de Israel guarden este día por sus generaciones y no sean desarraigados de la tierra; porque es día santo y día bendito.
\par 28 Y todo aquel que lo observe y guarde el sábado en todo su trabajo, será santo y bendito todos los días como nosotros.
\par 29 Declara y di a los hijos de Israel la ley de este día, para que guarden el sábado y para que no lo abandonen por el error de su corazón; (y) que no es lícito hacer en él ningún trabajo que sea indecoroso, hacer en él su propio placer, y que no deben preparar nada para comer o beber, y (que no es lícito) sacar agua, ni introducir ni sacar por sus puertas ninguna carga que no hubieran preparado para sí el sexto día en sus moradas.
\par 30 Y ese día no entrarán ni sacarán de casa en casa; porque aquel día es más santo y bendito que cualquier día de jubileo de los jubileos; sobre esto guardamos el sábado en los cielos antes de que fuera conocido a cualquier carne guardar el sábado allí en la tierra.
\par 31 Y el Creador de todas las cosas la bendijo, pero no santificó a todos los pueblos y naciones para que guardaran el sábado en ellos, sino sólo a Israel: sólo a ellos les permitió comer y beber y guardar el sábado en la tierra.
\par 32 Y el Creador de todas las cosas bendijo este día que había creado para bendición, santidad y gloria sobre todos los días.
\par 33 Esta ley y este testimonio fueron dados a los hijos de Israel como ley perpetua para sus generaciones.

\chapter{3}

\par \textit{Adán nombra todas las criaturas, 1-3. Creación de Eva y promulgación de las leyes levíticas de purificación, 4-14. Adán y Eva en el paraíso: su pecado y expulsión, 15-29. Se promulga la ley de cubrir la propia vergüenza, 30-2. Adán y Eva viven en Êldâ, 32-5. (Cf.Gen. ii.18-25, iii.)}

\par 1 Y en los seis días de la segunda semana, conforme a la palabra de Dios, trajimos a Adán todos los animales, y todos los ganados, y todas las aves, y todo lo que se mueve sobre la tierra, y todo lo que se mueve. en el agua, según sus especies, y según sus tipos: las bestias el primer día; el ganado el segundo día; los pájaros al tercer día; y todo lo que se mueve sobre la tierra al cuarto día; y el que se mueve en el agua el quinto día.
\par 2 Y Adán los nombró a todos por sus respectivos nombres, y según los llamó, así fue su nombre.
\par 3 Y en esos cinco días Adán vio a todos estos, hombres y mujeres, según todo tipo de seres que hay en la tierra, pero estaba solo y no encontró ayuda idónea para él.
\par 4 Y el Señor nos dijo: «No es bueno que el hombre esté solo; busquemos una ayuda para él».
\par 5 Y el Señor nuestro Dios hizo caer sobre él un sueño profundo, y se durmió, y tomó para la mujer una costilla de entre sus costillas, y esta costilla fue el origen de la mujer de entre sus costillas, y Él edificó la carne en su lugar y edificó a la mujer.
\par 6 Y despertó a Adán de su sueño y al despertar se levantó al sexto día, y se la trajo, y él la conoció, y le dijo: 'Esto ahora es hueso de mis huesos y carne de mi carne; ella será llamada [mi] esposa; porque le fue arrebatada a su marido.'
\par 7 Por tanto, el marido y la mujer serán uno, y por eso el hombre dejará a su padre y a su madre y se unirá a su mujer, y serán una sola carne.
\par 8 En la primera semana fue creado Adán, y la costilla fue su esposa; en la segunda semana se la mostró; por eso se les dio el mandamiento de guardar en su impureza, al varón siete días, y al varón, siete días. una hembra dos veces siete días.
\par 9 Y cuando Adán cumplió cuarenta días en la tierra donde había sido creado, lo llevamos al jardín del Edén para que lo cultivara y lo cuidara, pero a su esposa la trajeron al día ochenta, y después de esto ella entró en el jardín del Edén.
\par 10 Por esta razón, en las tablas celestiales está escrito el mandamiento respecto de la que da a luz: «Si da a luz un varón, permanecerá en su impureza siete días, según la primera semana de días, y treinta y tres días». días permanecerá en la sangre de su purificación, y no tocará ninguna cosa santificada, ni entrará en el santuario, hasta que cumpla estos días que (están prescritos) en el caso de un niño varón.'
\par 11 «Pero en el caso de una niña, ella permanecerá en su impureza dos semanas de días, según las dos primeras semanas, y sesenta y seis días en la sangre de su purificación, y serán en total ochenta días.'
\par 12 Y cuando hubo cumplido estos ochenta días, la llevamos al jardín del Edén, porque es más santo que toda la tierra y todo árbol que en él se planta es santo.
\par 13 Por lo tanto, a la mujer que dé a luz varón o hembra se le prescribió el estatuto de aquellos días: que no toque cosa sagrada ni entre en el santuario hasta que se cumplan estos días para el varón o la hembra.
\par 14 Ésta es la ley y el testimonio escrito para Israel, para que la cumplan todos los días.
\par 15 Y en la primera semana del primer jubileo, [1-7 a.m.] Adán y su esposa estuvieron en el jardín del Edén durante siete años, labrándolo y cuidándolo, y le dimos trabajo y le ordenamos que hiciera todo lo que es apto para labranza.
\par 16 Y labrando (el jardín), y estaba desnudo y no lo sabía, y no se avergonzaba, y protegió el jardín de las aves, y de las bestias y del ganado, y recogió su fruto, y comió, y guardó el resto. para él y para su esposa [y dejó a un lado lo que se guardaba].
\par 17 Y cuando se cumplieron los siete años que había cumplido allí, exactamente siete años, en el mes segundo, a los diecisiete días, vino la serpiente y se acercó a la mujer, y la serpiente dijo a la mujer: ¿Te ha mandado Dios, diciendo: No comeréis de todo árbol del jardín?
\par 18 Y ella le dijo: De todos los frutos de los árboles del huerto Dios nos ha dicho: Comed; pero del fruto del árbol que está en medio del huerto Dios nos ha dicho: No comeréis de él, ni lo tocaréis, para que no muráis.'
\par 19 Y la serpiente dijo a la mujer: 'No moriréis ciertamente; porque Dios sabe que el día que comáis de él, se os abrirán los ojos y seréis como dioses, y conoceréis el bien y el bien. demonio.'
\par 20 Y la mujer vio que el árbol era agradable y agradable a la vista, y que su fruto era bueno para comer, y lo tomó y comió.
\par 21 Y cuando ella cubrió su vergüenza con hojas de higuera, se las dio a Adán y él comió, y se le abrieron los ojos y vio que estaba desnudo.
\par 22 Y tomando hojas de higuera, las cosió, se hizo un delantal y cubrió su vergüenza.
\par 23 Y Dios maldijo a la serpiente y se enojó contra ella para siempre. . .
\par 24 Y se enojó contra la mujer, porque ella obedeció la voz de la serpiente y comió; y Él le dijo: 'Multiplicaré en gran manera tus dolores y tus dolores; con dolor darás a luz los hijos, y tu regreso será a tu marido, y él se enseñoreará de ti.'
\par 25 Y también a Adán le dijo: «Por cuanto obedeciste la voz de tu mujer y comiste del árbol del que te ordené que no comieras, maldita será la tierra por tu causa: espinas y Te producirá abrojos, y comerás tu pan con el sudor de tu frente, hasta que vuelvas a la tierra de donde fuiste tomado; porque tierra eres, y a la tierra volverás.'
\par 26 Y les hizo túnicas de piel, los vistió y los sacó del jardín del Edén.
\par 27 Y el día en que Adán salió del jardín, ofreció en olor grato una ofrenda de incienso, gálbano, estacte y especias aromáticas, por la mañana, al salir el sol, desde el día en que cubrió su rostro. lástima.
\par 28 Y aquel día se cerró la boca de todos los animales, de los ganados, de las aves, de todos los que caminan y se mueven, de modo que ya no podían hablar; porque todos habían hablado unos con otros con un labio y con una lengua.
\par 29 Y envió fuera del Jardín del Edén a toda carne que había en el Jardín del Edén, y toda carne fue esparcida según sus especies y según sus tipos en los lugares que habían sido creados para ellos.
\par 30 Y sólo a Adán le dio (los medios) para cubrir su vergüenza, entre todas las bestias y ganados.
\par 31 Por eso está escrito en las Tablas Celestiales que todos los que conocen el juicio de la ley deben cubrir su vergüenza y no descubrirse como se descubren los gentiles.
\par 32 Y en la luna nueva del cuarto mes, Adán y su esposa salieron del Jardín del Edén y habitaron en la tierra de Elda, en la tierra de su creación.
\par 33 Y Adán llamó el nombre de su esposa Eva.
\par 34 Y no tuvieron hijo hasta el primer jubileo, y después de esto él la conoció.
\par 35 Ahora labraba la tierra tal como le habían enseñado en el Jardín del Edén.

\chapter{4}

\par \textit{Caín y Abel y otros hijos de Adán, 1-12. Enós, Cainán, Mahalalel, Jared, 13-15. Enoc y su historia, 16-25. Cuatro lugares sagrados, 26. Matusalén, Lamec, Noé, 27, 28. Muerte de Adán y Caín, 29-32. Sem, Cam y Jafet, 32. (Cf. Gen. iv-v.)}

\par 1 Y en la tercera semana del segundo jubileo [64-70 a.m.] dio a luz a Caín, y en la cuarta [71-77 a.m.] dio a luz a Abel, y en el quinto [78-84 a.m.] dio a luz a su hija Âwân.
\par 2 Y en el primer (año) del tercer jubileo [99-105 a.m.], Caín mató a Abel porque (Dios) aceptó el sacrificio de Abel, y no aceptó la ofrenda de Caín.
\par 3 Y lo mató en el campo; y su sangre lloró desde la tierra hasta el cielo, quejándose de haberlo matado.
\par 4 Y el Señor reprendió a Caín a causa de Abel, porque lo había matado, y lo hizo fugitivo en la tierra a causa de la sangre de su hermano, y lo maldijo en la tierra.
\par 5 Y por esto está escrito en las tablas celestiales: 'Maldito el que hiere a su prójimo alevosamente; y todos los que lo hayan visto y oído, digan: Así sea; y el hombre que lo haya visto y no lo haya declarado, sea anatema como el otro.'
\par 6 Y por eso anunciamos, cuando nos presentamos ante el Señor nuestro Dios, todos los pecados que se cometen en el cielo y en la tierra, en la luz y en las tinieblas y en todas partes.
\par 7 Y Adán y su esposa lloraron a Abel durante cuatro semanas de años, [99-127 AM] y en el cuarto año de la quinta semana [130 AM] se alegraron, y Adán conoció a su esposa otra vez, y ella lo dio a luz. un hijo, y llamó su nombre Set; porque dijo 'DIOS nos ha levantado una segunda simiente en la tierra en lugar de Abel; porque Caín lo mató.'
\par 8 Y en la sexta semana [134-40 AM] engendró a su hija Azûrâ.
\par 9 Y Caín tomó a su hermana Âwân por esposa y ella le dio a luz a Enoc al final del cuarto jubileo. [190-196 AM] Y en el primer año de la primera semana del quinto jubileo, [197 AM] se edificaron casas en la tierra, y Caín edificó una ciudad, y le puso el nombre de su hijo Enoc.
\par 10 Y Adán conoció a Eva su esposa y ella dio a luz todavía nueve hijos.
\par 11 Y en la quinta semana del quinto jubileo [225-31 AM] Set tomó a Azûrâ su hermana por esposa, y en el cuarto (año de la sexta semana) [235 AM] ella le dio a luz a Enós.
\par 12 Comenzó a invocar el nombre del Señor en la tierra.
\par 13 Y en el séptimo jubileo de la tercera semana [309-15 AM] Enós tomó a Nôâm su hermana por esposa, y ella le dio a luz un hijo en el tercer año de la quinta semana, y él llamó su nombre Kenan.
\par 14 Y al final del octavo jubileo [325, 386-3992 AM] Cainán tomó a su hermana Mûalêlêth por esposa, y ella le dio a luz un hijo en el noveno jubileo, en la primera semana del tercer año de este semana, [395 AM] y llamó su nombre Mahalaleel.
\par 15 Y en la segunda semana del décimo jubileo [449-55 AM] Mahalalel tomó por esposa a DinaH, hija de Barakiel, hija del hermano de su padre, y ella le dio a luz un hijo en la tercera semana del sexto. año, [461 AM] y llamó su nombre Jared, porque en sus días los ángeles del Señor descendieron sobre la tierra, los que se llaman los Vigilantes, para que instruyeran a los hijos de los hombres, y para que hicieran juicio y rectitud en la tierra.
\par 16 Y en el undécimo jubileo [512-18 AM] Jared tomó para sí una esposa, y su nombre era Baraka, la hija de Râsûjâl, una hija del hermano de su padre, en la cuarta semana de este jubileo, [522 AM ] y ella le dio a luz un hijo en el quinto septenario, en el cuarto año del jubileo, y él llamó su nombre Enoc.
\par 17 Y él fue el primero entre los hombres nacidos en la tierra que aprendió a escribir, a tener conocimiento y sabiduría, y que escribió en un libro los signos del cielo según el orden de los meses, para que los hombres pudieran conocer las estaciones de los años. según el orden de sus meses separados.
\par 18 Y él fue el primero en escribir un testimonio, y dio testimonio a los hijos de los hombres entre las generaciones de la tierra, y contó las semanas de los jubileos, y les dio a conocer los días de los años, y los puso en orden. los meses y contó los sábados de los años tal como se los dimos a conocer.
\par 19 Y en una visión mientras dormía vio lo que era y lo que será, como les sucederá a los hijos de los hombres a lo largo de sus generaciones hasta el día del juicio; vio y comprendió todo, y escribió su testimonio, y puso el testimonio en la tierra para todos los hijos de los hombres y para sus generaciones.
\par 20 Y en el duodécimo jubileo, [582-88] en el séptimo septenario, tomó para sí una esposa, cuyo nombre era Edna, hija de Danel, hija del hermano de su padre, y en el sexto año En esta semana [587 AM] ella le dio un hijo y él llamó su nombre Matusalén.
\par 21 Además estuvo con los ángeles de Dios durante estos seis jubileos de años, y ellos le mostraron todo lo que hay en la tierra y en el cielo, el gobierno del sol, y él lo escribió todo.
\par 22 Y dio testimonio a los Vigilantes, que habían pecado con las hijas de los hombres; porque éstas habían comenzado a unirse, para contaminarse, con las hijas de los hombres, y Enoc testificó contra (ellas) todas.
\par 23 Y fue tomado de entre los hijos de los hombres, y lo condujimos al Jardín del Edén con majestad y honor, y he aquí que allí escribe la condenación y el juicio del mundo, y toda la maldad de los hijos de hombres.
\par 24 Y a causa de ello (Dios) trajo las aguas del diluvio sobre toda la tierra del Edén; porque allí fue puesto como señal y para que testificara contra todos los hijos de los hombres, para que contara todos los hechos de las generaciones hasta el día de la condenación.
\par 25 Y quemó el incienso del santuario, especias aromáticas agradables delante del Señor en el monte.
\par 26 Porque el Señor tiene cuatro lugares en la tierra: el Jardín del Edén, el Monte del Oriente, y este monte en el que estás hoy, el Monte Sinaí y el Monte Sión, que serán santificados en el nuevo mundo. creación para una santificación de la tierra; por medio de él será santificada la tierra de toda (su) culpa y de su inmundicia a través de las generaciones del mundo.
\par 27 Y en el jubileo catorce [652 AM] Matusalén tomó para sí una esposa, Edna hija de Azrial, hija del hermano de su padre, en el tercer septenario, en el primer año de este septenario, [701-7 AM ] y engendró un hijo y llamó su nombre Lamec.
\par 28 Y en el jubileo decimoquinto, en el tercer septenario, Lamec tomó para sí una esposa, y su nombre era Betenos, hija de Baraki'il, hija del hermano de su padre, y en este septenario ella le dio a luz un hijo y él lo llamó. su nombre Noé, diciendo: Éste me consolará de mi angustia y de todo mi trabajo, y de la tierra que el Señor ha maldecido.
\par 29 Y al final del jubileo decimonoveno, en la séptima semana del sexto año [930 a.m.] del mismo, Adán murió, y todos sus hijos lo sepultaron en la tierra de su creación, y él fue el primero en ser sepultado. en la tierra.
\par 30 Y le faltaron setenta años de mil años; porque mil años son como un día en el testimonio de los cielos y por eso fue escrito acerca del árbol del conocimiento: 'El día que de él comáis, moriréis.' Por esto no cumplió los años de este día; porque murió durante el mismo.
\par 31 Al terminar este jubileo, Caín fue asesinado después de él en el mismo año; porque su casa cayó sobre él y murió en medio de su casa, y fue muerto a pedradas; porque con una piedra había matado a Abel, y con una piedra fue muerto en justo juicio.
\par 32 Por esta razón está escrito en las Tablas Celestiales: 'Con el instrumento con el que un hombre mata a su prójimo, con él será asesinado; Según la manera en que lo hirió, de la misma manera harán con él.'
\par 33 Y en el jubileo veinticinco [1205 AM] Noé tomó para sí una esposa, y su nombre era \`Emzârâ, la hija de Râkê'êl, la hija del hermano de su padre, en el primer año del quinto. semana [1207 AM]: y en el tercer año de ella le dio a luz a Sem, en el quinto año de ella [1209 AM] le dio a luz a Cam, y en el primer año de la sexta semana [1212 AM] le dio a luz a Jafet.

\chapter{5}

\par \textit{Los Ángeles de Dios se casan con las hijas de los hombres, 1. Corrupción de toda la creación, 2-3. Castigo de los ángeles caídos y sus hijos, 4-9a. Se anuncia el juicio final, 9b-16. Día de Expiación, 17-18. El diluvio predicho, Noé construye el arca, el diluvio, 19-32. (Cf. Gen.vi-viii.19.)}

\par 1 Y aconteció que cuando los hijos de los hombres comenzaron a multiplicarse sobre la faz de la tierra y les nacieron hijas, los ángeles de Dios las vieron en un año de este jubileo, y eran hermosas a la vista. al; y tomaron esposas de todas las que eligieron, y les dieron hijos y eran gigantes.
\par 2 Y la maldad aumentó sobre la tierra y toda carne corrompió sus caminos, así los hombres, los animales, las bestias, las aves y todo lo que camina sobre la tierra; todos ellos corrompieron sus caminos y sus órdenes, y comenzaron a devorarse unos a otros. y la anarquía aumentó en la tierra y toda imaginación de los pensamientos de todos los hombres (era) así continuamente mala.
\par 3 Y Dios miró la tierra y vio que estaba corrompida, y toda carne había corrompido sus órdenes, y todos los que estaban sobre la tierra habían cometido toda clase de males ante sus ojos.
\par 4 Y dijo que destruiría al hombre y a toda carne sobre la faz de la tierra que había creado.
\par 5 Pero Noé halló gracia ante los ojos del Señor.
\par 6 Y se enojó mucho contra los ángeles que había enviado sobre la tierra, y dio orden de desarraigarlos de todo su dominio, y nos mandó que los atáramos en las profundidades de la tierra, y he aquí que están atados en medio de ellos, y están (mantenidos) separados.
\par 7 Y contra sus hijos salió una orden de delante de él: serían heridos con espada y eliminados de debajo del cielo.
\par 8 Y Él dijo: 'Mi espíritu no permanecerá para siempre en el hombre; porque ellos también son carne y sus días serán ciento veinte años.
\par 9 Y envió su espada entre ellos para que cada uno matara a su prójimo, y comenzaron a matarse unos a otros hasta que todos cayeron a espada y fueron destruidos de la tierra.
\par 10 Y sus padres fueron testigos (de su destrucción), y después de esto fueron atados en las profundidades de la tierra para siempre, hasta el día de la gran condenación, cuando se ejecutará el juicio sobre todos aquellos que han corrompido sus caminos y sus obras delante del Señor.
\par 11 Y destruyó a todos de sus lugares, y no quedó ninguno de ellos a quien no juzgara según todas sus maldades.
\par 12 Y creó para todas sus obras una naturaleza nueva y justa, para que nunca pecaran en toda su naturaleza, sino que fueran todos justos cada uno en su especie para siempre.
\par 13 Y el juicio de todos está ordenado y escrito en las tablas celestiales con justicia, incluso (el juicio de) todos los que se apartan del camino que les ha sido ordenado andar; y si no caminan en él, el juicio está escrito para toda criatura y para toda especie.
\par 14 Y no hay nada en el cielo ni en la tierra, ni en la luz, ni en las tinieblas, ni en el Seol, ni en lo profundo, ni en el lugar de las tinieblas, que no sea juzgado; y todos sus juicios están ordenados, escritos y grabados.
\par 15 Él juzgará a todos, al grande según su grandeza, al pequeño según su pequeñez y a cada uno según su camino.
\par 16 Y Él no es alguien que tenga en cuenta la persona (de nadie), ni es alguien que recibirá regalos, si dice que ejecutará juicio sobre cada uno: si uno diera todo lo que hay en la tierra, Él No hagas caso de los regalos ni de la persona (de nadie), ni aceptes nada de sus manos, porque Él es un juez justo.
\par 17 [Y de los hijos de Israel está escrito y ordenado: Si se vuelven a él con rectitud, Él perdonará todas sus transgresiones y perdonará todos sus pecados.
\par 18 Está escrito y ordenado que Él mostrará misericordia a todos los que se arrepientan de toda su culpa una vez al año.]
\par 19 Y en cuanto a todos aquellos que corrompieron sus caminos y sus pensamientos antes del diluvio, ninguna persona fue aceptada sino sólo la de Noé; porque su persona fue aceptada en nombre de sus hijos, a quienes (Dios) salvó de las aguas del diluvio por su cuenta; porque su corazón era recto en todos sus caminos, conforme a lo que le había sido ordenado, y no se había apartado de nada de lo que le estaba ordenado.
\par 20 Y el Señor dijo que destruiría todo lo que había sobre la tierra, tanto hombres como ganado, y
\par 21 bestias, aves del cielo y todo lo que se mueve sobre la tierra. Y ordenó a Noé que le hiciera un arca para salvarse de las aguas del diluvio.
\par 22 Y Noé hizo el arca en todos sus aspectos como le había ordenado, en el jubileo veintisiete de años, en el quinto septenario del quinto año (en la luna nueva del primer mes). [1307 a.m.]
\par 23 Y entró en el sexto (año) del mismo, [1308 AM] en el segundo mes, en la luna nueva del segundo mes, hasta el decimosexto; y entró él, y todo lo que le llevamos, en el arca, y el Señor la cerró por fuera en la decimoséptima tarde.
\par 24 Y el Señor abrió siete compuertas del cielo,  
\par     Y las bocas de las fuentes del gran abismo, siete bocas en total.
\par 25 Y las compuertas comenzaron a derramar agua del cielo durante cuarenta días y cuarenta noches,  
\par     Y también las fuentes del abismo hicieron subir aguas, hasta que todo el mundo se llenó de agua.
\par 26 Y las aguas aumentaron sobre la tierra.  
\par     Quince codos subieron las aguas sobre todos los montes altos,  
\par     Y el arca fue levantada sobre la tierra,  
\par     Y se movía sobre la faz de las aguas.
\par 27 Y las aguas estuvieron sobre la faz de la tierra durante cinco meses y ciento cincuenta días.
\par 28 Y el arca fue y se detuvo en la cima de Lubar, una de las montañas de Ararat.
\par 29 Y (en la luna nueva) en el cuarto mes se cerraron las fuentes del gran abismo y se cerraron las compuertas del cielo; y en la luna nueva del séptimo mes se abrieron todas las bocas de los abismos de la tierra, y el agua comenzó a descender a las profundidades de abajo.
\par 30 Y en la luna nueva del mes décimo se vieron las cimas de las montañas, y en la luna nueva del mes primero se hizo visible la tierra.
\par 31 Y las aguas desaparecieron de encima de la tierra en el quinto septenario de su séptimo año, y en el decimoséptimo día del segundo mes la tierra se secó.
\par 32 Y el día veintisiete del mismo día abrió el arca y sacó de ella bestias, ganado, aves y todo ser movible.

\chapter{6}

\par \textit{Sacrificio de Noé, 1-3 (cf. Gén. vii.20-2). El pacto de Dios con Noé, prohibido comer sangre, 4-10 (cf. Gén. ix. 1-17). Moisés ordenó renovar esta ley contra el consumo de sangre, 11-14. Arco colocado en las nubes en busca de una señal, 15-16. Fiesta de las semanas instituida, historia de sus observancias, 17-22. Fiesta de las lunas nuevas, 23-8. División del año en 364 días, 29-38.}

\par 1 Y en la luna nueva del tercer mes, salió del arca y edificó un altar en aquel monte.
\par 2 E hizo expiación por la tierra, tomó un cabrito y con su sangre hizo expiación por toda la culpa de la tierra; porque todo lo que había en ella había sido destruido, excepto los que estaban en el arca con Noé.
\par 3 Y puso su sebo sobre el altar, tomó un buey, una cabra, una oveja, un cabrito, sal, una tórtola y una cría de paloma, y ​​ofreció un holocausto. sobre el altar, y derramó sobre él una ofrenda mezclada con aceite, y roció vino, y derramó incienso sobre todo, y hizo surgir un olor agradable, agradable delante del Señor.
\par 4 Y el Señor olió el agradable olor e hizo con él un pacto de que no volvería a haber diluvios que destruyeran la tierra; que todos los días de la tierra, la siembra y la cosecha, nunca cesen; El frío y el calor, el verano y el invierno, el día y la noche no deben cambiar su orden ni cesar para siempre.
\par 5 'Y vosotros, multiplicaos y multiplicaos sobre la tierra, multiplicaos sobre ella y sed sobre ella una bendición. El temor y el pavor de ti inspiraré en todo lo que hay en la tierra y en el mar.'
\par 6 'Y he aquí, os he dado todos los animales, y todos los seres alados, y todo lo que se mueve sobre la tierra, y los peces en el agua, y todo lo que sirve de alimento; como las hierbas verdes, os he dado todo para comer.'
\par 7 'Pero la carne, con su vida, y con la sangre, no comeréis; porque la vida de toda carne está en la sangre, para que no sea requerida la sangre de vuestras vidas. De la mano de cada hombre, de la mano de cada (bestia) demandaré la sangre del hombre.'
\par 8 «Quien derrama sangre de hombre por el hombre, su sangre será derramada, porque a imagen de Dios hizo al hombre».
\par 9 «Y vosotros, multiplicaos y multiplicaos en la tierra».
\par 10 Y Noé y sus hijos juraron que no comerían sangre alguna que hubiera en ninguna carne, e hizo un pacto delante del Señor Dios para siempre por todas las generaciones de la tierra en este mes.
\par 11 Por esto te dijo que en este mes harías un pacto con los hijos de Israel sobre la montaña con juramento, y que rociarías sangre sobre ellos por todas las palabras del pacto que el Señor hizo con ellos para siempre.
\par 12 Y este testimonio está escrito acerca de vosotros para que lo guardéis continuamente, de modo que no comáis en ningún día sangre de animales, ni de aves, ni de ganado vacuno, durante todos los días de la tierra, y el hombre que come la sangre de bestia, o ganado, o aves, durante todos los días de la tierra, él y su descendencia serán desarraigados de la tierra.
\par 13 Y manda a los hijos de Israel que no coman sangre, para que sus nombres y su descendencia estén continuamente delante del Señor nuestro Dios.
\par 14 Y esta ley no tiene límite de días, porque es para siempre. Lo observarán a lo largo de sus generaciones, para que continúen suplicando por vosotros con sangre ante el altar; todos los días, a la hora de la mañana y de la tarde, buscarán de vosotros perdón perpetuamente delante del Señor, para guardarlo y no ser desarraigados.
\par 15 Y dio a Noé y a sus hijos una señal de que no volvería a haber diluvio sobre la tierra.
\par 16 Puso su arco en la nube como señal del pacto eterno de que no habrá más diluvio sobre la tierra para destruirla por todos los días de la tierra.
\par 17 Por esta razón está ordenado y escrito en las tablas celestiales que se celebre la fiesta de las semanas en este mes una vez al año, para renovar el pacto cada año.
\par 18 Y toda esta fiesta se celebró en el cielo desde el día de la creación hasta los días de Noé: veintiséis jubileos y cinco semanas de años [1309-1659 AM]: y Noé y sus hijos la observaron durante siete jubileos y un semana de años, hasta el día de la muerte de Noé, y desde el día de la muerte de Noé sus hijos lo destruyeron hasta los días de Abraham, y comieron sangre.
\par 19 Pero Abraham lo observó, e Isaac, Jacob y sus hijos lo observaron hasta tus días, y en tus días los hijos de Israel lo olvidaron hasta que lo celebrasteis de nuevo en este monte.
\par 20 Y ordenas a los hijos de Israel que celebren esta fiesta en todas sus generaciones, como mandamiento para ellos: un día del año en este mes celebrarán la fiesta.
\par 21 Porque es la fiesta de las semanas y la fiesta de las primicias. Esta fiesta es doble y de doble naturaleza: celébrala según lo que está escrito y grabado acerca de ella.
\par 22 Porque en el libro de la primera ley, en el que te he escrito, he escrito que la celebrarás a su tiempo, un día al año, y te expliqué sus sacrificios que los hijos de Israel debería recordarlo y celebrarlo a lo largo de sus generaciones en este mes, un día cada año.
\par 23 Y en la luna nueva del primer mes, en la luna nueva del cuarto mes, en la luna nueva del séptimo mes y en la luna nueva del décimo mes serán los días de recordación, y el días de las estaciones en las cuatro divisiones del año. Estos están escritos y ordenados como testimonio para siempre.
\par 24 Y Noé las dispuso para sí mismo como fiestas para siempre por las generaciones, de manera que le sirvieron de memorial.
\par 25 Y en la luna nueva del primer mes se le ordenó que se hiciera un arca, y ese (día) la tierra se secó y él abrió (el arca) y vio la tierra.
\par 26 Y en la luna nueva del cuarto mes se cerraron las bocas de las profundidades del abismo que había debajo. Y en la luna nueva del séptimo mes se abrieron todas las bocas de los abismos de la tierra, y las aguas comenzaron a descender en ellas.
\par 27 Y en la luna nueva del décimo mes se vieron las cimas de las montañas, y Noé se alegró.
\par 28 Y por eso las instituyó para sí como fiestas conmemorativas para siempre, y así son instituidas.
\par 29 Y los colocaron en las tablas celestiales, cada una de las cuales tenía trece semanas; de uno a otro (pasaron) su memoria, del primero al segundo, y del segundo al tercero, y del tercero al cuarto.
\par 30 Y todos los días del mandamiento serán cincuenta y dos semanas de días, y esto completará el año entero. Así está grabado y ordenado en las tablas celestiales.
\par 31 Y no se puede descuidar (este mandamiento) ni por un solo año ni de año en año.
\par 32 Y ordena a los hijos de Israel que observen los años según este cómputo: trescientos sesenta y cuatro días, y (estos) constituirán un año completo, y no alterarán su tiempo desde sus días y desde sus fiestas; porque todo les sucederá según su testimonio, y no dejarán de lado ningún día ni perturbarán ninguna fiesta.
\par 33 Pero si los descuidan y no los observan según Su mandamiento, entonces perturbarán todas sus estaciones y los años serán desalojados de este (orden), [y perturbarán las estaciones y los años serán desalojados ] y descuidarán sus ordenanzas.
\par 34 Y todos los hijos de Israel se olvidarán y no encontrarán el camino de los años, y se olvidarán de las lunas nuevas, de las estaciones y de los sábados, y se equivocarán en todo el orden de los años.
\par 35 Porque lo sé y desde ahora te lo declararé, y no es de mi propia invención; porque el libro (yace) escrito delante de mí, y en las tablas celestiales está ordenada la división de los días, para que no se olviden de las fiestas del pacto y anden según las fiestas de los gentiles, en pos de su error y de su ignorancia.
\par 36 Porque habrá quienes seguramente harán observaciones de la luna: cómo perturba las estaciones y llega de año en año diez días antes.
\par 37 Por eso vendrán sobre ellos años en los cuales perturbarán el orden y harán abominable el día del testimonio, y el día inmundo en fiesta, y confundirán todos los días, los santo con los inmundos, y el día inmundo con los santos; porque se equivocarán en cuanto a los meses, los sábados, las fiestas y los jubileos.
\par 38 Por eso te mando y te testifico, para que tú les testifiques; porque después de tu muerte tus hijos los perturbarán, de modo que no harán el año trescientos sesenta y cuatro días solamente, y por esta razón se equivocarán en cuanto a las lunas nuevas y las estaciones y los sábados y las fiestas, y comerán toda clase de sangre con toda clase de carne.

\chapter{7}

\par \textit{Noé planta una viña y ofrece un sacrificio, 1-5. Se emborracha y expone su persona, 6-9. La maldición de Canaán y la bendición de Sem y Japeth, 10-12 (cf. Gén. ix.20-8). Los hijos y nietos de Noé y sus ciudades, 13-19. Noé enseña a sus hijos sobre las causas del diluvio y les advierte que eviten comer sangre y asesinar, que guarden la ley sobre los árboles frutales y que dejen la tierra en barbecho cada siete años, como había ordenado Enoc, 20-39.}

\par 1 Y en la séptima semana del primer año [1317 AM] de este jubileo, Noé plantó vides en la montaña sobre la que había descansado el arca, llamada Lubar, una de las montañas de Ararat, y produjeron frutos en el cuarto año, [1320 AM] y guardó su fruto, y lo recogió en este año en el mes séptimo.
\par 2 Y con ello hizo vino, lo puso en una vasija y lo guardó hasta el quinto año, hasta el primer día de la luna nueva del primer mes.
\par 3 Y celebró con alegría el día de esta fiesta, e hizo en holocausto al Señor un novillo, un carnero, siete ovejas de un año cada una y un cabrito, para poder hacer expiación por sí mismo y por sus hijos.
\par 4 Primero preparó el cabrito, y puso un poco de su sangre sobre la carne que estaba sobre el altar que él había hecho, y toda la grasa puso sobre el altar donde hizo el holocausto, y el buey y el carnero y las ovejas, y puso toda su carne sobre el altar.
\par 5 Y puso encima todas sus ofrendas mezcladas con aceite, y después roció vino sobre el fuego que previamente había encendido sobre el altar, y puso incienso sobre el altar, e hizo subir un olor suave, agradable ante el Señor. su Dios.
\par 6 Y él se regocijó y bebió de este vino con alegría, él y sus hijos.
\par 7 Al anochecer, entró en su tienda y, borracho, se acostó y durmió, y se descubrió en su tienda mientras dormía.
\par 8 Y Cam vio desnudo a Noé, su padre, y salió y se lo contó a sus dos hermanos que estaban afuera.
\par 9 Entonces Sem tomó su manto y se levantaron, él y Jafet, y se pusieron el manto sobre sus hombros y retrocedieron y cubrieron la vergüenza de su padre, y sus rostros estaban hacia atrás.
\par 10 Y Noé despertó de su sueño y supo todo lo que su hijo menor le había hecho, y maldijo a su hijo y dijo: 'Maldito sea Canaán; un siervo será para sus hermanos.'
\par 11 Y bendijo a Sem, y dijo: «Bendito sea el Señor Dios de Sem, y Canaán será su siervo».
\par 12 «Dios engrandecerá a Jafet, y Dios habitará en la morada de Sem, y Canaán será su siervo».
\par 13 Cam se dio cuenta de que su padre había maldecido a su hijo menor y se disgustó por haberlo maldecido. y se separó de su padre, él y con él sus hijos, Cus, Mizraim, Fut y Canaán.
\par 14 Y edificó para sí una ciudad y le puso el nombre de su esposa Nelatamauk.
\par 15 Al verlo Jafet, tuvo envidia de su hermano y también él edificó una ciudad y la llamó Adatanes, como el de su esposa.
\par 16 Y Sem vivió con su padre Noé, y él edificó una ciudad cerca de su padre en la montaña, y también la llamó por el nombre de su esposa Sedeqetelebab.
\par 17 Y he aquí estas tres ciudades están cerca del monte Lubar; Sedeqetelebab frente a la montaña por el este; y Na'eltama'uk al sur; 'Adatan'eses hacia el oeste.
\par 18 Y estos son los hijos de Sem: Elam, Asur, Arpajshad (este hijo nació dos años después del diluvio), Lud y Aram.
\par 19 Los hijos de Jafet: Gomer, Magog, Madai, Javán, Tubal, Mesec y Tiras: estos son los hijos de Noé.
\par 20 Y en el jubileo veintiocho [1324-1372 AM] Noé comenzó a imponer a los hijos de sus hijos las ordenanzas y mandamientos, y todos los juicios que él conocía, y exhortó a sus hijos a observar la justicia y a cubrir la vergüenza de su carne, y bendecir a su Creador, honrar al padre y a la madre, amar al prójimo y guardar sus almas de la fornicación, la inmundicia y toda iniquidad.
\par 21 Porque por estas tres cosas vino el diluvio sobre la tierra, es decir, por la fornicación con la que los Vigilantes de la ley de sus ordenanzas se prostituyeron tras las hijas de los hombres y tomaron esposas entre todas las que escogieron. e hicieron el principio de la inmundicia.
\par 22 Y engendraron hijos, los Naphidim, y todos eran diferentes, y se devoraban unos a otros. Y los Gigantes mataron a los Naphil, y los Naphil mataron a los Eljo, y a los Eljo a la humanidad, y a un hombre a otro.
\par 23 Y cada uno se vendió a cometer iniquidad y a derramar mucha sangre, y la tierra se llenó de iniquidad.
\par 24 Y después de esto pecaron contra las bestias, las aves y todo lo que se mueve y camina sobre la tierra; y se derramó mucha sangre sobre la tierra, y toda imaginación y deseo de los hombres imaginaba continuamente la vanidad y el mal.
\par 25 Y el Señor destruyó todo lo que había sobre la faz de la tierra; por la maldad de sus obras y por la sangre que habían derramado en medio de la tierra, Él destruyó todo.
\par 26 Y quedamos yo y vosotros, mis hijos, y todo lo que entró con nosotros en el arca, y he aquí, veo vuestras obras delante de mí, de que no andáis con rectitud: porque habéis comenzado el camino de destrucción. andar, y os apartáis unos de otros, y tenéis envidia unos de otros, y (así sucede) que no estáis en armonía, hijos míos, cada uno con su hermano.
\par 27 Porque veo y he aquí que los demonios han comenzado sus seducciones contra vosotros y contra vuestros hijos, y ahora temo por vosotros que, después de mi muerte, derramaréis sangre de los hombres sobre la tierra, y que vosotros, también serán destruidos de la faz de la tierra.
\par 28 Porque todo aquel que derrame sangre de hombre y quien coma sangre de cualquier carne, todos serán destruidos de la tierra.
\par 29 Y no quedará ningún hombre que coma sangre o que derrame sangre de hombre sobre la tierra, ni le quedará descendencia ni descendencia que viva bajo el cielo; Porque al Seol irán, y al lugar de condenación descenderán, y a las tinieblas del abismo todos serán arrastrados por una muerte violenta.
\par 30 No se verá sobre vosotros sangre de toda la sangre que habrá en todos los días en que matéis animales, ganado o cualquier cosa que vuele sobre la tierra, y hagáis una buena obra para vuestras almas cubriendo lo que ha sido derramada sobre la faz de la tierra.
\par 31 Y no seréis como el que come sangre, sino guardaos de que nadie coma sangre delante de vosotros; cubrid la sangre, porque así se me ha ordenado testificar a vosotros y a vuestros hijos, junto con toda carne.
\par 32 Y no permitáis que el alma sea comida con la carne, para que vuestra sangre, que es vuestra vida, no sea requerida de mano de ninguna carne que la derrame sobre la tierra.
\par 33 Porque la tierra no quedará limpia de la sangre que sobre ella se ha derramado; porque (sólo) por la sangre del que la derramó será purificada la tierra por todas sus generaciones.
\par 34 Ahora pues, hijos míos, prestad atención: obrad juicio y justicia, para que plantéis con justicia sobre la faz de toda la tierra y se alce vuestra gloria delante de mi Dios, que me salvó de las aguas del diluvio.
\par 35 Y he aquí, iréis y edificaréis ciudades, y plantaréis en ellas todas las plantas que hay sobre la tierra, y además todos los árboles frutales.
\par 36 Durante tres años no se recogerá el fruto de todo lo que se come; y en el cuarto año su fruto será considerado santo [y ofrecerán las primicias], aceptable ante el Dios Altísimo, que creó el cielo. y la tierra y todas las cosas. Ofrezcan en abundancia las primicias del vino y del aceite (como) primicias sobre el altar del Señor, que lo recibe, y lo que quede, que los siervos de la casa del Señor coman delante del altar que lo recibe. ).
\par 37 Y en el quinto año haréis la liberación para que la liberéis con justicia y rectitud, y seréis justos y todo lo que plantéis prosperará.
\par 38 Porque así ordenó Enoc, el padre de vuestro padre, a Matusalén, su hijo, y Matusalén, su hijo, Lamec, y Lamec a mí me ordenó todas las cosas que sus padres le ordenaron.
\par 39 Y yo también os daré mandamientos, hijos míos, como Enoc mandó a su hijo en los primeros jubileos: cuando aún vivía, el séptimo de su generación, mandó y testificó a su hijo y a los hijos de su hijo hasta el día de su muerte.

\chapter{8}

\par \textit{Kâinâm descubre una inscripción relacionada con el sol y las estrellas, 1-4. Sus hijos, 5-8. Los hijos de Noé y Noé dividen la tierra, 10-11. La herencia de Sem, 12-21: la de Cam, 22-4: la de Jafet, 25-30. (Cf. Gen. x.)}

\par 1 En el jubileo vigésimo noveno, en la primera semana, [1373 AM] al comienzo del mismo, Arpachshad tomó para sí una esposa y su nombre era Rasu'eja, la hija de Susan, la hija de Elam, y ella dio a luz. Le dio un hijo en el tercer año en esta semana, [1375 AM] y llamó su nombre Kainam.
\par 2 Y el hijo creció, y su padre le enseñó a escribir, y fue a buscar un lugar donde poder apoderarse de una ciudad.
\par 3 Y encontró una escritura que las generaciones anteriores habían grabado en la roca, leyó lo que había en ella, la transcribió y pecó a causa de ella. porque contenía la enseñanza de los Vigilantes según la cual solían observar los presagios del sol, la luna y las estrellas en todos los signos del cielo.
\par 4 Y él lo escribió y no dijo nada al respecto; porque tenía miedo de hablar con Noé acerca de esto para que no se enojara con él a causa de ello.
\par 5 Y en el jubileo trigésimo, [1429 AM] en el segundo septenario, en su primer año, tomó para sí una esposa, y su nombre era Melka, hija de Madai, hijo de Jafet, y en el el cuarto año [1432 AM] engendró un hijo y llamó su nombre Sela; porque dijo: «En verdad he sido enviado».
\par 6 [Y en el cuarto año nació], y Sela creció y tomó para sí una esposa, y su nombre era Mu'ak, la hija de Kesed, el hermano de su padre, en el treinta y un jubileo, en la quinta semana, en el primer año [1499 AM] del mismo.
\par 7 Y ella le dio a luz un hijo en el quinto año [1503 AM] del mismo, y él llamó su nombre Eber; y tomó para sí una esposa, y su nombre era 'Azûrâd, la hija de Nebrod, en el treinta- segundo jubileo, en el séptimo septenario, en su tercer año. [1564 a.m.]
\par 8 Y en el sexto año de aquel mes, ella le dio a luz un hijo, y él llamó su nombre Peleg; porque en los días en que nació, los hijos de Noé comenzaron a dividirse la tierra entre sí: por eso llamó su nombre Peleg.
\par 9 Y lo dividieron en secreto entre ellos y se lo contaron a Noé.
\par 10 Y aconteció que al comienzo del trigésimo tercer jubileo [1569 AM] dividieron la tierra en tres partes, para Sem, Cam y Jafet, según la herencia de cada uno, en el primer año del primera semana, cuando uno de nosotros que habíamos sido enviados, estaba con ellos.
\par 11 Y llamó a sus hijos, y ellos se acercaron a él, ellos y sus hijos, y dividió la tierra en las suertes que sus tres hijos debían tomar en posesión, y ellos extendieron sus manos y tomaron las tierras. escribiendo desde el seno de Noé, su padre.
\par 12 Y en la escritura salió como suerte para Sem la mitad de la tierra, que él debería tomar como herencia para él y para sus hijos por las generaciones de la eternidad, desde el medio de la cordillera de Rafa, desde la desembocadura. del agua del río Tina, y su porción va hacia el occidente por medio de este río, y se extiende hasta llegar al agua de los abismos, de donde sale este río y vierte sus aguas en el mar Me' en, y este río desemboca en el gran mar. Y todo lo que está hacia el norte es de Jafet, y todo lo que está hacia el sur pertenece a Sem.
\par 13 Y se extiende hasta llegar a Karaso: esto está en el seno de la lengua que mira hacia el sur.
\par 14 Y su porción se extiende a lo largo del gran mar, y se extiende en línea recta hasta llegar al oeste de la lengua que mira hacia el sur: porque este mar se llama lengua del mar de Egipto.
\par 15 Y gira desde aquí hacia el sur hacia la desembocadura del gran mar en la orilla de (sus) aguas, y se extiende hacia el oeste hasta 'Afra, y se extiende hasta llegar a las aguas del río Gihón, y al sur de las aguas de Gihón, hasta la ribera de este río.
\par 16 Y se extiende hacia el oriente, hasta llegar al Jardín del Edén, al sur del mismo, y desde el oriente de toda la tierra del Edén y de todo el oriente, gira hacia el oriente y Continúa hasta llegar al este del monte llamado Rafa, y desciende hasta la orilla de la desembocadura del río Tina.
\par 17 Esta porción les tocó a Sem y a sus hijos por suerte, para que la poseyeran para siempre, por sus generaciones y para siempre.
\par 18 Y Noé se alegró de que esta porción llegara a Sem y a sus hijos, y se acordó de todo lo que había dicho con su boca en profecía; porque había dicho: 'Bendito sea el Señor Dios de Sem, y habite el Señor en la morada de Sem'.
\par 19 Y supo que el Jardín del Edén es el lugar santísimo y la morada del Señor, y el monte Sinaí, el centro del desierto, y el monte Sión, el centro del ombligo de la tierra: estos tres fueron creados. como lugares santos uno frente al otro.
\par 20 Y bendijo al Dios de los dioses, que había puesto la palabra del Señor en su boca, y al Señor para siempre.
\par 21 Y supo que Sem y sus hijos habían recibido una porción bendita y una bendición por las generaciones para siempre: toda la tierra del Edén y toda la tierra del Mar Rojo, y toda la tierra del este y la India, y sobre el Mar Rojo y sus montañas, y toda la tierra de Basán, y toda la tierra del Líbano y las islas de Kaftur, y todas las montañas de Sanir y 'Amana, y las montañas de Asur en el norte, y todas la tierra de Elam, Asur y Babel, Susan y Ma'edai, y todas las montañas de Ararat, y toda la región más allá del mar, que está más allá de las montañas de Asur hacia el norte, una tierra bendita y espaciosa, y todo lo que contiene es muy bueno.
\par 22 Y para Cam salió la segunda porción, más allá de Gihón hacia el sur a la derecha del Jardín, y se extiende hacia el sur y se extiende a todas las montañas de fuego, y se extiende hacia el oeste hasta el mar. de 'Atel y se extiende hacia el oeste hasta llegar al mar de Ma'uk -ese (mar) al que desciende todo lo que no se destruye.
\par 23 Y sale hacia el norte hasta los límites de Gadir, y sale a la costa de las aguas del mar, a las aguas del gran mar, hasta llegar al río Gihón, y va a lo largo del río. Gihón hasta llegar a la derecha del Jardín del Edén.
\par 24 Y esta es la tierra que le surgió a Cam como la porción que él había de ocupar para siempre para él y sus hijos, por sus generaciones y para siempre.
\par 25 Y para Jafet sale la tercera parte más allá del río Tina al norte de la salida de sus aguas, y se extiende al noreste hasta toda la región de Gog y a toda la tierra al este de ella.
\par 26 Y se extiende hacia el norte, hacia el norte, y se extiende hasta las montañas de Qelt hacia el norte, y hacia el mar de Ma'uk, y sale hacia el este de Gadir hasta la región de las aguas de el mar.
\par 27 Y se extiende hasta acercarse al oeste de Fara y regresa hacia 'Aferag, y se extiende hacia el este hasta las aguas del mar de Me'at.
\par 28 Y se extiende hasta la región del río Tina en dirección noreste hasta acercarse al límite de sus aguas hacia el monte Rafa, y gira hacia el norte.
\par 29 Esta es la tierra que le correspondió a Jafet y a sus hijos como parte de su herencia que él y sus hijos debían poseer por sus generaciones para siempre; cinco grandes islas y una gran tierra al norte.
\par 30 Pero hace frío, y la tierra de Cam es caliente, y la tierra de Sem no es ni caliente ni fría, sino que es una mezcla de frío y calor.

\chapter{9}

\par \textit{Subdivisión de las tres porciones entre los nietos de Noé. Entre los hijos de Cam, 1: los de Sem, 2-6: los de Jafet, 7-13. Juramento prestado por los hijos de Noé, 14-15.}

\par 1 Y Cam dividió entre sus hijos, y la primera porción salió para Cus hacia el este, y al oeste de él para Mizraim, y al oeste de él para Fut, y al oeste de él [y hacia el al oeste de allí] en el mar hacia Canaán.
\par 2 Y Sem también dividió entre sus hijos, y la primera porción salió para Cam y sus hijos, al este del río Tigris hasta que se acerca al este, toda la tierra de la India, y en el Mar Rojo en su costa. , y las aguas de Dedán, y todas las montañas de Mebri y Ela, y toda la tierra de Susan y todo lo que está del lado de Farnak hasta el Mar Rojo y el río Tina.
\par 3 Y para Asur salió la segunda porción, toda la tierra de Asur y Nínive y Sinar y hasta la frontera de la India, y sube y bordea el río.
\par 4 Y de Arpachshad salió la tercera parte, toda la tierra de la región de los caldeos al este del Éufrates, limitando con el Mar Rojo, y todas las aguas del desierto cerca de la lengua del mar que mira hacia hacia Egipto, toda la tierra del Líbano y Sanir y 'Amana hasta la frontera del Éufrates.
\par 5 Y para Aram salió la cuarta porción, toda la tierra de Mesopotamia entre el Tigris y el Éufrates al norte de los caldeos hasta el límite de las montañas de Asur y la tierra de Arara.
\par 6 Y la quinta porción salió para Lud, las montañas de Asiria y todo lo que les pertenece hasta llegar al Gran Mar, y hasta llegar al este de Asiria su hermano.
\par 7 Y Jafet también dividió la tierra de su herencia entre sus hijos.
\par 8 Y la primera parte salía hacia Gomer hacia el este, desde el lado norte hasta el río Tina; y en el norte salía para Magog todas las porciones interiores del norte hasta llegar al mar de Meat.
\par 9 Y Madai salió como su porción para poseer desde el oeste a sus dos hermanos hasta las islas y hasta las costas de las islas.
\par 10 Y para Javán salió la cuarta parte, cada isla y las islas que están hacia la frontera de Lud.
\par 11 Y para Tubal salió la quinta porción en medio de la lengua que se acerca hacia el límite de la porción de Lud a la segunda lengua, desde la región más allá de la segunda lengua a la tercera lengua.
\par 12 Y para Mesec salió la sexta porción, toda la región más allá de la tercera lengua hasta llegar al este de Gadir.
\par 13 Y para Tiras surgió la séptima porción, cuatro grandes islas en medio del mar, que llegan hasta la porción de Cam [y las islas de Kamaturi salieron por sorteo para los hijos de Arpachshad como su herencia].
\par 14 Y así los hijos de Noé se dividieron entre sus hijos en presencia de Noé su padre, y él los obligó a todos con un juramento, imprecando una maldición sobre todo el que intentara apoderarse de la porción que no le había tocado. por su suerte.
\par 15 Y todos dijeron: 'Que así sea; así sea para ellos y para sus hijos por los siglos de sus generaciones, hasta el día del juicio, en el cual el Señor Dios los juzgará con espada y con fuego por toda la inmunda maldad de sus errores, con que han llenado la tierra. transgresión, inmundicia, fornicación y pecado.

\chapter{10}

\par \textit{Los espíritus malignos extravían a los hijos de Noé, 1-2. La oración de Noé, 3-6. A Mastêmâ se le permitió retener una décima parte de sus espíritus súbditos, 7-11. Noé enseñó a los ángeles el uso de hierbas para resistir a los demonios, 12-14. Noé muere, 15-17. La construcción de Babel y la confusión de lenguas, 18-27. Canaán se apodera de Palestina, 29-34. Madai recibe a Media, 33-6.}

\par 1 Y en la tercera semana de este jubileo, los demonios inmundos comenzaron a extraviar a los hijos de Noé, a extraviarlos y destruirlos.
\par 2 Los hijos de Noé fueron a ver a Noé, su padre, y le contaron acerca de los demonios que engañaban, cegaban y mataban a los hijos de sus hijos.
\par 3 Y oró delante del Señor su Dios, y dijo:
\par    
\par     'Dios de los espíritus de toda carne, que has tenido misericordia de mí  
\par     Y nos salvaste a mí y a mis hijos de las aguas del diluvio,  
\par     Y no me has hecho perecer como hiciste con los hijos de perdición;
\par    
\par     Porque grande ha sido tu gracia para conmigo,  
\par     Y grande ha sido tu misericordia para con mi alma;
\par    
\par     Que tu gracia se alce sobre mis hijos,  
\par     Y no dejes que los espíritus malignos los dominen.  
\par     Para que no los exterminen de la tierra.
\par    
\par 4 Pero bendíceme a mí y a mis hijos, para que podamos crecer y multiplicarnos y llenar la tierra.
\par 5 Y Tú sabes cómo actuaron en mis días tus Vigilantes, los padres de estos espíritus; y en cuanto a estos espíritus que están vivientes, aprisionalos y mantenlos firmes en el lugar de condenación, y no dejes que traigan destrucción sobre el mundo. hijos de tu siervo, Dios mío; porque estos son malignos y creados para destruir.
\par 6 Y no se enseñoreen de los espíritus de los vivientes; porque sólo Tú puedes ejercer dominio sobre ellos. Y que no tengan poder sobre los hijos de los justos desde ahora y para siempre.'
\par 7 Y el Señor nuestro Dios nos ordenó que ataramos a todos.
\par 8 Y el jefe de los espíritus, Mastêmâ, vino y dijo: 'Señor, Creador, que algunos de ellos permanezcan delante de mí, y que escuchen mi voz, y hagan todo lo que les diga; porque si algunos de ellos no me son dejados, no podré ejecutar el poder de mi voluntad sobre los hijos de los hombres; porque estos son para corrupción y extravío ante mi juicio, porque grande es la maldad de los hijos de los hombres.'
\par 9 Y dijo: «Que la décima parte de ellos permanezca delante de él, y que las nueve partes desciendan al lugar de la condenación».
\par 10 Y a uno de nosotros nos ordenó que le enseñáramos a Noé todas sus medicinas; porque sabía que no andarían en rectitud, ni lucharían en justicia.
\par 11 E hicimos conforme a todas sus palabras: atamos a todos los malvados en el lugar de condenación y dejamos la décima parte de ellos para que se sometieran ante Satanás en la tierra.
\par 12 Y le explicamos a Noé todas las medicinas para sus enfermedades, junto con sus seducciones, y cómo podría curarlos con hierbas de la tierra.
\par 13 Y Noé anotó todo en un libro, mientras le instruíamos sobre cada tipo de medicina. De esta manera se impidió que los espíritus malignos (dañaran) a los hijos de Noé.
\par 14 Y entregó todo lo que había escrito a Sem, su hijo mayor; porque lo amaba mucho más que a todos sus hijos.
\par 15 Y Noé durmió con sus padres y fue sepultado en el monte Lubar, en la tierra de Ararat.
\par 16 Cumplió novecientos cincuenta años de su vida, diecinueve jubileos y dos semanas y cinco años. [1659 a.m.]
\par 17 Y en su vida en la tierra superó a los hijos de los hombres excepto a Enoc, debido a la justicia en la que era perfecto. Porque el oficio de Enoc fue ordenado para testimonio a las generaciones del mundo, para que él pudiera contar todos los hechos de generación en generación, hasta el día del juicio.
\par 18 Y en el jubileo trigésimo trigésimo tercero, en el año primero, en el segundo septenario, Peleg tomó para sí una esposa, cuyo nombre era Lomna, hija de Sinaar, y ella le dio a luz un hijo en el año cuarto de este semana, y llamó su nombre Reu; porque dijo: 'He aquí, los hijos de los hombres se han vuelto malos por el malvado propósito de construirse una ciudad y una torre en la tierra de Sinar.'
\par 19 Porque partieron de la tierra de Ararat hacia el este, hacia Sinar; porque en sus días edificaron la ciudad y la torre, diciendo: 'Id, subamos por ella al cielo'.
\par 20 Y comenzaron a construir, y en la cuarta semana hicieron ladrillos al fuego, y los ladrillos les servían de piedra, y el barro con que los unían era asfalto que sale del mar y del fuentes de agua en la tierra de Sinar.
\par 21 Y lo construyeron: cuarenta y tres años [1645-1688 AM] estuvieron edificandolo; su ancho era de 203 ladrillos, y la altura (de un ladrillo) era la tercera parte de uno; su altura era de 5433 codos y 2 palmos, y (la extensión de un muro era) de trece estadios (y de los otros treinta estadios).
\par 22 Y el Señor nuestro Dios nos dijo: 'He aquí, son un solo pueblo, y comienzan a hacer esto, y ahora nada les será negado. Vayamos, bajemos y confundamos su lengua, para que no entiendan el habla de los demás, y sean dispersos en ciudades y naciones, y ya no habrá un mismo propósito en ellos hasta el día del juicio.'
\par 23 Y el Señor descendió, y nosotros descendimos con él para ver la ciudad y la torre que los hijos de los hombres habían construido.
\par 24 Y él confundió su lengua, y ya no entendían el habla de los demás, y entonces dejaron de construir la ciudad y la torre.
\par 25 Por eso toda la tierra de Sinar se llama Babel, porque allí el Señor confundió toda la lengua de los hijos de los hombres, y desde allí fueron dispersados ​​en sus ciudades, cada uno según su lengua y su nación.
\par 26 Y el Señor envió un fuerte viento contra la torre y la derribó por tierra. Y he aquí que estaba entre Asur y Babilonia en la tierra de Sinar, y llamaron su nombre Derrocamiento.
\par 27 En el cuarto septenario del primer año, al principio del mismo, en el jubileo treinta y cuatro, fueron dispersados ​​de la tierra de Sinar.
\par 28 Cam y sus hijos entraron en la tierra que él había de ocupar y que había adquirido como su porción en la tierra del sur.
\par 29 Y Canaán vio que la tierra del Líbano hasta el río de Egipto era muy buena, y no entró en la tierra de su herencia al oeste (es decir, hacia) el mar, y habitó en la tierra de Líbano, al oriente y al occidente desde el límite del Jordán y desde el límite del mar.
\par 30 Y Cam, su padre, y Cus y Mizraim, sus hermanos, le dijeron: 'Tú te has establecido en una tierra que no es tuya, y que no nos tocó en suerte; no lo hagas; porque si lo haces, tú y tus hijos caeréis en la tierra y serán maldecidos por la sedición; porque por la sedición os habéis establecido, y por la sedición caerán tus hijos, y tú serás desarraigada para siempre.'
\par 31 'No habitéis en la morada de Sem; porque a Sem y a sus hijos les tocó en suerte.'
\par 32 «Maldito serás, y maldito serás entre todos los hijos de Noé, por la maldición con la que nos vinculamos mediante juramento en presencia del santo juez y en presencia de nuestro padre Noé».
\par 33 Pero él no los escuchó y habitó en la tierra del Líbano, desde Hamat hasta la entrada de Egipto, él y sus hijos hasta el día de hoy.
\par 34 Por eso esa tierra se llama Canaán.
\par 35 Y Jafet y sus hijos fueron hacia el mar y habitaron en la tierra de su porción, y Madai vio la tierra del mar y no le agradó, y pidió una (porción) a Cam, Asiria y Arpajshad, hermano de su mujer, y habitó en tierra de Media, cerca del hermano de su mujer, hasta el día de hoy.
\par 36 Y llamó Media, a su morada y a la de sus hijos, Media, según el nombre de su padre Madai.

\chapter{11}

\par \textit{Reu y Serug, 1 (cf. Gén. xi.20, 21). Aumento de la guerra, el derramamiento de sangre, el consumo de sangre y la idolatría, 2-7. Nacor y Taré, 8-14 (cf. Gén. xi.22-30). El conocimiento de Abram sobre Dios y las obras maravillosas, 15-24.}

\par 1 Y en el jubileo trigésimo quinto, en la tercera semana, en el primer año [1681 AM] del mismo, Reu tomó para sí una esposa, y su nombre era 'Ôrâ, la hija de 'Ûr, el hijo de Kesed. , y ella le dio a luz un hijo, y él llamó su nombre Sêrôh, en el séptimo año de este septenario en este jubileo. [1687 a.m.]
\par 2 Y los hijos de Noé comenzaron a guerrear unos contra otros, a tomar cautivos y a matarse unos a otros, y a derramar sangre de hombres sobre la tierra, y a comer sangre, y a construir ciudades fuertes, y murallas, y torres y personas (comenzaron) a exaltarse sobre la nación, y a fundar principios de reinos, y a ir a la guerra pueblo contra pueblo, y nación contra nación, y ciudad contra ciudad, y todos (comenzaron) a hacer el mal, y a adquirir armas y a enseñar a sus hijos la guerra, y comenzaron a tomar ciudades y a vender esclavos y esclavas.
\par 3 Y 'Ûr, hijo de Kesed, edificó la ciudad de 'Ara de los caldeos, y le puso el nombre de su propio nombre y del nombre de su padre. Y se hicieron imágenes de fundición, y adoraron cada uno al ídolo, la imagen de fundición que se habían hecho, y comenzaron a hacer imágenes talladas y simulacros inmundos, y espíritus malignos los ayudaban y seducían a cometer transgresión e inmundicia.
\par 4 Y el príncipe Mastêmâ se esforzó en hacer todo esto, y envió otros espíritus, los que estaban bajo su mano, para cometer toda clase de maldad y pecado, y toda clase de transgresión, para corromper y destruir, y para derramar sangre sobre la tierra.
\par 5 Por esta razón llamó el nombre de Sêrôh, Serug, porque todos se volvían a cometer toda clase de pecado y transgresión.
\par 6 Creció y habitó en Ur de los caldeos, cerca del padre de la madre de su mujer, y adoró ídolos y tomó mujer en el jubileo trigésimo sexto, en el quinto septenario, en el primer año del mismo, [1744 AM] y su nombre era Melka, hija de Kaber, hija del hermano de su padre.
\par 7 Y ella le dio a luz a Nacor, en el primer año de esta semana, y él creció y habitó en Ur de los caldeos, y su padre le enseñó las investigaciones de los caldeos para adivinar y augurar, según los signos del cielo.
\par 8 Y en el jubileo trigésimo séptimo, en el sexto septenario, en su primer año, [1800 a.m.] tomó para sí una esposa, y su nombre era Ijaska, hija de Nestag de los caldeos.
\par 9 Y ella dio a luz a Taré en el séptimo año de esta semana. [1806 a.m.]
\par 10 Y el príncipe Mastêmâ envió cuervos y pájaros para devorar la semilla que se había sembrado en la tierra, para destruir la tierra y robar a los hijos de los hombres sus trabajos. Antes de que pudieran arar la semilla, los cuervos la recogieron de la superficie del suelo.
\par 11 Y por eso llamó su nombre Taré, porque los cuervos y las aves los redujeron a la miseria y devoraron su semilla.
\par 12 Y los años comenzaron a ser estériles a causa de los pájaros, y devoraban de los árboles todos los frutos de los árboles: sólo con gran esfuerzo podían salvar en sus vidas un poco de todos los frutos de la tierra. días.
\par 13 Y en este jubileo trigésimo noveno, en la segunda semana del primer año, [1870 AM] Taré tomó para sí una esposa, y su nombre era Edna, hija de Abram, hija de la hermana de su padre. Y en el séptimo año de esta semana [1876 AM] ella le dio a luz un hijo, y llamó su nombre Abram, por el nombre del padre de su madre;
\par 14 porque murió antes de que su hija concibiera un hijo.
\par 15 Y el niño comenzó a comprender los errores de la tierra que todos se extraviaban tras las imágenes talladas y la inmundicia, y su padre le enseñó a escribir, y tenía dos semanas de años, [1890 AM] y se separó de su padre, para que no adorara ídolos con él.
\par 16 Y comenzó a orar al Creador de todas las cosas para que lo salvara de los errores de los hijos de los hombres y para que su porción no cayera en el error de la inmundicia y la vileza.
\par 17 Y llegó el tiempo de sembrar la semilla en la tierra, y todos salieron juntos para proteger su semilla contra los cuervos, y Abram salió con los que iban, y el niño era un niño de catorce años.
\par 18 Y una nube de cuervos vino para devorar la semilla, y Abram corrió a su encuentro antes de que se posaran en el suelo, y les gritó antes de que se posaran en el suelo para devorar la semilla, y dijo: «No desciendas; volved». al lugar de donde vinisteis', y procedieron a regresar.
\par 19 E hizo retroceder las nubes de cuervos aquel día setenta veces, y de todos los cuervos que había en toda la tierra donde estaba Abram, no se posó allí ni uno solo.
\par 20 Y todos los que estaban con él en toda la tierra lo vieron gritar, y todos los cuervos retrocedieron, y su nombre se hizo grande en toda la tierra de los caldeos.
\par 21 Y este año vinieron a él todos los que querían sembrar, y él fue con ellos hasta que terminó el tiempo de la siembra; y sembraron su tierra, y ese año trajeron suficiente grano a casa, comieron y se saciaron.
\par 22 Y en el primer año de la quinta semana [1891 AM] Abram enseñó a los que hacían herramientas para bueyes, a los artífices de madera, y construyeron una vasija sobre el suelo, frente al marco del arado, para poner y la semilla cayó de allí sobre la reja del arado, y quedó escondida en la tierra, y ya no temieron a los cuervos.
\par 23 Y de esta manera construyeron recipientes sobre el suelo sobre todos los marcos de los arados, y sembraron y labraron toda la tierra, tal como Abram les había ordenado, y ya no temieron a las aves.

\chapter{12}

\par \textit{Abram busca librar a Taré de la idolatría, 1-8. Se casa con Sarai, 9. Harán y Nacor, 9-11. Abram quema los ídolos: muerte de Harán, 12-14 (cf. Gén. xi.28). Taré y su familia van a Harán, 15. Abram observa las estrellas y ora, 16-21. Se le ordena ir a Canaán y ser bendecido, 22-4. Se le dio el poder de hablar hebreo, 25-7. Deja Harán hacia Canaán, 28-31. (Cf. Gen. xi.31-xii.3.)}

\par 1 Y aconteció en la sexta semana, en su séptimo año, que Abram dijo a Taré su padre, diciendo: '¡Padre!'
\par 2 Y él dijo: «Heme aquí, hijo mío». Y él dijo,
\par    
\par     '¿Qué ayuda y provecho tenemos de esos ídolos que tú adoras?  
\par     ¿Y ante cuál te inclinas?
\par    
\par 3 Porque no hay espíritu en ellos,  
\par     Porque son formas mudas y que extravían el corazón.  
\par     No los adoréis:
\par    
\par 4 Adorad al Dios del cielo,  
\par     ¿Quién hace que la lluvia y el rocío desciendan sobre la tierra?  
\par     Y hace todo lo que hay en la tierra,
\par    
\par     Y ha creado todo por su palabra,  
\par     Y toda vida proviene de delante de Su rostro.
\par    
\par 5 ¿Por qué adoráis cosas que no tienen espíritu en ellas?  
\par     Porque son obra de manos (de hombres),
\par    
\par     Y sobre vuestros hombros los lleváis,  
\par     Y no tenéis ayuda de ellos,
\par    
\par     Pero son motivo de gran vergüenza para quienes los hacen,  
\par     y extravío del corazón de los que los adoran:  
\par     No los adoréis.
\par    
\par 6 Y su padre le dijo: Yo también lo sé, hijo mío, pero ¿qué haré con un pueblo que me ha hecho servir delante de ellos?
\par 7 Y si les digo la verdad, me matarán; porque su alma se adhiere a ellos para adorarlos y honrarlos.
\par 8 Guarda silencio, hijo mío, para que no te maten. Y habló estas palabras a sus dos hermanos, y ellos se enojaron contra él y él guardó silencio.
\par 9 Y en el jubileo cuadragésimo, en el segundo septenario, en su año séptimo, [1925 AM] Abram tomó para sí una esposa, y su nombre era Sarai, hija de su padre, y ella fue su esposa.
\par 10 Y Harán, su hermano, tomó mujer en el tercer año de la tercera semana, [1928 AM] y ella le dio a luz un hijo en el séptimo año de esta semana, [1932 AM] y él llamó su nombre. Lote.
\par 11 Y Nacor, su hermano, tomó mujer.
\par 12 Y en el año sesenta de la vida de Abram, es decir, en la cuarta semana, en su cuarto año, [1936 AM] Abram se levantó de noche, y quemó la casa de los ídolos, y quemó todo lo que Estaba en la casa y ningún hombre lo sabía.
\par 13 Y se levantaron de noche y trataron de salvar a sus dioses de en medio del fuego.
\par 14 Y Harán se apresuró a salvarlos, pero el fuego ardió sobre él, y fue quemado en el fuego, y murió en Ur de los Caldeos delante de Taré su padre, y lo sepultaron en Ur de los Caldeos.
\par 15 Y Teraj salió de Ur de los caldeos, él y sus hijos, para ir a la tierra del Líbano y a la tierra de Canaán, y habitó en la tierra de Harán, y Abram habitó con su padre Taré en Harán. dos semanas de años.
\par 16 Y en la sexta semana, en su quinto año, [1951 AM] Abram se sentó toda la noche en la luna nueva del séptimo mes para observar las estrellas desde la tarde hasta la mañana, para ver qué sucedería. ser el carácter del año con respecto a las lluvias, y estaba solo mientras se sentaba y observaba.
\par 17 Y una palabra vino a su corazón y dijo: Todos los signos de las estrellas, y los signos de la luna y del sol están todos en la mano del Señor. ¿Por qué los busco?
\par    
\par 18 Si Él quiere, hace llover por la mañana y por la tarde;  
\par     Y si quiere, lo retiene,  
\par     Y todas las cosas están en su mano.
\par    
\par 19 Y aquella noche oró y dijo:  
\par     'Dios mío, Dios Altísimo, sólo tú eres mi Dios,  
\par     Y a ti y a tu dominio te he elegido.  
\par     Y tú has creado todas las cosas,  
\par     Y todas las cosas que son obra de tus manos.
\par    
\par 20 Líbrame de las manos de los espíritus malignos que dominan los pensamientos del corazón de los hombres,  
\par     Y que no me desvíen de ti, Dios mío.
\par    
\par     Y establecenos a mí y a mi descendencia para siempre.  
\par     Que no nos extraviemos desde ahora y para siempre.'
\par    
\par 21 Y él dijo: «¿Volveré a Ur de los caldeos que buscan mi rostro para volver a ellos? ¿Me quedaré aquí en este lugar?» El camino recto delante de Ti prospere en las manos de Tu siervo para que él lo cumpla y para que yo no camine en el engaño de mi corazón, oh Dios mío.'
\par 22 Y cuando terminó de hablar y de orar, he aquí, la palabra del Señor le fue enviada a través de mí, diciendo: 'Vete de tu tierra, de tu parentela y de la casa de tu padre, a un tierra que te mostraré, y haré de ti una nación grande y numerosa.
\par    
\par 23 Y te bendeciré  
\par     Y engrandeceré tu nombre,  
\par     Y serás bendito en la tierra,  
\par     Y en ti serán benditas todas las familias de la tierra,  
\par     Y bendeciré a los que te bendigan,  
\par     Y maldice a los que te maldicen.
\par    
\par 24 Y seré un Dios para ti y para tu hijo, y para el hijo de tu hijo, y para toda tu descendencia. No temas, desde ahora y para todas las generaciones de la tierra, yo soy tu Dios.'
\par 25 Y el Señor Dios dijo: «Abre su boca y sus oídos, para que oiga y hable con su boca en la lengua que le ha sido revelada». porque había cesado de la boca de todos los hijos de los hombres desde el día de la destrucción (de Babel).
\par 26 Y abrí su boca, sus oídos y sus labios, y comencé a hablar con él en hebreo, en la lengua de la creación.
\par 27 Y tomó los libros de sus padres, que estaban escritos en hebreo, los transcribió y comenzó desde entonces a estudiarlos, y yo le hice saber lo que no podía entender, y él los estudió durante los seis meses de lluvia.
\par 28 Y aconteció en el séptimo año de la sexta semana [1953 AM] que habló con su padre y le informó que dejaría Harán para ir a la tierra de Canaán, para verla y regresar a él.
\par 29 Y Taré su padre le dijo: Ve en paz:
\par    
\par     Que el Dios eterno enderece tu camino.  
\par     Y el Señor [(estar) contigo y] protegerte de todo mal,  
\par     Y te conceda gracia, misericordia y favor delante de los que te vean,  
\par     Y que ninguno de los hijos de los hombres tenga poder sobre ti para hacerte daño;  
\par     Ve en paz.
\par    
\par 30 Y si ves una tierra agradable a tus ojos para habitar, levántate y llévame contigo y toma contigo a Lot, el hijo de tu hermano Harán, como a tu propio hijo: el Señor esté contigo.
\par 31 Y tu hermano Nacor, vete conmigo hasta que vuelvas en paz, y nos iremos contigo todos juntos.

\chapter{13}

\par \textit{Abram viaja de Harán a Siquem en Canaán, de allí a Hebrón y de allí a Egipto, 1-14a. Regresa a Canaán donde Lot se separa de él, recibe la promesa de Canaán y viaja a Hebrón, 14b-21. El ataque de Quedorlaomer a Sodoma y Gomorra: Lot hecho cautivo, 22-4. Se promulga la ley de diezmos, 25-9. (Cf. Gen. xii.4-10, 15-17, 19-20; xiii.11-18; xiv.8-14; 21-4.)}

\par 1 Abram partió de Harán y tomó a Sarai, su esposa, y a Lot, hijo de su hermano Harán, a la tierra de Canaán, y llegó a Asur, y se dirigió a Siquem, y habitó cerca de una encina alta.
\par 2 Y vio que la tierra era muy agradable desde la entrada de Hamat hasta la encina alta.
\par 3 Y el Señor le dijo: «A ti y a tu descendencia daré esta tierra».
\par 4 Y edificó allí un altar, y sobre él ofreció un holocausto al Señor que se le había aparecido.
\par 5 Y de allí se fue a la montaña. . Betel al occidente y Hai al oriente, y plantó allí su tienda.
\par 6 Y vio y vio que la tierra era muy amplia y buena, y en ella crecía de todo: vides, higueras, granados, encinas, encinas, encinas, olivos, cedros, cipreses, dátiles y todos los árboles de el campo, y había agua en los montes.
\par 7 Y bendijo al Señor que lo había sacado de Ur de los caldeos y lo había traído a esta tierra.
\par 8 Y aconteció en el primer año, en la séptima semana, en la luna nueva del primer mes, 1954 AM], que edificó un altar en este monte, e invocó el nombre del Señor: 'Tú , el Dios eterno, eres mi Dios.'
\par 9 Y ofreció sobre el altar un holocausto al Señor, para que estuviera con él y no lo desamparara todos los días de su vida.
\par 10 Y salió de allí y se dirigió hacia el sur, y llegó a Hebrón, y Hebrón estaba entonces edificada, y habitó allí dos años, y de allí se fue a la tierra del sur, a Bealot, y hubo hambre en la tierra.
\par 11 Y Abram llegó a Egipto en el tercer año del septenario, y permaneció en Egipto cinco años antes de que le quitaran a su esposa.
\par 12 En aquel tiempo se construyó Tanais en Egipto, siete años después de Hebrón.
\par 13 Y aconteció que cuando Faraón se apoderó de Sarai, la esposa de Abram, el Señor castigó a Faraón y a su casa con grandes plagas a causa de Sarai, la esposa de Abram.
\par 14 Y Abram era muy glorioso por sus posesiones en ovejas, vacas, asnos, caballos, camellos, siervos y siervas, y en plata y oro en abundancia. Y Lot también el hijo de su hermano, era rico.
\par 15 Y Faraón devolvió a Sarai, la esposa de Abram, y lo envió fuera de la tierra de Egipto, y él se dirigió al lugar donde había levantado su tienda al principio, al lugar del altar, con Hai. al oriente, y a Betel al occidente, y bendijo al Señor su Dios, que le había hecho volver en paz.
\par 16 Y aconteció en el jubileo cuadragésimo primero, en el tercer año de la primera semana, [1963 AM] que regresó a este lugar y ofreció sobre él un holocausto, e invocó el nombre del Señor, y dijo: 'Tú, Dios Altísimo, eres mi Dios por los siglos de los siglos'.
\par 17 Y en el cuarto año de esta semana [1964 AM] Lot se separó de él, y Lot habitó en Sodoma, y ​​los hombres de Sodoma eran extremadamente pecadores.
\par 18 Y le dolió en el corazón que el hijo de su hermano se hubiera separado de él; porque no tenía hijos.
\par 19 En aquel año en que Lot fue llevado cautivo, el Señor dijo a Abram, después que Lot se separó de él, en el cuarto año de este septenario: «Levanta tus ojos desde el lugar donde habitas, hacia el norte y hacia el sur». , y hacia el oeste y hacia el este.
\par 20 Porque toda la tierra que ves te la daré a ti y a tu descendencia para siempre, y haré tu descendencia como la arena del mar; aunque un hombre cuente el polvo de la tierra, tu descendencia será no estar numerado.
\par 21 Levántate, camina por la tierra a lo largo y a lo ancho de ella, y míralo todo; porque a tu descendencia la daré. Y Abram fue a Hebrón y habitó allí.
\par 22 Y en este año vinieron Quedorlaomer, rey de Elam, y Amrafel, rey de Sinar, y Arioc, rey de Selasar, y Tergal, rey de las naciones, y mataron al rey de Gomorra, y el rey de Sodoma huyó, y muchos Cayó a causa de las heridas en el valle de Siddim, junto al Mar Salado.
\par 23 Y tomaron cautivos a Sodoma, a Adán y a Zeboim, y también a Lot, hijo del hermano de Abram, y todos sus bienes, y fueron a Dan.
\par 24 Uno de los que había escapado vino y le dijo a Abram que el hijo de su hermano había sido llevado cautivo y armó a sus sirvientes. . .
\par 25 . . . . para Abram y para su descendencia, el décimo de las primicias de los frutos para el Señor, y el Señor lo ordenó como orden perpetua para que lo dieran a los sacerdotes que servían delante de Él, para que lo poseyeran para siempre.
\par 26 Y esta ley no tiene límite de días; porque Él ha ordenado por generaciones para siempre que den al Señor el diezmo de todo, de la semilla, del vino, del aceite, del ganado vacuno y de las ovejas.
\par 27 Y lo dio a sus sacerdotes para que comieran y bebieran con alegría delante de él.
\par 28 Y el rey de Sodoma se acercó a él, se inclinó ante él y le dijo: «Nuestro Señor Abram, danos las almas que has rescatado, pero que el botín sea tuyo».
\par 29 Y Abram le dijo: 'Levanto mis manos al Dios Altísimo, para que desde un hilo hasta la correa de un zapato no tome nada que sea tuyo, no sea que digas: «He enriquecido a Abram». excepto lo que comieron los jóvenes y la porción de los hombres que fueron conmigo: Aner, Escol y Mamre. Estos tomarán su parte.'

\chapter{14}

\par \textit{Abram recibe la promesa de un hijo y de una descendencia innumerable, 1-7. Ofrece un sacrificio y se le dice que su descendencia está en Egipto, 8-17. El pacto de Dios con Abram, 18-20. Agar da a luz a Ismael, 21-4. (Cf. Gen. xv.; xvi.1-4, 11.)}


\par 1 Después de estas cosas, en el año cuarto de esta semana, en la luna nueva del tercer mes, vino palabra del Señor a Abram en sueños, diciendo: 'No temas, Abram; Soy tu defensor y tu recompensa será sumamente grande.
\par 2 Y él dijo: «Señor, Señor, ¿qué me darás, ya que me voy de aquí sin hijos, y el hijo de Maseq, el hijo de mi sierva, es el damasek Eliezer? Él será mi heredero, y para mí no has dado semilla.'
\par 3 Y él le dijo: 'Éste no será tu heredero, sino uno que saldrá de tus propias entrañas; él será tu heredero.'
\par 4 Y lo llevó afuera y le dijo: «Mira hacia el cielo y cuenta las estrellas, si puedes contarlas».
\par 5 Y miró hacia el cielo y vio las estrellas. Y le dijo: 'Así será tu descendencia'.
\par 6 Y creyó en el Señor, y le fue contado por justicia.
\par 7 Y Él le dijo: 'Yo soy el Señor que te saqué de Ur de los caldeos, para darte la tierra de los cananeos para que la poseas para siempre; y seré Dios para ti y para tu descendencia después de ti.'
\par 8 Y él dijo: «Señor, Señor, ¿por qué sabré que lo heredaré?»
\par 9 Y él le dijo: «Tómame una novilla de tres años, una cabra de tres años, una oveja de tres años, una tórtola y un palomo».
\par 10 Y tomó todo esto a mediados de mes y habitó en el roble de Mamre, que está cerca de Hebrón.
\par 11 Y edificó allí un altar y sacrificó todos estos; y derramó su sangre sobre el altar, y los partió por la mitad, y los puso uno frente al otro; pero las aves no las dividieron.
\par 12 Y los pájaros cayeron sobre los pedazos, y Abram los ahuyentó, sin permitir que los pájaros los tocasen.
\par 13 Y aconteció que cuando se puso el sol, un éxtasis cayó sobre Abram, y ¡he aquí! un horror de gran oscuridad cayó sobre él, y se dijo a Abram: 'Sabe con certeza que tu descendencia será extranjera en una tierra (que) no es de ellos, y los traerán en servidumbre, y los afligirán cuatro cien años.'
\par 14 «Y también juzgaré a la nación a la cual serán esclavos, y después de eso saldrán de allí con mucha riqueza».
\par 15 «Y volverás en paz con tus padres y serás sepultado en una buena vejez».
\par 16 'Pero en la cuarta generación volverán acá; porque la iniquidad de los amorreos aún no ha sido cumplida.'
\par 17 Y despertó de su sueño, se levantó y el sol se había puesto; y había una llama, y ​​¡he aquí! Un horno humeaba y una llama de fuego pasaba entre los pedazos.
\par 18 Y aquel día el Señor hizo un pacto con Abram, diciendo: 'A tu descendencia daré esta tierra, desde el río de Egipto hasta el río grande, el río Éufrates, a los ceneos, a los cenezeos, a los cadmoneos, los ferezeos, los refaítas, los facoreos, los heveos, los amorreos, los cananeos, los gergeseos y los jebuseos.
\par 19 Y pasó el día, y Abram ofreció los pedazos, las aves, sus ofrendas de frutas y sus libaciones, y el fuego los devoró.
\par 20 Ese día hicimos un pacto con Abram, como habíamos hecho con Noé en este mes; y Abram renovó para sí la fiesta y el rito para siempre.
\par 21 Y Abram se regocijó y le contó todas estas cosas a Sarai su esposa; y él creyó que tendría descendencia, pero ella no dio a luz.
\par 22 Y Sarai aconsejó a Abram su marido, y le dijo: «Entra con Agar, mi sierva egipcia; tal vez yo te edifique descendencia junto a ella».
\par 23 Y Abram escuchó la voz de Sarai su esposa y le dijo: «Hazlo». Y Sarai tomó a Agar su sierva egipcia, y se la dio a Abram su marido por mujer.
\par 24 Y él entró a ella, y ella concibió y le dio a luz un hijo, y él llamó su nombre Ismael, en el quinto año de esta semana [1965 AM]; y este fue el año ochenta y seis en la vida de Abram.

\chapter{15}

\par \textit{Abram celebra la fiesta de las primicias, 1-2: su nombre cambió y se instituyó la circuncisión, 3-14. El nombre de Sarai cambió e Isaak prometió, 15-21. Abraham, Ismael y toda su casa circuncidados, 22-4. La circuncisión es una ordenación eterna, 25, 26. Israel comparte este honor con los ángeles más elevados que fueron creados circuncidados, 27-9. Israel sujeto sólo a Dios: otras naciones a los ángeles, 30-2. Futura infidelidad de Israel, 33-4. (Cf. Gen. xvii.)}

\par 1 Y en el quinto año de la cuarta semana de este jubileo, [1979 AM] en el tercer mes, a la mitad del mes, Abram celebró la fiesta de las primicias de la cosecha del grano.
\par 2 Y ofreció nuevas ofrendas sobre el altar, las primicias de los productos al Señor, una novilla, una cabra y una oveja sobre el altar como holocausto al Señor; sus ofrendas de frutos y sus libaciones las ofreció sobre el altar con incienso.
\par 3 Y el Señor se apareció a Abram y le dijo: 'Yo soy el Dios Todopoderoso; apruebate delante de mí y sé perfecto.'
\par 4 'Y haré mi pacto entre mí y ti, y te multiplicaré en gran manera.'
\par 5 Y Abram cayó sobre su rostro, y Dios habló con él y le dijo:
\par    
\par 6 «Mira, mi ordenanza está contigo,  
\par     Y serás padre de muchas naciones.
\par    
\par 7 Ya no se llamará más tu nombre Abram,  
\par     Pero tu nombre desde ahora y para siempre será Abraham.  
\par     Porque te he hecho padre de muchas naciones.
\par    
\par 8 Y te haré muy grande,  
\par     Y te convertiré en naciones,  
\par     Y de ti saldrán reyes.
\par    
\par 9 Y estableceré mi pacto entre mí y ti, y tu descendencia después de ti, por sus generaciones, como pacto eterno, para que yo sea un Dios para ti y para tu descendencia después de ti.
\par 10 (Y te daré a ti y a tu descendencia después de ti) la tierra donde has sido peregrino, la tierra de Canaán, para que la poseas para siempre, y yo seré su Dios.'
\par 11 Y el Señor dijo a Abraham: 'Y tú, guarda mi pacto, tú y tu descendencia después de ti; y circuncidad a todo varón entre vosotros, y circuncidad vuestros prepucios, y será una señal de pacto eterno entre Yo y vosotros.'
\par 12 «Y circuncidaréis al niño, al octavo día, a todo varón de vuestras generaciones, al nacido en casa, o al que habéis comprado con dinero a algún extraño, al que habéis adquirido que no sea de vuestra descendencia .'
\par 13 «El que nazca en tu casa será circuncidado, y los que hayas comprado con dinero serán circuncidados, y mi pacto estará en tu carne como estatuto eterno».
\par 14 «Y el varón incircunciso que no esté circuncidado en la carne de su prepucio al octavo día, esa alma será cortada de su pueblo, porque ha violado Mi pacto».
\par 15 Y Dios dijo a Abraham: «En cuanto a Sarai tu esposa, su nombre no se llamará más Sarai, sino que Sara será su nombre».
\par 16 «Y la bendeciré y te daré un hijo de ella, y lo bendeciré, y se convertirá en una nación, y de él procederán reyes de naciones».
\par 17 Y Abraham cayó sobre su rostro, y se alegró, y dijo en su corazón: «¿A un hombre de cien años le nacerá un hijo, y Sara, que tiene noventa años, dará a luz?»
\par 18 Y Abraham dijo a Dios: '¡Ojalá Ismael viviera delante de ti!'
\par 19 Y dijo Dios: «Y Sara también te dará a luz un hijo, y llamarás su nombre Isaac, y estableceré mi pacto con él, pacto eterno, y para su descendencia después de él».
\par 20 «Y en cuanto a Ismael también te he oído, y he aquí lo bendeciré, lo engrandeceré y lo multiplicaré en gran manera, y engendrará doce príncipes, y haré de él una gran nación».
\par 21 «Pero estableceré mi pacto con Isaac, el hijo que Sara te dará a luz en estos días el año que viene».
\par 22 Y dejó de hablar con él, y Dios subió de allí de Abraham.
\par 23 E hizo Abraham tal como Dios le había dicho: tomó a Ismael su hijo y a todos los nacidos en su casa y a los que había comprado con su dinero, a todos los varones de su casa, y circuncidó la carne de su prepucio.
\par 24 Y aquel mismo día fue circuncidado Abraham, y fueron circuncidados con él todos los hombres de su casa, y los nacidos en la casa, y todos los que había comprado con dinero de los hijos del extranjero.
\par 25 Esta ley es para todas las generaciones y para siempre, y no hay circuncisión de los días, ni omisión de un día de los ocho días; porque es ordenanza eterna, ordenada y escrita en las tablas celestiales.
\par 26 Y todo aquel que nace sin circuncidar la carne del prepucio al octavo día, no pertenece a los hijos del pacto que el Señor hizo con Abraham, sino a los hijos de perdición; ni tampoco hay en él señal alguna de que sea del Señor, sino que (está destinado) a ser destruido y muerto de la tierra, y a ser desarraigado de la tierra, porque ha quebrantado el pacto del Señor nuestro. Dios.
\par 27 Porque todos los ángeles de la presencia y todos los ángeles de la santificación fueron creados así desde el día de su creación, y delante de los ángeles de la presencia y de los ángeles de la santificación santificó a Israel, para que estén con él. y con sus santos ángeles.
\par 28 Y ordena a los hijos de Israel que guarden la señal de este pacto para sus generaciones como ordenanza eterna, y no serán desarraigados de la tierra.
\par 29 Porque el mandamiento está establecido como pacto para que lo cumplan para siempre entre todos los hijos de Israel.
\par 30 A Ismael, a sus hijos, a sus hermanos y a Esaú, el Señor no hizo que se le acercaran, y no los eligió porque fueran hijos de Abraham, sino porque los conocía, sino que eligió a Israel como su pueblo.
\par 31 Y Él lo santificó y lo reunió de entre todos los hijos de los hombres; porque hay muchas naciones y muchos pueblos, y todos son suyos, y sobre todo ha puesto espíritus con autoridad para desviarlos de él.
\par 32 Pero sobre Israel no nombró ningún ángel ni espíritu, porque sólo Él es su gobernante, y los preservará y los exigirá de la mano de sus ángeles y de sus espíritus, y de la mano de todos sus poderes para para que Él los preserve y los bendiga, y que ellos sean suyos y Él sea suyo desde ahora en adelante para siempre.
\par 33 Y ahora te anuncio que los hijos de Israel no cumplirán esta ordenanza ni circuncidarán a sus hijos conforme a toda esta ley; porque en la carne de su circuncisión omitirán esta circuncisión de sus hijos, y todos ellos, hijos de Beliar, dejarán a sus hijos incircuncisos como nacieron.
\par 34 Y habrá gran ira de parte del Señor contra los hijos de Israel. porque han abandonado su pacto y se han apartado de su palabra, y se han irritado y blasfemado por no observar la ordenanza de esta ley; porque han tratado a sus miembros como a los gentiles, para ser removidos y desarraigados de la tierra. Y ya no habrá perdón ni perdón para ellos [para que haya perdón y perdón] por todo el pecado de este error eterno.

\chapter{16}

\par \textit{Los ángeles se le aparecen a Abraham en Hebrón e Isaac nuevamente lo prometió, 1-4. Destrucción de Sodoma y liberación de Lot, 5-9. Abraham en Beerseba: nacimiento y circuncisión de Isaac, cuya descendencia sería la porción de Dios, 10-19. Institución de la fiesta de los Tabernáculos, 20-31. (Cf. Gen. xviii.1, 10, 12; xix.24, 29, 33-7; xx.1, 4, 8; xxi. 1-4.)}

\par 1 Y en la luna nueva del cuarto mes nos aparecieron a Abraham en el encinar de Mamre, y hablamos con él, y le anunciamos que Sara su esposa le daría un hijo.
\par 2 Y Sara se rió, porque escuchó que habíamos hablado estas palabras con Abraham, y la amonesté, y ella tuvo miedo y negó que se hubiera reído a causa de esas palabras.
\par 3 Y le dijimos el nombre de su hijo, tal como su nombre está ordenado y escrito en las tablas celestiales, es decir, Isaac,
\par 4 Y (que) cuando volviéramos a ella a la hora señalada, ella habría concebido un hijo.
\par 5 Y en este mes el Señor ejecutó sus juicios sobre Sodoma, Gomorra, Zeboim y toda la región del Jordán, y los quemó con fuego y azufre, y los destruyó hasta el día de hoy, tal como [lo] Te he declarado todas sus obras, que son malvados y pecadores en gran manera, que se contaminan y cometen fornicación en su carne, y hacen inmundicia en la tierra.
\par 6 Y de la misma manera Dios ejecutará juicio en los lugares donde hayan hecho conforme a la inmundicia de los sodomitas, como el juicio de Sodoma.
\par 7 Pero a Lot salvamos; porque Dios se acordó de Abraham y lo envió fuera de en medio de la destrucción.
\par 8 Y él y sus hijas cometieron pecados en la tierra, como no los hubo en la tierra desde los días de Adán hasta su tiempo; porque el hombre se acostó con sus hijas.
\par 9 Y he aquí, estaba ordenado y grabado sobre toda su descendencia, en las tablas celestiales, quitarlas y desarraigarlas, y ejecutar sobre ellas juicio como el juicio de Sodoma, y ​​no dejar descendencia de aquel hombre. en la tierra el día de la condenación.
\par 10 Y en este mes Abraham salió de Hebrón y partió y habitó entre Cades y Shur, en las montañas de Gerar.
\par 11 Y a mediados del quinto mes partió de allí y habitó junto a la fuente del Juramento.
\par 12 Y a mediados del sexto mes el Señor visitó a Sara e hizo con ella tal como le había dicho y ella concibió.
\par 13 Y ella dio a luz un hijo en el tercer mes, y a mediados del mes, en el tiempo que el Señor había dicho a Abraham, en la fiesta de las primicias de la cosecha, nació Isaac.
\par 14 Y Abraham circuncidó a su hijo al octavo día; él fue el primero en circuncidar según el pacto establecido para siempre.
\par 15 Y en el sexto año del cuarto septenario llegamos a Abraham, al Pozo del Juramento, y nos aparecimos a él [como le habíamos dicho a Sara que regresaríamos a ella, y ella habría concebido un hijo.
\par 16 Y regresamos en el mes séptimo y encontramos a Sara encinta delante de nosotros] y lo bendecimos y le anunciamos todas las cosas que se habían decretado acerca de él, que no moriría hasta que engendrara seis hijos. más, y debería verlos antes de morir; pero (que) en Isaac debería llamarse su nombre y descendencia:
\par 17 Y que toda la descendencia de sus hijos fuera gentil y fuera contada entre los gentiles; pero de los hijos de Isaac uno debería llegar a ser una simiente santa, y no debería ser contado entre los gentiles.
\par 18 Porque él se convertiría en porción del Altísimo, y toda su descendencia habría caído en posesión de Dios, para que fuera para el Señor un pueblo en posesión sobre todas las naciones y que se convirtiera en un reino. y sacerdotes y una nación santa.
\par 19 Y nos pusimos en camino y le anunciamos a Sara todo lo que le habíamos dicho, y ambas se regocijaron con gran alegría.
\par 20 Y edificó allí un altar al Señor que lo había liberado y que lo alegraba en la tierra de su peregrinación, y celebró una fiesta de alegría durante siete días en este mes, cerca del altar que había construido. en el Pozo del Juramento.
\par 21 Y en esta fiesta construyó tabernáculos para él y para sus siervos, y fue el primero en celebrar la fiesta de las Tiendas en la tierra.
\par 22 Y durante estos siete días trajo cada día al altar un holocausto al Señor: dos bueyes, dos carneros, siete ovejas y un macho cabrío, como ofrenda por el pecado, para expiar por sí mismo y por los demás. su semilla.
\par 23 Y como ofrenda de acción de gracias, siete carneros, siete cabritos, siete ovejas y siete machos cabríos, con sus ofrendas de frutas y sus libaciones; y quemó toda su grosura sobre el altar, ofrenda escogida al Señor en olor fragante.
\par 24 Y por la mañana y por la tarde quemaba sustancias aromáticas: incienso, gálbano, tabaco, nardo, mirra, especias y especias; Todos estos siete los ofreció machacados, mezclados en partes iguales (y) puros.
\par 25 Y celebró esta fiesta durante siete días, regocijándose con todo su corazón y con toda su alma, él y todos los que estaban en su casa, y no había con él ningún extraño ni incircunciso.
\par 26 Y bendijo a su Creador, que lo había creado en su generación, porque lo había creado según su buena voluntad; porque sabía y percibía que de él surgiría la planta de justicia para las generaciones eternas, y de él una simiente santa, para que llegara a ser semejante a Aquel que había hecho todas las cosas.
\par 27 Y bendijo y se regocijó, y llamó el nombre de esta fiesta Fiesta del Señor, alegría agradable al Dios Altísimo.
\par 28 Y lo bendecimos para siempre, y a toda su descendencia después de él, por todas las generaciones de la tierra, porque celebró esta fiesta en su tiempo, según el testimonio de las tablas celestiales.
\par 29 Por esta razón está establecido en las tablas celestiales acerca de Israel que celebrarán con alegría durante siete días, en el séptimo mes, la fiesta de las Tiendas de Tabernáculos, aceptable ante el Señor, estatuto perpetuo por sus generaciones cada año.
\par 30 Y para esto no hay límite de días; porque está ordenado para siempre para Israel que lo celebren y habiten en cabañas, se pongan coronas en la cabeza y tomen ramas frondosas y sauces del arroyo.
\par 31 Y Abraham tomó ramas de palmeras y frutos de árboles hermosos, y cada día, dando vueltas al altar con las ramas siete veces por la mañana, alababa y daba gracias a su Dios por todas las cosas en alegría.

\chapter{17}

\par \textit{Expulsión de Agar e Ismael, 1-14. Mastêmâ propone que Dios exija a Abraham que sacrifique a Isaac para poner a prueba su amor y su obediencia: Las diez pruebas de Abraham, 15-18. (Cf. Génesis xxi.8-21.)}

\par 1 Y en el primer año de la quinta semana Isaac fue destetado en este jubileo, [1982 AM] y Abraham hizo un gran banquete en el tercer mes, el día en que su hijo Isaac fue destetado.
\par 2 Y Ismael, el hijo de Agar la egipcia, estaba delante de Abraham, su padre, en su lugar, y Abraham se alegró y bendijo a Dios porque había visto a sus hijos y no había muerto sin hijos.
\par 3 Y se acordó de las palabras que le había dicho el día en que Lot se separó de él, y se alegró porque el Señor le había dado simiente en la tierra para heredar la tierra, y bendijo con toda su boca. el Creador de todas las cosas.
\par 4 Y Sara vio a Ismael jugando y bailando, y a Abraham regocijándose con gran alegría, y tuvo celos de Ismael y dijo a Abraham: 'Echa fuera a esta esclava y a su hijo; porque el hijo de esta esclava no será heredero con mi hijo Isaac.'
\par 5 Y a Abraham le pareció grave, a causa de su sierva y de su hijo, que los echara de sí.
\par 6 Y dijo Dios a Abraham: 'No te parezca grave a causa del niño y de la esclava; En todo lo que Sara te ha dicho, escucha sus palabras y hazlas; porque en Isaac será llamado tu nombre y tu descendencia.'
\par 7 «Pero en cuanto al hijo de esta esclava, haré de él una gran nación, porque es de tu descendencia».
\par 8 Abraham se levantó muy de mañana, tomó pan y un odre de agua, los puso sobre los hombros de Agar y de la niña y la despidió.
\par 9 Ella se fue y vagó por el desierto de Beerseba; se acabó el agua del odre y el niño tuvo sed, no pudo seguir adelante y se cayó.
\par 10 Su madre lo tomó y lo arrojó debajo de un olivo, y fue y se sentó frente a él, a una distancia de un tiro de arco; porque dijo: «No me dejes ver la muerte de mi hijo», y mientras estaba sentada, lloró.
\par 11 Y un ángel de Dios, uno de los santos, le dijo: «¿Por qué lloras, Agar?» Levántate, toma al niño y sostenlo en tu mano; porque Dios ha oído tu voz y ha visto al niño.'
\par 12 Y abrió los ojos y vio una fuente de agua, y fue y llenó su odre con agua, y dio de beber a su hijo, y se levantó y se fue hacia el desierto de Parán.
\par 13 Y el niño creció y se hizo arquero, y Dios estaba con él, y su madre tomó para él una esposa de entre las hijas de Egipto.
\par 14 Y ella le dio a luz un hijo, y él llamó su nombre Nebaiot; porque ella dijo: «El Señor estaba cerca de mí cuando lo invoqué».
\par 15 Y aconteció que en el séptimo septenario de su primer año, en el primer mes de este jubileo, el día doce de este mes, se oyeron voces en el cielo acerca de Abraham, que decía que era fiel. en todo lo que le decía, y que amaba al Señor, y que en cada aflicción era fiel.
\par 16 Y vino el príncipe Mastêmâ y dijo delante de Dios: 'He aquí, Abraham ama a Isaac su hijo, y se deleita en él sobre todas las cosas; Dile que lo ofrezca en holocausto sobre el altar, y verás si cumple este mandamiento, y sabrás si es fiel en todo lo que le pruebas.
\par 17 Y el Señor supo que Abraham era fiel en todas sus aflicciones; porque lo había probado en su país y con hambre, y lo había probado con las riquezas de los reyes, y lo había probado nuevamente a través de su esposa, cuando ella fue arrancada (de él), y con la circuncisión; y lo había probado a través de Ismael y Agar, su sierva, cuando los despidió.
\par 18 Y en todo lo que le había probado, fue hallado fiel, y su alma no se impacientó ni tardó en actuar; porque él era fiel y amante del Señor.

\chapter{18}

\par \textit{Sacrificio de Isaac: Mastêmâ avergonzado, 1-13. Abraham nuevamente bendijo: regresa a Beersheba 14-19. (Cf. Génesis XXII. 1-19.)}

\par 1 Y Dios le dijo: 'Abraham, Abraham'; y él dijo: 'Heme aquí'.
\par 2 Y él dijo: «Toma a tu amado hijo, a quien amas, Isaac, y ve a la montaña y ofrécelo sobre uno de los montes que yo te indicaré».
\par 3 Se levantó muy de mañana, ensilló su asno, tomó consigo a sus dos jóvenes y a su hijo Isaac, cortó la leña del holocausto y al tercer día fue al lugar y vio el lugar a lo lejos.
\par 4 Y llegó a un pozo de agua y dijo a sus jóvenes: «Quedaos aquí con el asno, y yo y el muchacho iremos allá, y cuando hayamos adorado volveremos a vosotros».'
\par 5 Y tomó la leña del holocausto y la puso sobre Isaac su hijo, y tomó en su mano el fuego y el cuchillo, y ambos fueron juntos a aquel lugar.
\par 6 E Isaac dijo a su padre: «Padre». y él dijo: 'Aquí estoy, hijo mío'. Y él le dijo: 'He aquí el fuego, el cuchillo y la leña; pero ¿dónde está la oveja para el holocausto, padre?
\par 7 Y él dijo: «Dios se proveerá una oveja para el holocausto, hijo mío». Y se acercó al lugar del monte de Dios.
\par 8 Y edificó un altar, puso la leña sobre el altar, ató a Isaac su hijo, lo puso sobre la leña que estaba sobre el altar y extendió su mano para tomar el cuchillo y matar a Isaac su hijo.
\par 9 Y me paré delante de él, y delante del príncipe Mastêmâ, y el Señor dijo: 'Dile que no ponga su mano sobre el muchacho, ni que le haga nada, porque he demostrado que teme al Señor.'
\par 10 Y lo llamé desde el cielo y le dije: «Abraham, Abraham». y él, aterrorizado, dijo: 'He aquí, (aquí) estoy'.
\par 11 Y yo le dije: 'No pongas tu mano sobre el muchacho, ni le hagas nada; porque ahora he demostrado que temes al Señor, y no me has rehusado tu hijo, tu primogénito.'
\par 12 Y el príncipe Mastêmâ quedó avergonzado; y Abraham alzó sus ojos y miró, y he aquí un carnero prendido. . . por sus cuernos, y Abraham fue, tomó el carnero y lo ofreció en holocausto en lugar de su hijo.
\par 13 Y Abraham llamó a aquel lugar «El Señor ha visto», de modo que se dice \textit{en el monte} el Señor ha visto: es decir, el monte Sión.
\par 14 Y el Señor llamó por segunda vez a Abraham por su nombre desde el cielo, haciéndonos aparecer para hablarle en el nombre del Señor.
\par 15 Y él dijo: Por mí mismo he jurado, dice el Señor,
\par    
\par     Porque has hecho esto,  
\par     ¿Y no me has negado a tu hijo, tu amado hijo,  
\par     que en bendición te bendeciré,
\par    
\par     Y al multiplicarme multiplicaré tu simiente  
\par     Como las estrellas del cielo, y como la arena que está a la orilla del mar.
\par    
\par     Y tu descendencia heredará las ciudades de sus enemigos,  
\par    
\par 16 Y en tu descendencia serán benditas todas las naciones de la tierra;
\par    
\par     Porque has obedecido mi voz,  
\par     Y he mostrado a todos que me eres fiel en todo lo que te he dicho:
\par    
\par     Ve en paz.'
\par    
\par 17 Y Abraham fue donde sus jóvenes, y ellos se levantaron y fueron juntos a Beerseba, y Abraham [2010 AM] habitó junto al Pozo del Juramento.
\par 18 Y celebraba esta fiesta cada año durante siete días con alegría, y la llamaba fiesta del Señor, según los siete días en los que iba y regresaba en paz.
\par 19 Por eso está ordenado y escrito en las tablas celestiales acerca de Israel y su descendencia que celebren esta fiesta durante siete días con la alegría de la fiesta.

\chapter{19}

\par \textit{Regreso de Abraham a Hebrón. Muerte y entierro de Sara, 1-9. Matrimonio de Isaac y segundo matrimonio de Abraham. Nacimiento de Esaú y Jacob, 10-14. Abraham recomienda a Jacob a Rebeca y lo bendice, 15-31. (Cf. Gen. xxiii.1-4, 11-16; xxiv.15; xxv.1-2, 25-7; xiii. 16.)}

\par 1 Y en el primer año del primer septenario del jubileo cuadragésimo segundo, Abraham regresó y habitó frente a Hebrón, es decir, Quiriat Arba, durante dos semanas de años.
\par 2 Y en el año primero de la tercera semana de este jubileo se cumplieron los días de la vida de Sara, y murió en Hebrón.
\par 3 Y Abraham fue a llorar por ella y a sepultarla, y lo probamos [para ver] si su espíritu era paciente y no se indignaba con las palabras de su boca; y se le halló paciente en esto, y no se turbó.
\par 4 Porque con paciencia de espíritu conversó con los hijos de Het, para que le dieran un lugar donde enterrar a su muerta.
\par 5 Y el Señor le dio gracia delante de todos los que lo vieron, y él rogó con dulzura a los hijos de Het, y ellos le dieron la tierra de la doble cueva frente a Mamre, es decir, Hebrón, por cuatrocientas piezas de plata.
\par 6 Y ellos le rogaron, diciendo: Te lo daremos gratis; pero no se lo quitó de las manos por nada, pues les dio el precio del lugar, el dinero completo, y se inclinó ante ellos dos veces, y después de esto enterró a su muerta en la doble cueva.
\par 7 Y fueron todos los días de la vida de Sara ciento veintisiete años, es decir, dos jubileos, cuatro semanas y un año: estos son los días de los años de la vida de Sara.
\par 8 Esta es la décima prueba en que fue probado Abraham, y fue hallado fiel y paciente de espíritu.
\par 9 Y no dijo ni una sola palabra sobre el rumor que había en la tierra de que Dios había dicho que se lo daría a él y a su descendencia después de él, y pidió un lugar allí para enterrar a su muerta; porque fue hallado fiel y quedó registrado en las tablas celestiales como amigo de Dios.
\par 10 Y en el cuarto año tomó mujer para su hijo Isaac, y su nombre era Rebeca [2020 AM] [hija de Betuel, hijo de Nacor, hermano de Abraham], hermana de Labán e hija de Betuel. ; y Betuel era hijo de Melca, la cual era esposa de Nacor, hermano de Abraham.
\par 11 Y Abraham tomó para sí una tercera esposa, que se llamaba Cetura, de entre las hijas de sus sirvientes, porque Agar había muerto antes que Sara. Y ella le dio a luz seis hijos, Zimram, Jocsán, Medan, Madián, Isbac y Súa, en dos semanas de años.
\par 12 Y en el sexto septenario, en su segundo año, Rebeca dio a luz a Isaac dos hijos, Jacob y Esaú,
\par 13 y [2046 AM] Jacob era un hombre lisonjero y recto, y Esaú era fiero, hombre del campo y peludo, y Jacob habitaba en tiendas.
\par 14 Los jóvenes crecieron y Jacob aprendió a escribir; pero Esaú no aprendió, porque era hombre del campo y cazador, y aprendió la guerra, y todas sus hazañas eran feroces.
\par 15 Y Abraham amaba a Jacob, pero Isaac amaba a Esaú.
\par 16 Y Abraham vio los hechos de Esaú y supo que en Jacob se llamaría su nombre y su descendencia; y llamó a Rebeca y dio mandamiento respecto a Jacob, porque sabía que ella (también) amaba a Jacob mucho más que Esaú.
\par 17 Y él le dijo:
\par    
\par     Hija mía, cuida de mi hijo Jacob,  
\par     Porque él ocupará mi lugar en la tierra,  
\par     Y para bendición en medio de los hijos de los hombres,  
\par     Y para la gloria de toda la descendencia de Sem.
\par    
\par 18 Porque sé que el Señor lo escogerá como su pueblo en posesión, entre todos los pueblos que hay sobre la faz de la tierra.
\par 19 Y he aquí, mi hijo Isaac ama a Esaú más que a Jacob, pero yo veo que tú amas verdaderamente a Jacob.
\par    
\par 20 Aumenta aún más tu bondad para con él,  
\par     Y estén sobre él tus ojos con amor;  
\par     Porque él será una bendición para nosotros en la tierra desde ahora en adelante y para todas las generaciones de la tierra.
\par    
\par 21 Sean fuertes tus manos  
\par     Y alégrese tu corazón en tu hijo Jacob;  
\par     Porque lo he amado mucho más que a todos mis hijos.
\par    
\par     Será bendito por siempre,  
\par     Y su descendencia llenará toda la tierra.
\par    
\par 22 Si un hombre puede contar la arena de la tierra,  
\par     Su descendencia también será contada.
\par    
\par 23 Y todas las bendiciones con las que el Señor me ha bendecido a mí y a mi descendencia serán para Jacob y su descendencia para siempre.
\par 24 Y en su descendencia será bendito mi nombre y el nombre de mis padres, Sem, Noab, Enoc, Mahalaleel, Enós, Set y Adán.
\par 25 Y éstos servirán
\par    
\par     Para poner los cimientos del cielo,  
\par     Y para fortalecer la tierra,  
\par     Y renovar todas las lumbreras que están en el firmamento.
\par    
\par 26 Y llamó a Jacob delante de su madre Rebeca, lo besó, lo bendijo y dijo:
\par 27 'Jacob, mi amado hijo, a quien ama mi alma, que Dios te bendiga desde lo alto del firmamento, y te dé todas las bendiciones con las que bendijo a Adán, a Enoc, a Noé y a Sem; y que todas las cosas que me dijo, y todas las cosas que prometió darme, las haga permanecer unidas a ti y a tu descendencia para siempre, conforme a los días del cielo sobre la tierra.'
\par 28 'Y los Espíritus de Mastêmâ no se enseñorearán de ti ni de tu descendencia para apartarte del Señor, que es tu Dios desde ahora en adelante para siempre.'
\par 29 «Y que el Señor Dios sea padre para ti y para tu hijo primogénito, y para el pueblo para siempre».
\par 30 'Vete en paz, hijo mío'. Y ambos salieron juntos de Abraham.
\par 31 Y Rebeca amaba a Jacob con todo su corazón y con toda su alma, mucho más que Esaú; pero Isaac amaba a Esaú mucho más que a Jacob.

\chapter{20}

\par \textit{Abraham amonesta a sus hijos y a los hijos de sus hijos a obrar con justicia, observar la circuncisión y abstenerse de la impureza y la idolatría, 1-10. Los despide con regalos, 11. Moradas de los ismaelitas y de los hijos de Keturah, 12-13. (Cf. Gen. xxv. 5-6.)}

\par 1 Y en el jubileo cuadragésimo segundo, en el primer año del séptimo septenario, Abraham llamó a Ismael [2052 (2045?) AM] y a sus doce hijos, a Isaac, a sus dos hijos y a los seis hijos de Keturah. y sus hijos.
\par 2 Y les ordenó que observaran el camino del Señor; que obren con justicia, amen cada uno a su prójimo y actúen de esta manera entre todos los hombres; que cada uno debería caminar con respecto a ellos como para hacer juicio y justicia en la tierra.
\par 3 Que circuncidaran a sus hijos, según el pacto que Él había hecho con ellos, y no se desviaran ni a derecha ni a izquierda de todos los caminos que el Señor nos había ordenado; y que nos guardemos de toda fornicación e inmundicia, [y renunciemos de entre nosotros a toda fornicación e inmundicia].
\par 4 Y si alguna mujer o sierva entre vosotros comete fornicación, quemadla en el fuego y no forniquen con ella según sus ojos y su corazón; y no tomen para sí mujeres de las hijas de Canaán; porque la simiente de Canaán será desarraigada de la tierra.
\par 5 Y les contó el juicio de los gigantes y el juicio de los sodomitas, cómo habían sido juzgados por su maldad y habían muerto a causa de su fornicación, su impureza y su mutua corrupción por la fornicación.
\par    
\par 6 'Y guardaos de toda fornicación e inmundicia,  
\par     Y de toda contaminación del pecado,
\par    
\par     Para que no hagáis de nuestro nombre una maldición,  
\par     Y toda tu vida un silbido,
\par    
\par     y todos tus hijos serán destruidos a espada,  
\par     Y seréis anatemas como Sodoma,  
\par     Y todo vuestro remanente como hijos de Gomorra.
\par    
\par 7 Os imploro, hijos míos, que améis al Dios del cielo  
\par     Y guardad todos sus mandamientos.
\par    
\par     Y no andéis tras sus ídolos ni tras sus inmundicias,  
\par    
\par 8 Y no os hagáis dioses de fundición ni de escultura;
\par    
\par     porque son vanidad,  
\par     Y no hay espíritu en ellos;
\par    
\par     Porque son obra de manos (de hombres),  
\par     Y todo el que confía en ellos, no confía en nada.
\par    
\par 9 No les sirváis ni los adoréis,  
\par     Pero servid al Dios Altísimo y adorarlo continuamente.  
\par     Y esperar siempre en su rostro,  
\par     Y obrad rectitud y justicia delante de él,
\par    
\par     Para que se complazca en ti y te conceda su misericordia,  
\par     y que haga llover sobre vosotros mañana y tarde,
\par    
\par     Y bendecid todas vuestras obras que habéis hecho en la tierra,  
\par     Y bendecir tu pan y tu agua,
\par    
\par     Y bendice el fruto de tu vientre y el fruto de tu tierra,  
\par     Y las manadas de tus vacas y los rebaños de tus ovejas.
\par    
\par 10 Y seréis una bendición para la tierra,  
\par     Y todas las naciones de la tierra te desearán,
\par    
\par     Y bendice a tus hijos en mi nombre,  
\par     Para que sean bendecidos como yo.'
\par    
\par 11 Y dio regalos a Ismael, a sus hijos y a los hijos de Keturah, y los despidió de Isaac su hijo, y se lo dio todo a Isaac su hijo.
\par 12 Ismael y sus hijos, los hijos de Cetura y sus hijos fueron juntos y habitaron desde Parán hasta la entrada de Babilonia, en toda la tierra que está hacia el oriente, frente al desierto.
\par 13 Y éstos se mezclaron unos con otros, y llamaron su nombre árabes e ismaelitas.

\chapter{21}

\par \textit{Últimas palabras de Abraham a Isaac sobre la idolatría, el comer sangre, la ofrenda de diversos sacrificios y el uso de sal, 1-11. También en cuanto a las maderas que se utilizarán en el sacrificio y el deber de lavarse antes del sacrificio y de cubrir la sangre, etc., 12-25.}

\par 1 Y en el año sexto de la séptima semana de este jubileo, Abraham llamó a Isaac su hijo, y [2057 (2050?) AM] le mandó, diciendo: «Estoy viejo y no sé el día de mi muerte, y estoy lleno de mis días.'
\par 2 «Y he aquí, tengo ciento setenta y cinco años, y durante todos los días de mi vida me he acordado del Señor y he procurado con todo mi corazón hacer su voluntad y andar con rectitud en todas sus cosas. maneras.'
\par 3 «Mi alma aborreció los ídolos (y desprecié a los que los servían, y entregué mi corazón y mi espíritu) para procurar hacer la voluntad de Aquel que me creó».
\par 4 'Porque Él es el Dios viviente, y Él es santo y fiel, y Él es más justo que todos, y en Él no se aceptan personas (de los hombres) ni se aceptan regalos; porque Dios es justo y ejecuta juicio sobre todos los que traspasan sus mandamientos y desprecian su pacto.'
\par 5 «Y tú, hijo mío, observa sus mandamientos, sus ordenanzas y sus juicios, y no andes tras las abominaciones, ni tras las imágenes talladas ni tras las imágenes fundidas».
\par 6 «Y no comáis sangre alguna de animales, ni de ganado, ni de ninguna ave que vuele en el cielo».
\par 7 «Y si sacrificas una víctima como ofrenda de paz aceptable, mátala y derramarás su sangre sobre el altar, y toda la grasa de la ofrenda sobre el altar con flor de harina y la ofrenda mezclada con el aceite con su libación, ofrécelos todos juntos sobre el altar del holocausto; es un olor grato delante del Señor.'
\par 8 «Y ofrecerás el sebo del sacrificio de agradecimiento sobre el fuego que está sobre el altar, el sebo que está sobre el vientre, y todo el sebo de las entrañas y los dos riñones, y todo el sebo que está sobre el vientre. que está sobre ellos, y sobre los lomos y el hígado quitarás, junto con los riñones.'
\par 9 «Y ofreced todo esto en olor grato, agradable delante del Señor, con su presente y su libación, en olor grato, el pan de la ofrenda al Señor».
\par 10 'Y comerás su carne ese día y el segundo día, y no se pondrá el sol sobre él el segundo día hasta que se coma, y ​​no dejará nada que sobre para el tercer día; porque no es acepto [porque no está aprobado] y no se comerá más, y todo el que de él coma traerá pecado sobre sí; porque así lo encontré escrito en los libros de mis antepasados, y en las palabras de Enoc y en las palabras de Noé.'
\par 11 «Y echarás sal sobre todas tus ofrendas, y no faltará la sal del pacto en todas tus ofrendas delante del Señor».
\par 12 «Y en cuanto a la madera de los sacrificios, ten cuidado de no traer para el altar otras maderas además de éstas: ciprés, laurel, almendro, abeto, pino, cedro, sabina, higo, olivo, mirra, laurel. , aspalato.'
\par 13 'Y de estas clases de madera, ponlas sobre el altar debajo del sacrificio, tal como hayan sido probadas en cuanto a su apariencia, y no pongas (sobre ellas) ninguna madera partida u oscura, (sino) dura y limpia, sin defecto. , un crecimiento sano y nuevo; y no pongas (sobre ella) madera vieja, [porque su fragancia ha desaparecido] porque ya no hay en ella fragancia como antes.'
\par 14 «Además de estas clases de madera, no hay ninguna otra que coloques (sobre el altar), porque la fragancia se dispersa, y el olor de su fragancia no sube al cielo».
\par 15 «Observa este mandamiento y hazlo, hijo mío, para que seas recto en todas tus obras».
\par 16 'Y en todo momento sé limpio en tu cuerpo, y lávate con agua antes de acercarte a ofrecer sobre el altar, y lava tus manos y tus pies antes de acercarte al altar; y cuando termines de sacrificar, lávate de nuevo las manos y los pies.'
\par 17 'Y no aparezca sangre sobre ti ni sobre tu ropa; Guárdate, hijo mío, de la sangre, guárdate mucho; cúbrelo con polvo.'
\par 18 'Y no comáis sangre, porque es el alma; No comas sangre en absoluto.
\par 19 'Y no toméis regalos por la sangre del hombre, no sea que sea derramada impunemente, sin juicio; porque es la sangre derramada la que hace pecar a la tierra, y la tierra no puede ser limpiada de la sangre del hombre sino por la sangre del que la derramó.'
\par 20 'Y no aceptes presente ni dádiva por la sangre de hombre: sangre por sangre, para que seas acepto delante del Señor, Dios Altísimo; porque él es el defensor de los buenos, y para que tú seas preservado de todo mal, y él te salve de toda especie de muerte.'
\par    
\par 21 Ya veo, hijo mío,  
\par     Que todas las obras de los hijos de los hombres son pecado y maldad,  
\par     Y todas sus obras son inmundicia, abominación y contaminación,  
\par     Y no hay justicia en ellos.
\par    
\par 22 Cuídate, no vayas a seguir sus caminos  
\par     Y seguir sus caminos,  
\par     Y pecar pecado de muerte ante el Dios Altísimo.
\par    
\par     De lo contrario Él [esconderá Su rostro de ti  
\par     Y] devolverte en manos de tu transgresión,  
\par     Y desarraigarte de la tierra, y tu descendencia también de debajo del cielo,  
\par     Y tu nombre y tu descendencia perecerán de toda la tierra.
\par    
\par 23 Apártate de todas sus obras y de toda su inmundicia,  
\par     Y guardad la ordenanza del Dios Altísimo,  
\par     Y haz su voluntad y sé recto en todo.
\par    
\par 24 Y Él te bendecirá en todas tus obras,  
\par     Y levantará de ti planta de justicia por toda la tierra, por todas las generaciones de la tierra,  
\par     Y mi nombre y el tuyo no serán olvidados bajo el cielo para siempre.
\par    
\par 25 Ve, hijo mío, en paz.  
\par     Que el Dios Altísimo, Dios mío y Dios tuyo, te fortalezca para hacer su voluntad,  
\par     Y bendiga toda tu descendencia y el remanente de tu descendencia por las generaciones y para siempre, con todas las bendiciones justas,  
\par     Para que seas una bendición en toda la tierra.
\par    
\par 26 Y salió de allí gozoso.

\chapter{22}

\par \textit{Isaac, Ismael y Jacob celebran la fiesta de las primicias en Beerseba con Abraham, 1-5. Oración de Abraham, 6-9. Últimas palabras de Abraham y bendiciones de Jacob, 10-30.}

\par 1 Y aconteció en la primera semana del jubileo cuadragésimo cuarto, en el año segundo, es decir, el año en que murió Abraham, que Isaac e Ismael vinieron del pozo del Juramento para celebrar la fiesta de semanas -es decir, la fiesta de las primicias de la cosecha- a Abraham, su padre, y Abraham se alegró porque sus dos hijos habían venido.
\par 2 Porque Isaac tenía muchas posesiones en Beerseba, e Isaac solía ir a ver sus posesiones y regresar con su padre.
\par 3 En aquellos días, Ismael fue a ver a su padre y se reunieron ambos, e Isaac ofreció un sacrificio en holocausto y lo presentó sobre el altar de su padre que él había hecho en Hebrón.
\par 4 Y ofreció una ofrenda de agradecimiento e hizo un banquete de alegría delante de Ismael, su hermano; y Rebeca hizo nuevas tortas con el grano nuevo y se las dio a Jacob, su hijo, para que se las llevara a Abraham, su padre, de las primicias de la tierra, para comer y bendecir al Creador de todas las cosas antes de morir.
\par 5 También Isaac envió a Abraham, por mano de Jacob, la mejor ofrenda de agradecimiento para que comiera y bebiera.
\par 6 Y comió y bebió, y bendijo al Dios Altísimo,
\par    
\par     Quien creó el cielo y la tierra,  
\par     ¿Quién hizo todas las cosas ricas de la tierra?  
\par     Y se los dio a los hijos de los hombres  
\par     Para que pudieran comer y beber y bendecir a su Creador.
\par    
\par 7 «Y ahora te doy gracias, Dios mío, porque me has hecho ver este día. He aquí, soy ciento setenta y quince años, soy anciano y lleno de días, y todos mis días han pasado. ha sido para mí paz.'
\par 8 «La espada del adversario no me ha vencido en todo lo que me has dado a mí y a mis hijos, todos los días de mi vida hasta el día de hoy».
\par 9 'Dios mío, que tu misericordia y tu paz sean sobre tu siervo y sobre la descendencia de sus hijos, para que sean para ti una nación escogida y una herencia entre todas las naciones de la tierra, desde ahora en adelante para todos. los días de las generaciones de la tierra, por todos los siglos.'
\par 10 Y llamó a Jacob y le dijo: «Hijo mío, Jacob, que el Dios de todos te bendiga y te fortalezca para hacer justicia y su voluntad delante de Él, y que te escoja a ti y a tu descendencia para que llegues a ser un pueblo para Su herencia según Su voluntad para siempre.'
\par 11 «Y tú, hijo mío Jacob, acércate y bésame». Y él se acercó y lo besó, y dijo:
\par    
\par     'Bendito sea mi hijo Jacob  
\par     Y todos los hijos del Dios Altísimo, por todos los siglos:
\par    
\par     Que Dios te dé una semilla de justicia;  
\par     Y santifique a algunos de tus hijos en medio de toda la tierra;
\par    
\par     Que las naciones te sirvan,  
\par     Y todas las naciones se postran ante tu descendencia.
\par    
\par 12 Sé fuerte delante de los hombres,  
\par     Y ejercer autoridad sobre toda la descendencia de Set.
\par    
\par     Entonces tus caminos y los caminos de tus hijos serán justificados,  
\par     Para que lleguen a ser una nación santa.
\par    
\par 13 Que el Dios Altísimo te dé todas las bendiciones  
\par     con que me ha bendecido
\par    
\par     Y con el cual bendijo a Noé y a Adán;  
\par     Que descansen sobre la sagrada cabeza de tu descendencia de generación en generación para siempre.
\par    
\par 14 Y que Él te limpie de toda injusticia e impureza,  
\par     Para que te sean perdonadas todas las transgresiones; que has cometido por ignorancia.
\par    
\par     Y que Él te fortalezca,  
\par     Y te bendiga.  
\par     Y heredarás toda la tierra,
\par    
\par 15 Y que Él renueve su pacto contigo.  
\par     para que seas para él una nación por herencia para todos los siglos,  
\par     Y para que Él sea para ti y para tu descendencia un Dios en verdad y justicia durante todos los días de la tierra.
\par    
\par 16 Y tú, hijo mío Jacob, recuerda mis palabras,  
\par     Y observa los mandamientos de Abraham, tu padre:
\par    
\par     Apártate de las naciones,  
\par     Y no comas con ellos:
\par    
\par     Y no según sus obras,  
\par     Y no os hagáis cómplices de ellos;
\par    
\par     Porque sus obras son inmundas,  
\par     Y todos sus caminos son contaminación, abominación e inmundicia.
\par    
\par 17 Ofrecen sus sacrificios a los muertos  
\par     Y adoran a los espíritus malignos,
\par    
\par     Y comen sobre las tumbas,  
\par     Y todas sus obras son vanidad y nada.
\par    
\par 18 No tienen corazón para entender  
\par     Y sus ojos no ven cuáles son sus obras,
\par    
\par     Y cómo se equivocan al decir a un trozo de madera: «Tú eres mi Dios».  
\par     Y a una piedra: 'Tú eres mi Señor y eres mi libertador'.  
\par     [Y no tienen corazón.]
\par    
\par 19 Y tú, hijo mío Jacob,  
\par     Que el Dios Altísimo te ayude  
\par     Y el Dios del cielo te bendiga  
\par     Y líbrate de su inmundicia y de todo su error.
\par    
\par 20 Ten cuidado, hijo mío Jacob, de tomar esposa de alguna de las hijas de Canaán;
\par    
\par     Porque toda su simiente será desarraigada de la tierra.
\par    
\par 21 Porque por la transgresión de Cam, Canaán se extravió,  
\par     Y toda su descendencia será destruida de sobre la tierra y todo su resto,  
\par     Y ninguno que surja de él será salvo en el día del juicio.
\par    
\par 22 Y todos los adoradores de ídolos y los profanos  
\par     (b) No habrá esperanza para ellos en la tierra de los vivientes;  
\par     (c) Y no habrá memoria de ellos en la tierra;  
\par     (c) Porque descenderán al Seol,  
\par     (d) Y al lugar de condenación irán,
\par    
\par     Cuando los hijos de Sodoma fueron quitados de la tierra  
\par     Así serán quitados todos los que adoran ídolos.
\par    
\par 23 No temas, hijo mío Jacob,  
\par     Y no desmayes, oh hijo de Abraham:
\par    
\par     Que el Dios Altísimo te guarde de la destrucción,  
\par     Y de todos los caminos del error que él te libre.
\par    
\par 24 Esta casa me he construido para poner mi nombre sobre ella en la tierra, y se llamará casa de Abraham; te es dado a ti y a tu descendencia para siempre; porque tú edificarás mi casa y afirmarás mi nombre delante de Dios para siempre; tu descendencia y tu nombre permanecerán por todas las generaciones de la tierra.'
\par    
\par 25 Y dejó de mandarle y de bendecirle.
\par 26 Y los dos se acostaron juntos en una cama, y ​​Jacob durmió en el seno de Abraham, el padre de su padre, y él lo besó siete veces, y su afecto y su corazón se regocijaron sobre él.
\par 27 Y lo bendijo con todo su corazón y dijo: «El Dios Altísimo, Dios de todo y Creador de todo, que me sacó de Ur de los caldeos para darme esta tierra y heredarla para siempre. para siempre, y para establecer una simiente santa: bendito sea el Altísimo por los siglos.'
\par 28 Y bendijo a Jacob y dijo: «Hijo mío, por quien me regocijo con todo mi corazón y mi afecto, que tu gracia y tu misericordia sean alzadas sobre él y sobre su descendencia para siempre».
\par 29 «Y no lo abandones ni lo desprecies desde ahora hasta los días de la eternidad, y que tus ojos se abran sobre él y sobre su descendencia, para que lo preserves, lo bendigas y lo santifiques. como nación para tu herencia;'
\par 30 «Y bendícelo con todas tus bendiciones desde ahora en adelante hasta todos los días de la eternidad, y renueva tu pacto y tu gracia con él y con su descendencia según todo tu beneplácito por todas las generaciones de la tierra».

\chapter{23}

\par \textit{Muerte y entierro de Abraham, 1-8 (cf. Gén. xxv.7-10). Años menguantes y corrupción creciente de la humanidad: ayes mesiánicos: lucha universal: los fieles se levantan en armas para traer de vuelta a los infieles: Israel invadido por pecadores de los gentiles, 11-25. Estudio renovado de la ley y renovación de la humanidad: Reino mesiánico: bendita inmortalidad de los justos, 26-31.}

\par 1 Y puso dos dedos de Jacob sobre sus ojos, y bendijo al Dios de los dioses, y se cubrió el rostro, estiró los pies y durmió el sueño de la eternidad, y se reunió con sus padres.
\par 2 Y a pesar de todo esto, Jacob yacía en su seno y no sabía que Abraham, el padre de su padre, había muerto.
\par 3 Y Jacob despertó de su sueño, y he aquí que Abraham estaba frío como el hielo, y dijo: «Padre, padre»; pero nadie habló, y supo que estaba muerto.
\par 4 Entonces él se levantó de su seno y corrió a contárselo a Rebeca, su madre; Y Rebeca fue a Isaac esa noche y le contó; y fueron juntos, y Jacob con ellos, y una lámpara estaba en su mano, y cuando entraron, encontraron a Abraham tendido muerto.
\par 5 Isaac se postró sobre su padre, lloró y lo besó.
\par 6 Y se oyeron voces en la casa de Abraham, e Ismael su hijo se levantó y fue a Abraham su padre, y lloró por Abraham su padre, él y toda la casa de Abraham, y lloraron con gran llanto.
\par 7 Y sus hijos Isaac e Ismael lo sepultaron en la doble cueva, cerca de Sara su esposa, y lloraron por él cuarenta días, todos los hombres de su casa, Isaac e Ismael, y todos sus hijos, y todos los hijos. de Cetura en sus lugares; y terminaron los días de llanto por Abraham.
\par 8 Y vivió tres jubileos y cuatro semanas de años, ciento setenta y cinco años, y cumplió los días de su vida, siendo viejo y lleno de días.
\par 9 Porque los días de la vida de los antepasados ​​fueron diecinueve jubileos; y después del Diluvio comenzaron a tener menos de diecinueve jubileos, y a disminuir en jubileos, y a envejecer rápidamente, y a estar llenos de sus días a causa de las múltiples tribulaciones y de la maldad de sus caminos, con excepción de Abraham.
\par 10 Porque Abraham fue perfecto en todas sus obras para con el Señor, y complacido en justicia todos los días de su vida; y he aquí, no cumplió cuatro jubileos en su vida, cuando envejeció a causa de la maldad, y estaba lleno de sus días.
\par 11 Y todas las generaciones que surgirán desde este tiempo hasta el día del gran juicio envejecerán rápidamente, antes de completar dos jubileos, y su conocimiento los abandonará a causa de su vejez. desaparecer].
\par 12 Y en aquellos días, si un hombre vive un jubileo de años y medio, dirán de él: «Ha vivido mucho tiempo, y la mayor parte de sus días son dolor, tristeza y tribulación, y hay no paz:'
\par 13 «Porque calamidad tras calamidad, y herida sobre herida, y tribulación sobre tribulación, y mala noticia sobre mala noticia, y enfermedad sobre enfermedad, y todos los juicios malos como estos, uno con otro, enfermedad y destrucción, y nieve. y escarcha y hielo, y fiebre, y escalofríos, y letargo, y hambre, y muerte, y espada, y cautiverio, y toda clase de calamidades y dolores.'
\par 14 Y todo esto vendrá sobre una generación mala, que transgrede la tierra: sus obras son inmundicia, fornicación, inmundicia y abominaciones.
\par 15 Entonces dirán: 'Los días de los antepasados ​​fueron muchos, hasta mil años, y buenos; pero he aquí, los días de nuestra vida, si un hombre ha vivido muchos, son sesenta años y diez, y, si es fuerte, sesenta años, y esos malos, y no hay paz en los días de esta generación mala.'
\par 16 Y en esa generación los hijos convencerán a sus padres y a sus mayores de pecado e injusticia, de las palabras de sus bocas y de las grandes maldades que cometen, y de su abandono del pacto que el Señor hizo entre ellos y Él. , que deben observar y hacer todos sus mandamientos y sus ordenanzas y todas sus leyes, sin apartarse ni a la derecha ni a la izquierda.
\par 17 Porque todos han hecho lo malo, y toda boca habla iniquidad, y todas sus obras son inmundicia y abominación, y todos sus caminos son contaminación, inmundicia y destrucción.
\par 18 He aquí, la tierra será destruida a causa de todas sus obras, y no habrá semilla de vid ni aceite; porque sus obras son totalmente infieles, y todos perecerán a una, bestias, ganado y aves, y todos los peces del mar, a causa de los hijos de los hombres.
\par 19 Y se pelearán unos con otros, el joven con el viejo, el viejo con el joven, el pobre con el rico, el humilde con el grande, el mendigo con el príncipe, a causa de la ley y del pacto; porque han olvidado el mandamiento, el pacto, las fiestas, los meses, los sábados, los jubileos y todos los juicios.
\par 20 Y se levantarán con arcos y espadas y harán guerra para volverlos al camino; pero no volverán hasta que haya sido derramada mucha sangre sobre la tierra, uno por otro.
\par 21 Y los que hayan escapado no volverán de su maldad al camino de la justicia, sino que todos se exaltarán con el engaño y las riquezas, para tomar cada uno de lo que es de su prójimo y nombrar el gran nombre. pero no en verdad ni en justicia, y contaminarán el lugar santísimo con su inmundicia y la corrupción de su contaminación.
\par 22 Y un gran castigo caerá sobre las obras de esta generación por parte del Señor, y Él los entregará a la espada, al juicio y al cautiverio, y serán saqueados y devorados.
\par 23 Y Él despertará contra ellos a los pecadores de los gentiles, que no tienen misericordia ni compasión, y que no respetarán a nadie, ni al viejo ni al joven, ni a nadie, porque son más malvados y más fuertes para hacer. malos que todos los hijos de los hombres.
\par    
\par     Y usarán violencia contra Israel y transgresión contra Jacob,  
\par     Y mucha sangre será derramada sobre la tierra,  
\par     Y no habrá quien recoger ni quien enterrar.
\par    
\par 24 En aquellos días gritarán a gritos,  
\par     Y llamad y orad para que sean salvos de la mano de los pecadores, los gentiles;  
\par     Pero ninguno será salvo.
\par    
\par 25 Y las cabezas de los niños serán blancas y con canas,  
\par     Y el niño de tres semanas parecerá viejo como el hombre de cien años,  
\par     Y su estatura será destruida por la tribulación y la opresión.
\par    
\par 26 Y en aquellos días los niños comenzarán a estudiar las leyes,  
\par     Y para buscar los mandamientos,  
\par     Y volver al camino de la justicia.
\par    
\par 27 Y los días comenzarán a multiplicarse y multiplicarse entre aquellos hijos de los hombres.  
\par     Hasta que sus días se acerquen a los mil años.  
\par     Y a mayor número de años que (antes) era el número de los días.
\par    
\par 28 Y no habrá ningún anciano  
\par     Ni el que (no) está satisfecho con sus días,  
\par     Porque todos serán (como) niños y jóvenes.
\par    
\par 29 Y completarán todos sus días y vivirán en paz y alegría,  
\par     Y no habrá Satanás ni ningún mal destructor;  
\par     Porque todos sus días serán días de bendición y curación.
\par    
\par 30 En aquel tiempo el Señor sanará a sus siervos,  
\par     Y se levantarán y verán gran paz,  
\par     Y expulsar a sus adversarios.
\par    
\par     Y los justos lo verán y estarán agradecidos,  
\par     Y regocíjense con alegría por los siglos de los siglos,  
\par     Y verán todos sus juicios y todas sus maldiciones sobre sus enemigos.
\par    
\par 31 Y sus huesos reposarán en la tierra,  
\par     Y su espíritu tendrá mucha alegría,  
\par     Y sabrán que es el Señor quien ejecuta el juicio,  
\par     Y muestra misericordia a cientos y miles y a todos los que lo aman.
\par    
\par 32 Y tú, Moisés, escribe estas palabras; porque así están escritos y registrados en las tablas celestiales, para testimonio por las generaciones y para siempre.

\chapter{24}

\par \textit{Isaac en el Pozo de la Visión, 1 (cf. Gén. xxv. 11). Esaú vende su primogenitura, 2-7 (cf. Gén. xxv.29-34).}

\par 1 Y aconteció que después de la muerte de Abraham, el Señor bendijo a Isaac su hijo, y él se levantó de Hebrón y fue y habitó en el Pozo de la Visión en el primer año de la tercera semana [2073 AM] de este jubileo, siete años.
\par 2 Y en el primer año de la cuarta semana comenzó una hambruna en la tierra, además de la primera hambruna que hubo en los días de Abraham.
\par 3 Jacob preparó lentejas y Esaú volvió hambriento del campo. Y dijo a Jacob su hermano: 'Dame de este guisado rojo'. Y Jacob le dijo: 'Véndeme tu [primogenitura, esta] primogenitura y te daré pan, y también un poco de este guiso de lentejas.'
\par 4 Y Esaú dijo en su corazón: 'Moriré; ¿De qué me sirve esta primogenitura?
\par 5 Y dijo a Jacob: «Te lo doy». Y Jacob dijo: 'Júrame hoy', y él le juró.
\par 6 Y Jacob le dio a Esaú su hermano pan y guisados, y él comió hasta saciarse, y Esaú menospreció su primogenitura; por esta razón el nombre de Esaú fue llamado Edom, a causa del guiso rojo que Jacob le dio como primogenitura.
\par 7 Y Jacob envejeció, y Esaú perdió su dignidad.
\par 8 Y el hambre se apoderó de la tierra, e Isaac partió para descender a Egipto en el segundo año de este septenario, y se dirigió al rey de los filisteos en Gerar, en Abimelec.
\par 9 Y el Señor se le apareció y le dijo: 'No desciendas a Egipto; Habita en la tierra que yo te diré, y mora en esta tierra, y estaré contigo y te bendeciré.'
\par 10 «Porque a ti y a tu descendencia daré toda esta tierra, y confirmaré el juramento que hice a tu padre Abraham, y multiplicaré tu descendencia como las estrellas del cielo, y daré a tu descendencia toda esta tierra.'
\par 11 'Y en tu descendencia serán benditas todas las naciones de la tierra, porque tu padre obedeció mi voz y guardó mi precepto, mis mandamientos, mis leyes, mis ordenanzas y mi pacto; y ahora obedece mi voz y habita en esta tierra.'
\par 12 Y habitó en Gelar tres semanas de años.
\par 13 Y Abimelec acusó contra él [2080-2101 AM] y contra todo lo que era suyo, diciendo: 'Cualquier hombre que lo toque o algo que es suyo, ciertamente morirá'.
\par 14 Isaac se hizo fuerte entre los filisteos y adquirió muchas posesiones: bueyes, ovejas, camellos, asnos y una gran familia.
\par 15 Y sembró en tierra de los filisteos y produjo cien veces más. Isaac se hizo enormemente grande y los filisteos le tenían envidia.
\par 16 Todos los pozos que los siervos de Abraham habían cavado durante la vida de Abraham, los filisteos los habían cerrado después de la muerte de Abraham y los habían llenado de tierra.
\par 17 Y Abimelec dijo a Isaac: «Aléjate de nosotros, porque tú eres mucho más poderoso que nosotros». Isaac partió de allí en el primer año del séptimo septenario y habitó como peregrino en los valles de Gerar.
\par 18 Y volvieron a cavar los pozos de agua que los siervos de Abraham, su padre, habían cavado y que los filisteos habían cerrado después de la muerte de Abraham su padre, y él los llamó por el nombre que Abraham su padre los había llamado.
\par 19 Y los siervos de Isaac cavaron un pozo en el valle y encontraron agua viva, y los pastores de Gerar pelearon con los pastores de Isaac, diciendo: «El agua es nuestra»; e Isaac llamó el nombre del pozo 'Perversidad', porque habían sido perversos con nosotros.
\par 20 Y cavaron un segundo pozo, y lucharon por sacarlo también, y llamó su nombre Enemistad. Y él se levantó de allí y cavaron otro pozo, y por eso no discutieron, y llamó su nombre 'Habitación', e Isaac dijo: 'Ahora el Señor nos ha hecho lugar, y hemos aumentado en la tierra.'
\par 21 Y de allí subió a la Fuente del Juramento en el primer año de la primera semana del jubileo cuadragésimo cuarto.
\par 22 Y el Señor se le apareció aquella noche, en la luna nueva del primer mes, y le dijo: 'Yo soy el Dios de Abraham tu padre; No temas, porque yo estoy contigo y te bendeciré, y ciertamente multiplicaré tu descendencia como la arena de la tierra, por amor de Abraham mi siervo.'
\par 23 Y edificó allí el altar que Abraham su padre había construido primero, e invocó el nombre del Señor y ofreció sacrificios al Dios de Abraham su padre.
\par 24 Y cavaron un pozo y encontraron agua viva.
\par 25 Y los siervos de Isaac cavaron otro pozo y no encontraron agua, y fueron y le dijeron a Isaac que no habían encontrado agua, e Isaac dijo: «Hoy he jurado a los filisteos y esto ha sido anunciado a a nosotros.'
\par 26 Y llamó el nombre de aquel lugar Pozo del Juramento; porque allí había jurado a Abimelec y a Ahuzzat su amigo y a Ficol el prefecto o su anfitrión.
\par 27 Y ese día Isaac supo que, bajo obligación, les había jurado hacer las paces con ellos.
\par 28 Aquel día Isaac maldijo a los filisteos y dijo: 'Malditos los filisteos hasta el día de la ira y de la indignación de entre todas las naciones; que Dios los convierta en escarnio y maldición y en objeto de ira e indignación en manos de los pecadores los gentiles y en manos de los Kittim.
\par 29 Y a quien escape de la espada del enemigo y de los Kittim, que la nación justa lo desarraigue en juicio de debajo del cielo; porque ellos serán enemigos y adversarios de mis hijos a lo largo de sus generaciones sobre la tierra.
\par    
\par 30 Y no les quedará ningún resto,  
\par     Ni aquel que será salvo en el día de la ira del juicio;  
\par     Porque destrucción, desarraigo y expulsión de la tierra es toda la descendencia de los filisteos (reservada),  
\par     Y no quedará más para estos caftoreos nombre ni descendencia sobre la tierra.
\par    
\par 31 Aunque suba al cielo,  
\par     De allí será derribado,
\par    
\par     Y aunque se haga fuerte en la tierra,  
\par     De allí será arrastrado,
\par    
\par     Y aunque se esconda entre las naciones,  
\par     Incluso de allí será desarraigado;
\par    
\par     Y aunque descienda al Seol,  
\par     Allí también será grande su condenación,  
\par     Y allí tampoco tendrá paz.  
\par    
\par 32 Y si va en cautiverio,  
\par     Por manos de los que buscan su vida lo matarán en el camino,  
\par     Y no le quedará nombre ni descendencia en toda la tierra;  
\par     Porque partirá hacia la maldición eterna.'
\par    
\par 33 Y así está escrito y grabado acerca de él en las tablas celestiales, lo que se hará con él en el día del juicio, para que sea desarraigado de la tierra.

\chapter{25}

\par \textit{Rebeca amonestó a Jacob que no se casara con una mujer cananea, 1-3. Jacob promete casarse con una hija de Labán a pesar de las urgentes peticiones de Esaú de que se casara con una mujer cananea, 4-10. Rebeca bendice a Jacob, 11-23. (Cf. Gen. xxviii.1-4.)}

\par 1 Y en el segundo año de esta semana en este jubileo, Rebeca llamó a Jacob su hijo, y le habló [2109 AM], diciendo: 'Hijo mío, no tomes para ti una esposa de las hijas de Canaán, como Esaú. , tu hermano, que tomó para sí dos mujeres de las hijas de Canaán, y han amargado mi alma con todas sus obras inmundas; porque todas sus obras son fornicación y concupiscencia, y no hay justicia en ellas, porque (sus obras) son demonio.'
\par 2 Y yo, hijo mío, te amo muchísimo, y mi corazón y mi cariño te bendicen en cada hora del día y en las vigilias de la noche.
\par 3 Ahora pues, hijo mío, escucha mi voz y haz la voluntad de tu madre, y no tomes esposa de las hijas de esta tierra, sino sólo de la casa de mi padre y de los parientes de mi padre. Tomarás para ti mujer de la casa de mi padre, y el Dios Altísimo te bendecirá, y tus hijos serán generación justa y descendencia santa.'
\par 4 Entonces Jacob habló con Rebeca, su madre, y le dijo: «Mira, madre, tengo nueve semanas de años y no conozco ni he tocado a ninguna mujer, ni me he desposado con ninguna. ni se te ocurra tomarme una esposa de las hijas de Canaán.'
\par 5 «Porque recuerdo, madre, las palabras de nuestro padre Abraham, quien me ordenó que no tomara esposa de las hijas de Canaán, sino que tomara una esposa de la descendencia de la casa de mi padre y de mi parentela.'
\par 6 «Ya he oído antes que a Labán, tu hermano, le han nacido hijas, y he puesto mi corazón en ellas para tomar de entre ellas una esposa».
\par 7 'Y por esta razón me he guardado en mi espíritu de no pecar ni corromperme en todos mis caminos durante todos los días de mi vida; porque en cuanto a la concupiscencia y la fornicación, Abraham, mi padre, me dio muchos mandamientos.'
\par 8 'Y a pesar de todo lo que me ha ordenado, estos veintidós años mi hermano ha estado luchando conmigo, y me hablaba frecuentemente y decía: 'Hermano mío, toma por esposa a una hermana de mis dos esposas'; pero me niego a hacer lo que él ha hecho.'
\par 9 «Te juro, madre, que en todos los días de mi vida no tomaré mujer de las hijas de la descendencia de Canaán, ni actuaré mal como lo hizo mi hermano».
\par 10 'No temas, madre; Ten la seguridad de que haré tu voluntad y caminaré en rectitud y no corromperé mis caminos para siempre.'
\par 11 Entonces ella alzó su rostro al cielo, extendió los dedos de sus manos, abrió la boca y bendijo al Dios Altísimo, que había creado los cielos y la tierra, y le dio gracias y alabanza.
\par 12 Y ella dijo: 'Bendito sea el Señor Dios, y bendito sea su santo nombre por los siglos de los siglos, que me dio a Jacob por hijo puro y por simiente santa; porque tuyo es, y tuya será su descendencia continuamente y por todas las generaciones, para siempre.'
\par 13 'Bendícelo, Señor, y pon en mi boca la bendición de la justicia, para que pueda bendecirlo.
\par 14 Y en aquella hora, cuando el espíritu de justicia descendió a su boca, puso ambas manos sobre la cabeza de Jacob y dijo:
\par    
\par 15 Bendito eres tú, Señor de justicia y Dios de los siglos.  
\par     Y que Él te bendiga más allá de todas las generaciones de los hombres.
\par    
\par     Que Él te dé, Hijo mío, el camino de la justicia,  
\par     Y revela justicia a tu descendencia.
\par    
\par 16 Y que Él haga muchos a tus hijos durante tu vida,  
\par     Y que surjan según el número de los meses del año.  
\par     Y que sus hijos lleguen a ser muchos y grandes más allá de las estrellas del cielo,  
\par     Y su número será mayor que la arena del mar.
\par    
\par 17 Y que Él les dé esta hermosa tierra, como dijo que se la daría a Abraham y a su descendencia después de él para siempre.  
\par     Y que lo tengan como posesión para siempre.
\par    
\par 18 Y que pueda ver (nacer) a ti, hijo mío, hijos benditos durante mi vida,  
\par     Y sea toda tu descendencia una simiente bendita y santa.
\par    
\par 19 Y como tú has refrescado el espíritu de tu madre durante su vida,  
\par     El vientre de la que te dio a luz te bendice así,
\par    
\par     [Mi cariño] y mis pechos te bendigan  
\par     Y mi boca y mi lengua te alaban mucho.
\par    
\par 20 Crece y se extiende sobre la tierra,  
\par     Y que tu descendencia sea perfecta en el gozo del cielo y de la tierra para siempre;
\par    
\par     Y que tu descendencia se regocije,  
\par     Y que en el gran día de la paz tenga paz.
\par    
\par 21 Y que tu nombre y tu descendencia permanezcan para siempre,  
\par     Y que el Dios Altísimo sea su Dios,
\par    
\par     Y que el Dios de justicia more con ellos,  
\par     Y por ellos sea edificado su santuario para todos los siglos.
\par    
\par 22 Bendito el que te bendice,  
\par     Y toda carne que te maldiga falsamente, sea maldita.'
\par    
\par 23 Y ella lo besó y le dijo:  
\par     'Que el Señor del mundo te ame'  
\par     'Como el corazón de tu madre y su cariño se regocijan en ti y te bendicen'.
\par     Y ella dejó de bendecir.

\chapter{26}

\textit{Isaac corre a Esaú en busca de venado, 1-4. Rebeca instruye a Jacob para que obtenga la bendición, 5-9. Jacob bajo la persona de Esaú lo obtiene, 10-24. Esaú trae su venado y por su importunidad obtiene una bendición, 25-34. Amenaza a Jacob, 35. (Cf. Gen.xxvii.)}

\par 1 Y en el séptimo año de esta semana, Isaac llamó a Esaú, su hijo mayor, y le dijo: 'Soy viejo, hijo mío, y he aquí, mis ojos están nublados para ver, y no conozco el el día de mi muerte.'
\par 2 «Y ahora toma tus armas de caza, tu aljaba y tu arco, y sal al campo, caza y cógeme (venado), hijo mío, y hazme un guisado como a mi alma le gusta, y tráelo. a mí para que coma, y ​​para que mi alma te bendiga antes de morir.'
\par 3 Pero Rebeca escuchó a Isaac hablar con Esaú.
\par 4 Y Esaú salió temprano al campo para cazar y pescar y llevárselos a su padre.
\par 5 Y Rebeca llamó a Jacob, su hijo, y le dijo: 'He aquí, oí a Isaac, tu padre, hablar con Esaú, tu hermano, diciéndole: «Cázame, y prepárame un guisado, y tráemelo». ) para mí eso'
\par 6 «Puedo comer y bendecirte delante del Señor antes de morir.» Y ahora, hijo mío, obedece mi voz en lo que te mando: Ve a tu rebaño y tráeme dos buenos cabritos de las cabras, y yo Hará con ellos un guisado para tu padre, como a él le gusta, y se lo llevarás a tu padre para que coma y te bendiga delante de Jehová antes de que muera, y para que seas bendecido.'
\par 7 Y Jacob dijo a Rebeca su madre: 'Madre, no le negaré nada de lo que mi padre comería y que le agradaría; sólo temo, madre mía, que reconozca mi voz y quiera tocarme. '
\par 8 «Y tú sabes que yo soy liso, y Esaú, mi hermano, es peludo, y apareceré ante sus ojos como un malhechor, y haré una acción que él no me había ordenado, y él se enojará con yo, y traeré sobre mí maldición y no bendición.'
\par 9 Y Rebeca, su madre, le dijo: «Sobre mí tu maldición, hijo mío, sólo obedece mi voz».
\par 10 Y Jacob obedeció la voz de Rebeca, su madre, y fue a buscar dos cabritos de cabras buenos y gordos, y se los llevó a su madre, y su madre les preparó una carne sabrosa como a él le gustaba.
\par 11 Y Rebeca tomó la ropa hermosa de Esaú, su hijo mayor, que estaba con ella en la casa, y vistió a Jacob, su hijo menor, (con ellos), y puso las pieles de los cabritos sobre sus manos y en las partes expuestas de su cuello.
\par 12 Y ella entregó la carne y el pan que había preparado en manos de su hijo Jacob.
\par 13 Y Jacob se acercó a su padre y le dijo: «Soy tu hijo; he hecho lo que me ordenaste; levántate, siéntate y come de lo que he pescado, padre, para que tu alma me bendiga».
\par 14 Isaac dijo a su hijo: «¿Cómo has encontrado tan pronto, hijo mío?»
\par 15 Y Jacob dijo: «Porque (el Señor) tu Dios me hizo encontrar».
\par 16 E Isaac le dijo: «Acércate y te palparé, hijo mío, si eres mi hijo Esaú o no».
\par 17 Y Jacob se acercó a Isaac, su padre, lo palpó y dijo: «La voz es la voz de Jacob, pero las manos son las manos de Esaú».
\par 18 Pero él no lo distinguió, porque era una dispensa del cielo para quitarle el poder de percepción, e Isaac no lo distinguió, porque sus manos eran peludas como las de su hermano Esaú, de modo que lo bendijo.
\par 19 Y él dijo: «¿Eres tú mi hijo Esaú?» ' y él dijo: 'Soy tu hijo': y él dijo: 'Acércate a mí para que coma de lo que has pescado, hijo mío, para que mi alma te bendiga'.
\par 20 Y él se acercó a él, y él comió, le trajo vino y bebió.
\par 21 Entonces Isaac, su padre, le dijo: «Acércate y bésame, hijo mío».
\par 22 Y él se acercó y lo besó. Y olió el olor de su vestido, y lo bendijo y dijo: 'He aquí, el olor de mi hijo es como el olor de un campo (lleno) que el Señor ha bendecido.
\par    
\par 23 Y que el Señor te dé del rocío del cielo  
\par     Y del rocío de la tierra, y de abundancia de trigo y de aceite:
\par    
\par     Deja que las naciones te sirvan,  
\par     Y los pueblos se inclinan ante ti.
\par    
\par 24 Sé señor de tus hermanos,  
\par     Y se postren ante ti los hijos de tu madre;
\par    
\par     Y que todas las bendiciones con que el Señor me ha bendecido a mí y a Abraham mi padre;  
\par     Sea impartido a ti y a tu descendencia para siempre:
\par    
\par     Maldito el que te maldiga,  
\par     Y bendito sea el que te bendiga.'
\par    
\par 25 Y aconteció que tan pronto como Isaac terminó de bendecir a su hijo Jacob, y Jacob se hubo alejado de Isaac su padre, se escondió y Esaú, su hermano, volvió de su caza.
\par 26 También preparó un guisado, se lo llevó a su padre y le dijo: «Que mi padre se levante y coma de mi caza para que tu alma me bendiga».
\par 27 Entonces Isaac, su padre, le dijo: «¿Quién eres tú?» 'Y él le dijo: 'Yo soy tu primogénito, tu hijo Esaú; he hecho como me has mandado.'
\par 28 Isaac quedó muy asombrado y dijo: «¿Quién es el que cazó, lo capturó y me lo trajo, y yo comí de todo antes de que tú vinieras, y lo bendijo? (y) él será sea ​​bendita, y toda su descendencia por los siglos.'
\par 29 Y aconteció que cuando Esaú oyó las palabras de su padre Isaac, lanzó un grito muy grande y amargo, y dijo a su padre: «Bendíceme a mí también, padre».
\par 30 Y él le dijo: «Tu hermano vino con engaño y te ha quitado tu bendición». Y él dijo: 'Ahora sé por qué se llama su nombre Jacob: he aquí, él me ha suplantado estas dos veces: me quitó mi primogenitura, y ahora me ha quitado mi bendición.'
\par 31 Y él dijo: «¿No me has reservado una bendición, padre?» e Isaac respondió y dijo a Esaú:
\par    
\par     "He aquí, yo lo he puesto por señor tuyo,  
\par     Y a todos sus hermanos le he dado por siervos,  
\par     Y con mucho maíz, vino y aceite lo fortalecí:
\par    
\par     ¿Y ahora qué haré por ti, hijo mío?
\par    
\par 32 Y Esaú dijo a Isaac su padre:  
\par     '¿Tienes sólo una bendición, oh padre?  
\par     Bendíceme, (incluso) a mí también, padre: '
\par    
\par 33 Y Esaú alzó su voz y lloró. Y Isaac respondió y le dijo:
\par    
\par     'He aquí, lejos del rocío de la tierra estará tu morada,  
\par     Y lejos del rocío del cielo desde arriba.
\par    
\par 34 Y por tu espada vivirás,  
\par     Y servirás a tu hermano.
\par    
\par     Y sucederá que cuando seas grande,  
\par     y sacudirás su yugo de tu cuello,  
\par     Pecarás un pecado completo de muerte,  
\par     Y tu descendencia será desarraigada de debajo del cielo.
\par    
\par 35 Y Esaú siguió amenazando a Jacob a causa de la bendición con la que su padre lo bendijo, y dijo en su corazón: «Que vengan ahora los días de duelo por mi padre, para que pueda matar a mi hermano Jacob».

\chapter{27}

\par \textit{Rebeca, alarmada por las amenazas de Esaú, convence a Isaac para que envíe a Jacob a Mesopotamia, 1-12. Isaac consuela a Rebeca por la partida de Jacob, 13-18. El sueño y el voto de Jacob en Betel, 19-27. (Cf. Gen. xxviii.)}

\par 1 Y fueron dichas a Rebeca en sueños las palabras de Esaú, su hijo mayor, y Rebeca envió y llamó a Jacob, su hijo menor,
\par 2 y le dijo: «Mira, Esaú, tu hermano, se vengará de ti y te matará».
\par 3 «Ahora pues, hijo mío, obedece mi voz, levántate y huye a Labán, mi hermano, a Harán, y quédate con él algunos días hasta que la ira de tu hermano se calme y él quite su ira contra ti. y olvida todo lo que has hecho; entonces te enviaré y te buscaré desde allí.'
\par 4 Y Jacob dijo: 'No tengo miedo; Si quiere matarme, lo mataré.'
\par 5 Pero ella le dijo: «No permitas que me quede sin mis dos hijos en un mismo día».
\par 6 Y Jacob dijo a Rebeca su madre: 'Mira, tú sabes que mi padre ha envejecido y no ve porque tiene los ojos apagados, y si lo dejo, será malo ante sus ojos, porque lo dejo. y se alejará de vosotros, y mi padre se enojará y me maldecirá. No voy a ir; Sólo cuando él me envíe, entonces iré.'
\par 7 Y Rebeca dijo a Jacob: «Entraré y hablaré con él, y él te despedirá».
\par 8 Entonces Rebeca entró y dijo a Isaac: 'Aborrezco mi vida a causa de las dos hijas de Het, que Esaú le ha tomado por esposas; y si Jacob toma mujer como estas de entre las hijas de la tierra, ¿para qué seguiré viviendo, porque las hijas de Canaán son malas?
\par 9 Isaac llamó a Jacob, lo bendijo, le amonestó y le dijo: 'No tomes mujer de ninguna de las hijas de Canaán;
\par 10 Levántate y ve a Mesopotamia, a casa de Betuel, padre de tu madre, y toma de allí mujer de las hijas de Labán, hermano de tu madre.
\par 11 «Y Dios Todopoderoso te bendiga, te aumente y te multiplique, para que llegues a ser una multitud de naciones, y te dé las bendiciones de mi padre Abraham, a ti y a tu descendencia después de ti, para que heredes la tierra de tu estancias y toda la tierra que Dios dio a Abraham: ve, hijo mío, en paz.'
\par 12 Isaac despidió a Jacob y éste se fue a Mesopotamia, a ver a Labán, hijo de Betuel el arameo, hermano de Rebeca, la madre de Jacob.
\par 13 Y aconteció que después que Jacob se levantó para ir a Mesopotamia, el espíritu de Rebeca se entristeció por su hijo, y ella lloró.
\par 14 Isaac dijo a Rebeca: «Hermana mía, no llores por mi hijo Jacob; porque en paz se va, y en paz volverá.'
\par 15 'El Dios Altísimo lo preservará de todo mal y estará con él; porque no lo desamparará en todos sus días;'
\par 16 «Porque sé que sus caminos serán prosperados en todo dondequiera que vaya, hasta que regrese en paz a nosotros y lo veamos en paz».
\par 17 'No temas por él, hermana mía, porque va por el camino recto y es un hombre perfecto; es fiel y no perecerá. No llores.'
\par 18 Isaac consoló a Rebeca a causa de su hijo Jacob y lo bendijo.
\par 19 Y Jacob salió del Pozo del Juramento para ir a Harán el primer año del segundo septenario del jubileo cuadragésimo cuarto, y llegó a Luz, en la montaña, es decir, a Betel, en la luna nueva del el primer mes de esta semana, [2115 AM] y llegó al lugar al atardecer y se desvió del camino hacia el oeste del camino esa noche: y durmió allí; porque el sol se había puesto.
\par 20 Y tomó una de las piedras de aquel lugar y la puso debajo del árbol, y mientras viajaba solo, se durmió.
\par 21 Aquella noche soñó que había una escalera apoyada en la tierra y que su extremo llegaba al cielo; y he aquí que los ángeles del Señor subían y descendían por ella; y he aquí, el Señor estaba sobre ella.
\par 22 Y habló a Jacob y dijo: 'Yo soy el Señor Dios de Abraham, tu padre, y el Dios de Isaac; La tierra donde duermes, te la daré a ti, y a tu descendencia después de ti.'
\par 23 «Y tu descendencia será como el polvo de la tierra, y crecerás hacia el oeste y hacia el este, hacia el norte y el sur, y en ti y en tu descendencia serán todas las familias de las naciones. bendecido.'
\par 24 «Y he aquí, yo estaré contigo y te guardaré dondequiera que vayas, y te traeré de nuevo a esta tierra en paz; porque no te dejaré hasta que haya hecho todo lo que te dije.'
\par 25 Y Jacob se despertó de su sueño y dijo: «Verdaderamente este lugar es la casa del Señor, y yo no lo sabía». Y tuvo miedo y dijo: 'Espantoso es este lugar que no es otro que la casa de Dios, y ésta es la puerta del cielo'.
\par 26 Y Jacob se levantó muy de mañana, tomó la piedra que había puesto debajo de su cabeza, la erigió como señal como señal y derramó aceite sobre ella. Y llamó el nombre de aquel lugar Betel; pero el nombre del lugar era Luz al principio.
\par 27 Y Jacob hizo un voto al Señor, diciendo: «Si el Señor está conmigo y me guarda en este camino en el que voy, y me da pan para comer y vestido para vestirme, para volver de nuevo a la casa de mi padre en paz, entonces el Señor será mi Dios, y esta piedra que he puesto como columna de señal en este lugar, será la casa del Señor, y de todo lo que tú me des, lo haré. Te doy el décimo, Dios mío.

\chapter{28}

\par \textit{Jacob se casa con Lea y Raquel, 1-10. Sus hijos de Lea y Raquel y de sus siervas, 11-24. Jacob busca dejar a Labán, 25: pero se queda con un salario determinado, 26-8. Jacob se hace rico, 29-30. (Cf. Gen. xxix.1, 17, 18, 21-35; xxx.1-13,17-22, 24, 25, 28, 32, 39, 43; xxxi.1, 2.)}

\par 1 Y siguió su viaje y llegó a la tierra del oriente, a Labán, hermano de Rebeca, y estuvo con él, y le sirvió para Raquel su hija una semana.
\par 2 Y en el primer año de la tercera semana [2122 AM] le dijo: 'Dame mi esposa, por quien te he servido siete años'; y Labán dijo a Jacob: 'Te daré tu esposa'.
\par 3 Y Labán hizo un banquete, y tomó a Lea su hija mayor, y la dio a Jacob por esposa, y le dio a su sierva Zilpa por sierva; y Jacob no lo sabía, porque pensaba que ella era Raquel.
\par 4 Y él entró a ella, y he aquí, ella era Lea; y Jacob se enojó con Labán, y le dijo: '¿Por qué me has hecho así? ¿No te serví por Raquel y no por Lea? ¿Por qué me has hecho mal?
\par 5 'Toma a tu hija y yo iré; porque me has hecho mal.' Porque Jacob amaba a Raquel más que a Lea; porque los ojos de Leah eran débiles, pero su forma era muy hermosa; pero Raquel tenía unos ojos hermosos y una figura hermosa y muy hermosa.
\par 6 Y Labán dijo a Jacob: «No se hace así en nuestro país, dar el menor antes que el mayor». Y no está bien hacer esto; porque así está ordenado y escrito en las tablas celestiales, que nadie dé a su hija menor antes que a la mayor; pero la mayor, se da primero y después la menor; y al hombre que así lo hace, le echan culpa en el cielo, y no es justo el que hace esto, porque este hecho es malo ante el Señor.
\par 7 Y ordena a los hijos de Israel que no hagan esto; No tomen ni entreguen al menor antes de haber dado al mayor, porque es muy malo.
\par 8 Y Labán dijo a Jacob: «Pasarán los siete días de la fiesta de éste, y te daré a Raquel, para que me sirvas otros siete años, y puedas apacentar mis ovejas como lo hiciste en el la semana anterior.'
\par 9 Y el día que habían pasado los siete días de la fiesta de Lea, Labán le dio a Raquel a Jacob para que le sirviera otros siete años, y a Raquel le dio a Bilha, hermana de Zilpa, como sierva.
\par 10 Y sirvió otros siete años por Raquel, porque Lea le había sido entregada a cambio de nada.
\par 11 Y el Señor abrió el vientre de Lea, y ella concibió y dio a luz a Jacob un hijo, y él llamó su nombre Rubén, el día catorce del noveno mes, en el primer año del tercer septenario. [2122 a.m.]
\par 12 Pero el vientre de Raquel se cerró, porque el Señor vio que Lea era odiada y Raquel amada.
\par 13 Y Jacob volvió a encontrarse con Lea, y ella concibió y le dio a Jacob un segundo hijo, y llamó su nombre Simeón, el veintiuno del mes décimo, en el año tercero de este septenario. [2124 a.m.]
\par 14 Y Jacob volvió a encontrarse con Lea, y ella concibió y le dio a luz un tercer hijo, y llamó su nombre Leví, en la luna nueva del primer mes del sexto año de este septenario. [2127 a.m.]
\par 15 Y Jacob volvió a unirse a ella, y ella concibió y le dio a luz un cuarto hijo, y llamó su nombre Judá, el día quince del tercer mes, en el primer año del cuarto septenario. [2129 a.m.]
\par 16 A causa de todo esto, Raquel tenía envidia de Lea, porque no daba a luz, y dijo a Jacob: «Dame hijos». y Jacob dijo: '¿Te he negado los frutos de tu vientre? ¿Te he abandonado?
\par 17 Y cuando Raquel vio que Lea le había dado a Jacob cuatro hijos, Rubén, Simeón, Leví y Judá, le dijo: «Entra a Bilha mi sierva, y ella concebirá y me dará un hijo». (Y ella le dio a Bilha su sierva por esposa).
\par 18 Y él entró a ella, y ella concibió, y le dio a luz un hijo, y él llamó su nombre Dan, el día nueve del mes sexto, en el año sexto del tercer septenario. [2127 a.m.]
\par 19 Y Jacob volvió a entrar a Bilha por segunda vez, y ella concibió y dio a luz a Jacob otro hijo, y Raquel llamó su nombre Naftalí, el día cinco del séptimo mes, en el segundo año del cuarto septenario. [21:30 a. m.]
\par 20 Y cuando Lea vio que se había vuelto estéril y no podía dar a luz, tuvo envidia de Raquel, y también dio a su sierva Zilpa a Jacob por esposa, y ella concibió y dio a luz un hijo, y Lea llamó su nombre Gad, en el día doce del mes octavo, en el año tercero de la cuarta semana. [2131 a.m.]
\par 21 Y él volvió a ella, y ella concibió y le dio a luz un segundo hijo, y Lea le puso por nombre Aser, el día dos del mes undécimo, en el año quinto del cuarto septenario. [2133 a.m.]
\par 22 Y Jacob llegó a Lea, y ella concibió y dio a luz un hijo, y llamó su nombre Isacar, el cuarto del quinto mes, en el cuarto año del cuarto septenario, y le dio a una enfermera.
\par 23 Y Jacob volvió a ella, y ella concibió y dio a luz dos hijos, un hijo y una hija, y llamó el nombre de su hijo Zabulón, y el nombre de su hija Dina, el séptimo de el séptimo mes, en el sexto año de la cuarta semana. [2134 a.m.]
\par 24 Y el Señor tuvo misericordia de Raquel, y abrió su vientre, y ella concibió y dio a luz un hijo, y llamó su nombre José, en la luna nueva del cuarto mes, en el sexto año de este cuarto septenario. [2134 a.m.]
\par 25 Y en los días en que nació José, Jacob dijo a Labán: 'Dame mis mujeres y mis hijos, y déjame ir a mi padre Isaac, y déjame hacerme una casa; porque he cumplido los años que te serví por tus dos hijas, y volveré a la casa de mi padre.'
\par 26 Y Labán dijo a Jacob: «Quédate conmigo por tu salario, y vuelve a pastorear mis rebaños para mí, y recibe tu salario».
\par 27 Y acordaron entre sí que le daría como salario los corderos y los cabritos que nacieran negros, manchados y blancos, (estos) serían su salario.
\par 28 Y todas las ovejas produjeron corderos manchados, moteados y negros, con diversas marcas, y volvieron a producir corderos como ellos; y todos los que estaban manchados eran de Jacob y los que no, de Labán.
\par 29 Y los bienes de Jacob se multiplicaron enormemente: tuvo bueyes, ovejas, asnos, camellos, siervos y siervas.
\par 30 Y Labán y sus hijos tenían envidia de Jacob, y Labán le quitó sus ovejas y lo observó con malas intenciones.

\chapter{29}

\par \textit{Jacob, sale en secreto, 1-4. Labán lo persigue, 5-6. Pacto de Jacob y Labán, 7-8. Moradas de los amorreos (antiguamente de los refaítas) destruidas en la época del escritor, 9-11. Labán se marcha, 12. Jacob se reconcilia con Esaú, 13. Jacob envía provisiones de alimentos a sus padres cuatro veces al año a Hebrón, 14-17, 19-20. Esaú se vuelve a casar, 18. (Cf. Gen. xxxi.3, 4, 10, 13, 19, 21, 23, 24, 46, 47; xxxii.22; xxxiii.10, 16.)}

\par 1 Y aconteció que cuando Raquel dio a luz a José, Labán fue a esquilar sus ovejas; porque estaban lejos de él un camino de tres días.
\par 2 Y vio Jacob que Labán iba a esquilar sus ovejas, y llamó a Lea y a Raquel, y les dijo amablemente que vinieran con él a la tierra de Canaán.
\par 3 Porque él les contó cómo había visto todo en sueños, incluso todo lo que le había dicho que regresara a la casa de su padre, y ellos dijeron: 'A cualquier lugar a donde vayas, iremos contigo. '
\par 4 Y Jacob bendijo al Dios de Isaac su padre, y al Dios de Abraham, el padre de su padre; y se levantó, montó sobre sus mujeres y sus hijos, tomó todas sus posesiones, cruzó el río y llegó a la tierra de Galaad. , y Jacob ocultó su intención a Labán y no le dijo nada.
\par 5 Y en el séptimo año de la cuarta semana, Jacob se volvió hacia Galaad en el mes primero, el día veintiuno del mismo. [2135 AM] Y Labán lo persiguió y alcanzó a Jacob en el monte de Galaad en el mes tercero, el día trece del mismo.
\par 6 Y el Señor no le permitió herir a Jacob; porque se le apareció en sueños de noche. Y Labán habló a Jacob.
\par 7 Y el quince de aquellos días, Jacob hizo un banquete para Labán y para todos los que venían con él, y Jacob juró a Labán ese día, y Labán también a Jacob, que ninguno cruzaría la montaña de Galaad al otro lado. con mal propósito.
\par 8 E hizo allí un montón para que sirviera de testimonio; por lo que el nombre de aquel lugar se llama: 'El Montón del Testimonio', en honor a este montón.
\par 9 Pero antes llamaban a la tierra de Galaad tierra de los refaítas; porque era tierra de los refaítas, y nacieron (allí) refaítas, gigantes cuya altura era de diez, nueve, ocho hasta siete codos.
\par 10 Y su habitación se extendió desde la tierra de los hijos de Amón hasta el monte Hermón, y las sedes de su reino fueron Karnaim, Astarot, Edrei, Misur y Beón.
\par 11 Y el Señor los destruyó a causa de la maldad de sus obras; porque eran muy malignos, y los amorreos habitaron en su lugar, malvados y pecadores, y no hay pueblo hoy que haya cometido todos sus pecados al máximo, y ya no tienen más duración de vida sobre la tierra.
\par 12 Entonces Jacob despidió a Labán, que partió hacia Mesopotamia, la tierra del Oriente, y Jacob regresó a la tierra de Galaad.
\par 13 Y pasó el Jaboc en el mes noveno, el día undécimo del mismo. Y aquel día, Esaú, su hermano, vino a él, y se reconcilió con él, y se fue de él a la tierra de Seir, pero Jacob habitaba en tiendas.
\par 14 Y en el primer año del quinto septenario de este jubileo [2136 AM] cruzó el Jordán y habitó al otro lado del Jordán, y apacentó sus ovejas desde el mar del Montón hasta Betsán, y hasta Dotán y hasta el bosque de Akrabbim.
\par 15 Y envió a su padre Isaac todos sus bienes: ropa, comida, carne, bebida, leche, mantequilla, queso y algunos dátiles del valle.
\par 16 Y también a su madre Rebeca cuatro veces al año, entre los meses, entre el arado y la siega, entre el otoño y la lluvia, y entre el invierno y la primavera, a la torre de Abraham.
\par 17 Porque Isaac había regresado del pozo del juramento y había subido a la torre de su padre Abraham, y habitó allí separado de su hijo Esaú.
\par 18 Porque en los días en que Jacob fue a Mesopotamia, Esaú tomó como esposa a Mahalat, hija de Ismael, y juntó todos los rebaños de su padre y de sus mujeres, y subió y habitó en el monte Seír, y dejó solo a Isaac, su padre, en el Pozo del Juramento.
\par 19 Isaac subió del pozo del Juramento y habitó en la torre de su padre Abraham, en las montañas de Hebrón.
\par 20 Y allí envió Jacob todo lo que enviaba a su padre y a su madre de vez en cuando, todo lo que necesitaban, y bendijeron a Jacob con todo su corazón y con toda su alma.

\chapter{30}

\par \textit{Dina violada, 1-3. Matanza de los siquemitas, 4-6. Leyes contra los matrimonios mixtos entre Israel y los paganos, 7-17. Leví elegido para el sacerdocio a causa de la matanza de los siquemitas, 18-23. Dina se recuperó, 24. La reprensión de Jacob, 25-6. (Cf. Gen. xxxiii.18, xxxiv.2, 4, 7, 13-14, 25-30, xxxv.5.)}

\par 1 Y en el primer año de la sexta semana [2143 AM] subió en paz a Salem, al este de Siquem, en el cuarto mes.
\par 2 Y allí se llevaron a Dina, hija de Jacob, a casa de Siquem, hijo de Hamor, heveo, príncipe de la tierra, y él se acostó con ella y la contaminó, y ella era un poco niña, una niña de doce años.
\par 3 Y él rogó a su padre y a sus hermanos que se la dieran por esposa. Y Jacob y sus hijos se enojaron a causa de los hombres de Siquem; porque habían contaminado a Dina su hermana, y les hablaron con malas intenciones, los trataron con engaño y los engañaron.
\par 4 Y Simeón y Leví llegaron inesperadamente a Siquem y ejecutaron juicio sobre todos los hombres de Siquem, y mataron a todos los hombres que encontraron en ella, y no dejaron ni uno solo en ella; los mataron a todos en tormentos porque habían deshonraron a su hermana Dina.
\par 5 Y así no se vuelva a hacer de ahora en adelante que una hija de Israel sea contaminada; porque en el cielo está ordenado juicio contra ellos, que destruirán a espada a todos los hombres de los siquemitas, porque habían avergonzado a Israel.
\par 6 Y el Señor los entregó en manos de los hijos de Jacob para que los exterminaran con la espada y ejecutaran juicio sobre ellos, y para que no volviera a suceder en Israel que una virgen de Israel fuera contaminada.
\par 7 Y si hay algún hombre en Israel que quiera dar su hija o su hermana a algún hombre que sea de la descendencia de los gentiles, ciertamente morirá, y lo apedrearán; porque ha avergonzado a Israel; y quemarán a la mujer al fuego, porque ha deshonrado el nombre de la casa de su padre, y será desarraigada de Israel.
\par 8 Y que no se encuentre adúltera ni impureza en Israel durante todos los días de las generaciones de la tierra; porque Israel es santo para el Señor, y todo hombre que lo haya contaminado, ciertamente morirá: lo apedrearán.
\par 9 Porque así está ordenado y escrito en las tablas celestiales acerca de toda la descendencia de Israel: Quien la contamine ciertamente morirá y será apedreado.
\par 10 Y esta ley no tiene límite de días, ni remisión ni expiación; pero el hombre que ha profanado a su hija será desarraigado de entre todo Israel, porque ha dado de su descendencia a Moloc. , y obró impíamente para contaminarlo.
\par 11 Y tú, Moisés, ordena a los hijos de Israel y exhortales a no dar sus hijas a los gentiles, ni a tomar para sus hijos a ninguna de las hijas de los gentiles, porque esto es abominable ante el Señor.
\par 12 Por esta razón te he escrito en las palabras de la Ley todos los hechos que los siquemitas hicieron contra Dina, y cómo hablaron los hijos de Jacob, diciendo: «No daremos nuestra hija a ningún hombre». quien es incircunciso; porque eso fue un reproche para nosotros.'
\par 13 Y es un oprobio para Israel, para los que viven y para los que toman a las hijas de los gentiles; porque esto es inmundo y abominable para Israel.
\par 14 Y Israel no quedará libre de esta inmundicia si tiene una esposa de las hijas de los gentiles, o si entrega alguna de sus hijas a un hombre que sea de alguna de las naciones.
\par 15 Porque habrá plaga tras plaga, y maldición sobre maldición, y todo juicio, plaga y maldición caerá \textit{sobre él} si hace esto, o esconde sus ojos de los que cometen inmundicia, o de aquellos que los que profanan el santuario del Señor, o los que profanan su santo nombre, (entonces) toda la nación junta será juzgada por toda la inmundicia y profanación de este hombre.
\par 16 Y no habrá acepción de personas, ni recibirán de sus manos frutos, ofrendas, holocaustos, grasas ni olores de olor grato para aceptarlos. Que le pase a todo hombre o mujer en Israel que profane el santuario.
\par 17 Por eso te he mandado decir: «Da este testimonio a Israel: mira cómo les fue a los siquemitas y a sus hijos: cómo fueron entregados en manos de dos hijos de Jacob, y los mataron con tormentos, y les fue contado por justicia, y por justicia les está escrito.'
\par 18 Y la descendencia de Leví fue escogida para el sacerdocio y para ser levitas, para que pudieran ministrar delante del Señor, como nosotros, continuamente, y para que Leví y sus hijos fueran benditos para siempre; porque era celoso de ejecutar justicia, juicio y venganza sobre todos los que se levantaron contra Israel.
\par 19 Y así, como testimonio de su favor, escriben en las tablas celestiales bendición y justicia ante el Dios de todos:
\par 20 Y nos acordamos de la justicia que el hombre cumplió durante su vida, en todas las épocas del año; hasta mil generaciones lo registrarán, y vendrá a él y a su descendencia después de él, y ha sido registrado en las tablas celestiales como amigo y hombre justo.
\par 21 Todo este relato te lo he escrito y te he ordenado que digas a los hijos de Israel que no cometan pecados, ni transgredan las ordenanzas, ni rompan el pacto que se les ha establecido, sino que deben cumplirlo y quedar registrados como amigos.
\par 22 Pero si transgreden y cometen impurezas en cualquier forma, quedarán registrados en las tablas celestiales como adversarios, serán eliminados del libro de la vida y quedarán registrados en el libro de los que serán destruidos y con los que serán desarraigados de la tierra.
\par 23 Y el día que los hijos de Jacob mataron a Siquem, se escribió a su favor en el cielo que habían hecho justicia, rectitud y venganza sobre los pecadores, y fue escrito para bendición.
\par 24 Y sacaron a Dina, su hermana, de la casa de Siquem, y tomaron cautivo todo lo que había en Siquem, sus ovejas, sus bueyes, sus asnos, y todas sus riquezas, y todos sus rebaños, y se los llevaron. todo a Jacob su padre.
\par 25 Y les reprochó que habían pasado a espada la ciudad, porque temía a los habitantes de la tierra, a los cananeos y a los ferezeos.
\par 26 Y el temor del Señor se apoderó de todas las ciudades que están alrededor de Siquem, y no se levantaron para perseguir a los hijos de Jacob; porque el terror había caído sobre ellos.

\chapter{31}

\par \textit{Jacob va a Betel a ofrecer sacrificio, 1-3 (cf. Gén. xxxv.2-4, 7, 14). Isaac bendice a Leví, 4-17, y a Judá, 18-22. Jacob le cuenta a Isaac cómo Dios lo prosperó, 24. Jacob va a Betel con Rebeca y Débora, 26-30. Jacob bendice al Dios de sus padres, 31-2.}

\par 1 Y en la luna nueva del mes, Jacob habló a toda la gente de su casa. diciendo: 'Purifícate y cámbiate de ropa, y levantémonos y subamos a Betel, donde le hice un voto el día que huí de mi hermano Esaú, porque él había estado conmigo y me había traído. Entrad en paz a esta tierra, y eliminad los dioses extraños que están entre vosotros.
\par 2 Y abandonaron los dioses extraños, lo que tenían en las orejas y en el cuello, y los ídolos que Raquel había robado a su padre Labán, se los dio enteramente a Jacob. Y los quemó, los desmenuzó y los destruyó, y los escondió debajo de una encina que está en la tierra de Siquem.
\par 3 Y en la luna nueva del mes séptimo subió a Betel. Y edificó un altar en el lugar donde había dormido, y levantó allí una columna, y envió a decir a su padre Isaac que vinieran a él a su sacrificio, y a su madre Rebeca.
\par 4 Isaac dijo: «Que venga mi hijo Jacob y que yo lo vea antes de morir».
\par 5 Y Jacob fue a su padre Isaac y a su madre Rebeca, a la casa de su padre Abraham, y tomó consigo a dos de sus hijos, Leví y Judá, y vino a su padre Isaac y a su madre Rebeca.
\par 6 Y Rebeca salió de la torre al frente de ella para besar a Jacob y abrazarlo; porque su espíritu había revivido cuando oyó: 'He aquí, Jacob tu hijo ha venido'; y ella lo besó.
\par 7 Y ella vio a sus dos hijos, los reconoció y le dijo: «¿Son estos tus hijos, hijo mío?» y ella los abrazó, los besó y los bendijo, diciendo: 'En vosotros será esclarecida la descendencia de Abraham, y seréis una bendición en la tierra.'
\par 8 Y Jacob entró donde Isaac su padre, a la cámara donde yacía, y sus dos hijos estaban con él, y tomó la mano de su padre, e inclinándose lo besó, e Isaac se aferró al cuello de Jacob su hijo, y lloró sobre su cuello.
\par 9 Y la oscuridad desapareció de los ojos de Isaac, y vio a los dos hijos de Jacob, Leví y Judá, y dijo: «¿Son estos tus hijos, hijo mío?» porque son como tú.'
\par 10 Y él le dijo que eran verdaderamente sus hijos: «Y tú has visto verdaderamente que son verdaderamente mis hijos».
\par 11 Cuando se acercaron a él, él se volvió, los besó y los abrazó a ambos.
\par 12 Y el espíritu de profecía descendió a su boca, y tomó a Leví con su mano derecha y a Judá con su izquierda.
\par 13 Y se volvió primero hacia Leví y comenzó a bendecirlo primero, y le dijo: Que el Dios de todos, el mismo Señor de todos los siglos, te bendiga a ti y a tus hijos por todos los siglos.
\par 14 Y que el Señor te dé a ti y a tu descendencia grandeza y gran gloria, y haga que tú y tu descendencia, de entre toda carne, os acerquéis a Él para servir en su santuario como ángeles de la presencia y como santos. (Incluso) como ellos, la descendencia de tus hijos será para gloria, grandeza y santidad, y Él los engrandezca por todos los siglos.
\par 15 Y ellos serán jueces, príncipes y jefes de toda la descendencia de los hijos de Jacob;
\par    
\par     Hablarán la palabra del Señor con justicia,  
\par     Y juzgarán todos sus juicios con justicia.
\par    
\par     Y declararán mis caminos a Jacob  
\par     Y Mis caminos hacia Israel.
\par    
\par     La bendición del Señor será dada en su boca.  
\par     Para bendecir toda la semilla del amado.
\par    
\par 16 Tu madre llamó tu nombre Leví,  
\par     Y con justicia ha llamado tu nombre;
\par    
\par     Estarás unido al Señor  
\par     Y sé compañero de todos los hijos de Jacob;
\par    
\par     Que su mesa sea tuya,  
\par     Y comed de él tú y tus hijos;
\par    
\par     Y que tu mesa esté llena por todas las generaciones,  
\par     Y tu alimento no faltará para siempre.
\par    
\par 17 Y caigan ante ti todos los que te aborrecen,  
\par     Y sean desarraigados y perezcan todos tus adversarios;
\par    
\par     Y bendito sea el que te bendiga,  
\par     Y maldita sea toda nación que te maldiga.
\par    
\par 18 Y a Judá le dijo:  
\par     'Que el Señor te dé fuerza y ​​poder
\par    
\par     para hollar a todos los que te odian;  
\par     Serás príncipe, tú y uno de tus hijos, sobre los hijos de Jacob;
\par    
\par     Que tu nombre y el nombre de tus hijos salgan y atraviesen toda tierra y región.  
\par     Entonces las naciones temerán delante de ti,
\par    
\par     Y todas las naciones temblarán  
\par     [Y todos los pueblos temblarán].
\par    
\par 19 En ti estará la ayuda de Jacob,  
\par     Y en ti se encuentre la salvación de Israel.
\par    
\par 20 Y cuando te sientes en el trono de honor de tu justicia  
\par     Habrá gran paz para toda la descendencia de los hijos del amado;
\par    
\par     Bendito el que te bendiga,  
\par     Y todos los que te odian, te afligen y te maldicen.  
\par     Serán desarraigados y destruidos de la tierra y serán anatemas.'
\par    
\par 21 Y volviéndose, lo besó otra vez, lo abrazó y se alegró mucho; porque había visto en verdad a los hijos de Jacob su hijo.
\par 22 Y él salió de entre sus pies, se postró y se inclinó ante él, y los bendijo y descansó allí esa noche con Isaac su padre, y comieron y bebieron con alegría.
\par 23 E hizo dormir a los dos hijos de Jacob, uno a su derecha y el otro a su izquierda, y le fue contado por justicia.
\par 24 Y Jacob le contó a su padre durante la noche cómo el Señor le había mostrado gran misericordia, cómo lo había prosperado en todos sus caminos y lo había protegido de todo mal.
\par 25 E Isaac bendijo al Dios de su padre Abraham, que no había retirado su misericordia y su justicia de los hijos de su siervo Isaac.
\par 26 Y por la mañana, Jacob contó a su padre Isaac el voto que le había hecho al Señor, y la visión que había tenido, y que había construido un altar, y que todo estaba listo para el sacrificio que se haría antes. el Señor como había hecho voto, y que había venido a montarlo en un asno.
\par 27 Isaac dijo a su hijo Jacob: «No puedo ir contigo; porque soy viejo y no puedo soportar el camino: ve, hijo mío, en paz; porque hoy cumplo ciento sesenta y cinco años; Ya no puedo viajar; Pon a tu madre (sobre un asno) y déjala ir contigo.'
\par 28 «Y sé, hijo mío, que has venido por mi causa, y que sea bendito este día en el que me has visto vivo, y yo también te he visto a ti, hijo mío».
\par 29 'Que prosperes y cumplas el voto que has hecho; y no pospongas tu voto; porque serás llamado a rendir cuentas del voto; ahora, pues, apresúrate a cumplirlo, y que se complazca Aquel que hizo todas las cosas, a quien le has hecho el voto.'
\par 30 Y dijo a Rebeca: «Ve con Jacob tu hijo»; y Rebeca fue con su hijo Jacob, y Débora con ella, y llegaron a Betel.
\par 31 Y Jacob se acordó de la oración con que su padre lo había bendecido a él y a sus dos hijos, Leví y Judá, y se alegró y bendijo al Dios de sus padres, Abraham e Isaac.
\par 32 Y él dijo: «Ahora sé que tengo una esperanza eterna, y también mis hijos, delante del Dios de todos.» y así se ordena respecto de los dos; y lo registran como testimonio eterno para ellos en las tablas celestiales de cómo Isaac los bendijo.

\chapter{32}

\par \textit{El sueño de Leví en Betel, 1. Leví elegido para el sacerdocio, como décimo hijo, 2-3. Jacob celebra la fiesta de los tabernáculos y ofrece los diezmos a través de Leví: también el segundo diezmo, 4-9. Ley de diezmos ordenada, 10-15. Las visiones de Jacob en las que Jacob lee en las tablas celestiales su propio futuro y el de sus descendientes, 16-26. Celebra el día ochenta de la fiesta de los tabernáculos, 27-9. Muerte de Débora, 30. Nacimiento de Benjamín y muerte de Raquel, 33-4. (Cf. Gen. xxxv.8,10, 11, 13, 16-20.)}

\par 1 Y se quedó esa noche en Betel, y Leví soñó que lo habían ordenado y hecho sacerdote del Dios Altísimo, a él y a sus hijos para siempre; y despertó de su sueño y bendijo al Señor.
\par 2 Y Jacob se levantó temprano en la mañana, el catorce de este mes, y dio el diezmo de todo lo que venía con él, tanto de hombres como de ganado, tanto del oro como de todos los utensilios y vestidos, y también dio el diezmo. de todo.
\par 3 En aquellos días Raquel quedó embarazada de su hijo Benjamín. Y Jacob contó sus hijos desde él hacia arriba y Leví cayó a la porción del Señor, y su padre lo vistió con las vestiduras del sacerdocio y llenó sus manos.
\par 4 Y el quince de este mes, trajo al altar catorce bueyes de entre las vacas, veintiocho carneros, cuarenta y nueve ovejas, siete corderos y veintiún cabritos para quemarlo. -Ofrenda sobre el altar del sacrificio, de olor grato y agradable delante de Dios.
\par 5 Esta fue su ofrenda, como consecuencia del voto que había hecho, de dar el décimo, con sus ofrendas de frutas y sus libaciones.
\par 6 Y cuando el fuego lo consumió, quemó incienso sobre el fuego sobre el fuego, y en ofrenda de acción de gracias dos bueyes, cuatro carneros, cuatro ovejas, cuatro machos cabríos y dos ovejas de un año, y dos cabritos de las cabras; y así hizo diariamente durante siete días.
\par 7 Y él y todos sus hijos y sus hombres estuvieron allí comiendo con alegría durante siete días y bendiciendo y dando gracias al Señor, que lo había librado de todas sus tribulaciones y le había hecho su voto.
\par 8 Y diezmó todos los animales limpios e hizo un holocausto, pero los animales inmundos (no) los dio a Leví su hijo, y le dio todas las almas de los hombres.
\par 9 Y Leví ejerció el sacerdocio en Betel delante de su padre Jacob, antes que sus diez hermanos, y él era sacerdote allí, y Jacob hizo su voto: así diezmó de nuevo el diezmo al Señor y lo santificó, y se volvió santo para él.
\par 10 Y por esta razón está establecido en las tablas celestiales como ley para el diezmo volver a comer delante del Señor de año en año, en el lugar donde se ha elegido que habite su nombre, y a esta ley No hay límite de días para siempre.
\par 11 Esta ordenanza está escrita para que se cumpla de año en año al comer el segundo diezmo delante del Señor en el lugar elegido, y no quedará nada de él de este año al año siguiente.
\par 12 Porque en su año se comerá la semilla hasta los días de la recolección de la semilla del año, y el vino hasta los días del vino, y el aceite hasta los días de su estación.
\par 13 Y todo lo que quede de él y se vuelva viejo, será considerado contaminado y quemado en el fuego, porque es inmundo.
\par 14 Y así lo comerán juntos en el santuario, y no permitirán que se envejezca.
\par 15 Y todos los diezmos de los bueyes y de las ovejas serán consagrados al Señor y pertenecerán a sus sacerdotes, los cuales comerán delante de Él de año en año; porque así está ordenado y grabado con respecto al diezmo en las tablas celestiales.
\par 16 Y la noche siguiente, el día veintidós de este mes, Jacob resolvió edificar aquel lugar, rodear el atrio con un muro, santificarlo y santificarlo para siempre, para él y para sus hijos. niños después de él.
\par 17 Y el Señor se le apareció de noche, lo bendijo y le dijo: «No se llamará tu nombre Jacob, sino que llamarán tu nombre Israel».
\par 18 Y le dijo otra vez: «Yo soy el Señor que creó los cielos y la tierra, y te multiplicaré y te multiplicaré en gran manera, y reyes saldrán de ti, y juzgarán en todo lugar dondequiera que pise el pie de los hijos de los hombres han hollado.'
\par 19 «Y daré a tu descendencia toda la tierra que está debajo del cielo, y juzgarán a todas las naciones según sus deseos, y después tomarán posesión de toda la tierra y la heredarán para siempre».
\par 20 Y cuando acabó de hablar con él, se alejó de él. y Jacob miró hasta que ascendió al cielo.
\par 21 Y vio en una visión nocturna, y he aquí que un ángel descendía del cielo con siete tablas en sus manos, y se las dio a Jacob, y él las leyó y supo todo lo que en ellas estaba escrito y que le sucedería a él y sus hijos a lo largo de todos los siglos.
\par 22 Y le mostró todo lo que estaba escrito en las tablas, y le dijo: 'No edifiques este lugar, ni lo hagas un santuario eterno, ni habites aquí; porque este no es el lugar. Ve a la casa de Abraham tu padre y habita con Isaac tu padre hasta el día de la muerte de tu padre.'
\par 23 «Porque en Egipto morirás en paz, y en esta tierra serás sepultado con honor en el sepulcro de tus padres, con Abraham e Isaac».
\par 24 'No temas, porque como lo has visto y leído, así será todo; y escribe todo lo que has visto y leído.'
\par 25 Y Jacob dijo: 'Señor, ¿cómo podré recordar todo lo que he leído y visto? 'Y él le dijo: 'Te recordaré todas las cosas'.
\par 26 Y se levantó de él, y despertó de su sueño, y se acordó de todo lo que había leído y visto, y escribió todas las palabras que había leído y visto.
\par 27 Y celebró allí otro día más, y en él sacrificó como todo lo que había sacrificado en los días anteriores, y llamó su nombre «Adición», porque este día fue añadido, y a los días anteriores los llamó «Fiesta».
\par 28 Y así se manifestó que debía ser, y está escrito en las tablas celestiales; por lo cual se le reveló que debía celebrarlo y añadirlo a los siete días de la fiesta.
\par 29 Y se llamó su nombre Adición, porque estaba inscrito entre los días de las fiestas, según el número de los días del año.
\par 30 Y en la noche del veintitrés de este mes, murió la nodriza de Débora Rebeca, y la enterraron debajo de la ciudad, bajo la encina del río, y llamó el nombre de este lugar: Río de Débora. ,' y el roble, 'El roble del luto de Débora'.
\par 31 Y Rebeca fue y regresó a su casa con su padre Isaac, y Jacob le envió por mano carneros, ovejas y machos cabríos para que preparara para su padre la comida que él deseaba.
\par 32 Y siguió a su madre hasta que llegó a la tierra de Kabratan, y habitó allí.
\par 33 Y Raquel dio a luz un hijo esa noche, y llamó su nombre Hijo de mi dolor; porque ella sufrió al darlo a luz; pero su padre llamó su nombre Benjamín, el día once del mes octavo, en el primero del sexto septenario de este jubileo. [2143 a.m.]
\par 34 Allí murió Raquel y fue sepultada en la tierra de Efrata, que es Belén, y Jacob edificó una columna sobre la tumba de Raquel, en el camino encima de su tumba.

\chapter{33}

\par \textit{Rubén peca con Bilha, 1-9 (cf. Gén. xxxv.21, 22). Leyes relativas al incesto, 10-20. Los hijos de Jacob, 22. (Cf. Gen. xxxv.23-7.)}

\par 1 Y Jacob fue y habitó al sur de Magdaladraef. Y volvió a su padre Isaac, él y Lea su mujer, en la luna nueva del décimo mes.
\par 2 Y Rubén vio a Bilha, la sierva de Raquel, concubina de su padre, bañándose en agua en un lugar secreto, y se enamoró de ella.
\par 3 Y se escondió por la noche, y entró en la casa de Bilhah, y la encontró durmiendo sola en una cama en su casa.
\par 4 Y él se acostó con ella, y ella despertó y vio, y he aquí que Rubén estaba acostado con ella en la cama, y ​​ella descubrió el borde de su manto, lo agarró y gritó, y descubrió que era Rubén.
\par 5 Ella, avergonzada de él, le soltó la mano y él huyó.
\par 6 Y ella se lamentó mucho por esto, y no se lo contó a nadie.
\par 7 Y cuando Jacob volvió y la buscó, ella le dijo: 'No estoy limpio para ti, porque he sido contaminado contigo; porque Rubén me ha contaminado, y se ha acostado conmigo durante la noche, y yo dormía, y no lo descubrí hasta que me descubrió la falda y se acostó conmigo.'
\par 8 Y Jacob se enojó mucho contra Rubén porque se había acostado con Bilha, porque había descubierto el manto de su padre.
\par 9 Y Jacob no volvió a acercarse a ella porque Rubén la había contaminado. Y cualquier hombre que descubre el manto de su padre, su obra es muy mala, porque es abominable ante el Señor.
\par 10 Por eso está escrito y ordenado en las tablas celestiales que un hombre no debe acostarse con la mujer de su padre, ni descubrir el manto de su padre, porque esto es inmundo; ciertamente morirán juntos, el hombre que se acuesta con ella. la mujer de su padre y también la mujer, porque han hecho inmundicia en la tierra.
\par 11 Y no habrá nada inmundo delante de nuestro Dios en la nación que Él ha elegido para sí como posesión.
\par 12 Y otra vez está escrito por segunda vez: «Maldito el que se acueste con la mujer de su padre, porque ha descubierto la vergüenza de su padre». y todos los santos del Señor dijeron: 'Así sea; que así sea.'
\par 13 Y tú, Moisés, ordena a los hijos de Israel que observen esta palabra; porque (implica) un castigo de muerte; y es inmundo, y no hay expiación para siempre para el hombre que ha cometido esto, sino que será ejecutado y degollado, y apedreado, y desarraigado de en medio del pueblo de nuestro Dios.
\par 14 Porque a nadie que haga esto en Israel se le permitirá permanecer con vida un solo día en la tierra, porque es abominable e inmundo.
\par 15 Y que no digan: A Rubén se le concedió la vida y el perdón después de haberse acostado con la concubina de su padre, y a ella también aunque tenía marido, y su marido Jacob, su padre, aún vivía.
\par 16 Porque hasta entonces no había sido revelado el ordenamiento, el juicio y la ley en su totalidad para todos, sino en tus días (ha sido revelado) como ley de las estaciones y de los días, y ley eterna para los siglos de los siglos. generaciones.
\par 17 Y para esta ley no hay consumación de días ni expiación por ella, sino que ambos deben ser desarraigados en medio de la nación: el día en que la cometieron, los matarán.
\par 18 Y tú, Moisés, escríbelo para Israel, para que lo observen y hagan conforme a estas palabras, y no cometan pecado de muerte; porque el Señor nuestro Dios es juez, que no hace acepción de personas ni acepta dádivas.
\par 19 Y diles estas palabras del pacto, para que las oigan y las observen, y estén en guardia con respecto a ellos, y no sean destruidos ni desarraigados de la tierra; porque inmundicia, abominación, contaminación y contaminación son todos los que hacen esto en la tierra delante de nuestro Dios.
\par 20 Y no hay mayor pecado que la fornicación que cometen en la tierra; porque Israel es nación santa para Jehová su Dios, y nación de herencia, y nación sacerdotal y real, y para posesión (su propia); y no aparecerá tal inmundicia en medio de la nación santa.
\par 21 Y en el tercer año de esta sexta semana [2145 AM] Jacob y todos sus hijos fueron y habitaron en la casa de Abraham, cerca de Isaac su padre y Rebeca su madre.
\par 22 Y estos fueron los nombres de los hijos de Jacob: el primogénito, Rubén, Simeón, Leví, Judá, Isacar, Zabulón, los hijos de Lea; y los hijos de Raquel, José y Benjamín; y los hijos de Bilhah, Dan y Neftalí; y los hijos de Zilpa, Gad y Aser; y Dina, hija de Lea, hija única de Jacob.
\par 23 Y vinieron y se inclinaron ante Isaac y Rebeca, y cuando los vieron bendijeron a Jacob y a todos sus hijos, e Isaac se alegró mucho porque vio a los hijos de Jacob, su hijo menor, y los bendijo.

\chapter{34}

\par \textit{Guerra de los reyes amorreos contra Jacob y sus hijos, 1-9. Jacob envía a José a visitar a sus hermanos, 10. José vendido y llevado a Egipto, 11-12 (cf. Gén. xxxvii.14, 17, 18, 25, 32-6). Muertes de Bilha y Dina, 15. Jacob llora por José, 13, 14, 17. Institución del Día de la Expiación el día en que llegó la noticia de la muerte de José, 18-19. Esposas del hijo de Jacob, 20-1.}

\par 1 Y en el sexto año de esta semana de este cuadragésimo cuarto jubileo [2148 AM] Jacob envió a sus hijos a apacentar sus ovejas, y a sus siervos con ellos a los pastos de Siquem.
\par 2 Y los siete reyes amorreos se juntaron contra ellos para matarlos, escondiéndose debajo de los árboles y para apoderarse de sus ganados.
\par 3 Y Jacob, Leví, Judá y José estaban en casa con Isaac su padre; porque su espíritu estaba triste, y no podían dejarlo; y Benjamín era el menor, y por esto se quedó con su padre.
\par 4 Y vinieron el rey de Taphu, el rey de Aresa, el rey de Seragán, el rey de Selo y el rey de Ga. as, y el rey de Bethorón, y el rey de Ma'anisakir, y todos los que habitan en estas montañas (y) que habitan en los bosques en la tierra de Canaán.
\par 5 Y anunciaron esto a Jacob, diciendo: «He aquí, los reyes de los amorreos han rodeado a tus hijos y saqueado sus rebaños».
\par 6 Y se levantó de su casa, él y sus tres hijos y todos los sirvientes de su padre, y sus propios sirvientes, y fue contra ellos con seis mil hombres que llevaban espadas.
\par 7 Y los mató en los pastos de Siquem, y persiguió a los que huían, y los mató a filo de espada, y mató a 'Aresa, Taphu, Saregan, Selo, Amani-sakir y Ga[ga]. ]'as, y recuperó sus rebaños.
\par 8 Y él los venció y les impuso tributo, cinco productos del fruto de su tierra, y edificó Robel y Tamnatares.
\par 9 Y él regresó en paz e hizo las paces con ellos, y fueron sus siervos hasta el día en que él y sus hijos descendieron a Egipto.
\par 10 Y en el séptimo año de esta semana [2149 a.m.] envió a José a informarse sobre el bienestar de sus hermanos desde su casa a la tierra de Siquem, y los encontró en la tierra de Dotán.
\par 11 Y lo traicionaron y tramaron un complot contra él para matarlo, pero cambiando de opinión, lo vendieron a mercaderes ismaelitas, lo llevaron a Egipto y lo vendieron a Potifar, eunuco de Faraón, jefe de los cocineros, sacerdote de la ciudad de 'Elew.
\par 12 Y los hijos de Jacob sacrificaron un cabrito, mojaron la túnica de José en la sangre y se la enviaron a su padre Jacob el diez del mes séptimo.
\par 13 Y estuvo de luto toda esa noche, porque se lo habían traído por la tarde, y se puso febril de luto por su muerte, y dijo: «Una mala bestia ha devorado a José». y todos los miembros de su casa estuvieron en duelo con él aquel día, y estuvieron enlutados y enlutados con él todo aquel día.
\par 14 Sus hijos y su hija se levantaron para consolarlo, pero él no quiso ser consolado por su hijo.
\par 15 Y ese día Bilha se enteró de que José había fallecido, y murió llorando por él, y vivía en Qafratef, y también Dina, su hija, murió después de la muerte de José.
\par 16 Y en un mes cayeron sobre Israel estos tres duelos. Y enterraron a Bilha frente al sepulcro de Raquel, y también a Dina. su hija, la enterraron allí.
\par 17 Y estuvo de luto por José un año y no cesó, porque decía: «Déjame bajar al sepulcro llorando por mi hijo».
\par 18 Por esta razón está ordenado a los hijos de Israel que se aflijan el día diez del mes séptimo, el día en que llegó a su padre Jacob la noticia que le hizo llorar por José, para hacer expiación. para sí sobre él con un cabrito el día diez del séptimo mes, una vez al año, por sus pecados; porque habían entristecido el cariño de su padre hacia José su hijo.
\par 19 Y se ha ordenado este día para que se lamenten en él por sus pecados, por todas sus transgresiones y por todos sus errores, para poder limpiarse en ese día una vez al año.
\par 20 Y después de la muerte de José, los hijos de Jacob tomaron para sí mujeres. El nombre de la esposa de Rubén es 'Ada; y el nombre de la esposa de Simeón es 'Adlba'a, cananea; y el nombre de la esposa de Leví es Melka, de las hijas de Aram, de la simiente de los hijos de Taré; y el nombre de la esposa de Judá, Betasu'el, cananea; y el nombre de la esposa de Isacar, Hezaqa; y el nombre de la esposa de Zabulón, Ni'iman; y el nombre de la esposa de Dan, 'Egla; y el nombre de la esposa de Neftalí, Rasu'u, de Mesopotamia; y el nombre de la esposa de Gad, Maka; y el nombre de la esposa de Aser, 'Ijona; y el nombre de la esposa de José, Asenat, la egipcia; y el nombre de la esposa de Benjamín, 'Ijasaka.
\par 21 Entonces Simeón se arrepintió y tomó como hermanos a una segunda esposa de Mesopotamia.

\chapter{35}

\par \textit{La amonestación de Rebeca a Jacob y su respuesta, 1-8. Rebeca le pide a Isaac que haga jurar a Esaú que no dañará a Jacob, 9-12. Isaac consiente, 13-17. Esaú presta juramento y también Jacob, 18-26. Muerte de Rebeca, 27.}

\par 1 Y en el primer año de la primera semana del jubileo cuadragésimo quinto [2157 AM] Rebeca llamó a Jacob, su hijo, y le ordenó respecto a su padre y a su hermano, que los honrara todos los días de su vida.
\par 2 Y Jacob dijo: 'Haré todo como me has mandado; porque esto me será honra y grandeza, y justicia delante del Señor, para honrarlos.'
\par 3 'Y tú también, madre, sabes desde que nací hasta el día de hoy, todas mis obras y todo lo que hay en mi corazón, que siempre pienso bien para con todos.'
\par 4 «¿Cómo no voy a hacer lo que me has mandado, honrar a mi padre y a mi hermano?»
\par 5 «Dime, madre, qué perversidad has visto en mí y me apartaré de ella y la misericordia será conmigo».
\par 6 Y ella le dijo: 'Hijo mío, no he visto en ti ninguna acción perversa, sino (sólo) recta. Y, sin embargo, te diré la verdad, hijo mío: moriré este año y no sobreviviré este año de mi vida; porque he visto en sueños el día de mi muerte, que no viviría más de ciento cincuenta y cinco años; y he aquí, he cumplido todos los días de mi vida que he de vivir.'
\par 7 Y Jacob se rió de las palabras de su madre. porque su madre le había dicho que debía morir; y ella estaba sentada frente a él en posesión de su fuerza, y no estaba debilitada en su fuerza; porque entraba y salía y veía, y sus dientes eran fuertes, y ninguna enfermedad la había tocado en todos los días de su vida.
\par 8 Y Jacob le dijo: 'Bienaventurada soy, madre, si mis días se acercan a los días de tu vida, y mi fuerza permanece conmigo como tu fuerza, y no morirás, porque estás bromeando conmigo. con respecto a tu muerte.'
\par 9 Y ella fue donde Isaac y le dijo: 'Una petición te hago: haz que Esaú jure que no dañará a Jacob ni lo perseguirá con enemistad; porque tú conoces los pensamientos de Esaú, que son perversos desde su juventud, y no hay bien en él; porque desea matarlo después de tu muerte.'
\par 10 'Y tú sabes todo lo que ha hecho desde el día que su hermano Jacob fue a Harán hasta el día de hoy: cómo nos ha abandonado de todo corazón y nos ha hecho mal; Se ha apoderado de tus rebaños y se ha llevado todos tus bienes de delante de ti.'
\par 11 «Y cuando le imploramos y suplicamos por lo que era nuestro, él actuó como un hombre que se apiada de nosotros».
\par 12 «Y está amargado contra ti porque bendijiste a Jacob, tu hijo perfecto y recto; porque no hay maldad sino sólo bondad en él, y desde que vino de Harán hasta el día de hoy no nos ha robado nada, porque siempre nos trae todo a su debido tiempo, y se regocija con todo su corazón cuando tomamos de sus manos. y él nos bendice, y no se ha separado de nosotros desde que vino de Harán hasta el día de hoy, y permanece con nosotros continuamente en casa, honrándonos.'
\par 13 E Isaac le dijo: 'Yo también conozco y veo las obras de Jacob, que está con nosotros, cómo nos honra con todo su corazón; pero antes amaba a Esaú más que a Jacob, porque era el primogénito; pero ahora amo a Jacob más que a Esaú, porque ha hecho muchas malas acciones, y no hay justicia en él, porque todos sus caminos son injusticia y violencia, [y no hay justicia a su alrededor.]'
\par 14 «Y ahora mi corazón está turbado por todas sus obras, y ni él ni su descendencia podrán salvarse, porque ellos son los que serán destruidos de la tierra y los que serán desarraigados de debajo del cielo, porque él ha abandonado al Dios de Abraham y se ha ido en pos de sus mujeres, en pos de sus inmundicias y en pos de sus extravíos, él y sus hijos.'
\par 15 'Y me pides que le haga jurar que no matará a Jacob su hermano; Incluso si jura, no cumplirá su juramento y no hará el bien, sino sólo el mal.'
\par 16 «Pero si quiere matar a Jacob, su hermano, en manos de Jacob será entregado, y no escapará de sus manos, porque descenderá a sus manos».
\par 17 'Y no temas por causa de Jacob; porque el guardián de Jacob es grande, poderoso y honrado, y más alabado que el guardián de Esaú.'
\par 18 Entonces Rebeca envió y llamó a Esaú, y él vino a ella, y ella le dijo: «Tengo una petición que hacerte, hijo mío, y me prometes hacerla, hijo mío».
\par 19 Y él dijo: «Haré todo lo que me digas y no rechazaré tu petición».
\par 20 Y ella le dijo: 'Te pido que el día de mi muerte me recibas y me entierres cerca de Sara, la madre de tu padre, y que tú y Jacob os améis y ninguno de los dos desee el mal contra el otro, sino sólo amor mutuo, y (así) prosperaréis, hijos míos, y seréis honrados en medio de la tierra, y ningún enemigo se regocijará sobre vosotros, y seréis una bendición y una misericordia a los ojos de todos aquellos que te aman.'
\par 21 Y él dijo: «Haré todo lo que me has dicho y te enterraré el día que mueras cerca de Sara, la madre de mi padre, como has deseado que sus huesos estén cerca de los tuyos».
\par 22 'Y también amaré a mi hermano Jacob más que a toda carne; porque no tengo hermano en toda la tierra sino sólo él: y esto no es gran mérito para mí si lo amo; porque él es mi hermano, y juntos fuimos sembrados en tu cuerpo, y juntos salimos de tu vientre, y si no amo a mi hermano, ¿a quién amaré?
\par 23 «Y yo te ruego que exhortes a Jacob respecto a mí y a mis hijos, porque sé que él seguramente será rey sobre mí y sobre mis hijos, porque el día que mi padre lo bendijo, lo hizo más alto y más alto. yo el inferior.
\par 24 Y te juro que lo amaré y no desearé ningún mal contra él en todos los días de mi vida, sino sólo el bien.
\par 25 Y él le juró sobre todo este asunto. Y llamó a Jacob delante de los ojos de Esaú, y le dio mandamiento conforme a las palabras que le había hablado a Esaú.
\par 26 Y él dijo: 'Haré tu voluntad; Créanme que ningún mal procederá de mí ni de mis hijos contra Esaú, y seré el primero en nada excepto en el amor.'
\par 27 Y esa noche comieron y bebieron ella y sus hijos, y ella murió, tres jubileos, una semana y un año, esa noche; y sus dos hijos, Esaú y Jacob, la sepultaron en la doble cueva cerca de Sarah, la madre de su padre.

\chapter{36}

\par \textit{Isaac da instrucciones a sus hijos en cuanto a su sepultura: los exhorta a amarse unos a otros y les hace imprecar destrucción al que hiera a su hermano, 1-11. Divide sus posesiones, le da la porción mayor a Jacob y muere, 12-18. Muere Lea: los hijos de Jacob vienen a consolarlo, 21-4.}

\par 1 Y en el sexto año de esta semana [2162 AM] Isaac llamó a sus dos hijos Esaú y Jacob, y vinieron a él, y él les dijo: 'Hijos míos, voy a seguir el camino de mis padres, para la casa eterna donde están mis padres.'
\par 2 'Por tanto, entiérrame cerca de Abraham mi padre, en la doble cueva en el campo de Efrón el hitita, donde Abraham compró un sepulcro para sepultarlo; en el sepulcro que cavé para mí, entiérrame allí.
\par 3 «Y esto os mando, hijos míos, que practiquéis la justicia y la rectitud en la tierra, para que el Señor pueda traer sobre vosotros todo lo que el Señor dijo que haría con Abraham y con su descendencia».
\par 4 'Y amaos unos a otros, hijos míos, vuestros hermanos como un hombre que ama su propia alma, y ​​cada uno busque lo que pueda beneficiar a su hermano, y actúen juntos en la tierra; y que se amen unos a otros como a sus propias almas.'
\par 5 'Y en cuanto a la cuestión de los ídolos, os mando y advierto que los rechacéis, los odiéis y no los améis, porque están llenos de engaño para quienes los adoran y para quienes se inclinan ante ellos.'
\par 6 'Acordaos, hijos míos, del Señor Dios de Abraham vuestro padre, y de cómo yo también lo adoré y le serví con justicia y con alegría, para que él os multiplique y aumente vuestra descendencia como las estrellas del cielo en multitud. y establecerte sobre la tierra como planta de justicia que no será desarraigada por todas las generaciones para siempre.'
\par 7 «Y ahora os haré hacer un gran juramento, porque no hay juramento mayor que el del nombre glorioso, honorable, grande, espléndido, maravilloso y poderoso, que creó los cielos y la tierra y todas las cosas. juntos, para que le temáis y le adoréis.'
\par 8 «Y que cada uno ame a su hermano con cariño y rectitud, y que no desee ningún mal contra su hermano desde ahora y para siempre todos los días de su vida, para que prosperen en todas sus obras y no sean destruidos».
\par 9 «Y si alguno de vosotros trama el mal contra su hermano, sepan que desde ahora en adelante cualquiera que planee el mal contra su hermano caerá en sus manos y será desarraigado de la tierra de los vivientes, y su descendencia será destruida. de debajo del cielo.'
\par 10 «Pero el día de la turbulencia, de la execración, de la indignación y de la ira, con llama de fuego devorador, como quemó a Sodoma, así también quemará su tierra y su ciudad y todo lo que es suyo, y será borrado del mundo. libro de la disciplina de los hijos de los hombres, y no será registrado en el libro de la vida, sino en el que está destinado a destrucción, y pasará a la execración eterna; para que su condenación sea siempre renovada en odio y en execración y en ira y en tormento y en indignación y en plagas y en enfermedad para siempre.'
\par 11 «Os digo y testifico, hijos míos, conforme al juicio que caerá sobre el que quiera hacer daño a su hermano».
\par 12 «Y aquel día dividió todos sus bienes entre los dos y le dio la mayor parte al primogénito, y la torre y todo lo que la rodeaba, y todo lo que Abraham poseía junto al pozo de el juramento.'
\par 13 Y él dijo: «Esta porción mayor se la daré al primogénito».
\par 14 Y Esaú dijo: «He vendido a Jacob y le he dado mi primogenitura; a él se le dé, y no tengo ni una sola palabra que decir al respecto, porque es suyo.'
\par 15 Y dijo Isaac: «Que hoy, hijos míos, la bendición caiga sobre vosotros, hijos míos, y sobre vuestra descendencia, porque me habéis dado descanso, y mi corazón no se aflige por la primogenitura, para que no cometáis maldad por causa de él.'
\par 16 «Que el Dios Altísimo bendiga al hombre que hace justicia, a él y a su descendencia para siempre».
\par 17 Y terminó de mandarles y bendecirlos, y comieron y bebieron juntos delante de él, y él se alegró porque había una sola mente entre ellos, y salieron de él y descansaron ese día y durmieron.
\par 18 Y aquel día Isaac durmió gozoso en su cama; y durmió el sueño eterno, y murió de ciento ochenta años. Cumplió veinticinco semanas y cinco años; y sus dos hijos Esaú y Jacob lo sepultaron.
\par 19 Y Esaú fue a la tierra de Edom, a las montañas de Seir, y habitó allí.
\par 20 Y Jacob habitó en las montañas de Hebrón, en la torre de la tierra de las estancias de su padre Abraham, y adoró al Señor con todo su corazón y según los mandamientos visibles, tal como Él había dividido los días de su generaciones.
\par 21 Y murió Lea su esposa en el año cuarto de la segunda semana del jubileo cuadragésimo quinto, y él la enterró en la doble cueva cerca de su madre Rebeca, a la izquierda de la tumba de Sara, la hija de su padre. madre
\par 22 Y todos sus hijos y sus hijos vinieron con él a llorar por Lea su esposa y a consolarlo acerca de ella, porque él estaba llorando por ella porque la amaba mucho después de la muerte de su hermana Raquel.
\par 23 porque ella era perfecta y recta en todos sus caminos y honraba a Jacob, y en todos los días que vivió con él él no escuchó de su boca palabra dura, porque ella era amable, pacífica, recta y honorable.
\par 24 Y él se acordó de todos los hechos que ella había hecho durante su vida y se lamentó mucho; porque él la amaba con todo su corazón y con toda su alma.

\chapter{37}

\par \textit{Los hijos de Esaú le reprochan su subordinación a Jacob y lo obligan a guerrear con la ayuda de 4.000 mercenarios contra Jacob, 1-15. Jacob reprende a Esaú, 16-17. Respuesta de Esaú, 18-25.}

\par 1 Y el día que murió Isaac el padre de Jacob y Esaú, [2162 AM] los hijos de Esaú oyeron que Isaac había dado la porción del mayor a su hijo menor, Jacob, y se enojaron mucho.
\par 2 Y discutieron con su padre, diciendo: «¿Por qué tu padre le dio a Jacob la porción del mayor y te pasó por alto, siendo tú el mayor y Jacob el menor?»
\par 3 Y él les dijo: «Porque vendí mi primogenitura a Jacob por una pequeña porción de lentejas, y el día que mi padre me envió a cazar y pescar y a traerle algo para que comiera y me bendijera, vino con engaño y trajo a mi padre comida y bebida, y mi padre lo bendijo y me puso bajo su mano.'
\par 4 «Y ahora nuestro padre nos ha hecho jurar, a él y a mí, que no tramaremos el mal ni contra su hermano, y que permaneceremos en amor y en paz cada uno con su hermano y no seguiremos nuestros caminos. corrupto.'
\par 5 Y ellos le dijeron: 'No te escucharemos para hacer la paz con él; porque nuestra fuerza es mayor que su fuerza, y nosotros somos más poderosos que él; iremos contra él y lo mataremos y lo destruiremos a él y a sus hijos. Y si no vienes con nosotros, también te haremos daño.
\par 6 «Y ahora escúchennos: enviemos a Aram, a Filistea, a Moab y a Amón, y escojamos hombres escogidos y ardientes para la batalla, y vayamos contra ellos y peleemos contra ellos, y dejemos que Exterminémoslo de la tierra antes de que se fortalezca.'
\par 7 Y su padre les dijo: «No vayáis ni le peleéis, no sea que caigáis delante de él».
\par 8 Y ellos le dijeron: 'Esto también es exactamente lo que has hecho desde tu juventud hasta el día de hoy, y estás poniendo tu cuello bajo su yugo.
\par 9 No escucharemos estas palabras.' Y enviaron a Aram y a 'Aduram al amigo de su padre, y contrataron junto con ellos mil guerreros, hombres de guerra escogidos.
\par 10 Y de Moab y de los hijos de Amón vinieron a ellos mil hombres escogidos, y de Filistea, mil hombres de guerra escogidos, y de Edom y de los horeos, mil guerreros escogidos. , y de los Kittim, valientes guerreros.
\par 11 Y dijeron a su padre: «Ve con ellos y guíalos, de lo contrario te mataremos».
\par 12 Y se llenó de ira y de indignación al ver que sus hijos lo obligaban a ir delante de ellos para liderarlos contra su hermano Jacob.
\par 13 Pero después se acordó de todo el mal que había escondido en su corazón contra su hermano Jacob; y no se acordó del juramento que había hecho a su padre y a su madre de que no tramaría ningún mal en todos sus días contra Jacob su hermano.
\par 14 Y a pesar de todo esto, Jacob no sabía que venían contra él a la batalla, y estaba de luto por Lea, su esposa, hasta que se acercaron muy cerca de la torre con cuatro mil guerreros y hombres de guerra escogidos.
\par 15 Y los hombres de Hebrón enviaron a decirle: «He aquí, tu hermano ha venido contra ti para pelear contra ti, con cuatro mil armados con espada, y llevan escudos y armas». porque amaban a Jacob más que a Esaú. Entonces le dijeron; porque Jacob era un hombre más liberal y misericordioso que Esaú.
\par 16 Pero Jacob no creyó hasta que llegaron muy cerca de la torre.
\par 17 Y cerró las puertas de la torre; y se paró sobre las almenas y habló a su hermano Esaú, y dijo: 'Noble es el consuelo con que has venido a consolarme por mi mujer que ha muerto. ¿Es este el juramento que le hiciste a tu padre y a tu madre antes de que murieran? Has roto el juramento, y en el momento en que lo juraste a tu padre fuiste condenado.
\par 18 Entonces Esaú respondió y le dijo: Ni los hijos de los hombres ni las bestias de la tierra tienen ningún juramento de justicia que al jurar hayan hecho (un juramento válido) para siempre; pero cada día idean el mal unos contra otros, y cómo cada uno puede matar a su adversario y enemigo.
\par 19 Y me odiarás a mí y a mis hijos para siempre. Y no se puede observar el vínculo de hermandad contigo.
\par 20 Oye estas palabras que te declaro,
\par    
\par     Si el jabalí puede cambiar su piel y hacer que sus cerdas sean tan suaves como la lana,  
\par     O si puede hacer brotar en su cabeza cuernos como los de un ciervo o de una oveja,  
\par     Entonces observaré el vínculo de hermandad contigo  
\par     Y si los pechos se separaron de su madre, porque no has sido mi hermano.  
\par    
\par 21 Y si los lobos hacen las paces con los corderos para no devorarlos ni agredirlos,  
\par     Y si su corazón está hacia ellos para siempre,  
\par     Entonces habrá paz en mi corazón hacia ti.
\par    
\par 22 Y si el león se hace amigo del buey y hace las paces con él  
\par     Y si está sujeto con él bajo un yugo y ara con él,  
\par     Entonces haré las paces contigo.  
\par    
\par 23 Y cuando el cuervo se vuelva blanco como la raza,  
\par     Entonces sabes que te he amado  
\par     Y haré las paces contigo  
\par     Serás desarraigado,  
\par     Y tus hijos serán desarraigados,  
\par     Y no habrá paz para ti'
\par    
\par 24 Y cuando Jacob vio que estaba tan mal dispuesto hacia él con su corazón y con toda su alma como para matarlo, y que había venido saltando como el jabalí que ataca la lanza que atraviesa y mata. y no retrocede ante ello;
\par 25 Entonces dijo a los suyos y a sus siervos que lo atacaran a él y a todos sus compañeros.

\chapter{38}

\par \textit{Guerra entre Jacob y Esaú. Muerte de Esaú y derrocamiento de sus fuerzas, 1-10. Edom reducido a servidumbre 'hasta el día de hoy', 11-14. Reyes de Edom, 15-24. (Cf. Gen. xxxvi.31-9.)}

\par 1 Después de esto, Judá habló a Jacob, su padre, y le dijo: 'Entiende tu arco, padre, y lanza tus flechas y derriba al adversario y mata al enemigo; y que tengas el poder, porque no mataremos a tu hermano, porque él es como tú, y él es como tú, démosle (este) honor.'
\par 2 Entonces Jacob tensó su arco y lanzó la flecha, hirió a su hermano Esaú (en su pecho derecho) y lo mató.
\par 3 Y otra vez lanzó una flecha y hirió a Adoran el arameo en el pecho izquierdo, y lo empujó hacia atrás y lo mató.
\par 4 Entonces los hijos de Jacob, ellos y sus siervos, salieron dividiéndose en grupos a los cuatro lados de la torre.
\par 5 Y Judá avanzó al frente, y Neftalí y Gad con él y cincuenta siervos con él en el lado sur de la torre, y mataron a todos los que encontraron delante de ellos, y ninguno de ellos escapó.
\par 6 Y Leví, Dan y Aser salieron al lado oriental de la torre, y cincuenta hombres con ellos, y mataron a los guerreros de Moab y Amón.
\par 7 Rubén, Isacar y Zabulón, acompañados de cincuenta hombres, salieron por el lado norte de la torre y mataron a los guerreros filisteos.
\par 8 Y Simeón, Benjamín y Enoc, hijo de Rubén, salieron al lado occidental de la torre, y cincuenta (hombres) con ellos, y mataron de Edom y de los horeos a cuatrocientos hombres, valientes guerreros; y seiscientos huyeron, y cuatro de los hijos de Esaú huyeron con ellos, y dejaron a su padre muerto, como había caído en la colina que está en Aduram.
\par 9 Y los hijos de Jacob los persiguieron hasta las montañas de Seir. Y Jacob sepultó a su hermano en la colina que está en Aduram, y regresó a su casa.
\par 10 Y los hijos de Jacob presionaron duramente a los hijos de Esaú en las montañas de Seir, e inclinaron sus cuellos para convertirse en siervos de los hijos de Jacob.
\par 11 Y enviaron a preguntar a su padre si debían hacer las paces con ellos o matarlos.
\par 12 Y Jacob envió un mensaje a sus hijos para que hicieran la paz, y ellos hicieron la paz con ellos, y pusieron sobre ellos el yugo de servidumbre, de modo que pagaran tributo a Jacob y a sus hijos para siempre.
\par 13 Y continuaron pagando tributo a Jacob hasta el día en que descendió a Egipto.
\par 14 Y los hijos de Edom no se han liberado del yugo de servidumbre que los doce hijos de Jacob les habían impuesto hasta el día de hoy.
\par 15 Estos son los reyes que reinaron en Edom antes de que reinara rey sobre los hijos de Israel en la tierra de Edom.
\par 16 Y Balaq hijo de Beor reinó en Edom, y el nombre de su ciudad era Danaba.
\par 17 Y murió Balaq, y en su lugar reinó Jobab, hijo de Zara de Boser.
\par 18 Y murió Jobab, y reinó en su lugar Asam, de la tierra de Temán.
\par 19 Y murió Asam, y en su lugar reinó Adath, el hijo de Barad, que mató a Madián en el campo de Moab, y el nombre de su ciudad era Avith.
\par 20 Y Adath murió, y Salmán, de Amaseqa, reinó en su lugar.
\par 21 Y murió Salmán, y reinó en su lugar Saúl de Raaboth (junto al río).
\par 22 Y murió Saúl, y reinó en su lugar Baelunán, hijo de Acbor.
\par 23 Y murió Baelunan, hijo de Acbor, y Adath reinó en su lugar, y el nombre de su esposa era Maitabith, hija de Matarat, hija de Metabedzaab.
\par 24 Estos son los reyes que reinaron en la tierra de Edom.

\chapter{39}

\par \textit{José se encargó de la casa de Potifar, 1-4. Su pureza y encarcelamiento, 5-13. Encarcelamiento del jefe de los coperos y del jefe de los panaderos de Faraón, cuyos sueños interpreta José, 14-18. (Cf. Gen.xxxvii.2; xxxix.3-8, 12-15, 17-23; xl.1-5, 21-3; xli.1.)}

\par 1 Y Jacob habitó en la tierra de peregrinación de su padre, en la tierra de Canaán. Estas son las generaciones de Jacob.
\par 2 José tenía diecisiete años cuando lo llevaron a la tierra de Egipto, y lo compró Potifar, eunuco de Faraón, el jefe de cocina.
\par 3 Y puso a José sobre toda su casa y la bendición del Señor vino sobre la casa del egipcio a causa de José, y el Señor lo prosperó en todo lo que hizo.
\par 4 Y el egipcio entregó todo en manos de José; porque vio que el Señor estaba con él, y que el Señor lo prosperaba en todo lo que hacía.
\par 5 Y la apariencia de José era hermosa [y muy hermosa era su apariencia], y la esposa de su amo alzó los ojos y vio a José, y lo amó y le rogó que se acostara con ella.
\par 6 Pero él no entregó su alma, y ​​se acordó del Señor y de las palabras que Jacob, su padre, solía leer de entre las palabras de Abraham, que nadie debería fornicar con una mujer que tiene marido; que para él ha sido ordenado en los cielos delante del Dios Altísimo el castigo de muerte, y el pecado quedará registrado contra él en los libros eternos continuamente delante del Señor.
\par 7 José se acordó de estas palabras y no quiso acostarse con ella.
\par 8 Ella le rogó durante un año, pero él se negó y no la escuchó.
\par 9 Pero ella lo abrazó y lo retuvo en la casa para obligarlo a acostarse con ella, y cerró las puertas de la casa y lo retuvo; pero él dejó su manto en sus manos, irrumpió por la puerta y huyó fuera de su presencia.
\par 10 Y viendo la mujer que él no quería acostarse con ella, lo calumnió en presencia de su señor, diciendo: Tu siervo hebreo, a quien amas, trató de obligarme a acostarse conmigo; y aconteció que cuando alcé mi voz, él huyó y dejó su manto en mis manos cuando lo retuve, y entró por la puerta.'
\par 11 Y el egipcio vio el manto de José y la puerta rota, y escuchó las palabras de su esposa, y encarceló a José, en el lugar donde estaban los prisioneros que el rey encarcelaba.
\par 12 Y él estaba allí en la cárcel; y el Señor le dio a José favor ante los ojos del jefe de los guardias de la prisión y compasión ante él, porque vio que el Señor estaba con él, y que el Señor hacía prosperar todo lo que hacía.
\par 13 Y entregó todo en sus manos, y el jefe de los guardias de la prisión no sabía nada de lo que tenía, porque José hacía todo, y el Señor lo perfeccionó.
\par 14 Y permaneció allí dos años. Y en aquellos días Faraón rey de Egipto se enojó contra sus dos eunucos, contra el jefe de los mayordomos y contra el jefe de los panaderos, y los puso bajo custodia en la casa del jefe de cocina, en la prisión donde estaba José.
\par 15 Y el jefe de los guardias de la prisión nombró a José para que les sirviera; y sirvió delante de ellos.
\par 16 Y ambos tuvieron un sueño, el jefe de los coperos y el jefe de los panaderos, y se lo contaron a José.
\par 17 Y tal como les interpretó, así les sucedió, y Faraón restituyó al jefe de los coperos en su puesto y mató al (jefe) panadero, tal como José les había interpretado.
\par 18 Pero el jefe de los mayordomos se olvidó de José en la cárcel, aunque le había informado de lo que le sucedería, y no se acordó de informar a Faraón lo que José le había contado, porque se olvidó.

\chapter{40}

\par \textit{Los sueños del Faraón y su interpretación, 1-4. Elevación y matrimonio de José, 5-13. (Cf. Gen. xli.1-5, 7-9, 14 ss., 25, 29-30, 34, 36, 38-43, 45-6, 49.)}

\par 1 En aquellos días Faraón tuvo dos sueños en una noche acerca del hambre que había en todo el país, y despertó de su sueño y llamó a todos los intérpretes de sueños que había en Egipto y a los magos, y les contó sus dos sueños, y no pudieron contarlos.
\par 2 Entonces el jefe de los mayordomos se acordó de José y habló de él al rey, quien lo sacó de la prisión y le contó sus dos sueños delante de él.
\par 3 Y dijo ante Faraón que sus dos sueños eran uno, y le dijo: «Vendrán siete años (en los cuales habrá) abundancia en toda la tierra de Egipto, y después siete años de hambre, tales hambre como nunca la hubo en toda la tierra.
\par 4 'Y ahora que Faraón nombre supervisores en toda la tierra de Egipto, y almacene alimentos en cada ciudad para los días de los años de abundancia, y habrá alimentos para los siete años de hambre, y la tierra No perecerá por el hambre, porque será muy severa.'
\par 5 Y el Señor le dio a José favor y misericordia ante los ojos de Faraón, y Faraón dijo a sus siervos. «No encontraremos un hombre tan sabio y discreto como este, porque el espíritu del Señor está con él».
\par 6 Y lo nombró segundo en todo su reino, le dio autoridad sobre todo Egipto y lo hizo montar en el segundo carro de Faraón.
\par 7 Y lo vistió con ropas de biso, y le puso una cadena de oro en el cuello, y (un heraldo) proclamó delante de él 'El 'El wa' Abirer, y puso un anillo en su mano y lo hizo gobernante sobre todo. su casa, y le engrandecieron, y le dijeron. «Sólo en el trono seré mayor que tú».
\par 8 Y José gobernaba sobre toda la tierra de Egipto, y todos los príncipes de Faraón, y todos sus siervos, y todos los que hacían los negocios del rey lo amaban, porque caminaba con rectitud, porque no tenía orgullo ni arrogancia, y No respetaba a las personas ni aceptaba regalos, sino que juzgaba con rectitud a todo el pueblo de la tierra.
\par 9 Y la tierra de Egipto estaba en paz delante de Faraón a causa de José, porque el Señor estaba con él, y le había dado favor y misericordia para todas sus generaciones antes que todos los que lo conocían y los que oían acerca de él y del reino de Faraón. estaba bien ordenado, y no había ningún Satanás ni ninguna persona malvada (en él).
\par 10 Y el rey llamó a José Sefantifán, y le dio por mujer a la hija de Potifar, hija del sacerdote de Heliópolis, jefe de la cocina.
\par 11 Y el día que José se presentó ante Faraón, tenía treinta años.
\par 12 Y en aquel año murió Isaac. Y aconteció que como José había dicho en la interpretación de sus dos sueños, según lo había dicho, fueron siete años de abundancia en toda la tierra de Egipto, y la tierra de Egipto produjo abundantemente, una medida (produciendo) mil ochocientas medidas.
\par 13 Y José recogió alimentos en cada ciudad hasta que se llenaron de trigo, hasta el punto de que ya no podían contarlo ni medirlo según su multitud.

\chapter{41}

\par \textit{Los hijos de Judá y Tamar, 1-7. El incesto de Judá con Tamar, 8-18. Tamar tiene gemelos, 21-2. Judá perdonado, porque pecó por ignorancia y se arrepintió cuando fue condenado, y porque el matrimonio de Tamar con sus hijos no había sido consumado, 23-8. (Cf. Gen. xxxviii.6-18, 20-6, 29-30; xli.13.)}

\par 1 Y en el jubileo cuadragésimo quinto, en el segundo septenario del segundo año, Judá tomó para su primogénita a Er, una mujer de las hijas de Aram, llamada Tamar.
\par 2 Pero él la odiaba y no se acostaba con ella, porque su madre era de las hijas de Canaán, y quería tomar para él una esposa de los parientes de su madre, pero Judá, su padre, no se lo permitía.
\par 3 Y este Er, el primogénito de Judá, era malvado y el Señor lo mató.
\par 4 Entonces Judá dijo a su hermano Onán: «Ve a la mujer de tu hermano y cumple con ella el deber de hermano de tu marido, y levanta descendencia a tu hermano».
\par 5 Y sabiendo Onán que la semilla no sería suya, sino sólo de su hermano, entró en la casa de la mujer de su hermano y derramó la semilla en la tierra, y fue malvado ante los ojos del Señor. , y lo mató.
\par 6 Y Judá dijo a Tamar su nuera: «Quédate viuda en casa de tu padre hasta que crezca Sela mi hijo, y yo te daré a él por mujer».
\par 7 Y creció; pero Bedsuel, la esposa de Judá, no permitió que su hijo Sela se casara. Y Bedsu'el, mujer de Judá, murió [2168 a. m.] en el año quinto de esta semana.
\par 8 Y en el sexto año, Judá subió a Timnat a trasquilar sus ovejas. [2169 AM] Y dijeron a Tamar: 'He aquí, tu suegro sube a Timnat a esquilar sus ovejas.'
\par 9 Entonces se quitó sus ropas de viuda, se puso un velo, se atavió y se sentó a la puerta que está junto al camino de Timná.
\par 10 Y yendo Judá, la encontró y, tomándola por ramera, le dijo: «Déjame entrar a ti». y ella le dijo: «Entra», y él entró.
\par 11 Y ella le dijo: «Dame mi salario»; y él le dijo: 'No tengo nada en mi mano excepto mi anillo que está en mi dedo, y mi collar, y mi bastón que está en mi mano.'
\par 12 Y ella le dijo: 'Dámelos hasta que me envíes mi salario'. Y él le dijo: 'Te enviaré un cabrito de cabra'; y él se los dio, \textit{y entró a ella}, y ella concibió de él.
\par 13 Entonces Judá volvió a sus ovejas, y ella a la casa de su padre.
\par 14 Entonces Judá envió un cabrito de las cabras por mano de su pastor adullamita, y no la encontró; y preguntó a la gente del lugar, diciendo: '¿Dónde está la ramera que estaba aquí?' Y le dijeron; «Aquí no hay ninguna ramera entre nosotros».
\par 15 Y él regresó y le informó, y le dijo que no la había encontrado: 'Pregunté a la gente del lugar, y me dijeron: «Aquí no hay ninguna ramera». '
\par 16 Y él dijo: «Que ella se quede con ellos, para que no seamos motivo de burla». Y cuando cumplió tres meses, se vio que estaba encinta, y avisaron a Judá, diciendo: 'He aquí, Tamar, tu nuera, está encinta de fornicación.'
\par 17 Entonces Judá fue a la casa de su padre y dijo a su padre y a sus hermanos: «Sáquenla y quemenla, porque ha hecho inmundicia en Israel».
\par 18 Y aconteció que cuando la sacaron para quemarla, ella envió a su suegro el anillo, el collar y el bastón, diciendo: 'Descubre de quién son estos, porque por él estoy con él'. niño.'
\par 19 Entonces Judá se dio cuenta y dijo: «Tamar es más justa que yo».
\par 20 'Y por tanto, no la quemen'. Por eso ella no fue entregada a Sela, y él no volvió a acercarse a ella.
\par 21 Y después dio a luz dos hijos, Pérez [2170 AM] y Zera, en el séptimo año de esta segunda semana.
\par 22 Y entonces se cumplieron los siete años de fecundidad de que José habló a Faraón.
\par 23 Y Judá reconoció que su acción era mala, porque se había acostado con su nuera, y lo consideró odioso ante sus ojos, y reconoció que había transgredido y extraviado, porque había descubierto el manto de su hijo, y éste comenzó a lamentarse y a suplicar ante el Señor a causa de su transgresión.
\par 24 Y le dijimos en sueños que le había sido perdonado, porque había suplicado mucho y se lamentó, y no volvió a cometerlo.
\par 25 Y recibió perdón porque se había apartado de su pecado y de su ignorancia, pues había transgredido mucho ante nuestro Dios; y a todo el que así obra, a todo aquel que se acuesta con su suegra, que le quemen en el fuego para que arda en él, porque hay sobre ellos inmundicia y contaminación, que le quemen con el fuego.
\par 26 Y manda a los hijos de Israel que no haya impureza entre ellos, porque cualquiera que se acuesta con su nuera o con su suegra ha cometido impureza; quemen con fuego al hombre que se acostó con ella, y también a la mujer, y él apartará de Israel la ira y el castigo.
\par 27 Y a Judá le dijimos que sus dos hijos no se habían acostado con ella, y por eso su descendencia estaba establecida para una segunda generación y no sería desarraigada.
\par 28 Porque, con sinceridad, había ido a buscar el castigo; es decir, según el juicio que Abraham había ordenado a sus hijos, Judá había tratado de quemarla en el fuego.

\chapter{42}

\par \textit{Debido a la hambruna, Jacob envía a sus hijos a Egipto por maíz, 1-4. José los reconoce y retiene a Simeón, y les exige que traigan a Benjamín cuando regresen, 5-12. A pesar de la desgana de Jacob, sus hijos llevan a Benjamín con ellos en su segundo viaje y son agasajados por José, 13-25. (Cf. Gen. xli.54, 56; xlii.7-9, 13, 17, 20, 24-5, 29-30, 34-8; xliii.1-2, 4-5, 8-9, 11 , 15, 23, 26, 29, 34; xliv. 1-2.)}

\par 1 Y en el primer año de la tercera semana del jubileo cuadragésimo quinto comenzó el hambre en la tierra [2171 AM], y la lluvia no quiso caer sobre la tierra, porque no cayó nada.
\par 2 Y la tierra se volvió estéril, pero en la tierra de Egipto había comida, porque José había recogido la semilla de la tierra en los siete años de abundancia y la había conservado.
\par 3 Y los egipcios vinieron a José para que les diera comida, y él abrió los almacenes donde estaba el grano del primer año, y lo vendió a la gente de la tierra por oro.
\par 4 (El hambre era muy grave en la tierra de Canaán), y Jacob oyó que había comida en Egipto, y envió a sus diez hijos para que le procuraran comida en Egipto; pero a Benjamín no envió, y (los diez hijos de Jacob) llegaron (a Egipto) entre los que fueron (allí).
\par 5 Y José los reconoció, pero ellos no lo reconocieron, y él les habló y les preguntó, y les dijo: '¿No sois espías y no habéis venido a explorar los accesos a la tierra? Y los puso bajo custodia.
\par 6 Después de esto los liberó nuevamente, detuvo solo a Simeón y despidió a sus nueve hermanos.
\par 7 Y llenó sus costales de grano y puso en ellos el oro, y ellos no lo supieron.
\par 8 Y les ordenó que trajeran a su hermano menor, porque le habían dicho que su padre y su hermano menor vivían.
\par 9 Y subieron de la tierra de Egipto y llegaron a la tierra de Canaán; y contaron a su padre todo lo que les había sucedido, y cómo el señor del país les había hablado duramente y había prendido a Simeón hasta que trajeran a Benjamín.
\par 10 Y Jacob dijo: '¡Me habéis privado de mis hijos! José no existe, y Simeón tampoco, y a Benjamín os llevaréis. Sobre mí ha venido tu maldad.
\par 11 Y él dijo: 'Mi hijo no bajará con vosotros, no sea que caiga enfermo; porque su madre dio a luz dos hijos, y uno murió, y éste también me lo quitaréis. Si acaso cogiera fiebre en el camino, llevaréis mi vejez con dolor hasta la muerte.
\par 12 Porque vio que el dinero había sido devuelto a cada uno de los hombres que estaban en su costal, y por eso temió enviarlo.
\par 13 Y el hambre aumentó y se agravó en la tierra de Canaán y en todas las tierras excepto en la tierra de Egipto, porque muchos de los hijos de los egipcios habían almacenado su semilla para comer desde el momento en que vieron a José recolectar. juntar la semilla y guardarla en almacenes y conservarla para los años de hambre.
\par 14 Y los egipcios se alimentaron de ello durante el primer año de su hambre.
\par 15 Pero cuando Israel vio que el hambre era muy grave en la tierra y que no había salvación, dijo a sus hijos: «Volved y procuradnos comida para que no muramos».
\par 16 Y ellos dijeron: 'No iremos; A menos que nuestro hermano menor vaya con nosotros, no iremos.'
\par 17 Y vio Israel que si no lo enviaba con ellos, todos perecerían a causa del hambre.
\par 18 Y Rubén dijo: «Démelo en mis manos, y si no te lo devuelvo, mata a mis dos hijos en lugar de su alma».
\par 19 Y él le dijo: «Él no irá contigo». Y se acercó Judá y dijo: 'Envíalo conmigo, y si no te lo traigo de vuelta, yo llevaré la culpa delante de ti todos los días de mi vida.'
\par 20 Y lo envió con ellos en el segundo año de esta semana, el [2172 Am] primer día del mes, y llegaron a la tierra de Egipto con todos los que fueron, y (tenían) regalos en sus manos, estacte, almendras, terebintos y miel pura.
\par 21 Y fueron y se presentaron ante José, y él vio a Benjamín su hermano, y lo reconoció, y les dijo: «¿Es éste vuestro hermano menor?» Y ellos le dijeron: 'Es él'. Y él dijo: '¡El Señor tenga misericordia de ti, hijo mío!'
\par 22 Y lo envió a su casa y les trajo a Simeón y les hizo un banquete, y ellos le presentaron el presente que habían traído en sus manos.
\par 23 Y comieron delante de él y él les dio a todos una porción, pero la porción de Benjamín era siete veces mayor que la de cualquiera de ellos.
\par 24 Y comieron y bebieron, se levantaron y se quedaron con sus asnos.
\par 25 Y José ideó un plan mediante el cual podría conocer sus pensamientos sobre si los pensamientos de paz prevalecían entre ellos, y le dijo al mayordomo que estaba a cargo de su casa: 'Llena todos sus costales con comida y devuélveles su dinero. en sus vasijas, y mi copa, la copa de plata en la que bebo, métela en el costal de los más jóvenes y despídelos.'

\chapter{43}

\par \textit{El plan de José para seguir siendo sus hermanos, 1-10. La súplica de Judá, 11-13. José se da a conocer a sus hermanos y los envía de regreso a buscar a su padre, 14-24. (Cf. Gen. xliv.3-10, 12-18, 27-8, 30-2; xlv.1-2, 5-9, 12, 18, 20-1, 23, 25-8.)}

\par 1 E hizo lo que José le había dicho: llenó todos sus costales con comida, puso el dinero en ellos y puso la copa en el costal de Benjamín.
\par 2 Y muy temprano en la mañana partieron, y aconteció que cuando se habían ido de allí, José dijo al mayordomo de su casa: «Persíguelos, corre y apresálos, diciendo: »Para bien tenéis Me has pagado con mal; me has robado la copa de plata en la que bebe mi señor. Y tráiganme a su hermano menor, y tráiganlo rápidamente antes de que vaya a mi tribunal.'
\par 3 Y corrió tras ellos y les dijo conforme a estas palabras.
\par 4 Y ellos le dijeron: «Dios no permita que tus siervos hagan esto, y roben de la casa de tu señor cualquier utensilio, y también el dinero que encontramos en nuestros costales la primera vez, tus siervos lo trajimos de vuelta». de la tierra de Canaán.'
\par 5 '¿Cómo entonces vamos a robar cualquier utensilio? He aquí, aquí estamos nosotros y nuestros costales, y dondequiera que encuentres la copa en el costal de cualquier hombre entre nosotros, que lo maten, y nosotros y nuestros asnos serviremos a tu señor.
\par 6 Y él les dijo: «No así: al hombre con quien me encuentre, a él sólo lo tomaré como siervo, y vosotros volveréis en paz a vuestra casa».
\par 7 Y mientras buscaba en sus vasos, comenzando por el mayor y terminando por el menor, lo encontró en el costal de Benjamín.
\par 8 Y rasgaron sus vestidos, cargaron sus asnos, regresaron a la ciudad y llegaron a la casa de José, y todos se postraron rostro en tierra delante de él.
\par 9 Y José les dijo: «Habéis hecho lo malo». Y dijeron: '¿Qué diremos y cómo nos defenderemos? Nuestro señor ha descubierto la transgresión de sus siervos; he aquí que somos siervos de nuestro señor, y también nuestras asnas.'
\par 10 Y José les dijo: 'Yo también temo al Señor; En cuanto a vosotros, id a vuestras casas y dejad que vuestro hermano sea mi siervo, porque habéis hecho lo malo. ¿No sabéis que el hombre se deleita con su copa como yo con esta copa? Y, sin embargo, me lo habéis robado.
\par 11 Y Judá dijo: «Señor mío, te ruego que tu sierva hable una palabra al oído de mi señor: los dos hermanos que la madre de tu sierva dio a luz a nuestro padre; uno se fue y se perdió y no ha sido encontrado». , y él solo queda de su madre, y tu siervo nuestro padre lo ama, y ​​su vida también está ligada a la vida de este (muchacho).'
\par 12 «Y sucederá que cuando vayamos a tu siervo, nuestro padre, y el muchacho no esté con nosotros, morirá, y llevaremos a nuestro padre a la muerte con dolor».
\par 13 «Ahora mejor, déjame a mí, tu siervo, quedarme en lugar del niño como siervo de mi señor, y dejar que el muchacho se vaya con sus hermanos, porque yo fui fiador de él por mano de tu siervo nuestro padre, y si Si no lo hago volver, tu siervo oirá la culpa contra nuestro padre para siempre.'
\par 14 Y José vio que todos eran conformes en bondad unos con otros, y no pudo contenerse, y les dijo que él era José.
\par 15 Y habló con ellos en lengua hebrea y se echó sobre sus cuellos y lloró.
\par 16 Pero ellos no lo reconocieron y comenzaron a llorar. Y él les dijo: 'No lloréis por mí, sino apresuraos y traedme a mi padre; y veis que es mi boca la que habla y los ojos de mi hermano Benjamín ven.'
\par 17 «Porque he aquí el segundo año de hambre, y aún quedan cinco años sin cosecha, ni fruto de los árboles, ni arado».
\par 18 «Desciendan rápidamente ustedes y sus familias, para que no perezcan de hambre ni se aflijan por sus bienes, porque el Señor me envió delante de ustedes para arreglar las cosas de manera que mucha gente pudiera vivir».
\par 19 «Y dile a mi padre que todavía estoy vivo, y veréis que el Señor me ha puesto por padre para Faraón y gobernante de su casa y de toda la tierra de Egipto».
\par 20 «Y cuéntale a mi padre toda mi gloria y todas las riquezas y la gloria que el Señor me ha dado».
\par 21 Y por orden de boca de Faraón, les dio carros y provisiones para el camino, y les dio a todos vestidos multicolores y plata.
\par 22 Y envió a su padre vestidos y plata y diez asnos que transportaban trigo, y los despidió.
\par 23 Y subieron y dijeron a su padre que José estaba vivo y que estaba midiendo trigo para todas las naciones de la tierra, y que él era gobernante de toda la tierra de Egipto.
\par 24 Pero su padre no lo creía, porque estaba fuera de sí; pero cuando vio los carros que José había enviado, la vida de su espíritu revivió y dijo: 'Me basta con que José viva; Bajaré a verlo antes de morir.

\chapter{44}

\par \textit{Jacob celebra la fiesta de las primicias y, animado por una visión, desciende a Egipto, 1-10. Nombres de sus descendientes, 11-34. (Cf. Gen. xlvi.1-28.)}

\par 1 Israel partió de Harán desde su casa en la luna nueva del tercer mes, se dirigió al camino del Pozo del Juramento y ofreció un sacrificio al Dios de su padre Isaac el día séptimo. de este mes.
\par 2 Entonces Jacob se acordó del sueño que había visto en Betel y tuvo miedo de bajar a Egipto.
\par 3 Y mientras pensaba en avisar a José para que viniera a él, y que no bajaría, permaneció allí siete días, por si acaso podía tener una visión que le dijera si debía quedarse o bajar.
\par 4 Y celebró la fiesta de la cosecha de las primicias con el grano viejo, porque en toda la tierra de Canaán no había ni un puñado de semillas, porque el hambre se apoderaba de todos los animales, ganado y aves. , y también sobre el hombre.
\par 5 Y el día dieciséis se le apareció el Señor y le dijo: «Jacob, Jacob». y él dijo: 'Aquí estoy'. Y le dijo: 'Yo soy el Dios de tus padres, el Dios de Abraham y de Isaac; No temas descender a Egipto, porque allí haré de ti una gran nación, descenderé contigo y te haré subir (otra vez), y en esta tierra serás sepultado, y José pondrá sus manos. sobre tus ojos.'
\par 6 'No temáis; baja a Egipto.'
\par 7 Entonces sus hijos y los hijos de sus hijos se levantaron y colocaron a su padre y sus bienes en carros.
\par 8 E Israel se levantó del pozo del Juramento el día dieciséis de este tercer mes y se dirigió a la tierra de Egipto.
\par 9 E Israel envió a Judá delante de él a su hijo José para que explorara la tierra de Goshen, porque José había dicho a sus hermanos que debían venir y habitar allí para estar cerca de él.
\par 10 Y ésta era la tierra más hermosa de la tierra de Egipto, y cercana a él, para todos y también para el ganado.
\par 11 Estos son los nombres de los hijos de Jacob que fueron a Egipto con su padre Jacob.
\par 12 Rubén, el primogénito de Israel; y estos son los nombres de sus hijos Enoc, Pallu, Hezrón y Carmicinco.
\par 13 Simeón y sus hijos; Y estos son los nombres de sus hijos: Jemuel, Jamín, Ohad, Jaquín, Zohar y Saúl, hijo de la mujer sefatita, siete.
\par 14 Leví y sus hijos; Y estos son los nombres de sus hijos: Gersón, Coat y Merari, cuatro.
\par 15 Judá y sus hijos; y estos son los nombres de sus hijos: Sela, Pérez y Zera, cuatro.
\par 16 Isacar y sus hijos; y estos son los nombres de sus hijos: Tola, Phua, Jasub y Simrón cinco.
\par 17 Zabulón y sus hijos; y estos son los nombres de sus hijos: Sered, Elón y Jahleel, cuatro.
\par 18 Estos son los hijos de Jacob y los hijos que Lea le dio a Jacob en Mesopotamia: seis, y su única hermana, Dina, y todas las almas de los hijos de Lea y sus hijos, que fueron con su padre Jacob a Egipto, eran veintinueve, y estando Jacob su padre con ellos, eran treinta.
\par 19 Y los hijos de Zilpa, sierva de Lea, esposa de Jacob, quien le dio a Jacob Gad y Asur.
\par 20 Estos son los nombres de sus hijos que fueron con él a Egipto. Los hijos de Gad: Zifion, Hagui, Suni, Ezbón, Eri, Areli y Arodi, ocho.
\par 21 Los hijos de Aser fueron Imná, Isvá, Beriá y Seraj, su única hermana, seis.
\par 22 Todas las personas fueron catorce, y todos los de Lea, cuarenta y cuatro.
\par 23 Los hijos de Raquel, mujer de Jacob: José y Benjamín.
\par 24 Y a José le nacieron en Egipto, antes de que su padre llegara a Egipto, los que le dio a luz Asenat, hija de Potifar, sacerdote de Heliópolis: Manasés y Efraín, tres.
\par 25 Los hijos de Benjamín: Bela, Bequer, Asbel, Gera, Naamán, Ehi, Rosh, Muppim, Huppim y Ard-once.
\par 26 Y el total de almas de Raquel fueron catorce.
\par 27 Y los hijos de Bilha, la sierva de Raquel, la esposa de Jacob, que ella le dio a Jacob, fueron Dan y Neftalí.
\par 28 Estos son los nombres de sus hijos que fueron con ellos a Egipto. Y los hijos de Dan fueron Hushim, Samón y Asudi. y 'Ijaka, y Salomon-seis.
\par 29 Y murieron el año en que entraron en Egipto, y Dan Hushim quedó solo.
\par 30 Estos son los nombres de los hijos de Neftalí, Jahziel, Guni, Jezer, Salum e Iv.
\par 31 Y 'Iv, que nació después de los años de hambre, murió en Egipto.
\par 32 Y todas las almas de Raquel fueron veintiséis.
\par 33 Y todas las personas de Jacob que fueron a Egipto fueron setenta personas. Estos son sus hijos y los hijos de sus hijos, en total setenta, pero cinco murieron en Egipto antes que José y no tuvieron hijos.
\par 34 Y en la tierra de Canaán murieron dos hijos de Judá, Er y Onán, y no tuvieron hijos; y los hijos de Israel enterraron a los que perecieron, y fueron contados entre las setenta naciones gentiles.

\chapter{45}

\par \textit{José recibe a Jacob y le entrega Gosén, 1-7. José adquiere toda la tierra y sus habitantes para Faraón, 8-12. Jacob muere y es sepultado en Hebrón, 13-15. Sus libros entregados a Leví, 16. (Cf. Gen. xlvi.28-30; xlvii.11-13, 19, 20, 23, 24, 28; l.13.)}

\par 1 E Israel entró en el país de Egipto, en la tierra de Gosén, en la luna nueva del cuarto día. mes, en el año segundo de la tercera semana del jubileo cuadragésimo quinto.
\par 2 Y José fue al encuentro de su padre Jacob, a la tierra de Gosén, y se echó sobre el cuello de su padre y lloró.
\par 3 E Israel dijo a José: «Ahora déjame morir, ya que te he visto, y ahora sea bendito el Señor Dios de Israel, el Dios de Abraham y el Dios de Isaac, que no ha negado su misericordia y su gracia de su siervo Jacob.'
\par 4 'Me basta con haber visto tu rostro mientras aún estoy vivo; sí, verdadera es la visión que tuve en Betel. Bendito sea el Señor mi Dios por los siglos de los siglos, y bendito sea su nombre.'
\par 5 Y José y sus hermanos comieron pan delante de su padre y bebieron vino, y Jacob se regocijó con gran alegría al ver a José comer con sus hermanos y beber delante de él, y bendijo al Creador de todas las cosas que lo había preservado. y le había reservado a sus doce hijos.
\par 6 Y José había dado como regalo a su padre y a sus hermanos el derecho de habitar en la tierra de Goshen y en Ramsés y en toda la región circundante, que él gobernaba delante de Faraón. E Israel y sus hijos habitaron en la tierra de Goshen, la mejor parte de la tierra de Egipto e Israel tenía ciento treinta años cuando entró en Egipto.
\par 7 Y José alimentó a su padre y a sus hermanos, y también a sus bienes, con pan suficiente para los siete años de hambre.
\par 8 Y la tierra de Egipto sufrió a causa del hambre, y José adquirió toda la tierra de Egipto para Faraón a cambio de alimentos, y se apoderó del pueblo y de su ganado y de todo para Faraón.
\par 9 Y cuando se cumplieron los años del hambre, José dio al pueblo en la tierra semillas y alimentos para que pudieran sembrar (la tierra) en el octavo año, porque el río había inundado toda la tierra de Egipto.
\par 10 Porque en los siete años de la hambruna no se había desbordado y sólo había irrigado algunos lugares a orillas del río, pero ahora se desbordó y los egipcios sembraron la tierra, y ese año produjo mucho grano.
\par 11 Y este fue el primer año de [2178 AM] la cuarta semana del jubileo cuadragésimo quinto.
\par 12 Y José tomó del grano de la cosecha la quinta parte para el rey, y les dejó cuatro partes para alimento y para semilla, y José lo promulgó como ordenanza para la tierra de Egipto hasta el día de hoy.
\par 13 Y vivió Israel en la tierra de Egipto diecisiete años, y todos los días que vivió fueron tres jubileos, ciento cuarenta y siete años, y murió en el año cuarto [2188 AM] del quinto septenario del cuadragésimo quinto jubileo.
\par 14 Israel bendijo a sus hijos antes de morir y les contó todo lo que les sucedería en la tierra de Egipto. y les hizo saber lo que les sobrevendría en los últimos días, y los bendijo y dio a José dos porciones de la tierra.
\par 15 Y durmió con sus padres y fue sepultado en la doble cueva en la tierra de Canaán, cerca de Abraham su padre, en la tumba que él cavó para sí en la doble cueva en la tierra de Hebrón.
\par 16 Y dio todos sus libros y los libros de sus padres a Leví su hijo para que los conservara y los renovara para sus hijos hasta el día de hoy.

\chapter{46}

\par \textit{Prosperidad de Israel en Egipto, 1-2. Muerte de José, 3-5. Guerra entre Egipto y Canaán durante la cual los huesos de todos los hijos de Jacob excepto José son enterrados en Hebrón, 6-11. Egipto oprime a Israel, 12-16. (Cf. Gen. l.22, 25-6; Éxodo. i.6-14.)}

\par 1 Y aconteció que después de la muerte de Jacob, los hijos de Israel se multiplicaron en la tierra de Egipto, y llegaron a ser una nación grande, y eran unánimes en corazón, de modo que el hermano amaba al hermano y cada uno ayudaba a su hermano. , y crecieron abundantemente y se multiplicaron en gran manera, diez [2242 AM] semanas de años, todos los días de la vida de José.
\par 2 Y no hubo Satanás ni ningún mal en todos los días de la vida de José que vivió después de su padre Jacob, porque todos los egipcios honraron a los hijos de Israel durante todos los días de la vida de José.
\par 3 Y José murió siendo de ciento diez años; Diecisiete años vivió en la tierra de Canaán, y diez años fue siervo, y tres años en prisión, y ochenta años estuvo bajo el rey, gobernando toda la tierra de Egipto.
\par 4 Y murió él y todos sus hermanos y toda aquella generación.
\par 5 Y antes de morir, ordenó a los hijos de Israel que llevaran consigo sus huesos cuando salieran de la tierra de Egipto.
\par 6 Y les hizo jurar sobre sus huesos, porque sabía que los egipcios no volverían a sacarlo y enterrarlo en la tierra de Canaán, porque Makamaron, rey de Canaán, mientras habitaba en la tierra de Asiria, peleó en el valle con el rey de Egipto y lo mataron allí, y persiguieron a los egipcios hasta las puertas de Ermón.
\par 7 Pero no pudo entrar, porque otro nuevo rey había llegado a ser rey de Egipto, y era más fuerte que él, y regresó a la tierra de Canaán, y las puertas de Egipto se cerraron y nadie salió y nadie entró en Egipto.
\par 8 Y José murió en el jubileo cuadragésimo sexto, en el sexto septenario, en el año segundo, y lo sepultaron en la tierra de Egipto, y [2242 AM] todos sus hermanos murieron después de él.
\par 9 Y el rey de Egipto salió a la guerra contra el rey de Canaán [2263 AM] en el jubileo cuadragésimo séptimo, en el segundo septenario del segundo año, y los hijos de Israel sacaron todos los huesos de los niños. de Jacob salvo los huesos de José, y los sepultaron en el campo, en la doble cueva del monte.
\par 10 Y la mayoría regresó a Egipto, pero unos pocos se quedaron en las montañas de Hebrón, y Amram tu padre se quedó con ellos.
\par 11 Y el rey de Canaán venció al rey de Egipto y cerró las puertas de Egipto.
\par 12 E ideó un malvado plan para afligir a los hijos de Israel, y dijo al pueblo de Egipto: «He aquí, el pueblo de los hijos de Israel ha aumentado y se ha multiplicado más que nosotros».
\par 13 'Venid y tratemos con ellos sabiamente antes de que se vuelvan demasiados, y los sometamos a esclavitud antes de que nos sobrevenga la guerra y antes de que ellos también luchen contra nosotros; de lo contrario, se unirán a nuestros enemigos y los sacarán de nuestra tierra, porque sus corazones y rostros están hacia la tierra de Canaán.'
\par 14 Y puso sobre ellos capataces para someterlos a esclavitud; y edificaron ciudades fuertes para Faraón, Pitón y Ramsés y edificaron todos los muros y todas las fortificaciones que habían caído en las ciudades de Egipto.
\par 15 Y los obligaron a servir con rigor, y cuanto más mal los trataban, más crecían y se multiplicaban.
\par 16 Y el pueblo de Egipto abominó a los hijos de Israel.

\chapter{47}

\par \textit{Nacimiento de Moisés, 1-4. Adoptado por la hija de Faraón, 5-9. Mata a un egipcio y huye (a Madián), 10-12. (Cf. Éxodo i.22; ii. 2-15.)}

\par 1 Y en el séptimo septenario, en el año séptimo, en el jubileo cuadragésimo séptimo, tu padre salió [2303 AM] de la tierra de Canaán, y tú naciste en el cuarto septenario, en su año sexto, en el [2330 AM] cuadragésimo octavo jubileo; este fue el tiempo de tribulación para los hijos de Israel.
\par 2 Y Faraón, rey de Egipto, les ordenó que arrojaran al río a todos sus hijos varones que nacieran.
\par 3 Y los arrojaron allí durante siete meses, hasta el día en que naciste.
\par 4 Y tu madre te escondió durante tres meses, y hablaron de ella. Y ella te hizo un arca, y la cubrió con brea y asfalto, y la puso en las losas a la orilla del río, y te puso en ella siete días, y vino tu madre de noche y te amamantó, y por Día Miriam, tu hermana, te guardó de las aves.
\par 5 Y en aquellos días Tarmut, la hija de Faraón, vino a bañarse en el río, y oyó tu voz llorando, y dijo a sus doncellas que te sacaran, y ellas te trajeron a ella.
\par 6 Y ella te sacó del arca y tuvo compasión de ti.
\par 7 Y tu hermana le dijo: «¿Quieres que vaya y llame a una de las mujeres hebreas para que te críe y amamante a este niño?»
\par 8 Y ella le dijo: 'Ve'. Y ella fue y llamó a tu madre Jocabed, y ella le dio su salario y te crió.
\par 9 Y después, cuando ya eras mayor, te llevaron a la hija de Faraón, y fuiste su hijo, y Amram tu padre te enseñó a escribir, y después de haber cumplido tres semanas te llevaron a la corte real.
\par 10 Y estuviste en la corte durante tres semanas de años, hasta el momento [2351-] en que saliste de la corte real y viste a un egipcio golpeando a tu amigo que era [2372 AM] de los hijos de Israel, y tú Lo mataste y lo escondiste en la arena.
\par 11 Y el segundo día, tú y dos de los hijos de Israel peleaban entre sí, y le dijiste al que estaba haciendo el mal: «¿Por qué golpeas a tu hermano?»
\par 12 Y él, enojado e indignado, dijo: «¿Quién te ha puesto por príncipe y juez sobre nosotros?» ¿Piensas matarme como mataste ayer al egipcio? Y tuviste miedo y huiste a causa de estas palabras.

\chapter{48}

\par \textit{Moisés regresa de Madián a Egipto. Mastêmâ busca matarlo en el camino, 1-3. Las diez plagas, 4-11. Israel sale de Egipto: la destrucción de los egipcios en el Mar Rojo, 12-19. (Cf. Éxodo ii.15; iv.19, 24; vii. seqq.)}

\par 1 Y en el año sexto del tercer septenario del jubileo cuadragésimo noveno, partiste y habitaste (en [2372 AM] la tierra de Madián), cinco semanas y un año. Y volviste a Egipto en el segundo septenario del segundo año del jubileo quincuagésimo.
\par 2 Y tú mismo sabes lo que Él te habló en el [2410 AM] Monte Sinaí, y lo que el príncipe Mastêmâ deseaba hacer contigo cuando regresabas a Egipto (en el camino cuando lo encontraste en el lugar de alojamiento).
\par 3 ¿No intentó él con todo su poder matarte y librar a los egipcios de tus manos, cuando vio que habías sido enviado para ejecutar juicio y venganza sobre los egipcios?
\par 4 Y yo te libré de su mano, y tú hiciste las señales y prodigios que fuiste enviado a hacer en Egipto contra Faraón, y contra toda su casa, y contra sus siervos y su pueblo.
\par 5 Y el Señor ejecutó sobre ellos una gran venganza por amor de Israel, y los hirió con (plagas de) sangre y ranas, piojos y moscas, y llagas malignas que brotaban en las úlceras; y su ganado por muerte; y con granizo destruyó todo lo que para ellos crecía; y por langostas que devoraron el residuo que había dejado el granizo y la oscuridad; y (por la muerte) de los primogénitos de los hombres y de los animales, y de todos sus ídolos el Señor se vengó y los quemó al fuego.
\par 6 Y todo fue enviado por tu mano para que declararas (estas cosas) antes de que se hicieran, y hablaste con el rey de Egipto delante de todos sus siervos y delante de su pueblo.
\par 7 Y todo sucedió según tus palabras; Diez juicios grandes y terribles vinieron sobre la tierra de Egipto, para que tú ejecutaras en ella venganza por Israel.
\par 8 Y el Señor hizo todo por amor de Israel, y según el pacto que había establecido con Abraham, de vengarse de ellos por haberlos sometido por la fuerza a servidumbre.
\par 9 Y el príncipe Mastêmâ se levantó contra ti y trató de arrojarte en manos de Faraón, y ayudó a los hechiceros egipcios,
\par 10 y ellos se levantaron y obraron ante ti los males que les permitimos hacer, pero los remedios que no permitimos que hicieran sus manos.
\par 11 Y el Señor los hirió con úlceras malignas y no podían sostenerse en pie, porque los destruimos para que no pudieran realizar ni una sola señal.
\par 12 Y a pesar de todas estas señales y prodigios, el príncipe Mastêmâ no se avergonzó porque se animó y gritó a los egipcios que te persiguieran con todas las fuerzas de los egipcios, con sus carros y sus caballos, y con todos los ejércitos de los pueblos de Egipto.
\par 13 Y yo me puse entre los egipcios e Israel, y libramos a Israel de su mano y de la mano de su pueblo, y el Señor los hizo pasar por en medio del mar como si fuera tierra seca.
\par 14 Y a todos los pueblos que él trajo para perseguir a Israel, el Señor nuestro Dios los arrojó en medio del mar, en las profundidades del abismo debajo de los hijos de Israel, así como el pueblo de Egipto había arrojado a sus hijos. en el río se vengó de un millón de ellos, y mil hombres fuertes y enérgicos fueron destruidos a causa de un niño de pecho de los hijos de tu pueblo que habían arrojado al río.
\par 15 Y el día catorce, el día quince, el dieciséis, el diecisiete y el decimoctavo, el príncipe Mastêmâ fue atado y encarcelado detrás de los hijos de Israel para no acusarlos.
\par 16 Y el día diecinueve los soltamos para que ayudaran a los egipcios y persiguieran a los hijos de Israel.
\par 17 Y él endureció sus corazones y los hizo tercos, y el Señor nuestro Dios ideó la idea de derrotar a los egipcios y arrojarlos al mar.
\par 18 Y el día catorce lo atamos para que no acusara a los hijos de Israel el día en que pidieron a los egipcios vasos y vestidos, vasos de plata, vasos de oro y vasos de bronce, para despojarlos. los egipcios a cambio de la esclavitud en la que los habían obligado a servir.
\par 19 Y no sacamos a los hijos de Israel de Egipto con las manos vacías.

\chapter{49}

\par \textit{La Pascua: normas relativas a su celebración. (Cf. Éxodo xii.6, 9, 11, 13, 22-3, 30, 46; xv.22.)}

\par 1 Acuérdate del mandamiento que el Señor te mandó acerca de la Pascua: que la celebres en su tiempo, el día catorce del primer mes, que la mates antes de que llegue la tarde y que la coman por la noche del día 1. la tarde del día quince desde la hora de la puesta del sol.
\par 2 Porque en esta noche, el comienzo de la fiesta y el comienzo de la alegría, estabais comiendo la pascua en Egipto, cuando todos los poderes de Mastêmâ se habían desatado para matar a todos los primogénitos en la tierra de Egipto. , desde el primogénito de Faraón hasta el primogénito de la sierva cautiva en el molino, y hasta el ganado.
\par 3 Y esta es la señal que les dio el Señor: En cada casa en cuyos dinteles vieran la sangre de un cordero de un año, no entrarían a matar, sino que pasarían. (it), para que se salvaran todos los que estaban en la casa porque la señal de la sangre estaba en sus dinteles.
\par 4 Y los poderes del Señor hicieron todo tal como el Señor les había ordenado, y pasaron por alto a todos los hijos de Israel, y la plaga no cayó sobre ellos para destruir ninguna alma de entre ellos, ni de ganado, ni de hombre, ni perro.
\par 5 Y la plaga fue muy grave en Egipto, y no había casa en Egipto donde no hubiera un solo muerto, y llanto y lamento.
\par 6 Y todo Israel comía la carne del cordero pascual y bebía el vino, alababa, bendicía y daba gracias al Señor, Dios de sus padres, y se disponía a salir del yugo de Egipto. , y de la malvada servidumbre.
\par 7 Y recuerda este día todos los días de tu vida, y obsérvalo de año en año todos los días de tu vida, una vez al año, en su día, conforme a toda su ley, y no lo pospongas. ) de día a día o de mes a mes.
\par 8 Porque es un precepto eterno, grabado en las tablas celestiales, para todos los hijos de Israel que lo observen cada año en su día una vez al año, a lo largo de todas sus generaciones; y no hay límite de días, porque esto está ordenado para siempre.
\par 9 Y el hombre que está libre de impureza y no viene a observarla en su día, para traer una ofrenda aceptable delante del Señor, y comer y beber delante del Señor en el día de su festival, el hombre limpio y cercano será cortado; por no haber ofrecido la ofrenda del Señor en su tiempo señalado, cargará sobre sí mismo la culpa.
\par 10 Vengan los hijos de Israel y celebren la Pascua en el día de su fecha fijada, el día catorce del mes primero, entre las tardes, desde la tercera parte del día hasta la tercera parte de la noche, porque dos partes del día se dedican a la luz y una tercera parte a la tarde.
\par 11 Esto es lo que el Señor te ordenó que lo observaras entre las tardes.
\par 12 Y no está permitido sacrificarlo durante ningún período de luz, sino al atardecer, y comerlo al atardecer, hasta la tercera parte de la noche, y lo que sea necesario. sobrante de toda su carne desde la tercera parte de la noche en adelante, que la quemen al fuego.
\par 13 Y no lo cocerán con agua, ni lo comerán crudo, sino asado al fuego; lo comerán con diligencia; su cabeza con sus entrañas y sus pies asarán al fuego, y no romperán. cualquier hueso del mismo; porque de los hijos de Israel ningún hueso será quebrado.
\par 14 Por esta razón el Señor ordenó a los hijos de Israel que celebraran la Pascua en el día de su tiempo señalado, y que no les quebraran ni un solo hueso; porque es día festivo, y día prescrito, y no se podrá pasar de día en día, ni de mes en mes, sino que se observará en el día de su fiesta.
\par 15 Y ordenas a los hijos de Israel que celebren la Pascua durante todos sus días, cada año, una vez al año en el día de su fecha fijada, y será un memorial agradable delante del Señor, y no habrá plaga. ven sobre ellos para matar o herir en ese año en que celebran la pascua en su tiempo en todos los aspectos según Su mandato.
\par 16 Y no lo comerán fuera del santuario del Señor, sino delante del santuario del Señor, y todo el pueblo de la congregación de Israel lo celebrará en su tiempo señalado.
\par 17 Y todo hombre que llegue en su día, de veinte años arriba, lo comerá en el santuario de vuestro Dios delante del Señor; porque así está escrito y ordenado que lo coman en el santuario del Señor.
\par 18 Y cuando los hijos de Israel entren en la tierra que van a poseer, en la tierra de Canaán, y erijan el tabernáculo del Señor en medio de la tierra en una de sus tribus hasta el santuario del Señor ha sido edificada en la tierra, vengan y celebren la pascua en medio del tabernáculo del Señor, y celebrenla delante del Señor de año en año.
\par 19 Y en los días en que se edifique la casa en el nombre del Señor en la tierra de su heredad, irán allí y sacrificarán la pascua por la tarde, al ponerse el sol, a la tercera parte del día.
\par 20 Y ofrecerán su sangre en el umbral del altar, y pondrán su grasa sobre el fuego que está sobre el altar, y comerán su carne asada al fuego en el atrio de la casa santificada en el nombre del Señor.
\par 21 Y no podrán celebrar la Pascua en sus ciudades ni en ningún lugar excepto delante del tabernáculo del Señor o delante de su casa, donde habitó su nombre; y no se extraviarán del Señor.
\par 22 Y tú, Moisés, ordena a los hijos de Israel que observen las ordenanzas de la pascua, tal como te fue ordenado a ti; Diles cada año y el día de sus días, y la fiesta de los panes sin levadura, que coman panes sin levadura durante siete días, (y) que observen su fiesta, y que traigan una ofrenda cada día durante esos siete días. días de alegría delante del Señor en el altar de tu Dios.
\par 23 Porque celebrasteis esta fiesta apresuradamente desde que salisteis de Egipto hasta que entrasteis en el desierto de Shur; porque en la orilla del mar lo completasteis.

\chapter{50}

\par \textit{Leyes relativas a los jubileos, 1-5, y al sábado, 6-13.}

\par 1 Y después de esta ley te hice saber los días de los sábados en el desierto de Sinai, que está entre Elim y Sinaí.
\par 2 Y os he hablado de los sábados de la tierra en el monte Sinaí, y de los años del jubileo en sábados de años; pero no os he dicho el año en cuestión hasta que entréis en la tierra que vais a poseer.
\par 3 Y la tierra también guardará sus sábados mientras habiten en ella, y sabrán el año del jubileo.
\par 4 Por eso te he ordenado los años, las semanas, los años y los jubileos: hay cuarenta y nueve jubileos desde los días de Adán hasta el día de hoy, y una semana y dos años; cuarenta años por venir (lit. 'distantes') para aprender los mandamientos [2450 AM] del Señor, hasta que pasen a la tierra de Canaán, cruzando el Jordán hacia el oeste.
\par 5 Y los jubileos pasarán hasta que Israel sea limpiado de toda culpa de fornicación, inmundicia, contaminación, pecado y error, y habite con confianza en toda la tierra, y ya no habrá más Satanás ni cualquier maligno, y la tierra será limpia desde entonces para siempre.
\par 6 Y he aquí el mandamiento sobre los sábados (te los he escrito) y todos los preceptos de sus leyes.
\par 7 Seis días trabajarás, pero el séptimo día es sábado para el Señor tu Dios. En ella ningún trabajo haréis, ni vosotros, ni vuestros hijos, ni vuestros siervos, ni vuestras siervas, ni todo vuestro ganado, ni el extranjero que está con vosotros.
\par 8 Y el hombre que trabaje en él, morirá: cualquiera que profane ese día, cualquiera que se acueste con (su) esposa, o cualquiera que diga que hará algo en él, emprenderá un viaje por él con respecto a cualquier compra o venta; y cualquiera que saque de ella agua que no haya preparado para sí en el sexto día, y cualquiera que tome cualquier carga para sacarla de su tienda o de su casa, morirá.
\par 9 Ningún trabajo haréis en el día de reposo, excepto lo que os habéis preparado para el sexto día, para comer, beber y descansar, y guardar el sábado de todo trabajo en ese día, y bendecir el Señor tu Dios, que te ha dado día de fiesta y día santo; y día del reino santo para todo Israel es hoy entre sus días para siempre.
\par 10 Porque grande es el honor que el Señor ha dado a Israel de que puedan comer y beber y quedar saciados en este día festivo, y descansar en él de todo trabajo que pertenece al trabajo de los hijos de los hombres, excepto quemar incienso y traer oblaciones y sacrificios delante del Señor por días y sábados.
\par 11 Sólo esta obra se hará en los días de reposo en el santuario del Señor tu Dios; para que puedan expiar a Israel con sacrificio continuo de día en día, como memoria agradable delante de Jehová, y para que Él los reciba siempre de día en día, como te ha sido mandado.
\par 12 Y todo el que trabaja en él, sale de viaje o labra su granja, ya sea en su casa o en cualquier otro lugar, enciende fuego, monta algún animal o viaja en barco. el mar, y cualquiera que golpee o mate algo, o degüelle un animal o un pájaro, o cualquiera que capture un animal o un pájaro o un pez, o cualquiera que ayune o haga la guerra en sábado:
\par 13 El hombre que haga cualquiera de estas cosas en sábado morirá, de modo que los hijos de Israel observarán los sábados conforme a los mandamientos relativos a los sábados de la tierra, como está escrito en las tablas que Él dio en mis manos para escribirte las leyes de las estaciones, y las estaciones según la división de sus días.
\par    
\par     Aquí se completa el relato de la división de los días.

\end{document}