\begin{document}

\title{Testamento de Abraham}

\part{Versión 1}

\chapter{1}

\par 1 Abraham vivió la medida de su vida, novecientos noventa y cinco años, y habiendo vivido todos los años de su vida en tranquilidad, mansedumbre y rectitud, el justo fue sumamente hospitalario;

\par 2 Porque, plantando su tienda en el cruce de caminos junto a la encina de Mamre, recibió a todos, ricos y pobres, reyes y gobernantes, mancos y desvalidos, amigos y extraños, vecinos y viajeros, todos por igual. el devoto, todo santo, justo y hospitalario Abraham entretiene.

\par 3 Pero también a él le sobrevino la común, inexorable y amarga suerte de la muerte y el incierto fin de la vida.

\par 4 Entonces el Señor Dios, llamando a su arcángel Miguel, le dijo: «Desciende, capitán Miguel, a donde Abraham y habla con él acerca de su muerte, para que pueda arreglar sus asuntos,

\par 5 Porque lo he bendecido como las estrellas del cielo y como la arena a la orilla del mar, y gozará de larga vida y de muchos bienes, y se enriquecerá en gran manera. Más allá de todos los hombres, además, es justo en toda bondad, hospitalario y amoroso hasta el fin de su vida;

\par 6 pero ve, arcángel Miguel, a Abraham, mi amado amigo, y anúnciale su muerte y asegúrale así:

\par 7 En este momento partirás de este mundo vano, dejarás el cuerpo e irás a tu Señor entre los buenos.

\chapter{2}

\par 1 Y el capitán se alejó de la presencia de Dios y descendió hacia Abraham, al encinar de Mamre, y encontró al justo Abraham en el campo cercano, sentado junto a las yuntas de bueyes que araban, junto con sus hijos. de Masek y otros sirvientes, en número de doce.

\par 2 Y he aquí que el capitán se acercó a él, y Abraham, viendo al capitán Miguel venir de lejos, como un guerrero muy hermoso, se levantó y salió a su encuentro como era su costumbre, encontrando y entreteniendo a todos los extranjeros.

\par 3 Y el capitán en jefe lo saludó y dijo: «Salve, honorable padre, alma justa elegida de Dios, verdadero hijo del celestial».

\par 4 Abraham dijo al capitán en jefe: «Salve, guerrero muy honorable, brillante como el sol y más hermoso que todos los hijos de los hombres; de nada;

\par 5 Por tanto, ruego a tu presencia que me digas de dónde ha venido el joven de tu edad; Enséñame, tu suplicante, de dónde y de qué ejército y de qué viaje ha venido hasta aquí tu belleza.

\par 6 El capitán dijo: «Yo, el justo Abraham, vengo de la gran ciudad. El gran rey me ha enviado para ocupar el lugar de un buen amigo suyo, porque el rey lo ha convocado».

\par 7 Y Abraham dijo: «Ven, Señor mío, ven conmigo hasta mi campo». El capitán mayor dijo: «Ya voy»;

\par 8 y entrando al campo de labranza, se sentaron junto a la compañía.

\par 9 Y Abraham dijo a sus siervos, los hijos de Masek: «Vayan a la manada de caballos y traigan dos caballos, tranquilos, mansos y mansos, para que yo y este extraño podamos sentarnos en ellos».

\par 10 Pero el capitán dijo: «No, Señor mío Abraham, que no traigan caballos, porque yo me abstengo de montarme en ningún animal de cuatro patas.

\par 11 ¿No es mi rey rico en muchas mercancías, y que tiene poder sobre los hombres y sobre toda clase de ganado? Pero me abstengo de sentarme jamás sobre ningún animal de cuatro patas.

\par 12 Vamos, pues, alma justa, caminando con paso ligero hasta llegar a tu casa. Y Abraham dijo: «Amén, así sea».


\chapter{3}

\par 1 Y mientras salían del campo hacia su casa,

\par 2 Junto a ese camino había un ciprés,

\par 3 y por orden del Señor el árbol clamó con voz humana, diciendo: Santo, santo, santo es el Señor Dios que se llama a los que lo aman;

\par 4 Pero Abraham ocultó el misterio, pensando que el capitán no había oído la voz del árbol.

\par 5 Y llegando a la casa, se sentaron en el atrio, e Isaac, viendo el rostro del ángel, dijo a Sara su madre: «Señora madre, he aquí, el hombre que está sentado con mi padre Abraham no es hijo de raza de los que habitan la tierra».

\par 6 Isaac corrió, lo saludó y cayó a los pies del Incorporal, y el Incorporal lo bendijo y dijo: «El Señor Dios te concederá la promesa que le hizo a tu padre Abraham y a su descendencia, y también te concederá la preciosa oración de tu padre y de tu madre».

\par 7 Abraham dijo a Isaac su hijo: «Hijo mío, Isaac, saca agua del pozo y tráemela en la vasija para que lavemos los pies de este extraño, porque está cansado después de haber venido a nosotros desde lejos. un largo viaje.»

\par 8 Isaac corrió al pozo, sacó agua de la vasija y se la llevó.

\par 9 Y Abraham subió y lavó los pies del capitán Miguel, y el corazón de Abraham se conmovió y lloró por el extraño.

\par 10 Isaac, al ver llorar a su padre, lloró también; y el capitán, al verlos llorar, lloró también con ellos.

\par 11 Y las lágrimas del capitán cayeron sobre la vasija en el agua de la vasija y se convirtieron en piedras preciosas.

\par 12 Y Abraham, al ver la maravilla, se asombró y tomó las piedras en secreto, y escondió el misterio, guardándolo solo en su corazón.

\chapter{4}

\par 1 Y Abraham dijo a Isaac su hijo: «Ve, hijo amado, a la cámara interior de la casa y embellecela. Tened allí dos lechos para nosotros, uno para mí y otro para este hombre que está con nosotros este día.

\par 2 Prepáranos allí un asiento, un candelero y una mesa con abundancia de todo bien. Embellece la cámara, hijo mío, y extiende debajo de nosotros lino, púrpura y lino fino.

\par 3 Quemad allí todo incienso precioso y excelente, y traed del jardín plantas aromáticas, y llenad de ellas nuestra casa. Enciende siete lámparas llenas de aceite, para que nos regocijemos, porque este hombre que es nuestro hoy es más glorioso que los reyes o gobernantes, y su apariencia sobrepasa a la de todos los hijos de los hombres».

\par 4 E Isaac preparó bien todas las cosas, y Abraham, tomando al arcángel Miguel, entró en la cámara, y ambos se sentaron en los divanes, y entre ellos puso una mesa con abundante de todo lo bueno.

\par 5 Entonces el capitán se levantó y salió, como si su vientre estuviera obligado a producir agua, y en un abrir y cerrar de ojos ascendió al cielo, se presentó ante el Señor y le dijo:

\par 6 «Señor y Maestro, haz saber a tu poder que no puedo recordarle a ese justo su muerte, porque no he visto en la tierra a un hombre como él, compasivo, hospitalario, justo, veraz, devoto, que se abstenga de cada mala acción. Y ahora debes saber, Señor, que no puedo recordarle su muerte».

\par 7 Y el Señor dijo: «Ve, capitán en jefe Miguel, a ver a mi amigo Abraham, y haz lo que él te diga, y come con él todo lo que él coma.

\par 8 Y enviaré mi Espíritu Santo sobre su hijo Isaac, y pondré el recuerdo de su muerte en el corazón de Isaac, para que incluso él en sueños pueda ver la muerte de su padre, e Isaac le contará el sueño. , y tú lo interpretarás, y él mismo conocerá su fin».

\par 9 Y el capitán dijo: «Señor, todos los espíritus celestiales son incorpóreos y no comen ni beben, y este hombre ha puesto delante de mí una mesa con abundancia de todos los bienes terrenales y corruptibles. Ahora Señor, ¿qué haré? ¿Cómo escaparé de él, sentándome a la misma mesa con él? »

\par 10 El Señor dijo: «Desciende a él y no te preocupes por esto, porque cuando te sientes con él, enviaré sobre ti un espíritu devorador, que consumirá por tus manos y por tu boca todo eso está sobre la mesa. Regocíjate con él en todo,

\par 11 sólo tú interpretarás bien las cosas de la visión, para que Abraham conozca la hoz de la muerte y el fin incierto de la vida, y pueda disponer de todos sus bienes, porque yo lo he bendecido más que la arena del mar y como las estrellas del cielo».

\chapter{5}

\par 1 Entonces el capitán en jefe descendió a casa de Abraham, se sentó con él a la mesa e Isaac les servía.

\par 2 Y terminada la cena, Abraham oró según su costumbre, y el capitán oró con él, y cada uno se acostó a dormir en su lecho.

\par 3 E Isaac dijo a su padre: «Padre, yo también quisiera dormir contigo en esta cámara, para poder también escuchar tu discurso, porque me encanta escuchar la excelencia de la conversación de este hombre virtuoso».

\par 4 Abraham dijo: «No, hijo mío, sino vete a tu aposento y duerme en tu lecho, para que no seamos molestos para este hombre».

\par 5 Entonces Isaac, después de recibir la oración de ellos y después de bendecirlos, fue a su cámara y se acostó en su lecho.

\par 6 Pero el Señor puso en el corazón de Isaac el pensamiento de la muerte, como en un sueño,

\par 7 Y alrededor de la hora tercera de la noche, Isaac se despertó, se levantó de su lecho y vino corriendo a la cámara donde dormía su padre con el arcángel.

\par 8 Entonces Isaac, al llegar a la puerta, gritó diciendo: «Padre mío Abraham, levántate y ábreme pronto, para que entre y cuelgue de tu cuello y te abrace antes de que te aparten de mí».

\par 9 Entonces Abraham se levantó y le abrió, y entró Isaac, se colgó de su cuello y comenzó a llorar a gran voz.

\par 10 Entonces Abraham, conmovido de corazón, también lloró a gran voz; y el capitán, al verlos llorar, lloró también.

\par 11 Estando Sara en su habitación, oyó su llanto, vino corriendo hacia ellos y los encontró abrazados y llorando.

\par 12 Y Sara dijo llorando: «Señor mío Abraham, ¿qué es esto que lloras?

\par 13 Dime, Señor mío, ¿te ha traído este hermano que ha sido hospedado por nosotros hoy la noticia de que Lot, el hijo de tu hermano, ha muerto? ¿Es por esto que te afliges tanto?

\par 14 El capitán respondió y le dijo: «No, hermana mía Sara, no es como tú dices, sino que tu hijo Isaac, según creo, tuvo un sueño y vino a nosotros llorando, y nosotros, al verlo, nos conmovimos. en nuestros corazones y lloramos».

\chapter{6}

\par 1 Entonces Sara, al oír la excelente conversación del capitán, supo al instante que era un ángel del Señor el que hablaba.

\par 2 Entonces Sara indicó a Abraham que saliera hacia la puerta y le dijo: «Señor Abraham, ¿sabes quién es este hombre?»

\par 3 Abraham dijo: «No lo sé».

\par 4 Sara dijo: «Tú conoces, Señor mío, a los tres hombres del cielo que fueron hospedados por nosotros en nuestra tienda junto al roble de Mamre, cuando mataste al cabrito sin defecto y pusiste una mesa delante de ellos.

\par 5 Después de comer la carne, el cabrito se levantó otra vez y mamando a su madre con gran alegría. ¿No sabes, Señor mío Abraham, que por promesa nos dieron a Isaac como fruto del vientre? De estos tres santos varones éste es uno».

\par 6 Abraham dijo: «Sara, en esto dices la verdad. Gloria y alabanza de nuestro Dios y Padre. Porque ya avanzada la tarde, cuando lavé sus pies en la palangana, dije en mi corazón: Estos son los pies de uno de los tres hombres que lavé entonces;

\par 7 y sus lágrimas que caían en el recipiente se convirtieron en piedras preciosas. Y sacándolos de su regazo, se los dio a Sara, diciendo: Si no me crees, mira ahora esto.

\par 8 Sara, al recibirlos, se inclinó, saludó y dijo: «¡Gloria a Dios, que nos muestra cosas maravillosas! ¡Y ahora sabe, mi Señor Abraham, que hay entre nosotros la revelación de algo, ya sea malo o bueno!


\chapter{7}

\par 1 Y Abraham dejó a Sara, entró en la cámara y dijo a Isaac: «Ven acá, hijo amado, dime la verdad, qué fue lo que viste y qué te sucedió para que vinieras tan apresuradamente a nosotros».

\par 2 Y respondiendo Isaac, comenzó a decir: «Vi, Señor mío, esta noche el sol y la luna sobre mi cabeza, rodeándome con sus rayos y alumbrándome.

\par 3 Mientras contemplaba esto y me regocijaba, vi el cielo abierto y de él descendía un hombre que llevaba una luz que brillaba más que siete soles.

\par 4 Y vino este hombre como el sol, y me quitó el sol de la cabeza, y subió al cielo de donde había venido, pero me entristeció mucho que me quitara el sol.

\par 5 Poco después, estando todavía yo afligido y angustiado, vi a este hombre descender del cielo por segunda vez, y me quitó también la luna de mi cabeza.

\par 6 y lloré mucho y llamé a aquel hombre de luz, y le dije: Señor mío, no me quites mi gloria; Ten piedad de mí y escúchame, y si me quitas el sol, déjame la luna.

\par 7 Él dijo: «Dejad que los lleven al rey de arriba, porque él los quiere allí». Y me los quitó, pero sobre mí dejó los rayos».

\par 8 El capitán dijo: «Oye, justo Abraham; El sol que vio tu hijo eres tú su padre, y la luna también es Sara su madre. El hombre portador de luz que descendió del cielo, éste es el enviado de Dios que ha de quitaros vuestra alma justa.

\par 9 Ahora, pues, debes saber, oh honradísimo Abraham, que en este tiempo dejarás esta vida mundana y te trasladarás a Dios.

\par 10 Abraham dijo al capitán en jefe: «¡Oh, la más extraña de las maravillas! ¿Y ahora eres tú el que me quitará el alma?

\par 11 El capitán en jefe le dijo: «Soy el capitán en jefe Miguel, que está delante del Señor, y fui enviado a ti para recordarte tu muerte, y luego partiré hacia él como se me ordenó. .»

\par 12 Abraham dijo: Ahora sé que eres un ángel del Señor y que fuiste enviado para tomar mi alma, pero no iré contigo; pero haz lo que se te ordene».

\chapter{8}

\par 1 El capitán, al oír estas palabras, desapareció inmediatamente y, subiendo al cielo, se presentó ante Dios y contó todo lo que había visto en la casa de Abraham;

\par 2 Y el capitán dijo también esto a su Señor: Así dice tu amigo Abraham: No iré contigo, sino que haré lo que te mande;

\par 3 y ahora, oh Señor Todopoderoso, ¿ordena algo tu gloria y tu reino inmortal?

\par 4 Dios dijo al capitán Miguel: «Ve otra vez a ver a mi amigo Abraham y háblale así:

\par 5 Así dice el Señor tu Dios, el que te trajo a la tierra prometida, el que te bendijo más que la arena del mar y más que las estrellas del cielo:

\par 6 que abrió el vientre de esterilidad de Sara y te dio a Isaac como fruto del vientre en la vejez,

\par 7 De cierto os digo que con bendición os bendeciré, y multiplicando multiplicaré vuestra descendencia, y os daré todo lo que me pidáis, porque yo soy el Señor vuestro Dios, y fuera de mí no hay otro.

\par 8 Dime por qué te has rebelado contra mí, y por qué hay dolor en ti, y por qué te rebelaste contra mi arcángel Miguel.

\par 9 ¿No sabéis que todos los que vinieron de Adán y Eva han muerto, y que ninguno de los profetas ha escapado de la muerte? Ninguno de los que gobiernan como reyes es inmortal; Ninguno de tus antepasados ​​ha escapado al misterio de la muerte. Todos murieron, todos partieron al Hades, todos fueron recogidos por la hoz de la muerte.

\par 10 Pero sobre ti no he enviado la muerte, no he permitido que te sobrevenga ninguna enfermedad mortal, no he permitido que la hoz de la muerte te encuentre, no he permitido que las redes del Hades te envuelvan, he Nunca deseé que te encontraras con ningún mal.

\par 11 Pero para tu comodidad te he enviado a mi capitán Miguel, para que sepas tu partida del mundo, y pongas en orden tu casa y todo lo que te pertenece, y bendigas a Isaac, tu amado hijo. Y ahora sabed que he hecho esto sin querer entristeceros.

\par 12 ¿Por qué entonces le dijiste a mi capitán en jefe: «No iré contigo?» ¿Por qué has hablado así? ¿No sabes que si doy permiso a la muerte y ella viene sobre ti, entonces debería ver si vienes o no?

\chapter{9}

\par 1 Y el capitán, recibiendo las exhortaciones del Señor, descendió a Abraham, y al verlo, el justo cayó rostro en tierra como un muerto.

\par 2 y el capitán le contó todo lo que había oído del Altísimo. Entonces el santo y justo Abraham, levantándose con muchas lágrimas, se postró a los pies del Incorporal, y le suplicó, diciendo:

\par 3 «Te lo ruego, capitán en jefe de los ejércitos de arriba, ya que te has dignado venir a mí como un pecador y en todo tu servidor indigno, te ruego incluso ahora, oh capitán en jefe, que lleves mi palabra otra vez al Altísimo, y le dirás:

\par 4 Así dice Abraham tu siervo: Señor, Señor, en cada obra y palabra que te he pedido, me has oído y has cumplido todos mis consejos.

\par 5 Ahora, Señor, no resisto a tu poder, porque también yo sé que no soy inmortal sino mortal. Por tanto, puesto que a tus órdenes todas las cosas se someten, y temen y tiemblan ante tu poder, yo también temo, pero te pido una petición:

\par 6 y ahora, Señor y Maestro, escucha mi oración, porque mientras todavía estoy en este cuerpo deseo ver toda la tierra habitada y todas las creaciones que tú estableciste con una sola palabra, y cuando las vea, entonces si podré apartaré de la vida y estaré sin tristeza».

\par 7 Entonces el capitán volvió otra vez, se presentó ante Dios y le contó todo, diciendo: Así dice tu amigo Abraham: Yo deseaba contemplar toda la tierra durante mi vida antes de morir.

\par 8 Y oyendo esto el Altísimo, volvió a ordenar al capitán Miguel, y le dijo: Toma una nube de luz y los ángeles que tienen poder sobre los carros, y desciende, y lleva al justo Abraham sobre un carro de querubines, y lo exaltaremos al aire de los cielos para que pueda contemplar toda la tierra».

\chapter{10}

\par 1 Entonces el arcángel Miguel descendió y tomó a Abraham en un carro de querubines, lo exaltó al aire del cielo y lo llevó sobre la nube junto con sesenta ángeles, y Abraham ascendió en el carro sobre toda la tierra.

\par 2 Y Abraham vio el mundo tal como era aquel día: unos arando, otros conduciendo carros, en un lugar pastoreando rebaños, en otro vigilándolos de noche y bailando, tocando y tocando arpas, en otro lugar luchando y contendiendo por la ley, en otras partes hombres llorando y recordando a los muertos.

\par 3 Vio también a los recién casados ​​recibidos con honores y, en una palabra, vio todas las cosas que se hacen en el mundo, tanto las buenas como las malas.

\par 4 Entonces Abraham, al pasar por encima de ellos, vio a unos hombres que llevaban espadas y empuñaban espadas afiladas en las manos, y Abraham preguntó al capitán mayor: «¿Quiénes son estos?»

\par 5 El capitán dijo: «Estos son ladrones que quieren matar, hurtar, quemar y destruir».

\par 6 Abraham dijo: «Señor, Señor, escucha mi voz y ordena que las fieras salgan del bosque y los devoren».

\par 7 Y mientras él hablaba, salieron fieras del bosque y los devoraron.

\par 8 Y vio en otro lugar a un hombre y una mujer que fornicaban el uno con el otro,

\par 9 y dijo: «Señor, Señor, ordena que la tierra se abra y se los trague, y en seguida la tierra se partió y se los tragó».

\par 10 Y vio en otro lugar a unos hombres cavando en una casa y llevándose las posesiones ajenas,

\par 11 y dijo: «Señor, Señor, ordena que descienda fuego del cielo y los consuma. Y mientras él hablaba, descendió fuego del cielo y los consumió».

\par 12 Y en seguida vino una voz del cielo al capitán mayor, diciendo así: ¡Oh capitán Miguel, ordena que se detenga el carro y haz que Abraham se vaya para que no vea toda la tierra!

\par 13 Porque si ve a todos los que viven en la maldad, destruirá toda la creación. Porque he aquí, Abraham no ha pecado, y no tiene compasión de los pecadores,

\par 14 Pero yo he hecho el mundo y no deseo destruir a ninguno de ellos, sino esperar la muerte del pecador hasta que se convierta y viva.

\par 15 Pero lleva a Abraham a la primera puerta del cielo, para que vea allí los juicios y las recompensas, y se arrepienta de las almas de los pecadores que él ha destruido.

\chapter{11}

\par 1 Entonces Miguel hizo girar el carro y llevó a Abraham hacia el este, a la primera puerta del cielo;

\par 2 Y Abraham vio dos caminos, uno angosto y angosto, el otro ancho y espacioso,

\par 3 y allí vio dos puertas, una ancha en el camino ancho y otra estrecha en el camino angosto.

\par 4 Y fuera de las dos puertas vio a un hombre sentado sobre un trono dorado, y el aspecto de aquel hombre era terrible, como el del Señor.

\par 5 Y vieron muchas almas conducidas por ángeles y conducidas por la puerta ancha, y otras almas, pocas en número, que eran conducidas por los ángeles por la puerta estrecha.

\par 6 Y cuando el maravilloso que estaba sentado en el trono de oro vio que pocos entraban por la puerta estrecha y muchos por la ancha, en seguida aquel maravilloso se rasgó los cabellos de su cabeza y las laterales de su barba, y se arrojó en tierra desde su trono, llorando y lamentándose.

\par 7 Pero cuando vio muchas almas entrar por la puerta estrecha, se levantó del suelo y se sentó en su trono con gran alegría, regocijándose y regocijándose.

\par 8 Y Abraham preguntó al capitán: «Señor, capitán, ¿quién es este hombre tan maravilloso, adornado con tanta gloria, que a veces llora y se lamenta, y a veces se regocija y se regocija?»

\par 9 El incorpóreo dijo: «Este es el primer Adán creado, que está en tanta gloria y mira al mundo porque todos nacen de él,

\par 10 y cuando ve muchas almas pasar por la puerta estrecha, se levanta y se sienta en su trono, gozoso y exultante de alegría, porque esta puerta estrecha es la de los justos, que lleva a la vida, y los que entran por ella. ir al Paraíso. Por esto, pues, se alegra el primer Adán creado, porque ve las almas salvadas.

\par 11 Pero cuando ve muchas almas entrar por la puerta ancha, se arranca los cabellos y se tira al suelo llorando y lamentándose amargamente, porque la puerta ancha es la de los pecadores, que conduce a la destrucción y a la destrucción. castigo eterno. Y por esto el primer Adán cae de su trono llorando y lamentándose por la destrucción de los pecadores, porque son muchos los que se pierden, y son pocos los que se salvan.

\par 12 Porque entre siete mil apenas se encuentra una persona salva, que sea justa e inmaculada.

\chapter{12}

\par 1 Mientras todavía me decía estas cosas, he aquí dos ángeles, de aspecto ardiente, de mente despiadada y de mirada severa, y arremetían contra miles de almas, azotándolas sin piedad con correas de fuego.

\par 2 El ángel apresó a una de ellas, y a todas las expulsó por la puerta ancha hacia la destrucción.

\par 3 Nosotros también fuimos con los ángeles y entramos por la puerta ancha.

\par 4 y entre las dos puertas había un trono de terrible aspecto, de terrible cristal, resplandeciente como fuego,

\par 5 y sobre él estaba sentado un hombre maravilloso, brillante como el sol, semejante al Hijo de Dios.

\par 6 Delante de él había una mesa como de cristal, toda de oro y lino fino,

\par 7 Y sobre la mesa había un libro de seis codos de espesor y diez codos de ancho.

\par 8 y a derecha e izquierda estaban dos ángeles que tenían papel, tinta y pluma.

\par 9 Ante la mesa estaba sentado un ángel de luz, sosteniendo en su mano una balanza,

\par 10 y a su izquierda estaba sentado un ángel todo fogoso, despiadado y severo, que tenía en su mano una trompeta que tenía en su interior un fuego devorador para probar a los pecadores.

\par 11 El hombre maravilloso que estaba sentado en el trono juzgaba y sentenciaba las almas,

\par 12 Y los dos ángeles de la derecha y de la izquierda escribieron: el de la derecha la justicia y el de la izquierda la maldad.

\par 13 El que estaba sentado a la mesa, que sostenía la balanza, pesaba las almas,

\par 14 y el ángel de fuego que sostenía el fuego probó las almas.

\par 15 Y Abraham preguntó al capitán Miguel: «¿Qué es esto que estamos viendo?» Y el capitán mayor dijo: «Estas cosas que ves, santo Abraham, son el juicio y la recompensa.

\par 16 Y he aquí el ángel que tenía el alma en la mano y la llevó ante el juez.

\par 17 Y el juez dijo a uno de los ángeles que le servían: Ábreme este libro y encuéntrame los pecados de esta alma.

\par 18 Y al abrir el libro, encontró que sus pecados y su justicia estaban igualmente equilibrados, y no se lo dio a los verdugos ni a los que se habían salvado, sino que lo puso en medio.


\chapter{13}

\par 1 Y Abraham dijo: «Señor capitán, ¿quién es este juez tan maravilloso? ¿Y quiénes son los ángeles que escriben? ¿Y quién es el ángel como el sol que sostiene la balanza? ¿Y quién es el ángel de fuego que sostiene el fuego?

\par 2 El capitán dijo: «¿Ves, santísimo Abraham, al hombre terrible sentado en el trono? Éste es el hijo del primer Adán creado, llamado Abel, a quien mató el malvado Caín,

\par 3 y así se sienta para juzgar a toda la creación y examina a los justos y a los pecadores. Porque Dios ha dicho: No os juzgaré, sino que todo hombre nacido de hombre será juzgado.

\par 4 Por eso le ha dado juicio, para juzgar al mundo hasta su grande y gloriosa venida, y entonces, oh justo Abraham, será el juicio perfecto y la recompensa, eterna e inmutable, que nadie podrá alterar.

\par 5 Porque cada hombre proviene de la primera creación, y por eso aquí son juzgados primero por su hijo,

\par 6 y en la segunda venida serán juzgados por las doce tribus de Israel, todo aliento y toda criatura.

\par 7 Pero la tercera vez serán juzgados por el Señor Dios de todos, y entonces, en verdad, el fin de ese juicio está cerca, y la sentencia es terrible, y no hay nadie que los libere.

\par 8 Ahora bien, el juicio del mundo y la recompensa se dictan mediante tres tribunales, y por eso un asunto no es finalmente confirmado por uno o dos testigos, sino que por tres testigos todo se establece.

\par 9 Los dos ángeles de la derecha y de la izquierda son los que escriben los pecados y las justicias, el de la derecha escribe las justicias y el de la izquierda los pecados.

\par 10 El ángel semejante al sol, que sostiene la balanza en su mano, es el arcángel; Dokiel, el justo pesador, y pesa las justicias y los pecados con la justicia de Dios.

\par 11 El ángel ardiente y despiadado que sostiene el fuego en su mano es el arcángel Puruel, que tiene poder sobre el fuego y prueba las obras de los hombres mediante el fuego.

\par 12 Y si el fuego consume la obra de alguno, inmediatamente el ángel del juicio lo apresa y lo lleva al lugar de los pecadores, al lugar de castigo más amargo.

\par 13 Pero si el fuego aprueba la obra de alguien y no se apodera de ella, ese hombre queda justificado, y el ángel de la justicia lo toma y lo lleva para ser salvo en la suerte de los justos.

\par 14 Y así, justo Abraham, todas las cosas en todos los hombres son probadas por el fuego y la balanza.

\chapter{14}

\par 1 Y Abraham dijo al capitán: «Señor, capitán, el alma que el ángel tenía en su mano, ¿por qué fue juzgada puesta en medio?»

\par 2 El capitán dijo: «Escucha, justo Abraham. Porque el juez encontró sus pecados. y sus justicias iguales, no la entregó a juicio ni para salvación, hasta que venga el juez de todos».

\par 3 Abraham dijo al capitán: ¿Y qué falta todavía para que el alma se salve?

\par 4 El capitán dijo: «Si obtiene una justicia por encima de sus pecados, entrará en la salvación».

\par 5 Abraham dijo al capitán: «Ven acá, capitán Miguel, oremos por esta alma y veamos si Dios nos escucha. El capitán mayor dijo: Amén, así sea;

\par 6 e hicieron oración y súplica por el alma, y ​​Dios los escuchó, y cuando se levantaron de su oración no vieron al alma parada allí.

\par 7 Y Abraham dijo al ángel: «¿Dónde está el alma que tenías en medio?»

\par 8 Y el ángel respondió: «Ha sido salvado por tu justa oración, y he aquí, un ángel de luz lo ha tomado y lo ha llevado al Paraíso».

\par 9 Abraham dijo: «Glorifico el nombre de Dios Altísimo y su infinita misericordia».

\par 10 Y Abraham dijo al capitán: «Te ruego, arcángel, que escuches mi oración e invoquemos aún al Señor,

\par 11 y suplicad su compasión, y suplicad su misericordia por las almas de los pecadores a quienes antes, en mi ira, maldije y destruí, a quienes la tierra devoró, y las fieras despedazaron, y el fuego consumió por mis palabras. .

\par 12 Ahora sé que he pecado ante el Señor nuestro Dios. Ven, pues, oh Miguel, capitán en jefe de los ejércitos de arriba, ven, invoquemos a Dios con lágrimas para que me perdone mis pecados y me los conceda».

\par 13 Cuando el capitán lo escuchó, rogaron al Señor, y después de haberlo invocado por largo tiempo, vino una voz del cielo que decía:

\par 14 «Abraham, Abraham, he escuchado tu voz y tu oración, y te perdono tu pecado, y a aquellos que crees que destruí, los he llamado y los he resucitado con mi gran bondad, porque por un tiempo Les he pagado con juicio, y a aquellos a quienes destruya que viven en la tierra, no les pagaré con la muerte».

\chapter{15}

\par 1 Y la voz del Señor dijo también al capitán Miguel: Miguel, mi siervo, haz regresar a Abraham a su casa, porque he aquí que su fin se ha acercado y la medida de su vida se ha cumplido para que pueda Pon todas las cosas en orden, y luego tómalo y tráelo a mí.

\par 2 Entonces el capitán, haciendo girar el carro y la nube, llevó a Abraham a su casa,

\par 3 y entrando en su cámara, se sentó en su lecho.

\par 4 Y Sara, su esposa, se acercó y abrazó los pies del Incorpóreo, y habló con humildad, diciendo: «Te doy gracias, mi Señor, por haber traído a mi Señor Abraham, porque he aquí, pensábamos que había sido arrebatado de entre nosotros». .»

\par 5 Y vino también su hijo Isaac y se echó sobre su cuello, y de la misma manera todos sus esclavos y esclavas rodearon a Abraham y lo abrazaron, glorificando a Dios.

\par 6 Y el Incorporal les dijo: «Escuchen, justo Abraham. He aquí tu esposa Sara, he aquí tu amado hijo Isaac, he aquí también todos tus siervos y siervas que te rodean».

\par 7 Dispón de todo lo que tienes, porque se ha acercado el día en que dejaréis el cuerpo e iréis al Señor una vez para siempre.

\par 8 Abraham dijo: «¿Lo ha dicho el Señor o lo dices tú mismo?»

\par 9 El capitán respondió: «Escucha, justo Abraham. El Señor lo ha ordenado y yo os lo digo».

\par 10 Abraham dijo: «No iré contigo».

\par 11 El capitán, al oír estas palabras, salió inmediatamente de la presencia de Abraham, subió al cielo y se presentó ante el Dios Altísimo.

\par 12 y dijo: «Señor Todopoderoso, he aquí que he escuchado a tu amigo Abraham en todo lo que te ha dicho y he cumplido sus peticiones. Le he mostrado tu poder y toda la tierra y el mar que está debajo del cielo. Le he mostrado juicio y recompensa por medio de nubes y carros, y otra vez dice: No iré contigo».

\par 13 Y el Altísimo dijo al ángel: ¿Vuelve a decir mi amigo Abraham lo mismo: No iré contigo? »

\par 14 El arcángel dijo: «Señor Todopoderoso, así dice y yo me abstengo de imponerle las manos, porque desde el principio es tu amigo y ha hecho todo lo que te agrada».

\par 15 No hay hombre como él en la tierra, ni siquiera Job, el hombre maravilloso, y por eso me abstengo de ponerle las manos encima. Ordena, pues, Rey Inmortal, lo que se hará.

\chapter{16}

\par 1 Entonces el Altísimo dijo: «Llámame aquí Muerte, que se llama rostro desvergonzado y mirada despiadada».

\par 2 Y fue Miguel el Incorporal y dijo a la Muerte: Ven acá; el Señor de la creación, el rey inmortal, te llama».

\par 3 Y la Muerte, al oír esto, se estremeció y tembló, poseída por un gran terror, y acercándose con gran miedo se paró ante el padre invisible, temblando, gimiendo y temblando, esperando la orden del Señor.

\par 4 Por eso el Dios invisible dijo a la Muerte: Ven acá, nombre amargo y feroz del mundo, esconde tu furia, cubre tu corrupción, y arroja de ti tu amargura, y vístete de tu hermosura y de toda tu gloria.

\par 5 y desciende donde Abraham mi amigo, tómalo y tráelo a mí. Pero ahora también os digo que no le asustéis, sino que lo traigáis con buenas palabras, porque es mi amigo.

\par 6 Al oír esto, la Muerte salió de la presencia del Altísimo, se vistió con un manto resplandeciente y se apareció como el sol, y se volvió bella y hermosa sobre los hijos de los hombres, tomando forma de arcángel, con las mejillas ardiendo de fuego, y se fue a Abraham.

\par 7 Entonces el justo Abraham salió de su aposento y se sentó bajo los árboles de Mamre, con la barbilla en la mano, esperando la venida del arcángel Miguel.

\par 8 Y he aquí, llegó hasta él un olor suave y un resplandor de luz; y Abraham se volvió y vio a la Muerte que venía hacia él con gran gloria y hermosura. Y Abraham se levantó y fue a su encuentro, pensando que era el Príncipe de Dios,

\par 9 Y la Muerte, mirándolo, lo saludó, diciendo: Alégrate, precioso Abraham, alma justa, verdadero amigo del Dios Altísimo y compañero de los santos ángeles.

\par 10 Abraham dijo a la Muerte: Te saludo, de apariencia y forma como el sol, gloriosísimo ayudante, portador de luz, hombre maravilloso, ¿de dónde viene a nosotros tu gloria, y quién eres tú, y de dónde vienes? »

\par 11 Entonces la Muerte dijo: «Justísimo Abraham, he aquí te digo la verdad. Soy el destino amargo de la muerte».

\par 12 Abraham le dijo: «No, sino que tú eres la hermosura del mundo, eres la gloria y la belleza de los ángeles y de los hombres, eres más hermoso en forma que cualquier otro, y ¿dices: Yo soy la suerte amarga? de la muerte, y no más bien, soy más justo que todo bien».

\par 13 La muerte dijo: «Os digo la verdad. Lo que el Señor me ha puesto, eso también os lo digo».

\par 14 Abraham dijo: «¿Para qué has venido aquí?»

\par 15 La muerte dijo: «Por tu santa alma vengo».

\par 16 Entonces Abraham dijo: «Sé lo que quieres decir, pero no iré contigo; y la Muerte guardó silencio y no le respondió palabra.»

\chapter{17}

\par 1 Entonces Abraham se levantó y entró en su casa, y allí también lo acompañó la Muerte. Y Abraham subió a su aposento, y la Muerte subió con él. Y Abraham se acostó en su lecho, y la Muerte vino y se sentó a sus pies.

\par 2 Entonces Abraham dijo: «Apartaos, apártate de mí, porque deseo descansar en mi lecho».

\par 3 La muerte dijo: «No me iré hasta que os quite el espíritu».

\par 4 Abraham le dijo: «Por el Dios inmortal, te encargo que me digas la verdad. ¿Estás muerto?

\par 5 La Muerte le dijo: «Yo soy la Muerte. Soy el destructor del mundo».

\par 6 Abraham dijo: Te ruego, ya que eres la Muerte, que me digas si llegas así a todos con tanta justicia, gloria y hermosura.

\par 7 La muerte dijo: «No, mi Señor Abraham, porque tus justicias, el mar ilimitado de tu hospitalidad y la grandeza de tu amor hacia Dios se han convertido en una corona sobre mi cabeza, y en belleza, gran paz y dulzura yo acércate a los justos,

\par 8 pero a los pecadores vengo con gran corrupción y fiereza y la mayor amargura y con mirada fiera y despiadada.

\par 9 Abraham dijo: «Te ruego que me escuches y muéstrame tu crueldad y toda tu corrupción y amargura».

\par 10 Y la Muerte dijo: «No puedes contemplar mi fiereza, justo Abraham».

\par 11 Abraham dijo: «Sí, podré contemplar toda vuestra ferocidad en el nombre del Dios vivo, porque el poder de mi Dios que está en los cielos está conmigo».

\par 12 Entonces la muerte se despojó de toda su hermosura y hermosura, de toda su gloria y de la forma semejante al sol con que estaba revestida,

\par 13 y se vistió con un manto de tirano, y se hizo ver más sombrío y más feroz que toda clase de fieras, y más inmundo que toda inmundicia.

\par 14 Y mostró a Abraham siete cabezas de serpientes ardientes y catorce rostros, (uno) de fuego llameante y de gran fiereza, y un rostro de oscuridad, y un rostro de víbora muy sombrío, y un rostro de terrible precipicio, y un rostro más feroz que el de un áspid, y un rostro de león terrible, y un rostro de cerastes y de basilisco.

\par 15 Le mostró también un rostro de cimitarra de fuego, un rostro con espada, un rostro de relámpago terrible y un ruido de trueno espantoso.

\par 16 Le mostró también la otra cara de un mar embravecido y tempestuoso, y de un río impetuoso, de una terrible serpiente de tres cabezas y de una copa mezclada con venenos,

\par 17 y en resumen le mostró una gran fiereza y una amargura insoportable, y toda enfermedad mortal como olor a muerte.

\par 18 Y a causa de la gran amargura y furia, murieron siervos y sirvientas en número de unos siete mil,

\par 19 Y el justo Abraham cayó en la indiferencia de la muerte, de modo que le falló el espíritu.

\chapter{18}

\par 1 Y el santísimo Abraham, viendo estas cosas así, dijo a la Muerte: «Te ruego, Muerte destructora de todo, oculta tu fiereza y reviste tu belleza y la forma que tenías antes».

\par 2 Y al instante la Muerte ocultó su furia y se vistió con la belleza que antes tenía.

\par 3 Y Abraham dijo a la Muerte: «¿Por qué has hecho esto, que has matado a todos mis siervos y siervas? ¿Te ha enviado Dios aquí hoy para este fin?

\par 4 La muerte dijo: «No, mi Señor Abraham, no es como tú dices, sino que por tu causa fui enviado acá».

\par 5 Abraham dijo a la Muerte: «¿Cómo, pues, han muerto éstos? ¿No lo ha dicho el Señor?»

\par 6 Dijo la muerte: «Cree, justo Abraham, que esto también es maravilloso: que tampoco tú fuiste llevado con ellos. Sin embargo te digo la verdad,

\par 7 Porque si la diestra de Dios no hubiera estado con vosotros en aquel momento, también vosotros habríais tenido que partir de esta vida.

\par 8 El justo Abraham dijo: Ahora sé que he llegado a la indiferencia de la muerte, de modo que mi espíritu desfallece,

\par 9 pero yo te ruego, Muerte destructora de todo, ya que mis siervos han muerto antes de tiempo, ven y oremos al Señor nuestro Dios para que nos escuche y resucite a los que murieron antes de tiempo por tu crueldad.

\par 10 Y la Muerte dijo: «Amén, así sea». Entonces Abraham se levantó y cayó sobre la faz de la tierra en oración, y la Muerte junto con él,

\par 11 Y el Señor envió un espíritu de vida sobre los que estaban muertos, y volvieron a vivir. Entonces el justo Abraham dio gloria a Dios.

\chapter{19}

\par 1 Y subiendo a su aposento se acostó, y la Muerte vino y se puso delante de él.

\par 2 Y Abraham le dijo: «Apártate de mí, porque deseo descansar, porque mi espíritu está indiferente».

\par 3 La muerte dijo: «No me apartaré de ti hasta tomar tu alma».

\par 4 Y Abraham, con semblante severo y mirada enojada, dijo a la Muerte: «¿Quién te ha ordenado que digas esto?

\par 5 Dices estas palabras con jactancia y no iré contigo hasta que venga a verme el capitán en jefe Miguel, y yo iré con él. Pero esto también os digo, si queréis que os acompañe, explicadme todos vuestros cambios, las siete cabezas de serpientes de fuego y lo que es la cara del precipicio, y lo que es la espada aguda, y lo que es el gran rugido. río, y qué mar tempestuoso que brama con tanta fiereza.

\par 6 Enséñame también los truenos insoportables, los relámpagos terribles y la copa maloliente mezclada con veneno. Enséñame acerca de todo esto».

\par 7 Y la Muerte respondió: «Escucha, justo Abraham. Durante siete siglos destruyo el mundo y conduzco a todos al Hades, reyes y gobernantes, ricos y pobres, esclavos y libres, los convoyo al fondo del Hades, y para esto os mostré las siete cabezas de serpientes.

\par 8 Os mostré el rostro del fuego, porque muchos mueren consumidos por el fuego, y contemplan la muerte a través del rostro del fuego.

\par 9 Os mostré la cara del precipicio, porque muchos hombres mueren descendiendo de las copas de los árboles o de terribles precipicios y perdiendo la vida, y ven la muerte en forma de terrible precipicio.

\par 10 Os mostré el rostro de la espada, porque muchos mueren a espada en las guerras y ven la muerte como una espada.

\par 11 Os mostré la cara del gran río caudaloso, porque muchos se ahogan y perecen arrebatados por el paso de muchas aguas y arrastrados por grandes ríos, y ven la muerte antes de tiempo.

\par 12 Os mostré el rostro del mar furioso y furioso, porque muchos en el mar, cayendo en grandes olas y naufragando, son tragados y contemplan la muerte como el mar.

\par 13 Os mostré los truenos insoportables y los relámpagos terribles, porque muchos hombres, en el momento de la ira, se topan con truenos insoportables y relámpagos terribles que vienen a apoderarse de los hombres y ven así la muerte.

\par 14 También te mostré las fieras venenosas, los áspides y los basiliscos, los leopardos, los leones, los cachorros de leones, los osos y las víboras, y en resumen, te mostré el rostro de todas las fieras, oh justo, porque muchos hombres son destruidos. por las fieras,

\par 15 y otros por serpientes venenosas, víboras, áspides, cerastes, basiliscos y víboras, exhalan su vida y mueren.

\par 16 También os mostré las copas destructoras mezcladas con veneno, porque muchos hombres, a los que otros les dan a beber veneno, se marchan inmediatamente.

\chapter{20}

\par 1 Abraham dijo: «Os ruego: ¿hay también una muerte inesperada? Dime.»

\par 2 La muerte dijo: «De cierto, de cierto os digo en la verdad de Dios que hay setenta y dos muertes. Se trata de la muerte justa, comprando su tiempo fijado, y muchos hombres en una hora entran en la muerte siendo entregados al sepulcro.

\par 3 He aquí, ya te he dicho todo lo que me pedías; ahora te digo, justo Abraham, que desestimes todo consejo y dejes de pedir nada de una vez por todas, y ven, ve conmigo, como Dios y juez. de todos me ha mandado.»

\par 4 Abraham dijo a la Muerte: «Apártate todavía un poco de mí, para que pueda descansar en mi lecho, porque estoy muy débil de corazón,

\par 5 Porque desde que te vi con mis ojos me faltaron las fuerzas, todos los miembros de mi carne me parecen como un peso de plomo, y mi espíritu está muy angustiado. Sal por un rato; porque he dicho que no puedo soportar ver tu forma».

\par 6 Entonces llegó su hijo Isaac y cayó sobre su pecho llorando, y Sara su mujer se acercó y abrazó sus pies, lamentándose amargamente.

\par 7 Llegaron también sus esclavos y sus esclavas y rodearon su lecho, lamentándose mucho. Y Abraham quedó indiferente a la muerte,

\par 8 Y la Muerte dijo a Abraham: «Ven, toma mi mano derecha, y que te llegue la alegría, la vida y la fuerza».

\par 9 Porque la muerte engañó a Abraham, y éste tomó su mano derecha, y al instante su alma se adhirió a la mano de la Muerte.

\par 10 E inmediatamente vino el arcángel Miguel con una multitud de ángeles y tomó en sus manos su preciosa alma envuelta en un lienzo de lino divinamente tejido,

\par 11 y cuidaron el cuerpo del justo Abraham con ungüentos y perfumes divinos hasta el tercer día después de su muerte, y lo sepultaron en la tierra prometida, en la encina de Mamre,

\par 12 Pero los ángeles recibieron su preciosa alma y ascendieron al cielo, cantando el himno tres veces santo al Señor, Dios de todos, y la pusieron allí para adorar al Dios y Padre.

\par 13 Y después de haber sido dadas grandes alabanza y gloria al Señor, y Abraham se postró para adorar, vino la voz pura del Dios y Padre diciendo así:

\par 14 Lleva, pues, a mi amigo Abraham al paraíso, donde están las moradas de mis justos y las moradas de mis santos Isaac y Jacob en su seno, donde no hay angustia, ni tristeza, ni gemido, sino paz, alegría y vida interminable.

\par 15 (Y nosotros también, amados hermanos míos, imitemos la hospitalidad del patriarca Abraham y alcancemos su virtuoso modo de vivir, para que seamos considerados dignos de la vida eterna, glorificando al Padre, al Hijo y al Espíritu Santo). ; a quien sea la gloria y el poder por los siglos. Amén.).

\part{Versión 2}

\chapter{21}

\par 1 Aconteció que cuando se acercaban los días de la muerte de Abraham, el Señor dijo a Miguel:

\par 2 Levántate y ve a Abraham, mi siervo, y dile: Dejarás la vida, porque ¡he aquí!

\par 3 Los días de tu vida temporal se han cumplido, para que él pueda ordenar su casa antes de morir.

\chapter{22}

\par 1 Entonces Miguel fue y llegó donde Abraham, y lo encontró sentado delante de sus bueyes arando, y tenía un aspecto muy viejo y tenía a su hijo en brazos.

\par 2 Entonces Abraham, al ver al arcángel Miguel, se levantó de la tierra y lo saludó, sin saber quién era,

\par 3 y le dijo: «El Señor te guarde. Que tu viaje sea próspero contigo».

\par 4 Y Michael le respondió: «Eres amable, buen padre».

\par 5 Abraham respondió y le dijo: «Ven, acércate a mí, hermano, y siéntate un poco, y yo ordenaré que traigan una bestia para que vayamos a mi casa y tú descanses conmigo. porque ya es tarde,

\par 6 y levántate por la mañana y ve a donde quieras, no sea que alguna bestia mala te encuentre y te haga daño.

\par 7 Y Miguel preguntó a Abraham, diciendo: «Dime tu nombre antes de entrar en tu casa, para que no te resulte gravoso».

\par 8 Abraham respondió y dijo: Mis padres me llamaron Abram, y el Señor me llamó Abraham, diciendo: Levántate y sal de tu casa y de tu parentela, y vete a la tierra que te mostraré.

\par 9 Y cuando fui a la tierra que el Señor me mostró, él me dijo: «Tu nombre no se llamará más Abram, sino que tu nombre será Abraham».

\par 10 Respondió Miguel y le dijo: «Perdóname, padre mío, hombre experimentado de Dios, porque soy un extranjero y he oído de ti que recorriste cuarenta estadios y trajiste una cabra y la mataste, hospedando a los ángeles en tu casa, para que allí descansen».

\par 11 Hablando así, se levantaron y se dirigieron hacia la casa.

\par 12 Entonces Abraham llamó a uno de sus siervos y le dijo: «Ve, tráeme un animal para que el extranjero se siente en él, porque está cansado del camino».

\par 13 Y Miguel dijo: «No molestes al joven, sino que vayamos con cuidado hasta llegar a la casa, porque amo tu compañía».

\chapter{23}

\par 1 Levantándose, prosiguieron y, a medida que se acercaban a la ciudad,

\par 2 Como a tres estadios de allí, encontraron un gran árbol que tenía trescientas ramas, como un tamarisco.

\par 3 Y oyeron una voz que cantaba desde sus ramas: Santo eres, porque has guardado el propósito para el cual fuiste enviado.

\par 4 Y Abraham oyó la voz y escondió el misterio en su corazón, diciendo dentro de sí: ¿Cuál es el misterio que he oído?

\par 5 Al entrar en casa, Abraham dijo a sus siervos: Levantaos, salid a los rebaños, traed tres ovejas, matadlas pronto y preparadlas para que comamos y bebamos, porque hoy es un día. fiesta para nosotros.

\par 6 Y los sirvientes trajeron las ovejas, y Abraham llamó a su hijo Isaac, y le dijo: Hijo mío, Isaac, levántate y pon agua en la vasija para que podamos lavar los pies de este extraño. Y lo trajo como se le había ordenado,

\par 7 Y Abraham dijo: Ya veo, y así será, que nunca más lavaré en esta palangana los pies de ningún hombre que venga a visitarnos.

\par 8 E Isaac, al oír a su padre decir esto, lloró y le dijo: Padre mío, ¿qué es esto que dices? ¿Esta es la última vez que le lavo los pies a un extraño? Y Abraham, viendo llorar a su hijo, también lloró mucho.

\par 9 Y Miguel, al verlos llorar, lloró también, y las lágrimas de Miguel cayeron sobre la vasija y se convirtieron en una piedra preciosa.

\chapter{24}

\par 1 Cuando Sara, estando dentro de su casa, escuchó su llanto, salió y dijo a Abraham: Señor, ¿por qué lloras así?

\par 2 Abraham respondió y le dijo: «No es ningún mal. Entra en tu casa y haz tu propio trabajo, no sea que seamos molestos para ese hombre».

\par 3 Y Sara se fue, cuando estaba a punto de preparar la cena.

\par 4 Cuando el sol estaba a punto de ponerse, Miguel salió de la casa y fue elevado al cielo para adorar delante de Dios.

\par 5 porque al atardecer todos los ángeles adoran a Dios y el mismo Miguel es el primero de los ángeles.

\par 6 Y todos le adoraron y fueron cada uno a su lugar,

\par 7 Pero Miguel habló delante del Señor y dijo: ¡Señor, ordena que me interroguen ante tu santa gloria!

\par 8 Y el Señor dijo a Miguel: «¡Anuncia lo que quieras!»

\par 9 Y el Arcángel respondió y dijo: Señor, tú me enviaste a Abraham para decirle: Deja tu cuerpo y deja este mundo; el Señor os llama;

\par 10 Y no me atrevo, Señor, a revelarme a él, porque él es tu amigo, y un hombre justo y que recibe a los extraños.

\par 11 Pero te ruego, Señor, que hagas que el recuerdo de la muerte de Abraham entre en su corazón, y no me pidas que se lo cuente,

\par 12 Porque es gran brusquedad decir: Deja el mundo, y especialmente dejar el propio cuerpo,

\par 13 porque tú lo creaste desde el principio para que tuviera piedad de las almas de todos los hombres.

\par 14 Entonces el Señor dijo a Miguel: «Levántate y ve a ver a Abraham, y quédate con él.

\par 15 Y todo lo que le veas comer, come también, y dondequiera que duerma, duerme allí también.

\par 16 Porque pondré en sueños el pensamiento de la muerte de Abraham en el corazón de su hijo Isaac.

\chapter{25}

\par 1 Entonces Miguel entró aquella tarde en casa de Abraham y los encontró preparando la cena, comieron y bebieron y se alegraron.

\par 2 Y Abraham dijo a su hijo Isaac: «Levántate, hijo mío, extiende el lecho del hombre para que duerma y pon la lámpara sobre el candelero».

\par 3 E Isaac hizo lo que su padre le había ordenado,

\par 4 Isaac dijo a su padre: «Yo también voy a dormir junto a ti».

\par 5 Abraham le respondió: «No, hijo mío, para que no seamos una molestia para este hombre, sino ve a tu aposento y duerme».

\par 6 E Isaac, no queriendo desobedecer la orden de su padre, se fue y durmió en su cámara.

\chapter{26}

\par 1 Y sucedió que alrededor de la hora séptima de la noche, Isaac se despertó y llegó a la puerta de la cámara de su padre, gritando y diciendo: «Abre, padre, para que pueda tocarte antes de que te aparten de mí».

\par 2 Abraham se levantó y le abrió, y Isaac entró y, llorando, se colgó del cuello de su padre y lo besó con lamentos.

\par 3 Y Abraham lloró junto con su hijo, y Miguel los vio llorar y lloró también.

\par 4 Y Sara, oyéndoles llorar, los llamó desde su alcoba:

\par 5 diciendo: «Señor mío Abraham, ¿por qué lloras así? ¿Te ha dicho el extraño que Lot, el hijo de tu hermano, ha muerto? ¿O nos ha sucedido algo más?

\par 6 Miguel respondió y dijo a Sara: «No, Sara, no he traído noticias de Lot, pero sabía de toda tu bondad de corazón, que en esto superas a todos los hombres sobre la tierra, y el Señor se ha acordado de ti».

\par 7 Entonces Sara dijo a Abraham: ¿Cómo te atreves a llorar cuando el hombre de Dios ha entrado a ti?

\par 8 ¿Y por qué derraman lágrimas vuestros ojos, porque hoy hay gran alegría? Abraham le dijo:

\par 9 ¿Cómo sabes que éste es un hombre de Dios?

\par 10 Sara respondió y dijo: «Porque digo y declaro que este es uno de los tres hombres que fueron hospedados por nosotros en el roble de Mamre, cuando uno de los sirvientes fue y trajo un cabrito y lo mataste,

\par 11 y me dijo: Levántate y prepárate para que comamos con estos hombres en nuestra casa.

\par 12 Abraham respondió y dijo: «Bien has percibido, oh mujer,

\par 13 Porque también yo, cuando le lavé los pies, supe en mi corazón que estos eran los pies que había lavado en la encina de Mamre, y cuando comencé a preguntarle sobre su viaje, él me dijo: Voy a preservar Lot, tu hermano, de entre los hombres de Sodoma, y ​​entonces supe el misterio».


\chapter{27}

\par 1 Entonces Abraham dijo a Miguel: Dime, hombre de Dios,

\par 2 y muéstrame por qué has venido aquí.

\par 3 Y Miguel dijo: «Tu hijo Isaac te lo mostrará».

\par 4 Y Abraham dijo a su hijo: «Hijo mío, amado, cuéntame lo que has visto hoy en tu sueño y te asustaste. Cuéntamelo».

\par 5 Isaac respondió a su padre: «Vi en mi sueño el sol y la luna, y había una corona sobre mi cabeza,

\par 6 Y vino del cielo un hombre grande y resplandeciente como la luz que llaman el padre de la luz.

\par 7 Quitó el sol de mi cabeza, pero dejó sus rayos conmigo.

\par 8 Y lloré y dije: Te ruego, Señor mío, que no quites la gloria de mi cabeza, ni la luz de mi casa, ni toda mi gloria.

\par 9 Y el sol, la luna y las estrellas se lamentaban, diciendo: No nos quites la gloria de nuestro poder.

\par 10 Y el hombre resplandeciente respondió y me dijo: No llores porque tomo la luz de tu casa, porque ella pasa de las tribulaciones al descanso, de un estado bajo a uno alto;

\par 11 lo levantan de lo angosto a lo ancho; lo elevan de las tinieblas a la luz.

\par 12 Y le dije: Señor, te ruego que lleves también los rayos.

\par 13 Me dijo: El día tiene doce horas y luego tomaré todos los rayos.

\par 14 Mientras el hombre resplandeciente decía esto, vi el sol de mi casa ascender al cielo, pero ya no vi esa corona.

\par 15 y ese sol era como tú, mi padre».

\par 16 Y Miguel dijo a Abraham: Tu hijo Isaac ha dicho la verdad, porque irás y serás llevado a los cielos,

\par 17 pero vuestro cuerpo permanecerá en la tierra hasta que se cumplan siete mil edades, porque entonces resucitará toda carne.

\par 18 Ahora pues, Abraham, pon en orden tu casa y a tus hijos, porque has oído plenamente lo que está decretado acerca de ti.

\par 19 Abraham respondió y dijo a Miguel: «Te ruego, Señor, que si me aparto de mi cuerpo, he deseado ser recogido en mi cuerpo para poder ver las criaturas que el Señor mi Dios ha creado en el cielo. y en la tierra».

\par 20 Miguel respondió y dijo: «Esto no me corresponde a mí hacer, pero iré y se lo contaré al Señor, y si me lo ordena, os mostraré todas estas cosas».

\chapter{28}

\par 1 Entonces Miguel subió al cielo y habló delante del Señor acerca de Abraham:

\par 2 y el Señor respondió a Miguel: «Ve y toma a Abraham en el cuerpo, y muéstrale todas las cosas, y todo lo que él te diga, haz con él como a mi amigo».

\par 3 Entonces Miguel salió, tomó a Abraham en cuerpo sobre una nube y lo llevó al río Océano.

\chapter{29}

\par \textit{Versión del Capítulo 10 en la Versión 1}

\par 1 Y cuando Abraham vio el lugar del juicio, la nube lo descendió sobre el firmamento de abajo,

\par 2 Y Abraham, mirando hacia la tierra, vio a un hombre que cometía adulterio con una mujer desposada.

\par 3 Entonces Abraham, volviéndose, dijo a Miguel: «¿Ves esta maldad? Pero, Señor, envía fuego del cielo para consumirlos».

\par 4 Y al momento descendió fuego y los consumió,

\par 5 porque el Señor le había dicho a Miguel: Haz todo lo que Abraham te pida que hagas por él.

\par 6 Abraham volvió a mirar y vio a otros hombres insultando a sus compañeros,

\par 7 y dijo: «Que se abra la tierra y se los trague».

\par 8 y mientras hablaba, la tierra se los tragó vivos.

\par 9 Otra vez la nube lo llevó a otro lugar, y Abraham vio a unos que iban a un lugar desierto para cometer un asesinato.

\par 10 y dijo a Miguel: «¿Ves esta maldad? Pero que salgan del desierto las fieras y los despedacen».

\par 11 Y en aquel mismo momento salieron del desierto fieras y los devoraron.

\par 12 Entonces el Señor Dios habló a Miguel, diciendo: Haz volver a Abraham a su casa y no le dejes recorrer toda la creación que yo he hecho, porque no tiene compasión de los pecadores.

\par 13 pero tengo compasión de los pecadores, para que se conviertan y vivan, se arrepientan de sus pecados y sean salvos.

\chapter{30}

\par \textit{Versión del Capítulo 11 en la Versión 1}

\par 1 Y Abraham miró y vio dos puertas, una pequeña y otra grande,

\par 2 y entre las dos puertas había un hombre sentado sobre un trono de gran gloria, y una multitud de ángeles alrededor de él,

\par 3 y él lloraba y volvía a reír, pero su llanto era siete veces mayor que su risa.

\par 4 Y Abraham dijo a Miguel: ¿Quién es éste que está sentado entre las dos puertas en gran gloria? ¿A veces ríe y a veces llora, y su llanto supera siete veces a su risa?

\par 5 Entonces Miguel dijo a Abraham: «¿No sabes quién es?»

\par 6 Y él dijo: «No, Señor».

\par 7 Y Miguel dijo a Abraham: «¿Ves estas dos puertas, la pequeña y la grande?

\par 8 Estos son los que conducen a la vida y a la destrucción.

\par 9 Este hombre que está sentado entre ellos es Adán, el primer hombre que creó el Señor,

\par 10 y ponlo en este lugar para que vea toda alma que sale del cuerpo, y que todas proceden de él.

\par 11 Por tanto, cuando lo veáis llorar, sabed que ha visto muchas almas llevadas a la perdición.

\par 12 pero cuando lo ves reír, ha visto muchas almas ser conducidas a la vida.

\par 13 ¿Ves cómo su llanto supera a su risa? Puesto que ve que la mayor parte del mundo es conducida a través de la puerta ancha hacia la destrucción, su llanto supera siete veces su risa».

\chapter{31}

\par 1 Y Abraham dijo: Y el que no puede entrar por la puerta estrecha, ¿no podrá entrar en la vida?

\par 2 Entonces Abraham lloró y dijo: ¡Ay de mí! ¿Qué haré?

\par 3 Porque soy un hombre corpulento, ¿y cómo podré entrar por la puerta estrecha, por la que no puede entrar un niño de quince años?

\par 4 Respondió Miguel y dijo a Abraham: «No temas, padre, ni te aflijas, porque entrarás por ella sin obstáculos, junto con todos los que son como tú».

\par 5 Y mientras Abraham estaba de pie y maravillado, he aquí un ángel del Señor que llevaba a la destrucción a sesenta mil almas de pecadores.

\par 6 Y Abraham dijo a Miguel: ¿Todos estos van a la perdición?

\par 7 Y Miguel le dijo: «Sí, pero vayamos y busquemos entre estas almas, si hay entre ellas un solo justo».

\par 8 Y cuando fueron, encontraron un ángel que tenía en su mano el alma de una mujer de entre estas sesenta mil, porque había encontrado que sus pecados pesaban lo mismo que todas sus obras, y no estaban ni en movimiento ni en reposo. pero en un estado intermedio;

\par 9 pero a las demás las llevó a la perdición.

\par 10 Abraham dijo a Miguel: Señor, ¿es éste el ángel que saca las almas del cuerpo o no? Miguel respondió y dijo: «Esto es muerte, y los lleva al lugar del juicio, para que el juez los juzgue».

\chapter{32}

\par 1 Y Abraham dijo: «Señor mío, te ruego que me lleves al lugar del juicio para que yo también pueda ver cómo son juzgados».

\par 2 Entonces Miguel tomó a Abraham sobre una nube y lo llevó al paraíso.

\par 3 y cuando llegó al lugar donde estaba el juez, vino el ángel y le entregó esa alma.

\par 4 Y el alma dijo: Señor, ten misericordia de mí.

\par 5 Y el juez dijo: ¿Cómo tendré misericordia de ti, si tú no tuviste misericordia de la hija que tuviste, fruto de tu vientre? ¿Por qué la mataste?

\par 6 Ella respondió: «No, Señor, yo no he matado, sino que mi hija ha mentido sobre mí».

\par 7 Pero el juez ordenó que viniera el que escribía las actas,

\par 8 y he aquí querubines que llevaban dos libros. Y estaba con ellos un hombre de gran estatura, que tenía en su cabeza tres coronas,

\par 9 y una corona era más alta que las otras dos. Éstas se llaman coronas de testimonio.

\par 10 Y el hombre tenía en su mano una pluma de oro, y el juez le dijo: Muestra el pecado de esta alma.

\par 11 Y aquel hombre, abriendo uno de los libros de los querubines, buscó el pecado del alma de la mujer y lo encontró.

\par 12 Y el juez dijo: «Miserable alma, ¿por qué dices que no has matado?

\par 13 ¿No fuiste tú, después de la muerte de tu marido, y cometiste adulterio con el marido de tu hija, y la mataste?

\par 14 Y él también la condenó por sus otros pecados y por todo lo que había hecho desde su juventud.

\par 15 Al oír estas cosas, la mujer gritó y dijo: ¡Ay de mí! Todos los pecados que cometí en el mundo los olvidé, pero aquí no fueron olvidados.

\par 16 Luego se la llevaron también a ella y la entregaron a los verdugos.

\chapter{33}

\par 1 Y Abraham dijo a Miguel: Señor, ¿quién es este juez y quién es el otro que condena los pecados?

\par 2 Entonces Miguel dijo a Abraham: «¿Ves al juez? Este es Abel, el primero en testificar, y Dios lo trajo acá para juzgar,

\par 3 Y el que aquí da testimonio es el maestro del cielo y de la tierra, y escriba de justicia, Enoc,

\par 4 porque el Señor los envió acá para escribir los pecados y las justicias de cada uno.

\par 5 Abraham dijo: «¿Y cómo puede Enoc soportar el peso de las almas, sin haber visto la muerte? ¿O cómo podrá dar sentencia a todas las almas?»

\par 6 Miguel dijo: Si dicta sentencia sobre las almas, no le está permitido; pero el propio Enoc no dicta sentencia,

\par 7 pero es el Señor quien lo hace, y no tiene más que hacer que escribir.

\par 8 Porque Enoc oró al Señor diciendo: No deseo, Señor, dictar sentencia sobre las almas, para no ser gravoso a nadie;

\par 9 y el Señor dijo a Enoc: Te ordenaré que escribas los pecados del alma que hace expiación y entrará en la vida,

\par 10 y si el alma no hace expiación y no se arrepiente, encontrará sus pecados escritos y será arrojada al castigo. Y alrededor de la hora novena, Miguel llevó a Abraham a su casa. Pero Sara su esposa, al no ver lo que había sido de Abraham, se consumió de dolor y entregó el espíritu, y después del regreso de Abraham la encontró muerta y la enterró.

\chapter{34}

\par 1 Pero cuando se acercaba el día de la muerte de Abraham, el Señor Dios dijo a Miguel:

\par 2 La muerte no se atreverá a acercarse para quitarle el alma a mi siervo, porque es mi amigo, sino que ve y adorna a la muerte con gran hermosura y envíala así a Abraham, para que pueda verla con sus ojos.

\par 3 Y luego Miguel, tal como se le había ordenado, adornó a la muerte con gran belleza y la envió así a Abraham para que pudiera verla.

\par 4 Y se sentó junto a Abraham, y Abraham, al ver la Muerte sentada junto a él, tuvo mucho miedo.

\par 5 Y la Muerte dijo a Abraham: «¡Salve, alma santa! ¡Salve, amiga del Señor Dios! ¡Salve, consuelo y entretenimiento de los viajeros!

\par 6 Y Abraham dijo: «De nada, siervo del Altísimo. Dios. Te lo ruego, dime quién eres; y entrando en mi casa, come y bebe, y apartaos de mí, porque desde que te vi sentado cerca de mí, mi alma está turbada.

\par 7 Porque no soy digno de acercarme a ti, porque tú eres un espíritu exaltado y yo soy carne y sangre,

\par 8 y por eso no puedo soportar tu gloria, porque veo que tu hermosura no es de este mundo.

\par 9 Y la Muerte dijo a Abraham: «Te digo que en toda la creación que Dios ha hecho, no se ha encontrado uno como tú,

\par 10 porque ni siquiera el Señor mismo, buscándolo, ha encontrado a nadie así en toda la tierra.

\par 11 Y Abraham dijo a la Muerte: «¿Cómo te atreves a mentir? Porque veo que tu belleza no es de este mundo».

\par 12 Y la Muerte dijo a Abraham: «No pienses, Abraham, que esta belleza es mía, ni que así vengo a todos los hombres. No, pero si alguno es justo como tú, así tomo coronas y vengo a él, pero si es pecador vengo en gran corrupción, y de su pecado hago una corona para mi cabeza, y los sacudo con gran temor, de modo que quedaron consternados».

\par 13 Entonces Abraham le preguntó: «¿Y de dónde viene tu hermosura?»

\par 14 Y la Muerte dijo: No hay otro más corrupto que yo.

\par 15 Abraham le dijo: «¿Y eres tú realmente aquel a quien llaman Muerte?»

\par 16 Él le respondió y dijo: «Yo soy el nombre amargo. Estoy llorando...»

\chapter{35}

\par 1 Y Abraham dijo a la Muerte: «Muéstranos tu corrupción».

\par 2 Y la Muerte manifestó su corrupción; y tenía dos cabezas,

\par 3 uno tenía cara de serpiente y por ella algunos mueren al instante por los áspides,

\par 4 y la otra cabeza era como una espada; por ella algunos mueren a espada como a arcos.

\par 5 Aquel día los siervos de Abraham murieron por miedo a la muerte, y Abraham, al verlos, oró al Señor y él los resucitó.

\par 6 Pero Dios volvió y, como en un sueño, quitó el alma de Abraham, y el arcángel Miguel la llevó al cielo.

\par 7 E Isaac sepultó a su padre junto a su madre Sara, glorificando y alabando a Dios, porque a él es debida la gloria, la honra y la adoración del Padre, del Hijo y del Espíritu Santo, ahora y siempre y por toda la eternidad. Amén.

\end{document}