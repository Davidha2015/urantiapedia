\begin{document}

\title{Apocalipsis de Adán}

\chapter{1}

\par 1 La revelación del origen de Adán contada a su hijo Set

\par 2 La revelación que Adán enseñó a su hijo Set el año setecientos, diciendo: Escucha mis palabras, hijo mío Set. Cuando Dios me creó de la tierra junto con Eva, tu madre, anduve con ella en una gloria que ella había visto en el Aeón del que habíamos surgido. Ella me enseñó una palabra de conocimiento del Dios eterno. Y éramos parecidos a los grandes ángeles, porque éramos superiores al Dios que nos había creado, y a los poderes que están con él, a quienes no conocíamos.

\par 3 Entonces Dios, el gobernante de los Aeones y los poderes, nos dividió en ira. Luego nos convertimos en dos Aeones. Y la gloria en nuestros corazones nos abandonó, a mí y a tu madre Eva, junto con el primer conocimiento que sopló dentro de nosotros. Y la gloria huyó de nosotros; no de este Aeón del cual habíamos salido yo y Eva tu madre. Pero el conocimiento entró en la semilla de grandes Aeones. Por eso yo mismo te he llamado por el nombre de aquel hombre que es la semilla de la gran generación o de quien viene. Después de aquellos días el conocimiento eterno del Dios de la verdad se retiró de mí y de tu madre Eva. Desde entonces aprendimos acerca de las cosas muertas, como los hombres. Entonces reconocimos al Dios que nos había creado. Porque no éramos ajenos a sus poderes. Y le servimos en temor y esclavitud. Y después de estos acontecimientos nuestro corazón se oscureció. Ahora dormí en el pensamiento de mi corazón.

\par 4 Y vi delante de mí a tres hombres cuya semejanza no pude reconocer, porque no provenían de los poderes del Dios que nos había creado. Superaron la gloria y a los hombres, diciéndome: «Levántate, Adán, del sueño de la muerte y oye hablar del Eón y de la simiente de aquel hombre a quien ha llegado la vida, que vino de ti y de Eva, tu esposa».

\par 5 Cuando escuché estas palabras de los grandes hombres que estaban delante de mí, entonces suspiramos yo y Eva en nuestros corazones. Y el Señor, el Dios que nos había creado, se presentó ante nosotros. Él nos dijo , «Adán, ¿por qué ambos suspiraban en sus corazones? ¿No sabéis que yo soy el Dios que os creó? Y soplé en vosotros espíritu de vida como alma viviente». Entonces la oscuridad cubrió nuestros ojos.

\par 6 Entonces el Dios que nos creó creó un hijo de sí mismo y de Eva, tu madre. Conocí el dulce deseo de tu madre. Entonces el vigor de nuestro conocimiento eterno se destruyó en nosotros, y la debilidad nos persiguió. Por eso los días de nuestra vida se hicieron pocos, porque sabía que estaba bajo el dominio de la muerte.

\par 7 Ahora bien, hijo mío Set, te revelaré las cosas que me revelaron aquellos hombres que vi antes de mí al principio, después de que haya cumplido los tiempos de esta generación y los años de la generación.

\par 8 Porque las lluvias de Dios Todopoderoso se derramarán para destruir toda la carne de Dios Todopoderoso, para destruir toda carne de la tierra por medio de lo que la rodea, junto con aquellos. de la simiente de los hombres a quienes pasó la vida del conocimiento, que vino de mí y de Eva, tu madre. Porque eran extraños para él. Después vendrán los grandes ángeles en nubes altas, quienes traerán a esos hombres al lugar. donde el espíritu de vida habita allí en gloria. Entonces toda la multitud de la carne quedará atrás en las aguas.

\par 9 Entonces Dios descansará de su ira, y arrojará su poder sobre las aguas, y dará poder de su poder a sus hijos y a sus mujeres por medio del arca, junto con los animales que él quiera, y las aves. del cielo, al cual llamó y soltó sobre la tierra. Y Dios le dirá a Noé, a quien las generaciones llamarán Deucalión: «He aquí, yo te he protegido en el arca junto con tu esposa y tus hijos y sus esposas y sus animales y las aves del cielo, que tú llamaste y soltaste sobre el tierra. Por tanto, te daré la tierra a ti, a ti y a tus hijos. Como rey la gobernarás, tú y tus hijos. Y no saldrá de ti descendencia de los hombres que no estarán en mi presencia en otra gloria.»

\par 10 Entonces se convertirán en la nube de una gran luz. Vendrán aquellos hombres que han sido expulsados ​​del conocimiento del gran Aeón y de los ángeles. Se presentarán ante Noé y los Aeones. Y Dios le dirá a Noé , «¿Por qué os habéis apartado de lo que os dije? Has creado otra generación para que desprecies mi poder». Entonces Noé dirá: «Testificaré ante tu poder que la generación de estos hombres no vino de mí ni de mis hijos».

\par 11 Y traerá a esos hombres a su propia tierra y les construirá una morada santa. Y serán llamados con ese nombre y habitarán allí seiscientos años con un conocimiento de incorruptibilidad. Y los ángeles de la gran Luz morarán con ellos. Ninguna mala acción morará en sus corazones, sino sólo el conocimiento del Dios verdadero.

\par 12 Noé dividirá toda la tierra entre sus hijos Cam, Jafet y Sem. Y les dirá: «Hijos míos, escuchad mis palabras. He aquí, yo he repartido la tierra entre vosotros. Pero sírvele con temor y esclavitud todos los días de tu vida. No dejes que tu descendencia se aparte de la faz de Dios todopoderoso. Mi semilla será agradable delante de ti y delante de tu poder. Séllalo con tu mano fuerte con temor y mandamiento, para que toda la simiente que de mí salió no se incline lejos de ti y de Dios todopoderoso, sino que sirva con humildad y temor de su conocimiento».

\par 13 Luego vendrán otros del linaje de Cam y Jafet, cuatrocientos mil hombres, y entrarán en otra tierra y morarán con aquellos hombres que surgieron del gran conocimiento eterno. Porque la sombra de su poder protegerá a los que han peregrinado con ellos de toda cosa mala y de todo deseo inmundo. Entonces la descendencia de Cam y de Jafet formará doce reinos, y también su descendencia entrará en el reino de otro pueblo, y tomen consejo de los grandes eones de la imperecebilidad. E irán a Sacla, su Dios. Los que entran a los poderes, acusando a los grandes hombres que están en su gloria.

\par 14 Dirán a Sacla: «¿Cuál es el poder de estos hombres que estuvieron delante de ti, que fueron tomados de la descendencia de Cam y Jafet, que serán cuatrocientos mil hombres? Han sido recibidos en otro eón del cual habían surgido, y han derribado toda la gloria de tu poder y el dominio de tu mano. Porque la simiente de Noé a través de su hijo ha hecho toda tu voluntad, y también todos los poderes en los Aeones sobre los cuales tu poder gobierna, mientras que tanto esos hombres como los que son extranjeros en su gloria no han hecho tu voluntad. has desviado a toda tu multitud».

\par 15 Entonces el Dios de los Eones les dará algunos de los que le sirven. Llegarán a esa tierra donde estarán los grandes hombres que no han sido contaminados, ni serán contaminados por ningún deseo. Porque su alma no vino de una mano contaminada, sino que vino de un gran mandamiento del ángel eterno. Entonces fuego, azufre y asfalto serán arrojados sobre esos hombres, y fuego y niebla cegadora caerán sobre esos Eones, y los ojos de los poderes de los iluminadores se oscurecerán, y los Eones no los verán en esos días. Y los grandes descenderán nubes de luz, y otras nubes de luz descenderán sobre ellos desde los grandes Aeones.

\par 16 Abrasax, Sablo y Gamaliel descenderán y sacarán a esos hombres del fuego y de la ira, y los llevarán por encima de los Aeones y los Gobernantes de los poderes, y los llevarán allí, con los santos ángeles y los Aeones. Los hombres serán como esos ángeles, porque no les son extraños, sino que obran en la semilla incorruptible.

\par 17 Una vez más, por tercera vez, el iluminador del conocimiento pasará con gran gloria, para dejar algo de la simiente de Noé y de los hijos de Cam y Jafet, para dejar para sí árboles frutales. Y él redimirá sus almas desde el día de la muerte. Porque toda la creación que surgió de la tierra muerta estará bajo la autoridad de la muerte. Pero aquellos que reflexionen en su corazón sobre el conocimiento del Dios eterno no perecerán. Porque No han recibido espíritu sólo de este reino, sino que lo han recibido de uno de los ángeles eternos. El iluminador vendrá. Y hará señales y prodigios para despreciar a los poderes y a su gobernante.

\par 18 Entonces el Dios de los poderes se perturbará y dirá: «¿Cuál es el poder de este hombre que es más alto que nosotros?» Entonces despertará una gran ira contra ese hombre. Y la gloria se retirará y habitará en las casas santas que ella misma ha elegido. Y las potestades no la verán con sus ojos, ni tampoco verán al iluminador. Entonces castigar la carne del hombre sobre quien ha venido el espíritu santo.

\par 19 Entonces los ángeles y todas las generaciones de los poderes usarán el nombre erróneamente y preguntarán: «¿De dónde vino el error?» o «¿De dónde vienen las palabras de engaño, que todos los poderes no han podido descubrir?»

\par 20 Ahora el primer reino dice de él.....[]
\par 21 Fue nutrido en los cielos.
\par 22 Él recibió la gloria y el poder de aquel.
\par 23 Llegó al seno de su madre.
\par 24 Y así llegó al agua.
\par 25 Y el segundo reino dice de él que vino de un gran profeta. Y vino un pájaro, tomó al niño que había nacido y lo llevó a un monte alto. Y fue alimentado por el pájaro del cielo. Un ángel apareció allí y le dijo: «¡Levántate! Dios os ha dado gloria»

\par 26 Él recibió gloria y fuerza.
\par 27 Y así llegó al agua.
\par 28 El tercer reino dice de él que nació de un vientre virgen. Fue expulsado de su ciudad, él y su madre; fue llevado a un lugar desierto.

\par 29 Allí fue alimentado.
\par 30 Y así llegó al agua.
\par 31 El cuarto reino dice de él que vino de una virgen... Salomón la buscó, él, Phersalo, Sauel y sus ejércitos que habían sido enviados. El propio Salomón envió su ejército de demonios a buscar a la virgen. Y no encontraron a la que buscaban, sino a la virgen que se les había dado. A ella fue a quien buscaron. Salomón la tomó. La virgen quedó embarazada y dio a luz allí al niño.

\par 32 Ella lo alimentó en los confines del desierto. Cuando fue nutrido, recibió gloria y poder de la semilla de la que fue engendrado. Y así llegó al agua.

\par 33 Y el quinto reino dice de él que vino de una gota del cielo, que fue arrojado al mar, que el abismo lo recibió, lo engendró y lo llevó al cielo.
\par 34 Él recibió gloria y poder.
\par 35 Y así llegó al agua.
\par 36 Y el sexto reino dice que [.....] hasta el Aeon que está abajo, para recoger flores. Ella quedó embarazada del deseo de las flores. Ella lo dio a luz en ese lugar. Los ángeles del jardín de flores lo alimentaron. Allí recibió gloria y poder. Y así llegó al agua.

\par 37 Y el séptimo reino dice de él que es una gota. Vino del cielo a la tierra. Los dragones lo llevaron a las cuevas. Se convirtió en un niño. Un espíritu vino sobre él y lo llevó a lo alto, al lugar donde había salido una gota. Allí recibió gloria y poder. Y así llegó al agua.

\par 38 Y el octavo reino dice de él que una nube cayó sobre la tierra y envolvió una roca. De allí salió. Los ángeles que estaban sobre la nube lo alimentaron. Allí recibió gloria y poder. Y así llegó al agua.

\par 39 Y el noveno reino dice de él que una de las nueve Musas se separó. Ella llegó a una montaña alta y permaneció allí sentada algún tiempo, de modo que deseaba estar sola para volverse andrógina. Ella cumplió su deseo y se convirtió en embarazada de su deseo. Él nació. Los ángeles que estaban sobre el deseo lo alimentaron. Allí recibió gloria y poder.
\par 40 Y así llegó al agua.
\par 41 El décimo reino dice de él que su dios amaba una nube de deseo. Engendró en su mano y arrojó sobre la nube sobre él parte de la gota, y nació. Allí recibió gloria y poder.
\par 42 Y así llegó al agua.
\par 43 El undécimo reino dice de él que el padre deseaba a su propia hija. Ella misma quedó embarazada de su padre. Ella arrojó [...] Allí el ángel lo alimentó.
\par 44 Y así llegó al agua.

\par 45 Y el duodécimo reino dice de él que vino de dos iluminadores. Allí se alimentó.
\par 46 Él recibió gloria y poder.
\par 47 Y así llegó al agua.

\par 48 Y el decimotercer reino dice de él que cada nacimiento de su gobernante es una palabra. Y esta palabra recibió allí un mandato. Recibió gloria y poder.
\par 49 Y así llegó al agua.
\par 50 Pero la generación que no tiene rey dice que Dios lo eligió entre todos los eones. Hizo que llegara a existir en él un conocimiento del inmaculado de la verdad. Él dijo: «De un aire extraño, de un gran Aeón, surgió el gran iluminador. E hizo brillar a la generación de aquellos hombres que él había elegido para sí, para que brillaran sobre todo el Aeón».

\par 51 Entonces la descendencia, aquellos que recibirán su nombre sobre el agua y el de todos ellos, lucharán contra el poder. Y una nube de oscuridad vendrá sobre ellos.

\par 52 Entonces los pueblos gritarán a gran voz, diciendo: «¡Bienaventurada el alma de esos hombres porque han conocido a Dios con conocimiento de la verdad! Vivirán para siempre, porque no se han corrompido por sus deseos, junto con los ángeles, ni han realizado las obras de los poderes, sino que han permanecido en su presencia con un conocimiento de Dios como una luz que sale del fuego. y sangre. Pero hemos cometido todos los actos de las potencias sin sentido. Nos hemos jactado de la transgresión de todas nuestras obras. Hemos clamado contra el Dios de verdad, porque toda su obra es eterna. Estas van contra nuestro espíritu. Porque ahora sabemos que nuestras almas morirán.

\par 53 Entonces les llegó una voz que decía: «Micheu, Michar y Mnesinous, que están a cargo del santo bautismo y del agua viva, ¿por qué clamaban contra el Dios vivo con voces sin ley y lenguas sin ley sobre ellos, y almas sin ley? lleno de sangre y malas acciones? Estás lleno de obras que no son de la verdad, pero tus caminos están llenos de alegría y regocijo. Habiendo contaminado el agua de la vida, la has atraído dentro de la voluntad de los poderes a quienes te han sido dados para servirles. Y vuestro pensamiento no es como el de aquellos hombres a quienes perseguís. Sus frutos no se marchitan. Pero serán conocidos hasta los grandes Aeones, porque las palabras que habían guardado, del Dios de los Aeones, no fueron encomendadas al libro, ni fueron escritos. Pero los traerán seres angélicos, que todas las generaciones de los hombres no conocerán. Porque estarán en la alta montaña, sobre la roca de la verdad. Por eso serán llamadas 'Las palabras de la Imperecedera y de la Verdad' para aquellos que conocen al Dios eterno en sabiduría, conocimiento y enseñanza de los ángeles para siempre, porque él conoce todas las cosas.»

\par 54 Estas son las revelaciones que Adán le hizo saber a su hijo Set, y su hijo enseñó sobre ellas a su descendencia. Este es el conocimiento oculto de Adán, que le dio a Set, que es el santo bautismo de quienes conocen el conocimiento eterno a través de los nacidos del verbo y de los iluminadores imperecederos, que vinieron de la simiente santa: Yeseo. Mazareus Yessedekeus, el agua viva.


\end{document}