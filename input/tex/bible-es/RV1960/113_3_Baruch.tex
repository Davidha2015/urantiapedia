\begin{document}

\title{3 Baruc}

\chapter{1}

\par \textit{Prólogo}

\par 1 Narración y revelación de Baruc acerca de las cosas inefables que vio por orden de Dios. Bendito seas, oh Señor.

\par 2 Revelación de Baruc, que estaba junto al río Gel llorando por el cautiverio de

\par 3 Jerusalén, cuando también Abimelec fue preservado por la mano de Dios, en la granja de Agripa. Y estaba así sentado a las hermosas puertas, donde yacía el Lugar Santísimo.

\chapter{1b}

\par \textit{Prólogo}

\par 1 En verdad, yo, Baruc, estaba llorando en mi mente y entristecido por el pueblo, y eso

\par 2 Dios permitió al rey Nabucodonosor destruir su ciudad, diciendo: Señor, ¿por qué incendiaste tu viña y la devastaste? ¿Por qué hiciste esto? Y por qué, Señor, no nos pagaste con otro castigo. , sino que nos entregaste a naciones como éstas, para que

\par 3 reprocharnos y decir: ¿Dónde está su Dios? Y he aquí, mientras yo lloraba y decía estas cosas, vi un ángel del Señor que venía y me decía: Entiende, oh hombre, muy amado, y no te preocupes tanto acerca de la salvación de Jerusalén, porque así dice el Señor Dios:

\par 4 el Todopoderoso. Porque Él me envió delante de ti, para darte a conocer y mostrarte todas (las cosas)

\par 5 [...]

\par 6 de Dios. Porque tu oración fue oída delante de él, y entró en los oídos del Señor Dios. Y cuando me dijo estas cosas, me quedé en silencio. Y el ángel me dijo: Deja de provocar

\par 7 Dios, y te mostraré otros misterios, mayores que estos. Y yo, Baruc, dije: Vive Jehová el Señor, que si me muestras, y oigo una palabra tuya, no hablaré más.

\par 8 Dios aumentará mi juicio en el día del juicio, si hablo más adelante. Y el ángel de los poderes me dijo: Ven, y te mostraré los misterios de Dios.

\chapter{2}

\par \textit{El Primer Cielo.}

\par 1 Y me tomó y me llevó a donde el firmamento estaba firme, y donde había un río que nadie puede cruzar, ni ninguna brisa extraña de todas las que Dios creó. Y me tomó y me llevó al primer cielo, y me mostró una puerta de gran tamaño. Y me dijo: Entremos

\par 2 [...]

\par 3 por él, y entramos como llevados con alas, un camino de unos treinta días. Y me mostró dentro del cielo una llanura; y habitaban en ella hombres con rostros de

\par 4 bueyes, cuernos de ciervo, patas de machos cabríos y ancas de corderos. Y yo, Baruc, pregunté al ángel: Te ruego que me hagas saber cuál es la densidad del cielo por donde viajamos.

\par 5 o cuál es su extensión, o cuál es su llanura, para que también pueda decírselo a los hijos de los hombres. Y el ángel cuyo nombre es Famael me dijo: Esta puerta que ves es la puerta del cielo, y como Grande como es la distancia de la tierra al cielo, tan grande también es su espesor; y además, tan grande como es la distancia (de Norte a Sur, tan grande) es la longitud de la llanura que viste. Y otra vez me dijo el ángel de los poderes: Ven, y te mostraré misterios mayores. Pero

\par 6 [...]

\par 7 Dije: Te ruego que me muestres quiénes son estos hombres. Y me dijo: Estos son los que construyeron la torre de la contienda contra Dios, y el Señor los desterró.

\chapter{3}

\par \textit{El Segundo Cielo.}

\par 1 Y el ángel del Señor me tomó y me llevó al segundo cielo. Y me mostró allí

\par 2 también una puerta como la primera y dijo: Entremos por ella. Y entramos, llevados con alas

\par 3 una distancia de unos sesenta días de viaje. Y me mostró allí también una llanura, y estaba llena de

\par 4 hombres, cuyo aspecto era como el de los perros, y cuyos pies eran como los de los ciervos, y le pregunté

\par 5 el ángel: Te ruego, Señor, que me digas quiénes son éstos. Y él dijo: Estos son los que dieron consejo para construir la torre, porque los que ves expulsaron multitud de hombres y mujeres para hacer ladrillos; entre los cuales, a una mujer que hacía ladrillos no se le permitía ser liberada en la hora del parto, sino que daba a luz mientras hacía ladrillos, y llevaba a su hijo en su delantal, y

\par 6 continuaron haciendo ladrillos. Y se les apareció el Señor y les confundió el habla, cuando

\par 7 había construido la torre hasta una altura de cuatrocientos sesenta y tres codos. Y tomando una barrena, procuraban perforar el cielo, diciendo: Veamos si el cielo es de barro o de

\par 8 de latón o de hierro. Cuando Dios vio esto, no se lo permitió, sino que los hirió con ceguera y confusión de palabra, y les presentó como tú ves.

\chapter{4}

\par \textit{El Tercer Cielo.}

\par 1 Y yo, Baruc, dije: He aquí, Señor, me has mostrado cosas grandes y maravillosas; y ahora

\par 2 muéstrame todas las cosas por amor del Señor. Y el ángel me dijo: Ven, procedamos. (Y partí) con el ángel de aquel lugar como ciento ochenta y cinco días

\par 3 viajes. Y me mostró una llanura y una serpiente, que parecía tener doscientas pletras de largo.

\par 4 Y me mostró el Hades, y su aspecto era oscuro y abominable. Y yo dije,

\par 5 ¿Quién es este dragón, y quién es este monstruo que lo rodea? Y el ángel dijo: El dragón es él.

\par 6 que come los cuerpos de los que viven malvadamente, y de ellos se alimenta. Y este es el Hades, que también se le parece mucho en que también bebe alrededor de un codo de

\par 7 el mar, que no se hunde en absoluto. Baruc dijo: ¿Y cómo (sucede esto)? Y el ángel dijo: Escuchen, el Señor Dios hizo trescientos sesenta ríos, de los cuales el principal de

\par 8 todos son Alfias, Abiro y Gerico; y por ellos el mar no se hunde. Y dije: Te ruego que me muestres cuál es el árbol que extravió a Adán. Y el ángel me dijo: Es la vid que plantó el ángel Sammael, por la cual se enojó el Señor Dios, y lo maldijo a él y a su planta, aunque también por esto no permitió que Adán la tocara, y por eso

\par 9 El diablo, teniendo envidia, lo engañó a través de su vid. [Y yo, Baruc, dije: Puesto que también la vid ha sido causa de tan grande mal, y está bajo el juicio de la maldición de Dios, y era la

\par 10 destrucción de la primera creación, ¿cómo es ahora tan útil? Y el ángel dijo: Bien preguntas. Cuando Dios causó el diluvio sobre la tierra, y destruyó toda carne, y cuatrocientos nueve mil gigantes, y el agua subió quince codos sobre las montañas más altas, entonces el agua entró en el paraíso y destruyó toda flor; pero eliminó totalmente sin límites el rodaje

\par 11 de la vid y échala fuera. Y cuando la tierra surgió del agua, y salió Noé

\par 12 del arca, comenzó a plantar de las plantas que encontraba. Pero encontró también el sarmiento de la vid; y él lo tomó, y pensaba dentro de sí: ¿Qué, pues, es esto? Y me acerqué y le hablé.

\par 13 él las cosas relativas a ello. Y él dijo: ¿La plantaré o qué haré? Ya que Adán fue destruido por causa de esto, no permita que yo también sufra la ira de Dios por causa de ello; y diciendo

\par 14 Oró con estas cosas para que Dios le revelara lo que debía hacer al respecto. Y cuando hubo terminado la oración que duró cuarenta días, y habiendo suplicado muchas cosas y llorado,

\par 15 Dijo: Señor, te ruego que me reveles lo que haré con esta planta. Pero Dios envió a su ángel Sarasael, y le dijo: Levántate, Noé, y planta el sarmiento de la vid, porque así dice el Señor: Su amargura se cambiará en dulzura, y su maldición se convertirá en bendición, y lo que es producido de él se convertirá en la sangre de Dios; y así como por ella el género humano obtuvo condenación, así también por Jesucristo Emanuel recibirán en él la

\par 16 llamada a lo alto y entrada al paraíso]. Sepa, pues, oh Baruc, que así como Adán por este mismo árbol obtuvo la condenación y fue despojado de la gloria de Dios, así también los hombres que ahora beben insaciablemente el vino que de él se engendra, transgreden peor que Adán, y están lejos de el

\par 17 gloria de Dios, y se entregan al fuego eterno. Porque (ningún) bien surge de ello. Porque los que lo beben hasta saciarse hacen estas cosas: ni el hermano se compadece del hermano, ni el padre del hijo, ni los hijos de los padres, sino que del beber vino vienen todos los males, como homicidios, adulterios, fornicaciones, perjurios, robos y cosas por el estilo. Y con ello no se establece nada bueno.

\chapter{5}

\par 1 Y yo, Baruc, dije al ángel:

\par 2 Déjame preguntarte una cosa, Señor. Desde que me dijiste

\par 3 que el dragón bebe un codo del mar, dime también qué grande es su vientre. Y el ángel dijo: Su vientre es el Hades; y por más que trescientos hombres lanzan una plomada, así de grande es su vientre. Ven, pues, para que yo te muestre también obras mayores que éstas.

\chapter{6}

\par 1 Y él me tomó y me llevó hacia donde sale el sol;

\par 2 y me mostró un carro y cuatro, debajo de los cuales ardía un fuego, y en el carro estaba sentado un hombre que llevaba una corona de fuego, (y) el carro (era) tirado por cuarenta ángeles. Y he aquí un pájaro dando vueltas delante del sol, como a las nueve

\par a 3 codos de distancia. Y le dije al ángel: ¿Qué es esta ave? Y él me dijo: Esta es la

\par 4 [...]

\par 5 guardián de la tierra. Y dije: Señor, ¿cómo es él el guardián de la tierra? Enséñame. Y el ángel me dijo: Esta ave vuela junto al sol, y extendiendo sus alas recibe su fuego.

\par 6 rayos. Porque si no los recibiera, no se conservaría el género humano, ni ningún otro

\par 7 criaturas vivientes. Pero Dios designó a esta ave para eso. Y extendió sus alas, y vi en su ala derecha letras muy grandes, tan grandes como el espacio de una era, del tamaño de unas cuatro

\par 8 mil modii; y las letras eran de oro. Y el ángel me dijo: Léelos. Y leo

\par 9 y corrían así: Ni la tierra ni el cielo me sacarán, sino las alas de fuego me sacarán. Y dije: Señor, ¿qué es esta ave, y cómo se llama? Y el ángel me dijo: Se llama su nombre.

\par 10 [...]

\par 11 Fénix. (Y yo dije): ¿Y qué come? Y me dijo: El maná del cielo y

\par 12 el rocío de la tierra. Y dije: ¿El pájaro excreta? Y me dijo: Excreta un gusano, y el excremento del gusano es canela, que usan los reyes y los príncipes. Pero espera y lo harás

\par 13 veréis la gloria de Dios. Y mientras conversaba conmigo, se escuchó como un trueno, y el lugar donde estábamos se estremeció. Y le pregunté al ángel: Mi Señor, ¿qué es este sonido? Y el ángel me dijo: Incluso ahora los ángeles están abriendo las trescientas sesenta y cinco puertas.

\par 14 del cielo, y la luz se separa de las tinieblas. Y vino una voz que dijo: Luz

\par 15 dador, da al mundo resplandor. Y cuando oí el ruido del pájaro, dije: Señor, ¿qué es esto?

\par 16 ruido Y él dijo: Éste es el pájaro que despierta del sueño a los gallos en la tierra. Porque lo que hacen los hombres por la boca, así también el gallo significa para los del mundo, en su propia palabra. Porque los ángeles preparan el sol y el gallo canta.

\chapter{7}

\par 1 Y dije: ¿Y dónde comienza el sol sus trabajos después de que canta el gallo?

\par 2 Y el ángel me dijo: Escucha, Baruc: todo lo que te he mostrado está en el primer y segundo cielo, y en el tercer cielo pasa el sol y alumbra al mundo. Pero espera, y tú

\par 3 verás la gloria de Dios. Y mientras conversaba con él, vi el pájaro, y apareció

\par 4 al frente, y creció cada vez menos, y finalmente volvió a su tamaño completo. Y detrás de él vi el sol resplandeciente, y los ángeles que lo dibujaban, y una corona en su frente, cuya vista estábamos

\par 5 incapaz de mirar y contemplar. Y tan pronto como brilló el sol, el Fénix también extendió sus alas. Pero yo, al ver tanta gloria, me sentí abatido por un gran temor, y huí y

\par 6 se escondió en las alas del ángel. Y el ángel me dijo: No temas, Baruc, sino espera y verás también su puesta.

\chapter{8}

\par 1 Y él me tomó y me llevó hacia el oeste; y cuando llegó el momento de la puesta, vi de nuevo el pájaro que venía delante de ella, y tan pronto como llegó vi a los ángeles, y levantaron la corona.

\par 2 [...]

\par 3 de su cabeza. Pero el pájaro se quedó exhausto y con las alas contraídas. Y mirando estas cosas, dije: Señor, ¿por qué levantaron la corona de la cabeza del sol, y por qué está

\par 4 el pájaro estaba tan exhausto Y el ángel me dijo: La corona del sol, cuando haya pasado el día, cuatro ángeles la toman, la llevan al cielo y la renuevan, porque ella y sus rayos han sido contaminado en la tierra; además se renueva cada día. Y yo, Baruc, dije: Señor, ¿y por qué?

\par 5 Sus rayos están contaminados sobre la tierra. Y el ángel me dijo: Porque ve la iniquidad y la injusticia de los hombres, es decir, fornicaciones, adulterios, hurtos, extorsiones, idolatrías, borracheras, homicidios, contiendas, celos, malas palabras, murmuraciones. , susurros, adivinaciones y cosas semejantes que no agradan a Dios. Por estas cosas se contamina, y por eso se renueva.

\par 6 Pero tú preguntas por el pájaro, cómo está exhausto. Porque al retener los rayos del sol a través del fuego y el calor abrasador de todo el día, se agota. Porque, como dijimos antes, a menos que sus alas protegieran los rayos del sol, ningún ser viviente sería preservado.

\chapter{9}

\par 1 Cuando se retiraron, también cayó la noche, y al mismo tiempo llegó el carro de la luna con las estrellas.

\par 2 Y yo, Baruc, dije: Señor, muéstrame también, te ruego, cómo

\par 3 sale, de dónde sale y en qué forma avanza. Y el ángel dijo: Espera, y también lo verás pronto. Y al día siguiente también lo vi en forma de mujer, sentado en un carro con ruedas. Y había delante de él bueyes y corderos en el carro, y una multitud de

\par 4 ángeles de la misma manera. Y dije: Señor, ¿qué son los bueyes y los corderos? Y él me dijo:

\par 5 Ellos también son ángeles. Y de nuevo pregunté: ¿Por qué en un momento aumenta y en otro aumenta?

\par 6 el tiempo disminuye Y (me dijo): Escucha, oh Baruc: Esto que ves fue escrito

\par 7 por Dios hermosa como ninguna otra. Y en la transgresión del primer Adán, estuvo cerca de Sammael cuando tomó la serpiente como vestido. Y no se escondió, sino que aumentó, y Dios fue

\par 8 se enojó con ella, la afligió y acortó sus días. Y dije: ¿Y cómo no brilla también siempre, sino sólo en la noche? Y el ángel dijo: Escucha: como en presencia de un rey, los cortesanos no pueden hablar libremente, así la luna y las estrellas no pueden brillar en presencia del sol; porque las estrellas siempre están suspendidas, pero el sol las protege, y la luna, aunque ilesa, es consumida por el calor del sol.

\chapter{10}

\par \textit{El Cuarto Cielo.}

\par 1 Y cuando supe todas estas cosas del arcángel, él me tomó y me llevó a una cuarta

\par 2 [...]

\par 3 cielo. Y vi una llanura monótona, y en medio de ella un charco de agua. Y había en él multitud de aves de todas clases, pero no como las que hay aquí en la tierra. Pero vi una grúa tan grande como

\par 4 grandes bueyes; y todas las aves eran mayores que las del mundo. Y le pregunté al ángel: ¿Qué

\par 5 es la llanura, y el estanque, y la multitud de pájaros que la rodean. Y el ángel dijo: Escucha, Baruc: La llanura que contiene el estanque y otras maravillas es el lugar donde

\par 6 vienen las almas de los justos, cuando conversan, y conviven en coros. Pero el agua es

\par 7 lo que reciben las nubes, y llueve sobre la tierra, y aumentan los frutos. Y dije otra vez al ángel del Señor: Pero (¿qué) son estas aves? Y él me dijo: Son las que

\par 8 Cantad continuamente alabanzas al Señor. Y dije: Señor, ¿y cómo dicen los hombres que el agua que

\par 9 La lluvia que desciende es del mar. Y el ángel dijo: El agua que desciende en lluvia también es del mar y de las aguas de la tierra; pero aquello que estimula los frutos es (sólo) de

\par 10 la última fuente. Sepan, pues, desde ahora que de esta fuente procede lo que se llama el rocío del cielo.

\chapter{11}

\par \textit{El Quinto Cielo.}

\par 1 Y el ángel me tomó y me llevó de allí al quinto cielo. Y la puerta se cerró. Y dije: Señor, ¿no está abierta esta puerta para que podamos entrar? Y el ángel me dijo: No podemos entrar hasta que venga Miguel, que tiene las llaves del Reino de los Cielos; pero espera y verás

\par 2 [...]

\par 3 la gloria de Dios. Y se escuchó un gran sonido, como un trueno. Y dije: Señor, ¿qué es este sonido?

\par 4 Y me dijo: Ahora mismo Miguel, el príncipe de los ángeles, desciende para recibir el

\par 5 oraciones de los hombres. Y he aquí vino una voz: ¡Abran las puertas! Y los abrieron, y

\par 6 hubo un estruendo como de trueno. Y vino Miguel, y el ángel que estaba conmigo vino cara a cara.

\par 7 se enfrentó a él y le dijo: ¡Salve, comandante mío y de toda nuestra orden! Y dijo el comandante Miguel: Salve tú también, hermano nuestro, e intérprete de las revelaciones a los que pasan por la vida.

\par 8 virtuosamente. Y saludándose así unos a otros, se detuvieron. Y vi al comandante Miguel, sosteniendo un barco sumamente grande; su profundidad era tan grande como la distancia del cielo al

\par 9 la tierra, y su anchura era tan grande como la distancia que hay de norte a sur. Y dije: Señor, ¿qué es eso que sostiene el arcángel Miguel? Y me dijo: Por aquí entran los méritos de los justos, y las buenas obras que hacen, que son escoltadas delante del Dios celestial.

\chapter{12}

\par 1 Y mientras hablaba con ellos, he aquí vinieron ángeles trayendo cestas llenas de flores. Y

\par 2 se los dieron a Michael. Y le pregunté al ángel: Señor, ¿quiénes son éstos, y cuáles son las cosas?

\par 3 traído acá de junto a ellos Y él me dijo: Estos son ángeles (que) están sobre el

\par 4 [...]

\par 5 justos. Y el arcángel tomó las cestas y las arrojó en la vasija. Y el angel

\par 6 me dijo: Estas flores son los méritos de los justos. Y vi otros ángeles llevando cestas que no estaban (ni) vacías ni llenas. Y comenzaron a lamentarse, y no se atrevían a acercarse,

\par 7 porque no tenían los premios completos. Y Miguel lloró y dijo: Venid también acá vosotros.

\par 8 ángeles, traed lo que habéis traído. Y se entristeció mucho Miguel y el ángel que estaba conmigo, porque no llenaban la vasija.

\chapter{13}

\par 1 Y entonces vinieron de la misma manera otros ángeles llorando y lamentándose, y diciendo con temor: Mira cómo estamos nublados, oh Señor, porque fuimos entregados a hombres malos, y queremos apartarnos de

\par 2 de ellos. Y Miguel dijo: No podéis apartaros de ellos, para que el enemigo no prevalezca sobre vosotros.

\par 3 el final; pero decidme lo que pedís. Y ellos dijeron: Te rogamos, Miguel nuestro comandante, que nos alejes de ellos, porque no podemos tolerar a hombres malvados e insensatos, porque no hay nada bueno.

\par 4 en ellos, sino toda clase de injusticia y avaricia. Porque no los vemos entrar [en la Iglesia en absoluto, ni entre padres espirituales, ni] en ninguna buena obra. Pero donde hay asesinato, también los hay en medio, y donde hay fornicaciones, adulterios, hurtos, calumnias, perjurios, celos, borracheras, contiendas, envidias, murmuraciones, murmuraciones, idolatrías, adivinaciones y cosas semejantes.

\par 5 entonces son trabajadores de tales obras, y de otras peores. Por lo que rogamos que nos apartemos de ellos. Y Miguel dijo a los ángeles: Esperad hasta que sepa del Señor lo que sucederá.

\chapter{14}

\par 1 Y en esa misma hora Michael salió, y las puertas se cerraron. Y hubo un sonido como

\par 2 truenos. Y le pregunté al ángel: ¿Cuál es el sonido? Y él me dijo: Incluso ahora Miguel está presentando los méritos de los hombres a Dios.

\chapter{15}

\par 1 Y en aquella misma hora descendió Miguel y se abrió la puerta; y trajo aceite.

\par 2 Y a los ángeles que traían las cestas llenas, las llenó de aceite, diciendo: Llévatelo, recompensad cien veces más a nuestros amigos y a los que con mucho trabajo han hecho buenas obras.

\par 3 Porque los que sembraron virtuosamente, también cosecharán virtuosamente. Y dijo también a los que traían las cestas medio vacías: Venid también vosotros acá; Tomad la recompensa según la que habéis traído, y

\par 4 entregádselo a los hijos de los hombres. [Entonces dijo también a los que traían las cestas llenas y a los que traían las medio vacías: Id y bendecid a nuestros amigos, y decid a los que así dice el Señor: En poco sois fieles, os pondré sobre muchas cosas; entra en el gozo de tu Señor.]

\chapter{16}

\par 1 Y volviéndose, dijo también a los que no traían nada: Así dice el Señor: No estéis tristes por

\par 2 rostro, y no lloréis, ni dejéis solos a los hijos de los hombres. Pero ya que me enojaron con sus obras, ve y haz que envidien y se enojen y se irriten contra un pueblo que no es pueblo, un

\par 3 personas que no tienen entendimiento. Además, envíad además la oruga, la langosta sin alas, el mildiú y la langosta común, y granizo con relámpagos y con ira, y

\par 4 castígalos severamente con espada y muerte, y a sus hijos con demonios. Porque no oyeron mi voz, ni observaron mis mandamientos, ni los cumplieron, sino que menospreciaron mis mandamientos y se rebelaron contra los sacerdotes que les anunciaban mis palabras.

\chapter{17}

\par 1 Y mientras él aún hablaba, se cerró la puerta y nos retiramos.

\par 2 Y el ángel me tomó y

\par 3 me devolvió al lugar donde estaba al principio. Y habiendo vuelto en mí, di gloria

\par 4 a Dios, que me tuvo por digno de tal honor. Por tanto, hermanos que habéis recibido tal revelación, también vosotros glorificad a Dios, para que él también os glorifique a vosotros ahora y siempre, y por toda la eternidad. Amén.

\end{document}