\begin{document}

\title{Martirio de Isaías}

\chapter{1}

\par 1 Y aconteció que en el año veintiséis del reinado de Ezequías, rey de Judá,

\par 2 llamó a Manasés su hijo. Ahora él era el único. Y lo llamó delante del profeta Isaías hijo de Amoz; y en presencia de Josab hijo de Isaías.

\par 3 [...]

\par 4 [...]

\par 5 [...]

\par 6 Y mientras él (Ezequías) daba órdenes, estando presente Josab, hijo de Isaías, Isaías dijo al rey Ezequías, pero no sólo en presencia de Manasés le dijo: «Vive el Señor, cuyo nombre ha no he sido enviado a este mundo, [y como vive el Amado de mi Señor], y como vive el Espíritu que habla en mí, todos estos mandamientos y estas palabras serán anulados por Manasés tu hijo, y por medio de la agencia de De sus manos partiré en medio de la tortura de

\par 7 [...]

\par 8 mi cuerpo. Y Sammael Malchira servirá a Manasés, y ejecutará todo su deseo, y él

\par 9 conviértete en un seguidor de Beliar en lugar de mí. Y a muchos en Jerusalén y en Judea hará abandonar la verdadera fe, y Beliar habitará en Manasés, y por sus manos seré

\par 10 cortado en pedazos.» Y cuando Ezequías oyó estas palabras, lloró muy amargamente y rasgó sus vestidos.

\par 11 y puso tierra sobre su cabeza y cayó sobre su rostro. E Isaías le dijo: «El consejo de

\par 12 Sammael contra Manasés ha sido consumado: nada te servirá. Y aquel día Ezequías

\par 13 resolvió en su corazón matar a su hijo Manasés. E Isaías dijo a Ezequías: [«El Amado ha invalidado tu designio, y] el propósito de tu corazón no se cumplirá, porque con este llamamiento he sido llamado [y heredaré la herencia del Amado»] .

\chapter{2}

\par 1 Y aconteció que después que murió Ezequías y Manasés comenzó a reinar, no se acordó de los mandamientos de su padre Ezequías, sino que los olvidó, y Sammael se quedó en Manasés.

\par 2 y se aferró rápidamente a él. Y dejó Manasés el servicio del Dios de su padre, y sirvió

\par 3 Satanás y sus ángeles y sus poderes. Y desvió la casa de su padre que había sido

\par 4 delante de Ezequías palabras de sabiduría y del servicio de Dios. Y Manasés desvió su corazón para servir a Beliar; porque el ángel del pecado, que es el gobernante de este mundo, es Beliar, cuyo nombre es Matanbuchus. Y se deleitó en Jerusalén a causa de Manasés, y lo fortaleció en la apostasía (Israel) y en la iniquidad que se difundió en Jerusalén.

\par 5 Y aumentaron las hechicerías y los hechizos, la adivinación, la adivinación, la fornicación y el adulterio, y la persecución de los justos por parte de Manasés, de Tobías el cananeo y de Juan.

\par 6 de Anatot, y por (Sadok) el jefe de las obras. Y el resto de los hechos, he aquí están escritos

\par 7 en el libro de los reyes de Judá y de Israel. Y cuando Isaías hijo de Amoz vio la iniquidad que se estaba perpetrando en Jerusalén y el culto a Satanás y su libertinaje,

\par 8 se retiraron de Jerusalén y se establecieron en Belén de Judá. Y allí también hubo mucho

\par 9 y saliendo de Belén, se instaló en un monte en un lugar desierto. [Y el profeta Micaías, el anciano Ananías, Joel, Habacuc, su hijo Josab y muchos de los fieles que creían en la ascensión al cielo, se retiraron y se establecieron en el monte.]

\par 10 Todos estaban vestidos con vestiduras de pelo y todos eran profetas. Y no tenían nada consigo sino que estaban desnudos, y todos se lamentaban con gran lamentación a causa de la marcha.

\par 11 descarriados de Israel. Y éstos no comieron sino hierbas silvestres que recogieron en los montes, y cocinándolas, vivieron de ellas junto con el profeta Isaías. Y pasaron dos años de

\par 12 días en montañas y colinas. Y después de esto, mientras estaban en el desierto, hubo en Samaria cierto hombre llamado Belchlra, de la familia de Sedequías, hijo de Quenaán, falso profeta que moraba en Belén. Y Ezequías hijo de Canani, que era hermano de su padre, y en los días de Acab rey de Israel, había sido maestro de los cuatrocientos profetas de Baal,

\par 13 él mismo fue herido y reprendió a Micaías, hijo del profeta Amada. Y él, Micaías, había sido reprendido por Acab y encarcelado. (Y estaba) con el profeta Sedequías: estaban

\par 14 con Ocozías hijo de Acab, rey de Samaria. Y Elías el profeta de Tebón de Galaad estaba reprendiendo a Ocozías y a Samaria, y profetizó acerca de Ocozías que moriría en su lecho de enfermedad, y que Samaria sería entregada en manos de Leba Nasr porque había matado a

\par 15 los profetas de Dios. Y cuando los falsos profetas que estaban con Ocozías hijo de Acab y

\par 16 Cuando lo oyó su maestro Gemarias, del monte Joel, que era hermano de Sedequías, persuadieron a Ocozías, rey de Aguarón, y mataron a Micaías.

\chapter{3}

\par 1 Y Belchlra reconoció y vio el lugar de Isaías y de los profetas que estaban con él; porque habitaba en la región de Belén y era partidario de Manasés. Y profetizó mentira en Jerusalén, y muchos de Jerusalén se confederaron con él, y él era samaritano.

\par 2 Y aconteció que cuando Alagar Zagar, rey de Asiria, vino y capturó Samaria y tomó cautivas a las nueve (y media) tribus, y las llevó a las montañas de los medos y a los

\par 3 ríos de Tazón; este (Belchira) siendo aún joven, había escapado y venido a Jerusalén en los días de Ezequías rey de Judá, pero no anduvo en los caminos de su padre de Samaria; porque temía

\par 4 Ezequías. Y fue hallado en los días de Ezequías hablando palabras de iniquidad en Jerusalén.

\par 5 Los siervos de Ezequías lo acusaron y él escapó a la región de Belén.

\par 6 Y ellos persuadieron. . . Y Belchlra acusó a Isaías y a los profetas que estaban con él, diciendo: «Isaías y los que están con él profetizan contra Jerusalén y contra las ciudades de Judá que serán arrasadas y (contra los hijos de Judá y) Benjamín también que irá en cautiverio, y también contra ti, oh señor rey, que irás (atado) con ganchos

\par 7 [...]

\par 8 y cadenas de hierro; pero profetizan mentira contra Israel y Judá. Y el mismo Isaías tiene

\par 9 dijo: «Veo más que el profeta Moisés». Pero Moisés dijo: «Ningún hombre puede ver a Dios y vivir»:

\par 10 E Isaías dijo: «He visto a Dios y he aquí que vivo». Sepa, pues, oh rey, que miente. Y a Jerusalén también llamó Sodoma, y ​​a los príncipes de Judá y a Jerusalén los declaró pueblo de Gomorra. Y presentó muchas acusaciones contra Isaías y los

\par 11 profetas antes de Manasés. Pero Beliar habitó en el corazón de Manasés y en el corazón de los

\par 12 príncipes de Judá y de Benjamín, y de los eunucos y de los consejeros del rey. Y las palabras de Belchira le agradaron [en gran manera], y envió y prendió a Isaías.

\chapter{4}

\par \textit{No hay contenido para este capítulo}

\par 1 [...]

\chapter{5}

\par 1 Y lo cortó en pedazos con una sierra para madera. Y cuando Isaías estaba siendo aserrado en pedazos, Balchlra se levantó, acusándolo, y todos los falsos profetas se levantaron, riendo y regocijándose porque

\par 2 [...]

\par 3 de Isaías. Y Balchlra, con la ayuda de Mechembechus, se puso de pie ante Isaías, [risas]

\par 4 burlándose; Y Belchlra dijo a Isaías: «Di: «He mentido en todo lo que he hablado, y de la misma manera

\par 5 Los caminos de Manasés son buenos y rectos. Y los caminos también de Balchlra y de sus asociados son

\par 6 bueno.»» Y esto le dijo cuando comenzó a ser aserrado. Pero Isaías estaba (absorbido)

\par 7 [...]

\par 8 en una visión del Señor, y aunque tenía los ojos abiertos, los vio. Y Balchlra habló así a Isaías: «Di lo que te digo y haré volver su corazón, y obligaré a Manasés

\par 9 y a los príncipes de Judá, al pueblo y a toda Jerusalén para que te reverencian. Y Isaías respondió y dijo: «En lo que tengo expresión (digo): Maldito y maldito seas tú y todos tus poderes y

\par 10 toda tu casa. Porque no puedes quitarme (de mí) nada excepto la piel de mi cuerpo.» Y ellos

\par 11 [...]

\par 12 Agarró y cortó a Isaías, hijo de Amoz, con una sierra para madera. Y Manasés y

\par 13 Balchlra y los falsos profetas, los príncipes y el pueblo [y] todos estaban mirando. Y a los profetas que estaban con él, antes de ser aserrado, dijo: «Id a la región

\par 14 de Tiro y Sidón; porque sólo para mí Dios ha mezclado la copa.» Y cuando Isaías era aserrado en pedazos, no gritó ni lloró, sino que sus labios hablaron con el Espíritu Santo hasta que fue aserrado en dos.

\end{document}