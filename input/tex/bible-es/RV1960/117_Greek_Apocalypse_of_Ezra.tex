\begin{document}

\title{Apocalipsis Griego de Esdras}

\chapter{1}

\par \textit{Introducción y oración}

\par 1 Aconteció que el día veintidós del mes del año treinta, 

\par 2 yo estaba en mi casa y clamé, diciendo al Altísimo: Señor, concédeme gloria para que pueda ver tus misterios.


\par 3 Cuando cayó la noche, vino el ángel Miguel, el arcángel, y me dijo: «Profeta Esdras, reserva pan para setenta semanas». Y ayuné tal como él me dijo.

\par 4 Y vino el archistrategos Rafael y me dio un bastón de estoraque, 

\par 5 y ayuné dos veces sesenta semanas, y vi los misterios de Dios y sus ángeles.

\par 6 Y les dije: «Quiero suplicar a Dios por el pueblo cristiano. Más vale que el hombre no nazca que que entre en el mundo».

\par \textit{Esdras llevada al cielo: su oración pidiendo misericordia}

\par 7 Por tanto, fui llevado al cielo y vi en el primer cielo un gran orden de los ángeles y ellos me llevaron a los juicios.

\par 8 Y escuché una voz que decía a mí: «Ten piedad de nosotros, Esdras, el elegido de Dios».

\par 9 Entonces comencé a decir: «¡Ay de los pecadores, cuando vean al justo (elevado) por encima de los ángeles, y sean por la ardiente Gehena».

\par 10 Y Esdras dijo: Ten piedad de las obras de tus manos, misericordioso y muy compasivo.

\par 11 Condenadme a mí antes que a las almas de los pecadores, porque es mejor castigar a un alma y no llevar al mundo entero a la destrucción.»

\par 12 Y Dios dijo: «Daré descanso a los justos en el Paraíso y soy misericordioso».

\par 13 Y Esdras dijo: «Señor, ¿por qué muestras favor a los justos?

\par 14 Porque como el jornalero cumple su tiempo de servicio y se va, y el esclavo sirve nuevamente a sus amos para recibir su salario, así el justo recibe su recompensa en los cielos. 

\par 15 Pero ten piedad de los pecadores porque sabemos que tú eres misericordioso».

\par 16 Y Dios dijo: «No tengo manera de ser misericordioso con ellos». 

\par 17 Y Esdras dijo: «(Sé misericordioso) porque ellos no pueden soportar tu ira».

\par 18 Y Dios dijo: «(Estoy enojado) porque tales (son los desiertos) de tales (hombres) como estos».

\par 19 Y Dios dijo: «Quiero guardarte como a Pablo y a Juan.

\par 20 Me has dado incorrupto el tesoro inviolable, el tesoro de la virginidad, el muro de los hombres».

\par \textit{La segunda oración de Esdras}

\par 21 Y Esdras dijo: Mejor sería que el hombre no naciera; estaría bien si no estuviera vivo.

\par 22 Las bestias mudas son mejores que el hombre, porque no tienen castigo

\par 23 Nos tomaste y nos entregaste al juicio. 

\par 24 ¡Ay de los pecadores en el mundo venidero, porque su condenación es interminable y la llama no se apaga!

\chapter{2}

\par \textit{Esdras protesta ante Dios: el pecado de Adán}

\par 1 Mientras le decía esto, se acercaron Miguel y Gabriel y todos los apóstoles y dijeron:

\par 2 «¡Saludos!» 

\par 3 [Y Esdras dijo: «¡Hombre fiel de Dios!] 

\par 4 Levántate y ven acá conmigo, oh Señor, al juicio». Y dijo Dios: He aquí, os doy mi pacto, tanto mío como tuyo, para que lo aceptes».

\par 5 Y Esdras dijo: «Nosotros defenderemos nuestro caso en vuestro(s) oído(s)».

\par 6 Y Dios dijo: Pregunta a tu padre Abraham qué clase de hijo demanda a su padre y ven y defiende el caso ante nosotros.»

\par 7 Y Esdras dijo: «Vive el Señor, que nunca dejaré de defender el caso con vosotros a causa del pueblo cristiano.

\par 8 ¿Dónde están tus antiguas misericordias, oh Señor? ¿Dónde está tu gran sufrimiento?

\par 9 Y Dios dijo: «Como hice la noche y el día, hice al justo y al pecador, y conviene que te conduzcas como el hombre justo».

\par 10 Y el profeta dijo: «¿Quién hizo a Adán, el protoplasto, el primero?»

\par 11 Y dijo Dios: «Mis manos inmaculadas, y lo puse en el Paraíso para guardar la región del árbol de la vida».

\par 12 «Ya que el que estableció la desobediencia hizo que este (hombre) pecara».

\par 13 Y el profeta dijo: ¿No estaba él vigilado por un ángel?

\par 14 ¿Y no fue preservada la vida por los querubines para la era eterna?

\par 15 ¿Y cómo fue engañado aquel que estaba custodiado por ángeles a quienes tú ordenaste estar presente pase lo que pase? ¡Atiende también a lo que digo!

\par 16 Si no le hubieras dado a Eva, la serpiente nunca la habría engañado.

\par 17 Si salvas a quien quieras, también destruirás a quien quieras».

\par \textit{Esdras protesta ante Dios: los pecados de los hombres}

\par 18 Y el profeta dijo: «Oh Señor mío, pasemos a un segundo juicio».

\par 19 Y Dios dijo: «Arrojo fuego sobre Sodoma y Gomorra». 

\par 20 Y el profeta dijo: «Señor, tú traes sobre nosotros lo que merecemos».

\par 21 Y Dios dijo: «Tus pecados exceden mi bondad.»

\par 22 Y el profeta dijo: Acordaos de la Escritura, padre mío, que midió Jerusalén y la reconstruyó.

\par 23 Compadece, Señor, de los pecadores, compadécete de los tuyos, ten misericordia de tus obras.

\par 24 Entonces Dios se acordó de sus obras y dijo al profeta: «¿Cómo podré tener misericordia de ellos?

\par 25 Les dieron a beber vinagre y hiel y [...] se arrepintieron».

\par \textit{El día del juicio}

\par 26 Y el profeta dijo: «Revelad vuestros querubines y vayamos juntos al juicio,

\par 27 y muéstrame cuál será el carácter del día del juicio. 

\par 28 Y Dios dijo: «Tú te has desviado, Esdras,

\par 29 porque tal es el día del juicio en el que no lloverá sobre la tierra,

\par 30 porque aquel día habrá juicio misericordioso». 

\par 31 Y el profeta dijo: «Nunca dejaré de discutir con vosotros hasta que vea el día de la consumación.»

\par 32 (Y Dios dijo:) «Cuenta las estrellas y la arena del mar y si puedes contar esto, también podrás discutir el caso conmigo».

\chapter{3}

\par 1 Y el profeta dijo: «Señor, tú sabes que soy portador de carne humana. 

\par 2 ¿Y cómo puedo contar las estrellas del cielo y la arena del mar?»

\par 3 Y Dios dijo: «Oh mi profeta elegido, nadie sabrá ese gran día y la manifestación que prevalecerá hasta juzgar el mundo.

\par 4 Por ti, profeta mío, te dije el día, pero la hora no te la dije».

\par 5 Y el profeta dijo: «Señor, dime también los años». 

\par 6 Y Dios dijo: «Si veo que la justicia del mundo se ha hecho abundante, seré paciente para con ellos. Si no, extenderé mi mano y agarraré al mundo habitado por sus cuatro confines y los reuniré a todos en el valle de Josafat y exterminaré a la raza humana y el mundo no existirá más».

\par 7 Y el profeta dijo: «¿Y cómo será glorificada tu diestra?»

\par 8 Y dijo Dios: «Seré glorificado por mis ángeles».

\par \textit{¿Por qué fue creado el hombre?}

\par 9 Y el profeta dijo: «Señor, si este fue tu cálculo, ¿por qué formaste al hombre?

\par 10 Dijiste a Abraham nuestro padre: «Ciertamente multiplicaré tu descendencia como las estrellas del cielo y como la arena a la orilla del mar.» ¿Y dónde está tu promesa?

\par \textit{Señales del fin}

\par 11 Y dijo Dios: Primero haré sacudir la caída de los cuadrúpedos y los hombres.

\par 12 Y cuando veas que el hermano entrega a la muerte al hermano y a los hijos se levantará contra los padres, y la esposa abandonará a su propio marido;

\par 13 y cuando una nación se levantará contra otra nación en guerra, entonces sabréis que el fin está cerca.

\par 14 Y entonces el hermano no tendrá misericordia del hermano, ni el hombre de su esposa, ni hijos sobre padres, ni amigos sobre amigos, ni esclavo sobre amo.

\par 15 Porque el mismo adversario de los hombres subirá del Tártaro y mostrará muchas cosas para hombres.

\par 16 ¿Qué te haré, Ezra, y discutirás el caso conmigo?

\chapter{4}

\par \textit{Esdras desciende al Tártaro}

\par 1 Y el profeta dijo: «Señor, nunca dejaré de discutir contigo».

\par 2 Y dijo Dios: Cuenta las flores de la tierra. 

\par 3 Si puedes contarlos, también podrás discutir el caso conmigo».

\par 4 Y el profeta dijo: «Señor, no puedo contarlos; llevo carne humana, pero tampoco dejaré de discutir el caso contigo.

\par 5 Deseo, Señor, ver las partes bajas del Tártaro.»

\par 6 Y Dios dijo: «¡Desciende y mira!» 

\par 7 Y me dio a Miguel y Gabriel y otros treinta y cuatro ángeles,

\par 8 y bajé ochenta y cinco escalones y ellos me hicieron descender quinientos escalones.

\par \textit{El castigo de Herodes}

\par 9 Y vi un trono de fuego y a un anciano sentado en él, y su castigo era despiadado.

\par 10 Y dije a los ángeles: «¿Quién es éste y cuál es su pecado?»

\par 11 Y me dijeron, «Este es Herodes, que fue rey por un tiempo, y mandó matar los infantes de dos años o menos.»

\par 12 Y dije: «¡Ay de su alma!»

\par \textit{Los desobedientes y el abismo}

\par 13 Y de nuevo me hicieron bajar treinta escalones. Y vi allí fuegos hirviendo, y un multitud de pecadores en ellos.

\par 14 Y oí sus voces, pero no percibí sus formas.

\par 15 Y me llevaron a muchos escalones más profundos que no podía contar.

\par 16 Vi allí a unos ancianos con ejes de fuego girando sobre sus orejas.

\par 17 Y dije: «¿Quiénes son estos y cuál es su pecado?»

\par 18 Y me dijeron: «Estos son los espías.»

\par 19 Y de nuevo me hicieron bajar quinientos escalones más. 

\par 20 Y allí vi el gusano que no dormía y el fuego que consumía a los pecadores.

\par 21 Y me llevaron hasta los cimientos de Apoleia (Destrucción) y allí vi las doce veces golpe del abismo.

\par 22 Y me llevaron hacia el sur y allí vi a un hombre que colgaban de sus párpados y los ángeles lo golpeaban.

\par 23 Y le pregunté: «¿Quién es éste y cuál es su pecado?»

\par 24 Y Miguel el archistrategos me dijo: «Este hombre es incestuoso; habiendo llevado a cabo una pequeña lujuria, a este hombre se le ordenó que lo ahorcaran».

\par \textit{El Anticristo}

\par 25 Y me llevaron hacia el norte y vi allí a un hombre atado con barras de hierro.

\par 26 Y pregunté: «¿Quién es este?» Y me dijo: 

\par 27 Éste es el que dice: Yo soy el Hijo de Dios y el que hizo las piedras el pan y el agua el vino.

\par 28 Y el profeta dijo: «Hazme saber qué apariencia tiene e informaré a la raza humana para que no crean en él.

\par 29 Y él me dijo: «El aspecto de su rostro es como el de un hombre salvaje. Su ojo derecho es como una estrella que se eleva al amanecer y el otro está inmóvil.

\par 30 Su boca es un codo, sus dientes un palmo de largo,

\par 31 sus dedos como guadañas, las plantas de sus pies de dos palmos, y en su frente una inscripción «Anticristo».

\par 32 Fue exaltado hasta el cielo, descenderá tan lejos

\par 33 como Hades. Una vez será un niño, otra un anciano». 

\par 34 Y el profeta dijo: «Señor, ¿cómo permites que la raza de los hombres se extravíe?»

\par 35 Y Dios dijo: «¡Oye, profeta mío! Se convierte en niño y en anciano y que nadie le crea él es mi hijo amado.

\par 36 Y después de estas cosas se tocará la trompeta, y los sepulcros serán abiertos y los muertos resucitarán incorruptos.

\par 37 Entonces el oponente, habiendo escuchado la terrible amenaza, se esconderá en las tinieblas exteriores.

\par 38 Entonces el cielo y la tierra y el mar perecerán.

\par 39 Entonces quemaré el cielo por ochenta codos y la tierra ochocientos codos.

\par 40 Y el profeta dijo: «¿Y (en) qué pecó el cielo?»

\par 41 Y Dios dijo: «Ya que [...] es el mal». 

\par 42 Y el profeta dijo: Señor, ¿en qué pecó la tierra?

\par 43 Y Dios dijo: «Ya que el oponente, habiendo escuchado mi terrible amenaza, se esconderá (en ella), y por eso derretiré la tierra y con ella a los rebeldes de la raza de los hombres».

\chapter{5}

\par \textit{Castigos adicionales}

\par 1 Y el profeta dijo: «Apiádate, Señor, del linaje de los cristianos». 

\par 2 Y vi un mujeres suspendidas y cuatro fieras chupaban sus pechos.

\par 3 Y los ángeles me dijeron: «A ella le disgustó dar su leche, pero también arrojó a los niños en el río.»

\par 4 Y vi una oscuridad terrible y una noche sin estrellas ni luna. 

\par 5 No hay ni joven ni viejo, ni hermano con hermano ni madre con niño ni esposa con marido.

\par 6 Y lloré y dije: «Oh Señor, Señor, ten misericordia de los pecadores».

\par \textit{Esdras llevado al cielo}

\par 7 Y mientras decía estas cosas vino una nube, se apoderó de mí y me levantó de nuevo a los cielos.

\par 8 Y vi muchos juicios y lloré amargamente y dije: 

\par 9 «Es mejor sería si el hombre no saliera del vientre de su madre.»

\par 10 Los que estaban en el castigo gritaban, diciendo: Desde que llegaste aquí, santo de Dios, hemos obtenido un ligero respiro».

\par 11 Y el profeta dijo: «Bienaventurados los que lamentan sus propios pecados».

\par \textit{El nacimiento y su finalidad}

\par 12 Y Dios dijo: «¡Escucha a Esdras, amado! Así como un granjero arroja la semilla en la tierra, así el hombre echa su semilla en lugar de la mujer.

\par 13 En el primero (mes) está entero, en el segundo está hinchado, en el tercero le crece pelo, en el cuarto le crecen uñas, en el quinto se vuelve lechoso, en el sexto está listo y vivificado, en el en el séptimo se prepara, [en el octavo...], en el noveno se abren los cerrojos de las puertas de la mujer y nace sana en la tierra».

\par 14 Y el profeta dijo: «Más le valdría al hombre no haber nacido.

\par 15 ¡Ay, oh raza humana, en ese momento cuando vengas al juicio!

\par 16 Y le dije al Señor, «Señor, ¿por qué creaste al hombre y lo entregaste al juicio?»

\par 17 Y Dios dijo en su exaltada declaración: «No perdonaré a los que transgreden mi pacto».

\par 18 Y el profeta dijo: «Señor, ¿dónde está tu bondad?» 

\par 19 Y dijo Dios: Todo lo preparé por causa del hombre y el hombre no guarda mis mandamientos.

\par \textit{Castigos y recompensas}

\par 20 Y el profeta dijo: «Señor, revélame los castigos y el Paraíso».

\par 21 Y los ángeles me llevaron hacia el oriente y vi el árbol de la vida.

\par 22 Y vi allí a Enoc y Elías y Moisés y Pedro y Pablo y Lucas y Mateo y todos los justos y los patriarcas.

\par 23 Y vi allí el castigo del aire y el soplo de los vientos y los depósitos del hielo y el eterno castigos.

\par 24 Y vi allí a un hombre colgado de su cráneo. 

\par 25 Y me dijeron, «Este transfirió límites».

\par 26 Y allí vi grandes juicios y dije a al Señor: «Oh Señor, Señor, ¿quién de los hombres, habiendo nacido, no pecó?»

\par 27 Y me llevaron más abajo en el Tártaro y vi a todos los pecadores lamentándose y llanto y luto malvado.

\par 28 Y yo también lloré al ver la raza de los hombres así castigada.

\chapter{6}

\par 1 Entonces Dios me dijo: Esdras, ¿sabes los nombres de los ángeles que están sobre la consumación:

\par 2 Miguel, Gabriel, Uriel, Rafael, Gabutelón, Aker, Arphugitonos, Beburos, Zebuleon?

\par \textit{Ezra lucha por su alma}

\par 3 Entonces vino a mí una voz: «¡Ven aquí, muere, Esdras, amado mío! devolver eso que te ha sido confiado (a ti)».

\par 4 Y el profeta dijo: «¿Y de dónde has dado a luz mi alma?

\par 5 Y los ángeles dijeron: «Podemos expulsarlo a través de tu boca».

\par 6 Y el profeta dijo: «Hablé boca a boca con Dios y no saldrá de allí».

\par 7 Y los ángeles dijeron. «Lo sacaremos a luz a través tus fosas nasales».

\par 8 Y el profeta dijo: «Mis narices olían la gloria de Dios».

\par 9 Y los ángeles dijeron: «Podemos sacarlo a través de tus ojos». 

\par 10 Y el profeta dijo: «Mis ojos han visto la espalda de Dios».

\par 11 Y los ángeles dijeron: «Podemos traerlo a través de tu cabeza».

\par 12 Y los ángeles dijeron: «Podemos sacarlo a través de tus pies». Y el profeta dijo: Caminé con Moisés sobre la montaña, y no saldrá de allí».

\par 13 Y los ángeles dijeron: «Podemos lanzarlo a través de las puntas de las uñas (de los pies)».

\par 14 Y el profeta dijo: «Mis pies entraron en el santuario». 

\par 15 Y los ángeles se marcharon sin éxito, diciendo: Señor, nosotros no podemos recibir su alma.»

\par 16 Luego le dijo a su hijo unigénito: «Desciende, hijo amado mío, con un gran ejército de ángeles, y toma el alma de mi amado Esdras».

\par 17 Porque el Señor, tomando un gran ejército de muchos ángeles, dijo al profeta: «Dame el depósito que te he confiado. La corona está lista para ti».

\par 18 Y el profeta dijo: Señor, si me quitas el alma, ¿a quién le dejarás para abogar por la raza de los hombres?»

\par 19 Y Dios dijo: «Tú que eres mortal y terrenal, no me defiendan este caso».

\par 20 Y el profeta dijo: «Nunca dejaré de suplicar.»

\par 21 Y Dios dijo: «Dad, mientras tanto, lo que os ha sido confiado (para ti). La corona está lista para ti.».

\par 22 Ven aquí, muere, para que puedas alcanzar eso.»

\par 23 Entonces el profeta comenzó a hablar entre lágrimas: «Oh Señor, ¿de qué me sirve que yo suplique el caso contigo y yo caeré a la tierra?

\par 24 ¡Ay, ay! porque lo haré ser consumido por los gusanos.

\par 25 Lloren por mí, todos santos y piadosos, les suplico mucho y

\par ¡26 am entregado a la muerte! Llorad por mí, todos santos y justos, porque he entrado en la copa del Hades».

\chapter{7}

\par \textit{Alma y cuerpo}

\par 1 Y Dios le dijo: «Oye, Esdras, mi amado. Yo, siendo inmortal, recibí una cruz, probé vinagre y hiel, fui sepultado en una tumba.

\par 2 Y levanté a mis elegidos y llamé a Adán del Hades para que la raza de los hombres

\par 3 [...]. Por tanto, no temáis a la muerte. Porque lo que es de mí, que es el alma, parte para el cielo. Lo que es de la tierra, es decir el cuerpo, parte hacia la tierra de la cual fue tomada».

\par 4 Y el profeta dijo: «¡Ay, ay! ¿Qué debo hacer? ¿Cómo debo actuar? Yo no sé.»

\par \textit{Oración finalb}

\par 5 Y entonces el bienaventurado Esdras comenzó a decir: «¡Oh Dios eterno, Creador de toda la creación, que con un palmo midiste los cielos y en tu mano contenías la tierra,

\par 6 que impulsa a los querubines, que llevó al profeta Elías al cielo en un carro de fuego,

\par 7 que sustenta a toda carne, a quien todas las cosas temen y tiemblan frente a tu poder,

\par 8 escúchame, que suplica mucho 

\par 9 y da a todos los que copian este  dales bendición del cielo.libro y consérvalo y recuerda mi nombre y preserva mi memoria plenamente,

\par 10 Y bendecir todas sus cosas, así como los fines de José.

\par 11 Y no te acuerdes de sus pecados anteriores en el día de su juicio. 

\par 12 Aquellos que no crean en este libro serán quemados como Sodoma y Gomorra.

\par 13 Y vino a él una voz que decía: Esdras. Amado mío, concederé a cada uno lo que me pediste».

\par \textit{Muerte y sepultura de Esdras}

\par 14 E inmediatamente entregó su preciosa alma con mucho honor el día dieciocho del mes de octubre.

\par 15 Y lo sepultaron con incienso y salmos. Su precioso y santo cuerpo proporciona incesantemente fortalecimiento de almas y cuerpos a quienes se acercan a él voluntariamente.

\par 16 Gloria, poder, honra y adoración a aquel a quien conviene, al Padre, al Hijo y al Espíritu Santo, ahora y siempre, y por los siglos de los siglos. Amén.

\end{document}