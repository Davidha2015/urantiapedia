\begin{document}

\title{Preguntas de Esdras}

\chapter{1}

\par \textit{Recensión A}

\par \textit{¿Cuál es el destino de los justos y los pecadores?}

\par 1 El profeta Esdras vio al ángel de Dios y le hizo una pregunta tras otra.

\par 2 Y el ángel se acercó a él y le dijo lo que sucederá en el momento de la consumación. El profeta preguntó al ángel y le dijo: «¿Qué ha preparado Dios para los justos y los pecadores? Y en el momento en que llegue el día del fin, ¿qué será de ellos? ¿A dónde van, a los honores o a las torturas?

\par 3 El ángel respondió y dijo al profeta: Gran gozo y luz eterna están preparadas para los justos y para los pecadores están preparadas las tinieblas exteriores y el fuego eterno.»

\par 4 El profeta dijo al ángel: Señor, ¿quién de los vivientes no ha pecado contra Dios?

\par 5 Y si es así, bienaventurados los animales y los pájaros que lo hacen no esperaron la resurrección, ni esperaron el fin.

\par 6 Si coronas a los justos, que han soportado todos los tormentos, y a los profetas y a los mártires cuando tomaban piedras y con un martillo se golpeaban el rostro hasta que se vieron sus entrañas,

\par 7 fueron torturadas por tu causa. Ten piedad de nosotros, pecadores, que hemos sido ocupados y apresados ​​por Satanás».

\par \textit{El profeta reprendió}

\par 8 El ángel respondió y dijo: Si hay alguien por encima de ti, no hables con él. De lo contrario, os sobrevendrá un gran mal».

\par 9 El profeta le dijo al ángel: «Señor, quisiera hablar un poco más contigo, ¡respóndeme!

\par 10 Cuando llegue el día del fin y tome el alma, ¿la asignará al lugar de castigo o al lugar de honor hasta la Parusía? [...]»

\par \textit{El día del fin}

\par 11 El ángel respondió y dijo: No esperes hasta el día del fin, sino como un águila voladora se apresura a hacer buenas obras y misericordia.

\par 12 Porque ese día es temible, urgente, y exigente.

\par 13 No permite el cuidado de niños ni de posesiones. Viene y llega de repente como alguien despiadado e imparcial, toma cautivo inesperadamente, seguramente. Ya sea que llore o se lamente, no tendrá piedad».

\par \textit{Los ángeles buenos y malos}

\par 14 «Pero cuando llega el día del fin, viene un ángel bueno al alma buena y uno malo al alma mala. Como alguien enviado por los reyes a los hacedores de malas acciones.

\par 15 y las buenas obras recompensan con el bien a los buenos y con el mal a los malos, así también el ángel bueno llega al alma buena y el ángel malo al alma mala. No que el ángel es malo, pero las obras de cada uno (son malas).

\par 16 Toma el alma y la lleva al este; pasan por escarcha, por nieve, por oscuridad, por granizo, por hielo, por tormenta, por huestes de Satanás, por arroyos, por vientos de lluvias terribles, por senderos terribles y asombrosos, por angostos desfiladeros y por altas montañas.

\par 17 Oh camino maravilloso, porque un pie está detrás del otro y delante de él hay ríos de fuego!

\par 18 El profeta quedó asombrado y dijo: «¡Oh, ese camino maravilloso y terrible!»

\par \textit{Los siete pasos hacia la Divinidad}

\par 19 El ángel dijo: «Hacia aquel camino hay siete campamentos y siete gradas hasta la Divinidad, si puedo hacer que (alguien) la transmita.

\par 20 Porque las primeras moradas son malas y maravillosas; el segundo temible e indescriptible; el tercer infierno y frío glacial; el cuarto peleas y wers; en el quinto, pues, la investigación: si es justo, brilla, y si es pecador, se oscurece; en el Sexto, entonces, el alma del justo brilla como el sol;

\par 21 en el séptimo, pues, habiéndolo traído, lo hago acercar al gran trono de la Divinidad, frente al jardín, de cara a la gloria de Dios, donde está la luz sublime».

\par \textit{Dios no se puede ver}

\par 22 El profeta dijo al ángel: «Señor mío, cuando le haces pasar por tales terrores, por riñas, por guerras, por calor abrasador, ¿por qué no le haces encontrarse con la Divinidad, en lugar de hacerle llegar? acercarse sólo a el trono?

\par 23 El ángel dijo al profeta: «Tú eres uno de los hombres insensatos y no pensar según la naturaleza humana.

\par 24 Soy un ángel y sirvo perpetuamente a Dios, y no he visto el rostro de Dios. ¿Cómo se dice que el hombre pecador debe ser causado para encontrarse con la Divinidad?

\par 25 Porque la Divinidad es temible y maravillosa y ¿quién se atreve a mirar hacia la Divinidad increada?

\par 26 Si un hombre mira, se derretirá como cera ante el rostro de Dios: porque la Divinidad es ardiente y maravillosa. Porque tales guardianes están alrededor del trono de la Divinidad».

\par \textit{Aquellos alrededor del trono divino}

\par 27 «Hay estaciones, [...] huecos, ardientes, ceñidores, (y) faroles.

\par 28 En ese lugar hay truenos, terremotos, riñas, guerras, calor abrasador, portadores defuego, enjambres de llamas y huestes ardientes.

\par 29 Alrededor de él están los serafines incorpóreos, querubines de seis alas: con dos alas cubren su rostro, y con dos alas sus pies, y volando con dos, claman: «Santo, Santo, (Santo) Señor de Huestes, el cielo y la tierra están llenos de vuestra gloria».

\par 30 Tales guardianes están alrededor del trono de la Divinidad».

\par \textit{Liberación del alma de Satanás}

\par 31 El profeta preguntó al ángel y le dijo: Señor, ¿qué será de nosotros, que todos somos pecadores y estamos en manos de Satanás? Ahora bien, ¿por qué medios nos librará o quién nos sacará de sus manos?

\par 32 El ángel respondió y dijo: Si alguno queda después de la muerte, padre o madre o hermano o hermana o hijo o hija o cualquier otro cristiano, y ofrece oraciones con ayunos durante cuarenta días, habrá gran descanso y misericordia por el sacrificio de Cristo.

\par 33 Porque Cristo fue sacrificado por nosotros en la cruz y por seis siglos libró (nuestra) alma de las manos de Satanás.

\par 34 Cómo se libera el alma a través de lo ofrecido con reverencia por un sacerdote, si cumple los cuarenta días de la manera que agrada a Dios.

\par 35 Durante cuarenta días permanecerá en la iglesia, no yendo a los lugares públicos, sino de de vez en cuando recitará los Salmos de David junto con oraciones.

\par 36 Esto es lo que nos libra de las manos de Satanás. Si no, dáselo a los pobres».

\par \textit{La naturaleza de la oración}

\par 37 «Porque vuestras oraciones son así: así como el labrador sale, viene a sembrar, y el retoño brota alegre y gracioso y desea producir muchos frutos, y también brotan espinas y malas hierbas que lo ahogan y no dejes que sean frutos numerosos ensamblados.

\par 38 De la misma manera también tú, cuando entras en la iglesia y deseas ofrecer oraciones ante la Divinidad, las preocupaciones de este mundo y los engaños de la grandeza (las riquezas) salen y os ahogan, y no dejéis que se siembren muchos frutos.

\par 39 Porque si tu oración fuera como la de Moisés, llorarías cuarenta días y hablarías con la boca de Dios a la boca,


\par 40 asimismo Elías fue llevado al cielo en un carro de fuego, asimismo Daniel también oró en el foso del león...»

\par \textit{(Ver Recensión B, 10—14.)}

\chapter{2}

\par \textit{Recensión B}

\par 1 Vio al ángel de Dios y preguntó por los justos y los pecadores, cuándo salen de este mundo.

\par 2 El ángel dijo: «Para los justos hay luz y descanso, vida eterna, pero para los pecadores, tormentos interminables».

\par 3 Esdras dijo: «Si esto es así, entonces bienaventurados los animales y las bestias del campo y las cosas que reptan y las aves del cielo que no esperan la resurrección ni el juicio».

\par 4 El ángel dijo: «Pecas al decir esto, porque Dios ha hecho todas las cosas por amor al hombre, y el hombre por amor a Dios. Y aquellas cosas en las que Dios encuentra al hombre, por estos son los que juzgó».

\par 5 Esdras dijo: «Cuando toméis las almas de los hombres, ¿dónde los traes?

\par 6 El ángel dijo: «Yo traigo las almas de los justos para que adoren a Dios y los establezco en la atmósfera superior, y las almas de los pecadores son capturados por los demonios que están aprisionados en la atmósfera».

\par 7 Y Esdras dijo: «¿Y cuándo será liberada el alma apresada por Satanás?»

\par 8 El ángel dijo: «Cuando el alma tiene a alguien como buen recuerdo en este mundo, (este) se libera quitárselo a Satanás mediante la oración y (actos de) misericordia».

\par 9 Esdras dijo: «¿Por qué medios?» El ángel dijo: «Con oración, con (actos de) misericordia y con sacrificios». (Esdras dijo)

\par 10 «Si el alma del pecador no tiene un buen recuerdo que le ayude, ¿qué sucederá? para él?

\par 11 El ángel le dijo: «Éste está en manos de Satanás hasta que la venida de Cristo, cuando suene la trompeta de Gabriel.

\par 12 Entonces las almas son liberadas de las manos de Satanás y se elevan desde la atmósfera.

\par 13 Y vienen y se unen cada uno con su cuerpo que había sido convertido en polvo y al que el sonido de la trompeta había construido, despertado y renovado.

\par 14 Y lo levanta delante de Cristo nuestro Dios, que viene a juzgar (a los que están en) la tierra, es decir, a los justos y a los impíos, y a recompensar a cada uno por sus obras». Por petición de tus profetas divinamente narrados, ten piedad de los lectores de este escrito.

\end{document}