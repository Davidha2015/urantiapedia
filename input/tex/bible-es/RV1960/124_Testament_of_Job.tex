\begin{document}


\title{Testamento de Job}

\chapter{1}

\par 1 El día que enfermó y supo que tendría que abandonar su morada corporal, reunió a sus siete hijos y a sus tres hijas y les dijo lo siguiente:

\par 2 «Hijitos, formen un círculo a mi alrededor y escuchen, y les contaré lo que el Señor hizo por mí y todo lo que me sucedió.

\par 3 Porque yo soy Job tu padre.

\par 4 Sabed, pues, hijos míos, que sois generación de un elegido y prestad atención a vuestro noble nacimiento.

\par 5 Porque yo soy de los hijos de Esaú. Mi hermano es Nacor y tu madre es Dina. Por ella he llegado a ser vuestro padre.

\par 6 Porque mi primera esposa murió con mis otros diez hijos en una muerte amarga.

\par 7 Escuchen ahora, hijos, y les revelaré lo que me pasó.

\par 8 Yo era un hombre muy rico que vivía en Oriente, en la tierra de Ausitis (Utz), y antes que el Señor me pusiera por nombre Job, me llamaba Jobab.

\par 9 El comienzo de mi prueba fue así. Cerca de mi casa estaba el ídolo de uno adorado por el pueblo; y vi constantemente que le traían holocaustos como a un dios.

\par 10 Entonces reflexioné y me dije a mí mismo: «¿Es éste el que hizo el cielo y la tierra, el mar y todos nosotros? ¿Cómo sabré la verdad?»

\par 11 Y esa noche, mientras dormía, vino una voz y me llamó: '¡Jobab! Jobab! levántate y te diré quién es el que quieres conocer.

\par 12 Pero éste a quien el pueblo ofrece holocaustos y libaciones no es Dios, sino que es el poder y la obra del Seductor (Satanás) con el que engaña al pueblo.

\par 13 Y cuando oí esto, caí al suelo y me postré diciendo:

\par 14 'Oh Señor mío, que hablas por la salvación de mi alma. Te lo ruego, si este es el ídolo de Satanás, te lo ruego, déjame ir de aquí y destruirlo y purificar este lugar.

\par 15 Porque no hay nadie que pueda prohibirme hacer esto, ya que soy el rey de esta tierra, para que los que habitan en ella ya no sean extraviados.

\par 16 Y la voz que hablaba desde la llama me respondió: 'Tú puedes purificar este lugar.

\par 17 Pero he aquí, te anuncio lo que el Señor me ordenó decirte: Porque soy el arcángel de Dios.

\par 18 Y dije: «Todo lo que se le diga a su siervo». Lo escucharé'.

\par 19 Y el arcángel me dijo: «Así habla el Señor: Si te propones destruir y quitar la imagen de Satanás, él con ira se lanzará a hacer la guerra contra ti y desplegará contra ti todos sus malicia.

\par 20 Él traerá sobre ti muchas plagas terribles y te quitará todo lo que tienes.

\par 21 Él te quitará a tus hijos y te causará muchos males.

\par 22 Entonces deberás luchar como un atleta y resistir el dolor, seguro de tu recompensa, superar las pruebas y las aflicciones.

\par 23 Pero cuando perseveres, haré renombrado tu nombre por todas las generaciones de la tierra hasta el fin del mundo.

\par 24 Y te devolveré todo lo que habías tenido, y te será dada la doble parte de lo que pierdas, para que sepas que Dios no tiene en cuenta a la persona, sino que da a cada uno el bien que merece. .

\par 25 Y también a ti te será dado, y te pondrás una corona de amaranto.

\par 26 Y en la resurrección despertarás para la vida eterna. Entonces sabrás que el Señor es justo, verdadero y poderoso.

\par 27 A lo cual, hijos míos, respondí: «Por amor de Dios, soportaré hasta la muerte todo lo que me sobrevenga, y no retrocederé».

\par 28 Entonces el ángel puso su sello sobre mí y me dejó.

\chapter{2}

\par 1 Después de esto me levanté de noche, tomé cincuenta esclavos y fui al templo del ídolo y lo destruí hasta los cimientos.

\par 2 Entonces volví a mi casa y ordené que cerraran bien la puerta; diciendo a mis porteros:

\par 3 'Si alguien pregunta por mí, no me traigas ningún informe, sino dile: Él investiga los asuntos urgentes. Él está dentro'.

\par 4 Entonces Satanás se disfrazó de mendigo y llamó fuertemente a la puerta, diciendo al portero:

\par 5 'Preséntate ante Job y dile que deseo encontrarlo',

\par 6 Y entró el portero y me dijo eso, pero me enteré que estaba estudiando.

\par 7 El Maligno, al no haber logrado esto, se fue, tomó sobre su hombro una canasta vieja y rota, entró y habló al portero, diciendo: «Dile a Job: Dame pan de tus manos para que pueda comer».

\par 8 Y cuando oí esto, le di pan quemado para que se lo diera, y le dije: «No esperes comer de mi pan, porque te está prohibido».

\par 9 Pero la portera, avergonzada de darle el pan quemado y cubierto de cenizas, porque no sabía que era Satanás, tomó de su propio pan fino y se lo dio.

\par 10 Pero él lo tomó y, sabiendo lo que pasaba, dijo a la doncella: «Vete, mala sierva, y tráeme el pan que te dieron en la mano».

\par 11 Y el siervo lloró y habló con tristeza: 'Dices la verdad, diciendo que soy un mal siervo. porque no he hecho como me ordenó mi maestro'.

\par 12 Y él se volvió, le trajo el pan quemado y le dijo: «Así dice mi señor: No comerás más de mi pan, porque te está prohibido».

\par 13 Y esto me lo dio [diciendo: Esto lo doy] para que no se me acuse de no haberlo dado al enemigo que me lo pidió.')

\par 14 Y cuando Satanás oyó esto, envió al criado de regreso a mí, diciendo: 'Así como ves este pan todo quemado, así pronto quemaré tu cuerpo para que quede así'.

\par 15 Y yo respondí: 'Haz lo que desees y realiza todo lo que te propongas. Porque estoy dispuesto a soportar cualquier cosa que me traigas'.

\par 16 Y cuando el diablo oyó esto, me dejó y, subiendo hasta debajo del cielo, juró al Señor que tendría poder sobre todos mis bienes.

\par 17 Y después de haber tomado el poder, fue y al instante me quitó todas mis riquezas.

\chapter{3}

\par 1 Porque tenía ciento treinta mil ovejas, y de ellas separé siete mil para vestir a los huérfanos y a las viudas, a los necesitados y a los enfermos.

\par 2 Tenía un rebaño de ochocientos perros que cuidaban mis ovejas y además de estos doscientos vigilaban mi casa.

\par 3 Y tenía nueve molinos que trabajaban para toda la ciudad y barcos para transportar mercancías, y los instalé en cada ciudad y en cada aldea para los débiles, los enfermos y los desafortunados.

\par 4 Y tenía trescientos cuarenta mil asnos nómadas, de los cuales reservé quinientos, y ordené vender sus crías y dar el producto a los pobres y necesitados.

\par 5 Porque de todos los países vinieron a mi encuentro pobres.

\par 6 Porque las cuatro puertas de mi casa estaban abiertas, cada una de ellas a cargo de un centinela, que tenía que ver si venía alguien a pedir limosna y si me veían sentado a una de las puertas, para poder salir por el otro y tomar lo que necesitaban.

\par 7 También tenía treinta mesas fijas preparadas a todas horas para los extranjeros solos, y también tenía doce mesas preparadas para las viudas.

\par 8 Y si alguno venía pidiendo limosna, encontraba en mi mesa comida para llevarse todo lo que necesitaba, y yo no rechazaba a nadie para que saliera de mi puerta con el estómago vacío.

\par 9 También tenía tres mil quinientas yuntas de bueyes, y de estos quinientos escogí y los hice arar.

\par 10 Y con ellos hice todo el trabajo en cada campo por parte de aquellos que querían encargarse de él, y los ingresos de sus cosechas los puse a un lado para los pobres en su mesa.

\par 11 También tenía cincuenta panaderías desde donde enviaba [el pan] a la mesa de los pobres.

\par 12 Y hice seleccionar esclavos para su servicio.

\par 13 También hubo algunos extraños que vieron mi buena voluntad; ellos mismos querían servir como camareros.

\par 14 Otros, estando en apuros y sin poder ganarse la vida, vinieron con la petición diciendo:

\par 15 Te rogamos, ya que también nosotros podemos desempeñar este oficio de camareros (diáconos) y no tenemos posesión, ten piedad de nosotros y adelantanos dinero para que podamos ir a las grandes ciudades y vender mercancías.

\par 16 Y el excedente de nuestras ganancias lo daremos como ayuda a los pobres, y luego te devolveremos el tuyo (dinero).

\par 17 Y cuando oí esto, me alegré de que me quitaran todo esto para dedicarme a la caridad para los pobres.

\par 18 Y de buen corazón les di lo que querían y acepté su garantía escrita, pero no quise aceptar de ellos ninguna otra garantía excepto el documento escrito.

\par 19 Y se fueron al extranjero y, en la medida en que tuvieron éxito, dedicaron tiempo a los pobres.

\par 20 Sin embargo, con frecuencia algunas de sus mercancías se perdían en el camino o en el mar, o eran despojadas de ellas.

\par 21 Entonces venían y decían: 'Te rogamos que actúes con generosidad hacia nosotros para que podamos ver cómo podemos devolverte lo tuyo'.

\par 22 Cuando oí esto, me compadecí de ellos y les entregué su bono, y después de leerlo muchas veces delante de ellos, lo rompí y los liberé de su deuda, diciéndoles:

\par 23 «Lo que he consagrado para el beneficio de los pobres, no te lo quitaré».

\par 24 Por eso no acepté nada de mi deudor.

\par 25 Y un hombre de corazón alegre vino a mí y me dijo: No necesito que me obliguen a ser un trabajador asalariado para los pobres.

\par 26 Pero yo quiero servir a tu mesa a los necesitados», y él aceptó trabajar y comió su parte.

\par 27 Sin embargo, le di su salario y regresé a casa contento.

\par 28 Y como él no quiso tomarlo, lo obligué a hacerlo, diciéndole: «Sé que eres un trabajador que busca y espera su salario, y debes tomarlo».

\par 29 Nunca postergué el pago del salario del asalariado ni de ningún otro, ni retuve en mi casa ni una sola noche el salario que le correspondía.

\par 30 Los que ordeñaban las vacas y las ovejas hacían señales a los transeúntes para que tomaran su parte.

\par 31 Porque la leche corría en tanta abundancia que se cuajaba hasta convertirse en mantequilla en las colinas y al lado del camino; y junto a las peñas y a los collados yacía el ganado que había parido a su descendencia.

\par 32 Porque mis siervos se cansaron de guardar la carne de las viudas y de los pobres y de dividirla en trozos pequeños.

\par 33 Porque maldecían y decían: «¡Ojalá tuviéramos de su carne para poder estar satisfechos!», aunque yo era muy amable con ellos,

\par 34 También tenía seis arpas [y seis esclavos para tocar las arpas] y también una cítara, un decacordio, y lo tocaba durante el día.

\par 35 Y tomé la cítara, y las viudas respondieron después de comer.

\par 36 Y con el instrumento musical les recordé a Dios que debían alabar al Señor.

\par 37 Y cuando mis esclavas murmuraban, entonces tomaba los instrumentos musicales y tocaban tanto como ellas hubieran podido hacerlo por su salario, y les daba un respiro de su trabajo y de sus suspiros.

\chapter{4}

\par 1 Y mis hijos, después de encargarse del servicio, comían cada día con sus tres hermanas, comenzando por el hermano mayor, e hacían un banquete.

\par 2 Y me levanté por la mañana y ofrecí por ellos cincuenta carneros y diecinueve ovejas como expiación, y lo que sobró lo consagré a los pobres.

\par 3 Y les dije: «Tomen esto como residuo y oren por mis hijos.

\par 4 Quizás mis hijos pecaron delante del Señor, diciendo con altivez de espíritu: Somos hijos de este hombre rico. Nuestros son todos estos bienes; ¿Por qué deberíamos ser servidores de los pobres?

\par 5 Y al hablar así con espíritu altivo, es posible que hayan provocado la ira de Dios, porque el orgullo arrogante es abominación ante el Señor.

\par 6 Entonces llevé bueyes como ofrenda al sacerdote en el altar, diciendo: «Que mis hijos nunca piensen mal de Dios en su corazón».

\par 7 Mientras vivía de esta manera, el Seductor no pudo soportar ver el bien que [yo hacía] y exigió la guerra de Dios contra mí.

\par 8 Y vino sobre mí cruelmente.

\par 9 Primero quemó la gran cantidad de ovejas, luego los camellos, luego quemó el ganado vacuno y todos mis rebaños; o fueron capturados no sólo por los enemigos sino también por los que habían recibido beneficios de mí.

\par 10 Y vinieron los pastores y me lo anunciaron.

\par 11 Pero cuando lo oí, alabé a Dios y no blasfemé.

\par 12 Y cuando el Seductor conoció mi fortaleza, trazó nuevas cosas contra mí.

\par 13 Se disfrazó de rey de Persia y sitió mi ciudad, y después de sacar a todos los que allí había, les habló con malicia, diciendo en lenguaje jactancioso:

\par 14 «Este Job, que se apoderó de todos los bienes de la tierra y no dejó nada para los demás, destruyó y derribó el templo de Dios.

\par 15 Por tanto, le pagaré lo que ha hecho a la casa del gran dios.

\par 16 Ahora ven conmigo y saquearemos todo lo que quede en su casa.

\par 17 Ellos respondieron y le dijeron: Tiene siete hijos y tres hijas.

\par 18 Tened cuidado, no sea que huyan a otras tierras y se conviertan en nuestros tiranos y luego vengan sobre nosotros con fuerza y ​​nos maten.

\par 19 Y él dijo: No temáis en absoluto. Sus rebaños y sus riquezas destruí con fuego, y el resto los capturé, y he aquí, a sus hijos mataré».

\par 20 Y habiendo dicho esto, fue y arrojó la casa sobre mis hijos y los mató.

\par 21 Y mis conciudadanos, al ver que lo que él decía se había cumplido, vinieron y me persiguieron, y me robaron todo lo que había en mi casa.

\par 22 Y vi con mis propios ojos el saqueo de mi casa, y hombres sin cultura y sin honor se sentaban a mi mesa y en mis sofás, y no podía protestar contra ellos.

\par 23 Porque estaba exhausta como una mujer con los lomos sueltos por muchos dolores, recordando principalmente que esta guerra me había sido predicha por el Señor por medio de su ángel.

\par 24 Y me volví como aquel que, al ver el mar embravecido y los vientos adversos, cuando la carga del barco en medio del océano es demasiado pesada, arroja la carga al mar, diciendo:

\par 25 «Quiero destruir todo esto sólo para llegar sano y salvo a la ciudad y poder tomar como beneficio el barco rescatado y lo mejor de mis cosas».

\par 26 Así me las arreglé para administrar mis propios asuntos.

\par 27 Pero vino otro mensajero y me anunció la ruina de mis propios hijos, y me estremecí de terror.

\par 28 Y rasgué mis vestidos y dije: El Señor ha dado, el Señor ha quitado. Como le pareció mejor al Señor, así ha llegado a ser. Que el nombre del Señor sea bendito».

\chapter{5}

\par 1 Y cuando Satanás vio que no podía desesperarme, fue y pidió mi cuerpo al Señor para infligirme plaga, porque el Maligno no podía soportar mi paciencia.

\par 2 Entonces el Señor me entregó en sus manos para que usara mi cuerpo como quisiera, pero no le dio poder sobre mi alma.

\par 3 Y vino a mí mientras yo estaba sentado en mi trono, todavía llorando por mis hijos.

\par 4 Y como un gran huracán, volcó mi trono y me arrojó al suelo.

\par 5 Y estuve acostado en el suelo durante tres horas. y me hirió con una plaga dura desde la coronilla hasta la punta de los pies.

\par 6 Y con gran terror y aflicción salí de la ciudad y me senté en un muladar, con el cuerpo devorado por los gusanos.

\par 7 Y mojé la tierra con la humedad de mi cuerpo dolorido, porque de mi cuerpo fluía materia y muchos gusanos lo cubrían.

\par 8 Y cuando un solo gusano salió de mi cuerpo, lo puse de nuevo diciendo: «Permanece en el lugar donde has sido colocado hasta que Aquel que te envió te ordene a otra parte».

\par 9 Así estuve varios años sentado en un muladar, fuera de la ciudad, mientras estaba azotado por la peste.

\par 10 Y vi con mis propios ojos a mis hijos anhelados [llevados por ángeles al cielo]

\par 11 Y mi humilde esposa, que había sido llevada a su cámara nupcial con tanto lujo y con lanceros como guardaespaldas. La vi hacer el trabajo de aguador como esclava en casa de un hombre común para ganar un poco de pan y traérmelo.

\par 12 Y en mi dolorosa aflicción dije: «¡Oh, si estos gobernantes fanfarrones de la ciudad, a quienes nunca pensé que fueran iguales a mis perros pastores, ahora emplearan a mi esposa como sirvienta!»

\par 13 Y después de esto volví a tener valor.

\par 14 Sin embargo, después le retuvieron incluso el pan, para que sólo tuviera su propio sustento.

\par 15 Pero ella lo tomó y lo repartió entre ella y yo, diciendo con tristeza: «¡Ay de mí! En seguida ya no podrá alimentarse de pan, ni podrá ir al mercado a pedir pan a los panaderos para traérmelo y comer».

\par 16 Y cuando Satanás se enteró de esto, tomó la apariencia de un vendedor de pan, y fue como por casualidad que mi esposa lo encontró y le pidió pan pensando que era esa clase de hombre.

\par 17 Pero Satanás le dijo: «Dame el valor y luego toma lo que quieras».

\par 18 A lo cual ella respondió diciendo: ¿De dónde sacaré dinero? ¿No sabes qué desgracia me ha sucedido? Si tienes piedad, muéstramela; si no, ya lo verás».

\par 19 Y él respondió diciendo: «Si no merecieras esta desgracia, no habrías sufrido todo esto.

\par 20 Ahora bien, si no tienes moneda de plata en tu mano, dame el cabello de tu cabeza y toma por él tres panes, para que vivas allí tres días.

\par 21 Entonces se dijo a sí misma: «¿Cuál es el cabello de mi cabeza en comparación con el de mi marido hambriento?»

\par 22 Entonces ella, después de haber reflexionado sobre el asunto, le dijo: «Levántate y córtame el cabello».

\par 23 Entonces tomó unas tijeras, le cortó el cabello en presencia de todos y le dio tres panes.

\par 24 Entonces ella los tomó y me los trajo. Y Satanás iba detrás de ella por el camino, ocultándose mientras caminaba, y perturbando mucho su corazón.

\chapter{6}

\par 1 E inmediatamente mi esposa se me acercó y, llorando y llorando, dijo: «¡Job! ¡Trabajo! ¡Cuánto tiempo estarás sentado en el muladar fuera de la ciudad, reflexionando todavía por un tiempo y esperando obtener tu tan esperada salvación!

\par 2 Y he estado vagando de un lugar a otro, vagando como un jornalero, y he aquí que su memoria ya se ha extinguido de la tierra.

\par 3 Y mis hijos y las hijas que llevé en mi seno y los trabajos y dolores que soporté han sido en vano

\par 4 Y tú estás sentado en el estado maloliente de dolor y gusanos, pasando las noches en el aire frío.

\par 5 Y he pasado por todas las pruebas, angustias y dolores, día y noche, hasta lograr traerte pan.

\par 6 Porque ya no me es permitido vuestro excedente de pan; y como apenas puedo tomar mi propia comida y repartirla entre nosotros, pensé en mi corazón que no era justo que tuvieras dolor y hambre de pan.

\par 7 Y entonces me aventuré a ir al mercado sin vergüenza. y cuando el panadero me dijo: «Dame dinero. y tendrás pan». Le revelé nuestro estado de angustia.

\par 8 Entonces le oí decir: «Si no tienes dinero, pásame el pelo de tu cabeza y toma tres panes para que puedas vivir de ellos durante tres días».

\par 9 Y yo cedí ante el mal y le dije: «¡Levántate y córtame el cabello!» y se levantó y, avergonzado, me cortó con las tijeras el pelo de la cabeza en la plaza del mercado, mientras la multitud estaba parada y maravillada.

\par 10 ¿Quién entonces no se asombraría diciendo: «¿Es ésta Sitis, la esposa de Job, que tenía catorce cortinas para cubrir su sala interior y puertas dentro de las puertas, para que fuera muy honrado quien se acercara a ella, y ¡Mira ahora que ella cambia su cabello por pan!

\par 11 Que tenían camellos cargados de mercancías. y fueron llevados a tierras remotas a los pobres, ¡y ahora ella vende su cabello por pan!

\par 12 ¡Mira a aquella que tenía en su casa siete mesas puestas inmóviles en las que comía cada pobre y cada extraño, y ahora vende su cabello por pan!

\par 13 He aquí aquella que tenía una palangana para lavarse los pies hecha de oro y plata, y ahora camina sobre la tierra y [¡vende su cabello por pan!]

\par 14 ¡Mira a aquella que tenía sus vestidos hechos de biso entretejidos con oro, y ahora cambia sus cabellos por pan!

\par 15 ¡Mira a aquella que tenía lechos de oro y de plata y ahora vende sus cabellos por pan!

\par 16 En resumen, Job, después de tantas cosas que me han dicho, ahora te digo en una palabra:

\par 17 »Ya que la debilidad de mi corazón ha aplastado mis huesos, levántate entonces, toma estos panes y disfrútalos, y luego habla alguna palabra contra el Señor y muere.

\par 18 Porque yo también cambiaría el letargo de la muerte por el sustento de mi cuerpo».

\par 19 Pero yo le respondí: He aquí, he estado siete años azotada por la peste, y he soportado los gusanos de mi cuerpo, y todos estos dolores no me han agobiado en el alma.

\par 20 Y en cuanto a la palabra que dices: «¡Di alguna palabra contra Dios y muere!», yo soportaré contigo el mal que ves. y soportemos la ruina de todo lo que tenemos.

\par 21 Sin embargo, deseas que digamos alguna palabra contra Dios y que Él sea cambiado por el gran Plutón [el dios del mundo inferior].

\par 22 ¿Por qué no te acuerdas de los grandes bienes que poseíamos? Si estos bienes provienen de las tierras del Señor, ¿no deberíamos nosotros también soportar males y ser altivos en todo hasta que el Señor tenga nuevamente misericordia y se apiada de nosotros?

\par 23 ¿No ves al Seductor detrás de ti y confunde tus pensamientos para engañarme?

\par 24 Y se volvió hacia Satanás y le dijo: «¿Por qué no vienes abiertamente a mí? Deja de esconderte, miserable,

\par 25 ¿El león muestra su fuerza en la jaula de la comadreja o el pájaro vuela en la canasta? Ahora te digo: Vete y haz tu guerra contra mí».

\par 26 Entonces salió de detrás de mi esposa y se puso delante de mí llorando y dijo: He aquí, Job, cedo y cedo ante ti, que no eres más que carne, mientras que yo soy espíritu.

\par 27 Tú estás apestado, pero yo estoy en gran angustia.

\par 28 Porque soy como un luchador que lucha con un luchador que, en un combate con una sola mano, derribó a su antagonista, lo cubrió de polvo y le rompió todos los miembros, mientras que el otro que yace debajo, habiendo mostrado su valentía, emite sonidos de triunfo que dan testimonio de su propia excelencia superior.

\par 29 Así tú, Job, estás abatido y azotado por la plaga y el dolor, y sin embargo has obtenido la victoria en la lucha conmigo, y he aquí, me rindo ante ti».

\par 30 Entonces me dejó avergonzado.

\par 31 Ahora bien, hijos míos, mostrad también vosotros un corazón firme en todo el mal que os suceda, porque mayor que todas las cosas es la firmeza de corazón.

\chapter{7}

\par 1 En ese momento los reyes oyeron lo que me había pasado y se levantaron y vinieron a mí. cada uno desde su tierra para visitarme y consolarme.

\par 2 Y cuando se acercaron a mí, gritaron a gran voz y cada uno rasgó sus vestidos.

\par 3 Y después de postrarse, tocando la tierra con la cabeza, se sentaron a mi lado durante siete días y siete noches, y ninguno habló una palabra.

\par 4 Eran cuatro en número: Eliplaz, rey de Temán, Balad, Sofar y Elilhú.

\par 5 Y cuando tomaron asiento, hablaron de lo que me había sucedido.

\par 6 Cuando por primera vez vinieron a mí y les mostré mis piedras preciosas, se sorprendieron y dijeron:

\par 7 «Si de nosotros tres reyes todas nuestras posesiones se juntaran en una sola, no alcanzaría las piedras preciosas del reino (corona) de Jobab. Porque eres de mayor nobleza que todos los pueblos de Oriente.

\par 8 Y cuando vinieron a visitarme a la tierra de Ausitis «Uz», preguntaron en la ciudad: «¿Dónde está Jobab, el gobernante de toda esta tierra?»

\par 9 Y les contaron de mí: «Está sentado en un muladar fuera de la ciudad, porque hace siete años que no entra en ella».

\par 10 Y luego volvieron a preguntar por mis bienes, y se les reveló todo lo que me había sucedido.

\par 11 Cuando supieron esto, salieron de la ciudad con los habitantes, y mis conciudadanos me señalaron.

\par 12 Pero éstos protestaron y dijeron: «Seguramente éste no es Jobab».

\par 13 Y mientras ellos dudaban, dijo Elifaz. el rey de Temán: «Venid, acerquémonos y veamos».

\par 14 Y cuando se acercaron, me acordé de ellos y lloré mucho cuando supe el propósito de su viaje.

\par 15 Y arrojé tierra sobre mi cabeza y, mientras sacudía la cabeza, les revelé que yo era [Job].

\par 16 Y cuando me vieron menear la cabeza, se arrojaron al suelo, todos abrumados por la emoción.

\par 17 Y mientras sus ejércitos estaban alrededor, vi a los tres reyes tirados en el suelo como muertos durante tres horas.

\par 18 Entonces se levantaron y se dijeron unos a otros: No podemos creer que éste sea Jobab».

\par 19 Y finalmente, después de siete días de indagar por todo lo referente a mí y buscar mis rebaños y otras posesiones, dijeron:

\par 20 ¿No sabemos cuántos bienes envió a las ciudades y a los pueblos de los alrededores para dárselos a los pobres, aparte de todo lo que regaló dentro de su propia casa? ¡Qué estado de perdición y de miseria!

\par 21 Y después de los siete días, Eliú dijo a los reyes: «Venid, acerquémonos y examinémoslo con precisión, si realmente es Jobab o no».

\par 22 Y ellos, estando a menos de media milla (estadio) de su cuerpo maloliente, se levantaron y se acercaron, llevando perfume en sus manos, mientras sus soldados iban con ellos y arrojaban a su alrededor incienso aromático para que pudieran venir. cerca de mí.

\par 23 Y después de haber transcurrido así tres horas, cubriendo de olor el camino, se acercaron.

\par 24 Entonces Elifaz comenzó a decir: «¿Eres tú, Job, nuestro rey compañero? ¿A ti te pertenece la gran gloria?

\par 25 ¿Eres tú el que una vez brilló como el sol del día sobre toda la tierra? ¿Eres tú el que una vez se pareció a la luna y a las estrellas refulgiendo durante toda la noche?

\par 26 Yo le respondí y dije: «Yo soy». Entonces todos lloraron y se lamentaron, y cantaron un cántico real de lamentación, y todo su ejército se unió a ellos en un coro.

\par 27 Y otra vez Elifaz me dijo: «¿Eres tú el que ordenó que se dieran siete mil ovejas para vestir a los pobres? ¿Adónde se ha ido la gloria de tu trono?

\par 28 ¿Eres tú quien ordenó a tres mil cabezas de ganado que araran el campo para el pobre Wither? ¡Entonces tu gloria se habrá ido!

\par 29 ¿Eres tú el que tenía lechos de oro y ahora estás sentado sobre un monte de estiércol?

\par 30 ¿Eres tú el que tenía sesenta mesas puestas para los pobres? ¿Eres tú el que tenía un incensario para el fino perfume hecho de piedras preciosas, y ahora estás en un estado maloliente? ¿A dónde se ha ido entonces tu gloria?

\par 31 ¿Eres tú el que tenía candelabros de oro sobre pedestales de plata? y ahora debes anhelar el brillo natural de la luna [«¡Adónde se ha ido entonces tu gloria!»]

\par 32 ¿Eres tú el que tenía un ungüento hecho con especias de incienso y ahora estás en un estado repulsivo? [«¡Adónde pues se ha ido tu gloria!»]

\par 33 ¿Eres tú quien se burlaba de los malhechores y pecadores y ahora te has convertido en el hazmerreír de todos? [«¿Adónde pues se ha ido tu gloria?»]

\par 34 Y como Elifaz estuvo llorando y lamentándose durante mucho tiempo, mientras todos los demás se le unían, de modo que el alboroto era muy grande, yo les dije:

\par 35 Guarda silencio y te mostraré mi trono y la gloria de su esplendor: Mi gloria será eterna.

\par 36 El mundo entero perecerá y su gloria se desvanecerá, y todos los que se aferren a él permanecerán abajo, pero mi trono está en el mundo superior y su gloria y esplendor estarán a la derecha del Salvador en los cielos.

\par 37 Mi trono existe en la vida de los «santos» y su gloria en el mundo imperecedero.

\par 38 Porque los ríos se secarán y su arrogancia descenderá al fondo del abismo, pero los arroyos de mi tierra en la que está erigido mi trono no se secarán, sino que permanecerán intactos en fuerza.

\par 39 Los reyes perecen y los gobernantes desaparecen, y su gloria y orgullo son como la sombra en un espejo, pero mi Reino dura por los siglos de los siglos, y su gloria y hermosura están en el carro de mi Padre.

\chapter{8}

\par 1 Cuando les hablé así, Ehifaz se enojó y dijo a los demás amigos: «¿Para qué hemos venido aquí con nuestros anfitriones para consolarlo? He aquí, él nos reprende. Por tanto, volvamos a nuestros países.

\par 2 Este hombre se sienta aquí en la miseria, carcomido en medio de un estado de putrefacción insoportable, y sin embargo desafía su salvación: 'Los reinos perecerán y sus gobernantes, pero mi Reino, dice, durará para siempre'».

\par 3 Entonces Elifaz se levantó muy alborotado y, apartándose de ellos con gran furor, dijo: «Me voy de aquí. Es cierto que hemos venido a consolarlo, pero él nos declara la guerra a la vista de nuestros ejércitos».

\par 4 Pero entonces Baldad lo tomó de la mano y le dijo: «No se debe hablar así a un hombre afligido, y especialmente a uno que ha sido afligido por tantas plagas.

\par 5 He aquí, nosotros, que gozamos de buena salud, no nos atrevimos a acercarnos a él a causa del olor desagradable, excepto con la ayuda de un abundante aroma fragante. Pero tú, Elifaz. Arte olvidadizo de todo esto.

\par 6 Déjame hablar claramente. Seamos magnánimos y aprendamos cuál es la causa ¿Debe él, al recordar sus pasados ​​días de felicidad, volverse loco en su mente?

\par 7 ¿Quién no debería quedar completamente perplejo al verse caer así en desgracias y plagas? Pero déjame acercarme a él para descubrir por qué está así».

\par 8 Y Baldad se levantó y se acercó a mí diciendo: «¿Eres tú Job?» y dijo: «¿Está todavía en buen estado tu corazón?»

\par 9 Y dije: «No me aferré a las cosas terrenales, ya que la tierra y todos los que la habitan son inestables. Pero mi corazón se aferra al cielo, porque en el cielo no hay angustia».

\par 10 Entonces Baldad respondió y dijo: «Sabemos que la tierra es inestable, porque cambia según las estaciones. A veces se encuentra en estado de paz y a veces en estado de guerra. Pero del cielo escuchamos que está perfectamente estable.

\par 11 Pero, ¿estás verdaderamente en un estado de calma? Por tanto, déjame preguntar y hablar, y cuando me respondas a mi primera palabra, tendré una segunda pregunta que hacerte, y si de nuevo respondes con palabras bien formuladas, será manifiesto que tu corazón no ha sido desequilibrado».

\par 12 Y dije: «¿En qué pones tu esperanza?» Y dije: «En el Dios vivo».

\par 13 Y él me dijo: «¿Quién te privó de todo lo que poseías y quién te infligió estas plagas?» Y yo dije: «Dios».

\par 14 Y él dijo: «Si todavía pones tu esperanza en Dios, ¿cómo puede Él hacer mal en tu juicio, habiendo traído sobre ti estas plagas y desgracias, y habiéndote quitado todas tus posesiones?

\par 15 Y puesto que Él los tomó, está claro que no te ha dado nada. Ningún rey deshonrará a su soldado que le ha servido bien como guardaespaldas»

\par 16 [Y respondí diciendo]: «¿Quién entiende las profundidades del Señor y de Su sabiduría para poder acusar a Dios de injusticia»

\par 17 [Y Baldad dijo]: «Respóndeme, oh Job, a esto. Nuevamente te digo: 'Si estás en un estado de razón tranquila, enséñame si tienes sabiduría:

\par 18 ¿Por qué vemos el sol salir por el Este y ponerse por el Oeste? Y nuevamente cuando sale por la mañana lo encontramos salir por el Este. Dime lo que piensas sobre esto»

\par 19 Entonces dije: «¿Por qué traicionaré (balbucearé) los grandes misterios de Dios y mi boca tropezará al revelar cosas que pertenecen al Maestro? ¡Nunca!

\par 20 ¿Quiénes somos nosotros para entrometernos en asuntos relacionados con el mundo superior, siendo sólo de carne, más aún, de tierra y cenizas?

\par 21 Para que sepas que mi corazón está sano, escucha lo que te pido:

\par 22 Por el estómago viene el alimento, y por la boca se bebe el agua, y luego corre por la misma garganta, y cuando los dos descienden hasta convertirse en excremento, se separan de nuevo; quién efectúa esta separación».

\par 23 Y Baldad dijo: «No lo sé». Y yo repliqué y le dije: «Si no entiendes ni siquiera las salidas del cuerpo, ¿cómo podrás entender los circuitos celestes?»

\par 24 Entonces Sofar respondió y dijo: «No investigamos nuestros propios asuntos, pero deseamos saber si estás en buen estado, y he aquí, vemos que tu razón no ha sido sacudida.

\par 25 ¿Qué quieres ahora que hagamos por ti? He aquí, hemos venido aquí y hemos traído a los médicos de tres reyes, y si quieres, puedes curarlo con ellos.

\par 26 Pero respondí y dije: «Mi curación y mi restauración vienen de Dios, el Hacedor de los médicos».

\chapter{9}

\par 1 Y mientras les hablaba así, he aquí, mi esposa Sitis vino corriendo, vestida con harapos. del servicio del amo para quien era empleada como esclava, aunque se le había prohibido salir, para que los reyes, al verla, la tomaran cautiva.

\par 2 Y cuando llegó, se postró a sus pies, llorando y diciendo: «Recordad». Elifaz y otros amigos, lo que fui una vez con vosotros, y cómo he cambiado, cómo estoy ahora vestido para recibiros.

\par 3 Entonces los reyes prorrumpieron en grandes llantos y, doblemente perplejos, guardaron silencio. Pero Elifaz tomó su manto púrpura y se lo echó sobre ella para envolverse con él.

\par 4 Pero ella le preguntó diciendo: «Les pido como favor, señores míos, que ordenen a sus soldados que excaven entre las ruinas de nuestra casa que cayeron sobre mis hijos, para que sus huesos puedan ser traídos en un perfecto estado a los sepulcros.

\par 5 Pero como no tenemos poder alguno a causa de nuestra desgracia, al menos podremos ver sus huesos.

\par 6 Porque tengo como un bruto el sentimiento maternal de las fieras salvajes de que mis diez hijos hubieran muerto en un día y a ninguno de ellos pudiera darle un entierro digno.

\par 7 Y los reyes ordenaron que se desenterraran las ruinas de mi casa. Pero lo prohibí, salvando

\par 8 «No os hagáis problemas en vano; porque mis hijos no serán encontrados, porque están bajo el cuidado de su Hacedor y Gobernante''.

\par 9 Y los reyes respondieron y dijeron: «¿Quién podrá negar que está loco y delira?

\par 10 Porque aunque deseamos recuperar los huesos de sus hijos, él nos lo prohíbe diciendo: «Han sido tomados y puestos bajo custodia de su Hacedor». Por tanto, pruébanos la verdad».

\par 11 Pero yo les dije: «Levantadme para que pueda estar de pie», y ellos me levantaron, levantando mis brazos a ambos lados.

\par 12 Y me puse de pie y pronuncié primero la alabanza de Dios y después de la oración les dije: «Mirad con vuestros ojos hacia el Este».

\par 13 Y miraron y vieron a mis hijos con coronas cerca de la gloria del Rey, el Soberano del cielo.

\par 14 Y cuando mi esposa Sitis vio esto, cayó al suelo y se postró ante Dios, diciendo: «Ahora sé que mi memoria permanece con el Señor».

\par 15 Y después de haber dicho esto, cuando llegó la tarde, volvió a la ciudad, volvió al amo a quien servía como esclavo, se acostó en el pesebre del ganado y allí murió de cansancio.

\par 16 Y cuando su despótico amo la buscó y no la encontró, llegó al redil de sus rebaños y allí la vio muerta tendida en el pesebre, mientras todos los animales que la rodeaban lloraban por ella.

\par 17 Y todos los que la vieron lloraron y se lamentaron, y el clamor se extendió por toda la ciudad.

\par 18 Y la gente la bajó, la envolvieron y la enterraron junto a la casa que había caído sobre sus hijos.

\par 19 Y los pobres de la ciudad hicieron gran duelo por ella y dijeron: «He aquí esta Sitis, cuya nobleza y gloria no se encuentran en ninguna mujer. Pobre de mí ! ¡No fue encontrada digna de una tumba adecuada!

\par 20 El canto fúnebre por ella lo encontrarás en el acta.

\chapter{10}

\par 1 Pero Elifaz y los que estaban con él quedaron asombrados de estas cosas, y se sentaron conmigo y, respondiéndome, hablaron de mí con palabras jactanciosas durante veintisiete días.

\par 2 Me repetían una y otra vez que yo sufría con razón por haber cometido muchos pecados y que ya no me quedaba esperanza, pero yo mismo respondía a estos hombres con afán de contienda.

\par 3 Y ellos se levantaron enojados, dispuestos a partir con espíritu de ira. Pero Eliú los conjuró para que se quedaran todavía un poco de tiempo hasta que les mostrara de qué se trataba.

\par 4 «Porque tantos días pasaste, dijo, permitiendo que Job se jactara de ser justo. Pero ya no lo sufriré más.

\par 5 Porque desde el principio seguí llorando por él, recordando su antigua felicidad. Pero ahora habla con jactancia y con orgullo arrogante dice que tiene su trono en los cielos.

\par 6 Por tanto, escúchame y te diré cuál es la causa de su destino.

\par 7 Entonces, imbuido del espíritu de Satanás. Eliú pronunció palabras duras que están escritas en los registros que dejó Eliú.

\par 8 Y cuando terminó, Dios se me apareció en una tormenta y en las nubes, y habló, culpando a Eliú y mostrándome que el que había hablado no era un hombre, sino una bestia salvaje.

\par 9 Y cuando Dios terminó de hablarme, el Señor habló a Elifaz: «Tú y tus amigos habéis pecado por no haber dicho la verdad acerca de mi siervo Job.

\par 10 Levántate, pues, y hazle traer por ti una ofrenda por el pecado, para que tus pecados sean perdonados; porque si no fuera por él, os habría destruido».

\par 11 Entonces me trajeron todo lo que era necesario para el sacrificio, y yo lo tomé y les presenté una ofrenda por el pecado, y el Señor la recibió favorablemente y les perdonó su agravio.

\par 12 Entonces, cuando Elifaz, Baldad y Sofar vieron que Dios les había perdonado sus pecados por medio de su siervo Job, pero que no se dignaba perdonar a Eliú, entonces Elifaz comenzó a cantar un himno, mientras los demás respondían, y también sus soldados. unirse estando de pie junto al altar.

\par 13 Y Elifaz habló así: «El pecado ha sido quitado y nuestra injusticia ha desaparecido;

\par 14 Pero Eliú, el maligno, no tendrá memoria entre los vivientes; su lumbrera se ha apagado y ha perdido su luz.

\par 15 La gloria de su lámpara se anunciará ante él, porque es hijo de las tinieblas. y no de luz.

\par 16 Los porteros del lugar de las tinieblas le darán en parte su gloria y hermosura; Su Reino se ha desvanecido, su trono se ha desmoronado, y el honor de su estatura está en (Seol) Hades.

\par 17 Porque ha amado la belleza de la serpiente y las escamas (piel) del dragón, su hiel y su veneno pertenecen al del Norte (Zphuni = Víbora).

\par 18 Porque ni se reconocía ante el Señor ni le temía, sino que odiaba a los que él había elegido.

\par 19 Así Dios se olvidó de él, y los santos lo abandonaron; su ira y su enojo serán para él desolación y no tendrá misericordia en su corazón ni paz, porque tenía veneno de víbora en su lengua. .

\par 20 Justo es el Señor, y sus juicios son verdaderos. Para él no hay preferencia de persona, porque juzga a todos por igual.

\par 21 ¡He aquí que el Señor viene! He aquí, los 'santos' están preparados: ¡las coronas y los premios de los vencedores los preceden!

\par 22 Que se alegren los santos y que se regocije su corazón; porque recibirán la gloria que les está reservada.

\par \textit{Estribillo.}

\par 23 Nuestros pecados son perdonados, nuestra injusticia ha sido limpiada, pero Eliú no tiene memoria entre los vivos».

\par 24 Cuando Elifaz terminó el himno, nos levantamos y volvimos a la ciudad, cada uno a su casa.

\par 25 Y el pueblo me hizo un banquete en agradecimiento y deleite de Dios, y todos mis amigos volvieron a mí.

\par 26 Y todos los que me habían visto en mi anterior estado de felicidad, me preguntaban diciendo: «¿Cuáles son esas tres cosas aquí entre nosotros?»

\chapter{11}

\par 1 Pero yo, deseoso de retomar mi obra de beneficencia hacia los pobres, les pregunté diciendo:

\par 2 «Dadme a cada uno un cordero para vestir a los pobres en su estado de desnudez, y cuatro dracmas (monedas) de plata u oro»

\par 3 Entonces el Señor bendijo todo lo que me quedaba, y al cabo de unos días volví a hacerme rico en mercancías, en rebaños y en todo lo que había perdido, y volví a recibirlo todo en doble cantidad.

\par 4 Entonces también tomé por esposa a tu madre y te engendré a ti diez en lugar de los diez hijos que habían muerto.

\par 5 Y ahora, hijos míos, permítanme advertirles: «He aquí, muero. Tomarás mi lugar.

\par 6 Sólo que no abandonéis al Señor. Sed caritativos con los pobres; No despreciéis a los débiles. No toméis para vosotros esposas de extraños.

\par 7 He aquí, hijos míos, repartiré entre vosotros lo que poseo, para que cada uno tenga dominio sobre lo suyo y tenga pleno poder para hacer el bien con su parte».

\par 8 Y habiendo dicho esto, trajo todos sus bienes y los repartió entre sus siete hijos, pero no dio nada de sus bienes a sus hijas.

\par 9 Entonces dijeron a su padre: «¡Señor y padre nuestro! ¿No somos también nosotros tus hijos? ¿Por qué, entonces, no nos das también una parte de tus bienes?

\par 10 Entonces Job dijo a sus hijas: «No os enojéis, hijas mías. No te he olvidado. He aquí, os he guardado una posesión mejor que la que tomaron vuestros hermanos».

\par 11 Y llamó a su hija cuyo nombre era Day (Yemima) y le dijo: «Toma este anillo doble que sirve de llave y ve a la casa del tesoro y tráeme el cofre de oro, para que te entregue tu posesión. ».

\par 12 Ella fue y se lo trajo, y él lo abrió y sacó unos cinturones de tres hilos, de cuyo aspecto nadie puede hablar.

\par 13 Porque no eran trabajos terrenales, sino que chispas de luz celestial brillaban a través de ellos como los rayos del sol.

\par 14 Y dio un cordón a cada una de sus hijas y dijo: «Póntelos como cinturones para que te rodeen todos los días de tu vida y te colmen de todo bien».

\par 15 Y la otra hija, cuyo nombre era Kassiah, dijo: «¿Es ésta la posesión que tú dices que es mejor que la de nuestros hermanos? ¿Ahora qué podemos vivir de esto?»

\par 16 Y su padre les dijo: «No sólo tenéis aquí lo suficiente para vivir, sino que esto os llevará a un mundo mejor para vivir, en los cielos.

\par 17 ¿O no sabéis, hijos míos, el valor de estas cosas aquí? ¡Oíd pues! Cuando el Señor me consideró digno de tener compasión de mí y quitar de mi cuerpo las plagas y los gusanos, me llamó y me entregó estas tres cuerdas.

\par 18 Y me dijo: «Levántate y ciñe tus lomos como un hombre que te demandaré y me lo declararás».

\par 19 Y los tomé y los ciñí alrededor de mis lomos, e inmediatamente los gusanos abandonaron mi cuerpo, y lo mismo hicieron las plagas, y todo mi cuerpo tomó nuevas fuerzas gracias al Señor, y así seguí adelante, como si hubiera nunca sufrió.

\par 20 Pero también en mi corazón me olvidé de los dolores. Entonces el Señor me habló en Su gran poder y me mostró todo lo que era y será.

\par 21 Ahora bien, hijos míos, al guardar esto, no tendréis al enemigo conspirando contra vosotros ni intenciones [malas] en vuestra mente porque esto es un hechizo (Phylacterion) del Señor.

\par 22 Levántate, pues, y ciñe esto a tu alrededor antes de que muera, para que puedas ver a los ángeles venir en mi despedida, para que puedas contemplar con asombro los poderes de Dios».

\par 23 Entonces se levantó la que se llamaba Day (Yemima) y se ciñó; e inmediatamente partió de su cuerpo, como había dicho su padre, y se vistió de otro corazón, como si nunca le importaran las cosas terrenas.

\par 24 Y cantaba himnos angelicales con voz de ángeles, y cantaba alabanzas angelicales a Dios mientras danzaba.

\par 25 Entonces la otra hija, llamada Kassia, se puso el cinturón y su corazón se transformó, de modo que ya no deseaba las cosas mundanas.

\par 26 Y su boca asumió el dialecto de los gobernantes celestiales (Arcontes) y cantó la donología de la obra del Lugar Alto y si alguien desea conocer la obra de los cielos puede comprender los himnos de Kassia. .

\par 27 Entonces se ciñó la otra hija llamada el Cuerno de Amaltea (Keren Happukh), y su boca hablaba en el idioma de los de lo alto; porque su corazón fue transformado, siendo elevado por encima de las cosas del mundo.

\par 28 Ella habló en el dialecto de los querubines, cantando alabanzas al Soberano de los poderes (virtudes) cósmicos y ensalzando su (Su) gloria.

\par 29 Y quien desee seguir los vestigios de la «Gloria del Padre» los encontrará escritos en las Oraciones del Cuerno de Amaltea.

\chapter{12}

\par 1 Después de que estos tres terminaron de cantar himnos. ¿Me senté junto a él, Nacor (Neros), hermano de Job, mientras él se acostaba?

\par 2 Y escuché las cosas maravillosas (grandes) de las tres hijas de mi hermano, una sucediéndose siempre a la otra en medio de un silencio terrible.

\par 3 Y escribí este libro que contiene los himnos excepto los himnos y signos de la [santa] Palabra, porque estas eran las grandes cosas de Dios.

\par 4 Y Job, enfermo, se acostó en su lecho, pero sin dolor ni sufrimiento, porque el dolor no se apoderaba de él con fuerza a causa del encanto del cinto que se había enrollado.

\par 5 Pero después de tres días, Job vio que los santos ángeles venían por su alma, y ​​al instante se levantó, tomó la cítara y se la dio a su hija Day (Yemima).

\par 6 Y a Kassia le dio un incensario (con perfume = Kassia), y al cuerno de Amaltea (= música) le dio un pandero para que pudieran bendecir a los santos ángeles que vinieron por su alma.

\par 7 Y tomándolos, cantaron, tocaron el salterio y alabaron y glorificaron a Dios en santo dialecto.

\par 8 Después de esto vino el que estaba sentado en el gran carro y besó a Job, mientras sus tres hijas miraban, pero las demás no lo vieron.

\par 9 Y tomó el alma de Job y se elevó hacia arriba, tomándola (el alma) por el brazo y llevándola en el carro, y se dirigió hacia el Este.

\par 10 Pero su cuerpo fue llevado al sepulcro, mientras las tres hijas marchaban delante, vestidas con sus cinturones y cantando himnos de alabanza a Dios.

\par 11 Entonces Nacor, su hermano Nereo, y sus siete hijos, con el resto del pueblo y los pobres, los huérfanos y los débiles, hicieron gran duelo sobre él, diciendo:

\par 12 ¡Ay de nosotros! porque hoy nos ha sido quitada la fuerza de los débiles, la luz de los ciegos, el padre de los huérfanos;

\par 13 Le ha sido quitado al líder de los extraviados el receptor de los extraños, la cobertura de los desnudos. el escudo de las viudas. ¿Quién no lloraría por el hombre de Dios?

\par 14 Y como estaban de luto en tal o cual forma, no permitieron que lo enterraran.

\par 15 Sin embargo, después de tres días, finalmente fue puesto en la tumba, como quien duerme dulcemente, y recibió el nombre del bueno (hermoso) que seguirá siendo famoso por todas las generaciones del mundo.

\par 16 Dejó siete hijos y tres hijas, y no se encontraron hijas en la tierra tan hermosas como las hijas de Job.

\par 17 El nombre de Job antes era Jobab, y el Señor lo llamó Job.

\par 18 Había vivido antes de la plaga ochenta y cinco años, y después de la plaga recibió el doble de todo; de ahí que también duplicó su año, que es 170 años. Así vivió en total 255 años.

\par 19 Y vio hijos de sus hijos hasta la cuarta generación. Está escrito que se levantará con aquellos a quienes el Señor despertará. A nuestro Señor por la gloria. Amén.

\end{document}