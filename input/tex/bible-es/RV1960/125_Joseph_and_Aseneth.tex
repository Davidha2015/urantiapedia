\begin{document}


\title{José y Asenet}

\chapter{1}

\par \textit{La confesión y oración de Asenat, hija del sacerdote Pentefres}

\par \textit{Asenath es buscado en matrimonio por el Hijo del Rey y muchos otros.}

\par 1 En el primer año de abundancia, en el segundo mes, el día cinco del mes, Faraón envió a José a recorrer toda la tierra de Egipto; y en el mes cuarto del año primero, a los dieciocho del mes,

\par 2 José llegó a los límites de Heliópolis,

\par 3 y recogía el trigo de aquella tierra como la arena del mar.

\par 4 Y había en aquella ciudad un hombre llamado Pentefres, que era sacerdote de Heliópolis y sátrapa de Faraón, y jefe de todos los sátrapas y príncipes de Faraón;

\par 5 Y este hombre era sumamente rico y muy sabio y gentil, y también era consejero de Faraón, porque era más prudente que todos los príncipes de Faraón.

\par 6 Y tenía una hija virgen, llamada Asenat, de dieciocho años, alta y hermosa, y de una belleza incomparablemente mayor que cualquier virgen sobre la tierra.

\par 7 La propia Asenat no se parecía en nada a las vírgenes hijas de los egipcios, sino que era en todo semejante a las hijas de los hebreos.

\par 8 alta como Sara, hermosa como Rebeca y hermosa como Raquel;

\par 9 y la fama de su belleza se difundió por toda aquella tierra y hasta los confines del mundo, de modo que por esto todos los hijos de los príncipes y los sátrapas quisieron cortejarla, y también los hijos de los reyes también, todos ellos jóvenes y valientes,

\par 10 Y hubo gran contienda entre ellos a causa de ella, y trataron de pelear unos contra otros.

\par 11 También el hijo primogénito de Faraón se enteró de ella y siguió rogando a su padre que se la diera por esposa.

\par 12 y diciéndole: «Dame, padre, por esposa a Asenat, hija de Pentefres, el primer hombre de Heliópolis».

\par 13 Y su padre Faraón le dijo: «¿Por qué buscas una esposa menor que tú, siendo rey de toda esta tierra?

\par 14 No, pero ¡he aquí! La hija de Joacim, rey de Moab, está desposada contigo, y ella misma es reina y muy hermosa de contemplar. Toma entonces ésta para ti como esposa».

\chapter{2}

\par \textit{Se describe la Torre en la que vive Asenath.}

\par 1 Pero Asenat menospreciaba y despreciaba a todos, siendo jactanciosa y altiva, y nunca nadie la había visto, ya que Pentefres tenía en su casa una torre contigua, grande y muy alta,

\par 2 y encima de la torre había un loft que contenía diez cámaras.

\par 3 Y la primera cámara era grande y muy hermosa, y estaba pavimentada con piedras de púrpura, y sus paredes estaban revestidas con piedras preciosas y de muchos colores.

\par 4 y también el techo de aquella cámara era de oro. Y dentro de aquella cámara estaban fijados los dioses de los egipcios, de los cuales no había número, oro y plata,

\par 5 Y Asenat adoraba a todos aquellos, y ella los temía, y les ofrecía sacrificios todos los días.

\par 6 Y en la segunda cámara también estaban todos los adornos y los cofres de Asenat,

\par 7 y había en él oro, mucha plata y vestidos de oro sin límites, y piedras escogidas y preciosas,

\par 8 y vestidos finos de lino y todos los adornos de su virginidad estaban allí.

\par 9 Y la tercera cámara era el almacén de Asenat, que contenía todos los bienes de la tierra.

\par 10 Y las siete cámaras restantes las ocuparon las siete vírgenes que servían a Asenat,

\par 11 cada uno tenía una cámara, porque eran de la misma edad, nacieron la misma noche que Asenat, y ella los amaba mucho; y también eran extremadamente hermosas como las estrellas del cielo, y nunca ningún hombre conversó con ellas ni con un niño varón.

\par 12 La gran cámara de Asenat, donde se preservaba su virginidad, tenía tres ventanas;

\par 13 Y la primera ventana era muy grande y daba al atrio hacia el oriente; y el segundo miraba hacia el sur, y el tercero miraba hacia la calle.

\par 14 Y en la cámara que miraba hacia el oriente había una cama de oro;

\par 15 Y la cama estaba cubierta de tela de púrpura entretejida con oro, y la cama estaba tejida de tela escarlata, carmesí y lino fino.

\par 16 En esta cama durmió sola Asenat, y nunca se sentó en ella ningún hombre ni ninguna mujer.

\par 17 Y había también un gran atrio alrededor de la casa, y un muro muy alto alrededor del atrio, construido con grandes piedras rectangulares;

\par 18 También había en el atrio cuatro puertas revestidas de hierro, cada una de las cuales estaba custodiada por dieciocho jóvenes fuertes y armados;

\par 19 y también se plantaron a lo largo del muro árboles hermosos, de todas clases y de todos los que daban fruto, y sus frutos estaban maduros, porque era el tiempo de la cosecha;

\par 20 y había también una rica fuente de agua que brotaba a la derecha del mismo atrio; y debajo de la fuente había una gran cisterna que recibía el agua de aquella fuente, de donde salía como un río por en medio del atrio y regaba todos los árboles de aquel atrio.

\chapter{3}

\textit{José anuncia su acuñación a Pentefres.}

\par 1 Y aconteció que en el primer año de los siete años de abundancia, en el mes cuarto, el día veintiocho del mes, José llegó a los límites de Heliópolis recogiendo el trigo de aquella región.

\par 2 Y cuando José llegó cerca de esa ciudad, envió doce hombres delante de él a Pentefres, el sacerdote de Heliópolis, diciendo: «Hoy entraré a ti, porque es la hora del mediodía y de la tarde. merienda de mediodía,

\par 3 y hay mucho calor del sol, y puedo refrescarme bajo el techo de tu casa.

\par 4 Y Pentefres, al oír estas cosas, se alegró con gran alegría y dijo:

\par 5 «Bendito sea el Señor Dios de José, porque mi señor José me ha tenido por digno». Y Pentefres llamó al mayordomo de su casa y le dijo:

\par 6 «Date prisa, prepara mi casa y prepara una gran comida, porque hoy viene a nosotros José, el Poderoso de Dios».

\par 7 Y cuando Asenat supo que su padre y su madre habían regresado de la posesión de su herencia,

\par 8 Ella se alegró mucho y dijo: «Iré a ver a mi padre y a mi madre, porque han venido de la posesión de nuestra herencia» (pues era la época de la cosecha).

\par 9 Entonces Asenat se apresuró a entrar en su cámara, donde estaban sus ropas, y se vistió con un manto de lino fino hecho de tela carmesí y entretejido de oro, y se ciñó un cinto de oro y brazaletes alrededor de sus manos; y a sus pies se puso borceguíes de oro,

\par 10 Y alrededor de su cuello se puso un adorno de gran precio y piedras preciosas, las cuales estaban adornadas por todos lados, y también tenía grabados los nombres de los dioses de los egipcios por todas partes, tanto en los brazaletes como en las piedras;

\par 11 y se puso también una tiara en la cabeza, se ciñó una diadema alrededor de las sienes y se cubrió la cabeza con un manto.

\chapter{4}

\textit{Pentefres propone darle Asenat a José en matrimonio.}

\par 1 Entonces ella se apresuró a bajar las escaleras de su desván y se acercó a su padre y a su madre y los besó.

\par 2 Y Pentefres y su esposa se regocijaron con gran alegría por su hija Asenat, porque la vieron adornada y embellecida como la esposa de Dios;

\par 3 y sacaron todos los bienes que habían traído de la posesión de su herencia y se los dieron a su hija;

\par 4 Y Asenat se regocijó por todos los bienes, por las frutas tardías del verano, por las uvas, por los dátiles, por las palomas, por las moras y por los higos, porque todos eran hermosos y agradables al paladar.

\par 5 Y Pentefres dijo a su hija Asenat: «Niña». Y ella dijo: «Aquí estoy, mi señor».

\par 6 Y él le dijo: «Siéntate entre nosotros y te hablaré mis palabras». Y se sentó entre su padre y su madre,

\par 7 Y Pentefres, su padre, tomó con la suya la mano derecha de ella, la besó tiernamente y le dijo: «Querida hija». Y ella le dijo: «Aquí estoy, mi señor padre».

\par 8 Y Pentefres le dijo: «¡He aquí! José, el Poderoso de Dios, viene hoy a nosotros, y este hombre es gobernante de toda la tierra de Egipto; y el rey Faraón lo nombró gobernante de toda nuestra tierra y rey, y él mismo da trigo a todo este país, y lo salva de la hambruna venidera;

\par 9 Y este José es un hombre que adora a Dios, y discreto y virgen como tú hoy, y un hombre poderoso en sabiduría y conocimiento, y el espíritu de Dios está sobre él y la gracia del Señor está en él. a él.

\par 10 Ven, hija querida, y te entregaré a él por esposa, y tú serás para él por esposa, y él mismo será tu esposo para siempre.

\par 11 Cuando Asenat oyó estas palabras de su padre, un gran sudor se derramó sobre su rostro y se enojó con gran ira.

\par 12 Y ella miró de reojo a su padre y dijo: «¿Por qué, señor padre, hablas estas palabras? ¿Quieres entregarme cautivo a un extranjero, a un fugitivo y a un vendido?

\par 13 ¿No es éste el hijo del pastor de la tierra de Canaán? y él mismo ha sido dejado atrás por él.

\par 14 ¿No es éste el que se acostó con su ama, y ​​su señor lo arrojó en la prisión de las tinieblas, y Faraón lo sacó de la prisión, interpretando su sueño, como también lo interpretan las ancianas de los egipcios?

\par 15 No, sino que me casaré con el hijo primogénito del rey, porque él mismo es rey de toda la tierra.

\par 16 Al oír estas cosas, Pentefres se avergonzó de seguir hablando de José con su hija Asenat, porque ella le respondía con jactancia y enojo.

\chapter{5}

\textit{José llega a la casa de Pentefres.}

\par 1 ¡Y he aquí! Entonces entró un joven de los siervos de Pentefres y le dijo:

\par 2 «¡Mira! José está ante las puertas de nuestro patio». Y cuando Asenat escuchó estas palabras, huyó de la presencia de su padre y de su madre y subió al desván, entró en su cámara y se paró junto a la gran ventana mirando hacia el este para ver a José entrando en la casa de su padre.

\par 3 Y Pentefres salió al encuentro de José con su mujer y todos sus parientes y sus sirvientes;

\par 4 Y cuando se abrieron las puertas del atrio que daba al oriente, entró José sentado en el segundo carro de Faraón;

\par 5 y estaban unidos cuatro caballos blancos como la nieve, con frenos de oro, y el carro era todo de oro puro.

\par 6 Y José estaba vestido con una túnica blanca y preciosa, y el manto que lo rodeaba era púrpura, hecho de lino fino entretejido con oro, y una corona de oro estaba sobre su cabeza, y alrededor de su corona había doce piedras escogidas. , y sobre las piedras doce rayos dorados,

\par 7 y en su mano derecha un bastón real, en el que había una rama de olivo extendida y sobre ella abundante fruto.

\par 8 Cuando José entró en el atrio y se cerraron las puertas,

\par 9 y todos los hombres y mujeres extraños se quedaron fuera del atrio, por lo que los guardias de las puertas se acercaron y cerraron las puertas,

\par 10 Vinieron Pentefres, su mujer y todos sus parientes, excepto su hija Asenat, y se postraron sobre sus rostros en tierra ante José;

\par 11 José descendió de su carro y los saludó con la mano.

\chapter{6}

\textit{Asenath ve a José desde la ventana.}

\par 1 Cuando Asenat vio a José, sintió un gran dolor en el alma y su corazón destrozado.

\par 2 y sus rodillas se aflojaron y todo su cuerpo tembló y tuvo mucho miedo, y luego gimió y dijo en su corazón: «¡Ay de mí, miserable! ¿Adónde ahora me iré yo, el desdichado? ¿O dónde me esconderé de su presencia? ¿O cómo me verá José, hijo de Dios, que por mi parte he hablado mal de él? ¡Ay de mí, miserable!

\par 3 ¿A dónde iré y me esconderé, porque él mismo ve cada escondite y todo lo sabe, y nada oculto se le escapa a causa de la gran luz que hay en él?

\par 4 Y ahora que el Dios de José tenga misericordia de mí, porque por ignorancia he hablado malas palabras contra él.

\par 5 ¿Qué debo ahora seguir yo, el desdichado? ¿No he dicho: 'Viene José, el hijo del pastor, de la tierra de Canaán'? Ahora, pues, ha venido a nosotros en su carro como el sol del cielo, y ha entrado hoy en nuestra casa, y brilla en ella como luz sobre la tierra.

\par 6 Pero yo soy necio y atrevido, porque lo desprecié y hablé mal de él, y no sabía que José era hijo de Dios.

\par 7 ¿Quién entre los hombres engendrará jamás tal belleza, o qué vientre de mujer dará a luz tal luz? Desdichada soy y necia, porque hablé malas palabras a mi padre.

\par 8 Ahora, pues, que mi padre me dé a José por sierva y esclava, y seré esclava de él para siempre.

\chapter{7}

\par \textit{José ve a Asenat en la ventana.}

\par 1 Entonces José entró en casa de Pentefres y se sentó en una silla. Y le lavaron los pies, y pusieron una mesa delante de él aparte, para que José no comiera con los egipcios, ya que esto le era abominación.

\par 2 Y José miró hacia arriba y vio a Asenat asomándose, y le dijo a Pentefres: «¿Quién es esa mujer que está parada en el desván junto a la ventana? Que se vaya de esta casa».

\par 3 Porque José temía, diciendo: «No sea que ella también me moleste». Porque todas las mujeres y las hijas de los príncipes y de los sátrapas de toda la tierra de Egipto lo molestaban para acostarse con él;

\par 4 Pero también muchas mujeres e hijas de los egipcios, todas las que veían a José, se angustiaban a causa de su hermosura;

\par 5 y a los enviados que las mujeres le enviaron con oro, plata y regalos preciosos, José los devolvió con amenazas e insultos, diciendo: «No pecaré ante los ojos del Señor Dios y ante el rostro de mi padre Israel».

\par 6 Porque José tenía a Dios siempre ante sus ojos y siempre recordaba los mandatos de su padre; porque Jacob hablaba muchas veces y amonestaba a su hijo José y a todos sus hijos: «Hijos, guardaos seguros de la mujer extraña, para no tener comunión con ella, porque la comunión con ella es ruina y destrucción».

\par 7 Entonces José dijo: «Dejad que esa mujer se vaya de esta casa».

\par 8 Y Pentefres le dijo: «Señor, esa mujer que has visto parada en el desván no es una extraña, sino que es nuestra hija, una que odia a todos los hombres, y ningún otro hombre la ha visto jamás excepto tú. hoy;

\par 9 y si quieres, señor, ella vendrá y te hablará, porque nuestra hija es como tu hermana.

\par 10 Y José se alegró con gran alegría porque Pentefres dijo: «Es una virgen que odia a todos los hombres».

\par 11 Y José dijo a Pentefres y a su esposa: «Si es vuestra hija y es virgen, que venga, porque es mi hermana, y desde hoy la amo como a mi hermana».

\chapter{8}


\par \textit{José bendice a Asenat.}

\par 1 Entonces su madre subió al desván y llevó a Asenat a José, y Pentefres le dijo: «Besa a tu hermano, porque él también es virgen como tú hoy, y odia a toda mujer extraña como tú odias». cada hombre extraño».

\par 2 Y Asenat dijo a José: «Salve, Señor, bendito del Dios Altísimo». Y José le dijo: «Dios que vivifica todas las cosas, te bendecirá, doncella».

\par 3 Pentefres dijo entonces a su hija Asenat: «Ven y besa a tu hermano».

\par 4 Cuando Asenat se acercó a besar a José, José extendió su mano derecha y la puso sobre su pecho entre sus dos pechos (porque sus pechos ya estaban erectos como hermosas manzanas), y José dijo:

\par 5 No es propio del hombre que adora a Dios, que bendice con su boca al Dios vivo, y come el bendito pan de vida, y bebe la copa bendita de la inmortalidad, y es ungido con la bendita unción de la incorrupción, besar a una mujer extraña, que bendice con su boca ídolos muertos y sordos y come de su mesa el pan de ahogo y bebe de su libación el cáliz del engaño y es ungida con unción de destrucción;

\par 6 pero el hombre que adora a Dios besará a su madre, a la hermana nacida de su madre, a la hermana nacida de su tribu y a la esposa que comparte su lecho, quienes bendicen con su boca al Dios vivo.

\par 7 Tampoco le conviene a una mujer que adora a Dios besar a un extraño, porque esto es abominación ante los ojos del Señor Dios.

\par 8 Y cuando Asenat oyó estas palabras de José, se angustió mucho y gimió; y mientras miraba fijamente a José con los ojos abiertos, estos se llenaron de lágrimas.

\par 9 Y José, al verla llorar, se compadeció mucho de ella, porque era manso, misericordioso y temeroso del Señor.

\par 10 Entonces él levantó su mano derecha sobre su cabeza y dijo:

\par «Señor Dios de mi padre Israel, Dios Altísimo y Fuerte,
\par que vivificaste todas las cosas y llamaste de las tinieblas a la luz
\par y del error a la verdad y de la muerte a la vida,
\par Bendice también a esta virgen,

\par 11 y vivificala y renuévala con tu espíritu santo,
\par y que coma el pan de tu vida y beba el cáliz de tu bendición,
\par y cuéntala con tu pueblo, a quien escogiste antes de que todas las cosas fueran hechas,
\par y déjala entrar en tu reposo que preparaste para tus elegidos,
\par y déjala vivir en tu vida eterna para siempre».

\chapter{9}

\par \textit{Asenat se retira y José se prepara para partir.}

\par 1 Y Asenat se regocijó con gran alegría por la bendición de José. Entonces ella se apresuró y subió sola a su altillo, y cayó en su cama enferma, porque había en ella alegría y tristeza y gran temor; y un sudor continuo corría sobre ella cuando oyó estas palabras de José, y cuando él le hablaba en el nombre del Dios Altísimo.

\par 2 Entonces lloró con gran y amargo llanto, y, arrepentida, se apartó de los dioses que solía adorar y de los ídolos que despreciaba, y esperó la llegada de la tarde.

\par 3 Pero José comió y bebió; y ordenó a sus siervos que uncieran los caballos a sus carros y rodearan toda la tierra.

\par 4 Y Pentefres dijo a José: «Deja que mi señor se aloje aquí hoy y por la mañana te irás».

\par 5 Y José dijo: «No, pero hoy me iré, porque éste es el día en que Dios comenzó a hacer todas sus cosas creadas, y al octavo día también volveré a vosotros y me alojaré aquí.»

\chapter{10}

\par \textit{Asenath rechaza a los dioses egipcios y se humilla.}

\par 1 Y cuando José salió de la casa, también Pentefres y todos sus parientes se fueron a su heredad.

\par 2 y Asenat se quedó sola con las siete vírgenes, desganada y llorando hasta que se puso el sol; y no comía pan ni bebía agua, sino que mientras todos dormían ella sola estaba despierta y llorando y golpeándose frecuentemente el pecho con la mano.

\par 3 Y después de estas cosas, Asenat se levantó de su cama y bajó silenciosamente las escaleras del desván, y al llegar a la puerta encontró a la portera durmiendo con sus hijos;

\par 4 Entonces ella se apresuró a quitar de la puerta la cubierta de cuero de la cortina, la llenó de cenizas, la llevó al desván y la puso en el suelo.

\par 5 Y entonces cerró bien la puerta y la aseguró con el cerrojo de hierro del costado, y gimió con grandes gemidos y con mucho y muy grande llanto.

\par 6 Pero la virgen a quien Asenat amaba más que todas las vírgenes, al oír su gemido, se apresuró a llegar a la puerta, después de despertar también a las otras vírgenes, y la encontró cerrada.

\par 7 Y cuando hubo escuchado los gemidos y el llanto de Asenat, le dijo, estando afuera: «¿Qué pasa, señora mía, y por qué estás triste? ¿Y qué es lo que te preocupa?

\par 8 Ábrenos y déjanos verte. Y Asenath le dijo, estando encerrada por dentro: «Un dolor grande y doloroso ha atacado mi cabeza, y estoy descansando en mi cama, y ​​no puedo levantarme y abrirme a ti, por lo que estoy débil en todos mis miembros. Id, pues, cada uno de vosotros a su aposento y durmid, y déjame en paz.

\par 9 Y cuando las vírgenes se fueron cada una a su aposento, Asenat se levantó, abrió silenciosamente la puerta de su alcoba y se fue a su segundo aposento, donde estaban los cofres de sus adornos, abrió su cofre y tomó una túnica negra y sombría que se puso y lloró cuando murió su hermano primogénito.

\par 10 Tomó, pues, esta túnica, la llevó a su cámara, cerró de nuevo la puerta y echó el cerrojo a un lado.

\par 11 Entonces Asenat se quitó su manto real, se puso la túnica de luto, se desató el cinto de oro, se ciñó con una cuerda y se quitó la tiara, es decir, la mitra, de su cabeza, y también la diadema, y ​​las cadenas de sus manos y de sus pies también fueron puestas en el suelo.

\par 12 Entonces tomó su manto escogido, el cinto de oro, la mitra y su diadema, y ​​los arrojó por la ventana que daba al norte, a los pobres.

\par 13 Entonces tomó todos los dioses que había en su cámara, dioses de oro y de plata, de los cuales no había número, los partió en pedazos y los arrojó por la ventana a los pobres y a los mendigos.

\par 14 Y Asenat tomó de nuevo su cena real, los animales gordos, el pescado y la carne de novilla, y todos los sacrificios de sus dioses, y los vasos del vino de la libación, y los arrojó todo por la ventana que daba al norte como alimento para los perros.

\par 15 Y después de esto, tomó la cubierta de cuero que contenía las brasas y las derramó sobre el suelo;

\par 16 Entonces ella se vistió de cilicio y se ciñó los lomos; y se soltó también la red del cabello de su cabeza y esparció ceniza sobre su cabeza. Y esparció también cenizas por el suelo,

\par 17 y cayó sobre las brasas, y se golpeaba constantemente el pecho con las manos y lloraba toda la noche con gemidos hasta la mañana.

\par 18 Y cuando Asenat se levantó por la mañana y vio, ¡he aquí! las cenizas estaban debajo de ella como arcilla de sus lágrimas,

\par 19 volvió a caer de bruces sobre las cenizas hasta que se puso el sol.

\par 20 Así hizo Asenat durante siete días, sin probar nada.

\chapter{11}

\par \textit{Asenat decide orar al Dios de los hebreos.}

\par 1 Y al octavo día, cuando amaneció y los pájaros cantaban ya y los perros ladraban a los transeúntes, Asenat levantó un poco su cabeza del suelo y de las brasas sobre las que estaba sentada, para poder Estaba extremadamente cansada y había perdido la fuerza de sus miembros debido a su gran humillación;

\par 2 porque Asenat estaba cansada y desfallecida y sus fuerzas le fallaban, y entonces se volvió hacia la pared, sentándose debajo de la ventana que miraba hacia el este;

\par 3 y recostó la cabeza sobre su seno, entrelazando los dedos de sus manos sobre su rodilla derecha;

\par 4 y su boca estuvo cerrada, y no la abrió durante los siete días y las siete noches de su humillación.

\par 5 Y ella dijo en su corazón, sin abrir la boca: «¿Qué haré, yo la humilde, o adónde iré? ¿Y en quién volveré a encontrar refugio en el futuro? ¿O a quién hablaré, a la virgen huérfana y desolada y abandonada de todos y odiada?

\par 6 Ahora todos me odian, y entre ellos incluso mi padre y mi madre, porque desprecié a los dioses con aborrecimiento, los deseché y los entregué a los pobres para que los destruyeran los hombres. Porque mi padre y mi madre dijeron: «Asenath no es nuestra hija».

\par 7 Pero también todos mis parientes y todos los hombres han llegado a odiarme por haber entregado sus dioses a la destrucción. Y he odiado a todos los hombres y a todos los que me cortejaban, y ahora en esta mi humillación he sido odiado por todos y se alegran de mi tribulación.

\par 8 Pero el Señor y Dios del poderoso José aborrece a todos los que adoran ídolos, porque es un Dios celoso y terrible, como he oído, contra todos los que adoran a dioses extraños; por lo que también me ha odiado, porque adoré a ídolos muertos y sordos y los bendije.

\par 9 Pero ahora he rechazado sus sacrificios, y mi boca se ha alejado de su mesa, y no tengo valor para invocar al Señor Dios del cielo, el Altísimo y poderoso del poderoso José, para que mi boca está contaminada por los sacrificios de los ídolos.

\par 10 Pero he oído a muchos decir que el Dios de los hebreos es un Dios verdadero, un Dios vivo, un Dios misericordioso, compasivo, sufrido, lleno de misericordia y gentil, y que no toma en cuenta el pecado de del hombre humilde, y especialmente del que peca por ignorancia, y no convence de iniquidad en el tiempo de la aflicción del hombre afligido;

\par 11 Por eso también yo, el humilde, seré valiente y me volveré a él y buscaré refugio en él, le confesaré todos mis pecados y derramaré mi petición delante de él, y él tendrá misericordia de mi miseria.

\par 12 Porque ¿quién sabe si verá mi humillación y la desolación de mi alma y se compadecerá de mí, y verá también la orfandad de mi miseria y virginidad y me defenderá?

\par 13 porque, según he oído, él mismo es padre de huérfanos, consuelo de los afligidos y ayuda de los perseguidos. Pero en cualquier caso también yo, el humilde, seré atrevido y clamaré a él.

\par 14 Entonces Asenat se levantó del muro donde estaba sentada, se puso de rodillas hacia el oriente, dirigió sus ojos al cielo, abrió su boca y dijo a Dios:

\chapter{12}

\par \textit{La oración de Asenat}

\par 1 La oración y confesión de Asenat:

\par 2 «Señor Dios de los justos, que creaste los siglos y diste vida a todas las cosas,
\par que diste el aliento de vida a toda tu creación,
\par quien sacó a la luz las cosas invisibles,
\par quien hiciste todas las cosas y manifestaste las cosas que no aparecían,

\par 3 el que alzó los cielos y fundó la tierra sobre las aguas,
\par que fijaste las grandes piedras sobre el abismo del agua,
\par que no serán sumergidos sino que están hasta el fin haciendo tu voluntad,
\par porque tú Señor, dijiste la palabra y todas las cosas vinieron a existir, y tu palabra, Señor, es la vida de todas tus criaturas, a ti acudo en busca de refugio,

\par 4 Señor, Dios mío, desde ahora a ti clamaré, Señor,
\par y a ti confesaré mis pecados, a ti derramaré mi petición, Maestro,
\par y a ti revelaré mis rebeliones.

\par 5 Perdóname, Señor, perdóname por haber cometido muchos pecados contra ti,
\par cometí iniquidad e impiedad,
\par He hablado cosas indecibles, y malas delante de ti;
\par Mi boca, Señor, ha sido contaminada por los sacrificios de los ídolos de Egipto,
\par y de la mesa de sus dioses:

\par 6 Pequé, Señor, pequé ante ti, tanto en conocimiento como en ignorancia.
\par Hice impiedad adorando ídolos muertos y sordos,
\par y no soy digno de abrirte mi boca, Señor,

\par 7 Yo, la miserable Asenat, hija del sacerdote Pentefres, virgen y reina,
\par que en otro tiempo fue orgulloso y altivo y que prosperó en las riquezas de mi padre sobre todos los hombres,
\par pero ahora huérfano y desolado y abandonado de todos los hombres.
\par A ti acudo, Señor, y a ti te ofrezco mi petición,
\par y a ti clamaré.

\par 8 Líbrame de los que me persiguen, Maestro, antes de que me apresen;
\par porque, como un niño que, temeroso de alguien, huye hacia su padre y su madre,
\par y su padre extiende sus manos y lo agarra contra su pecho,
\par así también tú, Señor, extiende sobre mí tus manos inmaculadas y terribles, como un padre amante de los niños,
\par y cógeme de la mano del enemigo suprasensual.

\par 9 ¡Por lo! Me persigue el león viejo, salvaje y cruel,
\par porque es padre de los dioses de los egipcios,
\par y los dioses de los idólatras son sus hijos,
\par y llegué a aborrecerlos, y los destruí,
\par porque son hijos de león,
\par y eché de mí todos los dioses de los egipcios y los aniquilé,
\par y el león, o su padre el diablo, enojado contra mí, intenta devorarme.

\par 10 Pero tú, Señor, líbrame de sus manos,
\par y seré librado de su boca,
\par no sea que me desgarre y me arroje en la llama del fuego,
\par y el fuego me arrojó en tormenta,
\par y la tormenta se apoderará de mí en la oscuridad y me arrojará a lo profundo del mar,
\par y la gran bestia que es desde la eternidad me tragará,
\par y perezco para siempre.

\par 11 Líbrame, Señor, antes de que me sobrevengan todas estas cosas;
\par líbrame, Maestro, al desolado e indefenso,
\par por eso mi padre y mi madre me han negado y dicho:
\par 'Asenath no es nuestra hija'
\par porque desmenuzé a sus dioses y los aniquilé,
\par como si los hubiera odiado por completo. Y ahora estoy huérfano y desolado, y no tengo otra esperanza salvo ti.

\par 12 Señor, ni otro refugio sino tu misericordia, amigo de los hombres,
\par porque tú sólo eres padre de los huérfanos y paladín de los perseguidos y auxiliador de los afligidos.
\par Ten piedad de mi. Señor, y mantenme pura y virgen,
\par los abandonados y huérfanos, por eso sólo tú.
\par Señor, eres un padre dulce, bueno y gentil.
\par ¿Qué padre es dulce y bueno como tú, Señor?
\par ¡Por he aquí! todas las casas de mi padre Pentefres
\par que él me ha dado por herencia son por un tiempo y van a desaparecer;
\par pero las casas de tu herencia, Señor, son incorruptibles y eternas».

\chapter{13}

\par \textit{La oración de Asenat (continuación).}

\par 1 «Visita, Señor, mi humillación y ten piedad de mi orfandad y compadécete de mí, el afligido. ¡Por he aquí! Yo, Maestro, huí de todos y busqué refugio en ti, el único amigo de los hombres.

\par 2 ¡Mira! Dejé todas las cosas buenas de la tierra y busqué refugio en ti. Señor, vestido de cilicio y de ceniza, desnudo y solitario.

\par 3 ¡Mira! Ahora me despojo de mi manto real de lino fino y de tela carmesí entretejida de oro, y me pongo una túnica negra de luto. ¡Mira! Me desaté el cinto de oro y lo arrojé de mí y me ciñí con cuerda y cilicio.

\par 4 ¡Mira! mi diadema y mi mitra he echado de mi cabeza y me he rociado con cenizas,

\par 5 ¡Mira! El suelo de mi cámara, que estaba pavimentado con piedras multicolores y púrpura, que antes estaba humedecido con ungüentos y secado con lienzos de lino brillante, ahora está humedecido con mis lágrimas y ha sido deshonrado porque está cubierto de cenizas.

\par 6 ¡He aquí, Señor mío, de las cenizas y de mis lágrimas se ha formado mucho barro en mi cámara como en un camino ancho!

\par 7 ¡He aquí, Señor mío, mi comida real y las viandas que he dado a los perros!

\par 8 ¡Mira! También yo, Maestro, he estado ayunando siete días y siete noches y no comí pan ni bebí agua, y mi boca está seca como una rueda y mi lengua como cuerno y mis labios como un tiesto, y mi rostro se ha encogido y mis ojos. no han dejado de derramar lágrimas.

\par 9 Pero tú, Señor Dios mío, líbrame de mis muchas ignorancias y perdóname porque, siendo virgen y sin saberlo, me he descarriado. ¡Mira! ahora todos los dioses que antes adoraba en ignorancia ahora he conocido que eran ídolos sordos y muertos, y los rompí en pedazos y los entregué para que fueran pisoteados por todos los hombres, y los ladrones los saquearon, que eran oro y plata. , y en ti busqué refugio. Señor Dios, el único compasivo y amigo de los hombres.

\par 10 Perdóname, Señor, porque cometí muchos pecados contra ti por ignorancia y pronuncié palabras blasfemas contra mi señor José, sin saber, el miserable, que es tu hijo. Señor, ya que los malvados impulsados ​​por la envidia me dijeron: José es hijo de un pastor de la tierra de Canaán, y yo, el miserable, les he creído y me he descarriado, y lo he despreciado y he hablado cosas malas de él. , sin saber que es tu hijo.

\par 11 ¿Quién entre los hombres engendró o engendrará tal belleza? ¿O quién más es como él, sabio y poderoso como el todo hermoso José? Pero a ti. Señor, lo encomiendo, porque por mi parte lo amo más que a mi alma.

\par 12 Guárdalo en la sabiduría de tu gracia, y encoméndame a él como sierva y esclava, para que le lave los pies, le haga la cama, le sirva y le sirva, y seré su esclava. él para los momentos de mi vida».

\chapter{14}

\par \textit{El Arcángel Miguel visita Asenath.}

\par 1 Y cuando Asenat dejó de confesarse al Señor, ¡he aquí! la estrella de la mañana también surgió del cielo en el oriente;

\par 2 Y Asenat lo vio y se alegró y dijo: «¿Ha escuchado el Señor Dios mi oración? porque esta estrella es mensajera y heralda de la luz del gran día».

\par 3 ¡Y he aquí! Muy cerca de la estrella de la mañana, el cielo se rasgó y apareció una luz grande e inefable.

\par 4 Cuando ella vio esto, Asenat cayó de bruces sobre las brasas, y en seguida se le acercó un hombre del cielo, lanzando rayos de luz, y se paró sobre su cabeza. Y mientras ella yacía boca abajo, el ángel divino le dijo: «Asenat, levántate».

\par 5 Y ella dijo: «¿Quién es el que me llamó para que la puerta de mi cámara esté cerrada y la torre sea alta, y cómo, pues, ha entrado en mi cámara?».

\par 6 Y volvió a llamarla por segunda vez, diciendo: Asenat, Asenat. Y ella dijo: «Aquí estoy, Señor, dime quién eres».

\par 7 Y él dijo: «Yo soy el capitán en jefe del Señor Dios y el comandante de todo el ejército del Altísimo: levántate y ponte sobre tus pies para hablarte mis palabras».

\par 8 Y ella alzó su rostro y vio, ¡y he aquí! un hombre en todo semejante a José, en manto, corona y bastón real,

\par 9 salvo que su rostro era como un relámpago, y sus ojos como la luz del sol, y los cabellos de su cabeza como la llama de fuego de una antorcha encendida, y sus manos y sus pies como hierro resplandeciente en el fuego, porque Como si de sus manos y de sus pies salieran chispas.

\par 10 Al ver estas cosas, Asenat tuvo miedo y cayó de bruces, sin poder ni siquiera mantenerse en pie, porque tuvo mucho miedo y todos sus miembros temblaban.

\par 11 Y el hombre le dijo: «Ten ánimo, Asenat, y no temas; sino levántate y ponte sobre tus pies, para que yo te hable mis palabras».

\par 12 Entonces Asenat se levantó y se puso de pie, y el ángel le dijo:

\par 13 Entra sin impedimento en tu segunda cámara, y deja a un lado la túnica negra con la que estás vestido, y quita el cilicio de tus lomos, sacude las brasas de tu cabeza y lava tu cara y tus manos con agua pura. y viste un manto blanco, intacto, y ciñe tus lomos con el cinto resplandeciente de la virginidad, el doble,

\par 14 y vuelve a mí y te hablaré las palabras que te envíe el Señor.

\par 15 Entonces Asenat se apresuró y entró en su segunda cámara, donde estaban los cofres de sus adornos, abrió su cofre, tomó un manto blanco, hermoso e intacto y se lo vistió, después de haberse quitado el manto negro.

\par 16 y se desató también la cuerda y el cilicio de sus lomos y se ciñó con un doble cinto brillante de su virginidad, un cinto alrededor de sus lomos y otro cinto alrededor de su pecho.

\par 17 Y se sacudió también las brasas de su cabeza, se lavó las manos y la cara con agua pura, tomó un manto bellísimo y fino y se cubrió la cabeza con un velo.

\chapter{15}

\par \textit{Miguel le dice a Asenat que ella será la esposa de José.}

\par 1 Y entonces ella vino al divino capitán en jefe y se paró ante él, y el ángel del Señor le dijo: «Quita ahora el manto de tu cabeza, porque hoy eres una virgen pura, y tu cabeza es como de un joven.»

\par 2 Y Asenat se lo quitó de la cabeza. Y nuevamente el ángel divino le dijo: «Ten ánimo, Asenath, virgen y pura, porque he aquí, el Señor Dios escuchó todas las palabras de tu confesión y de tu oración, y ha visto también la humillación y aflicción de la Siete días de tu abstinencia, porque de tus lágrimas se ha formado mucho barro delante de tu rostro sobre estas brasas.

\par 3 Por lo tanto, ¡alégrate, Asenat, la virgen y pura, porque he aquí! tu nombre ha sido escrito en el libro de la vida y no será borrado para siempre;

\par 4 pero a partir de este día serás renovado, remodelado y reavivado, y comerás el bendito pan de la vida, beberás una copa llena de inmortalidad y serás ungido con la bendita unción de la incorrupción.

\par 5 ¡Ten ánimo, Asenat, virgen y pura, he aquí! Jehová Dios te ha dado hoy por esposa a José, y él mismo será tu esposo para siempre.

\par 6 Y nunca más te llamarán Asenat, sino que tu nombre será Ciudad de Refugio, porque en ti muchas naciones buscarán refugio y se alojarán bajo tus alas, y muchas naciones encontrarán refugio por medio de ti, y sobre tus muros serán guardados seguros los que se adhieren al Dios Altísimo mediante la penitencia;

\par 7 porque esa Penitencia es hija del Altísimo, y ella misma ruega cada hora al Dios Altísimo por ti y por todos los que se arrepienten, ya que él es padre de la Penitencia,

\par 8 y ella misma es la consumación y la guardiana de todas las vírgenes, amándote en gran manera y suplicando al Altísimo por ti cada hora, y a todos los que se arrepientan les proporcionará un lugar de descanso en los cielos, y renovará a todos los que se hayan arrepentido. arrepentido. Y la Penitencia es sumamente hermosa, una virgen pura, gentil y apacible; y por eso el Dios Altísimo la ama, y ​​todos los ángeles la reverencian, y yo la amo mucho, porque también ella es mi hermana, y como ella os ama vírgenes, yo también os amo.

\par 9 ¡Y he aquí! por mi parte, iré a José y le hablaré todas estas palabras acerca de ti, y él vendrá a ti hoy y te verá y se regocijará por ti y te amará y será tu novio, y tú serás su novia amada para siempre. alguna vez.

\par 10 Por tanto, escúchame, Asenat, y vístete el traje de bodas, el antiguo y primero traje que aún está guardado en tu cámara desde la antigüedad, y ponte también todos tus adornos escogidos, y atrápate como una buena novia. y prepárate para recibirlo;

\par 11 por lo ! él mismo vendrá a ti hoy y te verá y se regocijará».

\par 12 Y cuando el ángel del Señor en forma de hombre terminó de hablar estas palabras a Asenat, ella se regocijó con gran alegría por todo lo que él había dicho.

\par 13 y cayó rostro en tierra, se postró ante sus pies y le dijo: «Bendito sea el Señor tu Dios, que te envió para librarme de las tinieblas y sacarme de los cimientos del abismo mismo. a la luz, y bendito sea tu nombre por los siglos. Entonces, si he hallado gracia, señor mío, ante tus ojos, y sé que harás todas las palabras que me has dicho para que se cumplan, que te hable tu sierva. Y el ángel le dijo: «Di».

\par 14 Y ella dijo: «Te ruego, Señor, que te sientes un poco en esta cama, porque esta cama es pura e inmaculada, porque nunca otro hombre ni otra mujer se sentó en ella, y pondré delante de ti una mesa y pan, y comerás, y también te traeré vino añejo y bueno, cuyo olor llegará hasta el cielo, y lo beberás, y después te irás por tu camino». Y él le dijo: «Date prisa y tráelo pronto».

\chapter{16}

\par \textit{Asenath encuentra un panal en su almacén.}

\par 1 Entonces Asenat se apresuró a poner delante de él una mesa vacía; y, cuando iba a buscar pan, el divino ángel le dijo: «Tráeme también un panal de miel». Y ella se quedó quieta y estaba perpleja y afligida por no tener ni un panal en su almacén. Y el ángel divino le dijo: «¿Por qué estás quieta?»

\par 2 Y ella dijo: «Mi señor, enviaré un muchacho al suburbio, porque la posesión de nuestra herencia está cerca, y él vendrá y traerá rápidamente uno de allí, y lo pondré delante de ti».

\par 3 El ángel divino le dijo: «Entra en tu almacén y encontrarás un panal sobre la mesa; tómalo y tráelo acá». Y ella dijo: «Señor, no hay ningún panal en mi almacén». Y él dijo: «Ve y encontrarás».

\par 4 Entonces Asenat entró en su almacén y encontró un panal de miel sobre la mesa; y el panal era grande y blanco como la nieve y lleno de miel, y esa miel era como rocío del cielo, y su olor como olor de vida. Entonces Asenath se quedó perplejo y dijo para sí: «¿Este peine procede de la boca de este hombre?»

\par 5 Y Asenat tomó ese peine, lo trajo y lo puso sobre la mesa, y el ángel le dijo: «¿Por qué dices: 'No hay panal de miel en mi casa', y he aquí? ¿Me lo has traído?

\par 6 Y ella dijo: «Señor, nunca he puesto un panal de miel en mi casa, sin que así haya sido hecho como tú dijiste. ¿Salió esto de tu boca? porque su olor es como olor de ungüento».

\par 7 Y el hombre sonrió ante la comprensión de la mujer. Entonces la llamó a sí, y cuando ella llegó, extendió su mano derecha y la tomó por la cabeza, y cuando sacudió su cabeza con su mano derecha, Asenat temió mucho la mano del ángel, porque de ella salían chispas. Sus manos parecían hierro candente, y por eso ella estaba todo el tiempo mirando con mucho miedo y temblando la mano del ángel.

\par 8 Y él sonrió y dijo: «Bendito eres, Asenath, porque los inefables misterios de Dios te han sido revelados; y bienaventurados todos los que se unen al Señor Dios en arrepentimiento, porque comerán de este panal, porque este panal es el espíritu de vida, y esto lo han hecho las abejas del paraíso del deleite con el rocío de las rosas de la vida. que están en el paraíso de Dios y cada flor, y de ella comen los ángeles y todos los escogidos de Dios y todos los hijos del Altísimo, y el que de ella come, no morirá para siempre».

\par 9 Entonces el ángel divino extendió su mano derecha y tomó un pedacito del panal y comió, y con su propia mano puso lo que quedaba en la boca de Asenat y le dijo: «Come», y ella comió. Y el ángel le dijo: «¡He aquí! ahora has comido el pan de vida y has bebido la copa de la inmortalidad y has sido ungido con la unción de la incorrupción;

\par 10 lo ! ahora hoy tu carne produce flores de vida de la fuente del Altísimo, y tus huesos serán engordados como los cedros del paraíso del deleite de Dios y poderes infatigables te sustentarán;

\par 11 Por tanto, tu juventud no verá la vejez, ni tu belleza decaerá para siempre, sino que serás como una ciudad madre amurallada para todos.

\par 12 Y el ángel incitó el panal, y de las celdillas de aquel panal surgieron muchas abejas, y las celdillas eran innumerables, decenas de miles, decenas de miles y miles de miles.

\par 13 Y las abejas también eran blancas como la nieve, y sus alas como una sustancia púrpura, carmesí y escarlata; y además tenían picaduras agudas y no hirieron a nadie.

\par 14 Entonces todas aquellas abejas rodearon a Asenath desde los pies hasta la cabeza, y otras abejas grandes como sus reinas surgieron de las celdas, y rodearon su cara y sus labios, y formaron un peine sobre su boca y sobre sus labios como el peine que yacía delante del ángel; y todas aquellas abejas comieron del panal que estaba sobre la boca de Asenat.

\par 15 Y el ángel dijo a las abejas: «Id ahora a vuestro lugar».

\par 16 Entonces todas las abejas se levantaron, volaron y se fueron al cielo; pero todos los que quisieron herir a Asenath cayeron al suelo y murieron. Y entonces el ángel extendió su bastón sobre las abejas muertas.

\par 17 y les dijo: «Levántense y váyanse también ustedes a su lugar». Entonces todas las abejas muertas se levantaron y partieron hacia el patio contiguo a la casa de Asenath y se alojaron en los árboles frutales.

\chapter{17}

\par \textit{Michael se marcha.}

\par 1 Y el ángel dijo a Asenat: ¿Has visto esto? Y ella dijo: «Sí, señor mío, he visto todas estas cosas».

\par 2 El ángel divino le dijo: «Así serán todas mis palabras, tantas como te he hablado hoy».

\par 3 Entonces el ángel del Señor extendió por tercera vez su mano derecha y tocó el costado del peine, e inmediatamente salió fuego de la mesa y devoró el peine, pero la mesa no dañó ni un ápice.

\par 4 Y cuando la combustión del peine desprendía un gran aroma que llenaba la cámara, Asenat dijo al ángel divino: «Señor, tengo siete vírgenes que fueron criadas conmigo desde mi juventud y nacieron en una noche conmigo, quienes me esperan, y las amo a todas como a mis hermanas. Yo los llamaré y tú también los bendecirás como me bendijiste a mí».

\par 5 Y el ángel le dijo: «Llámalos». Entonces Asenat llamó a las siete vírgenes y las puso delante del ángel, y el ángel les dijo: «El Señor Dios Altísimo os bendecirá, y seréis [columnas] de refugio para siete ciudades, y para todos los escogidos de esa ciudad. quienes habitan juntos [sobre ti descansarán] para siempre».

\par 6 Y después de estas cosas, el ángel divino dijo a Asenat: «Quita esta mesa». Y cuando Asenat se volvió para quitar la mesa, inmediatamente se apartó de sus ojos, y Asenat vio como un carro con cuatro caballos que iba hacia el oriente hacia el cielo, y el carro era como una llama de fuego, y los caballos como relámpagos. , y el ángel estaba de pie encima de ese carro.

\par 7 Entonces Asenat dijo: «¡Tonto y tonto soy yo, el humilde, porque he hablado como si un hombre hubiera entrado en mi cámara desde el cielo! No sabía que Dios había entrado en ello; y he aquí! ahora regresa al cielo a su lugar». Y ella dijo para sí: «Sé misericordioso, Señor, con tu esclava, y perdona a tu sierva, porque por mi parte he hablado temerariamente delante de ti».

\chapter{18}

\par \textit{El Rostro de Asenath se transforma.}

\par 1 Y mientras Asenat aún estaba hablando estas palabras para sí misma, ¡he aquí! un joven, uno de los siervos de José, diciendo: «José, el valiente de Dios, viene a vosotros hoy».

\par 2 E inmediatamente Asenat llamó al mayordomo de su casa y le dijo: «Apresúrate, prepara mi casa y prepara una buena comida, porque José, el poderoso hombre de Dios, viene hoy a nosotros».

\par 3 Y el mayordomo de la casa, cuando la vio (porque su rostro se había encogido por los siete días de aflicción, llanto y abstinencia), se entristeció y lloró; y él tomó su mano derecha y la besó tiernamente y dijo: «¿Qué te pasa, señora mía, que tienes la cara así encogida?» Y ella dijo: «Tuve un gran dolor en mi cabeza, y el sueño se apartó de mis ojos». Entonces el mayordomo de la casa se fue y preparó la casa y la comida.

\par 4 Y Asenat se acordó de las palabras del ángel y de sus mandatos, y se apresuró y entró en su segunda cámara, donde estaban los cofres de sus adornos, y abrió su gran cofre y sacó su primer manto como un rayo para mirarlo y ponérselo; y se ciñó también con un cinto resplandeciente y real, de oro y piedras preciosas,

\par 5 y se puso brazaletes de oro en las manos, y borceguíes de oro en los pies, y en el cuello un adorno precioso, y se puso una corona de oro en la cabeza; y en la corona como en su frente había una gran piedra de zafiro, y alrededor de la gran piedra seis piedras de gran precio, y con un manto muy maravilloso cubrió su cabeza. Y cuando Asenat se acordó de las palabras del mayordomo de su casa, que le había dicho que su rostro se había encogido, ella se entristeció mucho, y gimió y dijo: «¡Ay de mí, la humilde, porque mi rostro está encogido! José me verá así y seré despreciado por él».

\par 6 Y ella dijo a su sierva: Tráeme agua pura de la fuente. Y cuando lo trajo, lo derramó en la palangana e inclinándose para lavarse la cara, vio su rostro brillando como el sol, y sus ojos como el lucero de la mañana cuando sale, y sus mejillas.como estrella del cielo, y sus labios como rosas rojas, los cabellos de su cabeza eran como la vid que florece entre sus frutos en el paraíso de Dios, su cuello como un ciprés abigarrado. Y Asenat, cuando vio estas cosas, se maravilló de lo que veía y se regocijó con gran alegría y no se lavó la cara, porque dijo: «Para que no me borre esta belleza tan grande y hermosa».

\par 7 Entonces el mayordomo de su casa volvió y le dijo: «Se ha hecho todo lo que tú ordenaste»; y cuando la vio, tuvo mucho miedo y estuvo mucho tiempo temblando, y cayó a sus pies y comenzó a decir: «¿Qué es esto, señora mía? ¿Cuál es esta belleza grande y maravillosa que te rodea? ¿Te ha elegido el Señor Dios del cielo como esposa para su hijo José?

\chapter{19}

\par \textit{José regresa y es recibido por Asenat.}

\par 1 Y mientras aún hablaban estas cosas, vino un niño y le dijo a Asenat: «¡Mira! José está ante las puertas de nuestro patio». Entonces Asenat se apresuró y bajó las escaleras de su desván con las siete vírgenes para encontrarse con José y se paró en el pórtico de su casa.

\par 2 Y cuando José entró en el atrio, se cerraron las puertas y todos los extranjeros se quedaron fuera. Y Asenat salió del pórtico para encontrarse con José, y al verla se maravilló de su belleza, y le dijo: «¿Quién eres tú, doncella? Dímelo rápido».

\par 3 Y ella le dijo: «Yo, Señor, soy tu sierva Asenat; Todos los ídolos los he desechado y perecieron. Y hoy vino a mí un hombre del cielo y me dio pan de vida y comí y bebí una copa bendita, y me dijo: Te he dado por esposa a José, y él mismo será tu esposo para siempre; y tu nombre no se llamará Asenat, sino que se llamará 'Ciudad de Refugio', y el Señor Dios reinará sobre muchas naciones, y en ti buscarán refugio en el Dios Altísimo». Y el hombre dijo: «Iré también a José para hablarle al oído estas palabras acerca de ti. Y ahora sabes, señor, si aquel hombre ha venido a ti y si te ha hablado de mí.

\par 4 Entonces José dijo a Asenat: «Bendita tú, mujer, del Dios Altísimo, y bendito sea tu nombre para siempre, porque el Señor Dios ha puesto los cimientos de tus muros, y los hijos del Dios viviente serán Habita en tu ciudad de refugio, y el Señor Dios reinará sobre ellos para siempre. Porque aquel hombre vino hoy a mí del cielo y me dijo estas palabras acerca de ti. Y ahora ven acá a mí, virgen y pura, ¿y por qué estás tan lejos?

\par 5 Entonces José extendió sus manos y abrazó a Asenat y a Asenat José, y se besaron durante mucho tiempo, y ambos vivieron nuevamente en su espíritu. Y José besó a Asenat y le dio el espíritu de vida, luego la segunda vez le dio el espíritu de sabiduría, y la tercera vez la besó tiernamente y le dio el espíritu de verdad.

\chapter{20}

\par \textit{Pentefres regresa y desea desposar a Asenat con José, pero José decide pedirle la mano a Faraón.}

\par 1 Y cuando estuvieron abrazados durante mucho tiempo y entrelazaron las cadenas de sus manos, Asenat dijo a José: «Ven acá, señor, y entra en nuestra casa, porque yo he preparado nuestra casa por mi parte. y una gran cena.»

\par 2 Ella lo tomó de la mano derecha, lo llevó a su casa y lo sentó en la silla de su padre Pentefres. y ella trajo agua para lavarle los pies. Y José dijo: «Que venga una de las vírgenes y me lave los pies».

\par 3 Y Asenat le dijo: «No, señor, porque de ahora en adelante tú eres mi señor y yo tu esclava. ¿Y por qué buscas esto, que otra virgen te lave los pies? porque tus pies son mis pies, y tus manos mis manos, y tu alma mi alma, y ​​otro no lavará tus pies». Y ella lo contuvo y le lavó el fect.

\par 4 Entonces José la tomó de la mano derecha y la besó tiernamente, y Asenat le besó tiernamente la cabeza y entonces la sentó a su derecha.

\par 5 Su padre y su madre y todos sus parientes salieron entonces de la posesión de su herencia y la vieron sentada con José y vestida de boda. Y se maravillaron de su belleza y se regocijaron y glorificaron a Dios que da vida a los muertos. Y después de estas cosas comieron y bebieron;

\par 6 Pentefres, todos alegres, dijo a José: Mañana llamaré a todos los príncipes y sátrapas de toda la tierra de Egipto, y te celebraré una boda, y tomarás por esposa a mi hija Asenat. »

\par 7 Pero José dijo: «Mañana iré al rey Faraón, porque él mismo es mi padre y me nombró gobernante sobre toda esta tierra, y le hablaré acerca de Asenat, y él me la dará por esposa.» Y Pentefres le dijo: «Ve en paz».

\chapter{21}

\par \textit{Las bodas de José y Asenat.}

\par 1 Y José se quedó aquel día con Pentefres, y no entró en Asenat, porque solía decir: «No es conveniente que un hombre que adora a Dios se acueste con su mujer antes de casarse». Y José se levantó temprano y fue a Faraón y le dijo: «Dame a Asenat, hija de Pentefres, sacerdote de Heliópolis, por esposa». Y Faraón se regocijó con gran alegría, y dijo a José: «¡He aquí! ¿No está éste desposado contigo por esposa desde la eternidad? Por tanto, déjala ser tu esposa desde ahora y por los siglos de los siglos».

\par 2 Entonces Faraón envió y llamó a Pentefres, y Pentefres trajo a Asenat y la presentó ante Faraón;

\par 3 y Faraón, cuando la vio, se maravilló de su belleza y dijo: «El Señor Dios de José te bendecirá, niña, y esta tu belleza permanecerá para la eternidad, porque el Señor Dios de José te eligió como esposa para él. : porque José es como el hijo del Altísimo, y tú serás llamada «su esposa desde ahora y para siempre».

\par 4 Después de esto, Faraón tomó a José y a Asenat y puso sobre sus cabezas coronas de oro que estaban en su casa desde tiempos antiguos y antiguos; y Faraón puso a Asenat a la derecha de José. Y Faraón puso sus manos sobre sus cabezas y dijo:

\par 5 «El Señor Dios Altísimo os bendecirá y os multiplicará, engrandecerá y glorificará por los siglos de los siglos».

\par 6 Entonces Faraón los hizo volver uno frente al otro, los acercó boca a boca y se besaron. Y Faraón hizo una boda para José y una gran cena y mucha bebida durante siete días, y convocó a todos los gobernantes de Egipto y a todos los reyes de las naciones,

\par 7 habiendo hecho proclamar en la tierra de Egipto, diciendo: Todo aquel que trabaje durante los siete días de las bodas de José y Asenat, ciertamente morirá.

\par 8 Y mientras se celebraban las bodas y terminada la cena, José llegó a Asenat, y Asenat concibió de José y dio a luz a Manasés y a Efraín su hermano en casa de José.

\chapter{22}

\par \textit{Asenat es presentada a Jacob.}


\par 1 Y cuando pasaron los siete años de abundancia, comenzaron a llegar los siete años de hambre.

\par 2 Y cuando Jacob oyó hablar de José su hijo, vino a Egipto con todos sus parientes en el segundo año del hambre, en el mes segundo, el día veintiuno del mes, y se estableció en Gesem.

\par 3 Y Asenat dijo a José: «Iré a ver a tu padre, porque tu padre Israel es como mi padre y Dios». Y José le dijo: «Irás conmigo y verás a mi padre».

\par 4 Y José y Asenat llegaron a Jacob en la tierra de Gesem, y los hermanos de José los recibieron y se postraron ante ellos sobre sus rostros en la tierra. Entonces ambos se llegaron a Jacob; y Jacob estaba sentado en su cama, y ​​él mismo era un anciano en plena vejez. Y cuando Asenat lo vio, se maravilló de su hermosura, porque Jacob era sumamente hermoso a la vista, y su vejez como la juventud de un hombre hermoso, y toda su cabeza era blanca como la nieve, y todos los cabellos de su cabeza eran apretado y muy tupido, y su barba blanca que le llegaba hasta el pecho, sus ojos alegres y brillantes, sus tendones y sus hombros y sus brazos como de ángel, sus muslos y sus pantorrillas y sus pies como de gigante. Entonces Asenat, cuando lo vio así, se maravilló y cayó e hizo reverencia sobre su rostro en tierra. Y Jacob dijo a José: «¿Es ésta mi nuera, tu esposa? Bendita será ella del Dios Altísimo».

\par 5 Entonces Jacob llamó a Asenat, la bendijo y la besó tiernamente; y Asenat extendió sus manos y tomó el cuello de Jacob y se colgó de su cuello y lo besó tiernamente.

\par 6 Y después de esto comieron y bebieron.

\par 7 Entonces José y Asenat fueron a su casa; y Simeón y Leví, los hijos de Lea, solos los condujeron fuera, pero los hijos de Balla y Zelfa, las siervas de Lea y Raquel, no se unieron para conducirlos, porque los envidiaban y detestaban. Y Leví estaba a la derecha de Asenat y Simeón a su izquierda.

\par 8 Y Asenat tomó a Leví de la mano, porque lo amaba más que a todos los hermanos de José, como a un profeta, adorador de Dios y temeroso del Señor. Porque él era hombre comprensivo y profeta del Altísimo, y él mismo veía cartas escritas en el cielo y las leía y se las revelaba a Asenat en secreto; por eso el mismo Leví también amó mucho a Asenat y vio el lugar de su descanso en las alturas.

\chapter{23}

\par \textit{El hijo del faraón intenta inducir a Simeón y Leví a matar a José.}


\par 1 Y aconteció que al pasar José y Asenat, cuando se dirigían a Jacob, el hijo primogénito de Faraón los vio desde el muro, y:

\par 2 Cuando vio a Asenat, se enojó con ella a causa de su extraordinaria belleza. Entonces el hijo de Faraón envió mensajeros y llamó a Simeón y a Leví; y cuando llegaron y se presentaron ante él,

\par 3 El hijo primogénito de Faraón les dijo: «Yo sé que hoy sois hombres más poderosos que todos los hombres de la tierra, y que con estas manos derechas vuestras fue derribada la ciudad de los sicimitas, y con vuestras dos manos espadas, 30.000 guerreros fueron asesinados.

\par 4 Y hoy os tomaré conmigo como compañeros y os daré mucho oro y plata y sirvientes y sirvientas y casas y grandes heredades, y vosotros contendéis de mi parte y hacedme bondad; por eso recibí gran desprecio de tu hermano José, ya que él mismo tomó a Asenat por esposa, y esta mujer estaba desposada conmigo desde antiguo.

\par 5 Y ahora venid conmigo, y pelearé contra José para matarlo con mi espada, y tomaré a Asenat por esposa, y vosotros seréis para mí como hermanos y amigos fieles.

\par 6 Pero si no escucháis mis palabras, os mataré con mi espada. Y dicho esto, desenvainó su espada y se la mostró.

\par 7 Y Simeón era un hombre valiente y atrevido, y pensó en poner su mano derecha sobre la empuñadura de su espada y sacarla de su vaina e herir al hijo de Faraón por haberles hablado palabras duras.

\par 8 Entonces Leví vio el pensamiento de su corazón, porque era profeta, y pisó con su pie derecho el pie derecho de Simeón, y lo apretó para que cesara su ira.

\par 9 Y Leví decía en voz baja a Simeón: «¿Por qué estás enojado contra este hombre? Somos hombres que adoramos a Dios y no nos conviene pagar mal por mal».

\par 10 Entonces Leví dijo abiertamente y con apacibilidad de corazón al hijo de Faraón: «¿Por qué habla nuestro señor estas palabras? Somos hombres que adoramos a Dios, y nuestro padre es amigo del Dios Altísimo, y nuestro hermano es como hijo de Dios. ¿Y cómo haremos esta maldad, pecando ante los ojos de nuestro Dios y de nuestro padre Israel, y ante los ojos de nuestro hermano José? Y ahora escucha mis palabras. No es propio que un hombre que adora a Dios dañe a otro de ninguna manera; y si alguno quiere hacer daño a un hombre que adora a Dios, ese hombre que adora a Dios no se venga de él, porque no tiene espada en sus manos.

\par 11 Y ten cuidado de no volver a decir estas palabras acerca de nuestro hermano José.

\par 12 Pero si continúas en tu mal consejo, ¡he aquí! nuestras espadas están desenvainadas contra ti».

\par 13 Entonces Simeón y Leví desenvainaron sus espadas y dijeron: «¿Ves ahora estas espadas? Con estas dos espadas castigó el Señor el desprecio de los sicimitas, con el que habían despreciado a los hijos de Israel por medio de nuestra hermana Dina, a quien Siquem hijo de Hamor había profanado.

\par 14 Y el hijo de Faraón, cuando vio las espadas desenvainadas, tuvo mucho miedo y tembló por todo su cuerpo, porque brillaban como llamas de fuego, y sus ojos se nublaron, y cayó de bruces en el suelo debajo de sus espadas. pies.

\par 15 Entonces Leví extendió su mano derecha y lo agarró, diciendo: «Levántate y no temas, solo ten cuidado de hablar más malas palabras contra nuestro hermano José».

\par 16 Entonces Simeón y Leví salieron de delante de él.

\chapter{24}

\par \textit{El Hijo del Faraón conspira con Dan y Gad para matar a José y apoderarse de Asenat.}


\par 1 El hijo de Faraón continuó lleno de temor y tristeza por tener miedo de los hermanos de José, y nuevamente se enojó mucho a causa de la belleza de Asenat, y se entristeció mucho.

\par 2 Entonces sus sirvientes le dicen al oído: «¡Mira! los hijos de Balla y los hijos de Zelfa, las siervas de Lea y Raquel, esposas de Jacob, están en gran enemistad contra José y Asenat y los odian; éstas serán para ti en todo según tu voluntad».

\par 3 Inmediatamente el hijo de Faraón envió mensajeros y los llamó, y ellos vinieron a él a primera hora de la noche, se presentaron ante él y él les dijo: «He aprendido de muchos que sois hombres valientes».

\par 4 Y Dan y Gad, los hermanos mayores, le dijeron: «Que mi señor hable ahora a sus siervos lo que quiera, para que tus siervos oigan y hagamos según tu voluntad».

\par 5 Entonces el hijo de Faraón se regocijó con gran alegría y dijo a sus sirvientes: «Apartaos ahora de mí por un corto tiempo, porque tengo que hablar en secreto con estos hombres». Y todos se retiraron.

\par 6 Entonces el hijo de Faraón mintió y les dijo: «¡He aquí! ahora la bendición y la muerte están ante vuestros rostros; Tomad, pues, la bendición antes que la muerte,

\par 7 porque sois hombres valientes y no moriréis como mujeres; pero sed valientes y vengaos de vuestros enemigos.

\par 8 Porque he oído a tu hermano José decir a mi padre Faraón: 'Dan, Gad, Neftalí y Aser no son mis hermanos, sino hijos de las siervas de mi padre.

\par 9 Espero, pues, la muerte de mi padre y los borraré de la tierra y de toda su descendencia, para que no hereden con nosotros, porque son hijos de esclavas.

\par 10 Porque también éstos me vendieron a los ismaelitas, y les pagaré de nuevo según el desprecio que cometieron contra mí; sólo mi padre morirá.'

\par 11 Y mi padre Faraón lo elogió por estas cosas y le dijo: «Has hablado bien, niño». Por tanto, toma de mí hombres valientes y procede contra ellos según lo que hicieron contra ti, y yo seré tu ayuda'».

\par 12 Y cuando Dan y Gad oyeron estas cosas del hijo de Faraón, se turbaron mucho y se entristecieron mucho, y le dijeron: Te rogamos, señor, que nos ayudes; Yo, de ahora en adelante, somos tus esclavos y siervos y moriremos contigo». Y el hijo de Faraón dijo: «Yo os seré una ayuda si también vosotros escucháis mis palabras». Y le dijeron: «Mándanos lo que quieras y haremos según tu voluntad».

\par 13 Y el hijo de Faraón les dijo: «Mataré a mi padre Faraón esta noche, porque ese Faraón es como el padre de José y le dijo que le ayudaría contra vosotros; y matéis a José, y yo tomaré a Asenat por mujer, y vosotros seréis mis hermanos y coherederos de todos mis bienes. Sólo haz esto».

\par 14 Dan y Gad le dijeron: «Hoy somos tus servidores y haremos todo lo que tú nos has ordenado. Y hemos oído a José decir a Asenat: 'Ve mañana a la posesión de nuestra herencia, que es la época de la vendimia'; y envió a la guerra contra ella seiscientos hombres poderosos y cincuenta precursores. Ahora pues, escúchanos y hablaremos a nuestro señor». Y le dijeron todas sus palabras secretas.

\par 15 Entonces el hijo de Faraón dio a los cuatro hermanos quinientos hombres a cada uno y les nombró sus jefes y líderes.

\par 16 Y Dan y Gad le dijeron: «Hoy somos tus servidores y haremos todas las cosas que tú nos has ordenado, y saldremos de noche y acecharemos en el barranco y nos esconderemos. en la espesura de los juncos;

\par 17 Si tomas cincuenta arqueros a caballo y avanzas un largo camino delante de nosotros, Asenat vendrá y caerá en nuestras manos, y mataremos a los hombres que están con ella.

\par 18 y ella misma huirá delante con su carro y caerá en tus manos y tú harás con ella lo que tu alma desee;

\par 19 Y después de esto mataremos también a José mientras está de luto por Asenat; Asimismo también mataremos a sus hijos delante de sus ojos».

\par 20 Entonces el hijo primogénito de Faraón, al oír estas cosas, se alegró mucho y los envió, y con ellos dos mil guerreros. Y cuando llegaron al barranco se escondieron en la espesura de los juncos, y se dividieron en cuatro compañías, y se apostaron al otro lado del barranco como en la parte delantera quinientos hombres de este lado del camino. y en aquel, y en el lado cercano del barranco igualmente quedaron los demás,

\par 21 Y ellos también se apostaron en la espesura de los juncos, quinientos hombres de un lado y del otro del camino; y entre ellos había un camino ancho y ancho.


\chapter{25}

\par \textit{El Hijo del Faraón va a matar a su Padre, pero no es admitido. Neftalí y Aser protestan ante Dan y Gad contra la Conspiración.}


\par 1 Entonces el hijo de Faraón se levantó esa misma noche y fue al dormitorio de su padre para matarlo a espada. Entonces los guardias de su padre le impidieron entrar donde su padre y le dijeron: «¿Qué mandas, señor?»

\par 2 Y el hijo de Faraón les dijo: «Quiero ver a mi padre, porque voy a recoger la vendimia de mi viña recién plantada».

\par 3 Y los guardias le dijeron: «Tu padre sufre dolores y estuvo despierto toda la noche y ahora descansa, y nos dijo que nadie debía entrar a él, ni siquiera si fuera mi hijo primogénito».

\par 4 Y él, al oír estas cosas, se fue enojado y luego tomó cincuenta arqueros a caballo en total y se fue delante de ellos, como Dan y Gad le habían dicho.

\par 5 Y los hermanos menores, Neftalí y Aser, hablaron a sus hermanos mayores, Dan y Gad, diciendo: «¿Por qué de nuevo hacéis maldad contra vuestro padre Israel y contra vuestro hermano José? Y Dios lo preserva como a la niña de sus ojos. ¡Mira!

\par 6 ¿No vendisteis ni una sola vez a José? y él es hoy rey ​​de toda la tierra de Egipto y dador de alimento.

\par 7 Ahora pues, si de nuevo queréis hacer maldad contra él, él invocará al Altísimo y enviará fuego del cielo y os devorará, y los ángeles de Dios lucharán contra vosotros.

\par 8 Entonces los hermanos mayores se enojaron contra ellos y dijeron: «¿Y moriremos como mujeres? Lejos de ello». Y salieron al encuentro de José y de Asenat.

\chapter{26}

\par \textit{Los Conspiradores matan a los Guardias de Asenath y ella huye.}


\par 1 Y Asenat se levantó por la mañana y dijo a José: «Voy a tomar posesión de nuestra herencia como has dicho; pero mi alma teme mucho que te apartes de mí».

\par 2 Y José le dijo: «Ten ánimo y no temas, sino más bien vete gozoso, sin tener miedo de nadie, porque el Señor está contigo y él mismo te preservará como a la manzana de un árbol. ojo de todo mal.

\par 3 Y saldré a dar comida y se la daré a todos los hombres de la ciudad, y nadie morirá de hambre en la tierra de Egipto.

\par 4 Entonces Asenat se fue y José se fue para darle de comer.

\par 5 Y cuando Asenat llegó al lugar del barranco con los seiscientos hombres, de repente los que estaban con el hijo de Faraón salieron de su emboscada y se enfrentaron a los que estaban con Asenat, y los mataron a todos con sus espadas. y mataron a todos sus antepasados,

\par 6 pero Asenat huyó con su carro. Entonces Leví, hijo de Lea, supo todas estas cosas por espíritu como profeta, y anunció a sus hermanos el peligro de Asenat.

\par 7 E inmediatamente cada uno de ellos tomó su espada sobre su muslo, sus escudos sobre sus brazos y las lanzas en sus manos derechas, y persiguieron a Asenat a gran velocidad.

\par 8 Y mientras Asenat huía antes, ¡he aquí! El hijo de Faraón salió a su encuentro y cincuenta jinetes con él; y Asenat, cuando lo vio, tuvo mucho miedo y tembló, e invocó el nombre de Jehová su Dios.

\chapter{27}

\par \textit{Los hombres que estaban con el hijo de Faraón y los que estaban con Dan y Gad son asesinados; y los cuatro Hermanos huyen al Barranco y sus Espadas son arrancadas de sus Manos.}


\par 1 Benjamín estaba sentado con ella en el carro, al lado derecho; y Benjamín era un muchacho fuerte de unos diecinueve años,

\par 2 y sobre él había una belleza inefable y un poder como el de un cachorro de león, y también era uno que temía a Dios en gran manera.

\par 3 Entonces Benjamín saltó del carro, tomó una piedra redonda del barranco, metió su mano y la arrojó contra el hijo de Faraón, le golpeó la sien izquierda y lo hirió con una herida grave, y cayó del caballo sobre el tierra medio muerta.

\par 4 Entonces Benjamín, corriendo hacia una roca, le dijo al hombre del carro de Asenat: «Dame piedras del barranco». Y le dio cincuenta piedras.

\par 5 Y Benjamín arrojó las piedras y mató a los cincuenta hombres que estaban con el hijo de Faraón, hundiéndose todas las piedras en sus sienes.

\par 6 Entonces los hijos de Lea, Rubén y Simeón, Leví y Judá, Isacar y Zabulón, persiguieron a los hombres que habían acechado a Asenat y cayeron sobre ellos sin darse cuenta y los mataron a todos. y los seis hombres mataron a dos mil setenta y seis hombres.

\par 7 Y los hijos de Balla y Selfa huyeron de ellos y dijeron: «Hemos perecido a manos de nuestros hermanos, y el hijo de Faraón también murió a manos del muchacho Benjamín, y todos los que estaban con él perecieron por la mano del niño Benjamín. Por tanto, venid, matemos a Asenat y a Benjamín y huyamos a la espesura de estos juncos».

\par 8 Y vinieron contra Asenat con las espadas desenvainadas y cubiertas de sangre. Y Asenath cuando los vio tuvo mucho miedo y dijo: «Señor Dios, que me vivificaste y me libraste de los ídolos y de la corrupción de la muerte, así como me dijiste que mi alma vivirá para siempre, líbrame ahora también de estos hombres malvados». Y el Señor Dios escuchó la voz de Asenat, y al instante las espadas de los adversarios cayeron de sus manos sobre la tierra y se convirtieron en cenizas.

\chapter{28}

\par \textit{Dan y Gad se salvan en la Súplica de Asenath.}


\par 1 Y los hijos de Balla y Zelfa, cuando vieron el extraño milagro que se había realizado, temieron y dijeron: «El Señor lucha contra nosotros en nombre de Asenat».

\par 2 Entonces se postraron sobre sus rostros en tierra, rindieron homenaje a Asenat y dijeron: «Ten piedad de nosotros, tus siervos, porque eres nuestra señora y reina. Cometimos malas acciones contra ti y contra nuestro hermano José, pero el Señor nos pagó según nuestras obras.

\par 3 Por eso, nosotros, tus siervos, te rogamos que tengas misericordia de nosotros, los humildes y miserables, y líbranos de las manos de nuestros hermanos, porque ellos se harán vengadores del agravio que te han hecho y sus espadas están contra nosotros. Por tanto, sé misericordiosa con tus siervos, señora, delante de ellos.

\par 4 Y Asenat les dijo: «Tened buen ánimo y no temáis a vuestros hermanos, porque ellos mismos son hombres que adoran a Dios y temen al Señor;

\par 5 pero id a la espesura de estos juncos hasta que yo los apacigue en vuestro nombre y calme su ira por los grandes crímenes que vosotros, por vuestra parte, os habéis atrevido a cometer contra ellos. Pero el Señor ve y juzga entre tú y yo».

\par 6 Entonces Dan y Gad huyeron a la espesura de los juncos; y sus hermanos, los hijos de Lea, vinieron corriendo como ciervos a gran prisa contra ellos.

\par 7 Y Asenat descendió del carro que la escondía y les tendió la mano derecha con lágrimas en los ojos,

\par 8 y ellos, postrándose, se postraron ante ella en el suelo y lloraron a gran voz;

\par 9 y seguían pidiendo a sus hermanos, los hijos de las siervas, que los mataran.

\par 10 Y Asenat les dijo: «Os ruego que perdonéis a vuestros hermanos y no les dejéis mal por mal. Porque el Señor me salvó de ellos y les arrancó de las manos los puñales y las espadas, ¡y he aquí! se derritieron y fueron reducidos a cenizas sobre la tierra como cera ante el fuego,

\par 11 y esto nos basta con que el Señor pelee por nosotros contra ellos. Por tanto, perdonad a vuestros hermanos, porque son vuestros hermanos y la sangre de vuestro padre Israel».

\par 12 Y Simeón le dijo: «¿Por qué nuestra señora habla bien de sus enemigos? No, más bien los cortaremos miembro a miembro con nuestras espadas,

\par 13 porque idearon cosas malas contra nuestro hermano José y nuestro padre Israel, y contra ti, nuestra señora, hoy.

\par 14 Entonces Asenat extendió su mano derecha, tocó la barba de Simeón, lo besó tiernamente y dijo: «Hermano, no des ninguna manera de pagar mal por mal a tu prójimo, porque así el Señor vengará tu afrenta. Ellos mismos, ya sabéis, son vuestros hermanos y la descendencia de vuestro padre Israel, y huyeron de lejos de delante de vosotros. Concédeles, pues, perdón».

\par 15 Entonces Leví se acercó a ella y le besó la mano derecha con ternura, porque sabía que ella quería salvar a los hombres de la ira de sus hermanos para que no los mataran. Y ellos mismos estaban cerca en la espesura del cañaveral; y sabiendo esto su hermano Leví, no lo declaró a sus hermanos, porque temía que en su ira, cortaran a sus hermanos.

\chapter{29}

\par \textit{Muere el hijo del faraón. Faraón también muere y José lo sucede.}

\par 1 Y el hijo de Faraón se levantó de la tierra, se sentó y escupió sangre de su boca; porque la sangre corría desde su sien hasta su boca.

\par 2 Entonces Benjamín corrió hacia él, tomó su espada y la sacó de la vaina del hijo de Faraón (porque Benjamín no llevaba espada en el muslo) y quiso herir al hijo de Faraón en el pecho.

\par 3 Entonces Leví corrió hacia él, lo tomó de la mano y le dijo: «De ninguna manera, hermano, hagas esto, porque somos hombres que adoramos a Dios, y no es propio que un hombre que adora a Dios devolver mal por mal, ni pisotear al que ha caído, ni aplastar por completo a su enemigo, incluso hasta la muerte. Y ahora vuelve a poner la espada en su lugar,

\par 4 y venid y ayúdame, y curémosle de esta herida; y si vive, será nuestro amigo y su padre Faraón será nuestro padre».

\par 5 Entonces Leví levantó de la tierra al hijo de Faraón, le lavó la sangre de la cara, le puso una venda en la herida, lo subió a su caballo y lo llevó ante su padre, Faraón.

\par 6 contándole todas las cosas que habían sucedido y acontecido.

\par 7 Y Faraón se levantó de su trono y se postró ante Leví en la tierra y lo bendijo.

\par 8 Pasado el tercer día, el hijo de Faraón murió a causa de la piedra con que Benjamín lo hirió.

\par 9 Y Faraón se lamentó mucho por su hijo primogénito,

\par 10 de donde Faraón enfermó y murió a los 109 años, y dejó su diadema al todo hermoso José.

\par 11 José reinó solo en Egipto cuarenta y ocho años; Y después de estas cosas, José devolvió la diadema al hijo menor de Faraón, que estaba de pecho cuando murió el anciano Faraón.

\par 12 Y desde entonces José fue padre del hijo menor de Faraón en Egipto hasta su muerte, glorificando y alabando a Dios.


\end{document}