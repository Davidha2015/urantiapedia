\begin{document}

\title{5 Macabeos}


\chapter{1}

\par \textit {El atentado de Heliodoro contra el tesoro. Ir fue ordenado por los reyes de los griegos}

\par 1 Los reyes de los gentiles griegos ordenaron que cada año se enviaran grandes sumas de dinero a la ciudad santa y se entregaran a los sacerdotes para que lo añadieran al tesoro de la casa de Dios. , como dinero para los receptores de limosnas [huérfanos] y para las viudas.

\par 2 Seleuco era rey en Macedonia y tenía un amigo, uno de sus capitanes, que se llamaba Heliodoro. Este hombre fue enviado a saquear el tesoro y a tomar todo el dinero que hubiera en él.

\par 3 Cuando esto se difundió en el extranjero, causó gran tristeza entre los ciudadanos; y temieron que Heliodoro llegara a mayores extremos;

\par 4 ya que no tenían poder suficiente para impedirle ejecutar sus órdenes.

\par 5 Por lo tanto, todos acudieron a Dios en busca de ayuda, y ordenaron un ayuno general y suplicaron con humildad, doblando las rodillas y grandes lamentos;

\par 6 se vistieron de cilicio y se revolcaron en cenizas, junto con el sumo sacerdote Onías y los demás príncipes, y los ancianos, incluso el pueblo, las mujeres y los niños.

\par 7 Y al día siguiente, Heliodoro entró en la casa de Dios con un séquito de seguidores; y entró en la casa con sus soldados de a pie, él mismo a caballo, y iba en busca del dinero.

\par 8 Pero el Dios grande y bueno envió sobre él una voz fuerte y terrible; y vio a una persona armada con armas de guerra, montada en un gran caballo, y avanzando contra él:

\par 9 Por lo tanto, sintió miedo y temblor; y aquel hombre se acercó a él, lo arrancó de la silla y lo arrojó violentamente al suelo.

\par 10 De modo que, aterrorizado y aterrorizado, quedó mudo.

\par 11 Pero cuando sus servidores vieron lo que le había sucedido y no pudieron reconocer a nadie que le hubiera hecho estas cosas, lo llevaron apresuradamente a su casa.

\par 12 y permaneció varios días sin hablar ni comer nada.

\par 13 Por lo tanto, los principales de sus amigos fueron al sacerdote Onías, suplicándole que se apaciguara con él y que implorara al Dios grande y bueno que no lo castigara.

\par 14 Lo que hizo Onías; y Heliodoro fue sanado de su enfermedad.

\par 15 Y vio en visión a la persona que había visto en el santuario, mandándole que fuera al sacerdote Onías, lo saludara y le rindiera los honores que correspondía; diciéndole que el Dios grande y bueno había oído sus oraciones y lo había sanado a petición de Onías.

\par 16 Entonces Heliodoro corrió hacia el sacerdote Onías, a quien postrándose saludó; ¡y le dio dinero de diversas clases!, pidiéndole que lo sumase a lo que había en el tesoro.

\par 17 Luego salió de Jerusalén al país de Macedonia y contó al rey Seleuco lo que le había sucedido; suplicando que no lo obligaría a convertirse en su representante en Jerusalén.

\par 18 Por eso el rey se maravilló de las cosas que Heliodoro le contaba; y le ordenó que los publicara al mundo.

\par 19 Y se preocupó de que sus hombres fueran expulsados ​​y enviados fuera de Jerusalén, aumentando los regalos que solía enviar allí anualmente, a causa de lo que había sucedido a Heliodoro.

\par 20 Y los reyes añadieron más dinero al dinero que habían ordenado dar a los sacerdotes, para gastarlo en los huérfanos y las viudas; también a lo que se iba a gastar en los sacrificios.

\chapter{2}

\par \textit{La historia de la traducción de los veinticuatro libros de la lengua hebrea a la lengua griega, por Ptolomeo rey de Egipto.}

\par 1 Había un hombre de Macedonia llamado Ptolomeo, dotado de conocimiento y entendimiento; a quien, mientras habitaba en Egipto, los egipcios hicieron rey sobre el país de Egipto.

\par 2 Por eso, poseído por el deseo de buscar diversos conocimientos, reunió todos los libros de los sabios de todas partes.

\par 3 Y, ansioso por obtener «los veinticuatro libros», escribió al sumo sacerdote de Jerusalén para que le enviara setenta ancianos de entre los más versados ​​en esos libros; y envió una carta al sacerdote con un presente.

\par 4 Cuando llegó la carta del rey al sacerdote, escogió a setenta hombres eruditos y los envió junto con un hombre llamado Eleazar, muy destacado en religión, ciencia y erudición, el cual partió a Egipto.

\par 5 Y cuando el rey supo que se acercaban, ordenó que se prepararan setenta alojamientos y que se hospedara a los hombres allí.

\par 6 También ordenó que se nombrara para cada uno un secretario que tomara nota de la interpretación de estos libros en caracteres y lengua griega.

\par 7 Prohibió asimismo que cualquiera de ellos mantuviera comunicación con cualquiera de sus compañeros; no sea que se pongan de acuerdo para hacer algún cambio en esos libros.

\par 8 Entonces los secretarios tomaron de cada uno de ellos la traducción de los «Veinticuatro Libros».

\par 9 Y cuando estuvieron terminadas las traducciones, Eleazar las llevó al rey; y los compararon en su presencia: en cuya comparación se encontró que estaban de acuerdo.

\par 10 Por lo cual el rey se alegró mucho y ordenó que se repartiera una gran suma de dinero entre el grupo. Pero Eleazar Hanase(fi) le recompensó con una generosa recompensa.

\par 11 También aquel día liberó a todos los cautivos que se encontraban en Egipto, de la tribu de Judá y de Benjamín, para que volvieran a su patria, Siria.

\par 12 El número de ellos era como ciento treinta mil.

\par 13 Además, ordenó que se repartiera dinero entre ellos, de modo que a cada uno le correspondieran varios denarios; quienes, al recibirlos, partieron a su propia tierra.

\par 14 Entonces mandó que se hiciera una gran mesa de oro purísimo, que fuera lo suficientemente grande como para contener una representación de toda la tierra de Egipto y una imagen del Nilo, desde el comienzo de su corriente hasta el final de la misma en Egipto, con sus diversas divisiones a través del país, y cómo cubre toda la tierra.

\par 15 También ordenó que se adornara la mesa con muchas piedras preciosas.

\par 16 Y se hizo esta mesa; y fue terminada su talla, y engastada de piedras preciosas, y fue llevada a la ciudad de Jerusalén, como presente a la magnífica casa.

\par 17 Y cuando llegó sano y salvo, fue colocado en la casa, según la orden del rey. Y verdaderamente los hombres nunca han sido considerados como tales, por la belleza de las imágenes y la excelencia de la mano de obra.

\chapter{3}

\par \textit{La historia de los judíos. Una relación de lo que les sucedió a los judíos bajo el rey Antíoco; y qué batallas hubo entre ellos y sus capitanes; y hasta dónde llegó finalmente.}

\par 1 Había un hombre de los reyes de Macedonia, que se llamaba Antíoco; entre cuyos hechos estaba éste:

\par 2 que cuando murió Ptolomeo, el mencionado rey de Egipto, fue con sus ejércitos a atacar al segundo Ptolomeo. ¿Y habiendo conquistado y matado a Ptolomeo, ganó su país? Egipto, y tomó posesión de él.

\par 3 Desde allí, a medida que sus asuntos cobraban fuerza, sometió gran parte de la tierra; el rey de Persia y otros le rindieron obediencia.

\par 4 Por eso se enalteció su corazón y, envanecido de orgullo, mandó que se hicieran imágenes a su semejanza; que los hombres los adoren, para su glorificación y honra.

\par 5 Y cuando estuvieron hechas, envió mensajeros a todas las regiones de su imperio, ordenando que fueran adoradas y adoradas. A estas órdenes las naciones asintieron, temiendo y temiendo su tiranía.

\par 6 Había entonces en Judea tres hombres, los peores de todos los mortales; y cada uno de ellos tenía, por así decirlo, una conexión en el mismo tipo de vicio. El nombre de uno de estos tres era Menelao; del segundo, Simeón; del tercero, Alcimo.

\par 7 Y en aquel tiempo aparecieron ciertas imágenes que los ciudadanos de Jerusalén contemplaron en el aire durante cuarenta días: eran figuras de hombres montados en caballos de fuego, peleando entre sí.

\par 8 Entonces aquellos hombres impíos fueron a ver a Antíoco para obtener de él alguna autoridad que les permitiera cometer con facilidad lo que quisieran, prostituyendo y saqueando los bienes de los hombres; y, en resumen, podría gobernar al resto y mantenerlos en sujeción. Y le dijeron:

\par 9 Oh rey, últimamente han aparecido en el aire sobre Jerusalén fieros jinetes, contendientes entre sí; y por eso los hebreos se regocijaron, diciendo: «Esto presagiaba la muerte del rey Antíoco».

\par 10 Creyendo estas palabras, el rey, lleno de ira, se dirigió a Jerusalén en el menor tiempo posible; y vino sobre la nación sin estar advertida de su llegada.

\par 11 Y sus hombres atacaron a los habitantes y los mataron a espada, causando una gran matanza; Hirieron también a muchos, y llevaron en cautiverio a una gran multitud.

\par 12 Pero algunos de los que escaparon huyeron a las montañas y a los bosques, donde permanecieron mucho tiempo alimentándose de hierbas.

\par 13 Después de esto, Antíoco decidió partir del país.

\par 14 Pero el mal que había hecho a la nación no le bastó: dejó como sustituto a un hombre llamado Félix, y le ordenó que obligara a los judíos a adorar su imagen y a comer carne de cerdo.

\par 15 Lo cual hizo Félix, mandando llamar al pueblo para que obedecieran al rey en lo que él le había ordenado.

\par 16 Pero ellos se negaron a hacer aquello para lo que fueron llamados; por lo que mató a una gran multitud de ellos; preservar a esos malvados orn y su familia, y elevar su dignidad.

\chapter{4}

\par \textit{La historia de la muerte del sacerdote Eleazar}

\par 1 Después fue apresado Eleazar, que había ido con los doctores a Ptolomeo», y era entonces un hombre muy anciano, de noventa años de edad; y fue puesto delante de Félix;

\par 2 El cual le dijo: Eleazar, en verdad eres un hombre sabio y prudente; y en verdad te amo desde hace muchos años, y por eso no debería desear «tu muerte»:

\par 3 obedeced, pues, al rey, adorad su imagen, comed sus sacrificios y partid sanos y salvos.

\par 4 A quien respondió Eleazar; «No voy a abandonar mi obediencia a Dios para obedecer al rey».

\par 5 Y acercándose Félix, le susurró: «Ten cuidado de enviar a alguien que te traiga carne de tus propias ofrendas, que pones sobre mi mesa:

\par 6 y come una parte de ello en presencia del pueblo, para que sepan que has obedecido al rey, y salvarás tu vida, sin ningún daño a tu religión.

\par 7 Eleazar le respondió: «No obedezco a Dios bajo ningún tipo de fraude, sino que soportaré esta violencia tuya. Porque siendo yo un anciano de noventa años, mis huesos ahora están debilitados y mi cuerpo se ha consumido.

\par 8 Si, pues, sufro con ánimo valiente esos tormentos, ante los cuales incluso los jóvenes más valientes retroceden con miedo; mi pueblo y los jóvenes de mi nación me imitarán valientemente, y dirán;

\par 9 «¿Cómo es posible que no podamos soportar los dolores que ha sufrido uno que es inferior a nosotros en fuerza y ​​menos sustancial en carne y huesos?»

\par 10 lo cual, en verdad, será mejor para mí que engañarlos con una fingida obediencia al rey.

\par 11 porque entonces dirán: «Si ese viejo decrépito, sabio y prudente como es, se aferra a la vida y vencido por el dolor de las cosas temporales, abdicando de su religión; En verdad, nos será lícito lo que a él le fue lícito, ya que es un hombre viejo y sabio, y a quien debemos seguir.»

\par 12 Por eso preferiría morir, dejándoles constancia en la religión y paciencia contra la tiranía; que vivir; después de haber debilitado su constancia en obedecer a su Señor y seguir sus mandamientos; para que a través de mí «sean felices y no infelices».

\par 13 Cuando Félix escuchó la determinación de Eleazar, se enfureció violentamente contra él y ordenó que lo torturaran de diversas maneras, de modo que entró en la más desesperada lucha mortal y dijo:

\par 14 «Tú, oh Dios, sabes que podría haberme librado de las dificultades en las que he caído si obedeciera a otro en lugar de a Ti.

\par 15 Pero esto no lo he hecho; pero he preferido obedecerte y he estimado como ligera toda la violencia que se me ofrecía, en aras de la constancia en la obediencia a Ti.

\par 16 Y ahora pienso poco en las cosas que me han sucedido según tu buena voluntad y las apoyo lo mejor que puedo.

\par 17 Por tanto, te ruego que aceptes esto de mí y me hagas morir antes de que mi resistencia se debilite.

\par 18 Y Dios escuchó sus oraciones, e inmediatamente murió.

\par 19 Pero dejó a su pueblo dedicado al culto de su Dios, y lo dotó de sólida fortaleza, perseverancia en la religión y paciencia para resistir las pruebas que les esperaban.

\chapter{5}

\par \textit{La historia de la muerte de los siete hermanos.}

\par 1 Después de esto, fueron apresados ​​siete hermanos y su madre; y fueron enviados al rey; porque aún no se había alejado mucho de Jerusalén.

\par 2 Y cuando los llevaron ante el rey, uno de ellos fue llevado ante su presencia; a quien ordenó renunciar a su religión:

\par 3 pero él se negó y le dijo: Si piensas enseñarnos la verdad por primera vez, no es así.

\par 4 Porque la verdad es la que hemos aprendido de nuestros padres y por la cual nos hemos obligado a abrazar únicamente el culto de Dios y a observar constantemente la ley; y de esto no nos apartaremos de ninguna manera».

\par 5 El rey Antieco, enojado por estas palabras, ordenó que trajeran una sartén de hierro y la pusieran al fuego.

\par 6 Entonces ordenó que le cortaran la lengua al joven, que le cortaran las manos y los pies, que le quitaran la piel de la cabeza y que la pusieran en la cacerola. Y así lo hicieron.

\par 7 Luego ordenó que trajeran un gran caldero de bronce y lo pusieran sobre el fuego, en el que arrojarían el resto de su cuerpo.

\par 8 Y cuando el hombre estaba a punto de morir, ordenó que le quitaran el fuego para poder seguir siendo torturado, con la intención de aterrorizar a su madre y a sus hermanos con estos actos.

\par 9 Pero, en realidad, con esto les dio más valor y fuerza para mantener su religión con constancia y soportar todos los tormentos que la tiranía podía infligirles.

\par 10 Cuando murió el primero, trajeron ante él al segundo, a quien algunos de los sirvientes le dijeron: «Obedece las órdenes que te dará el rey, para que no perezcas como pereció tu hermano».

\par 11 Pero él respondió: «No soy más débil en espíritu que mi hermano, ni detrás de él en mi fe. Presentad vuestro fuego y vuestra espada; y no menosprecies lo que le hiciste a mi hermano». Y le hicieron como le habían hecho a su hermano.

\par 12 Entonces llamó al rey y le dijo: «Escucha, oh monstruo de crueldad hacia los hombres, y sabe que no obtienes nada de nosotros excepto nuestros cuerpos; pero nuestras almas no las obtienes de ningún modo; y estos pronto irán a su Creador,

\par 13 a quienes Él restaurará en sus cuerpos, cuando resucitará a los muertos de su nación y a los muertos de su pueblo».

\par 14 Y el tercero fue sacado; quien haciendo señas con la mano dijo al rey; «¿Por qué nos asustas, oh enemigo?

\par 15 Sepan que esto nos es enviado del cielo, y que también lo padecemos dando gracias a Dios, y de Él esperamos nuestra recompensa.

\par 16 Y el rey y los que estaban a su lado admiraron el valor del joven, la firmeza de su mente y su bella palabra. Luego dio órdenes y fue asesinado.

\par 17 Y fue sacado el cuarto, que dijo: Por la religión de Dios ponemos en venta nuestras vidas y las alquilamos para exigirle el pago, el día en que no tendréis excusa en el juicio y no podré soportar vuestros tormentos».

\par 18 El rey ordenó y lo mataron.

\par 19 Y el quinto fue sacado, quien le dijo; «No pienses dentro de ti mismo que Dios nos ha desamparado a causa de las cosas que nos ha enviado.

\par 20 Pero en verdad su voluntad es mostrarnos honor y amor con estas cosas; y Él nos vengará de ti y de tu posteridad».

\par 21 Entonces el rey ordenó y lo mataron.

\par 22 Y salió el sexto, que decía: «Confieso ciertamente mis ofensas a Dios, pero creo que me serán perdonadas por su muerte.

\par 23 Pero ahora os habéis opuesto a Dios, matando a los que abrazan su religión; y ciertamente Él os pagará según vuestras obras y os desarraigará de su tierra. Y él dio órdenes sobre él, y fue asesinado.

\par 24 Y salió el séptimo, que era un niño.

\par 25 Entonces su madre se levantó, intrépida e impasible, y miró¢ los cadáveres de sus hijos:

\par 26 Y ella dijo: Hijos míos, no sé cómo concebí a cada uno de vosotros, cuando lo concibí. Tampoco tenía poder para darle aliento; o de sacarlo a la luz de este mundo; o de otorgarle coraje y comprensión:

\par 27 Pero, en verdad, el mismo Dios, grande y bueno, lo formó según su voluntad y le dio forma según su buena voluntad.

\par 28 y con su poder lo trajo al mundo; designándole un término de vida, buenas reglas y una dispensación de religión, según le plazca.

\par 29 Pero ahora habéis vendido a Dios vuestros cuerpos, que él mismo formó, y vuestras almas, que él creó, y habéis aceptado sus juicios que él ha decretado.

\par 30 Por tanto, sois felices con las cosas que habéis obtenido felizmente; y bienaventurados sois por las cosas en las que «habéis salido victoriosos».

\par 31 Antíoco, al verla levantarse, pensó que lo había hecho porque temía por su hijo; y él pensó completamente que ella estaba a punto de ordenarle obediencia al rey, para que no pereciera como habían perecido sus hermanos.

\par 32 Pero él, al oír sus palabras, se avergonzó y se sonrojó, y mandó que le trajeran al niño; para que pueda exhortarlo y persuadirlo a amar la vida y disuadirlo de la muerte:

\par 33 no sea que todos ellos parezcan oponerse a su autoridad y muchos otros sigan su ejemplo.

\par 34 Por lo tanto, cuando fue llevado ante él, lo exhortó con discursos, le prometió riquezas y le juró que lo nombraría virrey para sí.

\par 35 Pero como el muchacho no se conmovió en absoluto por sus palabras, ni les hizo caso; El rey se volvió hacia su madre y le dijo;

\par 36 «Feliz mujer, compadécete de este tu hijo, a quien has sobrevivido solo; y exhortarlo a cumplir mis órdenes y a escapar de los sufrimientos que han acontecido a sus hermanos».

\par 37 Y ella dijo: Traedlo acá, para que yo le exhorte con las palabras de Dios.

\par 38 Y cuando se lo trajeron, ella se apartó de la multitud, lo besó y se rió, burlándose de las cosas que le había dicho Antíoco:

\par 39 y luego le dijo; Hijo mío, ven ahora, sé obediente a mí, porque yo te he parido, te he amamantado, te he educado y te he enseñado la religión divina.

\par 40 «Mirad ahora al cielo, a la tierra, al agua y al fuego; y entender que el único Dios verdadero mismo creó estos; y formó al hombre de carne y sangre, que vive poco tiempo, y luego morirá.

\par 41 Por tanto, teme al Dios verdadero, que no muere, y obedece al Ser verdadero,

\par 42 que no cambia sus promesas, y no temáis a este simple gigante, y morid por la religión de Dios, como han muerto vuestros hermanos.

\par 43 Pues si pudieras ver, hijo mío, su honorable morada y la luz de su morada, y la gloria que han alcanzado, no soportarías no seguirlos.

\par 44 Y en verdad también espero que el Dios grande y bueno me prepare y te siga de cerca.

\par 45 Entonces dijo el niño; «Sabed que obedezco bien a Dios, y no obedeceré los mandamientos de Antíoco; por tanto, no tardes en dejarme seguir a mis hermanos; No me impidáis ir al lugar a donde ellos han ido».

\par 46 Entonces dijo al rey: «¡Ay de ti de parte de Dios! ¿Adónde huirás de Él? ¿Dónde buscarás refugio? ¿O a quién pedirás ayuda para que no se vengue de ti?

\par 47 En verdad, nos has hecho un favor cuando te propusiste hacernos el mal: hiciste el mal a tu propia alma y la destruiste, mientras pensabas hacerle el bien.

\par 48 Ahora estamos en camino hacia una vida a la que la muerte nunca seguirá; y habitará en una luz que la oscuridad nunca borrará.

\par 49 Pero vuestra morada será en las regiones infernales, con castigos exquisitos de Dios.

\par 50 Y confío en que la ira de Dios se apartará de su pueblo por lo que hemos sufrido por ellos.

\par 51 sino que Él os atormentará en este mundo y os llevará a una muerte miserable; y que después partiréis a los tormentos eternos».

\par 52 Antíoco se enojó al ver que el muchacho se oponía a su autoridad; por lo que mandó que lo torturaran aún más que a sus hermanos. Y así fue hecho, y murió.

\par 53 Pero su madre oró a Dios y le rogó que pudiera seguir a sus hijos; e inmediatamente ella murió.

\par 54 Entonces Antíoco se fue a su tierra, Macedonia, y escribió a Félix y a los demás gobernadores de Siria que mataran a todos los judíos, excepto a los que abrazaran su religión.

\par 55 Y sus siervos obedecieron su orden y mataron a una multitud de hombres.

\chapter{6}

\par \textit{La historia del sumo sacerdote Matatías, hijo de Jojanán, hijo del sacerdote Hesmai}

\par 1 Un hombre llamado Matatías, hijo de Jojanán, huyó a una de las montañas que estaban fortificadas. Y los hombres que estaban dispersos huyeron allí hacia él, y algunos se escondieron en lugares apartados.

\par 2 Pero después que Antíoco se hubo alejado del país, Matatías envió en secreto a su hijo Judas a las ciudades de Judá;

\par 3 para certificarles de su propia salud y la de su pueblo, y desear que todos los que estaban inspirados por el valor, la magnanimidad y el celo por la religión, por sus esposas y sus hijos, vinieran a él.

\par 4 Y salieron a él algunos de los altos cargos del pueblo que se habían quedado atrás; los cuales, cuando llegaron a él, les dijeron:

\par 5 «No nos queda más que la oración a Dios, la confianza en Él y la lucha con nuestros enemigos, si tal vez Dios nos ayude y nos dé la victoria sobre ellos».

\par 6 Y el pueblo aceptó la opinión de Matatías y obraron conforme a ella.

\par 7 Y le fue dicho a Félix; y marchó contra ellos con un gran ejército.

\par 8 Y, mientras iba de camino, le llegaron noticias de que unos mil judíos, entre hombres y mujeres, estaban reunidos y habitaban en una cueva para poder conservar su propia forma de adoración.

\par 9 Y se dirigió hacia ellos con parte de sus tropas, enviando a los comandantes de sus hombres con el resto del ejército contra Matatías.

\par 10 Félix exigió a los que estaban en la cueva que salieran a él y aceptaran entrar en su religión; pero ellos se negaron.

\par 11 Entonces él los amenazó con echarles humo debajo; y ellos soportaron eso, y no salieron a él; y puso humo debajo de ellos, y todos murieron.

\par 12 Y cuando los generales de su ejército marchaban contra Matatías y llegaron hasta él, estando él preparado para la batalla;

\par 13 Uno de los generales, de sangre noble, se acercó a él y le propuso obedecer al rey y no oponerse a su autoridad; para que él y los que con él estaban vivieran, y no perecieran.

\par 14 A quien dijo; «Yo a la verdad obedezco a Dios, el verdadero rey; pero tú obedece a tu rey y haz lo que bien te parezca». Y dejó de hablar.

\par 15 Y comenzaron a tenderle trampas.

\par 16 Y vino uno de los peores judíos que estaban con ellos y los incitó a marchar contra él y preparar la guerra.

\par 17 Entonces Matatías se abalanzó sobre él con su espada desenvainada y le cortó la cabeza; luego hirió al general con quien el judío estaba hablando y lo mató también a él.

\par 18 Pero los compañeros de Matatías, al ver lo que había hecho, corrieron hacia él; e irrumpieron en el campamento del enemigo, mataron a muchos de ellos y los hicieron huir; después persiguieron a los fugitivos, hasta matarlos a todos.

\par 19 Después de esto, Matatías tocó la trompeta y proclamó una expedición contra Félix. Y él y sus compañeros entraron en la tierra de Judá y tomaron posesión de muchas de sus ciudades.

\par 20 Y el Dios Altísimo les dio por su mano descanso de los generales de Antíoco, y volvieron a practicar su propia religión, y las bandas de sus enemigos se retiraron delante de ellos.

\chapter{7}

\par \textit{El relato de la muerte de Matatías, y los hechos de Judas su hijo después de él.}

\par 1 Entonces Matatías enfermó. Y cuando estaba a punto de morir, llamó a sus hijos, que eran cinco, y les dijo:

\par 2 «Sé con certeza que muchas y grandes guerras se encenderán en la tierra de Judá, por causa de aquellos asuntos por los cuales el grande y buen Dios nos ha incitado a hacer la guerra contra nuestros enemigos.

\par 3 Pero yo te mando que temas a Dios, que confíes en él y que seas celoso de la ley, del santuario y del pueblo;

\par 4 y preparaos para hacer la guerra contra sus enemigos; y no temáis la muerte, porque, sin duda, esto está decretado para todos los hombres.

\par 5 De modo que, si Dios te da la victoria, obtendrás inmediatamente lo que anhelabas; pero si caes, eso no será una pérdida para ti ante sus ojos.

\par 6 Y Matatías murió y fue sepultado; y sus hijos hicieron conforme a lo que él les había mandado. Y acordaron poner a su hermano Judas como líder.

\par 7 Su hermano Judas era el mejor en consejo y el más valiente en fuerza de todos.

\par 8 Y Félix envió contra ellos un ejército al mando de un hombre llamado Serón, a quien Judas hizo huir con su compañía, y mató a muchos.

\par 9 Y la fama de Judas se difundió y aumentó mucho en los oídos de los hombres, y todas las naciones que lo rodeaban le temían en gran manera.

\par 10 Y el rey Antíoco fue informado de lo que habían hecho Matatías y su hijo Judas.

\par 11 La noticia de esto llegó también al rey de los persas; de modo que jugó en falso con Antíoco, apartándose de su amistad, siguiendo el ejemplo de Judas.

\par 12 Ante lo cual Antíoco se inquietó mucho, llamó a uno de sus criados, llamado Lisias, hombre fuerte y valiente, y le dijo:

\par 13 Ahora he decidido ir a la tierra de Persia para hacer la guerra; y deseo dejar detrás de mí a mi hijo en mi lugar; y tomar conmigo la mitad de mi ejército, y dejar el resto con mi hijo:

\par 14 Y he aquí, te he dado el gobierno de mi hijo y el gobierno de los hombres que le dejo.

\par 15 Y en verdad vosotros sabéis lo que Matatías y Judas han hecho a mis amigos y a mis súbditos.

\par 16 Por tanto, envía uno para que dirija un ejército poderoso a la tierra de Judá; y ordenarle que ataque a espada la tierra de Judá, que los desarraigue, que derribe sus viviendas y que destruya todo rastro de ellos».

\par 17 Entonces Antíoco partió hacia el país de Persia.

\par 18 Pero Lisias preparó a tres generales valientes y valientes, hábiles en la guerra; de los cuales uno fue nombrado. Ptolomeo, un segundo Nicanor y el tercero Gorgias.

\par 19 Y con ellos envió cuarenta mil soldados escogidos y siete mil jinetes. También les encargó que trajeran consigo un ejército de sirios y filisteos; y les ordenó extirpar por completo a los judíos.

\par 20 Y partieron, llevando consigo una multitud de mercaderes, para venderles los cautivos que iban a capturar de entre los judíos.

\par 21 Pero la noticia de esto llegó a Judas, hijo de Matatías; y fue a la casa del Dios grande y bueno;

\par 22 y reunió a sus hombres y les ordenó ayuno, súplicas y oraciones al Dios grande y bueno; y les encargó que le suplicaran la victoria contra sus enemigos; qué cosa hicieron.

\par 23 Después de esto, Judas reunió a sus hombres y nombró un jefe sobre cada mil, y también sobre cada cien, y sobre cada cincuenta, y sobre cada diez.

\par 24 Entonces ordenó que se hiciera proclamar con trompeta en todo su ejército, que todo aquel que tuviera miedo y a quien Dios ordenara que fuera despedido del ejército, regresara a su casa.

\par 25 Y muchos regresaron; y quedaron con ellos siete mil hombres fuertes y valientes, diestros en las guerras y acostumbrados a ellas; Ninguno de ellos había huido jamás, y marcharon contra sus enemigos.

\par 26 Pero cuando se acercaron a ellos, Judas oró a su Señor, rogándole que apartara de él la malicia de su enemigo; y que Él lo ayudaría y lo haría victorioso.

\par 27 Entonces ordenó a los sacerdotes que tocaran las trompetas, lo cual hicieron; y todos sus hombres invocaron a Dios y se lanzaron contra el ejército de Nicanor.

\par 28 Y Dios les dio la victoria sobre ellos, y lo hicieron huir a él y a sus hombres, matando a nueve mil hombres, y el resto se dispersó.

\par 29 Entonces Judas y su compañía regresaron al campamento de Nicanor y lo saquearon; y saqueó muchas propiedades de los comerciantes y las envió para dividirlas entre los enfermos.

\par 30 Esta batalla tuvo lugar el sexto día de la semana; Por tanto, Judas y sus hombres permanecieron en el mismo lugar hasta que pasó el día de reposo.

\par 31 Luego marcharon contra Tolomeo y Gorgias, a quienes encontraron, los derrotaron y los derrotaron, matando a veinte mil de sus soldados.

\par 32 Y Ptolomeo y Gorgias huyeron; a quien persiguieron Judas y su compañía; sin embargo, no pudo alcanzarlos, porque se metieron en una ciudad de dos ídolos, y allí se fortificaron con el resto de su ejército.

\par 33 Y Judas atacó a Félix; y fue puesto en fuga delante de él. Y Judas lo persiguió. Los cuales, llegando a una casa que estaba cerca, entraron en ella y cerraron las puertas, porque era una casa fortificada.

\par 34 Entonces Judas ordenó y le prendió fuego; y la casa fue quemada, y Félix fue quemado en ella. Entonces Judas se vengó de él por Hleazar y los demás a quienes Feelix había matado.

\par 35 Después el pueblo volvió junto a los muertos y tomó sus despojos y sus armas; pero enviaron lo mejor de la presa a Tierra Santa.

\par 36 Pero Nicanor partió disfrazado y desconocido, volvió a Lisias y le contó todo lo que le había sucedido a él y a su compañía.

\chapter{8}

\par \textit{Relato del regreso de Antíoco, y de su entrada a la tierra de Judá, y de la enfermedad que le sobrevino y de la cual murió en el camino.}

\par 1 Pero Antíoco regresó huyendo del país de Persia, con su ejército desmantelado.

\par 2 Cuando supo lo que le había sucedido a su ejército que Lisias había enviado, y a todos sus hombres, salió con un gran ejército y marchó hacia la tierra de Judá.

\par 3 Cuando llegó a la mitad del camino, Dios hirió a sus tropas con armas poderosas:

\par 4 pero esto no pudo detenerlo de su viaje; pero él persistió en ello, pronunciando toda clase de insolencias contra Dios y diciendo que nadie podía desviarlo ni impedirle sus determinados propósitos.

\par 5 Por lo que el Dios grande y bueno también lo hirió con úlceras que atacaron todo su cuerpo; pero aún así él no desistió ni se abstuvo de su viaje;

\par 6 pero estaba más lleno de ira y inflamado por un deseo ardiente de obtener lo que había decidido y llevar a cabo su resolución.

\par 7 Ahora bien, había en su ejército muchísimos elefantes. Sucedió que uno de ellos se escapó y lanzó un bramido; ante lo cual los caballos que tiraban del lecho en el que yacía Antíoco, salieron corriendo y lo arrojaron fuera.

\par 8 Y como era gordo y corpulento, sus miembros quedaron magullados y algunas de sus articulaciones se dislocaron.

\par 9 Y el mal olor de sus úlceras, que ya desprendían un olor fétido, aumentó tanto que ni él mismo ni los que se acercaban a él podían soportarlo por más tiempo.

\par 10 Cuando cayó, sus sirvientes lo levantaron y lo llevaron sobre sus hombros; pero como el mal olor se hizo peor, lo arrojaron al suelo y se alejaron.

\par 11 Por lo tanto, al ver los males que lo rodeaban, creyó con certeza que todo aquel castigo le había venido del Dios grande y bueno; a causa de la injuria y la tiranía que había usado hacia los hebreos, y el injusto derramamiento de su sangre.

\par 12 Entonces, temeroso, se volvió a Dios y, confesando sus pecados, dijo: «Oh Dios, en verdad merezco lo que me has enviado; y en verdad eres justo en tus juicios;

\par 13 Tú humillas al exaltado y humillas al envanecido; pero tuyo es la grandeza, la magnificencia, la majestad y la proeza.

\par 14 En verdad, lo reconozco: he oprimido al pueblo, y he actuado y decretado tiránicamente contra él.

\par 15 Te ruego que perdones, oh Dios, este mi error; y borra mi pecado, y concédeme mi salud; y mi preocupación será llenar el tesoro de tu casa de oro y plata.

\par 16 y cubrirás el suelo de la casa de tu santuario con vestiduras de púrpura; y ser circuncidado; y proclamar por todo mi reino, que Tú sólo eres el Dios verdadero, sin ningún compañero, y que no hay Dios fuera de ti».

\par 17 Pero Dios no escuchó sus oraciones ni aceptó sus súplicas, sino que sus problemas aumentaron tanto sobre él que vació sus entrañas y sus úlceras aumentaron hasta tal punto que su carne se desprendió de su cuerpo.

\par 18 Luego murió y fue sepultado en su lugar. Y reinó en su lugar su hijo, que se llamaba Eupátor.

\chapter{9}

\par \textit{La historia de los ocho días de dedicación}

\par 1 Cuando Judas hizo huir a Tolomeo, a Nicanor y a Gorgias, y mató a sus hombres; él mismo y sus tropas regresaron al campo» de la santa casa.

\par 2 Y ordenó que se destruyeran todos los altares que Antíoco había ordenado construir.

\par 3 y quitó todos los ídolos que había en el santuario, y edificaron un altar nuevo, y ordenó que se ofrecieran sacrificios sobre él.

\par 4 También rogaron al Dios grande y bueno que hiciera surgir el fuego santo que permanecería sobre el altar.

\par 5 y salió fuego de algunas piedras del altar, y quemó la leña y los sacrificios; y desde allí el fuego continuó sobre el altar hasta el tercer traslado en cautiverio'.

\par 6 Y entonces celebraron la fiesta del nuevo altar durante ocho días, comenzando el día veinticinco del mes Casleu.

\par 7 Luego pusieron pan sobre la mesa de la casa de Dios y encendieron las lámparas del candelero.

\par 8 Y cada uno de estos ocho días se reunieron para orar y alabar, y además lo fijaron como ordenanza para cada año siguiente.

\chapter{10}

\par \textit{La historia de las batallas de Judas con Gorgias y Ptolomeo}

\par 1 Después de los días de la dedicación, Judas marchó al país de los Idumzan, a la montaña de Sara, porque allí se alojaba Gorgias.

\par 2 Y Gorgias salió contra él con un gran ejército, y hubo entre ellos duras batallas; y cayeron de los hombres de Gorgias veinte mil.

\par 3 Y Gorgias huyó a Ptolomeo a la tierra del oeste (porque Antíoco lo había nombrado gobernador de ese país y allí se alojaba), y le contó lo que le había sucedido.

\par 4 Entonces Tolomeo salió con un ejército de ciento veinte mil hombres de Macedonia y del este.

\par 5 Y continuó hasta llegar a la región de Giares, es decir, Galaad, y las partes adyacentes; y mató a muchos judíos.

\par 6 Entonces escribieron a Judas contándole lo que les había sucedido y rogándole que viniera a derrotar a Ptolomeo y lo expulsara de ellos.

\par 7 Y la carta de ellos le llegó al mismo tiempo que le llegó una carta de los habitantes de la montaña de Galilea. Yo también le informé cómo los macedonios; Los que estaban en Tiro y Sidón se habían unido contra ellos y los habían atacado, matando a varios.

\par 8 Cuando Judas hubo leído ambas cartas, reunió a sus hombres, les mostró el contenido de las cartas y designó ayuno y súplica.

\par 9 Después de esto, ordenó a su hermano Simeón que tomara consigo a tres mil hombres judíos y que marchara a toda velocidad hacia la montaña de Galilea, para derrotar a los macedonios que estaban allí.

\par 10 Y Simeón fue. Pero Judas se apresuró a encontrarse con Ptolomeo.

\par 11 Entonces Simeón atacó inesperadamente a los macedonios, mató a ocho mil hombres y dio tranquilidad a los galilzanos.

\par 12 Pero Judas siguió adelante hasta llegar con Gorgias y Ptolomeo; presionándolos y sitiándolos; y se encontraron los dos ejércitos, y hubo entre ellos batallas muy feroces.

\par 13 Porque Ptolomeo encabezaba un grupo de hombres numerosos, fuertes y valientes. Pero Judas iba acompañado de un grupo muy pequeño:

\par 14 Sin embargo, como el pueblo que estaba con él estaba formado por las tropas más valientes y fuertes, resistió firmemente, y la batalla entre ellos duró mucho y se hizo muy encarnizada.

\par 15 Entonces Judas invocó al Dios grande y bueno e invocó su ayuda.

\par 16 Y contó que había visto a cinco jóvenes jinetes, tres de los cuales luchaban contra el ejército de Ptolomeo, y dos estaban cerca de él.

\par 17 Los cuales, cuando los miró atentamente, le parecieron ángeles de Dios.

\par 18 Por lo cual se consoló su corazón y el de sus compañeros; y atacando con frecuencia a los enemigos, los hicieron huir y mataron a grandes multitudes.

\par 19 Y el número de los muertos del ejército de Tolomeo, desde el principio de esta batalla hasta el final, fue de veinte mil quinientos.

\par 20 Después de esto, Tolomeo y sus hombres huyeron a la costa del mar; Mientras Judas los perseguía, mataba a cuantos capturaba.

\par 21 Pero Tolomeo huyó a Gaza y permaneció allí; y los hombres de Chalisam vinieron a él.

\par 22 Y Judas marchó contra ellos; y cuando los encontró, los derrotó; y los hombres de Ptolomeo se dispersaron, pero él mismo huyó a Gaza, y allí se fortificó.

\par 23 Los hombres de Judas persiguieron al cuerpo que huía y mataron a muchos de ellos. Y Judas y los hombres que estaban con él fueron directamente a Gaza, y plantó su campamento y la sitió.

\par 24 Y los hombres de Judas regresaron a él; y los que quedaban de las fuerzas de Ptolomeo subieron a la fortificación y abusaron de Judas con muchos insultos.

\par 25 Y la lucha entre ellos y las tropas de Judas duró cinco días. Pero cuando llegó el quinto día, el pueblo continuó lanzando reproches a Judas y vilipendiando su religión:

\par 26 Entonces veinte de los hombres de Judas se enojaron; Los cuales, tomando escudos en la mano izquierda y espadas en la derecha, y teniendo consigo un hombre que llevaba una escalera que habían hecho, marcharon hasta llegar al muro.

\par 27 Y dieciocho de ellos se levantaron y lanzaron dardos a los que estaban en la muralla; y dos, apresurándose hacia la pared, subieron la escalera y subieron por ella.

\par 28 Pero algunos de los que estaban allí, al ver que habían subido y que sus compañeros los habían seguido, y que también habían bajado del muro a la ciudad, descendieron del muro tras ellos; a quienes derrotaron los hombres de Judas, matando gran número de sus enemigos.

\par 29 Pero el ejército de Judas avanzó hasta la puerta de la ciudad; y los veinte comenzaron a correr hacia la puerta para abrirla, pero fueron expulsados ​​de allí con gran violencia; Por eso dieron grandes gritos.

\par 30 Entonces Judas y sus hombres supieron que se habían acercado a la puerta, y la batalla se encarnizó tanto fuera como dentro de la puerta.

\par 31 Judas y sus hombres atacaron la puerta con fuego, y ésta cayó; y el pueblo pereció, y los hombres que habían injuriado a Judas fueron apresados, y él mandó sacarlos y quemarlos.

\par 32 Además, ordenó que la ciudad fuera completamente herida a espada; y la matanza continuó en él durante dos días, y luego fue consumida por el fuego.

\par 33 Pero Tolomeo huyó; ni se oyeron noticias de él en ese momento; porque que se había cambiado de ropa, y se había escondido en uno de los hoyos, y no se tenía cuenta de él.

\par 34 Pero sus dos hermanos fueron apresados ​​y llevados ante Judas; y ordenó que los decapitaran.

\par 35 Después de esto entró en la tierra del santuario, con mucho botín; y tanto él como su compañía ofrecieron en él oraciones, dando gracias a Dios por los beneficios que habían recibido.


\chapter{11}

\par \textit{La relación de la batile entre Judas y Lisias general de Eupátor, tras la muerte del rey Antíoco}

\par 1 El nombre de Antíoco, de quien ya hemos hablado anteriormente, era Epifanio; pero el nombre de su hijo que reinó después de él era Eupátor, que también se llamaba Antíoco.

\par 2 Y cuando tuvieron lugar las batallas de Judas con estos generales, ellos» escribieron sobre el tema a Eupator; quien envió con Lisias, el hijo de su primo, un gran ejército, en el que había ochenta mil jinetes y ochenta elefantes.

\par 3 Los cuales, llegando a una ciudad llamada Betner, acamparon alrededor de ella y la sitiaron, porque era una ciudad grande y había en ella mucha gente.

\par 4 Lisias levantó armas de guerra a su alrededor y comenzó a sitiar a sus habitantes.

\par 5 Dicho esto, Judas se dirigió con su compañía a unos montes fortificados;

\par 6 y allí se quedaron; no fuera que, si permanecían en alguna ciudad, viniera Lisias y la asediara y los dominara.

\par 7 Entonces Judas reunió a su compañía y resolvió marchar con ellos al campamento de Lisias, después de haber ido a la casa de Dios y ofrecer sacrificios en ella;

\par 8 suplicando al Dios grande y bueno que apartara de ellos la malicia de sus enemigos y les concediera la victoria sobre ellos: lo cual hicieron.

\par 9 Después de esto, partieron desde la región de la Casa Santa hacia Betner. Porque habían planeado atacar repentinamente al ejército y derrotarlo sin lucha.

\par 10 Ahora se dice que se le apareció a Judas un personaje entre el cielo y la tierra, montado en un caballo de fuego y sosteniendo en su mano una gran lanza, con la que derrotó al ejército de los gentiles.

\par 11 De modo que lo que habían visto les dio más valor y ánimo. Y se apresuraron y cargaron contra el ejército, y mataron a muchos de sus hombres.

\par 12 Por lo tanto, el ejército enemigo se vio perturbado y sumido en la mayor confusión, y todo él se lanzó a una huida confusa.

\par 13 Y la espada de Judas y su compañía los presionaba dolorosamente; y mató de ellos once mil hombres de a pie y mil seiscientos hombres de a caballo.

\par 14 También Lisias fue perseguido con su compañía a un lugar lejano, donde permaneció a salvo.

\par 15 Y envió a Judas, pidiéndole que se sometiera al rey y conservara su religión y la de su pueblo:

\par 16 a quien Judas consintió en este asunto, hasta que se escribiera palabra al rey y se recibiera respuesta de su conformidad.

\par 17 Y Judas escribió acerca de este asunto: Lisias también escribió al rey, informándole de lo sucedido y de las pruebas que había tenido de la fuerza y ​​valentía de la nación judía;

\par 18 y que una continuación de las guerras con ellos exterminaría a sus hombres, como habían sido exterminados los antes mencionados: le dijo también su acuerdo, y el suyo esperando hasta recibir una carta que le dijera lo que debía hacer.

\par 19 A lo cual el rey respondió que le parecía bien hacer la paz con la nación de los judíos, eliminando ese obstáculo en el ejercicio de su religión, porque precisamente esto los había incitado a las rebeliones y a los ataques dirigidos a sus predecesores.

\par 20 También le ordenó que firmara con ellos un tratado de paz y obediencia; para que no se les pongan obstáculos en materia de religión.

\par 21 También escribió lo siguiente a Judas y a todos los judíos que estaban en la tierra de Judá, y esta paz continuó entre ellos por algún tiempo.


\chapter{12}

\par \textit{Un relato del comienzo del poder de los romanos y de la ampliación de su imperio.}

\par 1 En el mismo tiempo del que hemos hablado, los asuntos de los romanos comenzaron a ser exaltados, para que el Dios grande y bueno cumpliera lo que el profeta Daniel (la paz sea con él) había predicho acerca del cuarto imperio.

\par 2 Había también en aquella época en África un rey muy generoso, cuyo nombre era Aníbal. Y la sede real de su imperio era Cartago. Decidió tomar posesión del reino de los romanos:

\par 3 por lo que se unieron para oponerse a él; y se multiplicaron las guerras entre ellos, de modo que pelearon dieciocho? batallas en el espacio de diez años; y no pudieron expulsarlo de su país, a causa de su innumerable ejército y pueblo.

\par 4 Por lo tanto, decidieron reunir una gran fuerza seleccionada entre sus tropas y ejércitos más valientes, y atacar a Aníbal en la guerra, y perseverar hasta que pudieran alejar sus fuerzas de ellos.

\par 5 Lo cual verdaderamente hicieron: y pusieron al frente de sus ejércitos a dos hombres muy ilustres; el nombre de uno era Aimilius, y el del otro Varro.

\par 6 Quien, al encontrarse con Aníbal, se comprometió con él; y fueron muertos de su ejército noventa mil hombres; y del ejército de Aníbal murieron cuarenta mil hombres.

\par 7 Pero Varrón huyó a una ciudad muy grande y fuerte llamada Venusia; Aníbal no lo persiguió; pero marchó a Roma para tomarla y quedarse allí.

\par 8 Así que permaneció allí ocho días y comenzó a construir casas frente a él;

\par 9 Lo cual, cuando los ciudadanos lo vieron, deliberaron sobre hacer paz y tratado con él y entregarle el país.

\par 10 Había entre ellos un joven llamado Escipión (porque en aquel tiempo los romanos no tenían rey, y toda la administración de sus asuntos estaba encomendada a trescientos veinte hombres, los cuales presidía una persona que era llamado mayor o anciano.)

\par 11 Entonces Escipión se acercó a ellos y los convenció de que no confiaran en Aníbal ni se sometieran a él. A quienes respondieron que no confiaban en él, pero que no podían resistirle.

\par 12 A quien dijo; El país de África está totalmente desprovisto de soldados, porque todos están aquí con Aníbal: dame, pues, una tropa de hombres escogidos, para que pueda ir a África.

\par 13 y haré tales hazañas en él, que cuando le llegue la noticia de ellas, tal vez él os abandone, y os liberéis de él y estéis en paz; y habiendo recuperado y fortalecido vuestros recursos, si Si debe prepararse para regresar, podrás oponerte a él.

\par 14 Y les pareció correcto el consejo de Escipión; y le entregaron treinta mil de sus hombres más valientes.

\par 15 Y se fue a África. Y salió al encuentro Asdrúbal, hermano de Aníbal, y peleó con él; a quien Escipión derrotó, le cortó la cabeza, la tomó con el resto de la presa y regresó a Roma.

\par 16 Y subiendo a la muralla, llamó a Aníbal y le dijo: ¿Cómo podrás prevalecer contra este nuestro país, si no puedes expulsarme de tu propia tierra a la que he ido? la destruiste, y mataste a tu hermano, y le cortaste la cabeza.

\par 17 Entonces le arrojó la cabeza. Lo cual, llevado a Aníbal y reconocido por él, se enfureció y se enojó contra el pueblo, y juró que no partiría hasta haber tomado Roma.

\par 18 Pero los ciudadanos, para apartarlo de ellos y tenerlo bajo control, decidieron enviar a Escipión de regreso para sitiar y atacar a Cartago.

\par 19 Y Escipión regresó con su ejército a África, y acamparon alrededor de Cartago y la sitiaron con un asedio muy activo.

\par 20 Entonces los habitantes escribieron a Aníbal, diciéndole: «Estás codiciando una tierra extranjera, que no sabes si podrás conquistar o no; pero ha llegado a tu propia tierra uno que intenta apoderarse de ella».

\par 21 Por lo tanto, si tardas en venir, le entregaremos el país, y le entregaremos a tu familia, todas tus riquezas y tus tesoros; para que nosotros y nuestra propiedad salgamos ilesos.

\par 22 Cuando le trajeron esta carta, salió de Roma; y se apresuró hasta llegar a África:

\par 23 Escipión avanzó y salió a su encuentro, y libró contra él tres veces encarnizada batalla, y murieron cincuenta mil de sus hombres.

\par 24 Pero Aníbal, puesto en fuga, se retiró a la tierra de Egipto; a quien Escipión persiguió, lo tomó prisionero y regresó a África.

\par 25 Y estando allí, Aníbal desdeñó ser visto por los africanos; por lo que tomó veneno y murió.

\par 26 Y Escipión conquistó el país de África y se apoderó de todos los bienes, servidores y tesoros de Aníbal.

\par 27 De este modo se engrandeció la fama de los romanos y desde entonces comenzó a aumentar su poder.

\chapter{13}

\par \textit{Un relato de la carta de los romanos a Judas, y del tratado que tuvo lugar entre ellos.}

\par 1 «Desde el anciano y trescientos veinte gobernadores, hasta Judas, general del ejército, y los judíos.

\par 2 La salud sea para ti. Ya hemos oído hablar de vuestras victorias, de vuestro valor y de vuestra resistencia en la guerra; de lo cual nos regocijamos. También hemos entendido que has llegado a un acuerdo con Antíoco.

\par 3 Os escribimos para que seáis amigos de nosotros y no de los griegos que os han hecho daño. Además, pretendemos ir a Antioquía y hacer la guerra a sus habitantes.

\par 4 Por tanto, apresúrate a decirnos con quién estás enemistado y con quién tienes amistad; para que podamos actuar en consecuencia».

\par 5 LA COPIA DEL TRATADO. «Este es el tratado que hicieron los ancianos y trescientos veinte gobernadores con Judas, general del ejército, y los judíos; que se unan a los romanos, y que los romanos y los judíos puedan ser de una sola opinión en las guerras y victorias para siempre.

\par 6 Ahora bien, si sobreviene la guerra contra los romanos, Judas y su pueblo los ayudarán, y no ayudarán a los enemigos de los romanos ni con provisiones ni con ningún tipo de armas.

\par 7 Y cuando llegue la guerra a los judíos, los romanos los ayudarán en la medida de sus fuerzas, y no prestarán ayuda alguna a sus enemigos.

\par 8 Y así como los judíos están ligados a los romanos, así también los romanos están ligados a los judíos, sin aumento ni disminución.

\par 9 Y Judas y su pueblo aceptaron esto; y el tratado entre ellos y los romanos se mantuvo y continuó durante mucho tiempo.

\chapter{14}

\par \textit{Un relato del batile que tuvo lugar entre Judas, Ptolomeo y Gorgias.}

\par 1 Después de esto, Tolomeo reunió ciento veinte mil hombres y mil jinetes, y fueron tras Judas. Y Judas salió a su encuentro con diez mil hombres y lo derrotaron, y muchos de los hombres de Ptolomeo fueron muertos.

\par 2 Y suplicó a Judas, y humildemente le rogó que le dejara escapar con vida; y juró que nunca más haría guerra contra él, y que mostraría bondad a los judíos que estaban en todos sus países.

\par 3 Entonces Judas tuvo compasión de él y lo dejó ir; y Ptolomeo cumplió su juramento.

\par 4 Pero Gorgias, reuniendo tres mil hombres del monte Sara (es decir, de Idumea) y cuatrocientos jinetes, salió al encuentro de Judas y mató al capitán de su ejército y a algunos de sus hombres.

\par 5 Entonces Judas y sus hombres avanzaron hacia ellos; y Gorgias fue puesto en fuga, y la mayor parte de su ejército murió o huyó; y lo buscaron, y no se supo nada de él; pero se dice que cayó en la batalla.

\chapter{15}

\par \textit{Relato de la disolución del tratado que Antíoco había hecho con Judas, y de su marcha (junto con Lisias, el hijo de su primo) con un gran ejército, y de sus guerras.}

\par 1 Pero cuando Antíoco Eupátor recibió la noticia de que Judas se había fortalecido y de las victorias que había obtenido, se enojó mucho;

\par 2 y rompió el pacto que había hecho con Judas y reunió un gran ejército, en el que había veintidós elefantes.

\par 3 y marchó con Lisias, el hijo de su primo, a la tierra de Judá, dirigiéndose a la ciudad de Betnerá, delante de la cual acampó y la sitió.

\par 4 Cuando Judas supo esto, se reunieron él y todos los ancianos de los hijos de Israel y oraron al Dios grande y bueno, ofreciendo muchos sacrificios;

\par 5 Terminado esto, Judas se fue con los jefes de su ejército y entró en el campamento de noche, y atacólo repentinamente, y mató de los enemigos a cuatro mil hombres y a uno de los elefantes; y volvió a su propio campamento hasta que comenzara a rayar el alba.

\par 6 Entonces cada ejército se separó y la batalla se encarnizó entre ellos.

\par 7 Y Judas vio uno de los elefantes con arreos de oro y supuso que el rey estaba sentado sobre él. Entonces llamó a sus hombres y les dijo: ¿Quién de vosotros saldrá y matará a este elefante?

\par 8 Y un joven, uno de sus siervos, que se llamaba Eleazar, salió y se abalanzó sobre las líneas enemigas, matando a derecha e izquierda, de modo que los hombres se desviaron fuera de su vista;

\par 9 y avanzó hasta llegar al elefante; y arrastrándose debajo de él, le abrió el vientre; y el elefante cayó sobre él y murió. Entonces el rey, al darse cuenta de esto, ordenó dar la orden de retirarse; y así fue.

\par 10 Y el número de hombres de alto rango que murieron ese día en la batalla fue de ochocientos hombres, sin contar los de la gente común que fueron asesinados y los que habían muerto durante la noche.

\par 11 Entonces le avisaron al rey que uno de sus amigos, llamado Filipo, se había rebelado contra él y que Demetrio, hijo de Seleuco, había salido de Roma con un gran ejército de romanos con la intención de apoderarse del reino. Su mano.

\par 12 Ante lo cual, muy asustado, envió un mensaje a Judas para que hiciera las paces entre ellos. Judas aceptó; y Antíoco y Lisias, el hijo de su primo, le juraron que nunca más le harían la guerra.

\par 13 Y el rey sacó una gran suma de dinero y se la dio a Judas como presente para la casa de Dios.

\par 14 El rey también ordenó que arrestaran a Menelao, uno de los tres malvados que habían causado el mal a los judíos en los días de su padre Antíoco; y ordenó que lo llevaran a una torre alta, y que desde allí lo arrojaran de cabeza; lo cual se hizo.

\par 15 Porque con esto el rey quería complacer a los judíos, ya que este hombre era uno de sus principales enemigos y había matado a muchos de ellos.

\chapter{16}

\par \textit{La historia de la llegada a Antioquía de Demetrio, hijo de Seleuco, y de su derrota contra Eupátor.}

\par 1 Después de esto, el rey Eupátor marchó hacia el país de Macedonia y luego regresó a Antioquía.

\par 2 A quien Demetrio atacó con un ejército de romanos, lo derrotó y mató, junto con Lisias, el hijo de su primo; y reinó en Antioquía.

\par 3 ¿Pero fue hacia él Alcimo, el líder de aquellos tres? hombres malvados; quien, llegando a su presencia, se postró ante él, lloró con mucha vehemencia y dijo;

\par 4 «Oh rey, Judas y su compañía nos han matado a muchos; porque, habiendo abandonado su religión, hemos abrazado la religión del rey. Por tanto, oh rey, ayúdanos contra ellos y vénganos de ellos».

\par 5 Entonces hizo ir a él a los judíos y lo enfureció; sugiriéndoles cosas que podrían provocar a Demetrio e irritarlo para que preparara un ejército para vencer a Judas.

\par 6 A quien el rey, atendiendo, envió un general llamado Nicanor, con un gran ejército y abundante armamento de guerra.

\par 7 Y cuando Nicanor llegó a Tierra Santa, envió mensajeros a Judas para que vinieran a él; y no reveló que había venido a conquistar la nación,

\par 8 pero afirmó que había venido sólo a causa de la paz que se había hecho entre él y la nación, y que ellos también estaban bajo obediencia a los romanos.

\par 9 Entonces Judas salió hacia él con algunos de sus hombres, que estaban dotados de fuerza y ​​coraje, y les ordenó que no se alejaran de él, para que Demetrio no le pusiera una trampa.

\par 10 Entonces, cuando se encontró con Demetrio, lo saludó; y disponiendo un asiento para cada uno, se sentaron y Demetrio conversó con él como quiso. Después cada uno de ellos entró en una tienda que las tropas le habían levantado.

\par 11 Nicanor y Judas partieron a la Ciudad Santa, y allí vivieron juntos, y nació entre ellos una gran amistad.

\par 12 Cuando Alcimo se enteró de esto, fue a ver a Demetrio y lo enfureció contra Judas, y lo persuadió para que escribiera y ordenara a Nicanor que le enviara a Judas encadenado.

\par 13 Pero la noticia de esto llegó a Judas, y salió de la ciudad de noche, se fue a Sebaste y envió a sus compañeros para que vinieran a él.

\par 14 Cuando llegaron, tocó la trompeta y les ordenó que se prepararan para atacar a Nicanor.

\par 15 Pero Nicanor buscó a Judas con gran diligencia, y no pudo saber nada de él.

\par 16 Entonces fue a la casa de Dios y pidió a los sacerdotes que se lo entregaran para enviarlo atado con cadenas al rey; pero ellos juraron que no había entrado en la casa de Dios.

\par 17 Entonces insultó a ellos y a la casa de Dios, habló insolentemente del templo y amenazó con demolerlo desde sus cimientos; y se fue furioso. También se encargó de registrar todas las casas de la Ciudad Santa.

\par 18 Asimismo envió a sus hombres a la casa de un hombre excelente que había sido apresado en tiempos de Antíoco y sometido a torturas extremas; pero después de la muerte de Antíoco los judíos aumentaron su autoridad y lo honraron mucho.

\par 19 Y cuando los mensajeros de Nicanor vinieron a él, temió recibir el mismo trato que había recibido de Antíoco; por lo que se impuso las manos.

\par 20 Cuando Judas se enteró de esto, se entristeció mucho y se entristeció mucho, y envió a decir a Nicanor: «No me busquéis en la ciudad, porque no estoy allí; venid, pues, a mí, y nos encontraremos, o en la llanura, o en la montaña, según vuestra elección».

\par 21 Y Nicanor salió hacia él, y Judas le salió al encuentro con estas palabras: «Oh Dios, fuiste tú quien exterminó el ejército del rey Senaquerib; y él en verdad era mayor que este hombre, en fama, en imperio, y en la multitud de su ejército:

\par 22 Y tú libraste de él a Ezequías, rey de Judá, cuando confió en ti y oró a ti. Líbranos, te ruego, oh Dios, de su malicia, y haznos victoriosos sobre él.

\par 23 Entonces se preparó para la batalla y avanzó hacia Nicanor, diciendo: «Cuídate, a ti vengo».

\par 24 Nicanor se volvió y huyó, pero Judas, que lo perseguía, le hirió en los hombros y se los partió; y sus hombres fueron puestos en fuga.

\par 25 Y aquel día cayeron de ellos treinta mil; y los habitantes de las ciudades salieron y los mataron, de modo que no quedó ninguno de ellos.

\par 26 Y decretaron que aquel día fuera cada año un día de acción de gracias al Dios grande y bueno, y un día de alegría, de fiesta y de bebida. [Hasta aquí está terminado el Segundo Libro de la traducción de los Hebreos.]

\chapter{17}

\par \textit{Un relato de la muerte de Judas}

\par 1 Pero cuando casi llegaba la misma estación del año, Báquides salió con treinta mil de los macedonios más valientes;

\par 2 y se encontró con Judas, sin que le llegara noticia alguna, cuando estaba en una ciudad llamada Lalis, con tres mil hombres.

\par 3 Por lo que la mayoría de los que estaban con él huyeron; y quedaron con él ochocientos hombres, y sus hermanos Simeón y Jonatán.

\par 4 Pero los que se quedaron con Judas eran los más fuertes y valientes, y los que ya habían sufrido mucho en las diversas batallas que había librado.

\par 5 Entonces Judas y su compañía salieron al encuentro de Báquides y su ejército.

\par 6 Y Báquides dividió su ejército, colocando quince mil a la derecha de Judas y su compañía, y quince mil a la izquierda.

\par 7 Entonces cada parte gritó contra Judas y su compañía. El cual, mirando atentamente a cada uno, vio que a la derecha estaban las tropas enemigas más fuertes y valientes, y descubrió que entre ellos estaba el propio Báquides.

\par 8 Judas también dividió su grupo, tomó consigo a los más valientes y dio el resto a sus hermanos. Luego atacó a los de la derecha y él y su compañía mataron a unos dos mil hombres.

\par 9 Entonces, al ver a Báquides, dirigió sus ojos y sus pasos hacia él y mató a todos los hombres más valientes que estaban a su alrededor.

\par 10 Y él mismo con su compañía sostuvo a la multitud que lo apretaba, derribando a la mayor parte de ellos, y llegó cerca de Báquides.

\par 11 Al cual, cuando Báquides vio venir hacia él como un león, blandiendo en su mano una gran espada manchada de sangre, tuvo mucho miedo de él, tembló y huyó fuera de su vista.

\par 12 Judas y su compañía lo persiguieron y mataron a espada a su pueblo, de modo que mataron a la mayor parte de aquellos quince mil; y Báquides huyó hasta Asdod.

\par 13 Y los quince mil que estaban a la izquierda de Judas lo siguieron y atacaron a Judas, a quien ya habían llegado sus hermanos y los que estaban con ellos, muy fatigados.

\par 14 Y aquellos quince mil se abalanzaron sobre ellos, y hubo una gran batalla entre ellos y Judas; y cayeron de ambos lados cierto número de muertos, entre los cuales estaba Judas.

\par 15 A quien sus hermanos llevaron y enterraron junto al sepulcro de su padre Matatías (Dios tenga misericordia de ellos); y los hijos de Israel lo lloraron muchos años.

\par 16 El tiempo de su gobierno fue de siete años, y su hermano Jonatán lo sucedió en el gobierno.

\chapter{18}

\par \textit{La historia de Jonatán hijo de Matatías}

\par 1 Jonatán sucedió a su hermano y se fue al Jordán con un pequeño número de hombres; Cuando Báquides se enteró, se dirigió hacia él con un gran ejército.

\par 2 Y cuando Jonatán lo vio, sus hombres cruzaron el Jordán a nado; y Báquides y su ejército los siguieron y los rodearon.

\par 3 Pero Jonatán se abalanzó sobre Báquides; Y cuando los hombres dieron paso a Jonatán, él y su compañía salieron de en medio de ellos y partieron hacia Beerseba.

\par 4 y Simeón su hermano se le unió y se quedaron allí; y repararon lo que de las fortificaciones se había caído, y se fortificaron allí.

\par 5 Pero Báquides marchó hacia ellos y los sitió. Jonatán, su hermano y los que estaban con ellos salieron hacia él de noche y mataron a muchos de su ejército y quemaron los arietes y las máquinas de guerra;

\par 6 y su ejército se dispersó, y Báquides huyó al desierto. Y Jonatán y Simeón, y los hombres que estaban con él, lo persiguieron y lo prendieron.

\par 7 El cual, cuando vio a Jonatán, supo que su muerte estaba cerca; por eso proclamó la paz con Jonatán, y juró que nunca más le haría la guerra, y además, que restauraría a todos los cautivos que había había tomado del ejército de Judas.

\par 8 Y Jonatán le dio la mano y se alejó de él; después de esto no hubo más guerra entre ellos. Y poco después murió Jonatán, y le sucedió su hermano Simeón.


\chapter{19}

\par \textit{La historia de Simeón hijo de Matatías}

\par 1 Entonces Simeón, hijo de Matatías, sucedió en el gobierno; y reunió a todos los que quedaban del ejército de Judas:

\par 2 y sus negocios prosperaron y sometió a todos los que se habían opuesto a los judíos después de la muerte de su hermano Judas; y se portó bien con su pueblo, y los asuntos de su país estaban correctamente ordenados.

\par 3 ¿Por qué Antioco? lo atacó, y también a Demetrio, hijo de Seleuco; y envió un gran ejército contra él:

\par 4 a su encuentro salieron Simeón y sus dos hijos; y dividió su ejército en dos partes, una de las cuales guardó para sí y la otra dio a sus hijos.

\par 5 Entonces él y los que estaban con él fueron al ejército; y envió a sus dos hijos y a sus seguidores por otro camino, y los designó para atacar al ejército en un momento determinado.

\par 6 Después de esto, se encontró con el ejército de Antíoco, lo atacó y comenzó a prevalecer contra él. Y sus dos hijos llegaron cuando la batalla ya había comenzado, y la lucha se volvió feroz, y rodearon la retaguardia del ejército.

\par 7 Y el ejército de Antíoco, colocado entre dos ejércitos, fue destrozado, y ninguno de ellos escapó; ni Antíoco volvió más para pelear con Simeón.

\par 8 Y la paz y la tranquilidad continuaron entre los judíos durante todos los días de Simeón. Y el tiempo de su gobierno fue de dos años.

\par 9 Entonces su yerno Tolomeo se abalanzó sobre él y lo mató en una fiesta en la que él estaba presente. Y se apoderó de su mujer y de sus dos hijos. Y el hijo de Simeón, que se llamaba Hircano, fue puesto en lugar de su padre.

\par [Aquí termina la historia tal como se presenta en los dos libros que normalmente se adjuntan a nuestras Biblias.]

\chapter{20}

\par \textit{La historia de Hircano hijo de Simeón}

\par 1 Simeón, cuando aún vivía, había nombrado capitán a su hijo Jojanán; y reuniendo muchas tropas, le envió a vencer a cierto hombre que había salido contra él y se llamaba Hircano.

\par 2 Ahora bien, era un hombre de gran fama, poderoso en fuerza y ​​de antiguo dominio.

\par 3 A quien Jonatán encontró y derrotó; por eso Simeón llamó a su hijo Jojanan Hircano; a causa de haber matado a Hircano y haber obtenido una victoria sobre él.

\par 4 Pero cuando este Hircano se enteró de que Tolomeo había matado a su padre, tuvo miedo de Tolomeo y huyó a Gaza, y Tolomeo lo persiguió con muchos seguidores.

\par 5 Pero los habitantes de Gaza ayudaron a Hircano, cerraron las puertas de su ciudad e impidieron a Ptolomeo llegar a Hircano.

\par 6 Y Tolomeo regresó y partió hacia Dagón, llevando consigo a la madre de Hircano y a sus dos hermanos. Castillo fuertemente fortificado. Ahora Dagón tenía en ese momento un castillo fuertemente fortificado.

\par 7 Pero Hircano fue a la Santa Casa, ofreció sacrificios y sucedió a su padre; reunió un gran ejército y fue a atacar a Tolomeo. Por lo tanto, Ptolomeo cerró las puertas de Dagón para él y su compañía, y se fortificó allí.

\par 8 Hircano lo sitió e hizo un ariete de hierro para derribar la muralla y abrirla; y la batalla entre ellos duró mucho tiempo.

\par 9 Hircano venció a Ptolomeo, se acercó al castillo y estuvo a punto de tomarlo.

\par 10 Tolomeo, al ver esto, ordenó que sacaran a la madre de Hircano y a sus dos hermanos a la muralla y los torturaran con crueldad; que se les hizo.

\par 11 Pero Hircano, al ver esto, se detuvo; y temiendo que los mataran, desistieron de luchar.

\par 12 A quien su madre llamó y dijo: «Hijo mío, no te dejes llevar por el amor y la piedad filial hacia mí y hacia tus hermanos, con preferencia a tu padre:

\par 13 Ni a causa de nuestro cautiverio os debilitéis en vuestro deseo de vengarlo; pero exige satisfacción por los derechos de tu padre y los míos, en la medida de tus posibilidades.

\par 14 Pero lo que teméis por nosotros de ese tirano, él necesariamente nos lo hará en cualquier caso; por lo tanto, continuad con vuestro asedio sin interrupción.

\par 15 Cuando Hircano escuchó las palabras de su madre, inició el asedio; por lo que Ptolomeo aumentó los tormentos de su madre y de sus hermanos; y juró que los arrojaría de cabeza desde el castillo cada vez que Hircano se acercara a la muralla.

\par 16 Por eso Hircano temió ser la causa de su muerte; y regresó a su campamento, continuando aún «el asedio de Ptolomeo.

\par 17 Y aconteció que estaba cerca la fiesta de las Tiendas; Por tanto, Hircano fue a la ciudad de la Santa Casa para estar presente en la fiesta, la solemnidad y los sacrificios.

\par 18 Y cuando Tolomeo supo que había partido hacia la Ciudad Santa y que allí estaba detenido, apresó a la madre de Hircano y a sus hermanos y los mató; y huyó a un lugar al que Hircano no podía llegar.



\chapter{21}

\par \textit{La historia de la subida de Antíoco a la ciudad de la Santa Casa, para luchar contra Hircano.}

\par 1 Cuando Antíoco supo que Simeón había muerto, reunió un ejército y marchó hasta llegar a la ciudad de la Santa Casa.

\par 2 y acampó alrededor de ella, y la sitió, con intención de tomarla por la fuerza, pero no pudo a causa de la altura y la fuerza de las murallas y de la multitud de guerreros que había en ella.

\par 3 Pero la voluntad de Dios le impidió conquistarla: porque se había dirigido al lado norte de la ciudad y había construido allí ciento treinta torres frente a la muralla;

\par 4 y habían hecho subir a ellos hombres para luchar contra los que intentaran subir a las murallas de la ciudad.

\par 5 También ordenó a unos hombres que excavaran la tierra en un lugar determinado, hasta llegar a los cimientos del muro; al ver que era de madera, la quemaron al fuego, y una gran parte del muro se derrumbó.

\par 6 Los hombres de Hircano se opusieron a ellos y les impidieron la entrada, manteniendo guardia sobre la parte arruinada;

\par 7 Hircano salió con la mayor parte de sus guerreros contra el ejército de Antíoco y los derrotó con gran matanza.

\par 8 Y Antíoco y sus hombres fueron derrotados; a quien Hircano con sus tropas persiguió hasta que los expulsaron de la ciudad.

\par 9 Luego, volviendo a las torres que Antíoco había construido, las destruyeron; y morará en la ciudad y alrededor de ella.

\par 10 Pero Antíoco acampó en un lugar que distaba unos dos estadios de la ciudad de la casa de Dios.

\par 11 Y al acercarse la fiesta de las Tiendas, Hircano le envió embajadores para pedirle una tregua hasta que pasara la solemnidad; que le concedió; y envió víctimas, y oro y plata», a la casa de Dios.

\par 12 Hircano ordenó a los sacerdotes que recibieran lo que Antíoco había enviado; y así lo hicieron.

\par 13 Cuando Hircano y los sacerdotes vieron la reverencia de Antíoco hacia el templo de Dios, le envió embajadores para tratar de lograr la paz.

\par 14 A lo cual Antíoco estuvo de acuerdo; Y él fue a Jerusalén; y al encontrarse Hircano con él, entraron juntos en la ciudad.

\par 15 Hirecano hizo un banquete para Antíoco y sus príncipes; y comieron y bebieron juntos, y él le hizo un presente de trescientos talentos de oro.

\par 16 y cada uno de ellos acordó con su compañero la paz y la ayuda, y Antíoco partió a su país.

\par 17 Pero se cuenta que Hircano abrió el tesoro que habían hecho algunos reyes de los hijos de David, y sacó de allí una gran suma de dinero, y dejó otra cantidad en relegándolo a su antiguo estado de secreto.

\par 18 Luego edificó y reparó la parte del muro que se había caído; y se ocupó cuidadosamente de la conveniencia y ventaja de su rebaño, y se comportó rectamente con ellos.

\par 19 Cuando Antíoco llegó a su país, decidió ir a luchar contra el rey de Persia, porque se había rebelado desde los tiempos del primer Antíoco.

\par 20 y envió embajadores a Hircano para que fuera a él; Hircano fue con él y partió hacia el país de Persia.

\par 21 Y un ejército de los persas salió a su encuentro y peleó con él; a quien Antíoco, haciendo huir, derrotó y pasó a espada.

\par 22 Luego se quedó en el lugar donde estaba y erigió un edificio maravilloso, para que sirviera de memoria de él en su país.

\par 23 Y después de algún tiempo salió al encuentro del rey de los persas; e Hircano se quedó atrás a causa del sábado que siguió inmediatamente a Pentecostés.

\par 24 Y el rey de Persia y Antíoco se encontraron; y tuvieron lugar entre ellos grandes batallas, en las que murieron Antíoco y muchos de su ejército.

\par 25 Cuando Hiricano recibió la noticia de esto, marchó hacia el país de Siria y en su viaje sitió Halepo.

\par 26 y los ciudadanos se rindieron a él y le pagaron tributo; y partiendo de ellos, volvió a la Ciudad Santa, y permaneció allí algunos días.

\par 27 Luego partió hacia el país de Samaria y peleó contra Neápolis; pero los ciudadanos le impidieron entrar.

\par 28 Y destruyó todos los edificios que tenían en el monte Jezabel y el templo; lo cual fue hecho doscientos años después que Sanbalat el Samaritano lo había construido. También mató a los sacerdotes que estaban en Sebaste.

\par 29 Y marchó hacia el país de Idumeza, es decir, las montañas de Sara, y ellos se rindieron a él: con quienes estipuló que debían circuncidarse y adoptar la religión de la Torá (o la ley mosaica).

\par 30 Ellos estuvieron de acuerdo con él, se circuncidaron y se hicieron judíos, y fueron confirmados en esta práctica hasta la destrucción de la segunda casa.

\par 31 Y Hireano | pasó a todas las naciones vecinas; y todos se sometieron a él, y al mismo tiempo hicieron un acuerdo de paz y obediencia.

\par 32 También envió embajadores a los romanos, hablándoles de la renovación de la alianza que había entre ellos.

\par 33 Cuando sus embajadores llegaron a los romanos, los honraron; y les nombró un asiento digno; y prestó atención a la embajada a causa de la cual habían venido; y despachó sus asuntos y respondió a su carta.

\chapter{22}

\textit{La copia de la carta de los romanos a Hircano}

\par 1 Del anciano y de sus trescientos veinte gobernadores, a Hircano, rey de Judá, salud.

\par 2 Acabamos de recibir tu carta, cuya lectura nos alegra; y hemos interrogado a vuestros embajadores sobre el estado de vuestros asuntos.

\par 3 También hemos reconocido su lugar de dignidad en la ciencia, la disciplina moral y las virtudes; y los honramos y los hicimos sentar en presencia de nuestro mayor:

\par 4 que ha sido cuidadoso en gestionar todos sus negocios, dando orden de que os sean devueltas todas las ciudades que Antíoco había tomado por la fuerza;

\par 5 y que todo obstáculo al ejercicio de vuestra religión sea eliminado; y que se anule todo lo que Antíoco había decretado contra vosotros.

\par 6 También ha ordenado que todas las ciudades que había tomado permanezcan fieles a vosotros; Asimismo ha dado órdenes por carta a todas sus provincias, para que vuestros embajadores sean tratados con respeto y honor.

\par 7 Además, os ha enviado con ellos un embajador llamado Cinzeus, con una carta; a quien también ha confiado una embajada para tratar con vosotros personalmente».

\par 8 Por eso, cuando esta epístola de los romanos llegó a Hircano, comenzó a ser llamado rey, siendo antes llamado sumo sacerdote, y así se unieron en él las dignidades real y sacerdotal.

\par 9 Y él fue el primero en ser llamado rey entre los jefes de los judíos en el tiempo de la segunda casa.


\chapter{23}

\par \textit{La historia de las guerras de Hircano con los samaritanos}

\par 1 Hircano marchó hacia Sebaste y sitió a los samaritanos que allí se encontraban durante mucho tiempo; hasta que los redujo a tal situación que se vieron obligados a alimentarse de todo tipo de cadáveres.

\par 2 Sin embargo, lo soportaron con paciencia, temiendo su espada y confiando en los macedonios y egipcios, cuya ayuda habían implorado.

\par 3 Mientras tanto, llega el gran ayuno, durante el cual Hircano debe estar presente en la Santa Casa para ofrecer sacrificios ese día.

\par 4 Por lo que sustituyó a sus dos hijos, Antígono y Aristóbulo, como comandantes del ejército; dejándoles órdenes de asediar a los samaritanos y reducirlos a extremidades.

\par 5 Asimismo ordenó al ejército que obedecieran a sus hijos y ejecutaran sus órdenes, y partió hacia la ciudad de la Santa Casa.

\par 6 Además, Antíoco el macedonio marchó en ayuda de los habitantes de Sebaste; y la noticia llegó a los dos hijos de Hircano;

\par 7 quien, habiendo sustituido a un general para dirigir el sitio de Sebaste, fue al encuentro de Antíoco; a quien encontraron y derrotaron, y regresaron a Sebaste.

\par 8 También Litra, hijo de la reina Cleopatra, salió de Egipto para ayudar a los samaritanos.

\par 9 Cuando Hircano se enteró de esto, salió a su encuentro, ya pasada la solemnidad. Cuando lo encontró, lo encontró con fiereza y mató a muchos de sus hombres.

\par 10 y Litra fue puesta en fuga; Los egipcios tampoco volvieron después de este regreso para ayudar a los samaritanos.

\par 11 Y el rey Hircanis regresó a Sebaste y la atacó con dureza, hasta que la tomó con la espada, mató a los que quedaban de sus ciudadanos, la destruyó por completo y derribó sus murallas.

\chapter{24}

\par \textit{La historia de Litra, hijo de Cleopatra, y de su marcha contra su madre en Egipto.}

\par 1 Litra, hijo de Clegpatra, fortalecido en bienes y en hombres, se rebeló contra su madre Cleopatra; los principales hombres del reino eran sus cómplices.

\par 2 Entonces Cleopatra envió a llamar a dos judíos, uno de los cuales se llamaba Chelcias y el otro Hananias, y los puso a la cabeza de los príncipes de Egipto que habían quedado de su lado, y los nombró a ambos generales del ejército egipcio.

\par 3 Ahora manejaban bien todos los asuntos con la gente común y dirigían los asuntos del imperio con sabiduría. Cleopatra los envió a luchar con Lythras;

\par 4 Los cuales, acercándose a él, hicieron la guerra y lo derrotaron, haciendo huir a sus hombres; y él huyó a Chipre, donde se quedó con unos pocos que se le unieron.

\chapter{25}

\par \textit{Un relato de las sectas judías en este momento.}

\par 1 En aquel tiempo había entre los judíos tres sectas. Uno, de los fariseos, es decir, los «separados» o religiosos;

\par 2 cuya norma era observar lo que estaba contenido en la ley, según las explicaciones de sus antepasados.

\par 3 El segundo, el de los saduceos; y estos son seguidores de cierto hombre de los doctores, de nombre Sadoc;

\par 4 cuya regla era mantener conforme a las cosas que se encuentran en el texto de la ley, y de lo cual está demostrado en la Escritura misma; pero no lo que no existe en el texto ni se prueba en él.

\par 5 La tercera secta era la de los Hasdanim, o los que estudiaban las virtudes: pero el autor de este libro no hizo mención de su gobierno, ni sabemos excepto en la medida en que se descubre por su nombre:

\par 6 porque se aplicaban a prácticas que se acercaban a las virtudes más eminentes; es decir, seleccionar de esas otras dos reglas la que fuera más segura, más segura y cautelosa.

\par 7 Hircano era al principio uno de los fariseos; después se pasó a los saduceos;

\par 8 porque uno de los fariseos le había dicho: No te es lícito ser sumo sacerdote, porque tu madre estaba cautiva antes de darte a luz, en los días de Antíoco; pero no conviene que el hijo del cautivo sea sumo sacerdote.

\par 9 Y esta conversación tuvo lugar en presencia de los principales de los fariseos; que fue la causa de su paso al gobierno de los saduceos.

\par 10 Los saduceos estaban enemistados con los fariseos; por lo que mantuvieron diferencias entre ellos, y lo prevalecieron hasta el punto de matar a un gran número de fariseos.

\par 11 Y el problema llegó a tal punto que continuaron entre ellos guerras y muchos males durante mucho tiempo.

\chapter{26}

\par \textit{El relato de la muerte de Hircano y de la época de su reinado}

\par 1 Hircano tuvo tres hijos: Antígono, Aristóbulo y Alejandro.

\par 2 Hircano amaba a Antígono y a Aristóbulo; pero Alejandro le era odioso.

\par 3 Y una vez vio en sueños que Alejandro reinaría entre sus hijos después de su muerte; y esto le produjo inquietud.

\par 4 Y mientras vivía, no le pareció conveniente levantar a ninguno de los hijos a quienes amaba, a causa de su visión;

\par 5 ni nombrar rey a Alejandro, porque no le agradaba. Por lo que aplazó el asunto; para que después de su muerte pudiera tomar ese giro que agradaría al gran y buen Dios.

\par 6 Ahora bien, en tiempos de su padre y de sus tíos, los judíos estaban unidos en afecto hacia ellos; y pronto para obedecerlos, debido a que sometieron a sus enemigos y las excelentes hazañas que realizaron.

\par 7 También ellos permanecieron unidos en afecto a Hircano; hasta que cometió la matanza de los fariseos, el desarraigo de los judíos y las guerras civiles por causa de la religión.

\par 8 De ahí surgieron perpetuas enemistades, males incesantes y muchos asesinatos. Por esta razón muchos detestaban a Hircano.

\par 9 ¿Y fue el tiempo de su reinado treinta y un años? años y murió.

\chapter{27}

\par \textit{La historia de Aristóbulo, hijo de Hircano}

\par 1 Muerto Hircano, le sucedió en el trono su hijo Aristóbulo; quien mostró altivez, orgullo y poder; y puso sobre su cabeza una gran corona, en desprecio de la corona del sagrado sacerdocio.

\par 2 Ahora sentía afecto por su hermano Antígono, a quien prefería a todos sus amigos; pero a su hermano lo mantuvo en prisión, como también a su madre, a causa de su amor por Alejandro.

\par 3 Y envió a su hermano Antígono, que luchó contra él y lo venció, con todos sus cómplices y tropas, a los que hizo huir, y regresó a la ciudad de la Santa Casa. Esto sucedió mientras Aristóbulo yacía enfermo.

\par 4 Cuando Antígono iba camino a la ciudad, le informaron de la enfermedad de su hermano; quien, entrando en la ciudad, fue a la casa de Dios, para dar gracias por la misericordia mostrada en su liberación del enemigo, y para suplicar al Dios grande y bueno que le devolviera la salud a su hermano.

\par 5 Por lo tanto, algunos de los que eran adversarios y enemigos de Antígono fueron a Aristóbulo y le dijeron:

\par 6 En verdad, la noticia de tu enfermedad llegó a tu hermano, y he aquí que viene armado con sus partidarios; y ahora ha entrado en el santuario para hacerse amigos, para venir repentinamente sobre vosotros y mataros.

\par 7 Y el rey Aristóbulo temía tomar alguna medida apresurada contra su hermano respecto a lo que le había dicho, hasta que supiera la exactitud de la noticia.

\par 8 Por lo tanto, ordenó a todos sus servidores que se apostaran armados en un lugar determinado, del cual cualquiera que llegara a su palacio no pudiera desviarse.

\par 9 También ordenó que se proclamara públicamente que nadie con armas de ningún tipo entraría en el patio al rey sin estar escondido.

\par 10 Después de esto, envió un mensaje a Antígono, ordenándole que fuera a su encuentro; entonces Antígono se quitó las armas en obediencia al rey.

\par 11 Mientras tanto, llega a él un mensajero de la mujer de su hermano Aristóbulo, que lo odiaba, y le dice:

\par 12 El rey te dice: «Ya he oído hablar de la belleza de tu vestimenta cuando entraste en la ciudad, y deseo verte así vestida; Por tanto, venid a mí en esa forma, para que pueda sentirme complacido de veros».

\par 13 Y Antígono no dudó que este mensaje provenía del rey, tal como el mensajero había informado;

\par 14 y que no quería ponerlo al mismo nivel que los demás en cuanto a la deposición de las armas; y se acercó a él con esa manera y vestido.

\par 15 Y cuando llegó al lugar donde el rey Aristóbulo había ordenado a sus hombres apostarse, con orden de matar a cualquiera que llegara allí armado,

\par 16 y cuando los hombres lo vieron armado, se abalanzaron sobre él y al instante lo mataron; y su sangre fluyó sobre el pavimento de mármol de ese lugar.

\par 17 Y el clamor de los hombres se hizo más fuerte, y su llanto y lamentación aumentaron, afligidos por la muerte de Antígono, por su belleza, y la elegancia de su discurso y sus hazañas.

\par 18 Entonces el rey, al oír el ruido de los hombres, preguntó al respecto; y descubrió que Antígono había sido asesinado;

\par 19 lo cual le causó gran dolor, tanto por el cariño que le tenía como porque no merecía esta suerte; y comprendió que le habían tendido una trampa:

\par 20 y gritó y lloró mucho; y golpeaba su pecho sin cesar; de modo que algunos vasos sanguíneos de su pecho se reventaron y la sangre salió de su boca.

\par 21 Pero sus servidores y los principales de sus amigos vinieron a él para consolarlo, apaciguarlo y tranquilizarlo para impedirlo de esta acción;

\par 22 temiendo morir, porque estaba débil y estaba a punto de morir por lo que ya había hecho.

\par 23 Y tomaron una vasija de oro para recibir la sangre que brotaba de su boca;

\par 24 Y enviaron la palangana con la sangre que había en ella, por mano de uno de los asistentes, a un médico para que la viera y aconsejara lo que había que hacer con él.

\par 25 Y el paje se fue con la palangana; y cuando llegó al lugar donde habían matado a Antígono, y su sangre había corrido, el paje resbaló y cayó; y derramó la sangre del rey que estaba en la palangana sobre la sangre de su hermano asesinado.

\par 26 Y el paje volvió con la palangana y contó a los cortesanos lo que había sucedido; quien abusó de él y lo injuriaba; mientras se justificaba y juraba que no lo había hecho intencionalmente ni voluntariamente.

\par 27 Pero cuando el rey los escuchó pelear, pidió hacerlo. Se les dijo lo que decían, y se callaron; pero cuando los amenazó, se lo dijeron.

\par 28 Quien entonces dijo: «¡Alabado sea el juez justo, que ha derramado la sangre del opresor sobre la sangre del oprimido!»

\par 29 Entonces gimió y al momento expiró. Y el tiempo de su reinado fue un año completo.

\par 30 Y todo su rebaño se lamentó por él; porque era noble, victorioso y liberal; y su hermano Alejandro reinó en su lugar.

\chapter{28}

\par \textit{El relato de Alejandro hijo de Hircano}

\par 1 DESPUÉS de la muerte de Aristóbulo, su hermano Alejandro fue liberado de sus cadenas; y siendo sacado de la cárcel, le sucedió en el trono.

\par 2 Ahora bien, el gobernador de la ciudad de Acche (que es Tolemaida) se había rebelado; y había enviado mensajeros a Lythras, hijo de Cleopatra, pidiéndole que lo ayudara y lo tomara bajo su protección;

\par 3 pero él se negó durante mucho tiempo, temiendo que se repitieran las cosas que ya había sufrido a causa de Hircano.

\par 4 Pero el mensajero le dio ánimos mediante los socorros prometidos por los señores de Tiro, de Sidón y otros. Y Lythras marchó con treinta mil hombres:

\par 5 y la noticia de esto fue comunicada a Alejandro, quien se anticipó a él en Tolemaida y la atacó; y los ciudadanos de Ptolemaida le cerraron la puerta en la cara y trataron de impedirle la entrada.

\par 6 Por lo que Alejandro los puso en aprietos y continuó sitiándolos; hasta que fue informado de la marcha de Lythras; luego se retiró de delante de ellos, estando Lythras con sus tropas cerca.

\par 7 Había entre los ciudadanos de Tolemaida un anciano de reconocida autoridad, que persuadió a los ciudadanos para que no permitieran a Lythras entrar en su ciudad ni asumir su obediencia a él, ya que era de otra religión.

\par 8 También les dijo: Mucho más ventajoso para vosotros será en todos los sentidos la sumisión a Alejandro, que es de la misma religión, que la sumisión a Litra. Y no cesó hasta que ellos aceptaron sus sentimientos.

\par 9 E impidieron a Litra entrar en Tolemaida, negándose a someterse a él. Y Lythras estaba perplejo en sus asuntos, y no consultó qué era lo mejor para él.

\par 10 Y esto fue informado al rey de Sidón, y éste le envió mensajeros para que le ayudara en la guerra contra Alejandro; que podrían derrotarlo o falsificar algunas de sus ciudades y así castigarlo;

\par 11 y así Lythras podría regresar a su propio país, después de realizar actos que podrían volverlo formidable; lo cual en verdad sería más ventajoso para él que regresar sin haber realizado su propósito.

\par 12 Y esto fue dicho a Alejandro; quien envió a Lythras una embajada honorable con un presente muy valioso, y le propuso no ayudar al rey de Sidón.

\par 13 Y Litra aceptó el regalo de Alejandro y accedió a su petición.

\par 14 Pero Alejandro marchó a Sidón y peleó contra su soberano; y Dios lo hizo victorioso sobre él, y mató a muchos de sus hombres; y habiéndolo hecho huir, tomó posesión de su país.

\par 15 Después de esto, Alejandro envió mensajeros a Cleopatra para que viniera con un ejército contra Lythras su hijo; y que él también marcharía con su ejército contra él, y le entregaría prisionero.

\par 16 Cuando Lytra se enteró, se fue a la montaña de Galilea, mató a muchos de sus habitantes y se llevó cautivos a diez mil; también murieron muchos de sus propios hombres.

\par 17 Desde allí marchó hasta llegar al Jordán, y acampó allí; para que sus hombres y caballos descansaran y luego pudiera marchar a Jerusalén para luchar con Alejandro.

\par 18 Esto le fue dicho a Alejandro; quienes fueron contra él con cincuenta mil hombres, de los cuales seis mil tenían escudos de bronce: y se dice que cada uno de ellos podía resistir cualquier número de hombres.

\par 19 Y lo atacó en el Jordán y allí se enfrentó a él; pero no obtuvo la victoria, porque confiaba en sus hombres y había puesto su confianza en su número.

\par 20 Pero junto a Lythras había hombres muy hábiles en las batallas y en formar ejércitos; quien le aconsejó dividir sus fuerzas en dos partes, de modo que una pudiera estar con Lythras y su compañía preparada para la batalla, y la otra parte pudiera estar con otro capitán de su compañía.

\par 21 Y luchó hasta el mediodía, y muchos de sus hombres fueron asesinados.

\par 22 Y su amigo avanzó con el resto del ejército que estaba con él, cuyas fuerzas aún estaban completas, contra Alejandro y sus hombres, que ya estaban abrumados por el cansancio.

\par 23 Hizo con ellos lo que quiso y mató a muchos de ellos; y Alejandro y los hombres que se habían quedado con él huyeron a la ciudad de la Santa Casa.

\par 24 También Litra partió al anochecer hacia una ciudad cercana; y por casualidad le encontraron unas mujeres judías con sus hijos;

\par 25 y ordenó que mataran a algunos de los niños y que les prepararan la carne, fingiendo que había algunos en su ejército que se alimentaban de carne humana; planeando con estos actos golpear a los habitantes del país con temor a sus tropas.

\par 26 Después de esto vino Cleopatra; a quien Alejandro conoció, le contó lo que Lythras le había hecho a su ejército y le pidió que fuera con ella a buscarlo.

\par 27 Dicho esto a Lythras, éste huyó a un lugar donde estaban atracados sus barcos; Subiendo a bordo del cual, regresó a Chipre; y Cleopatra regresó a Egipto.

\par 28 Pero al final del año Alejandro marchó contra Gaza; porque su jefe se había rebelado contra él y había enviado a cierto rey de los árabes llamado Hartasi para que lo ayudara; quien consintió en hacerlo, y marchó hacia Gaza:

\par 29 Esto le fue dicho a Alejandro; quien, dejando algunos de sus hombres delante de Gaza, marchó contra Hartas, lo enfrentó y lo puso en fuga.

\par 30 Luego regresó a Gaza y, adolorido sobre ella, la tomó al cabo de un año.

\par 31 Pero la causa de que la tomara fue el hermano de aquel jefe; el cual, acercándose repentinamente a él, lo mató.

\par 32 Cuando los ciudadanos querían matarlo, reunió a sus amigos, se dirigió a la puerta de la ciudad y se dirigió a Alejandro, rogándole que, dando seguridad por su vida y la de sus amigos, entraría en la ciudad;

\par 33 Alejandro, como prometió, entró en Gaza y mató a sus habitantes, derribó el templo que había en ella y quemó el ídolo dorado que había en el templo.

\par 34 Después de lo cual partió hacia la ciudad de la Santa Casa, y allí celebró la fiesta de las Tiendas.

\par 35 Y cuando pasó la fiesta, se preparó contra Hartas, a quien encontró, y mató a un gran número de sus hombres.

\par 36 y los asuntos de Hartas estaban muy apurados y paralizados, y él temía su propia extinción total. Por lo que, demandando a Alejandro por su vida, le rindió obediencia y le pagó tributos.

\par 37 Entonces Alejandro se apartó de él y marchó contra Hemat y Tiro, y las tomó; y habiendo recibido tributo de los habitantes, volvió a la ciudad de la Santa Casa.

\chapter{29}

\par \textit{Un relato de las batallas que tuvieron lugar entre los fariseos y los saduceos.}

\par 1 Después surgieron males entre los fariseos y los saduceos, que continuaron por espacio de seis años.

\par 2 Alejandro ayudó a los saduceos contra los fariseos, de los cuales en seis años murieron cincuenta mil.

\par 3 Por lo tanto, entre estas dos sectas el estado de cosas se redujo a la destrucción total y su enemistad se confirmó por completo.

\par 4 Entonces Alejandro, enviando a buscar a los ancianos de cada secta, les habló amablemente y les aconsejó una reconciliación.

\par 5 Pero ellos le respondieron: «En verdad, en nuestra opinión, eres digno de muerte por la abundancia de sangre inocente que has derramado; por tanto, no haya nada entre nosotros excepto la espada».

\par 6 Después de esto, comenzaron a mostrar abiertamente su enemistad, enviando mensajeros a Demetrio, rey de Macedonia, para que viniera a ellos con un ejército;

\par 7 prometiendo que lo ayudarían contra Alejandro y su grupo, y someterían a los hebreos a los macedonios. Y Demetrio marchó hacia ellos con un gran ejército.

\par 8 Lo cual también fue dicho a Alejandro; quien envió a alguien a contratar seis mil macedonios, los cuales, uniendo sus propias fuerzas, avanzó contra Demetrio.

\par 9 También muchos judíos fariseos se pasaron a Demetrio.

\par 10 Y Demetrio envió secretamente personas a los macedonios que estaban con Alejandro, para alejarlos de él; pero no le escucharon.

\par 11 Alejandro también envió hombres en secreto a los judíos que estaban con Demetrio, para volverlos a su lado; pero estos tampoco hicieron lo que él quería.

\par 12 Entonces Alejandro y Demetrio se encontraron y pelearon una batalla; en el cual cayeron todos los hombres de Alejandro, y él escapó solo a la tierra de Judá.

\par 13 Pero cuando sus hombres oyeron que se rumoreaba que había escapado sano y salvo, y descubrieron el lugar donde se encontraba;

\par 14 Se reunieron con él unos seis mil hombres de los más valientes de los hijos de Israel; y muchos de los que se habían rebelado contra Demetrio se unieron a él.

\par 15 Después acudieron a él hombres de todas partes; y volvió para dar batalla a Demetrio con una fuerza numerosa, y lo puso en fuga; y Demetrio regresó a su propio país.

\par 16 Y Alejandro marchó contra él hasta Antioquía, y la sitió durante tres años; y cuando Demetrio salió a pelear, Alejandro lo venció y lo mató.

\par 17 Y salió de la ciudad y volvió a Jerusalén con sus ciudadanos; quienes lo engrandecieron, honrándolo y alabandolo por haber vencido a sus enemigos.

\par 18 Los judíos acordaron someterse a él, y su corazón se tranquilizó, y envió sus ejércitos contra todos sus enemigos, a quienes hizo huir y obtuvo la victoria sobre ellos.

\par 19 También se apoderó de las montañas de Sara, de la tierra de Amón, de Moab, de la tierra de los filisteos y de todas las partes que estaban en manos de los árabes que peleaban con él, hasta los límites del desierto.

\par 20 Y los asuntos de su reino fueron gobernados correctamente; y puso a su pueblo y a su país en estado de seguridad.

\chapter{30}

\par \textit{El relato de la muerte de Alejandro hijo de Hircano}

\par 1 Después el rey Alejandro enfermó de fiebre cuartana durante tres años enteros.

\par 2 Pero cuando el gobernador de una ciudad llamada Ragaba se rebeló contra él, condujo allí un ejército poderoso, llevando consigo a su esposa y a su familia, y sitió la ciudad.

\par 3 Pero cuando estaba a punto de ser tomado, su enfermedad aumentó y sus fuerzas disminuyeron; y su esposa, que se llamaba Alejandra, perdió toda esperanza de su recuperación:

\par 4 el cual, acercándose a él, le dijo; «Ahora sabes cuáles son las diferencias entre tú y los fariseos: y tus dos hijos son niños, y yo soy mujer, y en absoluto podremos resistirlos: ¿qué consejo, pues, nos das a mí y a ellos? »

\par 5 Él le dijo: «Mi consejo es que perseveres contra la ciudad hasta que sea tomada, lo cual será pronto.

\par 6 Y cuando la hayas conquistado, establecerás su gobierno según lo establecido en las demás ciudades.

\par 7 Pero ante toda esta gente finge que estoy enfermo; y hagas lo que hagas, finge que lo haces por sugerencia mía; y revela mi muerte a aquellos servidores en quienes puedas confiar.

\par 8 Y cuando hayas terminado esto, ve a la ciudad de la Santa Casa, después de haber secado y embalsamado mi cuerpo con especias; y llena el lugar donde estoy acostado de muchos perfumes, para que no salga de mí ningún olor desagradable.

\par 9 Y cuando los asuntos del país estén resueltos, ve de allí, envuélveme en abundantes perfumes y llévame al palacio como si estuviera enfermo.

\par 10 y cuando esté allí, mandad llamar a los principales de los fariseos; y cuando vengan, hónralos y diles buenas palabras:

\par 11 Entonces di: Alejandro ya está muerto, y he aquí te lo entrego, haz con él lo que te parezca bien y desde ahora se comportará contigo como quieras.

\par 12 Porque si hacéis esto, sé muy bien que no nos harán nada a mí ni a vosotros, excepto lo que es bueno; y el pueblo los seguirá, y tus asuntos se ordenarán inmediatamente después de mi muerte, y reinarás seguro hasta que tus dos hijos crezcan».

\par 13 Después de esto, Alejandro murió; y su esposa ocultó su muerte; y cuando la ciudad fue tomada, ella volvió a Jerusalén; y habiendo llamado a los principales de los fariseos, les habló como Alejandro le había aconsejado.

\par 14 A ellos respondieron que Alejandro había sido su rey y ellos su pueblo; y le hablaron con todo cariño, y le prometieron ponerla al frente de su gobierno.

\par 15 Entonces salieron y reunieron hombres; y tomando el cuerpo de Alejandro, lo llevaron magníficamente a su entierro: y enviaron hombres para nombrar reina a Alejandra; con cuyo consentimiento fue designada.

\par 16 Y fueron los años del reinado de Alejandro veintisiete.

\chapter{31}

\par \textit{La historia de la reina Alejandra}

\par 1 Mientras reinaba Alejandra, llamó a los jefes de los fariseos y les ordenó que escribieran a todos los de su secta que habían huido a Egipto y a otras partes, en los días de Hircano y de Alejandro, que debería regresar a la tierra de Judá.

\par 2 Y ella les mostró su inclinación favorable hacia ellos y no se opuso a sus ritos ni prohibió sus ceremonias, como les habían prohibido Alejandro e Hircano.

\par 3 También liberó a todos los que estaban detenidos en prisión.

\par 4 Y se reunieron de todas partes; y los saduceos se abstuvieron de hacerles violencia.

\par 5 Y sus asuntos estaban bien ordenados y su condición mejoró al eliminarse las contiendas.

\par 6 Pero cuando Hircano y Aristóbulo, los dos hijos de Alejandro, crecieron, la reina nombró sumo sacerdote a Hircano, porque era manso, apacible y honesto:

\par 7 pero nombró general del ejército a Aristóbulo, porque era fuerte, valiente y alegre; y ella también le dio el ejército de los saduceos; pero no consideró conveniente nombrarlo rey, porque aún era un niño.

\par 8 Además, envió a todos los que pagaban tributo a Alejandro y tomó a los hijos de sus reyes, a quienes retuvo cerca de ella como rehenes; y continuaron ininterrumpidamente en su obediencia a ella, pagándole tributo cada año.

\par 9 Y ella caminó rectamente con su pueblo, impartiendo justicia y ordenando a su pueblo que hiciera lo mismo. Por lo tanto hubo una paz duradera entre las partes y ella se ganó su buena voluntad.

\chapter{32}

\par \textit{Un relato de las cosas que los fariseos hicieron a los saduceos en tiempos de Alejandra.}

\par 1 Había entre los saduceos un jefe que había sido ascendido por Alejandro, llamado Diógenes, quien anteriormente lo había inducido a matar a ochocientos hombres de los fariseos.

\par 2 Entonces los jefes de los fariseos vinieron a Alejandra, le recordaron lo que había hecho Diógenes y le pidieron permiso para matarle; que ella les dio, y tomándolo, mataron juntamente con él a muchos saduceos.

\par 3 Lo cual los saduceos, muy en serio, fueron a Aristóbulo; y llevándolo consigo, fueron a la reina y le dijeron:

\par 4 «Tú sabes qué cosas terribles y pesadas hemos sufrido, y las muchas guerras y batallas que hemos librado, en ayuda de Alejandro y su padre Hircano.

\par 5 Por lo tanto, no era justo pisotear nuestros derechos, ni alzar sobre nosotros la mano de nuestros enemigos, ni rebajar nuestras dignidades;

\par 6 Porque un asunto de este tipo no se ocultará a Hartas ni a otros de tus enemigos; que han experimentado nuestra valentía, y no han podido resistirnos, y sus corazones se han llenado del temor de nosotros.

\par 7 Por tanto, cuando se den cuenta de lo que nos has hecho, se imaginarán que nuestro corazón trama planes contra ti; de los cuales, cuando sean certificados, confía en que te engañarán.

\par 8 Ni toleraremos que los fariseos nos maten como a ovejas.

\par 9 Por lo tanto, o apartad de nosotros su malicia o nos permitís salir de la ciudad a algunas de las ciudades de Judá.

\par 10 Y ella les dijo: «Haced esto para evitar que os molesten».

\par 11 Y los saduceos salieron de la ciudad; y partieron sus jefes con los hombres de guerra que se les adhirieron; y fueron con sus ganados a las ciudades de Judá que habían escogido, y habitaron en ellas;

\par 12 y se unieron a ellos los dedicados a la virtud (es decir, los jasdanim).

\chapter{33}

\par \textit{El relato de la muerte de Alexandra}

\par 1 Después de esto, Alejandra cayó en una enfermedad de la que murió.

\par 2 Y cuando ya era casi imposible que se recuperara, su hijo Aristóbulo salió de Jerusalén de noche, acompañado de su criado:

\par 3 y partió hacia Gabatá, donde estaba un jefe de los saduceos, uno de sus amigos;

\par 4 y llevándolo consigo, se dirigió a las ciudades donde habitaban los saduceos; y les reveló su propósito, y los exhortó a salir con él, y a ser sus aliados en la guerra contra su hermano y los fariseos, y a nombrarlo rey.

\par 5 A quienes ellos consintieron, jugaron abiertamente en falso con Alejandra, reuniendo hombres de pie para unirse a Aristóbulo.

\par 6 Cuando la fama de estas cosas llegó a oídos de Hircano, hijo de Alejandra, el sumo sacerdote, y los ancianos de los fariseos, fueron a ver a Alejandra, que estaba enferma, y ​​le contaron el asunto.

\par 7 acentuándola por el gran temor que tenían por ella y su hijo Hircano, por parte de Aristóbulo y de los que con él estaban.

\par 8 A quien ella respondió; «Verdaderamente estoy cerca de la muerte, por lo que me es más propio y provechoso ocuparme de mis propios asuntos; ¿Qué, pues, puedo hacer estando así situado?

\par 9 Pero mis hombres, mis bienes y mis armas están con vosotros y en vuestras manos; por tanto, ordena el negocio como te parezca bien, implorando la ayuda de Dios en tus asuntos, y pidiéndole liberación». Luego ella murió.

\par 10 Su edad era setenta y tres años; y el tiempo de su reinado nueve años,

\chapter{34}

\par \textit{El relato del ataque de Aristóbulo a su hermano Hircano, después de la muerte de Alejandra.}

\par 1 Cuando Aristóbulo salió de Jerusalén en tiempos de Alejandra, dejó a su mujer y a sus hijos en Jerusalén.

\par 2 Pero cuando Alejandra tuvo noticia de su partida, los encerró en una casa y les puso guardia.

\par 3 Pero cuando Alejandra murió, Hircano los llamó, los trató con bondad y los cuidó; para librarlo de su hermano, si acaso lo conquistara.

\par 4 Entonces Aristóbulo condujo un gran ejército hasta el Jordán; e Hircano salió contra él con un ejército de fariseos.

\par 5 Y cuando los dos ejércitos se encontraron, muchos del ejército de Hircano fueron asesinados; Hircano y el resto de su ejército huyeron.

\par 6 Aristóbulo y sus tropas, a quienes perseguían, mataron a todos los que capturaron, excepto a los que se entregaron.

\par 7 Entonces Hircano se retiró a la Ciudad Santa; adonde también llegó Aristóbulo y su ejército; y la rodeó por todos lados con sus tiendas e intentó mediante estratagema destruir la fortificación.

\par 8 Entonces los ancianos de Judá y los ancianos de los sacerdotes salieron a él y le prohibieron hacer lo que había planeado; solicitándole que descartara de su mente cualquier sentimiento hostil que tuviera hacia su hermano: propuesta a la cual él asintió.

\par 9 Entonces se acordó entre ellos que Aristóbulo sería rey de Judá e Hircano sería sumo sacerdote en la casa de Dios, y junto al rey en dignidad.

\par 10 Y Aristóbulo aceptó estas condiciones, entró en la ciudad y se entrevistó con su hermano en la casa de Dios; y juntos prestaron juramento para ratificar los términos que los ancianos habían acordado mutuamente.

\par 11 Entonces Aristóbulo fue nombrado rey, e Hircano fue nombrado rey después de él.

\par 12 Y los hombres estaban en paz, los asuntos de estos dos hermanos estaban correctamente ordenados y el estado de su pueblo y de su país se volvió tranquilo.

\chapter{35}

\par \textit{El relato de Antípatro (es decir, el rey Herodes) y de las sediciones y batallas que encendió entre Hircano y Aristóbulo.}

\par 1 Había un hombre judío, de los hijos de algunos de los que subieron de Babilonia con el sacerdote Esdras, llamado Antípater.

\par 2 Y era sabio, prudente, sagaz, valiente y altivo, de buen carácter, bondadoso y cortés; también rico y poseedor de muchas casas, bienes y rebaños.

\par 3 A este hombre el rey Alejandro había nombrado gobernador del país de los idumzanos, de donde había tomado esposa; de quien tuvo cuatro hijos: Faselo, Herodes, que reinó sobre Judá, Feroras y Josefo.

\par 4 Después de haber salido de los montes de Sara, es decir, de la tierra de los idumeos, en tiempos de Alejandro, habitó en la ciudad de la Santa Casa.

\par 5 Hircano lo amaba y se inclinaba mucho hacia él; por eso Aristóbulo intentó matarlo; lo cual, sin embargo, no logró.

\par 6 Por eso Antípatro tenía mucho miedo de Aristóbulo y por eso comenzó a conspirar en secreto contra el reino de Aristóbulo.

\par 7 Se dirigió, pues, a los principales hombres del reino y, habiendo obtenido de ellos una promesa de secreto sobre lo que iba a comunicar,

\par 8 Comenzó a hablarles de la infame vida de Aristóbulo, de su tiranía, de su impiedad, del derramamiento de sangre que había causado y de su usurpación del trono; de lo cual su hermano mayor era más digno.

\par 9 Entonces les ordenó que tuvieran cuidado con el Dios grande y bueno, a menos que le quitaran la mano gobernante al tirano y le devolvieran lo que le correspondía a su legítimo soberano.

\par 10 No quedó ni uno solo de los principales a quien no se extralimitó e inclinó a someterse a Hircano, seduciéndolos de su obediencia a Aristóbulo, sin que Hircano supiera nada de esto.

\par 11 pero ¿se atribuye Antípatro? todo esto a él, no queriendo decírselo antes de haber establecido la cosa.

\par 12 Entonces, cuando hubo resuelto este asunto con el pueblo, fue a Hircano y le dijo:

\par 13 En verdad, tu hermano te tiene mucho miedo, porque ve que su patrimonio no estará seguro mientras vivas; por lo que busca una oportunidad para matarte y no permitirá que vivas.

\par 14 Pero Hircano no le dio crédito debido a la bondad y sinceridad de su corazón. Por lo que Antípatro le repitió este discurso una y otra vez.

\par 15 También dio grandes sumas de dinero a las personas en quienes Hircano confiaba, y acordó con ellos que le contarían cosas similares a las que Antípatro había mencionado;

\par 16 sólo teniendo cuidado de que no creyera que sabían que Antípatro le había hablado del asunto.

\par 17 Entonces Hircano creyó en sus palabras; y se vio inducido a idear un plan mediante el cual podría liberarse de su hermano.

\par 18 Cuando Antípatro volvió a hablarle del asunto, le informó que el gc. La verdad de sus palabras ahora le era manifiesta y sabía que le había aconsejado bien; y le pidió consejo en este asunto.

\par 19 Y Antípatro le aconsejó que saliera de la ciudad en busca de alguien en quien pudiera confiar y que pudiera ayudarle y asistirle.

\par 20 Antípatro fue a Hartam y acordó con él que recibiría a Hircano como huésped cuando viniera, ya que tenía miedo de vivir con su hermano.

\par 21 Por lo cual Hartam se alegró, aceptó el plan y acordó con Antípatro que en ningún caso entregaría a Hircano y a Antípatro en manos de sus enemigos, y que los ayudaría y protegería.

\par 22 Y regresó a Jerusalén y le contó a Hircano lo que había hecho y cómo había acordado con Hartam que debían ir a él.

\par 23 Entonces ambos salieron de la ciudad de noche y fueron a Hartam y permanecieron con él algún tiempo.

\par 24 Entonces Antípatro comenzó a persuadir a Hartam para que liderara un ejército con Hircano para reducir y capturar a su hermano Aristóbulo.

\par 25 Pero Hartam se negó a llevar a cabo este plan, temiendo no tener fuerzas para resistir a Aristóbulo.

\par 26 Pero Antípatro no dejaba de demostrarle que el negocio con Aristóbulo era fácil, y de incitarlo a ello con argumentos del tesoro que había de ganar, de la grandeza de la gloria que adquiriría y de la memoria que tendría que dejar atrás él:

\par 27 hasta que consintió en marchar; pero con la condición de que Hircano le devolviera todas las ciudades y pueblos que le pertenecían y que su padre Alejandro le había quitado.

\par 28 A lo que Hircano accedió y cumplió el tratado, Hartam marchó (y Hircano con él) con cincuenta mil soldados de a caballo y de infantería, dirigiéndose hacia el país de Judá, contra quien Aristóbulo salió y los enfrentó.

\par 29 Y cuando la lucha se volvió feroz, muchos del ejército de Aristóbulo se pasaron a Hircano.

\par 30 Al verlo Aristóbulo, dio la orden de retirarse y regresó a su campamento, temiendo que todo su ejército se escapara poco a poco al enemigo y él mismo cayera prisionero.

\par 31 Pero cuando llegaba la noche, Aristóbulo salió solo del campamento y se dirigió a la Ciudad Santa.

\par 32 Y cuando, al amanecer, el ejército supo su partida, la mayor parte de ellos se unieron a Hircano, y el resto se dispersó y se fue.

\par 33 Pero Hircano, Hartam y Antípater fueron directamente a la ciudad de la Santa Casa, llevando consigo un gran ejército;

\par 34 y encontraron a Aristóbulo ya preparado para el sitio; porque había cerrado las puertas de la ciudad y había colocado hombres en las murallas para defenderlas.

\par 35 Hircano y Hartam acamparon con sus fuerzas frente a la ciudad y la sitiaron.


\chapter{36}

\par \textit{La historia de Gneus, general del ejército de los romanos.}

\par 1 Aconteció que Gneo, general del ejército de los romanos, salió a luchar contra Tyrcanes el armenio:

\par 2 Porque los ciudadanos de Damasco, Hames, Halepum y el resto de Siria, que pertenecen a los armenios, recientemente se habían rebelado contra los romanos:

\par 3 Por eso Gneo había enviado a Escauro a Damasco y a sus territorios para tomar posesión de ellos; lo que se dijo a Aristóbulo y a Hircano.

\par 4 Por lo tanto, Aristóbulo envió embajadores a Escauro y mucho dinero, pidiéndole que viniera a él con un ejército y lo ayudara contra Hircano.

\par 5 Hircano también le envió embajadores pidiéndole ayuda contra Aristóbulo; pero no le envió ningún regalo.

\par 6 Pero Escauro se negó a ir a ninguno de los dos, sino que escribió a Hartam, ordenándole que se retirara con su ejército de la ciudad de la Santa Casa, y le prohibió ayudar a Hircano contra su hermano;

\par 7 y amenazó con entrar en su país con un ejército de romanos y sirios, si no obedecía.

\par 8 Cuando esta carta llegó a Hartam, él inmediatamente se retiró de la ciudad:

\par 9 Hircano también se retiró; a quienes Aristóbulo persiguió con un cierto número de sus tropas, y los alcanzó y se enfrentó a ellos': y un gran número de árabes murieron en esa batalla, y muchísimos judíos: y Aristóbulo regresó a la Ciudad Santa.

\par 10 Mientras tanto, Gneo llegó a Damasco; a quien Aristóbulo envió, por mano de un hombre llamado Nicomedes, un huerto y una viña de oro, que pesaban en total quinientos talentos, con un regalo muy rico; y le rogó que le ayudara contra Hircano.

\par 11 Hircano también envió a Antípatro a Pompeyo con la misma petición.

\par 12 Y Pompeyo (que es Gneo) se sintió inclinado a ayudar a Aristóbulo.

\par 13 Antípatro, al verlo, aguardó la oportunidad de hablar a solas con Pompeyo y le dijo:

\par 14 En verdad, no es necesario que se le devuelva el regalo que has recibido de Aristóbulo, aunque no le ayudes;

\par 15 Sin embargo, Hircano os ofrece el doble: y Aristóbulo no podrá someter a los judíos a vosotros, pero este Hircano sí.

\par 16 Y Pompeyo supuso que todo era así, tal como Antípatro había dicho; y se regocijó al pensar que podría poner a los judíos bajo su dominio.

\par 17 Por lo que dijo a Antípatro: Ayudará a tu amigo contra Aristóbulo; aunque pueda pretender ayudarlo contra ti, para que se confíe a mí.

\par 18 Porque estoy seguro de que tan pronto como se entere de que estoy ayudando a su hermano contra él, actuará en falso con todos sus hombres, se cuidará solo y su negocio será mucho más demorado.

\par 19 Pero enviaré a buscarlo e iré con él a la Ciudad Santa, y luego haré lo posible para que tu amigo obtenga su derecho; pero con la condición de que nos pague un tributo anual».

\par 20 EL MENSAJERO DE ARISTÓBULO. Después de esto, llamando a Nicomedes, le dijo: «Ve a tu amo y dile que he accedido a su petición; y llévale mi carta y dile que debe venir a verme lo antes posible, porque lo estoy esperando.

\par 21 Y escribió una carta a Aristóbulo, de la cual ésta es una copia:

\par 22 «Desde Gneo, general del ejército romano, hasta el rey Aristóbulo, heredero del trono y del sumo sacerdocio, la salud sea para vosotros.

\par 23 Han llegado tu huerto y tu vid de oro; y los he recibido y los he enviado a los «ancianos y gobernadores; que han aceptado «¡y han colocado en el templo! en Roma, dándote las gracias.

\par 24 Además, me han escrito que te ayudaré y te nombraré rey sobre los judíos.

\par 25 Por lo tanto, si crees conveniente «venir a mí lo antes posible, para que pueda subir contigo a la Ciudad Santa y cumplir tus deseos, así lo haré».

\par 26 Y Nicomedes partió hacia Aristóbulo con la carta de Gneo. Y Antípatro, regresando a Hircano, le habló de la promesa de Gneo, aconsejándole que fuera a Damasco.

\par 27 Entonces Hircano fue a Damasco, y Aristóbulo también; y se encontraron en Damasco en la sala de audiencias de Pompeyo, es decir, Gneo; y Antípatro y los ancianos de los judíos dijeron a Gneo:

\par 28 «Sepa, «ilustre general», que esto. Aristóbulo nos ha traicionado y ha usurpado con la espada el reino de su hermano Hircano, que es más digno de él que él, por ser el hermano mayor y llevar una vida mejor y más correcta.

\par 29 Y no le bastó oprimir a su hermano, sino que oprimió a todas las naciones que nos rodean; derramando su sangre y saqueando injustamente sus bienes, y manteniendo enemistades entre nosotros y ellos, cosa que aborrecemos».

\par 30 Entonces se levantaron mil ancianos, dando fe de la verdad de sus palabras.

\par 31 Y Aristóbulo dijo: En verdad, este mi hermano es mejor hombre que yo; pero no busqué el trono, hasta que vi que todos los que habían estado sujetos a nuestro padre Alejandro, después de su muerte, nos engañaban, sabiendo la incapacidad de mi hermano.

\par 32 Lo cual, cuando lo examiné, comprendí que era mi deber asumir la soberanía, ya que yo era mejor que él en asuntos de guerra y por eso estaba mejor preparado para preservar la monarquía:

\par 33 Y fui a la guerra contra todos los que nos engañaban y los sometí a la obediencia; y ésta fue la orden de nuestro padre antes de su muerte.

\par 34 Y presentó testigos que dieron fe de la verdad de sus palabras.

\par 35 Después de esto, Pompeyo salió de la ciudad de Damasco y se dirigió a la Santa Casa.

\par 36 Pero Antípatro envió en privado a los habitantes de las ciudades conquistadas por Aristóbulo, incitándolos a quejarse ante Gneo, exponiéndoles la tiranía que había ejercido sobre ellos; qué cosa hicieron.

\par 37 Y Gneo le ordenó que les escribiera un testimonio de su libertad y que les dijera que no los molestaría más; lo cual verdaderamente hizo, y las naciones fueron liberadas de su obediencia a los judíos.

\par 38 Pero cuando Aristóbulo vio lo que Gneo le había hecho, él y sus hombres se alejaron de noche del ejército de Gneo, sin avisarle, y se dirigieron a la ciudad de la Santa Casa.

\par 39 Y Gneo lo siguió hasta que llegó a la ciudad de la Santa Casa, alrededor de la cual acampó.

\par 40 Pero cuando vio la altura de las murallas, la solidez de sus edificios, la multitud de hombres que había en ellas y las montañas que las rodeaban, comprendió que la adulación y la astucia serían más útiles contra Aristóbulo que actos de provocación:

\par 41 Por lo cual le envió embajadores para que vinieran a él, prometiéndole un salvoconducto. Y Aristóbulo salió hacia él; a quien Gneo recibió amablemente, sin decir una palabra sobre sus acciones anteriores. Después de esto, Aristóbulo dijo a Gneo:

\par 42 «Quisiera que me ayudaras contra mi hermano, sin darles poder a mis enemigos sobre mí; y para esto tendrás todo lo que desees».

\par 43 Gneo respondió: «Si quieres esto, tráeme todo el dinero y las piedras preciosas que haya en el templo, y te daré posesión de lo que deseas». Y Aristóbulo le dijo:

\par 44 «Sin duda esto haré». Y Gneo envió un capitán llamado Gabinio con un gran número de hombres, para recibir todo el oro y las joyas que había en el templo.

\par 45 Pero los ciudadanos y los sacerdotes se negaron a permitirlo, por lo que resistieron a Gabinio, mataron a muchos de sus hombres y amigos y lo expulsaron de la ciudad.

\par 46 Entonces Gneo, enojado con Aristóbulo, lo encarceló. »

\par 47 Luego marchó con su ejército para forzar su entrada en la ciudad y entrar en ella. Pero un gran número de ciudadanos que se marchaban le impidieron hacerlo, matando a gran número de sus hombres.

\par 48 Y en verdad, el número, el espíritu y la valentía de la nación que había visto, lo asustaron; De modo que, alarmado por estos, había decidido retirarse de ellos, si no hubieran surgido en la ciudad peleas maliciosas entre los amigos de Aristóbulo y los amigos de Hircano.

\par 49 Porque algunos querían abrir las puertas a Pompeyo, pero otros se oponían a ello. Por eso llegaron a las manos por este motivo; y como este estado de cosas aumentó en lugar de disminuir, la guerra continuó.

\par 50 Al darse cuenta de lo cual Pompeyo, sitió con su ejército la puerta de la ciudad; y como algunos del pueblo le abrieron una portilla, entró y se apoderó del palacio del rey; pero no pudo entrar al templo, porque los sacerdotes habían cerrado las puertas y habían asegurado los accesos con hombres armados.

\par 51 Envió hombres contra ellos para atacarlos por todos lados y los hicieron huir. Y sus amigos, llegando al templo, subieron al muro, descendieron al mismo y abrieron sus puertas, después de matar a una multitud de sacerdotes.

\par 52 Entonces llegó Gneo y entró en ella, y admiró mucho la belleza y la magnificencia que contemplaba, y se asombró al ver sus riquezas y las piedras preciosas que había en ella.

\par 53 y se abstuvo de sacar nada de él; y ordenó a los sacerdotes que limpiaran la casa de los muertos y ofrecieran sacrificios según las ceremonias de su país.

\chapter{37}

\par \textit{El relato del nombramiento de Hircano, hijo de Alejandro, como rey de los judíos, y del regreso a Roma del general del ejército romano.}

\par 1 Habiendo arreglado todo esto, Pompeyo nombró rey a Hircano; y se llevó a su hermano Aristóbulo encadenado:

\par 2 También ordenó que los judíos no tuvieran dominio sobre aquellas naciones que habían sido sometidas por sus reyes antes de su llegada;

\par 3 y exigió un tributo a la ciudad de la Santa Casa; y pactó con Hircano que recibiría la toma de posesión de los romanos cada año.

\par 4 Y partió, llevándose consigo a Aristóbulo, dos de sus hijos y sus hijas; y le quedó un hijo llamado Alejandro, a quien Pompeyo no pudo prender porque había huido.

\par 5 Entonces Pompeyo colocó en su habitación de la ciudad de la Santa Casa a Hircano y a Antípatro, y con ellos a su propio colega Escauro.

\chapter{38}

\par \textit{La historia de Alejandro hijo de Aristóbulo}

\par 1 Cuando Pompeyo partió hacia Roma, Hircano y Antípatro marcharon contra los árabes para someterlos al dominio de los romanos.

\par 2 A lo cual los árabes se sometieron, confiando en su intimidad con Antípatro y prestando mucha atención a sus consejos; mediante los cuales Antípatro actuó para reconciliar a los romanos con él.

\par 3 Por lo tanto, cuando Alejandro, hijo de Aristóulo, se dio cuenta de la expedición de Hircano, Antípatro y Escauro contra los árabes, y que se habían alejado a gran distancia de la Ciudad Santa,

\par 4 caminó hasta llegar allí; y entrando en palacio, sacó de allí dinero para reparar la muralla de la ciudad que Pompeyo había derribado.

\par 5 Y reunió para sí un ejército y dispuso todo lo que deseaba antes de que Hircano y su grupo regresaran a la ciudad de la Santa Casa; y cuando regresaron,

\par 6 salió a su encuentro, los atacó y los hizo huir.


\chapter{39}

\par \textit{La historia de Gabinio y de Alejandro hijo de Aristóbulo.}

\par 1 Gabinio había salido de Roma para habitar en la tierra de Siria y cuidarla;

\par 2 y le contaron lo que había hecho Alejandro, hijo de Aristóbulo, edificando lo que Pompeyo había derribado, oponiéndose a su sucesor y matando a sus amigos.

\par 3 Por lo tanto, siguió derecho hasta llegar a Jerusalén; e Hircano y su grupo se unieron a él.

\par 4 Contra ellos salió Alejandro con diez mil soldados de infantería y mil quinientos caballos, y los encontró.

\par 5 y lo derrotaron y mataron a algunos de sus amigos; y huyó a una ciudad de la tierra de Judá, llamada Alejandría, en la cual se fortificó con su compañía.

\par 6 Hircano, Gabinio y sus fuerzas marcharon contra él y lo sitiaron.

\par 7 Alejandro salió contra ellos, los enfrentó y mató a muchos de sus hombres.

\par 8 Y Marco, llamado Antonio, marchó contra él y le obligó a huir de nuevo a Alejandría.

\par 9 Y la madre de Alejandro salió a ver a Gabinio, desaprobando su ira y rogándole que le concediera la vida a su hijo Alejandro:

\par 10 a quien Gabinio asintió en este punto; y Alejandro salió a él; y Gabinio le dio muerte; y consideró apropiado dividir los territorios de Judá en cinco porciones.

\par 11 Uno es la tierra de Jerusalén y sus partes adyacentes; y sobre esta parte Hircano fue nombrado gobernador. Otra parte es Gadira y sus lugares.

\par 12 El tercero es Jericó y las llanuras. El cuarto es Hamat en la tierra de Judá. Y el quinto es Séforis.

\par 13 Con estos medios pretendía eliminar las guerras y las sediciones de la tierra de Judá; pero de ninguna manera fueron eliminados.

\chapter{40}

\par \textit2{La historia de la lucha de Aristóbulo y su hijo Antígono desde Roma, y ​​su regreso a la tierra de Judá: también, un relato de la muerte de Aristóbulo}

\par 1 Entonces Aristóbulo ideó planes hasta que logró escapar de Roma con su hijo Antígono y llegó a la ciudad de Judá.

\par 2 Y cuando Aristóbulo se mostró en público, una gran multitud de hombres acudió a su alrededor; De entre los cuales escogió ocho mil, y marchó contra Gabinio y lo enfrentó; y fueron muertos del ejército romano un número muy grande:

\par 3 También cayeron de sus propios hombres siete mil, pero mil escaparon; y el ejército enemigo lo persiguió; pero él y los que le quedaron no cesaron de resistir ni siquiera hasta la destrucción total de sus hombres;

\par 4 y no quedó nadie sino él solo; y luchó furiosamente hasta que cayó vencido por las heridas, y fue apresado y conducido ante Gabinio; quien ordenó que lo cuidaran hasta que sanara.

\par 5 Luego lo envió encadenado a Roma.

\par [Y permaneció encerrado en prisión hasta el reinado de César; quien lo sacó de prisión y lo colmó de regalos y favores;

\par 6 y dándole dos generales y doce mil hombres, lo envió a la tierra de Judá, [BC 49.] para separar a los judíos del grupo de Pompeyo y hacerlos obedecer a César; porque Pompeyo en ese tiempo era gobernador de la tierra de Egipto.

\par 7 Y el informe de Aristóbulo y su grupo llegó a Hircano; Éste tuvo mucho miedo y escribió a Antípatro para que le desviara su poder mediante sus habituales artimañas.

\par 8 Entonces Antípatro envió a algunos de los principales de Jerusalén, y le dieron veneno a uno de ellos, encargándole que se lo administrara astutamente a Aristóbulo.

\par 9 Y lo recibieron en la tierra de Siria, como si fueran embajadores de la Ciudad Santa; y él los recibió con alegría, y comieron y bebieron con él.

\par 10 Y aquellos hombres tramaron conspiraciones hasta darle el veneno; y murió, y fue sepultado en tierra de Siria.

\par 11 El tiempo de su reinado, hasta que fue hecho prisionero por primera vez, fue de tres años y medio; y era un hombre de valor, peso y excelente disposición. ]

\par 12 Gabinio había escrito al Senado para que enviara a sus dos hijos a su madre, ya que ella lo había pedido; lo cual hicieron.

\par 13 Pero aconteció que cuando Pompeyo se había alejado de Jerusalén a gran distancia, rompieron su compromiso de obediencia a los romanos:

\par 14 Por eso Gabinio fue contra ellos, los encontró, los venció y los sometió nuevamente a los romanos.

\par 15 Mientras tanto, la tierra de Egipto se rebeló contra Ptolomeo y lo expulsó de su ciudad real, negándose a pagar tributo a los romanos.

\par 16 Entonces Tolomeo escribió a Gabinio pidiéndole que viniera a ayudarlo contra los egipcios, para volver a someterlos a los romanos.

\par 17 Gabinio salió del país de Siria y escribió a Hircano diciéndole que le encontraría con un ejército para ir a Tolomeo.

\par 18 Antípatro fue con un gran ejército a Gabinio y lo encontró en Damasco, felicitándole por la victoria que había obtenido sobre los persas.

\par 19 Gabinio le ordenó que se apresurara a ir a Tolomeo, lo cual hizo, y peleó contra los egipcios, y mató a un gran número de ellos.

\par 20 Después subió Gabinio, reemplazó a Ptolomeo en el trono, regresó a la Ciudad Santa, renovó el poder de Hircano y regresó a Roma.


\chapter{41}

\par \t text{La historia de Craso}

\par 1 Cuando Gabinio regresó a Roma, los persas actuaron en contra de los romanos;

\par 2 Craso marchó con un gran ejército a Siria y llegó a Jerusalén, pidiendo a los sacerdotes que le entregaran todo el dinero que había en la casa de Dios.

\par 3 ¿A quién respondieron: ¿Cómo te será lícito esto, cuando Pompeyo, Gabinio y otros lo han considerado ilícito? Pero él respondió: Debo hacerlo en cualquier caso.

\par 4 Y el sacerdote Eleazar le dijo: Júrame que no pondrás tu mano sobre nada de lo que le pertenece, y te daré trescientas minas de oro.

\par 5 Y le juró que no tomaría nada del tesoro de la casa de Dios si le entregaba lo que le había dicho.

\par 6 Y Eleazar le dio una barra de oro labrado, cuya parte superior había sido insertada en la pared del tesoro del templo, sobre la cual se colocaban cada año los velos viejos de la casa, siendo sustituidos por otros nuevos.

\par 7 Y la barra pesaba trescientas minas de oro, y estaba cubierta con los velos que se habían ido acumulando durante muchos años, sin que nadie la supiera excepto Eleazar.

\par 8 Entonces Craso, habiendo recibido este impedimento, rompió su palabra y se retractó del acuerdo hecho con Eleazar; y tomó todos los tesoros del templo, y saqueó todo el dinero que había en ellos, hasta la cantidad de dos mil talentos.

\par 9 porque este dinero se había acumulado desde la construcción del templo hasta entonces, del botín de los reyes de Judá y de sus ofrendas, y también de los presentes que habían enviado los reyes de las naciones;

\par 10 y se multiplicaron y aumentaron en el transcurso de los años; todo lo que tomó.

\par 11 Entonces el vil Craso se fue con el dinero y su ejército al país de los persas; y lo derrotaron a él y a su ejército en batalla, matándolos en un solo día:

\par 12 Y el ejército persa se llevó todo lo que había en el campamento de Craso.

\par 13 Después de esta hazaña, marcharon hacia Siria, la cual conquistaron, y se separaron de su sometimiento a los romanos.

\par 14 Al enterarse los romanos, enviaron a un general famoso llamado Casio con un gran ejército, quien, al llegar al país de Siria, expulsó a los persas que estaban en él.

\par 15 Luego, dirigiéndose a la Ciudad Santa, libró a Hircano de la guerra que los judíos libraban contra él, reconciliando a las partes.

\par 16 Después, pasando el Éufrates, luchó contra los persas y los hizo volver a someterse a los romanos:

\par 17 ¡También sometió a los veintidós reyes! a quien Pompeyo había sometido; y redujo bajo obediencia a los romanos todo lo que había en los países del este.

\chapter{42}

\par \textit{La historia de César, rey de los romanos}

\par 1 Se cuenta que había en Roma una mujer que estaba embarazada, la cual, estando cerca de dar a luz, y atormentada por los dolores más violentos del parto, murió:

\par 2 pero mientras el niño estaba en movimiento, se abrió el vientre de la madre, y de allí nació, y vivió, y creció, y llamó su nombre Julio, porque nació en el quinto mes; y se llamaba César,

\par 3 porque el vientre de su madre, de donde lo sacaron, fue desgarrado. (Lat. caesa.)

\par 4 Pero cuando el anciano de Roma envió a Pompeyo al este, también envió a César al oeste para someter a algunas naciones que se habían rebelado contra los romanos.

\par 5 Y César fue, los venció, los sometió a la obediencia a los romanos y regresó a Roma con gran gloria.

\par 6 y su fama aumentó, y sus negocios se hicieron muy famosos, y un exceso de orgullo se apoderó de él; por lo que pidió a los romanos que lo nombraran rey.

\par 7 Pero los ancianos y los gobernadores le respondieron: «En verdad, nuestros padres prestaron juramento en los días del rey Tarquino, que había tomado por la fuerza a la esposa de otro hombre, que se había impuesto las manos a sí misma para no poder disfrutar de ella,

\par 8 —que no darían el título de rey a ninguno de los que estuvieran a la cabeza de sus asuntos; por cuyo juramento (dijeron) no podemos satisfaceros en este particular».

\par 9 Por lo tanto, provocó sediciones y libró furiosas batallas en Roma, matando a muchos pueblos, hasta apoderarse del trono de los romanos y hacerse rey, poniéndose una diadema en la cabeza.

\par 10 Desde entonces fueron llamados reyes de los romanos, por su reino: también fueron llamados Césares.

\par 11 Cuando Pompeyo se enteró de que César había matado a trescientos veinte gobernadores, reunió sus ejércitos y marchó hacia Capadocia.

\par 12 César, yendo a su encuentro, lo enfrentó, lo venció y lo mató, y se apoderó de todo el territorio de los romanos.

\par 13 Después de esto, César se fue a la provincia de Siria; a quien Mitrídates el armenio se reunió con su ejército, asegurándole que venía con designios pacíficos y que estaba listo para atacar a cualquier enemigo que le mandara.

\par 14 César le ordenó partir hacia Egipto; y Mitrídates marchó hasta llegar a Ascalón.

\par 15 Hircano temía mucho a César, porque era conocida su sumisión a Pompeyo, a quien César había matado.

\par 16 Por lo tanto, envió rápidamente a Antípatro con un ejército valiente para ayudar a Mitrídates; y Antípatro marchó hacia él y lo ayudó contra cierta ciudad de Egipto, y la tomaron.

\par 17 Pero al salir de allí, encontraron un ejército de judíos que habitaban en Egipto, que estaban haciendo resistencia a la entrada para impedir que Mitrídates entrara en Egipto.

\par 18 Antípatro les mostró una carta de Hircano, ordenándoles que desistieran y no se opusieran a Mitrídates, el amigo de César. Y se abstuvieron.

\par 19 Pero los demás marcharon hasta llegar a la ciudad del entonces rey reinante; quien salió a ellos con todos los ejércitos de los egipcios, y cuando se comprometieron con él, los conquistó y los derrotó;

\par 20 Mitrídates se volvió y huyó; a quien, cuando «estaba rodeado por las tropas egipcias, Antípatro salvó de la muerte:

\par 21 Antípatro y sus hombres no cesaron de resistir en la batalla a los egipcios, a quienes derrotó y venció, y conquistó todo el país de Egipto.

\par 22 Mitrídates escribió a César contándole lo que Antípatro había hecho, las batallas que había soportado y las heridas que había recibido;

\par 23 y que la conquista del país no se debía atribuir a él sino a Antípatro, y que había reducido a los egipcios a la obediencia a César.

\par 24 Y cuando César leyó la carta de Mitrídates, elogió a Antípatro por sus hazañas y resolvió promoverlo y exaltarlo.

\par 25 Después de estos hechos, Mitrídates y Antípatro fueron a César, que entonces estaba en Damasco; y obtuvo de César todo lo que quiso, y le prometió todo lo que quiso.

\chapter{43}

\par \textit{Relato de la llegada de Antígono, hijo de Aristóbulo, a César, quejándose de Antípatro, que había causado la muerte de su padre.}

\par 1 Pero Antígono, hijo de Aristóbulo, fue a César y le contó la expedición de su padre Aristóbulo para atacar a Pompeyo, y lo obediente y servil que se había mostrado con él.

\par 2 Entonces le dijo que Hircano y Antípatro habían enviado en secreto a un hombre a su padre para matarlo con veneno, con la intención (dijo él) de ayudar a Pompeyo contra tus amigos.

\par 3 Entonces César envió a Antípatro y le preguntó sobre este asunto; a quien respondió Antipater;

\par 4 «Ciertamente obedecí a Pompeyo, porque entonces él era el gobernante y me confería beneficios; pero no peleé ahora con los egipcios por Pompeyo, que ya está muerto;

\par 5 ni tuve dificultades para derrotarlos y reducirlos a la obediencia a Pompeyo; pero lo hice por deber hacia César y para poder reducir a Lins a la obediencia a él.

\par 6 Entonces Antípatro se descubrió la cabeza y las manos y dijo: «Estas heridas que tengo en la cabeza y en el cuerpo atestiguan que mi cariño y obediencia a César son mayores que mi cariño y obediencia a Pompeyo;

\par 7 Porque no me expuse en los días de Pompeyo, a las cosas a las que me expuse en los días del rey César.

\par 8 Y César le dijo: «Paz a ti y a todos tus amigos, oh el más valiente de los judíos, porque verdaderamente has mostrado esta fortaleza, magnanimidad, obediencia y afecto hacia nosotros».

\par 9 Y desde entonces César aumentó su afecto hacia Antípatro, lo destacó por encima de todos sus amigos, lo ascendió a general de sus ejércitos y lo llevó consigo al país de los persas.

\par 10 y vio por su valentía y sus exitosas hazañas, que cada vez más despertaba en él un anhelo y afecto por él:

\par 11 finalmente lo hizo regresar a la tierra de Judá, cubierto de honores y coronado con un puesto de autoridad.

\par 12 Y César partió hacia Roma, habiendo arreglado los asuntos de Hircano; Quien construyó las murallas de la Ciudad Santa y se condujo hacia la gente de la manera más excelente:

\par 13 porque era un hombre bueno, dotado de virtudes, de vida intachable, pero su incapacidad en las guerras era notoria para todos los hombres.


\chapter{44}

\par \textit{El relato de la embajada de Hircano a César, pidiendo una renovación del tratado entre ellos; y de la copia del tratado que le envió Hircano.}

\par 1 Por lo tanto, Hircano envió embajadores a César con una carta relativa a la renovación del tratado que había entre él y los romanos.

\par 2 Cuando los embajadores de Hircano llegaron a César, éste les ordenó que se sentaran en su presencia; un honor que no había conferido a ninguno de los embajadores de los reyes que solían acudir a él.

\par 3 Además, se mostró bondadoso con ellos, agilizó sus negocios y ordenó que se respondiera a la carta de Hircano; a quien también escribió el tratado, del cual copiamos a continuación.

\par 4 «Desde César, rey de reyes, hasta los príncipes de los romanos que están en Tiro y Sidón, la paz sea con vosotros.

\par 5 Os hago saber que me han traído una carta de Hircano, hijo de Alejandro, ambos reyes de los judíos;

\par 6 de cuya llegada me alegré, por la continua buena voluntad que tanto él como su pueblo declaran tener hacia mí y la nación romana.

\par 7 Y en verdad, con esto he probado la verdad de sus palabras; que antes envió a Antípatro, capitán de los judíos, y su caballería, con mi amigo Mitrídates, a quien atacaron las tropas de Egipto;

\par 8 y salvó de la muerte a Mitrídates, habiendo conquistado para nosotros el país de Egipto y sometiendo a los egipcios a la obediencia de los romanos; también marchó conmigo al país de los persas, sirviendo como voluntario.

\par 9 Por eso ordeno que todos los habitantes de la costa del mar, desde Gaza hasta Sidón, paguen todos los años todos los tributos que nos deben a la casa del gran Dios que está en Jerusalén;

\par 10 excepto los ciudadanos de Sidón; y éstos le paguen, según la designación de su tributo, veinte mil quinientas cincuenta vibras de trigo cada año.

\par 11 También ordeno que Laodicea y sus posesiones, y todo lo que estaba en manos de los reyes de Judá, hasta la orilla del Éufrates;

\par 12 con todos los lugares que los asmoneos ganaron al pasar el Jordán, serán devueltos a Hircano, hijo de Alejandro, rey de Judá.

\par 13 Porque sus padres habían conquistado todas estas cosas con la espada, pero Pompeyo se las había arrebatado injustamente en tiempos de Aristóbulo:

\par 14 y desde ahora y en adelante serán propiedad de Hircano y de los siguientes reyes de Judá.

\par 15 Y este tratado es para mí y para cada uno de los reyes de Roma, mis sucesores: cualquiera que lo rompa, o cualquier parte de él, que Dios lo destruya a espada, y que su casa y su gobierno sean reparados. ¡Desolado y cortado!

\par 16 Y cuando leas esta mi epístola, escríbela con letras grabadas en tablas de bronce, en lengua romana y en sus caracteres, y en lengua griega y en sus caracteres:

\par 17 y colocarán las mesas en los lugares visibles de los templos que están en Tiro y Sidón; para que todos puedan verlos y comprender lo que he designado para «Hircano y los judíos».

\chapter{45}

\par \textit{La historia de la muerte de César}

\par 1 Estaban con César dos amigos de Pompeyo; De los cuales uno se llamaba Casio y el otro Bruto; quien tramó un complot para matar a César.

\par 2 ¿Con qué propósito se escondieron en el templo? en Roma, donde él mismo se había apartado para orar.

\par 3 Cuando llegó, descuidado, seguro y sin preocuparse, se abalanzaron sobre él y lo mataron.

\par 4 Y Casio tomó posesión del trono, reunió un gran ejército y lo transportó más allá del mar; temiendo al partido de César si continuaba residiendo en Roma.

\par 5 Y marchó hacia la tierra de Asia, y la devastó; de allí pasó al país de Judá.

\par 6 y Antípatro quiso atacarlo; pero viendo que sus fuerzas no estaban a la altura de la tarea, hizo las paces con él.

\par 7 Y Casio impuso un tributo de setecientos talentos de oro en la tierra de Judá; y Antípatro se comprometió como fiador del dinero;

\par 8 y encargó a su hijo Herodes que lo levantara en el país de Judá y se lo llevara a Casio, quien, al recibirlo, marchó al país de Macedonia y permaneció allí por miedo a los romanos.

\chapter{46}

\par \textit{La historia de la muerte de Antípatro}

\par 1 Los príncipes de Judá habían decidido matar a Antípatro; y para ello había se| Engañadamente puso sobre él a un hombre que se llamaba Malquías.

\par 2 Y Malquías lo intentó, pero su ejecución se retrasó mucho tiempo.

\par 3 Y la noticia llegó a Antípater, quien buscó a Malquías para matarlo.

\par 4 Pero Malquías se aclaró ante Antípatro de las cosas que le habían acusado; y le juró que la noticia era infundada; y Antípatro le creyó, apartando de él toda sospecha.

\par 5 Pero Malquías, después de haber dado una gran suma de dinero al copero de Hircano, acordó con él darle veneno a Antípatro mientras él estaba en el banquete en presencia del rey.

\par 6 Y el copero hizo esto, y el rey Antípater murió aquel mismo día, sin que el rey lo supiera ni lo hubiera planeado. Y cuando Antípatro murió, Hircano sustituyó a Malquías en su lugar.

\chapter{47}

\par \textit{La historia de la muerte de Malquías}

\par 1 Cuando Herodes, hijo de Antípatro, fue informado de que Malquías había causado la muerte de su padre, pensó en atacar abiertamente a Malquías; pero su hermano se lo impidió, aconsejándole que se lo llevaran mediante una estratagema.

\par 2 Entonces Herodes fue a ver a Casio y le contó lo que había hecho Malquías; a lo cual el otro respondió: Cuando yo vaya a Tiro, e Hircano esté conmigo, y con él Malquías, entonces apresúrate contra él y mátalo.

\par 3 Cuando Casio fue a Tiro e Hircano fue a reunirse con él, llevando consigo a Malquías; y estaban juntos en presencia de Casio, en cierta fiesta a la que Casio los había invitado con todos sus amigos:

\par 4 (ahora Casio había dado órdenes a sus siervos para que hicieran todo lo que Herodes les ordenara:)

\par 5 También Herodes estaba con su hermano entre los compañeros de Hircano, y Herodes acordó con algunos de los sirvientes matar a Malquías, cuando se les diera una señal con un guiño de ojos.

\par 6 Cuando Hircano hubo comido y bebido con sus amigos, se fueron a dormir por la tarde.

\par 7 Y cuando despertaron del sueño, Hircano ordenó que le prepararan un lecho al aire libre, ante la entrada de la sala del banquete en la que habían dormido:

\par 8 Y él mismo se sentó y ordenó a Malquías que se sentara con él, y también ordenó que se sentaran Herodes y su hermano.

\par 9 y los sirvientes de Casio estaban cerca de Hircano; a quien Herodes le guiñó un ojo contra Malquías, e inmediatamente se abalanzaron sobre él y lo mataron:

\par 10 Hircano se asustó mucho y se desmayó.

\par 11 Pero cuando los servidores de Casio se retiraron y sacaron a Malquías asesinado, Hircano volvió en sí y preguntó a Herodes la causa de la muerte de Malquías.

\par 12 Y Herodes respondió: «Soy completamente ignorante, ni conozco la «causa de la cosa». E Hircano guardó silencio y nunca volvió a preguntar más sobre el asunto.

\par 13 Y Casio marchó a Macedonia para encontrarse con Octaviano, hijo del hermano de César, y Antonio, general de su ejército, pues habían partido de Roma con un gran ejército en busca de Casio.

\chapter{48}

\par \textit{La historia de Octaviano (el mismo es Augusto, hijo del hermano de César) y de Antonio, general de su ejército, y de la muerte de Casio.}

\par 1 Cuando Octavio entró en Macedonia, Casio salió a su encuentro y se enfrentó a él; y Casio fue puesto en fuga;

\par 2 a quien Octavio, persiguiendo, lo derrotó y mató por completo; y Octavio ganó el reino en lugar de su tío César; y también le pusieron por sobrenombre César, por el nombre de su tío.

\par 3 Cuando Hircano supo la muerte de Casio, envió embajadores con regalos, dinero y joyas a Augusto y Antonio:

\par 4 y le escribió pidiéndole que renovara el tratado que había hecho con César;

\par 5 y que ordenaría que todos los cautivos de Judá que estaban en su reino, y los que habían sido cautivos en los días de Casio, fueran liberados;

\par 6 y que permitiría a todos los judíos que estaban en el país de los griegos y en la tierra de Asia regresar al país de Judá,

\par 7 sin exigir ningún rescate, ni redención, ni ningún obstáculo que nadie ponga en el camino.

\par 8 Cuando los embajadores de Hircano llegaron a Augusto con sus cartas y regalos, él los honró,

\par 9 y aceptó los regalos y accedió a todo lo que Hircano le había pedido; escribiéndole una carta, de la cual ésta es la copia.

\par 10 «Desde Augusto, rey de reyes, y Antonio su colega, hasta Hircano, rey de Judá; La salud sea para ti.

\par 11 Ahora nos ha llegado tu carta, por la cual nos regocijamos; y hemos enviado lo que quisiste, respetando la renovación del tratado y la escritura, a todas nuestras provincias, que se extienden desde el país de las Indias hasta el océano occidental.

\par 12 Pero lo que nos retrasó en escribiros antes acerca de la renovación del tratado fue nuestra ocupación de someter a Casio, ese inmundo tirano;

\par 13 quienes, actuando malvadamente contra César,

\par 14 Por eso hemos luchado contra él con todas nuestras fuerzas, hasta que el Dios grande y bueno nos hizo vencedores y lo hizo caer en nuestras manos;

\par 15 a quienes hemos matado. También hemos matado a Bruto, su colega; y hemos librado de su mano la tierra de Asia, después de que la asoló y exterminó a sus habitantes.

\par 16 Tampoco aceptó ningún compromiso; ni honrar ningún templo; ni hacer justicia a los oprimidos; ni compadecer a un judío, o cualquier otro de nuestros súbditos:

\par 17 pero con sus seguidores hizo perversamente muchos males a todos los hombres mediante la opresión y la tiranía:

\par 18 Por eso Dios ha vuelto su malicia sobre sus cabezas, entregándolos junto con sus confederados.

\par 19 Ahora, pues, alegraos, rey Hircano, y los demás judíos, los habitantes de la Región Santa y los sacerdotes que estáis en el templo de Jerusalén.

\par 20 y que acepten el presente que hemos enviado al templo más glorioso y oren por Augusto.

\par 21 También hemos escrito a todas nuestras provincias, que en ellas no quede ningún judío, sea siervo o sirvienta, que no sea despedido a todos, sin precio ni rescate.

\par 22 y que nadie les impidiera regresar a la tierra de Judá; y esto por orden de Augusto, y también de Antonio su colega».

\par 23 Además, escribió a sus amigos que estaban en Tiro, en Sidón y en otros lugares, para que devolvieran todo lo que habían tomado de la tierra de Judá en los días del inmundo Casio:

\par 24 y tratar pacíficamente a los judíos, y no oponerse a ellos en nada, y hacer por ellos todo lo que César había decretado en su tratado con ellos.

\par 25 Antonio permaneció en el país de Siria; y vino a él Cleopatra reina de Egipto, a la cual tomó por mujer.

\par 26 Ella era una mujer sabia, experta en artes mágicas y propiedades de las cosas: de modo que lo sedujo y se apoderó de su corazón hasta tal punto que él no podía negarle nada.

\par 27 En aquel mismo tiempo, cien hombres de los principales judíos fueron a ver a Antonio y se quejaron de Herodes y de su hermano Faselo, hijos de Antípatro, diciendo:

\par 28 Ahora se han apoderado de todo lo que pertenece a Hircano, y del reino no le queda nada excepto el nombre; y el ocultamiento de este asunto es prueba del cautiverio de su señor.

\par 29 Pero cuando Antonio preguntó a Hircano la verdad de lo que le habían dicho, Hircano declaró que hablaban mentira; limpiando a Herodes y a su hermano de lo que les habían imputado.

\par 30 Y Antonio se alegró de esto; porque él estaba muy inclinado hacia ellos y los amaba.

\par 31 En otra ocasión, otras personas se quejaron ante él de Herodes y de su hermano, cuando estaba en Tiro:

\par 32 Pero él no sólo se negó a aceptar sus palabras, sino que a algunos los mató y a los demás los metió en la cárcel.

\par 33 y engrandeció a Herodes y a su hermano, prestándoles servicios, y los envió de regreso a Jerusalén con grandes honores. Pero el propio Antonio; entró en el país de los persas, los derrotó, los sometió y regresó a Roma.

\chapter{49}

\par \textit{La historia de Antígono, hijo de Aristóbulo, y de su capedición contra su tío Hircano, y del socorro que obtuvo del rey de los persas.}

\par 1 Cuando Augusto y Antonio llegaron a Roma, Antígono fue al rey de los persas y le prometió mil talentos en oro acuñado y ochocientas vírgenes de las hijas de Judá y de sus príncipes, hermosas y sabias;

\par 2 si enviara con él un general al frente de un gran ejército contra Jerusalén, y le ordenara que lo nombrara rey de Judá, y tomara prisionero a su tío Hircano y matara a Herodes y a su hermano.

\par 3 A lo cual el rey aceptó y envió con él un general con un gran ejército:

\par 4 Y marcharon hasta llegar a la tierra de Siria; y mataron a un amigo de Antonio y a ciertos romanos que allí habitaban.

\par 5 Desde allí marcharon contra la Ciudad Santa; profesando seguridad y paz, y que Antígono sólo había venido a orar en el santuario, y luego regresaría con sus propios amigos.

\par 6 Y entraron en la ciudad; Cuando entraron, hicieron malas prácticas y comenzaron a matar hombres y a saquear la ciudad, conforme a las órdenes que les había dado el rey de Persia.

\par 7 Herodes y sus hombres corrieron a defender el palacio de Hircano, pero él envió a su hermano y le ordenó que vigilara el camino que conduce desde las murallas al palacio.

\par 8 Y cuando se hubo apoderado de cada posición, escogió algunos de sus hombres y marchó contra los persas que estaban en la ciudad;

\par 9 y su hermano lo siguió con un cierto número de sus hombres; y mataron a la mayor parte de los persas que estaban en la ciudad, pero el resto huyó de la ciudad.

\par 10 Y cuando el general de los persas vio que las cosas no se le habían ocurrido, envió mensajeros a Herodes y a su hermano para tratar de paz;

\par 11 informándoles que ahora estaba satisfecho de su valor y valentía, y que debían ser preferidos a Antígono; y que por esa razón persuadiría a sus tropas para que ayudaran a Hircano y a ellos en lugar de a Antígono:

\par 12 y confirmó este deseo con los más solemnes juramentos, de modo que Hircano y Faselo le creyeron, pero no Herodes.

\par 13 Hircano y Faselo, acudiendo al general de los persas, le manifestaron su confianza en él; y les aconsejó que fueran a ver a su colega que estaba en Damasco; y se fueron.

\par 14 Y cuando vinieron a él, los recibió honorablemente, hizo alarde de tenerlos en gran estima y los trató con cortesía; aunque en secreto había dado órdenes de que los hicieran prisioneros.

\par 15 Y viniendo a ellos algunos de los principales hombres del país, les contaron este mismo plan; aconsejándoles que huyan, con la promesa de ayudarlos a escapar.

\par 16 Pero ellos no confiaban en estos hombres, temiendo que se tratara de algún complot contra ellos; por lo que se quedaron.

\par 17 Y cuando llegó la noche, fueron apresados: Faselo efectivamente se impuso sus manos; pero Hircano fue encadenado y por orden del general de los persas le cortaron la oreja para no volver a ser sumo sacerdote nunca más;

\par 18 y lo envió a Herak, al rey de los persas; Cuando llegó a él, el rey ordenó que le quitaran las cadenas y le mostró bondad;

\par 19 Y permaneció en Herak cargado de honores, hasta que Herodes lo demandó del rey de los persas; y cuando fue devuelto a Herodes, le sucedieron las cosas que le sucedieron.

\par 20 Después de esto, el general subió con Antígono a la Ciudad Santa, y le contaron a Herodes lo que le habían hecho. Hircano y Phaselus:

\par 21 Entonces, tomando a su madre Cipris, a su esposa Mariamna, hija de Aristóbulo, y a su madre Alejandra, las envió con caballos y mucho equipaje a José, su hermano, para que montara a Sara.

\par 22 Pero él, con un ejército de mil hombres, marchaba lentamente y esperaba a los persas que intentaran perseguirlo.

\par 23 Y el general de los persas lo persiguió con su ejército; a quien Herodes atacó, venció y puso en fuga.

\par 24 Después de esto, las tropas de Antígono también lo persiguieron y pelearon con él encarnizadamente; y a éstos los derrotó y mató a muchos de ellos.

\par 25 Luego marchó hacia las montañas de Sara; y encontró a su hermano Josefo, a quien ordenó que asegurara a las familias en un lugar seguro y les proporcionara todas las cosas necesarias:

\par 26 Y les dio mucho dinero para que, en caso de necesidad, pudieran comprarse provisiones.

\par 27 Y dejando a sus hombres con su hermano Josefo, él mismo con algunos compañeros se fue a Egipto para tomar un barco y partir hacia el país de los romanos.

\par 28 Cleopatra lo atendió cortésmente y le pidió que tomara el mando de sus ejércitos y la gestión de todos sus asuntos; a quien notificó que era muy necesario que fuera a Roma.

\par 29 Ella le dio dinero y barcos, y él fue hasta Roma, se quedó con Antonio y le contó lo que Antígono había hecho y lo que había cometido contra Hircano y su hermano, con la ayuda del rey de Persas:

\par 30 Antonio fue con él a Augusto y al Senado y les dijo lo mismo.

\chapter{50}

\par \textit{La historia de Herodes cuando los romanos lo nombraron rey sobre los judíos, y su salida de Roma con un ejército para luchar contra la Santa Casa.}

\par 1 Agusto y el Senado, informados de lo que había hecho Antígono, de común acuerdo nombraron a Herodes rey sobre los judíos;

\par 2 mandándole que se pusiera una diadema de oro en la cabeza y montara en un caballo, y que proclamaran con trompetas delante de él: «Herodes es rey sobre los judíos y sobre la ciudad santa de Jerusalén», lo cual se hizo.

\par 3 Y volviendo a Augusto, cabalgó, y Augusto y Antonio; y fueron a casa de Antonio, que había invitado al Senado y a todos los ciudadanos de Roma a un banquete que él había preparado.

\par 4 Y comieron y bebieron, y se regocijaron con gran alegría por Herodes, habiendo firmado con él un tratado grabado en tablas de bronce; y fue colocado en los templos.

\par 5 Y escribieron ese día como el primero del reinado de Herodes, y desde entonces fue tomado como un zera, por el cual se cuentan los tiempos.

\par 6 Después de esto, Antonio y Herodes partieron por mar con un ejército grande y abundante, y cuando llegaron a Antioquía, dividieron sus fuerzas:

\par 7 y Antonio tomó una parte y la condujo al país de los persas, que es Herak» y las partes adyacentes; y Herodes, tomando otra parte, fue derecho hasta llegar a Tolemaida.

\par 8 Entonces Antígono, al enterarse de que Antonio había hecho una expedición al país de los persas y que Herodes había llegado a Tolemaida, salió de la Santa Casa al monte Sara para tomar a Josefo, el hermano de Herodes, y a los que estaban allí con él.

\par 9 a quienes atacó y sitió; y habiendo cortado un canal, interceptaron el agua que fluía hacia ellos: de modo que prevaleció la sed entre ellos, y sus asuntos se redujeron a grandes dificultades.

\par 10 Por lo cual Josefo decidió huir; y las familias habían deliberado sobre entregarse a Antígono, si Josefo huía.

\par 11 Pero Dios les envió una lluvia abundante, que llenó todas sus cisternas y vasos; por lo que sus corazones se animaron y su condición mejoró;

\par 12 Josefo continuó rechazando a Antonio y a sus hombres de la fortaleza, sin que éstos pudieran obtener ninguna ventaja sobre él.

\par 13 Pero Herodes se dirigió directamente al monte Sara para traer a su hermano, a sus familias y a los hombres que estaban con él a Jerusalem.

\par 14 Y encontró a Antígono sitiando a su hermano; sobre quien hizo un ataque repentino; y Josefo y sus hombres salieron hacia ellos, y la mayor parte del ejército de Antígono fue destruido, y él huyó a Jerusalén.

\par 15 A quien Herodes persiguió con un gran ejército de judíos, que habían venido a él de todas partes, cuando descubrieron que había regresado; y estaba bien provisto de ayuda, de modo que tenía menos necesidad del ejército de los romanos.

\par 16 Cuando Herodes llegó a la Ciudad Santa, Antígono le cerró las puertas en la cara; y luchó contra él; y envió mucho dinero a los jefes del ejército de los romanos, pidiéndoles que no ayudaran a Herodes, lo cual hicieron por él.

\par 17 Por lo cual la guerra duró mucho tiempo entre Antígono y Herodes, sin que ninguno de los dos prevaleciera sobre su compañero [es decir, el antagonista].

\chapter{51}

\par \textit{La historia de la magnanimidad de algunos de los hombres de Herodes y de su valentía.}

\par 1 Ahora bien, en tiempos de Antígono se habían multiplicado los ladrones y los que codiciaban la propiedad ajena;

\par 2 y se dirigieron a unas cuevas en las montañas, a las que sólo podía acceder un hombre a la vez, por ciertos lugares preparados para ello por ellos y que sólo ellos conocían.

\par 3 y aunque otros los conocieran, no pudieron subir a la cueva; porque siempre había un hombre listo en la boca que, con muy poco esfuerzo, podía fácilmente repeler a una persona que subía.

\par 4 Y ahora algunos de estos hombres habían conseguido en esa cueva abundancia de armas, provisiones y bebida, y todo lo que necesitaban;

\par 5 junto con todo el botín que habían obtenido atacando a quienes encontraron, y lo que habían tomado con razón o sin ella.

\par 6 Cuando, pues, Herodes se enteró de sus procedimientos y descubrió que era probable que sus asuntos causaran demora»; También que los hombres no podían actualmente subir a ellos por escaleras, ni de hecho subir de ninguna manera:

\par 7 Hizo uso de grandes cofres de madera, ensamblados y ensamblados, y los llenó con hombres, añadiendo comida y agua, y que llevaban lanzas larguísimas y ganchudas:

\par 8 y ordenó que bajaran aquellos cofres desde la cima de las montañas, en medio de las cuales estaban las cuevas, hasta colocarlos frente a sus bocas:

\par 9 y cuando se encontraron frente a ellos, pidió que sus hombres los atacaran cuerpo a cuerpo con espadas, y que desde lejos los arrastraran con aquellas lanzas.

\par 10 Y se hicieron los cofres y se llenaron de hombres.

\par 11 Y cuando algunos de ellos fueron bajados y se encontraban frente a las entradas de aquellas cuevas, sin que se hubiera dado información a los que allí vivían, uno de los hombres que estaban en los cofres se precipitó hacia las cuevas, seguido de sus compañeros;

\par 12 y mataron a los ladrones que estaban dentro de ellos y a sus seguidores, y los arrojaron abajo en los valles; todos los hombres que Herodes había enviado, emulando a estos jist.

\par 13 Y en esta hazaña, su coraje, su valentía y su audacia fueron tan notorios que nunca se vio nada igual: y exterminaron por completo a los ladrones de todas aquellas partes.

\chapter{52}

\par \textit{Un relato del regreso de Antonio del país de los persas después de matar al rey de los persas, y su encuentro con Herodes.}

\par 1 Entonces Antonio, dejando a Herodes, marchó de Antioquía al país de los persas, y peleó con el rey de los persas, lo venció, lo mató y conquistó su tierra;

\par 2 y, habiendo obligado a los persas a obedecer a los romanos, se desvió hacia el Éufrates.

\par 3 Cuando Herodes conoció su fama, salió a felicitarlo por su victoria; y pedirle que venga con él al País Santo.

\par 4 Y encontró reunida una gran multitud que deseaba acercarse a Antonio; a lo que se habían opuesto muchos grupos de árabes, impidiéndole llegar a la presencia de Antonio.

\par 5 Entonces Herodes avanzó contra los árabes y los mató, abriendo paso a todos los que quisieran acercarse a Antonio.

\par 6 Esto fue informado a Antonio antes de que llegara Herodes, y le envió una diadema de oro y muchos caballos.

\par 7 Pero cuando llegó Herodes, Antonio lo recibió cortésmente y lo elogió por sus hazañas contra los árabes; y le acompañó a Sosio, general de su ejército, con un gran ejército, y le ordenó que lo acompañara a la ciudad de Santa Casa:

\par 8 dándole también cartas para toda la tierra de Siria, desde Damasco hasta el Éufrates, y desde el Éufrates hasta la tierra de Armenia;

\par 9 diciéndoles: Augusto, rey de reyes, Antonio, su colega, y el Senado romano han nombrado ahora a Herodes rey sobre los judíos; y desean que encabeces a todos «tus hombres de guerra con Herodes para ayudarlo: si «por tanto actúas en contra de esto, debes ir a la guerra con nosotros».

\par 10 Entonces Antonio marchó hacia la costa del mar y de allí a Egipto; pero Herodes y Sosio con su ejército comandaban las fuerzas de Siria.

\par 11 Pero cuando Herodes llegó cerca de Damasco, descubrió que su hermano Josefo había salido de la Santa Casa con un ejército de romanos para sitiar Jericó y cortar su trigo.

\par 12 Contra él salió Pappus, general de las tropas de Antígono, y mató a treinta mil de ellos, habiendo matado también al hermano de Josefo Herodes.

\par 13 Y cuando su cabeza fue presentada a Antígono, su hermano Feroras la compró por quinientos talentos y la enterró en el sepulcro de sus padres.

\par 14 y oyó también que Antígono y Pappus avanzaban contra él con un gran ejército.

\par 15 Herodes, enterado de lo cual, decidió atacar a Antígono y aplastarlo inesperadamente.

\par 16 y acordó con Sosio que tomaría doce mil romanos y veinte mil judíos y marcharía contra Antígono, pero que el otro seguiría lentamente sus pasos con el resto del ejército.

\par 17 Entonces Herodes marchó con sus tropas en grupo y se encontró con Antígono en las montañas de Galilea, y pelearon con él desde el mediodía hasta la noche.

\par 18 Entonces el ejército se dispersó; y Herodes con algunos de sus hombres pasó la noche en cierta casa, y la casa cayó sobre ellos; pero todos escaparon de la ruina con vida, sin que a ninguno de ellos se le rompiera un hueso.

\par 19 Poco después, Herodes se apresuró a pelear con Antígono, y hubo una gran batalla entre ellos, y Antígono huyó a la Santa Casa; Mientras tanto, Pappus resistió valientemente y continuó la lucha, porque era muy animado y muy valiente.

\par 20 Aquel día murió la mayor parte del ejército de Antígono; También fue asesinado Pappus, a quien Feroras le cortó la cabeza, y la llevaron a Herodes, quien ordenó que la enterraran.

\par 21 Como no quedó nadie del ejército de Antígono, excepto prisioneros o fugitivos, Herodes ordenó a sus hombres que descansaran, comieran y bebieran.

\par 22 Pero él mismo fue a un baño que estaba en la ciudad vecina, y entró en el baño desarmado.

\par 23 Ahora yacían escondidos en el baño tres hombres fuertes y valientes, con espadas desenvainadas en las manos; los cuales, cuando lo vieron entrar desarmado en el baño, se apresuraron a salir uno tras otro, por miedo de él; y así escapó.

\par 24 Después de esto vino Sosio; y marcharon juntos a la ciudad de la Santa Casa, la cual rodearon con una trinchera; Y entre ellos y Antígono se produjeron feroces batallas:

\par 25 y muchos de los hombres de Sosio fueron asesinados, y Antígono los vencía frecuentemente; pero no pudo hacerlos huir a causa de su firmeza y resistencia para soportar los ataques.

\par 26 Entonces Herodes venció a Antígono; y Antígono huyó, y entrando en la ciudad cerró las puertas contra Herodes, y Herodes lo sitió por mucho tiempo.

\par 27 Pero una noche los guardias de la puerta se quedaron dormidos; al descubrirlo algunos de los hombres de Herodes, veinte de ellos corrieron, tomaron unas escaleras, las colocaron contra la pared y, subiendo a ellas, mataron a los guardias.

\par 28 Entonces Herodes se apresuró con sus hombres a la puerta de la ciudad que estaba frente a ellos, la irrumpió y entró en la ciudad.

\par 29 Los romanos, tomando esto, comenzaron a masacrar a los ciudadanos; Ante lo cual Herodes, turbado, dijo a Sosio: Si destruyes a todo mi pueblo, ¿sobre quién me nombrarás rey?

\par 30 y Sosio ordenó que se hiciera proclamar la espada debería permanecer; ni ninguna persona fue asesinada después de la proclamación.

\par 31 Pero los capitanes de Sosio, ávidos de presa, corrieron a saquear la casa de Dios; pero Herodes, que estaba a la puerta, con una espada desenvainada en la mano, se lo impidió; y envió a Sosio para contener a sus hombres, prometiéndoles dinero.

\par 32 Entonces Sosio ordenó que se hiciera proclamar a sus hombres que se abstuvieran de saquear, y ellos se abstuvieron. Y buscaron a Antígono y lo encontraron, y Antígono fue hecho prisionero.

\par 33 Después de esto, Sosio se dirigió a Egipto, donde su colega Antonio, llevando consigo a Antígono encadenado.

\par 34 Pero Herodes envió a Antonio un regalo muy grande y hermoso, pidiéndole que matara a Antígono; y Antonio lo mató; y esto fue en el tercer año del reinado de Herodes, que también era el tercer año de Antígono.

\chapter{53}

\par \textit{La historia de Herodes después de la muerte de Antígono}

\par 1 Cuando Herodes se enteró de la muerte de Antígono, se sintió seguro de que ningún miembro de la familia real de Asmonzan contendería con él:

\par 2 Por lo tanto, se dedicó a «promover las dignidades, en bondades y promociones, de aquellos que estaban bien inclinados hacia él y obedecían su voluntad».

\par 3 También se esforzó en destruir a aquellos que se le habían opuesto junto con sus familias, y en saquear su ganado y sus bienes, prestándole ayuda contra él.

\par 4 Y oprimió a la gente, quitándoles sus propiedades y despojando a todos los que habían dejado de obedecer a los judíos; y mató a los que se le resistieron, y saqueó sus bienes.

\par 5 También hizo un acuerdo con todos los que le obedecían, de que le pagarían dinero.

\par 6 También puso guardias a las puertas de la Santa Casa, quienes podían registrar a los que salían, tomar el oro o la plata que encontraran sobre cada uno y llevárselo.

\par 7 También ordenó que se registraran los ataúdes de los muertos; y cualquier dinero que cualquier persona pueda intentar llevar a cabo mediante estratagema, será tomado.

\par 8 Y reunió tanto dinero como ninguno de los reyes de la segunda casa había amasado.

\chapter{54}

\par \textit{Historia de Hircano, hijo de Alejandro, tío de Antígono, y de su regreso a Jerusalén a petición de Herodes, y de la muerte que le impuso.}

\par 1 Hircano, después de que el rey de los persas lo puso en libertad, permaneció en Herakin en una condición muy respetable y con grandes honores.

\par 2 Por eso Herodes temió que algo pudiera inducir al rey de los persas a nombrarlo rey y enviarlo a la tierra de Judá.

\par 3 Por lo tanto, deseando tranquilizar su mente, trazó planes para este negocio; y envió al rey de los persas un regalo muy grande y una carta;

\par 4 en el que hizo mención de los méritos y bondades de Hircano hacia él; y cómo había ido a Roma a causa de lo que le había hecho el hijo de su hermano Antígono;

\par 5 y que, habiendo alcanzado ya el trono y estando sus asuntos en orden, quería recompensarle adecuadamente por los beneficios que le había concedido.

\par 6 Entonces el rey de los persas envió un mensajero a Hircano, diciendo: «Si quieres volver a la tierra de Judá, vuélvete:

\par 7 pero os advierto que «cuidado con Herodes; y os informo claramente, que él no busca que os hagáis ningún bien», sino que su designio es estar seguro, ya ​​que no queda nadie a quien teme, excepto vosotros: por tanto, tened cuidado de él muy diligentemente con cuidado, y no os dejéis llevar por lazo».

\par 8 También se le acercaron los judíos de Babilonia y le dijeron lo mismo. Nuevamente le dicen:

\par 9 Ahora eres un hombre viejo y no apto para ejercer el oficio de sumo sacerdote a causa de la mancha que te hizo tu sobrino:

\par 10 pero Herodes es un hombre malo y derramador de sangre; y sólo os recuerda porque os teme; y no os falta nada entre nosotros, y estáis con nosotros en aquel puesto en que conviene estar.

\par 11 Y allí tu familia está en las mejores condiciones; Por tanto, quédate con nosotros y no ayudes a tu enemigo contra ti mismo.

\par 12 Pero Hircano no accedió a sus palabras; ni escuchó el consejo de quien bien le aconsejaba.

\par 13 Y partió y caminó hasta llegar a la Ciudad Santa, debido al gran anhelo que tenía por la casa de Dios, su familia y su país.

\par 14 Cuando llegó cerca de la ciudad, Herodes le salió al encuentro, mostrando tal honor y magnificencia, que Hircano se engañó y confió en él.

\par 15 Y Herodes en la asamblea pública y delante de sus amigos solía llamarlo «Padre», pero sin embargo no cesaba de idear conspiraciones en su corazón, sólo para que no le fueran imputadas.

\par 16 Por eso Alejandra y su hija Mariamna van a ver a Hircano, infundiéndole miedo a Herodes y aconsejándole que se cuide;

\par 17 Pero tampoco asistió a ellos, aunque le repitieron esto una y otra vez, aconsejándole que huyera donde alguno de los reyes de Arabia:

\par 18 Sin embargo, no prestó atención a todas estas cosas, hasta que lo obligaron a hacerlo con repetidas advertencias y alarmas.

\par 19 Entonces escribió al rey de Arabia: y llamando a cierto hombre (cuyo hermano Herodes había matado, confiscado sus bienes y le había causado muchos males), le dijo que quería comunicarle cierto secreto, rogándole que no lo dijera a cualquiera;

\par 20 y entregándole dinero y la carta para el rey de Arabia, le comunicó lo que pedía en la carta.

\par 21 Entonces el mensajero, al recibir la carta, pensó que, si le comunicaba el asunto a Herodes, conseguiría un alto cargo ante Herodes y se libraría del mal que continuamente temía de sus manos;

\par 22 y que esto le sería más provechoso que guardar el secreto de Hircano; ya que en el otro caso no estaba seguro, y seguro de que el asunto no sería revelado a Herodes en un momento u otro, y así ser la causa de su destrucción.

\par 23 Entonces llevó la carta a Herodes y le contó todo el asunto, quien le dijo: Lleva la carta tal como está al rey de Arabia y tráeme su respuesta, para que sepa. él:

\par 24 Dime también el lugar donde estarán los hombres que enviará el rey de Arabia para que Hircano regrese con ellos.

\par 25 Entonces el mensajero fue y llevó la carta de Hircano al rey de Arabia; quien se alegró y envió algunos de sus hombres;

\par 26 les ordenó que fueran a un lugar cercano a la Ciudad Santa y esperaran allí hasta que Hircano viniera a ellos; y luego acompañar a Hircano hasta que lo trajeran a su presencia.

\par 27 También escribió a Hircano una respuesta a su carta y la envió por medio de un mensajero.

\par 28 Entonces los hombres se dirigieron con el mensajero al lugar señalado y allí esperaron; pero el mensajero llevó la carta a Herodes, quien supo su contenido y le dijo también el lugar de los hombres a quienes Herodes había enviado personas para que los llevaran.

\par 29 Después enviaron a buscar a setenta ancianos de los ancianos de los judíos y también a Hircano; Cuando llegó, le dijo: ¿Hay algún intercambio de cartas entre tú y el rey de Arabia?

\par 30 Hircano respondió: No. Entonces le dijo: ¿Enviaste para huir a él? y él dijo: No.

\par 31 Entonces Herodes ordenó a su mensajero que se acercara, junto con los árabes y los caballos; También sacó la respuesta a su carta, y fue leída.

\par 32 Entonces ordenó que le cortaran la cabeza a Hircano; y le cortaron la cabeza, y nadie se atrevía a decir palabra por él.

\par 33 Hircano libró a Herodios de la muerte que justamente le había sido concedida en la asamblea del juicio, ordenando que la asamblea se aplazara hasta el día siguiente y despidiendo a Herodes esa misma noche.

\par 34 De donde estaba destinado a convertirse en su asesino, independientemente de los servicios que le había prestado a él y a su padre.

\par 35 Hircano fue ejecutado cuando tenía ochenta años y reinó cuarenta años; y ninguno de los reyes de la raza asmonza tuvo una conducta más loable ni una forma de vida más honorable.

\chapter{55}

\par \textit{La historia de Aristóbulo, hijo de Hircano}

\par 1 Aristóbulo, hijo de Hircano, era de tal belleza de figura, de tal figura y entendimiento exquisito, que no se conocía a nadie igual.

\par 2 También su hermana Mariamna, esposa de Herodes, era igual a él en belleza; y Herodes estaba maravillosamente apegado a ella.

\par 3 Pero Herodes se resistía a nombrar a Aristóbulo sumo sacerdote en lugar de su padre; no sea que los judíos, estando apegados a él por el afecto que sienten hacia su padre, en algún momento futuro lo hagan rey.

\par 4 Por lo cual nombró sumo sacerdote a uno de entre los sacerdotes comunes, que no era de la familia de los asmoneos.

\par 5 Ante lo cual Alejandra, la madre de Aristóbulo, enojada, escribió a Cleopatra; solicitando una carta de Antonio para Herodes, pidiéndole que destituyera al sacerdote que había elevado y nombrara en su lugar a su hijo Aristóbulo sumo sacerdote.

\par 6 Y Cleopatra lo concedió; y pidió a Antonio que escribiera una carta a Herodes sobre este tema y que se la enviara por medio de algún jefe de sus servidores.

\par 7 Entonces Antonio escribió una carta y la envió por su siervo Gelio; y llegando Gelio a Herodes, le entregó la carta de Antonio.

\par 8 Pero Herodes se abstuvo de hacer lo que Antonio había ordenado por escrito, afirmando que no era costumbre entre los judíos destituir a ningún sacerdote de su cargo.

\par 9 Sucedió que Gelio vio a Aristóbulo y quedó muy impresionado por la belleza de su forma y la perfección de su porte que vio.

\par 10 Por lo cual pintó un cuadro a su semejanza y se lo envió a Antonio, escribiendo así debajo del cuadro; que ningún hombre había engendrado a Aristóbulo, sino que un ángel que convivía con Alejandra lo engendró de ella.

\par 11 Por lo tanto, cuando Antonio recibió el cuadro, sintió un deseo vehemente de ver a Aristóbulo.

\par 12 Y escribió una carta a Herodes, recordándole cómo lo había nombrado rey y cómo lo había ayudado contra sus enemigos, contándole sus bondades hacia él:

\par 13 añadiendo la petición de que le enviara a Aristóbulo; y lo amenazó en este negocio por las palabras que le había enviado.

\par 14 Pero cuando le llevaron la epístola de Antonio a Herodes, éste se negó a enviar a Aristóbulo, sabiendo lo que Antonio planeaba; y por eso desdeñó hacerlo, y rápidamente depuso al sumo sacerdote que había nombrado, estableciendo a Aristóbulo en su lugar.

\par 15 Y luego escribió a Antonio, informándole que ya había ejecutado lo que le había escrito antes acerca de colocar a Aristóbulo en el puesto de su padre, antes de que llegara su última carta:

\par 16 Este asunto lo había demorado por aquel tiempo, porque fue necesario discutir el asunto con los sacerdotes y judíos, después de algunos días de intervalo, por ser la cosa inusual; pero habiendo transcurrido según su deseo, inmediatamente lo había nombrado.

\par 17 Pero ahora que había sido nombrado, no le era lícito salir de Jerusalén; Como no era rey, sino un sacerdote adscrito al servicio del templo:

\par 18 Y cuantas veces quiso obligarlo a salir, los judíos se negaron y no se lo permitieron, aunque matara a la mayor parte de ellos.

\par 19 Por lo tanto, cuando Antonio recibió la carta de Herodes, desistió de preguntar por Aristóbulo; y Aristóbulo fue nombrado sumo sacerdote.

\par 20 Entonces llegó la fiesta de las Tiendas; Y los hombres, reunidos ante la casa de Dios, vieron a Aristóbulo vestido con las túnicas sacerdotales de pie junto al altar, y lo oyeron bendecirlos:

\par 21 y agradaba tanto a los hombres, que le mostraban su afecto de manera muy marcada.

\par 22 Al enterarse Herodes de lo cual, se entristeció mucho; y temió que, cuando el partido de Aristóbulo ganara fuerza, le demandaría el reino, si su vida se prolongaba; por lo que comenzó a tramar su muerte.

\par 23 Era costumbre que los reyes salieran, después de la fiesta de los tabernáculos, a algunas residencias de lujo en Jericó que habían hecho los reyes anteriores.

\par 24 Y hay muchos jardines contiguos entre sí, en los cuales había estanques anchos y profundos para peces, a los cuales conducían corrientes de agua, y habían levantado en aquellos jardines hermosos edificios; también habían construido en Jericó hermosos palacios y hermosos edificios.

\par 25 Ahora bien, el autor del libro cuenta que en Jericó crecían abundantes árboles bálsamos; y que no se encontraron en ningún otro lugar que allí; y que muchos reyes los habían llevado de allí a su propio país, pero ninguno creció, excepto los que fueron llevados a Egipto;

\par 26 y que no fracasaron en Jericó hasta después de la destrucción de la segunda Casa; pero luego se marchitaron y nunca más volvieron a brotar.

\par 27 Entonces Herodes fue a Jericó en busca de placeres, y Aristóbulo lo siguió.

\par 28 Y cuando llegaron a Jericó, Herodes mandó a algunos de sus siervos que bajaran a los estanques de los peces y jugaran como era costumbre; y que si Aristóbulo bajaba a ellos, jugaran con él algún rato y luego lo ahogaran.

\par 29 Pero Herodes estaba sentado en una sala de banquete que se había preparado para sentarse; entonces Herodes mandó llamar a Aristóbulo y lo hizo sentarse a su lado; y también se sentaron en su presencia los jefes de sus servidores y de sus amigos.

\par 30 y ordenó que trajeran comida y bebida; y comieron y bebieron; y los sirvientes se apresuraron a bajar a las aguas según la costumbre, y se divirtieron.

\par 31 Y Aristóbulo tuvo grandes deseos de bajar con ellos al agua, ya dominados por el vino, y pidió permiso a Herodes para hacerlo, quien respondió:

\par 32 Esto no te conviene ni a ti ni a nadie como tú; y cuando le apremiaba, le amonestó y se lo prohibió; pero cuando Aristóbulo le repitió su petición, le dijo: Haz lo que quieras.

\par 33 Entonces Herodes, levantándose, fue a un palacio para dormir allí.

\par 34 Y Aristóbulo descendió a las aguas y jugó mucho tiempo con los asistentes, quienes, cuando vieron que, agotado y cansado, quería subir, lo retuvieron bajo el agua, lo mataron y lo llevaron muerto.

\par 35 Y hubo gran alboroto del pueblo, gritos y lamentaciones.

\par 36 Entonces Herodes, corriendo, salió para ver lo que había sucedido, y cuando vio a Aristóbulo muerto, se lamentó y lloró sobre él con mucha ternura, con un torrente de lágrimas muy vehemente.

\par 37 Luego ordenó que lo llevaran a la Ciudad Santa y lo acompañó hasta que llegó a la ciudad, y obligó al pueblo a asistir a sus funerales, y no hubo ningún punto del más alto honor que no le rindió.

\par 38 Y murió siendo un joven de dieciséis años de edad, y su sumo sacerdocio duró sólo unos días.

\par 39 Por esta razón surgió enemistad entre su madre Alejandra y su hija Mariamna, esposa de Herodes, y la madre y hermana de Herodes.

\par 40 Y se conocieron las maldiciones y ultrajes que Mariamne les infligía; Y aunque estos llegaron a Herodes, él no la prohibió ni la reprendió, a través de su gran afecto por ella:

\par 41 También temía que ella imaginara en su mente que él tenía buenas intenciones hacia los demás: de ahí que estas cosas duraran mucho entre estas mujeres.

\par 42 Y la hermana de Herodes, que estaba dotada de la mayor malicia y de un consumado artificio, comenzó a conspirar contra Mariamna:

\par 43 pero Mariamne era religiosa, recta, modesta y virtuosa; pero estaba un poco teñida de altivez, orgullo y odio hacia su marido.

\chapter{56}

\par \textit{La historia de Antonio, y de su expedición contra Augusto, y de la ayuda que pidió a Herodes. Y relato del terremoto que hubo en la tierra de Judá, y de la batalla que hubo entre ellos y los árabes.}

\par 1 Cleopatra, la reina de Egipto, era la esposa de Antonio, y descubrió tales métodos de adornarse y pintarse, con los que las mujeres suelen seducir a los hombres, como ninguna otra mujer en el mundo había descubierto:

\par 2 de modo que, siendo mujer de avanzada edad, parecía una muchachita soltera, y aún más delicada y más hermosa.

\par 3 Antonio también encontró en ella aquellos métodos de belleza y esos medios para crear placer, que nunca había encontrado en la gran cantidad de mujeres de las que había disfrutado. Por eso se apoderó tan completamente del corazón de Antonio, que no quedó en él lugar para el afecto hacia ninguna otra persona.

\par 4 Ella, por tanto, lo convenció de que desconcertara a ciertos reyes que estaban sujetos a los romanos, por sus propias consideraciones privadas; y él la obedeció en esto, dando muerte a ciertos reyes a instancia de ella; y a algunos los dejó con vida por orden de ella, haciéndolos sirvientes y esclavos de ella.

\par 5 Y esto fue dicho a Augusto; quien le escribió, abominando tal conducta y deseando que no volviera a ser culpable de algo similar.

\par 6 Y Antonio contó a Cleopatra lo que Augusto le había escrito; y ella le aconsejó que se rebelara contra Augusto, y le mostró que la cosa era muy fácil.

\par 7 A cuya opinión accedió, jugó abiertamente en falso con Augusto; y reunió un ejército y provisiones para poder ir por mar a Antioquía, y desde allí marchar por tierra para encontrarse con Augusto dondequiera que pudiera encontrarlo.

\par 8 Envió también a buscar a Herodes para que lo acompañara. Y Herodes fue a él con un ejército muy poderoso y provisiones muy completas.

\par 9 Y cuando llegó a él, Antonio le dijo: La razón correcta nos aconseja hacer una expedición contra los árabes y enfrentarnos a ellos: porque de ninguna manera estamos seguros de que no hagan una incursión contra los judíos y la tierra de Egipto, tan pronto como les hayamos dado la espalda.

\par 10 Antonio partió por mar, pero Herodes atacó a los árabes y Cleopatra envió a un general llamado Atenio con un gran ejército para ayudar a Herodes a someter a los árabes.

\par 11 Ella le ordenó que pusiera a Herodes y a sus hombres en primera fila y que llegara a un acuerdo con el rey de Arabia para que juntos rodearan a Herodes y despedazaran a sus hombres.

\par 12 A esto la llevó el deseo de apoderarse de todo lo que Herodes valía:

\par 13 También Alejandra algún tiempo antes le había pedido que indujera a Antonio a matar a Herodes; lo cual efectivamente había hecho, pero Antonio se negó a cometer este acto.

\par 14 A esto se añadió la circunstancia de que Cleopatra había deseado en otro tiempo a Herodes y había deseado en algún momento tener relaciones con él; pero él se contuvo, porque era casto. Y éstas fueron las causas que la habían inducido a esta línea de conducta.

\par 15 Entonces Atenio, llegando a Herodes, por orden de Cleopatra, envió a concertar un acuerdo con el rey de Arabia para rodearlo.

\par 16 Y cuando Herodes y sus árabes se encontraron y se enfrentaron, Atenio y sus hombres atacaron a Herodes, quien fue interceptado entre los dos ejércitos, y la batalla contra él se encarnizó tanto por delante como por detrás.

\par 17 Pero Herodes, al ver lo que había sucedido, reunió a sus hombres y luchó con todas sus fuerzas hasta que, después de un gran esfuerzo, estuvieron fuera del alcance de ambos ejércitos; y volvió a la Santa Casa.

\par 18 Y hubo un gran terremoto en la tierra de Judá, como no había sucedido desde los tiempos del rey Harba, en el que pereció una gran cantidad de hombres y animales.

\par 19 Esto alarmó mucho a Herodes, le causó gran temor y desanimó su espíritu. Por lo tanto, consultó con los ancianos de Judá acerca de hacer un acuerdo con todas las naciones circundantes; diseñar la paz y la tranquilidad, y la eliminación de las guerras y el derramamiento de sangre.

\par 20 También envió embajadores sobre estos asuntos a las naciones vecinas, quienes aceptaron la paz a la que los había invitado, excepto el rey de los árabes;

\par 21 quien ordenó matar a los embajadores que Herodes le había enviado; porque supuso que Herodes había hecho esto porque sus hombres habían sido destruidos en el terremoto y, por lo tanto, debilitado, se había dedicado a hacer la paz.

\par 22 Por lo cual decidió ir a la guerra contra Herodes; y habiendo reunido un ejército numeroso y bien provisto, marchó contra él.

\par 23 Y esto fue dicho a Herodes; y estaba muy molesto por dos razones: una, por la matanza de sus embajadores, acto que ninguno de los reyes había cometido hasta entonces; otro, porque se había atrevido a atacarlo, imaginando en su mente su debilidad y falta de tropas.

\par 24 Pero él quería mostrarle que las cosas eran diferentes: que todos aquellos a quienes había enviado embajadores para tratar de paz supieran que no lo había hecho por temor o debilidad, sino por el deseo de lograr que era amable y bueno; para que nadie se atreviera a atentar contra los judíos, o imaginar en su mente que eran débiles.

\par 25 Además, quería vengarse del rey de Arabia a causa de sus embajadores, por lo que decidió marchar a toda prisa contra él.

\par 26 Entonces reunió tropas de la tierra de Judá y les dijo: «Ustedes están al tanto de la matanza de nuestros embajadores perpetrada por ese árabe; un acto que ningún rey hasta ahora ha «cometido:

\par 27 porque piensa que hemos sido debilitados y nos hemos vuelto impotentes; y se ha atrevido a provocarnos, y piensa que obtendrá sobre nosotros todos sus deseos, y no dejará de hacernos la guerra continuamente.

\par 28 Por lo tanto, debéis luchar contra las dificultades, para poder demostrar vuestra valentía y poder dominar a vuestros enemigos y llevaros sus despojos».

\par 29 aunque la fortuna unas veces pueda mostrarse favorable, otras adversa, según las costumbres y vicisitudes habituales de este mundo.

\par 30 En verdad, debes emprender inmediatamente una expedición para vengarte de esos opresores y frenar la audacia de todos los que te tienen en poca estima.

\par 31 Pero si decís que este terremoto nos ha descorazonado y ha destruido a muchos de nosotros; sabes muy bien que no ha destruido a ninguno de los combatientes, pero sí a algunos otros.

\par 32 Tampoco debemos pensar que es en absoluto irrazonable que haya destruido a los peores de nuestra nación, pero haya dejado a los mejores para sobrevivir. También es indudable que esto ha mejorado su ánimo y sus sentimientos internos.

\par 33 Pero el deber de aquel a quien Dios ha salvado de la destrucción y preservado de la ruina, es que le obedezca y haga lo que es bueno y correcto.

\par 34 Y en verdad, ninguna obediencia es más honorable y gloriosa que pedir al opresor la reparación del oprimido; y someter a los enemigos de Dios y de su religión y nación, ayudando a quienes le muestran obediencia y atención.

\par 35 Tampoco sabéis lo que nos sucedió últimamente con aquellos árabes, cuando nos rodearon con Atenio; y cómo el Dios grande y bueno nos ayudó contra ellos, y nos libró de ellos.

\par 36 Temed, pues, a Dios, siguiendo vuestra antigua costumbre y la laudable costumbre de vuestros antepasados; y.preparaos contra este enemigo antes de que se prepare contra vosotros, y estad de antemano con él antes de que se anticipe a vosotros: y Dios os suministrará ayuda y socorro contra vuestro enemigo».

\par 37 Cuando los hombres oyeron las palabras de Herodes, respondieron que estaban dispuestos a emprender la expedición y que no se demorarían.

\par 38 Y dio gracias a Dios y a ellos por ello, y ordenó que se ofrecieran muchos sacrificios; también ordenó que se levantara un ejército; y se reunió una gran multitud de la tribu de Judá y de Benjamín.

\par 39 Entonces Herodes, marchando contra el rey de los árabes, lo encontró; y la batalla se encarnizó entre ellos, y murieron cinco mil árabes.

\par 40 Hubo otra batalla y murieron cuatro mil árabes; por lo que los árabes regresaron a su campamento y permanecieron allí; y Herodes nada pudo hacer contra ellos, porque el lugar estaba fortificado; pero él permaneció con su ejército, sitiándolos en el mismo lugar y no dejándolos salir.

\par 41 Y permanecieron cinco días en este estado; Y les sobrevino una sed tremenda; Por lo tanto, enviaron embajadores a Herodes con un regalo muy valioso, pidiéndole una tregua y libertad para sacar agua para beber; pero él no los escuchó, sino que continuó en la misma furiosa hostilidad.

\par 42 Entonces dijeron los árabes: Salgamos contra esta nación; porque es mejor para nosotros vencer o morir, que perecer de sed.

\par 43 Y salieron contra ellos; y el grupo de Herodes los venció y mató a nueve mil de ellos; y Herodes con sus hombres persiguió a los árabes que huían, matando a muchos de ellos; y sitió sus ciudades y las tomó.

\par 44 Por lo que demandaron por sus vidas, prometiendo obediencia; A lo cual él accedió, se retiró de ellos y regresó a la Santa Casa.

\par 45 Ahora bien, los árabes mencionados en este libro son los árabes que habitaron desde el país de Sara hasta Hegiaz y las partes adyacentes; y eran de gran renombre y en gran número.


\chapter{57}

\par \textit{La historia de la batalla de Antonio contra Augusto, y de la muerte de Antonio, y del viaje de Herodes a Augusto.}

\par 1 Cuando Antonio salió de Egipto hacia el país de los romanos y se encontró con Augusto, se produjeron entre ellos batallas muy duras, en las que la victoria fue del lado de Augusto y Antonio cayó en la batalla;

\par 2 y Augusto tomó posesión de su campamento y de todo lo que había en él. Hecho esto, se dirigió a Rodas, para embarcarse allí y pasar a Egipto.

\par 3 Herodes recibió la noticia y se turbó mucho por la muerte de Antonio y temía muchísimo a Augusto; y resolvió ir a él, saludarlo y felicitarlo.

\par 4 Por lo que envió a su madre y a su hermana con su hermano a una fortaleza que tenía en el monte Sara; también envió a su esposa Mariamne y a su madre Alejandra a Alejandrium, bajo el cuidado de Josefo de Tiro; conjurándole a matar a su esposa y a su madre, tan pronto como se le informara de su muerte.

\par 5 Después de esto, fue a ver a Augusto con un regalo muy valioso. Ahora Augusto ya había decidido dar muerte a Herodes;

\par 6 porque había sido amigo y partidario de Antonio, y porque antes había deliberado» marchar con Antonio para atacarlo.

\par 7 Cuando, pues, Augusto fue notificado de la llegada de Herodes, le hizo comparecer ante su presencia vestido con el hábito real que llevaba; excepto la diadema, que había mandado apartar de su cabeza.

\par 8 El cual, estando delante de él, se quitó la diadema, como Augusto le había ordenado, y dijo:

\par 9 «Oh rey, tal vez a causa de mi amor hacia Antonio te has enojado tanto conmigo que me has quitado la diadema de la cabeza;

\par 10 ¿o fue por alguna otra causa? Pues si os enojáis conmigo por mi adhesión a Antonio, en verdad os digo que me uní a él porque me merecía bien y puse sobre mi cabeza la diadema que me habéis quitado.

\par 11 Y en verdad él había pedido mi ayuda contra ti, y yo se la di; incluso «como él también muchas veces me prestó su ayuda:

\par 12 Pero no me tocó estar presente en la batalla que él peleó contigo, ni desenvainé mi espada contra ti, ni peleé contra ti; cuya causa fue que yo estaba ocupado en someter a los árabes.

\par 13 Pero nunca dejé de proporcionarle ayuda de hombres, armas y provisiones, como lo requería su amistad y sus buenas obras hacia mí. Y en verdad lamento haberlo dejado; para que los hombres no concibieran que abandoné a mi amigo cuando necesitaba mi ayuda.

\par 14 Ciertamente, si hubiera estado con él, lo habría ayudado con todas mis fuerzas; y lo habría animado si hubiera tenido miedo, y lo habría fortalecido si hubiera sido debilitado, y lo habría levantado si hubiera caído, hasta que Dios hubiera gobernado las cosas como quisiera.

\par 15 Y esto realmente me hubiera sido menos doloroso que pensar que había fallado a un hombre que había implorado mi ayuda, y así sucediera que mi amistad fuera poco estimada.

\par 16 En mi opinión, ciertamente fracasó por su propia mala política al ceder ante esa hechicera Cleopatra; a quien le había aconsejado que matara, y así quitarle su malicia; pero él no asintió.

\par 17 Pero ahora, si me habéis quitado la diadema de la cabeza, ciertamente no me quitaréis mi inteligencia ni mi coraje; y sea lo que sea, seré amigo de mis amigos y enemigo de mis enemigos».

\par 18 Augusto le respondió: «A Antonio, en verdad, lo hemos vencido con nuestras tropas; pero a ti te dominaremos atrayéndote hacia nosotros; y nos encargaremos, por nuestros buenos oficios hacia ti, de que tu afecto hacia nosotros sea duplicado, porque eres digno de ello.

\par 19 Y así como Antonio se desvió del consejo de Cleopatra, por la misma razón se comportó con nosotros de manera ingrata; volviendo por nuestras bondades males, y por nuestros favores rebelión.

\par 20 Pero nosotros nos alegramos de la guerra que habéis hecho contra los árabes, que son nuestros enemigos; porque cualquiera que sea vuestro enemigo, también lo es nuestro; y el que os rinde obediencia, a nosotros también nos la hace.»

\par 21 Entonces Augusto ordenó que se pusiera una diadema de oro sobre la cabeza de Herodes y que se le añadieran tantas provincias como las que ya tenía.

\par 22 Herodes acompañó a Augusto a Egipto; y todas las cosas que Antonio había destinado a Cleopatra le fueron entregadas. Y Augusto partió a Roma; pero Herodes volvió a la Ciudad Santa.

\chapter{58}

\par \textit{La historia del asesinato que Herodes cometió contra su esposa Mariamne.}

\par 1 Josefo, el marido de la hermana de Herodes, le había revelado a Mariamna que Herodes le había ordenado matarla a ella y a su madre, tan pronto como él mismo pereciera al subir a Augusto.

\par 2 Y ella ya tenía aversión hacia Herodes, desde el momento en que mató a su padre y a su hermano; y a esto se añadió no poco odio cuando fue informada de las órdenes que él había dado contra ella.

\par 3 Por lo tanto, cuando Herodes salió de Egipto, la encontró totalmente abrumada por el odio hacia él; por lo cual, muy turbado, trató de reconciliarla con él por todos los medios posibles.

\par 4 Pero un día llegó su hermana, después de algunas disputas que habían tenido lugar entre ella y Mariamna, y le dijo: Ciertamente José mi marido se ha ido con Mariamna.

\par 5 Pero Herodes no prestó atención a sus palabras, sabiendo cuán pura y casta era Mariamna.

\par 6 Después de esto, Herodes fue a ver a Mariamna la noche siguiente a aquel día, y se portó con ella con bondad y afecto, contándole su amor por ella, diciendo mucho sobre este tema:

\par 7 a quien ella dijo: «¿Has visto alguna vez a un hombre amar a otro y ordenarle que lo mataran? ¿Y es un enemigo a menos que muestre tales pruebas?

\par 8 Entonces Herodes se dio cuenta de que Josefo había descubierto a Mariamna el secreto que él le había confiado; y creyó que él no habría hecho eso, a menos que ella se hubiera entregado a él:

\par 9 y creyó lo que su hermana le había dicho sobre este asunto; y apartándose inmediatamente de Mariamne, la odió y detestó.

\par 10 Al enterarse su hermana, fue al copero, le dio dinero, le dio veneno y le dijo: Llevad esto al rey y decidle: Mariamna, esposa del rey, me dio este veneno y este dinero, y me mandó mezclarlo en la bebida del rey.

\par 11 Esto hizo el copero. Y el rey, al ver el veneno, no dudó de la veracidad de la cosa: entonces da orden de decapitar inmediatamente a Josefo, su cuñado; y también ordena que encadenen a Mariamne, hasta que los setenta ancianos estén presentes y le dicte la debida sentencia.

\par 12 Entonces la hermana de Herodes temió que se descubriera lo que había hecho y ella misma pereciera si Mariamna era liberada; entonces le dijo: Oh rey, si pospones la muerte de Mariamna para mañana, no podrá en absoluto efectuarlo:

\par 13 Porque tan pronto como se sepa que quieres matarla, vendrá toda la casa de su padre, y todos sus sirvientes y vecinos, y se interpondrán; y no podréis obtener su muerte hasta después de grandes tumultos.

\par 14 Entonces Herodes dijo: Haz lo que mejor te parezca.

\par 15 Entonces la hermana de Herodes envió a toda prisa un hombre para llevar a Mariamna al lugar de la matanza, echando sobre ella a sus criadas y a otras mujeres, para insultarla y reprenderla con toda clase de indecencia:

\par 16 pero ella nada respondió a ninguno de ellos, ni siquiera movió la cabeza en lo más mínimo: ni su color cambió con todo este tratamiento, ni apareció en ella miedo o confusión, ni se alteró su andar;

\par 17 pero con su manera habitual se dirigió al lugar donde la llevaban para matarla; y doblando las rodillas, extendió voluntariamente el cuello:

\par 18 y partió de esta vida, famoso por la religión y la castidad, marcado por «sin crimen, marcado sin culpa»; sin embargo, no estaba del todo libre de altivez, según la costumbre de su familia.

\par 19 Y de esto no fue la menor causa la obsequiosa atención y afecto de Herodes hacia ella, debido a la elegancia de su forma; de donde no sospechaba ningún cambio en él hacia ella.

\par 20 Herodes había engendrado de ella dos hijos, Alejandro y Aristóbulo; quienes, cuando su madre fue asesinada, vivían en Roma; porque los había enviado allí para aprender la literatura y el idioma de los romanos.

\par 21 Después Herodes se arrepintió de haber matado a su mujer; y se sintió tan afectado por su muerte, que contrajo una enfermedad de la que estuvo a punto de morir.

\par 22 Como Mariamna había muerto, su madre Alejandra trazó planes para matar a Herodes; Lo cual, al saberlo, la abandonó.

\chapter{59}

\par \textit{La historia de la venida de los dos hijos de Herodes, Alejandro y Ari, era tan pronto como se enteraron de que su madre había sido ejecutada por Herodes.}

\par 1 Cuando Alejandro y Aristóbulo recibieron la noticia del asesinato de su madre por Herodes, se sintieron abrumados por un dolor excesivo;

\par 2 y partiendo de Roma llegaron a la Ciudad Santa, sin respetar a su padre Herodes como solían hacer antes, por el odio que sentían hacia él por la muerte de su madre.

\par 3 Alejandro se había casado con la hija del rey Arquelao, y Aristóbulo se había casado con la hija de la hermana de Herodes.

\par 4 Entonces Herodes, al ver que no le respetaban, se dio cuenta de que lo odiaban y los evitaba; y esto no pasó desapercibido para los jóvenes y su familia.

\par 5 El rey Herodes se había casado antes que Mariamna con una mujer llamada Dositea, con quien tuvo un hijo llamado Antípatro.

\par 6 Cuando Herodes, pues, tuvo la seguridad de sus dos hijos, como ya hemos dicho, llevó a su esposa Dositea a su palacio, y se unió a su hijo Antípater, encomendándole todos sus negocios; y lo nombró por testamento su sucesor.

\par 7 Y que Antípatro persiguió a sus hermanos Alejandro y Aristóbulo, con la intención de procurarse la paz para él mientras vivía su padre, para no tener rival después de su muerte.

\par 8 Entonces dijo a su padre: «En verdad, mis hermanos buscan una herencia a causa de la familia de su madre, porque es más noble que la familia de mi madre; y por tanto tienen mejor derecho que yo a la fortuna de que el rey me ha juzgado digno:

\par 9 por eso pretenden matarte a ti, y a mí también me matarán poco después.

\par 10 Y esto se lo repetía muchas veces a Herodes, enviándole también en secreto personas para insinuarle cosas que pudieran producir en él un mayor odio hacia ellos.

\par 11 Mientras tanto, Herodes va a Roma, donde se encuentra Augusto, llevando consigo a su hijo Alejandro. «Y cuando llegó a la presencia de Augusto, Herodes se quejó con él de su hijo, pidiéndole que lo reprendiera.

\par 12 Pero Alejandro dijo; «En verdad no niego mi angustia por el asesinato de mi madre sin culpa alguna; Porque incluso los animales brutos muestran mucho más afecto a sus madres que los hombres y las aman más:

\par 13 pero todo plan de parricidio lo niego por completo y me limpio de ello ante Dios, porque tengo los mismos sentimientos hacia mi padre que hacia mi madre.

\par 14 Tampoco soy de esa clase de hombres que me acarrea culpa por el crimen contra mis padres, y más especialmente tormentos eternos.

\par 15 Entonces Alejandro lloró con llanto amargo y muy vehemente; y Augusto se compadeció de él, y todos los jefes de los romanos que estaban cerca, también lloraron.

\par 16 Entonces Augusto pidió a Herodes que restituyera a sus hijos la bondad y la intimidad anteriores, y pidió a Alejandro que besara los pies de su padre, quien así lo hizo. También ordenó a Herodes que lo abrazara y besara, y Herodes le obedeció.

\par 17 Después Augusto encargó un magnífico presente para Herodes, y se lo llevaron; y después de pasar algunos días con él, Herodes regresó a la Santa Casa; y llamando a él a los ancianos de Judá, dijo:

\par 18 Sabed que Antípatro es mi hijo mayor y mi primogénito, pero su madre es de una familia innoble; pero la madre de mis hijos Alejandro y Aristóbulo es de la familia de los sumos sacerdotes y de los reyes.

\par 19 Además, Dios ha ensanchado mi reino y ha extendido mi poder; y por tanto: me parece bien poner a estos mis tres hijos en igual autoridad; de modo que Antípatro no tendrá mando sobre sus hermanos, ni sus hermanos tendrán mando sobre él.

\par 20 Obedeced, pues, a los tres, oh asamblea de hombres, y no interfiráis en nada en lo que sus mentes puedan estar de acuerdo; ni proponer nada que pueda producir confusión y desacuerdo entre ellos.

\par 21 Y no bebáis con ellos ni habléis demasiado con ellos. Porque a partir de allí sucederá que alguno de ellos, imprudentemente, os contará los designios que tiene contra su hermano:

\par 22 sobre el cual, para conciliarlos contigo, seguirás tu acuerdo con cada uno de ellos, según lo que a él le parezca bien; y los destruiréis, y vosotros también seréis destruidos.

\par 23 En verdad, os corresponde a vosotros, hijos míos, ser obedientes a Dios y a mí; para que vivas muchos años y prosperen tus negocios». Poco después los abrazó y besó y ordenó al pueblo que se retirara.

\par 24 Pero lo que hizo Herodes no tuvo ningún resultado feliz, ni los corazones de sus hijos se unieron. Antípatro quería que todo le fuera puesto en sus manos, tal como su padre lo había ordenado anteriormente, y a sus hermanos no les parecía en absoluto justo que se le considerara igual a ellos.

\par 25 Antípatro estaba dotado de perseverancia y de toda mala y fingida amistad; pero no así sus dos hermanos: Antípatro, pues, puso espías a sus hermanos, que le llevarían noticias de ellos; también puso a otros que llevarían informes falsos sobre ellos a Pilato».

\par 26 Pero cuando Antípatro estaba delante del rey y oyó a alguien contar tales cosas de sus hermanos, rechazó la acusación de ellos, declarando que los autores no eran dignos de crédito y rogándole al rey que no creyera en los rumores.

\par 27 Lo cual hizo Antípatro para no inspirar al rey ninguna duda o sospecha de sí mismo.

\par 28 Por lo tanto, el rey no tuvo ninguna duda de que estaba bien inclinado hacia sus hermanos, y no les deseaba ningún mal.

\par 29 Al enterarse Antípatro, se inclinó hacia su propósito, Feroras, su tío y su tía (pues estaban enemistados con sus hermanos por causa de su madre), ofreciendo a Feroras un presente muy valioso, rogándole que informara al rey que Alejandro y Aristóbulo habían trazado un plan para asesinar al rey.

\par 30 (Pero Herodes tenía buena simpatía por su hermano Feroras y prestaba atención a todo lo que decía, ya que cada año le pagaba una gran suma con cargo a las provincias que gobernaba a orillas del Éufrates).

\par 31 Y esto hizo Feroras. Después Antípatro fue a Herodes y le dijo: «Oh rey, en s mis hermanos han tramado un complot para destruirme».

\par 32 Antípatro dio además dinero a los tres eunucos del rey para que dijeran: «Alejandro nos ha dado dinero para abusar de nosotros y matarte nosotros con la muerte».

\par 33 Y el rey se enojó contra Alejandro y ordenó que lo encadenaran; y apresó y sometió a tortura a todos los servidores de Alejandro, hasta que confesaran lo que sabían acerca del complot de Alejandro para asesinarlo.

\par 34 Y muchos de ellos, aunque murieron bajo el tormento, nunca dijeron una mentira acerca de Alejandro; pero algunos de ellos, no pudiendo soportar la violencia del tormento, inventaron mentiras por el deseo de liberarse;

\par 35 afirmando que Alejandro y Aristóbulo habían planeado atacar al rey, matarlo y huir a Roma; y habiendo recibido un ejército de Augusto, marcharían contra la Santa Casa, matarían a su hermano Antípater y tomarían el trono de Judza.

\par 36 Y el rey ordenó que apresaran a Aristóbulo y lo encadenaran, y lo ataron y lo pusieron con su hermano.

\par 37 Pero cuando su suegro Arquelao recibió noticias de Alejandro, éste fue a ver a Herodes, fingiendo estar muy enojado contra Alejandro:


\par 38 como si, al enterarse de la intención de parricidio, hubiera venido expresamente a ver si su hija, la esposa de Alejandro, estaba al tanto del negocio, y no se lo había revelado para poder matarla: pero que, si ella no estaba al tanto de nada por el estilo, podría separarla de Alejandro y llevarla a su propia casa.

\par 39 Este Arquelao era un hombre prudente, sabio y elocuente. Y cuando Herodes escuchó sus palabras y quedó satisfecho de su prudencia y honestidad, se apoderó maravillosamente de su corazón; y se confió a él, y confió en él sin la menor vacilación.

\par 40 Entonces Arquelao, viendo la inclinación de Herodes hacia él, después de una larga intimidad, le dijo un día que se habían retirado juntos:

\par 41 «En verdad, oh rey, reflexionando sobre tus asuntos he descubierto que, siendo ahora de edad avanzada, te falta mucho reposo mental y consuelo en tus hijos; mientras que, por el contrario, de ellos habéis sacado pena y ansiedad.

\par 42 Además, he pensado en estos dos hijos tuyos, y no encuentro que hayas sido deficiente en merecerlos bien; porque los has promovido y los has hecho reyes, y no has dejado nada sin hacer que pudiera impulsarlos perversamente a tramar tu muerte, ni tienen ninguna causa para entrar en este negocio.

\par 43 Pero tal vez esto provenga de alguna persona maliciosa que desea el mal contra ti y contra ellos, o que por envidia o enemistad te ha inducido a aborrecerlos.

\par 44 Si, pues, ha adquirido influencia sobre ti, que eres un hombre anciano, dotado de conocimiento, conocimiento y experiencia, transformándote de la mansedumbre paterna a la crueldad y la furia contra tus hijos;

\par 45 ¡Cuánto más fácil habría podido ser con ellos, que son jóvenes, inexpertos, desprotegidos y sin conocimiento de los hombres ni de sus engaños, para obtener de ellos lo que deseaba en este asunto!

\par 46 Considera, pues, tus asuntos, oh rey; y no prestéis oído a las palabras de los delatores, ni hagáis nada precipitado contra vuestros hijos; y preguntad quién es el que ha estado tramando el mal contra vosotros y contra ellos.

\par 47 Y el rey le respondió: «De hecho, la cosa es como usted ha mencionado: desearía saber quién los ha inducido a hacer esto». Arquelao respondió: «Este es tu hermano Pheroyas». El rey respondió: «Puede que sea así».

\par 48 Después de esto, el rey cambió mucho en su comportamiento hacia Feroras, lo cual, al verlo Feroras, le tuvo miedo; y llegando a Arquelao, le dijo;

\par 49 «Veo cómo el rey se ha cambiado hacia mí; por lo que te ruego que reconcilies su mente conmigo, eliminando los sentimientos que alberga en su corazón contra mí».

\par 50 A quien respondió Arquelao; «Lo haré en verdad, si prometes revelar al rey la verdad sobre los complots que has tramado contra Alejandro y Aristóbulo». Y él asintió.

\par 51 Y después de algunos días, Arquelao dijo al rey: «Oh rey, verdaderamente los parientes del hombre son para él como sus propios miembros: y como le conviene al hombre, si alguno de sus miembros resulta afectado por alguna enfermedad que le sobreviene, restaurarlo con medicinas, aunque pueda causarle dolor;

\par 52 y no es bueno cortarlo, no sea que el dolor aumente, el cuerpo se debilite y los miembros desfallezcan; y así, por la pérdida de ese miembro, debería sentir la falta de muchas comodidades:

\par 53 pero que soporte los dolores del tratamiento médico, para que su miembro mejore y sane, y su cuerpo vuelva a su antigua perfección y fuerza.

\par 54 Así es conveniente que un hombre, cada vez que alguno de sus parientes se haya alejado de él por cualquier causa abominable, lo reconcilie consigo mismo;

\par 55 incitándolo a la cortesía y la amistad, admitiendo sus excusas y desestimando los cargos en su contra; y que no lo ejecute apresuradamente ni lo aleje demasiado tiempo de su presencia.

\par 56 Porque los parientes del hombre son sus sostenedores y ayudantes, y en ellos consiste su honor y su gloria; y a través de ellos obtiene lo que de otro modo no podría obtener.

\par 57 Feroras es verdaderamente hermano del rey, e hijo de su padre y de su madre; y confiesa su falta, suplicando al rey que lo perdone y deseche de su mente su error. Y el rey respondió: «Esto haré».

\par 58 Y ordenó a Feroras que viniera delante de él; quien estando presente, le dijo; «He pecado ahora ante los ojos del Dios grande y bueno, y ante el rey, tramando maldades y planes que podrían perjudicar los asuntos del rey y de sus hijos, con falsedades mentirosas.

\par 59 Pero lo que me indujo a actuar así fue que el rey me quitó a cierta mujer, mi concubina, y nos separó a ella y a mí.

\par 60 El rey dijo a Arquelao: «Ahora he perdonado a Feroras, tal como me pediste, porque he descubierto que has curado la enfermedad que había en nuestros asuntos con tus métodos calmantes, así como un médico ingenioso cura las corrupciones de un cuerpo enfermo.

\par 61 Por tanto, te ruego que perdones a Alejandro, reconciliando a tu hija con su marido; pues tenla por hija mía, sabiendo que ella es más prudente que él, y que con su prudencia y sus exhortaciones lo aparta de muchas cosas.

\par 62 Por eso te ruego que no los separes ni lo destruyas, porque él está de acuerdo con ella y obtiene muchas ventajas de su guía.

\par 63 Pero Arquelao respondió: «Mi hija es la esclava del rey, pero a él mi alma últimamente lo aborrece a causa de sus malvados designios. Permítame, pues, el rey separarlo de mi hija, y el rey podrá unirla a quien de sus siervos quiera.

\par 64 A quien respondió el rey; «No vayas más allá de mi petición; y deja que tu hija se quede con él, y no me contradigas. Y Arquelao dijo; «Seguramente lo haré; y no contradeciré al rey en nada de lo que me ordene».

\par 65 Poco después, Herodes ordena que liberen a Alejandro y a Aristóbulo de sus cadenas y se presenten ante él; los cuales, estando en su presencia, se postraron ante él, confesando sus faltas, excusándose y pidiéndole perdón y perdón.

\par 66 Y les ordenó que se levantaran, y haciéndoles acercarse a él, los besó y les ordenó que se fueran a sus casas y regresaran al día siguiente. Y vinieron a comer y a beber con él, y él los restituyó en un lugar de mayor honor.

\par 67 Y a Arquelao le dio setenta talentos y un lecho de oro, y ordenó también a todos los principales de sus amigos que le ofrecieran regalos valiosos, y así lo hicieron.

\par 68 Cumplido esto, Arquelao partió de la ciudad de la Santa Casa a su país; a quien Herodes acompañó, y finalmente, habiéndose despedido de él, regresó a la Santa Casa.

\par 69 Sin embargo, Antípatro no dejó de conspirar contra sus hermanos para hacerlos odiosos.

\par 70 Aconteció que vino a Herodes un hombre que tenía algunos objetos valiosos y hermosos, con los que se suele ganar a los reyes;

\par 71 Estos se los presentó al rey, quien, tomándolos, se los pagó; Y aquel hombre obtuvo un lugar muy alto en sus afectos, y habiendo sido incorporado a su séquito, gozó de su confianza: este hombre se llamó Euricles.

\par 72 Cuando Antípatro se dio cuenta de que este hombre se había granjeado el favor de su padre, le ofreció dinero y le pidió que insinuara hábilmente a Herodes y le dijera que sus dos hijos, Alejandro y Aristóbulo, planeaban asesinarlo; que el hombre le prometió hacer.

\par 73 Poco después fue a ver a Alejandro, y se hizo tan íntimo y familiar con él, que se supo que era amigo suyo, y se le hizo saber al rey que tenía intimidad con él.

\par 74 Después de esto, se fue con el rey y le dijo: «Ciertamente tienes este derecho sobre mí, oh rey, de que nada debe impedirme darte buenos consejos: y en verdad tengo un asunto que el rey debe saber, y que debo revelarte».

\par 75 El rey le dijo: «¿Qué tienes?» El hombre le respondió: Oí a Alejandro decir: «En verdad, Dios ha diferido la venganza de mi padre por la muerte de mi madre, de mi abuelo y de mis parientes, sin ningún crimen, para que se realice por mi mano; y Espero vengarme de ellos en él.

\par 76 Y ahora ha acordado con algunos jefes atacaros, y quería implicarme en los planes que había ideado; pero lo consideré un crimen, a causa de los actos de bondad del rey hacia mí, y su liberalidad.

\par 77 Pero mi intención es amonestarlo bien y contárselo, porque tiene ojos y entendimiento.

\par 78 Y cuando el rey escuchó estas palabras, no las desestimó, sino que rápidamente comenzó a investigar su verdad:

\par 79 pero no encontró nada en lo que pudiera confiar, excepto una carta falsificada en nombre de Alejandro y Aristóbulo dirigida al gobernador de cierta ciudad.

\par 80 Y en la carta decía: «Queremos matar a nuestro padre y huir a ti; por tanto, prepáranos un lugar donde podamos permanecer hasta que el pueblo se reúna a nuestro alrededor y nuestros asuntos estén arreglados».

\par 81 Y esto fue confirmado al rey, y parecía probable: por lo que apresó al gobernador de esa ciudad y lo sometió a tortura para que confesara lo que estaba escrito en esa carta.

\par 82 Lo cual este hombre negó, eximiéndose de la acusación: ni se les demostró nada en este asunto ni en cualquier otra cosa que el delator les había acusado.

\par 83 Pero Herodes ordenó que los apresaran y los ataran con cadenas y grillos. Luego fue a Tiro, y de Tiro a Cesarea, llevándolos consigo encadenados.

\par 84 Y todos los capitanes y todos los soldados se compadecieron de ellos, pero nadie intercedió por ellos ante el rey, para que no confesara que era verdad de sí mismo lo que el delator había afirmado.

\par 85 Había en el ejército un viejo guerrero que tenía un hijo al servicio de Alejandro. Por lo tanto, cuando el anciano vio la miserable condición de los dos hijos de Herodes, se compadeció maravillosamente de su cambio de suerte y gritó con tanta voz como pudo: «¡Se ha ido la compasión! la bondad y la piedad han desaparecido; la verdad es eliminada del mundo».

\par 86 Entonces dijo al rey: «¡Oh, tú, despiadado con tus hijos, enemigo de tus amigos y amigo de tus enemigos, que recibes las palabras de los delatores y de las personas que no te desean ningún bien!»

\par 87 Y los enemigos de Alejandro y Aristóbulo corrieron hacia él, lo reprendieron y dijeron al rey: «Oh rey, no es el amor hacia ti y hacia «tus hijos» lo que ha inducido a este hombre a hablar así;

\par 88 pero ha querido balbucear el odio que sentía en su corazón hacia ti y hablar mal de tu consejo y administración, como si fuera un consejero fiel.

\par 89 Y algunos observadores nos han informado de él que ya había pactado con el barbero del rey matarlo con la navaja mientras lo afeitaba.

\par 90 Y el rey ordenó que arrestaran al anciano, a su hijo y al barbero; y el anciano y el barbero serían azotados con varas hasta que confesaran. Y fueron golpeados con varas cruelmente y sometidos a diversas clases de torturas; pero no confesaron nada de lo que no habían hecho.

\par 91 Cuando el hijo del anciano vio la triste condición de su padre y el estado en el que había llegado, se compadeció de él y pensó que sería liberado si él mismo confesara lo que habían puesto a su padre, después de recibir del rey una promesa por su vida.

\par 92 Entonces dijo al rey: «Oh rey, dame seguridad para mi padre y para mí, para que pueda decirte lo que buscas». Y el rey dijo: «Puedes tener esto».

\par 93 A quien dijo; «Alejandro ya había acordado con mi padre que te mataría, pero mi padre estuvo de acuerdo con el barbero, como te he dicho».

\par 94 Entonces el rey ordenó que mataran al anciano, a su hijo y al barbero. Asimismo ordenó que llevaran a sus hijos Alejandro y Aristóbulo a Sebaste, donde los mataran y los fijaran en un eibbet; y los llevaron, los mataron y los fijaron en una horca.

\par 95 Alejandro dejó de la hija del rey Arquelao dos hijos que le sobrevivieron, a saber, Tirones y Alejandro; y Aristóbulo dejó tres hijos, a saber, Aristóbulo, Agripa y Herodes.

\par 96 Pero la historia de Antípatro, hijo de Herodes, ya está descrita en nuestros relatos anteriores.

\end{document}