\begin{document}

\title{Pseudo-Filón}

\chapter{1}

\par 1 El comienzo del mundo. Adán engendró tres hijos y una hija, Caín, Noaba, Abel y Set.

\par 2 Y Adán vivió, después de engendrar a Set, setecientos años y engendró doce hijos y ocho hijas.

\par 3 Y estos son los nombres de los varones: Eliseel, Suris, Elamiel, Brabal, Naat, Zarama, Zasam, Maathal y Anath.

\par 4 Y estas son sus hijas: Phua, Iectas, Arebica, Sifa, Tecia, Saba y Asin.

\par 5 Y Set vivió 105 años y engendró a Enós. Y vivió Set, después que engendró a Enós, 707 años, y engendró 3 hijos y 2 hijas.

\par 6 Y estos son los nombres de sus hijos: Elidia, Phonna y Matha; y de sus hijas, Malida y Thila.

\par 7 Enós vivió ciento ochenta años y engendró a Cainán. Y vivió Enós, después que engendró a Cainán, 715 años, y engendró dos hijos y una hija.

\par 8 Y estos son los nombres de sus hijos: Phoë y Thaal; y de la hija, Catennath.

\par 9 Y vivió Cainán 520 años y engendró a Malalec. Y vivió Cainán, después que engendró a Malalec, 730 años, y engendró 3 hijos y 2 hijas.

\par 10 Y estos son los nombres de los varones: Athach, Socer, Lopha; y los nombres de las hijas, Ana y Leua.

\par 11 Y vivió Malalec ciento sesenta y cinco años y engendró a Jaret. Y vivió Malalech, después que engendró a Jareth, 730 años, y engendró 7 hijos y 5 hijas.

\par 12 Y estos son los nombres de los varones: Leta, Matha, Cethar, Melie, Suriel, Lodo y Othim. Y estos son los nombres de las hijas: Ada y Noa, Iebal, Mada, Sella.

\par 13 Y Jareth vivió 172 años y engendró a Enoc. Y Jareth vivió después de engendrar a Enoc 800 años y engendró 4 hijos y 2 hijas.

\par 14 Y estos son los nombres de los varones: Lead, Anac, Soboac e Iectar; y de las hijas, Tetzeco, Lesse.

\par 15 Y Enoc vivió 165 años y engendró a Matusalam. Y Enoc vivió, después que engendró a Matusalam, 200 años, y engendró 5 hijos y 3 hijas.

\par 16 Pero Enoc agradó a Dios en aquel momento y no fue encontrado, porque Dios lo trasladó.

\par 17 Los nombres de sus hijos son: Anaz, Zeum, Achaún, Pheledi, Elith; y de las hijas, Theiz, Lefith y Leath.

\par 18 Matusalam vivió 187 años y engendró a Lamec. Y Mathusalam vivió, después que engendró a Lamec, 782 años, y engendró 2 hijos y 2 hijas.

\par 19 Y estos son los nombres de los varones: Inab y Rapho; y de las hijas, Aluma y Amuga.

\par 20 Y Lamec vivió 182 años y engendró un hijo, y lo llamó Noé, según su nacimiento, diciendo: Este niño nos dará descanso a nosotros y a la tierra de aquellos que están en ella, sobre quienes (o en el día en que) se hará una visita a causa de la iniquidad de sus malas acciones.

\par 21 Y vivió Lamec después de engendrar a Noé 585 años.

\par 22 Noé vivió trescientos años y engendró tres hijos: Sem, Cam y Jafet.

\chapter{2}

\par 1 Pero Caín permaneció en la tierra temblando, tal como Dios le había ordenado después de que mató a su hermano Abel; y el nombre de su mujer era Themec.

\par 2 Y Caín conoció a Temac su esposa y ella concibió y dio a luz a Enoc.

\par 3 Caín tenía quince años cuando hizo estas cosas; y desde entonces comenzó a edificar ciudades, hasta fundar siete ciudades. Y estos son los nombres de las ciudades: El nombre de la primera ciudad según el nombre de su hijo Enoc. El nombre de la segunda ciudad Mauli, y de la tercera Leeth, y el nombre de la cuarta Teze, y el nombre de la quinta Iesca; el nombre del sexto Celeth, y el nombre del séptimo Iebbath.

\par 4 Y Caín vivió, después de engendrar a Enoc, 715 años y engendró 3 hijos y 2 hijas. Y estos son los nombres de sus hijos: Olad, Lizaph, Fosal; y de sus hijas Cita y Maac. Y fueron todos los días de Caín 730 años, y murió.

\par 5 Entonces Enoc tomó una mujer de las hijas de Set, la cual le dio a luz a Ciram, Cuuth y Madab. Pero Ciram engendró a Matusael, y Matusael engendró a Lamec.

\par 6 Pero Lamec tomó para sí dos esposas: el nombre de una era Ada y el nombre de la otra Sella.

\par 7 Y Ada dio a luz a Yobab, padre de todos los que habitan en tiendas y de todos los que pastorean rebaños. Y de nuevo dio a luz a Iobal, que fue el primero en enseñar todo tipo de instrumentos (lit. cada salmo de órganos).

\par 8 Y en aquel tiempo, cuando los habitantes de la tierra comenzaron a hacer el mal, cada uno con la mujer de su prójimo, profanándolos, Dios se enojó. Y comenzó a tocar el laúd (kinnor) y el arpa y todo instrumento de dulce salmodia (lit. salterio), y a corromper la tierra.

\par 9 Pero Sella dio a luz a Tubal, a Misa y a Theffa, y este es aquel Tubal que mostró a los hombres las artes del plomo, del estaño, del hierro, del cobre, de la plata y del oro; y entonces los habitantes de la tierra comenzaron a hacer imágenes talladas y a adorar. a ellos.

\par 10 Entonces Lamec dijo a sus dos esposas, Ada y Sella: Oíd mi voz, esposas de Lamec, prestad atención a mi precepto; porque he corrompido a los hombres para mí, y he quitado a los niños de pecho para poder mostrarles. hijos míos cómo hacer el mal, y los habitantes de la tierra. Y ahora será castigado siete veces a Caín, y a Lamec setenta veces siete.

\chapter{3}

\par 1 Y aconteció que cuando los hombres comenzaron a multiplicarse sobre la tierra, les nacieron hermosas hijas. Y vieron los hijos de Dios que las hijas de los hombres eran sumamente hermosas, y tomaron para sí mujeres de entre todas las que habían escogido.

\par 2 Y dijo Dios: Mi espíritu no juzgará jamás entre estos hombres, porque son de carne; pero sus años serán 120. Sobre quienes él puso (o en quienes he puesto) los confines del mundo, y en sus manos no fueron apagadas las maldades (ni la ley no será apagada).

\par 3 Y vio Dios que en todos los habitantes de la tierra se cumplían las obras de maldad; y como todos sus días pensaban en la iniquidad, dijo Dios: Borraré al hombre y a todo lo que ha florecido sobre la tierra, porque Me arrepiento de haberlo hecho.

\par 4 Pero Noé halló gracia y misericordia ante el Señor, y estas son sus generaciones. Noé, que era un hombre justo y sin mancha en su generación, agradó al Señor. A lo cual dijo Dios: Ha llegado el tiempo de todos los hombres que habitan sobre la tierra, porque sus obras son muy malas. Hazte ahora un arca de madera de cedro, y así la harás. Su longitud será de trescientos codos, su anchura será de trescientos codos y su altura será de treinta codos. Y entrarás en el arca tú, y tu mujer, y tus hijos, y las mujeres de tus hijos contigo. Y haré mi pacto contigo, de destruir a todos los moradores de la tierra. Ahora bien, de los animales limpios y de las aves del cielo que sean limpias, tomarás de siete en siete, machos y hembras, para que su descendencia se mantenga viva sobre la tierra. Pero de los animales inmundos y de las aves te tomarás de dos en dos, macho y hembra, y harás provisión para ti y también para ellos.

\par 5 Y Noé hizo lo que Dios le había ordenado y entró en el arca, él y todos sus hijos con él. Y aconteció que después de 7 días el agua del diluvio comenzó a caer sobre la tierra. Y en aquel día se abrieron todos los abismos y el gran manantial de agua y las ventanas de los cielos, y hubo lluvia sobre la tierra durante cuarenta días y cuarenta noches.

\par 6 Y era entonces el año 1652 (1656) desde el momento en que Dios había hecho los cielos y la tierra en el día en que la tierra se corrompió con sus habitantes a causa de la iniquidad de sus obras.

\par 7 Y cuando el diluvio duró 140 días sobre la tierra, sólo Noé y los que estaban con él en el arca quedaron con vida; y cuando Dios se acordó de Noé, hizo que el agua disminuyera.

\par 8 Y aconteció que al nonagésimo día Dios secó la tierra y dijo a Noé: Sal del arca, tú y todos los que están contigo, y crece y multiplícate sobre la tierra. Y salió Noé del arca, él y sus hijos y las mujeres de sus hijos, y todos los animales, reptiles, aves y ganado, sacó consigo, como Dios le había mandado. Entonces Noé edificó un altar al Señor, y tomó de todo el ganado y de las aves limpias, y ofreció holocaustos sobre el altar; y fue aceptado por el Señor como olor de reposo.

\par 9 Y dijo Dios: No volveré a maldecir la tierra por causa del hombre, porque la apariencia del corazón del hombre ha desaparecido desde su juventud. Por tanto, no volveré a destruir a todos los vivientes como lo he hecho. Pero sucederá que cuando los moradores de la tierra hayan pecado, yo los juzgaré con hambre, o con espada, o con fuego, o con pestilencia (lit. muerte), y habrá terremotos, y serán esparcidos en lugares deshabitados. (o, los lugares de su habitación serán esparcidos). Pero no volveré a arruinar la tierra con aguas de diluvio, y en todos los días de la tierra, la siembra y la cosecha, el frío y el calor, el verano y el otoño, el día y la noche, no cesarán hasta que me acuerde de los que habitan en la tierra, hasta que los tiempos se cumplan.

\par 10 Pero cuando se cumplan los años del mundo, entonces cesará la luz y se apagarán las tinieblas; y yo daré vida a los muertos y resucitaré de la tierra a los que duermen; y el infierno pagará su deuda y la destrucción dará devolver lo que le ha sido encomendado, para pagar a cada uno según sus obras y según el fruto de sus imaginaciones, hasta juzgar entre el alma y la carne. Y el mundo descansará, la muerte será apagada y el infierno cerrará su boca. Y la tierra no será sin nacimiento, ni estéril para los que en ella habitan; y ninguno de los que en mí han sido justificados será contaminado. Y habrá otra tierra y otro cielo, una habitación eterna.

\par 11 Y el Señor habló más a Noé y a sus hijos, diciendo: He aquí, haré mi pacto contigo y con tu descendencia después de ti, y no volveré a arruinar la tierra con el agua de un diluvio. Y todo lo que en él vive y se mueve, os será para comida. Sin embargo, no comeréis la carne ni la sangre del alma. Porque el que derrama sangre de hombre, su sangre será derramada; porque a imagen de Dios fue hecho el hombre. Y vosotros, creced y multiplicaos y llenad la tierra como la multitud de peces que se multiplican en las aguas. Y dijo Dios: Este es el pacto que he hecho entre mí y vosotros; y sucederá que cuando cubra el cielo con nubes, mi arco aparecerá en las nubes, y será para un memorial del pacto entre yo y tú, y todos los moradores de la tierra.

\chapter{4}

\par 1 Los hijos de Noé que salieron del arca fueron Sem, Cam y Jafet.

\par 2 Los hijos de Jafet: Gomer, Magog, Madai, Nidiazec, Tubal, Mocteras, Cenez, Rifat, Togorma, Elisa, Dessin, Cetin y Tudant.

\par Los hijos de Gomer fueron Thelez, Lud y Deberlet.

\par Los hijos de Magog: Cesé, Tifá, Faruta, Amiel, Fimei, Goloza y Samanac.

\par Y los hijos de Duden: Sallus, Phelucta Phallita.

\par Y los hijos de Tubal: Fanatonova, Eteva.

\par Los hijos de Tiras fueron Maac, Tabel, Ballana, Samplameac y Elaz.

\par Los hijos de Melec: Amboradat, Urach y Bosara.

\par Y los hijos de [As]cenez: Jubal, Zaraddana, Anac.

\par Los hijos de Heri fueron Phuddet, Doad, Dephadzeat y Enoc.

\par Los hijos de Togorma fueron Abiud, Safat, Asapli y Zepthir.

\par Y los hijos de Elisa: Etzaac, Zenez, Mastisa, Rira.

\par Y los hijos de Zepti: Macziel, Temna, Aela y Phinon.

\par Y los hijos de Tessis: Meccul, Somorgujo y Zelataban.

\par Los hijos de Duodennin fueron Iteb, Beat y Fenech.

\par 3 Estos son los que se dispersaron y habitaron en la tierra con los persas y los medos, y en las islas que están en el mar. Y Fenech, hijo de Dudeni, subió y ordenó que se construyeran barcos de mar: y entonces se dividió la tercera parte de la tierra.

\par 4 Domeret y sus hijos tomaron a Ladec; y Magog y sus hijos tomaron a Degal; La señora y sus hijos se llevaron a Besto; Iuban (sc. Javan) y sus hijos tomaron Ceel; Tubal y sus hijos tomaron Feed; Misech y sus hijos tomaron Nepthi; [T]iras y sus hijos tomaron [Rôô]; Duodennut y sus hijos tomaron a Goda; Rifat y sus hijos tomaron a Bosarra; Torgoma y sus hijos tomaron a Fud; Elisa y sus hijos tomaron Thabola; Tesis (sc. Tarsis) y sus hijos tomaron Marecham; Cetim y sus hijos tomaron Thaan; Dudennin y sus hijos tomaron Caruba.

\par 5 Y entonces comenzaron a labrar la tierra y a sembrar en ella; y cuando la tierra tuvo sed, sus habitantes clamaron al Señor y él los escuchó y hizo llover abundantemente, y así fue, cuando la lluvia descendió sobre la tierra, que el arco apareció en la nube, y los moradores de la tierra vieron el memorial del pacto y se postraron sobre sus rostros y sacrificaron, ofreciendo holocaustos al Señor.

\par 6 Los hijos de Cam fueron Chus, Mestra, Phuni y Canaán.

\par Y los hijos de Chus: Saba y . . . Tudán.

\par Y los hijos de Phuni: [Effuntenus], Zeleutelup, Geluc, Lephuc.

\par Los hijos de Canaán fueron Sidona, Endain, Racin, Simmin, Uruin, Nenugin, Amathin, Nephiti, Telaz, Elat y Cusin.

\par 7 Y Chus engendró a Nembrot. Comenzó a enorgullecerse ante el Señor.

\par Pero Mestram engendró a Ludin, Megimin, Labin, Latuin, Petrosonoin y Ceslun: de allí surgieron los filisteos y los capadocios.

\par 8 Y entonces comenzaron también a construir ciudades, y estas son las ciudades que construyeron: Sidón y sus alrededores, es decir, Resun, Beosa, Maza, Gerara, Ascalón, Dabir, Camo, Tellun, Lacis, Sodoma y Gomorra, Adama y Seboim.

\par 9 Los hijos de Sem: Elam, Asur, Arfaxa, Luzi y Aram. Y los hijos de Aram: Gedrum, Ese. Y Arfaxa engendró a Sale, Sale engendró a Heber, y a Heber le nacieron dos hijos: el nombre de uno era Falech, porque en sus días fue dividida la tierra, y el nombre de su hermano era Jectan.

\par 10 Y Jectan engendró a Helmadam, Salastra, Mazaam, Rea, Dura, Uzia, Deglabal, Mimoel, Sabthfin, Evilac y Iubab.

\par Los hijos de Falej: Ragau, Refut, Zepheram, Aculon, Sachar, Sifaz, Nabi, Suri, Seciur, Falaco, Rafo, Faltía, Zaldefal, Zaphis, Artemán y Helifas. Estos son los hijos de Falech, y estos son sus nombres, y tomaron mujeres de las hijas de Jectan y engendraron hijos e hijas y llenaron la tierra.

\par 11 Pero Ragau lo tomó por esposa, Melca, hija de Rut, y ella engendró a Seruc. Y cuando llegó el día de su parto, dijo: De este niño nacerá en la cuarta generación uno que pondrá su morada en alto, y será llamado perfecto e inmaculado, y será padre de naciones, y su Su pacto no será roto, y su descendencia se multiplicará para siempre.

\par 12 Y Ragau vivió, después que engendró a Seruch, 119 años y engendró 7 hijos y 5 hijas. Y estos son los nombres de sus hijos: Abiel, Obed, Salma, Dedasal, Zeneza, Accur, Nefes. Y estos son los nombres de sus hijas: Cedema, Derisa, Seipha, Pherita, Theila.

\par 13 Y vivió Seruc veintinueve años y engendró a Nacor. Y Seruc vivió, después que engendró a Nacor, 67 años y engendró 4 hijos y 3 hijas. Y estos son los nombres de los varones: Zela, Zoba, Dica y Phodde. Y estas son sus hijas: Tephila, Oda, Selipha.

\par 14 Najor vivió treinta y cuatro años y engendró a Tara. Y Nacor vivió, después que engendró a Thara, 200 años y engendró 8 hijos y 5 hijas. Y estos son los nombres de los varones: Recap, Dediap, Berechap, Iosac, Sithal, Nisab, Nadab, Camoel. Y estas son sus hijas: Esca, Thipha, Bruna, Ceneta.

\par 15 Y Tara vivió setenta años y engendró a Abram, Nacor y Aram. Y Aram engendró a Loth.

\par 16 Entonces los habitantes de la tierra comenzaron a mirar las estrellas, y comenzaron a pronosticar con ellas, a hacer adivinaciones y a hacer pasar a sus hijos e hijas por el fuego. Pero Seruc y sus hijos no anduvieron conforme a ellos.

\par 17 Y estas son las generaciones de Noé sobre la tierra según sus lenguas y sus tribus, de las cuales se dividieron las naciones sobre la tierra después del diluvio.

\chapter{5}

\par 1 Entonces vinieron los hijos de Cham y pusieron a Nembrot por príncipe sobre ellos; pero los hijos de Jafet pusieron a Fenec por jefe; y los hijos de Sem se reunieron y pusieron sobre ellos a Jectan como príncipe.

\par 2 Y cuando estos tres se reunieron, decidieron mirar y tener en cuenta al pueblo de sus seguidores. Y esto aconteció cuando Noé aún vivía, que todos los hombres se reunieran; y vivían uno con el otro, y la tierra estaba en paz.

\par 3 En el año 340 de la salida de Noé del arca, después que Dios secó el diluvio, los príncipes tomaron en cuenta a su pueblo.

\par 4 Y Fenec, hijo de Jafet, los miró primero.

\par Los hijos de Gomer, todos los que pasaron según los cetros de sus capitanías, fueron contados cinco mil ochocientos.

\par Pero de los hijos de Magog, todos los que pasaron con los cetros que llevaban al frente, fueron 6.200.

\par De los hijos de Madai, todos los que pasaron según los cetros de sus capitanías, fueron cinco mil setecientos.

\par Y los hijos de Tubal. todos los que pasaban según los cetros de sus capitanías, eran en número de nueve mil cuatrocientos.

\par Y los hijos de Mesca, todos los que pasaron según los cetros de sus capitanías, fueron contados cinco mil seiscientos.

\par Los hijos de Thiras, todos los que pasaron según los cetros de sus capitanías, fueron contados 12.300.

\par Y los hijos de Ripha[t] que pasaron con los cetros de sus capitanías fueron en número de 14.500.

\par Y los hijos de Thogorma que pasaron según los cetros de su capitanía fueron 14.400.

\par Pero los hijos de Elisa que pasaron con los cetros de su capitanía fueron 14.900.

\par Los hijos de Tersis, todos los que pasaron según los cetros de su capitanía, fueron contados 12.100.

\par Los hijos de Cethin, todos los que pasaron según los cetros de su capitanía, fueron contados 17.300.

\par Los hijos de Doin que pasaron según los cetros de sus capitanías fueron contados 17.700.

\par Y el número del campamento de los hijos de Jafet, todos ellos hombres valientes y todos armados, que estaban delante de sus capitanes, fue de 140.202, sin contar las mujeres y los niños.

\par La cuenta completa de Jafet fue de 142.000.

\par 5 Y pasó Nembrot, él y los hijos de Cham, todos ellos pasando según los cetros de sus capitanías, fueron hallados en número de 24.800.

\par Los hijos de Phua, todos los que pasaron según los cetros de sus capitanías, fueron contados veintisiete mil setecientos.

\par Y los hijos de Canaán, todos ellos pasando según los cetros de sus capitanías, fueron hallados en número de 32.800.

\par Los hijos de Soba, todos ellos pasando según los cetros de sus capitanías, fueron hallados en número de cuatro mil trescientos.

\par Los hijos de Lebilla todos ellos pasando según los cetros de sus capitanías, fueron hallados en número de veintidós mil trescientos.

\par Y los hijos de Sata, todos ellos pasando según los cetros de sus capitanías, fueron hallados en número de veinticinco mil trescientos.

\par Y los hijos de Remma, todos ellos pasando según los cetros de sus capitanías, fueron encontrados en número de 30.600.

\par Y los hijos de Sabaca, todos ellos pasando según los cetros de sus capitanías, fueron hallados en número de cuarenta y seis mil cuatrocientos.

\par Y el número del campamento de los hijos de Cham, todos ellos hombres valientes y armados con armas, que estaban delante de sus capitanías, era de 244.900, sin contar las mujeres y los niños.

\par 6 Y Jectan hijo de Sem miró a los hijos de Elam, y todos ellos pasaban según el número de los cetros de sus capitanías en número de 47.000.

\par Y los hijos de Asur, todos ellos pasando según los cetros de sus capitanías, fueron hallados en número de setenta y tres mil.

\par Y los hijos de Aram, todos ellos pasando según los cetros de sus capitanías, fueron hallados en número de 87.300.

\par Los hijos de Lud, todos ellos pasando según los cetros de sus capitanías, fueron hallados en número de 30.600.

\par [El número de los hijos de Cham era setenta y tres mil.]

\par Pero los hijos de Arfaxat, todos los que pasaron según los cetros de sus capitanías, fueron contados 114.600.

\par Y el número total de ellos fue 347.600.

\par 7 El número del campamento de los hijos de Sem, todos ellos avanzando con valor y con orden de guerra a la vista de sus capitanías, era † ix †, sin contar las mujeres y los niños.

\par 8 Y estas son las generaciones de Noé separadas, de las cuales el número total fue de 914.000. Y todo esto fue contado mientras Noé aún vivía, y en presencia de Noé 350 años después del diluvio. Y fueron todos los días de Noé 950 años, y murió.

\chapter{6}

\par 1 Entonces todos los que estaban divididos y habitaban en la tierra se reunieron allí después y habitaron juntos; y partieron del Oriente y encontraron una llanura en tierra de Babilonia; y habitaron allí, y decían cada uno a su prójimo: He aquí, sucederá que seremos esparcidos cada uno. de su hermano, y en los últimos días estaremos peleando unos contra otros. Ahora, pues, venid y edifiquémonos una torre, cuya punta llegue al cielo, y nos haremos nombre y renombre en la tierra.

\par 2 Y dijeron cada uno a su vecino: Tomemos ladrillos (lit. piedras), y escribamos cada uno nuestros nombres en los ladrillos y quememos al fuego; y lo que esté completamente quemado será para mortero. y ladrillo. (Tal vez, lo que no esté completamente quemado será para mortero, y lo que sí, para ladrillo.)

\par 3 Y tomaron cada uno sus ladrillos, salvo 12 hombres que no quisieron tomarlos, y estos son sus nombres: Abraham, Nachor, Loth, Ruge, Tenute, Zaba, Armodat, Iobab, Esar, Abimahel, Saba, Auphin.

\par 4 Y la gente del país les echó mano y los llevó ante sus príncipes y dijo: Estos son los hombres que han transgredido nuestros consejos y no andarán en nuestros caminos. Y los príncipes les dijeron: ¿Por qué no habéis puesto cada uno vuestros ladrillos con la gente de la tierra? Y ellos respondieron y dijeron: No pondremos ladrillos contigo, ni nos uniremos a tu deseo. Un Señor conocemos y a él adoramos. Y si nos echáis al fuego con vuestros ladrillos, no os lo consentiremos.

\par 5 Y los príncipes se enojaron y dijeron: Haz con ellos como han dicho, y si no aceptan poner ladrillos contigo, los quemarás al fuego junto con tus ladrillos.

\par 6 Entonces Jectan, que era el primer príncipe de los capitanes, respondió: No es así, sino que se les dará un plazo de siete días. Y sucederá que si se arrepienten de sus malos consejos y ponen ladrillos junto a nosotros, vivirán; pero si no, que sean quemados según tu palabra. Pero buscaba cómo salvarlos de las manos del pueblo; porque él era de su tribu y servía a Dios.

\par 7 Y habiendo dicho esto, los tomó y los encerró en la casa del rey; y al anochecer, el príncipe mandó que llamaran a cincuenta hombres valientes y les dijo: Salid y tomad -Noche estos hombres que están encerrados en mi casa, y ponen provisiones para ellos desde mi casa sobre 10 bestias, y los hombres traéis a mí, y su provisión junto con las bestias os lleváis a las montañas y esperáis allí: y sabed esto, que si alguno supiere lo que os he dicho, os quemaré en el fuego.

\par 8 Y los hombres partieron e hicieron todo lo que su príncipe les ordenó, y los sacaron de su casa por la noche; y tomó provisiones y las puso sobre las bestias y las llevó a la montaña como él les mandó.

\par 9 Y el príncipe llamó a aquellos doce hombres y les dijo: Tened ánimo y no temáis, porque no moriréis. Porque Dios en quien confiáis es poderoso, y por tanto estad confirmados en él, porque él os librará y salvará. Y ahora he aquí, he mandado a unos hombres que te saquen con provisiones de mi casa, y vayan delante de ti a la región montañosa y te esperen en el valle; y te daré otros cincuenta hombres que te guiarán allí: Id, pues, y escondeos allí en el valle, tomando para beber el agua que brota de las peñas; permaneced allí treinta días, hasta que se aplaque la ira del pueblo de la tierra, y hasta que Dios envíe sobre ellos su ira y los rompa. a ellos. Porque sé que el consejo de iniquidad que acordaron ejecutar no permanecerá, porque su pensamiento es vano. Y será que cuando se cumplan 7 días y os busquen, les diré: Han salido y han roto la puerta de la cárcel donde estaban encerrados y han huido de noche, y he enviado 100 hombres a buscarlos. Así los apartaré de la locura que los sobreviene.

\par 10 Y once de los hombres le respondieron, diciendo: Tus siervos han hallado gracia ante tus ojos, al liberarnos de las manos de estos hombres orgullosos.

\par 11 Pero Abram se limitó a guardar silencio, y el príncipe le dijo: ¿Por qué no me respondes, Abram, siervo de Dios? Abram respondió y dijo: He aquí, hoy huyo a la región montañosa, y si escapo del fuego, saldrán fieras de las montañas y nos devorarán. O nos faltarán víveres y moriremos de hambre; y seremos encontrados huyendo del pueblo de la tierra y caeremos en nuestros pecados. Y ahora, viviendo aquel en quien confío, no me moveré del lugar en que me han puesto; y si en algún pecado mío fuere quemado, hágase la voluntad de Dios. Y el príncipe le dijo: Tu sangre sea sobre tu cabeza, si rehúsas salir con estos. Pero si consientes, serás liberado. Sin embargo, si quieres permanecer, permanece como eres. Y Abram dijo: No saldré, sino que quedaré aquí.

\par 12 Y el príncipe tomó a aquellos once hombres y envió con ellos a otros cincuenta, y les ordenó diciendo: Esperad también vosotros quince días en la región montañosa con los cincuenta que fueron enviados antes que vosotros; y después volveréis y diréis: No los hemos encontrado, como dije a los primeros. Y sabed que si alguno transgrede alguna de todas estas palabras que os he hablado, será quemado en el fuego. Entonces los hombres salieron, tomó a Abram solo y lo encerró donde había estado encerrado antes.

\par 13 Y transcurridos siete días, el pueblo se reunió y habló a su príncipe, diciendo: Devuélvenos a los hombres que no quisieron aceptarnos, para que los quememos con fuego. Y enviaron capitanes para traerlos, y no los encontraron, excepto Abram solo. Y los reunieron todos junto a su príncipe, diciendo: Los hombres que vosotros encerrasteis han huido y han escapado de lo que habíamos aconsejado.

\par 14 Y Phenech y Nemroth dijeron a Jectan: ¿Dónde están los hombres a quienes encerraste? Pero él dijo: Han roto la prisión y han huido de noche; pero he enviado 100 hombres a buscarlos, y les he ordenado que, si los encuentran, no sólo los quemen con fuego sino que entreguen sus cuerpos a las aves del cielo y así que destrúyelos.

\par 15 Entonces dijeron: A este hombre que se encuentre solo, quememoslo. Y tomaron a Abram y lo llevaron ante sus príncipes y le dijeron: ¿Dónde están los que estaban contigo? Y él dijo: En verdad dormí de noche, y cuando desperté no los encontré.

\par 16 Lo tomaron y construyeron un horno, lo encendieron y pusieron en el horno ladrillos cocidos al fuego. Entonces el príncipe Jectan, asombrado (lit. derretido) en su mente, tomó a Abram y lo metió con los ladrillos en el horno de fuego.

\par 17 Pero Dios provocó un gran terremoto, y el fuego brotó del horno y se convirtió en llamas y chispas de fuego, y consumió a todos los que estaban alrededor a la vista del horno; y todos los que fueron quemados aquel día fueron 83.500. Pero Abram no sufrió el menor daño por el fuego.

\par 18 Y Abram se levantó del horno, y el horno de fuego cayó, y Abram se salvó. Y fue a los 11 hombres que estaban escondidos en la región montañosa y les contó todo lo que le había sucedido, y descendieron con él de la región montañosa regocijándose en el nombre del Señor, y nadie salió a su encuentro para asustarlos. ese día. Y llamaron aquel lugar por nombre Abram, y en lengua de los caldeos Deli, que se traduce, Dios.



\chapter{7}

\par 1 Y aconteció después de estas cosas, que el pueblo de la tierra no se apartó de sus malos pensamientos; y se reunieron nuevamente con sus príncipes y dijeron: El pueblo no será vencido para siempre; y ahora vamos a venir. juntos y edificadnos una ciudad y una torre que nunca será removida.

\par 2 Y cuando comenzaron a construir, Dios vio la ciudad y la torre que los hijos de los hombres estaban construyendo, y dijo: He aquí, este es un solo pueblo y su discurso es uno, y esto que han comenzado a construir la tierra no la sustentará, ni el cielo la tolerará al contemplarla; y será, si ahora no se les impide, que se atreverán a todo lo que se propongan hacer.

\par 3 Por eso, he aquí, dividiré su lengua y los esparciré por todos los países, para que no conozcan cada uno a su hermano, ni cada uno entienda la lengua de su prójimo. Y los entregaré a las peñas, y se construirán tabernáculos de rastrojo y de paja, y cavarán cuevas y habitarán en ellas como bestias del campo, y así permanecerán delante de mí para siempre, para que nunca idear tales cosas. Y los estimaré como gota de agua, y los compararé a saliva: y a unos de ellos les llegará el fin por el agua, y a otros se secarán de sed.

\par 4 Y delante de todos elegiré a mi siervo Abram, lo sacaré de su tierra y lo conduciré a la tierra que mis ojos han mirado desde el principio, cuando todos los habitantes de la tierra pecaron ante mí. , y traje sobre ellos las aguas del diluvio; y entonces no destruí aquella tierra, sino que la conservé. Por tanto, no brotaron en él las fuentes de mi ira, ni descendió sobre él el agua de mi destrucción. Porque allí haré habitar a mi siervo Abram, y haré con él mi pacto, y bendeciré su descendencia, y seré llamado su Dios para siempre.

\par 5 Pero cuando el pueblo que habitaba en la tierra comenzó a construir la torre, Dios dividió su discurso y cambió su semejanza. Y no conocían cada uno a su hermano, ni cada uno entendía la lengua de su prójimo. Y aconteció que cuando los constructores mandaban a sus ayudantes que trajeran ladrillos, ellos traían agua, y si pedían agua, los demás les traían paja. Y así su consejo fue roto y cesaron de edificar la ciudad: y Dios los dispersó desde allí sobre la faz de toda la tierra. Por eso se llamó aquel lugar Confusión, porque allí Dios confundió su habla, y desde allí los dispersó sobre la faz de toda la tierra.

\chapter{8}

\par 1 Pero Abram salió de allí y habitó en la tierra de Canaán, y tomó consigo a Loth, el hijo de su hermano, y a Sarai, su esposa. Y como Sarai era estéril y no tenía descendencia, Abram tomó a Agar su sierva, y ella le dio a luz a Ismahel. E Ismahel engendró doce hijos.

\par 2 Entonces Loth partió de Abram y habitó en Sodoma (pero Abram habitó en la tierra de Cam). Y los hombres de Sodoma eran muy malvados y pecadores en gran manera.

\par 3 Y se apareció Dios a Abraham, diciendo: A tu descendencia daré esta tierra; y tu nombre se llamará Abraham, y Sarai tu esposa se llamará Sara. Ana te daré de ella una simiente eterna y haré mi pacto contigo. Y conoció Abraham a Sara su mujer, la cual concibió y dio a luz a Isaac.

\par 4 Isaac tomó para sí una mujer mesopotámica, hija de Batuel, la cual concibió y dio a luz a Esaú y a Jacob.

\par 5 Y Esaú tomó por esposas a Judín hija de Bereu, a Basemat hija de Elón, a Elibema hija de Anan y a Manem hija de Samael.

\par Y [Basemath] dio a luz a Adelifan, y los hijos de Adelifan fueron Temar, Omar, Seffor, Getan, Tenaz y Amalec. Y Judín dio a luz a Tenacis, Ieruebemas, Bassemen y Rugil; y los hijos de Rugil fueron Naizar y Samaza; y Elibema desnuda Auz, Iollam, Coro.

\par Manem desnudo Tenetde, Thenatela.

\par 6 Y Jacob tomó por esposas a las hijas del general de Labán el sirio, Lía y Raquel, y dos concubinas, Bala y Zelfa. Y Lía le dio a luz a Rubén, Simeón, Leví, Judá, Isacar, Zabulón y Dina su hermana.

\par Pero Raquel dio a luz a José y a Benjamín.

\par Bala dio a luz a Dan y a Neptalim, y Zelfa dio a luz a Gad y a Aser.

\par Estos son los doce hijos de Jacob y una hija.

\par 7 Y Jacob habitó en la tierra de Canaán, y Siquem, hijo de Emor el correano, forzó a su hija Dina y la humilló. Y Simeón y Leví, hijos de Jacob, entraron y mataron a toda su ciudad a filo de espada, y tomaron a Dina su hermana, y salieron de allí.

\par 8 Y después Job la tomó por esposa y engendró de ella 14 hijos y 6 hijas, es decir, 7 hijos y 3 hijas antes de ser herido por la aflicción, y después, cuando quedó sano, 7 hijos y 3 hijas. Y estos son sus nombres: Eliphac, Erinoe, Diasat, Philias, Diffar, Zellud, Thelon; y sus hijas Meru, Litaz, Zeli. Y como eran los nombres de los primeros, así eran también los de los segundos.

\par 9 Jacob y sus doce hijos habitaban en la tierra de Canaán; y sus hijos odiaban a su hermano José, a quien también entregaron a Egipto, a Petefres, jefe de los cocineros de Faraón, y permaneció con él catorce años.

\par 10 Y aconteció que después que el rey de Egipto tuvo un sueño, le hablaron de José, y él le contó los sueños. Y aconteció que después que contó sus sueños, Faraón lo hizo príncipe sobre toda la tierra de Egipto. En aquel tiempo hubo hambre en toda la tierra, tal como José lo había previsto. Y sus hermanos descendieron a Egipto a comprar alimentos, porque sólo en Egipto había alimentos. Y José conoció a sus hermanos, y se les dio a conocer, y no hizo mal con ellos. Y envió y llamó a su padre de la tierra de Canaán, y descendió a él.

\par 11 Estos son los nombres de los hijos de Israel que descendieron con Jacob a Egipto, cada uno con su casa. Los hijos de Rubén, Enoc y Falud, Esrom y Carmín; los hijos de Simeón, Namuhel, Iamin, Dot y Iachin, y Saúl, hijo de una mujer cananea.

\par Los hijos de Leví, Gersón, Caat y Merari; pero los hijos de Judá, Auna, Selón, Fares y Zerami.

\par Los hijos de Isacar, Tola y Phua, Job y Sombram. Los hijos de Zabulón, Sarelón y Iaillil. Y Dina su hermana dio a luz 14 hijos y 6 hijas.

\par Y éstas son las generaciones de Lía que ella dio a luz a Jacob. Todas las almas de hijos e hijas fueron 72.

\par 12 Los hijos de Dan fueron Usinam. Los hijos de Neptalim, Betaal, Neemmu, Surem, Optisariel. Y estas son las generaciones de Balla que ella
\par desnudo para Jacob. Todas las almas eran 8.

\par 13 Pero los hijos de Gad: . . . Sariel, Sua, Visui, Mophat y Sar: su hermana la hija de Seriebel, Melchiel. Estas son las generaciones de Zelfa, la esposa de Jacob, que ella le dio a luz. Y todas las almas de hijos e hijas estaban en número 10.

\par 14 Los hijos de José fueron Efraín y Manasén. Benjamín engendró a Gela, Esbel, Abocmefec y Utundeus. Y estas fueron las almas que Raquel dio a luz a Jacob, 14.

\par Y descendieron a Egipto y permanecieron allí doscientos diez años.

\chapter{9}

\par 1 Y aconteció que después de la partida de José, los hijos de Israel se multiplicaron y crecieron mucho. Y se levantó en Egipto otro rey que no conocía a José, y dijo a su pueblo: He aquí, este pueblo se ha multiplicado más que nosotros. Venid, consultemos contra ellos para que no se multipliquen. Y el rey de Egipto mandó a todo su pueblo diciendo: Todo hijo que le naciere, échalo al río, pero mantén vivas a las hembras. Y los egipcios respondieron a su rey, diciendo: Matemos a sus machos y conservemos a sus hembras, para darlas a nuestros siervos por esposas; y el que nazca de ellos será siervo y nos servirá. Y esto es lo que parecía más malvado ante el Señor.

\par 2 Entonces los ancianos del pueblo reunieron al pueblo en duelo, y se lamentaron y se lamentaron, diciendo: Un parto prematuro ha sufrido el vientre de nuestras esposas. Nuestro fruto ha sido entregado a nuestros enemigos y ahora estamos exterminados. Sin embargo, establezcamos una ordenanza para que nadie se acerque a su esposa, no sea que el fruto de su vientre se contamine y nuestras entrañas sirvan a los ídolos; porque es mejor morir sin hijos, hasta que sepamos lo que Dios hará.

\par 3 Y Amram respondió y dijo: Más pronto sucederá que la era será completamente abolida y el mundo inconmensurable caerá, o el corazón de las profundidades tocará las estrellas, que que la raza de los hijos de Israel disminuya. Y será, cuando se cumpla el pacto que Dios hizo cuando habló a Abraham, diciendo: Ciertamente tus hijos habitarán en tierra ajena, y serán sometidos a servidumbre y aflicción por 400 años.—Y he aquí, desde entonces se pasó la palabra que Dios habló a Abraham, son 350 años. (Y) desde que hemos estado en esclavitud en Egipto han pasado 130 años.

\par 4 Ahora, pues, no cumpliré lo que vosotros determináis, sino que entraré, tomaré a mi mujer y engendraré hijos, para que seamos muchos en la tierra. Porque Dios no continuará en su ira, ni se olvidará para siempre de su pueblo, ni desechará a la raza de Israel sobre la tierra, ni en vano hizo su pacto con nuestros padres; sí, cuando aún no éramos , Dios habló de estas cosas.

\par 5 Ahora, pues, iré y tomaré a mi mujer, y no consentiré en el mandamiento de este rey. Y si os parece bien, así hagamos todos nosotros, porque sucederá que cuando nuestras mujeres conciban, no se considerarán que están encinta hasta que se cumplan tres meses, como también lo hizo nuestra madre Tamar. , porque su intención no era fornicar, sino que como no quería separarse de los hijos de Israel, pensó y dijo: Mejor me es morir por pecar con mi suegro, que unirme a los gentiles. Y escondió el fruto de su vientre hasta el tercer mes, porque entonces era percibido. Y cuando iba a ser ejecutada, lo afirmó diciendo: El hombre de quien es este báculo y este anillo y esta piel de cabra, de él he concebido. Y su dispositivo la liberó de todo peligro.

\par 6 Ahora pues, hagamos también nosotros lo mismo. Y será que cuando llegue el tiempo de dar a luz, si es posible, no echaremos el fruto de nuestro vientre. ¿Y quién sabe si con ello Dios será provocado para librarnos de nuestra humillación?

\par 7 Y la palabra que Amram tenía en su corazón fue agradable ante Dios, y Dios dijo: Porque el pensamiento de Amram es agradable ante mí, y él no ha anulado el pacto hecho entre yo y sus padres, por eso, he aquí. ahora, lo que de él es engendrado me servirá para siempre, y por él haré prodigios en la casa de Jacob, y haré por él señales y prodigios para mi pueblo que no he hecho con ningún otro, y haré en ellos mi gloria y les declararé mis caminos.

\par 8 Yo, el Señor, le encenderé mi lámpara para que more en él, le mostraré mi alianza, que ningún hombre ha visto, le manifestaré mi grandeza, mi justicia y mis juicios, y le haré brillar como luz perpetua. Porque en la antigüedad pensaba en él, diciendo: Mi espíritu no será mediador entre estos hombres para siempre, porque son carne, y sus días serán 120 años.

\par 9 Y Amram de la tribu de Leví salió y tomó una esposa de su tribu, y cuando él la tomó, los demás siguieron a él y tomaron sus esposas. Ahora tenía un hijo y una hija, y sus nombres eran Aarón y María.

\par 10 Y el espíritu de Dios vino sobre María de noche, y ella tuvo un sueño, y por la mañana contó a sus padres, diciendo: Vi esta noche, y he aquí un hombre vestido con una túnica de lino se puso de pie y me dijo: Ve y Di a tus padres: He aquí, lo que nacerá de ti será arrojado al agua, porque por él se secará el agua, y por él haré señales, y salvaré a mi pueblo, y él tendrá la capitanía. del mismo siempre. Y cuando María contó su sueño sus padres no le creyeron.

\par 11 Pero la palabra del rey de Egipto prevaleció contra los hijos de Israel, y fueron humillados y oprimidos en el trabajo de los ladrillos.

\par 12 Pero Jocabeth concibió a Amram y escondió al niño en su vientre durante tres meses, porque no podía ocultarlo por más tiempo, porque el rey de Egipto había designado supervisores de la región, para que cuando las mujeres hebreas dieran a luz, echaran a los varones. al río inmediatamente. Y tomó a su hijo, le hizo un arca con corteza de pino y la puso a la orilla del río.

\par 13 El niño nació en el pacto de Dios y en el pacto de su carne.

\par 14 Y aconteció que cuando lo echaron fuera, todos los ancianos se reunieron y discutieron con Amram, diciendo: ¿No son éstas las palabras que dijimos, diciendo: «Es mejor para nosotros morir sin hijos que que nuestro fruto se deshaga de nosotros?». ¿Ser arrojado al agua? Y cuando dijeron esto, Amram no los escuchó.

\par 15 Pero la hija de Faraón descendió a lavarse al río como había visto en un sueño, y sus criadas vieron el arca, y envió a una de ellas, la tomó y la abrió. Y cuando vio al niño y miró el pacto, es decir, el testamento en su carne, dijo: Es de los hijos de los hebreos.

\par 16 Ella lo tomó y lo crió, y él fue su hijo, y llamó su nombre Moisés. Pero su madre lo llamó Melquiel. Y el niño fue nutrido y llegó a ser glorioso sobre todos los hombres, y por él Dios libró a los hijos de Israel, como había dicho.

\chapter{10}

\par 1 Cuando murió el rey de Egipto, se levantó otro rey y afligió a todo el pueblo de Israel. Pero ellos clamaron a Jehová y él los escuchó, y envió a Moisés y los libró de la tierra de Egipto; y Dios envió también sobre ellos diez plagas y los hirió. Estas fueron las plagas, a saber, sangre, ranas y toda clase de moscas, granizo, muerte de ganado, langostas y mosquitos, tinieblas que se podían sentir y la muerte de los primogénitos.

\par 2 Cuando salieron de allí y se pusieron en camino, el corazón de los egipcios se endureció nuevamente y continuaron persiguiéndolos, y los encontraron junto al mar Rojo. Y los hijos de Israel clamaron a su Dios y hablaron a Moisés diciendo: He aquí, ahora ha llegado el tiempo de nuestra destrucción, porque el mar está delante de nosotros y la multitud de enemigos detrás de nosotros, y nosotros en medio. ¿Para esto nos sacó Dios, o son estos los pactos que hizo con nuestros padres diciendo: A vuestra descendencia daré la tierra en la que habitáis? y ahora haga con nosotros lo que bien le parezca.

\par 3 Entonces los hijos de Israel dividieron sus consejos en tres divisiones, a causa del temor del tiempo. Por la tribu de Rubén y de Isacar y. de Zabulón y de Simeón dijeron: Venid, echémonos al mar, porque es mejor para nosotros morir en el agua que ser asesinados por nuestros enemigos. Y la tribu de Gad y de Aser y de Dan y Neptalim dijeron: No, pero volvamos con ellos, y si nos dan la vida, les serviremos. Pero la tribu de Leví, de Judá, de José y de la tribu de Benjamín dijeron: No así, sino que tomemos nuestras armas y peleemos contra ellos, y Dios estará con nosotros.

\par 4 Moisés también clamó al Señor y dijo: Oh Señor, Dios de nuestros padres, ¿no me dijiste: Ve y diles a los hijos de Lía: Dios me ha enviado a ti? Y ahora, he aquí, has llevado a tu pueblo al borde del mar, y el enemigo los sigue; pero tú, Señor, recuerda tu nombre.

\par 5 Y dijo Dios: Por cuanto has clamado a mí, toma tu vara y golpea el mar, y se secará. Y cuando Moisés hizo todo esto, Dios reprendió al mar, y el mar se secó: los mares de las aguas se detuvieron y aparecieron los abismos de la tierra, y los cimientos de la morada quedaron desnudos ante el ruido del miedo. de Dios y al soplo de la ira de mi Señor.

\par 6 E Israel pasó por tierra seca en medio del mar. Y los egipcios lo vieron y fueron tras ellos, y Dios endureció su mente, y no sabían que entraban en el mar. Y así fue que mientras los egipcios estaban en el mar, Dios mandó otra vez al mar, y dijo a Moisés: Golpea el mar una vez más. Y así lo hizo. Y el Señor ordenó al mar, y éste volvió a sus olas, y cubrió a los egipcios y sus carros y su gente de a caballo hasta el día de hoy.

\par 7 Pero a su pueblo los llevó al desierto: durante cuarenta años les hizo llover pan del cielo, les trajo codornices del mar, y detrás de ellos les sacó un pozo de agua. Y en una columna de nube los guiaba de día y en una columna de fuego de noche les alumbraba.

\chapter{11}

\par 1 Y en el tercer mes del viaje de los hijos de Israel desde la tierra de Egipto, llegaron al desierto de Sinaí. Y Dios se acordó de su palabra y dijo: Daré luz al mundo, e iluminaré los lugares habitables, y haré mi pacto con los hijos de los hombres, y glorificaré a mi pueblo sobre todas las naciones, porque a ellos les presentaré una exaltación eterna. lo cual será para ellos una luz, pero para los impíos un castigo.

\par 2 Y dijo a Moisés: He aquí, mañana te llamaré; prepárate y dile a mi pueblo: «Durante tres días ningún hombre se acercará a su mujer», y al tercer día le hablaré. ti y a ellos, y después subirás a mí. Y pondré mis palabras en tu boca y tú iluminarás a mi pueblo. Porque he puesto en tus manos una ley eterna por la cual juzgaré a todo el mundo. Porque esto servirá para testimonio. Porque si los hombres dicen: «No te hemos conocido, y por eso no te hemos servido», por eso me vengaré de ellos, porque no han conocido mi ley.

\par 3 Moisés hizo lo que Dios le había ordenado, santificó al pueblo y les dijo: Estad preparados para el tercer día, porque después de tres días Dios hará su pacto con vosotros. Y el pueblo fue santificado.

\par 4 Y aconteció que al tercer día, he aquí, se oyeron voces de truenos (es decir, los que sonaban), y resplandor de relámpagos, y sonido de instrumentos que resonaban con fuerza. Y hubo temor sobre todo el pueblo que estaba en el campamento. Y Moisés sacó al pueblo al encuentro de Dios.

\par 5 Y he aquí los montes ardieron en fuego y la tierra tembló y los collados fueron removidos y los montes derribados; los abismos hirvieron y todos los lugares habitables se estremecieron; y los cielos se plegaron y las nubes arrastraron aguas. Y brillaron llamas de fuego y se multiplicaron truenos y relámpagos y rugieron vientos y tempestades: las estrellas se juntaron y los ángeles corrieron delante, hasta que Dios estableció la ley de un pacto eterno con los hijos de Israel, y les dio un mandamiento eterno que no debe pasar.

\par 6 Y en aquel tiempo el Señor habló a su pueblo todas estas palabras, diciendo: Yo soy el Señor tu Dios, que te saqué de la tierra de Egipto, de casa de servidumbre. No te harás dioses tallados, ni te harás ninguna imagen abominable del sol o de la luna, ni de ninguno de los adornos del cielo, ni semejanza de todo lo que hay sobre la tierra, ni de los que se arrastran en las aguas. o sobre la tierra. Yo soy el Señor tu Dios, Dios celoso, que pago los pecados de los que duermen sobre los hijos vivos de los impíos, si andan en los caminos de sus padres; hasta la tercera y cuarta generación, haciendo (o haciendo) misericordia hasta 1000 generaciones a los que me aman y guardan mis mandamientos.

\par 7 No tomarás el nombre del Señor tu Dios en vano, para que mis caminos no sean en vano. Porque Dios abomina al que toma su nombre en vano.

\par 8 Guardad el día de reposo para santificarlo. Seis días haz tu trabajo, pero el séptimo día es sábado del Señor. En él no harás ningún trabajo, tú y todos tus trabajadores, salvo que en él alabéis al Señor en la congregación de los ancianos y glorifiquéis al Poderoso en el trono de los ancianos. Porque en seis días hizo el Señor los cielos y la tierra, el mar y todo lo que en ellos hay, y todo el mundo, el desierto inhabitado, y todo lo que trabaja, y todo el orden de los cielos, y Dios descansó. el séptimo día. Por eso Dios santificó el séptimo día, porque en él descansó.

\par 9 Amarás a tu padre y a mi madre y los temerás; entonces nacerá tu luz, y yo mandaré al cielo y él te pagará su lluvia, y la tierra apresurará su fruto y tus días serán muchos. , y habitarás en tu tierra, y no quedarás sin hijos, porque no faltará tu descendencia, ni siquiera la de los que en ella habitan.

\par 10 No cometerás adulterio, porque tus enemigos no cometieron adulterio contigo, pero tú saliste con mano alta.

\par 11 No matarás, porque tus enemigos no lograron dominarte para matarte, pero tú viste su muerte.

\par 12 No darás falso testimonio contra tu prójimo, hablando mentira, para que tus centinelas no hablen mentira contra ti.

\par 13 No codiciarás la casa de tu prójimo ni sus propiedades, para que otros no codicien también tu tierra.

\par 14 Y cuando el Señor terminó de hablar, el pueblo tuvo mucho miedo; y vieron el monte ardiendo con antorchas de fuego, y dijeron a Moisés: Háblanos, y no dejes que Dios nos hable, no sea que tal vez morimos. Porque he aquí, hoy sabemos que Dios habla con el hombre cara a cara, y el hombre vivirá. Y ahora hemos percibido de verdad cómo la tierra llevaba con temblor la voz de Dios. Y Moisés les dijo: No temáis, porque para esto vino a vosotros esta voz, para que no pequéis (o, para esto, para probaros, Dios vino a vosotros, para que recibáis el temor de él). a vosotros, para que no pequéis).

\par 15 Y todo el pueblo se quedó a distancia, pero Moisés se acercó a la nube, sabiendo que Dios estaba allí. Y entonces Dios le habló de su justicia y juicios, y lo guardó junto a él cuarenta días y cuarenta noches. Y allí le mandó muchas cosas, y le mostró el árbol de la vida, del cual cortó y tomó y lo metió en Mara, y el agua de Mara se endulzó y los siguió por el desierto 40 años, y subió al desierto. con ellos las colinas y descendieron a la llanura. También le ordenó acerca del tabernáculo y del arca de Jehová, y del sacrificio de los holocaustos y del incienso, y del ordenamiento de la mesa y del candelero, y acerca de la fuente y su base, y de la hombrera y del el pectoral y las piedras preciosas, para que los hijos de Israel los hicieran así; y le mostró la figura de ellos para hacerlos conforme al modelo que había visto. Y le dijo: Hazme un santuario y el tabernáculo de mi gloria estará entre ti.

\chapter{12}

\par 1 Moisés descendió y, mientras estaba cubierto de una luz invisible (pues había descendido al lugar donde brillan el sol y la luna), la luz de su rostro venció el brillo del sol y de la luna. y él no lo sabía. Y aconteció que cuando descendió a los hijos de Israel, le vieron y no le reconocieron. Pero cuando habló, entonces lo reconocieron. Y esto fue como lo que sucedió en Egipto cuando José conocía a sus hermanos pero ellos no lo conocían a él. Y aconteció después de esto, que cuando Moisés supo que su rostro había llegado a ser glorioso, le hizo un velo para cubrir su rostro.

\par 2 Pero mientras él estaba en el monte, el corazón del pueblo se corrompió y se acercaron a Aarón y le dijeron: Haznos dioses para que les sirvamos, como también lo hacen las otras naciones. Porque este Moisés, que hizo las maravillas antes que nosotros, nos ha sido quitado. Y Aarón les dijo: Tened paciencia, porque Moisés vendrá y traerá el juicio a nosotros, y nos iluminará una ley, y expondrá de su boca la gran excelencia de Dios, y establecerá juicios para nuestro pueblo.

\par 3 Y cuando dijo esto, no le escucharon, para que se cumpliera la palabra que se pronunció el día en que el pueblo pecó al construir la torre, cuando Dios dijo: Y ahora, si no se lo prohíbo, lo harán. aventura todo lo que se proponen hacer, y cosas peores. Pero Aarón tuvo miedo, porque el pueblo se fortaleció mucho, y les dijo: Traednos los zarcillos de vuestras mujeres. Y los hombres buscaron cada uno su mujer, y en seguida se las dieron, y las pusieron en el fuego y les hicieron una figura, y salió un becerro fundido.

\par 4 Y el Señor dijo a Moisés: Date prisa, porque el pueblo está corrupto y ha obrado con engaño en mis caminos que les ordené. ¿Qué y si se acaban las promesas que hice a sus padres cuando dije: A vuestra descendencia daré esta tierra en la que habitáis? Porque he aquí, el pueblo aún no ha entrado en la tierra; aunque llevan mis juicios, me han abandonado. Y por eso sé que si entran en la tierra harán iniquidades aún mayores. Ahora, pues, yo también los abandonaré, y volveré y haré paz con ellos, para que me edifiquen casa en medio de ellos; y también aquella casa será aniquilada, porque pecarán contra mí, y la raza de los hombres me será como gota de cántaro, y me será contada como saliva.

\par 5 Entonces Moisés se apresuró y descendió y vio el becerro, y miró las tablas y vio que no estaban escritas; y se apresuró y las quebró; y sus manos se abrieron y quedó como una mujer que está dando a luz a su primogénito, el cual cuando es presa de sus dolores, sus manos están sobre su seno, y no tendrá fuerzas para ayudarla a dar a luz.

\par 6 Y aconteció que al cabo de una hora decía para sí: La amargura no prevalece para siempre, ni el mal domina para siempre. Ahora, pues, me levantaré y fortaleceré mis lomos; porque aunque hayan pecado, no serán en vano estas cosas que me fueron declaradas arriba.

\par 7 Entonces se levantó, quebró el becerro, lo arrojó en el agua y dio de beber al pueblo. Y fue así: si alguno tenía la voluntad en su mente de que se hiciera el becerro, se le cortaba la lengua; pero si alguno se había visto obligado a hacerlo por el miedo, su rostro resplandecía.

\par 8 Entonces Moisés subió al monte y oró al Señor, diciendo: He aquí ahora, tú eres el Dios que plantaste esta viña, pusiste sus raíces en lo profundo y extendiste sus retoños hasta tu lugar más alto. Míralo en este tiempo, porque la viña ha dado su fruto y no ha conocido al que la labraba. Y ahora, si te enojas contra tu viña y la arrancas del abismo, y secas los renuevos de tu altísimo trono eterno, no vendrá más el abismo a nutrirla, ni tu trono a refrescar tu viña que tú. has quemado.

\par 9 Porque tú eres todo luz, y has adornado tu casa con piedras preciosas y oro y perfumes y especias (o jaspe), y madera de bálsamo y canela, y con raíces de mirra y costumbre has esparcido tu casa, y con diversas comidas y dulzuras de muchas bebidas la has saciado. Así que, si no tienes piedad de tu viña, en vano serán todas estas cosas, Señor, y no tendrás quien te glorifique. Porque aunque plantes otra viña, ésta tampoco confiará en ti, porque tú destruiste la primera. Porque si en verdad abandonas el mundo, ¿quién hará por ti lo que has dicho como Dios? Y ahora que tu ira se retenga más sobre tu viña por lo que has dicho y por lo que aún queda por decir, y no sea en vano tu trabajo, ni tu herencia se desgarre en humillación.

\par 10 Y Dios le dijo: He aquí, me he vuelto misericordioso según tus palabras. Corta, pues, dos tablas de piedra del lugar donde labraste la primera, y escribe de nuevo en ellas mis juicios que estaban sobre las primeras.

\chapter{13}

\par 1 Moisés se apresuró a hacer todo lo que Dios le había ordenado; descendió e hizo las mesas y sus utensilios, el arca, las lámparas, la mesa, el altar de los holocaustos y el altar. del incienso, la hombrera, el pectoral, las piedras preciosas, la fuente, las bases y todo lo que le era mostrado. Y ordenó todas las vestiduras de los sacerdotes, los cintos y demás, la mitra, la plancha de oro y la corona santa: hizo también el aceite de la unción de los sacerdotes, y santificó a los sacerdotes mismos. Y cuando todo estuvo consumado, la nube los cubrió a todos.

\par 2 Entonces Moisés clamó al Señor, y Dios le habló desde el tabernáculo, diciendo: Ésta es la ley del altar, mediante la cual me sacrificaréis y oraréis por vuestras almas. Pero en cuanto a lo que habéis de ofrecerme, ofreced de vacas el becerro, la oveja y la cabra; y de las aves, la tórtola y la paloma.

\par 3 Y si hay lepra en vuestra tierra, y el leproso queda limpio, tomarán para el Señor dos polluelos vivos, y madera de cedro, hisopo y escarlata; y vendrá al sacerdote, y matará a uno, y guardará al otro. Y ordenará al leproso conforme a todo lo que yo he mandado en mi ley.

\par 4 Y cuando llegue el momento para vosotros, me santificaréis con una fiesta, y os alegraréis delante de mí en la fiesta de los panes sin levadura, y pondréis pan delante de mí, celebrando una fiesta conmemorativa, porque en aquella ocasión el día que salisteis de la tierra de Egipto.

\par 5 Y en la fiesta de las semanas pondréis pan delante de mí y me haréis una ofrenda por vuestros frutos.

\par 6 Pero la fiesta de las trompetas será una ofrenda para vuestros observadores, porque en ella supervisé mi creación, para que os acordéis del mundo entero. Al principio del año, cuando me los mostréis, contaré el número de los muertos y de los nacidos, y el ayuno de misericordia. Porque ayunaréis a mí por vuestras almas, para que se cumplan las promesas de vuestros padres.

\par 7 Traedme también la fiesta de las Tiendas: tomaréis para mí los frutos agradables de los árboles, ramas de palmera, sauces, cedros y ramas de mirra; y me acordaré de toda la tierra bajo la lluvia. , y se establecerá la medida de las estaciones, y ordenaré las estrellas y ordenaré a las nubes, y sonarán los vientos y correrán los relámpagos, y habrá tormenta de truenos, y esto será por señal perpetua. También las noches producirán rocío, como hablé después del diluvio de la tierra.

\par 8 cuando yo (o él) le di precepto sobre el año de la vida de Noé, y le dije: Estos son los años que ordené después de las semanas en que visité la ciudad de los hombres, en qué momento les mostré (o él) el lugar de nacimiento y el color (o y la serpiente), y yo (o él) dije: Esto es. el lugar del cual enseñé al primer hombre diciendo: Si no transgredes lo que te ordené, todas las cosas te estarán sujetas. Pero él transgredió mis caminos y se convenció de su esposa, y ella fue engañada por la serpiente. Y entonces fue ordenada la muerte para las generaciones de los hombres.

\par 9 Y además el Señor le mostró (o, Y el Señor dijo además: Yo le mostré) los caminos del paraíso y le dijo: Estos son los caminos que los hombres han perdido al no caminar en ellos, porque han pecado contra mí.

\par 10 Y el Señor le ordenó acerca de la salvación de las almas del pueblo, y dijo: Si siguen mis caminos, no los abandonaré, sino que siempre seré misericordioso con ellos y bendeciré su descendencia y la tierra. se apresurará a dar su fruto, y habrá lluvia para que aumenten sus ganancias, y la tierra no será estéril. Sin embargo, en verdad sé que corromperán sus caminos, y yo los abandonaré, y ellos olvidarán los pactos que hice con sus padres. Sin embargo, no los olvidaré para siempre: porque en los últimos días sabrán que a causa de sus pecados su descendencia fue abandonada; porque soy fiel en mis caminos.

\chapter{14}

\par 1 En aquel tiempo le dijo Dios: Comienza a contar a mi pueblo desde veinte años arriba hasta cuarenta años, para que pueda mostrar a tus tribus todo lo que les dije a sus padres en tierra extraña. Porque a la quincuagésima parte de ellos los saqué de la tierra de Egipto, pero cuarenta y nueve partes de ellos murieron en la tierra de Egipto.

\par 2 Cuando los hayas ordenado y contado (o, Mientras estés allí, y cuando los hayas contado, etc.), escribe la historia de ellos, hasta que cumpla todo lo que hablé con sus padres, y los ponga en orden. firmemente en su propia tierra; porque no disminuiré palabra alguna de las que he hablado a sus padres, de las que les dije: Vuestra descendencia será como las estrellas del cielo en multitud. En número entrarán en la tierra, y en poco tiempo serán innumerables.

\par 3 Entonces Moisés descendió y los contó, y el número del pueblo era 604.550. Pero no contó entre ellos a la tribu de Leví, porque así le había sido mandado; sólo contó a los mayores de 50 años, de los cuales el número fue 47.300. También contó a los menores de 20 años, y el número de ellos fue 850.850. Y miró sobre la tribu de Leví y el número total de ellos era CXX. CCXD. DCXX. CC. DCCC.

\par 4 Y Moisés declaró a Dios el número de ellos; y Dios le dijo: Estas son las palabras que hablé a sus padres en la tierra de Egipto, y puse un número de 210 años para todos los que vieron mis maravillas. Ahora el número de todos ellos era 9000 veces 10000, 200 veces 95000 hombres, sin las mujeres, y a toda la multitud de ellos los maté porque no me creyeron, y a la 50 parte de ellos me quedé y los santifiqué para mí. Por tanto, ordeno a la generación de mi pueblo que me dé los diezmos de sus frutos, para que estén delante de mí en memoria de la gran opresión que les he quitado.

\par 5 Y cuando Moisés descendió y contó estas cosas al pueblo, ellos se lamentaron y se lamentaron y permanecieron en el desierto dos años.

\chapter{15}

\par 1 Y Moisés envió espías para reconocer la tierra, doce hombres, tal como se lo había ordenado. Y cuando subieron y vieron la tierra, volvieron a él trayendo los frutos de la tierra, y turbaron el corazón del pueblo, diciendo: No podréis heredar la tierra, porque está cerrada con hierro. rejas por sus valientes.

\par 2 Pero dos hombres de entre los 12 no hablaron así, sino que dijeron: Como el hierro duro puede vencer a las estrellas, o como las armas pueden vencer los relámpagos, o las aves del cielo apagar los truenos, así estos hombres pueden resistir. El Señor. Porque vieron cómo, mientras subían, los relámpagos de las estrellas brillaban y los truenos los seguían resonando con ellos.

\par 3 Y estos son los nombres de los hombres: Caleb hijo de Jefone, hijo de Beri, hijo de Batuel, hijo de Galifa, hijo de Zenen, hijo de Selimún, hijo de Selón, hijo de Judá. El otro, Jesús hijo de Naue, hijo de Elifhat, hijo de Gal, hijo de Nephelien, hijo de Emon, hijo de Saúl, hijo de Dabra, hijo de Efrén, hijo de José.

\par 4 Pero el pueblo no escuchó la voz de los dos, sino que se turbó mucho y habló diciendo: ¿Son éstas las palabras que Dios nos habló, diciendo: Os llevaré a una tierra que mana leche y miel? ¿Y cómo ahora nos hará subir para que caigamos sobre la espada, y nuestras mujeres vayan en cautiverio?

\par 5 Y cuando dijeron esto, la gloria de Dios apareció de repente, y dijo a Moisés: ¿Este pueblo persiste así en escucharme en absoluto? He aquí que el consejo que he salido de mí no será en vano. Enviaré sobre ellos el ángel de mi ira, para que desmenuce sus cuerpos con fuego en el desierto. Y daré mandamiento a mis ángeles que los guardan, que no oren por ellos, porque encerraré sus almas en los tesoros de las tinieblas, y diré a mis siervos, sus padres: He aquí, ésta es la simiente a la cual Hablé diciendo: Vuestra descendencia vendrá a tierra ajena, y yo juzgaré a la nación a la que servirán. Y cumplí mis palabras e hice que sus enemigos se derritieran, y puse ángeles debajo de sus pies, y puse una nube para cubrir sus cabezas, y mandé al mar, y los abismos se rompieron ante su rostro y se levantaron muros de agua.

\par 6 Y no ha habido nada igual a esta palabra desde el día en que dije: Júntense las aguas bajo el cielo en un solo lugar, hasta el día de hoy. Y los saqué, maté a sus enemigos y los conduje delante de mí al monte Sina. Y incliné los cielos y bajé para encender una lámpara para mi pueblo y para poner límites a todas las criaturas. Y les enseñé a hacerme un santuario para habitar entre ellos. Pero ellos me han abandonado y se han vuelto infieles a mis palabras, y su mente ha desmayado, y he aquí que ahora vendrán días en que les haré lo que han deseado y arrojaré sus cuerpos al desierto.

\par 7 Y Moisés dijo : Antes de que tomaras semilla para hacer al hombre sobre la tierra, ¿ordené yo sus caminos? Por tanto, ahora tu misericordia nos soporte hasta el fin, y tu piedad por muchos días.

\chapter{16}

\par 1 En aquel momento le dio órdenes sobre las zonas periféricas; entonces se rebeló Coreb y doscientos hombres con él, y habló diciendo: ¿Y si se nos impone una ley que no podemos soportar?

\par 2 Y Dios se enojó y dijo: Yo ordené a la tierra y ella me dio al hombre, y a él le nacieron los dos primeros hijos. Y el mayor se levantó y mató al menor, y la tierra se apresuró y tragó su sangre. Pero expulsé a Caín, maldije la tierra y hablé a Sión diciendo: No tragarás más sangre. Y ahora los pensamientos de los hombres están muy contaminados.

\par 3 He aquí, yo mandaré a la tierra, que se tragará el cuerpo y el alma a la vez, y su morada será en oscuridad y destrucción, y no morirán, sino que desfallecerán hasta que yo me acuerde del mundo y renueve la tierra. Y entonces morirán y no vivirán, y su vida será quitada del número de todos los hombres; ni el infierno los vomitará otra vez, y la destrucción no se acordará de ellos, y su partida será como la de la tribu de las naciones de las cuales dije: «No me acordaré de ellas», es decir, del campamento de los egipcios, y del pueblo que destruí con el agua del diluvio. Y la tierra se los tragará, y no les haré más.

\par 4 Y cuando Moisés habló todas estas palabras al pueblo, Coreb y sus hombres todavía no creían. Y Coreb envió a llamar a sus siete hijos que no estaban de acuerdo con él.

\par 5 Pero ellos le enviaron una respuesta, diciendo: Así como el pintor no muestra una imagen hecha por su arte sin haber sido previamente instruido, así también nosotros, cuando recibimos la ley del Poderoso que nos enseña sus caminos, no lo hicimos. ingresar . en él salvo que podamos caminar en él. Nuestro padre [no] nos engendró, pero el Todopoderoso nos formó, y ahora, si andamos en sus caminos, seremos sus hijos. Pero si no crees, sigue tu camino. Y no subieron a él.

\par 6 Y aconteció después de esto que la tierra se abrió ante ellos, y sus hijos enviaron a decirle: Si tu locura todavía está sobre ti, ¿quién te ayudará en el día de tu destrucción? y él no los escuchó. Y la tierra abrió su boca y se los tragó a ellos y a sus casas, y cuatro veces se movieron los cimientos de la tierra para tragarse a los hombres, como le había sido ordenado. Y después Choreb y su compañía gimieron, hasta que el firmamento de la tierra fuera devuelto.

\par 7 Pero las asambleas del pueblo dijeron a Moisés: No podemos quedarnos en los alrededores de este lugar donde Choreb y sus hombres han sido tragados. Y él les dijo. Alzad vuestras tiendas alrededor de ellos, y no os unáis a sus pecados. Y así lo hicieron.

\chapter{17}

\par 1 Entonces se declaró el linaje de los sacerdotes de Dios mediante la elección de una tribu, y se dijo a Moisés: Toma para cada tribu una vara y ponla en el tabernáculo, y luego la vara de aquel a quien mi la gloria hablará, florecerá, y quitaré la murmuración de mi pueblo.

\par 2 Así lo hizo Moisés, colocó doce varas y la vara de Aarón salió, floreció y dio semillas de almendras.

\par 3 Y esta semejanza que nació allí era semejante a la obra que Israel hizo mientras estaba en Mesopotamia con Labán el sirio, cuando tomó varas de almendras y las puso en la reunión de aguas, y el ganado vino a beber. y fueron divididas entre las varas peladas, y dieron a luz [cabritos] blancos, moteados y multicolores.

\par 4 Por eso la sinagoga del pueblo se hizo semejante a un rebaño de ovejas, y como el ganado se criaba según las varas de almendro, así se establecía el sacerdocio por medio de las varas de almendro.

\chapter{18}

\par 1 En aquel tiempo Moisés mató a Seón y a Og, reyes de los amorreos, y repartió toda su tierra entre su pueblo, y ellos habitaron en ella.

\par 2 Pero Balac, rey de Moab, que vivía frente a ellos, tuvo mucho miedo y envió a Balaam, hijo de Beor, intérprete de sueños, que habitaba en Mesopotamia, y le encargó, diciendo: He aquí, yo sé cómo que en el reinado de mi padre Sefor, cuando los amorreos peleaban contra él, los maldijiste y fueron entregados delante de él. Y ahora ven y maldice a este pueblo, porque son muchos, más que nosotros, y te harán gran honor.

\par 3 Y Balaam dijo: He aquí, esto le parece bien a Balac, pero él no sabe que el consejo de Dios no es como el consejo de los hombres. Y no sabe que el espíritu que se nos da es por un tiempo, y que nuestros caminos no son guiados excepto la voluntad de Dios. Ahora pues, quedaos aquí, y veré lo que el Señor me dice esta noche.

\par 4 Y por la noche Dios le dijo: ¿Quiénes son los hombres que han venido a ti? Y Balaam dijo: ¿Por qué, Señor, tientas a la raza humana? Por lo tanto, no pueden sostenerlo, porque tú sabías más que ellos todo lo que había en el mundo antes de fundarlo. Y ahora ilumina a tu siervo si es justo que vaya con ellos.

\par 5 Y Dios le dijo: ¿No era acerca de este pueblo que hablé a Abraham en visión, diciendo: Tu descendencia será como las estrellas del cielo, cuando lo levanté sobre el firmamento y le mostré todos los ordenamientos de las estrellas, y pidió de él a su hijo en holocausto? y lo trajo para ponerlo sobre el altar, pero yo se lo devolví a su padre. Y como no resistió, su ofrenda fue aceptable delante de mis ojos, y por su sangre elegí a este pueblo. Y entonces dije a los ángeles que obran con astucia: ¿No dije yo de él: ¿Le revelaré a Abraham todo lo que hago?

\par 6 También Jacob, cuando luchaba en el polvo con el ángel que estaba sobre las alabanzas, no lo soltó hasta que lo bendijo. Y ahora, he aquí, piensas ir con estos y maldecir a los que yo he escogido. Pero si los maldices, ¿quién te bendecirá?

\par 7 Y Balaam se levantó por la mañana y dijo: Vete, porque Dios no quiere que vaya contigo. Y fueron y contaron a Balac todo lo que se decía de Balaam. Y Balac envió de nuevo otros hombres a Balaam, diciendo: He aquí, yo sé que cuando ofrezcas holocaustos a Dios, Dios se reconciliará con el hombre, y ahora pide otra vez a tu Señor, y suplica con holocaustos, a cuantos él quiera. voluntad. Porque si acaso se aplaca en mi necesidad, tendrás tu recompensa, si es que Dios acepta tus ofrendas.

\par 8 Y Balaam les dijo: He aquí, el hijo de Séfor es un tonto y no sabe que habita junto a (lit. alrededor) de los muertos. Y ahora quédate aquí esta noche y veré lo que Dios te dirá. a mí. Y Dios le dijo: Ve con ellos, y tu viaje será un escándalo, y Balac mismo irá a la destrucción. Y él se levantó y fue con ellos.

\par 9 Y su asna vino por el camino del desierto y vio al ángel, y abrió los ojos de Balaam y vio al ángel y lo adoró en la tierra. Y el ángel le dijo: Date prisa y sigue adelante, porque con él se cumplirá lo que dices.

\par 10 Y vino a la tierra de Moab y edificó un altar y ofreció sacrificios; y cuando vio a una parte del pueblo, el espíritu de Dios no moraba en él, y tomó su parábola y dijo: He aquí, Balac me ha traído acá al monte, diciendo: Ven, corre al fuego de estos hombres. [He aquí] No puedo soportar ese [fuego] que las aguas apagan, pero ese fuego que consume las aguas, ¿quién lo soportará? Y él le dijo: Es más fácil quitar los cimientos y toda la parte superior de ellos, y apagar la luz del sol y oscurecer el resplandor de la luna, que el que quiere arrancar la planta del Poderoso o arruinar su viña. Y el mismo Balac no lo supo, porque su mente se envaneció, con la intención de que su destrucción llegue pronto.

\par 11 Porque he aquí, veo la herencia que el Altísimo me mostró en la noche, y he aquí que vienen días en que Moab se asombrará de lo que le sucederá, porque Balac deseaba persuadir al Altísimo con regalos y comprar decisión. Con dinero. ¿No deberías haber preguntado qué envió sobre Faraón y sobre su tierra para someterlos a servidumbre? He aquí una vid que nos cubre, deseable en extremo; ¿y quién tendrá celos de ella si no se seca? Pero si alguno dice en su consejo que el Más Poderoso ha trabajado en vano o lo ha elegido sin propósito, he aquí que ahora veo la salvación de la liberación que ha de llegar a ellos. Estoy restringido en el habla de mi voz y no puedo expresar lo que veo con mis ojos, porque sólo me queda un poco del espíritu santo que mora en mí, ya que sé que en eso fui persuadido de Balac. He perdido los días de mi vida:

\par 12 He aquí, veo de nuevo la herencia de la morada de este pueblo, y su luz brilla más que el resplandor del relámpago, y su carrera es más rápida que las flechas. Y vendrá el tiempo en que Moab gemirá, y los que sirven a Cham (¿Chemosh?) serán débiles, incluso los que tomaron este consejo contra ellos. Pero rechinaré los dientes porque fui engañado y transgredí lo que me dijeron en la noche. Sin embargo, mi profecía permanecerá manifiesta, y mis palabras vivirán, y los sabios y prudentes recordarán mis palabras, porque cuando maldije perecí, y aunque bendije, no fui bendecido. Y cuando hubo dicho esto, guardó silencio. Y Balac dijo: Tu Dios te ha defraudado de muchos regalos de mi parte.

\par 13 Entonces Balaam le dijo: Ven y déjanos aconsejarles lo que les harás. Escoged a las mujeres más hermosas que hay entre vosotros y que están en Madián y ponedlas delante de ellas desnudas y adornadas con oro y joyas, y será que cuando las vean y se acuesten con ellas, pecarán contra su Señor y caen en tus manos, porque de lo contrario no podrás someterlos.

\par 14 Y diciendo esto, Balaam se dio la vuelta y regresó a su lugar. Y después el pueblo se extravió tras las hijas de Moab, porque Balac hizo todo lo que Balaam le había mostrado.

\chapter{19}

\par 1 En aquel tiempo Moisés mató a las naciones, dio la mitad del botín al pueblo y comenzó a declararles las palabras de la ley que Dios les había hablado en Oreb.

\par 2 Y les habló diciendo: He aquí que duermo con mis padres y volveré a mi pueblo. Pero yo sé que os levantaréis y abandonaréis las palabras que os he ordenado, y Dios se enojará contra vosotros y os desamparará y se irá de vuestra tierra, y traerá contra vosotros a los que os odian, y tendrán dominio. sobre vosotros, pero no hasta el fin, porque se acordará del pacto que hizo con vuestros padres.

\par 3 Pero entonces os levantaréis vosotros, y vuestros hijos, y todas vuestras generaciones después de vosotros, y buscaréis el día de mi muerte, y dirán en su corazón: ¿Quién nos dará un pastor como Moisés, u otro juez como éste para los hijos de Israel, para orar en todo momento por nuestros pecados y ser escuchado por nuestras iniquidades?

\par 4 Sin embargo, hoy pongo por testigos contra vosotros al cielo y a la tierra, porque el cielo oirá esto y la tierra lo escuchará con sus oídos: que Dios ha revelado el fin del mundo, para poder hacer un pacto con vosotros. sobre sus lugares altos, y ha encendido entre vosotros lámpara eterna. Acordaos, impíos, que cuando os hablé, respondisteis diciendo: Todo lo que Dios nos ha dicho, lo oiremos y lo haremos. Pero si transgredimos o corrompimos nuestros caminos, él llamará testigo contra nosotros y nos destruirá.

\par 5 Pero sabed que comisteis el pan de los ángeles durante cuarenta años. Y ahora he aquí que bendigo a vuestras tribus, antes de que llegue mi fin. Pero vosotros, conoced mi trabajo con el que he trabajado con vosotros desde el día que subisteis de la tierra de Egipto.

\par 6 Y habiendo dicho esto, Dios le habló por tercera vez, diciendo: He aquí, tú te vas a dormir con tus padres, y este pueblo se levantará y me buscará, y se olvidará de mi ley con la que los he iluminado. y abandonaré su simiente por un tiempo.

\par 7 Pero a ti te mostraré la tierra antes de que mueras, pero no entrarás en ella en este siglo, para que no veas las imágenes talladas con las que este pueblo será engañado y desviado del camino. Te mostraré el lugar donde me servirán 740 (l. 850) años. Y después será entregada en manos de sus enemigos, y la destruirán, y extraños la rodearán, y será en aquel día como fue el día en que rompí las tablas del pacto que hice. contigo en Oreb; y cuando pecaron, lo que en ellos estaba escrito se desvaneció. Ese día era el día 17 del cuarto mes.

\par 8 Y Moisés subió al monte Oreb, tal como Dios le había ordenado, y oró diciendo: He aquí, he cumplido el tiempo de mi vida, 120 años. Y ahora te ruego que sea tu misericordia con tu pueblo y que tu compasión continúe sobre tu herencia, Señor, y tu longanimidad en tu lugar sobre la raza que has elegido, porque los has amado más que a todos.

\par 9 Y tú sabes que yo era pastor de ovejas, y cuando apacientaba el rebaño en el desierto, las llevaba a tu monte Oreb, y entonces vi por primera vez a tu ángel en fuego desde la zarza; pero me llamaste de la zarza, y tuve miedo y volví mi rostro, y me enviaste a ellos, y los libraste de Egipto, y a sus enemigos hundiste en el agua. Y les diste una ley y juicios según los cuales debían vivir. Porque ¿qué hombre es el que no ha pecado contra ti? ¿Cómo se afirmará tu herencia si no tienes misericordia de ellos? ¿O quién nacerá todavía sin pecado? Sin embargo, los corregirás por un tiempo, pero no con ira.

\par 10 Entonces el Señor le mostró la tierra y todo lo que en ella hay, y dijo: Ésta es la tierra que daré a mi pueblo. Y le mostró el lugar de donde las nubes sacan agua para regar toda la tierra, y el lugar de donde el río recibe su agua, y la tierra de Egipto, y el lugar del firmamento, de donde sólo bebe la tierra santa. Le mostró también el lugar desde donde llovía maná para el pueblo, e incluso hasta los senderos del paraíso. Y le mostró las medidas del santuario, y el número de las ofrendas, y la señal mediante la cual los hombres deben interpretar (lit. comenzar a mirar; sobre) el cielo, y dijo: Estas son las cosas que estaban prohibidas a los hijos de hombres porque pecaron.

\par 11 Y ahora, tu vara con la que se hicieron las señales será un testimonio entre mí y mi pueblo. Y cuando pequen, me enojaré con ellos y me acordaré de mi vara, y los perdonaré según mi misericordia, y tu vara estará delante de mí en memoria todos los días, y será como el arco con el que hice un pacto. con Noé cuando salió del arca, diciendo: Pondré mi arco en la nube, y será señal entre mí y los hombres de que el agua del diluvio no habrá más sobre la tierra.

\par 12 Pero a ti te sacaré de aquí y te haré dormir con tus padres, te haré descansar en tu sueño y te sepultaré en paz, y todos los ángeles se lamentarán por ti, y las huestes del cielo se entristecerán. Pero nadie, ni ángeles ni hombres, conocerá tu sepulcro en el que serás sepultado, sino que descansarás en él hasta que yo visite el mundo y te levante a ti y a tus padres de la tierra [de Egipto] en la que estaréis. Dormid, y os reuniréis y habitaréis en una habitación inmortal que no está sujeta al tiempo.

\par 13 Pero este cielo será ante mis ojos como una nube fugaz, y como ayer cuando pasó, y será cuando me acerque para visitar el mundo, ordenaré los años y ordenaré los tiempos, y ellos se acortarán, y las estrellas se apresurarán, y la luz del sol se apresurará a ponerse, ni la luz de la luna durará, porque me apresuraré a levantaros ese sueño, que en el lugar de santificación que os mostré. en ti habitarán en ella todos los que puedan vivir.

\par 14 Y Moisés dijo: Si puedo pedirte aún una cosa, oh Señor, según la multitud de tu misericordia, no te enojes conmigo. Y muéstrame cuánto tiempo ha pasado y cuánto queda.

\par 15 Y el Señor le dijo: Un instante, la palma de una mano, la plenitud de un momento y la gota de una copa. Y el tiempo lo ha cumplido todo. Porque han pasado 4½ y quedan 2½.

\par 16 Moisés, cuando lo oyó, se llenó de entendimiento y su semejanza fue transformada gloriosamente; y murió en gloria según la boca del Señor, y lo sepultó como le había prometido, y los ángeles se lamentaron ante su muerte. muerte, y relámpagos, antorchas y flechas iban delante de él al unísono. Y aquel día no se dijo el himno de los ejércitos a causa de la partida de Moisés. Ni ha habido día igual a éste desde que el Señor hizo al hombre sobre la tierra, ni lo habrá para siempre, que haga cesar el himno de los ángeles por causa de un hombre; porque lo amaba mucho; y lo sepultó con sus propias manos en un lugar alto de la tierra, y a la luz del mundo entero.

\chapter{20}

\par 1 Y en aquel tiempo Dios hizo su pacto con Jesús, hijo de Naue, el resto de los hombres que reconocían la tierra: porque les había tocado la suerte de no ver la tierra porque hablaban mal de ella, y por esta causa murió aquella generación.

\par 2 Entonces dijo Dios a Jesús, hijo de Naue: ¿Por qué te lamentas y esperas en vano, pensando que Moisés vivirá aún? Ahora, pues, esperas en vano, porque Moisés ha muerto. Toma las vestiduras de su sabiduría y vístetelas, y ciñe tus lomos con el cinto de su conocimiento, y serás transformado y serás otro hombre. ¿No hablé yo por ti a Moisés mi siervo, diciendo: «Él guiará a mi pueblo tras de ti, y en su mano entregaré a los reyes de los amorreos»?

\par 3 Entonces Jesús tomó las vestiduras de la sabiduría, se las vistió y ciñó sus lomos con el cinto de la inteligencia. Y aconteció que cuando se lo puso, se encendió su mente y se agitó su espíritu, y dijo al pueblo: He aquí, la primera generación murió en el desierto porque hablaron contra su Dios. Y he aquí ahora, capitanes todos, sabed hoy que si vais por los caminos de vuestro Dios, vuestras sendas serán enderezadas.

\par 4 Pero si no obedecéis su voz y sois como vuestros padres, vuestras obras se arruinarán y vosotros mismos quebrantados, y vuestro nombre perecerá de la tierra; y entonces, ¿dónde estarán las palabras que Dios habló a vuestros hijos? padres? Porque incluso si los paganos dicen: Puede ser que Dios haya fallado, porque no ha liberado a su pueblo, sin embargo, aunque perciban que ha elegido para sí otros pueblos, obrando para ellos grandes maravillas, comprenderán que el Más Poderoso no acepta. personas. Pero como pecasteis por vanidad, él os quitó su poder y os sometió. Y ahora levántate y dispone tu corazón para caminar en los caminos de tu Señor y él te dirigirá.

\par 5 Y el pueblo le dijo: He aquí, hoy vemos lo que Eldad y Modat profetizaron en los días de Moisés, diciendo: Después que Moisés descanse, la capitanía de Moisés será dada a Jesús, el hijo de Naue. Y Moisés no tuvo envidia, sino que se alegró al oírlos; y desde entonces todo el pueblo creyó que tú los guiarías y les repartirías la tierra en paz; y ahora también, si hay conflicto, sé fuerte y hazlo con valentía, porque sólo tú serás líder en Israel.

\par 6 Al oír esto, Jesús pensó en enviar espías a Jericó. Y llamó a Cenez y a Senamías su hermano, los dos hijos de Calef, y les habló diciendo: Tu padre y yo fuimos enviados por Moisés al desierto y subimos con otros diez hombres; y ellos regresaron y hablaron mal de las tierras. y derritió el corazón del pueblo, y fueron esparcidos y el corazón del pueblo con ellos. Pero tu padre y yo sólo cumplimos la palabra del Señor, y he aquí, hoy estamos vivos. Y ahora os enviaré a reconocer la tierra de Jericó. Haced como vuestro padre y vosotros también viviréis.

\par 7 Y subieron y exploraron la ciudad. Y cuando trajeron la noticia, el pueblo subió y sitió la ciudad y la quemó a fuego.

\par 8 Y después de la muerte de Moisés, el maná dejó de descender para los hijos de Israel, y entonces comenzaron a comer los frutos de la tierra. Y estas son las tres cosas que Dios dio a su pueblo por amor a tres personas, es decir, el pozo de agua de Mara por amor a María, y la columna de nube por amor a Aarón, y el maná por amor a Moisés. Y cuando estos tres llegaron a su fin, esos tres regalos les fueron quitados.

\par 9 El pueblo y Jesús pelearon contra los amorreos, y cuando la batalla contra sus enemigos se recrudeció durante todos los días de Jesús, 30 y 9 reyes que habitaban en la tierra fueron exterminados. Y Jesús dio la tierra por suertes al pueblo, a cada tribu según las suertes, según había recibido el mandamiento.

\par 10 Entonces Calef se acercó a él y le dijo: Tú sabes que a nosotros dos fuimos enviados por suerte por Moisés para ir con los espías, y como cumplimos la palabra del Señor, he aquí que hoy estamos vivos. Si te agrada, que se le dé a mi hijo Cenez una porción del territorio de las tres (o la tribu de las) torres. Y Jesús lo bendijo, y así lo hizo.

\chapter{21}

\par 1 Y cuando Jesús envejeció y se hizo viejo, Dios le dijo: He aquí, tú envejeciste y envejeciste con los días, y la tierra se ha vuelto muy grande, y no hay quien la divida (o tómalo por suerte), y será que después de tu partida este pueblo se mezclará con los habitantes de la tierra y se extraviará tras otros dioses, y yo los abandonaré como testifiqué en mi palabra a Moisés; pero dales testimonio antes de morir.

\par 2 Y Jesús dijo: Tú sabes más que todos, oh Señor, lo que mueve el corazón del mar antes de que se enfurezca, y has rastreado las constelaciones y contado las estrellas y ordenado la lluvia. Tú conoces la mente de todas las generaciones antes de que nazcan. Y ahora, Señor, da a tu pueblo un corazón de sabiduría y una mente de prudencia, y será que cuando des estas ordenanzas a tu herencia, no pecarán delante de ti y no te enojarás con ellos.

\par 3 ¿No son estas las palabras que hablé delante de ti, Señor, cuando Acar robó la maldición y el pueblo fue entregado ante ti, y oré en tu presencia y dije: Si no fuera mejor para nosotros, oh Señor? , si hubiéramos muerto en el Mar Rojo, donde ahogaste a nuestros enemigos? ¿O si hubiéramos muerto en el desierto, como nuestros padres, que ser entregados en manos de los amorreos para ser destruidos para siempre?

\par 4 Pero si tu palabra se refiere a nosotros, ningún mal nos sobrevendrá; porque aunque nuestro fin sea quitado a la muerte, tú vives, que eres antes y después del mundo; y mientras que un hombre no puede idear cómo anteponer una generación a otra, dice: «Dios ha destruido a su pueblo que eligió»: y he aquí, estaremos en el infierno; sin embargo, tú darás vida a tu palabra. Y ahora que la plenitud de tus misericordias tenga paciencia con tu pueblo, y escoge para tu herencia un varón que gobierne a tu pueblo, él y su generación.

\par 5 ¿No fue por esto que nuestro padre Jacob habló, diciendo: Un príncipe no se apartará de Judá, ni un líder de sus lomos? Y ahora confirma las palabras dichas antes, para que las naciones de la tierra y las tribus del mundo sepan que tú eres eterno.

\par 6 Y añadió además: Oh Señor, he aquí que vendrán días y la casa de Israel será como una paloma que pone a sus polluelos en el nido y no los abandonará ni olvidará su lugar. Así también éstos se apartarán de sus obras y lucharán contra la salvación que les nacerá.

\par 7 Entonces Jesús descendió de Galgalá y edificó un altar de piedras muy grandes, pero no puso sobre ellas hierro, como Moisés había ordenado, y levantó grandes piedras en el monte Gebal, las blanqueó y escribió en ellas las palabras del ley muy claramente: y reuniendo a todo el pueblo, leyeron en sus oídos todas las palabras de la ley.

\par 8 Y descendió con ellos y ofreció sacrificios de paz sobre el altar, y ellos cantaron muchas alabanzas y sacaron del tabernáculo el arca de la alianza del Señor con panderos, danzas, flautas, arpas, salterios y todos instrumentos de dulce sonido.

\par 9 Y los sacerdotes y los levitas subían delante del arca y se regocijaban con salmos, y pusieron el arca delante del altar, y alzaron sobre ella aún muchas ofrendas de paz, y toda la casa de Israel cantó al unísono. una gran voz decía: He aquí, nuestro Señor ha cumplido lo que habló con nuestros padres, diciendo: A vuestra descendencia daré una tierra en la que habitar, una tierra que mana leche y miel. Y he aquí, él nos ha traído. en la tierra de nuestros enemigos y los ha librado delante de nosotros de los quebrantados de corazón, y él es el Dios que envió a nuestros padres a los lugares secretos de las almas, diciendo: He aquí, el Señor ha hecho todo lo que nos habló. Y ahora sabemos de verdad que Dios ha confirmado todas las palabras de la ley que nos habló en Oreb; y si nuestro corazón guarda sus caminos, nos irá bien a nosotros y a nuestros hijos después de nosotros.

\par 10 Y Jesús los bendijo y dijo: El Señor conceda a vuestro corazón permanecer en él (o en él) todos los días, y si no os apartáis de su nombre, el pacto del Señor perdurará con vosotros. Y conceda que no se corrompa, sino que la morada de Dios sea edificada entre vosotros, como él habló cuando os envió a su herencia con alegría y alegría.



\chapter{22}

\par 1 Y aconteció después de estas cosas, cuando Jesús y todo Israel oyeron que los hijos de Rubén y los hijos de Gad y la media tribu de Manasés que habitaban alrededor del Jordán habían construido para ellos un altar, y ofrecían sacrificios sobre él, y Después de nombrar sacerdotes para el santuario, todo el pueblo se turbó en gran medida y vino a ellos en Silón.

\par 2 Y Jesús y todos los ancianos les hablaron, diciendo: ¿Cuáles son estas obras que se hacen entre vosotros, mientras todavía no estamos habitados en nuestra tierra? ¿No son estas las palabras que Moisés os habló en el desierto, diciendo: Mirad que cuando entréis en la tierra no estropeéis vuestras obras y corrompáis a todo el pueblo? Y ahora, ¿por qué han abundado tanto nuestros enemigos, sino porque vosotros corrompéis vuestros caminos y habéis causado todos estos problemas, y por eso se reunirán contra nosotros y nos vencerán?

\par 3 Y los hijos de Rubén y los hijos de Gad y la media tribu de Manasés dijeron a Jesús y a todo el pueblo de Israel: He aquí, ahora Dios ha ensanchado el fruto del vientre de los hombres, y ha puesto una luz que El que está en tinieblas puede ver, porque sabe lo que hay en los lugares secretos del abismo, y con él permanece la luz. Ahora el Señor Dios de nuestros padres sabe si alguno de nosotros, o si nosotros mismos hemos hecho esto en camino de iniquidad, pero sólo por amor a nuestra posteridad, para que su corazón no se separe del Señor nuestro Dios, no sea que nos digan: He aquí ahora, nuestros hermanos que están al otro lado del Jordán tienen un altar para hacer ofrendas sobre él; pero nosotros, que estamos en este lugar, que no tenemos altar, nos apartaremos de Jehová nuestro Dios, porque nuestro Dios nos ha apartado de sus caminos. que no debemos servirle.

\par 4 Y entonces, en verdad, dijimos entre nosotros: Hagamos de nosotros un altar para que tengan celo por buscar al Señor. Y en verdad hay algunos de nosotros que permanecemos quietos y sabemos que somos tus hermanos y estamos libres de culpa ante ti. Haced, pues, lo que es agradable a los ojos del Señor.

\par 5 Y Jesús dijo: ¿No es el Señor nuestro Rey más poderoso que los sacrificios de cortejo? ¿Y por qué no enseñasteis a vuestros hijos las palabras del Señor que oísteis de nosotros? Porque si vuestros hijos se hubieran ocupado en la meditación de la ley del Señor, su mente no se habría desviado tras un santuario hecho de mano. ¿O no sabéis que cuando el pueblo fue abandonado por un momento en el desierto, cuando Moisés subió a recibir las mesas, su mente se extravió y se hicieron ídolos? Y si la misericordia del Dios de vuestros padres no nos hubiera guardado, todas las sinagogas habrían sido objeto de burla, y todos los pecados del pueblo habrían sido quemados a causa de vuestra necedad.

\par 6 Ahora, pues, id y excavad los santuarios que habéis construido, y enseñad la ley a vuestros hijos, y ellos meditarán en ellos día y noche, para que el Señor esté con ellos por testigo y juez para todos ellos. los días de su vida. Y Dios será testigo y juez entre yo y vosotros, y entre mi corazón y vuestro corazón, que si habéis hecho esto con astucia, será vengado de vosotros, porque queréis destruir a vuestros hermanos; pero si lo habéis hecho por ignorancia como decís, Dios tendrá misericordia de vosotros por amor a vuestros hijos. Y todo el pueblo respondió: Amén, Amén.

\par 7 Entonces Jesús y todo el pueblo de Israel ofrecieron por ellos mil carneros como ofrenda por el pecado (literalmente, palabra de disculpa), oraron por ellos y los despidieron en paz; y fueron y destruyeron el santuario, y Ayunaron y lloraron ellos y sus hijos, y oraron y dijeron: Oh Dios de nuestros padres, que conoces el corazón de todos los hombres, tú sabes que nuestros caminos no fueron obrados con iniquidad ante tus ojos, ni nos hemos desviado de tus ojos. caminos, sino que todos nosotros te hemos servido, porque obra de tus manos somos; ahora, pues, acuérdate de tu pacto con los hijos de tus siervos.

\par 8 Después de esto, Jesús subió a Galgala y levantó el tabernáculo del Señor, el arca del pacto y todos sus utensilios, y lo instaló en Silo, y puso allí la Demostración y la Verdad (es decir, el Urim y Tumim). Y en aquel tiempo Eleazar el sacerdote que servía el altar enseñaba mediante la Demostración a todos los del pueblo que venían a consultar al Señor, porque así les era mostrado, pero en el nuevo santuario que estaba en Galgala, Jesús nombró incluso hasta el día de hoy los holocaustos que ofrecían los hijos de Israel cada año.

\par 9 Porque hasta que se construyó la casa del Señor en Jerusalén y mientras se hicieron las ofrendas en el nuevo santuario, al pueblo no se le prohibió ofrecer allí, porque la Verdad y la Demostración revelaron todas las cosas en Silo. Y hasta que Salomón puso el arca en el santuario del Señor, continuaron sacrificando allí hasta aquel día. Pero Eleazar hijo de Aarón, sacerdote del Señor, ministraba en Silo.



\chapter{23}

\par 1 Jesús, hijo de Naué, ordenó al pueblo y les repartió la tierra, siendo un hombre valiente y valiente. Y estando aún los adversarios de Israel en la tierra, se acercaban los días de Jesús en que moriría, y envió y llamó a todo Israel por toda su tierra con sus mujeres y sus hijos, y les dijo: Reuníos antes. el arca del pacto de Jehová en Silo y haré pacto con vosotros antes de morir.

\par 2 Y cuando todo el pueblo estaba reunido en Silo, el día 16 del mes tercero, delante del Señor, con sus mujeres y sus hijos, Jesús les dijo: Oíd, oh Israel, he aquí, yo hago con vosotros el pacto de esta ley que el Señor ordenó con nuestros padres en Oreb, y por tanto, quedaos aquí esta noche y ved lo que Dios me dice acerca de vosotros.

\par 3 Y mientras el pueblo esperaba allí aquella noche, el Señor se apareció a Jesús en visión y le habló diciendo: Conforme a todas estas palabras hablaré a este pueblo.

\par 4 Y Jesús vino por la mañana y reunió a todo el pueblo y les dijo: Así dice el Señor: Había allí una roca de donde saqué a vuestro padre, y del corte de esa roca surgieron dos hombres, cuyos nombres eran Abraham y Nacor, y del cincel de aquel lugar nacieron dos mujeres que se llamaban Sara y Melca. Y habitaron juntos más allá del río. Y Abraham tomó a Sara por mujer y Nacor tomó a Melca.

\par 5 Y cuando la gente de la tierra fue descarriada, cada uno según sus propios designios, Abraham creyó en mí y no se dejó llevar tras ellos. Y lo salvé del fuego, lo tomé y lo llevé a toda la tierra de Canaán. Y le hablé en visión diciendo: A tu descendencia daré esta tierra. Y me dijo: He aquí ahora me has dado mujer y es estéril. ¿Y cómo tendré simiente de ese vientre cerrado?

\par 6 Y le dije: Tómame un becerro de tres años, una cabra de tres años, un carnero de tres años, una tórtola y un palomino. Y él los tomó como yo le mandé. Y envié sueño sobre él y lo rodeé de miedo, y puse delante de él el lugar de fuego donde serán vengadas las obras de los que cometen iniquidad contra mí, y le mostré las antorchas de fuego con las que los justos que han cometido creyeron en mí serán iluminados.

\par 7 Y le dije: Estos servirán de testimonio entre mí y tú de que te daré simiente de la matriz que está encerrada. Y te compararé a la paloma, porque has recibido para mí la ciudad que tus hijos (comenzarán a) edificar ante mis ojos. Pero compararé la tórtola con los profetas que nacerán de ti. Y compararé el carnero con los sabios que nacerán de ti y alumbrarán a tus hijos. Pero compararé el becerro con la multitud de pueblos que por medio de ti se multiplicarán. Y compararé la cabra con las mujeres cuyos vientres abriré y darán a luz. Estas cosas serán para testimonio entre nosotros de que no transgrediré mis palabras.

\par 8 Y le entregué a Isaac y lo formé en el vientre de la que lo dio a luz, y le ordené que lo restaurara rápidamente y me lo entregara en el séptimo mes. Y por esto toda mujer que dé a luz en el séptimo mes, su hijo vivirá: porque sobre él llamé mi gloria, y mostré la nueva era.

\par 9 Y a Isaac le di a Jacob y a Esaú, y a Esaú le di en herencia la tierra de Seír. Y Jacob y sus hijos descendieron a Egipto. Y los egipcios humillaron a vuestros padres, como sabéis, y yo me acordé de vuestros padres, y envié a mi amigo Moisés y los libré de allí y derroté a sus enemigos.

\par 10 Y los saqué con mano alta y los conduje a través del Mar Rojo, puse la nube bajo sus pies, los saqué a través de las profundidades y los llevé debajo del monte Sina, e incliné los cielos y descendió, y congelé la llama del fuego, y tapé los manantiales del abismo, e impidí el curso de las estrellas, y dominé el sonido del trueno, y apagué la plenitud; del viento, y reprendí a la multitud de las nubes, y detuve sus movimientos, e interrumpí la tormenta de los ejércitos, para que yo no rompiera mi pacto, porque todas las cosas fueron conmovidas con mi descenso, y todas las cosas fueron vivificadas con mi advenimiento, y no permití que mi pueblo fuera esparcido, sino que les di mi ley y los iluminé, para que si hicieran estas cosas, vivieran y tuvieran muchos días y no murieran.

\par 11 Y os he traído a esta tierra y os he dado viñas. Habitáis en ciudades que no construisteis. Y he cumplido el pacto que hablé con vuestros padres.

\par 12 Y ahora, si obedecéis a vuestros padres, pondré mi corazón sobre vosotros para siempre y os cubriré con mi sombra, y vuestros enemigos no volverán a luchar contra vosotros, y vuestra tierra será famosa en todo el mundo y vuestra descendencia será elegidos en medio de los pueblos, los cuales dirán: He aquí el pueblo fiel; porque creyeron en el Señor, por eso el Señor los libró y los plantó. Y por tanto te plantaré como a viña deseable y te regirá como a rebaño amado, y encargaré la lluvia y el rocío, y te saciarán todos los días de tu vida.

\par 13 Y sucederá que al final la suerte de cada uno de ustedes será la vida eterna, tanto para ustedes como para su descendencia, y yo recibiré sus almas y las guardaré en paz, hasta el tiempo de la era. se ha cumplido, y os restituiré a vuestros padres y a vuestros padres a vosotros, y sabrán de vuestra mano que no en vano os he escogido. Estas son las palabras que el Señor me ha hablado esta noche.

\par 14 Y todo el pueblo respondió y dijo: El Señor es nuestro Dios, y a él sólo serviremos. Y todo el pueblo hizo aquel día un gran banquete y una renovación del mismo por veintiocho días.

\chapter{24}

\par 1 Y después de estos días Jesús, hijo de Naue, reunió de nuevo a todo el pueblo y les dijo: He aquí ahora el Señor os ha testificado hoy: He llamado al cielo y a la tierra por testigos de que si permanecéis firmes, para servir al Señor seréis para él un pueblo peculiar. Pero si no queréis servirle y obedecer a los dioses de los amorreos en cuya tierra habitáis, decidlo hoy delante del Señor y salid. Pero yo y mi casa serviremos al Señor.

\par 2 Y todo el pueblo alzó la voz y lloró, diciendo: Quizás el Señor nos tendrá por dignos, y es mejor que muramos temiéndolo que ser destruidos de la tierra.

\par 3 Y Jesús, hijo de Naue, bendijo al pueblo, los besó y les dijo: Que vuestras palabras sean misericordiosas delante de nuestro Señor, y que él envíe su ángel y os proteja: Acordaos de mí después de mi muerte, y acordaos de vosotros. Moisés el amigo del Señor. Y no se aparten de ti las palabras del pacto que él ha hecho contigo en todos los días de tu vida. Y él los despidió y partieron cada uno a su heredad.

\par 4 Pero Jesús se acostó en su cama y envió a llamar a Finees, hijo de Eleazar, el sacerdote, y le dijo: He aquí, ahora veo con mis ojos la transgresión de este pueblo, en la cual comenzarán a engañar; pero tú, fortalece. tus manos en el tiempo que estés con ellos, y lo besó a él y a su padre y a sus hijos y lo bendijo y dijo: El Señor Dios de tus padres dirige tus caminos y los caminos de este pueblo.

\par 5 Y cuando terminó de hablarles, metió los pies en la cama y se acostó con sus padres. Y sus hijos pusieron sus manos sobre sus ojos.

\par 6 Y entonces todo Israel se reunió para enterrarlo, y lo lloraron con gran lamentación, y dijeron así en su lamentación: Llorad por el ala de esta veloz águila, porque se ha alejado de nosotros. Y llorad por la fuerza de este cachorro de león, porque está escondido de nosotros. ¿Quién irá ahora a contar a Moisés el justo, que durante cuarenta años hemos tenido un líder como él? Y cumplieron su duelo y lo enterraron con sus propias manos en el monte de Effraim y regresaron cada uno a su tienda. Y después de la muerte de Jesús, la tierra de Israel estaba en reposo.

\chapter{25}

\par 1 Los filisteos intentaron pelear contra los hombres de Israel, y consultaron al Señor y dijeron: ¿Subimos y peleamos contra los filisteos? y Dios les dijo: Si subís con un corazón puro, pelead; pero si tu corazón está contaminado, no subas. Y volvieron a preguntar diciendo: ¿Cómo sabremos si todos los corazones del pueblo son iguales? y Dios les dijo: Echen suertes entre vuestras tribus, y a cada tribu que entre en suerte se la apartará en una sola suerte, y entonces sabréis de quién es el corazón limpio y de quién está contaminado.

\par 2 Y el pueblo dijo: Primero designemos un príncipe sobre nosotros y luego echemos suertes. Y el ángel del Señor les dijo: Designad. Y el pueblo dijo: ¿A quién nombraremos que sea digno, Señor? Y el ángel del Señor les dijo: Echad suerte sobre la tribu de Caleb, y el que resulte de la suerte será vuestro príncipe. Y echaron suertes para la tribu de Caleb y salió sobre Cenez, y lo hicieron gobernante sobre Israel.

\par 3 Y Cenez dijo al pueblo: Traed a mí vuestras tribus y oíd ​​la palabra del Señor. Y se juntó el pueblo, y Cenez les dijo: Vosotros sabéis lo que Moisés, amigo del Señor, os mandó, que no transgredáis la ley ni a diestra ni a siniestra. Y también Jesús, el que iba después de él, os encargó lo mismo. Y ahora, he aquí, hemos oído de boca del Señor que vuestro corazón está contaminado. Y el Señor nos ha encargado que echemos suertes entre vuestras tribus para saber quién se ha apartado del corazón del Señor nuestro Dios. ¿No vendrá sobre el pueblo el furor de la ira? Pero yo os prometo hoy que incluso si un hombre de mi casa sale en la suerte del pecado, no será salvo con vida, sino que será quemado en el fuego. Y el pueblo dijo: Buen consejo has dado para ponerlo en práctica.

\par 4 Y fueron llevadas ante él las tribus, y se encontraron de la tribu de Judá 345 hombres, de la tribu de Rubén 560, de la tribu de Simeón 775, de la tribu de Leví 150 y de los tribu de Zabulón 655 (o 645), y de la tribu de Isacar 665, y de la tribu de Gad 380, de la tribu de Aser 665, y de la tribu de Manasés 480, y de la tribu de Efraín 468, y de los tribu de Benjamín 267 Y todo el número de los que fueron hallados por la suerte del pecado fue 6110 Y Cenez los tomó a todos y los encerró en prisión, hasta que se supiera qué se debía hacer con ellos.

\par 5 Y dijo Cenez: ¿No habló de esto Moisés, amigo del Señor, diciendo: Hay entre vosotros una raíz fuerte que produce hiel y amargura? Ahora bien, bendito sea el Señor que ha revelado todas las maquinaciones de estos hombres, y no les ha permitido corromper a su pueblo con sus malas obras. Traed, pues, aquí la demostración y la verdad, y llamad al sacerdote Eleazar, y consultemos al Señor por él.

\par 6 Entonces Cenez y Eleazar y todos los ancianos y toda la sinagoga oraron unánimes, diciendo: Señor Dios de nuestros padres, revela a tus siervos la verdad, porque no creemos en las maravillas que hiciste con nuestros padres desde entonces. los sacaste de la tierra de Egipto hasta el día de hoy. Y el Señor respondió y dijo: Preguntad primero a los que fueron encontrados, y que confiesen las obras que hicieron con astucia, y después serán quemados en el fuego.

\par 7 Entonces Cenez los sacó y les dijo: Ahora sabéis que Achiar confesó cuando le tocó la suerte y contó todo lo que había hecho. Y ahora declaradme todas vuestras maldades y vuestras invenciones: ¿quién sabe si nos decís la verdad, aunque muráis ahora, Dios tendrá misericordia de vosotros cuando resucite a los muertos?

\par 8 Y uno de ellos, llamado Elas, le dijo: ¿No vendrá ahora sobre nosotros la muerte, y moriremos en el fuego? Sin embargo, os digo, Señor mío, que no hay inventos como estos que hemos hecho malvadamente. Pero si quieres buscar la verdad claramente, pregunta individualmente a los hombres de cada tribu, y así alguno de los que estén presentes percibirá la diferencia de sus pecados.

\par 9 Y Cenez preguntó a los de su tribu y le dijeron: Queríamos imitar y hacer el becerro que ellos hicieron en el desierto. Y después preguntó a los hombres de la tribu de Rubén, los cuales dijeron: Queríamos ofrecer sacrificios a los dioses de los que habitan en la tierra. Y preguntó a los hombres de la tribu de Leví, los cuales dijeron: Probaríamos el tabernáculo si era santo. Y preguntó al resto de la tribu de Isacar, los cuales dijeron: Íbamos a investigar por los espíritus malignos de los ídolos, para ver si se revelaban claramente; y preguntó a los hombres de la tribu de Zabulón, los cuales dijeron: Deseábamos comer. la carne de nuestros hijos y saber si Dios se preocupa por ellos. Y preguntó al resto de la tribu de Dan, el cual dijo: Los amorreos nos enseñaron lo que ellos hacían, para que nosotros enseñáramos a nuestros hijos. Y he aquí, están escondidos debajo de la tienda de Elas, quien te dijo que nos consultaras. Envía, pues, y los encontrarás. Y Cenez envió y los encontró.

\par 10 Después preguntó a los que habían quedado de la tribu de Gad, y dijeron: Hemos cometido adulterio con las mujeres de otros. Y preguntó luego a los hombres de la tribu de Aser, los cuales dijeron: Encontramos siete imágenes de oro que los amorreos llamaban santas ninfas, y las tomamos con las piedras preciosas que estaban puestas sobre ellas, y las escondimos; y he aquí, ahora Están depositados debajo de la cima del monte Siquem. Envía, pues, y los encontrarás. Y Cenez envió hombres y los sacó de allí.

\par 11 Éstas son las ninfas que, cuando eran llamadas, mostraban a los amorreos sus obras a cada hora. Porque estos son los que idearon siete hombres malos después del diluvio, cuyos nombres son estos: [? Cham] Canaán, Fut, Selat, Nembrot, Elat, Desuat. Tampoco volverá a haber en el mundo ninguna semejanza parecida tallada por la mano del artífice y adornada con variedad de pinturas, sino que fueron colocadas y fijadas para la consagración (es decir, ¿el lugar santo?) de los ídolos. Ahora bien, las piedras eran preciosas, traídas de la tierra de Euilath, entre las cuales había un cristal y un prase (o uno cristalino y otro verde), y mostraban su forma, siendo tallados a la manera de una piedra calada, y otro de ellos estaba grabado en la parte superior, y otro como marcado con manchas (o como una crisoprasa manchada) brillaba tanto con su grabado como si mostrara el agua del abismo que yacía debajo.

\par 12 Y estas son las piedras preciosas que los amorreos tenían en sus lugares santos, y su precio era superior a cualquier cálculo. Porque cuando alguien entraba de noche, no necesitaba la luz de una lámpara, tanto brillaba la luz natural de las piedras. Donde daba mayor luz aquel que estaba tallado en forma de piedra calada y limpiada con cerdas; porque si alguno de los amorreos era ciego, iba y ponía sus ojos sobre él y recobraba la vista. Cuando Cenez los encontró, los separó y los guardó hasta que supiera qué sería de ellos.

\par 13 Después preguntó a los que quedaban de la tribu de Manasés, y dijeron: Nosotros sólo profanamos los sábados del Señor. Y preguntó a los desamparados de la tribu de Effraim, los cuales dijeron: Queríamos pasar a nuestros hijos y a nuestras hijas por el fuego, para saber si era manifiesto lo que se decía. Y preguntó a los desamparados de la tribu de Benjamín, los cuales dijeron: Queríamos examinar en este tiempo el libro de la ley, si Dios había escrito claramente lo que en él había, o si Moisés lo había enseñado por sí mismo.

\chapter{26}

\par 1 Y cuando Cenez tomó todas estas palabras y las escribió en un libro y las leyó delante del Señor, Dios le dijo: Toma a los hombres y lo que se encontró con ellos y todos sus bienes y ponlos en la cama de río Fisón, y quemarlos en el fuego, para que mi ira se calme de ellos.

\par 2 Y Cenez dijo: ¿Quemaremos también en el fuego estas piedras preciosas o te las santificaremos, porque entre nosotros no hay ninguna igual a ellas? Y Dios le dijo: Si Dios recibiera en su nombre algo del anatema, ¿qué debería hacer el hombre? Por tanto, ahora toma estas piedras preciosas y todo lo que se encontró, tanto libros como hombres: y cuando así trates con los hombres, separa estas piedras con los libros, porque el fuego no servirá para quemarlos, y después te mostraré cómo debes destruirlos. Pero a los hombres y todo lo que fue hallado los quemarás al fuego. Y reunirás a todo el pueblo, y les dirás: Así se hará a todo aquel cuyo corazón se aparte de su Dios.

\par 3 Y cuando el fuego haya consumido a esos hombres, entonces los libros y las piedras preciosas que no pueden quemarse con el fuego, ni cortarse con hierro, ni borrarse con agua, los pondrán en la cima del monte, junto al nuevo altar; y ordenaré a una nube, que irá y tomará rocío y lo derramará sobre los libros, y borrará lo que en ellos está escrito, porque no pueden ser borrados con otra agua que la que nunca ha servido a los hombres. Y después enviaré mi rayo, que quemará los libros mismos.

\par 4 Pero en cuanto a las piedras preciosas, ordenaré a mi ángel que las tome, irá y las arrojará a las profundidades del mar, y cargaré el abismo y las tragará, porque no podrán continuar. en el mundo porque han sido contaminados por los ídolos de los amorreos, y mandaré a otro ángel, y tomará para mí doce piedras del lugar de donde fueron tomadas estas siete; y tú, cuando las encuentres en la cima del monte donde las pondrá, tómalas y ponlas en la hombrera frente a las doce piedras que Moisés puso allí en el desierto, y santifícalas en el pectoral (lit. oráculo) según las doce tribus; y no digas: ¿Cómo sabré qué piedra pondré para cada tribu? He aquí, te diré el nombre de la tribu que corresponde al nombre de la piedra, y encontrarás esculpidas una y otra.

\par 5 Y Cenez fue y tomó todo lo que habían encontrado y a los hombres que estaban con él, y reunió de nuevo a todo el pueblo, y les dijo: He aquí, habéis visto todas las maravillas que Dios nos ha mostrado hasta el día de hoy, y he aquí , cuando descubrimos a todos los que habían maquinado mal contra el Señor y contra Israel, Dios los ha revelado según sus obras, y ahora, hermanos, maldito todo aquel que entre vosotros piensa hacer algo parecido. Y todo el pueblo respondió Amén, Amén. Y dicho esto, quemó al fuego a todos los hombres, y todo lo que con ellos se halló, salvo las piedras preciosas.

\par 6 Entonces Cenez quiso probar si las piedras se podían quemar en el fuego y las arrojó al fuego. Y fue así que cuando cayeron allí, al instante se apagó el fuego. Y Cenez tomó hierro para quebrarlos, y cuando la espada los tocó, el hierro del mismo se derritió; y después, al menos, borraba los libros con agua; pero aconteció que el agua que cayó sobre ellos se congeló. Y cuando vio esto, dijo: Bendito sea Dios que ha hecho tan grandes maravillas con los hijos de los hombres, e hizo a Adán el primero creado y le mostró todas las cosas; que cuando Adán hubiera pecado por ello, entonces debería negarle todas estas cosas, no sea que, si las mostraba a la raza humana, éstas tuvieran dominio sobre ellas.

\par 7 Y dicho esto, tomó los libros y las piedras y los puso en la cima del monte, junto al nuevo altar, como el Señor le había ordenado, tomó una ofrenda de paz y un holocausto y ofreció sobre el nuevo altar 2000, ofreciéndolos a todos en holocausto. Y aquel día hicieron una gran fiesta él y todo el pueblo juntamente.

\par 8 Y esa noche Dios hizo lo que le había dicho a Cenez, porque ordenó a una nube que fue y tomó rocío del hielo del paraíso y lo derramó sobre los libros y los borró. Y después vino un ángel y las quemó, y otro ángel tomó las piedras preciosas y las arrojó en el corazón del mar, y cargó en la profundidad del mar, y se las tragó. Y fue otro ángel y trajo doce piedras y las puso junto al lugar de donde había tomado aquellas siete. Y grabó en él los nombres de las doce tribus.

\par 9 Y al día siguiente Cenez se levantó y encontró aquellas doce piedras en la cima del monte donde él mismo había puesto las siete. Y la talla de ellos era tal como si sobre ellos estuviera representada la forma de ojos.

\par 10 Y la primera piedra, en la que estaba escrito el nombre de la tribu de Rubén, era como una piedra de sardina. La segunda piedra estaba grabada con un diente (o marfil), y en ella estaba grabado el nombre de la tribu de Simeón, y se veía en ella la figura de un topacio; y en la tercera piedra estaba grabado el nombre de la tribu de Leví, y era semejante a una esmeralda. Pero la cuarta piedra se llamaba cristal, y en ella estaba grabado el nombre de la tribu de Judá, y era parecida a un carbunclo. La quinta piedra era verde, y en ella estaba grabado el nombre de la tribu de Isacar, y en ella había una piedra del color de un zafiro. Y de la sexta piedra la talla era como si hubiera sido inscrita (o como crisoprasa) moteada con diversas marcas, y en ella estaba escrita la tribu de Zabulón, y la piedra de jaspe era semejante a ella.

\par 11 La inscripción de la séptima piedra resplandecía y mostraba en sí misma, como si encerrara el agua del abismo, y en ella estaba escrito el nombre de la tribu de Dan, la piedra era como una ligura. Pero la octava piedra estaba tallada con diamante, y en ella estaba escrito el nombre de la tribu de Neptalim, y era como una amatista. Y de la novena piedra se abrió la talla, y era del monte Ofir, y en ella estaba escrita la tribu de Gad, y una piedra de ágata era semejante a ella. Y de la décima piedra se hizo una talla, y se hizo la semejanza de una piedra de Temán, y allí estaba escrita la tribu de Aser, y se le asemejaba un crisólito. Y la undécima piedra era una piedra elegida del Líbano, y en ella estaba escrito el nombre de la tribu de José, y había un berilo. comparado con eso. Y la duodécima piedra fue cortada de lo alto de Sión (o de la cantera), y en ella estaba escrita la tribu de Benjamín; y la piedra de ónice era semejante a ella.

\par 12 Y Dios dijo a Cenez: Toma estas piedras y ponlas en el arca del pacto del Señor con las tablas del pacto que di a Moisés en Oreb, y estarán allí con ellas hasta que Jahel se levante para construir. una casa a mi nombre, y luego los pondrá delante de mí sobre los dos querubines, y estarán delante de mí por memorial de la casa de Israel.

\par 13 Y sucederá que cuando los pecados de mi pueblo se hayan cumplido y sus enemigos tengan dominio sobre su casa, tomaré estas piedras y las primeras junto con las tablas, y las pondré en el lugar de donde vinieron. fueron creados en el principio, y estarán allí hasta que yo recuerde el mundo y visite a los habitantes de la tierra. Y entonces los tomaré a ellos y a muchos otros mejores que ellos, de ese lugar que ojo no vio, ni oído oyó, ni ha subido al corazón del hombre, hasta que suceda lo mismo en el mundo, y los justos tengan no necesitarán la luz del sol ni el brillo de la luna, porque la luz de las piedras preciosas será su luz.

\par 14 Entonces Cenez se levantó y dijo: Mirad qué bienes ha hecho Dios con los hombres, y a causa de sus pecados han sido privados de todo ello. Y ahora sé hoy que la raza de los hombres es débil, y su vida será contada como nada.

\par 15 Y diciendo esto, tomó las piedras del lugar donde estaban colocadas y, al tomarlas, la luz del sol se derramaba sobre ellas, y la tierra brillaba con su luz. Y Cenez los puso en el arca del pacto del Señor con las tablas como le fue mandado, y allí están hasta el día de hoy.

\chapter{27}

\par 1 Después de esto armó del pueblo a 300.000 hombres y subió a luchar contra los amorreos, y mató el primer día a 800.000 hombres, y el segundo día mató a unos 500.000.

\par 2 Y cuando llegó el tercer día, algunos hombres del pueblo hablaron mal contra Cenez, diciendo: He aquí, ahora Cenez yace solo en su casa con su mujer y sus concubinas, y nos envía a la batalla para que seamos destruidos antes. nuestros enemigos.

\par 3 Cuando los servidores de Cenez lo oyeron, le avisaron. Y mandó un capitán de cincuenta, y de ellos trajo treinta y siete hombres que hablaron contra él y los encerraron en la cárcel.

\par 4 Y sus nombres son estos: Le y Uz, Betul, Efhal, Dealma, Anaph, Desac, Besac, Gethel, Anael, Anazim, Noac, Cehec, Boac, Obal, Iabal, Enat, Beat, Zelut, Éfor, Ezet. , Desaph, Abidán, Esar, Moab, Duzal, Azat, Felac, Igat, Zofal, Eliesor, Ecar, Zebath, Sebath, Nesach y Zere. Y cuando el capitán de cincuenta los hubo encerrado como Cenez había ordenado, Cenez dijo: Cuando el Señor haya hecho la salvación para su pueblo por mi mano, entonces castigaré a estos hombres.

\par 5 Y diciendo esto, Cenez ordenó al capitán de cincuenta, diciendo: Ve y escoge de mis servidores a trescientos hombres y otros tantos caballos, y que ningún hombre del pueblo sepa la hora en que saldré a la batalla; pero sólo a qué hora te diré, prepara los hombres para que estén listos esta noche.

\par 6 Y Cenez envió mensajeros espías para ver dónde estaba la multitud del campamento de los amorreos. Y los mensajeros fueron y espiaron, y vieron que la multitud del campamento de los amorreos se movía entre las rocas pensando en venir a pelear contra Israel. Y los mensajeros regresaron y le dijeron conforme a esta palabra. Y Cenez se levantó de noche, y con él trescientos hombres de a caballo, y tomando una trompeta en su mano, comenzó a descender con los trescientos hombres. Y aconteció que cuando estaba cerca del campamento de los amorreos, dijo a sus siervos: Quedaos aquí y yo bajaré solo y veré el campamento de los amorreos. Y será que si toco la trompeta descenderéis, pero si no, espérame aquí.

\par 7 Y Cenez descendió solo, y antes de bajar oró y dijo: Oh Señor Dios de nuestros padres, has mostrado a tu siervo las cosas maravillosas que has preparado para hacer por tu pacto en los últimos días: y ahora, envía a tu siervo una de tus maravillas, y venceré a tus adversarios, para que sepan ellos y todas las naciones y tu pueblo que el Señor no libra con la multitud de un ejército, ni con la fuerza de caballos, cuando percibirán la señal de liberación que tú harás para mí hoy (o jinetes, y que tú, Señor, harás conmigo hoy señal de salvación). He aquí, desenvainaré mi espada y brillará en el campamento de los amorreos; y será que, si los amorreos entienden que soy yo, Cenez, entonces sabré que los has entregado en mis manos. Pero si no ven que soy yo, y piensan que soy otro, entonces sabré que no me has escuchado, sino que me has entregado a mis enemigos. Pero si en verdad soy entregado a la muerte, sabré que a causa de mis iniquidades el Señor no me escuchó, y me entregó a mis enemigos; pero él no destruirá su herencia con mi muerte.

\par 8 Y después de haber orado, se puso en camino y escuchó a la multitud de los amorreos decir: Levantémonos y peleemos contra Israel, porque sabemos que nuestras santas ninfas están allí entre ellos y las entregarán en nuestras manos.

\par 9 Y Cenez se levantó, porque el espíritu del Señor lo vistió como un vestido, y desenvainó su espada, y cuando su luz brilló sobre los amorreos como un rayo agudo, lo vieron y dijeron: ¿No es éste el espada de Cenez que ha hecho muchos nuestros heridos? Ahora queda justificada la palabra que dijimos, diciendo que nuestras santas Ninfas los han entregado en nuestras manos. He aquí ahora que hoy habrá banquete para los amorreos, cuando nuestro enemigo nos sea entregado. Ahora, pues, levántense y cada uno se ciña su espada y comience la batalla.

\par 10 Y aconteció que cuando Cenez oyó sus palabras, se invistió de un espíritu de poder y se transformó en otro hombre, y descendió al campamento de los amorreos y comenzó a herirlos. Y el Señor envió delante de su rostro al ángel Ingethel (o Gethel), que está encargado de las cosas ocultas, y obra en secreto, (y otro) ángel poderoso que lo ayudaba; e Ingethel hirió a los amorreos con ceguera, de modo que cada hombre que vio a su prójimo, los consideró sus adversarios, y se mataron unos a otros. Y el ángel Zeruel, que está sobre la fuerza, desnudó los brazos de Cenez para que no lo vieran; y Cenez hirió de los amorreos a cuarenta y cinco mil hombres, y ellos mismos se golpearon unos a otros, y cayeron cuarenta y cinco mil hombres.

\par 11 Y cuando Cenez hirió a una gran multitud, soltó su mano de su espada, porque el mango de la espada no se podía soltar, y su mano derecha había tomado en ella la fuerza de la espada.

\par Entonces los que quedaron de los amorreos huyeron a las montañas; pero Cenez buscaba cómo soltar su mano; y miró con sus ojos y vio a un hombre amorreo que huía, y lo atrapó y le dijo: Sé que los amorreos son astutos; ahora, pues, muéstrame cómo puedo soltarlo. Quita mi mano de esta espada, y te dejaré ir. Y el amorreo dijo: Ve y toma a un hombre de los hebreos y mátalo, y mientras su sangre aún esté caliente, toma tu mano debajo y recibe su sangre, así tu mano quedará suelta. Y Cenez dijo: Vive el Señor, que si hubieras dicho: Toma a un hombre de los amorreos, yo habría tomado a uno de ellos y te habría salvado la vida; pero ya que dijiste «de los hebreos» para que pudieras mostrar tu odio, tu boca será contra ti mismo, y tal como has dicho, así haré contigo. Y cuando hubo dicho esto, Cenez lo mató, y mientras su sangre aún estaba caliente, puso su mano debajo y la recibió en ella, y se soltó.

\par 12 Entonces Cenez salió, se desnudó, se arrojó al río y se lavó, volvió a subir, se cambió de ropa y volvió con sus jóvenes. Y el Señor hizo caer sobre ellos un sueño pesado durante la noche, y durmieron y no supieron nada de todo lo que Cenez había hecho. Y vino Cenez y los despertó del sueño; y ellos miraron con sus ojos y vieron, y he aquí, el campo estaba lleno de cadáveres; y estaban atónitos en su mente, y miraban cada uno a su prójimo. Y Cenez les dijo: ¿Por qué os maravilláis? ¿Son los caminos del Señor como los caminos de los hombres? Porque entre los hombres prevalece la multitud, pero entre Dios lo que él designa. Y, por tanto, si Dios ha querido obrar la liberación de este pueblo por mis manos, ¿por qué os maravilláis? Levántense y cíñense cada uno sus espadas, y regresaremos a casa con nuestros hermanos.

\par 13 Y cuando todo Israel oyó la liberación obrada por las manos de Cenez, todo el pueblo salió unánimemente a su encuentro y dijeron: Bendito sea el Señor que te ha puesto por gobernante sobre su pueblo y te ha mostrado que son ciertas las cosas que él te habló; lo que oímos con la palabra, ahora lo vemos con nuestros ojos, porque la obra de la palabra de Dios es manifiesta.

\par 14 Y Cenez les dijo: Preguntad ahora a vuestros hermanos, y que os digan cuánto trabajaron conmigo en la batalla. Y los hombres que estaban con él dijeron: Vive el Señor, que no peleamos, ni supimos nada, salvo que cuando despertamos, vimos el campo lleno de cadáveres. Y el pueblo respondió: Ahora sabemos que cuando el Señor designa obrar liberación para su pueblo, no tiene necesidad de multitud, sino sólo de santificación.

\par 15 Y Cenez dijo al capitán de cincuenta que había encerrado a aquellos hombres en prisión: Saca a esos hombres para que podamos escuchar sus palabras. Y cuando los hubo sacado, Cenez les dijo: Dime, ¿qué viste en mí para que murmuraras entre el pueblo? Y ellos dijeron: ¿Por qué nos preguntas? ¿Por qué nos preguntas? Ahora pues, ordena que seamos quemados en fuego, porque no morimos por este pecado que ahora hemos hablado, sino por el primero en que fueron apresados ​​aquellos hombres que fueron quemados en sus pecados; porque entonces consentimos en su pecado, diciendo: Quizás el pueblo no nos perciba; y luego logramos escapar de la gente. Pero ahora nuestros pecados nos han dado (con razón) un ejemplo público al caer en calumnias contra ti. Y Cenez dijo: Si vosotros, pues, testificáis contra vosotros mismos, ¿cómo tendré compasión de vosotros? Y Cenez ordenó que los quemaran en el fuego y arrojaran sus cenizas en el lugar donde habían quemado a la multitud de los pecadores, en el arroyo Phison.

\par 16 Y Cenez reinó sobre su pueblo cincuenta y siete años, y todos sus enemigos tuvieron miedo durante todos sus días.

\chapter{28}

\par 1 Y cuando se acercaba el día de su muerte, Cenez envió y llamó a todos los hombres (o a todos los ancianos), y a los dos profetas Jabis y Finees, y a Finees, hijo del sacerdote Eleazar, y les dijo : He aquí ahora, el Señor me ha mostrado todas sus maravillas que ha preparado para hacer por su pueblo en los últimos días.

\par 2 Y ahora haré mi pacto con vosotros hoy, de que no abandonaréis al Señor vuestro Dios después de mi partida. Porque habéis visto todas las maravillas que sobrevinieron a los que pecaron, y todo lo que declararon, confesando sus pecados por su propia voluntad, y cómo el Señor nuestro Dios acabó con ellos por haber transgredido su pacto. Por tanto, ahora perdonad a los de vuestra casa y a vuestros hijos, y permaneced en los caminos del Señor vuestro Dios, para que el Señor no destruya su herencia.

\par 3 Y Finees, hijo del sacerdote Eleazar, dijo: Si el gobernante Cenez, junto con los profetas, el pueblo y los ancianos, me lo ordenan, hablaré la palabra que oí de mi padre cuando agonizaba. y no callaré el mandamiento que me mandó cuando fue recibida su alma. Y dijo el gobernante Cenez y los profetas: Que siga hablando Finees. ¿Hablará algún otro delante del sacerdote que guarda los mandamientos de Jehová nuestro Dios, y que de su boca sale la verdad, y de su corazón sale una luz resplandeciente?

\par 4 Entonces dijo Finees: Mi padre, cuando estaba agonizando, me mandó diciendo: Así dirás a los hijos de Israel cuando estén reunidos en asamblea: El Señor se me apareció el tercer día antes de este. en sueños de noche, y me dijo: He aquí, tú has visto, y tu padre delante de ti, cuánto he trabajado por mi pueblo; y será después de tu muerte que este pueblo se levantará y corromperá sus caminos, apartándose de mis mandamientos, y me enojaré en gran manera contra ellos. Sin embargo, me acordaré del tiempo que fue antes de los siglos, incluso en el tiempo en que no había hombre, y en él no había iniquidad, cuando dije que el mundo sería, y los que vendrían me alabarían en él, y yo Plantaré una gran viña, y de ella escogeré una planta, la ordenaré y la llamaré por mi nombre, y será mía para siempre. Pero cuando haya hecho todo lo que he dicho, sin embargo mi planta, que lleva mi nombre, no me conocerá a mí, quien la plantó, sino que corromperá su fruto, y no me dará su fruto. Estas son las cosas que mi padre me mandó que hablara a este pueblo.

\par 5 Entonces Cenez alzó la voz, los ancianos y todo el pueblo al unísono, y lloraron con gran lamentación hasta el anochecer, y dijeron: ¿Destruirá el pastor su rebaño en vano, a menos que continúe en pecado contra ¿a él? ¿Y no será él quien perdonará según la abundancia de su misericordia, ya que ha trabajado mucho en nosotros?

\par 6 Mientras estaban sentados, el espíritu santo que habitaba en Cenez saltó sobre él y le quitó los sentidos corporales, y comenzó a profetizar, diciendo: He aquí, ahora veo lo que no esperaba y percibo que No lo sabía. Escuchad ahora, moradores de la tierra, como profetizaron antes de mí los que moraban en ella, cuando vieron esta hora, incluso antes de que la tierra fuera corrompida, para que sepáis las profecías establecidas desde antes, todos los que moráis en ella.

\par 7 He aquí ahora veo llamas que no arden, y oigo manantiales de agua que despiertan del sueño, y no tienen fundamento, ni contemplo las cimas de las montañas, ni el dosel del firmamento, sino todas las cosas que no aparecen. e invisibles, que no tienen lugar alguno, y aunque mis ojos no saben lo que ven, mi corazón descubrirá lo que pueda aprender (o decir).

\par 8 Y de la llama que vi, y que no ardía, miré, y he aquí, surgió una chispa, y como si se construyera un piso debajo del cielo, y la semejanza de su piso era como una araña que hila. , a modo de escudo. Y cuando estaban puestos los cimientos, miré, y de aquel manantial se levantó como si fuera espuma hirviente, y he aquí, se transformó como si fuera otro cimiento; y entre los dos cimientos, el superior y el inferior, se acercaron desde la luz del lugar invisible como formas de hombres, y caminaban de un lado a otro; y he aquí, una voz que decía: Estos serán para un fundamento a los hombres y habitarán en él 7000 años.

\par 9 Y el fundamento inferior era un pavimento y el superior era de espuma, y ​​los que surgieron de la luz del lugar invisible, esos son los que habitarán allí, y el nombre de ese hombre es [Adán]. Y sucederá que cuando él (o ellos) hayan pecado contra mí y se cumpla el tiempo, la chispa se apagará y la primavera cesará, y así serán transformados.

\par 10 Y aconteció que después que Cenez hubo dicho estas palabras, se despertó y recobró el sentido; pero no sabía lo que había dicho ni lo que había visto, sino que sólo dijo al pueblo: Si los demás justos sean tales después de muertos, más les vale morir al mundo corruptible, para que no vean el pecado. Y cuando Cenez hubo dicho esto, murió y durmió con sus padres, y el pueblo estuvo de luto por él durante treinta días.



\chapter{29}

\par 1 Después de esto, el pueblo nombró a Zebul gobernante sobre ellos, y en aquel momento reunió al pueblo y les dijo: He aquí, ahora sabemos todo el trabajo con el que Cenez trabajó con nosotros en los días de su vida. Ahora bien, si hubiera tenido hijos, habrían sido príncipes sobre el pueblo, pero como sus hijas aún están vivas, que reciban mayor herencia entre el pueblo, porque su padre en vida se negó a dársela, para que no Hay que llamarlo codicioso y codicioso de ganancias. Y el pueblo dijo: Haz todo lo que bien te parezca.

\par 2 Y Cenez tenía tres hijas, cuyos nombres son estos: Ethema la primogénita, la segunda Pheila, la tercera Zelfa. Y Zebul dio al primogénito todo lo que había alrededor de la tierra de los fenicios, y al segundo le dio el olivar de Accaron, y al tercero toda la tierra cultivada que estaba alrededor de Azoto. Y les dio por maridos: a Elisephan, la primogénita, a Odiel, al segundo, y a Doel, el tercero.

\par 3 En aquellos días, Zebul levantó un tesoro para el Señor y dijo al pueblo: He aquí, si alguno quiere santificar para el Señor oro y plata, que los lleve al tesoro del Señor en Silo; pero que nadie el que tiene cosas pertenecientes a ídolos, piense en santificarlas para los tesoros del Señor, porque el Señor no desea las abominaciones de las cosas anatemas, para que no perturbéis la sinagoga del Señor, porque la ira que pasa es suficiente. Y todo el pueblo trajo lo que su corazón les impulsó a traer, tanto hombres como mujeres, oro y plata. Y se pesó todo lo traído, y fueron veinte talentos de oro y doscientos cincuenta talentos de plata.

\par 4 Y Zebul juzgó al pueblo veinticinco años. Y cuando cumplió su plazo, envió y llamó a todo el pueblo y dijo: He aquí que ahora parto a morir. Mirad los testimonios que testificaron los que fueron antes de nosotros, y no sea vuestro corazón como las olas del mar, sino como las olas del mar no entienden sino sólo las cosas que están en el mar, así vuestro corazón tampoco piensa en nada más que en las cosas que pertenecen a la ley. Y durmió Zebul con sus padres, y fue sepultado en el sepulcro de su padre.

\chapter{30}

\par 1 Entonces los hijos de Israel no tenían a nadie a quien nombrar juez sobre ellos; y desfalleció su corazón, y olvidaron la promesa, y transgredieron los caminos que Moisés y Jesús, los siervos del Señor, les habían mandado, y fueron llevados tras las hijas de los amorreos y sirvieron a sus dioses.

\par 2 Y el Señor, enojado con ellos, envió su ángel y les dijo: He aquí, yo me elegí un pueblo entre todas las tribus de la tierra, y dije que mi gloria permanecería con ellos en este mundo, y yo Les envié a Moisés mi siervo, para declararles mi gran majestad y mis juicios, y han transgredido mis caminos. Ahora pues, he aquí, yo incitaré a sus enemigos y ellos los gobernarán, y entonces todo el pueblo dirá: Por cuanto hemos transgredido los caminos de Dios y de nuestros padres, por eso nos sobrevendrán estas cosas. Sin embargo, una mujer los dominará y les dará luz durante 40 años.

\par 3 Y después de estas cosas, el Señor incitó contra ellos a Jabín, rey de Asor, y comenzó a luchar contra ellos, y tenía como capitán de su fuerza a Sísara, que tenía ocho mil carros de hierro. Y vino al monte Efrén y peleó contra el pueblo, e Israel le temió mucho, y el pueblo no pudo resistir en todos los días de Sísara.

\par 4 Y cuando Israel estaba muy abatido, todos los hijos de Israel se reunieron unánimes en el monte de Judá y dijeron: Nos decíamos más bienaventurados que todos los pueblos, y ahora, he aquí, estamos tan abatidos, más que todas las naciones, que no podemos habitar en nuestra tierra, y nuestros enemigos se enseñorean de nosotros. ¿Y ahora quién nos ha hecho todo esto? ¿No son nuestras iniquidades, por haber abandonado al Señor Dios de nuestros padres, y haber andado en cosas que no nos podían aprovechar? Venid, pues, ahora a ayunar siete días, tanto hombres como mujeres, y desde el más pequeño (sic) hasta el niño de pecho. ¿Quién sabe si Dios se reconciliará con su herencia, y no destruirá la plantación de su viña?

\par 5 Y después de que el pueblo había ayunado siete días, sentados en cilicio, el Señor les envió el séptimo día a Débora, quien les dijo: ¿Puede la oveja destinada al matadero responder ante el que la mata, cuando ambos el que mata [. . . ] ¿Y el que es asesinado guarda silencio, cuando a veces se le provoca contra ello? Ahora nacisteis para ser rebaño delante de nuestro Señor. Y os llevó a lo alto de las nubes, y sometió a los ángeles bajo vuestros pies, y os impuso una ley, y os dio mandamientos por medio de profetas, y os castigó por gobernantes, y os mostró no pocas maravillas, y por amor a vosotros. ordenaron a las luminarias y se detuvieron en los lugares donde les habían ordenado, y cuando vuestros enemigos vinieron sobre vosotros, llovió sobre ellos granizo y los destruyó, y Moisés, Jesús, Cenez y Zebul os dieron mandamientos. Y no los habéis obedecido.

\par 6 Porque mientras ellos vivieron, vosotros os mostrásteis como obedientes a vuestro Dios; pero cuando ellos murieron, también murió vuestro corazón. Y habéis llegado a ser semejantes al hierro que se mete en el fuego, que cuando se derrite con la llama se vuelve como agua, pero cuando sale del fuego vuelve a su dureza. Así también vosotros, mientras los que os amonestan os queman, mostráis el efecto, y cuando estén muertos os olvidéis de todas las cosas.

\par 7 Y ahora, he aquí, el Señor tendrá compasión de vosotros hoy, no por vosotros, sino por el pacto que hizo con vuestros padres y por el juramento que hizo de no desampararos por alguna vez. Pero sabed que después de mi muerte comenzaréis a pecar en vuestros últimos días. Por tanto, el Señor hará maravillas entre vosotros, y entregará a vuestros enemigos en vuestras manos. Porque vuestros padres están muertos, pero Dios, que hizo pacto con ellos, es vida.

\chapter{31}

\par 1 Y Débora envió y llamó a Barac y le dijo: Levántate y ciñe tus lomos como un hombre, y desciende y pelea contra Sísara, porque veo las constelaciones muy conmovidas en sus filas y preparándose para luchar por ti. Veo también los relámpagos inamovibles en su curso, y puestos a detener las ruedas de los carros de los que se jactan del poder de Sísara, el cual dice: Ciertamente descenderé con el brazo de mi fuerza para pelear contra Israel, y Repartiré el botín de ellos entre mis siervos, y tomaré para mí sus hermosas mujeres por concubinas. Por tanto, el Señor ha dicho acerca de él que el brazo de la mujer débil lo vencerá, y las doncellas tomarán su botín, y él también caerá en manos de una mujer.

\par 2 Y cuando Débora, el pueblo y Barac descendieron al encuentro de sus enemigos, inmediatamente el Señor perturbó el rumbo de sus estrellas y les habló diciendo: Apresúrate y id, porque nuestros (o vuestros) enemigos caen sobre vosotros. confunde sus brazos y quebranta la fortaleza de sus corazones, porque yo he venido para que mi pueblo prevalezca. Porque aunque mi pueblo haya pecado, tendré misericordia de él. Y dicho esto, las estrellas salieron como se les había ordenado y quemaron a sus enemigos. Y el número de los que fueron reunidos (o quemados) y asesinados en una hora fue multiplicado por 97.000 hombres. Pero a Sísara no la destruyeron, porque así se les había ordenado.

\par 3 Y cuando Sísara había huido en su caballo para librar su alma, Jahel, la esposa de Aber el cineo, se atavió con sus adornos y salió a su encuentro. Ahora bien, la mujer era muy hermosa; y cuando lo vio, dijo: Entra, toma comida y duerme; y al anochecer enviaré a mis siervos contigo, porque sé que te acordarás de mí y me recompensarás. Y entró Sísara, y cuando vio rosas esparcidas sobre la cama, dijo: Si soy liberado, oh Jahel, iré a mi madre y tú (o Jahel serás) mi esposa.

\par 4 Entonces Sísara tuvo sed y dijo a Jahel: Dame un poco de agua, porque estoy desmayado y mi alma arde a causa de la llama que vi en las estrellas. Y Jahel le dijo: Descansa un poco y luego beberás.

\par 5 Y cuando Sísara se durmió, Jahel fue al rebaño y ordeñó leche de él. Y mientras ordeñaba, dijo: He aquí, recuerda, oh Señor, que cuando dividiste cada tribu y nación sobre la tierra, ¿no escogiste sólo a Israel, y no lo comparaste con ninguna bestia, excepto sólo con el carnero que anda? delante del rebaño y lo guía? He aquí, pues, y ved cómo Sísara ha pensado en su corazón diciendo: Iré y castigaré el rebaño del Poderoso. Y he aquí, tomaré de la leche de las bestias con las cuales comparaste a tu pueblo, e iré y le daré de beber, y cuando la haya bebido se debilitará, y después lo mataré. Y esta será la señal que me darás, oh Señor, que mientras Sísara duerme, cuando yo entre, si al despertar me pide, diciendo: Dame agua para beber, entonces sabré que mi oración ha sido cumplida. sido escuchado.

\par 6 Entonces Jahel regresó y entró, y Sísara se despertó y le dijo: Dame de beber, porque me quemo mucho y mi alma está inflamada. Y Jahel tomó vino y lo mezcló con leche y le dio de beber, y él bebió y se durmió.

\par 7 Pero Jahel tomó un palo en su mano izquierda y se acercó a él, diciendo: Si el Señor me da esta señal, sabré que Sísara caerá en mis manos. He aquí, yo lo arrojo en tierra desde el lecho en que duerme, y si no se da cuenta, sabré que ha sido entregado. Y Jahel tomó a Sísara y lo empujó fuera de la cama al suelo, pero él no se dio cuenta, porque estaba muy débil. Y Jahel dijo: Fortalece en mí, oh Señor, mi brazo hoy por amor a ti y a tu pueblo, y a los que en ti confían. Y Jahel 'tomó la estaca y la puso sobre su sien y golpeó con el martillo. Y mientras moría, Sísara dijo a Jahel: He aquí, el dolor ha venido sobre mí, Jahel, y muero como una mujer. Y Jahel le dijo: Ve y jactate delante de tu padre en el infierno, y dile que has caído en (o di: he sido entregado en) manos de una mujer. Y ella acabó y lo mató y puso su cuerpo allí hasta que Barac regresara.

\par 8 La madre de Sísara se llamaba Themec y envió a sus amigas a decir: Venid, salgamos juntas a encontrarnos con mi hijo, y veréis a las hijas de los hebreos que mi hijo traerá aquí para ser suyas. concubinas.

\par 9 Pero Barac volvió de seguir a Sísara y se enojó mucho porque no lo encontró, y Jahel salió a su encuentro y le dijo: Ven, entra, bendito de Dios, y te libraré de tu enemigo a quien has atacado. seguido después y no has encontrado. Y Barac entró y encontró a Sísara muerta, y dijo: Bendito sea el Señor que envió su espíritu y dijo: En manos de una mujer será entregada Sísara. Y dicho esto, cortó la cabeza de Sísara y la envió a su madre, y le dio este mensaje diciendo: Recibe a tu hijo a quien esperabas para que viniera con el botín.



\chapter{32}

\par 1 Entonces Débora y Barac, hijo de Abino, y todo el pueblo juntos cantaron un himno al Señor aquel día, diciendo: He aquí, desde lo alto el Señor nos ha mostrado su gloria, como lo hizo antes cuando envió Emitió su voz para confundir las lenguas de los hombres. Y escogió nuestra nación, y sacó del fuego a Abraham nuestro padre, y lo escogió entre todos sus hermanos, y lo guardó del fuego y lo libró de los ladrillos de la construcción de la torre, y le dio un hijo en los últimos días de su vejez, y lo sacó del vientre estéril, y todos los ángeles tuvieron celos de él, y los capitanes de los ejércitos le tuvieron envidia.

\par 2 Y aconteció que cuando tuvieron celos contra él, Dios le dijo: Mata por mí el fruto de tu vientre y ofrece por mí lo que te he dado. Y Abraham no lo contradijo y partió inmediatamente. Y saliendo, dijo a su hijo: He aquí ahora, hijo mío, te ofrezco en holocausto y te entrego en las manos del que me dio.

\par 3 Y el hijo dijo a su padre: Escúchame, padre. Si un cordero del rebaño es aceptado como ofrenda al Señor en olor de dulzura, y si por las iniquidades de los hombres las ovejas son destinadas al matadero, pero el hombre está destinado a heredar el mundo, ¿cómo, pues, me dices ahora a mí? : ¿Venir y heredar una vida segura y un tiempo que no se puede medir? ¿Qué y si no hubiera nacido en el mundo para ser ofrecido en sacrificio al que me hizo? Y será mi bendición más allá de todos los hombres, porque no habrá otra cosa semejante; y en mí serán instruidas las generaciones, y por mí entenderán los pueblos que el Señor ha tenido por digna el alma del hombre para serle sacrificio.

\par 4 Y cuando su padre lo hubo ofrecido sobre el altar y le había atado los pies para matarlo, el Más Poderoso se apresuró y envió su voz desde lo alto, diciendo: No mates a tu hijo, ni destruyas el fruto de tu cuerpo; porque Ahora me he manifestado para aparecer a los que no me conocen, y he cerrado la boca de los que siempre hablan mal de ti. Y tu memoria estará delante de mí para siempre, y tu nombre y el nombre de este tu hijo, de generación en generación.

\par 5 Y a Isaac le dio dos hijos, que también estaban encerrados en el vientre, pues en aquel tiempo su madre estaba en el tercer año de su matrimonio. Y no será así con ninguna otra mujer, ni se jactará así ninguna mujer que se acerque a su marido en el tercer año. Y le nacieron dos hijos, Jacob y Esaú. Y Dios amó a Jacob, pero aborreció a Esaú a causa de sus obras.

\par 6 Y aconteció que en la vejez de su padre, Isaac bendijo a Jacob y lo envió a Mesopotamia, y allí engendró doce hijos, los cuales descendieron a Egipto y habitaron allí.

\par 7 Y cuando sus enemigos trataron mal con ellos, el pueblo clamó al Señor, y su oración fue escuchada, y él los sacó de allí, los condujo al monte Sina, y les mostró el fundamento del entendimiento que él se había preparado desde el nacimiento del mundo; y entonces se movieron los cimientos, las huestes lanzaron relámpagos sobre sus carreras, y los vientos resonaron en sus almacenes, y la tierra se agitó desde sus cimientos, y las montañas y las rocas temblaron en sus ataduras, y las nubes se levantaron. levantan sus ondas contra la llama del fuego para que no consuma al mundo.

\par 8 Entonces el abismo despertó de sus fuentes, y todas las olas del mar se juntaron. Entonces el Paraíso dio el aliento de sus frutos, y los cedros del Líbano fueron arrancados de sus raíces. Y las bestias del campo se aterrorizaron en las moradas de los bosques, y todas sus obras se reunieron para contemplar al Señor cuando establecía un pacto con los hijos de Israel. Y todo lo que dijo el Más Poderoso, esto lo ha observado, teniendo por testigo a Moisés su amado.

\par 9 Y cuando agonizaba, Dios le designó el firmamento y le mostró estos testigos que ahora tenemos, diciendo: Que el cielo en el que has entrado y la tierra en la que has caminado hasta ahora sean testigos entre mí y ti. y mi gente. Porque el sol, la luna y las estrellas serán ministros para nosotros (o para vosotros).

\par 10 Y cuando Jesús se levantó para gobernar al pueblo, aconteció que el día en que peleaba contra los enemigos, cuando se acercaba la tarde, cuando aún la batalla era fuerte, Jesús dijo al sol y a la luna: Oh vosotros, ministros que fuisteis nombrados entre el Más Poderoso y sus hijos, he aquí ahora que la batalla continúa, ¿y abandonáis vuestro cargo? Deteneos, pues, hoy y dad luz a sus hijos, y poned tinieblas sobre nuestros enemigos. Y así lo hicieron.

\par 11 Y en estos días Sísara se levantó para hacernos sus siervos, y clamamos al Señor nuestro Dios, y él ordenó a las estrellas y dijo: Apartaos de vuestras filas y quemad a mis enemigos, para que conozcan mi poder. Y las estrellas descendieron y derribaron su campamento y nos mantuvieron a salvo sin ningún trabajo.

\par 12 Por eso no dejaremos de cantar alabanzas, ni nuestra boca callará al hablar de sus maravillas; porque él se acordó de sus promesas, nuevas y antiguas, y nos mostró su salvación; y por eso se jacta Jahel. entre las mujeres, porque ella sola ha llevado al éxito este buen camino, al matar con sus propias manos a Sísara.

\par 13 Oh tierra, id, cielos y relámpagos, id, ángeles y ejércitos, [id] y decid a los padres en los tesoros de sus almas, y decid: El Más Poderoso no ha olvidado el y mucho menos las promesas que nos hizo, cuando dijo: Muchas maravillas haré con tus hijos. Y ahora desde este día en adelante se sabrá que todo lo que Dios ha dicho a los hombres que hará, lo hará, aunque el hombre muera.

\par 14 Canta alabanzas, canta alabanzas, oh Débora (o, si el hombre tarda en cantar alabanzas a Dios, canta tú, oh Debora), y que la gracia del espíritu santo despierte en ti y comience a alabar las obras de Jehová: porque no volverá a surgir día tal en que las estrellas lleven noticias y venzan a los enemigos de Israel, como les fue ordenado. Desde ahora en adelante, si Israel cae en apuros, llamará a estos sus testigos junto con sus ministros, e irán en embajada ante el Altísimo, y él se acordará de este día, y enviará liberación a su pacto.

\par 15 Y tú, Débora, comienza a contar lo que viste en el campo: cómo el pueblo caminaba y salía sano y salvo, y las estrellas luchaban por su parte (o cómo, como pueblos que caminan, así salían los estrellas y lucharon). Alégrate, oh tierra, por los que en ti habitan, porque en ti está el conocimiento del Señor que en ti edifica su fortaleza. Porque era justo que Dios tomara de ti la costilla del primero que fue formado, sabiendo que de su costilla nacería Israel. Y tu formación será para testimonio de lo que el Señor ha hecho por su pueblo.

\par 16 ¡Oh horas del día, deteneos y no os apresuréis a declarar lo que nuestro entendimiento puede producir, porque la noche vendrá sobre nosotros! Y será como la noche en que Dios hirió al primogénito de los egipcios por causa de su primogénito.

\par 17 Y entonces dejaré de cantar mi himno porque el tiempo será apresurado (o preparado) para sus justos. Porque le cantaré como en la renovación de la creación, y el pueblo se acordará de esta liberación, y les será de testimonio. Que también el mar y sus profundidades den testimonio, porque no sólo Dios lo secó delante de la faz de nuestros padres, sino que también derribó el campamento desde su lugar y venció a nuestros enemigos.

\par 18 Y cuando Débora terminó de hablar, subió con el pueblo a Silo, y ofrecieron sacrificios y holocaustos y tocaron las trompetas. Y cuando tocaron y hubieron ofrecido los sacrificios, Débora dijo: Esto será para testimonio de las trompetas entre las estrellas y del Señor de ellas.

\chapter{33}

\par 1 Debora descendió de allí y juzgó a Israel durante cuarenta años. Y aconteció que cuando se acercaba el día de su muerte, ella envió y reunió a todo el pueblo, y les dijo: Escuchen ahora, pueblo mío. He aquí, te amonesto como a mujer de Dios, y te doy luz como a una de la raza de las mujeres; Obedéceme ahora como a tu madre, y presta oído a mis palabras, como hombres que vosotros mismos moriréis.

\par 2 He aquí, yo voy a morir por el camino de toda carne, por el cual también vosotros iréis; con sólo dirigir vuestro corazón al Señor vuestro Dios en el tiempo de vuestra vida, porque después de vuestra muerte no podréis arrepentiros de aquellas cosas en las que vivís.

\par 3 Porque la muerte ya está sellada y cumplida, y la medida, el tiempo y los años han devuelto lo que les había sido encomendado. Porque incluso si buscáis hacer el mal en el infierno después de vuestra muerte, no podréis, porque el deseo del pecado cesará, y la creación maligna perderá su poder, y el infierno, que recibe lo que se le ha encomendado, lo hará. no restituirlo a menos que lo exija el que lo cometió. Ahora, pues, hijos míos, obedeced mi voz mientras tenéis el tiempo de la vida y la luz de la ley, y enderezad vuestros caminos.

\par 4 Y cuando Débora pronunció estas palabras, todo el pueblo alzó la voz a una y lloró, diciendo: He aquí, madre, que mueres y abandonas a tus hijos; ¿Y a quién los encomiendas? Ruega, pues, por nosotros, y después de tu partida, tu alma se acordará de nosotros para siempre.

\par 5 Entonces Débora respondió y dijo al pueblo: Mientras un hombre vive, puede orar por sí mismo y por sus hijos; pero después de su fin no podrá suplicar ni recordar a ningún hombre. Por tanto, no confiéis en vuestros padres, porque de nada os aprovecharán a menos que seáis hallados como ellos. Pero entonces vuestra semejanza será como las estrellas del cielo que se os han manifestado en este tiempo.

\par 6 Y Débora murió y durmió con sus padres y fue sepultada en la ciudad de sus padres, y el pueblo la lloró durante setenta días. Y mientras la lloraban, así hablaban en lamentación, diciendo: He aquí, ha perecido una madre de Israel, y una santa que reinó en la casa de Jacob, la que cercó el cerco alrededor de su generación, y su generación será buscarla. Y después de su muerte la tierra descansó siete años.

\chapter{34}

\par 1 En aquel tiempo se acercó un tal Aod de los sacerdotes de Madián, que era un mago, y habló a Israel, diciendo: ¿Por qué prestáis oído a vuestra ley? Ven y te mostraré algo que tu ley no es. Y el pueblo dijo: ¿Qué puedes mostrarnos que nuestra ley no tenga? Y dijo al pueblo: ¿Habéis visto alguna vez el sol de noche? Y ellos dijeron: No. Y él dijo: Cuando queréis, os lo mostraré, para que sepáis que nuestros dioses tienen poder, y no engañarán a los que les sirven. Y ellos dijeron: Muéstranos.

\par 2 Y partió y obró con su magia, dando órdenes a los ángeles encargados de los hechiceros, porque durante mucho tiempo les ofrecía sacrificios.

\par 3 [Porque esto antes estaba en poder de los ángeles y fue] realizado por los ángeles antes de ser juzgados, y habrían destruido el mundo inconmensurable; y debido a que transgredieron, aconteció que los ángeles ya no tenían poder. Porque cuando fueron juzgados, entonces el poder no fue confiado a los demás: y por estos signos (o poderes) obran los que ministran a los hombres en hechicerías, hasta que llegue la era inconmensurable.

\par 4 Y en ese momento Aod por arte de magia mostró al pueblo el sol por la noche. Y el pueblo quedó asombrado y decía: ¡Mirad qué grandes cosas pueden hacer los dioses de los madianitas, y nosotros no lo sabíamos!

\par 5 Y Dios, queriendo probar a Israel si aún estaba en iniquidad, soportó a los ángeles, y su obra tuvo buen éxito, y el pueblo de Israel fue engañado y comenzó a servir a los dioses de los madianitas. Y dijo Dios: Los entregaré en manos de los madianitas, ya que por ellos han sido engañados. Y él los entregó en sus manos, y los madianitas comenzaron a someter a Israel a servidumbre.

\chapter{35}

\par 1 Gedeón era hijo de Joat, el hombre más valiente entre todos sus hermanos. Y cuando llegó el tiempo de verano, vino al monte, trayendo consigo gavillas, para trillarlas allí y escapar de los madianitas que lo oprimían. Y el ángel del Señor salió a su encuentro y le dijo: ¿De dónde vienes y por dónde entras?

\par 2 Él le dijo: ¿Por qué me preguntas de dónde vengo? porque la angustia me rodea, porque Israel ha caído en aflicción, y en verdad han sido entregados en manos de los madianitas. ¿Y dónde están las maravillas que nos contaron nuestros padres, diciendo: El Señor escogió solo a Israel entre todos los pueblos de la tierra? He aquí que ahora nos ha entregado y ha olvidado las promesas que hizo a nuestros padres. Porque preferiríamos ser entregados a muerte de una vez por todas, que que su pueblo fuera castigado así una y otra vez.

\par 3 Y el ángel del Señor le dijo: No es por nada que estáis entregados, sino que vuestras propias invenciones han traído estas cosas sobre vosotros, porque así como habéis abandonado las promesas que recibisteis del Señor, Estos males os han sobrevenido, y no os habéis acordado de los mandamientos de Dios que os ordenaron los que fueron antes de vosotros. Por eso habéis entrado en desagrado de vuestro Dios. Pero él tendrá misericordia de vosotros, como nadie la tiene, ni siquiera del linaje de Israel, y no por vosotros, sino por los que están dormidos.

\par 4 Ahora pues, ven, yo te enviaré y librarás a Israel de la mano de los madianitas. Porque así dice el Señor: Aunque Israel no sea justo, sin embargo, por cuanto los madianitas son pecadores, por tanto, conociendo la iniquidad de mi pueblo, los perdonaré, y después los reprenderé por haber hecho mal, pero sobre el Madianitas, pronto seré vengado.

\par 5 Y Gedeón dijo: ¿Quién soy yo y cuál es la casa de mi padre para ir a la batalla contra los madianitas? Y el ángel le dijo: Quizás pienses que como es el camino del hombre, así es el camino de Dios. Porque los hombres miran la gloria del mundo y las riquezas, pero Dios mira lo recto y lo bueno, y la mansedumbre. Ahora pues, ve, ciñe tus lomos, y el Señor estará contigo, porque te ha elegido para vengarte de sus enemigos, tal como he aquí, él te ha ordenado.

\par 6 Y Gedeón le dijo: No se enoje mi Señor si digo una palabra. He aquí Moisés, el primero de todos los profetas, rogó al Señor una señal, y le fue dada. Pero ¿quién soy yo, sino el Señor que me ha escogido, que me dé una señal para saber que voy por buen camino? Y el ángel del Señor le dijo: Corre y sácame agua del hoyo de allá y derrámala sobre esta roca, y te daré una señal. Y él fue y la tomó como le mandó.

\par 7 Y el ángel le dijo: Antes de verter el agua sobre la roca, pregunta en qué quieres que se convierta: sangre, fuego o que no aparezca en absoluto. Y Gedeón dijo: Que se convierta en mitad de sangre y mitad de fuego. Y Gedeón derramó el agua sobre la roca, y aconteció que cuando la hubo derramado, la mitad se convirtió en llama, y ​​la mitad en sangre, y se mezclaron, es decir, el fuego y la sangre, pero la sangre no apagó el fuego, ni el fuego consumió la sangre. Y cuando Gedeón vio esto, pidió aún otras señales, y se las dieron. ¿No están esto escritos en el libro de los Jueces?

\chapter{36}

\par 1 Gedeón tomó trescientos hombres y partió y llegó al extremo del campamento de Madián, y oyó a cada uno hablar con su vecino y decir: Veréis una confusión incalculable de la espada de Gedeón que viene sobre nosotros, porque Dios ha entregado en sus manos el campamento de los madianitas, y comenzará a acabar con nosotros, incluso con la madre con los hijos, porque nuestros pecados están llenos, así como también nuestros dioses nos han mostrado y nosotros No les creí. Y ahora levantémonos, socorramos nuestras almas y volemos.

\par 2 Y cuando Gedeón oyó estas palabras, inmediatamente se invistió del espíritu del Señor y, dotado de poder, dijo a los trescientos hombres: Levantaos y ceñios cada uno su espada, porque los madianitas son entregados en nuestras manos. Y los hombres descendieron con él, y él se acercó y comenzó a pelear. Y tocaron la trompeta y clamaron a una, y dijeron: La espada del Señor está sobre nosotros. Y mataron de los madianitas a unos ciento veinte mil hombres, y el resto de los madianitas huyó.

\par 3 Después de estas cosas vino Gedeón, reunió a los hijos de Israel y les dijo: He aquí, el Señor me envió a pelear vuestra batalla, y fui tal como él me había ordenado. Y ahora te pido una petición: no apartes tu rostro; y que cada uno de vosotros me dé los brazaletes de oro que tenéis en las manos. Y Gedeón extendió una túnica, y cada uno echó sobre ella sus brazaletes, y todos fueron pesados, y se halló que su peso era de 12 talentos (o 12.000 siclos). Y Gedeón los tomó, y con ellos hizo ídolos y los adoró.

\par 4 Y dijo Dios: En verdad está establecido un camino: no reprender a Gedeón mientras viva, incluso porque cuando destruyó el santuario de Baal, entonces todos dijeron: ¡Que Baal se vengue! Ahora, pues, si le castigo por haber hecho mal contra mí, diréis: No fue Dios el que le castigó, sino Baal, porque pecó antes contra él. Por tanto, ahora Gedeón morirá en buena vejez, y no tendrán de qué hablar. Pero después que Gedeón muera, lo castigaré una vez más, porque se ha rebelado contra mí. Y Gedeón murió en buena vejez y fue sepultado en su propia ciudad.



\chapter{37}

\par 1 Y tuvo un hijo de una concubina que se llamaba Abimelec; Éste mató a todos sus hermanos, queriendo gobernar al pueblo.

\par [Se ha perdido una hoja.]

\par 2 Entonces todos los árboles del campo se reunieron junto a la higuera y dijeron: Ven, reina sobre nosotros. Y la higuera dijo: ¿Nací realmente en el reino o en el señorío de los árboles? ¿O fui plantado para eso y reinar sobre vosotros? Y por tanto, así como yo no puedo reinar sobre vosotros, tampoco Abimelec obtendrá continuidad en su gobierno. Después los árboles se juntaron junto a la vid y dijeron: Ven, reina sobre nosotros. Y la vid dijo: Fui plantada para dar a los hombres la dulzura del vino, y soy preservada dándoles mi fruto. Pero así como yo no puedo reinar sobre vosotros, así será demandada de vuestras manos la sangre de Abimelec. Y después los árboles se acercaron al manzano y le dijeron: Ven, reina sobre nosotros. Y él dijo: Me fue mandado dar a los hombres un fruto de olor agradable. Por tanto, no puedo reinar sobre vosotros, y Abimelec morirá apedreado.

\par 3 Entonces los árboles se acercaron a las zarzas y dijeron: Venid, reinad sobre nosotros. Y la zarza dijo: Cuando nació la espina, la verdad brilló con la apariencia de una espina. Y cuando nuestro primer padre fue condenado a muerte, la tierra fue condenada a producir espinos y cardos. Y cuando la verdad iluminó a Moisés, fue a través de una zarza que lo iluminó. Ahora pues, será que por mí la verdad será oída de vosotros. Ahora bien, si habéis hablado con sinceridad a la zarza para que en verdad reine sobre vosotros, sentaos bajo su sombra; pero si con disimulo, entonces salga el fuego y devore y consuma los árboles del campo. Porque el manzano fue hecho para los castigadores, y la higuera para el pueblo, y la viña para los que fueron antes de nosotros.

\par 4 Y ahora la zarza os será como Abimelec, que mató a sus hermanos con injusticia y desea gobernarse sobre vosotros. Si Abimelec es digno de aquellos (o sea Abimelec un fuego para ellos) a quienes desea gobernar, sea como la zarza que se hizo para reprender a los insensatos del pueblo. Y de la zarza salió fuego y devoró los árboles que están en el campo.

\par 5 Después de esto, Abimelec reinó sobre el pueblo durante un año y seis meses, y murió gravemente junto a una torre, desde donde una mujer arrojó sobre él la mitad de una piedra de molino.

\par [Un espacio de longitud incierta en el texto.]

\chapter{38}

\par 1 (Entonces Jair juzgó a Israel durante veintidós años). Este edificó un santuario a Baal, y extravió al pueblo, diciendo: Todo el que no ofrezca sacrificios a Baal, morirá. Y cuando todo el pueblo sacrificó, sólo siete hombres no quisieron sacrificar, cuyos nombres son estos: Defal, Abiesdrel, Getalibal, Selumi, Assur, Jonadali y Memihel.

\par 2 Éste respondió y dijo a Jair: He aquí, nos acordamos de los preceptos que nos ordenaron los que nos precedieron, y de Débora nuestra madre, diciendo: Mira, no desvíes tu corazón ni a derecha ni a izquierda. , sino que atendáis a la ley del Señor día y noche. Ahora pues, ¿por qué corrompes al pueblo del Señor y lo engañas, diciendo: Baal es Dios, adorémosle? Y ahora, si es Dios como tú dices, que hable como Dios, y entonces le ofreceremos sacrificios.

\par 3 Y Jair dijo: Quemadlos en el fuego, porque han blasfemado contra Baal. Y sus siervos los tomaron para quemarlos al fuego. Y cuando los echaron sobre el fuego, salió Natanael, el ángel que está sobre el fuego, y apagó el fuego y quemó a los siervos de Jair; pero a los siete hombres los hizo escapar, de modo que ningún hombre del pueblo los vio. , porque había herido al pueblo con ceguera.

\par 4 Y cuando Jair llegó al lugar (o llegó al lugar de Jair), él también fue quemado. Pero antes de quemarlo, el ángel del Señor le dijo: Oye la palabra del Señor antes de que mueras. Así dice el Señor: Yo te levanté de la tierra de Egipto y te nombré gobernante de mis pueblos. Pero tú te has levantado y has corrompido mi pacto, y los has extraviado, y has procurado quemar en la llama a mis siervos, porque te reprendieron, los cuales, aunque quemados con fuego corruptible, ahora son vivificados con fuego vivo y son entregado. Pero tú morirás, dice el Señor, y en el fuego en que morirás, allí tendrás tu morada. Y después lo quemó, y llegó hasta la columna de Baal y la derribó, y quemó a Baal con el pueblo que estaba allí, es decir, 1000 hombres.

\chapter{39}

\par 1 Después de esto vinieron los hijos de Amón y comenzaron a luchar contra Israel y tomaron muchas de sus ciudades. Y estando el pueblo en gran angustia, se reunieron en Masfat, diciendo cada uno a sus vecinos: He aquí ahora vemos el estrecho que nos rodea, y el Señor se ha apartado de nosotros, y ya no está con nosotros, ni con nuestros enemigos. han tomado nuestras ciudades, y no hay líder que entre y salga delante de nosotros. Ahora, pues, veamos a quién podemos poner sobre nosotros para pelear nuestra batalla.

\par 2 Jeptán galaadita era un hombre valiente y valiente, y por envidia de sus hermanos, lo echaron de su tierra, y se fue y habitó en la tierra de Tobi. Y los vagabundos se reunieron con él y se quedaron con él.

\par 3 Y aconteció que cuando Israel fue vencido en la batalla, llegaron a la tierra de Tobi a Jeptán y le dijeron: Ven, domina al pueblo. Porque ¿quién sabe si por eso fuiste preservado hasta el día de hoy o si por eso fuiste librado de las manos de tus hermanos para que en este tiempo pudieras gobernar a tu pueblo?

\par 4 Y Jeptán les dijo: ¿Vuelve así el amor después del odio, o el tiempo vence todas las cosas? Porque me echasteis de mi tierra y de la casa de mi padre; ¿Y ahora venís a mí estando en apuros? Y ellos le dijeron: Si el Dios de nuestros padres no se acordó de nuestros pecados, sino que nos libró cuando habíamos pecado contra él y nos había entregado delante de nuestros enemigos, y éramos oprimidos por ellos, ¿por qué quieres eso? ¿Puede un hombre mortal recordar las iniquidades que nos sucedieron en el tiempo de nuestra aflicción? Por tanto, no sea así delante de ti, señor.

\par 5 Y Jeptán dijo: Dios ciertamente puede olvidar nuestros pecados, ya que tiene tiempo y lugar para descansar de su paciencia, porque él es Dios; pero yo soy mortal, hecho de tierra: ¿a dónde volveré, y dónde arrojaré mi ira y el mal con que me habéis injuriado? Y el pueblo le dijo: Deja que te enseñe la paloma con la cual Israel era comparado, porque aunque le sean quitados sus crías, no se aparta de su lugar, sino que desprecia su agravio y lo olvida como si fuera en el fondo de las profundidades.

\par 6 Entonces Jeptán se levantó y fue con ellos, reunió a todo el pueblo y les dijo: Vosotros sabéis que cuando nuestros príncipes estaban vivos, nos advirtieron que siguiéramos nuestra ley. Y Ammón y sus hijos desviaron al pueblo del camino por el que andaban, para servir a otros dioses que los destruirían. Ahora pues, fijad vuestro corazón en la ley del Señor vuestro Dios, y supámosle unánimes. Y así lucharemos contra nuestros adversarios, y confiaremos y esperaremos en el Señor que no nos entregará para siempre. Porque aunque nuestros pecados abundan, sin embargo su misericordia llena toda la tierra.

\par 7 Y todo el pueblo oró unánimemente, tanto hombres como mujeres, niños y niños de pecho. Y cuando oraron, dijeron: Mira, oh Señor, al pueblo que has elegido, y no estropees la vid que plantó tu diestra; para que sea en herencia delante de ti este pueblo que has poseído desde el principio, y al cual siempre has preferido, y por cuyo amor hiciste las habitaciones habitables, y los metiste en la tierra que les juraste; No nos entregues delante de los que te odian, oh Señor.

\par 8 Y Dios se arrepintió de su ira y fortaleció el espíritu de Jeptán. Y envió un mensaje a Getal rey de los hijos de Amón, y le dijo: ¿Por qué afliges nuestra tierra y tomas mis ciudades, o por qué nos afliges? No te ha mandado el Dios de Israel que destruyas a los que habitan en la tierra. Ahora pues, devuélveme mis ciudades, y mi ira cesará contra ti. Pero si no, sabe que subiré a ti y te pagaré lo primero, y retribuiré tu maldad sobre tu cabeza. ¿No te acuerdas de cómo trataste con engaño a los hijos de Israel en el desierto? Y los mensajeros de Jeptán hablaron estas palabras al rey de los hijos de Amón.

\par 9 Y Getal dijo: ¿Se preocupó Israel cuando tomó la tierra de los amorreos? Di, pues: Sabed que ahora quitaré de ti el resto de tus ciudades y te pagaré tu maldad y me vengaré de los amorreos a quienes has agraviado. Y Jeptán envió otra vez al rey de los hijos de Amón, diciendo: En verdad veo que Dios te ha traído acá para destruirte, a menos que descanses de tu iniquidad con la que afliges a Israel. Y por eso vendré a ti y me mostraré a ti. Porque no son, como decís, dioses los que os han dado la herencia que poseéis. Pero porque os habéis extraviado tras las piedras, el fuego os seguirá para venganza.

\par 10 Y como el rey de los hijos de Amón no quiso escuchar la voz de Jeptán, éste se levantó y armó a todo el pueblo para salir y pelear en las fronteras, diciendo: Cuando los hijos de Amón sean entregados en mis manos y yo sea Cuando regrese, el que primero se encuentre conmigo será en holocausto al Señor.

\par 11 Y el Señor se enojó mucho y dijo: He aquí, Jeptán ha prometido ofrecerme lo que encuentre primero. Ahora bien, si un perro encuentra primero a Jeptán, ¿me será ofrecido un perro? Y ahora sea el voto de Jeptán sobre su primogénito, sobre el fruto de su vientre, y su oración sobre su hija unigénita. Pero en verdad libraré a mi pueblo en este tiempo, no por él, sino por la oración que Israel ha hecho.

\chapter{40}

\par 1 Y Jeptán vino y peleó contra los hijos de Amón, y el Señor los entregó en su mano, y derrotó a sesenta de sus ciudades. Y Jepthan regresó en paz. Y las mujeres salieron a recibirle con bailes. Y tenía una hija unigénita; la misma salió primero en los bailes para encontrarse con su padre. Y cuando Jeptán la vio, se desmayó y dijo: Con razón se llama tu nombre Seila, para que seas ofrecida en sacrificio. ¿Y ahora quién pondrá en la balanza mi corazón y pesará mi alma? ¿Y estaré de pie y veré si uno supera al otro, el gozo que viene o la aflicción que sobre mí? porque en lo que abrí mi boca a mi Señor en el cántico de mis votos, no puedo volver a llamarla.

\par 2 Y su hija Seila le dijo: ¿Y quién puede entristecerse por su muerte cuando ve al pueblo liberado? ¿No te acuerdas de lo que sucedió en los días de nuestros padres, cuando el padre ofrecía a su hijo en holocausto y él no lo contradecía, sino que consentía en recibirlo gozoso? Y el que fue ofrecido estaba listo, y el que lo ofreció se alegró.

\par 3 Ahora, pues, no anules nada de lo que has prometido, sino concédeme una sola oración. Te pido antes de morir una pequeña petición: te ruego que antes de entregar mi alma, pueda ir a las montañas y vagar (o morar) entre las colinas y caminar entre las rocas, yo y las vírgenes que están. mis compañeros, y derramaré allí mis lágrimas y contaré la aflicción de mi juventud; y los árboles del campo lamentarán por mí y las bestias del campo se lamentarán por mí; porque no estoy triste por morir, ni me duele entregar mi alma; pero si mi padre fue superado en su voto, [y] si no me ofrezco voluntariamente en sacrificio, temo que mi muerte no será aceptable y que perderé la vida sin ningún propósito. Estas cosas contaré a los montes, y después volveré. Y su padre dijo: Ve.

\par 4 Y salió Seila, hija de Jeftán, ella y las vírgenes que eran sus compañeras, y vino y lo contó a los sabios del pueblo. Y ningún hombre pudo responder a sus palabras. Y después de esto fue al monte Stelac, y de noche el Señor pensó en ella y dijo: He aquí ahora he cerrado la lengua de los sabios entre mi pueblo antes de esta generación, para que no puedan responder a la palabra del hija de Jeptán, para que se cumpliera mi palabra, y no se perdiera el consejo que había ideado; y he visto que ella es más sabia que su padre, y más prudente que todos los sabios que están aquí. Y ahora que se le dé la vida a petición suya, y su muerte será preciosa ante mis ojos en todo tiempo.

\par 5 Y cuando la hija de Jeptán llegó al monte Stelac, comenzó a lamentarse. Y este es su lamento con el que se lamentó y se lamentó antes de partir, y dijo: Escuchen, oh montañas, mi lamento, y miren, oh colinas, las lágrimas de mis ojos, y sean testigos, oh rocas, en el lamento de mi alma. He aquí cómo soy acusado, pero mi alma no será quitada en vano. Que mis palabras salgan a los cielos, y que mis lágrimas se escriban ante la faz del firmamento, para que el padre no venza (ni pelee) a su hija, a quien ha prometido ofrecer, para que su gobernante oiga que su Se promete a la hija unigénita para un sacrificio.

\par 6 Sin embargo, no me he saciado de mi lecho nupcial, ni me he llenado de las guirnaldas de mis bodas. Porque no me he vestido de esplendor, sentado en mi virginidad; No he usado mi ungüento precioso, ni mi alma ha disfrutado del aceite de la unción que me fue preparado. Oh madre mía, en vano has dado a luz a tu unigénita y la has engendrado en la tierra, porque el infierno se ha convertido en mi cámara nupcial. Que se derrame toda la mezcla de aceite que me has preparado, y que el manto blanco que me tejió mi madre, se lo coma la polilla, y que se seque la corona de flores que mi nodriza antes me tejió. y el manto que ella tejió de violeta y púrpura para mi virginidad, que el gusano lo estropee; y cuando las vírgenes amigas mías hablen de mí, me llorarán con gemidos durante muchos días.

\par 7 ¡Oh árboles, inclinad vuestras ramas y lamentad mi juventud! Venid, bestias del bosque, y pisotead mi virginidad. Porque mis años han sido cortados, y los días de mi vida envejecen en oscuridad.

\par 8 Y cuando ella hubo dicho esto, Seila volvió a su padre, y éste cumplió todo lo que había prometido y ofreció holocaustos. Entonces se reunieron todas las doncellas de Israel y sepultaron a la hija de Jeftán y lloraron por ella. Y los hijos de Israel hicieron una gran lamentación y fijaron en aquel mes, el día 14 del mes, que se reunieran cada año y lamentaran por la hija de Jeftán cuatro días. Y llamaron el nombre de su sepulcro según su propio nombre, Seila.

\par 9 Y Jeptán juzgó a los hijos de Israel diez años, y murió, y fue sepultado con sus padres.

\chapter{41}

\par 1 Después de él se levantó un juez en Israel, Addo, hijo de Elec de Pratón, que también juzgó a los hijos de Israel durante ocho años. En sus días, el rey de Moab envió mensajeros a él, diciendo: He aquí, ahora sabes que Israel ha tomado mis ciudades; ahora, pues, devuélvelas en pago. Y Addo dijo: ¿No estáis todavía instruidos por lo que ha acontecido a los hijos de Amón, a menos que tal vez se cumplan los pecados de Moab? Y Addo envió y tomó del pueblo 20.000 hombres y vino contra Moab, y peleó contra ellos y mató de ellos a 45.000 hombres. Y el resto huyó delante de él. Y Addo volvió en paz y ofreció holocaustos y sacrificios a su Señor, y murió, y fue sepultado en Efrata su ciudad.

\par 2 En aquel tiempo el pueblo escogió a Elón y lo puso como juez sobre ellos, y él juzgó a Israel veinte años. En aquellos días pelearon contra los filisteos y tomaron de ellos doce ciudades. Y murió Elón y fue sepultado en su ciudad.

\par 3 Pero los hijos de Israel se olvidaron del Señor su Dios y sirvieron a los dioses de los habitantes de la tierra. Por tanto, fueron entregados a los filisteos y les sirvieron cuarenta años.

\chapter{42}

\par 1 Había un hombre de la tribu de Dan, cuyo nombre era Manue, hijo de Edoc, hijo de Odo, hijo de Eriden, hijo de Phadesur, hijo de Dema, hijo de Susi, el hijo de Dan. Y tenía una esposa que se llamaba Eluma, hija de Remac. Y ella era estéril y no le dio hijos. Y cuando Manue su marido le decía de día en día: He aquí, Jehová ha cerrado tu vientre, para que no concibas; Déjame, pues, libre para que pueda tomar otra mujer, no sea que muera sin descendencia. Y ella dijo: No me ha impedido el Señor dar a luz, sino a ti, para que no dé fruto. Y él le dijo: Que la ley aclare nuestra prueba.

\par 2 Y como discutían día tras día, y ambos estaban muy afligidos porque les faltaba fruto, una noche la mujer subió al aposento alto y oró diciendo: Tú, oh Señor Dios de toda carne, revela a Yo sé si a mi marido o a mí no me es dado engendrar hijos, o a quien está prohibido o a quien se le permite dar fruto, para que a quien le está prohibido, llore por sus pecados, porque continúa sin fruto. O si ambos somos privados, revélanos esto también, para que carguemos con nuestro pecado y guardemos silencio delante de ti.

\par 3 Y el Señor escuchó su voz y le envió su ángel por la mañana, y le dijo: Tú eres la estéril que no da a luz, y tú eres el vientre que está prohibido dar fruto. Pero ahora el Señor ha oído tu voz y ha visto tus lágrimas y ha abierto tu vientre. Y he aquí, concebirás y darás a luz un hijo, y llamarás su nombre Sansón, porque será santo para tu Señor. Pero mirad que no pruebe ningún fruto de la vid, ni coma cosa inmunda, porque como él mismo ha dicho, él librará a Israel de la mano de los filisteos. Y cuando el ángel del Señor hubo dicho estas palabras, se apartó de ella.

\par 4 Y ella entró en casa de su marido y le dijo: He aquí, pongo mi mano sobre mi boca y guardaré silencio delante de ti todos mis días, porque en vano me jactaba y no creía en tus palabras. palabras. Porque el ángel del Señor vino a mí hoy y me mostró, diciendo: Eluma, eres estéril, pero concebirás y darás a luz un hijo.

\par 5 Y Manuel no creyó a su esposa. Y él, avergonzado y entristecido, subió también él al aposento alto y oró diciendo: He aquí, no soy digno de oír las señales y prodigios que Dios ha hecho en nosotros, ni de ver el rostro de su mensajero.

\par 6 Y aconteció que mientras él hablaba así, el ángel del Señor vino otra vez a su esposa. Ahora ella estaba en el campo y Manue en su casa. Y el ángel le dijo: Corre y llama a tu marido, porque Dios lo ha tenido por digno de oír mi voz.

\par 7 Y la mujer corrió y llamó a su marido, y él se apresuró y vino donde el ángel en el campo en Ammo (?), el cual le dijo: Entra con tu esposa y haz rápidamente todas estas cosas. Pero él le dijo: Mira, Señor, que se cumpla tu palabra sobre tu siervo. Y él dijo: Así será.

\par 8 Y Manue le dijo: Si pudiera, te convencería de que entraras en mi casa y comieras pan conmigo, y saber que cuando te vayas te daría regalos para que los lleves contigo y puedas ofrecer un sacrificio al Señor tu Dios. Y el ángel le dijo: No entraré contigo en tu casa, ni comeré tu pan, ni recibiré tus presentes. Porque si ofreces un sacrificio de lo que no es tuyo, no puedo mostrarte gracia.

\par 9 Y Manue edificó un altar sobre la roca y ofreció sacrificios y holocaustos. Y aconteció que cuando cortó la carne y la puso en el lugar santo, el ángel extendió su mano y la tocó con la punta de su cetro. Y salió fuego de la roca y consumió los holocaustos y sacrificios. Y el ángel subió de él con la llama del fuego.

\par 10 Pero Manue y su mujer, al ver esto, cayeron sobre sus rostros y dijeron: Ciertamente moriremos, porque hemos visto al Señor cara a cara. Y no me bastó verlo, sino que también pregunté su nombre, sin saber que era ministro de Dios. El ángel que vino se llamaba Padahel.



\chapter{43}

\par 1 Y aconteció en aquellos días que Eluma concibió y dio a luz un hijo y llamó su nombre Sansón. Y el Señor estaba con él. Y cuando ya era mayor y procuraba pelear contra los filisteos, tomó para sí mujer de los filisteos. Y los filisteos la quemaron al fuego, porque Sansón los había humillado mucho.

\par 2 Y después de esto Sansón entró en Azoto (o se enfureció contra él). Y lo encerraron, y rodearon la ciudad, y dijeron: He aquí, ahora nuestro adversario ha sido entregado en nuestras manos. Ahora pues, reunámonos y socorramos las almas unos a otros. Y cuando Sansón se levantó de noche y vio la ciudad cerrada, dijo: He aquí ahora estas pulgas me han encerrado en su ciudad. Y ahora el Señor estará conmigo, y saldré por sus puertas y pelearé contra ellos.

\par 3 Y fue y puso su mano izquierda debajo del cerrojo de la puerta, la sacudió y derribó la puerta del muro. Una de las puertas la tomó en su mano derecha como escudo, y la otra la puso sobre sus hombros y la quitó, y como no tenía espada, persiguió con ella a los filisteos, y mató con ella a veinticinco mil hombres. Y levantó todos los objetos de la puerta y los puso sobre un monte.

\par 4 En cuanto al león que mató, y la quijada del asno con que hirió a los filisteos, y las ataduras que rompió de sus brazos como si fueran propias, y las zorras que cazó, ¿no son estas cosas? escrito en el libro de los Jueces?

\par 5 Entonces Sansón descendió a Gerara, ciudad de los filisteos, y vio allí a una ramera que se llamaba Dalila, y fue llevado tras ella y la tomó por mujer. Y dijo Dios: He aquí, ahora Sansón se ha descarriado por sus ojos y se ha olvidado de los milagros que he hecho con él, y se ha mezclado con las hijas de los filisteos, y no ha considerado a mi siervo José, que estaba en tierra extraña. y fue corona para sus hermanos, porque no quiso afligir a su descendencia. Ahora, pues, su concupiscencia será tropezadero para Sansón, y su mezcla será su destrucción, y yo lo entregaré a sus enemigos, y ellos lo cegarán. Sin embargo, en la hora de su muerte me acordaré de él y lo vengaré una vez más contra los filisteos.

\par 6 Y después de estas cosas su mujer le insistió, diciéndole: Muéstrame tu fuerza, y en qué consiste tu poder. Así sabré que me amas. Y cuando Sansón la engañó tres veces, y ella insistía todos los días en importunarlo, la cuarta vez le mostró su corazón. Pero ella lo emborrachó, y cuando él se durmió llamó a un barbero, y él le afeitó los siete cabellos de la cabeza, y su poder se apartó de él, porque así mismo se lo había revelado. Y ella llamó a los filisteos, y ellos hirieron a Sansón, lo cegaron y lo metieron en la cárcel.

\par 7 Y aconteció que el día del banquete llamaron a Sansón para burlarse de él. Y él, atado entre dos columnas, oró diciendo: Oh Señor, Dios de mis padres, escúchame todavía esta vez, y fortaléceme para que muera con estos filisteos; porque esta visión de los ojos que me han quitado me ha sido dada gratuitamente. a mí por ti. Y Sansón añadió diciendo: Sal, alma mía, y no te entristezcas. Muere, oh cuerpo mío, y no llores por ti mismo.

\par 8 Y agarrándose a las dos columnas de la casa, las sacudió. Y la casa se derrumbó y todo lo que había en ella, y mató a todos los que estaban alrededor de ella, y el número de ellos fue de cuarenta mil hombres y mujeres. Y descendieron los hermanos de Sansón y toda la casa de su padre, y lo tomaron y lo sepultaron en el sepulcro de su padre. Y juzgó a Israel veinte años.

\chapter{44}

\par 1 En aquellos días no había ningún príncipe en Israel, sino que cada uno hacía lo que le agradaba.

\par 2 En aquel tiempo se levantó Mijás, hijo de Dedila, madre de Heliu, y tenía mil dracmas de oro, cuatro cuñas de oro fundido y cuarenta didracmas de plata. Y su madre Dedila le dijo: Hijo mío, oye mi voz y te harás un nombre antes de tu muerte: toma ese oro y derrímelo, y te harás ídolos, y serán para ti dioses, y tú serás para ellos sacerdote.

\par 3 Y sucederá que cualquiera que pregunte por ellos, vendrá a ti y tú les responderás. Y en tu casa se edificará un altar y una columna, y del oro que tienes te comprarás incienso para quemar y ovejas para los sacrificios. Y el que ofrezca sacrificio dará por las ovejas siete didracmas, y por el incienso, si lo quema, dará una didracma de plata de su peso completo. Y tu nombre será Sacerdote, y serás llamado adorador de los dioses.

\par 4 Y Michas le dijo: Madre mía, me has aconsejado bien cómo puedo vivir; y ahora tu nombre será mayor que el mío, y en los últimos días se te pedirán estas cosas.

\par 5 Y Michas fue e hizo todo lo que su madre le había ordenado. Y talló y se hizo tres imágenes de niños, de becerros, de un león, de un águila, de un dragón y de una paloma. Y acontecía que todos los descarriados venían a él, y si alguno quería pedir esposas, le preguntaban por medio de la paloma; y si para hijos, por la imagen de los muchachos; pero el que pedía riquezas se aconsejaba por la imagen del águila, y el que pedía fuerza por la imagen del león; otra vez, si pedían hombres y doncellas Preguntaban por imágenes de becerros, pero si por días largos, preguntaban por la imagen del dragón. Y su iniquidad fue de muchas formas, y su impiedad estaba llena de engaño.

\par 6 Por eso, cuando los hijos de Israel se apartaron del Señor, el Señor dijo: He aquí, yo desarraigaré la tierra y destruiré toda la raza de los hombres, porque cuando dispuse grandes cosas en el monte Sina, me mostré a los hijos de Israel en la tempestad y dije que no se hicieran ídolos, y ellos consintieron en que no tallaran figura de dioses. Y les ordené que no tomaran mi nombre en vano, y eligieron esto, ni siquiera tomar mi nombre en vano. Y les mandé que guardaran el día de reposo, y ellos consentieron en santificarse. Y les dije que honraran a su padre y a su madre: y ellos prometieron que así lo harían. Y les ordené que no hurtaran, y ellos consintieron. Y les mandé que no mataran, y lo recibieron, que no lo hicieran. Y les mandé que no cometieran adulterio, y no se negaron. Y les ordené que no dieran falso testimonio, y que cada uno no codiciara la mujer de su prójimo, ni su casa, ni ninguna cosa suya; y ellos lo aceptaron.

\par 7 Y ahora, mientras yo les había dicho que no hicieran ídolos, ellos han hecho las obras de todos esos dioses que nacen de la corrupción con el nombre de imágenes talladas. Y también de aquellos por quienes se han corrompido todas las cosas. Porque los hombres mortales los hicieron, y el fuego sirvió para fundirlos: el acto de los hombres los produjo, y las manos los labraron, y el entendimiento los ideó. Y mientras los recibieron, tomaron mi nombre en vano, y dieron mi nombre a imágenes talladas, y en el día de reposo que aceptaron para guardarlo, hicieron abominaciones con él. Por cuanto les dije que amaran a su padre y a su madre, me han deshonrado a mí, su hacedor. Y por eso les dije que no robaran; han cometido robos en su entendimiento con imágenes talladas. Y si bien dije que no debían matar, sí los matan cuando engañan. Y cuando les mandé que no cometieran adulterio, se hicieron adúlteros con sus celos. Y cuando escogieron no dar falso testimonio, recibieron falso testimonio de los que habían expulsado, y codiciaron mujeres extrañas.

\par 8 Por tanto, he aquí que aborrezco la raza de los hombres, y para poder desarraigar mi creación, los que mueran se multiplicarán más que los que nacen. Porque la casa de Jacob está contaminada con iniquidades y las impiedades de Israel se han multiplicado y no puedo [algunas palabras perdidas] destruir por completo a la tribu de Benjamín, porque fueron los primeros en ser llevados tras Micas. Y el pueblo de Israel tampoco quedará impune, sino que será para ellos un escándalo para siempre, en la memoria de todas las generaciones.

\par 9 Pero a Micas entregaré al fuego. Y su madre desfallecerá delante de él, viviendo en la tierra, y de su cuerpo saldrán gusanos. Y cuando hablen el uno con el otro, ella dirá como una madre reprendiendo a su hijo: Mira qué pecado has cometido. Y él responderá como un hijo obediente a su madre y obrando con astucia: Y tú has cometido una iniquidad aún mayor. Y la figura de la paloma que hizo será para sacarle los ojos, y la figura del águila será para derramar fuego de sus alas, y las imágenes de los niños que hizo serán para rasparle los costados, y porque la imagen del león que él hizo, le será como poderosos que lo atormentan.

\par 10 Y así haré no sólo con Micas, sino también con todos los que pecan contra mí. Y ahora vamos a la carrera. de los hombres saben que no me provocarán con sus propias invenciones. No sólo a los que hacen ídolos vendrá este castigo, sino que a cada uno se le juzgará por el pecado que haya cometido. Por tanto, si delante de mí hablan mentira, yo mandaré al cielo y él les quitará la lluvia. Y si alguno codicia los bienes de su prójimo, ordenaré la muerte y le negará el fruto de su cuerpo. Y si juran en mi nombre en falso, no soportaré su oración. Y cuando el alma se separe del cuerpo, entonces dirán: No nos lamentemos por lo que hemos padecido, sino porque todo lo que hemos ideado, eso también lo recibiremos.

\chapter{45}

\par 1 Y aconteció en aquel tiempo que un hombre de la tribu de Leví llegó a Gabaón, y cuando quiso quedarse allí, se puso el sol. Y cuando entraba allí, los que allí habitaban no le permitían. Y dijo a su muchacho: Anda, guía la mula, y vamos a la ciudad de Noba, tal vez nos permitan entrar allí. Y llegó allí y se sentó en la calle de la ciudad. Y nadie le dijo: Entra en mi casa.

\par 2 Pero estaba allí un levita que se llamaba Betac. Este lo vio y le dijo: ¿Eres Beel de mi tribu? Y él dijo: Yo soy. Y él le dijo: ¿No conoces la maldad de los que habitan en esta ciudad? ¿Quién te aconsejó entrar aquí? Date prisa y sal de aquí, y entra en mi casa en la que habito, y moraré allí hoy, y el Señor cerrará su corazón delante de nosotros, como cerró a los hombres de Sodoma delante de Lot. Y entró en la ciudad y se quedó allí aquella noche.

\par 3 Y se reunieron todos los habitantes de la ciudad y dijeron a Bethac: Saca a los que han venido a ti hoy; si no, los quemaremos a ellos y a ti en el fuego. Y saliendo a ellos, les dijo: ¿No son ellos nuestros hermanos? No hagamos mal con ellos, no sea que nuestros pecados se multipliquen contra nosotros. Y ellos respondieron: Nunca fue así que los extraños dieran órdenes a los habitantes. Y entraron con violencia y lo sacaron a él y a su concubina y los echaron fuera, y ellos: 'Dejad ir al hombre, pero abusaron de su concubina hasta que murió; porque ella había transgredido contra su marido en un tiempo al pecar con los amalecitas, y por eso el Señor Dios la entregó en manos de los pecadores.

\par 4 Y cuando ya era de día, Beel salió y encontró muerta a su concubina. Y él la cargó sobre la mula y se apresuró a salir y llegó a Gades. Y tomó su cuerpo y lo dividió y lo envió en todas partes (o por porciones) a las doce tribus, diciendo: Estas cosas me fueron hechas en la ciudad de Noba, porque los moradores de allí se levantaron contra mí para matarme y Tomó a mi concubina, me hizo callar y la mató. Y si esto os agrada, guardad silencio y que el Señor juzgue; pero si queréis vengarlo, el Señor os ayudará.

\par 5 Y todos los hombres, hasta las doce tribus, estaban confundidos. Y se reunieron en Silo y dijeron cada uno a su prójimo: ¿Se ha hecho tal iniquidad en Israel?

\par 6 Y el Señor dijo al adversario: ¿Ves cómo se perturba este pueblo insensato? En la hora en que debían haber muerto, incluso cuando Michas hizo astucia para engañar al pueblo con estos, es decir, con la paloma y el águila y con la imagen de hombres y becerros y de un león y de un dragón, entonces fueron no movido. Y por tanto, como no fueron provocados a ira, sean ahora vanos sus consejos y se conmueva su corazón, para que los que permiten el mal sean consumidos así como los pecadores.

\chapter{46}

\par 1 Y cuando se hizo de día, los hijos de Israel se conmovieron mucho y dijeron: Subamos y investiguemos el pecado cometido, para que la iniquidad sea quitada de nosotros. Y ellos hablaron así, y dijeron: Preguntemos primero al Señor y sepamos si entregará a nuestros hermanos en nuestras manos. Y si no, dejémoslo de lado. Y Finees les dijo: Ofrezcamos la Demostración y la Verdad. Y el Señor les respondió y dijo: Subid, que yo los entregaré en vuestras manos. Pero él los engañó para poder cumplir su palabra.

\par 2 Y subieron a la batalla y llegaron a la ciudad de Benjamín y enviaron mensajeros diciendo: Envíanos a los hombres que han hecho esta maldad y te perdonaremos, pero pagaremos a cada uno por su maldad. Y el pueblo de Benjamín endureció su corazón y dijo al pueblo de Israel: ¿Por qué debemos entregaros a nuestros hermanos? Si no los perdonáis, incluso lucharemos contra vosotros. Y el pueblo de Benjamín salió contra los hijos de Israel y los persiguió, y los hijos de Israel cayeron delante de ellos y derrotaron a 45.000 hombres.

\par 3 Y el corazón del pueblo se entristeció mucho, y vinieron llorando y lamentándose a Silo y dijeron: He aquí, el Señor nos ha entregado delante de los habitantes de Noba. Consultemos ahora al Señor quién de nosotros ha pecado. Y consultaron al Señor y él les dijo: Si queréis, subid y peleéis, y serán entregados en vuestras manos; y entonces se os dirá por qué os postrasteis ante ellos. Y al segundo día se fueron a pelear contra ellos. Y los hijos de Benjamín salieron y persiguieron a Israel y derrotaron a 46.000 hombres.

\par 4 Y el corazón del pueblo se derritió por completo y dijeron: ¿Ha querido Dios engañar a su pueblo? ¿O ha ordenado que, a causa del mal cometido, caigan tanto los inocentes como los que hacen el mal? Y hablando así, se postraron delante del arca del pacto de Jehová, y rasgaron sus vestidos y pusieron ceniza sobre sus cabezas, ellos y Finees hijo del sacerdote Eleazar, el cual oró y dijo: ¿Cuál es este engaño con que has cometido? ¿Nos engañaste, oh Señor? Si te parece justo lo que han hecho los hijos de Benjamín, ¿por qué no nos lo dijiste para que lo consideráramos? Pero si no te agradó, ¿por qué nos permitiste caer ante ellos?



\chapter{47}

\par 1 Y Finees añadió y dijo: Oh Dios de nuestros padres, escucha mi voz y di hoy a tu siervo si te parece bien, o si tal vez el pueblo ha pecado y tú quieres destruir su mal, para que tú no lo hagas. corrige también entre nosotros a los que pecaron contra ti. Porque recuerdo en mi juventud cuando Jambri pecó en los días de Moisés tu siervo, y en verdad entré, y fui celoso en mi alma, y ​​los levanté a ambos sobre mi espada, y el remanente se habría levantado contra mí para poner fin. y me mataste, y enviaste tu ángel, e hiciste de ellos veinticuatro mil hombres, y me libraste de sus manos.

\par 2 Y ahora enviaste a las once tribus y las trajiste aquí, diciendo: Id y heridlos. Y cuando se fueron, fueron entregados. Y ahora dicen que las declaraciones de tu verdad están ante ti. Ahora pues, Señor Dios de nuestros padres, no lo ocultes a tu siervo, sino dinos por qué has hecho esta iniquidad contra nosotros.

\par 3 Y cuando el Señor vio que Finees oraba intensamente ante él, le dijo: Por mí mismo he jurado, dice el Señor, que si no lo hubiera jurado, no me habría acordado de ti en lo que has hablado, ni tampoco Te he respondido hoy. Y ahora di al pueblo: Levántate y oye la palabra del Señor,

\par 4 Así dice el Señor: Había un león poderoso en medio del bosque, y todas las bestias le encomendaron el bosque para que él lo guardara con su poder, para que no vinieran otras bestias y lo devastaran. Y mientras el león lo guardaba, vinieron fieras del campo de otro bosque y devoraron a todas las crías de las bestias y devastaron el fruto de su cuerpo, y el león lo vio y calló. Ahora las bestias estaban en paz, porque habían confiado el bosque al león y no se daban cuenta de que sus crías habían sido destruidas.

\par 5 Y al cabo de un tiempo se levantó una bestia muy pequeña de las que habían entregado el bosque al león y devoró al más pequeño de los cachorros de otra bestia muy malvada. Y he aquí, el león gritó y despertó a todas las bestias del bosque, y pelearon entre sí, y cada uno peleó contra su prójimo.

\par 6 Y cuando muchos animales habían sido destruidos, otro cachorro de otro bosque parecido a él lo vio y dijo: ¿No has destruido otros tantos animales? ¿Qué iniquidad es ésta, que en el principio, cuando muchas bestias y sus crías eran destruidas injustamente por otras malas bestias, y cuando todas las bestias deberían haber sido movidas a vengarse, viendo que el fruto de sus cuerpos era despojado sin propósito, entonces tú guardaste silencio y no hablaste, pero ahora un cachorro de una bestia malvada ha perecido, y has agitado todo el bosque para que todas las bestias se devoren unas a otras sin causa, y el bosque se vea disminuido. Ahora, pues, primero debes ser destruido tú, y así se establecerá el remanente. Y cuando las crías de las bestias oyeron esto, mataron primero al león, y pusieron sobre ellos un cachorro en su lugar, y así las demás bestias quedaron sujetas juntas.

\par 7 Se levantó Mijás y os enriqueció con lo que había cometido, tanto él como su madre. Y hubo cosas malas y perversas, que nadie había ideado antes de ellos, pero con su astucia hizo imágenes talladas, que no habían sido hechas hasta ese día, y nadie se irritó, sino que todos fuisteis extraviados, y visteis el fruto. de tu cuerpo despojado, y callaste como ese león malvado.

\par 8 Y cuando visteis cómo moría la concubina de este hombre que padecía mal, os conmovisteis todos y vinisteis a mí diciendo: ¿Entregarás a los hijos de Benjamín en nuestras manos? Por eso os engañé y dije: Os los entregaré. Y ahora he destruido a los que entonces callaban, y así me vengaré de todos los que han hecho maldad contra mí. Pero vosotros subid ahora, que yo os los entregaré.

\par 9 Y todo el pueblo se levantó unánimemente y se fueron. Y los hijos de Benjamín salieron contra ellos, pensando que los vencerían como antes. Y no sabían que su maldad se había cumplido en ellos. Y cuando habían avanzado como al principio y los perseguían, el pueblo huyó de delante de ellos para darles lugar, y entonces se levantaron de sus emboscadas, y los hijos de Benjamín estaban en medio de ellos.

\par 10 Entonces los que huían se volvieron, y los hombres de la ciudad de Noba fueron asesinados, tanto hombres como mujeres, es decir, 85.000 hombres, y los hijos de Israel quemaron la ciudad, tomaron el botín y destruyeron todo a filo. de la espada. Y de los hijos de Benjamín no quedó ningún hombre, salvo sólo 600 hombres que huyeron y no fueron encontrados en la batalla. Y todo el pueblo volvió a Silo, y Finees hijo del sacerdote Eleazar con ellos.

\par 11 Estos son los que quedaron del linaje de Benjamín, los príncipes de la tribu, de diez familias cuyos nombres son estos: de la primera familia: Ezbaile, Zieb, Balac, Reindebac, Belloch; y de la segunda familia: Nethac, Zenip, Phenoch, Demec, Geresaraz; y de la tercera familia: Jerimuth, Veloth, Amibel, Genuth, Nephut, Phienna; y de la cuarta ciudad: Gemuf, Eliel, Gemot, Soleph, Raphaph y Doffo; y de la quinta familia: Anuel, Code, Fretan, Remmon, Peccan, Nabath; y de la sexta familia: Rephaz, Sephet, Araphaz, Metach, Adhoc, Balinoc; y de la séptima familia: Benin, Mephiz, Araph, Ruimel, Belon, Iaal, Abac; y (de) la (8ª, 9ª y) 10ª familia: Enophlasa, Melec, Meturia, Meac; y los demás príncipes de la tribu que quedaron, en número de sesenta.

\par 12 Y en aquel tiempo el Señor correspondió a Micas y a su madre todo lo que había dicho. Y Michas se derritió en el fuego y su madre desfallecía, tal como el Señor había hablado acerca de ellos.

\chapter{48}

\par 1 En aquel tiempo también Finees se acostó para morir, y el Señor le dijo: He aquí, has sobrepasado los 120 años que estaban prescritos para todos los hombres. Y ahora levántate y vete de aquí y habita en el monte Danaben y permanece allí muchos años, y yo le daré órdenes a mi águila y ella te alimentará allí, y no descenderás más a los hombres hasta que llegue el tiempo y seas probado en el tiempo. Y entonces cerrarás el cielo, y a tu palabra se abrirá. Y después serás alzado al lugar donde fueron alzados los que fueron antes de ti, y allí estarás hasta que me acuerde del mundo. Y entonces os traeré y probaréis lo que es la muerte.

\par 2 Y Finees subió e hizo todo lo que el Señor le ordenó. Y en los días que lo nombró sacerdote, lo ungió en Silo.

\par 3 Y en aquel tiempo, cuando él subió, aconteció que los hijos de Israel, mientras celebraban la pascua, ordenaron a los hijos de Benjamín, diciendo: Subid y tomad mujeres por la fuerza... porque no podemos daros. vosotros, hijas nuestras, porque juramos en el tiempo de nuestra ira; y no es posible que una tribu perezca de Israel. Y los hijos de Benjamín subieron y tomaron mujeres y se edificaron Gabaón, y comenzaron a habitar allí.

\par 4 Mientras tanto los hijos de Israel estaban tranquilos, en aquellos días no tenían príncipe y cada uno hacía lo que bien le parecía.

\par 5 Estos son los mandamientos, los juicios, los testimonios y las manifestaciones que hubo en los días de los jueces de Israel, antes de que reinara un rey sobre ellos.

\chapter{49}

\par 1 En aquel tiempo los hijos de Israel comenzaron a consultar al Señor, y dijeron: Vamos; todos echamos suertes, para que veamos quién puede gobernarnos como Cenez, porque tal vez encontraremos un hombre que pueda librarnos de nuestras aflicciones, porque no es conveniente que el pueblo se quede sin príncipe.

\par 2 Y echaron suertes y no encontraron a nadie; y el pueblo se entristeció mucho y decía: El pueblo no es digno de ser oído por el Señor, porque no nos ha respondido. Ahora pues, echemos suertes también por tribus, si acaso Dios será apaciguado por una multitud, porque sabemos que se reconciliará con los que son dignos de él. Y echaron suertes por tribus, y sobre ninguna tribu salió la suerte. E Israel dijo: Escojamos a uno de nosotros, porque estamos en apuros, porque percibimos que Dios aborrece a su pueblo, y que su alma está disgustada con nosotros.

\par 3 Y uno respondió y dijo al pueblo, que se llamaba Nethes: No es él quien nos odia, sino que nosotros mismos nos hemos hecho aborrecer, para que Dios nos abandone. Y por tanto, aunque muramos, no lo abandonemos, sino huyamos a él en busca de refugio; porque hemos andado en nuestros malos caminos y no hemos conocido al que nos hizo, y por tanto nuestras intenciones serán en vano. Porque sé que Dios no nos desechará para siempre, ni aborrecerá a su pueblo por todas las generaciones; por tanto, ahora sed fuertes y oremos una vez más y echemos suertes por las ciudades, porque aunque nuestros pecados se aumenten, aún así los suyos serán mayores. la paciencia no falla.

\par 4 Echaron suertes por ciudades y la suerte recayó sobre Armathem. Y el pueblo dijo: ¿Armathem es considerado justo entre todas las ciudades de Israel, por haberla escogido así entre todas las ciudades? Y cada uno dijo a su prójimo: En esa misma ciudad que ha salido por suerte, echemos suertes por hombres, y veamos a quién ha elegido el Señor de ella.

\par 5 Y echaron suertes por hombres, y no tomó a nadie excepto a Elchana, porque sobre él saltó la suerte, y el pueblo lo tomó y dijo: Venid y sé gobernante sobre nosotros. Y Elchana dijo al pueblo: Yo no puedo ser príncipe sobre este pueblo, ni puedo juzgar quién puede ser príncipe sobre vosotros. Pero si mis pecados me han descubierto y la suerte cae sobre mí, me mataré para que no me contamineis; porque es justo que muera sólo por mis propios pecados y no tenga que soportar el peso del pueblo.

\par 6 Y cuando el pueblo vio que no era la voluntad de Elchana tomar el liderazgo sobre ellos, oraron de nuevo al Señor diciendo: Oh Señor Dios de Israel, ¿por qué has abandonado a tu pueblo en la victoria del enemigo y ¿Descuidaste tu heredad en el tiempo de angustia? He aquí, incluso el que fue tomado por suerte no ha cumplido tu mandamiento; pero sólo sucedió esto: que la suerte se echó sobre él y creímos que teníamos un príncipe. Y he aquí, él también compite contra la suerte. ¿A quién necesitaremos todavía, o hacia quién huiremos, y dónde está el lugar de nuestro descanso? Porque si son verdaderas las ordenanzas que hiciste con nuestros padres, diciendo: Ensancharé tu descendencia, y ellos sabrán de esto, entonces sería mejor que nos dijeras: Cortaré tu descendencia, que que tú tengas sin tener en cuenta nuestra raíz.

\par 7 Y Dios les dijo: Si en verdad os he pagado según vuestras malas obras, no debería escuchar a vuestro pueblo; pero ¿qué haré, porque mi nombre viene a ser invocado sobre vosotros? Y ahora sabed que Elchana sobre quien ha caído la suerte no puede gobernar sobre vosotros, sino que es más bien su hijo el que nacerá de él; él será príncipe sobre vosotros y profetizará; y desde ahora en adelante no os faltará príncipe por muchos años.

\par 8 Y el pueblo dijo: He aquí, Señor, Elchana tiene diez hijos, y ¿quién de ellos será príncipe o profetizará? Y dijo Dios: Ninguno de los hijos de Fenena puede ser príncipe sobre el pueblo, pero el que nace de la mujer estéril que yo le he dado por esposa, será profeta delante de mí, y yo lo amaré hasta como amé a Isaac, y su nombre estará delante de mí para siempre. Y el pueblo dijo: He aquí ahora, puede ser que Dios se haya acordado de nosotros, para librarnos de mano de los que nos aborrecen. Y aquel día ofrecieron ofrendas de paz y festejaron a sus órdenes.

\chapter{50}

\par 1 Ahora bien, [mientras que] Elchana tenía dos esposas, el nombre de una era Ana y el nombre de la otra Fenena. Y como Fenena tenía hijos y Ana no tenía ninguno, Fenena la reprochó, diciendo: ¿De qué te aprovecha que Elchana tu marido te ame? pero tú eres un árbol seco. Sé además que me amará, porque se deleita en ver a mis hijos en pie a su alrededor como la plantación de un olivar.

\par 2 Y aconteció que cuando ella la reprochaba todos los días, y Ana estaba muy afligida de corazón, y temía a Dios desde su juventud, aconteció que cuando se acercaba el buen día de la Pascua, su marido subió. hacer sacrificio, que Fenena injuriaba a Ana diciendo: Una mujer no es amada aunque su marido la ame a ella o a su belleza. Por tanto, no se jacte Ana de su belleza, pero el que se jacta, que se jacte cuando vea su descendencia delante de su rostro; y cuando no sea así entre las mujeres, incluso el fruto de su vientre, entonces el amor quedará en vano. ¿De qué le sirvió a Raquel que Jacob la amara? Si no le hubieran dado el fruto de su vientre, ¿seguramente su amor habría sido en vano? Y cuando Anna escuchó eso, su alma se derritió dentro de ella y sus ojos se llenaron de lágrimas.

\par 3 Y su marido la vio y dijo: ¿Por qué estás triste y no comes, y por qué está abatido tu corazón dentro de ti? ¿No es tu conducta mejor que la de los diez hijos de Fenena? Y Ana le escuchó, se levantó después de haber comido y vino a Silo, a la casa del Señor, donde moraba el sacerdote Heli, a quien Finees, hijo del sacerdote Eleazar, había presentado como le había sido ordenado.

\par 4 Y Ana oró y dijo: ¿No has examinado, oh Señor, el corazón de todas las generaciones antes de que formaras el mundo? Pero ¿qué es el vientre que nace abierto, o el que está cerrado, que muere, si no lo quieres? Y ahora suba mi oración ante ti hoy, para que no baje de aquí vacío, porque tú conoces mi corazón, cómo he andado delante de ti desde los días de mi juventud.

\par 5 Y Ana no oraba en voz alta como lo hacen todos los hombres, porque en ese momento pensó: No sea que no sea digna de ser escuchada, y Fenena me envidiará aún más y me reprochará como lo hace cada día. dice: ¿Dónde está tu Dios en quien confías? Y sé que no la que tiene muchos hijos se enriquece, ni la que carece de ellos es pobre, sino la que abunda en la voluntad de Dios, se enriquece. Porque los que saben por qué he orado, si ven que no soy escuchado en mi oración, blasfemarán. Y no sólo tendré testimonio en mi propia alma, porque también mis lágrimas son esclavas de mis oraciones.

\par 6 Y mientras ella oraba, el sacerdote Heli, viendo que estaba afligida de ánimo y que se comportaba como una ebria, le dijo: Ve, quita tu vino de ti. Y ella dijo: ¿Es tan escuchada mi oración que me llaman ebria? En verdad estoy ebrio de dolor y he bebido la copa de mi llanto.

\par 7 Y el sacerdote Helí le dijo: Cuéntame tu afrenta. Y ella le dijo: Soy esposa de Elchana, y porque Dios ciertamente ha cerrado mi vientre, por eso oré delante de él para que no pudiera partir de este mundo hacia él sin fruto, ni morir sin dejar mi propia imagen. Y el sacerdote Heli le dijo: Ve, porque yo sé por qué has orado, y. tu oración es escuchada.

\par 8 Pero el sacerdote Helí no quiso decirle que de ella estaba previsto que naciera un profeta, porque había oído cuando el Señor hablaba de él. Y Ana llegó a su casa y se consoló de su dolor, pero a nadie contó aquello por lo que había orado.

\chapter{51}

\par 1 Y en aquellos días concibió y dio a luz un hijo, y llamó su nombre Samuel, que significa Poderoso, tal como Dios llamó su nombre cuando profetizó de él. Y Ana se sentó y amamantó al niño hasta que tuvo dos años, y cuando lo hubo destetado, subió con él llevando regalos en sus manos, y el niño era muy hermoso y el Señor estaba con él.

\par 2 Y Ana puso al niño delante de Elí y le dijo: Este es el deseo que deseaba, y esta es la petición que pedí. Y Heli le dijo: No sólo tú lo buscaste, sino que también el pueblo oró por esto. No es sólo tu petición, sino que fue prometida en tiempo pasado a las tribus; y en este niño es justificado tu vientre, para que pongas profecía delante del pueblo, y pongas la leche de tus pechos como fuente para las doce tribus.

\par 3 Y cuando Ana oyó esto, oró y dijo: Venid a mi voz, pueblos todos, y escuchad mis palabras, reinos todos, porque mi boca está abierta para que pueda hablar, y mis labios están ordenados. para que pueda cantar alabanzas al Señor. Gotad, oh pechos míos, y dad vuestros testimonios, porque a vosotros os está destinado amamantar. Porque será levantado el que por ti es amamantado, y por sus palabras los pueblos serán iluminados, y él mostrará a las naciones sus límites, y su cuerno será en gran manera exaltado.

\par 4 Por eso hablaré abiertamente, porque de mí surgirá la orden del Señor y todos los hombres encontrarán la verdad. No os apresuréis a hablar con soberbia, ni a proferir palabras altivas de vuestra boca, sino deléitaos en gloriaros cuando salga la luz de la cual nacerá la sabiduría, para que no sean llamados ricos los que más poseen, ni los que porque la estéril ha sido saciada, y la que se multiplicó en hijos quedó vacía;

\par 5 Porque el Señor mata con juicio y vivifica con misericordia, porque los impíos están en este mundo; por eso, él da vida a los justos cuando quiere, pero a los impíos los encierra en las tinieblas. Pero para los justos él preserva su luz, y cuando los impíos mueran, entonces perecerán, y cuando los justos duerman, entonces serán libertados. Y así perdurará todo juicio hasta que se manifieste quién lo posee.

\par 6 Habla, habla, oh Ana, y no calles; canta alabanzas, oh hija de Batuel, a causa de las maravillas que Dios ha hecho contigo. ¿Quién es Ana para que de ella salga un profeta? ¿O quién es la hija de Batuel, para que haga nacer una luz contra los pueblos? Levántate tú también, Elchana, y ciñe tus lomos. Cantad alabanzas a las señales de Jehová: Porque de tu hijo Asaf profetizó en el desierto, diciendo: Moisés y Aarón entre sus sacerdotes, y Samuel entre ellos. He aquí la palabra se cumple y la profecía se cumple. Y estas cosas permanecen así, hasta que den un cuerno a su ungido, y el poder se adhiera al trono de su rey. Sin embargo, que mi hijo esté aquí y ministre, hasta que surja una luz para este pueblo.

\par 7 Y partieron de allí y se pusieron en camino con alegría, regocijándose y regocijándose de corazón por toda la gloria que Dios había obrado con ellos. Pero el pueblo descendió unánimemente a Silo con panderos y danzas, con flautas y arpas, y vino al sacerdote Heli, y le ofreció a Samuel, al cual pusieron delante de la faz del Señor y lo ungieron, y dijeron: Que el profeta viva entre el pueblo, y sea por mucho tiempo luz para esta nación.

\chapter{52}

\par 1 Pero Samuel era muy pequeño y no sabía nada de todas estas cosas. Y mientras él servía delante del Señor, los dos hijos de Heli, que no andaban en los caminos de sus padres, comenzaron a hacer maldades con el pueblo y a multiplicar sus iniquidades. Y habitaron junto a la casa de Betac, y cuando el pueblo se reunió para sacrificar, Ofni y Finees vinieron y provocaron a ira al pueblo, apoderándose de las oblaciones antes de que las cosas santas fueran ofrecidas al Señor.

\par 2 Y esto no agradó al Señor, ni al pueblo ni a su padre. Y su padre les habló así: ¿Qué es esta noticia que oigo de vosotros? ¿No sabéis que he recibido el lugar que Finees me encomendó? Y si desperdiciamos lo que hemos recibido, ¿qué diremos si el que lo cometió nos lo exige nuevamente y nos aflige por lo que nos ha confiado? Ahora pues, enderezad vuestros caminos y andad por buenas sendas, y vuestras obras perdurarán. Pero si me contradicen y no se abstienen de sus malas intenciones, se destruirán a sí mismos, y el sacerdocio será en vano, y lo que fue santificado quedará en nada. Y entonces dirán: En vano brotó la vara de Aarón, y la flor que de ella nació se desvaneció.

\par 3 Por lo tanto, hijos míos, mientras todavía podáis, corregid que habéis hecho mal, y los hombres contra quienes habéis pecado orarán por vosotros. Pero si no queréis, sino que perseveráis en vuestras iniquidades, seré inocente, y no sólo me entristeceré por no (o y ahora no borraré estos grandes males en vosotros, no sea que) me entere del día de vuestra muerte antes. Yo muero, pero también si esto sucede (o incluso si esto no sucede), seré libre de culpa: y aunque yo sea afligido, vosotros pereceréis.

\par 4 Pero sus hijos no le obedecieron, porque el Señor les había dado sentencia de muerte por haber pecado; porque cuando su padre les dijo: Arrepiéntanse de su mal camino, ellos dijeron: Cuando envejezcamos , entonces nos arrepentiremos. Y por esta causa no les fue concedido que se arrepintieran cuando fueron reprendidos por su padre, porque siempre habían sido rebeldes y habían obrado muy injustamente al despojar a Israel. Pero el Señor se enojó contra Elí.

\chapter{53}

\par 1 Pero Samuel estaba ministrando delante del Señor y aún no sabía cuáles eran las palabras del Señor; porque aún no había oído las palabras del Señor, porque tenía ocho años.

\par 2 Pero cuando Dios se acordó de Israel, reveló sus palabras a Samuel, y Samuel durmió en el templo del Señor. Y aconteció que cuando Dios lo llamó, él consideró primero, y dijo: He aquí ahora, Samuel es joven para ser (o aunque sea) amado delante de mis ojos; sin embargo, por cuanto aún no ha oído la voz de Jehová, ni está confirmado a la voz del Altísimo, sin embargo es como Moisés mi siervo; pero a Moisés hablé cuando él tenía 80 años, pero Samuel tiene 8. años. Y Moisés vio el fuego primero y su corazón tuvo miedo. Y si Samuel ve ahora el fuego, ¿cómo lo soportará? Por tanto, ahora vendrá a él una voz como de hombre, y no como de Dios. Y cuando él entienda, entonces le hablaré como a Dios.

\par 3 Y a medianoche lo llamó una voz del cielo; y Samuel se despertó y oyó como si fuera la voz de Heli, y corrió hacia él y le habló diciendo: ¿Por qué me has despertado, padre? Porque tuve miedo porque nunca me llamaste en la noche. Y dijo Heli: ¡Ay de mí! ¿Será que un espíritu inmundo ha engañado a mi hijo Samuel? Y él le dijo: Ve y duerme, porque no te llamé. Sin embargo, dime, si te acuerdas, cuántas veces lloró el que te llamó. Y él dijo: Dos veces. Y Heli le dijo: Di ahora, ¿de quién escuchaste la voz, hijo mío? Y él dijo: De ti, por tanto, corrí hacia ti.

\par 4 Y Elí dijo: En ti veo la señal que los hombres tendrán desde hoy en adelante para siempre: si uno llama a otro dos veces en la noche o al mediodía, sabrán que es un espíritu maligno. Pero si llama por tercera vez, sabrán que es un ángel. Y Samuel se fue y durmió.

\par 5 Y oyó por segunda vez una voz del cielo, y se levantó y corrió hacia Heli y le dijo: ¿Por qué me llamó, si oí la voz de Elchana mi padre? Entonces comprendió Helí que Dios sí había comenzado a llamarlo. Y Elí dijo: En esas dos voces con que Dios te llamó, se comparó a tu padre y a tu señor, pero ahora la tercera vez hablará como Dios.

\par 6 Y él le dijo: Con el oído derecho escucha y con el izquierdo abstente. Porque el sacerdote Finees nos mandó, diciendo: El oído derecho oye al Señor de noche, y el oído izquierdo al ángel. Por tanto, si escuchas con tu oído derecho, di así: Habla lo que quieras, porque yo te escucho, porque tú me formaste; pero si oyes con el oído izquierdo, ven y dímelo. Y Samuel se fue y durmió como Heli le había mandado.

\par 7 Y el Señor añadió y habló por tercera vez, y el oído derecho de Samuel se llenó de la voz. Y cuando entendió que las palabras de su padre habían llegado hasta él, Samuel se volvió hacia el otro lado y dijo: Si puedo, habla, porque tú me has formado (o sabes bien acerca de mí).

\par 8 Y Dios le dijo: En verdad yo iluminé a la casa de Israel en Egipto y elegí en aquel tiempo a Moisés mi siervo como profeta, y por medio de él hice maravillas para mi pueblo, y los vengué de mis enemigos como lo hice yo. Quiso, y llevé a mi pueblo al desierto y los iluminé mientras contemplaban.

\par 9 Y cuando una tribu se levantó contra otra tribu, diciendo: ¿Por qué son santos sólo los sacerdotes? No quise destruirlos, sino que les dije: Dad cada uno su vara, y a aquel cuya vara florezca, yo lo he elegido para el sacerdocio. Y cuando todos hubieron dado sus varas como yo ordené, entonces ordené a la tierra del tabernáculo que la vara de Aarón floreciera, para que su linaje se manifestara por muchos días. Y ahora los que florecían han aborrecido mis cosas santas.

\par 10 Por tanto, he aquí, vendrán días en que cortaré (literalmente, detendré) la flor que brotó en aquel tiempo, y saldré contra ellos porque transgreden la palabra que hablé a mi siervo Moisés. , diciendo: Si encuentras un nido, no tomarás a la madre con las crías, por eso les sucederá que las madres morirán con los hijos, y los padres perecerán con los hijos.

\par 11 Cuando Samuel oyó estas palabras, su corazón se derritió y dijo: ¿Ha ocurrido así en mi juventud que profetizara para perdición de aquel que me crió? ¿Y cómo entonces me fue concedido a petición de mi madre? ¿Y quién es el que me crió? ¿Cómo me ha encargado que lleve malas nuevas?

\par 12 Samuel se levantó por la mañana y no quiso contárselo a Helí. Y Heli le dijo: Oye ahora, hijo mío. He aquí, antes de que nacieras, Dios prometió a Israel que te enviaría a ellos para profetizar. Y ahora, cuando tu madre vino acá y oró, porque no sabía lo que había sucedido, yo le dije: Ve, porque lo que nacerá de ti será un hijo para mí. Así hablé a tu madre, y así ha dirigido el Señor tu camino. Y aunque castigues a tu padre que amamanta, vive el Señor, no me ocultes lo que has oído.

\par 13 Entonces Samuel tuvo miedo y le contó todas las palabras que había oído. Y él dijo: ¿Puede la cosa formada responder al que la formó? Así tampoco puedo responder cuando me quitará lo que me ha dado, el dador fiel, el santo que ha profetizado, porque estoy sujeto a su poder.



\chapter{54}

\par 1 En aquellos días los filisteos reunieron su campamento para luchar contra Israel, y los hijos de Israel salieron a luchar contra ellos. Y cuando el pueblo de Israel fue puesto en fuga en la primera batalla, dijeron: Hagamos subir el arca del pacto del Señor, tal vez ella pelee con nosotros, porque en ella están los testimonios del Señor que él ordenado a nuestros padres en Oreb.

\par 2 Y mientras el arca subía con ellos, cuando llegó al campamento, el Señor tronó y dijo: Este tiempo será semejante a lo que hubo en el desierto, cuando tomaron el arca sin mi orden, y la destrucción acontecerlos. Así también en este tiempo caerá el pueblo, y el arca será tomada, para que yo castigue a los adversarios de mi pueblo a causa del arca, y reprenda a mi pueblo por haber pecado.

\par 3 Cuando el arca llegó a la batalla, los filisteos salieron al encuentro de los hijos de Israel y los derrotaron. Y estaba allí un tal Golia, filisteo, que llegó hasta el arca, y Ofni y Finees hijos de Heli, y Saúl hijo de Cis, sostenían el arca. Y Golia la tomó con su mano izquierda y mató a Ofni y a Finees.

\par 4 Pero Saúl, como era de pies ligeros, huyó de delante de él; y rasgó sus vestidos, se puso ceniza en la cabeza y vino al sacerdote Heli. Y Heli le dijo: Dime ¿qué ha sucedido en el campamento? Y Saúl le dijo: ¿Por qué me preguntas estas cosas? porque el pueblo está vencido, y Dios ha desamparado a Israel. Sí, y también los sacerdotes son muertos a espada, y el arca es entregada a los filisteos.

\par 5 Y cuando Heli se enteró de que se habían llevado el arca, dijo: He aquí, Samuel profetizó de mí y de mis hijos que moriríamos juntos, pero no me nombró el arca. Y ahora los testimonios son entregados al enemigo, ¿y qué más puedo decir? He aquí, Israel ha perecido de la verdad, porque los juicios le han sido quitados. Y como Heli estaba completamente desesperado, se cayó de su asiento. Y murieron en un día sus hijos Heli, Ofni y Finees.

\par 6 Y la esposa del hijo de Heli estaba sentada y estaba de parto; y al oír estas cosas, todas sus entrañas se derritieron. Y la partera le dijo: Ten ánimo, no desmayes tu alma, que te ha nacido un hijo. Y ella le dijo: He aquí que ahora nace un alma y morimos nosotros cuatro, es decir, mi padre y sus dos hijos y su nuera. Y llamó su nombre: ¿Dónde está la gloria? diciendo: La gloria de Dios ha perecido en Israel porque el arca del Señor fue llevada cautiva. Y cuando hubo dicho esto, entregó el espíritu.

\chapter{55}

\par 1 Pero Samuel no sabía nada de todo esto, porque tres días antes de la batalla Dios lo despidió, diciéndole: Ve y mira el lugar de Arimata, allí estará tu morada. Y cuando Samuel oyó lo que había acontecido a Israel, vino y oró al Señor, diciendo: He aquí, ahora en vano se me niega el entendimiento para poder ver la destrucción de mi pueblo. Y ahora temo que mis días envejezcan en el mal y mis años terminen en tristeza, porque si el arca del Señor no está conmigo, ¿por qué habría de vivir todavía?

\par 2 Y el Señor le dijo: No te entristezcas, Samuel, porque se lleven el arca. La traeré de nuevo, y a los que la tomaron los derribaré, y vengaré a mi pueblo de sus enemigos. Y Samuel dijo: He aquí, aunque los vengas a tiempo, según tu paciencia, ¿qué haremos nosotros los que ahora morimos? Y Dios le dijo: Antes de que mueras verás el fin que traeré sobre mis enemigos, en el cual los filisteos perecerán y serán asesinados por escorpiones y por toda clase de reptiles pestilentes.

\par 3 Y cuando los filisteos habían colocado el arca del Señor que había sido llevada en el templo de Dagón su dios, y fueron a preguntarle a Dagón acerca de su salida, lo encontraron caído sobre su rostro, con las manos y los pies tendido. delante del arca. Y salieron la primera mañana, después de crucificar a sus sacerdotes. Y el segundo día vinieron y hallaron como el día anterior, y la destrucción se multiplicó en gran manera entre ellos.

\par 4 Entonces los filisteos se reunieron en Acarón y dijeron cada uno a su prójimo: He aquí ahora vemos que la destrucción ha aumentado entre nosotros y el fruto de nuestra carne perece, porque los reptiles que son enviados sobre nosotros destruyen. las que están encintas y las que maman y también las que dan de mamar. Y dijeron: Veamos por qué la mano del Señor es fuerte contra nosotros. ¿Es por el bien del arca? porque cada día nuestro dios es encontrado caído sobre su rostro ante el arca, y hemos matado a nuestros sacerdotes sin ningún propósito una y otra vez.

\par 5 Y los sabios de los filisteos dijeron: He aquí, ¿podemos saber ahora si el Señor ha enviado sobre nosotros la destrucción por causa de su arca, o si una aflicción casual nos ha sobrevenido por un tiempo?

\par 6 Y ahora, mientras que todas las que están encintas y amamantan mueren, y las que amamantan quedan sin hijos, y las que son amamantadas perecen, nosotros también tomaremos vacas que amamantan y las unciremos a un carro nuevo, y puso el arca sobre ella, y encerró a las crías de las vacas. Y sucederá que, si las vacas salen y no vuelven a sus crías, sabremos que hemos sufrido estas cosas por causa del arca; pero si se niegan a ir, añorando a sus crías, sabremos que el tiempo de nuestra caída ha llegado sobre nosotros.

\par 7 Y algunos de los sabios y adivinos respondieron: No probéis sólo esto, sino que coloquemos las vacas al principio de los tres caminos que rodean Accaron. Porque el camino del medio lleva a Acarón, el camino de la derecha a Judea, y el camino de la izquierda a Samaria. Y encaminad las vacas que llevan el arca por el camino intermedio. Y si por el camino de la derecha van directamente a Judea, sabremos que en verdad el Dios de los judíos nos ha asolado; pero si siguen esos otros caminos, sabremos que nos ha sobrevenido un tiempo malo (lit. poderoso), porque ahora hemos negado a nuestros dioses.

\par 8 Y los filisteos tomaron vacas lecheras, las uncieron a un carro nuevo, colocaron encima el arca y las colocaron al principio de los tres caminos, y encerraron a sus crías en casa. Y las vacas, aunque mugieron y añoraron a sus crías, siguieron adelante por el camino derecho que lleva a Judea. Y entonces supieron que por causa del arca habían sido arrasados.

\par 9 Entonces se reunieron todos los filisteos y trajeron de nuevo el arca a Silo con panderos, flautas y danzas. Y a causa de los repugnantes reptantes que los asolaban, hicieron asientos de oro y santificaron el arca.

\par 10 Y en la plaga de los filisteos, el número de los que murieron encinta fue de 75.000, y de los que mamaron 65.000, y de los que amamantaron 55.000, y de los hombres 25.000. Y la tierra reposó siete años.

\chapter{56}

\par 1 En aquel tiempo los hijos de Israel necesitaban un rey en su justo. Y se reunieron con Samuel y le dijeron: He aquí, ahora estás viejo, y tus hijos no andan en los caminos del Señor; Ahora, pues, nombra sobre nosotros un rey que juzgue entre nosotros, porque se ha cumplido la palabra que Moisés habló a nuestros padres en el desierto, diciendo: Ciertamente nombrarás sobre ti un príncipe de tus hermanos.

\par 2 Y cuando Samuel oyó hablar del reino, se entristeció mucho en su corazón y dijo: He aquí, ahora veo que ya no hay para nosotros (o todavía no) un tiempo de reino perpetuo, ni de edificación del casa de Jehová nuestro Dios, por cuanto éstos desean rey antes de tiempo. Y ahora, si el Señor lo rechaza del todo (o incluso si el Señor así lo quiere), me parece que no se puede establecer un rey.

\par 3 Y el Señor le dijo aquella noche: No te entristezcas, porque les enviaré un ataque que los arrasará, y él mismo será arrasado después. Ahora bien, el que venga a ti mañana a la hora sexta, éste reinará sobre ellos.

\par 4 Y al día siguiente, Saúl, hijo de Cis, venía del monte Efrén en busca de las asnas de su padre; y cuando llegó a Armathem, entró para pedirle a Samuel las asnas. Ahora bien, caminaba junto a Baam, y Saúl le dijo: ¿Dónde está el que ve? Porque en aquel tiempo un profeta se llamaba Vidente. Y Samuel le dijo: Yo soy el que ve. Y él dijo: ¿Puedes hablarme de las asnas de mi padre? porque están perdidos.

\par 5 Y Samuel le dijo: Relájate conmigo hoy y mañana te diré lo que viniste a preguntar. Y Samuel dijo al Señor: Dirige, oh Señor, a tu pueblo, y revélame lo que has determinado acerca de ellos. Y Saúl reposó aquel día con Samuel y se levantó por la mañana. Y Samuel le dijo: He aquí, conoce que Jehová te ha escogido para ser príncipe sobre su pueblo en este tiempo, y ha levantado tus caminos, y tu tiempo será ordenado.

\par 6 Y Saúl dijo a Samuel: ¿Quién soy yo, y cuál es la casa de mi padre, para que mi señor me hable así? Porque no entiendo lo que dices, porque soy joven. Y Samuel dijo a Saúl: ¿Quién permitirá que tu palabra se cumpla por sí sola, para que vivas muchos días? pero considera esto, que tus palabras serán semejantes a las palabras de un profeta, cuyo nombre será Hieremias.

\par 7 Y mientras Saúl se iba ese día, el pueblo vino a Samuel, diciendo: Danos un rey como nos prometiste. Y él les dijo: He aquí, el rey vendrá a vosotros dentro de tres días. Y he aquí vino Saúl. Y le sucedieron todas las señales que Samuel le había contado. ¿No están escritas estas cosas en el libro de los Reyes?

\chapter{57}

\par 1 Entonces Samuel envió y reunió a todo el pueblo, y les dijo: He aquí, vosotros y vuestro rey estáis aquí, y yo estoy entre vosotros, como el Señor me ordenó.

\par 2 Por eso os digo, delante de vuestro rey, como mi señor Moisés: el siervo de Dios, dijo a vuestros padres en el desierto, cuando la sinagoga de Core se levantó contra él: Vosotros sabéis que nada os he quitado, ni he hecho daño a ninguno de vosotros; y por cuanto algunos mintieron en aquel tiempo y dijeron: Tú tomaste, la tierra se los tragó.

\par 3 Ahora pues, vosotros, a quienes el Señor no ha castigado, responded delante del Señor y de su ungido, si por esto habéis requerido un rey, porque os he tratado mal, y el Señor será vuestro testigo. Pero si ahora se cumple la palabra del Señor, soy libre y soy la casa de mi padre.

\par 4 Y el pueblo respondió: Nosotros somos tus siervos y nuestro rey entre nosotros; porque somos indignos de ser juzgados por un profeta, por eso dijimos: Nombrad un rey sobre nosotros para que nos juzgue. Y todo el pueblo y el rey lloraron con gran lamentación, y dijeron: Viva el profeta Samuel. Y cuando el rey fue nombrado, ofrecieron sacrificios al Señor.

\par 5 Después de esto, Saúl peleó contra los filisteos un año, y la batalla prosperó mucho.

\chapter{58}

\par 1 En aquel tiempo el Señor dijo a Samuel: Ve y di a Saúl: Tú eres enviado a destruir a Amalec, para que se cumplan las palabras que habló mi siervo Moisés, diciendo: Destruiré el nombre de Amalec de la tierra. de lo cual hablé en mi ira. Y no olvides destruir cada alma de ellos como se te ha ordenado.

\par 2 Y Saúl partió y peleó contra Amalec, y salvó la vida a Agag, rey de Amalec, porque le dijo: Te mostraré los tesoros escondidos. Por eso lo perdonó, lo salvó con vida y lo llevó a Armathem.

\par 3 Y Dios dijo a Samuel: ¿Has visto cómo el rey se corrompe con el dinero en un instante y ha salvado con vida a Agag, rey de Amalec, y a su esposa? Ahora, pues, deja que Agag y su mujer se reúnan esta noche, y mañana lo matarás; pero a su mujer la guardarán hasta que dé a luz un hijo varón, y entonces ella también morirá, y el que de ella naciere será por escándalo a Saúl. Pero tú levántate mañana y mata a Agag, porque el pecado de Saúl está escrito delante de mí para siempre.

\par 4 Al día siguiente, cuando Samuel se levantó, Saúl salió a su encuentro y le dijo: El Señor ha entregado a nuestros enemigos en nuestras manos, como él había dicho. Y Samuel dijo a Saúl: ¿A quién ha agraviado Israel? porque antes de que llegara el tiempo para que un rey gobernara sobre él, te demandó por su rey, y tú, cuando fuiste enviado a hacer la voluntad del Señor, la transgrediste. Por tanto, el que por ti fue salvado con vida, ahora morirá, y aquellos tesoros escondidos de que habló no te los mostrará, y el que de él nacerá te será una ofensa. Y Samuel vino a Agag con una espada y lo mató, y volvió a su casa.

\chapter{59}

\par 1 Y el Señor le dijo: Ve y unge al que yo te diré, porque se ha cumplido el tiempo en que vendrá su reino. Y. Samuel dijo: He aquí, ¿borrarás ahora el reino de Saúl? Y él dijo: Lo borraré.

\par 2 Y Samuel fue a Betel y santificó a los ancianos, a Isaí y a sus hijos. Y vino Eliab, primogénito de Jesé. Y Samuel dijo: He aquí ahora el santo, el ungido de Jehová. Y el Señor le dijo: ¿Dónde está la visión que ha visto tu corazón? ¿No eres tú el que decía a Saúl: Yo soy el que ve? ¿Y cómo no sabes a quién debes ungir? Ahora pues, bástate esta reprensión, y busca al pastor, al más pequeño de todos, y úngelo.

\par 3 Entonces Samuel dijo a Isaí: Escucha, Isaí, envía y trae a tu hijo del rebaño, porque a él ha escogido Dios. Y Jesé envió y trajo a David, y Samuel lo ungió en medio de sus hermanos. Y el Señor estuvo con él desde aquel día en adelante.

\par 4 Entonces David comenzó a cantar este salmo, y dijo: Desde los confines de la tierra comenzaré a glorificarlo, y hasta los días eternos cantaré alabanzas. Abel al principio, cuando apacentó las ovejas, su sacrificio fue más aceptable que el de su hermano. Y su hermano tuvo envidia de él y lo mató. Pero no es así conmigo, porque Dios me guardó, y me entregó a sus ángeles y a sus vigilantes para que me guardaran, porque mis hermanos me tenían envidia, y mi padre y mi madre me despreciaron, y cuando el profeta Vinieron y no me llamaron, y cuando fue proclamado el ungido del Señor, se olvidaron de mí. Pero Dios se acercó a mí con su diestra y con su misericordia; por tanto, no dejaré de cantar alabanzas todos los días de mi vida.

\par 5 Mientras David aún hablaba, he aquí un león feroz del bosque y una osa del monte se apoderaron de los toros de David. Y David dijo: He aquí, esto me será por señal de un poderoso comienzo de mi victoria en la batalla. Yo saldré tras ellos y entregaré lo que se ha raptado y los mataré. Y David salió tras ellos, tomó piedras del bosque y los mató. Y Dios le dijo: He aquí, con piedras te he librado estas fieras delante de tus ojos. Y esto te será por señal de que en adelante matarás a piedras al adversario de mi pueblo.

\chapter{60}

\par 1 En aquel tiempo el espíritu del Señor fue quitado de Saúl, y un espíritu maligno lo oprimió (lit. lo ahogó). Y Saúl envió a buscar a David, y él tocó un salmo con su arpa esa noche. Y este es el salmo que cantó a Saúl para que el espíritu maligno se alejara de él.

\par 2 Hubo oscuridad y silencio antes de que existiera el mundo, y el silencio habló, y las tinieblas se hicieron visibles. Y entonces fue creado tu nombre, incluso en la unión de lo que estaba extendido, de lo cual lo superior se llamaba cielo y lo inferior se llamaba tierra. Y se mandó al superior que lloviese según su estación, y al inferior que produjera alimento para el hombre que se haría. Y después de eso se formó la tribu de vuestros espíritus.

\par 3 Ahora, pues, no seas perjudicial, ya que eres una segunda creación; pero si no, recuerda el infierno (literalmente, recuerda el Tártaro), en el que caminaste. ¿O no te basta oír que con lo que resuena ante ti canto a muchos? ¿O olvidas que de un eco que rebota en el abismo (o caos) nació tu creación? Pero te reprenderá ese nuevo vientre, del cual yo nací, del cual al cabo de un tiempo nacerá de mis lomos el que te sujetará.

\par Y cuando David cantó alabanzas, el espíritu perdonó a Saúl.

\chapter{61}

\par 1 Después de esto, los filisteos vinieron a pelear contra Israel. Y David regresó al desierto para alimentar a sus ovejas, y los madianitas vinieron y quisieron tomar sus ovejas, y él descendió hacia ellos y peleó contra ellos y mató a 15.000 hombres. Esta es la primera batalla que peleó David, estando en el desierto.

\par 2 Y del campamento de los filisteos salió un hombre llamado Golia, que miró a Saúl y a Israel y dijo: ¿No eres tú Saúl, el que huyó delante de mí cuando te quité el arca y maté a tus sacerdotes? Y ahora que reinas, ¿descenderás a mí como un hombre y un rey y pelearás contra nosotros? Si no, vendré a ti y haré que tú seas cautivo y que tu pueblo sirva a nuestros dioses. Y cuando Saúl e Israel oyeron esto, temieron mucho. Y el filisteo dijo: Conforme al número de los días que Israel festejó cuando recibió la ley en el desierto, cuarenta días, yo los afrentaré, y después pelearé contra ellos.

\par 3 Y aconteció que cuando se cumplieron los cuarenta días, cuando David vino a ver la batalla de sus hermanos, oyó las palabras que hablaba el filisteo, y dijo: ¿Es este el tiempo en que Dios me dijo? : ¿Entregaré al adversario de mi pueblo en tu mano con piedras?

\par 4 Y Saúl oyó estas palabras y envió, lo tomó y le dijo: ¿Qué fue lo que hablaste al pueblo? Y David dijo: No temas, oh rey, porque iré y pelearé contra los filisteos, y Dios quitará el odio y el oprobio de Israel.

\par 5 Y David salió, tomó siete piedras y escribió en ellas los nombres de sus padres, Abraham, Isaac, Jacob, Moisés y Aarón, y su propio nombre, y el nombre del Poderoso. Y Dios envió a Cervihel, el ángel que está sobre la fuerza.

\par 6 Y David fue a Golia y le dijo: Oye una palabra antes de que mueras. ¿No eran hermanas las dos mujeres de las que nacimos tú y yo? y tu madre era Orfa y mi madre era Rut. Y Orfa eligió para sí los dioses de los filisteos y fue tras ellos, pero Rut eligió para sí los caminos del Más Poderoso y caminó en ellos. Y ahora tú y tus hermanos sois nacidos de Orfa, y como tú te has levantado hoy y vienes a devastar a Israel, he aquí, yo también, que he nacido de tu parentela, he venido a vengar a mi pueblo. Porque también tus tres hermanos caerán en mis manos después de tu muerte. Y entonces diréis a vuestra madre: El que nació de tu hermana no nos ha perdonado.

\par 7 Entonces David puso una piedra en su honda y golpeó al filisteo en la frente, y corrió hacia él, desenvainó su espada y le arrancó la cabeza. Y Golia le dijo mientras aún tenía vida en él: Apresúrate y mátame y regocíjate.

\par 8 Y David le dijo: Antes de que mueras, abre tus ojos y contempla al asesino que te ha matado. Y el filisteo miró y vio al ángel y dijo: No me has matado tú solo, sino al que estaba contigo, cuyo aspecto no es como el de un hombre. Y entonces David le quitó la cabeza.

\par 9 Y el ángel del Señor levantó el rostro de David y nadie lo conoció. Y cuando Saúl vio a David, le preguntó quién era, y no había nadie que supiera quién era.

\chapter{62}

\par 1 Después de esto, Saúl tuvo envidia de David y trató de matarlo. Pero David y Jonatán, hijo de Saúl, hicieron juntos un pacto. Y cuando David vio que Saúl procuraba matarlo, huyó a Armathem; y Saúl salió tras él.

\par 2 Y el espíritu reposó en Saúl, y él profetizó, diciendo: ¿Por qué te engañas, oh Saúl, o a quién persigues en vano? El tiempo de tu reino se ha cumplido. Ve a tu lugar, porque morirás y David reinará. ¿No moriréis tú y tu hijo juntos? Y entonces aparecerá el reino de David. Y el espíritu se apartó de Saúl, y no supo lo que había profetizado.

\par 3 Pero David se acercó a Jonatán y le dijo: Ven y hagamos un pacto antes de que nos separemos el uno del otro. Porque Saúl, tu padre, busca matarme sin causa. Y como ha visto que me amas, no te dice lo que trama acerca de mí.

\par 4 Pero él me odia por esto: porque tú me amas y para que yo no reine en su lugar. Y mientras le he hecho bien, él me recompensa con mal. Y mientras que maté a Golia por palabra del Más Poderoso, mira qué fin se propone para mí. Porque ha decidido destruir la casa de mi padre. Y ojalá se pusiera en la balanza el juicio de la verdad, para que la multitud de los prudentes oyera la sentencia.

\par 5 Y ahora temo que me mate y pierda su vida por mi causa. Porque nunca derramará sangre inocente sin castigo. ¿Por qué mi alma debe sufrir persecución? Porque yo era el más pequeño entre mis hermanos al pastorear las ovejas, y ¿por qué estoy en peligro de muerte? Porque soy justo y no tengo ninguna iniquidad. ¿Y por qué me odia tu padre? Sin embargo, la justicia de mi padre me ayudará a no caer en manos de tu padre. Y como soy joven y de corta edad, en vano Saúl me tiene envidia.

\par 6 Si le hubiera hecho daño, le pediría que me perdonara el pecado. Porque si Dios perdona la iniquidad, ¿cuánto más tu padre, que es de carne y hueso? He caminado en su casa con un corazón perfecto, sí, crecí ante su rostro como un águila veloz, puse mis manos en el arpa y lo bendije con canciones, y él pensó matarme, y como un gorrión que Huyo ante la faz del halcón, así he huido yo ante su faz.

\par 7 ¿A quién he hablado esto, o a quién he contado lo que he padecido sino a ti y a tu hermana Melcol? Porque en cuanto a nosotros dos, vayamos juntos en la verdad.

\par 8 Y sería mejor, hermano mío, que yo muriera en la batalla que caer en manos de tu padre; porque en la batalla mis ojos miraban a todos lados para defenderlo de sus enemigos. Oh hermano mío Jonatán, escucha mis palabras, y si hay iniquidad en mí, repréndeme.

\par 9 Y Jonatán respondió y dijo: Ven a mí, hermano David, y te declararé tu justicia. Mi alma suspira dolorosamente por tu tristeza porque ahora estamos separados el uno del otro. Y a esto nos obligan nuestros pecados, a separarnos unos de otros. Pero acordémonos unos de otros día y noche mientras vivamos. Y aunque la muerte nos separe, sé que nuestras almas se conocerán unas a otras. Porque tuyo es el reino en este mundo, y de ti será el principio del reino, y vendrá a su tiempo.

\par 10 Y ahora, como un niño que es destetado de su madre, así será nuestra separación. Sea testigo el cielo y testigo la tierra de lo que hemos hablado juntos. Y lloremos unos con otros y juntemos nuestras lágrimas en un vaso y encomendemos el vaso a la tierra, y será un testimonio para nosotros. ii. Y se lamentaron dolorosamente el uno por el otro, y se besaron. Pero Jonatán tuvo miedo y dijo a David: Acordémonos, hermano mío, del pacto que hemos hecho entre nosotros y del juramento puesto en nuestro corazón. Y si muero antes que ti y tú reinas, como el Señor ha dicho, no te acuerdes de la ira de mi padre, sino del pacto que se ha hecho entre mí y ti. Ni pienses en el odio con que mi padre te odia en vano sino en el amor con que yo te he amado. No pienses en aquello en lo que mi padre te fue desagradecido, sino recuerda la mesa en la que hemos comido juntos. Ni te acuerdes de la envidia con que mi padre te envidiaba perversamente, sino de la fe que tú y yo guardamos. No te preocupes por la mentira con que mintió Saúl, sino por los juramentos que nos hemos hecho unos a otros. Y se besaron. Y después de esto David se fue al desierto, y Jonatán entró en la ciudad.

\chapter{63}

\par 1 En aquel tiempo los sacerdotes que habitaban en Noba contaminaban las cosas sagradas del Señor y convertían las primicias en oprobio para el pueblo. Y se enojó Dios y dijo: He aquí, yo exterminaré a los sacerdotes que habitan en Noba, porque andan en los caminos de los hijos de Elí.

\par 2 En aquel tiempo Doech el sirio, que era mayordomo de las mulas de Saúl, vino a Saúl y le dijo: ¿No sabes que el sacerdote Abimelec ha consultado a David, le ha dado una espada y lo ha despedido en paz? Y Saúl envió y llamó a Abimelec, y le dijo: Ciertamente morirás, porque has consultado con mi enemigo. Y Saúl mató a Abimelec y a toda la casa de su padre, y no se salvó ni uno de su tribu, salvo sólo Abiatar su hijo. Este vino a David y le contó todo lo que le había sucedido.

\par 3 Y dijo Dios: He aquí, en el año en que Saúl comenzó a reinar, cuando Jonatán había pecado y quería matarlo, este pueblo se levantó y no lo dejó, y ahora, cuando los sacerdotes fueron asesinados, incluso 385 Los hombres guardaron silencio y no dijeron nada. Por tanto, he aquí, pronto vendrán días en que los entregaré en manos de sus enemigos y caerán heridos, ellos y su rey.

\par 4 Y a Doech el sirio le dijo así el Señor: He aquí, pronto vendrán días en que el gusano subirá a su lengua y le hará desfallecer, y su morada estará con Jair para siempre en el fuego que no se apaga.

\par 5 Todo lo que hizo Saúl, el resto de sus palabras y cómo persiguió a David, ¿no está escrito en el libro de los reyes de Israel?

\par 6 Después de esto murió Samuel, y todo Israel se reunió, lo lloraron y lo sepultaron.

\chapter{64}

\par 1 Entonces Saúl pensó y dijo: Ciertamente quitaré a los hechiceros de la tierra de Israel. Así me recordarán los hombres después de mi partida. Y Saúl dispersó a todos los hechiceros de la tierra. Y dijo Dios: He aquí, Saúl ha quitado de la tierra a los hechiceros, no por miedo a mí, sino para hacerse un nombre. He aquí, a los que él ha dispersado, que acuda a ellos y reciba de ellos adivinación, porque no tiene profetas.

\par 2 En aquel tiempo los filisteos decían cada uno a su vecino: He aquí, el profeta Samuel ha muerto y no hay nadie que ore por Israel. También David, que peleaba por ellos, se convirtió en adversario de Saúl y no está con ellos. Ahora, pues, levantémonos y luchemos poderosamente contra ellos, y venguemos la sangre de nuestros padres. Y los filisteos se reunieron y subieron a la batalla.

\par 3 Y cuando Saúl vio que Samuel había muerto y que David no estaba con él, se le soltaron las manos. Y consultó al Señor, y no le escuchó. Y buscó profetas, y ninguno se le apareció. Y Saúl dijo al pueblo: Busquemos un adivino y preguntémosle lo que tengo en mente. Y el pueblo le respondió: He aquí ahora hay una mujer llamada Sedecla, hija de Debin (o Adod) madianita, la cual engañaba a los hijos de Israel con hechicerías; y he aquí que ella habita en Endor.

\par 4 Entonces Saúl se vistió con ropas horribles y fue a ella, él y dos hombres con él, de noche, y le dijo: Levántame a Samuel. Y ella dijo: Tengo miedo del rey Saúl. Y Saúl le dijo: No recibirás ningún daño de Saúl en este negocio. Y Saúl dijo para sí: Cuando yo era rey en Israel, aunque los gentiles no me veían, sabían que yo era Saúl. Y Saúl preguntó a la mujer, diciendo: ¿Has visto a Saúl alguna vez? Y ella dijo: Muchas veces. Y salió Saúl y lloró y dijo: He aquí ahora sé que mi hermosura ha cambiado, y que la gloria de mi reino ha pasado de mí.

\par 5 Y aconteció que cuando la mujer vio que Samuel se acercaba y vio a Saúl con él, gritó y dijo: He aquí, tú eres Saúl, ¿por qué me has engañado? Y él le dijo: No temas, pero cuéntame lo que viste. Y ella dijo: He aquí, estos 40 años he resucitado muertos para los filisteos, pero esta aparición no se ha visto, ni se verá más adelante.

\par 6 Y Saúl le dijo: ¿Cuál es su forma? Y ella dijo: Tú me preguntas acerca de los dioses. Porque he aquí, su forma no es forma de hombre. Porque está vestido con un manto blanco y tiene un manto sobre él, y dos ángeles lo guían. Y Saúl se acordó del manto que Samuel había rasgado mientras vivía, y juntó sus manos y se arrojó en tierra.

\par 7 Y Samuel le dijo: ¿Por qué me has inquietado para que me haga subir? Pensé que había llegado el momento de recibir la recompensa de mis obras. Por tanto, no te jactes, oh rey, ni tú, oh mujer. Porque no sois vosotros los que me habéis criado, sino el precepto que Dios me habló mientras aún vivía, de que viniera y os dijera que habías pecado por segunda vez al descuidar a Dios. Por esto están turbulentos mis huesos, después que entregué mi alma, para hablarte, y estando muerto, se me oiría como a un vivo.

\par 8 Ahora pues, mañana estarás conmigo tú y tus hijos, cuando el pueblo sea entregado en manos de los filisteos. Y por cuanto tus entrañas se han conmovido de celos, por tanto, lo que es tuyo te será quitado. Y oyó Saúl las palabras de Samuel, y se desmayó su alma, y ​​dijo: He aquí, yo voy a morir con mis hijos, por si acaso mi destrucción puede ser expiación de mis iniquidades. Y Saúl se levantó y se fue de allí.

\chapter{65}

\par 1 Y los filisteos pelearon contra Israel. Y Saúl salió a la batalla. E Israel huyó delante de los filisteos; y viendo Saúl que la batalla se hacía muy dura, dijo en su corazón: ¿Por qué te esfuerzas para vivir, siendo que Samuel te ha anunciado la muerte a ti y a tus hijos?

\par 2 Y Saúl dijo al que llevaba su armadura: Toma tu espada y mátame antes de que vengan los filisteos y me insulten. Y el que llevaba sus armas no le pondría las manos encima.

\par 3 Y él mismo se inclinó sobre su espada y no pudo morir. Y miró hacia atrás y vio a un hombre que corría y lo llamó y le dijo: Toma mi espada y mátame. Porque mi vida aún está en mí.

\par 4 Y vino a matarlo. Y Saúl le dijo: Antes que me mates, dime, ¿quién eres? Y él le dijo: Yo soy Edab, hijo de Agag rey de los amalecitas. Y Saúl dijo: He aquí ahora han venido sobre mí las palabras de Samuel, como él dijo: El que nacerá de Agag te será por tropiezo.

\par 5 Pero ve tú y dile a David: He matado a tu enemigo. Y le dirás: Así dice Saúl: No te acuerdes de mi odio, ni de mi injusticia. . . .



\end{document}