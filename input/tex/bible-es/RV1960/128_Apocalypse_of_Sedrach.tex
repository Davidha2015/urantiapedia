\begin{document}

\title{Apocalipsis de Sedrach}

\chapter{1}

\par \textit{Un sermón del santo y bendito Sedrach sobre el amor, el arrepentimiento, los cristianos ortodoxos y la segunda venida de nuestro Señor Jesucristo. Maestro, concede (tu) bendición.}

\par 1 Amados, nada debemos preferir más que el amor no fingido.

\par 2 Cometemos muchas faltas a cada hora, de día y de noche, y por eso adquiramos el amor, porque cubre multitud de pecados.

\par 3 ¿Qué ganamos, hijos míos, si lo poseemos todo y no tenemos el amor salvador?

\par 4 ¿De qué sirve, hijos míos, dar un gran banquete e invitar al rey y a los nobles y preparar todo tipo de manjares caros para que no falte nada? sin embargo, si no hay sal, no se puede comer ese banquete; y uno no sólo corre con los gastos sino que también desperdicia (sus) esfuerzos y es deshonrado por los invitados.

\par 5 Lo mismo ocurre en nuestra situación, hermanos míos; ¿De qué nos aprovecharemos, pues qué gracia poseemos sin amor?

\par 6 Cada una de nuestras acciones es falsa, incluso si uno es virginal y ayuna, vela, ora y ofrece un banquete a los pobres.

\par 7 Y si uno trae regalos a Dios, u ofrece las primicias de todos sus bienes, o construye iglesias o hace cualquier otra cosa sin amor, Dios lo tendrá por nada, porque (estas cosas) no son aceptables.

\par 8 Así dice el profeta: «El sacrificio de los impíos es abominación al Señor».

\par 9 No os aconsejen hacer nada sin amor.

\par 10 Si dices: «Odio a mi hermano, pero amo a Cristo», eres un mentiroso, y Juan el Teólogo te reprende: ¿cómo puede alguien que no ama a su hermano, a quien ha visto, amar a Dios, a quien ha visto? ¿no visto?

\par 11 Es evidente que cualquiera que odia a su hermano, pero piensa que ama a Cristo, es un mentiroso y se engaña a sí mismo.

\par 12 Porque Juan el Teólogo dice que tenemos este mandamiento de Dios: el que ama a Dios, ame también a su hermano.

\par 13 Y nuevamente el Señor mismo dice: De estos dos (mandamientos) depende toda la ley y los profetas.

\par 14 ¡Oh, qué extraordinario y paradójico es el milagro de que quien tiene amor cumpla toda la ley; el amor es el cumplimiento de la ley.

\par 15 Oh, poder del amor más allá de la imaginación; ¡Oh, poder del amor sin medida!

\par 16 No hay nada más honroso que el amor, ni hay nada más grande ni en el cielo ni en la tierra.

\par 17 Este amor divino es la capital (virtud); entre todas las virtudes el amor es la perfección más elevada del mundo.

\par 18 Habitó en el corazón de Abel; trabajó junto con los Patriarcas; protegió a Moisés; hizo de David la morada del Espíritu Santo; fortaleció a José.

\par 19 Pero ¿por qué digo estas cosas?

\par 20 Lo más importante es que este amor hizo descender del cielo al Hijo de Dios.

\par 21 Por el amor se revelaron todas las cosas buenas; la muerte fue pisoteada, el Hades fue hecho cautivo, Adán fue llamado (de la muerte) y, a través del amor, se formó a partir de entonces un solo rebaño de ángeles y hombres.

\par 22 Por el amor se ha abierto el Paraíso; se promete el reino de los cielos; los lugares desiertos los convirtió en ciudades, y llenó de canciones las montañas y las cuevas; enseñó a hombres y mujeres que caminaban por el camino angosto y doloroso.

\par 23 ¿Pero hasta cuándo prolongaremos este sermón sobre los logros del amor que ni siquiera los ángeles pueden realizar?

\par 24 ¡Oh, bendito amor que concedes todos los bienes!

\par 25 Bienaventurado el hombre que posee fe verdadera y amor no fingido; porque, como dijo el Maestro, nada es mayor que el amor por el cual un hombre da la vida por sus amigos.

\chapter{2}

\par 1 ¿Y escuchó algo escondido? voz en sus oídos: «Aquí, Sedrach, tú que deseas y deseas hablar con Dios y pedirle que te revele las cosas que deseas preguntar».

\par 2 Y Sedrach dijo: «¿Qué es, mi Señor?»

\par 3 Y la voz le dijo: «Fui enviado a ti para llevarte al cielo».

\par 4 Y él dijo: Quiero hablar con Dios cara a cara, pero no puedo, Señor, subir a los cielos.

\par 5 Pero el ángel, extendiendo sus alas, lo tomó y subió al cielo, y lo llevó hasta el tercer cielo, y allí estaba la llama de la divinidad.

\chapter{3}

\par 1 Y el Señor le dijo: «Bienvenido, querido Sedrach.

\par 2 ¿Qué queja tenéis contra el Dios que os creó, porque habéis dicho: «Quiero hablar con Dios cara a cara»?

\par 3 Sedrach le dijo: «En verdad, el hijo tiene una queja contra el Padre: Señor mío, ¿para qué creaste la tierra?»

\par 4 El Señor le dijo: «Para el hombre».

\par 5 Sedrach dijo: «¿Para qué creasteis el mar y por qué esparcisteis todo lo bueno sobre la tierra?»

\par 6 El Señor dijo: «Para el hombre».

\par 7 Sedrach le dijo: Si has hecho estas cosas, ¿por qué destruiste al hombre?

\par 8 Y el Señor dijo: «El hombre es mi obra y la criatura de mis manos, y lo castigo como lo considero correcto».

\chapter{4}

\par 1 Sedrach le dijo: «¿Tu disciplina? Qué es el castigo? y fuego; y son muy amargas, mi Señor.

\par 2 Sería mejor para el hombre si no naciera.

\par 3 En verdad, ¿qué has hecho, Señor mío? ¿Por qué trabajaste con tus manos sin mancha y creaste al hombre, si no quisiste tener misericordia de él?

\par 4 Dios le dijo: «Yo creé al primer hombre, Adán, y lo puse en el Paraíso, en medio del (que es) el árbol de la vida, y le dije: «Come de todo el fruto, sólo que ten cuidado del árbol de la vida, porque si comes de él, ciertamente morirás.»

\par 5 Sin embargo, desobedeció mi mandamiento y, engañado por el diablo, comió del árbol.

\chapter{5}

\par 1 Sedrach le dijo: «Por tu voluntad Adán fue engañado, mi Maestro.

\par 2 Tú ordenaste a tus ángeles que adoraran a Adán, pero el que era el primero entre los ángeles desobedeció tu orden y no lo adoró; y así lo desterrasteis, porque transgredió vuestro mandamiento y no salió (a adorar) la creación de vuestras manos.

\par 3 Si amasteis al hombre, ¿por qué no matasteis al diablo, artífice de toda iniquidad?

\par 4 ¿Quién puede luchar contra un espíritu invisible?

\par 5 Él entra como el humo en el corazón de los hombres y les enseña toda clase de pecados.

\par 6 Él lucha incluso contra ti, el Dios inmortal, y ¿qué puede hacer el hombre compasivo contra él?

\par 7 Sin embargo, ten piedad, Maestro, y destruye el castigo; de otra manera recíbeme también con los pecadores, porque si no eres misericordioso con los pecadores, ¿dónde están tus misericordias y dónde está tu compasión, oh Señor?»


\chapter{6}

\par 1 Y Dios le dijo: «Sabes que todo lo que he ordenado al hombre que haga está a su alcance.

\par 2 Lo hice sabio y heredero del cielo y de la tierra, y sometí todo a él y todo ser viviente huye de él y de su rostro.

\par 3 Sin embargo, habiendo recibido mis dones, se convirtió en un extranjero, en un adúltero y en un pecador.

\par 4 Dime, ¿qué clase de padre daría una herencia a su hijo y, después de recibir el dinero, se marcha dejando a su padre y se convierte en un extranjero y al servicio de extranjeros?

\par 5 Entonces el padre, al ver que el hijo lo ha abandonado (y se ha ido), oscurece su corazón y, al irse, recupera sus riquezas y destierra a su hijo de su gloria porque abandonó a su padre.

\par 6 ¿Cómo es que yo, el Dios maravilloso y celoso, se lo he dado todo, pero él, habiéndolo recibido, se ha hecho adúltero y pecador?

\chapter{7}

\par 1 Sedrach le dijo: «Tú, Maestro, creaste al hombre; ¿Conoces el bajo estado de su voluntad y su conocimiento? y enviáis al hombre al castigo con un pretexto falso; así que quítalo.

\par 2 ¿Se supone que yo solo debo llenar los reinos celestiales?

\par 3 Si no es así, Señor, salva también al hombre.

\par 4 Un hombre lamentable ha transgredido tu voluntad, oh Señor.»

\par 5 «¿Por qué arrojas palabras a mi alrededor como si fueran una red, Sedrach?

\par 6 Creé a Adán y su esposa y el sol y dije: «Mírense unos a otros (para ver) quién está iluminado».

\par 7 Y el sol y Adán eran de la misma naturaleza, pero la esposa de Adán era más brillante en belleza que la luna, y ella le dio vida.

\par 8 Sedrach dijo: «¿De qué sirven las cosas bellas si se marchitan y se convierten en polvo?

\par 9 ¿Cómo es que dijiste, Señor: «No devuelvas mal por mal»?

\par 10 ¿Cómo es, Maestro, que la palabra de tu divinidad nunca miente?

\par 11 ¿Y por qué pagasteis así al hombre, si no queréis devolver mal por mal?

\par 12 Sé que entre los cuadrúpedos el mulo es un animal astuto, no es otro; sin embargo, con la brida la giramos hacia donde queremos.

\par 13 Tienes ángeles; envíalos a vigilar (sobre el hombre) y cuando haga un movimiento hacia el pecado, agárrate de su pie, y no irá a donde quiere».

\chapter{8}

\par 1 Dios le dijo: «Si toco su pie, él dice: «No me has dado ninguna gracia en el mundo», y por eso lo dejé a sus propios deseos porque lo amaba y por eso envié a mis ángeles justos para vigilarlo día y noche».

\par 2 Sedrach dijo: «Sé que entre tus propias criaturas, Maestro, amaste primero al hombre; entre los cuadrúpedos, las ovejas; entre los árboles, el olivo; entre las plantas que dan fruto, la vid; entre las cosas que vuelan, la abeja; entre los ríos, (el) Jordán; entre las ciudades, Jerusalén.

\par 3 Pero el hombre también ama todo esto, Maestro.

\par 4 Dios dijo a Sedrach: «Te voy a preguntar una cosa, Sedrach; si puedes responderme, entonces con razón me has desafiado, aunque hayas tentado a tu creador».

\par 5 Sedrach dijo: «Habla».

\par 6 El Señor Dios dijo: «Desde que creé todo, ¿cuántos han nacido, cuántos han muerto, cuántos morirán y cuántos cabellos tienen?

\par 7 Dime, Sedrach, desde que fueron creados el cielo y la tierra, ¿cuántos árboles se han creado en el mundo, y cuántos caerán, cuántos se harán y cuántas hojas tendrán?

\par 8 Dime, Sedrach, desde que hice el mar, ¿cuántas olas se han agitado, cuántas se han agitado ligeramente, cuántas se levantarán y cuántos vientos soplarán cerca de la orilla del mar?

\par 9 Dime, Sedrach, desde la creación del mundo de los siglos cuando el aire está lleno de lluvia, ¿cuántas gotas han caído sobre el mundo y cuántas caerán?

\par 10 Y Sedrach dijo: «Señor, sólo tú sabes todas estas cosas; sólo tú estás familiarizado con todo esto; Sólo te ruego que liberes al hombre del castigo, porque de lo contrario yo mismo voy a ser castigado y no seré separado de nuestra raza».

\chapter{9}

\par 1 Y dijo Dios a su Hijo unigénito: «Ve, toma el alma de mi amado Sedrach y ponla en el Paraíso».

\par 2 El Hijo unigénito dijo a Sedrach: «Dame lo que nuestro Padre depositó en el vientre de tu madre en tu santa morada desde que naciste».

\par 3 Sedrach dijo: «No te daré mi alma».

\par 4 Dios le dijo: «¿Y por qué fui enviado y por qué vine aquí, y tú finges? ¿a mi?

\par 5 Mi padre me ordenó que no dudara en tomar tu alma; por tanto, dame tu alma más deseada».

\chapter{10}

\par 1 Y Sedrach dijo a Dios: ¿De dónde sacarás mi alma y de qué miembro?

\par 2 Y Dios le dijo: «¿No sabes que está colocado en medio de tus pulmones y de tu corazón y que se extiende a todos los miembros?

\par 3 Se elimina por la faringe, la laringe y la boca; y cuando debe salir (del cuerpo) se saca con dificultad al principio y al juntarse de las uñas y de todos los miembros hay, necesariamente, una gran tensión? en estar separados del cuerpo y desprendidos del corazón».

\par 4 Cuando escuchó todas estas cosas y recordó el recuerdo de la muerte, Sedrach se turbó mucho y dijo a Dios: «Señor, dame un poco de tiempo para llorar, porque he oído que las lágrimas logran mucho y puede llegar a ser una cura suficiente para el humilde cuerpo de vuestras criaturas».

\chapter{11}


\par 1 Y llorando y lamentándose comenzó a decir: «¡Oh cabeza maravillosa, adornada como el cielo! Oh luz del sol sobre el cielo y la tierra; tu cabello se conoce por Theman, tus ojos por Bosra, tus oídos por el trueno, tu lengua por la corneta, y tu cerebro es una pequeña creación; la cabeza, el movimiento de todo el cuerpo, es digno de confianza y muy hermoso, amado por todos pero tan pronto como cae en la tierra no es reconocido.

\par 2 ¡Oh manos que sostienen tan bien, que son fáciles de enseñar y trabajadoras, a través de las cuales se alimenta el cuerpo!

\par 3 ¡Oh manos tan hábiles que reunisteis materiales y juntas adornasteis las casas!

\par 4 Oh dedos, hermosísimos y adornados de oro y plata; incluso las grandes estructuras se hacen con los dedos; las tres coyunturas estiran las palmas y juntan cosas buenas; pero ahora os habéis convertido en extraños en este mundo.

\par 5 Oh pies, que tan bien caminan, moviéndose solos tan rápidamente e incansablemente.

\par 6 Oh rodillas así unidas, sin ti el cuerpo no se mueve; los pies corren juntos con el sol y la luna, noche y día, reuniendo todas las cosas, alimento y bebida que nutren el cuerpo.

\par 7 ¡Oh pies, rápidos y ágiles, que remueven la faz de la tierra y adornan las casas con todo bien!

\par 8 ¡Oh pies que sostenéis todo el cuerpo y que vais directamente a las sienes! arrepentíos y suplicad a los santos, y ahora de repente debéis permanecer impasibles.

\par 9 Oh cabeza, manos y pies, hasta ahora te he retenido.

\par 10 Oh alma, ¿qué te puso en el cuerpo humilde y miserable?

\par 11 Sin embargo, ahora, separados de él, subís adonde el Señor os llama y el cuerpo miserable va al juicio.

\par 12 ¡Oh hermoso cuerpo, cabellos derramados por las estrellas, cabeza adornada como el cielo!

\par 13 ¡Oh rostro perfumado, ojos como ventanas, voz como sonido de clarín, lengua que habla tan fácilmente, barba bien recortada, cabello como las estrellas, cabeza alta como el cielo, cuerpo adornado, el iluminador elegante y renombrada, pero ahora, después de caer dentro de la tierra, tu belleza debajo de la tierra no se ve».

\chapter{12}

\par 1 Cristo le dijo: «Detente, Sedrach, ¿hasta cuándo derramarás lágrimas y gemirás?   

\par 2 Se te ha abierto el paraíso y después de morir vivirás.

\par 3 Sedrach le dijo: «Una vez más te hablaré, Señor, mientras viva, antes de morir; y no ignores mi súplica».

\par 4 El Señor le dijo: «Habla, Sedrach».

\par 5 (Y Sedrach dijo:) «Si el hombre vive ochenta, noventa o cien años, y los vive en pecado, pero al final se convierte y el hombre vive en arrepentimiento, ¿cuántos días de arrepentimiento le perdonaréis? ) sus pecados?»

\par 6 Dios le dijo: «Si regresa después de vivir ciento u ochenta años y se arrepiente durante tres años y da fruto de justicia y le alcanza la muerte, entonces no me acordaré de todos sus pecados».

\chapter{13}

\par 1 Sedrach le dijo: «Tres años son demasiados, mi Señor.

\par 2 Quizás llegue su muerte y no cumplirá su arrepentimiento.

\par 3 Ten piedad, Señor, de tu imagen y sé compasivo, porque tres años son demasiados.

\par 4 Dios le dijo: «¿Si un hombre vive después de cien años y se acuerda de su muerte y se confiesa delante de los hombres, y después de un año lo encuentro? Perdonaré todos sus pecados».

\par 5 Sedrach volvió a decir: «Señor, te pido nuevamente misericordia para tu criatura; un año es mucho, y tal vez llegue su muerte y de repente se lo arrebate».

\par 6 El Salvador le dijo: «Sedrach, amado mío, te haré una pregunta y luego podrás continuar con tus preguntas; Si el pecador se arrepiente durante cuarenta días, ¿no me acordaré yo de todos los pecados que ha cometido?

\chapter{14}

\par 1 Y Sedrach dijo al arcángel Miguel: «Escúchame, fuerte protector; ayúdame e intercede para que Dios sea misericordioso con el mundo».

\par 2 Y cayendo rostro en tierra, rogaron a Dios, diciendo: Señor, enséñanos de qué manera y mediante qué arrepentimiento el hombre puede salvarse, o mediante qué trabajo.

\par 3 Dios dijo: «Con arrepentimiento, súplica y liturgia, con lágrimas abundantes y gemidos fervorosos.

\par 4 ¿No sabéis que mi profeta David (fue salvo) gracias a las lágrimas, y que los demás se salvaron en un momento?

\par 5 Tú sabes, Sedrach, que hay naciones que no tienen ley, pero la cumplen; no son bautizados, pero mi espíritu divino entra en ellos y se convierten a mi bautismo, y los recibo con mis justos en el seno de Abraham.

\par 6 Y hay algunos que son bautizados con mi bautismo y ungidos con mi divina mirra, pero se han desesperado y no cambian de opinión.

\par 7 Sin embargo, los espero con mucha piedad y mucha misericordia para que se arrepientan.

\par 8 Pero hacen lo que mi divinidad odia, y no escucharon al sabio que preguntó y dijo: «Nosotros de ninguna manera justificamos al pecador».

\par 9 ¿No sabéis que está escrito: «Y los que se arrepientan no verán castigo»?

\par 10 Y no escucharon ni a los apóstoles ni mi palabra en los Evangelios y causan tristeza a mis ángeles, y ciertamente en mis reuniones y en mis liturgias no escuchan a mi ángel y no están en mis santas iglesias; están de pie y no se postran (se postran) con miedo y temblor, sino que pronuncian largas palabras que ni yo ni mis ángeles aceptamos».

\chapter{15}

\par 1 Sedrach dijo a Dios: «Señor, sólo tú eres sin pecado y muy misericordioso, mostrando compasión y gracia a los pecadores, pero tu divinidad dijo: «No he venido a llamar a los justos, sino a los pecadores al arrepentimiento.» »

\par 2 Y el Señor dijo a Sedrach: «¿No sabes, Sedrach, que el ladrón, después de cambiar de opinión, se salvó en un instante?

\par 3 ¿No sabéis que incluso mi apóstol y evangelista fue salvo en un instante? [. . . pero los pecadores no se salvan] porque sus corazones son como piedra podrida; son los que caminan por caminos impíos y que perecen con el Anticristo».

\par 4 Sedrach dijo: «Señor mío, tú también dijiste: «Mi espíritu divino entró en las naciones que, aunque no tienen ley, hacen las cosas de la ley».

\par 5 Sin embargo, como el ladrón, el apóstol y el evangelista y los demás que han tropezado en tu reino, Señor mío, ¿perdonas de la misma manera a los que en estos últimos días? He pecado contra ti, Señor, porque la vida está llena de trabajo y es obstinada».

\chapter{16}

\par 1 El Señor dijo a Sedrach: «Hice al hombre en tres etapas; cuando es joven, paso por alto sus errores debido a su juventud; de nuevo, cuando es hombre yo vigilo su mente; de nuevo, cuando envejezca, lo preservaré para que se arrepienta».

\par 2 Sedrach dijo: «Señor, tú sabes y estás familiarizado con todo esto; sin embargo, ten compasión de los pecadores».

\par 3 El Señor le dijo: «Mi amado Sedrach, prometo tener compasión incluso menos de cuarenta días, hasta veinte, y quien recuerde tu nombre no verá el lugar del castigo, sino que estará con los justos en un lugar de refrigerio y descanso, y el pecado del que copia este admirable sermón no será contado por los siglos de los siglos».

\par 4 Y Sedrach dijo: «Señor, también a quien celebre una liturgia en honor de tu siervo, líbralo, Señor, de todo mal». Y el siervo de Dios, Sedrach, dijo: «Ahora, Maestro, toma mi alma».

\par 5 Y Dios lo tomó y lo puso en el Paraíso con todos los santos. A él sea la gloria y el poder por los siglos de los siglos. Amén.




\es{document}