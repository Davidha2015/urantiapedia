\begin{document}

\title{Esimene Moosese raamat}


\chapter{1}

\par 1 Alguses lõi Jumal taeva ja maa.
\par 2 Maa oli tühi ja paljas ja pimedus oli sügavuse peal ja Jumala Vaim hõljus vete kohal.
\par 3 Ja Jumal ütles: „Saagu valgus!” Ja valgus sai.
\par 4 Ja Jumal nägi, et valgus oli hea, ja Jumal lahutas valguse pimedusest.
\par 5 Ja Jumal nimetas valguse päevaks ja pimeduse ta nimetas ööks. Siis sai õhtu ja sai hommik - esimene päev.
\par 6 Ja Jumal ütles: „Saagu laotus vete vahele ja see lahutagu veed vetest!”
\par 7 Ja nõnda sündis: Jumal tegi laotuse ja lahutas veed, mis olid laotuse all, vetest, mis olid laotuse peal.
\par 8 Ja Jumal nimetas laotuse taevaks. Siis sai õhtu ja sai hommik - teine päev.
\par 9 Ja Jumal ütles: „Veed kogunegu taeva all ühte paika, et kuiva näha oleks!” Ja nõnda sündis.
\par 10 Ja Jumal nimetas kuiva pinna maaks ja veekogu ta nimetas mereks. Ja Jumal nägi, et see oli hea.
\par 11 Ja Jumal ütles: „Maast tärgaku haljas rohi, seemet kandvad taimed, viljapuud, mille viljas on nende seeme, nende liikide järgi maa peale!” Ja nõnda sündis:
\par 12 maa laskis võrsuda haljast rohtu, seemet kandvaid taimi nende liikide järgi, ja viljapuid, mille viljas on nende seeme, nende liikide järgi. Ja Jumal nägi, et see oli hea.
\par 13 Siis sai õhtu ja sai hommik - kolmas päev.
\par 14 Ja Jumal ütles: „Saagu valgused taevalaotusse eraldama päeva ööst! Tähistagu need seatud aegu, päevi ja aastaid,
\par 15 olgu nad valgusteks taevalaotuses, valgustuseks maale!” Ja nõnda sündis:
\par 16 Jumal tegi kaks suurt valgust: suurema valguse valitsema päeval ja väiksema valguse valitsema öösel, ning tähed.
\par 17 Ja Jumal pani need taevalaotusse, et nad valgustaksid maad
\par 18 ja valitseksid päeval ja öösel ja eraldaksid valguse pimedusest. Ja Jumal nägi, et see oli hea.
\par 19 Siis sai õhtu ja sai hommik - neljas päev.
\par 20 Ja Jumal ütles: „Vesi kihagu elavaist olendeist, ja maa peal lennaku linnud taevalaotuse poole!”
\par 21 Ja Jumal lõi suured mereloomad ja kõiksugu elavad olendid, kellest vesi kihab, nende liikide järgi, ja kõiksugu tiibadega linnud nende liikide järgi. Ja Jumal nägi, et see oli hea.
\par 22 Ja Jumal õnnistas neid ja ütles: „Olge viljakad ja teid saagu palju, täitke mere vesi, ja lindusid saagu palju maa peale!”
\par 23 Siis sai õhtu ja sai hommik - viies päev.
\par 24 Ja Jumal ütles: „Maa toogu esile elavad olendid nende liikide järgi, kariloomad ja roomajad ja metsloomad nende liikide järgi!” Ja nõnda sündis:
\par 25 Jumal tegi metsloomad nende liikide järgi, ja kariloomad nende liikide järgi, ja kõik roomajad maa peal nende liikide järgi. Ja Jumal nägi, et see oli hea.
\par 26 Ja Jumal ütles: „Tehkem inimesed oma näo järgi, meie sarnaseks, et nad valitseksid kalade üle meres, lindude üle taeva all, loomade üle ja kogu maa üle ja kõigi roomajate üle, kes maa peal roomavad!”
\par 27 Ja Jumal lõi inimese oma näo järgi, Jumala näo järgi lõi ta tema, ta lõi tema meheks ja naiseks.
\par 28 Ja Jumal õnnistas neid, ja Jumal ütles neile: „Olge viljakad ja teid saagu palju, täitke maa ja alistage see enestele; ja valitsege kalade üle meres, lindude üle taeva all ja kõigi loomade üle, kes maa peal liiguvad!”
\par 29 Ja Jumal ütles: „Vaata, mina annan teile kõik seemet kandvad taimed kogu maal, ja kõik puud, mis kannavad vilja, milles on nende seeme; need olgu teile roaks!
\par 30 Ja kõigile loomadele maa peal ja kõigile lindudele taeva all ja kõigile roomajaile maa peal, kelles on elav hing, annan ma kõiksugu haljast rohtu toiduks.” Ja nõnda sündis.
\par 31 Ja Jumal vaatas kõike, mis ta oli teinud, ja vaata, see oli väga hea. Siis sai õhtu ja sai hommik - kuues päev.

\chapter{2}

\par 1 Nõnda on taevas ja maa ning kõik nende väed valmis saanud.
\par 2 Ja Jumal oli lõpetanud seitsmendaks päevaks oma töö, mis ta tegi, ja hingas seitsmendal päeval kõigist oma tegudest, mis ta oli teinud.
\par 3 Ja Jumal õnnistas seitsmendat päeva ja pühitses seda, sest ta oli siis hinganud kõigist oma tegudest, mis Jumal luues oli teinud.
\par 4 See on lugu taeva ja maa sündimisest, kui need loodi. Sel ajal, kui Issand Jumal tegi maa ja taeva,
\par 5 kui ainsatki väljapõõsast ei olnud veel maa peal ja ainsatki väljarohtu ei olnud veel tärganud, sest Issand Jumal ei olnud lasknud vihma sadada maa peale, ja inimest ei olnud põldu harimas,
\par 6 tõusis udu maast ja kastis kogu mullapinda.
\par 7 Ja Issand Jumal valmistas inimese, kes põrm on, mullast, ja puhus tema ninasse eluhinguse: nõnda sai inimene elavaks hingeks.
\par 8 Ja Issand Jumal istutas Eedeni rohuaia päevatõusu poole ja pani sinna inimese, kelle ta oli valmistanud.
\par 9 Ja Issand Jumal laskis maast tõusta kõiksugu puid, mis olid armsad pealtnäha ja millest oli hea süüa, ja elupuu keset aeda, ning hea ja kurja tundmise puu.
\par 10 Ja Eedenist sai alguse jõgi, mis kastis rohuaeda, jagunedes sealtpeale neljaks haruks:
\par 11 esimese nimi on Piison, see voolab ümber kogu Havilamaa, kus on kulda;
\par 12 selle maa kuld on hea, seal on bedolavaiku ja karneoolikive.
\par 13 Ja teise jõe nimi on Giihon, see voolab ümber kogu Kuusimaa.
\par 14 Ja kolmanda jõe nimi on Hiddekel, see voolab hommiku pool Assurit; ja neljas jõgi on Frat.
\par 15 Ja Issand Jumal võttis inimese ja pani ta Eedeni aeda harima ja hoidma.
\par 16 Ja Issand Jumal keelas inimest ja ütles: „Kõigist aia puudest sa võid küll süüa,
\par 17 aga hea ja kurja tundmise puust sa ei tohi süüa, sest päeval, mil sa sellest sööd, pead sa surma surema!”
\par 18 Ja Issand Jumal ütles: „Inimesel ei ole hea üksi olla; ma tahan teha temale abi, kes tema kohane on.”
\par 19 Ja Issand Jumal valmistas mullast kõik loomad väljal ja kõik linnud taeva all ning tõi inimese juurde, et näha, kuidas tema neid nimetab. Ja kuidas inimene igat elavat olendit nimetas, nõnda pidi selle nimi olema.
\par 20 Ja inimene pani nimed kõigile kariloomadele ja lindudele taeva all ja kõigile metsloomadele, aga inimesele ei leidunud abilist, kes tema kohane oleks.
\par 21 Siis Issand Jumal laskis tulla raske une inimese peale ja see jäi magama; siis ta võttis ühe tema küljeluudest ning sulges selle aseme taas lihaga.
\par 22 Ja Issand Jumal ehitas küljeluu, mille ta inimesest oli võtnud, naiseks ja tõi tema Aadama juurde.
\par 23 Ja Aadam ütles: „See on nüüd luu minu luust ja liha minu lihast. Teda peab hüütama mehe naiseks, sest ta on mehest võetud!”
\par 24 Seepärast jätab mees maha oma isa ja ema ning hoiab oma naise poole, ja nemad on üks liha!
\par 25 Ja nad olid mõlemad alasti, Aadam ja tema naine, ega häbenenud.

\chapter{3}

\par 1 Aga madu oli kavalam kõigist loomadest väljal, kelle Issand Jumal oli teinud, ja ta ütles naisele: „Kas Jumal on tõesti öelnud, et te ei tohi süüa mitte ühestki rohuaia puust?”
\par 2 Ja naine vastas maole: „Me sööme küll rohuaia puude vilja,
\par 3 aga selle puu viljast, mis on keset aeda, on Jumal öelnud: Te ei tohi sellest süüa ega selle külge puutuda, et te ei sureks!”
\par 4 Ja madu ütles naisele: „Te ei sure, kindlasti mitte,
\par 5 aga Jumal teab, et päeval, mil te sellest sööte, lähevad teie silmad lahti ja te saate Jumala sarnaseks, tundes head ja kurja.”
\par 6 Ja naine nägi, et puust oli hea süüa, ja see tegi ta silmadele himu, ja et puu oli ihaldusväärne, sest see pidi targaks tegema. Siis ta võttis selle viljast ja sõi ning andis ka oma mehele, ja tema sõi.
\par 7 Siis nende mõlema silmad läksid lahti ja nad tundsid endid alasti olevat, ja nad õmblesid viigilehti kokku ning tegid enestele põlled.
\par 8 Ja nad kuulsid Issanda Jumala häält, kes rohuaias sinna ja tänna käis, kui päev viluks läks, ja Aadam ja tema naine peitsid endid Issanda Jumala palge eest rohuaia puude keskele.
\par 9 Ja Issand Jumal hüüdis Aadamat ning ütles temale: „Kus sa oled?”
\par 10 Ja tema vastas: „Ma kuulsin su häält rohuaias ja kartsin, sest ma olen alasti. Sellepärast ma peitsin enese ära.”
\par 11 Siis ta küsis: „Kes on sulle teada andnud, et sa alasti oled? Või oled sa söönud puust, millest ma sind keelasin söömast?”
\par 12 Ja Aadam vastas: „Naine, kelle sa mulle kaasaks andsid, tema andis mulle puust ja ma sõin.”
\par 13 Ja Issand Jumal küsis naiselt: „Miks sa seda tegid?„ Ja naine vastas: ”Madu pettis mind, ja ma sõin.”
\par 14 Siis Issand Jumal ütles maole: „Et sa seda tegid, siis ole sa neetud kõigi koduloomade ja kõigi metsloomade seas! Sa pead roomama oma kõhu peal ja põrmu sööma kogu eluaja!
\par 15 Ja ma tõstan vihavaenu sinu ja naise vahele, sinu seemne ja tema seemne vahele, kes purustab su pea, aga kelle kanda sa salvad.”
\par 16 Naisele ta ütles: „Sinule ma saadan väga palju valu, kui sa lapseootel oled: sa pead valuga lapsi ilmale tooma! Sa himustad küll oma meest, aga tema valitseb su üle.”
\par 17 Aga Aadamale ta ütles: „Et sa kuulasid oma naise sõna ja sõid puust, millest mina olin sind keelanud, öeldes, et sa ei tohi sellest süüa, siis olgu maapind neetud sinu üleastumise pärast! Vaevaga pead sa sellest sööma kogu eluaja!
\par 18 Ta peab sulle kasvatama kibuvitsu ja ohakaid, ja põllutaimed olgu sulle toiduks!
\par 19 Oma palge higis pead sa leiba sööma, kuni sa jälle mullaks saad, sest sellest sa oled võetud! Tõesti, sa oled põrm ja pead jälle põrmuks saama!”
\par 20 Ja Aadam pani oma naisele nimeks Eeva, sest ta sai kõigi elavate emaks.
\par 21 Ja Issand Jumal tegi Aadamale ja ta naisele nahkriided ning pani neile selga.
\par 22 Ja Issand Jumal ütles: „Vaata, inimene on saanud nagu üheks meie hulgast, tundes head ja kurja. Aga nüüd, et ta oma kätt ei sirutaks ega võtaks ka elupuust ega sööks ega elaks igavesti!”
\par 23 Siis saatis Issand Jumal tema Eedeni rohuaiast välja, et ta hariks maad, millest ta oli võetud.
\par 24 Ja ta ajas Aadama välja ja pani hommikupoole Eedeni rohuaeda keerubid ja tuleleegina sähviva mõõga, et need valvaksid elupuu teed.

\chapter{4}

\par 1 Ja Aadam sai ühte oma naise Eevaga, kes jäi lapseootele ja tõi Kaini ilmale ning ütles: „Ma olen Issanda abiga mehe ilmale toonud.”
\par 2 Ja tema sünnitas taas: ta venna Aabeli. Ja Aabel oli lambakarjane, Kain aga oli põllumees.
\par 3 Ja mõne aja pärast juhtus, et Kain tõi Issandale roaohvri maaviljast,
\par 4 ja ka Aabel tõi oma lammaste esimesest soost ning nende rasvast, ja Issand vaatas Aabeli ja tema roaohvri peale,
\par 5 aga Kaini ja tema roaohvri peale ta ei vaadanud. Siis Kain vihastus väga ja lõi pilgu maha.
\par 6 Ja Issand küsis Kainilt: „Mispärast sa vihastud? Ja mispärast sa pilgu maha lööd?
\par 7 Eks ole: kui sa head teed, siis on su pilk tõstetud üles? Aga kui sa head ei tee, siis luurab patt ukse ees ja himustab sind. Kuid sina pead tema üle valitsema!”
\par 8 Ja Kain ütles oma vennale Aabelile: „Lähme väljale!” Ja kui nad väljal olid, tungis Kain oma venna Aabeli kallale ja tappis tema.
\par 9 Aga Issand küsis Kainilt: „Kus on su vend Aabel?„ Ja tema vastas: ”Ei mina tea. Kas ma olen oma venna hoidja?”
\par 10 Ja tema ütles: „Mis sa oled teinud? Sinu venna vere hääl kisendab maa pealt minu poole!
\par 11 Aga nüüd ole sa neetud siit maa pealt, mis oma suu on avanud, su venna verd sinu käest vastu võttes!
\par 12 Kui sa harid maad, siis see ei anna sulle enam oma rammu. Sa pead maa peal olema hulkur ja põgenik!”
\par 13 Ja Kain ütles Issandale: „Mu karistus on suurem, kui ma suudan kanda!
\par 14 Vaata, sa oled mind täna ära ajanud siit maalt ja ma pean varjule minema su palge eest ning maa peal olema hulkur ja põgenik. Ja igaüks, kes mind leiab, tapab mu.”
\par 15 Ja Issand ütles temale: „Ei, sugugi mitte, vaid igaühele, kes Kaini tapab, peab seitsmekordselt kätte makstama!” Ja Issand pani Kainile märgi, et leidja teda maha ei lööks.
\par 16 Ja Kain läks ära Issanda palge eest ning elas Noodimaal, hommiku pool Eedenit.
\par 17 Ja Kain sai ühte oma naisega, kes jäi lapseootele ja tõi ilmale Hanoki; ja ta ehitas ühe linna ning nimetas selle linna oma poja Hanoki nime järgi.
\par 18 Ja Hanokile sündis Iirad, Iiradile sündis Mehuujael, Mehuujaelile sündis Metuusael, Metuusaelile sündis Lemek.
\par 19 Ja Lemek võttis enesele kaks naist: ühe nimi oli Aada ja teise nimi oli Silla.
\par 20 Ja Aada tõi ilmale Jaabali, kes sai nende isaks, kes elasid telkides ja kasvatasid karja.
\par 21 Ja tema venna nimi oli Juubal, kes sai kõigi kandlelööjate ja vilepuhujate isaks.
\par 22 Ja Silla tõi ilmale Tuubal-Kaini, kes õpetas sepist taguma kõiki, kes tegid vask- ja raudsepa tööd; ja Tuubal-Kaini õde oli Naama.
\par 23 Ja Lemek ütles oma naistele: „Aada ja Silla, kuulge mu häält, te Lemeki naised, pange tähele mu kõnet! Jah, haava pärast ma tapan mehe ja verme pärast nooruki!
\par 24 Kui Kaini pärast makstakse kätte seitse korda, siis Lemeki pärast seitsekümmend seitse korda!”
\par 25 Ja Aadam sai taas ühte oma naisega, kes tõi ilmale poja ja pani temale nimeks Sett, öeldes: „Jumal andis mulle teise järeltulija Aabeli asemele, kuna Kain ta tappis.”
\par 26 Ja Setile sündis ka poeg, ja ta pani temale nimeks Enos. Sel ajal hakati Issanda nime appi hüüdma.

\chapter{5}

\par 1 See on Aadama sünniraamat. Sel päeval, mil Jumal inimese lõi, tegi ta tema Jumala sarnaseks.
\par 2 Ta lõi tema meheks ja naiseks, ja ta õnnistas neid ning andis neile loomisepäeval nime: inimene.
\par 3 Kui Aadam oli elanud sada kolmkümmend aastat, siis sündis temale poeg, kes oli tema sarnane, tema nägu, ja ta pani sellele nimeks Sett.
\par 4 Ja Aadama elupäevi oli pärast Seti sündimist kaheksasada aastat, ja temale sündis poegi ja tütreid.
\par 5 Nõnda oli kõiki Aadama elupäevi, mis ta elas, üheksasada kolmkümmend aastat; siis ta suri.
\par 6 Kui Sett oli elanud sada viis aastat, siis sündis temale Enos.
\par 7 Ja Sett elas pärast Enose sündimist kaheksasada seitse aastat, ja temale sündis poegi ja tütreid.
\par 8 Nõnda oli kõiki Seti elupäevi üheksasada kaksteist aastat; siis ta suri.
\par 9 Kui Enos oli elanud üheksakümmend aastat, siis sündis temale Keenan.
\par 10 Ja Enos elas pärast Keenani sündimist kaheksasada viisteist aastat, ja temale sündis poegi ja tütreid.
\par 11 Nõnda oli kõiki Enose elupäevi üheksasada viis aastat; siis ta suri.
\par 12 Kui Keenan oli elanud seitsekümmend aastat, siis sündis temale Mahalalel.
\par 13 Ja Keenan elas pärast Mahalaleli sündimist kaheksasada nelikümmend aastat, ja temale sündis poegi ja tütreid.
\par 14 Nõnda oli kõiki Keenani elupäevi üheksasada kümme aastat; siis ta suri.
\par 15 Kui Mahalalel oli elanud kuuskümmend viis aastat, siis sündis temale Jered.
\par 16 Ja Mahalalel elas pärast Jeredi sündimist kaheksasada kolmkümmend aastat, ja temale sündis poegi ja tütreid.
\par 17 Nõnda oli kõiki Mahalaleli elupäevi kaheksasada üheksakümmend viis aastat; siis ta suri.
\par 18 Kui Jered oli elanud sada kuuskümmend kaks aastat, siis sündis temale Eenok.
\par 19 Ja Jered elas pärast Eenoki sündimist kaheksasada aastat, ja temale sündis poegi ja tütreid.
\par 20 Nõnda oli kõiki Jeredi elupäevi üheksasada kuuskümmend kaks aastat; siis ta suri.
\par 21 Kui Eenok oli elanud kuuskümmend viis aastat, siis sündis temale Metuusala.
\par 22 Ja Eenok kõndis pärast Metuusala sündimist koos Jumalaga kolmsada aastat, ja temale sündis poegi ja tütreid.
\par 23 Nõnda oli kõiki Eenoki elupäevi kolmsada kuuskümmend viis aastat.
\par 24 Eenok kõndis koos Jumalaga, ja siis ei olnud teda enam, sest Jumal võttis tema ära.
\par 25 Kui Metuusala oli elanud sada kaheksakümmend seitse aastat, siis sündis temale Lemek.
\par 26 Ja Metuusala elas pärast Lemeki sündimist seitsesada kaheksakümmend kaks aastat, ja temale sündis poegi ja tütreid.
\par 27 Nõnda oli kõiki Metuusala elupäevi üheksasada kuuskümmend üheksa aastat; siis ta suri.
\par 28 Kui Lemek oli elanud sada kaheksakümmend kaks aastat, siis sündis temale poeg
\par 29 ja ta pani sellele nimeks Noa ning ütles: „See trööstib meid meie töös ja käte vaevas maa pärast, mille Issand on neednud!”
\par 30 Ja Lemek elas pärast Noa sündimist viissada üheksakümmend viis aastat, ja temale sündis poegi ja tütreid.
\par 31 Nõnda oli kõiki Lemeki elupäevi seitsesada seitsekümmend seitse aastat; siis ta suri.
\par 32 Ja kui Noa oli viissada aastat vana, siis sündisid Noale Seem, Haam ja Jaafet.

\chapter{6}

\par 1 Ja kui inimesi hakkas maa peal palju saama ja neile sündis tütreid,
\par 4 Sel ajal, ja veel pärastpoolegi, kui Jumala pojad heitsid inimeste tütarde juurde ja need neile lapsi ilmale tõid, olid hiiglased maa peal: needsamad vägimehed, kes muistsest ajast on kuulsad mehed.
\par 6 siis Issand kahetses, et ta inimese oli teinud maa peale, ja ta süda valutas.
\par 7 Ja Issand ütles: „Ma tahan inimese, kelle ma olen loonud, maa pealt kaotada, niihästi inimesed kui loomad ja roomajad ja taeva linnud, sest ma kahetsen, et ma nad olen teinud!”
\par 8 Aga Noa leidis armu Issanda silmis.
\par 9 See on jutustus Noa soost: Noa oli üks õige mees, täiesti vaga oma rahvapõlve seas; Noa kõndis koos Jumalaga.
\par 10 Ja Noale sündis kolm poega - Seem, Haam ja Jaafet.
\par 11 Aga maa oli Jumala palge ees raisku läinud ja vägivald täitis maad.
\par 12 Ja Jumal vaatas maad, ja näe, see oli raisku läinud, sest kõik liha maa peal oli oma eluviisides raiskunud.
\par 13 Ja Jumal ütles Noale: „Ma olen otsustanud teha lõpu kõigele lihale, sest maa on täis nende vägivalda, ja seepärast, vaata, ma hävitan nemad koos maaga.
\par 14 Tee enesele laev goferipuust; laev tee kambritega ja pigita seda seest ning väljast maapigiga!
\par 15 Ja tee see nõndaviisi: laeva pikkus olgu kolmsada küünart, laius viiskümmend küünart ja kõrgus kolmkümmend küünart;
\par 16 valmista see küünramõõdu järgi; tee laevale katus peale ja tee uks laeva küljesse; tee sellele alumine, keskmine ja ülemine lagi!
\par 17 Sest vaata, ma saadan veeuputuse maa peale ja hävitan taeva alt kõik liha, kus eluvaim sees on; kõik, mis maa peal on, peab hinge heitma!
\par 18 Aga sinuga ma teen lepingu ja sina pead minema laeva, sina ja su pojad, su naine ja su poegade naised sinuga.
\par 19 Ja sa pead viima laeva kõigist elavaist olendeist, kõigest lihast, igast liigist kaks, et nad koos sinuga jääksid elama: need olgu isane ja emane.
\par 20 Lindudest nende liikide järgi ja loomadest nende liikide järgi, kõigist roomajaist maa peal nende liikide järgi, kõigist peavad kaks tulema sinu juurde, et nad võiksid jääda elama.
\par 21 Ja sina võta enesele kõigest roast, mida süüakse, ja kogu enese juurde, et see oleks toiduks sinule ja neile.”
\par 22 Ja Noa tegi kõik. Nii nagu Jumal teda käskis, nõnda ta tegi.

\chapter{7}

\par 1 Ja Issand ütles Noale: „Mine sina ja kogu su pere laeva, sest ma olen näinud, et sa selle rahvapõlve seas minu ees õige oled.
\par 2 Võta enesele kõigist puhtaist loomadest seitse paari, isane ja emane; ja loomadest, kes puhtad ei ole, kaks - isane ja emane.
\par 3 Nõndasamuti lindudest taeva all seitse paari, isane ja emane, et nende sugu jääks elama kogu maa peale.
\par 4 Sest juba seitsme päeva pärast ma lasen vihma sadada maa peale nelikümmend päeva ja nelikümmend ööd, ja ma kaotan maapinnalt kõik olendid, keda ma olen teinud!”
\par 5 Ja Noa tegi kõik nõnda, nagu Issand teda käskis.
\par 6 Ja Noa oli kuussada aastat vana, kui uputusvesi maa peale tuli.
\par 7 Ja Noa läks laeva, ja ta pojad ja ta naine ja ta poegade naised temaga, veeuputuse eest.
\par 8 Puhtaist loomadest ja loomadest, kes puhtad ei ole, ja lindudest ja kõigist, kes maa peal roomavad,
\par 9 tulid kahekesi Noa juurde laeva isane ja emane, nõnda nagu Jumal Noale oli käsu andnud.
\par 10 Ja seitsme päeva pärast tuli veeuputus maa peale.
\par 11 Sel aastal, mil Noa kuussada aastat vanaks sai, teise kuu seitsmeteistkümnendal päeval, otse selsamal päeval puhkesid kõik suure sügavuse allikad ja taevaluugid tehti lahti.
\par 12 Ja sadu tuli maa peale nelikümmend päeva ja nelikümmend ööd.
\par 13 Otse selsamal päeval läksid Noa ja Noa pojad Seem ja Haam ja Jaafet ning Noa naine ja kolm ta poegade naist üheskoos laeva,
\par 14 nemad ja kõik metsloomad oma liikide järgi, ja kõiksugu kariloomad oma liikide järgi, ja kõiksugu roomajad, kes maa peal roomavad, oma liikide järgi, ja kõiksugu lendajad oma liikide järgi, kõik linnud, kõik tiivulised.
\par 15 Ja need tulid Noa juurde laeva kahekaupa kõigest lihast, kus eluvaim sees on.
\par 16 Ja need, kes sisse läksid, olid isane ja emane kõigest lihast, nõnda nagu Jumal temale oli käsu andnud. Ja Issand sulges ukse tema tagant.
\par 17 Siis tuli nelikümmend päeva veeuputust maa peale; vesi tõusis ja tõstis laeva, nõnda et see kerkis maast kõrgele.
\par 18 Ja vesi võttis võimust ning seda sai maa peal väga palju, ja laev liikus veepinnal.
\par 19 Ja vesi võttis maa peal väga võimust ja kõik kõrged mäed kogu taeva all kaeti.
\par 20 Vesi tõusis neist viisteist küünart kõrgemale, nõnda et mäed olid kaetud.
\par 21 Siis heitis hinge kõik liha, mis maa peal liikus, niihästi linnud kui kariloomad ja metsloomad ja kõik roomajad, kes maa peal roomasid, ja kõik inimesed ka.
\par 22 Kõik, kellel eluvaimu hingus ninas oli, kõik, kes olid kuival maal, need surid.
\par 23 Nõnda hävitati kõik olendid, kes maa peal olid; niihästi inimesed kui loomad ja roomajad, ja linnud taeva all hävitati maa pealt, järele jäid ainult Noa ja need, kes temaga laevas olid.
\par 24 Ja vesi võimutses maa peal sada viiskümmend päeva.

\chapter{8}

\par 1 Siis Jumal mõtles Noale ja kõigile metsloomadele ja kõigile kariloomadele, kes temaga laevas olid; ja Jumal laskis tuult puhuda üle maa ja vesi alanes.
\par 2 Ja sügavuse allikad ja taevaluugid suleti, ja sadu taevast keelati.
\par 3 Ja vesi taganes maa pealt, taganes üha, ja saja viiekümne päeva pärast oli vesi vähenenud.
\par 4 Ja seitsmenda kuu seitsmeteistkümnendal päeval peatus laev Ararati mägede kohal.
\par 5 Ja vesi vähenes üha kümnenda kuuni; kümnenda kuu esimesel päeval paistsid mägede tipud.
\par 6 Ja kui nelikümmend päeva oli möödunud, siis Noa avas laeva akna, mille ta oli teinud,
\par 7 ja laskis välja ühe kaarna; see lendas sinna ja tänna, kuni vesi maa pealt oli kuivanud.
\par 8 Siis ta laskis enese juurest välja ühe tuvi, et näha, kas vesi on maa pealt kahanenud.
\par 9 Aga tuvi ei leidnud oma jalavarvastele puhkepaika ja tuli tagasi tema juurde laeva, sest vesi oli veel kogu maa peal; siis ta pistis oma käe välja ja võttis tema ning pani enese juurde laeva.
\par 10 Ja ta ootas veel teist seitse päeva ning laskis taas ühe tuvi laevast välja.
\par 11 Ja õhtul tuli tuvi tema juurde, ja vaata, tal oli nokas õlipuu haljas leht. Siis Noa mõistis, et vesi oli maa pealt kahanenud.
\par 12 Ja ta ootas veel teist seitse päeva ning laskis ühe tuvi välja, aga see ei tulnud enam tagasi tema juurde.
\par 13 Ja kuuesaja esimesel Noa eluaastal, esimese kuu esimesel päeval, oli vesi maa pealt kuivanud. Ja Noa võttis ära laeva katuse ja vaatas, ja ennäe, maapind oli tahenenud.
\par 14 Ja teise kuu kahekümne seitsmendal päeval oli maa täiesti kuiv.
\par 15 Ja Jumal kõneles Noaga ning ütles:
\par 16 „Mine laevast välja, sina ja su naine ja su pojad ja su poegade naised koos sinuga!
\par 17 Kõik loomad, kes su juures on, kõik liha, niihästi linnud kui loomad, ja kõik roomajad, kes maa peal roomavad, vii enesega koos välja, et nad sigineksid maa peal, oleksid viljakad ja et neid maa peale saaks palju!”
\par 18 Ja Noa läks välja ja ta pojad ja ta naine ja ta poegade naised koos temaga.
\par 19 Kõik loomad, kõik linnud ja kõik roomajad, kes liiguvad maa peal, läksid laevast välja sugukondade kaupa.
\par 20 Ja Noa ehitas Issandale altari ja võttis kõigist puhtaist loomadest ja kõigist puhtaist lindudest ning ohverdas altaril põletusohvreid.
\par 21 Ja Issand tundis meeldivat lõhna ja Issand mõtles oma südames: „Ma ei nea enam maad inimese pärast, sest inimese südame mõtlemised on kurjad ta lapsepõlvest peale; ma ei hävita ka enam kõike, mis elab, nõnda nagu ma olen teinud.
\par 22 Niikaua kui püsib maa, ei lõpe külv ega lõikus, külm ega kuum, suvi ega talv, päev ega öö.”

\chapter{9}

\par 1 Ja Jumal õnnistas Noad ja tema poegi ning ütles neile: „Olge viljakad, teid saagu palju ja täitke maa!
\par 2 Teid peavad kartma ja pelgama kõik maa loomad ja kõik taeva linnud, kõik, kes maa peal liiguvad, ja kõik mere kalad; need on teie kätte antud.
\par 3 Kõik, mis liigub ja elab, olgu teile roaks; kõik selle annan ma teile nagu halja rohugi.
\par 4 Kummatigi ei tohi te liha süüa ta hingega, see on: ta verega!
\par 5 Tõepoolest, teie eneste verd ma nõuan taga: ma nõuan seda kõigilt loomadelt, ja ma nõuan inimestelt vastastikku inimese hinge!
\par 6 Kes valab inimese vere, selle vere valab inimene, sest inimene on tehtud Jumala näo järgi!
\par 7 Ja teie olge viljakad, teid saagu palju, siginege maa peal ja paljunege seal!”
\par 8 Ja Jumal rääkis Noa ja tema poegadega, kes ta juures olid, ning ütles:
\par 9 „Mina, vaata, teen lepingu teiega ja teie järglastega pärast teid,
\par 10 ja iga elava hingega, kes teie juures on: lindudega, kariloomadega ja kõigi metsloomadega, kes teie juures on, kõigiga, kes laevast välja tulid, kõigi maa loomadega.
\par 11 Ma teen teiega lepingu, et enam ei hävitata kõike liha uputusveega ja veeuputus ei tule enam maad rikkuma.”
\par 12 Ja Jumal ütles: „Lepingu tähis, mille ma teen enese ja teie ja iga teie juures oleva elava hinge vahel igavesteks põlvedeks, on see:
\par 13 ma panen pilvedesse oma vikerkaare ja see on lepingu tähiseks minu ja maa vahel.
\par 14 Kui ma kogun maa kohale pilvi ja pilvedes nähakse vikerkaart,
\par 15 siis ma mõtlen oma lepingule, mis on minu ja teie ja iga elava hinge vahel kõiges lihas, ja vesi ei saa enam kõike liha hävitavaks uputuseks.
\par 16 Kui pilvedes on vikerkaar, siis ma vaatan seda ja mõtlen igavesele lepingule Jumala ja iga elava hinge vahel maa peal olevas kõiges lihas.”
\par 17 Ja Jumal ütles Noale: „See on selle lepingu tähis, mille ma olen teinud enese ja kõige liha vahel, mis maa peal on.”
\par 18 Ja Noa pojad, kes laevast välja tulid, olid Seem, Haam ja Jaafet; ja Haam oli Kaanani isa.
\par 19 Need kolm olid Noa pojad ja neist rahvastus kogu maa.
\par 20 Noa hakkas põllumeheks ja istutas viinamäe.
\par 21 Ta jõi veini, jäi joobnuks ja ajas oma telgis enese paljaks.
\par 22 Aga Haam, Kaanani isa, nägi oma isa paljast ihu ja rääkis sellest oma kahele vennale õues.
\par 23 Siis Seem ja Jaafet võtsid vaiba ja panid enestele õlgadele, läksid tagurpidi ja katsid kinni oma isa palja ihu; nende näod olid ära pööratud, nõnda et nad oma isa paljast ihu ei näinud.
\par 24 Kui Noa ärkas oma veiniuimast ja sai teada, mis ta noorem poeg oli teinud,
\par 25 siis ta ütles: „Neetud olgu Kaanan, saagu ta oma vendadele sulaste sulaseks!”
\par 26 Ta ütles veel: „Kiidetud olgu Issand, Seemi Jumal, ja Kaanan olgu tema sulane!
\par 27 Jumal andku avarust Jaafetile, ta elagu Seemi telkides ja Kaanan olgu tema sulane!”
\par 28 Ja Noa elas pärast veeuputust kolmsada viiskümmend aastat.
\par 29 Ja kõiki Noa elupäevi oli üheksasada viiskümmend aastat; siis ta suri.

\chapter{10}

\par 1 Ja need on Noa poegade Seemi, Haami ja Jaafeti järeltulijad; pärast veeuputust sündisid neile pojad.
\par 2 Jaafeti pojad olid Gomer, Maagoog, Maadai, Jaavan, Tubal, Mesek ja Tiiras.
\par 3 Ja Gomeri pojad olid Askenas, Riifat ja Toogarma.
\par 4 Ja Jaavani pojad olid Eliisa ja Tarsis, kittid ja rodanlased;
\par 5 neist eraldusid saarte rahvad. Need olid Jaafeti järeltulijad nende maade järgi, igaühel oma keel, nende suguvõsade kaupa, vastavalt neile rahvastele.
\par 6 Ja Haami pojad olid Kuus, Mitsraim, Puut ja Kaanan.
\par 7 Ja Kuusi pojad olid Seba, Havila, Sabta, Raema ja Sabteka; ja Raema pojad olid Seeba ja Dedan.
\par 8 Ja Kuusile sündis Nimrod, kes oli esimene vägev mees maa peal.
\par 9 Tema oli vägev kütt Issanda ees, seepärast öeldakse: „vägev kütt Issanda ees nagu Nimrod.”
\par 10 Ja tema kuningriigi alguseks olid Paabel, Erek, Akad ja Kalne Sinearimaal.
\par 11 Sellelt maalt läks ta Assurisse ja ehitas Niineve, Rehobot-Iiri ja Kelahi,
\par 12 ja Reseni Niineve ja Kelahi vahele - see on see suur linn.
\par 13 Ja Mitsraimile sündisid luudlased, anamlased, lehablased, naftuhlased,
\par 14 patruuslased ja kasluhlased, kellest vilistid on lähtunud, ja kaftoorlased.
\par 15 Ja Kaananile sündisid Siidon, tema esmasündinu, ja Heet,
\par 16 jebuuslased, emorlased, girgaaslased,
\par 17 hiivlased, arklased, siinlased,
\par 18 arvadlased, semarlased ja hamatlased; hiljem harunesid kaananlaste suguvõsad.
\par 19 Ja kaananlaste maa-ala oli Siidonist Gerari suunas Assani, Soodoma, Gomorra, Adma ja Seboimi suunas Lesani.
\par 20 Need olid Haami järeltulijad nende suguvõsade, keelte, maade ja rahvaste järgi.
\par 21 Ja Seemile sündisid ka pojad; tema oli kõigi Eeberi poegade esiisa, Jaafeti vanem vend.
\par 22 Seemi pojad olid Eelam, Assur, Arpaksad, Luud ja Aram.
\par 23 Ja Arami pojad olid Uuts, Huul, Geter ja Maas.
\par 24 Ja Arpaksadile sündis Selah, Selahile sündis Eeber.
\par 25 Ja Eeberile sündis kaks poega: ühe nimi oli Peleg, sest tema päevil jagunes maa; ja tema venna nimi oli Joktan.
\par 26 Ja Joktanile sündisid Almodad, Selef, Hasarmavet, Jerah,
\par 27 Hadoram, Uusal, Dikla,
\par 28 Oobal, Abimael, Seeba,
\par 29 Oofir, Havila ja Joobab; need kõik olid Joktani pojad.
\par 30 Nende elukohad olid Meesast Sefaara suunas idapoolses mäestikus.
\par 31 Need olid Seemi järeltulijad nende suguvõsade, keelte, maade ja rahvaste järgi.
\par 32 Need olid Noa poegade suguvõsad nende põlvnemiste ja rahvaste järgi; neist harunesid rahvad maa peal pärast veeuputust.

\chapter{11}

\par 1 Kogu maailmas oli aga üks keel ja ühesugused sõnad.
\par 2 Ja sündis, kui nad hommiku poolt teele läksid, et nad Sinearimaal leidsid oru ja jäid sinna elama.
\par 3 Nad ütlesid üksteisele: „Tehkem nüüd telliskive ja põletagem neid hästi.” Siis olid telliskivid neile ehituskivideks ja maapigi oli sideaineks.
\par 4 Ja nad ütlesid: „Tulge, ehitagem enestele linn ja torn, mille tipp oleks taevas, ja tehkem enestele nimi, et me ei hajuks üle kogu maailma!”
\par 5 Aga Issand tuli alla vaatama linna ja torni, mida inimlapsed ehitasid.
\par 6 Ja Issand ütles: „Vaata, rahvas on üks ja neil kõigil on üks keel, ja see on alles nende tegude algus. Nüüd ei ole neil võimatu ükski asi, mida nad kavatsevad teha!
\par 7 Mingem nüüd alla ja segagem seal nende keel, et nad üksteise keelt ei mõistaks!”
\par 8 Ja Issand pillutas nad sealt üle kogu maailma ja nad jätsid linna ehitamata.
\par 9 Seepärast pandi sellele nimeks Paabel, sest seal segas Issand ära kogu maailma keele ja sealt pillutas Issand nad üle kogu maailma.
\par 10 Need olid Seemi järeltulijad: kui Seem oli sada aastat vana, siis sündis temale Arpaksad kaks aastat pärast veeuputust.
\par 11 Ja Seem elas pärast Arpaksadi sündimist viissada aastat, ja temale sündis poegi ja tütreid.
\par 12 Kui Arpaksad oli elanud kolmkümmend viis aastat, siis sündis temale Selah.
\par 13 Ja Arpaksad elas pärast Selahi sündimist nelisada kolm aastat, ja temale sündis poegi ja tütreid.
\par 14 Kui Selah oli elanud kolmkümmend aastat, siis sündis temale Eeber.
\par 15 Ja Selah elas pärast Eeberi sündimist nelisada kolm aastat, ja temale sündis poegi ja tütreid.
\par 16 Kui Eeber oli elanud kolmkümmend neli aastat, siis sündis temale Peleg.
\par 17 Ja Eeber elas pärast Pelegi sündimist nelisada kolmkümmend aastat, ja temale sündis poegi ja tütreid.
\par 18 Kui Peleg oli elanud kolmkümmend aastat, siis sündis temale Reu.
\par 19 Ja Peleg elas pärast Reu sündimist kakssada üheksa aastat, ja temale sündis poegi ja tütreid.
\par 20 Kui Reu oli elanud kolmkümmend kaks aastat, siis sündis temale Serug.
\par 21 Ja Reu elas pärast Serugi sündimist kakssada seitse aastat, ja temale sündis poegi ja tütreid.
\par 22 Kui Serug oli elanud kolmkümmend aastat, siis sündis temale Naahor.
\par 23 Ja Serug elas pärast Naahori sündimist kakssada aastat, ja temale sündis poegi ja tütreid.
\par 24 Kui Naahor oli elanud kakskümmend üheksa aastat, siis sündis temale Terah.
\par 25 Ja Naahor elas pärast Terahi sündimist sada üheksateist aastat, ja temale sündis poegi ja tütreid.
\par 26 Kui Terah oli elanud seitsekümmend aastat, siis sündisid temale Aabram, Naahor ja Haaran.
\par 27 Ja need olid Terahi järeltulijad: Terahile sündisid Aabram, Naahor ja Haaran; ja Haaranile sündis Lott.
\par 28 Aga Haaran suri enne kui ta isa Terah oma sünnimaal Kaldea Uuris.
\par 29 Ja Aabram ja Naahor võtsid enestele naised; Aabrami naise nimi oli Saarai ja Naahori naise nimi oli Milka, Haarani tütar; Haaran oli Milka ja Jiska isa.
\par 30 Aga Saarai oli viljatu, temal ei olnud last.
\par 31 Ja Terah võttis oma poja Aabrami ja Haarani poja Loti, oma pojapoja, ja Saarai, oma minia, oma poja Aabrami naise, ja lahkus koos nendega Kaldea Uurist, et minna Kaananimaale; ja nad jõudsid Haaranini ning jäid sinna elama.
\par 32 Ja Terahi elupäevi oli kakssada viis aastat, ja Terah suri Haaranis.

\chapter{12}

\par 1 Ja Issand ütles Aabramile: „Mine omalt maalt, omast sugukonnast ja isakojast maale, mille ma sulle näitan!
\par 2 Ma teen sind suureks rahvaks ja õnnistan sind, ma teen su nime suureks, et sa oleksid õnnistuseks!
\par 3 Siis ma õnnistan neid, kes sind õnnistavad, panen vande alla selle, kes sind neab, ja sinu nimel õnnistavad endid kõik suguvõsad maa peal!”
\par 4 Ja Aabram läks, nagu Issand teda käskis, ja Lott läks koos temaga; Aabram oli seitsekümmend viis aastat vana, kui ta Haaranist lahkus.
\par 5 Ja Aabram võttis oma naise Saarai ja oma vennapoja Loti ja kogu nende varanduse, mis nad olid soetanud, ja Haaranis hangitud hingelised, ja nad läksid teele Kaananimaa poole. Ja nad jõudsid Kaananimaale.
\par 6 Ja Aabram käis maa läbi Sekemi püha paigani, Moore tammeni; kaananlased olid siis veel sellel maal.
\par 7 Ja Issand ilmutas ennast Aabramile ning ütles: „Sinu soole ma annan selle maa!” Siis ta ehitas sinna altari Issandale, kes oli ennast temale ilmutanud.
\par 8 Sealt ta liikus edasi mäestikku Peetelist hommiku poole ja lõi oma telgi üles, nõnda et Peetel jäi õhtu ja Ai hommiku poole; ja ta ehitas sinna altari Issandale ning hüüdis appi Issanda nime.
\par 9 Ja Aabram läks teele, rändas üha ja siirdus Lõunamaale.
\par 10 Aga maal oli nälg. Siis Aabram läks alla Egiptusesse, et seal võõrana elada, sest maal oli suur nälg.
\par 11 Ja kui ta minnes Egiptusele ligines, ütles ta oma naisele Saaraile: „Vaata, ma tean, et sa oled ilusa välimusega naine.
\par 12 Aga kui egiptlased sind näevad, ütlevad nad: „See on tema naine!” Siis nad tapavad minu, aga sinu jätavad elama.
\par 13 Ütle siis, et oled minu õde, et mu käsi sinu tõttu võiks hästi käia ja mu hing sinu pärast ellu jääks!”
\par 14 Kui Aabram jõudis Egiptusesse, siis nägid egiptlased, et naine oli väga ilus.
\par 15 Ja kui vaarao vürstid teda nägid, siis nad ülistasid teda vaaraole ja naine võeti vaarao kotta.
\par 16 Vaarao tegi Aabramile tema pärast head: ta sai lambaid ja kitsi, veiseid ja eesleid, sulaseid ja teenijaid, emaeesleid ja kaameleid.
\par 17 Aga Issand nuhtles vaaraod ja tema koda suurte nuhtlustega Saarai, Aabrami naise pärast.
\par 18 Siis vaarao kutsus Aabrami ning ütles: „Miks sa mulle seda tegid? Miks sa ei teatanud mulle, et ta on sinu naine?
\par 19 Miks sa ütlesid: Ta on mu õde, nõnda et ma tema enesele naiseks võtsin? Aga nüüd, vaata, seal on su naine, võta tema ja mine!”
\par 20 Ja vaarao andis tema pärast meestele käsu, et nad saadaksid minema tema ja ta naise ja kõik, mis tal oli.

\chapter{13}

\par 1 Ja Aabram läks Egiptusest üles Lõunamaale, tema ja ta naine ja kõik, mis tal oli; ja Lott oli koos temaga.
\par 2 Ja Aabram oli väga rikas karja, hõbeda ja kulla poolest.
\par 3 Ja ta rändas peatuspaigast teise, Lõunamaalt Peeteli poole, sinna paika, kus ta telk enne oli olnud, Peeteli ja Ai vahel,
\par 4 altari paika, mille ta varem sinna oli teinud; ja Aabram hüüdis seal appi Issanda nime.
\par 5 Aga ka Lotil, kes rändas koos Aabramiga, oli lambaid ja kitsi, veiseid ja telke.
\par 6 Kuid maa ei suutnud neid toita, et üheskoos elada, sest nende varandus oli nii suur, et neil oli võimatu üheskoos elada.
\par 7 Ja Aabrami loomade karjaste ja Loti loomade karjaste vahel tekkis riid; kaananlased ja perislased elasid siis veel sellel maal.
\par 8 Siis Aabram ütles Lotile: „Ärgu olgu riidu minu ja sinu vahel, minu karjaste ja sinu karjaste vahel. Meie, mehed, oleme ju vennad!
\par 9 Eks ole kogu maa su ees lahti? Mine nüüd minu juurest ära, lähed sina vasakut kätt, lähen mina paremat kätt; lähed sina paremat kätt, lähen mina vasakut kätt.”
\par 10 Siis Lott tõstis oma silmad üles ja nägi, et kogu Jordani piirkond oli kõikjal veerikas; enne kui Issand Soodoma ja Gomorra hävitas, oli see kuni Soarini otsekui Issanda rohuaed, samasugune nagu Egiptusemaa.
\par 11 Ja Lott valis enesele kogu Jordani piirkonna; Lott läks teele hommiku poole ja nad lahkusid teineteisest.
\par 12 Aabram jäi Kaananimaale ja Lott asus piirkonna linnadesse ning lõi oma telgid üles Soodomani.
\par 13 Aga Soodoma mehed olid väga pahad ja patused Issanda ees.
\par 14 Ja Issand ütles Aabramile, pärast seda kui Lott tema juurest oli lahkunud: „Tõsta nüüd oma silmad üles ja vaata paigast, kus sa oled, põhja ja lõuna ja hommiku ja õhtu poole,
\par 15 sest kogu maa, mida sa näed, ma annan sinule ja su soole igaveseks ajaks!
\par 16 Ja ma teen su soo maapõrmu sarnaseks: kui keegi suudab maapõrmu ära lugeda, siis on sinugi sugu äraloetav.
\par 17 Võta kätte, käi maa läbi pikuti ja põiki, sest ma annan selle sinule!”
\par 18 Ja Aabram võttis telgid ja tuli ning elas Mamre tammikus, mis on Hebroni juures; ja ta ehitas sinna altari Issandale.

\chapter{14}

\par 1 Ja Sineari kuninga Amrafeli, Ellasari kuninga Arjoki, Eelami kuninga Kedorlaomeri ja Goojimi kuninga Tideali päevil sündis,
\par 2 et nad alustasid sõda Soodoma kuninga Bera, Gomorra kuninga Birsa, Adma kuninga Sineabi, Seboimi kuninga Semeeberi ja Bela, see on Soari kuninga vastu.
\par 3 Need kõik kogunesid Siddimi orgu, kus nüüd on Soolameri.
\par 4 Kaksteist aastat olid nad Kedorlaomerit orjanud, aga kolmeteistkümnendal aastal nad tõstsid mässu.
\par 5 Ja neljateistkümnendal aastal tulid Kedorlaomer ja need kuningad, kes olid koos temaga, ja lõid refalasi Astarot-Karnaimis, susiite Haamis, emiite Kirjataimi tasandikul
\par 6 ja horiite nende mäestikus Seiris kuni Eel-Paaranini, mis on kõrbe ääres.
\par 7 Siis nad pöördusid tagasi ja tulid Een-Mispatti, see on Kaadesisse, ja vallutasid kogu amalekkide väljade ala, samuti võitsid nad emorlasi, kes elasid Haseson-Taamaris.
\par 8 Aga Soodoma kuningas, Gomorra kuningas, Adma kuningas, Seboimi kuningas ja Bela, see on Soari kuningas, läksid välja ja valmistusid tapluseks nende vastu Siddimi orus:
\par 9 Eelami kuninga Kedorlaomeri, Goojimi kuninga Tideali, Sineari kuninga Amrafeli ja Ellasari kuninga Arjoki vastu - neli kuningat viie vastu.
\par 10 Aga Siddimi org oli täis maapigi auke. Kui Soodoma ja Gomorra kuningad põgenesid, siis nad langesid neisse, kuna ülejäänud põgenesid mäestikku.
\par 11 Ja nad võtsid kogu Soodoma ja Gomorra varanduse ja kogu nende toiduse ning läksid ära.
\par 12 Ja ära minnes nad võtsid kaasa ka Loti, Aabrami vennapoja, ja tema varanduse; ta elas ju Soodomas.
\par 13 Aga üks põgenik tuli ja teatas Aabramile, heebrealasele, kes elas emorlase Mamre, Eskoli ja Aaneri venna tammikus; ja need olid Aabrami liitlased.
\par 14 Kui Aabram kuulis, et ta vennapoeg oli vangi viidud, siis ta viis välja oma kodakondsed, kes tema peres olid sündinud, arvult kolmsada kaheksateist, ja ajas vaenlasi taga kuni Daanini.
\par 15 Ta jaotas öösel oma sulased nende vastu, lõi neid ja jälitas neid kuni Hoobani, mis on Damaskusest vasakut kätt.
\par 16 Ja ta tõi tagasi kogu varanduse; ka Loti, oma vennapoja, ja tema varanduse ta tõi tagasi, samuti naised ja rahva.
\par 17 Ja kui ta Kedorlaomerit ja koos temaga olevaid kuningaid löömast tagasi tuli, läks Soodoma kuningas temale vastu Saave orgu, see on Kuningaorgu.
\par 18 Ja Melkisedek, Saalemi kuningas, tõi leiba ja veini, sest tema oli kõige kõrgema Jumala preester,
\par 19 ja õnnistas teda ning ütles: „Olgu õnnistatud Aabram, kõige kõrgema Jumala, taeva ja maa Looja poolt!
\par 20 Olgu kiidetud kõige kõrgem Jumal, kes sinu vaenlased su kätte andis!” Ja Aabram andis temale kümnist kõigest.
\par 21 Ja Soodoma kuningas ütles Aabramile: „Anna hingelised mulle, aga varandus võta enesele!”
\par 22 Kuid Aabram ütles Soodoma kuningale: „Ma tõstan oma käe üles Issanda, kõige kõrgema Jumala poole, kes on taeva ja maa Looja,
\par 23 et ma ei võta lõngaotsa ega jalatsipaelagi kõigest sellest, mis on sinu oma, et sa ei saaks öelda: Mina olen Aabrami rikkaks teinud!
\par 24 Mul pole midagi vaja - ainult, mis poisid sõid, ja mehed, kes koos minuga käisid - Aaner, Eskol ja Mamre -, need võtku oma osa!”

\chapter{15}

\par 1 Pärast neid lugusid tuli Aabramile nägemuses Issanda sõna, kes ütles: „Ära karda, Aabram! Mina olen sulle kilbiks. Sinu tasu on väga suur!”
\par 2 Aga Aabram ütles: „Issand Jumal! Mida sa mulle saad anda? Mina ju lähen ära lastetuna ja mu koja valitsejaks on Elieser Damaskusest!”
\par 3 Ja Aabram ütles: „Vaata, mulle sa ei ole ihuvilja andnud, aga näe, minu kojas sündinu saab mu pärijaks.”
\par 4 Ja vaata, temale tuli Issanda sõna, kes ütles: „Tema ei ole sinu pärija, vaid see, kes tuleb välja su oma ihust, on su pärija.”
\par 5 Ja ta viis tema õue ning ütles: „Vaata nüüd taeva poole ja loe tähti, kui sa suudad neid lugeda!„ Ja ta ütles temale: ”Nõnda saab olema sinu sugu!”
\par 6 Ja ta uskus Issandat ning see arvati temale õiguseks.
\par 7 Ja ta ütles temale: „Mina olen Issand, kes tõi sind Kaldea Uurist, et anda sulle päranduseks see maa!”
\par 8 Aga ta küsis: „Issand Jumal, millest ma ära tunnen, et ma selle pärin?”
\par 9 Siis ta vastas temale: „Too mulle kolmeaastane mullikas, kolmeaastane kits, kolmeaastane jäär ja turteltuvi koos lennuvõimelise pojaga.”
\par 10 Ja ta tõi temale kõik need, lõikas need keskelt lõhki ja pani vastavad pooled vastakuti; aga lindusid ta ei lõiganud lõhki.
\par 11 Kui röövlinnud laskusid korjuste peale, siis peletas Aabram need minema.
\par 12 Aga kui päike loojus, vajus Aabram sügavasse unne, ja vaata, suur ja pime hirm haaras teda.
\par 13 Ja Issand ütles Aabramile: „Sa pead teadma, et su järglased on võõrastena maal, mis ei ole nende oma; nad tehakse orjadeks ja neid vaevatakse nelisada aastat.
\par 14 Aga ka rahvast, keda nad orjavad, ma karistan, ja selle järel nad tulevad ära suure varandusega.
\par 15 Sina ise aga lähed rahuga oma vanemate juurde, sind maetakse heas vanuses.
\par 16 Alles neljas põlv tuleb siia tagasi, sest emorlaste süü ei ole tänini veel küllaldane.”
\par 17 Ja kui päike oli loojunud ja kui oli pime, siis nähti suitsevat küpsetusahju ja tuleleeki, mis nende lõigatud tükkide vahelt läbi käis.
\par 18 Selsamal päeval tegi Issand Aabramiga lepingu ja ütles: „Sinu soole ma annan selle maa Egiptuseojast suure jõeni, Frati jõeni,
\par 19 keenlased, kenislased, kadmonlased,
\par 20 hetid, perislased, refalased,

\chapter{16}

\par 1 Ja Saarai, Aabrami naine, ei toonud temale last ilmale; aga tal oli teenijaks egiptlanna, nimega Haagar.
\par 2 Ja Saarai ütles Aabramile: „Vaata, Issand on mind keelanud last saamast. Heida nüüd mu teenija juurde, vahest saan järeltulija temalt!” Ja Aabram kuulas Saarai sõna.
\par 3 Ja Saarai, Aabrami naine, võttis egiptlanna Haagari, oma teenija, pärast seda kui Aabram oli kümme aastat elanud Kaananimaal, ja andis ta oma mehele Aabramile naiseks.
\par 4 Ja tema heitis Haagari juurde ja see jäi lapseootele; aga kui see nägi, et ta oli lapseootel, siis oli ta emand tema silmis nagu alam.
\par 5 Ja Saarai ütles Aabramile: „Mulle sündinud ülekohus tulgu sinu peale! Ma andsin oma teenija sinu sülle, aga kui ta nüüd näeb, et ta on lapseootel, siis olen mina tema silmis nagu alam. Issand mõistku õigust minu ja sinu vahel!”
\par 6 Aga Aabram ütles Saaraile: „Vaata, su teenija on sinu käes! Talita temaga, nagu sa heaks arvad!” Siis Saarai alandas teda, aga seejärel ta põgenes tema juurest.
\par 7 Ja Issanda ingel leidis tema veeallika juurest kõrbes, Suuri tee ääres oleva allika juurest.
\par 8 Ja ta ütles: „Haagar, Saarai teenija! Kust sa tuled ja kuhu sa lähed?„ Ja tema vastas: ”Ma põgenen oma emanda Saarai eest.”
\par 9 Siis ütles Issanda ingel temale: „Mine tagasi oma emanda juurde ja alanda ennast tema käte alla!”
\par 10 Ja Issanda ingel ütles temale: „Ma teen sinu soo nõnda arvurikkaks, et see paljuse pärast pole loetav!”
\par 11 Ja Issanda ingel ütles temale: „Vaata, sa oled lapseootel ja tood poja ilmale! Pane temale nimeks Ismael, sest Issand on kuulnud su alandusest!
\par 12 Temast tuleb mees nagu metseesel - tema käsi on igaühe vastu ja igaühe käsi on tema vastu - ta elab vaenus kõigi oma vendadega.”
\par 13 Siis ta nimetas Issandat, kes temaga oli rääkinud, nimega „Sina oled nähtav Jumal„, sest ta ütles: ”Kas ma siin ikka veel näen pärast oma nägemust?”
\par 14 Seepärast nimetatakse seda kaevu Lahhai-Roi kaevuks; vaata, see on Kaadesi ja Baaredi vahel.
\par 15 Ja Haagar tõi Aabramile poja ilmale; ja Aabram pani oma pojale, kelle Haagar oli sünnitanud, nimeks Ismael.
\par 16 Ja Aabram oli kaheksakümmend kuus aastat vana, kui Haagar Aabramile Ismaeli ilmale tõi.

\chapter{17}

\par 1 Kui Aabram oli üheksakümmend üheksa aastat vana, siis Issand ilmutas ennast Aabramile ja ütles temale: „Mina olen Kõigeväeline Jumal, käi minu palge ees ja ole vaga!
\par 2 Ma teen lepingu enese ja sinu vahel ja teen sind väga paljuks.”
\par 3 Siis Aabram heitis silmili maha ja Jumal rääkis temaga, öeldes:
\par 4 „See olen mina! Vaata, mu leping sinuga on, et sina saad paljude rahvaste isaks.
\par 5 Sinu nime ei hüüta siis enam Aabramiks, vaid su nimi olgu Aabraham, sest ma teen sind paljude rahvaste isaks!
\par 6 Ma teen sind väga viljakaks ja lasen sind rahvaiks saada, kuningadki põlvnevad sinust.
\par 7 Ma teen lepingu enese ja sinu vahel, ja sinu soo vahel pärast sind, igaveseks lepinguks sugupõlvedele, et ma olen Jumalaks sinule ja su soole pärast sind.
\par 8 Ja ma annan sinule ja su soole pärast sind selle maa, kus sa võõrana elad, kogu Kaananimaa, igaveseks omandiks. Ja mina olen neile Jumalaks!”
\par 9 Ja Jumal ütles Aabrahamile: „Ja sina pead mu lepingut pidama, sina ja su sugu pärast sind põlvest põlve.
\par 10 See on minu leping minu ja teie ning sinu soo vahel pärast sind, mida te peate pidama: kõik meesterahvad tuleb teil ümber lõigata!
\par 11 Te peate oma eesnaha liha ümber lõikama ja see olgu minu ja teie vahelise lepingu märgiks.
\par 12 Kaheksapäevastena tuleb teil ümber lõigata kõik teie meesterahvad põlvkondade viisi, olgu peres sündinud, olgu raha eest ostetud ükskõik missuguselt võõralt, kes sinu soost ei ole,
\par 13 kindlasti tuleb ümber lõigata niihästi su peres sündinu kui su raha eest ostetu. Minu leping peab teie ihu küljes olema igavese lepinguna!
\par 14 Aga eesnahaga meesterahvas, kelle eesnaha liha ei ole ümber lõigatud, tuleb hävitada oma rahva seast: ta on tühistanud minu lepingu!”
\par 15 Ja Jumal ütles Aabrahamile: „Saaraid, oma naist, ära hüüa enam Saaraiks, vaid tema nimi olgu Saara!
\par 16 Ma õnnistan teda ja annan ka temalt sulle poja. Ma õnnistan teda nõnda, et ta saab rahvaiks, rahvaste kuningadki põlvnevad temast.”
\par 17 Siis Aabraham heitis silmili maha, naeris ja ütles oma südames: „Kas peaks saja-aastasele poeg sündima? Või peaks üheksakümneaastane Saara sünnitama?”
\par 18 Ja Aabraham ütles Jumalale: „Kui ainult Ismaelgi sinu ees jääks elama!”
\par 19 Siis ütles Jumal: „Siiski, su naine Saara sünnitab sulle poja ja sa pead panema temale nimeks Iisak. Ja ma teen temaga lepingu, igaveseks lepinguks tema soole pärast teda.
\par 20 Aga ka Ismaeli pärast ma olen sind kuulnud. Vaata, ma õnnistan teda ja teen ta viljakaks ning väga arvurikkaks. Temast sünnib kaksteist vürsti ja ma teen ta suureks rahvaks.
\par 21 Kuid oma lepingu teen ma Iisakiga, kelle Saara sulle sünnitab tuleval aastal selsamal ajal.”
\par 22 Kui Jumal oli lõpetanud kõneluse Aabrahamiga, siis ta läks tema juurest üles.
\par 23 Ja Aabraham võttis oma poja Ismaeli ja kõik oma peres sündinud, ja kõik raha eest ostetud, kõik meesterahvad Aabrahami pere inimeste hulgast, ja lõikas ümber nende eesnaha liha, otse selsamal päeval, nõnda nagu Jumal teda oli käskinud.
\par 24 Aabraham oli üheksakümmend üheksa aastat vana, kui ta eesnaha liha ümber lõigati.
\par 25 Ja Ismael, tema poeg, oli kolmteist aastat vana, kui ta eesnaha liha ümber lõigati.
\par 26 Otse selsamal päeval lõigati ümber Aabraham ja tema poeg Ismael;
\par 27 ja kõik ta pere mehed, peres sündinud ja võõrastelt raha eest ostetud, lõigati ümber koos temaga.

\chapter{18}

\par 1 Ja Issand ilmutas ennast temale Mamre tammikus, kui ta istus telgi ukse ees kõige palavamal päevaajal.
\par 2 Ta tõstis oma silmad üles ja vaatas, ja ennäe, kolm meest seisid ta ees. Ja nähes neid, tõttas ta telgi ukse juurest neile vastu ja kummardas maani
\par 3 ning ütles: „Issand, kui ma sinu silmis armu leian, siis ära mine oma sulasest mööda!
\par 4 Toodagu nüüd pisut vett, peske jalgu ja nõjatuge puu alla!
\par 5 Ma toon palukese leiba, kinnitage südant, enne kui edasi lähete, kui juba kord olete oma sulase kaudu käimas!„ Ja nemad vastasid: „Tee nõnda, nagu sa oled rääkinud!”
\par 6 Ja Aabraham tõttas telki Saara juurde ning ütles: „Võta ruttu kolm mõõtu nisujahu, sõtku ja tee kooke!”
\par 7 Ja Aabraham jooksis karja juurde, võttis ühe noore ja ilusa vasika, andis poisi kätte ja see tõttas seda valmistama.
\par 8 Ja ta tõi võid, piima ja vasika, mis oli valmistatud, ja pani nende ette; ta ise seisis nende juures puu all, kui nad sõid.
\par 9 Siis nad küsisid temalt: „Kus su naine Saara on?„ Ja ta vastas: ”Seal telgis.”
\par 10 Siis üks neist ütles: „Ma tulen sinu juurde kindlasti tagasi aasta pärast samal ajal, ja vaata, su naisel Saaral saab olema poeg!” Ja Saara kuulis seda tema selja taga oleva telgi ukse juures.
\par 11 Aga Aabraham ja Saara olid vanad ja elatanud; Saaral oli lakanud olemast ka see, mis muidu naistele on omane.
\par 12 Ja Saara naeris iseeneses ja mõtles: „Nüüd, kui ma olen vanaks jäänud, peaks mul veel himu olema! Ja ka mu isand on vana.”
\par 13 Aga Issand ütles Aabrahamile: „Miks Saara naerab ja ütleb: Kas ma tõesti peaksin sünnitama, kuna ma ju olen vana?
\par 14 Kas peaks Issandal midagi olema võimatu? Ma tulen su juurde tagasi aasta pärast samal ajal, ja Saaral saab olema poeg!”
\par 15 Kuid Saara salgas, öeldes: „Mina ei naernud.„ Sest ta kartis. Tema aga ütles: ”Sa naersid küll!”
\par 16 Siis mehed tõusid sealt üles ja vaatasid alla Soodoma poole; ja Aabraham läks koos nendega, neid saatma.
\par 17 Ja Issand ütles: „Kas peaksin varjama Aabrahami eest, mida tahan teha?
\par 18 Aabraham saab ometi suureks ja vägevaks rahvaks ja tema kaudu õnnistatakse kõiki maailma rahvaid.
\par 19 Sest ma tean temast, et ta käsib oma poegi ja järeltulevat sugu hoida Issanda teed ning teha, mis õige ja kohus, et Issand võiks anda Aabrahamile, mis ta temale on tõotanud.”
\par 20 Siis ütles Issand: „Hädakisa Soodoma ja Gomorra pärast on suur ja nende patud on väga rasked!
\par 21 Seepärast ma lähen alla ja vaatan, kas minuni jõudnud kisa kohaselt on nad teinud kõike seda või mitte. Ma tahan seda teada!”
\par 22 Ja mehed pöördusid sealt ära ja läksid Soodomasse, aga Aabraham jäi veel seisma Issanda ette.
\par 23 Ja Aabraham astus ligi ning ütles: „Kas tahad tõesti hävitada õige koos õelaga?
\par 24 Vahest on linnas viiskümmend õiget? Kas tahad siis need hävitada ega taha paigale andeks anda nende viiekümne õige pärast, kes seal on?
\par 25 Jäägu sinust kaugele see tegu, et tapad õige koos õelaga, et õigel käib käsi nagu õelalgi! Jäägu see sinu poolt tegemata! Kas kogu maailma kohtumõistja ei peaks tegema õigust?”
\par 26 Ja Issand ütles: „Kui ma Soodoma linnast leian viiskümmend õiget, siis annan nende pärast andeks kogu paigale.”
\par 27 Aga Aabraham kostis ning ütles: „Vaata, ma olen nõuks võtnud siiski Issandaga rääkida, kuigi olen põrm ja tuhk.
\par 28 Vahest puudub viiekümnest õigest viis? Kas tahad siis nende viie pärast hävitada kogu linna?„ Ja tema vastas: „Ma ei hävita, kui leian sealt nelikümmend viis.”
\par 29 Ja ta jätkas veelgi kõnelust temaga ning ütles: „Vahest leidub seal nelikümmend?„ Ja tema vastas: ”Ma ei tee seda neljakümne pärast.”
\par 30 Aga ta ütles: „Ärgu süttigu põlema Issanda viha, et ma veel räägin! Vahest leidub seal kolmkümmend?„ Ja tema vastas: ”Ma ei tee seda, kui leian sealt kolmkümmend.”
\par 31 Siis ta ütles: „Vaata, ma olen nõuks võtnud siiski Issandaga rääkida. Vahest leidub seal kakskümmend?„ Ja tema vastas: ”Ma ei hävita kahekümne pärast.”
\par 32 Aga ta ütles: „Ärgu süttigu põlema Issanda viha, et ma veel üksainus kord räägin! Vahest leidub seal kümme?„ Ja tema vastas: ”Ma ei hävita kümne pärast.”
\par 33 Ja Issand läks ära, kui oli lõpetanud kõneluse Aabrahamiga; ja Aabraham läks koju.

\chapter{19}

\par 1 Ja need kaks inglit jõudsid õhtul Soodomasse. Lott istus parajasti Soodoma väravas. Kui Lott neid nägi, siis ta tõusis neile vastu, kummardas silmili maha
\par 2 ja ütles: „Ennäe mu isandaid! Astuge ometi oma sulase kotta, jääge öömajale ja peske oma jalad! Hommikul võite vara üles tõusta ja oma teekonda jätkata!„ Kuid nad vastasid: ”Ei, me jääme ööseks välja.”
\par 3 Aga ta käis neile väga peale. Siis nad põikasid tema juurde ning tulid ta kotta. Ja ta valmistas neile pidusöögi, küpsetas hapnemata leibu ja nad sõid.
\par 4 Nad ei olnud veel magama heitnud, kui linna mehed, Soodoma mehed, niihästi noored kui vanad, kogu rahvas viimseni, ümbritsesid koja
\par 5 ja hüüdsid Lotti ning ütlesid temale: „Kus on mehed, kes öösel tulid sinu juurde? Too nad välja meie kätte, et saaksime neid katsuda!”
\par 6 Siis Lott läks välja nende juurde ukse ette, sulges ukse enese järel
\par 7 ja ütles: „Mu vennad, ärge ometi tehke kurja!
\par 8 Vaadake, mul on kaks tütart, kes mehest veel midagi ei tea. Ma toon need välja teie kätte ja talitage nendega, nagu teie silmis hea on. Neile meestele ärge tehke midagi, sest nad on tulnud varjule minu katuse alla!”
\par 9 Kuid nad vastasid: „Käi minema!„ Ja nad ütlesid: ”Ise on tulnud võõrana elama, aga tahab olla kohtumõistjaks! Nüüd me kohtleme sind halvemini kui neid!” Ja nad kippusid väga mehe, Loti kallale ning asusid ust maha murdma.
\par 10 Siis mehed sirutasid käed välja, tõmbasid Loti enese juurde kotta ja sulgesid ukse.
\par 11 Aga mehi, kes koja ukse taga olid, lõid nad pimestusega, niihästi väikesi kui suuri, nõnda et need väsisid ust otsimast.
\par 12 Ja mehed ütlesid Lotile: „Kes sul veel siin on, väimees ja pojad ja tütred ja kõik, kes sul linnas on, vii siit paigast ära,
\par 13 sest me hävitame selle paiga, kuna hädakisa nende pärast on Issanda ees suur; ja Issand on meid läkitanud seda hävitama!”
\par 14 Siis Lott läks välja ja kõneles oma väimeestega, kes pidid võtma ta tütred, ning ütles: „Tõuske, minge siit paigast ära, sest Issand hävitab selle linna!” Aga oma väimeeste meelest heitis ta nalja.
\par 15 Ja kui hakkas koitma, kiirustasid inglid Lotti, öeldes: „Tõuse, võta oma naine ja kaks tütart, kes siin on, et sa ei hukkuks linna süü pärast!”
\par 16 Ja kui ta veel kõhkles, siis haarasid mehed kinni ta käest, ja ta naise käest ja ta mõlema tütre käest, sest Issand tahtis teda säästa; ja nad viisid ta ära ning jätsid väljapoole linna.
\par 17 Ja neid välja viies nad ütlesid: „Päästa oma hing! Ära vaata selja taha ja ära peatu mitte kuskil ümbruskonnas! Põgene mäestikku, et sa ei hukkuks!”
\par 18 Ja Lott vastas neile: „Muidugi mitte, mu isand!
\par 19 Vaata, su sulane on küll sinu silmis armu leidnud ja suur on su heldus, mida sa oled mulle osutanud, jättes mu hinge elama. Aga ma ei suuda põgeneda mäestikku, et õnnetus mind ei tabaks ja ma ei sureks.
\par 20 Vaata, see linn taamal on nii ligidal, et sinna võiks põgeneda, ja on pisitilluke. Kui ma ometi sinna pääseksin - eks ole, see on ju pisitilluke? -, siis võiks mu hing jääda elama!”
\par 21 Ja ta vastas temale: „Vaata, ma võtan sind kuulda ka selles asjas ega paiska segi linna, millest sa räägid.
\par 22 Tõtta, põgene sinna, sest ma ei või midagi teha enne, kui sa sinna oled jõudnud!” Seepärast pandi sellele linnale nimeks Soar.
\par 23 Päike oli just tõusnud maa kohale, kui Lott jõudis Soari.
\par 24 Ja Issand laskis sadada Soodoma ja Gomorra peale väävlit ja tuld Issanda juurest taevast
\par 25 ning hävitas need linnad ja kogu ümbruskonna, kõik linnade elanikud ja maa taimestiku.
\par 26 Aga Loti naine, kes ta järel käis, vaatas tagasi ja muutus soolasambaks.
\par 27 Ja Aabraham läks hommikul vara paika, kus ta Issanda ees oli seisnud,
\par 28 heitis pilgu alla Soodoma ja Gomorra poole ja kogu ümbruskonna maapinnale, vaatas, ja ennäe, maast tõusis suits otsekui sulatusahju suits.
\par 29 Kui Jumal oli segi paisanud ümbruskonna linnad, mõtles Jumal Aabrahamile ja saatis Loti ära selle hävituse keskelt, millega ta segi paiskas need linnad, kus Lott oli elanud.
\par 30 Ja Lott läks Soarist üles ning elas mäestikus koos oma kahe tütrega, sest ta kartis elada Soaris; ta elas ühes koopas, tema ise ja ta mõlemad tütred.
\par 31 Kord ütles vanem nooremale: „Meie isa on vana ja maal pole ühtegi meest, kes tuleks Meie juurde kogu maailma kombe järgi.
\par 32 Tule, joodame oma isa veiniga ja magame temaga, et saaksime oma isalt järeltulija!”
\par 33 Ja nad jootsid oma isa sel ööl veiniga ja vanem läks ning magas oma isaga; see ei märganudki, millal too maha heitis või üles tõusis.
\par 34 Ja järgmisel päeval ütles vanem nooremale: „Vaata, ma magasin eile oma isaga. Joodame ka sel ööl teda veiniga, siis mine maga sina temaga, et saaksime oma isalt järeltulija!”
\par 35 Ja nad jootsid ka sel ööl oma isa veiniga ja noorem läks ning magas temaga; see ei märganudki, millal too maha heitis või üles tõusis.
\par 36 Ja Loti mõlemad tütred jäid oma isast lapseootele.
\par 37 Ja vanem sünnitas poja ning pani temale nimeks Moab; see on moabide isa tänapäevani.
\par 38 Ja noorem sünnitas ka poja ning pani temale nimeks Ben-Ammi; see on ammonlaste isa tänapäevani.

\chapter{20}

\par 1 Ja Aabraham siirdus sealt Lõunamaale ning asus elama Kaadesi ja Suuri vahele; ta elas Geraris võõrana.
\par 2 Ja Aabraham ütles oma naise Saara kohta: „Ta on minu õde.” Siis Gerari kuningas Abimelek läkitas järele ja võttis Saara.
\par 3 Aga Jumal tuli Abimeleki juurde öösel unes ja ütles temale: „Vaata, sa pead surema naise pärast, kelle sa enesele võtsid, sest ta on abielunaine!”
\par 4 Abimelek aga ei olnud temasse puutunud ja vastas: „Issand, kas tahad surmata ka õiget rahvast?
\par 5 Eks ta öelnud mulle: Ta on minu õde? Ja tema ütles ka ise: Ta on mu vend. Ma tegin seda vaga südame ja süütute kätega.”
\par 6 Siis ütles Jumal temale unes: „Minagi tean, et sa tegid seda vaga südamega ja ma hoidsin sind ka minu vastu pattu tegemast: sellepärast ma ei lasknud sind temasse puutuda.
\par 7 Ja nüüd anna mehele naine tagasi, sest ta on prohvet ja ta palvetab sinu pärast, et sa jääksid elama. Aga kui sa tagasi ei anna, siis tea, et sina ja kõik, kes sul on, peate surema!”
\par 8 Ja Abimelek tõusis hommikul vara, kutsus kõik oma sulased ning rääkis kõik need sõnad nende kuuldes; ja mehed kartsid väga.
\par 9 Siis Abimelek kutsus Aabrahami ja ütles temale: „Mis sa meile tegid! Millega ma sinu vastu pattu tegin, et sa tõid suure süü minu ja mu kuningriigi peale? Sa oled minuga teinud sündmatuid tegusid!”
\par 10 Ja Abimelek küsis Aabrahamilt: „Mida sa mõtlesid seda asja tehes?”
\par 11 Ja Aabraham vastas: „Mina mõtlesin ainult, et selles paigas ei ole jumalakartust ja mind tapetakse mu naise pärast.
\par 12 Ja tema ongi tõepoolest mu õde: mu isa tütar, kuigi mitte mu ema tütar; seetõttu ta sai mu naiseks.
\par 13 Aga kui Jumal saatis mind isamajast rändama, ütlesin ma temale: Osuta mulle seda armastust, et sa kõigis paigus, kuhu tuleme, minu kohta ütled: Ta on mu vend.”
\par 14 Siis Abimelek võttis lambaid, kitsi ja veiseid, sulaseid ja teenijaid ja andis Aabrahamile; ja ta andis temale tagasi Saara, ta naise.
\par 15 Ja Abimelek ütles: „Vaata, mu maa on lahti sinu ees, ela, kus sulle meeldib!”
\par 16 Ja Saarale ta ütles: „Näe, ma annan su vennale tuhat hõbetükki. Vaata, see olgu sulle hüvituseks kõigi ees, kes su juures on. Sa oled kõiges õigeks osutunud.”
\par 17 Ja Aabraham palus Jumalat ja Jumal tegi terveks Abimeleki, ta naise ja ta teenijad, nõnda et nad said lapsi,
\par 18 sest Issand oli sulgenud kõvasti kõik emaihud Abimeleki kojas Saara, Aabrahami naise pärast.

\chapter{21}

\par 1 Ja Issand hoolitses Saara eest, nõnda nagu ta oli lubanud. Issand toimis Saaraga, nõnda nagu ta oli öelnud:
\par 2 Saara jäi lapseootele ja tõi Aabrahamile poja ilmale ta vanas eas, määratud ajal, millest Jumal temaga oli rääkinud.
\par 3 Ja Aabraham pani oma pojale, kes temale sündis, kelle Saara temale ilmale tõi, Iisak nimeks.
\par 4 Ja Aabraham lõikas ümber oma poja Iisaki, kui see kaheksapäevane oli, nõnda nagu Jumal teda oli käskinud.
\par 5 Ja Aabraham oli sada aastat vana, kui ta poeg Iisak temale sündis.
\par 6 Aga Saara ütles: „Jumal pani mind naerma! Igaüks, kes sellest kuuleb, naerab mind!”
\par 7 Ja ta ütles: „Kes oleks võinud Aabrahamile kuulutada, et Saara imetab veel lapsi? Ometi tõin ma temale poja ilmale ta vanas eas!”
\par 8 Laps kasvas ja võõrutati; ja Aabraham tegi suure peo Iisaki võõrutamispäeval.
\par 9 Kui Saara nägi mängimas egiptlanna Haagari poega, kelle see Aabrahamile oli ilmale toonud,
\par 10 siis ta ütles Aabrahamile: „Kihuta minema see teenija ja tema poeg, sest selle teenija poeg ei või pärida koos minu poja Iisakiga!”
\par 11 See kõne aga oli Aabrahami silmis ta poja kohta väga paha.
\par 12 Aga Jumal ütles Aabrahamile: „Ärgu olgu see su silmis paha ei poisi ega su teenija kohta. Kõiges, mis Saara sulle ütleb, kuula ta sõna, sest Iisakist loetakse sinu sugu!
\par 13 Aga ka teenija poja teen ma rahvaks, sest ta on ju sinu järeltulija.”
\par 14 Ja Aabraham tõusis hommikul vara, võttis leiva ja veelähkri ning andis Haagarile, pannes need temale selga, samuti lapse ja saatis ta minema. Ja tema läks ning eksles Beer-Seba kõrbes.
\par 15 Kui vesi lähkrist lõppes, siis ta jättis lapse ühe põõsa alla
\par 16 ja läks ning istus temaga kohastikku, ammulaske kauguses, sest ta ütles: „Ma ei või näha lapse surma!” Nõnda ta istus temaga kohastikku, tõstis häält ja nuttis.
\par 17 Aga Jumal kuulis poisi häält ja Jumala ingel hüüdis taevast Haagarit ning ütles temale: „Mis sul viga on, Haagar? Ära karda, sest Jumal on kuulnud poisi häält seal, kus ta on.
\par 18 Tõuse, tõsta poiss üles ja võta ta käekõrvale, sest ma tahan temast teha suure rahva!”
\par 19 Ja Jumal tegi ta silmad lahti, nõnda et ta nägi ühte veekaevu; ta läks ning täitis lähkri veega ja andis poisile juua.
\par 20 Ja Jumal oli poisiga; ta kasvas ja elas kõrbes, ja temast sai ammukütt.
\par 21 Ta elas Paarani kõrbes; ja ta ema võttis temale Egiptusemaalt naise.
\par 22 Sel ajal rääkisid Abimelek ja tema väepealik Piikol Aabrahamiga, öeldes: „Jumal on sinuga kõiges, mis sa teed.
\par 23 Ja nüüd vannu mulle siin Jumala juures, et sa ei peta mind ega mu lapsi ja lapselapsi! Heateo pärast, mis ma sulle tegin, tee seda mulle ja maale, kus sa võõrana elad!”
\par 24 Ja Aabraham ütles: „Ma vannun!”
\par 25 Aabraham aga noomis Abimelekit veekaevu pärast, mille Abimeleki sulased väevõimuga olid ära võtnud.
\par 26 Kuid Abimelek vastas: „Mina ei tea, kes seda tegi. Sina pole mulle sellest teatanud ja mina ise pole ka midagi kuulnud enne kui täna.”
\par 27 Siis Aabraham võttis lambaid, kitsi ja veiseid ja andis Abimelekile; ja nad mõlemad sõlmisid lepingu.
\par 28 Kui Aabraham pani seitse lambatalle eraldi,
\par 29 küsis Abimelek Aabrahamilt: „Milleks on siin need seitse talle, keda sa oled eraldi pannud?”
\par 30 Ja tema vastas: „Sa pead need seitse talle minu käest võtma tõendina minu poolt, et mina olen kaevanud selle kaevu!”
\par 31 Seepärast hüütakse seda paika Beer-Sebaks, sest nad mõlemad andsid seal vande.
\par 32 Nõnda sõlmisid nad Beer-Sebas lepingu. Siis Abimelek ja tema väepealik Piikol tõusid ja läksid tagasi vilistite maale.
\par 33 Aabraham aga istutas Beer-Sebasse ühe tamariskipuu ja hüüdis seal appi Issanda, igavese Jumala nime.
\par 34 Ja Aabraham elas võõrana vilistite maal kaua aega.

\chapter{22}

\par 1 Pärast neid sündmusi pani Jumal Aabrahami proovile ning ütles temale: „Aabraham!„ Ja ta vastas: ”Siin ma olen!”
\par 2 Ja tema ütles: „Võta nüüd Iisak, oma ainus poeg, keda sa armastad, ja mine Morijamaale ning ohverda ta seal põletusohvriks ühel neist mägedest, mis ma sulle nimetan!”
\par 3 Ja Aabraham tõusis hommikul vara, saduldas oma eesli, võttis enesega kaasa kaks noort meest ja oma poja Iisaki, lõhkus põletusohvri puud, seadis minekule ja läks paika, millest Jumal temale oli rääkinud.
\par 4 Kolmandal päeval tõstis Aabraham oma silmad üles ja nägi seda paika kaugelt.
\par 5 Ja Aabraham ütles oma noortele meestele: „Jääge teie eesliga siia! Mina ja poiss läheme sinna, kummardame ja tuleme siis tagasi teie juurde.”
\par 6 Ja Aabraham võttis põletusohvri puud, pani need oma pojale Iisakile õlale, võttis enda kätte tule ja noa ning mõlemad läksid üheskoos.
\par 7 Ja Iisak rääkis oma isa Aabrahamiga ning ütles: „isa!„ Ja tema vastas: „Siin ma olen, mu poeg!” Siis ta ütles: ”Näe, siin on tuli ja puud, aga kus on ohvritall?”
\par 8 Ja Aabraham vastas: „Küllap Jumal vaatab enesele ohvritalle, mu poeg!” Nõnda läksid mõlemad üheskoos.
\par 9 Ja kui nad jõudsid paika, millest Jumal temale oli rääkinud, ehitas Aabraham sinna altari, ladus puud, sidus kinni oma poja Iisaki ja pani ta altarile puude peale.
\par 10 Ja Aabraham sirutas käe ja võttis noa, et tappa oma poeg.
\par 11 Aga Issanda ingel hüüdis teda taevast ja ütles: „Aabraham, Aabraham!„ Ja tema vastas: ”Siin ma olen!”
\par 12 Siis ta ütles: „Ära pane kätt poisi külge ja ära tee temale midagi, sest nüüd ma tean, et sa kardad Jumalat ega keela mulle oma ainsat poega!”
\par 13 Ja Aabraham tõstis oma silmad üles, vaatas, ja ennäe, üks jäär oli rägastikus sarvipidi kinni. Ja Aabraham läks ning võttis jäära ja ohverdas selle põletusohvriks oma poja asemel.
\par 14 Ja Aabraham pani sellele paigale nimeks „Issand näeb„. Seepärast öeldakse tänapäevalgi: ”Issanda mäel ta näitab end.”
\par 15 Ja Issanda ingel hüüdis Aabrahami teist korda taevast
\par 16 ning ütles temale: „Ma vannun iseenese juures, ütleb Issand: sellepärast et sa seda tegid ega keelanud mulle oma ainsat poega,
\par 17 ma õnnistan sind tõesti ja teen su soo väga paljuks - nagu tähti taevas ja nagu liiva mere ääres - ja su sugu vallutab oma vaenlaste väravad!
\par 18 Ja sinu soo nimel õnnistavad endid kõik maailma rahvad, sellepärast et sa võtsid kuulda mu häält!”
\par 19 Siis Aabraham läks tagasi oma noorte meeste juurde, ja nad tõusid ning läksid üheskoos Beer-Sebasse. Ja Aabraham jäi elama Beer-Sebasse.
\par 20 Ja pärast neid sündmusi teatati Aabrahamile ja öeldi: „Vaata, Milka on ka su vennale Naahorile poegi ilmale toonud:
\par 21 Uusi, tema esmasündinu, Buusi, selle venna, Kemueli, Arami isa,
\par 22 Kesedi, Haso, Pildase, Jidlafi ja Betueli.”
\par 23 Ja Betuelile sündis Rebeka. Need kaheksa tõi Milka ilmale Naahorile, Aabrahami vennale.
\par 24 Ja tema liignaine, Reuma nimi, sünnitas ka - Teba, Gahami, Tahase ja Maaka.

\chapter{23}

\par 1 Ja Saara elas sada kakskümmend seitse aastat; nii palju oli Saaral eluaastaid.
\par 2 Saara suri Kirjat-Arbas, see on Hebronis, Kaananimaal; ja Aabraham tuli Saarat leinama ja taga nutma.
\par 3 Siis Aabraham tõusis üles oma surnu juurest ja rääkis hettidega, öeldes:
\par 4 „Mina olen teie juures võõras ja majaline. Andke mulle eneste juurde pärandhaud, et saaksin matta surnu oma silma eest.”
\par 5 Ja hetid vastasid Aabrahamile, öeldes temale:
\par 6 „Kuule meid, isand! Sina oled Jumala vürst meie keskel. Mata oma surnu meie kõige paremasse hauda! Ükski meist ei keela sulle oma hauda su surnu matmiseks.”
\par 7 Aga Aabraham tõusis ning kummardas maa rahva, hettide ees
\par 8 ja rääkis nendega, öeldes: „Kui see on teile meeltmööda, et ma oma surnu oma silma eest ära matan, siis kuulge mind ja mõjustage minu kasuks Efronit, Sohari poega,
\par 9 et ta mulle annaks temale kuuluva Makpela koopa, mis on ta välja servas. Ta andku see mulle täie hinna eest pärushauaks teie keskel!”
\par 10 Aga Efron istus hettide hulgas; ja hett Efron vastas Aabrahamile hettide kuuldes, kõigi juuresolekul, kes ta linna väravast läbi käisid, ja ütles:
\par 11 „Ei, mu isand! Kuule mind! Ma annan sulle välja, samuti ma annan sulle koopa, mis sellel asub; ma annan selle sulle oma rahvuskaaslaste silma all. Mata oma surnu!”
\par 12 Kuid Aabraham kummardas maa rahva ees,
\par 13 rääkis Efroniga maa rahva kuuldes ja ütles: „Kui sa ometi ise mind kuulda võtaksid! Mina maksan välja eest raha, võta see minult, et saaksin oma surnu sinna matta!”
\par 14 Siis Efron vastas Aabrahamile, öeldes temale:
\par 15 „Mu isand, kuule mind! Maa väärtus on nelisada hõbeseeklit - mis tähtsus sellel ongi minu ja sinu vahekorra juures. Mata aga oma surnu!”
\par 16 Ja Aabraham kuulas Efronit; Aabraham vaagis Efronile raha, mida see hettide kuuldes oli nimetanud, nelisada hõbeseeklit, kaubanduses käibel olevaid.
\par 17 Nõnda said Efroni väli Makpelas Mamre kohal, väli ja selles olev koobas ja kõik väljal olevad puud, mis olid ümber kogu selle maa-ala,
\par 18 Aabrahami omandiks hettide nähes, kõigi juuresolekul, kes linna väravast läbi käisid.
\par 19 Ja seejärel Aabraham mattis oma naise Saara Makpela välja koopasse Mamre kohal, see on Hebronis Kaananimaal.
\par 20 Nõnda said väli ja selles olev koobas hettidelt Aabrahamile pärushauaks.

\chapter{24}

\par 1 Kui Aabraham oli vana ja elatanud ning Issand oli Aabrahami kõigiti õnnistanud,
\par 2 siis ütles Aabraham vanimale sulasele oma peres, kes valitses kõige üle, mis tal oli: „Pane nüüd oma käsi mu puusa alla!
\par 3 Mina vannutan sind Issanda, taeva Jumala ja maa Jumala juures, et sa mu pojale ei võta naist kaananlaste tütreist, kelle keskel ma elan,
\par 4 vaid et sa lähed minu maale ja mu sugulaste juurde ning võtad sealt naise mu pojale Iisakile!”
\par 5 Aga sulane ütles temale: „Võib-olla naine ei taha mulle järgneda siia maale. Kas ma siis tõesti pean su poja viima tagasi maale, kust sa oled ära tulnud?”
\par 6 Siis ütles Aabraham temale: „Hoia, et sa ei vii mu poega sinna tagasi!
\par 7 Issand, taeva Jumal, kes minu võttis mu isakojast ja mu sünnimaalt ja kes mulle rääkis ning vandus, öeldes: Sinu soole ma annan selle maa! - tema ise läkitab oma ingli sinu ees, et saaksid sealt mu pojale naise võtta.
\par 8 Aga kui naine ei taha sulle järgneda, siis oled sa sellest mu vandest vaba. Ainult ära vii mu poega sinna tagasi!”
\par 9 Siis sulane pani käe oma isanda Aabrahami puusa alla ja vandus temale selle kõne kohaselt.
\par 10 Ja sulane võttis oma isanda kaamelitest kümme kaamelit ning läks, ja tal oli oma isandalt kaasas kõiksugu kalleid asju; ta võttis kätte ja läks Mesopotaamiasse Naahori linna.
\par 11 Seal laskis ta õhtul kaamelid põlvili heita väljaspool linna veekaevu juures sel ajal, kui veeviijad välja tulid,
\par 12 ning ütles: „Issand, minu isanda Aabrahami Jumal, lase ometi see mul täna korda minna ja tee head mu isandale Aabrahamile!
\par 13 Vaata, ma seisan veeallika juures ja linnaelanike tütred tulevad vett viima.
\par 14 Sündigu siis, et tütarlaps, kellele ma ütlen: Kalluta oma kruusi, et ma saaksin juua! ja kes vastab: Joo, ja ma joodan ka su kaameleid! - on see, kelle sa oled määranud oma sulasele Iisakile. Sellest ma siis tean, et sa mu isandale oled head teinud.”
\par 15 Ja veel enne kui ta rääkimise oli lõpetanud, vaata, siis tuli välja Rebeka, kes oli sündinud Betuelile, Aabrahami venna Naahori naise Milka pojale; ja tal oli kruus õlal.
\par 16 Ja tütarlaps oli väga ilusa välimusega, alles neitsi ja mehe poolt puutumata. Ta läks alla allika juurde, täitis kruusi ja tuli üles.
\par 17 Siis jooksis sulane temale vastu ning ütles: „Anna mulle oma kruusist pisut vett rüübata!”
\par 18 Tema vastas: „Joo, mu isand!” Ja ta tõstis kähku kruusi alla oma käele ning andis temale juua.
\par 19 Ja olles temale juua andnud, ütles ta: „Ma ammutan ka su kaamelitele, kuni needki on joonud.”
\par 20 Ja ta tühjendas kähku oma kruusi künasse ning jooksis jälle kaevule vett ammutama ja ammutas kõigile ta kaamelitele.
\par 21 Mees aga silmitses teda vaikides, et mõista, kas Issand oli ta teekonna lasknud korda minna või mitte.
\par 22 Ja kui kaamelid olid joonud, võttis mees kuldrõnga, pooleseeklilise, ja kaks käevõru ta käte jaoks, kümme kuldseeklit väärt,
\par 23 ning küsis: „Kelle tütar sa oled? Ütle ometi mulle! On su isa kojas meile ööbimispaika?”
\par 24 Ja ta vastas temale: „Mina olen Betueli, Naahori ja Milka poja tütar.”
\par 25 Ta ütles temale veel: „Niihästi õlgi kui muud loomasööta on meil küllalt, ööbimispaikki on olemas.”
\par 26 Siis mees kummardas ja heitis Issanda ette
\par 27 ning ütles: „Kiidetud olgu Issand, mu isanda Aabrahami Jumal, kes mu isandale ei ole keelanud oma heldust ja tõde! Issand on mind teekonnal juhtinud mu isanda vendade kotta!”
\par 28 Aga tütarlaps jooksis ning teatas oma ema perele, mis oli sündinud.
\par 29 Ja Rebekal oli vend, Laaban nimi; ja Laaban jooksis välja, mehe juurde allikale.
\par 30 Sest kui ta nägi rõngast ja käevõrusid oma õe kätel, ja kui ta oli kuulnud oma õe Rebeka sõnu, kes ütles: „Nõnda rääkis mulle see mees,” - siis ta läks selle mehe juurde, ja ennäe, see seisis kaamelite juures allikal.
\par 31 Ja ta ütles: „Tule sisse, Issanda õnnistatu! Miks sa väljas seisad? Ma olen korda seadnud koja ja kaamelite paiga.”
\par 32 Mees tuli siis kotta ja Laaban päästis kaamelid lahti, andis õlgi ja toitu kaamelitele ning tõi vett tema ja temaga kaasas olevate meeste jalgade pesemiseks.
\par 33 Siis pandi temale rooga ette. Aga ta ütles: „Ma ei söö enne, kui ma oma asja olen rääkinud.„ Ja talle vastati: ”Räägi!”
\par 34 Siis ta ütles: „Mina olen Aabrahami sulane.
\par 35 Issand on mu isandat väga õnnistanud, nõnda et ta on jõukaks saanud: ta on temale andnud lambaid, kitsi ja veiseid, hõbedat ja kulda, sulaseid ja teenijaid, kaameleid ja eesleid.
\par 36 Ja Saara, mu isanda naine, on vanas eas mu isandale poja ilmale toonud, ja sellele on ta andnud kõik, mis tal on.
\par 37 Ja mu isand vannutas mind, öeldes: Sa ei tohi võtta mu pojale naist kaananlaste tütarde seast, kelle maal ma elan,
\par 38 vaid sa pead minema mu isakotta ja mu suguvõsa juurde ning sealt võtma naise mu pojale!
\par 39 Aga mina ütlesin oma isandale: Võib-olla naine ei tule minuga?
\par 40 Siis ta vastas mulle: Issand, kelle palge ees ma olen käinud, läkitab oma ingli sinuga ja laseb su teekonna korda minna, et saad mu pojale naise võtta minu suguvõsast ja minu isakojast.
\par 41 Mu vandest sa vabaned sel juhul, kui sa tuled mu suguvõsa juurde, aga nemad ei anna sulle, siis oled mu vandest vaba.
\par 42 Ma jõudsin täna allika juurde ja ütlesin: Issand, mu isanda Aabrahami Jumal, kui sa nüüd tahad korda saata mu teekonna, mida ma käin,
\par 43 siis vaata, ma seisan veeallika juures. Kui üks neitsi tuleb vett viima ja ma temale ütlen: Anna mulle oma kruusist pisut vett juua!
\par 44 ja kui tema mulle vastab: Joo ise, ja ma ammutan ka su kaamelitele!, siis on tema see naine, kelle Issand on määranud mu isanda pojale.
\par 45 Veel enne kui olin kõneluse iseenesega lõpetanud, vaata, siis tuli Rebeka, kruus õlal, ja läks alla allika juurde ning ammutas vett. Ja ma ütlesin temale: Anna mulle juua!
\par 46 Siis ta tõstis kähku kruusi õlalt alla ning ütles: Joo, ja ma joodan ka su kaameleid! Ja mina jõin ning tema jootis mu kaameleid.
\par 47 Ja ma küsisin temalt ning ütlesin: Kelle tütar sa oled? Ja tema vastas: Betueli, Naahori ja Milka poja tütar. Siis ma panin temale rõnga ninasse ja käevõrud kätele,
\par 48 kummardasin ning heitsin Issanda ette ja kiitsin Issandat, oma isanda Aabrahami Jumalat, kes mind oli juhatanud õigele teele, et saaksin võtta oma isanda vennatütre ta pojale.
\par 49 Ja kui te nüüd tahate osutada heldust ja truudust mu isandale, siis öelge mulle. Aga kui mitte, öelge sedagi mulle, ja ma pöördun siis kas paremat või vasakut kätt!”
\par 50 Seepeale vastasid Laaban ja Betuel ning ütlesid: „Issandalt on see asi alguse saanud. Me ei või sulle sõnagi lausuda, ei halba ega head.
\par 51 Vaata, Rebeka on su ees, võta tema ja mine! Saagu ta naiseks su isanda pojale, nõnda nagu Issand on öelnud!”
\par 52 Kui Aabrahami sulane kuulis nende sõnu, siis ta kummardas maani Issanda ees.
\par 53 Ja sulane võttis välja hõbe- ja kuldriistad ja riided ning andis Rebekale; ka tema vennale ja emale andis ta kalleid asju.
\par 54 Siis nad sõid ja jõid, tema ja mehed, kes koos temaga olid, ja nad ööbisid seal. Aga hommikul, kui nad olid tõusnud, ütles ta: „Saatke mind nüüd mu isanda juurde!”
\par 55 Aga tütarlapse vend ja ema vastasid: „Jäägu tütarlaps veel mõneks ajaks, kas või kümneks päevakski meie juurde. Siis sa võid minna.”
\par 56 Kuid tema ütles neile: „Ärge mind viivitage, sest Issand on lasknud mu teekonna korda minna. Saatke mind teele, et saaksin minna oma isanda juurde!”
\par 57 Siis nad ütlesid: „Me kutsume tütarlapse ja küsime tema suust.”
\par 58 Ja nad kutsusid Rebeka ning küsisid temalt: „Kas tahad minna koos selle mehega?„ Ja ta vastas: ”Ma lähen!”
\par 59 Siis nad saatsid ära oma õe Rebeka ja tema imetaja, ja Aabrahami sulase ja tema mehed.
\par 60 Ja nad õnnistasid Rebekat ning ütlesid temale: „Õeke, sinust tulgu tuhat korda kümme tuhat, ja sinu sugu vallutagu oma vihameeste väravad!”
\par 61 Siis Rebeka ja tema tüdrukud tõusid, istusid kaamelite selga ning järgnesid mehele. Nõnda võttis sulane Rebeka ja läks.
\par 62 Iisak aga oli tulemas Lahhai-Roi kaevu poolt, sest ta elas Lõunamaal.
\par 63 Ja Iisak oli vastu õhtut läinud väljale mõtisklema; ja kui ta oma silmad üles tõstis ja vaatas, ennäe, siis tulid kaamelid.
\par 64 Kui Rebeka oma silmad üles tõstis ja nägi Iisakit, siis ta laskus kaameli seljast
\par 65 ning küsis sulaselt: „Kes on see mees, kes meile väljal vastu tuleb?„ Ja sulane vastas: ”See on mu isand!” Siis Rebeka võttis loori ja kattis ennast.
\par 66 Ja sulane jutustas Iisakile kõigest, mis ta oli teinud.
\par 67 Ja Iisak viis Rebeka oma ema Saara telki; ja ta võttis Rebeka, see sai tema naiseks ja ta armastas teda. Nõnda leidis Iisak troosti pärast oma ema surma.

\chapter{25}

\par 1 Ja Aabraham võttis taas naise, nimega Ketuura.
\par 2 Ja see tõi temale ilmale Simrani, Joksani, Medani, Midjani, Jisbaki ja Suuahi.
\par 3 Ja Joksanile sündisid Seeba ja Dedan; ja Dedani järeltulijad olid assüürlased, letuuslased ja leumlased.
\par 4 Ja Midjani pojad olid Eefa, Eefer, Hanok, Abiida ja Eldaa. Need kõik olid Ketuura järeltulijad.
\par 5 Ja Aabraham andis kõik, mis tal oli, Iisakile.
\par 6 Aga liignaiste poegadele, kes Aabrahamil olid, andis Aabraham ande ja saatis nad veel oma eluajal oma poja Iisaki juurest ära hommiku poole, Hommikumaale.
\par 7 Ja Aabrahami eluea aastaid, mis ta elas, oli sada seitsekümmend viis aastat.
\par 8 Ja Aabraham heitis hinge ning suri heas vanuses, vana ning elatanud, ja ta koristati oma rahva juurde.
\par 9 Ja ta pojad Iisak ja Ismael matsid tema Makpela koopasse, hett Sohari poja Efroni väljal, mis on Mamre kohal,
\par 10 väljale, mille Aabraham hettidelt oli ostnud, maeti Aabraham ja tema naine Saara.
\par 11 Ja pärast Aabrahami surma õnnistas Jumal ta poega Iisakit. Ja Iisak elas Lahhai-Roi kaevu juures.
\par 12 Ja need olid Ismaeli, Aabrahami poja järeltulijad, keda egiptlanna Haagar, Saara teenija, Aabrahamile ilmale tõi.
\par 13 Ismaeli poegade nimed, nimetatud nende sündimise järjekorras, olid need: Nebajot, Ismaeli esmasündinu, siis Keedar, Adbeel, Mibsam,
\par 14 Misma, Duuma, Massa,
\par 15 Hadad, Teema, Jetuur, Naafis ja Keedma.
\par 16 Need olid Ismaeli pojad ja need olid nende nimed vastavalt nende asulatele ja leeridele: nende suguharude kaksteist vürsti.
\par 17 Ja need olid Ismaeli eluaastad: sada kolmkümmend seitse aastat; siis ta heitis hinge ja suri, ja ta koristati oma rahva juurde.
\par 18 Ja nad asusid Havilast kuni vastu Egiptust oleva Suurini Assuri suunas, tungides kallale kõigile oma vendadele.
\par 19 Ja need olid Iisaki, Aabrahami poja järeltulijad: Aabrahamile sündis Iisak.
\par 20 Ja Iisak oli neljakümneaastane, kui ta võttis enesele naiseks Rebeka, süürlase Betueli tütre Mesopotaamiast, süürlase Laabani õe.
\par 21 Ja Iisak palus Issandat oma naise pärast, sest see oli viljatu; ja Issand kuulis ta palvet ja ta naine Rebeka jäi lapseootele.
\par 22 Aga kui lapsed ta ihus tõuklesid, ütles ta: „Mispärast on see minuga nõnda?” Ja ta läks Issandalt küsima.
\par 23 Ja Issand vastas temale: „Su ihus on kaks rahvast, kaks erinevat hõimu saab su üsast alguse: üks rahvas on vägevam teisest - vanem orjab nooremat.”
\par 24 Ja kui tema sünnitamise aeg saabus, vaata, siis olid ta ihus kaksikud.
\par 25 See, kes sündis esimesena, oli punakas, täiesti nagu karune kuub; ja temale pandi nimeks Eesav.
\par 26 Seejärel sündis ta vend, kelle käsi hoidis kinni Eesavi kannast; ja temale pandi nimeks Jaakob. Iisak oli kuuskümmend aastat vana, kui nad sündisid.
\par 27 Ja poisid kasvasid suureks. Eesavist sai osav kütt, väljal uitaja, aga Jaakob oli vagane mees, kes elas telkides.
\par 28 Ja Iisak armastas Eesavit, sest jahisaak oli temale suupärane; aga Rebeka armastas Jaakobit.
\par 29 Kord keetis Jaakob leent, Eesav aga tuli väljalt ning oli väsinud.
\par 30 Ja Eesav ütles Jaakobile: „Anna mulle ometi süüa seda punast, seda punast leent, sest ma olen väsinud!” Sellepärast hakati teda kutsuma Edomiks.
\par 31 Aga Jaakob ütles: „Enne müü mulle oma esmasünniõigus!”
\par 32 Ja Eesav vastas: „Vaata, mina ju suren niikuinii, milleks mulle siis veel esmasünniõigus!”
\par 33 Siis ütles Jaakob: „Vannu mulle enne!” Ja ta vandus temale ning müüs oma esmasünniõiguse Jaakobile.
\par 34 Ja Jaakob andis Eesavile leiba ja läätseleent; ja tema sõi ja jõi, tõusis üles ja läks ära. Nii vähe hoolis Eesav esmasünniõigusest.

\chapter{26}

\par 1 Aga maal oli nälg pärast seda eelmist nälga, mis Aabrahami päevil oli olnud. Ja Iisak läks vilistite kuninga Abimeleki juurde Gerarisse.
\par 2 Ja Issand ilmutas ennast temale ning ütles: „Ära mine alla Egiptusesse! Jää maale, kuhu ma sind käsin!
\par 3 Ela võõrana siin maal, ja ma olen sinuga ning õnnistan sind, sest sinule ja sinu soole ma annan kõik need maad ning pean vannet, mille ma olen vandunud su isale Aabrahamile.
\par 4 Ja ma teen su soo paljuks nagu taevatähed ja annan su soole kõik need maad, ja sinu soo nimel õnnistavad endid kõik maailma rahvad,
\par 5 sellepärast et Aabraham kuulas mu sõna ja pidas, mis ma käskisin pidada - mu käske, seadlusi ja õpetusi.”
\par 6 Ja Iisak jäi elama Gerarisse.
\par 7 Kui kohalikud mehed küsisid tema naise kohta, siis ta ütles: „See on mu õde.„ Sest ta kartis öelda: „Mu naine”, mõeldes ise: ”Muidu kohalikud mehed tapavad mu Rebeka pärast, sest ta on ilusa välimusega.”
\par 8 Aga kui ta seal pikemat aega oli viibinud, vaatas Abimelek, vilistite kuningas, kord aknast välja ja nägi, et Iisak hellitas oma naist Rebekat.
\par 9 Siis Abimelek kutsus Iisaki ja ütles: „Vaata, ta on tõepoolest su naine! Kuidas sa siis võisid öelda: Ta on mu õde?„ Ja Iisak vastas temale: ”Ma mõtlesin, et muidu ma ehk pean tema pärast surema.”
\par 10 Aga Abimelek ütles: „Miks sa meile seda tegid? Kui kergesti oleks võinud keegi rahva hulgast magada su naisega ja sa oleksid meie peale toonud süü!”
\par 11 Ja Abimelek andis käsu kogu rahvale, öeldes: „Kes puudutab seda meest ja tema naist, peab surema!”
\par 12 Ja Iisak külvas seal maal ja sai sel aastal sajakordselt, sest Issand õnnistas teda.
\par 13 Ja mees läks rikkaks, läks üha rikkamaks, kuni ta oli läinud väga rikkaks.
\par 14 Ja temal oli lamba- ja kitsekarju ja veisekarju ja palju peret, nõnda et vilistid teda kadestasid.
\par 15 Ja vilistid matsid kinni ja täitsid mullaga kõik kaevud, mis tema isa sulased olid kaevanud tema isa Aabrahami päevil.
\par 16 Ja Abimelek ütles Iisakile: „Mine ära meie juurest, sest sa oled saanud meist palju vägevamaks!”
\par 17 Siis Iisak läks sealt ära ja lõi oma telgid üles Gerari orgu ning elas seal.
\par 18 Ja Iisak kaevas uuesti need veekaevud, mis tema isa Aabrahami päevil olid kaevatud ja mis vilistid pärast Aabrahami surma olid kinni matnud; ja ta pani neile needsamad nimed, mis tema isa neile oli pannud.
\par 19 Aga kui Iisaki sulased kaevasid orus ja leidsid seal voolava veega kaevu,
\par 20 siis Gerari karjased riidlesid Iisaki karjastega, öeldes: „Vesi on meie oma!” Ta pani siis kaevule nimeks Eesek, sellepärast et nad temaga olid tülitsenud.
\par 21 Siis nad kaevasid teise kaevu, ja selle pärast riidlesid nad ka; ja sellele ta pani nimeks Sitna.
\par 22 Sealt ta siirdus edasi ja kaevas veel ühe kaevu, aga selle pärast nad ei riielnud; ja sellele ta pani nimeks Rehobot ning ütles: „Nüüd on Issand andnud meile avarust, et võiksime siin maal olla viljakad.”
\par 23 Ja sealt ta läks üles Beer-Sebasse.
\par 24 Ja Issand ilmutas ennast temale selsamal ööl ning ütles: „Mina olen su isa Aabrahami Jumal. Ära karda, sest ma olen sinuga ja õnnistan sind! Ma teen su soo paljuks oma sulase Aabrahami pärast.”
\par 25 Siis ta ehitas sinna altari, hüüdis appi Issanda nime ja lõi sinna oma telgi üles; ja Iisaki sulased kaevasid sinna kaevu.
\par 26 Ja Abimelek tuli Gerarist tema juurde ühes oma sõbra Ahusati ja väepealik Piikoliga.
\par 27 Aga Iisak ütles neile: „Miks tulete minu juurde? Te ju vihkate mind ja olete mind eneste juurest ära saatnud!”
\par 28 Ja nemad vastasid: „Me näeme selgesti, et Issand on sinuga. Seepärast me ütleme: Olgu meie vahel vanne, meie ja sinu vahel, ja me teeme sinuga lepingu,
\par 29 et sa meile kurja ei tee, nõnda nagu me sinusse ei ole puutunud, vaid oleme sulle ainult head teinud ja sind rahuga ära saatnud. Sina oled ju nüüd Issanda õnnistatu!”
\par 30 Siis ta tegi neile võõruspeo ning nad sõid ja jõid.
\par 31 Ja nad tõusid hommikul vara ning andsid üksteisele vande. Siis Iisak saatis nad ära ja nad läksid ta juurest rahuga.
\par 32 Ja selsamal päeval tulid Iisaki sulased ning teatasid temale kaevust, mille nad olid kaevanud, ja ütlesid temale: „Me leidsime vett.”
\par 33 Siis ta nimetas selle Sibaks; seepärast on linna nimeks tänapäevani Beer-Seba.
\par 34 Kui Eesav oli nelikümmend aastat vana, võttis ta naiseks Juuditi, hett Beeri tütre, ja Baasmati, hett Eeloni tütre.
\par 35 Aga need olid meelehärmiks Iisakile ja Rebekale.



\chapter{27}

\par 1 Kui Iisak oli vanaks jäänud ja ta silmanägemine oli tuhmunud, siis ta kutsus Eesavi, oma vanema poja, ja ütles temale: „Mu poeg!„ Ja see vastas temale: ”Siin ma olen!”
\par 2 Ja ta ütles: „Vaata, ma olen vanaks jäänud ega tea oma surmapäeva.
\par 3 Võta nüüd oma jahiriistad, nooletupp ja amb, mine väljale ja küti mulle mõni jahiloom!
\par 4 Valmista siis mulle maitsvat rooga, mida ma armastan, ja too mulle süüa, et mu hing sind õnnistaks, enne kui ma suren!”
\par 5 Aga Rebeka kuulis, kui Iisak rääkis oma poja Eesaviga. Ja kui Eesav oli läinud väljale küttima ja jahisaaki tooma,
\par 6 siis rääkis Rebeka oma poja Jaakobiga, öeldes: „Vaata, ma kuulsin su isa rääkivat su venna Eesaviga ja ütlevat:
\par 7 Too mulle jahisaaki ja valmista mulle maitsvat rooga, et ma söön ja sind Issanda ees õnnistan, enne kui ma suren.
\par 8 Ja nüüd, mu poeg, kuula mu sõna ja tee, mida ma sind käsin:
\par 9 mine karja juurde ja võta mulle sealt kaks head sikutalle ja ma valmistan need su isale maitsvaks roaks, mida ta armastab.
\par 10 Sina vii need siis oma isale, et ta sööks ja sind õnnistaks, enne kui ta sureb!”
\par 11 Aga Jaakob ütles oma emale Rebekale: „Vaata, mu vend Eesav on karune, aga mina olen sile.
\par 12 Võib-olla katsub isa mind käega, siis oleksin tema silmis nagu petis ja tooksin enesele needuse, mitte õnnistuse.”
\par 13 Aga ta ema ütles talle: „Sinu needmine tulgu minu peale, mu poeg! Kuula ainult mu sõna ja mine too mulle!”
\par 14 Siis ta läks ja võttis need ning tõi oma emale; ja ta ema valmistas maitsva roa, mida ta isa armastas.
\par 15 Ja Rebeka võttis oma vanema poja Eesavi parimad riided, mis olid ta juures kodus, ja pani need selga oma nooremale pojale Jaakobile.
\par 16 Tema käte ja sileda kaela ümber aga pani ta sikutallede nahad.
\par 17 Siis ta andis maitsva roa ja leiva, mille ta oli valmistanud, oma poja Jaakobi kätte,
\par 18 ja see läks oma isa juurde ning ütles: „Mu isa!„ Ja tema vastas: ”Siin ma olen! Kumb sa oled, mu poeg?”
\par 19 Ja Jaakob ütles oma isale: „Mina olen Eesav, sinu esmasündinu! Ma tegin, nagu sa mind käskisid. Tõuse, istu ja söö mu jahisaaki, et su hing mind õnnistaks!”
\par 20 Aga Iisak küsis oma pojalt: „Kuidas sa nii kähku leidsid, mu poeg?„ Ja tema vastas: ”Issand, sinu Jumal, saatis mulle ette.”
\par 21 Siis Iisak ütles Jaakobile: „Tule ometi ligemale, et ma sind käega katsun, mu poeg, kas sa oled mu poeg Eesav või mitte?”
\par 22 Ja Jaakob astus oma isa Iisaki juurde, ja tema katsus teda käega ning ütles: „Hääl on Jaakobi Hääl, aga käed on Eesavi käed!”
\par 23 Ja ta ei tundnud teda ära, sest tema käed olid karused nagu ta venna Eesavi käed; ja ta õnnistas teda.
\par 24 Ta küsis veel kord: „Kas sa oled tõesti mu poeg Eesav?„ Ja ta vastas: ”Olen!”
\par 25 Siis ta ütles: „Ulata mulle ja ma söön oma poja jahisaaki, et mu hing sind õnnistaks!” Ja ta ulatas temale selle, ja ta sõi; ja ta tõi temale veini, ja ta jõi.
\par 26 Seejärel ütles ta isa Iisak temale: „Tule nüüd ligemale ja anna mulle suud, mu poeg!”
\par 27 Ja ta astus ligi ning andis temale suud; siis ta tundis tema riiete lõhna ja ta õnnistas teda ning ütles: „Näe, mu poja lõhn - otsekui välja lõhn, mida Issand on õnnistanud!
\par 28 Jumal andku sulle taeva kastet ja maa rammu, ning külluses vilja ja veini!
\par 29 Rahvad orjaku sind, rahvahõimud kummardagu sind! Ole oma vendade isand, su ema pojad kummardagu sind! Neetud olgu, kes sind neab, õnnistatud, kes sind õnnistab!”
\par 30 Ja kui Iisak oli Jaakobit õnnistanud ja kui Jaakob oli just ära läinud oma isa Iisaki juurest, siis tuli tema vend Eesav küttimast.
\par 31 Ja temagi valmistas maitsva roa ja viis oma isa juurde ning ütles isale: „Tõuse, mu isa, ja söö oma poja jahisaaki, et su hing mind õnnistaks!”
\par 32 Aga tema isa Iisak küsis temalt: „Kes sa oled?„ Ja ta vastas: ”Mina olen su poeg Eesav, su esmasündinu!”
\par 33 Siis Iisak värises väga suurest ärritusest ja ütles: „Kes oli siis see, kes jahilooma küttis ja mulle tõi? Ja mina sõin kõike, enne kui sa tulid, ning õnnistasin teda! Õnnistatuks ta jääbki!”
\par 34 Kui Eesav kuulis oma isa sõnu, siis ta kisendas väga valjusti ja kibedasti ning ütles oma isale: „Õnnista ka mind, mu isa!”
\par 35 Aga ta vastas: „Su vend tuli kavalusega ja võttis su õnnistuse.”
\par 36 Siis ta ütles: „Eks ole temale nimeks pandud Jaakob? Juba teist korda on ta mind petnud: ta võttis mu esmasünniõiguse, ja vaata, nüüd ta võttis ka mu õnnistuse!„ Ja ta küsis: ”Kas sul pole hoitud õnnistust eraldi minu jaoks?”
\par 37 Aga Iisak vastas ning ütles Eesavile: „Vaata, ma olen pannud ta sinu isandaks ja olen andnud kõik ta vennad temale sulaseiks, ja ma olen teda varustanud vilja ja veiniga. Mida võiksin siis nüüd teha sinu heaks, mu poeg?”
\par 38 Ja Eesav ütles oma isale: „Ons see sul ainus õnnistus, mu isa? Õnnista ka mind, mu isa!” Ja Eesav tõstis häält ning nuttis.
\par 39 Siis vastas tema isa Iisak ning ütles temale: „Vaata, su eluase on eemal rammusast maast ja ilma taeva kasteta ülalt.
\par 40 Sa elad oma mõõga varal ja pead oma venda orjama. Ometi sünnib, kui end raputad, et rebid tema ikke oma kaelast.”
\par 41 Ja Eesav hakkas Jaakobit vihkama õnnistuse pärast, millega ta isa teda oli õnnistanud; ja Eesav mõtles iseeneses: „Küllap tulevad mu isa leinamise päevad, siis ma tapan oma venna Jaakobi!”
\par 42 Kui Rebekale tehti teatavaks ta vanema poja Eesavi mõtted, siis ta laskis kutsuda oma noorema poja Jaakobi ja ütles temale: „Vaata, sinu vend Eesav trööstib ennast sellega, et ta su tapab.
\par 43 Aga nüüd, mu poeg, kuula mu sõna! Võta kätte ja põgene mu venna Laabani juurde Haaranisse
\par 44 ja jää tema juurde mõneks ajaks, kuni su venna raev on raugenud,
\par 45 kuni su venna viha sinu pärast on möödunud ja ta unustab, mis sa temale oled teinud. Siis ma läkitan sulle järele ja lasen sind sealt ära tuua. Miks peaksin teid mõlemaid kaotama ühel ja samal päeval?”
\par 46 Ja Rebeka ütles Iisakile: „Ma olen elust tüdinud hetitaride pärast. Kui Jaakob võtab naise hetitaride hulgast, selle maa tütreist, niisuguse nagu need, mis elu mul siis on?”

\chapter{28}

\par 1 Siis Iisak kutsus Jaakobi ja õnnistas teda; ja ta keelas teda ning ütles temale: „Ära võta naist kaananlaste tütreist!
\par 2 Võta kätte, mine Mesopotaamiasse oma emaisa Betueli kotta ja võta sealt enesele naine oma ema venna Laabani tütreist.
\par 3 Kõigeväeline Jumal õnnistagu sind, tehku sind viljakaks ja paljuks, et sinust tuleks hulk rahvaid!
\par 4 Ta andku sulle Aabrahami õnnistust, sinule ja su soole koos sinuga, et sa päriksid maa, kus sa võõrana elad, mille Jumal on andnud Aabrahamile!”
\par 5 Ja Iisak saatis Jaakobi teele ning see läks Mesopotaamiasse süürlase Betueli poja Laabani juurde, kes oli Jaakobi ja Eesavi ema Rebeka vend.
\par 6 Kui Eesav nägi, et Iisak oli õnnistanud Jaakobit ja oli saatnud ta Mesopotaamiasse sealt enesele naist võtma, olles teda õnnistanud ja keelanud, öeldes: „Ära võta naist kaananlaste tütreist!”
\par 7 ja et Jaakob oli kuulanud oma isa ja ema ja oli läinud Mesopotaamiasse,
\par 8 siis Eesav mõistis, et kaananlaste tütred olid pahad ta isa Iisaki silmis,
\par 9 ja Eesav läks Ismaeli juurde ning võttis oma naiste kõrvale enesele naiseks Mahalati, Aabrahami poja Ismaeli tütre, Nebajoti õe.
\par 10 Jaakob aga lahkus Beer-Sebast ja läks Haarani poole.
\par 11 Ta sattus ühte paika ja ööbis seal, sest päike oli loojunud; ta võttis selle paiga kividest ühe, pani enesele peaaluseks ja heitis sinna paika magama.
\par 12 Ja ta nägi und, ja vaata, maa peal seisis redel, mille ots ulatus taevasse, ja ennäe, Jumala inglid astusid sedamööda üles ja alla.
\par 13 Ja vaata, Issand seisis tema ees ning ütles: „Mina olen Issand, su isa Aabrahami Jumal ja Iisaki Jumal. Maa, mille peal sa magad, ma annan sinule ja su soole.
\par 14 Ja sinu sugu saab maapõrmu sarnaseks ja sa levid õhtu ja hommiku, põhja ja lõuna poole, ja sinu ja su soo nimel õnnistavad endid kõik maailma suguvõsad.
\par 15 Ja vaata, mina olen sinuga ja hoian sind kõikjal, kuhu sa lähed, ning toon sind taas sellele pinnale, sest ma ei jäta sind maha, kuni olen teinud, mis ma sulle olen öelnud!”
\par 16 Siis Jaakob ärkas unest ja ütles: „Issand on tõesti selles paigas, mina aga ei teadnud seda!”
\par 17 Ja ta kartis ning ütles: „Küll on see paik kardetav! See pole muud midagi kui Jumala koda ja taeva värav!”
\par 18 Ja Jaakob tõusis hommikul vara ning võttis kivi, mille ta oli pannud enesele peaaluseks, ja pani selle sambaks püsti ning valas selle otsa peale õli.
\par 19 Ja ta pani sellele paigale nimeks Peetel; enne aga oli selle linna nimi Luus.
\par 20 Ja Jaakob andis tõotuse, öeldes: „Kui Jumal on minuga ja hoiab mind teel, mida käin, ja annab mulle leiba süüa ja riided selga,
\par 21 ja mina võin rahuga pöörduda oma isakotta, siis on Issand mulle Jumalaks,
\par 22 ja see kivi, mille ma panin sambaks, saab Jumala kojaks. Ja kõigest, mis sa mulle annad, ma annan sulle täpselt kümnist.”

\chapter{29}

\par 1 Ja Jaakob läks teele ning jõudis hommikumaa poegade maale.
\par 2 Ta vaatas, ja ennäe, väljal oli kaev. Ja vaata, sealsamas, selle kõrval, lebas kolm lamba- ja kitsekarja, sest sellest kaevust joodeti karju; kaevu suul aga oli suur kivi.
\par 3 Kui kõik karjad olid kogunenud sinna, siis veeretati kivi kaevu suult, joodeti lambaid ja kitsi ning seati kivi tagasi oma paika kaevu suul.
\par 4 Ja Jaakob küsis neilt: „Vennad, kust te olete?„ Ja nad vastasid: ”Me oleme Haaranist.”
\par 5 Siis ta küsis neilt: „Kas tunnete Laabanit, Naahori poega?„ Ja nad vastasid: ”Tunneme küll!”
\par 6 Ta küsis neilt: „Kuidas ta käsi käib?„ Ja nad vastasid: ”Hästi! Ja näe, sealt tuleb tema tütar Raahel karjaga.”
\par 7 Ta ütles: „Vaata, päike on alles kõrgel, pole veel aeg karja kokku ajada. Jootke loomi ja minge söötke neid!”
\par 8 Aga nad vastasid: „Me ei saa, enne kui kõik karjad on koos. Siis veeretatakse kivi kaevu suult ja me saame joota lambaid.”
\par 9 Kui ta alles nendega rääkis, tuli Raahel oma isa karjaga, sest ta oli neid hoidmas.
\par 10 Ja kui Jaakob nägi Raahelit, oma ema venna Laabani tütart, ja oma ema venna Laabani karja, siis Jaakob astus ligi ja veeretas kivi kaevu suult ning jootis oma ema venna Laabani lambaid.
\par 11 Siis Jaakob suudles Raahelit, tõstis häält ja nuttis.
\par 12 Ja Jaakob andis Raahelile teada, et ta on tema isa sugulane ja Rebeka poeg; ja Raahel jooksis ning teatas oma isale.
\par 13 Ja kui Laaban kuulis sõnumit oma õepojast Jaakobist, siis ta jooksis temale vastu, kaelustas ja suudles teda ning viis ta oma kotta; ja ta jutustas Laabanile kõik, mis oli sündinud.
\par 14 Siis ütles Laaban temale: „Sa oled tõesti minu luu ja liha!” Ja ta jäi tema juurde kuuks ajaks.
\par 15 Ja Laaban ütles Jaakobile: „Kas sa sellepärast, et oled mu sugulane, peaksid mind teenima ilma palgata? Nimeta mulle oma palk!”
\par 16 Laabanil aga oli kaks tütart; vanema nimi oli Lea ja noorema nimi oli Raahel.
\par 17 Leal olid läiketa silmad, aga Raahel oli jumekas ja ilusa välimusega.
\par 18 Jaakob armastas Raahelit, seepärast ta ütles: „Ma teenin sind seitse aastat su noorema tütre Raaheli pärast!”
\par 19 Laaban vastas: „Ma annan ta parem sinule kui mõnele teisele mehele. Jää minu juurde!”
\par 20 Ja Jaakob teenis Raaheli pärast seitse aastat, ja need olid tema silmis nagu üksikud päevad, sellepärast et ta teda armastas.
\par 21 Siis Jaakob ütles Laabanile: „Anna mu naine mulle kätte, sest aeg on täis, et ma võin minna tema juurde!”
\par 22 Laaban koguski kokku kõik selle paiga mehed ja tegi peo.
\par 23 Aga õhtul ta võttis oma tütre Lea ja viis selle tema juurde; ja Jaakob heitis ta juurde.
\par 24 Ja Laaban andis oma teenija Silpa oma tütrele Leale teenijaks.
\par 25 Jõudis hommik, ja vaata, see oli Lea! Siis ütles Jaakob Laabanile: „Mis sa mulle oled teinud! Eks ma ole Raaheli pärast sind teeninud? Mispärast sa mind petsid?”
\par 26 Aga Laaban vastas: „Ei ole meie pool kombeks anda noorem enne vanemat.
\par 27 Pea sellega pulmanädal ära, siis me anname ka teise sulle teenistuse eest, kui sa mind veel teist seitse aastat teenid!”
\par 28 Ja Jaakob tegi nõnda ning pidas sellega pulmanädala ära, siis ta andis oma tütre Raaheli temale naiseks.
\par 29 Ja Laaban andis oma teenija Billa oma tütrele Raahelile teenijaks.
\par 30 Siis Jaakob heitis ka Raaheli juurde, ta armastas ju Raahelit ikkagi rohkem kui Lead; ja ta teenis Laabanit veel teist seitse aastat.
\par 31 Ent kui Issand nägi, et Lea hüljati, siis ta avas tema üsa; Raahel aga oli viljatu.
\par 32 Ja Lea jäi lapseootele ja tõi poja ilmale ning pani temale nimeks Ruuben, sest ta ütles: „Issand on mu alandust näinud. Küllap mu mees hakkab nüüd mind armastama!”
\par 33 Ja ta jäi taas lapseootele ja tõi poja ilmale ning ütles: „Issand on kuulnud, et mind hüljati. Seepärast on ta mulle ka selle andnud.” Ja ta pani temale nimeks Siimeon.
\par 34 Ja tema jäi taas lapseootele ja tõi poja ilmale ning ütles: „Nüüd viimaks mu mees kiindub minusse, sest ma olen temale kolm poega ilmale toonud!” Seepärast pandi sellele nimeks Leevi.
\par 35 Ja tema jäi taas lapseootele ja tõi poja ilmale ning ütles: „Nüüd ma kiidan Issandat!” Seepärast ta pani temale nimeks Juuda. Siis ta lakkas sünnitamast.

\chapter{30}

\par 1 Kui Raahel nägi, et ta ei toonud Jaakobile lapsi ilmale, siis Raahel kadestas oma õde ja ütles Jaakobile: „Muretse mulle lapsi, muidu ma suren!”
\par 2 Aga Jaakobi viha süttis põlema Raaheli vastu ja ta küsis: „Kas mina olen Jumala asemik, kes sulle ihuvilja keelab?”
\par 3 Ja Raahel vastas: „Vaata, seal on mu orjatar Billa. Heida tema juurde, et ta sünnitaks lapsi mu põlvede peale ja minagi saaksin nõnda temalt järglasi!”
\par 4 Ja ta andis temale naiseks oma teenija Billa ning Jaakob heitis selle juurde.
\par 5 Ja Billa jäi lapseootele ning tõi Jaakobile poja ilmale.
\par 6 Siis ütles Raahel: „Jumal tegi mulle õigust. Ta kuulis ka mu häält ja andis mulle poja.” Seepärast ta pani temale nimeks Daan.
\par 7 Ja Billa, Raaheli teenija, jäi taas lapseootele ning tõi Jaakobile teise poja ilmale.
\par 8 Siis ütles Raahel: „Ma olen oma õega võidelnud Jumala võitlust ja olen võitnud.” Ja ta pani temale nimeks Naftali.
\par 9 Kui Lea nägi, et ta oli lakanud sünnitamast, siis ta võttis oma teenija Silpa ja andis selle Jaakobile naiseks.
\par 10 Ja Silpa, Lea teenija, tõi Jaakobile poja ilmale
\par 11 ning Lea ütles: „Õnneks!” Ja ta pani temale nimeks Gaad.
\par 12 Ja Silpa, Lea teenija, tõi Jaakobile teise poja ilmale
\par 13 ning Lea ütles: „Ma olen õnnelik. Tõesti, naised kiidavad mind õnnelikuks.” Ja ta pani temale nimeks Aaser.
\par 14 Kord läks Ruuben nisulõikuse ajal ja leidis väljalt lemmemarju ja tõi neid oma emale Leale. Ja Raahel ütles Leale: „Anna ka minule oma poja lemmemarju!”
\par 15 Aga ta vastas temale: „Kas on veel vähe, et sa võtsid mu mehe? Nüüd tahad sa ka mu poja lemmemarjad ära võtta!„ Siis ütles Raahel: ”Vastutasuks magagu ta täna öösel sinu juures su poja lemmemarjade eest!”
\par 16 Kui Jaakob tuli õhtul väljalt, siis läks Lea temale vastu ja ütles: „Sa pead minu juurde heitma, sest ma olen sind tinginud tasu eest, oma poja lemmemarjade eest!” Ja tema magas sel ööl ta juures.
\par 17 Ja Jumal kuulis Lead, ja Lea jäi lapseootele ja tõi Jaakobile viienda poja ilmale.
\par 18 Ja Lea ütles: „Jumal tasus mulle, et ma andsin oma teenija oma mehele.” Ja ta pani temale nimeks Issaskar.
\par 19 Ja Lea jäi taas lapseootele ja tõi Jaakobile kuuenda poja ilmale.
\par 20 Ja Lea ütles: „Jumal valmistas mulle ilusa kingituse. Nüüd mu mees hakkab mind sallima, sest ma olen temale kuus poega ilmale toonud!” Ja ta pani temale nimeks Sebulon.
\par 21 Ja pärastpoole ta tõi tütre ilmale ning pani temale nimeks Diina.
\par 22 Aga Jumal mõtles Raahelile, ja Jumal kuulis teda ning avas tema üsa.
\par 23 Ja ta jäi lapseootele ja tõi poja ilmale ning ütles: „Jumal võttis ära mu teotuse!”
\par 24 Ja ta pani temale nimeks Joosep, öeldes: „Annaks Issand mulle lisaks veel teisegi poja!”
\par 25 Ja kui Raahel oli Joosepi ilmale toonud, siis Jaakob ütles Laabanile: „Lase mind, et saaksin minna koju ja oma kodumaale!
\par 26 Anna mu naised ja lapsed, kelle pärast ma sind olen teeninud, ja ma lähen, sest sa tead ju ise, kuidas ma sind olen teeninud!”
\par 27 Ja Laaban vastas temale: „Kui ma nüüd sinu silmis armu leiaksin! Märgid näitavad mulle, et Issand on mind sinu pärast õnnistanud.”
\par 28 Ja ta ütles: „Nimeta mulle oma palk ja ma annan selle!”
\par 29 Siis ta vastas temale: „Sina tead ise, kuidas ma sind olen teeninud ja mis on saanud su karjast minu juures.
\par 30 Sest pisut oli seda, mis sul oli enne mind. See on aga ohtrasti kasvanud ja Issand on sind õnnistanud minu sammude läbi. Millal ma siis nüüd saan hoolitseda ka oma pere eest?”
\par 31 Siis ta küsis: „Mis ma sulle pean andma?” Ja Jaakob vastas: ”Ära anna mulle midagi. Kui sa lubad mulle seda, siis ma karjatan ja hoian veelgi su lambaid ja kitsi:
\par 32 ma käisin täna läbi kõik su lamba- ja kitsekarjad, lahutades kõik tähnilised ja kirjud uted ja kõik mustad uted su tallede seast, samuti kirjud ja tähnilised kitsede hulgast. Need olgu mulle palgaks
\par 33 ja mu õigus kostku minu eest tulevikus, kui sa tuled mu palka vaatama: kõik, kes ei ole tähnilised ja kirjud kitsede hulgas ja mustad tallede seas, loetagu minu poolt varastatuiks!”
\par 34 Ja Laaban vastas: „Hästi, sündigu tõesti su sõna järgi!”
\par 35 Ja ta lahutas selsamal päeval vöödilised ja kirjud sikud ja kõik tähnilised ja kirjud kitsed, kellel oli valget küljes, ja kõik mustad tallede seas, andis need oma poegade hooleks
\par 36 ning jättis kolme päeva tee enese ja Jaakobi vahele; Jaakob aga jäi karjatama Laabani ülejäänud lambaid ja kitsi.
\par 37 Ja Jaakob võttis enesele papli-, mandli- ja plataanipuu tooreid keppe ja kooris neile valged vöödid, paljastades keppide valge puu.
\par 38 Siis ta pani kooritud kepid lammaste ja kitsede ette rennidesse ja veekünadesse, kuhu loomad tulid jooma; ja joomas olles nad paaritusid.
\par 39 Ja kui lambad ja kitsed keppide juures paaritusid, siis nad sünnitasid tallesid: vöödilisi, tähnilisi ja kirjusid.
\par 40 Ja Jaakob eraldas noored jäärad: ta pööras isaloomade pead vöödiliste poole ja kõigi mustade poole Laabani karja hulgas. Nõnda tegi ta enesele eraldi karjad ega pannud neid Laabani lammaste ja kitsede sekka.
\par 41 Ja iga kord, kui tugevamad loomad paaritusid, pani Jaakob kepid künadesse nende silme ette, et nad paarituksid keppide juures.
\par 42 Aga kui loomad olid nõrgemad, siis ta ei pannud. Nõnda said nõrgad Laabanile ja tugevamad Jaakobile.
\par 43 Ja mees kosus väga ja tal oli palju lambaid ja kitsi, teenijaid ja sulaseid, kaameleid ja eesleid.

\chapter{31}

\par 1 Aga ta kuulis Laabani poegade kõnelusi, kes ütlesid: „Jaakob on ära võtnud kõik, mis oli meie isa päralt. Sellest, mis oli meie isa päralt, on ta enesele soetanud kõik selle rikkuse.”
\par 2 Ja Jaakob nägi Laabani palet, ja vaata, see ei olnud enam ta vastu nagu enne.
\par 3 Siis Issand ütles Jaakobile: „Mine tagasi oma isade maale ja oma sugulaste seltsi. Mina olen sinuga!”
\par 4 Ja Jaakob läkitas sõna ning käskis kutsuda oma karja juurde väljale Raaheli ja Lea
\par 5 ning ütles neile: „Ma näen teie isa palgest, et ta ei ole enam mu vastu nagu enne. Aga mu isa Jumal oli mu juures.
\par 6 Te ju teate, et ma olen teeninud teie isa kõigest väest.
\par 7 Kuid teie isa narritas mind ja muutis mu palka kümme korda. Jumal aga ei ole lubanud teda mulle kurja teha.
\par 8 Kui ta ütles nõnda: Tähnilised saagu sinule palgaks, siis kõik lambad ja kitsed poegisid tähnilisi. Ja kui ta ütles nõnda: Vöödilised saagu sinule palgaks, siis kõik loomad poegisid vöödilisi.
\par 9 Nõnda võttis Jumal teie isa karja ja andis mulle.
\par 10 Lammaste ja kitsede innaajal tõstsin ma oma silmad üles ja nägin unes, vaata, et isased, kes kargasid emaseid, olid vöödilised, tähnilised ja laigulised.
\par 11 Ja Jumala ingel ütles mulle unes: Jaakob! Ja ma vastasin: Siin ma olen!
\par 12 Siis ta ütles: Tõsta ometi oma silmad üles ja vaata: kõik isased, kes kargavad emaseid, on vöödilised, tähnilised ja laigulised, sest ma olen näinud kõike, mis Laaban sulle teeb!
\par 13 Mina olen Peeteli Jumal, kus sa võidsid samba, kus sa andsid mulle tõotuse. Võta nüüd kätte, lahku siit maalt ja mine tagasi oma sünnimaale!”
\par 14 Siis Raahel ja Lea vastasid ning ütlesid temale: „Kas meil ongi enam osa või omandit meie isakojas?
\par 15 Eks ta ole pidanud meid võõraks, kuna ta meid müüs ja ise muidugi ka meie hinna ära sõi!
\par 16 Jah, kõik see rikkus, mille Jumal meie isalt ära võttis, on meie ja meie laste oma. Ja nüüd tee kõik, mis Jumal sulle on öelnud!”
\par 17 Ja Jaakob võttis kätte, tõstis oma lapsed ja naised kaamelite selga
\par 18 ja saatis teele kogu oma karja ja kõik oma varanduse, mis ta oli kogunud, oma karjavaranduse, mis ta Mesopotaamias oli soetanud, et minna oma isa Iisaki juurde Kaananimaale.
\par 19 Aga Laaban oli läinud lambaid niitma. Ja Raahel varastas oma isa teeravikujud.
\par 20 Jaakob kasutas süürlase Laabani teadmatust ega andnud temale märku, et ta põgeneb.
\par 21 Nõnda ta siis põgenes koos kõigega, mis tal oli, võttis kätte ja läks üle jõe ning siirdus Gileadi mäestiku poole.
\par 22 Aga kolmandal päeval anti Laabanile teada, et Jaakob oli põgenenud.
\par 23 Tema võttis siis enesega kaasa oma suguvennad ja ajas teda taga seitse päevateekonda ning jõudis Gileadi mäestikus temale järele.
\par 24 Kuid Jumal tuli süürlase Laabani juurde öösel unes ja ütles temale: „Hoia, et sa Jaakobile ei ütle head ega halba!”
\par 25 Kui Laaban Jaakobile järele jõudis, oli Jaakob mäestikus telgi üles löönud, ja Laabangi suguvendadega lõi telgi üles Gileadi mäestikku.
\par 26 Ja Laaban ütles Jaakobile: „Mis sa oled teinud? Sa kasutasid mu teadmatust ja viisid ära mu tütred, nagu oleksid nad olnud mõõga abil vangistatud.
\par 27 Miks sa põgenesid salaja ja vargsel viisil ega teatanud mulle, et oleksin saanud sind rõõmsasti ära saata laulude, trummi ja kandlega?
\par 28 Sa ei lasknud mind suudelda oma poegi ja tütreid! Sa oled nüüd talitanud mõistmatult.
\par 29 Mul oleks meelevald teha teile kurja. Aga teie isa Jumal rääkis minuga eile öösel, öeldes: Hoia, et sa Jaakobile ei ütle head ega halba!
\par 30 Nüüd oled sa küll läinud oma teed, sellepärast et sa igatsesid nii väga oma isakoja järele. Aga mispärast sa varastasid mu jumalad?”
\par 31 Ja Jaakob vastas ning ütles Laabanile: „Sellepärast et ma kartsin. Sest ma mõtlesin, et sa röövid minult oma tütred.
\par 32 See, kelle juurest sa leiad oma jumalad, ärgu jäägu elama! Meie suguvendade ees otsi läbi, mis mul kaasas on, ja võta ära, mis on sinu!” Aga Jaakob ei teadnud, et Raahel oli need varastanud.
\par 33 Ja Laaban läks Jaakobi telki ja Lea telki ja mõlema teenija telki, aga ei leidnud midagi; ja Lea telgist välja tulnud, läks ta Raaheli telki.
\par 34 Kuid Raahel oli võtnud teeravid ja oli pannud need kaameli sadula tasku ning istus ise nende peal. Ja Laaban kompas läbi kogu telgi, aga ei leidnud midagi.
\par 35 Ja Raahel ütles oma isale: „Ärgu süttigu viha mu isanda silmis, et ma ei saa su ees üles tõusta, sest mul on naiste asjad!” Nõnda ta otsis läbi, aga teeraveid ta ei leidnud.
\par 36 Siis Jaakob vihastus ja riidles Laabaniga. Ja Jaakob kostis ning ütles Laabanile: „Milles seisneb mu üleastumine? Mis on mu patt, et oled mind nii tulisi jalu taga ajanud?
\par 37 Kuna sa oled läbi otsinud kogu mu kraami, siis missuguse oma koja riista oled sa leidnud? Pane siia minu suguvendade ja oma suguvendade ette, et nad võiksid õigust mõista meie mõlema vahel!
\par 38 Ma olin sinu juures kakskümmend aastat. Su lambad ja kitsed ei heitnud loodet ja jäärasid su karjast ma ei söönud.
\par 39 Murtut ma sulle ei toonud, ma pidin selle hüvitama. Sa nõudsid minult niihästi päeval kui öösel varastatut.
\par 40 Päeval piinas mind palavus ja öösel külm, ja uni põgenes mu silmist.
\par 41 Nüüd ma olen olnud su kojas kakskümmend aastat. Neliteist aastat ma teenisin sind su kahe tütre pärast ja kuus aastat lammaste ja kitsede pärast, ja sa muutsid mu palka kümme korda.
\par 42 Kui minuga ei oleks olnud mu isa Jumal, Aabrahami Jumal, Iisaki Kartus, siis oleksid sa mind nüüd tühje käsi ära saatnud. Jumal on näinud mu häda ja mu kätevaeva ja on eile öösel teinud otsuse.”
\par 43 Siis Laaban kostis ja ütles Jaakobile: „Tütred on minu Tütred ja pojad on minu pojad ja kari on minu kari, ja kõik, mis sa näed, on minu! Aga mida ma saaksin praegu teha oma tütarde heaks või nende poegade heaks, keda nad on ilmale toonud?
\par 44 Aga tule nüüd, tehkem leping, mina ja sina, ja see olgu tunnistajaks minu ja sinu vahel!”
\par 45 Siis Jaakob võttis ühe kivi ja pani sambaks püsti.
\par 46 Ja Jaakob ütles oma suguvendadele: „Korjake kive!” Ja need võtsid kive ning kuhjasid kivikangru; ja nad sõid seal kivikangru peal.
\par 47 Ja Laaban pani sellele nimeks Jegar-Sahaduuta; Jaakob aga nimetas selle Galeediks.
\par 48 Ja Laaban ütles: „See kivikangur olgu täna tunnistajaks minu ja sinu vahel!” Seepärast ta pani sellele nimeks Galeed
\par 49 ja Mispa, sest ta ütles: „Issand valvab minu ja sinu vahel, kui me teineteist enam ei näe.
\par 50 Kui sa kohtled mu tütreid halvasti või võtad mu tütarde kõrvale teisi naisi, ilma et ükski inimene oleks meie juures, vaata, siis on Jumal ometi tunnistajaks minu ja sinu vahel.”
\par 51 Siis ütles Laaban Jaakobile: „Vaata, see kivikangur, ja Vaata, see sammas, mille ma püstitasin enese ja sinu vahele, -
\par 52 see kivikangur olgu tunnistajaks, samuti olgu see sammas tunnistajaks, et mina ei tohi tulla sellest kivikangrust mööda sinu juurde ja et sina ei tohi tulla sellest kivikangrust ja sambast mööda minu juurde kurja tegema!
\par 53 Aabrahami Jumal ja Naahori Jumal, nende vanemate Jumal, see mõistku kohut meie vahel!” Ja Jaakob vandus oma isa Iisaki Kartuse juures.
\par 54 Ja Jaakob ohverdas mäe peal tapaohvri ning kutsus oma suguvennad leiba võtma. Ja nad võtsid leiba ning jäid ööseks mäele.

\chapter{32}

\par 1 Aga Laaban tõusis hommikul vara, suudles oma poegi ja tütreid ning õnnistas neid; siis Laaban läks teele ja pöördus tagasi koju.
\par 2 Ka Jaakob läks oma teed. Aga temale tulid vastu Jumala inglid.
\par 3 Ja Jaakob ütles, kui ta neid nägi: „See on Jumala leer!” Ja ta pani sellele paigale nimeks Mahanaim.
\par 4 Siis Jaakob läkitas käskjalad enese eel oma venna Eesavi juurde Seirimaale Edomi väljadele
\par 5 ja andis neile käsu, öeldes nõnda: „Öelge mu isandale Eesavile: Nõnda ütleb su sulane Jaakob: Ma olen tänini Laabani juures võõrana elanud ja viibinud.
\par 6 Mul on härgi ja eesleid, lambaid ja kitsi, sulaseid ja teenijaid, ja ma läkitan seda teatama oma isandale, et su silmis armu leida.”
\par 7 Käskjalad tulid tagasi Jaakobi juurde ja ütlesid: „Me jõudsime su venna Eesavi juurde. Ta juba tulebki sulle vastu koos neljasaja mehega.”
\par 8 Siis Jaakob kartis väga ja tal oli kitsas käes; ja ta jaotas rahva, kes koos temaga oli, samuti lambad ja kitsed, veised ja kaamelid, kahte leeri,
\par 9 sest ta mõtles: Kui Eesav tuleb ühe leeri kallale ja lööb selle maha, siis teine leer pääseb.
\par 10 Ja Jaakob ütles: „Mu isa Aabrahami Jumal ja mu isa Iisaki Jumal, Issand, kes mulle ütlesid: Mine tagasi oma maale ja oma sugulaste seltsi, siis ma teen sulle head!
\par 11 Mina pole väärt kõiki neid heategusid ja kõike seda truudust, mida sa oma sulasele oled osutanud. Sest kepp käes läksin ma üle Jordani, aga nüüd on mul kaks leeri.
\par 12 Päästa mind ometi mu venna Eesavi käest, sest ma kardan, et ta tuleb ja lööb mind maha koos emade ja lastega!
\par 13 Sina ise aga oled öelnud: Ma teen sulle tõesti head ja lasen su soo saada mereliiva sarnaseks, mida ei saa ära lugeda selle rohkuse pärast!”
\par 14 Ja ta jäi selleks ööks sinna ning võttis sellest, mis oli saanud tema omaks, oma vennale Eesavile kingituseks
\par 15 kakssada kitse ja kakskümmend sikku, kakssada emalammast ja kakskümmend jäära,
\par 16 kolmkümmend imetajat kaamelit ja nende varssa, nelikümmend lehma ja kümme härjavärssi, kakskümmend emaeeslit ja kümme eeslitäkku.
\par 17 Ja ta andis need oma sulaste kätte, iga karja eraldi, ja ütles oma sulastele: „Minge minu eel ja jätke vahemaa iga karja vahele!”
\par 18 Ja ta käskis esimest, öeldes: „Kui mu vend Eesav tuleb sulle vastu, küsib sinult ja ütleb: Kelle oma sa oled ja kuhu sa lähed, ja kelle omad on need, kes su ees on?,
\par 19 siis vasta: Need on su sulase Jaakobi omad, kingitus, mis läkitatakse mu isandale Eesavile, ja vaata, ka tema ise tuleb meie taga!”
\par 20 Ja ta käskis ka teist ja kolmandat ja kõiki muid, kes karjade järel käisid, öeldes: „Te peate Eesavile ütlema sedasama, kui te teda kohtate!
\par 21 Ja öelge ka: Vaata, su sulane Jaakob tuleb meie taga!” Sest ta mõtles: Ma lepitan teda kingitusega, mis mu eel läheb. Alles pärast seda ilmun ma ise tema ette, vahest võtab ta mind siis armulikult vastu.
\par 22 Nõnda läks kingitus tema eel, aga ta ise jäi selleks ööks leeri.
\par 23 Kuid veel selsamal ööl ta tõusis üles ja võttis oma mõlemad naised ja mõlemad teenijad ja oma üksteist poega ja läks läbi Jabboki koolme.
\par 24 Ta võttis nad ja viis üle jõe, samuti viis ta üle, mis tal oli.
\par 25 Aga Jaakob ise jäi üksinda maha. Siis heitles üks mees temaga, kuni hakkas koitma.
\par 26 Aga kui see nägi, et ta ei saanud võimust tema üle, siis ta lõi tema puusaliigest; ja Jaakobi puusaliiges nihkus paigast, kui ta heitles temaga.
\par 27 Ja mees ütles: „Lase mind lahti, sest juba koidab!„ Aga tema vastas: ”Ei ma lase sind mitte, kui sa mind ei õnnista!”
\par 28 Siis ta küsis temalt: „Mis su nimi on?„ Ja ta vastas: ”Jaakob.”
\par 29 Seepeale ütles tema: „Su nimi ärgu olgu enam Jaakob, vaid olgu Iisrael, sest sa oled võidelnud Jumala ja inimestega ja oled võitnud!”
\par 30 Siis küsis Jaakob ja ütles: „Ütle nüüd mulle oma nimi!„ Aga ta vastas: ”Miks sa mu nime küsid?” Ja ta õnnistas teda seal.
\par 31 Ja Jaakob pani sellele paigale nimeks Penuel, sest ta ütles: „Kuigi ma nägin Jumalat palgest palgesse, pääses siiski mu hing!”
\par 32 Päike tõusis, kui ta puusast longates Penuelist edasi läks.

\chapter{33}

\par 1 Kui Jaakob oma silmad üles tõstis ja vaatas, ennäe, siis tuli Eesav ja koos temaga nelisada meest. Ta jaotas nüüd lapsed Lea ja Raaheli ja mõlema teenija vahel,
\par 2 seadis teenijad ja nende lapsed ette, Lea ja tema lapsed nende järele, Raaheli ja Joosepi viimaseiks.
\par 3 Ta ise aga läks nende eel ja kummardas seitse korda maani, kuni ta jõudis oma venna juurde.
\par 4 Aga Eesav jooksis temale vastu ja süleles teda, langes temale kaela ja suudles teda; ja nad nutsid.
\par 5 Siis ta tõstis oma silmad üles ja nägi naisi ja lapsi, ja ta küsis: „Kes need sul on?„ Ja tema vastas: ”Need on lapsed, keda Jumal su sulasele armulikult on andnud.”
\par 6 Ka teenijad astusid ligi, nemad ise ja nende lapsed, ja nad kummardasid.
\par 7 Siis astus ligi ka Lea koos oma lastega ja nad kummardasid; lõppeks astusid ligi Joosep ja Raahel ja kummardasid.
\par 8 Siis ta küsis: „Mida sa kavatsed kogu selle leeriga, keda ma kohtasin?„ Ja tema vastas: ”Oma isanda silmis armu leida!”
\par 9 Aga Eesav ütles: „Mul eneselgi on küllalt, mu vend. Jäägu sulle, mis sul on!”
\par 10 Kuid Jaakob vastas: „Sugugi mitte! Kui ma nüüd su silmis olen armu leidnud, siis võta mu kingitus minult vastu! Sest ma olen ju tohtinud näha su palet, otsekui näeks Jumala palet, ja sa oled olnud mu vastu lahke.
\par 11 Võta nüüd minu tervituskink, mis sulle toodi, sest Jumal on olnud mu vastu armuline ja mul on kõike küllalt!” Ja ta käis temale peale, kuni ta võttis.
\par 12 Siis ütles Eesav: „Hakkame liikuma ja lähme, ja mina käin sinuga rinnu.”
\par 13 Aga Jaakob vastas temale: „Mu isand näeb ju, et lapsed on väetid ja minu hooleks on imetajad lambad ja lehmad; kui neid liiga kiiresti aetakse ühegi päeva, siis sureb kogu kari.
\par 14 Mingu aga mu isand oma sulase eel ja mina liigun pikkamisi oma ees käiva karja kannul ja laste kannul, kuni ma jõuan oma isanda juurde Seiri.”
\par 15 Eesav ütles: „Ma jätan siis sinu juurde osa rahvast, kes koos minuga on.„ Aga tema vastas: ”Mispärast nõnda? Kui ma ainult oma isanda silmis armu leiaksin!”
\par 16 Ja Eesav läks selsamal päeval oma teed tagasi Seiri.
\par 17 Aga Jaakob liikus Sukkotti, ehitas enesele koja ja tegi oma karjadele lehtkatused; seepärast pandi sellele paigale nimeks Sukkot.
\par 18 Ja Jaakob jõudis Mesopotaamiast tulles õnnelikult Sekemi linna, mis on Kaananimaal, ja lõi linna ees leeri üles.
\par 19 Ta ostis selle väljaosa, kuhu ta oma telgi oli üles löönud, Sekemi isa Hamori lastelt saja rahatüki eest.
\par 20 Ja ta püstitas sinna altari ning pani sellele nimeks „Jumal, Iisraeli Jumal”.

\chapter{34}

\par 1 Kord läks Diina, Lea tütar, kelle Lea oli Jaakobile ilmale toonud, maa tütreid vaatama.
\par 2 Aga maa vürsti hiivlase Hamori poeg Sekem nägi teda, võttis tema, magas ta juures ja naeris ta ära.
\par 3 Kuid ta hing kiindus Diinasse, Jaakobi tütresse, ja ta armastas tütarlast ning rääkis tütarlapsele meelitusi.
\par 4 Ja Sekem rääkis oma isa Hamoriga, öeldes: „Võta see tüdruk mulle naiseks!”
\par 5 Jaakob aga sai kuulda, et ta tütar Diina oli ära teotatud. Aga et ta pojad olid tema karjaga väljal, siis Jaakob vaikis, kuni nad koju tulid.
\par 6 Ja Hamor, Sekemi isa, läks Jaakobi juurde, et temaga rääkida.
\par 7 Jaakobi pojad tulid väljalt, kui nad sellest olid kuulda saanud; mehed olid nördinud ja nende viha süttis väga põlema, et ta Jaakobi tütre juures magades oli Iisraelis teinud häbiteo; sest nõnda ei tohtinud teha.
\par 8 Ja Hamor rääkis nendega, öeldes: „Mu poja Sekemi hing on kiindunud teie tütresse. Andke ta ometi temale naiseks!
\par 9 Saage meiega langudeks, andke meile oma tütreid ja võtke enestele meie tütreid,
\par 10 ja jääge meie juurde elama! Maa on lahti teie ees, elage, rännake ja kodunege siin!”
\par 11 Ja Sekem ütles tema isale ja vendadele: „Kui ma teie silmis armu leian, siis annan teile, mida te minult nõuate.
\par 12 Pange mulle peale ükskõik kui palju mõrsjahinda ja kingitusi: ma annan, mida te minult nõuate. Andke ainult tütarlaps mulle naiseks!”
\par 13 Siis kostsid Jaakobi pojad Sekemile ja tema isale Hamorile, rääkides aga kavalasti, sellepärast et ta nende õe Diina oli ära teotanud,
\par 14 ja ütlesid neile: „Me ei või teha niisugust asja, et anname oma õe mehele, kellel on eesnahk, sest see oleks meile häbiks.
\par 15 Ainult sel tingimusel oleme teiega nõus, kui saate meie sarnaseiks, lastes kõik oma meesterahvad ümber lõigata.
\par 16 Siis me anname oma tütreid teile ja võtame teie tütreid enestele, elame üheskoos teiega ja saame üheks rahvaks.
\par 17 Aga kui te ei võta meid kuulda ega lase endid ümber lõigata, siis võtame oma tütre ja läheme ära.”
\par 18 Ja nende sõnad olid head Hamori ja Hamori poja Sekemi silmis.
\par 19 Ja noor mees ei kõhelnud nõnda tegemast, sest ta ihaldas Jaakobi tütart; ja tema oli lugupeetum kui kõik teised ta isakojas.
\par 20 Siis Hamor ja ta poeg Sekem läksid oma linna väravasse ja rääkisid oma linna meestega, öeldes:
\par 21 „Need mehed on rahuarmastajad meie suhtes. Elagu ja rännaku nad siin maal! Sest maa, vaata, laiub ju nende ees igat kätt. Siis võtame enestele naisteks nende tütreid ja anname oma tütreid neile.
\par 22 Aga ainult sel tingimusel on mehed meiega nõus meie juures elama ja saama meiega üheks rahvaks, kui meil kõik meesterahvad ümber lõigatakse, nõnda nagu nemad ise on ümber lõigatud.
\par 23 Nende karjad ja varandus ja kõik nende veoloomad, eks need saa siis meile. Olgem ainult nendega nõus, siis nad asuvad meie juurde elama!”
\par 24 Ja nad võtsid kuulda Hamorit ja tema poega Sekemit, kõik, kes ta linna väravast läbi käisid; ja kõik meesterahvad lõigati ümber, kõik, kes ta linna väravast läbi käisid.
\par 25 Aga kolmandal päeval, kui nad olid valudes, võtsid kaks Jaakobi poega, Siimeon ja Leevi, Diina vennad, kumbki oma mõõga ja läksid takistamatult linna ja tapsid ära kõik meesterahvad.
\par 26 Nad tapsid mõõgateraga ka Hamori ja tema poja Sekemi, võtsid Sekemi kojast Diina ja läksid ära.
\par 27 Jaakobi pojad tulid haigetele kallale ja riisusid linna, sellepärast et nad nende õe olid ära teotanud.
\par 28 Nad võtsid ära nende lambad ja kitsed, veised ja eeslid ja mis iganes oli linnas või väljal.
\par 29 Nad viisid ära kõik nende varanduse ja kõik nende lapsed ja naised, ja riisusid kõik, mis kodades oli.
\par 30 Aga Jaakob ütles Siimeonile ja Leevile: „Te saadate mind õnnetusse, sellepärast et te mind olete teinud vihatavaks maa elanike, kaananlaste ja perislaste hulgas! Mul on vähe mehi: kui nad kogunevad mu vastu, siis nad löövad mind ja me hukkume, niihästi mina kui mu pere!”
\par 31 Kuid nemad vastasid: „Kas ta siis tohtis talitada meie õega nagu hooraga?”

\chapter{35}

\par 1 Ja Jumal ütles Jaakobile: „Võta kätte, mine üles Peetelisse, ela seal ja tee sinna altar Jumalale, kes sulle ennast ilmutas, kui sa põgenesid oma venna Eesavi eest!”
\par 2 Ja Jaakob ütles oma perele ja kõigile, kes olid koos temaga: „Kõrvaldage võõrad jumalad, kes teie keskel on, ja puhastage endid ning vahetage riided!
\par 3 Ja me võtame kätte ning läheme üles Peetelisse ja teeme sinna altari Jumalale, kes mind kuulis mu ahastuse ajal ja oli minuga teel, mida käisin.”
\par 4 Siis nad andsid Jaakobile kõik nende käes olevad võõrad jumalad ja kõrvarõngad, mis neil kõrvus olid, ja Jaakob mattis need maha Sekemi ligidal oleva tamme alla.
\par 5 Ja nad läksid teele; aga hirm Jumala ees lasus ümberkaudseil linnadel ja need ei ajanud taga Jaakobi poegi.
\par 6 Ja Jaakob jõudis Luusi, see on Peetelisse Kaananimaal, tema ja kogu rahvas, kes oli koos temaga.
\par 7 Ja ta ehitas sinna altari ning pani sellele paigale nimeks „Peeteli Jumal”, sest seal oli Jumal ennast temale ilmutanud, kui ta põgenes oma venna eest.
\par 8 Aga Deboora, Rebeka imetaja, suri, ja ta maeti ühe tamme alla allpool Peetelit, ja sellele pandi nimeks Nututamm.
\par 9 Ja Jumal ilmutas ennast taas Jaakobile, kui see Mesopotaamiast tuli, ja õnnistas teda.
\par 10 Ja Jumal ütles temale: „Jaakob on su nimi. Ärgu hüütagu su nime enam Jaakobiks, vaid su nimi olgu Iisrael!” Ja ta pani temale nimeks Iisrael.
\par 11 Ja Jumal ütles temale: „Mina olen Kõigeväeline Jumal. Ole viljakas ja paljune! Sinust saab rahvas, jah, rahvaste hulk, ja sinu niudeist tulevad kuningad.
\par 12 Ja maa, mille ma andsin Aabrahamile ja Iisakile, ma annan sinule; ka sinu soole pärast sind ma annan selle maa.”
\par 13 Ja Jumal läks tema juurest üles paigast, kus ta temaga oli rääkinud.
\par 14 Ja Jaakob püstitas samba sinna paika, kus ta temaga oli rääkinud, kivisamba, ja kallas selle peale joogiohvri ning valas õli.
\par 15 Ja Jaakob nimetas paiga, kus Jumal temaga oli rääkinud, Peeteliks.
\par 16 Siis nad läksid Peetelist teele. Aga kui veel tükk maad oli minna Efratani, pidi Raahel sünnitama, ja tal oli raske sünnitus.
\par 17 Tema raske sünnituse ajal ütles aitajanaine temale: „Ära karda, sest ka seekord on sul poeg!”
\par 18 Ja kui ta hing oli välja minemas, sest varsti ta surigi, pani ta temale nimeks Ben-Ooni; aga tema isa kutsus teda Benjaminiks.
\par 19 Ja Raahel suri ning ta maeti Efrata tee äärde, see on Petlemma.
\par 20 Ja Jaakob püstitas tema hauale samba; see Raaheli hauasammas on alles tänapäevani.
\par 21 Ja Iisrael läks teele ning lõi oma telgi üles teisele poole Karjatorni.
\par 22 Ja kui Iisrael elas seal maal, juhtus, et Ruuben läks ja magas oma isa liignaise Billa juures. Ja Iisrael sai sellest kuulda ja see oli paha tema silmis.
\par 23 Jaakobil oli kaksteist poega; Lea pojad: Ruuben, Jaakobi esmasündinu, Siimeon, Leevi, Juuda, Issaskar ja Sebulon;
\par 24 Raaheli pojad: Joosep ja Benjamin;
\par 25 Billa, Raaheli teenija pojad: Daan ja Naftali;
\par 26 Silpa, Lea teenija pojad: Gaad ja Aaser. Need olid Jaakobi pojad, kes sündisid temale Mesopotaamias.
\par 27 Ja Jaakob jõudis oma isa Iisaki juurde Mamresse, Kirjat-Arbasse, see on Hebronisse, kus Aabraham ja Iisak olid võõrastena elanud.
\par 28 Ja Iisaki elupäevi oli sada kaheksakümmend aastat.
\par 29 Siis Iisak heitis hinge ja suri; ta koristati oma rahva juurde, vana ja elatanud; ta pojad Eesav ja Jaakob matsid tema.

\chapter{36}

\par 1 Ja need olid Eesavi, see on Edomi järeltulijad:
\par 2 Eesav võttis oma naised Kaanani tütreist: Aada, hett Eeloni tütre, ja Oholibama, hiivlase Sibeoni poja Ana tütre,
\par 3 ja Baasmati, Ismaeli tütre, Nebajoti õe.
\par 4 Aada tõi Eesavile ilmale Eliifase, ja Baasmat tõi ilmale Reueli.
\par 5 Ja Oholibama tõi ilmale Jeusi, Jalami ja Korahi; need olid Eesavi pojad, kes temale sündisid Kaananimaal.
\par 6 Ja Eesav võttis oma naised, pojad ja tütred ja kõik oma pere hingelised, karja, kõik veoloomad ja kogu varanduse, mis ta Kaananimaal oli soetanud, ja läks teisele maale, ära oma venna Jaakobi juurest.
\par 7 Sest nende varandus oli liiga suur üheskoos elamiseks, ja maa, kus nad võõrastena elasid, ei suutnud neid toita nende karjade pärast.
\par 8 Ja Eesav asus elama Seiri mäestikku; Eesav on Edom.
\par 9 Ja need olid Eesavi, edomlaste isa järeltulijad Seiri mäestikus:
\par 10 need olid Eesavi poegade nimed: Eliifas, Eesavi naise Aada poeg, Reuel, Eesavi naise Baasmati poeg.
\par 11 Ja Eliifase pojad olid: Teeman, Oomar, Sefo, Gatam ja Kenas.
\par 12 Ja Timna oli Eesavi poja Eliifase liignaine ja tema tõi Eliifasele ilmale Amaleki; need olid Eesavi naise Aada järeltulijad.
\par 13 Ja need olid Reueli pojad: Nahat, Serah, Samma ja Missa; need olid Eesavi naise Baasmati järeltulijad.
\par 14 Ja need olid Eesavi naise Oholibama, Sibeoni poja Ana tütre pojad: tema tõi Eesavile ilmale Jeusi, Jalami ja Korahi.
\par 15 Need olid Eesavi poegade vürstid: Eliifase, Eesavi esmasündinu pojad: vürst Teeman, vürst Oomar, vürst Sefo, vürst Kenas,
\par 16 vürst Gatam, vürst Amalek. Need olid Eliifasest põlvnevad vürstid Edomimaal, need olid Aada järeltulijad.
\par 17 Ja need olid Eesavi poja Reueli pojad: vürst Nahat, vürst Serah, vürst Samma, vürst Missa. Need olid Reuelist põlvnevad vürstid Edomimaal, need olid Eesavi naise Baasmati järeltulijad.
\par 18 Ja need olid Eesavi naise Oholibama pojad: vürst Jeus, vürst Jalam, vürst Korah. Need olid Eesavi naisest, Ana tütrest Oholibamast põlvnevad vürstid.
\par 19 Need olid Eesavi, see on Edomi järeltulijad, ja need olid nende vürstid.
\par 20 Need olid horiit Seiri pojad, selle maa elanikud: Lootan, Soobal, Sibeon, Ana,
\par 21 Diison, Eeser ja Diisan; need olid horiitide vürstid, Seiri pojad Edomimaal.
\par 22 Ja Lootani pojad olid Hori ja Heemam; ja Lootani õde oli Timna.
\par 23 Ja need olid Soobali pojad: Alvan, Maanahat, Eebal, Sefo ja Oonam.
\par 24 Ja need olid Sibeoni pojad: Ajja ja Ana; Ana oli see, kes kõrbes leidis kuumaveeallikaid, kui ta karjatas oma isa Sibeoni eesleid.
\par 25 Ja need olid Ana lapsed: Diison ja Oholibama, Ana tütar.
\par 26 Ja need olid Diisoni pojad: Hemdan, Esban, Jitran ja Keran.
\par 27 Need olid Eeseri pojad: Bilhan, Saavan ja Akan.
\par 28 Need olid Diisani pojad: Uuts ja Aran.
\par 29 Need olid horiitide vürstid: vürst Lootan, vürst Soobal, vürst Sibeon, vürst Ana,
\par 30 vürst Diison, vürst Eeser, vürst Diisan. Need olid horiitide vürstid nende vürstide kaupa Seirimaal.
\par 31 Ja need olid kuningad, kes valitsesid Edomimaal, enne kui ükski kuningas valitses Iisraeli laste üle:
\par 32 Bela, Beori poeg, oli kuningaks Edomis, ja tema linna nimi oli Dinhaba.
\par 33 Kui Bela suri, sai tema asemel kuningaks Joobab, Serahi poeg Bosrast.
\par 34 Kui Joobab suri, sai tema asemel kuningaks Huusam teemanlaste maalt.
\par 35 Kui Huusam suri, sai tema asemel kuningaks Hadad, Bedadi poeg, kes lõi midjanlasi Moabi väljadel; ja tema linna nimi oli Aviit.
\par 36 Kui Hadad suri, sai tema asemel kuningaks Samla Masreekast.
\par 37 Kui Samla suri, sai tema asemel kuningaks Saul jõeäärsest Rehobotist.
\par 38 Kui Saul suri, sai tema asemel kuningaks Baal-Haanan, Akbori poeg.
\par 39 Kui Baal-Haanan, Akbori poeg, suri, sai tema asemel kuningaks Hadar; tema linna nimi oli Pau; ja tema naise nimi oli Mehetabel, Mee-Sahabi tütre Matredi tütar.
\par 40 Ja need olid Eesavi vürstide nimed nende suguvõsade kaupa, nimeliselt nende asupaikade järgi: vürst Timna, vürst Alva, vürst Jetet,
\par 41 vürst Oholibama, vürst Eela, vürst Piinon,
\par 42 vürst Kenas, vürst Teeman, vürst Mibsar,
\par 43 vürst Magdiel, vürst Iiram. Need olid Edomi, see on Eesavi, edomlaste isa vürstid nende elukohtade järgi nende pärusmaal.

\chapter{37}

\par 1 Aga Jaakob elas maal, kus ta isa oli võõrana elanud, Kaananimaal.
\par 2 Need on Jaakobi suguvõsa lood: Kui Joosep oli seitsmeteistkümneaastane, siis oli ta koos oma vendadega lammaste ja kitsede karjane; tema oli abilisena oma isa naiste Billa ja Silpa poegade juures. Ja Joosep kandis isale ette nende halva kuulsuse.
\par 3 Iisrael armastas Joosepit enam kui kõiki oma poegi, sest ta oli tema vana ea poeg, ja ta tegi temale kirju kuue.
\par 4 Kui ta vennad nägid, et nende isa armastas teda enam kui kõiki tema vendi, siis nad vihkasid teda ega suutnud rääkida temaga sõbralikult.
\par 5 Kord nägi Joosep unenäo ja jutustas selle oma vendadele; seejärel hakkasid need teda veel enam vihkama.
\par 6 Ta nimelt ütles neile: „Kuulge ometi seda unenägu, mis ma unes nägin!
\par 7 Jah, vaadake, me olime väljal vihke sidumas, ja ennäe, minu vihk tõusis üles ning jäigi püsti seisma. Aga vaata, teie vihud ümbritsesid seda ja kummardasid minu vihu ees.”
\par 8 Siis ta vennad ütlesid temale: „Kas sina tahad saada meile kuningaks ja hakata meie üle valitsema?” Ja nad vihkasid teda veelgi enam tema unenägude ja kõnede pärast.
\par 9 Ja ta nägi veel teise unenäo, jutustas selle oma vendadele ja ütles: „Vaata, ma nägin veel ühe unenäo, ja ennäe, päike, kuu ja üksteist tähte kummardasid minu ees.”
\par 10 Aga kui ta seda jutustas oma isale ja vendadele, siis ta isa sõitles teda ning ütles temale: „Mis unenägu see küll on, mis sa nägid! Kas mina ja su ema ja vennad tõesti peame tulema ja sinu ees maani kummardama?”
\par 11 Ta vennad said temale kadedaks, aga ta isa pidas meeles selle loo.
\par 12 Kord olid ta vennad läinud Sekemisse oma isa karja hoidma.
\par 13 Ja Iisrael ütles Joosepile: „Eks ole su vennad Sekemis karja hoidmas? Tule, ma läkitan sind nende juurde!„ Ja tema vastas: ”Siin ma olen!”
\par 14 Siis ta ütles temale: „Mine ometi vaatama, kas su vendade käsi käib hästi ja kas kari on korras, ja too mulle sõna!” Ta läkitas teda Hebroni orust ja ta tuli Sekemisse.
\par 15 Ja üks mees kohtas teda, kui ta oli väljal ümber ekslemas. Ja mees küsis temalt, öeldes: „Mida sa otsid?”
\par 16 Ja tema vastas: „Ma otsin oma vendi. Ütle mulle ometi, kus nad karja hoiavad?”
\par 17 Ja mees ütles: „Nad on siit edasi läinud, sest ma kuulsin neid ütlevat: Läki Dotanisse!” Ja Joosep läks järele oma vendadele ning leidis nad Dotanis.
\par 18 Aga nad nägid teda kaugelt ja enne kui ta jõudis nende juurde, võtsid nad õelalt nõuks ta tappa.
\par 19 Nad ütlesid üksteisele: „Näe, sealt tuleb see unenägude sepitseja!
\par 20 Tulgem nüüd, tapkem ta ära, visakem ta mõnda kaevu ja öelgem, et kuri loom sõi tema ära! Siis saame näha, mis ta unenäod tähendavad!”
\par 21 Kui Ruuben seda kuulis, siis ta tahtis teda nende käest päästa ja ütles: „Ärgem võtkem temalt hinge!”
\par 22 Ja Ruuben ütles neile: „Ärge valage verd, visake ta siia kõrbes olevasse kaevu, aga ärge pange oma kätt tema külge!” Sest ta tahtis tema päästa nende käest ja saata tagasi isa juurde.
\par 23 Ja kui Joosep tuli oma vendade juurde, siis kiskusid need Joosepil kuue seljast, kirju kuue, mis tal seljas oli,
\par 24 ning võtsid ja viskasid ta kaevu; aga kaev oli tühi, selles ei olnud vett.
\par 25 Seejärel nad istusid leiba võtma. Ja kui nad oma silmad üles tõstsid ja vaatasid, ennäe, siis tuli ismaeliitide karavan Gileadist. Nende kaamelid kandsid mitmesugust vaiku, palsamit ja lõhnaainest, ja nad olid sellega teel alla Egiptusesse.
\par 26 Ja Juuda ütles oma vendadele: „Mis kasu sellest on, kui me tapame oma venna ja katame kinni tema vere?
\par 27 Tulge, müüme tema ismaeliitidele, aga meie käed ärgu puudutagu teda, sest ta on meie lihane vend!” Ja ta vennad kuulasid teda.
\par 28 Kui siis Midjani mehed, kaupmehed, mööda läksid, tõmbasid nad Joosepi kaevust välja ja müüsid Joosepi kahekümne hõbetüki eest ismaeliitidele; ja need viisid Joosepi Egiptusesse.
\par 29 Kui Ruuben tuli tagasi kaevu juurde, vaata, siis ei olnud Joosepit enam kaevus. Siis ta käristas oma riided lõhki
\par 30 ja läks tagasi oma vendade juurde ning ütles: „Poissi ei ole enam! Ja mina, kuhu ma nüüd lähen?”
\par 31 Siis nad võtsid Joosepi kuue ja tapsid ühe siku ning kastsid kuue verre.
\par 32 Ja nad saatsid kirju kuue, tulid oma isa juurde ning ütlesid: „Selle me leidsime! Tunnista nüüd, kas see on su poja kuub või mitte?”
\par 33 Ja ta tundis selle ära ning ütles: „See on mu poja kuub! Kuri loom on ta ära söönud, Joosep on tõesti maha murtud!”
\par 34 Ja Jaakob käristas oma riided lõhki, kinnitas kotiriide niuete ümber ja leinas oma poega kaua aega.
\par 35 Kõik ta pojad ja tütred püüdsid teda trööstida, kuid ta ei lasknud ennast trööstida, vaid ütles: „Ma lähen tõesti leinates oma poja juurde hauda!” Ja tema isa nuttis teda taga.
\par 36 Aga midjanlased müüsid tema Egiptuses Pootifarile, vaarao hoovkondlasele, ta ihukaitse pealikule.

\chapter{38}

\par 1 Sel ajal läks Juuda ära oma vendade juurest ja siirdus ühe Adullami mehe juurde, kelle nimi oli Hiira.
\par 2 Ja Juuda nägi seal kaananlase, Suua-nimelise mehe tütart, võttis selle ning heitis ta juurde.
\par 3 Ja see jäi lapseootele ning tõi poja ilmale; ja ta pani sellele nimeks Eer.
\par 4 Ja tema jäi taas lapseootele ning tõi poja ilmale; ja ta pani sellele nimeks Oonan.
\par 5 Ja tema tõi veel ühe poja ilmale ja pani sellele nimeks Seela; ta oli Kesibis, kui ta selle ilmale tõi.
\par 6 Ja Juuda võttis Eerile, oma esmasündinule, naise nimega Taamar.
\par 7 Aga Eer, Juuda esmasündinu, oli Issanda silmis paha ja Issand laskis tema surra.
\par 8 Siis Juuda ütles Oonanile: „Heida oma venna naise juurde, ole temale mehe eest ja soeta oma vennale sugu!”
\par 9 Kuna aga Oonan teadis, et sugu ei pidanud saama temale, siis heites oma venna naise juurde ta hävitas oma seemne maha pillates, et mitte anda sugu oma vennale.
\par 10 Aga see, mis ta tegi, oli Issanda silmis paha ja ta laskis surra ka tema.
\par 11 Siis Juuda ütles oma miniale Taamarile: „Jää lesena oma isakotta elama, kuni mu poeg Seela on kasvanud suuremaks!„ Sest ta mõtles: ”Muidu sureb seegi nagu ta vennad.” Ja Taamar läks ning jäi oma isakotta elama.
\par 12 Mõne aja pärast suri Suua tütar, Juuda naine. Kui Juuda leinaaeg oli möödunud, siis ta läks üles Timnasse oma lammaste niitjate juurde, tema ja ta sõber Hiira, Adullami mees.
\par 13 Ja Taamarile anti teada ning öeldi: „Vaata, su äi läheb üles Timnasse lambaid niitma.”
\par 14 Siis ta võttis lesepõlve riided seljast ära, varjas ennast looriga ja kattis enese kinni ning istus Eenaimi väravasse, kust tee viib Timnasse, sest ta oli näinud, et Seela oli kasvanud suureks, teda aga ei olnud antud temale naiseks.
\par 15 Kui Juuda teda nägi, siis ta pidas teda hooraks, sest ta oli oma näo kinni katnud.
\par 16 Ja ta pöördus teelt tema poole ning ütles: „Lase ma heidan su juurde!„ Sest ta ei teadnud, et see oli tema minia. Aga too vastas: ”Mis sa mulle annad, kui sa heidad mu juurde?”
\par 17 Ta ütles: „Ma läkitan sulle karjast ühe sikutalle.„ Ja naine vastas: ”Jah, kui sa annad mulle pandi, seniks kui sa läkitad.”
\par 18 Siis ta küsis: „Mis võiks olla pandiks, mille ma pean sulle andma?„ Ja tema vastas: ”Su pitsat ja vöö ja kepp, mis sul käes on.” Ja ta andis need temale ning heitis ta juurde; ja naine jäi temast lapseootele.
\par 19 Siis ta tõusis ja läks ära ning võttis eneselt loori ja pani lesepõlve riided selga.
\par 20 Ja Juuda läkitas oma sõbra, Adullami mehega sikutalle, et võtta pant naise käest; aga see ei leidnud teda.
\par 21 Siis ta küsis meestelt seal paigas, öeldes: „Kus on see liiderlik naine, kes oli Eenaimi tee ääres?„ Aga need vastasid: ”Siin pole liiderlikku naist olnud.”
\par 22 Ja ta tuli tagasi Juuda juurde ning ütles: „Ma ei leidnud teda. Ja mehedki seal paigas ütlesid: Siin pole liiderlikku naist olnud.”
\par 23 Siis ütles Juuda: „Pidagu siis enesele, et me ei satuks pilke alla! Vaata, ma läkitasin selle siku, aga sina ei leidnud teda.”
\par 24 Aga kolme kuu pärast teatati Juudale ja öeldi: „Su minia Taamar on hooranud, ja vaata, ta on hooratööst jäänud ka lapseootele.„ Siis ütles Juuda: ”Tooge ta välja, et ta põletataks!”
\par 25 Kui ta välja toodi, siis ta läkitas sõna oma äiale: „Sellest mehest, kelle omad need on, olen ma lapseootel!„ Ja ta ütles: ”Tunnista nüüd, kelle see pitsat ja vöö ja kepp on!”
\par 26 Ja Juuda tundis need ära ning ütles: „Tema on minust õigem! Sest ma ei ole teda andnud oma pojale Seelale.” Ja ta ei ühtinud enam temaga.
\par 27 Aga sünnitamise ajal, vaata, olid ta ihus kaksikud.
\par 28 Ja kui ta sünnitas, sirutus üks käsi välja; ja aitajanaine võttis ning sidus käe ümber helepunase lõnga, öeldes: „See tuleb esmalt välja!”
\par 29 Aga kui ta oma käe tagasi tõmbas, vaata, siis tuli välja ta vend. Ja aitajanaine ütles: „Missuguse lõhe sa küll enesele oled rebestanud!” Ja temale pandi nimeks Perets.
\par 30 Ja pärast tuli välja tema vend, kelle käe ümber oli helepunane lõng; ja temale pandi nimeks Serah.

\chapter{39}

\par 1 Ja Joosep viidi alla Egiptusesse; ja egiptlane Pootifar, vaarao hoovkondlane, ihukaitsepealik, ostis tema ismaeliitidelt, kes olid ta sinna toonud.
\par 2 Aga Issand oli Joosepiga ja tema oli mees, kellel kõik asjad korda läksid; ja ta elas oma isanda, egiptlase kojas.
\par 3 Ka tema isand nägi, et Issand oli temaga ja et kõik, mis ta tegi, laskis Issand tema käes korda minna.
\par 4 Ja Joosep leidis armu tema silmis ning teenis teda; ja ta pani tema oma koja üle, ja kõik, mis tal oli, ta andis tema kätte.
\par 5 Ja sellest ajast peale, kui ta oli pannud tema oma koja üle, ja kõige üle, mis tal oli, Issand õnnistas egiptlase koda Joosepi pärast, ja Issanda õnnistus oli kõigega, mis tal oli kojas ja väljal.
\par 6 Ja ta jättis kõik, mis tal oli, Joosepi kätte ega teadnud millestki, kui ainult leivast, mida ta sõi. Joosep oli jumekas ja ilusa välimusega.
\par 7 Nende asjaolude tõttu sündis, et ta isanda naine Joosepile silma heitis ja ütles: „Maga mu juures!”
\par 8 Aga ta keeldus ja ütles oma isanda naisele: „Vaata, mu isand ise ei tea, mis kojas on, ja kõik, mis tal on, ta on andnud minu kätte.
\par 9 Ei ole ükski selles kojas suurem minust ja tema ei ole mulle keelanud midagi muud kui sind, sest sa oled tema naine. Kuidas tohiksin siis teha seda suurt kurja ja pattu oma Jumala vastu?”
\par 10 Ja kuigi naine rääkis Joosepile iga päev, ei kuulanud see teda, et ta magaks ta kõrval ja oleks ta juures.
\par 11 Aga ühel päeval, kui Joosep tuli koju oma toimetusi tegema ja ühtegi inimest pererahvast ei olnud kodus,
\par 12 haaras ta kinni tema kuuest, öeldes: „Maga mu juures!” Aga Joosep jättis oma kuue tema kätte, põgenes ja läks õue.
\par 13 Ja kui naine nägi, et ta oma kuue oli jätnud tema kätte ja oli põgenenud õue,
\par 14 siis ta hüüdis pererahvast ja rääkis neile, öeldes: „Näete, ta on toonud meie juurde heebrea mehe meid ära naerma! Ta tuli mu juurde, et minuga magada, aga ma hüüdsin suure häälega,
\par 15 ja kui ta kuulis, et ma häält tõstsin ja hüüdsin, siis ta jättis oma kuue mu juurde, põgenes ja läks õue.”
\par 16 Ja ta pani tema kuue enese kõrvale seniks, kui ta isand koju tuli.
\par 17 Ja ta rääkis temale needsamad sõnad, öeldes: „Heebrea sulane, kelle sa oled meie juurde toonud, tuli mu juurde mind ära naerma.
\par 18 Aga kui ma häält tõstsin ja hüüdsin, siis ta jättis oma kuue mu juurde ja põgenes õue.”
\par 19 Ja kui ta isand kuulis oma naise jutustust, mida see temale rääkis, öeldes: „Nõndaviisi on su sulane minuga teinud”, siis ta viha süttis põlema.
\par 20 Ja Joosepi isand võttis tema ja pani vangihoonesse, sinna paika, kus kuninga vange kinni peeti. Siis oli ta seal vangihoones.
\par 21 Aga Issand oli Joosepiga ja pööras tema poole oma helduse ning tegi, et ta sai vangihoone ülema soosikuks.
\par 22 Ja vangihoone ülem andis Joosepi kätte kõik vangid, kes olid vangihoones, ja kõik, mis seal pidi tehtama, tegi tema.
\par 23 Vangihoone ülem ei vaadanud millegi järele, mis oli tema käes, sest Issand oli temaga ja mis ta tegi, Issand laskis korda minna.

\chapter{40}

\par 1 Ja pärast seda lugu sündis, et Egiptuse kuninga joogikallaja ja pagar eksisid oma isanda, Egiptuse kuninga vastu.
\par 2 Ja vaarao sai väga kurjaks oma kahe hoovkondlase peale, joogikallajate ülema ja pagarite ülema peale.
\par 3 Ja ta andis nad vahi alla ihukaitsepealiku kotta, vangihoonesse, samasse paika, kus Joosep kinni oli.
\par 4 Ja ihukaitsepealik pani Joosepi nende juurde, et ta neid teeniks; ja nad olid vahi all kaua aega.
\par 5 Ja need mõlemad nägid ühel ööl unenäo, kumbki oma unenäo, kumbki oma tähendusega unenäo, Egiptuse kuninga joogikallaja ja pagar, kes vangihoones kinni olid.
\par 6 Kui Joosep tuli hommikul nende juurde ja nägi neid, vaata, siis olid nad nukra näoga.
\par 7 Ja ta küsis vaarao hoovkondlastelt, kes olid koos temaga vahi all ta isanda kojas, öeldes: „Mispärast on teil täna nii kurvad näod?”
\par 8 Ja nad vastasid temale: „Me nägime unenägusid, aga ei ole kedagi, kes need seletaks.„ Ja Joosep ütles neile: ”Eks seletused ole Jumala käes? Siiski jutustage mulle!”
\par 9 Ja joogikallajate ülem jutustas oma unenäo Joosepile ning ütles temale: „Mu unenäos oli nõnda: vaata, mu ees oli viinapuu
\par 10 ja viinapuul oli kolm oksa; see lehistus, õitses ja marjakobarad valmisid;
\par 11 mul oli käes vaarao karikas ja ma võtsin viinamarju ja pigistasin neid vaarao karikasse ja andsin karika vaarao kätte.”
\par 12 Ja Joosep ütles temale: „Selle seletus on niisugune: kolm oksa on kolm päeva.
\par 13 Enne kui kolm päeva on möödunud, tõstab vaarao su pea üles ja paneb sind tagasi su ametisse ja sa annad vaaraole karikat kätte endist viisi, nagu siis, kui olid ta joogikallaja.
\par 14 Aga pea mind meeles, kui su käsi hästi käib, ja tee mulle siis head ning tuleta mind vaaraole meelde ja vii mind siit hoonest välja,
\par 15 sest mind on vargsel viisil varastatud heebrealaste maalt ja ma pole siingi teinud midagi, et mind vangiurkasse pandaks.”
\par 16 Kui pagarite ülem nägi, et ta oli hästi seletanud, siis ta ütles Joosepile: „Ka mina nägin und, ja vaata, mu pea kohal oli kolm punutud korvi.
\par 17 Kõige ülemises korvis oli kõiksugust vaaraole valmistatud pagarirooga, aga linnud sõid selle mu peapealsest korvist.”
\par 18 Ja Joosep vastas ning ütles: „Selle seletus on niisugune: kolm korvi on kolm päeva.
\par 19 Enne kui kolm päeva on möödunud, võtab vaarao sul pea otsast ja poob sind puusse ning linnud söövad liha su pealt.”
\par 20 Ja kolmandal päeval, vaarao sünnipäeval, kui ta tegi kõigile oma sulastele suure peo, sündis, et ta tõstis üles joogikallajate ülema pea, samuti pagarite ülema pea oma sulaste seast,
\par 21 ja pani joogikallajate ülema tema joogikallajaametisse vaaraole karikat kätte ulatama,
\par 22 aga pagarite ülema ta laskis puua, nõnda nagu Joosep neile oli seletanud.
\par 23 Ent joogikallajate ülem ei pidanud Joosepit meeles, vaid unustas tema ära.

\chapter{41}

\par 1 Ja kahe aasta pärast sündis, et vaarao nägi und, ja vaata, ta seisis Niiluse jõe ääres.
\par 2 Ja näe, jõest tõusis seitse lehma, ilusad näha ja lihavad liha poolest, ja sõid roostikus.
\par 3 Ja vaata, teist seitse lehma tõusis nende järel jõest, pahad näha ja lahjad liha poolest, ja need seisid teiste lehmade kõrval jõe kaldal.
\par 4 Ja lehmad, kes olid pahad näha ja lahjad liha poolest, sõid ära need seitse lehma, kes olid ilusad näha ja lihavad. Siis vaarao ärkas üles.
\par 5 Aga ta jäi magama ja nägi teist puhku und, ja vaata, seitse viljapead tõusis ühest kõrrest, jämedad ja head.
\par 6 Ja vaata, seitse peenikest ja hommikutuulest kõrvetatud viljapead võrsus nende järel.
\par 7 Ja peenikesed viljapead neelasid ära need seitse jämedat ja täit viljapead. Ja vaarao ärkas üles, ja vaata, see oli olnud unenägu!
\par 8 Aga hommikul oli ta vaim rahutu ja ta läkitas järele ning kutsus kõik Egiptuse ennustajad ja targad, ja vaarao jutustas neile oma unenäo, aga ükski ei suutnud seda vaaraole seletada.
\par 9 Siis rääkis joogikallajate ülem vaaraoga, öeldes: „Täna meenub mulle mu patt.
\par 10 Vaarao sai väga kurjaks oma sulaste peale ja pani mind vahi alla ihukaitsepealiku kotta, minu ja pagarite ülema.
\par 11 Ja me nägime ühel ööl unenägu, mina ja tema, kumbki nägime unenäo, millel oli oma tähendus.
\par 12 Ja sealsamas oli meiega heebrea noormees, ihukaitsepealiku sulane, ja me jutustasime temale ja tema seletas meile meie unenäod, kummalegi ta unenäo tähenduse.
\par 13 Ja nõnda nagu ta meile seletas, nõnda sündis: mina sain tagasi oma ametisse ja teine poodi üles.”
\par 14 Siis vaarao läkitas järele ja laskis kutsuda Joosepi; ja Joosep toodi kiiresti vangiurkast välja; ta ajas habeme, vahetas riided ja tuli vaarao juurde.
\par 15 Ja vaarao ütles Joosepile: „Ma nägin unenäo, kuid keegi ei oska seda seletada. Aga ma olen kuulnud sinust räägitavat, et kui sina unenägu kuuled, siis sa seletad selle.”
\par 16 Ja Joosep vastas vaaraole, öeldes: „Mitte mina! Küllap Jumal annab vaaraole õige vastuse!”
\par 17 Ja vaarao jutustas Joosepile: „Mu unenäos oli nõnda: vaata, ma seisin Niiluse jõe kaldal.
\par 18 Ja näe, jõest tõusis seitse lehma, lihavad liha poolest ja ilusad näha, ja sõid roostikus.
\par 19 Ja vaata, teist seitse lehma tõusis nende järel, väetid ja väga pahad näha ning lahjad liha poolest. Ei ole ma kogu Egiptusemaal näinud selliseid pahu.
\par 20 Ja lahjad ning pahad lehmad sõid ära need seitse esimest lihavat lehma.
\par 21 Ja need läksid nende kõhtu, aga ei olnud tundagi, et need olid läinud nende kõhtu ja nende välimus oli paha nagu ennegi. Ja ma ärkasin üles.
\par 22 Siis ma nägin und, ja vaata, seitse viljapead tõusis ühest kõrrest, täis ja head.
\par 23 Ja vaata, seitse kuiva, peenikest, hommikutuulest kõrvetatud viljapead võrsus nende järel.
\par 24 Ja peenikesed viljapead neelasid ära need seitse head viljapead. Ma olen seda rääkinud oma ennustajaile, aga ükski ei oska mulle seletada.”
\par 25 Ja Joosep ütles vaaraole: „Vaarao unenäod tähendavad ühte ja sedasama. Jumal on vaaraole teada andnud, mida ta kavatseb teha.
\par 26 Seitse head lehma on Seitse aastat ja Seitse head viljapead on Seitse aastat; unenägudel on sama tähendus.
\par 27 Ja seitse lahjat ja paha lehma, kes tõusid nende järel, on seitse aastat, ja seitse peenikest, hommikutuulest kõrvetatud viljapead on seitse nälja-aastat.
\par 28 See ongi see asi, mis ma vaaraole rääkisin: mida Jumal kavatseb teha, seda on ta vaaraole näidanud.
\par 29 Vaata, tuleb seitse aastat, millal on suur küllus kogu Egiptusemaal.
\par 30 Aga neile järgneb seitse nälja-aastat, mil Egiptusemaal ununeb kõik küllus ja näljahäda ulatub üle kogu maa.
\par 31 Maal ei ole enam küllust tundagi nälja tõttu, mis tuleb selle järel, sest see on väga kange.
\par 32 Ja kuna vaaraol on unenägu kaks korda kordunud, siis on see asi Jumala poolt kindel ja Jumal laseb selle varsti sündida.
\par 33 Ja nüüd vaarao vaadaku välja üks mõistlik ja tark mees ja seadku Egiptusemaa üle.
\par 34 Vaarao tehku nõnda: pangu ülevaatajad maale ja lasku Egiptusemaad maksta viiendikku neil seitsmel külluseaastal.
\par 35 Ja kogutagu kõik nende tulevaste heade aastate toidus ja varutagu vilja linnadesse vaarao käe alla toiduseks ning säilitatagu seda.
\par 36 See vili olgu maale tagavaraks seitsmeks nälja-aastaks, mis tulevad Egiptusemaale, et maad ei tabaks näljaajal hukatus.”
\par 37 See kõne oli hea vaarao silmis ja kõigi ta sulaste silmis.
\par 38 Ja vaarao ütles oma sulastele: „Kas leiame veel sellise mehe, kelles on Jumala Vaim?”
\par 39 Ja vaarao ütles Joosepile: „Et Jumal on sulle kõike seda teada andnud, siis ei ole keegi nii mõistlik ja tark kui sina!
\par 40 Sina pead olema mu koja üle ja kogu mu rahvas suudelgu sind suu peale. Ainult aujärje poolest tahan ma olla sinust suurem!”
\par 41 Ja vaarao ütles Joosepile: „Vaata, ma olen pannud su kogu Egiptusemaa üle!”
\par 42 Ja vaarao võttis sõrmest pitserisõrmuse ning pani Joosepi sõrme; ja ta pani temale kallid linased riided selga ja riputas kuldkee kaela.
\par 43 Ja ta laskis teda sõita oma teises tõllas, mis tal oli, ja hüüda tema ees: „Põlvili!” Ja ta pani tema kogu Egiptusemaa üle.
\par 44 Ja vaarao ütles Joosepile: „Mina olen vaarao, aga ilma sinu tahtmata ei tohi keegi tõsta kätt ega jalga kogu Egiptusemaal.”
\par 45 Ja vaarao pani Joosepile nimeks Safnat-Paneah ja andis temale naiseks Asnati, Ooni preestri Pooti-Fera tütre; ja Joosep läks välja Egiptusemaale.
\par 46 Joosep oli kolmekümneaastane, kui ta astus vaarao, Egiptuse kuninga ette; ja Joosep läks ära vaarao juurest ning käis läbi kogu Egiptusemaa.
\par 47 Ja maa kandis seitsmel külluseaastal ühest ivast peotäie.
\par 48 Ja ta kogus kõik põllusaagi seitsmel aastal Egiptusemaal ja talletas selle linnadesse; igasse linna andis ta ümbruskonna põldude saagi.
\par 49 Ja Joosep kuhjas väga palju vilja, nagu liiva mere ääres, kuni lakati seda mõõtmast, sest see ei olnud enam mõõdetav.
\par 50 Ja Joosepile sündis enne nälja-aasta tulekut kaks poega. Need tõi temale ilmale Asnat, Ooni preestri Pooti-Fera tütar.
\par 51 Ja Joosep pani esmasündinule nimeks Manasse, sest ta ütles: „Jumal on mind lasknud unustada kõik mu vaeva ja kogu mu isakoja.”
\par 52 Ja teisele ta pani nimeks Efraim, sest ta ütles: „Jumal on mind teinud viljakaks mu viletsusemaal!”
\par 53 Kui seitse külluseaastat, mis Egiptusemaale tulid, lõppesid,
\par 54 siis hakkasid tulema seitse nälja-aastat, nõnda nagu Joosep oli öelnud; ja nälg oli kõigis maades, ent kogu Egiptusemaal oli leiba.
\par 55 Aga kui kogu Egiptusemaa hakkas tundma nälga, siis rahvas kisendas vaarao poole leiva pärast; ja vaarao ütles kõigile egiptlastele: „Minge Joosepi juurde! Mis tema teile ütleb, seda tehke!”
\par 56 Ja nälg oli üle kogu maa ja Joosep avas kõik viljaaidad, mis olid nende juures, ning müüs egiptlastele vilja, sest nälg võttis Egiptusemaal võimust.
\par 57 Ja kõigist maadest tuldi Egiptusesse Joosepilt vilja ostma, sest nälg oli võtnud võimust kõigis maades.

\chapter{42}

\par 1 Kui Jaakob kuulis, et Egiptuses oli vilja, siis Jaakob ütles oma poegadele: „Miks te üksteisele otsa vaatate?”
\par 2 Ja ta ütles: „Vaata, ma olen kuulnud, et Egiptuses on vilja. Minge sinna ja ostke meile sealt, et jääksime elama ega sureks!”
\par 3 Siis läks kümme Joosepi venda Egiptusest vilja ostma.
\par 4 Aga Benjamini, Joosepi venda, Jaakob ei läkitanud koos vendadega, sest ta ütles: „Et temale õnnetust ei juhtuks!”
\par 5 Ja Iisraeli pojad tulid vilja ostma teiste tulijate seltsis, sest Kaananimaal oli nälg.
\par 6 Joosep oli maal valitsejaks; tema müüs vilja kogu maa rahvale. Ja Joosepi vennad tulid ning kummardasid tema ette silmili maha.
\par 7 Kui Joosep nägi oma vendi, siis ta tundis nad ära, aga tegi ennast neile võõraks ja rääkis nendega karmilt ning küsis neilt: „Kust te tulete?„ Ja nad vastasid: ”Kaananimaalt, et leiba osta.”
\par 8 Joosep tundis oma vendi, aga nemad ei tundnud teda.
\par 9 Siis meenusid Joosepile unenäod, mis ta neist unes oli näinud, ja ta ütles neile: „Te olete maakuulajad. Olete tulnud vaatama, kust maa lahti on!”
\par 10 Aga nemad ütlesid: „Ei, mu isand, su sulased on tulnud leiba ostma.
\par 11 Me kõik oleme ühe mehe pojad, oleme ausad mehed, su sulased ei ole maakuulajad.”
\par 12 Tema aga ütles neile: „Ei, küllap olete tulnud vaatama, kust maa lahti on!”
\par 13 Siis nad vastasid: „Meie, su sulased, olime kaksteist venda, ühe mehe pojad Kaananimaalt. Ja vaata, noorim on praegu Meie isa juures, aga ühte ei ole enam olemas.”
\par 14 Aga Joosep ütles neile: „See on nõnda, nagu ma teile olen rääkinud ja öelnud: te olete maakuulajad!
\par 15 Selles asjas ma katsun teid läbi: nii tõesti kui vaarao elab, ei pääse te siit minema enne, kui teie noorim vend on siia tulnud!
\par 16 Läkitage eneste hulgast üks, et ta tooks teie venna. Teid aga jäetakse vangi ja teie kõnesid uuritakse, kas neis on tõtt. Muidu olete, nii tõesti kui vaarao elab, maakuulajad!”
\par 17 Ja ta pani nad üheskoos kolmeks päevaks vahi alla.
\par 18 Ja kolmandal päeval ütles Joosep neile: „Tehke seda, siis jääte elama! Mina olen jumalakartlik:
\par 19 kui olete ausad mehed, jäägu üks teie vendadest aheldatuna hoonesse, kus te olite vahi all, aga teised minge viige vilja näljahäda pärast teie kodudes
\par 21 Ja nad ütlesid üksteisele: „Me oleme tõesti süüdi oma venna pärast, kelle hinge kitsikust me nägime, kui ta palus meilt armu ja meie ei võtnud teda kuulda. Sellepärast on see kitsikus meile tulnud!”
\par 22 Ja Ruuben vastas neile, öeldes: „Eks ma rääkinud teile ja öelnud: Ärge tehke pattu poisi vastu! Aga teie ei kuulanud. Vaata, nüüd nõutaksegi tema verd!”
\par 23 Aga nad ei teadnud, et Joosep seda mõistis, sest ta rääkis nendega tõlgi kaudu.
\par 24 Ja ta pöördus ära nende juurest ning nuttis; siis ta tuli tagasi nende juurde ja rääkis nendega ning võttis nende hulgast Siimeoni ja laskis tema nende silme ees vangistada.
\par 25 Ja Joosep käskis täita nende kotid viljaga, panna igaühe raha tagasi ta kotti ja anda neile teemoona; ja neile tehti nõnda.
\par 26 Siis nad tõstsid vilja eeslite selga ja läksid sealt ära.
\par 27 Kui üks oma koti avas, et öömajal eeslile toitu anda, siis ta nägi oma raha, ja vaata, see oli koti suus.
\par 28 Ja ta ütles oma vendadele: „Mu raha on tagasi antud, ja vaata, see on mu kotis!„ Siis kadus nende julgus ja nad värisesid, öeldes isekeskis: ”Miks on Jumal meile seda teinud?”
\par 29 Ja nad tulid oma isa Jaakobi juurde Kaananimaale ning andsid temale teada kõik, mis neile oli juhtunud, öeldes:
\par 30 „See mees, maa isand, rääkis meiega karmilt ja pidas meid salajasiks maakuulajaiks.
\par 31 Aga me ütlesime temale: Me oleme ausad mehed, me ei ole maakuulajad.
\par 32 Meid oli kaksteist venda, oma isa pojad. Ühte ei ole enam ja noorim on praegu meie isa juures Kaananimaal.
\par 33 Ja see mees, maa isand, ütles meile: Sellest ma saan teada, kas te olete ausad: jätke üks oma vendadest minu juurde ja võtke vilja näljahäda pärast teie kodudes ja minge
\par 34 ning tooge oma noorim vend minu juurde, et saaksin teada, et te ei ole maakuulajad, vaid olete ausad mehed. Siis ma annan teile venna tagasi ja te võite maal liikuda vabalt.”
\par 35 Ja kui nad oma kotte tühjendasid, vaata, siis oli igaühe rahakukkur ta kotis! Ja oma rahakukruid nähes hakkasid nemad ja nende isa kartma.
\par 36 Ja nende isa Jaakob ütles neile: „Te võtate minult lapsed! Joosepit ei ole enam ja Siimeoni ei ole enam ja Benjamini tahate ka võtta. See kõik on tulnud minu peale!”
\par 37 Aga Ruuben rääkis oma isale, öeldes: „Võid surmata mu mõlemad pojad, kui ma ei too teda sinu juurde! Anna ta minu hooleks ja ma toon ta tagasi sinu juurde!”
\par 38 Kuid tema ütles: „Minu poeg ei lähe koos teiega alla, sest ta vend on surnud ja tema üksi on üle jäänud. Kui temaga teel, mida te käite, õnnetus juhtub, siis saadate mu hallid juuksed murega hauda!”

\chapter{43}

\par 1 Aga maal oli kange nälg.
\par 2 Ja kui nad olid ära söönud vilja, mis nad Egiptusest olid toonud, siis nende isa ütles neile: „Minge ostke meile pisut leiba!”
\par 3 Ja Juuda vastas temale, öeldes: „See mees kinnitas meile väga ja ütles: Te ei saa näha mu nägu, kui teie vend ei ole koos teiega.
\par 4 Kui sa nüüd läkitad meie venna koos meiega, siis läheme ja ostame sulle leiba.
\par 5 Aga kui sa teda ei läkita, siis me ei lähe, sest see mees ütles meile: Te ei saa näha mu nägu, kui teie vend ei ole koos teiega.”
\par 6 Ja Iisrael ütles: „Miks te tegite mulle seda kurja ja andsite mehele teada, et teil on veel üks vend?”
\par 7 Ja nad vastasid: „See mees päris väga meie ja meie suguseltsi järele, küsides: Kas teie isa elab veel? On teil veel mõni vend? Ja me andsime temale teada, nagu asi on. Kas me võisime teada, et ta ütleb: Tooge oma vend siia?”
\par 8 Ja Juuda ütles oma isale Iisraelile: „Saada poiss minuga ja me võtame kätte ning läheme, et jääksime elama ega sureks, ei meie ega sina ega meie väetid lapsed.
\par 9 Mina olen temale käemeheks, nõua teda minult. Kui ma ei too teda tagasi sinu juurde ega sea sinu palge ette, siis jään su ees alatiseks süüdlaseks.
\par 10 Tõesti, kui me ei oleks viivitanud, oleksime nüüd juba teist korda tagasi tulnud.”
\par 11 Siis ütles neile Iisrael, nende isa: „Kui see nõnda on, siis tehke seda! Võtke maa parimast oma kottidesse ja viige sellele mehele meeleheaks pisut palsamit ja pisut mett, kalleid rohte ja mürri, pähkleid ja mandleid.
\par 12 Ja võtke kaasa kahekordne raha; ka see raha, mis teie kottide suus tagasi tuli, võtke jälle kaasa - vahest oli see eksitus.
\par 13 Võtke ka oma vend, asuge teele ja minge tagasi selle mehe juurde!
\par 14 Kõigeväeline Jumal lasku teid leida halastust selle mehe ees, et ta teiega ära saadaks teie teise venna ja Benjamini! Aga kui jään lastest ilma, siis jään.”
\par 15 Ja mehed võtsid selle kingituse, võtsid kaasa kahekordse raha ja Benjamini, asusid teele ja läksid alla Egiptusesse ning astusid Joosepi ette.
\par 16 Kui Joosep nägi, et Benjamin oli nendega kaasas, siis ta ütles oma kojaülemale: „Vii need mehed mu kotta ja tapa tapaveis ning valmista see, sest need mehed söövad minuga lõunat!”
\par 17 Ja mees tegi, nagu Joosep oli öelnud; ja mees viis mehed Joosepi kotta.
\par 18 Aga mehed kartsid, kui neid Joosepi kotta viidi, ja ütlesid: „Meid viiakse raha pärast, mis eelmisel korral meie kottides tagasi tuli, et veeretada meie peale süüd ja kippuda meile kallale ning võtta meid orjadeks ja omandada meie eeslid!”
\par 19 Ja nad astusid mehe juurde, kes oli Joosepi kojaülem, ja rääkisid temaga koja ukse ees
\par 20 ning ütlesid: „Oh mu isand, me kord juba käisime siin leiba ostmas.
\par 21 Aga kui me seejärel jõudsime öömajale ja tegime oma kotid lahti, vaata, siis oli igamehe raha ta koti suus, meie raha selle täiskaalus. Me tõime selle nüüd tagasi.
\par 22 Ja me tõime ka teise raha leiva ostmiseks enestega kaasa. Me ei tea, kes oli pannud meie raha meie kottidesse.”
\par 23 Aga tema ütles: „Rahustuge, ärge kartke! Teie Jumal ja teie isa Jumal on pannud teile varanduse kottidesse. Teie raha ma olen saanud.” Ja ta tõi Siimeoni välja nende juurde.
\par 24 Siis mees viis mehed Joosepi kotta ja andis neile vett ning nad pesid oma jalgu; ja ta andis toitu nende eeslitele.
\par 25 Ja nad seadsid kingitused valmis, kuni Joosep tuli lõunale, sest nad olid kuulnud, et nad seal pidid leiba võtma.
\par 26 Kui Joosep koju tuli, siis nad viisid temale kotta kingituse, mis neil kaasas oli, ja kummardasid maha tema ette.
\par 27 Ja tema küsis neilt, kuidas nende käsi käib, ja ütles: „Kas teie vana isa, kellest te rääkisite, käsi käib hästi? On ta veel elus?”
\par 28 Ja nad vastasid: „Sinu sulase, meie isa käsi käib hästi, ta elab veel.” Ja nad kummardasid ning heitsid maha.
\par 29 Kui ta oma silmad üles tõstis ja nägi oma venda Benjamini, oma ema poega, siis ta küsis: „On see teie noorim vend, kellest te mulle rääkisite?„ Ja ta ütles: ”Jumal olgu sulle armuline, mu poeg!”
\par 30 Aga Joosep vaikis äkki, sest ta oli südamest liigutatud oma venna pärast ja otsis võimalust nutmiseks. Ta läks ühte siseruumi ja nuttis seal.
\par 31 Ja kui ta oma silmi oli pesnud, siis ta tuli välja, valitses enese üle ja ütles: „Pange toit lauale!”
\par 32 Ja temale pandi eraldi ja neile eraldi, ja egiptlastele, kes tema juures sõid, eraldi, sest egiptlased ei söö leiba üheskoos heebrealastega - see on egiptlastele jäledus.
\par 33 Ja nad pandi istuma temaga vastamisi, esmasündinu esimesena ja noorim viimasena, ja mehed panid seda isekeskis imeks.
\par 34 Ja ta laskis tõsta rooga enda eest nende ette, aga Benjamini osa oli viis korda suurem kui kõigi teiste osad. Ja nad jõid koos temaga ning jäid joobnuks.

\chapter{44}

\par 1 Siis ta käskis oma kojaülemat, öeldes: „Täida meeste kotid viljaga, nii palju kui nad jaksavad kanda, ja pane igamehe raha ta koti suhu.
\par 2 Ja minu karikas, see hõbekarikas, pane noorima venna koti suhu koos ta viljarahaga!” Ja ta tegi, nagu Joosep käskis.
\par 3 Hommikul, kui läks valgeks, lasti minema mehed ja nende eeslid.
\par 4 Kui nad olid linnast lahkunud ega olnud veel kaugele jõudnud, ütles Joosep oma kojaülemale: „Võta kätte, aja mehi taga, ja kui oled nad tabanud, siis ütle neile: Miks tasute head kurjaga?
\par 5 Eks see ole see, millest mu isand joob ja millest ta endeid otsib? Olete käitunud halvasti, et tegite nõnda!”
\par 6 Ja kui ta nad tabas, siis ta rääkis neile needsamad sõnad.
\par 7 Aga nad vastasid temale: „Miks räägib mu isand nõndaviisi? Su sulastele oleks teotuseks nõnda teha.
\par 8 Vaata, raha, mis me leidsime oma kottide suust, tõime sulle tagasi Kaananimaalt. Kuidas võiksime siis varastada su isanda kojast hõbedat või kulda?
\par 9 Kelle juurest su sulaste hulgast see leitakse, surgu, ja me teised jääme su isandale orjadeks!”
\par 10 Siis ta ütles: „Olgu see nüüd ka nõnda teie sõnade järgi! Kelle juurest see leitakse, jäägu mulle sulaseks, ja te teised olete süüta!”
\par 11 Ja nad tõttasid ning tõstsid igaüks oma koti maha ja iga mees tegi oma koti lahti.
\par 12 Ja ta otsis nad läbi, alates vanemast ja lõpetades nooremaga - ja karikas leiti Benjamini kotist.
\par 13 Siis nad käristasid oma riided lõhki, iga mees pani oma eeslile koorma selga ja nad läksid tagasi linna.
\par 14 Ja Juuda läks koos oma vendadega Joosepi kotta, kes oli alles seal, ja nad heitsid maha tema ette.
\par 15 Ja Joosep ütles neile: „Mis tegu see on, mis te olete teinud? Eks te tea, et niisugune mees nagu mina oskab endeid seletada?”
\par 16 Siis vastas Juuda: „Mida me oma isandale ütleme? Kuidas peame rääkima ja kuidas endid õigustama? Jumal on avastanud su sulaste süü. Vaata, me jääme oma isandale sulaseiks, niihästi meie kui see, kelle käest karikas leiti.”
\par 17 Aga ta ütles: „Ei tule mulle mõttessegi seda teha! Mees, kelle käest karikas leiti, peab jääma mulle sulaseks, te teised aga minge rahuga oma isa juurde!”
\par 18 Siis astus Juuda tema juurde ja ütles: „Mu isand, luba ometi oma sulast rääkida üks sõna mu isanda kõrva ees ja su viha ärgu süttigu põlema oma sulase vastu, sest sina oled nagu vaarao ise!
\par 19 Mu isand küsis oma sulaseilt, öeldes: On teil isa või mõni vend?
\par 20 Ja meie vastasime oma isandale: Meil on vana isa ja üks nooruk, ta vanas eas sündinud, alles noor; aga selle vend on surnud ja ta on üksi oma emast järele jäänud, ja ta isa armastab teda.
\par 21 Ja sa ütlesid oma sulaseile: Tooge ta minu juurde, et ma näeksin teda oma silmaga!
\par 22 Ja me vastasime oma isandale: Nooruk ei või isast lahkuda. Kui ta lahkuks oma isast, siis see sureks.
\par 23 Ja sa ütlesid oma sulaseile: Kui teie noorim vend ei tule koos teiega, siis te ei saa enam näha mu nägu.
\par 24 Ja kui me läksime su sulase, meie isa juurde ja andsime temale teada oma isanda sõnad,
\par 25 siis ütles meie isa: Minge jälle tagasi, ostke meile pisut leiba!
\par 26 Aga me vastasime: Me ei või minna. Kui meie noorim vend on koos meiega, siis me läheme, sest me ei saa näha selle mehe nägu, kui meie noorim vend ei ole koos meiega.
\par 27 Ja su sulane, minu isa, ütles meile: Te teate, et mu naine on mulle kaks poega ilmale toonud.
\par 28 Aga üks läks mu juurest ära ja ma ütlesin: Ta on tõesti maha murtud! Ja ma pole teda tänini enam näinud.
\par 29 Kui te nüüd võtate ka selle mu silma alt ja temaga juhtub õnnetus, siis saadate mu hallid juuksed kurvastusega hauda.
\par 30 Ja kui ma nüüd tuleksin su sulase, oma isa juurde ja koos meiega ei oleks poiss, kelle hingesse tema hing on kiindunud,
\par 31 ja kui ta siis näeb, et poissi ei ole, ta sureb ja su sulased on sinu sulase, meie isa hallid juuksed saatnud murega hauda,
\par 32 sest su sulane on poisile käemeheks mu isa juures, kuna ma olen öelnud: Kui ma ei too teda sinu juurde, siis ma jään oma isa ees alatiseks süüdlaseks.
\par 33 Ja nüüd siis jäägu su sulane poisi asemel sulaseks mu isandale ja poiss mingu koos oma vendadega!
\par 34 Sest kuidas võiksin minna oma isa juurde, kui poiss ei oleks koos minuga? Et ma ei näeks seda õnnetust, mis juhtub mu isaga!”

\chapter{45}

\par 1 Siis Joosep ei suutnud enam enese üle valitseda kõigi nende ees, kes ta juures seisid, ja ta hüüdis: „Minge kõik mu juurest ära!” Ja ükski ei seisnud tema juures, kui Joosep ennast vendadele tunda andis.
\par 2 Ja ta puhkes valjusti nutma, nii et Egiptus seda kuulis ja kuulis vaarao koda.
\par 3 Ja Joosep ütles oma vendadele: „Mina olen Joosep! Kas mu isa veel elab?” Aga vennad ei suutnud temale vastata, nõnda hirmunud olid nad tema palge ees.
\par 4 Siis ütles Joosep oma vendadele: „Astuge ligemale!” Ja nad astusid ligemale ning ta ütles: ”Mina olen Joosep, teie vend, kelle te müüsite Egiptusesse.
\par 5 Aga nüüd ärge kurvastage ja ärgu süttigu teil isekeskis viha, et müüsite mind siia, sest teie elu säilitamiseks läkitas Jumal mind eele.
\par 6 Sest kaks aastat on olnud nüüd nälg maal ja on veel viis aastat, mil ei ole kündi ega lõikust.
\par 7 Seepärast Jumal läkitas mind teie eele kindlustama teile järeltulijaid maa peal ja hoidma teid elus, pääsemiseks paljudele.
\par 8 Niisiis ei ole teie mind läkitanud siia, vaid Jumal, ja tema on mind pannud vaaraole isaks ja isandaks kogu ta kojale ning valitsejaks kogu Egiptusemaale.
\par 9 Tõtake ja minge mu isa juurde ning öelge temale: Nõnda ütleb su poeg Joosep: Jumal on mind pannud isandaks kogu Egiptusele. Tule minu juurde, ära viivita!
\par 10 Sa võid elada Gooseni maakonnas ja olla mu läheduses, sina ja su pojad ja su poegade pojad, ja su lambad, kitsed ja veised ja kõik, mis sul on.
\par 11 Ma hoolitsen seal sinu eest, kuna veel viis aastat on näljahäda, et ei jääks vaeseks sina ega su pere ega keegi, kes sul on.
\par 12 Ja vaata, te näete oma silmaga, samuti näeb mu vend Benjamin oma silmaga, et ma tõepoolest ise teiega räägin.
\par 13 Jutustage mu isale kõigest minu aust Egiptuses ja kõigest, mida olete näinud, ja tõtake ning tooge mu isa siia!”
\par 14 Siis ta langes oma vennale Benjaminile kaela ja nuttis, ja Benjamin nuttis tema kaela ümber.
\par 15 Ja ta andis suud kõigile oma vendadele ning nuttis üheskoos nendega; ja seejärel ta vennad rääkisid temaga.
\par 16 Kuuldus kostis ka vaarao kotta, et Joosepi vennad olid tulnud, ja see oli hea vaarao silmis ja tema sulaste silmis.
\par 17 Ja vaarao ütles Joosepile: „Ütle oma vendadele: Tehke nõnda: koormake oma veoloomad ja minge Kaananimaale,
\par 18 võtke oma isa ja oma pered ning tulge minu juurde, siis ma annan teile Egiptusemaa parimat ja te saate süüa maa rasva!
\par 19 Ja sul tuleb anda käsk: Tehke nõnda - võtke endile Egiptusemaalt vankrid väetite laste ja naiste jaoks ja tooge oma isa ning tulge!
\par 20 Teie silm ärgu kurvastagu teie asjade pärast, sest parim kogu Egiptusemaal peab olema teie päralt!”
\par 21 Ja Iisraeli pojad tegid nõnda ja Joosep andis neile vaarao käsu peale vankrid, samuti andis ta neile teemoona.
\par 22 Ta andis neile kõigile peoriided, aga Benjaminile andis ta kolmsada hõbeseeklit ja viied peoriided.
\par 23 Samuti läkitas ta oma isale kümme eeslit, kes olid koormatud Egiptuse parimate kaupadega, ja kümme emaeeslit, kes kandsid vilja, leiba ja moona ta isale teekonna tarvis.
\par 24 Siis ta saatis oma vennad minema ja nad läksid. Ja ta ütles neile: „Teel ärge riielge!”
\par 25 Ja nad läksid ära Egiptusest ning tulid Kaananimaale oma isa Jaakobi juurde.
\par 26 Ja nad jutustasid temale ning ütlesid: „Joosep elab alles, ta on nimelt kogu Egiptusemaa valitseja!” Aga ta süda jäi külmaks, sest ta ei uskunud neid.
\par 27 Siis nad jutustasid temale kõigest, mis Joosep nendega oli rääkinud. Ja kui ta nägi vankreid, mis Joosep oli läkitanud teda ära tooma, siis nende isa Jaakobi vaim elustus taas.
\par 28 Ja Iisrael ütles: „Küllalt! Mu poeg Joosep elab alles! Ma tahan minna ja teda näha, enne kui suren!”

\chapter{46}

\par 1 Nõnda läks Iisrael teele koos kõigega, mis tal oli, ja tuli Beer-Sebasse ning ohverdas tapaohvreid oma isa Iisaki Jumalale.
\par 2 Ja Jumal rääkis Iisraeliga öistes nägemustes; ta ütles: „Jaakob, Jaakob!„ Ja see vastas: ”Siin ma olen!”
\par 3 Siis ta ütles: „Mina olen Jumal, sinu isa Jumal, ära karda minna Egiptusesse, sest ma teen sind seal suureks rahvaks!
\par 4 Mina lähen koos sinuga Egiptusesse ja ma toon sind sealt tagasi, ja Joosep suleb oma käega su silmad.”
\par 5 Ja Jaakob asus Beer-Sebast minekule; Iisraeli pojad panid oma isa Jaakobi ja oma väetid lapsed ja naised vankritesse, mis vaarao oli läkitanud teda tooma.
\par 6 Ja nad võtsid oma karjad ja varanduse, mis nad Kaananimaal olid soetanud, ja tulid Egiptusesse, Jaakob ja kõik ta sugu koos temaga.
\par 7 Oma pojad ja poegade pojad, tütred ja poegade tütred ja kogu oma soo viis ta koos enesega Egiptusesse.
\par 8 Ja need on Iisraeli laste nimed, kes Egiptusesse tulid: Jaakob ja tema pojad. Jaakobi esmasündinu oli Ruuben.
\par 9 Ruubeni pojad olid Hanok, Pallu, Hesron ja Karmi.
\par 10 Siimeoni pojad olid Jemuel, Jaamin, Ohad, Jaakin, Sohar ja Saul, kaananlanna poeg.
\par 11 Leevi pojad olid Geerson, Kehat ja Merari.
\par 12 Juuda pojad olid Eer, Oonan, Seela, Perets ja Serah; aga Eer ja Oonan surid Kaananimaal. Peretsi pojad olid Hesron ja Haamul.
\par 13 Issaskari pojad olid Toola, Puvva, Jaasub ja Simron.
\par 14 Sebuloni pojad olid Sered, Eelon ja Jahleel.
\par 15 Need olid Lea pojad, keda ta Jaakobile ilmale tõi Mesopotaamias; peale nende oli tal tütar Diina. Kõiki ta poegi ja tütreid oli ühtekokku kolmkümmend kolm hinge.
\par 16 Gaadi pojad olid Sefon, Haggi, Suuni, Esbon, Eeri, Arodi ja Areli.
\par 17 Aaseri pojad olid Jimna, Jisva, Jisvi ja Berija; nende õde oli Serah. Beria pojad olid Heber ja Malkiel.
\par 18 Need olid selle Silpa pojad, kelle Laaban andis oma tütrele Leale ja kes tõi need Jaakobile ilmale, kuusteist hinge.
\par 19 Jaakobi naise Raaheli pojad olid Joosep ja Benjamin.
\par 20 Joosepile sündisid Egiptusemaal pojad, keda temale ilmale tõi Asnat, Ooni preestri Pooti-Fera tütar: Manasse ja Efraim.
\par 21 Benjamini pojad olid Bela, Beker, Asbel, Geera, Naaman, Eehi, Ross, Muppim, Huppim ja Aard.
\par 22 Need olid Raaheli pojad, kes Jaakobile sündisid, ühtekokku neliteist hinge.
\par 23 Daani poeg oli Husim.
\par 24 Naftali pojad olid Jahsel, Guuni, Jeeser ja Sillem.
\par 25 Need olid selle Billa pojad, kelle Laaban andis oma tütrele Raahelile ja kes tõi need Jaakobile ilmale, ühtekokku seitse hinge.
\par 26 Kõiki hingi, kes Jaakobiga Egiptusesse tulid, kes tema niudeist olid väljunud, peale Jaakobi poegade naiste, oli ühtekokku kuuskümmend kuus hinge.
\par 27 Ja Joosepi poegi, kes temale Egiptuses olid sündinud, oli kaks hinge; kõiki Jaakobi soo hingi, kes Egiptusesse tulid, oli seitsekümmend.
\par 28 Ja ta läkitas Juuda enese eel Joosepi juurde, et see teda juhataks Goosenisse; nõnda nad tulid Gooseni maakonda.
\par 29 Ja Joosep laskis hobused oma vankri ette rakendada ning läks Goosenisse vastu oma isale Iisraelile; kui ta teda nägi, siis ta langes temale kaela ja nuttis kaua tema kaelas.
\par 30 Ja Iisrael ütles Joosepile: „Nüüd ma võin surra, sest olen näinud su nägu ja tean, et oled veel elus.”
\par 31 Ja Joosep ütles oma vendadele ja oma isa perele: „Ma lähen ja teatan vaaraole ning ütlen temale: Mu vennad ja mu isa pere, kes olid Kaananimaal, on minu juurde tulnud.
\par 32 Ja need mehed on karjased, sest nad olidki karjakasvatajad, ja nad on oma lambad, kitsed ja veised ja kõik, mis neil oli, kaasa toonud.
\par 33 Kui juhtub, et vaarao teid kutsub ja küsib: Mis teie amet on?,
\par 34 siis vastake: Su sulased on olnud karjakasvatajad noorest põlvest kuni tänini, niihästi meie kui meie isad - et saaksite elada Gooseni maakonnas, sest egiptlased põlgavad kõiki lamba- ja kitsekarjaseid.”

\chapter{47}

\par 1 Ja Joosep läks ja teatas vaaraole ning ütles: „Mu isa ja vennad ja nende lambad, kitsed ja veised ja kõik, mis neil oli, on tulnud Kaananimaalt, ja vaata, nad on Gooseni maakonnas.”
\par 2 Ja ta võttis oma vendade hulgast viis meest ja tõi need vaarao ette.
\par 3 Ja vaarao küsis tema vendadelt: „Mis teie amet on?„ Ja nad vastasid vaaraole: ”Su sulased on lambakarjased, niihästi meie ise kui meie isad.”
\par 4 Ja nad ütlesid vaaraole: „Me oleme maale tulnud võõrastena elama, sest su sulaste karjale ei olnud sööta, kuna Kaananimaal on kange nälg. Luba siis nüüd oma sulaseid elada Gooseni maakonnas!”
\par 5 Ja vaarao rääkis Joosepiga, öeldes: „Sinu isa ja vennad on su juurde tulnud.
\par 6 Egiptusemaa on su ees lahti, pane oma isa ja vennad elama parimasse maakonda. Elagu nad Gooseni maakonnas, ja kui sa tunned nende hulgast tublisid mehi, siis pane need mu karja ülevaatajaiks!”
\par 7 Siis Joosep tõi sisse oma isa Jaakobi ja pani seisma vaarao ette; ja Jaakob õnnistas vaaraod.
\par 8 Ja vaarao küsis Jaakobilt: „Kui palju sul eluaastaid on?”
\par 9 Ja Jaakob vastas vaaraole: „Aastaid, mis ma võõrana olen elanud, on sada kolmkümmend aastat. Piskud ja kurjad on olnud mu eluaastad ja need ei ulatu mu isade eluaastateni nende võõrsiloleku ajal.”
\par 10 Siis Jaakob õnnistas vaaraod ja läks ära vaarao juurest.
\par 11 Ja Joosep paigutas oma isa ja vennad elama ja andis neile maaomandi Egiptusemaal kõige paremas maakonnas, Raamsese maakonnas, nagu vaarao oli käskinud.
\par 12 Ja Joosep hoolitses leivaga oma isa ja vendade ja kogu isa pere eest, vastavalt nende väetite laste suudele.
\par 13 Aga kogu maal ei olnud leiba, sest näljahäda oli väga kange, ja Egiptusemaa ja Kaananimaa olid näljast nõrkemas.
\par 14 Ja Joosep kogus kokku kõik Egiptusemaal ja Kaananimaal leiduva raha vilja eest, mida osteti; ja Joosep viis raha vaarao kotta.
\par 15 Kui raha oli lõppenud Egiptusemaalt ja Kaananimaalt, siis tulid kõik egiptlased Joosepi juurde, öeldes: „Anna meile leiba! Kas peame su silma ees surema, sellepärast et raha on otsas?”
\par 16 Ja Joosep vastas: „Andke oma loomad ja mina annan teile nende eest, kui raha on otsas.”
\par 17 Ja nad tõid oma loomad Joosepile ja Joosep andis neile leiba hobuste, lamba- ja kitsekarjade, veisekarjade ja eeslite eest; nõnda muretses ta neile sel aastal leiba kõigi nende loomade eest.
\par 18 Kui see aasta lõppes, siis järgmisel aastal tulid nad tema juurde ja ütlesid temale: „Me ei saa oma isandale salata, et raha on otsas ja loomakarjad on meie isanda käes. Meil ei ole isanda ees muud üle jäänud kui ainult meie ihud ja põllumaa.
\par 19 Kas peame su silma ees surema, niihästi me ise kui meie põllud? Osta meid ja meie põllud leiva eest, et me oma põldudega saaksime vaaraole orjadeks! Anna meile seemet, et jääksime elama ega sureks ja et põllud ei jääks tühjaks!”
\par 20 Siis Joosep ostis vaaraole kogu Egiptuse põllumaa, sest egiptlased müüsid igaüks oma põllu, sellepärast et nälg ahistas neid. Nõnda sai maa vaarao omaks.
\par 21 Ja ta tegi rahva tema orjaks, Egiptuse ühest äärest teiseni.
\par 22 Ainult preestrite põldusid ta ei ostnud, sest preestritel oli sissetulek vaaraolt ja nemad elatusid sissetulekust, mida vaarao neile andis; seepärast nemad ei müünud oma põldusid.
\par 23 Ja Joosep ütles rahvale: „Vaata, ma olen nüüd ostnud teid ja teie põllud vaaraole. Näe, siin on teile seemet, külvake see põldudele!
\par 24 Aga saagist peate andma viiendiku vaaraole, kuna neli osa jäägu teie kätte teile põlluseemneks, samuti toiduks teile ja neile, kes teie peredes on, ja toiduks teie väetitele lastele.”
\par 25 Ja nad vastasid: „Sina oled meid elus hoidnud! Kui leiame armu oma isanda silmis, siis jääme vaaraole orjadeks.”
\par 26 Ja Joosep tegi selle seaduseks Egiptuse põllumaa kohta tänapäevani, et vaaraole saab viiendik; ainult preestrite põllud ei saanud vaarao omaks.
\par 27 Ja Iisrael jäi elama Egiptusemaale Gooseni maakonda; nad jäid sinna paigale, olid viljakad ja neid sai väga palju.
\par 28 Ja Jaakob elas Egiptusemaal seitseteist aastat, ja Jaakobi päevi, tema eluaastaid, oli sada nelikümmend seitse aastat.
\par 29 Kui Iisraeli surmapäev ligines, siis ta kutsus oma poja Joosepi ning ütles temale: „Kui ma nüüd olen su silmis armu leidnud, siis pane oma käsi mu puusa alla ja osuta mulle heldust ja truudust: ära mata mind Egiptusesse,
\par 30 sest ma tahan magada oma vanemate juures. Vii mind Egiptusest ära ja mata nende hauda!„ Ja ta vastas: „Ma teen su sõna järgi!”
\par 31 Ja tema ütles: „Vannu mulle!” Ja ta vandus temale. Siis Iisrael kummardas voodi peatsi poole.

\chapter{48}

\par 1 Pärast seda sündmust öeldi Joosepile: „Vaata, su isa on haige!” Siis ta võttis oma kaks poega enesega, Manasse ja Efraimi.
\par 2 Ja Jaakobile anti teada ning öeldi: „Näe, su poeg Joosep tuleb sinu juurde!” Iisrael võttis siis jõu kokku ning tõusis voodis istukile.
\par 3 Ja Jaakob ütles Joosepile: „Kõigeväeline Jumal ilmutas ennast mulle Luusis Kaananimaal, ja õnnistas mind
\par 4 ning ütles mulle: Vaata, ma teen sind viljakaks ja paljuks ja teen sinust rahvaste hulga ja annan selle maa sinu soole pärast sind igaveseks omandiks.
\par 5 Ja nüüd olgu su kaks poega, kes sulle Egiptusemaal on sündinud, enne kui ma tulin sinu juurde Egiptusesse, minu omad: Efraim ja Manasse olgu minu omad nagu Ruuben ja Siimeongi.
\par 6 Aga su järeltulijad, kes sulle pärast neid sünnivad, olgu sinu päralt ja neid nimetatagu nende vendade nime järgi nende pärisosades.
\par 7 Kui ma Mesopotaamiast tulin, suri mul Raahel Kaananimaal, tee peal, kui veel tükk maad oli minna Efratasse. Ja ma matsin tema Efrata tee äärde, see on Petlemma.”
\par 8 Kui Iisrael nägi Joosepi poegi, siis ta küsis: „Kes need on?”
\par 9 Ja Joosep vastas oma isale: „Need on mu pojad, keda Jumal mulle siin on andnud!„ Siis ta ütles: ”Too nad minu juurde, et ma neid õnnistaksin!”
\par 10 Aga Iisraeli silmad olid vanadusest tuhmid ja ta ei näinud enam. Siis Joosep viis nad tema juurde ja tema andis neile suud ning süleles neid.
\par 11 Ja Iisrael ütles Joosepile: „Ei oleks uskunud, et saan näha su nägu, aga vaata, Jumal on mind lasknud näha ka su järglasi.”
\par 12 Siis Joosep võttis nad ära tema põlvilt ja kummardas silmili maha.
\par 13 Ja Joosep võttis nad mõlemad, Efraimi oma paremale käele, Iisraelist vasakule poole, ja Manasse oma vasakule käele, Iisraelist paremale poole, ja viis nad tema juurde.
\par 14 Aga Iisrael sirutas oma parema käe ja pani Efraimi pea peale, kes oli noorem, ja oma vasaku käe Manasse pea peale; ta pani oma käed ristamisi, sest Manasse oli esmasündinu.
\par 15 Ja ta õnnistas Joosepit ning ütles: „Jumal, kelle palge ees mu isad Aabraham ja Iisak on rännanud, Jumal, kes on olnud mu karjane kogu mu elu kuni tänapäevani,
\par 16 ingel, kes mind on päästnud kõigest kurjast, õnnistagu neid poisse; neid nimetades nimetatagu minu nime ja mu isade Aabrahami ja Iisaki nime! Ja nad siginegu rohkesti keset maad!”
\par 17 Aga kui Joosep nägi, et ta isa oma parema käe asetas Efraimi pea peale, siis ta pani seda pahaks ja haaras kinni oma isa käest, et seda Efraimi pea pealt tõsta Manasse pea peale.
\par 18 Ja Joosep ütles oma isale: „Mitte nõnda, mu isa, sest see on mu esmasündinu! Pane oma parem käsi tema pea peale!”
\par 19 Aga ta isa keeldus ja ütles: „Ma tean, mu poeg, ma tean, ka tema peab saama rahvaks ja temagi peab olema suur! Ometi peab ta noorem vend saama temast suuremaks ja selle sugu olema rahvarohke!”
\par 20 Ja ta õnnistas neid sel päeval, öeldes: „Sinu nimel õnnistatagu Iisraelis, öeldagu: Jumal tehku sind Efraimi ja Manasse sarnaseks!” Nõnda seadis ta Efraimi Manassest ettepoole.
\par 21 Ja Iisrael ütles Joosepile: „Vaata, ma suren, aga Jumal on teiega ja viib teid tagasi teie isade maale.
\par 22 Ja ma annan sulle ühe mäeseljandiku rohkem kui su vendadele, mille olen mõõga ja ammuga võtnud emorlastelt.”

\chapter{49}

\par 1 Ja Jaakob kutsus oma pojad enese juurde ning ütles: „Tulge kokku, siis ma kuulutan teile, mis teiega sünnib tulevasil päevil!
\par 2 Kogunege ja kuulge, Jaakobi pojad, võtke kuulda Iisraeli, oma isa!
\par 3 Ruuben, sina oled mu esmasündinu, mu rammu ja sigitusjõu esikpoeg, väljapaistev väärikuselt ja väljapaistev võimult.
\par 4 Veena oled sa vooganud - esikohale sa ei jää, sest sa tõusid oma isa sängi! Sel korral sa rüvetasid selle, sina, kes tõusid mu voodile.
\par 5 Vennaksed Siimeon ja Leevi, nende noad on vägivalla riistad.
\par 6 Nende nõusse ei astu mu hing, nende seltsiga ei liitu mu süda. Sest oma vihas nad tapsid mehi ja meelevallatuses halvasid härgi.
\par 7 Olgu neetud nende kange viha ja nende metsik raev! Ma jaotan nad Jaakobis ja hajutan Iisraelis!
\par 8 Juuda, sind ülistavad su vennad. Sinu käsi on su vaenlaste turjal, sind kummardavad su isa pojad.
\par 9 Juuda on lõvikutsikas - saagi kallalt, mu poeg, oled tõusnud. Ta on heitnud maha, ta lebab nagu lõvi, nagu metsik lõvi - kes julgeks teda äratada?
\par 10 Ei lahku valitsuskepp Juudast ega sau tema jalgelt, kuni tuleb Juuda poeg Siilo, ja teda võtavad rahvad kuulda.
\par 11 Ta seob oma eesli viinapuu külge, hea viinapuu külge oma eeslivarsa; ta peseb oma kuube veiniga ja oma vammust viinamarjade verega.
\par 12 Ta silmad on veinist hämused ja hambad piimast valged.
\par 13 Sebulon elab mererannal, ta saab laevade rannikuks ja tema selg on pööratud Siidoni poole.
\par 14 Issaskar on kondine eesel, kes lebab sadulakorvide vahel.
\par 15 Kui ta nägi head hingamispaika ja meeldivat maad, ta langetas oma turja koormat kandma ja sai tööorjaks.
\par 16 Daan mõistab kohut oma rahvale, üks Iisraeli suguharu on temagi.
\par 17 Daan on madu teel, rästik raja peal, kes salvab hobuse kandu, nõnda et ratsanik kukub selili.
\par 18 Ma ootan päästet sinult, Issand!
\par 19 Gaad - röövjõugud ründavad teda, aga ta ise ründab neid, olles neil kannul.
\par 20 Aaserilt tuleb rammus roog ja temal on anda kuninglikke maiuspalu.
\par 21 Naftali on nobe emahirv, kes toob kuuldavale ilusaid sõnu.
\par 22 Joosep on viljapuu poeg, viljapuu poeg allikal, oksad ulatuvad üle müüri.
\par 23 Ammukütid ahistavad teda, ründavad ja rõhuvad teda,
\par 24 aga tema amb jääb kindlaks ja ta käsivarred on nõtked Jaakobi Vägeva abiga, Karjase, Iisraeli Kalju nime abiga,
\par 25 su isa Jumala abiga, kes sind aidaku, Kõigeväelise abiga, kes sind õnnistagu õnnistustega ülalt taevast, õnnistustega all asuvast põhjaveest, õnnistustega emarindadest ja üskadest!
\par 26 Su isa õnnistused ületavad igaveste mägede õnnistused, ürgsete küngaste ihaldusväärsed annid. Need tulgu Joosepi pea peale, oma vendade vürsti pealaele!
\par 27 Benjamin on kiskjalik hunt. Hommikul ta sööb saaki ja õhtul jaotab röövitut.”
\par 28 Need kõik olid Iisraeli suguharud, neid oli kaksteist, ja see oli, mis nende isa neile rääkis, kui ta neid õnnistas: ta õnnistas igaüht temale kohase õnnistusega.
\par 29 Ja ta käskis neid ning ütles neile: „Mind koristatakse mu rahva juurde. Matke mind mu isade juurde koopasse, mis on hett Efroni väljal,
\par 30 sellesse koopasse, mis on Makpela väljal Mamre kohal Kaananimaal, mille Aabraham ostis koos väljaga hett Efronilt pärushauaks.
\par 31 Sinna on maetud Aabraham ja tema naine Saara, sinna on maetud Iisak ja tema naine Rebeka, ja sinna ma olen matnud Lea.
\par 32 Väli ja seal olev koobas on hettidelt ostetud.”
\par 33 Kui Jaakob oli oma poegadele käsu andnud, siis ta sirutas voodis oma jalad välja ja heitis hinge; ja ta koristati oma rahva juurde.

\chapter{50}

\par 1 Siis Joosep langes oma isa palge vastu, nuttis tema kohal ja suudles teda.
\par 2 Ja Joosep käskis oma teenistuses olevaid arste tema isa palsameerida; ja arstid palsameerisid Iisraeli.
\par 3 Selleks kulus nelikümmend päeva, sest nii palju päevi kulub palsameerimiseks; ja egiptlased nutsid teda seitsekümmend päeva.
\par 4 Kui tema nutupäevad olid möödunud, siis Joosep rääkis vaarao hoovkonnaga, öeldes: „Kui ma nüüd teie silmis olen armu leidnud, siis rääkige vaarao kõrva ees ja öelge:
\par 5 Minu isa laskis mind vanduda, öeldes: Vaata, ma suren. Mata mind mu hauda, mille ma enesele olen kaevanud Kaananimaal! Ma tahaksin nüüd minna, oma isa matta ja siis tagasi tulla.”
\par 6 Ja vaarao ütles: „Mine ja mata oma isa, nõnda nagu ta sind on lasknud vanduda!”
\par 7 Ja Joosep läks oma isa matma; ja temaga koos läksid kõik vaarao sulased, tema hoovkonna vanemad ja kõik Egiptusemaa vanemad,
\par 8 ja kogu Joosepi pere ja tema vennad ja ta isa pere; ainult oma väetid lapsed ja lambad, kitsed ja veised jätsid nad Gooseni maakonda.
\par 9 Ja temaga koos läksid niihästi vankrid kui ratsanikud, ja see oli väga suur karavan.
\par 10 Kui nad jõudsid Atadi rehealuse juurde, mis on teisel pool Jordanit, siis nad panid seal toime väga suure ja mõjuva leinakaebuse, ja ta pidas oma isa peiesid seitse päeva.
\par 11 Kui maa elanikud, kaananlased, nägid neid peiesid Atadi rehealuse juures, siis nad ütlesid: „Need on egiptlastel suured peied.” Seepärast pandi sellele paigale nimeks Aabel-Mitsraim; see on teisel pool Jordanit.
\par 12 Ja ta pojad tegid temaga nõnda, nagu ta neid oli käskinud:
\par 13 ta pojad viisid tema Kaananimaale ja matsid ta Makpela välja koopasse Mamre kohal, mille Aabraham koos väljaga oli ostnud pärandhauaks hett Efronilt.
\par 14 Ja Joosep läks tagasi Egiptusesse, tema ja ta vennad ja kõik, kes koos temaga olid läinud ta isa matma, pärast seda kui ta oma isa oli matnud.
\par 15 Kui Joosepi vennad nägid, et nende isa oli surnud, siis nad ütlesid: „Kui Joosep meid nüüd vihkab ja tõesti tasub meile kätte kõige kurja eest, mis me temale oleme teinud?”
\par 16 Ja nad käskisid Joosepile öelda: „Su isa andis käsu, enne kui ta suri, ja ütles:
\par 17 Öelge Joosepile nõnda: Anna ometi andeks oma vendade üleastumine ja patt, et nad sulle on kurja teinud! Seepärast anna siis nüüd andeks oma isa Jumala sulaste üleastumine!” Ja Joosep nuttis, kui temale seda räägiti.
\par 18 Siis tulid ka Joosepi vennad ise, heitsid maha ta ette ja ütlesid: „Vaata, me jääme sulle orjadeks!”
\par 19 Aga Joosep vastas neile: „Ärge kartke! Kas mina olen Jumala asemik?
\par 20 Te mõtlesite küll mu vastu kurja, aga Jumal pööras selle heaks, et teha, mis tänapäeval ongi tehtud: hoida palju rahvast elus.
\par 21 Ja nüüd ärge kartke, mina toidan teid ja teie väeteid lapsi!” Ja ta trööstis ning rahustas neid.
\par 22 Ja Joosep jäi Egiptusesse, tema ja ta isa pere. Ja Joosep elas saja kümne aastaseks.
\par 23 Ja Joosep nägi Efraimi lapsi kolm põlve; ka Manasse pojast Maakirist sündis lapsi Joosepi põlvede peale.
\par 24 Ja Joosep ütles oma vendadele: „Mina suren, aga Jumal hoolitseb kindlasti teie eest ja viib teid siit maalt sellele maale, mille ta vandega on tõotanud anda Aabrahamile, Iisakile ja Jaakobile!”
\par 25 Ja Joosep vannutas Iisraeli poegi, öeldes: „Kui Jumal tõesti hoolitseb teie eest, siis viige ka minu luud siit ära!”
\par 26 Ja Joosep suri, sada kümme aastat vana, ja ta palsameeriti ja pandi kirstu Egiptuses.




\end{document}
