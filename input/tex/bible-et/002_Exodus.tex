\begin{document}

\title{Teine Moosese raamat}


\chapter{1}

\par 1 Ja need on Iisraeli poegade nimed, kes olid Egiptusesse tulnud; koos Jaakobiga oli igaüks tulnud oma perega:
\par 2 Ruuben, Siimeon, Leevi, Juuda,
\par 3 Issaskar, Sebulon, Benjamin,
\par 4 Daan, Naftali, Gaad ja Aaser.
\par 5 Kõiki hingi, kes Jaakobi niudeist oli lähtunud, oli seitsekümmend hinge; ja Joosep oli juba Egiptuses.
\par 6 Joosep suri, samuti kõik ta vennad ja kogu see sugupõlv.
\par 7 Aga Iisraeli lapsed olid viljakad, nad siginesid ja paljunesid ning said väga vägevaiks; ja maa oli neid täis.
\par 8 Egiptuses tõusis uus kuningas, kes Joosepit ei tundnud,
\par 9 ja see ütles oma rahvale: „Vaata, Iisraeli laste rahvast on rohkem ja nad on meist vägevamad.
\par 10 Olgem seepärast nende vastu kavalad, et nad ei saaks paljuneda. Sest kui peaks tulema sõda, siis nad liituvad nendega, kes meid vihkavad, ja nad sõdivad meie vastu ning lähevad maalt ära.”
\par 11 Siis nad panid nende üle teoorjusele sundijaid, et need rõhuksid neid raske teoga; vaaraole ehitati varaaitade linnu - Pitomit ja Raamsest.
\par 12 Aga mida rohkem nad neid rõhusid, seda rohkem neid sai ja seda laiemale nad levisid; ja Iisraeli laste ees hakati hirmu tundma.
\par 13 Ja egiptlased panid väevõimuga Iisraeli lapsed töötama.
\par 14 Nad tegid nende elu kibedaks raske orjatööga savi ja telliskivide kallal ning kõiksugu orjusega põllul, kõiksugu tööga, mida nad väevõimuga sundisid neid tegema.
\par 15 Ja Egiptuse kuningas rääkis heebrealaste ämmaemandatega, kellest ühe nimi oli Sifra ja teise nimi Puua,
\par 16 ning ütles: „Kui te heebrea naisi aitate sünnitamisel ja näete sugutunnuseist, et on poeglaps, siis surmake see, aga tütarlaps võib jääda elama!”
\par 17 Aga ämmaemandad kartsid Jumalat ega teinud nõnda, nagu Egiptuse kuningas neid käskis, vaid jätsid poeglapsed elama.
\par 18 Siis Egiptuse kuningas kutsus ämmaemandad ja ütles neile: „Mispärast teete nõnda ja jätate poeglapsed elama?”
\par 19 Ja ämmaemandad vastasid vaaraole: „Sellepärast et heebrealaste naised pole nagu egiptlaste naised, vaid nad on tublimad: enne kui ämmaemand jõuab nende juurde, on nad sünnitanud.”
\par 20 Ja Jumal tegi ämmaemandatele head; rahvas aga suurenes ja nad said väga arvurikkaks:
\par 21 et ämmaemandad kartsid Jumalat, siis ta andis neile suured pered.
\par 22 Aga vaarao andis käsu kogu oma rahvale, öeldes: „Kõik poeglapsed, kes sünnivad, peate viskama jõkke, kõik tütarlapsed aga võite jätta elama!”

\chapter{2}

\par 1 Üks mees, Leevi soost, läks ja võttis ühe Leevi tütre.
\par 2 Ja naine jäi lapseootele ning tõi poja ilmale; ta nägi, et see oli ilus, ja ta peitis teda kolm kuud.
\par 3 Aga kui ta enam ei saanud teda peita, siis ta võttis tema jaoks pilliroost laeka, pigitas selle maapigi ja vaiguga, pani sellesse lapse ja asetas jõe äärde kõrkjaisse.
\par 4 Ta õde võttis eemal aset, et teada saada, mis temaga juhtub.
\par 5 Siis tuli vaarao tütar alla jõe äärde ennast pesema; ta toaneitsid aga kõndisid jõe kaldal. Kui ta nägi laegast kõrkjate sees, siis ta läkitas oma teenija ja laskis selle ära tuua.
\par 6 Kui ta selle avas ja nägi last, vaata, siis üks poeglaps nuttis. Aga ta halastas tema peale ning ütles: „See on üks heebrealaste poeglaps.”
\par 7 Selle õde ütles siis vaarao tütrele: „Kas pean minema ja kutsuma sulle heebrea naiste hulgast imetaja, et ta sulle last imetaks?”
\par 8 Ja vaarao tütar vastas temale: „Mine!” Ja tüdruk läks ning kutsus lapse ema.
\par 9 Ja vaarao tütar ütles sellele: „Vii see laps ja imeta teda minu jaoks, ja ma annan sulle tasu!” Ja naine võttis lapse ning imetas teda.
\par 10 Kui laps oli kasvanud, siis ta tõi selle vaarao tütrele ja see võttis tema enesele pojaks; ta pani temale nimeks Mooses ja ütles: „Sest ma olen ta veest välja tõmmanud!”
\par 11 Ja see sündis neil päevil, kui Mooses oli suureks kasvanud, et ta läks välja oma suguvendade juurde ja nägi nende teoorjust; ja ta nägi ühte egiptuse meest peksvat heebrea meest tema suguvendade hulgast.
\par 12 Ja kui ta vaatas sinna ja tänna ja nägi, et seal kedagi ei olnud, siis ta lõi egiptlase maha ja mattis liivasse.
\par 13 Kui ta teisel päeval läks välja, vaata, siis taplesid kaks heebrea meest, ja ta ütles õelale: „Miks sa peksad oma ligimest?”
\par 14 Aga see vastas: „Kes on pannud sind meile ülemaks ja kohtumõistjaks? Kas mõtled tappa ka mind, nagu sa tapsid egiptlase?„ Siis Mooses kartis ja mõtles: ”Asi on tõesti ilmsiks tulnud!”
\par 15 Ka vaarao kuulis sellest loost ja ta püüdis Moosest tappa. Aga Mooses põgenes vaarao eest ja asus Midjanimaale. Kord istus ta seal ühe kaevu ääres.
\par 16 Midjani preestril oli seitse tütart; need tulid ja ammutasid vett ning täitsid künad, et joota oma isa lambaid ja kitsi.
\par 17 Aga karjased tulid ja tõrjusid nad eemale; siis Mooses tõusis ja aitas neid ning jootis nende loomi.
\par 18 Kui nad tulid oma isa Reueli juurde, küsis see: „Kuidas te täna jõudsite nii kiiresti?”
\par 19 Nad vastasid: „Keegi egiptuse mees päästis meid karjaste käest; ta ammutas meile ka vett ning jootis lambaid ja kitsi.”
\par 20 Siis ta ütles oma tütardele: „Kus ta on? Miks jätsite mehe sinna? Kutsuge ta leiba võtma!”
\par 21 Ja Mooses otsustas jääda selle mehe juurde; see andis oma tütre Sippora Moosesele.
\par 22 Temale sündis poeg ja ta pani sellele nimeks Geersom, sest ta ütles: „Ma olen võõras võõral maal.”
\par 23 Alles hulga aja pärast juhtus, et Egiptuse kuningas suri. Ent Iisraeli lapsed ohkasid ja kaebasid orjuse pärast; ja nende hädakisa orjuse pärast tõusis Jumalani.
\par 24 Ja Jumal kuulis nende ägamist, ja Jumal mõtles oma lepingule Aabrahami, Iisaki ja Jaakobiga.
\par 25 Jumal vaatas Iisraeli laste peale ja Jumal mõistis neid.

\chapter{3}

\par 1 Mooses karjatas oma äia, Midjani preestri Jitro lambaid ja kitsi. Kord ajas ta karja kõrbe taha ja jõudis Jumala mäe Hoorebi juurde.
\par 2 Seal ilmutas ennast temale Issanda ingel tuleleegis keset kibuvitsapõõsast, ja ta vaatas, ja ennäe, kibuvitsapõõsas põles tules, aga kibuvitsapõõsas ei põlenud ära.
\par 3 Ja Mooses mõtles: „Ma põikan kõrvale ja vaatan seda imet, miks kibuvitsapõõsas ära ei põle.”
\par 4 Kui Issand nägi, et ta pöördus vaatama, siis Jumal hüüdis teda kibuvitsapõõsast ja ütles: „Mooses, Mooses!„ Ja tema vastas: ”Siin ma olen!”
\par 5 Siis ta ütles: „Ära tule siia, võta jalatsid jalast, sest paik, kus sa seisad, on püha maa!”
\par 6 Ja ta jätkas: „Mina olen sinu vanemate Jumal, Aabrahami Jumal, Iisaki Jumal ja Jaakobi Jumal!” Aga Mooses kattis oma näo, sest ta kartis Jumalale otsa vaadata.
\par 7 Ja Issand ütles: „Ma olen küllalt näinud oma rahva viletsust, kes on Egiptuses, ja ma olen kuulnud nende kisendamist sundijate pärast; seetõttu ma tean nende valu
\par 8 ja olen alla tulnud neid egiptlaste käest päästma ja neid sellelt maalt viima heale ja avarale maale, maale, mis piima ja mett voolab, kaananlaste, hettide, emorlaste, perislaste, hiivlaste ja jebuuslaste asupaika.
\par 9 Vaata, nüüd on Iisraeli laste hädakisa jõudnud minuni ja ma olen ka näinud rõhumist, millega egiptlased neid rõhuvad.
\par 10 Tule nüüd, ma läkitan su vaarao juurde, ja vii mu rahvas, Iisraeli lapsed, Egiptusest välja!”
\par 11 Kuid Mooses ütles Jumalale: „Kes olen mina, et võiksin minna vaarao juurde ja viia Iisraeli lapsed Egiptusest välja?”
\par 12 Aga tema kostis: „Mina olen sinuga, ja see olgu sulle tähiseks, et mina sind olen läkitanud: kui sa rahva Egiptusest oled välja viinud, siis te teenite Jumalat sellel mäel.”
\par 13 Siis Mooses ütles Jumalale: „Vaata, kui ma lähen Iisraeli laste juurde ja ütlen neile: Teie vanemate Jumal on mind läkitanud teie juurde, aga nemad küsivad minult: Mis ta nimi on?, mis ma siis neile pean vastama?”
\par 14 Ja Jumal ütles Moosesele: „Ma olen see, kes ma Olen!” Ja ta jätkas: „Ütle Iisraeli lastele nõnda: „Ma Olen” on mind läkitanud teie juurde.”
\par 15 Ja Jumal ütles Moosesele veel: „Ütle Iisraeli lastele nõnda: Jahve, teie vanemate Jumal, Aabrahami Jumal, Iisaki Jumal ja Jaakobi Jumal, on mind läkitanud teie juurde; see on igavesti mu nimi ja nõnda peab mind hüütama põlvest põlve!
\par 16 Mine ja kogu kokku Iisraeli vanemad ja ütle neile: Issand, teie vanemate Jumal, on ennast mulle ilmutanud, Aabrahami, Iisaki ja Jaakobi Jumal, ja on öelnud: Ma olen tõesti pidanud silmas teid ja seda, mis teiega Egiptuses on tehtud.
\par 17 Ja ma olen öelnud: Mina viin teid välja Egiptuse viletsusest kaananlaste, hettide, emorlaste, perislaste, hiivlaste ja jebuuslaste maale, maale, mis piima ja mett voolab.
\par 18 Siis nad kuulavad su sõna; sina ja Iisraeli vanemad aga peate minema Egiptuse kuninga juurde ja temale ütlema: Issand, heebrealaste Jumal, kohtas meid. Lase meid nüüd minna kolme päeva tee kõrbesse ja oma Jumalale ohverdada!
\par 19 Ma tean, et Egiptuse kuningas ei lase teid minna, isegi mitte vägeva käe sunnil.
\par 20 Aga ma sirutan oma käe välja ja löön Egiptust kõiksugu imetegudega, mis ma seal tahan teha; pärast seda ta laseb teid minna.
\par 21 Ja ma annan sellele rahvale armu egiptlaste silmis, nõnda et te ära minnes ei lähe mitte tühje käsi:
\par 22 iga naine küsigu oma naabrinaiselt ja võõrana ta kojas elavalt naiselt hõbe- ja kuldriistu ning riideid; pange need selga oma poegadele ja tütardele ja võtke nõnda egiptlastelt saaki!”

\chapter{4}

\par 1 Ja Mooses vastas ning ütles: „Aga kui nemad mind ei usu ja mu sõna ei kuula, vaid ütlevad: Issand ei ole ennast sulle ilmutanud?”
\par 2 Siis Issand küsis: „Mis see on, mis sul käes on?„ Ta vastas: ”Kepp.”
\par 3 Ja tema ütles: „Viska see maha!” Ta viskas selle maha ja see muutus maoks. Ja Mooses põgenes selle eest.
\par 4 Ja Issand ütles Moosesele: „Siruta oma käsi ja haara temal sabast,” - ja ta sirutas oma käe ning võttis temast kinni, ja see muutus ta pihus kepiks -
\par 5 „et nad usuksid, et sulle on ennast ilmutanud Issand, nende vanemate Jumal, Aabrahami Jumal, Iisaki Jumal ja Jaakobi Jumal!”
\par 6 Ja Issand ütles temale veel: „Pane oma käsi põue!” Ta pani oma käe põue ja tõmbas välja, ja vaata, käsi oli pidalitõvest valge nagu lumi.
\par 7 Ja ta ütles: „Pane oma käsi tagasi põue!” Ta pani oma käe põue ja tõmbas välja, ja vaata, see oli jälle nagu ta ihu.
\par 8 „Kui nad sind ei usu ega kuula su sõna esimese imeteo põhjal, siis nad usuvad teise imeteo sõnumit.
\par 9 Ja kui nad ka neid kahte imetegu ei usu ega kuula su sõna, siis võta Niiluse jõe vett ja vala kuivale maale! Vesi, mida sa Niiluse jõest võtad, muutub siis kuival maal vereks.”
\par 10 Aga Mooses ütles Issandale: „Oh Issand, mina ei ole sõnakas mees, ei varemast ajast ega ka mitte sellest peale, kui sa oma sulasega oled rääkinud, sest mul on raskevõitu suu ja raskevõitu keel.”
\par 11 Siis Issand ütles temale: „Kes on teinud inimesele suu, või kes teeb keeletu, kurdi, nägija või pimeda? Kas mitte mina, Issand?
\par 12 Ja nüüd mine, ja mina olen abiks sinu suule ning õpetan sind, mida sa pead rääkima!”
\par 13 Aga ta ütles: „Oh Issand, läkita, keda läkitad, ainult mitte mind!”
\par 14 Siis Issanda viha süttis põlema Moosese vastu ja ta ütles: „Eks ole leviit Aaron sinu vend? Ma tean, et ta räägib hästi; ja vaata, ta tulebki sulle vastu. Kui ta sind näeb, siis ta rõõmutseb südamest.
\par 15 Sina pead temaga rääkima ja temale sõnad suhu panema, aga mina olen sinu suuga ja tema suuga, ja ma õpetan teid, mida te peate tegema.
\par 16 Tema rääkigu sinu asemel rahvaga ja see olgu nõnda: tema on sulle suuks ja sina oled temale Jumalaks.
\par 17 Ja võta kätte see kepp, millega sa imetegusid teed!”
\par 18 Siis Mooses läks ära ja tuli tagasi oma äia Jitro juurde ning ütles temale: „Ma tahan minna tagasi oma vendade juurde, kes on Egiptuses, et näha, kas nad on veel elus!„ Ja Jitro ütles Moosesele: ”Mine rahuga!”
\par 19 Issand ütles Moosesele Midjanis: „Pöördu tagasi Egiptusesse, sest kõik need mehed, kes nõudsid su hinge, on surnud!”
\par 20 Ja Mooses võttis oma naise ja pojad ning pani need eesli selga ja läks tagasi Egiptusemaale; ja Mooses võttis Jumala kepi oma kätte.
\par 21 Issand ütles Moosesele: „Kui sa lähed ja tuled tagasi Egiptusesse, siis vaata, et sa teed vaarao ees kõik need imeteod, mis mina sinu kätte olen andnud. Aga mina teen ta südame kõvaks ja tema ei lase rahvast ära.
\par 22 Kuid ütle vaaraole: Nõnda ütleb Issand: Iisrael on minu esmasündinud poeg.
\par 23 Ja ma ütlen sulle: Saada mu poeg ära, et ta mind teeniks! Aga kui sa keeldud teda saatmast, vaata, siis ma tapan su esmasündinud poja!”
\par 24 Aga kui ta oli teel öömajale, tuli Issand temale vastu ja püüdis teda surmata.
\par 25 Siis Sippora võttis kivinoa ja lõikas ära oma poja eesnaha, puudutas sellega ta häbet ning ütles: „Sa oled tõesti mu verepeigmees!”
\par 26 Siis see jättis tema rahule; sel ajal öeldi verepeigmees ümberlõikamise tähenduses.
\par 27 Ja Issand ütles Aaronile: „Mine kõrbesse Moosesele vastu!” Ta läks ning kohtas teda Jumala mäe juures ja andis temale suud.
\par 28 Ja Mooses tegi Aaronile teatavaks kõik Issanda sõnad, kes teda oli läkitanud, ja kõik imeteod, mis ta teda oli käskinud teha.
\par 29 Siis Mooses läks koos Aaroniga ja nad kogusid kokku kõik Iisraeli laste vanemad.
\par 30 Ja Aaron rääkis kõik need sõnad, mis Issand oli Moosesele rääkinud, ja ta tegi imetegusid rahva silma ees.
\par 31 Ja rahvas uskus; kui nad kuulsid, et Issand oli tundnud muret Iisraeli laste pärast, ja et ta oli näinud nende viletsust, siis nad kummardasid ja heitsid silmili maha.

\chapter{5}

\par 1 Ja seejärel Mooses ja Aaron tulid ning ütlesid vaaraole: „Nõnda ütleb Issand, Iisraeli Jumal: Lase mu rahvas minna, et nad kõrbes peaksid mulle püha!”
\par 2 Aga vaarao vastas: „Kes on Issand, et peaksin kuulama ta sõna ja laskma Iisraeli ära minna? Mina ei tunne Issandat ega lase ka Iisraeli ära minna!”
\par 3 Siis nad ütlesid: „Heebrealaste Jumal tuli meile vastu. Lase meid minna kolme päeva tee kõrbesse ja ohverdada Issandale, meie Jumalale, et tema meid ei tabaks katku või mõõgaga!”
\par 4 Aga Egiptuse kuningas vastas neile: „Mooses ja Aaron, mispärast tahate rahva tööst vabastada? Minge oma töökohustuste juurde!”
\par 5 Vaarao mõtles: „Vaata, arvukas on nüüd maa rahvas, ja teie tahate nad töökohustusest vabastada!”
\par 6 Ja vaarao käskis selsamal päeval rahva sundijaid ja ülevaatajaid, öeldes:
\par 7 „Ärge koguge ega andke enam rahvale õlgi telliskivide tegemiseks nagu varem. Nad mingu ja korjaku ise endile õlgi!
\par 8 Ja telliskivide määr, mis nad tänini on teinud, pange neile peale, ärge seda vähendage, sest nad on laisad! Sellepärast nad kisendavad ja ütlevad: Me tahame minna oma Jumalale ohverdama!
\par 9 Saagu orjus meestele raskemaks, et neil oleks sellega tegemist ega kuulataks valekõnesid!”
\par 10 Ja rahva sundijad ja ülevaatajad läksid välja ning rääkisid rahvaga, öeldes: „Nõnda ütleb vaarao: Mina ei anna teile enam õlgi.
\par 11 Minge ise, võtke endile õlgi, kust aga leiate, ent teie orjusest ei vähendata midagi!”
\par 12 Siis rahvas hajus üle kogu Egiptusemaa korjama kõrsi õlgede asemel.
\par 13 Ja sundijad kihutasid tagant, öeldes: „Tehke oma töö valmis, oma igapäevane jagu nagu siis, kui oli õlgi!”
\par 14 Ja Iisraeli laste ülevaatajaid, kes vaarao sundijate poolt olid pandud nende üle, peksti, öeldes: „Mispärast te ei ole täitnud telliskivide määra endiselt, ei eile ega täna?”
\par 15 Siis Iisraeli laste ülevaatajad tulid ja kisendasid vaarao poole, öeldes: „Mispärast sa teed oma sulastega nõnda?
\par 16 Õlgi su sulastele ei anta, aga siiski öeldakse meile: Tehke telliskive! Vaata, su sulaseid pekstakse ja su rahvas teeb pattu!”
\par 17 Aga tema vastas: „Laisad olete! Te olete laisad, seepärast te ütlete: Me tahame minna Issandale ohverdama!
\par 18 Nüüd minge tööle! Õlgi teile ei anta, aga määratud hulga telliskive te peate andma!”
\par 19 Siis Iisraeli laste ülevaatajad nägid endid olevat täbaras olukorras, kui neile öeldi: „Te ei tohi ühelgi päeval vähendada oma telliskivide päevaosa!”
\par 20 Ja kui nad vaarao juurest tulid, siis nad kohtasid Moosest ja Aaronit, kes olid neid oodanud,
\par 21 ja ütlesid neile: „Issand karistagu teid ja mõistku kohut, sest te olete meid viinud halba kuulsusesse vaarao ja tema sulaste silmis ja neile mõõga meie tapmiseks kätte andnud!”
\par 22 Siis Mooses pöördus jälle Issanda poole ning ütles: „Issand, mispärast sa oled sellele rahvale kurja teinud? Milleks sa siis mind oled läkitanud?
\par 23 Alates sellest, kui ma tulin vaarao juurde sinu nimel rääkima, on ta sellele rahvale kurja teinud ja sina ei ole oma rahvast tõesti mitte päästnud!”

\chapter{6}

\par 1 Ja Issand ütles Moosesele: „Nüüd sa saad näha, mida ma vaaraole teen; sest vägeva käe sunnil ta laseb nad minna ja vägeva käe sunnil ta ajab nad ära oma maalt.”
\par 2 Ja Jumal rääkis Moosesega ning ütles temale: „Mina olen Issand.
\par 3 Mina olen ennast ilmutanud Aabrahamile, Iisakile ja Jaakobile Kõigeväelise Jumalana, aga oma nime Jahve ei ole ma neile teatavaks teinud.
\par 4 Mina olen nendega teinud ka oma lepingu, et ma annan neile Kaananimaa, nende rännakute maa, kus nad on võõrana elanud.
\par 5 Mina olen ka kuulnud Iisraeli laste ägamist, et egiptlased neid orjastavad, ja ma olen mõelnud oma lepingule.
\par 6 Seepärast ütle Iisraeli lastele: Mina olen Issand ja mina viin teid ära egiptlaste teoorjuse alt; ma päästan teid nende orjatööst ja lunastan teid oma väljasirutatud käsivarrega ning suurte kohtupidamistega.
\par 7 Mina võtan teid enesele rahvaks ja olen teile Jumalaks, ja teie peate tundma, et mina olen Issand, teie Jumal, kes teid viib välja egiptlaste teoorjusest.
\par 8 Mina viin teid sellele maale, mille pärast ma oma käe üles tõstsin, et ma annan selle Aabrahamile, Iisakile ja Jaakobile. Mina annan selle teie omandiks, mina, Issand.”
\par 9 Ja Mooses rääkis nõnda Iisraeli lastele, aga nemad ei kuulanud Moosest rõhutud meeleolu ja ränga orjuse tõttu.
\par 10 Ja Issand rääkis Moosesega, öeldes:
\par 11 „Mine ütle vaaraole, Egiptuse kuningale, et ta laseks Iisraeli lapsed oma maalt ära minna!”
\par 12 Aga Mooses rääkis Issanda ees, öeldes: „Vaata, Iisraeli lapsed ei kuula mind, kuidas siis vaarao mind kuulda võtab? Mina olen ju huultelt ümberlõikamata.”
\par 13 Aga Issand rääkis Moosesega ja Aaroniga ning andis neile käsu minna Iisraeli laste ja vaarao, Egiptuse kuninga juurde, et viia Iisraeli lapsed Egiptusemaalt välja.
\par 14 Need olid nende perekondade peamehed: Ruubeni, Iisraeli esmasündinu pojad olid: Hanok, Pallu, Hesron ja Karmi; need olid Ruubeni suguvõsad.
\par 15 Siimeoni pojad olid Jemuel, Jaamin, Ohad, Jaakin, Sohar ja Saul, kaananlanna poeg; need olid Siimeoni suguvõsad.
\par 16 Need olid Leevi poegade nimed nende sünnijärgluses: Geerson, Kehat ja Merari; ja Leevi eluaastaid oli sada kolmkümmend seitse aastat.
\par 17 Geersoni pojad olid Libni ja Simei oma suguvõsade kaupa.
\par 18 Kehati pojad olid Amram, Jishar, Hebron ja Ussiel; ja Kehati eluaastaid oli sada kolmkümmend kolm aastat.
\par 19 Merari pojad olid Mahli ja Muusi; need olid Leevi suguvõsad oma sünnijärgluses.
\par 20 Amram võttis enesele naiseks Jookebedi, oma isa õe, ja see tõi temale ilmale Aaroni ja Moosese; ja Amrami eluaastaid oli sada kolmkümmend seitse aastat.
\par 21 Jishari pojad olid Korah, Nefeg ja Sikri.
\par 22 Ussieli pojad olid Miisael, Elsafan ja Sitri.
\par 23 Aaron võttis Eliseba, Amminadabi tütre, Nahsoni õe, enesele naiseks, ja see tõi temale ilmale Naadabi, Abihu, Eleasari ja Iitamari.
\par 24 Korahi pojad olid Assir, Elkana ja Abiasaf; need olid korahlaste suguvõsad.
\par 25 Eleasar, Aaroni poeg, võttis enesele naise Puutieli tütreist, ja see tõi temale ilmale Piinehasi; need olid leviitide perekondade peamehed nende suguvõsade kaupa.
\par 26 Need olid Aaron ja Mooses, kellele Issand ütles: „Viige Iisraeli lapsed nende väehulkade kaupa Egiptusemaalt välja!”
\par 27 Nemad olid, kes rääkisid vaaraoga, Egiptuse kuningaga, et Iisraeli lapsi Egiptusest välja viia - Mooses ja Aaron.
\par 28 Sel päeval, mil Issand rääkis Moosesega Egiptusemaal,
\par 29 Issand rääkis Moosesele, öeldes: „Mina olen Issand! Räägi vaaraole, Egiptuse kuningale kõike, mis ma sulle räägin!”
\par 30 Ja Mooses ütles Issanda ees: „Vaata, ma olen huultelt ümberlõikamata. Kuidas siis vaarao mind kuulda võtaks?”

\chapter{7}

\par 1 Ja Issand ütles Moosesele: „Vaata, ma panen sind vaaraole jumalaks ja su vend Aaron olgu sulle prohvetiks.
\par 2 Sina räägi kõik, mida ma sind käsin, ja su vend Aaron rääkigu vaaraole, et ta laseks Iisraeli lapsed oma maalt ära minna.
\par 3 Aga mina teen kõvaks vaarao südame ja teen palju imetegusid ja tunnustähti Egiptusemaal.
\par 4 Vaarao ei kuula teid, aga mina panen oma käe Egiptuse vastu ja viin oma väehulgad, oma rahva, Iisraeli lapsed Egiptusemaalt välja suurte nuhtluste abil.
\par 5 Ja Egiptus peab tundma, et mina olen Issand, kui ma sirutan oma käe Egiptuse kohale ja viin Iisraeli lapsed nende keskelt välja!”
\par 6 Ja Mooses ja Aaron tegid nõnda; nagu Issand neid oli käskinud, nõnda nad tegid.
\par 7 Mooses oli kaheksakümmend aastat vana ja Aaron oli kaheksakümmend kolm aastat vana, kui nad vaaraoga rääkisid.
\par 8 Ja Issand rääkis Moosese ja Aaroniga, öeldes:
\par 9 „Kui vaarao räägib teiega ja ütleb: Tehke mõni tunnustäht!, siis ütle Aaronile: Võta oma kepp ja viska vaarao ette, siis see muutub maoks!”
\par 10 Siis Mooses ja Aaron läksid vaarao juurde ja tegid nõnda, nagu Issand oli käskinud. Aaron viskas oma kepi vaarao ja ta sulaste ette ja see muutus maoks.
\par 11 Aga vaaraogi kutsus targad ja nõiad, ja Egiptuse võlurid tegid oma salakunstidega ka sedasama:
\par 12 igaüks neist viskas oma kepi maha ja need muutusid madudeks; aga Aaroni kepp neelas nende kepid ära.
\par 13 Ja vaarao süda jäi kõvaks ja ta ei kuulanud neid, nagu Issand oli öelnud.
\par 14 Siis Issand ütles Moosesele: „Vaarao süda on kalk, ta keeldub rahvast ära laskmast.
\par 15 Mine hommikul vaarao juurde, kui ta läheb vee äärde, ja astu temale vastu Niiluse jõe ääres. Võta kätte kepp, mis muutus maoks,
\par 16 ja ütle temale: Issand, heebrealaste Jumal, on mind läkitanud sinu juurde, et ma ütleksin: Lase mu rahvas minna ja mind kõrbes teenida! Aga vaata, sa ei ole senini kuulda võtnud.
\par 17 Seepärast ütleb Issand nõnda: Sellest sa tunned, et mina olen Issand: vaata, ma löön kepiga, mis mul käes on, jõe vett, ja see muutub vereks.
\par 18 Kalad jões surevad ja jõgi hakkab haisema, nõnda et egiptlased jõest vett juues tunnevad tülgastust.”
\par 19 Ja Issand ütles Moosesele: „Ütle Aaronile: Võta oma kepp ja siruta käsi egiptlaste vete kohale, nende jõgede, kanalite, tiikide ja kõigi veekogude kohale, et need muutuksid vereks. Siis on kogu Egiptusemaal veri, niihästi puu- kui kiviastjais.”
\par 20 Ja Mooses ja Aaron tegid nõnda, nagu Issand oli käskinud. Ta tõstis kepi üles ning lõi Niiluse jõe vett vaarao ja ta sulaste silma ees, ja kõik vesi jões muutus vereks.
\par 21 Ja kalad jões surid, jõgi haises ja egiptlased ei saanud jõest vett juua; ja veri oli kogu Egiptusemaal.
\par 22 Aga Egiptuse võlurid tegid oma salakunstidega sedasama. Ja vaarao süda jäi kõvaks ning ta ei kuulanud neid, nagu Issand oli öelnud.
\par 23 Ja vaarao pöördus ümber ning läks oma kotta ega võtnud ka seda südamesse.
\par 24 Aga kõik egiptlased kaevasid Niiluse jõe ümbruses joogivett otsides, sest nad ei saanud jõe vett juua.
\par 25 Möödus seitse päeva Niiluse jõe löömisest Issanda poolt.

\chapter{8}

\par 1 Ja Issand ütles Moosesele: „Ütle Aaronile: Siruta oma käsi kepiga välja jõgede, kanalite ja tiikide kohale, ja lase tulla konni Egiptusemaale!”
\par 2 Ja Aaron sirutas oma käe Egiptuse vete kohale ning konnad ronisid üles ja katsid Egiptusemaa.
\par 3 Aga võlurid tegid oma salakunstidega sedasama ja lasksid tulla konni Egiptusemaale.
\par 4 Siis vaarao kutsus Moosese ja Aaroni ning ütles: „Paluge Issandat, et ta võtaks ära konnad minu ja mu rahva kallalt, siis ma lasen rahva minna Issandale ohverdama!”
\par 5 Aga Mooses vastas vaaraole: „Osuta mulle seda au: millal ma pean palvetama sinu ja su sulaste ning su rahva pärast, et konnad kaotataks sinu ja su kodade kallalt, et nad jääksid üksnes jõkke?”
\par 6 Ja tema vastas: „Homme.” Siis ütles Mooses: ”Sinu sõna peale! Et sa teaksid, et keegi ei ole niisugune nagu Issand, meie Jumal.
\par 7 Konnad eemaldatakse sinu ja su kodade, su sulaste ja su rahva kallalt. Nad jäävad üksnes jõkke.”
\par 8 Siis Mooses läks koos Aaroniga vaarao juurest välja. Ja Mooses hüüdis Issanda poole konnade pärast, keda ta oli vaaraole saatnud.
\par 9 Ja Issand tegi Moosese sõna järgi ning konnad surid kodadest, õuedest ja põldudelt,
\par 10 neid kuhjati hunnikute viisi ja maa hakkas haisema.
\par 11 Kui vaarao nägi, et ta oli saanud kergendust, siis ta tegi oma südame kõvaks ega kuulanud neid - nagu Issand oli öelnud.
\par 12 Siis Issand ütles Moosesele: „Ütle Aaronile: Siruta oma kepp välja ja löö maa põrmu, et sellest tuleks sääski kogu Egiptusemaale!”
\par 13 Ja nad tegid nõnda. Aaron sirutas oma käe kepiga välja ja lõi maa põrmu; siis tulid sääsed inimeste ja loomade kallale; kõik maa põrm muutus sääskedeks kogu Egiptusemaal.
\par 14 Ka võlurid tegid oma salakunstidega sedasama, et tekitada sääski, aga ei suutnud; ja sääsed olid inimeste ja loomade kallal.
\par 15 Siis võlurid ütlesid vaaraole: „See on Jumala sõrm!” Aga vaarao süda jäi kõvaks ja ta ei kuulanud neid - nagu Issand oli öelnud.
\par 16 Ja Issand ütles Moosesele: „Tõuse hommikul vara ja astu vaarao ette, kui ta läheb vee äärde, ja ütle temale: Nõnda ütleb Issand: Lase mu rahvas minna ja mind teenida!
\par 17 Sest kui sa ei lase mu rahvast minna, vaata, siis ma läkitan parmud sinu ja su sulaste ja su rahva kallale ning su kodadesse. Egiptlaste kojad täituvad parmudega, nõndasamuti ka maapind, millel need on.
\par 18 Aga ma eraldan sel päeval Gooseni maakonna, kus asub mu rahvas, nõnda et parme ei tule sinna, selleks et sa teaksid, et mina olen Issand keset seda maad.
\par 19 Ma teen vahe oma rahva ja sinu rahva vahele. Homme sünnib see imetegu.”
\par 20 Ja Issand tegi nõnda: parme tuli rängasti vaarao kotta ja tema sulaste kodadesse; ja kogu Egiptusemaal kannatas maa parmude tõttu.
\par 21 Siis vaarao kutsus Moosese ja Aaroni ning ütles: „Minge ohverdage oma Jumalale siin maal!”
\par 22 Aga Mooses vastas: „Ei ole sünnis nõnda teha, sest see, mis me ohverdame Issandale, oma Jumalale, on egiptlastele vastik. Vaata, kui me ohverdame seda, mis egiptlaste meelest on vastik, nende silme ees, eks nad viska meid siis kividega?
\par 23 Me tahame minna kolme päeva tee kõrbesse ja ohverdada Issandale, oma Jumalale, nagu tema meid on käskinud.”
\par 24 Ja vaarao ütles: „Ma lasen teid minna ja ohverdada Issandale, teie Jumalale kõrbes. Ainult ärge minge väga kaugele. Palvetage minu eest!”
\par 25 Siis ütles Mooses: „Vaata, ma lähen su juurest ja palun Issandat, et parmud homme kaoksid vaarao, ta sulaste ja rahva kallalt. Ainult ärgu vaarao enam petku, laskmata rahvast minna Issandale ohverdama!”
\par 26 Ja Mooses läks ära vaarao juurest ning palus Issandat.
\par 27 Ja Issand tegi Moosese sõna järgi ning kaotas parmud vaarao, ta sulaste ja rahva kallalt; ühtainsatki ei jäänud alles.
\par 28 Aga vaarao tegi oma südame kõvaks ka seekord ega lasknud rahvast minna.

\chapter{9}

\par 1 Siis Issand ütles Moosesele: „Mine vaarao juurde ja räägi temale: Nõnda ütleb Issand, heebrealaste Jumal: Lase mu rahvas minna ja mind teenida!
\par 2 Sest kui sa keelad neid minemast ja pead neid veel kinni,
\par 3 vaata, siis on Issanda käsi su karja peal, kes on väljal: hobuste, eeslite, kaamelite, veiste, lammaste ja kitsede peal väga raske katkuga.
\par 4 Aga Issand eraldab Iisraeli karja ja egiptlaste karja, ja Iisraeli laste omadest ei sure ühtainsatki.”
\par 5 Ja Issand määras aja, öeldes: „Homme teeb Issand seda siin maal.”
\par 6 Ja Issand tegi järgmisel päeval nõnda, ja egiptlaste kogu kari suri, ent Iisraeli laste karjast ei surnud ühtainsatki.
\par 7 Ja kui vaarao läkitas vaatama, ennäe, siis ei olnud Iisraeli karjast surnud ühtainsatki. Aga vaarao süda jäi kõvaks ja ta ei lasknud rahvast minna.
\par 8 Siis Issand ütles Moosesele ja Aaronile: „Võtke endile mõlemad pihud täis sulatusahju tahma ja Mooses puistaku seda vaarao silma ees vastu taevast.
\par 9 See muutub siis tolmuks üle kogu Egiptusemaa ning inimestele ja loomadele kogu Egiptusemaal tulevad mädavillideks arenevad paised.”
\par 10 Ja nad võtsid sulatusahju tahma ning astusid vaarao ette; Mooses puistas seda vastu taevast ja see muutus mädavillideks arenevaiks paiseiks inimestel ja loomadel.
\par 11 Ja võlurid ei suutnud seista Moosese ees paisete pärast, sest paised olid võlureil ja kõigil egiptlastel.
\par 12 Aga Issand tegi kõvaks vaarao südame ja too ei kuulanud neid, nagu Issand oli Moosesele öelnud.
\par 13 Siis Issand ütles Moosesele: „Tõuse hommikul vara ja astu vaarao ette ning ütle temale: Nõnda ütleb Issand, heebrealaste Jumal: Lase mu rahvas minna ja mind teenida!
\par 14 Sest seekord ma saadan kõik oma nuhtlused sulle enesele ja su sulastele ning su rahvale, et sa teaksid, et minu sarnast ei ole kogu maailmas.
\par 15 Kui ma nüüd oma käe välja sirutasin ja sind ja su rahvast lõin katkuga, siis oleksid sa maa pealt kaotatud olnud,
\par 16 aga ma jätsin sind alles just selleks, et näidata sulle oma väge ja teha kuulsaks oma nimi kogu maailmas.
\par 17 Kui sa veel ülbe oled mu rahva vastu ega lase neid minna,
\par 18 vaata, siis ma lasen homme sadada väga rasket rahet, millist ei ole Egiptuses olnud ta asustamisajast tänini.
\par 19 Ja nüüd läkita järele, päästa oma kari ja kõik, kes sul väljal on! Kõigi inimeste ja loomade peale, kes on väljal ega ole viidud koju, langeb rahe ja nad surevad.”
\par 20 Kes vaarao sulaseist kartis Issanda sõna, päästis oma sulased ja karja koju.
\par 21 Aga kes ei võtnud Issanda sõna südamesse, jättis oma sulased ja karja väljale.
\par 22 Ja Issand ütles Moosesele: „Siruta oma käsi taeva poole, siis tuleb rahet kogu Egiptusemaale, inimeste ja loomade peale, ja kõigi taimede peale Egiptusemaal!”
\par 23 Ja Mooses sirutas oma kepi taeva poole ning Issand andis müristamist ja rahet, ja tuli lõi maha; ja Issand laskis Egiptusemaale rahet sadada.
\par 24 Ja rahe ja tuli, mis oli rahega segamini, olid väga rängad, milliseid ei ole olnud kogu Egiptusemaal selle asustamisest alates.
\par 25 Ja rahe lõi maha kogu Egiptusemaal kõik, kes olid väljal, niihästi inimesed kui loomad; ja rahe lõi maha kogu rohu ning murdis kõik puud väljal.
\par 26 Ainult Gooseni maakonnas, kus olid Iisraeli lapsed, ei olnud rahet.
\par 27 Siis vaarao läkitas järele ja kutsus Moosese ja Aaroni ning ütles neile: „Ma olen seekord pattu teinud. Issand on õiglane, aga mina ja mu rahvas oleme õelad.
\par 28 Paluge Issandat, sest on küllalt Jumala müristamisest ja rahest. Ma lasen teid minna ja teil pole enam vaja jääda!”
\par 29 Ja Mooses vastas temale: „Kui ma olen linnast välja läinud, siis ma sirutan oma käed Issanda poole: müristamine lakkab ja rahet ei ole enam, et sa teaksid, et maa on Issanda päralt.
\par 30 Aga sinust ja su sulaseist ma tean, et te veelgi ei karda Jumalat Issandat.”
\par 31 Ja lina ja oder löödi maha, sest oder oli loonud ja lina oli kupras;
\par 32 nisu ja okasnisu aga ei löödud maha, sest need olid hilised.
\par 33 Ja Mooses läks ära vaarao juurest, linnast välja, ja sirutas oma käed Issanda poole: müristamine ja rahe lakkasid ning vihma ei sadanud enam maa peale.
\par 34 Kui vaarao nägi, et vihm, rahe ja müristamine lakkasid, siis ta patustas edasi ja tegi oma südame kõvaks, tema ja ta sulased.
\par 35 Vaarao süda jäi kõvaks ja ta ei lasknud Iisraeli lapsi minna - nagu Issand oli Moosese läbi öelnud.

\chapter{10}

\par 1 Siis Issand ütles Moosesele: „Mine vaarao juurde, sest ma olen teinud kõvaks tema südame ja ta sulaste südamed, et teha tema juures oma imetegusid,
\par 2 ja selleks, et sa saaksid jutustada oma poja ja pojapoja kuuldes, kuidas ma olen näidanud egiptlastele oma jõudu, ja mu imetegudest, mis ma neile olen teinud, et te teaksite, et mina olen Issand!”
\par 3 Siis Mooses ja Aaron läksid vaarao juurde ning ütlesid temale: „Nõnda ütleb Issand, heebrealaste Jumal: Kui kaua sa tõrgud alistumast mu ees? Lase mu rahvas minna ja mind teenida!
\par 4 Sest kui sa keelad mu rahvast minemast, vaata, siis ma toon homme su maale rohutirtsud.
\par 5 Need katavad maapinna, nõnda et maad pole näha, ja nad söövad jäänused säilinust, mis teile rahest üle jäi, ja nad söövad ära kõik puud, mis teil väljal kasvavad.
\par 6 Ja nad täidavad su kojad ja kõik su sulaste kojad ja kõik egiptlaste kojad, mida ei ole näinud su isad ja su isade isad oma maa peale tuleku päevast tänini.” Ja ta pöördus ümber ning läks ära vaarao juurest.
\par 7 Siis vaarao sulased ütlesid temale: „Kui kaua on see meile püüniseks? Lase mehed minna, et nad teeniksid Issandat, oma Jumalat! Kas sa veelgi ei mõista, et Egiptus hukkub?”
\par 8 Siis toodi Mooses ja Aaron tagasi vaarao juurde ja tema ütles neile: „Minge teenige Issandat, oma Jumalat! Aga kes need minejad õieti on?”
\par 9 Ja Mooses vastas: „Me läheme oma noorte ja vanadega, oma poegade ja tütardega, me läheme oma lammaste, kitsede ja veistega, sest meil on Issanda püha.”
\par 10 Tema aga ütles neile: „Issand olevat siis teie juures, kui ma lasen ära teid ja teie väetid lapsed! Te näete ka ise, et teil on kuri nõu.
\par 11 Nõnda ei sünni! Mingu ainult mehed ja teenigu Issandat, sest te olete ju seda nõudnud!” Ja nad aeti vaarao eest ära.
\par 12 Siis Issand ütles Moosesele: „Siruta oma käsi Egiptusemaa kohale rohutirtsude pärast, et need tuleksid Egiptusemaale ja sööksid ära kõik maa rohu, kõik, mis rahe on alles jätnud!”
\par 13 Ja Mooses sirutas oma kepi välja Egiptusemaa kohale ning Issand saatis maale idatuule kogu selle päeva ja kogu selle öö; kui hommik tuli, tõi idatuul kaasa rohutirtsud.
\par 14 Rohutirtsud tulid kogu Egiptusemaale ja laskusid väga suurel hulgal kõigisse Egiptuse paigusse; enne seda ei ole rohutirtse sel määral olnud ega ole neid olnud ka enam pärast seda.
\par 15 Need katsid kogu maapinna, nõnda et maa mustas; nad sõid ära kõik maa rohu ja kõik puude vilja, mis rahe oli üle jätnud. Ja kogu Egiptusemaal ei jäänud üle midagi haljast puudel ega väljarohtudel.
\par 16 Siis vaarao kutsus kiiresti Moosese ja Aaroni ning ütles: „Ma olen pattu teinud Issanda, teie Jumala, ja teie vastu.
\par 17 Nüüd aga andke mu patt veel seekord andeks ja paluge Issandat, oma Jumalat, et ta ometi võtaks minult selle surma!”
\par 18 Ja ta läks ära vaarao juurest ning palus Issandat.
\par 19 Siis Issand pööras tuule väga kangeks läänetuuleks, see tõstis rohutirtsud üles ja ajas need Kõrkjamerre; ühtainsatki rohutirtsu ei jäänud alles kogu Egiptuse maa-alale.
\par 20 Aga Issand tegi kõvaks vaarao südame ja too ei lasknud Iisraeli lapsi minna.
\par 21 Siis Issand ütles Moosesele: „Siruta oma käsi taeva poole, siis tuleb Egiptusemaale niisugune pimedus, et katsu või käega!”
\par 22 Ja Mooses sirutas oma käe taeva poole ning kogu Egiptusemaale tuli kolmeks päevaks pilkane pimedus:
\par 23 üks ei näinud teist ja ükski ei liikunud paigast kolmel päeval. Aga kõigil Iisraeli lastel oli oma asupaikades valge.
\par 24 Siis vaarao kutsus Moosese ning ütles: „Minge teenige Issandat! Ainult teie lambad, kitsed ja veised jäägu paigale. Ka teie väetid lapsed mingu koos teiega!”
\par 25 Aga Mooses vastas: „Sina pead meile kaasa andma ka tapa- ja põletusohvrid, et saaksime neid tuua Issandale, oma Jumalale!
\par 26 Meie karigi peab tulema koos meiega, sõrgagi ei tohi maha jääda, sest sellest me ju võtame Issanda, oma Jumala teenimiseks. Me ei tea isegi, millega peame teenima Issandat, enne kui jõuame sinna.”
\par 27 Aga Issand tegi kõvaks vaarao südame, nõnda et ta ei tahtnud neid ära lasta.
\par 28 Ja vaarao ütles temale: „Mine ära mu juurest! Hoia, et sa enam ei ilmu mu palge ette! Sest päeval, mil sa ilmud mu palge ette, sa sured!”
\par 29 Ja Mooses vastas: „Õigesti oled rääkinud! Enam ma ei ilmu su palge ette!”

\chapter{11}

\par 1 Siis Issand ütles Moosesele: „Veel ühe nuhtluse ma saadan vaaraole ja Egiptusele: pärast seda ta laseb teid siit ära minna. Kui ta lõppeks laseb minna, siis ta otse ajab teid siit ära.
\par 2 Räägi nüüd rahva kuuldes, et iga mees küsiks oma naabrilt ja iga naine oma naabrinaiselt hõbe- ja kuldriistu!”
\par 3 Ja Issand andis rahvale armu egiptlaste silmis; ka oli Mooses väga suur mees Egiptusemaal vaarao sulaste ja rahva silmis.
\par 4 Ja Mooses ütles: „Nõnda ütleb Issand: Keskööl ma lähen läbi egiptlaste keskelt:
\par 5 Egiptusemaal peavad surema kõik esmasündinud, aujärjel istuva vaarao esmasündinust kuni käsikivitaguse teenija esmasündinuni, samuti kõik kariloomade esmasündinud.
\par 6 Siis on kogu Egiptusemaal suur hädakisa, millist ei ole olnud ja millist ei tule enam.
\par 7 Aga ühegi Iisraeli lapse vastu ei liiguta koergi oma keelt, ei inimese ega karilooma vastu, et te teaksite, et Issand teeb vahet Egiptuse ja Iisraeli vahel.
\par 8 Siis tulevad kõik need sinu sulased alla minu juurde ja kummardavad mind, öeldes: Mine ära, sina ja kogu rahvas, kes su kannul käib! Ja seejärel ma lähen.” Ja ta läks ära vaarao juurest vihast õhetades.
\par 9 Aga Issand ütles Moosesele: „Vaarao ei võta teid kuulda, et Egiptusemaal sünniks rohkesti mu tunnustähti.”
\par 10 Ja Mooses ja Aaron tegid kõik need tunnustähed vaarao ees. Aga Issand tegi kõvaks vaarao südame ja too ei lasknud Iisraeli lapsi oma maalt ära minna.

\chapter{12}

\par 1 Ja Issand rääkis Moosese ja Aaroniga Egiptusemaal, öeldes:
\par 2 „See kuu olgu teil esimeseks kuuks; see olgu teil aasta esimeseks kuuks.
\par 3 Rääkige kogu Iisraeli kogudusega ja öelge: Selle kuu kümnendal päeval võtku iga perevanem tall, igale perele tall.
\par 4 Aga kui pere on talle jaoks väike, siis võtku tema ja ta naaber, kes ta kojale on lähemal, vastavalt hingede arvule; kui palju igaüks jõuab süüa, sellele vastavalt arvake neid talle kohta.
\par 5 Tall olgu teil veatu, isane, üheaastane; võtke see lammastest või kitsedest.
\par 6 Säilitage see enestele kuni selle kuu neljateistkümnenda päevani; siis kogu Iisraeli kogunenud kogudus tapku see õhtul!
\par 7 Ja nad võtku verd ning võidku ukse mõlemat piitjalga ja pealispuud kodades, kus nad seda söövad.
\par 8 Ja nad söögu liha selsamal ööl; tulel küpsetatult koos hapnemata leiva ja kibedate rohttaimedega söögu nad seda!
\par 9 Te ei tohi seda süüa toorelt või vees keedetult, vaid ainult tulel küpsetatult pea, jalgade ja sisikonnaga.
\par 10 Te ei tohi sellest midagi üle jätta hommikuks; mis aga sellest hommikuks üle jääb, põletage tulega!
\par 11 Ja sööge seda nõnda: teil olgu vöö vööl, jalatsid jalas ja kepp käes; ja sööge seda rutuga - see on paasatall Issanda auks!
\par 12 Siis käin mina selsamal ööl Egiptusemaa läbi ja löön maha kõik esmasündinud Egiptusemaal, niihästi inimesed kui loomad, ja ma mõistan kohut kõigi Egiptuse jumalate üle, mina, Issand.
\par 13 Aga veri olgu teil tundemärgiks kodadel, kus te asute; kui ma näen verd, siis ma lähen teist mööda ja nuhtlus ei saa teile hukatuseks, kui ma löön Egiptusemaad.
\par 14 See päev aga jäägu teile mälestuseks ja pühitsege seda Issanda pühana: oma sugupõlvede kaupa pühitsege seda igavese seadlusena!
\par 15 Seitse päeva sööge hapnemata leiba; juba esimesel päeval kõrvaldage haputaigen oma kodadest, sest igaüks, kes esimesest päevast seitsmenda päevani sööb hapnenut, selle hing hävitatakse Iisraelist.
\par 16 Esimesel päeval olgu teil pühalik kokkutulek, samuti olgu seitsmendal päeval pühalik kokkutulek: neil päevil ei tohi teha ühtki tööd, ainult mida iga hing sööb, üksnes seda valmistage!
\par 17 Te peate pidama seda hapnemata leibade püha, sest just sel päeval ma viin teie väehulgad Egiptusemaalt välja; seepärast pidage seda päeva kui igavest seadlust teie sugupõlvedele!
\par 18 Esimese kuu neljateistkümnenda päeva õhtul sööge hapnemata leiba kuni kuu kahekümne esimese päeva õhtuni.
\par 19 Seitse päeva ärgu leidugu haputaignat teie kodades, sest igaüks, kes sööb hapnenut, selle hing tuleb hävitada Iisraeli kogudusest, olgu võõras või maa päriselanik.
\par 20 Midagi hapnenut ärge sööge, vaid kõigis oma asupaikades sööge hapnemata leiba!”
\par 21 Ja Mooses kutsus kõik Iisraeli vanemad ning ütles neile: „Minge ja võtke enestele suguvõsade kaupa talled ja tapke paasatall!
\par 22 Ja võtke iisopikimbuke, kastke kausis olevasse verre ja määrige ukse pealispuud ning mõlemat piitjalga kausis oleva verega! Ja ükski teist ärgu väljugu hommikuni oma koja uksest,
\par 23 sest Issand läheb egiptlasi nuhtlema. Aga kui ta näeb verd ukse pealispuul ja mõlemal piitjalal, siis Issand ruttab sellest uksest mööda ega lase hävitajat tulla teie kodadesse nuhtlema.
\par 24 Pidage see asi meeles: see olgu igaveseks seadluseks sinule ja su lastele!
\par 25 Ja kui te tulete maale, mille Issand teile annab, nagu ta on öelnud, siis pidage seda teenistust!
\par 26 Ja kui teie lapsed teilt küsivad: Mis teenistus see teil on?,
\par 27 siis vastake: See on paasaohver Issanda auks, kes ruttas mööda Iisraeli laste kodadest Egiptuses, kui ta nuhtles egiptlasi ja päästis meie kojad.” Siis rahvas kummardas ning heitis silmili maha.
\par 28 Ja Iisraeli lapsed läksid ning tegid; nagu Issand oli Moosesele ja Aaronile käsu andnud, nõnda nad tegid.
\par 29 Ja see sündis keskööl, et Issand lõi maha kõik esmasündinud Egiptusemaal, aujärjel istuva vaarao esmasündinust alates kuni vangiurkas oleva vangi esmasündinuni, ja kõik kariloomade esmasündinud.
\par 30 Siis vaarao tõusis öösel üles, tema ja kõik ta sulased, ning kõik egiptlased, ja Egiptuses oli suur hädakisa, sest ei olnud ainsatki koda, kus ei olnud surnut.
\par 31 Ja ta kutsus öösel Moosese ja Aaroni ning ütles: „Võtke kätte ja minge ära mu rahva keskelt, niihästi teie kui ka Iisraeli lapsed, ja minge teenige Issandat, nagu te olete rääkinud!
\par 32 Võtke ka niihästi oma lambad ja kitsed kui veised, nagu te olete rääkinud, ja minge! Ja õnnistage ka mind!”
\par 33 Ja egiptlased käisid rahvale peale, et nad kiiresti lahkuksid maalt, sest nad ütlesid: „Me sureme kõik.”
\par 34 Ja rahvas viis ära oma taigna, enne kui see oli hapnenud; neil olid nende leivakünad üleriietesse seotuina õlal.
\par 35 Ja Iisraeli lapsed olid teinud Moosese sõna järgi ning olid palunud egiptlastelt hõbe- ja kuldriistu ning riideid.
\par 36 Issand oli rahvale armu andnud egiptlaste silmis ja need nõustusid; nii nad riisusid egiptlasi.
\par 37 Ja Iisraeli lapsed läksid teele Raamsesest Sukkotti, ligi kuussada tuhat jalameest, peale väetite laste.
\par 38 Ja ka hulk segarahvast läks koos nendega, ning lambaid, kitsi ja veiseid väga suur kari.
\par 39 Ja nad küpsetasid taignast, mis nad Egiptusest olid toonud, hapnemata leivakakkusid; see polnud ju hapnenud, sellepärast et nad Egiptusest välja aeti ja nad ei võinud viivitada, samuti mitte enestele teerooga valmistada.
\par 40 Iisraeli laste elamisaega, mis nad Egiptuses olid elanud, oli nelisada kolmkümmend aastat.
\par 41 Kui need nelisada kolmkümmend aastat lõppesid, siis just selsamal päeval sündis see, et kõik Issanda väehulgad läksid Egiptusemaalt välja.
\par 42 See oli valvamisöö Issandale, nende väljaviimiseks Egiptusemaalt; see on Issandale kuuluv öö, valvamiseks kõigile Iisraeli laste sugupõlvedele.
\par 43 Ja Issand ütles Moosesele ja Aaronile: „See on paasatalle seadlus: ükski võõras ei tohi seda süüa!
\par 44 Aga raha eest ostetud iga sulane võib seda süüa siis, kui oled tema ümber lõiganud.
\par 45 Majaline ja palgaline ärgu seda söögu!
\par 46 Ühes ja samas kojas tuleb seda süüa, lihast ei tohi midagi viia kojast välja õue, ja luid ei tohi sellel murda!
\par 47 Kogu Iisraeli kogudus pidagu seda!
\par 48 Ja kui su juures viibib mõni võõras ning tahab valmistada Issandale paasatalle, siis tuleb kõik ta meesterahvad ümber lõigata; alles siis tohib ta ligi tulla seda valmistama ja on nagu maa päriselanik; aga ükski ümberlõikamatu ei tohi seda süüa!
\par 49 Seadlus on üks päriselanikule ja muulasele, kes võõrana teie keskel elab.”
\par 50 Ja kõik Iisraeli lapsed tegid, nagu Issand Moosest ja Aaronit oli käskinud; nõnda nad tegid.
\par 51 Ja just selsamal päeval Issand viis Iisraeli lapsed väehulkadena Egiptusemaalt välja.

\chapter{13}

\par 1 Ja Issand rääkis Moosesega, öeldes:
\par 2 „Pühitse mulle kõik esmasündinud, kõik Iisraeli laste emakodade avajad inimestest ja kariloomadest - need olgu minu!”
\par 3 Ja Mooses ütles rahvale: „Mõelge päevale, mil te tulite välja Egiptusest, orjusekojast, sest vägeva käega tõi Issand teid sealt välja! Sellepärast ei tohi hapnenut süüa!
\par 4 Täna, aabibikuus, te lähete välja.
\par 5 Ja kui Issand sind viib kaananlaste, hettide, emorlaste, hiivlaste ja jebuuslaste maale, mille ta vandega su vanemaile on tõotanud sulle anda, maale, mis piima ja mett voolab, siis pead sa selles kuus pidama seda teenistust.
\par 6 Seitse päeva söö hapnemata leiba ja seitsmendal päeval olgu Issanda püha!
\par 7 Hapnemata leiba söödagu seitse päeva; ärgu nähtagu su juures hapnenut ja ärgu nähtagu haputaignat kogu su maa-alal!
\par 8 Ja sel päeval seleta oma pojale, öeldes: See on ühenduses sellega, mis Issand mulle tegi, kui ma Egiptusest lahkusin.
\par 9 See olgu sulle nagu märgiks käe peal ja meeldetuletuseks silmade vahel, et Issanda Seadus oleks su suus; sest Issand tõi sind vägeva käega Egiptusest välja.
\par 10 Seda seadlust pead sa täitma määratud ajal aastast aastasse!
\par 11 Ja kui Issand sind on viinud kaananlaste maale, nagu ta sulle ja su vanemaile on vandunud, ja on andnud selle sulle,
\par 12 siis vii Issandale kõik, kes avavad emakoja; ja kõik isased esmikud su kariloomade kandest olgu Issanda päralt!
\par 13 Iga eesli esmik lunasta ühe tallega; aga kui sa teda ei lunasta, siis murra tal kael! Ja lunasta iga inimese esmasündinu oma poegade hulgas!
\par 14 Ja kui su poeg sinult tulevikus küsib, öeldes: Mida see tähendab?, siis vasta temale: Vägeva käega tõi Issand meid välja Egiptusest, orjusekojast.
\par 15 Siis kui vaarao oli kalgilt meie mineku vastu, tappis Issand kõik esmasündinud Egiptusemaal, inimeste esmasündinuist kariloomade esmasündinuteni. Seepärast ma ohverdan Issandale kõik emakoja avajad isased ja lunastan kõik oma poegade esmasündinud.
\par 16 See olgu märgiks su käe peal ja naastuks su silmade vahel; sest vägeva käega tõi Issand meid Egiptusest välja!”
\par 17 Aga kui vaarao oli lasknud rahva minna, siis ei viinud Jumal neid mööda vilistite maa teed, kuigi see oli ligem, sest Jumal mõtles, et sõda nähes rahvas kahetseb ja pöördub tagasi Egiptusesse,
\par 18 vaid Jumal laskis rahva pöörduda kõrbeteed Kõrkjamere poole; ja võitlusvalmilt läksid Iisraeli lapsed Egiptusemaalt välja.
\par 19 Ja Mooses võttis enesega kaasa Joosepi luud, sest tema oli Iisraeli lapsi vandega kohustanud, öeldes: „Jumal hoolitseb kindlasti teie eest. Siis viige ka minu luud siit enestega kaasa!”
\par 20 Ja nad läksid teele Sukkotist ning lõid leeri üles Eetamisse, kõrbe äärde.
\par 21 Ja Issand käis nende ees, päeval pilvesambas juhatamas neile teed, ja öösel tulesambas, andes neile valgust, et nad said minna päeval ja öösel.
\par 22 Ei lahkunud pilvesammas päeval ega tulesammas öösel rahva eest.

\chapter{14}

\par 1 Ja Issand rääkis Moosesega, öeldes:
\par 2 „Ütle Iisraeli lastele, et nad pöörduksid tagasi ja lööksid leeri üles Pii-Hahiroti kohale, Migdoli ja mere vahele, Baal-Sefoni ette, sellega vastakuti; lööge leer üles mere äärde!
\par 3 Sest vaarao mõtleb, et Iisraeli lapsed ekslevad mööda maad, kõrb peab nad kinni.
\par 4 Mina teen kõvaks vaarao südame ja ta ajab neid taga. Aga mina ilmutan oma au vaarao ja kogu tema sõjaväe arvel, et egiptlased tunneksid, et mina olen Issand.” Ja nad tegid nõnda.
\par 5 Kui Egiptuse kuningale anti teada, et rahvas oli põgenenud, siis vaarao ja ta sulaste süda pöördus rahva vastu, ja nad ütlesid: „Miks tegime nõnda, et lasksime Iisraeli meid orjamast!”
\par 6 Ja ta laskis rakendada hobused sõjavankrite ette ning võttis oma rahva enesega kaasa.
\par 7 Ta võttis kuussada valitud sõjavankrit ja kõik muud Egiptuse sõjavankrid, ja võitlejaid nende kõigi jaoks.
\par 8 Ja Issand tegi kõvaks vaarao, Egiptuse kuninga südame, ning too ajas taga Iisraeli lapsi; Iisraeli lapsed aga olid välja läinud ülestõstetud käe kaitsel.
\par 9 Ja egiptlased, kõik vaarao hobused, sõjavankrid ja tema ratsanikud ning sõjavägi, ajasid neid taga ning jõudsid neile järele, kui nad olid leeris mere ääres Pii-Hahirotis, Baal-Sefoni kohal.
\par 10 Kui vaarao oli ligidal, siis Iisraeli lapsed tõstsid oma silmad üles, ja vaata, egiptlased olid tulemas neile järele. Siis Iisraeli lapsed kartsid väga ja kisendasid Issanda poole.
\par 11 Ja nad ütlesid Moosesele: „Kas ei olnud siis Egiptuses haudu, et tõid meid kõrbe surema? Miks tegid meile seda, tuues meid Egiptusest välja?
\par 12 Eks see just olegi, mis me rääkisime sulle Egiptuses, öeldes: Jäta meid rahule, et võiksime teenida egiptlasi! Sest meil on parem egiptlasi teenida kui kõrbes surra.”
\par 13 Aga Mooses vastas rahvale: „Ärge kartke, püsige paigal, siis te näete Issanda päästet, mille ta täna teile valmistab! Sest egiptlasi, keda te näete täna, ei näe te edaspidi enam iialgi.
\par 14 Issand sõdib teie eest, aga teie vaikige!”
\par 15 Ja Issand ütles Moosesele: „Miks sa minu poole kisendad? Ütle Iisraeli lastele, et nad läheksid edasi!
\par 16 Sina aga tõsta oma kepp üles, siruta käsi mere kohale ja lõhesta see, et Iisraeli lapsed saaksid minna kuiva mööda läbi mere!
\par 17 Ja mina, vaata, teen siis kõvaks egiptlaste südamed ja nad tulevad teile järele. Aga mina ilmutan oma au vaarao ja kogu ta sõjaväe, ta sõjavankrite ja ratsanike arvel,
\par 18 et egiptlased tunneksid, et mina olen Issand, kui ma ilmutan oma au vaarao, tema sõjavankrite ja ratsanike arvel.”
\par 19 Siis Jumala ingel, kes oli käinud Iisraeli leeri ees, siirdus ning läks nende taha; ja pilvesammas siirdus nende eest ning seisis nende taga,
\par 20 tulles egiptlaste leeri ja Iisraeli leeri vahele; siis oli pilv ühele pime, aga teisele valgustas ööd, ja üks ei pääsenud teisele ligi kogu öö.
\par 21 Siis Mooses sirutas oma käe mere kohale ja Issand laskis mere taanduda tugevast idatuulest kogu öö ning tegi mere kuivaks - vesi lõhenes.
\par 22 Ja Iisraeli lapsed läksid läbi mere kuiva mööda ja vesi oli neil müürina paremal ja vasakul pool.
\par 23 Ja egiptlased ajasid neid taga ning tulid neile järele, kõik vaarao hobused, tema sõjavankrid ja ratsanikud, keset merd.
\par 24 Kui saabus hommikune vahikord, siis Issand vaatas egiptlaste leeri peale tule- ja pilvesambast, ja viis egiptlaste leeri segadusse.
\par 25 Ja ta kiilus kinni nende sõjavankrite rattad ning takistas nende sõitu. Siis ütlesid egiptlased: „Põgenegem Iisraeli eest, sest Issand sõdib nende poolt egiptlaste vastu!”
\par 26 Aga Issand ütles Moosesele: „Siruta oma käsi välja mere kohale, et vesi tuleks tagasi egiptlaste, nende sõjavankrite ja ratsanike peale!”
\par 27 Ja Mooses sirutas oma käe välja mere kohale ning koiduajal meri pöördus tagasi oma paika, egiptlased aga põgenesid sellele vastu; ja Issand paiskas egiptlased keset merd.
\par 28 Ja vesi tuli tagasi ning kattis sõjavankrid ja ratsanikud, kogu vaarao sõjaväe, kes oli tulnud neile merre järele; ei jäänud neist üle ühtainsatki.
\par 29 Iisraeli lapsed aga läksid kuiva mööda läbi mere, ja vesi oli neil müürina paremal ja vasakul pool.
\par 30 Nõnda päästis Issand sel päeval Iisraeli egiptlaste käest ja Iisrael nägi egiptlasi surnuina mere rannal.
\par 31 Ja Iisrael nägi seda suurt kätt, mida Issand näitas egiptlastele, ja rahvas kartis Issandat ning nad uskusid Issandat ja Moosest, tema sulast.

\chapter{15}

\par 1 Mooses ja Iisraeli lapsed laulsid siis Issandale selle laulu; nad ütlesid nõnda: „Ma laulan Issandale, sest tema on Ülikõrge, hobused ja ratsanikud heitis ta merre.
\par 2 Mu tugevus ja mu kiituslaul on Issand, tema oli mulle päästeks. Tema on mu Jumal ja ma ülistan teda, tema on mu isa Jumal ja ma kiidan teda kõrgeks.
\par 3 Issand on sõjamees, Issand on ta nimi.
\par 4 Vaarao sõjavankrid ja väe heitis ta merre, selle valitud võitlejad uputati Kõrkjameres.
\par 5 Vetevood katsid nad, nad vajusid kivina sügavusse.
\par 6 Issand, su parem käsi näitas oma jõudu; Issand, su parem käsi purustas vaenlase.
\par 7 Suurima üleolekuga sa rebisid vastased maha, sa läkitasid oma vihaleegi, see põletas nad kõrtena.
\par 8 Su vihapuhang paisutas vee, vallina seisis voolus, vood tardusid mere südames.
\par 9 Vaenlane mõtles: „Ajan taga, võtan kinni, jaotan saagi - mu hing täitub sellest. Tõmban oma mõõga, oma käega hävitan nad.”
\par 10 Sina puhusid tuult, meri kattis nad, tinana vajusid nad võimsasse vette.
\par 11 Kes on sinu sarnane jumalate keskel, Issand? Kes on sinu sarnane, pühakute keskel ülistatu, kardetava kuulsusega imetegija?
\par 12 Sina sirutasid oma parema käe, maa neelas nad.
\par 13 Oma armus sa juhtisid seda rahvast, kelle sa lunastasid; oma väes sa talutasid teda oma püha eluaseme juurde.
\par 14 Rahvad kuulsid ja värisesid, ahastus haaras Vilistimaa elanikke.
\par 15 Siis Edomi pealikud ehmusid, Moabi vürste valdas värin, kõik Kaanani elanikud vabisesid.
\par 16 Heitumus ja hirm tabas neid, nad tummusid kivina su võimsa käsivarre pärast, kui su rahvas, Issand, läks läbi, kui läks läbi see rahvas, kelle sina oled loonud.
\par 17 Sa viid selle ja istutad oma pärisosa mäele, paika, mille sina, Issand, oled teinud oma asupaigaks, pühamusse, Issand, mille valmistavad sinu käed.
\par 18 Issand on kuningas ikka ja igavesti!”
\par 19 Kui siis vaarao hobused, ta sõjavankrid ja ratsanikud läksid merre ja Issand tõi tagasi nende peale mere vee, Iisraeli lapsed aga käisid kuiva mööda keset merd,
\par 20 siis naisprohvet Mirjam, Aaroni õde, võttis trummi kätte, ja kõik naised käisid tema järel trummidega ja ringtantsu tantsides.
\par 21 Ja Mirjam laulis neile: „Laulge Issandale, sest tema on Ülikõrge, hobused ja ratsanikud heitis ta merre!”
\par 22 Siis Mooses käskis Iisraeli Kõrkjamere äärest edasi minna; nad läksid Suuri kõrbesse ja käisid kõrbes kolm päeva ega leidnud vett.
\par 23 Nad jõudsid Maarasse, aga ei saanud Maara vett juua, sest see oli kibe; seepärast pandi sellele nimeks Maara.
\par 24 Ja rahvas nurises Moosesega, öeldes: „Mida me joome?”
\par 25 Aga tema hüüdis Issanda poole ja Issand näitas temale ühte puud; siis ta heitis selle vette ja vesi muutus magusaks. Seal andis Issand rahvale seaduse ja õiguse, ja seal ta katsus teda läbi.
\par 26 Ja ta ütles: „Kui sa tõesti kuulad Issanda, oma Jumala häält ja teed, mis õige on tema silmis, paned tähele tema käske ja täidad kõiki tema korraldusi, siis ma ei pane su peale ainsatki neist tõbedest, mis ma panin egiptlaste peale, sest mina olen Issand, su ravija.”
\par 27 Siis nad tulid Eelimisse; seal oli kaksteist veeallikat ja seitsekümmend palmipuud. Ja seal nad lõid leeri üles vee äärde.

\chapter{16}

\par 1 Nad läksid teele Eelimist ja kogu Iisraeli laste kogudus jõudis Siini kõrbe, mis on Eelimi ja Siinai vahel, teise kuu viieteistkümnendal päeval, pärast Egiptusemaalt lahkumist.
\par 2 Ja kogu Iisraeli laste kogudus nurises kõrbes Moosese ja Aaroni vastu,
\par 3 ja Iisraeli lapsed ütlesid neile: „Oleksime ometi võinud surra Issanda käe läbi Egiptusemaal, kus me istusime lihapottide juures, kus me sõime leiba kõhud täis! Teie aga olete meid toonud siia kõrbesse, et kogu seda kogudust nälga suretada.”
\par 4 Siis Issand ütles Moosesele: „Vaata, ma lasen taevast sadada teile leiba ja rahvas mingu ning kogugu iga päev oma osa, sest ma panen nad proovile: kas nad käivad minu Seaduse järgi või mitte?
\par 5 Aga kui nad kuuendal päeval valmistavad, mis nad on koju toonud, siis on seda kahekordselt rohkem, kui nad iga päev on kogunud.”
\par 6 Siis Mooses ja Aaron ütlesid kõigile Iisraeli lastele: „Täna õhtul te saate teada, et see on Issand, kes teid tõi välja Egiptusemaalt,
\par 7 ja hommikul te näete Issanda auhiilgust. Tema on kuulnud teie nurisemist Issanda vastu. Aga kes oleme meie, et te nurisete ka meie vastu?”
\par 8 Ja Mooses ütles: „Issand annab teile täna õhtul liha toiduks ja hommikul leiba kõhutäiteks. Issand on kuulnud teie nurisemist, kuidas te olete nurisenud tema vastu. Aga kes oleme meie? Teie nurisemine ei ole meie vastu, vaid on Issanda vastu.”
\par 9 Ja Mooses ütles Aaronile: „Ütle kogu Iisraeli laste kogudusele: Tulge Issanda palge ette, sest tema on teie nurisemist kuulnud!”
\par 10 Ja sündis, et kui Aaron oli rääkinud kogu Iisraeli laste kogudusega ja nad pöördusid kõrbe poole, vaata, siis nähti pilves Issanda auhiilgust.
\par 11 Ja Issand rääkis Moosesega, öeldes:
\par 12 „Ma olen kuulnud Iisraeli laste nurisemist. Räägi nendega ja ütle: Täna õhtul te sööte liha ja hommikul leiba kõhud täis. Siis te mõistate, et mina olen Issand, teie Jumal.”
\par 13 Ja õhtul tulid vutid ning katsid leeri; ja hommikul oli kastekord leeri ümber.
\par 14 Ja kui kastekord oli haihtunud, vaata, siis oli kõrbe pinnal midagi õhukese soomuse taolist, peenikest nagu härmatis maas.
\par 15 Kui Iisraeli lapsed seda nägid, siis nad küsisid üksteiselt: „Mis see on?” Sest nad ei teadnud, mis see oli. Aga Mooses vastas neile: ”See on leib, mida Issand annab teile süüa.
\par 16 Issand käskis nõnda: Igaüks kogugu sellest niipalju, kui ta sööb, kann iga pea kohta, vastavalt teie hingede arvule; igaüks võtku nende jaoks, kes tema telgis on!”
\par 17 Ja Iisraeli lapsed tegid nõnda, ja nad kogusid, üks rohkem ja teine vähem.
\par 18 Aga kui nad mõõtsid kannuga, siis ei olnud ülearu sellel, kes oli kogunud rohkem, ega tundnud puudust see, kes oli kogunud vähem; igaüks oli kogunud nõnda palju, kui ta sõi.
\par 19 Ja Mooses ütles neile: „Ükski ärgu jätku sellest midagi homseks!”
\par 20 Aga nad ei kuulanud Moosest, vaid mõningad jätsid sellest järele järgmiseks hommikuks; see täitus ussidega ja haises. Siis Mooses vihastus nende pärast.
\par 21 Nõnda nad kogusid seda igal hommikul, igaüks niipalju, kui ta sõi; aga kui päike läks palavaks, siis see sulas.
\par 22 Kuuendal päeval nad kogusid leiba kahekordselt, igaühele kaks kannu; siis tulid kõik koguduse vanemad ja teatasid sellest Moosesele.
\par 23 Ja tema ütles neile: „See ongi, millest Issand rääkis. Homme on puhkus, Issanda püha hingamispäev. Mida küpsetate, seda küpsetage, ja mida keedate, seda keetke! Aga kõik, mis teil üle jääb, pange endile homseks tallele!”
\par 24 Ja nad talletasid selle homseks, nagu Mooses käskis; see ei läinud haisema ega tulnud sellesse usse.
\par 25 Ja Mooses ütles: „Sööge seda täna, sest täna on Issanda hingamispäev; täna te seda väljalt ei leia.
\par 26 Kuus päeva saate seda koguda, aga seitsmes päev on hingamispäev, siis seda ei ole.”
\par 27 Seitsmendal päeval läksid ometi mõned rahva hulgast koguma, aga nad ei leidnud midagi.
\par 28 Siis Issand ütles Moosesele: „Kui kaua te tõrgute pidamast minu käske ja Seadust?
\par 29 Vaadake, Issand on andnud teile hingamispäeva; sellepärast ta annab teile kuuendal päeval kahe päeva leiva. Igaüks jäägu paigale, ükski ärgu väljugu kodunt seitsmendal päeval!”
\par 30 Ja rahvas puhkas seitsmendal päeval.
\par 31 Ja Iisraeli sugu pani sellele nimeks „manna”; see oli valge nagu koriandri seeme ja maitses nagu mesikook.
\par 32 Ja Mooses ütles: „Issand käskis nõnda: Üks kannutäis sellest jäägu säilitamiseks teie sugupõlvedele, et nad näeksid leiba, millega mina teid söötsin kõrbes, kui ma teid tõin välja Egiptusemaalt.”
\par 33 Ja Mooses ütles Aaronile: „Võta üks nõu ja pane sellesse kannutäis mannat ning aseta see Issanda ette, säilitamiseks teie sugupõlvedele!”
\par 34 Ja Aaron asetas selle tunnistuse ette, säilitamiseks, nagu Issand oli Moosesele käsu andnud.
\par 35 Ja Iisraeli lapsed sõid mannat nelikümmend aastat, kuni nad jõudsid asustatud maale; nad sõid mannat, kuni nad jõudsid Kaananimaa piirile.
\par 36 Kann on kümnes osa poolest vakast.

\chapter{17}

\par 1 Ja kogu Iisraeli laste kogudus läks liikvele Siini kõrbest, peatuspaigast teise Issanda käsu järgi; ja nad lõid leeri üles Refidimi, aga seal ei olnud rahval vett juua.
\par 2 Siis rahvas riidles Moosesega ja ütles: „Andke meile vett juua!„ Aga Mooses vastas neile: ”Miks te riidlete minuga? Miks te kiusate Issandat?”
\par 3 Ent rahval oli veejanu ja rahvas nurises Moosese vastu ning ütles: „Mispärast sa tõid meid Egiptusest siia, mind ja minu lapsi ning mu karja janusse surema?”
\par 4 Siis Mooses hüüdis Issanda poole, öeldes: „Mis ma pean selle rahvaga tegema? Vähe puudub, et nad viskavad mind kividega surnuks!”
\par 5 Ja Issand ütles Moosesele: „Mine edasi rahva ees ja võta enesega mõningad Iisraeli vanemaist! Võta kätte oma kepp, millega sa lõid Niiluse jõge, ja mine!
\par 6 Vaata, mina seisan seal su ees Hoorebi kaljul. Löö kaljut, siis tuleb sellest vesi välja ja rahvas saab juua!” Ja Mooses tegi nõnda Iisraeli vanemate nähes.
\par 7 Ja ta pani sellele paigale nimeks Massa ja Meriba, Iisraeli laste riiu pärast, ja et nad olid Issandat kiusanud, öeldes: „Ons Issand meie keskel või ei ole?”
\par 8 Siis tulid amalekid ja sõdisid Iisraeli vastu Refidimis.
\par 9 Ja Mooses ütles Joosuale: „Vali meile mehi ja mine sõdi homme amalekkide vastu! Mina seisan kõrgendiku tipus, Jumala kepp käes.”
\par 10 Ja Joosua tegi, nagu Mooses temale ütles, ning sõdis amalekkide vastu; ja Mooses, Aaron ja Huur läksid kõrgendiku tippu.
\par 11 Ja sündis, et niikaua kui Mooses hoidis oma käe ülal, oli Iisrael võidukas, aga kui ta laskis oma käe vajuda, oli Amalek võidukas.
\par 12 Aga kui Moosese käed väsisid, siis nad võtsid kivi, asetasid selle temale alla ja ta istus selle peale ning Aaron ja Huur toetasid tema käsi, üks siitpoolt ja teine sealtpoolt; siis ta käed seisid kindlalt kuni päikeseloojakuni.
\par 13 Ja Joosua võitis mõõgateraga Amaleki ning tema rahva.
\par 14 Ja Issand ütles Moosesele: „Kirjuta see meenutuseks raamatusse ja pane Joosuale kõrva taha, et ma pühin Amaleki mälestuse taeva alt sootuks!”
\par 15 Siis Mooses ehitas altari ning pani sellele nimeks „Minu lipp Issand”.
\par 16 Ja ta ütles: „Et käsi on olnud Issanda trooni poole, on Issandal sõda Amaleki vastu põlvest põlve!”

\chapter{18}

\par 1 Kui Midjani preester Jitro, Moosese äi, kuulis kõike seda, mida Jumal Moosesele ja oma Iisraeli rahvale oli teinud, kuidas Issand Iisraeli oli Egiptusest välja viinud,
\par 2 siis Moosese äi Jitro võttis enesega kaasa Sippora, Moosese naise, kelle see oli tagasi saatnud,
\par 3 ja tema kaks poega, kellest ühe nimi oli Geersom, kelle puhul Mooses oli öelnud: „Ma olen võõras võõral maal”,
\par 4 ja teise nimi Elieser, kelle puhul Mooses oli öelnud: „Mu isa Jumal oli mulle abiks ja päästis mind vaarao mõõga eest.”
\par 5 Ja Jitro, Moosese äi, tuli Moosese juurde koos ta poegade ja naisega sinna kõrbesse, kus ta oli leeris Jumala mäe juures.
\par 6 Ja ta käskis Moosesele öelda: „Mina, su äi Jitro, tulen sinu juurde, samuti tuleb su naine ja koos temaga ta kaks poega.”
\par 7 Siis Mooses läks välja oma äiale vastu, kummardas ja andis temale suud; kui nad olid küsinud teineteise käekäigu järele, siis nad läksid telki.
\par 8 Ja Mooses jutustas oma äiale kõigest, mida Issand vaaraole ja egiptlastele Iisraeli pärast oli teinud, ja kõigist vaevadest, mis teekonnal olid olnud, ja kuidas Issand nad oli päästnud.
\par 9 Ja Jitro tundis rõõmu kõigest sellest heast, mida Issand oli Iisraelile teinud, päästes tema egiptlaste käest.
\par 10 Ja Jitro ütles: „Kiidetud olgu Issand, kes teid päästis egiptlaste ja vaarao käest!
\par 11 Nüüd ma tean, et Issand on suurim kõigist jumalaist, sest ta päästis rahva Egiptuse käe alt, kui nad olid ülbed nende vastu.”
\par 12 Ja Jitro, Moosese äi, tõi põletus- ja tapaohvreid Jumalale; ja Aaron ja kõik Iisraeli vanemad tulid, et Jumala ees leiba võtta üheskoos Moosese äiaga.
\par 13 Järgmisel päeval, kui Mooses istus rahvale kohut mõistma ja rahvas seisis Moosese juures hommikust õhtuni,
\par 14 nägi Moosese äi kõike, mida ta rahvaga tegi, ja ütles: „Mis see on, mis sa rahvaga teed? Mispärast sina üksi istud, aga kõik rahvas seisab su juures hommikust õhtuni?”
\par 15 Ja Mooses vastas äiale: „Rahvas tuleb ju minu juurde Jumalalt nõu küsima.
\par 16 Kui neil on mingi asi, siis tullakse minu juurde ja mina mõistan kohut nende vahel ning teen teatavaks Jumala määrused ja tema Seaduse.”
\par 17 Siis Moosese äi ütles temale: „See asi pole hea, mis sa teed.
\par 18 Sa nõrked päris ära, niihästi sina kui ka see rahvas, kes su juures on, sest see tegevus on sulle raske. Sa ei jaksa seda üksinda teha.
\par 19 Kuule nüüd mu häält, ma annan sulle nõu, ja küllap on ka Jumal sinuga: sina ole rahva asemik Jumala juures ja vii asjad Jumala ette.
\par 20 Sina seleta neile määrusi ja Seadust ja anna neile teada tee, mida neil tuleb käia, ja töö, mida nad peavad tegema.
\par 21 Aga sina vali kogu rahvast tublisid mehi, kes kardavad Jumalat, ustavaid mehi, kes vihkavad ahnust, ja pane need neile pealikuiks iga tuhande, saja, viiekümne ja kümne üle.
\par 22 Nemad mõistku rahvale kohut igal ajal! Ja olgu nõnda: nad toogu kõik suured asjad sinu ette ja otsustagu kõik väikesed asjad ise. Tee nõnda oma koorem kergemaks ja nemad kandku seda koos sinuga!
\par 23 Kui sa seda teed ja Jumal sind käsib, siis sa jaksad vastu pidada, samuti läheb ka kogu see rahvas rahul olles koju.”
\par 24 Ja Mooses kuulas oma äia sõna ning tegi kõik, mis see oli öelnud.
\par 25 Ja Mooses valis tublisid mehi kogu Iisraelist ning pani nad juhtideks rahvale, pealikuiks iga tuhande, saja, viiekümne ja kümne üle.
\par 26 Ja need mõistsid rahvale kohut igal ajal; raskemad asjad tõid nad Moosese ette, aga kõik väikesed asjad otsustasid nad ise.
\par 27 Siis Mooses saatis teele oma äia ja see läks oma maale.

\chapter{19}

\par 1 Kolmandal kuul pärast Iisraeli laste lahkumist Egiptusemaalt, just sel ajal, tulid nad Siinai kõrbesse,
\par 2 sest nad olid Refidimist teele läinud, Siinai kõrbesse tulnud ja kõrbes leeri üles löönud; Iisrael oli seal leeris mäe jalamil.
\par 3 Mooses läks üles Jumala juurde ja Issand hüüdis teda mäelt, öeldes: „Ütle nõnda Jaakobi soole ja kuuluta Iisraeli lastele:
\par 4 Te olete näinud, mida ma olen teinud egiptlastele, kuidas ma teid olen kandnud kotka tiibadel ja kuidas ma teid olen toonud enese juurde.
\par 5 Ja kui te nüüd tõesti kuulate minu häält ja peate minu lepingut, siis te olete minu omand kõigi rahvaste hulgast, sest minu päralt on kogu maailm.
\par 6 Te olete mulle preestrite kuningriigiks ja pühaks rahvaks. Need on sõnad, mis sa pead Iisraeli lastele rääkima!”
\par 7 Ja Mooses tuli ning kutsus rahva vanemad ja pani nende ette kõik need sõnad, nagu Issand teda oli käskinud.
\par 8 Ja kogu rahvas vastas üksmeelselt ning ütles: „Me teeme kõik, mis Issand on öelnud.” Ja Mooses viis rahva vastuse Issandale.
\par 9 Ja Issand ütles Moosesele: „Vaata, mina tulen su juurde paksus pilves, et rahvas kuuleks, kui ma sinuga räägin, ja nad usuksid ka sind igavesti.” Ja Mooses kuulutas Issandale rahva vastuse.
\par 10 Siis Issand ütles Moosesele: „Mine rahva juurde, pühitse neid täna ja homme, ja nad pesku oma riided!
\par 11 Ja nad olgu valmis kolmandaks päevaks, sest kolmandal päeval astub Issand kogu rahva silma ees alla Siinai mäele.
\par 12 Aga hoia rahvast igast küljest tagasi, öeldes: Hoiduge mäele üles minemast ja selle jalamit puudutamast! Igaühte, kes mäge puudutab, karistatakse surmaga!
\par 13 Kellegi käsi ei tohi seda puudutada, vaid ta visatagu kividega surnuks või lastagu maha; olgu loom või inimene, ta ei tohi jääda elama! Alles kui pikalt sarve puhutakse, võivad nad minna mäele.”
\par 14 Ja Mooses tuli mäelt alla rahva juurde; ta pühitses rahvast ja nad pesid oma riided.
\par 15 Ja ta ütles rahvale: „Olge valmis kolmandaks päevaks! Ärge minge naise ligi!”
\par 16 Ja kolmandal päeval, kui hommik oli saabunud, sündis, et hakkas müristama ja välku lööma: mäe kohal oli ränk pilv ja kostis väga vali sarvehääl, nõnda et kogu rahvas, kes oli leeris, värises.
\par 17 Siis Mooses viis rahva leerist välja Jumalale vastu; ja nad jäid mäe jalamile.
\par 18 Ja kogu Siinai mägi suitses, kui Issand laskus sinna tule sees; selle suits tõusis üles nagu sulatusahju suits, ja kogu mägi vabises kõvasti.
\par 19 Ja sarvehääl läks üha valjemaks; Mooses rääkis ja Jumal vastas temale valjusti.
\par 20 Ja Issand laskus Siinai mäele, mäetippu; Issand kutsus Moosese mäetippu ja Mooses läks üles.
\par 21 Ja Issand ütles Moosesele: „Mine alla, hoiata rahvast, et nad ei tungiks Issanda juurde teda vaatama, et paljud neist ei langeks!
\par 22 Ja preestridki, kes liginevad Issandale, peavad endid pühitsema, et Issand neid ei kohtleks karmilt!”
\par 23 Ja Mooses ütles Issandale: „Rahvas ei või tõusta Siinai mäele, sest sina oled meid hoiatanud, öeldes: Märgi piir ümber mäe ja kuuluta see pühaks!”
\par 24 Ja Issand ütles temale: „Mine alla ja tule taas üles, sina ja Aaron koos sinuga! Preestrid ja rahvas aga ärgu tungigu üles Issanda juurde, et tema neid ei kohtleks karmilt!”
\par 25 Siis Mooses läks alla rahva juurde ja kõneles nendega.

\chapter{20}

\par 1 Ja Jumal kõneles kõik need sõnad, öeldes:
\par 2 „Mina olen Issand, sinu Jumal, kes sind tõi välja Egiptusemaalt, orjusekojast.
\par 3 Sul ei tohi olla muid jumalaid minu palge kõrval!
\par 4 Sa ei tohi enesele teha kuju ega mingisugust pilti sellest, mis on ülal taevas, ega sellest, mis on all maa peal, ega sellest, mis on maa all vees!
\par 5 Sa ei tohi neid kummardada ega neid teenida, sest mina, Issand, sinu Jumal, olen püha vihaga Jumal, kes vanemate süü nuhtleb laste kätte kolmanda ja neljanda põlveni neile, kes mind vihkavad,
\par 6 aga kes heldust osutab tuhandeile neile, kes mind armastavad ja mu käske peavad!
\par 7 Sa ei tohi Issanda, oma Jumala nime asjata suhu võtta, sest Issand ei jäta seda nuhtlemata, kes tema nime asjata suhu võtab!
\par 8 Pea meeles, et sa pead hingamispäeva pühitsema!
\par 9 Kuus päeva tee tööd ja toimeta kõiki oma talitusi,
\par 10 aga seitsmes päev on Issanda, sinu Jumala hingamispäev. Siis sa ei tohi toimetada ühtegi talitust, ei sa ise ega su poeg ja tütar, ega su sulane ja teenija, ega su veoloom ega võõras, kes su väravais on!
\par 11 Sest kuue päevaga tegi Issand taeva ja maa, mere ja kõik, mis neis on, ja ta hingas seitsmendal päeval: seepärast Issand õnnistas hingamispäeva ja pühitses selle.
\par 12 Sa pead oma isa ja ema austama, et su elupäevi pikendataks sellel maal, mille Issand, su Jumal, sulle annab!
\par 13 Sa ei tohi tappa!
\par 14 Sa ei tohi abielu rikkuda!
\par 15 Sa ei tohi varastada!
\par 16 Sa ei tohi tunnistada oma ligimese vastu valetunnistajana!
\par 17 Sa ei tohi himustada oma ligimese koda! Sa ei tohi himustada oma ligimese naist, sulast ega teenijat, härga ega eeslit ega midagi, mis su ligimese päralt on!”
\par 18 Ja kogu rahvas kuulis ja nägi müristamist, tuleleeke, sarvehäält ja mäe suitsemist. Kui rahvas seda nägi, siis ta vabises ja jäi eemale seisma.
\par 19 Ja nad ütlesid Moosesele: „Räägi sina meiega, siis me kuulame! Ainult ärgu Jumal meiega rääkigu, et me ei sureks!”
\par 20 Aga Mooses vastas rahvale: „Ärge kartke, sest Jumal on tulnud teid katsuma, et teil oleks tema kartus silme ees, selleks et te pattu ei teeks!”
\par 21 Ja rahvas püsis eemal; aga Mooses ligines pimedusele, kus oli Jumal.
\par 22 Ja Issand ütles Moosesele: „Ütle Iisraeli lastele nõnda: Te nägite, et ma taevast teiega rääkisin.
\par 23 Minu kõrvale ei tohi te midagi teha: te ei tohi enestele teha hõbe- ja kuldjumalaid!
\par 24 Tee mulle mullast altar ja ohverda selle peal oma põletus- ja tänuohvreid, oma lambaid, kitsi ja veiseid! Kõigis paigus, kus ma käsin oma nime kuulutada, tulen ma sinu juurde ja õnnistan sind.
\par 25 Aga kui sa teed mulle kivialtari, siis ära ehita seda tahutud kividest, sest peitliga raiudes sa rüvetad selle!
\par 26 Ja ära astu mu altari juurde üles astmeid mööda, et su ihu selle ees ei paljastuks!

\chapter{21}

\par 1 Ja need on seadused, mis sa pead panema nende ette:
\par 2 kui sa ostad sulaseks heebrealase, siis ta teenigu kuus aastat, aga seitsmendal saagu ta vabaks lunamaksuta.
\par 3 Kui ta tuli üksinda, mingu ta ka üksinda; kui ta oli naisemees, mingu naine koos temaga.
\par 4 Kui ta isand andis temale naise ja see sünnitas temale poegi või tütreid, siis naine ja lapsed jäävad isandale, aga tema mingu üksinda.
\par 5 Aga kui sulane ütleb kindla meelega: Ma armastan oma isandat, naist ja lapsi, ma ei taha vabaks saada,
\par 6 siis tema isand viigu ta Jumala ette ja seadku ukse või piitjala juurde: ta isand torgaku tal kõrv naaskliga läbi ja ta teenigu igavesti.
\par 7 Kui keegi müüb oma tütre teenijaks, siis see ei tohi sulaste taoliselt ära minna.
\par 8 Kui ta ei meeldi oma isandale, kes tema on määranud enesele, siis lasku see teda lunastada; aga tal ei ole luba teda müüa võõrale rahvale, kui ta tema hülgab.
\par 9 Aga kui ta määrab tema oma pojale, siis ta peab temale andma tütre õigused.
\par 10 Kui ta võtab enesele veel teise naise, siis ei tohi esimesele vähendada toitu, katet ja abieluõigust.
\par 11 Aga kui ta tema suhtes ei täida neid kolme tingimust, siis ta võib maksuta ära minna, ilma lunarahata.
\par 12 Kes teist inimest lööb, nõnda et see sureb, seda tuleb karistada surmaga!
\par 13 Aga kui ta teda ei ole varitsenud, vaid Jumal on lasknud ta tema kätte sattuda, siis ma määran sulle ühe paiga, kuhu ta võib põgeneda.
\par 14 Aga kui keegi on riivatu oma ligimese vastu, tappes tema kavalusega, siis sa pead tema võtma surmamiseks isegi mu altari eest!
\par 15 Kes lööb oma isa või ema, seda karistatagu surmaga!
\par 16 Kes röövib inimese, kas ta selle müüb või see leitakse tema käest, seda karistatagu surmaga!
\par 17 Kes neab oma isa või ema, seda karistatagu surmaga!
\par 18 Kui mehed riidlevad ja üks lööb teist kivi või tööriistaga, nõnda et teine ei sure, vaid lamab asemel,
\par 19 siis on lööja süüta, kui teine tõuseb üles ja saab õues kepi najal käia; ometi peab ta temale tasuma viidetud aja eest ja hoolitsema, et ta saaks terveks.
\par 20 Kui keegi lööb kepiga oma sulast või teenijat, nõnda et ta tema käe all sureb, siis tuleb teda karistada.
\par 21 Aga kui too jääb veel päevaks või paariks elama, siis ei tule teda karistada, sest too oli ju ostetud tema raha eest.
\par 22 Kui mehed taplevad ja tõukavad lapseootel naist, nõnda et see enneaegselt sünnitab, muud kahju aga ei teki, siis tuleb süüdlast rahaga karistada; nõnda nagu naise mees temalt nõuab, nõnda andku ta seda nurisünnituse pärast.
\par 23 Aga kui tekib kahju, siis tuleb anda hing hinge vastu,
\par 24 silm silma vastu, hammas hamba vastu, käsi käe vastu, jalg jala vastu,
\par 25 põletus põletuse vastu, haav haava vastu, vorp vorbi vastu.
\par 26 Kui keegi lööb oma sulase või teenija silma ja rikub selle, siis ta peab tema silma pärast vabaks laskma.
\par 27 Ja kui ta lööb oma sulasel või teenijal hamba välja, siis ta peab tema hamba pärast vabaks laskma.
\par 28 Kui härg kaevleb surnuks mehe või naise, siis tuleb härg kividega surnuks visata ja tema liha ei tohi süüa, härja omanik aga on süüta.
\par 29 Aga kui seesama härg oli varem kaevleja ja selle omanikku hoiatati juba ammu, ent tema ei takista seda ja härg tapab mehe või naise, siis visatagu härg kividega surnuks ja surmatagu ka ta omanik!
\par 30 Kui temale aga määratakse lunastushind, siis ta peab andma niipalju hinge lunaraha, kui temale määratakse.
\par 31 Kui see härg kaevleb poega või tütart, siis tuleb temaga talitada sellesama seaduse järgi.
\par 32 Kui härg kaevleb sulast või teenijat, siis tuleb nende isandale anda kolmkümmend hõbeseeklit, härg aga visatagu kividega surnuks!
\par 33 Kui keegi jätab kaevu lahti või kaevab kaevu, aga ei kata seda, mille tõttu langeb sinna härg või eesel,
\par 34 siis peab kaevu omanik andma hüvituse; ta tasugu looma omanikule rahas, aga korjus jäägu temale.
\par 35 Kui kellegi härg kaevleb surnuks teise mehe härja, siis tuleb elus härg müüa ja raha pooleks jagada; ka surnud härg tuleb poolitada.
\par 36 Aga kui on teada, et see härg oli juba varem kaevleja ja selle omanik ei ole seda takistanud, siis ta peab selle täiesti asendama: härg härja vastu; aga korjus jäägu temale.

\chapter{22}

\par 1 Kui varas tabatakse sissemurdmiselt ja lüüakse surnuks, siis tapjal ei ole veresüüd.
\par 2 Aga kui päike oli juba tõusnud, siis on tapjal veresüü. Varas peab tasuma kõik; kui tal midagi ei ole, võib teda ennast müüa varguse pärast.
\par 3 Kui varastatu, olgu härg või eesel või lammas leitakse tema käest elusana, siis ta peab tasuma kahekordselt.
\par 4 Kui keegi söödab ära põllu või viinamäe, laseb karja lahti ja söödab teise põllul, siis ta peab asemele andma parimat oma põllult ja viinamäelt.
\par 5 Kui tuli pääseb lahti ja hakkab kibuvitste külge ning põletab ära nabrad või lõikamata vilja või põllu, siis süütaja peab põlenu eest täielikult tasuma.
\par 6 Kui keegi teisele annab hoiule raha või riistu ja need varastatakse selle kojast, siis peab varas, kui ta leitakse, tasuma kahekordselt.
\par 7 Aga kui varast ei tabata, viidagu koja omanik Jumala ette: kas ta ise ei ole pannud oma kätt ligimese vara külge?
\par 8 Kõigis süütegudes härja, eesli, lamba, riiete ja kõigi kadunud asjade pärast, mida keegi ütleb enese oma olevat, tuleb mõlema asi viia Jumala ette: kelle Jumal süüdi mõistab, see peab teisele tasuma kahekordselt.
\par 9 Kui keegi teisele annab hoiule eesli või härja või lamba või mõne muu looma ja see sureb või saab vigastada või viiakse ära kellegi nägemata,
\par 10 siis peab nende mõlema vahel toimuma vanne Issanda ees: kas ta pole pannud oma kätt teise omandi külge? Omanik peab vandega leppima ja teine ärgu tasugu.
\par 11 Kui see aga tõepoolest on temalt varastatud, siis ta peab omanikule tasuma.
\par 12 Kui see aga tõepoolest on maha murtud, siis ta toogu see tõendiks; mahamurtut ta ei tarvitse tasuda.
\par 13 Kui keegi teiselt laenab looma ja see saab vigastada või sureb, ilma et omanik juures oleks, siis ta peab selle eest temale tasuma.
\par 14 Aga kui omanik juures oli, siis tal ei ole vaja tasuda; kui see oli renditud, siis asendab selle rendihind.
\par 15 Kui keegi võrgutab neitsi, kes ei ole kihlatud, ja magab temaga, siis ta peab mõrsjahinna eest tema enesele naiseks võtma.
\par 16 Aga kui selle isa keeldub teda temale andmast, siis mees vaagigu raha, nagu on mõrsjahind neitsi eest.
\par 17 Sa ei tohi nõida ellu jätta!
\par 18 Igaühte, kes ühtib loomaga, tuleb karistada surmaga!
\par 19 Kes ohverdab jumalatele ja mitte üksnes Issandale, see tuleb hukata!
\par 20 Ära rõhu võõrast ja ära tee temale häda, sest te ise olete olnud võõrad Egiptusemaal!
\par 21 Te ei tohi halvasti kohelda ühtegi lesknaist ega vaeslast!
\par 22 Kui sa neid halvasti kohtled, nii et nad mind appi hüüavad, siis ma kuulen tõesti nende hüüdu,
\par 23 mu viha süttib põlema ja ma tapan teid mõõgaga, nõnda et teie naised jäävad leskedeks ja lapsed orbudeks.
\par 24 Kui sa laenad raha minu rahvale, mõnele, kes on vaene, siis ära ole tema suhtes liigkasuvõtja: te ei tohi temalt kasu võtta!
\par 25 Kui sa oled võtnud pandiks oma ligimese üleriide, siis sa pead selle temale tagasi andma, enne kui päike loojub,
\par 26 sest see ta üleriie on tema ihu ainsaks katteks. Millega ta muidu magaks? Ma kuulen, kui ta mind appi hüüab, sest mina olen halastaja.
\par 27 Sa ei tohi needa Jumalat ja sajatada oma rahva vürsti!
\par 28 Oma külluse ja ülevooluga ära viivita! Sa pead esmasündinu oma poegadest minule andma!
\par 29 Nõndasamuti tee oma veiste, lammaste ja kitsede esmasündinutega: seitse päeva olgu ta oma ema juures, kaheksandal päeval pead ta minule andma.
\par 30 Ja te peate olema minule pühitsetud inimesed: väljal mahamurtu liha te ei tohi süüa, visake see koertele!

\chapter{23}

\par 1 Ära levita valekuuldusi! Ära löö kätt õelaga, et sa ei saaks valetunnistajaks!
\par 2 Ära järgne jõugule kurja tegema ja ära kosta riiuasjas jõugu järele paindudes, et sa õigust ei väänaks!
\par 3 Ära ole viletsa kasuks erapoolik tema riiuasjas!
\par 4 Kui sa kohtad oma vihamehe eksinud härga või eeslit, siis vii see temale tagasi!
\par 5 Kui sa näed oma vihamehe eeslit koorma all lamamas, siis ära jäta teda maha, vaid aita ta lahti päästa!
\par 6 Ära vääna su juures oleva vaese õigust tema riiuasjas!
\par 7 Pettusasjast hoia eemale; ja sa ei tohi tappa süütut ega õiget, sest mina ei mõista õigeks ühtki õelat!
\par 8 Ära võta meelehead, sest meelehea pimestab nägijaid ja teeb õigete asjad segaseks!
\par 9 Võõrale ära tee häda: te ju mõistate võõra hinge, sest te ise olete olnud võõrad Egiptusemaal!
\par 10 Kuus aastat külva oma maale ja kogu sellelt saaki,
\par 11 aga seitsmendal jäta see koristamata ja harimata, et su rahva vaesed saaksid sealt süüa; ja mis neist järele jääb, seda söögu metsloomad! Samuti talita ka oma viinamäe ja õlipuudega!
\par 12 Kuus päeva tee oma tööd, aga seitsmendal päeval puhka, et su härg ja eesel kosuksid, ja su teenija poeg ning võõras saaksid hingata!
\par 13 Kõik, mis ma teile olen rääkinud, pidage meeles! Teiste jumalate nimesid ei tohi te nimetada, ärgu seda kuuldagu sinu suust!
\par 14 Kolm korda aastas pead mulle püha pidama!
\par 15 Pea hapnemata leibade püha: seitse päeva söö hapnemata leiba, nõnda nagu ma sind olen käskinud, määratud ajal aabibikuus, sest siis sa tulid Egiptusest välja; aga tühje käsi ei tohi tulla mu palge ette!
\par 16 Ja pea lõikuspüha oma varajase vilja saagile, mille oled külvanud põllule; ja vilja kokkupanemise püha aasta lõpus, kui sa oma saagi põllult koristad!
\par 17 Kolm korda aastas peab kogu su meessugu tulema Issanda Jumala palge ette!
\par 18 Sa ei tohi ohverdada mu tapaohvri verd haput leiba süües, ja mu pühade ohvrirasv ei tohi seista üle öö hommikuni!
\par 19 Parim oma põllu uudseviljast vii Issanda, oma Jumala kotta! Ära keeda kitsetalle ta ema piimas!
\par 20 Vaata, mina läkitan sinu eel ingli sind tee peal hoidma ja sind viima paika, mille ma olen valmistanud.
\par 21 Ole valvel tema ees ja kuula ta häält, ära vihasta teda; sest ta ei andesta teie üleastumist, sellepärast et temas on minu nimi!
\par 22 Aga kui sa tõesti kuulad tema häält ja teed kõike, mida ma käsin, siis ma olen sinu vaenlaste vaenlane ja vastaste vastane.
\par 23 Sest minu ingel läheb sinu eel ja viib sind emorlaste, hettide, perislaste, kaananlaste, hiivlaste ja jebuuslaste juurde ja mina hävitan need.
\par 24 Ära kummarda nende jumalaid ja ära teeni neid! Ja ära tee ka nende rahvaste tegude järgi, vaid kisu maha sootuks nende jumalad ja purusta täiesti nende ebaususambad!
\par 25 Te peate aga teenima Issandat, oma Jumalat, siis ta õnnistab sinu leiba ja vett! Ja ma hoian tõve sinust eemal.
\par 26 Enneaegselt sünnitajat ega sigimatut ei ole su maal. Ja ma teen täielikuks sinu päevade arvu.
\par 27 Ma läkitan sinu eele hirmu ja viin segadusse kõik rahvad, kelle juurde sa tuled, ja teen, et kõik su vaenlased näitavad sulle selga.
\par 28 Ma läkitan sinu eele masenduse, et see ajaks su eest ära hiivlased, kaananlased ja hetid.
\par 29 Ma ei aja neid su eest ära ühe aastaga, et maa ei jääks tühjaks ja sulle ei sigineks metsloomi.
\par 30 Ma ajan nad sinu eest ära vähehaaval, seni kui sa sigined ja pärid maa.
\par 31 Ja ma annan sulle maa-ala Kõrkjamerest vilistite mereni ja kõrbest kuni Frati jõeni, sest ma annan maa elanikud teie kätte ja sina ajad need ära enese eest.
\par 32 Sa ei tohi teha lepingut nendega ega nende jumalatega!
\par 33 Nemad ei tohi jääda elama su maale, et nad ei mõjustaks sind minu vastu pattu tegema; sest kui sa nende jumalaid teenid, siis saab see sulle püüdepaelaks!”

\chapter{24}

\par 1 Ja ta ütles Moosesele: „Astuge üles Issanda juurde, sina ise, Aaron, Naadab, Abihu ja seitsekümmend Iisraeli vanemaist, ja kummardage eemalt!
\par 2 Mooses üksi liginegu Issandale, teised aga ei tohi lähedale tulla ja rahvas ei tohi minna üles koos temaga!”
\par 3 Ja Mooses tuli ning jutustas rahvale kõigist Issanda sõnadest ja kõigist seadlustest. Ja kogu rahvas vastas ühehäälselt, öeldes: „Me tahame teha iga sõna järgi, mis Issand on rääkinud!”
\par 4 Ja Mooses kirjutas üles kõik Issanda sõnad. Ja ta tõusis hommikul vara ning püstitas mäe alla altari ja kaksteist sammast vastavalt Iisraeli kaheteistkümnele suguharule.
\par 5 Ja ta läkitas Iisraeli laste noori mehi ohverdama põletusohvreid ja tapma härjavärsse tänuohvriks Issandale.
\par 6 Ja Mooses võttis poole verest ning pani kausikestesse, aga teise poole verest ta piserdas altarile.
\par 7 Siis ta võttis seaduseraamatu ja luges rahva kuuldes. Ja nad ütlesid: „Kõik, mida Issand on käskinud, me teeme ja võtame kuulda!”
\par 8 Ja Mooses võttis verd ning piserdas rahva peale, öeldes: „Vaata, see on selle lepingu veri, mille Issand teiega on teinud kõigi nende sõnade põhjal!”
\par 9 Seejärel Mooses, Aaron, Naadab, Abihu ja seitsekümmend Iisraeli vanemaist läksid üles,
\par 10 ja nad nägid Iisraeli Jumalat: tema jalge all oli nagu safiiridest tehtud põrand - otsekui taevas ise oma selguses.
\par 11 Aga ta ei tõstnud oma kätt Iisraeli laste ülemate vastu, vaid need võisid Jumalat vaadata ning süüa ja juua.
\par 12 Ja Issand ütles Moosesele: „Tule üles minu juurde mäele ja ole siin, siis ma annan sulle kivilauad ning Seaduse ja käsud, mis ma olen kirjutanud neile õpetuseks!”
\par 13 Ja Mooses läks teele koos oma teenri Joosuaga; ja Mooses läks üles Jumala mäele.
\par 14 Aga vanemaile ta ütles: „Jääge teie siia, kuni me tuleme tagasi teie juurde! Ja vaata, Aaron ja Huur on teie juures. Kellel on mingi põhjus, mingu nende juurde!”
\par 15 Ja Mooses läks mäele ning pilv kattis mäe.
\par 16 Ja Issanda auhiilgus laskus Siinai mäele ja pilv kattis seda kuus päeva; seitsmendal päeval hüüdis ta pilve seest Moosest.
\par 17 Ja Issanda auhiilguse ilmutus oli Iisraeli laste silmis otsekui hävitav tuli mäetipus.
\par 18 Mooses läks pilve keskele ning tõusis üles mäele. Ja Mooses oli mäe peal nelikümmend päeva ja nelikümmend ööd.

\chapter{25}

\par 1 Ja Issand rääkis Moosesega, öeldes:
\par 2 „Ütle Iisraeli lastele, et nad tooksid mulle tõstelõivu; igaühelt, kes annab heast südamest, võtke minu tõstelõivu.
\par 3 Ja see on tõstelõiv, mida te peate neilt võtma: kulda, hõbedat ja vaske,
\par 4 sinist, purpurpunast ja helepunast lõnga ning peent linast lõime, kitsekarvu,
\par 5 punaseid jääranahku, merilehmanahku, akaatsiapuud,
\par 6 valgustusõli, palsameid võideõliks ja healõhnalisi suitsutusrohte,
\par 7 karneoolikive ja ilustuskive õlarüü ja rinnakilbi jaoks.
\par 8 Ja nad tehku mulle pühamu, siis ma asun elama nende keskele.
\par 9 Tehke täpselt eeskuju järgi, mida ma sulle näitan, elamu ja kõigi selle riistade mudeli järgi!
\par 10 Laegas tehtagu akaatsiapuust, kaks ja pool küünart pikk, poolteist küünart lai ja poolteist küünart kõrge.
\par 11 Kata see puhta kullaga, kata seest- ja väljastpoolt ja tee sellele kuldpärg ümber!
\par 12 Vala selle jaoks neli kuldrõngast ja kinnita need ta nelja jala külge, kaks rõngast ühte külge ja kaks rõngast teise külge!
\par 13 Tee kandekangid akaatsiapuust ja karda need kullaga!
\par 14 Pista kangid rõngastesse laeka külgedel, et nendega saaks laegast kanda!
\par 15 Kangid jäägu laeka rõngastesse, neid ärgu sealt välja võetagu!
\par 16 Pane laekasse tunnistus, mille ma sulle annan!
\par 17 Tee puhtast kullast lepituskaas, kaks ja pool küünart pikk ja poolteist küünart lai!
\par 18 Tee kullast kaks keerubit; tee need sepisena lepituskaane kumbagi otsa!
\par 19 Pane üks keerub ühte otsa ja teine keerub teise otsa; pange keerubid lepituskaane kumbagi otsa!
\par 20 Keerubid sirutagu oma tiivad ülespoole, et nad tiibadega kataksid lepituskaant, ja nende palged olgu vastamisi; keerubite palged olgu lepituskaane poole!
\par 21 Aseta lepituskaas laekale peale ja pane laekasse tunnistus, mille ma sulle annan!
\par 22 Seal ma siis ilmutan ennast sulle ja kõnelen sinuga lepituskaanel mõlema keerubi vahel, mis on tunnistuslaeka peal, kõigest, mida ma sind käsin Iisraeli lastele öelda.
\par 23 Tee akaatsiapuust laud, kaks küünart pikk, küünar lai ja poolteist küünart kõrge!
\par 24 Karda see puhta kullaga ja tee sellele kuldpärg ümber!
\par 25 Tee sellele kämblalaiune põõn ümber ja tee kuldpärg ümber selle põõna!
\par 26 Tee sellele neli kuldrõngast ja kinnita rõngad nelja nurga külge selle nelja jala juures!
\par 27 Rõngad olgu otse põõna kõrval kangide asemeiks laua kandmisel.
\par 28 Tee laua kandmiseks akaatsiapuust kangid ja karda need kullaga!
\par 29 Tee sellele vaagnad ja kausid, kannud ja peekrid joogiohvri toomiseks; tee need puhtast kullast!
\par 30 Ja pane alati ohvrileibu lauale mu palge ette!
\par 31 Tee puhtast kullast lambijalg; see lambijalg olgu sepistatud töö aluse ja harudega; selle karikakesed, nupud ja õiekesed olgu sellega ühest tükist!
\par 32 Selle küljest lähtugu kuus haru: ühest küljest kolm lambijala haru ja teisest küljest kolm lambijala haru!
\par 33 Ühel harul olgu kolm mandliõiekujulist karikakest nupu ja õiekesega, samuti olgu teisel harul kolm mandliõiekujulist karikakest nupu ja õiekesega; nõnda olgu neil kuuel harul, mis lambijalast lähtuvad.
\par 34 Aga lambijalal enesel olgu neli mandliõiekujulist karikakest nupu ja õiekesega:
\par 35 iga harupaari all olgu üks nupp neil kuuel harul, mis lambijalast lähtuvad.
\par 36 Nupud ja harud olgu sellega ühest tükist terviklik sepisetöö puhtast kullast.
\par 37 Tee sellele seitse lampi; lambid seatagu üles nõnda, et need valgustaksid oma esist!
\par 38 Tahikäärid ja tahikarbid olgu puhtast kullast!
\par 39 See tehtagu ühest talendist puhtast kullast koos kõigi nende riistadega!
\par 40 Ja vaata, et sa need teed nende eeskujude järgi, mis sulle mäel näidati!

\chapter{26}

\par 1 Tee elamu kümnest vaibast, mis on korrutatud linasest lõimest, sinisest, purpurpunasest ja helepunasest lõngast; tee need kunstipäraselt sissekootud keerubitega!
\par 2 Iga vaiba pikkus olgu kakskümmend kaheksa küünart ja iga vaiba laius neli küünart; kõigil vaipadel olgu samad mõõdud!
\par 3 Viis vaipa seotagu üksteisega kokku, samuti seotagu ka teised Viis vaipa üksteisega kokku!
\par 4 Tee sinised aasad esimese vaiba servale, äärmisele esimeses reastuses; samasugused tee ka teise reastuse äärmise vaiba servale!
\par 5 Tee viiskümmend aasa esimesele vaibale ja tee viiskümmend aasa vaiba servale, mis on teises reastuses; aasad olgu üksteisega kohakuti!
\par 6 Tee ka viiskümmend kuldhaaki ja ühenda nende haakidega vaibad üksteise külge, et neist saaks täielik elamu!
\par 7 Ja tee kitsekarvadest vaibad telgiks elamu peale; neid vaipu tee üksteist!
\par 8 Iga vaiba pikkus olgu kolmkümmend küünart ja iga vaiba laius neli küünart; üheteistkümnel vaibal olgu samad mõõdud!
\par 9 Seo kokku viis vaipa eraldi ja kuus vaipa eraldi, aga kuues vaip pane kahekordselt telgi suule!
\par 10 Tee viiskümmend aasa esimese vaiba servale, äärmisele reastuses, ja viiskümmend aasa vaiba servale teises reastuses!
\par 11 Tee viiskümmend vaskhaaki, pista haagid aasadesse ja ühenda telk üheks tervikuks!
\par 12 Telgi vaipade liigse ülerippuvuse korral rippugu pool liigsest vaibast elamu tagaküljes!
\par 13 Küünar siitpoolt ja küünar sealtpoolt telgi vaipade liigsest pikkusest rippugu üle elamu külgede, kattes seda siit- ja sealtpoolt!
\par 14 Tee telgile kate punakaist jääranahkadest ja selle peale veel teine kate merilehmanahkadest!
\par 15 Tee elamule akaatsiapuust püstipandavad lauad!
\par 16 Iga laua pikkus olgu kümme küünart ja iga laua laius poolteist küünart!
\par 17 Igal laual olgu kaks tappi, mis üksteisega on ühendatud; nõnda tee kõik elamu lauad!
\par 18 Tee elamule laudu: kakskümmend lauda lõunapoolse külje jaoks lõuna poole!
\par 19 Tee kahekümnele lauale alla nelikümmend hõbejalga: kaks jalga iga laua alla mõlema tapi jaoks!
\par 20 Ja elamu teise külje jaoks, põhja poole, kakskümmend lauda
\par 21 ja nende nelikümmend hõbejalga: kaks jalga iga laua all.
\par 22 Aga elamu läänepoolse tagakülje jaoks tee kuus lauda!
\par 23 Ja tee kaks lauda elamu tagakülje nurkade jaoks!
\par 24 Need moodustagu kaksiklauad, mis on ühendatud alt üles kuni esimese rõngani; nõnda olgu see nende mõlemaga, neist saagu mõlemad nurgad!
\par 25 Neid on siis kaheksa lauda ja nende hõbejalgu on kuusteist jalga: kaks jalga iga laua all.
\par 26 Tee akaatsiapuust põiklatid: viis latti elamu ühe külje laudade jaoks,
\par 27 viis latti elamu teise külje laudade jaoks ja viis latti elamu küljelaudade jaoks läänepoolses tagaküljes!
\par 28 Keskmine põiklatt keset laudu kulgegu servast servani!
\par 29 Karda lauad kullaga ja tee neile kullast rõngad lattide asemeiks; karda ka latid kullaga!
\par 30 Siis püstita elamu plaani järgi, mis sulle mäel näidati!
\par 31 Ja tee eesriie sinisest, purpurpunasest ja helepunasest lõngast ning korrutatud linasest lõimest, sisse kududes keerubite kujud!
\par 32 Riputa see kullaga karratud nelja akaatsiapuust samba külge, millel on kuldhaagid ja mis seisavad neljal hõbejalal!
\par 33 Riputa eesriie haakide külge ja vii sinna eesriide taha tunnistuslaegas! Ja eesriie eraldagu teile püha ja kõige pühamat paika!
\par 34 Pane lepituskaas tunnistuslaekale peale kõige pühamas paigas!
\par 35 Aseta laud eesriide ette ja lauaga kohakuti lambijalg elamu lõunapoolsesse külge; laud aga pane põhjapoolsesse külge!
\par 36 Tee kate ka telgi uksele sinisest, purpurpunasest ja helepunasest lõngast ning korrutatud linasest lõimest kunstipäraselt kootuna!
\par 37 Katte jaoks tee viis akaatsiapuust sammast ja karda need kullaga; nende haagid olgu kullast; ja vala neile viis vaskjalga!

\chapter{27}

\par 1 Tee akaatsiapuust altar, viis küünart pikk ja viis küünart lai; altar olgu neljanurgeline ja kolm küünart kõrge!
\par 2 Tee selle neljale nurgale sarved; sarved olgu sellega ühest tükist ja karda see vasega!
\par 3 Tee selle juurde kuuluvad nõud tuha koristamiseks, labidad, piserdusnõud, hargid ja sütepannid; kõik selle riistad tee vasest!
\par 4 Tee sellele võrestik, võrgukujuline töö vasest; tee võrgule neli vaskrõngast nelja nurga külge!
\par 5 Pane see allapoole altari äärt, et võrk ulatuks altpoolt vaadates poole altarini!
\par 6 Tee altarile kandekangid, akaatsiapuust kangid, ja karda need vasega!
\par 7 Kangid pistetagu rõngastesse; kangid olgu kummalgi pool altarit, kui seda kantakse!
\par 8 Tee see laudadest õõnsakujulisena; nõnda nagu sulle mäel näidati, nõnda tehtagu see!
\par 9 Tee elamule õu: lõunakaares olgu õuel eesriided korrutatud linasest lõimest, saja küünra pikkuses ühe külje jaoks;
\par 10 ja selle kakskümmend sammast ja kakskümmend vaskjalga; sammaste haagid ja põrgad olgu hõbedast!
\par 11 Nõndasamuti olgu ka põhjapoolses pikemas küljes eesriideid saja küünra pikkuses, ja nende tarvis kakskümmend sammast ja kakskümmend vaskjalga; sammaste haagid ja põrgad olgu hõbedast!
\par 12 Ja vastavalt õue laiusele, läänepoolses küljes, olgu viiskümmend küünart eesriideid, nende kümme sammast ja kümme jalga!
\par 13 Õue laius esiküljes, ida pool, olgu viiskümmend küünart!
\par 14 Seal olgu ühel pool viisteist küünart eesriideid, nende kolm sammast ja kolm jalga,
\par 15 samuti olgu teisel pool viisteist küünart eesriideid, nende kolm sammast ja kolm jalga!
\par 16 Õueväraval olgu kahekümneküünrane kate sinisest, purpurpunasest ja helepunasest lõngast ning korrutatud linasest lõimest kunstipäraselt kootud; sellel olgu neli sammast ja neli jalga!
\par 17 Kõigil sambail ümber õue olgu hõbedast põrgad, samuti hõbedast haagid, aga vasest jalad!
\par 18 Õue pikkus olgu sada küünart ja laius mõlemalt poolt viiskümmend küünart; eesriie olgu viis küünart kõrge, korrutatud linasest lõimest; ja jalad olgu vasest!
\par 19 Kõik elamu riistad kõigeks selle teenistuseks, kõik vaiad ja kõik õue vaiad olgu vasest!
\par 20 Ja sina käsi Iisraeli lapsi, et nad tooksid sulle valgustuse jaoks puhast tambitud oliiviõli, lampide alaliseks ülesseadmiseks.
\par 21 Kogudusetelgis, väljaspool eesriiet, mis on tunnistuslaeka ees, peab Aaron oma poegadega seda korraldama Issanda ees õhtust hommikuni. See olgu igaveseks seadluseks Iisraeli laste tulevastele põlvedele!

\chapter{28}

\par 1 Sina aga lase enese ette astuda Iisraeli laste seast oma vend Aaron ja tema pojad, et nad oleksid mulle preestriteks: Aaron, Naadab, Abihu, Eleasar ja Iitamar, Aaroni pojad.
\par 2 Ja tee oma vennale Aaronile pühad riided auks ning iluks!
\par 3 Räägi kõigi südamest tarkadega, keda ma olen täitnud tarkuse vaimuga, et nad teeksid Aaronile riided, et teda saaks pühitseda mulle preestriks!
\par 4 Ja need on riided, mis nad peavad tegema: rinnakilp, õlarüü, ülekuub, kirjatud särk, peakate ja vöö; nad tehku pühad riided su vennale Aaronile ja tema poegadele, et nad saaksid olla mulle preestriteks!
\par 5 Nad võtku kulda ja sinist, purpurpunast ja helepunast lõnga ning linast lõime
\par 6 ja tehku õlarüü kuldsest, sinisest, purpurpunasest ja helepunasest lõngast ning korrutatud linasest lõimest kunstipäraselt kootuna!
\par 7 Sellel olgu kaks ühendatud õlatükki, mõlemast otsast seotud!
\par 8 Ja kunstipärane kinnitusvöö selle küljes olgu samasugune töö ning sellega ühest tükist: kuldsest, sinisest, purpurpunasest ja helepunasest lõngast ning korrutatud linasest lõimest.
\par 9 Võta siis kaks karneoolikivi ja uurenda neisse Iisraeli poegade nimed:
\par 10 kuus nende nimedest ühte kivisse ja kuus ülejäänud nime teise kivisse nende sünnijärgluses.
\par 11 Otsekui kivinikerdaja uurendab pitsatit, nõnda uurenda Iisraeli poegade nimed neisse mõlemasse kivisse; valmista need ümbritseva kuldäärisega!
\par 12 Aseta need kaks kivi õlarüü õlatükkidele kui mälestuskivid Iisraeli poegadele; ja Aaron kandku kummalgi õlal nende nimesid mälestuseks Issanda ees!
\par 13 Tee kullast ääris
\par 14 ja puhtast kullast kaks ketti; tee need punutuina, nööritaoliselt valmistatuina, ja kinnita need punutud ketid äärise külge!
\par 15 Ja tee kohtu-rinnakilp kunstipäraselt kootuna; tee see nõnda, nagu on tehtud õlarüügi: tee see kuldsest, sinisest, purpurpunasest ja helepunasest lõngast ning korrutatud linasest lõimest!
\par 16 See olgu neljanurgeline ja kahekordne, vaksapikkune ja vaksalaiune.
\par 17 Kata see kalliskivipealistusega, neli rida kive: rubiin, topaas, smaragd ridamisi esimeses reas;
\par 18 teises reas: türkiis, safiir, jaspis;
\par 19 kolmandas reas: hüatsint, ahhaat, ametüst;
\par 20 neljandas reas: krüsoliit, karneool, nefriit; kuldäärisest ümbritsetuina olgu need oma asemeis!
\par 21 Kivid olgu vastavalt Iisraeli poegade nimedele nende kaheteistkümne nimega, igal pitsatitaoliselt uurendatud nimi vastavalt kaheteistkümnele suguharule!
\par 22 Tee rinnakilbile puhtast kullast ketid, nööritaoliselt keerutatud!
\par 23 Tee rinnakilbile kaks kuldrõngast ja pane need kaks rõngast rinnakilbi kahe nurga külge!
\par 24 Ja pane need kaks kuldnööri mõlemasse rõngasse rinnakilbi nurkadel!
\par 25 Mõlema nööri mõlemad otsad kinnita mõlema äärise külge ja kinnita need õlatükkidele õlarüü esiküljes!
\par 26 Tee kaks kuldrõngast ja pane need rinnakilbi kahte nurka, selle ääre külge, mis on seespool vastu õlarüüd!
\par 27 Ja tee veel kaks kuldrõngast ja pane need õlarüü mõlema õlatüki külge, selle esikülje allosasse, ühenduskohale ülespoole õlarüü vööd!
\par 28 Rõngastega rinnakilp seotagu sinise nööriga õlarüü rõngaste külge, et see oleks ülalpool õlarüü vööd ja et rinnakilp ei tuleks lahti õlarüü küljest!
\par 29 Nõnda kandku Aaron, kui ta läheb pühamusse, Iisraeli poegade nimesid kohtu-rinnakilbis oma südame peal alaliseks mälestuseks Issanda ees!
\par 30 Pane kohtu-rinnakilpi ka uurim ja tummim, need olgu Aaroni südame peal, kui ta astub Issanda palge ette! Nõnda kandku Aaron alati Issanda ees Iisraeli laste õigusemõistu oma südame peal!
\par 31 Tee õlarüü ülekuub üleni sinisest lõngast
\par 32 ja sellel olgu keskel avaus pea jaoks; avause ümber olgu kootud äär, see olgu nagu raudrüü avaus, et see ei rebeneks.
\par 33 Tee palistuse külge granaatõunad sinisest, purpurpunasest ja helepunasest lõngast ümber palistuse ja kuldkellukesed ümberringi nende vahele:
\par 34 kuldkelluke ja granaatõun, kuldkelluke ja granaatõun ümber ülekuue palistuse!
\par 35 See olgu Aaronil teenides seljas ja selle helinat olgu kuulda, kui ta läheb pühamusse Issanda ette või tuleb sealt välja, et ta ei sureks!
\par 36 Tee puhtast kullast laubaehe ja uurenda sellesse nagu pitsatisse uurendatakse: „Issandale pühitsetud!”
\par 37 Kinnita see sinise nööriga peakatte külge; see olgu peakatte esiküljes!
\par 38 See olgu Aaroni laubal, et Aaron kannaks süüd pühade andide puhul, mida Iisraeli lapsed pühitsevad, kõigi nende pühade ohvriandide puhul; see olgu alati ta laubal, et neid teha armsaiks Issanda ees!
\par 39 Koo linasest lõimest särk, samuti tee linasest riidest peakate; ja tee vöö kirjatud tegumoes!
\par 40 Tee Aaroni poegadele särgid ja tee neile vööd; ja tee neile peakatted auks ning iluks!
\par 41 Pane need selga oma vennale Aaronile ja tema poegadele; ja võia neid, täida nende käed ja pühitse nad mulle preestriteks!
\par 42 Tee neile linased püksid palja ihu katteks; need ulatugu puusadest alla reiteni;
\par 43 need olgu jalas Aaronil ja tema poegadel, kui nad lähevad kogudusetelki või astuvad pühamus teenides altari ette, et nad ei saaks süüdlasteks ega sureks. See olgu igaveseks seadluseks temale ja ta soole pärast teda!

\chapter{29}

\par 1 Ja see on kirjeldus, mida sa nendega pead tegema, pühitsedes neid mulle preestriteks: võta üks noor härjavärss ja kaks veatut jäära,
\par 2 hapnemata leiba, õliga segatud hapnemata kakke ja õliga võitud õhukesi hapnemata kooke; tee need peenest nisujahust!
\par 3 Pane need ühte korvi ja too need korviga kaasa, siis kui tood härjavärsi ja kaks jäära!
\par 4 Too Aaron ja tema pojad kogudusetelgi ukse juurde ja pese neid veega!
\par 5 Võta riided ja pane Aaronile selga: särk ja õlarüü ülekuub, õlarüü ja rinnakilp, ja seo temale ümber õlarüü vöö!
\par 6 Pane temale peakate pähe ja kinnita püha laubaehe peakatte külge!
\par 7 Võta võideõli ja vala temale pähe ja võia teda!
\par 8 Too esile ta pojad ja pane neile särgid selga!
\par 9 Vööta nad vööga, Aaron ja tema pojad, ning seo nende peakatted! Preestriamet kuulugu neile igavese seaduse järgi. Ja täida Aaroni ja tema poegade käed!
\par 10 Too härjavärss kogudusetelgi ette ning Aaron ja tema pojad pangu oma käed härjavärsi pea peale!
\par 11 Tapa härjavärss Issanda ees kogudusetelgi ukse juures!
\par 12 Võta härjavärsi verd ja määri sõrmega altari sarvedele; kogu veri aga vala altari jalale!
\par 13 Ja võta kõik sisikonda kattev rasv, maksarasv, mõlemad neerud ja rasv, mis nende küljes on, ja süüta altaril põlema!
\par 14 Aga härjavärsi liha, nahk ja sisikond põleta tulega väljaspool leeri; see on patuohver!
\par 15 Võta üks jääradest ning Aaron ja tema pojad pangu oma käed jäära pea peale!
\par 16 Tapa see jäär ja võta ta verd ning piserda altarile ümberringi!
\par 17 Raiu jäär tükkideks, pese ta sisikond ja sääred ning pane need ta tükkide ja pea peale!
\par 18 Siis süüta kogu see jäär altaril põlema; see on põletusohver Issandale, see on healõhnaline tuleohver Issandale!
\par 19 Seejärel võta teine jäär ning Aaron ja tema pojad pangu oma käed jäära pea peale!
\par 20 Tapa see jäär ja võta ta verd ning määri Aaroni parema kõrva lestale ja tema poegade parema kõrva lestale, ja nende parema käe pöidlale ja parema jala suurele varbale; muu veri aga piserda altarile ümberringi!
\par 21 Võta altaril olevat verd ja võideõli ja piserda Aaroni ja tema riiete peale, nõndasamuti tema poegade ja poegade riiete peale; siis saavad pühaks tema ja ta riided, ja samuti tema pojad ja poegade riided!
\par 22 Võta siis jäära rasv ja rasvane saba, sisikonna võrkkile rasv ja maksarasv, mõlemad neerud ja rasv, mis nende küljes on, ja parempoolne saps, sest see on pühitsusjäär,
\par 23 üks päts leiba, üks õliga segatud leivakakk ja üks õhuke koogike hapnemata leibade korvist, mis on Issanda ees,
\par 24 ja pane need kõik Aaroni käte peale ja tema poegade käte peale ning kõiguta neid kõigutusohvrina Issanda ees!
\par 25 Siis võta need nende käte pealt ja põleta altaril põletusohvri peal meeldivaks lõhnaks Issanda ees; see on tuleohver Issandale!
\par 26 Võta rinnaliha Aaroni pühitsusjäärast ja kõiguta seda kõigutusohvrina Issanda ees; ja see saagu siis sinule!
\par 27 Pühitse kõigutusrinda ja tõstesapsu, mida on kõigutatud ja mida on tõstetud Aaroni ja tema poegade pühitsusohvri jäärast,
\par 28 ja need saagu Iisraeli lastelt Aaronile ja tema poegadele igavese seaduse järgi, sest see on tõstelõiv; ja kui tõstelõiv on see Iisraeli lastelt nende tänu-tapaohvritest, tõstelõiv Issandale!
\par 29 Aaroni pühad riided saagu pärast teda ta poegadele, et neidki nende sees võitaks ja nende käed täidetaks!
\par 30 Seitse päeva kandku neid oma seljas see ta poegadest, kes saab preestriks, kes läheb kogudusetelki, et pühamus teenida!
\par 31 Võta pühitsusjäär ja keeda selle liha pühas paigas!
\par 32 Ja Aaron ja tema pojad söögu jäära liha ja leiba, mis on korvis, kogudusetelgi ukse ees!
\par 33 Nad söögu seda, millega lepitust toodi, kui nende käed täideti ja neid pühitseti; aga võõras ei tohi seda süüa, sest see on püha!
\par 34 Ja kui pühitsusohvri lihast ja leivast midagi jääb üle hommikuks, siis põleta see jääk tulega: seda ei tohi süüa, sest see on püha!
\par 35 Tee Aaroni ja tema poegadega kõik nõnda, nagu ma sind olen käskinud; nende pühitsus kestku seitse päeva!
\par 36 Ohverda iga päev patuohvri härjavärss lepituseks ja puhasta patust altar, tuues selle peal lepitust, ja võia seda pühitsuseks!
\par 37 Seitse päeva toimeta altari lepitust ja pühitse seda; siis altar saab väga pühaks: igaüks, kes altarit puudutab, saab pühaks!
\par 38 Ja see on, mida sa pead ohverdama altaril alaliselt, iga päev: kaks aastast talle.
\par 39 Üks tall ohverda hommikul ja teine tall ohverda õhtul,
\par 40 ja kann peent jahu, segatud kolme kortli tambitud õliga, ja joogiohvriks kolm kortlit veini ühe talle kohta!
\par 41 Teine tall ohverda õhtul; valmista see nagu hommikulgi roaohvri ja joogiohvriga healõhnaliseks tuleohvriks Issandale!
\par 42 See olgu alaliseks põletusohvriks Issanda ees teie sugupõlvedele kogudusetelgi ukse ees, seal, kus ma ennast teile ilmutan, et sinuga rääkida!
\par 43 Seal ma ilmutan ennast Iisraeli lastele ja see saab pühaks minu auhiilguse läbi.
\par 44 Mina pühitsen kogudusetelgi ja altari; ja ma pühitsen enesele preestriteks Aaroni ja tema pojad.
\par 45 Ma tahan elada Iisraeli laste keskel ja olla neile Jumalaks.
\par 46 Ja nemad peavad tundma, et mina olen Issand, nende Jumal, kes tõi nad Egiptusemaalt välja, et elada nende keskel. Mina olen Issand, nende Jumal!

\chapter{30}

\par 1 Tee suitsutusaltar suitsutusohvri toomiseks; tee see akaatsiapuust!
\par 2 See olgu küünar pikk ja küünar lai, neljanurgeline ja kaks küünart kõrge; selle sarved olgu sellega ühest tükist!
\par 3 Karda see puhta kullaga, selle pealis, küljed ümberringi ja sarved; ja tee sellele kuldäär ümber!
\par 4 Tee kaks kuldrõngast selle ääre alla kahele poole; tee need kumbagi külge; need olgu asemeiks kangidele, millega seda kantakse!
\par 5 Tee kangid akaatsiapuust ja karda need kullaga!
\par 6 Pane see tunnistuslaeka ees oleva eesriide ette, kohastikku tunnistuslaeka peal oleva lepituskaanega, kus ma ennast sulle ilmutan!
\par 7 Ja Aaron põletagu selle peal healõhnalist suitsutusrohtu; ta põletagu seda igal hommikul, kui ta lampe korraldab!
\par 8 Ja kui Aaron õhtul lampe üles seab, siis ta põletagu nõndasamuti; see olgu teie tulevastele põlvedele alaline suitsutusohver Issanda palge ees!
\par 9 Ärge ohverdage selle peal võõrast suitsutusrohtu ega põletus- või roaohvrit; selle peale ärge valage ka joogiohvrit!
\par 10 Ja kord aastas toimetagu Aaron selle sarvede peal lepitust: põlvest põlve toimetagu ta kord aastas selle lepitust patu-lepitusohvri verega! See on väga püha Issandale.”
\par 11 Ja Issand rääkis Moosesega, öeldes:
\par 12 „Kui sa arvestad Iisraeli laste päid, neid, kes ära loetakse, siis andku iga mees oma hinge eest lunaraha Issandale, et neid ei tabaks nuhtlus, kui nad ära loetakse!
\par 13 Igaüks, kes astub äraloetavate hulka, andku pool seeklit püha seekli järgi, kakskümmend geera seeklis; tõstelõiv Issandale on pool seeklit.
\par 14 Igaüks, kes astub äraloetavate hulka, kakskümmend aastat vana ja üle selle, peab andma Issandale tõstelõivu!
\par 15 Rikas ärgu andku rohkem ja kehv ärgu andku vähem kui pool seeklit, kui te annate Issandale tõstelõivu lepituseks oma hingede eest!
\par 16 Võta Iisraeli lastelt lepitusraha ja kasuta seda kogudusetelgi teenistuseks; see meenutagu Issanda ees Iisraeli lapsi, et saaksite lepitust oma hingedele!”
\par 17 Ja Issand rääkis Moosesega, öeldes:
\par 18 „Tee pesemise jaoks vasknõu ja selle vaskjalg; pane see kogudusetelgi ja altari vahele ja vala sellesse vett!
\par 19 Aaron ja tema pojad pesku selles oma käsi ja jalgu!
\par 20 Kui nad lähevad kogudusetelki, siis nad peavad endid veega pesema, et nad ei sureks; nõndasamuti, kui nad astuvad teenistuseks altari juurde, et süüdata tuleohvrit Issandale.
\par 21 Nad peavad pesema oma käsi ja jalgu, et nad ei sureks. See olgu neile igaveseks seadluseks, temale ja ta soole põlvest põlve!”
\par 22 Ja Issand rääkis Moosesega, öeldes:
\par 23 „Ja sina võta enesele parimaid palsameid: viissada seeklit sula mürri, ja pool osa sellest, kakssada viiskümmend seeklit, healõhnalist kaneeli, ja kakssada viiskümmend seeklit lõhnavat kalmust,
\par 24 ja viissada seeklit kassiat püha seekli järgi, ja kolm toopi oliiviõli,
\par 25 ja valmista sellest püha võideõli, rohusegajate viisil segatud salvi; see olgu pühaks võideõliks!
\par 26 Võia sellega kogudusetelki ja tunnistuslaegast,
\par 27 lauda ja kõiki selle riistu, lambijalga ja selle riistu, suitsutusaltarit,
\par 28 põletusohvri altarit ja kõiki selle riistu, pesemisnõu ja selle jalga!
\par 29 Ja pühitse neid, et need oleksid väga pühad: igaüks, kes neid puudutab, saab pühaks.
\par 30 Ja võia Aaronit ja tema poegi ja pühitse nad mulle preestriteks!
\par 31 Ja räägi Iisraeli lastega ning ütle: See olgu teile põlvest põlve minu püha võideõli!
\par 32 Tavalise inimese ihu peale ei tohi seda valada ja niisugust segu ei tohi te järele teha: see on püha ja see olgu püha ka teile!
\par 33 Igaüks, kes valmistab niisugust võiet ja annab seda mõnele võõrale, kaotatagu oma rahva seast!”
\par 34 Ja Issand ütles Moosesele: „Võta enesele healõhnalisi aineid: lõhnavat vaiku, teokarpe ja galbanit - healõhnalisi aineid ja puhast viirukit võrdsetes osades -
\par 35 ja valmista neist rohusegajate viisil suitsutusrohi: soolane, puhas, püha!
\par 36 Osa sellest hõõru peeneks ja pane tunnistuslaeka ette kogudusetelgis, sinna, kus ma ennast sulle ilmutan; see olgu teile kõige püham!
\par 37 Suitsutusrohtu, mida sa teed selle segu kohaselt, ei tohi te endile teha: Issandale kuuluvana olgu see sulle püha!
\par 38 Igaüks, kes teeb midagi niisugust, et seda mõnuga nautida, kaotatagu oma rahva seast!”

\chapter{31}

\par 1 Ja Issand rääkis Moosesega, öeldes:
\par 2 „Vaata, ma olen nimepidi kutsunud Betsaleeli, Huuri poja Uuri poja Juuda suguharust.
\par 3 Ma olen tema täitnud Jumala Vaimuga, tarkuse, mõistuse ja teadmistega ning igasugu tööoskusega,
\par 4 et kujundada kunstipäraseid töid kullast, hõbedast ja vasest,
\par 5 uurendada kive raamistuse jaoks ja nikerdada puud, tehes igasugu tööd.
\par 6 Ja mina, vaata, olen andnud temale abiks Oholiabi, Ahisamaki poja Daani suguharust. Ja ma olen andnud kõigi arukate südamesse oskuse teha kõike, mida ma sind olen käskinud:
\par 7 kogudusetelgi, tunnistuslaeka ja lepituskaane, mis on selle peal, ja kõik telgi riistad,
\par 8 laua ja selle riistad, puhtast kullast lambijala ja kõik selle riistad, suitsutusaltari,
\par 9 põletusohvri altari ja kõik selle riistad, pesemisnõu ja selle jala,
\par 10 kootud riided, preester Aaroni pühad riided ja ta poegade preestriameti riided,
\par 11 võideõli ja healõhnalised suitsutusrohud pühamu jaoks. Nad tehku kõik nõnda, nagu ma sind olen käskinud!”
\par 12 Ja Issand rääkis Moosesega, öeldes:
\par 13 „Ja sina räägi Iisraeli lastega ning ütle: Te peate kindlasti pidama mu hingamispäevi, sest see on tähiseks minu ja teie vahel, teadmiseks põlvest põlve, et mina olen Issand, kes teid pühitseb.
\par 14 Pidage siis hingamispäeva, sest see on teile püha; kes seda rikub, seda karistatagu surmaga; sest kes siis iganes tööd teeb, selle hing kaotatagu oma rahva seast!
\par 15 Kuus päeva tehtagu tööd, aga seitsmendal päeval on täielik hingamispäev, Issanda püha; kes iganes hingamispäeval tööd teeb, seda karistatagu surmaga!
\par 16 Ja Iisraeli lapsed pidagu hingamispäeva nõnda, et nad teeksid hingamispäeva igaveseks seaduseks oma sugupõlvedele!
\par 17 Minu ja Iisraeli laste vahel on see igaveseks tähiseks, sest kuue päevaga tegi Issand taeva ja maa, aga seitsmendal päeval ta hingas ja puhkas.”
\par 18 Ja kui ta oli lõpetanud kõneluse Moosesega Siinai mäel, siis ta andis temale kaks tunnistuslauda, kivilauda, mille peale oli Jumala sõrmega kirjutatud.

\chapter{32}

\par 1 Kui rahvas nägi, et Mooses viivitas mäelt tulekuga, siis rahvas kogunes Aaroni ümber ja ütles temale: „Võta kätte ja tee meile jumalaid, kes käiksid meie ees, sest me ei tea, mis on juhtunud Moosesega, selle mehega, kes tõi meid Egiptusemaalt välja!”
\par 2 Siis Aaron ütles neile: „Rebige ära kuldrõngad, mis on kõrvus teie naistel, poegadel ja tütardel, ja tooge need mulle!”
\par 3 Ja kogu rahvas rebis ära kuldrõngad, mis neil kõrvus olid, ja tõi need Aaronile.
\par 4 Tema võttis need vastu nende käest, töötles lõikeriistaga ja valmistas valatud vasika. Siis nad ütlesid: „See on su jumal, Iisrael, kes tõi sind Egiptusemaalt välja!”
\par 5 Kui Aaron seda nägi, siis ta ehitas selle ette altari; ja Aaron hüüdis ning ütles: „Homme on Issanda püha!”
\par 6 Järgmisel päeval tõusid nad vara ning ohverdasid põletusohvreid ja tõid tänuohvreid; ja rahvas istus maha sööma ja jooma ning tõusis üles mängima.
\par 7 Siis Issand rääkis Moosesega: „Mine astu alla, sest su rahvas, kelle sa tõid Egiptusemaalt välja, on teinud pahasti!
\par 8 Nad on kähku pöördunud teelt, mida ma neil käskisin käia. Nad on enestele teinud valatud vasika. Nad on seda kummardanud ja sellele ohverdanud, ja nad on öelnud: See on su jumal, Iisrael, kes tõi sind Egiptusemaalt välja.”
\par 9 Ja Issand ütles Moosesele: „Ma olen näinud seda rahvast, ja vaata, see on kangekaelne rahvas.
\par 10 Jäta nüüd mind üksi, et mu viha saaks süttida põlema nende vastu ja ma saaksin nad hävitada! Aga sind ma teen suureks rahvaks.”
\par 11 Mooses aga anus Issanda, oma Jumala ees ja ütles: „Issand, miks süttib su viha põlema oma rahva vastu, kelle sa tõid Egiptusemaalt välja oma suure võimsuse ja vägeva käega?
\par 12 Miks peaksid egiptlased rääkima ja ütlema: Kurja kavatsusega viis ta nad välja, et neid mägedes tappa ja maa pealt hävitada. Pöördu oma tulisest vihast ja kahetse kurja, mida sa kavatsed teha oma rahvale!
\par 13 Meenuta oma sulaseid Aabrahami, Iisakit ja Iisraeli, kellele sa oled iseenese juures vandunud ja öelnud: Ma teen teie soo nõnda paljuks nagu tähti taevas; ja kogu selle maa, millest ma olen rääkinud, annan ma teie soole ja nad pärivad selle igaveseks.”
\par 14 Ja Issand kahetses kurja, mida ta oli ähvardanud teha oma rahvale.
\par 15 Ja Mooses pöördus ning astus mäelt alla, ja tal oli käes kaks tunnistuslauda, lauad, mille mõlema poole peale oli kirjutatud; neile oli kirjutatud nii ühe kui teise külje peale.
\par 16 Lauad olid Jumala tehtud ja Jumala kirjutatud oli kiri, mis laudadesse oli uurendatud.
\par 17 Kui Joosua kuulis rahva häält, valju karjumist, siis ta ütles Moosesele: „Leeris on sõjakisa!”
\par 18 Aga tema vastas: „Ei see ole võiduhüüd, ei see ole kaotusekisa - mina kuulen lauluhäält.”
\par 19 Ja kui Mooses ligines leerile ja nägi vasikat ning ringtantsu, siis ta viha süttis põlema ja ta viskas lauad käest ning purustas need mäejalamil.
\par 20 Seejärel ta võttis vasika, mille nad olid teinud, ja põletas seda tules ning jahvatas, kuni see sai peeneks; siis ta puistas selle vette ja andis Iisraeli lastele juua.
\par 21 Ja Mooses ütles Aaronile: „Mida see rahvas sulle on teinud, et sa oled nende peale toonud nii suure patu?”
\par 22 Aaron vastas: „Ärgu süttigu mu isanda viha põlema! Sina ise tead, et see rahvas on paha.
\par 23 Nad ütlesid mulle: Tee meile jumalad, kes käiksid meie ees, sest me ei tea, mis on juhtunud Moosesega, selle mehega, kes tõi meid Egiptusemaalt välja!
\par 24 Ja mina ütlesin neile: Kellel on kulda, rebigu ära! Ja nad andsid selle minule, mina viskasin tulle ja välja tuli see vasikas.”
\par 25 Kui Mooses nägi, et rahvas oli ohjest valla, sest Aaron oli ta valla päästnud kahjurõõmuks nende vaenlastele,
\par 26 siis Mooses astus leeri väravasse ja ütles: „Kes on Issanda poolt, tulgu minu juurde!” Ja tema juurde kogunesid kõik Leevi pojad.
\par 27 Ja ta ütles neile: „Nõnda ütleb Issand, Iisraeli Jumal: Igaüks pangu oma mõõk vööle! Käige edasi ja tagasi leeri väravast väravani ja tapke igaüks, olgu see ka oma vend, sõber või sugulane!”
\par 28 Ja Leevi pojad tegid, nagu Mooses käskis; ja sel päeval langes rahva seast ligi kolm tuhat meest.
\par 29 Ja Mooses ütles: „Nad täitsid täna oma käed Issandale, sest igaüks oli oma poja ja venna vastu, selleks et teile täna õnnistus antaks!”
\par 30 Ja järgmisel päeval ütles Mooses rahvale: „Te olete teinud väga suurt pattu. Aga ma lähen nüüd üles Issanda juurde, vahest saan lepitada teie patu.”
\par 31 Ja Mooses läks jälle Issanda juurde ning ütles: „Oh häda! See rahvas on teinud suurt pattu ja on enesele valmistanud kuldjumalad.
\par 32 Kui sa nüüd siiski annaksid andeks nende patu! Aga kui mitte, siis kustuta mind oma raamatust, mille oled kirjutanud!”
\par 33 Aga Issand vastas Moosesele: „Kes minu vastu on pattu teinud, selle ma kustutan oma raamatust.
\par 34 Nüüd aga mine, juhi rahvas sinna paika, millest ma sulle olen rääkinud! Vaata, minu ingel käib su eel. Oma karistuspäeval ma karistan neid nende patu pärast.”
\par 35 Ja Issand lõi rahvast, sellepärast et nad olid teinud vasika, mille Aaron valmistas.

\chapter{33}

\par 1 Ja Issand ütles Moosesele: „Mine, lahku siit, sina ja rahvas, kelle sa tõid Egiptusemaalt välja, maale, mille ma vandega olen tõotanud Aabrahamile, Iisakile ja Jaakobile, öeldes: Sinu soole ma annan selle!
\par 2 Ma läkitan sinu eel ingli ja ajan ära kaananlased, emorlased, hetid, perislased, hiivlased ja jebuuslased,
\par 3 et sa jõuaksid maale, mis piima ja mett voolab, sest mina ise ei lähe koos sinuga, et sind mitte hävitada teel, kuna sa oled kangekaelne rahvas.”
\par 4 Kui rahvas kuulis seda kurja kõnet, siis nad leinasid ja ükski ei pannud enesele ehteid ümber.
\par 5 Ja Issand ütles Moosesele: „Ütle Iisraeli lastele: Te olete kangekaelne rahvas. Kui ma läheksin ühe silmapilgugi koos sinuga, peaksin sinu hävitama. Võta nüüd ära oma ehted, siis ma mõtlen, mis ma sinuga teen!”
\par 6 Ja Iisraeli lapsed rebisidki endilt ehted Hoorebi mäe juurest lahkumisel.
\par 7 Mooses aga võttis telgi ja püstitas selle väljapoole leeri, leerist kaugemale, ja nimetas selle kogudusetelgiks; ja igaüks, kes otsis Issandat, läks kogudusetelgi juurde, mis oli väljaspool leeri.
\par 8 Iga kord, kui Mooses läks välja telgi juurde, tõusis kogu rahvas püsti ja igamees seisis oma telgi uksel ning vaatas Moosesele järele, kuni ta oli läinud telki.
\par 9 Ja iga kord, kui Mooses oli läinud telki, laskus pilvesammas alla ning seisis telgi ukse kohal; ja ta kõneles Moosesega.
\par 10 Ja kui kogu rahvas nägi pilvesammast seisvat telgi ukse kohal, siis kogu rahvas tõusis üles ja nad kummardasid igaüks oma telgi uksel.
\par 11 Ja Issand kõneles Moosesega palgest palgesse, nagu räägiks mees oma sõbraga. Siis Mooses tuli tagasi leeri, aga tema teener Joosua, Nuuni poeg, noor mees, ei lahkunud telgist.
\par 12 Ja Mooses ütles Issandale: „Vaata, sa ütled mulle: Vii see rahvas sinna! Aga sa ei ole mulle teada andnud, keda sa koos minuga läkitad. Ometi oled sa ise öelnud: Ma tunnen sind nimepidi ja sa oled ka armu leidnud minu silmis.
\par 13 Aga kui ma nüüd olen armu leidnud sinu silmis, siis õpeta mulle oma teed, et ma tunneksin sind ja leiaksin armu su silmis, sest vaata, see rahvas on sinu rahvas.”
\par 14 Ta vastas: „Minu pale läheb kaasa ja ma annan sulle rahu.”
\par 15 Siis Mooses ütles temale: „Kui su pale ei tule kaasa, siis ära vii meid siit ära!
\par 16 Sest millest muidu tuntakse, et oleme armu leidnud sinu silmis, mina ja su rahvas, kui sellest, et sina käid koos meiega, nõnda et meie, mina ja su rahvas, erineme kogu rahvast, kes maa peal on?”
\par 17 Ja Issand vastas Moosesele: „Mina teengi nõnda, nagu oled soovinud, sest sa oled armu leidnud minu silmis ja ma tunnen sind nimepidi.”
\par 18 Aga Mooses ütles: „Näita siis mulle oma auhiilgust!”
\par 19 Ja tema vastas: „Ma lasen sinu eest mööduda kogu oma ilu ja kuulutan sinu ees Issanda nime. Ja ma olen armuline, kellele olen armuline, ja halastan, kelle peale halastan.”
\par 20 Ja ta ütles veel: „Sa ei tohi näha mu palet, sest ükski inimene ei või mind näha ja jääda elama!”
\par 21 Siis ütles Issand: „Vaata, siin mu juures on üks paik; astu selle kalju peale!
\par 22 Kui mu auhiilgus möödub, siis ma panen sind kaljulõhesse ja katan sind oma käega, kuni ma olen möödunud.
\par 23 Kui ma siis oma käe ära võtan, näed sa mind selja tagant, aga mu palet ei tohi keegi näha!”

\chapter{34}

\par 1 Ja Issand ütles Moosesele: „Raiu enesele kaks kivilauda, esimeste sarnased, ja ma kirjutan laudade peale sõnad, mis olid esimestel laudadel, mis sa purustasid!
\par 2 Ole hommikuks valmis! Mine hommikul üles Siinai mäele ja seisa seal mu ees mäetipus!
\par 3 Aga ükski ei tohi koos sinuga üles tulla ja kogu mäel ärgu nähtagu ka mitte kedagi, isegi lambaid, kitsi ja veised ei tohi selle mäe ees karjas olla!”
\par 4 Siis Mooses raius kaks kivilauda, esimeste sarnased. Ja Mooses tõusis hommikul vara ning läks üles Siinai mäele, nõnda nagu Issand teda oli käskinud, ja võttis kätte need kaks kivilauda.
\par 5 Ja Issand laskus alla pilve sees. Mooses astus seal tema juurde ja hüüdis Issanda nime.
\par 6 Ja Issand möödus tema eest ning hüüdis: „Issand, Issand on halastaja ja armuline Jumal, pika meelega ja rikas heldusest ning tõest,
\par 7 kes säilitab heldust tuhandeile, annab andeks ülekohtu ja üleastumised ning patu, aga kes siiski ei jäta süüdlast karistamata, vaid nuhtleb vanemate patud lastele ja lastelastele kolmanda ja neljanda põlveni!”
\par 8 Siis Mooses kummardas kiiresti maani, heitis silmili maha
\par 9 ja ütles: „Issand, kui ma nüüd olen armu leidnud sinu silmis, siis käigu Issand meie keskel! Kuigi see on kangekaelne rahvas, anna siiski andeks meie pahateod ja patt ja võta meid oma pärisosaks!”
\par 10 Ja tema vastas: „Vaata, ma teen lepingu; ma teen kogu su rahva ees imetegusid, milliseid ei ole tehtud ühelgi maal ega ühegi rahva juures. Kogu see rahvas, kelle keskel sa oled, saab näha Issanda tegu, ja kui hirmuäratav on see, mis ma sulle teen.
\par 11 Pane tähele, mida ma täna käsin sind teha! Vaata, ma ajan su eest ära emorlased, kaananlased, hetid, perislased, hiivlased ja jebuuslased.
\par 12 Hoia, et sa ei tee lepingut selle maa elanikega, kuhu sa tuled, et nad ei saaks püüdepaelaks teie keskel,
\par 13 vaid lammutage nende altarid, purustage sambad ja raiuge maha viljakustulbad,
\par 14 sest sa ei tohi kummardada teist jumalat, sellepärast et Issanda nimeks on „Püha viha”! Ta on püha vihaga Jumal.
\par 15 Sellepärast ära tee lepingut maa elanikega, sest need käivad hoora viisil oma jumalate järel, ohverdavad oma jumalatele ja kutsuvad sindki sööma nende ohvreist!
\par 16 Ja kui sa oma poegadele võtad naisi nende tütreist ja nende tütred käivad hoora viisil oma jumalate järel, siis nad mõjustavad ka sinu poegi käima hoora viisil nende jumalate järel.
\par 17 Sa ei tohi enesele teha valatud jumalaid!
\par 18 Pea hapnemata leibade püha: seitse päeva söö hapnemata leiba, nõnda nagu ma sind olen käskinud, määratud ajal aabibikuus, sest aabibikuus tulid sa Egiptusest välja!
\par 19 Kõik, kes avavad emakoja, on minu päralt; samuti kõik su isased kariloomad, veiste ja lammaste esmikud.
\par 20 Aga eesli esmik lunasta ühe tallega; kui sa ei lunasta, siis murra tal kael! Lunasta kõik oma esmasündinud pojad! Ja tühje käsi ei tohi tulla mu palge ette!
\par 21 Kuus päeva tee tööd, aga seitsmendal päeval puhka! Ka künni- ja lõikusajal pea puhkust!
\par 22 Pea nädalatepüha, kui lõikad uudsenisu, ja vilja kokkupanemise püha aasta lõpus!
\par 23 Kolm korda aastas peab kogu su meessugu tulema Issanda, Iisraeli Jumala palge ette!
\par 24 Jah, ma ajan rahvad ära sinu eest ja laiendan su maa-ala, ja ükski ei himusta su maad, kui sa lähed, et tulla kolm korda aastas Issanda, oma Jumala palge ette.
\par 25 Sa ei tohi ohverdada mu tapaohvri verd haput leiba süües, ja paasapüha tapaohver ei tohi seista üle öö hommikuni!
\par 26 Parim osa oma põllu uudseviljast vii Issanda, oma Jumala kotta! Ära keeda kitsetalle ta ema piimas!”
\par 27 Ja Issand ütles Moosesele: „Kirjuta enesele üles need sõnad, sest nende sõnade põhjal annan ma seaduse sinule ja Iisraelile!”
\par 28 Ja ta oli seal Issanda juures nelikümmend päeva ja nelikümmend ööd leiba söömata ja vett joomata, ja ta kirjutas laudade peale seaduse sõnad, need kümme käsusõna.
\par 29 Ja kui Mooses astus alla Siinai mäelt ja Moosesel oli tulles käes kaks tunnistuslauda, siis Mooses ise ei teadnudki, et ta palge nahk hiilgas kõneluse tõttu Issandaga.
\par 30 Kui Aaron ja kõik Iisraeli lapsed nägid Moosest, vaata, siis hiilgas tema palge nahk ja nad kartsid temale ligineda.
\par 31 Aga Mooses hüüdis neid; siis Aaron ja kõik koguduse ülemad tulid jälle tema juurde ja Mooses rääkis nendega.
\par 32 Seejärel kõik Iisraeli lapsed tulid lähedale ja ta käskis neid teha kõike, mida Issand temale oli öelnud Siinai mäel.
\par 33 Kui Mooses oli lõpetanud nendega kõnelemise, siis ta pani katte oma näo ette.
\par 34 Ja iga kord, kui Mooses läks Issanda palge ette temaga rääkima, pani ta katte ära kuni väljatulekuni; ja olles välja tulnud, rääkis ta Iisraeli lastele, mida oli kästud.
\par 35 Siis Iisraeli lapsed vaatasid Moosese palet, sest Moosese palge nahk hiilgas. Aga Mooses pani jälle katte oma näo ette, kuni ta läks sisse, et temaga rääkida.

\chapter{35}

\par 1 Ja Mooses kogus kokku kogu Iisraeli laste koguduse ning ütles neile: „Need on sõnad, mille järgi Issand on teid käskinud teha:
\par 2 Kuus päeva tehtagu tööd, aga seitsmes päev olgu teile püha, täielik Issanda hingamispäev; kes siis iganes tööd teeb, seda karistatagu surmaga!
\par 3 Ärge süüdake tuld hingamispäeval, kus te iganes elate!”
\par 4 Ja Mooses rääkis kogu Iisraeli laste kogudusega, öeldes: „Issand on käskinud ja öelnud nõnda:
\par 5 Võtke sellest, mis teil on, Issandale tõstelõivu; igaüks, kes heast südamest tahab anda, toogu Issandale tõstelõivuks kulda, hõbedat ja vaske,
\par 6 sinist, purpurpunast ja helepunast lõnga ning peent linast lõime, kitsekarvu,
\par 7 punaseid jääranahku, merilehmanahku, akaatsiapuud,
\par 8 valgustusõli, palsamit võideõliks, healõhnalisi suitsutusrohte,
\par 9 karneoolikive ja ilustuskive õlarüü ja rinnakilbi jaoks.
\par 10 Ja igaüks, kes teist on osav, tulgu ja tehku kõik, mida Issand on käskinud:
\par 11 elamu, selle telk ja kate, haagid ja lauad, põiklatid, sambad ja jalad,
\par 12 laegas ja selle kandekangid, lepituskaas ja kattev eesriie,
\par 13 laud, selle kandekangid ja kõik selle riistad, ohvrileivad,
\par 14 lambijalg valgustuseks ja selle riistad, lambid ja valgustusõli,
\par 15 suitsutusaltar ja selle kandekangid, võideõli, healõhnalised suitsutusrohud, ukse kate elamu uksele,
\par 16 põletusohvri altar ja vaskvõrk, mis kuulub selle juurde, selle kandekangid ja kõik riistad, pesemisnõu ja selle jalg,
\par 17 õue eesriided, sambad ja jalad, õuevärava kate,
\par 18 elamu vaiad, õue vaiad ja nende nöörid,
\par 19 ametiriided pühamu teenistuseks, preester Aaroni pühad riided ja tema poegade preestriameti riided.”
\par 20 Ja kogu Iisraeli laste kogudus läks ära Moosese juurest.
\par 21 Siis nad tulid tagasi, igaüks, keda süda sundis, ja igaüks, kes hea meelega tahtis anda, tõi Issandale tõstelõivu kogudusetelgi ehitamiseks, kõige selle teenistuse ja pühade riiete jaoks.
\par 22 Nad tulid, niihästi mehed kui naised, igaüks, kes heast südamest tahtis anda, ja tõid sõlgi, kõrvarõngaid, sõrmuseid ja keesid, igasugu kuldriistu, kõik, kes tõid Issandale kõigutusohvriks kulda.
\par 23 Ja kõik, kellel leidus sinist, purpurpunast ja helepunast lõnga ning linast lõime, kitsekarvu, punaseid jääranahku ja merilehmanahku, tõid neid.
\par 24 Igaüks, kes sai tõstelõivuks annetada hõbedat või vaske, tõi Issandale tõstelõivu; ja igaüks, kellel leidus akaatsiapuud, tõi seda kõigi tööde tarviduseks.
\par 25 Ja kõik osavad naised ketrasid käsitsi ning tõid kedratud sinist, purpurpunast ja helepunast lõnga ning linast lõime.
\par 26 Ja kõik naised, keda süda selleks sundis ja kes oskasid, ketrasid kitsekarvu.
\par 27 Aga ülemad tõid karneoolikive ja ilustuskive õlarüü ja rinnakilbi jaoks,
\par 28 palsamit, õli valgustuseks ning võideõliks ja healõhnaliseks suitsutusrohuks.
\par 29 Kõik Iisraeli lapsed, mehed ja naised, keda süda sundis tooma kõige valmistamiseks, mida Issand oli Moosese läbi käskinud teha, tõid Issandale vabatahtliku ohvri.
\par 30 Ja Mooses ütles Iisraeli lastele: „Vaadake, Issand on nimepidi kutsunud Betsaleeli, Huuri poja Uuri poja Juuda suguharust,
\par 31 ja on teda täitnud Jumala Vaimuga, tarkuse, mõistuse ja teadmistega ning igasugu tööoskusega,
\par 32 et kujundada kunstipäraseid töid kullast, hõbedast ja vasest,
\par 33 uurendada kive raamistuse jaoks ja nikerdada puud, tehes igasugu kunstipärast tööd.
\par 34 Ta on temale ja Oholiabile, Ahisamaki pojale Daani suguharust, andnud oskuse teisi õpetada.
\par 35 Ta on neid täitnud oskusega teha igasugu tööd, olgu sepa- või kuduja- või kangakirjajatöö sinisest, purpurpunasest ja helepunasest lõngast ning linasest lõimest, või olgu kangrutöö, teha kõiki töid ja olla leidlik kunstipärastes töödes.

\chapter{36}

\par 1 Betsaleel ja Oholiab ning kõik oskusega mehed, kellele Issand on andnud tarkuse ja mõistuse, et nad teaksid, kuidas teha kõiki töid pühamu ehitamisel, tehku kõik, nagu Issand on käskinud!”
\par 2 Siis Mooses kutsus seda tegema Betsaleeli ja Oholiabi ning kõik osavad mehed, kellele Issand oli andnud tarkust südamesse, igaühe, keda ta süda sundis tööle minema.
\par 3 Ja nemad võtsid Mooseselt vastu kogu tõstelõivu, mida Iisraeli lapsed olid toonud pühamu ehitustööks, mis neil tuli teha. Aga temale toodi igal hommikul veel vabatahtlikke ande.
\par 4 Siis tulid kõik need osavad, kes tegid kõiki pühamu töid, igaüks oma töö juurest, mida ta parajasti oli tegemas,
\par 5 ja rääkisid Moosesega, öeldes: „Rahvas toob rohkem, kui on tarvis selle töö tegemiseks, mida Issand on käskinud teha.”
\par 6 Siis Mooses käskis leeris kuulutada ja öelda: „Ükski mees ja naine ärgu töötagu enam pühamu tõstelõivu jaoks!” Siis rahvas enam ei toonud,
\par 7 sest tagavara oli küllaldane kogu tööks, mida tuli teha, ja jäi ülegi.
\par 8 Siis tegid kõik oskusega mehed töötegijate hulgast elamu kümnest vaibast, mis olid korrutatud linasest lõimest ning sinisest, purpurpunasest ja helepunasest lõngast kunstipäraselt sissekootud keerubitega.
\par 9 Iga vaiba pikkus oli kakskümmend kaheksa küünart ja iga vaiba laius neli küünart; kõigil vaipadel olid samad mõõdud.
\par 10 Viis vaipa seoti üksteisega kokku, samuti seoti ka Viis teist vaipa üksteisega kokku.
\par 11 Siis tehti sinised aasad esimese vaiba servale, äärmisele esimeses reastuses; samasugused tehti ka teise reastuse äärmise vaiba servale.
\par 12 Viiskümmend aasa tehti esimesele vaibale ja Viiskümmend aasa tehti vaiba servale, mis oli teises reastuses; aasad olid üksteisega kohakuti.
\par 13 Siis tehti viiskümmend kuldhaaki ja vaibad ühendati üksteise külge; nõnda sai täielik elamu.
\par 14 Siis tehti kitsekarvadest vaibad telgiks elamu kohale; neid vaipu tehti üksteist.
\par 15 Iga vaiba pikkus oli kolmkümmend küünart ja iga vaiba laius neli küünart; üheteistkümnel vaibal olid samad mõõdud.
\par 16 Siis seoti kokku viis vaipa eraldi ja kuus vaipa eraldi.
\par 17 Vaiba servale, äärmisele reastuses, tehti viiskümmend aasa, samuti tehti vaiba servale teises reastuses viiskümmend aasa.
\par 18 Siis tehti viiskümmend vaskhaaki telgi ühendamiseks üheks tervikuks.
\par 19 Ja telgile tehti kate punastest jääranahkadest ning selle peale veel kate merilehmanahkadest.
\par 20 Elamule tehti püstipandavad lauad akaatsiapuust.
\par 21 Iga laua pikkus oli kümme küünart ja iga laua laius oli poolteist küünart.
\par 22 Igal laual oli kaks tappi, mis olid üksteisega ühendatud; nõnda tehti kõik elamu lauad.
\par 23 Ja elamule tehti laudu: kakskümmend lauda lõunapoolse külje jaoks lõuna suunas.
\par 24 Kahekümnele lauale tehti alla nelikümmend hõbejalga: kaks jalga iga laua alla mõlema tapi jaoks.
\par 25 Elamu teise külje jaoks põhja suunas tehti kakskümmend lauda
\par 26 ja nende nelikümmend hõbejalga: kaks jalga iga laua all.
\par 27 Aga elamu läänepoolse tagakülje jaoks tehti kuus lauda.
\par 28 Ja kaks lauda tehti elamu tagakülje nurkade jaoks.
\par 29 Need moodustasid kaksiklauad, mis olid ühendatud alt üles kuni esimese rõngani; nõnda tehti nende mõlemaga mõlema nurga jaoks.
\par 30 Neid oli siis kaheksa lauda ja nende hõbejalgu oli kuusteist jalga: kaks jalga iga laua all.
\par 31 Siis tehti akaatsiapuust põiklatid: viis latti elamu ühe külje laudade jaoks,
\par 32 viis latti elamu teise külje laudade jaoks ja viis latti elamu laudade jaoks läänepoolses tagaküljes.
\par 33 Keskmine põiklatt tehti kulgema keset laudu servast servani.
\par 34 Lauad karrati kullaga ja neile tehti kuldrõngad lattide asemeiks; ka latid karrati kullaga.
\par 35 Siis tehti eesriie sinisest, purpurpunasest ja helepunasest lõngast ning korrutatud linasest lõimest; see tehti kunstipäraselt sissekootud keerubitega.
\par 36 Sellele tehti neli akaatsiapuust sammast ja need karrati kullaga; nende haagid olid kullast ja neile valati neli hõbejalga.
\par 37 Siis tehti telgi uksele kate sinisest, purpurpunasest ja helepunasest lõngast ning korrutatud linasest lõimest kunstipäraselt kootuna,
\par 38 ja selle viis sammast ning nende haagid. Nende nupud ja põrgad karrati kullaga; nende viis jalga tehti vasest.

\chapter{37}

\par 1 Ja Betsaleel tegi akaatsiapuust laeka, kaks ja pool küünart pika, poolteist küünart laia ja poolteist küünart kõrge.
\par 2 Ta kardas selle puhta kullaga seest- ja väljastpoolt ja tegi sellele kuldpärja ümber.
\par 3 Ta valas sellele neli kuldrõngast nelja jala jaoks: kaks rõngast ühte külge ja kaks rõngast teise külge.
\par 4 Ta tegi akaatsiapuust kandekangid ja kardas need kullaga.
\par 5 Ta pistis kangid rõngastesse laeka külgedel, laeka kandmiseks.
\par 6 Ta tegi puhtast kullast lepituskaane, kaks ja pool küünart pika ja poolteist küünart laia.
\par 7 Ta tegi kullast kaks keerubit; ta tegi need sepisena lepituskaane kumbagi otsa,
\par 8 ühe keerubi ühte ja teise keerubi teise otsa; ta tegi kummaski otsas olevad keerubid lepituskaanega ühest tükist.
\par 9 Keerubid sirutasid oma tiivad ülespoole, kattes oma tiibadega lepituskaant, ja nende palged olid vastamisi; keerubite palged olid lepituskaane poole.
\par 10 Ja ta tegi akaatsiapuust laua, kaks küünart pika, küünar laia ja poolteist küünart kõrge.
\par 11 Ta kardas selle puhta kullaga ja tegi sellele kuldpärja ümber.
\par 12 Ta tegi sellele kämblalaiuse põõna ümber ja tegi ümber põõna kuldpärja.
\par 13 Ta valas sellele neli kuldrõngast ja kinnitas rõngad nelja nurga külge selle nelja jala juures.
\par 14 Rõngad olid otse põõna kõrval kangide asemeiks laua kandmisel.
\par 15 Ta tegi akaatsiapuust kangid laua kandmiseks ja kardas need kullaga.
\par 16 Ja ta tegi puhtast kullast riistad, mis olid laua peal: vaagnad ja kausid, peekrid ja kannud joogiohvri toomiseks.
\par 17 Ja ta tegi puhtast kullast lambijala; ta tegi selle lambijala koos aluse ja harudega sepisetööna; selle karikakesed, nupud ja õiekesed olid sellega ühest tükist.
\par 18 Selle küljest lähtus kuus haru: ühest küljest kolm lambijala haru ja teisest küljest kolm lambijala haru.
\par 19 Ühel harul oli kolm mandliõiekujulist karikakest nupu ja õiekesega, samuti oli teisel harul kolm mandliõiekujulist karikakest nupu ja õiekesega; nõnda oli neil kuuel harul, mis lambijalast lähtusid.
\par 20 Aga lambijalal enesel oli neli mandliõiekujulist karikakest nupu ja õiekesega:
\par 21 iga harupaari all oli üks nupp neil kuuel harul, mis lambijalast väljusid.
\par 22 Nupud ja harud olid sellega ühest tükist, tervikliku sepisetööna puhtast kullast.
\par 23 Ja ta tegi sellele seitse lampi ning puhtast kullast tahikäärid ja tahikarbid.
\par 24 Ta tegi selle koos kõigi selle riistadega ühest talendist puhtast kullast.
\par 25 Ja ta tegi akaatsiapuust suitsutusaltari, küünar pika ja küünar laia, neljanurgelise ja kaks küünart kõrge; sarved olid sellega ühest tükist.
\par 26 Ta kardas selle puhta kullaga, selle pealise, küljed ümberringi ja sarved; ja ta tegi sellele kuldääre ümber.
\par 27 Ta tegi sellele kaks kuldrõngast selle ääre alla, kahele poole, kumbagi külge asemeiks kangidele, millega seda kanti.
\par 28 Ta tegi kangid akaatsiapuust ja kardas need kullaga.
\par 29 Ja ta valmistas püha võideõli ja puhast, healõhnalist suitsutusrohtu rohusegajate viisil.

\chapter{38}

\par 1 Ja ta tegi akaatsiapuust põletusohvrialtari, viis küünart pika ja viis küünart laia, neljanurgelise ja kolm küünart kõrge.
\par 2 Ta tegi selle neljale nurgale sarved; sarved olid sellega ühest tükist; ja ta kardas selle vasega.
\par 3 Ta tegi kõik altari riistad, tuhanõud, labidad, piserdusnõud, hargid ja sütepannid; kõik selle riistad ta tegi vasest.
\par 4 Ta tegi altarile võrestiku, võrgukujulise töö vasest, ääre alla, altpoolt vaadates poole altarini.
\par 5 Ta valas neli rõngast vaskvõrestiku neljale nurgale kangide asemeiks.
\par 6 Ta tegi akaatsiapuust kandekangid ja kardas need vasega.
\par 7 Ta pistis kandekangid rõngastesse altari külgedel; ta tegi selle laudadest õõnsakujulisena.
\par 8 Ja ta tegi vaskpesemisnõu ja selle vaskjala nende teenistuses olevate naiste peeglitest, kes teenisid kogudusetelgi ukse juures.
\par 9 Ja ta tegi õue: lõunapoolses küljes, keskpäeva pool, oli õuel sada küünart korrutatud linasest lõimest eesriideid,
\par 10 nende kakskümmend sammast ja kakskümmend vaskjalga; sammaste haagid ja põrgad olid hõbedast.
\par 11 Nõndasamuti oli põhjapoolses küljes sada küünart eesriideid, nende kakskümmend sammast ja kakskümmend vaskjalga; sammaste haagid ja põrgad olid hõbedast.
\par 12 Läänepoolses küljes oli viiskümmend küünart eesriideid, nende kümme sammast ja kümme jalga; sammaste haagid ja põrgad olid hõbedast.
\par 13 Idapoolses küljes, päikesetõusu pool, oli viiskümmend küünart eesriideid:
\par 14 ühel pool oli viisteist küünart eesriideid, nende kolm sammast ja kolm jalga,
\par 15 samuti oli teisel pool; ühel ja teisel pool õueväravat oli võrdselt viisteist küünart eesriideid, nende kolm sammast ja kolm jalga.
\par 16 Kõik eesriided ümber õue olid korrutatud linasest lõimest.
\par 17 Sammaste jalad olid vasest, sammaste haagid ja põrgad hõbedast, nende nupud hõbedaga karratud; kõigil õue sammastel olid hõbepõrgad.
\par 18 Õuevärava kate oli kirjatud töö sinisest, purpurpunasest ja helepunasest lõngast ning korrutatud linasest lõimest, kakskümmend küünart pikk, laiusele vastavalt viis küünart kõrge, nagu muud õue eesriided.
\par 19 Nende neli sammast ja neli jalga olid vasest, nende haagid hõbedast, nende nuppude kard ja põrgad samuti hõbedast.
\par 20 Kõik elamu ja ümber oleva õue vaiad olid vasest.
\par 21 See on elamu, tunnistuselamu kulude arvestus, mis tehti Moosese käsul; selle tegid leviidid preester Aaroni poja Iitamari juhatusel:
\par 22 Betsaleel, Huuri poja Uuri poeg Juuda suguharust on valmistanud kõik, milleks Issand oli Moosesele käsu andnud,
\par 23 ja koos temaga Oholiab, Ahisamaki poeg Daani suguharust, kui sepp ning osav kuduja ja kangakirjaja sinist, purpurpunast ja helepunast lõnga ning linast lõime kasutades.
\par 24 Kõike kulda, mis oli kõigutusohvri kuld, mida tööks tarvitati kõigi pühamu tööde juures, oli kakskümmend üheksa talenti ja seitsesada kolmkümmend seeklit püha seekli järgi.
\par 25 Ja hõbedat neilt, kes koguduse hulgast olid ära loetud, oli sada talenti ja tuhat seitsesada seitsekümmend viis seeklit püha seekli järgi,
\par 26 üks beka pea kohta, see on pool seeklit püha seekli järgi kõigilt, kes astusid äraloetute hulka, kahekümneaastased ja üle selle, kuuesaja kolme tuhande viiesaja viiekümnelt.
\par 27 See sada talenti hõbedat oli pühamu jalgade ja eesriide jalgade valamiseks, sada talenti sajaks jalaks, talent iga jala jaoks.
\par 28 Aga sellest tuhande seitsmesaja seitsmekümne viiest seeklist tegi ta sammastele haagid, kardas nende nupud ja valmistas põrgad.
\par 29 Kõigutusohvri vaske oli seitsekümmend talenti ja kaks tuhat nelisada seeklit.
\par 30 Sellest tehti kogudusetelgi ukse jalad, vaskaltar ja selle küljes olev vaskvõrestik, kõik altari riistad,
\par 31 ümberringi oleva õue jalad, õuevärava jalad, kõik telgi vaiad ja kõik ümberringi oleva õue vaiad.

\chapter{39}

\par 1 Ja kootud riided pühamu teenistuseks tehti sinisest, purpurpunasest ja helepunasest lõngast; samuti tehti Aaronile pühad riided, nagu Issand Moosest oli käskinud.
\par 2 Õlarüü tehti kuldsest, sinisest, purpurpunasest ja helepunasest lõngast ning korrutatud linasest lõimest.
\par 3 Kuldplaadid taoti õhukeseks ja lõigati kiududeks, et neid kunstipäraselt kududa sinise, purpurpunase ja helepunase lõnga ning linase lõime sekka.
\par 4 Õlarüüle tehti ühendatud õlatükid, mõlemast otsast seotud.
\par 5 Kunstipärane kinnitusvöö selle küljes oli sellega ühest tükist ja samasugune töö kuldsest, sinisest, purpurpunasest ja helepunasest lõngast ning korrutatud linasest lõimest, nagu Issand Moosest oli käskinud.
\par 6 Siis ümbritseti kuldäärisega karneoolikivid, millesse olid pitsatitaoliselt uurendatud Iisraeli poegade nimed.
\par 7 Need asetati õlarüü õlatükkidele kui mälestuskivid Iisraeli poegadele, nagu Issand Moosest oli käskinud.
\par 8 Rinnakilp tehti kunstipäraselt kootuna, nagu oli tehtud õlarüügi, kuldsest, sinisest, purpurpunasest ja helepunasest lõngast ning korrutatud linasest lõimest.
\par 9 See oli neljanurgeline; rinnakilp oli tehtud kahekordsena, vaksapikkune ja vaksalaiune kahekordselt.
\par 10 See kaeti nelja rea kalliskividega: rubiin, topaas, smaragd ridamisi esimeses reas;
\par 11 teises reas: türkiis, safiir, jaspis;
\par 12 kolmandas reas: hüatsint, ahhaat, ametüst;
\par 13 neljandas reas: krüsoliit, karneool, nefriit; kuldäärisest ümbritsetuna olid need oma asemeis.
\par 14 Kivid olid vastavalt Iisraeli poegade nimedele nende kaheteistkümne nimega, igal pitsatitaoliselt uurendatud nimi vastavalt kaheteistkümnele suguharule.
\par 15 Rinnakilbile tehti puhtast kullast nööritaoliselt keerutatud ketid.
\par 16 Siis tehti kaks kuldäärist ja kaks kuldrõngast; need mõlemad rõngad kinnitati rinnakilbi kahe nurga külge.
\par 17 Ja need kaks kuldnööri pandi mõlemasse rõngasse rinnakilbi nurkadel.
\par 18 Mõlema nööri mõlemad otsad kinnitati mõlema äärise külge ja need kinnitati õlatükkidele õlarüü esiküljes.
\par 19 Siis tehti kaks kuldrõngast ja need pandi rinnakilbi kahte nurka, selle ääre külge, mis on seespool vastu õlarüüd.
\par 20 Siis tehti veel kaks kuldrõngast ja need kinnitati õlarüü mõlema õlatüki külge, selle esikülje allosasse, ühenduskohale ülespoole õlarüü vööd.
\par 21 Rõngastega rinnakilp seoti sinise nööriga õlarüü rõngaste külge, nõnda et see oli ülalpool õlarüü vööd, ja et rinnakilp ei saanud lahti tulla õlarüü küljest, nagu Issand Moosest oli käskinud.
\par 22 Õlarüü ülekuub tehti üleni sinisest lõngast kootuna.
\par 23 Ülekuue pea avaus oli nagu raudrüü avaus: avause ümber oli äär, et see ei rebeneks.
\par 24 Ülekuue palistuse külge tehti granaatõunad sinisest, purpurpunasest ja helepunasest korrutatud lõngast.
\par 25 Siis tehti puhtast kullast kellukesed ja need kinnitati granaatõunte vahele ümber ülekuue palistuse, granaatõunte vahele:
\par 26 kelluke ja granaatõun, kelluke ja granaatõun ümber teenistusülekuue palistuse, nagu Issand Moosest oli käskinud.
\par 27 Aaronile ja tema poegadele tehti kootud linased särgid,
\par 28 linasest riidest peakate ja linasest riidest mähitavad peamähised, linased korrutatud lõngast püksid,
\par 29 ja korrutatud linasest, sinisest, purpurpunasest ja helepunasest lõngast kunstipäraselt tehtud vöö, nagu Issand Moosest oli käskinud.
\par 30 Puhtast kullast tehti laubaehe, püha kroon, ja selle peale kirjutati pitsatiuurenduse sarnaselt kiri: „Issandale pühitsetud.”
\par 31 Selle külge kinnitati sinine nöör, millega see seoti üles peakatte külge, nagu Issand Moosest oli käskinud.
\par 32 Nõnda valmis kogu töö: elamu, kogudusetelk. Iisraeli lapsed tegid kõik. Nõnda nagu Issand Moosest oli käskinud, nõnda nad tegid.
\par 33 Siis nad tõid Moosese juurde elamu, telgi ja kõik selle riistad, haagid, lauad, põiklatid, sambad ja jalad,
\par 34 katte punastest jääranahkadest, katte merilehmanahkadest, katva eesriide,
\par 35 tunnistuslaeka ja selle kangid, lepituskaane,
\par 36 laua kõigi selle riistadega, ohvrileivad,
\par 37 puhtast kullast lambijala lampidega lampide reastuses, ja kõik selle riistad, valgustusõli,
\par 38 kuldaltari, võideõli, healõhnalisi suitsutusrohte, telgi ukse katte,
\par 39 vaskaltari ja vaskvõrestiku selle küljes, selle kangid ja kõik riistad, pesemisnõu ja selle jala,
\par 40 õue eesriided, selle sambad ja jalad, õuevärava katte, selle nöörid ja vaiad, kõik riistad kogudusetelgi teenistuseks,
\par 41 ametiriided pühamu teenistuseks, preester Aaroni pühad riided ja tema poegade preestriameti riided.
\par 42 Täpselt, nagu Issand Moosest oli käskinud, nõnda olid Iisraeli lapsed teinud kõik selle töö.
\par 43 Ja Mooses vaatas kõike seda tööd ja näe, nad olid selle teinud, nagu Issand oli käskinud. Nad olid just nõnda teinud. Ja Mooses õnnistas neid.

\chapter{40}

\par 1 Ja Issand rääkis Moosesega, öeldes:
\par 2 „Esimese kuu esimesel päeval püstita elamu-kogudusetelk!
\par 3 Aseta sinna tunnistuslaegas ja varja laegast eesriidega!
\par 4 Vii sinna laud ja korralda, mis selle peal tuleb korraldada; vii sinna lambijalg ja sea lambid üles!
\par 5 Aseta kuldne suitsutusaltar tunnistuslaeka ette ja riputa kate elamu uksele!
\par 6 Aseta põletusohvri altar elamu-kogudusetelgi ukse ette!
\par 7 Aseta pesemisnõu kogudusetelgi ja altari vahele ja pane sellesse vett!
\par 8 Sea õu ümberringi üles ja riputa kate õueväravale!
\par 9 Võta võideõli ja võia elamut ja kõike, mis selles on; pühitse seda ja kõiki selle riistu, et see saaks pühaks!
\par 10 Võia põletusohvri altarit ja kõiki selle riistu, ja pühitse altarit, et altar saaks kõige pühamaks!
\par 11 Võia pesemisnõu ja selle jalga ja pühitse seda!
\par 12 Lase Aaron ja tema pojad astuda kogudusetelgi ukse juurde ja pese neid veega!
\par 13 Pane Aaronile selga pühad riided ja võia ning pühitse teda, ja tema olgu mulle preestriks!
\par 14 Lase ka tema pojad ette astuda ja pane neile särgid selga!
\par 15 Võia neid, nagu sa võidsid nende isa, ja nad saagu mulle preestriteks! See olgu neile igavesse preestriametisse võidmiseks põlvest põlve!”
\par 16 Ja Mooses tegi kõik. Nõnda nagu Issand teda oli käskinud, nõnda ta tegi.
\par 17 Elamu püstitati teise aasta esimese kuu esimesel päeval.
\par 18 Mooses püstitas elamu: ta seadis selle jalad, paigutas lauad, asetas põiklatid ja pani sambad püsti.
\par 19 Ta laotas elamu peale telgi ja pani selle peale telgikatte, nagu Issand Moosest oli käskinud.
\par 20 Siis ta võttis tunnistuse ja pani selle laekasse, pistis kandekangid laeka külge ja pani lepituskaane laekale peale.
\par 21 Ta viis laeka elamusse, riputas eesriide varjuks ja varjas tunnistuslaegast, nagu Issand Moosest oli käskinud.
\par 22 Ta pani laua kogudusetelki, elamu põhjapoolsesse külge, väljapoole eesriiet,
\par 23 ja seadis selle peal leivad Issanda ette korra kohaselt, nagu Issand Moosest oli käskinud.
\par 24 Ta asetas lambijala kogudusetelki, elamu lõunapoolsesse külge, lauaga kohakuti,
\par 25 ja seadis lambid üles Issanda ette, nagu Issand Moosest oli käskinud.
\par 26 Ta asetas kuldaltari kogudusetelki eesriide ette
\par 27 ja süütas selle peal healõhnalist suitsutusrohtu, nagu Issand Moosest oli käskinud.
\par 28 Ta riputas elamule uksekatte.
\par 29 Ta asetas põletusohvri altari elamu-kogudusetelgi ukse ette ja ohverdas selle peal põletus- ja roaohvreid, nagu Issand Moosest oli käskinud.
\par 30 Ta asetas pesemisnõu kogudusetelgi ja altari vahele ja pani sellesse vett pesemiseks.
\par 31 Mooses, Aaron ja tema pojad pesid selles oma käsi ja jalgu:
\par 32 nad pesid, kui nad läksid kogudusetelki või astusid altari juurde, nii nagu Issand Moosest oli käskinud.
\par 33 Elamu ja altari ümber ta püstitas õue ja riputas katte õueväravale. Nõnda lõpetas Mooses selle töö.
\par 34 Siis pilv kattis kogudusetelgi ja Issanda auhiilgus täitis elamu.
\par 35 Isegi Mooses ei võinud minna kogudusetelki, sest pilv oli laskunud selle peale ja Issanda auhiilgus täitis elamu.
\par 36 Ja kõigil oma rännakuil läksid Iisraeli lapsed teele alles siis, kui pilv elamu pealt üles tõusis.
\par 37 Aga kui pilv ei tõusnud üles, siis nad ei läinudki teele kuni selle päevani, mil see jälle üles tõusis.
\par 38 Jah, elamu peal oli päeval Issanda pilv, ja öösel oli selles tuli kogu Iisraeli soo nähes kõigil nende rännakuil.



\end{document}