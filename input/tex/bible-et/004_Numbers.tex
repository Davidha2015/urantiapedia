\begin{document}

\title{Neljas Moosese raamat}

\chapter{1}

\par 1 Ja Issand rääkis Moosesega Siinai kõrbes kogudusetelgis teise kuu esimesel päeval, teisel aastal pärast nende lahkumist Egiptusemaalt, öeldes:
\par 2 „Võta arvele kogu Iisraeli laste koguduse pead nende suguvõsade kaupa, nende perekondade kaupa nimekirja järgi kõik meesterahvad pea-pealt,
\par 3 kahekümneaastased ja üle selle, kõik, kes Iisraelist on kõlvulised minema sõtta. Sina ja Aaron lugege nad ära nende väeosade kaupa.
\par 4 Ja teiega olgu igast suguharust mees, kes on oma perekondade peamees.
\par 5 Ja need on nende meeste nimed, kes seisku teie juures: Ruubenist Elisuur, Sedeuuri poeg;
\par 6 Siimeonist Selumiel, Suurisaddai poeg;
\par 7 Juudast Nahson, Amminadabi poeg;
\par 8 Issaskarist Netaneel, Suuari poeg;
\par 9 Sebulonist Eliab, Heeloni poeg;
\par 10 Joosepi poegadest: Efraimist Elisama, Ammihudi poeg; Manassest Gamliel, Pedasuuri poeg;
\par 11 Benjaminist Abidan, Gideoni poeg;
\par 12 Daanist Ahieser, Ammisaddai poeg;
\par 13 Aaserist Pagiel, Okrani poeg;
\par 14 Gaadist Eljasaf, Deueli poeg;
\par 15 Naftalist Ahira, Eenani poeg.”
\par 16 Need olid kogudusest kutsutud, oma vanemate suguharude vürstid, Iisraeli tuhandete peamehed.
\par 17 Ja Mooses ja Aaron võtsid need mehed, kes olid nimeliselt näidatud,
\par 18 ja teise kuu esimesel päeval kogusid nad kokku terve koguduse, kes kanti sünnikirja oma suguvõsade, oma perekondade kaupa nimekirja järgi, pea-pealt, kahekümneaastased ja üle selle.
\par 19 Nõnda nagu Issand oli Moosest käskinud, nõnda ta luges nad ära Siinai kõrbes.
\par 20 Ruubeni, Iisraeli esmasündinu poegi, nende järglasi oma suguvõsade, oma perekondade kaupa nimekirja järgi pea-pealt, kõiki meesterahvaid, kahekümneaastasi ja üle selle, kõiki sõjakõlvulisi,
\par 21 neid Ruubeni suguharust loetuid oli nelikümmend kuus tuhat viissada.
\par 22 Siimeoni poegade järglasi oma suguvõsade, oma perekondade kaupa nimekirja järgi pea-pealt, kõiki meesterahvaid, kahekümneaastasi ja üle selle, kõiki sõjakõlvulisi,
\par 23 neid Siimeoni suguharust loetuid oli viiskümmend üheksa tuhat kolmsada.
\par 24 Gaadi poegade järglasi oma suguvõsade, oma perekondade kaupa nimekirja järgi, kahekümneaastasi ja üle selle, kõiki sõjakõlvulisi,
\par 25 neid Gaadi suguharust loetuid oli nelikümmend viis tuhat kuussada viiskümmend.
\par 26 Juuda poegade järglasi oma suguvõsade, oma perekondade kaupa nimekirja järgi, kahekümneaastasi ja üle selle, kõiki sõjakõlvulisi,
\par 27 neid Juuda suguharust loetuid oli seitsekümmend neli tuhat kuussada.
\par 28 Issaskari poegade järglasi oma suguvõsade, oma perekondade kaupa nimekirja järgi, kahekümneaastasi ja üle selle, kõiki sõjakõlvulisi,
\par 29 neid Issaskari suguharust loetuid oli viiskümmend neli tuhat nelisada.
\par 30 Sebuloni poegade järglasi oma suguvõsade, oma perekondade kaupa nimekirja järgi, kahekümneaastasi ja üle selle, kõiki sõjakõlvulisi,
\par 31 neid Sebuloni suguharust loetuid oli viiskümmend seitse tuhat nelisada.
\par 32 Joosepi pojad: Efraimi poegade järglasi oma suguvõsade, oma perekondade kaupa nimekirja järgi, kahekümneaastasi ja üle selle, kõiki sõjakõlvulisi,
\par 33 neid Efraimi suguharust loetuid oli nelikümmend tuhat viissada.
\par 34 Manasse poegade järglasi oma suguvõsade, oma perekondade kaupa nimekirja järgi, kahekümneaastasi ja üle selle, kõiki sõjakõlvulisi,
\par 35 neid Manasse suguharust loetuid oli kolmkümmend kaks tuhat kakssada.
\par 36 Benjamini poegade järglasi oma suguvõsade, oma perekondade kaupa nimekirja järgi, kahekümneaastasi ja üle selle, kõiki sõjakõlvulisi,
\par 37 neid Benjamini suguharust loetuid oli kolmkümmend viis tuhat nelisada.
\par 38 Daani poegade järglasi oma suguvõsade, oma perekondade kaupa nimekirja järgi, kahekümneaastasi ja üle selle, kõiki sõjakõlvulisi,
\par 39 neid Daani suguharust loetuid oli kuuskümmend kaks tuhat seitsesada.
\par 40 Aaseri poegade järglasi oma suguvõsade, oma perekondade kaupa nimekirja järgi, kahekümneaastasi ja üle selle, kõiki sõjakõlvulisi,
\par 41 neid Aaseri suguharust loetuid oli nelikümmend üks tuhat viissada.
\par 42 Naftali poegade järglasi oma suguvõsade, oma perekondade kaupa nimekirja järgi, kahekümneaastasi ja üle selle, kõiki sõjakõlvulisi,
\par 43 neid Naftali suguharust loetuid oli viiskümmend kolm tuhat nelisada.
\par 44 Need olid need loetud, keda Mooses luges koos Aaroni ja Iisraeli vürstidega, keda oli kaksteist meest, igaüks oma perekondade esindaja.
\par 45 Ja kõiki perekondade kaupa loetud Iisraeli lapsi, kahekümneaastasi ja üle selle, kõiki sõjakõlvulisi Iisraelis,
\par 46 kõiki neid loetuid oli kuussada kolm tuhat viissada viiskümmend.
\par 47 Aga leviite, oma vanemate suguharu järgi, ei loetud teiste hulka,
\par 48 sest Issand oli rääkinud Moosesega, öeldes:
\par 49 „Aga Leevi suguharu ära loe ja nende päid ära arvesta Iisraeli laste hulka,
\par 50 vaid pane leviidid tunnistuselamu üle, ja kõigi selle riistade üle, ja kõige üle, mis sellel on! Nemad kandku elamut ja kõiki selle riistu, ja nad teenigu seda ning löögu oma leer üles elamu ümber!
\par 51 Kui elamu läheb teele, siis leviidid võtku see maha, ja kui elamu jääb leeri, siis leviidid seadku see üles; aga võõras, kes sellele ligineb, surmatagu!
\par 52 Iisraeli lapsed löögu leer üles, igaüks oma leeripaika ja igaüks oma lipu juurde, rühmade kaupa.
\par 53 Aga leviidid löögu leer üles ümber tunnistuselamu, et Iisraeli laste kogudust ei tabaks viha; ja leviidid toimetagu, mida tunnistuselamus on vaja toimetada!”
\par 54 Ja Iisraeli lapsed tegid, nad tegid just nõnda, nagu Issand oli Moosest käskinud.

\chapter{2}

\par 1 Ja Issand rääkis Moosese ja Aaroniga, öeldes:
\par 2 „Iisraeli lapsed löögu leer üles, igaüks oma lipu juurde, oma perekonna märgi juurde! Nad löögu leer üles ümber kogudusetelgi, sellega kohakuti:
\par 3 esikülge, ida poole, löögu leer üles Juuda leeri lipkond oma väehulkade kaupa, Juuda poegade vürst Nahson, Amminadabi poeg,
\par 4 ja tema väehulk, milles on loetuid seitsekümmend neli tuhat kuussada.
\par 5 Tema kõrvale löögu leer üles Issaskari suguharu, Issaskari poegade vürst Netaneel, Suuari poeg,
\par 6 ja tema väehulk, milles on loetuid viiskümmend neli tuhat nelisada.
\par 7 Siis on Sebuloni suguharu, Sebuloni poegade vürst Eliab, Heeloni poeg,
\par 8 ja tema väehulk, milles on loetuid viiskümmend seitse tuhat nelisada.
\par 9 Juuda leeris on kõiki loetuid väehulkade kaupa sada kaheksakümmend kuus tuhat nelisada. Nemad lähevad teele esimestena.
\par 10 Ruubeni leeri lipkond oma väehulkade kaupa olgu lõuna pool, Ruubeni poegade vürst Elisuur, Sedeuuri poeg,
\par 11 ja tema väehulk, milles on loetuid nelikümmend kuus tuhat viissada.
\par 12 Tema kõrvale löögu leer üles Siimeoni suguharu, Siimeoni poegade vürst Selumiel, Suurisaddai poeg,
\par 13 ja tema väehulk, milles on loetuid viiskümmend üheksa tuhat kolmsada.
\par 14 Siis on Gaadi suguharu, Gaadi poegade vürst Eljasaf, Deueli poeg,
\par 15 ja tema väehulk, milles on loetuid nelikümmend viis tuhat kuussada viiskümmend.
\par 16 Ruubeni leeris on kõiki loetuid väehulkade kaupa sada viiskümmend üks tuhat nelisada viiskümmend. Nemad lähevad teele järgmistena.
\par 17 Seejärel mingu teele kogudusetelk koos leviitide leeriga teiste leeride keskel! Nõnda nagu nad olid leeri üles löönud, nõnda nad mingu ka teele, igaüks oma kohal, oma lipkondade kaupa!
\par 18 Lääne pool olgu Efraimi leeri lipkond oma väehulkade kaupa, Efraimi poegade vürst Elisama, Ammihudi poeg,
\par 19 ja tema väehulk, milles on loetuid nelikümmend tuhat viissada.
\par 20 Tema kõrval olgu Manasse suguharu, Manasse poegade vürst Gamliel, Pedasuuri poeg,
\par 21 ja tema väehulk, milles on loetuid kolmkümmend kaks tuhat kakssada.
\par 22 Siis on Benjamini suguharu, Benjamini poegade vürst Abidan, Gideoni poeg,
\par 23 ja tema väehulk, milles on loetuid kolmkümmend viis tuhat nelisada.
\par 24 Efraimi leeris on kõiki loetuid väehulkade kaupa sada kaheksa tuhat ükssada. Nemad lähevad teele kolmandatena.
\par 25 Põhja pool olgu Daani leeri lipkond oma väehulkade kaupa, Daani poegade vürst Ahieser, Ammisaddai poeg,
\par 26 ja tema väehulk, milles on loetuid kuuskümmend kaks tuhat seitsesada.
\par 27 Tema kõrvale löögu oma leer üles Aaseri suguharu, Aaseri poegade vürst Pagiel, Okrani poeg,
\par 28 ja tema väehulk, milles on loetuid nelikümmend üks tuhat viissada.
\par 29 Siis on Naftali suguharu, Naftali poegade vürst Ahira, Eenani poeg,
\par 30 ja tema väehulk, milles on loetuid viiskümmend kolm tuhat nelisada.
\par 31 Daani leeris on kõiki loetuid sada viiskümmend seitse tuhat kuussada. Nemad oma lipkondade kaupa lähevad teele viimastena.”
\par 32 Need olid oma perekondade kaupa loetud Iisraeli lapsed; kõiki leerides väehulkade kaupa loetuid oli kuussada kolm tuhat viissada viiskümmend.
\par 33 Leviite aga ei loetud Iisraeli laste hulka, nõnda nagu Issand oli Moosest keelanud.
\par 34 Ja Iisraeli lapsed tegid kõik nõnda, nagu Issand oli Moosest käskinud: nõnda nad lõid leeri üles oma lipkondade kaupa ja nõnda nad läksid teele, igaüks koos oma suguvõsaga perekondade kaupa.

\chapter{3}

\par 1 Need on Aaroni ja Moosese järglased ajast, mil Issand rääkis Moosesega Siinai mäel.
\par 2 Need on Aaroni poegade nimed: Naadab, esmasündinu, Abihu, Eleasar ja Iitamar;
\par 3 need on Aaroni poegade, võitud preestrite nimed, kelle käsi oli täidetud preestriametiks.
\par 4 Aga Naadab ja Abihu surid Issanda ees, kui nad Siinai kõrbes viisid Issanda ette võõra tule, ja neil ei olnud lapsi. Nii olid Eleasar ja Iitamar preestriametis oma isa Aaroni palge ees.
\par 5 Ja Issand rääkis Moosesega, öeldes:
\par 6 „Too Leevi suguharu ja sea see preester Aaroni ette teda teenima!
\par 7 Nemad teostagu, mis on tema ülesandeks ja mida terve kogudusetelgi ees olev kogudus peab tegema: toimetagu elamu teenistust!
\par 8 Nemad hooldagu kõiki kogudusetelgi riistu ja Iisraeli laste kohustusi, toimetades elamu teenistust!
\par 9 Anna leviidid Aaronile ja tema poegadele; nad on Iisraeli laste seast antud täielikult temale.
\par 10 Aga käsi Aaronit ja tema poegi, et nad peaksid oma preestriametit! Võõras, kes ligineb, surmatagu!”
\par 11 Ja Issand rääkis Moosesega, öeldes:
\par 12 „Mina, vaata, olen võtnud leviidid Iisraeli laste seast kõigi Iisraeli laste esmasündinute, emakoja avajate asemel. Leviidid olgu minu päralt!
\par 13 Sest minu päralt on iga esmasündinu: päeval, mil ma lõin maha kõik esmasündinud Egiptusemaal, pühitsesin ma enesele kõik esmasündinud Iisraelis, niihästi inimesed kui loomad. Nad on minu päralt. Mina olen Issand!”
\par 14 Ja Issand rääkis Moosesega Siinai kõrbes, öeldes:
\par 15 „Loe ära Leevi pojad nende perekondade kaupa, nende suguvõsade kaupa; loe ära kõik meesterahvad, ühe kuu vanused ja üle selle!”
\par 16 Ja Mooses luges nad ära Issanda sõna peale, nagu teda kästi.
\par 17 Ja need olid Leevi pojad nimeliselt: Geerson, Kehat ja Merari.
\par 18 Ja need olid Geersoni poegade nimed oma suguvõsade järgi: Libni ja Simei.
\par 19 Ja Kehati pojad oma suguvõsade järgi olid Amram, Jishar, Hebron ja Ussiel.
\par 20 Ja Merari pojad oma suguvõsade järgi olid Mahli ja Muusi. Need olid Leevi suguvõsad oma perekondade kaupa.
\par 21 Geersonist põlvnevad libnilaste suguvõsa ja simeilaste suguvõsa. Need olid geersonlaste suguvõsad.
\par 22 Neid, keda neist loeti, kõiki meesterahvaid nimekirja järgi, ühe kuu vanuseid ja üle selle, neid, keda neist loeti, oli seitse tuhat viissada.
\par 23 Geersonlaste suguvõsad lõid leeri üles elamu taha, lääne poole.
\par 24 Ja geersonlaste isakoja vürst oli Eljasaf, Laeli poeg.
\par 25 Geersoni poegade hooleks oli ühenduses kogudusetelgiga elamu ja telk, selle kate ja kogudusetelgi ukse eesriie,
\par 26 õue seinavaibad ning elamu ja altari ümber oleva õue ukse kate ja selle nöörid - kõik sellega ühenduses olev töö.
\par 27 Kehatist põlvnevad amramlaste suguvõsa, jisharlaste suguvõsa, hebronlaste suguvõsa ja ussiellaste suguvõsa. Need olid kehatlaste suguvõsad.
\par 28 Kõiki meesterahvaid nimekirja järgi, ühe kuu vanuseid ja üle selle, oli kaheksa tuhat kuussada, kes pidid toimetama pühamu teenistust.
\par 29 Kehatlaste suguvõsad lõid leeri üles elamu kõrvale, lõuna poole.
\par 30 Ja kehatlaste suguvõsade isakoja vürst oli Elisafan, Ussieli poeg.
\par 31 Nende hooleks olid laegas, laud, lambijalg, altarid ja pühad riistad, millega peetakse teenistust, eesriie ja kõik sellega ühenduses olev töö.
\par 32 Ja leviitide vürstide vürst oli Eleasar, preester Aaroni poeg, nende järelevaataja, kelle hooleks oli pühamu teenistus.
\par 33 Merarist põlvnevad mahlilaste suguvõsa ja muusilaste suguvõsa. Need olid Merari suguvõsad.
\par 34 Neid, keda neist loeti, kõiki meesterahvaid nimekirja järgi, ühe kuu vanuseid ja üle selle, oli kuus tuhat kakssada.
\par 35 Ja Merari suguvõsade isakoja vürst oli Suuriel, Abihaili poeg. Nemad lõid leeri üles elamu kõrvale, põhja poole.
\par 36 Merari poegade kohustuseks oli hoolekanne elamu laudade, põiklattide, sammaste, jalgade ja kõigi riistade eest, ja kõik sellega ühenduses olev töö,
\par 37 samuti hoolekanne õue ümber olevate postide, nende jalgade, vaiade ja nööride eest.
\par 38 Aga elamu ette, kogudusetelgi ette idapoolsesse esikülge, lõid leeri üles Mooses ja Aaron ning tema pojad, kes hoolitsesid pühamu talituste eest, mida Iisraeli lastel tuli toimetada. Lähenev võõras aga tuli surmata.
\par 39 Kõiki loetud leviite, keda Mooses luges Issanda käsul nende suguvõsade kaupa, kõiki meesterahvaid, ühe kuu vanuseid ja üle selle, oli kakskümmend kaks tuhat.
\par 40 Ja Issand ütles Moosesele: „Loe ära kõik Iisraeli laste meessoost esmasündinud, ühe kuu vanused ja üle selle, ja tee kindlaks nende nimede arv!
\par 41 Võta siis minule leviidid - mina olen Issand - kõigi Iisraeli laste esmasündinute asemel, ja leviitide loomad, kõigi Iisraeli laste lojuste esmasündinute asemel!”
\par 42 Ja Mooses luges ära kõik Iisraeli laste esmasündinud, nõnda nagu Issand teda käskis.
\par 43 Kõiki meessoost esmasündinuid nimekirja järgi, ühe kuu vanuseid ja üle selle, neid, keda loeti, oli kakskümmend kaks tuhat kakssada seitsekümmend kolm.
\par 44 Ja Issand rääkis Moosesega, öeldes:
\par 45 „Võta leviidid kõigi Iisraeli laste esmasündinute asemel ja leviitide loomad nende loomade asemel, ja leviidid kuulugu minule! Mina olen Issand!
\par 46 Aga nende kahesaja seitsmekümne kolme lunastamiseks, kelle võrra Iisraeli laste esmasündinuid on rohkem kui leviite,
\par 47 võta viis seeklit iga pea kohta; võta seda püha seekli järgi, kakskümmend geera seeklis.
\par 48 Anna see raha Aaronile ja tema poegadele lunahinnaks nende eest, kes on üleliigsed!”
\par 49 Ja Mooses võttis lunaraha neilt, kes pärast leviitidega lunastamist olid üle jäänud.
\par 50 Ta võttis selle raha, tuhat kolmsada kuuskümmend viis seeklit püha seekli järgi, Iisraeli laste esmasündinuilt.
\par 51 Ja Mooses andis selle lunaraha Aaronile ja tema poegadele, Issanda sõna peale, nagu Issand oli Moosest käskinud.

\chapter{4}

\par 1 Ja Issand rääkis Moosese ja Aaroniga, öeldes:
\par 2 „Loe ära Kehati pojad Leevi poegade hulgast nende suguvõsade, nende perekondade kaupa,
\par 3 kolmekümneaastased ja üle selle kuni viiekümneaastasteni, kõik teenistuskõlvulised kogudusetelgi teenistuse toimetamiseks.
\par 4 See on Kehati poegade teenistus kogudusetelgis: hoolekanne kõige pühamate asjade eest.
\par 5 Leeri teele asudes mingu Aaron ja ta pojad ja võtku maha katte-eesriie ja katku sellega tunnistuslaegas!
\par 6 Siis nad pangu selle peale merilehmanahast kate, laotagu selle peale tervenisti sinine riie ja asetagu paika selle kandekangid!
\par 7 Ja nad laotagu ohvrileibade lauale sinine riie ja pangu selle peale vaagnad ja kausid, kannud ja joogiohvri peekrid; ka alaline leib olgu selle peal!
\par 8 Siis nad laotagu nende peale helepunane riie, katku see merilehmanahast kattega ja asetagu paika selle kandekangid!
\par 9 Ja nad võtku sinine riie ja katku valgustuslambi jalg, selle lambid, tahikäärid, tahikarbid ja kõik selle õlinõud, millega teenistust toimetatakse!
\par 10 Nad pangu see ja kõik selle riistad merilehmanahast kattesse ning asetagu kanderaamile!
\par 11 Ja kuldaltarile nad laotagu sinine riie, katku see merilehmanahast kattega ja asetagu paika selle kandekangid!
\par 12 Ja nad võtku kõik teenistusriistad, millega nad pühamus teenistust toimetavad, pangu need sinisesse riidesse, katku need merilehmanahast kattega ja asetagu kanderaamile!
\par 13 Nad koristagu tuhk altarilt ja laotagu sellele purpurpunane riie!
\par 14 Siis nad pangu selle peale kõik riistad, millega nad seal teenistust toimetavad: sütepannid, kahvlid, labidad ja piserdusnõud, kõik altari riistad, laotagu nende peale kate merilehmanahast ja asetagu paika selle kandekangid!
\par 15 Kui leeri teele asudes Aaron ja tema pojad on kinni katnud pühamu ja kõik pühamu riistad, siis Kehati pojad tulgu kandma, aga nad ei tohi puudutada pühi asju, et nad ei sureks! Need on Kehati poegade kandamiks kogudusetelgist.
\par 16 Ja preester Aaroni poja Eleasari hooldada olgu valgustusõli, healõhnalised suitsutusrohud, alaline roaohver ja võideõli, hooldus kogu elamu üle ja kõige üle, mis selles on, pühamus ja selle riistades.”
\par 17 Ja Issand rääkis Moosese ja Aaroniga, öeldes:
\par 18 „Ärge laske hävida kehatlaste suguvõsade tüve leviitide seast!
\par 19 Seepärast toimige nendega nõnda, et nad jääksid elama ega sureks, kui nad liginevad kõige pühamatele asjadele: Aaron ja ta pojad mingu ja pangu neist igaüks oma teenistusse ja oma kandami juurde,
\par 20 aga nemad ise ei tohi minna vaatama pühi asju mitte silmapilgukski, et nad ei sureks!”
\par 21 Ja Issand rääkis Moosesega, öeldes:
\par 22 „Loe ära ka Geersoni pojad nende perekondade, nende suguvõsade kaupa,
\par 23 kolmekümneaastased ja üle selle kuni viiekümneaastasteni! Loe ära kõik teenistuskõlvulised teenistuse toimetamiseks kogudusetelgis!
\par 24 See olgu geersonlaste suguvõsade töö, mida nad peavad tegema ja mida kandma:
\par 25 nemad kandku elamu vaipu ja kogudusetelki, selle katet ja merilehmanahast katet, mis kui pealis on selle peal, ja kogudusetelgi ukse katet,
\par 26 ja õue vaipu ja elamu ja altari ümber oleva õue väravaavause katet ja nende nööre ja kõiki nende juurde kuuluvaid tööriistu; nad tehku kõike, mis neil seejuures tuleb teha!
\par 27 Aaroni ja tema poegade käsu järgi sündigu kogu geersonlaste poegade teenistus kõiges, mis neil on kanda, ja kõiges, mis neil on toimetada; andke nende hooleks kogu nende kandam!
\par 28 See on geersonlaste poegade suguvõsade teenistus kogudusetelgis; mis neil tuleb teha, sündigu Iitamari, preester Aaroni poja käe all.
\par 29 Merari pojad - loe nad ära nende suguvõsade, nende perekondade kaupa,
\par 30 kolmekümneaastased ja üle selle kuni viiekümneaastasteni; neist loe ära kõik teenistuskõlvulised kogudusetelgi teenistuse toimetamiseks!
\par 31 Ja kogu nende teenistuseks kogudusetelgis on nende kohustus kanda elamu laudu, latte, sambaid ja jalgu,
\par 32 ja ümberringi oleva õue sambaid, nende jalgu, vaiu ja nööre koos kõigi nende riistadega ja kogu selleks vajaliku tööga; nimetage nimeliselt riistad, mida nad on kohustatud kandma.
\par 33 See on Merari poegade suguvõsade teenistus, kogu nende teenistus kogudusetelgis Iitamari, preester Aaroni poja käe all.”
\par 34 Ja Mooses ja Aaron ja koguduse vürstid lugesid ära kehatlaste pojad nende suguvõsade ja nende perekondade kaupa,
\par 35 kolmekümneaastased ja üle selle kuni viiekümneaastasteni, kõik teenistuskõlvulised kogudusetelgi teenistuseks.
\par 36 Ja neid, nende suguvõsade kaupa loetuid, oli kaks tuhat seitsesada viiskümmend.
\par 37 Need olid need, kes ära loeti kehatlaste suguvõsadest, kõik kogudusetelgi teenistuses olijad, keda Mooses ja Aaron lugesid Moosese läbi antud Issanda käsul.
\par 38 Ja loetud Geersoni poegi nende suguvõsade ja perekondade kaupa,
\par 39 kolmekümneaastasi ja üle selle kuni viiekümneaastasteni, kõiki teenistuskõlvulisi kogudusetelgi teenistuseks,
\par 40 neid nende suguvõsade ja perekondade kaupa loetuid oli kaks tuhat kuussada kolmkümmend.
\par 41 Need olid need, kes ära loeti Geersoni poegade suguvõsadest, kõik kogudusetelgi teenistuses olijad, keda Mooses ja Aaron lugesid Issanda käsul.
\par 42 Ja loetuid Merari poegade suguvõsadest, nende suguvõsade ja perekondade kaupa,
\par 43 kolmekümneaastasi ja üle selle kuni viiekümneaastasteni, kõiki teenistuskõlvulisi kogudusetelgi teenistuseks,
\par 44 neid nende suguvõsade kaupa loetuid oli kolm tuhat kakssada.
\par 45 Need olid need, kes ära loeti Merari poegade suguvõsadest, keda Mooses ja Aaron lugesid Moosese läbi antud Issanda käsul.
\par 46 Kõiki loetuid, keda Mooses ja Aaron ja Iisraeli vürstid lugesid leviitidest nende suguvõsade ja perekondade kaupa,
\par 47 kolmekümneaastasi ja üle selle kuni viiekümneaastasteni, kõiki töökõlvulisi teenistuse toimetamiseks ja kandetööks kogudusetelgis,
\par 48 neid loetuid oli kaheksa tuhat viissada kaheksakümmend.
\par 49 Moosese läbi antud Issanda käsul loeti nad ära mees-mehelt, igaüks oma teenistuse või kandami jaoks. Nad loeti ära, nagu Issand oli Moosest käskinud.

\chapter{5}

\par 1 Ja Issand rääkis Moosesega, öeldes:
\par 2 „Käsi Iisraeli lapsi, et nad saadaksid leerist ära kõik pidalitõbised ja kõik, kellel on voolus, ja kõik, kes on roojased surnu pärast!
\par 3 Saatke ära niihästi meesterahvas kui naisterahvas; saatke nad väljapoole leeri, et nad ei roojastaks oma leeri, kus mina elan nende keskel!”
\par 4 Ja Iisraeli lapsed tegid nõnda ja saatsid need väljapoole leeri; nõnda nagu Issand oli Moosest käskinud, nõnda Iisraeli lapsed tegid.
\par 5 Ja Issand rääkis Moosesega, öeldes:
\par 6 „Ütle Iisraeli lastele: Kui mees või naine teeb mõne inimliku patu ega ole ustav Issandale, ja see hing saab süüdlaseks,
\par 7 siis ta peab tunnistama oma pattu, mis ta on teinud, ja ta hüvitagu oma eksimus täies ulatuses ning lisagu sellele viies osa ja andku sellele, kelle vastu ta eksis!
\par 8 Aga kui sel mehel ei ole pärijat, kelle kätte võlga tasuda, siis kuulub tasutav võlg Issandale, see tähendab preestrile, lisaks lepitusjäärale, kellega ta toimetab tema eest lepitust.
\par 9 Ka iga tõstelõiv kõigist Iisraeli laste pühadest andidest, mida nad toovad preestrile, on preestri oma.
\par 10 Iga mehe pühitsetud annid on preestri omad; mis keegi annab preestrile, on preestri oma.”
\par 11 Ja Issand rääkis Moosesega, öeldes:
\par 12 „Räägi Iisraeli lastega ja ütle neile: Kui mõne mehe naine rikub abielu ja on tema vastu truuduseta,
\par 13 ning keegi magab sugutades tema juures, mis jääb varjatuks tema mehe silma eest, ilma et naist avastataks, kuigi ta on ennast roojastanud, aga tema vastu ei ole tunnistajat ja teda ei ole tabatud,
\par 14 meest aga valdab armukadeduse vaim ja ta armukadestab oma naist, kes ennast on roojastanud, või valdab teda armukadeduse vaim ja ta armukadestab oma naist, kuigi see ei ole ennast roojastanud,
\par 15 siis viigu mees oma naine preestri juurde ja viigu tema eest ohvrianniks üks kann odrajahu; aga ta ärgu valagu selle peale õli ja ärgu pangu selle peale viirukit, sest see on armukadeduse-roaohver, meenutus-roaohver, mis meenutab süüd.
\par 16 Ja preester toogu naine ja pangu ta seisma Issanda ette!
\par 17 Ja preester võtku pühitsetud vett saviastjasse ja preester võtku elamu põrandalt põrmu ning pangu vette!
\par 18 Ja preester pangu naine seisma Issanda ette, päästku valla naise juuksed ning andku ta käte peale meenutus-roaohver, see on armukadeduse-roaohver; ja preestri käes olgu kibe needevesi!
\par 19 Ja preester vannutagu naist ning öelgu temale: Kui keegi ei ole maganud su juures ja kui sa ei ole ennast roojastades hüljanud oma meest, siis ärgu sul olgu mingit viga sellest needeveest!
\par 20 Aga kui sa oled hüljanud oma mehe, ja kui sa oled ennast roojastanud ning keegi teine mees, mitte su oma mees, on sind magatanud -
\par 21 siis preester vannutagu seda naist sajatusevandega ja preester öelgu naisele: Issand pangu sind sajatuseks ja vandumiseks su rahva sekka! Issand lasku su niuded kiduda ja su kõht paisuda!
\par 22 Mingu see needevesi su sisikonda, et su kõht paisuks ja niuded kiduksid! Ja naine öelgu: Aamen, aamen!
\par 23 Ja preester kirjutagu need sajatused raamatusse ning uhtku seda kibeda veega!
\par 24 Siis ta andku naisele juua kibedat needevett ja see needevesi mingu temasse kibedaks piinaks!
\par 25 Ja preester võtku naise käest armukadeduse-roaohver ja kõigutagu roaohvrit Issanda ees ning viigu see altarile!
\par 26 Ja preester võtku roaohvrist peotäis selle lõhnavaks ohvriosaks ning süüdaku altaril põlema; ja seejärel andku ta naisele vett juua!
\par 27 Ja kui ta on andnud temale vett juua, siis sünnib temaga nõnda: kui ta ennast on roojastanud ja on hüljanud oma mehe, siis läheb needevesi temasse kibedaks piinaks ning ta kõht paisub ja niuded kiduvad, ja naine saab sajatuseks oma rahva seas.
\par 28 Aga kui naine ei ole ennast roojastanud, vaid on puhas, siis ta on süüta ja võib sigitada ihuvilja.
\par 29 See on seadus armukadeduse puhul. Kui naine on hüljanud oma mehe ja on ennast roojastanud,
\par 30 või kui meest valdab armukadeduse vaim ja ta armukadestab oma naist, siis seadku ta naine Issanda ette ja preester talitagu naisega täiesti selle seaduse järgi!
\par 31 Mees olgu vaba süüst, aga niisugune naine kandku oma süüd!”

\chapter{6}

\par 1 Ja Issand rääkis Moosesega, öeldes:
\par 2 „Räägi Iisraeli lastega ja ütle neile: Kui mees või naine tõotab erilise nasiiritõotuse, et Issandale pühenduda,
\par 3 siis ta peab hoiduma veinist ja vägijoogist; ta ei tohi juua veini- ega vägijoogi äädikat; ta ärgu joogu ka mitte mingisugust viinamarjamahla ja ärgu söögu värskeid või kuivatatud viinamarju!
\par 4 Kõigil oma nasiiripõlve päevil ei tohi ta midagi süüa, mida toodetakse viinapuust, isegi mitte kivikesi ega kesti.
\par 5 Kõigil tema tõotatud nasiiripõlve päevil ei tohi habemenuga ta peast üle käia; seni kui aeg täis saab, milleks ta ennast Issandale on pühendanud, on ta püha: ta peab laskma juuksed peas pikaks kasvada.
\par 6 Kõigil oma Issandale pühenduse päevil ei tohi ta minna surnukeha juurde.
\par 7 Ta ei tohi ennast roojastada oma isa ja ema, venna ja õe pärast, kui need surevad, sest ta pea peal on pühitsus Jumalale.
\par 8 Kõigil oma nasiiripõlve päevil on ta Issandale pühitsetud!
\par 9 Ja kui keegi tema juures väga äkitselt sureb ja ta nõnda roojastab oma nasiiripea, siis ta peab puhastuspäeval oma pea paljaks ajama; ta ajagu see paljaks seitsmendal päeval!
\par 10 Ja kaheksandal päeval viigu ta kaks turteltuvi või kaks muud tuvi preestrile kogudusetelgi ukse juurde!
\par 11 Ja preester valmistagu üks patuohvriks ja teine põletusohvriks ning toimetagu temale lepitust, sellepärast et ta surnu tõttu pattu tegi; selsamal päeval ta pühitsegu oma pead,
\par 12 eraldagu taas Issandale oma nasiiripõlve päevad ja viigu aastane oinastall süüohvriks! Endised päevad aga langevad ära, sest ta nasiiripõli rüvetus.
\par 13 Ja niisugune on nasiiriseadus: päeval, mil tema nasiiriaeg täitub, viidagu ta kogudusetelgi ukse juurde,
\par 14 ja ta toogu oma ohvriannina Issandale üks veatu aastane oinastall põletusohvriks ja üks veatu aastane utetall patuohvriks ja üks veatu jäär tänuohvriks;
\par 15 ja korv hapnemata leiba, peenest jahust õliga segatud kooke ja õliga võitud hapnemata koogikesi; ja nende juurde kuuluv roaohver ning joogiohvrid!
\par 16 Ja preester viigu need Issanda ette ning toimetagu tema patu- ja põletusohver;
\par 17 jäär aga ohverdagu ta Issandale tänuohvriks koos korvitäie hapnemata leibadega; preester ohverdagu ka tema roa- ja joogiohver!
\par 18 Siis nasiir pügagu oma nasiiripõlve pea paljaks kogudusetelgi ees ja võtku oma nasiiripea juuksed ning pangu tänuohvri all olevasse tulle!
\par 19 Ja preester võtku keedetud jäärasaps, üks hapnemata kook korvist ja üks hapnemata koogike ning pangu nasiiri kätele, pärast seda kui see oma nasiirijuuksed on püganud.
\par 20 Ja preester kõigutagu neid Issanda ees kõigutusohvrina; see olgu preestrile pühitsetud koos kõigutusrinna ja tõstesapsuga; pärast seda võib nasiir veini juua.
\par 21 See on seadus nasiiri kohta, kes tõotab oma ohvrianni Issandale nasiiripõlve pärast lisaks sellele, mis ta jõud muidu lubab. Vastavalt tõotusele, mis ta tõotab, peab ta toimima oma nasiiriseaduse järgi.”
\par 22 Ja Issand rääkis Moosesega, öeldes:
\par 23 „Räägi Aaroni ja ta poegadega ning ütle: Õnnistades Iisraeli lapsi, öelge neile nõnda:
\par 24 Issand õnnistagu sind ja hoidku sind!
\par 25 Issand lasku oma pale paista sinu peale ja olgu sulle armuline!
\par 26 Issand tõstku oma pale sinu üle ja andku sulle rahu!
\par 27 Nõnda pandagu minu nimi Iisraeli laste peale ja mina õnnistan neid!”

\chapter{7}

\par 1 Sel päeval, mil Mooses oli lõpetanud elamu püstitamise ja oli seda võidnud ning pühitsenud, ja oli võidnud ning pühitsenud ka kõiki selle riistu ja altarit ja kõiki selle riistu,
\par 2 ohverdasid Iisraeli vürstid, nende perekondade pead; need olid suguharude vürstid, need, kes seisid äraloetute eesotsas.
\par 3 Nad tõid oma ohvriannina Issanda ette kuus kaetud vankrit ja kaksteist härga, vanker kahe ja härg ühe vürsti kohta; nad tõid need elamu ette.
\par 4 Ja Issand rääkis Moosesega, öeldes:
\par 5 „Võta neilt need ja need olgu kogudusetelgi teenistuse tarvis; anna need leviitidele, igaühele vastavalt ta teenistusele!”
\par 6 Ja Mooses võttis vankrid ja härjad ning andis need leviitidele.
\par 7 Kaks vankrit ja neli härga andis ta Geersoni poegadele, vastavalt nende teenistusele.
\par 8 Neli vankrit ja kaheksa härga andis ta Merari poegadele, vastavalt nende teenistusele preester Aaroni poja Iitamari käe all.
\par 9 Aga Kehati poegadele ei andnud ta ühtegi, sest nende hooleks olid pühad asjad, mida nad pidid õlal kandma.
\par 10 Ja vürstid tõid altari pühitsusanni sel päeval, mil seda võiti; ja vürstid viisid oma ohvrianni altari ette.
\par 11 Ja Issand ütles Moosesele: „Iga vürst viigu eri päeval oma ohvriand altari pühitsusanniks!”
\par 12 Siis tõi esimesel päeval oma ohvrianni Nahson, Amminadabi poeg Juuda suguharust.
\par 13 Tema ohvrianniks oli üks hõbevaagen, saja kolmekümne seekline, üks hõbekauss, seitsmekümneseekline püha seekli järgi, mõlemad täis õliga segatud peent jahu roaohvriks,
\par 14 üks nõu, kümne-kuldseekline, täis suitsutusrohtu,
\par 15 üks noor härjavärss, üks jäär, üks aastane oinastall põletusohvriks,
\par 16 üks noor sikk patuohvriks,
\par 17 ja tänuohvriks kaks härga, viis jäära, viis sikku, viis aastast oinastalle. See oli Nahsoni, Amminadabi poja ohvriand.
\par 18 Teisel päeval tõi Netaneel, Suuari poeg, Issaskari vürst.
\par 19 Tema tõi oma ohvriannina ühe hõbevaagna, saja kolmekümne seeklise, ühe hõbekausi, seitsmekümneseeklise püha seekli järgi, mõlemad täis õliga segatud peent jahu roaohvriks,
\par 20 ühe nõu, kümne-kuldseeklise, täis suitsutusrohtu,
\par 21 ühe noore härjavärsi, ühe jäära, ühe aastase oinastalle põletusohvriks,
\par 22 ühe noore siku patuohvriks,
\par 23 ja tänuohvriks kaks härga, viis jäära, viis sikku, viis aastast oinastalle. See oli Netaneeli, Suuari poja ohvriand.
\par 24 Kolmandal päeval Sebuloni poegade vürst Eliab, Heeloni poeg.
\par 25 Tema ohvrianniks oli üks hõbevaagen, saja kolmekümne seekline, üks hõbekauss, seitsmekümneseekline püha seekli järgi, mõlemad täis õliga segatud peent jahu roaohvriks,
\par 26 üks nõu, kümne-kuldseekline, täis suitsutusrohtu,
\par 27 üks noor härjavärss, üks jäär, üks aastane oinastall põletusohvriks,
\par 28 üks noor sikk patuohvriks,
\par 29 ja tänuohvriks kaks härga, viis jäära, viis sikku, viis aastast oinastalle. See oli Eliabi, Heeloni poja ohvriand.
\par 30 Neljandal päeval Ruubeni poegade vürst Elisuur, Sedeuuri poeg.
\par 31 Tema ohvrianniks oli üks hõbevaagen, saja kolmekümne seekline, üks hõbekauss, seitsmekümneseekline püha seekli järgi, mõlemad täis õliga segatud peent jahu roaohvriks,
\par 32 üks nõu, kümne-kuldseekline, täis suitsutusrohtu,
\par 33 üks noor härjavärss, üks jäär, üks aastane oinastall põletusohvriks,
\par 34 üks noor sikk patuohvriks,
\par 35 ja tänuohvriks kaks härga, viis jäära, viis sikku, viis aastast oinastalle. See oli Elisuuri, Sedeuuri poja ohvriand.
\par 36 Viiendal päeval Siimeoni poegade vürst Selumiel Suurisaddai poeg.
\par 37 Tema ohvrianniks oli üks hõbevaagen, saja kolmekümne seekline, üks hõbekauss, seitsmekümneseekline püha seekli järgi, mõlemad täis õliga segatud peent jahu roaohvriks,
\par 38 üks nõu, kümne-kuldseekline, täis suitsutusrohtu,
\par 39 üks noor härjavärss, üks jäär, üks aastane oinastall põletusohvriks,
\par 40 üks noor sikk patuohvriks,
\par 41 ja tänuohvriks kaks härga, viis jäära, viis sikku, viis aastast oinastalle. See oli Selumieli, Suurisaddai poja ohvriand.
\par 42 Kuuendal päeval Gaadi poegade vürst Eljasaf, Deueli poeg.
\par 43 Tema ohvrianniks oli üks hõbevaagen, saja kolmekümne seekline, üks hõbekauss, seitsmekümneseekline püha seekli järgi, mõlemad täis õliga segatud peent jahu roaohvriks,
\par 44 üks nõu, kümne-kuldseekline, täis suitsutusrohtu,
\par 45 üks noor härjavärss, üks jäär, üks aastane oinastall põletusohvriks,
\par 46 üks noor sikk patuohvriks,
\par 47 ja tänuohvriks kaks härga, viis jäära, viis sikku, viis aastast oinastalle. See oli Eljasafi, Deueli poja ohvriand.
\par 48 Seitsmendal päeval Efraimi poegade vürst Elisama, Ammihudi poeg.
\par 49 Tema ohvrianniks oli üks hõbevaagen, saja kolmekümne seekline, üks hõbekauss, seitsmekümneseekline püha seekli järgi, mõlemad täis õliga segatud peent jahu roaohvriks,
\par 50 üks nõu, kümne-kuldseekline, täis suitsutusrohtu,
\par 51 üks noor härjavärss, üks jäär, üks aastane oinastall põletusohvriks,
\par 52 üks noor sikk patuohvriks,
\par 53 ja tänuohvriks kaks härga, viis jäära, viis sikku, viis aastast oinastalle. See oli Elisama, Ammihudi poja ohvriand.
\par 54 Kaheksandal päeval Manasse poegade vürst Gamliel, Pedasuuri poeg.
\par 55 Tema ohvrianniks oli üks hõbevaagen, saja kolmekümne seekline, üks hõbekauss, seitsmekümneseekline püha seekli järgi, mõlemad täis õliga segatud peent jahu roaohvriks,
\par 56 üks nõu, kümne-kuldseekline, täis suitsutusrohtu,
\par 57 üks noor härjavärss, üks jäär, üks aastane oinastall põletusohvriks,
\par 58 üks noor sikk patuohvriks,
\par 59 ja tänuohvriks kaks härga, viis jäära, viis sikku, viis aastast oinastalle. See oli Gamlieli, Pedasuuri poja ohvriand.
\par 60 Üheksandal päeval Benjamini poegade vürst Abidan, Gideoni poeg.
\par 61 Tema ohvrianniks oli üks hõbevaagen, saja kolmekümne seekline, üks hõbekauss, seitsmekümneseekline püha seekli järgi, mõlemad täis õliga segatud peent jahu roaohvriks,
\par 62 üks nõu, kümne-kuldseekline, täis suitsutusrohtu,
\par 63 üks noor härjavärss, üks jäär, üks aastane oinastall põletusohvriks,
\par 64 üks noor sikk patuohvriks,
\par 65 ja tänuohvriks kaks härga, viis jäära, viis sikku, viis aastast oinastalle. See oli Abidani, Gideoni poja ohvriand.
\par 66 Kümnendal päeval Daani poegade vürst Ahieser, Ammisaddai poeg.
\par 67 Tema ohvrianniks oli üks hõbevaagen, saja kolmekümne seekline, üks hõbekauss, seitsmekümneseekline püha seekli järgi, mõlemad täis õliga segatud peent jahu roaohvriks,
\par 68 üks nõu, kümne-kuldseekline, täis suitsutusrohtu,
\par 69 üks noor härjavärss, üks jäär, üks aastane oinastall põletusohvriks,
\par 70 üks noor sikk patuohvriks,
\par 71 ja tänuohvriks kaks härga, viis jäära, viis sikku, viis aastast oinastalle. See oli Ahieseri, Ammisaddai poja ohvriand.
\par 72 Üheteistkümnendal päeval Aaseri poegade vürst Pagiel, Okrani poeg.
\par 73 Tema ohvrianniks oli üks hõbevaagen, saja kolmekümne seekline, üks hõbekauss, seitsmekümneseekline püha seekli järgi, mõlemad täis õliga segatud peent jahu roaohvriks,
\par 74 üks nõu, kümne-kuldseekline, täis suitsutusrohtu,
\par 75 üks noor härjavärss, üks jäär, üks aastane oinastall põletusohvriks,
\par 76 üks noor sikk patuohvriks,
\par 77 ja tänuohvriks kaks härga, viis jäära, viis sikku, viis aastast oinastalle. See oli Pagieli, Okrani poja ohvriand.
\par 78 Kaheteistkümnendal päeval Naftali poegade vürst Ahira, Eenani poeg.
\par 79 Tema ohvrianniks oli üks hõbevaagen, saja kolmekümne seekline, üks hõbekauss, seitsmekümneseekline püha seekli järgi, mõlemad täis õliga segatud peent jahu roaohvriks,
\par 80 üks nõu, kümne-kuldseekline, täis suitsutusrohtu,
\par 81 üks noor härjavärss, üks jäär, üks aastane oinastall põletusohvriks,
\par 82 üks noor sikk patuohvriks,
\par 83 ja tänuohvriks kaks härga, viis jäära, viis sikku, viis aastast oinastalle. See oli Ahira, Eenani poja ohvriand.
\par 84 Need olid altari pühitsusannid Iisraeli vürstidelt selle võidmispäeval: kaksteist hõbevaagnat, kaksteist hõbekaussi, kaksteist kuldnõu;
\par 85 iga hõbevaagen oli saja kolmekümne seekline ja iga hõbekauss seitsmekümnene; kogu riistade hõbedat oli kaks tuhat nelisada püha seekli järgi;
\par 86 kaksteist kuldnõu, täis suitsutusrohtu, iga nõu kümneseekline püha seekli järgi; kogu nõude kulda oli sada kakskümmend seeklit.
\par 87 Kõiki põletusohvri loomi oli kaksteist härjavärssi, kaksteist jäära, kaksteist aastast oinastalle ja nende juurde kuuluv roaohver; ja kaksteist noort sikku oli patuohvriks.
\par 88 Ja kõiki tänuohvri loomi oli kakskümmend neli härga, kuuskümmend jäära, kuuskümmend sikku, kuuskümmend aastast oinastalle. Need olid altari pühitsusannid pärast selle võidmist.
\par 89 Ja kui Mooses läks kogudusetelki Issandaga rääkima, siis ta kuulis häält, mis kõnetas teda tunnistuslaeka peal olevalt kaanelt, kahe keerubi vahelt. Ja Issand rääkis temaga.

\chapter{8}

\par 1 Ja Issand rääkis Moosesega, öeldes:
\par 2 „Räägi Aaroniga ja ütle temale: Kui sa lambid üles sead, siis need seitse lampi valgustagu lambijala esist!”
\par 3 Ja Aaron tegi nõnda: ta seadis lambijala lambid selle esise poole, nagu Issand oli Moosesele käsu andnud.
\par 4 Ja lambijalg oli valmistatud nõnda: kullast sepistatud, jalast kuni õiekeseni sepisetöö; eeskuju järgi, mida Issand oli näidanud Moosesele, oli ta lambijala valmistanud.
\par 5 Ja Issand rääkis Moosesega, öeldes:
\par 6 „Võta leviidid Iisraeli laste hulgast ja puhasta nad!
\par 7 Neid puhastades talita nendega nõnda: piserda nende peale patupuhastuse vett ja nad lasku endil käia habemenuga üle kogu ihu; nad pesku oma riided ja puhastagu endid!
\par 8 Siis nad võtku noor härjavärss ja selle roaohvrina õliga segatud peent jahu, sina aga võta teine noor härjavärss patuohvriks,
\par 9 lase leviidid astuda kogudusetelgi ette ja kogu kokku terve Iisraeli laste kogudus!
\par 10 Siis lase leviidid astuda Issanda ette ja Iisraeli lapsed pangu oma käed leviitide peale!
\par 11 Seejärel Aaron kõigutagu leviite Iisraeli laste kõigutusohvrina Issanda ees, et nad võiksid toimetada Issanda teenistust!
\par 12 Siis pangu leviidid oma käed härjavärsside peade peale: üks neist valmistatagu patuohvriks ja teine põletusohvriks Issandale lepituse toimetamiseks leviitide eest!
\par 13 Pane leviidid seisma Aaroni ja ta poegade ette ja kõiguta neid kõigutusohvrina Issandale!
\par 14 Eralda leviidid Iisraeli laste hulgast, et leviidid jääksid minule!
\par 15 Seejärel mingu leviidid kogudusetelgi teenistusse, kui sa oled neid puhastanud ja kõigutanud kõigutusohvrina,
\par 16 sest nad on antud täiesti minule Iisraeli laste hulgast; kõigi emakoja avajate asemel kõigist Iisraeli laste esmasündinuist olen ma võtnud nad enesele,
\par 17 kuna minu päralt on iga esmasündinu Iisraeli laste hulgast, niihästi inimestest kui loomadest: sel päeval, mil ma lõin maha kõik esmasündinud Egiptusemaal, pühitsesin mina nad enesele.
\par 18 Ma võtsin leviidid kõigi Iisraeli laste esmasündinute asemel
\par 19 ja andsin leviidid kingina Iisraeli laste hulgast Aaronile ja ta poegadele toimetama Iisraeli laste teenistust kogudusetelgi juures ja toimetama Iisraeli laste eest lepitust, et Iisraeli lapsi ei tabaks nuhtlus, kui Iisraeli lapsed liginevad pühamule.”
\par 20 Ja Mooses, Aaron ja terve Iisraeli laste kogudus talitasid nõnda leviitidega; Iisraeli lapsed talitasid nendega kõigiti vastavalt Issanda poolt Moosesele antud käsule leviitide kohta.
\par 21 Ja leviidid puhastasid endid ning pesid oma riided; Aaron kõigutas neid kõigutusohvrina Issanda ees ja Aaron toimetas nende eest lepitust nende puhastamiseks.
\par 22 Seejärel leviidid läksid toimetama oma teenistust kogudusetelgis Aaroni ja tema poegade ees. Nõnda nagu Issand oli Moosest leviitide pärast käskinud, nõnda nad talitasid nendega.
\par 23 Ja Issand rääkis Moosesega, öeldes:
\par 24 „See kehtib leviitide kohta: kes on kahekümne viie aastane ja üle selle, on teenistuskohustuslik kogudusetelgi töös.
\par 25 Aga alates viiekümnendast eluaastast saagu ta lahti teenistuskohustusest ja ärgu enam teenigu,
\par 26 vaid ta aidaku oma vendi kogudusetelgis toimetada, mis on tarvis, ent päris teenistust ta ärgu toimetagu! Talita nõnda leviitidega nende kohustuste poolest!”

\chapter{9}

\par 1 Ja Issand rääkis Moosesega Siinai kõrbes teisel aastal, esimeses kuus pärast Egiptusemaalt lahkumist, öeldes:
\par 2 „Iisraeli lapsed pidagu paasapüha selle seatud ajal:
\par 3 selle kuu neljateistkümnenda päeva õhtul pidage seda seatud ajal; te peate seda pidama kõigi selle eeskirjade ja seaduste järgi!”
\par 4 Ja Mooses ütles Iisraeli lastele, et nad peaksid paasapüha.
\par 5 Siis nad pidasid paasapüha esimese kuu neljateistkümnenda päeva õhtul Siinai kõrbes. Iisraeli lapsed tegid kõigiti nõnda, nagu Issand oli Moosest käskinud.
\par 6 Aga seal oli mehi, kes olid roojased inimese laiba pärast, ja need ei tohtinud pidada paasapüha sel päeval. Ometi astusid nad sel päeval Moosese ja Aaroni ette,
\par 7 ja need mehed ütlesid temale: „Me oleme roojased ühe inimese laiba pärast. Miks keelatakse meid toomast ohvriandi Issandale tema seatud ajal Iisraeli laste hulgas?”
\par 8 Ja Mooses vastas neile: „Seiske siin, kuni ma kuulen, mida Issand teie kohta käsib!”
\par 9 Ja Issand rääkis Moosesega, öeldes:
\par 10 „Räägi Iisraeli lastega ja ütle: Kui keegi teist või teie tulevastest põlvedest on roojane laiba pärast või viibib pikal teekonnal, võib ta ometi pidada paasapüha Issanda auks.
\par 11 Seda peetagu teise kuu neljateistkümnenda päeva õhtul; paasatalle söödagu koos hapnemata leiva ja kibedate rohttaimedega!
\par 12 Midagi ärgu jäetagu sellest üle hommikuks ja selle luid ärgu murtagu; seda peetagu kõigi paasapüha seaduste järgi!
\par 13 Aga see mees, kes on puhas ega ole teekonnal ja ometi jätab paasapüha pidamata, tuleb hävitada oma rahva hulgast, sest ta ei ole toonud Issandale ohvriandi selleks seatud ajal. See mees peab kandma oma pattu!
\par 14 Ja kui teie juures viibib mõni võõras, kes tahab pidada paasapüha Issanda auks, siis pidagu ta seda nõndasamuti paasapüha eeskirja ja seaduse järgi! Teil olgu üks seadus niihästi võõra kui maa päriselaniku jaoks!”
\par 15 Päeval, kui elamu oli püstitatud, kattis elamut tunnistustelgi kohal pilv, aga õhtust hommikuni oli elamu kohal otsekui tulepaistus.
\par 16 Nõnda oli see alaliselt: seda kattis pilv ja öine tulepaistus.
\par 17 Ja iga kord, kui pilv tõusis üles telgi pealt, läksid Iisraeli lapsed teele; ja paika, kuhu pilv jäi seisma, lõid Iisraeli lapsed leeri üles.
\par 18 Issanda käsul läksid Iisraeli lapsed teele ja Issanda käsul lõid nad leeri üles; nad jäid leeri nii kauaks, kuni pilv seisis elamu peal.
\par 19 Ja kui pilv jäi elamu peale seisma kauemaks ajaks, siis Iisraeli lapsed panid tähele Issanda korraldusi ega läinud teele.
\par 20 Kui juhtus, et pilv oli elamu peal ainult mõne päeva, siis nad jäid leeri Issanda käsul ja läksid teele Issanda käsul.
\par 21 Kui juhtus, et pilv püsis õhtust hommikuni ja hommikul tõusis üles, siis nad läksid teele; või kui see jäi päevaks ja ööks, siis nad läksid teele pilve tõustes.
\par 22 Või kui pilv jäi elamu peale kaheks päevaks või kuuks või veel kauemaks ajaks, seisatades selle peal, siis Iisraeli lapsed jäid leeri ega läinud teele; alles pilve tõustes nad läksid teele.
\par 23 Issanda käsul jäid nad leeri ja Issanda käsul läksid nad teele; nad panid tähele Issanda korraldust, Moosese läbi antud Issanda käsul.

\chapter{10}

\par 1 Ja Issand rääkis Moosesega, öeldes:
\par 2 „Valmista enesele kaks hõbepasunat; valmista need taotud tööna ja need olgu sulle koguduse kokkukutsumiseks ja leeride teelesaatmiseks!
\par 3 Kui puhutakse neid mõlemaid, siis peab terve kogudus kogunema su juurde kogudusetelgi ukse ette.
\par 4 Aga kui puhutakse ainult ühte, siis kogunegu su juurde vürstid, Iisraeli tuhatkondade peamehed!
\par 5 Kui te puhute märku, siis mingu teele leerid, mis on leeris ida pool.
\par 6 Kui te teist korda puhute märku, siis mingu teele leerid, mis on leeris lõuna pool; nende teeleminekuks tuleb puhuda märku.
\par 7 Kogudust kokku kutsudes puhuge ka, aga mitte märku!
\par 8 Aaroni pojad, preestrid, puhugu pasunaid! See olgu teile ja teie sugupõlvedele igaveseks seaduseks!
\par 9 Ja kui te oma maal lähete sõtta vaenlase vastu, kes tungib teile kallale, siis puhuge pasunatega märku, et te meenuksite Issandale, teie Jumalale, ja et teid päästetaks teie vaenlaste käest!
\par 10 Oma rõõmupäevil, seatud pühil ja uue kuu alguses puhuge pasunaid oma põletus- ja tänuohvrite juures; see meenutagu teid teie Jumala ees! Mina olen Issand, teie Jumal!”
\par 11 Teisel aastal, teise kuu kahekümnendal päeval tõusis pilv tunnistuselamu pealt
\par 12 ja Iisraeli lapsed läksid teele Siinai kõrbest rännakukorras; ja pilv peatus Paarani kõrbes.
\par 13 Kui nad esimest korda teele läksid Moosese kaudu antud Issanda käsul,
\par 14 siis läks esimesena teele Juuda järglaste leeri lipkond väehulkade kaupa; selle väe ülem oli Nahson, Amminadabi poeg.
\par 15 Issaskari järglaste suguharu väeülem oli Netaneel, Suuari poeg.
\par 16 Sebuloni järglaste suguharu väeülem oli Eliab, Heeloni poeg.
\par 17 Kui elamu oli maha võetud, siis läksid teele Geersoni järglased ja Merari järglased, elamu kandjad.
\par 18 Siis läks teele Ruubeni leeri lipkond oma väehulkade kaupa; selle väe ülem oli Elisuur, Sedeuuri poeg.
\par 19 Siimeoni järglaste suguharu väeülem oli Selumiel, Suurisaddai poeg.
\par 20 Gaadi järglaste suguharu väeülem oli Eljasaf, Deueli poeg.
\par 21 Siis läksid teele kehatlased, pühade asjade kandjad; elamu tuli püstitada enne nende saabumist.
\par 22 Siis läks teele Efraimi järglaste leeri lipkond oma väehulkade kaupa; selle väe ülem oli Elisama, Ammihudi poeg.
\par 23 Manasse järglaste suguharu väeülem oli Gamliel, Pedasuuri poeg.
\par 24 Benjamini järglaste suguharu väeülem oli Abidan, Gideoni poeg.
\par 25 Daani järglaste leeri lipkond oma väehulkade kaupa läks teele kõigi leeride järelväena; selle väe ülem oli Ahieser, Ammisaddai poeg.
\par 26 Aaseri järglaste suguharu väeülem oli Pagiel, Okrani poeg.
\par 27 Naftali järglaste suguharu väeülem oli Ahira, Eenani poeg.
\par 28 Niisugune oli Iisraeli laste teeleminekukord nende väehulkade kaupa; nõnda nad läksid teele.
\par 29 Ja Mooses ütles Hoobabile, Moosese äia, midjanlase Reueli pojale: „Me oleme teel sinna paika, mille kohta Issand on öelnud: Ma annan selle teile. Tule meiega ja me teeme sulle head, sest Issand on tõotanud Iisraelile head!”
\par 30 Aga too vastas temale: „Ma ei tule, sest ma lähen oma maale ja oma sugulaste juurde.”
\par 31 Siis ütles Mooses: „Ära jäta meid maha, sest sina ju tead, kuhu me saame kõrbes leeri üles lüüa! Jää meile silmaks!
\par 32 Kui sa tuled koos meiega ja tuleb see hea, mida Issand lubab meile saata, siis teeme ka meie sulle head.”
\par 33 Siis nad läksid teele Issanda mäe juurest, kolm päevateekonda, ja Issanda seaduselaegas käis kolm päevateekonda nende ees, otsides neile puhkepaika.
\par 34 Ja nende leerist teeleminekul oli Issanda pilv päeval nende kohal.
\par 35 Kui laegas läks teele, ütles Mooses: „Tõuse, Issand, et su vaenlased hajuksid ja su vihamehed põgeneksid su palge eest!”
\par 36 Kui laegas pandi maha, ütles tema: „Tule taas, Issand, Iisraeli tuhatkondade kümnete tuhandete juurde!”

\chapter{11}

\par 1 Aga rahvas nurises ja see oli Issanda kõrvus paha. Kui Issand seda kuulis, siis ta viha süttis põlema: nende keskel süttis Issanda tuli ja põletas leeri äärt.
\par 2 Siis rahvas kisendas Moosese poole. Mooses palus Issandat ja tuli kustus.
\par 3 Sellele paigale pandi nimeks Tabeera, sellepärast et Issanda tuli oli põlenud nende keskel.
\par 4 Aga segarahval, kes oli nende hulgas, tekkis himu, ja Iisraeli lapsed hakkasid ka jälle nutma ning ütlesid: „Kes annaks meile liha süüa?
\par 5 Meil on meeles kala, mida me sõime Egiptuses muidu, kurgid, melonid, laugud, sibulad ja küüslaugud.
\par 6 Aga nüüd kuivetub meie hing, ei ole enam midagi kõigest sellest. Meie silme ees on ainult see manna!”
\par 7 Manna oli nagu koriandri seeme, välimuselt otsekui pedolavaik.
\par 8 Rahvas uitas ringi ja korjas, jahvatas käsikivil või tampis uhmris, keetis potis või valmistas sellest kooke. See maitses nagu õliga küpsetatud maiuspala.
\par 9 Kui öösel langes kaste leeri peale, siis langes ka manna selle peale.
\par 10 Ja Mooses kuulis rahvast nutvat oma suguvõsade kaupa, igaüks oma telgi ukse ees. Siis Issanda viha süttis väga põlema ja see oli Moosese silmis paha.
\par 11 Ja Mooses ütles Issandale: „Miks oled teinud paha oma sulasele? Miks ei ole ma leidnud armu sinu silma ees, et paned kogu selle rahva koormaks minu peale?
\par 12 Kas olen mina kogu selle rahva pärast olnud lapseootel või olen mina tema sünnitanud, et sa ütled mulle: Kanna teda süles, nagu hoidja kannab imikut, maale, mille sa vandega tõotasid anda tema vanemaile?
\par 13 Kus on mul liha anda kogu sellele rahvale? Sest nad nutavad mu ees, öeldes: Anna meile liha süüa!
\par 14 Ei jaksa mina üksi kanda kogu seda rahvast, sest see on mulle liiga raske.
\par 15 Kui sa tahad mind kohelda nõnda, siis parem tapa mind, kui ma leian armu su silmis, et ma ei näeks oma viletsust!”
\par 16 Aga Issand ütles Moosesele: „Kogu mulle seitsekümmend meest Iisraeli vanemaist, kellest sa tead, et nad on tõesti rahvavanemad ja ülevaatajad; too need kogudusetelgi juurde ja nad seisku seal koos sinuga!
\par 17 Siis ma astun alla ja räägin seal sinuga, ja ma võtan sinu peal oleva Vaimu, ja panen nende peale, et nad koos sinuga kannaksid rahva koormat ja sul ei oleks vaja üksinda kanda.
\par 18 Aga rahvale ütle: Pühitsege endid homseks, siis te saate liha süüa, sest te olete nutnud Issanda kõrva ees, öeldes: Kes annaks meile liha süüa? Egiptuses oli meil ju hea olla! Issand annab teile nüüd liha süüa.
\par 19 Te ei söö seda mitte üks päev, ka mitte kaks päeva, viis päeva, kümme päeva ega kakskümmend päeva,
\par 20 vaid terve kuu, kuni see tuleb teil ninast välja ja muutub teile vastikuks, sellepärast et te olete põlanud Issandat, kes on teie keskel, ja olete nutnud ta palge ees, öeldes: Miks tulime siis Egiptusest ära?”
\par 21 Aga Mooses ütles: „Kuussada tuhat jalameest on seda rahvast, kelle keskel ma olen, ja sina ütled: Ma annan neile liha süüa terveks kuuks!
\par 22 Ons lambaid, kitsi ja veiseid tapmiseks, nõnda et neile jätkuks? Või tuleks neile koguda kõik mere kalad, et neile jätkuks?”
\par 23 Aga Issand vastas Moosesele: „Kas Issanda käsi on jäänud lühemaks? Nüüd sa saad näha, kas mu sõna sulle tõestub või mitte!”
\par 24 Ja Mooses tuli välja ning andis rahvale edasi Issanda sõnad; ta kogus seitsekümmend meest rahva vanemaist ja pani need seisma ümber telgi.
\par 25 Siis Issand astus alla pilve sees ja rääkis temaga ning võttis tema peal oleva Vaimu, ja pani nende seitsmekümne mehe peale, kes olid vanemad. Ja kui Vaim oli nende peal, siis nad rääkisid prohveti viisil, aga pärast seda mitte enam.
\par 26 Kaks meest aga oli jäänud leeri, ühe nimi oli Eldad ja teise nimi Meedad; nendegi peal oli Vaim, sest nad olid üleskirjutatute hulgast, kuid nad ei olnud läinud telgi juurde, vaid nad rääkisid leeris prohveti viisil.
\par 27 Siis jooksis üks poiss ja kuulutas Moosesele ning ütles: „Eldad ja Meedad räägivad leeris prohveti viisil!”
\par 28 Siis võttis sõna Joosua, Nuuni poeg, Moosese teener oma noorusest alates, ja ütles: „Mooses, mu isand, keela neid!”
\par 29 Aga Mooses vastas temale: „Kas sa ägestud minu eest? Kes küll annaks, et kogu Issanda rahvas saaks prohveteiks, et Issand paneks oma Vaimu nende peale!”
\par 30 Ja Mooses läks tagasi leeri, tema ja Iisraeli vanemad.
\par 31 Siis tõusis tuul Issanda juurest ja ajas mere poolt vutte, paisates need üle leeri, päevateekond siit- ja päevateekond sealtpoolt ümber leeri, ligi kahe küünra kõrguselt üle maapinna.
\par 32 Ja rahvas võttis kätte kogu selle päeva, kogu öö ja kogu järgmise päeva, ja nad korjasid vutte. Kes korjas pisut, sai viiskümmend vakka, ja nad laotasid need endile kuivama ümber leeri.
\par 33 Liha oli neil alles hammaste vahel, see ei olnud veel ära söödud, kui Issanda viha süttis põlema rahva vastu ja Issand lõi rahvast väga suure nuhtlusega.
\par 34 Siis pandi sellele paigale nimeks Kibrot-Hattaava, sest sinna maeti rahvas, kes oli olnud himukas.
\par 35 Kibrot-Hattaavast läks rahvas teele Haserotti; ja nad jäid Haserotti.

\chapter{12}

\par 1 Aga Mirjam ja Aaron rääkisid Moosesele vastu Etioopia naise pärast, kelle ta oli võtnud; sest ta oli võtnud naiseks etiooplanna.
\par 2 Ja nad ütlesid: „Kas Issand räägib ainult Moosese läbi? Eks ta räägi ka meie läbi?” Ja Issand kuulis seda.
\par 3 Aga mees, Mooses, oli väga alandlik, alandlikum kõigist inimestest maa peal.
\par 4 Siis Issand ütles äkitselt Moosesele, Aaronile ja Mirjamile: „Minge teie kolmekesi välja kogudusetelgi juurde!” Ja need kolmekesi läksid.
\par 5 Siis Issand astus alla pilvesambas, seisis telgi uksel ning kutsus Aaronit ja Mirjamit, ja mõlemad läksid.
\par 6 Ja ta ütles: „Kuulge ometi mu sõnu! Kui teie prohvet on Issanda oma, siis ma ilmutan ennast temale nägemuses, räägin temaga unenäos.
\par 7 Nõnda aga ei ole mu sulase Moosesega: tema on ustav kogu mu kojas.
\par 8 Temaga ma räägin suust suhu, ilmsi, mitte nägemuste ja mõistatuste läbi. Ja tema võib vaadata Issanda kuju. Mispärast te siis ei ole kartnud rääkida vastu mu sulasele Moosesele?”
\par 9 Ja Issanda viha süttis põlema nende vastu ning ta läks ära.
\par 10 Kui pilv oli lahkunud telgi pealt, ennäe, siis oli Mirjam pidalitõvest lumivalge. Aaron pöördus Mirjami poole, ja vaata, too oli pidalitõbine.
\par 11 Siis Aaron ütles Moosesele: „Oh mu isand! Ära pane meie peale seda pattu, et me olime rumalad ja eksisime!
\par 12 Ära lase teda jääda surnultsündinu sarnaseks, kellel juba emaihust tulles on pool ihu kõdunenud!”
\par 13 Siis Mooses kisendas Issanda poole, öeldes: „Oh Jumal! Tee ta ometi terveks!”
\par 14 Ja Issand vastas Moosesele: „Kui ta isa oleks sülitanud temale näkku, kas ta ei oleks siis pidanud häbenema seitse päeva? Olgu ta seitsmeks päevaks suletud väljapoole leeri ja seejärel ta võetagu tagasi!”
\par 15 Siis suleti Mirjam seitsmeks päevaks väljapoole leeri ja rahvas ei läinud teele enne, kui Mirjam oli tagasi võetud.
\par 16 Pärast seda läks rahvas teele Haserotist ja nad lõid leeri üles Paarani kõrbe.

\chapter{13}

\par 1 Ja Issand rääkis Moosesega, öeldes:
\par 2 „Läkita mehi uurima Kaananimaad, mille ma annan Iisraeli lastele; igast nende vanemate suguharust läkitage üks mees, igaüks neist olgu vürst!”
\par 3 Siis läkitas Mooses Issanda käsul nad Paarani kõrbest; kõik need mehed olid Iisraeli laste peamehed.
\par 4 Ja need on nende nimed: Ruubeni suguharust Sammua, Sakkuri poeg;
\par 5 Siimeoni suguharust Saafat, Hoori poeg;
\par 6 Juuda suguharust Kaaleb, Jefunne poeg;
\par 7 Issaskari suguharust Jigal, Joosepi poeg;
\par 8 Efraimi suguharust Hoosea, Nuuni poeg;
\par 9 Benjamini suguharust Palti, Raafu poeg;
\par 10 Sebuloni suguharust Gaddiel, Soodi poeg;
\par 11 Joosepi suguharust, nimelt Manasse suguharust, Gaddi, Suusi poeg;
\par 12 Daani suguharust Ammiel, Gemalli poeg;
\par 13 Aaseri suguharust Setuur, Miikaeli poeg;
\par 14 Naftali suguharust Nahbi, Vofsi poeg;
\par 15 Gaadi suguharust Geuel, Maaki poeg.
\par 16 Need olid nende meeste nimed, keda Mooses läkitas maad kuulama. Ja Mooses nimetas Hoosea, Nuuni poja, Joosuaks.
\par 17 Ja läkitades neid Kaananimaad uurima, ütles Mooses neile: „Minge sinna Lõunamaale ja minge üles mäestikku,
\par 18 vaadake maad, missugune see on, ja rahvast, kes seal elab: on ta vägev või väeti, on teda pisut või palju?
\par 19 Kas maa, kus ta elab, on hea või halb? Missugused on linnad, kus ta elab: kas leerid või kindlused?
\par 20 Ja missugune on maa: kas rammus või lahja, kas seal on puid või ei ole? Olge vaprad ja võtke kaasa maa vilju!” Oli parajasti esimeste viinamarjade aeg.
\par 21 Ja nad läksid ning uurisid maad Siini kõrbest kuni Rehobini, Hamati teelahkmeni.
\par 22 Ja nad läksid lõuna poole ning tulid kuni Hebronini; seal olid Ahiman, Seesai ja Talmai, Anaki järeltulijad. Hebron oli ehitatud seitse aastat enne Egiptuse Soani.
\par 23 Siis nad tulid Kobaraorgu ja lõikasid sealt viinapuuväädi üheainsa kobaraga ning kandsid seda kahe mehega põikpuus; nad võtsid ka granaatõunu ja viigimarju.
\par 24 See paik nimetati Kobaraoruks viinamarjakobara pärast, mille Iisraeli lapsed sealt lõikasid.
\par 25 Ja neljakümne päeva pärast tulid nad tagasi maad kuulamast.
\par 26 Nad tulid ning läksid Moosese ja Aaroni ja kogu Iisraeli laste koguduse juurde Paarani kõrbe Kaadesisse ning tõid sõnumeid neile ja kogu kogudusele ja näitasid neile maa vilju.
\par 27 Ja nad jutustasid temale ning ütlesid: „Me jõudsime sellele maale, kuhu sa meid läkitasid. See voolab tõesti piima ja mett, ja siin on selle viljad.
\par 28 Kuid rahvas, kes elab maal, on tugev, ja linnad on kindlustatud ja väga suured. Me nägime seal ka Anaki järeltulijaid.
\par 29 Lõunamaal elavad amalekid, mäestikus elavad hetid, jebuuslased ja emorlased, mererannas ja Jordani ääres elavad kaananlased.”
\par 30 Siis Kaaleb vaigistas rahvast Moosese ees ja ütles: „Mingem siiski sinna ja vallutagem see, sest me suudame selle alistada!”
\par 31 Aga mehed, kes olid käinud koos temaga, ütlesid: „Me ei või minna selle rahva vastu, sest ta on meist vägevam.”
\par 32 Ja nad levitasid Iisraeli laste ees laimu maa kohta, mida nad olid uurinud, öeldes: „maa, mille me uurides läbi käisime, on maa, mis neelab oma elanikud, ja kogu see rahvas, keda me seal nägime, on pikakasvulised inimesed.
\par 33 Me nägime seal hiiglasi, Anaki poegi hiiglaste soost: me olime iseendi silmis nagu rohutirtsud ja samasugused olime meie ka nende silmis.”

\chapter{14}

\par 1 Siis terve kogudus tõstis valjusti häält ja rahvas nuttis sel ööl.
\par 2 Kõik Iisraeli lapsed nurisesid Moosese ja Aaroni vastu ja terve kogudus ütles neile: „Oleksime ometi surnud Egiptusemaal või siin kõrbes! Oleksime ometi surnud!
\par 3 Miks Issand viib meid sellele maale, kus me langeme mõõga läbi ja meie naised ja lapsed jäävad saagiks? Kas meil ei oleks parem minna tagasi Egiptusesse?”
\par 4 Ja nad ütlesid üksteisele: „Valigem pealik ja mingem tagasi Egiptusesse!”
\par 5 Siis Mooses ja Aaron heitsid silmili maha kogu Iisraeli laste kokkutulnud koguduse ees,
\par 6 Joosua, Nuuni poeg, ja Kaaleb, Jefunne poeg, maakuulajate hulgast aga käristasid oma riided lõhki
\par 7 ja rääkisid kogu Iisraeli laste kogudusele, öeldes: „Maa, mille me läbi käisime, et seda uurida, on väga hea Maa.
\par 8 Kui Issandal on meist hea meel, siis ta viib meid sellele maale ja annab selle meile, maa, mis piima ja mett voolab.
\par 9 Ärge ainult pange vastu Issandale ja ärge kartke maa rahvast, sest nad on meile parajaks palaks: nende kaitsja on nad maha jätnud, aga Issand on meiega! Ärge kartke neid!”
\par 10 Siis ütles terve kogudus, et nad tuleks kividega surnuks visata! Aga Issanda auhiilgus ilmutas ennast kogudusetelgis kõigile Iisraeli lastele.
\par 11 Ja Issand ütles Moosesele: „Kui kaua see rahvas põlgab mind ja kui kaua nad ei usu mind hoolimata kõigist tunnustähtedest, mis ma nende keskel olen teinud?
\par 12 Ma löön teda katkuga ja hävitan tema, aga sinust ma teen suurema ja vägevama rahva kui tema!”
\par 13 Siis Mooses ütles Issandale: „Egiptlased on muidugi kuulnud, et sa oma rammuga oled selle rahva ära toonud nende keskelt,
\par 14 ja nad on seda rääkinud selle maa elanikele. Nemadki on siis kuulnud, et sina, Issand, oled selle rahva keskel, sina, Issand, kes ennast ilmutad silmast silma, ja et sinu pilv seisab nende kohal ning et sa käid nende ees päeval pilvesambas ja öösel tulesambas.
\par 15 Aga kui sa nüüd surmad selle rahva nagu üheainsa mehe, siis räägivad rahvad, kes sinu kuulsusest on kuulnud, ja ütlevad:
\par 16 Sellepärast et Issand ei suutnud viia seda rahvast maale, mille ta neile oli vandega tõotanud, tappis ta nad kõrbes.
\par 17 Nüüd aga saagu Issanda ramm suureks, nagu sa oled tõotanud, öeldes:
\par 18 Issand on pika meelega ja rikas heldusest, ta annab andeks patu ja üleastumise, aga kes ei jäta süüdlast karistamata, vaid nuhtleb vanemate patu laste kätte kolmanda ja neljanda põlveni.
\par 19 Anna siis andeks selle rahva patt oma suure helduse pärast ja nagu sa sellele rahvale oled andeks andnud Egiptusemaalt kuni siiani!”
\par 20 Ja Issand vastas: „Ma annan andeks, nagu oled palunud!
\par 21 Aga nii tõesti kui ma elan ja kogu maailm on täis Issanda auhiilgust:
\par 22 ükski neist meestest, kes on näinud mu auhiilgust ja tunnustähti, mis ma tegin Egiptuses ja kõrbes, aga kes sellest hoolimata on mind kiusanud kümme korda ega ole võtnud kuulda mu häält,
\par 23 ei saa näha maad, mille ma vandega olen tõotanud anda nende vanemaile; ükski, kes mind on põlanud, ei saa seda näha!
\par 24 Aga oma sulase Kaalebi viin ma sellele maale, kus ta on käinud, ja tema sugu pärib selle, sellepärast et temas on teistsugune vaim ja tema on kõiges mulle järgnenud!
\par 25 Amalekid ja kaananlased elavad ju orus. Homme pöörduge ümber ja minge teele kõrbe poole mööda Kõrkjamere teed!”
\par 26 Ja Issand rääkis Moosesega ja Aaroniga, öeldes:
\par 27 „Kui kaua peab mul olema kannatust selle halva kogudusega, kes nuriseb mu vastu? Iisraeli laste nurinaid, kuidas nad nurisevad mu vastu, ma olen kuulnud.
\par 28 Ütle neile: Nii tõesti kui ma elan, on Issanda sõna, et nõnda nagu te minu kuuldes olete rääkinud, nõnda ma teen teiega!
\par 29 Siia kõrbe langevad teie kehad, kõik teie äraloetud, nii palju kui teid on, kahekümneaastased ja üle selle, kes on nurisenud mu vastu.
\par 30 Ükski teist ei pääse sellele maale, mille pärast ma oma käe olen vandudes üles tõstnud, et ma asustan teid sinna, peale Kaalebi, Jefunne poja, ja Joosua, Nuuni poja.
\par 31 Aga teie lapsed, kelle kohta te ütlesite, et nad jäävad riisutavaiks, ma viin ja nad õpivad tundma maad, mida teie olete põlanud.
\par 32 Kuid teie kehad langevad siia kõrbe
\par 33 ja teie lapsed peavad olema kõrbes karjased nelikümmend aastat ja kandma teie uskmatuse süüd, kuni teie kehad on kõrbes hävinud.
\par 34 Vastavalt päevade arvule, mis te uurisite seda maad, nelikümmend päeva, iga päeva kohta aasta, peate te kandma oma pattu nelikümmend aastat ja tundma minu vastupanu.
\par 35 Mina, Issand, olen rääkinud! Tõesti, seda ma teen kogu selle halva kogudusega, kes on kogunenud mu vastu: nad peavad hukkuma selles kõrbes ja surema seal!”
\par 36 Ja need mehed, keda Mooses oli läkitanud maad kuulama ja kes olid tagasi tulles ässitanud terve koguduse nurisema tema vastu, levitades maa kohta laimu,
\par 37 need mehed, kes olid levitanud maa kohta halba laimu, surid Issanda ees äkitselt.
\par 38 Aga Joosua, Nuuni poeg, ja Kaaleb, Jefunne poeg, jäid elama neist meestest, kes olid käinud maad kuulamas.
\par 39 Kui Mooses rääkis needsamad sõnad kõigile Iisraeli lastele, siis rahvas kurvastas väga.
\par 40 Aga järgmisel hommikul tõusid nad vara ja läksid üles mäestikku, öeldes: „Vaata, siin me oleme ja me läheme paika, millest Issand on rääkinud, sest me oleme pattu teinud.”
\par 41 Aga Mooses ütles: „Miks te siis astute üle Issanda käsust? See ei õnnestu!
\par 42 Ärge minge sinna üles, sest Issand ei ole teie keskel, et te ei kannaks kaotust oma vaenlaste ees!
\par 43 Sest amalekid ja kaananlased on seal teie ees ja te langete mõõga läbi; sellepärast et te olete taganenud Issanda järelt, ei ole ka Issand teiega.”
\par 44 Aga nad olid ülemeelsed minema üles mäestikku, kuigi Issanda seaduselaegas ja Mooses ei läinud leerist välja.
\par 45 Amalekid ja kaananlased, kes elasid seal mäestikus, tulid siis alla ja lõid neid ning ajasid nad kuni Hormani.

\chapter{15}

\par 1 Ja Issand rääkis Moosesega, öeldes:
\par 2 „Räägi Iisraeli lastega ja ütle neile: Kui te jõuate maale, mille mina teile elamiseks annan,
\par 3 ja te tahate Issandale ohverdada tuleohvrit, põletus- või tapaohvrit, olgu erilise tõotusena või vabatahtliku annina, või teie seatud pühil, et valmistada Issandale meeldivat lõhna veistest või lammastest ja kitsedest,
\par 4 siis toogu see, kes toob oma ohvrianni Issandale, roaohvriks kaks toopi peent jahu, segatud kolme kortli õliga,
\par 5 ja joogiohvriks põletus- või tapaohvri juurde ohverda kolm kortlit veini ühe talle kohta!
\par 6 Aga jäära kohta ohverda roaohvriks kaks kannu peent jahu, segatud ühe toobi õliga,
\par 7 ja joogiohvriks too üks toop veini, Issandale meeldivaks lõhnaks!
\par 8 Aga kui sa ohverdad noore veise põletus- või tapaohvriks, kas erilise tõotusena või tänuohvrina Issandale,
\par 9 siis toodagu noore veise kohta roaohvriks pool külimittu peent jahu, segatud poolteise toobi õliga,
\par 10 ja joogiohvriks too poolteist toopi veini; see on healõhnaline tuleohver Issandale.
\par 11 Nõndasamuti tehtagu iga härja või jäära puhul, või talle puhul lammastest või kitsedest
\par 12 vastavalt nende arvule, keda te ohverdate; nõnda tehke iga üksiku puhul vastavalt nende arvule!
\par 13 Nõnda peab seda tegema iga pärismaalane, tuues healõhnalist tuleohvrit Issandale.
\par 14 Ja kui keegi, kes elab teie juures võõrana, või keegi, kes põlvede jooksul on teie keskel, tahab tuua Issandale healõhnalist tuleohvrit, nagu teie seda teete, siis tehku ka tema nõnda!
\par 15 Kogudusel olgu üks seadus niihästi teile kui võõrale, kes elab teie juures! See olgu igaveseks seaduseks teie sugupõlvedele: nõnda nagu teie, nõnda olgu ka võõras Issanda ees!
\par 16 Sama seadus ja sama õigus olgu teil ja võõral, kes elab teie juures!”
\par 17 Ja Issand rääkis Moosesega, öeldes:
\par 18 „Räägi Iisraeli lastega ja ütle neile: Kui te tulete sellele maale, kuhu mina teid viin,
\par 19 siis süües selle maa leiba, te peate Issandale andma tõstelõivu:
\par 20 oma uudsest sõmerast jahust andke kakuke tõstelõivuks; nõnda nagu rehealuse tõstelõivu, nõnda andke ka seda!
\par 21 Oma uudsest sõmerast jahust andke põlvest põlve Issandale tõstelõivu!
\par 22 Aga kui te eksite ega täida kõiki neid käske, mis Issand on andnud Moosesele,
\par 23 kõiki, mis Issand teile on andnud Moosese läbi alates päevast, mil Issand andis käsud, ja edaspidi põlvest põlve,
\par 24 siis peab, kui seda on tehtud koguduse teadmata, tahtmatult, terve kogudus ohverdama ühe noore härjavärsi põletusohvriks, Issandale meeldivaks lõhnaks koos selle juurde kuuluva roaohvri ja joogiohvriga vastava korra järgi, ja ühe noore siku patuohvriks.
\par 25 Ja preester toimetagu lepitust terve Iisraeli laste koguduse eest; siis antakse neile andeks, sest see oli eksitus, ja nad on toonud oma ohvrianni tuleohvriks Issandale ja oma patuohvri Issanda ette oma eksimuse pärast.
\par 26 Siis antakse andeks tervele Iisraeli laste kogudusele, ka võõrale, kes elab nende keskel, sest kogu rahvas on osaline eksimuses.
\par 27 Aga kui üksik inimene kogemata pattu teeb, siis ta toogu patuohvriks üks aastane emane kitsetall
\par 28 ja preester toimetagu lepitust eksinud hinge eest, kes kogemata on patustanud Issanda ees; kui tema eest on lepitust toimetatud, siis antakse temale andeks!
\par 29 Pärismaalasele Iisraeli laste seast ja võõrale, kes elab teie keskel, olgu teil ühesugune seadus, kui midagi on tehtud kogemata.
\par 30 Aga hing, kes midagi teeb meelega, olgu pärismaalane või võõras, pilkab Issandat; see hing hävitatagu oma rahva seast,
\par 31 sest ta on põlanud Issanda sõna ja on tühistanud tema käsu! See hing hävitatagu tõesti, tema peal on süü!”
\par 32 Kord, kui Iisraeli lapsed olid kõrbes, tabasid nad ühe mehe hingamispäeval puid korjamast.
\par 33 Need, kes tabasid tema puid korjamast, viisid ta Moosese ja Aaroni ja terve koguduse ette.
\par 34 Ja nemad panid ta vahi alla, sest ei olnud selgesti öeldud, mis temaga tuli teha.
\par 35 Aga Issand ütles Moosesele: „Seda meest karistatagu surmaga, terve kogudus visaku ta väljaspool leeri kividega surnuks!”
\par 36 Siis terve kogudus viis tema väljapoole leeri ja nad viskasid ta kividega surnuks, nagu Issand oli Moosest käskinud.
\par 37 Ja Issand rääkis Moosesega, öeldes:
\par 38 „Räägi Iisraeli lastega ja ütle neile, et nad põlvest põlve teeksid enestele tutid oma kuue hõlmadele ja paneksid hõlmatuttide külge sinise nööri.
\par 39 Need tutid olgu teile selleks, et te neid vaadates meenutaksite kõiki Issanda käske ja täidaksite neid ega käiks ringi oma südame- ja silmahimude järgi, mis ahvatlevad teid truudusetusele.
\par 40 Pidage siis meeles ja täitke kõiki mu käske ja olge pühad oma Jumala ees!
\par 41 Mina olen Issand, teie Jumal, kes tõi teid ära Egiptusemaalt, et olla teile Jumalaks. Mina olen Issand, teie Jumal!”

\chapter{16}

\par 1 Aga Korah, Jishari poeg, kes oli Leevi poja Kehati poeg, ja Daatan ja Abiram, Eliabi pojad, ja Oon, Peleti poeg Ruubeni poegadest, võtsid kätte
\par 2 ja tõusid üles Moosese palge ees, nõndasamuti kakssada viiskümmend meest Iisraeli lastest, koguduse vürstid, kogudusest kutsutud nimekad mehed,
\par 3 ja nad kogunesid Moosese ja Aaroni vastu ning ütlesid neile: „Nüüd on küllalt! Sest terve kogudus - nad kõik on pühad ja Issand on nende keskel! Mispärast tõstate siis teie endid Issanda koguduse üle?”
\par 4 Kui Mooses seda kuulis, siis ta heitis silmili maha
\par 5 ja rääkis Korahi ning kogu ta jõuguga, öeldes: „Homme ilmutab Issand, kes on tema oma, kes on püha ja keda ta laseb enesele ligineda. Kelle ta välja valib, seda ta laseb enesele ligineda.
\par 6 Tehke nõnda: võtke enestele suitsutuspannid, Korah ja kogu ta jõuk,
\par 7 tehke neisse tuli ja pange homme nende peale suitsutusrohtu Issanda ees; siis on see mees, kelle Issand välja valib, püha! Nüüd on sellest küllalt, Leevi pojad!”
\par 8 Ja Mooses ütles Korahile: „Kuulge ometi, Leevi pojad!
\par 9 Ons teil vähe sellest, et Iisraeli Jumal teid on eraldanud Iisraeli kogudusest, lastes teid enesele ligineda, toimetama Issanda elamu teenistust ja seisma koguduse ees, et seda teenida?
\par 10 Ta on lasknud ligineda sind ja kõiki su vendi, Leevi poegi, koos sinuga. Ja nüüd te nõuate ka preestriametit!
\par 11 Sellepärast olete sina ja kogu su jõuk kogunenud Issanda vastu, sest kes on Aaron, et te nurisete tema vastu?”
\par 12 Ja Mooses läkitas kutsuma Daatanit ja Abirami, Eliabi poegi. Aga need ütlesid: „Me ei tule!
\par 13 Ons veel vähe, et oled meid toonud ära maalt, mis piima ja mett voolab, et meid kõrbes suretada, et sa tahad ennast teha ka veel valitsejaks meie üle?
\par 14 Ei, sa ei ole meid viinud maale, mis piima ja mett voolab, ega ole meile andnud pärisosaks põlde ja viinamägesid! Kas tahad neil inimestel silmad välja torgata? Me ei tule!”
\par 15 Siis Moosese viha süttis väga põlema ja ta ütles Issandale: „Ära hooli nende roaohvreist! Ei ole ma neilt võtnud ühtegi eeslit ega ole ühelegi neist kurja teinud!”
\par 16 Ja Mooses ütles Korahile: „Sina ja kogu su jõuk olge homme Issanda ees, sina ja nemad ja Aaron,
\par 17 ja igamees võtku oma suitsutuspann, pangu selle peale suitsutusrohtu ja igamees toogu Issanda ette oma suitsutuspann, kakssada viiskümmend suitsutuspanni, ka sina ja Aaron kumbki oma suitsutuspann!”
\par 18 Siis nad võtsid igamees oma suitsutuspanni, tegid sellesse tule, panid selle peale suitsutusrohtu ja astusid kogudusetelgi ukse ette, ka Mooses ja Aaron.
\par 19 Ja Korah kogus nende vastu terve koguduse kogudusetelgi ukse ette. Siis Issanda auhiilgus ilmutas ennast tervele kogudusele.
\par 20 Ja Issand rääkis Moosese ja Aaroniga, öeldes:
\par 21 „Lahkuge sellest jõugust, siis ma hävitan nad silmapilkselt!”
\par 22 Aga nad heitsid silmili maha ja ütlesid: „Jumal, kõige liha vaimude Jumal! Kui üks mees pattu teeb, kas sa vihastud siis terve koguduse peale?”
\par 23 Ja Issand rääkis Moosesega, öeldes:
\par 24 „Räägi kogudusega ja ütle: Minge ära Korahi, Daatani ja Abirami elukoha ümbrusest!”
\par 25 Siis Mooses tõusis ja läks Daatani ja Abirami juurde, ja Iisraeli vanemad järgnesid temale.
\par 26 Ja ta rääkis kogudusega, öeldes: „Minge nüüd eemale nende jultunud meeste telkide juurest ja ärge puudutage midagi, mis on nende oma, et teid ei hävitataks kõigi nende pattude pärast!”
\par 27 Siis läksid nad ära Korahi, Daatani ja Abirami elukoha ümbrusest; aga Daatan ja Abiram tulid välja ja seisid oma telkide uste ees, ka nende naised, lapsed ja imikud.
\par 28 Ja Mooses ütles: „Et Issand on mind läkitanud tegema kõiki neid tegusid ja et ma ei tee neid omast tahtest, seda te saate teada sellest:
\par 29 kui nad surevad, nagu kõik inimesed surevad, ja kõigi muude inimeste karistus tabab ka neid, siis ei ole Issand mind läkitanud.
\par 30 Aga kui Issand laseb sündida midagi erakorralist ja maa avab oma suu ja neelab nemad ja kõik, mis neil on, nõnda et nad lähevad elusalt alla hauda, siis te teate, et need inimesed on põlanud Issandat!”
\par 31 Ja kui ta oli rääkinud kõik need sõnad, siis sündis, et nende all olev maa lõhkes
\par 32 ja maa avas oma suu ja neelas ära nemad ja nende kojad ja kõik inimesed, kes kuulusid Korahile, ja kogu varanduse.
\par 33 Nõnda läksid nemad ja kõik, mis neil oli, elusalt hauda ja maa kattis nad ning nad hävitati koguduse hulgast.
\par 34 Ja kogu Iisrael, kes oli nende ümber, põgenes nende kisa tõttu, sest nad ütlesid: „Et maa ei neelaks ka meid!”
\par 35 Ja Issanda juurest läks välja tuli ning põletas need kakssada viiskümmend meest, kes oli ohverdanud suitsutusrohtu.

\chapter{17}

\par 1 Ja Issand rääkis Moosesega, öeldes:
\par 2 „Ütle Eleasarile, preester Aaroni pojale, et ta võtaks suitsutuspannid tuleasemelt, ja pillu tuli laiali, sest need on saanud pühaks,
\par 3 nende patuste suitsutuspannid, nende hinge hinna eest! Neist valmistatagu taotud plaate altari katteks, sest nad tõid need Issanda ette ja nõnda said need pühaks. Need olgu hoiatusmärgiks Iisraeli lastele!”
\par 4 Ja preester Eleasar võttis vasksuitsutuspannid, mis põlenud mehed olid toonud, ja need taoti altari katteks,
\par 5 meeldetuletuseks Iisraeli lastele, et ükski võõras, kes ei ole Aaroni soost, ärgu liginegu süütama suitsutusrohtu Issanda ees, et ei juhtuks nõnda nagu Korahi ja tema jõuguga, mida Issand temale oli kuulutanud Moosese läbi.
\par 6 Aga järgmisel päeval nurises kogu Iisraeli laste kogudus Moosese ja Aaroni vastu, öeldes: „Te olete tapnud Issanda rahva!”
\par 7 Aga kui kogudus kogunes Moosese ja Aaroni vastu ja nad pöördusid kogudusetelgi poole, vaata, siis kattis seda pilv ja Issanda auhiilgus ilmutas ennast.
\par 8 Siis Mooses ja Aaron läksid kogudusetelgi ette.
\par 9 Ja Issand rääkis Moosesega, öeldes:
\par 10 „Minge ära selle koguduse keskelt ja ma hävitan nad silmapilkselt!” Siis nad heitsid silmili maha.
\par 11 Ja Mooses ütles Aaronile: „Võta suitsutuspann, pane selle peale suitsutusrohtu ja vii kiiresti koguduse juurde ning toimeta nende eest lepitust, sest Issandalt on lähtunud vihahoog, nuhtlus on alanud!”
\par 12 Ja Aaron tegi, nagu Mooses käskis, ja jooksis koguduse keskele, ja vaata, nuhtlus oli rahva seas juba alanud. Ta pani suitsutusrohtu pannile ja toimetas rahva eest lepitust.
\par 13 Ja kui ta seisis surnute ja elavate vahel, pandi nuhtlusele piir.

\chapter{18}

\par 1 Ja Issand ütles Aaronile: „Sina ja su pojad, koos sinuga su perekond, peate kandma süüd pühamu eest; ja sina, ja koos sinuga su pojad, peate kandma süüd oma preestriameti eest.
\par 2 Too ka oma vennad, Leevi suguharu, enesega kaasa ja nad seltsigu sinuga ja teenigu sind, kui sina, ja koos sinuga su pojad, olete tunnistustelgi ees!
\par 3 Nad toimetagu sinu ja kogu telgiteenistust, aga pühadele riistadele ja altarile nad ei tohi ligineda, et ei sureks nii nemad kui teie!
\par 4 Nad seltsigu sinuga ja toimetagu kogudusetelgi teenistust, kogu telgiteenistust; aga võõras ei tohi teile ligineda!
\par 5 Teie aga toimetage pühamu- ja altariteenistust, et suur viha ei tabaks enam Iisraeli lapsi!
\par 6 Mina, vaata, olen võtnud teie vennad leviidid Iisraeli laste hulgast: nad on teile kingituseks, mis on antud Issandale, et nad toimetaksid kogudusetelgi teenistust.
\par 7 Aga sina ja koos sinuga su pojad peate pidama preestriametit kõiges, mis puutub altarisse ja seespool eesriiet sooritatavaisse toiminguisse, ja teie peate toimetama teenistust! Teie preestriameti ma annan teile kingitusena, aga võõras, kes ligineb, surmatagu!”
\par 8 Ja Issand rääkis Aaroniga: „Mina, vaata, olen andnud sinule, mis mu tõstelõivudest tuleb üle jätta; kõigist Iisraeli laste pühadest andidest ma annan need osaks sinule ja su poegadele igavese seadusena.
\par 9 Kõige pühamaist andidest saagu sulle see, mida ei põletata tules: kõik nende ohvriannid kõigi nende roaohvrite ja kõigi nende patuohvrite ja kõigi nende süüohvrite puhul, mida nad mulle toovad; kui kõige püham kuulub see sinule ja su poegadele.
\par 10 Söö seda kõige pühamas paigas; kõik meesterahvad võivad seda süüa! See olgu sulle püha!
\par 11 Nende andidest kuulub sinule tõstelõivuna ka see: kõik Iisraeli laste kõigutusohvrid ma annan sinule ja su poegadele ja tütardele koos sinuga igavese seadusena; iga puhas su peres võib neid süüa.
\par 12 Kõik parima õlist, kõik parima värskest veinist ja nende uudseviljast, mida nad annavad Issandale, ma annan sinule.
\par 13 Uudsevili kõigest, mis nende maal on, mida nad toovad Issandale, olgu sinu; kõik puhtad su peres võivad seda süüa.
\par 14 Kõik pühendatu Iisraelis kuulub sinule.
\par 15 Iga emakoja avaja kõigest lihast, keda tuuakse Issandale inimestest ja veistest, olgu sinu; aga lase lunastada inimese esmasündinu, samuti lase lunastada roojase looma esmasündinu!
\par 16 Neid lunastatavaid lase lunastada ühe kuu vanuselt hindeväärtuse alusel: viis seeklit hõbedat püha seekli järgi, milles on kakskümmend geera.
\par 17 Aga härja või lamba või kitse esmasündinut ära lase lunastada: need on pühitsetud; nende veri piserda altarile ja nende rasv süüta põlema healõhnaliseks tuleohvriks Issandale!
\par 18 Nende liha aga saagu sinule; samuti kui kõigutusrind ja parempoolne saps saagu see sinule!
\par 19 Kõik pühad tõstelõivud, mida Iisraeli lapsed ohverdavad Issandale, annan ma sinule ja su poegadele ja tütardele koos sinuga kui igavesti kuuluva osa. See olgu igavene soolaosadus Issanda ees sinule ja su soole koos sinuga!”
\par 20 Ja Issand ütles Aaronile: „Nende maal ärgu olgu sul pärisosa ja ärgu olgu sul omandit nende seas! Mina olen su omand ja su pärisosa Iisraeli laste seas.
\par 21 Aga leviitidele, vaata, annan ma pärisosaks kõik kümnise Iisraelis tasuks nende teenistuse eest, mida nad toimetavad, kogudusetelgi-teenistuse eest.
\par 22 Ja Iisraeli lapsed ei tohi enam ligineda kogudusetelgile, et nad ei võtaks eneste peale pattu ega sureks,
\par 23 vaid leviidid toimetagu kogudusetelgi-teenistust ja nemad kandku süüd; see olgu teie sugupõlvedele igaveseks seaduseks! Aga Iisraeli laste seas nad ärgu saagu pärisosa,
\par 24 sest Iisraeli laste kümnise, mida nad ohverdavad Issandale tõstelõivuks, annan ma leviitidele pärisosaks. Seepärast ma ütlen neile, et nad ei tohi saada pärisosa Iisraeli laste seas!”
\par 25 Ja Issand rääkis Moosesega, öeldes:
\par 26 „Räägi leviitidega ja ütle neile: Kui te võtate Iisraeli lastelt kümnist, mille ma neilt olen andnud teile kui teie pärisosa, siis võtke sellest Issandale tõstelõiv - kümnis kümnistest -
\par 27 ja seda teie tõstelõivu arvestatagu teile nagu vilja rehealusest ja ülevoolavust surutõrrest!
\par 28 Nõnda ohverdage ka teie Issandale tõstelõivu kõigist oma kümnistest, mida te saate Iisraeli lastelt, ja see Issanda tõstelõiv andke preester Aaronile!
\par 29 Kõigist oma saadud andidest ohverdage Issandale täielik tõstelõiv, kõige parem osa neist kui temale pühitsetud and!
\par 30 Ja ütle neile: Kui te ohverdate sellest parima osa, siis arvestatagu seda leviitidele nagu rehealuse vilja ja surutõrre saadust!
\par 31 Ja te võite seda süüa igas paigas, teie ja teie pere, sest see on teie palk teenistuse eest kogudusetelgis.
\par 32 Ja kui te ohverdate sellest parima osa, siis te ei võta sellega eneste peale pattu ega teota Iisraeli laste pühi ande, ja te ei sure.”

\chapter{19}

\par 1 Ja Issand rääkis Moosese ja Aaroniga, öeldes:
\par 2 „See on Seaduse määrus, mille Issand on andnud, öeldes: Ütle Iisraeli lastele, et nad tooksid sulle ühe punase veatu lehma, kellel ei ole kehalist viga, kelle peale veel ei ole iket pandud.
\par 3 Andke see preester Eleasarile; see viidagu väljapoole leeri ja tapetagu tema silma all!
\par 4 Siis preester Eleasar võtku oma sõrmega verd ja piserdagu verd kogudusetelgi esikülje poole seitse korda!
\par 5 Ja lehm põletatagu tema silma all; selle nahk, liha ja veri põletatagu koos sisikonnaga!
\par 6 Ja preester võtku seedripuud, iisopit ja helepunast lõnga ja visaku need keset lehmapõletust!
\par 7 Siis preester pesku oma riided ja loputagu oma ihu veega; seejärel ta võib tulla leeri; aga preester on õhtuni roojane!
\par 8 Ka see, kes selle põletas, pesku oma riided veega ja loputagu oma ihu veega, ja ta on õhtuni roojane!
\par 9 Aga üks puhas mees koristagu kokku lehma tuhk ja pangu see väljapoole leeri ühte puhtasse paika; seda säilitatagu Iisraeli laste kogudusele puhastusvee jaoks! See on patuohver.
\par 10 Ja kes lehma tuha kokku koristab, pesku oma riided, ja ta on õhtuni roojane. See olgu igaveseks seaduseks Iisraeli lastele ja võõrale, kes elab nende keskel!
\par 11 Kes puudutab mõne inimese laipa, see on seitse päeva roojane.
\par 12 Ta puhastagu ennast kolmandal ja seitsmendal päeval, siis ta saab puhtaks; aga kui ta kolmandal ja seitsmendal päeval ennast ei puhasta, siis ta ei saa puhtaks.
\par 13 Igaüks, kes puudutab surnut, mõnda surnud inimest, aga ei puhasta ennast, see roojastab Issanda eluaset ja see hing tuleb hävitada Iisraelist, sest tema peale ei ole piserdatud puhastusvett: ta on roojane, tema roojasus on üha ta peal!
\par 14 Niisugune on seadus, kui inimene sureb telgis: igaüks, kes tuleb telki, ja igaüks, kes on telgis, on seitse päeva roojane.
\par 15 Ja iga lahtine astja, mis ei ole pealt kinni seotud, on roojane.
\par 16 Ja igaüks, kes väljal puudutab mõõgaga tapetut või muidu surnut, või inimese luud või hauda, on seitse päeva roojane.
\par 17 Aga seesuguse roojase pärast võetagu põletatud patuohvri põrmu ja valatagu selle peale astjasse värsket vett!
\par 18 Siis võtku üks puhas mees iisopit ja kastku vette ning piserdagu telgi ja kõigi riistade peale, samuti inimeste peale, kes seal olid, ja selle peale, kes puudutas luud või tapetut või muidu surnut või hauda!
\par 19 Ja see puhas piserdagu kolmandal ja seitsmendal päeval seda roojast ja puhastagu teda seitsmendal päeval; aga see roojane pesku oma riided ja loputagu ennast veega, siis ta saab õhtul puhtaks!
\par 20 Aga mees, kes saab roojaseks ega puhasta ennast, tuleb hävitada koguduse seast, sest ta on roojastanud Issanda pühamut; tema peale ei ole piserdatud puhastusvett, ta on roojane.
\par 21 See olgu neile igaveseks seaduseks! Ja kes piserdab puhastusvett, pesku oma riided, ja kes puudutab puhastusvett, on õhtuni roojane!
\par 22 Ja kõik, mida puudutab roojane, saab roojaseks, ja inimene, kes puudutab teda, on õhtuni roojane.”

\chapter{20}

\par 1 Ja Iisraeli lapsed, terve kogudus, tulid esimeses kuus Siini kõrbe ja rahvas asus Kaadesisse. Seal suri Mirjam ja ta maeti sinna.
\par 2 Kogudusel aga ei olnud vett. Siis nad kogunesid Moosese ja Aaroni vastu.
\par 3 Ja rahvas riidles Moosesega, nad rääkisid ning ütlesid nõnda: „Oleksime ometi meiegi hinge heitnud, siis kui meie vennad hinge heitsid Issanda ees!
\par 4 Miks tõite Issanda koguduse, meid ja meie loomad, siia kõrbe surema?
\par 5 Miks tõite meid ära Egiptusest, tuues meid siia pahasse paika? Ei ole see vilja ega viigimarja, viinapuu ega granaatõuna paik, ja joomiseks ei ole vett!”
\par 6 Aga Mooses ja Aaron tulid rahvakogu eest kogudusetelgi ukse ette ja heitsid silmili maha. Siis ilmutas ennast neile Issanda auhiilgus.
\par 7 Ja Issand rääkis Moosesega, öeldes:
\par 8 „Võta kepp ja kogu kokku kogudus, sina ja su vend Aaron, ja öelge nende nähes kaljule, et see annaks vett! Sina too neile kaljust vesi välja ning jooda kogudust ja nende loomi!”
\par 9 Siis Mooses võttis Issanda eest kepi, nagu teda oli kästud.
\par 10 Ja Mooses ja Aaron kogusid koguduse kalju ette; ja ta ütles neile: „Kuulge nüüd, te vastupanijad! Kas peame tooma teile vee välja sestsamast kaljust?”
\par 11 Siis Mooses tõstis oma käe üles ja lõi oma kepiga kaks korda kaljut, ja palju vett tuli välja ning kogudus ja nende loomad said juua.
\par 12 Aga Issand ütles Moosesele ja Aaronile: „Sellepärast et te ei uskunud minusse ega pidanud mind pühaks Iisraeli laste silme ees, ei saa teie viia seda kogudust sellele maale, mille mina neile annan!”
\par 13 See oli Meriba vesi, kus Iisraeli lapsed riidlesid Issandaga ja tema näitas ennast neile pühana.
\par 14 Kaadesist läkitas Mooses käskjalad Edomi kuninga juurde: „Nõnda ütleb su vend Iisrael: Sina tead kõiki raskusi, mis meid on tabanud,
\par 15 et meie vanemad läksid alla Egiptusesse ja me asusime kaua aega Egiptuses ja et egiptlased tegid kurja meile ja meie vanemaile.
\par 16 Aga me kisendasime Issanda poole ja tema kuulis meie häält ning läkitas ingli, kes tõi meid Egiptusest välja. Ja vaata, me oleme Kaadesis, linnas su maa piiril.
\par 17 Luba meid nüüd oma maast läbi minna! Me ei lähe läbi põldude ega viinamägede ja me ei joo kaevuvett. Me läheme kuningateed mööda ega pöördu paremat või vasakut kätt, kuni oleme läbinud su maa-ala.”
\par 18 Aga Edom vastas temale: „Sa ei tohi läbi minna! Muidu ma tulen mõõgaga su vastu.”
\par 19 Ja Iisraeli lapsed ütlesid temale: „Me läheme maanteed mööda ja kui me joome su vett, mina ja mu loomad, siis ma maksan selle eest. Ei midagi muud, kui et võiksin jala läbi minna!”
\par 20 Aga tema vastas: „Sa ei tohi läbi minna!” Ja Edom läks välja tema vastu hulga rahva ja kindla käega.
\par 21 Kui Edom keeldus lubamast Iisraeli oma maa-alast läbi minna, siis Iisrael pöördus temast eemale.
\par 22 Ja nad läksid teele Kaadesist. Ja Iisraeli lapsed, terve kogudus, tulid Hoori mäe juurde.
\par 23 Ja Issand rääkis Moosese ja Aaroniga Hoori mäe juures Edomi maa piiril, öeldes:
\par 24 „Aaron koristatakse oma rahva juurde, sest ta ei pääse sellele maale, mille ma annan Iisraeli lastele, sellepärast et te Meriba vee juures olite tõrksad mu käsu vastu.
\par 25 Võta Aaron ja tema poeg Eleasar ja vii nad Hoori mäele!
\par 26 Võta Aaronil riided seljast ja pane need selga ta pojale Eleasarile! Aaron koristatakse ära ja ta sureb seal.”
\par 27 Ja Mooses tegi nõnda, nagu Issand käskis, ja nad läksid üles Hoori mäele terve koguduse nähes.
\par 28 Ja Mooses võttis Aaronil riided seljast ning pani need selga ta pojale Eleasarile. Ja Aaron suri seal mäeharjal, aga Mooses ja Eleasar astusid mäelt alla.
\par 29 Ja kui terve kogudus nägi, et Aaron oli hinge heitnud, siis nad nutsid Aaroni pärast kolmkümmend päeva, kogu Iisraeli sugu.

\chapter{21}

\par 1 Kui Aradi kuningas, kaananlane, kes elas Lõunamaal, kuulis, et Iisrael tuli Atarimi teed mööda, siis ta sõdis Iisraeli vastu ja võttis neilt vange.
\par 2 Siis Iisrael andis Issandale tõotuse, öeldes: „Kui sa tõesti annad selle rahva minu kätte, siis ma hävitan nende linnad sootuks.”
\par 3 Ja Issand kuulis Iisraeli häält ja loovutas kaananlased ning nad hävitasid need ja nende linnad sootuks. Ja sellele paigale pandi nimeks Horma.
\par 4 Hoori mäe juurest läksid nad ära Kõrkjamere teed mööda, et minna ümber Edomimaa; aga rahvas tüdines teekonnal.
\par 5 Ja rahvas rääkis vastu Jumalale ja Moosesele: „Miks olete meid toonud Egiptusest kõrbe surema? Sest ei ole leiba ega vett ja meie hing tülkab seda viletsat toitu.”
\par 6 Siis Issand läkitas rahva sekka mürgiseid madusid ja need salvasid rahvast ning Iisraelis suri palju rahvast.
\par 7 Siis rahvas tuli Moosese juurde ja nad ütlesid: „Me tegime pattu, et rääkisime vastu Jumalale ja sinule. Palu Issandat, et ta võtaks meilt ära need maod!” Ja Mooses palvetas rahva eest.
\par 8 Ja Issand ütles Moosesele: „Tee enesele madu ja pane see ridva otsa, siis jääb elama iga salvatu, kes seda vaatab!”
\par 9 Ja Mooses tegi vaskmao ning pani selle ridva otsa. Kui siis madu oli salvanud kedagi, aga too vaatas vaskmadu, siis ta jäi elama.
\par 10 Ja Iisraeli lapsed läksid teele ja lõid leeri üles Obotisse.
\par 11 Ja nad läksid teele Obotist ja lõid leeri üles Ijje-Abarimi, Moabi ees olevasse kõrbe päikesetõusu pool.
\par 12 Sealt nad läksid teele ja lõid leeri üles Seredi nõkku.
\par 13 Sealt nad läksid teele ja lõid leeri üles teisele poole Arnoni jõge, mis on kõrbes ja mis lähtub emorlaste maa-alalt, sest Arnon on piiriks Moabi ja emorlaste vahel.
\par 14 Seepärast öeldakse „Issanda sõdade raamatus: ”Vaaheb Suufas ja Arnoni orud
\par 15 ja orgude nõlvakud, mis laskuvad Aari maa-alale ja naalduvad Moabi piirile.”
\par 16 Ja sealt nad läksid kaevu juurde. See oli see kaev, mille puhul Issand ütles Moosesele: „Kogu rahvas kokku, siis ma annan neile vett!”
\par 17 Tol korral laulis Iisrael seda laulu: „Kee üles, kaev! Laulgem temast!
\par 18 Kaev, mille vürstid kaevasid, mida rahva parimad süvendasid valitsuskepiga, oma karjasekeppidega.” Ja kõrbest läksid nad Mattanasse
\par 19 ja Mattanast Nahalieli ja Nahalielist Baamotti
\par 20 ja Baamotist sinna orgu, mis on Moabimaal Pisgaa mäetipu juures, mis vaatab alla kõrbepinnale.
\par 21 Ja Iisrael läkitas käskjalad Siihoni, emorlaste kuninga juurde ütlema:
\par 22 „Luba mind oma maast läbi minna! Me ei põika põldudele ega viinamägedele, me ei joo kaevuvett. Me läheme kuningateed mööda, kuni oleme läbinud su maa-ala.”
\par 23 Aga Siihon ei lubanud Iisraeli oma maa-alast läbi minna, vaid Siihon kogus kokku kogu oma rahva ja läks välja kõrbe Iisraeli vastu; ta tuli Jaasasse ja sõdis Iisraeli vastu.
\par 24 Aga Iisrael lõi teda mõõgateraga ja vallutas tema maa Arnonist kuni Jabbokini, kuni ammonlasteni, sest ammonlaste piir oli kindlustatud.
\par 25 Ja Iisrael võttis kõik need linnad ja elas kõigis emorlaste linnades, Hesbonis ja kõigis selle tütarlinnades.
\par 26 Sest Hesbon oli Siihoni, emorlaste kuninga linn ja tema oli sõdinud endise Moabi kuninga vastu ning oli tolle käest ära võtnud kogu tema maa kuni Arnonini.
\par 27 Seepärast ütlevad pilkelaulikud: „Tulge Hesboni! Ehitatagu üles ja rajatagu Siihoni linn!
\par 28 Sest tuli läks välja Hesbonist, leek Siihoni linnast. See põletas Moabi Aari, Arnoni küngaste isandad.
\par 29 Häda sulle, Moab! Kadunud oled, Kemosi rahvas! Ta tegi oma pojad põgenikeks ja oma tütred vangideks emorlaste kuningale Siihonile.
\par 30 Aga meie ambusime neid, Hesbon on hävinud kuni Diibonini; me laastasime, kuni sai süüdatud tuli, mis ulatus Meedebani.”
\par 31 Ja Iisrael jäi elama emorlaste maale.
\par 32 Siis Mooses läkitas uurima Jaaserit ja nad vallutasid selle tütarlinnad; emorlased, kes seal olid, aeti ära.
\par 33 Siis nad pöördusid ja läksid üles mööda Baasani teed. Aga Baasani kuningas Oog läks välja nende vastu, tema ja kogu ta rahvas, taplusesse Edreis.
\par 34 Aga Issand ütles Moosesele: „Ära karda teda, sest ma annan su kätte tema ja kogu ta rahva ja maa! Ja talita temaga, nagu sa talitasid Siihoniga, emorlaste kuningaga, kes elas Hesbonis!”
\par 35 Ja nad lõid maha tema ja ta pojad ja kogu ta rahva, laskmata põgeneda ainsatki. Ja nad vallutasid ta maa.

\chapter{22}

\par 1 Ja Iisraeli lapsed läksid teele ning lõid leeri üles Moabi lagendikele teisele poole Jordanit Jeeriko kohale.
\par 2 Aga Baalak, Sippori poeg, nägi kõike, mida Iisrael oli teinud emorlastega.
\par 3 Ja moabid kartsid väga seda rahvast, sest see oli suur; ja moabid värisesid hirmust Iisraeli laste ees.
\par 4 Ja moabid ütlesid Midjani vanemaile: „Nüüd sööb see jõuk paljaks kogu meie ümbruse, otsekui härg näsib väljalt rohu.” Baalak, Sippori poeg, oli sel ajal Moabi kuningas.
\par 5 Tema läkitas käskjalad Bileami, Beori poja juurde Petoori, mis on Frati jõe ääres Amau poegade maal, teda kutsuma, käskides öelda: „Vaata, Egiptusest on tulnud üks rahvas; näe, see ujutab maa üle ja on asunud elama mu naabrusesse.
\par 6 Tule nüüd ja nea mulle ära see rahvas, sest ta on minust vägevam! Vahest suudan, et me lööme teda ja ma ajan ta maalt välja, sest ma tean, et keda sina õnnistad, see on õnnistatud, ja keda sina nead, see on neetud.”
\par 7 Siis läksid Moabi vanemad ja Midjani vanemad ja neil oli kaasas tõekuulutaja tasu; nad tulid Bileami juurde ja andsid temale edasi Baalaki sõnad.
\par 8 Ja tema ütles neile: „Jääge selleks ööks siia, siis ma saadan teiega tagasi sõna, mis Issand mulle ütleb!” Ja Moabi vürstid jäid Bileami juurde.
\par 9 Ja Jumal tuli Bileami juurde ning küsis: „Kes on need mehed su juures?”
\par 10 Ja Bileam vastas Jumalale: „Baalak, Sippori poeg, Moabi kuningas, on läkitanud minu juurde:
\par 11 vaata, Egiptusest on tulnud üks rahvas ja ujutab maa üle. Tule nüüd, nea ta mulle ära, vahest suudan teda võitluses võita ja minema ajada!”
\par 12 Aga Jumal ütles Bileamile: „Ära mine koos nendega, sa ei tohi needa seda rahvast, sest ta on õnnistatud!”
\par 13 Ja Bileam tõusis hommikul üles ning ütles Baalaki vürstidele: „Minge oma maale, sest Issand ei luba mind minna koos teiega!”
\par 14 Ja Moabi vürstid võtsid kätte ja tulid Baalaki juurde ning ütlesid: „Bileam keeldus meiega kaasa tulemast.”
\par 15 Aga Baalak läkitas uuesti vürste, rohkem ja neist auväärsemaid.
\par 16 Ja need tulid Bileami juurde ning ütlesid temale: „Nõnda ütleb Baalak, Sippori poeg: Ära keeldu tulemast mu juurde,
\par 17 sest ma tahan osutada sulle väga suurt au ja ma teen kõik, mis sa minult nõuad. Tule siis ja nea mulle ära see rahvas!”
\par 18 Aga Bileam vastas ja ütles Baalaki sulastele: „Isegi kui Baalak annaks mulle oma koja täie hõbedat ja kulda, ei või ma üle astuda Issanda, oma Jumala käsust ei väikeses ega suures asjas.
\par 19 Aga jääge siis nüüd ka selleks ööks siia, et saaksin teada, mis Issand mulle veel ütleb!”
\par 20 Ja Jumal tuli öösel Bileami juurde ning ütles temale: „Kui need mehed on tulnud sind kutsuma, siis võta kätte, mine koos nendega, aga tee ainult seda, mis ma sulle ütlen!”
\par 21 Ja Bileam tõusis hommikul üles, saduldas oma emaeesli ja läks koos Moabi vürstidega.
\par 22 Aga kui ta läks, süttis Jumala viha põlema ja Issanda ingel seadis ennast teel temale vastu; tema aga ratsutas oma emaeesli seljas, ja ta kaks poissi olid koos temaga.
\par 23 Kui emaeesel nägi Issanda inglit tee peal seisvat, paljastatud mõõk käes, siis emaeesel põikas teelt ja läks väljale; aga Bileam lõi emaeeslit, et ta pöörduks teele.
\par 24 Siis seisis Issanda ingel viinamägede kitsasteel, millel oli müür kummalgi pool.
\par 25 Kui emaeesel nägi Issanda inglit, siis ta surus ennast seina ligi ja pigistas Bileami jala vastu seina; ja Bileam lõi teda veel korra.
\par 26 Siis Issanda ingel läks edasi ja seisis kitsas kohas, kus ei olnud teed pöördumiseks paremale ega vasakule.
\par 27 Kui emaeesel nägi Issanda inglit, siis ta heitis maha Bileami all; aga Bileami viha süttis põlema ja ta lõi emaeeslit kepiga.
\par 28 Siis Issand avas emaeesli suu ja too küsis Bileamilt: „Mis ma sulle olen teinud, et sa nüüd lõid mind kolm korda?”
\par 29 Ja Bileam vastas emaeeslile: „Sellepärast et sa tembutasid minuga. Kui mul oleks mõõk käes, ma tõesti tapaksin su nüüd!”
\par 30 Aga emaeesel ütles Bileamile: „Eks ma ole sinu emaeesel, kelle seljas sa oled ratsutanud kogu aja kuni tänapäevani? Kas mul on olnud viisiks sulle nõnda teha?„ Ja tema vastas: ”Ei!”
\par 31 Siis Issand tegi lahti Bileami silmad ja ta nägi Issanda inglit tee peal seisvat, paljastatud mõõk käes. Siis ta kummardas ja heitis silmili maha.
\par 32 Ja Issanda ingel ütles temale: „Mispärast sa lõid oma emaeeslit kolm korda? Vaata, ma olen su vastu välja astunud, sest hukutav on see tee minu ees.
\par 33 Emaeesel nägi mind ja põikas mu eest need kolm korda. Kui ta ei oleks pöördunud mu eest, ma tõesti oleksin nüüd tapnud sinu ja jätnud tema elama.”
\par 34 Siis Bileam ütles Issanda inglile: „Ma tegin pattu, sest ma ei teadnud, et sina seisid tee peal mu vastas. Kui see nüüd sinu silmis on paha, siis ma lähen tagasi.”
\par 35 Aga Issanda ingel vastas Bileamile: „Mine koos meestega, aga räägi ainult seda, mis mina sulle ütlen!” Ja Bileam läks koos Baalaki vürstidega.
\par 36 Kui Baalak kuulis, et Bileam tuli, siis ta läks temale vastu Moabimaa linna, mis oli Arnoni piiril, maa kaugeimas osas.
\par 37 Ja Baalak ütles Bileamile: „Kas ma ei ole tungivalt läkitanud su juurde sind kutsuma? Miks sa ei tulnud mu juurde? Kas ma tõesti ei suuda sind austada?”
\par 38 Aga Bileam vastas Baalakile: „Vaata, ma olen tulnud su juurde! Kas ma aga ise oskan nüüd midagi rääkida? Ma räägin seda, mis Jumal mulle suhu paneb.”
\par 39 Siis Bileam läks koos Baalakiga ja nad tulid Kirjat-Husotti.
\par 40 Ja Baalak tappis veiseid, lambaid ja kitsi ning läkitas Bileamile ja vürstidele, kes olid koos temaga.
\par 41 Aga järgmisel hommikul võttis Baalak Bileami ja viis tema üles Baali kõrgendikele, kust ta nägi tolle rahva äärt.

\chapter{23}

\par 1 Ja Bileam ütles Baalakile: „Ehita mulle siia seitse altarit ja muretse mulle siia seitse härjavärssi ja seitse jäära!”
\par 2 Ja Baalak tegi, nagu Bileam ütles; ja Baalak ja Bileam ohverdasid igal altaril härjavärsi ja jäära.
\par 3 Ja Bileam ütles Baalakile: „Jää oma põletusohvri juurde, mina aga lähen sinna! Vahest Issand tuleb mulle vastu. Ja mida tema mulle ilmutab, seda ma ütlen sinule.” Ja ta läks lagedale künkale.
\par 4 Ja Jumal tuli vastu Bileamile, kes ütles temale: „Ma püstitasin need seitse altarit ning ohverdasin igal altaril härjavärsi ja jäära.”
\par 5 Ja Issand pani Bileamile sõnad suhu ning ütles: „Mine tagasi Baalaki juurde ja räägi nõnda!”
\par 6 Siis ta läks tagasi tema juurde, ja vaata, ta seisis oma põletusohvri juures, tema ja kõik Moabi vürstid.
\par 7 Siis ta hakkas lausuma ja ütles: „Aramist tõi mind Baalak, ida mägedelt Moabi kuningas: „Tule, nea mulle Jaakob, tule sajata Iisraeli!”
\par 8 Kuidas ma võiksin needa, keda Jumal ei nea? Ja kuidas ma võiksin sajatada, keda Issand ei sajata?
\par 9 Sest ma näen teda kaljude tipust ja vaatan küngastelt: ennäe rahvast, kes elab eraldi ega arva ennast paganate hulka.
\par 10 Kes loendab Jaakobi põrmu ja kes arvab kokku Iisraeli tolmukübemed? Mu hing surgu õiglaste surma ja mu lõpp olgu nagu temal!”
\par 11 Siis Baalak ütles Bileamile: „Mis sa mulle tegid? Ma tõin sind oma vaenlasi needma, ja vaata, sa oled neid hoopis õnnistanud!”
\par 12 Aga tema vastas ja ütles: „Kas ma ei peaks ustavalt rääkima seda, mis Issand mu suhu paneb?”
\par 13 Siis Baalak ütles temale: „Tule nüüd koos minuga teise paika, kust sa näed seda rahvast. Sa näed küll ainult ta äärt, sa ei näe teda kogu ulatuses, aga nea ta mulle sealt!”
\par 14 Ja ta võttis tema kaasa valvurite väljakule, Pisgaa mäetippu, ehitas seitse altarit ning ohverdas igal altaril härjavärsi ja jäära.
\par 15 Siis Bileam ütles Baalakile: „Jää siia oma põletusohvri juurde, mina lähen sinna kohtamisele!”
\par 16 Ja Issand tuli Bileamile vastu, pani temale sõnad suhu ja ütles: „Mine tagasi Baalaki juurde ja räägi nõnda!”
\par 17 Siis ta tuli tema juurde, ja vaata, ta seisis oma põletusohvri juures ja koos temaga Moabi vürstid. Ja Baalak küsis temalt: „Mis Issand rääkis?”
\par 18 Siis ta hakkas rääkima ja ütles: „Tõuse, Baalak, ja kuule, pane mind tähele, Sippori poeg!
\par 19 Jumal ei ole inimene, et ta valetaks, inimlaps, et ta kahetseks. Kas tema ütleb, aga ei tee, või räägib, aga ei vii täide?
\par 20 Vaata, mind on kästud õnnistada. Tema on õnnistanud, mina ei saa seda tagasi võtta.
\par 21 Ei märgata häda Jaakobis ega nähta õnnetust Iisraelis. Issand, ta Jumal, on temaga ja rõõmuhääl kuningast tema keskel.
\par 22 Jumal, kes tõi nad Egiptusest, on neile otsekui sarved metshärjale.
\par 23 Sellepärast ei ole lausumist Jaakobi vastu ega loitsimist Iisraeli vastu. Küll öeldakse Jaakobist ja Iisraelist: Vaata, mis Jumal on teinud!
\par 24 Vaata, rahvas tõuseb nagu emalõvi, ajab ennast püsti nagu lõvi. Ei ta heida maha enne, kui saak on söödud ja tapetute veri on joodud.”
\par 25 Aga Baalak ütles Bileamile: „Kui sa teda ei nea, siis ära teda ometi õnnista!”
\par 26 Bileam aga vastas ja ütles Baalakile: „Eks ma ole sulle rääkinud ja öelnud: Ma pean tegema kõike, mida Issand käsib.”
\par 27 Siis Baalak ütles Bileamile: „Tule nüüd, ma võtan sind kaasa teise paika. Vahest on see õige Jumala silmis, et sa nead rahva mulle sealt!”
\par 28 Ja Baalak võttis Bileami enesega kaasa mäetippu, mis kerkib üle kõrbepinna.
\par 29 Siis Bileam ütles Baalakile: „Ehita mulle siia seitse altarit ja muretse mulle siia seitse härjavärssi ja seitse jäära!”
\par 30 Ja Baalak tegi, nagu Bileam ütles, ja ta ohverdas igal altaril härjavärsi ja jäära.

\chapter{24}

\par 1 Kui Bileam nägi, et Iisraeli õnnistamine oli Issanda silmis hea, siis ei läinud ta mitte nagu eelmistel kordadel otsima endemärke, vaid pööras oma näo kõrbe poole.
\par 2 Ja kui Bileam oma silmad üles tõstis, siis ta nägi Iisraeli asuvat leeris oma suguharude kaupa. Ja tema peale tuli Jumala Vaim.
\par 3 Ta hakkas rääkima ja ütles: „Nõnda kuulutab Bileam, Beori poeg, nõnda kõneleb avatud silmaga mees,
\par 4 nõnda kuulutab Jumala kõnede kuulaja, kes näeb Kõigeväelise nägemusi, mahalangenuna avatud silmil:
\par 5 kui kaunid on su telgid, Jaakob, su eluasemed, oh Iisrael!
\par 6 Nagu laiuvad orud, nagu rohuaiad jõe kaldal, nagu Issanda istutatud aaloepuud, nagu seedripuud vete ääres.
\par 7 Vesi ta astjaist voolab üle ja ta külvil on palju vett. Tema kuningas on Agagist vägevam ja ta kuningriik ülendab ennast.
\par 8 Jumal, kes tõi tema Egiptusest, on talle otsekui sarved metshärjale. Ta neelab rahvaid, oma vaenlasi, ta murrab nende luid, oma nooltega purustab neid.
\par 9 Ta on laskunud lebama, ta lamab nagu lõvi või emalõvi, kes julgeks teda äratada? Õnnistatud olgu, kes sind õnnistavad, neetud, kes sind neavad!”
\par 10 Siis Baalaki viha süttis põlema Bileami vastu ja ta lõi oma käed kokku. Ja Baalak ütles Bileamile: „Ma kutsusin sind needma mu vaenlasi, ja vaata, sa oled neid kolm korda õnnistanud.
\par 11 Mine nüüd, põgene koju! Ma kavatsesin sind väga austada, aga näe, Issand on keelanud sind austada!”
\par 12 Ja Bileam vastas Baalakile: „Eks ma rääkinud juba su käskjalgadele, keda sa saatsid minu juurde, öeldes:
\par 13 Isegi kui Baalak annaks oma koja täie hõbedat ja kulda, ei võiks ma üle astuda Issanda käsust, tehes head või kurja omaenese südame järgi. Mis Issand räägib, seda räägin ka mina!
\par 14 Ja nüüd, vaata, ma lähen oma rahva juurde. Tule, ma kuulutan sulle, mida see rahvas teeb sinu rahvaga tulevasil päevil!”
\par 15 Ja ta hakkas rääkima ning ütles: „Nõnda kuulutab Bileam, Beori poeg, nõnda kõneleb avatud silmaga mees,
\par 16 nõnda kuulutab Jumala kõnede kuulaja ja Kõigekõrgema tarkuse tundja, kes näeb Kõigeväelise nägemusi, mahalangenuna avatud silmil.
\par 17 Ma näen teda, aga mitte nüüd, ma silmitsen teda, aga mitte ligidalt: Jaakobist tõuseb täht, Iisraelist kerkib valitsuskepp. See purustab Moabi oimud ja kõigi Seti poegade pealaed.
\par 18 Edom saab alistatud maaks ja Seir alluvaks oma vaenlastele, sest Iisrael teeb vägitegusid.
\par 19 Jaakobist tuleb valitseja ja ta hävitab linnast põgenenu.”
\par 20 Ja kui ta nägi Amalekki, siis ta hakkas rääkima ning ütles: „Rahvaist esimene on Amalek, aga ta lõpp on häving igavesti!”
\par 21 Ja kui ta nägi keenlasi, siis ta hakkas rääkima ning ütles: „Su eluase on püsiv ja su pesa on pandud kaljule.
\par 22 Aga siiski hävitatakse Kain! Kui kaua veel? Ja Assur viib sind vangi!”
\par 23 Ja ta hakkas rääkima ning ütles: „Oh häda! Kes jääb elama, kui Jumal seda teeb?
\par 24 Need, kes lähevad välja kittide rannast, alandavad Assurit ja alandavad Eberit, aga ka tema hävib igavesti.”
\par 25 Seejärel Bileam võttis kätte ja asus teele ning läks tagasi koju. Ka Baalak läks oma teed.

\chapter{25}

\par 1 Kui Iisrael elas Sittimis, siis hakkas rahvas tegema hooratööd Moabi tütardega,
\par 2 kes kutsusid rahvast oma jumalate ohvriteenistustele; ja rahvas sõi ning hakkas kummardama nende jumalaid.
\par 3 Nõnda hoidis Iisrael Baal-Peori poole. Aga Issanda viha süttis põlema Iisraeli vastu
\par 4 ja Issand ütles Moosesele: „Võta kõik rahva peamehed ja poo nad päikese käes Issandale, et Issanda tuline viha pöörduks Iisraeli pealt!”
\par 5 Ja Mooses ütles Iisraeli kohtumõistjaile: „Tapke igaüks oma meestest see, kes on hoidnud Baal-Peori poole!”
\par 6 Ja vaata, keegi mees Iisraeli lastest tuli ja tõi oma vendade juurde Midjani naise, Moosese ja kogu Iisraeli laste koguduse silma all, kui need parajasti nutsid kogudusetelgi ukse ees.
\par 7 Kui Piinehas, preester Aaroni poja Eleasari poeg, seda nägi, siis ta tõusis üles koguduse keskelt, võttis piigi kätte
\par 8 ja läks Iisraeli mehele järele naistekambrisse ning torkas mõlemale kõhtu, niihästi Iisraeli mehele kui sellele naisele. Siis võeti nuhtlus ära Iisraeli laste pealt.
\par 9 Sellesse nuhtlusesse surnuid oli kakskümmend neli tuhat.
\par 10 Ja Issand rääkis Moosesega, öeldes:
\par 11 „Piinehas, preester Aaroni poja Eleasari poeg, on pööranud ära mu vihaleegi Iisraeli laste pealt, olles nende keskel vihastunud minu asemel, nõnda et minul ei olnud vaja oma vihas teha lõppu Iisraeli lastele.
\par 12 Seepärast ütle: Vaata, ma teen temaga oma rahulepingu.
\par 13 See olgu temale ja ta järeltulevale soole igaveseks preestriameti lepinguks, sellepärast et ta vihastus oma Jumala asemel ja toimetas lepitust Iisraeli laste eest.”
\par 14 Surmatud Iisraeli mehe nimi, kes surmati ühes Midjani naisega, oli Simri, Salu poeg, siimeonlaste perekonna vürst.
\par 15 Ja surmatud Midjani naise nimi oli Kosbi, Suuri tütar; Suur oli Midjani sugukondade ühe perekonna peamees.
\par 16 Ja Issand rääkis Moosesega, öeldes:
\par 17 „Tungige kallale midjanlastele ja lööge nad maha!
\par 18 Sest nad on teile kallale tunginud oma salakavalusega, mida nad tarvitasid teie vastu Peori juhtumi puhul, ja Kosbi, Midjani vürsti tütre, nende kaasmaalanna juhtumi puhul, kes surmati nuhtlusepäeval, mis oli Peori juhtumi pärast.”

\chapter{26}

\par 1 Pärast seda nuhtlust rääkis Issand Moosesega ja preester Aaroni poja Eleasariga, öeldes:
\par 2 „Võtke arvele kogu Iisraeli laste koguduse pead, kahekümneaastased ja üle selle, kõik sõjakõlvulised nende perekondade kaupa Iisraelis!”
\par 3 Ja Mooses ja preester Eleasar rääkisid nendega Moabi lagendikel Jordani ääres Jeeriko kohal, öeldes:
\par 4 Kahekümneaastasi ja üle selle - nagu Issand oli Moosest käskinud - ja Egiptusemaalt tulnud Iisraeli lapsi oli:
\par 5 Ruuben, Iisraeli esmasündinu; Ruubeni järeltulijad olid: Hanokist hanoklaste suguvõsa, Pallust pallulaste suguvõsa;
\par 6 Hesronist hesronlaste suguvõsa, Karmist karmlaste suguvõsa.
\par 7 Need olid ruubenlaste suguvõsad ja neist loetuid oli nelikümmend kolm tuhat seitsesada kolmkümmend.
\par 8 Pallu poeg oli Eliab.
\par 9 Eliabi pojad olid Nemuel, Daatan ja Abiram; need olid Daatan ja Abiram, kogudusest kutsutud mehed, kes riidlesid Moosesega ja Aaroniga Korahi jõugus, kui need riidlesid Issandaga
\par 10 ja maa avas oma suu ja neelas nemad ja Korahi - siis kui see jõuk suri ja kakssada viiskümmend meest põles tules - ja nad said hoiatusmärgiks.
\par 11 Aga Korahi pojad ei surnud.
\par 12 Siimeoni järeltulijad olid oma suguvõsade kaupa: Nemuelist nemuellaste suguvõsa, Jaaminist jaaminlaste suguvõsa, Jaakinist jaakinlaste suguvõsa;
\par 13 Serahist serahlaste suguvõsa, Saulist saullaste suguvõsa.
\par 14 Need olid siimeonlaste suguvõsad - kakskümmend kaks tuhat kuussada.
\par 15 Gaadi järeltulijad olid oma suguvõsade kaupa: Sefonist sefonlaste suguvõsa, Haggist haggilaste suguvõsa, Suunist suunlaste suguvõsa;
\par 16 Osnist osnilaste suguvõsa, Eerist eerlaste suguvõsa;
\par 17 Arodist arodlaste suguvõsa, Areelist areellaste suguvõsa.
\par 18 Need olid Gaadi poegade suguvõsad; neist loetuid oli nelikümmend tuhat viissada.
\par 19 Juuda pojad olid Eer ja Oonan, aga Eer ja Oonan surid Kaananimaal.
\par 20 Ja need olid Juuda järeltulijad oma suguvõsade kaupa: Seelast seelalaste suguvõsa, Peretsist peretslaste suguvõsa, Serahist serahlaste suguvõsa.
\par 21 Ja need olid Peretsi järeltulijad: Hesronist hesronlaste suguvõsa, Haamulist haamullaste suguvõsa.
\par 22 Need olid Juuda suguvõsad; neist loetuid oli seitsekümmend kuus tuhat viissada.
\par 23 Issaskari järeltulijad olid oma suguvõsade kaupa: Toolast toolalaste suguvõsa, Puuast puualaste suguvõsa;
\par 24 Jaasubist jaasublaste suguvõsa, Simronist simronlaste suguvõsa.
\par 25 Need olid Issaskari suguvõsad; neist loetuid oli kuuskümmend neli tuhat kolmsada.
\par 26 Sebuloni järeltulijad olid oma suguvõsade kaupa: Seredist seredlaste suguvõsa, Eelonist eelonlaste suguvõsa, Jahleelist jahleellaste suguvõsa.
\par 27 Need olid sebulonlaste suguvõsad; neist loetuid oli kuuskümmend tuhat viissada.
\par 28 Joosepi pojad olid Manasse ja Efraim; oma suguvõsade kaupa
\par 29 olid Manasse järeltulijad: Maakirist maakirlaste suguvõsa. Maakirile sündis Gilead; Gileadist on gileadlaste suguvõsa.
\par 30 Need olid Gileadi järeltulijad: Jeserist jeserlaste suguvõsa, Heelekist heeleklaste suguvõsa;
\par 31 Asrielist asriellaste suguvõsa, Sekemist sekemlaste suguvõsa;
\par 32 Semidast semidlaste suguvõsa ja Heeferist heeferlaste suguvõsa.
\par 33 Aga Selofhadil, Heeferi pojal, ei olnud poegi, vaid olid ainult tütred; ja Selofhadi tütarde nimed olid Mahla, Noa, Hogla, Milka ja Tirsa.
\par 34 Need olid Manasse suguvõsad; neist loetuid oli viiskümmend kaks tuhat seitsesada.
\par 35 Need olid Efraimi järeltulijad oma suguvõsade kaupa: Suutelahist suutelahlaste suguvõsa, Bekerist bekerlaste suguvõsa, Tahanist tahanlaste suguvõsa.
\par 36 Need olid Suutelahi järeltulijad: Eeranist eeranlaste suguvõsa.
\par 37 Need olid Efraimi poegade suguvõsad; neist loetuid oli kolmkümmend kaks tuhat viissada.
\par 38 Benjamini järeltulijad olid oma suguvõsade kaupa: Belast belalaste suguvõsa, Asbelist asbellaste suguvõsa, Ahiramist ahiramlaste suguvõsa;
\par 39 Suufamist suufamlaste suguvõsa, Huufamist huufamlaste suguvõsa.
\par 40 Aga Bela pojad olid Ard ja Naaman: Ardist ardlaste suguvõsa, Naamanist naamanlaste suguvõsa.
\par 41 Need olid Benjamini järeltulijad oma suguvõsade kaupa; neist loetuid oli nelikümmend viis tuhat kuussada.
\par 42 Need olid Daani järeltulijad oma suguvõsade kaupa; Suuhamist suuhamlaste suguvõsa. Need olid Daani suguvõsad oma suguvõsade kaupa.
\par 43 Kõiki suuhamlaste suguvõsadest loetuid oli kuuskümmend neli tuhat nelisada.
\par 44 Aaseri järeltulijad olid oma suguvõsade kaupa: Jimnast jimnalaste suguvõsa, Jisvist jisvilaste suguvõsa, Berijast berijalaste suguvõsa.
\par 45 Berija järeltulijad olid: Heberist heberlaste suguvõsa, Malkielist malkiellaste suguvõsa.
\par 46 Ja Aaseri tütre nimi oli Saarah.
\par 47 Need olid Aaseri järeltulijate suguvõsad; neist loetuid oli viiskümmend kolm tuhat nelisada.
\par 48 Naftali järeltulijad olid oma suguvõsade kaupa: Jahselist jahsellaste suguvõsa, Guunist guunlaste suguvõsa;
\par 49 Jeeserist jeeserlaste suguvõsa, Sillemist sillemlaste suguvõsa.
\par 50 Need olid Naftali suguvõsad oma suguvõsade kaupa; neist loetuid oli nelikümmend viis tuhat nelisada.
\par 51 Neid, keda Iisraeli lastest loeti, oli kuussada üks tuhat seitsesada kolmkümmend.
\par 52 Ja Issand rääkis Moosesega, öeldes:
\par 53 „Maa jaotatagu neile pärisosaks vastavalt nimede arvule:
\par 54 suuremale suguharule anna pärisosaks rohkem ja väiksemale vähem; igaühele antagu ta pärisosa vastavalt ta loetuile!
\par 55 Aga maa jaotatagu liisu läbi; nad saagu oma pärisosad oma vanemate suguharude nimede järgi!
\par 56 Pärisosa jaotatagu nende vahel liisu läbi, olgu neid palju või pisut!”
\par 57 Ja need olid leviitidest loetud oma suguvõsade kaupa: Geersonist geersonlaste suguvõsa, Kehatist kehatlaste suguvõsa, Merarist merarlased.
\par 58 Need oli leviitide suguvõsad: libnilaste suguvõsa, hebronlaste suguvõsa, mahlilaste suguvõsa, muusilaste suguvõsa, korahlaste suguvõsa. Ja Kehatile oli sündinud Amram.
\par 59 Ja Amrami naise nimi oli Jookebed, Leevi tütar, kes Leevile oli sündinud Egiptuses, ja too tõi Amramile ilmale Aaroni ja Moosese ja nende õe Mirjami.
\par 60 Ja Aaronile sündisid Naadab, Abihu, Eleasar ja Iitamar.
\par 61 Aga Naadab ja Abihu surid, kui nad viisid võõra tule Issanda ette.
\par 62 Ja neist oli loetuid kakskümmend kolm tuhat, kõik meesterahvad ühe kuu vanustest ja üle selle, sest neid ei olnud loetud Iisraeli laste hulka, kuna neile ei olnud antud pärisosa Iisraeli laste seas.
\par 63 Need olid ära loetud Moosese ja preester Eleasari poolt, kui nad lugesid Iisraeli lapsi Moabi lagendikel Jordani ääres Jeeriko kohal.
\par 64 Nende seas ei olnud kedagi, kes oleks olnud ära loetud Moosese ja preester Aaroni poolt, kui nad Iisraeli lapsi lugesid Siinai kõrbes,
\par 65 sest Issand oli neile öelnud, et nad peavad surema kõrbes; seepärast ei olnud neist enam alles kedagi peale Kaalebi, Jefunne poja, ja Joosua, Nuuni poja.

\chapter{27}

\par 1 Siis astusid ette Selofhadi tütred; Selofhad oli Heeferi poeg, kes oli Gileadi poeg, kes oli Maakiri poeg, kes oli Manasse poeg Joosepi poja Manasse suguvõsast; ja need olid tema tütarde nimed: Mahla, Noa, Hogla, Milka ja Tirsa.
\par 2 Ja nad seisid Moosese ja preester Eleasari ja vürstide ja terve koguduse ees kogudusetelgi ukse juures, öeldes:
\par 3 „Meie isa suri kõrbes, aga ta ei olnud selles jõugus, kes kogunes Issanda vastu, Korahi jõugus, vaid ta suri oma patu pärast ja tal ei olnud poegi.
\par 4 Miks peaks kaduma meie isa nimi tema suguvõsa keskelt, sellepärast et tal ei olnud poega? Anna meile pärisosa meie isa vendade keskel!”
\par 5 Ja Mooses viis nende nõudeasja Issanda ette.
\par 6 Ja Issand rääkis Moosesega, öeldes:
\par 7 „Selofhadi tütred räägivad õigesti. Anna neile tõesti maaomand pärisosaks nende isa vendade keskel; anna neile üle nende isa pärisosa!
\par 8 Ja räägi Iisraeli lastega, öeldes: Kui keegi sureb ja tal ei ole poega, siis andke ta pärisosa üle tema tütrele!
\par 9 Aga kui tal ei ole tütart, siis andke ta pärisosa tema vendadele!
\par 10 Aga kui tal ei ole vendi, siis andke ta pärisosa tema isa vendadele!
\par 11 Aga kui ta isal ei ole vendi, siis andke tema pärisosa ligemale veresugulasele ta suguvõsast, ja tema võtku see oma valdusesse! See olgu Iisraeli lastele seadluseks Issanda poolt Moosesele antud käsu kohaselt!”
\par 12 Ja Issand ütles Moosesele: „Mine üles sellele Abarimi mäele ja vaata maad, mille ma annan Iisraeli lastele!
\par 13 Ja kui sa oled seda vaadanud, siis koristatakse sind su rahva juurde, nagu koristati su vend Aaron,
\par 14 sellepärast et te Siini kõrbes, kui kogudus riidles, panite vastu mu käsule ega pühitsenud mind nende nähes vee muretsemisega.” See oli Kaadesi Meriba vesi Siini kõrbes.
\par 15 Ja Mooses rääkis Issandaga, öeldes:
\par 16 „Pangu Issand, kõige liha vaimude Jumal, koguduse üle üks mees,
\par 17 kes läheks välja nende ees ja kes tuleks tagasi nende ees, kes viiks nad välja ja kes tooks nad tagasi, et Issanda kogudus ei oleks nagu kari, kellel ei ole karjast!”
\par 18 Ja Issand ütles Moosesele: „Võta Joosua, Nuuni poeg, mees, kelles on Vaim, ja pane oma käsi tema peale!
\par 19 Ja pane ta seisma preester Eleasari ja terve koguduse ette ning anna temale nende nähes kohustus!
\par 20 Pane tema peale osa oma väärikusest, et terve Iisraeli laste kogudus võtaks teda kuulda!
\par 21 Siis ta astugu preester Eleasari ette ja too küsigu temale uurimi otsust Issanda ees; tema käsul nad mingu välja ja tema käsul nad tulgu tagasi, tema ise ja kõik Iisraeli lapsed koos temaga, terve kogudus!”
\par 22 Ja Mooses tegi, nagu Issand oli teda käskinud, ja võttis Joosua ning pani ta seisma preester Eleasari ja terve koguduse ette.
\par 23 Ja ta pani oma käed tema peale ja andis temale kohustuse, nagu Issand oli käskinud Moosese läbi.

\chapter{28}

\par 1 Ja Issand rääkis Moosesega, öeldes:
\par 2 „Käsi Iisraeli lapsi ja ütle neile: Pidage meeles, et te peate tooma selleks seatud ajal mu ohvrianni, mu leiva, mulle healõhnaliseks tuleohvriks!
\par 3 Ja ütle neile: See tuleohver, mille te peate tooma Issandale, on niisugune: iga päev kaks aastast veatut oinastalle alaliseks põletusohvriks;
\par 4 üks tall ohverda hommikul ja teine tall ohverda õhtul;
\par 5 ja roaohvriks olgu kaks toopi peent jahu, segatud kolme kortli tambitud õliga!
\par 6 See on alaline põletusohver, mis Siinai mäe juures seati healõhnaliseks tuleohvriks Issandale.
\par 7 Ja selle juurde olgu joogiohvriks kolm kortlit ühe talle kohta; joogiohver vägijoogist valatagu Issandale pühamus!
\par 8 Ja teine tall ohverda õhtul, samuti nagu hommikune koos roaohvri ja selle joogiohvriga; ohverda see healõhnaliseks tuleohvriks Issandale!
\par 9 Aga hingamispäeval olgu kaks aastast veatut oinastalle ja roaohvriks kaks kannu õliga segatud peent jahu koos selle juurde kuuluva joogiohvriga!
\par 10 See on hingamispäeva põletusohver igal hingamispäeval, peale alalise põletusohvri ja selle juurde kuuluva joogiohvri.
\par 11 Ja iga kuu esimesel päeval tooge Issandale põletusohvriks kaks noort härjavärssi, üks jäär ja seitse aastast veatut oinastalle,
\par 12 ja pool külimittu õliga segatud peent jahu roaohvriks ühe härjavärsi kohta ja kaks kannu õliga segatud peent jahu roaohvriks jäära kohta,
\par 13 ja üks kann õliga segatud peent jahu roaohvriks ühe talle kohta; see on põletusohver, healõhnaline tuleohver Issandale.
\par 14 Ja nende juurde kuuluvaks joogiohvriks olgu poolteist toopi veini härjavärsi, toop jäära ja kolm kortlit talle kohta; see on noorkuu põletusohver igas aasta kuus.
\par 15 Ja peale alalise põletusohvri ning selle juurde kuuluva joogiohvri tuleb ohverdada Issandale patuohvriks üks noor sikk.
\par 16 Ja esimese kuu neljateistkümnendal päeval on paasapüha Issanda auks.
\par 17 Ja sama kuu viieteistkümnendal päeval on püha; seitse päeva söödagu hapnemata leiba!
\par 18 Esimesel päeval on pühalik kokkutulek; ühtegi argipäevatööd ärgu tehtagu!
\par 19 Ja tooge tuleohvriks, põletusohvriks Issandale kaks noort härjavärssi, üks jäär ja seitse aastast oinastalle; need olgu teil veatud!
\par 20 Ja nende juurde kuuluvaks roaohvriks ohverdage õliga segatud peent jahu, pool külimittu härjavärsi ja kaks kannu jäära kohta!
\par 21 Aga iga talle kohta neist seitsmest tallest ohverda üks kann!
\par 22 Ja teie eest lepituse toimetamiseks olgu patuohvriks üks sikk!
\par 23 Need ohverdage peale hommikuse põletusohvri, mis on alaliseks põletusohvriks!
\par 24 Sel viisil ohverdage seitse päeva iga päev leiba, healõhnalist tuleohvrit Issandale; ohverdage seda koos selle juurde kuuluva joogiohvriga lisaks alalisele põletusohvrile!
\par 25 Ja seitsmendal päeval olgu teil pühalik kokkutulek; ühtegi argipäevatööd ärgu tehtagu!
\par 26 Ja uudsevilja päeval, kui te toote uudse roaohvri Issandale oma nädalatepühal, olgu teil pühalik kokkutulek; ühtegi argipäevatööd ärgu tehtagu!
\par 27 Siis tooge healõhnaliseks põletusohvriks Issandale kaks noort härjavärssi, üks jäär ja seitse aastast oinastalle
\par 28 ning nende juurde kuuluvaks roaohvriks õliga segatud peent jahu, pool külimittu härjavärsi, kaks kannu jäära
\par 29 ja kann iga talle kohta neist seitsmest tallest;
\par 30 ja üks noor sikk teie eest lepituse toimetamiseks!
\par 31 Ohverdage need lisaks alalisele põletusohvrile ja selle juurde kuuluvale roaohvrile - need olgu teil veatud - ja nende juurde kuulugu joogiohvrid!

\chapter{29}

\par 1 Ja seitsmenda kuu esimesel päeval olgu teil pühalik kokkutulek; ühtegi argipäevatööd ärgu tehtagu; see olgu teile sarve puhumise päevaks!
\par 2 Siis ohverdage põletusohvriks, Issandale meeldivaks lõhnaks, üks noor härjavärss, üks jäär, seitse aastast veatut oinastalle
\par 3 ja nende juurde kuuluvaks roaohvriks õliga segatud peent jahu, pool külimittu härjavärsi, kaks kannu jäära
\par 4 ja üks kann iga talle kohta neist seitsmest tallest,
\par 5 ja üks noor sikk patuohvriks, teie eest lepituse toimetamiseks!
\par 6 See olgu lisaks noorkuu põletusohvrile ja selle juurde kuuluvale roaohvrile, samuti alalisele põletusohvrile ja selle juurde kuuluvale roaohvrile ning nende juurde kuuluvaile kohustuslikele joogiohvritele kui healõhnaline tuleohver Issandale!
\par 7 Ja sellesama seitsmenda kuu kümnendal päeval olgu teil pühalik kokkutulek, siis alandage oma hinged, ühtegi tööd ärge tehke!
\par 8 Siis ohverdage põletusohvriks, Issandale meeldivaks lõhnaks, üks noor härjavärss, üks jäär, seitse aastast oinastalle; need olgu teil veatud!
\par 9 Ja nende juurde kuulugu roaohvrina õliga segatud peen jahu, pool külimittu härjavärsi, kaks kannu jäära
\par 10 ja üks kann iga talle kohta neist seitsmest tallest!
\par 11 Üks noor sikk olgu patuohvriks peale lepitus-patuohvri ja alalise põletusohvri koos selle juurde kuuluva roaohvri ja joogiohvritega.
\par 12 Ja seitsmenda kuu viieteistkümnendal päeval olgu teil pühalik kokkutulek; ühtegi argipäevatööd ärge tehke, ja pidage seda püha Issanda auks seitse päeva!
\par 13 Siis ohverdage põletusohvriks, healõhnaliseks tuleohvriks Issandale, kolmteist noort härjavärssi, kaks jäära, neliteist aastast oinastalle; need olgu veatud!
\par 14 Ja nende juurde kuulugu roaohvrina õliga segatud peen jahu, pool külimittu iga härjavärsi kohta neist kolmeteistkümnest härjavärsist, kaks kannu iga jäära kohta neist kahest jäärast
\par 15 ja üks kann iga talle kohta neist neljateistkümnest tallest!
\par 16 Ja üks noor sikk olgu patuohvriks, peale alalise põletusohvri, selle juurde kuuluva roaohvri ja joogiohvri!
\par 17 Ja teisel päeval: kaksteist noort härjavärssi, kaks jäära, neliteist aastast veatut oinastalle
\par 18 ja nende juurde kuuluv roaohver koos joogiohvritega, vastavalt härjavärsside, jäärade ja tallede arvule eeskirja kohaselt.
\par 19 Ja üks noor sikk olgu patuohvriks, peale alalise põletusohvri, selle juurde kuuluva roaohvri ja joogiohvri!
\par 20 Ja kolmandal päeval: üksteist härjavärssi, kaks jäära, neliteist aastast veatut oinastalle
\par 21 ja nende juurde kuuluv roaohver koos joogiohvritega, vastavalt härjavärsside, jäärade ja tallede arvule eeskirja kohaselt.
\par 22 Ja üks noor sikk olgu patuohvriks, peale alalise põletusohvri, selle juurde kuuluva roaohvri ja joogiohvri!
\par 23 Ja neljandal päeval: kümme härjavärssi, kaks jäära, neliteist aastast veatut oinastalle
\par 24 ja nende juurde kuuluv roaohver koos joogiohvritega, vastavalt härjavärsside, jäärade ja tallede arvule eeskirja kohaselt.
\par 25 Ja üks noor sikk olgu patuohvriks, peale alalise põletusohvri, selle juurde kuuluva roaohvri ja joogiohvri!
\par 26 Ja viiendal päeval: üheksa härjavärssi, kaks jäära, neliteist aastast veatut oinastalle
\par 27 ja nende juurde kuuluv roaohver koos joogiohvritega, vastavalt härjavärsside, jäärade ja tallede arvule eeskirja kohaselt.
\par 28 Ja üks noor sikk olgu patuohvriks, peale alalise põletusohvri, selle juurde kuuluva roaohvri ja joogiohvri!
\par 29 Ja kuuendal päeval: kaheksa härjavärssi, kaks jäära, neliteist aastast veatut oinastalle
\par 30 ja nende juurde kuuluv roaohver koos joogiohvritega, vastavalt härjavärsside, jäärade ja tallede arvule eeskirja kohaselt.
\par 31 Ja üks noor sikk olgu patuohvriks, peale alalise põletusohvri, selle juurde kuuluva roaohvri ja joogiohvri!
\par 32 Ja seitsmendal päeval: seitse härjavärssi, kaks jäära, neliteist aastast veatut oinastalle
\par 33 ja nende juurde kuuluv roaohver koos joogiohvritega, vastavalt härjavärsside, jäärade ja tallede arvule eeskirja kohaselt.
\par 34 Ja üks noor sikk olgu patuohvriks, peale alalise põletusohvri, selle juurde kuuluva roaohvri ja joogiohvri!
\par 35 Kaheksandal päeval olgu teil lõpetuspüha; ühtegi argipäevatööd ärgu tehtagu!
\par 36 Siis ohverdage põletusohvriks, healõhnaliseks tuleohvriks Issandale, üks härjavärss, üks jäär, seitse aastast veatut oinastalle
\par 37 ja nende juurde kuuluv roaohver koos joogiohvritega, vastavalt härjavärsi, jäära ja tallede arvule eeskirja kohaselt!
\par 38 Ja üks noor sikk olgu patuohvriks, peale alalise põletusohvri, selle juurde kuuluva roaohvri ja joogiohvri!
\par 39 Need ohverdage Issandale oma seatud pühil peale oma tõotus- ja vabatahtlike ohvrite põletusohvreiks, roa-, joogi- ja tänuohvreiks!”

\chapter{30}

\par 1 Ja Mooses rääkis Iisraeli lastele kõik, mida Issand oli Moosest käskinud.
\par 2 Ja Mooses rääkis Iisraeli laste suguharude peameestega, öeldes: „Nõnda on Issand käskinud:
\par 3 Kui keegi mees annab Issandale tõotuse või vannub vande, võttes oma hinge peale loobumistõotuse, siis ei tohi ta murda oma sõna: ta peab tegema kõik, mis ta suust on välja tulnud!
\par 4 Ja kui keegi naine annab Issandale tõotuse ja võtab enesele loobumistõotuse, olles noorpõlves oma isakojas,
\par 5 ja tema isa kuuleb ta lubadust ja loobumistõotust, mille ta on võtnud oma hinge peale, aga tema isa ei ütle temale sõnagi, siis peavad kõik ta lubadused jääma jõusse, ja kõik loobumistõotused, mis ta on võtnud oma hinge peale, peavad jääma jõusse!
\par 6 Aga kui ta isa keelab teda päeval, mil ta seda kuuleb: siis ei jää jõusse ükski ta lubadus ja loobumistõotus, mille ta on võtnud oma hinge peale, ja Issand annab temale andeks, sellepärast et ta isa on teda keelanud.
\par 7 Ja kui ta saab mehele ja temal on täita tõotus või mõtlemata sõna huultelt, mille ta on võtnud oma hinge peale,
\par 8 ja ta mees kuuleb seda, aga ei lausu temale ühtegi sõna sel päeval, mil ta seda kuuleb, siis peavad jääma jõusse tema lubadused ja loobumistõotused, mis ta on võtnud oma hinge peale!
\par 9 Aga kui ta mees sel päeval, mil ta seda kuuleb, keelab teda, siis ta tühistab tema tõotuse, mis tal oli täita, ja selle mõtlemata sõna huultelt, mille ta oli võtnud oma hinge peale, ja Issand annab temale andeks.
\par 10 Aga lese või hüljatu tõotus, kõik, mis ta on võtnud oma hinge peale, jäägu jõusse!
\par 11 Ja kui naine tõotab oma mehe kojas või võtab vandega oma hinge peale loobumistõotuse
\par 12 ja tema mees kuuleb, aga ei lausu temale sõnagi ega keela teda, siis jäävad jõusse kõik ta lubadused ja kõik loobumistõotused, mis ta on võtnud oma hinge peale.
\par 13 Aga kui ta mees sel päeval, mil ta neid kuuleb, tühistab need täiesti, siis ei jää jõusse ükski ta huultelt tulnud lubadus või hinge peale võetud loobumistõotus. Ta mees on need tühistanud ja Issand annab temale andeks.
\par 14 Kõiki ta tõotusi ja kõiki vandega antud lubadusi hinge alandamiseks võib ta mees jätta jõusse või tühistada.
\par 15 Aga kui ta mees päevast päeva ei lausu temale sõnagi, siis jätab ta jõusse tema lubadused või kõik ta loobumistõotused, mis tal on täita. Ta jätab need jõusse, sest ta ei ole lausunud temale sõnagi päeval, mil ta neid kuulis.
\par 16 Ja kui ta need siiski tühistab, pärast seda kui ta neid on kuulnud, aga alles hiljem, siis ta peab kandma naise süüd!”

\chapter{31}

\par 1 Ja Issand rääkis Moosesega, öeldes:
\par 2 „Tasu kätte midjanlastele Iisraeli laste eest! Pärast seda koristatakse sind su rahva juurde.”
\par 3 Siis Mooses rääkis rahvaga, öeldes: „Varustage eneste hulgast mehi sõjaks, et nad läheksid Midjani vastu Issanda poolt Midjanile kätte tasuma!
\par 4 Tuhat meest igast suguharust, kõigist Iisraeli suguharudest läkitage sõtta!”
\par 5 Siis valiti Iisraeli tuhandeist tuhat meest igast suguharust, kaksteist tuhat sõjaks varustatud meest.
\par 6 Ja Mooses läkitas sõtta need tuhat meest igast suguharust, nemad ja preester Eleasari poja Piinehasi, kellel olid kaasas pühad riistad ja märgupasunad.
\par 7 Ja nad sõdisid Midjani vastu, nagu Issand oli Moosest käskinud, ja nad tapsid ära kõik meesterahvad.
\par 8 Hukatute hulgas nad tapsid Midjani kuningad Evi, Rekemi, Suuri, Huuri ja Reba, viis Midjani kuningat; ka Bileami, Beori poja, nad tapsid mõõgaga.
\par 9 Ja Iisraeli lapsed võtsid vangi Midjani naised ja nende lapsed, nad riisusid ära kõik nende veoloomad ja kõik nende karjad ja kogu nende varanduse.
\par 10 Ja kõik linnad nende asupaikades ja kõik nende telklaagrid nad põletasid tulega.
\par 11 Ja nad võtsid kogu saagi ja kõik võetava inimestest ja loomadest
\par 12 ning tõid vangid, võetu ja saagi Moosese ja preester Eleasari ja Iisraeli laste koguduse juurde leeri, mis oli Moabi lagendikel Jordani ääres Jeeriko kohal.
\par 13 Siis läksid Mooses ja preester Eleasar ja kõik koguduse vürstid neile vastu väljapoole leeri.
\par 14 Aga Mooses vihastus väeülemate peale, tuhande- ja sajapealikute peale, kes tulid sõjakäigult.
\par 15 Ja Mooses ütles neile: „Kas olete kõik naised jätnud elama?
\par 16 Vaata, nemad ju olid, kes Bileami nõu järgi saatsid Iisraeli lapsed Issandale truudust murdma Peori pärast, nõnda et Issanda kogudust tabas nuhtlus.
\par 17 Seepärast tapke nüüd kõik poeglapsed, samuti tapke iga naine, kes on meest tunda saanud mehega magades!
\par 18 Aga kõik tütarlapsed, kes ei ole õppinud tundma mehega magamist, jätke endile elama!
\par 19 Ja te ise lööge leer üles väljapoole leeri seitsmeks päevaks; igaüks, kes on tapnud inimese, ja igaüks, kes on puudutanud mahalöödut, puhastagu ennast kolmandal päeval ja seitsmendal päeval, te ise ja teie vangid!
\par 20 Ja puhastage kõik riided, kõik nahkriistad ja kõik asjad, mis on tehtud kitsekarvust, ja kõik puuastjad!”
\par 21 Ja preester Eleasar ütles sõjameestele, kes olid käinud sõjas: „See on Seaduse määrus, mille Issand andis Moosesele:
\par 22 kuld, hõbe, vask, raud, tina ja seatina,
\par 23 kõik, mis kannatab tuld, laske tulest läbi käia, siis on see puhas; ometi tuleb seda veel puhastada puhastusveega, samuti laske veest läbi käia kõik, mis ei kannata tuld!
\par 24 Ja seitsmendal päeval peske oma riided, siis olete puhtad, ja pärast seda võite leeri tulla!”
\par 25 Ja Issand rääkis Moosesega, öeldes:
\par 26 „Lugege üle inimestest ja loomadest võetud saak, sina ja preester Eleasar ja koguduse perekondade peamehed,
\par 27 ja jaota saak pooleks sõjas käinud sõjameeste ja terve koguduse vahel!
\par 28 Siis võta sõjas käinud sõjameestelt andam Issandale, üks hingeline viiesajast, niihästi inimestest kui veistest, eeslitest, lammastest ja kitsedest!
\par 29 Võta see nende osapoolest ja anna preester Eleasarile, tõstelõivuks Issandale!
\par 30 Ja Iisraeli laste osapoolest võta üks viiekümnest: inimestest, veistest, eeslitest, lammastest ja kitsedest, kõigist loomadest, ja anna need leviitidele, kes toimetavad Issanda elamu teenistust!”
\par 31 Ja Mooses ja preester Eleasar tegid, nagu Issand oli Moosest käskinud.
\par 32 Ja saak, ülejääk kokkuriisutust, mida sõjamehed olid riisunud, oli: lambaid ja kitsi kuussada seitsekümmend viis tuhat;
\par 33 veiseid seitsekümmend kaks tuhat;
\par 34 eesleid kuuskümmend üks tuhat;
\par 35 ja inimhingi naistest, kes ei olnud tundnud mehega magamist, kõiki kokku kolmkümmend kaks tuhat.
\par 36 Pool sellest, sõjas käinute osa, oli: lambaid ja kitsi kolmsada kolmkümmend seitse tuhat viissada;
\par 37 andam Issandale lammastest ja kitsedest oli kuussada seitsekümmend viis;
\par 38 veiseid oli kolmkümmend kuus tuhat, ja neist andam Issandale: seitsekümmend kaks;
\par 39 eesleid oli kolmkümmend tuhat viissada, ja neist andam Issandale: kuuskümmend üks;
\par 40 inimhingi oli kuusteist tuhat, ja neist andam Issandale: kolmkümmend kaks hinge.
\par 41 Ja Mooses andis Issanda tõstelõivu andami preester Eleasarile, nagu Issand oli Moosest käskinud.
\par 42 Ja Iisraeli laste osapool, mille Mooses oli sõjameestelt eraldanud,
\par 43 see koguduse osapool oli: lambaid ja kitsi kolmsada kolmkümmend seitse tuhat viissada;
\par 44 veiseid kolmkümmend kuus tuhat;
\par 45 eesleid kolmkümmend tuhat viissada;
\par 46 inimhingi kuusteist tuhat.
\par 47 Ja Mooses võttis Iisraeli laste osapoolest, inimestest ja loomadest, ühe viiekümnest ja andis need leviitidele, kes toimetasid Issanda elamu teenistust, nagu Issand oli Moosest käskinud.
\par 48 Ja väeülemad, kes olid sõjaväe tuhandete üle, tuhande- ja sajapealikud, astusid Moosese juurde
\par 49 ning ütlesid Moosesele: „Su sulased on üle lugenud sõjamehed, kes olid meie käe all, ja meist pole ainsatki vajaka.
\par 50 Seepärast me toome Issandale ohvrianniks, mida iga mees on leidnud kuldasjadest: käevõrusid, randmekette, sõrmuseid, kõrvarõngaid ja kaelakeesid, et teha oma hingede eest lepitust Issanda ees.”
\par 51 Ja Mooses ja preester Eleasar võtsid neilt kulla, kõiksugu valmisasjad.
\par 52 Kogu tõstelõivu kulda, mida nad Issandale võtsid igalt tuhandepealikult ja igalt sajapealikult, oli kuusteist tuhat seitsesada viiskümmend seeklit.
\par 53 Sõjamehed aga olid riisunud igaüks iseenesele.
\par 54 Ja Mooses ja preester Eleasar võtsid kulla tuhande- ja sajapealikuilt ning viisid selle kogudusetelki Iisraeli laste meenutamiseks Issanda ees.

\chapter{32}

\par 1 Ruubenlastel ja gaadlastel oli palju karja, väga palju. Kui nad nägid Jaaserimaad ja Gileadimaad, vaata, siis oli see sobiv paik karjale.
\par 2 Ja gaadlased ja ruubenlased tulid ning rääkisid Moosesega ja preester Eleasariga ja koguduse vürstidega, öeldes:
\par 3 „Atarot, Diibon, Jaaser, Nimra, Hesbon, Elaale, Sebam, Nebo ja Beon,
\par 4 maa, mida Issand lõi Iisraeli koguduse ees, on karjamaa, ja su sulastel on karja.”
\par 5 Ja nad ütlesid veel: „Kui me su silmis oleme armu leidnud, siis antagu see maa omandiks su sulastele! Ära vii meid üle Jordani!”
\par 6 Aga Mooses vastas gaadlastele ja ruubenlastele: „Kas peavad teie vennad minema sõtta, teie aga jääte siia?
\par 7 Miks võtate Iisraeli lastelt julguse minna sellele maale, mille Issand on neile andnud?
\par 8 Nõnda tegid teie vanemad, kui ma Kaades-Barneast läkitasin neid maad vaatama:
\par 9 kui nad jõudsid Kobaraorgu ja nägid maad, siis nad võtsid Iisraeli lastelt julguse minna maale, mille Issand oli neile andnud.
\par 10 Sel päeval süttis Issanda viha põlema ja ta vandus, öeldes:
\par 11 Need mehed, kes on tulnud Egiptusest, kahekümneaastased ja üle selle, ei saa näha maad, mille ma vandega tõotasin Aabrahamile, Iisakile ja Jaakobile, sest nad ei ole ustavalt käinud mu järel,
\par 12 peale Kaalebi, kenislase Jefunne poja, ja Joosua, Nuuni poja, sest nemad on ustavalt käinud Issanda järel.
\par 13 Ja Issanda viha süttis põlema Iisraeli vastu ja ta laskis neid hulkuda kõrbes nelikümmend aastat, kuni oli otsa saanud kogu see sugupõlv, kes oli kurja teinud Issanda silmis.
\par 14 Ja vaata, te olete nüüd tõusnud oma vanemate asemele, te patuste inimeste sigidikud, et veelgi õhutada Issanda vihaleeki Iisraeli vastu.
\par 15 Kui te taganete tema järelt, siis ta jätab rahva veel kauemaks kõrbe ja te hukkate kogu selle rahva.”
\par 16 Siis nad astusid tema juurde ja ütlesid: „Me ehitame siia karjatarad oma loomadele ja linnad oma väetitele,
\par 17 me ise aga varustume kärmesti käima Iisraeli laste ees, kuni oleme nad nende paika viinud. Aga meie väetid jäävad kindlustatud linnadesse maa elanike pärast.
\par 18 Me ei pöördu koju tagasi enne, kui igamees Iisraeli lastest on saanud oma pärisosa,
\par 19 sest me ei taha koos nendega saada pärisosa teiselt poolt Jordanit ega kaugemalt, kuna me saame oma pärisosa siitpoolt Jordanit, ida poolt.”
\par 20 Ja Mooses vastas neile: „Kui te teete nõnda, kui te Issanda ees varustate endid sõjaks
\par 21 ja iga teie varustatu läheb Issanda ees üle Jordani, kuni ta oma vaenlased on enese eest ära ajanud
\par 22 ja maa on alistatud Issanda ees, ja kui te alles seejärel pöördute tagasi, siis te olete süüta Issanda ja Iisraeli ees ning see maa jääb Issanda ees teie omandiks.
\par 23 Aga kui te ei tee nõnda, vaata, siis te teete pattu Issanda vastu ja saate tunda oma karistust, mis teid tabab!
\par 24 Ehitage oma väetitele linnad, lammastele ja kitsedele tarad ja tehke, mida olete lubanud!”
\par 25 Ja gaadlased ja ruubenlased rääkisid Moosesega, öeldes: „Su sulased teevad, nagu meie isand käsib.
\par 26 Meie lapsed, meie naised, meie kari ja kõik meie muud loomad jäävad siia Gileadi linnadesse,
\par 27 aga su sulased, kõik sõjaks varustatud, lähevad sinna üle, et sõdida Issanda ees, nagu mu isand on käskinud.”
\par 28 Siis Mooses andis nende pärast käsu preester Eleasarile ja Joosuale, Nuuni pojale, ja Iisraeli laste suguharude perekondade peameestele,
\par 29 ja Mooses ütles neile: „Kui gaadlased ja ruubenlased lähevad koos teiega üle Jordani, kõik, kes on varustatud sõjaks Issanda ees, ja kui maa alistub teile, siis andke Gileadimaa neile omandiks!
\par 30 Aga kui nad ei lähe varustatult koos teiega sinna üle, siis nad peavad elama Kaananimaal teie keskel!”
\par 31 Siis vastasid gaadlased ja ruubenlased, öeldes: „Mida Issand on öelnud su sulastele, seda me teeme.
\par 32 Me läheme sõjaks varustatult Issanda ees üle Kaananimaale, et meie pärisosa jääks meile siiapoole Jordanit.”
\par 33 Ja Mooses andis neile, gaadlastele ja ruubenlastele ja Joosepi poja Manasse poolele suguharule emorlaste kuninga Siihoni kuningriigi ja Baasani kuninga Oogi kuningriigi, maa ja selle linnad ühes maa-aladega, selle maa linnad ümberkaudu.
\par 34 Ja gaadlased ehitasid üles Diiboni, Ataroti, Aroeri,
\par 35 Atrot-Soofani, Jaaseri, Jogbeha,
\par 36 Beet-Nimra ja Beet-Haarani kindlustatud linnad ja karjatarad.
\par 37 Ja ruubenlased ehitasid üles Hesboni, Elaale, Kirjataimi,
\par 38 Nebo ja Baal-Meoni, nende nimed muudeti, ja Sibma. Nad panid linnadele, mis nad üles ehitasid, omad nimed.
\par 39 Ja Manasse poja Maakiri pojad läksid Gileadi, vallutasid selle, ja emorlased, kes seal olid, aeti ära.
\par 40 Ja Mooses andis Gileadi Manasse pojale Maakirile ning too asus sinna.
\par 41 Ja Manasse poeg Jair läks ning vallutas nende telklaagrid ja nimetas need „Jairi telklaagriteks”.
\par 42 Ja Nobah läks ning vallutas Kenati ja selle tütarlinnad ja nimetas selle Nobahiks enese nime järgi.

\chapter{33}

\par 1 Need on Iisraeli laste rännakud, kui nad lahkusid Egiptusemaalt väehulkadena Moosese ja Aaroni juhtimisel.
\par 2 Mooses kirjutas Issanda käsul üles nende rännakute peatuspaigad. Ja need on nende rännakud peatuspaikade järgi:
\par 3 nad läksid teele Raamsesest esimeses kuus, esimese kuu viieteistkümnendal päeval; päeval pärast paasapüha läksid Iisraeli lapsed välja kõrgele tõstetud käe abiga kõigi egiptlaste nähes.
\par 4 Ja egiptlased matsid, keda Issand oli nende seast maha löönud, kõik esmasündinud, ja Issand mõistis kohut nende jumalate üle.
\par 5 Iisraeli lapsed läksid teele Raamsesest ja lõid leeri üles Sukkotti.
\par 6 Ja nad läksid teele Sukkotist ja lõid leeri üles Eetamisse, mis on kõrbe ääres.
\par 7 Ja nad läksid teele Eetamist ja pöördusid tagasi Pii-Hahirotti, mis on vastu Baal-Sefonit, ja lõid leeri üles Migdoli kohale.
\par 8 Ja nad läksid teele Pii-Hahirotist ja läksid merest läbi kõrbesse, käisid kolm päevateekonda Eetami kõrbes ja lõid leeri üles Maarasse.
\par 9 Ja nad läksid teele Maarast ja jõudsid Eelimisse; Eelimis oli kaksteist veeallikat ja seitsekümmend palmipuud; ja nad lõid seal leeri üles.
\par 10 Ja nad läksid teele Eelimist ja lõid leeri üles Kõrkjamere äärde.
\par 11 Ja nad läksid teele Kõrkjamere äärest ja lõid leeri üles Siini kõrbe.
\par 12 Ja nad läksid teele Siini kõrbest ja lõid leeri üles Dofkasse.
\par 13 Ja nad läksid teele Dofkast ja lõid leeri üles Aalusesse.
\par 14 Ja nad läksid teele Aalusest ja lõid leeri üles Refidimi, aga seal ei olnud rahval vett joomiseks.
\par 15 Ja nad läksid teele Refidimist ja lõid leeri üles Siinai kõrbe.
\par 16 Ja nad läksid teele Siinai kõrbest ja lõid leeri üles Kibrot-Hattaavasse.
\par 17 Ja nad läksid teele Kibrot-Hattaavast ja lõid leeri üles Haserotti.
\par 18 Ja nad läksid teele Haserotist ja lõid leeri üles Ritmasse.
\par 19 Ja nad läksid teele Ritmast ja lõid leeri üles Rimmon-Peretsi.
\par 20 Ja nad läksid teele Rimmon-Peretsist ja lõid leeri üles Libnasse.
\par 21 Ja nad läksid teele Libnast ja lõid leeri üles Rissasse.
\par 22 Ja nad läksid teele Rissast ja lõid leeri üles Kehelatasse.
\par 23 Ja nad läksid teele Kehelatast ja lõid leeri üles Saaferi mäe juurde.
\par 24 Ja nad läksid teele Saaferi mäe juurest ja lõid leeri üles Haraadasse.
\par 25 Ja nad läksid teele Haraadast ja lõid leeri üles Makhelotti.
\par 26 Ja nad läksid teele Makhelotist ja lõid leeri üles Taahatti.
\par 27 Ja nad läksid teele Taahatist ja lõid leeri üles Taarahisse.
\par 28 Ja nad läksid teele Taarahist ja lõid leeri üles Mitkasse.
\par 29 Ja nad läksid teele Mitkast ja lõid leeri üles Hasmonasse.
\par 30 Ja nad läksid teele Hasmonast ja lõid leeri üles Moserotti.
\par 31 Ja nad läksid teele Moserotist ja lõid leeri üles Bene-Jaakanisse.
\par 32 Ja nad läksid teele Bene-Jaakanist ja lõid leeri üles Hor-Hagidgadisse.
\par 33 Ja nad läksid teele Hor-Hagidgadist ja lõid leeri üles Jotbatasse.
\par 34 Ja nad läksid teele Jotbatast ja lõid leeri üles Abronasse.
\par 35 Ja nad läksid teele Abronast ja lõid leeri üles Esjon-Geberisse.
\par 36 Ja nad läksid teele Esjon-Geberist ja lõid leeri üles Siini kõrbe, see on Kaadesisse.
\par 37 Ja nad läksid teele Kaadesist ja lõid leeri üles Hoori mäe juurde Edomimaa piirile.
\par 38 Ja preester Aaron läks Issanda käsul üles Hoori mäele ning suri seal neljakümnendal aastal pärast Iisraeli laste lahkumist Egiptusemaalt, viienda kuu esimesel päeval.
\par 39 Ja Aaron oli sada kakskümmend kolm aastat vana, kui ta suri Hoori mäel.
\par 40 Ja Aradi kuningas, kaananlane, kes elas Kaananimaa lõunaosas, sai kuulda Iisraeli laste tulekust.
\par 41 Ja nad läksid teele Hoori mäe juurest ja lõid leeri üles Salmonasse.
\par 42 Ja nad läksid teele Salmonast ja lõid leeri üles Puunoni.
\par 43 Ja nad läksid teele Puunonist ja lõid leeri üles Obotti.
\par 44 Ja nad läksid teele Obotist ja lõid leeri üles Ijje-Abarimi Moabi maa-alal.
\par 45 Ja nad läksid teele Ijjimist ja lõid leeri üles Diibon-Gaadi.
\par 46 Ja nad läksid teele Diibon-Gaadist ja lõid leeri üles Almon-Diblataimi.
\par 47 Ja nad läksid teele Almon-Diblataimist ja lõid leeri üles Abarimi mäestikku Nebo ette.
\par 48 Ja nad läksid teele Abarimi mäestikust ja lõid leeri üles Moabi lagendikele, Jordani äärde Jeeriko kohale.
\par 49 Ja Jordani ääres lõid nad leeri üles Beet-Jesimotist kuni Aabel-Sittimini Moabi lagendikel.
\par 50 Ja Issand rääkis Moosesega Moabi lagendikel Jordani ääres Jeeriko kohal, öeldes:
\par 51 „Räägi Iisraeli lastega ja ütle neile: Kui te lähete üle Jordani Kaananimaale,
\par 52 siis ajage eneste eest ära kõik maa elanikud ja purustage kõik nende jumalakujud; hävitage kõik nende valatud kujud ja tehke maatasa kõik nende ohvrikünkad!
\par 53 Võtke maa oma valdusesse ja asuge sinna, sest ma annan selle maa teile päranduseks!
\par 54 Maa jaotatagu liisu läbi teile pärisosaks, vastavalt teie suguvõsadele: suuremale andke rohkem pärisosaks ja väiksemale vähem pärisosaks - kuhu temale liisk langeb, see saagu temale; te peate saama pärisosa oma vanemate suguharude järgi.
\par 55 Aga kui te ei aja ära maa elanikke eneste eest, siis saavad need, keda te neist alles jätate, okkaiks teie silmadele ja astlaiks teie külgedele, ja nad ohustavad teid sellel maal, kus te asute.
\par 56 Ja siis sünnib, et ma talitan teiega nõnda, nagu ma mõtlesin talitada nendega.”

\chapter{34}

\par 1 Ja Issand rääkis Moosesega, öeldes:
\par 2 „Käsi Iisraeli lapsi ja ütle neile: Kui te tulete Kaananimaale, see on maa, mis langeb teile pärisosaks, Kaananimaa otsast otsani,
\par 3 siis olgu teie lõunakülg Siini kõrbest piki Edomit; idas alaku teie lõunapoolne piir Soolamere äärest,
\par 4 seejärel pöördugu teie piir lõuna poolt Skorpioni astangut, kulgegu Siinini ja selle lõpp olgu lõuna pool Kaades-Barnead; sealt lähtugu see Hasar-Addarisse ja kulgegu Asmonasse;
\par 5 Asmonast pöördugu piir Egiptuseojani ja selle lõpp olgu meri!
\par 6 Teie läänepoolseks piiriks olgu suur meri; see olgu teie läänepoolne piir!
\par 7 Teie põhjapoolseks piiriks olgu see: suurest merest märgistage endile Hoori mäeni;
\par 8 Hoori mäelt märgistage Hamati teelahkmeni ja piir lõppegu Sedadis;
\par 9 sealt mingu piir Sifroni ja selle lõpp olgu Hasar-Eenanis; see olgu teie põhjapoolne piir!
\par 10 Idapoolne piir märgistage endile Hasar-Eenanist Sefamisse;
\par 11 Sefamist laskugu piir Riblasse ida pool Aini; piir laskugu edasi ja puudutagu nõlvakut ida pool Kinnereti järve;
\par 12 siis laskugu piir Jordanini ja selle lõpp olgu Soolameri! See olgu teie maa, ja need on piirid, mis seda ümbritsevad!”
\par 13 Ja Mooses käskis Iisraeli lapsi, öeldes: „See on maa, mille te saate liisu läbi endile pärisosaks, mille Issand on käskinud anda üheksale ja poolele suguharule,
\par 14 sest ruubenlaste suguharu oma perekondade kaupa ja gaadlaste suguharu oma perekondade kaupa ja Manasse pool suguharu on juba saanud oma pärisosad.
\par 15 Kaks ja pool suguharu on saanud oma pärisosad siinpool Jordanit Jeeriko kohal, ida pool, päikesetõusu pool.”
\par 16 Ja Issand rääkis Moosesega, öeldes:
\par 17 „Need on nende meeste nimed, kes peavad andma maa teile pärisosaks: preester Eleasar ja Joosua, Nuuni poeg.
\par 18 Ja võtke igast suguharust üks vürst maad pärisosaks andma!
\par 19 Ja need on nende meeste nimed: Juuda suguharust Kaaleb, Jefunne poeg;
\par 20 siimeonlaste suguharust Semuel, Ammihudi poeg;
\par 21 Benjamini suguharust Elidad, Kisloni poeg;
\par 22 daanlaste suguharust vürst Bukki, Jogli poeg;
\par 23 Joosepi järeltulijaist: manasselaste suguharust vürst Hanniel, Eefodi poeg;
\par 24 efraimlaste suguharust vürst Kemuel, Siftani poeg;
\par 25 sebulonlaste suguharust vürst Elisafan, Parnaki poeg;
\par 26 issaskarlaste suguharust vürst Paltiel, Assani poeg;
\par 27 aaserlaste suguharust vürst Ahihud, Selomi poeg;
\par 28 naftalilaste suguharust vürst Pedahel, Ammihudi poeg.”
\par 29 Need olid need, keda Issand käskis anda Iisraeli lastele pärisosa Kaananimaal.

\chapter{35}

\par 1 Ja Issand rääkis Moosesega Moabi lagendikel Jordani ääres Jeeriko kohal, öeldes:
\par 2 „Käsi Iisraeli lapsi, et nad neile kuuluvast pärisosast annaksid leviitidele linnu elamiseks; ja andke leviitidele ka karjamaad nende linnade ümber!
\par 3 Need linnad olgu neile elamiseks ja karjamaad olgu nende veoloomadele, veistele ja kõigile muudele loomadele!
\par 4 Ja nende linnade karjamaad, mis te annate leviitidele, olgu väljaspool linnamüüri tuhandeküünrase ringina!
\par 5 Mõõtke väljaspool linna idapoolses küljes kaks tuhat küünart ja lõunapoolses küljes kaks tuhat küünart ja läänepoolses küljes kaks tuhat küünart ja põhjapoolses küljes kaks tuhat küünart ja linn olgu keskpaigas; see olgu neile linnade karjamaaks!
\par 6 Ja neist linnadest, mis te annate leviitidele, olgu kuus tükki pelgulinnadeks, mis te annate selleks, et sinna võiks põgeneda tapja; ja neile lisaks andke nelikümmend kaks linna!
\par 7 Ühtekokku olgu linnu, mis te peate andma leviitidele, nelikümmend kaheksa linna, need ja nende karjamaad!
\par 8 Ja neid linnu, mis te annate Iisraeli laste omanditest, andke suuremailt suguharudelt rohkem ja väiksemailt vähem; igaüks andku oma linnadest leviitidele vastavalt pärisosale, mille ta saab!”
\par 9 Ja Issand rääkis Moosesega, öeldes:
\par 10 „Räägi Iisraeli lastega ja ütle neile: Kui te lähete üle Jordani Kaananimaale,
\par 11 siis valige endile linnad; need olgu teile pelgulinnadeks, et sinna võiks põgeneda tapja, kes kogemata on kellegi surmanud!
\par 12 Need linnad olgu teile pelgupaikadeks veritasunõudja eest, et tapja ei sureks, enne kui ta on seisnud koguduse kohtu ees!
\par 13 Linnadest, mis te annate, olgu kuus tükki teile pelgulinnadeks:
\par 14 kolm linna andke siinpool Jordanit ja kolm linna andke Kaananimaal; need olgu pelgulinnadeks!
\par 15 Iisraeli lastele ning võõrale ja majalisele nende seas olgu need kuus linna pelgupaikadeks, et sinna võiks põgeneda igaüks, kes kogemata on kellegi tapnud!
\par 16 Aga kui ta lööb teda raudriistaga, nõnda et ta sureb, siis ta on meelega tapja: seda tapjat karistatagu surmaga!
\par 17 Või kui ta lööb teda käes oleva kiviga, millega saab surmata, nõnda et ta sureb, siis ta on meelega tapja: seda tapjat karistatagu surmaga!
\par 18 Või kui ta lööb teda käes oleva puuriistaga, millega saab surmata, nõnda et ta sureb, siis ta on meelega tapja: seda tapjat karistatagu surmaga!
\par 19 Veritasunõudja surmaku see tapja; kui ta teda kohtab, siis ta surmaku ta!
\par 20 Ja kui ta viha pärast teda tõukab või varitsedes viskab midagi tema peale, nõnda et ta sureb,
\par 21 või vaenulikkusest lööb teda oma käega, nõnda et ta sureb, siis karistatagu lööjat surmaga: ta on meelega tapja; veritasunõudja surmaku tapja, kui ta teda kohtab.
\par 22 Aga kui ta äkitselt, ilma vaenulikkuseta, tõukab või viskab ta peale mõne asja, ilma et ta oleks teda varitsenud,
\par 23 või laseb ilma teda nägemata langeda ta peale mõne kivi, millega saab surmata, nõnda et ta sureb, kuid ta ei ole olnud tema vihamees ega ole püüdnud temale kurja teha,
\par 24 siis kogudus mõistku kohut lööja ja veritasunõudja vahel nende seadluste järgi!
\par 25 Ja kogudus päästku tapja veritasunõudja käest ja kogudus saatku ta tagasi pelgulinna, kuhu ta oli põgenenud; ja ta asugu seal püha õliga võitud ülempreestri surmani!
\par 26 Aga kui tapja siiski läheb välja oma pelgulinna maa-alalt, kuhu ta on põgenenud,
\par 27 ja veritasunõudja leiab tema väljaspool ta pelgulinna maa-ala ja veritasunõudja tapab tapja, siis ei ole tal veresüüd,
\par 28 sest tapja peab jääma oma pelgulinna kuni ülempreestri surmani; pärast ülempreestri surma võib tapja minna tagasi oma pärisosa maale.
\par 29 Need olgu teile seadusteks põlvest põlve, kus te ka iganes elate.
\par 30 Kui keegi lööb maha mõne hingelise, siis surmatagu tapja tunnistajate ütluse järgi; aga ühe tunnistaja tunnistuse järgi ärgu surmatagu kedagi!
\par 31 Te ei tohi võtta lunahinda tapja hinge eest, kes on surma väärt, vaid teda karistatagu surmaga!
\par 32 Te ei tohi võtta lunahinda sellelt, kes on põgenenud pelgulinna, et ta saaks pöörduda tagasi maale elama enne preestri surma!
\par 33 Ärge rüvetage maad, kus te olete, sest veri rüvetab maad ja maale ei saa toimetada lepitust vere eest, mis seal on valatud, kui ainult selle valaja verega!
\par 34 Ära rüveta maad, kus te elate, mille keskel mina elan, sest mina, Issand, elan Iisraeli laste keskel!”

\chapter{36}

\par 1 Siis astusid ette gileadlaste suguvõsa perekondade peamehed - üks Joosepi poegade suguvõsadest, sest Gilead oli Manasse poja Maakiri poeg - ja nad rääkisid Moosese ja vürstide ees, kes olid Iisraeli laste perekondade peamehed,
\par 2 ning ütlesid: „Issand on mu isandat käskinud anda maa liisu läbi Iisraeli lastele pärisosaks, ja mu isand sai Issandalt käsu anda meie venna Selofhadi pärisosa tema tütardele.
\par 3 Aga kui nad saavad naisteks Iisraeli laste teiste suguharude poegadele, siis võetakse nende pärisosa ära meie vanemate pärisosast ja see liidetakse selle suguharu pärisosaga, kuhu nad hakkavad kuuluma; aga meie pärandus-liisuosa vähendatakse.
\par 4 Isegi kui Iisraeli lastele saabub juubeliaasta, siis liidetakse nende pärisosa ometi selle suguharu pärisosaga, kuhu nad hakkavad kuuluma, ja meie vanemate suguharu pärisosast võetakse ära nende pärisosa.”
\par 5 Ja Mooses andis Issanda käsul Iisraeli lastele käsu, öeldes: „Joosepi poegade suguharu räägib õigesti.
\par 6 Issand on andnud Selofhadi tütarde pärast niisuguse käsu, öeldes: Nad saagu naiseks neile, kes nende silmis head on, ainult et nad saaksid naiseks oma vanemate suguharu suguvõsasse
\par 7 ja et Iisraeli laste pärisosa ei läheks ühest suguharust teise, vaid Iisraeli lastest peab igaüks kinni hoidma oma vanemate suguharu pärisosast.
\par 8 Iga tütarlaps, kes pärib pärisosa mõnes Iisraeli laste suguharus, saagu naiseks kellelegi oma isa suguharu suguvõsadest, et Iisraeli lastest saaks igaüks pärida oma vanemate pärisosa
\par 9 ja et pärisosa ei läheks ühest suguharust teise, vaid et iga Iisraeli laste suguharu saaks kinni hoida oma pärisosast.”
\par 10 Selofhadi tütred tegid nõnda, nagu Issand andis Moosesele käsu:
\par 11 Mahla, Tirsa, Hogla, Milka ja Noa, Selofhadi tütred, said naisteks oma isa vendade poegadele.
\par 12 Nad said naisteks Joosepi poja Manasse poegade suguvõsadesse ja nende pärisosa jäi nende isa suguvõsa suguharu juurde.
\par 13 Need on need käsud ja seadlused, mis Issand andis Moosese kaudu Iisraeli lastele Moabi lagendikel Jordani ääres Jeeriko kohal.

\end{document}