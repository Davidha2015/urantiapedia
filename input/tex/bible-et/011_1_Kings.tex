\begin{document}

\title{Esimene Kuningate raamat}

\chapter{1}

\par 1 Kuningas Taavet oli jäänud vanaks ja elatanuks, ja kuigi teda kaeti vaipadega, ei saanud ta sooja.
\par 2 Siis ütlesid ta sulased temale: „Otsitagu mu isandale kuningale tütarlaps, kes on neitsi, ja ta seisku kuninga teenistuses ning olgu tema eest hoolitsejaks! Ta magagu su süles, et mu isand kuningas saaks sooja!”
\par 3 Nad otsisid siis ilusat tütarlast Iisraeli kõigist paigust ja leidsid suunemlanna Abisagi ning tõid ta kuninga juurde.
\par 4 Tütarlaps oli väga ilus; ta sai kuninga eest hoolitsejaks ja teenis teda; aga kuningas hoidus temast.
\par 5 Adonija, Haggiti poeg, aga ülistas iseennast ja ütles: „Ma tahan saada kuningaks!” Ja ta muretses enesele vankreid ja ratsanikke ning viiskümmend meest, kes jooksid tema ees.
\par 6 Tema isa ei olnud oma eluajal teda kunagi laitnud, et ta oleks öelnud: „Mispärast sa teed nõnda?” Adonijagi oli väga ilusa välimusega; ta oli sündinud pärast Absalomi.
\par 7 Ta pidas nõu Joabiga, Seruja pojaga, ja preester Ebjatariga; ja need hoidsid Adonija poole.
\par 8 Aga preester Saadok ja Benaja, Joojada poeg, samuti prohvet Naatan ja Simei ja Rei ning Taaveti kangelased ei olnud Adonija poolt.
\par 9 Kord tappis Adonija lambaid ja kitsi, veiseid ja nuumloomi Soheleti kivi juures, mis on Rogeli allika kõrval, ja kutsus kõik oma vennad, kuningapojad, ja kõik Juuda mehed, kuninga sulased,
\par 10 aga prohvet Naatanit, Benajat, ihukaitsjaid ja oma venda Saalomoni ta ei kutsunud.
\par 11 Siis rääkis Naatan Batsebaga, Saalomoni emaga, öeldes: „Kas sa ei ole kuulnud, et Haggiti poeg Adonija on saanud kuningaks, ilma et meie isand Taavet seda teaks?
\par 12 Aga tule nüüd, küllap ma annan sulle nõu, et saaksid päästa oma hinge ja oma poja Saalomoni hinge!
\par 13 Tule, mine kuningas Taaveti juurde ja küsi temalt: Eks sina, mu isand kuningas, ole vandunud oma teenijale ja öelnud: Sinu poeg Saalomon saab pärast mind kuningaks ja tema istub minu aujärjele? Mispärast on siis Adonija saanud kuningaks?
\par 14 Vaata, kui sa seal alles kuningaga kõneled, tulen mina sinu järel sisse ja kinnitan su sõnu.”
\par 15 Ja Batseba läks kuninga juurde kambrisse; kuningas oli nüüd väga vana ja suunemlanna Abisag teenis kuningat.
\par 16 Ja Batseba põlvitas ning kummardas kuninga ees. Ja kuningas küsis: „Mida sa soovid?”
\par 17 Ja ta vastas temale: „Mu isand, sa oled Issanda, oma Jumala juures vandunud oma teenijale: Sinu poeg Saalomon saab pärast mind kuningaks ja istub mu aujärjele.
\par 18 Aga vaata, nüüd on Adonija saanud kuningaks ja sina, mu isand kuningas, ei tea sellest midagi.
\par 19 Ta on tapnud palju härgi, nuumloomi, lambaid ja kitsi ning on kutsunud kõik kuningapojad, preester Ebjatari ja väepealik Joabi; aga su sulast Saalomoni ei ole ta kutsunud.
\par 20 Sinu poole, mu isand kuningas, vaatavad nüüd kogu Iisraeli silmad, et sa neile teada annaksid, kes istub mu isanda kuninga aujärjele pärast teda.
\par 21 Muidu juhtub, et kui mu isand kuningas läheb magama oma vanemate juurde, siis jääme süüdlasteks, mina ja mu poeg Saalomon.”
\par 22 Ja vaata, kui ta alles kuningaga rääkis, tuli prohvet Naatan.
\par 23 Ja kuningale anti teada ning öeldi: „Vaata, prohvet Naatan on siin.” Ja tema tuli kuninga ette ning kummardas silmili maha kuninga ees.
\par 24 Ja Naatan ütles: „Mu isand kuningas! Sa oled muidugi öelnud: Adonija saab pärast mind kuningaks ja istub minu aujärjele?
\par 25 Sest ta läks täna ja tappis palju härgi, nuumloomi, lambaid ja kitsi ning kutsus kõik kuningapojad ja väepealikud ja preester Ebjatari; vaata, nad söövad ja joovad tema ees ning ütlevad: „Elagu kuningas Adonija!”
\par 26 Aga mind, su sulast, ja preester Saadokit ja Benajat, Joojada poega, ja su sulast Saalomoni ei ole ta kutsunud.
\par 27 Kas see on sündinud mu isanda kuninga poolt, ilma et sa oleksid oma sulastele teada andnud, kes istub mu isanda kuninga aujärjele pärast teda?”
\par 28 Siis kostis kuningas Taavet ja ütles: „Kutsuge mu juurde Batseba!” Ja too tuli kuninga ette ning seisis kuninga ees.
\par 29 Ja kuningas vandus ning ütles: „Nii tõesti kui elab Issand, kes mu hinge igast hädast on päästnud:
\par 30 nõnda nagu ma sulle olen vandunud Issanda, Iisraeli Jumala juures ja olen öelnud: Saalomon, sinu poeg, saab pärast mind kuningaks ja istub minu asemel mu aujärjele - nõnda ma teengi täna!”
\par 31 Siis Batseba kummardas silmili maha, heitis kuninga ette ja ütles: „Mu isand, kuningas Taavet, elagu igavesti!”
\par 32 Ja kuningas Taavet ütles: „Kutsuge mu juurde preester Saadok, prohvet Naatan ja Joojada poeg Benaja!” Ja need tulid kuninga ette.
\par 33 Ja kuningas ütles neile: „Võtke enestega kaasa oma isanda sulased ja pange mu poeg Saalomon minu muula selga ning viige ta alla Giihoni äärde!
\par 34 Seal võidku preester Saadok ja prohvet Naatan ta Iisraeli kuningaks! Ja puhuge sarve ning ütelge: „Elagu kuningas Saalomon!”
\par 35 Tulge siis tema järel üles, ja tema tulgu ning istugu mu aujärjele ja olgu minu asemel kuningas! Mina olen käskinud teda olla Iisraeli ja Juuda vürstiks.”
\par 36 Siis kostis Benaja, Joojada poeg, kuningale ja ütles: „Aamen! Nõnda öelgu ka Issand, mu isanda kuninga Jumal!
\par 37 Nõnda nagu Issand on olnud mu isandaga, kuningaga, nõnda olgu ta Saalomoniga ja ta tehku tema aujärg suuremaks, kui on mu isanda, kuningas Taaveti aujärg!”
\par 38 Siis läksid preester Saadok, prohvet Naatan ja Benaja, Joojada poeg, ning kreedid ja pleedid ja panid Saalomoni kuningas Taaveti muula selga ning viisid ta Giihoni äärde.
\par 39 Ja preester Saadok võttis telgist õlisarve ning võidis Saalomoni; nad puhusid sarve ja kogu rahvas hüüdis: „Elagu kuningas Saalomon!”
\par 40 Ja kogu rahvas läks tema järel üles, rahvas puhus vilet ja kõik olid väga rõõmsad, nõnda et maa lõhkes nende kärast.
\par 41 Aga Adonija ja kõik kutsutud, kes olid ta juures, kuulsid seda just siis, kui nad söömise olid lõpetanud. Ja kui Joab kuulis sarvehäält, siis ta küsis: „Mispärast kostab linnast niisugune kära?”
\par 42 Kui ta alles rääkis, vaata, siis tuli Joonatan, preester Ebjatari poeg, ja Adonija ütles: „Tule sisse, sest sa oled tubli mees ja tood häid sõnumeid!”
\par 43 Aga Joonatan kostis ja ütles Adonijale: „Vastupidi! Meie isand, kuningas Taavet, on tõstnud Saalomoni kuningaks.
\par 44 Kuningas läkitas koos temaga preester Saadoki ja prohvet Naatani ja Benaja, Joojada poja, ning kreedid ja pleedid, ja nad panid ta kuninga muula selga.
\par 45 Siis võidsid preester Saadok ja prohvet Naatan ta Giihoni ääres kuningaks, ja nad tulid sealt rõõmustades üles ning kogu linn oli liikvel. See oligi kära, mida te kuulsite.
\par 46 Saalomon istubki juba kuninglikul aujärjel.
\par 47 Ka kuninga sulased tulid õnnitlema meie isandat, kuningas Taavetit, öeldes: „Tehku su Jumal Saalomoni nimi sinu nimest kuulsamaks ja tehku ta tema aujärg sinu aujärjest suuremaks!” Ja kuningas kummardas oma voodis.
\par 48 Kuningas ütles veel nõnda: „Kiidetud olgu Issand, Iisraeli Jumal, kes täna on andnud mu aujärjele istuja, ja et ma näen seda oma silmaga!”
\par 49 Siis värisesid ja tõusid üles kõik need kutsutud, kes olid Adonija juures, ja läksid igaüks oma teed.
\par 50 Ja Adonija kartis Saalomoni; ta tõusis üles ja läks ning haaras kinni altari sarvedest.
\par 51 Ja Saalomonile anti teada ning öeldi: „Vaata, Adonija kardab kuningas Saalomoni, ja näe, ta hoiab kinni altari sarvedest, öeldes: „kuningas Saalomon vandugu mulle täna, et ta ei surma mõõgaga oma sulast!””
\par 52 Siis ütles Saalomon: „Kui ta jääb tubliks meheks, siis ei lange temalt juuksekarvagi maha, aga kui temas leitakse kurja, siis peab ta surema.”
\par 53 Ja kuningas Saalomon läkitas talle järele ning laskis ta altarilt alla tuua. Siis ta tuli ning kummardas kuningas Saalomoni ees. Ja Saalomon ütles temale: „Mine oma kotta!”

\chapter{2}

\par 1 Kui Taaveti päevad liginesid surmale, siis andis ta oma pojale Saalomonile käsu, öeldes:
\par 2 „Mina lähen kõige maailma teed, ole siis vahva ja ole mees!
\par 3 Pea, mida Issand, su Jumal, on käskinud pidada, käies tema teedel, pidades tema määrusi, käske, seadlusi ja manitsusi, nõnda nagu Moosese Seaduses on kirjutatud, et sa võiksid targasti teha kõike, mida sa teed, ja kõikjal, kuhu sa pöördud,
\par 4 et Issand võiks kinnitada oma sõna, mis ta minu kohta on rääkinud, öeldes: Kui su pojad hoiavad oma teed, käies minu ees ustavalt, kõigest oma südamest ja kõigest oma hingest, siis - ütles tema - ei puudu sul mees Iisraeli aujärjel.
\par 5 Ja sina tead ka seda, mis Joab, Seruja poeg, on teinud minule, mis ta tegi kahele Iisraeli väepealikule, Abnerile, Neeri pojale, ja Amaasale, Jeteri pojale, ja kuidas ta tappis nad ning valas rahuajal sõjaverd, määrides sõjaverega oma vöö, mis tal oli puusadel, ja jalatsid, mis tal olid jalas.
\par 6 Seepärast talita oma tarkust mööda ja ära lase teda hallipäisena rahuga hauda minna!
\par 7 Aga gileadlase Barsillai poegadele tee head, et nad saaksid su lauas sööjate seltsi, sest nad tulid sel viisil mulle appi, kui ma põgenesin su venna Absalomi eest.
\par 8 Ja vaata, su juures on Simei, Geera poeg, benjaminlane Bahuurimist, kes sajatas mind kibeda sajatusega sel päeval, kui ma läksin Mahanaimi; aga kui ta tuli alla Jordani äärde mulle vastu, siis vandusin ma temale Issanda juures ja ütlesin: Ma ei surma sind mõõgaga.
\par 9 Ent nüüd ära jäta teda karistamata, sest sa oled tark mees ja tead küll, mida sa temale pead tegema, et võiksid saata ta hallpea verisena hauda!”
\par 10 Siis Taavet läks magama oma vanemate juurde ja ta maeti Taaveti linna.
\par 11 Ja aega, mis Taavet valitses Iisraeli üle, oli nelikümmend aastat; Hebronis valitses ta seitse aastat ja Jeruusalemmas valitses ta kolmkümmend kolm aastat.
\par 12 Saalomon istus oma isa Taaveti aujärjele ja ta kuningavõim oli väga kindel.
\par 13 Aga Adonija, Haggiti poeg, tuli Saalomoni ema Batseba juurde. Ja Batseba küsis: „Kas tood rahu?„ Ja ta vastas: ”Rahu.”
\par 14 Siis ta ütles: „Mul on sulle midagi rääkida.„ Ja Batseba vastas: ”Räägi!”
\par 15 Ja ta rääkis: „Sa tead, et kuningriik oli minu ja et kogu Iisrael oli pööranud oma silmad minu poole, et mina saaksin kuningaks. Aga kuningriik on läinud mu vennale, sest ta sai selle Issandalt.
\par 16 Ja nüüd on mul üksainus palve, mida ma sinult palun. Ära tõuka mind ära!„ Ja Batseba vastas temale: „Räägi!”
\par 17 Siis ta ütles: „Räägi ometi kuningas Saalomoniga, sest sind ta ei tõuka ära, et ta annaks mulle naiseks suunemlanna Abisagi!”
\par 18 Ja Batseba vastas: „Hea küll! Ma räägin su pärast kuningaga.”
\par 19 Ja Batseba läks kuningas Saalomoni juurde, et temaga rääkida Adonija pärast. Ja kuningas tõusis, läks temale vastu, kummardas tema ees ja istus siis oma aujärjele. Kuninga emale seati iste ja ta istus tema paremale käele.
\par 20 Siis ta ütles: „Mul on sinult paluda üksainus pisike palve, ära tõuka mind ära!„ Ja kuningas ütles temale: ”Palu, mu ema, sest sind ma ei tõuka ära!”
\par 21 Ja Batseba ütles: „Antagu suunemlanna Abisag naiseks su vennale Adonijale!”
\par 22 Aga kuningas Saalomon vastas ja ütles oma emale: „Mispärast sa palud Adonijale ainult suunemlannat Abisagi? Palu temale ka kuningriiki, sest ta on mu vanem vend, temale ja preester Ebjatarile ja Joabile, Seruja pojale!”
\par 23 Ja kuningas Saalomon vandus Issanda juures, öeldes: „Jumal tehku minuga ükskõik mida, kui Adonija ei ole seda rääkinud oma hinge hinnaga!
\par 24 Ja nüüd, nii tõesti kui elab Issand, kes mind on kinnitanud ja pannud mu isa Taaveti aujärjele ja kes mulle on teinud koja, nagu ta lubas: Adonija tuleb täna surmata!”
\par 25 Ja kuningas Saalomon läkitas Benaja, Joojada poja, kes tungis Adonijale kallale, nõnda et ta suri.
\par 26 Preester Ebjatarile ütles kuningas: „Mine Anatotti oma põllule, sest sa oled surma väärt! Aga täna ei taha ma sind surmata, sest sa oled kandnud Issanda Jumala laegast mu isa Taaveti ees ja oled kannatanud kõike, mida mu isa kannatas.”
\par 27 Nõnda ajas Saalomon Ebjatari ära ega lasknud teda olla Issanda preester, et läheks täide Issanda sõna, mis ta Siilos oli rääkinud Eeli soo kohta.
\par 28 Kui kuuldus sellest jõudis Joabini, sest Joab oli hoidnud Adonija poole, kuigi ta ei olnud hoidnud Absalomi poole, siis põgenes Joab Issanda telgi juurde ja haaras kinni altari sarvedest.
\par 29 Ja kui kuningas Saalomonile teatati, et Joab oli põgenenud Issanda telgi juurde, ja et vaata, ta on altari ääres, siis läkitas Saalomon Benaja, Joojada poja, öeldes: „Mine tungi temale kallale!”
\par 30 Ja kui Benaja jõudis Issanda telgi juurde, siis ta ütles temale: „Nõnda ütleb kuningas: Tule ära!„ Aga ta vastas: „Ei, sest ma tahan siin surra!” Ja Benaja viis kuningale vastuse, öeldes: ”Nõnda rääkis Joab ja nõnda vastas ta mulle.”
\par 31 Ja kuningas ütles temale: „Tee, nagu ta ütles, tungi temale kallale ja mata ta, et sa kõrvaldaksid minult ja mu isakojalt selle süütu vere, mille Joab on valanud!
\par 32 Issand lasku tema veri tulla tagasi ta enese pea peale, sellepärast et ta tungis kallale kahele mehele, kes olid temast õiglasemad ja paremad, ja tappis need mõõgaga, ilma et mu isa Taavet seda oleks teadnud: Abneri, Neeri poja, Iisraeli väepealiku, ja Amaasa, Jeteri poja, Juuda väepealiku.
\par 33 Nende veri tulgu igavesti Joabi pea peale ja tema soo pea peale! Aga Taavetile ja tema soole ja kojale ja aujärjele tulgu Issandalt igavesti rahu!”
\par 34 Siis läks Benaja, Joojada poeg, ja tungis temale kallale ning surmas tema; ja ta maeti oma kotta kõrbe.
\par 35 Ja kuningas pani Benaja, Joojada poja, tema asemele sõjaväe ülemaks; ja Ebjatari asemele pani kuningas preester Saadoki.
\par 36 Ja kuningas läkitas kutsujad Simei järele ning ütles temale: „Ehita enesele koda Jeruusalemma ja ela seal, aga ära mine sealt välja, ei sinna ega tänna!
\par 37 Sest sel päeval, kui sa välja lähed ja ületad Kidroni jõe, olgu sul hästi teada, et sa pead surema, su veri tuleb su oma pea peale.”
\par 38 Ja Simei ütles kuningale: „See on hea kõne. Nõnda nagu mu isand kuningas on rääkinud, nõnda su sulane teeb.” Ja Simei elas Jeruusalemmas kaua aega.
\par 39 Aga kolme aasta pärast juhtus, et kaks Simei sulast põgenesid Gati kuninga Maaka poja Aakise juurde. Ja Simeile teatati ning öeldi: „Vaata, su sulased on Gatis.”
\par 40 Siis Simei võttis kätte, saduldas oma eesli ja läks Gatti Aakise juurde oma sulaseid otsima. Nõnda läks Simei ja tõi oma sulased Gatist ära.
\par 41 Aga kui Saalomonile teatati, et Simei oli Jeruusalemmast Gatti läinud ja tagasi tulnud,
\par 42 siis kuningas läkitas kutsujad Simei järele ning ütles temale: „Kas ma ei ole sind Issanda juures vannutanud ja hoiatanud, öeldes: Sel päeval, kui sa lähed välja ja käid siin ning seal, olgu sul hästi teada, et sa pead surema!? Ja sa ütlesid mulle: See on hea kõne, ma kuulan!
\par 43 Mispärast sa ei ole siis teinud Issanda vande ja keelu kohaselt, mille ma sulle andsin?”
\par 44 Ja kuningas ütles veel Simeile: „Sina tead ise oma südames kõike seda kurja, mis sa oled teinud mu isale Taavetile. Sellepärast Issand laseb su kurjuse tagasi tulla su oma pea peale.
\par 45 Aga kuningas Saalomon olgu õnnistatud ja Taaveti aujärg Issanda ees igavesti kindel!”
\par 46 Siis kuningas andis käsu Benajale, Joojada pojale, ja too läks välja ning tungis temale kallale, nõnda et ta suri. Ja kuningriik jäi kindlalt Saalomoni kätte.

\chapter{3}

\par 1 Ja Saalomon sai vaaraoga, Egiptuse kuningaga, languks: ta võttis vaarao tütre ja viis selle Taaveti linna, seniks kui ta oli valmis ehitanud oma koja, Issanda koja ja müüri ümber Jeruusalemma.
\par 2 Aga rahvas ohverdas alles ohvriküngastel, sest Issanda nimele ei olnud kuni selle ajani koda ehitatud.
\par 3 Saalomon armastas küll Issandat, käies oma isa Taaveti seadluste järgi, aga ta ohverdas ja suitsutas alles ohvriküngastel.
\par 4 Nõnda läks kuningas Gibeoni, et seal ohverdada, sest see oli suur ohvripaik; tuhat põletusohvrit ohverdas Saalomon sealsel altaril.
\par 5 Gibeonis ilmutas Issand ennast Saalomonile öösel unes; ja Jumal ütles: „Palu, mida ma sulle peaksin andma!”
\par 6 Ja Saalomon vastas: „Sina osutasid oma sulasele, mu isale Taavetile, suurt heldust, kuna ta su ees käis tões ja õiguses ning oli õiglase südamega sinu vastu. Ja sina säilitasid temale selle suure helduse ning andsid temale poja, kes ta aujärjel istuks, nagu praegu ongi.
\par 7 Ja nüüd, Issand, mu Jumal, sa oled tõstnud oma sulase kuningaks mu isa Taaveti asemele; aga mina olen nagu pisike poiss - ei mõista minna ega tulla.
\par 8 Ja su sulane on su rahva keskel, kelle sa oled ära valinud - suur rahvas, keda ei saa ära lugeda ega arvutada rohkuse pärast.
\par 9 Anna seepärast oma sulasele sõnakuulelik süda, et ta võiks su rahvale kohut mõista ning vahet teha hea ja kurja vahel; sest kes suudaks muidu kohut mõista sellele sinu suurele rahvale?”
\par 10 Et Saalomon just seda palus, siis oli see kõne Issanda silmis hea.
\par 11 Ja Jumal ütles temale: „Sellepärast et sa palusid seda ega palunud enesele pikka iga, rikkust ja oma vaenlaste hinge, vaid palusid mõistust, et kuulata, mis õige on,
\par 12 siis ma teen, vaata, nagu sa ütled: näe, ma annan sulle targa ja mõistliku südame, nõnda et sinu sarnast ei ole olnud enne sind ega tõuse sinu sarnast ka mitte pärast sind.
\par 13 Aga ka seda, mida sa ei ole palunud, ma annan sulle, niihästi rikkust kui au, nõnda et kogu su eluajal ei ole ükski kuningas sinu sarnane.
\par 14 Ja kui sa käid minu teedel, pidades minu määrusi ja käske, nõnda nagu käis su isa Taavet, siis ma pikendan su elupäevi.”
\par 15 Ja Saalomon ärkas, ja vaata, see oli olnud unenägu. Kui ta tuli Jeruusalemma, siis ta astus Issanda seaduselaeka ette, ohverdas põletusohvreid ja valmistas tänuohvreid ning tegi kõigile oma sulaseile suure peo.
\par 16 Kord tulid kaks hooranaist kuninga juurde ja seisid tema ees.
\par 17 Ja üks naine ütles: „Oh mu isand! Mina ja see naine elasime samas kojas. Ja mina sünnitasin kojas tema juuresolekul.
\par 18 Ja kolmandal päeval, kui ma olin sünnitanud, sünnitas ka see naine. Me olime üheskoos, ühtegi võõrast ei olnud meiega seal kojas; ainult me mõlemad olime kojas.
\par 19 Aga selle naise poeg suri öösel, sest ta oli tema ära maganud.
\par 20 Siis ta tõusis keset ööd ja võttis minu poja mu kõrvalt, kui su teenija magas, ja pani oma sülle, aga oma surnud poja pani ta minu sülle.
\par 21 Ja kui ma hommikul tõusin oma poega imetama, vaata, siis oli ta surnud. Aga kui ma teda hommikul tunnistasin, vaata, siis ei olnudki see minu poeg, kelle ma olin sünnitanud.”
\par 22 Siis ütles teine naine: „Ei ole nõnda, vaid minu poeg elab ja sinu poeg on surnud.„ Aga esimene vastas: ”Ei ole nõnda, sinu poeg on surnud ja minu poeg elab.” Nõnda sõnelesid nad kuninga ees.
\par 23 Siis ütles kuningas: „Üks ütleb: See on minu poeg, kes on elus, ja sinu poeg on surnud. Ja teine ütleb: Ei ole nõnda, sinu poeg on surnud ja minu poeg elab.”
\par 24 Ja kuningas ütles: „Tooge mulle mõõk!” Ja mõõk toodi kuninga ette.
\par 25 Siis ütles kuningas: „Raiuge elus laps pooleks ja andke üks pool ühele ja teine pool teisele!”
\par 26 Aga naine, kelle poeg oli elus, rääkis kuningale, sest ta süda põles poja pärast, ja ütles: „Oh mu isand, andke elus laps temale, ärge surmake teda!„ Ent teine ütles: ”See ei pea saama ei minule ega sinule - raiuge pooleks!”
\par 27 Siis kostis kuningas ja ütles: „Andke elus laps temale ja ärge surmake teda mitte - see seal on ta ema!”
\par 28 Ja kogu Iisrael kuulis kohtust, mida kuningas oli mõistnud, ja nad kartsid kuningat, sest nad nägid, et temal oli Jumala tarkus õigust teha.

\chapter{4}

\par 1 Kuningas Saalomon oli kogu Iisraeli Kuningas.
\par 2 Ja need olid tema kõrgemad ametimehed: Asarja, Saadoki poeg, oli ülempreester;
\par 3 Elihoref ja Ahija, Siisa pojad, olid kirjutajad; Joosafat, Ahiluudi poeg, oli nõunik;
\par 4 Benaja, Joojada poeg, oli väeülem; Saadok ja Ebjatar olid preestrid;
\par 5 Asarja, Naatani poeg, oli asevalitsejate ülem; Saabud, Naatani poeg, preester, oli kuninga sõber;
\par 6 Ahisar oli kojaülem; Adoniram, Abda poeg, oli orjatöö ülevaataja.
\par 7 Ja Saalomonil oli üle kogu Iisraeli kaksteist asevalitsejat, kes muretsesid toidust kuningale ja tema kojale; igaühel oli aastas üks kuu toidust muretseda.
\par 8 Need olid nende nimed: Huuri poeg Efraimi mäestikus;
\par 9 Dekeri poeg Maakatsis, Saalbimis, Beet-Semesis ja Eelon-Beet-Haananis;
\par 10 Hesedi poeg Arubbotis, temal oli Sooko ja kogu Heeferimaa;
\par 11 Abinadabi poeg kogu Doori mäeseljandikul, Taafat, Saalomoni tütar, oli tal naiseks;
\par 12 Baana, Ahiluudi poeg, Taanakis, Megiddos ja kogu Beet-Seanis, mis on Saaretani kõrval allpool Jisreeli, Beet-Seanist kuni Aabel-Meholani teisel pool Jokmeami;
\par 13 Geberi poeg Gileadi Raamotis, temal olid Manasse poja Jairi telklaagrid, mis olid Gileadis, temal oli Argobi piirkond, mis on Baasanis, kuuskümmend suurt linna müüride ja vaskriividega;
\par 14 Ahinadab, Iddo poeg, Mahanaimis;
\par 15 Ahimaats Naftalis, temagi võttis naiseks Saalomoni tütre - Baasemati;
\par 16 Baana, Huusai poeg, Aaseris ja Bealotis;
\par 17 Joosafat, Paaruahi poeg, Issaskaris;
\par 18 Simei, Eela poeg, Benjaminis;
\par 19 Geber, Uuri poeg, Gileadimaal, emorlaste kuninga Siihoni ja Baasani kuninga Oogi maal, kui ainus asevalitseja maal.
\par 20 Juudas ja Iisraelis oli palju rahvast, rohkuse poolest nagu liiva mere ääres. Nad sõid ja jõid ning olid rõõmsad.

\chapter{5}

\par 1 Ja Saalomon valitses kõigi kuningriikide üle Frati jõest kuni vilistite maani ja Egiptuse piirini; need tõid ande ja teenisid Saalomoni kogu tema eluaja.
\par 2 Saalomoni igapäevaseks moonaks oli kolmkümmend koori peent jahu ja kuuskümmend koori muud jahu;
\par 3 kümme nuumveist, kakskümmend karjaveist ja sada lammast; peale selle hirved, gasellid ja kitsed ning nuumatud linnud.
\par 4 Sest ta valitses kõikjal siinpool Frati jõge, Tifsahst kuni Assani, kõiki kuningaid siinpool Frati jõge; ja tal oli rahu igalt poolt ümberkaudu.
\par 5 Juuda ja Iisrael elasid julgesti, igamees oma viinapuu ja viigipuu all, Daanist kuni Beer-Sebani, kogu Saalomoni eluaja.
\par 6 Saalomonil oli nelikümmend tuhat latrit vankrihobuste tarvis ja kaksteist tuhat ratsanikku.
\par 7 Need asevalitsejad muretsesid toidust kuningas Saalomonile ja kõigile, kes kuulusid kuninga lauda, igaüks oma kuus, jätmata midagi vajaka.
\par 8 Ja nad tõid, igaüks siis, kui oli tema kord, otri ja õlgi hobustele ja veoloomadele paika, kuhu oli määratud.
\par 9 Ja Jumal andis Saalomonile väga palju tarkust ja mõistust ja taipamisvõimet - nagu liiva mere ääres,
\par 10 nõnda et Saalomoni tarkus oli suurem kui kõigi hommikumaalaste ja kogu Egiptuse tarkus.
\par 11 Ta oli targem kõigist inimestest, targem kui esrahlane Eetan ja Heeman, Kalkol ja Darda, Maaholi pojad; ja ta oli kuulus kõigi rahvaste keskel ümberkaudu.
\par 12 Ta kõneles kolm tuhat õpetussõna ja ta laule oli tuhat viis.
\par 13 Ta kõneles puudest, Liibanoni seedritest müüril kasvava iisopini; ja ta kõneles loomadest ja lindudest, roomajatest ja kaladest.
\par 14 Saalomoni tarkust tuldi kuulama kõigist rahvaist ja kõigi maa kuningate hulgast, kes olid kuulnud tema tarkusest.
\par 15 Ja Tüürose kuningas Hiiram läkitas oma sulased Saalomoni juurde, kui ta kuulis, et tema oli võitud kuningaks oma isa asemele, sest Hiiram oli alati armastanud Taavetit.
\par 16 Ja Saalomon läkitas Hiiramile ütlema:
\par 17 „Sina tead ise, et mu isa Taavet ei suutnud ehitada koda Issanda, oma Jumala nimele sõja pärast, millega teda ümbritseti, kuni Issand andis vaenlased tema jalataldade alla.
\par 18 Aga nüüd on Issand, mu Jumal, andnud mulle rahu ümberkaudu, ei ole vaenlast ega hädaohu ähvardust.

\chapter{6}

\par 1 Ja neljasaja kaheksakümnendal aastal pärast Iisraeli laste Egiptusemaalt tulekut, Saalomoni neljandal valitsemisaastal Iisraeli üle, siivikuus, see on teises kuus, hakkas ta Issandale koda ehitama.
\par 2 Koda, mille kuningas Saalomon Issandale ehitas, oli kuuskümmend küünart pikk, kakskümmend küünart lai ja kolmkümmend küünart kõrge.
\par 3 Eeskoda koja pealöövi esiküljes oli kakskümmend küünart pikk, vastavalt koja laiusele, ja kümme küünart lai - koja pikkusele lisaks.
\par 4 Ta tegi kojale võretatud aknad.
\par 5 Ta ehitas koja seina külge külgehitise ümberringi, ümber koja seinte - pealöövi ja tagaruumi - ja tegi külgkambrid ümberringi.
\par 6 Külgehitise alumise korruse laius oli viis küünart, keskmise korruse laius kuus küünart ja kolmanda korruse laius seitse küünart, sest ta tegi koja seina välispidi ümberringi järkmeliselt, et poleks vaja kinnitust koja seintes.
\par 7 Ja kui koda ehitati, siis ehitati see kivimurrus töödeldud kividest; ei kuuldud kojas vasaraid ega kirkasid, mitte ühtegi raudriista, kui seda ehitati.
\par 8 Külgehitise keskmise korruse uks oli koja lõunapoolses küljes; treppe mööda tõusti keskmisele korrusele ja keskmiselt kolmandale.
\par 9 Nõnda ehitas ta koja. Ja kui ta selle oli valmis saanud, siis ta vooderdas koja seedriplankude ja -laudadega.
\par 10 Ja ehitanud külgehitise kogu kojale, viis küünart kõrgete korrustega, kattis ta koja seedripuuga.
\par 11 Ja Saalomonile tuli Issanda sõna, kes ütles:
\par 12 „Selle kojaga, mida sa ehitad, on lugu nõnda: kui sa käid minu määruste järgi ja täidad minu seadlusi ning pead kõiki minu käske nende järgi käies, siis ma teen tõeks oma sõna, mis ma ütlesin su isale Taavetile:
\par 13 Mina asun elama Iisraeli laste keskele ega jäta maha oma Iisraeli rahvast.”
\par 14 Nõnda ehitas Saalomon koda ja sai selle valmis.
\par 15 Ta kattis koja seinad seestpoolt seedrilaudadega; koja vooderdas ta seestpoolt puuga, põrandast kuni laetaladeni; koja põranda kattis ta aga küpressilaudadega.
\par 16 Ja ta kattis kakskümmend küünart koja tagumisest osast seedrilaudadega, põrandast taladeni, ja ehitas selle tagaruumiks - kõige pühamaks paigaks.
\par 17 Koda, see on pealööv selle ees, oli nelikümmend küünart pikk.
\par 18 Seedripuu selle koja sees oli nikerdatud metskõrvitsate ja puhkenud õite kujuliselt; kõik oli seedripuu, kivi ei olnudki näha.
\par 19 Koja sisemuses seadis ta korda kõige pühama paiga, et sinna asetada Issanda seaduselaegas.
\par 20 Kõige pühama paiga ette, mis oli kakskümmend küünart pikk, kakskümmend küünart lai ja kakskümmend küünart kõrge ja mille ta kattis puhta kullaga, tegi ta seedripuust altari.
\par 21 Ja Saalomon kattis koja seestpoolt puhta kullaga, tõmbas kuldketid kõige pühama paiga ette ja kattis selle kullaga.
\par 22 Ta kattis kullaga kogu koja, kogu koja tervenisti; nõndasamuti kattis ta kullaga kogu altari, mis oli kõige pühama paiga ees.
\par 23 Ja ta valmistas kõige pühamasse paika kaks õlipuust keerubit, kümme küünart kõrged.
\par 24 Üks keerubi tiib oli viis küünart ja teine keerubi tiib oli viis küünart - ühe tiiva otsast teise tiiva otsani oli kümme küünart.
\par 25 Ka teine keerub oli kümneküünrane - mõlemal keerubil oli ühesugune mõõt ja ühesugune kuju.
\par 26 Üks keerub oli kümme küünart kõrge ja nõndasamuti teine keerub.
\par 27 Ta pani keerubid keset koja sisemist osa; ja keerubite tiivad laotusid nõnda, et ühe tiib puudutas ühte ja teise keerubi tiib puudutas teist seina, aga keset koda puutusid nende tiivad teineteise külge.
\par 28 Ja ta kattis keerubid kullaga.
\par 29 Ja ta nikerdas kõigi koja seinte peale ümberringi keerubeid ja palme ja puhkenud õiekesi niihästi sisemises kui välimises ruumis.
\par 30 Ja ta kattis koja põranda kullaga niihästi sisemises kui välimises ruumis.
\par 31 Ja kõige pühama paiga sissekäigule tegi ta õlipuust uksed; piitjala postid olid viietahulised.
\par 32 Ja mõlema õlipuust uksepoole peale nikerdas ta nikerdusi - keerubeid ja palme ja puhkenud õiekesi; ta kattis need kullaga, tagudes kulla keerubite ja palmide peale.
\par 33 Ja nõnda ta tegi ka pealöövi uksele õlipuust postid, mis olid neljatahulised,
\par 34 ja kaks küpressipuust uksepoolt: ühe uksepoole mõlemad tahvlid olid pööratavad, nõndasamuti olid ka teise uksepoole mõlemad tahvlid pööratavad.
\par 35 Ja ta nikerdas nende peale keerubeid ja palme ja puhkenud õiekesi ning kattis nikerdatu pealt siledaks taotud kullaga.
\par 36 Ja ta ehitas siseõue - kolm rida tahutud kive ja rida seedripalke.
\par 37 Neljandal aastal siivikuus pandi Issanda kojale alus,
\par 38 ja üheteistkümnendal aastal buulikuus, see on kaheksandas kuus, sai koda kõigis oma osades valmis, just nagu see pidi olema. Ta ehitas seda seitse aastat.

\chapter{7}

\par 1 Oma koda ehitas aga Saalomon kolmteist aastat ja sai siis valmis kogu Oma koja.
\par 2 Ta ehitas Liibanonimetsakoja: sada küünart pikk, viiskümmend küünart lai ja kolmkümmend küünart kõrge, nelja rea seedrisammaste peal, sammaste peal tahutud seedripalgid.
\par 3 Ülalt oli see kaetud seedrilaudadega kandetalade peal, mis toetusid neljakümne viiele sambale, viisteist igas reas.
\par 4 Aknaid oli kolm rida, valguseava valguseava kohal kolmekordselt.
\par 5 Kõik uksed ja uksepiidad koos pealisega olid nelinurksed; ukseava oli ukseava kohal kolmekordselt.
\par 6 Ja ta tegi sammassaali, viiskümmend küünart pika ja kolmkümmend küünart laia, ja selle ette eeskoja; ja selle ees olid sambad ja ehiskatus.
\par 7 Siis ta tegi troonisaali, kus ta kohut mõistis, kohtusaali, ja vooderdas selle seedripuuga põrandast laeni.
\par 8 Ja tema koda, kus ta ise elas, teises õues, seespool saali, oli tehtud selsamal viisil; ja ta tegi koja, samasuguse kui oli see saal, vaarao tütrele, kelle Saalomon oli naiseks võtnud.
\par 9 Kõik need ehitised alusmüürist kuni räästani ja väljaspool kõik kuni suure õueni olid hinnalistest kividest, mis olid tahutud mõõdu järgi, kivisaega saetud seest- ja väljastpoolt.
\par 10 Alusmüür oli rajatud hinnalistest kividest, suurtest, kümneküünrastest ja kaheksaküünrastest kividest.
\par 11 Selle peal olid hinnalised kivid, mõõdu järgi tahutud, ja seedripalgid.
\par 12 Ja suurel õuel oli ümberringi kolm rida tahutud kive ja rida tahutud seedripalke; nõnda oli ka Issanda koja sisemisel õuel ja paleesaali õuel.
\par 13 Ja kuningas Saalomon läkitas teate ning laskis tuua Tüürosest Hiirami.
\par 14 Too oli lesknaise poeg, Naftali suguharust, ja tema isa oli olnud Tüürose mees, vasksepp. Temal oli külluslikult tarkust ja taipu ja oskust kõiksugu vasktööde tegemiseks. Ja ta tuli kuningas Saalomoni juurde ning tegi ära kõik tema tööd.
\par 15 Ta valas kaks vasksammast: sammaste kõrgus oli kaheksateist küünart ja kaheteistkümneküünrane mõõdunöör ulatus ümber kummagi samba.
\par 16 Ja ta valas vasest kaks nuppu sammaste otsa asetamiseks: ühe nupu kõrgus oli viis küünart, samuti oli teise nupu kõrgus viis küünart;
\par 17 ja võrgukujuliselt tehtud võrestikud, ketikujuliselt põimitud kaunistused nuppude jaoks, mis olid sammaste otsas: seitse ühele nupule ja seitse teisele nupule.
\par 18 Ja ta valmistas sambad, et ühel nuppude kattevõrestikul oleks kaks rida granaatõunu ümberringi; nõndasamuti valmistas ta need teise nupu jaoks.
\par 19 Nupud, mis olid eeskoja sammaste otsas, olid õiekujulised, neljaküünrased.
\par 20 Mõlemal sambanupul oli ka veel ülalt tihedasti mööda võrestiku kumerust kakssada granaatõuna ridastikku ümberringi, niihästi ühel kui teisel nupul.
\par 21 Ta püstitas sambad pealöövi eeskoja ette; ta püstitas samba paremale poole ja nimetas selle „Jaakiniks„; ta püstitas samba vasakule poole ja nimetas selle ”Boaseks”.
\par 22 Sammaste otsad olid tehtud õiekujuliselt. Nõnda viidi lõpule töö sammastega.
\par 23 Ja ta valmistas valatud vaskmere, kümme küünart äärest ääreni, täiesti ümmarguse, viis küünart kõrge; kolmekümneküünrane mõõdunöör ulatus selle ümber.
\par 24 Selle ääre all olid metskõrvitsad, mis seda ümbritsesid: kümme igal küünral, ringi ümber vaskmere: kõrvitsad olid kahes reas, selle valuga koos valatud.
\par 25 See seisis kaheteistkümnel härjal: kolm vaatasid põhja poole, kolm vaatasid lääne poole, kolm vaatasid lõuna poole ja kolm vaatasid ida poole; vaskmeri oli ülal nende peal ja neil kõigil olid tagumised pooled sissepoole.
\par 26 See oli kämblapaksune ja selle äär oli tehtud karikaääre sarnaselt liilia õiena; see mahutas kaks tuhat batti.
\par 27 Ja ta valmistas kümme vaskalust; iga alus oli kümme küünart pikk, neli küünart lai ja kolm küünart kõrge.
\par 28 Alused olid tehtud nõnda: neil olid tahud, ja tahud olid põikliistude vahel.
\par 29 Tahkude peal, mis olid põikliistude vahel, olid lõvid, veised ja keerubid, samuti põikliistude peal; ülal ja all, lõvide ja veiste juures, olid vanikud, sepisetöö.
\par 30 Igal alusel oli neli vaskratast ja vasktelge; ja nende nelja nurga küljes olid pesunõualused toed, valatud toed, igaühe kohal vanikud.
\par 31 Alusel oli seespool tugesid suuline, küünrakõrgune; see suuline oli ümmargune, tehtud kandealuseks, poolteiseküünrane; ka suulise peal olid nikerdused. Aluse tahud olid nelinurksed, mitte ümmargused.
\par 32 Need neli ratast olid tahkude all ja rataste teljed olid kinni aluse küljes; iga ratta kõrgus oli poolteist küünart.
\par 33 Rattad olid tehtud vankriratta sarnaselt; nende teljed, pöiad, kodarad ja rummud olid kõik valatud.
\par 34 Igal alusel oli neli tuge selle nelja nurga küljes; toed olid alusega ühest tükist.
\par 35 Aluse pealmises osas oli poole küünra kõrgune kumm, ringikujuline; ja aluse peal olevad toed ja tahud olid sellega ühest tükist.
\par 36 Ja selle tugede pindade ja tahkude peale nikerdas ta keerubid, lõvid ja palmipuud, kus iganes oli vaba koht, ja ümberringi vanikud.
\par 37 Nõndaviisi tegi ta need kümme alust; need kõik olid ühesuguselt valatud, ühemõõdulised, samakujulised.
\par 38 Ja ta valmistas kümme vaskpesunõu; iga nõu mahutas nelikümmend batti, iga nõu läbimõõt oli neli küünart; kõigi kümne aluse peal oli oma pesunõu.
\par 39 Ta asetas viis alust koja paremasse tiiba ja viis alust koja vasakusse tiiba; ja vaskmere asetas ta koja paremasse tiiba kagusse.
\par 40 Ja Hiiram valmistas tuhanõud, -labidad ja piserduskausid. Nõnda sai Hiiram valmis kõik tööd, mis ta pidi tegema kuningas Saalomonile Issanda koja jaoks:
\par 41 kaks sammast ja kaks kausikujulist nuppu, mis olid sammaste otsas; ja kaks võrestikku mõlema kausikujulise nupu katteks, mis olid sammaste otsas;
\par 42 ja nelisada granaatõuna mõlema võrestiku jaoks, kaks rida granaatõunu kummalegi võrestikule, katteks mõlemale kausikujulisele nupule, mis olid sammaste otsas;
\par 43 ja kümme alust ja kümme pesunõu aluste peale;
\par 44 ja vaskmere ja kaksteist härga vaskmere alla;
\par 45 ja tuhanõud, -labidad ja piserduskausid. Kõik need riistad, mis Hiiram tegi kuningas Saalomonile Issanda koja jaoks, olid hiilgavast vasest.
\par 46 Kuningas laskis need valada Jordani uhtmaal savimaa sees Sukkoti ja Saartani vahel.
\par 47 Ja Saalomon jättis kõik riistad vaagimata, sest neid oli üpris palju; vase kaalu ei arvestatudki.
\par 48 Saalomon valmistas ka kõik need riistad, mis pidid olema Issanda kojas: kuldaltari ja kuldlaua, mille peal olid ohvrileivad;
\par 49 ja puhtast kullast lambijalad, viis paremale ja viis vasakule poole kõige pühama paiga ette; ja kullast õied, lambid ja tahikäärid;
\par 50 ja puhtast kullast liuad, noad, peekrid, suitsutusrohupannid ja sütepannid; ja kullast esiküljed kõige pühama paiga ustel koja sisemuses, samuti pealöövi ustel.
\par 51 Kui kõik tööd, mis kuningas Saalomon tegi Issanda koja jaoks, olid valmis saanud, viis Saalomon kotta kõik, mis ta isa Taavet oli pühitsenud: ta pani hõbeda ja kulla ja riistad Issanda koja varanduste hulka.

\chapter{8}

\par 1 Siis kogus Saalomon Iisraeli vanemad ja kõik suguharude peamehed, Iisraeli laste perekondade eestseisjad, kuningas Saalomoni juurde Jeruusalemma Issanda seaduselaegast Taaveti linnast, see on Siionist, üles tooma.
\par 2 Ja kõik Iisraeli mehed kogunesid pühadeks kuningas Saalomoni juurde eetanimikuus, mis on seitsmes kuu.
\par 3 Ja kui kõik Iisraeli vanemad olid tulnud, siis tõstsid preestrid laeka
\par 4 ja tõid üles Issanda laeka, kogudusetelgi ja kõik pühad riistad, mis telgis olid; preestrid ja leviidid tõid need üles.
\par 5 Kuningas Saalomon oli laeka ees, ja koos temaga oli kogu Iisraeli kogudus, kes oli kogunenud tema juurde; nad ohverdasid nii palju lambaid, kitsi ja veiseid, et neid ei saadud lugeda ega kokku arvata.
\par 6 Ja preestrid tõid Issanda seaduselaeka selle paika koja tagaruumi, kõige pühamasse paika keerubite tiibade alla,
\par 7 sest keerubid laotasid oma tiivad laeka paiga üle; keerubid katsid laegast ja selle kandekange pealtpoolt.
\par 8 Ja kangid olid nii pikad, et kangide otsi võis näha pühamust, kõige pühama paiga eest, ent neid ei nähtud väljast; ja need on seal tänapäevani.
\par 9 Laekas ei olnud muud kui need kaks kivilauda, mis Mooses oli Hoorebil sinna pannud, kui Issand tegi Iisraeli lastega lepingu nende tulles Egiptusemaalt.
\par 10 Ja kui preestrid pühamust väljusid, siis sündis, et pilv täitis Issanda koja
\par 11 ja preestrid ei võinud jääda teenima pilve pärast: sest Issanda auhiilgus oli täitnud Issanda koja.
\par 12 Siis kõneles Saalomon: „Issand on öelnud, et ta tahab elada pimeduses.
\par 13 Mina olen sulle ehitanud valitsuskoja, su igavese eluaseme paiga.”
\par 14 Ja kuningas pööras oma palge ja õnnistas kogu Iisraeli kogudust, ja kogu Iisraeli kogudus seisis.
\par 15 Ja ta ütles: „Kiidetud olgu Issand, Iisraeli Jumal, kes oma käega on tõeks teinud, mida ta oma suuga on kõnelnud mu isale Taavetile, öeldes:
\par 16 Sellest päevast alates, kui ma tõin oma Iisraeli rahva ära Egiptusest, ei ole ma üheltki Iisraeli suguharult valinud ühtegi linna, kuhu ehitada koda, kus oleks mu nimi; aga ma valisin Taaveti, et ta valitseks mu Iisraeli rahva üle.
\par 17 Mu isal Taavetil oli küll südame peal ehitada koda Issanda, Iisraeli Jumala nimele,
\par 18 aga Issand ütles mu isale Taavetile: Et sul südame peal on ehitada mu nimele koda, siis oled sa hästi teinud, kui see sul südame peal on.
\par 19 Ometi ei ehita sina seda koda, vaid su poeg, kes su niudeist välja tuleb, ehitab mu nimele koja.
\par 20 Issand on pidanud oma sõna, mis ta kõneles, ja mina olen tõusnud oma isa Taaveti asemele ja istun Iisraeli aujärjel, nõnda nagu Issand on kõnelnud, ja ma olen ehitanud koja Issanda, Iisraeli Jumala nimele.
\par 21 Ja ma olen valmistanud paiga laekale, mille sees on Issanda leping, mille ta tegi meie vanematega, kui ta tõi nad ära Egiptusemaalt.”
\par 22 Siis astus Saalomon Issanda altari ette kogu Iisraeli koguduse juuresolekul ja sirutas oma käed taeva poole
\par 23 ning ütles: „Issand, Iisraeli Jumal! Sinu sarnast jumalat ei ole ülal taevas ega all maa peal: sina pead lepingut ja osadust oma sulastega, kes käivad su ees kõigest südamest,
\par 24 sina oled pidanud oma sulasele, mu isale Taavetile, mis sa temale olid lubanud. Jah, mis sa oma suuga oled kõnelnud, selle oled sa oma käega tõeks teinud, nõnda nagu see täna on sündinud.
\par 25 Ja nüüd, Issand, Iisraeli Jumal, pea oma sulasele, mu isale Taavetile, mis sa temale tõotasid, öeldes: Ei puudu sul minu palge ees mees, kes istub Iisraeli aujärjel, kui ainult su pojad peavad oma teed, käies mu ees, nõnda nagu sina oled käinud mu ees.
\par 26 Ja nüüd, Iisraeli Jumal, saagu ometi tõeks su sõnad, mis sa oled kõnelnud oma sulasele, mu isale Taavetile!
\par 27 Aga kas Jumal tõesti peaks elama maa peal? Vaata, taevas ja taevaste taevas ei mahuta sind, veel vähem siis see koda, mille ma olen ehitanud.
\par 28 Aga pöördu oma sulase palve ja ta anumise poole, Issand, mu Jumal, et sa kuuleksid kaebehüüdu ja palvet, mida su sulane täna palvetab sinu ees,
\par 29 et su silmad oleksid lahti ööd ja päevad selle koja kohal, selle paiga kohal, mille kohta sa oled öelnud: Seal peab olema minu nimi!, et sa kuuleksid palvet, mida su sulane palvetab selle paiga poole!
\par 30 Kuule siis oma sulase ja oma Iisraeli rahva anumist, kuidas nad palvetavad selle paiga poole! Jah, Kuule paigast, kus sa elad - taevast! Ja kui sa kuuled, siis anna andeks!
\par 31 Kui keegi oma ligimese vastu pattu teeb ja talle pannakse peale vanne teda vannutades, ja vandeasi tuleb sinu altari ette siia kotta,
\par 32 siis kuule sina taevast ja tee ning mõista õigust oma sulastele: mõista süüdlane süüdi, pane tema teod ta pea peale, õige aga mõista õigeks, anna temale ta õigust mööda!
\par 33 Ja kui su Iisraeli rahvas lüüakse maha vaenlase ees, sellepärast et nad sinu vastu on pattu teinud, ent nad pöörduvad tagasi su juurde ja kiidavad sinu nime, palvetavad su poole ja anuvad sind selles kojas,
\par 34 siis kuule sina taevast ja anna andeks oma Iisraeli rahva patt ja too nad tagasi maale, mille sa oled andnud nende vanematele!
\par 35 Kui taevas on suletud ja vihma ei ole, sellepärast et nad on sinu vastu pattu teinud, aga nad palvetavad selle paiga poole ja kiidavad sinu nime ja pöörduvad oma patust, sellepärast et sa neid oled alandanud,
\par 36 siis kuule sina taevast ja anna andeks oma sulaste ja oma Iisraeli rahva patt, sest sina õpetad neile head teed, mida nad peavad käima, ja anna vihma oma maale, mille sa oled andnud pärisosaks oma rahvale!
\par 37 Kui maale tuleb nälg, kui tuleb katk, kui tulevad viljakõrvetus või -rooste, rohutirtsud ja mardikad, kui vaenlane teda rõhub ta maa väravais, kui tabab mingi nuhtlus või mingi haigus,
\par 38 kui siis iganes palvetatakse või anutakse, olgu ükskõik missuguse inimese või kogu su Iisraeli rahva poolt, kui nad igaüks tunnevad oma südame häda ja sirutavad oma käed selle koja poole,
\par 39 siis kuule sina taevast, oma asupaigast, ja anna andeks ning tee nõnda, et sa annad igaühele tema tegusid mööda, nagu sa tunned tema südant, sest sina üksi tunned kõigi inimlaste südameid,
\par 40 et nad kardaksid sind kõigil neil päevil, mil nad elavad maal, mille sa oled andnud meie vanemaile!
\par 41 Aga ka võõramaalast, kes ei ole sinu Iisraeli rahva hulgast, tuleb aga kaugelt maalt sinu nime pärast
\par 42 - sest nad kuulevad sinu suurest nimest, sinu vägevast käest ja sinu väljasirutatud käsivarrest - ja ta tuleb ning palvetab selle koja poole,
\par 43 kuule sina taevast, oma asupaigast, ja tee kõike, mille pärast võõras sinu poole hüüab, et kõik maa rahvad õpiksid tundma sinu nime ja kardaksid sind, nõnda nagu su Iisraeli rahvas, ja et nad teaksid, et sinu nimi on pandud kojale, mille ma olen ehitanud!
\par 44 Kui su rahvas läheb sõtta oma vaenlase vastu, teekonnale, kuhu sa nad läkitad, ja nad palvetavad Issanda poole selle linna suunas, mille sina oled valinud, ja koja suunas, mille mina olen ehitanud su nimele,
\par 45 siis kuule taevast nende palvet ja anumist ja tee neile õigust!
\par 46 Kui nad sinu vastu pattu teevad - sest pole inimest, kes pattu ei tee - ja sina vihastad nende peale ning annad nad vaenlase kätte, nõnda et nad viiakse vangi vaenlase maale, kaugemale või lähemale,
\par 47 aga kui nad siis seda südamesse võtavad maal, kuhu nad on vangi viidud, ja pöörduvad ning anuvad sind oma vangistusemaal, öeldes: Me oleme pattu teinud, oleme eksinud ja saanud süüdlasteks,
\par 48 ja pöörduvad sinu poole kõigest oma südamest ja kõigest oma hingest nende vaenlaste maal, kes on nad vangi viinud, ja paluvad sind oma maa suunas, mille sa oled andnud nende vanematele, linna suunas, mille sina oled valinud, ja koja suunas, mille mina olen ehitanud sinu nimele,
\par 49 siis kuule taevast, oma asupaigast, nende palvet ja anumist, tee neile õigust
\par 50 ja anna oma rahvale andeks see, milles nad sinu vastu on pattu teinud, ja kõik nende vastuhakkamised, millega nad sulle on vastu hakanud, ja lase neil leida halastust oma vangiviijate ees, et need halastaksid nende peale!
\par 51 Sest nad on sinu rahvas ja sinu pärisosa, kelle sa tõid ära Egiptusest, rauasulatusahjust.
\par 52 Olgu su silmad lahti su sulase anumise ja su Iisraeli rahva anumise poole, et sa neid kuuleksid kõiges, mille pärast nad sind hüüavad!
\par 53 Sest sina oled nad eraldanud enesele pärisosaks kõigi maa rahvaste hulgast, nõnda nagu sa rääkisid oma sulase Moosese läbi, kui sa tõid meie vanemad ära Egiptusest, oh Issand Jumal!”
\par 54 Ja kui Saalomon oli lõpetanud palvetamise Issanda poole, kõik need palved ja anumised, siis ta tõusis üles Issanda altari eest, kus ta oli põlvitanud, käed sirutatud taeva poole,
\par 55 ja astus ette ning õnnistas kogu Iisraeli kogudust, öeldes valju häälega:
\par 56 „Kiidetud olgu Issand, kes on andnud rahu oma Iisraeli rahvale, nõnda nagu ta on öelnud! Ainsatki sõna ei ole langenud tühja kõigist ta headest sõnadest, mis ta oma sulase Moosese läbi on rääkinud.
\par 57 Olgu Issand, meie Jumal, meiega, nõnda nagu ta on olnud meie vanematega; tema ärgu jätku meid maha ja ärgu loobugu meist,
\par 58 vaid ta pööraku meie südamed enese poole, et käiksime kõigil ta teedel ja peaksime ta käske, määrusi ja seadlusi, nagu ta meie vanemaid on käskinud!
\par 59 Ja olgu need minu sõnad, millega ma olen anunud Issanda ees, Issanda, meie Jumala juures päevad ja ööd, et ta teeks õigust oma sulasele ja oma Iisraeli rahvale, nagu iga päev on tarvis,
\par 60 nõnda et kõik maa rahvad saaksid teada, et Issand on Jumal, aga mitte keegi teine!
\par 61 Ja teie süda olgu siiras Issanda, meie Jumala vastu, et te käiksite tema määruste järgi ja peaksite tema käske nagu täna!”
\par 62 Siis ohverdas kuningas ja kogu Iisrael koos temaga Issanda ees tapaohvreid.
\par 63 Saalomon ohverdas tänuohvriks, mida ta Issandale ohverdas, kakskümmend kaks tuhat veist ning sada kakskümmend tuhat lammast ja kitse. Nõnda pühitsesid kuningas ja kõik Iisraeli lapsed Issanda koja.
\par 64 Selsamal päeval pühitses kuningas Issanda koja ees oleva õue keskpaiga, sest ta ohverdas seal põletusohvri ja roaohvri ning tänuohvri rasvu, kuna vaskaltar Issanda ees oli liiga väike, et mahutada põletus- ja roaohvrit ning tänuohvri rasvu.
\par 65 Nõnda pidas Saalomon sel ajal Issanda, meie Jumala ees püha, ja kogu Iisrael koos temaga - üks suur kogudus Hamati teelahkmest kuni Egiptuseojani - seitse päeva ja seitse päeva, see on neliteist päeva.
\par 66 Kaheksandal päeval saatis ta rahva ära; nad õnnistasid kuningat ja läksid oma telkide juurde rõõmsatena ja heas meeleolus selle hea pärast, mida Issand oli teinud oma sulasele Taavetile ja oma Iisraeli rahvale.

\chapter{9}

\par 1 Kui Saalomon oli lõpetanud Issanda koja ja kuningakoja ehitamise ja täitnud kõik Saalomoni südamesoovid, mis ta igatses teoks teha,
\par 2 siis ilmutas Issand ennast Saalomonile teist korda, nagu ta oli ennast ilmutanud temale Gibeonis.
\par 3 Ja Issand ütles temale: „Ma olen kuulnud su palvet ja anumist, mis sa oled saatnud minu poole. Ma olen pühitsenud selle koja, mille sa ehitasid, et ma võiksin panna sinna oma nime igaveseks ajaks. Mu silmad ja mu süda on alati seal.
\par 4 Kui sa käid minu palge ees, nõnda nagu käis su isa Taavet, tehes vaga ja õiglase südamega kõike, mida ma sind käsin, ja pead mu määrusi ja seadlusi,
\par 5 siis ma kinnitan igaveseks su kuningriigi aujärje Iisraelis, nõnda nagu ma olen rääkinud su isale Taavetile, öeldes: Sul ei puudu mees Iisraeli aujärjel.
\par 6 Aga kui te taganete minu järelt, teie ja teie lapsed, ega pea mu käske ja seadlusi, mis ma teile olen andnud, vaid lähete ning teenite teisi jumalaid ja kummardate neid,
\par 7 siis ma hävitan Iisraeli sellelt maalt, mille ma neile olen andnud, ja koja, mille ma olen pühitsenud oma nimele, ma tõukan ära oma palge eest. Ja Iisrael saab kõnekäänuks ja pilkesõnaks kõigi rahvaste keskel.
\par 8 Ja kui kõrge see koda ka oli, ometi kohkub igaüks, kes sellest mööda läheb, ja vilistab. Ja kui küsitakse: Mispärast talitas Issand nõnda selle maa ja selle kojaga?,
\par 9 siis vastatakse: Sellepärast, et nad jätsid maha Issanda, oma Jumala, kes tõi nende vanemad ära Egiptusemaalt, ja haarasid teiste jumalate järele ning kummardasid ja teenisid neid. Sellepärast on Issand lasknud nende peale tulla kogu selle õnnetuse.”
\par 10 Möödus kakskümmend aastat, mille jooksul Saalomon ehitas need mõlemad kojad - Issanda koja ja kuningakoja.
\par 11 Et Hiiram, Tüürose kuningas, oli Saalomoni toetanud seedri- ja küpressipuude ja kullaga, niipalju kui ta soovis, siis andis kuningas Saalomon Hiiramile kakskümmend linna Galileamaal.
\par 12 Aga kui Hiiram tuli Tüürosest vaatama neid linnu, mis Saalomon temale oli andnud, siis ei olnud need tema silmis head,
\par 13 ja ta ütles: „Mis linnad need on, mis sa mulle oled andnud, mu vend?„ Seepärast hüütakse neid tänapäevani ”Kabuuli maaks”.
\par 14 Hiiram oli aga kuningale saatnud sada kakskümmend talenti kulda.
\par 15 Ja niisugune oli lugu töökohustusega, mille kuningas Saalomon oli peale pannud, et ehitada Issanda koda, oma koda, kindlust, Jeruusalemma müüre, Haasorit, Megiddot ja Geserit:
\par 16 vaarao, Egiptuse kuningas, oli tulnud ja vallutanud Geseri, põletanud selle tulega ja tapnud kaananlased, kes linnas elasid, ja oli selle andnud kaasavaraks oma tütrele, Saalomoni naisele.
\par 17 Saalomon ehitas siis üles Geseri ja alumise Beet-Hooroni,
\par 18 Baalati ja Taamari maal olevas kõrbes
\par 19 ja kõik varustuslinnad, mis Saalomonil olid, ja sõjavankrite linnad ja ratsanike linnad ja muud, mida Saalomon soovis ehitada Jeruusalemmas ja Liibanonil ja kõikjal oma valitsusmaal.
\par 20 Kogu selle rahva, kes oli alles jäänud emorlastest, hettidest, perislastest, hiivlastest ja jebuuslastest, need, kes ei olnud Iisraeli laste hulgast,
\par 21 nende järglased, kes pärast neid olid maale alles jäänud, keda Iisraeli lapsed ei olnud suutnud sootuks hävitada, need pani Saalomon teoorjusesse kuni tänapäevani.
\par 22 Aga Iisraeli lastest ei teinud Saalomon mitte kedagi orjaks, vaid neist said sõjamehed, tema sulased ja pealikud, tema vankrivõitlejad, sõjavankrite ja ratsanike pealikud.
\par 23 Neid asevalitsejate ametimehi, kes olid Saalomoni tööde ülevaatajad, oli viissada viiskümmend; nemad valitsesid tööd tegeva rahva üle.
\par 24 Vaarao tütar tuli siis Taaveti linnast üles oma kotta, mille Saalomon temale oli ehitanud; sel ajal ehitas ta ka kindluse.
\par 25 Ja Saalomon ohverdas kolm korda aastas põletus- ja tänuohvreid altaril, mille ta Issandale oli ehitanud, ja suitsutas altaril, mis oli Issanda ees. Nõnda sai ta koja valmis.
\par 26 Kuningas Saalomon ehitas ka laevu Esjon-Geberis, mis on Eeloti lähedal Kõrkjamere rannas Edomimaal.
\par 27 Hiiram läkitas aga laevadesse oma sulaseid, laevamehi, kes tundsid merd, koos Saalomoni sulastega.
\par 28 Need läksid Oofiri ja võtsid sealt nelisada kakskümmend talenti kulda ning tõid kuningas Saalomonile.

\chapter{10}

\par 1 Kui Seeba kuninganna kuulis Saalomoni kuulsusest Issanda nime tõttu, siis ta tuli teda mõistuküsimustega kimbutama.
\par 2 Ta tuli Jeruusalemma väga suure saatjaskonnaga ja kaamelitega, kes kandsid palsameid, väga palju kulda ja kalliskive. Ja kui ta jõudis Saalomoni juurde, siis ta kõneles temaga kõigest, mis tal südame peal oli.
\par 3 Aga Saalomon vastas kõigile tema küsimustele; ükski tema küsimus ei olnud kuningale liiga raske, ta suutis vastata kõigile.
\par 4 Kui Seeba kuninganna nägi kõike Saalomoni tarkust, ja koda, mille ta oli ehitanud,
\par 5 rooga ta laual, kuidas ta sulased istusid, tema teenrite teenimist ja nende riietust, tema jooke ja tema põletusohvreid, mis ta ohverdas Issanda kojas, siis ta jäi otse hingetuks.
\par 6 Ja ta ütles kuningale: „See kõne oli tõsi, mis ma oma maal kuulsin sinust ja sinu tarkusest.
\par 7 Aga mina ei uskunud neid jutte enne, kui ma tulin ja nägin oma silmaga; ja vaata, mulle ei olnud räägitud pooltki. Sinul on rohkem tarkust ja vara kui kuulduses, mida ma olin kuulnud.
\par 8 Õnnelikud on su mehed, õnnelikud on need su sulased, kes seisavad alati su ees ja kuulevad su tarkust.
\par 9 Kiidetud olgu Issand, su Jumal, kellele sa nõnda meeldisid, et ta pani su Iisraeli aujärjele! Sellepärast et Issand Iisraeli igavesti armastab, on ta sind pannud kuningaks, et sa teeksid, mis on kohus ja õige.”
\par 10 Ja ta andis kuningale sada kakskümmend talenti kulda ning väga palju palsameid ja kalliskive; nii suurt hulka palsameid kui see, mis Seeba kuninganna andis kuningas Saalomonile, ei toodud enam kunagi.
\par 11 Ka Hiirami laevad, mis olid toonud kulda Oofirist, tõid Oofirist väga palju almugipuud ja kalliskive.
\par 12 Ja kuningas valmistas almugipuust tugiosad Issanda kojale ja kuningakojale, samuti kandleid ja naableid lauljaile; niisugust almugipuud ei ole toodud ega nähtud enam tänapäevani.
\par 13 Ja kuningas Saalomon andis Seeba kuningannale kõike, mida ta soovis ja palus, lisaks sellele, mis temale oli antud kuningas Saalomoni jõu kohaselt; siis kuninganna pöördus ümber ja läks tagasi oma maale, tema ja ta sulased.
\par 14 Ja kulla kaal, mis tuli Saalomoni kätte ühe aasta jooksul, oli kuussada kuuskümmend kuus talenti kulda,
\par 15 lisaks see, mis tuli suurkaupmeestelt ja kaubitsejate tulust, samuti kõigilt Araabia kuningailt ja maa asevalitsejailt.
\par 16 Ja kuningas Saalomon valmistas kakssada suurt kilpi taotud kullast, tarvitades kuussada seeklit kulda üheks kilbiks;
\par 17 ja kolmsada väiksemat kilpi taotud kullast, tarvitades kolm miini kulda üheks kilbiks; ja kuningas pani need Liibanonimetsakotta.
\par 18 Ja kuningas valmistas suure elevandiluust aujärje ning kattis selle puhta kullaga.
\par 19 Aujärjel oli kuus astet ja aujärje tagakülg oli pealt ümmargune; kummalgi pool istepaika olid käetoed ja käetugede kõrval seisid kaks lõvi.
\par 20 Ja seal seisid kaksteist lõvi kummalgi pool kuuel astmel; niisugust ei ole valmistatud mitte üheski muus kuningriigis.
\par 21 Ja kõik kuningas Saalomoni joogiriistad olid kullast, samuti olid kõik Liibanonimetsakoja riistad puhtast kullast; hõbedasi ei olnudki, seda ei peetud Saalomoni ajal mikski,
\par 22 sest kuningal olid merel Tarsise laevad koos Hiirami laevadega: igal kolmandal aastal tulid Tarsise laevad ja tõid kulda ja hõbedat, elevandiluud, pärdikuid ja paabulinde.
\par 23 Ja kuningas Saalomon sai rikkuse ja tarkuse poolest suurimaks kõigist kuningaist maa peal.
\par 24 Ja kogu maailm püüdis näha saada Saalomoni palet, et kuulda tema tarkust, mille Jumal oli pannud tema südamesse.
\par 25 Ja nad tõid igaüks oma anni: hõbe- ja kuldriistu, riideid, relvi, palsameid, hobuseid ja muulasid; nõnda aastast aastasse.
\par 26 Ja Saalomon kogus sõjavankreid ja ratsanikke, ja tal oli tuhat nelisada sõjavankrit ja kaksteist tuhat ratsanikku; need ta paigutas vankrilinnadesse ja kuninga juurde Jeruusalemma.
\par 27 Ja kuningas hoolitses, et Jeruusalemmas oli hõbedat nagu kive, ja seedripuid nõnda palju nagu metsviigipuid Madalmaal.
\par 28 Saalomoni hobused olid toodud Egiptusest ja Kiliikiast; kuninga ülesostjad tõid neid Kiliikiast kindla hinna eest.
\par 29 Egiptusest toodi vanker kuuesaja hõbeseekli eest ja hobune saja viiekümne eest, ja nõnda toodi neid nende vahendusel kõigile hettide ja süürlaste kuningaile.

\chapter{11}

\par 1 Aga kuningas Saalomon armastas vaarao tütre kõrval paljusid võõramaa naisi - moabe, ammonlasi, edomlasi, siidonlasi ja hette -
\par 2 rahvaist, kelle kohta Issand oli öelnud Iisraeli lastele: „Ärge minge nende sekka ja need ärgu tulgu teie sekka, sest nad pööravad tõesti teie südamed oma jumalate poole!” Neisse kiindus Saalomon armastusega.
\par 3 Temal oli seitsesada vürstisoost naist ja kolmsada liignaist; ta naised aga pöörasid tema südame.
\par 4 See sündis Saalomoni vanas eas, et ta naised pöörasid tema südame teiste jumalate poole; ta süda ei olnud siiras Issanda, oma Jumala vastu, nagu oli olnud tema isa Taaveti süda.
\par 5 Saalomon käis siis siidonlaste jumalanna Astarte ja ammonlaste jäleduse Milkomi järel.
\par 6 Saalomon tegi kurja Issanda silmis ega käinud ustavalt Issanda järel nagu tema isa Taavet.
\par 7 Sel ajal ehitas Saalomon ohvrikünka Kemosele, moabide jäledusele, Jeruusalemma ees oleva mäe peale, ja Moolokile, ammonlaste jäledusele.
\par 8 Seda tegi ta kõigi oma võõramaalastest naiste heaks, kes suitsutasid ja ohverdasid oma jumalatele.
\par 9 Aga Issand vihastas Saalomoni peale, et ta oma südame oli ära pööranud Issandast, Iisraeli Jumalast, kes oli ennast temale kaks korda ilmutanud
\par 10 ja oli temale andnud käsu selle asja pärast, et ta ei käiks teiste jumalate järel; ent tema ei olnud pidanud, mis Issand oli käskinud.
\par 11 Ja Issand ütles Saalomonile: „Sellepärast et sinuga on nõnda sündinud ja et sa ei ole pidanud minu lepingut ja minu seadusi, mis ma sulle andsin, kisun ma tõesti kuningriigi sinult ja annan sinu sulastele.
\par 12 Aga su isa Taaveti pärast ei tee ma seda sinu päevil: ma kisun selle ära sinu poja käest.
\par 13 Ometi ei kisu ma ära kogu kuningriiki: ühe suguharu ma annan su pojale oma sulase Taaveti pärast ja Jeruusalemma pärast, mille ma olen välja valinud.”
\par 14 Ja Issand tõstis Saalomonile vastase, edomlase Hadadi; too oli Edomi kuningasoost.
\par 15 Kui Taavet oli löönud Edomit ja väepealik Joab läks mahalööduid matma ja hävitas Edomis kogu meessoo -
\par 16 sest Joab ja kogu Iisrael viibisid seal kuus kuud, kuni ta Edomis oli hävitanud kogu meessoo -,
\par 17 siis põgenes Hadad, tema ja mõned Edomi mehed ta isa sulaste hulgast koos temaga, et minna Egiptusesse; Hadad oli alles väike poiss.
\par 18 Nad läksid Midjanist teele ja tulid Paaranisse; Paaranist võtsid nad mehi enestega kaasa ja tulid Egiptusesse vaarao, Egiptuse kuninga juurde; vaarao andis Hadadile koja, lubas temale leiba ja andis temale maad.
\par 19 Hadad leidis suurt armu vaarao silmis, kes andis temale naiseks oma naise õe, kuninganna Tahpenesi õe.
\par 20 Tahpenesi õde tõi temale ilmale ta poja Genubati ja Tahpenes võõrutas tolle vaarao kojas; ja Genubat jäi vaarao kotta, vaarao poegade seltsi.
\par 21 Kui Hadad Egiptuses kuulis, et Taavet oli läinud magama oma vanemate juurde ja et väepealik Joab oli surnud, siis ütles Hadad vaaraole: „Lase mind, et võiksin minna oma maale!”
\par 22 Aga vaarao küsis temalt: „Mis sul minu juures puudub, et sa nüüd tahad minna oma maale?„ Ja ta vastas: ”Mitte midagi, kuid lase mind siiski minna!”
\par 23 Ja Jumal tõstis temale vastaseks ka Resoni, Eljada poja, kes oli põgenenud oma isanda, Sooba kuninga Hadadeseri juurest.
\par 24 Tema kogus enese juurde mehi ja sai röövjõugu pealikuks; pärast seda kui Taavet neid oli tapnud, läksid nad Damaskusesse, elasid seal ja valitsesid Damaskuses.
\par 25 Tema oli Iisraeli vastane kogu Saalomoni eluaja, tehes kurja nagu Hadadki; ta põlgas Iisraeli ja ta sai Süüria kuningaks.
\par 26 Ka Saalomoni sulane Jerobeam, Nebati poeg, efraimlane Seredast, kelle ema oli lesknaine Seruua, tõstis oma käe kuninga vastu.
\par 27 Ja see on põhjus, miks ta tõstis oma käe kuninga vastu: Saalomon ehitas kindlust ja sulges läbimurdekoha oma isa Taaveti linnas.
\par 28 Jerobeam oli väga tubli mees, ja kui Saalomon nägi, kuidas see noor mees tööd tegi, siis ta pani tema kogu Joosepi soo töökohustuse ülevaatajaks.
\par 29 Aga sel ajal sündis, et Jerobeam läks Jeruusalemmast välja ja prohvet, siilolane Ahija, kellel oli uus kuub seljas, juhtus teel temale vastu; ainult nemad kahekesi olid väljal.
\par 30 Siis Ahija haaras kinni sellest uuest kuuest, mis tal seljas oli, ja kiskus selle kaheteistkümneks tükiks
\par 31 ning ütles Jerobeamile: „Võta enesele kümme tükki, sest nõnda ütleb Issand, Iisraeli Jumal: Vaata, ma kisun kuningriigi Saalomoni käest ja annan kümme suguharu sinule.
\par 32 Aga üks suguharu jäägu temale mu sulase Taaveti pärast ja Jeruusalemma linna pärast, mille ma olen valinud kõigi Iisraeli suguharude seast,
\par 33 sellepärast et nad jätsid mind maha ja kummardasid siidonlaste jumalannat Astartet, moabide jumalat Kemost ja ammonlaste jumalat Milkomi ega käinud minu teedel, et teha, mis õige on minu silmis minu määruste ja seadluste järgi, nõnda nagu tema isa Taavet.
\par 34 Mina aga ei võta tema käest siiski mitte tervet kuningriiki, vaid lasen teda olla vürstiks kogu ta eluaja oma sulase Taaveti pärast, kelle ma valisin, kes pidas mu käske ja määrusi.
\par 35 Aga tema poja käest ma võtan kuningriigi ja annan selle sinule - need kümme suguharu.
\par 36 Tema pojale ma annan ühe suguharu, et mu sulasel Taavetil võiks alatiselt olla lamp mu palge ees Jeruusalemmas, linnas, mille ma enesele olen valinud, et panna sinna oma nimi.
\par 37 Aga sind ma võtan, et sa valitseksid kõige üle, mida su hing himustab: sina saad Iisraeli kuningaks!
\par 38 Ja kui sa võtad kuulda kõike, mis ma sind käsin, ja käid minu teedel ning teed, mis minu silmis õige on, pidades mu määrusi ja käske, nõnda nagu mu sulane Taavet tegi, siis ma olen sinuga ja ehitan sulle püsiva koja, nõnda nagu ma ehitasin Taavetile, ja annan Iisraeli sinule.
\par 39 Sel põhjusel ma alandan Taaveti sugu, aga mitte alatiseks.”
\par 40 Siis püüdis Saalomon Jerobeami surmata; aga Jerobeam võttis kätte ja põgenes Egiptusesse, Egiptuse kuninga Siisaki juurde; ja ta jäi Egiptusesse kuni Saalomoni surmani.
\par 41 Ja mis veel tuleks öelda Saalomonist ja kõigest, mis ta tegi, ja tema tarkusest, eks sellest ole kirjutatud Saalomoni Tegude raamatus?
\par 42 Ja aega, mis Saalomon Jeruusalemmas valitses kogu Iisraeli üle, oli nelikümmend aastat.
\par 43 Siis Saalomon läks magama oma vanemate juurde ja ta maeti oma isa Taaveti linna. Ja tema poeg Rehabeam sai tema asemel kuningaks.

\chapter{12}

\par 1 Ja Rehabeam läks Sekemisse, sest kogu Iisrael oli tulnud Sekemisse teda kuningaks tõstma.
\par 2 Seda kuulis Jerobeam, Nebati poeg, sest ta oli alles Egiptuses, kuhu ta oli põgenenud kuningas Saalomoni eest, ja Jerobeam elas Egiptuses.
\par 3 Ja nad läkitasid käskjalad ning kutsusid tema; ja Jerobeam ning kogu Iisraeli kogudus tulid, ja nad rääkisid Rehabeamiga, öeldes:
\par 4 „Sinu isa tegi meie ikke raskeks. Aga kergenda nüüd sina oma isa rasket teenistust ja tema ränka iket, mille ta meile on peale pannud, siis me teenime sind!”
\par 5 Aga tema vastas neile: „Minge veel kolmeks päevaks ära ja tulge siis tagasi mu juurde!” Ja rahvas läks.
\par 6 Ja kuningas Rehabeam pidas nõu vanematega, kes olid seisnud tema isa Saalomoni teenistuses, kui ta veel elas, öeldes: „Mis nõu te annate, et sellele rahvale midagi vastata?”
\par 7 Nad vastasid temale, öeldes: „Kui sa nüüd tahad olla sulaseks sellele rahvale ja teenida neid, vastad neile ja räägid neile häid sõnu, siis jäävad nad sulle igavesti sulasteks.”
\par 8 Aga ta hülgas vanemate nõu, mida need temale andsid, ja pidas nõu noorematega, kes olid kasvanud koos temaga ja seisid tema teenistuses.
\par 9 Ja ta ütles neile: „Mis nõu te annate, et saaksime midagi vastata sellele rahvale, kes minuga kõneles, öeldes: Kergenda iket, mille su isa on meile peale pannud?”
\par 10 Ja nooremad, kes olid kasvanud koos temaga, vastasid temale, öeldes: „Ütle nõnda rahvale, kes sinuga on kõnelnud ja sulle on öelnud: Su isa tegi meie ikke raskeks, aga kergenda sina seda meile - Ütle neile nõnda: Minu väike sõrm on jämedam kui mu isa puusad!
\par 11 Ja kui nüüd mu isa on teile peale pannud ränga ikke, siis mina annan teie ikkele veel lisa. Kui mu isa karistas teid piitsadega, siis mina karistan teid okaspiitsadega.”
\par 12 Ja Jerobeam ja kogu rahvas tulid kolmandal päeval Rehabeami juurde, nagu kuningas oli käskinud, öeldes: „Tulge kolmandal päeval tagasi mu juurde!”
\par 13 Ja kuningas vastas rahvale karmilt ning hülgas vanemate nõu, mis need olid andnud,
\par 14 ja kõneles neile nooremate nõu järgi, öeldes: „Mu isa on teie ikke teinud rängaks, aga mina annan teie ikkele veel lisa. Mu isa karistas teid piitsadega, aga mina karistan teid okaspiitsadega.”
\par 15 Ja kuningas ei võtnud kuulda rahvast, sest see pööre oli Issandalt, et läheks tõeks tema sõna, mis Issand oli kõnelnud siilolase Ahija läbi Jerobeamile, Nebati pojale.
\par 16 Kui kogu Iisrael nägi, et kuningas ei võtnud neid kuulda, siis vastas rahvas kuningale, öeldes nõnda: „Mis osa on meil Taavetis? Ei ole meil pärisosa Iisai pojas. Oma telkidesse, Iisrael! Karjata nüüd omaenese sugu, Taavet!” Ja Iisrael läks Oma telkidesse.
\par 17 Siis Rehabeam valitses nende Iisraeli laste üle, kes elasid Juuda linnades.
\par 18 Ja kui kuningas Rehabeam läkitas Adorami, kes oli sunnitöö ülevaataja, siis kogu Iisrael viskas teda kividega, nõnda et ta suri; kuningas Rehabeam ise astus aga kärmesti vankrisse ja põgenes Jeruusalemma.
\par 19 Nõnda on Iisrael Taaveti soost taganenud kuni tänapäevani.
\par 20 Ja kui kogu Iisrael kuulis, et Jerobeam oli tagasi tulnud, siis läkitasid nad talle järele ja kutsusid ta koguduse juurde ning tõstsid ta kuningaks kogu Iisraelile; Taaveti soo järel ei käinud muud kui üksnes Juuda suguharu.
\par 21 Ja kui Rehabeam tuli Jeruusalemma, siis ta kogus kokku kogu Juuda soo ja Benjamini suguharu, sada kaheksakümmend tuhat valitud sõjakõlvulist meest, et sõdida Iisraeli soo vastu ja taastada kuningriik Rehabeamile, Saalomoni pojale.
\par 22 Aga jumalamehele Semajale tuli Jumala sõna, kes ütles:
\par 23 „Räägi Saalomoni poja Rehabeamiga, Juuda kuningaga, ja kogu Juuda ja Benjamini sooga ja ülejäänud rahvaga ning ütle:
\par 24 Nõnda ütleb Issand: Te ei tohi minna ega sõdida oma vendade Iisraeli laste vastu! Igamees mingu tagasi koju, sest see lugu on lastud sündida minu poolt!” Ja nad võtsid kuulda Issanda sõna, pöördusid ümber ja läksid, nagu Issand oli käskinud.
\par 25 Siis Jerobeam kindlustas Sekemi Efraimi mäestikus ja elas seal; ja sealt ta läks ning kindlustas Penueli.
\par 26 Ja Jerobeam mõtles oma südames: „Nüüd läheb kuningriik tagasi Taaveti soole.
\par 27 Kui see rahvas läheb ohverdama tapaohvreid Issanda kojas Jeruusalemmas, siis pöördub selle rahva süda nende isanda, Juuda kuninga Rehabeami poole ja nad tapavad minu ning pöörduvad Juuda kuninga Rehabeami poole.”
\par 28 Ja kuningas pidas nõu, valmistas siis kaks kuldvasikat ning ütles rahvale: „Saagu teile küllalt Jeruusalemma minekust! Vaata, Iisrael, siin on sinu jumalad, kes tõid sind ära Egiptusemaalt.”
\par 29 Ja ta pani ühe Peetelisse ja teise Daani.
\par 30 Sellest aga tuli patt, et rahvas läks neist ühe juurde kuni Daanini.
\par 31 Ja Jerobeam ehitas ohvriküngastele kodasid ning seadis preestreid igasugu rahvast, kes ei olnud leviidid.
\par 32 Ja ta seadis kaheksanda kuu viieteistkümnendale päevale samasuguse püha, nagu oli Juudamaal, ja läks ise üles altarile; nõnda tegi ta Peetelis, et ohverdada vasikaile, mis ta oli teinud. Ja ta seadis Peetelis preestrid ohvriküngaste tarvis, mis ta oli teinud.
\par 33 Ta tõusis altarile, mille ta Peetelis oli teinud, kaheksanda kuu viieteistkümnendal päeval, selles kuus, mille ta ise oli määranud ja seadnud pühaks Iisraeli lastele; ta tõusis altarile suitsutama.

\chapter{13}

\par 1 Ja vaata, Issanda käsul tuli üks jumalamees Juudast Peetelisse, siis kui Jerobeam seisis altari juures, et suitsutada.
\par 2 Ta hüüdis Issanda käsul altari poole ja ütles: „Altar, Altar! Nõnda ütleb Issand: Vaata, Taaveti soole sünnib poeg, Joosija nimi, ja tema tapab su peal ohvriküngaste preestrid, kes suitsutavad su peal, ja su peal põletatakse inimluid!”
\par 3 Ja ta andis sel päeval tunnustähe, öeldes: „See on tunnustäheks, et Issand on kõnelnud: Vaata, altar lõhkeb ja tuhk, mis selle peal on, pillutatakse laiali.”
\par 4 Kui kuningas kuulis jumalamehe sõnu, mis too hüüdis Peeteli altari poole, siis sirutas Jerobeam altari juurest oma käe välja ja hüüdis: „Võtke ta kinni!” Aga käsi, mille ta sirutas tema poole, kuivetus ja ta ei suutnud seda tagasi tõmmata.
\par 5 Ja altar lõhkes ning tuhk pillutati altarilt, nagu oli tunnustäht, mille jumalamees oli andnud Issanda käsul.
\par 6 Siis kuningas hakkas kõnelema ja ütles jumalamehele: „Leevenda ometi Issanda, oma Jumala palet ja palveta minu eest, et ma saaksin käe tagasi!” Ja jumalamees leevendas Issanda palge ning kuningas sai oma käe tagasi ja see oli nagu ennegi.
\par 7 Ja kuningas ütles jumalamehele: „Tule koos minuga koju ja võta kehakinnitust, siis ma annan sulle ühe kingituse!”
\par 8 Aga jumalamees vastas kuningale: „Kui sa annaksid mulle ka poole oma kojast, ei tuleks ma sinuga kaasa; siin paigas ei söö ma leiba ega joo vett.
\par 9 Sest nõnda on mind kästud Issanda sõnaga, kes ütleb: Ära söö leiba ja ära joo vett, ja ära mine tagasi sedasama teed, mida sa tulid!”
\par 10 Ja ta läks teist teed ega läinud tagasi seda teed, mida mööda ta oli Peetelisse tulnud.
\par 11 Aga Peetelis elas keegi vana prohvet; tema pojad tulid ja jutustasid temale kõigist tegudest, mis jumalamees sel päeval Peetelis oli teinud, ja sõnadest, mis ta kuningale oli öelnud. Kui nad olid jutustanud oma isale,
\par 12 siis küsis nende isa neilt: „Missugust teed ta läks?” Ja ta pojad olid näinud teed, mida mööda läks jumalamees, kes oli tulnud Juudamaalt.
\par 13 Siis ta ütles oma poegadele: „Saduldage mulle eesel!” Nad saduldasid temale eesli ja ta istus selle selga,
\par 14 läks jumalamehele järele ning leidis tolle istumas tamme all; ta küsis temalt: „Kas sina oled see jumalamees, kes tuli Juudamaalt?„ Ja ta vastas: ”Olen.”
\par 15 Siis ta ütles temale: „Tule koos minuga koju ja võta leiba!”
\par 16 Aga ta vastas: „Ma ei või sinuga tagasi minna ega sinuga kaasa tulla; samuti ei söö ma leiba ega joo vett siin paigas koos sinuga.
\par 17 Sest mulle on öeldud Issanda sõna läbi: Sa ei tohi seal leiba süüa ega vett juua; sa ei tohi tulla tagasi sedasama teed, mida sa läksid!”
\par 18 Aga ta ütles temale: „Ka mina olen prohvet nagu sinagi, ja ingel kõneles mulle Issanda sõna läbi, öeldes: Vii ta enesega oma koju, et ta saaks leiba süüa ja vett juua!” Aga seda ta valetas temale.
\par 19 Siis läks jumalamees koos temaga tagasi ning sõi ta kojas leiba ja jõi vett.
\par 20 Aga kui nad lauas istusid, tuli Issanda sõna prohvetile, kes oli ta tagasi toonud,
\par 21 ja ta hüüdis jumalamehele, kes oli tulnud Juudast, öeldes: „Nõnda ütleb Issand: Sellepärast et sa oled vastu pannud Issanda suule ega ole pidanud käsku, mis Issand, su Jumal, sulle andis,
\par 22 vaid oled tagasi tulnud ja söönud leiba ja joonud vett paigas, mille kohta ma sulle ütlesin, et ära söö leiba ja ära joo vett, muidu ei saa su laip su vanemate hauda.”
\par 23 Pärast seda kui ta oli söönud leiba, ja Pärast seda kui ta oli joonud, saduldati temale eesel, prohveti oma, kes oli ta tagasi toonud.
\par 24 Ja kui ta läks, siis juhtus temale tee peal vastu lõvi ja see surmas tema; ta laip oli paisatud teele ja eesel seisis selle kõrval, ka lõvi seisis laiba kõrval.
\par 25 Ja vaata, mehed läksid mööda ja nägid teele paisatud laipa ja lõvi seisvat laiba kõrval. Nad tulid ja jutustasid sellest linnas, kus vana prohvet elas.
\par 26 Kui prohvet, kes oli ta teelt tagasi toonud, kuulis seda, siis ta ütles: „See on see jumalamees, kes pani vastu Issanda sõnale. Sellepärast andis Issand tema lõvi küüsi, ja see on tema maha murdnud ning surmanud, nagu oli Issanda sõna, mis ta oli temale öelnud.”
\par 27 Ja ta rääkis oma poegadega, öeldes: „Saduldage mulle eesel!” Ja nad saduldasid.
\par 28 Siis ta läks ning leidis tema laiba teele paisatuna, laiba kõrval aga seisid eesel ja lõvi; lõvi ei olnud laipa söönud ega eeslit maha murdnud.
\par 29 Prohvet tõstis siis jumalamehe laiba üles, pani selle eesli selga ja viis tagasi; ja vana Prohvet tuli linna leinakaebust tegema ja teda matma.
\par 30 Ta pani tema laiba oma hauda ja nad tegid temale leinakaebust: „Oh häda, mu vend!”
\par 31 Ja pärast seda kui ta oli tema matnud, rääkis ta oma poegadega, öeldes: „Kui mina suren, siis matke mind hauda, kuhu jumalamees on maetud; pange minu luud tema luude kõrvale!
\par 32 Sest see asi sünnib tõesti, mis ta Issanda käsul hüüdis Peetelis oleva altari ja kõigi Samaaria linnades olevate ohvrikünkakodade vastu.”
\par 33 Pärast seda lugu ei pöördunud Jerobeam ometi mitte oma kurjalt teelt, vaid seadis jälle ohvriküngaste preestreid igasugu rahvast; kes aga soovis, selle käe ta täitis ja see sai ohvrikünka preestriks.
\par 34 Ja see asi sai patuks Jerobeami soole, põhjuseks seda hukata ja maa pealt hävitada.

\chapter{14}

\par 1 Sel ajal haigestus Abija, Jerobeami poeg.
\par 2 Ja Jerobeam ütles oma naisele: „Võta nüüd kätte ja moonda ennast, nõnda et ei tunta, et sa oled Jerobeami naine, ja mine Siilosse! Vaata, seal on prohvet Ahija, kes rääkis minust, et ma saan selle rahva kuningaks.
\par 3 Võta kaasa kümme leiba, korpe ja kruus mett ning mine tema juurde; küllap ta kuulutab sulle, mis poissi ootab!”
\par 4 Ja Jerobeami naine tegi nõnda, võttis kätte ja läks Siilosse ning tuli Ahija kotta; aga Ahija ei näinud enam, sest vanaduse tõttu olid ta silmad tuhmunud.
\par 5 Kuid Issand oli Ahijale öelnud: „Vaata, Jerobeami naine tuleb sinult küsima oma poja kohta, sest too on haige. Ütle temale nõnda ja nõnda! Kui ta tuleb, siis on ta ennast tundmatuks teinud.”
\par 6 Kui siis Ahija kuulis jalgade kobinat, kui naine uksest sisse astus, ütles ta: „Tule sisse, Jerobeami naine! Miks sa ennast tundmatuks teed? Mind on läkitatud su juurde kõva sõnaga.
\par 7 Mine ütle Jerobeamile: Nõnda ütleb Issand, Iisraeli Jumal: Sellepärast et ma rahva hulgast olen sind ülendanud ja pannud vürstiks oma Iisraeli rahvale
\par 8 ning olen kiskunud kuningriigi Taaveti soo käest ja andnud sinule - sina aga ei ole olnud nagu mu sulane Taavet, kes pidas mu käske ja kes käis mu järel kõigest oma südamest, tehes ainult seda, mis õige oli minu silmis,
\par 9 vaid sa oled teinud rohkem kurja kui kõik, kes enne sind on olnud, ja sa oled läinud ning valmistanud enesele teisi jumalaid ja valatud kujusid, et mind vihastada, ja oled mind heitnud oma selja taha,
\par 10 sellepärast, vaata, saadan mina Jerobeami soole õnnetuse ja hävitan Jerobeamilt meesolendid, niihästi orjad kui vabad Iisraelis, ja ma pühin Jerobeami soo järelt, otsekui pühitakse sõnnikut, kuni see lõpeb.
\par 11 Kes Jerobeami omadest surevad linnas, neid söövad koerad, ja kes surevad väljal, neid söövad taeva linnud. Sest Issand on rääkinud.
\par 12 Sina aga tõuse, mine oma kotta! Niipea kui su jalg jõuab linna, sureb laps.
\par 13 Kogu Iisrael leinab teda ja nad matavad ta, sest Jerobeami omadest saab üksnes tema hauda, sellepärast et Jerobeami soos leidub ainult temas midagi head Issanda, Iisraeli Jumala ees.
\par 14 Aga Issand tõstab enesele Iisraeli kuninga, kes hävitab Jerobeami soo. See on see päev, ja mis nüüd veel?
\par 15 Issand lööb Iisraeli, nõnda et see kõigub nagu pilliroog vees, ja kitkub Iisraeli sellelt healt maalt, mille ta on andnud nende vanemaile, ja pillutab nad teisele poole Frati jõge, sellepärast et nad on teinud endile viljakustulbad Issanda vihastamiseks.
\par 16 Ja ta annab Iisraeli ära Jerobeami pattude pärast, mis ta on teinud ja millega ta on saatnud Iisraeli pattu tegema.”
\par 17 Siis Jerobeami naine tõusis üles ja läks ning tuli Tirsasse; aga kui ta jõudis koja läveni, siis suri poiss.
\par 18 Ta maeti, ja kogu Iisrael tegi tema pärast leinakaebust Issanda sõna järgi, nagu ta oli rääkinud oma sulase, prohvet Ahija läbi.
\par 19 Ja mis veel tuleks öelda Jerobeamist, kuidas ta sõdis ja kuidas ta valitses, vaata, sellest on kirjutatud Iisraeli kuningate Ajaraamatus.
\par 20 Ja aega, mis Jerobeam valitses, oli kakskümmend kaks aastat; siis ta läks magama oma vanemate juurde ja tema poeg Naadab sai tema asemel kuningaks.
\par 21 Aga Rehabeam, Saalomoni poeg, valitses Juudas: Rehabeam oli kuningaks saades nelikümmend üks aastat vana ja ta valitses seitseteist aastat Jeruusalemmas, linnas, mille Issand oli valinud kõigist Iisraeli suguharudest, et sinna panna oma nimi; ta ema oli ammonlanna Naama.
\par 22 Ja Juuda tegi kurja Issanda silmis; nad vihastasid teda oma pattudega, mida nad tegid rohkem, kui nende vanemad olid teinud.
\par 23 Nad tegid endile ka ohvrikünkaid, sambaid ja viljakustulpi igale kõrgemale künkale ja iga halja puu alla.
\par 24 Ja maal oli ka pühamu pordumehi. Nad tegid järele kõigi nende rahvaste jäledusi, keda Issand oli ära ajanud Iisraeli laste eest.
\par 25 Aga kuningas Rehabeami viiendal aastal tuli Egiptuse kuningas Siisak Jeruusalemma vastu
\par 26 ja võttis ära Issanda koja varandused ja kuningakoja varandused; ta võttis kõik ära. Ta võttis ära ka kõik kuldkilbid, mis Saalomon oli teinud.
\par 27 Siis tegi kuningas Rehabeam nende asemele vaskkilbid ja andis need ihukaitsepealikute hooleks, kes valvasid kuningakoja ust.
\par 28 Ja iga kord, kui kuningas läks Issanda kotta, kandsid ihukaitsjad neid ja viisid need siis jälle tagasi ihukaitse ruumi.
\par 29 Ja mis veel tuleks öelda Rehabeamist ja kõigest, mis ta tegi, eks sellest ole kirjutatud Juuda kuningate Ajaraamatus?
\par 30 Rehabeami ja Jerobeami vahel oli kogu aeg sõda.
\par 31 Siis Rehabeam läks magama oma vanemate juurde ja ta maeti oma vanemate juurde Taaveti linna; ta ema oli ammonlanna Naama. Ja tema poeg Abijam sai tema asemel kuningaks.

\chapter{15}

\par 1 Ja kuningas Jerobeami, Nebati poja kaheksateistkümnendal aastal sai Juuda kuningaks Abijam.
\par 2 Tema valitses Jeruusalemmas kolm aastat; ta ema nimi oli Maaka, Abisalomi tütar.
\par 3 Tema käis kõigis oma isa pattudes, mis too enne teda oli teinud; tema süda ei olnud siiras Issanda, oma Jumala vastu, nagu oli olnud ta isa Taaveti süda.
\par 4 Aga Taaveti pärast andis Issand, tema Jumal, temale Jeruusalemmas lambi, tõstes ta järglaseks tema poja ja jättes Jeruusalemma püsima,
\par 5 sest Taavet oli teinud, mis õige oli Issanda silmis, ega olnud kogu oma eluaja lahkunud kõigest sellest, mis ta temale oli käskinud, välja arvatud hett Uurija lugu.
\par 6 Rehabeami ja Jerobeami vaheline sõda kestis ka veel kogu ta eluaja.
\par 7 Ja mis veel tuleks öelda Abijamist ja kõigest, mis ta tegi, eks sellest ole kirjutatud Juuda kuningate Ajaraamatus? Abijami ja Jerobeami vahel oli sõda.
\par 8 Ja Abijam läks magama oma vanemate juurde ja ta maeti Taaveti linna. Ja tema poeg Aasa sai tema asemel kuningaks.
\par 9 Iisraeli kuninga Jerobeami kahekümnendal aastal sai Juuda kuningaks Aasa.
\par 10 Tema valitses Jeruusalemmas nelikümmend üks aastat; ta vanaema nimi oli Maaka, Abisalomi tütar.
\par 11 Aasa tegi, mis õige oli Issanda silmis, nõnda nagu ta isa Taavet,
\par 12 ja ta kaotas maalt pühamu pordumehed ning kõrvaldas kõik ebajumalad, mis ta vanemad olid teinud.
\par 13 Tema kõrvaldas ka oma vanaema Maaka kui valitsejanna, sellepärast et too oli teinud Aðera häbikuju; Aasa hävitas tema häbikuju ja põletas selle Kidroni orus.
\par 14 Aga ohvrikünkaid ei kaotatud; ometi oli Aasa süda siiras Issanda vastu kogu ta eluaja.
\par 15 Ja ta viis Issanda kotta, mis ta isa oli pühitsenud ja mis ta ise pühitses: hõbeda, kulla ja riistad.
\par 16 Aga Aasa ja Iisraeli kuninga Baesa vahel oli sõda kogu nende eluaja.
\par 17 Iisraeli kuningas Baesa tuli Juuda vastu ja kindlustas Raama, et Juuda kuningal Aasal ei oleks välja- ega sissepääsu.
\par 18 Siis võttis Aasa kõik hõbeda ja kulla, mis oli järele jäänud Issanda koja varanduste hulgast ja kuningakoja varanduste hulgast, ja andis selle oma sulaste kätte; ja kuningas Aasa läkitas need Ben-Hadadi juurde, kes oli Hesjoni poja Tabrimmoni poeg, Süüria kuningas, kes elas Damaskuses, öeldes:
\par 19 „Minu ja sinu vahel olgu leping, nagu oli minu isa ja sinu isa vahel! Vaata, ma läkitan sulle anniks hõbedat ja kulda. Mine, tühista oma leping Iisraeli kuninga Baesaga, et ta läheks ära minu kallalt!”
\par 20 Ja Ben-Hadad kuulas kuningas Aasat ning läkitas oma sõjaväepealikud Iisraeli linnade vastu ja vallutas Ijjoni, Daani ja Aabel-Beet-Maaka ja kogu Kinneroti koos kogu Naftalimaaga.
\par 21 Kui Baesa seda kuulis, siis ta loobus Raamat kindlustamast ja jäi Tirsasse.
\par 22 Aga kuningas Aasa kutsus kokku kogu Juuda, kedagi ei vabastatud, ja nad viisid Raamast ära kivid ja puud, millega Baesa seda oli kindlustanud; ja kuningas Aasa kindlustas nendega Benjamini Geba ja Mispa.
\par 23 Ja kõigest, mis veel tuleks öelda Aasast, kõigist ta vägitegudest ja kõigest, mis ta tegi, ja linnadest, mis ta ehitas, eks sellest ole kirjutatud Juuda kuningate Ajaraamatus? Aga oma vanas eas ta põdes jalgu.
\par 24 Siis Aasa läks magama oma vanemate juurde ja ta maeti oma vanemate juurde ta isa Taaveti linna. Ja tema poeg Joosafat sai tema asemel kuningaks.
\par 25 Ja Naadab, Jerobeami poeg, sai Iisraeli kuningaks Juuda kuninga Aasa teisel aastal; tema valitses Iisraeli üle kaks aastat.
\par 26 Tema tegi kurja Issanda silmis ja käis oma isa teedel ning tema patus, millega ta Iisraeli oli saatnud pattu tegema.
\par 27 Aga Baesa, Ahija poeg, Issaskari soost, pidas vandenõu tema vastu ja Baesa tappis ta Gibbetoni juures, mis oli vilistite päralt, sest Naadab ja kogu Iisrael piirasid Gibbetoni.
\par 28 Baesa surmas tema Juuda kuninga Aasa kolmandal aastal ja sai ise tema asemel kuningaks.
\par 29 Ja kui ta oli kuningaks saanud, siis ta lõi maha kogu Jerobeami soo, ta ei jätnud Jerobeamile ühtegi hingelist, kuni ta oli tema hävitanud Issanda sõna kohaselt, mis too oli rääkinud oma sulase, siilolase Ahija läbi
\par 30 Jerobeami pattude pärast, mis ta oli teinud ja millega ta Iisraeli oli saatnud pattu tegema, sügava haavamise pärast, millega ta oli haavanud Issandat, Iisraeli Jumalat.
\par 31 Ja mis veel tuleks öelda Naadabist ja kõigest, mis ta tegi, eks sellest ole kirjutatud Iisraeli kuningate Ajaraamatus?
\par 32 Aasa ja Iisraeli kuninga Baesa vahel oli sõda kogu nende eluaja.
\par 33 Juuda kuninga Aasa kolmandal aastal sai Ahija poeg Baesa Tirsas kuningaks kogu Iisraeli üle ja ta valitses kakskümmend neli aastat.
\par 34 Tema tegi kurja Issanda silmis ja käis Jerobeami teedel ning tema patus, millega ta Iisraeli oli saatnud pattu tegema.

\chapter{16}

\par 1 Ja Jehule, Hanani pojale, tuli Baesa kohta Issanda sõna, kes ütles:
\par 2 „Hoolimata sellest, et ma sind olen põrmust üles tõstnud ja pannud vürstiks oma Iisraeli rahvale, käid sina Jerobeami teed ja saadad mu Iisraeli rahva pattu tegema, nõnda et nad vihastavad mind oma pattudega,
\par 3 vaata, sellepärast ma siis pühin ära Baesa ja tema soo. Ja ma talitan sinu sooga nõnda nagu Nebati poja Jerobeami sooga.
\par 4 Kes Baesa omadest sureb linnas, selle söövad koerad, ja kes tal sureb väljal, selle söövad taeva linnud.”
\par 5 Ja mis veel tuleks öelda Baesast ja mis ta tegi ja tema vägitegudest, eks sellest ole kirjutatud Iisraeli kuningate Ajaraamatus?
\par 6 Ja Baesa läks magama oma vanemate juurde ja ta maeti Tirsasse; ja tema poeg Eela sai tema asemel kuningaks.
\par 7 Issanda sõna oli ju tulnud prohvet Jehu, Hanani poja läbi Baesa ja tema soo kohta kõige selle kurja pärast, mis ta oli teinud Issanda silmis, vihastades teda oma kätetööga ja saades nõnda Jerobeami soo sarnaseks, ja sellepärast et ta selle oli surmanud.
\par 8 Juuda kuninga Aasa kahekümne kuuendal aastal sai Baesa poeg Eela Tirsas kaheks aastaks Iisraeli kuningaks.
\par 9 Aga tema sulane Simri, kes oli poole sõjavankrite hulga pealik, pidas vandenõu tema vastu. Ja kord, kui Eela oli enese joobnuks joonud Tirsas Arsa kojas - Arsa oli Tirsas kojaülemaks -,
\par 10 tuli Simri ja lõi tema maha ning tappis tema Juuda kuninga Aasa kahekümne seitsmendal aastal; ja ta sai tema asemel kuningaks.
\par 11 Ja kui ta oli saanud kuningaks ja istus oma aujärjel, siis ta lõi maha kogu Baesa soo ega jätnud temale alles ainsatki meesolendit, ei sugulast ega sõpra.
\par 12 Nõnda hävitas Simri kogu Baesa soo Issanda sõna kohaselt, mis Issand oli rääkinud Baesa vastu prohvet Jehu läbi
\par 13 kõigi Baesa pattude pärast ja tema poja Eela pattude pärast, mis nad olid teinud ja millega nad Iisraeli olid saatnud pattu tegema, vihastades Issandat, Iisraeli Jumalat, oma ebajumalatega.
\par 14 Ja mis veel tuleks öelda Eelast ja kõigest, mis ta tegi, eks sellest ole kirjutatud Iisraeli kuningate Ajaraamatus?
\par 15 Juuda kuninga Aasa kahekümne seitsmendal aastal valitses Simri Tirsas seitse päeva; rahvas oli aga leeris Gibbetoni all, mis vilistitele kuulus.
\par 16 Kui leeris olev rahvas kuulis öeldavat: „Simri on pidanud vandenõu ja on ka kuninga maha löönud”, siis tõstis kogu Iisrael selsamal päeval leeris kuningaks Omri, Iisraeli väepealiku.
\par 17 Siis Omri läks Gibbetonist ära ja koos temaga kogu Iisrael, ja nad piirasid Tirsat.
\par 18 Ja kui Simri nägi, et linn oli vallutatud, siis ta läks kuningakoja torni, põletas tulega enese pealt kuningakoja ja suri
\par 19 oma pattude pärast, mis ta oli teinud, tehes kurja Issanda silmis, käies Jerobeami teel ja tema patus, mis too oli teinud, saates Iisraeli pattu tegema.
\par 20 Ja mis veel tuleks öelda Simrist ja tema vandenõust, mida ta pidas, eks sellest ole kirjutatud Iisraeli kuningate Ajaraamatus?
\par 21 Siis Iisraeli rahvas jagunes pooleks; pool rahvast käis Giinati poja Tibni järel, et tõsta teda kuningaks, ja pool Omri järel.
\par 22 Aga see rahvas, kes käis Omri järel, sai võimust rahva üle, kes käis Giinati poja Tibni järel. Tibni suri ja Omri sai kuningaks.
\par 23 Juuda kuninga Aasa kolmekümne esimesel aastal sai Omri kaheteistkümneks aastaks Iisraeli kuningaks; Tirsas valitses ta kuus aastat.
\par 24 Ta ostis Semerilt kahe hõbetalendi eest Samaaria mäe; ta ehitas mäe peale linna ja nimetas linna, mille ta oli ehitanud, Samaariaks, mäe omaniku Semeri nime järgi.
\par 25 Aga Omri tegi kurja Issanda silmis; ta tegi rohkem kurja kui kõik need, kes enne teda olid olnud.
\par 26 Ta käis kõigil Nebati poja Jerobeami teedel ja tema pattudes, millega too oli saatnud Iisraeli pattu tegema, vihastama Issandat, Iisraeli Jumalat, oma ebajumalatega.
\par 27 Ja mis veel tuleks öelda Omrist, mis ta tegi, ja tema vägitegudest, mis ta korda saatis, eks sellest ole kirjutatud Iisraeli kuningate Ajaraamatus?
\par 28 Ja Omri läks magama oma vanemate juurde ja ta maeti Samaariasse. Ja tema poeg Ahab sai tema asemel kuningaks.
\par 29 Ahab, Omri poeg, sai Iisraeli kuningaks Juuda kuninga Aasa kolmekümne kaheksandal aastal; ja Ahab, Omri poeg, valitses Samaarias Iisraeli üle kakskümmend kaks aastat.
\par 30 Aga Ahab, Omri poeg, tegi kurja Issanda silmis, rohkem kui kõik need, kes enne teda olid olnud.
\par 31 Kas sellest veel vähe oli, et ta käis Nebati poja Jerobeami pattudes? Ta võttis naiseks Iisebeli, siidonlaste kuninga Etbaali tütre, ja läks ning teenis Baali ja kummardas teda.
\par 32 Ja ta püstitas Baalile altari Baali kotta, mille ta Samaariasse oli ehitanud.
\par 33 Ja Ahab valmistas Aðera kuju. Ahab tegi Issanda, Iisraeli Jumala vihastamiseks veel rohkem kui kõik Iisraeli kuningad, kes enne teda olid olnud.
\par 34 Tema päevil ehitas peetellane Hiiel üles Jeeriko; ta rajas selle oma esmasündinu Abirami hinnaga ja ta pani sellele väravad ette oma noorima poja Seguubi hinnaga Issanda sõna kohaselt, mis ta oli öelnud Joosua, Nuuni poja läbi.

\chapter{17}

\par 1 Tisbelane Eelija, Gileadi Tisbest, ütles Ahabile: „Nii tõesti kui elab Issand, Iisraeli Jumal, kelle ees ma seisan: neil aastail ei ole kastet ega vihma muidu kui minu sõna peale!”
\par 2 Ja temale tuli Issanda sõna, kes ütles:
\par 3 „Mine siit ära ja pöördu ida poole ning ole varjul Kriti jõe ääres, mis on ida pool Jordanit!
\par 4 Jõest saad sa juua, ja ma olen käskinud kaarnaid seal sind toita.”
\par 5 Ja ta läks ning tegi Issanda sõna järgi: ta läks ja elas Kriti jõe ääres, mis on ida pool Jordanit.
\par 6 Ja hommikuti tõid kaarnad temale leiba ja liha, samuti õhtuti leiba ja liha, ja ta jõi jõest.
\par 7 Aga mõne aja pärast juhtus, et jõgi kuivas, sest maal ei olnud vihma.
\par 8 Ja temale tuli Issanda sõna, kes ütles:
\par 9 „Võta kätte, mine Sareptasse, mis kuulub Siidonile, ja ela seal! Vaata, ma olen käskinud ühte lesknaist seal sind toita.”
\par 10 Ja ta võttis kätte ning läks Sareptasse. Ja kui ta jõudis linna värava juurde, vaata, siis oli seal üks lesknaine puid korjamas. Ja ta hüüdis teda ning ütles: „Too mulle ometi pisut vett mõne nõuga, et saaksin juua!”
\par 11 Ja kui naine läks vett tooma, siis ta hüüdis temale järele ning ütles: „Too mulle ka paluke leiba!”
\par 12 Aga naine vastas: „Nii tõesti kui Issand, su Jumal, elab, ei ole mul kakukestki, vaid ainult peotäis jahu vakas ja pisut õli kruusis. Vaata, ma olen korjanud paar puutükki, ja ma lähen ning valmistan midagi enesele ja oma pojale, et saaksime veel süüa, enne kui sureme.”
\par 13 Siis ütles Eelija temale: „Ära karda! Mine tee, nagu sa oled öelnud! Aga esmalt valmista sellest mulle pisike kook ja too see minule! Pärast valmista enesele ja oma pojale!
\par 14 Sest nõnda ütleb Issand, Iisraeli Jumal: Jahu ei lõpe vakast ja õli ei vähene kruusist kuni päevani, mil Issand annab maale vihma.”
\par 15 Ja naine läks ning tegi Eelija sõna järgi. Ja temal, samuti Eelijal ja naise perel oli süüa kauaks ajaks:
\par 16 jahu ei lõppenud vakast ja õli ei vähenenud kruusist Issanda sõna peale, nagu ta Eelija läbi oli öelnud.
\par 17 Ja pärast seda lugu sündis, et naise, kojaemanda poeg jäi haigeks; ja tema haigus oli väga raske, nõnda et temasse ei jäänud enam hingeõhkugi.
\par 18 Siis ütles naine Eelijale: „Mis on mul sinuga tegemist, jumalamees? Sa tulid minu juurde mu süütegu meelde tuletama ja mu poega surma saatma.”
\par 19 Aga ta vastas temale: „Anna mulle oma poeg!” Ja ta võttis selle tema sülest ja viis ülakambrisse, kus ta elas, ja pani oma voodisse.
\par 20 Siis ta hüüdis Issanda poole ning ütles: „Issand, mu Jumal, kas sa oled tõesti lesele, kelle juures ma võõraks olen, teinud seda paha, et oled surmanud tema poja?”
\par 21 Seejärel ta sirutas ennast poisi üle kolm korda ja hüüdis Issanda poole ning ütles: „Issand, mu Jumal, lase ometi selle poisi hing tulla temasse tagasi!”
\par 22 Ja Issand kuulis Eelija häält ning poisi hing tuli temasse tagasi ja ta virgus ellu.
\par 23 Ja Eelija võttis poisi ning tõi ülakambrist alla kotta ja andis tema ta emale; ja Eelija ütles: „Vaata, su poeg elab!”
\par 24 Siis naine ütles Eelijale: „Nüüd ma tean, et sa oled jumalamees ja et Issanda sõna sinu suus on tõde.”

\chapter{18}

\par 1 Ja hulga aja pärast, kolmandal aastal, tuli Eelijale Issanda sõna, kes ütles: „Mine näita ennast Ahabile, siis ma annan maale vihma!”
\par 2 Ja Eelija läks ennast Ahabile näitama. Samaarias oli aga suur nälg
\par 3 ja Ahab kutsus Obadja, kes oli kojaülem. Obadja kartis väga Issandat,
\par 4 ja kui Iisebel hävitas Issanda prohveteid, siis oli Obadja võtnud sada prohvetit ja peitnud need viiekümne kaupa koobastesse ning toitnud neid leiva ja veega.
\par 5 Ja Ahab ütles Obadjale: „Käi maa läbi, kõik veeallikad ja kõik jõed! Vahest leiame rohtu, et saame elus hoida hobused ja muulad ega peaks hävitama karju?”
\par 6 Ja nad jaotasid isekeskis läbikäidava maa: Ahab läks omaette ühte teed ja Obadja läks omaette teist teed.
\par 7 Kui nüüd Obadja oli teel, vaata, siis tuli Eelija temale vastu. Teda ära tundes heitis ta silmili maha ja küsis: „Kas see oled sina, mu isand Eelija?”
\par 8 Ja ta vastas temale: „Olen. Mine ütle oma isandale: Vaata, Eelija on siin!”
\par 9 Aga tema ütles: „Mis pattu ma olen teinud, et sa tahad oma sulase anda Ahabi kätte, selleks et ta mu surmaks?
\par 10 Nii tõesti kui Issand, sinu Jumal, elab, ei ole rahvast ega kuningriiki, kuhu mu isand ei ole läkitanud sind otsima; ja kui nad ütlesid: Teda ei ole siin, siis ta laskis seda kuningriiki või rahvast vanduda, et sind ei ole leitud.
\par 11 Ja nüüd sa ütled: Mine ütle oma isandale: Vaata, Eelija on siin!
\par 12 Kui ma lähen sinu juurest ära ja Issanda Vaim viib su sinna, kuhu ma ei tea, mina aga tulen Ahabile teatama ja tema ei leia sind, siis ta tapab mu. Ometi kardab su sulane Issandat oma noorpõlvest peale.
\par 13 Kas mu isandale pole jutustatud, mis ma tegin, kui Iisebel tappis Issanda prohveteid? Ma peitsin siis Issanda prohvetitest sada meest viiekümne kaupa koobastesse ning toitsin neid leiva ja veega.
\par 14 Ja nüüd ütled sina: Mine ütle oma isandale: Vaata, Eelija on siin! Ta ju tapab mu!”
\par 15 Aga Eelija ütles: „Nii tõesti kui elab vägede Issand, kelle ees ma seisan, täna näitan ma ennast temale!”
\par 16 Siis läks Obadja Ahabile vastu ja jutustas temale sellest. Ahab läks siis Eelijale vastu.
\par 17 Ja kui Ahab nägi Eelijat, siis küsis Ahab temalt: „Kas sina oled see, kes saadab Iisraeli õnnetusse?”
\par 18 Ja ta vastas: „Mina ei saada Iisraeli õnnetusse, küll aga sina ja su isa sugu, sest te jätate maha Issanda käsud ja sina käid baalide järel!
\par 19 Aga nüüd läkita käsk, et minu juurde Karmeli mäele kogutaks kogu Iisrael ja need nelisada viiskümmend Baali prohvetit ja nelisada Aðera prohvetit, kes söövad Iisebeli lauas!”
\par 20 Ja Ahab läkitas käsu kõigile Iisraeli lastele ning kogus prohvetid Karmeli mäele.
\par 21 Nüüd astus Eelija kogu rahva ette ja ütles: „Kui kaua te lonkate kahe karguga? Kui Issand on Jumal, siis käige tema järel; aga kui Baal on see, siis käige tema järel!” Aga rahvas ei vastanud temale sõnagi.
\par 22 Siis Eelija ütles rahvale: „Mina olen ainus järelejäänud Issanda prohvet, aga Baali prohveteid on nelisada viiskümmend meest.
\par 23 Antagu meile kaks härjavärssi: valigu nemad enestele üks härjavärss, raiugu tükkideks ja pangu puude peale, aga nad ärgu süüdaku tuld; mina valmistan teise härjavärsi ja panen puude peale, ja minagi ei süüta tuld.
\par 24 Siis hüüdke teie oma jumala nime ja mina hüüan Issanda nime! See jumal, kes siis vastab tulega, on jumal!„ Ja kogu rahvas kostis ning ütles: „See kõne on hea!”
\par 25 Ja Eelija ütles Baali prohvetitele: „Valige enestele üks härjavärss ja valmistage see esimesena, sest teid on rohkem! Hüüdke siis oma jumala nime, aga ärge süüdake tuld!”
\par 26 Ja nad võtsid härjavärsi, kelle ta neile andis, ning valmistasid selle. Siis hüüdsid nad Baali nime hommikust lõunani, öeldes: „Baal, vasta meile!” Aga ei häält ega vastust. Ja nad karglesid ümber altari, mille nad olid teinud.
\par 27 Aga lõunaajal Eelija pilkas neid ning ütles: „Hüüdke valjema häälega, sest ta on ju jumal! Vahest on ta mõtteis, on läinud kõrvale või viibib teekonnal? Vahest ta magab? Aga küllap ta ärkab!”
\par 28 Siis nad hüüdsid valjema häälega ja täkkisid end mõõkadega ja piikidega oma viisi järgi, kuni neil veri hakkas voolama.
\par 29 Ja kui lõunaaeg oli möödunud, siis nad märatsesid kuni roaohvri toomise ajani; aga ei häält, ei vastust ega tähelepanu.
\par 30 Siis ütles Eelija kogu rahvale: „Astuge minu juurde!” Ja kogu rahvas astus tema juurde. Siis parandas ta Issanda altari, mis oli maha kistud.
\par 31 Ja Eelija võttis kaksteist kivi, vastavalt Jaakobi laste suguharude arvule, kellele oli tulnud Issanda sõna ja kellele oli öeldud: „Su nimi olgu Iisrael!”
\par 32 Ja ta ehitas kividest altari Issanda nimele ja tegi altari ümber kraavi, mis mahutas külimitu seemet.
\par 33 Siis ta ladus puud ja raius härjavärsi tükkideks ning pani puude peale
\par 34 ja ütles: „Täitke neli kruusi veega ja valage põletusohvri ja puude peale!„ Siis ta ütles: „Tehke seda teist korda!” Ja nad tegid teist korda. Siis ta ütles: ”Tehke kolmandat korda!” Ja nad tegid kolmandat korda.
\par 35 Vesi aga voolas ümber altari ja kraavgi täitus veega.
\par 36 Ja kui siis roaohvrit pidi ohverdatama, astus prohvet Eelija ette ja ütles: „Issand! Aabrahami, Iisaki ja Iisraeli Jumal! Saagu täna teatavaks, et sina oled Jumal Iisraelis ja mina olen sinu sulane ja et mina olen kõike seda teinud sinu sõna peale!
\par 37 Vasta mulle, Issand! Vasta mulle, et see rahvas saaks teada, et sina, Issand, oled Jumal ja et sina pöörad tagasi nende südamed!”
\par 38 Siis Issanda tuli langes alla ja sõi ära põletusohvri, puud, kivid ja põrmu ning lakkus ära vee, mis oli kraavis.
\par 39 Kui kogu rahvas seda nägi, siis heitsid nad silmili maha ja ütlesid: „Issand on Jumal! Issand on Jumal!”
\par 40 Aga Eelija ütles neile: „Võtke kinni Baali prohvetid, et ükski neist ei pääseks!” Ja rahvas võttis nad kinni. Ja Eelija viis nad alla Kiisoni jõe äärde ning tappis nad seal.
\par 41 Ja Eelija ütles Ahabile: „Mine üles, söö ja joo, sest vihma kohin kostab!”
\par 42 Ja Ahab läks sööma ja jooma. Aga Eelija läks üles Karmeli tippu, kummardas maha ja pani näo põlvede vahele.
\par 43 Siis ta ütles oma teenrile: „Mine nüüd ja vaata mere poole!„ Ja see läks ja vaatas, kuid ütles: „Ei ole midagi.” Tema aga ütles: ”Mine tagasi!” Nõnda seitse korda.
\par 44 Aga seitsmendal korral ütles teener: „Vaata, pisike pilv nagu mehe kämmal tõuseb merest.„ Siis ütles Eelija: ”Mine ütle Ahabile: Rakenda hobused ette ja mine, et sadu ei peaks sind kinni!”
\par 45 Ja vahepeal tumenes taevas pilvedest ja tuulest ning tuli suur sadu. Ahab aga sõitis ja läks Jisreeli.
\par 46 Ja Issanda käsi tuli Eelija peale: ta pani enesele vöö vööle ja jooksis Ahabile ette Jisreeli teelahkmeni.

\chapter{19}

\par 1 Ja Ahab jutustas Iisebelile kõik, mis Eelija oli teinud ja kuidas ta oli mõõgaga tapnud kõik prohvetid.
\par 2 Siis läkitas Iisebel käskjala Eelijale ütlema: „Jumalad tehku minuga ükskõik mida, kui ma homme sel ajal ei tee sinu hingega, nagu sündis kõigi nende hingedega!”
\par 3 Kui Eelija seda nägi, siis ta võttis kätte ja läks ära oma hinge pärast. Ta jõudis Beer-Sebasse, mis kuulub Juudale, ja jättis oma poisi sinna.
\par 4 Ta ise aga käis kõrbes ühe päevateekonna ja tuli ning istus ühe leetpõõsa alla; siis soovis ta oma hinges, et ta võiks surra, ja ütles: „Küllalt! Nüüd, Issand, võta mu hing, sest ma pole parem kui mu vanemad!”
\par 5 Ja ta heitis maha ning jäi magama leetpõõsa alla. Aga vaata, üks ingel puudutas teda ja ütles temale: „Tõuse üles, söö!”
\par 6 Ja kui ta vaatas, ennäe, siis oli ta peatsis üks kuumadel kividel küpsetatud kook ja kruus vett. Ta sõi ja jõi ning heitis jälle magama.
\par 7 Aga Issanda ingel tuli veel teist korda ja puudutas teda ning ütles: „Tõuse üles, söö, sest sul on veel pikk tee ees!”
\par 8 Siis ta tõusis, sõi ja jõi ning käis selle söömise rammuga nelikümmend päeva ja nelikümmend ööd kuni Jumala mäeni, Hoorebini.
\par 9 Seal läks ta ühte koopasse ja ööbis seal. Ja vaata, temale tuli Issanda sõna, kes küsis temalt: „Mis sa siin teed, Eelija?”
\par 10 Ja ta vastas: „Ma olen tõsiselt ägestunud Issanda, vägede Jumala pärast, sest Iisraeli lapsed on hüljanud sinu lepingu, nad on kiskunud maha sinu altarid ja on mõõgaga tapnud sinu prohvetid. Mina üksi olen üle jäänud ja nad püüavad võtta mu hinge.”
\par 11 Siis ta ütles: „Mine välja ja seisa mäe peal Issanda ees!” Ja vaata, Issand läks mööda, ja tugev ning võimas tuul, mis lõhestas mägesid ja purustas kaljusid, käis Issanda ees. Aga Issandat ei olnud tuules. Ja tuule järel tuli maavärisemine, aga Issandat ei olnud maavärisemises.
\par 12 Ja maavärisemise järel tuli tuli, aga Issandat ei olnud tules. Ja tule järel tuli vaikne, tasane sahin.
\par 13 Kui Eelija seda kuulis, siis ta kattis oma näo kuuega ja läks välja ning seisis koopasuus. Ja vaata, temale kostis üks hääl, kes küsis: „Mis sa siin teed, Eelija?”
\par 14 Ta vastas: „Ma olen tõsiselt ägestunud Issanda, vägede Jumala pärast, sest Iisraeli lapsed on hüljanud sinu lepingu, nad on kiskunud maha sinu altarid ja on mõõgaga tapnud sinu prohvetid. Mina üksi olen üle jäänud ja nad püüavad võtta mu hinge.”
\par 15 Ja Issand ütles temale: „Mine tagasi oma tuldud teed Damaskuse kõrbe! Mine ja võia Hasael Süüria kuningaks!
\par 16 Jehu, Nimsi poeg, aga võia Iisraeli kuningaks ja Eliisa, Saafati poeg Aabel-Meholast, võia enda asemel prohvetiks!
\par 17 Ja olgu nõnda: kes pääseb Hasaeli mõõga eest, selle surmab Jehu, ja kes pääseb Jehu mõõga eest, selle surmab Eliisa!
\par 18 Aga ma jätan Iisraelis üle seitse tuhat: kõik põlved, mis ei ole nõtkunud Baali ees, ja kõik suud, mis ei ole teda suudelnud.”
\par 19 Siis ta läks sealt ära ja leidis Eliisa, Saafati poja, kes oli kündmas; ta ees käis kaksteist härjapaari ja ta ise oli kaheteistkümnendaga. Kui Eelija temast mööda läks, siis ta heitis oma kuue tema peale.
\par 20 Ja Eliisa jättis veised ja jooksis Eelijale järele ning ütles: „Lase mind ometi oma isale ja emale suud anda, siis ma käin su järel!„ Ta vastas temale: ”Mine, aga tule tagasi, sest ära unusta, mis ma sulle olen teinud!”
\par 21 Ja ta läks tema juurest tagasi, võttis veistepaari ning tappis need; veiste iketega keetis ta liha ning andis rahvale, ja nad sõid. Siis ta tõusis ja käis Eelija järel ning teenis teda.

\chapter{20}

\par 1 Ben-Hadad, Süüria kuningas, kogus kokku kogu oma sõjaväe. Temaga oli kaasas kolmkümmend kaks kuningat ning hobuseid ja sõjavankreid. Ta läks ja piiras Samaariat ning sõdis selle vastu.
\par 2 Ta läkitas käskjalad linna Iisraeli kuninga Ahabi juurde
\par 3 ja käskis temale öelda: „Nõnda ütleb Ben-Hadad: Sinu hõbe ja kuld on minu, ja sinu ilusamad naised ja lapsed - ka need on minu!”
\par 4 Ja Iisraeli kuningas kostis ning ütles: „Nõnda nagu sa ütled, mu isand kuningas. Mina ise ja kõik, mis mul on, on sinu!”
\par 5 Aga käskjalad tulid taas ja ütlesid: „Nõnda kõneleb Ben-Hadad ja ütleb: Mina olen ju läkitanud sinule ütlema, et sa pead mulle andma oma hõbeda ja kulla, naised ja lapsed.
\par 6 Tõesti, homme sel ajal läkitan ma oma sulased sinu juurde ja nad otsivad läbi su koja ja su sulaste kojad; ja kõik, mis sinu silmis on väärtuslik, võtavad nad enestega kaasa ja toovad ära.”
\par 7 Siis Iisraeli kuningas kutsus kõik maa vanemad ja ütles: „Teadke nüüd ja nähke, et ta püüab kurja teha! Sest ta läkitab minu juurde mu naiste ja laste pärast, ja mu hõbeda ja kulla pärast, ning ma ei keela seda temale.”
\par 8 Kõik vanemad ja kogu rahvas ütlesid temale: „Ära kuula ja ära ole nõus!”
\par 9 Siis ta ütles Ben-Hadadi käskjalgadele: „Öelge mu isandale kuningale: Kõik, mille pärast sa esmalt läkitasid oma sulase juurde, ma teen; aga seda asja ei või ma teha.” Ja käskjalad läksid ning viisid sõna tagasi.
\par 10 Siis Ben-Hadad läkitas ta juurde ja ütles: „Jumalad tehku minuga ükskõik mida, kui Samaaria põrmust peaks jätkuma peotäieks kogu rahvale, kes mu kannul käib!”
\par 11 Aga Iisraeli kuningas vastas ja ütles: „Öelge: Vöö köitja ärgu kiidelgu nõnda nagu lahtipäästja!”
\par 12 Kui Ben-Hadad kuulis seda sõna, olles lehtmajades kuningate seltsis joomas, ütles ta oma sulastele: „Valmistuge!” Ja nad valmistusid rünnakuks linna vastu.
\par 13 Ja vaata, üks prohvet astus Iisraeli kuninga Ahabi juurde ja ütles: „Nõnda ütleb Issand: Kas sa näed kogu seda suurt rahvahulka? vaata, mina annan selle täna sinu kätte, et sa teaksid, et mina olen Issand!”
\par 14 Aga Ahab küsis: „Kelle abiga?„ Ja tema vastas: „Nõnda ütleb Issand: Asevalitsejate noorte meeste abiga.„ Ja Ahab küsis: ”Kes peab taplust alustama?” Ja tema vastas: ”Sina.”
\par 15 Siis ta luges ära asevalitsejate noored mehed, ja neid oli kakssada kolmkümmend kaks; ja nende järel luges ta ära kogu rahva, kõik Iisraeli lapsed: neid oli seitse tuhat.
\par 16 Ja nad läksid välja lõunaajal, kui Ben-Hadad oli lehtmajades enese joobnuks joonud, tema ja need kolmkümmend kaks kuningat, kes olid temale appi tulnud.
\par 17 Asevalitsejate noored mehed läksid välja esimestena. Ben-Hadad oli läkitanud maad kuulama ja temale teatati ning öeldi: „Samaariast tuleb mehi!”
\par 18 Aga tema ütles: „Kui nad tulevad rahuga, siis võtke nad elusalt kinni; tulevad nad aga sõjaga, siis võtke nad samuti elusalt kinni!”
\par 19 Aga need, kes tulid linnast välja, asevalitsejate noored mehed, ja sõjavägi, kes käis nende järel,
\par 20 lõid igamees maha oma vastase ja süürlased põgenesid ning Iisrael ajas neid taga. Aga Ben-Hadad, Süüria kuningas, pääses hobuse seljas koos ratsanikega.
\par 21 Siis Iisraeli kuningas läks välja ja lõi maha hobused ja sõjavankrid ning valmistas süürlastele suure kaotuse.
\par 22 Ja prohvet astus Iisraeli kuninga juurde ning ütles temale: „Mine, kinnita ennast, pane tähele ja vaata, mis sul tuleks teha, sest aasta pärast tuleb Süüria kuningas su vastu!”
\par 23 Aga Süüria kuningale ütlesid tema sulased: „Nende jumalad on mägede jumalad, sellepärast olid nad meist vägevamad; aga tapelgem nendega lagendikul, siis me võidame nad kindlasti!
\par 24 Ja tee nõnda: kõrvalda kuningad, igamees oma kohalt, ja pane nende asemele asevalitsejad!
\par 25 Sina aga vali enesele sõjavägi, niisama suur kui see, mille sa kaotasid, ja hobuseid niisama palju, kui oli hobuseid, ja vankreid niisama palju, kui oli vankreid, siis me tapleme nendega lagendikul ja võidame nad kindlasti!” Ja ta kuulas nende häält ning tegi nõnda.
\par 26 Ja aasta pärast luges Ben-Hadad süürlased üle ning läks Afekisse sõdima Iisraeli vastu.
\par 27 Ka Iisraeli lapsed loeti üle ja varustati, ja nad läksid neile vastu; Iisraeli lapsed lõid leeri üles nende ette, nagu kaks väikest kitsekarja, kuna süürlased täitsid maa.
\par 28 Taas astus ette jumalamees ja rääkis Iisraeli kuningaga, öeldes: „Nõnda ütleb Issand: Kuna süürlased on öelnud, et Issand on mägede Jumal ega ole orgude Jumal, siis ma annan selle suure rahvahulga sinu kätte, et te mõistaksite, et mina olen Issand!”
\par 29 Ja nad olid vastakuti leeris seitse päeva. Aga seitsmendal päeval läks tapluseks ja Iisraeli lapsed lõid süürlastest maha sada tuhat jalameest ühelainsal päeval.
\par 30 Ülejäänud põgenesid Afeki linna, aga müür langes kahekümne seitsme tuhande Ülejäänud mehe peale. Ja jõudnud linna, põgenes ka Ben-Hadad kambrist kambrisse.
\par 31 Siis ütlesid ta sulased temale: „Vaata ometi, me oleme kuulnud, et Iisraeli soo kuningad on armulikud kuningad: pangem siis endile kotiriided ümber niuete ja nöörid ümber pea ja mingem välja Iisraeli kuninga juurde; vahest ta jätab su hinge elama?”
\par 32 Ja nad vöötasid oma niuded kotiriidega ja panid nöörid ümber pea ja tulid Iisraeli kuninga juurde ning ütlesid: „Sinu sulane Ben-Hadad ütleb: Jäta siiski mu hing elama!„ Ja ta vastas: ”Kas ta elab veel? Ta on mu vend!”
\par 33 Mehed nägid selles head märki ja tõttasid pidama seda tema poolt kindlaks ning ütlesid: „Ben-Hadad on su vend!„ Ja tema ütles: ”Minge ja tooge ta!” Siis tuli Ben-Hadad välja tema juurde ja ta laskis tal astuda vankrisse.
\par 34 Ja Ben-Hadad ütles temale: „Linnad, mis mu isa võttis ära sinu isalt, annan ma tagasi. Korralda enesele kaubatänavad Damaskuses, nagu mu isa korraldas Samaarias!„ Ja Ahab vastas: ”Sel tingimusel ma lasen sinu vabaks.” Ja ta tegi temaga lepingu ning laskis tal minna.
\par 35 Siis ütles keegi prohveti jüngritest Issanda käsul oma sõbrale: „Löö mind ometi!” Aga mees keeldus teda löömast.
\par 36 Siis ta ütles temale: „Sellepärast et sa ei ole võtnud kuulda Issanda häält, vaata, siis murrab lõvi sind maha, kui sa minu juurest ära lähed.” Ja kui too tema juurest ära läks, siis juhtus lõvi temale vastu ja murdis ta maha.
\par 37 Siis ta leidis teise mehe ja ütles: „Löö mind ometi!” Ja mees lõi kõvasti ning haavas teda.
\par 38 Siis läks prohvet ja seisis kuninga teel, olles ennast moondanud silmile pandud sidemega.
\par 39 Ja kui kuningas mööda läks, siis ta kisendas kuningale ning ütles: „Sinu sulane läks sõtta, ja vaata, üks mees tuli sealt ja tõi teise mehe mu juurde ning ütles: Valva seda meest! Kui ta kaob, siis jääb sinu hing tema hinge asemele või sa pead vaagima talendi hõbedat!
\par 40 Aga su sulasel oli tegemist siin ja seal, ja siis ei olnud enam seda meest.„ Iisraeli kuningas ütles temale: „Samasugune on kohtuotsus sinu kohta. Sa ise oled selle määranud.”
\par 41 Seejärel kõrvaldas ta kähku sideme silmilt ja Iisraeli kuningas nägi, et ta oli prohvetite hulgast.
\par 42 Ja ta ütles kuningale: „Nõnda ütleb Issand: Et sa lasksid käest ära minu poolt vande alla pandud mehe, siis peab sinu hing olema tema asemel ja sinu rahvas tema rahva asemel.”
\par 43 Siis Iisraeli kuningas läks koju, tusane ja raevutsev, ja tuli Samaariasse.

\chapter{21}

\par 1 Pärast neid sündmusi juhtus järgmine lugu: jisreellasel Naabotil oli viinamägi, mis asus Jisreelis Samaaria kuninga Ahabi palee kõrval.
\par 2 Ja Ahab rääkis Naabotiga, öeldes: „Anna oma viinamägi mulle rohuaiaks, sest see on mu kojale nii lähedal; mina annan sulle selle asemele parema viinamäe, või kui sa tahad, siis ma annan sulle selle hinna rahas.”
\par 3 Aga Naabot vastas Ahabile: „Issand hoidku mind selle eest, et annaksin sulle oma vanemate pärisosa!”
\par 4 Ja Ahab tuli oma kotta, tusane ja raevutsev vastuse pärast, mille jisreellane Naabot oli temale andnud, öeldes: „Ma ei anna sulle oma vanemate pärisosa!” Ja ta heitis voodisse, pööras oma näo ära ega söönud leiba.
\par 5 Siis tuli ta naine Iisebel tema juurde ja küsis temalt: „Mispärast on su vaim nii vaevatud, et sa leibagi ei söö?”
\par 6 Ahab vastas temale: „Sellepärast et ma rääkisin jisreellase Naabotiga ja ütlesin temale: Anna mulle oma viinamägi raha eest, või kui sulle meeldib, siis ma annan sulle selle asemele teise viinamäe! Aga tema vastas: Ma ei anna sulle oma viinamäge!”
\par 7 Siis ta naine Iisebel ütles temale: „Vist ikka sina valitsed praegu Iisraeli üle? Tõuse üles, võta leiba ja su süda olgu rõõmus! Mina ise annan sulle jisreellase Naaboti viinamäe.”
\par 8 Ja ta kirjutas Ahabi nimel kirjad, sulges need tema pitseriga ja läkitas kirjad vanemaile ja suurnikele, kes elasid Naabotiga ühes linnas.
\par 9 Ja kirjades oli ta kirjutanud ning öelnud: „Kuulutage välja paast ja pange Naabot istuma rahva hulgas esikohale!
\par 10 Pange siis kaks kõlvatut meest istuma temaga vastamisi ja need tunnistagu ning öelgu: Sina oled teotanud Jumalat ja kuningat! Siis viige ta välja ja visake kividega surnuks!”
\par 11 Ja tema linna mehed, vanemad ja suurnikud, kes elasid ta linnas, tegid, nagu Iisebel neile oli sõna saatnud, nagu oli kirjutatud kirjades, mis ta neile oli läkitanud.
\par 12 Nad kuulutasid välja paastu ja panid Naaboti istuma rahva hulgas esikohale.
\par 13 Ja kaks kõlvatut meest tulid ning istusid temaga vastamisi; ja need kõlvatud mehed tunnistasid rahva ees Naaboti vastu, öeldes: „Naabot on teotanud Jumalat ja kuningat.” Siis viidi ta linnast välja ja visati kividega surnuks.
\par 14 Ja nad läkitasid Iisebelile ütlema: „Naabot on kividega surnuks visatud.”
\par 15 Kui Iisebel kuulis, et Naabot oli kividega surnuks visatud, siis ütles Iisebel Ahabile: „Tõuse üles, võta enesele jisreellase Naaboti viinamägi, mida ta ei tahtnud sulle anda raha eest, sest Naabot ei ole enam elus, vaid on surnud!”
\par 16 Kui Ahab kuulis, et Naabot oli surnud, siis tõusis ta üles, et minna jisreellase Naaboti viinamäele seda enesele võtma.
\par 17 Aga tisbelasele Eelijale tuli Issanda sõna, kes ütles:
\par 18 „Võta kätte, mine vastu Iisraeli kuningale Ahabile, kes on Samaarias! Vaata, ta on Naaboti viinamäel, kuhu ta läks, et võtta seda enesele.
\par 19 Räägi temaga ja ütle: Nõnda ütleb Issand: Kas oled tapnud ja omastanud? Siis Räägi temaga ja ütle: Nõnda ütleb Issand: Samas paigas, kus koerad lakkusid Naaboti verd, lakuvad koerad ka su enese verd!”
\par 20 Ja Ahab küsis Eelijalt: „Kas nüüd leidsid mind, mu vaenlane?” Ja ta vastas: ”Leidsin. Sellepärast et sa oled ennast müünud tegema, mis Issanda silmis on kuri.
\par 21 Vaata, ma lasen sulle tulla õnnetuse ja ma pühin sind ära, ma hävitan Ahabilt meesolendid, niihästi orjad kui vabad Iisraelis.
\par 22 Ja ma talitan sinu sooga nagu Nebati poja Jerobeami sooga ja nagu Ahija poja Baesa sooga, meelepaha pärast, mida sa mulle oled valmistanud, ja et sa oled saatnud Iisraeli pattu tegema.
\par 23 Aga Issand rääkis ka Iisebeli kohta, öeldes: Koerad söövad Iisebeli Jisreeli põllul.
\par 24 Kes Ahabi omadest sureb linnas, selle söövad koerad, ja kes sureb väljal, selle söövad taeva linnud.”
\par 25 Tõesti, ei ole olnud Ahabi sarnast, kes enese müüs tegema, mis Issanda silmis on kuri, sellepärast et ta naine Iisebel teda kihutas.
\par 26 Ta talitas väga põlastusväärselt, käies ebajumalate järel, nagu olid teinud emorlased, keda Issand oli ära ajanud Iisraeli laste eest.
\par 27 Aga kui Ahab kuulis neid sõnu, siis ta käristas oma riided lõhki, pani kotiriide ümber ihu ja paastus; ja ta magas kotiriides ning käis tasahilju.
\par 28 Ja tisbelasele Eelijale tuli Issanda sõna, kes ütles:
\par 29 „Kas oled näinud, kuidas Ahab on ennast alandanud minu ees? Sellepärast et ta on ennast alandanud minu ees, ei saada ma temale õnnetust tema päevil; ma saadan tema soole õnnetuse ta poja päevil.”

\chapter{22}

\par 1 Kolm aastat elati nõnda, et Süüria ja Iisraeli vahel ei olnud sõda.
\par 2 Aga kolmandal aastal läks Juuda kuningas Joosafat Iisraeli kuninga juurde.
\par 3 Ja Iisraeli kuningas ütles oma sulastele: „Te ju teate, et Gileadi Raamot on meie oma, aga meie ei tee midagi, et võtta see ära Süüria kuninga käest!”
\par 4 Ja ta küsis Joosafatilt: „Kas tuled koos minuga sõtta Gileadi Raamoti pärast?„ Ja Joosafat vastas Iisraeli kuningale: ”Nagu sina, nõnda mina, nagu sinu rahvas, nõnda minu rahvas, nagu sinu hobused, nõnda minu hobused.”
\par 5 Aga Joosafat ütles Iisraeli kuningale: „Küsi ometi enne Issanda sõna!”
\par 6 Siis Iisraeli kuningas kogus kokku prohvetid, ligi nelisada meest, ja küsis neilt: „Kas ma võin minna sõdima Gileadi Raamoti vastu või pean loobuma?„ Ja nad vastasid: ”Mine, ja Issand annab selle kuninga kätte!”
\par 7 Aga Joosafat küsis: „Kas ei ole siin veel mõnda Issanda prohvetit, et saaksime temalt küsida?”
\par 8 Ja Iisraeli kuningas vastas Joosafatile: „On küll veel üks mees, kellelt saaks küsida Issanda nõu. Aga mina vihkan teda, sest ta ei kuuluta mulle head, küll aga halba. See on Miika, Jimla poeg.„ Aga Joosafat ütles: ”kuningas ärgu rääkigu nõnda!”
\par 9 Siis Iisraeli kuningas kutsus ühe hoovkondlase ja ütles: „Kähku siia Miika, Jimla poeg!”
\par 10 Iisraeli kuningas ja Juuda kuningas Joosafat istusid kumbki oma aujärjel, mantlid seljas, rehepeksu väljakul Samaaria värava suus, ja kõik prohvetid kuulutasid nende ees.
\par 11 Sidkija, Kenaana poeg, oli teinud enesele raudsarved ja ütles: „Nõnda ütleb Issand: Nendega pead sa puskima süürlasi, kuni nad on hävitatud!”
\par 12 Ja kõik prohvetid kuulutasid sedasama ning ütlesid: „Mine Gileadi Raamotisse ja see õnnestub sul, sest Issand annab selle kuninga kätte!”
\par 13 Ja käskjalg, kes oli läinud Miikat kutsuma, rääkis temaga ning ütles: „Vaata ometi, prohvetite sõnad on nagu ühest suust kuningale head; olgu siis sinugi sõnad nagu ühel nende hulgast ja räägi head!”
\par 14 Aga Miika vastas: „Nii tõesti kui Issand elab, mina räägin, mis Issand mulle ütleb.”
\par 15 Ja kui ta tuli kuninga juurde, siis ütles kuningas temale: „Miika! Kas me võime minna sõdima Gileadi vastu või peame loobuma?„ Ja ta vastas temale: ”Mine, ja see õnnestub sul, sest Issand annab selle kuninga kätte!”
\par 16 Aga kuningas ütles temale: „Mitu korda ma pean sind vannutama, et sa ei räägiks mulle muud kui ainult tõtt Issanda nimel?”
\par 17 Siis ta ütles: „Ma nägin kogu Iisraeli hajali olevat mägedel nagu lambad, kellel ei ole karjast. Ja Issand ütles: Neil pole isandaid. Igaüks pöördugu rahuga koju!”
\par 18 Siis ütles Iisraeli kuningas Joosafatile: „Eks ma öelnud sulle, et ta ei kuuluta mulle head, küll aga halba!”
\par 19 Aga Miika ütles: „Seepärast kuule Issanda sõna! Ma nägin Issandat istuvat oma aujärjel ja kogu taeva sõjaväe seisvat ta juures temast paremal ja vasakul pool.
\par 20 Ja Issand ütles: Kes tahaks ahvatleda Ahabit, et ta läheks ja langeks Gileadi Raamotis? Siis rääkis üks nii ja teine rääkis naa.
\par 21 Aga üks vaim tuli ja seisis Issanda ees ning ütles: Mina ahvatlen teda! Ja Issand küsis temalt: Kuidas?
\par 22 Ja ta vastas: Ma lähen ja olen valevaim kõigi ta prohvetite suus. Siis ütles Issand: Sina oled ahvatleja ja küllap sa suudad. Mine ja tee nõnda!
\par 23 Ja nüüd, vaata, Issand on pannud valevaimu kõigi nende sinu prohvetite suhu ja Issand on kuulutanud sulle halba.”
\par 24 Siis astus ligi Sidkija, Kenaana poeg, ja lõi Miikat põse pihta ning küsis: „Missugust teed on Issanda Vaim minust läinud sinusse rääkima?”
\par 25 Ja Miika vastas: „Vaata, sa näed seda päeval, kui sa pead minema kambrist kambrisse, et ennast peita!”
\par 26 Siis ütles Iisraeli kuningas: „Võta Miika ja vii ta tagasi linnapealik Aamoni ja kuningapoeg Joase juurde
\par 27 ning ütle: Nõnda ütleb kuningas: Pange ta vangikotta ja toitke teda hädapäraselt leiva ja veega, kuni ma rahuga tagasi tulen!”
\par 28 Aga Miika ütles: „Kui sa tõesti rahuga tagasi tuled, siis ei ole Issand minu läbi rääkinud.„ Ja ta ütles veel: ”Kuulge, kõik rahvad!”
\par 29 Ja Iisraeli kuningas ja Juuda kuningas Joosafat läksid üles Gileadi Raamotisse.
\par 30 Ja Iisraeli kuningas ütles Joosafatile: „Mina panen sõtta minnes teised riided selga, aga sina kanna oma riideid!” Ja Iisraeli kuningas pani teised riided selga ning läks sõtta.
\par 31 Aga Süüria kuningas oli andnud käsu sõjavankrite pealikuile, neid oli tal kolmkümmend kaks, ja oli öelnud: „Ärge tapelge mitte ühegi muu, vähema või suurema vastu kui üksnes Iisraeli kuninga vastu!”
\par 32 Kui siis sõjavankrite pealikud nägid Joosafatti, ütlesid nad: „See on kindlasti Iisraeli kuningas.” Ja nad pöördusid tema vastu sõdima. Siis Joosafat hakkas kisendama.
\par 33 Kui sõjavankrite pealikud nägid, et see ei olnudki Iisraeli kuningas, siis nad pöördusid tagasi tema järelt.
\par 34 Aga keegi mees, kes oli huupi oma ammu vinna tõmmanud, tabas Iisraeli kuningat rihmade ja soomustuse vahele. Siis kuningas ütles oma sõjavankri juhile: „Pööra ümber ja vii mind võitlusest välja, sest ma olen haavatud!”
\par 35 Kuid taplus ägenes sel päeval üha ja kuningas pidi jääma vankrisse süürlaste vastu, aga õhtul ta suri; ja haavast oli veri voolanud vankri põhja.
\par 36 Päikeseloojakul käis hädakisa läbi leeri. Öeldi: „Iga mees oma linna! Iga mees oma maale!”
\par 37 Nõnda suri kuningas ja viidi Samaariasse; ja kuningas maeti Samaariasse.
\par 38 Ja kui vankrit loputati Samaaria tiigi ääres, siis lakkusid koerad tema verd ja hoorad pesid end selles, nagu oli olnud Issanda sõna, mis ta oli öelnud.
\par 39 Ja mis veel tuleks öelda Ahabist ja kõigest, mis ta tegi, ja elevandiluukojast, mille ta ehitas, ja kõigist linnadest, mis ta ehitas, eks sellest ole kirjutatud Iisraeli kuningate Ajaraamatus?
\par 40 Ja Ahab läks magama oma vanemate juurde ja tema poeg Ahasja sai tema asemel kuningaks.
\par 41 Joosafat, Aasa poeg, sai Juuda kuningaks Iisraeli kuninga Ahabi neljandal aastal.
\par 42 Joosafat oli kuningaks saades kolmkümmend viis aastat vana, ja ta valitses Jeruusalemmas kakskümmend viis aastat; ta ema nimi oli Asuuba, Silhi tütar.
\par 43 Ta käis täiesti oma isa Aasa teed, sellelt ta ei lahkunud, tehes, mis õige oli Issanda silmis.
\par 44 Aga ohvrikünkad ei kadunud, rahvas ohverdas ja suitsutas veelgi ohvriküngastel.
\par 45 Ja Joosafat tegi Iisraeli kuningaga rahu.
\par 46 Ja mis veel tuleks öelda Joosafatist ja tema vägitegudest, mis ta tegi, ja kuidas ta sõdis, eks sellest ole kirjutatud Juuda kuningate Ajaraamatus?
\par 47 Ja pühamu viimased pordumehed, kes tema isa Aasa päevil olid alles jäänud, pühkis ta maalt.
\par 48 Edomis ei olnud kuningat; kuningaks oli asevalitseja.
\par 49 Joosafat oli teinud Tarsise laevu, mis pidid minema Oofirisse kulla järele, ent need ei läinud, sest laevad hukkusid Esjon-Geberis.
\par 50 Siis ütles Ahasja, Ahabi poeg, Joosafatile: „Lase minu sulased lähevad koos sinu sulastega laevadesse!” Aga Joosafat ei tahtnud.
\par 51 Ja Joosafat läks magama oma vanemate juurde ja ta maeti oma vanemate juurde ta isa Taaveti linna. Ja tema poeg Jooram sai tema asemel kuningaks.
\par 52 Ahasja, Ahabi poeg, sai Samaarias Iisraeli kuningaks Juuda kuninga Joosafati seitsmeteistkümnendal aastal, ja ta valitses Iisraelis kaks aastat.
\par 53 Tema tegi kurja Issanda silmis ning käis oma isa ja ema teed, ja Nebati poja Jerobeami teed, kes Iisraeli oli saatnud pattu tegema.



\end{document}