\begin{document}

\title{Saalomoni Õpetus- e Vanasõnad}

\chapter{1}

\par 1 Saalomoni, Taaveti poja, Iisraeli kuninga õpetussõnad
\par 2 tarkuse ja õpetuse tundmaõppimiseks, mõistlike sõnade mõistmiseks,
\par 3 et võtta õpetust targaks käitumiseks, õigluseks ja õiguseks ning õigeks eluviisiks,
\par 4 et anda kogenematuile oidu, noortele teadmisi ja otsustusvõimet -
\par 5 kes tark on, see kuuleb seda ja võtab veelgi enam õpetust; kes aru saab, see omandab oskuse
\par 6 õpetus- ja tähendamissõnade, tarkade meeste ütluste ja nende mõistatuste mõistmiseks.
\par 7 Issanda kartus on tunnetuse algus, meeletud põlgavad tarkust ja õpetust.
\par 8 Kuule, mu poeg, oma isa õpetust ja ära jäta tähele panemata oma ema juhatust,
\par 9 sest need on su peale ilupärjaks ja kaela ümber keeks!
\par 10 Mu poeg! Kui patused sind ahvatlevad, siis ära nõustu!
\par 11 Kui nad ütlevad: „Tule meiega! Varitseme verd, luurame süütut ilma põhjuseta,
\par 12 neelame neid otsekui surmavald, elusalt ja tervelt nagu haudaminejaid,
\par 13 küllap me siis leiame kõiksugust kallist vara ja täidame oma kojad saagiga.
\par 14 Heida liisku meie keskel ja meil kõigil olgu ühine kukkur!” -
\par 15 siis, mu poeg, ära mine nendega seda teed, hoia oma jalg nende radadelt!
\par 16 Sest nende jalad jooksevad kurja poole ja nad tõttavad verd valama.
\par 17 Ent asjatu on võrku laotada kõigi tiivuliste nähes:
\par 18 sest nad varitsevad omaenese verd, luuravad omaenese hinge.
\par 19 Niisugune tee on igaühel, kes ahnitseb kasu: selle omanikult võetakse hing.
\par 20 Tarkus hüüab tänavail, annab oma häält kuulda turgudel,
\par 21 kisendab lärmakates paikades, kõneleb oma sõnu linnaväravate suus:
\par 22 „Kui kaua te, rumalad, armastate rumalust? Kui kaua meeldib hooplejail hoobelda ja alpidel vihata teadmist?
\par 23 Pöörduge minu noomimise alla! Vaata, ma lasen teile voolata oma vaimu, teen teile teatavaks oma sõnad.
\par 24 Et ma olen kutsunud, aga te olete tõrkunud, olen sirutanud oma käe, aga ükski pole tähele pannud,
\par 25 ja et te pole hoolinud mitte ühestki minu nõust ega ole tahtnud mu noomimist,
\par 26 siis naeran minagi teie õnnetust, pilkan, kui teie kohkumine tuleb,
\par 27 kui teie kohkumine tormina tuleb ja teie õnnetus tuulekeerisena saabub, kui häda ja ahastus teile kallale kipuvad.
\par 28 Siis nad hüüavad mind, aga mina ei vasta, otsivad mind, aga ei leia.
\par 29 Sellepärast et nad on vihanud tunnetust ega ole valinud Issanda kartust,
\par 30 et nad ei ole hoolinud minu nõust, vaid on põlastanud kõiki mu noomitusi,
\par 31 peavad nad siis sööma oma tegude vilja ja küllastuma oma kavatsustest.
\par 32 Sest rumalaid tapab nende taganemine ja alpe hukkab nende hooletus.
\par 33 Aga kes mind kuulab, võib julgesti elada ja olla muretu, kurja kartmata.”

\chapter{2}

\par 1 Mu poeg! Kui sa mu sõnad vastu võtad ja mu käsud enesele talletad,
\par 2 lased oma kõrva tarkust tähele panna, pöörad südame arukuse poole,
\par 3 jah, kui sa mõistuse appi kutsud ja tood oma hääle kuuldavale arukuse heaks,
\par 4 kui sa seda otsid nagu hõbedat ja püüad leida nagu peidetud varandust,
\par 5 siis sa mõistad Issanda kartust ja leiad Jumala tunnetuse.
\par 6 Sest Issand annab tarkuse, tema suust tuleb tunnetus ja arukus.
\par 7 Tema talletab õigetele edu, on kilbiks neile, kes elavad laitmatult,
\par 8 kaitstes õiguse radu ja valvates oma vagade teed.
\par 9 Siis sa mõistad õiglust ja õigust, ja seda, mis õige on - kõiki häid teid.
\par 10 Sest siis tuleb tarkus su südamesse ja tunnetus hakkab meeldima su hingele.
\par 11 Otsustusvõime valvab su üle, arukus kaitseb sind,
\par 12 päästes sind kurjalt teelt, meeste käest, kes räägivad pööraselt,
\par 13 kes hülgavad sirged rajad, et käia pimedatel teedel,
\par 14 kes rõõmustavad kurja tehes, hõiskavad pööraste roimade juures,
\par 15 kelle rajad on kõverad ja kelle teed on eksiteed.
\par 16 See hoiab sind ka võõra naise eest, muulasest naise eest, kes räägib meelitavaid sõnu,
\par 17 kes on hüljanud oma noorpõlve sõbra ja on unustanud oma Jumala seaduse.
\par 18 Sest tema koda vajub alla surma poole ja tema teed varjuderiiki.
\par 19 Ükski, kes tema juurde sisse läheb, ei tule tagasi ega jõua taas eluradadele.
\par 20 See on siis selleks, et sa võiksid käia heade teel ja hoida õigete radu,
\par 21 sest õiglased tohivad elada maa peal ja vagad sinna järele jääda,
\par 22 aga õelad hävitatakse maa pealt ja truudusemurdjad pühitakse ära.

\chapter{3}

\par 1 Mu poeg, ära unusta mu õpetust, vaid su süda hoidku alal mu käsud,
\par 2 sest need lisavad sulle pikka iga, eluaastaid ja rahu!
\par 3 Heldus ja tõde ärgu jätku sind maha! Seo need enesele ümber kaela, kirjuta need oma südamelauale,
\par 4 siis sa leiad armu ja heakskiitu Jumala ja inimeste silmis!
\par 5 Looda Issanda peale kõigest südamest ja ära toetu omaenese mõistusele!
\par 6 Õpi teda tundma kõigil oma teedel, siis ta teeb su teerajad tasaseks!
\par 7 Ära ole iseenese silmis tark! Karda Issandat ja tagane kurjast!
\par 8 See on su ihule terviseks ja luudele-liikmetele kosutuseks.
\par 9 Austa Issandat oma varandusega ja uudseviljaga kõigest oma saagist,
\par 10 siis su aidad täituvad küllusega ja veiniaamid ajavad üle!
\par 11 Mu poeg, ära põlga Issanda karistust ja ärgu olgu sulle vastumeelne tema noomimine,
\par 12 sest keda Issand armastab, seda ta karistab otsekui isa poega, kellele ta head tahab!
\par 13 Õnnis on inimene, kes leiab tarkuse, ja inimene, kes jõuab arusaamisele,
\par 14 sest sellest on rohkem kasu kui hõbedast ja rohkem tulu kui kullast.
\par 15 See on kallim kui pärlid, ja ükski asi, mida sa ihaldad, ei ole sellega võrreldav.
\par 16 Selle paremal käel on pikk iga ja vasakul käel rikkus ning au.
\par 17 Selle teed on armsad teed ja kõik selle rajad on rahu.
\par 18 See on elupuuks neile, kes sellest kinni haaravad, õnnis, kes seda suudab hoida.
\par 19 Issand on tarkusega rajanud maa, arukusega valmistanud taeva.
\par 20 Tema teadmisel lõhkevad sügavused ja pilved tilguvad kastet.
\par 21 Mu poeg, ära kaota neid silmist: säilita arukus ja otsustusvõime,
\par 22 siis on need su hingele eluks ja kaela ümber kaunistuseks.
\par 23 Siis sa võid käia oma teed julgesti, su jalg ei tarvitse komistada.
\par 24 Kui sa heidad puhkama, siis pole sul vaja karta, vaid sa saad magada ja su uni on magus.
\par 25 Ei sa siis karda äkilist hirmutust ega õelate õnnetust, kui see tuleb.
\par 26 Sest Issand on kõigil su teedel ja hoiab su jalga püünise eest.
\par 27 Ära keela head neile, kes seda vajavad, kui su käel on jõudu seda teha!
\par 28 Ära ütle oma ligimesele: „Mine praegu ära ja tule teinekord tagasi, küll ma homme annan”, kui sa seda kohe saad teha!
\par 29 Ära kavatse kurja oma ligimese vastu, kui ta elab usaldavalt su juures!
\par 30 Ära riidle ühegi inimesega põhjuseta, kui ta ei ole sulle kurja teinud!
\par 31 Ära ole kade mehele, kes on vägivaldne, ja ära vali ühtegi tema teedest,
\par 32 sest pöörane on Issanda meelest jäle, aga õigetega on tema osadus!
\par 33 Issanda needus on õela kojas, aga ta õnnistab õiglase eluaset.
\par 34 Pilkajaile on temagi pilkaja, aga alandlikele annab ta armu.
\par 35 Targad pärivad au, aga albid kannavad häbi.

\chapter{4}

\par 1 Kuulge, pojad, isa manitsust, ja pange tähele, et õpiksite arukust!
\par 2 Sest ma annan teile hea õpetuse, ärge hüljake mu juhatust!
\par 3 Kui ma alles olin oma isa poeg, õrn ja ainus oma ema hoole all,
\par 4 siis õpetas mind mu isa ja ütles mulle: Su süda hoidku minu sõnu, pea mu käske, siis sa jääd elama!
\par 5 Taotle tarkust, taotle arukust - ära seda unusta - ära kaldu kõrvale mu suu sõnadest!
\par 6 Ära seda hülga, siis see hoiab sind; armasta seda, siis see kaitseb sind!
\par 7 Tarkuse algus on see: taotle tarkust ja taotle arukust kogu oma varanduse hinnaga.
\par 8 Pea seda kõrgeks, siis see ülendab sind, kui sa seda süleled, siis see austab sind!
\par 9 See paneb sulle pähe ilupärja, kingib sulle kauni krooni.
\par 10 Kuula, mu poeg, ja võta vastu mu sõnad, siis saab sulle palju eluaastaid!
\par 11 Ma õpetan sulle tarkuseteed, ma juhin sind õigeisse rööpaisse.
\par 12 Kui sa seal käid, siis on su samm vaba, ja kui sa jooksed, siis sa ei komista.
\par 13 Haara kinni õpetusest, ära lase lahti, hoia seda, sest see on su elu!
\par 14 Ära mine õelate rajale ja ära astu kurjade teele!
\par 15 Jäta see, ära käi seal, hoidu sellest ja mine mööda!
\par 16 Sest nad ei saa magada, enne kui nad kurja on teinud; neilt võetakse uni, kui nad kedagi ei ole saanud panna komistama.
\par 17 Sest nad söövad õeluse leiba ja joovad vägivalla veini.
\par 18 Aga õigete rada on otsekui valgusepaistus, mis muutub üha selgemaks, kuni päev on saabunud.
\par 19 Õelate tee on nagu pilkane pimedus: nad ei tea, mille peale nad komistavad.
\par 20 Mu poeg, pane tähele mu sõnu, pööra oma kõrv mu kõnede poole!
\par 21 Ära kaota neid enesel silmist, hoia need oma südames;
\par 22 sest need on eluks sellele, kes need leiab, ja terviseks kogu ta ihule!
\par 23 Hoia oma südant enam kui kõike muud, mida tuleb hoida, sest sellest lähtub elu!
\par 24 Saada enesest ära valelik suu ja hoidu huulte väärusest!
\par 25 Su silmad vaadaku otse ja su pilk olgu suunatud ettepoole!
\par 26 Pane tähele oma jala suunda, siis on kõik su teed kindlad!
\par 27 Ära kaldu paremale ega vasakule, hoia oma jalga kurjast!

\chapter{5}

\par 1 Mu poeg, pane tähele mu tarkust, pööra oma kõrv minu arusaamise poole,
\par 2 et tallele panna head nõu ja et su huuled võiksid säilitada teadlikkuse!
\par 3 Sest võõra naise huuled tilguvad mett ja tema suulagi on libedam kui õli.
\par 4 Aga viimaks on ta kibe nagu koirohi, terav otsekui kaheterane mõõk.
\par 5 Ta jalad lähevad alla surma, ta sammud veavad põrguhaua poole.
\par 6 Ta ei pea silmas elurada, ta jäljed ekslevad, ilma et ta märkakski.
\par 7 Seepärast, mu pojad, kuulake mind, ja ärge lahkuge mu suu sõnadest!
\par 8 Hoia oma tee temast kaugel ja ära mine tema koja ukse ligi,
\par 9 et sa ei annaks oma au mitte teistele ega oma aastaid armutule,
\par 10 et võõrad ei küllastuks sinu varast ja su töövaev ei läheks muulase kotta,
\par 11 et sa viimaks ei hakkaks vinguma, kui su liha ja ihu on lõppenud,
\par 12 ega ütleks: „Miks ma küll vihkasin õpetust ja miks mu süda põlgas noomitust?
\par 13 Miks ma ei kuulanud juhatajate häält ega pööranud kõrva õpetajate poole?
\par 14 Peagi oleksin sattunud lausa õnnetusse keset kogukonda ja kogudust!”
\par 15 Joo vett oma kaevust, voolavat vett oma allikast!
\par 16 Kas peaksid su lätted valguma tänavale, su veeojad turgudele?
\par 17 Kuulugu need ainult sinule, aga mitte koos sinuga võõraile!
\par 18 Olgu õnnistatud su allikas ja tunne rõõmu oma noorpõlve naisest,
\par 19 kes on otsekui armas emahirv, kena kaljukits - ta rinnad joovastagu sind igal ajal, eksi alati tema kallistustesse.
\par 20 Miks peaksid, mu poeg, eksima võõra naise juurde, kaisutama võõramaa naise põue?
\par 21 Sest mehe teed on Issanda silme ees ja tema paneb tähele kõiki ta jälgi.
\par 22 Õelat tabavad tema enese süüteod ja teda peetakse kinni tema enese patuköitega.
\par 23 Ta sureb õpetuse puudusest ja läheb eksiteele suurest rumalusest.

\chapter{6}

\par 1 Mu poeg, kui oled hakanud käendajaks oma ligimesele, kui oled löönud kätt võõra heaks,
\par 2 kui oled oma suu sõnade pärast võrku mässitud, oma suu sõnade pärast kinni püütud,
\par 3 siis tee ometi, mu poeg, enese päästmiseks nõnda, sest sa oled ju sattunud oma ligimese võimusesse: mine, alanda ennast ja anu oma ligimest!
\par 4 Ära anna und oma silmadele ega suikumist laugudele!
\par 5 Päästa end gasellina lõksu eest või otsekui lind püüdja käest!
\par 6 Mine sipelga juurde, sina laisk, vaatle tema viise ja saa targaks!
\par 7 Kuigi tal ei ole pealikut, ülevaatajat ega valitsejat,
\par 8 valmistab ta siiski suvel oma leiva ja kogub lõikusajal oma toiduse.
\par 9 Kui kaua sa, laisk, magad, millal sa ärkad unest?
\par 10 Veel pisut und, pisut tukkumist, pisut pikutamist ristis kätega,
\par 11 siis tuleb vaesus sulle kallale otsekui röövel ja puudus nagu relvastatud mees.
\par 12 Kõlvatu inimene, nurjatu mees, käib, vale suus,
\par 13 pilgutab silmi, annab jalaga märku, viitab sõrmedega,
\par 14 tal on kavalus südames, ta kavatseb kurja, ta külvab alati riidu.
\par 15 Seepärast tuleb tema õnnetus äkitselt, ta murtakse silmapilkselt ja abi ei ole.
\par 16 Neid kuut asja vihkab Issand, jah, seitse on tema hingele jäledad:
\par 17 ülbed silmad, valelik keel, käed, mis valavad süütut verd,
\par 18 süda, mis sepitseb nurjatuid kavatsusi, jalad, mis kiiresti jooksevad kurja poole,
\par 19 valetunnistaja, kes väidab valet, ja see, kes külvab riidu vendade vahel.
\par 20 Pea, mu poeg, oma isa käsku ja ära hülga oma ema juhatust!
\par 21 Seo need alatiseks oma südame külge, mähi need enesele ümber kaela!
\par 22 Need juhtigu sind, kui sa kõnnid, valvaku sind, kui sa magad, ja kõnelgu sinuga, kui sa ärkad!
\par 23 Sest käsk on lamp ja õpetus on valgus, ja korralekutsuvad manitsused on elutee,
\par 24 et sind hoida halva naise eest, võõramaa naise libeda keele eest.
\par 25 Ära himusta oma südames tema ilu ja ära lase ennast kütkestada tema silmalaugudest!
\par 26 Sest hooranaise jaoks jätkub pätsist leivast, aga teise mehe naine püüab kallist hinge.
\par 27 Kas keegi võib kanda põues tuld, ilma et ta riided põleksid?
\par 28 Kas keegi võib käia tuliste süte peal, ilma et ta jalad kõrbeksid?
\par 29 Nõnda on sellega, kes läheb oma ligimese naise juurde: ei jää karistamata ükski, kes puutub temasse!
\par 30 Eks põlata varast, isegi kui ta varastab, et oma kõhtu täita, kui tal on nälg?
\par 31 Ja tabamise korral peab ta tasuma seitsmekordselt, ära andma kogu oma koja varanduse.
\par 32 Kes abielunaisega abielu rikub, on meeletu; seda teeb ainult see, kes oma hinge tahab hävitada.
\par 33 Teda tabab nuhtlus ning häbi ja tema teotust ei saa ära pühkida.
\par 34 Sest armukadedusest tekib mehe viha ja ta ei halasta kättemaksupäeval.
\par 35 Ta ei hooli mingist lepitushinnast ega rahuldu, kuigi sa lisaksid kingitusi.

\chapter{7}

\par 1 Mu poeg, hoia mu sõnu ja pane mu käsud enesele tallele!
\par 2 Pea mu käske, et sa jääksid elama, hoia mu õpetust kui oma silmatera!
\par 3 Seo need enesele sõrmede ümber, kirjuta need oma südamelauale!
\par 4 Ütle tarkusele: „Sa oled mu õde!” ja hüüa arukust sugulaseks,
\par 5 et see hoiaks sind võõra naise eest, võõramaa naise eest, kes räägib libedaid sõnu.
\par 6 Sest oma koja aknast, aknaavast ma vaatasin
\par 7 ja nägin kogenematute seas, märkasin poiste hulgas arutut noormeest:
\par 8 ta käis mööda tänavat kuni selle nurgani ja sammus siis naise koja poole,
\par 9 videvikus, kui päev jõudis õhtule, südaööl ja pimedas.
\par 10 Ja vaata, naine tuli temale vastu, hooraehtes ja kavala südamega.
\par 11 Ta oli rahutu ja isemeelne, ta jalad ei püsinud kodus:
\par 12 mõnikord oli ta tänaval, mõnikord turgudel, ja ta varitses iga nurga juures.
\par 13 Ta haaras temast kinni, suudles teda ja ütles temale häbitu näoga:
\par 14 „Ma pidin viima tänuohvreid ja ma tasusin täna oma tõotused.
\par 15 Seepärast ma tulin välja sulle vastu, sind otsima, ja ma leidsin su.
\par 16 Ma katsin oma voodi vaipadega, kirju Egiptuse lõuendiga.
\par 17 Ma riputasin voodisse mürri, aaloed ja kaneeli.
\par 18 Tule, joobume kallistustest hommikuni, tundkem rõõmu armastusest!
\par 19 Sest mu mees ei ole kodus, ta läks pikale teekonnale.
\par 20 Ta võttis rahakukru kaasa, ta tuleb koju alles täiskuu ajaks.”
\par 21 Ta võrgutas teda paljude meelitussõnadega, ahvatles oma libedate huultega.
\par 22 Äkitselt läks mees temale järele, nagu härg, keda viiakse tappa, otsekui jalarauad meeletu karistuseks,
\par 23 nagu lind, kes tõttab võrku ega tea, et see maksab tema hinge, kuni nool lõhestab ta maksa.
\par 24 Ja nüüd, pojad, kuulge mind, ja pange tähele mu suu sõnu!
\par 25 Ärgu pöördugu su süda tema teedele, ära eksi tema radadele!
\par 26 Sest palju on mahalööduid, keda tema on viinud langusele, rohkesti on neid, keda tema on tapnud.
\par 27 Tema koda on põrgutee - see viib alla surma kambritesse.

\chapter{8}

\par 1 Eks hüüa tarkus, eks tõsta arukus häält?
\par 2 Küngastel tee ääres, teelahkmetel, on ta võtnud aset.
\par 3 Väravate kõrval, linna sissekäigus, seal, kus ustest sisse minnakse, kisendab ta:
\par 4 „Teile, mehed, ma hüüan, ja mu hääl kõlab inimlastele!
\par 5 Õppige, rumalad, mõistlikkust, ja albid, hakake aru saama!
\par 6 Kuulge, sest ma räägin tõtt ja mu huuled on avatud õigeks asjaks!
\par 7 Jah, mu suulagi toob kuuldavale tõe, aga õelus on mu huultele jäle.
\par 8 Kõik mu suu sõnad on siirad, neis pole midagi kavalat ega väära.
\par 9 Need kõik on selged arukale ja õiged neile, kes on leidnud tunnetuse.
\par 10 Võtke vastu mu õpetus, aga mitte hõbedat, pigem teadlikkus kui valitud kuld!
\par 11 Sest tarkus on parem kui pärlid ja ükski ihaldatav asi ei ole sellega võrdne.
\par 12 Mina, tarkus, olen mõistlikkuse naaber, ja ma leian tunnetuse ning targa nõu.
\par 13 Issanda kartus on kurja vihkamine. Ma vihkan kõrkust ja ülbust, halbu eluviise ja pöörast suud.
\par 14 Mul on nõu ja kordaminek, ma olen mõistus, mul on võimus.
\par 15 Minu abiga valitsevad kuningad ja võimukandjad korraldavad õiglust.
\par 16 Minu abiga valitsevad vürstid ja suursugused, kõik õiged kohtumõistjad.
\par 17 Mina armastan neid, kes armastavad mind, ja kes otsivad mind, need leiavad minu.
\par 18 Minu juures on rikkus ja au, jääv varandus ja hüvang.
\par 19 Minu vili on kullast parem, parem koguni kui puhas kuld, ja mu saak on parem kui valitud hõbe.
\par 20 Ma jäin õigluse rajale, keset õiguse jalgteid,
\par 21 et anda jõukust neile, kes mind armastavad, ja et täita nende varakambreid.
\par 22 Issand lõi mind oma töö alguses, esimesena oma töödest muistsel ajal.
\par 23 Igavikust alates on mind eraldatud, ürgajast peale, enne maailma algust.
\par 24 Ma olen sündinud enne sügavusi, enne kui puhkesid allikad.
\par 25 Veel enne kui mäed paigale pandi, enne künkaid sündisin mina,
\par 26 kui ta veel ei olnud teinud maad ega maastikku, ei maailma esimest tolmukübetki.
\par 27 Kui tema valmistas taevad, olin mina seal, siis kui ta joonistas sõõri sügavuse pinnale,
\par 28 kui ta kinnitas pilved ülal, kui said jõuliseks sügavuse allikad,
\par 29 kui ta merele pani piiri, et veed ei astuks üle tema käsust, kui ta kinnitas maa alused,
\par 30 siis olin mina tema kõrval kui lemmiklaps, olin päevast päeva temale rõõmuks, mängisin igal ajal tema ees,
\par 31 mängisin tema maailma maa peal ja tundsin rõõmu inimlastest.
\par 32 Ja nüüd, pojad, kuulge mind! Õndsad on need, kes hoiavad minu teed.
\par 33 Kuulge õpetust ja saage targaks, ärge jätke seda tähele panemata!
\par 34 Õnnis on inimene, kes kuulab mind, kes päevast päeva valvab mu ustel, on vahiks mu uksepiitade juures.
\par 35 Sest kes leiab minu, leiab elu ja saab Issanda hea meele osaliseks.
\par 36 Aga kes ei taba mind, teeb kahju oma hingele; kõik, kes vihkavad mind, armastavad surma.”

\chapter{9}

\par 1 Tarkus ehitas enesele koja, raius oma seitse sammast.
\par 2 Ta tappis tapaloomad, segas veini, kattis lauagi valmis.
\par 3 Ta läkitas oma teenijad, laskis hüüda linna küngastel:
\par 4 „Kes on kogenematu, pöördugu siia!” Napiarulisele ta ütles:
\par 5 „Tulge, sööge mu leiba ja jooge mu segatud veini!
\par 6 Jätke maha rumalus, et jääksite elama, ja käige arukuse teel!”
\par 7 Kes õpetab pilkajat, toob iseenesele häbi, ja kes noomib õelat, sellele jääb külge halb märk.
\par 8 Ära noomi pilkajat, et ta sind ei vihkaks, noomi tarka, ja ta armastab sind!
\par 9 Anna targale, ja ta saab veelgi targemaks, õpeta õiglast, ja ta võtab veelgi enam õpetust!
\par 10 Issanda kartus on tarkuse algus ja Kõigepühama äratundmine on arukus.
\par 11 Sest minu läbi saavad su päevad paljuks ja su eluaastad jätkuvad.
\par 12 Kui oled tark, siis oled tark iseenese jaoks; ja kui oled pilkaja, siis pead sedagi üksinda kandma.
\par 13 Meeletusenaine on rahutu, rumalus ei tunne midagi.
\par 14 Ta istub oma koja uksel, istme peal linna küngastel,
\par 15 kutsudes möödaminejaid, neid, kes käivad otseradu:
\par 16 „Kes on kogenematu, pöördugu siia!” Ja napiarulisele ta ütleb:
\par 17 „Varastatud vesi on magus ja salajas söödud leib on hõrk.”
\par 18 Aga teine ei teagi, et seal on kadunud hinged, et tema kutsutud on põrgu sügavustes.

\chapter{10}

\par 1 Saalomoni õpetussõnad: Tark poeg rõõmustab isa, aga alp poeg on emale meelehärmiks.
\par 2 Ülekohtuselt saadud varandustest pole kasu, aga õiglus päästab surmast.
\par 3 Issand ei lase nälgida õiget hinge, aga ta tõrjub tagasi õelate himu.
\par 4 Laisk käsi toob vaesuse, aga virgad käed teevad rikkaks.
\par 5 Taibukas poeg kogub suvel, aga poeg, kes lõikusajal magab, teeb häbi.
\par 6 Õige pea peale tulevad õnnistused, aga õelate suus peitub vägivald.
\par 7 Mälestus õigest jääb õnnistuseks, aga õelate nimi kõduneb.
\par 8 Kes on südamelt tark, võtab käsud vastu, aga huultelt meeletu libastub.
\par 9 Kes elab laitmatult, elab julgesti, aga kelle teed on kõverad, selle käsi käib halvasti.
\par 10 Kes pilgutab silmi, põhjustab valu, ja huultelt meeletu libastub.
\par 11 Õiglase suu on eluallikas, aga õelate suus peitub vägivald.
\par 12 Vihkamine õhutab riidu, aga armastus katab kinni kõik üleastumised.
\par 13 Mõistliku huultel leidub tarkust, aga meeletu seljale on vitsa vaja.
\par 14 Targad panevad tallele, mida teavad, aga rumala suu toob hukatuse lähedale.
\par 15 Rikka varandus on tema tugev linn, vaestele on nende vaesus hukatuseks.
\par 16 Õige inimese töötasu on eluks, õela tulu on patuks.
\par 17 Kes hoiab õpetust, on elurajal, aga kes põlgab noomimist, eksib ära.
\par 18 Kes peab salaviha, selle huuled on valelikud, ja kes ajab laimujuttu, on alp.
\par 19 Kus on palju sõnu, seal ei puudu üleastumine, aga kes talitseb oma huuli, on mõistlik.
\par 20 Õiglase keel on valitud hõbe, õelate süda on vähe väärt.
\par 21 Õiglase huuled kosutavad paljusid, aga rumalad surevad arunappuse pärast.
\par 22 See on Issanda õnnistus, mis teeb rikkaks, ja oma vaev ei lisa sellele midagi juurde.
\par 23 Albile meeldib teha häbitegusid, aga mõistlikule mehele meeldib tarkus.
\par 24 Mida õel kardab, see tabab teda, aga õigete igatsus rahuldatakse.
\par 25 Kui torm üle käib, siis pole enam õelat, aga õigel on igavene alus.
\par 26 Nagu äädikas hammastele ja suits silmadele, nõnda on laisk sellele, kes tema läkitab.
\par 27 Issanda kartus jätkab elupäevi, aga õelate aastaid lühendatakse.
\par 28 Õigete ootus on rõõm, aga õelate lootus hukkub.
\par 29 Issanda tee on kindluseks laitmatule, aga kurjategijaile on ta hukatuseks.
\par 30 Õige ei kõigu iialgi, aga õelad ei jää maale elama.
\par 31 Õiglase suu edendab tarkust, aga valelik keel hävitatakse.
\par 32 Õiglase huuled teavad, mis on meelepärane, aga õelate suu on pöörane.

\chapter{11}

\par 1 Valed kaalud on Issanda meelest jäledus, aga õigest vihist on tal hea meel.
\par 2 Tuleb uhkus, Tuleb ka häbi, aga alandlikel on tarkus.
\par 3 Ausus juhib õiglasi, aga valelikkus hävitab truudusemurdjad.
\par 4 Varandusest pole kasu vihapäeval, aga õiglus päästab surmast.
\par 5 Õiglus tasandab vagale tema tee, aga õel langeb oma õeluse läbi.
\par 6 Õigete õiglus päästab neid, aga truudusemurdjaid vangistab nende oma himu.
\par 7 Kui õel inimene sureb, siis ta lootus hävib, ja rumalate ootus kaob.
\par 8 Õige päästetakse hädast, aga õel satub sinna tema asemel.
\par 9 Suuga hävitab jumalatu oma ligimese, aga õiged pääsevad teadlikkuse läbi.
\par 10 Õigete õnnest hõiskab linn ja õelate hukkumisest tuntakse rõõmu.
\par 11 Õigete õnnistusest kerkib linn, aga õelate suu kisub selle maha.
\par 12 Napiaruline on, kes halvustab oma ligimest, aga arukas mees on vait.
\par 13 Kes käib keelekandjana, reedab saladusi, aga kes on ustava vaimuga, varjab neid.
\par 14 Kus ei ole juhtimist, seal langeb rahvas, aga kus on palju nõuandjaid, seal on abi.
\par 15 Kes asub käendajaks võõrale, selle käsi käib halvasti, aga kes käendamist ei salli, võib olla muretu.
\par 16 Veetlev naine saavutab au ja virgad saavutavad rikkuse.
\par 17 Kes osutab halastust, teeb head iseenesele, aga kalk teeb valu oma ihule.
\par 18 Õel teenib petise palga, aga õigluse külvaja tõelise tasu.
\par 19 Kes püsib õigluses, jääb elama, aga kes taotleb kurja, saab surma.
\par 20 Need, kel valelik süda, on Issanda meelest jäledad, aga tal on hea meel neist, kelle tee on laitmatu.
\par 21 Käsi selle peale! Kuri ei jää karistuseta, aga õigete sugu pääseb.
\par 22 Otsekui kuldrõngas sea kärsas on ilus naine, kel puudub peenetundelisus.
\par 23 Õigete igatsuseks on ainult hea, õelate ootuseks viha.
\par 24 Üks jagab välja, aga saab üha lisa, teine hoiab varandust, ometi on puudus käes.
\par 25 Hing, kes jagab õnnistust, kosub, ja kes kastab teisi, seda ennastki kastetakse.
\par 26 Kes keelab vilja, seda sajatab rahvas, aga müüja peale tuleb õnnistus.
\par 27 Kes taotleb head, leiab tunnustuse, aga kes kavatseb kurja, seda ennast see tabab.
\par 28 Kes loodab oma rikkuse peale, langeb, aga õiged haljendavad nagu lehed.
\par 29 Kes oma koja hooletusse jätab, lõikab tuult, ja rumal saab targale sulaseks.
\par 30 Õiglase vili on elupuu, aga tark saab hinged.
\par 31 Vaata, õigele tasutakse maa peal, saati siis õelale ja patusele.

\chapter{12}

\par 1 Kes armastab hoiatust, armastab teadlikkust, aga kes vihkab noomimist, on juhm.
\par 2 Hea inimene saab Issandalt tunnustuse, aga kurja nõuga mehe ta mõistab hukka.
\par 3 Õeluse pärast ei jää inimene püsima, aga õigete juur on kõikumatu.
\par 4 Tubli naine on oma mehe kroon, aga häbitu on otsekui mädanik tema luudes.
\par 5 Õiglaste mõtted on õiged, õelate nõuanne on pettus.
\par 6 Õelate sõnad varitsevad verd, aga ausate suu päästab varitsetavad.
\par 7 Õelad kummutatakse ja neid ei ole enam, aga õigete koda jääb püsima.
\par 8 Meest kiidetakse tema mõistuse pärast, aga meelesegane on põlatud.
\par 9 Parem on olla tähtsusetu, kel ometi on ori, kui ennast laiutada ja olla leivata.
\par 10 Õige hoolib oma looma hingest, aga õelate halastuski on julm.
\par 11 Kes oma põldu harib, sel on leiba küllalt, aga kes tühja taga ajab, on meeletu.
\par 12 Õel himustab kurjade saaki, aga õigete juur kinnitatakse.
\par 13 Kurja püünis on huulte üleastumises, aga õige pääseb hädast.
\par 14 Oma suu viljast saab igaüks küllalt head, ja inimese kätetöö tuleb tagasi temale enesele.
\par 15 Rumala tee on ta enese silmis õige, aga tark võtab nõu kuulda.
\par 16 Rumala pahameel saab kohe teatavaks, aga tark katab solvangu kinni.
\par 17 Tõe kõneleja kuulutab õigust, aga valetunnistaja pettust.
\par 18 Mõni paiskab sõnu otsekui mõõgapisteid, aga tarkade keel on terviseks.
\par 19 Tõehuuled kestavad igavesti, aga valelik keel on üürikeseks.
\par 20 Kes kavatsevad kurja, neil on pettus südames, aga kes annavad rahu nõu, võivad rõõmu tunda.
\par 21 Õigele ei tule midagi halba, aga õelad on täis õnnetust.
\par 22 Valelikud huuled on Issanda meelest jäledad, aga ustavad on temale meelepärased.
\par 23 Tark inimene varjab, mida ta teab, aga alpide süda kuulutab nende rumalust.
\par 24 Kärmete käsi valitseb, aga laisk peab orjama.
\par 25 Mure mehe südames painutab teda, aga hea sõna teeb temale rõõmu.
\par 26 Õige hoiab eemale halvast, aga õelaid eksitab nende oma tee.
\par 27 Laisk ei saa küpsetada jahisaaki, aga inimese kallis vara on virkus.
\par 28 Õigluse rajal on elu, ja sillutatud teel ei ole surma.

\chapter{13}

\par 1 Tark poeg kuulab isa õpetust, aga pilkaja ei kuula sõitlustki.
\par 2 Oma suu viljast sööb mees head, aga autute igatsuseks on vägivald.
\par 3 Kes valvab oma suud, hoiab oma hinge, aga huulte ammuliajajat tabab hukatus.
\par 4 Laisa hing igatseb asjata, aga virkade hing kosub.
\par 5 Õige vihkab valelikkust, aga õel toimib vastikult ja häbiväärselt.
\par 6 Õigus kaitseb seda, kelle tee on laitmatu, aga õelus kukutab patuse.
\par 7 Üks näib rikkana, aga tal ei ole midagi, teine näib vaesena, aga tal on palju vara.
\par 8 Mehe hinge lunahind on tema rikkus, aga vaesel pole vaja sõitlust kuulda.
\par 9 Õigete valgus paistab rõõmsasti, aga õelate lamp kustub ära.
\par 10 Ülbusest tõuseb riid, aga kes nõu kuulda võtavad, on targad.
\par 11 Hõlpsalt saadud varandus kahaneb, aga kes kogub vähehaaval, kasvatab seda.
\par 12 Pikk ootus teeb südame haigeks, aga täideläinud igatsus on elupuu.
\par 13 Kes põlgab sõna, teeb enesele kahju, aga kes kardab käsku, saab tasu.
\par 14 Targa õpetus on eluallikas surmapaeltest pääsemiseks.
\par 15 Peenetundelisus toob poolehoidu, aga hoolimatute tee on hukatus.
\par 16 Taibukas teeb kõike teadlikult, aga alp näitab oma rumalust.
\par 17 Õel käskjalg toob õnnetuse, aga ustav saadik toob tervise.
\par 18 Vaesus ja häbi on sellel, kes ei hooli õpetusest, aga kes noomimist tähele paneb, seda austatakse.
\par 19 Täideläinud igatsus on hingele armas, kurjast hoidumine on alpide meelest hirmus.
\par 20 Kes tarkadega läbi käib, saab targaks, aga kes alpidega seltsib, selle käsi käib halvasti.
\par 21 Õnnetus ajab taga patuseid, aga õiged saavad palgaks õnne.
\par 22 Hea inimene jätab pärandi lastelastelegi, aga patuse vara talletatakse õigele.
\par 23 Kehvade uudismaa annab palju leiba, aga mõni hukkub ülekohtu tõttu.
\par 24 Kes vitsa ei tarvita, vihkab oma poega, aga kes teda armastab, karistab teda aegsasti.
\par 25 Õige sööb nii palju, kui ta süda kutsub, aga õelate kõht on tühi.

\chapter{14}

\par 1 Tark naine ehitab enesele koja, aga rumalus kisub selle maha oma kätega.
\par 2 Ausasti elab, kes Issandat kardab, aga eksiteedel käib, kes teda põlgab.
\par 3 Rumala suus on vits ta uhkusele, aga tarku kaitsevad nende huuled.
\par 4 Kus pole härgi, seal puudub vili, aga härja rammuga saab palju saaki.
\par 5 Tõetruu tunnistaja ei valeta, aga valetunnistaja sepitseb valesid.
\par 6 Pilkaja otsib tarkust ilmaaegu, aga arukale on tunnetus kerge.
\par 7 Mine ära albi mehe juurest, sest seal sa ei märka teadlikke huuli.
\par 8 Mõistliku mehe tarkus on tunda oma teed, aga alpide rumalus eksitab.
\par 9 Rumalad pilkavad süüohvrit, aga õigete keskel on hea meel.
\par 10 Süda tunneb omaenese kibedust ja võõras ei saa segada tema rõõmu.
\par 11 Õelate koda hävitatakse, aga õigete telk haljendab.
\par 12 Mehe meelest on mõnigi tee õige, aga lõpuks on see surmatee.
\par 13 Naerdeski võib süda valutada ja rõõmule võib järgneda meelehärm.
\par 14 Truudusetu süda küllastub oma viisidest, aga hea mees oma tegudest.
\par 15 Lihtsameelne usub iga sõna, aga taiplik paneb tähele oma samme.
\par 16 Tark kardab ja hoidub kurjast, aga alp on jultunud ja enesekindel.
\par 17 Kes on äkilise vihaga, talitab meeletult, salasepitsejat meest vihatakse.
\par 18 Lihtsameelsed pärivad rumaluse, aga tarku kroonitakse teadlikkusega.
\par 19 Kurjad peavad kummardama heade ees ja õelad õigete väravais.
\par 20 Kehva vihkab tema sõbergi, aga rikast armastavad paljud.
\par 21 Kes põlgab oma ligimest, teeb pattu, aga kes halastab hädaliste peale, on õnnis.
\par 22 Eks eksi ju need, kes kavatsevad kurja? Aga heldus ja tõde on nendega, kes kavatsevad head.
\par 23 Igast vaevanägemisest on kasu, aga tühjast kõnest ainult kahju.
\par 24 Tarkade kroon on nende rikkus, alpide rumalus jääb rumaluseks.
\par 25 Tõetruu tunnistaja päästab hingi, aga kes sepitseb valesid, petab.
\par 26 Issanda kartuses on tugeva lootus ja varjupaik tema lastele.
\par 27 Issanda kartus on eluallikas surmapaeltest pääsemiseks.
\par 28 Rahva rohkus on kuninga uhkus, aga rahva vähesus on vürsti hukk.
\par 29 Pikameelsel on palju arukust, aga kannatamatu näitab suurimat rumalust.
\par 30 Südamerahu on ihule eluks, aga kadedus on otsekui mädanik luudes.
\par 31 Kes rõhub viletsat, teotab tema Loojat, aga kes halastab vaese peale, austab Loojat.
\par 32 Õel kukutatakse tema kurjuse pärast, aga õigel on surmaski varjupaik.
\par 33 Mõistliku südames hingab tarkus, aga alpide sees pole seda tunda.
\par 34 Õiglus ülendab rahvast, aga patt on teotuseks rahvahõimudele.
\par 35 Kuningal on hea meel targast sulasest, aga tema viha tabab häbiväärselt toimijat.

\chapter{15}

\par 1 Rahulik vastus vaigistab raevu, aga haavav sõna õhutab viha.
\par 2 Tarkade keel teeb tundmise heaks, aga alpide suu laseb voolata rumalust.
\par 3 Issanda silmad on igas paigas, valvates kurje ja häid.
\par 4 Keele mahedus on elupuu, aga selle valelikkus murrab vaimu.
\par 5 Meeletu laidab oma isa õpetust, aga kes noomimist tähele paneb, teeb targasti.
\par 6 Õige kojas on palju vara, aga õela saak jääb tarvitamata.
\par 7 Tarkade huuled külvavad teadmisi, aga alpide süda ei tee nõnda.
\par 8 Õelate ohver on Issandale jäle, aga õigete palve on temale meelepärane.
\par 9 Õela tee on Issanda meelest jäle, aga ta armastab õigusenõudjat.
\par 10 Rajalt lahkuja karistus on karm, noomimise vihkaja peab surema.
\par 11 Surmavald ja kadupaik on lahti Issanda ees, saati siis inimlaste südamed.
\par 12 Pilkaja ei armasta, et teda noomitakse, tarkade juurde ta ei lähe.
\par 13 Rõõmus süda teeb näo rõõmsaks, aga südamevalus pekstakse vaim rusuks.
\par 14 Mõistlik süda otsib tunnetust, aga alpide suu leiab toitu rumalusest.
\par 15 Viletsal on kõik päevad pahad, aga rõõmsal südamel on alati pidu.
\par 16 Parem pisut Issanda kartuses kui suur varandus ja rahutus selle juures.
\par 17 Parem taimetoit armastusega kui nuumhärg, mille juures on vihkamine.
\par 18 Raevutsev mees õhutab tüli, aga pikameelne vaigistab riidu.
\par 19 Laisa tee on nagu kibuvitsahekk, aga õigete rada on sillutatud.
\par 20 Tark poeg rõõmustab isa, aga alp inimene põlgab ema.
\par 21 Rumalus on rõõmuks sellele, kel puudub aru, aga arukas mees käib otse.
\par 22 Nõupidamiseta nurjuvad kavatsused, aga lähevad korda paljude nõuandjate abiga.
\par 23 Inimesel on rõõm, kui ta suu oskab vastata, ja sõna õigel ajal - küll see on hea.
\par 24 Targal läheb elurada ülespidi, et ta pääseks põrgust, mis on all.
\par 25 Issand kisub maha kõrkide koja, aga ta kinnitab lesknaise raja.
\par 26 Kurjad kavatsused on Issandale jäledad, aga heatahtlikud sõnad on puhtad.
\par 27 Kes ahnitseb kasu, jätab oma koja hooletusse, aga kes vihkab meelehead, jääb elama.
\par 28 Õige süda mõtleb, mida vastata, aga õelate suu purskab kurjust.
\par 29 Issand on õelatest kaugel, aga ta kuuleb õigete palveid.
\par 30 Silmade särast rõõmustab süda, hea sõnum kosutab luid-liikmeid.
\par 31 Kõrv, mis kuulab eluks vajalikku noomimist, jääb tarkade seltsi.
\par 32 Kes ei hooli õpetusest, põlgab oma hinge, aga kes kuulab noomimist, saab targa südame.
\par 33 Issanda kartus on tarkuse kool, aga enne au on alandus.

\chapter{16}

\par 1 Inimesel on küll südamesoovid, aga Issandalt tuleb, mida keel peab kostma.
\par 2 Kõik mehe teed on ta enese silmis selged, aga Issand katsub vaimud läbi.
\par 3 Veereta oma tööd Issanda peale, siis su kavatsused lähevad korda.
\par 4 Issand on kõik teinud otstarbekohaselt, nõnda ka õela õnnetuspäeva jaoks.
\par 5 Iga südamelt ülbe on Issanda meelest jäle, käsi selle peale - seesugune ei jää karistuseta!
\par 6 Armastuse ja ustavuse abiga lepitatakse süü, ja Issanda kartuse tõttu hoidutakse kurjast.
\par 7 Kui mehe teed meeldivad Issandale, siis ta paneb tema vaenlasedki temaga rahu tegema.
\par 8 Parem pisut õiglusega kui palju tulu ülekohtuga.
\par 9 Inimese süda kavandab oma teed, aga Issand juhib tema sammu.
\par 10 Otsus on kuninga huultel: kohut mõistes tema suu ei eksi.
\par 11 Õige margapuu ja vaekausid on Issanda päralt, kõik vihid kukrus on tema tehtud.
\par 12 Ülekohut teha on kuningate meelest jäle, sest aujärg kinnitatakse õigluses.
\par 13 Õiglased huuled on kuningale meelepärased ja ta armastab seda, kes õigust kõneleb.
\par 14 Kuninga viha on surma käskjalg, aga tark mees püüab seda lepitada.
\par 15 Kuninga lahke nägu tähendab elu ja tema poolehoid on otsekui kevadine vihmapilv.
\par 16 Kui palju parem on hankida tarkust kui kulda, ja arukuse hankimine on hõbedast olulisem.
\par 17 Õigete tee on hoidumine kurjast; oma hinge hoiab, kes oma teed tähele paneb.
\par 18 Uhkus on enne langust ja kõrkus enne komistust.
\par 19 Parem olla koos vaestega alandlik kui kõrkidega saaki jagada.
\par 20 Kes sõna tähele paneb, leiab õnne, ja kes loodab Issanda peale, on õnnis.
\par 21 Kel tark süda, seda nimetatakse mõistlikuks, ja mahedad huuled edendavad õpetust.
\par 22 Mõistus on omanikule eluallikaks, aga rumalus on rumalatele karistuseks.
\par 23 Targa süda teeb tema suu targaks ja edendab ta huultel õpetust.
\par 24 Sõbralikud sõnad on nagu kärjemesi, magusad hingele ja kosutuseks kontidele.
\par 25 Mehe meelest on mõnigi tee õige, aga lõpuks on see surmatee.
\par 26 Nälg aitab töömeest töös, sest ta oma suu sunnib teda.
\par 27 Kõlvatu mees kaevab hukatuseaugu ja tema huultel on otsekui kõrvetav tuli.
\par 28 Salakaval mees tõstab riidu ja keelepeksja lahutab sõbrad.
\par 29 Mees, kes kasutab vägivalda, ahvatleb oma ligimest ja viib ta teele, mis ei ole hea.
\par 30 Kes silmi pilgutab, see mõtleb riukaid; kes huuled kokku pigistab, sel on pahategu otsustatud.
\par 31 Hallid juuksed on ilus kroon, mis saavutatakse, kui ollakse õigluse teel.
\par 32 Pikameelne on parem kui kangelane, ja kes valitseb iseenese üle, on parem kui linna vallutaja.
\par 33 Liisku heidetakse kuuehõlmas, aga selle otsus on Issandalt.

\chapter{17}

\par 1 Parem kuiv paluke rahus kui kojatäis ohvriliha riiuga.
\par 2 Tark sulane valitseb häbitu perepoja üle ning jagab pärisosa nagu üks vendadest.
\par 3 Sulatuspott on hõbeda ja ahi kulla jaoks, aga Issand katsub südamed läbi.
\par 4 Kurjategija paneb tähele nurjatuid huuli, petis kuulab kurjakuulutavat keelt.
\par 5 Kes kehva pilkab, teotab tema Loojat, ei jää karistuseta, kes tunneb rõõmu teise õnnetusest.
\par 6 Lapselapsed on vanade kroon ja laste uhkuseks on nende vanemad.
\par 7 Selge jutt ei sobi rumalale, veel vähem siis vale kõne vürstile.
\par 8 Meelehea on andja silmis nagu võlukivi: kuhu ta iganes pöördub, seal on tal edu.
\par 9 Kes üleastumise kinni katab, otsib armastust, aga kes seda meelde tuletab, lahutab ennast sõbrast.
\par 10 Noomitus mõjub arukale rohkem kui sada hoopi albile.
\par 11 Õel inimene püüab aina vastu hakata, ent temale läkitatakse kallale julm käskjalg.
\par 12 Pigemini tulgu mehele vastu karu, kellelt pojad on röövitud, kui alp oma rumalusega.
\par 13 Kes head kurjaga tasub, selle kojast ei lahku õnnetus.
\par 14 Kes alustab tüli, päästab otsekui vee valla: seepärast jäta järele, enne kui puhkeb riid!
\par 15 Õela õigeks- ja õige hukkamõistja - need mõlemad on Issanda meelest jäledad.
\par 16 Mis kasu on rahast albi käes, kui tal ei ole soovi omandada tarkust?
\par 17 Tõeline sõber armastab igal ajal ja hädas tuleb ilmsiks, kes on vend.
\par 18 Arutu on inimene, kes kätt lööb, kes hakkab käendajaks oma ligimese ees.
\par 19 Kes armastab tüli, armastab üleastumist; kes teeb oma ukse kõrgeks, otsib hukatust.
\par 20 Kel valelik süda, ei leia õnne, ja kes oma keelega keerutab, langeb õnnetusse.
\par 21 Alp on meelehärmiks oma sünnitajale ja rumala isa ei tunne rõõmu.
\par 22 Rõõmus süda toob head tervenemist, aga rõhutud vaim kuivatab luudki.
\par 23 Õel võtab põuest vastu meelehead, et teha kõveraks õiguse teid.
\par 24 Arukal on tarkus näo ees, aga albi silmad sihivad maailma otsa.
\par 25 Alp poeg on kurvastuseks isale ja meelekibeduseks oma sünnitajale.
\par 26 Ei ole hea karistada õiget ega peksta õilsat tema õigluse pärast.
\par 27 Kes oma sõnu peatab, on arukas, ja mõistlik mees on külmavereline.
\par 28 Isegi rumalat peetakse targaks, kui ta vaikib, ja mõistlikuks, kui ta oma huuled kinni peab.

\chapter{18}

\par 1 Kes end eraldab, otsib, mida ta himustab; kõige jõuga tülitseb ta.
\par 2 Albil pole head meelt arukusest, küll aga oma arvamuste avaldamisest.
\par 3 Kui tuleb ülekohtune, tuleb ka põlgus ja koos häbiga teotus.
\par 4 Sõnad mehe suust on sügav vesi, tarkuseallikas on vulisev oja.
\par 5 Ei ole hea olla õela poolt, et õiget kohtus väärata.
\par 6 Albi huuled toovad riidu ja tema suu kutsub lööke.
\par 7 Albi suu on hukatuseks temale enesele ja ta huuled on püüdepaelaks ta hingele.
\par 8 Keelepeksja sõnad on nagu maiuspalad ja lähevad otse sisikonna soppidesse.
\par 9 Isegi see, kes oma töös on loid, on hävitaja vend.
\par 10 Issanda nimi on tugev torn: sinna jookseb õige ja leiab varju.
\par 11 Rikka varandus on tema tugev linn ja ta enese kujutluses kõrge müüri sarnane.
\par 12 Enne langust läheb inimese süda ülbeks, aga Enne au on alandus.
\par 13 Kui keegi vastab enne ärakuulamist, siis on see tema rumalus ja häbi.
\par 14 Mehine meel talub haigust, aga kes võiks kannatada rusutud vaimu?
\par 15 Mõistliku süda hangib teadlikkust ja tarkade kõrv otsib teadmisi.
\par 16 And avab inimesele tee ja viib teda suurte ette.
\par 17 Oma riiuasjas on esimesel õigus, kuni tuleb teine ja teda läbi katsub.
\par 18 Liisk lõpetab tülid ja otsustab vägevate vahel.
\par 19 Petetud venda on raskem võita kui tugevat linna, ja tülid on otsekui kindluse riiv.
\par 20 Oma suu viljast saab mehe kõht täis, ta küllastub oma huulte saagist.
\par 21 Surm ja elu on keele võimuses, ja kes seda armastab, saab süüa selle vilja.
\par 22 Kes leiab naise, leiab õnne ja saab Issanda hea meele osaliseks.
\par 23 Kehv kõneleb anudes, aga rikas vastab karmilt.
\par 24 On sõpru, kes üksteist maha löövad, aga mõni sõber on ustavam kui vend.

\chapter{19}

\par 1 Parem vaene, kes elab vagaduses, kui see, kel valelikud huuled ja ise on alp.
\par 2 Arutu agarus pole hea, ja kelle jalgadel on kiire, eksib ära.
\par 3 Inimese teed eksitab tema oma rumalus, aga ta süda kibestub Issanda vastu.
\par 4 Varandus toob palju sõpru, aga vaene peab sõbrast lahkuma.
\par 5 Valetunnistaja ei jää karistuseta, ja kes sepitseb valesid, ei pääse.
\par 6 Paljud lipitsevad ülemuse ees ja andja sõbrad on kõik.
\par 7 Vaest vihkavad kõik ta vennad, veel enam hoiduvad sõbrad temast eemale; ajab ta taga sõnu, siis neid ei ole.
\par 8 Kes hangib enesele tarkust, armastab oma hinge; kes hoiab arukust, leiab õnne.
\par 9 Valetunnistaja ei jää karistuseta, ja kes sepitseb valesid, hukkub.
\par 10 Ei sobi albil nautida elu, veel vähem siis sulasel valitseda vürstide üle.
\par 11 Arukus teeb inimese pikameelseks ja temale on auks üleastumine andeks anda.
\par 12 Kuninga viha on nagu noore lõvi möirgamine, aga tema lembus on otsekui kaste rohu peal.
\par 13 Alp poeg on oma isale õnnetuseks, ja naise riidlemine on otsekui katuse alaline läbitilkumine.
\par 14 Koda ja vara on isade pärand, aga arukas naine on Issandalt.
\par 15 Laiskus langetab sügavasse unne ja hooletu hing näeb nälga.
\par 16 Kes peab käsku, hoiab oma hinge; kes sellest ei hooli, peab surema.
\par 17 Kes halastab kehva peale, laenab Issandale ja tema tasub talle ta heateo.
\par 18 Karista oma poega, kuni veel on lootust, ja ära soovi tema surma!
\par 19 Kelle viha on suur, peab kandma karistust, sest kui sa ta sellest vabastad, siis tuleb sul teha seda korduvalt.
\par 20 Kuule nõu ja võta õpetust, et sa tulevikus oleksid targem!
\par 21 Inimsüdames on palju kavatsusi, aga Issanda nõu saab teoks.
\par 22 Inimesele on kasuks ta heldus, ja parem on olla vaene kui valelik mees.
\par 23 Issanda kartus on eluks: siis võib magada rahulikult, ilma et õnnetus tabaks.
\par 24 Laisk pistab käe kaussi, aga ei vii tagasi suu juurde.
\par 25 Peksa pilkajat, siis saab kohtlane targaks, ja manitse mõistlikku, siis ta saab teadlikuks!
\par 26 Kes kohtleb halvasti oma isa, ajab ära oma ema, see on häbitu ja häbematu poeg.
\par 27 Loobu, mu poeg, õpetust kuulmast, kui tahad eemale eksida tarkuse sõnadest!
\par 28 Kõlvatu tunnistaja pilkab õigust ja õelate suu ajab välja nurjatust.
\par 29 Pilkajate jaoks on valmis kohtuotsused ja alpide seljale hoobid.

\chapter{20}

\par 1 Vein paneb pilkama, vägijook lärmama, ja ükski, keda see paneb taaruma, pole tark.
\par 2 Kuninga ähvardus on otsekui noore lõvi möirgamine: kes teda vihastab, teeb pattu oma hinge vastu.
\par 3 Mehele on auks hoiduda riiust, aga kõik meeletud hakkavad peale.
\par 4 Laisk ei künna sügisel: lõikusajal ta ootab asjata.
\par 5 Nõu on mehe südames nagu sügav vesi, aga arukas mees oskab seda ammutada.
\par 6 Paljud inimesed kuulutavad omaenese headust, aga kes leiaks ühe ustava mehe?
\par 7 Õige elab oma vagaduses, õnnelikud on ta lapsed pärast teda.
\par 8 Kuningas, kes istub kohtujärjel, hajutab oma silmadega kõik kurja.
\par 9 Kes võiks öelda: „Ma olen oma südame puhtaks teinud, ma olen oma patust puhas”?
\par 10 Kahesugune kaaluviht ja kahesugune vakk - ka need mõlemad on Issanda meelest jäledad.
\par 11 Juba poisikese tegudest tuntakse, kas ta loomus on puhas ja õige.
\par 12 Kuulja kõrva ja nägija silma - ka need mõlemad on Issand teinud.
\par 13 Ära armasta und, et sa ei jääks vaeseks; hoia silmad lahti, siis saad leiba küllalt!
\par 14 „Halb, halb!” ütleb ostja, aga ära minnes ta kiitleb.
\par 15 Olgu kulda ja palju pärleid, aga teadlikud huuled on veel kallimad.
\par 16 Võta kuub sellelt, kes on hakanud käendajaks võõrale, muulaste pärast võta temalt pant!
\par 17 Valega saadud leib on mõnele magus, aga pärast on ta suu täis sõmerat.
\par 18 Nõu pidades lähevad kavatsused korda, ja ainult targal juhtimisel pea sõda.
\par 19 Kes käib keelekandjana, paljastab saladusi: seepärast ära seltsi lobisejaga!
\par 20 Kes neab oma isa ja ema, selle lamp kustub pilkases pimeduses.
\par 21 Pärand, millega esmalt on kiire, ei too lõpuks õnnistust.
\par 22 Ära ütle: „Ma tasun kätte kurja eest!” Oota Issandat, küll tema aitab sind!
\par 23 Kahesugune kaaluviht on Issanda meelest jäle, ja vale kaal ei ole hea.
\par 24 Mehe sammud on Issandalt: kuidas võiks siis inimene tunda oma teed?
\par 25 Inimesele on püüdepaelaks kergemeelne ütlus: „See on pühitsetud!” ja alles pärast tõotusi hakata järele mõtlema.
\par 26 Tark kuningas puistab õelaid ja laseb ratta neist üle käia.
\par 27 Inimese vaim on Issanda lamp: see uurib läbi kõik südamesopid.
\par 28 Armastus ja ustavus hoiavad kuningat, ja ta toetab helduse läbi oma aujärge.
\par 29 Noorte meeste uhkuseks on jõud, aga vanade ehteks on hallid juuksed.
\par 30 Haavaarmid on kurja kasimiseks ja hoobid ihu soppide jaoks.

\chapter{21}

\par 1 Kuninga süda on Issanda käes nagu veeojad: tema juhib seda, kuhu ta iganes tahab.
\par 2 Kõik mehe teed on ta enese silmis õiged, aga Issand katsub südamed läbi.
\par 3 Teha, mis õige ja kohus - see on Issandale olulisem kui ohver.
\par 4 Ülbed silmad ja hooplev süda, õelate lamp, on patt.
\par 5 Virga kavatsused toovad küll kasu, aga kõik ruttajad saavad ometi kahju.
\par 6 Valeliku keelega varanduste soetamine on surma otsijate hajuv tuulevine.
\par 7 Õelate vägivald viib ära neid endid, sest nad ei taha õigust teha.
\par 8 Süüdlase tee on kõver, aga süütu eluviis on õige.
\par 9 Parem on elada katusenurgas kui riiaka naisega ühises kojas.
\par 10 Õela hing ihaldab kurja, ligimene ei leia armu tema silmis.
\par 11 Kui pilkajat karistatakse, saab kohtlane targaks, ja kui tarka õpetatakse, võtab ta õpetust.
\par 12 Õige pidagu silmas õela koda: õelad tõugatakse õnnetusse.
\par 13 Kes suleb oma kõrva viletsa hädakisale, see peab ka ise karjuma vastust saamata.
\par 14 Salajane and vaigistab viha ja kingitus põue - tugeva raevu.
\par 15 Õigele on rõõmuks, kui tehakse õigust, aga kurjategijaile on see ehmatuseks.
\par 16 Inimene, kes eksib tarkuse teelt, peab minema puhkama surnute seltsi.
\par 17 Kes armastab lõbu, jääb vaeseks meheks; kes armastab veini ja õli, ei saa rikkaks.
\par 18 Õel jääb lunahinnaks õige eest ja ausate asemele jääb autu.
\par 19 Parem on elada kõrbemaal kui riiaka ja pahura naisega.
\par 20 Targa kodus on kallist vara ja õli, aga alp inimene neelab need.
\par 21 Kes taotleb õiglust ja headust, leiab elu, õiguse ja au.
\par 22 Tark tõuseb vägevate linna ja kisub maha kantsi, mille peale nad lootsid.
\par 23 Kes valvab oma suud ja keelt, hoiab oma hinge hädade eest.
\par 24 Pilkaja on selle ülbe upsaka nimi, kes teeb oma tegusid määratus ülbuses.
\par 25 Laisale on ta ihkamine surmaks, sest ta käed ei taha tööd teha.
\par 26 Päev otsa himustab ta aina saada, aga õiglane annab ega keela.
\par 27 Õelate ohver on jäledus, eriti kui seda tuuakse häbiteo eest.
\par 28 Valetunnistaja hukkub, aga mees, kes midagi on kuulnud, võib alati rääkida.
\par 29 Õel mees näitab jultunud nägu, aga õige kinnitab oma teid.
\par 30 Issanda vastu ei aita tarkus, mõistus ega nõu.
\par 31 Hobune seatakse valmis tapluse päevaks, aga võit on Issanda käes.

\chapter{22}

\par 1 Aus nimi on kallim kui suur rikkus, hea kuulsus on parem kui hõbe ja kuld.
\par 2 Rikas ja vaene kohtuvad, Issand on loonud need mõlemad.
\par 3 Tark näeb hädaohtu ja poeb peitu, aga rumalad lähevad edasi ja saavad nuhelda.
\par 4 Alandlikkuse ja Issanda kartuse tasu on rikkus, au ja elu.
\par 5 Okkad ja paelad on valeliku teel: kes jääb neist eemale, hoiab oma hinge.
\par 6 Juhata poiss ta tee peale, siis ta ei lahku sellelt ka vanas eas!
\par 7 Rikas valitseb vaeste üle ja laenaja on laenuandja ori.
\par 8 Kes külvab ülekohut, lõikab viletsust ja tema viha vits saab otsa.
\par 9 Kes on helde, seda õnnistatakse, sest ta annab oma leivast kehvale.
\par 10 Aja pilkaja ära, siis lakkab riid, lõpeb tüli ning teotus!
\par 11 Kes armastab südamepuhtust ja kelle huuled on armsad, selle sõber on kuningas.
\par 12 Issanda silmad valvavad tunnetust, aga ta kummutab petise sõnad.
\par 13 Laisk ütleb: „Väljas on lõvi. Tapab mind viimaks keset turgu.”
\par 14 Võõraste naiste suu on sügav haud: keda Issand on hukka mõistnud, see langeb sinna.
\par 15 Meeletus on seotud poisi südamesse, aga karistusvits saadab selle temast kaugele.
\par 16 Kes rõhub viletsat, et rikkaks saada, ja kes annab rikkale, sellele tuleb vaesus.
\par 17 Pööra kõrv ja kuule tarkade sõnu ja pangu su süda tähele minu tundmist:
\par 18 sest see on armas, kui sa neid eneses säilitad, kui need kõik su huultel on valmis.
\par 19 Et Issand võiks olla su lootus, selleks ma õpetan täna sind, just sind.
\par 20 Kas ma ei ole sulle kirjutanud kolm korda nõuandeid ja teadmisi,
\par 21 et õpetada sulle tõde, tõe sõnu, et võiksid vastata oma läkitajale tõe sõnu?
\par 22 Ära riisu viletsat, sellepärast et ta on vilets, ja ära rõhu vaest väravas,
\par 23 sest Issand lahendab nende asja ja röövib nende röövijailt hinge.
\par 24 Ära pea sõprust vihastujaga ja ära käi läbi raevutseva mehega,
\par 25 et sa ei harjuks tema viisidega ega valmistaks püüdepaela oma hingele.
\par 26 Ära ole nende seas, kes kätt löövad, kes hakkavad käendajaks võlgade eest!
\par 27 Kui sul ei ole, millega maksta, miks peaks võetama su voodi su alt?
\par 28 Ära nihuta igivana piirimärki, mille su esiisad on seadnud!
\par 29 Kui näed meest, kes oma töös on tubli, siis on ta koht kuningate, mitte alama rahva teenistuses.

\chapter{23}

\par 1 Kui sa istud valitsejaga leiba võtma, siis pane hästi tähele, mis sul ees on,
\par 2 ja pane nuga kõri juurde, kui sa oled himukas.
\par 3 Ära ihalda tema maiuspalu, sest see on petlik leib!
\par 4 Ära näe vaeva rikkaks saamiseks, ära kuluta selleks oma mõistust:
\par 5 kui sa pöörad oma pilgu selle peale, siis ei ole seda enam, sest rikkus saab enesele tiivad otsekui kotkas ja lendab ära taeva poole.
\par 6 Ära söö kadeda leiba ja ära ihalda tema maiuspalu,
\par 7 sest ta on selline, kes arvestab oma hinges: „Söö ja joo!” ütleb ta sulle, aga ta süda ei ole sinuga.
\par 8 Sa pead söödud palukese välja oksendama ja su ilusad sõnad osutuvad raisatuks.
\par 9 Ära räägi albi kuuldes, sest ta põlgab su mõistlikke sõnu!
\par 10 Ära nihuta igivana piirimärki ja ära mine vaeslaste põldudele,
\par 11 sest nende lunastaja on vägev, tema lahendab nende riiuasja sinu vastu!
\par 12 Pööra oma süda õpetuse ja kõrvad tarkussõnade juurde!
\par 13 Ära hoidu poissi karistamast: kui sa teda vitsaga peksad, siis tal ei tule surra!
\par 14 Sa peksad teda küll vitsaga, aga päästad tema hinge põrgust.
\par 15 Mu poeg! Kui su süda saab targaks, siis rõõmustab minugi süda
\par 16 ja mu neerud hõiskavad, kui su huuled räägivad õigust.
\par 17 Su süda ärgu kadestagu patuseid, vaid karda alati Issandat,
\par 18 sest siis on sul tõesti tulevik ja su lootus ei kao.
\par 19 Kuule, mu poeg, ja saa targaks ning juhi oma süda õigele teele!
\par 20 Ära viibi veinijoojate ega lihaõgijate killas,
\par 21 sest joodik ja õgija jäävad vaeseks ja uimane olek sunnib riietuma räbalaisse!
\par 22 Kuule oma isa, kes sind on sigitanud, ja ära põlga oma ema, kui ta on vanaks saanud!
\par 23 Osta tõtt ja ära seda müü, osta tarkust, õpetust ja arukust!
\par 24 Õiglase isa võib tõesti hõisata, kes targa on sünnitanud, võib temast rõõmu tunda.
\par 25 Olgu su isal ja emal rõõm, kes sinu on sünnitanud, hõisaku!
\par 26 Anna, mu poeg, oma süda mulle ja su silmad tundku head meelt mu teest!
\par 27 Sest hoor on sügav haud ja võõras naine on kitsas kaev.
\par 28 Jah, ta varitseb otsekui röövel ja rohkendab truudusetuid inimeste seas.
\par 29 Kellel on häda? Kellel on halb? Kellel on tüli? Kellel on kaebus? Kellel on haavad põhjuseta? Kellel on tuhmid silmad?
\par 30 Neil, kes viibivad veini juures, kes lähevad maitsma segatud veini.
\par 31 Ära vaata veini, kuidas see punetab, kuidas see karikas sädeleb, hõlpsasti sisse läheb:
\par 32 see salvab viimaks maona ja mürgitab otsekui rästik!
\par 33 Su silmad näevad siis imelikke asju ja su süda räägib pöörasusi.
\par 34 Sa nagu lamaksid keset merd ja magaksid masti tipus.
\par 35 „Mind löödi, aga ma ei saanud haiget, mind peksti, aga ma ei tundnudki! Millal ma ärkan? Ma tahan veelgi otsida sedasama.”

\chapter{24}

\par 1 Ära kadesta kurje inimesi ja ära ihalda olla nende juures,
\par 2 sest nende süda kavatseb vägivalda ja nende huuled räägivad paha!
\par 3 Tarkusega ehitatakse koda ja arukusega kinnitatakse,
\par 4 tunnetusega täidetakse kambrid igasuguse kalli ja kauni varaga.
\par 5 Tark mees on tugev ja teadja mees tugevdab oma jõudu.
\par 6 Sest targal juhtimisel sa võid pidada sõda, ja võit on seal, kus on rohkesti nõuandjaid.
\par 7 Tarkus on rumalale kõrge - ta ei tee väravas oma suud lahti.
\par 8 Kes kavatseb kurja, seda hüütakse salasepitsejaks.
\par 9 Meeletu tegu on patt ja pilkaja on inimeste meelest jäle.
\par 10 Kui oled hädaajal arg, siis on su jõud vähene.
\par 11 Päästa need, keda viiakse surma, ja peata, kes vanguvad tapmisele!
\par 12 Kui sa ütled: „Vaata, me ei teadnud seda”, kas siis südamete läbikatsuja ei saa sellest aru? Su hinge hoidja teab seda ja tasub inimesele ta tegu mööda.
\par 13 Söö, mu poeg, mett, sest see on hea, ja kärjemesi on su suulaele magus!
\par 14 Tea, et nõnda on ka tarkus su hingele: kui sa selle leiad, siis on sul tulevik ja su lootus ei kao!
\par 15 Ära varitse, õel, õige eluaset, ära hävita tema puhkepaika!
\par 16 Sest õige langeb seitse korda ja tõuseb üles, aga õelad komistavad õnnetusse.
\par 17 Ära tunne rõõmu oma vihamehe langusest ja ärgu hõisaku su süda, kui ta komistab,
\par 18 et see ei oleks paha Issanda silmis, kui ta seda näeb, ja et ta ei pööraks oma viha tema pealt!
\par 19 Ära ärritu kurjade pärast, ära kadesta õelaid,
\par 20 sest kurjal ei ole tulevikku, õelate lamp kustub!
\par 21 Mu poeg, karda Issandat ja kuningat, ära seltsi mässajatega, nendega, kes mõtlevad teisiti!
\par 22 Sest äkitselt kerkib nendelt hukatus ja kes teab nende mõlema õnnetust?
\par 23 Needki on tarkade sõnad: Erapoolikus kohtus ei ole hea.
\par 24 Kes ütleb õelale: „Sa oled õige”, seda neavad inimesed, sajatavad rahvahulgad;
\par 25 aga neil, kes noomivad, käib käsi hästi ning neile tuleb õnn ja õnnistus.
\par 26 Õige vastus on otsekui suudlus huultele.
\par 27 Tee enne oma tööd väljas ja hari oma põldu, siis alles ehita enesele koda!
\par 28 Ära ole põhjuseta tunnistajaks oma ligimese vastu, või tahad sa ometi petta oma huultega?
\par 29 Ära ütle: „Nõnda nagu tema tegi minule, nõnda teen mina temale, ma tasun mehele ta tegu mööda.”
\par 30 Ma läksin mööda laisa põllust ja arutu inimese viinamäest,
\par 31 ja vaata, see oli üleni kasvanud umbrohtu, selle pinda katsid nõgesed ja kiviaed oli maha kistud.
\par 32 Kui ma seda nägin, siis panin südamesse, vaatasin ja võtsin õpetust:
\par 33 veel pisut und, pisut tukkumist, pisut pikutamist ristis kätega,
\par 34 siis tuleb vaesus sulle kallale otsekui röövel ja puudus nagu relvastatud mees.

\chapter{25}

\par 1 Needki on Saalomoni õpetussõnad, mis Juuda kuninga Hiskija mehed on edasi andnud:
\par 2 Jumala au on asja salajas hoida, aga kuningate au on asja uurida.
\par 3 Taeva kõrgus, maa sügavus ja kuningate süda on uurimatud.
\par 4 Eralda räbu hõbedast, siis tuleb sellest riist hõbesepa käes!
\par 5 Eemalda õel kuninga eest, siis kinnitub ta aujärg õigluses!
\par 6 Ära tee ennast tähtsaks kuninga ees ja ära asu suurte kohale,
\par 7 sest on parem, kui sulle öeldakse: „Tule siia üles!”, kui et sind alandatakse ülema ees, keda su silmad on näinud.
\par 8 Ära mine kiiresti kohtu ette, sest mis sa siis viimaks teed, kui su ligimene sind häbistab?
\par 9 Lahenda oma riiuasi ligimesega, aga teise saladust ära ilmuta,
\par 10 et kuulja sind ei laimaks ja sulle ei jääks halba kuulsust!
\par 11 Õigel ajal räägitud sõnad on otsekui kuldõunad hõbevaagnail.
\par 12 Otsekui kuldrõngas või kuldehe on tark noomija kuuljale kõrvale.
\par 13 Otsekui lume külmus lõikusajal on ustav käskjalg oma läkitajale: ta jahutab oma isanda hinge.
\par 14 Otsekui pilved ja tuul, mis ei too vihma, on mees, kes kiitleb annist, mis on ainult pettus.
\par 15 Kannatlikkusega saab veenda valitsejat ja mahe keel murrab luud.
\par 16 Kui sa leiad mett, siis söö mõõdukalt, et sa sellest ei küllastuks ja seda välja ei oksendaks!
\par 17 Astugu su jalg harva su sõbra kotta, et ta sinust ei tüdineks ega hakkaks sind vihkama!
\par 18 Mees, kes valet tunnistab oma ligimese vastu, on otsekui vasar, mõõk ja terav nool.
\par 19 Otsekui haige hammas või lonkav jalg on usaldus kelmi vastu hädaajal.
\par 20 Otsekui kisuks külmal päeval kuue seljast, või nagu äädikas leelise peale, on laulude laulmine kurvale südamele.
\par 21 Kui su vihamehel on nälg, anna temale leiba süüa, ja kui tal on janu, anna temale vett juua,
\par 22 sest nõnda sa kogud tuliseid süsi tema pea peale ja Issand tasub sinule selle eest!
\par 23 Põhjatuul sünnitab saju ja salalik keel vihaseid nägusid.
\par 24 Parem on elada katusenurgas kui riiaka naisega ühises kojas.
\par 25 Otsekui külm vesi väsinud hingele on hea sõnum kaugelt maalt.
\par 26 Otsekui sogaseks tehtud allikas või rikutud kaev on õela ees vankuv õige.
\par 27 Palju mett süüa ei ole hea ega ole auväärne otsida oma au.
\par 28 Otsekui mahakistud linn, millel pole müüri, on mees, kes ei talitse oma meelt.

\chapter{26}

\par 1 Nagu lumi suvel ja vihm lõikusajal, nõnda ei sobi albile au.
\par 2 Nagu koduta lind, nagu lendav pääsuke, nõnda on teenimata needus: see ei lähe täide.
\par 3 Hobusele piits, eeslile valjad ja alpide seljale vits.
\par 4 Ära vasta albile ta rumalust eeskujuks võttes, et sinagi ei saaks tema sarnaseks!
\par 5 Vasta albile ta rumaluse peale, et ta ei hakkaks ennast targaks pidama!
\par 6 Kes albiga sõna saadab, raiub eneselt jalad ja peab jooma vägivalda.
\par 7 Nagu lonkuri jalgade kõikumine on õpetussõna alpide suus.
\par 8 Otsekui asetaks kivi lingule, nõnda on albile au andmine.
\par 9 Otsekui joobnu kätte sattunud orjavits on õpetussõna alpide suus.
\par 10 Nagu kütt, kes haavab kõiki, on see, kes palkab albi või palkab möödujaid.
\par 11 Nagu oma okse juurde tagasi pöörduv koer on oma rumalust kordav alp.
\par 12 Kui näed meest, kes iseenese silmis on tark, siis on albil enam lootust kui temal.
\par 13 Laisk ütleb: Noor lõvi on tee peal! Kiskja keset turgu!
\par 14 Uks pöörleb hingedel, laisk oma asemel.
\par 15 Laisk pistab käe kaussi, aga ei viitsi seda suu juurde viia.
\par 16 Laisk on iseenese meelest targem kui seitse arukalt vastajat.
\par 17 Otsekui koera kõrvust haarab kinni see, kes mööda minnes vihastab riiu pärast, millega temal ei ole tegemist.
\par 18 Nagu meeletu, kes ammub tuliseid surmanooli,
\par 19 on mees, kes petab oma ligimest ja ütleb: Ma teen ju ainult nalja.
\par 20 Puude puudusel kustub tuli, ja kui ei ole keelepeksjat, vaibub tüli.
\par 21 Mida söed hõõgusele ja puud tulele, seda on riiakas mees tüli õhutamisel.
\par 22 Keelepeksja sõnad on nagu maiuspalad ja need lähevad otse sisikonna soppidesse.
\par 23 Põlevad huuled ja kuri süda on nagu hõbevaabaga kaetud potitükk.
\par 24 Vihkaja teeskleb huultega, aga südames ta haub pettust:
\par 25 kui ta teeb oma hääle mahedaks, siis ära usu teda, sest tal on seitse jäledust südames.
\par 26 Vihkamine katab ennast kavalasti, aga koguduses tuleb ta kurjus ilmsiks.
\par 27 Kes kaevab augu, langeb ise sinna sisse, ja kes paneb kivi veerema, selle peale see veereb tagasi.
\par 28 Valelik keel vihkab oma ohvreid, ja meelitaja suu valmistab hukatuse.

\chapter{27}

\par 1 Ära kiitle homsest päevast, sest sa ei tea, mida see päev toob!
\par 2 Kiitku sind keegi teine, aga mitte su oma suu, keegi võõras, aga mitte su oma huuled.
\par 3 Kivil on raskus ja liival kaal, aga arulageda viha on rängem neist mõlemast.
\par 4 Raev võib olla julm ja viha otsekui uputus, aga kes suudaks seista armukadeduse ees?
\par 5 Parem avalik noomitus kui salalik armastus.
\par 6 Sõbra löögid on mõeldud siiralt, aga vihamehe suudlused on võltsid.
\par 7 Kelle kõht on täis, see tallab kärjemee, aga näljasele on kõik kibegi magus.
\par 8 Nagu pesast pagev lind on mees, kes põgeneb oma kodukohast.
\par 9 Õli ja suitsutusrohi rõõmustavad südant, aga sõbra magusus tuleb südamlikust nõuandest.
\par 10 Ära hülga oma sõpra ja oma isa sõpra; ära mine oma venna kotta, kui sul on viletsuseaeg; parem naaber ligidal kui vend kaugel!
\par 11 Ole tark, mu poeg, ja rõõmusta mu südant, et võiksin vastata sellele, kes mind teotab!
\par 12 Tark näeb hädaohtu ja poeb peitu, aga rumalad lähevad edasi ja saavad nuhelda.
\par 13 Võta kuub sellelt, kes on hakanud käendajaks võõrale, muulaste pärast võta temalt pant!
\par 14 Kes vara hommikul oma ligimest valju häälega õnnistab, sellele loetakse see sajatamiseks.
\par 15 Katuse alaline läbitilkumine saju ajal ja riiakas naine on ühesugused:
\par 16 teda varjata on otsekui varjaks tuult või kahmaks õli oma parema käega.
\par 17 Raud ihub rauda ja inimene ihub teist.
\par 18 Kes hoolitseb viigipuu eest, saab süüa selle vilja, ja kes teenib oma isandat, seda austatakse.
\par 19 Otsekui vees pale peegeldab palet, nõnda vastab inimsüda inimesele.
\par 20 Surmavald ja kadupaik ei saa iialgi täis, samuti ei küllastu ka inimese silmad.
\par 21 Hõbeda jaoks on sulatuspott ja kulla jaoks ahi, ja meest hinnatakse tema kuulsuse järgi.
\par 22 Isegi kui sa rumalat tambiksid nuiaga uhmris terade seas, ei lahkuks temast rumalus.
\par 23 Pane tähele oma lammaste seisukorda, tee karjad oma südameasjaks,
\par 24 sest varandus ei kesta igavesti, või kas kroongi püsib põlvest põlve?
\par 25 Kui rohi on kadunud ja ädal on tulnud nähtavale, kui mägedelt on hein korjatud,
\par 26 siis on sul tallesid riietuseks ja sikke ostuhinnaks põllu eest.
\par 27 Ja sul enesel on küllalt kitsepiima toiduks, toiduks su perele ja elatiseks su teenijaile.

\chapter{28}

\par 1 Õel põgeneb, kuigi ei ole tagaajajat, aga õige on julge nagu noor lõvi.
\par 2 Üleastumise ajal on maal palju valitsejaid, aga mõistliku ja targa mehe puhul püsib kindel kord kaua.
\par 3 Kehv mees, kes rõhub viletsaid, on otsekui vihm, mis uhub ega jäta leiba.
\par 4 Seaduse hülgajad kiidavad õelaid, aga Seaduse täitjad võitlevad nende vastu.
\par 5 Kurjad inimesed ei mõista, mis on õige, aga kes Issandat otsivad, mõistavad täiesti.
\par 6 Parem vaene, kes elab vagaduses, kui rikas, kelle teed on kõverad.
\par 7 Kes peab Seadust, on mõistlik poeg, aga kes seltsib kergemeelsetega, teeb oma isale häbi.
\par 8 Kes suurendab oma varandust rendi ja kasu läbi, kogub seda sellele, kes halastab vaeste peale.
\par 9 Kes pöörab oma kõrva ära Seadust kuulmast, selle palvegi on jäledus.
\par 10 Kes eksitab õigeid kurjale teele, langeb omaenese auku, aga vagad saavad hea pärisosa.
\par 11 Rikas mees on iseenese silmis tark, aga arusaaja vaene näeb teda läbi.
\par 12 Kui õiged hõiskavad, on kõik suurepärane, aga kui õelad tõusevad, peavad inimesed peitu pugema.
\par 13 Kes oma üleastumisi varjab, ei jõua sihile, aga kes neid tunnistab ja need hülgab, leiab armu.
\par 14 Õnnis on inimene, kes alati kardab Jumalat, aga kes teeb oma südame kõvaks, langeb õnnetusse.
\par 15 Otsekui möirgav lõvi või kallale kargav karu on vaese rahva õel valitseja.
\par 16 Vürst, kel puudub arukus, on suur rõhuja, aga kes vihkab ahnust, pikendab oma päevi.
\par 17 Inimene, keda rõhub veresüü, olgu põgenik kuni hauani, ükski ärgu aidaku teda!
\par 18 Kes elab laitmatult, päästetakse, aga kes käib kõveraid teid, ükskord langeb.
\par 19 Kes oma maad harib, saab külluses leiba, aga kes tühja taga ajab, saab külluses vaesust.
\par 20 Ustav mees saab suure õnnistuse, aga kes ruttab rikkaks saama, ei jää karistuseta.
\par 21 Ei ole hea olla erapoolik, ent leivapalukese pärast saab mõnigi üleastujaks.
\par 22 Kade mees tahab kiiresti saada rikkaks, aga ta ei tea, et temale tuleb kätte puudus.
\par 23 Kes teist inimest noomib, leiab lõpuks enam tänu kui libekeelne.
\par 24 Kes oma isalt ja emalt riisub ja ütleb: „See pole üleastumine”, on röövlite seltsimees.
\par 25 Ablas hing õhutab riidu, aga kes loodab Issanda peale, kosub.
\par 26 Kes loodab oma südame peale, on alp, aga kes käib tarkuses, pääseb.
\par 27 Kes annab kehvale, sellele ei tule puudust, aga kes oma silmad suleb, saab palju sajatusi.
\par 28 Kui õelad tõusevad, poevad inimesed peitu, aga kui nad hukkuvad, rohkeneb õigete hulk.

\chapter{29}

\par 1 Kes noomimisest hoolimata jääb kangekaelseks, murtakse äkitselt ja paranemist ei ole.
\par 2 Õigete rohkusest rahvas rõõmustab, aga õela valitsedes rahvas ägab.
\par 3 Tarkuse armastaja rõõmustab oma isa, aga kes hooradega seltsib, raiskab varanduse.
\par 4 Kuningas kindlustab maad õiguse abil, aga pistisevõtja laostab selle.
\par 5 Mees, kes oma ligimese vastu on libekeelne, laotab tema jalgadele võrgu.
\par 6 Kurja inimese üleastumised on talle püüniseks, aga õige hõiskab ja rõõmustab.
\par 7 Õiglane tunneb viletsate kohtuasja, aga õel ei taha sellest aru saada.
\par 8 Pilkajad kihutavad linna mässama, aga targad vaigistavad viha.
\par 9 Kui tark mees rumalaga kohut käib, siis too kas vihastab või naerab, aga rahu ei saa.
\par 10 Verejanulised vihkavad vaga, aga ausameelsed püüavad hoida tema hinge.
\par 11 Alp paiskab välja kogu oma viha, aga tark mees vaigistab teda lõpuks.
\par 12 Kui valitseja paneb tähele valekõnesid, siis on kõik ta teenrid õelad.
\par 13 Vaene ja rõhuja kohtavad teineteist, Issand valgustab nende mõlema silmi.
\par 14 Kuningal, kes viletsaile ausasti kohut mõistab, on aujärg igavesti kindel.
\par 15 Vits ja noomimine annavad tarkust, aga omapead jäetud poiss teeb häbi oma emale.
\par 16 Kui õelaid saab palju, siis on ka üleastumisi palju, aga õiged saavad näha nende hukkumist.
\par 17 Karista oma poega, siis on sul temast rahu ja ta rõõmustab su hinge!
\par 18 Kui nägemus puudub, muutub rahvas ohjeldamatuks, aga Seadust pidades on ta õnnis.
\par 19 Sõnadega ei saa sulast õpetada, sest kuigi ta mõistab, ei võta ta kuulda.
\par 20 Kas näed nobeda kõnega meest? Ennem võib loota albile kui temale.
\par 21 Kui sulast noorelt hellitatakse, siis on ta viimaks tüliks.
\par 22 Viha pidav inimene õhutab riidu ja raevutsejal on palju üleastumisi.
\par 23 Inimest alandab ta oma ülbus, aga kes on alandliku vaimuga, saab au.
\par 24 Kes vargaga jagab, vihkab oma hinge: ta kuuleb küll needmist, aga ei avalda midagi.
\par 25 Inimeste kartmine paneb püünise, aga kes loodab Issanda peale, on kaitstud.
\par 26 Paljud otsivad valitseja poolehoidu, aga Issandalt tuleb igaühele õigus.
\par 27 Õigete meelest on jäle ülekohtune mees, aga õelate meelest on jäle õigel teel käija.

\chapter{30}

\par 1 Massalase Aguri, Jake poja sõnad; nõnda ütleb see mees Iitielile, Iitielile ja Uukalile:
\par 2 „Kui ma olen ka rumalam meestest ja mul ei ole inimmõistust
\par 3 ja ma ei ole õppinud tarkust, aga pühade teadmist ma tean.
\par 4 Kes on läinud taevasse ja on sealt alla tulnud? Kes on kogunud tuule oma pihkudesse? Kes on mähkinud vee vaibasse? Kes on paigale pannud kõik maaääred? Mis on ta nimi ja mis on ta poja nimi? Küllap sa tead!
\par 5 Jumala iga sõna on selge, tema on kilbiks neile, kes otsivad kaitset temalt.
\par 6 Ära lisa midagi tema sõnadele, et ta ei võtaks sind vastutusele ja sina ei jääks valelikuks!
\par 7 Kaht asja ma palun sinult, ära neid mulle keela, enne kui ma suren:
\par 8 pettus ja valekõne hoia minust eemal, vaesust ega rikkust ära mulle anna, toida mind aga vajaliku leivaga,
\par 9 et ma küllastudes ei hakkaks salgama ega ütleks: „Kes on Issand?” või et ma vaeseks jäädes ei hakkaks varastama ega patustaks oma Jumala nime vastu.
\par 10 Ära laima sulast ta isandale, et ta ei kiruks sind ja sina ei jääks süüdlaseks!
\par 11 On neid, kes neavad oma isa ega õnnista oma ema.
\par 12 On neid, kes iseenese silmis on puhtad, kuid ei ole oma saastast puhtaks pestud.
\par 13 On neid - ah, kui ülbed on nende silmad ja kui kõrgele on tõstetud nende silmalaud.
\par 14 On neid, kelle hambad on mõõgad ja lõualuud noad, et süüa viletsaid maalt ja vaeseid inimeste hulgast.
\par 15 Verekaanil on kaks tütart: „Anna! Anna!„ Kolm asja on, mis ei saa täis, neli, mis ei ütle: ”Küllalt!”:
\par 16 surmavald, viljatu üsk, maa, mis ei saa küllaldaselt vett, ja tuli, mis iialgi ei ütle: „Aitab!”
\par 17 Silma, mis irvitab isa ja põlgab ema sõna kuulamist, peavad kaarnad jõe ääres välja nokkima ja kotka pojad sööma.
\par 18 Kolm asja on mulle väga imelised, jah, neli on, mida ma ei mõista:
\par 19 kotka tee taeva all, mao tee kalju peal, laeva tee keset merd ja mehe tee neitsi juurde.
\par 20 Niisugune on abielurikkuja naise tee: ta sööb ja pühib suu puhtaks ning ütleb: „Ma pole kurja teinud!”
\par 21 Kolme asja all väriseb maa, jah, nelja all, mida see ei suuda taluda:
\par 22 sulase all, kui ta saab kuningaks, ja jõleda all, kui tal on külluses leiba,
\par 23 vihatud naise all, kui ta saab mehele, ja teenija all, kui ta kõrvale tõrjub oma emanda.
\par 24 Need neli on küll kõige pisemad maa peal, aga ometi on nad targemad kui targad:
\par 25 sipelgad on väeti hulk, ometi valmistavad nad suvel oma leiva;
\par 26 kaljumägrad on jõuetu hulk, ometi teevad nad oma kodud kaljusse;
\par 27 rohutirtsudel ei ole kuningat, ometi lähevad nad kõik rühmadena välja;
\par 28 sisalikku võib püüda kätega, ometi leidub teda kuninga paleedes.
\par 29 Neid on kolm, kelle astumine on rühikas, jah, neljal on rühikas käik:
\par 30 lõvi, kangelane loomade seas, kes ei tagane kellegi eest;
\par 31 uhkeldav kukk või sikk, ja kuningas, kes sammub oma rahva ees.
\par 32 Kui sa oled hoobeldes rumal olnud, siis mõtle järele, käsi suu peal!
\par 33 Sest piima kirnumisest saab võid, nina nuuskamisest tuleb verd ja viha õhutamisest puhkeb riid.”

\chapter{31}

\par 1 Lemueli, Massa kuninga sõnad, millega ta ema teda õpetas:
\par 2 „Mis oleks mul sulle öelda, mu poeg, mu üsa poeg, mu tõotuste poeg?
\par 3 Ära anna oma rammu naistele, oma teid nende hooleks, kes hävitavad kuningaid!
\par 4 Ei sünni kuningail, Lemuel, ei sünni kuningail juua veini ega vürstidel himustada vägijooki,
\par 5 et nad juues ei unustaks seadust ega väänaks kõigi vaeste õigust.
\par 6 Andke vägijooki norutajale ja veini sellele, kelle hing on kibestunud,
\par 7 et ta jooks ja unustaks oma vaesuse ega mõtleks enam oma vaevale!
\par 8 Ava oma suu keeletu heaks, õiguse tegemiseks kõigile põlatuile!
\par 9 Ava oma suu, mõista õiglast kohut, tee õigust viletsale ja vaesele!
\par 10 Tubli naine on palju enam väärt kui pärlid. Kes leiab tema?
\par 11 Ta mehe süda loodab tema peale ja tulust ei ole tal puudust.
\par 12 Kogu oma eluaja teeb ta mehele head ja mitte kurja.
\par 13 Ta muretseb villu ja linu ning töötab virkade kätega.
\par 14 Ta on kaupmehe laevade sarnane: ta toob oma leiva kaugelt.
\par 15 Ta tõuseb, kui on alles öö, ja annab oma perele rooga ja määratud osa oma teenijaile.
\par 16 Ta soovib põldu ja hangib selle, oma käte viljast ta istutab viinamäe.
\par 17 Ta paneb enesele vöö kõvasti vööle ja teeb oma käsivarred tugevaks.
\par 18 Ta märkab, et ta tulemused on head: ei kustu öösel ta lamp.
\par 19 Ta paneb oma käed koonlapuu külge ja ta pihud hoiavad kedervart.
\par 20 Ta avab oma pihu viletsale ja sirutab vaestele mõlemad käed.
\par 21 Ei ta karda lund oma pere pärast, sest kogu ta perel on kahekordsed riided.
\par 22 Ta valmistab enesele vaipu, ta riietus on linane ja purpurpunane.
\par 23 Ta mees on tuntud väravais, kui ta istub maa vanemate hulgas.
\par 24 Ta valmistab ja müüb särke ning annab kaupmeestele vöösid.
\par 25 Ta riided on tugevad ja ilusad ja ta vaatab rõõmsalt tulevikku.
\par 26 Ta avab oma suu targasti ja tema keelel on sõbralik õpetus.
\par 27 Ta valvab tegevust kojas ega söö laiskuse leiba.
\par 28 Ta pojad tõusevad ja nimetavad teda õnnelikuks, ja ta mees ülistab teda:
\par 29 „Palju on tütarlapsi, kes töös on tublid, aga sina ületad nad kõik!”
\par 30 Võluvus on petlik ja ilu on tühine, aga naine, kes Issandat kardab, on kiiduväärt.
\par 31 Andke temale ta käte vilja ja tema teod ülistagu teda väravais!”



\end{document}