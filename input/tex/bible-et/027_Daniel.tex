\begin{document}

\title{Taaniel}

\chapter{1}

\par 1 Juuda kuninga Joojakimi kolmandal valitsuseaastal tuli Paabeli kuningas Nebukadnetsar Jeruusalemma alla ja piiras seda.
\par 2 Ja Issand andis tema kätte Juuda kuninga Joojakimi ja osa Jumala koja riistu, ja ta viis need Sinearimaale oma jumalakotta; ta viis riistad oma jumala varakambrisse.
\par 3 Ja kuningas käskis Aspenast, oma kammerteenrite ülemat, et ta tooks Iisraeli lastest, niihästi kuninglikust soost kui suursuguste hulgast,
\par 4 noori mehi, kellel ei oleks ühtki kehalist viga ja kes oleksid ilusa välimusega, kes oleksid taibukad kõigis teadusis, targad ja arusaajad, ja kes oleksid kõlvulised teenima kuningakojas; ja neile pidi ta õpetama kaldea kirja ning keelt.
\par 5 Ja kuningas määras neile igapäevase osa kuninglikust roast ja viinast, mida ta ise jõi; ja nõnda pidi neid kasvatatama kolm aastat, et nad selle järele võiksid astuda kuninga teenistusse.
\par 6 Ja nende seas olid Juuda lastest Taaniel, Hananja, Miisael ja Asarja.
\par 7 Ja kammerteenrite ülem pani neile nimed: ta nimetas Taanieli Beltsassariks, Hananja Sadrakiks, Miisaeli Meesakiks ja Asarja Abednegoks.
\par 8 Aga Taaniel võttis südames ette ennast mitte roojastada kuninga roaga ja tema joodava viinaga, ja ta taotles kammerteenrite ülemalt, et tal ei oleks vaja ennast roojastada.
\par 9 Ja Jumal laskis Taanielil leida heldust ja armu kammerteenrite ülema ees.
\par 10 Aga kammerteenrite ülem ütles Taanielile: „Ma kardan, et mu isand kuningas, kes on määranud teie roa ja joogi, näeb siis teie palged olevat viletsama välimusega kui teistel teieealistel noortel meestel, ja nõnda te viite mu pea hädaohtu kuninga ees!”
\par 11 Siis ütles Taaniel ülevaatajale, kelle kammerteenrite ülem oli määranud Taanielile, Hananjale, Miisaelile ja Asarjale:
\par 12 „Tee ometi oma sulastega kümme päeva katset ja meile antagu taimetoitu süüa ja vett juua.
\par 13 Siis vaadatagu sinu juuresolekul meie välimust ja nende noorte meeste välimust, kes söövad kuninglikku rooga, ja talita siis oma sulastega, nagu sa heaks arvad!”
\par 14 Ja ta kuulas neid selles asjas ning katsetas nendega kümme päeva.
\par 15 Ja kümne päeva pärast näis nende välimus ilusam ja ihu lihavam kui ühelgi neist noortest meestest, kes sõid kuninglikku rooga.
\par 16 Siis ülevaataja võttis ära nende roa ja nende joodava viina ja andis neile taimetoitu.
\par 17 Ja Jumal andis neile neljale noorele mehele tarkust ja taipu igasuguses kirjas ja teaduses, ja Taaniel sai aru igasuguseist nägemusist ja unenägudest.
\par 18 Ja kui need aastad olid möödunud, mille järele kuningas oli käskinud neid ette tuua, tõi kammerteenrite ülem nad Nebukadnetsari ette.
\par 19 Ja kuningas rääkis nendega, ja nende kõigi hulgast ei leidunud Taanieli, Hananja, Miisaeli ja Asarja sarnast; nõnda tulid nad kuninga teenistusse.
\par 20 Ja kõigis teaduse ja tarkuse asjus, milles kuningas neid küsitles, leidis ta nad olevat kümme korda üle kõigist ennustajaist ja nõidadest, kes olid kogu ta kuningriigis.
\par 21 Ja Taaniel jäi sinna kuningas Koorese esimese aastani.

\chapter{2}

\par 1 Ja Nebukadnetsari teisel valitsuseaastal nägi Nebukadnetsar unenäo, ja ta vaim muutus rahutuks ning tal läks uni ära.
\par 2 Ja kuningas käskis kutsuda ennustajaid ja nõidu, posijaid ja Kaldea tarku, et need seletaksid kuningale ta unenäo; ja need tulid ning astusid kuninga ette.
\par 3 Ja kuningas ütles neile: „Ma nägin unenäo ja mu vaim on rahutu püüdes seda unenägu mõista!”
\par 4 Ja Kaldea targad rääkisid kuningale aramea keeli: „Kuningas elagu igavesti! Jutusta see unenägu oma sulastele, siis me anname sulle seletuse!”
\par 5 Kuningas kostis ja ütles Kaldea tarkadele: „Minu sõna on kindel: kui te ei tee mulle teatavaks unenägu ja selle tähendust, siis raiutakse teid tükkideks ja teie kojad tehakse rusuhunnikuks!
\par 6 Aga kui te teete mulle teatavaks unenäo ja selle tähenduse, siis saate minult ande ja kingitusi ning suurt au. Seepärast andke mulle teada unenägu ja selle tähendus!”
\par 7 Nad vastasid teist korda ja ütlesid: „Kuningas rääkigu oma sulastele, siis me anname sulle tähenduse teada!”
\par 8 Kuningas kostis ja ütles: „Ma mõistan tõesti, et te tahate aega võita, sest te näete, et minu sõna on kindel,
\par 9 et kui te ei tee unenägu mulle teatavaks, siis on teie kohta ainult see üks otsus. Te olete nõuks võtnud minu ees rääkida valelikke ja tühje sõnu, kuni ajad muutuvad. Seepärast öelge mulle unenägu, et ma teaksin, et te suudate ka selle tähenduse mulle teada anda!”
\par 10 Kaldea targad vastasid kuningale ja ütlesid: „Ei ole maa peal inimest, kes suudaks kuninga soovi rahuldada! Sest ükski suur ja vägev kuningas ei ole küsinud niisugust asja üheltki ennustajalt, nõialt või Kaldea targalt!
\par 11 Jah, asi, mida kuningas küsib, on raske ja ei ole muid, kes suudaksid seda kuningale seletada, kui jumalad, kelle eluase ei ole inimeste juures!”
\par 12 Selle peale kuningas ägestus ja vihastus väga ning käskis hukata kõik Paabeli targad.
\par 13 Kui käsk oli väljunud, et targad pidi tapetama, siis otsiti ka Taanieli ja tema seltsilisi, et neid tappa.
\par 14 Siis pöördus Taaniel targalt ja arukalt kuninga ihukaitsepealiku Arjoki poole, kes oli läinud Paabeli tarku tapma.
\par 15 Ta sõnas ja ütles Arjokile, kuninga pealikule: „Mispärast see vali käsk kuninga poolt?” Siis andis Arjok Taanielile asjaolu teada.
\par 16 Ja Taaniel läks sisse ning palus kuningat, et temale antaks aega tähenduse seletamiseks kuningale.
\par 17 Siis läks Taaniel oma kotta ja teatas asjaolust oma seltsilistele Hananjale, Miisaelile ja Asarjale,
\par 18 et need paluksid taeva Jumalalt halastust selle saladuse pärast, et Taaniel ja tema seltsilised ei hukkuks ühes teiste Paabeli tarkadega.
\par 19 Siis ilmutati saladus Taanielile öises nägemuses; ja Taaniel kiitis taeva Jumalat.
\par 20 Taaniel hakkas kõnelema ja ütles: „Jumala nimi olgu kiidetud igavikust igavikuni! Sest temal on tarkus ja võimus!
\par 21 Tema muudab aegu ja aastaid, tema tagandab kuningaid ja tõstab kuningaid; tema annab tarkadele tarkust ja mõistlikele mõistust!
\par 22 Tema paljastab sügavused ja saladused, tema teab, mis on pimeduses, ja tema juures elab valgus!
\par 23 Sind, mu vanemate Jumalat, kiidan ja ülistan mina, et sa mulle oled andnud tarkust ja väge ja et sa nüüd oled teinud mulle teatavaks, mida me sinult oleme palunud - oled kuninga asja teinud meile teatavaks!”
\par 24 Seejärel Taaniel läks Arjoki juurde, kelle kuningas oli pannud Paabeli tarku hukkama; ta tuli ja ütles temale nõnda: „Ära hukka Paabeli tarku; vii mind kuninga ette ja ma seletan kuningale tähenduse!”
\par 25 Siis Arjok viis Taanieli kiiresti kuninga ette ja ütles temale nõnda: „Ma olen leidnud Juuda vangide hulgast mehe, kes tahab kuningale teatavaks teha unenäo tähenduse!”
\par 26 Kuningas kostis ja ütles Taanielile, kelle nimeks oli Beltsassar: „Kas sina suudad mulle avaldada mu nähtud unenäo ja selle tähenduse?”
\par 27 Taaniel vastas kuningale ja ütles: „Saladust, mille kohta kuningas küsib, ei suuda kuningale seletada ei targad, nõiad, ennustajad ega täheteadlased!
\par 28 Aga taevas on Jumal, kes paljastab saladusi, ja tema ilmutab kuningas Nebukadnetsarile, mis tulevasil päevil sünnib. Su unenägu ja nägemused su peas, kui sa olid voodis, olid need:
\par 29 sinul, kuningas, tõusid voodis olles mõtted selle kohta, mis tulevikus sünnib, ja tema, kes paljastab saladusi, ilmutas sulle, mis sünnib.
\par 30 Ja minule on see saladus paljastatud mitte minu tarkuse läbi, kui mul olekski seda rohkem kui muil elavail, vaid seepärast, et selle tähendus saaks kuningale teatavaks ja et sa mõistaksid oma südame mõtteid.
\par 31 Sina, kuningas, nägid, ja vaata, oli üks suur kuju! See kuju oli suur ja selle hiilgus erakordne; see seisis su ees ja selle välimus oli hirmus!
\par 32 Kuju pea oli puhtast kullast, rind ja käsivarred hõbedast, kõht ja reied vasest,
\par 33 sääred rauast, jalad osalt rauast, osalt savist.
\par 34 Sa vaatasid, kuni üks kivi käte abita lahti murdus ja lõi kujule vastu jalgu, mis olid rauast ja savist, ja purustas need.
\par 35 Siis purunesid üheskoos raud, savi, vask, hõbe ja kuld ja said aganate sarnaseks suviselt rehealuselt: tuul viis need ära ja neist ei leidunud jälgegi. Aga kivist, mis oli tabanud kuju, sai suur mägi ja see täitis kogu maa.
\par 36 See oli unenägu, ja nüüd ütleme kuningale selle tähenduse:
\par 37 sina, kuningas, oled kuningate kuningas, kellele taeva Jumal on andnud kuningriigi, võimu, vägevuse ja au,
\par 38 ja kelle kätte ta on andnud inimesed, kus need ka iganes elavad, metselajad ja taeva linnud, pannes sind valitsema nende kõigi üle - sina oled see kullast pea!
\par 39 Aga sinu järele tõuseb teine kuningriik, väiksem kui sinu oma, ja veel kolmas kuningriik, vasest, mis valitseb kogu maa üle.
\par 40 Siis tuleb neljas kuningriik, tugev nagu raud - kuna ju raud kõike lõhub ja pihustab; ja nagu purustav raud, nii lõhub ja purustab ta need kõik!
\par 41 Aga et sa nägid jalad ja varbad olevat osalt potissepa savist ja osalt rauast, siis tähendab see jagatud kuningriiki; küll on selles raua tugevust, just nagu sa nägid rauda savimullaga segatult.
\par 42 Ja et jalgade varbad olid osalt rauast ja osalt savist, tähendab seda, et osa kuningriiki on tugev ja osa on rabe.
\par 43 Et sa nägid rauda savimullaga segatult, tähendab seda, et kuigi nad segunevad inimseemne poolest, nad ei püsi üksteise küljes, nagu raud ei segune saviga.
\par 44 Aga nende kuningate päevil püstitab taeva Jumal kuningriigi, mis jääb igavesti hävitamatuks ja mille valitsust ei anta teisele rahvale. See lõhub ja hävitab kõik need kuningriigid, aga ta ise püsib igavesti,
\par 45 just nagu sa nägid, et üks kivi käte abita mäest lahti murdus ja purustas raua, vase, savi, hõbeda ja kulla! Suur Jumal on andnud kuningale teada, mis tulevikus sünnib. Ja unenägu on tõsi ning selle tähendus kindel!”
\par 46 Siis kuningas Nebukadnetsar langes silmili maha ja kummardas Taanieli ning käskis temale ohverdada roaohvrit ja suitsutusohvrit.
\par 47 Kuningas kostis ja ütles Taanielile: „See on tõsi, et teie Jumal on jumalate Jumal ja kuningate Issand ning saladuste paljastaja, sest sina suutsid selle saladuse paljastada!”
\par 48 Siis kuningas ülendas Taanieli ja andis temale palju suuri ande ning pani tema valitsejaks üle kogu Paabeli linna ja ülem-eestseisjaks kõigile Paabeli tarkadele.
\par 49 Ja Taaniel palus kuningat ja see seadis Sadraki, Meesaki ja Abednego valitsema Paabeli linna üle; aga Taaniel jäi kuningahoovi.

\chapter{3}

\par 1 Kuningas Nebukadnetsar laskis valmistada kuldkuju, mille kõrgus oli kuuskümmend küünart ja laius kuus küünart; ta laskis selle püstitada Duura orgu Paabeli linnas.
\par 2 Ja kuningas Nebukadnetsar läkitas koguma asehaldureid, ülemaid ja maavalitsejaid, nõunikke, varahoidjaid, kohtunikke, korravalvureid ja kõiki linna ametikandjaid, et need tuleksid kuju pühitsemisele, mille kuningas Nebukadnetsar oli lasknud püstitada.
\par 3 Siis kogunesid asehaldurid, ülemad ja maavalitsejad, nõunikud, varahoidjad, kohtunikud, korravalurid ja kõik linna ametikandjad kuju pühitsemisele, mille kuningas Nebukadnetsar oli lasknud püstitada, ja asetusid kuju ette, mille Nebukadnetsar oli lasknud püstitada.
\par 4 Ja hüüdja hüüdis valjusti: „Teid, rahvad, suguvõsad ja keeled, kästakse:
\par 5 niipea kui te kuulete sarvede, vilede, kannelde, harfide, naablite, torupillide ja igasugu mänguriistade häält, heitke maha ja kummardage kuldkuju, mille kuningas Nebukadnetsar on lasknud püstitada!
\par 6 Aga kes ei heida maha ega kummarda, see heidetakse otsekohe tulisesse ahju!”
\par 7 Seepärast kõik rahvad, suguvõsad ja keeled heitsid maha ja kummardasid kuldkuju, mille kuningas Nebukadnetsar oli lasknud püstitada, niipea kui kõik rahvad kuulsid sarvede, vilede, kannelde, harfide, naablite ja igasugu mänguriistade häält.
\par 8 Seepeale astus otsekohe esile kaldea mehi ja need süüdistasid juute.
\par 9 Nad hakkasid rääkima ja ütlesid kuningas Nebukadnetsarile: „Kuningas elagu igavesti!
\par 10 Sina, kuningas oled andnud käsu, et iga inimene, kes kuuleb sarvede, vilede, kannelde, harfide, naablite, torupillide ja igasugu mänguriistade häält, peab heitma maha ja kummardama kuldkuju,
\par 11 aga kes ei heida maha ega kummarda, see heidetakse tulisesse ahju.
\par 12 On olemas juuda mehi, keda sa oled pannud valitsema Paabeli linna üle: Sadrak, Meesak ja Abednego. Need mehed ei hooli sinust, kuningas, nad ei teeni sinu jumalaid ega kummarda kuldkuju, mille sa oled lasknud püstitada!”
\par 13 Siis käskis Nebukadnetsar, vihas ja raevus, tuua Sadraki, Meesaki ja Abednego; ja need mehed toodi kuninga ette.
\par 14 Nebukadnetsar hakkas rääkima ja ütles neile: „Kas see on tõsi, Sadrak, Meesak ja Abednego, et teie ei teeni mu jumalaid ega kummarda kuldkuju, mille ma olen lasknud püstitada?
\par 15 Kui te nüüd olete valmis, otsekohe kui te kuulete sarvede, vilede, kannelde, harfide, naablite, torupillide ja igasugu mänguriistade häält, maha heitma ja kummardama kuju, mille ma olen lasknud valmistada, siis on see hea; aga kui te ei kummarda, siis heidetakse teid silmapilkselt tulisesse ahju! Ja kes oleks see jumal, kes teid päästaks minu käest!”
\par 16 Sadrak, Meesak ja Abednego kostsid ning ütlesid kuningale: „Nebukadnetsar! Selle peale pole meil tarvis sulle vastata sõnagi!
\par 17 Kui see peab olema, võib meie Jumal, keda me teenime, meid päästa: ta päästab meid tulisest ahjust ja sinu käest, oh kuningas!
\par 18 Aga kui mitte, siis olgu sul teada, kuningas, et meie ei teeni su jumalaid ega kummarda kuldkuju, mille sa oled lasknud püstitada!”
\par 19 Siis Nebukadnetsar täitus vihaga ja ta näojume muutus Sadraki, Meesaki ja Abednego pärast; ta kostis ja käskis ahju kütta seitse korda enam kui oli viisiks kütta.
\par 20 Ja ta käskis mõningaid tugevaid mehi oma sõjaväest siduda Sadraki, Meesaki ja Abednego, et heita need tulisesse ahju.
\par 21 Siis seoti need mehed oma kuubede, pükste, mütside ja muu riietusega tükkis ning heideti tulisesse ahju.
\par 22 Et kuninga käsk oli vali ja ahi oli ülemäära köetud, siis tappis tuleleek need mehed, kes Sadraki, Meesaki ja Abednego sinna viisid.
\par 23 Aga need kolm meest, Sadrak, Meesak ja Abednego, langesid seotuina tulisesse ahju.
\par 24 Siis kuningas Nebukadnetsar ehmus ja tõusis kähku üles, kostis ja ütles oma nõuandjaile: „Kas me ei heitnud kolm seotud meest tulle?” Nad kostsid ja ütlesid kuningale: „Tõepoolest, kuningas!”
\par 25 Tema kostis ja ütles: „Vaata, ma näen nelja meest vabalt tules käivat ja neil pole midagi viga, neljas aga on välimuselt jumalate poja samane!”
\par 26 Siis Nebukadnetsar läks tulise ahju ukse juurde, hakkas rääkima ja ütles: „Sadrak, Meesak ja Abednego, kõrgeima Jumala sulased, astuge välja ja tulge siia!” Siis tulid Sadrak, Meesak ja Abednego tulest välja.
\par 27 Ja asehaldurid, ülemad ja maavalitsejad ja kuninga nõuandjad kogunesid ning nägid, et tuli ei olnud saanud võimust nende meeste kehade üle; neil ei olnud juuksed peas kõrbenud, nende kuued ei olnud muutnud värvi ja neile ei olnud külge hakanud kõrbelõhna.
\par 28 Nebukadnetsar kostis ja ütles: „Kiidetud olgu Sadraki, Meesaki ja Abednego Jumal, kes läkitas oma ingli ja päästis oma sulased, kes lootsid tema peale ja astusid üle kuninga käsust, kes andsid pigemini oma ihud kui et teenida ja kummardada mõnd muud jumalat peale nende oma Jumala!
\par 29 Nüüd antakse minu poolt käsk, et igaüks, olgu mis tahes rahvast, suguvõsast ja keelest, kes kõneleb häbematult Sadraki, Meesaki ja Abednego Jumala kohta, raiutakse tükkideks ja tema koda tehakse rusuhunnikuks, sest ei ole muud jumalat, kes suudaks nõnda päästa!”
\par 30 Siis kuningas andis hea põlve Sadrakile, Meesakile ja Abednegole Paabeli linnas.
\section*{Nebukadnetsari hullumeelsus}

\par 31 „Kuningas Nebukadnetsar, kõigile rahvastele, suguvõsadele ja keeltele, kes elavad kogu maal: teie rahu olgu suur!
\par 32 Mul on hea meel teha teatavaks tunnustähti ja imetegusid, mis kõrgeim Jumal mulle on teinud!
\par 33 Kui suured on tema tunnustähed ja kui vägevad on tema imeteod! Tema kuningriik on igavene kuningriik ja tema valitsus on põlvest põlve!

\chapter{4}

\par 1 Mina, Nebukadnetsar, olin muretu oma kojas ja õnnelik oma palees.
\par 2 Ma nägin und ja see kohutas mind; mu voodis-oleku kujutlused ja peas sündinud nägemused tegid mulle hirmu.
\par 3 Ja ma andsin käsu tuua mu ette kõik Paabeli targad, et need seletaksid mulle unenäo tähenduse.
\par 4 Siis tulid ennustajad, nõiad, Kaldea targad ja täheteadlased, ja ma jutustasin neile unenäo, aga nad ei suutnud mulle seletada selle tähendust.
\par 5 Aga viimaks tuli mu ette Taaniel, kelle nimi on Beltsassar minu jumala nime järgi, ja kelles on pühade jumalate vaim, ja ma rääkisin temale unenäo:
\par 6 Beltsassar, ennustajate ülem, ma tean, et sinus on pühade jumalate vaim ja et ükski saladus ei tee sulle raskust! Siin on mu unenäo nägemused, mis ma nägin - ütle selle tähendus!
\par 7 Ja need on mu pea nägemused, mis mul voodis olles olid: ma vaatasin, ja ennäe, keset maad oli üks puu ja selle kõrgus oli suur!
\par 8 Puu kasvas ja muutus tugevaks, selle kõrgus ulatus taevani ja seda oli näha kogu maa ääreni.
\par 9 Sellel olid ilusad lehed ja palju vilja ning seal oli toidust kõigile; selle all oli varju loomadele ja selle okstel elasid taeva linnud ning kõik liha toitis ennast sellest.
\par 10 Ma nägin oma pea nägemusis, mis mul voodis olles olid, ja vaata, püha ingel astus taevast alla!
\par 11 Ta hüüdis valjusti ja ütles nõnda: „Raiuge puu maha ja laasige ta oksad, rabage temalt lehed ja puistake vili laiali, siis põgenevad loomad ta alt ja linnud okstelt!
\par 12 Aga tema juur jätke maa sisse, raud- ja vaskahelasse aasa rohu peale; teda kastetagu taeva kastega ja ühes loomadega olgu tal osa maa rohust.
\par 13 Tema inimsüda võetagu ja temale antagu looma süda, ja seitse aega käigu temast üle.
\par 14 See käsk oleneb inglite otsusest ja see asi pühade sõnast, selleks et elavad tunneksid, et Kõigekõrgem valitseb inimeste kuningriigi üle, annab selle kellele ta tahab, ja tõstab selle üle kõige alama inimese!”
\par 15 Selle unenäo nägin mina, kuningas Nebukadnetsar. Ja sina, Beltsassar, ütle, mida see tähendab, sest ükski mu kuningriigi tarkadest ei suuda mulle teatavaks teha selle tähendust. Sina aga suudad, sest sinus on pühade jumalate vaim!”
\par 16 Taaniel, kelle nimi oli Beltsassar, kohkus siis üheks silmapilguks ja ta mõtted ehmatasid teda. Kuningas rääkis ja ütles: „Beltsassar, ärgu unenägu ja selle tähendus sind ehmatagu!” Beltsassar kostis ja ütles: „Mu isand! Unenägu tabagu su vihkajaid ja selle tähendus su vaenlasi!
\par 17 Puu, mida sa nägid, mis kasvas ja sai tugevaks, mille kõrgus ulatus taevani ja mis oli nähtav kogu maal,
\par 18 millel olid ilusad lehed ja palju vilja ning kus oli toidust kõigile, mille all elasid välja loomad ja mille okstel asusid taeva linnud -
\par 19 see oled sina, kuningas, kes oled kasvanud ja saanud tugevaks; sinu suurus on kasvanud ja ulatub taevani ning sinu valitsus maailma ääreni!
\par 20 Ja et kuningas nägi püha ingli taevast alla astuvat ja ütlevat: „Raiuge puu maha ja hävitage see, aga jätke juur maa sisse, raud- ja vaskahelasse aasa rohu peale; seda kastetagu taeva kastega ja ühes välja loomadega olgu tal osa, kuni seitse aega on temast üle käinud,”
\par 21 selle tähendus, oh kuningas, ja Kõigekõrgema otsus, mis tabab mu isandat kuningat, on see:
\par 22 sind aetakse ära inimeste hulgast ja sul on eluase välja loomade juures; sulle antakse rohtu süüa nagu härgadele ja sind kastetakse taeva kastega, ja sinust käib üle seitse aega, kuni sa tunned, et Kõigekõrgem valitseb inimeste kuningriigi üle ja annab selle, kellele tahab!
\par 23 Ja et kästi puu juur alles jätta, see tähendab: su kuningriik jääb sinule, niipea kui sa mõistad, et taevas valitseb!
\par 24 Seepärast, oh kuningas, lase enesele meeldida mu nõu: vabasta ennast oma pattudest õigluse läbi, ja oma ülekohtutegudest, halastades viletsate peale, et su õnn võiks kesta!”
\par 25 Kõik see tabas kuningas Nebukadnetsarit.
\par 26 Kaheteistkümne kuu pärast, kui ta oli kõndimas Paabeli kuningliku palee katusel,
\par 27 kuningas hakkas rääkima ja ütles: „Eks see ole see suur Paabel, mille ma oma võimsa jõuga olen ehitanud kuninglikuks valitsuspaigaks ja oma väärikuse auks?”
\par 28 Sõna oli alles kuninga suus, kui taevast langes hääl: „Sinule, kuningas Nebukadnetsar, öeldakse: sinult võetakse kuningriik,
\par 29 sind aetakse ära inimeste juurest ja su asupaik on välja loomade juures; sulle antakse rohtu süüa nagu härgadele ja sinust käib üle seitse aega, kuni sa mõistad, et Kõigekõrgem valitseb inimeste kuningriigi üle ja annab selle, kellele tahab!”
\par 30 Selsamal silmapilgul sai see sõna tõeks Nebukadnetsari kohta ja ta aeti ära inimeste juurest; ta sõi rohtu nagu härjad ja ta ihu kasteti taeva kastega, kuni ta juuksed kasvasid pikaks nagu kotkasuled ja ta küüned olid nagu linnuküüned.
\par 31 „Aga pärast selle aja möödumist, mina, Nebukadnetsar, tõstsin oma silmad taeva poole ja mulle tuli mõistus tagasi; ja ma õnnistasin Kõigekõrgemat, kiitsin ja ülistasin teda, kes elab igavesti, kelle valitsus on igavene valitsus ja kelle kuningriik püsib põlvest põlve!
\par 32 Kõiki, kes elavad maa peal, ei tule panna mikski, ja tema talitab, nagu tahab, niihästi taeva väega kui maa elanikega ega ole seda, kes võiks lüüa tema käe peale ja öelda temale: „Mis sa teed?”
\par 33 Selsamal ajal tuli mu mõistus tagasi, ja mu toredus ja hiilgus tulid tagasi mu kuningriigi auks; mu ametikandjad ja suurnikud otsisid mind, ja mind pandi taas mu kuningriigi üle ja mulle anti veelgi suurem võim!
\par 34 Nüüd mina, Nebukadnetsar, kiidan ja ülistan ja austan taeva kuningat, sest kõik tema teod on tõde ja tema teed on õiged! Tema võib alandada neid, kes käivad kõrkuses!”

\chapter{5}

\par 1 Kuningas Belsassar tegi suure peo oma tuhandele suurnikule ja jõi viina selle tuhande ees.
\par 2 Viina mõjul käskis Belsassar tuua kuld- ja hõberiistad, mis tema isa Nebukadnetsar Jeruusalemma templist oli toonud, et kuningas ja tema suurnikud, tema naised ja liignaised, saaksid neist juua.
\par 3 Siis toodi need kuldriistad, mis olid toodud templist, Jumala kojast Jeruusalemmast, ja kuningas ning tema suurnikud, tema naised ja liignaised jõid nende seest.
\par 4 Nad jõid viina ja ülistasid kuld-, hõbe-, vask-, raud-, puu- ja kivijumalaid.
\par 5 Selsamal tunnil ilmusid inimkäe sõrmed ja kirjutasid kuninga palee lubjatud seinale, küünlajala kohale, ja kuningas nägi kirjutavat kätt.
\par 6 Siis kahvatas kuninga näojume ja ta mõtted kohutasid teda; ta niudeliigesed lõdisesid ja ta põlved peksid teineteise vastu!
\par 7 Kuningas hüüdis valjusti, et toodaks nõiad, Kaldea targad ja täheteadlased. Kuningas hakkas rääkima ja ütles Paabeli tarkadele: „Kes iganes seda kirja oskab lugeda ja seletab mulle selle tähenduse, see riietatakse purpurisse, temale pannakse kuldkee kaela ja ta saab kolmandaks valitsejaks kuningriigis!”
\par 8 Siis tulid kõik targad, aga need ei osanud kirja lugeda ega selle seletust kuningale teatavaks teha.
\par 9 Siis kohkus kuningas Belsassar väga ja ta näojume kahvatas, ja ta suurnikud sattusid segadusse.
\par 10 Kuninga ja tema suurnike sõnade peale tuli kuninga ema pidusaali. Kuninga ema hakkas rääkima ja ütles: „Kuningas elagu igavesti! Ärgu kohutagu sind su mõtted ja ärgu kahvatagu su näojume!
\par 11 Sinu kuningriigis on mees, kelles on pühade jumalate vaim” Su isa päevil leidus temal arusaamist, taipu ja tarkust, otsekui jumalate tarkust, ja kuningas Nebukadnetsar, su isa, tõstis tema ennustajate, nõidade, Kaldea tarkade ja täheteadlaste ülemaks - kuningas, su isa -
\par 12 sellepärast et temas, Taanielis, kellele kuningas oli pannud nimeks Beltsassar, leidus eriline vaim ja temal oli mõistust ning taipu unenägude seletamiseks, mõistatuste lahendamiseks ja sõlmede harutamiseks. Kutsutagu nüüd Taaniel, küll tema annab tähenduse teada!”
\par 13 Siis toodi Taaniel kuninga ette. Kuningas hakkas rääkima ja ütles Taanielile: „Kas sina oled Taaniel, Juuda vangide seltsist, keda kuningas, mu isa, Juudast tõi?
\par 14 Ma olen sinust kuulnud, et sinus on jumalate vaim, ja et sinul leidub arusaamist ja taipu ning erilist tarkust.
\par 15 Ja nüüd on mu ette toodud targad ja nõiad seda kirja lugema ning selle tähendust mulle teatavaks tegema, aga nad ei ole suutnud sellele asjale seletust anda.
\par 16 Sinust ma olen aga kuulnud, et sina suudad seletusi anda ja sõlmi harutada. Kui sa nüüd suudad kirja lugeda ja selle tähenduse mulle teatavaks teha, siis riietatakse sind purpurisse, sulle pannakse kuldkee kaela ja sa saad kolmandaks valitsejaks kuningriigis!”
\par 17 Siis kostis Taaniel ja ütles kuninga ees: „Su annid jäägu sulle enesele ja oma kingitused anna mõnele teisele. Aga kirja ma loen kuningale ja teen selle tähenduse temale teatavaks!
\par 18 Sina, kuningas! Kõigekõrgem Jumal oli andnud su isale Nebukadnetsarile kuningriigi ja võimu, au ja hiilguse.
\par 19 Võimu pärast, mille ta temale oli andnud, värisesid ja kartsid tema ees kõik rahvad, suguvõsad ja keeled. Tema tappis, keda tahtis, ja jättis ellu, keda tahtis; ta ülendas, keda tahtis, ja alandas, keda tahtis.
\par 20 Aga kui ta süda suurustas ja ta vaim paadus ülemeelseks käitumiseks, siis tõugati ta oma kuninglikult aujärjelt ja ta au võeti temalt ära.
\par 21 Ta aeti ära inimlaste hulgast ja ta süda muutus looma südame sarnaseks; tema eluase oli metseeslite juures ja temale anti rohtu süüa nagu härgadele ja ta ihu kasteti taeva kastega, kuni ta mõistis, et Kõigekõrgem Jumal valitseb inimeste kuningriiki ja tõstab selle üle, keda tema tahab.
\par 22 Aga sina, tema poeg Belsassar, ei ole alandanud oma südant, kuigi sa teadsid kõike seda,
\par 23 vaid oled tõusnud taeva Issanda vastu: tema koja riistad on toodud su ette, ja sina ja su suurnikud, su naised ja su liignaised olete joonud nende seest viina; sa oled ülistanud hõbe-, kuld-, vask-, raud-, puu-, ja kivijumalaid, kes ei näe, ei kuule ega mõista! Aga seda Jumalat, kelle käes on su hing ja kelle omad on kõik su teed, sa ei ole austanud!
\par 24 Sellepärast on tema poolt läkitatud see käsi ja kirjutatud see kiri!
\par 25 Ja see on see kirjutatud kiri: Menee, menee, tekeel, ufarsiin!
\par 26 Sõnade tähendus on niisugune: menee - Jumal on ära lugenud su kuningriigi päevad ja on teinud sellele lõpu!
\par 27 Tekeel - sind on vaekaussidega vaetud ja leitud kerge olevat!
\par 28 Ufarsiin - ja su kuningriik on tükeldatud ning antud meedlastele ja pärslastele!”
\par 29 Siis Belsassar andis käsu ja Taaniel riietati purpurisse, temale pandi kuldkee kaela ja temast kuulutati, et ta hakkab kuningriigis valitsema kolmandana.
\par 30 Selsamal ööl tapeti Belsassar, Kaldea kuningas!

\chapter{6}

\par 1 Ja meedlane Daarjaves sai kuningriigi enesele, kui ta oli kuuskümmend kaks aastat vana.
\par 2 Daarjaves arvas heaks seada kuningriigi üle sada kakskümmend asehaldurit, et neid oleks kogu kuningriigis,
\par 3 ja nende üle kolm ametikandjat, kellest üks oli Taaniel, kellele need asehaldurid pidid aru andma, et kuningale ei sünniks kahju.
\par 4 Siis oli see Taaniel silmapaistvam kui teised ametikandjad ja asehaldurid, sellepärast et temas oli eriline vaim; ja kuningas kavatses tema tõsta üle kogu kuningriigi.
\par 5 Siis need ametikandjad ja asehaldurid otsisid ettekäänet Taanielile kuningriigivastase süü leidmiseks; aga nad ei suutnud leida ühtki ettekäänet ega midagi halba, sellepärast et ta oli ustav ja mingit hooletust ega halba tema kohta ei leidunud.
\par 6 Siis ütlesid need mehed: „Me ei leia selle Taanieli vastu ühtki ettekäänet, kui me seda ei leia ühenduses tema Jumala seadusega.”
\par 7 Siis need ametikandjad ja asehaldurid tormasid kuninga juurde ja ütlesid temale nõnda: „Kuningas Daarjaves elagu igavesti!
\par 8 Kõik kuningriigi ametikandjad, maavalitsejad ja asehaldurid, nõunikud ja maavanemad on pidanud nõu, et kuningas annaks korralduse ja jõustaks keelu, et igaüks, kes kolmekümne päeva jooksul palub midagi mõnelt jumalalt või inimeselt, aga mitte sinult, kuningas, visatakse lõukoerte auku!
\par 9 Nüüd, kuningas, avalda keeld ja kirjuta kiri, mida meedlaste ja pärslaste muutmatu seaduse tõttu ei tohi tühistada!”
\par 10 Seepeale kuningas Daarjaves kirjutas kirja ja keelu.
\par 11 Aga kui Taaniel sai teada, et kiri oli kirjutatud, siis ta läks oma kotta, mille ülakambri aknad olid avatud Jeruusalemma poole. Ja kolm korda päevas heitis ta põlvili, palvetas ja kiitis oma Jumalat, nagu ta seda ennegi oli teinud.
\par 12 Siis need mehed tormasid sisse ja leidsid Taanieli palvetamast ja anumast oma Jumala ees.
\par 13 Selle järele nad astusid kuninga ette ja küsisid kuninga keelu kohta: „Kas sa pole mitte kirjutanud keelu, et iga inimene, kes kolmekümne päeva jooksul palub midagi mõnelt jumalalt või inimeselt, aga mitte sinult, kuningas, visatakse lõukoerte auku?” Kuningas kostis ja ütles: „Asi on kindel meedlaste ja pärslaste muutmatu seaduse järgi!”
\par 14 Siis nad kostsid ja ütlesid kuninga ees: „Taaniel, kes on Juuda vangide seltsist, ei hooli sinust, kuningas, ega keelust, mille sa oled kirjutanud, vaid ta palvetab kolm korda päevas oma palvet!”
\par 15 Kui kuningas seda kuulis, siis oli see temale väga ebameeldiv ja ta oli mures Taanieli pärast, kuidas teda päästa; ja ta nägi vaeva tema päästmiseks kuni päikeseloojakuni.
\par 16 Siis need mehed tormasid kuninga juurde ja ütlesid kuningale: „Tea, kuningas, et meedlaste ja pärslaste seaduseks on, et ühtki kuninga antud keeldu või korraldust ei tohi muuta?”
\par 17 Siis kuningas andis käsu ja Taaniel toodi ning visati lõukoerte auku. Kuningas rääkis ja ütles Taanielile: „Sinu Jumal, keda sa lakkamata teenid, päästku sind!”
\par 18 Siis toodi üks kivi ja pandi augu suule, ja kuningas pitseeris selle oma pitserisõrmusega ja oma suurnike pitserisõrmustega, et Taanieli asjas ei oleks muutust.
\par 19 Siis läks kuningas oma paleesse ja veetis öö paastudes ega lasknud mänguriistu enese ette tuua, ja tal ei olnud und.
\par 20 Koiduajal, kui valgeks oli läinud, kuningas tõusis ja läks kiiresti lõukoerte augu juurde.
\par 21 Ja kui ta jõudis augu juurde, kus Taaniel oli, ta hüüdis kurva häälega; ja kuningas rääkis ning ütles Taanielile: „Taaniel, elava Jumala sulane! Kas su Jumal, keda sa lakkamata oled teeninud, on suutnud sind päästa lõukoerte küüsist?”
\par 22 Siis Taaniel kõneles kuningaga: „Kuningas elagu igavesti!
\par 23 Minu Jumal läkitas oma ingli ja sulges lõukoerte suud, ja need ei teinud mulle kurja, sellepärast et mind leiti olevat tema ees süütu; ja nõnda ei ole ma ka sinu ees, kuningas, kurja teinud!”
\par 24 Siis oli kuningas tema pärast väga rõõmus ja käskis Taanieli august välja tuua; ja Taaniel toodi august välja ning tema küljes ei leitud ühtki viga, sellepärast et ta oli uskunud oma Jumalasse.
\par 25 Ja kuningas käskis tuua need mehed, kes olid Taanieli süüdistanud, ja visata lõukoerte auku, nemad, nende lapsed ja naised; ja nad ei olnud veel jõudnud augu põhja, kui lõukoerad said nende üle võimuse ja murdsid kõik nende kondid.
\par 26 Siis kuningas Daarjaves kirjutas kõigile rahvastele, suguvõsadele ja keeltele, kes elasid kogu maal: „Teie rahu olgu suur!
\par 27 Minu poolt on antud käsk, et kogu mu kuningriigi võimupiirkonnas tuleb väriseda ja karta Taanieli Jumala ees! Sest tema on elav Jumal ja püsib igavesti! Tema kuningriik ei hukku ja tema valitsus ei lõpe!
\par 28 Tema päästab ja vabastab, tema teeb tunnustähti ja imesid taevas ja maa peal, tema, kes päästis Taanieli lõukoerte küüsist!”
\par 29 Ja selle Taanieli käsi käis hästi Daarjavese kuningriigis ja pärslase Koorese kuningriigis.

\chapter{7}

\par 1 Paabeli kuninga Belsassari esimesel aastal nägi Taaniel voodis olles und ja nägemusi peas. Siis ta kirjutas unenäo üles, tegi sisu kokkuvõtte.
\par 2 Taaniel hakkas rääkima ja ütles: „Ma nägin oma nägemuses öösel, ja vaata, neli taeva tuult panid voogama suure mere!
\par 3 Ja neli suurt metsalist tõusid merest, üksteisest erinevad.
\par 4 Esimene oli nagu lõukoer, aga tal olid kotka tiivad; minu nähes katkuti talt tiivad ja ta tõsteti maast üles ning pandi jalgadele püsti nagu inimene, ja temale anti inimese süda.
\par 5 Ja vaata, tema järel teine metsaline, karu sarnane; aga see pandi püsti ühe külje peale ja tal oli kolm küljeluud suus hammaste vahel; ja temale öeldi nõnda: „Tõuse, söö palju liha!”
\par 6 Pärast seda ma nägin, ja vaata, üks teine, otsekui panter, ja tal oli seljas neli linnu tiiba; sel metsalisel oli neli pead ja temale anti valitsus.
\par 7 Pärast seda ma nägin öisis nägemusis, ja vaata, seal oli neljas metsaline, kole ja kohutav ja väga tugev; tal olid suured raudhambad, ta sõi ja näris ning tallas ülejäägi jalgadega. Ja see oli erinev kõigist endisist metsalisist, ja tal oli kümme sarve.
\par 8 Ma vaatlesin sarvi, ja vaata, veel üks pisuke sarv tõusis nende vahelt ja kolm endisist sarvist kisti selle eest välja; ja vaata, sellel sarvel olid silmad nagu inimese silmad, ja suu, mis suurustas!
\par 9 Kui ma seda vaatasin, asetati aujärjed ja Elatanu võttis istet. Tema kuub oli valge nagu lumi ja ta juuksed nagu puhas vill! Tema aujärjeks olid tuleleegid, selle rattaiks põlev tuli!
\par 10 Tulejõgi voolas ja väljus temast! Tuhat korda tuhat teenisid teda, kümme tuhat korda kümme tuhat seisid tema ees! Kohus võttis istet ja raamatud avati!
\par 11 Ma vaatasin siis nende suurte sõnade kõla pärast, mis sarv rääkis? Kui ma vaatasin, siis tapeti metsaline ja ta keha hävitati ning heideti tulle põlema!
\par 12 Ka teistelt metsalistelt võeti valitsus ära ja määrati kindlaks nende eluea aeg ja kestus!
\par 13 Ja ma nägin öisis nägemusis, ja vaata, taeva pilvedega tuli keegi, kes oli Inimesepoja sarnane! Ja ta saabus Elatanu juurde ning viidi tema ette.
\par 14 Ja temale anti valitsus ja au ning kuningriik, ja kõik rahvad, suguvõsad ja keeled teenisid teda! Tema valitsus on igavene valitsus, mis ei lakka, ja tema kuningriik ei hukku!
\par 15 Minu, Taanieli vaim oli seetõttu murelik ja nägemused mu peas kohutasid mind.
\par 16 Ma liginesin ühele seal seisjaist ja palusin temalt seletust kõigi nende asjade kohta; ja ta vastas mulle ning tegi nende asjade tähenduse mulle teatavaks:
\par 17 „Need suured metsalised, keda on neli, on neli kuningat, kes tõusevad maa peal!
\par 18 Aga Kõigekõrgema pühad saavad kuningriigi ja kuningriik jääb neile igaveseks - igavesest ajast igavesti!”
\par 19 Siis ma tahtsin seletust neljanda metsalise kohta, kes oli teistsugune kui kõik muud, kes oli väga kohutav, kellel olid raudhambad ja vaskküüned, kes sõi, näris ja tallas ülejäägi jalgadega,
\par 20 ja kümne sarve kohta ta peas ja selle teise kohta, mis tõusis, mille eest kolm maha langes, selle sarve kohta, millel olid silmad ja suurustav suu ja mis näis olevat suurem kui teised.
\par 21 Kui ma vaatasin, siis seesama sarv sõdis pühadega ja võitis nad,
\par 22 kuni Elatanu tuli ja kohus anti Kõigekõrgema pühade kätte ja jõudis aeg, millal pühad said kuningriigi endile!
\par 23 Ta ütles nõnda: „Neljas metsaline on neljas kuningriik maa peal, kõigist kuningriikidest erinev, see sööb kogu maa ja tallab ning purustab selle!
\par 24 Ja need kümme sarve: sellest kuningriigist tõuseb kümme kuningat, ja veel üks tõuseb pärast neid; aga see on teistsugune kui eelmised ja ta alandab kolm kuningat.
\par 25 Ja ta kõneleb sõnu Kõigekõrgema vastu ning piinab Kõigekõrgema pühi! Ja ta püüab muuta aegu ja seadust, ja need antakse tema kätte ajaks, aegadeks ja pooleks ajaks!
\par 26 Siis kohus võtab istet ja temalt võetakse ära valitsus ja ta hävitatakse ning hukatakse lõplikult!
\par 27 Ja kuningriik ning valitsus, ja võim kuningriikide üle kogu taeva all antakse Kõigekõrgema pühale rahvale! Tema kuningriik on igavene kuningriik ja kõik valitsused peavad teda teenima ning temale alistuma!”
\par 28 Siin lõpeb jutustus. Mind, Taanieli, kohutasid mu mõtted väga ja mu näojume kahvatas, aga selle asja ma pidasin meeles.

\chapter{8}

\par 1 Kuningas Belsassari valitsuse kolmandal aastal ilmutati mulle, Taanielile, nägemus peale selle, mis mulle varem oli ilmutatud.
\par 2 Kui ma nägemuses nägin, siis ma olin, kui ma vaatasin, Suusa palees Eelami maakonnas; ja ma nägin selles nägemuses, et ma olin Uulai jõe ääres.
\par 3 Ja ma tõstsin oma silmad üles ja vaatasin, ja ennäe, üks jäär seisis teispool jõge ja tal oli kaks sarve; ja sarved olid kõrged, aga üks oli teisest kõrgem, ja kõrgem tõusis viimasena.
\par 4 Ma nägin jäära kaevlevat lääne ja põhja ja lõuna poole, ja ükski loom ei suutnud temale vastu seista ega olnud tema käest päästjat; ta tegi, nagu temale meeldis, ja ta suurustas.
\par 5 Siis ma panin tähele, ja vaata, üks sikk tuli lääne poolt üle kogu maa, ilma et ta oleks puudutanud maad; ja sikul oli silmade vahel väljapaistev sarv.
\par 6 Ja ta tuli kahe sarvega jäära juurde, keda ma olin näinud seisvat teisel pool jõge, ja jooksis sellele kallale oma jõulises vihas.
\par 7 Ja ma nägin teda jõudvat jäära juurde; siis ta märatses selle vastu ja puskis jäära ning murdis selle mõlemad sarved; ja jääral ei olnud jõudu temale vastu seista, vaid too paiskas ta maha ja tallas teda; ja ei olnud kedagi, kes oleks päästnud jäära tema käest.
\par 8 Ja sikk suurustas üpris väga, aga kui ta oli vägevuse tipul, murdus suur sarv ja selle asemele kasvas neli väljapaistvat sarve taeva nelja tuule poole.
\par 9 Ja ühest neist puhkes veel üks pisuke sarv ja see kasvas väga suureks lõuna ja ida ja ilusa maa suunas!
\par 10 Ta kasvas suureks kuni taeva väeni ja paiskas maha mõned sellest väest ja tähtedest ning tallas neid!
\par 11 Ja ta tõstis ennast kuni selle väe vürstini; sellelt võeti ära alaline ohver ja ta pühamu paik tõugati ümber!
\par 12 Ja alalise ohvri asemele anti üleastumine; ta paiskas tõe maha, ja mida ta tegi, läks tal korda!
\par 13 Siis ma kuulsin üht püha kõnelevat, ja teine püha küsis kõnelejalt: „Kui kaua kestab see nägemus alalisest ohvrist ja hukutavast üleastumisest, pühamu ja väe tallata-andmisest?”
\par 14 Ja ta ütles mulle: „Kaks tuhat kolmsada õhtut-hommikut! Siis saab pühamu taas oma õiguse!”
\par 15 Ja kui mina, Taaniel, olin näinud seda nägemust ja püüdsin seda mõista, vaata, siis seisis mu ees keegi, kellel oli mehe välimus!
\par 16 Ja ma kuulsin inimese häält Uulai kallaste vahelt, ja see hüüdis ning ütles: „Gabriel, seleta see nägemus tollele seal!”
\par 17 Siis ta tuli sinna, kus ma seisin, ja kui ta tuli, siis ma kohkusin ja langesin silmili maha! Ja ta ütles mulle: „Inimesepoeg, pane tähele, et see nägemus tähendab viimset aega!”
\par 18 Ja kui ta minuga rääkis, olin ma sügavas unes, silmili maas; aga ta puudutas mind ja pani mind seisma sinna, kus ma olin seisnud,
\par 19 ja ütles: „Vaata, ma annan sulle teada, mis sünnib sajatuse viimsel ajal: seatud ajal tuleb lõpp!
\par 20 Jäär, keda sa nägid, kellel oli kaks sarve, on Meeda ja Pärsia kuningad.
\par 21 Karune sikk on Kreeka kuningas; suur sarv tema silmade vahel on esimene kuningas.
\par 22 Ja et see murdus ning neli tõusid selle asemele, on neli kuningriiki, mis tõusevad tema rahvast, aga mitte nii vägevad kui tema.
\par 23 Ja nende valitsuste lõpus, kui üleastujate mõõt on täis saanud, tõuseb üks kuningas, jultunud näoga ja salasepitsustes osav!
\par 24 Ja tema jõud on vägev, aga mitte ta omast jõust, ja ta valmistab tohutu õnnetuse; tal on kordaminek teos ja ta hävitab vägevad!
\par 25 Tema tarkus on püha rahva vastu ja ta käes õnnestub pettus; ta suurustab oma südames ja hävitab paljusid ootamatult! Vürstide vürstigi vastu ta tõuseb, ent ilma inimkäeta murtakse ta ise!
\par 26 Ja nägemus õhtuist ja hommikuist, millest oli kõne, on tõsi; aga sina hoia see nägemus saladuses, sest see on kaugeiks päeviks!”
\par 27 Ja mina, Taaniel, olin seejärel mõnd aega haige; siis ma tõusin üles ja toimetasin kuninga teenistust, aga ma olin kohkunud nägemuse pärast ega mõistnud seda.

\chapter{9}

\par 1 Daarjavese, Ahasverose poja, kes oli meedlaste soost ja oli saanud kaldealaste kuningriigi kuningaks, esimesel aastal,
\par 2 tema valitsuse esimesel aastal mina, Taaniel, panin raamatuis tähele aastate arvu, mis prohvet Jeremijale tulnud Jehoova sõna järgi pidi täide minema Jeruusalemma varemeis-oleku kohta: seitsekümmend aastat.
\par 3 Ja ma pöörasin oma näo Issanda Jumala poole, otsides teda palve ja anumistega, paastudes, kotiriides ja tuhas.
\par 4 Ma palusin Jehoovat, oma Jumalat, tunnistasin ja ütlesin: „Oh Issand, sina oled suur ja kardetav Jumal, kes peab lepingut ja osutab heldust neile, kes teda armastavad ja tema käske peavad!
\par 5 Me oleme pattu teinud ja eksinud, oleme olnud ülekohtused ja oleme hakanud vastu, oleme taganenud su käskudest ja kohtuseadustest,
\par 6 me pole kuulanud su sulaseid, prohveteid, kes sinu nimel kõnelesid meie kuningaile, vürstidele, vanemaile ja kogu maa rahvale!
\par 7 Sinul, Issand, on õigus, aga meil on häbi silmis, nõnda nagu tänapäeval on Juuda meestel ja Jeruusalemma elanikel ja kogu Iisraelil, lähedastel ja kaugetel kõigis maades, kuhu sa nad oled pillutanud truudusetuse pärast, mida nad sinule osutasid!
\par 8 Jehoova! Meil on häbi silmis, meie kuningail, meie vürstidel ja meie vanemail, et me oleme sinu vastu pattu teinud!
\par 9 Issandal, meie Jumalal on halastus ja andeksand, kuigi me oleme hakanud temale vastu
\par 10 ega ole kuulanud Jehoova, oma Jumala häält, et oleksime käinud tema õpetuste järgi, mis ta pani meie ette oma sulaste prohvetite läbi,
\par 11 vaid kogu Iisrael on astunud üle su käsuõpetusest ja on taganenud sellest, ilma et oleks võtnud kuulda su häält! Seepärast on meie peale valatud see sajatus ja vanne, mis Jumala sulase Moosese käsuõpetuses on kirja pandud! Sest me oleme tema vastu pattu teinud!
\par 12 Ja tema on tõeks teinud oma sõnad, mis ta meie ja meie valitsejate kohta, kes meie üle valitsesid, oli rääkinud, tuues meile nii suure õnnetuse, mille sarnast ei ole sündinud kogu taeva all, nagu see sündis Jeruusalemmas!
\par 13 Nõnda nagu Moosese käsuõpetuses on kirjutatud, nõnda tuli kogu see õnnetus meie peale. Aga me ei ole Jehoova, oma Jumala palet mitte leevendanud, et oleksime pöördunud oma süütegudest ja tähele pannud sinu tõde!
\par 14 Seepärast oli Jehoova valvas ja tõi selle õnnetuse meie peale; sest Jehoova, meie Jumal on õiglane kõigis oma tegudes, mis ta teeb! Meie aga ei ole kuulanud tema häält!
\par 15 Ja nüüd, Issand, meie Jumal, kes oled oma rahva Egiptusemaalt ära toonud oma vägeva käega ja oled teinud enesele nime, nagu see tänapäeval on: meie oleme pattu teinud ja õelad olnud!
\par 16 Issand, kõigi su õiglaste tegude pärast pöördugu ometi su viha ja raev su linnalt Jeruusalemmalt, su pühalt mäelt! Sest meie pattude ja meie vanemate süütegude pärast on Jeruusalemm ja sinu rahvas saanud teotuse aluseks kõigile meie naabritele!
\par 17 Ja nüüd, meie Jumal, kuule oma sulase palvet ja anumisi, ja lase oma pale paista oma hävitatud pühamu peale iseenese pärast, Issand!
\par 18 Pööra, mu Jumal, oma kõrv ja kuule, ava oma silmad ja vaata meie hävingut, ja linna, millele on pandud sinu nimi, sest me ei heida oma anumisi su palge ette mitte oma õiguse pärast, vaid sinu suure halastuse pärast!
\par 19 Issand, kuule, Issand, anna andeks, Issand, pane tähele ja tee teoks iseenese pärast, mu Jumal, ära viivita! Sest su linnale ja su rahvale on pandud sinu nimi!”
\section*{Ennustus seitsmekümnest aastanädalast}

\par 20 Ja kui ma alles rääkisin ja palvetasin ja tunnistasin oma pattu ja oma Iisraeli rahva pattu ja heitsin oma anumise Jehoova, oma Jumala palge ette mu Jumala püha mäe pärast,
\par 21 kui ma alles palves rääkisin, siis tuli see mees, Gabriel, keda ma nägemuses varem olin näinud, kiiresti lennates mu juurde õhtuse ohvri ajal.
\par 22 Ta tuli ja rääkis minuga ning ütles: „Taaniel, nüüd ma olen tulnud sind targaks tegema!
\par 23 Su anumiste algul läks Sõna välja ja mina olen tulnud teatama, et sa oled armastusväärne! Seepärast pane tähele Sõna ja õpi mõistma nägemust!
\par 24 Seitsekümmend aastanädalat on seatud su rahvale ja su pühale linnale üleastumise lõpetamiseks, patule piiri panemiseks ja süüteo lepitamiseks, igavese õiguse toomiseks, nägemuse ja prohveti kinnitamiseks ja Kõigepühama võidmiseks!
\par 25 Ja tea ning mõista: sõna väljumisest alates Jeruusalemma taastamiseks ja ülesehitamiseks kuni võitud vürstini on seitse aastanädalat; ja kuuskümmend kaks aastanädalat - siis on see jälle üles ehitatud turgude ja vallikraavidega, kuigi aegade surve all. Ja aegade lõpus,
\par 26 pärast kuutkümmend kaht aastanädalat, kaotatakse Võitu ja teda ei ole enam; linna ja pühamu hävitab ühe vürsti rahvas, kes tuleb; selle lõpp tuleb uputusega ja lõpuks on sõda: hävitus on määratud!
\par 27 Paljude meelest ta ei hooli lepingust ühe aastanädala jooksul ja ta lõpetab pooleks aastanädalaks tapa- ja roaohvri; hävitaja tuleb koletise tiibadel, kuni määratud lõpp voolab üle hävitaja!”

\chapter{10}

\par 1 Pärsia kuninga Koorese kolmandal aastal ilmutati Taanielile, kellele oli pandud nimeks Beltsassar, üks asi; ja see asi on tõsi ning tähendab suurt vaeva. Ta pani asja tähele ja õppis nägemust mõistma.
\par 2 „Neil päevil olin mina, Taaniel, leinanud kolm nädalat aega.
\par 3 Maiusrooga ma ei söönud ja liha ega viina ei tulnud mu suhu, ja ma ei võidnud ennast hoopiski mitte, seni kui kolme nädala päevad olid täis saanud.
\par 4 Ja esimese kuu kahekümne neljandal päeval olin ma suure jõe, Hiddekeli kaldal.
\par 5 Ja ma tõstsin oma silmad üles ning vaatasin, ja ennäe, seal oli üks mees, linased riided seljas ja niuded vöötatud Uufa kullaga!
\par 6 Ta ihu oli nagu tarsisekivi ja ta pale paistis välguna, ta silmad olid nagu tulelondid, ta käsivarred ja jalad hiilgasid vase sarnaselt; ja ta sõnad kõlasid nagu rahvahulga karjumine!
\par 7 Mina, Taaniel, nägin üksi seda nägemust, aga mehed, kes olid ühes minuga, seda nägemust ei näinud; ometi valdas neid suur hirm ja nad jooksid peitu.
\par 8 Ma jäin üksi. Ja kui ma nägin seda suurt nägemust, siis ei olnud mul enam rammu, mu tore välimus muutus hirmsaks ja ma jäin jõuetuks.
\par 9 Ja ma kuulsin tema sõnade kõla; ja kui ma tema sõnade kõla kuulsin, siis ma langesin raskes unes silmili maha, nägu vastu maad.
\par 10 Ja vaata, üks käsi puudutas mind ja raputas mind tõusma põlvili ja käpukile!
\par 11 Ja ta ütles mulle: „Taaniel, armas mees, pane tähele sõnu, mis ma sulle räägin, ja seisa seal, kus sa seisid, sest mind on nüüd läkitatud su juurde!” Ja kui ta minuga neid sõnu rääkis, seisin ma värisedes.
\par 12 Ja ta ütles mulle: „Ära karda, Taaniel, sest esimesest päevast peale, kui sa andsid oma südame, et mõista ja olla alandlik oma Jumala ees, võeti su sõnu kuulda; ja ma olen tulnud su sõnade pärast.
\par 13 Pärsia kuningriigi kaitseingel pani mulle vastu kakskümmend üks päeva; aga vaata, Miikael, üks peainglitest, tuli mulle appi ja ma jätsin tema sinna, Pärsia kuningate kaitseingli juurde
\par 14 ja tulin sulle õpetama seda, mis rahvale juhtub viimseil päevil, sest on veelgi üks nägemus neiks päeviks!”
\par 15 Ja kui ta minuga nõnda rääkis, lõin ma oma silmad maha ja vaikisin.
\par 16 Ja vaata, nagu oleks üks inimkäsi puudutanud mu huuli; ja ma avasin suu ja rääkisin ning ütlesin sellele, kes seisis mu kohal: „Mu isand, nägemuse tõttu on mulle tulnud valud ja mul ei ole enam jõudu!
\par 17 Ja kuidas võikski mu isanda seesugune sulane rääkida säärase isandaga? Sest nüüd ei ole mul enam jõudu ega ole mu sisse jäänud hingeõhku!”
\par 18 Siis puudutas see inimesesarnane mind jälle ja kinnitas mind.
\par 19 Ja ta ütles: „Ära karda, armas mees, rahu olgu sulle! Ole julge! Ole julge!” Ja kui ta minuga rääkis, siis ma tundsin ennast tugevana ja ütlesin: „Rääkigu mu isand, sest sa oled mind kinnitanud!”
\par 20 Ja tema ütles: „Kas sa tead, mispärast ma su juurde olen tulnud? Aga nüüd ma lähen tagasi Pärsia kaitseingliga võitlema, ja kui ma temaga valmis saan, vaata, siis tuleb Kreeka kaitseingel!
\par 21 Ometi tahan ma sulle teada anda, mis on kirjutatud tõeraamatusse! Ei ole ühtki, kes aitaks mind nende vastu kui ainult teie kaitseingel Miikael!

\chapter{11}

\par 1 Ja ma seisin meedlase Daarjavese esimesel aastal temale toeks ning kaitseks.
\par 2 Ja nüüd ma annan sulle teada tõe: vaata, veel kolm kuningat tõuseb Pärsias ja neljas soetab suurema rikkuse kui kõik muud; ja kui ta oma rikkuse tõttu on saanud vägevaks, paneb ta kõik liikuma Kreeka kuningriigi vastu.
\par 3 Siis tõuseb kangelaskuningas, valitseb suure võimuga ja teeb, mida tahab.
\par 4 Aga niipea kui ta on tõusnud, murdub ta kuningriik ja jaguneb taeva nelja tuule poole, aga mitte tema järglastele ja ilma selle võimuta, millega tema valitses; sest tema kuningriik kistakse ära ja saab teistele, aga mitte neile.
\par 5 Ja Lõuna kuningas saab vägevaks, samuti üks tema vürstidest; aga see saab temast vägevamaks ja valitseb; tema valitsus on suur võim.
\par 6 Ja aastate pärast nad teevad liidu ja Lõuna kuninga tütar läheb Põhja kuninga juurde rahu tegema; aga tema käsivarrel ei ole jõudu ja tema ja ta sugu ei jää püsima, vaid ta antakse ära, tema ja ta saatjaskond, tema laps ja see, kes teda kosis.
\par 7 Sel ajal tõuseb üks võsu tema juurest ta asemele ja tuleb sõjaväe vastu ning tungib Põhja kuninga kindlustesse, talitab nendega, nagu tahab, ja saab vägevaks.
\par 8 Nende jumaladki ühes valatud kujudega ja ihaldusväärsete asjadega hõbedast ja kullast ta viib saagina Egiptusesse; siis ta jätab Põhja kuninga mõneks aastaks rahule.
\par 9 See aga tuleb küll Lõuna kuninga riiki, ent peab omale maale tagasi minema.
\par 10 Aga tema poeg varustab ja kogub suuri sõjaväehulki; ja ta tuleb temale kallale, ujutab üle ja käib läbi; siis ta tuleb uuesti ja tungib kuni tema kindluseni.
\par 11 Siis vihastub Lõuna kuningas ja läheb ning sõdib sellega, Põhja kuningaga; too seab üles suure väehulga, aga see väehulk antakse tema kätte.
\par 12 Ja kui see väehulk on võetud, siis läheb ta süda ülbeks; ta tapab kümneid tuhandeid, aga võitu ei saa.
\par 13 Ja jälle seab Põhja kuningas üles väehulga, mis on suurem kui endine, ja mõne aasta pärast tuleb ta temale kallale suure sõjaväe ja rikkaliku vooriga.
\par 14 Ja sel ajal tõusevad paljud Lõuna kuninga vastu; ja sinu omast rahvast tõuseb vägivalla-mehi, et nägemus läheks tõeks, nad ise aga langevad.
\par 15 Ja Põhja kuningas tuleb ja kuhjab piiramisvalli ning vallutab kindlustatud linna; Lõuna sõjaväed ei pea vastu ega ole ta valitute väel püsimiseks jõudu.
\par 16 Ja see, kes tuleb tema vastu, talitab oma heaksarvamise järgi ja ükski ei suuda seisma jääda tema ees; ta asub ilusale maale ja see kõik saab tema kätte.
\par 17 Siis ta pöörab oma näo, et ta võiks tulla kogu oma kuningriigi võimsusega; ja ta teeb temaga rahu ning annab oma tütre temale naiseks, et siis seda maad hävitada; aga see ei toimu ega lähe tal korda.
\par 18 Siis ta pöörab oma näo saarte poole ja vallutab paljud; aga üks väepealik teeb ta pilkamisele lõpu, tõesti, ta tasub temale ta pilkamise eest!
\par 19 Siis ta pöörab näo tagasi maa kindluste poole, aga ta komistab ja langeb ning teda ei ole enam.
\par 20 Ja tema asemele tõuseb keegi, kes paneb maksunõudja kõige suursugusemat kuningriiki läbi käima, aga mõne aja pärast ta murdub, ent mitte viha läbi ega sõjas.
\par 21 Ja tema asemele tõuseb üks kõlvatu, kellele ei ole antud kuninglikku väärikust; aga ta tuleb ootamatult ja haarab libekeelsusega kuningriigi enesele.
\par 22 Sõjaväed uhutakse sootuks tema eest ja murtakse, nõndasamuti ka lepinguvürst.
\par 23 Niipea kui temaga on tehtud leping, teeb ta pettust; ja ta läheb ning saab vägevaks vähese rahvaga.
\par 24 Ootamatult tungib ta maa viljakamasse osasse ja teeb, mida ta isad ja isade isad ei ole teinud: pillub omadele kokkuröövitut ja saaki ning vara. Ja ta sepitseb plaane kindlustatud linnade vastu, aga ainult pisut aega.
\par 25 Ja ta õhutab oma jõudu ning julgust, et minna Lõuna kuninga vastu suure sõjaväega; ka Lõuna kuningas valmistub sõjaks suure ja väga tugeva sõjaväega, aga see ei suuda vastu panna, sest tema vastu peetakse vandenõu.
\par 26 Need, kes ta lauas söövad, murravad tema, ta sõjavägi uhutakse ära ja paljud langevad mahalööduina.
\par 27 Mõlemad kuningad mõtlevad südames kurja ja ühes lauas istudeski nad räägivad valet. Aga see ei õnnestu, sest lõpp tuleb alles määratud ajal.
\par 28 Ja ta läheb tagasi oma maale suure varandusega, aga ta süda on püha lepingu vastu; ta teeb, mida tahab, ja läheb tagasi oma maale.
\par 29 Määratud ajal tuleb ta jälle Lõunasse, aga teisel korral ei sünni nõnda nagu esimesel korral.
\par 30 Sest kitiitide laevad tulevad tema vastu ja ta lööb kartma ning läheb tagasi, sajatades püha lepingut; ja ta tegutseb ning paneb tähele neid, kes loobuvad pühast lepingust.
\par 31 Ja tema poolt saadetud sõjaväed tulevad ja teotavad pühamut, kindlat linna; nad kõrvaldavad alalise ohvri ja panevad sinna hävituse koletise.
\par 32 Ja oma libekeelsusega ta hukutab lepingu rikkujaid, aga see rahvas, kes tunneb oma Jumalat, jääb kindlaks ja tegutseb.
\par 33 Ja mõistlikud rahva hulgast õpetavad paljusid, kuigi need mõnd aega langevad mõõga ja tuleleegi kätte, vangi ja riisutavaiks.
\par 34 Ja kui nad langevad, siis nad saavad vähest abi, ja paljud liituvad nendega libekeelsuses.
\par 35 Ja mõistlikest langevad mõned, et neid sulatada ja puhastada ja valgeks teha lõpuajaks, sest määratud aeg viibib veel.
\par 36 Ja kuningas teeb, mida tahab, ja ta ülistab ennast ning suurustab iga jumala ees, isegi jumalate Jumala vastu räägib ta kuulmatuid asju; ja see õnnestub tal kuni sajatus on teostunud, sest mis on otsustatud, see sünnib.
\par 37 Ja ta ei hooli oma vanemate jumalaist, ka mitte sellest, keda naised armastavad, ja ta ei hooli ühestki muust jumalast, vaid suurustab nende kõigi ees
\par 38 ja austab selle asemel kindluste jumalat; jumalat, keda ta vanemad ei tundnud, austab ta kulla ja hõbedaga, kalliskivide ja väärtasjadega.
\par 39 Ja ta paneb kindlustatud linnadesse võõra jumala rahva; kes teda tunnustavad, neid ta austab väga ja paneb valitsema paljude üle ning jagab neile tasuks maa.
\par 40 Aga lõpuajal Lõuna kuningas põrkab temaga kokku ja Põhja kuningas tormab temale kallale sõjavankrite, ratsanike ja paljude laevadega; ta tungib maadesse, ujutab need üle ja käib neist läbi.
\par 41 Ja ta tungib ilusale maale ning kümned tuhanded langevad. Aga need pääsevad tema käest: Edom, Moab ja peamine osa ammonlastest.
\par 42 Ja ta sirutab käe maade järele ja Egiptusemaa ei pääse.
\par 43 Ja ta valitseb kulla ja hõbeda varandusi ja kõiki Egiptuse väärtasju; ja liibüalased ja etiooplased on tema saatjaskonnas.
\par 44 Aga sõnumid idast ja põhjast teevad temale hirmu ja ta läheb ära suures vihas, hukkama ja hävitama paljusid.
\par 45 Ja ta lööb oma toredad telgid üles mere ja püha ilumäe vahele; aga tema lõpp tuleb ja ükski ei aita teda!

\chapter{12}

\par 1 Ja sel ajal tõuseb Miikael, see suur vürst, kes seisab su rahva laste eest. Siis on kitsas aeg, millist ei ole olnud rahvaste algusest peale kuni selle ajani; aga sel ajal päästetakse su rahvas, kõik, keda leitakse olevat raamatusse kirjutatud!
\par 2 Ja paljud neist, kes magavad mulla põrmus, ärkavad: ühed igaveseks eluks ja teised teotuseks, igaveseks põlastuseks!
\par 3 Siis paistavad mõistlikud nagu taevalaotuse hiilgus, ja need, kes saadavad paljusid õiguse teele, otsekui tähed ikka ja igavesti!
\par 4 Aga sina, Taaniel, pea need sõnad saladuses ja pane raamat pitseriga kinni lõpuajaks! Siis uurivad paljud seda ja arusaamine kasvab!”
\par 5 Ja mina, Taaniel, vaatasin, ja ennäe, kaks teist inglit seisid seal, üks jõe siinpoolsel kaldal ja teine jõe sealpoolsel kaldal!
\par 6 Ja üks küsis linases riides mehelt, kes oli jõe vete kohal: „Kui kaua tuleb oodata nende imede lõppu?”
\par 7 Siis ma kuulsin linases riides meest, kes oli jõe vete kohal, kuidas ta oma paremat ja vasakut kätt taeva poole tõstes vandus selle juures, kes igavesti elab: „Tõesti, aeg, ajad ja pool aega! Ja kui püha rahva jõu hävitus on jõudnud lõpule, siis teostuvad need kõik!”
\par 8 Ma kuulsin, aga ei mõistnud, ja ma küsisin: „Mu isand, missugune on nende lõpp?”
\par 9 Aga ta ütles: „Mine, Taaniel, sest need sõnad jäävad saladusse ja pitseriga kinnipanduks kuni lõpuajani!
\par 10 Paljusid puhastatakse, tehakse valgeks ja sulatatakse, aga õelad teevad õelust ja ükski õel ei mõista seda; aga mõistlikud mõistavad küll!
\par 11 Ja alates sellest ajast, mil kõrvaldatakse alaline ohver ja sinna pannakse hävituse koletis, on tuhat kakssada üheksakümmend päeva.
\par 12 Õnnis on, kes ootab ja elab üle tuhat kolmsada kolmkümmend viis päeva!
\par 13 Aga sina mine lõpule vastu ja puhka ning tõuse oma liisuosaks päevade lõpus!”

\end{document}