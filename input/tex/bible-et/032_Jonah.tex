\begin{document}

\title{Joona}

\chapter{1}

\par 1 Ja Jehoova sõna tuli Joonale, Amittai pojale; ta ütles:
\par 2 „Võta kätte, mine Niinivesse, sinna suurde linna ja jutlusta sellele, sest nende kurjus on tõusnud mu palge ette!”
\par 3 Aga Joona tahtis Jehoova palge eest põgeneda Tarsisesse; ta läks alla Jafasse ja leidis laeva, mis oli Tarsisesse minemas; ta andis sõiduraha ja astus peale, et minna ühes nendega Tarsisesse, Jehoova palge eest ära.
\par 4 Ent Jehoova paiskas mere peale suure tuule, nõnda et merel tõusis suur torm ja näis, et laev hukkub.
\par 5 Ja meremehed kartsid ning hüüdsid igaüks oma jumala poole; ja nad heitsid merre laevas olevad asjad, et seda neist kergendada. Aga Joona oli läinud alla laevaruumi ja oli heitnud magama; ja ta magas sügavasti.
\par 6 Siis astus laevajuht tema juurde ja ütles temale: „Kuidas sa saad nõnda sügavasti magada? Tõuse üles, hüüa oma jumala poole, vahest jumal mõtleb meie peale ja me ei hukku!”
\par 7 Aga isekeskis nad ütlesid: „Lähme ja heidame liisku, et saaksime teada, kelle pärast on see õnnetus meile tulnud!” Ja nad heitsid liisku ja liisk langes Joonale.
\par 8 Siis nad ütlesid temale: „Seleta meile ometi, mispärast on see õnnetus meile tulnud? Mis on su amet ja kust sa tuled? Kus on su kodumaa ja missugusest rahvast oled sa pärit?”
\par 9 Ja ta vastas neile: „Mina olen heebrealane ja ma kardan Jehoovat, taeva Jumalat, kes on teinud mere ja kuiva maa!”
\par 10 Siis mehed hakkasid üpris väga kartma ja ütlesid temale: „Miks sa seda tegid?” Sest mehed teadsid, et ta oli Jehoova palge eest põgenemas, kuna ta oli neile sellest rääkinud.
\par 11 Ja nad küsisid temalt: „Mida peaksime sinuga tegema, et meri meie poolest võiks rahuneda?” Sest meri hakkas üha enam mässama.
\par 12 Ja ta vastas neile: „Võtke mind ja visake merre, siis rahuneb meri teie poolest! Sest ma tean, et minu pärast on see suur torm teile tulnud!”
\par 13 Mehed sõudsid küll, et pääseda tagasi kuivale maale, aga nad ei suutnud, sest meri hakkas üha enam mässama nende vastu.
\par 14 Siis nad hüüdsid Jehoova poole ja ütlesid: „Oh Jehoova, ära ometi lase meid hukkuda selle mehe hinge pärast, samuti ära pane meie peale süütu verd, sest sina, Jehoova, teed nõnda nagu on sinule meelepärane!”
\par 15 Ja nad võtsid Joona ning viskasid ta merre; siis rauges mere raev.
\par 16 Aga mehed kartsid Jehoovat üpris väga ja nad ohverdasid Jehoovale tapaohvri ning andsid tõotusi.

\chapter{2}

\par 1 Aga Jehoova saatis ühe suure kala, et see neelaks Joona. Ja Joona oli kala kõhus kolm päeva ja kolm ööd.
\par 2 Ja Joona palus Jehoovat, oma Jumalat kala kõhus
\par 3 ja ütles: „Ma hüüdsin oma kitsikuses Jehoova poole ja tema vastas mulle! Haua sisemuses ma hüüdsin appi, sina kuulsid mu häält!
\par 4 Sest sina heitsid mind sügavusse, merede südamesse; voolus ümbritses mind, kõik su vood ja su lained käisid minust üle!
\par 5 Siis ma mõtlesin: ma olen ära aetud su silma eest! Kas saan veel näha su püha templit?
\par 6 Vesi ümbritses mind kõrist saadik, süvavesi piiras mind, kõrkjad mähkisid mu pead!
\par 7 Ma vajusin alla mägede alusteni, maa riivid sulgusid mu kohal igaveseks! Aga sina tõid mu elu hauast üles, Jehoova, mu Jumal!
\par 8 Kui mu hing nõrkes mu sees, mõtlesin ma Jehoovale ja mu palve tuli sinu juurde su pühasse templisse!
\par 9 Need, kes austavad tühje ebajumalaid, hülgavad oma osaduse!
\par 10 Aga mina tahan ohverdada sinule tänulaulu kõlades! Mida olen tõotanud, seda täidan! Jehoova käes on pääste!”
\par 11 Siis Jehoova käskis kala ja see oksendas Joona kuivale maale.

\chapter{3}

\par 1 Ja Joonale tuli teist korda Jehoova sõna; ta ütles:
\par 2 „Võta kätte, mine Niinivesse, sinna suurde linna ja pea sellele jutlus, nagu ma sind käsin!”
\par 3 Siis Joona võttis kätte ja läks Jehoova sõna peale Niinivesse; Niinive oli suur linn isegi Jumala ees: kolm päevateekonda!
\par 4 Ja Joona hakkas linna läbi käima; käies esimese päeva teed, ta hüüdis ja ütles: „Veel nelikümmend päeva ja Niinive hävitatakse!”
\par 5 Ja Niinive mehed uskusid temasse, kuulutasid paastu ja riietusid kotiriidesse, nii suured kui väikesed.
\par 6 Ja kui sõna sellest jõudis Niinive kuningale, tõusis ta oma aujärjelt ja võttis mantli seljast, kattis ennast kotiriidega ja istus tuha peale!
\par 7 Siis ta laskis Niinives kuulutada ja öelda: „Kuninga ja tema ülikute käsul öeldakse: inimesed ja loomad, veised ja pudulojused ärgu söögu midagi, ärgu käigu karjas ja ärgu joogu vett,
\par 8 vaid nad katku endid kotiriidega, inimesed ja loomad, ja hüüdku võimsasti Jumala poole ja igaüks pöördugu oma kurjalt teelt ja vägivallast, mis nende käte küljes on!
\par 9 Kes teab, vahest Jumal pöördub ja kahetseb, pöördub oma tulisest vihast ja me ei hukku!”
\par 10 Kui Jumal nägi nende tegusid, et nad pöördusid oma kurjalt teelt, siis Jumal kahetses kurja, mis ta oli lubanud neile teha, ega teinud seda mitte.


\chapter{4}

\par 1 Aga see oli Joonale väga vastumeelt ja ta viha süttis põlema.
\par 2 Ja ta palus Jehoovat ning ütles: „Oh Jehoova! Eks olnud see minugi sõna, kui ma olin alles kodumaal? Sellepärast ma tahtsingi eelmisel korral põgeneda Tarsisesse, sest ma teadsin, et sina oled armuline ja halastaja Jumal, pika meelega ja rikas heldusest ja et sa kahetsed kurja!
\par 3 Seepärast, Jehoova, võta nüüd ometi minust mu hing, sest mul on parem surra kui elada!”
\par 4 Aga Jehoova küsis: „Kas sul on õigus vihastuda?”
\par 5 Siis Joona läks linnast välja ja asus ida poole linna; ta tegi enesele sinna lehtmajakese ja istus selle varjus, et näha, mis linnaga juhtub.
\par 6 Ja Jehoova Jumal käsutas ühe kiikajonipuu ja see kasvas Joona kohale, olles ta peale varjuks, päästes teda ta vaevast. Ja Joona rõõmutses kiikajonipuu pärast üpris väga.
\par 7 Aga järgmise päeva koites Jumal saatis ühe ussi, kes näris kiikajonipuud, nõnda et see kuivas.
\par 8 Ja kui päike tõusis, siis Jumal saatis kõrvetava idatuule ja päike pistis Joonat pähe, nõnda et ta oli minestamas; siis ta palus enesele surma ja ütles: „Mul on parem surra kui elada!”
\par 9 Aga Jumal küsis Joonalt: „Kas sul on õigus vihastuda kiikajonipuu pärast?” Tema vastas: „Küllap mul on õigus vihastuda surmani!”
\par 10 Siis ütles Jehoova: „Sina tahaksid armu anda kiikajonipuule, mille kallal sa ei ole näinud vaeva ja mida sa ei ole kasvatanud, mis ühe ööga sündis ja ühe ööga hukkus,
\par 11 aga mina ei peaks armu andma Niinivele, sellele suurele linnale, kus on enam kui kaksteist korda kümme tuhat inimest, kes ei oska vahet teha oma parema ja vasaku käe vahel, ja palju lojuseid!”




\end{document}