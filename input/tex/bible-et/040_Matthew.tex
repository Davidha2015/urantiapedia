\begin{document}

\title{Matteuse evangeelium}

\chapter{1}

\section*{Jeesuse esivanemad}

\par 1 Jeesuse Kristuse, Taaveti poja, Aabrahami poja, sünniraamat.
\par 2 Aabrahamile sündis Iisak, Iisakile sündis Jaakob, Jaakobile sündisid Juuda ja tema vennad;
\par 3 Juudale sündisid Taamarist Perets ja Serah, Peretsile sündis Hesrom, Hesromile sündis Aram;
\par 4 Aramile sündis Amminaadab, Amminaadabile sündis Nahson, Nahsonile sündis Salmon;
\par 5 Salmonile sündis Raahabist Boas, Boasele sündis Rutist Oobed, Oobedile sündis Jesse;
\par 6 Jessele sündis kuningas Taavet; Taavetile sündis Uurija naisest Saalomon;
\par 7 Saalomonile sündis Rehabeam, Rehabeamile sündis Abija, Abijale sündis Aasa;
\par 8 Aasale sündis Joosafat, Joosafatile sündis Jooram, Jooramile sündis Ussija;
\par 9 Ussijale sündis Jootam, Jootamile sündis Ahas, Ahasele sündis Hiskija;
\par 10 Hiskijale sündis Manasse, Manassele sündis Aamos, Aamosele sündis Joosija;
\par 11 Joosijale sündisid Jekonja ja tema vennad Baabüloni vangipõlve ajal.
\par 12 Pärast Baabüloni vangipõlve sündis Jekonjale Sealtiel, Sealtielile sündis Serubbaabel;
\par 13 Serubbaabelile sündis Abihuud, Abihuudile sündis Eljakim, Eljakimile sündis Asur;
\par 14 Asurile sündis Saadok, Saadokile sündis Aahim, Aahimile sündis Elihuud;
\par 15 Elihuudile sündis Eleasar, Eleasarile sündis Mattan, Mattanile sündis Jaakob;
\par 16 Jaakobile sündis Joosep, Maarja mees; ja Maarjast sündis Jeesus, keda nimetatakse Kristuseks.
\par 17 Kõiki põlvi Aabrahamist Taavetini on siis neliteist põlve; ja Taavetist Baabüloni vangipõlve ajani on neliteist põlve; ja Baabüloni vangipõlvest Kristuseni on neliteist põlve.

\section*{Jeesuse sünd}

\par 18 Aga Jeesuse Kristuse sündimine oli nõnda: kui ta ema Maarja oli kihlatud Joosepiga, siis ta leiti enne nende kokkusaamist käima peal olevat Pühast Vaimust.
\par 19 Aga et tema mees Joosep oli õiglane ega tahtnud teda saata häbisse, siis ta võttis nõuks salaja tema hüljata.
\par 20 Ent kui temal see mõte oli, vaata, siis ilmus talle Issanda ingel unes ja ütles: „Joosep, Taaveti poeg, ära karda Maarjat, oma naist, enese juurde võtta, sest mis temas on sündinud, on Pühast Vaimust.
\par 21 Ta toob ilmale poja ja sa pead temale nimeks panema Jeesus, sest tema päästab oma rahva nende pattudest!”
\par 22 Aga see kõik on sündinud, et läheks täide, mis Issand on rääkinud prohveti kaudu, kes ütleb:
\par 23 „Ennäe, neitsi saab käima peale ning toob poja ilmale, ja temale peab pandama nimeks Immaanuel”, see on meie keeli: Jumal meiega.
\par 24 Kui Joosep unest ärkas, siis ta tegi nõnda, kuidas Issanda ingel teda oli käskinud ning võttis oma naise enese juurde
\par 25 ega puutunud temasse, enne kui ta oli ilmale toonud poja. Ja ta pani temale nimeks Jeesus.

\chapter{2}

\section*{Targad hommikumaalt}

\par 1 Kui Jeesus oli sündinud Petlemmas Juudamaal kuningas Heroodese ajal, vaata, siis tulid targad hommikumaalt Jeruusalemma
\par 2 ja ütlesid: „Kus on see sündinud juutide kuningas? Sest me oleme näinud tema tähte hommikumaal ning oleme tulnud teda kummardama.”
\par 3 Kui kuningas Heroodes seda kuulis, ehmus ta väga ja kogu Jeruusalemm ühes temaga.
\par 4 Ja ta kogus kokku kõik rahva ülempreestrid ja kirjatundjad ning kuulas neilt, kus Kristus pidi sündima.
\par 5 Nemad ütlesid temale: „Petlemmas Juudamaal; sest nõnda on kirjutatud prohveti kaudu:
\par 6 Ja sina, Petlemm Juudamaal, ei ole mingil kombel kõige vähem Juuda vürstide seast, sest sinust lähtub valitseja, kes mu rahvast Iisraeli hoiab kui karjane!”
\par 7 Siis Heroodes kutsus targad salaja ja uuris neilt hoolega aja, mil täht oli ilmunud.
\par 8 Ja ta läkitas nad Petlemma ning ütles: „Minge kuulake hoolega lapsukese järele, ja kui te tema leiate, siis andke minule teada, et minagi läheksin teda kummardama!”
\par 9 Kui nemad olid kuningat kuulnud, siis nad läksid teele. Ja vaata, täht, mida nad olid näinud hommikumaal, käis nende eel, kuni ta tuli ja seisatas ülal seal kohal, kus lapsuke oli.
\par 10 Aga tähte nähes nad said üliväga rõõmsaks.
\par 11 Ja nad läksid sinna kotta ning nägid lapsukest ühes Maarjaga, tema emaga, ja heitsid maha ning kummardasid teda ja avasid oma varandused ning tõid temale ande, kulda ja viirukit ja mürri.
\par 12 Ja kui nad unes olid Jumalalt saanud käsu mitte minna tagasi Heroodese juurde, läksid nad teist teed tagasi omale maale.

\section*{Põgenemine Egiptusesse}

\par 13 Aga kui nad olid ära läinud, vaata, siis ilmus Issanda ingel unes Joosepile ja ütles: „Tõuse ning võta lapsuke ja tema ema enesega ning põgene Egiptusesse ja ole seal, kuni ma sinule ütlen; sest Heroodes hakkab otsima last, et teda hukata.”
\par 14 Siis ta tõusis ning võttis enesega lapsukese ja tema ema öösel ja läks ära Egiptusesse
\par 15 ja oli seal kuni Heroodese surmani, et läheks täide, mis Issand on rääkinud prohveti kaudu, kes ütleb: „Egiptusest ma kutsusin oma poja!”

\section*{Petlemma laste tapmine}

\par 16 Kui nüüd Heroodes nägi, et targad olid teda petnud, siis ta vihastus väga ja läkitas ning laskis tappa kõik poeglapsed Petlemmas ja kogu selle ümbruskonnas, kaheaastased ja nooremad, vastavalt ajale, mida ta tarkadelt oli hoolega uurinud.
\par 17 Siis läks täide, mis on öeldud prohvet Jeremija kaudu:
\par 18 „Raamast on kuuldud häält, palju nuttu ja kaebust, Raahel nutab oma lapsi ega taha lasta ennast trööstida, sest neid ei ole enam!”

\section*{Asumine Naatsaretti}

\par 19 Aga kui Heroodes oli surnud, vaata, siis ilmus Issanda ingel Joosepile unes Egiptuses
\par 20 ning ütles: „Tõuse ning võta lapsuke ja tema ema enesega ja mine Iisraelimaale; sest need on surnud, kes püüdsid lapsukese hinge!”
\par 21 Siis ta tõusis ning võttis lapsukese ja ta ema enesega ja tuli Iisraelimaale.
\par 22 Aga kui ta kuulis, et Arhelaos valitseb Judead oma isa Heroodese asemel, siis ta kartis sinna minna. Kuid ta sai unes Jumalalt käsu ja siirdus Galilea aladele
\par 23 ja tuli ning asus elama linna, mida kutsutakse Naatsaretiks, et läheks täide, mis on öeldud prohvetite kaudu: „Teda peab hüütama Naatsaretlaseks!”


\chapter{3}

\section*{Ristija Johannese jutlus}

\par 1 Neil päevil tuli Ristija Johannes ja kuulutas Judea kõrbes
\par 2 ning ütles: „Parandage meelt, sest taevariik on lähedal!”
\par 3 Tema on nimelt see, kellest prohvet Jesaja on rääkinud, öeldes: „Hüüdja hääl on kõrbes: valmistage Issanda teed, õgvendage tema teerajad!”
\par 4 Aga Johannesel oli riie kaameli karvust ja nahkvöö vööl; ja tema toit oli rohutirtsud ja metsmesi.
\par 5 Siis läks välja tema juurde Jeruusalemm ja kogu Judea ja kõik Jordani ümberkaudne maa;
\par 6 ja ta ristis nad Jordani jões ning nad tunnistasid oma patud.
\par 7 Aga nähes palju varisere ja sadusere tulevat ristimisele, ütles ta neile: „Te rästikute sigitis, kes on teile märku andnud põgeneda tulevase viha eest?
\par 8 Seepärast kandke õiget meeleparanduse vilja
\par 9 ja ärge arvake võivat öelda iseenestes: meil on Aabraham isaks; sest ma ütlen teile, et Jumal võib neist kividest äratada Aabrahamile lapsi.
\par 10 Ent kirves on juba puude juure küljes; iga puu nüüd, mis ei kanna head vilja, raiutakse maha ja visatakse tulle.
\par 11 Mina ristin teid küll veega meeleparanduseks, aga kes tuleb pärast mind, on vägevam minust ja mina ei ole kõlvuline temale jalatseidki kandma, tema ristib teid Püha Vaimu ja tulega.
\par 12 Ta visklabidas on tema käes ja ta puhastab oma rehealuse ja kogub oma nisud aita, kuid aganad ta põletab ära kustutamatu tulega.”

\section*{Jeesuse ristimine}

\par 13 Siis tuli Jeesus Galileast Jordani äärde Johannese juurde, et see teda ristiks.
\par 14 Kuid Johannes peatas teda öeldes: „Minule on vaja, et sina mind ristiksid, ja sina tuled minu juurde!”
\par 15 Jeesus vastas ning ütles temale: „Olgu nüüd nii; sest nõnda on meie kohus täita kõike õigust.” Siis Johannes andis temale järele.
\par 16 Kui nüüd Jeesus oli ristitud, tuli ta sedamaid veest välja. Ja ennäe, taevad avanesid ja ta nägi Jumala Vaimu laskuvat alla kui tuvi ja tulevat tema peale.
\par 17 Ja vaata, taevast kostis hääl, mis ütles: „See on minu armas Poeg, kellest mul on hea meel!”


\chapter{4}

\section*{Jeesust kiusatakse kõrbes}

\par 1 Siis Vaim viis Jeesuse kõrbe kuradi kiusata.
\par 2 Ja kui ta oli paastunud nelikümmend päeva ja nelikümmend ööd, tuli temale viimaks nälg kätte.
\par 3 Ja kiusaja tuli ta juurde ning ütles temale: „Kui sa oled Jumala Poeg, siis ütle, et need kivid leibadeks saaksid!”
\par 4 Aga tema vastas ning ütles: „Kirjutatud on: inimene ei ela ükspäinis leivast, vaid igaühest sõnast, mis lähtub Jumala suust!”
\par 5 Siis kurat võttis tema enesega pühasse linna ja asetas ta pühakoja harjale
\par 6 ja ütles temale: „Oled sa Jumala Poeg, siis kukuta ennast alla, sest kirjutatud on: tema annab oma inglitele käsu sinu pärast, ja nemad kannavad sind kätel, et sa oma jalga vastu kivi ei tõukaks!”
\par 7 Jeesus ütles temale: „Taas on kirjutatud: ära kiusa Issandat, oma Jumalat!”
\par 8 Jällegi võttis kurat ta enesega väga kõrgele mäele ja näitas talle kõik maailma kuningriigid ja nende hiilguse
\par 9 ning ütles temale: „Selle kõik ma annan sinule, kui sa maha langed ja mind kummardad!”
\par 10 Siis Jeesus ütles temale: „Tagane minust, saatan! Sest on kirjutatud: sina pead Issandat, oma Jumalat kummardama ja ükspäinis teda teenima!”
\par 11 Siis kurat jättis ta rahule. Ja vaata, ingleid tuli tema juurde, ja need teenisid teda.

\section*{Jeesus asub Kapernauma}

\par 12 Aga kui Jeesus kuulis, et Johannes oli pandud vangi, siis ta läks ära Galileasse.
\par 13 Ja jättes maha Naatsareti tuli ta ja asus Kapernauma, mis on mererannas Sebuloni ja Naftali aladel,
\par 14 et läheks täide, mis on öeldud prohvet Jesaja kaudu:
\par 15 „Sebulonimaa ja Naftalimaa; mereäärne tee, maa sealpool Jordanit, paganate Galilea -
\par 16 rahvas, kes istub pimeduses, näeb suurt valgust, ja surma maal ja varjus istujaile, neile tõuseb valgus!”

\section*{Jeesus hakkab kuulutama}

\par 17 Sellest ajast alates Jeesus hakkas kuulutama ning ütlema: „Parandage meelt, sest taevariik on lähedal!”

\section*{Jeesus kutsub oma esimesed jüngrid}

\par 18 Aga Galilea mere rannal kõndides ta nägi kaht venda, Siimonat, keda nimetatakse Peetruseks, ja Andreast, tema venda, võrku merre heitvat. Sest nad olid kalamehed.
\par 19 Ja ta ütles neile: „Tulge minu järele ja ma teen teid inimeste püüdjaiks!”
\par 20 Nemad jätsid kohe võrgud maha ning järgisid teda.
\par 21 Ja kui ta läks sealt eemale, nägi ta teist kaht venda, Jakoobust, Sebedeuse poega, ja Johannest, ta venda paadis ühes nende isa Sebedeusega parandamas oma võrke. Ja ta kutsus neid.
\par 22 Ja nad jätsid kohe maha paadi ja oma isa ning järgisid teda.

\section*{Jeesus teeb haigeid terveks}

\par 23 Ja Jeesus käis läbi kogu Galilea, õpetades nende kogudusekodades ja kuulutades kuningriigi evangeeliumi ja parandades igasugust tõbe ja vigadust rahva seas.
\par 24 Ja jutt temast levis üle kogu Süüria. Ja tema juurde toodi kõik vaevalised, mõnesuguse haiguse ja tõve põdejad, seestunud, kuutõbised ja halvatud. Ja ta tegi nad terveks.
\par 25 Ja palju rahvast järgis teda Galileast ja Dekapolist ja Jeruusalemmast ja Judeast ja sealtpoolt Jordanit.


\chapter{5}

\section*{Mäejutlus}

\par 1 Nähes rahvahulki, läks ta üles mäele. Ja kui ta oli maha istunud, tulid ta jüngrid tema juurde.
\par 2 Ja ta avas oma suu, õpetas neid ning ütles:

\section*{Õndsakskiitmised}

\par 3 „Õndsad on vaimust vaesed, sest nende päralt on taevariik.
\par 4 Õndsad on kurvad, sest nemad trööstitakse.
\par 5 Õndsad on tasased, sest nemad pärivad maa.
\par 6 Õndsad on need, kellel nälg ja janu on õiguse järele, sest nemad rahuldatakse.
\par 7 Õndsad on armulised, sest nemad saavad armu.
\par 8 Õndsad on puhtad südamelt, sest nemad saavad näha Jumalat.
\par 9 Õndsad on rahunõudjad, sest neid hüütakse Jumala lasteks.
\par 10 Õndsad on need, keda taga kiusatakse õiguse pärast, sest nende päralt on taevariik.
\par 11 Õndsad olete teie, kui inimesed teid minu pärast laimavad ja taga kiusavad ja valetades räägivad teist kõiksugust kurja.
\par 12 Olge rõõmsad ja ilutsege, sest teie palk on suur taevas; samuti on nad taga kiusanud prohveteid enne teid.

\section*{Maa sool ja maailma valgus}

\par 13 Teie olete maa sool. Aga kui sool tuimub, millega saab teda teha soolaseks? Ta ei kõlba enam millekski kui välja visata ja inimeste tallata.
\par 14 Teie olete maailma valgus. Ei saa jääda varjule linn, mis asetseb mäe otsas.
\par 15 Ei süüdata ka küünalt ega panda seda vaka alla, vaid küünlajalale ja see paistab kõikidele, kes majas on.
\par 16 Nõnda paistku teie valgus inimeste ees, et nad näeksid teie häid tegusid ja annaksid au teie Isale, kes on taevas.

\section*{Jeesus ja käsuõpetus}

\par 17 Ärge arvake, et ma olen tulnud tühistama käsuõpetust või prohveteid; ma ei ole tulnud neid tühistama, vaid täitma.
\par 18 Sest tõesti ma ütlen teile, kuni kaob taevas ja maa, ei kao käsuõpetusest mitte ühtki tähekest või ühtki märgikest, enne kui kõik on sündinud.
\par 19 Kes nüüd iganes tühistab ühe neist vähimaist käskudest ja nõnda õpetab inimesi, teda hüütakse vähimaks taevariigis; aga kes seda mööda teeb ja nõnda õpetab, teda hüütakse suureks taevariigis.
\par 20 Sest ma ütlen teile, kui teie õigus pole palju parem kirjatundjate ja variseride omast, siis te ei saa taevariiki.

\section*{Viha ja mõrv}

\par 21 Te olete kuulnud, et muistsele põlvele on öeldud: sa ei tohi tappa! ja kes iganes tapab, kuulub kohtu alla.
\par 22 Kuid mina ütlen teile, et igaüks, kes on oma vennale vihane, kuulub kohtu alla; aga kes iganes oma vennale ütleb „raka!” kuulub Suurkohtu alla; aga kes ütleb „Sa jõle!” kuulub põrgutulle.
\par 23 Sellepärast kui sa oma andi tood altarile ja seal meenub sulle, et su vennal on midagi sinu vastu,
\par 24 siis jäta oma and sinna altari ette ja mine lepi enne oma vennaga ja siis tule ja too oma and.
\par 25 Ole varsti järeleandlik oma vastasele, niikaua kui sa temaga teel oled, et vastane sind ei annaks kohtuniku kätte ja kohtunik sind ei annaks sulase kätte ja sind ei pandaks vangi.
\par 26 Tõesti ma ütlen sulle, sa ei pääse sealt, enne kui oled maksnud ära viimse veeringu!

\section*{Abielurikkumine ja abielulahutus}

\par 27 Te olete kuulnud, et on öeldud: sa ei tohi abielu rikkuda!
\par 28 Aga mina ütlen teile, et igaüks, kes naise peale vaatab teda himustades, on juba abielu rikkunud temaga oma südames.
\par 29 Ent kui su parem silm sind pahandab, siis kisu ta välja ja heida enesest ära, sest sulle on parem, et üks sinu liikmeist hukkub kui et kogu su ihu heidetakse põrgusse.
\par 30 Ja kui sinu parem käsi sind pahandab, siis raiu ta maha ja heida enesest ära, sest sulle on parem, et üks sinu liikmeist hukkub kui et kogu su ihu läheb põrgusse.
\par 31 Ka on öeldud, et kes iganes oma naise enesest lahutab, see andku temale lahutuskiri.
\par 32 Aga mina ütlen teile, et igaüks, kes oma naise enesest lahutab muidu kui hooruse pärast, see teeb, et naisega abielu rikutakse, ja kes iganes lahutatud naisega abiellub, rikub abielu.

\section*{Vanne ja tõearmastus}

\par 33 Taas te olete kuulnud, et muistsele põlvele on öeldud: sa ei tohi valet vanduda! ja: pea Issandale oma vanded!
\par 34 Aga mina ütlen teile: ärge üldse vanduge, ei taeva juures, sest see on Jumala aujärg;
\par 35 ega maa juures, sest see on tema jalgealune järg; ega Jeruusalemma juures, sest see on suure Kuninga linn.
\par 36 Ära vannu ka oma pea juures, sest sina ei või ühtki juuksekarva teha valgeks ega mustaks;
\par 37 vaid teie kõne olgu: jah, jah, või: ei, ei; aga mis üle selle on, see on kurjast.

\section*{Kättemaksmisest}

\par 38 Te olete kuulnud, et on öeldud: silm silma vastu ja hammas hamba vastu.
\par 39 Aga mina ütlen teile: ärge pange vastu kurjale, vaid kui keegi sind lööb vastu su paremat kõrva, siis kääna temale ka teine;
\par 40 ja sellele, kes tahab sinuga kohut käia ja võtta su vammuse, jäta ka kuub;
\par 41 ja kes sind sunnib kaasas käima ühe penikoorma, sellega mine kaks.
\par 42 Anna sellele, kes sult palub ja ära käändu kõrvale sellest, kes sult tahab laenata.

\section*{Ligimese ja vaenlase armastamisest}

\par 43 Te olete kuulnud, et on öeldud: armasta oma ligimest ja vihka oma vaenlast.
\par 44 Aga mina ütlen teile: armastage oma vaenlasi ja palvetage nende eest, kes teid taga kiusavad,
\par 45 et te saaksite oma Isa lasteks, kes on taevas, sest tema laseb oma päikest tõusta kurjade ja heade üle ja laseb vihma sadada õigete ja ülekohtuste peale.
\par 46 Sest kui te armastate neid, kes teid armastavad, mis palka te saate? Eks tölneridki tee sedasama?
\par 47 Ja kui te lahkesti tervitate ainult oma vendi, mida iseäralikku te siis teete? Eks paganadki tee sedasama?
\par 48 Teie olge siis täiuslikud, nõnda nagu teie taevane Isa on täiuslik.


\chapter{6}

\section*{Armastuseandide jagamisest}

\par 1 Hoidke, et te armastuseande ei jaga inimeste nähes selleks, et nemad teid vaatleksid; muidu ei ole teil palka oma Isalt, kes on taevas.
\par 2 Seepärast, kui sa armastuseande annad, siis ära lase enese ees sarve puhuda, nagu silmakirjatsejad teevad kogudusekodades ja tänavail, et inimesed neid ülistaksid. Tõesti ma ütlen teile, et neil on oma palk käes!
\par 3 Vaid kui sina armastuseande annad, siis ärgu su vasak käsi teadku, mida su parem käsi teeb,
\par 4 et su armastuseannid oleksid salajas; ja su Isa, kes näeb salajasse, tasub sinule.

\section*{Õige palvetamine}

\par 5 Ja kui te palvetate, siis ärge olge nagu silmakirjatsejad; sest need armastavad palvetada kogudusekodades ja tänavate nurkadel, et nad paistaksid inimestele silma. Tõesti ma ütlen teile, et neil on oma palk käes!
\par 6 Ent sina, kui sa palvetad, siis mine oma kambrisse ja sule uks, ja palu oma Isa, kes on salajas, ja su Isa, kes näeb salajasse, tasub sinule.
\par 7 Aga kui te palvetate, siis ärge palju lobisege, nõnda nagu paganad, sest nad arvavad, et neid kuuldakse nende paljude sõnade tõttu.
\par 8 Ärge siis saage nende sarnaseks, sest Jumal, teie Isa, teab, mida te vajate, enne kui te teda palute.

\section*{Meie Isa palve}

\par 9 Teie palvetage siis nõnda: meie Isa, kes oled taevas! Pühitsetud olgu sinu nimi;
\par 10 sinu riik tulgu; sinu tahtmine sündigu nagu taevas, nõnda ka maa peal;
\par 11 meie igapäevane leib anna meile tänapäev;
\par 12 ja anna meile andeks meie võlad, nagu meiegi andeks anname oma võlglastele;
\par 13 ja ära saada meid kiusatusse, vaid päästa meid kurjast; [sest sinu on riik ja vägi ja au igavesti! Aamen.]
\par 14 Sest kui te annate andeks inimestele nende eksimused, siis annab teie taevane Isa ka teile andeks.
\par 15 Aga kui te inimestele nende eksimusi andeks ei anna, siis ei anna ka teie Isa teie eksimusi andeks.

\section*{Õige paastumine}

\par 16 Aga kui te paastute, siis ärge jääge kurvanäoliseks, nõnda nagu silmakirjatsejad; sest nad teevad oma palge näotuks, et rahvas näeks neid paastuvat. Tõesti, ma ütlen teile, neil on oma palk käes!
\par 17 Vaid kui sina paastud, siis võia oma pea ja pese oma pale,
\par 18 et su paastumine ei oleks nähtav inimestele, vaid su Isale, kes on salajas. Ja su Isa, kes näeb salajasse, tasub sinule.

\section*{Tõeline varandus}

\par 19 Ärge koguge endile varandusi maa peale, kus koi ja rooste rikuvad ja kus vargad sisse murravad ning varastavad.
\par 20 Vaid koguge endile varandusi taevasse, kus koi ega rooste ei riku ja kus vargad sisse ei murra ega varasta.
\par 21 Sest kus su varandus on, seal on ka su süda!

\section*{Valgus ja pimedus}

\par 22 Ihu küünal on silm; kui su silm on terve, siis on kõik su ihu valguses.
\par 23 Aga kui su silm on rikkis, siis on kogu su ihu pimeduses. Kui nüüd su valgus, mis on sinus, on pimedus, kui suur on siis pimedus?

\section*{Mure ja lootus Jumala peale}

\par 24 Ükski ei või teenida kaht isandat, sest tema kas vihkab üht ja armastab teist või hoiab ühe poole ega hooli teisest. Te ei või teenida Jumalat ja mammonat!
\par 25 Sellepärast ma ütlen teile: ärge olge mures oma elu pärast, mida süüa ja mida juua, ega oma ihu pärast, millega riietuda. Eks elu ole enam kui toidus ja ihu enam kui riided?
\par 26 Pange tähele taeva linde, nad ei külva ega lõika ega pane kokku aitadesse ja teie taevane Isa toidab neid. Eks teie ole palju enam kui nemad?
\par 27 Aga kes teie seast võib muretsemisega oma pikkusele ühegi küünra jätkata?
\par 28 Ja miks te muretsete riietuse pärast? Pange tähele lilli väljal, kuidas nad kasvavad; nad ei tee tööd ega ketra.
\par 29 Ometi ma ütlen teile, et Saalomongi kõiges oma hiilguses pole olnud nõnda ehitud kui üks nendest!
\par 30 Kui nüüd Jumal rohtu väljal, mis täna on ja homme ahju visatakse, nõnda ehib, kas siis mitte palju enam teid, teie nõdrausulised?
\par 31 Ärge siis olge mures, küsides, mida me sööme? või mida me joome? või millega me riietume?
\par 32 Sest kõike seda taotlevad paganad. Teie taevane Isa teab ju, et te seda kõike vajate.
\par 33 Ent otsige esiti Jumala riiki ja tema õigust, siis seda kõike antakse teile pealegi!
\par 34 „Ärge siis olge mures homse pärast, sest küll homne päev muretseb enese eest. Igale päevale saab küllalt omast vaevast!


\chapter{7}

\section*{Kohtumõistmisest teiste üle}

\par 1 Ärge mõistke kohut, et teie üle ei mõistetaks kohut.
\par 2 Sest missuguse kohtuga te kohut mõistate, niisugusega mõistetakse teile kohut; ja missuguse mõõduga te mõõdate, niisugusega mõõdetakse ka teile.
\par 3 Aga miks sa näed pindu oma venna silmas, kuid palki omas silmas sa ei pane tähele?
\par 4 Või kuidas sa ütled oma vennale: lase ma tõmban pinnu sinu silmast!? Ja vaata, palk on su omas silmas!
\par 5 Sa silmakirjatseja! Tõmba esiti palk omast silmast ja siis sa seletad tõmmata pindu oma venna silmast.
\par 6 Ärge andke seda, mis on püha, koertele, ja ärge heitke oma pärleid sigade ette, et nad neid ei sõtkuks oma jalgadega ega käänduks ja teid ei kisuks.

\section*{Julgustus palveks}

\par 7 Paluge, siis antakse teile; otsige, siis te leiate; koputage, siis avatakse teile.
\par 8 Sest igaüks, kes palub, see saab ja kes otsib, see leiab, ja kes koputab, sellele avatakse.
\par 9 Või missugune inimene on teie seast, kellelt tema poeg palub leiba ja ta annaks temale kivi?
\par 10 Või kui ta palub kala ja ta annaks temale mao?
\par 11 Kui nüüd teie, kes olete kurjad, mõistate anda häid ande oma lastele, eks palju enam teie Isa taevas anna häid ande neile, kes teda paluvad.

\section*{Ülim käsk}

\par 12 Sellepärast kõik, mida te tahate, et inimesed teile teevad, tehke ka neile; sest see on käsuõpetus ja prohvetid.

\section*{Kaks teed}

\par 13 Minge sisse kitsast väravast; sest avar on värav ja lai on tee, mis viib hukatusse, ja palju on neid, kes sealt sisse lähevad.
\par 14 Ja kitsas on värav ja ahtake on tee, mis viib ellu, ja pisut on neid, kes selle leiavad.

\section*{Valeprohvetite tundmisest}

\par 15 Hoiduge valeprohveteist, kes tulevad teie juurde lammaste riideis, aga seestpidi on nad kiskjad hundid.
\par 16 Nende viljast te tunnete nad. Ega viinamarju nopita kibuvitstest või viigimarju ohakatest?
\par 17 Nõnda iga hea puu kannab head vilja, aga halb puu kannab halba vilja.
\par 18 Hea puu ei või kanda halba vilja ega halb puu kanda head vilja.
\par 19 Iga puu, mis ei kanna head vilja, raiutakse maha ja visatakse tulle.
\par 20 Nõnda siis nende viljast te tunnete nad ära.

\section*{Sõnad ja teod}

\par 21 Mitte igaüks, kes minule ütleb: Issand, Issand, ei saa taevariiki, vaid kes teeb mu Isa tahtmist, kes on taevas.
\par 22 Mitmed ütlevad minule tol päeval: Issand, Issand, kas me ei ole sinu nimel ennustanud ja sinu nimel ajanud välja kurje vaime ja sinu nimel teinud palju vägevaid tegusid?
\par 23 Ja siis ma tunnistan neile: ma ei ole elades teid tundnud, taganege minust, te ülekohtutegijad!

\section*{Kaks koja ehitajat}

\par 24 Igaüks nüüd, kes neid mu sõnu kuuleb ja teeb nende järgi, on võrreldav mõistliku mehega, kes ehitas oma koja kaljule.
\par 25 Ja ränk sadu tuli ja tulid vetevoolud ja tuuled puhusid ja sööstsid vastu seda koda; aga ta ei langenud, sest tema alus oli rajatud kaljule.
\par 26 Ja igaüks, kes neid minu sõnu kuuleb, aga ei tee nende järgi, on võrreldav jõleda mehega, kes oma koja ehitas liivale.
\par 27 Ja ränk sadu tuli ja tulid vetevoolud ja tuuled puhusid ja sööstsid vastu seda koda, ja ta langes ja tema langemine oli suur.”
\par 28 Ja sündis, kui Jeesus oli lõpetanud need kõned, et rahvahulgad hämmastusid tema õpetusest;
\par 29 sest ta õpetas neid nõnda nagu see, kellel on meelevald, ja mitte nõnda nagu nende kirjatundjad.


\chapter{8}

\section*{Pidalitõbise tervekstegemine}

\par 1 Aga kui tema astus mäelt alla, järgis teda palju rahvast.
\par 2 Ja vaata, pidalitõbine tuli ja heitis maha tema ette ning ütles: „Issand, kui sa tahad, võid sa mind puhtaks teha!”
\par 3 Siis Jeesus sirutas oma käe ja puudutas teda ning ütles: „Ma tahan, saa puhtaks!” Ja sedamaid sai ta oma pidalitõvest puhtaks.
\par 4 Ja Jeesus ütles temale: „Katsu, et sa seda ühelegi ei ütle, vaid mine näita ennast preestritele ja vii ohvriand, mille Mooses on seadnud, neile tunnistuseks!”

\section*{Kapernauma pealiku poisi tervekstegemine}

\par 5 Aga kui Jeesus saabus Kapernauma, tuli tema juurde sõjapealik, palus teda
\par 6 ning ütles: „Issand, minu poiss on kodus halvatuna maas suures valus!”
\par 7 Jeesus ütles temale: „Ma tulen ja teen ta terveks.”
\par 8 Aga sõjapealik kostis ja ütles: „Issand, mina pole väärt, et sa mu katuse alla tuled, vaid ütle ainult sõna, siis mu poiss paraneb.
\par 9 Sest ma isegi olen inimene valitsuse all, ja minu käsu all on sõjamehi. Ja kui ma ütlen ühele: mine! siis ta läheb ja teisele: tule! siis ta tuleb ja oma orjale: tee seda! siis ta teeb.”
\par 10 Kui Jeesus seda kuulis, imestas ta ja ütles neile, kes teda järgisid: „Tõesti ma ütlen teile, mitte üheltki Iisraelis ma ei ole leidnud nii suurt usku!
\par 11 Ent ma ütlen teile, et paljud tulevad idast ja läänest ja istuvad lauas Aabrahami, Iisaki ja Jaakobiga taevariigis,
\par 12 aga kuningriigi lapsed heidetakse välja kõige äärmisemasse pimedusse; seal on ulumine ja hammaste kiristamine!”
\par 13 Ja Jeesus ütles sõjapealikule: „Mine! Nagu sa oled uskunud, nõnda sündigu sulle!” Ja tema poiss sai terveks selsamal tunnil.

\section*{Jeesus teeb terveks Peetruse ämma ja teisi}

\par 14 Ja Jeesus tuli Peetruse kotta ja nägi tema ämma olevat maas palavikus.
\par 15 Siis Jeesus võttis ta käest kinni ja palavik lahkus temast. Ja ämm tõusis üles ja ümmardas teda.
\par 16 Aga õhtu tulles toodi tema juurde palju seestunuid; ja ta ajas vaimud välja sõnaga ja tegi haiged terveks,
\par 17 et läheks täide, mis on öeldud prohvet Jesaja kaudu: „Tema võttis enese peale meie haigused ja kandis meie tõved!”

\section*{Jüngriks saada tahtjad valiku ees}

\par 18 Aga kui Jeesus nägi rahvahulka enese ümber olevat, käskis ta minna teisele poole järve.
\par 19 Ja tema juurde tuli üks kirjatundja ja ütles temale: „Õpetaja, ma tahan sind järgida, kuhu sa iganes lähed!”
\par 20 Siis Jeesus ütles temale: „Rebastel on augud ja taeva lindudel on pesad, kuid Inimese Pojal ei ole, kuhu ta oma pea võiks panna!”
\par 21 Ka keegi teine jüngritest ütles temale: „Issand, luba mind enne minna ja matta oma isa.”
\par 22 Aga Jeesus ütles temale: „Järgi mind ja lase surnuid oma surnud matta!”

\section*{Jeesus vaigistab maru}

\par 23 Ja kui tema astus paati, läksid ta jüngrid tema järel.
\par 24 Ja vaata, suur maru tõusis merel, nõnda et paat lainetega kaeti. Aga tema magas.
\par 25 Siis jüngrid tulid tema juurde, äratasid ta üles ja ütlesid: „Issand aita! Me hukkume!”
\par 26 Tema ütles neile: „Miks te olete arad, te nõdrausulised?” Siis ta tõusis üles ja sõitles tuult ja merd. Ja meri jäi täiesti vaikseks.
\par 27 Aga inimesed imestasid ja ütlesid: „Mis mees see küll on, et isegi tuuled ja meri kuulevad tema sõna!”

\section*{Gadara seestunud}

\par 28 Ja kui ta jõudis teisele poole järve gadaralaste maale, kohtasid teda kaks haudadest tulnud seestunut, kes olid väga hirmsad, nii et ükski ei saanud käia seda teed.
\par 29 Ja vaata, nemad kisendasid ning ütlesid: „Mis on sul meiega tegemist, sa Jumala Poeg? Oled sa siia tulnud meid enneaegu vaevama?”
\par 30 Aga kaugel neist oli suur seakari söömas.
\par 31 Ja kurjad vaimud palusid teda ning ütlesid: „Kui sa meid välja ajad, siis läheta meid seakarja sisse!”
\par 32 Ja ta ütles neile: „Minge!” Siis nad läksid sigade sisse. Ja vaata, kogu kari kukutas enese ülepeakaela kaldalt merre ja suri vette.
\par 33 Aga karjased põgenesid ja läksid linna ning andsid teada kõik ja mis oli sündinud seestunutega.
\par 34 Ja vaata, kogu linn läks välja Jeesusele vastu. Ja kui nad teda nägid, palusid nad teda, et ta nende rajadest ära läheks.


\chapter{9}

\section*{Halvatu tervekstegemine}

\par 1 Ja ta astus paati ja tuli teisele poole järve ning saabus oma linna.
\par 2 Ja vaata, nad kandsid tema juurde halvatu, kes oli voodis maas. Kui Jeesus nende usku nägi, ütles ta halvatule: „Ole julge, mu poeg, sinu patud antakse sulle andeks!”
\par 3 Ja vaata, mõned kirjatundjaist ütlesid isekeskis: „See pilkab Jumalat.”
\par 4 Aga kui Jeesus nende mõtteid nägi, ütles ta: „Mispärast te mõtlete kurja oma südames?
\par 5 Sest mis on kergem öelda: su patud antakse sulle andeks, või öelda: tõuse üles ja kõnni?
\par 6 Aga et te teaksite, et Inimese Pojal on meelevald maa peal anda patud andeks, siis ta ütleb halvatule: tõuse üles, võta oma voodi ja mine koju!”
\par 7 Ja see tõusis üles ja läks koju.
\par 8 Aga kui rahvahulgad seda nägid, panid nad seda imeks ja andsid Jumalale au, kes annab inimestele niisuguse meelevalla.

\section*{Matteuse kutsumine}

\par 9 Ja sealt mööda minnes nägi Jeesus meest, Matteus nimi, tolli juures istuvat ja ütles temale: „Järgi mind!” Ja see tõusis ja järgis teda.
\par 10 Ja kui ta lauas istus tema kojas, vaata, siis tuli palju tölnereid ja patuseid ning istusid lauas ühes Jeesuse ja ta jüngritega.
\par 11 Seda nähes ütlesid variserid ta jüngritele: „Mispärast teie õpetaja sööb ühes tölnerite ja patustega!”
\par 12 Aga kui tema seda kuulis, ütles ta: „Arsti ei vaja terved, vaid haiged.
\par 13 Ent minge ja õppige, mis see on: ma tahan halastust ja mitte ohvrit. Sest ma pole tulnud kutsuma õigeid, vaid patuseid!”

\section*{Paastumise küsimus}

\par 14 Siis tulid Johannese jüngrid tema juurde ja küsisid: „Mispärast meie ja variserid paastume, aga sinu jüngrid ei paastu?”
\par 15 Jeesus ütles neile: „Ega peiupoisid või olla kurvad, niikaua kui peig on nende juures? Ent päevad tulevad, mil peig neilt võetakse ja siis nad paastuvad.
\par 16 Ükski ei pane vanutamata riidetükki paigaks vanale kuuele; sest niisugune augutäidis rebeneb küljest lahti ja lõhe läheb pahemaks.
\par 17 Ei valata ka mitte värsket viina vanadesse nahklähkritesse; muidu lähkrid lõhkevad ja viin voolab maha ning lähkrid saavad hukka; vaid värske viin valatakse uutesse lähkritesse, siis säilivad mõlemad.”

\section*{Ülema tütre ülesäratamine surnuist ja veritõbine naine}

\par 18 Kui tema seda neile rääkis, vaata, siis tuli üks ülem ja langes maha tema ette ning ütles: „Minu tütar suri praegu, aga tule ja pane oma käsi tema peale, siis ta saab elavaks!”
\par 19 Ja Jeesus tõusis ja läks ta järel, samuti ta jüngrid.
\par 20 Ja vaata, naine, kes oli kaksteist aastat veritõbe põdenud, tuli tema selja taha ja puudutas tema kuue palistust.
\par 21 Sest ta ütles iseeneses: „Saaksin ma vaid puudutada tema kuube, siis ma pääseksin!”
\par 22 Aga Jeesus pöördus, nägi teda ja ütles: „Ole julge, tütar, sinu usk on sind päästnud!” Ja naine sai terveks selsamal tunnil.
\par 23 Ja kui Jeesus ülema kotta jõudis ja nägi vilepuhujaid ja käratsevat rahvast,
\par 24 ütles ta: „Minge ära, sest tütarlaps pole surnud, vaid magab!” Ja nad naersid teda.
\par 25 Aga kui rahvahulk oli välja aetud, läks ta sisse ja võttis tema käest kinni, ja tütarlaps tõusis üles.
\par 26 Ja sõnum sellest levis üle kogu selle maa.

\section*{Kaks pimedat}

\par 27 Kui Jeesus sealt edasi läks, järgis teda kaks pimedat, karjudes ning öeldes: „Taaveti Poeg, halasta meie peale!”
\par 28 Kui ta tuppa astus, tulid need pimedad tema juurde, ja Jeesus ütles neile: „Kas te usute, et ma võin seda teha?” Nemad ütlesid temale: „Jah, Issand!”
\par 29 Siis ta puudutas nende silmi ning ütles: „Teile sündigu teie usku mööda!”
\par 30 Ja nende silmad avanesid. Ja Jeesus hoiatas neid kõvasti ning ütles: „Katsuge, et seda keegi ei saa teada!”
\par 31 Aga nemad läksid välja ja tegid tema kuulsaks kogu sellel maal.

\section*{Keeletu seestunu}

\par 32 Kui need olid välja läinud, vaata, siis toodi ta juurde keeletu mees, kes oli seestunud.
\par 33 Ja kui kuri vaim oli välja aetud, rääkis keeletu. Ja rahvas imestas ja ütles: „Seda pole veel iialgi nähtud Iisraelis!”
\par 34 Kuid variserid ütlesid: „Kurjade vaimude ülema abil ajab ta kurje vaime välja.”
\par 35 Ja Jeesus käis läbi kõik linnad ja külad, õpetas nende kogudusekodades ja kuulutas Jumala riigi evangeeliumi, parandas kõike tõbe ja kõike viga.

\section*{Jeesuse kaastunne rahva vastu}

\par 36 Aga kui ta rahvahulki nägi, oli tal hale meel nende pärast, sest nad olid piinatud ja vintsutatud otsekui lambad, kellel ei ole karjast.
\par 37 Siis ta ütles oma jüngritele: „Lõikust on palju, aga vähe töötegijaid.
\par 38 Paluge siis lõikuse Issandat, et ta läkitaks töötegijaid välja oma lõikusele.”


\chapter{10}

\section*{Apostlite läkitamine}

\par 1 Ja ta kutsus oma kaksteist jüngrit enese juurde ja andis neile meelevalla rüvedate vaimude üle, neid välja ajada ja parandada kõike tõbe ja kõike viga.
\par 2 Nende kaheteistkümne apostli nimed on need: esimene Siimon, nimetatud Peetruseks ja Andreas, tema vend; Jakoobus, Sebedeuse poeg ja Johannes, tema vend;
\par 3 Filippus ja Bartolomeus, Toomas ja Matteus, tölner; Jakoobus, Alfeuse poeg, ja Taddeus;
\par 4 Siimon Kaanast ja Juudas Iskariot, kes tema ka ära andis.

\section*{Apostlite ülesanne}

\par 5 Need kaksteist läkitas Jeesus, käskis neid ning ütles: „Ärge minge paganate teele ja ärge astuge samaarlaste linna,
\par 6 vaid minge ennemini Iisraeli koja kadunud lammaste juurde.
\par 7 Ja minnes kuulutage ning öelge: taevariik on lähedal!
\par 8 Tehke haigeid terveks, äratage surnuid üles, puhastage pidalitõbiseid, ajage välja kurje vaime. Muidu olete saanud, muidu andke!
\par 9 Ärge varuge kulda ega hõbedat ega vaskraha oma vöö vahele,
\par 10 ei pauna teekonnale ega kaht kuube ega jalatseid ega saua; sest töötegija on oma toiduse väärt.
\par 11 Aga kui te kuhugi linna või külasse astute, siis kuulake, kes seal on vääriline ja sinna jääge, seni kui sealt ära lähete.
\par 12 Aga sisse astudes majasse teretage seda.
\par 13 Ja kui see maja on selle vääriline, siis tulgu teie rahu ta peale; aga kui see ei ole vääriline, pöördugu teie rahu tagasi teie juurde.
\par 14 Ja kes iganes teid vastu ei võta ega kuule teie sõnu, sellest kojast või sellest linnast minge välja ja puistake tolm oma jalgadelt.
\par 15 Tõesti mina ütlen teile, Soodoma- ja Gomorramaal on kohtupäeval hõlpsam põli kui niisugusel linnal!

\section*{Apostlite tagakiusamisest}

\par 16 Vaata, mina läkitan teid nagu lambaid huntide keskele! Olge siis arukad nagu maod ja vagurad nagu tuvid!
\par 17 Aga hoiduge inimestest; sest nad annavad teid ära kohtute kätte ja piitsutavad teid kogudusekodades.
\par 18 Ja teid viiakse minu pärast ka maavalitsejate ja kuningate ette, neile ja paganaile tunnistuseks.
\par 19 Ja kui nad teid annavad kohtu kätte, siis ärge muretsege, kuidas või mida rääkida, sest teile antakse sel tunnil see, mida teil tuleb rääkida.
\par 20 Sest teie ei ole need, kes kõnelevad, vaid see on teie Isa Vaim, kes kõneleb teie sees.
\par 21 Siis vend annab venna surma ja isa oma lapse ja lapsed hakkavad vastu vanemaile ja tapavad nad.
\par 22 Ja te saate kõikide vihaalusteks minu nime pärast; aga kes jääb püsima otsani, see päästetakse.
\par 23 Aga kui nad teid taga kiusavad ühes linnas, siis põgenege teise. Sest tõesti ma ütlen teile, te ei jõua Iisraeli linnu õpetades läbi käia, enne kui Inimese Poeg tuleb!
\par 24 Jünger ei ole ülem oma õpetajast ega ori ülem oma isandast.
\par 25 Jünger olgu rahul sellega, et tema käsi käib nagu ta õpetajal ja orjal nagu ta isandal. Kui nad pereisandat on hüüdnud Peeltsebuliks, kui palju enam tema peret!

\section*{Keda karta, keda mitte}

\par 26 Ärge siis neid kartke! Sest midagi ei ole peidetud, mis ei tuleks ilmsiks, ja midagi pole salajas, mida ei saada teada.
\par 27 Mis ma teile ütlen pimedas, seda rääkige valges; ja mida te kuulete kõrva sisse räägitavat, seda kuulutage katustelt.
\par 28 Ja ärge kartke neid, kes ihu tapavad, aga hinge ei või tappa, vaid kartke ennemini teda, kes hinge ja ihu võib hukutada põrgus.
\par 29 Eks kaks varblast müüda veeringu eest? Ent ükski neist ei lange maha ilma teie Isata.
\par 30 Aga teie juuksekarvadki on kõik ära loetud.
\par 31 Ärge siis kartke; teie olete kallihinnalisemad kui palju varblasi!
\par 32 Igaüks nüüd, kes mind tunnistab inimeste ees, teda tunnistan minagi oma Isa ees, kes on taevas.
\par 33 Aga kes iganes mind ära salgab inimeste ees, teda salgan minagi oma Isa ees, kes on taevas!

\section*{Karm valik ja selle tasu}

\par 34 Ärge arvake, et ma olen tulnud tooma rahu maa peale; ma ei ole tulnud tooma rahu, vaid mõõka!
\par 35 Sest ma olen tulnud ajama inimest riidu tema isaga ja tütart tema emaga ja miniat tema ämmaga;
\par 36 ja inimese vaenlasiks saavad ta omad kodakondsed.
\par 37 Kes isa või ema enam armastab kui mind, see ei ole mind väärt; ja kes poega või tütart enam armastab kui mind, see ei ole mind väärt,
\par 38 ja kes ei võta oma risti enese peale ega järgi mind, ei ole mind väärt.
\par 39 Kes oma elu leiab, kaotab selle, ja kes oma elu kaotab minu pärast, leiab selle.
\par 40 Kes teid vastu võtab, võtab mind vastu; ja kes mind vastu võtab, võtab vastu selle, kes mind on läkitanud.
\par 41 Kes prohveti vastu võtab ta prohvetinime tõttu, saab prohveti palga; ja kes õige vastu võtab õige inimese nime tõttu, see saab õige inimese palga.
\par 42 Ja kes iganes üht neist vähemaist joodab karikatäie külma veega ta jüngri nime tõttu, tõesti ma ütlen teile, see ei jää ilma oma palgast!”


\chapter{11}

\section*{Karm valik ja selle tasu}

\par 1 Ja sündis, kui Jeesus oli lõpetanud käskude andmise oma kaheteistkümnele jüngrile, et ta läks sealt eemale õpetama ja jutlustama nende linnades.

\section*{Ristija Johannes}

\par 2 Aga kui Johannes vangihoones olles kuulis Kristuse tegudest, läkitas ta oma jüngrite kaudu
\par 3 temale ütlema: „Kas oled sina see, kes tuleb, või peame teist ootama?”
\par 4 Ja Jeesus vastas ning ütles neile: „Minge ja teatage Johannesele, mida te kuulete ja näete:
\par 5 pimedad saavad nägijaiks ja jalutud käivad, pidalitõbised tehakse puhtaks ja kurdid kuulevad, ja surnud äratatakse üles ja vaestele kuulutatakse evangeeliumi,
\par 6 ja õnnis on, kes iganes minust ei pahandu!”
\par 7 Aga kui need ära läksid, hakkas Jeesus rahvale rääkima Johannesest: „Mida te olete läinud välja kõrbe vaatama? Kas pilliroogu, mida tuul kõigutab?
\par 8 Või mida te olete välja läinud vaatama? Kas inimest, kes on riietatud peente riietega? Vaata, kes kannavad peeni riideid, need on kuningate kodades.
\par 9 Või mida te olete välja läinud vaatama? Kas prohvetit? Tõesti, ma ütlen teile, et ta on enamgi kui prohvet.
\par 10 Tema on see, kellest on kirjutatud: vaata, mina läkitan sinu palge eele oma ingli, kes sulle tee valmistab sinu ees!
\par 11 Tõesti, ma ütlen teile, ei ole naistest sündinute seast tõusnud suuremat kui Ristija Johannes, aga väiksem taevariigis on suurem temast!
\par 12 Ent Ristija Johannese päevist siiamaani rünnatakse taevariiki ja kes seda ründavad, kisuvad ta endile.
\par 13 Ja kõik prohvetid ja käsuõpetus on ennustanud Johanneseni,
\par 14 ja kui te tahate seda tõeks võtta: tema on Eelija, kes pidi tulema.
\par 15 Kellel kõrvad on, see kuulgu!
\par 16 Aga kellega ma võrdleksin seda sugupõlve? See on laste sarnane, kes turgudel istuvad ja hüüavad teistele
\par 17 ning ütlevad: me oleme teile vilet ajanud ja te pole tantsinud; me oleme teile nutulugu laulnud ja te pole nutnud.
\par 18 Sest Johannes tuli, ei söönud ega joonud; ja nad ütlevad: temas on kuri vaim!
\par 19 Inimese Poeg tuli, sööb ja joob; ja nad ütlevad: vaata, see inimene on söödik ja viinajoodik, tölnerite ja patuste sõber! Ometi mõistetakse tarkus õigeks ta tegudest.”

\section*{Jeesus sõitleb uskmatuid linnu}

\par 20 Siis ta hakkas sõitlema neid linnu, kus kõige rohkem oli sündinud tema vägevaid tegusid, sellepärast et nad ei olnud meelt parandanud:
\par 21 „Häda sulle, Korasin! Häda sulle, Betsaida! Sest kui Tüüroses ja Siidonis oleksid sündinud need vägevad teod, mis on sündinud teie juures, küll nemad ammugi oleksid kotis ja tuhas parandanud meelt.
\par 22 Aga ma ütlen teile: Tüürosel ja Siidonil on hõlpsam põli kohtupäeval kui teil!
\par 23 Ja sina, Kapernaum, kas sa ei olnud ülendatud taevani? Kuni põrguni sa langed alla! Sest kui Soodomas oleksid sündinud need vägevad teod, mis on sündinud sinus, seisaks see veel tänapäeval!
\par 24 Aga ma ütlen teile: Soodomamaal on hõlpsam põli kohtupäeval kui sinul!”

\section*{Lapselikust usaldusest}

\par 25 Sel ajal Jeesus hakkas rääkima ja ütles: „Ma ülistan sind, Isa, taeva ja maa Issand, et sa selle oled peitnud tarkade ja mõistlike eest ja oled selle ilmutanud väetitele!
\par 26 Tõesti, Isa, see on nõnda olnud su meele järgi!

\section*{Jeesuse eesmärk ja töö}

\par 27 Kõik on mu Isa andnud minu kätte ja ükski muu ei tunne Poega kui ainult Isa ega ükski tunne Isa kui ainult Poeg ja see, kellele iganes Poeg tahab seda ilmutada.
\par 28 Tulge minu juurde kõik, kes olete vaevatud ja koormatud ja mina annan teile hingamise!
\par 29 Võtke endi peale minu ike ja õppige minust, et mina olen tasane ja südamelt alandlik; ja te leiate hingamise oma hingedele.
\par 30 Sest minu ike on hea ja minu koorem on kerge!”


\chapter{12}

\section*{Hingamispäeva pühitsemisest}

\par 1 Sel ajal läks Jeesus viljast läbi hingamispäeval. Aga tema jüngritel oli nälg ja nad hakkasid päid katkuma ja sööma.
\par 2 Kui variserid seda nägid, ütlesid nad temale: „Vaata, sinu jüngrid teevad, mida ei sünni teha hingamispäeval!”
\par 3 Aga tema ütles neile: „Eks te ole lugenud, mis Taavet tegi, kui temal ja ta kaaslastel nälg oli,
\par 4 kuidas ta läks Jumala kotta ja nad sõid ära vaateleivad, mida süüa ei olnud luba temal ega ta kaaslastel, vaid ainult preestritel?
\par 5 Või eks te ole käsuõpetusest lugenud, et preestrid rikuvad hingamispäeval pühakojas hingamispäeva ja on ometigi süüta?
\par 6 Kuid ma ütlen teile, et siin on see, kes on suurem kui pühakoda!
\par 7 Aga kui te teaksite, mis see on: ma tahan halastust, aga mitte ohvrit! siis te ei oleks hukka mõistnud süütuid.
\par 8 Sest Inimese Poeg on hingamispäeva isand!”

\section*{Kuivanud käega mehe tervekstegemine}

\par 9 Ja kui ta läks sealt eemale, tuli ta nende kogudusekotta.
\par 10 Ja vaata, seal oli inimene kuivanud käega. Ja nad küsisid temalt ning ütlesid: „Kas sünnib terveks teha hingamispäeval?” Nad mõtlesid ta peale kaevata.
\par 11 Aga tema ütles neile: „Kes on teie seast inimene, kellel on üksainus lammas ja kui see kukub auku hingamispäeval, et ta temast ei haara kinni ega tõmba teda välja?
\par 12 Kui suur vahe on nüüd inimese ja lamba vahel! Sellepärast sünnib küll teha head hingamispäeval!”
\par 13 Siis ta ütles sellele inimesele: „Siruta oma käsi!” Ja ta sirutas. Ja see sai jälle terveks otsekui teine.
\par 14 Siis variserid läksid välja ja pidasid nõu tema vastu, kuidas teda hukka saata.

\section*{Valitud sulane}

\par 15 Aga kui Jeesus seda sai teada, läks ta sealt ära. Ja paljud järgisid teda, ja ta tegi nad kõik terveks;
\par 16 ja ta hoiatas neid kõvasti, et nad temast avalikult ei räägiks,
\par 17 et läheks täide, mis on öeldud prohvet Jesaja kaudu:
\par 18 „Vaata, see on mu sulane, kelle ma olen valinud, mu armas, kellest mu hingel on hea meel! Ma panen oma Vaimu tema peale ja ta kuulutab paganaile kohut.
\par 19 Ta ei riidle ega kisenda ja ta häält ei kuulda tänavail.
\par 20 Rudjutud pilliroogu ta ei murra katki ja suitsevat tahti ta ei kustuta ära, kuni ta on õigluse jalule seadnud;
\par 21 ja paganad panevad oma lootuse tema nime peale!”

\section*{Jeesus ei vaja saatana abi}

\par 22 Siis toodi tema juurde seestunu, kes oli pime ja keeletu. Ja ta tegi tema terveks, nõnda et keeletu rääkis ja nägi.
\par 23 Ja kõik rahvas hämmastus ning ütles: „Kas see vahest ei ole Taaveti poeg?”
\par 24 Aga kui variserid seda kuulsid, ütlesid nad: „See ei aja kurje vaime välja muidu kui Peeltsebuli, kurjade vaimude peamehe abil!”
\par 25 Et aga Jeesus nende mõtteid mõistis, ütles ta neile: „Iga kuningriik, mis on isekeskis riius, hävib; ja ükski linn või koda, mis on isekeskis riius, ei jää püsima.
\par 26 Kui nüüd saatan välja ajab saatana, siis on ta riius iseenesega; kuidas siis võib tema riik püsida?
\par 27 Ja kui mina Peeltsebuli abil kurje vaime välja ajan, kelle abil ajavad teie pojad need välja? Sellepärast peavad nemad olema teile kohtumõistjaiks.
\par 28 Aga kui mina Jumala Vaimuga kurjad vaimud välja ajan, siis on juba Jumala riik teie juurde tulnud.
\par 29 Või kuidas võib keegi minna vägeva kotta ja riisuda tema riistu, kui ta enne ei seo seda vägevat? Alles siis ta saab riisuda tema maja.
\par 30 Kes ei ole minuga, see on minu vastu; ja kes minuga ei kogu, see pillab laiali.
\par 31 Seepärast ma ütlen teile: iga patt ja jumalapilge antakse inimestele andeks; aga Püha Vaimu pilget ei anta inimestele andeks.
\par 32 Ja kes iganes räägib sõna Inimese Poja vastu, sellele antakse see andeks, aga kes iganes midagi räägib Püha Vaimu vastu, sellele ei anta seda andeks ei selles maailmas ega tulevases.
\par 33 Kas tunnistage puu heaks ja ta vili heaks või tunnistage puu halvaks ja ta vili halvaks. Sest puud tuntakse tema viljast.
\par 34 Te rästikute sigitis, kuidas te võite rääkida head, kui te olete kurjad? Sellest, mida süda on täis, sellest räägib suu.
\par 35 Hea inimene toob südame heast tagavarast esile head, ja halb inimene toob halvast tagavarast esile halba.
\par 36 Ent ma ütlen teile, et inimesed peavad kohtupäeval aru andma igaühest tühjast sõnast, mis nad on rääkinud.
\par 37 Sest su sõnadest arvatakse sind õigeks ja su sõnadest mõistetakse sind hukka.”

\section*{Jeesus ei hooli tunnustähe nõudmisest}

\par 38 Siis vastasid temale mõningad kirjatundjad ja variserid ning ütlesid: „Õpetaja, me tahame sinult näha tunnustähte.”
\par 39 Aga tema kostis ja ütles neile: „See kuri ja abielurikkuja tõug otsib tunnustähte, aga talle ei anta muud tunnustähte kui prohvet Joona tunnustäht.
\par 40 Sest otsekui Joona oli valaskala kõhus kolm päeva ja kolm ööd, nõnda peab Inimese Poeg olema maapõues kolm päeva ja kolm ööd.
\par 41 Niinive mehed astuvad kohtus esile ühes selle sugupõlvega ja süüdistavad teda, sest nad parandasid meelt Joona jutluse tõttu. Ja vaata, siin on enam kui Joona!
\par 42 Lõunamaa kuninganna tõuseb üles kohtus ühes selle sugupõlvega ja süüdistab teda, sest ta tuli ilmamaa otsast kuulama Saalomoni tarkust. Ja vaata, siin on enam kui Saalomon!

\section*{Puuduliku puhastuse hädaoht}

\par 43 Kui rüve vaim on inimesest väljunud, käib ta kuivi paiku mööda ja otsib hingamist, aga ei leia.
\par 44 Siis ta ütleb: ma lähen tagasi oma kotta, kust ma väljusin! Ja kui ta tuleb, leiab ta selle tühja olevat, pühitud ja ehitud.
\par 45 Siis ta läheb ja võtab enesega kaasa teist seitse vaimu, kes on temast kurjemad; ja kui nad sisse tulevad, elavad nad seal. Ja selle inimese viimne lugu läheb pahemaks kui esimene. Nõnda käib ka selle kurja sugupõlve käsi.”

\section*{Tõeline sugulus}

\par 46 Kui ta rahvale veel rääkis, vaata, siis seisid ta ema ja ta vennad õues ja tahtsid temaga rääkida.
\par 47 Siis ütles keegi temale: „Vaata, su ema ja su vennad seisavad õues ja tahavad sinuga rääkida!”
\par 48 Tema aga vastas ja ütles sellele, kes seda temale teatas: „Kes on mu ema ja kes on mu vennad?”
\par 49 Ja ta sirutas oma käe jüngrite poole ning ütles: „Vaata, siin on mu ema ja mu vennad!
\par 50 Sest kes iganes teeb mu Isa tahtmist, kes on taevas, see on mu vend ja õde ja ema!”


\chapter{13}

\section*{Tähendamissõna külvajast}

\par 1 Sel päeval väljus Jeesus sealt kojast ja istus maha mere äärde.
\par 2 Ja palju rahvast kogunes tema juurde, nii et ta pidi astuma paati ja maha istuma. Ja kõik rahvas seisis rannal.
\par 3 Ja ta rääkis neile palju tähendamissõnadega ning ütles: „Vaata, külvaja läks välja külvama.
\par 4 Ja kui ta külvas, kukkus muist tee äärde, ja linnud tulid ja sõid selle ära.
\par 5 Muist kukkus kaljuse maa peale, kus tal ei olnud palju mulda. Ja see tärkas ruttu, sest tal ei olnud sügavat maad.
\par 6 Aga kui päike tõusis, kõrbes ta, ja et tal ei olnud juurt, kuivas ta ära.
\par 7 Muist kukkus ohakate sekka ja ohakad tõusid ja lämmatasid selle.
\par 8 Aga muist kukkus hea maa peale ja kandis vilja, mõni iva sada seemet, mõni kuuskümmend ja mõni kolmkümmend.
\par 9 Kellel kõrvad on, see kuulgu!”

\section*{Tähendamissõnade otstarve}

\par 10 Siis jüngrid astusid ta juurde ja ütlesid temale: „Mispärast sa räägid neile tähendamissõnadega?”
\par 11 Tema kostis ning ütles: „Teile on antud mõista taevariigi saladusi, neile aga ei ole antud.
\par 12 Sest kellel on, sellele antakse, ja tal on küllalt; aga kellel ei ole, sellelt võetakse ka see, mis tal on!
\par 13 Sellepärast ma räägin neile tähendamissõnadega, et nad nähes ei näeks ja kuuldes ei kuuleks ega mõistaks!
\par 14 Ja nende kohta läheb täide prohvet Jesaja ennustus, mis ütleb: kuuldes te kuulete ega mõista, ja nähes te näete ega taipa!
\par 15 Sest selle rahva süda on tuimaks läinud ja nad kuulevad raskesti oma kõrvadega ja sulevad oma silmad, et nad silmadega ei näeks ja kõrvadega ei kuuleks ja südamega ei mõistaks ega pöörduks, et mina neid parandaksin!
\par 16 Aga õndsad on teie silmad, et nad näevad ja teie kõrvad, et nad kuulevad!
\par 17 Sest tõesti, ma ütlen teile, paljud prohvetid ja õiged on igatsenud näha, mida teie näete ega ole näinud, ja kuulda, mida teie kuulete ega ole kuulnud!

\section*{Jeesus seletab tähendamissõna külvajast}

\par 18 Kuulge siis teie nüüd tähendamissõna külvajast:
\par 19 kui keegi kuningriigi sõna kuuleb ega saa aru, siis tuleb tige ja kisub ära selle, mis oli külvatud ta südamesse. Seda tähendab see, mis külvati tee äärde.
\par 20 Mis kaljuse maa peale külvati tähendab seda, kes sõna kuuleb ja võtab selle kohe rõõmuga vastu;
\par 21 temal aga ei ole juurt eneses, vaid ta on ainult üürikeseks: kui viletsus ning kiusatus tuleb sõna pärast, taganeb ta varsti.
\par 22 Aga mis ohakate sekka külvati, tähendab seda, kes küll sõna kuuleb, maailma mure ja rikkuse pettus aga lämmatab sõna ära ja see jääb viljatuks.
\par 23 Ent mis külvati hea maa peale, tähendab seda, kes sõna kuuleb ning saab sellest aru ja kannab ka vilja; ja mõni annab sada seemet, mõni kuuskümmend, mõni kolmkümmend!”

\section*{Tähendamissõna põllulustest}

\par 24 Teise tähendamissõna esitas ta neile ning ütles: „Taevariik on mehe sarnane, kes oma põllule külvas hea seemne.
\par 25 Aga inimeste magades tuli ta vaenlane ja külvas lustet nisu sekka ja läks ära.
\par 26 Kui nüüd oras kasvas ja vili hakkas looma, siis tuli ka luste nähtavale.
\par 27 Aga majaisanda orjad tulid tema juurde ning ütlesid temale: isand, eks sa külvanud head seemet oma põllule, kust tuleb sellele nüüd lustet?
\par 28 Tema ütles neile: seda on teinud vaenulik inimene. Siis ütlesid orjad temale: kas sa nüüd tahad, et me läheme ja selle kokku kogume?
\par 29 Aga tema ütles: ei, et te lustet katkudes ühes sellega ei kisuks välja ka nisu.
\par 30 Laske mõlemad ühtlasi kasvada lõikuseks ja lõikuseajal ma ütlen lõikajaile: koguge enne lusted ja siduge vihku ärapõletamiseks, nisu aga koguge kokku mu aita!”

\section*{Tähendamissõnad sinepiivakesest ja haputaignast}

\par 31 Teise tähendamissõna ta esitas neile ning ütles: „Taevariik on sinepiivakese samane, mille mees võttis ning külvas oma põllule.
\par 32 See on küll kõige väiksem kõigist seemneist, aga kui see on kasvanud, on see suurim aiataimedest ja saab puuks, nii et taeva linnud tulevad ja teevad pesad ta okstele!”
\par 33 Teise tähendamissõna ta rääkis neile: „Taevariik on haputaigna sarnane, mille naine võttis ja segas kolme vaka jahu sekka, kuni kõik läks hapnema!”
\par 34 Seda kõike rääkis Jeesus rahvale tähendamissõnadega ja ilma tähendamissõnata ei rääkinud ta neile midagi,
\par 35 et läheks täide, mis on öeldud prohveti kaudu: „Ma avan oma suu tähendamissõnades ja kuulutan, mis on olnud salajas maailma algusest!”

\section*{Jeesus seletab tähendamissõna põllulustest}

\par 36 Siis Jeesus laskis rahvahulgad ära minna ja tuli koju. Ja ta jüngrid astusid tema juurde ning ütlesid: „Seleta meile tähendamissõna põllulustest.”
\par 37 Tema kostis ja ütles: „Kes head seemet külvab, on Inimese Poeg.
\par 38 Põld on maailm; hea seeme on kuningriigi lapsed, lusted aga on tigeda lapsed.
\par 39 Vaenlane, kes neid külvab, on kurat. Aga lõikus on maailma-ajastu lõpp; lõikajad on inglid.
\par 40 Otsekui nüüd umbrohi kogutakse ja tulega ära põletatakse, nõnda peab ka sündima selle maailma-ajastu lõpul.
\par 41 Inimese Poeg läkitab oma inglid ja nemad koguvad tema kuningriigist kõik pahandused ja kõik need, kes teevad, mis on käsu vastu,
\par 42 ja heidavad nad tuleahju; seal on ulumine ja hammaste kiristamine.
\par 43 Siis paistavad õiged nagu päike oma Isa kuningriigis. Kel kõrvad on, see kuulgu!

\section*{Tähendamissõnad peidetud varandusest, kallist pärlist ja noodast}

\par 44 Taevariik on põllusse peidetud varanduse sarnane, mille mees leidis ning kinni kattis ja läks sealt ära rõõmsana selle üle ja müüs ära kõik, mis tal oli ja ostis selle põllu.
\par 45 Taas on taevariik kaupmehe sarnane, kes otsis ilusaid pärleid,
\par 46 ja kui ta oli leidnud ühe kalli pärli, läks ta ja müüs kõik, mis tal oli, ning ostis selle.
\par 47 Taas on taevariik nooda sarnane, mis merre heideti ja kokku vedas kõiksugu kalu.
\par 48 Ja kui see täis sai, veeti ta randa, ja istuti maha ning koguti head astjatesse, aga halvad visati ära.
\par 49 Nõnda sünnib maailma-ajastu lõpul: inglid väljuvad ja eraldavad kurjad õigete hulgast
\par 50 ja heidavad nad tuleahju; seal on ulumine ja hammaste kiristamine.
\par 51 Kas te olete aru saanud kõigest sellest?„ Nad ütlesid temale: ”Jah!”
\par 52 Siis ta ütles neile: „Seepärast on iga kirjatundja, kes on õpetatud taevariigi jaoks, majaisanda sarnane, kes toob esile oma tagavarast uut ja vana!”

\section*{Jeesust põlatakse kodukohas}

\par 53 Ja sündis, kui Jeesus oli lõpetanud need tähendamissõnad, et ta läks sealt ära.
\par 54 Ja ta tuli oma kodukohta ja õpetas neid nende kogudusekojas, nii et nad hämmastusid ning ütlesid: „Kust on sellel see tarkus ja need vägevad teod?
\par 55 Eks tema ole see puusepa poeg? Eks ta ema nimetata Maarjaks ja ta vendi Jakoobuseks ja Jooseseks ja Siimonaks ja Juudaks?
\par 56 Ja eks ta õed kõik ole meie juures? Kust siis temale see kõik on tulnud?”
\par 57 Ja nad pahandusid temast. Aga Jeesus ütles neile: „Ei peeta prohvetist kuskil nii vähe lugu kui kodukohas ja ta omas majas!”
\par 58 Ja ta ei teinud seal mitte palju vägevaid tegusid nende uskmatuse pärast.


\chapter{14}

\section*{Ristija Johannese surm}

\par 1 Sel ajal kuulis nelivürst Heroodes Jeesusest räägitavat.
\par 2 Ja ta ütles oma teenijaile: „See on Ristija Johannes; tema on surnuist üles tõusnud ja sellepärast on imelised väed tegevad tema sees.”
\par 3 Oli ju Heroodes Johannese kinni võtnud, sidunud ja pannud ta vangi Heroodiase, oma venna Filippuse naise pärast.
\par 4 Sest Johannes oli temale öelnud: „Sul ei ole luba teda pidada.”
\par 5 Ja Heroodes oleks tahtnud Johannese tappa, aga kartis rahvast, sest nad pidasid teda prohvetiks.
\par 6 Aga kui Heroodese sünnipäev tuli, tantsis Heroodiase tütar nende keskel ja meeldis Heroodesele.
\par 7 Sellepärast tõotas ta vandega temale anda, mida ta iganes peaks paluma.
\par 8 Siis ütles tütar oma ema õhutusel: „Anna mulle siia vaagnal Ristija Johannese pea!”
\par 9 Ja kuningas sai kurvaks, kuid vande ja lauasistujate pärast ta käskis temale selle anda.
\par 10 Ta läkitas siis ja laskis vangihoones Johannese pea otsast raiuda.
\par 11 Ja tema pea toodi vaagnal ning anti neitsile; ja neitsi viis selle oma emale.
\par 12 Ja tema jüngrid tulid ning võtsid tema keha ja matsid ta maha ja tulid ning teatasid sellest Jeesusele.

\section*{Viie tuhande mehe söötmine}

\par 13 Kui Jeesus seda kuulis, läks ta sealt paadiga ära tühja paika üksipäini. Ja kui rahvahulgad seda kuulsid, käisid nad linnadest jala tema järel.
\par 14 Ja maale astudes ta nägi palju rahvast. Ja tal hakkas nendest hale meel ja ta tegi terveks nende põdejad.
\par 15 Ent õhtu tulles astusid jüngrid tema juurde ning ütlesid: „See on tühi paik ja aeg on juba möödas; lase rahvas minna, et nad läheksid küladesse ja ostaksid enestele toidupoolist.”
\par 16 Aga Jeesus ütles neile: „Nendel pole tarvis ära minna; andke teie neile süüa.”
\par 17 Nemad ütlesid temale: „Meil ei ole siin rohkem kui viis leiba ja kaks kala!”
\par 18 Kuid tema ütles: „Tooge mulle need siia!”
\par 19 Ja ta käskis rahvahulgad istuda murule, võttis need viis leiba ja kaks kala, vaatas üles taeva poole ja õnnistas ning murdis ja andis leivad jüngrite kätte, aga jüngrid andsid rahvale.
\par 20 Ja nad kõik sõid ja nende kõhud said täis. Ja nad korjasid kokku ülejäänud palukesi kaksteist korvitäit.
\par 21 Ent neid, kes olid söönud, oli ligi viis tuhat meest peale naiste ja laste.

\section*{Jeesus kõnnib vee peal}

\par 22 Ja sedamaid sundis Jeesus oma jüngreid astuma paati ja sõitma tema eele teisele poole, kuni ta laseb rahva minema.
\par 23 Kui ta siis rahva oli lasknud minema, läks ta ise üles mäele üksipäini palvetama. Ja õhtu tulles oli ta üksinda sealsamas.
\par 24 Aga paat oli juba palju vagusid maad kaldast eemal ja oli hädas lainetega, sest tuul oli vastu.
\par 25 Aga neljandal öövahikorral tuli Jeesus nende juurde, kõndides merel.
\par 26 Ja kui jüngrid teda nägid merel kõndivat, kohkusid nad ja ütlesid: „See on tont!” Ja nad kisendasid hirmu pärast.
\par 27 Aga sedamaid rääkis Jeesus nendega ning ütles: „Olge julged, mina olen see, ärge kartke!”
\par 28 Peetrus kostis temale ning ütles: „Issand, kui sina oled, siis käsi mind tulla enese juurde vee peale!”
\par 29 Tema ütles: „Tule!” Ja Peetrus astus paadist välja ning kõndis vee peal ja tuli Jeesuse juurde.
\par 30 Aga nähes tuult lõi ta kartma ja hakkas vajuma, kisendas ning ütles: „Issand, päästa mind!”
\par 31 Ja Jeesus sirutas sedamaid oma käe, haaras temast kinni ja ütles talle: „Sa nõdrausuline, miks sa kahtlesid?”
\par 32 Ja nad astusid paati, ja tuul rauges.
\par 33 Siis paadis olijad tulid ja kummardasid teda ning ütlesid: „Tõesti, sa oled Jumala Poeg!”

\section*{Jeesus teeb haigeid terveks Gennesareti mail}

\par 34 Ja kui nad olid jõudnud teisele kaldale, tulid nad Gennesareti maile.
\par 35 Ja kui selle koha mehed ta ära tundsid, läkitasid nad sõna kogu ümberkaudsele maale. Ja tema juurde toodi kõik haiged;
\par 36 ja need palusid teda, et nad saaksid vaid puudutada tema kuue palistust. Ja nii paljud, kui puudutasid, said terveks.


\chapter{15}

\section*{Jeesus vastab etteheiteile pärimuste rikkumise kohta}

\par 1 Siis tuli varisere ja kirjatundjaid Jeruusalemmast Jeesuse juurde ja need ütlesid:
\par 2 „Miks sinu jüngrid rikuvad vanemate pärimust? Sest nad ei pese käsi, kui nad hakkavad leiba võtma.”
\par 3 Aga tema kostis ja ütles neile: „Miks te ise rikute Jumala käsku oma pärimuse pärast?
\par 4 Sest Jumal on öelnud: sa pead oma isa ja ema austama, ja kes isa või ema neab, peab surma surema!
\par 5 Kuid teie ütlete, kes iganes isale või emale ütleb: ohvrianniks läheb see, mis sul iganes on õigus minult saada! see ei tarvitse oma isa või ema austada.
\par 6 Sellega te olete Jumala sõna tühjaks teinud oma pärimuse pärast.
\par 7 Te silmakirjatsejad, Jesaja on teist õigesti ennustanud, öeldes:
\par 8 see rahvas austab mind oma huultega, aga nende süda on minust kaugel;
\par 9 ilmaaegu teenivad nad mind, õpetades õpetusi, mis on inimeste käskimised!”
\par 10 Ja ta kutsus rahva enese juurde ning ütles neile: „Kuulge ja mõistke!
\par 11 Mitte see, mis suust sisse läheb, ei rüveta inimest, vaid see, mis suust väljub, rüvetab inimest.”
\par 12 Siis astusid tema jüngrid ta juurde ja ütlesid temale: „Kas sa tead, et variserid pahandusid kuuldes seda sõna?”
\par 13 Aga tema kostis ning ütles: „Iga taim, mida mu taevane Isa pole istutanud, kistakse välja.
\par 14 Jätke nad! Nad on pimedate pimedad teejuhid. Aga kui pime pimedat juhib, langevad mõlemad auku!”
\par 15 Siis kostis Peetrus ja ütles temale: „Seleta meile see võrdum!”
\par 16 Jeesus ütles: „Kas teiegi veel ei saa aru?
\par 17 Eks te mõista, et kõik, mis suust sisse läheb, läheb kõhtu ja väljub eri paika.
\par 18 Aga mis suust väljub, see lähtub südamest ja see rüvetab inimese.
\par 19 Sest südamest lähtub kurje mõtteid, mõrvu, abielurikkumisi, hoorust, vargusi, valetunnistusi, jumalapilget.
\par 20 Need on, mis rüvetavad inimese; kuid pesemata kätega söömine ei rüveta inimest.”

\section*{Jeesus teeb terveks Kaananea naise tütre}

\par 21 Ja Jeesus tuli sealt ära ning läks Tüürose ja Siidoni rajadesse.
\par 22 Ja vaata, neist paigust väljus Kaananea naine; see kisendas ning ütles talle: „Issand, Taaveti poeg, halasta mu peale! Mu tütart vaevab kuri vaim hirmsasti.”
\par 23 Aga ta ei vastanud temale sõnagi. Siis tema jüngrid astusid ta juurde, palusid teda ning ütlesid: „Saada ta ära, sest ta kisendab meie taga!”
\par 24 Ent tema kostis ning ütles: „Mind ei ole läkitatud muude kui Iisraeli soo kadunud lammaste juurde!”
\par 25 Kuid naine tuli, kummardas teda ning ütles: „Issand, aita mind!”
\par 26 Aga tema kostis ning ütles: „Ei ole ilus võtta laste leib ja heita koerakeste ette!”
\par 27 Siis ütles naine: „Jah, Issand, ometigi söövad koerakesed raasukesi, mis nende isandate laualt langevad!”
\par 28 Siis kostis Jeesus ning ütles temale: „Oh naine, su usk on suur! Sulle sündigu nõnda, kuidas sa tahad!” Ja tema tütar sai terveks sestsamast tunnist.

\section*{Jeesus teeb terveks palju vigaseid}

\par 29 Ja Jeesus läks sealt ära ja tuli Galilea mere äärde; ja ta läks mäele ja istus sinna.
\par 30 Ja tema juurde tuli palju rahvast, tuues enesega ühes jalutuid, vigaseid, pimedaid, keeletuid ja palju teisi; ja nad heitsid need tema jalge ette. Ja ta tegi nad terveks,
\par 31 nii et rahvas pani imeks, nähes keeletuid rääkivat, vigaseid olevat terved, jalutuid käivat ja pimedaid nägevat. Ja nad andsid au Iisraeli Jumalale.

\section*{Nelja tuhande mehe söötmine}

\par 32 Ja Jeesus kutsus oma jüngrid enese juurde ja ütles: „Mul on meel hale rahvast, sest nad on juba kolm päeva olnud minu juures ja neil ei ole midagi süüa; ma ei taha lasta neid ära minna ilma söömata, et nad teel ei nõrkeks.”
\par 33 Siis ütlesid ta jüngrid temale: „Kust me võtame kõrbes nii palju leibu, et nii hulga rahva kõhud täita?”
\par 34 Ja Jeesus ütles neile: „Mitu leiba teil on?” Nad vastasid: „Seitse, ja pisut kalu.”
\par 35 Siis ta käskis rahva maha istuda.
\par 36 Ja ta võttis need seitse leiba ja kalad, tänas, murdis ning andis jüngrite kätte ja jüngrid andsid rahvale.
\par 37 Ja kõik sõid ja nende kõhud said täis; ja nad korjasid kokku ülejäänud palukesi seitse korvitäit.
\par 38 Ent neid, kes olid söönud, oli neli tuhat meest peale naiste ja laste.
\par 39 Ja kui ta oli lasknud rahva ära minna, läks ta paati ja tuli Magdala maa-alale.


\chapter{16}

\section*{Jeesus ei hooli variseride tunnustähe nõudmisest}

\par 1 Siis tulid variserid ja saduserid ta juurde teda kiusama ja nõudsid, et ta neile näitaks mõne tunnustähe taevast.
\par 2 Aga ta kostis ning ütles neile: „Õhtu tulles te ütlete: head ilma tuleb, sest taevas punab.
\par 3 Ja hommikul te ütlete: täna tuleb rajuilm, sest taevas punab ja on pilves. Taeva nägu te oskate küll mõista, kuid aegade tunnustähti te ei suuda mõista.
\par 4 See kuri ja abielurikkuja tõug otsib tunnustähte, ja temale ei anta muud tähte kui prohvet Joona täht!” Ja ta jättis nad maha ning läks ära.

\section*{Jeesus noomib jüngreid taipamatuse pärast}

\par 5 Ja kui tema jüngrid olid jõudnud teisepoolsele kaldale, olid nad unustanud leiba kaasa võtta.
\par 6 Aga Jeesus ütles neile: „Olge ettevaatlikud ja hoiduge variseride ja saduseride haputaignast!”
\par 7 Siis nad arutasid isekeskis ja ütlesid: „See käib selle kohta, et me ei võtnud leiba kaasa.”
\par 8 Jeesus märkas seda ja ütles: „Te nõdrausulised, miks te arutate isekeskis, et te ei ole leiba kaasa võtnud?
\par 9 Kas te veel ei saa aru? Eks te mäleta neid viit leiba viiele tuhandele ja mitu korvitäit te saite,
\par 10 ja neid seitset leiba neljale tuhandele ja mitu korvitäit te saite?
\par 11 Kuidas te siis ei mõista, et ma teile ei rääkinud leivast, kui ma käskisin teid hoiduda variseride ja saduseride haputaignast?”
\par 12 Siis nad mõistsid, et ta neid ei käskinud hoiduda leiva haputaignast, vaid variseride ja saduseride õpetusest.

\section*{Peetrus tunnistab Jeesuse Kristuseks}

\par 13 Aga kui Jeesus tuli Filippuse Kaisarea rajale, küsis ta oma jüngritelt ning ütles: „Keda ütlevad inimesed mind, Inimese Poja olevat?”
\par 14 Nemad ütlesid: „Mõned ütlevad sind olevat Ristija Johannese, teised Eelija, aga teised Jeremija või ühe prohveteist.”
\par 15 Tema ütles neile: „Aga teie, keda teie mind ütlete olevat?”
\par 16 Siimon Peetrus vastas ning ütles: „Sina oled Kristus, elava Jumala Poeg!”
\par 17 Jeesus vastas ning ütles temale: „Õnnis oled sa, Siimon, Joona poeg, sest liha ja veri ei ole sulle seda ilmutanud, vaid mu Isa, kes on taevas.
\par 18 Ja mina ütlen sulle ka: sina oled Peetrus ja sellele kaljule ma ehitan oma koguduse ja surmavalla väravad ei võida seda!
\par 19 Ma annan sinule taevariigi võtmed ja mis sa maa peal seod, see on taevas seotud ja mis sa maa peal lahti päästad, on ka taevas lahti päästetud!”
\par 20 Siis ta keelas oma jüngreid, et nad kellelegi ei ütleks, et tema on Kristus.

\section*{Jeesus kuulutab ette oma surma ja ülestõusmist}

\par 21 Sellest ajast hakkas Jeesus Kristus teatama oma jüngritele, et ta peab minema Jeruusalemma ja palju kannatama vanemate ja ülempreestrite ja kirjatundjate poolt ja et ta tapetakse ja kolmandal päeval äratatakse üles.
\par 22 Siis Peetrus võttis tema isepäinis, hakkas teda noomima ja ütles: „Jumal hoidku, Issand! Ärgu seda sulle sündigu!”
\par 23 Aga ta pöördus ja ütles Peetrusele: „Tagane minust, saatan! Sa oled mulle pahanduseks; sest sa ei mõtle sellele, mis on Jumala, vaid mis on inimeste meelt mööda!”

\section*{Nõue iseenese salgamiseks}

\par 24 Siis Jeesus ütles oma jüngritele: „Kui keegi tahab minu järele tulla, siis ta salaku end ning võtku oma rist enese peale ja järgigu mind.
\par 25 Sest kes iganes oma hinge tahab päästa, see kaotab selle; aga kes oma hinge kaotab minu pärast, see leiab selle.
\par 26 Sest mis kasu on inimesel sellest, kui ta kogu maailma kasuks saaks, kuid oma hingele kahju teeks? Või mis lunastushinda võib inimene anda oma hinge eest?
\par 27 Sest Inimese Poeg tuleb oma Isa auhiilguses oma inglitega ja siis ta tasub igaühele tema tööd mööda.
\par 28 Tõesti, mina ütlen teile, neist, kes siin seisavad, on mõned, kes ei maitse surma, enne kui nad näevad Inimese Poja tulevat tema kuningriigis!”


\chapter{17}

\section*{Jeesuse muutumine}

\par 1 Ja kuue päeva pärast Jeesus võttis enesega Peetruse, Jakoobuse ja Johannese, tema venna, ja viis nad üles kõrgele mäele isepäinis.
\par 2 Ja ta muudeti nende ees; ja ta pale paistis otsekui päike ja ta riided läksid valgeks otsekui valgus.
\par 3 Ja vaata, Mooses ja Eelija näitasid endid neile ja kõnelesid temaga.
\par 4 Aga Peetrus hakkas rääkima ja ütles Jeesusele: „Issand, siin on hea olla; kui sa tahad, siis ma teen siia kolm telki, ühe sinule, ühe Moosesele ja ühe Eelijale?”
\par 5 Kui ta alles rääkis, vaata, siis varjas neid hele pilv. Ja vaata, hääl pilvest ütles: „See on mu armas Poeg, kellest mul on hea meel; teda kuulake!”
\par 6 Ja kui jüngrid seda kuulsid, langesid nad silmili maha ning kartsid väga.
\par 7 Ja Jeesus tuli nende juurde, puudutas neid ja ütles: „Tõuske üles ja ärge kartke!”
\par 8 Aga kui nad oma silmad üles tõstsid, ei näinud nad kedagi kui Jeesust üksi.
\par 9 Ja mäelt alla tulles keelas Jeesus neid ning ütles: „Ärge rääkige kellelegi sellest nägemusest, kuni Inimese Poeg surnuist üles tõuseb!”
\par 10 Ja jüngrid küsisid temalt ning ütlesid: „Miks siis kirjatundjad ütlevad, et Eelija peab tulema enne?”
\par 11 Tema kostis ja ütles: „Eelija tuleb küll enne ja seab kõik korda.
\par 12 Aga ma ütlen teile, et Eelija on juba tulnud ja nad ei ole teda ära tundnud, vaid on temale teinud, mida nad tahtsid. Nõnda peab ka Inimese Poeg kannatama nende poolt!”
\par 13 Siis jüngrid said aru, et ta neile rääkis Ristijast Johannesest.

\section*{Jeesus teeb terveks kuutõbise poisi}

\par 14 Ja kui nad tulid rahva juurde, astus tema ette inimene ja heitis põlvili ta ette
\par 15 ning ütles: „Issand, halasta mu poja peale, sest ta on kuutõbine ja suures vaevas; sagedasti langeb ta tulle ja sagedasti vette.
\par 16 Ja ma tõin ta sinu jüngrite juurde, kuid nad ei suutnud teda terveks teha!”
\par 17 Aga Jeesus kostis ning ütles: „Oh sina uskmatu ja pöörane tõug! Kui kaua pean ma olema teie juures? Kui kaua pean ma teiega kannatama? Tooge ta siia mu juurde!”
\par 18 Ja Jeesus sõitles teda, ja kuri vaim väljus temast, ja poiss sai terveks samast tunnist.
\par 19 Siis tulid jüngrid Jeesuse juurde isepäinis ja küsisid: „Miks meie ei võinud teda välja ajada?”
\par 20 Aga tema ütles neile: „Teie nõdra usu pärast. Sest tõesti ma ütlen teile, kui teil oleks usk nagu sinepiivake, siis te võiksite öelda sellele mäele: siirdu siit sinna! Ja ta siirduks sinna, ja miski ei oleks teile võimatu.
\par 21 [See sugu ei lähe muidu välja kui aga palve ja paastumisega!]

\section*{Jeesus kuulutab ette teist korda oma surma ja ülestõusmist}

\par 22 Aga kui nad üheskoos viibisid Galileas, ütles Jeesus neile: „Inimese Poeg antakse ära inimeste kätte;
\par 23 ja nad tapavad tema ja kolmandal päeval ta äratatakse üles!” Ja nad said väga kurvaks.

\section*{Jeesus annab maksuraha}

\par 24 Aga kui nad Kapernauma jõudsid, tulid Peetruse juurde maksuraha korjajad ja ütlesid: „Kas teie õpetaja maksuraha ei maksa?”
\par 25 Tema ütles: „Jah maksab!” Ja kui ta koju tuli, jõudis Jeesus talle küsimusega ette ning ütles: „Mis sa arvad, Siimon? Kellelt võtavad kuningad maa peal tolli või maksu, kas oma lastelt või võõrastelt?”
\par 26 Peetrus vastas temale: „Võõrastelt.” Jeesus ütles temale: „Siis on lapsed sellest vabad.
\par 27 Kuid et neid mitte pahandada, mine merele, heida õng sisse ja võta esimene kala, mis üles tuleb; ja kui sa tema suu avad, sa leiad hõberaha. Võta see ja anna neile minu ja enese eest!”


\chapter{18}

\section*{Õpetus alandlikkuseks}

\par 1 Sel ajal tulid jüngrid Jeesuse juurde ja ütlesid: „Kes on küll suurim taevariigis?”
\par 2 Ja Jeesus kutsus lapsukese enese juurde, asetas tema nende keskele
\par 3 ja ütles: „Tõesti ma ütlen teile, kui te ei pöördu ega saa kui lapsukesed, ei saa te mitte taevariiki!
\par 4 Kes nüüd iseennast alandab nagu see lapsuke, on suurem taevariigis.
\par 5 Ja kes iganes ühe niisuguse lapsukese vastu võtab minu nimel, see võtab mind vastu.

\section*{Hoiatus pahanduste eest}

\par 6 Aga kes pahandab ühe neist pisukesist, kes minusse usuvad, sellele oleks parem, et veskikivi tema kaela poodaks ja ta uputataks mere sügavusse.
\par 7 Häda maailmale pahanduste pärast! Sest pahandused peavad tulema, kuid häda sellele inimesele, kelle läbi pahandus tuleb!
\par 8 Aga kui su käsi või su jalg sind pahandab, siis raiu ta maha ja heida enesest ära. Parem on sulle, et sa vigasena või jalutuna lähed elusse, kui et sul on kaks kätt ja kaks jalga ja sind heidetakse igavesse tulle.
\par 9 Ja kui su silm sind pahandab, kisu ta välja ja viska enesest ära. Parem on sulle, et sa ühe silmaga lähed elusse kui et sul on kaks silma ja sind heidetakse põrgutulle.
\par 10 Vaadake ette, et te ühtki neist pisukesist ei põlga; sest ma ütlen teile, et nende inglid taevas alati näevad mu Isa palet, kes on taevas.
\par 11 [Sest Inimese Poeg on tulnud päästma seda, mis on kadunud.]

\section*{Tähendamissõna kadunud lambast}

\par 12 Mis te arvate? Kui kellelgi inimesel juhtub olema sada lammast ja üks neist eksib ära, eks ta jäta need üheksakümmend üheksa mägedele ja lähe otsima seda, kes on ära eksinud?
\par 13 Ja kui juhtub, et ta selle leiab, tõesti ma ütlen teile, et ta sellest tunneb rohkem rõõmu kui üheksakümne üheksast, kes ei olnud ära eksinud.
\par 14 Nõnda ei ole ka teie Isa tahtmine, kes on taevas, et üks neist pisukesist hukka läheks.

\section*{Süüdlase kohtlemisest}

\par 15 Aga kui su vend eksib sinu vastu, siis mine ja noomi teda nelja silma all. Kui ta sind kuulab, siis sa oled oma venna võitnud.
\par 16 Aga kui ta sind ei kuula, siis võta enesega veel üks või kaks, et iga asi kinnitataks kahe või kolme tunnistaja sõnaga.
\par 17 Aga kui ta neid ei kuula, siis ütle kogudusele; aga kui ta kogudustki ei kuula, siis ta olgu sinu meelest nagu pagan ja tölner.
\par 18 Tõesti ma ütlen teile, et mis te iganes maa peal seote, on ka taevas seotud, ja mis te iganes maa peal lahti päästate, on ka taevas lahti päästetud.

\section*{Ühine palve}

\par 19 Taas ütlen ma tõesti teile, et kui kaks teie seast ühel nõul on maa peal mingi asja pärast, mida nad iganes paluvad, siis see saab neile minu Isalt, kes on taevas.
\par 20 Sest kus kaks või kolm koos on minu nimel, seal olen mina nende keskel.”

\section*{Andeksandmise käsk ja tähendamissõna tigedast sulasest}

\par 21 Siis astus Peetrus tema juurde ja ütles: „Issand, mitu korda ma pean, kui mu vend mu vastu eksib, temale andeks andma? Ons küllalt seitsmest korrast?”
\par 22 Jeesus ütles temale: „Ma ei ütle sulle mitte seitse korda, vaid seitsekümmend korda seitse korda!
\par 23 Sellepärast on taevariik kuninga sarnane, kes oma sulastega tahtis aru teha.
\par 24 Aga kui ta hakkas aru tegema, toodi üks ta ette, kes oli talle kümme tuhat talenti võlgu.
\par 25 Aga kui tal ei olnud maksta, käskis isand müüa tema ja ta naise ja ta lapsed ja kõik, mis tal oli ja maksta.
\par 26 Siis sulane heitis maha, kummardas teda ja ütles: kannata minuga, ja ma maksan sulle kõik!
\par 27 Siis oli isandal hale meel sellest sulasest, ta laskis tema lahti ja kinkis talle võla.
\par 28 Aga seesama sulane läks välja ja leidis ühe oma kaassulase, kes temale oli võlgu sada teenarit. Ja ta võttis tema kinni, kägistas teda ja ütles: maksa, mis sa oled võlgu!
\par 29 Siis tema kaassulane heitis maha tema jalge ette, palus teda ning ütles: kannata minuga, ja ma maksan sulle kõik!
\par 30 Aga tema ei tahtnud, vaid läks ära ja heitis ta vangi kuni ta maksab oma võla.
\par 31 Aga kui tema kaassulased nägid, mis sündis, said nad väga kurvaks ja tulid ning kaebasid oma isandale kõik, mis oli sündinud.
\par 32 Siis kutsus ta isand tema enese ette ja ütles temale: sa tige sulane! Kõik selle võla ma kinkisin sulle, sellepärast et sa mind palusid,
\par 33 eks siis sinagi pidanud halastama oma kaassulase peale, nõnda nagu mina sinu peale halastasin?
\par 34 Ja ta isand sai vihaseks ja andis tema piinajate kätte, kuni ta maksab kõik, mis tal temaga oli võlgu.
\par 35 Nõnda teeb ka minu taevane Isa teile, kui te igaüks omast südamest andeks ei anna oma vennale!”


\chapter{19}

\section*{Abielu lahutamisest}

\par 1 Ja sündis, kui Jeesus need kõned oli lõpetanud, et ta läks ära Galileamaalt ja tuli Juudamaa alale teisele poole Jordanit.
\par 2 Ja palju rahvast järgis teda, ja ta tegi neid seal terveks.
\par 3 Siis tuli tema juurde varisere teda kiusama, ja need ütlesid: „Kas on luba oma naist enesest lahutada igasugusel põhjusel?”
\par 4 Tema kostis ja ütles: „Kas te ei ole lugenud, et Looja nad algusest lõi meheks ja naiseks,
\par 5 ja ütles: seepärast inimene jätab maha oma isa ja ema ning hoiab oma naise poole, ja need kaks on üks liha!?
\par 6 Nõnda ei ole nad siis enam kaks, vaid on üks liha. Mis nüüd Jumal on ühte pannud, seda inimene ärgu lahutagu!”
\par 7 Nad ütlesid temale: „Mispärast siis Mooses käskis anda lahutuskirja ja lahutada?”
\par 8 Ta ütles neile: „Mooses on teile teie südame kanguse pärast andnud loa oma naisi enestest lahutada; algusest ei ole see mitte nõnda olnud.
\par 9 Aga ma ütlen teile: kes oma naise enesest lahutab muidu kui hooruse pärast ja võtab teise, see rikub abielu, [ja kes võtab lahutatud naise, see rikub abielu.]
\par 10 Siis ütlesid jüngrid temale: „Kui inimese lugu naisega on nõnda, siis ei ole hea abielluda.”
\par 11 Aga tema ütles neile: „Kõik ei saa aru sellest sõnast, vaid ainult need, kellele see on antud.
\par 12 Sest on kohitsetuid, kes emaihust nõnda on sündinud, ja on kohitsetuid, kes inimeste poolt on kohitsetud ja on kohitsetuid, kes ise on ennast kohitsenud taevariigi pärast. Kes suudab aru saada, saagu aru!”

\section*{Jeesus õnnistab lapsi}

\par 13 Siis toodi tema juurde lapsukesi, et ta oma käed nende peale paneks ja palvetaks. Aga jüngrid sõitlesid toojaid.
\par 14 Ent Jeesus ütles: „Jätke lapsukesed rahule ja ärge keelake neid minu juurde tulemast, sest niisuguste päralt on taevariik!”
\par 15 Ja ta pani oma käed nende peale ja läks sealt ära.

\section*{Rikas noormees ja igavene elu}

\par 16 Ja vaata, üks mees astus tema juurde ja ütles temale: „Õpetaja, mis head ma pean tegema, et saaksin igavese elu?”
\par 17 Aga ta ütles talle: „Miks sa mult küsid, mis on hea? Üksainus on, kes on hea. Ent kui sa tahad minna elu sisse, siis pea käsud!”
\par 18 Tema küsis temalt: „Missugused?” Jeesus ütles: „Sa ei tohi tappa; sa ei tohi abielu rikkuda; sa ei tohi varastada; sa ei tohi valet tunnistada;
\par 19 sa pead oma isa ja ema austama; ja armasta oma ligimest nagu iseennast!”
\par 20 Noor mees ütles temale: „Seda kõike ma olen pidanud. Mis puudub mul veel?”
\par 21 Jeesus ütles talle: „Kui tahad olla täiuslik, siis mine müü oma varandus ja anna vaestele, siis on sul varandus taevas, ja tule ning järgi mind!”
\par 22 Kui noor mees seda sõna kuulis, läks ta ära kurva meelega, sest tal oli palju vara.

\section*{Rikkal on raske pääseda taevariiki}

\par 23 Siis Jeesus ütles oma jüngritele: „Tõesti, ma ütlen teile, rikkal on raske pääseda taevariiki.
\par 24 Ja taas ma ütlen teile: hõlpsam on kaamelil minna läbi nõelasilma kui rikkal pääseda Jumala riiki!”
\par 25 Kui jüngrid seda kuulsid, hämmastusid nad väga ning ütlesid: „Kes siis võib õndsaks saada?”
\par 26 Aga Jeesus vaatas neile otsa ja ütles neile: „Inimestel on see võimatu, kuid Jumalal on kõik võimalik!”
\par 27 Siis kostis Peetrus ning ütles temale: „Vaata, meie oleme jätnud maha kõik ja oleme järginud sind. Mis me sellest nüüd saame?”
\par 28 Aga Jeesus ütles neile: „Tõesti, ma ütlen teile, et teie, kes olete mind järginud, uuestisündimises, kui Inimese Poeg istub oma aujärjele, istute ka ise kaheteistkümnele aujärjele ja mõistate kohut kaheteistkümne Iisraeli suguharu üle!
\par 29 Ja igaüks, kes on jätnud maha majad või vennad või õed või isa või ema või lapsed või põllud minu nime pärast, saab mitu korda rohkem ja pärib igavese elu.
\par 30 Aga paljud esimesed jäävad viimseiks ja viimsed saavad esimesteks!


\chapter{20}

\section*{Tähendamissõna töötegijaist viinamäel}

\par 1 Sest taevariik on majaisanda sarnane, kes vara hommikul välja läks töötegijaid palkama oma viinamäele.
\par 2 Kui ta töötegijatega oli kokku leppinud ühe teenari peale päevapalgaks, läkitas ta nad oma viinamäele.
\par 3 Ja ta läks välja kolmandal tunnil ja nägi teisi turul tööta seisvat
\par 4 ja ütles neile: minge ka teie viinamäele ja mis iganes õige on, annan ma teile!
\par 5 Ja nad läksid ära. Taas läks ta välja kuuendal ja üheksandal tunnil ja tegi nõndasamuti.
\par 6 Aga üheteistkümnendal tunnil ta läks välja ja leidis teisi seisvat ja ütles neile: mis te siin kogu päeva tööta seisate?
\par 7 Nad ütlesid temale: meid ei ole keegi palganud! Tema ütles neile: minge teiegi viinamäele!
\par 8 Kui siis õhtu tuli, ütles viinamäe isand oma ülevaatajale: kutsu töötegijad ja anna neile palk alates viimastest kuni esimesteni.
\par 9 Kui nüüd need tulid, kes üheteistkümnendal tunnil olid palgatud, said nad igaüks ühe teenari.
\par 10 Aga kui esimesed tulid, arvasid nemad, et nad saavad rohkem: ja nemadki said igaüks ühe teenari.
\par 11 Kui nad olid saanud, nurisesid nad majaisanda vastu
\par 12 ning ütlesid: need viimased aga on ühe tunni tööd teinud ja sa oled nad teinud meie väärilisteks, kes me päeva koormat ja palavust oleme kandnud!
\par 13 Aga ta kostis ning ütles ühele nende seast: sõber, ma ei tee sulle ülekohut; eks sa leppinud minuga kokku ühe teenari peale?
\par 14 Võta, mis sinule kuulub, ja mine oma teed; aga ma tahan sellele viimasele anda nõnda nagu sinulegi.
\par 15 Kas ei ole mul luba enese omaga teha, mida ma tahan? Või on sinu silm kade, et mina hea olen?
\par 16 Nõnda saavad viimased esimesteks ja esimesed viimasteks. [Sest paljud on kutsutud, kuid vähesed on valitud!]

\section*{Jeesus kuulutab ette kolmandat korda oma surma ja ülestõusmist}

\par 17 Ja tahtes minna üles Jeruusalemma, võttis Jeesus need kaksteist isepäinis ja ütles teel olles neile:
\par 18 „Vaata, me läheme üles Jeruusalemma ja Inimese Poeg antakse ülempreestrite ja kirjatundjate kätte ja nad mõistavad tema surma
\par 19 ja annavad tema paganate kätte pilgata ja piitsutada ja risti lüüa; ja kolmandal päeval ta äratatakse üles!”

\section*{Igatsus au järele}

\par 20 Siis astus Sebedeuse poegade ema tema juurde ühes oma poegadega, kummardas ja palus talt midagi.
\par 21 Aga ta ütles naisele: „Mida sa tahad?” Tema ütles temale: „Ütle, et need mu kaks poega istuksid üks su paremale ja teine su vasakule käele sinu kuningriigis!”
\par 22 Aga Jeesus vastas ning ütles: „Te ei tea, mida te palute! Kas te võite juua seda karikat, mis minul tuleb juua?” Nad ütlesid talle: „Jah, võime!”
\par 23 Tema ütles neile: „Minu karika te küll joote, aga mu paremal ja vasakul käel istuda ei ole minu käes teile anda, vaid kellele mu Isa selle on valmistanud.”

\section*{Inimese tõeline suurus}

\par 24 Ja kui need kümme seda kuulsid, sai nende meel pahaseks nende kahe venna peale.
\par 25 Aga Jeesus kutsus nad enese juurde ja ütles: „Te teate, et paganate ülemad valitsevad nende üle ja suured isandad tarvitavad vägivalda nende kallal.
\par 26 Teie seas aga ei tohi nõnda olla; vaid kes iganes teie seast tahab suureks saada, see olgu teie teenija;
\par 27 ja kes teie seast tahab olla kõige esimene, see olgu teie ori;
\par 28 otsekui Inimese Poeg ei ole tulnud, et teda teenitaks, vaid teenima ja oma hinge andma lunaks paljude eest!”

\section*{Kaks pimedat tehakse nägijaiks}

\par 29 Ja kui nad Jeerikost välja läksid, järgis teda palju rahvast.
\par 30 Ja vaata, kaks pimedat istus tee ääres ja kui nad kuulsid, et Jeesus läheb mööda, kisendasid nad: „Issand, Taaveti Poeg, halasta meie peale!”
\par 31 Aga rahvas sõitles neid, et nad vait jääksid. Kuid nad kisendasid veel enam: „Issand, Taaveti Poeg, halasta meie peale!”
\par 32 Siis Jeesus jäi seisma, hüüdis neid ja ütles: „Mida te tahate, et ma teile teeksin?”
\par 33 Nad ütlesid temale: „Issand, et meie silmad läheksid lahti!”
\par 34 Siis Jeesusel hakkas hale meel ja ta puudutas nende silmi. Ja kohe nad nägid jälle ja järgisid teda.


\chapter{21}

\section*{Jeesus tuleb Jeruusalemma kui Kuningas}


\par 1 Ja kui nad Jeruusalemma ligi said ja Betfage poole Õlimäele tulid, siis läkitas Jeesus kaks jüngrit ning ütles neile:
\par 2 „Minge sinna alevisse, mis on teie ees, ja varsti te leiate kinniseotud emaeesli ja sälu ta juures; päästke need lahti ja tooge mu juurde.
\par 3 Ja kui teile keegi midagi ütleb, siis öelge: Issandal on neid tarvis! Siis ta lähetab nad sedamaid.
\par 4 Aga see on kõik sündinud, et läheks täide, mis on öeldud prohveti kaudu:
\par 5 Öelge Siioni tütrele: vaata, sinu Kuningas tuleb sulle tasane ja istub emaeesli seljas ja sälu seljas, kes on ikkealuse looma varss!”
\par 6 Siis jüngrid läksid ja tegid nõnda, nagu Jeesus neid oli käskinud,
\par 7 ning tõid emaeesli ja sälu ja panid riided nende peale ja panid ta riiete peale istuma.
\par 8 Aga suurem hulk laotas oma riided tee peale, aga teised raiusid oksi puudest ja puistasid tee peale.
\par 9 Aga rahvahulgad, kes ees ja taga käisid, hüüdsid: „Hoosianna Taaveti Pojale! Õnnistatud olgu, kes tuleb Issanda nimel! Hoosianna kõrges!”
\par 10 Ja kui ta Jeruusalemma sisse läks, tõusis kogu linn liikvele ja ütles: „Kes see on?”
\par 11 Aga rahvas ütles: „Tema on see prohvet Jeesus Galilea Naatsaretist!”

\section*{Jeesus puhastab templi}

\par 12 Ja Jeesus läks pühakotta ja ajas välja kõik, kes müüsid ja ostsid pühakojas ja viskas kummuli rahavahetajate lauad ja tuvimüüjate istmed,
\par 13 ja ütles neile: „Kirjutatud on: minu koda peab hüütama palvekojaks, aga teie teete ta röövliauguks!”
\par 14 Ja pühakojas tuli tema juurde pimedaid ja jalutuid, ja ta tegi nad terveks.
\par 15 Aga kui ülempreestrid ja kirjatundjad nägid neid imesid, mida ta tegi, ja poisse, kes pühakojas hüüdsid: „Hoosianna Taaveti Pojale!” sai nende meel pahaseks
\par 16 ja nad ütlesid temale: „Kas sa kuuled, mida need ütlevad?” Jeesus ütles neile: „Jah kuulen! Kas te iganes pole lugenud: laste ja imikute suust oled sa enesele valmistanud kiituse?”
\par 17 Ja ta jättis nad maha ja läks linnast välja Betaaniasse ja jäi sinna ööseks.

\section*{Jeesus neab viljatu viigipuu}

\par 18 Aga kui ta vara hommikul jälle linna läks, hakkas tal nälg.
\par 19 Ja nähes üht viigipuud tee ääres, läks ta selle juurde, aga ei leidnud sealt pealt midagi muud kui ainult lehti. Ja ta ütles puule: „Ärgu iialgi enam kasvagu sinust vilja!” Ja kohe kuivas viigipuu ära.
\par 20 Kui jüngrid seda nägid, panid nad imeks ning ütlesid: „Kuidas see viigipuu nii kohe ära kuivas?”
\par 21 Aga Jeesus kostis ning ütles neile: „Tõesti ma ütlen teile, kui teil oleks usku ja te ei mõtleks kaksipidi, siis te ei teeks mitte ainult seda, mis viigipuule on sündinud, vaid kui te ka ütleksite sellele mäele: tõuse paigast ja lange merre! siis see sünniks.
\par 22 Ja kõik, mida te iganes palves palute uskudes, seda te saate!”

\section*{Küsimus Jeesuse meelevallast}

\par 23 Ja kui nad tulid pühakotta, astusid ülempreestrid ja rahva vanemad tema juurde, kui ta õpetas, ja ütlesid: „Missuguse meelevallaga sa teed neid asju? Ja kes on sulle selle meelevalla andnud?”
\par 24 Aga Jeesus kostis ning ütles neile: „Ma tahan ka teilt küsida ühe asja; kui te mulle selle ütlete, siis ma ütlen ka teile, missuguse meelevallaga ma neid asju teen.
\par 25 Kust oli Johannese ristimine, kas taevast või inimestest?„ Aga nemad arutasid asja isekeskis ja ütlesid: ”Kui me ütleme: taevast! siis ta ütleb meile: mispärast te siis ei uskunud teda?
\par 26 Aga kui me ütleme: inimestest! siis meil tuleb karta rahvast, sest nad kõik peavad Johannest prohvetiks!”
\par 27 Ja nad kostsid Jeesusele ning ütlesid: „Me ei tea!” Siis ütles ka tema neile: „Ega minagi teile ütle, missuguse meelevallaga ma neid asju teen.

\section*{Tähendamissõna kahest pojast}

\par 28 Aga mis te arvate? Ühel inimesel oli kaks poega. Ja ta läks esimese juurde ning ütles: poeg, mine täna tööle mu viinamäele!
\par 29 Aga tema kostis ning ütles: küll ma lähen, isand! Ja ta ei läinud mitte.
\par 30 Siis isa läks ka teise juurde ja ütles nõndasamuti. See kostis ning ütles: ei mina taha! Pärast ta kahetses ja läks.
\par 31 Kumb neist tegi isa tahtmist?„ Nad ütlesid: ”Viimane.„ Jeesus ütles neile: ”Tõesti ma ütlen teile, et tölnerid ja hoorad saavad enne teid Jumala riiki.
\par 32 Sest Johannes tuli teie juurde õiguse teed ja te ei uskunud teda; ent tölnerid ja hoorad uskusid teda. Aga teie, ehk te küll seda nägite, ei kahetsenud pärastki, et teda uskuda.

\section*{Tähendamissõna viinamäest}

\par 33 Kuulge teist tähendamissõna: oli majaisand, kes istutas viinamäe ja tegi selle ümber aia ja kaevas sinna surutõrre ja ehitas torni ja andis selle rendile aednike kätte ja läks võõrale maale.
\par 34 Kui siis viinamarja korjamise aeg lähenes, läkitas ta oma sulased aednike juurde oma vilja vastu võtma.
\par 35 Aga aednikud võtsid tema sulased kinni; mõnd nad peksid, mõne nad tapsid, mõne nad viskasid kividega surnuks.
\par 36 Taas ta läkitas teised sulased, rohkem kui esimesi; ja nad tegid neile nõndasamuti.
\par 37 Viimaks ta läkitas oma poja nende juurde, mõeldes, küllap nad mu poega häbenevad.
\par 38 Aga kui aednikud nägid poega, ütlesid nad isekeskis: see on see pärija, tulge, tapame ta ära, siis saame tema pärandi enestele!
\par 39 Ja nad võtsid ja tõukasid ta välja viinamäest ning tapsid ta ära.
\par 40 Kui nüüd viinamäe isand tuleb, mis ta teeb nende aednikega?”
\par 41 Nad ütlesid temale: „Need kurjad ta hukkab ära kurjasti ja annab viinamäe teiste aednike kätte, kes annavad temale vilja omal ajal.”
\par 42 Jeesus ütles neile: „Kas te pole iialgi Kirjast lugenud: kivi, mille hooneehitajad kõrvale heitsid, on saanud nurgakiviks; Issandalt on see tulnud ja on imeasi meie silmis.
\par 43 Seepärast ma ütlen teile: Jumala riik võetakse teilt ära ja antakse sellele rahvale, kes selle vilja kannab.
\par 44 Ja kes selle kivi peale langeb, läheb rusuks; aga kelle peale iganes see langeb, selle ta teeb pihuks!”
\par 45 Ja kui ülempreestrid ja variserid tema tähendamissõnu kuulsid, mõistsid nad, et ta neist rääkis.
\par 46 Ja nad oleksid ta hea meelega kinni võtnud, aga kartsid rahvast, sest see pidas teda prohvetiks.


\chapter{22}

\section*{Tähendamissõna kuninglikust pulmast}

\par 1 Ja Jeesus algas kõnet ja rääkis neile jälle tähendamissõnadega ning ütles:
\par 2 „Taevariik on kuninga sarnane, kes oma pojale pulmad tegi.
\par 3 Ja ta läkitas oma sulased kutsutuid pulma kutsuma. Ja need ei tahtnud tulla.
\par 4 Taas läkitas ta teisi sulaseid ja ütles: öelge kutsutuile: vaata, oma söömaaja ma olen valmistanud, mu härjad ja nuumveised on tapetud ja kõik on valmis, tulge pulma!
\par 5 Aga nad ei hoolinud sellest, vaid läksid ära, kes oma põllule, kes oma kaubale;
\par 6 veel teised võtsid tema sulased kinni, kohtlesid neid ülbesti ja tapsid nad ära.
\par 7 Aga kuningas vihastus ja läkitas oma sõjaväed välja, hukkas need tapjad ja süütas nende linna põlema.
\par 8 Siis ta ütles oma sulastele: pulmad on küll valmis, aga need, kes olid kutsutud, ei olnud seda väärt.
\par 9 Minge nüüd teelahkmetele ja kutsuge pulma, keda te iganes leiate!
\par 10 Ja need sulased läksid välja teedele ja kogusid kokku kõik, keda nad leidsid, kurjad ja head. Ja pulmakoda sai täis lauasistujaid.
\par 11 Siis kuningas läks sisse lauavõõraid vaatama ja nägi seal inimest, kellel ei olnud pulmariiet seljas.
\par 12 Ja ta ütles temale: „Sõber, kuidas sa siia oled sisse tulnud, ilma et sul pulmariiet oleks?” Aga too ei saanud sõnagi suust.
\par 13 Siis kuningas ütles teenijaile: siduge tema jalad ja käed ja heitke ta kõige äärmisemasse pimedusse, seal on ulumine ja hammaste kiristamine!
\par 14 Sest paljud on kutsutud, kuid vähesed on valitud!”

\section*{Maksu maksmisest keisrile}

\par 15 Siis läksid variserid ja pidasid nõu, kuidas nad teda ta kõnest saaksid võrgutada.
\par 16 Ja nad läkitavad tema juurde oma jüngrid ühes Heroodese seltsiga ja lasevad öelda: „Õpetaja, me teame, et sa oled tõemeelne ja õpetad Jumala teed tões ega hooli ühestki, sest sa ei vaata inimese isikule.
\par 17 Ütle nüüd meile, mida sa arvad, kas on tarvis anda maksu keisrile või mitte?”
\par 18 Aga Jeesus mõistis nende tigedust ja ütles: „Te silmakirjatsejad, miks te mind kiusate?
\par 19 Näidake mulle maksuraha!” Ja nad tõid tema kätte ühe teenari.
\par 20 Tema ütles neile: „Kelle kuju ja pealkiri see on?”
\par 21 Nad vastasid temale: „Keisri.” Siis ta ütles neile: „Andke siis keisrile, mis kuulub keisrile, ja Jumalale, mis kuulub Jumalale!”
\par 22 Kui nad seda kuulsid, imestasid nad ja jätsid tema ning läksid ära.

\section*{Surnute ülestõusmise küsimus}

\par 23 Sel päeval tuli tema juurde sadusere, kes ütlevad, et ülestõusmist ei olegi ja küsisid talt:
\par 24 „Õpetaja, Mooses on öelnud: kui keegi sureb lasteta, peab ta vend abielluma ta naisega ja soetama oma vennale järglase.
\par 25 Meie juures oli seitse venda. Esimene abiellus ja suri, ja et tal ei olnud järglast, jättis ta oma naise oma vennale.
\par 26 Nõndasamuti ka teine ja kolmas kuni seitsmendani.
\par 27 Kõige viimaks suri ka naine.
\par 28 Ent ülestõusmises, kelle naiseks ta jääb nende seitsme seast, sest ta on kõikidel olnud?”
\par 29 Aga Jeesus kostis ning ütles neile: „Eksite, sest te ei mõista Kirja ega Jumala väge!
\par 30 Sest ülestõusmises ei võeta naisi ega minda mehele, vaid ollakse kui Jumala inglid taevas.
\par 31 Aga kas te ei ole surnute ülestõusmisest lugenud, mis Jumal teile on öelnud:
\par 32 mina olen Aabrahami Jumal ja Iisaki Jumal ja Jaakobi Jumal. Jumal ei ole mitte surnute, vaid on elavate Jumal!”
\par 33 Ja kui rahvas seda kuulis, hämmastus ta tema õpetusest.

\section*{Missugune käsk on suur?}

\par 34 Aga kui variserid said kuulda, et ta saduseride suu oli sulgenud, tulid nad ühel meelel kokku;
\par 35 ja üks käsutundja nende seast küsis talt teda kiusates:
\par 36 „Õpetaja, missugune käsk on suur käsuõpetuses?”
\par 37 Aga Jeesus ütles temale: „Armasta Issandat, oma Jumalat, kõigest oma südamest ja kõigest oma hingest ja kõigest oma meelest.
\par 38 See on suur ja esimene käsk.
\par 39 Aga teine on selle sarnane: armasta oma ligimest nagu iseennast.
\par 40 Neis kahes käsus on kogu käsuõpetus ja prohvetid koos!”

\section*{Kelle poeg on Kristus?}

\par 41 Aga kui variserid üheskoos olid, küsis neilt Jeesus
\par 42 ning ütles: „Mis teie arvate Kristusest, kelle poeg ta on?” Nad ütlesid temale: „Taaveti!”
\par 43 Tema ütles neile: „Kuidas siis Taavet hüüab teda vaimus Issandaks, kui ta ütleb:
\par 44 Issand on öelnud minu Issandale: istu mu paremale käele, kuni ma sinu vaenlased panen su jalge alla?
\par 45 Kui nüüd Taavet teda nimetab Issandaks, kuidas ta siis on tema poeg?”
\par 46 Ja ükski ei võinud temale sõnagi vastata ja ükski ei julgenud sellest päevast peale temalt midagi enam küsida.


\chapter{23}

\section*{Jeesus hurjutab kirjatundjaid ja varisere}

\par 1 Siis rääkis Jeesus rahvale ja oma jüngritele
\par 2 ning ütles: „Moosese istmel istuvad kirjatundjad ja variserid.
\par 3 Kõike nüüd, mis nad iganes teile ütlevad, seda tehke ja pidage, aga nende tegude järgi ärge tehke, sest nad ütlevad küll, aga ei tee;
\par 4 nad seovad kokku raskeid ja ränki koormaid ja panevad neid inimeste õlgadele, aga ise nad ei taha sõrmegagi neid liigutada.
\par 5 Kõik oma teod nad teevad selleks, et inimesed neid näeksid; sest nad teevad oma palvekaukad laiad ja oma rüüde tupsud suured.
\par 6 Nad armastavad ülemat paika lauas võõruspidudel ja esimesi istmeid kogudusekodades,
\par 7 ja teretusi turgudel ja et inimesed neid kutsuksid „Rabi!”
\par 8 Aga teie ärge laske endid hüüda rabiks, sest üks on teie õpetaja; aga teie kõik olete vennad.
\par 9 Ja ärge te kutsuge kedagi maa peal oma isaks, sest üks on teie Isa, kes on taevas.
\par 10 Ja ärge te laske ka endid kutsuda juhatajaks, sest üks on teie juhataja - Kristus!
\par 11 Aga kes on ülem teie seast, see olgu teie teenija.
\par 12 Sest kes ennast ise ülendab, seda alandatakse, ja kes ennast ise alandab, seda ülendatakse.
\par 13 Aga häda teile, kirjatundjad ja variserid, te silmakirjatsejad, et te sulete taevariigi inimeste eest! Ise te ei lähe sisse ega lase sisse minna neid, kes tahavad sisse minna.
\par 14 [Häda teile, kirjatundjad ja variserid, te silmakirjatsejad, et te sööte ära leskede hooned ja loete silmakirjaks pikki palveid! Sellepärast te langete seda raskema kohtu alla.]
\par 15 Häda teile, kirjatundjad ja variserid, te silmakirjatsejad, et te käite läbi mered ja maad, et saavutada ühegi, kes heidab teie usku; ja kui ta on heitnud, siis te teete temast põrgulapse, kaks korda hullema kui te ise olete!
\par 16 Häda teile, te sõgedad teejuhid, kes ütlete: kes iganes vannub templi juures, sellel ei ole sest midagi, aga kes vannub templi kulla juures, sellel on kohustus!
\par 17 Te jõledad ja sõgedad! Sest kumb on suurem, kuld või tempel, mis kulda pühitseb?
\par 18 Ja: kes iganes vannub altari juures, sellel ei ole sest midagi; aga kes vannub annetuse juures, mis on seal peal, sellel on kohustus!
\par 19 Te jõledad ja sõgedad! Sest kumb on suurem, ohver või altar, mis ohvrit pühitseb?
\par 20 Seepärast, kes vannub altari juures, vannub tema juures ja kõige selle juures, mis tema peal on.
\par 21 Ja kes vannub templi juures, see vannub tema ja selle juures, kes seal elab.
\par 22 Ja kes vannub taeva juures, see vannub Jumala aujärje juures ja tema juures, kes sellel istub.
\par 23 Häda teile, kirjatundjad ja variserid, te silmakirjatsejad, et te maksate kümnist mündist ja tillist ja köömnest ja jätate kõrvale, mis on tähtsaim käsuõpetuses, õigluse ja halastuse ja ustavuse! Seda tuleks teha, aga teist mitte jätta tegemata.
\par 24 Te sõgedad teejuhid, kes kurnate sääski, aga neelate alla kaameleid!
\par 25 Häda teile, kirjatundjad ja variserid, te silmakirjatsejad, et te karika ja vaagna teete puhtaks väljastpoolt, aga seestpoolt on need täis röövi ja aplust!
\par 26 Sa sõge variser! Tee esmalt karikas puhtaks seestpoolt, et see ka väljastpoolt saaks puhtaks!
\par 27 Häda teile, kirjatundjad ja variserid, te silmakirjatsejad, et te olete lubjatud haudade sarnased, mis küll väljastpoolt on nägusad, aga seestpoolt on täis surnute luid ja kõike räpasust!
\par 28 Nõnda olete ka teie küll väljastpoolt näha õiged inimeste ees, kuid seestpoolt te olete täis salalikku meelt ja ülekohut.
\par 29 Häda teile, kirjatundjad ja variserid, te silmakirjatsejad, et te ehitate prohvetite haudu ja kaunistate õigete hauasambaid
\par 30 ja ütlete: kui me oleksime elanud oma esiisade ajal, ei meie küll nende osalised oleks olnud prohvetite verd valamas!
\par 31 Nõnda te siis tunnistate iseeneste vastu, et te olete nende lapsed, kes tapsid prohveteid!
\par 32 Et täitke siis ka oma vanemate mõõt!
\par 33 Te maod, te rästikute sigitis, kuidas te põgenete põrgu hukatuse eest?
\par 34 Seepärast, vaata, ma läkitan teie juurde prohveteid ja tarku ja kirjatundjaid! Ja muist neist te tapate ja lööte risti, ja muist neist te piitsutate oma kogudusekodades ja kiusate neid taga ühest linnast teise,
\par 35 et teie peale tuleks kõik vaga veri, mis on valatud maa peal alates õige Aabeli verest Sakarja, Berekja poja vereni, kelle te tapsite templi ja altari vahel.
\par 36 Tõesti ma ütlen teile, see kõik tuleb selle sugupõlve peale!

\section*{Jeesus kurdab Jeruusalemma saatuse pärast}

\par 37 Jeruusalemm, Jeruusalemm, kes tapad prohvetid ja viskad kividega surnuks need, kes sinu juurde on läkitatud! Kui mitu korda ma olen tahtnud su lapsi koguda, otsekui kana kogub oma pojakesi tiibade alla, ja teie ei ole tahtnud!
\par 38 Vaata, teie koda jäetakse teil maha!
\par 39 Sest mina ütlen teile: nüüdsest peale ei saa te mind näha, seni kui te ütlete: õnnistatud olgu, kes tuleb Issanda nimel!”


\chapter{24}

\section*{Jeesus kuulutab ette templi hävitamist}

\par 1 Ja Jeesus tuli välja pühakojast ja läks edasi. Ja tema jüngrid astusid ta juurde temale näitama pühakoja hooneid.
\par 2 Aga Jeesus ütles neile: „Eks te näe seda kõike? Tõesti ma ütlen teile, siia ei jäeta kivi kivi peale, mida maha ei kistaks!”

\section*{Tulevased õnnetused ja hädad}

\par 3 Ja kui ta Õlimäel istus, astusid jüngrid tema juurde isepäinis ja ütlesid: „Ütle meile, millal see kõik sünnib ja mis on su tulemise ja maailma-ajastu lõpetuse tunnus?”
\par 4 Aga Jeesus kostis ning ütles neile: „Katsuge, et keegi teid ei eksita!
\par 5 Sest paljud tulevad minu nime all ja ütlevad: mina olen Kristus! ja eksitavad paljusid.
\par 6 Aga te saate kuulda sõdadest ja sõnumeid sõjast; katsuge, et te ei ehmuks! Sest see peab sündima, aga ots ei ole veel käes.
\par 7 Sest rahvas tõuseb rahva vastu ja kuningriik kuningriigi vastu, ja nälga ja maavärisemisi on paiguti.
\par 8 Aga see kõik on sünnivalude hakatus!
\par 9 Siis antakse teid viletsusse ja teid tapetakse ja te olete kõigi rahvaste all minu nime pärast.
\par 10 Ja siis taganevad paljud ja annavad üksteist ära ja vihkavad üksteist.
\par 11 Ja palju valeprohveteid tõuseb, ja need eksitavad paljusid.
\par 12 Ja et ülekohus läheb väga võimsaks, jaheneb paljude armastus.
\par 13 Aga kes otsani vastu peab, see pääseb!
\par 14 Ja seda kuningriigi evangeeliumi peab kuulutatama kogu maailmas tunnistuseks kõigile rahvaile, ja siis tuleb ots.
\par 15 Kui te siis näete pühas paigas seisvat hävituse koletist, millest on rääkinud prohvet Taaniel, - kes seda loeb, see pangu tähele -
\par 16 siis põgenegu need, kes on Juudamaal, mägedele;
\par 17 kes on katusel, ärgu tulgu maha midagi oma majast võtma;
\par 18 ja kes on väljal, ärgu mingu tagasi võtma oma kuube!
\par 19 Aga häda neile, kes on käima peal, ja neile, kes imetavad neil päevil!
\par 20 Ent paluge, et teie põgenemine ei juhtuks talvel ega hingamispäeval.
\par 21 Sest siis tuleb suur viletsus, mille sarnast ei ole olnud maailma algusest kuni praeguse ajani ega tulegi.
\par 22 Ja kui neid päevi ei lühendataks, ei pääseks mitte ükski liha; aga äravalitute pärast lühendatakse need päevad.
\par 23 Kui siis keegi teile ütleb: vaata, siin on Kristus! või: vaata seal! ärge uskuge.
\par 24 Sest valekristusi ja valeprohveteid tõuseb ja need teevad suuri tunnustähti ja imesid, et eksitada, kui võimalik, ka äravalituid.
\par 25 Vaata, ma olen teile seda ette öelnud!
\par 26 Kui teile siis öeldakse: vaata, ta on kõrbes! ärge minge välja; vaata, ta on kambrites! ärge uskuge.
\par 27 Sest otsekui välk sähvab ida poolt ja paistab läände, nõnda peab olema Inimese Poja tulemine.
\par 28 Sest kus on raibe, sinna kogunevad kotkad.

\section*{Ajastu lõpp}

\par 29 Aga varsti pärast nende päevade viletsust läheb päike pimedaks ja kuu ei anna oma valget, ja tähed langevad maha taevast ja taeva vägesid kõigutatakse.
\par 30 Ja siis ilmub Inimese Poja tunnustäht taevas, ja siis hakkavad kõik rahva suguvõsad maa peal halisema ja näevad Inimese Poja tulevat taeva pilvede peal suure väe ja auhiilgusega.
\par 31 Ja ta läkitab oma inglid suure pasunahäälega, ja nad koguvad kokku ta äravalitud neljast tuulest, ühest taeva otsast teise.

\section*{Vajadus valvsuseks}

\par 32 Ent viigipuust õppige võrdumit: kui ta oksad juba on pungas ja ajavad lehti, siis te tunnete, et suvi on ligi.
\par 33 Nõnda ka teie, kui te näete seda kõike, siis teadke, et see on ligi ukse ees.
\par 34 Tõesti ma ütlen teile, et selle põlve rahvas ei lõpe ära, enne kui see kõik sünnib!
\par 35 Taevas ja maa hävivad, aga minu sõnad ei hävi mitte!
\par 36 Aga sellest päevast ja tunnist ei tea ükski, ei inglidki taevas ega ka Poeg, muud kui Isa üksi.
\par 37 Sest nõnda nagu Noa päevad olid, nõnda peab olema Inimese Poja tulemine.
\par 38 Sest nõnda nagu inimesed olid neil päevil enne veeuputust: sõid ja jõid, võtsid naisi ja läksid mehele selle päevani, mil Noa läks laeva,
\par 39 ega saanud aru, enne kui tuli veeuputus ja võttis nad puha ära; nõnda on ka Inimese Poja tulemine.
\par 40 Siis on kaks põllul: üks võetakse vastu ja teine jäetakse maha.
\par 41 Kaks naist on jahvatamas veskil: üks võetakse vastu ja teine jäetakse maha.
\par 42 Siis valvake, sest te ei tea, mil päeval teie Issand tuleb!
\par 43 Aga seda teadke, et kui peremees teaks, mil öövahi-ajal varas tuleb, küll ta siis valvaks ega laseks oma majasse sisse murda.
\par 44 Sellepärast olge teiegi valmis, sest Inimese Poeg tuleb tunnil, mil te ei arva!

\section*{Tähendamissõna heast ja halvast sulasest}

\par 45 Kes on nüüd ustav ja mõistlik sulane, kelle ta isand on pannud oma pere üle neile rooga andma õigel ajal?
\par 46 Õnnis on see sulane, keda ta isand tulles leiab nõnda tegevat!
\par 47 Tõesti ma ütlen teile, et ta paneb tema üle kogu oma vara!
\par 48 Aga kui see halb sulane mõtleb oma südames: mu isand viibib tulles!
\par 49 ja hakkab lööma oma kaassulaseid, joomaritega sööma ja jooma,
\par 50 siis selle sulase isand tuleb päeval, mil ta teda ei oota, ja tunnil, mil ta ei tea arvata,
\par 51 ja raiub ta pooleks ja annab temale osa silmakirjatsejatega; seal on ulumine ja hammaste kiristamine.


\chapter{25}

\section*{Tähendamissõna kümnest neitsist}

\par 1 Siis on taevariik kümne neitsi sarnane, kes võtsid oma lambid ja läksid välja peigmehele vastu.
\par 2 Viis neist olid rumalad ja viis mõistlikud.
\par 3 Rumalad võtsid küll lambid, aga ei võtnud õli enesega.
\par 4 Aga mõistlikud võtsid õli astjatesse ühes oma lampidega.
\par 5 Kui peigmees viibis, jäid nad kõik uniseks ja uinusid magama.
\par 6 Aga keskööl kuuldi häält: ennäe, peigmees! Minge vastu!
\par 7 Siis tõusid kõik need neitsid üles ja valmistasid oma lambid.
\par 8 Aga rumalad ütlesid mõistlikele: andke meile oma õlist, sest meie lambid kustuvad!
\par 9 Kuid mõistlikud kostsid ja ütlesid: Ei ilmaski, sellest ei jätku meile ja teile; vaid minge pigemini kaupmeeste juurde ja ostke endile!
\par 10 Aga kui nad läksid ostma, tuli peigmees, ja kes olid valmis, läksid temaga pulma. Ja uks suleti.
\par 11 Pärast tulid ka teised neitsid ja ütlesid: Issand, Issand, ava meile!
\par 12 Aga tema vastas ning ütles: „Tõesti ma ütlen teile, ma ei tunne teid!”
\par 13 Seepärast valvake, sest te ei tea päeva ega tundi!

\section*{Tähendamissõna talentidest}

\par 14 Sest see on nõnda, nagu üks inimene võõrale maale minnes kutsus oma sulased ja usaldas nende kätte oma vara.
\par 15 Ühele ta andis viis talenti, teisele kaks ja kolmandale ühe, igaühele tema jõudu mööda, ja läks võõrale maale.
\par 16 Kes viis talenti oli saanud, läks kohe ja kauples nendega ning saavutas teist viis lisaks.
\par 17 Nõndasamuti, kes oli saanud kaks, saavutas teist kaks lisaks.
\par 18 Aga see, kes oli saanud ühe, läks ja kaevas selle maa sisse ja peitis ära oma isanda raha.
\par 19 Pika aja pärast tuli nende sulaste isand tagasi ja hakkas nendega aru pidama.
\par 20 Siis astus esile see, kes oli saanud viis talenti, ja lisas teist viis talenti juurde ja ütles: „Isand, viis talenti sa usaldasid mu kätte; vaata, ma olen nendega saavutanud teist viis!”
\par 21 Tema isand ütles talle: „See on hea, sa hea ja ustav sulane! Sa oled pisku üle ustav olnud, ma panen sind palju üle; mine oma Issanda rõõmusse!”
\par 22 Esile astus ka see, kes oli saanud kaks talenti, ja ütles: „Isand, sa usaldasid mu kätte kaks talenti; vaata, ma olen nendega saavutanud teist kaks talenti!”
\par 23 Tema isand ütles talle: „See on hea, sa hea ja ustav sulane! Sa oled pisku üle ustav olnud, ma panen sind palju üle; mine oma Issanda rõõmusse!”
\par 24 Siis astus esile ka see, kes oli saanud ühe talendi, ja ütles: „Isand, ma teadsin, et sa oled kalk inimene. Sa lõikad, kuhu sa ei ole külvanud, ja kogud sealt, kuhu sa ei ole puistanud.
\par 25 Ja ma kartsin ning läksin ja matsin su talendi maha. Vaata, siin on su oma!”
\par 26 Aga tema isand kostis ning ütles temale: „Sa paha ja laisk sulane! Sa teadsid, et ma lõikan, kuhu ma ei ole külvanud, ja kogun sealt, kuhu ma ei ole puistanud.
\par 27 Sa oleksid siis pidanud mu raha andma rahavahetajate kätte, ja tulles ma oleksin saanud enese oma kasuga tagasi.
\par 28 Sellepärast võtke talent ära tema käest ja andke see sellele, kellel on kümme talenti.
\par 29 Sest igaühele, kellel on, antakse, ja temal peab olema küllalt, aga kellel ei ole, selle käest võetakse ära ka see, mis tal on.
\par 30 Ja kõlvatu sulane heitke välja kõige äärmisemasse pimedusse; seal on ulumine ja hammaste kiristamine.”

\section*{Suur kohtupäev}

\par 31 Aga kui Inimese Poeg tuleb oma auhiilguses ja kõik inglid temaga, siis ta istub oma aujärjele;
\par 32 ja siis kogutakse tema ette kõik rahvad, ja ta eraldab nad üksteisest, nagu karjane eraldab lambad sikkudest.
\par 33 Ja ta asetab lambad oma paremale käele, aga sikud vasakule käele.
\par 34 Siis Kuningas ütleb neile, kes on ta paremal käel: tulge siia, minu Isa õnnistatud, pärige kuningriik, mis teile on valmistatud maailma asutamisest!
\par 35 Sest mul oli nälg, ja te andsite mulle süüa; mul oli janu, ja te jootsite mind; ma olin võõras, ja te võtsite mind vastu;
\par 36 ma olin alasti, ja te riietasite mind; ma olin haige, ja te tulite mind vaatama; ma olin vangis, ja te tulite mu juurde.
\par 37 Siis vastavad õiged temale: Issand, millal me nägime sind näljasena ja söötsime sind, või millal janusena ja jootsime sind?
\par 38 Millal me nägime sind võõrana ja võtsime sind vastu, või alasti ja riietasime sind?
\par 39 Millal me nägime sind haigena või vangis ja tulime su juurde?
\par 40 Siis vastab Kuningas ja ütleb neile: „Tõesti ma ütlen teile, et mida te iganes olete teinud ühele nende mu vähemate vendade seast, seda te olete minule teinud!”
\par 41 Siis ta ütleb ka neile, kes on vasakul käel: „Minge ära minu juurest, te neetud, igavesse tulle, mis on valmistatud kuradile ja tema inglitele!
\par 42 Sest mul oli nälg, ja te ei andnud mulle süüa; mul oli janu, ja te ei jootnud mind;
\par 43 ma olin võõras, ja te ei võtnud mind vastu; ma olin alasti, ja te ei riietanud mind; ma olin haige ja vang, ja te ei tulnud mind vaatama.”
\par 44 Siis vastavad ka need temale: „Issand, millal me nägime sind näljas või janus või võõrana või alasti või haigena või vangis, ja ei ole sind teeninud?”
\par 45 Siis ta vastab neile nõnda: „Tõesti ma ütlen teile, et mida te iganes ei ole teinud ühele nende vähemate seast, seda te ei ole minulegi teinud.”
\par 46 Ja need lähevad igavesse karistusse, aga õiged igavesse elusse!”


\chapter{26}

\section*{Vandenõu Jeesuse tapmiseks}

\par 1 Kui Jeesus kõik need kõned oli lõpetanud, ütles ta oma jüngritele:
\par 2 „Te teate, et kahe päeva pärast on paasapüha; siis antakse Inimese Poeg risti lüüa!”
\par 3 Siis tulid kokku ülempreestrid ja rahva vanemad ülempreestri kotta, kelle nimi oli Kaifas,
\par 4 ja pidasid isekeskis nõu Jeesus kavalusega kinni võtta ja ära tappa.
\par 5 Aga nad ütlesid: „Mitte pühade ajal, et mässu ei tõuseks rahva seas!”

\section*{Jeesuse võidmine Betaanias}

\par 6 Aga kui Jeesus oli Betaanias pidalitõbise Siimona kojas,
\par 7 tuli ta juurde naine, kel oli ühes alabasterriist väga kalli salviga, ja valas selle tema pea peale, kui ta lauas istus.
\par 8 Aga kui jüngrid seda nägid, sai nende meel pahaseks ja nad ütlesid: „Mistarvis on see raiskamine?
\par 9 Sest selle oleks võinud ära müüa hulga raha eest ja anda vaestele.”
\par 10 Aga kui Jeesus seda märkas, ütles ta neile: „Miks teete vaeva sellele naisele? Ta on ju mulle heateo teinud!
\par 11 Sest vaeseid on alati teie juures, mind aga ei ole teil mitte alati.
\par 12 Sest salvi minu ihu peale valades tegi ta seda minu matmiseks.
\par 13 Tõesti ma ütlen teile, et kus iganes seda evangeeliumi kuulutatakse kogu maailmas, seal räägitakse tema mälestuseks ka sellest, mis ta on teinud.”

\section*{Juudas reedab Jeesuse}

\par 14 Siis läks üks neist kaheteistkümnest, nimega Juudas Iskariot, ülempreestrite juurde
\par 15 ning ütles: „Mis te mulle tahate anda, kui ma ta annan teie kätte?” Nad pakkusid temale kolmkümmend hõbetükki.
\par 16 Ja sellest ajast ta otsis parajat aega teda ära anda.

\section*{Jeesuse viimne paasatalle söömine}

\par 17 Aga esimesel hapnemata leibade päeval tulid jüngrid Jeesuse juurde ja ütlesid temale: „Kus sa tahad, et me sulle valmistame paasasöömaaja?”
\par 18 Tema ütles: „Minge linna ühe mehe juurde ja öelge talle: Õpetaja ütleb: mu aeg on ligi, sinu juures ma pean paasasöömaaja oma jüngritega.”
\par 19 Ja jüngrid tegid nõnda, kuidas Jeesus neid oli käskinud, ja valmistasid paasatalle.
\par 20 Aga kui õhtu tuli, istus ta lauda oma kaheteistkümne jüngriga.
\par 21 Ja kui nemad sõid, ütles ta: „Tõesti ma ütlen teile, üks teie seast annab mind ära!”
\par 22 Ja nad said väga kurvaks ning hakkasid üksteise järel temale ütlema: „Ega ometi mina see ole, Issand?”
\par 23 Tema vastas ning ütles: „Kes minuga oma käe vaagnasse pistab, see annab mind ära.
\par 24 Inimese Poeg läheb küll ära, nõnda nagu temast on kirjutatud; aga häda sellele inimesele, kelle läbi Inimese Poeg ära antakse! Hea oleks sellele inimesele, kui ta ei oleks sündinud!”
\par 25 Siis kostis Juudas, ta äraandja, ning ütles: „Ega ometi mina see ole, rabi?” Tema ütles talle: „Sina jah!”
\par 26 Ja kui nad sõid, võttis Jeesus leiva, õnnistas ja murdis ja andis oma jüngritele ning ütles: „Võtke, sööge, see on minu ihu!”
\par 27 Ja ta võttis karika, tänas ja andis neile ning ütles: „Jooge kõik selle seest!
\par 28 Sest see on minu veri, uue lepingu veri, mis paljude eest valatakse pattude andeksandmiseks.
\par 29 Aga ma ütlen teile: nüüdsest peale ma ei joo enam viinapuu viljast kuni selle päevani, mil ma ühes teiega joon uut oma Isa riigis!”
\par 30 Ja kui nad kiituslaulu olid laulnud, läksid nad välja Õlimäele.

\section*{Jeesus kuulutab ette, et jüngrid jätavad ta maha}

\par 31 Siis ütleb Jeesus neile: „Sel ööl te kõik taganete minust; sest kirjutatud on: ma löön karjast ja karja lambad pillutatakse!
\par 32 Aga pärast oma ülestõusmist ma lähen teie eele Galileasse.”
\par 33 Siis Peetrus kostis ning ütles temale: „Kui ka kõik sinust taganevad, ei tagane mina mitte ilmaski!”
\par 34 Jeesus ütles talle: „Tõesti ma ütlen sulle, täna öösel, enne kui kukk on laulnud, salgad sa mind kolm korda!”
\par 35 Peetrus ütleb temale: „Kui ma sinuga peaksin ka surema, ei salga ma sind mitte!” Samuti ütlesid ka kõik jüngrid.

\section*{Jeesus Ketsemanis}

\par 36 Siis tuli Jeesus nendega paika, mida nimetatakse Ketsemaniks, ja ütles oma jüngritele: „Istuge siin niikaua, kui ma lähen sinna ja palvetan!”
\par 37 Ja ta võttis enese juurde Peetruse ja kaks Sebedeuse poega, ja hakkas kurvaks minema ning ahastust tundma.
\par 38 Siis ta ütles neile: „Minu hing on väga kurb surmani; jääge siia ja valvake minuga!”
\par 39 Ja ta läks pisut eemale, heitis silmili maha, palvetas ning ütles: „Minu Isa, kui on võimalik, siis mingu see karikas minust mööda! Ometi mitte nõnda, kuidas mina tahan, vaid kuidas sina tahad!”
\par 40 Ja ta tuli oma jüngrite juurde ja leidis nad magamast ja ütles Peetrusele: „Nii te siis ei suutnud ühtki tundi minuga valvata?
\par 41 Valvake ja paluge, et te kiusatusse ei satuks! Vaim on küll valmis, aga liha on nõder!”
\par 42 Taas läks ta teist korda ära, palvetas ning ütles: „Minu Isa, kui see ei või muidu mööda minna kui et ma selle joon, siis sündigu sinu tahtmine!”
\par 43 Ja ta tuli jälle ja leidis nad magamast, sest nende silmad olid rasked unest.
\par 44 Ja ta jättis nad ja läks ära ja palvetas kolmandat korda ja ütles jälle needsamad sõnad.
\par 45 Siis tuleb ta oma jüngrite juurde ja ütleb neile: „Te ikka veel magate ja puhkate! Vaata, tund on ligi ja Inimese Poeg antakse patuste kätte!
\par 46 Tõuske üles, lähme! Vaata, see on ligidal, kes mind ära annab!”

\section*{Jeesuse vangistamine}

\par 47 Ja kui ta alles rääkis, vaata, siis tuli Juudas, üks neist kaheteistkümnest, ja temaga ühes suur jõuk mõõkade ja nuiadega ülempreestrite ja rahva vanemate poolt.
\par 48 Aga ta äraandja oli neile andnud märgu ja öelnud: „Keda ma suudlen, see ta on, tema võtke kinni!”
\par 49 Ja kohe ta astus Jeesuse juurde ja ütles: „Tere, rabi!” ja andis temale suud.
\par 50 Aga Jeesus ütles talle: „Sõber, mispärast oled sa siin?” Siis nad tulid tema juurde, pistsid oma käed tema külge ja võtsid ta kinni.
\par 51 Ja vaata, üks neist, kes olid Jeesusega, sirutas oma käe ja tõmbas oma mõõga välja ja lõi ülempreestri sulast ning raius ta kõrva ära.
\par 52 Siis ütles Jeesus temale: „Pista oma mõõk tuppe tagasi, sest kõik, kes mõõga tõmbavad, saavad mõõga läbi hukka!
\par 53 Või arvad sa, et ma ei või oma Isa paluda, ja ta läkitab mulle kohe enam kui kaksteist leegioni ingleid?
\par 54 Kuidas siis läheksid kirjad täide, et see nõnda peab sündima?”
\par 55 Sel tunnil ütles Jeesus hulgale: „Otsekui röövli peale olete te välja tulnud mõõkade ja nuiadega mind kinni võtma! Ma olen iga päev istunud teie juures pühakojas ja õpetanud, ja te ei ole mind kinni võtnud.”
\par 56 Aga see kõik on sündinud, et läheksid täide prohvetite kirjad. Siis jätsid kõik jüngrid tema maha ja põgenesid ära.

\section*{Jeesus ülempreestri ees}

\par 57 Aga need, kes Jeesuse olid kinni võtnud, viisid ta ülempreester Kaifase juurde, kuhu kirjatundjad ja vanemad olid kokku tulnud.
\par 58 Aga Peetrus järgis teda eemalt ülempreestri kojani, ja ta läks sisse ja istus maha ühes sulastega, et näha, kuidas asi lõpeb.
\par 59 Aga ülempreestrid ja kõik Suurkohus otsisid valetunnistust Jeesuse vastu, et teda surmata,
\par 60 ega leidnud ühtki, ehk küll palju valetunnistajaid esile tuli. Viimaks ometi astus esile kaks
\par 61 ning need ütlesid: „Tema on öelnud: ma võin Jumala templi lammutada ja kolme päevaga üles ehitada!”
\par 62 Siis tõusis ülempreester üles ja ütles temale: „Kas sa midagi ei vasta selle peale, mida need tunnistavad sinu kohta?”
\par 63 Aga Jeesus jäi vait. Ja ülempreester ütles temale: „Ma vannutan sind elava Jumala juures, et sa meile ütled, kas sina oled Kristus, Jumala Poeg?”
\par 64 Jeesus ütleb temale: „Jah, olen! Ometi ma ütlen teile: sellest ajast te näete Inimese Poega istuvat Jumala väe paremal pool ja tulevat taeva pilvede peal!”
\par 65 Siis ülempreester käristas oma riided lõhki ja ütles: „Tema on Jumalat pilganud! Milleks meile veel on tunnistajaid vaja? Vaata, nüüd te olete tema jumalapilget kuulnud!
\par 66 Mis te arvate?„ Ent nemad vastasid ning ütlesid: ”Tema on surma väärt!”
\par 67 Siis nad sülitasid temale silmi ja lõid teda rusikatega, aga mõningad peksid teda kepiga
\par 68 ja ütlesid: „Ütle, Kristus, meile kui prohvet, kes see on, kes sind lõi!”

\section*{Peetrus salgab Jeesuse}

\par 69 Aga Peetrus istus väljas sisemises õues. Ja tema juurde tuli üks tüdruk ja ütles: „Ka sina olid ühes selle galilealase Jeesusega!”
\par 70 Kuid tema salgas kõikide ees ning ütles: „Ei mina saa aru, mida sa räägid!”
\par 71 Ja kui ta läks värava poole, nägi teda teine tüdruk ja ütles neile, kes seal olid: „Ka see oli naatsaretlase Jeesusega!”
\par 72 Tema salgas jälle vandega: „Mina ei tunne seda inimest!”
\par 73 Aga natukese aja pärast lähenesid need, kes seal seisid, ja ütlesid Peetrusele: „Tõesti, sinagi oled üks nende seast, sest su keelemurregi näitab, kust sa oled!”
\par 74 Siis ta hakkas väga needma ja vanduma: „Ei mina tunne seda inimest!” Ja sedamaid laulis kukk.
\par 75 Siis tuli Peetrusele meelde Jeesuse sõna, kes talle oli öelnud: „Enne kui kukk laulab, salgad sa mind kolm korda!” Ja ta läks välja ja nuttis kibedasti.


\chapter{27}

\section*{Jeesus antakse maavalitseja Pilaatuse kätte}

\par 1 Aga kui valgeks läks, tegid kõik ülempreestrid ja rahvavanemad Jeesuse kohta otsuseks tema ära tappa.
\par 2 Nad sidusid ta kinni, viisid ta ära ja andsid ta maavalitseja Pilaatuse kätte.

\section*{Juudase surm}

\par 3 Siis Juudas, kes oli ta ära andnud, nähes, et ta oli hukka mõistetud, kahetses ning tõi need kolmkümmend hõbetükki tagasi ülempreestrite ja vanemate kätte
\par 4 ning ütles: „Ma olen pattu teinud, et ma ära andsin süütu vere!” Aga nemad ütlesid: „Mis see meisse puutub? Vaata ise!”
\par 5 Ja ta viskas hõbetükid maha templisse ja läks kõrvale ja poos enese.
\par 6 Aga ülempreestrid võtsid hõbetükid ja ütlesid: „Neid ei sünni panna ohvrikirstu, sest see on vere hind!”
\par 7 Kuid nad pidasid nõu ning ostsid nende eest potissepa põllu matmispaigaks muulastele.
\par 8 Selle tõttu hüütakse seda põldu verepõlluks tänapäevani.
\par 9 Siis läks täide, mis on öeldud prohvet Jeremija kaudu: „Ja ma võtsin need kolmkümmend hõbetükki Iisraeli lastelt, selle hinnatud mehe hinna, kelle nad olid hinnanud,
\par 10 ja andsin need potissepa põllu eest, nõnda nagu Issand mind oli käskinud!”

\section*{Jeesus Pilaatuse ees}

\par 11 Aga Jeesus seati maavalitseja ette. Ja maavalitseja küsis temalt ning ütles: „Oled sa juutide kuningas?” Aga Jeesus ütles: „Jah, olen!”
\par 12 Kui siis ülempreestrid ja vanemad tema peale kaebasid, ei vastanud ta midagi.
\par 13 Siis ütles Pilaatus temale: „Eks sa kuule, mida kõike nad tunnistavad sinu vastu?”
\par 14 Ent tema ei vastanud talle ühtainustki sõna, nõnda et maavalitseja pani seda väga imeks.
\par 15 Aga tavaliselt laskis maavalitseja pühiks rahvale vabaks ühe vangi, keda nad tahtsid.
\par 16 Ja neil oli siis kuulus vang, nimega Barabas.
\par 17 Kui nad nüüd koos olid, ütles Pilaatus neile: „Kumma te tahate, et ma teile vabaks lasen, Barabase või Jeesuse, keda hüütakse Kristuseks?”
\par 18 Sest ta teadis, et nad olid ta andnud kadeduse pärast tema kätte.
\par 19 Aga kui maavalitseja istus kohtujärjel, läkitas tema naine ta juurde ütlema: „Ärgu sul olgu ühtki tegemist selle õigega; sest ma olen täna unes palju kannatanud tema pärast!”
\par 20 Aga ülempreestrid ja vanemad meelitasid rahvahulki lahti paluma Barabast, aga hukka mõistma Jeesust.
\par 21 Ja maavalitseja kostis ning ütles neile: „Kumma neist kahest te tahate, et ma teile vabaks lasen?” Nemad ütlesid: „Barabase!”
\par 22 Pilaatus ütles neile: „Mis ma siis pean tegema Jeesusega, keda hüütakse Kristuseks?” Nemad ütlesid kõik: „Löödagu ta risti!”
\par 23 Maavalitseja küsis: „Mis kurja ta siis on teinud?” Aga nad karjusid veel enam ja ütlesid: „Löödagu ta risti!”
\par 24 Kui Pilaatus nägi, et ta midagi ei võinud parata, vaid et kära läks veel suuremaks, võttis ta vett ja pesi oma käsi rahva nähes ning ütles: „Ma olen süüta selle verest! Küll te näete!”
\par 25 Siis vastas kõik rahvas ning ütles: „Tema veri tulgu meie peale ja meie laste peale!”
\par 26 Siis ta vabastas neile Barabase, aga Jeesust ta laskis rooskadega peksta ja andis ta risti lüüa.

\section*{Jeesuse ristilöömine}

\par 27 Siis võtsid maavalitseja sõjamehed Jeesuse enestega kohtukotta ja kogusid tema juurde kokku kõik oma jõugu.
\par 28 Nad võtsid ta riidest lahti ja panid temale selga purpurmantli,
\par 29 ja punusid kibuvitstest krooni ning panid selle temale pähe, andsid ta paremasse kätte pilliroo, heitsid ta ette põlvili maha, naersid teda ja ütlesid: „Tere, juutide kuningas!”
\par 30 Ja nad sülitasid tema peale ja võtsid roo ning lõid sellega temale pähe.
\par 31 Ja kui nemad teda olid naernud, võtsid nad mantli ta seljast ära ja panid ta omad riided temale selga ja viisid ta ära risti lüüa.
\par 32 Välja minnes leidsid nad Küreene mehe, nimega Siimon. Seda nad sundisid kandma tema risti.
\par 33 Ja kui nad jõudsid paika, mida hüütakse Kolgataks - see tähendab Pealae asemeks, -
\par 34 andsid nad temale juua sapiga segatud viina. Kui ta seda oli maitsnud, ei tahtnud ta juua.
\par 35 Aga kui nad tema olid risti löönud, jaotasid nad tema riided isekeskis ja heitsid liisku nende pärast
\par 36 ning istusid maha ja valvasid teda seal.
\par 37 Ja tema pea kohale nad panid pealkirja, millele oli kirjutatud ta süü: „See on Jeesus, juutide kuningas!”
\par 38 Siis löödi temaga ühes risti kaks röövlit, üks paremale ja teine vasakule poole.
\par 39 Aga möödaminejad pilkasid teda, vangutasid pead
\par 40 ning ütlesid: „Sina, kes templi lammutad ja kolme päevaga üles ehitad, aita iseennast! Kui sa oled Jumala Poeg, siis astu ristilt maha!”
\par 41 Nõndasamuti naersid teda ka ülempreestrid ühes kirjatundjate ja vanematega ning ütlesid:
\par 42 „Teisi ta on aidanud, iseennast ta ei või aidata; kui ta on Iisraeli kuningas, astugu ta nüüd ristilt maha, siis me usume temasse!
\par 43 Ta on lootnud Jumala peale, see päästku nüüd tema, kui ta tahab; sest ta on öelnud: ma olen Jumala Poeg!”
\par 44 Samuti teotasid teda röövlidki, kes ühes temaga olid risti löödud.

\section*{Jeesuse surm}

\par 45 Aga kuuendal tunnil tuli pimedus üle kogu maa üheksanda tunnini.
\par 46 Üheksandal tunnil kisendas Jeesus suure häälega ning ütles: „Elii! Elii! Lemaa sabahtaani” see on: „Mu Jumal! Mu Jumal! Miks sa mind oled maha jätnud?”
\par 47 Aga kui mõningad neist, kes seal seisid, seda kuulsid, ütlesid nad: „Tema kutsub Eelijat!”
\par 48 Ja varsti jooksis üks nende seast, võttis käsna, täitis selle äädikaga, pistis pilliroo otsa ja jootis teda.
\par 49 Aga teised ütlesid: „Oota, saab näha, kas Eelija tuleb teda aitama?”
\par 50 Aga Jeesus kisendas jälle suure häälega ja heitis hinge.
\par 51 Ja vaata, templi eesriie kärises lõhki kaheks, ülemisest otsast alumiseni! Ja maa värises ja kaljud lõhkesid,
\par 52 ja hauad läksid lahti, ja tõusis üles palju pühade ihusid, kes olid maganud,
\par 53 ja tulid välja haudadest ja läksid pärast tema ülestõusmist pühasse linna ja ilmusid paljudele!
\par 54 Aga kui pealik ja need, kes temaga ühes Jeesust valvasid, nägid maa värisevat ja seda, mis sündis, lõid nad väga kartma ja ütlesid: „Tõesti, see oli Jumala Poeg!”
\par 55 Ja seal oli palju naisi kaugelt vaatamas, kes Jeesust järgisid Galileast, teda teenides;
\par 56 nende seas oli Maarja Magdaleena ja Maarja, Jakoobuse ja Joosese ema, ja Sebedeuse poegade ema.

\section*{Jeesuse matmine}

\par 57 Aga kui õhtu kätte jõudis, tuli rikas mees Arimaatiast, Joosep nimi, kes ka oli Jeesuse jünger.
\par 58 See läks Pilaatuse juurde ja palus Jeesuse ihu. Siis Pilaatus käskis selle anda.
\par 59 Ja Joosep võttis ihu ja mähkis selle puhtasse lõuendisse
\par 60 ja pani tema oma uude hauda, mille ta oli lasknud kalju sisse raiuda; ja ta veeretas suure kivi haua uksele ja läks ära.
\par 61 Aga Maarja Magdaleena ja teine Maarja olid seal istumas haua kohal.
\par 62 Teisel päeval, mis järgneb valmistuspäevale, tulid ülempreestrid ja variserid kokku Pilaatuse juurde
\par 63 ning ütlesid: „Isand, meile meenub, et see petis veel elus olles ütles: ma tõusen kolme päeva pärast üles!
\par 64 Käsi nüüd hauda valve all pidada kolmanda päevani, et jüngrid ei tuleks ja teda ära ei varastaks ega ütleks rahvale: ta on surnuist üles tõusnud! Nii osutuks viimne pettus pahemaks kui esimene!”
\par 65 Pilaatus ütles neile: „Siin on teil valvesalk! Minge ja pidage valvet nii hästi kui oskate!”
\par 66 Nemad läksid ja võtsid haua valve alla, pannes kivi pitseriga kinni ja seades valvurid.


\chapter{28}

\section*{Jeesuse ülestõusmine}

\par 1 Kui hingamispäev oli möödunud ja hakkas koitma nädala esimese päeva hommikul, tulid Maarja Magdaleena ja teine Maarja hauda vaatama.
\par 2 Ja vaata, suur maavärisemine sündis, sest Issanda ingel astus taevast alla, tuli ja veeretas kivi kõrvale ja istus selle peale!
\par 3 Ta nägu oli otsekui välk ja ta riided valged nagu lumi!
\par 4 Aga hirmu pärast tema ees värisesid valvajad ja jäid otsekui surnuks!
\par 5 Ent ingel hakkas kõnelema ning ütles naistele: „Ärge kartke, sest ma tean, et te otsite Jeesust, kes oli risti löödud.
\par 6 Teda ei ole siin, sest ta on üles tõusnud, nõnda nagu ta ütles. Tulge siia, vaadake aset, kus ta on maganud,
\par 7 ja minge usinasti ütlema tema jüngritele, et ta on surnuist üles tõusnud. Ja vaata, ta läheb teie eele Galileasse; seal te saate teda näha. Vaata, ma olen teile seda öelnud!”
\par 8 Ja nad läksid rutusti haua juurest minema kartuse ja suure rõõmuga ning jooksid seda kuulutama tema jüngritele.
\par 9 Ja vaata, Jeesus tuli neile vastu ja ütles: „Tere!” Ja nad tulid ta juurde ja hakkasid tema jalgade ümbert kinni ja kummardasid teda.
\par 10 Siis Jeesus ütles neile: „Ärge kartke! Minge ning teatage mu vendadele, et nad läheksid Galileasse, seal nad saavad mind näha!”
\par 11 Aga kui nad olid minemas, vaata, siis tulid mõned valvureist linna ja teatasid ülempreestritele kõik, mis oli sündinud.
\par 12 Ja need tulid kokku ühes vanematega, pidasid isekeskis nõu ja andsid sõjameestele rohkesti raha
\par 13 ning ütlesid: „Öelge, et ta jüngrid tulid öösel ja varastasid ta ära, kui me magasime.
\par 14 Ja kui maavalitseja seda saab kuulda, küll me teda meelitame ja teeme, et te saate olla mureta!”
\par 15 Need aga võtsid raha ja tegid nõnda, kuidas neid õpetati. Ja see jutt on juutide keskel levinud tänase päevani.

\section*{Jeesus ilmub oma jüngritele Galileas}

\par 16 Ja need üksteist jüngrit läksid Galileasse sellele mäele, kuhu Jeesus neid oli käskinud minna.
\par 17 Ja kui nad teda nägid, kummardasid nad teda. Aga mõned olid kahevahel.
\par 18 Ja Jeesus tuli nende juurde ja rääkis nendega ning ütles: „Minule on antud kõik meelevald taevas ja maa peal.
\par 19 Minge siis ja tehke jüngriteks kõik rahvad, neid ristides Isa ja Poja ja Püha Vaimu nimesse
\par 20 ja neid õpetades pidama kõike, mida mina teid olen käskinud. Ja vaata, mina olen iga päev teie juures maailma-ajastu otsani!”






\end{document}