\begin{document}

\title{Luuka evangeelium}

\chapter{1}

\section*{Eessõna}

\par 1 Et juba mitmed on võtnud kätte kirjutada loo neist asjust, mis meie seas tõsikindlalt on sündinud,
\par 2 nõnda nagu meile on jutustanud need, kes ise on seda näinud algusest ja on olnud sõna sulased,
\par 3 siis olen minagi arvanud heaks algusest peale kõike hoolega uurida ja järgemööda kirja panna sinule, kõrgeauline Teofilos,
\par 4 et sa õpiksid tundma nende asjade kindla tõe, mida sulle on õpetatud.

\section*{Ristija Johannese sünni ettekuulutus}

\par 5 Heroodese, Judea kuninga ajal oli preester, nimega Sakarias, Abija järjekorrast, ja temal oli naine Aaroni tütardest, ja selle nimi oli Eliisabet.
\par 6 Nad olid mõlemad õiged Jumala ees ja elasid kõigis käskudes ja Issanda seadustes laitmatult.
\par 7 Ja neil ei olnud last, sest Eliisabet oli sigimatu ja nad mõlemad olid väga elatanud.
\par 8 Aga sündis, kui ta oma teenistuse järjekorras oli preestritalitusi toimetamas Jumala ees,
\par 9 et preestriameti kombe järgi langes liisk temale minna suitsutama, ja ta läks Issanda templisse.
\par 10 Ja kõik rahvahulk oli väljas palvetamas suitsutuse tunnil.
\par 11 Seal ilmus temale Issanda ingel, seistes suitsutusaltari paremal pool.
\par 12 Teda nähes Sakarias ehmus ja hirm valdas teda.
\par 13 Aga ingel ütles talle: „Ära karda, Sakarias, sest su palvet on kuuldud ja su naine Eliisabet toob sulle ilmale poja, ja sa paned temale nimeks Johannes.
\par 14 Ja temast on sul rõõm ning hea meel, ja paljud rõõmustuvad tema sündimisest.
\par 15 Sest ta peab suur olema Issanda ees; ja viina ega vägijooki ei joo ta mitte, ja ta täitub Püha Vaimuga juba oma ema ihus;
\par 16 ja ta pöörab palju Iisraeli lapsi Issanda, nende Jumala poole;
\par 17 ja ta käib tema eel Eelija vaimus ja väes, et käänata isade südamed laste poole ja sõnakuulmatud õigete meelsuse poole, et soetada Issandale valmistatud rahvast!”
\par 18 Ja Sakarias ütles inglile: „Millest ma seda võin ära tunda? Mina olen ju vana ja mu naine on oma päevade poolest väga elatanud!”
\par 19 Ingel vastas ning ütles temale: „Mina olen Gabriel, kes seisab Jumala ees, ja ma olen läkitatud rääkima sinuga ja kuulutama sulle seda rõõmusõnumit!
\par 20 Ja vaata, sa jääd keeletuks ega saa rääkida päevani, mil see sünnib, sellepärast et sa ei ole uskunud mu sõnu, mis lähevad täide omal ajal!”
\par 21 Ja rahvas oli ootamas Sakariast ja pani imeks, et ta nii kaua viibis templis.
\par 22 Aga väljudes ta ei saanud nendega rääkida. Siis nad said aru, et ta templis oli näinud nägemust. Ja tema andis neile käega märku ning jäi keeletuks.
\par 23 Ja kui tema teenistuskorra päevad täis said, siis ta läks koju.
\par 24 Aga pärast neid päevi sai tema naine Eliisabet käima peale ja pidas ennast varjul viis kuud, öeldes:
\par 25 „Nõnda on Issand mulle teinud neil päevil, mil tema on vaadanud minu peale, et minult ära võtta mu teotus inimeste seas!”

\section*{Jeesuse sünni ettekuulutus}

\par 26 Aga kuuendas kuus läkitas Jumal ingel Gabrieli Galilea linna, mille nimi on Naatsaret,
\par 27 neitsi juurde, kes oli kihlatud Joosepi-nimelise mehega Taaveti soost. Ja neitsi nimi oli Maarja.
\par 28 Ja tulles sisse tema juurde, ütles ingel: „Tere, sa armuleidnu, Issand olgu sinuga!”
\par 29 Aga tema ehmus väga sellest sõnast ja mõtles, mis see teretus võiks tähendada.
\par 30 Siis ingel ütles temale: „Ära karda, Maarja, sest sa oled leidnud armu Jumala juures!
\par 31 Ja vaata, sa saad käima peale ning tood ilmale poja ja paned temale nimeks Jeesus.
\par 32 Tema saab suur olema ja teda peab hüütama Kõigekõrgema Pojaks, ja Issand Jumal annab temale ta isa Taaveti aujärje,
\par 33 ja ta valitseb igavesti Jaakobi soo üle ning tema kuningriigil ei ole otsa!”
\par 34 Aga Maarja ütles inglile: „Kuidas see võib sündida, kuna ma mehest ei tea?”
\par 35 Ingel vastas ning ütles temale: „Püha Vaim tuleb su peale ja Kõigekõrgema vägi varjab sind; sellepärast peab ka Püha, kes sinust sünnib, nimetatama Jumala Pojaks.
\par 36 Ja vaata, su sugulane Eliisabet, temagi on pojaga käima peal oma vanas eas, ja see on kuues kuu temal, keda öeldi olevat sigimatu,
\par 37 sest Jumalal ei ole ükski asi võimatu!”
\par 38 Aga Maarja ütles: „Vaata, siin on Issanda ümmardaja; mulle sündigu su sõna järgi!” Ja ingel läks ära tema juurest.

\section*{Maarja külastab Eliisabetti. Maarja kiituslaul}

\par 39 Neil päevil asus Maarja teele ja läks rutuga mäestikku Juuda linna.
\par 40 Ja ta tuli Sakariase kotta ja teretas Eliisabetti.
\par 41 Ja sündis, kui Eliisabet kuulis Maarja teretust, et laps hüppas tema ihus. Ja Eliisabet sai täis Püha Vaimu
\par 42 ja hüüdis suure häälega ning ütles: „Õnnistatud oled sina naiste seas ja õnnistatud on sinu ihu sugu!
\par 43 Kust see tuleb, et minu Issanda ema on tulnud minu juurde?
\par 44 Sest vaata, kui su teretuse hääl kostis mu kõrvu, siis hüppas lapsuke suurest rõõmust mu ihus!
\par 45 Ja õnnis see, kes on uskunud, et läheb täide, mis Issand temale on öelnud!”
\par 46 Siis Maarja ütles: „Mu hing ülistab väga Issandat
\par 47 ja mu vaim rõõmutseb Jumalas, minu Õnnistegijas;
\par 48 sest ta on vaadanud oma ümmardaja alanduse peale! Vaata, nüüdsest peale kiidavad mind õndsaks kõik sugupõlved,
\par 49 sest minule on teinud suuri asju Vägev, ja tema nimi on püha,
\par 50 ja tema halastus kestab põlvest põlve neile, kes teda kardavad!
\par 51 Ta on oma käsivarrega teinud võimsa teo, ta on pillutanud need, kes on suurelised oma südame meele poolest;
\par 52 ta on tõuganud maha vägevad aujärgedelt ja ülendanud alandlikud;
\par 53 näljased on ta täitnud heade andidega, aga rikkad tühjalt minema saatnud;
\par 54 ta on vastu võtnud oma sulase Iisraeli, meelde tuletades oma halastust,
\par 55 nõnda nagu ta on rääkinud meie isadele, Aabrahamile ja tema soole kuni igavikuni!”
\par 56 Ja Maarja jäi tema juurde umbes kolm kuud ja läks siis tagasi koju.

\section*{Ristija Johannese sünd. Sakariase ülistussõnad}

\par 57 Aga Eliisabeti mahasaamise aeg tuli kätte, ja ta tõi poja ilmale.
\par 58 Kui tema üleaedsed ja sugulased kuulsid, et Issand oli temale osutanud suurt halastust, olid nad rõõmsad ühes temaga.
\par 59 Ja nad tulid kaheksandal päeval lapsukest ümber lõikama ja nimetasid tema ta isa järgi Sakariaseks.
\par 60 Aga tema ema kostis ning ütles: „Ei sugugi, vaid tema nimi peab olema Johannes!”
\par 61 Kuid nemad ütlesid temale: „Su suguvõsas pole ühtki, keda hüütakse selle nimega!”
\par 62 Ja viibates isale küsisid nad, kuidas tema tahaks teda nimetada.
\par 63 Tema nõudis lauakese ja kirjutas selle peale nõnda: „Johannes on tema nimi.” Ja nad kõik panid seda imeks.
\par 64 Aga sedamaid läks tema suu ning keel lahti, ja ta hakkas rääkima, kiites Jumalat.
\par 65 Ja kartus tuli kõikidele, kes elasid nende ümbruses, ja kõigist neist asjust kõneldi Juuda mäestikus.
\par 66 Ja kõik, kes seda kuulsid, võtsid seda oma südamesse ja ütlesid: „Mis saab küll sellest lapsest?” Sest Issanda käsi oli temaga.
\par 67 Ja Sakarias, tema isa, sai täis Püha Vaimu ja hakkas ennustama, öeldes:
\par 68 „Kiidetud olgu Issand, Iisraeli Jumal, et ta on tulnud katsuma oma rahvast ja on saatnud oma rahvale lunastuse
\par 69 ning on meile äratanud päästesarve Taaveti, oma sulase kojast
\par 70 - nõnda nagu ta maailma ajastu algusest on rääkinud oma pühade prohvetite suu kaudu -
\par 71 päästet meie vaenlasist ja kõigi nende käest, kes meid vihkavad,
\par 72 et osutada halastust meie esiisadele ja tuletada meelde oma püha lepingut,
\par 73 seda vannet, mille ta vandus meie isale Aabrahamile,
\par 74 et ta laseb meid päästetuina meie vaenlaste käest kartmatult
\par 75 teda teenida pühaduses ja õiguses tema ees kogu meie eluaja.
\par 76 Ja sina, lapsuke, sind peab hüütama Kõigekõrgema prohvetiks, sest sina lähed Issanda palge eel valmistama tema teid,
\par 77 et anda tema rahvale pääste tunnetus nende pattude andeksandmises
\par 78 meie Jumala südamliku halastuse läbi, millega meid on tulnud katsuma päevatõus kõrgest,
\par 79 paistma meile, kes istume pimeduses ja surma varjus, ja juhtima meie jalgu rahuteele!”
\par 80 Aga lapsuke kasvas ja sai tugevaks vaimus. Ja ta oli kõrbes selle päevani, mil ta esines Iisraelile.


\chapter{2}

\section*{Jeesuse sünd}

\par 1 Neil päevil sündis, et keiser Augustus andis käsu üles kirjutada kogu maailm.
\par 2 See üleskirjutus oli esimene ja toimus, kui Küreenius oli Süüria maavalitseja.
\par 3 Ja kõik läksid endid laskma kirja panna, igaüks oma linna.
\par 4 Siis läks ka Joosep Galileast Naatsareti linnast üles Juudamaale Taaveti linna, mida hüütakse Petlemmaks, sest ta oli pärit Taaveti kojast ning soost,
\par 5 ennast laskma kirja panna ühes Maarja, oma kihlatud naisega, kes oli käima peal.
\par 6 Aga nende seal olles sai aeg täis, et Maarja pidi maha saama,
\par 7 ja ta tõi ilmale oma esimese poja ja mähkis ta mähkmetesse ja asetas ta sõime, sest neil polnud muud aset majas.
\par 8 Ja seal paigus oli karjaseid väljal õitsil pidamas valvet öösel oma karja juures.
\par 9 Ja Issanda ingel astus nende ette ja Issanda auhiilgus paistis nende ümber, ja nemad kartsid üliväga.
\par 10 Ja ingel ütles neile: „Ärge kartke! Sest vaata, ma kuulutan teile suurt rõõmu, mis saab osaks kõigele rahvale:
\par 11 sest teile on täna Taaveti linnas sündinud Õnnistegija, kes on Issand Kristus!
\par 12 Ja see olgu teile tunnuseks: te leiate lapse mähitud ja sõimes magavat!”
\par 13 Ja äkitselt oli ingliga ühes taeva sõjaväe hulk; need kiitsid Jumalat ning ütlesid:
\par 14 „Au olgu Jumalale kõrges ja maa peal rahu inimeste seas, kellest temal on hea meel!”
\par 15 Kui siis inglid olid läinud nende juurest ära taevasse, ütlesid karjased üksteisele: „Läki nüüd Petlemma ja vaadakem seda asja, mis on sündinud, mis Issand on meile teada andnud!”
\par 16 Ja nad tulid tõtates ning leidsid Maarja ja Joosepi ja lapsukese, kes magas sõimes.
\par 17 Aga kui nad seda nägid, teatasid nad asjast, mis neile oli öeldud selle lapsukese kohta.
\par 18 Ja kõik, kes seda kuulsid, panid imeks, mida karjased neile rääkisid.
\par 19 Ent Maarja pidas kõik need sõnad meeles ning mõtles nendele oma südames.
\par 20 Ja karjased läksid tagasi Jumalale au andes ja teda kiites kõige eest, mis nad olid kuulnud ja näinud, nõnda nagu neile oli öeldud.
\par 21 Ja kui kaheksa päeva täis sai ja laps pidi ümber lõigatama, pandi temale nimeks Jeesus, mille ingel oli pannud, enne kui laps sai ema ihusse.

\section*{Jeesuse esitamine templis}

\par 22 Ja kui nende puhastuspäevad täis said Moosese käsuõpetuse järgi, viisid nad lapse Jeruusalemma, et teda seada Issanda ette -
\par 23 nagu on kirjutatud Issanda käsuõpetuses: „Iga poeglaps, kes on ema esmasündinu, loetagu pühitsetuks Issandale”
\par 24 - ja tuua ohver, nagu Issanda käsuõpetuses on öeldud: paar turteltuvi või kaks muud tuvi.

\section*{Siimeoni õnnistussõnad}

\par 25 Ja vaata, Jeruusalemmas oli mees, Siimeon nimi. See mees oli õige ja jumalakartlik ning ootas Iisraeli troosti; ja Püha Vaim oli tema peal.
\par 26 Temale oli Püha Vaim ilmutanud, et ta ei näe surma, enne kui ta on näinud Issanda Võitut.
\par 27 Ja ta tuli Vaimu sunnil pühakotta. Ja kui vanemad lapsukese Jeesuse sinna tõid, et tema pärast teha käsuõpetuse kombe järgi,
\par 28 siis ta võttis tema sülle, kiitis Jumalat ning ütles:
\par 29 „Issand, nüüd sa lased oma sulase rahus minna oma sõna järgi,
\par 30 sest mu silmad on näinud sinu päästet,
\par 31 mille sa oled valmistanud kõigi rahvaste nähes,
\par 32 valguseks, mis peab ilmuma paganaile ja auhiilguseks oma rahvale Iisraelile!”
\par 33 Ja ta isa ja ema panid imeks, mida temast räägiti.
\par 34 Ja Siimeon õnnistas neid ja ütles Maarjale, tema emale: „Vaata, see on seatud langemiseks ja tõusmiseks paljudele Iisraelis ja tähiseks, mille vastu räägitakse -
\par 35 ent sinu omastki hingest peab mõõk läbi tungima - et saaksid avalikuks paljude südamete mõtlemised!”

\section*{Naisprohvet Anna}

\par 36 Ja seal oli naisprohvet Anna, Fanueli tütar, Aaseri suguharust. See oli väga elatanud ja oli elanud oma mehega seitse aastat pärast oma neitsipõlve,
\par 37 ja tema oli lesk, ligi kaheksakümmend neli aastat vana. See ei läinud ära pühakojast, vaid teenis Jumalat paastumiste ja palvetamistega ööd ja päevad.
\par 38 Tema tuli sinna samal tunnil ja ülistas Jumalat ning kõneles temast kõikidele, kes ootasid Jeruusalemma lunastust.

\section*{Kaheteistaastane Jeesus templis}

\par 39 Ja kui nad olid lõpetanud kõik, mis tuli teha Issanda käsuõpetuse järgi, läksid nad tagasi Galileamaale oma Naatsareti linna.
\par 40 Aga lapsuke kasvas ja sai tugevaks ning täitus tarkusega. Ja Jumala arm oli tema peal.
\par 41 Ja tema vanemad käisid iga aasta Jeruusalemmas paasapühil.
\par 42 Ja kui ta sai kaheteistaastaseks, läksid nad üles Jeruusalemma pühadeaja kommet mööda.
\par 43 Ja kui pärast nende päevade möödumist nad olid kodu poole minemas, jäi laps Jeesus Jeruusalemma; ja tema vanemad ei pannud seda tähele.
\par 44 Nad arvasid tema olevat kaasteekäijate seas, ja käisid ühe päeva tee ning otsisid teda sugulaste ja tuttavate juurest;
\par 45 ja kui nad teda ei leidnud, läksid nad tagasi Jeruusalemma ja otsisid teda sealt.
\par 46 Ja kolme päeva pärast sündis, et nad ta leidsid pühakojas istumast õpetajate keskel ja neid kuulamast ja neilt küsimast.
\par 47 Aga kõik, kes teda kuulsid, panid imeks tema mõistust ja tema kostuseid.
\par 48 Ja teda nähes ehmusid tema vanemad väga, ja ta ema ütles temale: „Poeg, miks sa oled meile nõnda teinud? Vaata, su isa ja mina oleme valuga sind otsinud!”
\par 49 Siis ta ütles neile: „Miks te mind otsite? Eks te teadnud, et ma pean olema selles, mis on mu Isa oma?”
\par 50 Aga nemad ei mõistnud seda sõna, mis ta neile rääkis.
\par 51 Ja ta läks ühes nendega ja tuli Naatsaretti ja oli neile allaheitlik. Ja tema ema pidas kõik need sõnad oma südames.
\par 52 Ja Jeesus edenes tarkuses ja pikkuses ja armus Jumala ja inimeste juures.


\chapter{3}

\section*{Ristija Johannese jutlus}

\par 1 Keiser Tibeeriuse valitsuse viieteistkümnendal aastal, kui Pontius Pilaatus oli Judea maavalitseja ja Heroodes Galilea nelivürst ja Filippus, tema vend, Iturea ja Trahhoniitisemaa nelivürst ja Lüsaanias Abileene nelivürst,
\par 2 kui ülempreestreiks olid Annas ja Kaifas, sai Jumala sõna Sakariase poja Johannese kätte kõrbes.
\par 3 Ja ta tuli kogu Jordani ümberkaudsele maale ja kuulutas meeleparandusristimist pattude andekssaamiseks,
\par 4 nagu on kirjutatud prohvet Jesaja sõnade raamatus: „Hüüdja hääl on kõrbes: valmistage Issanda teed, õgvendage tema teerajad!
\par 5 Kõik kuristikud täidetagu ja kõik mäed ja mäekünkad tehtagu madalaks; mis on kõver, saagu sirgeks, ja mis on mätlik, siledaks teeks;
\par 6 siis kõik liha näeb Jumala päästet!”
\par 7 Siis ta ütles rahvale, kes tuli välja ennast laskma tema poolt ristida: „Te rästikute sigitis, kes on teile andnud märku põgeneda tulevase viha eest?
\par 8 Seepärast kandke õiget meeleparanduse vilja ja ärge hakake iseenestes ütlema: meil on isaks Aabraham! Sest ma ütlen teile, et Jumal võib neist kividest äratada Aabrahamile lapsi!
\par 9 Kuid kirves on ka juba puude juure küljes! Iga puu nüüd, mis head vilja ei kanna, raiutakse maha ja visatakse tulle!”
\par 10 Ja rahvas küsis temalt ning ütles: „Mis meil siis tuleb teha?”
\par 11 Tema vastas ning ütles neile: „Kellel on kaks vammust, see andku sellele, kellel ei ole, ja kellel on rooga, see tehku nõndasamuti!”
\par 12 Ent ka tölnereid tuli endid laskma ristida, ja need ütlesid temale: „Õpetaja, mis meil tuleb teha?”
\par 13 Aga ta ütles neile: „Ärge nõudke rohkem kui teile on seatud!”
\par 14 Ka küsisid temalt sõjamehed ning ütlesid: „Ja meie, mis siis meil tuleb teha?” Ja ta ütles neile: „Ärge tehke kellelegi liiga ja ärge survake kedagi, vaid olge rahul oma palgaga!”
\par 15 Aga kui rahvas ootas ja kõik oma südames mõtlesid Johannesest, kas tema vahest ei ole Kristus,
\par 16 siis kostis Johannes ja ütles kõikidele: „Mina ristin teid veega; aga on tulemas vägevam minust, kelle jalatsite paela ma ei ole kõlvuline lahti päästma; tema ristib teid Püha Vaimu ja tulega!
\par 17 Tema visklabidas on ta käes, et teha puhtaks oma rehealune ja koguda nisud oma aita, aga aganad põletada ära kustutamatu tulega!”
\par 18 Ka veel muude asjade pärast manitsedes kuulutas ta armuõpetust rahvale.
\par 19 Aga kui nelivürst Heroodes sai talt noomida Heroodiase, oma venna Filippuse naise pärast ja kõigi pahategude pärast, mis Heroodes oli teinud,
\par 20 sai ta peale kõige muu veel toime sellega, et ta Johannese pani kinni vanglasse.

\section*{Jeesuse ristimine}

\par 21 Ent sündis, kui kõike rahvast ristiti ja ka Jeesus oli ristitud ja palvetas, et taevas avati
\par 22 ja Püha Vaim tuli ihulikul kujul alla nagu tuvi tema peale, ja hääl kostis taevast: „Sina oled mu armas Poeg, sinust on mul hea meel!”

\section*{Jeesuse esivanemad}

\par 23 Ja Jeesus oli oma tegevust alates umbes kolmkümmend aastat vana ja oli, nagu arvati, Joosepi poeg, kes oli Eeli poeg.
\par 24 Eeli oli Mattati, see oli Leevi, see oli Melki, see oli Jannai, see oli Joosepi,
\par 25 see oli Matatia, see oli Aamose, see oli Nahumi, see oli Esli, see oli Nangai,
\par 26 see oli Mahati, see oli Matatia, see oli Simei, see oli Jooseki, see oli Jooda,
\par 27 see oli Joohanani, see oli Reesa, see oli Serubaabeli, see oli Sealtieli, see oli Neeri,
\par 28 see oli Melki, see oli Adi, see oli Koosami, see oli Elmadami, see oli Eeri,
\par 29 see oli Jeesuse, see oli Elieseri, see oli Joorimi, see oli Mattati, see oli Leevi,
\par 30 see oli Siimeoni, see oli Juuda, see oli Joosepi, see oli Joonami, see oli Eljakimi,
\par 31 see oli Melea, see oli Menna, see oli Matata, see oli Naatani, see oli Taaveti,
\par 32 see oli Jesse, see oli Oobedi, see oli Boase, see oli Salma, see oli Nahsoni,
\par 33 see oli Amminaadabi, see oli Admini, see oli Arni, see oli Hesromi, see oli Peretsi, see oli Juuda,
\par 34 see oli Jaakobi, see oli Iisaki, see oli Aabrahami, see oli Taara, see oli Naahori,
\par 35 see oli Serugi, see oli Reu, see oli Pelegi, see oli Eeberi, see oli Sela,
\par 36 see oli Keenani, see oli Arpaksadi, see oli Seemi, see oli Noa, see oli Lemeki,
\par 37 see oli Metuusala, see oli Eenoki, see oli Jeredi, see oli Mahalaleli, see oli Keenani,
\par 38 see oli Eenose, see oli Seti, see oli Aadama, see oli Jumala poeg.


\chapter{4}

\section*{Jeesust kiusatakse kõrbes}

\par 1 Siis tuli Jeesus, täis Püha Vaimu, tagasi Jordani äärest, ja ta aeti Vaimu läbi kõrbe;
\par 2 ja nelikümmend päeva kurat kiusas teda. Ja ta ei söönud midagi neil päevil. Ja kui need möödusid, tuli temale nälg.
\par 3 Siis ütles kurat temale: „Kui sa oled Jumala Poeg, siis ütle sellele kivile, et ta leivaks muutuks!”
\par 4 Jeesus vastas temale: „Kirjutatud on, et inimene ei ela ükspäinis leivast!”
\par 5 Siis kurat viis tema kõrgele ning näitas talle kõik maailma kuningriigid viivu ajaga.
\par 6 Ja kurat ütles temale: „Ma annan sulle meelevalla kõigi nende üle ja nende auhiilguse, sest see on antud minule ja mina annan selle, kellele ma iganes tahan.
\par 7 Kui sa nüüd kummardad minu ette, on kõik sinu!”
\par 8 Jeesus vastas ning ütles temale: „Kirjutatud on: sina pead Issandat, oma Jumalat kummardama ja ükspäinis teda teenima!”
\par 9 Ja ta viis tema Jeruusalemma ja asetas ta pühakoja harjale ning ütles talle: „Kui sa oled Jumala Poeg, siis kukuta ennast siit alla;
\par 10 sest on kirjutatud: ta annab oma inglitele käsu sinu pärast, et nad sind hoiaksid
\par 11 ja et nad sind kätel kannaksid, et sa oma jalga vastu kivi ei tõukaks!”
\par 12 Jeesus vastas ning ütles temale: „Öeldud on: sa ei tohi Issandat, oma Jumalat kiusata!”
\par 13 Ja kui kurat kõik kiusamise oli lõpetanud, läks ta temast eemale mõneks ajaks.

\section*{Jeesus hakkab kuulutama ja õpetama}

\par 14 Ja Jeesus läks tagasi Vaimu väes Galileasse; ja kuuldus temast levis kogu ümberkaudsele maale.
\par 15 Ja ta õpetas nende kogudusekodades, ja kõik andsid temale au.

\section*{Jeesust põlatakse kodukohas}

\par 16 Ja ta tuli Naatsaretti, kus ta oli üles kasvanud, ja läks oma harjumust mööda hingamispäeval kogudusekotta ja tõusis üles lugema.
\par 17 Ja tema kätte anti prohvet Jesaja raamat. Ja kui ta raamatu avas, leidis ta koha, kuhu oli kirjutatud:
\par 18 „Issanda Vaim on minu peal; seepärast on ta mind võidnud kuulutama evangeeliumi vaestele; ta on mind läkitanud kuulutama vabakssaamist seotuile ja nägemist pimedaile, laskma rõhutuid vabadusse,
\par 19 kuulutama Issanda meelepärast aastat!”
\par 20 Ja kui ta raamatu oli kokku keeranud, andis ta selle kojasulase kätte ja istus maha; ja kõikide silmad kogudusekojas olid sihitud tema peale.
\par 21 Ta hakkas neile kõnelema: „Täna on see kiri täide läinud teie kuuldes!”
\par 22 Ja kõik tunnustasid teda ja panid imeks neid armu sõnu, mis lähtusid ta suust, ning ütlesid: „Eks see ole Joosepi poeg?”
\par 23 Siis ta ütles neile: „Küllap te mulle ütlete selle vanasõna: arst, aita iseennast! Neid suuri asju, mida me oleme kuulnud Kapernaumas sündinud olevat, tee ka siin oma kodukohas!”
\par 24 Aga ta ütles: „Tõesti ma ütlen teile, ükski prohvet ei ole meelepärane oma kodumaal.
\par 25 Tõepoolest ma ütlen teile: palju lesknaisi oli Iisraelis Eelija ajal, mil taevas oli kinni pandud kolm aastat ja kuus kuud, kui suur nälg tuli üle kogu maa;
\par 26 aga ei ühegi juurde nende seast ei läkitatud Eelijat kui vaid Sareptasse Siidonimaale lesknaise juurde.
\par 27 Ja palju pidalitõbiseid oli Iisraelis prohvet Eliisa ajal, aga ükski neist ei saanud puhtaks kui vaid Naaman, see süürlane!”
\par 28 Seda kuuldes said kõik kogudusekojas täis viha,
\par 29 ja tõusid ning tõukasid tema linnast välja ja viisid ta üles mäerünkale, millele nende linn oli ehitatud, et teda kukutada ülepeakaela alla.
\par 30 Aga tema sammus nende keskelt läbi ja läks oma teed.

\section*{Vaimuhaige tervekstegemine}

\par 31 Ja ta tuli Kapernauma, Galilea linna, ja õpetas neid hingamispäevil.
\par 32 Ja nemad hämmastusid tema õpetusest, sest ta sõnal oli meelevald.
\par 33 Ja kogudusekojas oli mees, kellel oli rüveda pahareti vaim, see kisendas suure häälega:
\par 34 „Oeh, mis on meil sinuga tegemist, Jeesus Naatsaretlane? Oled sa tulnud meid hävitama? Ma tunnen sind, kes sa oled, Jumala Püha!”
\par 35 Siis Jeesus sõitles teda ning ütles: „Ole vait ja mine temast välja!” Ja kuri vaim viskas ta maha nende keskele ja väljus temast ega teinud talle ühtki kahju.
\par 36 Ja kohkumine valdas kõiki, ja nad rääkisid isekeskis ning ütlesid: „Mis lugu see on, et ta meelevalla ja väega käsib rüvedaid vaime, ja need väljuvad?”
\par 37 Ja kuuldus temast levis kõigisse ümberkaudse maa paikadesse.

\section*{Jeesus teeb terveks Siimona ämma ja teisi}

\par 38 Aga ta tõusis üles kogudusekojast ja läks Siimona kotta. Ja Siimona ämm oli haige raskes palavikus, ja nad palusid teda tema pärast.
\par 39 Ja ta astus tema juurde ja ähvardas palavikku, ja see lahkus temast. Ja sedamaid tõusis naine üles ja ümmardas neid.
\par 40 Aga kui päike looja läks, tõid kõik, kellel oli haigeid mõnesugustes tõbedes, need tema juurde, ja ta pani igaühe peale oma käed ning tegi nad terveks.
\par 41 Aga ka kurjad vaimud läksid paljudest välja kisendades ning öeldes: „Sina oled Jumala Poeg!” Ja tema sõitles neid ega lubanud neid rääkida, sest et nad teadsid tema olevat Kristuse.
\par 42 Ja päeva tulles väljus ta ja läks tühja paika; kuid rahvahulgad otsisid teda ning tulid tema juurde. Ja nad pidasid teda kinni, et ta ei läheks ära nende juurest.
\par 43 Aga tema ütles neile: „Ma pean ka muudele linnadele kuulutama Jumala riigi evangeeliumi, sest selleks ma olen läkitatud!”
\par 44 Ja ta jutlustas Galilea kogudusekodades.


\chapter{5}

\section*{Jeesus kutsub oma esimesed jüngrid}

\par 1 Aga sündis, kui rahvas tema juurde tungis Jumala sõna kuulama ja ta seisis Gennesareti järve ääres,
\par 2 et ta nägi kaht paati järve rannas seisvat; aga kalamehed olid neist väljunud ja loputasid võrke.
\par 3 Tema astus ühte paati, mis oli Siimona oma, ja palus teda natuke maad rannast eemale sõuda. Siis ta istus maha ja õpetas rahvast paadist.
\par 4 Aga kui ta lakkas kõnelemast, ütles ta Siimonale: „Sõua sügavale kohale ja heitke oma võrgud välja loomuse katseks!”
\par 5 Siimon vastas ning ütles temale: „Õpetaja, me oleme kogu öö rühelnud ega ole ühtki saanud, aga sinu sõna peale ma heidan võrgud välja!”
\par 6 Ja kui nad seda tegid, said nad suure hulga kalu; ja nende võrgud rebenesid.
\par 7 Ja nad viipasid oma kaaslastele teises paadis, et nad tuleksid ja neid aitaksid. Ja nad tulid ja täitsid mõlemad paadid nõnda, et need olid vajumas.
\par 8 Seda nähes heitis Siimon Peetrus maha Jeesuse põlvede ette ning ütles: „Issand, mine minu juurest ära, sest ma olen patune inimene!”
\par 9 Sest kohkumine oli vallanud teda ja kõiki, kes temaga olid, kalasaagi pärast, mis nad olid saanud;
\par 10 nõndasamuti ka Jakoobust ja Johannest, Sebedeuse poegi, kes olid Siimona kaaslased. Ja Jeesus ütles Siimonale: „Ära karda, nüüdsest alates pead sa inimesi püüdma!”
\par 11 Ja nad vedasid paadid maale ja jätsid kõik maha ja järgisid teda.

\section*{Pidalitõbise tervekstegemine}

\par 12 Ja kui ta viibis ühes linnas, vaata, siis oli seal mees täis pidalitõbe. Kui ta Jeesust nägi, heitis ta silmili maha, palus teda ning ütles: „Issand, kui sa tahad, võid sa mind puhtaks teha!”
\par 13 Ja ta sirutas oma käe ja puudutas teda ning ütles: „Ma tahan, saa puhtaks!” Ja sedamaid lahkus pidalitõbi temast.
\par 14 Ja tema keelas teda seda kellelegi ütlemast. „Mine vaid” - ütles ta -„ja näita ennast preestrile ja ohverda oma puhastuse eest, nõnda nagu Mooses on seadnud neile tunnistuseks!”
\par 15 Aga sõnum temast levis laiemale, ja palju rahvast tuli kokku teda kuulama ja temalt tervist saama tõbedest.
\par 16 Aga tema läks kõrvale tühjadesse paikadesse ja palvetas seal.

\section*{Halvatu tervekstegemine}

\par 17 Ja ühel päeval, kui ta õpetas, oli seal istumas varisere ja käsutundjaid, kes olid tulnud kõigist Galilea ja Judea küladest ja Jeruusalemmast; ja Issanda vägi oli seal selleks, et Jeesus terveks teeks.
\par 18 Ja vaata, mehed tõid voodiga inimese, kes oli halvatud; ja nad püüdsid teda sisse viia ja tema ette panna.
\par 19 Ja kui nad rahva pärast ei leidnud, kust nad ta oleksid saanud sisse viia, läksid nad üles katusele ja lasksid ta katuse kivide vahelt alla kõige voodiga otse Jeesuse ette maha.
\par 20 Kui tema nende usku nägi, ütles ta: „Inimene, sinu patud on sulle andeks antud!”
\par 21 Siis kirjatundjad ja variserid hakkasid mõtlema ning ütlesid: „Kes see on, kes nõnda Jumalat pilgates räägib? Kes muu võib patte andeks anda kui üksnes Jumal?”
\par 22 Aga Jeesus tundis ära nende mõtlemised ja kostust andes ta ütles neile: „Mida te mõtlete oma südames?
\par 23 Kumb on hõlpsam öelda: su patud on sulle andeks antud, või öelda: tõuse üles ja kõnni?
\par 24 Aga et te teaksite, et Inimese Pojal on meelevald patte andeks anda,„ - siis ütles ta halvatule: ”Ma ütlen sulle, tõuse üles ja võta oma voodi ning mine koju!”
\par 25 Ja sedamaid tõusis ta nende nähes üles, võttis oma voodi, millel ta oli lamanud, ja läks koju Jumalale au andes.
\par 26 Ja hämmastus valdas kõiki, ja nad andsid Jumalale au ja said täis kartust ning ütlesid: „Me oleme täna näinud imelisi asju!”

\section*{Leevi kutsumine}

\par 27 Pärast seda ta läks välja ja nägi tölnerit, Leevi nimi, tollihoones istuvat ja ütles temale: „Järgi mind!”
\par 28 Ja ta jättis kõik maha, tõusis ja järgis teda.
\par 29 Ja Leevi tegi temale suure võõruspeo oma majas, ja suur hulk tölnereid ja teisi oli ühes nendega lauas istumas.
\par 30 Siis variserid ja nende kirjatundjad nurisesid tema jüngrite vastu ning ütlesid: „Miks te sööte ja joote ühes tölnerite ja patustega?”
\par 31 Ja Jeesus vastas neile ning ütles: „Ei ole arsti vaja tervetele, ainult haigetele!
\par 32 Ma ei ole tulnud õigeid kutsuma meeleparandusele, vaid patuseid!”

\section*{Paastumise küsimus}

\par 33 Aga nemad ütlesid temale: „Johannese jüngrid paastuvad sagedasti ja peavad palveid, samuti variseridegi omad, aga sinu jüngrid söövad ja joovad!”
\par 34 Jeesus ütles neile: „Kas te võite peiupoisse panna paastuma sel ajal, kui peigmees on nende juures?
\par 35 Ent päevad tulevad, mil peigmees võetakse neilt ära; siis nad paastuvad neil päevil!”
\par 36 Aga ta ütles neile ka tähendamissõna: „Ükski ei lõika paika uue kuue küljest ega pane seda vana kuue peale, muidu ta lõikab katki ka uue kuue, ja pealegi ei sobi vanale kuuele paik uue küljest.
\par 37 Ka ei pane keegi värsket viina vanadesse nahklähkreisse, muidu värske viin ratkub lähkrid ja ta jookseb maha ning lähkrid lähevad rikki;
\par 38 vaid värske viin tuleb panna uutesse nahklähkritesse,
\par 39 siis keegi, kes joob vana, ei himusta värsket, vaid ütleb: vana on maitsev!”


\chapter{6}

\section*{Hingamispäeva pühitsemisest}

\par 1 Ja hingamispäeval ta juhtus minema läbi viljapõldude; ja tema jüngrid katkusid päid, hõõrusid neid käte vahel ja sõid.
\par 2 Ent mõningad variseridest ütlesid: „Miks te teete seda, mida hingamispäeval ei ole luba teha?”
\par 3 Aga Jeesus vastas ning ütles neile: „Eks te ole seda lugenud, mida Taavet tegi, kui temal ja ta kaaslastel oli nälg,
\par 4 kuidas ta läks Jumala kotta ja võttis vaateleivad, sõi ning andis ka oma kaaslasile, ehk küll neid ei olnud luba süüa muil kui üksnes preestritel!”
\par 5 Ja ta ütles neile: „Inimese Poeg on hingamispäeva isand!”
\par 6 Aga juhtus ühel teisel hingamispäeval, et ta läks kogudusekotta ja õpetas. Ja seal oli inimene, kelle parem käsi oli kuivanud.
\par 7 Aga kirjatundjad ja variserid varitsesid teda, kas ta hingamispäeval peaks terveks tegema, et nad leiaksid kaebust tema peale.
\par 8 Tema aga teadis nende mõtted ja ütles mehele, kellel oli kuivanud käsi: „Tõuse üles ja astu keskele!” Ja ta tõusis ja seisis seal.
\par 9 Siis ütles Jeesus neile: „Ma küsin teilt: kas hingamispäeval on luba teha head või kurja, hinge päästa või hukka saata?”
\par 10 Ja ta vaatas ümber nende kõikide peale ja ütles mehele: „Siruta oma käsi!” Ja see tegi nõnda. Ja tema käsi sai jälle terveks.
\par 11 Nemad aga sattusid meeletusse vimma ja rääkisid üksteisega sellest, mida nad küll võiksid teha Jeesusele.

\section*{Apostlite valimine}

\par 12 Neil päevil sündis ka, et ta läks välja mäele palvetama ja viibis kogu öö Jumala palumises.
\par 13 Ja kui valgeks läks, kutsus ta oma jüngrid enese juurde ja valis nende seast kaksteist, keda ta nimetas ka apostliteks:
\par 14 Siimona, keda ta nimetas ka Peetruseks, ja tema venna Andrease, Jakoobuse ja Johannese, Filippuse ja Bartolomeuse,
\par 15 Matteuse ja Tooma, Jakoobuse, Alfeuse poja, ja Siimona, keda hüütakse Selooteseks,
\par 16 ja Juuda, Jakoobuse poja, ja Juudas Iskarioti, kes sai äraandjaks.
\par 17 Nendega ta astus alla ja jäi peatuma lagedale paigale; ja seal oli suur hulk tema jüngreid ja kogu Juudamaalt ja Jeruusalemmast ning Tüürose ja Siidoni rannamaalt suur rahvakogu,
\par 18 kes oli tulnud teda kuulama ja ennast laskma parandada oma haigustest. Ja ka need, keda rüvedad vaimud vaevasid, said terveks.
\par 19 Ja kõik rahvas püüdis teda puudutada, sest vägi lähtus temast ja parandas kõik.

\section*{Õndsakskiitmised ja hukkamõistmised}

\par 20 Ja tema tõstis silmad oma jüngrite poole ja ütles: „Õndsad olete, vaesed, sest teie päralt on Jumala riik.
\par 21 Õndsad teie, kellel nüüd on nälg, sest teid peab toidetama. Õndsad teie, kes nüüd nutate, sest siis te naerate.
\par 22 Õndsad olete, kui inimesed teid vihkavad ja kui nad teid kogudusest välja heidavad ning laimavad ja teie nime kui paha nime hülgavad Inimese Poja pärast.
\par 23 Rõõmutsege sel päeval ja hüpake, sest vaata, teie palk on suur taevas! Tegid ju nõndasamuti prohvetitega nende isad.
\par 24 Kuid häda teile rikastele! Sest teil on troost käes.
\par 25 Häda teile, kes nüüd olete täidetud, sest teile tuleb nälg! Häda teile, kes nüüd naerate, sest te saate veel leinata ja nutta!
\par 26 Häda teile, kui kõik inimesed teist räägivad head, sest nõndasamuti tegid nende isad valeprohvetitele!

\section*{Vaenlase armastamisest}

\par 27 Aga teile, kes kuulete, ütlen ma: armastage oma vaenlasi, tehke head neile, kes teid vihkavad;
\par 28 õnnistage neid, kes teid neavad; paluge nende eest, kes teid siunavad.
\par 29 Kes sind lööb vastu kõrva, sellele paku ka teine; ja kes sult võtab kuue, sellele ära keela vammust.
\par 30 Anna igaühele, kes sind palub, ja sellelt, kes võtab sinu oma, ära nõua tagasi.
\par 31 Ja nõnda nagu te tahate, et inimesed teile teeksid, nõnda tehke neile.
\par 32 Ja kui te armastate neid, kes teid armastavad, mis tänu on teil sellest? Sest patusedki armastavad neid, kes neid armastavad.
\par 33 Ja kui te teete head oma heategijaile, mis tänu on teil sellest? Sest patusedki teevad sedasama.
\par 34 Ja kui te laenate neile, kellelt te loodate tagasi saada, mis tänu on teil sellest? Patusedki laenavad patustele, et nad samavõrra saaksid tagasi.
\par 35 Aga armastage oma vihamehi ja tehke head ja laenake ilma midagi selle vastu lootmata; siis teie palk on suur ja te olete Kõigekõrgema lapsed, sest tema on lahke ka tänamatute ja kurjade vastu.
\par 36 Olge armulised, nõnda nagu teie Isa on armuline.

\section*{Kohtumõistmisest teiste üle}

\par 37 Ja ärge mõistke kohut, siis ei mõisteta ka teie üle kohut. Ärge mõistke hukka, siis ei mõisteta ka teid hukka. Andke andeks, siis antakse ka teile andeks.
\par 38 Andke, siis antakse ka teile; hea, tuubitud ja raputatud ja kuhjaga mõõt antakse teie rüppe; sest sama mõõduga, millega teie mõõdate, mõõdetakse ka teile.”
\par 39 Aga ta ütles neile võrdumi: „Egas pime või pimedat talutada? Eks nad mõlemad kuku auku?
\par 40 Ei ole õpilane ülem oma õpetajast, ent iga täiesti õpetatu on samasugune kui ta õpetaja.
\par 41 Aga miks sa näed pindu oma venna silmas, kuid palki su omas silmas sa ei pane tähele?
\par 42 Ehk kuidas sa võid öelda oma vennale: vend, lase ma tõmban pinnu sinu silmast! ja sa ise ei näe palki omas silmas? Sa silmakirjatseja! Tõmba esmalt palk omast silmast ja siis sa seletad pindu välja tõmmata oma venna silmast!
\par 43 Sest ei ole head puud, mis kannaks halba vilja, ega jälle halba puud, mis kannaks head vilja.
\par 44 Sest igat puud tuntakse tema viljast. Ei nopita ju viigimarju kibuvitstest ega korjata viinamarjakobaraid ohakatest.
\par 45 Hea inimene toob esile head oma südame heast tagavarast ja halb inimene toob esile halba oma südame halvast tagavarast. Sest mida süda täis, sellest suu räägib!

\section*{Kaks maja ehitajat}

\par 46 Aga miks te mind hüüate: Issand, Issand! ega tee, mida ma ütlen?
\par 47 Igaüks, kes tuleb minu juurde ja kuuleb mu sõnu ning teeb nende järgi - ma näitan teile, kelle sarnane ta on.
\par 48 Ta on inimese sarnane, kes maja ehitades kaevas ja võttis sügavasse ning rajas tema aluse kaljule. Kui siis suurvesi tuli, murdus vool vastu seda maja ja ei suutnud seda kõigutada, sest see oli hästi ehitatud.
\par 49 Aga kes kuuleb ega tee, see on inimese sarnane, kes maja rajas maa peale ilma aluseta; ja voolav vesi rõhus selle vastu ja see langes kohe kokku; ja selle maja varisemine oli suur!”


\chapter{7}

\section*{Jeesus teeb terveks sõjapealiku sulase}

\par 1 Kui ta kõik oma kõned oli lõpetanud kuulavale rahvale, läks ta Kapernauma.
\par 2 Ja ühe sõjapealiku sulane oli haige ning suremisel; ja ta oli kallis tema meelest.
\par 3 Aga kui ta Jeesusest kuulis, läkitas ta juutide vanemad tema juurde ja palus teda, et ta tuleks ja teeks tema sulase terveks.
\par 4 Kui need saabusid Jeesuse juurde, palusid nad teda üliväga ning ütlesid: „Ta on seda väärt, et sa temale seda teed;
\par 5 sest ta armastab meie rahvast ja tema on meile ehitanud kogudusekoja.”
\par 6 Jeesus läks nendega. Aga kui ta enam ei olnud majast kaugel, läkitas pealik oma sõbrad talle ütlema: „Issand, ära tee enesele vaeva, sest mina ei ole väärt, et sina mu katuse alla tuled.
\par 7 Sellepärast ei ole mina ennast arvanud väärt tulema sinu juurde, vaid ütle aga sõna, siis mu sulane paraneb.
\par 8 Sest minagi olen inimene, kes on pandud valitsuse alla, ja minu all on sõjamehi, ja kui ma ütlen ühele: mine! siis ta läheb, ja teisele: tule! siis ta tuleb, ja oma sulasele: tee seda! siis ta teeb.”
\par 9 Kui Jeesus seda kuulis, imetles ta teda ja pöördudes rahva poole, kes järel käis, ütles ta: „Ma ütlen teile, ei ole ma Iisraelistki leidnud nii suurt usku!”
\par 10 Ja kui need, kes olid läkitatud, jälle koju said, leidsid nad sulase terve olevat.

\section*{Jeesus äratab surnuist Naini noormehe}

\par 11 Järgmisel päeval läks ta linna, mida hüütakse Nainiks, ja temaga ühes läksid tema jüngrid ja palju rahvast.
\par 12 Kui ta linna värava ligi jõudis, vaata, siis kanti välja surnut, oma ema ainust poega; ja ema oli lesk. Ja suur hulk linna rahvast oli temaga.
\par 13 Ja kui Issand teda nägi, oli tal väga hale meel temast ja ütles talle: „Ära nuta!”
\par 14 Ja ta astus ligi ja puudutas puusärki; aga kandjad jäid seisma. Ja ta ütles: „Noormees, ma ütlen sulle, tõuse üles!”
\par 15 Ja surnu tõusis istukile ja hakkas rääkima. Ja ta andis tema ta emale.
\par 16 Aga neid kõiki valdas hirm, ja nad andsid au Jumalale ning ütlesid: „Suur prohvet on meie seas tõusnud ja Jumal on tulnud oma rahvast katsuma!”
\par 17 Ja see jutt temast levis kogu Juudamaale ja kõigisse ümberkaudseisse paikadesse.

\section*{Ristija Johannes}

\par 18 Ja seda kõike kuulutasid Johannesele tema jüngrid. Ja Johannes kutsus enese juurde kaks oma jüngritest
\par 19 ja läkitas nad Issanda juurde küsima: „Kas oled sina see, kes tuleb, või peame ootama teist?”
\par 20 Aga kui mehed tema juurde jõudsid, ütlesid nad: „Ristija Johannes läkitas meid sinu juurde küsima: kas oled sina see, kes tuleb, või peame ootama teist?”
\par 21 Ent samal tunnil ta tegi paljud terveks tõbedest ja valudest ja kurjest vaimudest, ja paljudele pimedaile ta kinkis nägemise.
\par 22 Ja ta vastas ning ütles neile: „Minge ja kuulutage Johannesele, mida te olete näinud ja kuulnud: pimedad näevad jälle, jalutud käivad, pidalitõbised saavad puhtaks, kurdid kuulevad, surnud äratatakse üles, vaestele kuulutatakse evangeeliumi;
\par 23 ja õnnis on see, kes minust ei pahandu!”
\par 24 Aga kui Johannese käskjalad olid ära läinud, hakkas ta rahvale rääkima Johannesest: „Mida te olete läinud välja kõrbe vaatama? Pilliroogu, mida tuul kõigutab?
\par 25 Või mida te olete välja läinud vaatama? Inimest, pehmete riietega riietatud? Vaata, kes käivad toredas riietuses ja endid hellitavad, need on kuninglikes kodades!
\par 26 Või mida te olete välja läinud vaatama? Kas prohvetit? Tõesti, ma ütlen teile, ta on veel rohkem kui prohvet!
\par 27 Tema on see, kellest on kirjutatud: vaata, mina läkitan sinu palge eele oma ingli, kes sulle tee valmistab sinu ees!
\par 28 Mina ütlen teile, ei ole kedagi suuremat naisest sündinute seas kui Johannes, aga väikseim Jumala riigis on suurem temast!”
\par 29 Ja kõik rahvas, kes seda kuulis, ja tölnerid tunnistasid Jumala õigeks, lastes endid ristida Johannese ristimisega.
\par 30 Variserid ja käsutundjad aga tegid tühjaks Jumala tahte eneste kohta ega lasknud teda endid ristida.
\par 31 „Kellega ma siis võrdlen selle sugupõlve inimesi ja kelle sarnased nad on?
\par 32 Nad on laste sarnased, kes turul istuvad ja üksteisele hüüavad nõnda: me oleme teile vilet ajanud, ja te ei ole tantsinud; me oleme teile nutulaulu laulnud, ja te ei ole nutnud!
\par 33 Sest Ristija Johannes on tulnud, ja ta ei söönud leiba ega joonud viina; ja te ütlete: temal on kuri vaim!
\par 34 Inimese Poeg on tulnud, ja ta sööb ja joob, ja te ütlete: vaata, see inimene on söödik ja viinajoodik, tölnerite ja patuste sõber!
\par 35 Ometi mõistetakse tarkus õigeks kõigi tema laste poolt!”

\section*{Jeesuse jalgade võidmine ja tähendamissõna kahest võlgnikust}

\par 36 Aga keegi variseridest palus teda enesega leiba võtma. Ja ta läks variseri kotta ja istus lauda.
\par 37 Ja vaata, selles linnas oli naine, kes oli patune. Kui see teada sai, et ta istub lauas variseri kojas, tõi ta alabasterriistatäie kallist lõhnasalvi
\par 38 ja astus nuttes tema taha ta jalgade juurde ja hakkas tema jalgu kastma silmaveega ja kuivatas neid oma juustega, ja suudles tema jalgu ning võidis neid salviga.
\par 39 Aga kui variser, kes teda oli kutsunud, seda nägi, mõtles ta iseeneses nõnda: kui see oleks prohvet, küll ta siis ära tunneks, kes ja missugune see naine on, kes teda puudutab, et ta on patune.
\par 40 Ja Jeesus vastas ning ütles temale: „Siimon, mul on sulle midagi öelda!” Aga tema kostis: „Õpetaja, räägi!”
\par 41 „Ühel rahalaenajal oli kaks võlglast: üks oli võlgu viissada teenarit, teine viiskümmend.
\par 42 Aga kui neil ei olnud maksta, kinkis ta mõlemale selle. Kumb neist nüüd teda rohkem armastab?”
\par 43 Siimon vastas ning ütles: „Ma arvan see, kellele ta rohkem kinkis.” Tema ütles temale: „Sa otsustasid õieti!”
\par 44 Ja pöördudes naise poole, ütles ta Siimonale: „Kas sa näed seda naist? Ma tulin sinu majasse, sa ei andnud vett mu jalgade tarvis, tema aga on pisaratega mu jalgu kastnud ja oma juustega kuivatanud.
\par 45 Sa ei andnud mulle suud; tema aga ei ole sellest ajast, kui ta sisse tuli, lakanud minu jalgadele suud andmast.
\par 46 Sa ei võidnud mu pead õliga, tema aga on mu jalgu võidnud salviga.
\par 47 Sellepärast, ütlen ma sulle, on tema rohked patud andeks antud, sest ta on palju armastanud. Aga kellele pisut andeks antakse, see armastab pisut!”
\par 48 Ja ta ütles naisele: „Sinu patud on sulle andeks antud!”
\par 49 Siis need, kes temaga lauas istusid, hakkasid iseeneses mõtlema: kes see on, kes ka patud andeks annab?
\par 50 Aga tema ütles naisele: „Sinu usk on sind aidanud; mine rahuga!”


\chapter{8}

\section*{Naised, kes Jeesust aitasid}

\par 1 Pärast seda aega rändas tema linnast linna ja külast külla ning jutlustas ja kuulutas evangeeliumi Jumala riigist. Ja need kaksteist olid temaga,
\par 2 ja mõned naised, keda tema oli terveks teinud kurjadest vaimudest ja tõbedest: Maarja, keda hüütakse Magdaleenaks, kellest seitse kurja vaimu oli välja läinud,
\par 3 ja Johanna, Heroodese varahoidja Kuusa naine, ja Susanna ja palju muid, kes teda toetasid oma varaga.

\section*{Tähendamissõna külvajast}

\par 4 Aga kui palju rahvast kogunes ja kõigist linnadest tuli tema juurde, rääkis ta tähendamissõnaga:
\par 5 „Külvaja läks välja oma seemet külvama. Ja kui ta külvas, kukkus muist tee äärde ja tallati ära, ja taeva linnud sõid selle ära.
\par 6 Ja muist kukkus kaljule. Ja kui see tärkas, kuivas see ära, sest sellel ei olnud niiskust.
\par 7 Ja muist kukkus ohakate keskele, ja ohakad kasvasid ühtlasi üles ja lämmatasid selle.
\par 8 Ja muist kukkus heale maale, ja kui see tärkas, kandis see sajakordse vilja.„ Seda öeldes ta hüüdis: ”Kellel kõrvad on kuulda, see kuulgu!”

\section*{Jeesus seletab tähendamissõna külvajast}

\par 9 Siis küsisid tema jüngrid temalt, mida see tähendamissõna tähendab.
\par 10 Tema ütles: „Teile on antud mõista Jumala riigi saladusi, aga muile tähendamissõnades, et nad nähes ei näeks ja kuuldes ei mõistaks.
\par 11 Ent tähendamissõna on see: seeme on Jumala sõna.
\par 12 Teeäärsed on need, kes kuulevad; pärast tuleb kurat ja võtab sõna ära nende südamest, et nad ei usuks ega saaks õndsaks.
\par 13 Kaljupealsed on need, kes kuuldes sõna rõõmuga vastu võtavad, kuid neil ei ole juurt; ajutiseks nad usuvad ja kiusatuse ajal nad loobuvad.
\par 14 Mis ohakate sekka kukkus, need on need, kes kuulevad ja lähevad ära ning lämbuvad muredest ja rikkusest ja elu lõbudest ega kanna valmis vilja.
\par 15 Aga mis on heas maas, on need, kes sõna kuulevad, kaunis ja heas südames peavad ning kannavad vilja kannatlikkuses.

\section*{Kogemus küünlast küünlajalal}

\par 16 Ükski, kes süütab küünla, ei kata seda astjaga kinni ega pane voodi alla, vaid asetab selle küünlajalale, et sissetulijad näeksid valgust.
\par 17 Sest midagi ei ole varjul, mis ei saaks avalikuks, ei ole ka midagi salajas, mis ei saaks teatavaks ega tuleks ilmsiks.
\par 18 Katsuge siis, kuidas te kuulete; sest kellel iganes on, sellele antakse, ja kellel iganes ei ole, sellelt võetakse seegi, mis ta arvab enesel olevat.”

\section*{Tõeline sugulus}

\par 19 Aga tema juurde tulid ta ema ja vennad ega võinud saada tema ligi rahva pärast.
\par 20 Ja temale teatati: „Sinu ema ja su vennad seisavad õues ja tahavad sind näha!”
\par 21 Aga ta kostis ning ütles neile: „Minu ema ja mu vennad on need, kes Jumala sõna kuulevad ja teevad selle järgi!”

\section*{Jeesus vaigistab tormi}

\par 22 Ja ühel päeval astusid tema ja ta jüngrid paati, ja ta ütles neile: „Lähme teisele poole järve!” Ja nad sõudsid minema,
\par 23 Ja kui nad paadiga purjetasid, uinus ta magama. Siis tõusis järvel tuulispea ja paat uhtus vett täis, ja nad olid hädas.
\par 24 Siis nad tulid tema juurde, äratasid ta üles ja ütlesid: „Õpetaja, õpetaja, me hukkume!” Aga tema tõusis üles, ähvardas tuult ning vee lainetust, ja need raugesid ning ilm muutus vaikseks.
\par 25 Aga tema ütles neile: „Kus on teie usk?” Ent nad kartsid ja imestasid ning ütlesid üksteisele: „Kes see õieti on, et tema käsib tuuli ja vett, ja need kuulevad ta sõna?”

\section*{Jeesus teeb terveks vaimuhaige gerasalase}

\par 26 Ja nad purjetasid gerasalaste maale, mis on Galilea vastas.
\par 27 Aga kui tema astus maale, tuli linnast talle vastu mees, kelle sees oli kurje vaime; ja ta ei olnud ammust ajast kuube selga pannud ega asunud majas, vaid surnuhaudades.
\par 28 Jeesust nähes ta hakkas karjuma, heitis tema ette maha ning ütles suure häälega: „Mis on mul tegemist sinuga, Jeesus, kõige kõrgema Jumala Poeg? Ma palun sind, ära piina mind!”
\par 29 Sest ta käskis rüvedat vaimu väljuda inimesest. See oli teda juba kaua aega vaevanud, ja teda oli ahelatega seotud ja peetud jalgraudus, aga ta oli rauad katki kiskunud ja kuri vaim oli teda ajanud kõrbetesse.
\par 30 Jeesus küsis temalt: „Mis su nimi on?” Tema vastas: „Leegion!” Sest palju kurje vaime oli läinud tema sisse.
\par 31 Ja need palusid teda, et ta neid ei käsiks minna ära põrguhauda.
\par 32 Aga seal oli suur seakari mäel söömas. Ja nad palusid teda, et ta neid lubaks minna nende sisse. Ja ta andis neile loa.
\par 33 Aga kui kurjad vaimud olid inimesest väljunud, läksid nad sigade sisse, ja seakari kukutas enese ülepeakaela kaldalt järve ning uppus.
\par 34 Kui karjatajad nägid, mis sündis, põgenesid nad ja andsid seda teada linnas ja maal.
\par 35 Siis tuldi välja vaatama, mis oli sündinud, ja mindi Jeesuse juurde ja leiti inimene, kellest kurjad vaimud olid välja läinud, Jeesuse jalge ees istumas, riietatult ja selge aruga. Ja inimesed lõid kartma.
\par 36 Siis jutustasid neile need, kes olid näinud, kuidas seestunu oli terveks saanud.
\par 37 Ja kogu rahvahulk gerasalaste maa ümbruskonnast palus teda, et ta läheks ära nende juurest. Sest suur hirm oli neid vallanud. Tema aga astus paati ja pöördus tagasi.
\par 38 Aga mees, kellest kurjad vaimud olid välja läinud, palus teda, et ta võiks jääda tema juurde. Kuid Jeesus laskis ta minema, öeldes:
\par 39 „Mine tagasi koju ja jutusta, mis suuri asju Jumal sulle on teinud!” Ja ta läks ära ja kuulutas mööda kogu linna, kui suuri asju Jeesus temale oli teinud.

\section*{Jairuse tütar ja veritõbine naine}

\par 40 Aga kui Jeesus tagasi tuli, võttis rahvas teda vastu, sest nemad kõik olid teda ootamas.
\par 41 Ja vaata, siis tuli mees, Jairus nimi, ja see oli kogudusekoja ülem; ja ta heitis Jeesuse jalge ette ning palus teda tulla enese kotta.
\par 42 Sest tal oli ainus tütar, kaksteist aastat vana, ja see oli suremas. Aga tema sinna minnes tungis rahvas temale peale.
\par 43 Ja keegi naine, kes oli kaksteist aastat põdenud verejooksu ega olnud võinud ühegi käest saada tervist,
\par 44 tuli tema selja taha ja puudutas ta kuue palistust, ja sedamaid lakkas ta verejooks.
\par 45 Ja Jeesus ütles: „Kes puudutas mind?” Aga kui kõik salgasid, ütles Peetrus: „Õpetaja, rahvahulgad tungivad sulle peale ja rõhuvad sind!”
\par 46 Kuid Jeesus ütles: „Keegi puudutas mind, sest ma tundsin väe enesest välja minevat!”
\par 47 Kui nüüd naine nägi, et ta ei olnud varjule jäänud, tuli ta värisedes ja heitis tema ette maha ja teatas temale kõige rahva kuuldes, mis asja pärast ta oli teda puudutanud ja kuidas ta kohe oli terveks saanud.
\par 48 Siis ta ütles naisele: „Tütar, sinu usk on sind aidanud, mine rahuga!”
\par 49 Kui tema alles rääkis, tuli kogudusekoja ülema perest keegi ning ütles: „Su tütar on surnud, ära tülita enam õpetajat!”
\par 50 Aga seda kuuldes vastas Jeesus temale: „Ära karda, usu vaid, siis ta saab abi!”
\par 51 Kui ta siis jõudis majasse, ei lasknud ta enesega sisse tulla kedagi teist kui ainult Peetruse ja Johannese ja Jakoobuse ning lapse isa ja ema.
\par 52 Ja kõik nutsid ja leinasid tütarlast. Aga tema ütles: „Ärge nutke, ta ei ole surnud, vaid ta magab!”
\par 53 Ja nad naersid teda, teades, et lapsuke on surnud.
\par 54 Ent tema haaras lapse käest kinni ja ütles hüüdes: „Tütarlaps, tõuse üles!”
\par 55 Ja tema vaim tuli tagasi, ja ta tõusis kohe üles. Ja ta käskis talle süüa anda.
\par 56 Ja tütarlapse vanemad kohkusid; aga tema keelas neid ühelegi rääkimast seda, mis oli sündinud.


\chapter{9}

\section*{Apostlite läkitamine}

\par 1 Aga ta kutsus kokku need kaksteist ja andis neile väe ja meelevalla kõigi kurjade vaimude üle ja tõbesid parandada,
\par 2 ja läkitas nad välja kuulutama Jumala riiki ja tegema haigeid terveks,
\par 3 ja ütles neile: „Ärge võtke midagi teele kaasa, ei keppi ega pauna, ei leiba ega raha; ärgu olgu ka kellelgi kaht kuube.
\par 4 Ja kuhu majasse te iganes sisse lähete, sinna jääge ja sealt minge teele.
\par 5 Ja kes iganes teid vastu ei võta, sellest linnast minge välja ja puistake tolm oma jalgadelt tunnistuseks nende vastu!”
\par 6 Nii nad läksid teele ja käisid mööda külasid, kuulutades evangeeliumi ja tehes haigeid terveks igal pool.

\section*{Heroodese hämmeldus}

\par 7 Aga nelivürst Heroodes sai kuulda kõik, mis sündis, ja oli kahevahel, sest mõned ütlesid: „Johannes on surnuist üles tõusnud!”
\par 8 Aga teised: „Eelija on ilmunud!” Aga mõned jälle: „Keegi vanadest prohvetitest on üles tõusnud!”
\par 9 Ent Heroodes ütles: „Johannese pea ma lasksin maha raiuda; kes on siis see, kellest ma kuulen niisuguseid asju?” Ja ta püüdis teda näha saada.

\section*{Viie tuhande mehe söötmine}

\par 10 Ja kui apostlid tagasi tulid, jutustasid nad temale, mis nad olid teinud. Ja ta võttis nad enese juurde ja läks ära üksindusse, linna poole, mida hüütakse Betsaidaks.
\par 11 Aga kui rahvahulgad seda teada said, läksid nad temale järele. Ja ta võttis nad vastu ja rääkis neile Jumala riigist; ja kellele oli vaja ravitsemist, need ta tegi terveks.
\par 12 Siis päev hakkas veerema. Ja need kaksteist tulid ta juurde ja ütlesid talle: „Lase rahvas minema, et nad läheksid ümberkaudseisse küladesse ja asulaisse puhkama ja otsima rooga; sest me oleme siin tühjas paigas!”
\par 13 Aga ta ütles neile: „Andke teie neile süüa!” Nemad vastasid: „Meil pole rohkem kui viis leiba ja kaks kala, või me peaksime minema ja ostma toidust kogu sellele rahvale!”
\par 14 Ent neid oli ligi viis tuhat meest. Siis ta ütles oma jüngritele: „Seadke nad maha istuma salkadesse viiekümne kaupa!”
\par 15 Ja nad tegid nõnda ja seadsid kõik maha istuma einele.
\par 16 Ent ta võttis need viis leiba ja kaks kala, vaatas üles taeva poole, õnnistas neid ning murdis nad ja andis oma jüngritele ettepanemiseks rahvale.
\par 17 Ja kõik sõid ja kõikide kõhud said täis; ja neist ülejäänud palukesi korjati kokku kaksteist korvitäit.

\section*{Peetrus tunnistab Jeesuse Kristuseks}

\par 18 Ja sündis, kui ta üksipäini oli palvetamas, et ta jüngrid olid ühes temaga. Ja ta küsis neilt: „Keda rahvas ütleb mind olevat?”
\par 19 Aga nemad vastasid ning ütlesid: „Ristija Johannese, aga mõned Eelija, aga teised, et keegi vanadest prohvetitest on üles tõusnud!”
\par 20 Siis ta ütles neile: „Ent teie, keda teie ütlete mind olevat?” Peetrus vastas ning ütles: „Jumala Kristuse!”
\section*{Jeesus kuulutab ette oma surma ja ülestõusmist}

\par 21 Ja ta ähvardas neid kangesti ning keelas neid sellest ühelegi rääkimast,
\par 22 öeldes: „Inimese Poeg peab palju kannatama ja kõlbmatuna hüljatama vanemate ja ülempreestrite ja kirjatundjate poolt ja tapetama ja kolmandal päeval jälle üles äratatama!”

\section*{Nõue iseenese salgamiseks}

\par 23 Aga ta ütles kõikidele: „Kui keegi tahab minu järele tulla, see salaku end ja võtku oma rist iga päev enese peale ja järgigu mind.
\par 24 Sest kes iganes oma hinge tahab päästa, see kaotab selle, aga kes iganes oma hinge kaotab minu pärast, see päästab selle!
\par 25 Sest mis kasu on inimesel sellest, kui ta kogu maailma kasuks saaks, aga kaotaks iseenese või läheks hukka?
\par 26 Sest kes minu ja mu sõnade pärast häbeneb, selle pärast häbeneb Inimese Poeg, kui ta tuleb omas ja Isa ja pühade inglite auhiilguses.
\par 27 Aga ma ütlen teile tõesti: on mõningad siin seisjaist, kes ei maitse surma, kuni nad näevad Jumala riiki!”

\section*{Jeesuse muutumine}

\par 28 Ent arvata kaheksa päeva pärast neid kõnesid sündis, et ta võttis enesega Peetruse ja Johannese ja Jakoobuse ja läks mäele palvetama.
\par 29 Ja kui ta palvetas, muutus tema näoilme teistsuguseks ja tema riietus läks valgeks ja hiilgavaks.
\par 30 Ja vaata, kaks meest kõnelesid temaga; need olid Mooses ja Eelija.
\par 31 Nemad olid ilmunud auhiilguses ja kõnelesid tema eluotsast, mis tal Jeruusalemmas tuli täide saata.
\par 32 Aga Peetrus ja ta kaaslased olid suikunud raskesse unne; ent üles ärgates nad nägid tema auhiilgust ja neid kaht meest seisvat tema juures.
\par 33 Kui siis need temast olid lahkumas, ütles Peetrus Jeesusele: „Õpetaja, siin on meil hea olla! Teeme kolm telki, ühe sinule ja ühe Moosesele ja ühe Eelijale!” Kuid ta ei teadnud, mida ta rääkis.
\par 34 Aga kui ta seda rääkis, tekkis pilv ja varjas nad; ja nemad kartsid pilve sisse jõudes.
\par 35 Ja hääl kostis pilvest ning ütles: „See on minu äravalitud Poeg, teda kuulake!”
\par 36 Ja kui see hääl oli kostnud, leiti Jeesus üksi olevat. Ja nad olid vait ega kuulutanud neil päevil kellelegi midagi sellest, mida nad olid näinud.

\section*{Jeesus teeb terveks kurjast vaimust vaevatud poisi}

\par 37 Järgmisel päeval, kui nad mäelt alla läksid, tuli palju rahvast temale vastu.
\par 38 Ja vaata, üks mees rahvahulga seast kisendas ning ütles: „Õpetaja, ma palun sind vaadata mu poega, sest ta on mu ainuke!
\par 39 Ja vaata, vaim haarab teda, ja ta karjatab äkki, ja vaim raputab teda, nõnda et ta ajab suust vahtu; ja see lahkub vaevalt teda piinamast.
\par 40 Ja mina palusin sinu jüngreid, et nad ajaksid ta välja, ja nad ei suutnud!”
\par 41 Aga Jeesus vastas ning ütles: „Oh sa uskmatu ja pöörane tõug! Kui kaua ma pean olema teie juures ja kannatama teiega? Too siia oma poeg!”
\par 42 Kui ta alles oli tulemas, kiskus teda kuri vaim ja raputas teda. Aga Jeesus sõitles rüvedat vaimu ja tegi poisi terveks ning andis ta tagasi tema isale.

\section*{Jeesus kuulutab ette teist korda oma surma}

\par 43 Ent kõik hämmastusid Jumala suurest väest. Aga kui kõik panid imeks seda, mida ta tegi, ütles ta oma jüngritele:
\par 44 „Pange teie oma kõrvu need sõnad. Sest Inimese Poeg antakse inimeste kätte!”
\par 45 Ent nemad ei mõistnud seda sõna, ja see oli pandud varjule nende eest, et nad sellest aru ei saaks. Ja nad kartsid küsida temalt selle sõna kohta.

\section*{Õpetus alandlikkuseks}

\par 46 Aga nende seas tõusis vaidlus, kes küll nende seast peaks olema suurem.
\par 47 Jeesus, teades nende südame mõtlemisi, võttis lapsukese ja seadis selle enese kõrvale
\par 48 ning ütles neile: „Kes iganes selle lapsukese võtab vastu minu nimel, võtab mind vastu; ja kes iganes võtab vastu mind, võtab vastu selle, kes mind on läkitanud. Sest kes teist kõigist on vähem, see on suur!”

\section*{Sallivuse käsk}

\par 49 Siis Johannes hakkas rääkima ning ütles: „Õpetaja, me nägime üht meest sinu nimel kurje vaime välja ajavat; ja me keelasime teda, sellepärast et ta ühes meiega sind ei järgi!”
\par 50 Jeesus ütles talle: „Ärge keelake, sest kes ei ole meie vastu, see on meie poolt!”

\section*{Samaarlased ei võta Jeesust vastu}

\par 51 Aga sündis, kui tema ülesvõtmisepäevad olid täis saamas, et ta suundus kindlasti minema Jeruusalemma poole,
\par 52 ja ta läkitas käskjalad enese eele. Ja need läksid ja tulid ühte samaarlaste külla, valmistama temale öömaja.
\par 53 Ent need ei võtnud teda vastu, sest ta oli minemas Jeruusalemma poole.
\par 54 Kui ta jüngrid Jakoobus ja Johannes seda nägid, ütlesid nad: „Issand, kas tahad, et me käsime tulla tule taevast maha ja nad hävitada?”
\par 55 Aga ta pöördus ümber ja sõitles neid.
\par 56 Ja nad läksid teise alevisse.

\section*{Jüngriks saada tahtjad valiku ees}

\par 57 Ja kui nad olid sinna minemas, ütles teel keegi temale: „Issand, ma tahan sind järgida, kuhu sa iganes lähed!”
\par 58 Jeesus ütles talle: „Rebastel on augud ja taeva lindudel on pesad, aga Inimese Pojal ei ole aset, kuhu ta oma pea paneks!”
\par 59 Kellelegi teisele ta ütles: „Järgi mind!” Ent see ütles: „Issand, luba mind enne minna ja oma isa maha matta!”
\par 60 Aga ta ütles temale: „Lase surnuid oma surnud matta, sina aga mine ja kuuluta Jumala riiki!”
\par 61 Ja veel üks teine ütles: „Issand, ma tahan sind järgida! Aga luba mind enne jätta jumalaga oma kodakondsed!”
\par 62 Aga Jeesus ütles temale: „Ükski, kes paneb oma käe adra külge ja vaatab tagasi, ei kõlba Jumala riigile!”


\chapter{10}

\section*{Seitsmekümne jüngri läkitamine}

\par 1 Pärast seda määras Issand teised seitsekümmend ja läkitas nad kahekaupa enese eele igasse linna ja paika, kuhu ta ise mõtles minna.
\par 2 Ja ta ütles neile: „Lõikust on palju, aga töötegijaid vähe; seepärast paluge lõikuse Issandat, et ta läkitaks välja töötegijaid oma lõikusele!
\par 3 Minge! Vaata, ma läkitan teid kui tallesid huntide keskele!
\par 4 Ärge kandke kukrut ega pauna ega jalatseid ja teel ärge kedagi teretage.
\par 5 Kuhu majasse te aga iganes sisse lähete, seal öelge esiti: „Rahu olgu sellele kojale!”
\par 6 Ja kui iganes seal on rahulaps, siis hingab teie rahu tema peal; aga kui mitte, siis tuleb rahu tagasi teie peale.
\par 7 Aga sinna majasse jääge, sööge ja jooge, mis neil on, sest töötegija on oma palga väärt, ärge käige ühest majast teise.
\par 8 Ja kuhu linna te iganes lähete ja kus teid vastu võetakse, seal sööge, mida teile ette pannakse;
\par 9 ja tehke selle paiga haiged terveks ja öelge neile: Jumala riik on teie lähedal!
\par 10 Aga kuhu linna te iganes lähete ja kus teid vastu ei võeta, seal minge välja ta tänavaile ning öelge:
\par 11 tolmugi, mis teie linnast on hakanud meie jalgade külge, me pühime ära teile; ometi teadke, et Jumala riik on lähedal!
\par 12 Ma ütlen teile, et Soodomal on sel päeval hõlpsam põli kui tol linnal!

\section*{Jeesus sõitleb uskmatuid linnu}

\par 13 Häda sulle, Korasin, häda sulle, Betsaida! Sest kui Tüüroses ja Siidonis oleksid sündinud need vägevad teod, mis teie juures on sündinud, küll nad oleksid ammu kotiriides ja tuhas istudes parandanud meelt!
\par 14 Ometi peab Tüürosel ja Siidonil olema kohtus hõlpsam põli kui teil!
\par 15 Ja sina, Kapernaum, eks sa olnud ülendatud taevani? Sind tõugatakse põrguni alla!
\par 16 Kes teid kuuleb, see kuuleb mind, ja kes teid hülgab, see hülgab mind; aga kes hülgab mind, see hülgab selle, kes mind on läkitanud!”

\section*{Jüngrite tagasitulek}

\par 17 Siis need seitsekümmend tulid tagasi rõõmuga ja ütlesid: „Issand, ka kurjad vaimud alistuvad meile sinu nime tõttu!”
\par 18 Aga ta ütles neile: „Ma nägin saatana nagu välgu taevast maha langevat!
\par 19 Vaata, ma olen andnud teile meelevalla astuda madude ja skorpionide peale ja vaenlase kõige väe peale; ja miski ei tee teile kahju.
\par 20 Ometi ärge rõõmutsege, et vaimud teile alistuvad, vaid rõõmutsege palju enam sellest, et teie nimed on taevasse kirja pandud!”

\section*{Jeesus tänab Jumalat}

\par 21 Selsamal tunnil ta rõõmustus Pühas Vaimus ja ütles: „Mina tänan sind, Isa, taeva ja maa Issand, et sa oled tarkade ja mõistlike eest selle pannud varjule ja oled selle ilmutanud väetitele! Jah, Isa, see on nõnda olnud sulle meelepärast!
\par 22 Kõik on mu Isa mulle andnud! Ja ükski ei tunne, kes on Poeg, kui vaid Isa, ja kes on Isa, kui vaid Poeg ja kellele iganes Poeg tahab ilmutada!”
\par 23 Ja ta pöördus jüngrite poole eriti ning ütles: „Õndsad on silmad, kes näevad, mida teie näete!
\par 24 Sest ma ütlen teile, et palju prohveteid ja kuningaid on tahtnud näha, mida teie näete, ja ei ole näinud, ja kuulda, mida teie kuulete, ja ei ole kuulnud!”

\section*{Igavese elu pärimisest}

\par 25 Ja vaata, keegi käsutundja astus esile, kiusas teda ja ütles: „Õpetaja, mis ma pean tegema, et ma päriksin igavese elu?”
\par 26 Aga tema ütles talle: „Mis on käsuõpetuses kirjutatud? Kuidas sa loed?”
\par 27 Tema vastas ning ütles: „Armasta Issandat, oma Jumalat, kõigest oma südamest ja kõigest oma hingest ja kõigest oma jõust ja kõigest oma meelest, ja oma ligimest nagu iseennast.”
\par 28 Aga ta ütles temale: „Sa oled õieti vastanud; tee seda, ja sa pead elama!”

\section*{Halastaja samaarlane}

\par 29 Tema aga tahtis iseennast õigeks teha ja ütles Jeesusele: „Kes siis on mu ligimene?”
\par 30 Aga vastates Jeesus ütles: „Üks inimene läks Jeruusalemmast alla Jeerikosse ja sattus röövlite kätte. Kui need ta riided olid riisunud ja temale hoope andnud, läksid nad ära, jättes ta poolsurnuna maha.
\par 31 Juhtumisi tuli keegi preester sedasama teed mööda alla ja nägi teda ning läks mööda.
\par 32 Nõndasamuti ka leviit. Kui ta tuli sinna paika ja nägi teda, läks ta mööda.
\par 33 Aga üks samaarlane käis seda teed ja tuli tema juurde; ja kui ta teda nägi, läks ta meel haledaks.
\par 34 Ja ta astus ligi, sidus ta haavad ning valas peale õli ja viina, tõstis ta oma looma selga ning viis ta öömajale ja kandis hoolt tema eest.
\par 35 Ja teisel hommikul ära minnes võttis ta välja kaks teenarit, andis need peremehele ja ütles: kanna hoolt tema eest, ja mida sa veel peaksid kulutama, seda ma maksan sulle tagasi tulles.
\par 36 Kes neist kolmest oli sinu arvates ligimene sellele, kes oli sattunud röövlite kätte?”
\par 37 Aga tema ütles: „See, kes tema peale halastas!” Siis ütles Jeesus temale: „Mine ja tee sina nõndasamuti!”

\section*{Marta ja Maarja}

\par 38 Aga kui nad rändasid, läks ta ühte alevisse. Ent üks naine, Marta nimi, võttis tema vastu oma majasse.
\par 39 Ja temal oli õde, keda hüüti Maarjaks; seegi istus Issanda jalge ees ja kuulas tema kõnet.
\par 40 Marta aga tegi tegemist mitmesuguse talitusega. Ta tuli sinna ja ütles: „Issand, kas sa ei hooli sellest, et mu õde laseb mind üksi talitada? Ütle ometi temale, et ta mind aitaks!”
\par 41 Aga Issand vastas ning ütles talle: „Marta, Marta, sa muretsed ja teed enesele tüli paljude asjadega;
\par 42 kuid tarvis on vähe, õigupoolest üht! Maarja on selle hea osa valinud ja seda ei võeta temalt ära!”


\chapter{11}

\section*{Meie Isa palve}

\par 1 Ja kui ta ühes paigas oli palvetamas ja oli lõpetanud, ütles üks ta jüngreist temale: „Issand, õpeta meid palvetama nõnda nagu Johannes õpetas oma jüngreid!”
\par 2 Aga ta ütles neile: „Kui te palvetate, siis öelge: Isa! Pühitsetud olgu sinu nimi; sinu riik tulgu;
\par 3 meie igapäevane leib anna meile iga päev;
\par 4 ja anna meile andeks meie patud, sest meiegi anname andeks igaühele, kellel on meiega võlgu; ja ära saada meid kiusatusse!”

\section*{Tungiva palve mõjust}

\par 5 Ja ta ütles neile: „Kellel teie seast on sõber ja ta läheks südaöösel ta juurde ja ütleks temale: sõber, laena mulle kolm leiba,
\par 6 sest mu sõber on teelt tulnud mu juurde ja mul ei ole, mida talle ette paneksin!
\par 7 Kas tema peaks toast vastama ning ütlema: ära tee mulle tüli, uks on juba lukus ja mu lapsukesed on minuga voodis; ma ei või üles tõusta sulle andma!?
\par 8 Ma ütlen teile, kui ta üles ei tõuseks ja talle ei annaks, sellepärast et ta on tema sõber, siis tõuseks ta ometi üles tema pealekäimise pärast ning annaks talle, niipalju kui talle vaja.
\par 9 Ka mina ütlen teile: paluge, siis antakse teile; otsige, siis te leiate; koputage, siis avatakse teile.
\par 10 Sest igaüks, kes palub, see saab, ja kes otsib, see leiab, ja kes koputab, sellele avatakse.
\par 11 Ent missugune isa on teie seas, kellelt poeg palub kala, et ta annaks sellele kala asemel mao?
\par 12 Või kui ta palub muna, ta annaks talle skorpioni?
\par 13 Kui nüüd teie, kes olete kurjad, märkate anda häid ande oma lastele, kui palju enam Isa taevast annab Püha Vaimu neile, kes teda paluvad.”

\section*{Jeesus ei vaja saatana abi}

\par 14 Ja kord ta ajas kurja vaimu välja; ja see oli keeletu. Aga kui kuri vaim oli väljunud, hakkas keeletu rääkima ja rahvahulgad panid seda imeks.
\par 15 Aga mõningad nende seast ütlesid: „Ta ajab kurje vaime välja Peeltsebuli, kurjade vaimude ülema abil!”
\par 16 Aga teised kiusasid teda ning nõudsid temalt tunnustähte taevast.
\par 17 Ent tema mõistis ära nende mõtted ja ütles neile: „Iga kuningriik, mis on isekeskis riius, hävib ja koda langeb koja peale.
\par 18 Kui saatangi on riius isekeskis, kuidas võib tema kuningriik püsida? Te ju ütlete, et mina ajavat kurje vaime välja Peeltsebuli abil!
\par 19 Aga kui mina kurje vaime välja ajan Peeltsebuli abil, kelle abil ajavad teie pojad neid välja? Sellepärast saavad nemad teie kohtumõistjaiks.
\par 20 Ent kui mina kurje vaime välja ajan Jumala sõrmega, siis on Jumala riik jõudnud teie juurde.
\par 21 Kui vägev relvastatud mees valvab oma elamut, siis on ta vara rahus.
\par 22 Aga kui temast vägevam peale tuleb ja võidab tema, siis ta võtab ära kõik ta relvad, mille peale ta lootis, ja jagab temalt saadud saagi välja.
\par 23 Kes minu poolt ei ole, on minu vastu; ja kes minuga ei kogu, see hajutab.

\section*{Puuduliku puhastuse hädaoht}

\par 24 Kui rüve vaim on inimesest välja läinud, käib ta mööda põudseid paiku ja otsib hingamist. Ja kui ta ei leia, ütleb ta: ma pöördun tagasi oma kotta, kust ma väljusin!
\par 25 Ja kui ta tuleb, leiab ta selle pühitud ning ehitud olevat.
\par 26 Siis ta läheb ja võtab kaasa teist seitse vaimu, kes on kurjemad temast, ja nad tulevad sisse ning elavad seal. Ja selle inimese viimne lugu läheb pahemaks kui eelmine!”
\par 27 Kui ta seda rääkis, sündis, et üks naine rahva seast tõstis oma häält ning ütles temale: „Õnnis on ihu, mis sind on kandnud, ja rinnad, mida sina oled imenud!”
\par 28 Aga tema ütles: „Jah, õndsad on need, kes Jumala sõna kuulevad ja seda tallele panevad!”

\section*{Hoiatus tunnustähe otsijaile}

\par 29 Kui rahvahulki murdu kokku tuli, hakkas ta kõnelema: „See sugu on paha sugu; ta otsib tunnustähte, ja talle ei anta muud tähte kui vaid prohvet Joona täht.
\par 30 Sest nõnda nagu Joona sai täheks Niinive elanikele, nõnda peab ka Inimese Poeg olema sellele soole.
\par 31 Lõunamaa kuninganna tõuseb üles kohtupäeval selle soo meeste kõrval ja mõistab nad hukka; sest ta tuli ilmamaa otsast kuulama Saalomoni tarkust. Ja vaata, siin on rohkem kui Saalomon!
\par 32 Niinive mehed tõusevad üles kohtupäeval selle soo kõrval ja mõistavad selle hukka; sest nad parandasid meelt Joona jutluse tõttu. Ja vaata, siin on rohkem kui Joona!

\section*{Kogemus küünlast küünlajalal}

\par 33 Ükski ei süüta küünalt ega pane seda varjule ega vaka alla, vaid küünlajalale, et sissetulijad valgust näeksid.
\par 34 Ihu küünal on su silm. Kui su silm on korras, siis on ka kõik su ihu valguses; aga kui ta on rikkis, siis on ka su ihu pime.
\par 35 Katsu siis, et valgus, mis on sinu sees, ei oleks pimedus!
\par 36 Kui nüüd kogu su ihu on valguses, nõnda et tal sugugi ei ole pimedust, siis on ta täiesti valgustatud, nõnda nagu küünal valgustaks sind oma välkuva leegiga!”

\section*{Jeesus hurjutab varisere ja käsutundjaid}

\par 37 Kui tema alles nõnda rääkis, palus teda keegi variser enese juurde lõunat sööma. Ja ta läks sisse ja istus lauda.
\par 38 Aga variser pani imeks, kui ta nägi, et ta enne söömist ennast ei pesnud.
\par 39 Siis Issand ütles temale: „Teie, variserid, teete küll väljastpoolt puhtaks karika ja vaagna, aga seestpoolt te olete täis rüüstet ja kurjust!
\par 40 Te mõistmatud! Eks see, kes tegi välispidise, ole teinud ka seespidise?
\par 41 Ent andke armuanniks, mis teie sees on, ja vaata, siis on teile kõik puhas!
\par 42 Aga häda teile, variseridele, et te maksate kümnist mündist ja ruudist ja kõigest aiarohust ja lähete mööda õiglusest ja Jumala armastusest! Seda peab tegema ja teist ei tohi tegemata jätta!
\par 43 Häda teile, variseridele, et te armastate esimest istet kogudusekodades ja teretusi turgudel!
\par 44 Häda teile, et te olete nagu äravajunud hauad, ja inimesed käivad nende peal, ilma et nad teaksid!”
\par 45 Aga üks käsutundjaist kostis ja ütles temale: „Õpetaja, seda öeldes sa teotad ka meid!”
\par 46 Ent tema ütles: „Häda ka teile, käsutundjaile, sest te panete inimestele peale rängad koormad ja ise ei puutu ühe oma sõrmegagi neisse koormaisse!
\par 47 Häda teile, et te ehitate prohvetitele hauamärke, aga teie isad on nad tapnud!
\par 48 Nii te olete ise tunnistajad, et teil on hea meel oma isade tegudest; sest nemad tapsid nad, aga teie ehitate nende hauamärke!
\par 49 Sellepärast ütleb ka Jumala tarkus: ma läkitan nende juurde prohveteid ja apostleid, ja nad tapavad mõned neist ja kiusavad neid taga,
\par 50 et sellelt sugupõlvelt nõutaks kõigi prohvetite verd, mis on ära valatud alates maailma asutamisest,
\par 51 Aabeli verest Sakarja verest saadik, kes sai hukka altari ja Jumala koja vahel. Tõesti, ma ütlen teile, seda nõutakse sellelt sugupõlvelt!
\par 52 Häda teile, käsutundjaile, et te olete ära võtnud tunnetuse võtme! Ise te ei ole sisse läinud ja olete keelanud neid, kes sisse läheksid!”
\par 53 Ja kui ta sealt väljus, hakkasid kirjatundjad ja variserid teda kangesti kimbutama ja usutlema mitmeis asjus,
\par 54 et kavaldamisega midagi tema suust kätte saada.


\chapter{12}

\section*{Jeesus hoiatab silmakirjalikkuse eest}

\par 1 Kui sel ajal rahvast oli tuhandeti kokku tulnud, nõnda et nad üksteist tallasid, hakkas ta kõigepealt kõnelema oma jüngritele: „Hoiduge variseride haputaigna, see on silmakirjalikkuse eest!
\par 2 Ei ole midagi varjatud, mis ei tuleks ilmsiks, ega ole midagi salajas, mida ei saadaks teada.
\par 3 Sellepärast peab kõike, mida te ütlete pimedas, kuuldama valges; ja mida te üksteise kõrva räägite kambrites, seda peab kuulutatama katustelt!

\section*{Keda karta, keda mitte}

\par 4 Aga ma ütlen teile, oma sõpradele: ärge kartke neid, kes tapavad ihu ja pärast ei saa rohkem midagi teha.
\par 5 Ent ma näitan teile, keda teil tuleb karta: kartke teda, kellel on meelevald pärast tapmist heita põrgusse. Tõesti, ma ütlen teile, teda kartke!
\par 6 Eks viis varblast müüda kahe veeringu eest? Ja ükski neist ei ole unustatud Jumala ees!
\par 7 Aga teie juuksekarvadki on kõik ära loetud! Ärge kartke, teie olete hinnalisemad kui palju varblasi!
\par 8 Ent ma ütlen teile: kes iganes mind tunnistab inimeste ees, seda tunnistab ka Inimese Poeg Jumala inglite ees!
\par 9 Aga kes mind salgab inimeste ees, see salatakse Jumala inglite ees!
\par 10 Ja igaüks, kes Inimese Poja vastu räägib sõna, sellele antakse andeks; aga sellele, kes pilkab Püha Vaimu, ei anta mitte andeks!
\par 11 Aga kui teid viiakse kogudusekodadesse ja ülemate ja valitsejate ette, ärge olge mures, kuidas või mida te enese eest kostate või mida ütlete;
\par 12 sest Püha Vaim õpetab teile selsamal tunnil, mida tuleb öelda!”

\section*{Ahnusest; tähendamissõna rumalast rikkast mehest}

\par 13 Siis keegi rahva seast ütles talle: „Õpetaja, ütle mu vennale, et ta minuga jagaks pärandi!”
\par 14 Aga tema ütles talle: „Inimene, kes on mind seadnud teile kohtumõistjaks või jagajaks?”
\par 15 Ja ta ütles neile: „Vaadake ette ja hoiduge ahnuse eest; sest külluseski ei olene kellegi elu sellest, mis tal on!”
\par 16 Ja ta rääkis neile ka tähendamissõna, öeldes: „Ühe rikka mehe põllumaa oli hästi vilja kandnud.
\par 17 Ja ta mõtles iseeneses nõnda: mis ma teen? Sest mul ei ole, kuhu ma koguksin oma vilja.
\par 18 Ja ta ütles: seda ma teen: ma kisun maha oma aidad ja ehitan suuremad, ja sinna ma kogun kõik oma vilja ja oma vara,
\par 19 ja ütlen oma hingele: hing, sul on palju vara tagavaraks mitmeks aastaks; ole nüüd rahul, söö, joo ja ole rõõmus!
\par 20 Aga Jumal ütles temale: sina meeletu! Selsamal ööl nõutakse sult sinu hing; kellele saab siis, mis sa oled soetanud?
\par 21 Nõnda on selle lugu, kes enesele kogub tagavara ja ei ole rikas Jumalas!”

\section*{Mure ja lootus Jumala peale}

\par 22 Siis ta ütles oma jüngritele: „Seepärast ma ütlen teile: ärge olge mures oma hinge pärast, mida süüa, ega oma ihu pärast, millega riietuda!
\par 23 Sest hing on enam kui toidus ja ihu enam kui riided!
\par 24 Pange tähele kaarnaid! Nad ei külva ega lõika; neil ei ole varakambrit ega aita, ja Jumal toidab neid. Kui palju hinnalisemad olete teie kui on linnud!
\par 25 Aga kes teie seast võib muretsemisega oma pikkusele ühegi küünra jätkata?
\par 26 Kui te nüüd kõige vähematki ei suuda, miks te siis olete mures muu pärast?
\par 27 Pange tähele lilli, kuidas nad ei ketra ega koo; aga ma ütlen teile: Saalomongi kõiges oma hiilguses pole olnud nõnda ehitud kui üks nendest!
\par 28 Kui nüüd Jumal rohtu, mis täna kasvab ja homme visatakse ahju, nõnda ehib, kui palju enam teid, te nõdrausulised!
\par 29 Ka teie ärge küsige, mida süüa või mida juua, ja ärge olge kärsitud!
\par 30 Sest kõike seda nõuavad taga maailma rahvad; teie Isa teab ju, et te seda vajate.
\par 31 Otsige vaid tema riiki, ja see kõik antakse teile pealegi!
\par 32 Ära karda, sa pisuke karjuke, sest teie Isa on heaks arvanud anda teile kuningriigi!
\par 33 Müüge ära, mis teil on, ja andke armuande! Valmistage enestele kukrud, mis ei vanane, ja varandus taevas, mis ei vähene, kuhu varas ligi ei pääse ja mida koi ei riku!
\par 34 Sest kus teie varandus on, seal on ka teie süda!

\section*{Vajadus valvsuseks ning tähendamissõna heast ja halvast sulasest}

\par 35 Teie niuded olgu vöötatud ja küünlad põlegu!
\par 36 Ja teie olge inimeste sarnased, kes ootavad oma isandat, millal ta lahkub pulmast, et kui ta tuleb ja koputab, nad kohe temale avaksid.
\par 37 Õndsad on need sulased, keda isand tulles leiab valvamast! Tõesti, ma ütlen teile, tema vöötab enese ja paneb nad lauda istuma ja tuleb ja teenib neid!
\par 38 Ja tulgu ta teisel või kolmandal öövahikorral, kui ta nad leiab nõnda - õndsad on need sulased!
\par 39 Aga seda teadke: kui peremees teaks, mil tunnil varas tuleb, ta ei laseks oma kotta sisse murda!
\par 40 Olge siis ka teie valmis, sest Inimese Poeg tuleb tunnil, mil te ei arvagi!”
\par 41 Aga Peetrus küsis: „Issand, kas ütled selle tähendamissõna meile või kõikidele?”
\par 42 Ja Issand ütles: „Kes on siis ustav ja mõistlik majapidaja, keda isand seab oma pere üle neile õigel ajal andma määratud moona?
\par 43 Õnnis see sulane, keda isand tulles leiab nõnda tegemast!
\par 44 Tõesti, ma ütlen teile, ta paneb tema üle kõige oma vara!
\par 45 Aga kui see sulane ütleb oma südames: mu isand viibib tulles! ja ta hakkab peksma poisse ja tüdrukuid, ja sööma ja jooma ja purjutama,
\par 46 siis selle sulase isand tuleb päeval, mil ta ei oota teda, ja tunnil, mida ta ei tea, ja raiub ta pooleks ja annab temale osa ühes uskmatutega.
\par 47 Aga sulane, kes teadis oma isanda tahtmist, aga ei valmistanud ega teinud tema tahtmise järgi, saab palju hoope.
\par 48 Aga kes ei teadnud, kuid tegi, mis on hoopide väärt, saab pisut hoope. Sest kellele on palju antud, sellelt nõutakse palju; ja kelle hoolde on palju jäetud, sellelt päritakse veel rohkem!

\section*{Ajamärgid}

\par 49 Tuld ma tulin viskama maa peale; ja mis ma muud tahaksin, kui et see juba oleks süüdatud!
\par 50 Aga mind peab ristitama ristimisega, ja kuidas on mul kitsas käes, kuni see on lõpetatud!
\par 51 Kas mõtlete, et ma olen tulnud andma rahu maa peale? Mitte sugugi, ütlen ma teile, vaid lahkmeelt!
\par 52 Sest nüüdsest alates peavad viis ühes majas olema lahkmeeles, kolm kahe vastu ja kaks kolme vastu,
\par 53 isa poja vastu ja poeg isa vastu, ema tütre vastu ja tütar ema vastu, ämm oma minia vastu ja minia ämma vastu!”
\par 54 Aga ta ütles ka rahvale: „Kui te näete pilve tõusvat lääne poolt, siis te kohe ütlete: raske sadu tuleb! Ja see sünnib nõnda.
\par 55 Ja kui näete lõunatuult puhuvat, siis te ütlete: kuum tuleb! Ja see sünnib nõnda.
\par 56 Te silmakirjatsejad! Maa ja taeva nägu te oskate mõista, aga kuidas te ei mõista seda aega?
\par 57 Ja miks te ka iseenestest ei otsusta, mis on õige?
\par 58 Sellepärast kui sa oma vastasega lähed ülema ette, siis püüa temaga asi lahendada teel, et ta sind ei veaks kohtuniku ette ja kohtunik sind ei annaks kohtusulase kätte ja kohtusulane sind ei heidaks vanglasse.
\par 59 Ma ütlen sulle, et sa ei pääse sealt, enne kui maksad viimse leptoni!”


\chapter{13}

\section*{Manitsus patukahetsemiseks}

\par 1 Selsamal ajal olid saabunud mõned, kes jutustasid temale galilealastest, kelle vere Pilaatus oli seganud nende ohvritega.
\par 2 Ja Jeesus hakkas kõnelema ning ütles neile: „Kas arvate, et need galilealased olid suuremad patused kui kõik muu Galilea rahvas, et nad seda said kannatada?
\par 3 Ma ütlen teile, et mitte sugugi, vaid kui te ei paranda meelt, lähete kõik samuti hukka!
\par 4 Või arvate, et need kaheksateist, kelle peale langes Siiloa torn ja tappis nad, olid suuremad süüalused kui kõik inimesed, kes elavad Jeruusalemmas?
\par 5 Ma ütlen teile, et mitte sugugi, vaid kui te ei paranda meelt, siis lähete kõik samuti hukka!”

\section*{Tähendamissõna viljatust viigipuust}

\par 6 Ja ta ütles selle tähendamissõna: „Ühel inimesel oli viigipuu; see oli istutatud tema viinamäkke. Ja ta tuli ning otsis sellelt vilja, aga ei leidnud.
\par 7 Siis ta ütles viinamäe aednikule: vaata, kolm aastat ma käin otsimas sellelt viigipuult vilja, aga ei leia. Raiu ta maha! Miks ta maad raiskab?
\par 8 Aga see vastas ning ütles talle: isand, jäta ta veel sellekski aastaks, kuni ma ta ümber kaevan ja talle panen sõnnikut!
\par 9 Ehk ta hakkab edaspidi vilja kandma; aga kui mitte, siis raiu ta maha!”

\section*{Haige tervekstegemine hingamispäeval}

\par 10 Ja ta oli õpetamas ühes kogudusekojas hingamispäeval.
\par 11 Ja vaata, seal oli naine, kellel oli olnud haiguse vaim kaheksateist aastat; naine oli küürus ega saanud ennast sugugi sirgeks ajada.
\par 12 Teda nähes kutsus Jeesus ta enese juurde ja ütles talle: „Naine, sa oled lahti oma haigusest!”
\par 13 Ja ta pani oma käed tema peale; ja kohe ajas naine enese sirgeks ja andis Jumalale au.
\par 14 Aga kogudusekoja ülem, kes pahandus sellest, et Jeesus tegi terveks hingamispäeval, hakkas rääkima ning ütles rahvale: „Kuus päeva on, mil tööd tehakse; tulge ometi neil päevil ja laske endid ravida, aga mitte hingamispäeval!”
\par 15 Aga Issand vastas temale ning ütles: „Te silmakirjatsejad! Eks igaüks teie seast päästa oma härja või eesli sõime küljest hingamispäeval ja vii teda jooma?
\par 16 Kas siis teda, kes on Aabrahami tütar, keda saatan, vaata, kaheksateist aastat on sidunud, ei pidanud päästetama sellest köidikust hingamispäeval?”
\par 17 Ja kui ta seda ütles, häbenesid kõik ta vastased, ja kogu rahvas rõõmustus kõigist aulistest tegudest, mida tema tegi.

\section*{Tähendamissõnad sinepiivakesest ja haputaignast}

\par 18 Siis ta ütles: „Mille sarnane on Jumala riik ja millega ma võrdlen seda?
\par 19 See on sinepiivakese sarnane, mille inimene võttis ja külvas oma aeda; ja see kasvas ja sai puuks, ja taeva linnud tegid pesad tema okstele!”
\par 20 Ja ta ütles veel: „Millega ma võrdleksin Jumala riiki?
\par 21 See on haputaigna sarnane, mille üks naine võttis ja segas kolme vaka jahu hulka, kuni kõik läks hapnema!”

\section*{Kitsas elu uks}

\par 22 Ja ta rändas mööda linnu ja külasid õpetades ja edasi minnes teed Jeruusalemma.
\par 23 Siis küsis keegi temalt: „Issand, kas on pisut neid, kes saavad õndsaks?” Aga tema ütles neile:
\par 24 „Võidelge, et te läheksite sisse kitsast uksest, sest paljud, ütlen ma teile, püüavad sisse minna, aga ei jaksa mitte!
\par 25 Sellest ajast, kui peremees on üles tõusnud ja on pannud ukse lukku, hakkate te seisma õues ja koputama uksele, öeldes: Issand, ava meile! Ja tema vastab teile, öeldes: mina ei tunne teid, kust te olete!
\par 26 Siis te hakkate ütlema: me oleme sinu ees söönud ja joonud ja meie tänavais sa õpetasid!
\par 27 Ja tema ütleb teile: ma ei tunne teid, kust te olete! Jääge minust eemale, kõik ülekohtutegijad!
\par 28 Seal on ulumine ja hammaste kiristamine, kui te näete Aabrahami ja Iisakit ja Jaakobit ja kõiki prohveteid Jumala riigis, kuid endid olevat välja tõugatud.
\par 29 Ja tuleb inimesi idast ja läänest, põhjast ja lõunast, ja nad istuvad lauas Jumala riigis.
\par 30 Ja vaata, on viimseid, kes saavad esimesteks, ja esimesi, kes jäävad viimasteks!”

\section*{Heroodese vaenulikkus}

\par 31 Samal tunnil tulid mõned variserid ja ütlesid temale: „Välju ja mine siit ära, sest Heroodes tahab sind tappa!”
\par 32 Tema ütles neile: „Minge ja öelge sellele rebasele: vaata, ma ajan kurje vaime välja ja teen terveks täna ja homme ja kolmandal päeval ma jõuan eesmärgile!
\par 33 Siiski ma pean veel rändama täna ja homme ja tunahomme; sest see ei sobi, et prohvet saab hukka mujal kui Jeruusalemmas!

\section*{Kaebehüüd Jeruusalemma saatuse pärast}

\par 34 Jeruusalemm, Jeruusalemm, kes tapad prohvetid ja viskad kividega surnuks need, kes sinu juurde on läkitatud! Kui mitu korda ma olen tahtnud koguda sinu lapsi otsekui kana oma poegi tiibade alla, aga teie ei ole tahtnud!
\par 35 Vaata, teie koda jäetakse teil maha! Ja tõesti, ma ütlen teile, et te ei näe mind enne, kui tuleb aeg, mil te ütlete: õnnistatud olgu, kes tuleb Issanda nimel!”


\chapter{14}

\section*{Vesitõbise tervekstegemine hingamispäeval}

\par 1 Ja sündis, kui tema hingamispäeval tuli ühe variseride ülema kotta leiba võtma, et nad varitsesid teda.
\par 2 Ja vaata, üks vesitõbine inimene oli tema ees!
\par 3 Ja Jeesus hakkas rääkima kirjatundjatele ja variseridele ning ütles: „Kas on luba hingamispäeval terveks teha?”
\par 4 Aga nad jäid vait. Ja puudutades ta tegi tema terveks ja laskis ta minna.
\par 5 Siis ta ütles neile: „Kes on teie seast, kelle poeg või härg kukub kaevu, et ta teda kohe välja ei tõmba hingamispäeval?”
\par 6 Ja nad ei suutnud selle peale midagi vastata.

\section*{Alandlikkusest ja külalislahkusest}

\par 7 Aga tähele pannes, kuidas need, kes olid kutsutud, valisid ülemaid istekohti, rääkis ta neile tähendamissõna ning ütles neile:
\par 8 „Kui sind keegi on pulma kutsunud, siis ära istu ülemasse paika, sest vahest on tema poolt kutsutud keegi sinust aulisem,
\par 9 ja et see, kes sind ja teda kutsus, ei ütleks sulle: anna sellele aset! ja sa peaksid siis häbiga istuma alamasse paika.
\par 10 Aga kui sa oled kutsutud, siis mine ja istu alamasse paika, ja kui tuleb see, kes sind kutsus, ja ütleb sulle: sõber, mine sinna ülemale! siis on sul au nende ees, kes sinuga istuvad ühes lauas.
\par 11 Sest igaüht, kes ennast ülendab, alandatakse, ja kes ennast alandab, ülendatakse!”
\par 12 Aga ta ütles ka sellele, kes teda oli kutsunud: „Kui sa teed lõuna- või õhtusöömaaja, siis ära kutsu oma sõpru ega oma vendi ega oma sugulasi ega rikkaid naabreid, et nad sind jälle ei kutsuks endi juurde ja sina nõnda saaksid tasu,
\par 13 vaid kui sa teed võõruspeo, siis kutsu vaeseid, vigaseid, jalutuid, pimedaid;
\par 14 siis sa oled õnnis, sest neil ei ole midagi sulle tasuda, kuna see tasutakse sulle õigete ülestõusmises!”

\section*{Tähendamissõna suurest õhtusöömaajast}

\par 15 Kui üks lauas kaasistujaist seda kuulis, ütles ta temale: „Õnnis on, kes leiba sööb Jumala riigis!”
\par 16 Aga ta ütles temale: „Üks inimene tegi suure õhtusöömaaja ja kutsus paljusid.
\par 17 Ja õhtusöömaaja tunnil ta läkitas oma sulase kutsutuile ütlema: tulge, sest kõik on juba valmis!
\par 18 Ja nad hakkasid endid kõik ühest suust vabandama. Esimene ütles talle: ma olen ostnud põllu ja pean välja minema seda vaatama; ma palun sind, vabanda mind!
\par 19 Ja teine ütles: ma olen ostnud viis paari härgi ja lähen neid katsuma; ma palun sind, vabanda mind!
\par 20 Ja veel teine ütles: ma olen võtnud naise ega saa sellepärast tulla!
\par 21 Ja sulane tuli tagasi ja teatas seda oma isandale. Siis vihastus kojaisand ja ütles oma sulasele: mine kohe välja linna laiadele tänavatele ja kurudesse ja too siia sisse vaesed ja vigased ja jalutud ja pimedad!
\par 22 Ja sulane ütles: isand, see on tehtud, nõnda nagu sa käskisid; aga veel on ruumi!
\par 23 Ja isand ütles sulasele: mine välja teede peale ja aedade äärde ja sunni sisse tulema, et minu koda täis saaks!
\par 24 Sest ma ütlen teile, et ükski neist kutsutud meestest ei saa maitsta mu õhtusöömaaega!”

\section*{Jeesuse järel käimise hind}

\par 25 Aga temaga ühes käis palju rahvast ja tema pöördus ning ütles neile:
\par 26 „Kui keegi tuleb minu juurde ja ei vihka oma isa ja ema ja naist ja lapsi ja vendi ja õdesid ja veel pealegi oma elu, see ei või olla mu jünger.
\par 27 Ja kes ei kanna oma risti ega käi minu järel, see ei või olla mu jünger.
\par 28 Sest kes teie seast, kui ta tahab ehitada torni, ei istu enne maha ega arva kulu, kas temal on, mis vaja läheb teostamiseks?
\par 29 Et kui ta aluse on rajanud ega suuda lõpetada, kõik vaatajad teda ei hakkaks naerma,
\par 30 öeldes: see mees hakkas ehitama ega suuda lõpule viia!
\par 31 Või missugune kuningas, minnes sõtta võitlema teise kuningaga, ei istu enne maha ega pea nõu, kas ta võib minna kümne tuhandega selle vastu, kes kahekümne tuhandega ta peale tuleb?
\par 32 Ja kui mitte, eks ta läkita, kui teine alles on kaugel, ja palu rahu?
\par 33 Nõnda ka igaüks teie seast, kes mitte ei loobu kõigest, mis tal on, ei või olla mu jünger.
\par 34 Sool on hea; aga kui sool tuimub, millega seda teha soolaseks?
\par 35 Ei see kõlba maasse ega sõnnikusse; see visatakse välja! Kellel kõrvad on kuulda, see kuulgu!”


\chapter{15}

\section*{Tähendamissõna kadunud lambast}

\par 1 Aga kõik tölnerid ja patused lähenesid temale, teda kuulama.
\par 2 Ja variserid ja kirjatundjad nurisesid ning ütlesid: „See võtab patuseid vastu ja sööb ühes nendega!”
\par 3 Aga ta rääkis neile selle tähendamissõna ning ütles:
\par 4 „Kes teie seast, kui tal on sada lammast ja ühe neist kaotab, ei jäta need üheksakümmend üheksa kõrbe ega lähe kadunule järele, kuni ta tema leiab?
\par 5 Ja kui ta on leidnud, võtab ta rõõmuga tema oma õlgadele,
\par 6 ja kui ta koju tuleb, ta kutsub kokku sõbrad ja naabrid ning ütleb neile: rõõmutsege ühes minuga, sest ma leidsin oma lamba, kes oli kadunud!
\par 7 Ma ütlen teile, nõnda on taevas ühest patusest, kes meelt parandab, rohkem rõõmu kui üheksakümne üheksast õigest, kes meeleparandust ei vaja!

\section*{Tähendamissõna kadunud drahmirahast}

\par 8 Või kui ühel naisel on kümme drahmiraha ja ta kaotab ühe drahmi, eks ta süüta küünla ja pühi koda ja otsi hoolega, kuni ta selle leiab?
\par 9 Ja kui ta selle on leidnud, siis ta kutsub kokku sõbrad ja naabrid ja ütleb: rõõmutsege ühes minuga, sest ma olen leidnud drahmiraha, mille ma kaotasin!
\par 10 Nõnda, ütlen ma teile, tõuseb rõõm Jumala inglite keskel ühest patusest, kes meelt parandab!”

\section*{Tähendamissõna kadunud pojast}

\par 11 Veel ütles ta: „Ühel inimesel oli kaks poega.
\par 12 Ja noorem neist ütles isale: isa, anna mulle kätte see osa vara, mis saab minule! Ja isa jagas varanduse nende vahel.
\par 13 Ja ei möödunud palju päevi, kui noorem poeg kogus kokku kõik ja siirdus kaugele maale, ja seal ta pillas ära oma vara, elades kõlvatut elu.
\par 14 Aga kui ta kõik oli ära raisanud, tuli sinna maale kange nälg, ja puudus hakkas talle kätte tulema.
\par 15 Ja ta läks ja poetas enese ühe kodaniku juurde seal maal, ja see saatis ta oma väljadele sigu hoidma.
\par 16 Ja ta püüdis oma kõhtu täita kaunadega, mida sead sõid; ja ükski ei andnud temale.
\par 17 Aga kui ta keskenes endasse, ütles ta: kui palju palgalisi on mu isal ja neil on leiba küllalt, mina aga suren siia nälja kätte!
\par 18 Ma asun teele ning lähen oma isa juurde ja ütlen temale: isa, ma olen pattu teinud taeva vastu ja sinu ees
\par 19 ega ole enam väärt, et mind su pojaks hüütaks; pea mind kui üht oma palgalist!
\par 20 Ja ta läks teele ja tuli oma isa juurde. Aga kui ta alles kaugel oli, nägi teda tema isa ja tal hakkas hale meel. Ja ta jooksis ja hakkas temale ümber kaela ja andis temale suud.
\par 21 Aga poeg ütles talle: isa, ma olen pattu teinud taeva vastu ja sinu ees ega ole enam väärt, et mind su pojaks hüütaks!
\par 22 Ent isa ütles oma sulastele: tooge kõige kallim rüü ja pange temale selga ja andke temale sõrmus sõrme ja kingad jalga,
\par 23 ja tooge, veristage nuumvasikas ja söögem ning olgem rõõmsad,
\par 24 sest see mu poeg oli surnud ja on ellu virgunud, ta oli kadunud ja on leitud! Ja nemad hakkasid rõõmutsema.
\par 25 Aga tema vanem poeg oli väljal; ja kui ta sealt tulles jõudis maja ligi, kuulis ta pilli ja ringmängu.
\par 26 Siis ta kutsus ühe teenijaist enda juurde ja kuulas, mis see peaks olema.
\par 27 See ütles temale: su vend on tulnud ja su isa on veristanud nuumvasika, et ta on saanud tema tervisega tagasi!
\par 28 Aga ta vihastus ega tahtnud sisse minna. Siis tuli ta isa välja ja palus teda.
\par 29 Kuid tema vastas ning ütles isale: vaata, nii mitu aastat orjan mina sind ja ei ole iialgi astunud üle sinu käsu, ja sina ei ole mulle elades andnud üht sikkugi, et ma oleksin võinud rõõmus olla oma sõpradega.
\par 30 Aga kui tuli see sinu poeg, kes su vara on ära raisanud hooradega, veristasid sina temale nuumvasika!
\par 31 Tema ütles temale: laps, sina oled ikka minu juures ja kõik, mis on minu, on sinu!
\par 32 Nüüd oli tarvis pidu teha ja rõõmutseda, et see sinu vend oli surnud ja on jälle saanud elavaks, ja et ta oli kadunud ja on leitud!”


\chapter{16}

\section*{Tähendamissõna kavalast majapidajast}

\par 1 Aga ta ütles ka oma jüngritele: „Oli keegi rikas mees, kellel oli majapidaja, ja seda oli tema ees süüdistatud, nagu pillaks see tema vara.
\par 2 Ta kutsus tema ja ütles talle: kuidas ma seda kuulen sinust? Tee aru oma majapidamisest, sest sa ei või enam maja pidada!
\par 3 Aga majapidaja mõtles iseeneses: mis ma teen? Minu isand võtab mu käest majapidamise ära; kaevata ma ei jaksa, kerjata on mul häbi.
\par 4 Ma tean, mis ma teen, et kui ma majapidamisest lahti saan, mind vastu võetaks nende kodadesse.
\par 5 Ja ta kutsus enda juurde igaühe oma isanda võlglasist ja ütles esimesele: palju sul on võlgu minu isandaga?
\par 6 See ütles: sada vaati õli! Tema ütles talle: võta oma arveraamat, istu maha ning kirjuta kohe viiskümmend!
\par 7 Pärast ta ütles teisele: aga palju on sinul võlgu? Tema vastas: sada vakka nisu! Ja ta ütles talle: võta oma arveraamat ning kirjuta kaheksakümmend!
\par 8 Ja isand kiitis seda ülekohtust majapidajat, et ta arukalt oli teinud, sest selle maailma lapsed on oma sugupõlve suhtes arukamad kui valguse lapsed.

\section*{Varanduse õigest kasutamisest}

\par 9 Ja mina ütlen teile: tehke endile sõpru ülekohtuse mammonaga, et kui see saab otsa, nemad võtaksid teid vastu igavestesse telkidesse.
\par 10 Kes ustav on kõige vähemas, see on ustav ka suures; ja kes ülekohtune on kõige vähemas, see on ülekohtune ka suures.
\par 11 Kui te nüüd ülekohtuses mammonas ei ole ustavad olnud, kes võib teie kätte tõelist usaldada?
\par 12 Ja kui te võõra omaga ei ole ustavad olnud, kes võib teile kätte anda meie oma?
\par 13 Ükski sulane ei või teenida kaht isandat; sest kas ta vihkab üht ja armastab teist, või hoiab ühe poole ja põlgab teise ära; te ei või teenida Jumalat ja mammonat!”

\section*{Manitsus variseridele}

\par 14 Aga seda kõike kuulsid ka variserid, kes olid rahaahned; ja nad irvitasid teda.
\par 15 Ja ta ütles neile: „Teie olete need, kes endid teevad õigeks inimeste ees; aga Jumal tunneb teie südant. Sest mis inimeste keskel on kõrge, see on Jumala ees jäledus.
\par 16 Käsuõpetus ja prohvetid ulatavad Johanneseni; sellest ajast peale kuulutatakse Jumala riiki ja igaüks tungib väevõimul sinna sisse.
\par 17 Aga kergem on, et taevas ja maa hukka lähevad, kui et üks täheke käsuõpetusest hävib!
\par 18 Igaüks, kes oma naise enesest lahutab ja võtab teise, see rikub abielu; ja igaüks, kes võtab lahutatud naise, see rikub abielu.

\section*{Rikas mees ja vaene Laatsarus}

\par 19 Oli üks rikas mees. See riietus purpuri ja kalli lõuendiga ja elas iga päev rõõmsasti ja suuresti.
\par 20 Aga üks vaene, Laatsarus nimi, oli maas tema värava ees täis paiseid
\par 21 ja püüdis oma kõhtu täita leivaraasukestega, mis rikka laualt kukkusid. Ent koeradki tulid ja lakkusid tema paiseid.
\par 22 Siis sündis, et vaene suri, ja inglid kandsid ta Aabrahami sülle. Aga ka rikas suri ja maeti maha.
\par 23 Ja kui ta põrgus suures valus oli ja oma silmad üles tõstis, nägi ta Aabrahami kaugelt ja Laatsarust tema süles.
\par 24 Ja tema hüüdis ning ütles: isa Aabraham, halasta mu peale ja läkita Laatsarus, et ta oma sõrme otsa kastaks vette ja jahutaks mu keelt, sest ma tunnen suurt piina selles tuleleegis!
\par 25 Aga Aabraham ütles: laps, mõtle, et sa oled hea põlve oma elus kätte saanud ja Laatsarus nõndasamuti kurja. Nüüd rõõmustatakse teda siin ja sina oled piinas.
\par 26 Ja kõige selle lisaks on meie ja teie vahele seatud suur kuristik, et need, kes siit tahavad minna teie juurde, ei saa mitte, ega need, kes seal on, ei saa tulla meie juurde!
\par 27 Aga ta ütles: ma palun siis sind, isa, et sa läkitaksid tema mu isa majasse,
\par 28 sest mul on viis venda - et ta kinnitaks neile, et nemadki ei satuks siia piinapaika!
\par 29 Kuid Aabraham ütles: neil on Mooses ja prohvetid; kuulaku nad neid!
\par 30 Aga tema ütles: ei mitte, isa Aabraham, vaid kui keegi surnuist läheks nende juurde, siis nad parandaksid meelt!
\par 31 Siis Aabraham ütles talle: kui nad ei kuula Moosest ja prohveteid, ei nad veenduks ka, kui keegi surnuist üles tõuseks!”


\chapter{17}

\section*{Pahandusist ja leppimisest, usust ja tegudest}

\par 1 Ja ta ütles oma jüngritele: „Võimatu on, et pahandusi ei tuleks; aga häda sellele, kelle läbi need tulevad!
\par 2 Sellele oleks tulusam, et veskikivi pandaks temale kaela ja ta visataks merre, kui et ta pahandaks ühe neist pisukesist.
\par 3 Pidage endid silmas! Kui su vend pattu teeb, siis noomi teda; ja kui ta kahetseb, anna talle andeks.
\par 4 Ja kui ta seitse korda päevas eksib su vastu ja seitse korda päevas pöördub su poole ning ütleb: ma kahetsen! siis anna temale andeks!”
\par 5 Ja apostlid ütlesid Issandale: „Kasvata meie usku!”
\par 6 Aga Issand ütles: „Kui teil usku oleks sinepiivakese võrra ja te ütleksite sellele metsviigipuule: juuri end välja ja istuta end merrel siis see kuuleks teie sõna.
\par 7 Aga kes teie seast, kellel sulane on kündmas või karja hoidmas, ütleks temale, kui ta põllult tuleb: tule kohe siia ja istu lauda!?
\par 8 Eks ta pigemini ütle temale: valmista, mida ma õhtuks söön, ja pane vöö vööle ja talita mind, kuni ma saan söönud ja joonud; ja pärast seda söö ja joo sina.
\par 9 Kas ta seda sulast tänab, et see tegi, mida kästi?
\par 10 Nõnda ka teie: kui te olete kõik teinud, mis teid on kästud, siis öelge: me oleme kõlvatud sulased; me oleme teinud, mis meie kohus oli teha!”

\section*{Jeesus teeb terveks kümme pidalitõbist}

\par 11 Ja sündis, kui ta oli teel Jeruusalemma, et ta läks Samaaria ja Galilea vahelt läbi.
\par 12 Ja kui ta saabus ühte alevisse, tulid temale vastu kümme pidalitõbist meest ja jäid eemale seisma,
\par 13 ja nad tõstsid häält ning ütlesid: „Jeesus, õpetaja, halasta meie peale!”
\par 14 Neid nähes ütles ta neile: „Minge ja näidake endid preestritele!” Ja kui nad läksid, said nad puhtaks.
\par 15 Aga üks nende seast, nähes enese terveks saanud olevat, läks tagasi ja andis Jumalale suure häälega au
\par 16 ja heitis tema jalge ette silmili maha ja tänas teda. Ja see oli samaarlane.
\par 17 Aga Jeesus kostis ning ütles: „Eks kümme ole saanud puhtaks? Ent kus on need üheksa?
\par 18 Kas muid ei ole leitud, kes oleksid tulnud tagasi Jumalale au andma, kui aga see muulane?”
\par 19 Ja ta ütles temale: „Tõuse üles ja mine; su usk on sind aidanud!”

\section*{Jumala riigi tulekust}

\par 20 Aga kui variserid temalt küsisid, millal tuleb Jumala riik, vastas ta neile ning ütles: „Jumala riik ei tule tähelepanu äratades;
\par 21 ei öelda ka mitte: vaata siin! või: vaata seal! sest ennäe, Jumala riik on seespidi teie sees!”
\par 22 Ja ta ütles jüngritele: „Päevad tulevad, mil te himustate näha üht Inimese Poja päevist, ja te ei näe mitte!
\par 23 Siis öeldakse teile: vaata siin! või: vaata seal! Ärge minge ja ärge ajage seda taga!
\par 24 Sest otsekui välk, mis sähvatab teisel pool taeva serval ja paistab teisele poole taeva servale, nõnda on Inimese Poeg omal päeval!
\par 25 Kuid enne ta peab palju kannatama ja selle sugupõlve poolt kõlbmatuna ära heidetama.
\par 26 Ja nõnda nagu oli Noa päevil, nõnda on ka Inimese Poja päevil:
\par 27 nad sõid, jõid, võtsid naisi ja läksid mehele selle päevani, mil Noa läks laeva, ja tuli veeuputus ning hävitas nad kõik.
\par 28 Samuti ka, nagu oli Loti päevil: nad sõid, jõid, ostsid, müüsid, istutasid, ehitasid hooneid;
\par 29 aga sel päeval, mil Lott väljus Soodomast, sadas tuld ja väävlit taevast ning hävitas nad kõik.
\par 30 Otse nii peab olema sel päeval, mil Inimese Poeg ilmub!
\par 31 Kes sel päeval on katusel ja kelle riistad on toas, ärgu see tulgu maha neid võtma; ja nõndasamuti, kes on põllul, ärgu mingu koju tagasi.
\par 32 Mõtelge Loti naisele!
\par 33 Kes oma hinge püüab säästa, see kaotab selle, ja kes selle kaotab, see hoiab selle elus!
\par 34 Ma ütlen teile, selsamal ööl on kaks ühes voodis: üks võetakse vastu ja teine jäetakse maha;
\par 35 kaks naist jahvatavad üheskoos: üks võetakse vastu ja teine jäetakse maha;
\par 36 [kaks meest on väljal: üks võetakse vastu ja teine jäetakse maha!]
\par 37 Ja nad vastasid ning ütlesid temale: „Kus, Issand?” Aga ta ütles neile: „Kus on korjus, sinna kogunevad ka kotkad!”


\chapter{18}

\section*{Tähendamissõna lesknaisest ja kohtunikust}

\par 1 Aga ta ütles neile tähendamissõna, selleks et alati tuleks palvetada ja mitte tüdida.
\par 2 Ta ütles: „Ühes linnas oli kohtunik, kes ei kartnud Jumalat ega häbenenud inimesi.
\par 3 Samas linnas oli ka lesknaine. See tuli ta juurde ja ütles: kaitse mu õigust mu vastase vastu!
\par 4 Ja kaua aega ta ei tahtnud. Kuid pärast ta mõtles iseeneses: ehk ma küll Jumalat ei karda ja inimesi ei häbene,
\par 5 siis ometigi, kuna see lesk mind tüütab, muretsen ma temale õiguse, et ta viimaks ei tuleks ja mulle näkku ei lööks!”
\par 6 Siis ütles Issand: „Kuulge, mida see ülekohtune kohtunik ütleb!
\par 7 Kas siis Jumal ei peaks muretsema õigust oma äravalituile, kes tema poole kisendavad ööd ja päevad, ja kas ta peaks viivitama neid aidates?
\par 8 Ma ütlen teile: küll ta peatselt muretseb neile õiguse! Siiski, kui Inimese Poeg tuleb, kas ta leiab usku maa pealt?”

\section*{Variser ja tölner}

\par 9 Aga ka mõnede vastu, kes iseeneste peale lootsid, et nad on õiged, ja põlgasid muid, ütles ta selle tähendamissõna:
\par 10 „Kaks inimest läks üles pühakotta palvetama: üks oli variser ja teine tölner.
\par 11 Variser seisis ning palus iseeneses nõnda: oh Jumal, ma tänan sind, et mina ei ole nõnda nagu muud inimesed, röövijad, ülekohtused, abielurikkujad ega nõnda nagu see tölner!
\par 12 Ma paastun kaks korda nädalas; ma annan kümnist kõigest, mis saan!
\par 13 Aga tölner seisis eemal ega tahtnud oma silmigi tõsta taeva poole, vaid lõi enesele vastu rindu ning ütles: oh Jumal, ole mulle patusele armuline!
\par 14 Ma ütlen teile, et see läks alla oma kotta paremini õigeks mõistetuna kui teine! Sest igaüht, kes ennast ise ülendab, alandatakse; aga kes ennast ise alandab, seda ülendatakse!”

\section*{Jeesus õnnistab lapsi}

\par 15 Siis toodi ka lapsukesi tema juurde, et ta neid puudutaks. Aga seda nähes jüngrid sõitlesid toojaid.
\par 16 Jeesus aga kutsus lapsed enese juurde ja ütles: „Laske lapsukesed minu juurde tulla, ärge keelake neid, sest niisuguste päralt on Jumala riik!
\par 17 Tõesti, ma ütlen teile, kes Jumala riiki vastu ei võta nagu lapsuke, see ei pääse sinna sisse!”

\section*{Igavese elu tingimuseks on varandusest loobumine}

\par 18 Ja üks ülem küsis temalt ning ütles: „Hea õpetaja, mis ma pean tegema, et ma igavese elu päriksin?”
\par 19 Aga Jeesus ütles temale: „Miks sa mind nimetad heaks? Keegi muu ei ole hea kui ainult Jumal.
\par 20 Käsud sa tead: sa ei tohi abielu rikkuda; sa ei tohi tappa; sa ei tohi varastada; sa ei tohi valet tunnistada; sa pead oma isa ja ema austama!”
\par 21 Aga tema ütles: „Seda kõike ma olen pidanud oma noorpõlvest alates!”
\par 22 Aga kui Jeesus seda kuulis, ütles ta temale: „Üht asja on sulle vaja: müü ära kõik, mis sul on, ja jaga vaestele, ja sul on siis varandus taevas; ja tule ning järgi mind!”
\par 23 Aga seda kuuldes sai ta koguni kurvaks; sest ta oli väga rikas.
\par 24 Teda nõnda nähes, ütles Jeesus: „Kui raske on jõukail sisse minna Jumala riiki!
\par 25 Sest hõlpsam on, et kaamel läheb läbi nõelasilma, kui et rikas sisse läheb Jumala riiki!”
\par 26 Ent kuuljad ütlesid: „Kes siis võib õndsaks saada?”
\par 27 Aga tema ütles: „Mis inimestel on võimatu, on Jumalal võimalik!”
\par 28 Siis ütles Peetrus: „Vaata, meie oleme kõigest loobunud ja oleme sind järginud!”
\par 29 Aga tema ütles neile: „Tõesti, ma ütlen teile, et ei ole kedagi, kes on loobunud majast või naisest või vendadest või vanemaist või lapsist Jumala riigi pärast,
\par 30 kes mitte mitu korda rohkem tagasi ei saaks sellel ajal ja tulevases maailmas igavest elu!”

\section*{Jeesus kuulutab ette kolmandat korda oma surma ja ülestõusmist}

\par 31 Aga ta võttis need kaksteist enese juurde ja ütles neile: „Vaata, me läheme üles Jeruusalemma ja kõik viiakse lõpule, mis prohvetid on kirjutanud Inimese Pojast!
\par 32 Sest ta antakse ära paganate kätte ja teda naerdakse ja teotatakse ja tema peale sülitatakse,
\par 33 ja kui nad teda on rooskadega peksnud, tapavad nad ta, ja kolmandal päeval ta tõuseb üles!”
\par 34 Aga nemad ei saanud midagi aru sellest, ja see kõne oli varjul nende eest, ja nad ei mõistnud, mida öeldi.

\section*{Jeesus teeb pimeda nägijaks}

\par 35 Aga kui ta Jeerikole lähenes, istus üks pime tee ääres ja kerjas.
\par 36 Kuuldes rahvast mööda minevat, päris ta, mis see peaks olema.
\par 37 Talle teatati: „Jeesus Naatsaretlane läheb mööda!”
\par 38 Siis ta kisendas ning ütles: „Jeesus Taaveti Poeg, halasta mu peale!”
\par 39 Ja möödaminejad sõitlesid teda, et ta vait jääks. Tema aga karjus palju rohkem: „Taaveti Poeg, halasta minu peale!”
\par 40 Jeesus jäi seisma ja käskis talutada teda enese juurde. Kui ta ligidale jõudis, küsis ta temalt:
\par 41 „Mis sa tahad, et ma sulle teeksin?” Tema ütles: „Issand, et ma jälle näeksin!”
\par 42 Ja Jeesus ütles temale: „Näe jälle! Sinu usk on sind aidanud!”
\par 43 Ja kohe ta nägi jälle ja järgis teda ning andis Jumalale au! Ja seda nähes kõik rahvas kiitis Jumalat!


\chapter{19}

\section*{Sakkeus}

\par 1 Ja tema tuli Jeeriko linna ja läks sealt läbi.
\par 2 Ja vaata, seal oli mees, Sakkeus nimi, ja see oli tölnerite ülem ja oli rikas.
\par 3 Ja ta püüdis näha saada Jeesust, kes ta on, aga ei saanud rahva pärast, sest ta oli väikese kasvuga.
\par 4 Ja ta jooksis ettepoole ja ronis metsviigipuu otsa, et teda näha; sest Jeesus pidi sealtkaudu minema.
\par 5 Kui nüüd Jeesus sinna paika jõudis, vaatas ta üles ja nägi teda ning ütles temale: „Sakkeus, tule usinasti maha, sest täna ma pean jääma sinu kotta!”
\par 6 Ja ta tuli usinasti maha ja võttis teda vastu rõõmuga.
\par 7 Seda nähes nurisesid kõik ning ütlesid: „Tema on ühe patuse mehe juurde läinud jalgu puhkama!”
\par 8 Aga Sakkeus astus Issanda ette ning ütles: „Vaata, Issand, poole oma varandusest ma annan vaestele; ja kui ma kellelegi olen ülekohut teinud, annan ma neljakordselt tagasi!”
\par 9 Siis Jeesus ütles temale: „Täna on sellele kojale õnnistus tulnud, sest ka tema on Aabrahami poeg;
\par 10 sest Inimese Poeg on tulnud otsima ja päästma, mis on kadunud!”

\section*{Tähendamissõna kümnest naelast}

\par 11 Aga kui nad seda kuulsid, ütles ta sinna juurde veel tähendamissõna, sellepärast et ta oli Jeruusalemma lähedal ja et nemad arvasid Jumala riigi varsti ilmuvat.
\par 12 Sellepärast ta ütles: „Üks suursugune inimene läks teele kaugele maale, et omandada kuningriik ja tulla tagasi.
\par 13 Aga ta kutsus oma kümme sulast ja andis neile kümme naela raha ning ütles neile: kaubelge seni, kuni ma tagasi tulen.
\par 14 Aga tema kodanikud vihkasid teda ja läkitasid käskjalad temale järele ja lasksid öelda: me ei taha, et ta on kuningaks meie üle!
\par 15 Ja sündis, kui tema tuli tagasi ja oli omandanud selle riigi, et ta käskis kutsuda enese juurde need sulased, kellele ta raha oli andnud, et teada saada, missugust kasu igaüks kaubeldes oli saavutanud.
\par 16 Siis tuli sinna esimene ja ütles: isand, sinu nael raha on tootnud kümme naela!
\par 17 Tema ütles talle: see on hea, sa hea sulane! Et sa kõige vähemas oled ustav olnud, siis olgu sul valitsus kümne linna üle!
\par 18 Ja teine tuli ja ütles: isand, sinu nael on tootnud viis naela!
\par 19 Aga ta ütles sellelegi: ja sina valitse viie linna üle!
\par 20 Veelgi tuli teine ja ütles: isand, vaata, siin on su nael, mida ma olen hoidnud higirätikus;
\par 21 sest ma kartsin sind, et sa oled vali inimene; sa võtad, mida sa ei ole paigale pannud, ja lõikad, mida sa ei ole külvanud!
\par 22 Ta ütles temale: sinu suust ma mõistan sind hukka, sa paha sulane! Kui sa teadsid, et ma olen vali mees ja võtan, mida ma ei ole pannud paigale, ja lõikan, mida ma ei ole külvanud,
\par 23 miks sa siis ei andnud mu raha panka? Küll ma tulles oleksin selle tagasi nõudnud kasudega.
\par 24 Ja ta ütles juuresseisjaile: võtke nael ära ta käest ja andke sellele, kellel on kümme!
\par 25 Ja nad ütlesid talle: isand, temal on kümme naela!
\par 26 Ma ütlen teile: kellel on, sellele antakse, aga kellel ei ole, sellelt võetakse ka see, mis tal on!
\par 27 Aga need mu vaenlased, kes ei tahtnud mind kuningaks, tooge tänna ja tapke nad ära minu ees!”
\par 28 Ja kui ta seda oli rääkinud, läks ta edasi Jeruusalemma poole.

\section*{Jeesus tuleb Jeruusalemma kui Kuningas}

\par 29 Ja sündis, kui ta Betfage ja Betaania ligi jõudis, mäe juurde, mida kutsutakse Õlimäeks, et ta läkitas kaks oma jüngrit
\par 30 ning ütles: „Minge vastasolevasse alevisse, ja kui te sinna jõuate, leiate te kinniseotud sälu, kelle seljas ükski inimene ei ole istunud. Päästke ta lahti ja tooge siia.
\par 31 Ja kui keegi teilt küsib: miks te ta lahti päästate? siis öelge nii: Issandal on teda tarvis!”
\par 32 Aga need, kes olid läkitatud, läksid ja leidsid kõik nõnda, nagu ta neile oli öelnud.
\par 33 Kui nad sälgu lahti päästsid, ütlesid selle omanikud neile: „Miks te sälu lahti päästate?”
\par 34 Nemad vastasid: „Issandal on teda tarvis!”
\par 35 Ja nad tõid selle Jeesuse juurde, heitsid oma riided sälu peale ja panid Jeesuse tema selga istuma.
\par 36 Kui ta nüüd edasi liikus, laotasid nad oma riided tee peale.
\par 37 Aga kui ta lähenes Õlimäe kallakule, hakkas kogu jüngrite hulk rõõmustades suure häälega kiitma Jumalat kõigi vägevate tegude eest, mis nad olid näinud,
\par 38 ning ütlesid: „Kiidetud olgu kuningas, kes tuleb Issanda nimel, rahu taevas ja au kõrges!”
\par 39 Ja mõningad variserid rahva seast ütlesid temale: „Õpetaja, sõitle oma jüngreid!”
\par 40 Ta vastas ning ütles: „Ma ütlen teile, kui need vait jääksid, hakkaksid kivid kisendama!”

\section*{Jeesus nutab Jeruusalemma pärast}

\par 41 Ja kui ta ligi jõudis ja linna nägi, nuttis ta tema pärast
\par 42 ning ütles: „Kui sina teaksid sellel päeval, mis sinu rahule tarvis läheb! Ent nüüd see on pandud varjule sinu silmade eest.
\par 43 Sest päevad tulevad sinu peale, mil su vaenlased teevad sinu ümber valli ja piiravad sind ja vaevavad sind kõikepidi.
\par 44 Ja lõhuvad sind maha maatasa ja su lapsed sinu sees ega jäta kivi kivi peale, sellepärast et sa ei ole tundnud oma armukatsumisaega!”

\section*{Jeesus puhastab templi}

\par 45 Ja ta läks pühakotta ja hakkas välja ajama neid, kes seal müüsid ja ostsid,
\par 46 ning ütles neile: „Kirjutatud on: minu koda on palvekoda, aga teie olete ta teinud röövliauguks!”
\par 47 Ja ta oli iga päev õpetamas pühakojas; aga ülempreestrid ja kirjatundjad ja rahvaülemad püüdsid teda hukka saata,
\par 48 kuid ei leidnud, mis teha. Sest kõik rahvas rippus ta küljes teda kuulates.


\chapter{20}

\section*{Küsimus Jeesuse meelevallast}

\par 1 Ja ühel päeval, kui ta pühakojas rahvast õpetas ja evangeeliumi kuulutas, tulid ülempreestrid ja kirjatundjad vanematega ta juurde,
\par 2 kõnetasid teda ning ütlesid: „Ütle meile, missuguse meelevallaga sa teed seda või kes on see, kes sulle selle meelevalla on andnud?”
\par 3 Tema kostis ning ütles neile: „Mina küsin ka teilt ühe asja; öelge mulle:
\par 4 kas oli Johannese ristimine taevast või inimestest?”
\par 5 Nemad pidasid nõu isekeskis, öeldes: „Kui me ütleme taevast, siis ta ütleb: mispärast te ei ole siis teda uskunud?
\par 6 Aga kui me ütleme inimestest, siis kõik rahvas viskab meid kividega, sest see on veendunud, et Johannes on prohvet!”
\par 7 Nemad vastasid, et nad ei teadvat, kust.
\par 8 Siis Jeesus ütles neile: „Ega minagi ütle teile, missuguse meelevallaga ma seda teen!”

\section*{Tähendamissõna viinamäe aednikest}

\par 9 Aga ta hakkas rääkima rahvale seda tähendamissõna: „Üks inimene istutas viinamäe ja andis selle rendile aednike kätte ning läks välismaale kauaks ajaks.
\par 10 Ja parajal ajal ta läkitas sulase aednike juurde, et nad temale annaksid viinamäe viljast. Aga aednikud peksid teda ja saatsid ta tühjalt minema.
\par 11 Ja ta läkitas veel teise sulase. Aga nad peksid tedagi ja saatsid ta tühjalt minema.
\par 12 Ja ta läkitas veel kolmanda. Aga nad haavasid ka seda ja tõukasid ta välja.
\par 13 Siis ütles viinamäe isand: mis ma pean tegema? Ma läkitan oma armsa poja, vahest nad häbenevad teda!
\par 14 Aga kui aednikud teda nägid, pidasid nad nõu isekeskis ning ütlesid: see on pärija. Tapame ta ära, et pärand saaks meile!
\par 15 Ja nad tõukasid ta viinamäest välja ja tapsid ta ära. Mis teeb nüüd viinamäe isand nendega?
\par 16 Ta tuleb ja hukkab need aednikud ja annab viinamäe teiste kätte!„ Seda kuuldes nad ütlesid: ”Ärgu seda sündigu!”
\par 17 Aga tema vaatas neile otsa ja ütles: „Mis see siis õigupoolest tähendab, mis on kirjutatud: see kivi, mille hooneehitajad kui kõlbmatu ära heitsid, see on saanud nurgakiviks!
\par 18 Igaüks, kes selle kivi peale kukub, läheb rusuks; aga kelle peale see langeb, selle ta teeb pihuks!”
\par 19 Ja ülempreestrid ja kirjatundjad püüdsid selsamal tunnil pista käed tema külge, kuid nad kartsid rahvast, sest nad said aru, et ta nende kohta oli öelnud selle tähendamissõna.

\section*{Maksu maksmisest keisrile}

\par 20 Ja nad varitsesid teda ja läkitasid välja kavalad mehed, kes pidid teesklema õigeid, et teda tabada kõnest ja teda ära anda ülemusele ning maavalitseja meelevalda.
\par 21 Ja nad küsisid temalt ning ütlesid: „Õpetaja, me teame, et sa räägid ja õpetad õigesti ega arvesta ühegi isikut, vaid õpetad Jumala teed tõtt mööda.
\par 22 Kas meil sünnib keisrile anda maksuraha või ei sünni?”
\par 23 Aga tema mõistis ära nende kurikavaluse ja ütles neile: „Miks te mind kiusate?
\par 24 Näidake mulle teenariraha! Kelle kuju ja pealkiri sellel on?„ Nemad vastasid: ”Keisri!”
\par 25 Tema ütles neile: „Siis andke keisrile, mis kuulub keisrile, ja Jumalale, mis kuulub Jumalale!”
\par 26 Ja nad ei saanud teda kinni võtta rahva ees; ja imestades tema vastust, jäid nad vait.

\section*{Surnute ülestõusmise küsimus}

\par 27 Ent ta juurde tuli mõningaid sadusere, kes ütlesid, et ei ole ülestõusmist, ja küsisid temalt
\par 28 nõnda: „Õpetaja, Mooses on meile kirjutanud, et kui kellegi vend, kes oli naisemees, sureb ja tal ei olnud lapsi, siis võtku ta vend selle naine ja saatku oma vennale sugu.
\par 29 Oli kord seitse venda; ja esimene võttis naise ja suri lapseta.
\par 30 Ja teine samuti.
\par 31 Ja kolmas võttis tema, ja nõndasamuti ka kõik seitse ei jätnud lapsi järele ja surid.
\par 32 Kõige viimaks suri ka naine.
\par 33 Kelle naiseks nende seast nüüd see naine peab olema ülestõusmises? Ta on ju olnud naiseks neile seitsmele?”
\par 34 Jeesus kostis neile: „Selle maailma-ajastu lapsed võtavad naisi ja lähevad mehele,
\par 35 aga kes on väärt arvatud jõudma teise maailma ja surnuist üles tõusma, ei need võta naist ega lähe mehele.
\par 36 Sest nad ei või ka enam surra; nad on inglite sarnased ja nad on Jumala pojad, olles ülestõusmise pojad.
\par 37 Aga et surnud tõusevad üles, seda näitab ka Mooses kibuvitsapõõsa loos, kui ta nimetab Issandat Aabrahami Jumalaks ja Iisaki Jumalaks ja Jaakobi Jumalaks;
\par 38 Jumal aga ei ole mitte surnute, vaid on elavate Jumal; sest kõik elavad temale!”
\par 39 Siis mõned kirjatundjaist vastasid ning ütlesid: „Õpetaja, sa oled hästi rääkinud!”
\par 40 Ei nad julgenud temalt ka enam midagi küsida.

\section*{Kelle poeg on Kristus?}

\par 41 Aga ta ütles neile: „Kuidas öeldakse Kristus olevat Taaveti Poeg,
\par 42 kuna Taavet ise ütleb Laulude raamatus: Issand on öelnud minu Issandale: istu mu paremale poole,
\par 43 kuni ma panen su vaenlased su jalgealuseks järiks?
\par 44 Taavet nimetab siis teda Issandaks, ja kuidas ta on tema Poeg?”
\section*{Hoiatus kirjatundjate eest}

\par 45 Aga ta ütles kõige rahva kuuldes oma jüngritele:
\par 46 „Hoiduge kirjatundjate eest, kes tahavad käia pikis rüüdes ja armastavad teretusi turgudel ja esimesi istmeid kogudusekodades ja ülemat kohta lauas söömaaegadel!
\par 47 Nemad söövad leskede hooned ja loevad silmakirjaks pikki palveid. Need saavad seda raskema hukatuse!”


\chapter{21}

\section*{Lesknaise ohvriand}

\par 1 Aga ta vaatas ja nägi rikkaid oma ande ohvrikirstu panevat.
\par 2 Ent ta nägi ka üht vaest lesknaist kaks leptonit sinna sisse panevat.
\par 3 Ja ta ütles: „Tõesti ma ütlen teile, see vaene lesknaine pani rohkem kui kõik muud!
\par 4 Sest need kõik panid oma küllusest anni, kuid tema pani oma vaesusest kõik toiduse, mis tal oli!”
\section*{Jeesus kuulutab ette templi hävitamist}

\par 5 Ja kui mõningad rääkisid pühakojast, et see on ilusate kividega ja annetistega ehitud, ütles ta:
\par 6 „Päevad tulevad, et sellest kõigest, mida te näete, ei jäeta kivi kivi peale, mida maha ei kistaks!”

\section*{Tulevased õnnetused ja hädad}

\par 7 Siis küsiti temalt: „Õpetaja, millal see siis sünnib? Ja mis on selle tunnuseks, et see hakkab sündima?”
\par 8 Aga tema ütles: „Katsuge, et teid ei eksitataks; sest paljud tulevad minu nime all ja ütlevad: mina olen see! ja: aeg on lähenenud! Ärge minge nende järele!
\par 9 Ja kui te kuulete sõdadest ja mässudest, siis ärge kohkuge; sest see peab enne sündima; aga ots ei ole veel niipea käes!”
\par 10 Siis ta ütles neile: „Rahvas tõuseb rahva vastu ja kuningriik kuningriigi vastu
\par 11 ja suuri maavärisemisi on siis ja paiguti nälga ning katku; ja hirmsaid asju on näha ning suuri tunnustähti taeval!
\par 12 Aga enne kõike seda pistavad nad oma käed teie külge ja kiusavad teid taga, andes teid ära kogudusekodadesse ja vangi, viivad teid kuningate ja maavalitsejate ette minu nime pärast.
\par 13 Aga see annab teile juhust tunnistamiseks.
\par 14 Seepärast võtke südamesse, et te enneaegu ei muretseks, mida vastata enda kaitseks.
\par 15 Sest mina annan teile suu ja tarkuse, mille vastu ei saa panna ega rääkida ükski teie vastaseist.
\par 16 Ka teie vanemad ja vennad ja sugulased ning sõbrad annavad teid ära ja surmavad mõned teie seast;
\par 17 ja teid vihatakse kõikide poolt minu nime pärast.
\par 18 Aga mitte juuksekarvgi ei saa hukka teie peast!
\par 19 Oma püsivusega kannatustes te päästate oma hinged.
\par 20 Aga kui te näete Jeruusalemma olevat sõjaväe poolt ümber piiratud, siis tundke, et ta hävitus on ligidal!
\par 21 Siis põgenegu need, kes on Juudamaal, mägedele ja need, kes on linnas, mingu välja, ja kes on maal, ärgu mingu linna.
\par 22 Sest need on kättemaksu päevad, et läheks täide, mis on kirjutatud.
\par 23 Häda käima peal olijaile ja imetajaile neil päevil, sest suur häda on maa peal ja viha selle rahva vastu!
\par 24 Ja nad langevad mõõgatera läbi ja nad viiakse vangi kõigi rahvaste sekka, ja Jeruusalemm jääb paganrahvaste tallata, kuni paganate ajad täis saavad.

\section*{Ajastu lõpp}

\par 25 Ja päikeses, kuus ja tähtedes on ennustusmärke ja maa peal on rahvastel ahastus ja nõutu olek mere kohina ja veevoogude pärast.
\par 26 Ja inimesed lähevad rammetuks kartuse ja sündmuste ootamise pärast, mis maailma peale on tulemas; sest taeva vägesid kõigutatakse.
\par 27 Ja siis nad näevad Inimese Poja tulevat pilves suure väe ja auhiilgusega!
\par 28 Aga kui see kõik hakkab sündima, siis vaadake üles ja tõstke oma pead, sest teie lunastus läheneb!”

\section*{Vajadus valvsuseks}

\par 29 Ja ta ütles neile võrdumi: „Vaadake viigipuud ja kõiki puid!
\par 30 Kui nad juba pakatavad ja te näete seda, siis tunnete iseenesest, et suvi on juba ligidal.
\par 31 Nõnda ka teie: kui te näete seda sündivat, siis tundke, et Jumala riik on ligidal!
\par 32 Tõesti ma ütlen teile, see sugupõlv ei lõpe ära, enne kui see kõik sünnib!
\par 33 Taevas ja maa hävivad, kuid minu sõnad ei hävi mitte!
\par 34 Aga hoidke, et te oma südameid ei koormaks liigsöömise ega purjutamisega ega peatoiduse muredega ja et see päev ei tuleks teie peale äkitselt
\par 35 otsekui linnupael! Sest ta tuleb kõikide peale, kes kogu maapinnal asuvad!
\par 36 Siis valvake ja paluge igal ajal, et teid arvataks väärt põgenema kõige selle eest, mis tuleb, ja seisma Inimese Poja ees!”
\par 37 Ja ta õpetas päeva ajal pühakojas, aga ööseks ta läks välja ning ööbis mäel, mida kutsutakse Õlimäeks.
\par 38 Ja kõik rahvas tuli vara hommikul tema juurde pühakotta teda kuulama.


\chapter{22}

\section*{Äraandja Juudas}

\par 1 Aga hapnemata leibade päev, mida kutsutakse paasapühaks, oli ligidal.
\par 2 Ja ülempreestrid ja kirjatundjad pidasid aru, kuidas teda surmata; sest nad kartsid rahvast.
\par 3 Aga saatan oli läinud Juudase sisse, keda liignimega hüüti Iskariotiks, kes oli nende kaheteistkümne arvust.
\par 4 Ja ta läks ära ja rääkis ülempreestrite ja sõjapealikutega, kuidas ta võiks tema ära anda nende kätte.
\par 5 Ja nad said rõõmsaks ja leppisid kokku, et nad annavad temale raha.
\par 6 Tema nõustus sellega ja otsis parajat juhust tema äraandmiseks neile ilma rahva teadmata.

\section*{Ettevalmistused paasatalle söömiseks}

\par 7 Nii jõudis kätte hapnemata leibade päev, mil paasatall pidi tapetama.
\par 8 Ja ta läkitas Peetruse ja Johannese ning ütles: „Minge ja valmistage meile paasatall, et me seda sööksime.”
\par 9 Aga nad küsisid temalt: „Kus sa tahad, et me selle valmistame?”
\par 10 Tema vastas neile: „Vaata, kui te linna sisse lähete, siis tuleb teile vastu inimene, kes kannab veekruusi; minge tema järele sinna majasse, kuhu ta sisse läheb,
\par 11 ja öelge selle maja isandale: Õpetaja ütleb sulle: kus on võõrastetuba, kus ma võiksin süüa paasatalle oma jüngritega?
\par 12 Siis ta näitab teile suure ülemise toa; seal valmistage.”
\par 13 Ent kui nad ära läksid, leidsid nad nõnda, nagu ta neile oli öelnud. Ja nad valmistasid paasatalle.

\section*{Jeesuse viimne paasatalle söömine}

\par 14 Ja kui tund kätte jõudis, istus ta maha ja apostlid ühes temaga.
\par 15 Ja ta ütles neile: „Ma olen südamest igatsenud seda paasatalle süüa ühes teiega, enne kui ma kannatan.
\par 16 Sest ma ütlen teile, et ma ei söö enam sellest, kuni kõik on täide läinud Jumala riigis!”
\par 17 Ja ta võttis karika, tänas ning ütles: „Võtke see ja jagage eneste vahel.
\par 18 Sest ma ütlen teile, et mina ei joo enam viinapuu viljast, enne kui tuleb Jumala riik!”
\par 19 Ja ta võttis leiva, tänas ja murdis ja andis neile ning ütles: „See on minu ihu, mis teie eest antakse; seda tehke minu mälestuseks!”
\par 20 Samuti ka karika pärast õhtusöömaaega ja ütles: „See karikas on uus leping minu veres, mis teie eest valatakse.
\par 21 Aga vaata, mu äraandja käsi on minuga lauas!
\par 22 Ja Inimese Poeg läheb küll ära, nõnda nagu see on määratud; ometi häda sellele inimesele, kes ta ära annab!”
\par 23 Ja nemad hakkasid küsima üksteiselt, kes see küll peaks nende seast olema, kes seda teeb.

\section*{Inimese tõeline suurus}

\par 24 Aga nende seas tõusis vaidlus ka selle kohta, keda neist tuleks arvata suuremaks.
\par 25 Siis tema ütles neile: „Rahvaste kuningad valitsevad isandaina nende üle ja nende võimumehi hüütakse armulisiks isandaiks.
\par 26 Kuid teie ärge olge nõnda, vaid suurem teie seast olgu nagu noorem, ja ülem nõnda nagu see, kes teenib.
\par 27 Sest kumb on suurem, kas see, kes istub lauas, või see, kes teenib? Eks see, kes istub lauas? Ent mina olen teie seas nõnda nagu see, kes teenib.
\par 28 Aga teie olete need, kes minu juurde on jäänud mu kiusatustes;
\par 29 ja mina sean teile riigi, nõnda nagu minu Isa selle mulle on seadnud,
\par 30 et te sööksite ja jooksite minu lauas minu riigis ja istuksite aujärgedel ja mõistaksite kohut Iisraeli kaheteistkümne suguharu üle!”

\section*{Jeesus kuulutab ette, et Peetrus teda salgab}

\par 31 „Siimon, Siimon, vaata, saatan on väga püüdnud sõeluda teid nagu nisu!
\par 32 Aga mina olen sinu eest palunud, et su usk ära ei lõpeks, ja kui sa pärast pöördud, siis kinnita oma vendi!”
\par 33 Aga tema ütles talle: „Issand, ma olen valmis sinuga minema niihästi vangi kui surma!”
\par 34 Kuid ta ütles: „Ma ütlen sulle, Peetrus: kukk ei laula täna mitte enne, kui sina oled kolm korda salanud, et sa mind tunned!”
\par 35 Ja ta ütles neile: „Kui ma teid läkitasin kukruta ja paunata ja jalatseita, kas oli teil midagi puudu?” Nemad vastasid: „Ei midagi!”
\par 36 Siis ta ütles neile: „Aga nüüd, kellel on kukkur, see võtku ta kaasa, nõndasamuti ka paun; ja kellel ei ole, see müügu ära oma kuub ja ostku mõõk!
\par 37 Sest ma ütlen teile, et minus peab veel täide minema, mis on kirjutatud: „Ja teda arvati üleastujate hulka!” Sest mis minusse puutub, jõuab lõpule!”
\par 38 Aga nemad ütlesid: „Issand, ennäe, siin on kaks mõõka!” Tema ütles neile: „Sellest jätkub!”

\section*{Jeesus Õlimäel}

\par 39 Ja ta läks välja ja tuli oma viisi järgi Õlimäele. Ent ka tema jüngrid järgisid teda.
\par 40 Ja kui ta jõudis kohale, ütles ta neile: „Palvetage, et te ei satuks kiusatusse!”
\par 41 Ta ise läks neist eemale, niikaugele kui kiviga jõuab visata, heitis põlvili maha ja palvetas,
\par 42 öeldes: „Isa, kui sa tahad, siis võta see karikas minult ära; ometi ärgu sündigu minu, vaid sinu tahtmine!”
\par 43 Siis ilmus temale ingel taevast ja kinnitas teda.
\par 44 Ja heideldes raskesti, palvetas ta veel pinevamalt; ja tema higi oli nagu verepisarad, mis langesid maa peale.
\par 45 Ja ta tõusis üles palvetamast ning tuli oma jüngrite juurde ja leidis nad magamast kurbuse pärast.
\par 46 Ja ta ütles neile: „Miks te magate? Tõuske üles ja palvetage, et te ei satuks kiusatusse!”

\section*{Jeesuse vangistamine}

\par 47 Kui tema alles rääkis, vaata, siis tuli rahvast ja see, keda hüüti Juudaseks, üks neist kaheteistkümnest, käis nende ees ja tuli Jeesuse ligi teda suudlema.
\par 48 Aga Jeesus ütles temale: „Juudas, annad sa Inimese Poja suudlusega ära?”
\par 49 Kui nüüd kaaslased seda nägid, mis oli tulemas, ütlesid nad: „Issand, kas me peame mõõgaga sekka lööma?”
\par 50 Ja üks neist lõi ülempreestri sulast ning raius tema parema kõrva ära.
\par 51 Aga Jeesus kostis ning ütles: „Jätke sellega!” Ja ta puudutas tema kõrva ja tegi ta terveks.
\par 52 Siis Jeesus ütles ülempreestritele ja pühakoja pealikuile ja vanemaile, kes tema vastu olid tulnud: „Te olete tulnud välja otsekui röövli vastu mõõkade ja nuiadega!
\par 53 Kui ma iga päev teie juures olin pühakojas, ei ole te pistnud käsi mu külge. Ent see on teie tund ja pimeduse võimus!”

\section*{Peetrus salgab Jeesuse}

\par 54 Aga nad võtsid ta kinni ja viisid ning tõid tema ülempreestri kotta. Aga Peetrus järgis teda kaugelt.
\par 55 Ja kui nad keset õue süütasid tule ja üheskoos maha istusid, istus ka Peetrus nende keskele.
\par 56 Siis nägi üks ümmardaja teda tule ääres istuvat ning jäi temale otsa vaatama ja ütles: „Ka see oli ühes temaga!”
\par 57 Aga tema salgas ning ütles: „Naine, mina teda ei tunne!”
\par 58 Üürikese aja pärast nägi teda teine, keegi mees, ning ütles: „Sinagi oled nende seast!” Kuid Peetrus ütles: „Inimene, mina ei ole mitte!”
\par 59 Ja umbes tund aega hiljem kinnitas seda keegi teine ning ütles: „Tõepoolest, see oli temaga, sest ta on ju galilealane!”
\par 60 Aga Peetrus ütles: „Inimene, ma ei mõista, mida sa räägid!” Ja sedamaid laulis kukk!
\par 61 Ja Issand pöördus ja vaatas Peetrusele; ja Peetrusele tuli meelde Issanda sõna, kuidas ta temale oli öelnud: „Enne kui kukk täna laulab, salgad sa mind kolm korda!”
\par 62 Ja ta läks välja ja nuttis kibedasti.

\section*{Jeesust pilgatakse ja pekstakse}

\par 63 Aga mehed, kes Jeesust kinni hoidsid, naersid ja peksid teda,
\par 64 ja nad katsid ta silmad kinni ja küsisid temalt ning ütlesid: „Mõista kui prohvet, kes see on, kes sind lõi!”
\par 65 Ja palju muid pilkesõnu nad rääkisid tema vastu.

\section*{Jeesus Suurkohtu ees}

\par 66 Ja kui valgeks läks, tulid rahvavanemad, ülempreestrid ja kirjatundjad kokku ja viisid ta üles oma Suurkohtu ette
\par 67 ning ütlesid: „Kui sina oled Kristus, siis ütle meile!” Tema vastas neile: „Kui ma teile ütlen, ei usu te ju mitte,
\par 68 ja kui ma küsin, ei vasta te mitte!
\par 69 Ent nüüdsest peale hakkab Inimese Poeg istuma Jumala väe paremal poolel!”
\par 70 Aga nad kõik ütlesid: „Kas siis sina oled Jumala Poeg?” Tema ütles neile: „Jah, ma olen!”
\par 71 Siis nad ütlesid: „Mis tunnistust me veel vajame? Oleme ju ise seda kuulnud tema suust!”


\chapter{23}

\section*{Jeesus maavalitseja Pilaatuse ees}

\par 1 Ja kogu nende hulk tõusis ning viis tema Pilaatuse juurde.
\par 2 Ja nad hakkasid kaebama tema peale ning ütlesid: „Me oleme leidnud, et tema rahvast eksitab ja keelab andmast maksuraha keisrile ning ütleb enese kuningas Kristuse olevat!”
\par 3 Siis Pilaatus küsis temalt ning ütles: „Kas sina oled juutide kuningas?” Ja ta vastas temale, öeldes: „Jah, olen!”
\par 4 Pilaatus aga ütles ülempreestritele ja rahvale: „Ma ei leia sellest inimesest ühtki süüd!”
\par 5 Aga nemad ajasid peale ja ütlesid: „Tema ässitab rahvast ja õpetab mööda kõike Juudamaad, alates Galileast kuni siiani!”

\section*{Jeesus Heroodese ees}

\par 6 Aga kui Pilaatus seda kuulis, küsis ta, kas see inimene ei ole mitte galilealane?
\par 7 Ja saades teada, et ta on Heroodese valitsuse alt, saatis ta tema Heroodese juurde, kes ka viibis Jeruusalemmas neil päevil.
\par 8 Kui siis Heroodes Jeesust nägi, sai ta väga rõõmsaks, sest ta oli hea meelega tahtnud teda juba ammugi näha, kuna ta oli temast palju kuulnud; ka lootis ta näha saada mõnd imetähte, mida ta ehk teeb.
\par 9 Ta küsis temalt palju asju, aga tema ei vastanud talle midagi.
\par 10 Aga ülempreestrid ja kirjatundjad seisid ja kaebasid väga valjusti tema peale.
\par 11 Kui siis Heroodes oma sõjameestega teda oli halvaks pannud ja pilganud ning riietanud toreda riidega, saatis ta tema tagasi Pilaatuse juurde.
\par 12 Ja sel päeval said Pilaatus ja Heroodes teineteisega sõbraks, sest enne nad olid olnud teineteise vihamehed.

\section*{Jeesus või Barabas?}

\par 13 Siis Pilaatus kutsus kokku ülempreestrid ning rahvaülemad ja rahva
\par 14 ning ütles neile: „Te olete toonud selle inimese minu juurde otsekui rahva eksitaja. Ja vaata, ma olen teda üle kuulanud teie ees ega ole sellest inimesest leidnud ühtki süüd, milles te teda süüdistate,
\par 15 ega Heroodeski, vaid ta on tema saatnud tagasi meie juurde. Ja vaata, tema pole teinud midagi, mis oleks surma väärt!
\par 16 Sellepärast ma tahan teda karistada ja vabaks lasta!”
\par 17 [Aga temal oli kohustuseks neile pühiks vabaks lasta üks vang.]
\par 18 Siis kogu hulk kisendas väga ning ütles: „Hukka ära, lase meile Barabas vabaks!”
\par 19 See oli mingi linnas sündinud mässu ja tapmise pärast vangi heidetud.
\par 20 Siis Pilaatus tõstis jälle häält nende vastu, tahtes vabaks lasta Jeesuse.
\par 21 Aga nemad karjusid: „Löö risti, löö ta risti!”
\par 22 Siis ta ütles neile kolmandat puhku: „Mis ta siis on kurja teinud? Ma pole temast leidnud ühtki surmasüüd. Sellepärast ma karistan teda ja lasen ta vabaks!”
\par 23 Aga nad käisid peale, suure kisendamisega nõudes, et Jeesus löödaks risti. Ja nende hüüded võtsid võimust.
\par 24 Siis Pilaatus tegi otsuse, et sünniks nende palve järgi.
\par 25 Ta laskis neile vabaks selle, kes mässamise ja tapmise pärast oli vangitorni heidetud ja keda nemad vabaks palusid. Aga Jeesuse ta andis nende meelevalla kätte.

\section*{Jeesuse ristilöömine}

\par 26 Ja teda välja viies said nad kätte kellegi Siimona Küreenest, kes tuli väljalt; ja tema peale nad panid risti, et ta seda kannaks Jeesuse järel.
\par 27 Aga teda järgis suur hulk rahvast ja ka naisi, kes kaebasid ja teda nutsid.
\par 28 Siis Jeesus pöördus nende poole ja ütles: „Jeruusalemma tütred, ärge nutke mind, vaid nutke iseendid ja oma lapsi!
\par 29 Sest vaata, päevad tulevad, mil öeldakse: õndsad on sigimatud ja ihud, mis ei ole ilmale kandnud, ja rinnad, mis ei ole imetanud!
\par 30 Siis hakatakse ütlema mägedele: langege meie peale! ja mäekinkudele: katke meid!
\par 31 Sest kui seda tehakse toore puuga, mis sünnib siis kuivaga!”
\par 32 Aga ka kaks muud kurjategijat viidi ühes temaga välja hukkamiseks.
\par 33 Ja kui nad jõudsid sinna paika, mida hüütakse Pealaeks, lõid nad tema sinna risti ja need kurjategijad, ühe paremale, teise vasakule poole.
\par 34 Aga Jeesus ütles: „Isa, anna neile andeks, sest nad ei tea, mida nad teevad!” Ja nad jagasid tema riided ning heitsid liisku nende pärast.
\par 35 Ja rahvas seisis seal vaatamas, ja ka ülemad irvitasid ühes nendega ning ütlesid: „Muid ta on aidanud, aidaku iseennast, kui ta on Jumala Võitu, see äravalitu!”
\par 36 Ka sõjamehed naersid teda, läksid tema juurde ja viisid temale äädikat
\par 37 ning ütlesid: „Kui sina oled juutide kuningas, siis aita iseennast!”
\par 38 Aga oli ka pealkiri tema kohal: „See on juutide kuningas!”
\par 39 Siis teine poodud kurjategijaist pilkas teda ning ütles: „Eks sa ole Kristus? Aita iseennast ja meid!”
\par 40 Aga teine kostis ja sõitles teda ning ütles: „Kas sinagi ei karda Jumalat, kuna sa ise oled ju sama karistuse all?
\par 41 Meie küll õiguse poolest, sest me saame kätte, mis meie teod on ära teeninud, aga see ei ole teinud midagi ebakohast!”
\par 42 Ja ta ütles: „Jeesus, mõtle minule, kui sa oma kuningriiki tuled!”
\par 43 Ja Jeesus ütles temale: „Tõesti ma ütlen sulle, täna pead sa minuga olema paradiisis!”

\section*{Jeesuse surm}

\par 44 Ja oli juba umbes kuues tund, siis tekkis pimedus üle kogu maa üheksandast tunnist saadik,
\par 45 sest päike pimenes. Ja templi eesriie kärises keskelt lõhki!
\par 46 Ja Jeesus kisendas suure häälega ning ütles: „Isa, sinu kätte ma annan oma vaimu!” Ja kui ta seda oli öelnud, heitis ta hinge.
\par 47 Aga kui pealik nägi, mis sündis, andis ta Jumalale au ning ütles: „See inimene oli tõesti õige!”
\par 48 Ja kui kõik rahvahulgad, kes olid kokku tulnud seda vaatama, nägid, mis sündis, lõid nad enestele vastu rindu ja läksid tagasi.
\par 49 Aga kõik ta tuttavad ja ka naised, kes olid järginud teda Galileamaalt, seisid eemal ning nägid seda.

\section*{Jeesuse matmine}

\par 50 Ja vaata, mees nimega Joosep, Suurkohtu liige, hea ja õiglane mees,
\par 51 kes ei olnud ühes nõus nende otsusega ega teoga, Arimaatiast, Juuda rahva linnast, mees, kes ootas Jumala riiki,
\par 52 see läks Pilaatuse juurde ja palus enesele Jeesuse ihu.
\par 53 Ja ta võttis selle maha ja mähkis ta kallisse lõuendisse. Ja ta pani ta kaljusse raiutud hauda, kuhu ei olnud veel iialgi kedagi pandud.
\par 54 Ja see oli valmistuspäev, ja hingamispäev oli tulemas.
\par 55 Aga naised, kes Galileast olid teda järginud, käisid kaasas ja vaatasid hauda ja kuidas tema ihu sinna pandi.
\par 56 Ja kui nad olid tagasi tulnud, valmistasid nad lõhnarohte ja salvi. Aga hingamispäeval nad seisid rahul käsu järgi.


\chapter{24}

\section*{Jeesuse ülestõusmine}

\par 1 Aga nädala esimesel päeval puhteajal läksid naised hauale, viies ühes lõhnarohud, mis nad olid valmistanud.
\par 2 Ja nad leidsid kivi haualt ära veeretatud.
\par 3 Siis nad läksid sisse, kuid ei leidnud mitte Issanda Jeesuse ihu.
\par 4 Ja kui nad selle pärast kahevahel olid, vaata, siis seisid kaks meest nende kõrval säravais riideis!
\par 5 Aga kui naised hakkasid kartma ja oma silmad maha lõid, ütlesid mehed neile: „Miks te elavat otsite surnute juurest?
\par 6 Tema ei ole siin, vaid on üles tõusnud! Tuletage meelde, mis ta teile rääkis veel Galileas olles,
\par 7 kui ta teile ütles, et Inimese Poeg antakse patuste inimeste kätte ja lüüakse risti ja kolmandal päeval ta tõuseb jälle üles!”
\par 8 Siis nad tuletasid meelde tema sõnad
\par 9 ja läksid tagasi haua juurest ja kuulutasid seda kõike neile üheteistkümnele ja kõigile teistele.
\par 10 Aga Maarja Magdaleena ja Johanna ja Maarja, Jakoobuse ema, ja teised nende kaaslased jutustasid seda apostlitele.
\par 11 Ent need kõned olid nende meelest nagu tühi jutt ja nad ei uskunud naisi.
\par 12 Aga Peetrus võttis kätte ja jooksis hauale. Ja kui ta kummargile sisse vaatas, nägi ta üksnes surnulinad maas olevat. Ja ta läks ära ja pani seda imeks, mis oli sündinud.

\section*{Jeesus ilmub jüngritele Emmause teel}

\par 13 Ja vaata, kaks nende seast olid minemas selsamal päeval külasse, mis on ligi kuuskümmend vagu maad Jeruusalemmast ja mille nimi on Emmaus.
\par 14 Ja nad kõnelesid isekeskis kõigest, mis oli sündinud.
\par 15 Ja nende kõneldes ja vaieldes lähenes ka Jeesus ise neile ja käis nendega.
\par 16 Aga nende silmad peeti, nii et nad teda ei tundnud.
\par 17 Ja ta ütles neile: „Mis kõned need on, mis te käies kõnelete isekeskis?” Siis nad seisatasid kurvanäolistena.
\par 18 Aga teine, nimega Kleopas, vastas ning ütles temale: „Sinaks üksi elad kui võõras Jeruusalemmas ega tea, mis neil päevil seal on sündinud?”
\par 19 Ta küsis neilt: „Mis?” Aga nad ütlesid temale: „See, mis sündis Jeesus Naatsaretlasega, kes oli prohvet, vägev teolt ja sõnalt Jumala ja kõige rahva ees,
\par 20 kuidas meie ülempreestrid ja vanemad on andnud ta surma mõista ja on ta risti löönud.
\par 21 Ent meie lootsime tema olevat selle, kes Iisraeli rahva lunastab; aga peale selle kõige on täna kolmas päev, kui see sündis!
\par 22 Ka mõned naised meie seast, kes puhteajal haual käisid, on meid ehmatanud:
\par 23 kui nad tema ihu ei leidnud, tulid nad ja ütlesid endid näinud olevat ka inglite nägemust, kes ütlevad tema elavat!
\par 24 Ja mõningad neist, kes olid meiega, läksid hauale ja leidsid nõnda olevat, nagu naised olid öelnud; kuid teda ennast nad ei näinud!”
\par 25 Tema ütles neile: „Oh te mõistmatud ja südamest pikaldased uskuma seda kõike, mis prohvetid on rääkinud!
\par 26 Eks Kristus pidanud seda kannatama ja oma auhiilgusesse minema?”
\par 27 Ja ta hakkas peale Moosesest ja kõigist prohveteist ja seletas neile, mis temast kõigis kirjades oli öeldud.
\par 28 Ja kui nad lähenesid külale, kuhu nad olid minemas, tegi ta enese eemale minema.
\par 29 Ja nad käisid temale peale ning ütlesid: „Jää meie juurde, sest õhtu jõuab ja päev veereb!” Ja tema läks sisse, et nende juurde jääda.
\par 30 Ja sündis, kui ta nendega lauas istus, et ta võttis leiva, õnnistas ja murdis ning andis neile.
\par 31 Siis nende silmad läksid lahti ja nad tundsid tema ära! Ja tema kadus nende silmist.
\par 32 Ja nad ütlesid üksteisele: „Eks meie süda põlenud meie sees, kui ta teel meiega rääkis ja meile kirju seletas?”
\par 33 Ja nad tõusid samal tunnil ja läksid tagasi Jeruusalemma ja leidsid koos olevat need üksteist ja kaaslased,
\par 34 kes ütlesid: „Issand on tõesti üles tõusnud ja Siimonale ilmunud!”
\par 35 Ja nemad ise jutustasid, mis teel oli sündinud ja kuidas nad tema olid leiva murdmisest ära tundnud.

\section*{Jeesus ilmub jüngritele Jeruusalemmas}

\par 36 Aga kui nad seda rääkisid, seisis Jeesus ise nende keskel ja ütles neile: „Rahu olgu teile!”
\par 37 Nemad kohkusid ja lõid kartma ning arvasid vaimu nägevat.
\par 38 Ent tema ütles neile: „Miks te olete nii väga ehmunud? Ja mispärast tõuseb niisuguseid mõtteid teie südamest?
\par 39 Vaadake mu käsi ja jalgu, et mina see olen! Katsuge mind kätega ja nähke, sest vaimul ei ole liha ega luid, nõnda nagu te näete minul olevat!”
\par 40 Ja seda öeldes näitas ta neile oma käsi ja jalgu.
\par 41 Aga kui nad rõõmu pärast veel ei uskunud ja imeks panid, ütles ta neile: „Kas teil on siin midagi süüa?”
\par 42 Ja nad panid tema ette pala küpsetatud kala.
\par 43 Ja tema võttis ja sõi nende ees.
\par 44 Siis ta ütles neile: „Need on kõned, mis ma teile rääkisin veel teie juures olles; sest kõik peab täide minema, mis on kirjutatud minu kohta Moosese käsuõpetuses ja Prohveteis ja Lauludes!”
\par 45 Siis ta avas nende mõistuse, nõnda et nad kirjadest aru said.
\par 46 Ja ta ütles neile: „Nõnda on kirjutatud ja nõnda pidi Kristus kannatama ja surnuist üles tõusma kolmandal päeval,
\par 47 ja tema nimel peab kuulutatama meeleparandust pattude andekssaamiseks kõigi rahvaste seas, alates Jeruusalemmast.
\par 48 Teie olete nende asjade tunnistajad!
\par 49 Ja vaata, mina läkitan teie peale oma Isa tõotuse! Ent teie jääge sellesse linna seni, kuni teid ehitakse väega kõrgest!”

\section*{Jeesuse taevaminek}

\par 50 Siis ta viis nad välja Betaania lähedale ja tõstis oma käed üles ja õnnistas neid.
\par 51 Ja sündis, et ta neid õnnistades lahkus neist ja võeti üles taevasse.
\par 52 Ja nemad kummardasid teda ning läksid tagasi Jeruusalemma suure rõõmuga
\par 53 ja olid alati pühakojas, tänades Jumalat!






\end{document}