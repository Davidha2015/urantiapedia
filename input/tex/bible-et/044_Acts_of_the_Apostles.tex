\begin{document}

\title{Apostlite teod}

\chapter{1}

\section*{Eeskõne}

\par 1 Esimese raamatu, Teofilos, ma kirjutasin kõigest sellest, mida Jeesus algusest peale tegi ja õpetas
\par 2 selle päevani, mil ta võeti üles pärast seda kui ta Püha Vaimu läbi oli andnud käsu apostlitele, keda ta oli valinud,
\par 3 kellele tema ka pärast oma kannatamist oli mitme tõestusmärgiga ennast näidanud elavana, neile ilmudes nelikümmend päeva ning neile rääkides Jumala riigist.
\par 4 Ja kui ta nendega koos oli, keelas ta neid Jeruusalemmast ära minemast, käskis aga oodata Isa tõotust, „mis teie,” ütles tema, „olete kuulnud minult.
\par 5 Sest Johannes ristis veega, aga teid peab ristitama Püha Vaimuga mitte kaua pärast neid päevi!”

\section*{Jeesuse taevaminek}

\par 6 Siis küsisid temalt need, kes olid kokku tulnud, ning ütlesid: „Issand, kas sa sel ajal jälle ehitad Iisraelile kuningriigi?”
\par 7 Tema ütles neile: „Teile ei sünni teada aegu ega hetki, mis Isa omas meelevallas on määranud;
\par 8 aga te saate Püha Vaimu väe, kes tuleb teie peale, ja peate olema minu tunnistajad Jeruusalemmas ja kõigel Juuda- ja Samaariamaal ja maailma otsani!”
\par 9 Kui ta seda oli öelnud, siis tõsteti ta üles nende nähes ja pilv viis ta nende silme eest ära.
\par 10 Ja kui nad vaatasid üksisilmi taeva poole, kuidas ta ära läks, vaata, siis seisid nende juures kaks meest valgeis riideis,
\par 11 ja need ütlesid: „Galilea mehed, miks te seisate ja vaatate üles taeva poole? See Jeesus, kes teilt üles võeti taevasse, tuleb samal kombel kui te nägite teda taevasse minevat!”

\section*{Kaheteistkümnenda apostli valimine}

\par 12 Siis nad läksid tagasi Jeruusalemma mäelt, mida kutsutakse Õlimäeks ja mis on Jeruusalemma lähedal ühe hingamispäeva teekonna maad.
\par 13 Ja kui nad olid jõudnud linna, siis nad läksid ülemisse tuppa, kus nad olid asumas, Peetrus ja Jakoobus ja Johannes ja Andreas, Filippus ja Toomas, Bartolomeus ja Matteus, Jakoobus, Alfeuse poeg ja Siimon Selootes ja Juudas, Jakoobuse vend.
\par 14 Need kõik olid alati ühel meelel üheskoos palves naistega ja Maarjaga, Jeesuse emaga, ja Jeesuse vendadega.
\par 15 Neil päevil tõusis Peetrus vendade seas üles - rahvast aga oli koos umbes sada kakskümmend isikut - ning ütles:
\par 16 „Mehed, vennad, täide on pidanud minema kirjasõna, mis Püha Vaim on ette kuulutanud Taaveti suu kaudu Juuda kohta, kes hakkas juhiks Jeesuse kinnivõtjaile.
\par 17 Sest ta oli arvatud meie sekka ja oli saanud selle ameti osaliseks.
\par 18 Aga tema hankis enesele põllu ülekohtu palgaga, ja kui ta ülepeakaela alla kukkus, lõhkes ta keskpaigast ja kogu ta sisikond vajus välja.
\par 19 Ja see on saanud teatavaks kõigile Jeruusalemma elanikele, nõnda et seda põldu nende oma keelemurdes hüütakse Akeldaamaks, see on Verepõlluks.
\par 20 Sest Laulude raamatus on kirjutatud: „Tema elamu saagu tühjaks ja ärgu olgu selles elanikku, ja tema ülevaatajaamet saagu teisele!”
\par 21 Sellepärast peab nüüd üks neist meestest, kes on käinud ühes meiega kõige selle aja, mil Issand Jeesus meie juures käis sisse ja välja,
\par 22 alates Johannese ristimisest kuni selle päevani, mil ta meilt võeti üles, saama tema ülestõusmise tunnistajaks ühes meiega!”
\par 23 Ja nad seadsid üles kaks meest: Joosepi, keda hüütakse Barsabaseks, lisanimega Justuseks, ja Mattiase.
\par 24 Ja nad palvetasid ning ütlesid: „Sina, Issand, kõikide südametundja, näita meile, kumma neist kahest sina oled valinud
\par 25 saama selle tunnistuse ja apostliameti koha, millest Juudas ära astus, et minna oma paika!”
\par 26 Ja nad heitsid liisku nende kohta; ja liisk langes Mattiasele, ja tema arvati nende üheteistkümne apostli juurde.


\chapter{2}

\section*{Püha Vaimu tulek nelipühapäeval}

\par 1 Ja kui nelipühapäev kätte tuli, olid kõik ühes paigas koos.
\par 2 Ja taevast sündis äkitselt kohin, otsekui oleks kange tuul puhunud, ja täitis kõik koja, kus nad istusid.
\par 3 Ja neile ilmus nagu lõhestatud tulekeeli, ja need langesid üksikult igaühe peale nende seast.
\par 4 Ja nad kõik said täis Püha Vaimu ja hakkasid rääkima teisi keeli, nõnda nagu Vaim neile andis rääkida.
\par 5 Aga Jeruusalemmas oli elamas juute, jumalakartlikke mehi kõigist rahvaist taeva all.
\par 6 Kui nüüd see hääl tekkis, tuli rahvahulk kokku ja kohkus ära, sest igaüks kuulis neid rääkivat oma keelemurret.
\par 7 Ja nad hämmastusid kõik ja panid imeks ning ütlesid üksteisele: „Vaata, eks need kõik, kes räägivad, ole Galilea mehed?
\par 8 Kuidas me siis kuuleme igaüks oma keelemurret, mille sees me oleme sündinud?
\par 9 Meie, partlased ja meedlased ja eelamlased ja kes elame Mesopotaamias, Judeas ning Kapadookias, Pontoses ja Aasias,
\par 10 Früügias ja Pamfüülias, Egiptuses ja Küreene-poolses Liibüas, ja siin asuvad roomlased, juudid ühes nende usku pöördunutega,
\par 11 kreetalased ja araablased, kuidas me kuuleme neid rääkivat meie omis keeltes Jumala suuri asju?”
\par 12 Nad kõik olid hämmastunud ja kahevahel ning küsisid üksteiselt: „Mis see küll peaks tähendama?”
\par 13 Aga teised irvitasid ning ütlesid: „Nad on täis magusat viina!”

\section*{Peetruse jutlus}

\par 14 Siis Peetrus astus esile ühes nende üheteistkümnega, tõstis oma häält ning seletas neile: „Juuda mehed ja kõik teie, Jeruusalemma elanikud, see olgu teile teada ja pange tähele mu sõnu!
\par 15 Ei ole need sugugi, nagu te arvate, joobnud, sest see on kolmas päevatund;
\par 16 vaid see on, mis on öeldud prohvet Joeli kaudu:
\par 17 „Ja viimseil päevil sünnib, ütleb Jumal, et ma valan oma Vaimu kõige liha peale! Ja teie pojad ja tütred hakkavad ennustama, teie noored mehed näevad nägemusi, ja teie vanemad uinuvad unenägusid nähes!
\par 18 Ja neil päevil ma valan oma Vaimu oma sulaste ja ümmardajate peale ja nemad ennustavad;
\par 19 ja ma annan näha imesid ülal taevas ja tunnustähti all maa peal, verd ja tuld ja suitsusambaid!
\par 20 Päike muutub pimedaks ja kuu vereks, enne kui tuleb Issanda päev, suur ja auline!
\par 21 Ja sünnib, et igaüks, kes appi hüüab Issanda nime, pääseb!”
\par 22 Iisraeli mehed, kuulge neid sõnu: Jeesuse Naatsaretlase, mehe, kellest Jumal andis tunnistuse väe ja imede ja tunnustähtedega, mis Jumal tegi tema läbi teie seas, nõnda nagu isegi teate,
\par 23 tema te olete, kui ta Jumala määratud otsusel ja etteteadmist mööda oli loovutatud teie kätte, ülekohtuste käte läbi risti naelutanud ning tapnud;
\par 24 kuid Jumal on tema üles äratanud, päästes teda surma valudest, sest ei olnud ju võimalik, et surm oleks teda kinni pidanud.
\par 25 Sest Taavet ütleb tema kohta: „Ma pean Issandat alati oma silma ees, sest ta on mu paremal pool, et ma ei kõiguks!
\par 26 Sellepärast rõõmutseb mu süda ja mu keel ilutseb, ja mu ihugi võib hingata lootuses!
\par 27 Sest sina ei jäta mu hinge surmavalda ega lase oma Püha näha kõdunemist!
\par 28 Sa annad mulle teada elutee, sa täidad mind rõõmuga oma palge ees!”
\par 29 Mehed, vennad, olgu lubatud teie peavanemast Taavetist lausa öelda, et ta on surnud ja maha maetud ja tema haud on meie juures tänapäevani.
\par 30 Et ta nüüd oli prohvet ja teadis, et Jumal temale oli vandega tõotanud seada tema ihusoost järglase tema aujärjele,
\par 31 siis ta rääkis ette nähes Kristuse ülestõusmisest, et tema hinge ei jäeta surmavalda ja et tema liha ei näe kõdunemist.
\par 32 Selle Jeesuse on Jumal surnuist üles äratanud; selle tunnistajad oleme meie kõik.
\par 33 Et ta nüüd Jumala parema käe läbi on ülendatud ja on Isalt saanud Püha Vaimu tõotuse, siis on tema selle välja valanud, nõnda nagu te nüüd näete ning kuulete.
\par 34 Sest Taavet ei ole läinud taevasse, aga tema ütleb: „Issand on öelnud minu Issandale: istu minu paremale käele,
\par 35 kuni ma panen su vaenlased sinu jalajäriks!”
\par 36 Kindlasti teadku nüüd kogu Iisraeli kodakond, et Jumal on teinud tema Issandaks ning Kristuseks, selle Jeesuse, kelle te risti lõite!”
\par 37 Aga kui nad seda kuulsid, läks see neil südamest läbi ja nad ütlesid Peetrusele ja teistele apostlitele: „Mehed, vennad, mis me peame tegema?”
\par 38 Aga Peetrus ütles neile: „Parandage meelt ja igaüks teist lasku ennast ristida Jeesuse Kristuse nimesse pattude andekssaamiseks, ja siis te saate Püha Vaimu anni.
\par 39 Sest teie ja teie laste päralt on see tõotus ja kõikide päralt, kes on kaugel, keda iganes Issand, meie Jumal, kutsub enese juurde.”
\par 40 Ja veel paljude muude sõnadega tunnistas ja manitses ta ning ütles: „Laske endid päästa sellest pöörasest soost!”
\par 41 Kes nüüd vastu võtsid tema sõna, need ristiti, ja nõnda lisati sel päeval nende juurde ligi kolm tuhat hinge.

\section*{Kristuse koguduse usuline elu}

\par 42 Aga nemad jäid alati apostlite õpetusse ja osadusse ja leivamurdmisse ja palvetesse.
\par 43 Aga igale hingele tuli kartus; ja palju imesid ja tunnustähti sündis apostlite läbi.
\par 44 Aga kõik need, kes uskusid, olid üheskoos ja kõik, mis neil oli, oli neil ühine.
\par 45 Omandi ja vara nad müüsid ära ja jagasid, mis saadi, igaühele sedamööda, kuidas keegi vajas.
\par 46 Ja nad viibisid iga päev ühel meelel pühakojas ja murdsid leiba kodasid mööda ja võtsid toidust ilutsemise ja siira südamega,
\par 47 kiites Jumalat ja leides armastust kõige rahva juures. Aga Issand lisas iga päev nende ühendusele juurde neid, kes päästeti.


\chapter{3}

\section*{Peetrus teeb terveks halvatud mehe}

\par 1 Aga Peetrus ja Johannes läksid pühakotta palvusele, mida peeti üheksandal tunnil.
\par 2 Ja üks mees, kes oli emaihust alates halvatu, toodi sinna. Ta pandi iga päev pühakoja ukse ette, mida hüütakse Ilusaks, ande paluma pühakojas käijailt.
\par 3 Kui see nägi, et Peetrus ja Johannes tahavad minna pühakotta, palus ta neilt andi.
\par 4 Siis Peetrus ühes Johannesega vaatas temale otsa ning ütles: „Vaata meie poole!”
\par 5 Tema jäi teraselt neile otsa vaatama, oodates, et ta neilt midagi saab.
\par 6 Aga Peetrus ütles: „Hõbedat ja kulda mul ei ole, aga mis mul on, seda ma annan sulle: Jeesuse Kristuse, Naatsaretlase nimel, tõuse üles ja kõnni!”
\par 7 Ja ta haaras kinni tema paremast käest ja aitas ta üles. Ja sedamaid said tema jalapöiad ja luupeksed tugevaks;
\par 8 ja ta kargas üles, seisis ja kõndis, ja läks ühes nendega pühakotta sisse, kõndides ja hüpates ning Jumalat kiites.
\par 9 Ja kõik rahvas nägi teda kõndivat ja Jumalat kiitvat.
\par 10 Nad tundsid tema, et ta on seesama, kes oli istunud kerjamas pühakoja Ilusa värava ees, ja nad panid seda väga imeks ja olid hämmastunud sellest, mis temale oli sündinud.
\par 11 Aga kui halvatu, kes oli terveks saanud, käis Peetruse ja Johannese kannul, jooksis kõik rahvas kokku nende juurde nõndanimetatud Saalomoni võlvitud hoonesse ja oli ärevil.

\section*{Peetruse jutlus pühakojas}

\par 12 Kui Peetrus seda nägi, hakkas ta kõnelema rahvale: „Iisraeli mehed, miks te panete seda imeks või miks te üksisilmi vaatate meie peale, otsekui oleksime meie oma väe või vagadusega selle pannud kõndima?
\par 13 Aabrahami ja Iisaki ja Jaakobi Jumal, meie esiisade Jumal, on austanud oma sulast Jeesust, kelle te ära andsite ja salgasite Pilaatuse ees, kui see tegi otsuseks teda vabaks lasta.
\par 14 Te salgasite ära Püha ja Õige ja palusite, et teile mõrtsukas vabaks antaks.
\par 15 Te tapsite elu ülima juhi, kelle Jumal on surnuist üles äratanud; selle tunnistajad oleme meie.
\par 16 Ja usu läbi tema nimesse on tema nimi teinud tugevaks selle, keda te näete ja tunnete; ja usk, mis tuleb temalt, on sellele andnud täie tervise teie kõikide nähes.
\par 17 Ja nüüd, vennad, tean ma, et te olete seda teinud teadmatusest nagu teie ülemadki.
\par 18 Aga Jumal on nõnda lasknud täide minna, mis tema kõigi oma prohvetite suu kaudu on ette kuulutanud, et tema võitu peab kannatama.
\par 19 Sellepärast parandage meelt ja pöörduge, et teie patud kustutataks, et hingamiseajad tuleksid Issanda palgest
\par 20 ja et ta läkitaks teie jaoks määratud Kristuse Jeesuse,
\par 21 keda taevas peab pidama enesele selle ajani, mil oma kohale asetatakse kõik, mis Jumal on rääkinud kõigi oma pühade prohvetite suu kaudu maailma ajastust alates.
\par 22 Sest Mooses on öelnud: „Ühe prohveti äratab teile Issand, teie Jumal, teie vendade hulgast, minu sarnase; teda peate kuulama kõiges, mida ta teile ütleb.
\par 23 Ja peab sündima, et iga hing, kes seda prohvetit ei kuula, kaotatakse ära rahva seast!”
\par 24 Ja kõik prohvetid Saamueli ajast alates ja pärast seda, nii mitu kui neid on rääkinud, on ka ette kuulutanud neid päevi.
\par 25 Teie olete prohvetite ja selle lepingu lapsed, mille Jumal tegi teie esiisadega, kui ta Aabrahamile ütles: „Ja sinu soos õnnistatakse kõiki suguvõsasid maa peal!”
\par 26 Teile kõigepealt on Jumal üles äratanud oma sulase Jeesuse ja on tema läkitanud teie juurde teid õnnistama, kui te pöördute igaüks oma tigedusest!”


\chapter{4}

\section*{Peetrus ja Johannes Suurkohtu ees}

\par 1 Aga kui nad rahvale rääkisid, astusid nende juurde preestrid ja pühakoja pealik ja saduserid,
\par 2 ja nende meel oli paha, et nemad õpetasid rahvast ja kuulutasid surnuist ülestõusmist Jeesuses.
\par 3 Ja nad pistsid käed nende külge ja panid nad vangi järgmiseks päevaks, sest oli juba õhtu.
\par 4 Aga paljud neist, kes Sõna kuulsid, uskusid; ja meeste arv tõusis ligi viie tuhandeni.
\par 5 Järgmisel päeval sündis, et rahva ülemad ja vanemad ja kirjatundjad tulid kokku Jeruusalemma,
\par 6 ka ülempreester Annas ja Kaifas ja Johannes ja Aleksandros ja niipalju kui neid oli ülempreestri soost.
\par 7 Ja nad lasksid nad ette tuua ning küsitlesid neid: „Missuguse väega või kelle nimel te tegite seda?”
\par 8 Siis Peetrus, täis Püha Vaimu, ütles neile: „Rahva ülemad ja vanemad!
\par 9 Kui me täna peame kohtus aru andma vigasele inimesele tehtud heateost ja kelle läbi ta on terveks saanud,
\par 10 siis olgu teile kõigile ja kogu Iisraeli rahvale teada, et Jeesuse Kristuse, Naatsaretlase nimel, kelle teie risti lõite ja kelle Jumal on surnuist üles äratanud, et tema läbi seisab see siin teie ees tervena.
\par 11 Tema on see kivi, mille teie, kojaehitajad, olete ära põlanud ja mis on saanud nurgakiviks!
\par 12 Ja ühegi muu sees ei ole päästet; sest ei ole antud taeva all inimestele ühtki muud nime, kelles meid päästetakse!”
\par 13 Aga kui nad Peetruse ja Johannese julgust nägid ja teada said, et nad on kirjatundmatud ja õppimatud mehed, panid nad seda imeks ja tundsid nad ära, et nad olid need, kes olid olnud Jeesusega.
\par 14 Ja kui nad inimest, kes oli terveks saanud, nende juures nägid seisvat, ei olnud neil midagi öelda selle vastu,
\par 15 vaid nad käskisid neid välja minna Suurkohtust, pidasid isekeskis nõu
\par 16 ning ütlesid: „Mis me peame tegema nende inimestega, sest avalik ime on sündinud nende läbi ja on teada kõigile, kes Jeruusalemmas elavad, nõnda et me ei või seda salata.
\par 17 Aga et seda laiemale ei levitataks rahva sekka, siis ähvardagem neid kõvasti, et nad enam ühelegi inimesele ei räägiks sel nimel!”
\par 18 Ja nad kutsusid nad sisse ja keelasid neid midagi rääkimast ja õpetamast Jeesuse nimel.
\par 19 Aga Peetrus ja Johannes kostsid neile ning ütlesid: „Kas on õige Jumala ees teid rohkem kuulata kui Jumalat? Otsustage ise.
\par 20 Sest me ei või jätta rääkimata seda, mida oleme näinud ja kuulnud!”
\par 21 Aga nad ähvardasid neid ja lasksid nad vabaks, sest nad ei leidnud midagi, mille eest neid karistada, rahva pärast, sest kõik kiitsid Jumalat selle eest, mis oli sündinud.
\par 22 Sest juba rohkem kui nelikümmend aastat vana oli inimene, kellele see tervekssaamise tunnustäht oli sündinud.

\section*{Peetrus ja Johannes tulevad tagasi kaaslaste juurde}

\par 23 Kui nad olid vabaks saanud, tulid nad omade juurde ja kuulutasid, mis ülempreestrid ja vanemad neile olid öelnud.
\par 24 Aga kui need seda kuulsid, tõstsid nad ühel meelel häält Jumala poole ja ütlesid: „Issand, sina oled see Jumal, kes on teinud taeva ja maa ja mere ja kõik, mis nende sees on,
\par 25 kes oled Püha Vaimu läbi oma sulase Taaveti suu kaudu öelnud: „Miks paganad möllavad ja rahvad mõtlevad tühja?
\par 26 Ilmamaa kuningad on kokku astunud ja ülemad on kogunenud ühte Issanda vastu ning tema Võitu vastu?”
\par 27 Sest tõepoolest on selles linnas sinu püha sulase Jeesuse vastu, kelle sa oled võidnud, kogunenud ühte paganatega ja Iisraeli rahvaga Heroodes ja Pontius Pilaatus,
\par 28 et teha seda, mis sinu käsi ja nõu oli enne määranud, et see pidi sündima.
\par 29 Ja nüüd vaata, Issand, nende ähvardamistele ja anna oma sulastele kõige julgusega rääkida sinu sõna,
\par 30 oma kätt sirutades selleks, et tervekssaamised ning tunnustähed ja imed sünniksid sinu püha sulase Jeesuse nime läbi!”
\par 31 Ja kui nad olid palvetanud, kõikus paik, kus nad koos olid, ning nad kõik said täis Püha Vaimu ja rääkisid Jumala sõna julgesti.

\section*{Ühine varandus}

\par 32 Ja usklike hulk oli üks süda ja üks hing; ja ükski ei öelnud oma varanduse kohta, et see on tema oma, vaid kõik oli neil ühine.
\par 33 Ja apostlid tunnistasid suure väega Issanda Jeesuse ülestõusmist, ja suur arm oli nende kõikide juures.
\par 34 Ei olnud ka ühtki vaest nende seas; sest kellel olid põllud või majad, need müüsid need ära ning tõid müüdud asjade hinna
\par 35 ja panid selle apostlite jalgade ette. Igaühele jagati siis sedamööda, kuidas ta vajas.
\par 36 Aga Joosep, keda apostlid lisanimega hüüdsid Barnabaseks - see on meie keeli Trööstija - leviit, Küprose saarelt pärit,
\par 37 kellel oli põld, müüs selle ära ja tõi hinna ja pani selle apostlite jalgade ette.


\chapter{5}

\section*{Ananias ja Safiira}

\par 1 Aga keegi mees, Ananias nimi, oma naise Safiiraga müüs oma varanduse
\par 2 ja mees toimetas oma naise teades muist hinda kõrvale ja muist ta tõi ning pani apostlite jalgade ette.
\par 3 Aga Peetrus ütles: „Ananias, mispärast on saatan täitnud su südame, et sa valet rääkisid Pühale Vaimule ja kõrvale toimetasid muist põllu hinda?
\par 4 Eks see su käes olles olnud sinu oma, ja kui see oli müüdud, eks see olnud ka siis sinu meelevallas? Mispärast oled sa seda ette võtnud oma südames? Sina ei ole valetanud inimestele, vaid Jumalale!”
\par 5 Kui nüüd Ananias kuulis neid sõnu, langes ta maha ja heitis hinge. Ja suur kartus tuli kõikide peale, kes seda kuulsid.
\par 6 Aga noored mehed tõusid ning koristasid ta ära ja kandsid ta välja ning matsid maha.
\par 7 Arvata kolme tunni pärast tuli tema naine ega teadnud, mis oli sündinud.
\par 8 Aga Peetrus küsis talt: „Ütle mulle, kas te selle hinnaga müüsite põllu?” Tema vastas: „Selle hinnaga jah!”
\par 9 Siis ütles Peetrus talle: „Mispärast on teil üks nõu olnud kiusata Issanda Vaimu? Vaata, nende jalad, kes su mehe matsid, on ukse ees ja kannavad sindki välja!”
\par 10 Siis langes naine kohe maha tema jalge ette ja heitis hinge. Ja kui noored mehed tulid sisse, leidsid nad ta surnud olevat ja kandsid ta välja ning matsid ta tema mehe kõrvale.
\par 11 Ja suur kartus tuli kõige koguduse peale ja kõikide peale, kes seda kuulsid.

\section*{Apostlid teevad tunnustähti ja imesid}

\par 12 Aga palju tunnustähti ja imesid sündis rahva seas apostlite käte läbi. Ja nad olid kõik koos ühel meelel Saalomoni võlvitud hoones.
\par 13 Muudest inimestest aga ei julgenud ükski liituda nende hulka, rahvas aga pidas neist suurt lugu.
\par 14 Seda enam aga lisandus neid, kes uskusid Issandasse, nii mehi kui naisi suurel hulgal!
\par 15 Kandsid ju inimesed haigeid tänavailegi ja asetasid neid vooditesse ja raamidele, et kui Peetrus tuleb, tema vari langeks mõne peale nendest.
\par 16 Ka tuli rahvast kokku Jeruusalemma ümberkaudseist linnadest, tuues haigeid ja rüvedaist vaimudest vaevatuid, ja need kõik said terveks.

\section*{Peetrus ja Johannes taas Suurkohtu ees}

\par 17 Aga ülempreester tõusis ning kõik ta kaaslased, saduseride lahk, ning said täis viha,
\par 18 ja pistsid oma käed apostlite külge ning panid nad üldisesse vanglasse.
\par 19 Aga Issanda ingel tegi öösel vangihoone uksed lahti ja viis nad välja ning ütles:
\par 20 „Minge, esinege ning rääkige rahvale pühakojas kõik selle elu sõnad!”
\par 21 Kui nad seda olid kuulnud, läksid nad koidu ajal pühakotta ja õpetasid. Aga ülempreester ja tema kaaslased tulid ja kutsusid kokku Suurkohtu ja kõik Iisraeli laste vanematekogu ja läkitasid sulased vangitorni neid tooma.
\par 22 Kui nüüd sulased kohale jõudsid, ei leidnud nad neid vangihoonest. Nad tulid tagasi ja andsid seda teada,
\par 23 öeldes: „Vangla me leidsime küll hästi hoolsasti lukustatud olevat ja hoidjad väljas uste ees seisvat, aga kui me avasime, ei leidnud me kedagi seest!”
\par 24 Kui pühakoja pealik ja ülempreestrid neid sõnumeid kuulsid, jäid nad kahevahele, mis sellest arvata.
\par 25 Siis tuli keegi ja jutustas neile: „Vaata, mehed, keda te panite vanglasse, seisavad pühakojas ja õpetavad rahvast!”
\par 26 Siis läks pealik sulastega ja tõi nad ära; aga mitte vägisi, sest nad kartsid, et rahvas nad kividega surnuks viskab.
\par 27 Ja nad tõid ning seadsid nad Suurkohtu ette. Ja ülempreester küsis neilt ning ütles:
\par 28 „Kas me teid ei ole kõvasti keelanud õpetamast selle nimel? Ja vaata, te olete Jeruusalemma täitnud oma õpetusega ja tahate selle inimese vere saata meie peale!”
\par 29 Aga Peetrus ja teised apostlid kostsid ning ütlesid: „Jumala sõna tuleb rohkem kuulda kui inimeste sõna.
\par 30 Meie esiisade Jumal on üles äratanud Jeesuse, kelle te olete poonud puu külge ning surmanud.
\par 31 Tema on Jumal oma parema käega tõstnud Juhiks ja Õnnistegijaks, andma Iisraelile meeleparandust ja pattude andeksandmist.
\par 32 Ja meie oleme kõigi nende asjade tunnistajad ja samuti Püha Vaim, keda Jumal on andnud neile, kes võtavad kuulda tema sõna!”
\par 33 Aga kui nad seda kuulsid, käis see nende südamest läbi, ja nad võtsid nõuks nemad ära tappa.
\par 34 Kuid Suurkohtus tõusis üles üks variser, Gamaaliel nimi, käsuõpetaja, keda kõik rahvas austas, ja käskis mehed välja viia natukeseks ajaks,
\par 35 ja ütles siis koosolijaile: „Iisraeli mehed, olge ettevaatlikud sellega, mida te nende inimestega mõtlete teha.
\par 36 Sest mitte ammu tagasi tõusis Teudas ning ütles enese midagi olevat, ja temaga liitus arvult ligi nelisada meest; tema tapeti ja kõik, kes nõustusid temaga, hajutati ja hävisid.
\par 37 Pärast teda astus üles galilealane Juudas rahvalugemise päevil ja ahvatles rahva enese järele; ka tema sai hukka ja kõik, kes heitsid tema nõusse, hajutati.
\par 38 Ja nüüd ma ütlen teile: jätke need inimesed rahule ja laske nad minna! Sest kui see nõu ja töö on inimestest, siis läheb see tühja,
\par 39 aga kui see on Jumalast, siis te ei või seda tühjaks teha, et te kuidagi ei osutuks jumalavastasteks!” Ja nad võtsid kuulda tema nõu.
\par 40 Ja kui apostlid olid sisse kutsutud, peksid nad neid ja keelasid neid rääkimast Jeesuse nimel ja lasksid nad minna.
\par 41 Aga nemad läksid Suurkohtu eest minema, rõõmsad sellest, et neid oli väärt arvatud kannatama teotust selle nime pärast.
\par 42 Ja nad ei lakanud iga päev pühakojas ja kodasid mööda õpetamast evangeeliumi Kristusest Jeesusest.


\chapter{6}

\section*{Hoolekandjad seatakse ametisse}

\par 1 Aga neil päevil kui jüngrite arv suurenes, tõusis nurin kreekakeelsete juutide seas heebrealaste vastu, et igapäevases abiandmises õieti ei hoolitud nende leskedest.
\par 2 Siis kutsusid need kaksteist eneste juurde jüngrite hulga ning ütlesid: „See ei sobi, et me kõrvale jätame Jumala sõna ja hoolt kanname toidulaudade eest.
\par 3 Sellepärast, vennad, vaadake endi seast seitse meest, kellel on hea tunnistus ja kes on täis Vaimu ja tarkust, kelle me seaksime sellesse ametisse.
\par 4 Meie aga tahame pidevalt olla palvetamises ja sõna teenistuses!”
\par 5 See kõne oli kogu rahvahulga meelt mööda. Ja nad valisid Stefanose, mehe täis usku ja Püha Vaimu, ja Filippuse ja Prokorose ja Nikaanori ja Timoni ja Parmena ja Nikolaose, juudiusku pöördunud antiohlase.
\par 6 Need nad seadsid apostlite ette ja palvetasid ning panid oma käed nende peale.
\par 7 Ja Jumala sõna kasvas; ja jüngrite arv sai väga suureks Jeruusalemmas; ja suur hulk preestreid sai sõnakuulelikuks usule.

\section*{Süüdistus Stefanose vastu}

\par 8 Aga Stefanos, täis armu ja väge, tegi imetegusid ja suuri tunnustähti rahva seas.
\par 9 Siis tõusid nõndanimetatud libertiinide ja küreenlaste ja aleksandrialaste kogudustest ja Kiliikia ja Aasia poolt mõned, kes vaidlesid Stefanosega,
\par 10 aga ei suutnud vastu seista tarkusele ja Vaimule, kelle läbi tema rääkis.
\par 11 Siis nad hankisid mehed, kes ütlesid: „Me oleme teda kuulnud pilkesõnu rääkivat Moosese ja Jumala vastu!”
\par 12 Ja nemad õhutasid rahvast ja vanemaid ja kirjatundjaid ja langesid ta kallale ning võtsid ta kinni ja viisid ta Suurkohtu ette.
\par 13 Nad esitasid valetunnistajaid, kes ütlesid: „See inimene ei lakka rääkimast sõnu selle püha paiga ja käsuõpetuse vastu.
\par 14 Sest me oleme kuulnud teda ütlevat, et Jeesus Naatsaretlane hävitavat selle paiga ja muutvat kombed, mis Mooses meile on andnud.”
\par 15 Ja kui kõik Suurkohtus istujad temale otsa vaatasid, nägid nad tema palge olevat otsekui ingli palge.


\chapter{7}

\section*{Stefanose jutlus kohtu ees}

\par 1 Siis ütles ülempreester: „Kas see on nõnda?”
\par 2 Aga tema ütles: „Mehed, vennad ja isad, kuulge! Au Jumal ilmus meie esiisale Aabrahamile, kui ta oli Mesopotaamias enne oma siirdumist Haaranisse,
\par 3 ning ütles temale: „Mine välja omalt maalt ja omast sugukonnast ja tule sinna maale, mille ma sulle näitan.”
\par 4 Siis ta läks kaldealaste maalt ja asus elama Haaranisse. Ja kui tema isa suri, saatis Jumal tema sealt siia maale, kus teie nüüd elate,
\par 5 ega andnud siin temale pärandiks mitte jalatäitki, aga tõotas selle anda omandiks temale ja tema järglasile pärast teda ajal, mil ta oli alles lasteta.
\par 6 Ent Jumal rääkis nõnda: tema järglased peavad olema võõrastena võõral maal ja seal nad tehakse orjadeks ja neid vaevatakse nelisada aastat,
\par 7 ja rahvast, keda nad orjavad, mina karistan, ütles Jumal, ja selle järel nad lähevad välja ja teenivad mind siin selles paigas.
\par 8 Ja ta andis temale ümberlõikamislepingu; ja nii sündis Aabrahamile Iisak ja ta lõikas tema ümber kaheksandal päeval. Ja Iisakile sündis Jaakob, ja Jaakobile need kaksteist peavanemat.
\par 9 Ja peavanemad kadestasid Joosepit ning müüsid ta ära Egiptusesse; aga Jumal oli temaga
\par 10 ja päästis tema kõigist ta viletsustest ja andis temale armu ja tarkust, kui ta oli Egiptuse kuninga, vaarao ees; ja see tõstis ta kogu Egiptuse ja oma koja ülemaks.
\par 11 Aga nälg tuli kogu Egiptusesse ja Kaananisse ja suur viletsus; ja meie esiisad ei leidnud toidust.
\par 12 Aga kui Jaakob kuulis, et Egiptuses oli vilja, läkitas ta meie esiisad sinna esimest korda;
\par 13 ja teisel korral andis Joosep ennast tunda oma vendadele ja vaarao sai teada Joosepi päritolu.
\par 14 Siis Joosep läkitas järele ja laskis kutsuda oma isa Jaakobi ja kogu oma suguvõsa, seitsekümmend viis hinge.
\par 15 Ja Jaakob läks alla Egiptusesse ja suri, tema ja meie esiisad,
\par 16 ja nad viidi Sekemisse ja pandi hauda, mille Aabraham oli ostnud raha eest Emori, Sekemi isa lastelt.
\par 17 Kui nüüd lähenes tõotuse aeg, mille Jumal Aabrahamile oli vandega tõotanud, kasvas rahvas ja sigis paljuks Egiptuses,
\par 18 seni kui tõusis teine kuningas, kes Joosepist midagi ei teadnud.
\par 19 See tarvitas kavalust meie soo kallal ja tegi paha meie esivanemaile neid sundides ära heitma oma lapsukesed, et need ei jääks ellu.
\par 20 Sel ajal sündis Mooses ja ta meeldis Jumalale. Teda toideti kolm kuud tema isa majas.
\par 21 Aga kui ta oli kõrvale heidetud, võttis vaarao tütar tema ja kasvatas ta enesele pojaks.
\par 22 Ja Moosesele õpetati kõike egiptlaste tarkust ja tema oli vägev sõnades ja tegudes.
\par 23 Aga kui ta oli nelikümmend aastat vanaks saanud, tuli tema südamesse mõte oma vendi, Iisraeli lapsi, vaatama minna.
\par 24 Ja nähes ühele ülekohut tehtavat, läks ta appi ja tasus kätte selle eest, kellele liiga tehti, ja lõi egiptlase maha.
\par 25 Ent ta mõtles, et tema vennad saavad sellest aru, et Jumal tema käe läbi annab neile pääste; aga nad ei saanud aru.
\par 26 Ja järgmisel päeval ta juhtus nende juurde, kui nad riidlesid, ja sundis neid rahule ning ütles: „Mehed, te olete vennad, miks teete liiga üksteisele?”
\par 27 Aga see, kes tegi liiga oma ligimesele, lükkas ta enesest ära ja ütles: „Kes on sind seadnud meie ülemaks ja kohtumõistjaks?
\par 28 Kas tahad tappa mind, nagu sa eile tapsid egiptlase?”
\par 29 Selle kõne peale Mooses põgenes ja elas võõrana Midjanimaal; seal sündis temale kaks poega.
\par 30 Ja kui nelikümmend aastat möödas oli, ilmus temale Siinai mäe kõrbes Issanda ingel tuleleegis kibuvitsapõõsas.
\par 31 Kui Mooses seda nägi, pani ta nägemust imeks. Ja kui ta ligi astus vaatama, kostis Issanda hääl:
\par 32 „Mina olen sinu vanemate Jumal, Aabrahami Jumal ja Iisaki Jumal ja Jaakobi Jumal!” Aga Mooses hakkas värisema ega julgenud sinna vaadata.
\par 33 Aga Issand ütles temale: „Võta kingad jalast, sest paik, kus sa seisad, on püha maa!
\par 34 Mina olen küllalt näinud oma rahva vaeva Egiptuses ja olen kuulnud nende ohkamist ning olen tulnud alla neid vabastama. Siis tule nüüd, ma läkitan sind Egiptusesse!”
\par 35 Selle Moosese, kelle nemad ära salgasid, öeldes: „Kes on sind seadnud ülemaks ja kohtumõistjaks?”, selle läkitas Jumal ülemaks ja lunastajaks selle ingli kaudu, kes temale ilmus kibuvitsapõõsas.
\par 36 See viis nad välja ja tegi imetegusid ja tunnustähti Egiptuses ja Punases meres ja kõrbes nelikümmend aastat.
\par 37 See on see Mooses, kes ütles Iisraeli lastele: „Ühe prohveti äratab teile Jumal teie vendade hulgast, minu sarnase”,
\par 38 tema on see, kes oli koguduse seas kõrbes ühes selle ingliga, kes temaga kõneles Siinai mäel, ja oli ühes meie esiisadega ja sai elavaid sõnu meile edasiandmiseks.
\par 39 Temale meie esivanemad ei tahtnud olla sõnakuulelikud, vaid tõukasid ta ära ja pöördusid oma südamega Egiptuse poole,
\par 40 öeldes Aaronile: „Tee meile jumalaid, kes käiksid meie ees, sest me ei tea, mis on sündinud selle Moosesega, kes meid tõi Egiptusest välja!”
\par 41 Ja nad tegid noil päevil vasika ja viisid sellele ebajumalale ohvreid ning rõõmustusid oma käte tegudest.
\par 42 Kuid Jumal pöördus neist ära ja jättis nad teenima taeva vägesid, nõnda nagu on kirjutatud prohvetite raamatus: „Kas teie, Iisraeli sugu, nelikümmend aastat kõrbes tõite mulle tapaohvreid ja muid ohvreid?
\par 43 Ei, vaid te kandsite Mooloki telki ja oma jumala Romfa tähte, kujusid, mis te olite teinud, et neid kummardada: sellepärast ma siirutan teid teispoole Baabüloni!”
\par 44 Meie esiisadel oli tunnistustelk kõrbes, nõnda nagu oli määranud see, kes Moosesega rääkis, et ta pidi selle valmistama eeskuju järgi, mida ta oli näinud.
\par 45 Selle meie esiisad ühes Joosuaga võtsid ning viisid paganate maa-alale, keda Jumal ajas minema meie esiisade eest kuni Taaveti päevini,
\par 46 kes leidis armu Jumala ees ja palus, et ta leiaks elamu Jaakobi kodakonnale.
\par 47 Ent Saalomon ehitas temale koja.
\par 48 Kuid Kõigekõrgem ei ela kätega tehtud kodades, nõnda nagu ütleb prohvet:
\par 49 „Taevas on minu aujärg ja maa minu jalgealune järg! Mis koda te mulle tahate ehitada”, ütleb Issand, „või mis on mu hingamise ase?
\par 50 Eks minu käsi ole selle kõik teinud?”
\par 51 Te kangekaelsed ja ümberlõikamatud südamest ja kõrvust! Te panete ikka vastu Pühale Vaimule! Nõnda nagu teie esiisad, nõnda teete teiegi.
\par 52 Keda prohvetitest teie esiisad ei ole taga kiusanud? Nad tapsid need, kes ennustasid selle Õige tulemist, kelle äraandjaiks ja tapjaiks nüüd olete saanud teie,
\par 53 kes saite käsuõpetuse inglite teenistuse kaudu ega pidanud seda mitte.

\section*{Stefanose surm}

\par 54 Aga kui nad seda kuulsid, lõikas see neile südamesse ja nad kiristasid hambaid ta peale.
\par 55 Tema aga täis Püha Vaimu vaatas üksisilmi taeva poole ja nägi Jumala auhiilgust ning Jeesust seisvat Jumala paremal poolel,
\par 56 ja ta ütles: „Ennäe! Ma näen taevad lahti olevat ja Inimese Poja seisvat Jumala paremal poolel!”
\par 57 Siis nad karjusid suure häälega ning pidasid oma kõrvad kinni ja kargasid kõik ühel meelel tema kallale
\par 58 ning lükkasid ta linnast välja ja viskasid teda kividega. Ja tunnistajad panid oma riided maha ühe noore mehe jalgade ette, kelle nimi oli Saulus.
\par 59 Ja nad viskasid Stefanost kividega, kuna tema palus Jumalat ja ütles: „Issand Jeesus, võta minu vaim vastu!”
\par 60 Ja ta heitis põlvili ning hüüdis suure häälega: „Issand, ära arva seda pattu neile süüks!” Ja kui ta seda oli öelnud, uinus ta.


\chapter{8}

\section*{Jeruusalemma koguduse tagakiusamine ja hajutamine}

\par 1 Aga Saulusel oli hea meel Stefanose surmast. Sel ajal algas suur Jeruusalemma koguduse tagakiusamine. Ja kõik peale apostlite hajutati Juuda- ja Samaariamaale.
\par 2 Aga mõned jumalakartlikud mehed matsid Stefanose maha ja leinasid väga tema pärast.
\par 3 Ent Saulus rüüstas kogudust, käies mööda kodasid, ja tõi välja mehi ja naisi ja andis nad vangi.
\par 4 Siis käisid hajaliolevad mööda maad ja kuulutasid evangeeliumi sõna.

\section*{Filippus Samaarias}

\par 5 Aga Filippus tuli Samaaria linna ja kuulutas neile Kristust.
\par 6 Ja rahvas pani ühel meelel seda tähele, mida Fifippus ütles, kuuldes teda ning nähes imetähti, mida ta tegi.
\par 7 Sest paljude seest, kellel olid rüvedad vaimud, läksid need välja suure häälega kisendades, ja palju halvatuid ja jalutuid sai terveks.
\par 8 Ja suur rõõm oli selles linnas.

\section*{Nõid Siimon}

\par 9 Aga üks mees, Siimon nimi, oli enne juba seal linnas ja nõidus ning eksitas Samaaria rahvast, öeldes enese midagi suurt olevat.
\par 10 Teda panid tähele kõik, nii pisukesed kui suured, ning ütlesid: „See on Jumala vägi, mida kutsutakse Suureks!”
\par 11 Ja nad hoidsid tema poole, sest et ta kaua aega oli neid nõidusega eksitanud.
\par 12 Aga kui nad Filippust uskusid, kes neile kuulutas evangeeliumi Jumala riigist ja Jeesuse Kristuse nimest, siis ristiti nii mehi kui naisi.
\par 13 Ka Siimon ise sai usklikuks, ja kui ta oli ristitud, jäi ta alati Filippuse juurde ja pani väga imeks, nähes sündivat imetähti ja suuri vägevaid tegusid.

\section*{Peetrus ja Johannes Samaarias}

\par 14 Aga kui Jeruusalemmas olevad apostlid said kuulda, et Samaaria linn oli Jumala sõna vastu võtnud, läkitasid nad nende juurde Peetruse ja Johannese.
\par 15 Kui need sinna said, palvetasid nad nende eest, et nad saaksid Püha Vaimu.
\par 16 Sest Vaim ei olnud veel langenud ühegi peale nende seast, vaid nemad olid ainult ristitud Issanda Jeesuse nimesse.
\par 17 Siis panid nad oma käed nende peale, ja nad said Püha Vaimu.
\par 18 Aga kui Siimon nägi, et apostlite käte pealepanemise läbi anti Püha Vaimu, pakkus ta neile raha
\par 19 ning ütles: „Andke ka minule see meelevald, et see, kelle peale ma iganes oma käed panen, saaks Püha Vaimu.”
\par 20 Aga Peetrus ütles talle: „Kadugu su raha sinuga tükkis, et sa mõtled Jumala andi saada raha eest!
\par 21 Sul ei ole jagu ega osa sellest sõnast, sest su süda ei ole avameelne Jumala ees.
\par 22 Sellepärast paranda meelt sellest oma kurjusest ja palu Issandat, et sinu südame mõte sulle vahest andeks antaks.
\par 23 Sest ma näen sind viha sapis ja ülekohtu sõlmes kinni olevat!”
\par 24 Aga Siimon vastas ning ütles: „Paluge teie Issandat minu pärast, et minu peale ei tuleks midagi sellest, mis te olete öelnud!”
\par 25 Kui nad nüüd olid tunnistanud ja rääkinud Issanda sõna, pöördusid nad tagasi Jeruusalemma ja kuulutasid evangeeliumi mitmes Samaariamaa alevis.

\section*{Filippus ja Etioopia kojaülem}

\par 26 Aga Issanda ingel rääkis Filippusega ja ütles: „Tõuse ja mine lõuna poole seda teed, mis Jeruusalemmast läheb alla Gaasa poole, see on kõrbetee!”
\par 27 Ja ta tõusis ning läks. Ja vaata, üks Etioopia mees, etiooplaste kuninganna Kandake suur kojaülem, kogu ta varanduse hooldaja, kes oli tulnud Jeruusalemma Jumalat kummardama,
\par 28 oli tagasi minemas ning istus oma tõllas ja luges prohvet Jesaja raamatut.
\par 29 Ja Vaim ütles Filippusele: „Mine ja astu selle tõlla lähedale!”
\par 30 Aga kui Filippus jooksis tõlla juurde, kuulis ta teda lugevat prohvet Jesaja raamatut ja küsis: „Kas sa ka mõistad, mida sa loed?”
\par 31 Tema ütles: „Kuidas ma võin mõista, kui keegi mind ei juhata?” Ja ta palus Filippust astuda tõlda ja istuda tema kõrvale.
\par 32 Ja kirjakoht, mida ta luges, oli see: „Nagu lammas viidi ta tappa, ja nagu tall oma niitja ees on vait, nõnda ei avanud ta oma suud!
\par 33 Teda alandades võeti kohus ta pealt! Kes kõneleb tema päritolust? Sest ta elu võeti ära maa pealt!”
\par 34 Aga kojaülem hakkas rääkima Filippusega ning ütles: „Ma palun sind, kellest räägib prohvet seda? Kas enesest või kellestki teisest?”
\par 35 Aga Filippus avas oma suu ja lähtudes sellest kirjakohast ta kuulutas temale evangeeliumi Jeesusest.
\par 36 Ent kui nad teed edasi läksid, jõudsid nad vee juurde ja kojaülem ütles: „Ennäe vett! Mis keelab, et mind ei peaks ristitama?”
\par 37 [Aga Filippus ütles: „Kui sa usud kõigest südamest, siis võib see sündida!” Tema vastas ning ütles: „Mina usun, et Jeesus Kristus on Jumala Poeg!”]
\par 38 Ja ta käskis tõlla peatada, ja nad astusid mõlemad maha vette, Filippus ja kojaülem; ja ta ristis teda.
\par 39 Aga kui nad veest välja tulid, võttis Issanda Vaim Filippuse ära, ja kojaülem ei näinud teda enam. Aga ta läks oma teed rõõmuga.
\par 40 Aga Filippus leiti Asdodist ja ta käis mööda maad ning kuulutas evangeeliumi kõigile linnadele, kuni ta jõudis Kaisareasse.


\chapter{9}

\section*{Sauluse pöördumine}

\par 1 Aga Saulus turtsus ikka veel ähvardamise ja tapmisega Issanda jüngrite vastu ja läks ülempreestri juurde
\par 2 ning palus temalt kirju Damaskuse kogudustele, et kui ta peaks leidma usuteel käijaid, olgu mehi või naisi, et ta võiks nad siduda ja tuua Jeruusalemma.
\par 3 Aga kui ta oli sinna minemas ja Damaskuse lähedale jõudis, sündis, et äkitselt paistis tema ümber valgus taevast;
\par 4 ja ta kukkus maha maa peale ning kuulis üht häält temale ütlevat: „Saul, Saul, miks sa mind taga kiusad?”
\par 5 Tema vastas: „Issand, kes sa oled?” Ja Issand ütles: „Mina olen Jeesus, keda sa taga kiusad!
\par 6 Aga tõuse ja mine linna, ja seal öeldakse sulle, mis sul tuleb teha!”
\par 7 Aga mehed, kes olid temaga teel, seisid ehmunult; nad kuulsid küll häält, aga ei näinud kedagi.
\par 8 Aga Saulus tõusis maast üles ja kui ta oma silmad avas, ei näinud ta enam! Siis nad talutasid teda kättpidi ja viisid ta Damaskusesse.
\par 9 Ja kolm päeva ta ei näinud, ei söönud ega joonud!

\section*{Saulus Damaskuses}

\par 10 Ent Damaskuses oli üks jünger, Ananias nimi. Sellele ütles Issand nägemuses: „Ananias!” Tema vastas: „Issand, vaata, siin ma olen!”
\par 11 Issand ütles temale: „Tõuse ja mine sinna tänavasse, mida kutsutakse Õigeks, ja kuula Juuda majas Sauluse-nimelise Tarsose mehe järele, sest vaata, ta palvetab
\par 12 ja on nägemuses näinud mehe, Ananias nimi, sisse tulevat ja käe ta peale panevat, et ta nägemise tagasi saaks!”
\par 13 Ananias vastas: „Issand, ma olen mitmelt kuulnud sellest mehest, kui palju ta on kurja teinud sinu pühadele Jeruusalemmas;
\par 14 ja tal on ülempreestrilt luba siin siduda kõiki, kes sinu nime appi hüüavad!”
\par 15 Aga Issand ütles temale: „Mine, sest ta on mulle valitud riist, et ta minu nime kannaks paganate ja kuningate ning Iisraeli laste ette,
\par 16 sest mina tahan temale näidata, kui palju ta peab minu nime pärast kannatama!”
\par 17 Siis Ananias läks ära ja astus sinna majja. Ja kui ta oma käe tema peale oli pannud, ütles ta: „Saul, vend, Issand on mind läkitanud, Jeesus, kes sulle ilmus teel, mida sa tulid, et sa nägemise tagasi saaksid ja täituksid Püha Vaimuga!”
\par 18 Ja sedamaid pudenes tema silmadelt otsekui soomuseid, ja ta nägi jälle ja tõusis ning laskis ennast ristida.
\par 19 Ja kui ta oli leiba võtnud, sai ta jälle tugevaks. Siis ta jäi mõneks päevaks jüngrite sekka Damaskusesse,
\par 20 ja kuulutas varsti kogudusekodades Kristust, tunnistades, et seesama on Jumala Poeg.
\par 21 Aga kõik, kes teda kuulsid, ehmusid ning ütlesid: „Eks see ole seesama, kes Jeruusalemmas rüüstas selle nime appihüüdjaid ja on siia tulnud selleks, et neid kinni siduda ja viia ülempreestrite kätte?”
\par 22 Aga Saulus läks järjest vägevamaks ja ajas juudid, kes Damaskuses elasid, kihama ning tõestas, et Jeesus on Kristus.
\par 23 Mõne päeva pärast pidasid juudid nõu teda ära tappa.
\par 24 Aga Saulus sai teada nende salanõu. Ent nad valvasid päeva ja öö väravais, et teda tappa;
\par 25 kuid jüngrid võtsid ta öösel ja aitasid ta üle müüri ning lasksid ta korviga alla.

\section*{Saulus Jeruusalemmas ja Tarsoses}

\par 26 Kui siis Saulus jõudis Jeruusalemma, püüdis ta liituda jüngrite hulka, aga nemad kõik kartsid teda ega uskunud, et ta on jünger.
\par 27 Kuid Barnabas võttis ta vastu ja viis ta apostlite juurde ja rääkis neile, kuidas tema teel olles oli Issandat näinud ning temaga kõnelnud ja kuidas ta Damaskuses oli avalikult kuulutanud evangeeliumi Jeesuse nimel.
\par 28 Nii ta jäi nende juurde sisse ja välja käima Jeruusalemmas ja kuulutas avalikult evangeeliumi Issanda nimel.
\par 29 Ta rääkis ka ja ajas juttu kreekakeelsete juutidega; aga need püüdsid teda tappa.
\par 30 Aga kui vennad seda teada said, viisid nad ta Kaisareasse ja saatsid ta sealt Tarsosesse.
\par 31 Siis oli nüüd kogudusel rahu kogu Juuda- ja Galilea- ja Samaariamaal, ja see kasvas ja täienes ja edenes Issanda kartuses ning Püha Vaimu julgustusel.

\section*{Peetrus teeb Lüddas haige terveks}

\par 32 Aga sündis, kui Peetrus kõigis paigus käimas oli, et ta tuli ka pühade juurde, kes elasid Lüddas.
\par 33 Seal ta leidis ühe inimese, Aineas nimi, kes oli kaheksa aastat olnud voodis maas ja halvatud.
\par 34 Ja Peetrus ütles talle: „Aineas! Jeesus Kristus teeb su terveks: tõuse üles ja sea oma ase korda!” Ja sedamaid ta tõusis üles.
\par 35 Ja teda nägid kõik, kes elasid Lüddas ja Saaronis; ja need pöördusid Issanda poole.

\section*{Peetrus äratab ellu naisjüngri Tabiita}

\par 36 Aga Joppes oli keegi naisjünger, nimega Tabiita, mis meie keeli tähendab Hirv. See oli rikas häist tegudest ja armastusandidest, mida ta tegi.
\par 37 Ja neil päevil sündis, et ta jäi haigeks ja suri. Ja nad pesid teda ja panid ta ülemisse tuppa.
\par 38 Aga Lüdda oli Joppe lähedal. Kui jüngrid kuulsid, et Peetrus seal on, läkitasid nad kaks meest tema juurde ja palusid teda, et ta viibimata tuleks nende juurde.
\par 39 Siis Peetrus tõusis ja läks nendega. Ja kui ta sinna jõudis, viisid nad ta ülemisse tuppa, ja kõik lesknaised astusid tema ümber, nutsid ja näitasid kuubi ja riideid, mis Hirv nende juures olles oli teinud.
\par 40 Aga Peetrus ajas kõik välja, heitis põlvili ja palus Jumalat ning pöördus surnu poole ja ütles: „Tabiita, tõuse üles!” Siis avas see oma silmad, ja kui ta Peetrust nägi, tõusis ta istuma.
\par 41 Aga Peetrus andis temale käe ja tõstis ta püsti. Siis ta kutsus sisse pühad ja lesed ja esitas teda neile elavana.
\par 42 Aga see sai teatavaks kogu Joppes ja paljud uskusid Issandasse.
\par 43 Ja Peetrus jäi mõneks päevaks Joppesse ühe nahkru Siimona juurde.


\chapter{10}

\section*{Peetrus ja Korneelius}

\par 1 Kaisarea linnas oli üks mees, Korneelius nimi, selle väesalga pealik, mida kutsuti Itaalia väesalgaks.
\par 2 Tema oli vaga ja jumalakartlik kogu oma perega ja jagas rahvale palju ande ning palus alati Jumalat.
\par 3 See nägi nägemuses selgesti, arvata üheksandal päevatunnil, et Jumala ingel tuli sisse tema juurde ning ütles talle: „Korneelius!”
\par 4 Tema aga vaatas inglile otsa, lõi kartma ning ütles: „Isand, mis on?” See ütles talle: „Sinu palved ja su annid on tõusnud Jumala ette sind meelde tuletama.
\par 5 Läkita nüüd mehi Joppesse ja lase kutsuda Siimon, kelle lisanimi on Peetrus;
\par 6 tema asub nahkur Siimona juures, kelle maja on mererannas!”
\par 7 Kui siis ingel oli ära läinud, kes Korneeliusega rääkis, kutsus ta kaks oma perest ja ühe vaga sõjamehe neist, kes alati ta juures olid,
\par 8 rääkis neile kõik ära ning läkitas nad Joppesse.
\par 9 Aga järgmisel päeval, kui nad teel olles lähenesid linnale, läks Peetrus katusele palvetama kuuendal tunnil.
\par 10 Ja temale tuli nälg ning ta tahtis süüa. Aga kui talle rooga valmistati, jäi ta otsekui enesest ära
\par 11 ning nägi taeva avatud olevat ja enese juurde alla tulevat anuma, otsekui suure linase riide, mida nelja nurka pidi alla lasti.
\par 12 Selle sees oli kõiksugu neljajalgseid ja roomajaid maa elajaid ja taeva linde.
\par 13 Ja temale ütles hääl: „Tõuse, Peetrus, verista ja söö!”
\par 14 Aga Peetrus ütles: „Ei ilmaski, Issand, sest ma pole veel söönud seda, mis on keelatud ja roojane!”
\par 15 Ja hääl ütles taas teist korda temale: „Mis Jumal on puhastanud, seda sina ära arva keelatuks!”
\par 16 Ja see sündis kolm korda, ja astja võeti jälle üles taevasse.
\par 17 Aga kui Peetrus oli iseeneses kahevahel, mis tema nähtud nägemus küll võiks tähendada, vaata, siis olid mehed, keda Korneelius oli läkitanud, küsitledes leidnud Siimona maja ja seisid värava taga.
\par 18 Nad hüüdsid ning küsisid, kas Siimon, kelle lisanimi on Peetrus, on seal võõrsil.
\par 19 Ent kui Peetrus alles oma meeles mõtles nägemusele, ütles Vaim temale: „Vaata, kolm meest otsivad sind!
\par 20 Aga tõuse üles ja astu alla ning mine nendega ilma kaksipidi mõtlemata, sest mina olen nad läkitanud!”
\par 21 Siis Peetrus tuli alla meeste juurde ning ütles: „Vaata, mina olen see, keda te otsite! Mis asja pärast te olete tulnud?”
\par 22 Nemad vastasid: „Pealik Korneelius, õige ja jumalakartlik mees, kellel on hea tunnistus kogu juuda rahvalt, on pühalt inglilt saanud sõna, et ta sind peab kutsuma oma kotta ja kuulama sinu kõnet!”
\par 23 Siis ta kutsus nad sisse ja pidas neid külalistena. Aga järgmisel päeval läks Peetrus nendega ära ja mõned vennad, kes Joppest olid, läksid ühes temaga.
\par 24 Järgmisel päeval nad saabusid Kaisareasse. Ja Korneelius ootas neid ja kutsus kokku oma sulased ja parimad sõbrad.
\par 25 Ja kui Peetrus oli sisse astunud, läks Korneelius talle vastu ning heitis tema jalge ette maha ja kummardas.
\par 26 Aga Peetrus tõstis ta üles ja ütles: „Tõuse üles, ka mina ise olen inimene!”
\par 27 Ja kui ta temaga oli kõnelnud, läks ta sisse ja leidis eest suure hulga inimesi
\par 28 ning ütles neile: „Te teate, et juuda mehel ei ole luba sõbrustada muulasega või minna tema juurde, aga Jumal on mulle näidanud, et ühtki inimest ei tohi pidada halvaks ega rüvedaks.
\par 29 Sellepärast olen ma ka tõrkumata tulnud, kui mind kutsuti. Siis küsin ma nüüd: mis asja pärast te olete mind kutsunud?”
\par 30 Korneelius ütles: „Neli päeva tagasi ma olin otse selsamal üheksandal tunnil oma kojas palvetamas ja vaata, üks mees seisis minu ees hiilgavais riideis
\par 31 ning ütles: Korneelius, sinu palvet on kuuldud ja su annid on Jumalale meelde tulnud.
\par 32 Siis läkita nüüd Joppesse ja kutsu Siimon, keda lisanimega hüütakse Peetruseks; tema on võõrsil nahkur Siimona majas mererannas.
\par 33 Siis ma läkitasin sedamaid sinu juurde, ja sa oled hästi teinud, et tulid. Nüüd me oleme kõik siin üheskoos Jumala ees kuulamas kõike, mis Jumal sind on käskinud!”

\section*{Peetruse jutlus Korneeliuse kojas}

\par 34 Siis Peetrus avas oma suu ning ütles: „Ma mõistan tõesti, et Jumal ei tee vahet isikute vahel,
\par 35 vaid kõige rahva seast on see, kes teda kardab ja teeb õigust, tema meele järgi.
\par 36 Selle sõna ta on läkitanud Iisraeli lastele, kuulutades evangeeliumi rahust Jeesuse Kristuse läbi, kes on kõikide Issand.
\par 37 Te teate seda asja, mis on sündinud tervel Juudamaal alates Galileast pärast ristimist, mida Johannes kuulutas,
\par 38 kuidas Jumal Jeesuse Naatsaretist oli võidnud Püha Vaimu ja väega ja kuidas tema käis mööda maad ja tegi head ning parandas kõiki, kelle üle kurat oli saanud võimuse; sest Jumal oli temaga.
\par 39 Ja meie oleme kõigi nende asjade tunnistajad, mis ta on teinud juutide maal ja Jeruusalemmas; ja nad poosid tema ristipuule ning surmasid.
\par 40 Selle on Jumal üles äratanud kolmandal päeval ja on teda lasknud saada ilmsiks,
\par 41 mitte kõigele rahvale, vaid Jumala ennevalitud tunnistajaile, meile, kes temaga oleme söönud ja joonud, pärast seda kui ta oli surnuist üles tõusnud.
\par 42 Ja tema on meid käskinud rahvale kuulutada ja tunnistada, et tema on Jumala poolt seatud elavate ja surnute kohtumõistja.
\par 43 Temast tunnistavad kõik prohvetid, et tema nime läbi igaüks, kes usub temasse, saab pattude andeksandmise!”

\section*{Paganadki saavad Püha Vaimu ja neid ristitakse}

\par 44 Kui Peetrus neid sõnu alles rääkis, langes Püha Vaim kõikide peale, kes seda sõna kuulsid!
\par 45 Ja usklikud ümberlõigatute hulgast, niipalju kui neid ühes Peetrusega oli tulnud, ehmusid, et Püha Vaimu and valati ka paganate peale,
\par 46 sest nad kuulsid neid võõraid keeli rääkivat ja Jumalat väga ülistavat. Siis Peetrus ütles:
\par 47 „Kas keegi võib vett keelata, et ei ristitaks neid, kes on saanud Püha Vaimu nõnda nagu meiegi?”
\par 48 Ja ta käskis neid ristida Issanda nimesse. Siis nad palusid teda, et ta jääks mõneks päevaks nende juurde.


\chapter{11}

\section*{Peetrus õigustab paganate ristimist}

\par 1 Siis apostlid ja vennad, kes Juudamaal siin ja seal elasid, said kuulda, et ka paganad olid vastu võtnud Jumala sõna.
\par 2 Aga kui Peetrus tuli Jeruusalemma, vaidlesid temaga ümberlõigatud
\par 3 ning ütlesid: „Sina oled läinud sisse ümberlõikamatute juurde ja oled söönud nendega!”
\par 4 Aga Peetrus hakkas rääkima ning seletas neile asja selle järjekorras nõnda:
\par 5 „Mina olin Joppe linnas palvetamas ja nägin otsekui enesest ära olles nägemuse: seal tuli alla anum, nagu suur linane riie, mida nelja nurka pidi lasti alla taevast, ja see tuli aiva minu juurde.
\par 6 Kui ma seda juureldes vaatasin, nägin ma maa neljajalgseid elajaid ja metsalisi ja roomajaid ja taeva linde,
\par 7 ja ma kuulsin ka häält mulle ütlevat: tõuse üles, Peetrus, verista ja söö!
\par 8 Aga ma ütlesin: ei ilmaski, Issand; sest minu suhu ei ole elades saanud midagi keelatut ehk roojast!
\par 9 Siis vastas mulle teist korda hääl taevast: mis Jumal on puhastanud, seda ära sina arva keelatuks!
\par 10 Ja see sündis kolm korda. Ja kõik tõmmati jälle üles taevasse.
\par 11 Ja vaata, samal hetkel seisid maja ees, kus me olime, kolm meest, kes olid Kaisareast läkitatud minu juurde.
\par 12 Ent Vaim ütles mulle, et ma läheksin ühes nendega ega mõtleks kaksipidi. Siis tulid ka need kuus venda minuga, ja me läksime selle mehe kotta.
\par 13 Ja ta kuulutas meile, kuidas ta oli näinud inglit oma majas seisvat ja ta talle ütlevat: läkita Joppesse ja kutsu Siimon, keda hüütakse lisanimega Peetruseks.
\par 14 Tema räägib sulle sõnu, mille läbi sina ja kõik su pere saate õndsaks.
\par 15 Aga kui ma hakkasin kõnelema, langes Püha Vaim nende peale, nagu meiegi peale alguses.
\par 16 Siis meenus mulle Issanda sõna, kuidas ta ütles: Johannes ristis küll veega, aga teid ristitakse Püha Vaimuga!
\par 17 Kui nüüd Jumal neile on andnud sama anni kui meilegi, kes usume Issandasse Jeesusesse Kristusesse, kes siis olin mina, et oleksin võinud Jumalat keelata?”
\par 18 Aga kui nad seda kuulsid, jäid nad vait ja andsid Jumalale au, öeldes: „Siis on Jumal ka paganaile andnud meeleparanduse eluks!”

\section*{Paganate kogudus Antiookias}

\par 19 Aga need, kes olid hajutatud viletsuse läbi, mis oli tekkinud Stefanose pärast, käisid maad läbi Foiniikiani ja Küprose saareni ja Antiookiani ega kuulutanud sõna kellelegi muule kui ainult juutidele.
\par 20 Aga nende hulgas olid mõned Küprose ja Küreene mehed; need läksid Antiookiasse ning rääkisid ka kreeklastele, jutlustades Issandat Jeesust.
\par 21 Ja Issanda käsi oli nendega ning suur hulk uskus ja pöördus Issanda poole.
\par 22 Aga teade neist kostis Jeruusalemmas oleva koguduse kõrvu, ja nad läkitasid Barnabase Antiookiasse.
\par 23 Kui ta sinna jõudis ja nägi Jumala armu, sai ta rõõmsaks ja manitses kõiki jääma kindla südamega Issanda juurde;
\par 24 sest ta oli õige mees, täis Püha Vaimu ning usku, ja palju rahvast koguti Issandale.
\par 25 Siis ta läks Tarsosesse Saulust otsima, ja kui ta tema leidis, tõi ta tema Antiookiasse.
\par 26 Ja nemad käisid terve aasta koos koguduses ja õpetasid suure hulga rahvast. Ja Antiookias hakati kõige enne nimetama jüngreid kristlasteks.

\section*{Abi Antiookiast vendadele Juudamaal}

\par 27 Noil päevil tuli Jeruusalemmast prohveteid Antiookiasse.
\par 28 Ja üks neist, Aagabus nimi, tõusis ja andis Vaimu läbi teada, et suur nälg oli tulemas üle kogu maailma. Ja see tuligi keiser Klaudiuse ajal.
\par 29 Siis jüngrid võtsid nõuks igaüks sedamööda, kuidas kellelgi jõudu oli, saata abi vendadele, kes elasid Juudamaal.
\par 30 Seda nad tegidki ja saatsid annid vanemate kätte Barnabase ja Sauluse käe läbi.


\chapter{12}

\section*{Tagakiusamine kuningas Heroodese poolt. Peetrus pääseb vangist}

\par 1 Aga sel ajal pistis kuningas Heroodes käe mõnede külge kogudusest, et neile paha teha.
\par 2 Ja ta tappis mõõgaga Jakoobuse, Johannese venna.
\par 3 Ja nähes, et see oli juutidele meelt mööda, võttis ta ette vangistada ka Peetruse. Sel ajal olid parajasti hapnemata leibade pühad.
\par 4 Ta võttis tema kinni ja heitis ta vangitorni ning andis tema nelja neljamehelise sõjameeste salga kätte hoida, kavatsedes teda pärast paasapüha tuua rahva ette.
\par 5 Nii peeti siis Peetrust vangis, aga kogudus tegi väsimata palvet Jumala poole tema eest.
\par 6 Aga kui Heroodes tahtis teda tuua kohtu ette, magas Peetrus sel ööl kahe sõjamehe vahel kahe ahelaga seotult, ja vahid ukse ees valvasid vangihoonet.
\par 7 Ja vaata, Issanda ingel seisis seal ja valgus paistis seal toas! Ja ingel tõukas Peetruse külge, äratas ta üles ning ütles: „Tõuse usinasti üles!” Ja ta ahelad langesid käte ümbert maha.
\par 8 Ja ingel ütles talle: „Pane vöö vööle ja kingad jalga!” Tema tegi nõnda. Veel ta ütles talle: „Pane kuub selga ja tule mu järel!”
\par 9 Ja ta tuli välja ja läks ta järele ega teadnud, et on ilmsi see, mis sünnib ingli kaudu, vaid mõtles nägevat nägemust.
\par 10 Aga kui nad jõudsid mööda esimesest ja teisest vahist, tulid nad raudvärava ette, mis viis linna. See läks iseenesest neile lahti ja nad tulid välja ja läksid edasi üht tänavat mööda. Ja ingel lahkus äkitselt temast.
\par 11 Kui siis Peetrus teadlikuks sai, ütles ta: „Nüüd ma tean tõesti, et Issand on läkitanud oma ingli ja on mind päästnud Heroodese käest ja kõigest, mida juuda rahvas ootas!”
\par 12 Ja märgates, kus ta on, tuli ta Maarja, Johannese, teise nimega Markuse ema maja juurde, kuhu mitmed olid kokku tulnud palvetama.
\par 13 Ent kui Peetrus koputas jalgväravale, tuli tüdruk, nimega Roode, välja kuulatama.
\par 14 Kui ta Peetruse hääle ära tundis, ei teinud ta väravat lahti rõõmu pärast, vaid jooksis sisse ja teatas, et Peetrus värava taga seisvat.
\par 15 Nemad ütlesid talle: „Sa jampsid!” Aga tema tunnistas kindlasti, et see on nõnda. Nad ütlesid: „See on tema ingel!”
\par 16 Aga Peetrus jäi veel koputama. Ja kui nad avasid, nägid nad teda ja ehmusid.
\par 17 Siis ta viitas käega neile, et nad jääksid vait, ja jutustas neile, kuidas Issand ta oli vangihoonest välja viinud, ning ütles: „Kuulutage seda Jakoobusele ja vendadele!” Ja ta väljus ning läks teise paika.
\par 18 Aga kui valgeks läks, ei olnud sõjameestel pisut muret sellest, mis Peetrusega oli sündinud.
\par 19 Ja kui Heroodes teda otsis ega leidnud, laskis ta valvuritele kohut mõista ja käskis nad ära hukata. Ise ta läks siis Juudamaalt Kaisareasse ja viibis seal.

\section*{Heroodese surm}

\par 20 Heroodes aga kandis viha tüüroslaste ja siidonlaste vastu. Need tulid ühel meelel tema juurde ja meelitasid Blastust, kuninga kambriülemat, ja palusid rahu. Sest nende maa sai oma toiduse kuninga maast.
\par 21 Siis pani Heroodes määratud päeval kuninglikud riided selga ja istus kohtujärjele ja pidas neile kõnet;
\par 22 aga rahvas kisendas: „See on Jumala, aga mitte inimese hääl!”
\par 23 Aga sedamaid lõi teda Issanda ingel, sest et ta ei andnud au Jumalale; ja ussid sõid ta ära ja ta heitis hinge.
\par 24 Aga Jumala sõna kasvas ja levis.
\par 25 Ja Barnabas ja Saulus tulid tagasi Jeruusalemmast, kui nad olid täitnud oma ülesande, ning võtsid enestega ühes ka Johannese, keda hüüti lisanimega Markuseks.



\chapter{13}

\section*{Antiookia kogudus}

\par 1 Aga Antiookias olevas koguduses oli prohveteid ja õpetajaid: Barnabas ja Siimon, keda hüüti Nigeriks, ja Luukius Küreenest ja Maanaen, kes nelivürst Heroodesega ühes oli üles kasvatatud, ja Saulus.
\par 2 Kui need Issandat teenisid ja paastusid, ütles Püha Vaim: „Eraldage mulle Barnabas ja Saulus tööle, milleks mina nad olen kutsunud!”
\par 3 Siis nad paastusid ja palusid Jumalat ja panid oma käed nende peale ja saatsid nad teele.

\section*{Barnabas ja Saulus Küproses}

\par 4 Need, Püha Vaimu poolt läkitatud, tulid nüüd Seleukiasse ja sõitsid sealt laevaga Küprosesse.
\par 5 Ja kui nad Salamisesse olid jõudnud, kuulutasid nad Jumala sõna juutidele kogudusekodades; neil oli ka Johannes käsiliseks ühes.
\par 6 Aga kui nad olid saare läbi käinud Paafosest saadik, kohtasid nad üht juudi nõida ja valeprohvetit, Barjeesus nimi;
\par 7 see oli maavalitseja Sergius Pauluse juures, kes oli mõistlik mees. Tema kutsus Barnabase ja Sauluse enese juurde ja püüdis kuulda Jumala sõna.
\par 8 Aga neile pani vastu Elümas, nõid - sest seda tähendab nimi tõlkes - püüdes maavalitsejat ära pöörata usust.
\par 9 Kuid Saulus, keda hüütakse ka Pauluseks, täis Püha Vaimu, vaatas teravasti talle otsa
\par 10 ning ütles: „Oh sind, kes oled täis kõike kavalust ja kõike tigedust, sa kuradi laps, sa kõige õiguse vaenlane, kas sa ei lakka kõverdamast Issanda sirgeid teid?
\par 11 Ja nüüd, vaata, Issanda käsi tuleb sinu peale ja sa jääd pimedaks ega näe päikest tükil ajal!” Ja sedamaid langes ta peale sõgedus ning pimedus, ja ta käis ümber ja otsis talutajaid.
\par 12 Kui maavalitseja nägi, mis sündis, uskus ta ja ehmus üliväga Issanda õpetusest.

\section*{Paulus ja Barnabas Pisiidia Antiookias ja Pauluse jutlus juutidele}

\par 13 Ja kui need, kes olid kaasas Paulusega, ära läksid Paafosest, saabusid nad Pergesse, Pamfüüliamaale. Seal Johannes lahkus neist ja läks tagasi Jeruusalemma.
\par 14 Aga nemad läksid edasi Pergest ja jõudsid Pisiidia Antiookiasse ning läksid kogudusekotta hingamispäeval ja istusid maha.
\par 15 Ja pärast seda, kui olid loetud käsk ja prohvetid, läkitasid kogudusekoja ülemad neile ütlema: „Mehed, vennad, kui teil on öelda mõni manitsussõna rahvale, siis öelge!”
\par 16 Siis Paulus tõusis üles ja viitas käega neile ning ütles: „Iisraeli mehed ja teie, kes kardate Jumalat, kuulge!
\par 17 Selle rahva, Iisraeli, Jumal valis ära meie esiisad ja ülendas rahva, kui nad võõrastena olid Egiptuses, ja tõi nad sealt välja kõrgele tõstetud käsivarrega,
\par 18 ja sallis nende kombeid ligi nelikümmend aastat kõrbes
\par 19 ja hävitas seitse rahvast Kaananimaal ja jagas nende maa neile pärandiks.
\par 20 See kõik sündis umbes neljasaja viiekümne aasta jooksul. Pärast seda ta andis neile kohtumõistjaid prohvet Saamueli ajani.
\par 21 Ja selle järele nad palusid enestele kuningat. Ja Jumal andis neile Sauli, Kiisi poja, Benjamini suguharust mehe. Nii möödus nelikümmend aastat.
\par 22 Ja kui ta selle oli tagandanud, äratas ta neile kuningaks Taaveti, kellest tema ka tunnistades ütles: „Ma olen leidnud Taaveti, Jesse poja, oma südamekohase mehe, kes teeb kõik mu tahtmist mööda!”
\par 23 Selle soost on Jumal äratanud oma tõotuse järgi Jeesuse Iisraeli Päästjaks,
\par 24 siis kui otse enne tema ilmumist Johannes kuulutas meeleparanduse ristimist kõigele Iisraeli rahvale.
\par 25 Ent kui Johannes oli oma elukäiku lõpetamas, ütles tema: „Mina ei ole see, kelleks te mind peate; aga vaata, pärast mind tuleb see, kelle kinga jalast päästma ma väärt ei ole!”
\par 26 Mehed, vennad, Aabrahami soo lapsed ja need, kes teie seast kardavad Jumalat, meile on selle pääste sõna läkitatud!
\par 27 Sest Jeruusalemma elanikud ja nende ülemad, Jeesust mitte tundes, saatsid hukkamõistmisega täide prohvetite sõnad, mida igal hingamispäeval loetakse;
\par 28 ja ehk nad küll ei leidnud ühkti surmasüüd, palusid nad Pilaatust, et ta surmataks.
\par 29 Ja kui nad olid täide saatnud kõik, mis temast oli kirjutatud, võtsid nad ta maha puu pealt ja panid hauda.
\par 30 Aga Jumal äratas tema surnuist üles!
\par 31 Ja tema ilmus palju päevi neile, kes temaga olid tulnud Galileast Jeruusalemma ja kes nüüd on ta tunnistajad rahva ees.
\par 32 Ja me kuulutame teile selle hea sõnumi, tõotuse, mis anti meie esiisadele,
\par 33 et Jumal selle on täitnud meile, nende lastele, äratades üles Jeesuse, nagu ka teises laulus on kirjutatud: „Sina oled mu Poeg, täna ma sünnitasin sind!”
\par 34 Et ta tema on surnuist üles äratanud, nii et tal iialgi ei ole tarvis kõduneda, sellest on ta öelnud nõnda: „Mina täidan teile need Taavetile antud pühad ja kindlad tõotused!”
\par 35 Sellepärast ta ütleb ka teises paigas: „Sina ei lase oma Püha näha kõdunemist!”
\par 36 Sest kui Taavet oma rahvapõlve ajal oli teinud Jumala tahte järgi, uinus ta magama ja pandi oma esiisade juurde ja nägi kõdunemist;
\par 37 kuid see, kelle Jumal üles äratas, ei näinud kõdunemist!
\par 38 Sellepärast olgu teil nüüd teada, mehed, vennad, et teile selle läbi kuulutatakse pattude andeksandmist,
\par 39 ja et igaüks, kes usub, mõistetakse õigeks tema sees kõigest sellest, millest te ei võinud õigeks saada Moosese käsuõpetuse kaudu.
\par 40 Katsuge nüüd, et teid ei tabaks, mis on öeldud prohvetite kirjades:
\par 41 „Vaadake, te põlgajad, ja pange imeks ja minge hukka! Sest ma teen teie päevil teo; teo, mida te ei usuks, kui keegi teile sellest jutustaks!”
\par 42 Kui nad siis väljusid, paluti neid järgmisel hingamispäeval neile rääkida neid sõnu.
\par 43 Aga kui kogudusekojast oli laiali mindud, järgis palju juute ja juudiusku pöördunud jumalakartlikke Paulust ja Barnabast, kes nendega kõnelesid ja neid õhutasid jääma Jumala armusse.

\section*{Apostlid pöörduvad paganate poole}

\par 44 Aga järgmisel hingamispäeval kogunes peaaegu kõik linn kuulama Jumala sõna.
\par 45 Kui siis juudid nägid rahvahulka, said nad täis viha ja vaidlesid vastu Pauluse kõnele ning pilkasid Jumalat.
\par 46 Aga Paulus ja Barnabas kõnelesid julgesti ning ütlesid: „Teile pidi Jumala sõna esiti räägitama; kuid et te selle enesest ära lükkate ega arva endid olevat igavest elu väärt, siis me pöördume paganate poole.
\par 47 Sest nõnda on meid Issand käskinud: mina olen sind pannud valguseks paganaile, et sa oleksid päästeks ilmamaa otsani!”
\par 48 Kui paganad seda kuulsid, said nad rõõmsaks ja ülistasid Issanda sõna ja uskusid, nii paljud kui olid määratud igaveseks eluks.
\par 49 Ja Issanda sõna levis üle kogu maakonna.
\par 50 Aga juudid ässitasid jumalakartlikke kõrgema seisuse naisi ja linna ülemaid ning äratasid vaenu Pauluse ja Barnabase vastu ja ajasid nad välja oma aladelt.
\par 51 Aga nemad puistasid oma jalgade põrmu nende peale ja tulid Ikoonioni.
\par 52 Ja jüngrid täitusid rõõmu ja Püha Vaimuga.


\chapter{14}

\section*{Paulus ja Barnabas Ikoonionis}

\par 1 Aga Ikoonionis sündis, et nad üheskoos läksid juutide kogudusekotta ja kõnelesid nõnda, et niihästi suur hulk juute kui kreeklasi sai usklikuks.
\par 2 Aga uskmatud juudid kihutasid üles ja ärritasid paganate meeled vihale vendade vastu.
\par 3 Siis nad viibisid seal kauemat aega, kuulutades julgesti Issandat, kes andis tunnistuse oma armusõnale ja laskis sündida tunnustähti ja imetegusid nende käte läbi.
\par 4 Aga linna rahvas jagunes kaheks: ühed olid juutide, teised apostlite poolt.
\par 5 Kui siis paganad ja juudid oma ülematega ühel nõul tahtsid tulla nende kallale, et neid teotada ja kividega visata,
\par 6 said nad seda teada ja põgenesid Lükaoonia linnadesse, Lüstrasse ja Derbesse ja nende ümbrusesse
\par 7 ja kuulutasid seal evangeeliumi.

\section*{Paulus ja Barnabas Lüstras}

\par 8 Ja Lüstras istus mees, võimetu seisma, oma emaihust jalutu, kes ei olnud elades kõndinud.
\par 9 See kuulis Pauluse kõnet. Ja kui Paulus teda silmas ning nägi, et tal oli usku terveks saada,
\par 10 ütles ta suure häälega: „Tõuse püsti oma jalgele!” Ja ta kargas üles ja kõndis.
\par 11 Aga kui rahvas nägi, mis Paulus oli teinud, siis nad tõstsid oma häält ning ütlesid lükaoonia keeli: „Jumalad on inimeste sarnastena alla tulnud meie juurde!”
\par 12 Ja nad nimetasid Barnabast Zeusiks ja Paulust Hermeseks, sest tema oli see, kes kõneles.
\par 13 Ja Zeusi preester linna aguli templist tõi härgi ja lillepärgi värava ette ja tahtis ohverdada ühes rahvaga.
\par 14 Aga kui apostlid Barnabas ja Paulus seda kuulsid, rebestasid nad oma riided, kargasid hüüdes rahva sekka
\par 15 ning ütlesid: „Mehed, miks te seda teete? Meie oleme niisamasugused nõdrad inimesed nagu teiegi ja kuulutame teile evangeeliumi, et te tühjest asjust pöörduksite elava Jumala poole, kes on teinud taeva ja maa ja mere ning kõik, mis nende sees on,
\par 16 kes endisel põlvel on lasknud kõik rahvad käia omi teid,
\par 17 ja siiski ei ole jätnud andmata enesest tunnistust, vaid on meile head teinud, ja on taevast meile andnud vihma ja head viljalist aega ja rooga ning on täitnud meie südame rõõmuga!”
\par 18 Nii kõneldes nad vaevalt vaigistasid rahva, et see neile ei ohverdaks.
\par 19 Aga Antiookiast ja Ikoonionist tuli sinna juute ja need meelitasid hulga eneste poole ja viskasid Paulust kividega ning vedasid ta linnast välja, arvates, et ta on surnud.
\par 20 Aga kui jüngrid olid tema ümber tulnud, tõusis ta üles ja läks linna. Ja järgmisel päeval ta läks ühes Barnabasega Derbesse.

\section*{Tagasi Antiookiasse}

\par 21 Ja kui nad selles linnas olid kuulutanud evangeeliumi ja saanud mõned jüngreiks, läksid nad tagasi Lüstrasse, Ikoonioni ja Antiookiasse ning
\par 22 kinnitasid jüngrite hingi, manitsedes neid jääma ususse ning seletades, et meil mitme viletsuse kaudu tuleb minna Jumala riiki.
\par 23 Ja kui nad neile igas koguduses käte pealepanemisega olid seadnud vanemad, jätsid nad paastudes ja palvetades nemad Issanda hooleks, kellesse need nüüd uskusid.
\par 24 Siis nad käisid läbi Pisiidiamaa ja saabusid Pamfüüliasse.
\par 25 Ja kui nad Perges olid sõna kuulutanud, läksid nad Ataaliasse.
\par 26 Sealt nad purjetasid Antiookiasse, kust nad olid väljunud Jumala armu hooleks antuina sellele tööle, mille nad olid lõpetanud.
\par 27 Ja kui nad olid saabunud, kogusid nad koguduse kokku ja kuulutasid, mis suuri asju Jumal oli teinud, olles nendega, ja et ta oli avanud paganaile usu ukse.
\par 28 Ja nad viibisid seal kauemat aega jüngrite juures.


\chapter{15}

\section*{Pagankristlased ja Moosese seadus}

\par 1 Ja mõned, kes olid tulnud Juudamaalt, õpetasid vendi nõnda: „Kui te ei lase endid ümber lõigata Moosese kombe järgi, siis te ei või õndsaks saada!”
\par 2 Kui sellest tõusis lahkmeel ja Paulusel ja Barnabasel ei olnud nendega mitte pisut vaidlemist, siis tehti otsuseks, et Paulus ja Barnabas ning mõned muud nende seast lähevad Jeruusalemma apostlite ja vanemate juurde selle tüliküsimuse pärast.
\par 3 Siis kogudus saatis nad teele, ja nemad läksid Foiniikiast ja Samaariast läbi ning jutustasid paganate pöördumisest ja tegid sellega suure rõõmu kõigile vendadele.
\par 4 Kui nad saabusid Jeruusalemma, võtsid kogudus, apostlid ja vanemad nad vastu. Ja nemad jutustasid, mis suuri asju Jumal oli teinud, olles nendega.
\par 5 Aga mõned variseride usulahust, kes olid saanud usklikuks, astusid esile ning ütlesid: „Neid tuleb ümber lõigata ja käskida pidada Moosese käsku!”

\section*{Apostlite koosolek Jeruusalemmas}

\par 6 Siis apostlid ja vanemad tulid kokku seda asja arutama.
\par 7 Ja kui asjast tekkis palju vaidlust, tõusis Peetrus ning ütles neile: „Mehed, vennad, teie teate, et Jumal juba ammusest ajast teie seast on ära valinud mind, et minu suu läbi paganad kuuleksid evangeeliumi sõna ja usuksid.
\par 8 Ja südametundja Jumal on neile andnud tunnistuse, andes neile Püha Vaimu otsekui meilegi,
\par 9 ega ole teinud mingisugust vahet meie ja nende vahel, puhastades nende südamed usu läbi.
\par 10 Miks te siis nüüd kiusate Jumalat, et panete jüngrite kaela ikke, mida meie esiisad ega meie ei ole suutnud kanda?
\par 11 Ent me usume Issanda Jeesuse Kristuse armu läbi saada õndsaks, otse nõnda kui nemadki!”
\par 12 Aga kogu rahvahulk jäi vait ja nad kuulasid, kuidas Barnabas ja Paulus jutustasid, mis suuri tunnustähti ja imesid Jumal paganate seas oli teinud nende läbi.
\par 13 Kui nad siis olid jutu lõpetanud, võttis Jakoobus sõna ning ütles: „Mehed, vennad, kuulge mind!
\par 14 Siimon on seletanud, kuidas Jumal esiti võttis eesmärgiks võita oma nimele rahvas paganate seast.
\par 15 Sellega sobivad ühte prohvetite sõnad, nagu on kirjutatud:
\par 16 „Pärast seda ma tulen jälle ja ehitan üles Taaveti lagunenud maja ning parandan ära tema varemed ja püstitan ta uuesti,
\par 17 et muudki inimesed otsiksid Issandat ja kõik rahvad, kelle üle on nimetatud minu nimi, ütleb Issand, kes seda teeb,
\par 18 mis on teada igavikust.”
\par 19 Sellepärast arvan mina, et neile, kes paganate seast pöörduvad Jumala poole, ei peaks tehtama raskusi,
\par 20 vaid neile kirjutatagu, et nad hoiduksid väärjumalate rüveduse, porduelu, lämbunu ja vere eest.
\par 21 Sest Moosesel on vanast ajast kõigis linnades küllalt neid, kes teda kuulutavad, ning teda loetakse igal hingamispäeval kogudusekodades.”

\section*{Apostlite kiri pagankristlastele}

\par 22 Siis apostlid ja vanemad ühes kõige kogudusega arvasid heaks valida endi seast mehed ning läkitada Antiookiasse ühes Pauluse ja Barnabasega, nimelt Juuda, lisanimega Barsabas, ja Siilase, kes olid juhatajad vendade seas.
\par 23 Nendega nad saatsid järgmise kirja: „Meie, apostlid ja vanemad ja vennad, saadame tervisi vendadele paganate seast, kes elavad Antiookia- ja Süüria- ja Kiliikiamaal.
\par 24 Olles kuulnud, et mõned meie seast on välja läinud ja on teid teinud kõnedega rahutuks ja on saatnud teie hinged kitsikusse, ilma et meie neid oleksime käskinud,
\par 25 arvasime üksmeelselt heaks valida mehed ja läkitada need teie juurde ühes meie armsate vendade Barnabase ja Paulusega,
\par 26 meestega, kes oma elu on pannud kaalule meie Issanda Kristuse nime pärast.
\par 27 Nii me oleme siis läkitanud Juuda ja Siilase, et nad teile suusõnal teataksid sedasama.
\par 28 Sest Püha Vaim ja meie oleme arvanud heaks, et teie peale ei tohi panna enam ühtki koormat kui aga need väga tarvilised määrused:
\par 29 hoiduda ebajumalate ohvriliha ja vere ja lämbunu ja porduelu eest. Kui te nendest hoidute, teete hästi. Jääge Jumalaga!”
\par 30 Nii nad saadeti teele ja nad saabusid Antiookiasse. Seal nad kogusid rahvahulga kokku ja andsid kirja neile kätte.
\par 31 Kui nad seda olid lugenud, said nad rõõmsaks selle troostist.
\par 32 Aga Juudas ja Siilas, kes ka ise olid prohvetid, manitsesid vendi mitme sõnaga ja kinnitasid neid.
\par 33 Ja kui nad seal mõne aja olid viibinud, saadeti nad rahuga vendade juurest nende juurde, kes nad olid läkitanud.
\par 34 [Ja Siilas arvas heaks jääda nende juurde.]
\par 35 Aga Paulus ja Barnabas viibisid Antiookias ning õpetasid ja kuulutasid ühes paljude teistega Issanda armuõpetuse sõna.

\section*{Paulus ja Barnabas lähevad lahku}

\par 36 Mõne päeva pärast ütles Paulus Barnabasele: „Lähme jälle vaatama vendi kõiki linnu mööda, kus me oleme kuulutanud Issanda sõna, kuidas nende käsi käib!”
\par 37 Ja Barnabas andis nõu võtta kaasa Johannes, keda hüütakse Markuseks.
\par 38 Aga Paulus arvas heaks mitte võtta enesega teda, kes neist oli lahkunud Pamfüülias ega olnud ühes nendega läinud tööle.
\par 39 Sellest tekkis äge vaidlus, nii et nad üksteisest lahkusid ja et Barnabas võttis enese juurde Markuse ning sõitis laevaga Küprosesse.
\par 40 Aga Paulus valis Siilase ja läks teele, ja vennad andsid ta Jumala armu hooleks.
\par 41 Ja ta käis läbi Süüria ja Kiliikia kinnitades kogudusi.


\chapter{16}

\section*{Timoteos saab Pauluse kaaslaseks}

\par 1 Ja ta saabus Derbesse ja Lüstrasse. Ja vaata, seal oli üks jünger, Timoteos nimi, uskliku juudi naise poeg; aga ta isa oli kreeklane.
\par 2 Sellele andsid Lüstras ja Ikoonionis olijad vennad hea tunnistuse.
\par 3 Teda tahtis Paulus enesele saatjaks teele. Ja ta võttis ning lõikas ta ümber juutide pärast, kes olid neis paigus; sest nad teadsid kõik, et tema isa oli kreeklane.
\par 4 Aga kui nad käisid linnast linna, käskisid nad usklikke pidada neid seadmisi, mis apostlid ja vanemad Jeruusalemmas olid otsustanud.
\par 5 Siis kinnitati kogudusi usus ja liikmete arv kasvas päev-päevalt.

\section*{Pauluse teekond läbi Väike-Aasia}

\par 6 Ja nemad käisid läbi Früügia- ja Galaatiamaa, sest Püha Vaim oli neid keelanud sõna rääkimast Aasias.
\par 7 Jõudes Müüsiani üritasid nad matkata Bitüüniasse, kuid Jeesuse Vaim ei lasknud neid mitte.
\par 8 Siis nad läksid mööda Müüsiast ja tulid alla Troasse.
\par 9 Ja Paulus nägi öösel nägemuse: Makedoonia mees seisis ja palus teda ning ütles: „Tule alla Makedooniasse ja aita meid!”
\par 10 Kui ta seda nägemust oli näinud, siis me püüdsime kohe välja minna Makedooniasse ja me olime julged selles, et Issand meid oli kutsunud neile evangeeliumi kuulutama.

\section*{Paulus Filipis}

\par 11 Me läksime Troast teele ja tulime otsekohe Samotraakesse ja järgmisel päeval Neapolisse
\par 12 ja sealt Filipisse, mis on esimesi linnu tolles Makedoonia osas, asulinn. Sinna linna me jäime peatama mõneks päevaks.
\par 13 Ja hingamispäeval me läksime linnast välja jõe äärde, kus neil oli kombeks käia palvusel, ja istusime maha ning kõnelesime naistele, kes olid kokku tulnud.
\par 14 Ja ka üks jumalakartlik naine, nimega Lüüdia, purpurimüüja Tüatiira linnast, kuulas jutlust. Tema südame avas Issand, et ta pani tähele seda, mida Paulus rääkis.
\par 15 Aga kui tema ja ta pere olid ristitud, palus ta meid ning ütles: „Kui te arvate, et ma olen saanud usklikuks Issandasse, siis tulge mu kotta ja jääge sinna!” Ja ta käis meile peale.
\par 16 Aga sündis, kui me läksime palvusele, et meile tuli vastu tüdruk, kellel oli lausuja vaim. See tõi oma isandaile palju kasu lausumisega.
\par 17 Tema tuli Paulusele ja meile järele, kisendas ning ütles: „Need inimesed on kõige kõrgema Jumala sulased, kes teile kuulutavad õndsuse teed!”
\par 18 Ja seda ta tegi mitu päeva. Aga Paulusel oli sellepärast meel haige, ja ta pöördus ümber ning ütles vaimule: „Ma käsin sind Jeesuse Kristuse nimel välja minna tema seest!” Ja vaim läks välja samal tunnil.

\section*{Pauluse ja Siilase vangistamine ja vabanemine}

\par 19 Aga kui tüdruku isandad nägid, et nende kasu lootus oli lõppenud, võtsid nad Pauluse ja Siilase kinni ja vedasid nad turule ülemate juurde.
\par 20 Ja nad viisid nad pealikute ette ning ütlesid: „Need inimesed teevad meie linna rahutuks ja nad on juudid!
\par 21 Nad kuulutavad kombeid, mida meil kui roomlasil ei sünni vastu võtta ega täita!”
\par 22 Ja rahvas tõusis nende vastu. Ja pealikud käskisid kiskuda nende riided seljast ja neid vitstega peksta.
\par 23 Ja kui nad neile olid palju hoope andnud, viskasid nad nemad vanglasse ja käskisid vangihoidjal neid hoolsasti valvata.
\par 24 Niisuguse kange käsu saanud, heitis ta nad kõige kindlamasse vangihoonesse ja pani nende jalad pakku.
\par 25 Aga keskööl palvetasid Paulus ja Siilas ning kiitsid lauldes Jumalat. Ja vangid kuulsid neid.
\par 26 Siis äkitselt sündis suur maavärisemine, nii et vangihoone alused vabisesid; ja sedamaid läksid kõik uksed lahti, ka kõikide köidikud pääsesid valla!
\par 27 Kui nüüd vangihoidja unest ärkas ja nägi vangitorni uksed avatud olevat, tõmbas ta mõõga ja tahtis enesele otsa teha, sest ta mõtles, et vangid on ära karanud.
\par 28 Aga Paulus hüüdis suure häälega ning ütles: „Ära tee enesele kurja, sest me oleme kõik siin!”
\par 29 Siis ta küsis tuld, kargas sisse, heitis värisedes Pauluse ja Siilase ette maha,
\par 30 tõi nad ka välja ning ütles: „Isandad, mis ma pean tegema, et ma õndsaks saaksin?”
\par 31 Aga nemad ütlesid: „Usu Issandasse Jeesusesse, siis saad õndsaks sina ja su pere!”
\par 32 Ja nad kuulutasid temale Issanda sõna ja kõigile, kes olid tema kojas.
\par 33 Tema võttis nad sel öötunnil enese juurde ja pesi nende haavad. Ja ta ristiti sedamaid ja kõik tema pere.
\par 34 Siis ta viis nad oma kotta, tõi toitu lauale ja oli väga rõõmus, et ta kogu oma perega oli saanud usklikuks Jumalasse.
\par 35 Kui valgeks läks, läkitasid pealikud kohtusulased ütlema: „Lase need inimesed lahti!”
\par 36 Siis vangihoidja teatas need sõnad Paulusele, öeldes: „Pealikud on siia läkitanud sõna, et teid tuleb lahti lasta; seepärast väljuge nüüd ja minge rahuga!”
\par 37 Aga Paulus ütles neile: „Nad on meid avalikult peksnud ilma kohut mõistmata, ja me oleme Rooma kodanikud; nad on meid heitnud vangitorni ja nüüd nad tahavad meid salaja välja lükata. Nõnda ei tohi olla! Vaid tulgu nad ise ja saatku meid välja!”
\par 38 Kohtusulased teatasid need sõnad pealikuile. Need lõid kartma, kui nad kuulsid neid Rooma kodanikud olevat.
\par 39 Ja nad tulid ning rääkisid nendega lahkesti, tõid nad välja vangist ja palusid neid linnast ära minna.
\par 40 Siis nad tulid vanglast välja ja läksid Lüüdia juurde. Ja kui nad olid vendi näinud, siis nad kinnitasid neid ja läksid ära.


\chapter{17}

\section*{Paulus ja Siilas Tessaloonikas}

\par 1 Ja nad läksid läbi Amfipolise ja Apolloonia ja tulid Tessaloonikasse, kus oli juutide kogudusekoda.
\par 2 Ja Paulus läks oma kombe järgi sisse nende juurde ja kõneles neile kolmel hingamispäeval Kirjast,
\par 3 seletades ja tõestades, et Kristus pidi kannatama ja surnuist üles tõusma, ja öeldes: „See Jeesus, keda mina teile kuulutan, on Kristus!”
\par 4 Ja mõned neist said usklikuks ja liitusid Pauluse ja Siilasega, nõndasamuti suur hulk jumalakartlikke kreeklasi ja mitte pisut suursuguseid naisi.
\par 5 Aga juudid said kadedaks ja võtsid enestega mõningaid hulguseid, nurjatuid mehi, kihutasid rahva üles ning tõstsid linnas möllu ja ründasid Jaasoni maja ning katsusid neid välja vedada rahva ette.
\par 6 Aga kui nad neid ei leidnud, vedasid nad Jaasoni ja mõned vennad linna ülemate ette ning kisendasid: „Need, kes kõiges maailmas tüli tõstavad, on ka siin!
\par 7 Neid on Jaason vastu võtnud ja nad kõik teevad keisri käskude vastu ja ütlevad ühe teise olevat kuninga, Jeesuse!”
\par 8 Seda kuuldes said rahvas ja linna ülemad rahutuks,
\par 9 kuid saades Jaasonilt ja teistelt tagatise, lasksid nad nemad lahti.

\section*{Paulus ja Siilas Beroias}

\par 10 Aga vennad saatsid sedamaid Pauluse ja Siilase läbi öö Beroiasse. Kui nemad olid saabunud, läksid nad juutide kogudusekotta.
\par 11 Need olid üllamad kui Tessaloonika juudid; nemad võtsid sõna vastu kõige hea meelega ja uurisid iga päev Kirjast, kas see on nõnda, nagu kuulutati.
\par 12 Siis uskusid nüüd paljud neist, ja mitte pisut kreeka lugupeetud naisi ja mehi.
\par 13 Aga kui Tessaloonika juudid said teada, et Paulus ka Beroias kuulutas Jumala sõna, tulid nad sinnagi rahvast ässitama ja rahutuks tegema.
\par 14 Siis vennad saatsid Pauluse sedamaid minema mere äärde; aga Siilas ja Timoteos jäid sinna.
\par 15 Ja Pauluse saatjad viisid ta Ateenani, ja kui nad olid saanud Siilase ja Timoteose jaoks käsu, et need nii ruttu kui võimalik tuleksid tema juurde, tulid nad sealt ära.

\section*{Paulus Ateenas}

\par 16 Aga kui Paulus Ateenas neid oli ootamas, ärritus ta vaim temas, kui ta nägi linna ebajumalakujusid täis olevat.
\par 17 Ent ta kõneles nüüd kogudusekojas juutidega ja jumalakartlike inimestega, ja turul iga päev nendega, keda ta kohtas.
\par 18 Ja mõned epikuurlaste ja stoalaste mõttetargad vaidlesid temaga. Ühed ütlesid: „Mida siis see lobasuu tahab öelda?” Aga teised: „Tema näib kuulutavat võõraid vaime!” Sest ta jutlustas neile Jeesusest ja ülestõusmisest.
\par 19 Ja nad võtsid ta kinni ja viisid ta Areopaagi ning ütlesid: „Kas me võiksime teada saada, mis uus õpetus see on, mida sa õpetad?
\par 20 Sest sa tood meie kõrvade ette võõrastavaid asju; sellepärast tahamegi nüüd teada, mis need õige on!”
\par 21 Sest kõik ateenlased ja muulased, kes nende juures elasid, ei olnud millelegi nii himukad kui jutustama ja kuulama uudiseid.

\section*{Pauluse jutlus Areopaagil}

\par 22 Aga Paulus seisis keset Areopaagi ning ütles: „Ateena mehed, ma näen, et te olete igapidi väga jumalakartlikud.
\par 23 Sest ma olen läbi minnes ja teie pühi paiku vaadeldes leidnud ka altari, mille peale oli kirjutatud: „Tundmatule Jumalale!” Keda te nüüd tundmata teenite, teda kuulutan mina teile.
\par 24 Jumal, kes on teinud maailma ja kõik, mis seal sees, tema, kes on taeva ja maa Issand, ei ela templis, mis on kätega tehtud;
\par 25 teda ei teenita ka mitte inimeste kätega, otsekui oleks temale midagi vaja; tema annab ise kõigile elu ja õhu ja kõik.
\par 26 Ja ta on ühest ainsast teinud kogu inimkonna kõige maa peale elama ning neile seadnud ennemääratud ajad ja nende asukohtade rajad,
\par 27 et nad otsiksid Jumalat, kas nad teda ehk saaksid kätega kombata ja leida, ehk tema küll ei ole kaugel mitte ühestki meist;
\par 28 sest tema sees elame ja liigume ja oleme meie, nagu ka mõned teie luuletajaist on öelnud: „Sest ka meie oleme tema sugu!”
\par 29 Kui me nüüd oleme Jumala sugu, siis me ei tohi mõelda, et jumalik olend on kulla või hõbeda või kivi sarnane, niisugune nagu inimese oskuse ja väljamõeldise läbi tekitatud kuju.
\par 30 Jumal on küll niisuguseid teadmatuse aegu sallinud, kuid nüüd ta kuulutab kõigile inimestele kõigis paigus, et nad peavad meelt parandama,
\par 31 sest ta on seadnud päeva, mil ta tahab kohut mõista maailma üle õigluses mehe läbi, kelle ta selleks on määranud, ja pakub usku kõigile pärast seda, kui ta on tema surnuist üles äratanud!”
\par 32 Aga kui nad surnute ülestõusmisest kuulsid, panid mõned seda naeruks, aga teised ütlesid: „Me tahame sind selles asjas kuulda veel teinegi kord!”
\par 33 Nõnda läks Paulus nende keskelt ära.
\par 34 Ent mõned mehed ühinesid temaga ja said usklikuks; nende hulgas oli ka Areopaagi liige Dionüüsios ja üks naine, nimega Daamaris, ja muid ühes nendega.



\chapter{18}

\section*{Paulus Korintoses}

\par 1 Pärast seda läks Paulus Ateenast ära ja tuli Korintosesse.
\par 2 Sealt ta leidis juudi, nimega Akvila; see oli Pontosest pärit ja oli hiljuti tulnud Itaaliast ühes oma naise Priskillaga, sest keiser Klaudius oli käskinud kõigil juutidel Roomast lahkuda. Nende juurde tuli tema.
\par 3 Ja et neil oli ühesugune amet, siis ta jäi nende juurde ja nad tegid ühes tööd; sest nad olid ametilt telgitegijad.
\par 4 Aga ta õpetas kogudusekojas igal hingamispäeval ning äratas juute ja kreeklasi usule.
\par 5 Kui siis Siilas ja Timoteos Makedooniast sinna jõudsid, oli Paulus andunud sõna kuulutamisele ning tunnistas juutidele, et Jeesus on Messias.
\par 6 Aga kui nad vastu panid ja pilkama hakkasid, puistas ta tolmu oma riietelt ning ütles neile: „Teie veri tulgu teie pea peale! Mina olen puhas. Sest ajast ma lähen paganate juurde!”
\par 7 Ja kui ta sealt ära läks, tuli ta ühe jumalakartliku mehe kotta, nimega Justus, kelle maja oli otse kogudusekoja kõrval.
\par 8 Aga kogudusekoja ülem Krispos sai usklikuks Issandasse kogu oma perega ja palju korintlasi sai kuulates usklikuks, ja nad ristiti.
\par 9 Siis Issand ütles öösel nägemuses Paulusele: „Ära karda, vaid kõnele ja ära ole vait!
\par 10 Sest mina olen sinuga ja ükski ei saa su kallale tulla sulle kurja tegema, sest mul on palju rahvast selles linnas!”
\par 11 siis ta viibis seal ühe aasta ja kuus kuud ja õpetas nende seas Jumala sõna.

\section*{Maavalitseja Gallio ja Paulus}

\par 12 Aga kui Gallio oli Ahhaia maavalitseja, tõusid juudid ühel meelel Pauluse vastu ja viisid ta kohtujärje ette
\par 13 ning ütlesid: „See siin meelitab inimesi Jumalat teenima käsuvastaselt!”
\par 14 Aga kui Paulus tahtis suu avada, ütles Gallio juutidele: „Kui oleks tehtud mõni ülekohtune tegu või ilge kuritöö, oh juudid, siis ma oleksin küll kohustatud teid üle kuulama;
\par 15 aga kui on vaidlust sõnade ja nimede ja käsuõpetuse pärast teie seas, siis katsuge ise toime saada; neis asjus ei taha mina olla kohtumõistja!”
\par 16 Ja ta ajas nad ära kohtujärje eest.
\par 17 Nemad aga läksid kõik kogudusekoja ülema Soostenese kallale ja peksid teda kohtujärje ees. Ja Gallio ei hoolinud sellest midagi.

\section*{Paulus läheb tagasi Süüriasse}

\par 18 Aga Paulus viibis seal veel mitu päeva. Siis ta jättis vennad jumalaga ja sõitis laevaga Süüriasse ja ühes temaga Priskilla ja Akvila, pärast seda kui ta Kenkreas oli lasknud oma pea pügada, sellepärast et ta oli andnud tõotuse.
\par 19 Ja nad saabusid Efesosse ja siin ta jättis nad maha, tema ise aga läks kogudusekotta ja vaidles juutidega.
\par 20 Aga kui nad teda palusid kauemaks eneste juurde jääda, ei olnud ta nõus,
\par 21 vaid jättis nad jumalaga ning ütles: „Ma tulen teinekord teie juurde tagasi, kui Jumal tahab!” Ja ta läks Efesosest ära.
\par 22 Ja kui ta Kaisareas oli maabunud, läks ta üles Jeruusalemma, tervitas kogudust ja läks siis alla Antiookiasse.
\par 23 Kui ta seal oli viibinud mõnd aega, läks ta teele ja käis järgemööda läbi Galaatia maakonna ja Früügia, kinnitades kõiki jüngreid.

\section*{Apollos Efesoses}

\par 24 Aga keegi juut, Apollos nimi, Aleksandriast pärit, vahva kõnemees ja vägev kirjaseletaja, tuli Efesosse.
\par 25 Teda oli õpetatud tundma Issanda teed ja ta oli tuline vaimult, rääkis ja õpetas õigesti Jeesusest, aga tundis ainult Johannese ristimist.
\par 26 Tema hakkas julgesti õpetama kogudusekojas. Aga kui Priskilla ja Akvila teda kuulsid, võtsid nad ta eneste juurde ja seletasid talle selgemini ära Jumala tee.
\par 27 Aga kui ta tahtis minna Ahhaiasse, julgustasid vennad teda selleks ning kirjutasid jüngritele, et nad tema vastu võtaksid. Kui ta sinna oli saabunud, saatis ta palju kasu neile, kes armu läbi olid saanud usklikuks.
\par 28 Sest suure osavusega lükkas ta avalikult juutide väited ümber ja näitas Kirjast, et Jeesus on Messias.


\chapter{19}

\section*{Ristija Johannese jüngrid Efesoses}

\par 1 Kui Apollos oli Korintoses, sündis, et Paulus käis läbi ülamaakonnad ja tuli Efesosse ning leidis sealt jüngreid.
\par 2 Ja ta küsis neilt: „Kas te saite Püha Vaimu, kui saite usklikuks?” Aga nemad ütlesid talle: „Me ei olegi kuulnud, kas Püha Vaimu on!”
\par 3 Ja tema ütles: „Mis ristimisega te siis olete ristitud?” Nad vastasid: „Johannese ristimisega!”
\par 4 Aga Paulus ütles: „Johannes ristis meeleparandusristimisega, öeldes rahvale, et nad usuksid temasse, kes pärast teda tuleb, see on Jeesusesse!”
\par 5 Kui nad seda kuulsid, lasksid nad endid ristida Issanda Jeesuse nimesse.
\par 6 Ja kui Paulus pani oma käed nende peale, tuli Püha Vaim nende peale ja nad rääkisid võõraid keeli ja ennustasid.
\par 7 Neid oli üldse ligi kaksteist meest.

\section*{Paulus Efesoses}

\par 8 Ja ta läks kogudusekotta ning kõneles seal avalikult kolm kuud, õpetades ja äratades nõudma Jumala riiki.
\par 9 Aga kui mõned tegid oma südame kõvaks ega uskunud, vaid rääkisid halba õpetusest rahva ees, läks ta ära nende juurest ning eraldas neist jüngrid ja pidas iga päev arutlusi Türannose koolis.
\par 10 See kestis kaks aastat, nii et kõik, kes Aasias elasid, niihästi juudid kui kreeklased, said kuulda Issanda Jeesuse sõna.
\par 11 Ja Jumal tegi iseäralikke imetegusid Pauluse käte läbi,
\par 12 nii et ka tema ihu pealt võetud higirätikuid ja põlli viidi haigete peale ja tõved lahkusid haigetest ning kurjad vaimud läksid välja.
\par 13 Aga mõningad ümberrändajad juudi lausujad hakkasid Issanda Jeesuse nime nimetama nende peale, kelles oli kurje vaime, öeldes: „Ma vannutan teid Jeesuse nimel, keda Paulus kuulutab!”
\par 14 Ja need olid ühe juudi ülempreestri Skeua seitse poega, kes seda tegid.
\par 15 Ent kuri vaim vastas ning ütles neile: „Jeesust ma tunnen ja Paulust ma tean, aga kes teie olete?”
\par 16 Ja inimene, kelles oli kuri vaim, kargas nende kallale, sai võimust nende üle ja võitis nad ära, nõnda et nad alasti ja haavatult põgenesid sellest kojast.
\par 17 Ja see sai teatavaks kõigile juutidele ja kreeklastele, kes elasid Efesoses. Ja hirm tuli nende kõikide peale, ja Issanda Jeesuse nime ülistati väga.
\par 18 Ja paljud neist, kes olid usklikuks saanud, tulid ning tunnistasid ja andsid üles oma teod.
\par 19 Aga mitmed neist, kes olid tegemist teinud nõiakunstiga, tõid kokku oma raamatud ja põletasid need ära kõikide nähes. Ja kui nende hind kokku arvati, leiti see olevat viiskümmend tuhat hõbetükki.
\par 20 Nii võimsasti kasvas Issanda sõna ja avaldas oma väge.

\section*{Pauluse kavatsused}

\par 21 Kui see kõik oli sündinud, võttis Paulus vaimus ette minna Makedoonia ja Ahhaia kaudu Jeruusalemma ning ütles: „Kui ma seal olen ära käinud, pean ma näha saama ka Rooma linna!”
\par 22 Ja ta läkitas kaks oma abilist, Timoteose ja Erastose, Makedooniasse ja jäi ise veel mõneks ajaks Aasiasse.

\section*{Möll Efesoses}

\par 23 Sel ajal tekkis suur tüli õpetuse pärast.
\par 24 Sest üks hõbesepp, Demeetrios nimi, tegi hõbedast Artemise templeid ja saatis sellega ametimeestele palju kasu.
\par 25 Need ta kogus kokku ja teisi, kes seda tööd tegid, ning ütles: „Mehed, te teate, et meil sellest tööst on hea tulu,
\par 26 ja te näete ning kuulete, et see Paulus mitte ainult Efesoses, vaid ka peaaegu kogu Aasias palju rahvast ära meelitab ja eksitab, öeldes, et need ei olevat jumalad, mis kätega tehakse.
\par 27 Ent mitte ükspäinis seda meie tööstust ei ähvarda halvakspanu, vaid ka suure jumalanna Artemise templit, nii et seda mikski ei panda ja et kaob ka jumalanna suur au, keda kummardavad kogu Aasia ja kõik maailm!”
\par 28 Kui nad seda kuulsid, said nad täis viha, kisendasid ning ütlesid: „Suur on efeslaste Artemis!”
\par 29 Ja kõik linn sai täis kära ja nad tormasid ühel meelel teatrisse ja vedasid ühes sinna Gaajuse ja Aristarhose, kaks makedoonlast, kes olid Pauluse matkakaaslased.
\par 30 Ent kui Paulus tahtis rahva sekka minna, ei lasknud jüngrid teda mitte.
\par 31 Ka mõned Aasiamaa ülemad, kes olid tema sõbrad, läkitasid tema juurde ning palusid teda, et ta ei läheks teatriväljakule.
\par 32 Seal kisendasid ühed seda, teised teist; sest rahvakogu oli segaduses ja suurem hulk ei teadnud, misjaoks nad olid kokku tulnud.
\par 33 Siis toodi rahva seast esile Aleksandros, kelle juudid ette lükkasid. Aleksandros viitas käega, märku andes, et ta tahab pidada kaitsekõnet rahva ees.
\par 34 Aga kui nad märkasid, et ta on juut, tõstsid nad kisa ja karjusid kõik nagu ühest suust ligi kaks tundi aega: „Suur on efeslaste Artemis!”
\par 35 Viimaks linnakirjutaja vaigistas rahva ning ütles: „Efesose mehed, kas on ühtki inimest, kes ei teaks, et efeslaste linn on suure Artemise ja tema taevast mahalangenud kuju hoidja?
\par 36 Et nüüd ükski sellele ei saa vastu rääkida, siis tuleb teil jääda vait ega tule midagi ette võtta uisapäisa.
\par 37 Sest te olete toonud siia need mehed, kes ei ole templi rüüstajad ega teie jumalanna teotajad.
\par 38 Kui nüüd Demeetriosel ja temaga ühes olevail ametimeestel on kellegi peale midagi kaebamist, siis on selleks kohtupäevi ja maavalitsejaid; kaevaku nad üksteist kohtusse.
\par 39 Ja kui teil on veel mingit muud nõudmist, siis seletatagu asi korrapärase rahvakogu ees.
\par 40 Sest meil tuleb tänase kära pärast karta, et meie peale kaebus tõstetakse, ilma et meil oleks mingit põhjust, millega seda rahva kokkujooksu õigustada!” Kui ta seda oli öelnud, saatis ta rahvakogu laiali.


\chapter{20}

\section*{Paulus Makedoonias, Kreekas ja Troas}

\par 1 Kui kära oli vaigistatud, kutsus Paulus jüngrid enese juurde, jättis nad jumalaga ja läks teele Makedooniasse.
\par 2 Ja kui ta need maakohad oli läbi käinud ja seal palju manitsussõnu rääkinud, tuli ta Kreekamaale.
\par 3 Seal ta viibis kolm kuud. Kui ta siis mõtles mereteed minna Süüriasse ja juudid salanõu pidasid ta vastu, võttis ta nõuks tagasi matkata Makedoonia kaudu.
\par 4 Teda saatsid beroialane Soopatros, Pürrose poeg, tessalooniklastest Aristarhos ja Sekundus, ja derbelane Gaajus, ja Timoteos, ja aasialased Tühhikos ja Trofimos.
\par 5 Need läksid ära eele ja ootasid meid Troas.
\par 6 Meie aga läksime laevaga pärast hapnemata leibade päevi Filipist teele ja saabusime nende juurde Troasse viiendal päeval, ja jäime sinna seitsmeks päevaks.

\section*{Eutühhos}

\par 7 Aga kui me esimesel nädalapäeval olime kokku tulnud leiba murdma, siis Paulus, kes tahtis teisel päeval ära minna, kõneles nendega ja jätkas oma kõnet keskööni.
\par 8 Ent mitu lampi põles ülemises toas, kus me koos olime.
\par 9 Aga üks noor mees, Eutühhos nimi, istus aknal; ja kui Paulus kauemini kõneles, tuli raske uni tema peale ja ta kukkus magades kolmandalt korralt alla ja tõsteti üles surnuna.
\par 10 Aga Paulus läks alla ja heitis tema peale, haaras ta ümbert kinni ning ütles: „Ärge hädaldage, sest tal on hing veel sees!”
\par 11 Siis ta läks üles ja murdis leiba ja sõi ja kõneles kaua kuni koiduni ja läks siis teele.
\par 12 Aga noor mees viidi minema elusana ja oldi suuresti trööstitud.

\section*{Teekond Mileetosesse}

\par 13 Me läksime eele laeva ja sõitsime Assosesse, kus mõtlesime oodata Paulust laeva, sest nõnda oli ta käskinud, ette võttes ise jala tulla.
\par 14 Kui ta nüüd Assoses meiega kokku sai, võtsime ta laeva ja tulime Mitüleenesse.
\par 15 Sealt me purjetasime edasi ja jõudsime järgmisel päeval Kiose saare kohta ja peatusime teisel päeval Saamoses ja saabusime järgmisel päeval Mileetosesse.
\par 16 Sest Paulusel oli nõu Efesosest mööda sõita, et mitte aega viita Aasias. Tõttas ta ju, et kui vähegi võimalik, nelipühapäevaks jõuda Jeruusalemma.

\section*{Pauluse jumalagajätukõne Efesose koguduse vanemaile}

\par 17 Aga Mileetosest ta läkitas sõna Efesosse ja kutsus koguduse vanemad enese juurde.
\par 18 Ja kui nad tema juurde tulid, ütles ta neile: „Te teate, kuidas ma esimesest päevast, kui ma tulin Aasiasse, kõige selle aja olin teie juures,
\par 19 teenides Issandat kõige alanduse ja pisarate ja kiusatustega, mis mind tabasid juutide salanõu tõttu;
\par 20 kuidas ma teie eest ei pidanud salajas midagi, mis teile tarvis oli, et oleksin selle jätnud teile kuulutamata ega oleks teid õpetanud avalikult ja kodasid mööda
\par 21 ja tunnistanud nii juutidele kui kreeklastele pöördumist Jumala poole ja usku meie Issandasse Jeesusesse.
\par 22 Ja nüüd, vaata, ma lähen vaimus seotuna Jeruusalemma ega tea, mis mind seal võib tabada;
\par 23 kuid seda ma tean, et Püha Vaim igas linnas mulle tunnistab ning ütleb, et köidikud ja viletsused ootavad mind.
\par 24 Ometi ma ei hooli sellest midagi ega pea ka oma elu kalliks, kui ma aga lõpetan oma elukorra ja ametikohused, mis ma olen saanud Issanda Jeesuse käest, et tunnistada Jumala armu evangeeliumi.
\par 25 Ja nüüd, vaata, ma tean, et te iialgi enam ei saa minu silmi näha, te kõik, kelle keskel ma olen käinud ja kuulutanud Jumala riiki.
\par 26 Sellepärast tunnistan ma teile tänasel päeval, et ma olen puhas kõikide verest;
\par 27 sest ma pole midagi teie eest salanud Jumala tahtest, et ma teile seda tervelt ei oleks kuulutanud.
\par 28 Sellepärast pange tähele iseendid ja kõike karja, kellele Püha Vaim on pannud teid ülevaatajaiks karjastena hoidma Jumala kogudust, mille ta on omandanud iseenese vere läbi.
\par 29 Ma tean seda, et pärast minu äraminekut tulevad teie sekka hirmsad hundid, kes karjale armu ei anna;
\par 30 ja teie eneste seast tõusevad mehed, kes kõnelevad pööraseid asju, et vedada jüngreid eneste järele.
\par 31 Sellepärast valvake ja mõelge sellele, et ma kolm aastat ööd ja päevad ei ole lakanud silmaveega manitsemast igaüht teie seast.
\par 32 Ja nüüd ma annan teid, vennad, Jumala ja tema armusõna hooleks, kes on vägev teid üles ehitama ja andma teile pärandi kõikide seas, kes on pühitsetud.
\par 33 Hõbedat ega kulda ega riietust ei ole ma ihaldanud.
\par 34 Te teate ise, et need mu käed on teinud tööd minu ja mu kaaslaste vajaduste rahuldamiseks.
\par 35 Mina olen teile kõigiti näidanud, et nõnda tööd tehes tuleb hoolt kanda nõrkade eest ja meeles pidada Issanda Jeesuse sõnu, mis ta on öelnud: õndsam on anda kui võtta!”
\par 36 Ja kui ta seda oli öelnud, langes ta põlvili ja palvetas ühes nende kõikidega.
\par 37 Siis puhkesid kõik valjusti nutma ja hakkasid Paulusel ümber kaela ja andsid temale suud.
\par 38 Kõige rohkem kurvastas neid sõna, mis ta oli öelnud, et nad enam ei saavat näha tema palet. Siis nad saatsid ta laeva.


\chapter{21}

\section*{Paulus Tüüroses}

\par 1 Aga kui me neist olime lahkunud ja laevaga teele läinud, purjetasime otsekohe Koossi, aga teisel päeval Roodosesse ja sealt Patarasse.
\par 2 Seal me leidsime laeva, mis läks Foiniikiasse, astusime sellesse ja purjetasime minema.
\par 3 Kui Küpros meile hakkas paistma, jätsime selle vasakut kätt ja purjetasime Süüriasse ning maabusime Tüüroses, sest seal pidi laev koormast tühjendatama.
\par 4 Seal me kohtasime jüngreid ning jäime nende juurde seitsmeks päevaks. Nemad ütlesid Paulusele Vaimu mõjul, et ta ei läheks Jeruusalemma.
\par 5 Aga kui meie aeg lõppes, läksime teele. Ja nad kõik naiste ja lastega saatsid meid linnast välja, ja me heitsime põlvili ranna äärde ning palvetasime.
\par 6 Ja kui me olime üksteise jätnud jumalaga, astusime laeva; aga nemad pöördusid tagasi koju.

\section*{Paulus Kaisareas}

\par 7 Tüürosest me sõitsime Ptolemaisse ja lõpetasime seal oma laevasõidu; me teretasime vendi ja jäime nende juurde üheks päevaks.
\par 8 Aga järgmisel päeval läksime sealt välja ja tulime Kaisareasse; me läksime evangelist Filippuse kotta, kes oli üks neist seitsmest, ja jäime tema juurde.
\par 9 Temal oli neli tütart; need olid neitsid ja ennustasid.
\par 10 Aga kui me mõned päevad seal viibisime, tuli sinna prohvet Judeast, Aagabus nimi.
\par 11 Tema tuli meie juurde ja võttis Pauluse vöö, sidus oma käed ja jalad kinni ja ütles: „Seda ütleb Püha Vaim: mehe, kelle oma on see vöö, seovad juudid Jeruusalemmas nõnda kinni ja annavad ta paganate kätte!”
\par 12 Kui me seda kuulsime, palusime teda meie ja ka teised, kes seal elasid, et ta ei läheks Jeruusalemma.
\par 13 Siis Paulus vastas: „Mis te teete, et nutate ja rõhute minu südant? Sest ma pole mitte üksnes valmis laskma ennast siduda, vaid ka surema Jeruusalemmas Issanda Jeesuse nime pärast!”
\par 14 Aga kui ta ei võtnud meid kuulda, jäime rahule ja ütlesime: „Issanda tahtmine sündigu!”

\section*{Paulus Jeruusalemmas}

\par 15 Pärast neid päevi me valmistusime teele ja läksime Jeruusalemma.
\par 16 Ja meiega ühes tulid mõned jüngrid Kaisareast, kes meid viisid ühe vana jüngri Mnaasoni, Küprose mehe juurde, kelle külaliseks me pidime jääma.
\par 17 Ja kui me jõudsime Jeruusalemma, võtsid vennad meid heal meelel vastu.

\section*{Paulus lepitab juutkristlasi}

\par 18 Järgmisel päeval läks Paulus meiega Jakoobuse juurde. Ja kõik vanemad tulid sinna.
\par 19 Ja kui ta neid oli teretanud, jutustas ta järgemööda kõik, mis Jumal oli teinud paganate seas tema ameti kaudu.
\par 20 Seda kuuldes nad andsid Jumalale au ning ütlesid talle: „Sa näed, vend, kui mitu tuhat juuti on, kes on saanud usklikuks, ja need kõik peavad kangesti käsu kombeid.
\par 21 Aga nad on saanud kuulda sinust, et sina õpetavat juute, kes elavad paganate seas, taganema Moosesest öeldes, et ei ole vaja lapsi ümber lõigata ega elada usukommete järgi.
\par 22 Mis nüüd teha? Igatahes nad saavad kuulda, et sa oled tulnud.
\par 23 Siis tee nüüd seda, mis me sulle ütleme: meil on neli meest, kes on võtnud eneste peale tõotuse.
\par 24 Need võta enese juurde ning puhasta ennast ühes nendega ja kanna nende eest nende kulud, et nad oma pea pügaksid; siis märkavad kõik, et see on tühi, mis nad sinust on kuulnud, vaid et ka sina ise elad käsku pidades.
\par 25 Aga paganate kohta, kes on saanud usklikuks, oleme otsustanud ja läkitanud kirja, et neil on tarvis hoiduda ebajumala ohvrilihast ja verest ja lämbunu lihast ja hooraelust!”
\par 26 Siis võttis Paulus mehed enese juurde ja järgmisel päeval ta laskis ennast ühes nendega puhastada, läks pühakotta ja teatas, et puhastuspäevad on lõppenud, ja jäi sinna, kuni nende igaühe eest oli ohver toodud.

\section*{Möll Jeruusalemmas ja Pauluse vangistamine}

\par 27 Aga kui need seitse päeva hakkasid lõpule jõudma, nägid teda Aasiast olevad juudid pühakojas ja ajasid kõik rahva kihama ning pistsid oma käed tema külge
\par 28 ja kisendasid: „Iisraeli mehed, tulge appi! Siin on see mees, kes igal pool õpetab kõiki inimesi meie rahva ja käsu ja selle paiga vastu; pealegi on ta toonud kreeklasi pühakotta ja selle paiga rüvetanud!”
\par 29 Sest nad olid enne näinud temaga linnas olevat efeslase Trofimose ja arvasid temast, et Paulus oli ta toonud pühakotta.
\par 30 Ja kogu linn tõusis liikvele ning rahvast jooksis kokku. Ja nad võtsid Pauluse kinni ning tõmbasid ta pühakojast välja, ja uksed pandi kohe lukku.
\par 31 Juba nad tahtsid teda tappa, kui ülempealikule teada anti, et kõik Jeruusalemm on kihamas.
\par 32 See võttis sedamaid sõjamehi ja pealikuid enesega ja tuli joostes nende juurde. Kui nad ülempealikut ja sõjamehi nägid, lakkasid nad Paulust peksmast.
\par 33 Siis astus ülempealik ette, võttis ta oma hoole alla, käskis ta kahe ahelaga kinni siduda ja küsis, kes ta on ja mis ta on teinud.
\par 34 Aga rahva seast kisendasid ühed seda, teised teist. Ja et ta ei võinud õieti asja teada saada kära pärast, käskis ta tema viia kindlusesse.
\par 35 Kui aga Paulus jõudis trepile, pidid sõjamehed teda kandma rahvatungi pärast,
\par 36 sest suur rahvahulk käis järel ja kisendas: „Hukka ta ära!”
\par 37 Ja kui Paulust juba oldi viimas kindlusesse, ütles ta ülempealikule: „Kas mul on luba midagi sulle rääkida?” Aga tema ütles: „Kas sa oskad kreeka keelt?
\par 38 Kas sina ei olegi see egiptlane, kes mõni aeg tagasi mässu tõstis ja viis välja kõrbe neli tuhat mõrtsukat?”
\par 39 Paulus ütles: „Mina olen juut Tarsosest, kuulsa Kiliikia linna kodanik; aga ma palun sind, anna mulle luba rahvale kõnelda!”
\par 40 Kui ta andis loa, astus Paulus astmeile ning andis käega rahvale märku. Kui kõik vait jäid, hakkas ta kõnelema heebrea keeli ning ütles:


\chapter{22}

\section*{Pauluse jutlus juutidele}

\par 1 „Mehed, vennad ja vanemad, kuulge nüüd, mida ma enese kaitseks teile räägin!”
\par 2 Kui nad kuulsid, et ta neile räägib heebrea keeli, jäid nad veel vaiksemaks. Ja tema ütles:
\par 3 „Mina olen Juuda mees, sündinud Tarsoses Kiliikiamaal ja kasvatatud siin selles linnas Gamaalieli jalgade ees, õpetatud kõige rangemaks esiisade käskude täitmiseks; ja ma olin agar Jumala nõudja, nõnda nagu teie kõik tänapäeval olete.
\par 4 Ja sellisena ma kiusasin taga seda õpetust vereni, köites ja viies vangi niihästi mehi kui naisi,
\par 5 nõnda nagu mulle võivad anda tunnistust ka ülempreester ja kogu vanematekogu, kellelt ma sain ka kirju viimiseks vendadele Damaskusesse; ja ma olin sinna minemas kinni siduma ka neid, kes seal olid, ja Jeruusalemma viima, et neid nuheldaks.
\par 6 Aga sündis, kui ma teel olin ja Damaskuse lähedale jõudsin lõunaajal, et äkitselt paistis mu ümber suur valgus taevast!
\par 7 Ma kukkusin maha ja kuulsin häält mulle ütlevat: Saul! Saul! Miks sa mind taga kiusad?
\par 8 Mina aga vastasin: kes sa oled, isand? Ja ta ütles mulle: mina olen Jeesus Naatsaretlane, keda sa taga kiusad!
\par 9 Need aga, kes olid minuga, nägid küll valgust ja kartsid, kuid selle häält, kes minuga rääkis, nad ei kuulnud!
\par 10 Siis ma ütlesin: Issand, mida ma pean tegema? Aga Issand ütles mulle: tõuse üles, mine Damaskusesse ja seal sulle öeldakse kõik, mis sulle on määratud teha.
\par 11 Ja kui ma selle heleda valguse pärast ei võinud näha, talutasid mind kättpidi mu kaasasolijad, ja ma jõudsin Damaskusesse.
\par 12 Siis keegi Ananias, käsuõpetuselt jumalakartlik mees, kellel oli hea tunnistus kõigi sealsete juutide poolt,
\par 13 tuli mu juurde, astus ligi ja ütles mulle: Saul, vend, näe jälle! Ja selsamal tunnil ma nägin jälle ning vaatasin talle otsa.
\par 14 Tema aga ütles: meie esiisade Jumal on sind enne valinud tundma tema tahtmist ja nägema seda Õiget ja kuulma häält tema suust,
\par 15 sest sa pead olema tema tunnistaja kõigi inimeste ees neis asjus, mida sa oled näinud ja kuulnud.
\par 16 Ja mida sa nüüd ootad? Tõuse üles ja lase ennast ristida ja oma patud ära pesta, appi hüüdes tema nime.
\par 17 Kui ma siis jälle tulin Jeruusalemma ja pühakojas palvetasin, sündis minuga, et ma otsekui enesest ära olin
\par 18 ja nägin teda. Ja tema ütles mulle: tõtta ja mine kohe ära Jeruusalemmast, sellepärast et nad sinu tunnistust minu kohta vastu ei võta!
\par 19 Ja mina ütlesin: Issand, nad teavad seda ise, et ma viisin vangi ja peksin kogudusekodades neid, kes uskusid sinusse.
\par 20 Ja kui Stefanose, su tunnistaja veri ära valati, siis minagi juures olles kiitsin seda heaks ja hoidsin nende riideid, kes teda tapsid.
\par 21 Ja tema ütles mulle: mine, sest ma tahan sind läkitada kaugele paganate sekka!”

\section*{Paulus toetub oma Rooma kodakondsusele}

\par 22 Sellest sõnast saadik nad kuulasid teda, ja siis nad tõstsid oma häält ning ütlesid: „Hukka niisugune ära maa pealt, tema ei tohi ellu jääda!”
\par 23 Ja kui nad kisendasid ja riideid seljast kiskusid ning tolmu vastu taevast viskasid,
\par 24 käskis ülempealik teda viia kindlusesse ning ütles, et teda tuleb üle kuulata ja piitsa abil kätte saada, mis süü pärast nad nõnda tema peale kisendasid.
\par 25 Aga kui nad tema olid maha sirutanud, et teda rihmadega peksta, ütles Paulus pealikule, kes seal juures seisis: „Kas teil on luba peksta Rooma kodanikku ja pealegi ilma kohtu otsuseta?”
\par 26 Kui pealik seda kuulis, läks ta ja teatas seda ülempealikule ning ütles: „Mis sa nüüd teed? See inimene on ju Rooma kodanik!”
\par 27 Siis tuli ülempealik Pauluse juurde ning ütles: „Ütle mulle, kas sa oled Rooma kodanik?” Tema ütles: „Olen küll!”
\par 28 Ja ülempealik vastas: „Ma olen suure rahaga nõutanud enesele selle Rooma kodanikuõiguse!” Ent Paulus ütles: „Aga mina olen sellena sündinud!”
\par 29 Siis läksid sedamaid tema juurest ära need, kes teda pidid piitsa abil üle kuulama. Ja ülempealik kartis, kui ta teada sai, et ta on Rooma kodanik, ja et ta tema oli kinni sidunud.

\section*{Paulus kaitseb ennast Suurkohtu ees}

\par 30 Aga järgmisel päeval, tahtes õieti teada saada, mis asja pärast juudid Paulust süüdistasid, päästis ta tema köidikuist lahti ja käskis ülempreestreid ja kogu nende Suurkohtu kokku tulla, viis Pauluse sinna ja seadis ta nende ette.


\chapter{23}

\section*{Paulus kaitseb ennast Suurkohtu ees}

\par 1 Aga Paulus pööras oma silmad Suurkohtu poole ning ütles: „Mehed, vennad, ma olen täiesti hea südametunnistusega Jumala ees elanud tänase päevani!”
\par 2 Aga ülempreester Ananias käskis neid, kes ta lähedal seisid, lüüa temale vastu suud.
\par 3 Siis Paulus ütles talle: „Küll Jumal sind lööb, sa lubjatud sein! Sina istud ja mõistad kohut minu üle käsku mööda ja käsid mind lüüa käsu vastu?”
\par 4 Lähedal seisjad aga ütlesid: „Kas sa teotad Jumala ülempreestrit?”
\par 5 Paulus ütles: „Vennad, ma ei teadnud, et ta on ülempreester, sest on kirjutatud: oma rahva ülemat ära sajata!”
\par 6 Kuna aga Paulus teadis, et üks osa neist oli sadusere ja teine varisere, siis ta hüüdis Suurkohtu ees: „Mehed, vennad, ma olen variser ja variseri poeg, lootuse ja surnute ülestõusmise pärast olen ma kohtu ees!”
\par 7 Kui ta seda oli rääkinud, tõusis kära variseride ja saduseride vahel ja nende hulk läks lahku.
\par 8 Sest saduserid ei ütle ülestõusmist olevat, samuti ingleid ja vaime; aga variserid tunnustavad mõlemaid.
\par 9 Siis tõusis suur karjumine, ja kirjatundjad variseride hulgast tõusid ja vaidlesid kangesti ning ütlesid: „Me ei leia ühtki paha sellest inimesest! Ehk aga on vaim või ingel temaga rääkinud?”
\par 10 Aga kui kära läks suureks, kartis ülempealik, et nad Pauluse lõhki kisuvad, ja käskis sõjaväe alla minna ja ta ära kiskuda nende seast ja viia kindlusesse.
\par 11 Aga järgmisel ööl seisis Issand ta juures ning ütles: „Ole julge, Paulus! Sest otsekui sa neid asju minust oled tunnistanud Jeruusalemmas, nõnda pead sa tunnistama ka Roomas!”

\section*{Juutide vandenõu Pauluse vastu}

\par 12 Kui nüüd valgeks läks, heitsid mõned juudid ühte nõusse ja andsid isekeskis vande, et nad ei söö ega joo, enne kui nad on Pauluse tapnud.
\par 13 Ja neid oli rohkem kui nelikümmend, kes nõnda olid heitnud vandeliitu.
\par 14 Need läksid ülempreestrite ja vanemate juurde ning ütlesid: „Me oleme needes vandunud mitte midagi suhu võtta, enne kui oleme tapnud Pauluse!
\par 15 Pange siis teie nüüd ühes Suurkohtuga ülempealikule ette, et ta homme tema tooks teie ette, otsekui tahaksite teie ta asja paremini üle kuulata; aga meie oleme valmis teda tapma, enne kui ta saab teie ette!”
\par 16 Aga Pauluse õepoeg sai kuulda varitsemisnõust ja läks alla ning tuli kindlusesse ja teatas seda Paulusele.
\par 17 Siis Paulus kutsus enese juurde ühe pealikuist ja ütles: „Vii see noormees ülempealiku juurde, sest tal on midagi temale öelda!”
\par 18 See võttis siis tema ja saatis ta ülempealiku juurde ja ütles: „Vang Paulus kutsus mind enese juurde ja palus noormehe tuua sinu juurde, sest tal on midagi sulle rääkida!”
\par 19 Aga ülempealik võttis ta kättpidi ja viis ta kõrvale ning küsis temalt: „Mis sul on mulle teatada?”
\par 20 Tema ütles: „Juudid on heitnud ühte nõusse sind paluda, et sa Pauluse homme laseksid viia Suurkohtu ette, otsekui sa mõtleksid paremini üle kuulata tema asja.
\par 21 Aga sina ära usu neid, sest rohkem kui nelikümmend meest nende seast varitsevad teda; need on vandunud mitte enne süüa ja juua kui nad tema on tapnud; ja nüüd on nad valmis ning ootavad sinu luba!”
\par 22 Siis laskis ülempealik noormehe ära minna ja keelas teda kellelegi rääkimast, et ta seda oli temale teada andnud.

\section*{Paulus saadetakse Kaisareasse}

\par 23 Ja ta kutsus enese juurde kaks pealikuist ning ütles: „Seadke valmis kakssada sõjameest Kaisareasse minemiseks ja seitsekümmend ratsameest ja kakssada piigimeest kolmandal öötunnil,
\par 24 samuti pidage ratsaloomad valmis, et ühe selga panna Paulus ja tema tervelt viia maavalitseja Feeliksi juurde.”
\par 25 Ka kirjutas ta kirja nende sõnadega:
\par 26 „Klaudius Lüüsias läkitab aulikule maavalitsejale Feeliksile palju tervist!
\par 27 Selle mehe olid juudid kinni võtnud ja tahtsid teda tappa; siis tulin mina sõjaväega vahele ja võtsin ta ära nende käest ja sain teada, et ta on Rooma kodanik.
\par 28 Ja et ma tahtsin teada saada, mis süü pärast nad ta peale kaebasid, viisin ma ta nende Suurkohtu ette
\par 29 ja leidsin, et ta peale kaevati nende käsuõpetusse puutuvate vaidluste pärast; aga ükski kaebus ei olnud niisugune, mis oleks olnud surma või vangistust väärt.
\par 30 Aga kui mulle teada anti, et juutidel on kaval nõu selle mehe vastu, saatsin ma tema sedamaid sinu juurde ja käskisin ka kaebajaid sinu ees kõnelda tema vastu!”
\par 31 Vastavalt saadud käsule võtsid siis sõjamehed Pauluse ja viisid ta ära läbi öö Antipatrisesse.
\par 32 Aga järgmisel päeval lasksid nad ratsamehed temaga edasi minna ja pöördusid ise tagasi kindlusesse.
\par 33 Kui need Kaisareasse jõudsid, andsid nad kirja ära maavalitseja kätte ja viisid ka Pauluse tema ette.
\par 34 Kui maavalitseja oli kirja lugenud, küsis ta, kust maakonnast see mees on, ja teada saades, et ta on Kiliikiast,
\par 35 ütles ta: „Ma tahan sind üle kuulata, kui su kaebajad ka siia jõuavad!” Ja ta käskis teda hoida Heroodese peavahis.


\chapter{24}

\section*{Paulus maavalitseja Feeliksi ees}

\par 1 Aga viie päeva pärast tuli alla ülempreester Ananias mõnede vanematega ja ühe kõnemehe Tertullusega, ja nad tõstsid maavalitseja ees kaebust Pauluse peale.
\par 2 Kui Paulus ette kutsuti, algas Tertullus oma kaebekõnet ning ütles:
\par 3 „Me oleme sinu kaudu, auline Feeliks, saanud maitsta rahu ja sinu hoolekandel on palju parandusi osaks saanud sellele rahvale. Seda me võtame kõikepidi ja kõigis paigus vastu suure tänuga.
\par 4 Aga et ma sind kauemini ei tülitaks, siis palun sind meid oma lahkust mööda lühidalt kuulata.
\par 5 Me oleme leidnud selle mehe kahjuliku olevat ja naatsaretlaste usulahu eestvedajana juute mässule kihutavat igal pool kogu maailmas.
\par 6 Ta on püüdnud rüvetada isegi pühakoda. Tema me võtsime kinni, [et tema üle kohut mõista meie käsuõpetuse järgi.
\par 7 Kuid ülempealik Lüüsias tuli ning võttis ta suure vägivallaga meie käest ära
\par 8 ja käskis tema süüdistajaid tulla sinu juurde.] Sa võid ise teda nüüd üle kuulata ja teada saada kõike, milles me teda süüdistame!”
\par 9 Ja juudid ühinesid sellega, tõendades, et see nõnda on.
\par 10 Kui siis maavalitseja Paulusele käega märku andis, et ta räägiks, vastas ta: „Et ma sind mitu aastat kui selle rahva kohtunikku tunnen, siis tahan seda julgemini kosta enese eest.
\par 11 Nagu sa võid teada saada, ei ole rohkem kui kaksteist päeva sellest, kui ma läksin üles Jeruusalemma Jumalat kummardama.
\par 12 Ei nad ole leidnud mind ühegi inimesega kõnelemast ega rahva seas tüli tõstmast, ei pühakojas, ei kogudusekodades ega linnas;
\par 13 nad ei või ka tõeks teha seda, milles nad nüüd mind süüdistavad.
\par 14 Aga seda ma tunnistan sulle, et ma seda õpetust mööda, mida nad kutsuvad usulahuks, nõnda teenin oma esiisade Jumalat, et ma usun kõike, mis käsuõpetuses ja prohvetite raamatutes on kirjutatud,
\par 15 ja et mul on see lootus Jumala peale, mida ka nemad ise ootavad, et tuleb õigete ja ülekohtuste ülestõusmine.
\par 16 Ja selles ma püüan alati hoida puhast südametunnistust Jumala ja inimeste ees.
\par 17 Aga nüüd ma tulin mitme aasta pärast tooma annetusi oma rahvale ja ohvriande.
\par 18 Seda tegemast leidsid mind mõned juudid Aasiamaalt, kui ma pühakojas ennast puhastasin ega andnud põhjust rahvamurruks või käratsemiseks.
\par 19 Need peaksid siin sinu ees olema ja kaebama, kui neil midagi oleks mu vastu.
\par 20 Või öelgu need siinolijad ise, kas nad on minust leidnud mingisugust süüd, kui ma seisin Suurkohtu ees,
\par 21 olgu siis selle ainsa sõna, et ma nende seas seistes hüüdsin: surnute ülestõusmise pärast te mõistate täna kohut minu üle!”

\section*{Feeliks ei kiirusta kohtupidamisega}

\par 22 Aga Feeliks lükkas asja edasi, sest ta teadis väga hästi, kuidas selle õpetusega lugu on, ja ütles: „Kui ülempealik Lüüsias tuleb siia, siis ma uurin teie asja!”
\par 23 Ja ta käskis pealikut hoida teda vangis, aga kerge vahi all ja mitte takistada omi temale abiks olemast või tema juurde tulemast.
\par 24 mõne päeva pärast tuli Feeliks ühes oma naise Drusillaga, kes oli juut, ja kutsus Pauluse enese ette ning kuulas ta kõnet usust Kristusesse Jeesusesse.
\par 25 Aga kui Paulus kõneles õigusest ja kasinusest ja tulevasest kohtust, ehmus Feeliks ja kostis: „Mine seks puhuks ära; kui mul aega on, kutsun ma sind jälle!”
\par 26 Ühtlasi ta ka lootis Pauluselt raha saada, mistõttu ta teda ka sagedasti kutsus enese juurde ja kõneles temaga.
\par 27 Aga kui kaks aastat täis sai, tuli Porkius Festus Feeliksi asemele; ja et Feeliks tahtis teha juutidele meelehead, siis jättis ta Pauluse vangi.

\chapter{25}

\section*{Paulus maavalitseja Festuse ees}

\par 1 Kui nüüd Festus oli astunud maavalitseja ametisse, läks ta kolme päeva pärast Kaisareast Jeruusalemma.
\par 2 Ja ülempreestrid ja juutide peamehed esitasid temale kaebuse Pauluse vastu ning pöördusid tema poole,
\par 3 paludes Paulusele seda armu, et ta kutsuks tema Jeruusalemma, sest nad olid nõuks võtnud teda varitseda ja teel ära tappa.
\par 4 Aga Festus vastas, et Paulust peetakse kinni Kaisareas ja et ta ise mõtleb peatselt sinna minna.
\par 5 „Kes nüüd teie seast”, ütles ta, „on mõjumehed, need tulgu ühes minuga alla, ja kui sellel mehel on süüd, kaevaku nad tema peale!”
\par 6 Ja kui ta mitte rohkem kui kaheksa või kümme päeva nende juures oli viibinud, läks ta alla Kaisareasse. Ja järgmisel päeval istus ta kohtujärjele ja käskis Pauluse ette tuua.
\par 7 Aga kui ta ette astus, asusid Jeruusalemmast tulnud juudid tema ümber ja tõstsid Pauluse vastu mitu rasket kaebust, mida nad ei suutnud tõeks teha.
\par 8 Sest Paulus kaitses ennast, öeldes: „Ei ma ole kuidagi viisi eksinud ei juutide käsu ega pühakoja vastu ega keisri vastu!”
\par 9 Aga Festus tahtis juutidele teha meelehead ning vastas Paulusele nõnda: „Kas sa tahad minna Jeruusalemma ja seal minu ees lasta enese üle kohut mõista neis asjus?”
\par 10 Paulus ütles: „Ma seisan keisri kohtujärje ees, kus mu üle peab kohut mõistetama; juutidele ei ole ma teinud ühtki ülekohut, nõnda nagu sa väga hästi tead.
\par 11 Sest kui ma midagi ülekohut olen teinud ja seda, mis on surma väärt, siis ma ei tõrgu suremast; aga kui selles, milles need mind süüdistavad, ei ole mitte midagi, siis ei või ükski mind nende kätte anda; mina nõuan keisri kohut!”
\par 12 Siis kõneles Festus Suurkohtu esindajatega ja vastas: „Keisri kohut oled sa nõudnud, keisri ette sa lähed!”

\section*{Agrippas ja Berniike}

\par 13 Aga mõne päeva pärast tulid kuningas Agrippas ja Berniike Kaisareasse Festust tervitama.
\par 14 Ja kui nad seal mitu päeva olid viibinud, jutustas Festus kuningale Pauluse asjast ning ütles: „Feeliks on siia kinni jätnud ühe mehe,
\par 15 kelle pärast minu Jeruusalemmas olles ülempreestrid ja juutide vanemad tulid mu ette ja nõudsid tema hukkamõistmist.
\par 16 Neile ma vastasin, et roomlastel ei ole kommet inimest anda hukata, enne kui süüalune on saanud olla suu suud vastu süüdistajatega ja leidnud võimaluse ennast kaitsta kaebuse vastu.
\par 17 Kui nad nüüd siia kokku tulid, siis ma viivitamata istusin järgmisel päeval kohtujärjele ja käskisin mehe ette tuua.
\par 18 Aga kui kaebajad seisid ta ümber, ei suutnud nad leida ühtki kurja süüd, mida ma ootasin,
\par 19 vaid neil oli mõningaid vaidlusi temaga nende omist usuküsimustest ja kellegi surnud Jeesuse pärast, keda Paulus ütles elavat.
\par 20 Kui ma siis nõutu olin, kuidas seda asja lahendada, küsisin temalt, kas ta ei tahaks minna Jeruusalemma ja seal lasta enese üle kohut mõista neis asjus.
\par 21 Aga kui Paulus nõudis, et teda niikaua kinni peetaks, kuni keisri majesteet teeb otsuse, siis ma käskisin teda kinni pidada seni, kui ma võin tema saata keisri juurde.”
\par 22 Siis ütles Agrippas Festusele: „Mina isegi tahaksin kuulda seda inimest!” Ja tema ütles: „Homme sa saad teda kuulda!”
\par 23 Kui nüüd Agrippas ja Berniike järgmisel päeval suure toredusega tulid ja kohtukotta läksid ühes ülempealikute ja meestega, kes linnas olid kõige ülemad, siis Festuse sõna peale toodi Paulus ette.
\par 24 Ja Festus ütles: „Kuningas Agrippas ja kõik mehed, kes olete meiega! Te näete siin seda, kelle pärast kogu juutide hulk Jeruusalemmas ja siin on mulle peale käinud kisendades, et ta ei peaks enam ellu jääma.
\par 25 Aga kui ma aru sain, et ta pole teinud midagi, mis on surma väärt, ja tema ka ise nõudis majesteedi kohut, tegin ma otsuseks saata tema sinna.
\par 26 Midagi kindlat mul ei ole temast oma isandale kirjutada. Sellepärast ma olen lasknud tema tuua teie ette ja kõigepealt sinu ette, kuningas Agrippas, et mul pärast ülekuulamist oleks midagi kirjutada.
\par 27 Sest minule näib mõttetuna lähetada vangi, ilma et tema vastu tõstetud süüdistust teada antaks!”

\chapter{26}

\section*{Paulus kaitseb ennast Agrippase ees}

\par 1 Aga Agrippas ütles Paulusele: „Sul on luba enese eest rääkida!” Siis Paulus sirutas oma käe ja kostis enese eest nõnda:
\par 2 „Ma arvan enesele õnneks, kuningas Agrippas, et ma täna sinu ees saan kosta kõigi nende kaebuste pärast, mis juudid tõstavad minu vastu,
\par 3 seda enam, et sa tunned kõiki juutide kombeid ja vaidlusküsimusi. Sellepärast ma palun sind mind kannatlikult kuulata.
\par 4 Minu elukorda mu noorest east alates, kuidas see mul algusest on olnud oma rahva seas Jeruusalemmas, teavad kõik juudid,
\par 5 sest nad tunnevad mind algusest peale, kui nad aga tahavad seda tunnistada, et mina kui variser olen elanud meie jumalateenistuse kõige rangemate kommete järgi.
\par 6 Ja nüüd ma seisan kohtu ees lootuse pärast selle tõotuse peale, mille Jumal on andnud meie esiisadele
\par 7 ja mille täitumist meie kaksteist suguharu ööd ja päevad alati Jumalat teenides loodavad näha. Selle lootuse pärast, kuningas, tõstavad juudid kaebust minu peale!
\par 8 Mis? Kas teil peetakse uskumatuks asjaks, et Jumal surnuid üles äratab?
\par 9 Ka mina ise olin arvamisel, et ma pean kõikepidi vastu panema Jeesuse Naatsaretlase nimele,
\par 10 mida ma ka tegin Jeruusalemmas. Ma panin palju pühi inimesi vangi, kui ma ülemailt preestreilt olin saanud selleks loa. Ja kui need tapeti, andsin mina oma hääle selle poolt.
\par 11 Ja ma nuhtlesin neid sageli kõigis kogudusekodades ja sundisin neid pilkama Jumalat ning olin päris hull nende vastu neid taga kiusates ka välismaa linnadeni.
\par 12 Ja kui ma ka neis asjus läksin Damaskusesse ülempreestrite volituse ja loaga,
\par 13 siis ma nägin, kuningas, keskpäevaajal tee peal valgust, mis oli heledam kui päike ja paistis taevast minu ja nende ümber, kes ühes minuga teel olid.
\par 14 Aga kui me kõik maha kukkusime, kuulsin ma üht häält mulle heebrea keeli ütlevat: Saul! Saul! Miks sa mind taga kiusad? Sulle läheb raskeks astla vastu takka üles lüüa!
\par 15 Mina aga ütlesin: kes sa oled, isand? Issand ütles: mina olen Jeesus, keda sa taga kiusad!
\par 16 Tõuse siis nüüd üles ja seisa oma jalgel, sest ma olen sulle ilmunud selleks, et ma sind seaksin oma sulaseks ja selle tunnistajaks, kuidas ma veel sulle ilmun ning
\par 17 päästan sind su rahva ja paganate käest, kelle juurde ma sind nüüd läkitan
\par 18 avama nende silmi, et nad pimedusest pöörduksid valguse poole ja saatana võimuse alt Jumala poole ning saaksid pattude andeksandmist ja osa nende seas, kes on pühitsetud usu läbi minusse!
\par 19 Sellepärast, kuningas Agrippas, ei olnud ma kuulmatu taevase nägemuse vastu,
\par 20 vaid jutlustasin esiti neile, kes elavad Damaskuses, ja siis neile, kes elavad Jeruusalemmas ja kogu Judeamaal, ja paganaile, et nad meelt parandaksid ja pöörduksid Jumala poole ning teeksid õigeid meeleparanduse tegusid.
\par 21 Selle asja pärast on juudid mind kinni võtnud pühakojas ja on püüdnud mind tappa.
\par 22 Aga et ma tänapäevani olen Jumalalt abi saanud, siis ma seisan veel siin ja tunnistan nii väikestele kui suurtele ega ütle muud midagi kui seda, mis prohvetid ja Mooses on rääkinud tulevasist asjust,
\par 23 et Kristus pidi kannatama ja esimesena surnuist üles tõusma ning kuulutama valgust niihästi juutidele kui paganaile!”
\par 24 Aga kui ta nõnda enese eest kostis, ütles Festus suure häälega: „Paulus, sa jampsid! Suur kirjatarkus paneb sind jampsima!”
\par 25 Aga Paulus ütles: „Auline Festus, ma ei jampsi mitte, vaid ma räägin tõelisi ja mõistlikke sõnu!
\par 26 Sest kuningas teab neid asju küll ja temale mina räägin ka julgesti, sest ma ei arva midagi temale teadmata olevat neist asjust; ei ole ju need nurgas sündinud.
\par 27 Kas sina, kuningas Agrippas, usud prohveteid? Ma tean, et sa usud!”
\par 28 Aga Agrippas ütles Paulusele: „Ei puudu palju, sa meelitad mind, et ma saaksin kristlaseks!”
\par 29 Aga Paulus ütles: „Ma sooviksin küll Jumalalt, olgu piskus või paljus, et mitte ainult sina, vaid ka kõik, kes mind täna kuulete, saaksite niisuguseiks nagu mina olen, aga ilma nende köidikuteta!”
\par 30 Siis tõusid üles kuningas ja maavalitseja ja Berniike ning need, kes nendega istusid,
\par 31 ja läksid kõrvale ning rääkisid isekeskis ja ütlesid: „See inimene ei tee midagi, mis oleks surma ja köidikute väärt!”
\par 32 Siis Agrippas ütles Festusele: „Selle inimese oleks võinud lahti lasta, kui ta mitte ei oleks nõudnud keisri kohut!”

\chapter{27}

\section*{Pauluse mereteekond Rooma}

\par 1 Kui oli otsustatud, et me laevaga sõidame Itaaliasse, anti Paulus ja mõned teised vangid ühe pealiku hoolde, nimega Juulius, kes oli üks keiserliku väesalga ülemaid.
\par 2 Ja me astusime Adramüti laeva, mille tee pidi minema Aasiamaa kohtade kaudu, ja purjetasime edasi. Meiega oli ka Aristarhos, makedoonlane Tessaloonikast.
\par 3 Ja teisel päeval me saabusime Siidonisse. Ja Juulius kohtles Paulust lahkesti ning lubas teda minna oma sõprade juurde, et need hoolitseksid tema eest.
\par 4 Ja kui me sealt olime läinud merele, purjetasime Küprose varju, sest tuuled olid vastu,
\par 5 ja kui olime purjetanud läbi Kiliikia ja Pamfüülia-äärse mere, saabusime Mürrasse Lüükiamaal.
\par 6 Seal leidis pealik Aleksandria laeva, mis läks Itaaliasse, ja pani meid sellesse.
\par 7 Aga kui me mitu päeva olime vähehaaval edasi purjetanud ja vaevaga saanud Kniidose kohale, ei lasknud tuul meid randa ja me purjetasime Kreeta varju Salmoone juures
\par 8 ja saime ainult vaevaga sealt mööda ning jõudsime ühte paika, mida nimetatakse Heaks Sadamaks ja mille lähedal oli Lasaia linn.

\section*{Torm merel ja laevahukk}

\par 9 Aga kui palju aega oli kulunud ja laevasõit juba läks kardetavaks ja ka paastuaeg oli möödunud, hoiatas Paulus neid
\par 10 ning ütles: „Mehed, ma näen, et laevasõit ähvardab tuua raskusi ja suurt hädaohtu mitte ainult koormale ja laevale, vaid ka meie elule!”
\par 11 Aga pealik usaldas rohkem tüürimeest ja laevajuhti kui seda, mis Paulus ütles.
\par 12 Ja et sinna sadamasse ei olnud hea üle talve jääda, siis oli suurema hulga nõu sealt ära minna ja kui võimalik, jõuda Foiniksisse, ühte Kreeta sadamasse vastu edela- ja loodetuult, ja sinna ületalve jääda.
\par 13 Aga kui lõunatuul hakkas puhuma ja nad arvasid oma nõu võivat täide viia, tõmbasid nad ankru üles ja liikusid edasi Kreeta ranna läheduses.
\par 14 Ent varsti pärast seda tõusis saare poolt marutuul, mida hüütakse kirdemaruks.
\par 15 Kui laev kisti sellega kaasa ega saanud tuulele vastu panna, andsime endid tuule ajada.
\par 16 Sattudes siis ühe väikese saare varju, mida hüütakse Klaudaks, suutsime hädavaevalt toime saada paadiga.
\par 17 Kui see oli üles tõmmatud, tarvitati veel teisi abinõusid ja laev seoti ümbert kinni. Ja et kardeti sattuda Sürti liivale, võeti purjed maha ja lasti ennast edasi ajada.
\par 18 Aga kui rajuilm meid väga vintsutas, heitsid nad teisel päeval muist koormat välja.
\par 19 Ja kolmandal päeval viskasid nad oma käega laeva riistad välja.
\par 20 Aga kui mitu päeva ei paistnud ei päikest ega tähti ja maru läks väga kangeks, lõppes viimaks kõik lootus veel pääseda.
\par 21 Kui nad siis kaua olid olnud söömata, astus Paulus nende keskele ja ütles: „Mehed, te oleksite pidanud kuulama minu nõu ja Kreetast mitte ära tulema, siis te oleksite vältinud selle häda ja kahju.
\par 22 Nüüd ma manitsen teid olla julged, sest ükski hing teie seast ei hukku, vaid üksnes laev.
\par 23 Sest sel ööl seisis minu juures selle Jumala ingel, kelle oma ma olen ja keda ma ka teenin,
\par 24 ning ütles: ära karda Paulus, sa pead saama keisri ette, ja vaata, Jumal on sulle kinkinud need kõik, kes on ühes sinuga laeval!
\par 25 Sellepärast, mehed, olge julges meeles, sest ma usun Jumalat, et nõnda sünnib, nagu mulle on öeldud!
\par 26 Aga me peame ajama ühele saarele.”
\par 27 Kui juba jõudis kätte neljateistkümnes öö, kui meid Aadria merel aeti sinna ja tänna, siis arvasid laevamehed kesköö ajal endid ühele maale lähenevat.
\par 28 Ja kui nad loodi vette heitsid, leidsid nad kakskümmend sülda vett. Ja kui nad pisut edasi olid jõudnud ja jälle loodi heitsid, leidsid nad viisteist sülda vett.
\par 29 Siis nad kartsid sattuda karile ja heitsid laeva tagumisest otsast välja neli ankrut ja jäid ootama päevavalgust.
\par 30 Aga kui laevamehed püüdsid laevast põgeneda ja paati merre lasksid, nagu tahaksid nad laeva ninast ankrut sisse lasta,
\par 31 ütles Paulus pealikule ja sõjameestele: „Kui need ei jää laeva, siis te ei pääse eluga!”
\par 32 Siis raiusid sõjamehed paadi köied katki ja lasksid selle merre kukkuda.
\par 33 Aga seni kui valgeks läks, manitses Paulus neid kõiki leiba võtta ja ütles: „Täna on neljateistkümnes päev, kui te ootate ja olete söömata ega ole midagi suhu võtnud.
\par 34 Sellepärast ma manitsen teid leiba võtta, sest see on tarvilik teie pääsemiseks; ei tohi ju ühelgi teie seast juuksekarvgi peast kaotsi minna!”
\par 35 Kui ta seda oli öelnud, võttis ta leiva, tänas Jumalat nende kõikide nähes, murdis ja hakkas sööma.
\par 36 Siis nende kõikide meel läks heaks ja nad võtsid ka leiba.
\par 37 Aga neid kõiki oli laeval kakssada seitsekümmend kuus hinge.
\par 38 Ja kui nad olid leiba võtnud, kergendasid nad veel laeva, heites vilja merre.
\par 39 Kui valgeks läks, siis nad ei tundnud, mis maa see oli, aga nad nägid üht lahte, millel oli sobiv rand. Sinna nad otsustasid, kui võimalik, laeva ajada.
\par 40 Nad raiusid ankrud katki ja lasksid need merre vajuda; siis nad päästsid tüüri köied lahti ja seadsid väikese purje tuule järele ning suunasid sõidu ranna poole.
\par 41 Ent nad sattusid neemele ja ajasid laeva sinna; laeva nina tungis sinna sisse ja jäi liikumata seisma, aga pära katkes kangete lainete käes.
\par 42 Ent sõjameestel oli nõu vangid ära tappa, et ükski neist ujudes ei pääseks põgenema.
\par 43 Kuid pealik, tahtes Paulust päästa, keelas neid seda nõu täitmast ja käskis neid, kes oskasid ujuda, esimestena hüpata vette ja väljuda maale,
\par 44 aga teisi ta käskis püüda randa jõuda laudadel ja muist laevatükkidel. Ja sel kombel pääsesid kõik tervelt maale.

\chapter{28}

\section*{Paulus Melitel (Maltal)}

\par 1 Kui olime pääsenud, saime teada, et saart hüütakse Meliteks.
\par 2 Aga umbkeelsed osutasid meile suurt lahkust; nad tegid tule maha ja võtsid meid kõiki selle äärde vihmasaju ja külma pärast.
\par 3 Aga kui Paulus sületäie hagu kokku riisus ja tulle pani, tuli kuumuse tõttu välja rästik ja hakkas tema käest kinni!
\par 4 Kui umbkeelsed nägid seda looma tema käe küljes rippuvat, ütlesid nad isekeskis: „Vist on see mees inimesetapja, et õiguse jumalanna teda ei lase elada, ehk ta küll on pääsenud merest!”
\par 5 Aga tema raputas eluka tulle ega tundnud mingit viga.
\par 6 Nad ootasid küll, et ta üles paistetab või äkitselt surnuna maha langeb. Aga kui nad olid kaua oodanud ja nägid, et talle ei sündinud mingit viga, mõtlesid nad teisiti ja ütlesid tema jumala olevat.
\par 7 Aga selle paiga läheduses oli saare tähtsamal mehel, nimega Puublius, mõis. Tema võttis meid vastu ja pidas meid kolm päeva lahkesti külalistena.
\par 8 Ja sündis, kui Puubliuse isa oli haige maas palavikus ja kõhutõves, et Paulus läks tema juurde, palvetas ja pani oma käed ta peale ning tegi tema terveks.
\par 9 Kui see oli sündinud, tulid ka teised haiged saarelt ligi ja said terveks.
\par 10 Ja nemad austasid meid mitmel viisil, ja kui me hakkasime ära minema, andsid nad teekonna jaoks, mis vaja oli.

\section*{Teekond Rooma jätkub}

\par 11 Aga kolme kuu pärast me läksime sealt minema Aleksandria laevaga, mis oli ületalve olnud sellel saarel ja mille tunnuseks oli Dioskuuride märk.
\par 12 Me maabusime Sürakuusas ja olime seal kolm päeva paigal.
\par 13 Sealt me sõitsime edasi ja jõudsime Reegioni, ja kui ühe päeva pärast tõusis lõunatuul, saabusime teisel päeval Puteolisse.
\par 14 Sealt me leidsime vendi, ja nende kutsel jäime seitsmeks päevaks nende juurde. Siis me läksime Rooma linna.
\par 15 Ja kui sealsed vennad meist kuulda said, tulid nad meile vastu Appiuse turuni ja Trestabernani. Neid nähes tänas Paulus Jumalat ja sai julgust.
\par 16 Aga kui saabusime Rooma, anti Paulusele luba jääda omaette elama ühes sõduriga, kes teda valvas.

\section*{Pauluse kõnelus Rooma juutidega}

\par 17 Aga kolme päeva pärast Paulus kutsus kokku juutide peamehed. Kui need kokku tulid, ütles ta neile: „Mehed, vennad, mina pole teinud midagi meie rahva ja esiisade kommete vastu, ja ometi ma olen Jeruusalemmast vangina antud üle roomlaste kätte;
\par 18 ja kui need mind olid üle kuulanud, tahtsid nad mind vabaks lasta, sest minul ei olnud ühtki surmasüüd.
\par 19 Aga kui juudid selle vastu rääkisid, siis ma pidin häda pärast nõudma keisri kohut, aga mitte, nagu oleks mul midagi kaebamist oma rahva peale.
\par 20 Selle asja pärast ma teid nüüd kutsusin enese juurde, et teid näha ja teiega rääkida; sest Iisraeli lootuse pärast ma olen kinni selles ahelas!”
\par 21 Aga nemad ütlesid talle: „Me ei ole saanud ühtki kirja Juudamaalt sinu kohta; ei ole ka tulnud siia ühtki venda, kes sinust oleks teatanud ehk rääkinud midagi paha.
\par 22 Ent meie soovime siiski kuulda sinult, mida sa mõtled; sest sellest väärusust on meil teada, et selle vastu igas paigas räägitakse!”
\par 23 Ja nad määrasid temale päeva ja siis tuli tema juurde majasse veel rohkem inimesi. Neile ta seletas asja ja tunnistas Jumala riiki ning meelitas hommikust õhtuni neid uskuma Moosese käsuõpetusest ja prohvetitest seda, mis Jeesusest on kirjutatud.
\par 24 Siis ühed veendusid tema sõnus, teised aga ei uskunud mitte.
\par 25 Aga kui nad isekeskis ei olnud üksmeelsed, läksid nad ära, kui Paulus oli öelnud selle sõna: „Püha Vaim on hästi rääkinud prohvet Jesaja kaudu teie esiisadele
\par 26 ja on öelnud: mine selle rahva juurde ja ütle: kuuldes te kuulete ega mõista, ja nähes te näete ega taipa!
\par 27 Sest selle rahva süda on tuimaks läinud ja nad kuulevad raskesti oma kõrvadega ja sulevad oma silmad, et nad silmadega ei näeks ja kõrvadega ei kuuleks ja südamega ei mõistaks ega pöörduks, et mina neid parandaksin!
\par 28 Siis olgu teile teada, et see Jumala pääste on läkitatud paganaile ja nemad võtavad seda kuulda!”
\par 29 [Ja kui ta seda oli öelnud, läksid juudid minema ja neil oli isekeskis palju vaidlemist.]

\section*{Paulus kuulutab evangeeliumi Roomas}

\par 30 Aga Paulus jäi kaheks terveks aastaks oma üürielamusse ja võttis vastu kõiki, kes tulid tema juurde,
\par 31 ja kuulutas Jumala riiki ning õpetas Issandast Jeesusest Kristusest kõige julgusega ilma takistamata.



\end{document}