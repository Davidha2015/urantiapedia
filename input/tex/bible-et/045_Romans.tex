\begin{document}

\title{Pauluse kiri roomlastele}

\chapter{1}

\section*{Tervitus}

\par 1 Paulus, Jeesuse Kristuse sulane, kutsutud apostliks, välja valitud kuulutama Jumala evangeeliumi,
\par 2 mille Jumal on enne tõotanud oma prohvetite kaudu pühades kirjades
\par 3 oma Pojast - kes liha poolest on sündinud Taaveti soost
\par 4 ja pühaduse vaimu poolest on seatud surnuist ülestõusmise läbi Jumala Pojaks väes - Jeesusest Kristusest, meie Issandast,
\par 5 kelle läbi me oleme saanud armu ja apostliameti, et äratada usu sõnakuulelikkust tema nime heaks kõigi rahvaste seas,
\par 6 kelle seast ka teie olete Jeesuse Kristuse poolt kutsutud,
\par 7 kõigile Roomas olevaile Jumala armastatuile, kutsutud pühadele: armu teile ja rahu Jumalalt, meie Isalt, ja Issandalt Jeesuselt Kristuselt!

\section*{Pauluse kavatsus külastada Roomat}

\par 8 Kõigepealt ma tänan oma Jumalat Jeesuse Kristuse läbi teie kõikide pärast, et teie usust kiitvalt kõneldakse kogu maailmas.
\par 9 Sest Jumal, keda ma oma vaimus teenin tema Poja evangeeliumi kuulutamisega, on mu tunnistaja, kuidas ma lakkamata pean teid meeles,
\par 10 aina oma palvetes anudes, kas ma juba kord Jumala tahtmisel saaksin tulla teie juurde.
\par 11 Sest ma igatsen teid näha, et võiksin jagada teile pisut vaimulikku annet teie kinnituseks;
\par 12 see on, et me teie juures üheskoos olles vastastikku ergutaksime üksteist ühise usu, teie ja minu oma läbi.
\par 13 Ja ma ei taha, vennad, et teil oleks teadmata, et ma sageli olen ette võtnud tulla teie juurde, et ma ka teie seas võiksin saata mõnd kasu, nõnda nagu muude paganate seas; aga ma olin tänini takistatud.
\par 14 Ma olen nii kreeklaste kui umbkeelsete, nii tarkade kui rumalate võlglane;
\par 15 siis olen ma omalt poolt valmis ka teile, Roomas asuvaile, evangeeliumi kuulutama.

\section*{Evangeeliumi olemus}

\par 16 Sest ma ei häbene evangeeliumi; sest see on Jumala vägi õndsakssaamiseks igaühele, kes usub, nii juudile esiti kui ka kreeklasele.
\par 17 Sest temas saab ilmsiks Jumala õigus usust usku, nõnda nagu on kirjutatud: „Õige elab usust!”

\section*{Paganate patusus}

\par 18 Sest Jumala viha saab ilmsiks taevast kõige inimeste jumalatuse ja ülekohtu vastu, nende vastu, kes tõde ülekohtuga kinni peavad,
\par 19 sellepärast et see, mida teatakse Jumalast, on avalik nende seas; sest Jumal on seda neile avaldanud.
\par 20 Sest tema nähtamatut olu, nii tema igavest väge kui jumalikku olemist, nähakse, kui neid pannakse tähele, tema tegudes maailma loomisest alates, nii et nad ei saa vabanduda,
\par 21 sellepärast et nad Jumalat tundes ei ole teda kui Jumalat austanud ega tänanud, vaid on oma mõtlemistes saanud tühiseks ja nende mõistmatu süda on läinud pimedaks.
\par 22 Kiites endid targaks, on nad saanud jõledaks
\par 23 ja on kadumatu Jumala au vahetanud kaduva inimese ja lindude ja neljajalgsete ja roomajate sarnase kuju vastu.
\par 24 Sellepärast on Jumal nad andnudki nende südamete himudes rüvedusse, oma ihusid ise pilastama.
\par 25 Nemad on Jumala tõe vahetanud vale vastu ja on austanud ning teeninud loodut enam kui Loojat, kes on ülistatud igavesti. Aamen.
\par 26 Sellepärast on Jumal nad andnud häbematuisse ihadesse; sest nende naised on vahetanud loomuliku sugulise käitlemise loomuvastasega,
\par 27 ja samuti on ka mehed loobunud loomulikust naise sugulisest käitlemisest ja on oma himus süttinud üksteisest ja teinud rõvedust mees mehega ja kätte saanud iseenestes oma eksimuse palga, mille pidid saama.
\par 28 Ja nagu nad heaks ei arvanud kinni pidada Jumala tunnetamisest, nii on Jumal nad andnud hoolimatu meele sisse, tegema seda, mis ei kõlba;
\par 29 nad on täis kõike ülekohut, kurjust, ahnust, tigedust, täis kadedust, mõrva, riidu, kurikavalust; nad on keelekandjad,
\par 30 laimajad, Jumala vihkajad, ülbed, suurelised, hooplejad, leidlikud kurjale, sõnakuulmatud vanemaile,
\par 31 mõistmatud, lepingu rikkujad, armuheitmatud, halastamatud,
\par 32 kes, ehk nad küll tunnevad Jumala õiguse määrust ja teavad, et need, kes teevad niisuguseid asju, on surma väärt, ei tee mitte ainult sedasama, vaid tunnevad head meelt neist, kes nõnda teevad.


\chapter{2}

\section*{Juudidki on patused ja süüdlased}

\par 1 Sellepärast, oh inimene, kes kohut mõistad, ole sa kes tahes, ei ole sul mingit vabandust! Sest milles sa kohut mõistad teise üle, sessamas sa mõistad iseenese hukka; sest sina, kes teist hukka mõistad, teed ise sedasama!
\par 2 Aga me teame, et Jumala kohus tõepoolest tabab neid, kes sellesarnast teevad.
\par 3 Või mõtled sa, oh inimene, kes kohut mõistad nende üle, kes sellesarnast teevad, ja teed ka sedasama, et sa pääsed Jumala kohtu eest?
\par 4 Või põlgad sa tema helduse ja kannatlikkuse ja pika meele rohkust ega saa aru, et Jumala heldus sind ajab meelt parandama?
\par 5 Ent sa kogud enesele viha oma kangusega ja patustpöördumata südamega vihapäevaks ja Jumala õige kohtu ilmumiseks,
\par 6 kes igaühele maksab tema tegude järgi,
\par 7 neile küll igavest elu, kes head tehes püsivusega otsivad au ja kiitust ja kadumatut põlve,
\par 8 aga neile, kes on riiakad ja tõele sõnakuulmatud, aga sõnakuulelikud ülekohtule, viha ja raevu!
\par 9 Viletsus ning ahastus tuleb igale inimese hingele, kes kurja teeb, nii juudile esiti kui ka kreeklasele,
\par 10 aga au ja kiitus ja rahu igaühele, kes head teeb, nii juudile esiti kui ka kreeklasele!
\par 11 Sest Jumal ei tee vahet isikute vahel!
\par 12 Sest kõik, kes ilma käsuta on pattu teinud, lähevad ka hukka ilma käsuta; ja kõik, kes käsu all on pattu teinud, nende üle mõistetakse ka kohut käsu järgi;
\par 13 sest käsu kuuljad ei ole õiged Jumala ees, vaid käsu tegijad mõistetakse õigeks -
\par 14 sest kui paganad, kellel ei ole käsku, loomu poolest täidavad käsu nõudeid, on nad, ehk küll neil ei ole käsku, iseenestele käsuks
\par 15 ja näitavad, et käsutegu on kirjutatud nende südametesse, kuna nende südametunnistus ühtlasi tunnistab seda ja nende mõtted isekeskis kas kaebavad nende peale või ka kostavad nende eest
\par 16 - sel päeval, kui Jumal kohut mõistab inimeste salajaste asjade üle Kristuse Jeesuse läbi minu evangeeliumi järgi.
\par 17 Ent kui sa nimetad ennast juudiks ja loodad käsu peale ning kiitled Jumalast
\par 18 ja tunned tema tahtmist, ja käsust õpetust saades katsud läbi, mis on kõige parem,
\par 19 ja pead ennast sõgedate teejuhiks, nende valguseks, kes on pimeduses,
\par 20 mõistmatute juhatajaks, alaealiste õpetajaks, kellel on käes täieks kujunenud tunnetus ja tõde käsuõpetuses;
\par 21 kuidas sa siis, kes teist õpetad, iseennast ei õpeta; kes kuulutad, et ei tohi varastada, ise varastad;
\par 22 kes ütled, et ei tohi abielu rikkuda, ise rikud abielu; kes pead ebajumalaid jäledaks asjaks, ise riisud pühamut;
\par 23 kes kiitled käsust, ise teotad Jumalat käsust üleastumisega?
\par 24 Pilgatakse ju Jumala nime teie pärast paganate seas, nõnda nagu on kirjutatud.
\par 25 Sest ümberlõikamisest on küll kasu, kui sa teed käsu järgi; aga kui sa oled käsust üleastuja, siis on sinu ümberlõikamine saanud eesnahaks.
\par 26 Kui nüüd ümberlõikamatu peab käsu nõudmisi, eks siis tema eesnahka peeta ümberlõikamiseks?
\par 27 Ja eks siis see, kes loomu poolest on ümberlõikamatu ja peab täiesti käsku, mõista kohut sinu üle, kes kirjatähe ja ümberlõikamise kaudu oled ometi käsust üleastuja?
\par 28 Ei ole ju juut see, kes seda on välispidi; ega ole ümberlõikamine see, mis sünnib välispidi lihas,
\par 29 vaid juut on see, kes seda on seespidi, ja südame ümberlõikamine on vaimus, mitte kirjatähes. Niisugune saab kiituse mitte inimestelt, vaid Jumalalt.


\chapter{3}

\section*{Vastulaused väiteile}

\par 1 Mis paremus on siis juudil? Või mis kasu on ümberlõikamisest?
\par 2 On paljugi kõikepidi! Kõigepealt see, et nende kätte on usaldatud, mida Jumal on rääkinud.
\par 3 Mis sellest, kui mõned olid uskmatud? Ega siis nende uskmatu meel tee tühjaks Jumala ustavust?
\par 4 Mitte sugugi! Kindel on, et Jumal on tõemeelne, olgugi, et iga inimene on valelik, nagu on kirjutatud: „Et sa oleksid õiglane oma sõnades ja võit jääks sinule, kui sinuga kohtus käiakse!”
\par 5 Aga kui meie ülekohus teeb avalikuks Jumala õigluse, mis peame siis ütlema? Kas Jumal ei ole ülekohtune, kui ta nuhtleb vihas? Ma räägin kui inimene.
\par 6 Mitte sugugi! Kuidas võiks siis Jumal kohut mõista maailma üle?
\par 7 Sest kui Jumala tõde minu vale läbi saab palju selgemaks tema auks, miks siis veel minu kui patuse üle kohut mõistetakse?
\par 8 Ja miks me ei teeks nõnda, nagu meist pilgates öeldakse ja nagu mõned meist räägivad, et meie ütlevat: „Tehkem kurja, et sellest tuleks head?” Niisugune nuhtlus on õige.

\section*{Üldinimlik patune olukord}

\par 9 Kuidas siis on? Kas me oleme paremas seisukorras? Mitte sugugi! Oleme ju varemini tõestanud, et niihästi juudid kui kreeklased on kõik patu all,
\par 10 nõnda nagu on kirjutatud: „Ei ole kedagi, kes oleks õige, ei ühtainustki;
\par 11 ei ole arusaajat; ei ole, kes otsiks Jumalat;
\par 12 nad kõik on läinud kõrvale, nad on puha saanud kõlvatuks, ei ole kedagi, kes teeb head, ei ühtainustki.
\par 13 Nende kurk on lahtine haud; oma keelega nad petavad; madude mürk on nende huulte taga.
\par 14 Nende suu on täis sajatamist ja kibedaid sõnu;
\par 15 nende jalad on nobedad verd valama;
\par 16 hävitus ja õnnetus on nende teedel,
\par 17 ja rahu teed nad ei tunne;
\par 18 ei ole Jumala kartust nende silmade ees!”
\par 19 Aga me teame, et mida iganes käsk ütleb, seda ta ütleb neile, kes on käsu all, et iga suu suletaks ja kogu maailm oleks süüalune Jumala ees;
\par 20 sellepärast et käsu tegudest ei mõisteta õigeks ühtki liha tema ees; sest käsu kaudu tuleb patu tundmine.

\section*{Õigeksmõistmine usu läbi}

\par 21 Aga nüüd on Jumala õigus, millest käsk ja prohvetid tunnistavad, ilmutatud ilma käsuta,
\par 22 see Jumala õigus, mis tuleb Jeesuse Kristuse usu kaudu kõikidele, kes usuvad; sest siin ei ole ühtki vahet;
\par 23 sest kõik on pattu teinud ja on Jumala aust ilma
\par 24 ning mõistetakse õigeks täiesti muidu, tema armust, lunastuse läbi, mis on Kristuses Jeesuses,
\par 25 kelle Jumal on seadnud lepitusvahendiks usu kaudu tema veres, oma õiguse osutamiseks enne tehtud pattude arvestamata jätmise pärast
\par 26 tol ajal, kui Jumal seda sallis, et osutada oma õigust praegusel ajal ning näidata, et tema on õige ja teeb õigeks selle, kes on Jeesuse usust.
\par 27 Kus on nüüd kiitlemine? See on kõrvaldatud. Missuguse käsu läbi? Kas tegude käsu läbi? Ei mitte, vaid usu käsu läbi.
\par 28 Sellepärast me arvame, et inimene mõistetakse õigeks usu läbi lahus käsu tegudest.
\par 29 Või on Jumal ükspäinis juutide Jumal? Eks ta ole ka paganate Jumal? Jah, tõesti ka paganate Jumal!
\par 30 Sest üks ja sama Jumal teeb usust õigeks ümberlõigatu ja usu läbi ümberlõikamatu.
\par 31 Kas me siis teeme käsu tühjaks usu läbi? Mitte sugugi, vaid me kinnitame käsku!



\chapter{4}

\section*{Näide Aabrahamist}

\par 1 Mis me siis ütleme oma lihase esiisa Aabrahami saavutanud olevat?
\par 2 Sest kui Aabraham on õigeks mõistetud tegudest, on temal kiitlemist, kuid mitte Jumala ees.
\par 3 Sest mis ütleb Kiri? „Aabraham uskus Jumalat ja see arvati temale õiguseks!”
\par 4 Aga tegude tegijale ei arvata palka mitte armust, vaid teenet mööda.
\par 5 Aga kes tegusid ei tee, vaid usub temasse, kes mõistab õigeks jumalatu, sellele arvatakse tema usk õiguseks,
\par 6 nõnda nagu Taavet kiidab õndsaks inimest, kellele Jumal arvab õiguse osaks ilma tegudeta:
\par 7 „Õndsad need, kelle ülekohus on andeks antud ja kelle patud on kinni kaetud.
\par 8 Õnnis see, kellele Jumal pattu ei arvesta!”
\par 9 Kas nüüd see õndsakskiitmine käib ainult ümberlõigatute kohta või ka ümberlõikamatute kohta? Me ju ütleme, et Aabrahamile usk arvati õiguseks.
\par 10 Kuidas see siis arvati? Kas siis, kui ta oli ümberlõigatu või kui ta oli ümberlõikamatu? Mitte kui ta oli ümberlõigatu, vaid kui ta oli ümberlõikamatu.
\par 11 Ja ümberlõikamise märgi ta sai selle usu õiguse kinnituseks, mis tal oli ümberlõikamata põlves, et ta oleks kõikide isa, kes usuvad ümberlõikamatutena, et õigus arvatakse ka neile osaks
\par 12 ja et ta oleks ka ümberlõigatute isa, nende isa, kes mitte ainult ei ole ümber lõigatud, vaid ka käivad meie isa Aabrahami usu jälgedes, mis tal oli juba ümberlõikamatuna.

\section*{Tõotus Aabrahamile teostub tema usu läbi}

\par 13 Sest see tõotus, et Aabraham saab maailma pärijaks, ei antud temale ega tema soole mitte käsu läbi, vaid usu õiguse kaudu.
\par 14 Sest kui need, kes on käsust, on pärijad, siis on usk tühistatud ja tõotus on tehtud mõjutuks.
\par 15 Sest käsk toob viha; aga kus ei ole käsku, seal ei ole ka üleastumist.
\par 16 Sellepärast on see usust, et see oleks armust, et tõotus oleks kindel kogu soole, mitte üksnes sellele, mis on pärit käsust, vaid ka sellele, mis on pärit Aabrahami usust, kes on meie kõikide isa -
\par 17 nagu on kirjutatud: „Ma olen sind pannud paljude rahvaste isaks!” - selle Jumala ees, keda ta uskus ja kes teeb elavaks surnud ja kutsub esile olematuid asju otsekui olevaid.
\par 18 Tema uskus üle lootuse lootuses, et ta saab paljude rahvaste isaks vastavalt sõnale: „Nõnda peab olema sinu sugu!”
\par 19 Ja ta ei läinud usus nõdraks tähele pannes oma ihu, mis oli elatanud - oli ta ju umbes saja-aastane - ja et Saara lapsekoda oli surnud.
\par 20 Ent Jumala tõotusest ta ei mõelnud uskmatuses kaksipidi, vaid sai vägevaks usus, andes Jumalale austust
\par 21 ja olles täiesti julge selles, et Jumal on vägev tegema, mis ta on tõotanud.
\par 22 Sellepärast see arvatigi temale õiguseks.
\par 23 Aga ei ole mitte üksnes tema pärast kirjutatud, et see arvati temale õiguseks,
\par 24 vaid ka meie pärast, kellele see peab arvatama õiguseks, sest me usume temasse, kes surnuist üles äratas meie Issanda Jeesuse,
\par 25 kes loovutati meie üleastumiste pärast ja üles äratati meie õigekssaamise pärast.


\chapter{5}

\section*{Usu tagajärg}

\par 1 Et me nüüd oleme usust õigeks saanud, siis on meil rahu Jumalaga meie Issanda Jeesuse Kristuse läbi,
\par 2 kelle läbi me oleme ka usus saanud ligipääsu sellele armule, milles me nüüd oleme, ja kiitleme Jumala au lootusest.
\par 3 Aga mitte üksnes sellest, vaid me kiitleme ka viletsustest, teades, et viletsus saadab kannatlikkuse,
\par 4 ja kannatlikkus saadab püsivuse ja püsivus lootuse;
\par 5 aga lootus ei jäta häbisse, sest Jumala armastus on välja valatud meie südameisse Püha Vaimu läbi, kes meile on antud.

\section*{Jumal osutab armastust Kristuse surma läbi}

\par 6 Sest Kristus on, kui me alles nõdrad olime, omal ajal surnud nende eest, kes alles olid jumalatud.
\par 7 Vaevalt ju keegi läheb surma õige eest; ehk mõni küll julgeks surra hea sõbra eest.
\par 8 Ent Jumal osutab oma armastust meie vastu sellega, et Kristus on surnud meie eest, kui me alles patused olime;
\par 9 kui palju enam me nüüd, olles õigeks saanud tema veres, pääseme tema kaudu viha eest.
\par 10 Sest kui meid Jumalaga lepitati tema Poja surma kaudu, kui me alles olime Jumala vaenlased, kui palju enam päästetakse meid tema elu läbi nüüd, kus me juba oleme lepitatud.
\par 11 Aga mitte ainult seda, vaid me kiitleme ka Jumalast meie Issanda Jeesuse Kristuse läbi, kelle kaudu me nüüd oleme saanud lepituse.

\section*{Surm Aadama, elu Kristuse läbi}

\par 12 Sellepärast, otsekui ühe inimese kaudu patt tuli maailma ja patu läbi surm, nõnda on ka surm tunginud kõigisse inimestesse, sest nad kõik on pattu teinud.
\par 13 Sest patt oli maailmas ka enne käsku, kuid pattu ei arvata süüks seal, kus ei ole käsku.
\par 14 Ometi valitses surm Aadamast Mooseseni ka neid, kes ei olnud pattu teinud samasuguse üleastumisega nagu Aadam, kes oli selle eeltähendus, kes pidi tulema.
\par 15 Aga ka armuanniga ei ole lugu nõnda nagu pattulangemisega; sest kui ühe inimese pattulangemise läbi paljud on surnud, siis on veel palju enam Jumala arm ja and selle ühe inimese Jeesuse Kristuse armu läbi ülirohkesti saanud osaks neile paljudele.
\par 16 Ja ka anniga ei ole nõnda nagu sellega, mis on tulnud ühe pattulangenu läbi; sest ühest ainsast pattulangemisest on kohus saanud hukkamõistmiseks, aga armuand mitmest pattulangemisest õigeksmõistmiseks.
\par 17 Sest kui ühe inimese pattulangemise tõttu surm on valitsenud selle ühe läbi, kui palju enam peavad need, kes saavad armu ja õiguse anni täiuse, valitsema elus selle ühe, Jeesuse Kristuse läbi.
\par 18 Nõnda siis, otsekui üks langemine on saanud hukkamõistmiseks kõigile inimestele, nõnda saab ka üks õiguse täitmine kõigile inimestele elu õigustuseks.
\par 19 Sest otsekui tolle ühe inimese sõnakuulmatuse läbi paljud on saanud patuseks, nõnda saavad ka selle ühe inimese sõnakuulelikkuse läbi paljud õigeks.
\par 20 Aga käsk tuli kõrvalt sisse, et langemine saaks suuremaks. Ent kus patt on suurenenud, seal on ka arm saanud ülirohkeks,
\par 21 et otsekui patt on valitsenud surmas, samuti ka arm valitseks õiguse kaudu igaveseks eluks Jeesuse Kristuse, meie Issanda läbi.


\chapter{6}

\section*{Ühendus Kristusega vabastab patust}

\par 1 Mis me nüüd ütleme? Kas peame jääma patusse, et arm suureneks?
\par 2 Ei milgi kombel! Meie, kes oleme ära surnud patule, kuidas peaksime veel elama selles?
\par 3 Või te ei tea, et nii mitu, kui meid on ristitud Kristusesse Jeesusesse, oleme ristitud tema surmasse?
\par 4 Me oleme siis surmasse ristimise kaudu ühes temaga maha maetud, et otsekui Kristus on surnuist üles äratatud Isa au läbi, nõnda meiegi käiksime uues elus.
\par 5 Sest kui me oleme kasvanud ühte tema surma sarnasusega, siis saame üheks ka tema ülestõusmise sarnasusega,
\par 6 kuna me seda teame, et meie vana inimene on ühes temaga risti löödud, et patuihu kaotataks, nõnda et me enam ei orjaks pattu;
\par 7 sest kes on surnud, see on õigeks mõistetud patust.
\par 8 Aga kui me ühes Kristusega oleme surnud, siis usume, et me ühes temaga saame ka elama.
\par 9 Sest me teame, et Kristus pärast seda, kui ta surnuist üles äratati, enam ei sure; surm ei valitse enam tema üle.
\par 10 Sest mis ta suri, seda ta suri patule ühe korra; aga mis ta elab, elab ta Jumalale.
\par 11 Nõnda arvake teiegi endid surnud olevat patule, aga elavat Jumalale Kristuses Jeesuses.
\par 12 Seepärast ärgu valitsegu siis patt teie surelikus ihus, et te kuulaksite teda tema himudes.
\par 13 Ärge ka mitte andke oma liikmeid ülekohtu relviks patule, vaid andke iseendid Jumalale kui need, kes surnuist on saanud elavaks, ja oma liikmed õiguse relviks Jumalale.
\par 14 Sest patt ei tohi teie üle valitseda, sellepärast et teie ei ole käsu all, vaid armu all.

\section*{Selgitav kogemus orjusest}

\par 15 Kuidas siis? Kas hakkame pattu tegema, sest me ei ole käsu all, vaid armu all? Mitte sugugi!
\par 16 Eks te tea, et kelle orjaks te hakkate sõnakuulmises, selle orjad te olete, kelle sõna te kuulete, kas patu orjad surmaks või sõnakuulmise orjad õiguseks?
\par 17 Aga tänu Jumalale, et te olite patu orjad, aga nüüd olete südamest saanud sõnakuulelikuks sellele õpetuse väljendusele, mille juurde teid on juhatatud,
\par 18 ja et te patust vabanenuina olete saanud õiguse orjadeks!
\par 19 Ma räägin inimese kombel teie liha nõtruse pärast. Sest otsekui te oma liikmed andsite orjadeks rüvedusele ja ülekohtule, selleks et käsust üle astuda, nõnda andke nüüd oma liikmed õiguse orjadeks pühitsusele.
\par 20 Sest kui te olite patu orjad, olite vabad õigusest.
\par 21 Mis vilja te kandsite siis? Niisugust, millest teil nüüd on häbi! Sest selle lõpp on surm!
\par 22 Aga nüüd, et te olete saanud patust vabaks ja Jumala orjadeks, on teie vili pühitsuseks ja selle lõpp on igavene elu.
\par 23 Sest patu palk on surm, aga Jumala armuand on igavene elu Kristuses Jeesuses, meie Issandas!


\chapter{7}

\section*{Selgitav näide abielust}

\par 1 Või eks te tea, vennad - mina räägin käsutundjaile - et käsk valitseb inimese üle, niikaua kui inimene elab.
\par 2 Nii on abielunaine käsuga seotud elusoleva mehe külge; aga kui mees sureb, on ta lahti mehe käsu alt.
\par 3 Sellepärast, kui naine mehe elus olles läheb teisele mehele, nimetatakse teda abielurikkujaks; aga kui mees sureb, on ta sellest käsust vaba, nii et ta ei ole abielurikkuja, kui ta läheb teisele mehele.
\par 4 Nõnda teiegi, mu vennad, olete surmatud käsule Kristuse ihu läbi, et saaksite teisele mehele, kes on surnuist üles äratatud, et me kannaksime vilja Jumalale.
\par 5 Sest kui me olime lihas, siis patuhimud, mis tõusid käsu läbi, tegutsesid meie liikmetes, et kanda vilja surmale.
\par 6 Aga nüüd me oleme käsust lahti ning oleme surnud sellele, mis meid kinni pidas, nii et me teenime Jumalat vaimu uues olemises ja mitte kirjatähe vanas olemises.

\section*{Käsk ja patt}

\par 7 Mis me siis ütleme? Ons käsk patt? Mitte sugugi! Ent pattu ma ei oleks tundnud muidu kui käsu kaudu; sest ma ei oleks kurjast himust midagi teadnud, kui käsk ei oleks öelnud: „Sa ei tohi himustada!”
\par 8 Aga patt, saades tõuget käsust, äratas minus kõiksugu himusid; sest ilma käsuta on patt surnud.
\par 9 Mina elasin enne ilma käsuta; aga kui käsusõna tuli, virgus patt ellu
\par 10 ja mina surin ära, ja nõnda leiti käsusõna, mis mulle pidi olema eluks, mulle olevat surmaks.
\par 11 Sest kui patt oli saanud käsusõnast tõuke, pettis ta mind ja surmas mind selle abil.
\par 12 Nõnda on siis käsk püha ja käsusõna püha ja õige ja hea.
\par 13 Kas siis nüüd see, mis on hea, on saanud mulle surmaks? Mitte sugugi! Vaid patt, et ta ilmneks patuna, on selle hea kaudu mulle toonud surma, et patt ise saaks üliväga patuseks käsusõna läbi.

\section*{Inimese kahesugune loomus}

\par 14 Sest me teame, et käsk on vaimne, aga mina olen lihalik ja müüdud patu alla.
\par 15 Ma ei tunne ju ära, mida ma teen; sest ma ei tee seda, mida ma tahan, vaid mida ma vihkan, seda ma teen.
\par 16 Kui ma aga teen, mida ma ei taha, siis ma möönan, et käsk on hea.
\par 17 Nii ei tee seda enam mitte mina, vaid patt, mis minus elab.
\par 18 Sest ma tean, et minus, see on minu lihas, ei ela head. Tahet mul on, aga head teha ma ei suuda;
\par 19 sest head, mida ma tahan, ma ei tee, vaid kurja, mida ma ei taha, ma teen!
\par 20 Kui ma nüüd teen seda, mida ma ei taha, siis ei tee seda enam mitte mina, vaid patt, mis minus elab.
\par 21 Nii ma leian eneses käsu, et kuigi ma tahan teha head, on mulle omane teha kurja.
\par 22 Sest seespidise inimese poolest on mul Jumala käsust hea meel,
\par 23 aga oma liikmetes ma tunnen teist käsku, mis paneb vastu mu mõistuse käsule ja võtab mind vangi patu käsu alla, mis on mu liikmetes.
\par 24 Oh mind viletsat inimest! Kes päästab mind sellest surma ihust?
\par 25 Tänu olgu Jumalale Jeesuse Kristuse, meie Issanda läbi! Nii ma siis teenin oma mõistusega Jumala käsku, aga lihaga patu käsku.


\chapter{8}

\section*{Elu Vaimus}

\par 1 Nii ei ole siis nüüd mingisugust hukkamõistmist neile, kes on Kristuses Jeesuses.
\par 2 Sest elu Vaimu käsk Kristuses Jeesuses on sind vabastanud patu ja surma käsust.
\par 3 Sest mis oli võimatu käsule, sellepärast et ta oli jõuetu liha tõttu, seda tegi Jumal, kui ta läkitas oma Poja patuse liha samasuses ja patu pärast ning mõistis patu hukka liha sees,
\par 4 et käsu õigus täide viidaks meie sees, kes me ei käi liha järgi, vaid Vaimu järgi.
\par 5 Sest kes elavad liha järgi, nende meel on suunatud lihalikele asjadele; aga kes elavad Vaimu järgi, neil vaimsetele asjadele.
\par 6 Sest liha mõtteviis, see on surm, aga Vaimu mõtteviis, see on elu ja rahu,
\par 7 sellepärast et liha mõtteviis on vaen Jumala vastu, sest ta ei alistu Jumala käsule ega võigi alistuda.
\par 8 Kes siis lihameeles on, need ei või olla Jumalale meelepärased.
\par 9 Aga teie ei ole lihameeles, vaid vaimus, kui Jumala Vaim tõepoolest asub teie sees. Aga kellel ei ole Kristuse Vaimu, see ei ole tema oma.
\par 10 Kui aga Kristus on teie sees, siis on küll ihu surnud patu pärast; aga vaim on elu õiguse pärast.
\par 11 Aga kui selle Vaim, kes Jeesuse on surnuist üles äratanud, teis elab, siis tema, kes Kristuse Jeesuse surnuist üles äratas, teeb ka teie surelikud ihud elavaks oma Vaimu läbi, kes teis elab.
\par 12 Sellepärast nüüd, vennad, oleme küll võlglased, aga mitte lihale, et elada liha järgi.
\par 13 Sest kui te liha järgi elate, siis te surete; aga kui te Vaimu läbi ihu teod suretate, siis te elate.

\section*{Jumala lapsed}

\par 14 Sest kõik, keda iganes Jumala Vaim juhib, on Jumala lapsed.
\par 15 Sest te ei ole saanud orjapõlve vaimu, et peaksite jälle kartma, vaid te olete saanud lapsepõlve Vaimu, kelles me hüüame: „Abba! Isa!”
\par 16 Seesama Vaim tunnistab ühes meie vaimuga, et me oleme Jumala lapsed.
\par 17 Kui me aga oleme lapsed, siis oleme ka pärijad, nii Jumala pärijad kui ka Kristuse kaaspärijad, et kui me ühes temaga kannatame, siis meid ühes temaga ka austatakse.

\section*{Tulevane au}

\par 18 Sest ma arvan, et selle aja kannatused ei ole midagi tulevase au vastu, mis meile peab ilmsiks saama.
\par 19 Sest kogu loodu ootab pikisilmi Jumala laste ilmsikssaamist.
\par 20 On ju kogu loodu heidetud kaduvuse alla - mitte vabatahtlikult, vaid allaheitja tahtest - ometi lootuse peale,
\par 21 sest ka kogu loodu ise päästetakse kord kaduvuse orjusest Jumala laste au vabadusse.
\par 22 Sest me teame, et kõik loodu ühtlasi ägab ja on aina sünnitusvaevas tänini;
\par 23 aga mitte üksnes seda, vaid isegi need, kellel on Vaimu esmaand, ka meie ise ägame enestes ning ootame lapseseisust, oma ihu lunastust.
\par 24 Sest me oleme päästetud lootuses. Ent lootus, mida nähakse, ei ole mingi lootus; sest mida keegi näeb, kuidas ta seda veel loodab?
\par 25 Aga kui me loodame seda, mida me ei näe, siis me ootame seda kannatlikkusega.
\par 26 Aga samuti tuleb ka Vaim appi meie nõtrusele; sest me ei tea seda, mida paluda, nõnda nagu peaks, ent Vaim ise palvetab meie eest sõnades väljendamatute ohkamistega.
\par 27 Aga südamete uurija teab, mis Vaimul on mõttes, et ta kooskõlas Jumalaga kostab pühade eest.
\par 28 Aga me teame, et neile, kes Jumalat armastavad, kõik ühtlasi heaks tuleb, neile, kes tema kavatsuse järgi on kutsutud.
\par 29 Sest keda ta on ette ära tundnud, need on ta ka ette ära määranud olema tema Poja näo sarnased, et tema oleks esmasündinu paljude vendade seas.
\par 30 Aga keda ta on ette ära määranud, need on ta ka kutsunud, ja keda ta on kutsunud, need on ta ka õigeks teinud, ent keda ta on õigeks teinud, neid on ta ka austanud.

\section*{Jumala armastus}

\par 31 Mis me siis ütleme selle kohta? Kui Jumal on meie poolt, kes võib olla meie vastu?
\par 32 Tema, kes oma Poegagi ei säästnud, vaid loovutas tema meie kõikide eest, kuidas ta ei peaks siis temaga meile kõike muud annetama?
\par 33 Kes võib süüdistada Jumala valituid? Jumal on, kes õigeks teeb.
\par 34 Kes on, kes võib hukka mõista? Kristus Jeesus on, kes suri, ja mis veel enam, kes üles äratati, kes on Jumala paremal käel, kes meie eest palub.
\par 35 Kes võib meid lahutada Kristuse armastusest? Kas viletsus, või ahastus, või tagakiusamine, või nälg, või alastiolek, või häda, või mõõk?
\par 36 Nõnda nagu on kirjutatud: „Sinu pärast surmatakse meid kogu päev, meid arvatakse tapalambaiks!”
\par 37 Aga selles kõiges me saame täie võidu tema läbi, kes meid on armastanud!
\par 38 Sest ma olen veendunud selles, et ei surm ega elu, ei inglid, ei vürstid, ei käesolev ega tulev, ei vägevad,
\par 39 ei kõrgus ega sügavus ega mingi muu loodu või meid lahutada Jumala armastusest, mis on Kristuses Jeesuses, meie Issandas!


\chapter{9}

\section*{Apostli kurbus Iisraeli laste uskmatuse pärast}

\par 1 Ma räägin tõtt Kristuses, mina ei valeta mitte; seda tunnistab mulle mu südametunnistus Pühas Vaimus,
\par 2 et mul on suur kurbus ja lõpmatu valu mu südames.
\par 3 Sest ma sooviksin ise saada äraneetavaks, Kristusest lahku oma vendade heaks, kes on mu sugulased liha poolest,
\par 4 kes on iisraellased, kellele kuuluvad lapseõigus ja au, ja lepingud ja käsuandmine, ja jumalateenistus ja tõotused,
\par 5 kelle omad on esiisad ja kellest Kristus on pärit liha poolest, tema, kes on Jumal üle kõige, kiidetud igavesti! Aamen.

\section*{Jumala tõotused ei ole tühistunud}

\par 6 Aga ei või olla nõnda, et Jumala sõna oleks läinud tühja; sest mitte kõik need, kes põlvnevad Iisraelist, ei ole Iisrael;
\par 7 ka ei ole kõik lapsed selle tõttu, et nad on Aabrahami sugu, vaid on öeldud: „Iisakist loetakse sinu sugu!”
\par 8 See on: lihased lapsed, mitte need ei ole Jumala lapsed, vaid tõotuse lapsed arvatakse sooks.
\par 9 Sest tõotuse sõna on see: „Ma tulen jälle aasta pärast samal ajal, ja Saaral saab olema poeg!”
\par 10 Aga mitte ainult temaga, vaid ka Rebekaga oli samane lugu, kui ta ühest, see on meie esiisast Iisakist, sai käima peale.
\par 11 Sest enne kui kaksikud olid sündinud ega olnud teinud midagi head ega halba, siis - selleks, et valikule vastav Jumala otsustus jääks kindlaks mitte tegude pärast, vaid kutsuja tõttu -
\par 12 öeldi temale: „Vanem orjab nooremat!”,
\par 13 nagu ju on kirjutatud: „Jaakobit ma armastasin, aga Eesavit ma vihkasin!”

\section*{Jumal ei ole ülekohtune}

\par 14 Mida me nüüd ütleme? Kas Jumal teeb ülekohut? Mitte sugugi!
\par 15 Sest ta ütleb Moosesele: „Ma olen armuline, kellele ma olen armuline, ja halastan, kelle peale ma halastan!”
\par 16 Nõnda siis ei olene see sellest, kes tahab, ega sellest, kes jookseb, vaid Jumalast, kes armu annab.
\par 17 Sest Kiri ütleb vaaraole: „Just selleks ma olen sind õhutanud, et sinus näidata oma väge ja teha kuulsaks oma nime kogu maailmas!”
\par 18 Nõnda ta siis annab armu, kellele tahab, ja paadutab, keda tahab.

\section*{Jumala tahe on otsustav}

\par 19 Sa ehk ütled nüüd mulle: „Miks ta siis veel sõitleb? Kes võib vastu panna tema tahtele?”
\par 20 Oh inimene, nii see on! Kes siis sina oled, et sa tahad vaielda Jumala vastu? Kas see, mis tehtud, ütleb oma tegijale: „Miks sa mu nõnda tegid?”
\par 21 Või ei ole potissepal meelevalda savi ühest ja samast segust teha ühe astja õilsaks, teise halvaks otstarbeks?
\par 22 Eks Jumal, tahtes osutada oma viha ja teha teatavaks oma väge, suure pikameelega sallinud vihaastjaid, mis olid valmistatud hukatuseks,
\par 23 ja teinud seda selleks, et ilmutada oma au rikkust armuastjate vastu, mis ta ette oli valmistanud aulisteks,
\par 24 nõnda nagu tema on kutsunud ka meid, mitte ainult juutide, vaid ka paganate seast?
\par 25 Nõnda nagu ta ütleb ka Hoosea raamatus: „Ma tahan hüüda omaks rahvaks neid, kes ei olnud mu rahvas, ja oma armsaks seda, kes ei olnud mulle armas;
\par 26 ja sünnib, et seal paigas, kus neile öeldi: teie ei ole minu rahvas! hüütakse neid elava Jumala lasteks!”
\par 27 Aga Jesaja hüüab Iisraeli kohta: „Kuigi Iisraeli laste arv oleks otsekui mereliiv, pääseb sellest ainult jääk;
\par 28 sest lõplikult ja kiiresti teostab Issand oma sõna maa peal!”
\par 29 Ja nõnda nagu Jesaja on ette öelnud: „Kui mitte vägede Issand ei oleks meile jätnud seemet, oleksime olnud nagu Soodom ja saanud Gomorra sarnaseks!”

\section*{Iisraeli ilmaoleku põhjus}

\par 30 Mis me siis nüüd ütleme? Seda, et paganad, kes õigust ei taotlenud, on kätte saanud õiguse, ent õiguse, mis tuleb usust.
\par 31 Aga Iisrael, kes taotles õiguse käsku, ei ole käsku kätte saanud!
\par 32 Mispärast? Sellepärast et taotlus ei olnud usust, vaid otsekui tegudest; nad tõukasid endid vastu komistuskivi,
\par 33 nõnda nagu on kirjutatud: „Vaata, ma panen Siionisse komistuskivi ja pahanduskalju; ja kes usub temasse, ei jää häbisse!”


\chapter{10}

\section*{Iisraeli ilmaoleku põhjus}

\par 1 Vennad, minu südame hea meel ja palve Jumala poole nende eest on, et nad õndsaks saaksid!
\par 2 Sest ma annan neile tunnistuse, et nad on väga agarad Jumala suhtes, kuid mitte õiget tunnetust mööda;
\par 3 sest kui nad ei mõista Jumala õigust ja püüavad üles seada oma õigust, siis ei ole nad alistunud Jumala õigusele.
\par 4 Sest käsu lõppsiht on Kristus õiguseks igaühele, kes usub.

\section*{Uus õiguse tee on igaühe jaoks}

\par 5 Sest Mooses kirjutab õigusest, mis tuleb käsust, et inimene, kes seda täidab, elab sellest.
\par 6 Aga õigus, mis tuleb usust, ütleb nõnda: „Ära ütle oma südames: kes läheb üles taevasse? See on: Kristust alla tooma;
\par 7 või: kes läheb alla sügavusse? See on: Kristust surnuist üles tooma!”
\par 8 Vaid mis ta ütleb? „Sõna on sinule ligidal, sinu suus ja sinu südames!” See on usu sõna, mida me kuulutame!
\par 9 Sest kui sa oma suuga tunnistad, et Jeesus on Issand, ja oma südames usud, et Jumal on tema surnuist üles äratanud, siis sa saad õndsaks!
\par 10 Sest südamega usutakse õiguseks, ent suuga tunnistatakse õndsuseks.
\par 11 Sest Kiri ütleb: „Ükski, kes temasse usub, ei jää häbisse!”
\par 12 Siin ei ole vahet juudi ja kreeklase vahel, sest üks ja sama on kõikide Issand, rikas kõikide heaks, kes teda appi hüüavad.
\par 13 Sest igaüks, kes hüüab appi Issanda nime, päästetakse.
\par 14 Kuidas nad nüüd saavad appi hüüda teda, kellesse nad ei ole uskunud? Ja kuidas nad võivad uskuda temasse, kellest nad ei ole kuulnud? Ja kuidas nad saavad kuulda ilma kuulutajata?
\par 15 Ja kuidas nad võivad kuulutada, kui neid ei läkitata? Nõnda nagu on kirjutatud: „Kui armsad on nende sammud, kes häid sõnumeid kuulutavad!”
\par 16 Aga mitte kõik ei ole võtnud kuulda evangeeliumi sõna. Sest Jesaja ütleb: „Issand, kes usub meie kuulutust?”
\par 17 Nii tuleb siis usk kuuldust, aga kuuldu Kristuse sõna kaudu.
\par 18 Aga ma küsin: kas nad siis ei ole kuulnud? On küll! „Üle kogu ilmamaa käib nende hääl ja maailma otsteni nende sõna!”
\par 19 Aga ma küsin: kas Iisrael ei ole teada saanud? Esimesena ütleb Mooses: „Mina ärritan teid rahvaga, kes ei ole rahvas, ja vihastan teid mõistmatu rahvaga!”
\par 20 Ent Jesaja räägib julgesti ja ütleb: „Mind on leidnud need, kes mind ei ole otsinud; ma olen saanud ilmsiks neile, kes mind ei ole nõudnud!”
\par 21 Aga Iisraeli kohta ta ütleb: „Kogu päeva ma sirutan käsi rahva poole, kes ei kuula sõna ja räägib vastu!”


\chapter{11}

\section*{Iisraeli kõrvalejätmine ei ole lõplik}

\par 1 Siis ma nüüd ütlen: kas vahest Jumal on oma rahva hüljanud? Mitte sugugi! Sest minagi olen Iisraeli mees, Aabrahami soost, Benjamini suguharust.
\par 2 Jumal ei ole hüljanud oma rahvast, kelle tema on ette ära tundnud. Või kas te ei tea, mis Kiri ütleb Eelijast, kuidas tema Jumala ees kaebab Iisraeli pärast, öeldes:
\par 3 „Issand, sinu prohvetid on nad tapnud ja sinu altarid maha kiskunud, ja mina olen üksi üle jäänud, ja nemad tahavad võtta mu elu!”
\par 4 Aga mis ütleb temale Jumala vastus? „Mina olen enesele üle jätnud seitse tuhat meest, kes ei ole nõtkutanud põlvi Baali ees!”
\par 5 Samuti on ka siis praegusel ajal jääk üle jäänud armuvalikut mööda.
\par 6 Aga kui see on armust, siis ei ole see enam tegudest; muidu arm ei oleks arm. Aga kui see on tegudest, siis ei ole see enam arm; muidu tegu ei oleks enam tegu.
\par 7 Mis nüüd? Mida Iisrael taotleb, seda ta ei ole saavutanud; ent äravalitud on selle saavutanud ja teised on paadunud,
\par 8 nõnda nagu on kirjutatud: „Jumal on neile andnud uimuse vaimu, silmad mitte nägema ja kõrvad mitte kuulma tänapäevani!”
\par 9 Ja Taavet ütleb: „Nende laud saagu neile püüdepaelaks ja lõksuks ja pahanduseks ja kättemaksuks!
\par 10 Nende silmad saagu pimedaks, et nad ei näeks, ja murra nende selg kõveraks jäädavalt!”
\par 11 Ma ütlen nüüd: ega nad ole komistanud selleks, et langeda? Ei sugugi mitte! Vaid nende eksimise läbi tuleb pääste paganaile, et paganad teeksid neid kadedaks.
\par 12 Aga kui nende eksimine on maailma rikkus ja nende kahju paganate rikkus, saati siis nende täisarv!

\section*{Paganad liituvad Jumala rahvaga otsekui pookoksad puutüve külge}

\par 13 Mina aga kõnelen teile, paganaile! Sest sedavõrd kui ma olen paganate apostel, austan ma oma ametit,
\par 14 et võiksin kuidagi õhutada neid, kes on mu liha ja veri, ja mõned nende seast päästa.
\par 15 Sest kui juba nende hülgamine on maailma lepitus, mis on siis nende vastuvõtt muud kui surnuist ellusaamine?
\par 16 Aga kui uudsevili on püha, siis on ka taigen püha; ja kui juur on püha, siis on ka oksad pühad.
\par 17 Kui nüüd okstest mõned on ära murtud ja sina, kes olid metsõlipuu, oled jätkatud nende asemele ja oled ühes nendega osa saanud õlipuu mahlakast juurest,
\par 18 siis ära kiitle okste vastu. Ja kui sa kiitled, siis mõtle, et sina ei kanna juurt, vaid juur kannab sind.
\par 19 Sina ehk ütled nüüd: „Oksad on ära murtud, et mind külge jätkataks!”
\par 20 Õige küll! Uskmatuse pärast on nad ära murtud; aga sina püsid usu läbi. Ära saa omas meeles suureliseks, vaid karda!
\par 21 Sest kui Jumal ei säästnud loodud oksi, ega ta sindki säästa!
\par 22 Sellepärast vaata Jumala heldust ja valjust; valjust küll nende vastu, kes on langenud; aga heldust sinu vastu, kui sa püsid helduses, muidu sindki raiutakse maha!
\par 23 Aga nemadki, kui nad ei jää uskmatusse, jätkatakse puu külge; sest Jumal võib nad jälle külge jätkata.
\par 24 Sest kui sina loomulikust metsõlipuust oled ära lõigatud ja loomu vastu oled jätkatud väärisõlipuu külge, saati siis, et nemad kui loodud oksad, jätkatakse iseeneste õlipuu külge.

\section*{Jumala eesmärgiks on arm kõigile}

\par 25 Sest mina ei taha, vennad, et teile jääks teadmatuks see saladus - et teie ei oleks eneste meelest targad - et Iisraelile osalt on tulnud paadumus, kuni paganate täisarv on läinud sisse,
\par 26 ja kogu Iisrael päästetakse, nõnda nagu on kirjutatud: „Siionist tuleb päästja ja kõrvaldab Jaakobist jumalatuse.
\par 27 Ja see on minu leping nendega, kui ma ära võtan nende patud!”
\par 28 Evangeeliumi poolest on nad küll vaenlased teie pärast, aga valiku poolest on nad armastatud esiisade pärast.
\par 29 Sest Jumal ei võta tagasi oma armuande ja kutsumist!
\par 30 Sest otsekui teie muiste olite Jumalale sõnakuulmatud, aga nüüd olete armu saanud nende sõnakuulmatuse läbi,
\par 31 nõnda on nüüd ka nemad sõnakuulmatud, et ka nemad nüüd teile osaks saanud armu kaudu saaksid armu.
\par 32 Sest Jumal on kõik inimesed kinni pannud sõnakuulmatuse alla, et ta armu annaks kõikidele.
\par 33 Oh seda Jumala rikkuse ja tarkuse ja tunnetuse sügavust! Kui väljauurimatud on tema kohtumõistmised ja äraarvamatud tema teed!
\par 34 Sest kes on ära tundnud Issanda meele? Või kes on olnud tema nõuandja?
\par 35 Või kes on temale midagi enne andnud, et temale peaks jälle tasutama?
\par 36 Sest tema seest ja tema läbi ja tema poole on kõik asjad. Temale olgu au igavesti! Aamen.


\chapter{12}

\section*{Uus elu}

\par 1 Mina manitsen teid nüüd, vennad, Jumala suure südamliku halastuse tõttu andma oma ihud elavaks pühaks ja Jumala meelepäraseks ohvriks. See olgu teie mõistlik jumalateenistus.
\par 2 Ja ärge saage selle maailma sarnaseks, vaid muutuge teiseks oma meele uuendamise teel, et te uuriksite, mis on Jumala hea ja meelepärane ja täielik tahtmine.

\section*{Vaimuannete õigest kasutamisest}

\par 3 Sest selle armu läbi, mis mulle on antud, ma ütlen teile igaühele, et ta ei mõtleks üleolevalt selle kohta, mida tuleb mõelda, vaid mõtleks nõnda, et saaks arukaks sedamööda, kuidas Jumal igaühele usu mõõdu on jaganud.
\par 4 Sest otsekui meil ühes ihus on palju liikmeid, aga kõigil liikmeil ei ole sama töö,
\par 5 nõnda oleme meiegi paljud üks ihu Kristuses, aga üksikult igaüks üksteise liikmed.
\par 6 Aga meil on armu mööda, mis meile antud, mõnesuguseid armuandeid: olgu prohvetliku kuulutamise anne, mis toimugu usu mõõtu mööda;
\par 7 Olgu hoolekanne, mis toimugu hoolekandeametis; olgu keegi õpetaja, siis toimigu ta õpetamisametis;
\par 8 manitseb keegi, siis ta olgu manitsemisametis; kes teistele annetab, andku siira südamega; kes teisi juhatab, olgu hoolas; kes vaeseid hooldab, tehku seda rõõmuga.

\section*{Kristliku elu juhised}

\par 9 Armastus ärgu olgu silmakirjalik. Hoiduge kurjast eemale, kiinduge heasse!
\par 10 Olge vennaarmastuse poolest hellasüdamelised üksteise vastu; vastastikuses austamises jõudke üksteisest ette!
\par 11 Ärge olge viitsimatud hoolsuses; olge tulised vaimus; teenige Issandat!
\par 12 Olge rõõmsad lootuses, kannatlikud ahastuses, püsivad palves!
\par 13 Võtke osa pühade puudusest; püüdke olla külalislahked.
\par 14 Õnnistage neid, kes teid taga kiusavad! Õnnistage ja ärge needke!
\par 15 Olge rõõmsad rõõmsatega, nutke nutjatega!
\par 16 Mõtelge ühesuguselt üksteise suhtes! Ärge nõudke kõrgeid asju, vaid hoiduge madaluse poole; ärge olge eneste meelest targad!
\par 17 Ärge tasuge ühelegi kurja kurjaga; mõtelge ikka sellele, mis hea on kõigi inimeste suhtes!
\par 18 Kui võimalik on ja niipalju kui teist oleneb, pidage rahu kõigi inimestega.
\par 19 Ärge makske ise kätte, armsad, vaid andke maad Jumala vihale, sest on kirjutatud: „Minu käes on kättemaks, mina tasun kätte! ütleb Issand.”
\par 20 „Kui nüüd sinu vaenlasel on nälg, sööda teda; kui tal on janu, jooda teda; sest seda tehes sa kogud tuliseid süsi tema pea peale!”
\par 21 Ära lase kurjal võitu saada enese üle, vaid võida sina kuri ära heaga!


\chapter{13}

\section*{Valitsejaile alistumisest}

\par 1 Iga hing olgu allaheitlik valitsemas olevaile ülemustele; sest ülemust ei ole muud kui Jumalalt; kus neid on, seal on nad Jumala poolt seatud.
\par 2 Kes nüüd ülemusele vastu paneb, see paneb vastu Jumala korrale; aga kes vastu panevad, saadavad eneste peale nuhtluse.
\par 3 Sest valitsejad ei ole hirmuks headele tegudele, vaid kurjadele. Kui sina aga ei taha tunda hirmu ülemuse ees, siis tee head, ja siis sa saad temalt kiitust.
\par 4 Sest tema on Jumala teener sinu heaks; aga kui sa kurja teed, karda, sest ta ei kanna mõõka asjata; ta on Jumala teenija, kättemaksja nuhtluseks sellele, kes kurja teeb.
\par 5 Siis on tarvis olla allaheitlik, mitte ainult nuhtluse, vaid ka südametunnistuse pärast.
\par 6 Sellepärast makske ka makse; sest nemad on Jumala sulased, kes selleks peavad oma ametit.
\par 7 Tasuge kõigile, mis teie kohus on: maksu, kellele maksu; tolliraha, kellele tolliraha; kartust, kellele kartust; au, kellele au tuleb anda!

\section*{Vennalikust armastusest}

\par 8 Ärgu olgu teil midagi kellegagi võlgu, kui et te üksteist armastate, sest kes teist armastab, on käsu täitnud.
\par 9 Sest käsk: „Sa ei tohi abielu rikkuda; sa ei tohi tappa; sa ei tohi varastada; sa ei tohi ülekohut tunnistada; sa ei tohi himustada”, ja iga teine käsk on kokku võetud sellesse sõnasse: „Armasta oma ligimest nagu iseennast!”
\par 10 Armastus ei tee ligimesele kurja; siis on armastus käsu täitmine.

\section*{Kristuse päev koidab}

\par 11 Ja seda tehke sellepärast et te teate tunni juba käes olevat unest üles ärgata; sest nüüd on meie pääste lähemal kui siis, kui me usklikuks saime.
\par 12 Öö on möödumas, aga päev on lähedal. Siis heitkem enestest ära pimeduse teod ja varustugem valguse relvadega!
\par 13 Elagem ausasti nagu päeva ajal, mitte öistes pidutsemistes ega joominguis, mitte kiimaluses ega iharuses, mitte riius ega kadeduses,
\par 14 vaid varustuge Issanda Jeesuse Kristusega ja ärge muutke liha eest hoolitsemist himude ärritamiseks!


\chapter{14}

\section*{Nõdrausuliste kohtlemisest}

\par 1 Usu poolest nõrgale olge vastutulelikud ilma vaidlust alustamata arvamiste kohta.
\par 2 Mõni usub, et ta võib süüa kõike, aga kes on nõrk, sööb taimetoitu.
\par 3 Kes sööb, ärgu pidagu halvaks seda, kes mitte ei söö; ja kes ei söö, ärgu mõistku hukka seda, kes sööb; sest Jumal on tema vastu võtnud.
\par 4 Kes oled sina, et sa mõistad hukka teise sulase? Tema seisab või langeb iseenese isandale. Aga tema jääb seisma, sest Jumal võib teda püsti hoida.
\par 5 Mõni arvab üht päeva paremaks teisest, aga teine peab kõik päevad ühesuguseks. Igaüks olgu enese mõttes kindel omas veendumuses.
\par 6 Kes paneb tähele päeva, paneb seda tähele Issandale; ja kes sööb, sööb Issandale, sest tema tänab Jumalat; ja kes mitte ei söö, jääb söömata Issandale ning tänab Jumalat.
\par 7 Sest keegi meist ei ela iseenesele ja keegi meist ei sure iseenesele.
\par 8 Sest kui me elame, siis elame Issandale, ja kui me sureme, siis sureme Issandale. Kas me siis elame või sureme, oleme Issanda omad!
\par 9 Sest Kristus suri selleks ja sai jälle elavaks, et ta oleks nii surnute kui ka elavate Issand.
\par 10 Aga sina, miks sa mõistad kohut oma venna üle? Või sina teine, miks sa paned oma venda halvaks? Meid kõiki ju seatakse Jumala kohtujärje ette.
\par 11 Sest on kirjutatud: „Nii tõesti kui ma elan, ütleb Issand, minu ees peab nõtkuma iga põlv ja iga keel peab tunnistama Jumalat!”
\par 12 Nõnda tuleb meil igaühel anda aru iseenesest Jumalale.
\par 13 Ärgem siis enam mõistkem kohut üksteise üle, vaid pigemini otsustagem hoiduda saamast vennale komistuseks ja pahanduseks.
\par 14 Ma tean ja olen veendunud Issandas Jeesuses, et midagi ei ole ebapüha iseenesest, vaid on ebapüha üksnes sellele, kes midagi peab ebapühaks.
\par 15 Aga kui sinu vend saab nukraks roa pärast, siis ei käi sina enam armastuse järgi. Ära saada hukka oma roaga seda, kelle eest Kristus on surnud.
\par 16 Sellepärast ärgu saagu hea, mis teil on, pilgatavaks.
\par 17 Sest Jumala riik ei ole mitte söömine ega joomine, vaid õigus ja rahu ja rõõm Pühas Vaimus.
\par 18 Sest kes neis asjus orjab Kristust, see on Jumalale meelepärane ja kõlvuline inimestele.
\par 19 Siis taotlegem nüüd seda, mis läheb tarvis rahuks ja vastastikuseks paranduseks.
\par 20 Ära riku roa pärast Jumala teost! Kõik on küll puhas, aga paha on see sellele inimesele, kes sööb pahandusega.
\par 21 Hea on liha mitte süüa ja viina mitte juua, ja mitte teha seda, mis su vennale on komistuseks.
\par 22 Usk, mis sul on, pea enesele Jumala ees. Õnnis on see, kes iseennast ei mõista hukka selles, mida ta peab õigeks;
\par 23 aga kui keegi kõheldes sööb, siis on ta hukka mõistetud, sest see ei tule usust. Aga kõik, mis ei ole usust, on patt!



\chapter{15}

\section*{Nõdrausuliste kohtlemisest}

\par 1 Aga meie, kes oleme tugevad, peame kandma jõuetute nõrkusi ega tohi elada iseeneste meele heaks.
\par 2 Igaüks meist olgu ligimesele meelepärane tema heaks, et teda parandada.
\par 3 Sest Kristuski ei elanud enese meele heaks, vaid nõnda nagu on kirjutatud: „Nende teotamised, kes sind teotavad, on langenud minu peale!”
\par 4 Sest mis iganes enne on kirjutatud, on kirjutatud meile õpetuseks, et meil kannatlikkuse ja Kirja troosti kaudu oleks lootust.
\par 5 Aga kannatlikkuse ja troosti Jumal andku teile üksmeelt isekeskis Kristuse Jeesuse järele,
\par 6 et te ühel meelel ja ühe suuga austaksite Jumalat ja meie Issanda Jeesuse Kristuse Isa!

\section*{Evangeelium paganaile}

\par 7 Sellepärast võtke vastu üksteist, nõnda nagu ka Kristus teid on vastu võtnud Jumala austuseks.
\par 8 Sest ma ütlen, et Jeesus Kristus sai ümberlõikamise sulaseks Jumala tõe pärast, et kinnitada isadele antud tõotused,
\par 9 aga et paganad ülistavad Jumalat tema halastuse pärast, nõnda nagu on kirjutatud: „Sellepärast ma tahan sind ülistada paganate seas ja laulda kiitust su nimele!”
\par 10 Ja taas ütleb tema: „Olge rõõmsad, paganad, ühes tema rahvaga!”
\par 11 Ja veel: „Kiitke Issandat, kõik paganad, ja ülistage teda, kõik rahvad!”
\par 12 Ja jälle ütleb Jesaja: „Jesse juur tuleb ja see, kes tõuseb valitsema paganrahvaid; tema peale hakkavad paganad lootma!”
\par 13 Aga lootuse Jumal täitku teid kõige rõõmu ja rahuga usus, et teil oleks rohkesti lootust Püha Vaimu väes!

\section*{Avameelsusest apostli kirjas}

\par 14 Mu vennad, mina isiklikult olen teie suhtes veendunud selles, et teie isegi olete täis headust, täidetud kõige teadmisega ning võite küll juhatada üksteist.
\par 15 Ometi olen ma üsna julgesti mõnd teile kirjutanud, vennad, otsekui teile meeldetuletuseks, selle armu pärast, mille Jumal mulle on andnud selleks,
\par 16 et ma oleksin Jeesuse Kristuse teenija paganate seas ja toimetaksin Jumala evangeeliumi püha teenistust, et paganad saaksid meelepäraseks ja Pühas Vaimus pühitsetud ohvriannetuseks.
\par 17 Sellepärast ma tohin Kristuses Jeesuses kiidelda Jumala asjust.
\par 18 Sest mina ei julge kõnelda millestki muust kui sellest, mis Kristus minu läbi paganate sõnakuulelikuks tegemiseks on korda saatnud sõna ja teoga
\par 19 tunnustähtede ja imetegude väel, Püha Vaimu väel, nõnda et mina Jeruusalemmast ja selle ümberkaudsest maast alates Illürikonini olen täide viinud Kristuse evangeeliumi kuulutamise;
\par 20 kuna ma nõnda olen pidanud oma auks armuõpetust mitte kuulutada seal, kus Kristust juba on nimetatud, et ma ei ehitaks teiste rajatud alusele,
\par 21 vaid nõnda nagu on kirjutatud: „Need, kellele ei ole temast midagi kuulutatud, saavad näha, ja need, kes ei ole kuulnud, mõistavad!”

\section*{Apostli isiklikest kavatsusist}

\par 22 See ongi see, mis mind mitu korda on takistanud teie juurde tulemast.
\par 23 Aga et mul nüüd ei ole neis paigus enam aset ja et ma nii mitu aastat väga olen igatsenud teie juurde tulla,
\par 24 siis ma ehk Hispaaniasse minnes tulen teie juurde, sest ma loodan teie kaudu matkates näha teid ja teie saatel jõuda sinna, niipea kui ma enne teie keskel ennast olen pisut rõõmustanud.
\par 25 Aga nüüd ma lähen Jeruusalemma püha rahvast abistama.
\par 26 Sest Makedoonia ja Ahhaia on heaks arvanud toime panna korjanduse neile, kes Jeruusalemmas elavate seas kehvad on.
\par 27 Nemad on seda heaks arvanud ja nad on ka nende võlglased. Sest kui paganad on osa saanud nende vaimulikest andidest, siis on ka nende kohus neid aidata ihulike andidega.
\par 28 Kui ma siis selle asja olen toimetanud ja neile selle saavutuse ustavalt kätte saatnud, tahan ma teie kaudu minna Hispaaniasse.
\par 29 Aga ma tean, kui ma teie juurde tulen, et ma siis tulen Kristuse täie õnnistusega.
\par 30 Ent mina manitsen teid, vennad, meie Issanda Jeesuse Kristuse läbi ja Vaimu armastuse läbi, minuga ühes võidelda eestpalvetega minu eest Jumala poole,
\par 31 et ma pääseksin uskmatu rahva käest Juudamaal ja et minu abistus Jeruusalemma heaks oleks armas pühade meelest,
\par 32 et ma Jumala tahtel võiksin rõõmuga tulla teie juurde ja teie keskel leida kosutust.
\par 33 Aga rahu Jumal olgu teie kõikidega! Aamen.


\chapter{16}

\section*{Nimelised tervitused}

\par 1 Mina annan teie hooleks meie õe Foibe, kes on Kenkrea koguduse käsiline,
\par 2 et te võtaksite teda vastu Issandas, nõnda nagu sobib pühadele, ja oleksite temale abiks kõiges, milles temal iganes teid on vaja. Sest tema on abiks olnud paljudele, ka mulle enesele.
\par 3 Tervitage minu kaastöölisi Kristuses Jeesuses, Priskat ja Akvilat,
\par 4 kes minu elu eest on pannud oma kaela tapavalmis, keda ei täna mitte mina üksi, vaid ka kõik paganate kogudused,
\par 5 ja kogudust nende majas. Tervitage Epainetost, minu armast, kes oli esimene Aasias pöörduma Kristuse poole.
\par 6 Tervitage Maarjat, kes on palju vaeva näinud teie eest.
\par 7 Tervitage Andronikost ja Juuniust, minu sugulasi ja kaasvange, kes on kuulsad apostlite hulgas ja kes enne mind on olnud Kristuses.
\par 8 Tervitage Ampliatost, mu armast Issandas.
\par 9 Tervitage Urbanust, meie kaastöölist Kristuses, ja mu armast Stahhüst.
\par 10 Tervitage Apellest, kes on läbi katsutud Kristuses. Tervitage Aristobulose perekonda.
\par 11 Tervitage Heroodioni, minu sugulast. Tervitage Narkissose perekonna liikmeid, kes on Issandas.
\par 12 Tervitage Trüfainat ja Trüfoosat, kes teevad tööd Issandas. Tervitage Persist, armast õde, kes on näinud palju vaeva Issandas.
\par 13 Tervitage Ruufust, Issandas äravalitut, ja tema ema, kes on ka minu ema.
\par 14 Tervitage Asünkritost, Flegonit, Hermest, Patrobast, Hermast ja vendi, kes on nende juures.
\par 15 Tervitage Filologost ja Juuliat, Neereust ja tema õde ja Olümpat ja kõiki pühasid, kes on nende juures.
\par 16 Tervitage üksteist püha suudlusega. Kõik Kristuse kogudused tervitavad teid.
\par 17 Aga mina manitsen teid, vennad, pidama silmas neid, kes tekitavad lõhesid ja pahandust õpetuse vastu, mis te olete õppinud, ja tõmbuge neist tagasi.
\par 18 Sest niisugused inimesed ei teeni mitte meie Issandat Jeesust, vaid oma kõhtu, ja petavad magusate sõnadega ja libedate kõnedega tõemeelsete südameid.
\par 19 Sest teie sõnakuulmine on saanud kõigile teatavaks; seepärast ma rõõmustun teist, kuid tahan, et teie oleksite targad heale, aga kohtlased pahale.
\par 20 Aga rahu Jumal tallab varsti saatana teie jalge alla. Meie Issanda Jeesuse Kristuse arm olgu teiega! Aamen.
\par 21 Teid tervitavad Timoteos, minu kaastööline, ja Luukius ja Jaason ja Sosipatros, minu sugulased.
\par 22 Mina, Tertius, kes ma selle kirja olen kirjutanud, tervitan teid Issandas.
\par 23 Teid tervitab Gaajus, kes on olnud külalislahke minule ja kõigele kogudusele. Teid tervitab Erastos, linna laekur, ja vend Kvartus.
\par 24 [Meie Issanda Jeesuse Kristuse arm olgu teie kõikidega! Aamen.]

\section*{Jumala ülistus}

\par 25 Ent sellele, kes teid võib kinnitada minu evangeeliumi ja Jeesuse Kristuse jutluse järgi selle saladuse ilmutust mööda, mis on olnud igavesest ajast varjul,
\par 26 aga nüüd on saanud avalikuks ja on prohvetite kirjade kaudu igavese Jumala käsul tehtud teatavaks kõigile rahvastele usu sõnakuulelikkuse saavutamiseks -
\par 27 ainsale targale Jumalale, Jeesuse Kristuse läbi, temale olgu au ajastute ajastuteni! Aamen.





\end{document}