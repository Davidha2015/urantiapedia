\begin{document}

\title{Pauluse Esimene kiri korintlastele}

\chapter{1}

\section*{Tervitus ja tänu}

\par 1 Paulus, Jumala tahtel kutsutud Kristuse Jeesuse apostel, ja vend Soostenes
\par 2 Korintoses olevale Jumala kogudusele, pühitsetuile Kristuses Jeesuses, kutsutud pühadele, ühes nende kõikidega, kes appi hüüavad meie Issanda Jeesuse Kristuse nime kõigis nende ja meie paigus.
\par 3 Armu teile ja rahu Jumalalt, meie Isalt ja Issandalt Jeesuselt Kristuselt.
\par 4 Ma tänan alati oma Jumalat teie pärast selle Jumala armu eest, mis teile on antud Kristuses Jeesuses,
\par 5 et te olete kõiges saanud rikkaks tema sees igasuguses õpetuse sõnas ja igasuguses tunnetuses,
\par 6 sedamööda nagu tunnistus Kristusest on kinnitatud teie sees,
\par 7 nõnda et teil ei ole puudu ühestki armuandest ja ootate meie Issanda Kristuse ilmumist,
\par 8 kes ka teid tahab kinnitada otsani, et te võiksite olla laitmatud meie Issanda Jeesuse Kristuse päeval.
\par 9 Jumal on ustav, kelle läbi te olete kutsutud tema Poja Jeesuse Kristuse, meie Issanda osadusse.

\section*{Lõhenemine koguduses}

\par 10 Aga ma manitsen teid, vennad, meie Issanda Jeesuse Kristuse nime läbi, et te kõik ühtviisi räägiksite ja et ei oleks lõhesid teie seas, vaid et oleksite kokku liidetud samas meeles ja samas mõttes.
\par 11 Sest ma olen Kloe pererahvalt teist, mu vennad, saanud kuulda, et teie seas on riidlemisi.
\par 12 Ma räägin sellest, et teie seast igaüks räägib omaviisi; üks: „Mina olen Pauluse poolt!” teine: „Mina olen Apollose poolt!” kolmas: „Mina olen Keefase poolt!” ja neljas: „Mina olen Kristuse poolt!”
\par 13 Kas Kristus on jaotatud mitmeks? Kas Paulus löödi risti teie eest? Või ristiti teid Pauluse nimesse?
\par 14 Ma tänan Jumalat, et ma kedagi muud teie seast ei ole ristinud kui üksnes Krispose ja Gaajuse,
\par 15 et keegi ei saaks öelda, ma olevat ristinud iseenese nimesse.
\par 16 Siiski ma ristisin ka Stefanase pere; muud kedagi ma ei tea ristinud olevat.
\par 17 Sest Kristus ei läkitanud mind ristima, vaid armuõpetust kuulutama, mitte sõna tarkuses, et Kristuse risti ei tehtaks tühiseks.

\section*{Risti vägi}

\par 18 Sest sõna ristist on jõledus neile, kes hukka lähevad, aga meile, kes päästetakse, on see Jumala vägi.
\par 19 Sest on kirjutatud: „Mina kaotan tarkade tarkuse ja teen tühjaks mõistlike mõistuse!”
\par 20 Kus on targad? Kus kirjatundjad? Kus selle ajastu arutlejad? Eks Jumal ole selle maailma tarkuse teinud jõleduseks?
\par 21 Sest maailm Jumala tarkuses ei õppinud tundma Jumalat tarkuse kaudu ja seepärast oli Jumalale meelepärast jõleda jutluse kaudu päästa need, kes usuvad.
\par 22 Sest juudid nõuavad tunnustähti ja kreeklased otsivad tarkust,
\par 23 aga meie kuulutame ristilöödud Kristust, kes on juutidele pahandus ja paganaile jõledus,
\par 24 kuid neile nende seast, kes on kutsutud, niihästi juutidele kui kreeklastele, me kuulutame Kristust, Jumala väge ja Jumala tarkust.
\par 25 Sest mis jõle on Jumala ees, on targem kui inimesed, ja mis nõder on Jumala ees, on vägevam kui inimesed.
\par 26 Vaadake ometi, vennad, oma kutsumist, et mitte palju tarku liha poolest, mitte palju vägevaid, mitte palju suursuguseid,
\par 27 vaid mis on jõle maailma ees, selle on Jumal valinud, et tema teeks häbi tarkadele; ja mis on nõder maailma ees, selle on Jumal valinud, et tema teeks häbi sellele, mis on vägev;
\par 28 ja mis maailma ees on halvast soost ja põlatud, selle on Jumal valinud, selle, mis midagi ei ole, et teha tühjaks seda, mis midagi on,
\par 29 et ükski liha ei kiitleks Jumala ees.
\par 30 Ent temast on teie olemine Kristuses Jeesuses, kes on saanud meile tarkuseks Jumalalt ja õiguseks ja pühitsuseks ja lunastuseks,
\par 31 et oleks nõnda, nagu on kirjutatud: „Kes kiitleb, see kiidelgu Issandast!”


\chapter{2}

\section*{Inimlik tarkus ja Jumala ilmutus}

\par 1 Ka mina, kui ma tulin teie juurde, vennad, ei tulnud mitte kõne või tarkuse hiilgega teile kuulutama Jumala tunnistust.
\par 2 Sest ma ei arvanud teie seas mitte midagi muud teadvat kui vaid Jeesust Kristust, ja teda kui ristilöödut.
\par 3 Ja mina olin teie juures nõtruses ja kartuses ja suures väristuses.
\par 4 Ja minu kõne ja minu jutlus ei olnud inimliku tarkuse meelitavais sõnus, vaid Vaimu ja väe osutamises,
\par 5 et teie usk ei oleks inimeste tarkuses, vaid Jumala väes.
\par 6 Ent me kuulutame tarkust täiuslike seas, aga mitte selle maailma ega selle ajastu ülemate tarkust, kes hävivad,
\par 7 vaid me kuulutame Jumala salajas olevat ja varjatud tarkust, mis Jumal on ette määranud enne maailma-ajastute algust meie auks,
\par 8 mida ükski selle maailma ülemaist ei ole tundnud; sest kui nad seda oleksid tundnud, ei oleks nad au Issandat risti löönud,
\par 9 vaid nõnda nagu on kirjutatud: „Mida silm ei ole näinud ja kõrv ei ole kuulnud ja mis inimese südamesse ei ole tõusnud, mis Jumal on valmistanud neile, kes teda armastavad!”
\par 10 Ent Jumal on selle meile ilmutanud Vaimu läbi; sest Vaim uurib kõik asjad, ka Jumala sügavused.
\par 11 Sest kes inimesist teab seda, mis on inimeses, kui aga inimese vaim, kes temas on? Samuti ei tea ka ükski, mis on Jumalas, kui aga Jumala Vaim.
\par 12 Aga meie ei ole saanud maailma vaimu, vaid Vaimu, kes on Jumalast, et teaksime, mis Jumal meile armust on andnud,
\par 13 ja seda me ka räägime mitte sõnadega, mida õpetab inimlik tarkus, vaid mida õpetab Vaim, ja seletame vaimseid asju vaimsetega.
\par 14 Aga maine inimene ei võta seda vastu, mis on Jumala Vaimust; sest see on temale jõledus ja ta ei või sellest aru saada, sellepärast et seda tuleb ära mõista vaimselt.
\par 15 Aga vaimulik inimene mõistab kõik ära, ent teda ennast ei mõista ükski ära.
\par 16 Sest kes on Issanda meele ära tundnud, et temale nõu anda? Aga meil on Kristuse meel!


\chapter{3}

\section*{Lahkmeel on hukkamõistetav}

\par 1 Ka mina, vennad, ei võinud teile rääkida kui vaimulikele, vaid kui lihalikele, kui väetitele lastele Kristuses.
\par 2 Ma jootsin teile piima ega andnud tahket rooga; sest seda te ei talunud sellal ega talu veel praegugi.
\par 3 Te olete ju alles lihalikud. Sest kui teie seas on kadedust ja riidu, eks te siis ole lihalikud ja eks te käitu inimeste taoliselt?
\par 4 Sest kui üks ütleb: „Mina olen Pauluse poolt!” ja teine: „Mina olen Apollose poolt!” eks te siis ole nagu inimesed kunagi?
\par 5 Mis on siis Apollos? Või mis on Paulus? Mis muud kui käsilised, kelle kaudu te olete saanud usklikuks, ja sedamööda, kuidas Issand kummalegi on andnud.
\par 6 Mina istutasin, Apollos kastis, aga Jumal laskis kasvada.
\par 7 Nõnda ei ole siis midagi see, kes istutab, ega see, kes kastab, vaid Jumal, kes laseb kasvada.
\par 8 Aga istutaja ja kastja on ühesugused; kuid kumbki saab oma palga oma töö järgi.
\par 9 Sest me oleme Jumala kaastöölised, teie olete Jumala põllumaa ja Jumala ehitus.

\section*{Õpetaja vastutus}

\par 10 Jumala armu järgi, mis minule on antud, olen ma kui tark hooneehitaja pannud aluse, ja teine ehitab selle peale; aga igaüks vaadaku, kuidas tema selle peale ehitab.
\par 11 Sest teist alust ei või keegi panna kui see, mis juba on pandud, see on Jeesus Kristus!
\par 12 Aga kui keegi sellele alusele ehitab hoone kullast, hõbedast, kalliskividest, puust, heintest, õlest,
\par 13 siis saab igaühe töö avalikuks; sest selle teeb selgeks Issanda päev, sest see saab tules ilmsiks ja tuli ise katsub läbi igaühe töö, missugune see on.
\par 14 Kui kellegi töö, mis tema sellele alusele on ehitanud, jääb püsima, siis ta saab palga;
\par 15 aga kui kellegi töö põleb ära, siis ta saab sellest kahju; kuid tema ise pääseb, aga ometi otsekui läbi tule.
\par 16 Eks te tea, et te olete Jumala tempel ja et Jumala Vaim elab teie sees?
\par 17 Kui keegi rikub Jumala templi, siis Jumal rikub tema; sest Jumala tempel on püha ja seesama olete teie.
\par 18 Ärgu ükski ennast petku! Kui keegi teie seast arvab enese targa olevat selles maailmas, siis ta saagu jõledaks, et ta saaks targaks.
\par 19 Sest selle maailma tarkus on jõledus Jumala ees. Sest on kirjutatud: „Tema tabab tarku nende kavaluses!”
\par 20 Ja jälle: „Issand tunneb tarkade mõtlemised, et need on tühised!”
\par 21 Ärgu siis ükski kiidelgu inimestest! Sest kõik on teie päralt,
\par 22 olgu Paulus, olgu Apollos, olgu Keefas, olgu maailm, olgu elu, olgu surm, olgu praegused asjad, olgu tulevased asjad, kõik on teie päralt;
\par 23 aga teie olete Kristuse päralt ja Kristus on Jumala päralt!


\chapter{4}

\section*{Apostli vastutus Jumala ees}

\par 1 Nõnda pidagu meid igaüks Kristuse sulaseiks ja Jumala, saladuste majapidajaiks.
\par 2 Ent majapidajailt nõutakse sel juhul kõigepealt, et igaüks leitaks ustav olevat.
\par 3 Aga minule ei tähenda see midagi, et teie või mõni inimlik kohtupäev minu üle kohut mõistate; ja mina isegi ei mõista kohut enese üle.
\par 4 Sest mina ei tea enesel ühtki süüd olevat; aga sellega ei ole ma veel õigeks mõistetud. Sest Issand on see, kes minu üle kohut mõistab.
\par 5 Sellepärast ärge mõistke ühtki kohut enneaegu, enne kui Issand tuleb, kes ka pimeduse salajased asjad toob valge ette ja südamete nõud teeb avalikuks; ja siis saab igaüks kiituse Jumalalt.
\par 6 Seda olen mina, vennad, tähendanud enese ja Apollose kohta teie pärast, et te meist õpiksite - mitte üle selle, mis on kirjutatud -, et keegi teist ei uhkustaks teise vastu.
\par 7 Sest kes eesõigustab sind? Ja mis on sinul, mida sina ei oleks saanud? Aga kui sa selle oled saanud, mis sa siis kiitled, otsekui ei oleks sa seda mitte saanud?
\par 8 Juba te olete täis söönud, juba olete saanud rikkaks; te olete saanud kuningaiks ilma meieta. Kui te ometi tõesti oleksite kuningad, et ka meie võiksime ühes teiega olla kuningad!
\par 9 Sest mulle näib, et Jumal meid, apostleid, on asetanud viimasteks, otsekui surmamõistetuiks; sest me oleme saanud silmaimeks maailmale, nii inglitele kui ka inimestele;
\par 10 meie oleme ogarad Kristuse pärast, aga teie olete arukad Kristuses; meie oleme nõdrad, aga teie olete tugevad; teie ausad, aga meie autud.
\par 11 Praeguse hetkenigi kannatame nälga ja janu ja oleme alasti, ja meid pekstakse rusikatega ja meil ei ole kuskil asu.
\par 12 Ja me näeme vaeva tööd tehes oma kätega; kui meid sõideldakse, siis me õnnistame; kui meid taga kiusatakse, siis me kannatame ära;
\par 13 kui meid sõimatakse, siis me räägime lahkesti. Me oleme saanud otsekui maailma pühkmeks ja kõikide roistuks tänapäevani.

\section*{Isalik nõuanne ja manitsus}

\par 14 Mina ei kirjuta seda teie häbistamiseks, vaid noomin teid kui oma armsaid lapsi.
\par 15 Sest ehkki teil oleks kümme tuhat kasvatajat Kristuses, ometi ei ole teil mitut isa, sest mina sünnitasin teid evangeeliumi läbi Kristuses Jeesuses.
\par 16 Sellepärast ma manitsen teid: võtke mind eeskujuks!
\par 17 Otse sellepärast ma läkitasin teile Timoteose, kes on mu armas ja ustav poeg Issandas. Tema tuletab teile meelde minu teed, mida ma käin Kristuses Jeesuses, nõnda nagu ma igal pool koguduses õpetan.
\par 18 Mõned on hakanud suurustama, otsekui ma ei tulekski teie juurde.
\par 19 Aga ma tulen varsti teie juurde, kui Issand tahab, ja siis saan selgusele mitte suurustajate sõnade, vaid väe kohta!
\par 20 Sest Jumala riik ei ole mitte sõnades, vaid väes!
\par 21 Mis te tahate: kas pean tulema teie juurde vitsaga või armastuse ning tasase vaimuga?


\chapter{5}

\section*{Ebakõlbelisest olukorrast koguduses}

\par 1 Üldse kuuldub teie seas olevat hoorust ja niisugust hoorust, millist ei leidu paganategi seas, et kellelgi on oma isa naine.
\par 2 Ja te olete pealegi suurelised ega ole pigemini saanud kurvaks, et see, kes niisuguse teo on teinud, lükataks teie seast välja.
\par 3 Aga mina, ehk ma küll olen ihulikult eemal, aga vaimus siiski ligi, olen otsekui teie juures viibides omalt poolt juba otsustanud, et see, kes niisuguse teo on teinud -
\par 4 pärast seda kui teie ja minu vaim ühes meie Issanda Jeesuse väega on kokku saanud -
\par 5 Issanda Jeesuse nimel üle antaks saatanale liha hukkamiseks, et ta vaim päästetaks Issanda päeval.
\par 6 Teie kiitlemine ei ole hea. Eks te tea, et pisut haputaignat teeb kogu taigna hapuks?
\par 7 Pühkige välja vana haputaigen, et oleksite uus taigen, nõnda nagu te oletegi hapnemata. Sest ka meie paasatall Kristus on tapetud.
\par 8 Siis pidagem pühi, mitte vana haputaignaga, mitte pahede ega kurjuse haputaignaga, vaid puhtuse ja tõe hapnemata taignaga.
\par 9 Ma kirjutasin oma kirjas, et teil ärgu olgu tegemist hoorajatega,
\par 10 muidugi mitte üldiselt selle maailma hoorajatega ja ahnetega ja riisujatega ja ebajumala kummardajatega, sest muidu te peaksite minema ära maailmast.
\par 11 Aga praegu ma kirjutasin teile, et teil ärgu olgu midagi tegemist niisugusega, keda nimetatakse vennaks ja kes on kas hooraja või ahne või ebajumala kummardaja või pilkaja või joodik või anastaja; niisugusega ärge ühes söögegi.
\par 12 Sest kas minu asi on mõista kohut nende üle, kes on väljas? Eks te mõista kohut nende üle, kes on sees?
\par 13 Ent nende üle, kes on väljas, mõistab Jumal kohut. Lükake siis kurjategija endi seast välja.


\chapter{6}

\section*{Kristlaste omavahelisest kohtuskäimisest}

\par 1 Kuidas julgeb keegi teie seast, kui tal on tegemist teisega, käia kohut ülekohtuste juures ja mitte pühade juures?
\par 2 Või kas te ei tea, et pühad mõistavad kohut maailma üle? Ja kui te mõistate kohut maailma üle, kas te siis ei kõlba otsuseid tegema kõige vähemais asjus?
\par 3 Eks te tea, et kord me mõistame kohut inglitegi üle, miks siis mitte ajalikes asjus?
\par 4 Kui teil nüüd on õiendusi ajalikes asjus, kas te siis seate kohtunikeks neid, kellel ei ole lugupidamist koguduses?
\par 5 Seda ma ütlen teile häbiks. Kas siis teie seas ei ole ühtainustki tarka, kes kohut mõistaks venna ja venna vahel?
\par 6 Vaid vend käib kohut vennaga ja pealegi uskmatute ees!
\par 7 Juba seegi on üldse veaks teie seas, et te üksteisega kohut käite. Miks te pigemini ei kannata ülekohut? Miks te pigemini ei lase enestele kahju teha?
\par 8 Vaid te teete ülekohut ja kahju, ja pealegi vendadele!
\par 9 Või kas te ei tea, et ülekohtused ei päri Jumala riiki? Ärge eksige: hoorajad ja ebajumala kummardajad ja abielurikkujad ja salajased ropud ja poisipilastajad
\par 10 ja vargad ja ahned, joodikud, pilkajad ja anastajad ei päri Jumala riiki!
\par 11 Ja sellised olid teist mõningad; aga te olete puhtaks pestud, olete pühitsetud, olete õigeks tehtud Issanda Jeesuse Kristuse nimes ja meie Jumala Vaimus.

\section*{Kõlvatu elu ei ole kristlasele kohane}

\par 12 Kõike on mul luba teha, aga kõigest ei ole kasu; kõike on mul luba teha, aga miski ei tohi saada meelevalda minu üle.
\par 13 Toit on kõhu jaoks ja kõht on toidu jaoks, ja Jumal teeb lõpu nii ühele kui teisele. Aga ihu ei ole hooruse jaoks, vaid Issanda jaoks ja Issand ihu jaoks.
\par 14 Ent Jumal on üles äratanud Issanda ja äratab üles ka meid oma väe läbi.
\par 15 Eks te tea, et teie ihud on Kristuse liikmed? Kas peaksin siis võtma Kristuse liikmed ja tegema nad hoora liikmeiks? Ärgu seda sündigu!
\par 16 Või te ei tea, et see, kes hoiab hoora poole, on temaga üks ihu? Sest „need kaks”, ütleb tema, „peavad olema üks liha!”
\par 17 Aga kes hoiab Issanda poole, see on üks vaim temaga.
\par 18 Põgenege hooruse eest! Iga patt, mida inimene iganes teeb, on väljaspool ihu, aga kes hoorab, teeb pattu omaenese ihu vastu.
\par 19 Või te ei tea, et teie ihu on teie sees oleva Püha Vaimu tempel, kelle te olete saanud Jumalalt, ja et te ei ole iseeneste omad?
\par 20 Sest te olete kalli hinnaga ostetud. Austage siis Jumalat oma ihus!


\chapter{7}

\section*{Abielu ja vallaspõlve küsimus}

\par 1 Aga mis puutub sellesse, millest kirjutasite, siis mehele on hea naist mitte puudutada.
\par 2 Kuid hooruspattude pärast olgu igal mehel oma naine ja igal naisel oma mees.
\par 3 Mees täitku oma kohust naise vastu ja nõndasamuti ka naine mehe vastu.
\par 4 Naisel ei ole meelevalda oma ihu üle, vaid mehel; ja nõndasamuti ei ole mehel meelevalda oma ihu üle, vaid naisel.
\par 5 Ärge tõmbuge eemale teineteisest muidu kui ehk mõlemate tahtes mõneks ajaks, et oleksite vabad palveks, ja tulge jälle ühte, et saatan teid ei kiusaks teie taltumatuse pärast.
\par 6 Ent seda ma ütlen andes luba, mitte käskides.
\par 7 Sest ma tahaksin, et kõik inimesed oleksid nõnda nagu minagi; aga igaühel on oma armuanne Jumalalt, ühel nõnda, teisel teisiti.
\par 8 Ent ma ütlen vallalistele ja lesknaistele: neile on hea, kui nad jäävad nõnda nagu minagi.
\par 9 Aga kui nad ei suuda endid talitseda, siis nad abiellugu, sest parem on abielluda kui himudes põleda.
\par 10 Aga abielulistele ei käsi mina, vaid Issand, et naine ärgu mingu mehest lahku.
\par 11 On aga keegi lahutatud, siis ta jäägu abielutuks või leppigu ära oma mehega. Ja mees ärgu hüljaku naist.
\par 12 Aga muile ütlen mina, mitte Issand: kui ühel vennal on uskmatu naine ja naine heal meelel tahab elada temaga, siis ta ärgu hüljaku teda.
\par 13 Samuti kui ühel naisel on uskmatu mees ja mees heal meelel tahab elada temaga, siis ta ärgu hüljaku meest.
\par 14 Sest uskmatu mees on pühitsetud naise läbi ja uskmatu naine on pühitsetud mehe läbi, sest muidu oleksid teie lapsed rüvedad, aga nüüd on nad pühad.
\par 15 Aga kui uskmatu pool lahkub, siis ta lahkugu; niisuguseis asjus ei ole vend ega õde orjastatud; ent Jumal on meid rahus kutsunud.
\par 16 Sest mis tead sina, naine, kas sa oma mehe päästad? Või mis tead sina, mees, kas sa oma naise päästad?
\par 17 Ainult nõnda nagu Jumal on jaganud igaühele, nõnda nagu Issand on igaühe kutsunud, nõnda ta käitugu ja nõnda korraldan mina kõigis kogudustes.
\par 18 On keegi kutsutud ümberlõigatuna, siis ärgu ta seda varjaku; on keegi kutsutud eesnahas, siis ärgu ta lasku ennast ümber lõigata.
\par 19 Ümberlõikamine ei ole midagi ja eesnahk ei ole midagi, vaid tähtis on Jumala käskude pidamine.
\par 20 Igaüks jäägu sellesse kutsesse, milles ta on kutsutud.
\par 21 Kui sa oled kutsutud orjana, siis ära hooli sellest. Aga kui sa võid ka vabaks saada, siis kasuta pigemini seda.
\par 22 Sest kes on kutsutud orjana Issandas, on Issandas vabakslastu; samuti, kes on kutsutud vabana, on Kristuse ori.
\par 23 Teie olete kallilt ostetud, ärge saage inimeste orjaks!
\par 24 Jäägu, vennad, igaüks Jumala ees sellesse, milles ta on kutsutud!
\par 25 Ent neitsite kohta ei ole mul Issanda käsku, aga ma annan nõu otsekui see, kes Issandalt on saanud armu olla usaldusväärne.
\par 26 Siis ma arvan eelseisva kitsikuse pärast hea olevat, et inimene jääb, nagu ta on.
\par 27 Kui sa oled naisega seotud, siis ära püüa vabaneda; kui oled vallaline, siis ära otsi naist.
\par 28 Aga kui sa ka abiellud, siis sa ei tee pattu; ja kui neitsi läheb mehele, siis ta ei tee pattu. Ent sellised saavad kannatada ihulikku viletsust, mina aga säästaksin teid.
\par 29 Ent seda ütlen, vennad, et aeg on lühike; sellepärast olgu nii, et needki, kellel on naised, olgu nagu need, kellel neid ei ole;
\par 30 ja olgu nii, et kes nutavad, nagu ei nutakski, ja kes on rõõmsad, nagu ei olekski rõõmsad, ja kes ostavad, nagu ei saakski seda enesele pidada,
\par 31 ja kes seda maailma tarvitavad, nagu ei tarvitakski seda; sest selle maailma nägu kaob.
\par 32 Ent mina tahaksin, et te oleksite muretud. Vallaline mees hoolitseb selle eest, mis kuulub Issandale, kuidas ta saaks olla Issanda meele järgi.
\par 33 Aga naisemees hoolitseb selle eest, mis kuulub maailmale, kuidas ta saaks olla naise meele järgi.
\par 34 Nii on ta jaotatud kaheks. Ja vallaline naine või neitsi hoolitseb selle eest, mis kuulub Issandale, et ta võiks olla püha nii ihu kui ka vaimu poolest; aga abielunaine hoolitseb selle eest, mis kuulub maailmale, kuidas tema saaks olla mehe meele järgi.
\par 35 Aga seda ma ütlen teie eneste kasuks, ei mitte, et köit panna teile kaela, vaid et te elaksite viisakalt ja püsiksite takistamatult Issandas.
\par 36 Aga kui keegi arvab vääriti käitlevat oma neitsit, kelle paras naimaiga on möödumas, ja kui see peab nõnda sündima, siis tehku ta, mida tahab; ta ei tee pattu. Nad abiellugu.
\par 37 Aga kes on kindel oma südames ja kellel seda ei ole vaja, vaid on meelevald oma tahtmise üle ning on otsustanud oma südames oma neitsi jätta neitsiks, see teeb hästi.
\par 38 Kes siis oma neitsi naib, teeb hästi, ja kes teda ei nai, teeb paremini.
\par 39 Naine on seotud niikaua kui tema mees elab; aga kui ta mees on läinud magama, on ta vaba minema mehele, kellele tahab, kui see vaid sünnib Issandas.
\par 40 Aga ta on õndsam, kui ta jääb, nagu ta on. See on minu arvamine. Ent ma mõtlen, et minulgi on Jumala Vaim.


\chapter{8}

\section*{Ebajumala ohvrilihast}

\par 1 Mis puutub ebajumala ohvrilihasse, siis teame: meil kõigil on tunnetus. Tunnetus teeb suureliseks, ent armastus ehitab.
\par 2 Kui keegi arvab midagi tunnetavat, ta ei tunneta veel nõnda, kuidas tuleb tunnetada,
\par 3 aga kes armastab Jumalat, selle on Jumal ära tundnud.
\par 4 Mis nüüd puutub ebajumala ohvriliha söömisse, siis teame, et maailmas ei ole ühtki ebajumalat ja et ei ole muud Jumalat kui üks.
\par 5 Sest kuigi on olemas niinimetatud jumalaid, olgu taevas või maa peal, nagu ju on olemas palju jumalaid ja palju isandaid,
\par 6 on meil ometigi ainult üks Jumal, Isa, kellest on kõik asjad ja kellesse me suundume, ja üks Issand Jeesus Kristus, kelle läbi on kõik ja ka meie tema läbi.

\section*{Mure nõdra venna südametunnistuse pärast}

\par 7 Aga kõikidel ei ole seda teadmist, vaid harjumusest ebajumalatega söövad mõned tänini ebajumala ohvriliha sellisena ja nende südametunnistus, mis on nõder, rüvetub.
\par 8 Roog ei vii meid lähemale Jumalale; kui me ei söö, ei ole meile sellest kahju, ja kui sööme, ei ole meil sellest kasu.
\par 9 Vaadake ainult, et see teie vabadus ei saaks nõtradele komistuseks.
\par 10 Sest kui keegi, kellel on tunnetus, näeb sind ebajumala kojas lauas istuvat, kas siis tema südametunnistus, kui see on nõrk, ei ärrita teda sööma ebajumala ohvriliha?
\par 11 Siis läheb ju sinu tunnetuse läbi hukka nõder vend, kelle pärast Kristus on surnud.
\par 12 Aga kui te nõnda teete pattu vendade vastu ja haavate nende nõrka südametunnistust, siis te teete pattu Kristuse vastu.
\par 13 Sellepärast: kui roog mu venda pahandab, siis ma ei söö iialgi enam liha, et ma oma venda ei pahandaks!


\chapter{9}

\section*{Apostel kaitseb oma õigust}

\par 1 Eks ma ole vaba? Eks ma ole apostel? Eks ma ole näinud Jeesust Kristust, meie Issandat? Eks teie ole minu töö Issandas?
\par 2 Kui ma apostel ei ole muile, siis ma olen seda ometi teile; sest teie olete minu apostliameti pitser Issandas.
\par 3 Minu kaitsekõne neile, mu arvustajaile, on see:
\par 4 kas meil ei ole õigust süüa ja juua?
\par 5 Kas meil ei ole õigust naiseks kaasa võtta usuõde, nagu muil apostlitel ning Issanda vendadel ja Keefasel?
\par 6 Või kas üksi minul ja Barnabasel pole õigust olla ilma kehaliku tööta?
\par 7 Kes iganes läheb omal kulul sõtta? Kes istutab viinamäe, aga ei söö selle vilja? Või kes hoiab karja, aga ei söö karja piima?
\par 8 Kas ma räägin seda inimeste viisil? Eks käskki ütle seda?
\par 9 On ju Moosese käsuõpetuses kirjutatud: „Ära seo kinni härja suud, kui ta pahmast tallab!” Ega Jumal härgade eest hoolitse?
\par 10 Eks ta seda kõige rohkem ütle meie pärast? Sest meie pärast on kirjutatud, et kündja peab kündma lootuses ja rehepeksja peksma lootuses osa saada.
\par 11 Kui me teile oleme külvanud vaimulikku seemet, kas on see siis suur asi, kui me lõikame teie ihulikku vara?
\par 12 Kui teistel on niisugune õigus teie kohta, mispärast mitte palju rohkem meil? Me ainult ei ole tarvitanud seda õigust, vaid me talume kõike, et me mingit takistust ei teeks Kristuse evangeeliumile.
\par 13 Eks te tea, et need, kes pühakojas tööd teevad, pühakojast ka söövad, et need, kes teenivad ohvrialtari juures, oma osa saavad altariga samal määral?
\par 14 Nõnda on ka Issand seadnud neile, kes evangeeliumi kuulutavad, et nad elaksid evangeeliumist.
\par 15 Aga mina ei ole sellest midagi tarvitanud. Ent ma ei ole seda kirjutanud, et mulle nõnda peaks sündima; sest mulle oleks palju parem surra, kui - ei, keegi ei saa mu kiitlemist tühjaks teha!
\par 16 Sest kui ma armuõpetust kuulutan, ei ole mul sellest kiitlemist, sest mu kohus on seda teha. Ja häda mulle, kui ma armuõpetust ei kuuluta!
\par 17 Sest kui ma teen seda vabast tahtest, siis ma saan palga; aga kui sunniviisil, siis on ometi amet usaldatud minu kätte.
\par 18 Mis on siis mu palk? See, et ma armuõpetust kuulutades tasuta levitan evangeeliumi, ilma et ma tarvitaksin oma õigust, mis evangeeliumis on lubatud.

\section*{Apostli agarus inimeste võitmiseks Kristusele}

\par 19 Sest ehk ma küll olen vaba kõikidest, olen ma ometi hakanud kõikide orjaks, et ma seda rohkem inimesi võiksin võita.
\par 20 Nii olen ma juutidele olnud otsekui juut, et ma võidaksin juudid; neile, kes on käsu all, olen ma otsekui käsualune, kuigi ma käsu all ei ole, et ma võidaksin käsualused.
\par 21 Käsuta olijaile ma olen otsekui käsuta - ehk ma küll ei ole ilma Jumala käsuta, vaid elan Kristusele käsus - et võita need, kes on käsuta.
\par 22 Nõtradele ma olen saanud nõdraks, et võita nõdrad. Ma olen kõikidele saanud kõigeks, et ma igapidi mõned päästaksin.
\par 23 Aga kõike ma teen evangeeliumi pärast, et mina sellest ka osa saaksin.

\section*{Võit nõuab pingutust}

\par 24 Eks te tea, et need, kes võidu jooksevad, jooksevad küll kõik, aga võiduanni saab üks? Jookske nõnda, et teie selle saate!
\par 25 Aga iga võidujooksja on kasin kõigis asjus, nemad küll selleks, et saavutada kaduvat pärga, aga meie, et saavutada kadumatut pärga.
\par 26 Sellepärast ma ei jookse nagu pimedast peast, ma ei võitle nõnda nagu see, kes tuult peksab,
\par 27 vaid ma talitsen oma ihu ja teen selle oma orjaks, et ma muile jutlust öeldes ise ei saaks kõlvatuks.


\chapter{10}

\section*{Hoiatavad lood Iisraeli laste minevikust}

\par 1 Sest ma ei taha, vennad, et teil oleks teadmata, et meie esiisad olid kõik pilve all ja läksid kõik merest läbi,
\par 2 ja ristiti kõik Moosesesse pilves ja meres,
\par 3 ja sõid kõik sama vaimulikku rooga
\par 4 ja jõid kõik sama vaimulikku jooki, sest nad jõid vaimulikust kaljust, mis neid järgis. Ent see kalju oli Kristus.
\par 5 Aga suurem hulk neist ei olnud Jumalale meelepärased, sest nad löödi maha kõrbes.
\par 6 Ent need lood sündisid meile eeltähenduseks, et me ei himustaks kurja, nagu nemad himustasid.
\par 7 Ärge ka saage ebajumala kummardajaiks, nõnda nagu mõned neist said, nagu on kirjutatud: „Rahvas istus maha sööma ja jooma ning tõusis üles mängima.”
\par 8 Ärgem pidagem ka hooraelu, nõnda nagu mõned neist pidasid ja langesid ühel päeval kakskümmend kolm tuhat.
\par 9 Ärgem ka kiusakem Issandat, nõnda nagu mõned neist kiusasid ja said surma madude läbi.
\par 10 Ärge ka nurisege, nõnda nagu mõned neist nurisesid ja said otsa hukkaja läbi.
\par 11 See sündis neile eeltähendavalt ja on kirjutatud meile manitsuseks, kellele maailma lõpu ajad on vastu jõudnud.
\par 12 Sellepärast, kes enese arvab seisvat, katsugu, et ta ei langeks!
\par 13 Teid ei ole veel tabanud muud kui inimlik kiusatus; aga ustav on Jumal, kes ei lase teid rohkem kiusata kui te suudate kanda, vaid ühes kiusatusega valmistab ka väljapääsu, nõnda et te suudaksite kanda.

\section*{Ebajumalateenistust ei tohi olla}

\par 14 Sellepärast, minu armsad, põgenege ebajumala teenistuse eest!
\par 15 Ma räägin nõnda kui arusaajaile; otsustage ise, mida ma räägin.
\par 16 Õnnistuse karikas, mida me õnnistame, eks see ole Kristuse vere osadus? Leib, mida me murrame, eks see ole Kristuse ihu osadus?
\par 17 Sellepärast et on ainult üks leib, siis me, kuigi paljud, oleme üks ihu; sest me kõik saame osa ühest leivast.
\par 18 Vaadake Iisraeli peale, nagu ta liha poolest on! Eks need, kes söövad ohvreid, ole altari osalised?
\par 19 Mis ma nüüd ütlen? Kas seda, et ebajumala ohver on midagi? Või et ebajumal on midagi?
\par 20 Ei! Vaid ma ütlen: mida paganad ohverdavad, seda nad ohverdavad kurjadele vaimudele, aga mitte Jumalale. Ent ma ei taha, et teil oleks osadust kurjade vaimudega.
\par 21 Te ei või juua Issanda karikat ja kurjade vaimude karikat; te ei või osa võtta Issanda lauast ja kurjade vaimude lauast!
\par 22 Või tahame Issandat ärritada? Kas me oleme vägevamad temast?
\par 23 Kõik on lubatav, kuid kõigest ei ole kasu; kõik on lubatav, kuid kõik ei ehita kogudust.
\par 24 Ärgu ükski otsigu oma kasu, vaid teise kasu!
\par 25 Kõike, mida lihaturul müüakse, sööge, südametunnistusest hoolimata.
\par 26 Sest Issanda päralt on maa ja selle täius!
\par 27 Kui keegi uskmatuist kutsub teid ja te tahate minna, siis sööge kõike, mis teile ette pannakse, südametunnistusest hoolimata.
\par 28 Aga kui keegi teile ütleb: „See on ohvriliha!”, siis ärge sööge selle inimese pärast, kes tegi märkuse, ja südametunnistuse pärast.
\par 29 Ent mina ei räägi sinu, vaid teise südametunnistusest. Sest milleks peaks teise südametunnistus otsustama minu vabaduse üle?
\par 30 Kui ma midagi vastu võtan tänuga, miks mind pilgatakse selle pärast, mille eest ma tänan?
\par 31 Kas te siis sööte või joote või midagi teete, seda tehke kõik Jumala auks!
\par 32 Ärge olge pahanduseks ei juutidele ega kreeklastele ega Jumala kogudusele,
\par 33 nagu minagi olen kõiges kõikidele meelepärane ega otsi oma, vaid paljude kasu, et nad päästetaks!


\chapter{11}

\section*{Ebajumalateenistust ei tohi olla}

\par 1 Võtke mind eeskujuks, nagu minagi võtan Kristuse.

\section*{Jumalateenistusel olgu naise pea kaetud}

\par 2 Aga ma kiidan teid, vennad, et te kõiges tuletate mind meelde ja peate mu seadmisi, nõnda nagu ma need teile olen andnud.
\par 3 Aga ma tahan, et te teaksite, et Kristus on iga mehe pea ja et mees on naise pea ja Jumal on Kristuse pea.
\par 4 Iga mees, kes palvetab või prohvetlikult räägib kaetud peaga, häbistab oma pead.
\par 5 Aga iga naine, kes palvetab või prohvetlikult räägib katmata peaga, häbistab oma pead, sest see on otse sama, kui oleks tema pea paljaks aetud.
\par 6 Sest kui naine ei kata oma pead, siis lõigaku ta ka juuksed maha. Aga et naisel on häbiks juukseid lõigata või pead paljaks ajada, siis ta katku oma pea.
\par 7 Mehel ei ole tarvis oma pead katta, sest tema on Jumala kuju ja aupaiste, naine aga on mehe aupaiste.
\par 8 Sest mees ei ole naisest, vaid naine on mehest.
\par 9 Meest ei loodud ka mitte naise pärast, vaid naine loodi mehe pärast.
\par 10 Sellepärast peab naisel olema pea peal meelevallatunnus inglite pärast.
\par 11 Ometi ei ole naine meheta ega mees naiseta Issandas.
\par 12 Sest otsekui naine mehest, nõnda on ka mees naise läbi, aga kõik on Jumalast.
\par 13 Otsustage iseenestes, kas naisele kõlbab katmata peaga paluda Jumalat.
\par 14 Eks ju loodus ise teid õpeta, et mehele on häbiks kanda pikki juukseid;
\par 15 ent naisele, kui ta kannab pikki juukseid, on see auks; sest pikad juuksed on temale antud liniku eest.
\par 16 Aga kui keegi on tülinorija, see teadku, et meil ei ole seesugust kommet ega ka mitte Jumala kogudustel.

\section*{Issanda õhtusöömaaja väärast pühitsemisest}

\par 17 Aga seda käskides ma ei kiida seda, et te tulete kokku mitte paremuseks vaid pahemuseks.
\par 18 Sest esmalt ma kuulen, et kui te kokku tulete Jumala koguduses, teil on lõhesid eneste keskel, ja osalt ma usun seda.
\par 19 Sest teie seas peab olema lahkõpetusi, et saaks ilmsiks, kes teie seast on püsivad katsumistes.
\par 20 Kui te nüüd kokku tulete, siis ei ole Issanda õhtusöömaaja pidamist;
\par 21 sest sööma asudes võtab igaüks enne iseenese toidu ära, ja mõni jääb ilma ja mõni joob liiaks.
\par 22 Kas teil ei ole kodasid, kus te võite süüa? Või kas põlgate Jumala kogudust ja teete häbi neile, kellel ei ole midagi? Mis ma pean teile ütlema? Kas pean teid kiitma? Selle poolest ma ei kiida teid!

\section*{Püha õhtusöömaaja õige kord}

\par 23 Sest mina olen Issandalt saanud selle, mis ma teilegi olen andnud, et Issand Jeesus sel ööl, mil ta ära anti, võttis leiva
\par 24 ja tänas ja murdis ning ütles: „See on minu ihu, mis teie eest antakse. Seda tehke minu mälestuseks!”
\par 25 Samuti ta võttis ka karika pärast söömaaega ning ütles: „See karikas on uus leping minu veres. Seda tehke nii sageli kui te iganes seda joote, minu mälestuseks!”
\par 26 Sest iga kord, kui te seda leiba sööte ja karikast joote, te kuulutate Issanda surma, kuni ta tuleb!

\section*{Õhtusöömaajast kõlvatult osasaamise tagajärg}

\par 27 Sellepärast: kes iganes seda leiba sööb või Issanda karikat joob kõlvatult, sellel on Issanda ihust ja verest süüd.
\par 28 Aga inimene katsugu ennast läbi ja nõnda söögu ta seda leiba ja joogu sellest karikast;
\par 29 sest kes sööb ja joob, see sööb ja joob enesele nuhtlust, kui ta enesele ei anna aru sellest ihust.
\par 30 Sellepärast on ka palju nõrku ja põdejaid teie seas ja paljud on läinud magama.
\par 31 Sest kui me enestest ise aru annaksime, siis meie üle ei mõistetaks kohut;
\par 32 aga kui meie üle kohut mõistetakse, siis Issand karistab meid, et meid ühes maailmaga hukka ei mõistetaks.
\par 33 Seepärast, mu vennad, kui te kokku tulete sööma, siis oodake üksteist.
\par 34 Kui keegi on näljane, söögu kodus, et te ei tuleks kokku nuhtluseks. Muud asjad ma korraldan, kui ma tulen.


\chapter{12}

\section*{Vaimuannetest}

\par 1 Mis puutub vaimuannetesse, vennad, ei taha ma teid jätta teadmatusse.
\par 2 Te teate, et alles paganad olles te käisite keeletute ebajumalate juures, nõnda nagu teid veeti.
\par 3 Sellepärast ma teen teile teatavaks, et ükski, kes räägib Jumala Vaimus, ei ütle: „Neetud on Jeesus!” Ja ükski ei või öelda: „Jeesus on Issand!” kui vaid Pühas Vaimus.
\par 4 Armuanded on küll mitmesugused, aga Vaim on sama;
\par 5 ja ametid on mitmesugused, aga Issand on sama.
\par 6 Ja väeavaldused on mitmesugused, aga sama on Jumal, kes teeb kõike kõikide sees.
\par 7 Aga igaühele antakse Vaimu ilmutus ühiseks kasuks.
\par 8 Nii antakse ühele Vaimu läbi tarkuse sõna, teisele aga tunnetuse sõna sama Vaimu kaudu;
\par 9 teisele antakse usku samas Vaimus, teisele andeid terveks teha samas Vaimus;
\par 10 teisele vägevaid tegusid, teisele prohvetlikku kuulutamist, teisele vaimude äratundmist, teisele mõnesuguste keelte kõnelemist, teisele keelte tõlgitsemist.
\par 11 Ent kõike seda teeb üks ja sama Vaim, jagades igaühele eriti, nõnda nagu tema tahab.

\section*{Näide kogudusest kui ihust ja ihuliikmeist}

\par 12 Sest otsekui ihu on üks ja tal on palju liikmeid, aga kõik ihu liikmed, kuigi neid on palju, on üks ihu, nõnda on ka Kristus:
\par 13 sest me kõik oleme ühe Vaimuga ristitud üheks ihuks, olgu juudid või kreeklased, olgu orjad või vabad, ja oleme kõik sama Vaimuga joodetud.
\par 14 Sest ihu ei ole üks liige, vaid palju liikmeid.
\par 15 Kui jalg ütleks: „Et ma ei ole käsi, siis ma ei kuulu ihusse!” kas ta sellepärast ei kuulu ihusse?
\par 16 Ja kui kõrv ütleks: „Et ma ei ole silm, siis ma ei kuulu ihusse!” kas ta sellepärast ei kuulu ihusse?
\par 17 Kui kõik ihu oleks silm, kuhu jääks siis kuulmine? Kui kõik oleks kuulmine, kuhu jääks siis haistmine?
\par 18 Aga nüüd Jumal asetaski liikmed, igaühe neist, ihu külge, nõnda nagu tema tahtis.
\par 19 Ja kui kõik need oleksid üks liige, kuhu jääks siis ihu?
\par 20 Nüüd aga ongi palju liikmeid, aga üks ihu!
\par 21 Silm ei või öelda käele: „Mulle ei ole sind tarvis” Või jälle pea jalgadele: „Mulle ei ole teid tarvis!”
\par 22 Vaid palju enam on tarvis neid ihu liikmeid, mis näivad olevat nõrgemad,
\par 23 ja neid, mida me peame autumaiks ihu küljes, katame iseäralise auga, ja liikmeid, millest meil on häbi, me ehime kõige enam;
\par 24 ent meie suursuguseil liikmeil pole seda tarvis. Sest Jumal on ihu nõnda ühte liitnud, et ta sellele, mis on alam, on andnud rohkem au,
\par 25 et ihus ei oleks lahkmeelt, vaid et liikmed üksmeeles kannaksid muret üksteise eest.
\par 26 Ja kui üks liige kannatab, siis kannatavad teisedki liikmed ühes temaga, ja kui üht liiget austatakse, siis teisedki liikmed rõõmutsevad ühes temaga.
\par 27 Aga teie olete Kristuse ihu ja igaüks omast kohast tema liikmed.
\par 28 Ja Jumal on seadnud koguduses mõned - esiteks apostleiks, teiseks prohveteiks, kolmandaks õpetajaiks; siis ta on andnud imetegusid, siis andeid terveks teha, abistada ja valitseda, rääkida mõnesuguseid keeli.
\par 29 Ega kõik ole apostlid? Ega kõik prohvetid? Ega kõik õpetajad? Ega kõik imetegijad?
\par 30 Ega kõigil ole tervekstegemise armuandeid? Ega kõik räägi keeltega? Ega kõik tõlgitse?
\par 31 Aga olge agarad püüdma suuremaid armuandeid! Ja ma näitan veel teile tee, mis on üle kõige.


\chapter{13}

\section*{Armastuse kiitus}

\par 1 Kui ma inimeste ja inglite keeltega räägiksin, aga mul poleks armastust, oleksin ma vaid kumisev vask ja kõlisev kelluke!
\par 2 Ja kui mul oleks prohvetianne ja ma teaksin kõik saladused ja kõik tunnetatu, ja kui mul oleks kõik usk, nõnda et võiksin mägesid teisale paigutada, aga mul poleks armastust, siis ei oleks minust ühtigi!
\par 3 Ja kui ma jagaksin kõik oma vara vaestele ja kui ma annaksin oma ihu põletada ja mul poleks armastust, siis ei oleks mul sellest mingit kasu!
\par 4 Armastus on pikameelne, armastus on täis heldust; ta ei ole kade, armastus ei suurustle, ta ei ole iseennast täis;
\par 5 ta ei ole viisakuseta, ta ei otsi omakasu, ta ei ärritu, ta ei pea meeles paha;
\par 6 ta ei rõõmutse ülekohtust, aga ta rõõmutseb ühes tõega;
\par 7 tema vabandab kõik, usub kõik, loodab kõik, sallib kõik!
\par 8 Armastus ei hävi ilmaski! Aga olgu prohveti ennustused, need kaovad; olgu keeled, need lakkavad; olgu tunnetus, see lõpeb ära.
\par 9 Sest poolik on, mida me tunnetame, ja poolik, mida me ennustame.
\par 10 Aga kui tuleb täiuslik asi, siis kaob see, mis on poolik!
\par 11 Kui ma olin väeti laps, siis ma rääkisin nagu väeti laps, ma mõtlesin nagu väeti laps ja arvasin nagu väeti laps; aga kui ma sain meheks, siis ma hülgasin selle, mis on omane väetile lapsele.
\par 12 Sest nüüd me näeme nagu peeglis tuhmi kujutist, aga siis palgest palgesse; nüüd ma tunnetan poolikult, aga siis ma tunnetan täiesti, nagu minagi olen täiesti tunnetatud.
\par 13 Ent nüüd jääb usk, lootus, armastus, need kolm; aga suurim neist on armastus!


\chapter{14}

\section*{Prohvetlikust kuulutamisest ja keeltega rääkimisest}

\par 1 Taotlege armastust ja olge agarad taotlema vaimuandeid, aga kõige enam prohvetlikku kõnelemist!
\par 2 Sest kes keeltega räägib, ei kõnele inimestele, vaid Jumalale; sest keegi ei saa temast aru, vaimust ta ju räägib saladusi.
\par 3 Aga prohvetlikult kõneleja räägib inimestele nende ehitamiseks ja manitsuseks ja troostiks.
\par 4 Keeltega rääkija ehitab iseennast; aga prohvetlikult kõneleja ehitab kogudust.
\par 5 Ma tahaksin küll, et te kõik räägiksite keeltega, aga veel enam, et te prohvetlikult kõneleksite; sest prohvetlikult kõneleja on suurem keeltega rääkijast, kui see mitte ühtlasi ei tõlgitse, nõnda et kõne saab koguduse ehitamiseks.
\par 6 Aga nüüd, vennad, kui ma tuleksin teie juurde ja räägiksin keeltega, mis kasu ma tooksin teile, kui ma teile ei räägiks kas ilmutuse või tunnetuse või ennustuse või õpetuse kaudu?
\par 7 Nõnda on ju lugu ka elutute asjadega, mis annavad häält, olgu vile või kannel; kui need ei annaks erinevaid hääli, kuidas teatakse, mida puhutakse vilel või mida mängitakse kandlel?
\par 8 Samuti kui pasun annab segast häält, kes siis hakkab valmistama sõjaks?
\par 9 Nõnda ka teie: kui te keeltega ei räägi arusaadavalt, kuidas siis teatakse, mida räägitakse? Te räägiksite ju tuulde!
\par 10 Maailmas on ei tea kui palju mitmesuguseid keelemurdeid ja ükski neist pole häälikuteta.
\par 11 Kui mina nüüd ei tunne hääliku tähendust, olen ma sellele, kes räägib, umbkeelne ja rääkija on mulle umbkeelne.
\par 12 Samuti teiegi olles agarad taotlema vaimuandeid, püüdke saada neid rohkesti koguduse ehitamiseks.
\par 13 Seepärast, kes keeltega räägib, palugu Jumalat, et ta võiks ka tõlgitseda.
\par 14 Sest kui ma keeltega rääkides palvetan, siis palvetab mu vaim, aga mu mõistus on viljatu.
\par 15 Mida siis teha? Ma tahan palvetada vaimus ja tahan palvetada ka mõistusega; ma tahan laulda vaimus ja tahan laulda ka mõistusega.
\par 16 Sest kui sa teed tänupalvet vaimus, kuidas siis see, kes võhiku kohal seisab, sinu tänu peale saab öelda: „Aamen!”; ta ju ei tea, mida sa ütled.
\par 17 Sest sa tänad küll hästi, aga teist see ei ehita.
\par 18 Ma tänan Jumalat, et mina räägin keeltega rohkem kui teie kõik!
\par 19 Aga ma tahan koguduses pigemini rääkida viis sõna oma mõistusega, et ka muid õpetada, kui kümme tuhat sõna keeltega.
\par 20 Vennad, ärge saage lasteks mõtteviisis, vaid olge lapsed kurjuses; kuid mõtteviisi poolest saage täisealiseks.
\par 21 Käsuõpetuses on kirjutatud: „Teiste keeltega ja teiste huultega ma räägin sellele rahvale ja siiski nad ei kuule mind!”, ütleb Issand.
\par 22 Nii ei ole siis keeled tunnuseks usklikele, vaid uskmatuile; aga prohvetlik kõne ei ole tunnuseks uskmatuile, vaid usklikele.
\par 23 Kui nüüd kõik kogudus tuleks kokku ühte paika ja kõik räägiksid keeltega, ja tuleksid sisse võhikud või uskmatud, kas nad ei ütleks, et te jampsite?
\par 24 Aga kui nemad kõik kõneleksid prohvetlikult ja tuleks uskmatu või võhik, siis nad kõik paljastaksid tema olukorra ja kõik arvustaksid teda;
\par 25 tema südame saladused tuleksid ilmsiks ja ta heidaks siis silmili maha ja kummardaks Jumalat ning tunnistaks, et Jumal on tõesti teie sees.

\section*{Jumalateenistuses valitsegu kord}

\par 26 Mis siis nüüd, vennad? Kui te kokku tulete, siis on igaühel midagi, kas laulu või õpetust või ilmutust või keeltega rääkimist või tõlgitsemist; sündigu see kõik koguduse ehitamiseks.
\par 27 Kui keeltega räägitakse, siis rääkigu kaks või kõige rohkem kolm ükshaaval, ja üks tõlgitsegu.
\par 28 Aga kui tõlgitsejat ei ole, olgu rääkija vait koguduses ja rääkigu iseenesele ja Jumalale.
\par 29 Aga prohveteist kõnelgu kaks või kolm ja teised arvustagu.
\par 30 Ja kui kellelegi teisele, kes istub, ilmutatakse midagi, siis olgu esimene vait.
\par 31 Sest te võite kõik ükshaaval prohvetlikult kõnelda, et kõik õpiksid ja saaksid troosti.
\par 32 Ja prohvetite vaimud alistuvad prohveteile.
\par 33 Sest Jumal ei ole mitte korratuse, vaid rahu Jumal. Nõnda nagu kõigis pühade kogudustes,
\par 34 olgu naised koguduses vait, sest neil ei ole luba kõnelda, vaid nad olgu allaheitlikud, nõnda nagu käskki ütleb.
\par 35 Aga kui nad tahavad midagi õppida, siis nad küsigu kodus oma meestelt. Sest naisele on häbiks kõnelda koguduses.
\par 36 Või kas Jumala sõna on teist lähtunud? Või kas see on tulnud üksnes teie juurde?
\par 37 Kui keegi arvab enese olevat prohveti või Vaimu mõjualuse, see teadku, et mida ma teile kirjutan, see on Issanda käsk!
\par 38 Aga kui keegi ei taha seda teada, ärgu teadku!
\par 39 Niisiis, vennad, taotlege prohvetlikku kõnelemist ja ärge ka keelake keeltega rääkimist.
\par 40 Kõik sündigu viisakalt ja korra järgi.


\chapter{15}

\section*{Kristuse surnuist ülestõusmisest}

\par 1 Ent ma juhin teie tähelepanu evangeeliumile, mida ma teile kuulutasin, mille te ka vastu võtsite ja milles ka püsite
\par 2 ning milles te ka õndsaks saate - kui te peate seda kinni nende sõnadega, millega mina teile seda kuulutasin - olgu siis, et asjata saite usklikuks.
\par 3 Sest ma olen teile kõigepealt teada andnud, mida ma olen saanud, et Kristus suri meie pattude eest kirjade järgi
\par 4 ja et ta maeti ja et ta üles äratati kolmandal päeval kirjade järgi
\par 5 ja et ta ilmus Keefasele, pärast seda neile kaheteistkümnele.
\par 6 Pärast ta ilmus ühekorraga rohkem kui viiesajale vennale, kellest suurem hulk veel praegugi on elus, aga mõned on läinud magama;
\par 7 pärast seda ta ilmus Jakoobusele, siis kõigile apostlitele;
\par 8 aga pärast kõike ta ilmus ka minule kui äbarikule.
\par 9 Sest mina olen apostlite seast kõige väiksem, kes ei ole väärt, et mind apostliks hüütakse, sest ma kiusasin taga Jumala kogudust.
\par 10 Aga Jumala armust olen mina, mis ma olen; ja see arm minu vastu ei ole olnud tühine, vaid ma olen palju rohkem tööd teinud kui nemad kõik; aga mitte mina, vaid Jumala arm, mis on minuga.
\par 11 Olgu nüüd mina või olgu nemad, nõnda me kuulutame ja nõnda olete saanud usklikuks.

\section*{Surnute ülestõusmisest}

\par 12 Aga kui Kristusest kuulutatakse, et ta on surnuist üles äratatud, kuidas siis mõned teie seast ütlevad, et ei olevatki surnute ülestõusmist?
\par 13 Aga kui ei ole surnute ülestõusmist, siis ei ole ka Kristus üles äratatud.
\par 14 Ent kui Kristus mitte ei ole üles äratatud, siis on meie jutlus tühine ja tühine on ka teie usk;
\par 15 ja meid leitakse siis Jumala valetunnistajad olevat, sest me oleme tunnistanud Jumala vastu, et ta on üles äratanud Kristuse, keda tema pole äratanud, kui surnuid üles ei äratata.
\par 16 Sest kui surnuid ei äratata, siis ei ole ka Kristust äratatud.
\par 17 Aga kui Kristust ei ole äratatud, siis on teie usk tühine ja te olete alles oma pattude sees.
\par 18 Siis on ka need, kes Kristuse sees on läinud hingama, hukka saanud.
\par 19 Kui meie selles elus oleme lootnud ainult Kristuse peale, siis me oleme kõigist inimestest kõige armetumad.
\par 20 Aga nüüd on Kristus surnuist üles äratatud ja on saanud esmaseks nende seast, kes on läinud hingama.
\par 21 Sest kui juba surm on tulnud inimese kaudu, siis on ka surnute ülestõusmine inimese kaudu.
\par 22 Sest nõnda nagu kõik surevad Aadamas, nõnda tehakse ka kõik elavaiks Kristuses,
\par 23 aga igaüks oma järjekorras: esimesena Kristus, selle järel Kristuse omad tema tulemises;
\par 24 siis tuleb ots, kui ta annab riigi Jumala ja Isa kätte pärast seda, kui ta on hävitanud kõik valitsuse ja kõik meelevalla ja võimu.
\par 25 Sest tema peab valitsema, kuni ta on pannud kõik vaenlased tema jalge alla.
\par 26 Viimne vaenlane, kellele ots tehakse, on surm!
\par 27 Sest „ta on kõik alistanud tema jalge alla!” Kui ta aga ütleb, et kõik on alistatud tema alla, siis on avalik, et ei ole alistatud see, kes kõik temale alistas.
\par 28 Aga kui kõik temale on alistatud, siis peab ka Poeg ise alistatama sellele, kes kõik temale on alistanud, et Jumal oleks kõik kõigis.
\par 29 Mis teevad muidu need, kes endid lasevad ristida surnute eest, kui surnuid koguni ei äratata? Mispärast siis neid ristitakse nende eest?
\par 30 Ja mispärast oleme meiegi hädaohus igal hetkel?
\par 31 Iga päev ma olen surmasuus, nii tõesti kui teie, vennad, olete minu kiitus, mis mul on Kristuses Jeesuses, meie Issandas.
\par 32 Kui ma inimeste kombel olen Efesoses võidelnud metsalistega, siis mis kasu on mul sellest? Kui surnuid ei äratata, siis „söögem ja joogem, sest homme me sureme!”
\par 33 Ärge eksige! „Kurjad kõned rikkuda võivad kombeid häid!”
\par 34 Kainestuge õieti ja ärge tehke pattu; sest mõned ei tea midagi Jumalast; seda ma ütlen teile häbiks.
\par 35 Aga mõni küsib: „Kuidas surnud ärkavad üles? Ja millise ihuga nad tulevad?”
\par 36 Rumal! See, mida sa külvad, ei saa elavaks, kui see enne ei sure.
\par 37 Ja mida sa külvad, sa ei külva selle ihuna, mis peab tõusma, vaid paljast iva, olgu nisuiva või mis tahes iva.
\par 38 Aga Jumal annab temale ihu, millise tahab, ja igale seemnele tema oma ihu.
\par 39 Kõik liha ei ole sama liha, vaid ise on inimeste liha ja ise veiste liha ja ise lindude ja ise kalade liha;
\par 40 ja on taevalikke ihusid ja maiseid ihusid; aga teistsugune on taevalike auhiilgus ja teistsugune maiste auhiilgus.
\par 41 Isesugune on päikese hiilgus ja isesugune kuu hiilgus ja isesugune tähtede hiilgus; sest tähe ja tähe hiilgusel on oma vahe.
\par 42 Nõnda on ka surnute ülestõusmine. Kaduvuses külvatakse, kadumatuses äratatakse üles;
\par 43 autuses külvatakse, auhiilguses äratatakse üles; nõtruses külvatakse, väes äratatakse üles;
\par 44 maine ihu külvatakse, vaimne ihu äratatakse üles; sest kui on olemas maine ihu, siis on ka olemas vaimne ihu.
\par 45 Nõnda on ka kirjutatud: „Esimene inimene Aadam sai elavaks hingeks.” Viimne Aadam sai vaimuks, kes elustab.
\par 46 Aga vaimne ihu ei ole esimene, vaid maine; selle järel on vaimne.
\par 47 Esimene inimene oli maast, muldne, teine inimene on taevast.
\par 48 Milline on muldne, sellised on ka muldsed; ja milline on taevane, sellised on ka taevased.
\par 49 Ja otse nõnda nagu me oleme kandnud muldse inimese kuju, nõnda me kanname kord ka taevast kuju.
\par 50 Aga seda ma ütlen, vennad, et liha ja veri ei või pärida Jumala riiki ega kaduvus pärida kadumatust.
\par 51 Vaata, ma ütlen teile saladuse: me kõik ei lähe magama, aga me kõik muutume,
\par 52 äkitselt ühe silmapilguga, viimse pasuna hüüdes. Sest pasun hüüab ja surnud tõusevad üles kadumatutena ja me muutume.
\par 53 Sest see kaduv peab riietuma kadumatusega ja see surev riietuma surematusega.
\par 54 Aga kui see kaduv riietub kadumatusega ja see surev riietub surematusega, siis saab tõeks sõna, mis on kirjutatud: „Surm on neelatud võidusse!
\par 55 Surm, kus on sinu võit? Surm, kus on sinu astel?”
\par 56 Aga surma astel on patt ja patu vägi on käsk.
\par 57 Ent tänu Jumalale, kes meile võidu annab meie Issanda Jeesuse Kristuse läbi!
\par 58 Sellepärast, mu armsad vennad, olge kindlad, vankumatud ja ikka innukad Issanda töös, teades, et teie vaevanägemine Issandas ei ole asjatu!


\chapter{16}

\section*{Korjandus Jeruusalemma koguduse heaks}

\par 1 Mis puutub korjandusse pühade heaks, siis tehke teiegi nõnda, nagu ma olen korraldanud Galaatia kogudustes.
\par 2 Igal esimesel nädalapäeval pangu igaüks teie seast midagi tallele sedamööda, kuidas temal on jõudu, et mitte alles siis raha ei kogutaks, kui ma tulen.
\par 3 Aga kui ma saabun, läkitan ma kirjadega need, keda te arvate kõlbavaks viima teie anni Jeruusalemma.
\par 4 Ja kui asi on seda väärt, et minagi läheksin, siis nad tulgu ühes minuga.

\section*{Apostli isiklikud asjaolud ja korraldused ning jumalagajätu sõnad}

\par 5 Ent ma mõtlen tulla teie juurde, kui ma olen käinud läbi Makedoonia; sest Makedooniast ma lähen läbi.
\par 6 Teie juures ma ehk viibin mõne aja või vahest kogu talvegi, et te mind siis saadaksite, kuhu ma iganes lähen.
\par 7 Sest ma ei taha teid seekord näha mitte ainult mööda minnes, vaid ma loodan jääda tükiks ajaks teie juurde, kui Issand lubab.
\par 8 Aga Efesosse ma jään nelipühani,
\par 9 sest mulle on seal avanenud suur uks viljarikkaks tööks ja seal on palju vastaseid.
\par 10 Kui Timoteos tuleb, siis katsuge, et ta võiks olla ilma kartuseta teie juures, sest ta teeb Issanda tööd nõnda nagu minagi.
\par 11 Ärgu siis ükski teda halvaks pangu, vaid saatke ta rahus teele, et ta tuleks minu juurde, sest ma ootan teda ühes vendadega.
\par 12 Mis aga puutub vend Apollosesse, siis olen ma temale sageli peale käinud, et ta tuleks teie juurde ühes vendadega, aga tal ei olnud üldse tahtmist nüüd tulla; ta tuleb, kui tal leidub parajat aega.
\par 13 Valvake, seiske usus, olge mehed, saage tugevaks!
\par 14 Kõik teie asjad sündigu armastuses!
\par 15 Ma palun teid, vennad: teie teate Stefanase peret, et see on Ahhaias esimene pöördunu ja et nemad on andunud pühade abistamisele;
\par 16 olge teiegi sõnakuulelikud niisugustele ja igaühele, kes töötab ühes ning näeb vaeva.
\par 17 Ma rõõmustun Stefanase ja Fortunaatuse ja Ahhaikose tulekust, sest nemad täidavad minule teie aset;
\par 18 nemad on ju jahutanud minu ja teie vaimu. Pidage siis niisugustest lugu!
\par 19 Aasia kogudused tervitavad teid. Teid tervitavad Issandas Akvila ja Priska ühes kogudusega nende majas.
\par 20 Teid tervitavad kõik vennad. Tervitage üksteist püha suudlusega!
\par 21 Tervitus on minu, Pauluse käega.
\par 22 Kui keegi Issandat Jeesust Kristust ei pea armsaks, siis olgu ta ära neetud! Maaran ata.
\par 23 Issanda Jeesuse Kristuse arm olgu teiega!
\par 24 Minu armastus on teie kõikidega Kristuses Jeesuses!





\end{document}