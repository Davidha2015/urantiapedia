\begin{document}

\title{Pauluse Teine kiri korintlastele}

\chapter{1}

\section*{Tervitus ja tänu}

\par 1 Paulus, Kristuse Jeesuse apostel Jumala tahtest, ja vend Timoteos Korintoses olevale Jumala kogudusele ja kõigile pühadele kogu Ahhaias.
\par 2 Armu teile ja rahu Jumalalt, meie Isalt ja Issandalt Jeesuselt Kristuselt.
\par 3 Kiidetud olgu Jumal ja meie Issanda Jeesuse Kristuse Isa, halastuse Isa ja kõige troosti Jumal,
\par 4 kes meid trööstib kõigis meie viletsusis, et meie võime trööstida neid, kes on kõiksuguses viletsuses, troostiga, millega Jumal meid endid trööstib.
\par 5 Sest otsekui Kristuse kannatamisi tuleb rohkesti meie peale, nõndasamuti tuleb meile rohkesti ka troosti Kristuse kaudu.
\par 6 Olgu meil nüüd viletsust, siis sünnib see teie trööstimiseks ja päästmiseks; või kui saame troosti, siis sünnib seegi teie trööstimiseks ja see teostub samade kannatamiste kandmiseks, mida meiegi kannatame;
\par 7 ja meie lootus teie kohta on kindel, sest me teame, et otsekui teil on osa kannatustest, nõnda on teil osa ka troostist.
\par 8 Sest me ei taha, vennad, et teil oleks teadmata, missuguses viletsuses me olime Aasias, kuidas meid ülemäära vaevati üle meie jõu, nõnda et me isegi olime kahevahel oma elu pärast.
\par 9 Ja me olime enestegi arvates surma mõistetud, et me ei loodaks eneste peale, vaid Jumala peale, kes surnud üles äratab.
\par 10 Tema päästis meid nii suurest surmast ja päästab veelgi; tema peale me loodame, et tema ka edaspidi päästab,
\par 11 kui ühtlasi ka teie appi tulete palves meie eest, et paljude isikute käest meie heaks tulnud armuanni eest tõuseks tänu paljude suust meie eest.

\section*{Apostli aus tahe}

\par 12 Sest meie kiitlus on see: meie südametunnistuse tõendus, et meie Jumala pühaduses ja puhtuses oleme elanud maailmas, mitte lihalikus tarkuses, vaid Jumala armus, ja iseäranis teie juures.
\par 13 Sest me ei kirjuta teile muud midagi kui aga seda, mida te siin loete või mida te ka tunnetate, ja mida te, nagu ma loodan, veel lõpuni saate tunnetada,
\par 14 nõnda nagu te ka meid olete osalt tundnud, et me oleme teie kiitlus, nagu ka teie olete meie kiitlus meie Issanda Jeesuse päeval.

\section*{Korintose külastus lükkub edasi}

\par 15 Ja niisuguse usaldusega ma mõtlesin juba ennemini tulla teie juurde, et teie saaksite veel teisegi armuanni osalisteks,
\par 16 ja teie kaudu siis matkata Makedooniasse ja Makedooniast tulla tagasi jälle teie juurde ja teilt saada saatjaid Judeasse.
\par 17 Kui ma nüüd nõnda kavatsesin, kas ma siis toimisin kergemeelselt? Või mis ma kavatsen, kas ma kavatsen liha järgi, selleks, et mu kõnes oleks „jah”, „jah” ja „ei”, „ei”?
\par 18 Kuid Jumal on ustav, nii et meie kõne teie vastu ei ole „jah” ja „ei” ühtlasi.
\par 19 Sest Jumala Poeg Jeesus Kristus, keda meie, mina, Silvaanus ja Timoteos, oleme teile kuulutanud, ei olnud mitte „jah” ja „ei”, vaid temas sai tõeks „jah”.
\par 20 Sest Jumala tõotused puha on tema sees „jah” ja sellepärast ka tema läbi „aamen” Jumala auks meie läbi.
\par 21 Aga kes meid ühes teiega kinnitab Kristusesse ja kes meid on võidnud, see on Jumal,
\par 22 kes meid ka on kinnitanud pitseriga ja on andnud Vaimu pandi meie südamesse.
\par 23 Aga ma kutsun Jumala tunnistajaks oma hinge peale, et ma tahtes teid säästa, ei ole veel tulnud Korintosesse.
\par 24 Ei mitte, et me tahame olla valitsejad teie usu üle, vaid me oleme teie rõõmu kaastegelased; sest teie püsite usus.


\chapter{2}

\section*{Korintose külastus lükkub edasi}

\par 1 Aga ma olin oma mõttes otsustanud mitte tulla jälle teie juurde kurvastusega.
\par 2 Sest kui mina teid kurvastan, kes on siis, kes mind rõõmustab, kui vaid see, kelle ma olen teinud kurvaks.
\par 3 Ja ma olen kirjutanud teile seda otse selleks, et mulle teie juurde tulles ei saaks osaks kurvastust neilt, kes mind pidid rõõmustama; sest mul on teist kõigist see lootus, et minu rõõm on teie kõikide rõõm.
\par 4 Sest ma kirjutasin teile suures viletsuses ja südame ahastuses paljude silmapisaratega, ei mitte selleks, et te kurvastuksite, vaid et te tunneksite selle armastuse, mida mul on rohkesti teie vastu.

\section*{Patukahetsejale tuleb andeks anda}

\par 5 Aga kui keegi on kurvastanud, ei ole ta mind kurvastanud, vaid teataval määral, et mitte liialdada, teid kõiki.
\par 6 Niisugusel on küllalt sellest noomitusest, mille ta on saanud paljude poolt,
\par 7 nõnda et teie hoopis vastupidi peate temale andeks andma ja teda kinnitama, et teda veel suurem kurvastus ära ei neelaks.
\par 8 Sellepärast ma manitsen teid osutada temale armastust.
\par 9 Sest selleks ma olengi kirjutanud, et teada saada teie meelt, kas te olete igapidi sõnakuulelikud.
\par 10 Aga kellele te midagi andeks annate, sellele ka mina annan andeks. Sest ka mina, mis ma andeks andsin, kui mul oli midagi andeks anda, andsin andeks teie pärast Kristuse palge ees,
\par 11 et saatan meid kavalasti ei petaks, sest tema mõtted ei ole meile teadmatud.

\section*{Evangeeliumi katsumusest ja võidust}

\par 12 Aga kui ma tulin Troasse kuulutama Kristuse evangeeliumi ja mulle oli uks avatud Issandas,
\par 13 ei leidnud ma siiski mitte rahu oma vaimus, sest ma ei kohanud oma venda Tiitust. Ma jätsin nad siis jumalaga ja läksin Makedooniasse.
\par 14 Aga tänu Jumalale, kes meile ikka annab võimust Kristuses ja teeb avalikuks oma tunnetuse lõhna meie kaudu kõigis paigus.
\par 15 Sest meie oleme Jumalale Kristuse hea lõhn niihästi nende seas, kes päästetakse, kui ka nende seas, kes hukka lähevad,
\par 16 ühtedele küll surmalehk surmaks, kuid teistele elulõhn eluks. Ja kes kõlbab selleks?
\par 17 Sest me ei ole nagu mitmed, kes Jumala sõnaga hangeldavad, vaid puhtast meelest, jah, otsekui Jumalast, me kõneleme Kristuses Jumala ees!


\chapter{3}

\section*{Apostel kui Vaimu sulane}

\par 1 Kas hakkame siis jälle iseendid soovitama? Või kas vajame, nagu mõningad, soovituskirju teie kätte või teie käest?
\par 2 Teie olete meie kiri, kirjutatud meie südamesse, kõigile inimestele tuttav ja loetav,
\par 3 sest teist on ju ilmsi näha, et te olete Kristuse kiri, mis meie teenimistöö läbi on valminud, kirjutatud mitte tindiga, vaid elava Jumala Vaimuga, ka mitte kivilauakestele, vaid lihastele südamelauakestele.
\par 4 Aga niisugune lootus on meil Kristuse läbi Jumala peale,
\par 5 mitte nii, et me oleksime iseenestest võimelised midagi mõtlema, nagu iseenesest, vaid meie võime tuleb Jumalalt,
\par 6 kes meid ka on teinud võimeliseks olema uue lepingu sulased, mitte kirjatähe, vaid Vaimu sulased; sest kirjatäht suretab, aga Vaim teeb elavaks.

\section*{Evangeeliumil on suurem au kui käsuõpetusel}

\par 7 Aga kui surma amet, mis kirjatähtedega oli uuristatud kividesse, esines nii suures aus, et Iisraeli lapsed ei suutnud vaadata Moosese palgesse ta palge hiilguse pärast, mis ometi oli kaduv,
\par 8 kuidas ei peaks palju rohkem Vaimu amet olema au sees?
\par 9 Sest kui hukkamõistmise ametil oli au, siis on õiguse amet veel palju suurem au poolest.
\par 10 Sest too, mis oli enne au sees, ei olegi enam niisugusel määral au sees selle üliväga suure au pärast.
\par 11 Sest kui see, mis on kaduv, on tekkinud au läbi, kui palju rohkem on siis au sees see, mis on püsiv.
\par 12 Et meil nüüd on niisugune lootus, siis me räägime täie julgusega
\par 13 ega tee nõnda nagu Mooses, kes pani katte oma palgele, et Iisraeli lapsed ei näeks selle lõppu, mis on kaduv.
\par 14 Ent nende meeled on paadunud; sest veel tänapäevalgi jääb sama kate Vana Testamendi lugemisel ära võtmata, sest alles Kristuses see kaob.
\par 15 Veel tänapäevalgi, kui Moosest loetakse, on kate nende südame peal;
\par 16 aga kui nende süda pöördub Issanda poole, siis kate võetakse ära.
\par 17 Sest Issand on Vaim, ja kus on Issanda Vaim, seal on vabadus!
\par 18 Aga me kõik, kes katmatu palgega vaatleme Issanda auhiilgust otsekui peeglis, muutume samasuguseks kujuks aust ausse nagu Issanda Vaimust.


\chapter{4}

\section*{Julge avameelsus evangeeliumi kuulutamisel}

\par 1 Sellepärast et meil on niisugune amet selle halastuse järgi, mis meile on osaks saanud, siis me ei lähe araks,
\par 2 vaid me oleme loobunud salajastest häbitegudest ega ela tigedas kavaluses ega võltsi Jumala sõna, vaid tõe ilmsikstoomisega me asetume kõigi inimeste südametunnistuse otsuse ette Jumala ees.
\par 3 Kui nüüd ka meie evangeelium on kinni kaetud, siis on see kinni kaetud nende eest, kes hukka lähevad,
\par 4 kelle uskmatud meeled selle maailma jumal on teinud sõgedaks, et neile ei paistaks Kristuse au evangeeliumi valgus, kes on Jumala kuju.
\par 5 Sest me ei kuuluta mitte iseendid, vaid Kristust Jeesust, et tema on Issand ja meie teie orjad Jeesuse pärast.
\par 6 Sest Jumal, kes ütles: „Paistku valgus pimedusest!”, on see, kes on hakanud paistma meie südametes, et tekiks tunnetuse valgus Jeesuse Kristuse isikus olevast Jumala aust.

\section*{Apostli nõtrus ja Jumala vägi}

\par 7 Aga meil on see aare saviriistades, et üliväga suur vägi oleks Jumala poolt ja mitte meist.
\par 8 Meid rõhutakse kõikepidi, kuid ei suruta maha; me oleme nõutud, aga me ei heida meelt;
\par 9 me oleme taga kiusatud, aga me ei ole hüljatud; me oleme maha vajutatud, aga me ei hukku.
\par 10 Me kanname alati enestega Issanda Jeesuse surma oma ihus, et ka Jeesuse elu meie ihus tuleks ilmsiks.
\par 11 Sest meid, kes elame, antakse alatasa surma Issanda Jeesuse pärast, et ka Jeesuse elu meie surelikus lihas saaks avalikuks.
\par 12 Nõnda siis on surm vägev meie sees, aga elu teie sees!
\par 13 Aga et meil on seesama usu Vaim sedamööda, nagu on kirjutatud: „Ma usun, sellepärast ma räägin!”, siis meiegi usume ja sellepärast me ka räägime,
\par 14 teades, et see, kes üles äratas Issanda Jeesuse, ka meid üles äratab ühes Jeesusega ja seab ühes teiega enese ette.
\par 15 Sest kõik sünnib teie pärast, et üha rohkemate kaudu üliväga suureks saav arm valmistaks ikka rohkemat tänu Jumala austuseks.

\section*{Ajutine viletsus ja igavene au}

\par 16 Sellepärast me ei loidu, vaid kuigi meie väline inimene kõduneb, uueneb sisemine inimene ometi päev-päevalt.
\par 17 Sest see silmapilkne kerge viletsus saavutab meile määratu suure ja rohke igavese au,
\par 18 meile, kes me ei vaata nähtavaile, vaid nähtamatuile asjadele; sest nähtavad asjad on ajalikud, nähtamatud aga on igavesed.


\chapter{5}

\section*{Ajutine viletsus ja igavene au}

\par 1 Sest me teame, et kui meie maine telkhoone maha kistakse, on meil elamu Jumala käest, mitte kätega tehtud igavene hoone taevas.
\par 2 Sellepärast me ka ohkame igatsedes, et kaetud saada taevase eluasemega,
\par 3 ometi nõnda, et meid kaetult kord ei leitaks alasti olevat.
\par 4 Sest meiegi, kes elame selles telgis, ohkame olles koorma all, sellepärast et me ei taha kattest vabaneda, vaid kaetud olla, et elu neelaks ära selle, mis on surelik.
\par 5 Aga meid on sellesama tarvis valmistanud Jumal, kes meile on andnud Vaimu pandi.
\par 6 Sellepärast me oleme julged igal ajal ja teame, et niikaua kui me kodus oleme ihus, oleme Issanda juurest ära;
\par 7 sest me käime usus ja mitte nägemises;
\par 8 aga me oleme julges meeles ja himustaksime ennemini ära olla ihust ja kodus olla Issanda juures.
\par 9 Sellepärast me ka püüame olla temale meelepärased, kas oleme kodus või võõrsil.
\par 10 Sest me kõik peame saama ilmsiks Kristuse kohtujärje ees, et igaühele tasutaks vastavalt sellele, mis ta ihus olles tegi, olgu see hea või kuri!

\section*{Kristuse armastus apostli sundijaks}

\par 11 Et me nüüd teame, mis on Issanda kartus, siis me meelitame inimesi uskuma, aga Jumalale me oleme tuntud, ja ma loodan, et me ka teie südametunnistuses oleme tuntud.
\par 12 Me ei soovita jälle endid teie ees, vaid me anname teile põhjust kiidelda meist, et teil oleks millega kiidelda nende ees, kes kiitlevad sellega, mis on silma ees, aga mitte sellega, mis on südames.
\par 13 Sest kui me oleme olnud arust ära, siis oleme seda olnud Jumalale; ja kui me oleme selge meelega, siis teile.
\par 14 Sest Kristuse armastus sunnib meid ning me otsustame nõnda: kui üks on surnud kõikide eest, siis on kõik surnud;
\par 15 ja ta on surnud kõikide eest, et need, kes elavad, ei elaks enam enestele, vaid temale, kes nende eest on surnud ja üles tõusnud.

\section*{Uus elu Kristuses}

\par 16 Sellepärast me ei tunne praegusest ajast peale kedagi liha poolest; ja kui me ka oleme tundnud Kristust liha poolest, siis me ei tunne teda nüüd mitte enam nõnda.
\par 17 Niisiis: kui keegi on Kristuses, siis ta on uus loodu; vana on möödunud, vaata, uus on tekkinud!
\par 18 Aga kõik on Jumalast, kes meid on iseenesega lepitanud Kristuse läbi ja on meile andnud lepitusameti.
\par 19 Sest Jumal oli Kristuses ja lepitas maailma iseenesega ega arvanud neile nende üleastumisi süüks ja on pannud meie sisse lepitussõna.

\section*{Manitsus leppimiseks Jumalaga}

\par 20 Seepärast me oleme nüüd käskjalad Kristuse asemel, otsekui manitseks Jumal meie läbi. Me palume Kristuse asemel: andke endid lepitada Jumalaga!
\par 21 Ta on tema, kes ei teadnud mingist patust, meie eest teinud patuks, et meie saaksime Jumala õiguseks tema sees.


\chapter{6}

\section*{Manitsus leppimiseks Jumalaga}

\par 1 Aga olles kaastegevad me ka manitseme, et te Jumala armu ilmaasjata vastu ei võtaks!
\par 2 Sest ta ütleb: „Ma olen sind kuulnud soodsal ajal ja aidanud päästepäeval!” Näe, nüüd on hästi soodus aeg, vaata, nüüd on päästepäev!
\par 3 Me ei anna üheski asjas mingit põhjust pahanduseks, et meie ametit ei laidetaks,
\par 4 vaid kõiges me näitame endid kui Jumala abilised: suures kannatuses, viletsustes, hädades, kitsikustes,
\par 5 hoopide all, vangis, mässudes, vaevanägemistes, valvamistes, paastumistes,
\par 6 puhtuses, tunnetuses, pikas meeles, helduses, Pühas Vaimus, silmakirjatsematus armastuses,
\par 7 tõe sõnas, Jumala väes, õiguse sõjariistade varal paremas ja vasakus käes,
\par 8 aus ja häbis, kurja kõne all ja hea all, kui eksitajad ja ometi tõtt rääkijad,
\par 9 kui tundmatud ja ometi küll tuntud, kui surijad, ja vaata, me elame; kui karistatud, kuid mitte surmatud;
\par 10 kui kurvastatud, kuid ikka rõõmsad; kui vaesed, kes aga teevad paljusid rikkaks; kui need, kellel ei ole midagi ja kelle päralt on kõik!

\section*{Palve, et korintlased osutaksid armastust apostlile}

\par 11 Meie suu on avatud teie vastu, korintlased, meie süda on avardunud;
\par 12 meie sees ei ole teil kitsas, vaid teil on kitsas teie omas südames.
\par 13 Vastutasuks tehke sedasama - ma ütlen teile kui lastele - avardage ka teie oma süda!

\section*{Hoiatus sõpruse eest uskmatutega}

\par 14 Ärge hakake võõras ikkes vedama ühes uskmatutega; sest mis on ühist õigusel ülekohtuga? Mis on ühist valgusel pimedusega?
\par 15 Kuidas sobib Kristus ühte Beliariga? Või mis osa on usklikul uskmatuga?
\par 16 Kuidas sünnib Jumala tempel ühte ebajumalatega? Sest meie oleme elava Jumala tempel, nõnda nagu Jumal on öelnud: „Ma elan nende sees ja käin ja olen nende Jumal, ja nad peavad olema mu rahvas!”
\par 17 Sellepärast „minge ära nende keskelt ja eralduge neist”, ütleb Issand, „ja ärge puudutage roojast, siis ma võtan teid vastu
\par 18 ja olen teile Isaks, ja teie olete mulle poegadeks ja tütardeks”, ütleb kõigeväeline Issand.


\chapter{7}

\section*{Hoiatus sõpruse eest uskmatutega}

\par 1 Et meil nüüd on niisugused tõotused, mu armsad, siis puhastagem endid kõigest liha ja vaimu rüvedusest ja tehkem täielikuks oma pühitsus Jumala kartuses.

\section*{Apostli raskused ja julgustus Tiituse tulekust}

\par 2 Andke meile maad oma südames! Me ei ole teinud kellelegi ülekohut, ei ole kellelegi saatnud kahju, ei ole kedagi petnud.
\par 3 Seda ma ei ütle teie hukkamõistmiseks; sest mina olen juba öelnud, et te olete meie südames, et me ühes sureksime ja ühes elaksime.
\par 4 Mul on palju lootust teie suhtes, mul on teist palju kiitlemist, ma olen täis troosti; ma olen üliväga rõõmus kõiges meie kitsikuses.
\par 5 Sest ka Makedooniasse saabudes ei olnud meie lihal mingit rahu, vaid meid vaevati kõikepidi; väljaspool oli võitlusi ja seespool kartust.
\par 6 Aga Jumal, kes masendatuid kinnitab, on meid kinnitanud Tiituse tulemisega;
\par 7 kuid mitte ainult tema tulemisega, vaid ka selle troostiga, mille ta oli teilt saanud, sest ta jutustas meile teie igatsemisest, teie kurtmisest, teie innust minu heaks, nõnda et ma rõõmustusin veel enam.
\par 8 Sest kuigi ma teid kirjaga kurvastasin, ei kahetse ma seda mitte; ja kuigi ma kahetsesin - ma näen ju, et see kiri teid on kurvastanud, olgugi lühikeseks ajaks -
\par 9 siis olen nüüd rõõmus, mitte sellest, et kurvastusite, vaid et kurvastusite meeleparanduseks; sest te kurvastusite Jumala meele järgi, et teil ei oleks mingit kahju meilt.
\par 10 Sest kurvastus Jumala meele järgi saavutab meeleparanduse, mis toob pääste, mida ei kahetseta; aga maailma kurvastus toob surma.
\par 11 Sest vaadake, kuidas otse see, et te saite kurvaks Jumala meele järgi, on teis tekitanud millise hoole, millise vabandamise, millise meelepaha, millise kartuse, millise igatsemise, millise innukuse, millise karistuse! Te olete kõikepidi osutanud endid puhtaiks selles asjas.
\par 12 Nõnda siis: kuigi ma teile kirjutasin, siis ei sündinud see mitte tema pärast, kes tegi paha, ega tema pärast, kellele tehti paha, vaid selleks, et teie hool meie heaks saaks avalikuks teie keskel Jumala ees.
\par 13 Selle poolest oleme trööstitud. Oma troostiks me rõõmustusime aga veel rohkem Tiituse rõõmust, et tema vaim kosutust on saanud teie kõikide poolt.
\par 14 Sest kui ma temale teist midagi olen kiidelnud, ei ole ma jäänud häbisse, vaid otsekui me teile oleme rääkinud tervet tõtt, nõnda on ka meie kiitlemine teist Tiitusele osutunud tõeks.
\par 15 Ja tema süda on seda rohkem teie poole, kui ta meenutab teie kõikide sõnakuulelikkust, kuidas te kartuse ja väristusega võtsite ta vastu.
\par 16 Olen rõõmus, et ma kõiges võin olla julge teie suhtes!


\chapter{8}

\section*{Korjandus Jeruusalemma vaeste heaks}

\par 1 Aga me teeme teile, vennad, teatavaks armu, mille Jumal on andnud Makedoonia kogudustele,
\par 2 et neil keset suurt kitsikuse katsumist oli rõõmu ülikülluses ja nende põhjatu vaesus kasvas nende südamlikkuse rikkuseks.
\par 3 Sest nad on andnud jõudu mööda - ma tunnistan seda, koguni üle jõu - ja seda vabatahtlikult,
\par 4 käies meile kangesti peale anumisega võtta vastu see and ja lubada neid osa võtta pühade abistamisest;
\par 5 ja mitte ainult nõnda nagu lootsime, vaid nad andsid iseendidki esiti Issandale, siis meile, Jumala tahtel,
\par 6 nii et me õhutasime Tiitust ka teie keskel selle anni korjamist nõnda lõpetama, nagu ta seda juba oli alustanud.
\par 7 Aga otsekui te kõiges olete rikkad, usus ja sõnas ja tunnetuses ja kõiksuguses usinuses ja armastuses meie poolt teie vastu, nõnda olge helde käega selleski armastustöös.

\section*{Jeesuse eeskuju}

\par 8 Ma ei ütle seda käskides, vaid seades teiste usinust teile eeskujuks, ma tahaksin teada saada, kas teiegi armastus on tõsine.
\par 9 Sest te tunnete meie Issanda Jeesuse Kristuse armu, et tema, kuigi ta oli rikas, sai vaeseks teie pärast, et teie tema vaesusest saaksite rikkaks.
\par 10 Selle asja poolest ma annan nõu; sest see tuleb kasuks teile, kes juba mullu olite esimesed mitte ainult tegema, vaid ka tahtma.
\par 11 Aga nüüd viige ka täide oma tegemine, et nõnda nagu teil oli himu tahta, nõnda te ka lõpetaksite oma teo, sedamööda kuidas teil jõudu on.
\par 12 Sest kui on head tahtmist anda, siis on see vastuvõetav sel määral kuidas jõud kannab, mitte üle jõu.
\par 13 Sest ei ole vaja, et mis teistele on kergenduseks, teile oleks koormaks, vaid tasakaalustuseks
\par 14 nende puudusele olgu see, mida teil praegusel ajal on ülearu, et ka see, mida neil on ülearu, aitaks teie puudust ja nõnda sünniks tasakaal,
\par 15 nagu on kirjutatud: „Sellel, kel oli palju, ei olnud ülearu, ja kel oli pisut, ei olnud puudust!”

\section*{Tiituse ja ta kaaslaste läkitamine}

\par 16 Aga tänu olgu Jumalale, kes niisuguse hoole teie heaks andis Tiituse südamesse.
\par 17 Sest ta võttis kuulda manitsust; ent ta enese hool oli veel suurem ja ta läks omast tahtest teele teie juurde.
\par 18 Me läkitasime ühes temaga venna, kelle kuulsus evangeeliumi kuulutamises on tuttav kõigis kogudustes;
\par 19 ja mitte ainult seda, vaid ta on ka koguduste poolt valitud meie teekonna kaaslaseks kohale viima seda armastuseandi, mida meie kogume, Issanda enese auks ja meie hea tahte tunnuseks.
\par 20 Nõnda me hoidume selle eest, et ükski meid ei laimaks selle rohke anni suhtes, mida me teostame.
\par 21 Sest me hoolitseme selle eest, mis on hea, mitte ainult Issanda ees, vaid ka inimeste ees.
\par 22 Me aga läkitasime ühes nendega veel oma venna, kelle innukust me oleme sageli saanud tunda paljudes asjades ja kes on nüüd veel palju innukam selle suure usalduse tõttu, mis tal teie vastu on.
\par 23 Mis Tiitusesse puutub, siis on ta minu seltsimees ja kaastööline teie juures; mis puutub meie vendadesse, siis on nemad koguduste apostlid ja Kristuse au.
\par 24 Niisiis andke neile koguduste ees tõestus oma armastusest ja sellest, et meie kiitlemine teie kohta on tõsi.


\chapter{9}

\section*{Tiituse ja ta kaaslaste läkitamine}

\par 1 Pühade abistamisest aga on mul tõesti ülearune teile kirjutada;
\par 2 sest ma tean teie head tahet ning kiitlen teist sellega makedoonlaste ees, et Ahhaia juba mullu on olnud valmis ja et teie innukus on paljusid õhutanud.
\par 3 Aga ma läkitasin need vennad, et meie kiitlemine teist selles asjas ei läheks tühja, vaid te oleksite valmis, nagu ma olen rääkinud;
\par 4 muidu, kui makedoonlasi peaks tulema ühes minuga ja nad ei leiaks teid valmis olevatena, meie - või kas pean ütlema „teie” - jääksime häbisse oma lootuses.
\par 5 Olen siis arvanud tarvilikuks õhutada vendi minema teele teie poole ette valmistama teie varemini lubatud tänuandi, et see oleks valmis tänuna ja mitte ihnutsemisena.

\section*{Helde annetamine ja Jumala vastutasu}

\par 6 Aga pange tähele: kes kasinasti külvab, see lõikab ka kasinasti, ja kes rohkesti külvab, see lõikab ka rohkesti.
\par 7 Igaüks andku nõnda, kuidas süda kutsub, mitte kurva meelega ega sunniviisil, sest Jumal armastab rõõmsat andjat.
\par 8 Aga Jumal on vägev teile andma kõike armu rohkesti, et teil ikka oleks kõike igati küllaldaselt ning oleksite rikkad iga hea teo tarvis,
\par 9 nõnda nagu on kirjutatud: „Ta puistab välja, ta jagab vaestele, tema õigus püsib igavesti!”
\par 10 Aga tema, kes annab seemet külvajale, annab ka leiba toiduseks ja rohkendab teie külvi ja kasvatab teie õiguse vilja,
\par 11 nõnda et te, saades rikkaks kõige poolest, võiksite helde südamega teha kõike head, mis meie läbi valmistab tänu Jumalale.
\par 12 Sest selle anni kaudu antud abi ei kõrvalda mitte ainult pühade puudust, vaid toob ka rohket õnnistust selle läbi, et paljud tänavad Jumalat
\par 13 ja selle abistamise tõestuse tõttu ülistavad Jumalat selle eest, et te nii allaheitlikult tunnistate Kristuse evangeeliumi ja nii helde südamega teete andide osalisteks neid ja kõiki,
\par 14 ja oma palvetes teie eest nad avaldavad igatsust teie järele Jumala üliväga suure armu tõttu, mis teile on osaks saanud.
\par 15 Tänu olgu Jumalale tema ütlematu suure anni eest!


\chapter{10}

\section*{Pauluse meelevald on Jumalalt}

\par 1 Aga mina, Paulus, manitsen teid isiklikult Kristuse tasase ja järeleandliku meele pärast, kes ma teie silma ees olles küll olen alandlik, kuid ära olles julge teie vastu,
\par 2 ja palun, et mul ei oleks tarvis teie juures viibides südi olla selle julgusega, millega ma mõtlen südikas olla nende vastu, kes meist arvavad, nagu käiksime meie liha järgi.
\par 3 Sest kuigi me käime lihas, ei sõdi me mitte liha viisi järgi.
\par 4 Sest meie võitluse relvad ei ole lihalikud, vaid nad on vägevad Jumala ees maha lõhkuma kindlustusi, hävitades kõik mõistuse targutused.
\par 5 Ja me lükkame ümber kõik kõrgistused, mis tõstetakse Jumala tunnetuse vastu, võttes vangi iga mõtte Kristuse sõnakuulelikkuse alla
\par 6 ja olles valmis nuhtlema kõike sõnakuulmatust, kui teie sõnakuulelikkus enne on saanud täielikuks.
\par 7 Vaadake õieti seda, mis on silme ees! Kui keegi on eneses veendunud, et ta on Kristuse päralt, siis ta mõelgu iseeneses ka seda, et otsekui tema on Kristuse päralt, nõnda oleme ka meie.
\par 8 Sest kui ma ka rohkem hakkaksin kiitlema meie meelevallast, mis Issand on andnud teie ehitamiseks ja mitte mahalõhkumiseks, siis ma ei jääks mitte häbisse,
\par 9 et ei näiks, nagu tahaksin ma teid hirmutada kirjadega.
\par 10 Sest öeldakse: „Tema kirjad on küll ranged ja võimsad, kuid ihulikult meie juures olles on ta nõrk ja ta kõne on kõhn!”
\par 11 Nõnda ütleja mõtelgu, et millised me oleme eemal olles kirjade kaudu sõnadelt, sellised me oleme ka teie juures olles tegudelt!
\par 12 Sest me ei julge arvata iseendid mõningate sekka või võrrelda nendega, kes iseendid soovitavad; ent nad ei saa aru, et nad endid mõõdavad iseenestega ja võrdlevad iseenestega.
\par 13 Aga me ei taha ennast kiita ülemäära, vaid seda mõõdupuud mööda, mille Jumal meile on mõõduks määranud, et me ulatuksime ka teieni.
\par 14 Sest me ei pinguta ennast liialt, nagu ei saakski me teie juurde, sest me oleme juba ulatunud teieni Kristuse evangeeliumiga.
\par 15 Me ei kiida ennast ülemäära võõra vaevaga, vaid meil on lootus, et teie usu kasvades ka meie oma mõõdupuule vastavalt saame üliväga suureks teie seas,
\par 16 et me neisse maadesse, mis on teist tagapool, saaksime viia evangeeliumi ega kiitleks sellega, mis on korda saadetud võõral tegevusalal.
\par 17 Aga kes kiitleb, see kiidelgu Issandast!
\par 18 Sest mitte see ei ole kõlvuline, kes ennast ise soovitab, vaid keda Issand soovitab.


\chapter{11}

\section*{Pauluse kui apostli õigusest}

\par 1 Oh, et te minult ometi salliksite natuke rumalust! Küllap te seda sallitegi ka minult!
\par 2 Sest ma olen armukade teie pärast jumaliku armukadedusega; olen ma ju teid kihlanud üheainsale mehele, Kristusele, et teid esitada temale kui puhast neitsit.
\par 3 Aga mina kardan, et nõnda nagu madu pettis Eevat oma kavalusega, vahest teiegi mõtted rikutakse ja need loobuvad siirast meelest ja kasinusest Kristuse suhtes.
\par 4 Sest kui keegi tuleb ja kuulutab mingisugust teist Jeesust, keda meie ei ole kuulutanud, või kui te saate teise vaimu, keda te ei ole saanud, või teise evangeeliumi kui see, mille te olete vastu võtnud, siis te seda küll salliksite.
\par 5 Sest ma ei arvanud ennast sugugi vähema olevat neist ülisuurtest apostlitest.
\par 6 Aga ehk ma küll olen õppimatu kõnelt, ei ole ma seda siiski mitte tunnetuselt, vaid me oleme kõikepidi seda kõigis asjus teile avalikuks teinud.
\par 7 Või olen ma teinud pattu sellega, et ma ennast alandasin, et teid ülendataks, ja teile ilma palgata kuulutasin Jumala evangeeliumi?
\par 8 Teisi kogudusi ma riisusin, neilt palka võttes, et teid teenida,
\par 9 ja kui mul oli puudust teie juures olles, ei olnud ma ühelegi vaevaks. Sest mis mul puudus, selle tasusid vennad, kes tulid Makedooniast, ja mingeis asjus ma ei tahtnud olla teile koormaks ega taha nüüdki.
\par 10 Nii tõesti kui Kristuse tõde on minus, ei tõkestata seda mu kiitlemist Ahhaiamaa kohtades.
\par 11 Miks? Kas sellepärast, et ma ei armasta teid? Jumal teab seda.
\par 12 Aga mida ma teen, tahan ma teha veelgi, et ära lõigata nende põhjust, kes otsivad põhjust, et neid leitaks neis asjus, millega nad kiitlevad, olevat samasugused nagu meie.
\par 13 Sest need inimesed on valeapostlid, petised töötegijad, kes endid moondavad Kristuse apostleiks.
\par 14 Ja see ei ole ime; sest saatan ise moondab ennast valguseingliks.
\par 15 Sellepärast ei ole suur asi, kui ka tema abilised endid moondavad õiguse abilisteks. Ent nende ots on nende tegude järgi.

\section*{Apostli ja ta vastaste võrdlemine}

\par 16 Veel ma ütlen: ükski ärgu arvaku mind rumala olevat; aga kui ometi, siis võtke mind vastu kui rumalat, et ka mina võiksin pisut kiidelda.
\par 17 Mida ma nüüd räägin, seda ma ei räägi Issanda nõu järgi, vaid otsekui rumaluses, selles kindlas arvamises, et mul on asja kiitlemiseks.
\par 18 Kuna paljud kiitlevad liha poolest, siis kiitlen minagi.
\par 19 Sest te sallite rumalaid heal meelel, olles ise mõistlikud.
\par 20 Sest te sallite, kui keegi teid orjastab, kui keegi teid paljaks sööb, kui keegi teid koorib, kui keegi on ülbe, kui keegi lööb teile vastu silmi.

\section*{Apostli kannatused ja surmaohud}

\par 21 Häbi pärast ma ütlen, et meie nagu oleksime olnud nõdrad. Aga mille üle keegi uhke on - ma räägin rumalusest - selle üle olen ka mina uhke.
\par 22 Nemad on heebrea mehed, mina ka; nemad on Iisraeli lapsed, mina ka; nemad on Aabrahami sugu, mina ka.
\par 23 Nemad on Kristuse teenijad - ma räägin pööraselt - mina olen rohkem. Ma olen rohkem näinud vaeva, olen rohkem olnud vangis, olen palju rohkem saanud hoope, olen sagedasti olnud surmahädas.
\par 24 Juutide käest ma olen saanud viis korda ühe hoobi vähem kui nelikümmend.
\par 25 Kolm korda on mind vitstega pekstud, üks kord kividega visatud, kolm korda on laev hukka läinud, terve öö ja päeva ma olen olnud mere sügavuses.
\par 26 Sagedasti olen ma oma teekondadel hädaohus olnud jõgedel, hädas mõrtsukate käes, hädas oma rahva seas, hädas paganate keskel, hädas linnas, hädas kõrbes, hädas merel, hädas valevendade seas,
\par 27 töös ning vaevas; sagedasti valvamises, näljas ja janus, sagedasti paastumistes, külmas ja alastiolekus;
\par 28 peale muude asjade igapäevane rahva kokkuvool, mure kõigi koguduste eest.
\par 29 Kes on nõder, ja mina ei peaks jääma nõdraks? Kes pahandab, ja minu hing ei peaks süttima?
\par 30 Kui tuleb kiidelda, siis ma kiitlen oma nõtrusest!
\par 31 Jumal ja meie Issanda Jeesuse Kristuse Isa, kes on igavesti kiidetud, teab, et ma ei valeta.
\par 32 Damaskuses valvas kuningas Aretase pealik damasklaste linna ja tahtis mind kinni võtta,
\par 33 ja mind lasti aknaaugust korviga üle müüri alla ja nõnda ma pääsesin tema käest.


\chapter{12}

\section*{Ülendavaist nägemustest ja alandavaist nõtrustest: nõdra vägevusest}

\par 1 Tuleb kiidelda, ehk mul sellest küll pole kasu. Aga nüüd ma siirdun nägemustesse ja Issanda ilmutustesse.
\par 2 Ma tunnen inimest Kristuses, keda neljateistkümne aasta eest tõmmati kolmanda taevani - kas ta oli ihus, seda ma ei tea, või kas ta oli ihust väljas, mina ei tea, Jumal teab.
\par 3 Ja ma tean, et sama inimene - kas ta oli ihus või ihust lahus, mina ei tea, Jumal teab -
\par 4 tõmmati paradiisi ja ta kuulis räägitamatuid sõnu, mida inimene ei tohi rääkida.
\par 5 Sellestsamast mehest ma tahangi kiidelda; aga iseenesest ma ei kiitle muuga kui oma nõtrustega.
\par 6 Sest kui ma tahaksingi kiidelda, ei oleks ma rumal, sest ma räägiksin tõtt. Aga ma keeldun sellest, et keegi ei arvaks minust rohkem kui ta mind näeb olevat, või mida ta minust kuuleb
\par 7 ka väga suurte ilmutuste tõttu. Sellepärast et ma ei läheks kõrgiks, on mulle liha sisse antud vai, saatana ingel, mind rusikaga lööma, et ma ei läheks kõrgiks.
\par 8 Sellesama pärast olen ma kolm korda Issandat palunud, et see minust lahkuks.
\par 9 Aga ta ütles mulle: „Sulle saab küllalt minu armust; sest vägi saab nõtruses täie võimuse!” Nii ma siis tahan meelsamini kiidelda oma nõtrustest, et Kristuse vägi asuks elama minusse.
\par 10 Sellepärast olen meeleldi nõtrustes, vägivalla all, hädades, tagakiusamistes ja kitsikustes Kristuse pärast; sest kui olen nõder, siis olen vägev!

\section*{Apostel vähendab oma kiitlemist}

\par 11 Ma olen läinud rumalaks; teie olete mind selleks sundinud. Sest mina oleksin pidanud saama kiitust teilt; sest ma pole sugugi vähem kui ülisuured apostlid, ehk ma küll ei ole midagi.
\par 12 Apostli tunnusteod on ju teie seas täide saadetud kõige kannatlikkusega, nii tunnustähtede kui imede ja väeavaldustega.
\par 13 Sest mille poolest teie olete olnud halvemad kui teised kogudused, kui vaid selle poolest, et mina ise ei ole olnud teile koormaks? Andke mulle see süü andeks!
\par 14 Vaata, ma olen nüüd kolmandat korda valmis tulema teie juurde ega tule teile koormaks; sest mina ei otsi teie oma, vaid teid endid; sest lapsed pole kohustatud koguma varandust vanemaile, vaid vanemad lastele.
\par 15 Aga ma teen meeleldi kulu ja kulutan iseennast teie hingede eest, kuigi mind, kes ma teid väga armastan, pisut armastatakse.
\par 16 Aga olgu peale, mina ei ole teid koormanud, kuid „ma olin kelm ja võtsin teid kavalusega kinni!”
\par 17 Kas ma kellegi kaudu neist, keda ma läkitasin teie juurde, olen teid koorinud?
\par 18 Mina õhutasin Tiitust tulema ja ühes temaga ma saatsin ühe venna. Kas Tiitus teid on koorinud? Kas me ei ole käinud ühesuguses vaimus? Ja kas ka mitte samades jälgedes?
\par 19 Teie mõtlete ammugi, et me kostame eneste eest teie ees? Me räägime Kristuses Jumala ees. Aga kõik, armsad vennad, teie ehitamiseks.
\par 20 Sest ma kardan, et kui ma tulen, ma ei leia teid niisugustena nagu tahan, ja et teie leiate mind seesugusena, nagu te ei taha. Ma kardan, et teie seas vahest on riidu, kadedust, viha, jonni, laimamisi, keelekandmist, hooplemist, korratust;
\par 21 ja et kui ma tulen, minu Jumal mind jälle alandab teie juures ja ma pean olema kurb mitme pärast, kes enne on teinud pattu ega ole pöördunud oma rüvedusest ja hoorusest ja kiimalusest, mida nad on harrastanud.


\chapter{13}

\section*{Manitsused apostli tulevase külastuse eel}

\par 1 Kolmandat korda juba tulen teie juurde. Kahe ja kolme tunnistaja ütlusega tehakse iga asi kindlaks.
\par 2 Olen enne öelnud ja ütlen veel ette ära - nagu ütlesin siis, kui teist korda teie juures olin, ja samuti nüüd, kui ma ära olen - neile, kes enne on pattu teinud, ja kõigile muile, et kui ma jälle tulen, ma ei säästa.
\par 3 Te otsite ju tõestust selle kohta, kas Kristus räägib minus. Tema ei ole nõder teie suhtes, vaid vägev teie sees.
\par 4 Sest kuigi tema löödi risti nõtruses, ometi elab ta Jumala väe läbi. Nõnda meiegi oleme küll nõdrad tema sees, kuid elame ühes temaga Jumala väest teie jaoks.
\par 5 Katsuge iseendid läbi, kas te olete usus. Uurige iseendid! Või te ehk ei tunne iseendist, et Jeesus Kristus on teie sees? Või ei ole te ehk kõlvulised?
\par 6 Ma loodan, et te tunnete ära, et me ei ole kõlbmatud.
\par 7 Aga me palume Jumalat, et teie ei teeks ühtki kurja, mitte et meie näiksime kõlvulised olevat, vaid et teie teeksite head ja meie oleksime otsekui kõlbmatud.
\par 8 Sest me ei suuda midagi tõe vastu, vaid üksnes tõe poolt!
\par 9 Sest me rõõmutseme, kui meie oleme nõdrad ja teie olete vägevad; ja seda me palumegi, et teie saaksite täiuslikuks.
\par 10 Sellepärast ma kirjutan seda eemal olles, et ei oleks vaja teie juures olles olla range selle meelevalla järgi, mille Issand on andnud minule koguduse ehitamiseks ja mitte mahakiskumiseks.
\section*{Lõppsõna}

\par 11 Lõppeks, vennad, olge rõõmsad, seadke oma asjad korda, võtke manitsust, olge üksmeelsed, pidage rahu. Küll armastuse ja rahu Jumal on siis teiega!
\par 12 Teretage üksteist püha suudlusega! Kõik pühad saadavad teile tervituse!
\par 13 Issanda Jeesuse Kristuse arm ja Jumala armastus ja Püha Vaimu osadus olgu teie kõikidega!





\end{document}