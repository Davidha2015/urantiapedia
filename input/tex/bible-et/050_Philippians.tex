\begin{document}

\title{Pauluse kiri filiplastele}

\chapter{1}

\section*{Tervitus}

\par 1 Paulus ja Timoteos, Kristuse Jeesuse sulased, kõigile pühadele Kristuses Jeesuses, kes elavad Filipis, ja ühtlasi koguduse juhatajaile ja abilistele.
\par 2 Armu teile ja rahu Jumalalt, meie Isalt, ja Issandalt Jeesuselt Kristuselt!

\section*{Apostli tänulikkus ja rõõm}

\par 3 Ma tänan oma Jumalat iga kord, kui ma teid meelde tuletan,
\par 4 alati igas oma palves teie kõikide eest rõõmuga palvetades;
\par 5 ma tänan teie osaduse pärast evangeeliumiga esimesest päevast siiani;
\par 6 ja olen kindel selles, et see, kes teis on alustanud hea töö, viib selle lõpule Kristuse Jeesuse päevani.
\par 7 Nõnda ongi mulle õige seda teist kõigist mõtelda, sest ma kannan teid oma südames, kes te kõik siis, kui ma olen ahelais, ja ka siis, kui kaitsen evangeeliumi ja seda kinnitan, ühes minuga olete armu osalised.
\par 8 Sest Jumal on mu tunnistaja, kuidas ma teid kõiki igatsen Kristuse Jeesuse südamlikkusega.

\section*{Palve kirja lugejate eest}

\par 9 Ja seda ma palun Jumalalt, et teie armastus veel rohkem ja rohkem kasvaks selges tunnetamises ja kõiges tajumises,
\par 10 et te võiksite kindlaks teha, mis on peaasi, ja oleksite puhtad ning laitmatud Kristuse päevaks,
\par 11 täis õiguse vilja, mis tuleb Jeesuse Kristuse läbi, Jumala austuseks ja kiituseks.

\section*{Apostli vangistuse tagajärg}

\par 12 Aga ma tahan teada anda teile, vennad, et see, mis minuga on juhtunud, on andnud evangeeliumile veel rohkem hoogu,
\par 13 nõnda et kogu kohtukojas ja kõigile muile on saanud avalikuks, et ma olen ahelais Kristuse pärast
\par 14 ja et suurem osa vendi mu ahelate tõttu on saanud vahvaks Issandas ja palju rohkem julgevad kartmatult rääkida Jumala sõna.
\par 15 Mõned küll kuulutavad Kristust kadeduse ja riiu pärast, kuid mõned ka hea meelega.
\par 16 Teised kuulutavad armastusest, teades, et ma viibin siin evangeeliumi kaitsmiseks,
\par 17 teised aga kuulutavad Kristust kiusu pärast, mitte puhtast meelest; nad ju mõtlevad suurendada minu vangipõlve viletsust.
\par 18 Aga mis sellest? Kui aga Kristust kuulutatakse igal kombel, olgu silmakirjaks, olgu tõemeelega, ja sellest ma rõõmutsen ja tahan ka edaspidi rõõmutseda;
\par 19 sest ma tean, et see tuleb mulle päästeks teie palve kaudu ja Jeesuse Kristuse Vaimu abil
\par 20 minu kindlat ootust ja lootust mööda, et ma milleski ei jää häbisse, vaid et Kristus nüüdki, nagu ikka, täie julgusega saab auliseks minu ihus, olgu elu või surma läbi.

\section*{Kumb on kasulikum: elu või surm?}

\par 21 Sest minule on elamine Kristus ja suremine kasu.
\par 22 Aga kui lihas elamine on tulusam mu tööle, siis ma ei tea, kumba valida.
\par 23 Mind tõmbab nii see kui teine: ma himustan siit lahkuda ja olla Kristuse juures, sest see on hoopis palju parem;
\par 24 aga lihasse jäämine on vajalikum teie pärast.
\par 25 Ja selles veendumuses ma tean, et ma jään lihasse ja jään teie kõikide juurde teie usu eduks ja rõõmuks,
\par 26 et teie kiitlemine oleks rohke Kristuses Jeesuses minu läbi selle tõttu, et ma jälle tulen teie juurde.

\section*{Julgustussõnad kristlastele}

\par 27 Ainult käituge Kristuse evangeeliumi vääriliselt, et ma, olgu tulles ja teid nähes või tulemata jäädes, kuuleksin teist, et te püsite ühes vaimus ja ühel meelel minuga võitlete evangeeliumi usu eest
\par 28 ega lase endid heidutada mitte millestki vastaste poolt. See on neile hukatuse, teile aga pääste tähiseks, ja nimelt Jumalalt.
\par 29 Sest teile on armust antud Kristuse pärast mitte ainult uskuda temasse, vaid ka kannatada tema pärast
\par 30 ja võidelda sama võitlust, mida te olete näinud minust ja nüüd minust kuulete.


\chapter{2}

\section*{Üleskutse üksmeeleks ja alandlikkuseks}

\par 1 Kui nüüd mingisugune manitsus Kristuses, kui mingisugune armastuse troost, kui mingisugune Vaimu osadus, kui mingisugune südamlikkus ja kaastundmus kehtib,
\par 2 siis tehke mu rõõm täielikuks sellega, et mõtlete sama ja peate ühesugust armastust, olles üksmeelsed ja ühemõttelised
\par 3 ega tee midagi riiu ega tühja au pärast, vaid arvate alanduses üksteist ülemaks kui iseennast,
\par 4 ja et ükski ei vaata selle peale, mis on tulus temale, vaid ka selle peale, mis on tulus teistele.

\section*{Jeesus alandlikkuse eeskujuks}

\par 5 Olgu teil samasugune meel, mis oli ka Kristusel Jeesusel,
\par 6 kes, kuigi ta oli Jumala nägu, ei arvanud saagiks olla Jumalaga ühesugune,
\par 7 vaid loobus iseenese olust ning võttis orja näo, saades inimeste sarnaseks; ja ta leiti välimuselt inimesena;
\par 8 ta alandas iseennast, saades sõnakuulelikuks surmani, pealegi ristisurmani.
\par 9 Sellepärast ongi Jumal tema ülendanud kõrgele ja andnud temale nime üle kõigi nimede,
\par 10 et Jeesuse nimes nõtkuksid kõik põlved, niihästi taevaliste kui maapealsete kui ka maa-aluste omad,
\par 11 ja iga keel tunnistaks, et Jeesus Kristus on Issand, Jumala Isa auks.

\section*{Lunastus sõltub laitmatust elust}

\par 12 Nõnda siis, mu armsad, otsekui te ikka olete olnud sõnakuulelikud, mitte üksnes nagu siis, kui ma olin teie juures, vaid ka nüüd veel rohkem minu ära olles, nõudke oma päästet kartuse ja värinaga!
\par 13 Sest Jumal on see, kes teis on tegev, et te tahate ja tegutsete tema hea meele järgi.
\par 14 Tehke kõik nurisemata ja kaksipidi mõtlemata,
\par 15 et te oleksite laitmatud ja puhtad, veatud Jumala lapsed keset tigedat ja pöörast sugupõlve, kelle seas teie paistate otsekui taevatähed maailmas,
\par 16 hoides elu sõna, minule kiitluseks Kristuse päeval, et ma ei ole tühja jooksnud ega teinud tühja tööd.
\par 17 Ja kuigi minu veri peaks kahjana valatama teie usu ohvri ja jumalateenistuse üle, siiski ma rõõmutsen ja olen rõõmus ühes teie kõikidega.
\par 18 Sellesama pärast olge ka teie rõõmsad ja rõõmutsege ühes minuga.

\section*{Timoteos}

\par 19 Ma loodan Issandas Jeesuses Timoteose varsti läkitada teie juurde, et ka minul oleks hea meel teada saades, kuidas teie käsi käib.
\par 20 Sest mul pole kedagi muud nii üksmeelset minuga, kes nõnda tõsiselt hoolitseks teie eest.
\par 21 Sest nad kõik otsivad eneste oma, aga mitte seda, mis on Kristuse Jeesuse oma.
\par 22 Aga tema ustavust te teate, sest nagu laps oma isaga, nõnda on ta minuga ühes töötanud evangeeliumi kasuks.
\par 23 Teda ma loodan nüüd läkitada, niipea kui näen, kuidas minu asi areneb.
\par 24 Ent mul on kindel lootus Issandas, et ma ka ise varsti tulen.

\section*{Epafroditos}

\par 25 Aga ma olen arvanud tarviliseks läkitada teie juurde vend Epafroditose, oma kaastöölise ja kaasvõitleja, teie apostli ja minu vajaduste eest hoolitseja,
\par 26 sellepärast et ta teid kõiki igatses ja suurt tuska tundis sellest, et olite kuulnud tema haigestumisest.
\par 27 Sest tema oli ka tõesti suremas haige; ent Jumal andis temale armu, aga mitte üksi temale, vaid ka minule, et mulle ei tuleks kurbust kurbuse peale.
\par 28 Sellepärast ma läkitasin tema seda nobedamini, et te teda nähes jälle rõõmustuksite ja mul oleks vähem kurbust.
\par 29 Siis võtke ta vastu Issandas kõige rõõmuga ja pidage niisuguseid mehi kalliks.
\par 30 Sest Kristuse töö pärast oli ta surma suus ega hoolinud oma elust, et täita seda, mis teil jäi tegemata hoolitsemises minu eest.


\chapter{3}

\section*{Väliste eesõiguste väike väärtus}

\par 1 Peale kõige, mu vennad, rõõmutsege Issandas! Üht ja sama kirjutades ei tüüta see mind, aga teid see kinnitab.
\par 2 Pidage silmas koeri, pidage silmas kurje töötegijaid, hoiduge katkilõikamise eest!
\par 3 Sest ümberlõigatud, need oleme meie, kes Jumalat teenime vaimus, ja kiitleme Kristusest Jeesusest ega looda liha peale,
\par 4 ehkki mul oleks lootust liha peale. Kui keegi muu arvab, et tema võib liha peale loota, siis mina veel rohkem!
\par 5 Ma olen ümber lõigatud kaheksandal päeval, ma olen Iisraeli soost, Benjamini suguharust, ma olen heebrealane, sündinud heebrealasist, käsuõpetuselt variser,
\par 6 innukas koguduse tagakiusaja, laitmatu käsuõiguse poolest.

\section*{Apostli igatsus ühineda Kristusega}

\par 7 Aga mis mulle oli kasuks, olen ma arvanud kahjuks Kristuse pärast.
\par 8 Jah tõesti, mina arvan kõik kahjuks oma Issanda Kristuse Jeesuse ülivõimsa tunnetuse vastu, kelle pärast ma olen minetanud kõik selle ja pean kõike pühkmeiks, et kasuks saada Kristust
\par 9 ja et mind leitaks tema seest ega oleks mul oma õigust, mis on käsust, vaid see õigus, mis tuleb Kristuse usu kaudu, see, mis tuleb Jumalalt usu varal,
\par 10 et ma tunneksin ära tema ja ta ülestõusmise väe ja tema kannatamise osaduse ning muutuksin tema surma sarnaseks,
\par 11 kui ma kuidagi pääsen ülestõusmisele surnute seast.
\par 12 Ei mitte, et ma selle juba oleksin kätte saanud või oleksin juba täiuslik, vaid ma püüan kätte saada seda, mille pärast Kristus Jeesus mind on kätte saanud.
\par 13 Vennad, mina ei arva seda juba kätte saanud olevat; ent üht ma ütlen: ma unustan ära, mis on taga, ja sirutun sinnapoole, mis on ees,
\par 14 ma pürin seatud eesmärgi poole, taevase kutsumise võiduhinna poole Kristuses Jeesuses.
\par 15 Nii mitu nüüd, kui meid on täiuslikku, mõelgem sama! Ja kui te midagi mõtlete teisiti, siis Jumal ilmutab teile ka selle.
\par 16 Ometi, kuhu oleme jõudnud, sealt käigem sama teed!

\section*{Järgitagu apostli eeskuju}

\par 17 Järgige minu eeskuju, vennad, ja vaadelge neid, kes elavad nõnda nagu meie teile eeskujuks oleme.

\section*{Kristuse risti vaenlased}

\par 18 Sest paljud, kellest ma teile olen sagedasti öelnud ja nüüd ütlen ka nuttes, elavad Kristuse risti vaenlastena.
\par 19 Nende ots on hukatus, nende jumal on nende kõht ja nende au on nende häbis; nad taotlevad maapealseid asju.
\par 20 Sest meie ühiskond on taevas, kust me ka ootame Issandat Jeesust Kristust kui Õnnistegijat,
\par 21 kes meie alanduse ihu muudab oma äraseletatud ihu sarnaseks väge mööda, millega ta võib ka kõik teha oma alamaks.


\chapter{4}

\section*{Manitsusi}

\par 1 Nõnda, mu armsad ja igatsetud vennad, minu rõõm ja pärg, seiske siis Issandas kindlatena, armsad!
\par 2 Euodiat ma manitsen ja ma manitsen Süntühhet olema ühemõttelised Issandas.
\par 3 Ja ma palun ka sind, tõsine kaastööline, aita neid naisi, kes on võidelnud evangeeliumi eest ühes minuga ja Kleemensiga ja mu teiste abimeestega, kelle nimed on eluraamatus.
\par 4 Olge ikka rõõmsad Issandas! Ja taas ma ütlen: olge rõõmsad!
\par 5 Teie leebus saagu teatavaks kõigile inimestele. Issand on lähedal!
\par 6 Ärge muretsege ühtigi, vaid laske kõiges oma palumised palve ja anumisega ühes tänuga saada Jumalale teatavaks.
\par 7 Ja Jumala rahu, mis on ülem kõigest mõistusest, hoiab teie südamed ja mõtted Kristuses Jeesuses.
\par 8 Viimaks, vennad, kõik, mis on tõsine, mis aus, mis õige, mis kasin, mis armas, mis on hea kuulda kui vooruslik komme ja kiituse väärt, sellele mõelge!
\par 9 Mis te ka olete õppinud ja saanud ja kuulnud ja näinud minult, seda tehke; ja rahu Jumal on siis teiega.

\section*{Apostli rõõm filiplaste lahkest abist}

\par 10 Väga suureks rõõmuks Issandas aga oli mulle see, et te juba uuesti virgusite mõtlema minule: sellele te mõtlesite ennegi, aga teil ei olnud juhust selleks.
\par 11 Ma ei ütle seda puuduse pärast; sest ma olen õppinud olema rahul sellega, mis mul on.
\par 12 Oskan elada vähesega ja oskan elada külluses, olen kõigega ja kõigi oludega harjunud: nii olema söönud kui ka nägema nälga, elama nii külluses kui ka puuduses.
\par 13 Ma suudan kõik temas, kes mind teeb vägevaks.
\par 14 Siiski te tegite hästi, et mind toetasite minu viletsuses.
\par 15 Ent teiegi, filiplased, teate, et armuõpetuse algusaegadest, kui ma läksin teele Makedooniast, ükski kogudus ei olnud osaduses minuga andmise ja võtmise arvepidamises kui ainult teie,
\par 16 sest Tessaloonikassegi te saatsite minu tarviduseks ühe ja teise korra.
\par 17 Mitte, et ma andi püüan, vaid ma otsin vilja, mis kasuks oleks teie arvele.
\par 18 Mul on nüüd kõike ja mul on rohkesti; ma sain küllalt, kui ma Epafroditoselt võtsin vastu, mis teie olite lähetanud magusa lõhnana ja armsa ohvrina, mis on Jumalale meelepärast.
\par 19 Aga küll minu Jumal täidab kõik teie vajaduse oma rikkust mööda auga Kristuses Jeesuses.
\par 20 Aga Jumalale ja meie Isale olgu austus ajastute ajastuteni! Aamen.

\section*{Jumalagajätt}

\par 21 Tervitage kõiki pühasid Kristuses Jeesuses! Teid tervitavad minu juures olevad vennad.
\par 22 Kõik pühad saadavad teile tervitusi, aga iseäranis need, kes on keisri perest.
\par 23 Meie Issanda Jeesuse Kristuse arm olgu teie vaimuga!




\end{document}