\begin{document}

\title{Pauluse Esimene kiri Timoteusele}

\chapter{1}

\section*{Tervitus ja hoiatus valeõpetuse eest}

\par 1 Paulus, Kristuse Jeesuse apostel Jumala, meie Õnnistegija, ja Kristuse Jeesuse, meie lootuse käsul,
\par 2 Timoteosele, oma tõelisele pojale usus. Armu, halastust, rahu Jumalalt Isalt ja Kristuselt Jeesuselt, meie Issandalt!
\par 3 Nõnda kuidas ma Makedooniasse minnes sind õhutasin jääma kauemaks Efesosse, et mõningaid manitseda, et nad ei õpetaks teisiti
\par 4 ega teeks tegemist tühjade juttudega ja lõpmatute suguvõsade üleslugemistega, mis tekitavad rohkem küsimusi kui Jumala majapidamisest arusaamist usus, nõnda tee.
\par 5 Sest käskimise eesmärk on armastus puhtast südamest ja heast südametunnistusest ja silmakirjatsematust usust.
\par 6 Neist on mõned ära taganenud ja kaldunud kõrvale tühja kõne poole,
\par 7 tahtes olla käsuõpetajad, kuigi nad ei mõista seda, mida nad räägivad, ega seda, mida nad kindlasti väidavad.
\par 8 Me ju teame, et käsk on hea, kui keegi seda käsitleb käsu kohaselt
\par 9 ja peab meeles, et käsku pole seatud õigele, vaid ülekohtustele ja kangekaelseile, jumalakartmatuile ja patustele, õelatele ja roojastele, isatapjaile ja ematapjaile, mõrtsukaile,
\par 10 hoorajaile, poisipilastajaile, inimesemüüjaile, valelikele, valevandujaile ja kõigele muule, mis käib õige õpetuse vastu -
\par 11 õndsa Jumala au evangeeliumi järgi, mis on usaldatud minu kätte.

\section*{Apostli tänu Jumala armu eest}

\par 12 Ma tänan teda, kes mind on teinud vägevaks, Kristust Jeesust, meie Issandat, et ta mind on pidanud ustavaks, pannes mind oma teenistusse,
\par 13 ehk ma küll enne olin pilkaja ja tagakiusaja ja vägivaldne. Aga ma olen armu saanud, sest ma olin seda teinud teadmata, uskmatus meeles.
\par 14 Meie Issanda arm aga on olnud üliväga suur usu ja armastusega, mis on Kristuses Jeesuses.
\par 15 Ustav on see sõna ja kõigiti vastuvõetav, et Kristus Jeesus on tulnud maailma päästma patuseid, kelle seast mina olen suurim;
\par 16 kuid sellepärast ma olen saanud armu, et Jeesus Kristus kõigepealt minus osutaks kõike oma pikka meelt eeskujuks neile, kes veel edaspidi saavad usklikuks temasse igaveseks eluks.
\par 17 Aga ajastute Kuningale, kadumatule, nähtamatule, ainule Jumalale olgu au ja austus ajastute ajastuteni! Aamen.
\par 18 Selle käsu ma panen su südamele, mu poeg Timoteos, vastavalt varemaile sinu kohta käivaile ennustustele, et sa nende varal võitleksid õilsat võitlust,
\par 19 omades usku ja head südametunnistust, mille mõningad on tõuganud enesest ära ja on selle tõttu usu suhtes läinud põhja nagu laev.
\par 20 Nende seast on Hümenaios ja Aleksandros, keda ma olen andnud saatana meelevalda, et nad karistuse all võõrduksid pilkamisest.


\chapter{2}

\section*{Palvet tehtagu kõigi inimeste eest}

\par 1 Ma manitsen siis nüüd kõigepealt tegema palveid, palvusi, eestpalveid, tänupalveid kõigi inimeste eest,
\par 2 kuningate ja kõigi ülemate eest, et võiksime vaikset ja rahulikku elu elada kõiges jumalakartuses ja aususes.
\par 3 Sest see on hea ja armas Jumala, meie Õnnistegija meelest,
\par 4 kes tahab, et kõik inimesed õndsaks saaksid ja tõe tunnetusele tuleksid.
\par 5 Sest Jumal on üks, ka vahemees üks Jumala ja inimeste vahel, inimene Kristus Jeesus,
\par 6 kes andis iseenese lunastushinnaks kõikide eest, et sellest antaks tunnistust parajail aegadel;
\par 7 ja selle tarvis olen mina pandud kuulutajaks ja apostliks - ma räägin tõtt Kristuses ega valeta mitte - paganate õpetajaks usus ja tões.
\par 8 Ma tahan siis, et mehed palvetaksid igas kogumispaigas ja tõstaksid üles pühad käed ilma vihata ja kahtlemiseta;
\par 9 nõndasamuti ka, et naised, viisakalt riietatud, endid ehiksid häbeliku ja mõistliku meelega, mitte juuksepalmikutega ega kullaga ega pärlitega ega kalliste riietega,
\par 10 vaid heade tegudega, nõnda kui sobib naistele, kes endid tunnistavad jumalakartlikeks.
\par 11 Naine õppigu vaiksel viisil, olles kõigiti alistuv.
\par 12 Aga naisele ma ei luba õpetada ega valitseda mehe üle, vaid ta elagu vaikselt.
\par 13 Sest Aadam loodi enne, siis Eeva;
\par 14 ega Aadamat petetud, vaid naist peteti ja ta sattus üleastumisse,
\par 15 aga ta saab õndsaks lastesünnitamise läbi, kui ta jääb ususse ja armastusse ning pühitsusse mõistliku meelega.


\chapter{3}

\section*{Nõuded koguduse ülevaatajate ja abiliste kohta}

\par 1 Ustav on see sõna: kui keegi püüab koguduse ülevaataja ametisse, siis ta igatseb kaunist tööd.
\par 2 Niisiis tuleb koguduse ülevaatajal olla laitmatu, ühe naise mees, kaine, mõistlik, viisakas, külalislahke, osav õpetama;
\par 3 mitte joodik, mitte riiakas, vaid leebe, rahunõudja, mitte rahaahne,
\par 4 vaid kes oma maja hästi valitseb, kes oma lapsi peab sõnakuulmises kõige aususega.
\par 5 Sest kui keegi iseenese maja ei oska juhtida, kuidas ta võib hoolt kanda Jumala koguduse eest?
\par 6 Ta ärgu olgu alles vastpöördunu, et ta ei läheks uhkeks ega langeks sama nuhtluse alla kui kurat.
\par 7 Ka olgu tal hea tunnistus neilt, kes on väljaspool kogudust, et ta ei satuks naeru alla ja kuradi köitesse.
\par 8 Nõndasamuti tuleb ka abilistel olla ausad, mitte kahekeelsed, mitte joomamehed, mitte liigkasuvõtjad, vaid sellised,
\par 9 kellel usu saladus on puhtas südametunnistuses.
\par 10 Ent nad katsutagu enne läbi ja siis, kui nad on laitmatud, astugu nad ametisse.
\par 11 Nõndasamuti olgu nende naised ausad, mitte keelepeksjad, kained, ustavad kõigis asjus.
\par 12 Koguduse abilised olgu ühe naise mehed, kes oma lapsi ja oma peret hästi juhivad.
\par 13 Sest need, kes on hästi pidanud oma ametit, saavutavad enestele kauni lugupidamise ja suure julguse usus, mis on Kristuses Jeesuses.

\section*{Meie usu suur ja sügav sisu}

\par 14 Seda kirjutan sulle, ehk ma küll loodan pea tulla sinu juurde,
\par 15 et kui ma viibin, sa teaksid, kuidas tuleb käituda Jumala kojas, mis on elava Jumala kogudus, tõe sammas ja alustugi.
\par 16 Ja vastuvaieldamatu suur on jumalakartuse saladus: Jumal on avalikuks saanud lihas, õigeks mõistetud Vaimus, ilmunud inglitele, kuulutatud paganate seas, usutud maailmas, üles võetud ausse!


\chapter{4}

\section*{Hoiatus eksiõpetuse levitajate eest}

\par 1 Aga Vaim ütleb selge sõnaga, et viimsel ajal mõned taganevad usust ja hoiavad eksitajate vaimude ja kurjade vaimude õpetuste poole,
\par 2 valekuulutajate silmakirjaliku vagaduse tõttu, kes oma südametunnistusse otsekui tulise märgi on põletanud
\par 3 ja keelavad abiellumast ning maitsmast rooga, mis Jumal on loonud selleks, et usklikud ja tõetundjad seda vastu võtaksid tänuga.
\par 4 Sest kõik, mis Jumal on loonud, on hea, ja miski ei ole hüljatav, kui seda vastu võetakse tänuga.
\par 5 Sest seda pühitsetakse Jumala sõna ja palve läbi.
\par 6 Kui sa seda vendadele ette paned, oled sa Kristuse Jeesuse hea abiline ja toidad ennast usu ja hea õpetuse sõnadega, mida sa oled järginud.
\par 7 Ent kõlvatut ja vananaiste vada väldi, aga harjuta ennast jumalakartuses.
\par 8 Sest ihulikust harjutusest on pisut kasu; aga jumalakartusest on kasu kõigile asjadele ja sellel on käesoleva ja tulevase elu tõotus.
\par 9 Ustav on see sõna ja kõigiti vastuvõetav.
\par 10 Sest selleks me näeme vaeva ja võitleme, et me oleme lootnud elava Jumala peale, kes on kõigi inimeste, iseäranis usklike, Õnnistegija.
\par 11 Seda käsi ja õpeta!

\section*{Timoteose isiklik elu ja õpetus}

\par 12 Ükski ärgu põlaku sinu noorust, vaid ole usklikele eeskujuks sõnas, elus, armastuses, usus, meelepuhtuses.
\par 13 Kuni ma tulen, ole hoolas ette lugema, manitsema ja õpetama.
\par 14 Ära jäta eneses peituvat armuannet hooletusse, mis sulle anti ennustuse läbi, kui vanemad panid oma käed sinu peale.
\par 15 Seda harrasta, selles ela, et sinu edenemine oleks ilmne kõikidele.
\par 16 Pane tähele iseennast ja õpetust; püsi kindlasti selles; sest kui sa seda teed, päästad sa enese ja need, kes sind kuulevad.


\chapter{5}

\section*{Timoteose isiklik elu ja õpetus}

\par 1 Vana meest ära sõitle valjusti, vaid manitse teda kui isa, nooremaid kui vendi,
\par 2 vanemaid naisi kui emasid, nooremaid kui õdesid kõiges meelepuhtuses.

\section*{Leskedest ja nende kohtlemisest koguduses}

\par 3 Austa leski, kes on õiged lesknaised.
\par 4 Aga kui kellelgi lesel on lapsi või lapselapsi, siis õppigu need esiti vagalt kohtlema oma majarahvast ja tasuma, mis nad on võlgu oma vanemaile. Sest see on meelepärane Jumala ees.
\par 5 Aga kes õige lesk on ja üksi maha jäänud, see loodab Jumala peale ja jääb kindlasti anumistesse ja palvetesse ööd ja päevad.
\par 6 Aga kes elab lihahimus, see on elusalt surnud.
\par 7 Ja sedagi pane südamele, et nad oleksid laitmatud.
\par 8 Aga kui keegi enese omaste ja iseäranis kodakondsete eest ei kanna hoolt, see on usu ära salanud ja on pahem kui uskmatu.
\par 9 Leskede hulka loetagu ainult niisugune, kes on vähemalt kuuskümmend aastat vana ja on olnud ühe mehe naine,
\par 10 ja kes on tunnustatud heade tegude poolest, kui ta on kasvatanud lapsi, kui ta on vastu võtnud võõraid, kui ta on pühade jalgu pesnud, kui ta on aidanud hädasolijaid, kui ta on püüdnud teha kõiksugust head.
\par 11 Kuid nooremaid leski keeldu vastu võtmast, sest kui iharus neid Kristuse kiuste vallutab, siis nad tahavad minna mehele,
\par 12 ja neid tuleb süüdistada selles, et nad on hüljanud esimese usu.
\par 13 Kui nad on pealegi jõude, harjuvad nad käima mööda maju, ja siis ei ole nad mitte ainult jõude, vaid ka lobisejad ning uudishimulikud ja räägivad sobimatuid asju.
\par 14 Ma tahan siis, et nooremad läheksid mehele ja tooksid lapsi ilmale, valitseksid majatalitusi ega annaks vastasele põhjust solvamiseks.
\par 15 Sest mõned ongi juba pöördunud saatana jälgedesse.
\par 16 Kui kellelgi usklikul mehel või naisel on leski omaste seas, siis nad hooldagu neid ja ärgu saagu need koormaks kogudusele, et see võiks hooldada tõelisi leski.

\section*{Koguduse vanemaist}

\par 17 Vanemaid, kes hästi juhatavad kogudusi, peetagu kahekordse austuse vääriliseks, eriti neid, kes töötavad sõnas ja õpetuses.
\par 18 Sest Kiri ütleb: „Ära seo kinni härja suud, kui ta pahmast tallab!” ja: „Töötegija on oma palga väärt!”
\par 19 Vanema vastu tõstetud kaebust ära võta kuulda, kui ei ole juures kaht või kolme tunnistajat.
\par 20 Kes pattu teevad, neid noomi kõikide ees, et ka teistel oleks hirm.
\par 21 Ma vannutan sind Jumala ja Kristuse Jeesuse ja äravalitud inglite ees, et sa seda pead silmas ilma eelarvamiseta ega tee midagi erapoolikult.
\par 22 Ära ole kärmas kellelegi käsi pea peale panema ja ära tee ennast teiste pattude osaliseks. Pea ennast puhtana.
\par 23 Ära joo enam ainult vett, vaid tarvita pisut viina oma kõhu pärast ja oma sageda põdemise pärast.
\par 24 Mõnede inimeste patud on kohe avalikud ja on kohe hukkamõistetavad, aga mõnede omad saavad tagantjärele ilmsiks.
\par 25 Samuti on ka head teod kohe avalikud, ja needki, mis seda ei ole, ei või jääda varjule.


\chapter{6}

\section*{Orjadest}

\par 1 Kõik need, kes on orjaikke all, pidagu oma isandaid kõige austuse väärt, et Jumala nime ja õpetust ei pilgataks.
\par 2 Aga kellel on usklikud isandad, ärgu pidagu neid halvemaks, sellepärast et nad on vennad, vaid orjaku neid veel parema meelega, sest et nad on usklikud ja armsad ning harrastavad heategevust. Seda õpeta ja manitse.

\section*{Valeõpetajaist}

\par 3 Kui keegi õpetab teisiti ega hoia meie Issanda Jeesuse Kristuse tervete sõnade ja õpetuse poole, mis viib jumalakartusele,
\par 4 siis on see iseennast täis ega saa aru millestki, vaid on haige küsimuste ja vaidlemiste poolest, millest tõuseb kadedust, riidu, laimu, kurje kahtlusi,
\par 5 alalisi tülitsusi inimeste vahel, kelle mõistus on rikutud ja kellel on tõde läinud käest, kes peavad jumalakartust tuluallikaks.
\section*{Tõelisest rikkusest}

\par 6 Ent jumalakartus on suur tuluallikas, kui ta on ühendatud rahulolemisega.
\par 7 Ei ole me ju midagi toonud maailma, seepärast me ei või midagi siit ära viia.
\par 8 Aga kui meil on peatoidust ja ihukatet, siis olgem sellega rahul.
\par 9 Ent kes tahavad rikkaks saada, need langevad kiusatusse ja võrku ja paljudesse rumalaisse ja kahjulikesse himudesse, mis suruvad inimesed alla hukatusse ja hävitusse.
\par 10 Sest rahaahnus on kõigi kurjade asjade juur; raha ihaldades on mitmed ära eksinud usust ja on iseendile valmistanud palju torkavat valu.

\section*{Isiklik manitsus Timoteosele}

\par 11 Aga sina, Jumala inimene, põgene selle eest! Taotle õigust, jumalakartust, usku, armastust, kannatlikkust, tasadust!
\par 12 Võitle head usuvõitlust, hakka kinni igavesest elust; selleks sa oled kutsutud ja oled tunnistanud head tunnistust paljude tunnistajate ees.
\par 13 Ma käsin sind Jumala ees, kes kõik teeb elavaks, ja Kristuse Jeesuse ees, kes Pontius Pilaatuse ees andis hea tunnistuse,
\par 14 et sa peaksid käsku kinni veatult ja laitmatult kuni meie Issanda Jeesuse Kristuse ilmumiseni,
\par 15 mille omal ajal toob nähtavale õnnis ja ainus vägev valitseja, kuningate Kuningas ja kõigi isandate Isand,
\par 16 kellel üksi on surematus, kes elab ligipääsmatus valguses, keda ükski inimene ei ole näinud ega võigi näha! Temale olgu austus ja igavene vägi! Aamen.
\par 17 Neid, kes on rikkad selles maailmas, manitse mitte olla suurelised ega loota kaduva rikkuse peale, vaid elava Jumala peale, kes meile annab kõike rohkesti tarvituseks,
\par 18 teha head ja rikkaks saada heade tegude poolest, olla helde käega ning jagada teistele
\par 19 ja koguda enestele raudvara heaks aluseks tuleviku jaoks, et saavutada tõelist elu.
\par 20 Oh Timoteos! Hoia enda hooleks antud vara ning pöördu ära kõlvatuist tühjest juttudest ja valenimelise tunnetuse väidetest,
\par 21 mida mõningad kiidavad enestel olevat ning on ära eksinud usust. Arm teiega!




\end{document}