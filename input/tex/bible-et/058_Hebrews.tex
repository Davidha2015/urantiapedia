\begin{document}

\title{Kiri heebrealastele}

\chapter{1}

\section*{Sissejuhatus}

\par 1 Kui Jumal muiste mitu korda ja mitmel kombel vanemaile oli rääkinud prohvetite kaudu,
\par 2 siis on ta neil viimseil päevil meile rääkinud Poja kaudu, kelle ta on pannud kõigi asjade pärijaks, kelle läbi ta on teinud maailmaajastud,
\par 3 kes, olles tema aupaistus ning ta olemuse kuju ja kandes kõike oma vägeva sõnaga, pärast seda kui ta iseenese läbi oli toimetanud meie pattude puhastuse, istus Ausuuruse paremale käele kõrguses
\par 4 ning sai seda ülemaks inglitest, mida kõrgema nime ta on pärinud kui nemad.

\section*{Jumala Poeg on ülem kui inglid}

\par 5 Sest missugusele inglile on ta iganes öelnud: „Sina oled mu Poeg, täna ma sünnitasin sind!”, või jälle: „Mina tahan olla temale Isaks ja tema peab olema mulle Pojaks!”?
\par 6 Ent sellest, kui ta jälle toob esmasündinu maailma, ütleb ta: „Ka kõik Jumala inglid kummardagu teda!”
\par 7 Ent inglitest ta ütleb küll: „Ta teeb oma inglid tuulteks ja oma teenijad tuleleegiks!”
\par 8 Aga Pojale ta ütleb: „Su aujärg, Jumal, on ikka ja igavesti, su valitsuskepp on õigluse kepp!
\par 9 Sa armastad õigust ja vihkad ülekohut; sellepärast on Jumal, sinu Jumal, sind võidnud rõõmuõliga enam kui su kaaslasi!”
\par 10 Ja: „Sina, Issand, panid maale aluse ja taevad on su käte tööd!
\par 11 Need hävivad, aga sina püsid; nad kõik kuluvad nagu kuub
\par 12 ja sa rullid nad kokku nagu mantli, nagu kuue, ja nad muudetakse; aga sina oled seesama ja sinu aastad ei lõpe!”
\par 13 Või missugusele inglile on ta kunagi öelnud: „Istu mu paremale käele, kuni ma panen su vaenlased su jalgealuseks järiks!”
\par 14 Eks nad kõik ole teenijad vaimud, läkitatud abistama neid, kes õndsuse pärivad?


\chapter{2}

\section*{Jumala Poeg on ülem kui inglid}

\par 1 Sellepärast tuleb meil palju hoolsamini silmas pidada seda, mida oleme kuulnud, et me kuidagi mööda ei libiseks.
\par 2 Sest kui inglite kaudu räägitud sõna püsis kindlana ja kõik üleastumine ja sõnakuulmatus sai oma õige palga,
\par 3 kuidas me siis võime pääseda pakku, kui me ei hooli nii suurest päästest, mis oma kuulutuse alguse sai Issandalt ja mille kinnitasid meisse need, kes seda kuulsid,
\par 4 kui Jumal ühes nendega andis tunnistust tunnustähtedega ja imedega ja mõnesuguste vägevate tegudega ja Püha Vaimu jagamistega oma tahte järgi.
\par 5 Sest mitte inglitele ei alistanud ta tulevast maailma, millest me räägime,
\par 6 vaid keegi on kuskil tunnistanud, öeldes: „Mis on inimene, et sa temale mõtled, või inimese poeg, et sa tema eest hoolitsed?
\par 7 Sa tegid ta pisut alamaks inglitest; sa ehtisid teda au ja austusega;
\par 8 kõik sa panid tema jalge alla!” Sest alistades kõik temale, ei jätnud ta midagi temale alistamata. Aga nüüd me ei näe veel kõike temale alistatuna.
\par 9 Ent teda, kes sai üürikeseks ajaks alamaks inglitest, Jeesust, me näeme tema surma kannatamise pärast au ja austusega ehitud, et ta Jumala armust igaühe eest maitseks surma.
\par 10 Sest temale, kelle pärast ja kelle läbi on kõik, sobis, et ta selle, kes palju lapsi ausse viib, nende õndsuse ülemjuhi, teeks täiuslikuks kannatuste kaudu.
\par 11 Sest niihästi pühitseja kui pühitsetavad on kõik üht Isa; sel põhjusel tema ka ei häbene neid nimetada vendadeks,
\par 12 öeldes: „Ma tahan su nime kuulutada oma vendadele, keset kogudust ma tahan laulda sulle kiitust!”
\par 13 Ja jälle: „Ma loodan tema peale!”; ja jälle: „Vaata, siin olen mina ja lapsed, keda Jumal mulle on andnud!”
\par 14 Et nüüd lapsed on liha ja vere osalised, siis temagi sai otse samal viisil osa sellest, et ta surma läbi kaotaks selle, kelle võimu all oli surm, see on kuradi,
\par 15 ja vabastaks need, kes surma kartusest olid kogu eluaja kinni orjapõlves.
\par 16 Sest ega ta ometi hoolitse inglite eest, vaid ta hoolitseb Aabrahami soo eest!
\par 17 Sellepärast pidi ta kõiges saama vendade sarnaseks, et ta oleks halastaja ja ustav ülempreester teenistuses Jumala ees rahva pattude lepitamiseks.
\par 18 Sest selles, milles ta on kannatanud kiusatud olles, võib ta aidata neid, keda kiusatakse.


\chapter{3}

\section*{Kristus on ülem kui Mooses}

\par 1 Sellepärast, pühad vennad, taevase kutsumise osalised, pange tähele meie tunnistuse apostlit ja ülempreestrit Jeesust,
\par 2 kes on ustav oma ametisseseadjale nagu Mooseski kogu tema kojas.
\par 3 Sest tema on Moosesest niivõrd suurema au vääriline, kuivõrd maja valmistajal on suurem au kui majal.
\par 4 Sest igal majal on oma valmistaja; aga kes kõik on valmistanud, see on Jumal.
\par 5 Ja Mooses oli küll ustav sulasena kogu tema kojas tunnistuseks sellest, mida pidi räägitama,
\par 6 aga Kristus kui Poeg on tema koja üle; ja tema koda oleme meie, kui me julguse ning lootuse kiitlemise hoiame otsani kindla.
\par 7 Sellepärast nõnda nagu Püha Vaim ütleb: „Täna, kui te kuulete tema häält,
\par 8 ärge paadutage oma südameid, nõnda nagu sündis nurina ajal, kiusatuse päeval kõrbes,
\par 9 kus teie isad mind kiusasid ning katsusid, kuigi nad olid näinud minu tegusid nelikümmend aastat,
\par 10 mispärast ma vihastusin selle sugupõlve peale ning ütlesin: nemad eksivad alati oma südamega, ent minu teid nad ei õppinud tundma;
\par 11 sellepärast ma vandusin oma vihas: nad ei pääse mitte minu hingamisse!”

\section*{Usk on vajalik lõpuaja eel}

\par 12 Katsuge, vennad, et kellelgi teie seast ei oleks kuri uskmatu süda, mis loobub elavast Jumalast,
\par 13 vaid manitsege üksteist iga päev, niikaua kui veel öeldakse „täna”, et ükski teie seast ei paaduks patu pettusest!
\par 14 Sest me oleme saanud Kristuse osalisiks, kui me vaid otsani kindla hoiame esimese usulootuse.
\par 15 Kui öeldakse: „Täna, kui te kuulete tema häält, ärge paadutage oma südameid, nõnda nagu sündis nurina ajal”, -
\par 16 kes siis olid need kuuljad, kes hakkasid nurisema? Kas need ei olnud kõik, kes olid väljunud Egiptusest Moosese juhatusel?
\par 17 Aga kellele ta oli vihane nelikümmend aastat? Kas mitte neile, kes olid pattu teinud, kelle kehad langesid kõrbes?
\par 18 Kellele ta siis vandus, et nad ei pääse mitte tema hingamisse? Kas mitte neile, kes olid sõnakuulmatud?
\par 19 Ja nii me näeme, et nad oma uskmatuse pärast ei võinud sisse saada.


\chapter{4}

\section*{Jumala hingamisse jõudmisest}

\par 1 Siis kartkem, et niikaua kui tõotus pääseda tema hingamisse on täitumata, keegi teist kuidagi ei osutuks hiljaks jäänuks.
\par 2 Sest rõõmusõnum on kuulutatud meile nõnda nagu neilegi. Ent kuuldud sõnast ei olnud neil kasu, sest see ei imbunud usu läbi neisse, kes seda kuulsid.
\par 3 Sest meie, kes usume, saame hingamisse, nõnda nagu tema on öelnud: „Siis ma vandusin oma vihas: nad ei pääse mitte minu hingamisse!”, ehk küll tema teod olid valmis maailma asutamisest alates.
\par 4 Sest kuskil on ta seitsmendast päevast öelnud nõnda: „Ja Jumal hingas seitsmendal päeval kõigist oma tegudest”;
\par 5 ja siin jälle: „Nad ei pääse mitte minu hingamisse!”
\par 6 Et nüüd asi nii on, et mõned pääsevad sinna ja et need, kellele rõõmusõnumit esmalt kuulutati, ei ole sinna sisse saanud sõnakuulmatuse pärast,
\par 7 siis ta määrab uuesti ühe päeva „tänapäevaks”, öeldes Taaveti kaudu nii pika aja pärast, nagu on enne öeldud: „Täna, kui te kuulete tema häält, ärge paadutage oma südameid!”
\par 8 Sest kui Joosua nad oleks viinud hingamisse, siis ta ei räägiks teisest, pärastisest päevast.
\par 9 Seega on Jumala rahval hingamisaeg veel ees.
\par 10 Sest kes tema hingamisse on pääsenud, see hingab ka ise oma tegudest, otsekui Jumal oma tegudest.
\par 11 Siis olgem agarad sellesse hingamisse minema, et keegi ei langeks sõnakuulmatusse samal eeskujul.
\par 12 Sest Jumala sõna on elav ja vägev ja teravam kui ükski kaheterane mõõk ning tungib läbi, kuni ta lõhestab hinge ja vaimu, liikmed ja üdi, ja on südame meelsuse ja mõtete hindaja;
\par 13 ja ükski loodu ei ole temale nähtamatu, vaid kõik on alasti ja paljastatud tema silma ees; ja temale tuleb meil aru anda.

\section*{Jeesus kui suur ülempreester}

\par 14 Et meil nüüd on suur ülempreester Jeesus, Jumala Poeg, kes taevad on läbinud, siis pidagem kinni tunnistusest!
\par 15 Sest meil pole niisugune ülempreester, kellel ei oleks kaastundmust meie nõtrustega, vaid kes kõiges on kiusatud otsekui meie, siiski ilma patuta.
\par 16 Läki siis julgusega armuaujärje ette, et me saaksime halastust ja leiaksime armu oma abiks õigeks ajaks.



\chapter{5}

\section*{Jeesus kui suur ülempreester}

\par 1 Sest iga ülempreester, kes võetakse inimeste seast, seatakse inimeste heaks toimetama teenistust Jumala ees, tooma ande ja ohvreid pattude eest;
\par 2 ja tema võib säästa mõistmatuid ning eksijaid; on ta ju ise ka nõtruse sees.
\par 3 Ja selle nõtruse pärast peab tema niisamuti nagu rahva eest, ka iseenese eest ohvrit tooma pattude pärast.
\par 4 Ka ei võta keegi enesele ise seda au, vaid Jumala kutsel nagu Aarongi.
\par 5 Nõnda Kristuski ei võtnud ise enesele au saada ülempreestriks, vaid ta sai selle temalt, kes temale ütles: „Sina oled mu Poeg, täna ma sünnitasin sind!”,
\par 6 nõnda nagu ta teiseski kohas ütleb: „Sina oled preester igavesti Melkisedeki korra järgi!”
\par 7 Tema ohverdas oma liha päevil palveid ja anumisi suure hüüdmise ja silmaveega sellele, kes teda võis päästa surmast, ja tema palvet kuuldi ta jumalakartuse pärast.
\par 8 Ja ehk ta oligi Poeg, õppis ta sõnakuulmist sellest, mida ta kannatas.
\par 9 Ja kui ta oli saanud täielikuks, sai tema igavese õndsuse alustajaks kõigile, kes on temale sõnakuulelikud,
\par 10 ja Jumal nimetas tema ülempreestriks Melkisedeki korra järgi.

\section*{Arengu vajadusest õpetuses}

\par 11 Sellest oleks meil palju rääkida ning seda on raske sõnadega seletada, sest te olete läinud tuimaks kuulma.
\par 12 Sest teie, kes aja poolest peaksite olema õpetajad, vajate jälle, et teile õpetataks Jumala sõnade esimesi algeid, ja olete saanud nende aruliseks, kellele läheb tarvis piima, aga mitte tahket rooga.
\par 13 Sest igaüks, kes veel tarvitab piima, ei saa õieti aru õiguse sõnast, sest ta on alles laps.
\par 14 Aga täisealiste jaoks on tahke roog, nende jaoks, kelle meeled vilumuse tõttu on harjunud vahet tegema hea ja kurja vahel.


\chapter{6}

\section*{Arengu vajadusest õpetuses}

\par 1 Sellepärast jätkem kõrvale algõpetus Kristusest ja püüdkem täiuslikkusele ning ärgem hakakem uuesti rajama alust pöördumisele surnud tegudest ja usule Jumalasse,
\par 2 õpetusele ristimisest ja käte pealepanemisest, surnute ülestõusmisest ja igavesest hukatusest.
\par 3 Ja nii me teemegi, kui Jumal lubab.
\par 4 Sest on võimatu neid, kes korra olid valgustatud ning maitsesid taevalist andi ja said osa Pühast Vaimust,
\par 5 ja maitsesid Jumala head sõna ja tulevase maailmaajastu väge
\par 6 ja siis ära taganesid, uuesti tuua meeleparandamisele, sellepärast et nad iseeneste kahjuks Jumala Poja risti löövad ja naeruks panevad.
\par 7 Sest maa, mis joob enesesse rikkalikult sadanud vihma ja kasvatab tarvilist vilja neile, kes seda harivad, saab õnnistuse Jumalalt;
\par 8 kuid maa, mis kannab kibuvitsu ja ohakaid, on kõlbmatu ja lähedal needusele, ja lõppeks ta põletatakse.
\par 9 Aga, armsad, teie suhtes me usume seda, mis on parem ja soodus õndsusele, kuigi me räägime nõnda.
\par 10 Sest Jumal ei ole ülekohtune, et ta unustaks ära teie teod ja armastuse, mida olete osutanud tema nimele, kui te pühadele abiks olite ja veelgi olete.
\par 11 Aga me ihaldame seda, et te igaüks osutaksite sama indu täieliku lootuse säilitamiseks otsani;
\par 12 et te ei läheks loiuks, vaid järgiksite neid, kes usu ja pika meele tõttu pärivad selle, mis on tõotatud.

\section*{Jumala tõotus annab lootuse}

\par 13 Sest kui Jumal Aabrahamile andis tõotuse, vandus ta iseenese juures, sest ei olnud kedagi suuremat, kelle juures ta oleks võinud vanduda,
\par 14 ning ütles: „Tõesti, õnnistusega ma õnnistan sind ja teen sind väga paljuks!”
\par 15 Ja nii Aabraham ootas kannatlikult ning sai tõotuse kätte.
\par 16 Sest inimesed vannuvad suurema juures ja vanne on neile kinnituseks kui kõige vasturääkimise lõpetus.
\par 17 Nii on Jumal selleks, et tõotuse pärijaile veel selgemini näidata oma nõu kindlust, vannet tarvitanud vahendiks,
\par 18 et kahe kõikumatu asja läbi, milles Jumalal on võimatu valetada, oleks vägev julgustus meil, kes oleme usaldanud kinni haarata eelolevast lootusest,
\par 19 mis meile on otsekui hinge ankur, kindel ja tugev ning ulatub sissepoole eesriide taha,
\par 20 kuhu eeljooksjana meie heaks on läinud sisse Jeesus, kui ta sai ülempreestriks igavesti Melkisedeki korra järgi.


\chapter{7}

\section*{Jeesus kui Melkisedeki sarnane ülempreester}

\par 1 Sest see Melkisedek - Saalemi kuningas, kõige kõrgema Jumala preester, kes tuli vastu Aabrahamile, kui see oli koju minemas pärast võitu kuningate üle, ja õnnistas teda,
\par 2 kellele ka Aabraham jagas kümnist kõigest saagist ja kes on esmalt, nagu teda tõlgitsetakse, „Õiguse kuningas”, aga siis ka Saalemi kuningas, see on „Rahu kuningas”,
\par 3 kes on isata, emata, ilma suguvõsata, ilma päevade alguseta ja ilma eluotsata, ent kes on sarnastatud Jumala Pojaga - jääb preestriks igavesti.
\par 4 Vaadake siis, kui suur on see, kellele ka peavanem Aabraham andis saagist kümnist.
\par 5 On ju neil Leevi poegadelgi, kes saavad preestriameti, käsk võtta kümnist käsuõpetuse järgi rahva, see on oma vendade käest, ehk küll nemadki on lähtunud Aabrahami niudeist.
\par 6 Kuid see, kes ei põlvnenudki nendest, võttis kümnise Aabrahami käest ja õnnistas teda, kellel olid tõotused.
\par 7 Ent vasturääkimata on selge, et alam saab õnnistuse ülemalt.
\par 8 Ja siin võtavad kümnist surelikud inimesed, aga seal see, kellest tunnistatakse, et ta elab.
\par 9 Ja Aabrahami kaudu on, niiöelda, Leevigi, kes võtab kümnist, pidanud kümnist maksma;
\par 10 sest ta oli veel isa niudeis, kui Melkisedek temale vastu tuli.

\section*{Kristuse preestriamet on ülem kui juutide preestriamet}

\par 11 Kui nüüd täiuslikkus oleks leviitide preestriameti kaudu olnud saavutatud - sest sellele on rajatud rahva käsuõpetus - mis vajadust siis veel oli, et teistsugune preester tõusis Melkisedeki korra järgi ega nimetatud teda Aaroni korra järgi?
\par 12 Sest preestriameti muutmisega sünnib paratamatult ka käsuõpetuse muutus.
\par 13 Sest see, kellest seda öeldakse, on pärit teisest suguharust, kellest veel ükski ei ole saanud altari juurde.
\par 14 On ju teada, et meie Issand on tärganud Juudast, kelle suguharule Mooses ei ole midagi kõnelnud preestriametist.
\par 15 Ja asi saab veel palju selgemaks, kui tõuseb teistsugune Melkisedeki sarnane preester,
\par 16 kes ei ole lihaliku käsusõna käsku mööda saanud preestriks, vaid hääbumatu elu väge mööda.
\par 17 Sest tunnistatakse: „Sina oled preester igavesti Melkisedeki korra järgi!”
\par 18 Endise käsusõna tühistus sünnib ju selle tõttu, et see oli jõuetu ja kasutu -
\par 19 ei ole ju käsk midagi viinud täiuslikkusele - ent asemele seatakse parem lootus, mille kaudu me tuleme Jumala ligi.
\par 20 Ja niivõrd kui see ei tekkinud ilma vandeta - sest need on saanud preestreiks ilma vandeta,
\par 21 tema aga vandega selle seadmisel, kes temale ütles: „Issand on vandunud ja tema ei kahetse seda mitte: Sina oled preester igavesti!” -
\par 22 sedavõrd on palju parem ka leping, mille käemeheks on saanud Jeesus.
\par 23 Oli ju neid palju, kes said preestreiks, sellepärast et surm neid keelas jäämast.
\par 24 Aga temal on selle tõttu, et ta jääb igavesti, preestriamet, mis ei lähe kellegi teise kätte.
\par 25 Sellepärast võib ta ka täielikult päästa need, kes tulevad tema läbi Jumala juurde, elades aina selleks, et kosta nende eest.
\par 26 Niisugune ülempreester peabki ju meil olema: püha, veatu, laitmatu, eraldatud patustest ja kõrgemale saanud kui taevad,
\par 27 kellel ei ole mitte iga päev vaja nõnda nagu ülempreestril esiti tuua ohvreid enese pattude eest ja siis rahva pattude eest; sest seda on ta igaks puhuks teinud, kui ta iseenese tõi ohvriks.
\par 28 Sest käsk seab nõdrad inimesed ülempreestreiks, aga vande sõna, mis on hilisem käsuõpetusest, seab Poja, kes on täiuslikuks saanud igaveseks.


\chapter{8}

\section*{Kristuse preestriamet ja uus leping}

\par 1 Aga peaasi selles, millest me kõneleme, on see: meil on selline ülempreester, kes istub Ausuuruse istme paremal käel taevas,
\par 2 pühamu ja tõelise telgi ametitalitaja, mille on ehitanud Jumal ja mitte inimene.
\par 3 Sest iga ülempreester seatakse ohverdama ande ja ohvreid, mispärast on paratamata vajalik, et tal oleks midagi ning seda, mida ohverdada.
\par 4 Kui ta nüüd oleks maa peal, ei ta siis olekski preester, sest on olemas neid, kes käsuõptuse järgi ohverdavad ande
\par 5 ja kes teenivad selles, mis on taevase pühamu kuju ja vari, nagu Moosesele ilmutati, kui ta hakkas telki valmistama. Sest nii öeldi talle: „Vaata, et sa kõik teed selle eeskuju järgi, mis sulle mäel näidati!”
\par 6 Aga nüüd on tema saanud seda kallima ameti, mida parema lepingu vahemees ta ka on, mis on rajatud parematele tõotustele.
\par 7 Sest kui esimene leping oleks olnud laitmatu, ei siis oleks otsitud aset teisele.
\par 8 Sest neid laites ütleb prohvet neile: „Vaata, päevad tulevad, ütleb Issand, mil ma teen Iisraeli sooga ja Juuda sooga uue lepingu:
\par 9 mitte selle lepingu sarnase, mille ma tegin nende vanematega sel päeval, mil ma võtsin nad kättpidi, et viia nad välja Egiptusemaalt - sest nad ei jäänud minu lepingusse ja nii minagi ei hoolinud neist, ütleb Issand -
\par 10 vaid leping, mille ma teen Iisraeli sooga pärast neid päevi, ütleb Issand, on niisugune: ma panen neile meelde oma käsud ja kirjutan need neile südamesse; siis ma olen neile Jumalaks ja nemad on mulle rahvaks!
\par 11 Siis ei õpeta keegi enam oma kaaskodanikku ega vend oma venda, öeldes: „Tunne Issandat!”, sest nad kõik tunnevad mind, niihästi nende pisukesed kui nende suured.
\par 12 Sest ma olen armuline nende ülekohtustele tegudele ega tuleta enam meelde nende patte!”
\par 13 Kui ta nimetab uut, siis on ta esimese tunnistanud vananenuks. Aga mis vananeb ja iganeb, on hävimisele ligi!


\chapter{9}

\section*{Vana lepingu telk uue lepingu telgi eeskujuks}

\par 1 Esimesel lepingul olid ju ka oma jumalateenistuse korraldused ja oma maine pühamu.
\par 2 Sest enne valmistati esimene telk, milles olid küünlajalg ja laud ja vaateleivad laual; seda kutsutakse „pühaks paigaks.”
\par 3 Ent teise eesriide taga oli telk, mida kutsutakse „kõige pühamaks paigaks.”
\par 4 Seal oli kuldne suitsutusaltar ja seaduselaegas, mis oli üleni kullaga karratud; selle sees oli kuldkruus mannaga ja Aaroni kepp, mis oli õitsenud, ja seaduselauad.
\par 5 Ent laeka peal olid auhiilguse keerubid, kes varjasid lepituskaant. Sellest kõigest ei ole nüüd tarvis üksikasjalikult rääkida.
\par 6 Et need kõik nõnda on korrastatud, siis lähevad eestelki alati preestrid, kes toimetavad jumalateenistusi,
\par 7 aga teise läheb korra aastas üksnes ülempreester, mitte ilma vereta, mille ta ohverdab oma ja rahva eksimuste eest.
\par 8 Sel kombel näitab Püha Vaim, et tee pühasse paika veel ei ole avatud, niikaua kui esimene telk veel on olemas.
\par 9 See on praeguse aja kohta võrdum, mille kohaselt ohverdatakse ande ja ohvreid, mis ei suuda teha südametunnistuse poolest täiuslikuks seda, kes toimetab teenistust,
\par 10 vaid mis samuti nagu toidud ja joogid ja mitmesugused pesemised on ainult liha korraldused, kehtivad seniks, kui kätte jõuab uuendatud korra aeg.

\section*{Kristuse ohvri lunastav vägi}

\par 11 Aga Kristus, kui tema tuli tulevaste heade andide ülempreestrina, suurema ja täiuslikuma telgiga, mis ei ole kätega tehtud, see on, mis ei ole pärit sellest loomisest,
\par 12 ja ka mitte sikkude ega vasikate verega, vaid iseenese verega, läks ükskord sinna pühasse paika ning saavutas igavese lunastuse.
\par 13 Sest kui sikkude ja härgade veri ja lehma tuhk, mis riputatakse nende peale, kes on rüvetunud, pühitseb liha puhtuseks,
\par 14 kui palju enam Kristuse veri, kes igavese Vaimu läbi iseenese veatuna ohverdas Jumalale, puhastab meie südametunnistuse surnud tegudest teenima elavat Jumalat.
\par 15 Ja sellepärast on tema uue lepingu vahemees, et pärast tema surma, mis oli lunastuseks esimese lepingu ajal olnud üleastumistest, need, kes on kutsutud, kätte saaksid igavese pärandi.
\par 16 Sest seal, kus on olemas testament, on hädavajalik, et tõendataks ka testamenditegija surm.
\par 17 Sest testament hakkab kehtima pärast surma; sest see ei ole kunagi jõus, niikaua kui testamenditegija on elus.
\par 18 Sellepärast ei ole ka esimest lepingut pühitsetud ilma vereta.
\par 19 Sest kui Mooses kõigele rahvale oli ette lugenud kogu käsu käsuõpetuse järgi, võttis ta vasikate ja sikkude vere ühes vee ja punase villa ja iisopiga ja piserdas seda niihästi raamatu enese kui ka kõige rahva peale
\par 20 ning ütles: „See on selle lepingu veri, mille Jumal on seadnud teie jaoks!”
\par 21 Ja samuti ta piserdas verd ka telgi ja kõigi jumalateenistuseriistade peale.
\par 22 Ja peaaegu kõik puhastatakse verega käsuõpetuse järgi, ja ilma verd valamata ei ole andeksandmist olemas.

\section*{Kristuse ohvri ainulaadsus}

\par 23 Nii on siis tarvis, et taevalike asjade kujud puhastatakse sel teel, aga taevasi asju endid puhastatakse paremate ohvritega kui need.
\par 24 Sest Kristus ei läinud kätega tehtud pühamusse kui tõelise püha paiga kujusse, vaid taevasse enesesse, et nüüd ilmuda Jumala palge ette meie eest,
\par 25 aga mitte ennast mitu korda ohverdama, nõnda nagu ülempreester igal aastal läks pühasse paika võõra verega,
\par 26 sest muidu oleks ta pidanud mitu korda kannatama maailma rajamisest alates; aga nüüd on ta maailmaajastute lõpul korra ilmunud, et oma ohvriga kaotada patt.
\par 27 Ja otsekui inimestele on määratud kord surra, aga pärast seda kohus,
\par 28 nõnda ka Kristus, kes üks kord tõi enese ohvriks, et ära võtta paljude patud, ilmub teist korda ilma patuta neile, kes teda ootavad õndsuseks.


\chapter{10}

\section*{Kristuse ohver on mõjuvam kui kõik teised}

\par 1 Käsk, olles ainult tulevaste hüvede vari, aga mitte asjade täiskuju ise, ei või iialgi samade igaaastaste ohvritega, mida ühtelugu ohverdatakse, teha täiuslikuks nende toojaid,
\par 2 sest eks muidu oleksid ju ohvrite toomised lõppenud, sellepärast et neil, kes toimetavad jumalateenistust, ei oleks enam mingit patust südametunnistust, kui nad kord on puhtaks tehtud.
\par 3 Ent ohvritega tuletatakse iga aasta patte meelde.
\par 4 Sest on võimatu, et härgade ja sikkude veri võib patud ära võtta.
\par 5 Sellepärast ta maailma tulles ütleb: „Ohvrit ega andi sa ei tahtnud, aga sa valmistasid mulle ihu:
\par 6 põletus- ja patuohvrid ei olnud sulle meelepärast!
\par 7 Siis ma ütlesin: vaata, ma tulen - rullraamatus on minust kirjutatud - tegema su tahtmist, oh Jumal!”
\par 8 Kuna ta eespool ütleb: „Ohvreid ja ande ning põletus- ja patuohvreid sa ei tahtnud ega olegi need su meele järgi, kuigi neid käsu järgi tuuakse”,
\par 9 siis ütleb ta pärast: „Vaata, ma tulen tegema su tahtmist!” Sellega ta tunnistab tühjaks esimese ohvri, et panna kehtima teine.
\par 10 Selles tahtmises oleme meiegi pühitsetud Jeesuse Kristuse ihu ohvriga ühel hoobil.
\par 11 Ja iga preester seisab päevast päeva oma ametis ja ohverdab sageli samu ohvreid, mis iialgi ei või patte ära võtta;
\par 12 aga tema, kui ta üheainsa ohvri oli viinud pattude eest, on istunud jäädavalt Jumala paremale käele
\par 13 ja ootab nüüd ainult, kuni ta vaenlased pannakse tema jalgealuseks järiks.
\par 14 Sest ühe ohvriga on ta jäädavalt teinud täiuslikeks need, keda pühitsetakse.
\par 15 Seda tunnistab meile ju ka Püha Vaim; sest kui ta oli öelnud:
\par 16 „Niisugune on leping, mille ma teen nendega pärast neid päevi, ütleb Issand: ma panen neile südamesse oma käsud ja kirjutan need neile meelde
\par 17 ega mäleta enam nende patte ja ülekohtutegusid!”
\par 18 Aga kus need on andeks antud, seal ei ole enam tarvis ohvrit nende eest.

\section*{Julgustus ja hoiatus}

\par 19 Nii on meil siis, vennad, Jeesuse veres julgus sissepääsuks pühasse paika;
\par 20 ja selle on ta meile avanud uueks ja elavaks teeks eesriide, see on oma liha kaudu;
\par 21 ja meil on suur preester Jumala koja üle.
\par 22 Astugem siis ligi tõelise südamega, täielikus usus, südame poolest piserdamisega puhastatud kurjast südametunnistusest,
\par 23 ja ihu poolest pestud puhta veega, pidagem kõikumata kinni lootuse tunnistusest. Sest ustav on see, kes seda on tõotanud.
\par 24 Ja pidagem üksteist silmas õhutamiseks armastusele ja headele tegudele.
\par 25 Ning ärgem jätkem maha oma koguduse kooskäimisi, nõnda nagu mõnel on kombeks, vaid manitsegem üksteist, ja seda enam, et näete selle päeva lähenevat.
\par 26 Sest kui me meelega pattu teeme, pärast seda kui oleme saanud tõe tunnetuse, siis ei jää enam mingit ohvrit üle pattude eest;
\par 27 vaid mingi hirmus kohtu ootamine ja äge tuli, mis tahab ära süüa vastupanijad.
\par 28 Kui keegi rikub Moosese käsku, siis peab ta ilma armuta surema kahe või kolme tunnistaja sõna peale,
\par 29 kui palju hirmsama nuhtluse te arvate siis ära teeninud olevat selle, kes Jumala Poega jalgadega tallab ega pea pühaks lepingu verd, millega ta on pühitsetud, ja teotab armu Vaimu?
\par 30 Me tunneme ju seda, kes ütleb: „Minu käes on kättemaks, mina tasun kätte!” ja taas: „Issand mõistab kohut oma rahvale!”
\par 31 Hirmus on elava Jumala kätte langeda!

\section*{Jätkuv kannatajate julgustamine}

\par 32 Aga tuletage meelde endisi päevi, mil teie, olles valgustatud, talusite palju kannatamiste võitlust,
\par 33 kord nii, et te teotamiste ja viletsuste läbi olite tehtud maailma naeruks, kord jälle nii, et saite nende osalisteks, kellel oli sama saatus.
\par 34 Sest te olete ühes vangidega kannatanud ja rõõmuga vastu võtnud oma vara riisumise, teades, et teil on parem ja jäädav vara taevas.
\par 35 Ärge siis heitke ära oma julgust, mis saab suure palga.
\par 36 Sest kannatlikkust läheb teile tarvis, et teie, tehes Jumala tahtmist, saaksite kätte tõotuse.
\par 37 Sest „veel üsna pisut, pisut aega, siis tuleb see, kes peab tulema ega viivita mitte!
\par 38 Aga minu õige peab usust elama, ja kui ta hakkab kõhklema, siis ei ole minu hingel temast head meelt!”
\par 39 Ent meie ei ole need, kes kõhklevad hukatuseks, vaid need, kes usuvad hinge päästmiseks!


\chapter{11}

\section*{Usukangelased}

\par 1 Aga usk on kindel usaldus selle vastu, mida oodatakse, ja veendumus selles, mida ei nähta.
\par 2 Selle kohta on ju vanad saanud tunnistuse.
\par 3 Usu kaudu me tunneme, et maailmad on valmistatud Jumala sõna läbi, nii et mitte sellest, mida võib näha, ei ole tekkinud see, mida nähakse.
\par 4 Usu läbi tõi Aabel Jumalale parema ohvri kui Kain, ja usu kaudu ta sai tunnistuse, et tema oli õige, kui Jumal andis tunnistuse tema andide kohta, ja usu kaudu ta räägib veel pärast surmagi.
\par 5 Usu läbi võeti ära Eenok, et ta surma ei näeks, ja teda ei leitud enam, sest Jumal oli ta ära võtnud. Sest enne äravõttu oli ta juba saanud tunnistuse, et ta oli Jumalale meelepärane.
\par 6 Aga ilma usuta on võimatu olla meelepärane; sest kes Jumala juurde tuleb, peab uskuma, et tema on olemas ja et ta annab palga neile, kes teda otsivad.
\par 7 Usu kaudu sai Noa ilmutuse sellest, mida veel ei olnud näha, ja ehitas pühas kartuses laeva oma perekonna päästmiseks; ja selle kaudu ta mõistis hukka maailma ning sai selle õiguse pärijaks, mis tuleb usust.
\par 8 Usu läbi kuulas Aabraham sõna, kui teda kutsuti välja minema paika, mille ta pidi saama pärandiks; ja ta läks välja ilma teadmata, kuhu ta läheb.
\par 9 Usu läbi asus ta võõrana elama tõotusemaale otsekui võõrale maale ja elas telkides ühes Iisaki ja Jaakobiga, kes olid sama tõotuse kaaspärijad;
\par 10 sest ta ootas linna, millel on alused ja mille ehitaja ning valmistaja on Jumal.
\par 11 Usu läbi sai isegi Saara rammu soo rajamiseks ja pealegi üle oma ea, sellepärast et ta pidas ustavaks teda, kes tõotuse oli andnud.
\par 12 Sellepärast sündis ka ühest ja pealegi elatanud mehest nii palju nagu taevas tähti ja nagu mere ääres liiva, mida ei saa ära lugeda.
\par 13 Usus need kõik surid ega saanud tõotusi kätte, vaid nägid neid kaugelt ja teretasid neid ning tunnistasid endid olevat võõrad ja majalised maa peal.
\par 14 Sest need, kes seda ütlevad, näitavad, et nad otsivad kodumaad.
\par 15 Ja kui nad oleksid mõelnud seda kodumaad, kust nad olid väljunud, siis oleks neil ju olnud juhust minna tagasi.
\par 16 Ent nüüd nad igatsevad paremat kodumaad, see on taevast; seepärast ei ole Jumalal neist häbi ja ta laseb ennast nimetada nende Jumalaks; sest ta on neile valmistanud linna.
\par 17 Usu läbi viis Aabraham ohvriks Iisaki, kui teda katsuti, ja ohverdas selle ainusündinu, kui ta oli saanud tõotused
\par 18 ning kui temale oli öeldud: „Iisakist loetakse sinu sugu!”
\par 19 Sest ta mõtles, et Jumal on võimeline ka surnuist üles äratama; selle mõistukujuks ta sai tema ka tagasi.
\par 20 Usu läbi õnnistas ka Iisak Jaakobit ja Eesavit tulevaste asjade suhtes.
\par 21 Usu läbi õnnistas Jaakob surres kumbagi Joosepi poegadest ja palvetas oma kepi pära najal.
\par 22 Usu läbi tuletas Joosep oma eluotsal meelde Iisraeli laste väljumist ja tegi korralduse oma luude kohta.
\par 23 Usu läbi varjasid vanemad Moosest pärast ta sündimist kolm kuud, sest nad nägid ta ilusa lapse olevat ega kartnud kuninga käsku.
\par 24 Usu läbi keeldus Mooses, kui ta suureks sai, kandmast vaarao tütrepoja nime,
\par 25 pidades paremaks ühes Jumala rahvaga kannatada viletsust kui üürikest aega nautida paturõõmu,
\par 26 arvates Kristuse teotust suuremaks rikkuseks kui Egiptuse aarded; sest ta tõstis oma silmad tasu poole.
\par 27 Usu läbi ta jättis maha Egiptuse ega kartnud kuninga viha, sest otsekui nähes teda, kes on nähtamatu, püsis ta kindlana.
\par 28 Usu läbi ta pidas paasapüha ja vere piserdust, et esmasündinute surmaja ei puudutaks neid.
\par 29 Usu läbi nad läksid Punasest merest läbi nagu kuiva maad mööda, mida ka egiptlased katsusid teha, aga uppusid.
\par 30 Usu läbi langesid Jeeriko müürid, kui seitse päeva oli käidud nende ümber.
\par 31 Usu läbi ei saanud hoor Raahab hukka ühes sõnakuulmatutega, kui ta salakuulajad oli rahuga vastu võtnud.
\par 32 Ja mida ma veel ütleksin? Mul puudub aeg jutustada Giideonist ja Baarakist ja Simsonist ja Jeftast ja Taavetist ja Saamuelist ja prohvetitest,
\par 33 kes usu läbi võitsid ära kuningriigid, viisid täide õiguse, said kätte tõotused, sulgesid lõukoerte lõuad,
\par 34 kustutasid tule väe, pääsesid ära mõõgatera eest, said nõtrusest tugevaks, said vägevaks sõjas, ajasid pakku võõraste sõjahulgad.
\par 35 Naised said tagasi oma surnud ülestõusmise läbi. Teised lasksid ennast piinata ega võtnud vastu vabastamist, et saaksid parema ülestõusmise osalisiks.
\par 36 Teised said kogeda pilget ja rooska, peale selle ahelaid ja vangitorni.
\par 37 Neid on kividega visatud, piinatud, lõhki saetud, mõõgaga surmatud; nad on lambanahas ja kitsenahas käinud ühest kohast teise, puuduses, viletsuses, kurja kannatades, nemad,
\par 38 kelle väärt maailm ei olnud. Nad eksisid ümber kõrbetes ja mägedel ja koobastes ja maa-aukudes.
\par 39 Ja need kõik, olles usu läbi saanud tunnistuse, ei saavutanud ometi mitte seda, mis oli tõotatud;
\par 40 sest Jumal oli meile varunud midagi paremat, et nad ilma meieta ei pääseks täielikkusesse.


\chapter{12}

\section*{Jeesus usu alustaja ja täidesaatja}

\par 1 Sellepärast meiegi, et nii suur pilv tunnistajaid on meie ümber, pangem maha kõik koorem ja meid nii hõlpsasti takerdav patt ning jookskem kannatlikkusega meile määratud võidujooksmist,
\par 2 vaadates usu alustajale ja täidesaatjale Jeesusele, kes risti kannatas temale oodatava rõõmu asemel, häbist hoolimata, ja on istunud Jumala aujärje paremale käele.

\section*{Karistuse eesmärk}

\par 3 Kujutlege ometi teda, kes niisugust vastuhakkamist enese vastu on saanud kannatada patustelt, et te ei väsiks ega läheks araks oma hinges.
\par 4 Teie ei ole veel vereni vastu pannud võideldes patuga
\par 5 ja olete unustanud manitsuse, mis teile otsekui lastele ütleb: „Mu poeg, ära põlga Issanda karistust ja ära saa araks, kui tema sind noomib!
\par 6 Sest keda Issand armastab, seda ta karistab; ta peksab igat poega, keda ta vastu võtab.”
\par 7 Taluge karistust kasvatuseks: Jumal kohtleb teid kui poegi; sest milline poeg on see, keda isa ei karista?
\par 8 Ent kui te olete ilma karistuseta, millest kõik on osa saanud, siis te olete värdjad ja mitte pojad.
\par 9 Pealegi on meile meie lihased isad olnud karistajaiks ja me oleme neid kartnud; kas me palju enam ei tahaks alistuda vaimude Isale ning elada?
\par 10 Sest nemad on küll meid mõnd päeva oma heaksarvamist mööda karistanud, aga tema karistab meid tõesti selle hea otstarbega, et me saaksime osa tema pühadusest.
\par 11 Aga mingi karistus, kui see on käes, ei näi olevat rõõmuks, vaid on kurbuseks; aga pärast toob see neile, kes sellega on õpetatud, õiguse rahuvilja.
\par 12 Seepärast ajage jälle sirgu lõdvad käed ja halvatud põlved
\par 13 ja õgvendage teerajad oma jalgadele, et see, mis lonkav, ei nikastuks, vaid ennemini saaks terveks.

\section*{Hoiatus Jumala armu hülgamise eest}

\par 14 Nõudke rahu kõikidega ja pühitsust; ilma selleta ei saa ükski Issandat näha.
\par 15 Ja pange tähele, et keegi ei jääks ilma Jumala armust, et ükski viha juur ei kasvaks üles ega tooks tüli ja selle läbi paljud ei rüvetuks,
\par 16 et keegi ei oleks hooraja või roojane, kes nagu Eesav üheainsa kõhutäie eest andis käest esmasünniõiguse.
\par 17 Sest te teate, kuidas ta pärast küll tahtis pärida õnnistust, aga siis hüljati kui kõlvatu; sest ta ei leidnud mahti meeleparanduseks, ehk ta seda küll silmapisaratega otsis.

\section*{Maise Siinai ja taevase Siioni erinevusest}

\par 18 Sest te ei ole astunud käega katsutava ja tules põleva mäe ligi, ei pimeduse ega pilkase pimeduse, ei maru,
\par 19 ei pasuna helina ega niisuguse sõnade hääle ligi, mille kuuljad palusid, et neile sõna enam ei räägitaks;
\par 20 sest nemad ei suutnud taluda seda keeldu: „Isegi kui üks loom peaks puutuma mäe külge, siis visatagu ta kividega surnuks!”;
\par 21 ja nii hirmus oli see nähtus, et Mooses ütles: „Ma olen ehmunud ja värisen!”;
\par 22 vaid te olete tulnud Siioni mäe ligi ja elava Jumala linna, taevase Jeruusalemma juurde ja lugematu hulga inglite juurde,
\par 23 ja esmasündinute piduliku kogu ning koguduse juurde, kes on kirja pandud taevasse, ja Jumala, kõikide kohtumõistja juurde, ja õigete vaimude juurde, kes on saanud täiuslikeks,
\par 24 ja uue lepingu vahemehe Jeesuse juurde ja piserdamisvere juurde, mis paremini räägib kui Aabeli veri.
\par 25 Katsuge, et teiegi teda ei hülga, kes räägib! Sest kui ei ole pääsenud need, kes maapealse kõneleja hülgasid, kui palju vähem pääseme meie, kui me pöördume ära temast, kes on taevastest,
\par 26 kelle hääl tookord pani kõikuma maa, aga kes nüüd on tõotanud ja öelnud: „Veel kord ma panen värisema mitte ainult maa, vaid ka taeva!”
\par 27 Ent „veel kord” näitab, et see, mis kõigub, peab muutuma, sest ta on loodud, et püsiks see, mida ei saa kõigutada.
\par 28 Seepärast, saades kuningriigi, mis ei kõigu, olgem tänulikud ja teenigem seega Jumalat tema meelt mööda pelglikkuse ja aukartusega.
\par 29 Sest meie Jumal on hävitav tuli!


\chapter{13}

\section*{Mitmesugused manitsused}

\par 1 Vendade armastus jäägu püsima!
\par 2 Ärge unustage külalislahkust, sest selle läbi on mõningad ilma teadmata külalisteks vastu võtnud ingleid.
\par 3 Pidage meeles vange, otsekui oleksite kaasvangid, ja vaevatuid, sest ka teie olete veel ihus.
\par 4 Abielu olgu igapidi au sees ja abieluvoodi rüvetamata! Sest Jumal nuhtleb hoorajaid ja abielurikkujaid.
\par 5 Teie meelelaad olgu rahaahnuseta; olge rahul sellega, mis teil on. Sest tema on öelnud: „Ma ei hülga sind ega jäta sind maha!”
\par 6 Nii võime siis julgesti öelda: „Issand on minu abimees, ei ma karda! Mis võib inimene mulle teha?”

\section*{Juhatajaile oldagu ustavad ja sõnakuulelikud}

\par 7 Mõelge oma juhatajaile, kes teile on rääkinud Jumala sõna, pannes tähele nende eluotsa järgige nende usku!
\par 8 Jeesus Kristus on seesama, eile ja täna ja igavesti!
\par 9 Ärge laske endid vintsutada mitmelaadiliste ja võõraste õpetustega; sest hea on, et süda kinnitatakse armu läbi, aga mitte toitudega, millest ei ole saanud kasu need, kes nende määruste järgi käivad.
\par 10 Meil on ohvrialtar, kust toidust võtta ei ole luba neil, kes telgis peavad teenistust.
\par 11 Sest nende loomade kehad, kelle vere ülempreester viib ohvriks patu eest pühasse paika, põletatakse väljaspool leeri.
\par 12 Seepärast on ka Jeesus, et pühitseda rahvast oma vere läbi, kannatanud väljaspool väravat.
\par 13 Siis mingem nüüd tema juurde väljapoole leeri ning kandkem tema teotust.
\par 14 Sest meil pole siin jäädavat linna, vaid meie otsime tulevast.
\par 15 Viigem siis nüüd tema kaudu alati Jumalale kiitusohvrit, see on nende huulte vilja, kes tunnistavad tema nime.
\par 16 Ärge unustage head teha ja osadust pidada, sest sellised ohvrid on Jumala meele järgi.
\par 17 Olge sõnakuulelikud oma juhatajaile ja alistuge neile; sest nemad valvavad teie hingi nõnda nagu need, kellel tuleb aru anda, et nad teeksid seda rõõmuga ja mitte ohates; sest teil ei oleks sellest kasu.

\section*{Manitsused ja õnnistussõnad jumalagajätuks}

\par 18 Palvetage meie eest! Sest me oleme veendunud selles, et meil on hea südametunnistus ja me tahame kõiges käituda kaunisti.
\par 19 Eriti ma palun seda teha selleks, et mind rutemini antaks teile tagasi.
\par 20 Aga rahu Jumal, kes surnuist on üles toonud igavese lepingu vere läbi lammaste suure karjase, meie Issanda Jeesuse,
\par 21 varustagu teid kõige heaga tegema tema tahtmist ja saatku teie sees korda seda, mis on tema meelt mööda, Jeesuse Kristuse läbi, kellele olgu austus ajastute ajastuteni! Aamen.
\par 22 Aga ma palun teid, vennad, võtke heaks see manitsussõna! Ma olen teile ju lühidalt kirjutanud.
\par 23 Teadke, et meie vend Timoteos on jälle vaba. Temaga ühes, kui ta varsti tuleb, saan ma teid näha.
\par 24 Tervitage kõiki oma juhatajaid ja kõiki pühasid! Teid tervitavad Itaalia vennad.
\par 25 Arm olgu teie kõikidega!






\end{document}