\begin{document}

\title{Peetruse Esimene kiri}

\chapter{1}

\section*{Tervitus}

\par 1 Peetrus, Jeesuse Kristuse apostel, Pontoses, Galaatias, Kapadookias, Aasias ja Bitüünias hajuvil asuvaile valitud majalistele,
\par 2 Jumala Isa etteteadmist mööda Vaimu pühitsemise kaudu sõnakuulmiseks ja piserdamiseks Jeesuse Kristuse verega! Arm ja rahu rohkenegu teile!

\section*{Tänu Jumalale lunastuse lootuse eest}

\par 3 Kiidetud olgu Jumal ja meie Issanda Jeesuse Kristuse Isa, kes oma suurt halastust mööda meid on uuesti sünnitanud elavaks lootuseks Jeesuse Kristuse ülestõusmise läbi surnuist
\par 4 kadumatu ja rüvetamatu ja närtsimatu pärandi saamiseks, mis taevas on tallel teie jaoks,
\par 5 mida Jumala väega valvatakse usu läbi õndsuse jaoks, mis on valmis ilmsiks tulema viimsel ajal!
\par 6 Sellest te väga rõõmutsete, kuigi te nüüd vajaduse korral pisut kurvastute mõnesugustes kiusatustes,
\par 7 et teie usk, kui see on läbi katsutud, leitaks kallihinnalisem olevat kullast, mis kaob ja siiski tules läbi katsutakse, ja oleks teile kiituseks ja hiilguseks ja auks Jeesuse Kristuse ilmumisel,
\par 8 keda te küll ei ole näinud, ja siiski armastate, kellesse te nüüd usute, kuigi te teda ei näe, ja rõõmutsete ütlematu ning väga kalli rõõmuga,
\par 9 kes saavutate usu eesmärgi, hingede õndsuse.
\par 10 Seda õndsust on uurinud ja juurelnud prohvetid, kes on ennustanud teile määratud armust,
\par 11 juureldes seda, mis ja millise aja kohta teateid andis Kristuse Vaim nende sees, kui ta ette ilmutas Kristust tabavaid kannatusi ja neile järgnevaid austusi.
\par 12 Neile ilmutati, et mitte neile endile, vaid teile oli kasuks see, mis nüüd on kuulutatud teile nende kaudu, kes teile taevast läkitatud Püha Vaimu läbi on toonud evangeeliumi; ja sellesse igatsevad inglidki vaadata.

\section*{Üleskutse pühaks eluks}

\par 13 Sellepärast vöötage oma meele niuded ja olge kained; ja lootke täiesti armu peale, mida teile pakutakse Jeesuse Kristuse ilmumises.
\par 14 Kui sõnakuuulelikkuse lapsed ärge anduge endistele himudele, nagu siis kui teil puudus õige teadmine;
\par 15 vaid nõnda nagu see, kes teid on kutsunud, on püha, saage ka ise pühaks kõigis eluviisides;
\par 16 sest on kirjutatud: „Olge pühad, sest mina olen püha!”
\par 17 Ja et te appi hüüate teda kui Isa, kes isikule vaatamata mõistab kohut igaühe tegu mööda, siis veetke kartuses oma majaliseks olemise aeg,
\par 18 teades, et teid ei ole kaduvaga, hõbeda või kullaga lunastatud teie tühiseist esivanemailt päritud eluviisidest,
\par 19 vaid Kristuse kui veatu ja laitmatu talle kalli verega,
\par 20 kes küll oli ette määratud enne maailma rajamist, aga aegade lõpul on saanud ilmsiks teie pärast,
\par 21 kes tema kaudu olete usklikud Jumalasse, kes tema surnuist üles äratas ja temale andis au, nii et teie usk on ka lootus Jumala peale.
\par 22 Et te olete oma hinged puhastanud tõe sõnakuulmises teesklematuks vennalikuks armastuseks, siis armastage üksteist innukalt ja südamest,
\par 23 sest et olete uuesti sünnitatud mitte kaduvast, vaid kadumatust seemnest, Jumala elava ning püsiva sõna läbi.
\par 24 Sest „kõik liha on nagu rohi ja kõik tema hiilgus nagu rohu õieke; rohi kuivab ära ja õieke variseb maha,
\par 25 aga Issanda sõna püsib igavesti!” Ja see on see sõna, mis teile rõõmusõnumina on kuulutatud.



\chapter{2}

\section*{Elav kivi ja püha rahvas}

\par 1 Siis pange nüüd maha kõik kurjus ja kõik kavalus ja silmakirjatsus ja kadedus ja kõik keelekandmine
\par 2 ning himustage otsekui vastsündinud lapsed vaimset täispiima, et te selle läbi võiksite kasvada õndsuseks,
\par 3 kui te olete maitsnud, et Issand on helde, tulles tema juurde
\par 4 kui elava kivi juurde, mis küll inimeste poolt on põlatud, aga Jumala juures on ära valitud ja väga kallis,
\par 5 ja teiegi ehituge üles kui elavad kivid vaimulikuks kojaks ja pühaks preesterkonnaks ohverdama vaimulikke ohvreid, mis on Jumalale meelepärased Jeesuse Kristuse kaudu.
\par 6 Sellepärast ongi lugeda Kirjas: „Vaata, mina panen Siionisse valitud kalli nurgakivi, ja kes temasse usub, see ei satu häbisse!”
\par 7 Teile nüüd, kes usute, on ta kallihinnaline; aga uskmatuile on ta „kivi, mille hooneehitajad põlgasid, kuid mis sai nurgakiviks”,
\par 8 ja „komistuskiviks ja pahanduskaljuks”. Nemad „tõukavad tema vastu”, sest nad ei kuule sõna, ja selleks on need ka pandud.
\par 9 Teie aga olete „valitud sugu, kuninglik preesterkond, püha rahvas, omandrahvas, et te kuulutaksite tema aulisi tegusid”, kes teid on kutsunud pimedusest oma imelise valguse juurde,
\par 10 teie, kes muiste olite „mitterahvas”, aga nüüd olete Jumala rahvas; kes „ei olnud armu saanud”, aga nüüd olete armu saanud!
\par 11 Armsad, ma manitsen teid kui „võõraid ja majalisi” hoiduda lihalikest himudest, mis sõdivad hinge vastu,
\par 12 ja elada vooruslikku elu paganate keskel, et nad selles, mille pärast nad teist kui kurjategijaist paha räägivad, teie häid tegusid nähes annaksid Jumalale austust „katsumispäeval”!

\section*{Ühiskondlikule korrale alistumisest}

\par 13 Alistuge kõigele inimeste seatud korrale Issanda pärast, olgu kuningale kui ülemale valitsejale,
\par 14 või pealikuile kui neile, keda tema on läkitanud kurjategijaile nuhtluseks, aga heategijaile kiituseks.
\par 15 Sest nõnda on Jumala tahtmine, et teie, tehes head, tõkestaksite mõistmatute inimeste rumalust -
\par 16 kui vabad ja mitte kui need, kellele vabadus on kurjuse katteks, vaid kui Jumala sulased.
\par 17 Austage kõiki, armastage vendi, kartke Jumalat, austage kuningat!

\section*{Alistumise ja teenimise mõttest}

\par 18 Sulased, olge kõige kartusega allaheitlikud oma isandaile, mitte ainult headele ja leplikele, vaid ka tigedaile.
\par 19 Sest see on arm, kui keegi südametunnistuse pärast Jumala ees talub viletsusi, kannatades süütult.
\par 20 Sest mis kiitus see on, kui teid pekstakse patu pärast ning te jääte kannatlikuks? Aga kui te head tehes ja kurja kannatades püsite kannatlikena, siis on see arm Jumala juures.
\par 21 Selleks te ju olete kutsutud, sest et Kristuski kannatas teie eest ja jättis teile eeskuju, et te käiksite tema jälgedes,
\par 22 „kes ei teinud pattu ja kelle suust ei leitud pettust”,
\par 23 kes ei sõimanud vastu, kui teda sõimati, kes ei ähvardanud, kui ta kannatas, vaid jättis selle hooleks, kes mõistab kohut õigesti,
\par 24 kes meie patud ise kandis üles ristipuule oma ihus, et meie pattudele ära sureksime ja elaksime õigusele, kelle vermete tõttu te olete terveks saanud.
\par 25 Sest te olite „nagu eksijad lambad”, aga nüüd te olete pöördunud oma hingede Karjase ja Hooldaja poole.


\chapter{3}

\section*{Manitsusi abielunaistele ja -meestele}

\par 1 Samuti teie, naised, olge allaheitlikud oma meestele, et ka need, kes ei ole sõnakuulelikud sõnale, naiste eluviiside läbi võidetaks ilma sõnata usule,
\par 2 kui nad vaatavad, kuidas te elate jumalakartuses puhast elu.
\par 3 Teie ehteks ärgu olgu välispidine juuste palmitsemine, kulla ümberriputamine ega toredate riietega riietumine,
\par 4 vaid varjul olev südame inimene, tasase ja vaikse vaimu kadumatuses, mis on kallis Jumala ees.
\par 5 Sest nõnda on ka ennevanasti pühad naised, kes lootsid Jumala peale, endid ehtinud ja olnud allaheitlikud oma meestele,
\par 6 nõnda nagu Saara oli sõnakuulelik Aabrahamile ja „hüüdis teda isandaks”, ja tema lasteks te olete saanud, kui te teete head ega karda ühtki hirmutust.
\par 7 Samuti teie, mehed, elage naisega targasti kui nõrgema astjaga ja osutage neile austust, sest nad on elu armuanni kaaspärijad nagu teiegi, et teie palved poleks tagajärjeta.

\section*{Kristlikust ühiselust}

\par 8 Aga viimaks, olge kõik üksmeelsed, kaastundlikud, vennaarmastajad, halastajad, alandlikud!
\par 9 Ärge tasuge kurja kurjaga ega sõimu sõimuga, vaid vastupidi, õnnistage, sest te teate, et olete kutsutud pärima õnnistust!
\par 10 Sest „kes armastab elu ja näha häid päevi, see hoidku oma keelt kurja eest ja oma huuli pettust rääkimast;
\par 11 ta pöördugu ära kurjast ning tehku head; ta otsigu rahu ja nõudku seda taga.
\par 12 Sest Issanda silmad on õigete poole ja tema kõrvad nende palvetamise poole; aga Issanda pale on pahategijate vastu!”

\section*{Süütu kannatamine ja Kristuse eeskuju}

\par 13 Ja kes peaks teile kurja tegema, kui te heategemises olete agarad?
\par 14 Aga kuigi te kannataksite õiguse pärast, siis olete ometi õndsad. „Ärge kartke nende hirmutamist ja ärge kohkuge”,
\par 15 vaid pühitsege Issandat Kristust oma südameis, olles alati valmis andma vastust igaühele, kes teilt nõuab seletust lootuse kohta, mis teis on.
\par 16 Aga vastake tasaduse ja kartusega, ja teil olgu hea südametunnistus, et jääksid häbisse selles, mida teist paha räägivad need, kes teotavad teie head eluviisi Kristuses.
\par 17 Sest parem on, kui nõnda on Jumala tahtmine, head tehes kannatada kui paha tehes.
\par 18 On ju ka Kristus kord surnud pattude pärast, õige ülekohtuste eest, et ta meid juhiks Jumala juurde, kui ta küll surmati liha poolest, kuid elavaks tehti vaimu poolest,
\par 19 kelles ta ka läks ära ja kuulutas vangis olevaile vaimudele,
\par 20 kes muiste ei kuulanud sõna, kui Jumala pikk meel ootas Noa päevil, kui ehitati laeva, millel mõned üksikud, see on kaheksa hinge, päästeti läbi vee.
\par 21 Sellele võrdluskujule vastavalt päästab teidki nüüd vesi ristimisena, mitte kui liha rüveduse kõrvaldamine, vaid kui hea südametunnistuse nõudmine Jumala ees Jeesuse Kristuse ülestõusmise kaudu,
\par 22 kes on läinud taevasse ja on Jumala paremal käel ja kellele on alistatud inglid ja võimud ja väed.



\chapter{4}

\section*{Kannatused lõpetavad patutegemise}

\par 1 Et nüüd Kristus on kannatanud lihas, siis relvastuge teiegi sama meelega; sest kes lihas on kannatanud, see on lakanud pattu tegemast,
\par 2 nõnda et ta aega, mis tal veel on elada lihas, enam ei ela inimeste himude järgi, vaid Jumala tahtmise järgi.
\par 3 Sest sellest on küllalt, et teil möödunud aeg kulus ära paganate tahte täitmiseks, kui te elasite lodevuses, himudes, viinajoomises, öistes olenguis, jootudes ja jõledais ebajumalateenistustes.
\par 4 Nemad võõrastavad seda, et teie nendega ühes ei jookse samas lodeva elu voolus, ning pilkavad teid.
\par 5 Nad peavad aru andma sellele, kes on valmis kohut mõistma elavate ja surnute üle.
\par 6 Sest selleks on surnuile kuulutatud evangeelium, et kohus nende üle oleks mõistetud nagu inimeste üle lihas ja nad elaksid nagu Jumal vaimus.
\par 7 Aga kõigi asjade lõpp on lähedal! Olge siis mõistlikud ja kained palveiks!
\par 8 Kõigepealt olgu teie armastus südamlik üksteise vastu, sest „armastus katab pattude hulga”.
\par 9 Olge külalislahked üksteise vastu ilma nurisemata.
\par 10 Teenigu igaüks teist selle andega, mille ta on saanud, kui Jumala mõnesuguse armu head majapidajad.
\par 11 Kui keegi räägib, siis ta rääkigu kui Jumala sõnu; kui keegi teenib, siis ta teenigu kui jõust, mida Jumal annab, et kõiges austataks Jumalat Jeesuse Kristuse läbi, kelle on au ja vägi ajastute ajastuteni! Aamen.

\section*{Kristuse pärast kannataja tasu}

\par 12 Armsad, ärge pidage võõraks tulekuumust enestes, mis teile on saanud katsumiseks, otsekui sünniks teile midagi võõrast,
\par 13 vaid sedamööda nagu teil on osa Kristuse kannatamistest, olge rõõmsad, et võiksite rõõmustuda ning hõisata tema au ilmumisel.
\par 14 Kui teid solvatakse Kristuse nime pärast, siis te olete õndsad, sest au ja Jumala Vaim hingab teie peal.
\par 15 Ärgu siis keegi teist kannatagu mõrtsukana või vargana või kurjategijana või salakaebajana.
\par 16 Aga kui keegi kannatab ristiinimesena, siis ta ärgu häbenegu, vaid andku Jumalale austust selle nimega.
\par 17 Sest aeg on kohtul alata Jumala kojast; kui see aga algab kõigepealt meist, missugune ots ootab neid, kes ei ole sõnakuulelikud Jumala evangeeliumile?
\par 18 Ja kui õige vaevalt pääseb, kuhu siis saab õel ja patune?
\par 19 Seepärast usaldagu need, kes Jumala tahtmise järgi kannatavad, oma hinged ustava Looja hooleks, tehes head.


\chapter{5}

\section*{Manitsus kogudusevanemaile}

\par 1 Kogudusevanemaid teie seas ma nüüd manitsen kui kaasvanem ja Kristuse kannatuste tunnistaja, kes ka olen tulevase au osaline, mis peab ilmuma:
\par 2 hoidke teile hoida antud Jumala karja mitte sundusest, vaid vabast tahtest Jumala meele järgi, mitte alatu kasu tõttu, vaid innust,
\par 3 mitte isandaina valitsedes kogudusi kui liisuosi, vaid olles karjale eeskujuks;
\par 4 siis te saate, kui Ülimkarjane ilmub, närtsimatu aupärja!
\par 5 Samuti teie, nooremad, alistuge vanemaile ja riietuge kõik alandlikkusega üksteise vastu; sest „Jumal paneb suurelistele vastu, aga alandlikele ta annab armu!”

\section*{Manitsus alanduda Jumala vägeva käe alla}

\par 6 Siis alanduge Jumala vägeva käe alla, et ta teid ülendaks omal ajal!
\par 7 Heitke kõik oma mure tema peale, sest tema peab hoolt teie eest!
\par 8 Olge kained, valvake, sest teie vastane, kurat, käib ümber nagu möirgaja lõukoer otsides, keda neelata!
\par 9 Tema vastu seiske kindlad usus ning teadke, et teie vendadel maailmas tuleb neidsamu kannatusi täielikult kanda.
\par 10 Aga kõige armu Jumal ise, kes meid on kutsunud oma igavesse ausse Kristuses, võtku teid, kui te pisut olete kannatanud, valmistada, kinnitada, teha tugevaks ja rajada!
\par 11 Tema päralt on võimus ajastute ajastuteni! Aamen.
\par 12 Silvaanuse, minu arvates ustava venna kaudu olen teile lühidalt kirjutanud, manitsedes ja tunnistades, et see on tõeline Jumala arm, milles te olete.
\par 13 Teid tervitab kaasvalitud kogudus Baabülonis ja Markus, mu poeg.
\par 14 Tervitage üksteist armastuse suudlusega! Rahu teile kõikidele, kes olete Kristuses!




\end{document}