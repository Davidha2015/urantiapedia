\begin{document}

\title{Peetruse Teine kiri}

\chapter{1}

\section*{Tervitus}

\par 1 Siimon Peetrus, Jeesuse Kristuse sulane ja apostel, neile, kes on saanud niisama kalli usu kui meiegi meie Jumala ja Õnnistegija Jeesuse Kristuse õiguses.
\par 2 Armu ja rahu lisatagu teile rohkesti Jumala ja Jeesuse, meie Issanda tunnetuses!

\section*{Kristlaste eesõigused ja kutsumus}

\par 3 Nõnda nagu tema jumalik vägi meile on annetanud kõik, mida vajatakse eluks ja jumalakartuseks, tema tunnetuse kaudu, kes meid on kutsunud oma au ja vooruslikkusega,
\par 4 millega on meile kingitud kallid ja suurimad tõotused, et te nende läbi saaksite osa jumalikust loomust, kui te olete põgenenud ära hukkumisest, mis on maailmas himude tõttu;
\par 5 siis kandke sellesama pärast kõikepidi hoolt ja osutage oma usus vooruslikkust ja vooruslikkuses tunnetust,
\par 6 tunnetuses taltumust, taltumuses kannatlikkust, kannatlikkuses jumalakartust,
\par 7 jumalakartuses vennaarmastust ja vennaarmastuses armastust kõikide vastu.
\par 8 Sest kui teil seda kõike on ja veel lisatakse, siis ei lase see teid olla viitsimatud ja viljatud meie Issanda Jeesuse Kristuse tunnetamises.
\par 9 Aga kellel see kõik puudub, on pime, lühinägelik, ja on unustanud puhastuse endistest pattudest.
\par 10 Sellepärast, vennad, püüdke seda enam teha kindlaks oma kutsumine ja valik; sest kui te seda teete, siis te ei komista iialgi.
\par 11 Sest nõnda avaneb teile rohkel mõõdul sissepääs meie Issanda ja Õnnistegija Jeesuse Kristuse igavesse kuningriiki.

\section*{Peetrus aimab oma peatset surma}

\par 12 Sellepärast ma tahan alati teile seda meelde tuletada, ehk te küll seda teate ja olete kindlasti rajatud tõele, mis teil on käes.
\par 13 Ma pean õigeks, niikaua kui ma selles ihulikus telgis olen, teid meeldetuletuse kaudu virgutada,
\par 14 sest et ma tean, et ma varsti maha panen oma telgi, nagu ka meie Issand Jeesus Kristus mulle on avaldanud.
\par 15 Ent ma tahan ka hoolt kanda, et teil pärast minu lahkumist alati oleks võimalik seda meelde tuletada.

\section*{Peetruse isiklik tunnistus ja prohvetite kuulutus}

\par 16 Sest me ei ole Issanda Jeesuse Kristuse väge ja tulemist teile teatavaks tehes mitte järginud kavalasti sepitsetud tühje jutte, vaid me oleme ise oma silmaga näinud tema suurt aulisust.
\par 17 Sest tema sai Jumalalt Isalt au ja austust kõige kõrgemalt ault, kui temale kostis niisugune hääl: „See on minu armas Poeg, kellest mul on hea meel!”
\par 18 Ja selle hääle me kuulsime taevast tulevat, olles ühes temaga pühal mäel.
\par 19 Ja meil on veel kindlam prohvetlik sõna, ja te teete hästi, et te seda tähele panete kui küünalt, mis paistab pimedas paigas, kuni päev jõuab kätte ning koidutäht tõuseb teie südameis.
\par 20 Ja see olgu teile kõigepealt teada, et ükski prohvetikuulutus Kirjas ei ole omavolilise seletuse jaoks.
\par 21 Sest inimese tahtel ei ole iialgi ühtki prohveti kuulutust esile toodud, vaid Pühast Vaimust kantuina on inimesed rääkinud, saades seda Jumala käest.


\chapter{2}

\section*{Valeprohvetid ja jumalakartmatu elu}

\par 1 Aga valeprohveteidki tõusis rahva seas, nõnda nagu teiegi seas saab olema lahkõpetajaid, kes salamahti toovad sisse hukutavaid valeõpetusi, salgavad ära peremehegi, kes nad on ostnud, ja tõmbavad eneste peale äkilise hukatuse.
\par 2 Ja paljud järgivad nende lodevat elu, kelle pärast pilgatakse tõe teed;
\par 3 ja ahnuses kasutavad nad teid petlike sõnadega. Nende kohus on juba ammu valvas ja nende hukatus ei tuku.
\par 4 Sest kui Jumal ei säästnud ingleidki, kes pattu tegid, vaid tõukas nad põrgupimeduse soppidesse säilitatavaiks kohtu jaoks,
\par 5 ja kui ta ei säästnud muistset maailma, vaid varjas ainult õigusekuulutajat Noad, neid kaheksakesi, kui ta laskis tulla veeuputuse jumalatute maailma peale;
\par 6 ja mõistis ka hukka Soodoma ja Gomorra linnad, põletades need tuhaks ja pannes hirmu täheks neile, kes tulevikus on jumalatud,
\par 7 ja päästis Loti, kes oli õige ja keda üleannetud inimesed piinasid oma kiimalise käitumisega -
\par 8 sest elades nende keskel see õige vaevas päevast päeva oma vaga hinge, nähes ja kuuldes ülekohtusi tegusid -
\par 9 nii Issand teab päästa jumalakartlikke kiusatusest, aga ülekohtusi säilitada kohtupäevaks, et neid nuhelda,
\par 10 iseäranis aga neid, kes liha järgi elavad roppudes himudes ja põlgavad Issanda valitsust. Need on jultunud, iseennast täis, ei karda pilgata aukandjaid vaime,
\par 11 kuna inglidki, kuigi nad küll võimuselt ja väelt on suuremad, ühtki pilkavat otsust ei lausu nende kohta Issanda ees.
\par 12 Ent need on nagu mõistmatud loomad, kes on oma loomu poolest sündinud püütavaiks ja hukutavaiks; nad pilkavad seda, mida nad ei tea, ja hukkuvad ka oma hukus,
\par 13 saades kätte ülekohtu palga; nad peavad lõbuks igapäevast laiutamist; nad on ebapuhtuse ja häbi laigud, kes oma pettuste varal lõbutsevad, kui nad pidutsevad ühes teiega;
\par 14 nende silmad õhkuvad abielurikkumisest ja nende iha ei lakka; nemad ahvatlevad kinnitamatuid hingi, neil on ahnusega vilunud süda, nad on needuse lapsed!
\par 15 Nad on hüljanud otsese tee, nad on ära eksinud, käies Bileami, Beori poja teed, kellele ülekohtu palk oli armas,
\par 16 aga kes oma üleastumise pärast sai karistada: keeletu koormakandja loom rääkis inimese häälega ja takistas prohveti jõledust.
\par 17 Need on ilma veeta lätted, marutuulest aetavad udupilved, kellele on varutud pilkane pimedus.
\par 18 Sest rääkides tühje ülerinna sõnu, ahvatlevad nad lodevate lihahimudega neid, kes vaevalt olid pääsenud pakku eksiteel käijate seast;
\par 19 nad tõotavad neile vabadust, olles ise kaduvuse orjad, sest kelle võidetud keegi on, selle ori ta on.
\par 20 Sest kui nad Issanda ja Õnnistegija Jeesuse Kristuse tunnetuse läbi pääsesid pakku maailma ebapuhtusest, kuid jälle segunevad sellega ja jäävad alla, siis on nende viimne lugu saanud pahemaks kui esimene.
\par 21 Sest neil oleks parem, et õiguse tee oleks jäänud neile tundmatuks, kui seda tundes ära pöörduda käsust, mis neile on antud.
\par 22 Neile on tulnud kätte, mis tõeline vanasõna ütleb: „Koer pöördub tagasi oma okse juurde” ja „pestud emis läheb porisse püherdama!”


\chapter{3}

\section*{Issanda päeva tulemine}

\par 1 See on, armsad, juba teine kiri, mille ma teile kirjutan, ja mõlematega ma hoian oma meeldetuletamisega ärkvel teie puhast meelt,
\par 2 et te meenutaksite neid sõnu, mis pühad prohvetid enne on rääkinud, ja Issanda ning Õnnistegija käsku, mille on andnud teie apostlid.
\par 3 Seda teadke kõigepealt, et viimseil päevil tuleb pilkesõnadega pilkajaid, kes elavad iseeneste himude järgi
\par 4 ning ütlevad: „Kus on tema tulemise tõotus? Sest sellest ajast, kui isad läksid hingama, jääb kõik nõnda, nagu oli loomise algusest!”
\par 5 Sest neil, kes seda tahavad, on teadmata, et taevad olid vanasti olemas ja maa koosnes veest ja vee läbi Jumala sõna väe tõttu
\par 6 ja et nende läbi tookordne maailm hukkus veeuputuses.
\par 7 Ent praegused taevad ja maa säilitatakse sama sõna läbi ning hoitakse alal tule jaoks jumalatute inimeste kohtu ja hukatuse päevaks.
\par 8 Aga see üks asi ärgu olgu teile, armsad, teadmata, et üks päev on Issanda juures nagu tuhat aastat ja tuhat aastat nagu üks päev!
\par 9 Issand ei viivita tõotust täitmast, nõnda nagu mõned peavad seda viivituseks, vaid tema on pika meelega teie vastu; sest ta ei taha, et keegi hukkuks, vaid et kõik tuleksid meeleparandusele!
\par 10 Aga Issanda päev tuleb otsekui varas öösel, mil taevad raginal hukkuvad ja maailma algained tules laostuvad ja ilmamaad ja kõiki seal tehtud töid ei leidu enam.

\section*{Tehtagu ettevalmistusi pääsemiseks}

\par 11 Kui see nüüd kõik nõnda laostub, missugused siis teie peate olema pühas elus ja jumalakartuses,
\par 12 oodates ja igatsedes Jumala päeva tulekut, mille pärast taevad süüdatult laostuvad ja maailma algained kuumuses ära sulavad?
\par 13 Aga meie ootame tema tõotuse järgi uusi taevaid ja uut maad, kus õigus elab!
\par 14 Sellepärast, armsad, seda oodates püüdke, et ta teid leiaks veatuina ja laitmatuina rahus;
\par 15 ja meie Issanda pikka meelt arvake õndsuseks, nõnda nagu ka meie armas vend Paulus teile on kirjutanud temale antud tarkust mööda,
\par 16 nagu ka kõigis kirjades, milles ta räägib neist asjust, kus on raske aru saada mõnest asjast, mida õppimatud ja kinnitamatud inimesed nagu muidki kirju väänavad iseeneste hukatuseks.
\par 17 Et teie nüüd, armsad, seda ette ära teate, siis hoiduge sellest, et üleannetute eksitused teid ei kisuks ühtlasi kaasa eksitusse ning te ei langeks ära oma kindlalt aluselt.
\par 18 Aga kasvage meie Issanda ja Õnnistegija Jeesuse Kristuse armus ja tunnetuses. Temale olgu austus nii nüüd kui igaviku päevil!




\end{document}