\begin{document}

\title{Johannese Ilmutusraamat}

\chapter{1}

\section*{Sissejuhatus ja tervitus}

\par 1 Jeesuse Kristuse ilmutus, mille Jumal temale andis, et ta näitaks tema sulastele, mis varsti peab sündima; ja seda ta näitas, läkitades oma ingli kaudu oma sulasele Johannesele,
\par 2 kes on tunnistanud Jumala sõna ja Jeesuse Kristuse tunnistust, kõike, mida ta on näinud.
\par 3 Õnnis on see, kes loeb, ja need, kes kuulevad selle prohvetliku kuulutuse sõnu ja peavad tallel, mis sellesse on kirja pandud; sest aeg on ligidal.
\par 4 Johannes seitsmele Aasia kogudusele: armu teile ja rahu sellelt, kes on ja kes oli ja kes tuleb; ja neilt seitsmelt vaimult, kes on tema aujärje ees;
\par 5 ja Jeesuselt Kristuselt, kes on ustav tunnistaja, esmasündinu surnute seast ja maailma kuningate valitseja. Temale, kes meid on armastanud ja meid on vabastanud meie pattudest oma verega,
\par 6 ja on meid teinud kuningriigiks, preestriks Jumalale ja oma Isale, temale olgu au ja vägi ajastute ajastuteni! Aamen.
\par 7 Vaata, tema tuleb pilvedega ja kõik silmad saavad teda näha, ja need, kes tema läbi pistsid, ja kõik maa suguharud tõstavad kaebehäält tema pärast. Tõepoolest, aamen!
\par 8 „Mina olen A ja O”, ütleb Issand Jumal, kes on ja kes oli ja kes tuleb, Kõigeväeline.

\section*{Nägemus Kristusest}

\par 9 Mina, Johannes, teie vend ja teie kaaslane viletsuses ja kuningriigis ja kannatlikkuses Jeesuses, olin saarel, mida kutsutakse Patmoseks, Jumala sõna ja Jeesuse Kristuse tunnistuse pärast.
\par 10 Ma olin vaimus Issanda päeval ja kuulsin oma selja taga suurt, otsekui pasuna häält,
\par 11 mis ütles: „Mida sa näed, kirjuta raamatusse ja läkita neile seitsmele kogudusele: Efesosse ja Smürnasse ja Pergamoni ja Tüatiirasse ja Sardesesse ja Filadelfiasse ja Laodikeasse!”
\par 12 Ja ma pöördusin ümber vaatama seda häält, mis minuga rääkis. Ja kui ma olin pöördunud, nägin ma seitset kuldküünlajalga.
\par 13 Ja keset neid seitset küünlajalga üht, kes oli Inimese Poja sarnane, riietatud pika rüüga ja vöötatud rinde alt kuldvööga.
\par 14 Tema pea ja juuksed olid valged nagu valge vill, nagu lumi, ja tema silmad nagu tuleleek;
\par 15 ja tema jalad olid otsekui ahjus hõõguv, hiilgav vasemaak ja tema hääl oli nagu suurte vete kohin.
\par 16 Ja temal oli paremas käes seitse taevatähte, ja tema suust läks välja terav kaheterane mõõk, ja tema pale oli otsekui päike, kui see paistab oma väes.
\par 17 Kui ma teda nägin, siis ma langesin maha tema jalgade ette nagu surnu. Ja ta pani oma parema käe mu peale ja ütles mulle: „Ära karda, mina olen esimene ja viimne
\par 18 ja Elav; ma olin surnud, ja vaata, ma olen elav ajastute ajastuteni, ja minu käes on surma ja surmavalla võtmed!
\par 19 Kirjuta siis, mida sa oled näinud ja mis nüüd on ja mis pärast seda sünnib!
\par 20 Nende seitsme taevatähe saladus, mida sa nägid minu paremas käes, ja nende seitsme kuldküünlajala saladus on see: need seitse tähte on seitsme koguduse inglid, ja need seitse küünlajalga on seitse kogudust!


\chapter{2}

\section*{Kiri Efesose kogudusele}

\par 1 Efesose koguduse inglile kirjuta: nõnda ütleb see, kes neid seitset tähte peab oma paremas käes, kes kõnnib seitsme kuldküünlajala vahel:
\par 2 Ma tean su tegusid ja sinu tööd ja su kannatlikkust ja et sa ei või sallida kurje ja oled läbi katsunud need, kes endid ütlevad apostlid olevat, aga ei ole, ning oled leidnud nad olevat valelikud;
\par 3 ja sinul on kannatlikkust ja sa oled kandnud raskusi minu nime pärast ega ole raugenud.
\par 4 Aga mul on sinu vastu, et sa oled maha jätnud oma esimese armastuse.
\par 5 Siis mõtle nüüd, kust sa oled langenud, ja paranda meelt ja tee esimesi tegusid; aga kui mitte, siis ma tulen su juurde ja lükkan su küünlajala tema asemelt ära, kui sa meelt ei paranda.
\par 6 Aga see sul on, et sa vihkad nikolaiitide tegusid, mida vihkan minagi.
\par 7 Kellel kõrv on, see kuulgu, mis Vaim kogudustele ütleb: kes võidab, sellele ma annan süüa elupuust, mis on Jumala paradiisis.

\section*{Kiri Smürna kogudusele}

\par 8 Ja Smürna koguduse inglile kirjuta: nõnda ütleb esimene ja viimne, kes oli surnud ja on saanud elavaks:
\par 9 Ma tean su viletsust ning vaesust - kuid sa oled siiski rikas - ja pilkamist nende poolt, kes ütlevad endid juudid olevat, aga ei ole, vaid on saatana kogudus!
\par 10 Ära karda seda, mida sul tuleb kannatada! Vaata, kurat tahab visata mõned teie seast vangitorni, et teid kiusataks; ja teil on viletsust kümme päeva. Ole ustav surmani, siis ma annan sulle elukrooni!
\par 11 Kellel kõrv on, see kuulgu, mis Vaim kogudustele ütleb: kes võidab, sellele ei tee teine surm mingit kahju!

\section*{Kiri Pergamoni kogudusele}

\par 12 Ja Pergamoni koguduse inglile kirjuta: nõnda ütleb see, kelle käes on terav kaheterane mõõk:
\par 13 Ma tean, kus sa elad: seal, kus on saatana aujärg! Ja sa pead kinni mu nimest ega ole minu usku salanud ka neil päevil, mil Antipas, mu tunnistaja, mu ustav, tapeti teie juures, kus saatan elab.
\par 14 Aga mul on pisut sinu vastu, et sul on seal neid, kes peavad kinni Bileami õpetusest, kes Baalakit õpetas võrgutama Iisraeli sööma ebajumalate ohvreid ja hoorama.
\par 15 Samuti on sul ka neid, kes samasuguselt kinni peavad nikolaiitide õpetusest.
\par 16 Paranda siis meelt; aga kui mitte, siis ma tulen su juurde nobedasti ja sõdin nende vastu oma suu mõõgaga!
\par 17 Kellel kõrv on, see kuulgu, mis Vaim kogudustele ütleb: kes võidab, sellele ma annan süüa varjule pandud mannat ja annan temale valge kivikese ja kivikese peale kirjutatud uue nime, mida ükski muu ei tunne kui aga see, kes selle saab.

\section*{Kiri Tüatiira kogudusele}

\par 18 Ja Tüatiira koguduse inglile kirjuta: nõnda ütleb Jumala Poeg, kellel on silmad nagu tuleleek ja jalad hiilgavad otsekui vasemaak:
\par 19 Ma tean sinu tegusid ja armastust ja usku ja teenimist ja kannatlikkust, ja et sinu viimseid tegusid on rohkem kui esimesi.
\par 20 Aga mul on sinu vastu, et sa sallid naist Iisebeli, kes ennast nimetab prohvetiks ja õpetab ning ahvatleb minu sulaseid hoorama ja sööma ebajumalate ohvreid.
\par 21 Ma olen temale andnud aega meeleparanduseks, aga ta ei taha pöörduda oma porduelust.
\par 22 Vaata, ma viskan ta haigevoodisse ja need, kes temaga abielu rikuvad, suurde viletsusse, kui nad ei pöördu tema tegudest!
\par 23 Ja tema lapsed ma lõpetan surmaga, ja kõik kogudused tunnevad ära, et mina olen see, kes katsub läbi neerud ja südamed, ja ma annan teile igaühele teie tegude järgi.
\par 24 Aga teile ma ütlen ja muile Tüatiiras olevaile, kellel ei ole seda õpetust ja kes ei ole, nagu nad ütlevad, ära tundnud saatana sügavusi: ma ei pane teile peale muud koormat,
\par 25 pidage vaid kinni, mis teil on, kuni ma tulen!
\par 26 Ja kes võidab ja otsani peab minu tegusid, sellele ma annan meelevalla paganate üle
\par 27 ja ta peab neid karjatama raudkepiga, nõnda nagu saviastjad pihuks lüüakse, otsekui minagi olen saanud meelevalla oma Isalt;
\par 28 ja mina annan temale koidutähe.
\par 29 Kellel kõrv on, see kuulgu, mis Vaim kogudustele ütleb!


\chapter{3}

\section*{Kiri Sardese kogudusele}

\par 1 Ja Sardese koguduse inglile kirjuta: nõnda ütleb see, kelle on seitse Jumala vaimu ja need seitse tähte: Ma tean sinu tegusid, et sul on nimi, et sa elad, aga oled surnud!
\par 2 Saa valvsaks ja kinnita muid, kes on suremas; sest mina ei ole leidnud, et su teod oleksid täielikud minu Jumala ees!
\par 3 Siis mõtle, kuidas sa sõna vastu võtsid ja kuulsid, ja pea seda ning paranda meelt! Kui sa nüüd ei valva, tulen ma kui varas ja sina ei tea, mil tunnil ma su peale tulen.
\par 4 Ometi on sul ka Sardeses mõned nimed, kes ei ole reostanud oma riideid, ja nad peavad minuga käima valgeis riideis, sest nad on seda väärt.
\par 5 Kes võidab, see riietatakse nõnda valgete riietega ja mina ei kustuta tema nime eluraamatust, ja ma tahan tunnistada tema nime oma Isa ees ja kõigi tema inglite ees.
\par 6 Kellel kõrv on, see kuulgu, mis Vaim kogudustele ütleb!

\section*{Kiri Filadelfia kogudusele}

\par 7 Ja Filadelfia koguduse inglile kirjuta: nõnda ütleb Püha, tõeline, kelle käes on Taaveti võti ja kes avab ja ükski ei sule; kes suleb ja ükski ei ava:
\par 8 Ma tean sinu tegusid. Vaata, ma olen seadnud su ette avatud ukse ja ükski ei suuda seda sulgeda; sest sul on pisut rammu, ja sa oled pidanud mu sõna ega ole salanud mu nime!
\par 9 Vaata, ma annan saatana kogudusest mõned, kes ütlevad endid juudid olevat, aga ei ole, vaid valetavad! Vaata, ma teen, et nad tulevad ja kummardavad sinu jalgade ette ja tunnevad ära, et ma sind olen armastanud!
\par 10 Et sa mu kannatlikkuse sõna oled pidanud, siis minagi tahan sind hoida kiusatustunni eest, mis on tulemas kogu maailma peale neid kiusama, kes maa peal elavad.
\par 11 Ma tulen nobedasti; pea kinni, mis sul on, et ükski sinu krooni ei võtaks!
\par 12 Kes võidab, selle ma teen sambaks oma Jumala templis, ja tema ei lähe sealt enam välja, ja ma kirjutan tema peale oma Jumala nime ja oma Jumala linna nime, Uue Jeruusalemma, mis maha tuleb taevast minu Jumala juurest, ja oma uue nime.
\par 13 Kellel kõrv on, see kuulgu, mis Vaim kogudustele ütleb!

\section*{Kiri Laodikea kogudusele}

\par 14 Ja Laodikea koguduse inglile kirjuta: nõnda ütleb Aamen, ustav ja tõeline tunnistaja, Jumala loodu algus:
\par 15 Ma tean sinu tegusid, et sa pole külm ega kuum. Oh, oleksid sa külm või kuum!
\par 16 Aga nüüd, et sa oled leige ja mitte külm ega kuum, siis tahan ma sind välja sülitada oma suust!
\par 17 Sa ju ütled: ma olen rikas ja mul on vara küllalt ega ole mul midagi vaja. Ja sa ei teagi, et sa oled vilets ja armetu ja vaene ja pime ja alasti!
\par 18 Ma annan sulle nõu osta minult kulda, mis tules on puhastatud, et sa võiksid rikkaks saada, ja valged riided, et sa nendega riietuksid ja ei nähtaks sinu alastuse häbi, ja silmasalvi võida silmi, et sa näeksid.
\par 19 Ma noomin ja karistan kõiki, keda ma armastan. Ole siis väga hoolas ja paranda meelt!
\par 20 Vaata, ma seisan ukse taga ja koputan: kui keegi mu häält kuuleb ja ukse avab, selle juurde ma lähen sisse ja söön õhtust ühes temaga ja tema minuga!
\par 21 Kes võidab, sellele ma annan istuda ühes minuga minu aujärjel, nõnda nagu minagi olen võitnud ja istunud ühes oma Isaga tema aujärjele.
\par 22 Kellel kõrv on, see kuulgu, mis Vaim kogudustele ütleb!


\chapter{4}

\section*{Nägemus taevast}

\par 1 Pärast seda ma nägin, ja vaata: uks oli avatud taevas ja endine hääl, mida olin kuulnud otsekui pasunat rääkivat minuga, ütles: „Astu siia üles, ja ma näitan sulle, mis pärast seda peab sündima!”
\par 2 Ja sedamaid olin mina vaimus. Ja vaata, aujärg seisis taevas, ja keegi istus aujärjel!
\par 3 Ja see, kes istus, oli jumelt otsekui jaspise- ja sardisekivi, ja vikerkaar oli aujärje ümber, jumelt sarnane smaragdikivile.
\par 4 Ja aujärje ümber oli kakskümmend neli aujärge; ja aujärgedel ma nägin istuvat kahtkümmend nelja vanemat, riietatud valgeisse riietesse ja kuldpärjad peas.
\par 5 Ja aujärjest väljus välke ja hääli ja piksemürinat, ja seitse tulist lampi põles aujärje ees; need on Jumala seitse vaimu.
\par 6 Ja aujärje ees oli otsekui klaasmeri, mägikristalli sarnane; ja keset aujärge ja aujärje ümber oli neli olendit, täis silmi eest ja tagant.
\par 7 Ja esimene olend oli lõukoera sarnane, ja teine olend oli härja sarnane, ja kolmandal olendil oli nägu otsekui inimesel, ja neljas olend oli lendava kotka sarnane.
\par 8 Ja neil neljal olendil oli igaühel kuus tiiba, ümberringi ja seestpoolt nad olid täis silmi. Nad ei lakka ööd ja päevad ütlemast: „Püha, püha, püha on Issand Jumal, Kõigeväeline, kes oli ja kes on ja kes tuleb!”
\par 9 Ja nii sageli kui olendid annavad austust, au ja tänu sellele, kes aujärjel istub ja elab ajastute ajastuteni,
\par 10 heidavad need kakskümmend neli vanemat maha selle ette, kes aujärjel istub, ja nad kummardavad teda, kes elab ajastute ajastuteni, ja panevad oma pärjad maha aujärje ette ning ütlevad:
\par 11 Sina, meie Issand ja Jumal, oled väärt võtma austust ja au ja väge, sest sina oled loonud kõik asjad, ja sinu tahte läbi on need olemas ja on loodud!”


\chapter{5}

\section*{Nägemus Jumala Tallest}

\par 1 Ja ma nägin selle paremas käes, kes istus aujärjel, rullraamatut, täis kirjutatud seestpoolt ja väljastpoolt ja kinni pandud seitsme pitseriga.
\par 2 Ja ma nägin vägevat inglit; see kuulutas suure häälega: „Kes on vääriline avama seda raamatut ja lahti võtma selle pitsereid?”
\par 3 Ja ükski ei võinud ei taevas ega maa peal ega maa all avada raamatut ega vaadata sinna sisse.
\par 4 Ja mina nutsin väga, et kedagi ei leitud väärt olevat avama raamatut ja vaatama sinna sisse.
\par 5 Ja üks vanemaist ütles mulle: „Ära nuta! Vaata, lõukoer Juuda suguharust, Taaveti juur, on võitnud nii, et tema võib avada raamatu ja lahti võtta selle seitse pitserit!”
\par 6 Ja ma nägin keset aujärge ja nelja olendit ja keset vanemaid seismas Talle, kes oli otsekui tapetud; ja tal oli seitse sarve ja seitse silma, mis on Jumala seitse vaimu, kes on läkitatud kogu ilmamaale.
\par 7 Ja ta tuli ja võttis raamatu selle paremast käest, kes istus aujärjel.
\par 8 Ja kui ta oli võtnud raamatu, siis need olendid ja need kakskümmend neli vanemat heitsid maha Talle ette, ja igaühel oli käes kannel ja kuldkausid, täis suitsutusrohte; need on pühade palved.
\par 9 Ja nad laulsid uut laulu ning ütlesid: „Sina oled väärt võtma raamatut ja lahti tegema selle pitsereid, sest sina oled olnud tapetud ja oled oma verega Jumalale ostnud inimesi kõigist suguharudest ja keeltest ja rahvaist ja paganahõimudest
\par 10 ja oled nad teinud kuningriigiks ja preestriks meie Jumalale, ja nad valitsevad maa peal.”
\par 11 Ja ma nägin ja kuulsin paljude inglite häält aujärje ja olendite ja vanemate ümbert; ja nende arv oli kümme tuhat korda kümme tuhat ja tuhat korda tuhat.
\par 12 Ja need ütlesid suure häälega: „Tall, kes on tapetud, on väärt võtma väge ja rikkust ja tarkust ja rammu ja au ja austust ja kiitust!”
\par 13 Ja kõike loodut, mis on taevas ja maa peal ja maa all ja meres, ja kõike, mis nendes on, kuulsin ma ütlevat: „Sellele, kes istub aujärjel, ja Tallele olgu kiitus ja au ja austus ja vägi ajastute ajastuteni!”
\par 14 Ja need neli olendit ütlesid: „Aamen!” Ja vanemad heitsid maha ja kummardasid.


\chapter{6}

\section*{Pitserite lahtitegemine}

\par 1 Ja ma nägin, kuidas Tall võttis lahti ühe seitsmest pitserist, ja ma kuulsin üht neljast olendist otsekui pikse häälega ütlevat: „Tule!”
\par 2 Ja ma vaatasin, ja näe: valge hobune; ja ta seljas istujal oli amb käes; ja talle anti pärg ja ta väljus võites ja võitu saavutama.
\par 3 Ja kui Tall võttis lahti teise pitseri, kuulsin ma teist olendit ütlevat: „Tule!”
\par 4 Ja väljus teine hobune, tulipunane, ja ta seljas istujal lasti rahu maa pealt ära võtta, et inimesed üksteist tapaksid; ja temale anti suur mõõk.
\par 5 Ja kui Tall võttis lahti kolmanda pitseri, kuulsin ma kolmandat olendit ütlevat: „Tule!” Ja ma vaatasin, ja näe: must hobune, ja tema seljas istujal olid kaalud käes.
\par 6 Ja ma kuulsin otsekui häält nelja olendi keskelt ütlevat: „Mõõt nisu teenari eest ja kolm mõõtu otri teenari eest; ja õlile ega viinale ära tee kahju!”
\par 7 Ja kui ta võttis lahti neljanda pitseri, kuulsin ma neljanda olendi häält ütlevat: „Tule!”
\par 8 Ja ma vaatasin, ja näe, tuhkur hobune, ja kes tema seljas istus, selle nimi oli Surm, ja Surmavald käis ühes temaga; ja neile anti meelevald neljanda osa ilmamaa üle, tappa mõõga ja nälja ja surmaga ja maapealsete kiskjate elajate abil.
\par 9 Ja kui ta võttis lahti viienda pitseri, siis ma nägin altari all nende hingi, kes olid tapetud Jumala sõna pärast ja tunnistuse pärast, mis neil oli.
\par 10 Ja nad kisendasid suure häälega ning ütlesid: „Oh püha ja tõeline Valitseja, kui kaua sa ei mõista kohut ega tasu kätte meie verd neile, kes elavad maa peal?”
\par 11 Ja neile anti igaühele pikk valge rüü, ja neile öeldi, et nad oleksid rahul veel natuke aega, kuni saab täis nende kaassulaste ja nende vendade arv, kes tapetakse nõnda nagu nemadki.
\par 12 Ja ma nägin, kui ta võttis lahti kuuenda pitseri, et siis sündis suur maavärisemine, ja päike läks mustaks nagu karvane kotiriie ja kuu läks kogunisti nagu vereks;
\par 13 ja taevatähed kukkusid maha, otsekui viigipuu ajab maha oma toored marjad, kui suur tuul teda raputab;
\par 14 ja taevas veeres ära nagu rullraamat, ja kõik mäed ja saared nihkusid oma paigust;
\par 15 ja ilmamaa kuningad ja suured isandad ja sõjapealikud ja rikkad ja võimumehed ja kõik orjad ja kõik vabad pugesid varjule koobastesse ja mägede kaljudesse,
\par 16 ja nad ütlesid mägedele ja kaljudele: „Langege meie peale ja varjake meid selle palge eest, kes aujärjel istub, ja Talle viha eest!
\par 17 Sest on tulnud tema suur vihapäev, ja kes võib püsida?”


\chapter{7}

\section*{Saja neljakümne nelja tuhande märkimine pitseriga}

\par 1 Pärast seda ma nägin nelja inglit seisvat maa neljal nurgal ning kinni pidavat nelja maatuult, et ükski tuul ei puhuks maa peale ega mere peale ega ühegi puu peale.
\par 2 Ja ma nägin teist inglit tõusvat päevatõusu poolt, ning ta käes oli elava Jumala pitsat, ja tema hüüdis suure häälega nendele neljale inglile, kellele oli antud teha kahju maale ja merele,
\par 3 ning ütles: „Ärge tehke kahju maale ega merele ega puudele, enne kui me oleme pannud pitseri oma Jumala sulaste otsaesisele!”
\par 4 Ja ma kuulsin nende arvu, kes olid pitseriga märgitud, sada nelikümmend neli tuhat pitseriga märgitut kõigist Iisraeli laste suguharudest:
\par 5 Juuda suguharust kaksteist tuhat pitseriga märgitut; Ruubeni suguharust kaksteist tuhat; Gaadi suguharust kaksteist tuhat;
\par 6 Aaseri suguharust kaksteist tuhat; Naftali suguharust kaksteist tuhat; Manasse suguharust kaksteist tuhat;
\par 7 Siimeoni suguharust kaksteist tuhat; Leevi suguharust kaksteist tuhat; Issaskari suguharust kaksteist tuhat;
\par 8 Sebuloni suguharust kaksteist tuhat; Joosepi suguharust kaksteist tuhat; Benjamini suguharust kaksteist tuhat pitseriga märgitut.

\section*{Nägemus lunastatuist taevas}

\par 9 Pärast seda ma nägin, ja vaata, suur hulk rahvast, keda ükski ei võinud ära lugeda kõigist rahvahõimudest ja suguharudest ja rahvaist ja keeltest; need seisid aujärje ees ja Talle ees, riietatud valgeisse rüüdesse ja palmioksad käes.
\par 10 Ja nad hüüdsid suure häälega ja ütlesid: „Õnnistus meie Jumalale, kes aujärjel istub, ja Tallele!”
\par 11 Ja kõik inglid seisid aujärje ja vanemate ja nelja olendi ümber ja heitsid aujärje ette silmili maha ja kummardasid Jumalat
\par 12 ning ütlesid: „Aamen, kiitus ja austus ja tarkus ja tänu ja au ja vägi ja ramm meie Jumalale ajastute ajastuteni! Aamen.”
\par 13 Ja üks vanemaist hakkas kõnelema ning ütles minule: „Need seal valgeis rüüdes, kes nad on ja kust nad tulid?”
\par 14 Ja ma ütlesin temale: „Mu isand, sina tead!” Ja ta ütles mulle: „Need on, kes tulevad suurest viletsusest ja on oma rüüd pesnud ja oma rüüd valgeks teinud Talle veres!
\par 15 Sellepärast on nad Jumala aujärje ees ja teenivad teda ööd ja päevad tema templis. Ja see, kes aujärjel istub, laotab oma telgi nende üle.
\par 16 Neil ei ole siis enam nälga ega janu; ka ei lange nende peale päikest ega mingisugust palavat;
\par 17 sest Tall, kes on keset aujärge, hoiab neid ja juhatab nad elava vee allikaile; ja Jumal pühib ära kõik pisarad nende silmist!”


\chapter{8}

\section*{Seitsme pasuna puhumine}

\par 1 Ja kui Tall võttis lahti seitsmenda pitseri, tekkis taevas vaikus ligi pooleks tunniks.
\par 2 Ja ma nägin seitset inglit, kes seisid Jumala ees, ja neile anti seitse pasunat.
\par 3 Ja teine ingel tuli ja astus altari äärde, kuldsuitsutusastja käes, ja talle anti palju suitsutusrohte, et ta neid lisaks kõigi pühade palvetele kuldaltaril, mis oli aujärje ees.
\par 4 Ja suitsutusrohtude suits pühade palvetega tõusis ingli käest Jumala ette.
\par 5 Ja ingel võttis suitsutusastja ja täitis selle tulega altarilt ja viskas ta maa peale. Siis sündis hääli ja mürinaid ja välke ning maavärisemine.
\par 6 Ja need seitse inglit, kelle käes oli seitse pasunat, valmistusid pasunat puhuma.
\par 7 Ja esimene puhus pasunat. Siis tuli rahet ja tuld verega segamini ja paisati maa peale. Ja kolmas osa maad põles ära ja kolmas osa puid põles ära ja kõik haljas rohi põles ära.
\par 8 Ja teine ingel puhus pasunat. Siis heideti merre otsekui suur mägi, mis tules põles, ja kolmas osa merd sai vereks.
\par 9 Ja kolmas osa mere loomi, kellel hing sees, suri ära ja kolmas osa laevu läks hukka.
\par 10 Ja kolmas ingel puhus pasunat. Siis kukkus taevast maha suur täht, mis põles otsekui tõrvalont ja langes kolmanda osa jõgede peale ja veeallikate peale.
\par 11 Ja tähe nimi on Koirohi. Ja kolmas osa vetest muutus koirohuks, ja palju inimesi suri vete kätte, sest need olid mõruks läinud.
\par 12 Ja neljas ingel puhus pasunat. Siis löödi kolmas osa päikest ja kolmas osa kuud ja kolmas osa tähti, nii et kolmas osa neist pimeneks ja kolmas osa päeva ei paistaks, ja ööd niisamuti.

\section*{Hädahüüd kolme tulevase nuhtluse pärast}

\par 13 Ja ma nägin ja kuulsin kotka lendavat kesktaeva all ning suure häälega ütlevat: „Häda, häda, häda neile, kes elavad maa peal, nende kolme ingli järgneva pasunahääle pärast, kes veel peavad pasunat puhuma!”


\chapter{9}

\section*{Hädahüüd kolme tulevase nuhtluse pärast}

\par 1 Ja viies ingel puhus pasunat. Siis ma nägin tähe taevast maa peale kukkunud olevat, ja temale anti sügavuse kaevu võti.
\par 2 Ja ta avas sügavuse kaevu, ja kaevust tõusis suits otsekui suure ahju suits. Ja päike ja õhk läksid pimedaks kaevu suitsust.
\par 3 Ja suitsust väljusid rohutirtsud maa peale ja neile anti meelevald, nagu skorpionidel on meelevald maa peal.
\par 4 Ja neile öeldi, et nad ei tohi kurja teha ei maa rohule ega millelegi, mis haljas on, ega ühelegi puule, vaid ainult inimestele, kellel ei ole Jumala pitserit otsaesisel.
\par 5 Ja neile anti käsk, et nad neid ei tapaks, vaid et nad neid piinaksid viis kuud; ja nende piinamine on otsekui skorpioni piinamine, kui ta inimest nõelab.
\par 6 Ja neil päevil inimesed otsivad surma ega leia seda mitte, ja püüavad surra, aga surm põgeneb nende eest!
\par 7 Ja rohutirtsud olid välimuselt sõjaks valmistatud hobuste sarnased ja neil olid peas pärjad otsekui kullast ja nende nägu oli nagu inimeste nägu;
\par 8 ja neil olid juuksed nagu naiste juuksed ja nende hambad olid otsekui lõukoerte hambad.
\par 9 Ja neil olid soomusrüüd, otsekui raudrüüd, ja nende tiibade kahin oli otsekui vankrite mürin, kui palju hobuseid jookseb sõtta.
\par 10 Ja neil olid sabad ja astlad nagu skorpionidel, ja nende sabades oli võim inimestele kahju teha viis kuud.
\par 11 Ja neil oli kuningaks sügavuse ingel; tema nimi on heebrea keeli Abadoon ja kreeka keeli Apollüon.
\par 12 Esimene häda on möödas; vaata, kaks häda tuleb veel pärast seda!
\par 13 Ja kuues ingel puhus pasunat. Siis ma kuulsin üht häält Jumala ees oleva kuldaltari neljast nurgast;
\par 14 see ütles kuuendale inglile, kelle käes oli pasun: „Päästa lahti need neli inglit, kes on seotud suure Eufrati jõe ääres!”
\par 15 Siis päästeti lahti neli inglit, kes on valmis tunniks ja päevaks ja kuuks ja aastaks, et tappa kolmas osa inimesi.
\par 16 Ja ratsaväelaste arv on kakskümmend tuhat korda kümme tuhat; ma kuulsin nende arvu.
\par 17 Ja nõnda ma nägin nägemuses hobuseid ja nende seljas istujaid, ja neil olid tulekarva ja purpursinised ja väävlikarva soomusrüüd; ja hobuste pead olid nagu lõukoerte pead, ja nende suust käis välja tuld ja suitsu ja väävlit.
\par 18 Neist kolmest nuhtlusest sai surma kolmas osa inimesi tule ja suitsu ja väävli läbi, mis nende hobuste suust välja käis.
\par 19 Sest hobuste võim on nende suus ja nende sabades; sest nende sabad on madude sarnased; neil on pead ja nendega nad teevad kahju.
\par 20 Aga järelejäänud inimesed, keda neis nuhtlustes ei surmatud, ei pöördunud siiski mitte oma käte tegudest, et nad poleks kummardanud kurje vaime ning kuld- ja hõbe- ja vask- ja kivi- ja puujumalaid, kes ei või näha ega kuulda ega kõndida.
\par 21 Ja nad ei pöördunud oma tapmistest ega oma nõidustest, ei oma hoorusest ega oma vargustest.



\chapter{10}

\section*{Vägev ingel ja raamatuke}

\par 1 Ja ma nägin teist vägevat inglit taevast maha tulevat; see oli riietatud pilvega ja vikerkaar oli tema pea kohal, ja tema pale oli otsekui päike ja tema jalad nagu tulesambad.
\par 2 Ja tema käes oli avatud raamatuke, ja ta pani oma parema jala mere ja vasaku maa peale.
\par 3 Ja ta kisendas suure häälega, otsekui lõukoer möirgab. Ja kui ta kisendas, siis hakkasid seitse pikset müristama oma häältega.
\par 4 Ja kui need seitse pikset oma häältega olid rääkinud, tahtsin ma kirjutada. Siis ma kuulsin häält taevast ütlevat: „Pane pitseriga kinni, mis seitse pikset on kõnelnud, ja ära kirjuta seda!”
\par 5 Ja ingel, keda ma nägin seisvat mere ja maa peal, tõstis oma käe taeva poole
\par 6 ja vandus selle juures, kes elab ajastute ajastuteni, kes on loonud taeva ja mis seal on ja maa ja mis seal on ja mere ja mis seal on, et aega ei saa enam olema,
\par 7 vaid seitsmenda ingli hääle päevil, kui ta hakkab pasunat puhuma, läheb täide Jumala saladus, nõnda nagu ta seda rõõmsat sõnumit on kuulutanud prohvetitele, oma sulastele!
\par 8 Ja hääl, mida ma taevast kuulsin, rääkis jälle minuga ja ütles: „Mine võta avatud raamatuke ingli käest, kes seisab mere ja maa peal!”
\par 9 Ja ma läksin ingli juurde ning ütlesin temale: „Anna mulle see raamatuke!” Ja tema ütles mulle: „Võta ja söö see ära! Ja see peab su kõhus olema mõru, aga su suus magus nagu mesi!”
\par 10 Ja ma võtsin raamatukese ingli käest ja sõin selle ära. Ja see oli mu suus magus nagu mesi. Ja kui ma selle olin söönud, sai see mu kõhus mõruks!
\par 11 Ja minule öeldi: „Sa pead veel ennustama rahvaste ja rahvahõimude ja keelte ja paljude kuningate kohta!”


\chapter{11}

\section*{Kaks tunnistajat}

\par 1 Siis anti mulle kepi sarnane pilliroog ning öeldi: „Tõuse ja mõõda Jumala tempel ja altar ja need, kes seal kummardavad;
\par 2 aga väljaspool templit olev õu jäta kõrvale ja ära seda mõõda, sest see on antud paganaile. Ja nad tallavad püha linna nelikümmend ja kaks kuud.
\par 3 Ja mina annan oma kahele tunnistajale meelevalla kotiriidesse riietatuina prohvetlikult kuulutada tuhat kakssada kuuskümmend päeva.
\par 4 Nemad on need kaks õlipuud ja kaks küünlajalga, mis seisavad ilmamaa Issanda ees.
\par 5 Ja kui keegi tahab neile kurja teha, siis väljub tuli nende suust ja sööb nende vaenlased; kui keegi tahab neile kurja teha, siis peab see nõnda tapetama.
\par 6 Neil on meelevald sulgeda taevas, et nende prohvetliku kuulutamise päevil ei sajaks vihma, ja neil on meelevald vete üle muuta neid vereks ja lüüa maad igasuguse nuhtlusega nii sageli kui nad vaid tahavad.
\par 7 Ja kui nad oma tunnistuse on lõpetanud, siis hakkab metsaline, kes tõuseb sügavusest, nendega sõda pidama ja võidab nad ära ja surmab nad.
\par 8 Ja nende kehad vedelevad suure linna tänavail, mida vaimulikult hüütakse Soodomaks ja Egiptuseks, kus ka nende Issand risti löödi.
\par 9 Ja rahvaste ja suguharude ja keelte ja rahvahõimude seast nähakse nende kehi kolm ja pool päeva ja nende kehi ei lasta hauda panna.
\par 10 Ja need, kes maa peal elavad, rõõmustuvad nendest ja on väga rõõmsad ning läkitavad ande üksteisele, sest et need kaks prohvetit vaevasid neid, kes maa peal elavad.”
\par 11 Ja pärast neid kolme ja poolt päeva läks elu vaim Jumalast nende sisse ja nad tõusid püsti oma jalgele, ja suur kartus tuli nende peale, kes neid vaatasid,
\par 12 Ja nad kuulsid suurt häält taevast neile ütlevat: „Tulge siia üles!” Ja nad läksid üles taevasse pilves, ja nende vaenlased vaatasid neid.
\par 13 Ja samal tunnil sündis suur maavärisemine, ja kümnes osa linna langes maha, ja selles maavärisemises sai surma inimesi seitse tuhat nime. Siis teised lõid kartma ja andsid taeva Jumalale austust.
\par 14 Teine häda on möödas, vaata, kolmas häda tuleb varsti!

\section*{Seitsmes pasun}

\par 15 Ja seitsmes ingel puhus pasunat. Siis sündisid suured hääled taevas ning ütlesid: „Maailma valitsus on saanud meie Issanda ja tema Kristuse omaks, ja tema valitseb ajastute ajastuteni!”
\par 16 Ja need kakskümmend neli vanemat, kes istuvad Jumala ees oma aujärgedel, heitsid silmili maha ja kummardasid Jumalat
\par 17 ning ütlesid: „Me täname sind, Issand, kõigeväeline Jumal, kes oled ja kes olid, et sa oled võtnud kätte oma suure väe ja oled saanud kuningaks.
\par 18 Ja paganrahvad on vihastunud, kuid sinu viha on tulnud, ja tulnud on aeg mõista kohut surnutele ja kätte anda palk su sulastele-prohvetitele ja pühadele ja neile, kes kardavad sinu nime, pisukestele ja suurtele, ja hävitada need, kes hävitavad maad!”
\par 19 Ja Jumala tempel taevas avanes ja tema seaduselaegast nähti tema templis. Siis sündis välke ja hääli ja mürinat ja maavärisemine ja suur rahesadu!


\chapter{12}

\section*{Nägemus naisest ja lohest}

\par 1 Ja suur imetäht ilmus taevas: naine, riietatud päikesega ja kuu tema jalge all, ja temal peas pärg kaheteistkümnest tähest.
\par 2 Ta oli käima peal ja kisendas lapsevaevas ja tal oli suur valu sünnitades.
\par 3 Ja teine imetäht ilmus taevas, ja vaata, suur tulipunane lohe, seitsme peaga ja kümne sarvega, ja ta peade peal seitse ehissidet.
\par 4 Tema saba pühkis ära kolmanda osa taevatähti ja viskas need maa peale. Ja lohe seisis naise ees, kes oli sünnitamas, et niipea kui ta on sünnitanud, neelata laps.
\par 5 Ja naine tõi ilmale poeglapse, kes peab raudkepiga karjatama kõiki paganrahvaid. Ja naise laps kisti Jumala ja tema aujärje juurde.
\par 6 Ja naine põgenes kõrbe, kus temal oli Jumalast valmistatud ase, et teda seal toidetaks tuhat kakssada kuuskümmend päeva.
\par 7 Ja sõda tõusis taevas: Miikael ja tema inglid sõdisid lohe vastu, ja lohe sõdis ja tema inglid.
\par 8 Ja ta ei saanud võimust ja nende aset ei leitud enam taevast.
\par 9 Ja suur lohe, see vana madu, keda hüütakse Kuradiks ja Saatanaks, kes eksitab kogu maailma, visati maa peale, ja tema inglid visati välja ühes temaga.
\par 10 Ja ma kuulsin suurt häält taevast ütlevat: „Nüüd on õnnistus ja vägi ja kuninglik valitsus saanud meie Jumala ja meelevald tema Kristuse kätte, sest välja on visatud meie vendade süüdistaja, kes nende peale kaebab meie Jumala ees päevad ja ööd!
\par 11 Ja nemad on tema võitnud Talle vere tõttu ja oma tunnistuse sõna tõttu ega ole oma elu armastanud surmani.
\par 12 Seepärast olge väga rõõmsad, taevad, ja teie, kes seal sees elate! Häda maale ja merele, sest kurat on tulnud maha teie juurde ja tal on suur viha, sest ta teab, et tal on pisut aega!”
\par 13 Ja kui lohe nägi, et ta oli visatud maa peale, ta kiusas taga naist, kes oli ilmale toonud poeglapse.
\par 14 Ja naisele anti kaks suure kotka tiiba, et ta lendaks kõrbe oma paika, kus teda toidetakse aeg ja ajad ja pool aega eemal mao palge eest.
\par 15 Ja madu purskas oma suust vett naisele järele otsekui jõge, et teda jõevooluga ära uhtuda.
\par 16 Aga maa aitas naist: maa avas oma suu ja neelas ära jõe, mille lohe oma suust välja ajas.
\par 17 Ja lohe sai vihaseks naise peale ja läks sõdima nendega, kes naise soost olid üle jäänud ja kes peavad Jumala käske ja kellel on Jeesuse tunnistus.
\par 18 Ja ta jäi seisma mere liivale.


\chapter{13}

\section*{Nägemus kahest metsalisest}

\par 1 Ja ma nägin metsalist tõusvat merest. Sel oli seitse pead ja kümme sarve ja tema sarvede peal kümme ehissidet ja tema peade peal Jumala pilkenimesid.
\par 2 Ja metsaline, keda ma nägin, oli pantri sarnane, ja tema jalad olid nagu karu jalad ja tema suu nagu lõukoera suu, ja lohe andis temale oma väe ja oma aujärje ja suure võimu.
\par 3 Ja ma nägin ühe tema peadest olevat nagu surmavalt haavatud; ja tema surmahaav paranes. Ja kogu ilmamaa imetles jälgides metsalist,
\par 4 ja nad kummardasid lohet, et ta metsalisele oli andnud selle võimu, ja kummardasid metsalist ning ütlesid: „Kes on metsalise sarnane ja kes suudab sõdida tema vastu?”
\par 5 Ja temale anti suu rääkida suurustamise ja jumalapilke sõnu, ja temale anti meelevald tegutseda nelikümmend ja kaks kuud.
\par 6 Ja ta avas oma suu pilkamiseks Jumala vastu, et pilgata tema nime ja tema telki, neid, kes taevas elavad.
\par 7 Ja temale anti võimus pidada sõda pühadega ja nad ära võita; ja temale anti meelevald iga suguharu ja rahva ja keele ja rahvahõimu üle.
\par 8 Ja teda hakkavad kummardama kõik, kes maa peal elavad, kelle nimed maailma algusest ei ole kirjutatud tapetud Talle eluraamatusse.
\par 9 Kui kellelgi on kõrv, siis ta kuulgu!
\par 10 Kui keegi viib vangi, siis ta ise satub vangi; kui keegi tapab mõõgaga, siis tapetakse teda ennast mõõgaga. Siin on pühade kannatlikkus ja usk.
\par 11 Ja ma nägin teist metsalist tõusvat maa seest; ja sel oli kaks sarve otsekui tallel ja ta rääkis nagu lohe.
\par 12 Ja ta teeb kõik esimese metsalise võimuga tema ees. Ja ta teeb, et maa ja need, kes seal peal elavad, kummardavad esimest metsalist, kelle surmahaav oli paranenud.
\par 13 Ja ta teeb suuri tunnustähti, nõnda et ta inimeste ees laseb tuldki tulla taevast alla maa peale.
\par 14 Ja ta eksitab neid, kes maa peal elavad, tunnustähtedega, mis temale on antud teha esimese metsalise ees, ning käsib neid, kes maa peal elavad, teha kuju metsalisele, kellel oli mõõga haav ja kes virgus ellu.
\par 15 Ja temale anti meelevald anda metsalise kujule vaim, et ka metsalise kuju räägiks ja teeks, et need, kes iialgi ei kummarda metsalise kuju, ära tapetaks.
\par 16 Ja tema teeb, et kõik, nii pisukesed kui suured, nii rikkad kui vaesed, nii vabad kui orjad, võtavad märgi oma parema käe peale või oma otsaesisele,
\par 17 ja et ükski muu ei saa osta ega müüa kui aga see, kellel on see märk: metsalise nimi või tema nime arv.
\par 18 Siin on tarkus! Kellel on mõistust, see arvaku ära metsalise arv; sest see on inimese arv. Ja tema arv on kuussada kuuskümmend kuus.


\chapter{14}

\section*{Nägemus Tallest ja lunastatuist}

\par 1 Ja ma nägin, ja vaata, Tall seisis Siioni mäel ja ühes temaga sada nelikümmend neli tuhat, kellele oli otsaesisele kirjutatud tema Isa nimi.
\par 2 Ja ma kuulsin häält taevast otsekui hulga vete kohisemist ja nagu suure müristamise häält; ja hääl, mida ma kuulsin, oli otsekui kandlelööjate hääl, kes löövad oma kandleid.
\par 3 Ja nemad laulsid uut laulu aujärje ees ja nende nelja olendi ja vanemate ees. Ja ükski ei võinud seda laulu õppida kui aga need sada nelikümmend neli tuhat, kes on ära ostetud ilmamaalt.
\par 4 Need on need, kes endid ei ole rüvetanud naistega, sest nad on neitsilikud; need on, kes Talle järgivad, kuhu ta iganes läheb; need on inimeste seast ära ostetud esianniks Jumalale ja Tallele,
\par 5 ja nende suust ei ole leitud kavalust; nemad on veatud.

\section*{Nägemus kolmest inglist}

\par 6 Ja ma nägin teist inglit lendavat kesktaeva kohal; sellel oli igavene evangeelium, et armuõpetust kuulutada neile, kes elavad maa peal, ja kõigile rahvahõimudele ja suguharudele ja keeltele ja rahvastele.
\par 7 Ja ta ütles suure häälega: „Kartke Jumalat ja andke temale austust, sest on tulnud tema kohtutund, ja kummardage teda, kes on teinud taeva ja maa ja mere ja veteallikad!”
\par 8 Ja veel teine ingel järgnes temale ning ütles: „Langenud, langenud on suur linn Baabülon, kes oma hooruskiima viinaga on jootnud kõiki rahvaid!”
\par 9 Ja veel kolmas ingel järgnes neile ning ütles suure häälega: „Kui keegi kummardab metsalist ja tema kuju ja võtab tema märgi oma otsaesisele või oma käe peale,
\par 10 siis see saab ka juua Jumala kange viha viinast, mis segamata on valatud tema viha karikasse; ja teda peab vaevatama tules ja väävlis pühade inglite ees ja Talle ees.
\par 11 Ja nende valu suits tõuseb üles ajastute ajastuteni; ja ei ole rahu päevad ega ööd neil, kes kummardavad metsalist ja tema kuju ega neil, kes vastu võtavad tema nime märgi.
\par 12 Siin on pühade kannatlikkus; siin on need, kes peavad Jumala käske ja Jeesuse usku!”
\par 13 Ja ma kuulsin häält taevast ütlevat: „Kirjuta: õndsad on surnud, kes Issandas surevad nüüdsest peale; tõesti, ütleb Vaim, nad hingavad oma vaevadest, sest nende teod lähevad nendega ühes!”

\section*{Maailm on küps lõikuseks}

\par 14 Ja ma nägin: vaata, valge pilv ja see, kes pilve peal istus, oli Inimese Poja sarnane; temal oli kuldpärg peas ja terav sirp käes.
\par 15 Ja templist väljus teine ingel ning hüüdis suure häälega sellele, kes istus pilve peal: „Pista sirp sisse ja põima, sest lõikuseaeg on tulnud, maa lõikus on ju valminud!”
\par 16 Ja see, kes istus pilve peal, laskis oma sirbi käia üle maa, ja maa lõigati paljaks.
\par 17 Ja teine ingel väljus taeva templist, ja temalgi oli terav sirp käes.
\par 18 Ja veel teine ingel väljus altarist, ja temal oli võimus tule üle. Ja ta hüüdis suure häälega sellele, kellel oli terav sirp käes, ning ütles: „Pista sisse oma terav sirp ja lõika maha maa viinapuu tarjad, sest selle kobarad on küpsed!”
\par 19 Ja ingel laskis käia oma sirbi üle maa ja lõikas maa viinamäe paljaks ja viskas marjad Jumala kange viha suurde surutõrde.
\par 20 Ja surutõrt sõtkuti väljaspool linna, ja tõrrest jooksis verd välja hobuste valjastest saadik tuhat kuussada vagu maad!


\chapter{15}

\section*{Nägemus seitsmest vihakausist ja seitsmest nuhtlusest}

\par 1 Ja ma nägin teist imetähte taevas, suurt ja imelist: seitse inglit, kelle käes oli seitse viimset nuhtlust; sest nendega sai Jumala kange viha täielikuks.
\par 2 Ja ma nägin otsekui klaasmerd segatud tulega ja neid, kes olid võitnud metsalise ja ta kuju ja ta nime arvu, seisvat klaasmere ääres, Jumala kandled käes.
\par 3 Ja nad laulsid Jumala sulase Moosese laulu ja Talle laulu, öeldes: „Suured ja imelised on sinu teod, Issand Jumal, Kõigeväeline! Õiged ja tõelised on sinu teed, sa rahvaste Kuningas!
\par 4 Kes ei peaks kartma sind, Issand, ja andma austust sinu nimele? Sest sina üksi oled püha; sest kõik paganad tulevad ja kummardavad sind, sellepärast et su õiged kohtuotsused on saanud avalikuks!”
\par 5 Ja pärast seda ma nägin: tunnistustelgi tempel taevas avati.
\par 6 Ja need seitse inglit, kelle käes oli seitse nuhtlust, väljusid templist, riietatud puhta ja hiilgava lõuendiga ja rinde ümbert vöötatud kuldvööga.
\par 7 Ja üks neist neljast olendist andis seitsmele inglile seitse kuldkaussi täis kanget Jumala viha, kes elab ajastute ajastuteni.
\par 8 Ja tempel sai täis suitsu Jumala aust ja tema väest, ja ükski ei võinud sinna templisse sisse minna enne kui oli täide saadetud seitsme ingli seitse nuhtlust.


\chapter{16}

\section*{Nägemus seitsmest vihakausist ja seitsmest nuhtlusest}

\par 1 Ja ma kuulsin suurt häält templis ütlevat seitsmele inglile: „Minge ja valage Jumala kange viha kausid välja maa peale!”
\par 2 Ja esimene läks ja valas oma kausi välja maa peale. Ja kurje ning ilgeid mädapaiseid tekkis inimeste külge, kellel oli metsalise märk ja kes kummardasid tema kuju.
\par 3 Ja teine ingel valas oma kausi välja merre, ja see sai otsekui surnu vereks, ja suri iga elav hing, kõik, kes meres olid.
\par 4 Ja kolmas ingel valas oma kausi välja jõgedesse ja veeallikaisse, ja need said vereks.
\par 5 Ja ma kuulsin vete inglit ütlevat: „Õige oled sina, kes oled ja kes olid, ja sina, Püha, et sa nõnda oled kohut mõistnud!
\par 6 Sest nemad on pühade ja prohvetite verd valanud, ja verd oled ka sina neile juua andnud; nad on seda väärt!”
\par 7 Ja ma kuulsin altarit ütlevat: tõesti, Issand Jumal, sa Kõigeväeline, tõelised ja õiged on sinu kohtuotsused!”
\par 8 Ja neljas ingel valas välja oma kausi päikese peale. Ja temale anti inimesi tulega kõrvetada.
\par 9 Ja suur kuumus kõrvetas inimesi ja nemad pilkasid Jumala nime, kelle meelevalla all on need nuhtlused, ega parandanud meelt, et nad temale oleksid austust andnud.
\par 10 Ja viies ingel valas välja oma kausi metsalise aujärje peale. Ja tema kuningriik pimenes; ja nad närisid oma keelt valu pärast
\par 11 ja pilkasid taeva Jumalat oma valude ja paisete pärast ega pöördunud oma tegudest.
\par 12 Ja kuues ingel valas välja oma kausi suure Eufrati jõe peale, ja selle vesi kuivas ära, et tee oleks valmis kuningaile, kes tulevad päevatõusu poolt.
\par 13 Ja ma nägin lohe suust ja metsalise suust ja valeprohveti suust väljuvat kolm rüvedat vaimu otsekui konnad.
\par 14 Need on nimelt kuradite tunnustähti tegevad vaimud, kes lähevad välja kõige ilmamaa kuningate juurde neid koguma sõtta kõigeväelise Jumala suureks päevaks. -
\par 15 Vaata, ma tulen kui varas! Õnnis see, kes valvab ja hoiab oma riideid, et ta ei käiks alasti ja et ei nähtaks tema häbi. -
\par 16 Ja need kogusid nad kokku ühte paika, mida heebrea keeles kutsutakse Harmagedooniks.
\par 17 Ja seitsmes ingel valas oma kausi välja õhku, ja suur hääl kostis taeva templist aujärjelt ning ütles: „See on sündinud!”
\par 18 Ja sündis hääli ja müristamisi ja välke ja tekkis suur maavärisemine, millist ei ole olnud sest ajast, kui inimesi on olnud maa peal; säärane suur maavärisemine oli see.
\par 19 Ja suur linn jagunes kolmeks osaks, ja paganrahvaste linnad langesid. Ja suurt Baabüloni tuletati meelde Jumala ees, et Jumal annaks temale oma raevuviina karika.
\par 20 Ja kõik saared põgenesid ja mägesid ei leidunud enam.
\par 21 Ja suuri raheteri, talendiraskusi, tuli taevast maha inimeste peale. Ja inimesed pilkasid Jumalat rahe nuhtluse pärast; sest selle nuhtlus oli väga vali.


\chapter{17}

\section*{Nägemus hoorast ja metsalisest}

\par 1 Siis tuli üks seitsmest inglist, kelle käes olid need seitse kaussi, ja kõneles minuga ning ütles: „Tule, ma tahan sulle näidata suure hoora nuhtlemist, kes istub suurte vete peal,
\par 2 kellega on hooranud ilmamaa kuningad ja kelle hooruse viinast on ilmamaa elanikud joobnuks saanud!”
\par 3 Ka tema viis mind vaimus ära kõrbe. Seal ma nägin naist istuvat helkjaspunase metsalise seljas, kes oli täis pilkenimesid ja kellel oli seitse pead ja kümme sarve.
\par 4 Ja naine oli riietatud purpuriga ja helkjaspunase riidega ning ehitud kulla ja kalliskivide ja pärlitega, ja temal oli käes kuldkarikas täis jälkusi ja oma hooruse saasta.
\par 5 Ja tema otsaesisele oli kirjutatud nimi, saladus: „Suur Baabülon, ilmamaa hoorade ja jälkuste ema!”
\par 6 Ja ma nägin naise olevat joobnud pühade verest ja Jeesuse tunnistajate verest. Ja ma imestasin üliväga teda nähes.
\par 7 Ja ingel ütles mulle: „Mispärast sa imestad? Ma ütlen sulle saladuse naisest ja metsalistest, kes naist kannab ja kellel on seitse pead ja kümme sarve.
\par 8 Metsaline, keda sa nägid, on olnud, ja teda ei ole enam; aga ta tuleb sügavusest üles ja läheb hukatusse; ja ilmamaa elanikud, kelle nimed ei ole kirjutatud eluraamatusse maailma asutamisest alates, imestavad metsalist nähes, sest ta oli ja teda ei ole, aga ta tuleb jälle!
\par 9 Siia kuulub mõistus, millel on tarkust. Need seitse pead on seitse mäge, millel naine istub, ja on seitse kuningat:
\par 10 viis on neist langenud, üks on alles, ja teine ei ole veel tulnud. Kui ta tuleb, peab ta jääma natukeseks ajaks.
\par 11 Ja metsaline, kes on olnud ja keda ei ole enam, on ise kaheksas ja ühtlasi üks neist seitsmest, ja ta läheb hukatusse.
\par 12 Ja need kümme sarve, mis sa nägid, on kümme kuningat, kes ei ole veel saanud valitsust, aga nad saavad võimu kui kuningad üheks tunniks ühes metsalisega.
\par 13 Neil on üks ja sama nõu, ja nad annavad metsalise kätte oma väe ja võimu.
\par 14 Nemad hakkavad sõdima Tallega, ja Tall võidab nad ära, sest tema on isandate Isand ja kuningate Kuningas, ja ühes temaga võidavad need, kes on kutsutud ja valitud ja ustavad!”
\par 15 Ja ta ütles mulle: „Veed, mis sa nägid seal, kus hoor istub, on rahvad ja rahvahulgad ja rahvahõimud ja keeled.
\par 16 Ja need kümme sarve, mis sa nägid, ja metsaline, need vihkavad hoora ja teevad tema puupaljaks ja alasti ja söövad tema liha ja põletavad ta ära tulega.
\par 17 Sest Jumal on neile andnud südamesse teha tema nõu järgi ja heita ühte nõusse ning anda oma kuninglik valitsus metsalise kätte, kuni täituvad Jumala sõnad.
\par 18 Ja naine, keda sa nägid, on suur linn, kelle käes on valitsus ilmamaa kuningate üle!”


\chapter{18}

\section*{Baabüloni langemine}

\par 1 Pärast seda ma nägin teist inglit taevast maha tulevat, ja temal oli suur meelevald; ja ilmamaa läks valgeks tema auhiilgusest.
\par 2 Ja tema hüüdis võimsa häälega ning ütles: „Langenud, langenud on suur Baabülon ja on saanud kurjade vaimude eluasemeks ja kõigi rüvedate vaimude ulualuseks ja kõigi rüvedate ja vihatud lindude pesapaigaks!
\par 3 Sest tema hooruskiima viina on joonud kõik rahvad, ja ilmamaa kuningad on temaga hooranud ja ilmamaa kaupmehed on tema liiast toredusest rikkaks saanud!”
\par 4 Ja ma kuulsin teist häält taevast ütlevat: „Minge välja temast, minu rahvas, et te ei saaks tema pattude osaliseks ja et teid ei tabaks tema nuhtlused!
\par 5 Sest tema patud on ulatunud taevani ja Jumal on tuletanud meelde tema ülekohtused teod.
\par 6 Makske temale kätte, nõnda nagu tema on kätte maksnud, ja tasuge temale kahekordselt tema tegusid mööda; karikasse, mille tema on täis valanud, valage temale kahekordselt;
\par 7 nii palju kui tema on ülistanud iseennast ja taga ajanud toredust, niisama palju tehke temale piina ja leina. Sest ta ütleb oma südames: ma istun kui kuninganna ega ole lesk ega saa kunagi näha kurvastust!
\par 8 Sellepärast tulevad ühel päeval tema nuhtlused: surm ja lein ja nälg, ja ta põletatakse ära tulega, sest vägev on Issand Jumal, kes on mõistnud kohut tema üle!
\par 9 Ja teda uluvad taga ja kaebavad ilmamaa kuningad, kes temaga on hooranud ja toredasti elanud, kui nad näevad tema põlemise suitsu,
\par 10 hirmu pärast tema piina ees kaugel seistes ning öeldes: „Häda, häda, Baabülon, sa suur linn, sa vägev linn, sest ühe silmapilguga on su kohus tulnud!”
\par 11 Ja ilmamaa kaupmehed nutavad ja leinavad tema pärast, sest keegi ei osta enam nende kaupa:
\par 12 ei kaubaks toodud kulda, ei hõbedat, ei kalliskive, ei pärleid, ei kallist lõuendit, ei purpurit, ei siidi, ei helkjaspunast riiet, ei mingisugust healõhnalist puud, ei mingisugust elevandiluust riista ega mingisugust kallimast puust riista, ei vasest, ei rauast ega marmorkivist riista,
\par 13 ei kaneeli, ei juuksevõiet, ei suitsutusrohte, ei salvi, ei viirukit, ei viina, ei õli, ei peenikest jahu, ei nisu, ei veiseid, ei lambaid, ei hobuseid, ei tõldu, ei orje ega inimhingi!
\par 14 Ja puuvili, mida su hing ihaldas, on su käest ära läinud, ja kõik, mis lihav ja läikiv, on su käest ära läinud, ja seda sa enam ei leia!
\par 15 Nende asjade kaupmehed, kes selle linna kaudu rikkaks said, seisavad kaugel hirmu pärast tema piina ees, nuttes ja leinates,
\par 16 ja ütlevad: „Häda, häda, sa suur linn, kes olid riietatud kalli lõuendi ja purpuri ja helkjaspunase riidega ning olid ilustatud kulla ja kalliskivide ja pärlitega, sest ühe silmapilguga on nii suur rikkus hävitatud!”
\par 17 Ja kõik tüürimehed ja kõik rannasõidu-laevnikud ja meremehed ja kõik, kes merel sõidavad, seisid eemal
\par 18 ja kisendasid, kui nad nägid tema põlemise suitsu, ning ütlesid: „Kes on selle suure linna sarnane?”
\par 19 Ja nad puistasid põrmu pea peale ja kisendasid nuttes ja leinates ning ütlesid: „Häda häda, sa suur linn, kelle varaga on rikkaks saanud kõik, kellel oli laevu merel - ühe silmapilguga on ta hävitatud!”
\par 20 Rõõmutsege temast, taevas ja pühad ja apostlid ja prohvetid! Sest Jumal on teie kohtu mõistnud tema kätte!
\par 21 Ja üks tugev ingel tõstis kivi, nagu suure veskikivi, ja viskas selle merre, öeldes: „Nõnda visatakse suur linn Baabülon äkitselt ära ja teda ei leita enam!
\par 22 Su sees ei ole enam kuulda kandlelööjate ja mängumeeste ja vileajajate ja pasunapuhujate häält ega leidu sinus enam ühtki ametimeest; ja enam ei kosta su sees veski mürin;
\par 23 su sees ei paista enam küünlavalgus, ei ole enam kuulda peigmehe ega pruudi häält su sees! Sest sinu kaupmehed olid maa suured isandad, sinu nõidusega on kõik rahvad viidud eksitusse!
\par 24 Temast on leitud prohvetite ja pühade verd ja kõikide verd, kes on tapetud maa peal!”


\chapter{19}

\section*{Võidurõõm taevas ja Talle pulmad}

\par 1 Pärast seda ma kuulsin otsekui hulga rahva suurt häält taevast ütlevat: „Halleluuja! Õnnistus ja austus ja vägi olgu meie Jumalale!
\par 2 Sest tõelised ja õiged on tema kohtud, et ta on kohut mõistnud suure hoora üle, kes rikkus ilmamaa oma hooraeluga, ja on oma sulaste vere temale kätte maksnud!”
\par 3 Ja nad ütlesid teist korda: „Halleluuja!” Ja tema suits tõuseb üles ajastute ajastuteni.
\par 4 Ja need kakskümmend neli olendit heitsid maha ja kummardasid Jumalat, kes aujärjel istub ja ütlesid: „Aamen, halleluuja!”
\par 5 Ja aujärjelt kostis hääl, mis ütles: „Kiitke meie Jumalat, kõik tema sulased, kes teda kardate, nii pisukesed kui suured!”
\par 6 Ja ma kuulsin otsekui hulga rahva häält ja otsekui suurte vete kohinat ja otsekui kange pikse müristamist, ütlevat: „Halleluuja! Sest Issand, meie kõigeväeline Jumal, on võtnud kuningliku valitsuse oma kätte!
\par 7 Rõõmutsegem ja hõisakem ning andkem temale austust! Sest Talle pulmad on tulnud ja tema naine on ennast valmistanud!
\par 8 Ja naisele anti riietumiseks hiilgav ja puhas lõuend. See lõuend on pühade õiged teod!”
\par 9 Ja ingel ütles mulle: „Kirjuta: õndsad on need, kes Talle pulma õhtusöömaajale on kutsutud!” Ja tema ütles veel mulle: „Need on tõelised Jumala sõnad!”
\par 10 Ja ma heitsin maha tema jalge ette, et teda kummardada, aga tema ütles mulle: „Vaata, ära tee seda! Ma olen sinu ja nende su vendade kaassulane, kellel on Jeesuse tunnistus. Kummarda Jumalat, sest Jeesuse tunnistus on prohvetikuulutamise vaim!”

\section*{Võitja Kristus}

\par 11 Ja ma nägin taeva avatud, ja vaata: valge hobune! Ja selle nimi, kes tema seljas istus, on Ustav ja Tõeline, ja tema mõistab kohut ja sõdib õiguses.
\par 12 Aga tema silmad olid tuleleek ja temal oli peas palju ehissidemeid; ja temale oli kirjutatud nimi, mida ei tea keegi muu kui tema ise.
\par 13 Ja temal oli seljas verega kastetud kuub ja tema nimeks on pandud Jumala Sõna.
\par 14 Ja sõjaväed taevas järgisid teda valgete hobuste seljas ja olid riietatud valge ja puhta lõuendiga.
\par 15 Ja tema suust tuli välja terav mõõk, et sellega lüüa rahvaid. Ja tema ise karjatab neid raudkepiga. Ja ta ise sõtkub kõigeväelise Jumala tulise viha viina surutõrt.
\par 16 Ja temale on tema kuue ja puusa peale kirjutatud nimi: „Kuningate Kuningas ja isandate Isand!”

\section*{Kristuse vaenlaste hävitamine}

\par 17 Ja ma nägin üht inglit seisvat päikeses ja see kisendas suure häälega ning hüüdis kõigile kesktaeva kohal lendavaile lindudele: „Tulge ja lennake kokku suurele Jumala söömaajale
\par 18 sööma kuningate liha ja ülemate pealikute liha ja vägevate liha ja hobuste ja nende seljas istujate liha ja kõigi vabade ja orjade ja pisukeste ja suurte liha!”
\par 19 Ja ma nägin metsalist ja ilmamaa kuningaid ja nende sõjaväge olevat kogunenud sõda pidama sellega, kes istus hobuse seljas, ja tema sõjaväega.
\par 20 Ja metsaline võeti kinni ja temaga ühes valeprohvet, kes tema ees oli teinud oma imeteod, millega ta oli eksitanud neid, kes metsalise märgi olid vastu võtnud ja kes kummardasid tema kuju. Nad mõlemad visati elusalt tulejärve, mis põleb väävliga.
\par 21 Ja teised tapeti mõõgaga, mis lähtus selle suust, kes istus hobuse seljas. Ja kõigi lindude kõhud said täis nende lihast.


\chapter{20}

\section*{Saatana tuhandeaastane vangistus}

\par 1 Ja ma nägin ingli taevast maha tulevat; sellel oli sügaviku võti ja suured ahelad käes.
\par 2 Ja ta võttis kinni lohe, vana mao, see on Kuradi ehk Saatana, ja sidus ta ahelaisse tuhandeks aastaks
\par 3 ja heitis ta sügavikku ja pani ta luku taha ja vajutas pealepoole teda pitseri, et ta enam ei eksitaks rahvaid, kuni need tuhat aastat otsa saavad. Pärast seda peab teda natukeseks ajaks lahti lastama.
\par 4 Ja ma nägin aujärgi ja neid, kes nendel istusid; ja kohus anti nende kätte; ja ma nägin nende hingi, kelle pead olid ära raiutud Jeesuse tunnistuse ja Jumala sõna pärast ja kes ei olnud kummardanud metsalist ega tema kuju ega olnud võtnud tema märki oma otsaesisele ja oma käe peale. Ja nad virgusid ellu usus ja valitsesid kuninglikult ühes Kristusega tuhat aastat.
\par 5 Aga muud surnud ei elustunud mitte, kuni need tuhat aastat otsa saavad. See on esimene ülestõusmine.
\par 6 Õnnis ja püha on see, kellel on osa esimesest ülestõusmisest; nende üle ei ole teisel surmal meelevalda, vaid nad peavad olema Jumala ja Kristuse preestrid ja valitsema kuninglikult ühes temaga need tuhat aastat.

\section*{Saatana vabastamine ja otsustav võitlus}

\par 7 Ja kui need tuhat aastat täis saavad, siis lastakse saatan lahti oma vangist.
\par 8 Ja ta läheb välja eksitama rahvaid, kes elavad ilmamaa neljas nurgas, Googi ja Maagoogi, neid sõtta koguma. Nende arv on nagu mereäärne liiv!
\par 9 Ja nad tulid üles ilmamaa lagendikule ja piirasid ümber pühade leeri ja armastatud linna. Siis langes tuli taevast maha ja sõi nad ära!
\par 10 Ja kurat, kes neid eksitas, heideti tule ja väävli järve, kus on ka metsaline ja valeprohvet. Ja neid vaevatakse päevad ja ööd ajastute ajastuteni!

\section*{Üldine surnute ülestõusmine ja viimne kohus}

\par 11 Ja ma nägin suurt valget aujärge ja seda, kes sellel istub, kelle palge eest põgenesid maa ja taevas, ja neile ei leitud aset.
\par 12 Ja ma nägin surnuid, suuri ja pisukesi, seisvat aujärje ees, ja raamatud avati. Ja teine raamat avati, see on eluraamat. Ja surnutele mõisteti kohut sedamööda kuidas raamatuisse oli kirjutatud, nende tegude järgi.
\par 13 Ja meri andis tagasi need surnud, kes temas olid, ja surm ja surmavald andsid tagasi surnud, kes neis olid, ja igaühele mõisteti kohut tema tegude järgi.
\par 14 Ja surm ja surmavald heideti tulejärve! See on teine surm, tulejärv.
\par 15 Ja keda ei leitud kirjutatud olevat eluraamatusse, heideti tulejärve!


\chapter{21}

\section*{Uus taevas ja uus maa}

\par 1 Ja ma nägin uut taevast ja uut maad; sest esimene taevas ja esimene maa olid kadunud ja merd ei olnud enam.
\par 2 Ja ma nägin püha linna, uut Jeruusalemma, taevast Jumala juurest maha tulevat, valmistunud otsekui pruut, kaunistatud oma mehele.
\par 3 Ja ma kuulsin suurt häält aujärjelt ütlevat: „Vaata, Jumala telk on inimeste juures! Ja tema asub nende juurde elama ja nemad on tema rahvad ja Jumal ise on nendega.
\par 4 Ja tema pühib ära kõik pisarad nende silmist, ja surma ei ole enam ega leinamist ega kisendamist ega vaeva ei ole enam. Sest endised asjad on möödunud!”
\par 5 Ja kes aujärjel istus, ütles: „Vaata, ma teen kõik uueks!” Ja ta ütles: „Kirjuta, sest need sõnad on ustavad ja tõelised!”
\par 6 Ja ta ütles mulle: „See on sündinud! Mina olen A ja O, algus ja ots. Mina annan sellele, kellel on janu, eluvee allikast ilma hinnata.
\par 7 Kes võidab, see pärib selle, ja mina olen temale Jumalaks ja tema on minule pojaks.
\par 8 Aga argade ja uskmatute ja jälkide ja tapjate ja hoorajate ja nõidade ja ebajumalateenijate ja kõigi valelike osa on tule ja väävliga põlevas järves! See on teine surm.”

\section*{Taevane Jeruusalemm}

\par 9 Ja minu juurde tuli üks seitsmest inglist, kelle käes olid need seitse kaussi, täis seitset viimset nuhtlust; ja ta rääkis minuga ning ütles: „Tule, ma näitan sulle mõrsjat, Talle naist!”
\par 10 Ja tema kandis mind vaimus suure ja kõrge mäe otsa ja näitas mulle linna, püha Jeruusalemma, mis oli maha tulemas taevast Jumala juurest
\par 11 ja millel on Jumala auhiilgus; ja tema valgus on kõige kallima kivi sarnane, otsekui jaspiskivi, mis hiilgab nagu mägikristall.
\par 12 Linnal on suur ja kõrge müür ja kaksteist väravat ja väravate peal kaksteist inglit; ja väravate peale on kirjutatud nimed. Need on kaheteistkümne Iisraeli laste suguharu nimed.
\par 13 Ida pool on kolm väravat, põhja pool kolm väravat, lõuna pool kolm väravat, lääne pool kolm väravat.
\par 14 Ja linna müüril on kaksteist aluskivi ja nende peal Talle kaheteistkümne apostli nimed.
\par 15 Ja sellel, kes minuga rääkis, oli käes kuldpilliroog, et mõõta linna ja tema väravaid ja tema müüri.
\par 16 Ja linn on nelinurkne ja tema pikkus on niisama suur kui tema laius. Ja ta mõõtis linna pillirooga kaksteist tuhat vagu. Tema pikkus ja laius ja kõrgus on võrdsed.
\par 17 Ja ta mõõtis tema müüri - sada nelikümmend neli küünart, inimese mõõdu järgi, mis on ingli mõõt.
\par 18 Ja tema müüri alusehitus oli jaspiskivist ja linn puhtast kullast, selge klaasi sarnane.
\par 19 Ja linna müüri aluskivid olid ehitud kõiksuguste kalliskividega. Esimene aluskivi oli jaspis, teine safiir, kolmas kaltsedoon, neljas smaragd,
\par 20 viies sardoonüks, kuues sardion, seitsmes krüsoliit, kaheksas berüll, üheksas topaas, kümnes krüsopraas, üheteistkümnes hüatsint, kaheteistkümnes ametüst.
\par 21 Ja need kaksteist väravat olid kaksteist pärlit: iga värav oli ühest pärlist ja linna tänav selgest kullast otsekui läbipaistev klaas.
\par 22 Ja templit ma seal ei näinud, sest Issand, kõigeväeline Jumal, on tema tempel, ja Tall.
\par 23 Ja linnale ei ole tarvis päikest ega kuud, et need seal paistaksid; sest Jumala auhiilgus valgustab teda ja tema lamp on Tall.
\par 24 Ja rahvad hakkavad käima tema valguses, ja ilmamaa kuningad toovad sinna oma auhiilguse.
\par 25 Ja tema väravaid ei suleta päeval, aga ööd seal ei olegi.
\par 26 Ja rahvaste auhiilgus ja kallisvarad tuuakse sinna sisse.
\par 27 Aga sinna sisse ei pääse midagi, mis ei ole puhas, ei keegi, kes teeb jälkusi ja valet, vaid üksnes need, kes on kirjutatud Talle eluraamatusse.


\chapter{22}

\section*{Taevane Jeruusalemm}

\par 1 Ja ta näitas mulle puhast eluveejõge, selget nagu mägikristall. See voolas välja Jumala ja Talle aujärjest.
\par 2 Keset linna tänavat ja mõlemal pool jõge oli elupuu; see kandis vilja kaksteist korda, igas kuus andes oma vilja. Ja puu lehed tulid terviseks rahvastele.
\par 3 Ja midagi neetut ei ole enam. Ja Jumala ja Talle aujärg on siis seal, ja tema sulased teenivad teda.
\par 4 Ja nad näevad tema palet, ja tema nimi on nende otsaesisel.
\par 5 Ja ööd ei ole enam, ja neile ei ole vaja lambivalgust ega päevavalget, sest Issand Jumal valgustab neid, ja nemad valitsevad ajastute ajastuteni.
\par 6 Ja ta ütles mulle: „Need on ustavad ja tõelised sõnad! Ja Issand, prohvetite hingede Jumal, on läkitanud oma ingli oma sulastele näitama, mis peab varsti sündima.
\par 7 Ja vaata, ma tulen nobedasti! Õnnis see, kes peab selle raamatu ennustuse sõnu!”
\par 8 Ja mina, Johannes, olen see, kes seda kuulis ja nägi. Ja kui ma seda olin kuulnud ja näinud, ma heitsin maha, kummardama ingli jalgade ette, kes mulle seda näitas.
\par 9 Ja ta ütles mulle: „Vaata, ära tee seda! Sest ma olen sinu ja su vendade, prohvetite ja nende kaassulane, kes selle raamatu sõnu peavad. Kummarda Jumalat!”
\par 10 Ja ta ütles mulle: „Ära pane pitseriga kinni selle raamatu ennustussõnu, sest aeg on ligidal!
\par 11 Kes ülekohut teeb, tehku veel ülekohut, ja kes on rüve, see rüvetugu veel enam; aga kes on õige, tehku veel õigust, ja kes on püha, see pühitsegu ennast veel enam!
\par 12 Vaata, ma tulen pea, ja mu palk on minuga tasuda kätte igaühele nõnda nagu tema tegu on!
\par 13 Mina olen A ja O, esimene ja viimne, algus ja ots!
\par 14 Õndsad need, kes oma rüüd pesevad, et neil oleks meelevald süüa elupuust ja väravaist linna sisse minna!
\par 15 Väljaspool on koerad ja nõiad ja hoorajad ja tapjad ja ebajumalateenijad ja kõik, kes valet armastavad ja teevad.
\par 16 Mina, Jeesus, läkitan oma ingli teile seda tunnistama kogudustes! Mina olen Taaveti juur ja sugu, helkjas koidutäht!”
\par 17 Ja Vaim ja pruut ütlevad: „Tule!” Ja kes seda kuuleb, öelgu: „Tule!” Ja kellel on janu, tulgu; ja kes tahab, võtku eluvett ilma hinnata!
\par 18 Mina tunnistan kõikidele, kes kuulevad selle raamatu ennustussõnu: kui keegi neile midagi juurde lisab, siis paneb Jumal tema peale nuhtlused, mis selles raamatus on kirja pandud!
\par 19 Ja kui keegi võtab midagi ära selle prohvetiraamatu sõnadest, siis Jumal võtab ära tema osa elupuust ja pühast linnast, millest on kirjutatud selles raamatus!
\par 20 Tema, kes seda tunnistab ütleb: „Tõesti, mina tulen varsti!” Aamen, tule, Issand Jeesus!
\par 21 Issanda Jeesuse Kristuse arm olgu kõikidega! Aamen.






\end{document}