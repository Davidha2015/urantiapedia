\begin{document}

\title{Jeesus Siirak}

\chapter{1}

\section*{Tarkuse saladusest}

\par 1 Palju ja tähtsat on meile antud Seaduse ja Prohvetite ning nende järeltulijate läbi, mistõttu peab Iisraeli ülistama õpetuse ja tarkuse pärast. Ometi ei pea targad olema ainult need, kes seda kõike on hästi tundma õppinud, vaid teadusejanuline olgu võimeline niihästi kõnes kui kirjas kasu tooma ka kodumaast eemal elavaile. Sellepärast minu vanaisa Jeesus, kes ennast oli pühendanud Seaduse ja Prohvetite ning meie isade muude raamatute uurimisele, saades nõnda küllaldase vilumuse, soovis ka ise kirjutada sellest, mis õpetusse ja tarkusse kuulub, et õpihimulised seda õpiksid ja nõnda üha rohkem kasvaksid Seaduse kohaseks eluks. Teid palutakse nüüd seda lugema hakata heatahtlikult ja tähelepanelikult ning olla andestajad, kui mõnikord näib, et minu hoolikalt tehtud tõlge paiguti on sõnastuselt puudulik. Nimelt ei ole heebrea keeles loetu alati just seesama teises keeles. Ja mitte ainult see raamat, vaid ka Seadus ja Prohvetid ning muud raamatud on algkeeles loetuna tunduvalt erinevad. Kui ma siis kuningas Euergetese kolmekümne kaheksandal aastal tulin Egiptusesse ja seal viibisin, leidsin eest suuresti erineva teadmiste taseme. Pidasin seepärast sel ajal vältimatuks ohverdada hoolsust ja vaeva, et suure valvsuse ja innuga valmistada selle raamatu tõlget, et seda valmis saada ja välja anda ka nende jaoks, kes võõrsil elades tahavad sellesse süveneda ja valmistuda õigesti elama Seaduse järgi. Kõik tarkus on Issandalt ja on igavesti tema juures.
\par 2 Kes suudaks lugeda mereliiva ja vihmapiisku ning igaviku päevi?
\par 3 Kes suudaks uurida taeva kõrgust, maa avarust ja allmaailma - või tarkust?
\par 4 Enne kõike on loodud tarkus, ja mõistlik mõistus on igavikust alates.
\par 5 Kõrgeima Jumala sõna on tarkuse allikas; ja tema teed on igavesed käsud.
\par 6 Kellele on ilmutatud tarkuse juur, ja kes on mõistnud tema imepäraseid tegusid?
\par 7 [Kellele on tarkuse tundmine avalikuks tehtud? ja kes on mõistnud tema suurt kogemust?]
\par 8 Üksainus on tark ja väga kardetav - tema, kes istub oma aujärjel.
\par 9 Issand ise lõi tarkuse, vaatas seda ja mõõtis ning valas selle välja kõigi oma tegude üle,
\par 10 ka kogu liha üle, nõnda kui oli ta and, ja andis seda neile, kes teda armastavad.

\section*{Jumalakartusest}

\par 11 Issanda kartus on au ja kuulsus ning hea meel ja rõõmupärg.
\par 12 Issanda kartus kosutab südant ning annab head meelt ja rõõmu ja pika ea.
\par 13 Kes Issandat kardab, sellel on lõpuks hea põli, ja oma surmapäeval on ta õnnistatud.
\par 14 Jumalakartus on tarkuse algus, see on usklikele kaasa sündinud emaihust alates.
\par 15 Tarkus rajas enesele inimeste keskel igavese aluse, mis usaldatakse nende järeltulijaile.
\par 16 Issanda kartus on täiuslik tarkus, ja see küllastab inimesi oma viljadega.
\par 17 Kallisvaraga täidab see kogu oma koja ja aidad oma andidega.
\par 18 Tarkuse kroon on Issanda kartus, see paneb õitsema rahu ja tervise.
\par 19 Issand vaatas ja mõõtis seda, laskis voolata arukust ja tarka tunnetust ning ülendas nende au, kes sellest kinni peavad.
\par 20 Tarkuse juur on Issanda kartus, ja tema võsud on pikk iga.

\section*{Kannatlikkusest ja enesevalitsusest}

\par 21 Issanda kartus ajab patud minema, ja kus see on, seal ta tõrjub viha.
\par 22 Ülekohtust viha ei saa õigustada, sest viha võimus on talle languseks.
\par 23 Kannatlik peab vastu õige ajani ja lõpuks antakse temale rõõmu.
\par 24 Õige ajani varjab ta oma sõnu, ja siis jutustavad paljude huuled tema tarkusest.

\section*{Tarkusest ja õiglusest}

\par 25 Tarkuse varade hulgas on tähendusrikkad õpetussõnad, aga patusele on jumalakartus jäleduseks.
\par 26 Kui tahad tarkust, pea käske, siis Issand annab sulle seda!
\par 27 Sest Issanda kartus on tarkus ja õpetus, ning usk ja vagadus on temale meelepärased.
\par 28 Ära ole sõnakuulmatu Issanda kartuse vastu ja ära ligine sellele kaksipidise südamega!
\par 29 Ära ole silmakirjalik inimeste ees ja pane tähele oma huuli!
\par 30 Ära ülenda iseennast, et sa ei langeks ega tooks häbi oma hingele! Sest Issand ilmutab sinu saladused ja tõukab sind maha koguduse keskel, sellepärast et sa ei ole tulnud Issanda kartusesse ja sinu süda on olnud täis valet.

\chapter{2}

\section*{Jumalakartusest kiusatustes}

\par 1 Poeg, kui tuled Issandat teenima, siis valmista oma hing kiusatuse vastu!
\par 2 Juhi oma südant ja pea vastu ja ahvatluse ajal ära tegutse ülepeakaela!
\par 3 Jää Issanda juurde ja ära tagane, et sa lõpuks saaksid suureks!
\par 4 Kõik, mis sulle juhtub, võta vastu ja ole kannatlik alanduse ajal!
\par 5 Sest kulda katsutakse läbi tules ja (Jumalale) armsaid inimesi alanduseahjus.
\par 6 Usu teda, ja ta võtab sind vastu, õgvenda oma teid ja looda tema peale!
\par 7 Teie, kes Issandat kardate, oodake tema halastust ja ärge põigelge, et te ei langeks!
\par 8 Teie, kes Issandat kardate, uskuge temasse ja teie tasu ei jää tulemata!
\par 9 Teie, kes Issandat kardate, lootke head ja igavest rõõmu ning halastust!
\par 10 Vaadake muistseid põlvkondi ja nähke: kes Issandasse uskujaist on jäänud häbisse? Või kes on püsinud tema kartuses ja on siis maha jäetud? Või kes on teda appi hüüdnud ja tema ei ole hüüdjast hoolinud?
\par 11 Sest Issand on kaastundlik ja halastaja, ta annab patud andeks ning päästab hädaajal.
\par 12 Häda argadele südametele ja lõtvadele kätele ning patusele, kes käib kaht teed!
\par 13 Häda ükskõiksele südamele! Et ta ei usu, siis teda ei kaitsta.
\par 14 Häda teile, kes olete kaotanud kannatlikkuse! Mida te teete, kui Issand tuleb nuhtlema?
\par 15 Kes Issandat kardavad, ei jäta tähele panemata tema sõnu, ja kes teda armastavad, püsivad tema teedel.
\par 16 Kes Issandat kardavad, otsivad, mis temale on meelepärane, ja kes teda armastavad, neile on küllalt Seadusest.
\par 17 Kes Issandat kardavad, valmistavad oma südant ja alandavad tema ees oma hinge:
\par 18 „Meie tahame langeda Issanda kätte, aga mitte inimeste kätte, sest nõnda suur kui on ta ise, on ka tema halastus.”

\chapter{3}

\section*{Laste kohustustest vanemate vastu}

\par 1 Lapsed, kuulge mind, oma isa, ja tehke nõnda, et teie käsi hästi käiks!
\par 2 Sest Issand tahab, et lapsed austaksid isa, ja on kinnitanud ema meelevalla poegade üle.
\par 3 Kes isast lugu peab, lepitab oma patte,
\par 4 ja ema austaja on varanduse talletajaga sarnane.
\par 5 Kes isast lugu peab, tunneb rõõmu oma lastest, ja päeval, mil ta palvetab, võetakse teda kuulda.
\par 6 Kes isa austab, sellel on pikk iga, ja kes Issanda sõna kuuleb, laseb ema puhata,
\par 7 ja ta teenib oma vanemaid otsekui isandaid.
\par 8 Teo ja sõnaga austa oma isa, et sinule tuleks tema õnnistus!
\par 9 Sest isa õnnistus kindlustab laste kodasid, ema needmine aga lammutab nende alusmüürid.
\par 10 Ära otsi enesele au oma isa häbist, sest isa häbi ei ole sulle auks!
\par 11 Sest inimese au tuleb tema isa auväärsusest ja autu ema on lastele teotuseks.
\par 12 Laps, aita oma isa tema vanas eas ja ära kurvasta teda, niikaua kui ta elab!
\par 13 Ja isegi kui ta mõistus kahaneb, siis saa sellest aru, ja ära põlga teda, kui ise oled täies jõus!
\par 14 Sest heategu isa vastu ei unustata ja arvestatakse sinu kasuks, olgugi sul patte.
\par 15 Sinu viletsuspäeval meenutatakse sind: nagu päikesepaistel jää, nõnda sulavad su patud.
\par 16 Isa hülgaja on otsekui jumalateotaja ja Issand neab seda, kes oma ema vihastab.

\section*{Alandlikkusest}

\par 17 Laps, oma tööd tee tasase meelega, siis armastavad sind Jumalale armsad inimesed!
\par 18 Mida suurem sa oled, seda rohkem alanda ennast, siis leiad armu Issanda ees!
\par 19 Paljud on kõrgel kohal ja kuulsad, kuid alandlikele on ilmutatud saladused.
\par 20 Sest Issanda võim on suur ja alandlikud austavad teda.
\par 21 Ära otsi, mis sulle on liiga raske, ja ära uuri, mis sulle ei ole jõukohane!
\par 22 Mõtle sellele, mida sul on kästud teha, sest mis on salajas, see ei ole sinu jaoks!
\par 23 Mis ei ole sinu töö, sellega ära ennast asjata vaeva, sest sinule on ju näidatud rohkem, kui inimesed suudavad mõista!
\par 24 Sest paljusid on eksitanud nende enda arvamus ja vale kujutlus on rikkunud nende mõtlemise.
\par 25 Ilma silmadeta tahad sa valgust: ära tunnista seepärast teadmist, mida sul ei ole.

\section*{Kõrkusest}

\par 26 Kõval südamel läheb lõpuks halvasti ja kes armastab hädaohtu, see hukkub selles.
\par 27 Kõva süda koormatakse valudega ja patune lisab patule patu.
\par 28 Katsumus ei ole kõrgile raviks, sest temas on juurdunud kurjuse taim.
\par 29 Mõistliku süda mõtiskleb tähendamissõnade üle ja targa igatsuseks on kuulev kõrv.

\section*{Armastus vaeste vastu}

\par 30 Vesi kustutab leegitseva tule ja heategevus lepitab patte.
\par 31 Kes heategusid tasub, mõtleb tulevikule ja leiab enesele tuge languse ajal.

\chapter{4}

\section*{Armastus vaeste vastu}

\par 1 Poeg, ära keela elatist kerjusele ja ära tõrju tagasi kehvade silmi!
\par 2 Näljast hinge ära kurvasta ja ära ärrita abitut inimest tema hädas!
\par 3 Kibestunud südant ära vaeva veelgi rohkem ja ära lase hädalist oodata andi!
\par 4 Ahastavat palujat ära lükka tagasi ja ära pööra oma palet vaese pealt!
\par 5 Ära pööra oma silma palujalt ja ära anna kellelegi põhjust sind sajatada!
\par 6 Sest kui ta oma hingekibeduses sind sajatab, siis tema Looja kuuleb ta palvet.
\par 7 Tee ennast kogudusele armsaks ja langeta pea suure isanda ees!
\par 8 Pööra oma kõrv vaese poole ja vasta temale sõbraliku rahuga!
\par 9 Päästa rõhutu rõhuja käest ja ära ole arg, kui mõistad kohut!
\par 10 Ole orbudele isaks ja nende emale mehe asemel! Siis oled sa kui Kõigekõrgema poeg, ja tema armastab sind rohkem kui su oma ema.

\section*{Tarkuse viljast}

\par 11 Tarkus ülendab oma poegi ja võtab vastu need, kes teda otsivad.
\par 12 Kes armastab teda, armastab elu, ja kes aegsasti tulevad tema juurde, need täidetakse rõõmuga.
\par 13 Kes temast kinni hoiab, pärib au, ja kus ta käib, seal õnnistab teda Issand.
\par 14 Kes teenivad teda, teenivad Püha, ja kes teda armastavad, neid armastab Issand.
\par 15 Kes teda kuulda võtab, mõistab rahvaile kohut, ja kes teda tähele paneb, võib julgesti elada.
\par 16 Kes ennast tema hooleks annab, on tema pärija, ja selle järeltulijadki on tema omandiks.
\par 17 Aga alguses käib tarkus temaga kõveraid teid, toob temale kartust ja argust, piinab teda oma karistusega ja proovib oma käskude läbi, kuni see hakkab usaldama tema hinge.
\par 18 Siis tuleb tarkus jälle otseteed tema juurde, rõõmustab teda ja ilmutab temale oma saladusi.
\par 19 Aga kui ta eksib, siis tarkus hülgab ta ja laseb tal langeda.

\section*{Valehäbist ja inimeste kohtlemisest}

\par 20 Pane tähele aega ja hoidu kurjast, et sa ei peaks häbenema oma hinge pärast!
\par 21 Sest on häbi, mis toob kaasa patu, ja on häbi, mis tähendab au ning armu.
\par 22 Ära ole erapoolik oma hinge kahjuks ja ära häbene oma languseks!
\par 23 Kui vaja, ära säästa sõna,
\par 24 sest sõnast tuntakse tarkust ja keele kõnest õpetust!
\par 25 Ära räägi tõele vastu, vaid häbene oma teadmatuse pärast!
\par 26 Ära häbene tunnistada oma patte ja ära püüa peatada jõe voolu!
\par 27 Ära alistu rumalale inimesele ja ära ole vägeva poolt!
\par 28 Tõe pärast võitle kuni surmani, siis Issand Jumal sõdib sinu eest!
\par 29 Ära ole hoopleva keelega ega laisk ja loid oma tegudes!
\par 30 Ära ole oma kodus nagu lõvi ja oma kodakondsete keskel tontide nägija!
\par 31 Ärgu olgu su käsi sirutatud võtmiseks ja suletud andmiseks!

\chapter{5}

\section*{Rikkusest ja ülbusest}

\par 1 Ära looda oma varanduse peale, ja ära ütle: „Minul on küllalt!”
\par 2 Ära järgi oma hinge ja oma jõudu, nii et käid oma südame himude järele!
\par 3 Ära ütle: „Kes on minu käskija?”, sest Issand nuhtleb sind tõesti!
\par 4 Ära ütle: „Ma olen pattu teinud, aga mis on mul sellest?”, sest Issand on pikameelne!
\par 5 Ära ole julge andestuse peale, et lisad pattudele patu!
\par 6 Ja ära ütle: „Tema halastus on suur, tema annab andeks minu rohked patud”, sest temal on halastus ja viha, ja tema meelepaha on patuste peal!
\par 7 Pöördu viivitamata Issanda poole ja ära lükka seda edasi päevast päeva, sest Issanda viha tuleb äkitselt, ja karistusajal sa hukkud!
\par 8 Ära looda ülekohtuse varanduse peale, sest sellest ei ole sul abi kohtupäeval!

\section*{Kindlameelsusest ja enesevalitsusest}

\par 9 Ära lase ennast tuulata igast tuulest ja ära käi igal rajal, sest nõnda teeb kahekeelne patune!
\par 10 Jää kindlaks oma veendumuses ja sinu kõne olgu alati üks!
\par 11 Ole kärme kuulama, aga vastus anna kaalutletult!
\par 12 Kui sa midagi tead, siis vasta ligimesele, aga kui mitte, siis pane käsi suu peale!
\par 13 Kõnes on au ja häbi, ja inimese keel võib saada hukatuseks temale.
\par 14 Hoidu, et sind ei peetaks kõrvapuhujaks, ja ära varitse oma keelega! Sest vargale saab osaks häbi ja kahekeelsele raske hukkamõist!
\par 15 Ära eksi suuremas ega väiksemas ja sõbrast ära muutu vaenlaseks!

\chapter{6}

\section*{Kindlameelsusest ja enesevalitsusest}

\par 1 Sest seesugune pärib halva nime, häbi ja teotuse - nõndasamuti ka kahekeelne patune.

\section*{Hoiatus ohjeldamatuse eest}

\par 2 Ära lase ennast eksitada himust, et see ei rüüstaks su hinge otsekui härg:
\par 3 sa sööd oma lehed ja hävitad oma viljad ning jätad iseenese kuivanud puuks.
\par 4 Kõlvatu hing hävitab inimese, kelle sees ta on, ja teeb ta vaenlaste naerualuseks.

\section*{Õigest ja valest sõprusest}

\par 5 Sõbralik suu teeb enesele palju sõpru ja hästi kõnelev keel saab palju häid vastuseid.
\par 6 Olgu küll palju neid, kes sinuga rahus elavad, aga üksainus tuhandeist olgu sulle nõuandjaks!
\par 7 Kui leiad enesele sõbra, siis pane ta proovile ja ära teda otsekohe usalda!
\par 8 Sest mõni on küll sõber, niikaua kui temale meeldib, aga ei jää selleks, kui sinul on raske aeg.
\par 9 Mõni sõber muutub vaenlaseks ja avalikustab riiu sinule teotuseks.
\par 10 Mõni on sõber lauakaaslasena, aga ei jää selleks, kui sinul on raske aeg:
\par 11 sinu õnnepõlves sarnaneb ta sinuga ja kohtleb siiralt sinu kodakondseid,
\par 12 aga kui su käsi halvasti käib, siis on ta sinu vastu ja peidab ennast sinu palge eest.
\par 13 Vaenlastest hoia eemale ja sõprade ees ole valvel!
\par 14 Ustav sõber on võimas kaitse, kes leiab selle, leiab varanduse.
\par 15 Ustav sõber on asendamatu ja miski ei kaalu üles tema väärtust.
\par 16 Ustav sõber on elu võlurohi, ja kes Issandat kardavad, leiavad niisuguse.
\par 17 Kes Issandat kardab, peab õiget sõprust, ja missugune ta ise on, niisugune on ka tema ligimene.

\section*{Manitsus tarkuse taotlemiseks}

\par 18 Laps, juba noorpõlves taotle haridust, siis leiad tarkust kuni raugaeani!
\par 19 Astu selle juurde otsekui kündja ja külvaja ja oota selle häid vilju! Selles töös on küll natuke vaeva, aga juba varsti saad sa süüa saaki.
\par 20 Jah, väga vaevaline on see harimatuile ja jõuetu ei püsi selle juures.
\par 21 See on tema peal raske katsekivina ja ta viskab selle ära jalamaid.
\par 22 Sest tarkus on oma nime kohane, temast ei saa paljud aru.
\par 23 Kuule, laps, ja võta vastu minu ettepanek ning ära põlga minu nõuannet!
\par 24 Pane oma jalad tarkuse kammitsaisse ja kael tema kaelarauda!
\par 25 Painuta oma õlgu ja kanna teda, ja ära tõrgu tema köidikute vastu!
\par 26 Mine tema juurde kogu hingega ja püsi tema teedel kõigest jõust!
\par 27 Otsi ja uuri, siis ta saab sulle tuttavaks, ja kui ta sul käes on, siis ära teda lahti lase!

\section*{Tarkuse kasust}

\par 28 Sest lõpuks sa leiad tarkuses rahu ja ta saab sulle rõõmuks.
\par 29 Siis on tema ahelad sulle tugevaks kaitseks ja tema kaelarauad aurüüks.
\par 30 Sest temal on kuldehe peas ja mähiseks sinine side.
\par 31 Aurüüna sa kannad teda ja paned kui rõõmupärja enesele pähe.
\par 32 Kui tahad, laps, siis õpetatakse sind, ja kui annad oma hinge, saad teadjaks!
\par 33 Kui armastad kuulata, siis õpid, ja kui pöörad oma kõrva, saad targaks.
\par 34 Astu vanemate hulka, ja kes tark on, sellega seltsi!
\par 35 Kõiki jumalikke kuulutusi kuula meelsasti ja mõistlikud õpetussõnad ärgu jäägu sinust eemale!
\par 36 Kui näed arusaajat, siis mine aegsasti tema juurde ja sinu jalg kulutagu tema lävesid!
\par 37 Pea meeles Issanda korraldusi ja tema käskude pärast ole alati mures! Tema kinnitab sinu südant ja sinu tarkuseihalus leiab rahuldust.

\chapter{7}

\section*{Manitsus õigeks käitumiseks ühiskonnas}

\par 1 Ära tee kurja, siis ei sünni kurja ka sinule!
\par 2 Hoidu ülekohtust, siis põgeneb see sinu juurest!
\par 3 Mu poeg, ära külva ülekohtu vagudesse, sest lõikama pead seda seitsmekordselt!
\par 4 Ära nõua Issanda käest ülemvalitsust ega kuningalt aujärge!
\par 5 Issanda ees ära pea ennast õigeks ja kuninga ees ära targuta!
\par 6 Ära püüa saada kohtumõistjaks: sina ehk ei suuda väärtegusid lõpetada; võib-olla kardad võimukandja palet ja annad oma õiglusele häbimärgi.
\par 7 Ära tee pattu linna elanike vastu ja ära alanda iseennast rahvahulga ees!
\par 8 Ära korda pattu, sest üheainsagi pärast ei jää sa karistuseta!
\par 9 Ära ütle: „Tema vaatab minu andide rohkuse peale, ja kui toon need Jumalale, Kõigekõrgemale, siis ta võtab need vastu.”
\par 10 Palvetades ära ole pelglik ja ära unusta almuste andmist!
\par 11 Ära pilka inimest, kellel on hingevalu, sest üks on, kes alandab ja ülendab!
\par 12 Ära külva valet oma venna vastu, ära tee seda ka oma sõbrale!
\par 13 Ära kunagi taha valet rääkida, sest sellega harjumine ei ole hea!
\par 14 Ära ole lobiseja vanade seltskonnas, ja oma palves ära korruta sõnu!
\par 15 Ära põlga töö vaeva, ära põlga põllutööd, mis on Kõigekõrgema poolt seatud!
\par 16 Ära lase ennast lugeda patuste kilda, pea meeles, et karistus ei viibi!
\par 17 Alanda oma hinge üliväga, sest tuli ja uss on jumalakartmatu karistuseks!

\section*{Õigest perekonnaelust}

\par 18 Ära vaheta sõpra raha vastu ega ausat venda Oofiri kulla vastu!
\par 19 Ära lahku targast ja heast naisest, sest tema armsus on ülem kui kuld!
\par 20 Ära tee paha ustavalt töötavale orjale ega palgalisele, kes annab oma hinge!
\par 21 Sinu hing armastagu mõistlikku orja, ära keela temale vabadust!
\par 22 On sul kariloomi, siis hoia neid, ja kui sulle on kasulik, siis pea neid edasi!

\section*{Laste kasvatamisest}

\par 23 On sul lapsi, siis kasvata neid, ja pane nad tööle noorpõlvest alates!
\par 24 On sul tütreid, siis valva nende ihu ja ära näita neile liiga lahket nägu!
\par 25 Tütart mehele pannes sooritad suure teo, aga anna ta mõistlikule mehele!
\par 26 On sul meelepärane naine, siis ära hülga teda, aga põlatule ära ennast usalda!

\section*{Vanemate austamisest}

\par 27 Austa oma isa kõigest südamest ja ära unusta oma ema sünnitusvalusid!
\par 28 Pea meeles, et oled neist sündinud - ja kuidas sa tasud neile, mis nemad on sinule teinud?
\section*{Ohvriannid Issanda teenreile}

\par 29 Karda Issandat kõigest hingest ja austa tema preestreid!
\par 30 Armasta oma Loojat kõigest jõust ja ära jäta maha tema teenreid!
\par 31 Karda Issandat ja austa preestrit ja anna temale, nagu sind on kästud, osa uudseviljast, süüohvrist, tõstelõivust, pühitsusohvrist ja anna esmaand sellest, mis on püha!

\section*{Vaeste ja abitute aitamisest}

\par 32 Siruta oma käsi ka vaesele, et sinu õnnistus oleks täielik!
\par 33 Helde and kõigile elavaile - aga ka surnule ära keela heldust!
\par 34 Ära puudu nutjate juurest ja leina leinajatega!
\par 35 Ära kõhkle külastamast haiget inimest, sest kui sa seda teed, siis armastatakse sind!
\par 36 Kõigis tegudes mõtle oma lõpule, siis sa ei tee iialgi pattu!

\chapter{8}

\section*{Hoiatus ohtliku riiu eest}

\par 1 Ära võitle vägeva inimesega, et sa ei langeks tema kätte!
\par 2 Ära riidle rikka inimesega, et ta ei seaks sulle vastu oma kaalu: sest kuld on paljusid hukutanud ja kuningate südameid pööranud!
\par 3 Ära vaidle sõnaka inimesega ja ära kuhja puid tema lõkkele!
\par 4 Ära naljata harimatu inimesega, et sinu esivanemaid ei teotataks!
\par 5 Ära sõitle inimest, kes patust pöördub, pea meeles, et meie kõik oleme süüalused!
\par 6 Ära halvusta inimest, kui ta on vana, sest meistki mõni saab vanaks!
\par 7 Ära tunne rõõmu kellegi surmast, mõtle, et me kõik peame surema!

\section*{Tarku mehi tuleb kuulata}

\par 8 Ära põlga tarkade juttu, ja käi nende õpetussõnade järgi, sest neilt sa õpid tarkust ja suurte isandate teenimist!
\par 9 Ära väldi vanade jutustusi, sest ka nemad on õppinud oma vanemailt, sest neilt sa õpid arukust ja vastuse andmist siis, kui on vaja!

\section*{Inimsuhetest}

\par 10 Ära puhu lõkkele patuse süsi, et sina ise ei põleks tema tuleleegis!
\par 11 Ära kaota enesevalitsust jultunu ees, et tema ei saaks seada otsekui lõksu sinu suhu!
\par 12 Ära laena inimesele, kes on sinust vägevam, ja kui oled laenanud, siis pea seda kadunuks!
\par 13 Ära käenda üle oma jõu, ja kui oled käendanud, siis mõtle maksmisele!
\par 14 Ära käi kohut kohtumõistjaga, sest tema üle mõistetakse kohut seisusekohaselt!
\par 15 Hulljulgega koos ära mine teele, et ta ei muutuks sulle tülikaks, sest ta teeb, mida tahab, ja tema mõtlematuse tõttu hukkud sinagi!
\par 16 Ära mine riidu äkilise mehega, ja ära mine koos temaga kõrbeteekonnale, sest veri ei maksa tema silmis midagi, ja kus pole abi, seal ta lööb sinu maha!
\par 17 Rumalaga ära pea nõu, sest tema ei suuda midagi salajas hoida!
\par 18 Võõra nähes ära tee midagi salajast, sest sa ei tea, mida tema mõtleb teha!
\par 19 Ära puista oma südant igaühele, tänu sa selle eest ei saa!

\chapter{9}

\section*{Naiste kohtlemisest}

\par 1 Ära ole armukade naisele, kes su süles on, ja ära anna temale halba õpetust, kahjuks sinule enesele!
\par 2 Ära anna naisele oma hinge, et ta ei tõuseks sinu võimu vastu!
\par 3 Ära seltsi ligitikkuva naisega, et sa ei langeks tema paeltesse!
\par 4 Ära ole kaua laulunaise juures, et sind ei vangistataks tema riugaste läbi!
\par 5 Ära heida silma neitsi peale, et sul ei tuleks pahandust tema kahjutasu pärast!
\par 6 Ära anna oma hinge hooradele, et sa ei raiskaks ära oma pärandit!
\par 7 Ära piilu linnatänavail ja ära luusi seal üksildasis paigus!
\par 8 Pööra pilk ära kaunilt naiselt ja ära silmitse võõrast ilu! Naise ilu on eksitanud paljusid ja sellest süttib armastus nagu tuli.
\par 9 Abielunaise seltsis ära istu iialgi ja ära pidutse koos temaga veini juures, et su hing ei kalduks tema poole ja sa vaimustuses ei langeks hukatusse!

\section*{Vana sõber}

\par 10 Vana sõpra ära jäta maha, sest uus ei ole temaga sarnane! Uus sõber - värske vein: kui see vanaks saab, siis jood seda hea meelega.

\section*{Hoiatus jumalakartmatute eest}

\par 11 Ära kadesta patuse hiilgust, sest sa ei tea, missugune on tema lõpp!
\par 12 Ära vaimustu jumalakartmatute õnnest, pea meeles, et nad enne hauda ei jää karistamata!
\par 13 Hoidu inimesest, kellel on võim tappa, siis pole sul tarvis tunda surmahirmu! Aga kui tuled tema lähedale, siis ära eksi, et ta ei võtaks sinult elu! Pea meeles, et käid keset püüniseid ja kõnnid linna müüripealseil!

\section*{Suhtlemisest tarkade ja õigetega}

\par 14 Jõudumööda õpi tundma oma ligimesi ja tarkadega pea nõu!
\par 15 Kes on mõistlikud, nendega räägi, ja kõik su kõne olgu Kõigekõrgema Seadusest!
\par 16 Õiglased mehed olgu sinu lauakaaslased ja Issanda kartuses seisnegu sinu kiitus!
\par 17 Tööd kiidetakse tegija käe järgi ja rahvajuhti targaks tema kõne pärast.
\par 18 Kardetud oma linnas on lobiseja mees, ja kõnes järelemõtlematut vihatakse.

\chapter{10}

\section*{Valitsejad olgu targad}

\par 1 Tark kohtumõistja õpetab rahvast ja mõistliku valitsus on hästi seatud.
\par 2 Milline on rahva kohtumõistja, sellised on ka tema sulased, ja milline on linna valitseja, sellised on ka kõik selle elanikud.
\par 3 Rumal kuningas hävitab oma rahva, linn ehitatakse aga valitsejate tarkusega.
\par 4 Issandal on meelevald maa üle, ja selle valitsejaks ta äratab tubli mehe õigel ajal.
\par 5 Mehe edukus on Issanda käes ja tema annab ametikandjale aupaiste.
\par 6 Ära ole ligimese peale vihane mitte ühegi kuriteo pärast ja vägivallategudele ära vasta!

\section*{Hoiatus ülbuse eest}

\par 7 Ülbus on vihatav Issanda ja inimeste ees ja ülekohus on patt mõlemate meelest!
\par 8 Valitsusvõim läheb rahvalt rahvale ülekohtu, vägivalla ja saamahimu pärast.
\par 9 Miks suurustab põrm ja tuhk, tema, kelle sisikond lüüakse segi juba ta eluajal?
\par 10 Pikaldane haigus pilkab arsti: täna kuningas, homme surnud.
\par 11 Sest kui inimene sureb, siis on tema pärisosaks roomajad, metsloomad ja ussid.
\par 12 Ülbuse alguseks on inimese taganemine Issandast ja tema südame loobumine oma Loojast.
\par 13 Sest ülbuse alguseks on patt, ja kes sellest kinni peab, sellest voolab nurjatust. Seepärast saadab Issand temale ootamatuid karistusi ning viimaks hävitab tema täiesti.
\par 14 Issand kummutab valitsejate aujärjed ja paneb alandlikud nende asemele.
\par 15 Issand juurib välja rahvad ja istutab alandlikud nende asemele.
\par 16 Issand lööb segi rahvaste asupaigad ja hävitab need maa põhjani.
\par 17 Ta viib sealt ära elanikud ning hävitab need ja kaotab maa pealt nende mälestuse.
\par 18 Uhkus ei ole loodud inimeste jaoks ega äge viha naisest sündinuile.

\section*{Õigest ja valest aust}

\par 19 Missugune sugu on auväärne? Inimsugu. Missugune sugu on auväärne? Kes Issandat kardavad. Missugune sugu on autu? Inimsugu. Missugune sugu on autu? Kes käskudest üle astuvad.
\par 20 Vendade keskel on auväärne nende juht, Issanda silmis aga need, kes teda kardavad.
\par 21 Issanda kartus käib enne võimu saamist, aga karmus ja uhkus on selle kaotamine.
\par 22 Rikas ja kuulus ja vaene - nad kiidelgu Issanda kartusest!
\par 23 Ei ole õige põlata tarka vaest ega sünnis austada patust meest.
\par 24 Väga suurt meest, kohtumõistjat ja valitsejat, austatakse, aga ükski neist ei ole suurem kui see, kes Issandat kardab.
\par 25 Tarka orja teenivad vabad, aga arukas mees ei nurise selle üle.
\par 26 Ära targuta, kui teed oma tööd, ja ära kiitle, kui oled kitsikuses!
\par 27 Parem on see, kes tööd teeb ja kellel on kõike küllalt, kui see, kes elab kiideldes, aga leivapuuduses.
\par 28 Laps, alandusega austa oma hinge ja hinda teda tema tõelise väärtuse järgi!
\par 29 Kes saab õigeks mõista seda, kes patustab oma hinge vastu? Ja kes austab seda, kes teotab oma elu?
\par 30 Vaest austatakse tema tarkuse pärast ja rikast austatakse tema rikkuse pärast.
\par 31 Kui kedagi austatakse vaesuses, kuivõrd rohkem siis rikkuses? Ja kes on austamata rikkuses, kuivõrd rohkem siis vaesuses?

\chapter{11}

\section*{Õigest ja valest aust}

\par 1 Alandliku tarkus tõstab üles ta pea ja paneb ta istuma suurte seltsi.

\section*{Õigest kiitusest}

\par 2 Ära kiida meest tema ilu pärast ja ära põlga inimest tema välimuse pärast:
\par 3 mesilane on tiivuliste hulgast üks väiksemaid, ometi on tema töövili magusaist magusaim.
\par 4 Ära uhkelda riietusega, mida kannad, ja oma aupäeval ära suurusta, sest Issanda teod on imelised ja tema tegevus on inimestele varjatud!
\par 5 Mitu vürsti on pidanud istuma põrandale, aga mõni, kellest ei arvatudki, on kandnud krooni.
\par 6 Palju vägevaid on väga teotatud ja auväärseid on antud teiste kätte.

\section*{Manitsus ettevaatlikkusele}

\par 7 Ära laida enne, kui oled uurinud, enne mõtle järele ja siis karista!
\par 8 Ära vasta enne, kui oled kuulanud, ja keset kõnet ära räägi vahele!
\par 9 Ära riidle asja pärast, mis sinusse ei puutu, ja ära ole kaasistujaks, kui patused kohut mõistavad!
\par 10 Laps, ära tee mitut asja ühekorraga, sest kui sa palju ette võtad, siis sa ei jää karistuseta: kuigi püüad, sa ometi ei saavuta, ja kui tahadki põgeneda, sa ei pääse!

\section*{Inimene ei ole oma elu peremees}

\par 11 Mõni rabeleb, näeb vaeva ja kiirustab, ometi jääb maha seda rohkem.
\par 12 Mõni on pikaldane ja vajab abi, jõudu on vähe ja vaesust on küllalt, Issanda silmad vaatavad ometi armsalt tema peale ja ta ülendab teda tema alandusest.
\par 13 Jah, ta tõstab üles tema pea ning paljud imetlevad teda.
\par 14 Hea ja paha, elu ja surm, vaesus ja rikkus on Issandalt.
\par 15 Tarkus, teadmine ja käsumõistmine on Issandalt, armastus ja heade tegude tee on temalt.
\par 16 Eksitus ja pimedus said alguse ühest patustega ja kurjus vananeb nendega, kes sellega hiilgavad.
\par 17 Issanda and jääb vagadele ja tema hea tahe toob alati kordamineku.
\par 18 Mõni rikastub oma ihnsuse ja ahnuse tõttu, aga tasu, mis ta selle eest saab, on see -
\par 19 ta ütleb küll: „Ma olen rahu leidnud ja nüüd ma söön oma varandusest”, ometi ta ei tea, missugune aeg tuleb, mil ta peab selle jätma teistele ja ise surema.
\par 20 Püsi oma ametis ja ela selles ning saa vanaks oma töös!
\par 21 Ära imetle patuse tegusid, usu Issandasse ja jää oma töö juurde, sest Issandal on kerge teha vaene äkitselt ja kiiresti rikkaks.
\par 22 Vagale on tasuks õnnistus Issandalt, kes üürikese ajaga paneb oma õnnistuse õitsema.
\par 23 Ära ütle: „Mida ma veel vajan ja mis võiks mulle nüüd veel heaks tulla?”
\par 24 Ära ütle: „Minul on küllalt, ja mis halba võiks mulle nüüd sündida?”
\par 25 Heal päeval unustatakse halb ja halval päeval ei meenutata head.
\par 26 Issandal on aga kerge tasuda inimesele tema surmapäeval tema käitumise kohaselt.
\par 27 Halb aeg paneb unustama heaolu ja inimese elu lõpp teeb avalikuks tema teod.
\par 28 Ära kiida kedagi õndsaks enne tema surma, meest tuntakse alles tema lastest!

\section*{Ettevaatust suhtlemises ligimesega!}

\par 29 Ära vii iga inimest oma kotta, sest kavalal on palju lõkse!
\par 30 Peibutuslind puuris - samasugune on jultunud inimese süda: ta varitseb hukatuseks otsekui salakuulaja.
\par 31 Sest kavalasti pöörab ta hea halvaks ja paneb parimailegi külge häbipleki.
\par 32 Sädemest süttib tulelõõm: patune inimene varitseb verd.
\par 33 Hoia ennast kelmi eest, sest ta sepitseb kurja, et ta ei paneks sulle külge häbiplekki igaveseks ajaks!
\par 34 Kotta elama võetud võõras teeb tüli ja võõrutab sind sinu omastest.

\chapter{12}

\section*{Ettevaatust suhtlemises ligimesega!}

\par 1 Kui sa head teed, siis vaata, kellele sa teed, et saaksid tänu oma heategude eest!
\par 2 Tee head vagale, siis saad tasu, ja kui mitte temalt, siis Kõigekõrgemalt!
\par 3 Heateod ei ole sellele, kes püsib kurjuses, ega sellele, kes heategude eest ei ole tänulik.
\par 4 Anna vagale, aga ära aita patust!
\par 5 Alandlikule tee head, aga jumalakartmatule ära anna, keela temale leiba ja ära anna temale midagi, et ta seeläbi ei saaks võimust sinu üle! Sest sa leiad kahekordselt kurja kõige hea eest, mida sa temale teed.
\par 6 Sest Kõigekõrgemgi vihkab patuseid ja karistab jumalakartmatuid.
\par 7 Anna heale, aga ära aita patust!

\section*{Õige ja vale sõber}

\par 8 Sõpra ei õpita tundma õnnes, ja õnnetuses ei jää vihamees tundmatuks.
\par 9 Kui mehe käsi hästi käib, siis tema vihamehed on murelikud, aga halbadel päevadel lahkub temast sõbergi.
\par 10 Ära iialgi usu oma vihameest, sest otsekui raud roostetab, nii on tema kurjus!
\par 11 Isegi kui ta ennast alandab ja küürus käib, ole valvel ja hoia ennast tema eest! Sina ole temale otsekui metallpeegli poleerija: siis sa näed, et tema rooste ei ole jäädav!
\par 12 Ära võta teda enese kõrvale, et ta sind ei kõrvaldaks ega asuks sinu kohale! Ära pane teda istuma oma paremale käele, et ta ei hakkaks ihaldama sinu istet, nõnda et sa viimaks mõistad minu sõnu ja minu hoiatused pistavad sind valusasti!
\par 13 Kellel on kaastunnet ussist salvatud lausuja vastu ja kõigi nende vastu, kes tegemist teevad metsloomadega?
\par 14 Nõnda on lugu ka sellega, kes läheb patuse mehe juurde ja segab end tema pattudesse.
\par 15 Mõneks ajaks jääb ta sinu juurde, aga kui hakkad langema, siis ta ei pea vastu.
\par 16 Vihamehe huulil on magusust, aga südames soovib ta sind tõugata kuristikku. Vihamehel on nutt silmis, aga kui ta leiab paraja juhuse, siis ta ei küllastu verest.
\par 17 Kui sulle juhtub õnnetus, siis leiad, et tema on kohal enne kui sina ise. Ja nagu tahtes aidata, paneb sulle jala taha.
\par 18 Ta vangutab pead ja plaksutab käsi, ta räägib taga ja moonutab nägu.

\chapter{13}

\section*{Hoiatus seltsimise eest rikka ja vägevaga}

\par 1 Kes puudutab pigi, määrib ennast, ja kes seltsib ülbega, saab temaga sarnaseks.
\par 2 Üle jõu käivat koormat ära kanna ja ära seltsi vägevama ja rikkamaga! Mis on potil tegemist pajaga? Üks tõukab ja teine puruneb.
\par 3 Rikas teeb ülekohut ja pealegi ähvardab, vaene kannatab ülekohut ja peab veel teda anuma.
\par 4 Kui oled kasulik, siis ta vajab sind, aga kui kannatad puudust, siis ta loobub sinust.
\par 5 Kui sul midagi on, siis ta elab koos sinuga, koorib sind, aga ise ei tööta.
\par 6 Kui ta sind vajab, siis ta petab sind, naeratab sulle ja annab lootust, räägib sinuga kenasti ning küsib: „Mis on sul vaja?”
\par 7 Vastukutset ootavate söömingutega häbistab ta sind kaks või kolm korda, kuni ta on su laostanud. Ja lõpuks ta pilkab sind. Hiljem, kui ta sind näeb, ei tee ta sinust väljagi, vaid vangutab sinu pärast pead.
\par 8 Hoia, et sind ei petetaks ega alandataks sinu mõtlematuses!
\par 9 Kui võimukandja sind kutsub, siis ole tagasihoidlik - seda enam ta sind kutsub!
\par 10 Ära torma, et sind ei tõrjutaks, ja ära jää liiga kaugele, et sind ei unustataks!
\par 11 Ära püüa kõnelda temaga kui omasugusega ja ära usu tema sõnaohtrust! Sest suure lobisemisega paneb ta sind proovile ja sulle naeratades teeb ta oma otsuse -
\par 12 halastamatu, kes ei säästa sõnu, ei väldi julma kohtlemist ega ahelaid.
\par 13 Ole valvel ja pane hästi tähele, sest sa oled lähedal oma langusele!
\par 14 Armasta Issandat kogu oma elu ja kutsu teda päästma.

\section*{Manitsus seltsimiseks omasugusega}

\par 15 Iga elusolend armastab omasugust ja iga inimene oma ligimest.
\par 16 Kõik elav ühineb oma liikide järgi ja inimene hoiab omataolise poole.
\par 17 Mis ühist on hundil lambaga või patusel vagaga?
\par 18 Mis rahu on hüäänil koeraga? Ja mis rahu on rikkal vaesega?
\par 19 Metseeslid kõrbes on lõvide saagiks, nõndasamuti on vaesed rikaste karjamaaks.
\par 20 Alandlikkus on ülbele vastik, nõndasamuti on vaene vastik rikkale.

\section*{Rikkusest}

\par 21 Rikast, kes vangub, toetavad sõbrad, aga kui kehv langeb, tõukavad teda veel sõbradki.
\par 22 Kui rikas vääratab, siis on temal palju aitajaid, kui ta räägib lubamatut, siis teda õigustatakse. Kui kehv vääratab, siis teda veel sõideldaksegi, ja kuigi ta targalt kõneleb, ei anta temale võimalust.
\par 23 Kui rikas räägib, siis vaikivad kõik ja tema sõna tõstetakse pilvedeni. Kui vaene räägib, siis küsitakse: „Kes ta on?” Ja kui ta komistab, siis teda tõugatakse veelgi.
\par 24 Hea on rikkus, mille juures ei ole pattu, ja vaesus on halb jumalakartmatute suus.
\par 25 Inimese süda muudab tema näo kas heaks või kurjaks.
\par 26 Õnneliku südame märk on rõõmus nägu, aga õpetussõnade leidmine nõuab vaevarikast mõtisklust.

\chapter{14}

\section*{Kes on õnnis?}

\par 1 Õnnis on mees, kes suuga ei libastu ja keda ei piina mure pattude pärast.
\par 2 Õnnis on see, keda tema oma hing ei süüdista ja kes ei ole kaotanud lootust.
\section*{Hoiatus ahnuse ja ihnsuse eest}

\par 3 Ihnsale mehele ei ole rikkus hea ja milleks kadedale inimesele raha?
\par 4 Kes kogub iseennast ilma jättes, kogub teistele, ja tema varandust priiskavad võõrad.
\par 5 Halb iseenese vastu - kelle vastu ta on siis hea? Iialgi ei tunne ta rõõmu oma varandusest.
\par 6 Ei ole midagi halvemat kui see, et ei raatsita midagi lubada iseenesele, ja see ongi tema pahateo palk.
\par 7 Kui tema ka head teeb, siis teeb ta seda kogemata, ja lõpuks ta näitab oma kurjust.
\par 8 Paha on see, kelle silmas on kadedus, kes näo ära pöörab ja teisi põlgab.
\par 9 Ahnitseja silm ei rahuldu oma osaga, ja nurjatu ülekohus kuivatab hinge.
\par 10 Kuri silm on kade leivagi pärast ja tunneb puudust omaenese lauas.

\section*{Elust tuleb rõõmu tunda}

\par 11 Laps, võimalust mööda tee head iseenesele ja vii Issandale väärikaid ohvreid!
\par 12 Pea meeles, et surm ei viivita, ja surmavalla lepingut ei ole sulle näidatud!
\par 13 Enne kui sured, tee sõbrale head, jõudumööda siruta käsi ja anna temale!
\par 14 Ära keela enesele head päeva ja osa õigest lõbust ärgu mingu sinust mööda!
\par 15 Eks sinul tule jätta oma töövili teisele ja vaevaga saadu jaotamiseks liisu läbi?
\par 16 Anna ja võta ja meelita oma hinge, sest surmavallas ei otsita enam mõnu.

\section*{Üldisest kaduvusest}

\par 17 Kõik liha vananeb nagu kuub, sest algusest peale on seadus: „Sa pead surma surema!”
\par 18 Otsekui haljad lehed kaharal puul: ühed varisevad, teised puhkevad - nõnda on ka liha ja vere sooga: üks sureb, teine sünnib.
\par 19 Kõik tehtu kõduneb ja lõpeb, ja tegija kaob koos sellega.

\section*{Tarkuse taotlemise kasust}

\par 20 Õnnis on mees, kes taotleb tarkust ja mõistusega arutleb,
\par 21 kes südames mõtleb tarkuse teedele ja mõtiskleb tema saladuste üle.
\par 22 Tema järel mine välja nagu piilur ja varitse tema teedel!
\par 23 Kes tema aknaist sisse vaatab ja kuulutab tema uste ees,
\par 24 kes peatub tema koja lähedal ja taob telgivaiu tema seinte äärde,
\par 25 see püstitab oma telgi tema kõrvale ja elab heaolu asupaigas,
\par 26 see annab oma lapsed tema kaitse alla ja ööbib ise tema okste all:
\par 27 seal kaitstakse seda meest kuuma eest ja ta elab tarkuse kirkuses.

\chapter{15}

\section*{Tarkuse õnnistusest}

\par 1 Kes Issandat kardab, teebki nõnda, ja kes Seadusest kinni peab, saab targaks.
\par 2 Ja tarkus tuleb talle vastu nagu ema ja võõrustab teda nagu neitsilik naine.
\par 3 Tarkus toidab teda arukuse leivaga ja annab talle juua oma tarkuse vett.
\par 4 Ta toetub sellele ega kõigu, ta hoiab sellest kinni ega jää häbisse.
\par 5 Tarkus ülendab teda ligimesest kõrgemale ja avab tema suu koguduse keskel.
\par 6 Ta leiab hea meele ja rõõmupärja ning pärib igavese nime.
\par 7 Arutud inimesed ei saa targaks ja patused mehed ei saa tarkust näha.
\par 8 Uhkusest on tarkus kaugel ja valelikud mehed ei mõtle tema peale.
\par 9 Kiituslaul ei sobi patuse suhu, sest Issand ei ole seda temale läkitanud.
\par 10 Sest kiituslaul võib kõlada ainult tarkuses, ja Issand juhatab seda hästi.

\section*{Tahtevabadusest}

\par 11 Ära ütle: „On Issanda süü, et olen taganenud” - sest mida tema vihkab, seda ära tee!
\par 12 Ära ütle: „Tema ise on mind eksitanud” - sest patust meest tema ei vaja.
\par 13 Issand vihkab kõike jäledat, ja see ei ole armas ka neile, kes teda kardavad.
\par 14 Tema lõi alguses inimese ja jättis sellele vaba tahte:
\par 15 kui tahad, siis pead käske, ja jääd ustavaks, kui meeldib.
\par 16 Tema on pannud sinu ette tule ja vee: kumba tahad - siruta käsi!
\par 17 Inimese ees on elu ja surm: kumma ta neist valib, see temale antakse.
\par 18 Sest Issanda tarkus on suur, ta on vägev valitsemises ja näeb kõike.
\par 19 Tema silmad on nende peal, kes teda kardavad, ja ta teab kõiki inimese tegusid.
\par 20 Tema ei ole kedagi käskinud olla jumalakartmatu ega ole kellelegi andnud luba pattu teha.

\chapter{16}

\section*{Mure halbade laste pärast}

\par 1 Ära taha, et sul oleks palju halbu lapsi, ja ära rõõmusta jumalakartmatute poegade pärast!
\par 2 On neid palju, siis ära tunne neist rõõmu, kui neis pole Issanda kartust!
\par 3 Ära looda, et nad jäävad elama, ja ära toetu sellele, et neid on palju! Sest parem üks kui tuhat ja parem on surra lastetuna, kui et sul on jumalakartmatud lapsed.
\par 4 Sest ühest mõistlikust saab linn rahva, jumalavallatute sugu jääb aga tühjaks.

\section*{Patune ei pääse karistusest}

\par 5 Palju seesugust olen ma näinud oma silmaga, ja veelgi raskemat on mu kõrv kuulnud:
\par 6 patuste koguduses süttib tuli ja viha hõõgub uskmatu rahva vastu.
\par 7 Jumal ei andestanud muistsetele hiiglastele, kes oma vägevuses temast taganesid.
\par 8 Ta ei säästnud Loti naabreid, keda ta jälestas nende ülbuse pärast.
\par 9 Ta ei halastanud hukatust vääriva rahva peale, nende peale, kes hävisid oma pattudes.
\par 10 Nõndasamuti sündis kuuesaja tuhande jalamehega, kes olid ühinenud oma südame kanguses.
\par 11 Ja olgu kas või üksainus kangekaelne, oleks ime, kui ta jääks karistamata. Sest temal on halastus ja viha, temal on võim andestada ja viha välja valada.
\par 12 Nõnda suur kui on tema halastus, nõnda suur on ka tema karistus: ta mõistab kohut inimese üle tema tegude järgi.
\par 13 Patune ei pääse pakku oma saagiga, ja vaga inimese lootus ei jää täitumata.
\par 14 Igale heateole annab ta võimaluse, igaüks saab oma tegude järgi.
\par 15 Issand tegi vaarao kõvaks, et ta ei tunneks teda, et tema vägevad teod saaksid maailmale teada.
\par 16 Tema halastus on ilmne igale loodule; ja ta on eraldanud oma valguse pimedusest kangekaelselt.

\section*{Jumal ei unusta kedagi}

\par 17 Ära ütle: „Eks ole ma Issanda eest varjul, kes küll kõrgustest meenutab mind? Suure rahva hulgas ei märgata mind, sest mis on minu hing mõõtmatus maailmas!”
\par 18 Vaata, taevas ja taevaste taevas, sügavik ja maa kõiguvad, kui tema karistab!
\par 19 Mäed ja maa alused vabisevad ja värisevad, kui tema nende peale vaatab.
\par 20 Aga süda neid asju ei mõista, ja kes hoolib tema teedest?
\par 21 Või nagu tuulehoog, mida inimene ei näe, nõnda on enamik tema tegudest salajased.
\par 22 „Tema õiguse teod - kes neist kuulutab? Kes jõuab neid ära oodata? Sest lepingu tähtaeg on kaugel!”
\par 23 Nõnda mõtiskleb see, kelle mõistus on piiratud, nõnda rumalasti mõtleb mõistmatu ja eksija mees.

\section*{Jumala suured loomistööd}

\par 24 Kuule mind, laps, ja õpi tarkust ning juhi oma süda minu sõnade juurde!
\par 25 Mina esitan kaalutud õpetust ja kuulutan täpset teadmist.
\par 26 Issanda enese nõu järgi on tema teod algusest alates ja kui ta neid tegi, siis ta määras nende osad.
\par 27 Ta seadis oma teod igaveseks ja nende võimupiirid põlvest põlve. Ei nad tunne nälga, ei nad väsi, ja nad ei lõpeta oma tegevust.
\par 28 Üks ei tõuka teist ja nad ei põlga iialgi tema sõna.
\par 29 Pärast seda vaatas Issand maad ja täitis selle oma hüvedega.
\par 30 Maapinna kattis ta igasugu elavate hingedega ja need pöörduvad jälle sinna tagasi.

\chapter{17}

\section*{Inimene kui loomingu kroon}

\par 1 Issand lõi inimese mullast ja saadab ta sinna tagasi.
\par 2 Inimestele andis ta loetud päevad ja seatud aja, ja andis neile meelevalla selle üle, mis maa peal on.
\par 3 Nagu iseeneselegi, pani ta neile rüüks rammu ja tegi nad omaenese kuju sarnaseks.
\par 4 Kartuse nende ees pani ta kogu loodu peale ning seadis nad valitsema loomade ja lindude üle.
\par 5 Nad said kasutada Issanda viit toimingut ja kuuendal kohal andis ta neile mõistmise ja seitsmendas kõnes nende mõtete tõlgi.
\par 6 Ta andis neile vaba tahte, keele ja silmad ja kõrvad ning südame järelemõtlemiseks.
\par 7 Ta täitis nad arukuse ja tarkusega ning näitas neile head ja kurja.
\par 8 Ta pani nende südamesse oma silma, et näidata neile oma tegude suurust,
\par 9 Ta andis neile igaveseks au oma imeliste tegudega, et nad kuulutaksid tema tegusid mõistusega.
\par 10 et nad püha nime ülistades kuulutaksid tema suuri tegusid.
\par 11 Ta andis neile tarkust ja pärandiks eluseaduse.
\par 12 Ta tegi nendega igavese lepingu ja näitas neile oma kohtumõistmisi.
\par 13 Nende silmad nägid suurt auhiilgust ja nende kõrvad kuulsid tema võimsat häält.
\par 14 Ta ütles neile: „Hoiduge igast ülekohtust!” ning andis neile käsu, kuidas omavahel olla.

\section*{Jumal on kohtumõistja}

\par 15 Inimeste teed on alati tema ees, ei jää need tema silmadele varjatuks.
\par 16 Iga mees on noorusest peale antud kurja kätte; samuti ei saanud nad endale kivisüdameid teha.
\par 17 Igale rahvale on ta seadnud valitseja, Iisrael on aga Issanda enese osa.
\par 18 Keda, olles tema esmasündinu, toidab ta distsipliiniga ja andes talle oma armastuse valgust, ei jäta teda maha.
\par 19 Kõik nende teod on tema ees nagu päike ja tema silmad jälgivad alati nende teid.
\par 20 Nende pahateod ei ole tema eest varjatud ja kõik nende patud on Issanda ees.
\par 21 Aga Issand, olles armuline ja teadis oma tööd, ei jätnud ega jätnud neid maha, vaid säästis neid.
\par 22 Mehe halastus on tema käes otsekui pitserisõrmus ja inimese headust hoiab ta nagu silmatera.
\par 23 Ükskord ta tõuseb ja tasub neile, ta tasub nende tasu neile pea peale.
\par 24 Neid, kes kahetsevad, lubab ta tagasi tulla, ja ta trööstib neid, kes on kaotanud lootuse.

\section*{Manitsus pöördumiseks}

\par 25 Pöördu Issanda poole ja hoidu pattudest, palveta tema palge ees ja väldi pahandusi!
\par 26 Pöördu jälle Kõigekõrgema poole, tagane ülekohtust, ja nurjatust vihka väga!
\par 27 Kes ülistab Kõigekõrgemat surmavallas, elavate asemel, kes temale kiitust toovad?
\par 28 Surnuil, kes on olematud, on kiitus lakanud: kes on elus ja terve, see kiitku Issandat!
\par 29 Kui suur on Issanda halastus ja lepitus neile, kes tema poole pöörduvad!
\par 30 Sest inimesed ise ei suuda kõike, inimlaps ei ole ju surematu.
\par 31 Mis on päikesest heledam? Seegi pimeneb. Ja liha ning veri mõtlevad vaid kurja.
\par 32 Issand näeb taevakõrguse väge, kõik inimesed on aga põrm ja tuhk.

\chapter{18}

\section*{Jumala suurusest}

\par 1 Tema, kes elab igavesti, on loonud viimse kui ühe.
\par 2 Üksnes Issand on õige.
\par 3 Kes valitseb maailma oma peopesaga ja kõik kuuletub tema tahtele, sest tema on kõigi Kuningas, kes oma väega eraldab pühad asjad nende vahel rüvetusest.
\par 4 Väge kuulutada tema töid ei ole antud kellelegi, ja kes suudakski uurida tema suuri tegusid?
\par 5 Kes saaks mõõta tema väe suurust? Ja kes võiks lisaks jutustada tema halastusest?
\par 6 Neist ei ole midagi vähendada ega suurendada ja Issanda imeteod ei ole uuritavad.
\par 7 Kui inimene arvab, et ta on jõudnud pärale, siis ta alles alustab, ja kui ta arvab, et ta on lõpetanud, siis ta on teadmatuses.

\section*{Inimese väiksusest}

\par 8 Mis on inimene ja milleks ta kõlbab? Mis on tema õnn ja mis on tema õnnetus?
\par 9 Inimese päevade hulk, kui neid on palju, on sada aastat.
\par 10 Nagu veetilk meres ja liivatera, nõnda on tema vähesed aastad ühe igaviku päeva kõrval.
\par 11 Seepärast on Issand nende vastu pikameelne ja valab nende peale oma halastust.
\par 12 Ta näeb ja teab, et nende lõpp on ränk, seepärast ongi ta teinud suureks oma lepituse.
\par 13 Inimese halastus puudutab tema ligimest, Issanda halastus aga kogu loodut. Tema karistab, kasvatab ja õpetab ja toob oma karja koju nagu karjane.
\par 14 Ta halastab nende peale, kes võtavad õpetust, ja nende peale, kes tungivalt nõuavad tema õigusemõistmist.

\section*{Annetaja vältigu haavavaid sõnu}

\par 15 Laps, head tehes ära tee etteheiteid ja anni juures ärgu olgu haavavaid sõnu!
\par 16 Eks kaste jahuta palavust? Nõnda on sõna parem kui and.
\par 17 Vaata, eks ole sõna rohkem kui hea and? Ja mehel, kes hästi teeb, on mõlemad.
\par 18 Rumal teeb tänamatult etteheiteid ja kadeda and paneb silmad vett jooksma.

\section*{Ettenägelikkusest ja ettevaatusest}

\par 19 Õpi, enne kui räägid, ja otsi tervist, enne kui haigus sind võtab!
\par 20 Enne kohtumõistmist katsu iseennast läbi, siis leiad armu karistuse ajal.
\par 21 Alandu, enne kui jääd haigeks, kui oled pattu teinud, pöördu!
\par 22 Ära lase ennast takistada tõotust õigeaegselt täitmast, ja ära viivita surmani, oodates sellest vabastamist!
\par 23 Enne kui tõotad, valmista ennast, ära ole inimese sarnane, kes Issandat kiusab!
\par 24 Lõpu päevil mõtle Issanda viha peale, ja karistuse ajale, kui ta oma palge sinult pöörab!
\par 25 Mõtle näljaajale külluse ajal, vaesuse ja puuduse peale rikkuse päevil!
\par 26 Hommikust õhtuni voolab aeg ja Issanda ees sünnib kõik kiiresti.
\par 27 Tark inimene on ettevaatlik kõiges ja kiusatuse päevil hoidub eksimast.
\par 28 Iga arusaaja tunneb tarkust ja kiidab seda, kes selle on leidnud.
\par 29 Kes õpetussõnadest aru saavad, on ka ise targad ja lasevad voolata tabavaid tähendamissõnu.

\section*{Enesevalitsusest}

\par 30 Ära anna järele oma ihadele ja talitse oma himusid!
\par 31 Kui annad oma hinge himude rahuldamisse, siis teeb see sind su vaenlaste naerualuseks.
\par 32 Ära tunne rõõmu suurest priiskamisest, ära seo ennast niisuguse seltskonnaga!
\par 33 Ära tee ennast vaeseks, pidutsedes laenatud rahaga, kui sul enesel ei ole midagi kukrus!

\chapter{19}

\section*{Vägijook on taunitav}

\par 1 Joodik töömees ei saa rikkaks, kes piskut põlgab, hukkub vähehaaval.
\par 2 Vein ja naised eksitavad tarku ja kes hooradega seltsib, läheb pööraseks.
\par 3 Kõdu ja ussid pärivad tema ja jultunud hing võetakse ära.

\section*{Hoidu kuulujuttudest ja talitse keelt!}

\par 4 Kes kiiresti usaldab, on kergeusklik, ja kes selles patustab, eksib oma hinge vastu.
\par 5 Kes kurja pärast südames rõõmustab, mõistetakse hukka,
\par 6 ja kes vihkab lobisemist, pääseb kurjast.
\par 7 Ära iialgi korda kuulujuttu, siis ei sünni sulle midagi halba!
\par 8 Ära jutusta seda sõbrale ega vaenlasele, ja kui see ei ole sulle süüks, siis ära anna midagi teada!
\par 9 Sest kui sinult seda kuuldakse, siis hoitakse sinust kõrvale, ja omal ajal vihatakse sind.
\par 10 Kui oled kuulujuttu kuulnud, siis surgu see koos sinuga, ole julge, see ei aja sind lõhki!
\par 11 Kuulujutu pärast on rumalal suured valud, otsekui sünnitajal ihuvilja pärast.
\par 12 Otsekui reielihasse tunginud nool on kuulujutt rumala sisimas.

\section*{Selgita kuuldusi!}

\par 13 Küsi sõbralt, kas vahest tema ei ole seda teinud, ja kui ta on teinud, ütle, et ta enam ei teeks!
\par 14 Küsi ligimeselt, kas vahest tema ei ole seda rääkinud, ja kui ta on rääkinud, ütle, et ta seda ei kordaks!
\par 15 Küsi sõbralt, sest sageli on ju tegemist laimuga, ja ära usu iga juttu!
\par 16 Mõnigi eksib, aga mitte tahtlikult; ja kes ei ole oma keelega patustanud?
\par 17 Küsi ligimeselt, enne kui ähvardad, ja anna aset Kõigekõrgema seadusele!
\par 18 Issanda kartus on esimene samm [tema] vastuvõtmiseks ja tarkus saab tema armastuse.
\par 19 Issanda käskude tundmine on eluõpetus ja need, kes teevad seda, mis talle meeldivad, saavad surematuse puu vilja.

\section*{Õigest ja ebaõigest tarkusest}

\par 20 Kogu tarkus on Issanda kartus ja iga tarkuse juurde kuulub seaduse täitmine.
\par 21 Kui sulane ütleb oma isandale: Ma ei tee seda, mis sulle meeldib; kuigi ta pärast seda teeb, vihastab ta selle, kes teda toidab.
\par 22 Oskus kurja teha ei ole tarkus ja arukust ei ole seal, kus on patuste nõu.
\par 23 On küll kavalus, aga see on jäledus, ja meeletu on, kellel tarkus puudub.
\par 24 Parem olla mõistusest vaene, aga jumalakartlik, kui rikas tarkuse poolest ja seadusest üleastuja.
\par 25 On peent kavalust, aga see on vale, ja on heatahtlikkuse väänajaid, et saada soovitud kohtuotsust.
\par 26 On kurjategijaid, kes küürutades käivad leinariides, seesmiselt on aga täis valelikkust.
\par 27 Ta käib silmad maas ja teeb ennast kurdiks, ja kui teda ära ei tunta, siis ta üllatab sind.
\par 28 Kui võimu puudus teda takistab pattu tegemast, siis ta teeb kurja, kui iganes leiab võimaluse.
\par 29 Meest tuntakse näost ja tarka tuntakse käitumisest.
\par 30 Mehe riietus, naer ja kõnnak näitavad, missugune inimene ta on.

\chapter{20}

\section*{Kõnelemisest ja vaikimisest}

\par 1 On noomimist, mis ei sünni õigel ajal, mõni aga vaikib ja see on tark.
\par 2 Parem noomida kui viha pidada.
\par 3 Kes üles tunnistab, seda hoitakse kahju eest.
\par 4 Otsekui eunuhh, kes himustab neitsit vägistada, on see, kes tahab õigust mõista vägivallaga.
\par 5 Mõni vaikib ja teda peetakse targaks, ja mõnda vihatakse, et ta palju lobiseb.
\par 6 Mõni vaikib, sest ta ei oska vastata, ja mõni vaikib, sest ta teab parajat aega.
\par 7 Tark inimene vaikib, kuni õige aeg tuleb, aga lobiseja ja alp ei hooli ajast.
\par 8 Kes palju sõnu teeb, seda vihatakse, ja omavolitseja on põlastust väärt.

\section*{Näilikkuse petlikkusest}

\par 9 Mõnele mehele on õnnetus õnneks ja võit võib olla kaotuseks.
\par 10 On ande, millest sul ei ole kasu, ja on ande, mis tasuvad kahekordselt.
\par 11 Mõni talub alandust au pärast ja mõni tõstab madalusest pea üles.
\par 12 Mõni ostab palju vähese eest, peab aga maksma seitsmekordselt.
\par 13 Tark teeb ennast armsaks sõnadega, aga rumalate lahkus on raisatud.

\section*{Kitsi inimese and on väärtuseta}

\par 14 Arutu annist ei ole sul kasu, sest ühe silma asemel on temal neid mitu.
\par 15 Ta annab vähe ja sõimab palju, ja avab oma suu nagu sõnumitooja. Ta laenab täna ja homme nõuab tagasi, vihatud on niisugune inimene.
\par 16 Rumal ütleb: „Minul ei ole sõpra ja minu heategude eest ei ole tänu.
\par 17 Neil, kes minu leiba söövad, on laisad keeled.” Kui paljud ja kui sageli on teda naernud!

\section*{Hoiatus keelepattude eest}

\par 18 Parem komistada maapinnal kui libastuda keelega: õela langus tuleb nii äkitselt.
\par 19 Ebameeldiv inimene - kohatu jutt, arutute suus on see tavaline.
\par 20 Õpetussõna rumala suust põlatakse, sest ta ei ütle seda õigel ajal.
\par 21 Mõnda takistab pattu tegemast vaesus ja oma rahus ei lase ta ennast häirida.
\par 22 Mõni hävitab oma hinge häbi pärast ja kahjustab iseennast mõistmatu käitumisega.
\par 23 Mõni annab häbi pärast sõbrale lubadusi ja teeb ta asjatult oma vaenlaseks.

\section*{Vale on taunitav}

\par 24 Vale on inimesel paha häbiplekk, arutute suus on see aga tavaline.
\par 25 Pigem varas kui alatine valetaja, aga mõlemad pärivad hukatuse.
\par 26 Valeliku inimese harjumus on vääritu ja tema häbi on alati temaga kaasas.

\section*{Tarkuse kasust ja kasutusest}

\par 27 Kes sõnades on tark, edutab iseennast, ja mõistlik inimene meeldib suurtele isandatele.
\par 28 Kes maad harib, teeb kõrgeks oma viljakuhilad, ja kes suurtele isandatele meeldib, lepitab ülekohut.
\par 29 Meelehea ja kingitused pimestavad tarkade silmi ja takistavad manitsust otsekui suukorv suu ees.
\par 30 Peidetud tarkus ja nähtamatu varandus - mis kasu on neist mõlemast?
\par 31 Pigem inimene, kes varjab rumalust, kui inimene, kes varjab tarkust.

\chapter{21}

\section*{Manitsus põgeneda patu eest}

\par 1 Laps, kui oled pattu teinud, siis ära enam tee, vaid palu andeks endised!
\par 2 Põgene patu eest otsekui mao eest, sest kui sa temaga kokku puutud, salvab ta sind! Tema hambad on nagu lõvi hambad: need tapavad inimeste hinge.
\par 3 Iga ülekohus on otsekui kaheterane mõõk, selle haav on ravimatu.
\par 4 Vägivald ja ülbus hävitavad rikkuse: selsamal põhjusel hävib kõrgi koda.
\par 5 Palve vaese suust jõuab Jumala kõrvadeni ja tema kohus tuleb ruttu.
\par 6 Kes vihkab manitsust, käib patuse jälgedes, aga kes Issandat kardab, pöördub südames.
\par 7 Suuresuulist tuntakse kaugelt ja tark teab, millal teine eksib.
\par 8 Kes oma koda ehitab võõra rahaga, on nagu see, kes kogub enesele hauakive.
\par 9 Patuste kogudus on takukoonal ja nende lõpp on tuleleegis.
\par 10 Patuste tee on kividega sillutatud, aga selle lõpp on surmavalla sügavuses.

\section*{Tark ja rumal}

\par 11 Kes Seadust järgib, valitseb oma mõtteid, ja täielik Issanda kartus on tarkus.
\par 12 Ei saa õpetada seda, kes ei ole taibukas; on aga taibukust, mis kasvatab kibedust.
\par 13 Targa tunnetus kasvab otsekui tõusev vesi ja tema nõuanne on elava allika sarnane.
\par 14 Rumala süda on nagu katkine anum, see ei pea ühtki tarkust.
\par 15 Kui taibukas kuuleb tarka sõna, siis ta kiidab seda, andes veel lisagi. Kuuleb seda aga kergemeelne, siis temale see ei meeldi ja ta heidab selle üle õla.
\par 16 Rumala jutt on otsekui koorem teekonnal, aga targa huultelt leitakse, mis on meelepärane.
\par 17 Targa suud otsitakse koguduses ja tema sõnadele mõteldakse südames.
\par 18 Rumalale on tarkus nagu lagunenud koda, ja arulageda teadmiseks on arusaamatud sõnad.
\par 19 Mõistmatule on õpetus nagu jalarauad jalgadel ja käerauad ümber parema käe.
\par 20 Rumal tõstab naerdes häält, tark mees ainult naeratab vaikselt.
\par 21 Targale on õpetus nagu kuldehe ja võru parema käe randmel.
\par 22 Rumala jalg astub kiiresti kotta, aga kogenud mees jääb tagasihoidlikult läve ette.
\par 23 Arutu piilub läbi ukse kotta, aga kasvatatud mees jääb välja seisma.
\par 24 Ukse taga kuulatamine on inimese kasvatamatus, targale on aga seesugune häbitus piinaks.
\par 25 Lobisejate huuled räägivad muidu, aga tarkade sõnu kaalutakse kaaluga.
\par 26 Rumalate süda on nende suus, aga tarkade suu on nende südames.
\par 27 Kui jumalakartmatu neab saatanat, siis ta neab omaenese hinge.
\par 28 Keelepeksja rüvetab omaenese hinge ja naabrid vihkavad teda.

\chapter{22}

\section*{Laiskusest}

\par 1 Laisk sarnaneb roojaga kaetud kiviga, tema häbituse pärast vilistab ta peale igaüks.
\par 2 Laisk sarnaneb sõnnikuhunnikuga, igaüks, kes seda tõstab, pühib kätt.

\section*{Kõlvatutest lastest}

\par 3 Isale häbiks sünnib poeg, kes ei võta õpetust, ja seesuguse tütre sünd on temale kahjuks.
\par 4 Tark tütar on oma mehele varanduseks, häbitu on aga oma sünnitajale mureks.
\par 5 Häbematu naine häbistab oma isa ja meest, seepärast põlatakse teda mõlema poolt.
\par 6 Nagu pillimäng leinaajal on kohatu jutt, vitsad ja karistus on aga igal ajal tarkuseks.
\par 7 Kes lolli õpetab, on nagu see, kes potikildu kokku liimib, ja nagu see, kes unest äratab.
\par 8 Kes lollile jutustab, räägib magama jäävale inimesele. Kui ta on oma jutu rääkinud, siis ta küsib: Mis on?

\section*{Targast ja rumalast}

\par 9 Potikilde liimib kokku, kes mõistmatut õpetab, ta äratab magajat sügavast unest.
\par 10 Kes jutustab juhmile, jutustab tukkujale, ja lõpuks see küsib: „Mis on?”
\par 11 Nuta surnu pärast, sest tema valgus on kadunud, ja nuta juhmi pärast, sest tema aru on läinud! Surnu pärast nuta vähem, sest tema on ju rahu saanud, juhmi elu on aga halvem kui surm.
\par 12 Lein surnu pärast kestab seitse päeva, aga juhmi ja jumalakartmatu pärast kõik tema elupäevad.
\par 13 Arutuga ära palju räägi ja arulagedaga ära seltsi, hoidu temast, et sul ei oleks vaeva ja sind ei rüvetataks sellega, mis ta suust välja ajab! Väldi teda, siis saad rahu ja sa ei muutu kärsituks tema meeletuse pärast!
\par 14 Mis on raskem kui seatina? Ja mis muu võiks olla ta nimeks kui „tobu”?
\par 15 Liiva, soola ja rauakamakat on kergem kanda kui arutut inimest.

\section*{Järelemõtlemise kasust}

\par 16 Seotud palkidega hoone ei purune maavärisemisel, nõnda ka süda, mis kindlaks jääb läbimõeldud otsusele, ei lähe iialgi araks.
\par 17 Süda, rajatud mõistlikule meelele, on nagu liivakrohv silutud seinal.
\par 18 Kõrgendikule püstitatud pihttara ei pea vastu tuule ees: nõnda ka rumala mõtteviisiga arg süda ei pea vastu ühegi ähvarduse ees.

\section*{Sõprusest}

\par 19 Kes torgib silma, paneb pisarad voolama; kes torkab südamesse, tekitab valu.
\par 20 Kes viskab kiviga linde, peletab need; kes teotab sõpra, lõpetab sõpruse!
\par 21 Kui oled tõstnud mõõga sõbra vastu, siis ära heida meelt, sest veel on tagasitee!
\par 22 Kui oled avanud suu sõbra vastu, siis ära muretse, sest on ka lepitus! Pea meeles: teotus ja kõrkus, saladuse reetmine ja salakaval löök - nende eest põgeneb iga sõber.
\par 23 Ole ustav oma ligimesele, kui ta on vaene, et siis, kui ta käsi jälle hästi käib, võiksid koos temaga rahul olla. Hädaajal jää tema juurde, et siis, kui ta pärib, võiksid koos temaga pärida!
\par 24 Enne tuld on ahjus aur ja suits, nõndasamuti on teotused enne verevalamist.
\par 25 Sõpra kaitsta ma ei häbene ja ma ei varja ennast tema palge eest.
\par 26 Kui mulle tema pärast halba sünnib, siis igaüks, kes sellest kuuleb, hoidub temast.

\section*{Valvsusest}

\par 27 Kes paneks minu suu ette valvuri ja minu huultele kohase pitseri, et ma ei langeks ja minu keel mind ei hävitaks?

\chapter{23}

\section*{Palve patust hoidumiseks}

\par 1 Issand, minu elu isa ja valitseja! Ära jäta mind nende nõu võimusesse, ära lase mind nende läbi langeda!
\par 2 Kes annaks vitsu minu mõtlemisele ja tarkuse karistust minu südamele, et need ei säästaks minu väärsamme ega jätaks karistamata nende patte,
\par 3 et minu väärsamme ei lisanduks ja minu patud ei suureneks, et ma ei langeks oma vastaste ees ja et mu vihamees minu pärast ei rõõmustaks?!

\section*{Kiusatusest hoidumisest}

\par 4 Issand, minu elu isa ja Jumal! Ära anna mulle suurelisi silmi
\par 5 ja pööra himu minust ära!
\par 6 Kõhu aplus ja himurus ärgu vallutagu mind ja ära jäta mind häbitu hinge võimusesse!

\section*{Vandumisest}

\par 7 Kuulge, lapsed, suu õpetamist, kes tähele paneb, seda ei vangista huuled.
\par 8 Huultest tabatakse patustaja, nendega komistavad teotaja ja ülbe.
\par 9 Ära harjuta oma suud vandeks ja ära võta kombeks nimetada Püha nime!
\par 10 Sest nii nagu ori, keda alatasa üle kuulatakse, ei parane vermeist, nõnda ka see, kes alati vannub ja nimetab Jumala nime, ei ole iialgi pattudest puhas.
\par 11 Mees, kes palju vannub, suurendab süüd ja tema kojast ei lahku nuhtlus. Kui ta eksib, siis on patt tema peal, ja kui ta sellest ei hooli, siis ta patustab kahekordselt. Kui ta valet vannub, ei mõisteta teda õigeks, vaid tema koda saab õnnetusi täis.

\section*{Kõlvatu kõne}

\par 12 On kõne, mis on võrdne surmaga: ärgu seesugust leidugu Jaakobi pärandis! Sest vagadest on see kõik kaugel ja nemad ennast pattudesse ei mähi.
\par 13 Ära harjuta oma suud jämedaks tooruseks, sest sellest tuleb patune kõne!
\par 14 Mõtle oma isale ja emale, ehkki istud suursuguste keskel, et sa nende ees seda ei unustaks ega käituks rumalasti oma harjumuse pärast, et sa ei peaks soovima, et oleksid jäänud sündimata, ega oleks sul vaja needa oma sündimispäeva!
\par 15 Inimene, kes enese harjutab rääkima sõimusõnu, ei võta enam õppust kõigil oma elupäevil.

\section*{Kõlblusvastastest tegudest}

\par 16 Kaks liiki inimesi teeb palju pattu ja kolmas tekitab viha.
\par 17 Kirglik hing on nagu põlev tuli: see ei kustu enne, kui on neelanud iseenese. Inimene, kes hoorab omaenese ihus, ei lõpeta enne, kui tuli on põlenud lõpuni. Hooravale inimesele on iga leib magus, ta ei väsi enne, kui sureb.
\par 18 Inimene, kes oma voodis patustab, ütleb endamisi: „Kes mind näeb? Minu ümber on pimedus, seinad varjavad mind, keegi ei näe mind. Mida ma kardan? Kõigekõrgem ei pane tähele minu patte.”
\par 19 Jah, inimeste silmi ta kardab ega tea, et Issanda silmad on kümme tuhat korda selgemad kui päike, need jälgivad kõiki inimeste teid ja näevad ka salajastesse paikadesse.
\par 20 Ta teab kõike juba enne tegu, niisamuti ka pärast, kui see on tehtud.
\par 21 Niisugust meest nuheldakse linnatänavail, võetakse kinni seal, kus ta ei aimagi.

\section*{Abielurikkujast naisest}

\par 22 Nõnda on lugu ka naisega, kes oma mehe maha jätab ja saab võõralt pärija.
\par 23 Sest esiteks on ta olnud sõnakuulmatu Kõigekõrgema seaduse vastu, teiseks on ta eksinud oma mehe vastu, kolmandaks on ta hoorates abielu rikkunud ja võõralt mehelt lapsi saanud.
\par 24 Ta viidagu koguduse ette ja karistus tuleb tema laste peale.
\par 25 Tema lapsed ei juurdu ja tema oksad ei kanna vilja.
\par 26 Mälestuse enesest jätab ta sajatuseks ja tema häbi ei kustutata ära.
\par 27 Ja järelejääjad teavad, et ei ole paremat kui Issanda kartus ega meeldivamat kui tähele panna Issanda käske.

\chapter{24}

\section*{Tarkus ülistab ennast}

\par 1 Tarkus ülistab ennast ja kiidab end oma rahva keskel.
\par 2 Kõigekõrgema koguduses avab ta suu ja kiidab end tema väe ees:
\par 3 „Mina tulin Kõigekõrgema suust ja katsin maad otsekui udu.
\par 4 Ma elasin kõrgustes ja minu iste oli pilvesambas.
\par 5 Mina üksi ringlesin taevakaartel ja käisin kuristike sügavustes.
\par 6 Mere laineil ja kogu maal, kõigi rahvaste ja rahvuste üle oli mul meelevald.
\par 7 Kõigi nende juures otsisin ma rahupaika: kelle pärisossa ma võiksin jääda?
\par 8 Siis kõige Looja käskis mind, tema, kes minu on loonud, andis rahu minu telgile ja ütles: „Jaakobis löö üles oma telk ja Iisraelis võta enesele pärisosa!”
\par 9 Alguses, enne aegade arvestust, on ta minu loonud ja iialgi ei lakka ma olemast.
\par 10 Pühas telgis teenisin ma tema ees, ja nõnda kinnistati mind Siionisse.
\par 11 Nõnda laskis ta mind rahu leida armsas linnas, minu valitsus on Jeruusalemmas.
\par 12 Ma juurdusin austatud rahvas, Issanda omandis, tema pärisosas.
\par 13 Ma kasvasin kõrgeks nagu seeder Liibanonil ja küpress Hermoni mäestikus.
\par 14 Ma kasvasin kõrgeks nagu palmipuu Een-Gedis ja roosipõõsad Jeerikos, nagu ilus õlipuu väljal. Ma kasvasin kõrgeks nagu plataan.
\par 15 Ma lõhnasin hästi nagu kaneel ja palsam, andsin meeldivat lõhna nagu valitud mürr, nagu galban ja oonüks ja stakte, nagu suitsutusrohu suits telgis.
\par 16 Ma laiutasin oma oksi nagu terebint, minu oksad olid ilusad ja armsad.
\par 17 Ma ajasin toredaid võrseid nagu viinapuu ja minu õitest tuli kaunis ning rikkalik vili.
\par 18 Ma olen õiglase armastuse, hirmu ja teadmise ja püha lootuse ema. Seepärast olen mina, kes olen igavene, antud kõigile mu lastele, kellele tema nimed on pandud.
\par 19 Tulge minu juurde, kes te ihaldate mind, ja küllastuge minu viljast!
\par 20 Sest mälestus minust on magusam kui mesi ja pärida mind on parem kui meevaha.
\par 21 Kes minust söövad, soovivad veelgi, ja kes minust joovad, janunevad üha.
\par 22 Kes mind kuulda võtab, ei jää häbisse, ja kes minu pärast vaeva näevad, ei tee pattu.”

\section*{Tarkusest ja seadusest}

\par 23 See kõik on öeldud Kõigekõrgema Jumala lepinguraamatu kohta, Seaduse kohta, mille Mooses meile andis pärisosaks Jaakobi kogudustele.
\par 24 Ärge minesta, et olla tugev Issandas; et ta teid kinnitaks, jääge tema juurde, sest kõigeväeline Issand on üksi Jumal ja peale tema pole kedagi teist Päästjat.
\par 25 Ta laseb voolata tarkusel, mis on nagu Piison ja Tigris kevadisel ajal,
\par 26 ta tekitab mõistuse tulva, mis on nagu Eufrat ja Jordan lõikuse päevil,
\par 27 ta ilmutab õpetuse valgust, mis on nagu Giihon viinamarjakoristuse ajal.
\par 28 Ei õppinud esimene teda täiesti tundma, ja nõnda ka viimane ei mõista teda.
\par 29 Sest ta mõte on suurem kui meri ja nõu sügavam kui tohutu kuristik.
\par 30 Mina olen nagu jõeharu ja veejuhe, mis paradiisist on lähtunud.
\par 31 Ma ütlesin: „Mina kastan oma aeda ja joodan oma peenart.” Ja vaata, minu kraavist sai jõgi ning minu jõest sai meri.
\par 32 Ma lasen oma juhatust ikka veel paista otsekui koitu ja valgustada kaugele.
\par 33 Ma valan ikka veel välja õpetust otsekui prohvetikuulutust, jättes selle tulevastele põlvedele igaveseks ajaks.
\par 34 Nähke, et mina ei ole vaeva näinud ainult iseenese heaks, vaid kõigi heaks, kes tarkust taotlevad.

\chapter{25}

\section*{Tarkusesõnu}

\par 1 Kolm asja meeldivad mulle ja need on ilusad Issanda ning inimeste ees: vendade üksmeel ja ligimesearmastus ning et naine ja mees üksteist mõistavad.
\par 2 Aga kolme vihkab mu hing, ja nende eluviisi pärast ma vihastan väga: ülbe vaene, valelik rikas ja abielurikkuja rauk, kellel puudub aru.

\section*{Vanadusest}

\par 3 Kui sa nooruses ei ole kogunud, kuidas sa siis vanas eas leiad?
\par 4 Kui ilus on, et hallpead otsustavad ja vanemad teavad nõu!
\par 5 Kui ilus on vanade tarkus ja auväärsete arukus ning nõukus!
\par 6 Vanade krooniks on suured kogemused ja kuulsuseks Issanda kartus.

\section*{Kümme kiiduväärt kogemust}

\par 7 Üheksat meeles olevat asja ma ülistan südames ja kümnendat kuulutan keelega: inimest, kes oma lastest rõõmu tunneb, ja seda, kes oma eluajal näeb vaenlaste langemist;
\par 8 õnnelikku, kes koos elab mõistliku naisega, ja seda, kes keelega ei libastu, ja seda, kes ei pea orjama vääritut;
\par 9 õnnelikku, kes on leidnud tarkuse, ja seda, kes võib kuulutada kuulavaile kõrvadele.
\par 10 Kuigi see on suur, kes on leidnud tarkuse, ei ole ta suurem kui see, kes kardab Issandat.
\par 11 Issanda kartus on üle kõige: kellega võiks võrrelda inimest, kes sellest kinni peab?
\par 12 Issanda kartus on tema armastuse algus ja usk on tema külge klammerdumise algus.

\section*{Tigedast naisest}

\par 13 Olgu haav, ainult mitte südame haav, olgu tigedus, ainult mitte naise tigedus!
\par 14 Olgu rünnak, ainult mitte vihkajate rünnak, olgu kättemaks, ainult mitte vaenlaste kättemaks!
\par 15 Ükski mürk ei ole kangem kui maomürk ja ükski viha ei ole suurem kui vaenlase viha.
\par 16 Pigem elada koos lõvide ja lohemadudega kui koos tigeda naisega.
\par 17 Naise tigedus muudab tema välimust ja teeb ta palge süngeks just nagu karul.
\par 18 Tema mees istub oma sõprade keskel ja neid kuulates ohkab ta kibestunult.
\par 19 Kõik halbus on väike naise halbuse kõrval - tabagu teda patuse liisk!
\par 20 Seesama, mis liivane mägitee vana mehe jalgadele, on lobisev naine vaiksele mehele.
\par 21 Ära lase ennast võluda naise ilust ja ära himusta naist!
\par 22 Viha, sündsusetus ja häbi on suur, kui naine toidab oma meest.
\par 23 Rõhutud meel, sünge pale ja südamevalu tuleb tigedalt naiselt. Lõdvad käed ja nõtkuvad põlved on sellelt, kes ei tee oma meest õnnelikuks.
\par 24 Naisest on patu algus ja tema pärast peame kõik surema.
\par 25 Veele ära anna vabalt voolata ega tigedale naisele sõnavoli!
\par 26 Kui ta ei taha käia sinuga käsikäes, siis lahuta ta oma ihust!

\chapter{26}

\section*{Heast naisest}

\par 1 Hea naise mees on õnnelik ja tema päevade arv kasvab kahekordselt.
\par 2 Tubli naine rõõmustab oma meest ja see veedab oma aastad rahus.
\par 3 Hea naine on hea osa ja antakse sellele, kes Issandat kardab:
\par 4 olgu mees rikas või vaene, tema süda on siis rahul, tema nägu on alati rõõmus.

\section*{Naise kolm halba omadust}

\par 5 Kolme asja kardab mu süda ja neljanda pärast ma palun silmili maas: kõmu linnas, jõukude kogunemine ja valesüüdistus - kõik halvemad kui surm! -
\par 6 südamevalu ja muret tegev naine, kes on armukade teisele naisele, ja keelepeksja, kes seda kõigile kuulutab.
\par 7 Kuri naine on nagu tõrges härjapaar, kes tema võtab, haaraks otsekui skorpioni.
\par 8 Väga vihatud on joobnud naine: ta ei varja oma sündsusetust.
\par 9 Naise hoorust tuntakse silmategemisest ja tema silmalaugudest.
\par 10 Kangekaelset tütart valva rangelt, et ta lõtvust leides seda ei kasutaks!
\par 11 Häbitu silma üle pea valvet ja ära imesta, kui ta sind võrgutab!
\par 12 Nagu janune teekäija avab suu ja joob iga ligiolevat vett, nõnda istub ta iga telgivaia juurde ja avab oma tupe noolele.

\section*{Kooselust naistega}

\par 13 Naise armsus rõõmustab tema meest ja tema osavus lisab mehe kontidele liha.
\par 14 Vaikiv naine on Issanda and ja haritud hinge eest ei ole küllalt lunaraha.
\par 15 Häbelik naine on armude arm ja tagasihoidlik hing ei ole millegagi kaalutav.
\par 16 Nõnda nagu päike tõuseb Issanda kõrgustes, nõnda on ka hea naise ilu ehteks mehe kojale.
\par 17 Nagu pühalt lambijalalt paistev lamp on ilus nägu sihvakal kehal.
\par 18 Nagu kuldsambad hõbealustel on kaunid jalad kindlatel kandadel.
\par 19 Mu poeg, hoia oma ajastu lille kõla; ja ära anna oma jõudu võõrastele.
\par 20 Kui sa oled saanud viljaka vara läbi kogu põllu, külva see oma seemega, lootes oma vilja headusele.
\par 21 Nii et su sugu, mille sa maha jätad, saab suureks, sest neil on kindlustunne nende heas päritolus.
\par 22 Hoort peetakse süljeks; aga abielunaine on oma mehele torn surma vastu.
\par 23 Õel naine antakse osaks õelale, aga jumalakartlik naine antakse sellele, kes kardab Issandat.
\par 24 Ebaaus naine taunib häbi, aga aus naine austab oma meest.
\par 25 Häbematut naist peetakse koeraks; aga see, kes on häbistatud, kardab Issandat.
\par 26 Naine, kes austab oma meest, mõistetakse kõigist targaks; aga see, kes teda oma uhkuses teotab, peetakse kõigi seas jumalakartmatuks.
\par 27 Vaenlaste eemale tõrjumiseks otsitakse valjult nutvat naist ja noomitust.

\section*{Kolm halba nähtust}

\par 28 Kahe asja pärast on minu süda murelik ja kolmanda pärast tõuseb minus viha: sõjamees, kes vaesuse tõttu puudust kannatab, arukad mehed, keda ei panda millekski, ja see, kes õigusest pöördub pattu: Issand valmistab teda mõõga jaoks.

\section*{Kaubitsemisest}

\par 29 Kaupmees saab vaevalt hoiduda süüteost ja äritsejat ei mõisteta patust lahti.

\chapter{27}

\section*{Kaubitsemisest}

\par 1 Raha pärast on paljud pattu teinud ja kes tahab palju saada, pöörab oma silmad Issandalt.
\par 2 Otsekui vai lüüakse kivide vahele, nõnda tungib patt müügi ja ostu vahele.
\par 3 Kes hoolsalt ei pea kinni Issanda kartusest, selle koda variseb varsti.

\section*{Kõnest}

\par 4 Sõela raputades jääb peale räbu, nõndasamuti jääb inimese saast tema mõtteisse.
\par 5 Potissepa tööd proovitakse ahjus ja inimest õpitakse tundma tema kõnest.
\par 6 Vili näitab hoolitsemist puu eest, nõndasamuti sõna inimsüdame mõtteid.
\par 7 Ära kiida meest, enne kui ta on kõnelnud, sest kõne on inimese katsekivi!

\section*{Õiglusest ja õigusest}

\par 8 Kui taotled õigust, siis saavutad selle ja ehid ennast sellega kui aukuuega.
\par 9 Linnud seltsivad omasugustega ja tõde tuleb jälle nende juurde, kes seda nõuavad.
\par 10 Lõvi varitseb saaki, nõnda ka patt ülekohtutegijaid.
\par 11 Jumalakartliku kõne on alati tark, albil aga muutub just nagu kuu.
\par 12 Arutute keskel mõõda aega, aga mõistlike seltsis ole alatasa!
\par 13 Rumalate kõne on põlastust väärt ja nende naer on patune ülemeelikus.
\par 14 Kes palju vannub, selle jutt ajab juuksed püsti ja seesuguste riid paneb kõrvad kinni.
\par 15 Ülbete riid on verevalamine ja nende sõimlemist on vastik kuulda.

\section*{Saladustest}

\par 16 Kes saladusi avaldab, kaotab usalduse ega leia enam sõpra oma hingele.
\par 17 Armasta sõpra ja ole talle truu, aga kui oled ta saladusi avaldanud, siis ära enam jookse talle järele!
\par 18 Sest otsekui inimene hävitab oma vaenlase, nõnda oled sina hävitanud oma ligimese sõpruse.
\par 19 Ja otsekui laseksid linnu lahti oma käest, nõnda oled ligimese lasknud minna ega saa teda enam kätte.
\par 20 Ära aja teda taga, sest ta on juba kaugel, põgenenud nagu gasell püünisest!
\par 21 Haava saab siduda ja teotust lepitada, aga saladuste väljarääkijal ei ole lootust.

\section*{Usaldamatusest}

\par 22 Kes silma pilgutab, kavatseb kurja, ja keegi ei pääse tema käest.
\par 23 Sinu silma ees räägib ta magusasti ja imetleb sinu sõnu, aga pärast väänab ta oma suud ja teeb sinu sõnadest pahanduse.
\par 24 On palju, mida ma vihkan, aga ei midagi nõnda nagu sellist - ka Issand vihkab teda!
\par 25 Kes viskab kivi kõrgele, viskab selle enesele pähe, ja salakaval hoop rebib haavad lahti.
\par 26 Kes kaevab augu, langeb sinna sisse, ja kes paneb püünise, seda püütakse sellega.
\par 27 Kes kurja teeb, sellele veereb see tagasi ja ta ei märka, kust see temale tuleb.
\par 28 Ülbe pilkab ja teotab, aga karistus varitseb teda otsekui lõvi.
\par 29 Püünisesse langevad, kes rõõmustavad vagade langusest, ja valu hävitab nad juba enne surma.
\par 30 Raev ja viha, ka need on hirmsad, patune mees aga püsib neis kindlalt.

\chapter{28}

\section*{Usaldamatusest}

\par 1 Kes kätte maksab, saab kättemaksu Issandalt ja tema patud talletatakse hoolsasti.
\par 2 Anna andeks oma ligimese ülekohus ja siis, kui sa palud, vabastatakse sind su pattudest.
\par 3 Inimene peab viha inimese vastu, aga otsib päästet Issandalt.
\par 4 Omataolise inimese vastu ei ole tal halastust, ta palub aga küll oma pattude pärast.
\par 5 Tema ise, lihalik olend, peab viha, kes peab lepitama tema patte?
\par 6 Mõtle lõpule ja jäta vihavaen, mõtle kaduvusele ja surmale ja pea käske!
\par 7 Mõtle käskudele ja ära raevutse ligimese vastu, mõtle Kõigekõrgema lepingule ja andesta eksimus!

\section*{Tülitsemisest}

\par 8 Hoidu tülist, siis teed vähem pattu, sest vihane inimene sütitab tüli!
\par 9 Patune mees ässitab sõpru ja paiskab kahtlustusi rahus elavate keskele.
\par 10 Kuidas tulepuud, nõnda on tuli, ja ägeduse kohaselt leegitseb tüli. Mida võimsam inimene, seda võimsam tema vimm, mida rikkam ta on, seda suurem ta viha.
\par 11 Õhutatud vaen süütab tule ja õhutatud tüli valab verd.
\par 12 Kui puhud sädet, hakkab see leegitsema, ja kui sülitad selle peale, siis see kustub. Ja mõlemad teod tulevad sinu suust.

\section*{Keelepattudest}

\par 13 Nea keelepeksjat ja kahekeelset, sest ta on hukka saatnud palju rahus elanuid.
\par 14 Kolmekeelne on kõigutanud paljusid ja on neid pillutanud rahva juurest rahva juurde, on hävitanud ka kindlustatud linnu ja maha kiskunud suurte meeste kodasid.
\par 15 Kolmekeelne on ära tõuganud tublisid naisi ja on neilt röövinud nende töö vilja.
\par 16 Kes seda kuulda võtab, ei saa iialgi rahu ega vaikselt elada.
\par 17 Piitsahoop tekitab vermeid, aga keelehoop murrab luid.
\par 18 Paljud on langenud mõõgatera läbi, aga mitte nõnda palju kui keele läbi.
\par 19 Õnnis on see, kes selle eest on kaitstud, kes ei ole kogenud selle tigedust, kes ei pea vedama selle ikkes ega ole seotud selle rakmetega.
\par 20 Sest selle ike on raudike ja selle rakmed on vaskrakmed.
\par 21 Surm selle läbi on kuri surm, ja surmavald on palju parem kui see.
\par 22 Vagade üle ei saa ta võimust ja need ei põle tema leegis.
\par 23 Sellesse langevad need, kes Issanda maha jätavad, see põleb neis ega kustu. Otsekui lõvi läkitatakse see nende kallale ja nagu panter hävitab see neid.
\par 24 Vaata, piira oma vara kibuvitstega, köida kokku oma hõbe ja kuld:
\par 25 oma sõnadele muretse kaal ja vihid, oma suule valmista uks ja riiv!
\par 26 Hoia, et sa oma keelega ei libastu ega lange selle ees, kes sind varitseb!

\chapter{29}

\section*{Laenamisest}

\par 1 Kellel on kaastunnet, see laenab ligimesele, ja kes peab käske, sirutab abistava käe.
\par 2 Laena ligimesele, kui temal tarvis on, ja vastupidi - ligimesele maksa tagasi lubatud ajal!
\par 3 Pea sõna ja ole temale ustav, siis saad igal ajal, mis sulle tarvis on!
\par 4 Mõned peavad laenu leiuks ja valmistavad vaeva oma aitajaile!
\par 5 Ta suudleb käsi, kuni ta on saanud, ja ligimese raha pärast räägib ta alandliku häälega, tasumise ajal ta aga viivitab, tasub jõhkrate sõnadega ja süüdistab aegu.
\par 6 Kui ta jaksab, siis saadakse tagasi vaevalt pool ja seda võib arvestada kui leitut. Aga kui mitte, siis on raha röövitud ja ilmaasjata on saadud vaenlane: sajatuse ja sõimuga tasub ta laenajale, austamise asemel on tasutud autusega.
\par 7 Paljud loobuvad niisuguse alatuse pärast, et hoiduda asjatust röövimisest.

\section*{Andidest}

\par 8 Ometi ole pikameelne viletsa vastu ja ära kõhkle temale almust andmast!
\par 9 Käsu pärast aita vaest ja ära lase teda tühjalt ära minna, kui ta hädas on!
\par 10 Venna ja sõbra pärast kuluta raha, ära lase seda kivi all roostetada ja hävida!
\par 11 Talleta enesele varandus Kõigekõrgema käskude järgi, sellest on sul rohkem kasu kui kullast!
\par 12 Talleta heateod oma varakambrisse, siis päästab see sind kõigist hädadest!
\par 13 Paremini kui tugev kilp ja raske oda sõdib see sinu eest vaenlaste vastu.

\section*{Käendusest}

\par 14 Hea mees käendab ligimest, ainult häbitu jätab tema hädasse.
\par 15 Ära unusta käendaja heatahtlikkust, sest tema on sinu eest andnud oma hinge!
\par 16 Patune hävitab käendaja õnne, tänamatu meelsus hülgab oma päästja.
\par 17 Käendamine on laostanud palju jõukaid, pannes nad kõikuma otsekui merelainetel.
\par 18 See on teinud kodutuks vägevaid mehi ja need on ekselnud võõraste rahvaste keskel.
\par 19 Patune, kes kasu pärast tõttab käendajaks, langeb kohtu alla.
\par 20 Aita ligimest jõudumööda, aga pane tähele, et sa ise sisse ei kukuks!

\section*{Külalislahkusest}

\par 21 Kõige vajalikumad eluks on vesi ja leib, koda ja riietus, et katta sündsusetut.
\par 22 Parem vaese elu oma pilbaskatuse all kui maiusroad võõrastes kodades.
\par 23 Pisut või palju, ole rahul, siis pääsed kuulmast häbistust, et oled võõras!
\par 24 Halb on elu kojast kotta, sest kus oled võõras, seal sa ei tohi suud lahti teha.
\par 25 Annad süüa ja juua tänuta ja kuuled ka haavavaid sõnu:
\par 26 „Tule, võõras, kata laud, ja kui sul midagi on, siis toida mind!”
\par 27 „Mine ära võõras, auväärsema eest, vend on külla tulnud, ma vajan koda!”
\par 28 Arusaajal inimesel on raske taluda etteheiteid päritolu pärast ja laimu võlausaldaja poolt.

\chapter{30}

\section*{Kasvatusest}

\par 1 Kes oma poega armastab, karistab teda alati, et ta lõpuks saaks temast rõõmu tunda.
\par 2 Kes oma poega karistab, saab temast kasu ja võib tuttavate keskel temast uhkusega rääkida.
\par 3 Kes oma poega õpetab, teeb oma vaenlase kadedaks ja võib sõprade ees tema pärast hõisata.
\par 4 Kui tema isa sureb, siis see nagu ei olekski surnud, sest temast on jäänud järele omasugune.
\par 5 Elades nägi ta teda ja oli rõõmus, ja surres ei olnud ta kurb.
\par 6 Vaenlaste vastu jättis ta kättemaksja, sõpradele aga tänuliku tasuja.
\par 7 Kes poega hellitab, peab siduma tema haavu, ja ta süda väriseb iga tema valuhüüde korral.
\par 8 Taltsutamata hobune läheb peruks ja ohjeldamata poeg läheb tõrksaks.
\par 9 Hellita last, siis ta paneb sind hirmu tundma, naljatle temaga, siis ta kurvastab sind!
\par 10 Ära naera koos temaga, et sa ei peaks koos temaga valu kannatama ja lõpuks hambaid kiristama!
\par 11 Ära anna temale voli noorpõlves!
\par 12 Painuta tema kaela, kui ta alles noor on, et ta ei muutuks kangekaelseks ega oleks sulle sõnakuulmatu!
\par 13 Kasvata oma poega ja pane ta tööle, et sinul ei tuleks pahandusi tema saamatuse pärast!

\section*{Tervisest}

\par 14 Parem olla vaene, kes on terve ja tugev, kui rikas, kellel on haige ihu.
\par 15 Tervis ja heaolu on parem kui kogu kuld ja terve keha on parem mõõtmatust rikkusest.
\par 16 Ei ole paremat rikkust kui ihu tervis ja ei ole suuremat rõõmu kui südame rõõm.
\par 17 Parem surm kui kibe elu, igavene puhkus on parem kui pikaldane haigus.
\par 18 Maiusroog suletud suu ees on nagu haua peale pandud toit.
\par 19 Mis kasu on ohvrist ebajumalale? Sest too ei söö ega tunne lõhna. Nõndasamuti on Issandast vaevatuga:
\par 20 ta näeb silmadega ja ohkab nagu kohitsetu, kes neitsit süleleb ja ohkab.

\section*{Rõõmust}

\par 21 Ära lase oma hinge kurvastada ja ära vaeva ennast oma mõtisklustega!
\par 22 Südame rõõm on inimesele eluks ja rõõmsameelsus pikendab mehe elupäevi.
\par 23 Armasta oma hinge ja tröösti oma südant, ja hoia kurbus enesest eemal! Sest kurvastamine on paljusid hukanud ja sellest ei ole midagi kasu.
\par 24 Kadedus ja viha lühendavad elupäevi ja mure teeb enneaegu vanaks.
\par 25 Helde ja toidust rõõmu tundev süda kannab hoolt oma roa eest.

\chapter{31}

\section*{Rikkusest}

\par 1 Rikkuse valvamine kurnab ihu ja mure selle pärast viib une.
\par 2 Murelik valvsus ei lase uinuda, äratab unest nagu raske haigus.
\par 3 Rikas näeb vaeva raha kogumisega ja puhates naudib ta oma küllust.
\par 4 Vaene näeb vaeva kokkuhoidliku eluga ja puhates jääb ta puudusesse.
\par 5 Kes armastab kulda, ei jää patuta, ja kes raha taga ajab, seda petetakse.
\par 6 Kuld on saanud õnnetuseks paljudele, nad on ise näinud oma hävingut.
\par 7 Kuld on selle kummardajale tõkkepuuks, kullaga püütakse iga meeletut narri.
\par 8 Õnnis on rikas, kes leitakse olevat laitmatu ja kes ei jookse kulla järele.
\par 9 Kus niisugune on, et saaksime teda ülistada? Sest ta on teinud imetlusväärseid tegusid oma rahva keskel.
\par 10 Kes on selles proovile pandud ja täiuslikuks saanud? Olgu see kiituseks temale, kes oleks võinud eksida, aga ei ole eksinud, kes oleks võinud kurja teha, aga ei ole teinud.
\par 11 Seepärast on tema varandus kindlustatud ja kogudus jutustab tema almustest.

\section*{Käitumisest võõruspeol}

\par 12 Kui istud suurniku lauas, siis ära aja suud ammuli ja ära ütle: „Siin on palju!”
\par 13 Pea meeles, et kuri silm on hädaohtlik! Kas on loodud midagi vaenulikumat kui silm? Seepärast ta põhjustab pisaraid igal palgel.
\par 14 Kuhu ta vaatab, sinna ära siruta kätt ja ära pista seda koos temaga vaagnasse!
\par 15 Hinda oma ligimest kui iseennast ja iga teo puhul mõtle järele!
\par 16 Söö nagu inimene seda, mis sulle ette pannakse, ja ära matsuta, et sind ei vihataks!
\par 17 Hea kombe pärast lõpeta esimesena ja ära ole täitmatu, et sa kedagi ei pahandaks!
\par 18 Kui sa istud paljude keskel, siis ära siruta kätt enne teisi!
\par 19 Jah, küll lepib viisakas inimene vähesega! Ja voodis olles ta ei hingelda.
\par 20 Kellel on mõõdukas isu, sellel on terve uni, ta tõuseb vara ja ta hing on rahul. Täitmatul inimesel on unetus, kõhuhäda ja soolekeerud.
\par 21 Aga kui oled pidanud sööma, siis tõuse üles ja pea vahet!
\par 22 Kuule mind, laps, ja ära põlga mind, siis leiad lõpuks, et minu sõnad on tõsi! Ole mõõdukas kõigis oma tegudes, siis ei tule sulle ühtki tõbe!
\par 23 Kes on leivaga helde, seda õnnistavad kõik, ja tunnistus tema headusest on tõsi.
\par 24 Kes on aga leivaga kitsi, selle vastu nuriseb linn, ja tunnistus tema kurjusest peab paika.

\section*{Veinist}

\par 25 Veini juures ära mängi vaprat, sest vein on paljusid viinud hukatusse!
\par 26 Sulatusahi proovib terast karastusvees, nõnda ka vein südameid ülbete tülis.
\par 27 Vein on inimesele otsekui elu, kui seda juuakse mõõdukalt. Mis oleks elu ilma veinita? Vein on loodud inimestele rõõmuks.
\par 28 Vein on südamele hõiskamiseks ja hingele rõõmuks, kui seda juuakse õigel ajal parajalt.
\par 29 Vein kibestab hinge, kui seda palju juuakse, kihutab riidlema ja komistama.
\par 30 Purjusolek suurendab albi viha eksisammuks, vähendab tema jõudu ja toob haavu.
\par 31 Veinijoomingul ära laida ligimest ja ära põlga teda, kui ta on rõõmus! Ära ütle temale sõimusõna ja ära kimbuta teda võla nõudmisega!

\chapter{32}

\section*{Võõruspeost}

\par 1 Kui sind on pandud eesistujaks, ära siis uhkusta, ole teiste hulgas kui üks neist, hoolitse nende eest ja istu siis ise!
\par 2 Kui oled teinud kõik, mis vajalik, alles siis võta istet, et võiksid teistest rõõmu tunda ja hea korralduse eest saaksid pärja!
\par 3 Räägi sina, vanem, sest see sobib sulle, põhjalikult järele mõteldes, ja ära eksita muusikat!
\par 4 Kui see kostab, siis peata kõne ja ära näita oma tarkust sobimatul ajal!
\par 5 Otsekui rubiinist pitsatkivi kuldehtes on muusika veini joomisel.
\par 6 Otsekui smaragdist pitsatkivi kuldraamistuses on muusikute laul magusa veini juures.
\par 7 Räägi, noormees, kui sul vaja on, ometi kõige rohkem kaks korda, kui sinult veel küsitakse!
\par 8 Oma kõne võta kokku: vähesega palju! Ole kui see, kes asja tunneb, aga vaikib ometi.
\par 9 Suursuguste seltsis ära pea ennast nende sarnaseks ja kui teine räägib, siis ära palju lobise!
\par 10 Välgusähvatusele järgneb müristamine, kuid tagasihoidliku ees käib lugupidamine.
\par 11 Tõuse üles õigel ajal ja ära jää viimaseks, mine kohe koju ja ära ole kergemeelne!
\par 12 Seal naljata ja tee, mis sulle meelde tuleb, aga ära tee pattu suurelise kõnega!
\par 13 Ja seejuures kiida oma Loojat, kes sind kosutab oma heade andidega!

\section*{Jumalakartusest}

\par 14 Kes Issandat kardab, võtab õpetust, ja kes teda aegsasti otsivad, leiavad tema heameele.
\par 15 Kes Seadust uurib, küllastub sellest, aga teesklejale on see pahanduseks!
\par 16 Kes Issandat kardavad, leiavad, mis õige on, ja süütavad õiguse otsekui valguse.
\par 17 Patune inimene ei salli laitust ja leiab oma tahtele õigustuse.
\par 18 Mees, kes järele mõtleb, ei põlga ühtki nõuannet, aga teistsugune ja ülbe ei karda üldse midagi.
\par 19 Ära tee midagi mõtlematult, siis sa tehes ei kahetse!
\par 20 Ära käi konarlikul teel, siis sa ei komista kividele!
\par 21 Ometi ära looda tasase tee peale
\par 22 ja hoia ennast oma lastegi eest!
\par 23 Kõiges, mis sa teed, looda iseenese peale, sest seegi on Seaduse täitmine!
\par 24 Kes Seadust usub, peab ka käske, ja kes Issanda peale loodab, sellel ei puudu midagi.

\chapter{33}

\section*{Jumalakartusest}

\par 1 Kes Issandat kardab, sellele ei sünni kurja, vaid Issand päästab ta kiusatusestki.
\par 2 Tark mees ei vihka Seadust, aga kes sellega silmakirjatseb, on otsekui laev tormis.
\par 3 Arukas inimene loodab Seaduse peale ja usub Seadust otsekui vastust uurimilt+ .
\par 4 Valmista ette oma kõne, siis sind kuulatakse, võta kokku oma teadmised ja vasta siis!
\par 5 Rumala süda on nagu vankriratas ja tema mõttekäik pöörleva rummu sarnane.
\par 6 Pilkav sõber on nagu tiirane mära, kes hirnub iga ratsutaja all.

\section*{Erinevustest ja vastuoludest}

\par 7 Mispärast on üks päev teisest parem, ometi tuleb ju kogu aasta päevavalgus päikeselt?
\par 8 Issanda tarkus on need erinevaiks teinud ja on seadnud ajad ning pühad.
\par 9 Mõned neist ta ülendas ja pühitses, aga mõned neist ta tegi argipäevadeks.
\par 10 Ja kõik inimesed on põrmust - Aadam loodi mullast.
\par 11 Aga Issand oma suures tarkuses kujundas nad erinevaiks ja määras neile mitmesugused teed.
\par 12 Mõningaid neist ta õnnistas ja ülendas, mõningaid pühitses ja lähendas enesele, mõningaid needis ja alandas ning ajas nad ära nende asupaigast.
\par 13 Otsekui savi potissepa käes - tema teeb kõiki oma töid, nagu ise tahab -, nõnda on inimesed oma Looja käes, et ta annaks neile, mida tema õigeks peab.
\par 14 Halva vastu on seatud hea ja surma vastu elu, nõnda on ka patune ja vaga.
\par 15 Ja nõnda vaata kõiki Kõigekõrgema töid: kaks ja kaks, teineteisega vastakuti!
\par 16 Mina ise olen viimasena valvel olnud nagu viinamarjakorjajate järelt noppija.
\par 17 Issanda õnnistusega olen aga teistest ette jõudnud ja täitnud surutõrre nagu viinamarjakorjaja.
\par 18 Nähke, et mina ei ole vaeva näinud ainult iseenese heaks, vaid kõigi heaks, kes õpetust otsivad!
\par 19 Kuulge mind, rahva suurmehed, ja koguduse juhid, pange tähele!

\section*{Sõltumatusest}

\par 20 Ära anna enese üle meelevalda pojale ja naisele, vennale ja sõbrale, niikaua kui sa elad! Ja oma vara ära anna kellelegi teisele, et kahetsuse korral poleks vaja tagasi nõuda!
\par 21 Niikaua kui sa elad ja hinges oled, ära müü ennast mitte ühelegi inimesele!
\par 22 Sest on parem, kui lapsed paluvad sinult, kui et sina vaatad oma poegade käte poole.
\par 23 Jää isandaks kõigis oma tegudes ja oma au peale ära lase tulla häbiplekki!
\par 24 Alles siis, kui su elupäevad lõpevad, surmatunnil, jaota oma pärand!

\section*{Orjadest}

\par 25 Sööt, piits ja koorem eeslile, leib, karistus ja töö orjale!
\par 26 Pane ori tööle, siis saad ise puhata, kui jätad tema käed jõude, siis ta nõuab vabadust!
\par 27 Ike ja ohjad painutavad turja, kuritahtlikule orjale piinapink ja pigistus!
\par 28 Sunni teda tööle, et ta ei oleks jõude, sest jõudeolek õpetab palju halba!
\par 29 Pane ta tööle, milleks ta on võimeline, ja kui ta on sõnakuulmatu, siis tee tema jalarauad raskemaks!
\par 30 Ometi ära tee liiga kellelegi ja ära tee midagi õigusvastast!
\par 31 Kui sul on ori, siis olgu tema nagu sina ise, sest sa oled ta hankinud oma verega.
\par 32 Kui sul on ori, siis kohtle teda nagu venda, sest sa vajad teda nagu oma hinge.
\par 33 Kui sa teda halvasti kohtled, ta põgeneb ja jookseb ära. Missugust teed lähed siis teda otsima?

\chapter{34}

\section*{Unenägudest}

\par 1 Arutul mehel on tühised ja petlikud lootused ja unenäod erutavad alpe.
\par 2 Otsekui see, kes varju püüab või tuult taga ajab, on see, kes unenägusid tähele paneb.
\par 3 Unenäod on vaid peegeldus - varjupilt näo asemel.
\par 4 Mis puhast saab tulla rüvedast või mis tõtt saab tulla valest?
\par 5 Ennustused, märkideseletused ja unenäod on tühised, otsekui sünnitusvaludes olija südame kujutlused.
\par 6 Kui Kõigekõrgem ei ole neid läkitanud juhatamiseks, siis ära võta neid südamesse!
\par 7 Sest unenäod on paljusid eksitanud, ja kes nende peale on lootnud, on pettunud.
\par 8 Seadus läheb täide pettuseta ja tarkus usaldusväärsest suust on täiuslik.

\section*{Rändamisest}

\par 9 Rännanud mees teab palju ja kes on väga kogenud, jutustab arukalt.
\par 10 Kellel pole kogemusi, teab vähe, aga kes on laialt liikunud, oskab palju.
\par 11 Palju olen näinud oma rännakuil ja mul on rohkem tarkust kui sõnu.
\par 12 Sageli olen olnud surmaohus, aga olen pääsenud selle tõttu:
\par 13 nende vaim, kes Issandat kardavad, jääb elama, sest nad loodavad oma päästja peale.
\par 14 Kes Issandat kardab, ei tunne hirmu, sest Issand on tema lootus.
\par 15 Kes Issandat kardab, selle hing on õnnis. Kellest ta kinni hoiab? Ja kes on tema tugi?
\par 16 Issanda silmad on nende peal, kes teda armastavad, tema on nende võimas kaitse ja kindel tugi, vari idatuule ja vari lõunakuumuse eest, valve komistuse ja abi languse vastu,
\par 17 hinge ülendaja ja silmade valgustaja, tervise, elu ja õnnistuse andja.

\section*{Ohverdamisest}

\par 18 Ohver ülekohtuselt saadust on ohvriand pilkeks ja patuste annid ei valmista heameelt.
\par 19 Kõigekõrgemale ei meeldi jumalatute ohvriannid, ta ei andesta patte ka mitte ohvrite rohkuse pärast.
\par 20 Kes ohvri toob vaeste omandist, on nagu see, kes tapab poja tema isa nähes.
\par 21 Vaestele on eluks kehv leib, kes selle röövib, on verejanuline.
\par 22 Ligimese tapab, kes tema elatise ära võtab, ja verd valab, kes päevilise palga kinni peab.
\par 23 Üks ehitab ja teine lammutab: mis kasu neil sellest on peale vaeva?
\par 24 Üks palvetab ja teine neab: kumma häält Issand võtab kuulda?
\par 25 Kui keegi ennast peseb surnu pärast ja puudutab seda jälle, mis kasu on siis temal pesemisest?
\par 26 Nõndasamuti on lugu inimesega, kes paastub oma pattude pärast, aga läheb ja teeb jälle sedasama: kes võtab kuulda tema palvet ja mis kasu temal on enda alandamisest?

\chapter{35}

\section*{Ohverdamisest}

\par 1 Kes Seadusest kinni peab, toob rikkalikult ohvriande, käskude tähelepanija on tänuohvri tooja.
\par 2 Kes heategu tasub, toob roaohvri, ja almuste andja ohverdab kiitusohvri.
\par 3 Loobumine kurjast on Issandale meelepärane ja loobumine ülekohtust on lepitusohver.
\par 4 Ära ilmu Issanda palge ette tühjade kätega, sest see kõik on käsu kohaselt!
\par 5 Õige inimese ohvriand teeb altari rasvaseks ja selle lõhn meeldib Kõigekõrgemale.
\par 6 Õige mehe ohver on armas ja tema meenutusohvrit ei unustata.
\par 7 Austa Issandat heatahtliku silmaga ja ära vähenda oma käte esmaande!
\par 8 Andes ole alati rõõmsa näoga ja pühitse kümnist hea meelega!
\par 9 Anna Kõigekõrgemale, nagu tema on andnud, heatahtliku silmaga ja käe jõu kohaselt!
\par 10 Sest Issand on tasuja ja tema tasub sulle seitsmekordselt.

\section*{Jumala õiglusest}

\par 11 Ära püüa anda meelehead, sest tema ei võta seda vastu, ja ära looda väära ohvri peale,
\par 12 sest Issand on kohtumõistja ja ükski ei ole auline tema ees.
\par 13 Tema ei eelista kedagi vaeste vastu, vaid võtab kuulda rõhutute palveid.
\par 14 Tema ei põlga vaeslapse anumist ega lesknaist, kui see oma kaebed kuuldavale toob.
\par 15 Eks lesknaisel voola pisarad palgeil, ja eks ta süüdista seda, kes neid on põhjustanud?
\par 16 Kes Jumalat kogu südamega teenib, võetakse vastu, ja tema palve ulatub pilvedeni.
\par 17 Alandliku palve tungib pilvest läbi, enne kui see on pärale jõudnud, ei ole tal lohutust,
\par 18 ta ei lõpeta, enne kui Kõigekõrgem tähele paneb, õiglaselt kohut mõistab ja õigust teeb.
\par 19 Jah, Issand ei viivita ega ole pikameelne halastamatute vastu,
\par 20 kuni ta on nende niuded purustanud ja paganaile kätte maksnud,
\par 21 kuni ta on hävitanud vägivallatsejate jõugu ja murdnud ülekohtuste valitsuskepi,
\par 22 kuni ta on tasunud igaühele tema tegude järgi ja inimeste töö nende kavatsuste kohaselt,
\par 23 kuni ta on toonud õiguse oma rahvale ja on teda oma halastusega rõõmustanud.
\par 24 Tema halastus on armas ahastuse päevil, just nagu vihmapilved põua ajal.

\chapter{36}

\section*{Palve Iisraeli vabastamiseks}

\par 1 Halasta meie peale, Issand, kõiksuse Jumal, ja vaata siia ning pane kõik paganad sind kartma!
\par 2 Tõsta oma käsi võõraste rahvaste vastu, et nad näeksid sinu väge!
\par 3 Nõnda nagu sa ennast näitasid pühana meie juures nende nähes, nõnda näita ennast suurena nende juures meie nähes,
\par 4 et nad õpiksid sind tundma, nõnda nagu meie tunneme, et ei ole muud Jumalat kui sina, Issand!
\par 5 Uuenda oma tunnustähti ja korda imetegusid, austa oma kätt ja paremat käsivart!
\par 6 Ärata oma raev ja vala välja viha, hävita vastane ja purusta vaenlane!
\par 7 Kiirenda aega ja meenuta vannet, et saaks jutustada sinu suurtest tegudest!
\par 8 Vihatules hävigu, kes üritab pääsu, ja hukkugu, kes sinu rahvale kurja teevad!
\par 9 Purusta vaenulike vürstide pead, kes ütlevad: „Ei ole kedagi peale meie!”
\par 10 Kogu kõik Jaakobi suguharud ja olgu nad pärijad nagu muistegi!
\par 11 Halasta, Issand, rahva peale, kellele on antud sinu nimi, ja Iisraeli peale, keda pead esmasündinuks!
\par 12 Halasta oma pühamu linna, Jeruusalemma, oma hingamispaiga peale!
\par 13 Täida Siion sinu ülistusega ja oma rahvas sinu auhiilgusega!
\par 14 Anna tunnistus oma esmalooduile ja tee tõeks, mis sinu nimel on ennustatud!
\par 15 Tasu neile, kes sinu peale loodavad, ja sinu prohvetid jäägu usaldusväärseks!
\par 16 Kuule, Issand, nende palvet, kes sind anuvad, sinu rahvale kuuluva Aaroni õnnistuse pärast,
\par 17 et kõik, kes maailmas elavad, tunneksid, et sina oled Issand, igavene Jumal!

\section*{Vahetegemisest}

\par 18 Kõht seedib igasugust rooga, aga üks roog on parem kui teine.
\par 19 Suulagi eristab jahisaagi maitset, nõnda ka arukas süda valelikke sõnu.
\par 20 Pöörane süda valmistab muret, aga kogenud mees tasub temale samaväärselt.

\section*{Naise valimisest}

\par 21 Naisele kõlbab iga mees, aga üks tütar on parem kui teine.
\par 22 Naise ilu teeb mehe näo rõõmsaks ja ületab kõik muud mehe himud.
\par 23 Kui tema keelel on armsus ja hellus, siis tema mehega võrdset ei ole inimesepoegade hulgas.
\par 24 Kes saab niisuguse naise, paneb varandusele aluse, saab enesele kohase abi ja toe, kellele toetuda.
\par 25 Kus ei ole tara, seal rüüstatakse põld, ja kus ei ole naist, seal eksleb mees ohates.
\par 26 Sest kes usaldab relvastatud röövlit, kes tõttab linnast linna?
\par 27 Nõnda on lugu mehega, kellel ei ole pesa ja kes ööbib seal, kus õhtu jõuab.

\chapter{37}

\section*{Sõprusest}

\par 1 Iga sõber ütleb: „Minagi olen sõber”, mõni sõber on aga sõber ainult nime poolest.
\par 2 Eks see ole mure surmani, kui kaaslane ja sõber muutub vaenlaseks?
\par 3 Oh, kuri kalduvus! Kust sa oled ilmunud katma maad valega?
\par 4 Mõni kaaslane tunnustab sõpra häil päevil, hädaajal on aga tema vastu.
\par 5 Mõni kaaslane, kes sõbraga jagab kõhumuresid, haarab võitluse tulles siiski ka kilbi.
\par 6 Sõpra ära unusta oma südames ja mõtle tema peale, kui oled rikas!

\section*{Nõuandjaist}

\par 7 Iga nõuandja annab nõu, mõni annab aga nõu iseenese kasuks.
\par 8 Hoia oma hinge nõuandja eest ja selgita enne, mida temal on vaja - sest vahest ta annab nõu iseenese kasuks, et ta sinule liisku heites
\par 9 ei ütleks sulle: „Oled õigel teel”, ise aga jääb eemale seisma, et näha, mis sinuga juhtub.
\par 10 Ära pea nõu sellega, kes sinule kõõrdi vaatab, ja varja oma kavatsust nende eest, kes sind kadestavad!
\par 11 Naisega tema võistleja pärast ja pelguriga sõja pärast, kaupmehega müügi pärast ja ostjaga kauba pärast, kadedaga tänu pärast ja halastamatuga heategemise pärast, tahtejõuetuga mõne ettevõtmise pärast ja sulasega töö lõpetamise pärast, laisa orjaga töö raskuse pärast - niisugustega ära pea üldse nõu!
\par 12 Pea aga alati nõu vaga mehega, kellest sa tead, et ta peab käske, kelle hing sarnaneb sinu hingega, ja kes, kui sa komistad, tunneb sulle kaasa!
\par 13 Ja otsustagu südame nõu, sest keegi ei ole sinule ustavam kui see!
\par 14 Sest mehe hing annab vahel rohkem teada kui seitse vahimeest, kes valve pidamiseks istuvad kõrges tornis.
\par 15 Aga kõige selle juures palu Kõigekõrgemat, et ta tões juhataks sinu teed.

\section*{Õige ja vale tarkus}

\par 16 Iga töö alguseks on arupidamine ja enne iga tegu olgu nõupidamine!
\par 17 Südame muutlikkuse pärast võrsub neli haru:
\par 18 hea ja kuri, elu ja surm. Aga keel valitseb neid alati.
\par 19 Mõni osav mees õpetab teisi, iseenese jaoks on aga kõlbmatu.
\par 20 Mõni targutab sõnadega, aga on vihatud. Seesugune on ilma igasugusest ülalpidamisest,
\par 21 sest Issand ei ole talle andnud armu: tarkus puudub tal täiesti.
\par 22 Mõni on tark iseenese meelest ja tema arukuse vilja saab usaldada ainult tema enese suus.
\par 23 Tark mees õpetab oma rahvast ja tema arukuse vilja saab usaldada.
\par 24 Tarka meest õnnistatakse rohkesti ja kõik, kes teda näevad, kiidavad teda õndsaks.
\par 25 Inimese elu kestab piiratud arv päevi, aga Iisraeli päevad on loendamatud.
\par 26 Kes oma rahva hulgas on tark, pälvib usalduse, ja tema nimi elab igavesti.

\section*{Mõõdukusest}

\par 27 Laps, elus katsu oma hing läbi, ja vaata, mis temale on kahjulik - ära seda temale anna!
\par 28 Sest kõik ei ole kasulik kõigile ja iga hing ei võta kõike heaks!
\par 29 Ära ole täitmatu üheski naudingus ja ära priiska toitudega!
\par 30 Sest suurest söömisest tuleb tõbi ja aplus toob kõhuvalu.
\par 31 Apluse pärast on paljud surnud, aga kes iseennast valitseb, pikendab elu.

\chapter{38}

\section*{Arst ja haigus}

\par 1 Austa arsti - sa ju vajad teda - austusega, mis temale kuulub, sest Issand on loonud ka tema!
\par 2 Sest tervekstegemine tuleb Kõigekõrgemalt ja arst saab kingitusi kuningalt.
\par 3 Arsti oskus tõstab tema pea püsti ja teda imetlevad suured isandad.
\par 4 Issand laseb maast võrsuda ravimtaimi ja mõistlik mees ei põlga neid.
\par 5 Eks puu mõjul muutunud vesi magusaks, et Issanda vägevus saaks tuntuks?
\par 6 Jah, tema on andnud inimesele oskuse, et teda austataks tema imetegude pärast.
\par 7 Nende taimede abil ta ravib ja vaigistab valu ja salvisegaja valmistab neist vajaliku võide.
\par 8 Jah, Issanda tegu ei lakka iialgi ja temalt tuleb õnnistus maa peale.
\par 9 Laps! Kui jääd haigeks, siis ära ole hooletu, vaid palu Issandat ja tema teeb sinu terveks!
\par 10 Hoidu eksimustest ja korralda oma käte teod ning puhasta süda kõigest patust!
\par 11 Ohverda healõhnaline ohver ja meenutusohver peenjahust, ka rasvane ohvriand - oma võimaluste järgi!
\par 12 Ka arstile anna võimalus, sest Issand on loonud ka tema, ja ära hoia teda enesest eemal, sest temagi on vajalik!
\par 13 On aegu, millal abi on nende käes.
\par 14 Sest ka nemad paluvad Issandat, et neil õnnestuks leevendus ja ravi, et säilitada elu.
\par 15 Kes pattu teeb oma Looja ees, see jõudku arsti käte alla!

\section*{Leinamisest}

\par 16 Laps! Surnu pärast vala pisaraid ja itke nagu raskesti kannataja! Kata tema ihu, nagu on talle kohane, ja ära jäta hooletusse tema matust!
\par 17 Nuta kibedasti ja sinu kaebus olgu suur, leina, nõnda nagu ta väärib, üks või kaks päeva, et vältida keelepeksu, ja siis lase ennast kurbuses trööstida!
\par 18 Sest kurvastusest tuleb surm ja südamevalu murrab jõu.
\par 19 Kurvastuses kestab valu ja vaese elu hakkab vastu südamele.
\par 20 Ära anna oma südant kurvastusele, tõrju seda, mõteldes lõpule!
\par 21 Ära seda unusta, sest tagasitulekut ei ole; surnule sa kasu ei too, aga teed kahju iseendale!
\par 22 Pea meeles: otsus minu kohta on seesama ka sinu kohta! „Eile minule, täna sinule!”
\par 23 Kui surnu puhkab, siis puhaku ka mälestus temast, ja lohuta ennast tema pärast, kui ta hing on lahkunud!

\section*{Tööst ja käsitööst}

\par 24 Kirjatundja tarkus tuleb jõudeajast, ja kes toimetusi vähendab, võib targaks saada.
\par 25 Kuidas saab targaks see, kes hoiab atra, kiitleb astlaga, ajab härgi ja juhib nende tööd ning räägib noortest sõnnidest?
\par 26 Tema süda on vagude ajamise juures ja tema hool on sööta vasikaid.
\par 27 Samasugune on lugu iga käsitöölise ja ehitusmeistriga, kes on tegevuses ööd ja päevad; ja sellega, kes teeb märke pitserisõrmusesse ja kannatlikult vahetab mustreid: tema südameasi on, kuidas kujundatavale anda sarnasust, ja tema mureks on töö valmimine.
\par 28 Samasugune lugu on sepaga, kes seisab alasi juures ja hoolsasti jälgib sepist; tule lõõsk kõrvetab ta ihu ja ta peab võitlema ääsi palavusega; vasara kõlksumine teeb ta kõrvad kurdiks ja ta silmad kiinduvad taotava kujusse: temal on südamel töö valmimine ja tema mureks on lõplik viimistlus.
\par 29 Samasugune lugu on potissepaga, kes istub oma töö juures ja jalgadega keerutab ketra: ta on alati mures oma töö pärast ja tema tegevuses sõltub kõik nõude arvust.
\par 30 Käega ta voolib savi ja jalgadega teeb selle pehmeks: temal on südamel vaaba valmistamine ja tema hool on puhastada ahju.
\par 31 Kõik nad loodavad oma kätele ja igaüks on osav oma töös.
\par 32 Ilma nendeta ei saa linna ehitada ja neil pole tarvis rännata ja elada võõrana.
\par 33 Aga rahva nõupidamisele neid ei kutsuta ja koguduses nad ei pääse etteotsa, nad ei istu kohtumõistja istmele ega mõista seaduse tahet.
\par 34 Nad ei seleta ka mitte karistust ja kohtuotsust ning neid ei leita õpetussõnade juurest, aga nad toetavad maailmakorda ja palvetavad oma töö pärast.

\chapter{39}

\section*{Kirjatundja}

\par 1 Teine on lugu sellega, kes pühendab oma hinge ja mõtiskleb Kõigekõrgema Seaduse üle: tema uurib kõigi endisaegsete tarkust ja tegeleb prohvetite poolt öelduga.
\par 2 Ta paneb tähele kuulsate meeste jutustusi ja tungib õpetussõnade tähendusse.
\par 3 Ta uurib tähendamissõnade saladust ja süveneb õpetussõnade mõistatustesse.
\par 4 Ta teenib suurte isandate keskel ja teda nähakse vürstide ees. Ta rändab võõraste rahvaste maal, sest ta tahab tundma õppida inimestes olevat head ja halba.
\par 5 Oma südame kingib ta varakult Issandale, oma Loojale, ja ta palvetab Kõigekõrgema ees. Ta avab suu palveks ja anub oma pattude pärast.
\par 6 Kui suur Issand tahab, siis ta täitub tarkuse vaimuga. Ta laseb voolata tarkusesõnu ja ülistab palves Issandat.
\par 7 Ta kasutab õigesti oma tahet ja mõistust ning mõtiskleb saladuste üle.
\par 8 Ta teeb teatavaks oma kasvatava õpetuse ja kiitleb Issanda lepingu seadusest.
\par 9 Paljud ülistavad tema taipu ja seda ei unustata iialgi. Mälestus temast ei kustu ja tema nimi elab põlvest põlve.
\par 10 Rahvad räägivad tema tarkusest ja kogudus kuulutab tema kuulsust.
\par 11 Kui ta elab, saab tuntuks ta nimi, ja kui ta sureb, suureneb see veelgi.

\section*{Manitsus Jumalat kiita}

\par 12 Tahan veelgi jutustada, millest olen mõtelnud, jah, ma olen täidetud, olen nagu täiskuu.
\par 13 Kuulake mind, vagad pojad, siis te võrsute nagu roos, mis kasvab veeoja ääres,
\par 14 lõhnate hästi nagu suitsutusrohi ja õilmitsete otsekui liilia! Levitage lõhna ja laulge kiituslaulu! Kiitke Issandat kõigi tema tegude pärast!
\par 15 Andke au tema nimele ja ülistage tema kuulsust lauludega huultelt ning kannelt mängides! Ja ülistades ütelge nõnda:
\par 16 „Kõik Issanda teod on väga head ja kõik, mis ta seab, sünnib õigel ajal! Ärgu öeldagu: „Mis see on? Milleks see?”, sest kõike uuritakse omal ajal!”
\par 17 Tema sõna läbi seisis vesi otsekui mäena ja tema suu ütlemisest sündisid vete kogunemiskohad.
\par 18 Tema käsul sünnib kõik, mis ta tahab, ja ei ole kedagi, kes takistaks tema lunastust.
\par 19 Tema ees on kõikide teod ja tema silmadele ei jää midagi varjatuks.
\par 20 Tema vaatleb igavikust igavikku ja miski ei ole temale imestusväärne.
\par 21 Ärgu öeldagu: „Mis see on? Milleks see?”, sest kõik on loodud otstarbe järgi.
\par 22 Tema õnnistus tulvab nagu jõgi, otse tulvaveena jootes kuiva maad.
\par 23 Rahvad pärivad tema viha selsamal kombel, nagu ta kord muutis veed soolakõrbeks.
\par 24 Tema teed on tasased jumalakartlikele, kurjategijaile aga täis komistuskive.
\par 25 Hea on algusest peale loodud headele, nõndasamuti kuri patustele.
\par 26 Kõige vajalikumad inimelus on: vesi, tuli, raud ja sool, nisujahu, piim ja mesi, viinamarjamahl, õli ja riietus.
\par 27 Need kõik tulevad heaks jumalakartlikele, patustele aga pöörduvad kurjaks.
\par 28 On tuuli, mis on loodud karistuseks, ja raevutsedes piitsutavad need valusasti. Kohtumõistmise ajal valavad nad välja oma jõu ja leevendavad selle viha, kes nad on loonud.
\par 29 Tuli ja rahe, nälg ja surm - need kõik on loodud kätte maksma.
\par 30 Kiskjate hambad, skorpionid ja maod ning tasumismõõk on jumalakartmatuile hukatuseks.
\par 31 Need rõõmustavad tema käsu pärast ja valmistuvad teenistuseks maa peal ega astu käsust üle, kui nende aeg tuleb.
\par 32 Sellepärast olen ma algusest peale veendunud olnud, olen selle üle järele mõtelnud ja kirja pannud:
\par 33 Issanda teod on kõik head, ja kõike vajalikku annab ta õigel ajal.
\par 34 Ja ärgu öeldagu: „Üks on halvem kui teine”, sest kõik on kõlblikud omal ajal!
\par 35 Ja nüüd laulge kogu südamest ja suust ning kiitke Issanda nime!

\chapter{40}

\section*{Inimlikust viletsusest}

\par 1 Igale inimesele on loodud suur vaev ja Aadama lastel on raske ike alates päevast, mil nad emaihust välja tulevad, kuni päevani, mil nad pöörduvad tagasi kõigi emasse:
\par 2 kõhklus ja südame kartus - tuleviku ootus, surmapäev.
\par 3 Aujärjel istujast põrmu ja tuhka alandatuni,
\par 4 purpuri ja krooni kandjast kotiriidesse riietatuni: viha, kadedus, rahutus ja mäss, surmahirm, vaen ja riid.
\par 5 Koguni puhkeajal voodis segab öine uni tema mõtlemist.
\par 6 Lühike, otse olematu on tema puhkus, ja uneski on siis nagu päeval valvates: teda kohutatakse oma südame nägemuses, nagu põgeneks ta sõja eest,
\par 7 ta ärkab aga õigel ajal ja imestab, et midagi ei ole karta.
\par 8 Kõigele elavale, inimesest loomani - ja patustele veel seitsmekordselt -
\par 9 on osaks surm ja veri, riid ja mõõk, õnnetused, nälg, purustus ja piin.
\par 10 Jumalakartmatute jaoks on see kõik loodud ja nende pärast tuli veeuputus.
\par 11 Kõik, mis on mullast, läheb tagasi mulda, ja veed voolavad jälle merre.

\section*{Mitmesugused kogemused}

\par 12 Iga kink ja ülekohtuselt saadu kaob, ausus aga kestab igavesti.
\par 13 Ülekohtuste varandused kuivavad nagu vihmanire, nagu vali äike, mis sajus tasaneb.
\par 14 Kes oma käed lahti hoiab, võib rõõmu tunda, aga üleastujad viimaks hukkuvad.
\par 15 Jumalakartmatute järglased ei kasvata oksi, ja nende kõdunenud juured on järsul kaljul:
\par 16 kõrkjad iga vee ja jõe kaldal kitkutakse enne kogu muud rohtu.
\par 17 Heategemine on aga nagu õnnistatud rohuaed ja halastus kestab igavesti.

\section*{Tähelepanekuid elust}

\par 18 Elu on magus rahulolevale ja töökale, aga varanduse leidja on mõlemast üle.
\par 19 Lapsed ja linna ehitamine annavad püsiva nime, aga rohkem kui kumbagi neist hinnatakse laitmatut naist.
\par 20 Vein ja muusika rõõmustavad südant, aga mõlemast üle on tarkuse armastus.
\par 21 Vile ja kannel kaunistavad laule, aga mõlemast üle on lahke kõne.
\par 22 Sinu silm ihaldab kenadust ja ilu, aga mõlemast üle on haljendav oras.
\par 23 Sõber ja kaaslane ilmuvad õigel ajal, aga mõlemast üle on naine mehe kõrval.
\par 24 Vennad ja aitajad päästavad kitsikuse ajal, aga neist paremini päästab halastus.
\par 25 Kuld ja hõbe annavad jalale kindlust, head nõu peetakse aga mõlemast paremaks.
\par 26 Rikkus ja võim ülendavad südant, aga mõlemast üle on Issanda kartus. Issandat kartes ei puudu midagi ja siis ei ole tarvis abi otsida.
\par 27 Issanda kartus on nagu õnnistatud rohuaed, mis on kaetud ülima kirkusega.

\section*{Kerjamisest}

\par 28 Laps, ära ela kerjuseelu, parem surra kui kerjata!
\par 29 Võõrast lauda piiluva mehe elu ei saa eluks pidada. Ta rüvetab oma hinge võõraste roogadega, aga mõistlik ja haritud mees hoidub sellest.
\par 30 Häbitu suus on kerjatud leib magus, aga tema kõhus põletab see nagu tuli.

\chapter{41}

\section*{Surmast}

\par 1 Oh, surm, kui kibe on mõte sinule inimesel, kes rahulikult elab oma kodus, mehel, kes muretult elab ja on õnnelik kõiges ning ikka veel on võimeline nautima pidurooga!
\par 2 Oh, surm, hea on sinu paratamatus inimesele, kes on kehv ja jõuetu, väga vana ja mures kõige pärast, kes usu ja kannatlikkuse on kaotanud!
\par 3 Ära karda surma paratamatust, mõtle nende peale, kes on olnud enne sind ja kes tulevad pärast sind!
\par 4 See paratamatus on Issandalt kõigele elavale, miks tõrgud siis Kõigekõrgema tahte vastu? Oli kümme või sada või tuhat aastat - surmavallas ei ole kaebamist elu pikkuse pärast.

\section*{Jumalakartmatute saatusest}

\par 5 Patuste lastest saavad põlatud lapsed ja need viibivad jumalakartmatute elupaigus.
\par 6 Patuste laste pärand hävib ja nende järeltulijaid saadab alati häbi.
\par 7 Lapsed süüdistavad jumalakartmatut isa, sest neid häbistatakse tema pärast.
\par 8 Häda teile, jumalakartmatud mehed, kes hülgate Kõigekõrgema Jumala Seaduse:
\par 9 sündides sünnite needuseks ja surres on needus teie osaks!
\par 10 Kõik, mis on mullast, läheb jälle mulda, nõnda ka jumalakartmatud needusest hukatusse.
\par 11 Inimesed leinavad ihu pärast, aga patuse halb nimigi pühitakse ära.
\par 12 Nime eest kanna hoolt, sest see jääb sulle kauemaks kui paljud tuhanded kulda!
\par 13 Hea elu päevad on loetud, hea nimi jääb aga igavesti.

\section*{Häbist}

\par 14 Lapsed! Minu õpetus talletage õnnistuseks! Varjatud tarkus ja nähtamatu varandus - mis kasu on neist mõlemast?
\par 15 Parem inimene, kes varjab oma rumalust, kui inimene, kes varjab oma tarkust.
\par 16 Seepärast tundke häbi, nagu olen seda ütelnud, sest mitte iga häbi ei ole häbi ja kõike ei sünni heaks kiita:
\par 17 häbenege isa ja ema ees hooramist, vürsti ja valitseja ees valet,
\par 18 kohtumõistja ja ülemuse ees eksimist, koguduse ja rahva ees pattu,
\par 19 kaasosalise ja sõbra ees ülekohut, naabrite ees vargust,
\par 20 Jumala tõe ja seaduse rikkumist ning küünarnukiga leivale toetumist,
\par 21 vastuvõetu ja väljaantu salgamist, tervitusele vastamata jätmist,
\par 22 hoorale pilgu heitmist ja sugulastelt palge pööramist,
\par 23 pärandiosa ja kaasavara äravõtmist, teise mehe naise ihaldamist,
\par 24 oma teenijatüdrukule lähenemist - ära astu tema voodi äärde -,
\par 25 sõprade ees solvavaid sõnu - kui oled andnud, siis ära tee etteheiteid -,
\par 26 kuulujutu levitamist ja saladuste avaldamist!
\par 27 Nõnda oled sa tõesti kombekas ja leiad armu kõigi inimeste ees.

\chapter{42}

\section*{Häbist}

\par 1 Aga nende pärast ära häbene, hoolimata isikust, sest see oleks patuks:
\par 2 Kõigekõrgema Seadus ja leping ning otsus, millega sa jumalakartmatule kohut mõistad,
\par 3 kokkulepe kaaslase ja teekäijaga ning pärandiosa andmine omastele,
\par 4 kaalu ja vihtide täpsus ning kasu, olgu seda palju või pisut,
\par 5 raha, mis on saadud kauplemisest, vali lastekasvatus ja nurjatu orja selja veriseks löömine.
\par 6 Alatu naise eest kaitseb pitser ja kus on palju käsi, seal lukusta!
\par 7 Mis sa välja annad, olgu loendatud ja kaalutud, väljaminek ja sissetulek - kõik olgu kirjas!
\par 8 Ära häbene õpetamast arutut ja rumalat ning vana meest, keda süüdistatakse hooratöös! Siis oled sa tõesti hästi kasvatatud ja austatud kõigi elavate ees.

\section*{Isa mure tütre pärast}

\par 9 Tütar on isale varjatud mureks ja rahutus tema pärast viib une: kui ta on noor, et ta ei närtsiks, ja kui ta on abielus, et mees teda ei vihkaks,
\par 10 kui ta on neitsi, et teda ei võrgutataks ja ta ei jääks rasedaks oma isakodus; kui ta on mehel, et ta ei eksiks ja abielus olles ei jääks viljatuks.
\par 11 Kangekaelset tütart valva rangelt, et ta ei teeks sind vaenlaste naerualuseks, kõmuks linnas ja laimatavaks rahvale, ega häbistaks sind suure kogu ees!

\section*{Naistest}

\par 12 Ära vaata ühtki inimest tema ilu pärast ja ära istu naiste seltskonnas!
\par 13 Sest riideist väljub koi ja naisest naise kurjus.
\par 14 Parem kuri mees kui edvistav naine, naine, kes toob häbi ja pilget.

\section*{Jumala tegude ülistus looduses}

\par 15 Ma tahan nüüd meenutada Issanda tegusid ja jutustada, mida olen näinud. Issanda teod on sündinud tema sõna läbi.
\par 16 Päike valgustab ja paistab kõigi peale ja Issanda teod on täis tema enese auhiilgust.
\par 17 Issand ei ole pühitsetuilegi võimaldanud üles lugeda kõiki tema imetegusid, mis Issand, Kõigeväeline, on kinnitanud, et maailm seisaks kindlana tema auhiilguses.
\par 18 Issand uurib sügavikku ja südant, ja mõistab nende riukaid, sest Kõigekõrgem on kõiketeadja ja näeb aegade märke.
\par 19 Tema kuulutab minevikku ja tulevikku ning ilmutab salaasjade jälgi.
\par 20 Ükski mõte ei lähe temal kaotsi, ükski sõna ei jää temale saladuseks.
\par 21 Oma tarkuse suured teod on ta hästi teinud, nõnda, et need jäävad igavikust igavikku. Midagi ei ole lisada, midagi ei ole ära võtta, ja temale ei ole vaja ühtki nõuandjat.
\par 22 Kui imetlusväärsed on kõik tema teod, ja mis neist tuntakse, on kübemeke.
\par 23 Need kõik elavad ja jäävad igavesti, igaks otstarbeks, ja kõik kuulavad tema sõna.
\par 24 Kõik on paari kaupa, teineteisega vastamisi, ja tema ei ole teinud midagi, mis on puudulik.
\par 25 Üks toetab teise hüvangut - ja kes küllastuks tema auhiilguse nägemisest?

\chapter{43}

\section*{Jumala tegude ülistus looduses}

\par 1 Kõrguse uhkuseks on selge taevalaotus, taeva ilu on hiilgav vaatepilt.
\par 2 Päike paistab, kuulutab tõustes - see on imeasi, Kõigekõrgema töö!
\par 3 Keskpäeval ta kuivatab maa ja kes jõuaks seista tema palavuses?
\par 4 Ahju lõõtsutatakse kuumust nõudvateks töödeks, aga kolm korda rohkem kõrvetab päike mägesid. Ta tekitab tulist auru ja kiirgab kiiri, mis pimestavad silmi.
\par 5 Suur on Issand, kes päikese on teinud, tema käsul tõttab see oma teekonnal.
\par 6 Ja kuu! Määratud ajal kõikjal, aja näitaja ning ajastute tähistaja.
\par 7 Kuu järgi märgitakse pühade aega, kuu on valgus, mis kahaneb, kuni kaob.
\par 8 Aeg on oma nime saanud Kuust. Muutudes kuu kasvab imeliselt. Taevalaotuses hiilates on ta kõrguse väehulkadele relvaks.
\par 9 Tähtede sära on taeva ilu, vilkuv kaunistus Issanda kõrgustes.
\par 10 Püha Jumala käsul seisavad tähed oma kohal ega väsi oma valvepostidel.
\par 11 Vaata vikerkaart ja ülista selle Loojat: särades on see väga ilus!
\par 12 See piirab taevast toreda kaarena, Kõigekõrgema käed on selle koolutanud.
\par 13 Tema käsul hakkab äkitselt lund sadama ja ta läkitab kiiresti oma kohtu välke.
\par 14 Selleks avanevad varaaidad ja pilved lendavad välja otsekui linnud.
\par 15 Oma võimsuses ta tihendab pilvi ja raheterad langevad kildudena alla.
\par 16 Tema ilmudes kõiguvad mäed, tema tahtel puhub lõunatuul.
\par 17 Tema piksemürin paneb maa värisema, sünnib tormav põhjatuul ja tuulispea.
\par 18 Ta puistab lund nagu lendlevaid linde, ja siis tuiskab, nagu laskuks rohutirtsuparv. Lume ilusat valevust imetleb silm ja selle sajust vaimustub süda.
\par 19 Ta riputab maa peale härmatist nagu soola, mis jäätub teravaiks okkaiks.
\par 20 Külm põhjatuul puhub ja vesi külmub jääks. See tekib iga veekogu peale ja vesi kattub otsekui soomusrüüga.
\par 21 Ta murendab mäed ja kuumutab kõrbe ning kõrvetab halja rohu otsekui tuli.
\par 22 Udu toob ruttu kosutust kõigile ja kaste langeb jahutuseks pärast palavust.
\par 23 Oma nõu järgi ta vaigistas meresügavused ja istutas sinna saared.
\par 24 Merel sõitjad jutustavad selle hädaohtlikkusest ja me imestame, kui meie kõrvad seda kuulevad.
\par 25 Ja seal on kummalised ning imelised olendid, mitmesugused looma- ja mereelukate liigid.
\par 26 Tänu tema võimsusele õnnestub tema käskjala teekond ja tema sõna läbi püsib kõik.
\par 27 Palju võiksime veel ütelda, ometi ei jõuaks me kuhugi, ja lõppsõnaks on: „Tema on kõik!”
\par 28 Kas suudame teda kiita? Sest tema on suurem kui kõik tema teod.
\par 29 Issand on kardetav ja väga suur ja tema võimsus on imeline.
\par 30 Kiitke Issandat, ülendage teda kõigest jõust: tema on ikkagi veel suurem! Ülendage teda suurema jõuga, ärge väsige: eesmärki ei saavuta te ometi!
\par 31 Kes on teda näinud, et saaks teda kirjeldada? Ja kes suudaks teda ülistada, nõnda nagu ta väärib?
\par 32 Palju on saladusi, suuremad kui need, sest tema tegudest oleme vähe näinud.
\par 33 Sest Issand on teinud kõik ja on tarkuse andnud jumalakartlikele.

\chapter{44}

\section*{Jumala tegude ülistus ajaloos}

\par 1 Kiitkem nüüd kuulsaid mehi ja meie isasid põlvest põlve!
\par 2 Issand on enesele valmistanud suure au ja on näidanud oma suurust igavikust alates:
\par 3 meie isad valitsesid oma kuningriikides ja olid võimu poolest nimekad mehed, kes otsustasid targalt, kes kuulutasid prohvetlikult -
\par 4 nõuandvad rahvajuhid, rahva kirjatundjad, tarkusesõnadega õpetajad,
\par 5 lauluviiside loojad, luuletuste kirjapanijad.
\par 6 Need olid rikkad mehed, kellele oli antud võim, kes elasid rahus oma asupaikades.
\par 7 Nad kõik olid oma sugupõlve ajal austatud ja olid oma elupäevil kuulsad.
\par 8 Mõned neist jätsid järele nime, nõnda et neist räägitakse kiitvalt.
\par 9 Aga on ka neid, keda ei mäletata, kes on kadunud, neid nagu ei oleks olnudki, jah, otsekui ei oleks nad sündinudki, nõndasamuti nende lapsed pärast neid.
\par 10 Teised olid aga õnnistatud mehed, kelle voorusi ei ole unustatud.
\par 11 Nende sooga säilib hea pärand, mis neist põlvneb.
\par 12 Nende järeltulijad peavad lepingut ja nende tõttu nende lapsed.
\par 13 Nende sugu jääb igavesti ja nende au ei kustutata.
\par 14 Nende ihud on rahus maetud ja nende nimi elab põlvest põlve.
\par 15 Rahvad jutustavad nende tarkusest ja kogudus kuulutab nende kuulsust.
\par 16 Eenok meeldis Issandale ja võeti ära: ta oli meeleparanduse eeskujuks tulevastele põlvedele.
\par 17 Noa leiti olevat täiesti õige, viha ajal sai ta lunahinnaks: sellepärast jäi ta maa peale alles, kui veeuputus tuli.
\par 18 Temaga tehti igavene leping, et kõike elavat ei hävitata enam veeuputusega.
\par 19 Aabraham oli paljude rahvaste suur isa, ja au poolest ei leidu temaga sarnast,
\par 20 tema pidas kinni Kõigekõrgema Seadusest ja astus temaga lepingusse. Ta kinnitas lepingu omaenese ihul ja katsumuses leiti ta olevat ustav.
\par 21 Sellepärast on temale vandega kinnitatud, et rahvaid õnnistatakse tema soo nimel, et ta tehakse paljuks otsekui liiv, et ta järeltulijad tõstetakse kõrgele nagu taevatähed ja neid lastakse pärida merest mereni ning jõest kuni ilmamaa ääreni.
\par 22 Ja nõnda on kinnitatud ka Iisakile tema isa Aabrahami pärast:
\par 23 õnnistus ja leping kõigile inimestele, mis pidi jääma Jaakobi pea peale. Issand märkis ta oma õnnistusega ja andis talle pärandi. Ja tema jagas selle osadeks, jaotades need kaheteistkümnele suguharule.

\chapter{45}

\section*{Jumala tegude ülistus ajaloos}

\par 1 Ja Issand laskis temast tõusta õnnistatud mehe, kes leidis armu kogu elava silmis, armas Jumalale ja inimestele: Moosese, kelle mälestus on õnnistuseks.
\par 2 Ta seadis tema au poolest võrdseks pühakutega ja vaenlaste hirmuks tegi tema suureks.
\par 3 Tema sõna läbi laskis ta sündida imetegusid, ta austas teda kuningate ees, andis temale käsud oma rahva jaoks ja näitas temale pisut oma auhiilgusest.
\par 4 Tema ustavuse ja alandlikkuse pärast ta pühitses teda, valis tema kõigi inimeste hulgast,
\par 5 laskis tal kuulda oma häält ja viis ta tumedasse pilve. Ja palgest palgesse andis ta temale käsud, elu ja tunnetuse seaduse, et ta õpetaks Jaakobile lepingut ja Iisraelile tema õiguskorda.
\par 6 Aaroni, tema venna Leevi suguharust, ülendas Issand pühaks, temaga sarnaseks.
\par 7 Ta tegi temaga igavese lepingu ja andis temale rahva preestriameti. Ta rõõmustas teda iluehtega ja pani temale selga aukuue.
\par 8 Ta riietas teda suurima toredusega ja määras temale kindlad esemed: püksid, ülekuue ja õlarüü.
\par 9 Ta ümbritses teda granaatõuntega, paljude kuldkellukestega ümberringi, mis helisesid, kui ta astus, et templis kuulduks kõla meeldetuletuseks tema rahva poegadele.
\par 10 Ja veel: püha kuuega, kullast, sinisest ja purpurpunasest lõngast, kirjatud töö; selle juurde: kohtu-rinnakilp uurimi ja tummimiga, helepunasest korrutatud lõngast, kunstipärane töö,
\par 11 pitsatitaoliselt uurendatud kalliskividega kuldraamistuses, kivikirjaja töö, meeldetuletuseks uurendatud kirjaga vastavalt Iisraeli suguharude arvule;
\par 12 ja kuldlaubaehe peakatte küljes, püha pitsati jäljendiga, uhke aumärk, suurepärane töö, silmailu, veetlev kaunistus.
\par 13 Enne Aaronit ei ole seesugust olnud, eales ei ole seda kandnud võõras, vaid ainult tema oma pojad ja tema järeltulijad üha edasi.
\par 14 Tema ohvrid põletati täiesti kaks korda päevas, alaliselt.
\par 15 Mooses täitis tema käed ja võidis teda püha õliga. Sellest tuli igavene leping temaga ja tema sooga taeva päeviks: teenida Issandat, olla preestriks ja õnnistada tema rahvast tema nimel.
\par 16 Ta valis Aaroni kõigi elavate hulgast Issandale ohvrit tooma: suitsutusohvrit ja healõhnalist mälestusohvrit lepituseks oma rahva eest.
\par 17 Ta andis temale oma käsud, meelevalla seaduste üle, et õpetada Jaakobile oma tunnistusi ja valgustada Iisraeli oma Seaduses.
\par 18 Võõrad astusid üles tema vastu ja kadestasid teda kõrbes, mehed Daatani ja Abirami ümber, ning Korahi jõuk vihas ja raevus.
\par 19 Issand nägi seda ja see ei meeldinud temale: nad hukati vihahoos. Ta tegi nendega imeteo: hävitas nad tuleleegis.
\par 20 Aga Aaronile lisas ta au ja andis temale pärisosa: jaotas temale esmaannid uudseviljast, valmistas külluses leiba kõigepealt.
\par 21 Jah, nad sõid ka Issanda ohvreid, neid, mida ta oli andnud temale ja tema soole.
\par 22 Aga rahva maast ei saanud ta pärisosa ega olnud temal omandit rahva keskel, sest: „Mina ise olen sinu omand ja pärisosa!”
\par 23 Ja Piinehas, Eleasari poeg, au poolest kolmas, oli innukas Issanda kartuses ja rahva taganedes jäi kindlaks õiges hinge otsustavuses ning tõi Iisraelile lepituse.
\par 24 Seetõttu kinnitati temaga rahuleping: ta pidi olema pühamu ja oma rahva eestseisja, nõnda et temale ja tema soole pidi igavesti jääma preestriameti auväärsus.
\par 25 Nõnda nagu oli leping Taavetil, Iisai pojal Juuda suguharust, et kuningriik on päritav ainult pojalt pojale, nõnda oli päritavus ka Aaronil ja tema sool.
\par 26 Andku Issand tarkust teie südamesse, et mõistaksite kohut tema rahvale õiguses, et nende õnn ei kaoks ja nende au jääks põlvest põlve!

\chapter{46}

\section*{Jumala tegude ülistus ajaloos}

\par 1 Joosua, Nuuni poeg, sõjakangelane ja prohvetina Moosese järglane, kes oma nime tõttu oli suur päästma Issanda valituid, kätte maksma kogunenud vaenlastele, et anda Iisraelile tema pärisosa.
\par 2 Kuidas oli ta austatud oma käsi tõstes ja mõõka tõmmates linnade vastu!
\par 3 Kes on enne teda olnud nõnda kange? Jah, tema pidas Issanda sõda!
\par 4 Eks tema läbi jäänud päike seisma ja ühest päevast sai kaks?
\par 5 Ta hüüdis appi kõrgeimat Valitsejat, kui vaenlased rõhusid teda igalt poolt, ja suur Issand kuulis teda: rahet nagu kive võimsa jõuga
\par 6 laskis ta langeda vaenuliku rahva peale ja hävitas vastupanijad mäenõlvakul, et paganad saaksid tunda tema relvi, et Joosua pidas oma sõda Issanda palge ees.
\par 7 Sest tema käis ju Valitseja järel ja juba Moosese päevil osutas ta vagadust, tema ja Kaaleb, Jefunne poeg, kui nad seisid kogudusele vastu, et hoida rahvast patust ja vaigistada kurjade nurinat.
\par 8 Need kaks jäeti ellu kuuesaja tuhande jalamehe hulgast, et neid viia pärisosale, maale, mis piima ja mett voolab.
\par 9 Ja Issand andis Kaalebile rammu, mis jäi temale kuni kõrge eani, et ta jõudis minna üles kõrgendikule ja tema sugu sai kätte pärisosa -
\par 10 et kõik Iisraeli lapsed näeksid, kui hea on käia Issanda järel.
\par 11 Ja kohtumõistjad, igaüks nimeliselt, kõik, kelle süda ei murdnud truudust ja kes ei taganenud Issandast: olgu õnnistatud nende mälestus!
\par 12 Nende luud elustugu oma rahupaikades ja nende nimi uuenegu nende kuulsais poegades!
\par 13 Oma Issandale armas oli Saamuel, Issanda prohvet, kes rajas kuningriigi ja võidis valitsejad tema rahvale.
\par 14 Ta mõistis kohut kogudusele Issanda Seaduse järgi, ja Issand hoolitses Jaakobi eest.
\par 15 Tema usu pärast tunnustati teda prohvetina ja tema kõnest tunti teda kui nägijat.
\par 16 Ta hüüdis appi Issandat, Kõigeväelist, kui vaenlased temale igalt poolt kallale tungisid, ja ohverdas piimatalle.
\par 17 Siis Issand müristas taevas ja andis kuulda oma häält võimsa kõlaga
\par 18 ning hävitas tüüroslaste valitsejad ja kõik vilistite vürstid.
\par 19 Enne uinumist igavesse unne tunnistas ta Issanda ja tema poolt võitu ees: „Ühtki asja, mitte kingagi, ei ole ma kelleltki võtnud!” Ja ükski inimene ei süüdistanud teda.
\par 20 Ka pärast oma surma kuulutas ta prohvetlikult ja näitas kuningale tema lõppu: ennustades andis ta kuulda oma häält maa seest, et lõpetada rahva seaduserikkumine.

\chapter{47}

\section*{Jumala tegude ülistus ajaloos}

\par 1 Ja pärast teda tõusis Naatan, kes prohvetlikult kuulutas Taaveti päevil.
\par 2 Just nagu rasv eraldatakse tänuohvrist, nõnda erines Taavet Iisraeli lastest.
\par 3 Ta mängis lõvidega nagu sokukestega ja karudega, nagu oleksid need lambatalled.
\par 4 Eks ta oma nooruses tapnud hiiglase ja võtnud teotuse rahva pealt, tõstes oma käe lingukiviga ja lüües maha Koljati hooplemise.
\par 5 Sest ta hüüdis appi Issandat, Kõigekõrgemat, ja Issand andis rammu tema paremale käele vägeva sõjamehe hävitamiseks ning oma rahva sarve+ ülendamiseks.
\par 6 Siis ülistati teda kümne tuhande pärast ja kiideti Issanda õnnistuse pärast temale aukrooni andes.
\par 7 Sest ta lõi ümberkaudseid vaenlasi ja purustas oma vastased vilistid, murdes katki nende sarve tänapäevani.
\par 8 Kõigis oma tegudes andis ta kiitust Pühale, Kõigekõrgemale, ülistussõnadega. Ta laulis kõigest südamest ja armastas oma Loojat.
\par 9 Ja ta paigutas altari ette lauljad ning neilt kõlasid võluvad viisid.
\par 10 Ta andis pidupäevadele hiilguse ja korraldas pühade ajad täiuslikult, nõnda et Issanda püha nime kiideti ja pühamu kajas varahommikust alates.
\par 11 Issand võttis ära tema patud ja ülendas tema sarve igaveseks ajaks, andes temale kuningriigi lepingu ja aujärje Iisraelis.
\par 12 Pärast teda tõusis tark poeg, kes tema tõttu kõikjal leidis rahu.
\par 13 Saalomon valitses rahu päevil, tema, kellele Jumal andis rahu igalt poolt, et ta ehitaks tema nimele koja ja valmistaks pühamu igaveseks ajaks.
\par 14 Kui tark sa olid noores eas ja taipu täis otsekui jõgi.
\par 15 Sinu vaim kattis maa ja selle sa täitsid mõistukõneliste tähendamissõnadega.
\par 16 Sinu nimi ulatus kaugetele saartele ja sind armastati sinu rahumeele pärast.
\par 17 Maad imetlesid sind sinu laulude, ütluste, tähendamissõnade ja seletuste pärast.
\par 18 Issanda Jumala nimel, keda nimetatakse Iisraeli Jumalaks, sa kogusid kulda nagu tina ja kuhjasid hõbedat, otsekui oleks see tinamaak.
\par 19 Aga sa kallutasid oma niuded naiste poole ja andsid oma ihu nende meelevalda.
\par 20 Sa soetasid häbipleki oma au külge ja rüvetasid oma seemne. Sa tõid oma laste peale viha ja suure kurvastuse, olles nõnda mõistmatu,
\par 21 et ainuvalitsus jagunes kaheks ja Efraimist sai alguse sõnakuulmatu kuningriik.
\par 22 Aga Issand ei lõpeta oma halastust ega lase ühtki oma sõna tühja minna, ei kaota ka mitte oma valitu järglasi ega hävita selle sugu, kes teda on armastanud. Ta jättis Jaakobile jäägi ja Taavetile temast väljunud juure.
\par 23 Ja Saalomon läks magama oma vanemate juurde ning jättis enese järeltulijaks oma soost Rehabeami, rumaluselt rikka ja mõistuselt puuduliku, kes oma nõu järgi sundis rahva taganema,
\par 24 ja Jerobeami, Nebati poja, kes saatis Iisraeli pattu tegema ning avas Efraimile patutee. Ja nende patte sai väga palju, sellepärast nad viidi ära oma maalt.
\par 25 Jah, nad leiutasid igasugu kurja, kuni karistus neid tabas.

\chapter{48}

\section*{Jumala tegude ülistus ajaloos}

\par 1 Ja Eelija tõusis, prohvet otsekui tuli, tema sõna põles nagu tõrvik.
\par 2 Tema saatis neile näljahäda, ja suures ägeduses vähendas nende sugu.
\par 3 Issanda sõnaga sulges ta taeva ja selsamal kombel laskis ta tule kolm korda alla tulla.
\par 4 Kuidas küll sind austati, Eelija, sinu imetegude pärast! Kes võiks kiidelda, et ta on sinuga sarnane,
\par 5 kes sa Kõigekõrgema sõnaga äratasid surnu surmast ja surmavallast,
\par 6 kes sa kihutasid kuningaid hukatusse ja auväärseid nende asemeilt,
\par 7 kes sa kuulsid Siinail ähvardust ja Hoorebil karistusotsuseid,
\par 8 kes sa võidsid kuningaid kättemaksjaiks ja prohveteid sinu järeltulijaiks,
\par 9 sind võeti üles tulemöllus tuliste hobustega vankris,
\par 10 sinust on kirjutatud, et sa noomid seatud ajal, vaigistad viha, enne kui see puhkeb, pöörad isade südamed poegade poole ja taastad Jaakobi suguharud.
\par 11 Õndsad, kes said sind näha ja rahuldustundega läksid magama. Aga meie ju veel elame!
\par 12 Kui tuulepööris oli peitnud Eelija, siis tema vaim täitis Eliisa. Oma elupäevil ei vankunud Eliisa ühegi vürsti ees ja keegi ei suutnud teda alistada.
\par 13 Ükski asi ei olnud talle võimatu ja tema keha rääkis prohvetlikult veel hauaski.
\par 14 Elades tegi ta imetegusid ja surnunagi olid tema teod imelised.
\par 15 Sellest kõigest hoolimata ei parandanud rahvas meelt ega jätnud maha oma patte, kuni nad vangidena viidi oma maalt ja pillutati kõigisse maadesse.
\par 16 Järele jäi ainult vähe rahvast ja üks vürst Taaveti kojale. Mõned neist tegid, mis õige oli, teised aga kuhjasid patte.
\par 17 Hiskija kindlustas oma linna ja juhtis sinna sisse vett. Ta murdis raudriistadega läbi kalju ja ehitas veehoidlad.
\par 18 Tema päevil tuli Sanherib üles ja läkitas Rabsake; too läks ja tõstis oma käe Siioni vastu ning hooples oma ülbuses.
\par 19 Siis värisesid nende südamed ja käed ja nad vaevlesid otsekui sünnitajad.
\par 20 Nad hüüdsid appi Issandat, halastajat, sirutades oma käsi tema poole. Ja Püha kuulis neid otsekohe taevas ning vabastas Jesaja käe läbi:
\par 21 ta lõi assüürlaste leeri ja tema ingel hävitas nad.
\par 22 Sest Hiskija tegi, mis oli Issandale meelepärane, ja jäi kindlaks oma isa Taaveti teedele, nagu oli käskinud prohvet Jesaja, suur ja usaldusväärne oma nägemustes.
\par 23 Tema päevil läks päike tagasi ja kuninga elu pikendati.
\par 24 Vaimus suurena nägi ta tulevikku ja trööstis leinajaid Siionis.
\par 25 Ta kuulutas, mis tuleb kuni igavikuni, ja salajasi asju enne nende sündimist.

\chapter{49}

\section*{Jumala tegude ülistus ajaloos}

\par 1 Joosija mälestus on nagu segatud suitsutusrohi, valmistatud rohusegajate viisil. Mesimagus on see igas suus, võrdne muusikaga veinipeol.
\par 2 Temal oli õnne rahva pööramisega ja ta kõrvaldas paganlikud jäledused.
\par 3 Ta pööras oma südame Issanda poole, jumalatute päevil süvendas jumalakartlikkust.
\par 4 Peale Taaveti ja Hiskija ja Joosija tegid kõik teised pattu, sest nad hülgasid Kõigekõrgema Seaduse. Siis tuli lõpp Juuda kuningaile.
\par 5 Sest nad andsid oma sarve teistele ja oma au võõrale rahvale.
\par 6 Need süütasid põlema valitud püha linna ja tühjendasid selle tänavad
\par 7 Jeremija pärast. Sest nad olid piinanud teda, kuigi ta juba emaihus oli pühitsetud prohvetiks, välja juurima ja kahju tegema ja hävitama, nõndasamuti ka ehitama ja istutama.
\par 8 Hesekiel nägi nägemust auhiilgusest, mida temale näidati keerubite vankril.
\par 9 Sest ta kujutas vaenlasi paduvihmana ja tõotas head neile, kes olid õigel teel.
\par 10 Ja kaksteist prohvetit - nende luud elustugu haudades! Sest nemad trööstisid Jaakobit ja lunastasid tema kindla lootusega.
\par 11 Kuidas küll peame ülistama Serubbaabelit? Tema oli ju otsekui pitserisõrmus paremas käes.
\par 12 Niisugune oli ka Joosua, Joosadaki poeg. Nende päevil ehitati koda ja püstitati püha tempel Issandale, valmistatud igavese aupaistuse jaoks.
\par 13 Rohkeid mälestusi on ka Nehemjast, kes meile kohendas varisenud müürid, pani paigale väravad ja riivid ning taastas meie kojad.
\par 14 Maa peale ei ole loodud ühtki Eenokiga sarnast, sest tema võeti maa pealt taevasse.
\par 15 Ei ole ka sündinud Joosepiga sarnast meest, kes oli oma vendade juht, rahva tugi, kelle luudegi eest kanti hoolt.
\par 16 Seem ja Sett olid inimeste hulgas austatud, aga üle kõigi loodud elavate olendite on Aadam.

\chapter{50}

\section*{Jumala tegude ülistus ajaloos}

\par 1 See oli ülempreester Siimon, Oniase poeg, kes oma eluajal kohendas koda ja kelle päevil tempel uuendati.
\par 2 Tema ehitas kaks korda kõrgemaks taastatud templi kõrge ringmüüri.
\par 3 Tema päevil rajati veehoidla, tiik, ümbermõõdult otsekui meri.
\par 4 Tema hoolitses oma rahva eest, et kaotust ei tuleks, ja kindlustas linna piiramise vastu.
\par 5 Kui hiilgav ta oli, pöördudes rahva poole koja eesriide tagant välja tulles.
\par 6 Ta oli nagu koidutäht pilvede vahel, nagu täiskuu omal ajal,
\par 7 nagu päike, kui see kiirgab Kõigekõrgema templi kohal, nagu vikerkaar, mis särab toredates pilvedes,
\par 8 nagu roosiõis kevadpäevil, nagu liilia veeallika ääres, nagu palsamipuu oks suvisel ajal,
\par 9 nagu tuli ja suitsutusrohi pannil, nagu meisterlikult valmistatud kuldastja, kaunistatud igasugu kalliskividega,
\par 10 nagu viljarikas õlipuu, nagu küpress, mis kasvab pilvedeni.
\par 11 Kui ta pani selga aukuue ja riietus kogu toredusega, siis astudes üles püha altari juurde, täitis ta pühamu eesõue aupaistusega.
\par 12 Kui ta võttis vastu ohvriosi preestrite kätelt, seistes ise altarilee ääres, siis olid vennad ringina tema ümber nagu noored seedrid Liibanonil, ümbritsedes teda nagu palmitüved.
\par 13 Ja kõik Aaroni pojad oma toreduses, käes Issanda ohvriannid, olid terve Iisraeli koguduse ees.
\par 14 Ja kui ta altariteenistuse oli lõpetanud, siis, et kroonida ohvrit Kõigekõrgemale, Kõigeväelisele,
\par 15 sirutas ta käe ohvrikarika järele ja ohverdas viinamarjade verd, valades seda altari alusele, meeldivaks lõhnaks Kõigekõrgemale, kõikide Kuningale.
\par 16 Siis hüüdsid Aaroni pojad ja puhusid seppade taotud pasunaid, lastes valjult kõlada häält meeldetuletuseks Kõigekõrgema ees.
\par 17 Kogu rahvas tõttas siis üheskoos ja heitis silmili maha kummardama oma Issandat, kõigeväelist Jumalat, Kõigekõrgemat.
\par 18 Lauljad oma häältega kiitsid, väga kaugele kõlas lummav laul.
\par 19 Ja rahvas anus Issandat, Kõigekõrgemat, palvetas Halastaja ees, kuni Issanda austamine oli lõpule viidud ja tema teenimine lõpetatud.
\par 20 Siis astus Siimon alla ja tõstis oma käed terve Iisraeli laste koguduse üle, et oma huultelt anda neile Issanda õnnistus, tundes ise uhkust tema nime pärast.
\par 21 Ja rahvas kummardas jälle, et võtta vastu õnnistus Kõigekõrgemalt.
\par 22 Nüüd kiitke kõik Jumalat, kes kõikjal teeb suuri tegusid, kes ülendab meie päevi emaihust alates ja talitab meiega oma halastust mööda.
\par 23 Tema andku meile südamerõõmu ja et meie päevil oleks Iisraelis rahu, nagu oli muistseil päevil.
\par 24 Tema halastus olgu alati meiega ja ta lunastagu meid meie elupäevil.
\par 25 Minu hing vihkab kaht rahvast, ja kolmas ei olegi rahvas:
\par 26 neid, kes asuvad Seiri mäestikus, ja vilisteid ja seda rumalat rahvast, kes elab Sekemis.

\section*{Kirjutaja lõppsõna}

\par 27 Arukuse ja mõistlikkuse õpetuse olen selles raamatus kirja pannud, mina, Jeesus, Siiraki poeg, Eleasari pojapoeg Jeruusalemmast, olles tarkust puistanud oma südamest.
\par 28 Õnnis on see, kes sellega tutvust teeb, ja kes seda südamesse võtab, saab targaks.
\par 29 Sest kui ta seda teeb, siis on temal jõudu kõigeks - on ju Issanda valgus tema teejuht.

\chapter{51}

\section*{Tänulaul}

\par 1 Ma ülistan sind, Issand, Kuningas, ja kiidan sind, Jumal, mu Lunastaja! Ma ülistan sinu nime!
\par 2 Sest sina oled olnud mu kaitsja ja aitaja, sa oled päästnud mu ihu hukatusest ja kurja keele püügipaelast, nende huultest, kes välja mõtlevad valet. Sa oled olnud mu abimees vaenlaste ees ning oled mu lunastanud
\par 3 oma suure halastuse ja oma nime pärast nende salvamistest, kes olid valmis mind neelama, nende käest, kes püüdsid mu hinge, paljudest viletsustest, mis mul on olnud,
\par 4 lämbumisest tules, mis mind ümbritses, ja tule keskelt, mida ma ise ei olnud süüdanud,
\par 5 surmavalla sügavast sülest ning kasimata keelest ja valelikust kõnest,
\par 6 ülekohtuse keele laimust kuninga ees. Mu hing ligines surmale ja mu elu oli lähedal surmavallale, mis on all.
\par 7 Mind piirati igalt poolt ja aitajat ei olnud, ma otsisin abi inimestelt, aga seda ei tulnud.
\par 8 Siis meenutasin ma sinu halastust, Issand, ja sinu tegusid igavikust alates, sest sina vabastad need, kes sinu peale loodavad, ja päästad nad vaenlaste käest.
\par 9 Ja ma lasksin tõusta maa pealt oma appihüüde ning palusin päästmist surmast.
\par 10 Ma hüüdsin appi Issandat, oma Issanda Isa, et ta mind ei jätaks maha viletsuse päevil, abituna ülbete meelevalda: „Ma ülistan lakkamata sinu nime ja laulan tänades!”
\par 11 Ja mu palvet võeti kuulda. Jah, sina päästsid mind hävingust ja võtsid mu ära kurjast ajast.
\par 12 Sellepärast ma ülistan ja kiidan sind, jah, ma austan Issanda nime!

\section*{Tarkuse taotlemisest}

\par 13 Olles veel noor, enne oma eksirännakuid, otsisin ma tarkust avalikult palvetades.
\par 14 Templi ees ma õppisin seda hindama ja ma otsisin seda elu lõpuni.
\par 15 Selle õitsengust rõõmustas mu süda nagu valmivast viinamarjakobarast, mu jalg astus sirget teed, noorusest alates olen seda otsinud.
\par 16 Ma pöörasin oma kõrva ja võtsin selle vastu, ma leidsin enesele rohkesti õpetust.
\par 17 Ma arenesin selle läbi. Temale, kes minule andis tarkuse, annan ma au.
\par 18 Sest ma võtsin nõuks seda saavutada ja püüdlesin hea poole ega jäänud häbisse.
\par 19 Mu hing on selle pärast võidelnud ja oma tegudes ma jälgisin täpselt seadust. Ma tõstsin käed kõrguse poole ja kahetsesin oma teadmatust.
\par 20 Ma juhtisin oma hinge selle poole ja puhtuse kaudu ma leidsin selle. Ma andsin oma südame algusest peale sellele, seepärast mind ei jäetagi maha.
\par 21 Mu sisemus kehutas seda otsima, seetõttu ma sain hea varanduse.
\par 22 Issand andis mulle tasuks minu keele ja sellega ma kiidan teda.
\par 23 Tulge minu juurde, õpetamatud, ja jääge õpetuse kotta!
\par 24 Ütelge, miks tahate sellest eemale jääda ja teie hing peab tundma suurt janu?
\par 25 Mina olen oma suu avanud ja ütelnud: „Ostke enesele ilma rahata!”
\par 26 Painutage oma kael ikke alla ja teie hing võtku õpetust: see on ligidalt leitav!
\par 27 Nähke oma silmaga, kuidas ma vähese vaevaga olen enesele saanud suure rahu!
\par 28 Saage õpetuse osaliseks rohke hõbeda hinnaga, siis saate selle eest palju kulda!
\par 29 Teie hing olgu rõõmus Issanda halastuse pärast ja ärge häbenege teda kiita!
\par 30 Tehke oma tööd, kuni on aega, siis ta annab teile tasu omal ajal!


\end{document}