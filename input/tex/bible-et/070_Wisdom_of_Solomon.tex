\begin{document}

\title{Saalomoni Tarkuseraamat}

\chapter{1}

\section*{Manitsus otsida Jumalat ja taganeda patust}

\par 1 Armastage õigust, maised kohtumõistjad, mõtelge ausalt Issanda peale ja otsige teda siira südamega!
\par 2 Sest tema leiavad need, kes teda ei kiusa, ja ta ilmutab ennast neile, kes temas ei kahtle.
\par 3 Väärad mõtted viivad eemale Jumalast, ja tema võimsus, kui seda proovitakse, karistab meeletuid.
\par 4 Ei, tarkus ei tule kurikavalasse hinge ega ela patuga leppivas ihus.
\par 5 Karistuse püha vaim põgeneb pettuse eest ja jääb eemale sõgedaist mõtteist - ülekohtu lähenemine peletab tema.
\par 6 Tarkus on inimest armastav vaim ega jäta teotajat karistamata tema huulte pärast. Jumal on ju tema sisemuse tunnistaja, tema südame tõeline valvur ja tema keele kuulja.
\par 7 Sest Issanda Vaim täidab maailma, ja tema, kes kõike koos hoiab, teab, mida räägitakse.
\par 8 Ei jää varjule ükski, kes valet räägib, ja õiguse karistus ei lähe temast mööda.
\par 9 Sest jumalakartmatu inimese kavatsusi uuritakse, sõnum tema sõnadest tuleb ju Issanda ette tema eksimuste karistamiseks.
\par 10 Armukadeda kõrv kuuleb kõike, sosistaja nurin ei jää salajaseks.
\par 11 Seepärast hoiduge kasutust nurisemisest ja keelake keelt laimamast, sest salajane kõne ei jää tagajärjeta, ja valelik suu tapab hinge.
\par 12 Ärge otsige surma oma elu eksiteel ja ärge tooge hukatust oma käte tegudega!
\par 13 Sest Jumal ei ole teinud surma ega tunne rõõmu elavate hukkumisest.
\par 14 Tema on ju kõik loonud, et see jääks, ja mis maailmas on sündinud, on tarvilik. Selles ei ole hukatuse mürki, maa peal ei ole surmavalla valitsust.
\par 15 Sest õigus on surematu.

\section*{Jumalakartmatute elukäsitusest}

\par 16 Jumalakartmatud aga kutsuvad käte ja sõnadega enese juurde surma. Pidades teda küll sõbraks, kaovad nad ometi. Nad teevad temaga lepingu, sest nad on väärt saama tema omaks. 

\chapter{2}

\section*{Jumalakartmatute elukäsitusest}

\par 1 Sest nad ütlevad endamisi, arutledes valesti: „Meie elu on lühike ja kurb, ei ole rohtu inimese surma vastu ega ole teada, kes päästaks surmavallast.
\par 2 Sest me oleme sündinud juhuse läbi, ja pärast seda elu oleme, otsekui ei oleks meid olnudki. Sest hingeõhk meie sõõrmeis on suits, ja mõtlemine ainult säde meie südame tuksumisest.
\par 3 Kui see kustub, siis meie ihu muutub tuhaks ja vaim haihtub otsekui hõre õhk.
\par 4 Ajapikku unustatakse meie nimi ja keegi ei meenuta meie tegusid. Meie elu möödub otsekui pilve vari ja hajub, nagu oleks see udu, taga aetud päikesekiirtest ja viidud nende palavusest.
\par 5 Sest meie elu on mööduv vari ega ole tagasitulekut surmast. Sest see on pitseriga kinni pandud ja keegi ei pööra seda.
\par 6 Tulge siis, saagem osa heast, mis on olemas, ja nautigem innuga maailma nagu nooruspäevil!
\par 7 Täitkem endid kalli veiniga ja salvidega, ja ärgu mingu meist mööda kevade õis!
\par 8 Pärjakem endid roosinuppudega, enne kui need närtsivad!
\par 9 Ükski meist ärgu hoidugu ülemeelikusest, jätkem kõikjale oma rõõmu jäljed, sest see on meie osa ja liisk!
\par 10 Rõhugem vaest õiget, ärgem heitkem armu lesknaisele, ärgem hooligem elatanud rauga hallidest juustest!
\par 11 Meie õiguse mõõduks olgu meie jõud, sest nõtra peetakse kõlbmatuks!
\par 12 Varitsegem õiget, sest ta on meile tülikas ja on vastu meie tegudele, laidab meid seaduserikkumise pärast ja heidab meile ette meie kasvatuse puudujääke.
\par 13 Tema kuulutab, et ta tunneb Jumalat, ja nimetab ennast Issanda lapseks.
\par 14 Tema on saanud etteheiteks meie mõtetele, juba pealtnäha on ta meile koormaks.
\par 15 Sest tema elu ei ole niisugune nagu teistel ja tema teed on erinevad.
\par 16 Tema peab meid võltsiks ja hoidub meie teedest kui rüvedaist. Tema kiidab õndsaks õigete lõppu ja kiitleb, et Jumal on tema isa.
\par 17 Vaadakem, kas tema sõnad on tõde, ja uurigem, missugune on ükskord tema enese lõpp!
\par 18 Sest kui see õige on Jumala poeg, küllap siis Jumal tema eest ka hoolitseb ja päästab vastaste käest.
\par 19 Pangem ta proovile pilke ja piinamisega, et tema tublidust tundma õppida ja vastupidavust uurida!
\par 20 Mõistkem ta häbistavasse surma, kuigi temal olevat kaitse, nagu ta ise ütleb.” 

\section*{Jumalakartmatute eksitus}

\par 21 Nõnda nad mõtlevad, aga eksivad, sest nende kurjus on nad pimestanud.
\par 22 Nad ei mõista Jumala saladusi, ei looda vagaduse tasu ega pea auasjaks laitmatuid hingi.
\par 23 Jumal ju lõi inimese surematuks, tehes ta sarnaseks omaenese olemusega.
\par 24 Aga kuradi kadeduse läbi tuli maailma surm, ja seda kogevad need, kes kuuluvad temale. 

\chapter{3} 

\section*{Õigete ja jumalakartmatute võrdlus}

\par 1 Õigete hinged on aga Jumala käes ja neid ei puuduta ükski piin.
\par 2 Rumalate silmis näivad nad olevat surnud, ja nende äraminekut peetakse õnnetuseks,
\par 3 lahkumist meie keskelt hukatuseks; nemad aga on rahus.
\par 4 Sest kuigi neid inimeste meelest nuheldakse, on nende lootuseks täielik surematus.
\par 5 Pärast kerget karistust saavad nad osa rohkest heast, sest Jumal on neid proovile pannud ja leidnud, et nad on tema väärilised.
\par 6 Ta on neid proovinud nagu kulda sulatusahjus, ja vastu võtnud kui täieliku põletusohvri.
\par 7 Oma läbikatsumise ajal nad hiilgavad ja on nagu õlgedes lendlevad sädemed.
\par 8 Nad mõistavad kohut paganate üle ning valitsevad rahvaid, ja Issand on igavesti nende kuningas.
\par 9 Kes tema peale loodavad, tunnetavad tõde, ja armastuses ustavad jäävad tema juurde, sest tema valituile saab osaks arm ja halastus.
\par 10 Aga jumalakartmatud saavad nuhtluse, mida nende mõtted väärivad, need, kes õigest ei ole hoolinud ja on Issandast taganenud.
\par 11 Sest õnnetu on, kes põlgab tarkust ja kasvatust. Nende lootus on tühine ja vaev asjatu, ja nende teod ei kõlba kuhugi.
\par 12 Nende naised on rumalad ja nende lapsed on pahad - neetud on nende sugu. 

\section*{Parem lastetus kui jumalakartmatud järglased}

\par 13 Aga õnnis on süütu sigimatu, kes ei tunne patust armatsemist - tema kannab vilja hingede läbikatsumise ajal.
\par 14 Ja õnnis on kohitsetu, kes oma käega ei ole teinud seadusevastast ega ole mõtelnud kurja Issanda vastu. Sest temale antakse usu valitud armuand ja kõige armsam osa Issanda templis.
\par 15 Jah, auline on vooruste vili ja laitmatu on tarkuse juur.
\par 16 Aga abielurikkujate lapsed ei saa täisealiseks, ja keelatud armatsemisest sündinud kaovad.
\par 17 Sest kui nad elaksidki kaua, neid ei pandaks millekski, ja nende vana iga on lõpuks autu.
\par 18 Kui nad ka varakult surevad, ei ole neil lootust ega troosti otsustuspäeval.
\par 19 Sest truudusetu soo lõpp on hirmus. 

\chapter{4}

\section*{Parem lastetus kui jumalakartmatud järglased}

\par 1 Parem on lastetus koos voorusega, sest vooruse mälestus on surematu - tunnustatakse seda ju Jumala ja inimeste poolt.
\par 2 Kui see on olemas, siis matkitakse seda, ja kui seda ei ole, siis igatsetakse seda. Ja igavikus uhkustab see pärga kandes, olles võitnud ausas võitluses.
\par 3 Jumalakartmatute lastekari on aga kasutu, ja kes on värdjalike võsude seast, ei juurdu sügavalt ega saa kindlat põhja.
\par 4 Sest kuigi selle oksad hetkeks haljendavad, varistab tuul neid kui varisemisvalmeid ja torm kisub nende juured välja.
\par 5 Õrnad oksad murtakse ümbertringi ja nende vili jääb kõlbmatuks, söömiseks tooreks ja kasutamata.
\par 6 Sest lapsed, kes on sündinud lubamatust armatsemisest, tunnistavad oma vanemate vastu, kui neid ükskord läbi uuritakse. 

\section*{Õige inimese varajasest surmast}

\par 7 Õige aga saab rahu, kuigi ta varakult sureb.
\par 8 Sest austatav vanus ei ole sama mis pikk iga, ja seda ei mõõdeta aastate arvu järgi,
\par 9 vaid tarkus on inimesele hallideks juusteks, ja laitmatu elu on tõeline vanaduspõlv.
\par 10 Jumal armastas teda, sellepärast et ta meeldis Jumalale, ja ta võeti ära, sellepärast et ta elas patuste keskel.
\par 11 Ta viidi ära, et kurjus tema mõistust ei muudaks ja kavalus tema hinge ei eksitaks.
\par 12 Sest kurja veetlevus varjab hea, ja himu hullus väänab vaga meele.
\par 13 Olles varakult küpsenud, saavutas ta pika eluea.
\par 14 Et tema hing oli Issandale meelepärane, siis tõttas see ära kurjuse keskelt. Inimesed näevad seda küll, aga ei saa aru ega võta südamesse,
\par 15 et tema valituile on arm ja halastus ja tema vagade eest kantakse hoolt.
\par 16 Õige mõistab surnunagi hukka elavaid jumalakartmatuid, ja kiiresti küpsenud noorus patuse kõrge eluea.
\par 17 Sest need näevad küll targa lõppu, aga ei mõista, mida Issand tema kohta on otsustanud ja miks ta teda on kaitsnud.
\par 18 Nad näevad, ja ei pane seda millekski, aga tema ise, Issand, naerab nende üle.
\par 19 Ja siis saavad neist põlatud korjused ning igavesti pilkealused surnute seas. Sest ta paiskab nad hääletuina põrmu ja tõukab aluselt. Nad hävitatakse sootuks ning nende osaks on piin. Ja mälestus neist kustub. 

\section*{Õige ja jumalakartmatu kohtu ees}

\par 20 Nad tulevad aralt, kui nende patte arvele võetakse, kui nende üleastumised tõusevad süüdistajatena nende vastu. 

\chapter{5}

\section*{Õige ja jumalakartmatu kohtu ees}

\par 1 Õige astub aga suure julgusega oma rõhujate ja teiste ette, kes tema hädast ei hoolinud.
\par 2 Seda nähes nad värisevad kohutavas hirmus ja jahmuvad tema ootamatu päästmise pärast.
\par 3 Kahetsedes räägivad nad omavahel, ohkavad hingeahastuses ja ütlevad:
\par 4 „See on see, keda me kunagi naeruks panime ja kes oli meile sõimusõnaks. Meie, meeletud, pidasime tema elu hulluseks ja tema surma häbistavaks.
\par 5 Kuidas küll teda on arvatud Jumala laste hulka ja miks on tema pärand püha rahva keskel?
\par 6 Jah, meie oleme eksinud tõe teelt! Ei ole meile paistnud õiguse valgus ega ole meile tõusnud selle päike.
\par 7 Meie leidsime rahuldust patu ja hukatuse radadel, rändasime teedeta kõrbetes, aga Issanda teed me ei tundnud.
\par 8 Mis kasu oli meil uhkusest, ja mis abi andis meile rikkus ja hooplemine?
\par 9 See kõik on kadunud nagu vari või nagu põgus kuuldus;
\par 10 või nagu laev, mis sõidab läbi lainetava vee: ei leidu jälge selle minekust ega kiilu rada lainetes;
\par 11 või nagu läbi õhu lendav lind, kelle teekonnast ei leita märki - ainult tiivalöögid piitsutavad mahedat tuult ja lõikavad seda läbilennul tiibadest sündinud hoogsa sahinaga - pärast seda aga ei leita sealt liikumise jälge;
\par 12 või nagu märki lastava noole puhul vajub lõhestatud õhk otsekohe kokku, nõnda et lennujoont enam ei tunta -
\par 13 nõnda oleme ka meie: vaevalt sündinud, juba surnud, mingit vooruse märki ei ole meil näidata, küll aga oleme kulunud oma kurjuses.”
\par 14 Sest jumalakartmatu lootus on nagu tuule viidud vaht, nagu tormi aetud peen härmatis, nagu tuule hajutatud suits, nagu mööduv mälestus ühepäevasest peatujast.
\par 15 Õiged aga elavad igavesti, nende tasu on Issandas, ja Kõigekõrgem kannab hoolt nende eest.
\par 16 Sellepärast nad saavad väärt kuningriigi ja kauni krooni Issanda käest, sest tema kaitseb neid oma parema käega ja tema käsivars on neile kilbiks.
\par 17 Tema võtab sõjavarustuseks oma püha viha ja ta relvastab kogu looduse kättemaksuks vaenlastele.
\par 18 Ta paneb selga õiguse soomusrüü, ja pähe äraostmatu kohtu kiivri.
\par 19 Ta võtab pühaduse võitmatu kilbi,
\par 20 mõõgaks ihub aga kange viha, ja maailm läheb koos temaga sõtta meeletute vastu.
\par 21 Tabavad välgunooled sähvatavad ja lendavad märki otsekui pilvede koolutatud vibust.
\par 22 Vihavalingust paiskub rohkesti rahet, merevesi raevutseb nende vastu, jõed aga uputavad armutult.
\par 23 Võimsuse vaim tõuseb nende vastu ja pillutab neid otsekui torm. Muidu ju ülekohus laastaks kogu maa ja kuritöö tõukaks ümber valitsejate aujärjed. 

\chapter{6}

\section*{Valitsejad taotlegu tarkust}

\par 1 Kuulge siis, kuningad, ja mõistke; õppige, ilmamaa kohtumõistjad!
\par 2 Pange tähele, rahvahulkade valitsejad, ja kes olete uhked paganate jõukuse pärast!
\par 3 Sest teie valitsus on Issanda antud, ja võim on Kõigekõrgemalt, kes teie tegusid kaalub ja kavatsusi uurib.
\par 4 Sest kuigi te olete tema kuningriigi teenrid, ei ole teie õigesti kohut mõistnud ega Seadust valvanud, ei ole teinud Jumala tahte kohaselt.
\par 5 Kohutavalt ja äkitselt astub ta teile vastu, sest ülemusi tabab karm kohus.
\par 6 Väikseimat küll vabandatakse halastuse pärast, vägevaid aga katsutakse vägevasti läbi.
\par 7 Sest kõigi Valitseja ei karda kedagi ega tagane jõu ees. Tema ise on ju teinud väikese ja suure ning kannab ühtviisi hoolt kõigi eest.
\par 8 Vägevat ootab aga range läbikatsumine.
\par 9 Teile siis, ainuvalitsejad, on minu sõnad, et te õpiksite tarkust ega eksiks.
\par 10 Sest kes pühadusi pühaks peavad, neid pühitsetakse, ja kes neis on õpetust saanud, neid kaitstakse.
\par 11 Paluge seepärast minu sõnu, igatsege neid, siis teid õpetatakse! 

\section*{Tarkus on leitav}

\par 12 Hiilgav ja kadumatu on tarkus, hõlpsasti nähtav neile, kes seda armastavad, ja leitav neile, kes seda otsivad.
\par 13 Juba ette ta annab ennast tunda neile, kes teda nõuavad.
\par 14 Kes seda varakult teeb, ei tarvitse vaeva näha, sest ta leiab selle istumas oma ukse ees.
\par 15 Jah, selle peale mõtlemine on mõistuse tipp, ja kes selle pärast on valvel, on peagi muretu.
\par 16 Sest tarkus ise käib ringi, otsides eneseväärseid, ilmutab ennast neile lahkesti nende teedel ja kohtab neid igas nende mõttes.
\par 17 Sest tarkuse tõeliseks alguseks on õpihimu, hool õppida oleneb aga armastusest.
\par 18 Ent armastus on tarkuse seaduste täitmine, seaduste tähelepanemine omakorda tagab surematuse.
\par 19 Surematus aga annab, et ollakse Jumala lähedal.
\par 20 Järelikult viib tarkuseihalus Jumala riiki.
\par 21 Kui nüüd teile, rahvaste valitsejad, meeldivad aujärjed ja valitsuskepid, siis austage tarkust, et saaksite igavesti valitseda! 

\section*{Tarkuse iseloomustus}

\par 22 Aga mis tarkus on ja kuidas see on sündinud, seda ma kuulutan ega varja teie eest saladusi, vaid jälgin seda selle sündimisest alates ja teen selgeks selle äratundmise ega lähe tõest mööda.
\par 23 Ka ei taha ma käia koos näriva kadedusega, sest sellel ei ole tarkusega midagi ühist.
\par 24 Aga tarkade rohkus on maailma lunastus ja tark kuningas on rahva tugevus.
\par 25 Seepärast võtke õpetust minu sõnadest, siis te saate sellest kasu! 

\chapter{7}

\section*{Ka Saalomon on ainult inimene}

\par 1 Minagi olen surelik inimene nagu kõik, ja selle järglane, kes esimesena mullast tehti. Emaihus sain ma lihaks
\par 2 kümne kuu kuludes, tihenedes veres mehe seemnest ja unega kaasnevast ihast.
\par 3 Minagi, olles sündinud, hingasin sisse üldist õhku ja langesin maa peale, nagu saab osaks kõigile. Minu esimene hääl oli nutt just nagu kõigil muilgi.
\par 4 Mähkmeis ja hooles kasvatati mind.
\par 5 Mitte ühelgi kuningal ei ole olnud elu algus teistsugune.
\par 6 Tulek ellu on kõigil seesama, äraminek nõndasamuti. 

\section*{Tarkuse ülistus}

\par 7 Sellepärast ma palvetasin, ja minule anti mõistus, hüüdsin, ja minule tuli tarkuse vaim.
\par 8 Seda ma eelistasin valitsuskeppidele ja aujärgedele, ja sellega võrreldes ei pannud ma rikkust millekski.
\par 9 Sellega võrdseks ma ei pidanud hindamatut kalliskivi, sest tarkuse kõrval on kõik kuld nagu pisut liiva, ja hõbedat tuleb arvata sõnnikuks.
\par 10 Ma armastasin seda rohkem kui tervist ja ilu ja tahtsin seda pidada valguse asemel, sest selle paistus on igavene.
\par 11 Aga koos sellega tuli minule ka kõik hea, ja sellel oli käes mõõtmatu rikkus.
\par 12 Ma tundsin rõõmu kõigest, et tarkus juhatab neid, aga ma ei teadnud, et see on ka nende sünnitaja.
\par 13 Ausalt õppisin, kadeduseta annan edasi, selle rikkust ma ei salga.
\par 14 Sest see on inimestele ammendamatuks varanduseks, ja need, kes selle saavad, sõlmivad sõpruse Jumalaga andide soovitusel, mida õpetus annab.
\par 15 Mind aga Jumal lubagu rääkida, nõnda nagu ma tahan, ja väärikalt mõtiskleda antu üle, sest tema ise on tarkuse teenäitaja ja tarkade juht.
\par 16 Sest tema käes oleme meie, ja on meie sõnadki, ka arukus ning tööoskus.
\par 17 Jah, tema andis mulle kindla teadmise sellest, mis on olemas, nõnda et ma tunnen maailma ehitust ja algainete mõjujõudu,
\par 18 aegade algust ja lõppu ja keset, pööripäevi ja aastaaegade vaheldumist,
\par 19 aastate veeremist ja tähtede seisu,
\par 20 loomade loomust ja kiskjate verejanu, vaimude väge ja inimeste mõtteid, taimede erinevust ja juurte jõudu.
\par 21 Mina tean, mis on peidetud ja mis on avalik, sest tarkus, kõige meister, on mind õpetanud.
\par 22 Sest temas on vaim: arukas, püha, ainulaadne, mitmekülgne, õrn, liikuv, läbinägev, rüvetamatu, kirgas, kahjustamatu, head armastav, terav,
\par 23 vaba, heategev, inimesearmastaja, kindel, usaldatav, muretu, ja läbitungiv kõikidest vaimudest - arukatest, puhastest, õrnadest.
\par 24 Sest tarkus on liikuvam igast liikumisest. Oma puhtuse tõttu ta läheb ja tungib läbi kõigest.
\par 25 Tema on ju Jumala võimsuse hingus ja Kõigeväelise auhiilguse selge kiirgus, seepärast ei tungi midagi rüvedat temasse.
\par 26 Sest ta on igavese valguse paistus ja Jumala mõjujõu puhas peegel ning tema headuse kuju.
\par 27 Kuigi ainus, suudab ta kõike, ja jäädes muutumatuks, uuendab kõike. Põlvest põlve läheb ta vagade hingedesse ja teeb nõnda Jumala sõpru ning prohveteid.
\par 28 Sest Jumal armastab ainult seda, kes elab koos tarkusega.
\par 29 Jah, tarkus on toredam kui päike ja on üle igast tähtede seisust. Valgusega võrreldes leitakse see olevat suurepärasem.
\par 30 Sest valgusele järgneb ju öö, aga tarkust ei võida kurjus. 

\chapter{8}

\section*{Tarkuse ülistus}

\par 1 Tarkus ulatub võimsasti maailma äärest ääreni ja valitseb kõike hästi. 

\section*{Tarkuses on kõik voorused}

\par 2 Tarkust olen ma armastanud ja otsinud oma noorpõlvest alates. Ma tahtsin teda pruudina koju tuua ja sain tema ilu armastajaks.
\par 3 Tema ülistab oma üllast päritolu, et ta elab koos Jumalaga ja kõige Valitseja armastab teda.
\par 4 Tarkus on pühendatud Jumala saladustesse ja on tema tegude valija.
\par 5 Kui elus on rikkus ihaldusväärt varanduseks, mis on siis veel suurem rikkus kui kõike saavutav tarkus?
\par 6 Ja kui mõistus midagi teeb, kes siis kogu maailmas on suurem meister kui tema?
\par 7 Kui keegi armastab õiglust, siis on tema tegudeks voorused. Sest ta õpetab mõõdukust ja arukust, õigust ja vaprust - inimestel ei ole elus midagi kasulikumat.
\par 8 Ja kui keegi ihaldab rikkalikke kogemusi, siis tarkus tunneb minevikku ja ennustab tulevikku. Ta mõistab vanasõnu ja mõistatusi, teab juba ette märke ja imetähti ning ajastute ja aegade kulgu. 

\section*{Valitsejaile on tarkus tarvilik}

\par 9 Sellepärast ma otsustasin tuua tarkuse koju elukaaslaseks, teades, et ta annab mulle head nõu ning trööstib mures ja kurbuses.
\par 10 Tema tõttu on mul kuulsus rahva hulgas ja au vanade juures, kuigi olen noor.
\par 11 Kohtus peetakse mind teravmeelseks, ja valitsejate silmis olen imetlusväärne.
\par 12 Kui ma vaikin, siis nad ootavad mind, kui ma häält tõstan, siis nad panevad mind tähele, ja kui ma pikalt räägin, siis nad panevad käe suu peale.
\par 13 Tarkuse pärast olen ma surematu ja jätan oma järeltulijaile igavese mälestuse.
\par 14 Ma valitsen rahvaid ja rahvahõimud alistuvad minule.
\par 15 Hirmuvalitsejad kardavad, kui minust kuulevad. Aga rahva hulgas osutan ma headust, ja sõjas olen vapper.
\par 16 Kui ma lähen oma kotta, siis otsin kosutust tarkuselt, sest temaga läbikäimine ei valmista meelepaha ega ole tema seltskonnas valu, vaid on rõõm ja rahulolu.

\section*{Tarkuse kasulikkusest}

\par 17 Seda mõteldes ja südames kaalutledes, et tarkusega on suguluses surematus,
\par 18 et sõprus temaga on suur nauding, et tema kätetööst tuleb ammendamatu rikkus, et harjumus suhelda temaga annab arukuse ja osasaamine tema sõnadest kuulsuse, käisin ma ringi ja otsisin, kuidas teda tuua enese juurde.
\par 19 Mina olin küll andekas laps ja olin saanud hea meelelaadi,
\par 20 või veel enam: et olin hea, siis olin tulnud puhtasse ihusse.
\par 21 Aga kui mõistsin, et ma ei või seda saada teisiti, kui et Jumal selle mulle annab - seega on tarkus teadmises, kellelt see armuand on -, siis ma kohtasin Issandat ja palusin teda ning ütlesin kõigest südamest: 

\chapter{9}

\section*{Palve tarkuse pärast}

\par 1 „Vanemate Jumal ja halastuse Issand, kes sa oma sõnaga oled teinud kõik,
\par 2 ja oma tarkuses oled valmistanud inimese, et ta valitseks sinu poolt loodut,
\par 3 korraldaks maailma jumalakartuse ja õiglusega ning mõistaks kohut ausa meelega.
\par 4 Anna mulle sinu aujärgede juures olevat tarkust ja ära arva mind välja sinu laste hulgast!
\par 5 Sest mina olen sinu sulane ja sinu teenija poeg, nõder inimene ja üürikese elueaga, liiga väeti aru saama kohtust ja Seadusest.
\par 6 Kui keegi olekski täiuslik inimeste hulgas, ometi ei pandaks teda millekski, kui temal puudub sinu tarkus.
\par 7 Sina oled minu valinud kuningaks sinu rahvale ja kohtumõistjaks sinu poegadele ja tütardele.
\par 8 Sina käskisid mind ehitada templi sinu pühale mäele ja altari sinu asupaiga linna, selle püha telgiga sarnaseks, mille sina esialgu valmistasid.
\par 9 Sinu juures on tarkus, ta tunneb su tegusid, ta oli juures siis, kui sina lõid maailma, ta teab, mis on meeldiv sinu silmis ja mis on õige sinu käskude kohaselt.
\par 10 Läkita tema pühadest taevastest ja saada oma hiilguse aujärjelt, et ta koos minuga tööd teeks ja ma teaksin, mis on sinule meelepärane!
\par 11 Sest tema teab ja mõistab kõike ning juhatab mind mõistlikult minu tegudes ja hoiab mind oma auhiilguses.
\par 12 Siis kiidetakse minu tegusid ja ma juhin sinu rahvast õigesti ning olen oma isa aujärje vääriline.
\par 13 Sest missugune inimene teab Jumala nõu? Või kes saab uurida, mida Issand tahab?
\par 14 Sest surelike mõtted on tühised ja meie kavatsused kõiguvad.
\par 15 Sest kaduv ihu koormab hinge ja maine telk rõhub mõtisklevat meelt.
\par 16 Vaevalt aimame maist ja suure vaevaga avastame käesolevat, kes suudaks siis uurida, mis taevas on?
\par 17 Kes aga tunneks sinu nõu - olgu siis, et sina annad tarkust ja läkitad kõrgusest oma püha Vaimu?
\par 18 Ainult nõnda õgvendatakse maa peal elavate teerajad, ja inimesed õpivad seda, mis sinule on meelepärane, jah, tarkuse läbi nad päästetakse.” 

\chapter{10}

\section*{Tarkus ajaloos}

\par 1 Seesama tarkus hoidis maailmale tehtud esimest isa, kes üksikuna oli loodud, ja päästis tema ta üleastumisest
\par 2 ning andis temale meelevalla valitseda kõige üle.
\par 3 Aga kui siis ülekohtune oma vihas tarkusest taganes, hukkus ta vennatapu raevuhoost.
\par 4 Kui maa oli tema pärast üle ujutatud, siis päästis selle jälle tarkus, juhtides õiget lihtsa puu varal.
\par 5 Ja kui rahvad üksmeelses kurjuses olid segadusse sattunud, oli tarkus see, kes tundis ära õige ja hoidis teda laitmatuna Jumala ees ning laskis tal kindlaks jääda, hoolimata armastusest poja vastu.
\par 6 Samuti päästis tarkus õige jumalakartmatute hukatusest, kui see põgenes tule eest, mis langes viie linna peale;
\par 7 nende kurjuse tunnistuseks on: veelgi suitsev laastatud maa, valel ajal viljakandvad taimed ja soolasammas, mis seisab uskmatu hinge mälestusmärgina.
\par 8 Sest tarkuse põlgajad tegid iseendile kahju mitte ainult sellepärast, et nad ei mõistnud head, vaid nad jätsid elavaile ka mälestuse oma rumalusest, et nende eksimused ei jääks unustusse.
\par 9 Tarkus päästis aga hädast need, kes teda teenisid.
\par 10 Tema juhatas tasastele teedele õige, kes põgenes venna viha eest, näitas temale Jumala riiki ja andis teadmisi pühadest asjadest, tegi ta rikkaks vaevapõlves ja suurendas tema töö vilja.
\par 11 Kui ahnuse pärast talle liiga tehti, siis aitas teda tarkus ja tegi ta rikkaks.
\par 12 See kaitses teda vaenlaste eest ja tegi julgeks varitsejate vastu. See juhtis teda läbi ägedast võitlusest, et ta mõistaks, et jumalakartus on võimsam kui kõik muu.
\par 13 Tarkus ei hüljanud müüdud õiget, vaid päästis tema patust.
\par 14 Tarkus läks koos temaga vangimajja ega jätnud teda maha, kui ta oli ahelais, kuni ta tõi temale valitsuskepi ja võimu tema rõhujate üle, näitas tema pilkajaid valelikena ja andis temale igavese au. 

\section*{Päästmine Egiptusest}

\par 15 Tarkus päästis pühitsetud rahva ja laitmatu soo rahva käest, kes neid rõhus.
\par 16 Ta läks Issanda sulase hinge, astudes vastu kardetavaile kuningaile imetegude ja ennustusmärkidega.
\par 17 Pühitsetuile andis ta tasu nende vaevade eest, ta juhatas neid imelisel teel, oli neile varjuks päeval ja tähtede säraks öösel.
\par 18 Ta viis nad läbi Punasest merest, juhtides läbi rohke vee.
\par 19 Aga nende vaenlased ta uputas ära, vee kobrutades üles põhjatust sügavusest.
\par 20 Sellepärast rüüstasid õiged jumalakartmatuid ja ülistasid, Issand, sinu püha nime, kiites üksmeelselt sinu kaitsvat kätt.
\par 21 Sest tarkus avas tummade suu ja tegi kõnekaks väetite keele. 

\chapter{11}

\section*{Päästmine Egiptusest}

\par 1 Ta laskis nende teod korda minna oma püha prohveti käe läbi.
\par 2 Nad läksid läbi asustamata kõrbe ja püstitasid telgid teedeta paikades.
\par 3 Nad astusid vastu vaenlastele ja tõrjusid vihamehi.
\par 4 Neil oli janu, ja nad hüüdsid sind appi, neile anti vett järsust kaljust, janu kustutust kõvast kivist.

\section*{Hukatuseks ühele, õnnistuseks teisele}

\par 5 Sest see, millega nende vaenlasi nuheldi, tuli neile kasuks, kui nad olid hädas.
\par 6 Alati voolava jõe allika asemel, mis mõrva verega oli sogaseks tehtud
\par 7 laste tapmise käsu karistuseks, andsid sa neile ootamatult külluses vett,
\par 8 näidates tookordse januga, kuidas sa vastaseid olid nuhelnud.
\par 9 Sest kui neid nõnda proovile pandi, olgugi kergelt karistades, mõistsid nad, kuidas piinati jumalakartmatuid, kellele vihas kohut mõisteti.
\par 10 Sest neid sa panid proovile, manitsedes nagu isa, aga neid teisi sa karistasid, süüdi mõistes nagu vali kuningas.
\par 11 Ja olid nad kaugel või lähedal, neid piinati ühtemoodi.
\par 12 Sest kahekordne mure valdas neid ja ägamine, kui nad meenutasid möödunut.
\par 13 Sest kui nad kuulsid, et nende endi nuhtlused tulid teistele kasuks, siis nad tundsid ära Issanda.
\par 14 Sest teda, kelle nad ükskord laituses olid kõrvale lükanud ja pilgates hüljanud, teda imetlesid nad lõpuks toimunu pärast, kui neil oli teistsugune janu kui õigetel. 

\section*{Jumal halastab egiptlastele}

\par 15 Nende jumalakartmatuse rumalate mõtete pärast, millest eksitatuina nad austasid mõistmatuid roomajaid ja viletsaid putukaid, läkitasid sa palju mõistmatuid loomi neile karistuseks,
\par 16 et nad saaksid aru, et igaüht nuheldakse sellesamaga, millega ta patustab.
\par 17 Sest sinu kõikvõimsal käel, mis lõi maailma vormitust ainest, ei oleks olnud võimatu läkitada neile rohkesti karusid või metsikuid lõvisid,
\par 18 või vastloodud, senitundmata vihaseid metsloomi, kes lõõtsuvad tuldpurskavat hingeõhku, või levitavad suhisevat suitsu, või välgutavad silmist kohutavaid sädemeid,
\par 19 loomi, kes oleksid suutnud neid hävitada mitte ainult murdes, vaid hukata juba oma kohutava välimusega.
\par 20 Ja ilma nendetagi oleksid nad võinud langeda ainsas tuulepuhangus, taga aetuna kättemaksust ja puistatuna sinu võimsuse hingusest. Aga sina oled kõik korraldanud mõõdu, arvu ja kaalu järgi.
\par 21 Sest sinul on alati suur jõud, ja kes suudaks vastu panna sinu vägevale käsivarrele?
\par 22 Sest kogu maailm on sinu ees nagu kübeke kaalukaussidel, otsekui kastetilk, mis hommikul maha langeb. 

\section*{Halastamise põhjus}

\par 23 Sina halastad aga kõigi peale, sest et sa suudad kõike, ja vaatad mööda inimeste pattudest, et nad meelt parandaksid.
\par 24 Sest sina armastad kõike, mis olemas on, ega põlga midagi, mida sa oled teinud, sest sina ei olegi loonud midagi, mida sa ise vihkaksid.
\par 25 Sest kuidas saaks midagi püsida, kui sina ei tahaks, või kuidas säiliks see, keda sina ei ole kutsunud?
\par 26 Sina aga säästad kõiki, sellepärast et nad on sinu omad, Issand, hingede armastaja. 

\chapter{12}

\section*{Halastamise põhjus}

\par 1 Sest sinu surematu Vaim on kõiges.
\par 2 Sellepärast sa karistad langevaid leebelt ja manitsed neid nende patte meenutades, et nad loobuksid kurjusest ja usuksid sinusse, Issand. 

\section*{Jumal säästis Kaanani}

\par 3 Sina vihkasid küll oma püha maa endisi asukaid,
\par 4 sellepärast et nad tegid vastikuid nõiduse tempe ja tõid jäledaid ohvreid
\par 5 halastuseta lastetapjaina ja inimliha sisikonna ja vere maiustajaina salaseltsi pühendatute keskel.
\par 6 Vanemaid, kes tapsid abituid lapsi, tahtsid sa hävitada meie isade käte läbi,
\par 7 et maa, mis sinu silmis oli kõige kallim, saaks väärika elanikkonna Jumala lastest.
\par 8 Aga sa säästsid ka neid kui inimesi, läkitades oma sõjaväe eelsalgana masenduse, et see neid vähehaaval hävitaks.
\par 9 Sina ei olnud võimetu andma neid jumalakartmatuid õigete kätte lahingus, või hirmsate metsloomade või valju sõna läbi ühekorraga hävitama,
\par 10 kuid sa mõistsid kohut pikkamisi, andes võimaluse meeleparanduseks, olgugi et sa teadsid, et nende päritolu on halb ja nende kurjus sünnipärane ja et nende mõtteviis iialgi ei muutu. 

\section*{Säästmise põhjus}

\par 11 Sest see oli algusest peale neetud sugu, ometi sa ei kartnud jätta karistamata nende patte.
\par 12 Sest kes tohiks ütelda: „Mis sa oled teinud?” Ja kes suudaks vastu panna sinu kohtuotsusele? Kes tohiks sind süüdistada rahvaste hävitamises, keda sa ise oled teinud? Või kes tahaks sinu vastu üles astuda, kätte maksma jumalakartmatute inimeste eest?
\par 13 Sest ei ole muud Jumalat kui sina, kes hoolitseb kõige eest, et peaksid veel tõestama, et sa ei ole ülekohtuselt kohut mõistnud.
\par 14 Mitte ükski kuningas või valitseja ei suudaks sulle vastu astuda nende pärast, keda sina oled karistanud.
\par 15 Kuna sa oled õige, siis sa korraldad kõike õigesti ega pea oma võimu vääriliseks hukka mõista seda, kes karistust ei ole teeninud.
\par 16 Sest sinu vägevus on sinu õiguse alguseks, ja see, et sa valitsed kõigi üle, lubab sind hoida kõiki.
\par 17 Sest sina näitad jõudu, kui ei usuta sinu võimu täiuslikkust, ja karistad neid, kes seda ei tunnusta.
\par 18 Sina aga, kes jõudu käsutad, mõistad kohut mõõdukalt ja valitsed meid suure hoidmisega, sest kui sa tahad, siis sa võid. 

\section*{Jumala halastuse õpetlikkus}

\par 19 Niisuguste tegudega oled sa õpetanud oma rahvast, et õige peab olema inimestearmastaja, ja oma lastele oled sa jätnud hea lootuse, et sa pärast patte annad meeleparanduse.
\par 20 Sest kui sa neid, kes olid su laste vaenlased ja surma väärt, karistasid niisuguse tähelepanu ja kannatlikkusega, et andsid neile aega ja mahti kurjusest loobumiseks,
\par 21 siis missuguse hoolitsusega sa küll mõistad kohut oma laste üle, kelle vanemaile sa andsid vande ja lepinguga nii head tõotused.
\par 22 Kui sa meid kasvatasid, siis sa piitsutasid meie vaenlasi kümme tuhat korda rohkem, et kohut mõistes mõtleksime sinu headusele, ja kui meile kohut mõistetakse, ootaksime halastust. 

\section*{Jumala leebusele järgneb karm karistus}

\par 23 Sellepärast sa oled ka neid jumalakartmatuid, kes elasid seda meeletut elu, piinanud nende endi hirmsate tegudega.
\par 24 Sest nad olid eksiteedel eksinud üha kaugemale, võttes jumalaiks põlatud, jälestusväärseid loomi, lastes endid petta kui rumalad lapsed.
\par 25 Sellepärast sa läkitasid neile kui mõistmatuile lastele karistuse, tehes nad pilkealuseks.
\par 26 Aga kes pilkamise karistusest õpetust ei võta, saavad Jumalalt kohase nuhtluse.
\par 27 Sest just nende läbi, kelle vastu nad kannatustes nurisesid, nende läbi, keda nad jumalaiks pidasid, nuheldi neid, ja nad nägid ning mõistsid, et see, keda nad enne olid salanud, on tõeline Jumal. Seepärast tabaski neid suurim karistus. 

\chapter{13} 

\section*{Taevatähtede ja loodusjõudude kummardamisest}

\par 1 Sest kõik inimesed olid ju loomu poolest tühised, neil puudus jumalatunnetus, ja nähtavaist heategudest ei osanud nad ära tunda teda, kes ON. Tegude märkajadki ei tundnud tegijat,
\par 2 vaid pidasid maailma valitsevaiks jumalaiks tuld või tuult või liikuvat õhku või tähtede sõõri või võimast vett või taevavalgusi.
\par 3 Kui nad, olles võlutud nende ilust, pidasid neid jumalaiks, siis oleksid nad pidanud ka teadma, kuivõrd parem on nende Valitseja, sest tema, kes on ilu algkuju, on need loonud.
\par 4 Kui neid hämmastas nende jõud ja mõju, siis oleksid nad pidanud mõtlema, kuivõrd vägevam on tema, kes need on valmistanud.
\par 5 Sest loodu suurusest ja ilust võib võrdpildina näha, missugune on Looja.
\par 6 Sellegipärast on vähe neile ette heita, sest kergesti eksivad ju ka need, kes otsivad Jumalat ja tahavad teda leida.
\par 7 Sest elades tema tegude keskel, nad uurivad neid, aga välimus eksitab, kuna nähtav on ilus.
\par 8 Teistpidi aga ei saa neid vabandada,
\par 9 sest kui nad nii palju suutsid tunnetada, et olid võimelised uurima maailma, miks nad siis rutemini ei leidnud nende Valitsejat? 

\section*{Ebajumalakujude kummardamisest}

\par 10 Aga armetud olid ja elutute peale lootsid need, kes inimeste kätetöid nimetasid jumalaiks: kuld ja hõbe, kunstipärane toode, loomakujud või kasutu kivi - muistse käe töö.
\par 11 Ja puusepa töö? Tema vestab tarbepuud: koorib siledalt kõik selle koore ja ilusasti töödeldes valmistab eluks vajaliku eseme.
\par 12 Tööst jäänud laaste tarvitab ta toidu tegemisel, et oma kõhtu täis süüa.
\par 13 Aga selle jäänuse, mis kuhugi ei kõlba, kõvera ja okslikuks kasvanud puutüki, ta võtab ja vestab seda oma jõudehetkil, ning õpitud oskusega kujundades valmistab sellest inimese kuju,
\par 14 või teeb selle lihtsa loomaga sarnaseks, võõpab värvimullaga ja värvib punaseks, maalides üle kõik selle laigud.
\par 15 Siis ta valmistab sellele sobiva hoiupaiga, mille ta raua abil seina külge kinnitab.
\par 16 Ta hoolitseb, et kuju ei kukuks maha, sest ta teab, et see ei saa iseennast aidata, kuna ta on üksnes kuju ja vajab abi.
\par 17 Aga palvetades oma varanduse ning naise ja laste pärast, ei häbene ta kõnetada elutut. Jah, tervise pärast ta hüüab appi jõuetut.
\par 18 Elu pärast palub surnut, abi pärast anub võimetut, teekonna pärast küsib selle käest, kes ise ei saa käia,
\par 19 kasu, töövilja ja kätetöö pärast nõuab ta jõudu sellelt, kelle käed on nõdrad.

\chapter{14}

\section*{Ebajumalakujude kummardamisest}

\par 1 Või jälle kui keegi meresõidu ette võtab ja tahab läbida mässavaid laineid, siis ta hüüab appi puud, mis on pehkinum kui teda kandev laev.
\par 2 Sest selle on kasuiha leiutanud ja meistri tarkus ehitanud.
\par 3 Aga sinu ettehoole, oh Isa, on see, mis selle läbi tüürib, sest sina oled ka mere peale teinud tee ja lainetesse kindla raja,
\par 4 tõestamaks, et sina võid päästa kõigest, koguni siis kui keegi ka ilma oskuseta astub laeva.
\par 5 Sina aga tahad, et sinu tarkuse teod ei jääks viljatuks, sellepärast usaldavad inimesed oma elu ka kõige pisema puu hoolde, ja lainemurdu läbides nad päästetakse parvel.
\par 6 Sest ka muiste, kui ülbed hiiglased hukkusid, põgenes maailma lootus viletsasse laeva ja säilitas uue sugupõlve seemne, kuna ju sinu käsi seda laeva tüüris.
\par 7 Sest õnnistatud on puu, mille läbi sünnib õigus.
\par 8 Aga kätega tehtu on neetud, niihästi see kui selle tegija; üks sellepärast, et ta selle on teinud, ja teine, et teda nimetatakse jumalaks, olgugi et ta on kaduv.
\par 9 Sest Jumal vihkab võrdselt jumalakartmatut ja tema jumalakartmatut tegu.
\par 10 Karistatakse ju tegu koos tegijaga.
\par 11 Sellepärast tabab karistus ka paganate ebajumalaid, sest need on saanud jäleduseks Jumala loomingu seas, pahanduseks inimhingedele ja püügipaelaks rumalate jalgadele. 

\section*{Ebajumalakujude kummardamise algusest}

\par 12 Sest ebajumalate väljamõtlemine on hooruse alguseks ja nende leiutamine on elule hukatuseks.
\par 13 Ei ole need ju olnud algusest peale ega jää need ka igavesti.
\par 14 Sest need on maailma tulnud inimeste tühise kujutluse tõttu ja seepärast on neile määratud lühike lõpp.
\par 15 Üks liiga varasest leinast piinatud isa valmistas oma äkitselt ära võetud lapsest kuju ja austas nüüd surnud inimest otsekui jumalat, seades oma perele salapärased teenistused ja kombed.
\par 16 Pärast sai see jumalavallatu harjumus aegamööda kinnituse ja seda peeti seaduseks.
\par 17 Nikerdatud kujusid teeniti ka valitsejate käsul. Siis kui inimesed kaugel elades ei saanud neid austada palge ees, jäljendasid nad eemalolija välimust ja tegid austatava kuninga nähtava kuju, et agaralt meelitada kaugelviibijat, just nagu oleks ta juuresolija.
\par 18 Kunstniku kuulsushimu õhutas üha innukamalt teenima ka neid, kes kujutatut ei tundnudki.
\par 19 Sest see, võib-olla tahtes meeldida valitsejale ja pingutades oma oskust, püüdis teha sarnasust veelgi lummavamaks.
\par 20 Aga rahvahulk, kuju kaunidusest võlutud, hakkas pidama pühaduseks nüüd seda, keda äsja olid austanud kui inimest.
\par 21 Ja sellest tuli elule hukatus, et õnnetuse või vägivalla orjuses olevad inimesed andsid kividele ja puudele neile sobimatu nime. 

\section*{Ebajumalakuju kummardamise tagajärg}

\par 22 Aga sellest ei olnud neile veel küllalt, et nad eksisid Jumala tundmises, vaid, ehkki elades suures rumaluse sõjas, nad nimetasid niisugust kurja tegu rahuks.
\par 23 Sest kas lapsetapu ohvritega või varjatud salategudega või pööraste, võõrastavate prassimistega pidutsedes,
\par 24 nad ei hoia puhtana elu ega abielu: üks hävitab teise salakavalalt või valmistab talle abielurikkumisega valu.
\par 25 Kõike üheskoos valitseb aga verevalamine ja mõrv, vargus ja pettus, hukatus, uskmatus, mäss, valevandumine,
\par 26 headuse mahakarjumine, tänu unustamine, hingede rüvetamine, loomuvastane suguelu, abielude korratus, hooratöö ja ohjeldamatus.
\par 27 Sest nimeta ebajumalate teenistus on kõige kurja algus ja põhjus ja lõpp.
\par 28 Sest rõõmutsetakse ju arutult või ennustatakse petlikult või elatakse jumalakartmatult või vannutakse kergesti valet.
\par 29 Kuna nad loodavad elutute ebajumalate peale, ei karda nad karistust valevande pärast.
\par 30 Aga kummagi pärast ootab neid kohus: et nad mõtlesid halvasti Jumalast, kuulates ebajumalaid, ja et nad kavalasti valet vandusid, põlates pühadust.
\par 31 Sest mitte nende võim, kelle nimel nad vannuvad, vaid patustele määratud karistus tabab alati jumalavallatute üleastumist. 

\chapter{15} 

\section*{Iisrael ei teeni ebajumalaid}

\par 1 Aga sina, meie Jumal, oled hea, aus ja pikameelne ning valitsed kõike halastusega.
\par 2 Sest kui me ka pattu teeme, oleme ometi sinu omad, tundes sinu väge. Meie aga ei taha pattu teha, teades, et meid arvatakse sinu omaks.
\par 3 Sest arusaamine sinust on täielik õigeksmõistmine, ja sinu väe tundmine on surematuse juur.
\par 4 Sest meid ei ole eksitanud inimeste kurikaval leiutis ega maalija viljatu vaev: kirevate värvidega määritud kuju,
\par 5 mille nägemine äratab rumala himu, nõnda et hakatakse ihaldama surnud kuju hingetut olemust.
\par 6 Kurja armastajad ja tühise lootuse väärilised on niihästi nende valmistajad kui ka ihaldajad ja austajad. 

\section*{Rumalad ebajumalakuju valmistajad}

\par 7 Sest ka potissepp sõtkub suure vaevaga pehmet savi ja vormib iga riista meie tarbeks. Ent samast savist vormib ta niihästi astjad, mis määratud puhtaiks toimetusiks, kui ka vastupidiseiks - kõik selsamal viisil. Aga milleks igaüht neist tarvis on, otsustab potissepp.
\par 8 Ja kurjasti kasutatud vaevaga vormib ta samast savist võimetu jumala, tema, kes ise äsja on maamullast sündinud ja peagi läheb jälle sinna, millest ta on võetud, kui laenuks antud hing tagasi nõutakse.
\par 9 Temale ei teegi muret see, et ta peab hääbuma, ega see, et temal on üürike elu, vaid ta võistleb hõbe- ja kullasseppadega, matkib vasevalajaid ja loeb enesele auks, et ta pettekujusid valmistab.
\par 10 Tema süda on tuhk ja tema lootus on mullast tühisem, tema elu on vähem väärt kui savi,
\par 11 sellepärast et ta ei tunne seda, kes tema on loonud, kes on temasse puhunud teovõimsa hinge ja sisendanud elava vaimu,
\par 12 vaid ta arvab, et meie elu on mäng, ja meie olemine kasutoov turupäev. „Sest,” nii ta ütleb, „on ometi vaja kasu saada, kas või kurjast.”
\par 13 Seesugune teab ju ise paremini kui kõik muud, et ta pattu teeb, valmistades muldsest ainest hapraid riistu ja nikerdatud kujusid. 

\section*{Iisraeli rõhujate rumalus: ebajumalateenistus}

\par 14 Aga nad kõik on rumalamad ja haletsusväärsemad kui lapse hing, need sinu rahva vaenlased, kes teda on rõhunud.
\par 15 Sest nad on pidanud jumalaiks kõiki paganlikke ebajumalakujusid, millel ei ole silmi nägemiseks, ei ninasõõrmeid õhu hingamiseks, ei kõrvu kuulmiseks, ei kätel sõrmi kompamiseks; ja nende jalad ei kõlba kõndimiseks.
\par 16 Sest need on inimese tehtud, vaimu laenukssaanu on need kujundanud. Aga ükski inimene ei suuda ometi valmistada Jumalat, kui see ka oleks ainult tema enesega sarnane.
\par 17 Kui surelik saab ta ju oma ülekohtuste kätega valmistada ainult surnut. Sest tema ise on ometi parem kui need, keda ta austab: tema ise ju elab, need aga mitte.
\par 18 Nad koguni austavad vastikuid loomi, kes teistega võrreldes on oma rumaluse tõttu veel armetumad.
\par 19 Kes ei ole ka nii ilusad, et nad meeldiksid, nagu muidu loomi nähes sünnib. Ei, need on jäänud ilma Jumala heakskiidust ja õnnistusest. 

\chapter{16} 

\section*{Egiptus ja Iisrael: karistus ja kasvatus}

\par 1 Sellepärast oli õige, et neid seesugustega nuheldi, vastikute putukate rohkusega piinates.
\par 2 Niisuguse nuhtluse asemel tegid sa oma rahvale head: suure isu täitmiseks valmistasid haruldase roa, andes neile toiduks vutte -
\par 3 olid ju egiptlased süüa tahtes kaotanud vajaliku isu neile kallale saadetud loomade vastiku välimuse tõttu -, kuna aga need, kes pisut aega puudust tundsid, said osa hoopis uuest roast.
\par 4 Sest neile, rõhujaile, pidi tulema paratamatu puudus, sinu rahvale aga pidi näidatama, kuidas nende vaenlasi piinati.
\par 5 Sest ka siis, kui elukate hirmus viha tuli sinu rahva peale ja neid hävitati vingerdavate madude salvamise läbi, ei kestnud sinu viha lõpuni,
\par 6 vaid neid hirmutati pisut aega hoiatuseks, ja nad said päästemärgi meenutama sinu Seaduse käsku.
\par 7 Sest kes pöördus, seda ei päästetud nähtu pärast, vaid sinu poolt, kõigi Päästja.
\par 8 Ja sellega panid sa ka meie vaenlased uskuma, et sina oled lunastaja kõigest kurjast.
\par 9 Neid tappis ju rohutirtsude ja parmude hammustus, ja nende hingele ei leidunud ravimit, sest nad olid seesugust nuhtlust väärt.
\par 10 Sinu lapsi aga ei kahjustanud mürkmadude hambad, sest sinu halastus astus vastu ja tegi nad terveks.
\par 11 Neid salvati sinu sõnade meenutuseks, ja nad päästeti ruttu, et nad ei langeks sügavasse unustusse ega jääks ilma sinu heategudest.
\par 12 Neid ei parandanud ravimtaimed ega plaastrid, vaid, oh Issand, sinu sõna, mis kõike parandab.
\par 13 Sest sinul on meelevald elu ja surma üle, sina viid alla surmavalla väravaisse ja tood jälle üles.
\par 14 Inimene ju tapab küll oma kurjuse tõttu, aga välja läinud vaimu ta tagasi ei too ja vangistatud hinge ta ei vabasta.
\par 15 Sinu käest põgeneda on aga võimatu.
\par 16 Sest jumalakartmatuid, kes ei tahtnud sind tunda, piitsutati sinu käsivarre rammuga, haruldased vihmavalingud ja rahehood ning vältimatud rajuilmad jälitasid neid, ja tuli tegi neile lõpu.
\par 17 Mis veelgi imelisem: kõike kustutavas vees oli tulel suurem võimus, sest loodus sõdib õigete eest.
\par 18 Teinekord tuleleek aga vaibus, et see ei põletaks loomi, kes olid läkitatud jumalakartmatute kallale, et need seda nähes mõistaksid, et neid piinab Jumala kohus.
\par 19 Teinekord põles see keset vett tavalisest tulest võimsamana, et hävitada ülekohtuse maa viljad.
\par 20 Selle asemel söötsid sa oma rahvast inglite roaga, ilma et nad oleksid vaeva näinud, andsid neile taevast valmis leiba, mis iga isu rahuldas ja igale maitsele vastas.
\par 21 Sest sinu olemus ilmutas sinu lastele sinu armsust, ja vastu tulles vastuvõtjate soovile muutus selleks, mida igaüks himustas.
\par 22 Aga lumi ja jää panid vastu tulele ega sulanud, mõistmiseks, et tuli, mis leegitses rahehoogudes ja vihmavalinguis välku lõi, hävitas ainult vaenlase vilja,
\par 23 ja teinekord - et õiged saaksid süüa - unustas oma loomupärase jõu.
\par 24 Sest loodus, mis teenib sind, Looja, tõuseb ülekohtuste karistamiseks, ja leebub, et head teha neile, kes sinu peale loodavad.
\par 25 Sellepärast loodus ka tookord igati muutudes teenis sinu kõiketoitvat andi palujate soovi kohaselt,
\par 26 et sinu lapsed, keda sina, Issand, armastad, õpiksid, et inimesi ei toida maaviljad, vaid et sinu sõna peab ülal neid, kes sinusse usuvad.
\par 27 Sest see, mida tuli ei hävitanud, sulas otsekohe, kui seda soojendas põgus päikesekiir,
\par 28 mõistmiseks, et sind tuleb tänada enne päikesetõusu ja sinu poole palvetada aovalgel.
\par 29 Sest tänamatu inimese lootus sulab otsekui talvine härmatis ja voolab ära kui kasutu vesi. 

\chapter{17} 

\section*{Egiptus ja Iisrael: pimedus ja valgus}

\par 1 Sest suured ja uurimatud on sinu kohtuotsused, seepärast eksisid õpetamata hinged.
\par 2 Sest nurjatud, arvates võivat rõhuda püha rahvast, olid ise pimeduse vangid ja pika öö ahelais, suletuina katuste alla, paos vääramatu ettemääramise eest.
\par 3 Sest kui nad oma salapattudega lootsid jääda peidetuks unustuse pimestava katte alla, siis pillutati neid hirmsas ehmatuses ja kohutati viirastustega.
\par 4 Sest ka mitte see nurgake, mis neid varjas, ei hoidnud neid kartusest vabana, ümbritses neid ju kohutav kära, neile ilmusid koletislikud, jubedalõustalised kummitused.
\par 5 Ei suutnud anda neile valgust mitte ühegi tule vägi, seda sünget ööd ei teinud heledamaks taevatähtede selge säragi.
\par 6 Neile paistis üksnes isesüttinud hirmuäratav tuli. Ja kohutatud sellest ennenägematust viirastusest, pidasid nad nähtut veelgi õudsemaks.
\par 7 Võlurite salakunstide oskus oli aga alla jäänud, ja hooplejate tarkus jäi haledasti häbisse.
\par 8 Sest need, kes tõotasid ära saata haigest hingest kartuse ja rahutuse, haigestusid ise naeruväärsesse argusesse.
\par 9 Ka siis, kui midagi hirmuäratavat ei olnudki neid peletamas, kohutas neid metsloomade jooks ja roomajate sisin, ja ehmatusest olid nad otse hukkumas, kartes vaadata tühja õhkugi, millest kuhugi ei pääse.
\par 10 Sest kurjus on loomu poolest arg, olles hukka mõistetud omaenese tunnistuse põhjal, ja kitsikuses olev südametunnistus teeb ju halva veelgi halvemaks.
\par 11 Sest kartus ei ole midagi muud kui loobumine arukast abist.
\par 12 Ja mida väiksem on seesmine lootus abile, seda suurem tundub teadmatus kannatuse põhjusest.
\par 13 Neid aga, kes selsamal võimuta ööl, mis oli tõusnud võimuta surmavalla sügavusest, tahtsid magada nagu ikka,
\par 14 neid kas vintsutati viirastuslike nägemustega või halvati hinge ahastusega: sest äkitselt ja ootamatult valdas neid hirm.
\par 15 Siis igaüks seal, kus ta iganes maha langes, oli otsekui vahi all ja suletud ahelateta vangikotta.
\par 16 Oli see põllumees või karjane, või keegi, kes kõrbes rasket tööd tegi, - ootamatult tabas teda paratamatu saatus, sest pimeduse ahelatega aheldati igaüks.
\par 17 Tuule tuhin või rõkkav linnulaul tihedais okstes, või võimsasti voolavate vete kohin, või langevate kaljupankade vali mürin,
\par 18 hüplevate loomade nähtamatu jooks, möirgavate lõvide kohutav hääl, või kaja, mis vastu kostab mägede koopaist, - see kõik hirmutas ja halvas neid.
\par 19 Sest kogu maailm oli heleda valgusega valgustatud ja tegi takistamatult oma töid,
\par 20 ainult neile oli tulnud raske öö, pilt sellest pimedusest, mis ükskord pidi neid haarama. Aga nad olid iseendile veel suuremaks koormaks kui see pimedus. 

\chapter{18} 

\section*{Egiptus ja Iisrael: pimedus ja valgus}

\par 1 Aga heledaim valgus oli sinu pühal rahval. Kui teised nende häält küll kuulsid, nende nägusid aga ei näinud, siis nad kiitsid neid õnnelikuks, et nemadki sedasama ei pidanud kannatama.
\par 2 Ja nad tänasid neid sellepärast, et nad neile kätte ei maksnud, hoolimata varem kannatatud ülekohtust, ja palusid andeks, et oli vaenus oldud.
\par 3 Selle asemel andsid sa neile leegitseva tulesamba teejuhiks tundmatul teekonnal, otsekui päikese, mis ei põletanud auküllasel rännakul võõrsil.
\par 4 Sest need teised olid väärt, et neilt võeti valgus ja peeti pimeduse vangis, sellepärast et nad olid kinni pidanud sinu lapsi, kelle läbi maailmale pidi antama Seaduse kustumatu valgus. 

\section*{Egiptus ja Iisrael: hukkaja tegutseb}

\par 5 Ja kui nad olid kavatsenud tappa püha rahva lapsukesi, kellest üks oli kõrvale asetatud ja päästeti, siis karistuseks võtsid sa neilt palju nende oma lapsi ja hukkasid nad kõik üheskoos võimsas vees.
\par 6 Seda ööd kuulutati ette meie vanemaile, et nad kindlasti teaksid, missugused on need vanded, mida nad uskusid, et olla julged.
\par 7 Nõnda ootas sinu rahvas õigete päästmist, aga vaenlaste hävitamist.
\par 8 Sest millega sa meie vastaseid nuhtlesid, sellesamaga sa kutsusid ja austasid meid.
\par 9 Sest vagade pühad lapsed ohverdasid salajas ja kohustusid üksmeelselt täitma jumalikku seadust, et nad samavõrd osa saaksid niihästi hüvedest kui hädaohtudest, kusjuures nad juba siis laulsid vanemate pühi kiituslaule.
\par 10 Vastu kajas vaenlaste metsik kisa ja kaikus laste pärast nutjate kaebehääl.
\par 11 Ühesuguse karistusega nuheldi sulast ja isandat, ja lihtrahvas kannatas sedasama mis kuningas.
\par 12 Samavõrd oli kõigil loendamatuid surnuid, kes olid samasugust surma surnud. Elavaid aga ei jätkunud nende matmiseks, sest ainsa silmapilguga olid hävitatud nende kõige kallimad võsud.
\par 13 Need, kes oma salakunstide pärast midagi ei olnud uskunud, tunnistasid nüüd, oma esmasündinute hukkudes, et see rahvas on Jumala poeg.
\par 14 Kui sügav vaikus haaras kõike ja öö kiires kulgemises oli jõudnud kesköö,
\par 15 siis sööstis sinu kõikvõimas sõna kuninglikult aujärjelt taevast otsekui karm sõjamees hukkumisele määratud maa keskele, tuues terava mõõgana sinu selge käsu.
\par 16 Sinna astudes täitis ta kõik surmaga, ja taevast puudutades käis ometi maa peal.
\par 17 Siis kohutasid neid äkitselt unenägude õudsed viirastused, ja ootamatu hirm valdas neid.
\par 18 Poolsurnuna langes üks siin, teine seal, ise teada andes oma surma põhjuse.
\par 19 Sest nägemused, mis neid hirmutasid, olid ette kuulutatud, et nad ei hukkuks teadmata, miks nad pidid seda õnnetust kannatama.
\par 20 Aga surmahaare puudutas ka õigeid, ja paljud hukkusid kõrbes. Ent viha ei kestnud kaua.
\par 21 Sest üks laitmatu mees tõttas võitlema rahva eest: ta kandis oma ametirelva - palvet tehes ja lepitavat suitsutusohvrit tuues astus ta vihale vastu ja pani õnnetusele piiri, osutades, et ta on sinu teener.
\par 22 Tema ei võitnud seda mässu ihurammu ega relva jõuga, vaid ta sundis nuhtlejat sõna läbi, meenutades vanemaile antud vandeid ja lepinguid.
\par 23 Sest kui surnud juba olid langenud hunnikuina üksteise peale, astus ta vahele ja keelas viha ning lõikas sellelt tee elavate juurde.
\par 24 Sest tema pika kuue peal oli kujutatud kogu maailm. Vanemate austus oli raiutud neljas reas olevaisse kividesse, ja sinu auhiilgus laubaehtesse, mis oli temal peas.
\par 25 Nende ees taganes hukkaja, neid ta kartis. Sest üksnes vihaproovist oli küllalt. 

\chapter{19} 

\section*{Egiptus ja Iisrael: Punane meri}

\par 1 Aga jumalakartmatute peale jäi halastamatu viha kuni lõpuni, sest Jumal teadis juba nende kavatsusi.
\par 2 Nimelt, lubades küll Iisraeli laste lahkumist ja kiirustades nende äraminekut, kahetsevad nad ja ajavad neid taga.
\par 3 Sest kui nad surnuhaudade juures alles leinasid ja kaeblesid, tegid nad juba teise meeletu otsuse ja jälitasid kui põgenikke neid, keda nad härdasti paludes olid ära ajanud.
\par 4 Sest teenitud saatus viis nad lõpuni ja laskis neil unustada sündinu, et nad täielikult kätte saaksid nende piinades veel puuduva karistuse,
\par 5 kusjuures sinu rahvas sooritaks imelise teekonna, teised leiaksid aga ebatavalise surma.
\par 6 Sest kogu looduse olemus loodi uuesti, teenima erakordseid käske, et sinu lapsi hoida kahju eest.
\par 7 Oli näha leeri varjavat pilve ja kuiva maad kerkivat seal, kus enne oli vesi: vaba tee läbi Punase mere ja haljendav väli võimsa lainetuse asemel.
\par 8 Seda mööda läks läbi kogu rahvas - need, keda kaitses sinu käsi, nähes haruldasi imetegusid.
\par 9 Sest neid viidi nagu hobuseid karjamaale ja nad lõid kepsu otsekui talled, kiites sind, Issand, kes sa olid nad päästnud.
\par 10 Veel mäletasid nad oma võõrsilolekut, kuidas maa tõi välja sääski, selle asemel et sünnitada loomi, ja kuidas jõgi paiskas üles kalade asemel palju konni.
\par 11 Aga hiljem nägid nad ka uut linnusugu, kui nad himust aetuna palusid maiusrooga.
\par 12 Sest meeleheaks neile tõusid merest vutid. 

\section*{Egiptlased ja soodomlased}

\par 13 Ja süüdlastele tulid nuhtlused, mille ennetena käisid vägevad välgud. Sest nemad kannatasid tõepoolest omaenese kuritegude pärast. Olid nad ju näidanud suurt viha võõraste vastu.
\par 14 Sest teised ei võtnud vastu tundmatuid tulijaid, nemad tegid aga orjaks võõrad, kes olid neile head teinud.
\par 15 Aga mitte ainult see, vaid karistus tabas neid veel seetõttu: kui teised võõraid vastu võtsid vaenulikult,
\par 16 siis nemad võtsid pidulikult. Kui aga võõrad olid saanud täieõiguslikeks, siis vaevati neid raskete töödega.
\par 17 Aga neid pimestati, nõnda nagu neid teisi õige mehe ukse ees, siis kui igaüks neist otsis tema ust sissepääsuks ümbritsevas pilkases pimeduses.
\par 18 Sest otsekui kandlehelid oma laadilt vahelduvad, kooskõla aga alati jääb, nõnda muutub ka algainete omavaheline kord, mis neist sündinud asjust selgesti näha on.
\par 19 Sest maaloomad muudeti veeloomadeks ja ujujad tulid maale.
\par 20 Tuli läks vees vägevamaks, kui ta oli, ja vesi unustas oma kustutamisvõime.
\par 21 Leegid omakorda ei neelanud kergesti kaduvat loomaliha, mis neisse oli jäänud, ega sulanud hõlpsasti sulav, jääga sarnane taevane roog.
\par 22 Sest sina, Issand, oled teinud oma rahva kõikepidi suureks ja auväärseks, ega ole teda põlanud, vaid oled igal ajal ja igas paigas seisnud tema kõrval. 



\end{document}