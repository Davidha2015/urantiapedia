\begin{document}

\title{Genesis}

 
\chapter{1}

\par 1 در ابتدا، خدا آسمانها و زمین را آفرید.
\par 2 وزمین تهی و بایر بود و تاریکی بر روی لجه. و روح خدا سطح آبها را فرو گرفت.
\par 3 و خدا گفت: «روشنایی بشود.» و روشنایی شد.
\par 4 و خدا روشنایی را دید که نیکوست و خداروشنایی را از تاریکی جدا ساخت.
\par 5 و خداروشنایی را روز نامید و تاریکی را شب نامید. وشام بود و صبح بود، روزی اول.
\par 6 و خدا گفت: «فلکی باشد در میان آبها و آبهارا از آبها جدا کند.»
\par 7 و خدا فلک را بساخت وآبهای زیر فلک را از آبهای بالای فلک جدا کرد. و چنین شد.
\par 8 و خدا فلک را آسمان نامید. و شام بود و صبح بود، روزی دوم.
\par 9 و خدا گفت: «آبهای زیر آسمان در یکجاجمع شود و خشکی ظاهر گردد.» و چنین شد.
\par 10 و خدا خشکی را زمین نامید و اجتماع آبها رادریا نامید. و خدا دید که نیکوست.
\par 11 و خداگفت: «زمین نباتات برویاند، علفی که تخم بیاوردو درخت میوه‌ای که موافق جنس خود میوه آوردکه تخمش در آن باشد، بر روی زمین.» و چنین شد.
\par 12 و زمین نباتات را رویانید، علفی که موافق جنس خود تخم آورد و درخت میوه داری که تخمش در آن، موافق جنس خود باشد. و خدادید که نیکوست.
\par 13 و شام بود و صبح بود، روزی سوم.
\par 14 و خدا گفت: «نیرها در فلک آسمان باشند تاروز را از شب جدا کنند و برای آیات و زمانها وروزها و سالها باشند.
\par 15 و نیرها در فلک آسمان باشند تا بر زمین روشنایی دهند.» و چنین شد.
\par 16 و خدا دو نیر بزرگ ساخت، نیر اعظم را برای سلطنت روز و نیر اصغر را برای سلطنت شب، وستارگان را.
\par 17 و خدا آنها را در فلک آسمان گذاشت تا بر زمین روشنایی دهند،
\par 18 و تاسلطنت نمایند بر روز و بر شب، و روشنایی را ازتاریکی جدا کنند. و خدا دید که نیکوست.
\par 19 وشام بود و صبح بود، روزی چهارم.
\par 20 و خدا گفت: «آبها به انبوه جانوران پر شودو پرندگان بالای زمین بر روی فلک آسمان پروازکنند.»
\par 21 پس خدا نهنگان بزرگ آفرید و همه جانداران خزنده را، که آبها از آنها موافق اجناس آنها پر شد، و همه پرندگان بالدار را به اجناس آنها. و خدا دید که نیکوست.
\par 22 و خدا آنها رابرکت داده، گفت: «بارور و کثیر شوید و آبهای دریا را پر سازید، و پرندگان در زمین کثیر بشوند.»
\par 23 و شام بود و صبح بود، روزی پنجم.
\par 24 و خدا گفت: «زمین، جانوران را موافق اجناس آنها بیرون آورد، بهایم و حشرات وحیوانات زمین به اجناس آنها.» و چنین شد.
\par 25 پس خدا حیوانات زمین را به اجناس آنهابساخت و بهایم را به اجناس آنها و همه حشرات زمین را به اجناس آنها. و خدا دید که نیکوست.
\par 26 و خدا گفت: «آدم را بصورت ما و موافق شبیه ما بسازیم تا بر ماهیان دریا و پرندگان آسمان وبهایم و بر تمامی زمین و همه حشراتی که بر زمین می‌خزند، حکومت نماید.»
\par 27 پس خدا آدم را بصورت خود آفرید. او رابصورت خدا آفرید. ایشان را نر و ماده آفرید.
\par 28 و خدا ایشان را برکت داد و خدا بدیشان گفت: «بارور و کثیر شوید و زمین را پر سازید و در آن تسلط نمایید، و بر ماهیان دریا و پرندگان آسمان وهمه حیواناتی که بر زمین می‌خزند، حکومت کنید.»
\par 29 و خدا گفت: «همانا همه علف های تخم داری که بر روی تمام زمین است و همه درختهایی که در آنها میوه درخت تخم دار است، به شما دادم تا برای شما خوراک باشد.
\par 30 و به همه حیوانات زمین و به همه پرندگان آسمان وبه همه حشرات زمین که در آنها حیات‌است، هر علف سبز را برای خوراک دادم.» و چنین شد.و خدا هر‌چه ساخته بود، دید و همانابسیار نیکو بود. و شام بود و صبح بود، روزششم.
\par 31 و خدا هر‌چه ساخته بود، دید و همانابسیار نیکو بود. و شام بود و صبح بود، روزششم.
 
\chapter{2}

\par 1 و آسمانها و زمین و همه لشکر آنها تمام شد.
\par 2 و در روز هفتم، خدا از همه کار خود که ساخته بود، فارغ شد. و در روز هفتم از همه کارخود که ساخته بود، آرامی گرفت.
\par 3 پس خدا روزهفتم را مبارک خواند و آن را تقدیس نمود، زیراکه در آن آرام گرفت، از همه کار خود که خداآفرید و ساخت.
\par 4 این است پیدایش آسمانها و زمین در حین آفرینش آنها در روزی که یهوه، خدا، زمین وآسمانها را بساخت.
\par 5 و هیچ نهال صحرا هنوز درزمین نبود و هیچ علف صحرا هنوز نروییده بود، زیرا خداوند خدا باران بر زمین نبارانیده بود وآدمی نبود که کار زمین را بکند.
\par 6 و مه از زمین برآمده، تمام روی زمین را سیراب می‌کرد.
\par 7 خداوند خدا پس آدم را از خاک زمین بسرشت و در بینی وی روح حیات دمید، و آدم نفس زنده شد.
\par 8 و خداوند خدا باغی در عدن بطرف مشرق غرس نمود و آن آدم را که سرشته بود، در آنجاگذاشت.
\par 9 و خداوند خدا هر درخت خوشنما وخوش خوراک را از زمین رویانید، و درخت حیات را در وسط باغ و درخت معرفت نیک و بدرا.
\par 10 و نهری از عدن بیرون آمد تا باغ را سیراب کند، و از آنجا منقسم گشته، چهار شعبه شد.
\par 11 نام اول فیشون است که تمام زمین حویله را که در آنجا طلاست، احاطه می‌کند.
\par 12 و طلای آن زمین نیکوست و در آنجا مروارید و سنگ جزع است.
\par 13 و نام نهر دوم جیحون که تمام زمین کوش را احاطه می‌کند.
\par 14 و نام نهر سوم حدقل که بطرف شرقی آشور جاری است. و نهر‌چهارم فرات.
\par 15 پس خداوند خدا آدم را گرفت و او را درباغ عدن گذاشت تا کار آن را بکند و آن رامحافظت نماید.
\par 16 و خداوند خدا آدم را امرفرموده، گفت: «از همه درختان باغ بی‌ممانعت بخور،
\par 17 اما از درخت معرفت نیک و بد زنهارنخوری، زیرا روزی که از آن خوردی، هرآینه خواهی مرد.»
\par 18 و خداوند خدا گفت: «خوب نیست که آدم تنها باشد. پس برایش معاونی موافق وی بسازم.»
\par 19 و خداوند خدا هر حیوان صحرا و هر پرنده آسمان را از زمین سرشت و نزدآدم آورد تا ببیند که چه نام خواهد نهاد و آنچه آدم هر ذی حیات را خواند، همان نام او شد.
\par 20 پس آدم همه بهایم و پرندگان آسمان و همه حیوانات صحرا را نام نهاد. لیکن برای آدم معاونی موافق وی یافت نشد.
\par 21 و خداوند خدا، خوابی گران بر آدم مستولی گردانید تا بخفت، و یکی از دنده هایش راگرفت و گوشت در جایش پر کرد.
\par 22 و خداوندخدا آن دنده را که از آدم گرفته بود، زنی بنا کرد ووی را به نزد آدم آورد.
\par 23 و آدم گفت: «همانااینست استخوانی از استخوانهایم و گوشتی ازگوشتم، از این سبب "نسا" نامیده شود زیرا که ازانسان گرفته شد.»
\par 24 از این سبب مرد پدر و مادرخود را ترک کرده، با زن خویش خواهد پیوست ویک تن خواهند بود.و آدم و زنش هر دو برهنه بودند و خجلت نداشتند.
\par 25 و آدم و زنش هر دو برهنه بودند و خجلت نداشتند.
 
\chapter{3}

\par 1 و مار از همه حیوانات صحرا که خداوندخدا ساخته بود، هشیارتر بود. و به زن گفت: «آیا خدا حقیقت گفته است که از همه درختان باغ نخورید؟»
\par 2 زن به مار گفت: «از میوه درختان باغ می‌خوریم،
\par 3 لکن از میوه درختی که در وسط باغ است، خدا گفت از آن مخورید و آن را لمس مکنید، مبادا بمیرید.»
\par 4 مار به زن گفت: «هر آینه نخواهید مرد،
\par 5 بلکه خدا می‌داند درروزی که از آن بخورید، چشمان شما باز شود ومانند خدا عارف نیک و بد خواهید بود.»
\par 6 و چون زن دید که آن درخت برای خوراک نیکوست وبنظر خوشنما و درختی دلپذیر دانش افزا، پس ازمیوه‌اش گرفته، بخورد و به شوهر خود نیز داد و اوخورد.
\par 7 آنگاه چشمان هر دو ایشان باز شد وفهمیدند که عریانند. پس برگهای انجیر به هم دوخته، سترها برای خویشتن ساختند.
\par 8 و آواز خداوند خدا را شنیدند که در هنگام وزیدن نسیم نهار در باغ می‌خرامید، و آدم و زنش خویشتن را از حضور خداوند خدا در میان درختان باغ پنهان کردند.
\par 9 و خداوند خدا آدم راندا در‌داد و گفت: «کجا هستی؟»
\par 10 گفت: «چون آوازت را در باغ شنیدم، ترسان گشتم، زیرا که عریانم. پس خود را پنهان کردم.»
\par 11 گفت: «که تورا آگاهانید که عریانی؟ آیا از آن درختی که تو راقدغن کردم که از آن نخوری، خوردی؟»
\par 12 آدم گفت: «این زنی که قرین من ساختی، وی از میوه درخت به من داد که خوردم.»
\par 13 پس خداوندخدا به زن گفت: «این چه‌کار است که کردی؟» زن گفت: «مار مرا اغوا نمود که خوردم.»
\par 14 پس خداوند خدا به مار گفت: «چونکه این کار کردی، از جمیع بهایم و از همه حیوانات صحرا ملعون تر هستی! بر شکمت راه خواهی رفت و تمام ایام عمرت خاک خواهی خورد.
\par 15 وعداوت در میان تو و زن، و در میان ذریت تو و ذریت وی می‌گذارم؛ او سر تو را خواهد کوبید وتو پاشنه وی را خواهی کوبید.»
\par 16 و به زن گفت: «الم و حمل تو را بسیار افزون گردانم؛ با الم فرزندان خواهی زایید و اشتیاق تو به شوهرت خواهد بود و او بر تو حکمرانی خواهد کرد.»
\par 17 و به آدم گفت: «چونکه سخن زوجه ات راشنیدی و از آن درخت خوردی که امرفرموده، گفتم از آن نخوری، پس بسبب توزمین ملعون شد، و تمام ایام عمرت از آن بارنج خواهی خورد.
\par 18 خار و خس نیز برایت خواهد رویانید و سبزه های صحرا را خواهی خورد،
\par 19 و به عرق پیشانی ات نان خواهی خوردتا حینی که به خاک راجع گردی، که از آن گرفته شدی زیرا که تو خاک هستی و به خاک خواهی برگشت.»
\par 20 و آدم زن خود را حوا نام نهاد، زیرا که اومادر جمیع زندگان است.
\par 21 و خداوند خدارختها برای آدم و زنش از پوست بساخت وایشان را پوشانید.
\par 22 و خداوند خدا گفت: «هماناانسان مثل یکی از ما شده است، که عارف نیک وبد گردیده. اینک مبادا دست خود را دراز کند و ازدرخت حیات نیز گرفته بخورد، و تا به ابد زنده ماند.»
\par 23 پس خداوند خدا، او را از باغ عدن بیرون کرد تا کار زمینی را که از آن گرفته شده بود، بکند.پس آدم را بیرون کرد و به طرف شرقی باغ عدن، کروبیان را مسکن داد و شمشیرآتشباری را که به هر سو گردش می‌کرد تا طریق درخت حیات را محافظت کند.
\par 24 پس آدم را بیرون کرد و به طرف شرقی باغ عدن، کروبیان را مسکن داد و شمشیرآتشباری را که به هر سو گردش می‌کرد تا طریق درخت حیات را محافظت کند.
 
\chapter{4}

\par 1 و آدم، زن خود حوا را بشناخت و او حامله شده، قائن را زایید. و گفت: «مردی از یهوه حاصل نمودم.»
\par 2 و بار دیگر برادر او هابیل رازایید. و هابیل گله بان بود، و قائن کارکن زمین بود.
\par 3 و بعد از مرور ایام، واقع شد که قائن هدیه‌ای ازمحصول زمین برای خداوند آورد.
\par 4 و هابیل نیزاز نخست زادگان گله خویش و پیه آنها هدیه‌ای آورد. و خداوند هابیل و هدیه او را منظورداشت،
\par 5 اما قائن و هدیه او را منظور نداشت. پس خشم قائن به شدت افروخته شده، سر خود رابزیر افکند.
\par 6 آنگاه خداوند به قائن گفت: «چراخشمناک شدی؟ و چرا سر خود را بزیرافکندی؟
\par 7 اگر نیکویی می‌کردی، آیا مقبول نمی شدی؟ و اگر نیکویی نکردی، گناه بر در، درکمین است و اشتیاق تو دارد، اما تو بر وی مسلطشوی.»
\par 8 و قائن با برادر خود هابیل سخن گفت. و واقع شد چون در صحرا بودند، قائن بر برادر خودهابیل برخاسته او را کشت.
\par 9 پس خداوند به قائن گفت: «برادرت هابیل کجاست؟» گفت: «نمی دانم، مگر پاسبان برادرم هستم؟»
\par 10 گفت: «چه کرده‌ای؟ خون برادرت از زمین نزد من فریادبرمی آورد!
\par 11 و اکنون تو ملعون هستی از زمینی که دهان خود را باز کرد تا خون برادرت را ازدستت فرو برد.
\par 12 هر گاه کار زمین کنی، هماناقوت خود را دیگر به تو ندهد. و پریشان و آواره در جهان خواهی بود.»
\par 13 قائن به خداوند گفت: «عقوبتم از تحملم زیاده است.
\par 14 اینک مراامروز بر روی زمین مطرود ساختی، و از روی تو پنهان خواهم بود. و پریشان و آواره درجهان خواهم بود و واقع می‌شود هر‌که مرایابد، مرا خواهد کشت.»
\par 15 خداوند به وی گفت: «پس هر‌که قائن را بکشد، هفت چندان انتقام گرفته شود.» و خداوند به قائن نشانی‌ای داد که هر‌که او را یابد، وی را نکشد.
\par 16 پس قائن از حضور خداوند بیرون رفت ودر زمین نود، بطرف شرقی عدن، ساکن شد. 
\par 17 و قائن زوجه خود را شناخت. پس حامله شده، خنوخ را زایید. و شهری بنا می‌کرد، و آن شهر را به اسم پسر خود، خنوخ نام نهاد.
\par 18 وبرای خنوخ عیراد متولد شد، و عیراد، محویائیل را آورد، و محویائیل، متوشائیل را آورد، ومتوشائیل، لمک را آورد.
\par 19 و لمک، دو زن برای خود گرفت، یکی را عاده نام بود و دیگری را ظله.
\par 20 و عاده، یابال را زایید. وی پدر خیمه نشینان وصاحبان مواشی بود.
\par 21 و نام برادرش یوبال بود. وی پدر همه نوازندگان بربط و نی بود.
\par 22 و ظله نیز توبل قائن را زایید، که صانع هر آلت مس وآهن بود. و خواهر توبل قائن، نعمه بود.
\par 23 و لمک به زنان خود گفت: «ای عاده و ظله، قول مرابشنوید! ای زنان لمک، سخن مرا گوش گیرید! زیرا مردی را کشتم بسبب جراحت خود، وجوانی را بسبب ضرب خویش.
\par 24 اگر برای قائن هفت چندان انتقام گرفته شود، هر آینه برای لمک، هفتاد و هفت چندان.»
\par 25 پس آدم بار دیگرزن خود را شناخت، و او پسری بزاد و او را شیث نام نهاد، زیرا گفت: «خدا نسلی دیگر به من قرارداد، به عوض هابیل که قائن او را کشت.»وبرای شیث نیز پسری متولد شد و او را انوش نامید. در آنوقت به خواندن اسم یهوه شروع کردند.
\par 26 وبرای شیث نیز پسری متولد شد و او را انوش نامید. در آنوقت به خواندن اسم یهوه شروع کردند.
 
\chapter{5}

\par 1 این است کتاب پیدایش آدم. در روزی که خدا آدم را آفرید، به شبیه خدا او راساخت.
\par 2 نر و ماده ایشان را آفرید. و ایشان رابرکت داد و ایشان را «آدم» نام نهاد، در روزآفرینش ایشان.
\par 3 و آدم صد و سی سال بزیست، پس پسری به شبیه و بصورت خود آورد، و او را شیث نامید.
\par 4 وایام آدم بعد از آوردن شیث، هشتصد سال بود، وپسران و دختران آورد.
\par 5 پس تمام ایام آدم که زیست، نهصد و سی سال بود که مرد.
\par 6 و شیث صد و پنج سال بزیست، و انوش را آورد.
\par 7 وشیث بعد از آوردن انوش، هشتصد و هفت سال بزیست و پسران و دختران آورد.
\par 8 و جمله ایام شیث، نهصد و دوازده سال بود که مرد.
\par 9 و انوش نود سال بزیست، و قینان را آورد.
\par 10 و انوش بعداز آوردن قینان، هشتصد و پانزده سال زندگانی کرد و پسران و دختران آورد.
\par 11 پس جمله ایام انوش نهصد و پنج سال بود که مرد.
\par 12 و قینان هفتاد سال بزیست، و مهللئیل را آورد.
\par 13 و قینان بعد از آوردن مهللئیل، هشتصد و چهل سال زندگانی کرد و پسران و دختران آورد.
\par 14 و تمامی ایام قینان، نهصد و ده سال بود که مرد.
\par 15 ومهللئیل، شصت و پنج سال بزیست، و یارد را آورد.
\par 16 و مهللئیل بعد از آوردن یارد، هشتصد وسی سال زندگانی کرد و پسران و دختران آورد.
\par 17 پس همه ایام مهللئیل، هشتصد و نود و پنج سال بود که مرد.
\par 18 و یارد صد و شصت و دو سال بزیست، و خنوخ را آورد.
\par 19 و یارد بعد از آوردن خنوخ، هشتصد سال زندگانی کرد و پسران ودختران آورد.
\par 20 و تمامی ایام یارد، نهصد وشصت و دو سال بود که مرد.
\par 21 و خنوخ شصت وپنج سال بزیست، و متوشالح را آورد.
\par 22 و خنوخ بعد از آوردن متوشالح، سیصد سال با خدا راه می‌رفت و پسران و دختران آورد.
\par 23 و همه ایام خنوخ، سیصد و شصت و پنج سال بود.
\par 24 وخنوخ با خدا راه می‌رفت و نایاب شد، زیرا خدااو را برگرفت.
\par 25 و متوشالح صد و هشتاد و هفت سال بزیست، و لمک را آورد.
\par 26 و متوشالح بعداز آوردن لمک، هفتصد و هشتاد و دو سال زندگانی کرد و پسران و دختران آورد.
\par 27 پس جمله ایام متوشالح، نهصد و شصت و نه سال بودکه مرد.
\par 28 و لمک صد و هشتاد و دو سال بزیست، و پسری آورد.
\par 29 و وی را نوح نام نهاده گفت: «این ما را تسلی خواهد داد از اعمال ما و ازمحنت دستهای ما از زمینی که خداوند آن راملعون کرد.»
\par 30 و لمک بعد از آوردن نوح، پانصدو نود و پنج سال زندگانی کرد و پسران و دختران آورد.
\par 31 پس تمام ایام لمک، هفتصد و هفتاد وهفت سال بود که مرد.و نوح پانصد ساله بود، پس نوح سام و حام و یافث را آورد.
\par 32 و نوح پانصد ساله بود، پس نوح سام و حام و یافث را آورد.
 
\chapter{6}

\par 1 و واقع شد که چون آدمیان شروع کردندبه زیاد شدن بر روی زمین و دختران برای ایشان متولد گردیدند،
\par 2 پسران خدا دختران آدمیان را دیدند که نیکومنظرند، و از هر کدام که خواستند، زنان برای خویشتن می‌گرفتند.
\par 3 وخداوند گفت: «روح من در انسان دائم داوری نخواهد کرد، زیرا که او نیز بشر است. لیکن ایام وی صد و بیست سال خواهد بود.»
\par 4 و در آن ایام مردان تنومند در زمین بودند. و بعد از هنگامی که پسران خدا به دختران آدمیان در‌آمدند و آنهابرای ایشان اولاد زاییدند، ایشان جبارانی بودندکه در زمان سلف، مردان نامور شدند.
\par 5 و خداونددید که شرارت انسان در زمین بسیار است، و هرتصور از خیالهای دل وی دائم محض شرارت است.
\par 6 و خداوند پشیمان شد که انسان را برزمین ساخته بود، و در دل خود محزون گشت.
\par 7 وخداوند گفت: «انسان را که آفریده‌ام، از روی زمین محو سازم، انسان و بهایم و حشرات وپرندگان هوا را، چونکه متاسف شدم از ساختن ایشان.»
\par 8 اما نوح در نظر خداوند التفات یافت.
\par 9 این است پیدایش نوح. نوح مردی عادل بود، و در عصر خود کامل. و نوح با خدا راه می‌رفت.
\par 10 و نوح سه پسر آورد: سام و حام و یافث.
\par 11 وزمین نیز بنظر خدا فاسد گردیده و زمین از ظلم پر شده بود.
\par 12 و خدا زمین را دید که اینک فاسدشده است، زیرا که تمامی بشر راه خود را بر زمین فاسد کرده بودند.
\par 13 و خدا به نوح گفت: «انتهای تمامی بشر به حضورم رسیده است، زیرا که زمین بسبب ایشان پر از ظلم شده است. و اینک من ایشان را با زمین هلاک خواهم ساخت.
\par 14 پس برای خودکشتی‌ای از چوب کوفر بساز، و حجرات درکشتی بنا کن و درون و بیرونش را به قیر بیندا.
\par 15 وآن را بدین ترکیب بساز که طول کشتی سیصدذراع باشد، و عرضش پنجاه ذراع و ارتفاع آن سی ذراع.
\par 16 و روشنی‌ای برای کشتی بساز و آن را به ذراعی از بالا تمام کن. و در کشتی را در جنب آن بگذار، و طبقات تحتانی و وسطی و فوقانی بساز.
\par 17 زیرا اینک من طوفان آب را بر زمین می‌آورم تاهر جسدی را که روح حیات در آن باشد، از زیرآسمان هلاک گردانم. و هر‌چه بر زمین است، خواهد مرد.
\par 18 لکن عهد خود را با تو استوارمی سازم، و به کشتی در خواهی آمد، تو وپسرانت و زوجه ات و ازواج پسرانت با تو.
\par 19 و ازجمیع حیوانات، از هر ذی جسدی، جفتی از همه به کشتی در خواهی آورد، تا با خویشتن زنده نگاه داری، نر و ماده باشند.
\par 20 از پرندگان به اجناس آنها، و از بهایم به اجناس آنها، و از همه حشرات زمین به اجناس آنها، دودو از همه نزدتو آیند تا زنده نگاه داری.
\par 21 و از هر آذوقه‌ای که خورده شود، بگیر و نزد خود ذخیره نما تابرای تو و آنها خوراک باشد.»پس نوح چنین کرد و به هرچه خدا او را امر فرمود، عمل نمود.
\par 22 پس نوح چنین کرد و به هرچه خدا او را امر فرمود، عمل نمود.
 
\chapter{7}

\par 1 و خداوند به نوح گفت: «تو و تمامی اهل خانه ات به کشتی در آیید، زیرا تو را در این عصر به حضور خود عادل دیدم.
\par 2 و از همه بهایم پاک، هفت هفت، نر و ماده با خود بگیر، و از بهایم ناپاک، دودو، نر و ماده،
\par 3 و از پرندگان آسمان نیزهفت هفت، نر و ماده را، تا نسلی بر روی تمام زمین نگاه داری.
\par 4 زیرا که من بعد از هفت روزدیگر، چهل روز و چهل شب باران می‌بارانم، وهر موجودی را که ساخته‌ام، از روی زمین محومی سازم.»
\par 5 پس نوح موافق آنچه خداوند او را امرفرموده بود، عمل نمود.
\par 6 و نوح ششصد ساله بود، چون طوفان آب بر زمین آمد.
\par 7 و نوح وپسرانش و زنش و زنان پسرانش با وی از آب طوفان به کشتی در‌آمدند.
\par 8 از بهایم پاک و ازبهایم ناپاک، و از پرندگان و از همه حشرات زمین،
\par 9 دودو، نر و ماده، نزد نوح به کشتی در‌آمدند، چنانکه خدا نوح را امر کرده بود.
\par 10 و واقع شدبعد از هفت روز که آب طوفان بر زمین آمد.
\par 11 و در سال ششصد از زندگانی نوح، در روزهفدهم از ماه دوم، در همان روز جمیع چشمه های لجه عظیم شکافته شد، و روزنهای آسمان گشوده.
\par 12 و باران، چهل روز و چهل شب بر روی زمین می‌بارید.
\par 13 در همان روز نوح وپسرانش، سام و حام و یافث، و زوجه نوح و سه زوجه پسرانش، با ایشان داخل کشتی شدند.
\par 14 ایشان و همه حیوانات به اجناس آنها، و همه بهایم به اجناس آنها، و همه حشراتی که بر زمین می‌خزند به اجناس آنها، و همه پرندگان به اجناس آنها، همه مرغان و همه بالداران.
\par 15 دودواز هر ذی جسدی که روح حیات دارد، نزد نوح به کشتی در‌آمدند.
\par 16 و آنهایی که آمدند نر و ماده از هر ذی جسد آمدند، چنانکه خدا وی را امرفرموده بود. و خداوند در را از عقب او بست.
\par 17 و طوفان چهل روز بر زمین می‌آمد، و آب همی افزود و کشتی را برداشت که از زمین بلندشد.
\par 18 و آب غلبه یافته، بر زمین همی افزود، وکشتی بر سطح آب می‌رفت.
\par 19 و آب بر زمین زیاد و زیاد غلبه یافت، تا آنکه همه کوههای بلندکه زیر تمامی آسمانها بود، مستور شد.
\par 20 پانزده ذراع بالاتر آب غلبه یافت و کوهها مستور گردید.
\par 21 و هر ذی جسدی که بر زمین حرکت می‌کرد، ازپرندگان و بهایم و حیوانات و کل حشرات خزنده بر زمین، و جمیع آدمیان، مردند.
\par 22 هرکه دم روح حیات در بینی او بود، از هر‌که در خشکی بود، مرد.
\par 23 و خدا محو کرد هر موجودی را که برروی زمین بود، از آدمیان و بهایم و حشرات وپرندگان آسمان، پس از زمین محو شدند. و نوح باآنچه همراه وی در کشتی بود فقط باقی ماند.وآب بر زمین صد و پنجاه روز غلبه می‌یافت.
\par 24 وآب بر زمین صد و پنجاه روز غلبه می‌یافت.
 
\chapter{8}

\par 1 بهایمی را که با وی در کشتی بودند، بیادآورد. و خدا بادی بر زمین وزانید و آب ساکن گردید.
\par 2 و چشمه های لجه و روزنهای آسمان بسته شد، و باران از آسمان باز ایستاد.
\par 3 و آب رفته رفته از روی زمین برگشت. و بعد از انقضای صد و پنجاه روز، آب کم شد،
\par 4 و روز هفدهم ازماه هفتم، کشتی بر کوههای آرارات قرار گرفت.
\par 5 و تا ماه دهم، آب رفته رفته کمتر می‌شد، و درروز اول از ماه دهم، قله های کوهها ظاهر گردید.
\par 6 و واقع شد بعد از چهل روز که نوح دریچه کشتی را که ساخته بود، باز کرد.
\par 7 و زاغ را رهاکرد. او بیرون رفته، در تردد می‌بود تا آب از زمین خشک شد.
\par 8 پس کبوتر را از نزد خود رها کرد تاببیند که آیا آب از روی زمین کم شده است.
\par 9 اماکبوتر چون نشیمنی برای کف پای خود نیافت، زیرا که آب در تمام روی زمین بود، نزد وی به کشتی برگشت. پس دست خود را دراز کرد و آن را گرفته نزد خود به کشتی در‌آورد.
\par 10 و هفت روز دیگر نیز درنگ کرده، باز کبوتر را از کشتی رها کرد.
\par 11 و در وقت عصر، کبوتر نزد وی برگشت، و اینک برگ زیتون تازه در منقار وی است. پس نوح دانست که آب از روی زمین کم شده است.
\par 12 و هفت روز دیگر نیز توقف نموده، کبوتر را رها کرد، و او دیگر نزد وی برنگشت.
\par 13 و در سال ششصد و یکم در روز اول از ماه اول، آب از روی زمین خشک شد. پس نوح پوشش کشتی را برداشته، نگریست، و اینک روی زمین خشک بود.
\par 14 و در روز بیست و هفتم از ماه دوم، زمین خشک شد.
\par 15 آنگاه خدا نوح رامخاطب ساخته، گفت:
\par 16 «از کشتی بیرون شو، توو زوجه ات و پسرانت و ازواج پسرانت با تو.
\par 17 وهمه حیواناتی را که نزد خود داری، هرذی جسدی را از پرندگان و بهایم و کل حشرات خزنده بر زمین، با خود بیرون آور، تا بر زمین منتشر شده، در جهان بارور و کثیر شوند.»
\par 18 پس نوح و پسران او و زنش و زنان پسرانش، با وی بیرون آمدند.
\par 19 و همه حیوانات و همه حشرات و همه پرندگان، و هر‌چه بر زمین حرکت می‌کند، به اجناس آنها، از کشتی به در شدند.
\par 20 و نوح مذبحی برای خداوند بنا کرد، و از هر بهیمه پاک واز هر پرنده پاک گرفته، قربانی های سوختنی برمذبح گذرانید. 
\par 21 و خداوند بوی خوش بویید و خداوند در دل خود گفت: «بعد از این دیگر زمین را بسبب انسان لعنت نکنم، زیرا که خیال دل انسان از طفولیت بد است، و بار دیگر همه حیوانات را هلاک نکنم، چنانکه کردم.مادامی که جهان باقی است، زرع و حصاد، و سرما و گرما، و زمستان و تابستان، و روز و شب موقوف نخواهد شد.»
\par 22 مادامی که جهان باقی است، زرع و حصاد، و سرما و گرما، و زمستان و تابستان، و روز و شب موقوف نخواهد شد.»
 
\chapter{9}

\par 1 و خدا، نوح و پسرانش را برکت داده، بدیشان گفت: «بارور و کثیر شوید و زمین را پر سازید.
\par 2 و خوف شما و هیبت شما بر همه حیوانات زمین و بر همه پرندگان آسمان، و بر هرچه بر زمین می‌خزد، و بر همه ماهیان دریا خواهدبود، به‌دست شما تسلیم شده‌اند.
\par 3 و هرجنبنده‌ای که زندگی دارد، برای شما طعام باشد. همه را چون علف سبز به شما دادم،
\par 4 مگرگوشت را با جانش که خون او باشد، مخورید.
\par 5 وهر آینه انتقام خون شما را برای جان شما خواهم گرفت. از دست هر حیوان آن را خواهم گرفت. واز دست انسان، انتقام جان انسان را از دست برادرش خواهم گرفت.
\par 6 هر‌که خون انسان ریزد، خون وی به‌دست انسان ریخته شود. زیرا خداانسان را به صورت خود ساخت.
\par 7 و شما بارور وکثیر شوید، و در زمین منتشر شده، در آن بیفزایید.»
\par 8 و خدا نوح و پسرانش را با وی خطاب کرده، گفت:
\par 9 «اینک من عهد خود را با شما و بعداز شما با ذریت شما استوار سازم،
\par 10 و با همه جانورانی که با شما باشند، از پرندگان و بهایم و همه حیوانات زمین با شما، با هر‌چه از کشتی بیرون آمد، حتی جمیع حیوانات زمین.
\par 11 عهدخود را با شما استوار می‌گردانم که بار دیگر هرذی جسد از آب طوفان هلاک نشود، و طوفان بعداز این نباشد تا زمین را خراب کند.»
\par 12 و خدا گفت: «اینست نشان عهدی که من می‌بندم، در میان خود و شما، و همه جانورانی که با شما باشند، نسلا بعد نسل تا به ابد:
\par 13 قوس خود را در ابر می‌گذارم، و نشان آن عهدی که درمیان من و جهان است، خواهد بود.
\par 14 و هنگامی که ابر را بالای زمین گسترانم، و قوس در ابر ظاهرشود،
\par 15 آنگاه عهد خود را که در میان من و شما وهمه جانوران ذی جسد می‌باشد، بیاد خواهم آورد. و آب طوفان دیگر نخواهد بود تا هرذی جسدی را هلاک کند.
\par 16 و قوس در ابرخواهد بود، و آن را خواهم نگریست تا بیاد آورم آن عهد جاودانی را که در میان خدا و همه جانوران است، از هر ذی جسدی که بر زمین است.»
\par 17 و خدا به نوح گفت: «این است نشان عهدی که استوار ساختم در میان خود و هرذی جسدی که بر زمین است.»
\par 18 و پسران نوح که از کشتی بیرون آمدند، سام و حام و یافث بودند. و حام پدر کنعان است.
\par 19 اینانند سه پسر نوح، و از ایشان تمامی جهان منشعب شد.
\par 20 و نوح به فلاحت زمین شروع کرد، وتاکستانی غرس نمود.
\par 21 و شراب نوشیده، مست شد، و در خیمه خود عریان گردید.
\par 22 و حام، پدر کنعان، برهنگی پدر خود را دید و دو برادرخود را بیرون خبر داد.
\par 23 و سام و یافث، ردا راگرفته، بر کتف خود انداختند، و پس رفته، برهنگی پدر خود را پوشانیدند. و روی ایشان بازپس بود که برهنگی پدر خود را ندیدند.
\par 24 و نوح از مستی خود به هوش آمده، دریافت که پسرکهترش با وی چه کرده بود.
\par 25 پس گفت: «کنعان ملعون باد! برادران خود را بنده بندگان باشد.»
\par 26 وگفت: «متبارک باد یهوه خدای سام! و کنعان، بنده او باشد.
\par 27 خدا یافث را وسعت دهد، و درخیمه های سام ساکن شود، و کنعان بنده او باشد.»
\par 28 و نوح بعد از طوفان، سیصد و پنجاه سال زندگانی کرد.پس جمله ایام نوح نهصد وپنجاه سال بود که مرد.
\par 29 پس جمله ایام نوح نهصد وپنجاه سال بود که مرد.
 
\chapter{10}

\par 1 این است پیدایش پسران نوح، سام وحام و یافث. و از ایشان بعد از طوفان پسران متولد شدند.
\par 2 پسران یافث: جومر و ماجوج و مادای ویاوان و توبال و ماشک و تیراس.
\par 3 و پسران جومر: اشکناز و ریفات و توجرمه.
\par 4 و پسران یاوان: الیشه و ترشیش و کتیم و رودانیم.
\par 5 از اینان جزایرامت‌ها منشعب شدند در اراضی خود، هر یکی موافق زبان و قبیله‌اش در امت های خویش.
\par 6 و پسران حام: کوش و مصرایم و فوط وکنعان.
\par 7 و پسران کوش: سبا و حویله و سبته ورعمه و سبتکا. و پسران رعمه: شبا و ددان.
\par 8 وکوش نمرود را آورد. او به جبار شدن در جهان شروع کرد.
\par 9 وی در حضور خداوند صیادی جبار بود. از این جهت می‌گویند: «مثل نمرود، صیاد جبار در حضور خداوند.»
\par 10 و ابتدای مملکت وی، بابل بود و ارک و اکد و کلنه در زمین شنعار.
\par 11 از آن زمین آشور بیرون رفت، و نینوا ورحوبوت عیر، و کالح را بنا نهاد،
\par 12 و ریسن را درمیان نینوا و کالح. و آن شهری بزرگ بود.
\par 13 ومصرایم لودیم و عنامیم و لهابیم و نفتوحیم راآورد.
\par 14 و فتروسیم و کسلوحیم را که از ایشان فلسطینیان پدید آمدند و کفتوریم را.
\par 15 و کنعان، صیدون، نخست زاده خود، وحت را آورد.
\par 16 ویبوسیان و اموریان و جرجاشیان را
\par 17 و حویان وعرقیان و سینیان را
\par 18 و اروادیان و صماریان وحماتیان را. و بعد از آن، قبایل کنعانیان منشعب شدند.
\par 19 و سرحد کنعانیان از صیدون به سمت جرار تا غزه بود، و به سمت سدوم و عموره وادمه و صبوئیم تا به لاشع.
\par 20 اینانند پسران حام برحسب قبایل و زبانهای ایشان، در اراضی وامت های خود.
\par 21 و از سام که پدر جمیع بنی عابر و برادریافث بزرگ بود، از او نیز اولاد متولد شد.
\par 22 پسران سام: عیلام و آشور و ارفکشاد و لود وارام.
\par 23 و پسران ارام: عوص و حول و جاتر وماش.
\par 24 و ارفکشاد، شالح را آورد، و شالح، عابررا آورد.
\par 25 و عابر را دو پسر متولد شد. یکی رافالج نام بود، زیرا که در ایام وی زمین منقسم شد. و نام برادرش یقطان.
\par 26 و یقطان، الموداد و شالف و حضرموت و یارح را آورد،
\par 27 و هدورام واوزال و دقله را،
\par 28 و عوبال و ابیمائیل و شبا را،
\par 29 و اوفیر و حویله و یوباب را. این همه پسران یقطان بودند.
\par 30 و مسکن ایشان از میشا بود به سمت سفاره، که کوهی از کوههای شرقی است.
\par 31 اینانند پسران سام برحسب قبایل و زبانهای ایشان، در اراضی خود برحسب امت های خویش.اینانند قبایل پسران نوح، برحسب پیدایش ایشان در امت های خود که از ایشان امت های جهان، بعد از طوفان منشعب شدند.
\par 32 اینانند قبایل پسران نوح، برحسب پیدایش ایشان در امت های خود که از ایشان امت های جهان، بعد از طوفان منشعب شدند.
 
\chapter{11}

\par 1 و تمام جهان را یک زبان و یک لغت بود.
\par 2 و واقع شد که چون از مشرق کوچ می‌کردند، همواری‌ای در زمین شنعار یافتند و درآنجا سکنی گرفتند.
\par 3 و به یکدیگر گفتند: «بیایید، خشتها بسازیم و آنها را خوب بپزیم.» و ایشان راآجر به‌جای سنگ بود، و قیر به‌جای گچ.
\par 4 وگفتند: «بیایید شهری برای خود بنا نهیم، و برجی را که سرش به آسمان برسد، تا نامی برای خویشتن پیدا کنیم، مبادا بر روی تمام زمین پراکنده شویم.»
\par 5 و خداوند نزول نمود تا شهر وبرجی را که بنی آدم بنا می‌کردند، ملاحظه نماید.
\par 6 و خداوند گفت: «همانا قوم یکی است و جمیع ایشان را یک زبان و این کار را شروع کرده‌اند، والان هیچ کاری که قصد آن بکنند، از ایشان ممتنع نخواهد شد.
\par 7 اکنون نازل شویم و زبان ایشان رادر آنجا مشوش سازیم تا سخن یکدیگر رانفهمند.»
\par 8 پس خداوند ایشان را از آنجا بر روی تمام زمین پراکنده ساخت و از بنای شهر بازماندند.
\par 9 از آن سبب آنجا را بابل نامیدند، زیرا که در آنجا خداوند لغت تمامی اهل جهان رامشوش ساخت. و خداوند ایشان را از آنجا برروی تمام زمین پراکنده نمود.
\par 10 این است پیدایش سام. چون سام صد ساله بود، ارفکشاد را دو سال بعد از طوفان آورد.
\par 11 وسام بعد از آوردن ارفکشاد، پانصد سال زندگانی کرد و پسران و دختران آورد.
\par 12 و ارفکشاد سی وپنج سال بزیست و شالح را آورد.
\par 13 و ارفکشادبعد از آوردن شالح، چهار صد و سه سال زندگانی کرد و پسران و دختران آورد.
\par 14 و شالح سی سال بزیست، و عابر را آورد.
\par 15 و شالح بعد از آوردن عابر، چهارصد و سه سال زندگانی کرد و پسران ودختران آورد.
\par 16 و عابر سی و چهار سال بزیست و فالج را آورد.
\par 17 و عابر بعد از آوردن فالج، چهار صد و سی سال زندگانی کرد و پسران ودختران آورد.
\par 18 و فالج سی سال بزیست، و رعورا آورد.
\par 19 و فالج بعد از آوردن رعو، دویست ونه سال زندگانی کرد و پسران و دختران آورد.
\par 20 ورعو سی و دو سال بزیست، و سروج را آورد.
\par 21 ورعو بعد از آوردن سروج، دویست و هفت سال زندگانی کرد و پسران و دختران آورد.
\par 22 وسروج سی سال بزیست، و ناحور را آورد.
\par 23 وسروج بعد از آوردن ناحور، دویست سال بزیست و پسران و دختران آورد.
\par 24 و ناحور بیست و نه سال بزیست، و تارح را آورد.
\par 25 و ناحور بعد ازآوردن تارح، صد و نوزده سال زندگانی کرد وپسران و دختران آورد.
\par 26 و تارح هفتاد سال بزیست، و ابرام و ناحور و هاران را آورد.
\par 27 و این است پیدایش تارح که تارح، ابرام وناحور و هاران را آورد، و هاران، لوط را آورد.
\par 28 وهاران پیش پدر خود، تارح در زادبوم خویش دراور کلدانیان بمرد.
\par 29 و ابرام و ناحور زنان برای خود گرفتند. زن ابرام را سارای نام بود. و زن ناحور را ملکه نام بود، دختر هاران، پدر ملکه وپدر یسکه.
\par 30 اما سارای نازاد مانده، ولدی نیاورد.
\par 31 پس تارح پسر خود ابرام، و نواده خودلوط، پسر هاران، و عروس خود سارای، زوجه پسرش ابرام را برداشته، با ایشان از اور کلدانیان بیرون شدند تا به ارض کنعان بروند، و به حران رسیده، در آنجا توقف نمودند.و مدت زندگانی تارح، دویست و پنج سال بود، و تارح درحران مرد.
\par 32 و مدت زندگانی تارح، دویست و پنج سال بود، و تارح درحران مرد.
 
\chapter{12}

\par 1 و خداوند به ابرام گفت: «از ولایت خود، و از مولد خویش و از خانه پدرخود بسوی زمینی که به تو نشان دهم بیرون شو،
\par 2 و از تو امتی عظیم پیدا کنم و تو را برکت دهم، ونام تو را بزرگ سازم، و تو برکت خواهی بود.
\par 3 وبرکت دهم به آنانی که تو را مبارک خوانند، ولعنت کنم به آنکه تو را ملعون خواند. و از توجمیع قبایل جهان برکت خواهند یافت.»
\par 4 پس ابرام، چنانکه خداوند بدو فرموده بود، روانه شد. و لوط همراه وی رفت. و ابرام هفتاد و پنج ساله بود، هنگامی که از حران بیرون آمد.
\par 5 و ابرام زن خود سارای، و برادرزاده خود لوط، و همه اموال اندوخته خود را با اشخاصی که در حران پیداکرده بودند، برداشته، به عزیمت زمین کنعان بیرون شدند، و به زمین کنعان داخل شدند.
\par 6 وابرام در زمین می‌گشت تا مکان شکیم تا بلوطستان موره. و در آنوقت کنعانیان در آن زمین بودند.
\par 7 وخداوند بر ابرام ظاهر شده، گفت: «به ذریت تو این زمین را می‌بخشم.» و در آنجا مذبحی برای خداوند که بر وی ظاهر شد، بنا نمود.
\par 8 پس، ازآنجا به کوهی که به شرقی بیت ئیل است، کوچ کرده، خیمه خود را برپا نمود. و بیت ئیل بطرف غربی و عای بطرف شرقی آن بود. و در آنجامذبحی برای خداوند بنا نمود و نام یهوه را خواند.
\par 9 و ابرام طی مراحل و منازل کرده، به سمت جنوب کوچید.
\par 10 و قحطی در آن زمین شد، و ابرام به مصرفرود آمد تا در آنجا بسر برد، زیرا که قحط درزمین شدت می‌کرد.
\par 11 و واقع شد که چون نزدیک به ورود مصر شد، به زن خود سارای گفت: «اینک می‌دانم که تو زن نیکومنظر هستی.
\par 12 همانا چون اهل مصر تو را بینند، گویند: "این زوجه اوست." پس مرا بکشند و تو را زنده نگاه دارند.
\par 13 پس بگو که تو خواهر من هستی تا به‌خاطر تو برای من خیریت شود و جانم بسبب توزنده ماند.» 
\par 14 و به مجرد ورود ابرام به مصر، اهل مصر آن زن را دیدند که بسیار خوش منظر است.
\par 15 و امرای فرعون او را دیدند، و او را درحضور فرعون ستودند. پس وی را به خانه فرعون در‌آوردند.
\par 16 و بخاطر وی با ابرام احسان نمود، و او صاحب میشها و گاوان وحماران و غلامان و کنیزان و ماده الاغان وشتران شد.
\par 17 و خداوند فرعون و اهل خانه اورا بسبب سارای، زوجه ابرام به بلایای سخت مبتلا ساخت.
\par 18 و فرعون ابرام را خوانده، گفت: «این چیست که به من کردی؟ چرا مرا خبرندادی که او زوجه توست؟
\par 19 چرا گفتی: اوخواهر منست، که او را به زنی گرفتم؟ و الان، اینک زوجه تو. او را برداشته، روانه شو!»آنگاه فرعون در خصوص وی، کسان خود راامر فرمود تا او را با زوجه‌اش و تمام مایملکش روانه نمودند.
\par 20 آنگاه فرعون در خصوص وی، کسان خود راامر فرمود تا او را با زوجه‌اش و تمام مایملکش روانه نمودند.
 
\chapter{13}

\par 1 و ابرام با زن خود، و تمام اموال خویش، و لوط، از مصر به جنوب آمدند.
\par 2 وابرام از مواشی و نقره و طلا، بسیار دولتمند بود.
\par 3 پس، از جنوب، طی منازل کرده، به بیت ئیل آمد، بدانجایی که خیمه‌اش در ابتدا بود، در میان بیت ئیل و عای،
\par 4 به مقام آن مذبحی که اول بنانهاده بود، و در آنجا ابرام نام یهوه را خواند.
\par 5 ولوط را نیز که همراه ابرام بود، گله و رمه و خیمه هابود.
\par 6 و زمین گنجایش ایشان را نداشت که دریکجا ساکن شوند زیرا که اندوخته های ایشان بسیار بود، و نتوانستند در یک جا سکونت کنند.
\par 7 و در میان شبانان مواشی ابرام و شبانان مواشی لوط نزاع افتاد. و در آن هنگام کنعانیان و فرزیان، ساکن زمین بودند.
\par 8 پس ابرام به لوط گفت: «زنهاردر میان من و تو، و در میان شبانان من و شبانان تونزاعی نباشد، زیرا که ما برادریم.
\par 9 مگر تمام زمین پیش روی تو نیست؟ ملتمس اینکه از من جداشوی. اگر به‌جانب چپ روی، من بسوی راست خواهم رفت و اگر بطرف راست روی، من به‌جانب چپ خواهم رفت.»
\par 10 آنگاه لوط چشمان خود را برافراشت، وتمام وادی اردن را بدید که همه‌اش مانند باغ خداوند و زمین مصر، به طرف صوغر، سیراب بود، قبل از آنکه خداوند سدوم و عموره راخراب سازد.
\par 11 پس لوط تمام وادی اردن را برای خود اختیار کرد، و لوط بطرف شرقی کوچ کرد، واز یکدیگر جدا شدند.
\par 12 ابرام در زمین کنعان ماند، و لوط در بلاد وادی ساکن شد، و خیمه خودرا تا سدوم نقل کرد.
\par 13 لکن مردمان سدوم بسیار شریر و به خداوند خطاکار بودند.
\par 14 و بعد از جداشدن لوط از وی، خداوند به ابرام گفت: «اکنون توچشمان خود را برافراز و از مکانی که در آن هستی، بسوی شمال و جنوب، و مشرق و مغرب بنگر
\par 15 زیرا تمام این زمین را که می‌بینی به تو وذریت تو تا به ابد خواهم بخشید.
\par 16 و ذریت تو رامانند غبار زمین گردانم. چنانکه اگر کسی غبارزمین را تواند شمرد، ذریت تو نیز شمرده شود.
\par 17 برخیز و در طول و عرض زمین گردش کن زیراکه آن را به تو خواهم داد.»و ابرام خیمه خودرا نقل کرده، روانه شد و در بلوطستان ممری که در حبرون است، ساکن گردید، و در آنجا مذبحی برای یهوه بنا نهاد.
\par 18 و ابرام خیمه خودرا نقل کرده، روانه شد و در بلوطستان ممری که در حبرون است، ساکن گردید، و در آنجا مذبحی برای یهوه بنا نهاد.
 
\chapter{14}

\par 1 و واقع شد در ایام امرافل، ملک شنعار، و اریوک، ملک الاسار، و کدرلاعمر، ملک عیلام، و تدعال، ملک امت‌ها،
\par 2 که ایشان بابارع، ملک سدوم، و برشاع ملک عموره، وشناب، ملک ادمه، و شمئیبر، ملک صبوئیم، وملک بالع که صوغر باشد، جنگ کردند.
\par 3 این همه در وادی سدیم که بحرالملح باشد، با هم پیوستند.
\par 4 دوازده سال، کدرلاعمر را بندگی کردند، و در سال سیزدهم، بر وی شوریدند.
\par 5 ودر سال چهاردهم، کدرلاعمر با ملوکی که با وی بودند، آمده، رفائیان را در عشتروت قرنین، وزوزیان را در هام، و ایمیان را در شاوه قریتین، شکست دادند.
\par 6 و حوریان را در کوه ایشان، سعیر، تا ایل فاران که متصل به صحراست.
\par 7 پس برگشته، به عین مشفاط که قادش باشد، آمدند، وتمام مرز و بوم عمالقه و اموریان را نیز که درحصون تامار ساکن بودند، شکست دادند.
\par 8 آنگاه ملک سدوم و ملک عموره و ملک ادمه و ملک صبوئیم و ملک بالع که صوغر باشد، بیرون آمده، با ایشان در وادی سدیم، صف آرایی نمودند.
\par 9 باکدرلاعمر ملک عیلام و تدعال، ملک امت‌ها وامرافل، ملک شنعار و اریوک ملک الاسار، چهارملک با پنج.
\par 10 و وادی سدیم پر از چاههای قیربود. پس ملوک سدوم و عموره گریخته، در آنجاافتادند و باقیان به کوه فرار کردند.
\par 11 و جمیع اموال سدوم و عموره را با تمامی ماکولات آنهاگرفته، برفتند.
\par 12 و لوط، برادرزاده ابرام را که درسدوم ساکن بود، با آنچه داشت برداشته، رفتند.
\par 13 و یکی که نجات یافته بود آمده، ابرام عبرانی را خبر داد. و او در بلوطستان ممری آموری که برادر اشکول و عانر بود، ساکن بود. وایشان با ابرام هم عهد بودند.
\par 14 چون ابرام ازاسیری برادر خود آگاهی یافت، سیصد و هجده تن از خانه زادان کارآزموده خود را بیرون آورده، در عقب ایشان تا دان بتاخت.
\par 15 شبانگاه، او وملازمانش، بر ایشان فرقه فرقه شده، ایشان راشکست داده، تا به حوبه که به شمال دمشق واقع است، تعاقب نمودند.
\par 16 و همه اموال را بازگرفت، و برادر خود، لوط و اموال او را نیز با زنان و مردان باز آورد.
\par 17 و بعد از مراجعت وی از شکست دادن کدرلاعمر و ملوکی که با وی بودند، ملک سدوم تابه وادی شاوه، که وادی الملک باشد، به استقبال وی بیرون آمد.
\par 18 و ملکیصدق، ملک سالیم، نان و شراب بیرون آورد. و او کاهن خدای تعالی بود،
\par 19 و او را مبارک خوانده، گفت: «مبارک باد ابرام ازجانب خدای تعالی، مالک آسمان و زمین.
\par 20 و متبارک باد خدای تعالی، که دشمنانت را به‌دستت تسلیم کرد.» و او را از هر چیز، ده‌یک داد.
\par 21 و ملک سدوم به ابرام گفت: «مردم را به من واگذار و اموال را برای خود نگاه دار.»
\par 22 ابرام به ملک سدوم گفت: «دست خود را به یهوه خدای تعالی، مالک آسمان و زمین، برافراشتم،
\par 23 که ازاموال تو رشته‌ای یا دوال نعلینی بر نگیرم، مباداگویی "من ابرام را دولتمند ساختم".مگر فقطآنچه جوانان خوردند و بهره عانر و اشکول وممری که همراه من رفتند، ایشان بهره خود رابردارند.»
\par 24 مگر فقطآنچه جوانان خوردند و بهره عانر و اشکول وممری که همراه من رفتند، ایشان بهره خود رابردارند.»
 
\chapter{15}

\par 1 بعد از این وقایع، کلام خداوند دررویا، به ابرام رسیده، گفت: «ای ابرام مترس، من سپر تو هستم، و اجر بسیار عظیم تو.»
\par 2 ابرام گفت: «ای خداوند یهوه، مرا چه خواهی داد، و من بی‌اولاد می‌روم، و مختارخانه‌ام، این العاذار دمشقی است؟»
\par 3 و ابرام گفت: «اینک مرا نسلی ندادی، و خانه زادم وارث من است.»
\par 4 در ساعت، کلام خداوند به وی دررسیده، گفت: «این وارث تو نخواهد بود، بلکه کسی‌که از صلب تو درآید، وارث تو خواهدبود.»
\par 5 و او را بیرون آورده، گفت: «اکنون بسوی آسمان بنگر و ستارگان را بشمار، هرگاه آنها راتوانی شمرد.» پس به وی گفت: «ذریت تو چنین خواهد بود.»
\par 6 و به خداوند ایمان آورد، و او، این را برای وی عدالت محسوب کرد.
\par 7 پس وی راگفت: «من هستم یهوه که تو را از اور کلدانیان بیرون آوردم، تا این زمین را به ارثیت، به تو بخشم.»
\par 8 گفت: «ای خداوند یهوه، به چه نشان بدانم که وارث آن خواهم بود؟»
\par 9 به وی گفت: «گوساله ماده سه ساله و بز ماده سه ساله و قوچی سه ساله و قمری و کبوتری برای من بگیر.»
\par 10 پس این همه را بگرفت، و آنها را از میان، دوپاره کرد، و هر پاره‌ای را مقابل جفتش گذاشت، لکن مرغان را پاره نکرد.
\par 11 و چون لاشخورها برلاشه‌ها فرود آمدند، ابرام آنها را راند.
\par 12 و چون آفتاب غروب می‌کرد، خوابی گران بر ابرام مستولی شد، و اینک تاریکی ترسناک سخت، اورا فرو گرفت.
\par 13 پس به ابرام گفت: «یقین بدان که ذریت تو در زمینی که از آن ایشان نباشد، غریب خواهند بود، و آنها را بندگی خواهند کرد، و آنهاچهارصد سال ایشان را مظلوم خواهند داشت.
\par 14 و بر آن امتی که ایشان بندگان آنها خواهند بود، من داوری خواهم کرد. و بعد از آن با اموال بسیاربیرون خواهند آمد.
\par 15 و تو نزد پدران خود به سلامتی خواهی رفت، و در‌پیری نیکو مدفون خواهی شد.
\par 16 و در پشت چهارم بدینجاخواهند برگشت، زیرا گناه اموریان هنوز تمام نشده است.»
\par 17 و واقع شد که چون آفتاب غروب کرده بودو تاریک شد، تنوری پر دود و چراغی مشتعل ازمیان آن پاره‌ها گذر نمود.
\par 18 در آن روز، خداوندبا ابرام عهد بست و گفت: «این زمین را از نهر مصرتا به نهر عظیم، یعنی نهر فرات، به نسل توبخشیده‌ام،
\par 19 یعنی قینیان و قنزیان و قدمونیان وحتیان و فرزیان و رفائیان،و اموریان و کنعانیان و جرجاشیان و یبوسیان را.»
\par 20 و اموریان و کنعانیان و جرجاشیان و یبوسیان را.»
 
\chapter{16}

\par 1 و سارای، زوجه ابرام، برای وی فرزندی نیاورد. و او را کنیزی مصری، هاجر نام بود.
\par 2 پس سارای به ابرام گفت: «اینک خداوند مرا از زاییدن باز داشت. پس به کنیز من درآی، شاید از او بنا شوم.» و ابرام سخن سارای را قبول نمود.
\par 3 و چون ده سال از اقامت ابرام درزمین کنعان سپری شد، سارای زوجه ابرام، کنیزخود هاجر مصری را برداشته، او را به شوهرخود، ابرام، به زنی داد.
\par 4 پس به هاجر درآمد و اوحامله شد. و چون دید که حامله است، خاتونش بنظر وی حقیر شد.
\par 5 و سارای به ابرام گفت: «ظلم من بر تو باد! من کنیز خود را به آغوش تو دادم وچون آثار حمل در خود دید، در نظر او حقیرشدم. خداوند در میان من و تو داوری کند.»
\par 6 ابرام به سارای گفت: «اینک کنیز تو به‌دست توست، آنچه پسند نظر تو باشد با وی بکن.» پس چون سارای با وی بنای سختی نهاد، او از نزد وی بگریخت.
\par 7 و فرشته خداوند او را نزد چشمه آب دربیابان، یعنی چشمه‌ای که به راه شور است، یافت.
\par 8 و گفت: «ای هاجر کنیز سارای، از کجا آمدی وکجا می‌روی؟» گفت: «من از حضور خاتون خودسارای گریخته‌ام.»
\par 9 فرشته خداوند به وی گفت: «نزد خاتون خود برگرد و زیر دست او مطیع شو.»
\par 10 و فرشته خداوند به وی گفت: «ذریت تو رابسیار افزون گردانم، به حدی که از کثرت به شماره نیایند.»
\par 11 و فرشته خداوند وی را گفت: «اینک حامله هستی و پسری خواهی زایید، و او را اسماعیل نام خواهی نهاد، زیرا خداوند تظلم تو را شنیده است.
\par 12 و او مردی وحشی خواهدبود، دست وی به ضد هر کس و دست هر کس به ضد او، و پیش روی همه برادران خود ساکن خواهد بود.»
\par 13 و او، نام خداوند را که با وی تکلم کرد، «انت ایل رئی» خواند، زیرا گفت: «آیا اینجانیز به عقب او که مرا می‌بیند، نگریستم.»
\par 14 از این سبب آن چاه را «بئرلحی رئی» نامیدند، اینک درمیان قادش و بارد است.
\par 15 و هاجر از ابرام پسری زایید، و ابرام پسر خود را که هاجر زایید، اسماعیل نام نهاد.و ابرام هشتاد و شش ساله بود چون هاجر اسماعیل را برای ابرام بزاد.
\par 16 و ابرام هشتاد و شش ساله بود چون هاجر اسماعیل را برای ابرام بزاد.
 
\chapter{17}

\par 1 و چون ابرام نود و نه ساله بود، خداوندبر ابرام ظاهر شده، گفت: «من هستم خدای قادر مطلق، پیش روی من بخرام و کامل شو،
\par 2 و عهد خویش را در میان خود و تو خواهم بست، و تو را بسیاربسیار کثیر خواهم گردانید.»
\par 3 آنگاه ابرام به روی در‌افتاد و خدا به وی خطاب کرده، گفت:
\par 4 «اما من اینک عهد من با توست و توپدر امت های بسیار خواهی بود.
\par 5 و نام تو بعد ازاین ابرام خوانده نشود بلکه نام تو ابراهیم خواهدبود، زیرا که تو را پدر امت های بسیار گردانیدم.
\par 6 و تو را بسیار بارور نمایم و امت‌ها از تو پدیدآورم و پادشاهان از تو به وجود آیند. 
\par 7 و عهدخویش را در میان خود و تو، و ذریتت بعد از تو، استوار گردانم که نسلا بعد نسل عهد جاودانی باشد، تا تو را و بعد از تو ذریت تو را خدا باشم.
\par 8 وزمین غربت تو، یعنی تمام زمین کنعان را، به تو و بعد از تو به ذریت تو به ملکیت ابدی دهم، وخدای ایشان خواهم بود.»
\par 9 پس خدا به ابراهیم گفت: «و اما تو عهد مرا نگاه دار، تو و بعد از توذریت تو در نسلهای ایشان.
\par 10 این است عهد من که نگاه خواهید داشت، در میان من و شما و ذریت تو بعد از تو هر ذکوری از شما مختون شود،
\par 11 وگوشت قلفه خود را مختون سازید، تا نشان آن عهدی باشد که در میان من و شماست.
\par 12 هر پسرهشت روزه از شما مختون شود. هر ذکوری درنسلهای شما، خواه خانه زاد خواه زرخرید، ازاولاد هر اجنبی که از ذریت تو نباشد،
\par 13 هرخانه زاد تو و هر زر خرید تو البته مختون شود تاعهد من در گوشت شما عهد جاودانی باشد.
\par 14 واما هر ذکور نامختون که گوشت قلفه او ختنه نشود، آن کس از قوم خود منقطع شود، زیرا که عهد مرا شکسته است.»
\par 15 و خدا به ابراهیم گفت: «اما زوجه توسارای، نام او را سارای مخوان، بلکه نام او ساره باشد.
\par 16 و او را برکت خواهم داد و پسری نیز ازوی به تو خواهم بخشید. او را برکت خواهم داد وامتها از وی به وجود خواهند آمد، و ملوک امتهااز وی پدید خواهند شد.»
\par 17 آنگاه ابراهیم به روی در‌افتاده، بخندید و در دل خود گفت: «آیابرای مرد صد ساله پسری متولد شود و ساره درنود سالگی بزاید؟»
\par 18 و ابراهیم به خدا گفت: «کاش که اسماعیل در حضور تو زیست کند.»
\par 19 خدا گفت: «به تحقیق زوجه ات ساره برای توپسری خواهد زایید، و او را اسحاق نام بنه، و عهدخود را با وی استوار خواهم داشت، تا با ذریت اوبعد از او عهد ابدی باشد.
\par 20 و اما در خصوص اسماعیل، تو را اجابت فرمودم. اینک او را برکت داده، بارور گردانم، و او را بسیار کثیر گردانم. دوازده رئیس از وی پدید آیند، و امتی عظیم ازوی بوجود آورم.
\par 21 لکن عهد خود را با اسحاق استوار خواهم ساخت، که ساره او را بدین وقت در سال آینده برای تو خواهد زایید.»
\par 22 و چون خدا از سخن‌گفتن با وی فارغ شد، از نزد ابراهیم صعود فرمود.
\par 23 و ابراهیم پسر خود، اسماعیل وهمه خانه زادان و زرخریدان خود را، یعنی هرذکوری که در خانه ابراهیم بود، گرفته، گوشت قلفه ایشان را در همان روز ختنه کرد، چنانکه خدا به وی امر فرموده بود.
\par 24 و ابراهیم نود و نه ساله بود، وقتی که گوشت قلفه‌اش مختون شد.
\par 25 و پسرش، اسماعیل سیزده ساله بود هنگامی که گوشت قلفه‌اش مختون شد.
\par 26 در همان روزابراهیم و پسرش، اسماعیل مختون گشتند.وهمه مردان خانه‌اش، خواه خانه زاد، خواه زرخرید از اولاد اجنبی، با وی مختون شدند.
\par 27 وهمه مردان خانه‌اش، خواه خانه زاد، خواه زرخرید از اولاد اجنبی، با وی مختون شدند.
 
\chapter{18}

\par 1 و خداوند در بلوطستان ممری، بروی ظاهر شد، و او در گرمای روز به درخیمه نشسته بود.
\par 2 ناگاه چشمان خود را بلندکرده، دید که اینک سه مرد در مقابل اوایستاده‌اند. و چون ایشان را دید، از در خیمه به استقبال ایشان شتافت، و رو بر زمین نهاد
\par 3 و گفت: «ای مولا، اکنون اگر منظور نظر تو شدم، از نزدبنده خود مگذر،
\par 4 اندک آبی بیاورند تا پای خودرا شسته، در زیر درخت بیارامید،
\par 5 و لقمه نانی بیاورم تا دلهای خود را تقویت دهید و پس از آن روانه شوید، زیرا برای همین، شما را بر بنده خودگذر افتاده است.» گفتند: «آنچه گفتی بکن.»
\par 6 پس ابراهیم به خیمه، نزد ساره شتافت و گفت: «سه کیل از آرد میده بزودی حاضر کن و آن را خمیرکرده، گرده‌ها بساز.»
\par 7 و ابراهیم به سوی رمه شتافت و گوساله نازک خوب گرفته، به غلام خودداد تا بزودی آن را طبخ نماید.
\par 8 پس کره و شیر وگوساله‌ای را که ساخته بود، گرفته، پیش روی ایشان گذاشت، و خود در مقابل ایشان زیردرخت ایستاد تا خوردند.
\par 9 به وی گفتند: «زوجه ات ساره کجاست؟» گفت: «اینک درخیمه است.»
\par 10 گفت: «البته موافق زمان حیات، نزد تو خواهم برگشت، و زوجه ات ساره را پسری خواهد شد.» و ساره به در خیمه‌ای که در عقب اوبود، شنید.
\par 11 و ابراهیم و ساره پیر و سالخورده بودند، و عادت زنان از ساره منقطع شده بود.
\par 12 پس ساره در دل خود بخندید و گفت: «آیا بعداز فرسودگی‌ام مرا شادی خواهد بود، و آقایم نیزپیر شده است؟»
\par 13 و خداوند به ابراهیم گفت: «ساره برای چه خندید؟» و گفت: «آیا فی الحقیقه خواهم زایید و حال آنکه پیر هستم؟»
\par 14 «مگرهیچ امری نزد خداوند مشکل است؟ در وقت موعود، موافق زمان حیات، نزد تو خواهم برگشت و ساره را پسری خواهد شد.»
\par 15 آنگاه ساره انکار کرده، گفت: «نخندیدم»، چونکه ترسید. گفت: «نی، بلکه خندیدی.»
\par 16 پس، آن مردان از آنجا برخاسته، متوجه سدوم شدند، و ابراهیم ایشان را مشایعت نمود.
\par 17 و خداوند گفت: «آیا آنچه من می‌کنم ازابراهیم مخفی دارم؟
\par 18 و حال آنکه از ابراهیم هرآینه امتی بزرگ و زورآور پدید خواهد آمد، وجمیع امت های جهان از او برکت خواهند یافت.
\par 19 زیرا او را می‌شناسم که فرزندان و اهل خانه خود را بعد از خود امر خواهد فرمود تا طریق خداوند را حفظ نمایند، و عدالت و انصاف را بجاآورند، تا خداوند آنچه به ابراهیم گفته است، به وی برساند.»
\par 20 پس خداوند گفت: «چونکه فریاد سدوم و عموره زیاد شده است، و خطایای ایشان بسیار گران،
\par 21 اکنون نازل می‌شوم تا ببینم موافق این فریادی که به من رسیده، بالتمام کرده اندوالا خواهم دانست.»
\par 22 آنگاه آن مردان از آنجابسوی سدوم متوجه شده، برفتند. و ابراهیم درحضور خداوند هنوز ایستاده بود.
\par 23 و ابراهیم نزدیک آمده، گفت: «آیا عادل را با شریر هلاک خواهی کرد؟
\par 24 شاید در شهر پنجاه عادل باشند، آیا آن را هلاک خواهی کرد و آن مکان را بخاطرآن پنجاه عادل که در آن باشند، نجات نخواهی داد؟
\par 25 حاشا از تو که مثل این کار بکنی که عادلان را با شریران هلاک سازی و عادل و شریر مساوی باشند. حاشا از تو آیا داور تمام جهان، انصاف نخواهد کرد؟»
\par 26 خداوند گفت: «اگر پنجاه عادل در شهر سدوم یابم هر آینه تمام آن مکان را به‌خاطر ایشان رهایی دهم.»
\par 27 ابراهیم در جواب گفت: «اینک من که خاک و خاکستر هستم جرات کردم که به خداوند سخن گویم.
\par 28 شاید از آن پنجاه عادل پنج کم باشد، آیا تمام شهر را بسبب پنج، هلاک خواهی کرد؟» گفت: «اگر‌چهل و پنج در آنجا یابم، آن را هلاک نکنم.»
\par 29 بار دیگر بدو عرض کرده، گفت: «هر گاه در آنجا چهل یافت شوند؟» گفت: «به‌خاطر چهل آن را نکنم.»
\par 30 گفت: «زنهار غضب خداوند افروخته نشود تاسخن گویم، شاید در آنجا سی پیدا شوند؟» گفت: «اگر در آنجا سی یابم، این کار را نخواهم کرد.»
\par 31 گفت: «اینک جرات کردم که به خداوندعرض کنم. اگر بیست در آنجا یافت شوند؟» گفت: «به‌خاطر بیست آن را هلاک نکنم.»
\par 32 گفت: «خشم خداوند، افروخته نشود تا این دفعه را فقط عرض کنم، شاید ده در آنجا یافت شوند؟» گفت: «به‌خاطر ده آن را هلاک نخواهم ساخت.»پس خداوند چون گفتگو را با ابراهیم به اتمام رسانید، برفت و ابراهیم به مکان خویش مراجعت کرد.
\par 33 پس خداوند چون گفتگو را با ابراهیم به اتمام رسانید، برفت و ابراهیم به مکان خویش مراجعت کرد.
 
\chapter{19}

\par 1 و وقت عصر، آن دو فرشته وارد سدوم شدند، و لوط به دروازه سدوم نشسته بود. و چون لوط ایشان را بدید، به استقبال ایشان برخاسته، رو بر زمین نهاد
\par 2 و گفت: «اینک اکنون‌ای آقایان من، به خانه بنده خود بیایید، و شب رابسر برید، و پایهای خود را بشویید و بامدادان برخاسته، راه خود را پیش گیرید.» گفتند: «نی، بلکه شب را در کوچه بسر بریم.»
\par 3 اما چون ایشان را الحاح بسیار نمود، با او آمده، به خانه‌اش داخل شدند، و برای ایشان ضیافتی نمود و نان فطیرپخت، پس تناول کردند.
\par 4 و به خواب هنوز نرفته بودند که مردان شهر، یعنی مردم سدوم، از جوان و پیر، تمام قوم از هر جانب، خانه وی را احاطه کردند.
\par 5 و به لوط ندا در‌داده، گفتند: «آن دو مرد که امشب به نزد تو درآمدند، کجا هستند؟ آنها رانزد ما بیرون آور تا ایشان را بشناسیم.»
\par 6 آنگاه لوط نزد ایشان، بدرگاه بیرون آمد و در را از عقب خود ببست
\par 7 و گفت: «ای برادران من، زنهار بدی مکنید.
\par 8 اینک من دو دختر دارم که مرد رانشناخته‌اند. ایشان را الان نزد شما بیرون آورم وآنچه در نظر شما پسند آید، با ایشان بکنید. لکن کاری بدین دو مرد ندارید، زیرا که برای همین زیرسایه سقف من آمده‌اند.»
\par 9 گفتند: «دور شو.» وگفتند: «این یکی آمد تا نزیل ما شود و پیوسته داوری می‌کند. الان با تو از ایشان بدتر کنیم.» پس بر آن مرد، یعنی لوط، بشدت هجوم آورده، نزدیک آمدند تا در را بشکنند.
\par 10 آنگاه آن دو مرد، دست خود را پیش آورده، لوط را نزد خود به خانه درآوردند و در رابستند.
\par 11 اما آن اشخاصی را که به در خانه بودند، از خرد و بزرگ، به کوری مبتلا کردند، که ازجستن در، خویشتن را خسته ساختند.
\par 12 و آن دومرد به لوط گفتند: «آیا کسی دیگر دراینجا داری؟ دامادان و پسران و دختران خود و هر‌که را درشهر داری، از این مکان بیرون آور،
\par 13 زیرا که مااین مکان را هلاک خواهیم ساخت، چونکه فریادشدید ایشان به حضور خداوند رسیده و خداوندما را فرستاده است تا آن را هلاک کنیم.»
\par 14 پس لوط بیرون رفته، با دامادان خود که دختران او راگرفتند، مکالمه کرده، گفت: «برخیزید و از این مکان بیرون شوید، زیرا خداوند این شهر را هلاک می‌کند.» اما بنظر دامادان مسخره آمد.
\par 15 و هنگام طلوع فجر، آن دو فرشته، لوط را شتابانیده، گفتند: «برخیز و زن خود را با این دودختر که حاضرند بردار، مبادا در گناه شهر هلاک شوی.»
\par 16 و چون تاخیر می‌نمود، آن مردان، دست او و دست زنش و دست هر دو دخترش راگرفتند، چونکه خداوند بر وی شفقت نمود و اورا بیرون آورده، در خارج شهر گذاشتند.
\par 17 وواقع شد چون ایشان را بیرون آورده بودند که یکی به وی گفت: «جان خود را دریاب و از عقب منگر، و در تمام وادی مایست، بلکه به کوه بگریز، مبادا هلاک شوی.»
\par 18 لوط بدیشان گفت: «ای آقاچنین مباد!
\par 19 همانا بنده ات در نظرت التفات یافته است و احسانی عظیم به من کردی که جانم را رستگار ساختی، و من قدرت آن ندارم که به کوه فرار کنم، مبادا این بلا مرا فرو‌گیرد و بمیرم.
\par 20 اینک این شهر نزدیک است تا بدان فرار کنم، ونیز صغیر است. اذن بده تا بدان فرار کنم. آیا صغیرنیست، تا جانم زنده ماند.»
\par 21 بدو گفت: «اینک دراین امر نیز تو را اجابت فرمودم، تا شهری را که سفارش آن را نمودی، واژگون نسازم.
\par 22 بدان جابزودی فرار کن، زیرا که تا تو بدانجا نرسی، هیچ نمی توانم کرد.» از این سبب آن شهر مسمی به صوغر شد.
\par 23 و چون آفتاب بر زمین طلوع کرد، لوط به صوغر داخل شد.
\par 24 آنگاه خداوند برسدوم و عموره، گوگرد و آتش، از حضورخداوند از آسمان بارانید.
\par 25 و آن شهرها، و تمام وادی، و جمیع سکنه شهرها و نباتات زمین راواژگون ساخت.
\par 26 اما زن او، از عقب خودنگریسته، ستونی از نمک گردید.
\par 27 بامدادان، ابراهیم برخاست و به سوی آن مکانی که در آن به حضور خداوند ایستاده بود، رفت.
\par 28 و چون به سوی سدوم و عموره، و تمام زمین وادی نظر انداخت، دید که اینک دود آن زمین، چون دود کوره بالا می‌رود.
\par 29 و هنگامی که خدا، شهرهای وادی را هلاک کرد، خدا، ابراهیم را به یاد آورد، و لوط را از آن انقلاب بیرون آورد، چون آن شهرهایی را که لوط در آنهاساکن بود، واژگون ساخت.
\par 30 و لوط از صوغر برآمد و با دو دختر خوددر کوه ساکن شد زیرا ترسید که در صوغر بماند. پس با دو دختر خود در مغاره سکنی گرفت. 
\par 31 ودختر بزرگ به کوچک گفت: «پدر ما پیر شده ومردی بر روی زمین نیست که برحسب عادت کل جهان، به ما در‌آید.
\par 32 بیا تا پدر خود را شراب بنوشانیم، و با او هم بستر شویم، تا نسلی از پدرخود نگاه داریم.»
\par 33 پس در همان شب، پدر خودرا شراب نوشانیدند، و دختر بزرگ آمده با پدرخویش همخواب شد، و او از خوابیدن وبرخاستن وی آگاه نشد.
\par 34 و واقع شد که روزدیگر، بزرگ به کوچک گفت: «اینک دوش با پدرم همخواب شدم، امشب نیز او را شراب بنوشانیم، و تو بیا و با وی همخواب شو، تا نسلی از پدرخود نگاه داریم.»
\par 35 آن شب نیز پدر خود راشراب نوشانیدند، و دختر کوچک همخواب وی شد، و او از خوابیدن و برخاستن وی آگاه نشد.
\par 36 پس هر دو دختر لوط از پدر خود حامله شدند.
\par 37 و آن بزرگ، پسری زاییده، او را موآب نام نهاد، و او تا امروز پدر موآبیان است.و کوچک نیز پسری بزاد، و او را بن عمی نام نهاد. وی تا بحال پدر بنی عمون است.
\par 38 و کوچک نیز پسری بزاد، و او را بن عمی نام نهاد. وی تا بحال پدر بنی عمون است.
 
\chapter{20}

\par 1 پس ابراهیم از آنجا بسوی ارض جنوبی کوچ کرد، و در میان قادش وشور، ساکن شد و در جرار منزل گرفت.
\par 2 وابراهیم در خصوص زن خود، ساره، گفت که «اوخواهر من است.» و ابی ملک، ملک جرار، فرستاده، ساره را گرفت.
\par 3 و خدا در رویای شب، بر ابی ملک ظاهر شده، به وی گفت: «اینک تومرده‌ای بسبب این زن که گرفتی، زیرا که زوجه دیگری می‌باشد.»
\par 4 و ابی ملک، هنوز به اونزدیکی نکرده بود. پس گفت: «ای خداوند، آیاامتی عادل را هلاک خواهی کرد؟
\par 5 مگر او به من نگفت که "او خواهر من است "، و او نیز خود گفت که "او برادر من است؟" به ساده دلی و پاک دستی خود این را کردم.»
\par 6 خدا وی را در رویا گفت: «من نیز می‌دانم که این را به ساده دلی خود کردی، و من نیز تو را نگاه داشتم که به من خطا نورزی، و از این سبب نگذاشتم که او را لمس نمایی.
\par 7 پس الان زوجه این مرد را رد کن، زیرا که او نبی است، وبرای تو دعا خواهد کرد تا زنده بمانی، و اگر او رارد نکنی، بدان که تو و هر‌که از آن تو باشد، هرآینه خواهید مرد.»
\par 8 بامدادان، ابی ملک برخاسته، جمیع خادمان خود را طلبیده، همه این امور را به سمع ایشان رسانید، و ایشان بسیار ترسان شدند.
\par 9 پس ابی ملک، ابراهیم را خوانده، بدو گفت: «به ما چه کردی؟ و به تو چه گناه کرده بودم، که بر من و بر مملکت من گناهی عظیم آوردی و کارهای ناکردنی به من کردی؟»
\par 10 و ابی ملک به ابراهیم گفت: «چه دیدی که این کار را کردی؟»
\par 11 ابراهیم گفت: «زیرا گمان بردم که خداترسی در این مکان نباشد، و مرا به جهت زوجه‌ام خواهند کشت.
\par 12 وفی الواقع نیز او خواهر من است، دختر پدرم، اما نه دختر مادرم، و زوجه من شد.
\par 13 و هنگامی که خدا مرا از خانه پدرم آواره کرد، او را گفتم: احسانی که به من باید کرد، این است که هر جابرویم، درباره من بگویی که او برادر من است.»
\par 14 پس ابی ملک، گوسفندان و گاوان و غلامان وکنیزان گرفته، به ابراهیم بخشید، و زوجه‌اش ساره را به وی رد کرد.
\par 15 و ابی ملک گفت: «اینک زمین من پیش روی توست، هر جا که پسند نظرت افتد، ساکن شو.»
\par 16 و به ساره گفت: «اینک هزار مثقال نقره به برادرت دادم، همانا او برای تو پرده چشم است، نزد همه کسانی که با تو هستند، و نزد همه دیگران، پس انصاف تو داده شد.»
\par 17 و ابراهیم نزدخدا دعا کرد. و خدا ابی ملک، و زوجه او وکنیزانش را شفا بخشید، تا اولاد بهم رسانیدند،زیرا خداوند، رحم های تمام اهل بیت ابی ملک را بخاطر ساره، زوجه ابراهیم بسته بود.
\par 18 زیرا خداوند، رحم های تمام اهل بیت ابی ملک را بخاطر ساره، زوجه ابراهیم بسته بود.
 
\chapter{21}

\par 1 و خداوند برحسب وعده خود، ازساره تفقد نمود، و خداوند، آنچه به ساره گفته بود، بجا آورد.
\par 2 و ساره حامله شده، ازابراهیم در‌پیری‌اش، پسری زایید، در وقتی که خدا به وی گفته بود.
\par 3 و ابراهیم، پسر مولود خودرا، که ساره از وی زایید، اسحاق نام نهاد.
\par 4 و ابراهیم پسر خود اسحاق را، چون هشت روزه بود، مختون ساخت، چنانکه خدا او را امر فرموده بود.
\par 5 و ابراهیم، در هنگام ولادت پسرش، اسحاق، صد ساله بود.
\par 6 و ساره گفت: «خدا خنده برای من ساخت، و هر‌که بشنود، با من خواهدخندید.»
\par 7 و گفت: «که بود که به ابراهیم بگوید، ساره اولاد را شیر خواهد داد؟ زیرا که پسری برای وی، در‌پیری‌اش زاییدم.»
\par 8 و آن پسر نموکرد، تا او را از شیر باز گرفتند. و در روزی که اسحاق را از شیر باز داشتند، ابراهیم ضیافتی عظیم کرد.
\par 9 آنگاه ساره، پسر هاجر مصری را که ازابراهیم زاییده بود، دید که خنده می‌کند.
\par 10 پس به ابراهیم گفت: «این کنیز را با پسرش بیرون کن، زیرا که پسر کنیز با پسر من اسحاق، وارث نخواهد بود.»
\par 11 اما این امر، بنظر ابراهیم، درباره پسرش بسیار سخت آمد.
\par 12 خدا به ابراهیم گفت: «درباره پسر خود و کنیزت، بنظرت سخت نیاید، بلکه هر‌آنچه ساره به تو گفته است، سخن او رابشنو، زیرا که ذریت تو از اسحاق خوانده خواهدشد.
\par 13 و از پسر کنیز نیز امتی بوجود آورم، زیراکه او نسل توست.»
\par 14 بامدادان، ابراهیم برخاسته، نان و مشکی از آب گرفته، به هاجر داد، و آنها را بر دوش وی نهاد، و او را با پسر روانه کرد. پس رفت، و در بیابان بئرشبع می‌گشت.
\par 15 وچون آب مشک تمام شد، پسر را زیر بوته‌ای گذاشت.
\par 16 و به مسافت تیر پرتابی رفته، در مقابل وی بنشست، زیرا گفت: «موت پسر را نبینم.» و در مقابل او نشسته، آواز خود را بلند کرد وبگریست.
\par 17 و خدا، آواز پسر را بشنید، و فرشته خدا از آسمان، هاجر را ندا کرده، وی را گفت: «ای هاجر، تو را چه شد؟ ترسان مباش، زیراخدا، آواز پسر را در آنجایی که اوست، شنیده است.
\par 18 برخیز و پسر را برداشته، او را به‌دست خود بگیر، زیرا که از او، امتی عظیم بوجودخواهم آورد.»
\par 19 و خدا چشمان او را باز کرد تاچاه آبی دید. پس رفته، مشک را از آب پر کرد، وپسر را نوشانید.
\par 20 و خدا با آن پسر می‌بود. و اونمو کرده، ساکن صحرا شد، و در تیراندازی بزرگ گردید.
\par 21 و در صحرای فاران، ساکن شد. ومادرش زنی از زمین مصر برایش گرفت.
\par 22 و واقع شد، در آن زمانی که ابی ملک، وفیکول، که سپهسالار او بود، ابراهیم را عرض کرده، گفتند که «خدا در آنچه می‌کنی با توست.
\par 23 اکنون برای من، در اینجا به خدا سوگند بخور، که با من و نسل من و ذریت من خیانت نخواهی کرد، بلکه برحسب احسانی که با تو کرده‌ام، با من و با زمینی که در آن غربت پذیرفتی، عمل خواهی نمود.»
\par 24 ابراهیم گفت: «من سوگند می‌خورم.»
\par 25 و ابراهیم ابی ملک را تنبیه کرد، بسبب چاه آبی که خادمان ابی ملک، از او به زور گرفته بودند.
\par 26 ابی ملک گفت: «نمی دانم کیست که این کار راکرده است، و تو نیز مرا خبر ندادی، و من هم تاامروز نشنیده بودم.»
\par 27 و ابراهیم، گوسفندان وگاوان گرفته، به ابی ملک داد، و با یکدیگر عهدبستند.
\par 28 و ابراهیم، هفت بره از گله جدا ساخت.
\par 29 گفت: «که این هفت بره ماده را از دست من قبول فرمای، تا شهادت باشدکه این چاه را من حفر نمودم.»
\par 30 از این سبب، آن مکان را، بئرشبع نامید، زیرا که در آنجا با یکدیگرقسم خوردند.
\par 31 و چون آن عهد را در بئرشبع بسته بودند، ابی ملک با سپهسالار خود فیکول برخاسته، به زمین فلسطینیان مراجعت کردند.
\par 32 و ابراهیم در بئرشبع، شوره کزی غرس نمود، ودر آنجا به نام یهوه، خدای سرمدی، دعا نمود.پس ابراهیم در زمین فلسطینیان، ایام بسیاری بسر برد.
\par 33 پس ابراهیم در زمین فلسطینیان، ایام بسیاری بسر برد.
 
\chapter{22}

\par 1 و واقع شد بعد از این وقایع، که خداابراهیم را امتحان کرده، بدو گفت: «ای ابراهیم!» عرض کرد: «لبیک.»
\par 2 گفت: «اکنون پسرخود را، که یگانه توست و او را دوست می‌داری، یعنی اسحاق را بردار و به زمین موریا برو، و او رادر آنجا، بر یکی از کوههایی که به تو نشان می‌دهم، برای قربانی سوختنی بگذران.»
\par 3 بامدادان، ابراهیم برخاسته، الاغ خود رابیاراست، و دو نفر از نوکران خود را، با پسرخویش اسحاق، برداشته و هیزم برای قربانی سوختنی، شکسته، روانه شد، و به سوی آن مکانی که خدا او را فرموده بود، رفت.
\par 4 و در روزسوم، ابراهیم چشمان خود را بلند کرده، آن مکان را از دور دید.
\par 5 آنگاه ابراهیم، به خادمان خودگفت: «شما در اینجا نزد الاغ بمانید، تا من با پسربدانجا رویم، و عبادت کرده، نزد شما بازآییم.»
\par 6 پس ابراهیم، هیزم قربانی سوختنی را گرفته، بر پسر خود اسحاق نهاد، و آتش و کارد را به‌دست خود گرفت؛ و هر دو با هم می‌رفتند.
\par 7 واسحاق پدر خود، ابراهیم را خطاب کرده، گفت: «ای پدر من!» گفت: «ای پسر من لبیک؟» گفت: «اینک آتش و هیزم، لکن بره قربانی کجاست؟»
\par 8 ابراهیم گفت: «ای پسر من، خدا بره قربانی رابرای خود مهیا خواهد ساخت.» و هر دو با هم رفتند.
\par 9 چون بدان مکانی که خدا بدو فرموده بود، رسیدند، ابراهیم در آنجا مذبح را بنا نمود، و هیزم را بر هم نهاد، و پسر خود، اسحاق را بسته، بالای هیزم، بر مذبح گذاشت.
\par 10 و ابراهیم، دست خودرا دراز کرده، کارد را گرفت، تا پسر خویش را ذبح نماید.
\par 11 در حال، فرشته خداوند از آسمان وی را ندا درداد و گفت: «ای ابراهیم! ای ابراهیم!» عرض کرد: «لبیک.»
\par 12 گفت: «دست خود را برپسر دراز مکن، و بدو هیچ مکن، زیرا که الان دانستم که تو از خدا می‌ترسی، چونکه پسر یگانه خود را از من دریغ نداشتی.»
\par 13 آنگاه، ابراهیم، چشمان خود را بلند کرده، دید که اینک قوچی، در عقب وی، در بیشه‌ای، به شاخهایش گرفتارشده. پس ابراهیم رفت و قوچ را گرفته، آن را درعوض پسر خود، برای قربانی سوختنی گذرانید.
\par 14 و ابراهیم آن موضع را «یهوه یری» نامید، چنانکه تا امروز گفته می‌شود: «در کوه، یهوه، دیده خواهد شد.»
\par 15 بار دیگر فرشته خداوند، به ابراهیم ازآسمان ندا در‌داد
\par 16 و گفت: «خداوند می‌گوید: به ذات خود قسم می‌خورم، چونکه این کار را کردی و پسر یگانه خود را دریغ نداشتی،
\par 17 هرآینه تو را برکت دهم، و ذریت تو را کثیر سازم، مانند ستارگان آسمان، و مثل ریگهایی که بر کناره دریاست. و ذریت تو دروازه های دشمنان خود رامتصرف خواهند شد.
\par 18 و از ذریت تو، جمیع امتهای زمین برکت خواهند یافت، چونکه قول مرا شنیدی.»
\par 19 پس ابراهیم نزد نوکران خودبرگشت، و ایشان برخاسته، به بئرشبع با هم آمدند، و ابراهیم در بئرشبع ساکن شد.
\par 20 و واقع شد بعد از این امور، که به ابراهیم خبر داده، گفتند: «اینک ملکه نیز برای برادرت ناحور، پسران زاییده است.
\par 21 یعنی نخست زاده او عوص، و برادرش بوز و قموئیل، پدر ارام،
\par 22 وکاسد و حزو و فلداش و یدلاف و بتوئیل.»
\par 23 وبتوئیل، رفقه را آورده است. این هشت را، ملکه برای ناحور، برادر ابراهیم زایید.و کنیز او که رومه نام داشت، او نیز طابح و جاحم و تاحش ومعکه را زایید.
\par 24 و کنیز او که رومه نام داشت، او نیز طابح و جاحم و تاحش ومعکه را زایید.
 
\chapter{23}

\par 1 و ایام زندگانی ساره، صد و بیست وهفت سال بود، این است سالهای عمرساره.
\par 2 و ساره در قریه اربع، که حبرون باشد، درزمین کنعان مرد. و ابراهیم آمد تا برای ساره ماتم و گریه کند.
\par 3 و ابراهیم از نزد میت خودبرخاست، و بنی حت را خطاب کرده، گفت:
\par 4 «من نزد شما غریب و نزیل هستم. قبری از نزد خود به ملکیت من دهید، تا میت خود را از پیش روی خود دفن کنم.»
\par 5 پس بنی حت در جواب ابراهیم گفتند:
\par 6 «ای مولای من، سخن ما را بشنو. تو درمیان ما رئیس خدا هستی. در بهترین مقبره های ما، میت خود را دفن کن. هیچ کدام از ما، قبرخویش را از تو دریغ نخواهد داشت که میت خودرا دفن کنی.» 
\par 7 پس ابراهیم برخاست، و نزد اهل آن زمین، یعنی بنی حت، تعظیم نمود.
\par 8 و ایشان را خطاب کرده، گفت: «اگر مرضی شما باشد که میت خود را از نزد خود دفن کنم، سخن مرابشنوید و به عفرون بن صوحار، برای من سفارش کنید،
\par 9 تا مغاره مکفیله را که از املاک او در کنارزمینش واقع است، به من دهد، به قیمت تمام، درمیان شما برای قبر، به ملکیت من بسپارد.»
\par 10 و عفرون در میان بنی حت نشسته بود. پس عفرون حتی، در مسامع بنی حت، یعنی همه که به دروازه شهر او داخل می‌شدند، در جواب ابراهیم گفت:
\par 11 «ای مولای من، نی، سخن مرا بشنو، آن زمین را به تو می‌بخشم، و مغاره‌ای را که در آن است به تو می‌دهم، بحضور ابنای قوم خود، آن رابه تو می‌بخشم. میت خود را دفن کن.»
\par 12 پس ابراهیم نزد اهل آن زمین تعظیم نمود،
\par 13 و عفرون را به مسامع اهل زمین خطاب کرده، گفت: «اگر تو راضی هستی، التماس دارم عرض مرا اجابت کنی. قیمت زمین را به تو می‌دهم، از من قبول فرمای، تا در آنجا میت خود را دفن کنم.»
\par 14 عفرون در جواب ابراهیم گفت:
\par 15 «ای مولای من، از من بشنو، قیمت زمین چهارصد مثقال نقره است، این در میان من و تو چیست؟ میت خود رادفن کن.»
\par 16 پس ابراهیم، سخن عفرون را اجابت نمود، و آن مبلغی را که در مسامع بنی حت گفته بود، یعنی چهارصد مثقال نقره رایج المعامله، به نزدعفرون وزن کرد.
\par 17 پس زمین عفرون، که درمکفیله، برابر ممری واقع است، یعنی زمین ومغاره‌ای که در آن است، با همه درختانی که در آن زمین، و در تمامی حدود و حوالی آن بود، مقررشد
\par 18 به ملکیت ابراهیم، بحضور بنی حت، یعنی همه که به دروازه شهرش داخل می‌شدند.
\par 19 ازآن پس، ابراهیم، زوجه خود، ساره را در مغاره صحرای مکفیله، در مقابل ممری، که حبرون باشد، در زمین کنعان دفن کرد.و آن صحرا، بامغاره‌ای که در آن است، از جانب بنی حت، به ملکیت ابراهیم، به جهت قبر مقرر شد.
\par 20 و آن صحرا، بامغاره‌ای که در آن است، از جانب بنی حت، به ملکیت ابراهیم، به جهت قبر مقرر شد.
 
\chapter{24}

\par 1 و ابراهیم پیر و سالخورده شد، وخداوند، ابراهیم را در هر چیز برکت داد.
\par 2 و ابراهیم به خادم خود، که بزرگ خانه وی، و بر تمام مایملک او مختار بود، گفت: «اکنون دست خود را زیر ران من بگذار.
\par 3 و به یهوه، خدای آسمان و خدای زمین، تو را قسم می‌دهم، که زنی برای پسرم از دختر کنعانیان، که در میان ایشان ساکنم، نگیری،
\par 4 بلکه به ولایت من و به مولدم بروی، و از آنجا زنی برای پسرم اسحاق بگیری.»
\par 5 خادم به وی گفت: «شاید آن زن راضی نباشد که با من بدین زمین بیاید؟ آیا پسرت را بدان زمینی که از آن بیرون آمدی، بازبرم؟»
\par 6 ابراهیم وی را گفت: «زنهار، پسر مرا بدانجا باز مبری.
\par 7 یهوه، خدای آسمان، که مرا از خانه پدرم و اززمین مولد من، بیرون آورد، و به من تکلم کرد، وقسم خورده، گفت: "که این زمین را به ذریت توخواهم داد." او فرشته خود را پیش روی توخواهد فرستاد، تا زنی برای پسرم از آنجا بگیری.
\par 8 اما اگر آن زن از آمدن با تو رضا ندهد، از این قسم من، بری خواهی بود، لیکن زنهار، پسر مرابدانجا باز نبری.»
\par 9 پس خادم دست خود را زیرران آقای خود ابراهیم نهاد، و در این امر برای اوقسم خورد.
\par 10 و خادم ده شتر، از شتران آقای خود گرفته، برفت. و همه اموال مولایش به‌دست او بود. پس روانه شده، به شهر ناحور در ارام نهرین آمد.
\par 11 وبه وقت عصر، هنگامی که زنان برای کشیدن آب بیرون می‌آمدند، شتران خود را در خارج شهر، برلب چاه آب خوابانید.
\par 12 و گفت: «ای یهوه، خدای آقایم ابراهیم، امروز مرا کامیاب بفرما، و باآقایم ابراهیم احسان بنما.
\par 13 اینک من بر این چشمه آب ایستاده‌ام، و دختران اهل این شهر، به جهت کشیدن آب بیرون می‌آیند.
\par 14 پس چنین بشود که آن دختری که به وی گویم: "سبوی خودرا فرودآر تا بنوشم "، و او گوید: "بنوش و شترانت را نیز سیراب کنم "، همان باشد که نصیب بنده خود اسحاق کرده باشی، تا بدین، بدانم که با آقایم احسان فرموده‌ای.»
\par 15 و او هنوز از سخن‌گفتن فارغ نشده بود که ناگاه، رفقه، دختر بتوئیل، پسر ملکه، زن ناحور، برادر ابراهیم، بیرون آمد و سبویی بر کتف داشت.
\par 16 و آن دختر بسیار نیکومنظر و باکره بود، ومردی او را نشناخته بود. پس به چشمه فرورفت، و سبوی خود را پر کرده، بالا آمد.
\par 17 آنگاه خادم به استقبال او بشتافت و گفت: «جرعه‌ای آب ازسبوی خود به من بنوشان.»
\par 18 گفت: «ای آقای من بنوش»، و سبوی خود را بزودی بر دست خودفرودآورده، او را نوشانید.
\par 19 و چون ازنوشانیدنش فارغ شد، گفت: «برای شترانت نیزبکشم تا از نوشیدن بازایستند.»
\par 20 پس سبوی خود را بزودی در آبخور خالی کرد و باز به سوی چاه، برای کشیدن بدوید، و از بهر همه شترانش کشید.
\par 21 و آن مرد بر وی چشم دوخته بود وسکوت داشت، تا بداند که خداوند، سفر او راخیریت اثر نموده است یا نه.
\par 22 و واقع شد، چون شتران از نوشیدن بازایستادند، که آن مرد حلقه طلای نیم مثقال وزن، ودو ابرنجین برای دستهایش، که ده مثقال طلا وزن آنها بود، بیرون آورد.
\par 23 و گفت: «به من بگو که دختر کیستی؟ آیا در خانه پدرت جایی برای ماباشد تا شب را بسر بریم؟»
\par 24 وی را گفت: «من دختر بتوئیل، پسر ملکه که او را از ناحور زایید، می‌باشم.»
\par 25 و بدو گفت: «نزد ما کاه و علف فراوان است، و جای نیز برای منزل.»
\par 26 آنگاه آن مرد خم شد، خداوند را پرستش نمود
\par 27 و گفت: «متبارک باد یهوه، خدای آقایم ابراهیم، که لطف و وفای خود را از آقایم دریغ نداشت، و چون من در راه بودم، خداوند مرا به خانه برادران آقایم راهنمایی فرمود.»
\par 28 پس آن دختر دوان دوان رفته، اهل خانه مادر خویش را از این وقایع خبر داد.
\par 29 و رفقه رابرادری لابان نام بود. پس لابان به نزد آن مرد، به‌سر چشمه، دوان دوان بیرون آمد.
\par 30 و واقع شدکه چون آن حلقه و ابرنجینها را بر دستهای خواهر خود دید، و سخنهای خواهر خود، رفقه را شنید که می‌گفت آن مرد چنین به من گفته است، به نزد وی آمد. و اینک نزد شتران به‌سر چشمه ایستاده بود.
\par 31 و گفت: «ای مبارک خداوند، بیا، چرا بیرون ایستاده‌ای؟ من خانه را و منزلی برای شتران، مهیا ساخته‌ام.»
\par 32 پس آن مرد به خانه درآمد، و لابان شتران را باز کرد، و کاه و علف به شتران داد، و آب به جهت شستن پایهایش و پایهای رفقایش آورد.
\par 33 و غذا پیش او نهادند. وی گفت: «تا مقصود خود را بازنگویم، چیزی نخورم.» گفت: «بگو.»
\par 34 گفت: «من خادم ابراهیم هستم.
\par 35 وخداوند، آقای مرا بسیار برکت داده و او بزرگ شده است، و گله‌ها و رمه‌ها و نقره و طلا و غلامان و کنیزان و شتران و الاغان بدو داده است.
\par 36 وزوجه آقایم ساره، بعد از پیر شدن، پسری برای آقایم زایید، و آنچه دارد، بدو داده است.
\par 37 وآقایم مرا قسم داد و گفت که "زنی برای پسرم ازدختران کنعانیان که در زمین ایشان ساکنم، نگیری.
\par 38 بلکه به خانه پدرم و به قبیله من بروی، و زنی برای پسرم بگیری."
\par 39 و به آقای خودگفتم: "شاید آن زن همراه من نیاید؟"
\par 40 او به من گفت: "یهوه که به حضور او سالک بوده‌ام، فرشته خود را با تو خواهد فرستاد، و سفر تو راخیریت اثر خواهد گردانید، تا زنی برای پسرم ازقبیله‌ام و از خانه پدرم بگیری.
\par 41 آنگاه از قسم من بری خواهی گشت، چون به نزد قبیله‌ام رفتی، هرگاه زنی به تو ندادند، از سوگند من بری خواهی بود."
\par 42 پس امروز به‌سر چشمه رسیدم و گفتم: "ای یهوه، خدای آقایم ابراهیم، اگر حال، سفر مراکه به آن آمده‌ام، کامیاب خواهی کرد.
\par 43 اینک من به‌سر این چشمه آب ایستاده‌ام. پس چنین بشودکه آن دختری که برای کشیدن آب بیرون آید، وبه وی گویم: "مرا از سبوی خود جرعه‌ای آب بنوشان "،
\par 44 و به من گوید: "بیاشام، و برای شترانت نیز آب می‌کشم "، او همان زن باشد که خداوند، نصیب آقازاده من کرده است.
\par 45 و من هنوز از گفتن این، در دل خود فارغ نشده بودم، که ناگاه، رفقه با سبویی بر کتف خود بیرون آمد، و به چشمه پایین رفت، تا آب بکشد. و به وی گفتم: "جرعه‌ای آب به من بنوشان."
\par 46 پس سبوی خودرا بزودی از کتف خود فروآورده، گفت: "بیاشام، و شترانت را نیز آب می‌دهم." پس نوشیدم وشتران را نیز آب داد.
\par 47 و از او پرسیده، گفتم: "تودختر کیستی؟" گفت: "دختر بتوئیل بن ناحور که ملکه، او را برای او زایید." پس حلقه را در بینی او، و ابرنجین‌ها را بر دستهایش گذاشتم.
\par 48 آنگاه سجده کرده، خداوند را پرستش نمودم. و یهوه، خدای آقای خود ابراهیم را، متبارک خواندم، که مرا به راه راست هدایت فرمود، تا دختر برادرآقای خود را، برای پسرش بگیرم.
\par 49 اکنون اگربخواهید با آقایم احسان و صداقت کنید، پس مراخبر دهید. و اگر نه مرا خبر دهید، تا بطرف راست یا چپ ره سپر شوم.»
\par 50 لابان و بتوئیل در جواب گفتند: «این امر ازخداوند صادر شده است، با تو نیک یا بدنمی توانیم گفت.
\par 51 اینک رفقه حاضر است، او رابرداشته، روانه شو تا زن پسر آقایت باشد، چنانکه خداوند گفته است.»
\par 52 و واقع شد که چون خادم ابراهیم سخن ایشان را شنید، خداوند را به زمین سجده کرد.
\par 53 و خادم، آلات نقره و آلات طلا و رختها رابیرون آورده، پیشکش رفقه کرد، و برادر و مادر اورا چیزهای نفیسه داد.
\par 54 و او و رفقایش خوردندو آشامیدند و شب را بسر بردند. و بامدادان برخاسته، گفت: «مرا به سوی آقایم روانه نمایید.»
\par 55 برادر و مادر او گفتند: «دختر با ما ده روزی بماند و بعد از آن روانه شود.»
\par 56 بدیشان گفت: «مرا معطل مسازید، خداوند سفر مرا کامیاب گردانیده است، پس مرا روانه نمایید تا بنزد آقای خود بروم.»
\par 57 گفتند: «دختر را بخوانیم و اززبانش بپرسیم.»
\par 58 پس رفقه را خواندند و به وی گفتند: «با این مرد خواهی رفت؟» گفت: «می‌روم.»
\par 59 آنگاه خواهر خود رفقه، و دایه‌اش را، با خادم ابراهیم و رفقایش روانه کردند.
\par 60 ورفقه را برکت داده به وی گفتند: «تو خواهر ماهستی، مادر هزار کرورها باش، و ذریت تو، دروازه دشمنان خود را متصرف شوند.»
\par 61 پس رفقه با کنیزانش برخاسته، بر شتران سوار شدند، و از عقب آن مرد روانه گردیدند. وخادم، رفقه را برداشته، برفت.
\par 62 و اسحاق از راه بئرلحی رئی می‌آمد، زیرا که او در ارض جنوب ساکن بود.
\par 63 و هنگام شام، اسحاق برای تفکر به صحرا بیرون رفت، و چون نظر بالا کرد، دید که شتران می‌آیند.
\par 64 و رفقه چشمان خود را بلندکرده، اسحاق را دید، و از شتر خود فرود آمد،
\par 65 زیرا که از خادم پرسید: «این مرد کیست که درصحرا به استقبال ما می‌آید؟» و خادم گفت: «آقای من است.» پس برقع خود را گرفته، خود راپوشانید.
\par 66 و خادم، همه کارهایی را که کرده بود، به اسحاق باز‌گفت.و اسحاق، رفقه را به خیمه مادر خود، ساره، آورد، و او را به زنی خود گرفته، دل در او بست. و اسحاق بعد از وفات مادر خود، تسلی پذیرفت.
\par 67 و اسحاق، رفقه را به خیمه مادر خود، ساره، آورد، و او را به زنی خود گرفته، دل در او بست. و اسحاق بعد از وفات مادر خود، تسلی پذیرفت.
 
\chapter{25}

\par 1 و ابراهیم، دیگر بار، زنی گرفت که قطوره نام داشت.
\par 2 و او زمران و یقشان و مدان و مدیان و یشباق و شوحا را برای او زایید.
\par 3 و یقشان، شبا و ددان را آورد. و بنی ددان، اشوریم و لطوشیم و لامیم بودند.
\par 4 و پسران مدیان، عیفا و عیفر و حنوک و ابیداع و الداعه بودند. جمله اینها، اولاد قطوره بودند.
\par 5 و ابراهیم تمام مایملک خود را به اسحاق بخشید.
\par 6 اما به پسران کنیزانی که ابراهیم داشت، ابراهیم عطایاداد، و ایشان را در حین حیات خود، از نزد پسرخویش اسحاق، به‌جانب مشرق، به زمین شرقی فرستاد. 
\par 7 این است ایام سالهای عمر ابراهیم، که زندگانی نمود: صد و هفتاد وپنج سال.
\par 8 و ابراهیم جان بداد، و در کمال شیخوخیت، پیر و سیر شده، بمرد. و به قوم خود ملحق شد.
\par 9 و پسرانش، اسحاق و اسماعیل، او را در مغاره مکفیله، درصحرای عفرون بن صوحارحتی، در مقابل ممری دفن کردند.
\par 10 آن صحرایی که ابراهیم، ازبنی حت، خریده بود. در آنجا ابراهیم و زوجه‌اش ساره مدفون شدند.
\par 11 و واقع شد بعد از وفات ابراهیم، که خدا پسرش اسحاق را برکت داد، واسحاق نزد بئرلحی رئی ساکن بود.
\par 12 این است پیدایش اسماعیل بن ابراهیم، که هاجر مصری، کنیز ساره، برای ابراهیم زایید.
\par 13 واین است نامهای پسران اسماعیل، موافق اسمهای ایشان به حسب پیدایش ایشان. نخست زاده اسماعیل، نبایوت، و قیدار و ادبیل ومبسام.
\par 14 و مشماع و دومه و مسا
\par 15 و حدار وتیما و یطور و نافیش و قدمه.
\par 16 اینانند پسران اسماعیل، و این است نامهای ایشان در بلدان وحله های ایشان، دوازده امیر، حسب قبایل ایشان.
\par 17 و مدت زندگانی اسماعیل، صد و سی و هفت سال بود، که جان را سپرده، بمرد. و به قوم خودملحق گشت.
\par 18 و ایشان از حویله تا شور، که مقابل مصر، به سمت آشور واقع است، ساکن بودند. و نصیب او در مقابل همه برادران او افتاد.
\par 19 و این است پیدایش اسحاق بن ابراهیم. ابراهیم، اسحاق را آورد.
\par 20 و چون اسحاق چهل ساله شد، رفقه دختر بتوئیل ارامی و خواهر لابان ارامی را، از فدان ارام به زنی گرفت.
\par 21 و اسحاق برای زوجه خود، چون که نازاد بود، نزد خداونددعا کرد. و خداوند او را مستجاب فرمود. وزوجه‌اش رفقه حامله شد.
\par 22 و دو طفل در رحم او منازعت می‌کردند. او گفت: «اگر چنین باشد، من چرا چنین هستم؟» پس رفت تا از خداوندبپرسد.
\par 23 خداوند به وی گفت: «دو امت در بطن تو هستند، و دو قوم از رحم تو جدا شوند. وقومی بر قومی تسلط خواهد یافت، و بزرگ، کوچک را بندگی خواهد نمود.»
\par 24 و چون وقت وضع حملش رسید، اینک توامان در رحم اوبودند.
\par 25 و نخستین، سرخ‌فام بیرون آمد، وتمامی بدنش مانند پوستین، پشمین بود. و او راعیسو نام نهادند.
\par 26 و بعد از آن، برادرش بیرون آمد، و پاشنه عیسو را به‌دست خود گرفته بود. واو را یعقوب نام نهادند. و درحین ولادت ایشان، اسحاق، شصت ساله بود.
\par 27 و آن دو پسر، نموکردند، و عیسو صیادی ماهر، و مرد صحرایی بود. و اما یعقوب، مرد ساده دل و چادرنشین.
\par 28 واسحاق، عیسو را دوست داشتی، زیرا که صید اورا می‌خورد. اما رفقه، یعقوب را محبت نمودی.
\par 29 روزی یعقوب، آش می‌پخت و عیسو وا مانده، از صحرا آمد.
\par 30 و عیسو به یعقوب گفت: «از این آش ادوم (یعنی سرخ ) مرا بخوران، زیرا که وامانده‌ام.» از این سبب او را ادوم نامیدند.
\par 31 یعقوب گفت: «امروز نخست زادگی خود را به من بفروش.»
\par 32 عیسو گفت: «اینک من به حالت موت رسیده‌ام، پس مرا از نخست زادگی چه فایده؟»
\par 33 یعقوب گفت: «امروز برای من قسم بخور.» پس برای او قسم خورد، و نخست زادگی خود را به یعقوب فروخت.و یعقوب نان وآش عدس را به عیسو داد، که خورد و نوشید وبرخاسته، برفت. پس عیسو نخست زادگی خود راخوار نمود.
\par 34 و یعقوب نان وآش عدس را به عیسو داد، که خورد و نوشید وبرخاسته، برفت. پس عیسو نخست زادگی خود راخوار نمود.
 
\chapter{26}

\par 1 و قحطی در آن زمین حادث شد، غیرآن قحط اول، که در ایام ابراهیم بود. واسحاق نزد ابی ملک، پادشاه فلسطینیان به جراررفت.
\par 2 و خداوند بر وی ظاهر شده، گفت: «به مصر فرود میا، بلکه به زمینی که به تو بگویم ساکن شو.
\par 3 در این زمین توقف نما، و با تو خواهم بود، وتو را برکت خواهم داد، زیرا که به تو و ذریت تو، تمام این زمین را می‌دهم، و سوگندی را که باپدرت ابراهیم خوردم، استوار خواهم داشت.
\par 4 وذریتت را مانند ستارگان آسمان کثیر گردانم، وتمام این زمینها را به ذریت تو بخشم، و از ذریت تو جمیع امتهای جهان برکت خواهند یافت.
\par 5 زیرا که ابراهیم قول مرا شنید، و وصایا و اوامر وفرایض و احکام مرا نگاه داشت.»
\par 6 پس اسحاق در جرار اقامت نمود.
\par 7 و مردمان آن مکان درباره زنش از او جویا شدند. گفت: «او خواهر من است، » زیرا ترسید که بگوید: «زوجه من است، » مبادا اهل آنجا، او را به‌خاطررفقه که نیکومنظر بود، بکشند.
\par 8 و چون در آنجامدتی توقف نمود، چنان افتاد که ابی ملک، پادشاه فلسطینیان، از دریچه نظاره کرد، و دید که اینک اسحاق با زوجه خود رفقه، مزاح می‌کند.
\par 9 پس ابی ملک، اسحاق را خوانده، گفت: «همانا این زوجه توست! پس چرا گفتی که خواهر من است؟» اسحاق بدو گفت: «زیرا گفتم که مبادابرای وی بمیرم.»
\par 10 ابی ملک گفت: «این چه‌کاراست که با ما کردی؟ نزدیک بود که یکی از قوم، بازوجه ات همخواب شود، و بر ما جرمی آورده باشی.»
\par 11 و ابی ملک تمامی قوم را قدغن فرموده، گفت: «کسی‌که متعرض این مرد و زوجه‌اش بشود، هر آینه خواهد مرد.»
\par 12 و اسحاق در آن زمین زراعت کرد، و در آن سال صد چندان پیدا نمود؛ و خداوند او را برکت داد.
\par 13 و آن مرد بزرگ شده، آن فان ترقی می‌نمود، تا بسیار بزرگ گردید.
\par 14 و او را گله گوسفندان و مواشی گاوان و غلامان کثیر بود، وفلسطینیان بر او حسد بردند.
\par 15 و همه چاههایی که نوکران پدرش در ایام پدرش ابراهیم، کنده بودند، فلسطینیان آنها را بستند، و از خاک پرکردند.
\par 16 و ابی ملک به اسحاق گفت: «از نزد مابرو، زیرا که از ما بسیار بزرگتر شده‌ای.»
\par 17 پس اسحاق از آنجا برفت، و در وادی جرارفرود آمده، در آنجا ساکن شد.
\par 18 و چاههای آب را، که در ایام پدرش ابراهیم کنده بودند، وفلسطینیان آنها را بعد از وفات ابراهیم بسته بودند، اسحاق از سر نو کند، و آنها را مسمی نمودبه نامهایی که پدرش آنها را نامیده بود.
\par 19 ونوکران اسحاق در آن وادی حفره زدند، و چاه آب زنده‌ای در آنجا یافتند.
\par 20 و شبانان جرار، باشبانان اسحاق، منازعه کرده، گفتند: «این آب ازآن ماست!» پس آن چاه را عسق نامید، زیرا که باوی منازعه کردند.
\par 21 و چاهی دیگر کندند، همچنان برای آن نیز جنگ کردند، و آن را سطنه نامید.
\par 22 و از آنجا کوچ کرده، چاهی دیگر کند، وبرای آن جنگ نکردند، پس آن را رحوبوت نامیده، گفت: «که اکنون خداوند ما را وسعت داده است، و در زمین، بارور خواهیم شد.»
\par 23 پس از آنجا به بئرشبع آمد.
\par 24 در همان شب، خداوند بر وی ظاهر شده، گفت: «من خدای پدرت، ابراهیم، هستم. ترسان مباش زیراکه من با تو هستم، و تو را برکت می‌دهم، و ذریت تو را بخاطر بنده خود، ابراهیم، فراوان خواهم ساخت.»
\par 25 و مذبحی در آنجا بنا نهاد و نام یهوه را خواند، و خیمه خود را برپا نمود، و نوکران اسحاق، چاهی در آنجا کندند.
\par 26 و ابی ملک، به اتفاق یکی از اصحاب خود، احزات نام، وفیکول، که سپهسالار او بود، از جرار به نزد اوآمدند.
\par 27 و اسحاق بدیشان گفت: «چرا نزد من آمدید، با آنکه با من عداوت نمودید، و مرا از نزدخود راندید؟»
\par 28 گفتند: «به تحقیق فهمیده‌ایم که خداوند با توست. پس گفتیم سوگندی در میان ماو تو باشد، و عهدی با تو ببندیم.
\par 29 تا با ما بدی نکنی چنانکه به تو ضرری نرساندیم، بلکه غیر ازنیکی به تو نکردیم، و تو را به سلامتی روانه نمودیم، و اکنون مبارک خداوند هستی.»
\par 30 آنگاه برای ایشان ضیافتی برپا نمود، وخوردند و آشامیدند.
\par 31 بامدادان برخاسته، بایکدیگر قسم خوردند، و اسحاق ایشان را وداع نمود. پس، از نزد وی به سلامتی رفتند.
\par 32 و درآن روز چنان افتاد که نوکران اسحاق آمده، او را ازآن چاهی که می‌کندند خبر داده، گفتند: «آب یافتیم!»
\par 33 پس آن را شبعه نامید. از این سبب آن شهر، تا امروز بئرشبع نام دارد.
\par 34 و چون عیسوچهل ساله بود، یهودیه، دختر بیری حتی، وبسمه، دختر ایلون حتی را به زنی گرفت.وایشان باعث تلخی جان اسحاق و رفقه شدند.
\par 35 وایشان باعث تلخی جان اسحاق و رفقه شدند.
 
\chapter{27}

\par 1 و چون اسحاق پیر شد، و چشمانش از دیدن تار گشته بود، پسر بزرگ خودعیسو را طلبیده، به وی گفت: «ای پسر من!» گفت: «لبیک.»
\par 2 گفت: «اینک پیر شده‌ام و وقت اجل خود را نمی دانم.
\par 3 پس اکنون، سلاح خود یعنی ترکش و کمان خویش را گرفته، به صحرا برو، ونخجیری برای من بگیر،
\par 4 و خورشی برای من چنانکه دوست می‌دارم ساخته، نزد من حاضر کن، تا بخورم و جانم قبل از مردنم تو را برکت دهد.»
\par 5 و چون اسحاق به پسر خود عیسو سخن می‌گفت، رفقه بشنید، و عیسو به صحرا رفت تانخجیری صید کرده، بیاورد.
\par 6 آنگاه رفقه پسرخود، یعقوب را خوانده، گفت: «اینک پدر تو راشنیدم که برادرت عیسو را خطاب کرده، می‌گفت:
\par 7 "برای من شکاری آورده، خورشی بساز تا آن را بخورم، و قبل از مردنم تو را درحضور خداوند برکت دهم."
\par 8 پس‌ای پسر من، الان سخن مرا بشنو در آنچه من به تو امر می‌کنم.
\par 9 بسوی گله بشتاب، و دو بزغاله خوب از بزها، نزد من بیاور، تا از آنها غذایی برای پدرت بطوری که دوست می‌دارد، بسازم.
\par 10 و آن را نزد پدرت ببر تا بخورد، و تو را قبل از وفاتش برکت دهد.»
\par 11 یعقوب به مادر خود، رفقه، گفت: «اینک برادرم عیسو، مردی مویدار است و من مردی بی‌موی هستم؛
\par 12 شاید که پدرم مرا لمس نماید، و در نظرش مثل مسخره‌ای بشوم، و لعنت به عوض برکت بر خود آورم.»
\par 13 مادرش به وی گفت: «ای پسر من، لعنت تو بر من باد! فقط سخن مرا بشنو و رفته، آن را برای من بگیر.»
\par 14 پس رفت و گرفته، نزد مادر خود آورد. و مادرش خورشی ساخت بطوری که پدرش دوست می‌داشت.
\par 15 و رفقه، جامه فاخر پسر بزرگ خودعیسو را، که نزد او در خانه بود گرفته، به پسر کهترخود، یعقوب پوشانید،
\par 16 و پوست بزغاله‌ها را، بر دستها و نرمه گردن او بست.
\par 17 و خورش ونانی که ساخته بود، به‌دست پسر خود یعقوب سپرد.
\par 18 پس نزد پدر خود آمده، گفت: «ای پدرمن!» گفت: «لبیک، تو کیستی‌ای پسر من؟»
\par 19 یعقوب به پدر خود گفت: «من نخست زاده توعیسو هستم. آنچه به من فرمودی کردم، الان برخیز، بنشین و از شکار من بخور، تا جانت مرابرکت دهد.»
\par 20 اسحاق به پسر خود گفت: «ای پسر من! چگونه بدین زودی یافتی؟» گفت: «یهوه خدای تو به من رسانید.»
\par 21 اسحاق به یعقوب گفت: «ای پسر من، نزدیک بیا تا تو را لمس کنم، که آیا تو پسر من عیسو هستی یا نه.»
\par 22 پس یعقوب نزد پدر خود اسحاق آمد، و او را لمس کرده، گفت: «آواز، آواز یعقوب است، لیکن دستها، دستهای عیسوست.»
\par 23 و او را نشناخت، زیرا که دستهایش مثل دستهای برادرش عیسو، موی دار بود، پس او را برکت داد.
\par 24 و گفت: «آیاتو همان پسر من، عیسو هستی؟» گفت: «من هستم.»
\par 25 پس گفت: «نزدیک بیاور تا از شکارپسر خود بخورم و جانم تو را برکت دهد.» پس نزد وی آورد و بخورد و شراب برایش آورد ونوشید.
\par 26 و پدرش، اسحاق به وی گفت: «ای پسر من، نزدیک بیا و مرا ببوس.»
\par 27 پس نزدیک آمده، او را بوسید و رایحه لباس او را بوییده، او رابرکت داد و گفت: «همانا رایحه پسر من، مانندرایحه صحرایی است که خداوند آن را برکت داده باشد.
\par 28 پس خدا تو را از شبنم آسمان و ازفربهی زمین، و از فراوانی غله و شیره عطا فرماید.
\par 29 قومها تو را بندگی نمایند و طوایف تو راتعظیم کنند، بر برادران خود سرور شوی، وپسران مادرت تو را تعظیم نمایند. ملعون باد هرکه تو را لعنت کند، و هر‌که تو را مبارک خواند، مبارک باد.»
\par 30 و واقع شد چون اسحاق، از برکت دادن به یعقوب فارغ شد، به مجرد بیرون رفتن یعقوب ازحضور پدر خود اسحاق، که برادرش عیسو ازشکار باز آمد.
\par 31 و او نیز خورشی ساخت، و نزدپدر خود آورده، به پدر خود گفت: «پدر من برخیزد و از شکار پسر خود بخورد، تا جانت مرابرکت دهد.» 
\par 32 پدرش اسحاق به وی گفت: «توکیستی؟» گفت: «من پسر نخستین تو، عیسوهستم.»
\par 33 آنگاه لرزه‌ای شدید بر اسحاق مستولی شده، گفت: «پس آن که بود که نخجیری صید کرده، برایم آورد، و قبل از آمدن تو از همه خوردم و او را برکت دادم، و فی الواقع او مبارک خواهد بود؟»
\par 34 عیسو چون سخنان پدر خود راشنید، نعره‌ای عظیم و بی‌نهایت تلخ برآورده، به پدر خود گفت: «ای پدرم، به من، به من نیز برکت بده!»
\par 35 گفت: «برادرت به حیله آمد، و برکت تورا گرفت.»
\par 36 گفت: «نام او را یعقوب بخوبی نهادند، زیرا که دو مرتبه مرا از پا درآورد. اول نخست زادگی مرا گرفت، و اکنون برکت مرا گرفته است.» پس گفت: «آیا برای من نیز برکتی نگاه نداشتی؟»
\par 37 اسحاق در جواب عیسو گفت: «اینک او را بر تو سرور ساختم، و همه برادرانش را غلامان او گردانیدم، و غله و شیره را رزق اودادم. پس الان‌ای پسر من، برای تو چه کنم؟»
\par 38 عیسو به پدر خود گفت: «ای پدر من، آیا همین یک برکت را داشتی؟ به من، به من نیز‌ای پدرم برکت بده!» و عیسو به آواز بلند بگریست.
\par 39 پدرش اسحاق در جواب او گفت: «اینک مسکن تو (دور) از فربهی زمین، و از شبنم آسمان از بالا خواهد بود.
\par 40 و به شمشیرت خواهی زیست، و برادر خود را بندگی خواهی کرد، وواقع خواهد شد که چون سر باز زدی، یوغ او را ازگردن خود خواهی انداخت.»
\par 41 و عیسو بسبب آن برکتی که پدرش به یعقوب داده بود، بر او بغض ورزید؛ و عیسو دردل خود گفت: «ایام نوحه گری برای پدرم نزدیک است، آنگاه برادر خود یعقوب را خواهم کشت.»
\par 42 و رفقه، از سخنان پسر بزرگ خود، عیسوآگاهی یافت. پس فرستاده، پسر کوچک خود، یعقوب را خوانده، بدو گفت: «اینک برادرت عیسو درباره تو خود را تسلی می‌دهد به اینکه تورا بکشد.
\par 43 پس الان‌ای پسرم سخن مرا بشنو وبرخاسته، نزد برادرم، لابان، به حران فرار کن.
\par 44 وچند روز نزد وی بمان، تا خشم برادرت برگردد.
\par 45 تا غضب برادرت از تو برگردد، و آنچه بدوکردی، فراموش کند. آنگاه می‌فرستم و تو را ازآنجا باز می‌آورم. چرا باید از شما هر دو در یک روز محروم شوم؟»و رفقه به اسحاق گفت: «بسبب دختران حت از جان خود بیزار شده‌ام. اگریعقوب زنی از دختران حت، مثل اینانی که دختران این زمینند بگیرد، مرا از حیات چه فایده خواهد بود.»
\par 46 و رفقه به اسحاق گفت: «بسبب دختران حت از جان خود بیزار شده‌ام. اگریعقوب زنی از دختران حت، مثل اینانی که دختران این زمینند بگیرد، مرا از حیات چه فایده خواهد بود.»
 
\chapter{28}

\par 1 و اسحاق، یعقوب را خوانده، او رابرکت داد و او را امر فرموده، گفت: «زنی از دختران کنعان مگیر.
\par 2 برخاسته، به فدان ارام، به خانه پدر مادرت، بتوئیل، برو و از آنجازنی از دختران لابان، برادر مادرت، برای خودبگیر.
\par 3 و خدای قادر مطلق تو را برکت دهد، و تورا بارور و کثیر سازد، تا از تو امتهای بسیار بوجودآیند.
\par 4 و برکت ابراهیم را به تو دهد، به تو و به ذریت تو با تو، تا وارث زمین غربت خود شوی، که خدا آن را به ابراهیم بخشید.»
\par 5 پس اسحاق، یعقوب را روانه نمود و به فدان ارام، نزد لابان بن بتوئیل ارامی، برادر رفقه، مادر یعقوب و عیسو، رفت.
\par 6 و اما عیسو چون دید که اسحاق یعقوب رابرکت داده، و او را به فدان ارام روانه نمود، تا ازآنجا زنی برای خود بگیرد، و در حین برکت دادن به وی امر کرده، گفته بود که «زنی از دختران کنعان مگیر، »
\par 7 و اینکه یعقوب، پدر و مادر خود را اطاعت نموده، به فدان ارام رفت.
\par 8 و چون عیسودید که دختران کنعان در نظر پدرش، اسحاق، بدند،
\par 9 پس عیسو نزد اسماعیل رفت، و محلت، دختر اسماعیل بن ابراهیم را که خواهر نبایوت بود، علاوه بر زنانی که داشت، به زنی گرفت.
\par 10 و اما یعقوب، از بئرشبع روانه شده، بسوی حران رفت.
\par 11 و به موضعی نزول کرده، در آنجاشب را بسر برد، زیرا که آفتاب غروب کرده بود ویکی از سنگهای آنجا را گرفته، زیر سر خود نهادو در همان جا بخسبید.
\par 12 و خوابی دید که ناگاه نردبانی بر زمین برپا شده، که سرش به آسمان می‌رسد، و اینک فرشتگان خدا، بر آن صعود ونزول می‌کنند.
\par 13 در حال، خداوند بر سر آن ایستاده، می‌گوید: «من هستم یهوه، خدای پدرت ابراهیم، و خدای اسحاق. این زمینی را که تو بر آن خفته‌ای به تو و به ذریت تو می‌بخشم.
\par 14 و ذریت تو مانند غبار زمین خواهند شد، و به مغرب ومشرق و شمال و جنوب منتشر خواهی شد، و ازتو و از نسل تو جمیع قبایل زمین برکت خواهندیافت.
\par 15 و اینک من با تو هستم، و تو را در هرجایی که روی، محافظت فرمایم تا تو را بدین زمین بازآورم، زیرا که تا آنچه به تو گفته‌ام، بجانیاورم، تو را رها نخواهم کرد.»
\par 16 پس یعقوب ازخواب بیدار شد و گفت: «البته یهوه در این مکان است و من ندانستم.»
\par 17 پس ترسان شده، گفت: «این چه مکان ترسناکی است! این نیست جز خانه خدا و این است دروازه آسمان.»
\par 18 بامدادان یعقوب برخاست و آن سنگی را که زیر سر خودنهاده بود، گرفت، و چون ستونی برپا داشت، وروغن بر سرش ریخت.
\par 19 و آن موضع را بیت ئیل نامید، لکن نام آن شهر اولا لوز بود.
\par 20 و یعقوب نذر کرده، گفت: «اگر خدا با من باشد، و مرا در این راه که می‌روم محافظت کند، و مرا نان دهد تابخورم، و رخت تا بپوشم،
\par 21 تا به خانه پدر خودبه سلامتی برگردم، هرآینه یهوه، خدای من خواهد بود.و این سنگی را که چون ستون برپاکردم، بیت الله شود، و آنچه به من بدهی، ده‌یک آن را به تو خواهم داد.»
\par 22 و این سنگی را که چون ستون برپاکردم، بیت الله شود، و آنچه به من بدهی، ده‌یک آن را به تو خواهم داد.»
 
\chapter{29}

\par 1 پس یعقوب روانه شد و به زمین بنی المشرق آمد.
\par 2 و دید که اینک درصحرا، چاهی است، و بر کناره‌اش سه گله گوسفند خوابیده، چونکه از آن چاه گله‌ها را آب می‌دادند، و سنگی بزرگ بر دهنه چاه بود.
\par 3 وچون همه گله‌ها جمع شدندی، سنگ را از دهنه چاه غلطانیده، گله را سیراب کردندی. پس سنگ را بجای خود، بر سر چاه باز گذاشتندی.
\par 4 یعقوب بدیشان گفت: «ای برادرانم از کجاهستید؟» گفتند: «ما از حرانیم.»
\par 5 بدیشان گفت: «لابان بن ناحور را می‌شناسید؟» گفتند: «می‌شناسیم.»
\par 6 بدیشان گفت: «بسلامت است؟» گفتند: «بسلامت، و اینک دخترش، راحیل، با گله او می‌آید.»
\par 7 گفت: «هنوز روز بلند است و وقت جمع کردن مواشی نیست، گله را آب دهید ورفته، بچرانید.»
\par 8 گفتند: «نمی توانیم، تا همه گله هاجمع شوند، و سنگ را از سر چاه بغلطانند، آنگاه گله را آب می‌دهیم.»
\par 9 و هنوز با ایشان در گفتگومی بود که راحیل، با گله پدر خود رسید. زیرا که آنها را چوپانی می‌کرد.
\par 10 اما چون یعقوب راحیل، دختر خالوی خود، لابان، و گله خالوی خویش، لابان را دید، یعقوب نزدیک شده، سنگ را از سر چاه غلطانید، و گله خالوی خویش، لابان را سیراب کرد.
\par 11 و یعقوب، راحیل را بوسید، وبه آواز بلند گریست.
\par 12 و یعقوب، راحیل را خبرداد که او برادر پدرش، و پسر رفقه است. پس دوان دوان رفته، پدر خود را خبر داد.
\par 13 و واقع شد که چون لابان، خبر خواهرزاده خود، یعقوب را شنید، به استقبال وی شتافت، و او را در بغل گرفته، بوسید و به خانه خود آورد، و او لابان را ازهمه این امور آگاهانید.
\par 14 لابان وی را گفت: «فی الحقیقه تو استخوان و گوشت من هستی.» و نزد وی مدت یک ماه توقف نمود.
\par 15 پس لابان، به یعقوب گفت: «آیاچون برادر من هستی، مرا باید مفت خدمت کنی؟ به من بگو که اجرت تو چه خواهد بود؟»
\par 16 ولابان را دو دختر بود، که نام بزرگتر، لیه و اسم کوچکتر، راحیل بود.
\par 17 و چشمان لیه ضعیف بود، و اما راحیل، خوب صورت و خوش منظربود.
\par 18 و یعقوب عاشق راحیل بود و گفت: «برای دختر کوچکت راحیل، هفت سال تو را خدمت می‌کنم.»
\par 19 لابان گفت: «او را به تو بدهم، بهتراست از آنکه به دیگری بدهم. نزد من بمان.»
\par 20 پس یعقوب برای راحیل هفت سال خدمت کرد. و بسبب محبتی که به وی داشت، در نظرش روزی چند نمود.
\par 21 و یعقوب به لابان گفت: «زوجه‌ام را به من بسپار، که روزهایم سپری شد، تا به وی درآیم.»
\par 22 پس لابان، همه مردمان آنجارا دعوت کرده، ضیافتی برپا نمود.
\par 23 و واقع شدکه هنگام شام، دختر خود، لیه را برداشته، او رانزد وی آورد، و او به وی درآمد.
\par 24 و لابان کنیزخود زلفه را، به دختر خود لیه، به کنیزی داد.
\par 25 صبحگاهان دید، که اینک لیه است! پس به لابان گفت: «این چیست که به من کردی؟ مگر برای راحیل نزد تو خدمت نکردم؟ چرا مرا فریب دادی؟»
\par 26 لابان گفت: «در ولایت ما چنین نمی کنند که کوچکتر را قبل از بزرگتر بدهند.
\par 27 هفته این را تمام کن و او را نیز به تو می‌دهیم، برای هفت سال دیگر که خدمتم بکنی.»
\par 28 پس یعقوب چنین کرد، و هفته او را تمام کرد، و دخترخود، راحیل را به زنی بدو داد.
\par 29 و لابان، کنیزخود، بلهه را به دختر خود، راحیل به کنیزی داد.
\par 30 و به راحیل نیز درآمد و او را از لیه بیشتردوست داشتی، و هفت سال دیگر خدمت وی کرد.
\par 31 و چون خداوند دید که لیه مکروه است، رحم او را گشود. ولی راحیل، نازاد ماند.
\par 32 و لیه حامله شده، پسری بزاد و او را روبین نام نهاد، زیراگفت: «خداوند مصیبت مرا دیده است. الان شوهرم مرا دوست خواهد داشت.»
\par 33 و بار دیگرحامله شده، پسری زایید و گفت: «چونکه خداوند شنید که من مکروه هستم، این را نیز به من بخشید.» پس او را شمعون نامید.
\par 34 و باز آبستن شده، پسری زایید و گفت: «اکنون این مرتبه شوهرم با من خواهد پیوست، زیرا که برایش سه پسر زاییدم.» از این سبب او را لاوی نام نهاد.وبار دیگر حامله شده، پسری زایید و گفت: «این مرتبه خداوند را حمد می‌گویم.» پس او را یهودانامید. آنگاه از زاییدن بازایستاد.
\par 35 وبار دیگر حامله شده، پسری زایید و گفت: «این مرتبه خداوند را حمد می‌گویم.» پس او را یهودانامید. آنگاه از زاییدن بازایستاد.
 
\chapter{30}

\par 1 یعقوب، اولادی نزایید، راحیل برخواهر خود حسد برد. و به یعقوب گفت: «پسران به من بده والا می‌میرم.»
\par 2 آنگاه غضب یعقوب برراحیل افروخته شد و گفت: «مگر من به‌جای خدا هستم که بار رحم را از تو باز داشته است؟»
\par 3 گفت: «اینک کنیز من، بلهه! بدو درآ تا بر زانویم بزاید، و من نیز از او اولاد بیابم.»
\par 4 پس کنیز خود، بلهه را به یعقوب به زنی داد. و او به وی درآمد.
\par 5 وبلهه آبستن شده، پسری برای یعقوب زایید.
\par 6 وراحیل گفت: «خدا مرا داوری کرده است، و آوازمرا نیز شنیده، و پسری به من عطا فرموده است.» پس او را دان نام نهاد.
\par 7 و بلهه، کنیز راحیل، بازحامله شده، پسر دومین برای یعقوب زایید.
\par 8 وراحیل گفت: «به کشتیهای خدا با خواهر خودکشتی گرفتم و غالب آمدم.» و او را نفتالی نام نهاد.
\par 9 و اما لیه چون دید که از زاییدن باز‌مانده بود، کنیز خود زلفه را برداشته، او را به یعقوب به زنی داد.
\par 10 و زلفه، کنیز لیه، برای یعقوب پسری زایید.
\par 11 و لیه گفت: «به سعادت!» پس او را جادنامید.
\par 12 و زلفه، کنیز لیه، پسر دومین برای یعقوب زایید.
\par 13 و لیه گفت: «به خوشحالی من! زیرا که دختران، مرا خوشحال خواهند خواند.» واو را اشیر نام نهاد.
\par 14 و در ایام درو گندم، روبین رفت و مهرگیاهها در صحرا یافت و آنها را نزدمادر خود لیه، آورد. پس راحیل به لیه گفت: «ازمهرگیاههای پسر خود به من بده.»
\par 15 وی را گفت: «آیا کم است که شوهر مرا گرفتی و مهر گیاه پسرمرا نیز می‌خواهی بگیری؟» راحیل گفت: «امشب به عوض مهر گیاه پسرت، با تو بخوابد.»
\par 16 ووقت عصر، چون یعقوب از صحرا می‌آمد، لیه به استقبال وی بیرون شده، گفت: «به من درآ، زیرا که تو را به مهرگیاه پسر خود اجیر کردم.» پس آنشب با وی همخواب شد. 
\par 17 و خدا، لیه را مستجاب فرمود که آبستن شده، پسر پنجمین برای یعقوب زایید.
\par 18 و لیه گفت: «خدا اجرت به من داده است، زیرا کنیز خود را به شوهر خود دادم.» و او را یساکار نام نهاد.
\par 19 و بار دیگر لیه حامله شده، پسر ششمین برای یعقوب زایید.
\par 20 و لیه گفت: «خدا عطای نیکو به من داده است. اکنون شوهرم با من زیست خواهد کرد، زیرا که شش پسر برای او زاییدم.» پس او را زبولون نامید.
\par 21 و بعد از آن دختری زایید، و او را دینه نام نهاد.
\par 22 پس خداراحیل را بیاد آورد، و دعای او را اجابت فرموده، خدا رحم او را گشود.
\par 23 و آبستن شده، پسری بزاد و گفت: «خدا ننگ مرا برداشته است.»
\par 24 و اورا یوسف نامیده، گفت: «خداوند پسری دیگربرای من مزید خواهد کرد.»
\par 25 و واقع شد که چون راحیل، یوسف رازایید، یعقوب به لابان گفت: «مرا مرخص کن، تا به مکان و وطن خویش بروم.
\par 26 زنان و فرزندان مراکه برای ایشان، تو را خدمت کرده‌ام به من واگذار، تا بروم زیرا خدمتی که به تو کردم، تو می‌دانی.»
\par 27 لابان وی را گفت: «کاش که منظور نظر توباشم، زیرا تفالا یافته‌ام که بخاطر تو، خداوند مرابرکت داده است.»
\par 28 و گفت: «اجرت خود را برمن معین کن تا آن را به تو دهم.»
\par 29 وی را گفت: «خدمتی که به تو کرده‌ام، خود می‌دانی، ومواشی ات چگونه نزد من بود.
\par 30 زیرا قبل ازآمدن من، مال تو قلیل بود، و به نهایت زیاد شد، وبعد از آمدن من، خداوند تو را برکت داده است. واکنون من نیز تدارک خانه خود را کی ببینم؟»
\par 31 گفت: «پس تو را چه بدهم؟» یعقوب گفت: «چیزی به من مده، اگر این کار را برای من بکنی، بار دیگر شبانی و پاسبانی گله تو را خواهم نمود.
\par 32 امروز در تمامی گله تو گردش می‌کنم، و هرمیش پیسه و ابلق و هر میش سیاه را از میان گوسفندان، و ابلق‌ها و پیسه‌ها را از بزها، جدامی سازم، و آن، اجرت من خواهد بود.
\par 33 و درآینده عدالت من، بر من شهادت خواهد داد، وقتی که بیایی تا اجرت مرا پیش خود ببینی، آنچه ازبزها، پیسه و ابلق، و آنچه از گوسفندان، سیاه نباشد، نزد من به دزدی شمرده شود.»
\par 34 لابان گفت: «اینک موافق سخن تو باشد.»
\par 35 و در همان روز، بزهای نرینه مخطط و ابلق، و همه ماده بزهای پیسه و ابلق، یعنی هر‌چه سفیدی در آن بود، و همه گوسفندان سیاه را جدا کرده، به‌دست پسران خود سپرد.
\par 36 و در میان خود و یعقوب، سه روز راه، مسافت گذارد. و یعقوب باقی گله لابان را شبانی کرد.
\par 37 و یعقوب چوبهای تر و تازه از درخت کبوده و بادام و چنار برای خود گرفت، و خطهای سفید در آنها کشید، و سفیدی را که در چوبها بود، ظاهر کرد.
\par 38 و وقتی که گله‌ها، برای آب خوردن می‌آمدند، آن چوبهایی را که خراشیده بود، درحوضها و آبخورها پیش گله‌ها می‌نهاد، تا چون برای نوشیدن بیایند، حمل بگیرند.
\par 39 پس گله هاپیش چوبها بارآور می‌شدند، و بزهای مخطط وپیسه و ابلق می‌زاییدند.
\par 40 و یعقوب، بزها را جداکرد، و روی گله‌ها را بسوی هر مخطط و سیاه درگله لابان واداشت، و گله های خود را جدا کرد و باگله لابان نگذاشت.
\par 41 و هرگاه حیوان های تنومندحمل می‌گرفتند، یعقوب چوبها را پیش آنها درآبخورها می‌نهاد، تا در میان چوبها حمل گیرند.
\par 42 و هر گاه حیوانات ضعیف بودند، آنها رانمی گذاشت، پس ضعیف‌ها از آن لابان، وتنومندها از آن یعقوب شدند.و آن مرد بسیارترقی نمود، و گله های بسیار و کنیزان و غلامان وشتران و حماران بهم رسانید.
\par 43 و آن مرد بسیارترقی نمود، و گله های بسیار و کنیزان و غلامان وشتران و حماران بهم رسانید.
 
\chapter{31}

\par 1 و سخنان پسران لابان را شنید که می گفتند: «یعقوب همه مایملک پدر مارا گرفته است، و از اموال پدر ما تمام این بزرگی رابهم رسانیده.»
\par 2 و یعقوب روی لابان را دید که اینک مثل سابق با او نبود.
\par 3 و خداوند به یعقوب گفت: «به زمین پدرانت و به مولد خویش مراجعت کن و من با تو خواهم بود.»
\par 4 پس یعقوب فرستاده، راحیل و لیه را به صحرا نزد گله خودطلب نمود.
\par 5 و بدیشان گفت: «روی پدر شما رامی بینم که مثل سابق با من نیست، لیکن خدای پدرم با من بوده است.
\par 6 و شما می‌دانید که به تمام قوت خود پدر شما را خدمت کرده‌ام.
\par 7 و پدرشما مرا فریب داده، ده مرتبه اجرت مرا تبدیل نمود ولی خدا او را نگذاشت که ضرری به من رساند.
\par 8 هر گاه می‌گفت اجرت تو پیسه‌ها باشد، تمام گله‌ها پیسه می‌آوردند، و هر گاه گفتی اجرت تو مخطط باشد، همه گله‌ها مخطط می‌زاییدند.
\par 9 پس خدا اموال پدر شما را گرفته، به من داده است.»
\par 10 و واقع شد هنگامی که گله‌ها حمل می‌گرفتند که در خوابی چشم خود را باز کرده، دیدم اینک قوچهایی که با میشها جمع می‌شدند، مخطط و پیسه و ابلق بودند.
\par 11 و فرشته خدا درخواب به من گفت: «ای یعقوب!» گفتم: «لبیک.»
\par 12 گفت: «اکنون چشمان خود را باز کن و بنگر که همه قوچهایی که با میشها جمع می‌شوند، مخططو پیسه و ابلق هستند زیرا که آنچه لابان به تو کرده است، دیده‌ام.
\par 13 من هستم خدای بیت ئیل، جایی که ستون را مسح کردی و با من نذر نمودی. الان برخاسته، از این زمین روانه شده، به زمین مولد خویش مراجعت نما.»
\par 14 راحیل و لیه در جواب وی گفتند: «آیا در خانه پدر ما، برای ما بهره یامیراثی باقیست؟
\par 15 مگر نزد او چون بیگانگان محسوب نیستیم، زیرا که ما را فروخته است و نقدما را تمام خورده.
\par 16 زیرا تمام دولتی را که خدااز پدر ما گرفته است، از آن ما و فرزندان ماست، پس اکنون آنچه خدا به تو گفته است، بجا آور.»
\par 17 آنگاه یعقوب برخاسته، فرزندان و زنان خود را بر شتران سوار کرد،
\par 18 و تمام مواشی واموال خود را که اندوخته بود، یعنی مواشی حاصله خود را که در فدان ارام حاصل ساخته بود، برداشت تا نزد پدر خود اسحاق به زمین کنعان برود.
\par 19 و اما لابان برای پشم بریدن گله خود رفته بود و راحیل، بتهای پدر خود را دزدید.
\par 20 و یعقوب لابان ارامی را فریب داد، چونکه او رااز فرار کردن خود آگاه نساخت.
\par 21 پس با آنچه داشت، بگریخت و برخاسته، از نهر عبور کرد ومتوجه جبل جلعاد شد.
\par 22 در روز سوم، لابان را خبر دادند که یعقوب فرار کرده است.
\par 23 پس برادران خویش را با خودبرداشته، هفت روز راه در عقب او شتافت، تا درجبل جلعاد بدو پیوست.
\par 24 شبانگاه، خدا درخواب بر لابان ارامی ظاهر شده، به وی گفت: «باحذر باش که به یعقوب نیک یا بد نگویی.»
\par 25 پس لابان به یعقوب دررسید و یعقوب خیمه خود رادر جبل زده بود، و لابان با برادران خود نیز درجبل جلعاد فرود آمدند.
\par 26 و لابان به یعقوب گفت: «چه کردی که مرا فریب دادی و دخترانم رامثل اسیران، شمشیر برداشته، رفتی؟
\par 27 چرامخفی فرار کرده، مرا فریب دادی و مرا آگاه نساختی تا تو را با شادی و نغمات و دف و بربطمشایعت نمایم؟
\par 28 و مرا نگذاشتی که پسران ودختران خود ببوسم؛ الحال ابلهانه حرکتی نمودی.
\par 29 در قوت دست من است که به شمااذیت رسانم. لیکن خدای پدر شما دوش به من خطاب کرده، گفت: «با حذر باش که به یعقوب نیک یا بد نگویی.»
\par 30 و الان چونکه به خانه پدرخود رغبتی تمام داشتی البته رفتنی بودی و لکن خدایان مرا چرا دزدیدی؟»
\par 31 یعقوب در جواب لابان گفت: «سبب این بود که ترسیدم و گفتم شایددختران خود را از من به زور بگیری،
\par 32 و اما نزدهر‌که خدایانت را بیابی، او زنده نماند. در حضوربرادران ما، آنچه از اموال تو نزد ما باشد، مشخص کن و برای خود بگیر.» زیرا یعقوب ندانست که راحیل آنها را دزدیده است.
\par 33 پس لابان به خیمه یعقوب و به خیمه لیه وبه خیمه دو کنیز رفت و نیافت، و از خیمه لیه بیرون آمده، به خیمه راحیل درآمد.
\par 34 اما راحیل بتها را گرفته، زیر جهاز شتر نهاد و بر آن بنشست و لابان تمام خیمه را جست وجو کرده، چیزی نیافت.
\par 35 او به پدر خود گفت: «بنظر آقایم بدنیاید که در حضورت نمی توانم برخاست، زیرا که عادت زنان بر من است.» پس تجسس نموده، بتهارا نیافت.
\par 36 آنگاه یعقوب خشمگین شده، با لابان منازعت کرد. و یعقوب در جواب لابان گفت: «تقصیر و خطای من چیست که بدین گرمی مراتعاقب نمودی؟
\par 37 الان که تمامی اموال مراتفتیش کردی، از همه اسباب خانه خود چه یافته‌ای، اینجا نزد برادران من و برادران خودبگذار تا در میان من و تو انصاف دهند.
\par 38 در این بیست سال که من با تو بودم، میشها و بزهایت حمل نینداختند و قوچهای گله تو را نخوردم.
\par 39 دریده شده‌ای را پیش تو نیاوردم؛ خود تاوان آن را می‌دادم و آن را از دست من می‌خواستی، خواه دزدیده شده در روز و خواه دزدیده شده درشب.
\par 40 چنین بودم که گرما در روز و سرما درشب، مرا تلف می‌کرد، و خواب از چشمانم می‌گریخت.
\par 41 بدینطور بیست سال در خانه ات بودم، چهارده سال برای دو دخترت خدمت توکردم، و شش سال برای گله ات، و اجرت مرا ده مرتبه تغییر دادی.
\par 42 و اگر خدای پدرم، خدای ابراهیم، و هیبت اسحاق با من نبودی، اکنون نیزمرا تهی‌دست روانه می‌نمودی. خدا مصیبت مراو مشقت دستهای مرا دید و دوش، تو را توبیخ نمود.»
\par 43 لابان در جواب یعقوب گفت: «این دختران، دختران منند و این پسران، پسران من واین گله، گله من و آنچه می‌بینی از آن من است. پس الیوم، به دختران خودم و به پسرانی که زاییده‌اند چه توانم کرد؟
\par 44 اکنون بیا تا من و توعهد ببندیم که در میان من و تو شهادتی باشد.»
\par 45 پس یعقوب سنگی گرفته، آن را ستونی برپانمود.
\par 46 و یعقوب برادران خود را گفت: «سنگهاجمع کنید.» پس سنگها جمع کرده، توده‌ای ساختند و در آنجا بر توده غذا خوردند.
\par 47 ولابان آن را «یجرسهدوتا» نامید ولی یعقوب آن راجلعید خواند.
\par 48 و لابان گفت: «امروز این توده در میان من و تو شهادتی است.» از این سبب آن را «جلعید» نامید.
\par 49 و مصفه نیز، زیرا گفت: «خداوند در میان من و تو دیده بانی کند وقتی که ازیکدیگر غایب شویم.
\par 50 اگر دختران مرا آزارکنی، و سوای دختران من، زنان دیگر بگیری، هیچکس در میان ما نخواهد بود. آگاه باش، خدادر میان من و تو شاهد است.»
\par 51 و لابان به یعقوب گفت: «اینک این توده و اینک این ستونی که در میان خود و تو برپا نمودم.
\par 52 این توده شاهد است و این ستون شاهد است که من از این توده بسوی تو نگذرم و تو از این توده و از این ستون به قصدبدی بسوی من نگذری.
\par 53 خدای ابراهیم وخدای ناحور و خدای پدر ایشان در میان ماانصاف دهند.» و یعقوب قسم خورد به هیبت پدرخود اسحاق.
\par 54 آنگاه یعقوب در آن کوه، قربانی گذرانید، و برادران خود را به نان خوردن دعوت نمود، و غذا خوردند و در کوه، شب را بسر بردند.بامدادان لابان برخاسته، پسران و دختران خودرا بوسید و ایشان را برکت داد و لابان روانه شده، به مکان خویش مراجعت نمود.
\par 55 بامدادان لابان برخاسته، پسران و دختران خودرا بوسید و ایشان را برکت داد و لابان روانه شده، به مکان خویش مراجعت نمود.
 
\chapter{32}

\par 1 و یعقوب راه خود را پیش گرفت وفرشتگان خدا به وی برخوردند.
\par 2 وچون یعقوب، ایشان را دید، گفت: «این لشکرخداست!» و آن موضع را «محنایم» نامید.
\par 3 پس یعقوب، قاصدان پیش روی خود نزدبرادر خویش، عیسو به دیار سعیر به بلاد ادوم فرستاد.
\par 4 و ایشان را امر فرموده، گفت: «به آقایم، عیسو چنین گویید که بنده تو یعقوب عرض می‌کند با لابان ساکن شده، تاکنون توقف نمودم،
\par 5 و برای من گاوان و الاغان و گوسفندان و غلامان و کنیزان حاصل شده است، و فرستادم تا آقای خود را آگاهی دهم و در نظرت التفات یابم.»
\par 6 پس قاصدان، نزد یعقوب برگشته، گفتند: «نزدبرادرت، عیسو رسیدیم و اینک با چهارصد نفر به استقبال تو می‌آید.»
\par 7 آنگاه یعقوب به نهایت ترسان و متحیر شده، کسانی را که با وی بودند با گوسفندان و گاوان و شتران به دو دسته تقسیم نمود 
\par 8 و گفت: «هر گاه عیسو به‌دسته اول برسد وآنها را بزند، همانا دسته دیگر رهایی یابد.»
\par 9 و یعقوب گفت: «ای خدای پدرم، ابراهیم وخدای پدرم، اسحاق، ای یهوه که به من گفتی به زمین و به مولد خویش برگرد و با تو احسان خواهم کرد.
\par 10 کمتر هستم از جمیع لطفها و ازهمه وفایی که با بنده خود کرده‌ای زیرا که باچوبدست خود از این اردن عبور کردم و الان (مالک ) دو گروه شده‌ام.
\par 11 اکنون مرا از دست برادرم، از دست عیسو رهایی ده زیرا که من از اومی ترسم، مبادا بیاید و مرا بزند، یعنی مادر وفرزندان را،
\par 12 و تو گفتی هرآینه با تو احسان کنم و ذریت تو را مانند ریگ دریا سازم که از کثرت، آن را نتوان شمرد.»
\par 13 پس آن شب را در آنجا بسر برد و از آنچه بدستش آمد، ارمغانی برای برادر خود، عیسوگرفت:
\par 14 دویست ماده بز با بیست بز نر ودویست میش با بیست قوچ،
\par 15 و سی شتر شیرده با بچه های آنها و چهل ماده گاو با ده گاو نر وبیست ماده الاغ با ده کره.
\par 16 و آنها را دسته دسته، جداجدا به نوکران خود سپرد و به بندگان خود گفت: «پیش روی من عبور کنید و در میان دسته‌ها فاصله بگذارید.»
\par 17 و نخستین را امر فرموده، گفت که «چون برادرم عیسو به تو رسد و از تو پرسیده، بگوید: از آن کیستی و کجا می‌روی و اینها که پیش توست ازآن کیست؟
\par 18 بدو بگو: این از آن بنده ات، یعقوب است، و پیشکشی است که برای آقایم، عیسوفرستاده شده است و اینک خودش نیز در عقب ماست.»
\par 19 و همچنین دومین و سومین و همه کسانی را که از عقب آن دسته‌ها می‌رفتند، امرفرموده، گفت: «چون به عیسو برسید، بدو چنین گویید،
\par 20 و نیز گویید: اینک بنده ات، یعقوب درعقب ماست.» زیرا گفت: «غضب او را بدین ارمغانی که پیش من می‌رود، فرو خواهم نشانید، وبعد چون روی او را بینم، شاید مرا قبول فرماید.»
\par 21 پس ارمغان، پیش از او عبور کرد و او آن شب را در خیمه گاه بسر برد.
\par 22 و شبانگاه، خودش برخاست و دو زوجه ودو کنیز و یازده پسر خویش را برداشته، ایشان رااز معبر یبوق عبور داد.
\par 23 ایشان را برداشت و ازآن نهر عبور داد، و تمام مایملک خود را نیز عبورداد.
\par 24 و یعقوب تنها ماند و مردی با وی تا طلوع فجر کشتی می‌گرفت.
\par 25 و چون او دید که بر وی غلبه نمی یابد، کف ران یعقوب را لمس کرد، وکف ران یعقوب در کشتی گرفتن با او فشرده شد.
\par 26 پس گفت: «مرا رها کن زیرا که فجرمی شکافد.» گفت: «تا مرا برکت ندهی تو را رهانکنم.»
\par 27 به وی گفت: «نام تو چیست؟» گفت: «یعقوب.»
\par 28 گفت: «از این پس نام تو یعقوب خوانده نشود بلکه اسرائیل، زیرا که با خدا و باانسان مجاهده کردی و نصرت یافتی.»
\par 29 ویعقوب از او سوال کرده، گفت: «مرا از نام خودآگاه ساز.» گفت: «چرا اسم مرا می‌پرسی؟» و او رادر آنجا برکت داد.
\par 30 و یعقوب آن مکان را «فنیئیل» نامیده، گفت: «زیرا خدا را روبرو دیدم وجانم رستگار شد.»
\par 31 و چون از «فنوئیل» گذشت، آفتاب بر وی طلوع کرد، و بر ران خودمی لنگید.از این سبب بنی‌اسرائیل تا امروز عرق النساء را که در کف ران است، نمی خورند، زیرا کف ران یعقوب را در عرق النسا لمس کرد.
\par 32 از این سبب بنی‌اسرائیل تا امروز عرق النساء را که در کف ران است، نمی خورند، زیرا کف ران یعقوب را در عرق النسا لمس کرد.
 
\chapter{33}

\par 1 پس یعقوب چشم خود را باز کرده، دید که اینک عیسو می‌آید و چهارصدنفر با او. آنگاه فرزندان خود را به لیه و راحیل و دوکنیز تقسیم کرد.
\par 2 و کنیزان را با فرزندان ایشان پیش داشت و لیه را با فرزندانش در عقب ایشان، وراحیل و یوسف را آخر.
\par 3 و خود در‌پیش ایشان رفته، هفت مرتبه رو به زمین نهاد تا به برادر خودرسید.
\par 4 اما عیسو دوان دوان به استقبال او آمد واو را در بر‌گرفته، به آغوش خود کشید، و او رابوسید و هر دو بگریستند.
\par 5 و چشمان خود را بازکرده، زنان و فرزندان را بدید و گفت: «این همراهان تو کیستند؟»
\par 6 آنگاه کنیزان با فرزندان ایشان نزدیک شده، تعظیم کردند.
\par 7 و لیه با فرزندانش نزدیک شده، تعظیم کردند. پس یوسف و راحیل نزدیک شده، تعظیم کردند.
\par 8 و او گفت: «از تمامی این گروهی که بدان برخوردم، چه مقصود داری؟» گفت: «تا در نظر آقای خود التفات یابم.»
\par 9 عیسوگفت: «ای برادرم مرا بسیار است، مال خود را نگاه دار.»
\par 10 یعقوب گفت: «نی، بلکه اگر در نظرت التفات یافته‌ام، پیشکش مرا از دستم قبول فرما، زیرا که روی تو را دیدم مثل دیدن روی خدا، ومرا منظور داشتی.
\par 11 پس هدیه مرا که به حضورت آورده شد بپذیر، زیرا خدا به من احسان فرموده است و همه‌چیز دارم.» پس او را الحاح نمود تا پذیرفت.
\par 12 گفت: «کوچ کرده، برویم و من همراه تو می‌آیم.»
\par 13 گفت: «آقایم آگاه است که اطفال نازکند وگوسفندان و گاوان شیرده نیز با من است، و اگرآنها را یک روز برانند، تمامی گله می‌میرند؛
\par 14 پس آقایم پیشتر از بنده خود برود و من موافق قدم مواشی که دارم. و به حسب قدم اطفال، آهسته سفر می‌کنم، تا نزد آقای خود به سعیربرسم.
\par 15 عیسو گفت: «پس بعضی از این کسانی را که با منند نزد تو می‌گذارم.» گفت: «چه لازم است، فقط در نظر آقای خود التفات بیابم.»
\par 16 در همان روز عیسو راه خود را پیش گرفته، به سعیرمراجعت کرد.
\par 17 و اما یعقوب به سکوت سفرکرد و خانه‌ای برای خود بنا نمود و برای مواشی خود سایبانها ساخت. از این سبب آن موضع به «سکوت» نامیده شد.
\par 18 پس چون یعقوب از فدان ارام مراجعت کرد، به سلامتی به شهر شکیم، در زمین کنعان آمد، و در مقابل شهر فرود آمد.
\par 19 و آن قطعه زمینی را که خیمه خود را در آن زده بود ازبنی حمور، پدر شکیم، به صد قسیط خرید.ومذبحی در آنجا بنا نمود و آن را ایل الوهی اسرائیل نامید.
\par 20 ومذبحی در آنجا بنا نمود و آن را ایل الوهی اسرائیل نامید.
 
\chapter{34}

\par 1 پس دینه، دختر لیه، که او را برای یعقوب زاییده بود، برای دیدن دختران آن ملک بیرون رفت.
\par 2 و چون شکیم بن حمورحوی که رئیس آن زمین بود، او را بدید، او را بگرفت و با او همخواب شده، وی را بی‌عصمت ساخت.
\par 3 و دلش به دینه، دختر یعقوب، بسته شده، عاشق آن دختر گشت، و سخنان دل آویز به آن دختر گفت.
\par 4 و شکیم به پدر خود، حمورخطاب کرده، گفت: «این دختر را برای من به زنی بگیر.»
\par 5 و یعقوب شنید که دخترش دینه رابی عصمت کرده است. و چون پسرانش با مواشی او در صحرا بودند، یعقوب سکوت کرد تا ایشان بیایند.
\par 6 و حمور، پدر شکیم نزد یعقوب بیرون آمد تا به وی سخن گوید.
\par 7 و چون پسران یعقوب این را شنیدند، از صحرا آمدند و غضبناک شده، خشم ایشان به شدت افروخته شد، زیرا که بادختر یعقوب همخواب شده، قباحتی دراسرائیل نموده بود و این عمل، ناکردنی بود.
\par 8 پس حمور ایشان را خطاب کرده، گفت: «دل پسرم شکیم شیفته دختر شماست؛ او را به وی به زنی بدهید.
\par 9 و با ما مصاهرت نموده، دختران خود را به ما بدهید و دختران ما را برای خودبگیرید.
\par 10 و با ما ساکن شوید و زمین از آن شماباشد. در آن بمانید و تجارت کنید و در آن تصرف کنید.»
\par 11 و شکیم به پدر و برادران آن دختر گفت: «در نظر خود مرا منظور بدارید و آنچه به من بگویید، خواهم داد.
\par 12 مهر و پیشکش هر قدرزیاده از من بخواهید، آنچه بگویید، خواهم دادفقط دختر را به زنی به من بسپارید.»
\par 13 اما پسران یعقوب در جواب شکیم و پدرش حمور به مکرسخن‌گفتند زیرا خواهر ایشان، دینه رابی عصمت کرده بود.
\par 14 پس بدیشان گفتند: «این کار را نمی توانیم کرد که خواهر خود را به شخصی نامختون بدهیم، چونکه این برای ما ننگ است.
\par 15 لکن بدین شرط با شما همداستان می‌شویم اگر چون ما بشوید، که هر ذکوری ازشما مختون گردد.
\par 16 آنگاه دختران خود را به شما دهیم و دختران شما را برای خود گیریم و باشما ساکن شده، یک قوم شویم.
\par 17 اما اگر سخن ما را اجابت نکنید و مختون نشوید، دختر خود رابرداشته، از اینجا کوچ خواهیم کرد.»
\par 18 و سخنان ایشان بنظر حمور و بنظر شکیم بن حمور پسند افتاد.
\par 19 و آن جوان در کردن این کار تاخیر ننمود، زیرا که شیفته دختر یعقوب بود، و او از همه اهل خانه پدرش گرامی تر بود.
\par 20 پس حمور و پسرش شکیم به دروازه شهرخود آمده، مردمان شهر خود را خطاب کرده، گفتند:
\par 21 «این مردمان با ما صلاح اندیش هستند، پس در این زمین ساکن بشوند، و در آن تجارت کنند. اینک زمین از هر طرف برای ایشان وسیع است؛ دختران ایشان را به زنی بگیریم و دختران خود را بدیشان بدهیم.
\par 22 فقط بدین شرط ایشان باما متفق خواهند شد تا با ما ساکن شده، یک قوم شویم که هر ذکوری از ما مختون شود، چنانکه ایشان مختونند.
\par 23 آیا مواشی ایشان و اموال ایشان و هر حیوانی که دارند، از آن ما نمی شود؟ فقط با ایشان همداستان شویم تا با ما ساکن شوند.»
\par 24 پس همه کسانی که به دروازه شهر اودرآمدند، به سخن حمور و پسرش شکیم رضادادند، و هر ذکوری از آنانی که به دروازه شهر اودرآمدند، مختون شدند.
\par 25 و در روز سوم چون دردمند بودند، دو پسر یعقوب، شمعون و لاوی، برادران دینه، هر یکی شمشیر خود را گرفته، دلیرانه بر شهر آمدند و همه مردان را کشتند.
\par 26 وحمور و پسرش شکیم را به دم شمشیر کشتند، ودینه را از خانه شکیم برداشته، بیرون آمدند.
\par 27 وپسران یعقوب بر کشتگان آمده، شهر را غارت کردند، زیرا خواهر ایشان را بی‌عصمت کرده بودند.
\par 28 و گله‌ها و رمه‌ها و الاغها و آنچه در شهرو آنچه در صحرا بود، گرفتند.
\par 29 و تمامی اموال ایشان و همه اطفال و زنان ایشان را به اسیری بردند. و آنچه در خانه‌ها بود تاراج کردند.
\par 30 پس یعقوب به شمعون و لاوی گفت: «مرا به اضطراب انداختید، و مرا نزد سکنه این زمین، یعنی کنعانیان و فرزیان مکروه ساختید، و من در شماره قلیلم، همانا بر من جمع شوند و مرا بزنند و من با خانه‌ام هلاک شوم.»گفتند: «آیا او با خواهر ما مثل فاحشه عمل کند؟»
\par 31 گفتند: «آیا او با خواهر ما مثل فاحشه عمل کند؟»
 
\chapter{35}

\par 1 و خدا به یعقوب گفت: «برخاسته، به بیت ئیل برآی، و در آنجا ساکن شو وآنجا برای خدایی که بر تو ظاهر شد، وقتی که ازحضور برادرت، عیسو فرار کردی، مذبحی بساز.»
\par 2 پس یعقوب به اهل خانه و همه کسانی که با وی بودند، گفت: «خدایان بیگانه‌ای را که درمیان شماست، دور کنید، و خویشتن را طاهرسازید، و رختهای خود را عوض کنید.
\par 3 تابرخاسته، به بیت ئیل برویم و آنجا برای آن خدایی که در روز تنگی من، مرا اجابت فرمود ودر راهی که رفتم با من می‌بود، مذبحی بسازم.»
\par 4 آنگاه همه خدایان بیگانه را که در دست ایشان بود، به یعقوب دادند، با گوشواره هایی که در گوشهای ایشان بود. و یعقوب آنها را زیر بلوطی که در شکیم بود دفن کرد.
\par 5 پس کوچ کردند، وخوف خدا بر شهرهای گرداگرد ایشان بود، که بنی یعقوب را تعاقب نکردند.
\par 6 و یعقوب به لوزکه در زمین کنعان واقع است، و همان بیت ئیل باشد، رسید. او با تمامی قوم که با وی بودند.
\par 7 ودر آنجا مذبحی بنا نمود و آن مکان را «ایل بیت ئیل» نامید. زیرا در آنجا خدا بر وی ظاهر شده بود، هنگامی که از حضور برادر خودمی گریخت.
\par 8 و دبوره دایه رفقه مرد. و او را زیر درخت بلوط تحت بیت ئیل دفن کردند، و آن را «الون باکوت» نامید.
\par 9 و خدا بار دیگر بر یعقوب ظاهر شد، وقتی که از فدان ارام آمد، و او را برکت داد.
\par 10 و خدا به وی گفت: «نام تو یعقوب است اما بعد از این نام تویعقوب خوانده نشود، بلکه نام تو اسرائیل خواهدبود.» پس او را اسرائیل نام نهاد.
\par 11 و خدا وی راگفت: «من خدای قادر مطلق هستم. بارور و کثیرشو. امتی و جماعتی از امتها از تو بوجود آیند، واز صلب تو پادشاهان پدید شوند.
\par 12 و زمینی که به ابراهیم و اسحاق دادم، به تو دهم؛ و به ذریت بعد از تو، این زمین را خواهم داد.» 
\par 13 پس خدا ازآنجایی که با وی سخن گفت، از نزد وی صعودنمود.
\par 14 و یعقوب ستونی برپا داشت، در جایی که باوی تکلم نمود. ستونی از سنگ و هدیه‌ای ریختنی بر آن ریخت، و آن را به روغن تدهین کرد.
\par 15 پس یعقوب آن مکان را، که خدا با وی در آنجا سخن گفته بود، «بیت ئیل» نامید.
\par 16 پس، از «بیت ئیل» کوچ کردند. و چون اندک مسافتی مانده بود که به افراته برسند، راحیل راوقت وضع حمل رسید، و زاییدنش دشوار شد.
\par 17 و چون زاییدنش دشوار بود، قابله وی را گفت: «مترس زیرا که این نیز برایت پسر است.»
\par 18 و درحین جان کندن، زیرا که مرد، پسر را «بن اونی» نام نهاد، لکن پدرش وی را «بن یامین» نامید.
\par 19 پس راحیل وفات یافت، و در راه افراته که بیت لحم باشد، دفن شد.
\par 20 و یعقوب بر قبر وی ستونی نصب کرد، که آن تا امروز ستون قبرراحیل است.
\par 21 پس اسرائیل کوچ کرد، و خیمه خود را بدان طرف برج عیدر زد.
\par 22 و در حین سکونت اسرائیل در آن زمین، روبین رفته، با کنیزپدر خود، بلهه، همخواب شد. و اسرائیل این راشنید. و بنی یعقوب دوازده بودند:
\par 23 پسران لیه: روبین نخست زاده یعقوب و شمعون و لاوی ویهودا و یساکار و زبولون.
\par 24 و پسران راحیل: یوسف و بن یامین.
\par 25 و پسران بلهه کنیز راحیل: دان و نفتالی.
\par 26 و پسران زلفه، کنیز لیه: جاد واشیر. اینانند پسران یعقوب، که در فدان ارام برای او متولد شدند.
\par 27 و یعقوب نزد پدر خود، اسحاق، در ممری آمد، به قریه اربع که حبرون باشد، جایی که ابراهیم و اسحاق غربت گزیدند.
\par 28 و عمراسحاق صد و هشتاد سال بود.و اسحاق جان سپرد و مرد، و پیر و سالخورده به قوم خویش پیوست. و پسرانش عیسو و یعقوب او را دفن کردند.
\par 29 و اسحاق جان سپرد و مرد، و پیر و سالخورده به قوم خویش پیوست. و پسرانش عیسو و یعقوب او را دفن کردند.
 
\chapter{36}

\par 1 و پیدایش عیسو که ادوم باشد، این است:
\par 2 عیسو زنان خود را از دختران کنعانیان گرفت: یعنی عاده دختر ایلون حتی، واهولیبامه دختر عنی، دختر صبعون حوی،
\par 3 وبسمه دختر اسماعیل، خواهر نبایوت.
\par 4 و عاده، الیفاز را برای عیسو زایید، و بسمه، رعوئیل رابزاد،
\par 5 و اهولیبامه یعوش، و یعلام و قورح رازایید. اینانند پسران عیسو که برای وی در زمین کنعان متولد شدند.
\par 6 پس عیسو زنان و پسران ودختران و جمیع اهل بیت، و مواشی و همه حیوانات، و تمامی اندوخته خود را که در زمین کنعان اندوخته بود، گرفته، از نزد برادر خودیعقوب به زمین دیگر رفت.
\par 7 زیرا که اموال ایشان زیاده بود از آنکه با هم سکونت کنند. و زمین غربت ایشان بسبب مواشی ایشان گنجایش ایشان نداشت.
\par 8 و عیسو در جبل سعیر ساکن شد. وعیسو همان ادوم است.
\par 9 و این است پیدایش عیسو پدر ادوم در جبل سعیر:
\par 10 اینست نامهای پسران عیسو: الیفاز پسرعاده، زن عیسو، و رعوئیل، پسر بسمه، زن عیسو.
\par 11 و بنی الیفاز: تیمان و اومار و صفوا و جعتام وقناز بودند.
\par 12 و تمناع، کنیز الیفاز، پسر عیسوبود. وی عمالیق را برای الیفاز زایید. اینانندپسران عاده زن عیسو.
\par 13 و اینانند پسران رعوئیل: نحت و زارع و شمه و مزه. اینانند پسران بسمه زن عیسو.
\par 14 و اینانند پسران اهولیبامه دختر عنی، دختر صبعون، زن عیسو که یعوش ویعلام و قورح را برای عیسو زایید.
\par 15 اینانند امرای بنی عیسو: پسران الیفازنخست زاده عیسو، یعنی امیر تیمان و امیر اومار وامیر صفوا و امیر قناز،
\par 16 و امیر قورح و امیرجعتام و امیر عمالیق. اینانند امرای الیفاز در زمین ادوم. اینانند پسران عاده.
\par 17 و اینان پسران رعوئیل بن عیسو می‌باشند: امیر نحت و امیر زارح و امیر شمه و امیر مزه. اینهاامرای رعوئیل در زمین ادوم بودند. اینانند پسران بسمه زن عیسو.
\par 18 و اینانند بنی اهولیبامه زن عیسو: امیریعوش و امیر یعلام و امیر قورح. اینها امرای اهولیبامه دختر عنی، زن عیسو می‌باشند.
\par 19 اینانند پسران عیسو که ادوم باشد و اینهاامرای ایشان می‌باشند.
\par 20 و اینانند پسران سعیر حوری که ساکن آن زمین بودند، یعنی: لوطان و شوبال و صبعون وعنی،
\par 21 و دیشون و ایصر و دیشان. اینانند امرای حوریان و پسران سعیر در زمین ادوم.
\par 22 و پسران لوطان: حوری و هیمام بودند وخواهر لوطان تمناع، بود.
\par 23 و اینانند پسران شوبال: علوان و منحت و عیبال و شفو و اونام.
\par 24 و اینانند بنی صبعون: ایه و عنی. همین عنی است که چشمه های آب گرم را در صحرا پیدانمود، هنگامی که الاغهای پدر خود، صبعون رامی چرانید.
\par 25 و اینانند اولاد عنی: دیشون واهولیبامه دختر عنی.
\par 26 و اینانند پسران دیشان: حمدان و اشبان و بتران و کران.
\par 27 و اینانند پسران ایصر: بلهان و زعوان و عقان.
\par 28 اینانند پسران دیشان: عوص و اران.
\par 29 اینها امرای حوریانند: امیر لوطان و امیرشوبال و امیر صبعون و امیر عنی،
\par 30 امیر دیشون و امیر ایصر و امیر دیشان. اینها امرای حوریانند به حسب امرای ایشان در زمین سعیر.
\par 31 و اینانند پادشاهانی که در زمین ادوم سلطنت کردند، قبل از آنکه پادشاهی بربنی‌اسرائیل سلطنت کند:
\par 32 و بالع بن بعور درادوم پادشاهی کرد، و نام شهر او دینهابه بود.
\par 33 وبالع مرد، و در جایش یوباب بن زارح از بصره سلطنت کرد.
\par 34 و یوباب مرد، و در جایش حوشام از زمین تیمانی پادشاهی کرد.
\par 35 وحوشام مرد، و در جایش هداد بن بداد، که درصحرای موآب، مدیان را شکست داد، پادشاهی کرد، و نام شهر او عویت بود.
\par 36 و هداد مرد، و درجایش سمله از مسریقه پادشاهی نمود.
\par 37 وسمله مرد، و شاول از رحوبوت نهر در جایش پادشاهی کرد.
\par 38 و شاول مرد، و در جایش بعل حانان بن عکبور سلطنت کرد.
\par 39 و بعل حانان بن عکبور مرد، و در جایش، هدار پادشاهی کرد. ونام شهرش فاعو بود، و زنش مسمی به مهیطبئیل دختر مطرد، دختر می‌ذاهب بود.
\par 40 و اینست نامهای امرای عیسو، حسب قبائل ایشان و اماکن و نامهای ایشان: امیر تمناع وامیر علوه و امیر یتیت،
\par 41 و امیر اهولیبامه و امیرایله و امیر فینون،
\par 42 و امیر قناز و امیرتیمان و امیرمبصار،و امیر مجدیئیل و امیر عیرام. اینان امرای ادومند، حسب مساکن ایشان در زمین ملک ایشان. همان عیسو پدر ادوم است.
\par 43 و امیر مجدیئیل و امیر عیرام. اینان امرای ادومند، حسب مساکن ایشان در زمین ملک ایشان. همان عیسو پدر ادوم است.
 
\chapter{37}

\par 1 و یعقوب در زمین غربت پدر خود، یعنی زمین کنعان ساکن شد.
\par 2 این است پیدایش یعقوب. چون یوسف هفده ساله بود، گله را با برادران خود چوپانی می‌کرد. و آن جوان باپسران بلهه و پسران زلفه، زنان پدرش، می‌بود. ویوسف از بدسلوکی ایشان پدر را خبر می‌داد.
\par 3 واسرائیل، یوسف را از سایر پسران خود بیشتردوست داشتی، زیرا که او پسر پیری او بود، وبرایش ردایی بلند ساخت.
\par 4 و چون برادرانش دیدند که پدر ایشان، او را بیشتر از همه برادرانش دوست می‌دارد، از او کینه داشتند، ونمی توانستند با وی به سلامتی سخن گویند.
\par 5 ویوسف خوابی دیده، آن را به برادران خود بازگفت. پس بر کینه او افزودند.
\par 6 و بدیشان گفت: «این خوابی را که دیده‌ام، بشنوید:
\par 7 اینک ما در مزرعه بافه‌ها می‌بستیم، که ناگاه بافه من برپا شده، بایستاد، و بافه های شماگرد آمده، به بافه من سجده کردند.»
\par 8 برادرانش به وی گفتند: «آیا فی الحقیقه بر ماسلطنت خواهی کرد؟ و بر ما مسلط خواهی شد؟» و بسبب خوابها و سخنانش بر کینه او افزودند.
\par 9 از آن پس خوابی دیگر دید، و برادران خود را ازآن خبر داده، گفت: «اینک باز خوابی دیده‌ام، که ناگاه آفتاب و ماه و یازده ستاره مرا سجده کردند.»
\par 10 و پدر و برادران خود را خبر داد، وپدرش او را توبیخ کرده، بوی گفت: «این چه خوابی است که دیده‌ای؟ آیا من و مادرت وبرادرانت حقیقت خواهیم آمد و تو را بر زمین سجده خواهیم نمود؟»
\par 11 و برادرانش بر او حسدبردند، و اما پدرش، آن امر را در خاطر نگاه داشت.
\par 12 و برادرانش برای چوپانی گله پدر خود، به شکیم رفتند.
\par 13 و اسرائیل به یوسف گفت: «آیابرادرانت در شکیم چوپانی نمی کنند، بیا تا تو رانزد ایشان بفرستم.» وی را گفت: «لبیک.»
\par 14 او راگفت: «الان برو و سلامتی برادران و سلامتی گله را ببین و نزد من خبر بیاور.» و او را از وادی حبرون فرستاد، و به شکیم آمد.
\par 15 و شخصی به اوبرخورد، و اینک، او در صحرا آواره می‌بود، پس آن شخص از او پرسیده، گفت: «چه می‌طلبی؟»
\par 16 گفت: «من برادران خود را می‌جویم، مرا خبرده که کجا چوپانی می‌کنند.»
\par 17 آن مرد گفت: «ازاینجا روانه شدند، زیرا شنیدیم که می‌گفتند: به دوتان می‌رویم.» پس یوسف از عقب برادران خود رفته، ایشان را در دوتان یافت.
\par 18 و او را ازدور دیدند، و قبل از آنکه نزدیک ایشان بیاید، باهم توطئه دیدند که اورا بکشند.
\par 19 و به یکدیگر گفتند: «اینک این صاحب خوابها می‌آید.
\par 20 اکنون بیایید او را بکشیم، و به یکی از این چاهها بیندازیم، و گوییم جانوری درنده او را خورد. و ببینیم خوابهایش چه می‌شود.»
\par 21 لیکن روبین چون این را شنید، او را از دست ایشان رهانیده، گفت: «او را نکشیم.»
\par 22 پس روبین بدیشان گفت: «خون مریزید، او را در این چاه که در صحراست، بیندازید، و دست خود رابر او دراز مکنید.» تا او را از دست ایشان رهانیده، به پدر خود رد نماید.
\par 23 و به مجرد رسیدن یوسف نزد برادران خود، رختش را یعنی آن ردای بلند را که دربرداشت، از او کندند.
\par 24 و او را گرفته، درچاه انداختند، اما چاه، خالی و بی‌آب بود.
\par 25 پس برای غذا خوردن نشستند، و چشمان خود را باز کرده، دیدند که ناگاه قافله اسماعیلیان از جلعاد می‌رسد، و شتران ایشان کتیرا و بلسان ولادن، بار دارند، و می‌روند تا آنها را به مصر ببرند.
\par 26 آنگاه یهودا به برادران خود گفت: «برادر خودرا کشتن و خون او را مخفی داشتن چه سود دارد؟
\par 27 بیایید او را به این اسماعیلیان بفروشیم، ودست ما بر وی نباشد، زیرا که او برادر و گوشت ماست.» پس برادرانش بدین رضا دادند.
\par 28 و چون تجار مدیانی در گذر بودند، یوسف را از چاه کشیده، برآوردند، و یوسف را به اسماعیلیان به بیست پاره نقره فروختند. پس یوسف را به مصر بردند.
\par 29 و روبین چون به‌سرچاه برگشت، و دید که یوسف در چاه نیست، جامه خود را چاک زد،
\par 30 و نزد برادران خودبازآمد و گفت: «طفل نیست و من کجا بروم؟»
\par 31 پس ردای یوسف را گرفتند، و بز نری راکشته، ردا را در خونش فرو بردند.
\par 32 و آن ردای بلند را فرستادند و به پدر خود رسانیده، گفتند: «این را یافته‌ایم، تشخیص کن که ردای پسرت است یا نه.»
\par 33 پس آن را شناخته، گفت: «ردای پسر من است! جانوری درنده او را خورده است، و یقین یوسف دریده شده است.»
\par 34 و یعقوب رخت خود را پاره کرده، پلاس دربر کرد، وروزهای بسیار برای پسر خود ماتم گرفت.
\par 35 وهمه پسران و همه دخترانش به تسلی اوبرخاستند. اما تسلی نپذیرفت، و گفت: «سوگوارنزد پسر خود به گور فرود می‌روم.» پس پدرش برای وی همی گریست.اما مدیانیان، یوسف را در مصر، به فوطیفار که خواجه فرعون و سردارافواج خاصه بود، فروختند.
\par 36 اما مدیانیان، یوسف را در مصر، به فوطیفار که خواجه فرعون و سردارافواج خاصه بود، فروختند.
 
\chapter{38}

\par 1 و واقع شد در آن زمان که یهودا ازنزد برادران خود رفته، نزد شخصی عدلامی، که حیره نام داشت، مهمان شد.
\par 2 و درآنجا یهودا، دختر مرد کنعانی را که مسمی به شوعه بود، دید و او را گرفته، بدو درآمد.
\par 3 پس آبستن شده، پسری زایید و او را عیر نام نهاد.
\par 4 وبار دیگر آبستن شده، پسری زایید و او را اونان نامید.
\par 5 و باز هم پسری زاییده، او را شیله نام گذارد. و چون او را زایید، یهودا در کزیب بود. 
\par 6 و یهودا، زنی مسمی به تامار، برای نخست زاده خود عیر گرفت.
\par 7 و نخست زاده یهودا، عیر، در نظر خداوند شریر بود، و خداونداو را بمیراند.
\par 8 پس یهودا به اونان گفت: «به زن برادرت درآی، و حق برادر شوهری را بجاآورده، نسلی برای برادر خود پیدا کن.»
\par 9 لکن چونکه اونان دانست که آن نسل از آن او نخواهدبود، هنگامی که به زن برادر خود درآمد، بر زمین انزال کرد، تا نسلی برای برادر خود ندهد.
\par 10 و این کار او در نظر خداوند ناپسند آمد، پس او را نیزبمیراند.
\par 11 و یهودا به عروس خود، تامار گفت: «در خانه پدرت بیوه بنشین تا پسرم شیله بزرگ شود.» زیرا گفت: «مبادا او نیز مثل برادرانش بمیرد.» پس تامار رفته، در خانه پدر خود ماند.
\par 12 و چون روزها سپری شد، دختر شوعه زن یهودا مرد. و یهودا بعد از تعزیت او با دوست خود حیره عدلامی، نزد پشم‌چینان گله خود، به تمنه آمد.
\par 13 و به تامار خبر داده، گفتند: «اینک پدرشوهرت برای چیدن پشم گله خویش، به تمنه می‌آید.»
\par 14 پس رخت بیوگی را از خویشتن بیرون کرده، برقعی به رو کشیده، خود را درچادری پوشید، و به دروازه عینایم که در راه تمنه است، بنشست. زیرا که دید شیله بزرگ شده است، و او را به وی به زنی ندادند.
\par 15 چون یهودااو را بدید، وی را فاحشه پنداشت، زیرا که روی خود را پوشیده بود.
\par 16 پس از راه به سوی او میل کرده، گفت: «بیا تابه تو درآیم.» زیرا ندانست که عروس اوست. گفت: «مرا چه می‌دهی تا به من درآیی.»
\par 17 گفت: «بزغاله‌ای از گله می‌فرستم.» گفت: «آیا گرومی دهی تا بفرستی.»
\par 18 گفت: «تو را چه گرودهم.» گفت: «مهر و زنار خود را و عصایی که دردست داری.» پس به وی داد، و بدو درآمد، و او ازوی آبستن شد.
\par 19 و برخاسته، برفت. و برقع را ازخود برداشته، رخت بیوگی پوشید.
\par 20 و یهودا بزغاله را به‌دست دوست عدلامی خود فرستاد، تا گرو را از دست آن زن بگیرد، امااو را نیافت.
\par 21 و از مردمان آن مکان پرسیده، گفت: «آن فاحشه‌ای که سر راه عینایم نشسته بود، کجاست؟» گفتند: «فاحشه‌ای در اینجا نبود.»
\par 22 پس نزد یهودا برگشته، گفت: «او را نیافتم، ومردمان آن مکان نیز می‌گویند که فاحشه‌ای دراینجا نبود.»
\par 23 یهودا گفت: «بگذار برای خودنگاه دارد، مبادا رسوا شویم. اینک بزغاله رافرستادم و تو او را نیافتی.»
\par 24 و بعد از سه ماه یهودا را خبر داده، گفتند: «عروس تو تامار، زناکرده است و اینک از زنا نیز آبستن شده.» پس یهودا گفت: «وی را بیرون آرید تا سوخته شود!»
\par 25 چون او را بیرون می‌آوردند نزد پدر شوهر خود فرستاده، گفت: «از مالک این چیزها آبستن شده‌ام، و گفت: «تشخیص کن که این مهر و زنار وعصا از آن کیست.»
\par 26 و یهودا آنها را شناخت، وگفت: «او از من بی‌گناه تر است، زیرا که او را به پسر خود شیله ندادم. و بعد او را دیگر نشناخت.
\par 27 و چون وقت وضع حملش رسید، اینک توامان در رحمش بودند.
\par 28 و چون می‌زایید، یکی دست خود را بیرون آورد که در حال قابله ریسمانی قرمز گرفته، بر دستش بست و گفت: «این اول بیرون آمد.»
\par 29 و دست خود را بازکشید. و اینک برادرش بیرون آمد و قابله گفت: «چگونه شکافتی، این شکاف بر تو باد.» پس او را فارص نام نهاد.بعد از آن برادرش که ریسمان قرمز را بردست داشت بیرون آمد، و او را زارح نامید.
\par 30 بعد از آن برادرش که ریسمان قرمز را بردست داشت بیرون آمد، و او را زارح نامید.
 
\chapter{39}

\par 1 اما یوسف را به مصر بردند، و مردی مصری، فوطیفار نام که خواجه و سردارافواج خاصه فرعون بود، وی را از دست اسماعیلیانی که او را بدانجا برده بودند، خرید.
\par 2 وخداوند با یوسف می‌بود، و او مردی کامیاب شد، و در خانه آقای مصری خود ماند.
\par 3 و آقایش دیدکه خداوند با وی می‌باشد، و هر‌آنچه او می‌کند، خداوند در دستش راست می‌آورد.
\par 4 پس یوسف در نظر وی التفات یافت، و او را خدمت می‌کرد، واو را به خانه خود برگماشت و تمام مایملک خویش را بدست وی سپرد.
\par 5 و واقع شد بعد ازآنکه او را بر خانه و تمام مایملک خود گماشته بود، که خداوند خانه آن مصری را بسبب یوسف برکت داد، و برکت خداوند بر همه اموالش، چه در خانه و چه در صحرا بود.
\par 6 و آنچه داشت به‌دست یوسف واگذاشت، و از آنچه با وی بود، خبر نداشت جز نانی که می‌خورد. و یوسف خوش اندام و نیک منظر بود.
\par 7 و بعد از این امور واقع شد که زن آقایش بریوسف نظر انداخته، گفت: «با من همخواب شو.»
\par 8 اما او ابا نموده، به زن آقای خود گفت: «اینک آقایم از آنچه نزد من در خانه است، خبر ندارد، وآنچه دارد، به‌دست من سپرده است.
\par 9 بزرگتری ازمن در این خانه نیست و چیزی از من دریغ نداشته، جز تو، چون زوجه او می‌باشی؛ پس چگونه مرتکب این شرارت بزرگ بشوم و به خدا خطاورزم؟»
\par 10 و اگرچه هر روزه به یوسف سخن می‌گفت، به وی گوش نمی گرفت که با او بخوابد یانزد وی بماند.
\par 11 و روزی واقع شد که به خانه درآمد، تا به شغل خود پردازد و از اهل خانه کسی آنجا درخانه نبود.
\par 12 پس جامه او را گرفته، گفت: «با من بخواب.» اما او جامه خود را به‌دستش رها کرده، گریخت و بیرون رفت.
\par 13 و چون او دید که رخت خود را به‌دست وی ترک کرد و از خانه گریخت،
\par 14 مردان خانه راصدا زد، و بدیشان بیان کرده، گفت: «بنگرید، مردعبرانی را نزد ما آورد تا ما را مسخره کند، و نزدمن آمد تا با من بخوابد، و به آواز بلند فریاد کردم،
\par 15 و چون شنید که به آواز بلند فریاد برآوردم، جامه خود را نزد من واگذارده، فرار کرد و بیرون رفت.»
\par 16 پس جامه او را نزد خود نگاه داشت، تاآقایش به خانه آمد.
\par 17 و به وی بدین مضمون ذکر کرده، گفت: «آن غلام عبرانی که برای ماآورده‌ای، نزد من آمد تا مرا مسخره کند،
\par 18 وچون به آواز بلند فریاد برآوردم، جامه خود راپیش من رها کرده، بیرون گریخت.»
\par 19 پس چون آقایش سخن زن خود را شنید که به وی بیان کرده، گفت: «غلامت به من چنین کرده است، » خشم او افروخته شد.
\par 20 و آقای یوسف، او را گرفته، در زندان خانه‌ای که اسیران پادشاه بسته بودند، انداخت و آنجا در زندان ماند.
\par 21 اماخداوند با یوسف می‌بود و بر وی احسان می‌فرمود، و او را در نظر داروغه زندان حرمت داد.
\par 22 و داروغه زندان همه زندانیان را که درزندان بودند، به‌دست یوسف سپرد. و آنچه درآنجا می‌کردند، او کننده آن بود.و داروغه زندان بدانچه در دست وی بود، نگاه نمی کرد، زیرا خداوند با وی می‌بود و آنچه را که او می‌کرد، خداوند راست می‌آورد.
\par 23 و داروغه زندان بدانچه در دست وی بود، نگاه نمی کرد، زیرا خداوند با وی می‌بود و آنچه را که او می‌کرد، خداوند راست می‌آورد.
 
\chapter{40}

\par 1 و بعد از این امور، واقع شد که ساقی و خباز پادشاه مصر، به آقای خویش، پادشاه مصر خطا کردند.
\par 2 و فرعون به دو خواجه خود، یعنی سردار ساقیان و سردار خبازان غضب نمود.
\par 3 و ایشان را در زندان رئیس افواج خاصه، یعنی زندانی که یوسف در آنجا محبوس بود، انداخت.
\par 4 و سردار افواج خاصه، یوسف را برایشان گماشت، و ایشان را خدمت می‌کرد، ومدتی در زندان ماندند.
\par 5 و هر دو در یک شب خوابی دیدند، هر کدام خواب خود را. هر کدام موافق تعبیر خود، یعنی ساقی و خباز پادشاه مصر که در زندان محبوس بودند.
\par 6 بامدادان چون یوسف نزد ایشان آمد، دید که اینک ملول هستند.
\par 7 پس، از خواجه های فرعون، که با وی در زندان آقای او بودند، پرسیده، گفت: «امروز چرا روی شما غمگین است؟»
\par 8 به وی گفتند: «خوابی دیده‌ایم و کسی نیست که آن را تعبیر کند.» یوسف بدیشان گفت: «آیا تعبیرها از آن خدانیست؟ آن را به من بازگویید.»
\par 9 آنگاه رئیس ساقیان، خواب خود را به یوسف بیان کرده، گفت: «در خواب من، اینک تاکی پیش روی من بود.
\par 10 و در تاک سه شاخه بود و آن بشکفت، و گل آورد و خوشه هایش انگور رسیده داد.
\par 11 و جام فرعون در دست من بود. و انگورها را چیده، در جام فرعون فشردم، وجام را به‌دست فرعون دادم.»
\par 12 یوسف به وی گفت: «تعبیرش اینست، سه شاخه سه روز است.
\par 13 بعد از سه روز، فرعون سر تو را برافرازد و به منصبت بازگمارد، و جام فرعون را به‌دست وی دهی به رسم سابق، که ساقی او بودی.
\par 14 و هنگامی که برای تو نیکوشود، مرا یاد کن و به من احسان نموده، احوال مرانزد فرعون مذکور ساز، و مرا از این خانه بیرون آور،
\par 15 زیرا که فی الواقع از زمین عبرانیان دزدیده شده‌ام، و اینجا نیز کاری نکرده‌ام که مرا درسیاه چال افکنند.»
\par 16 اما چون رئیس خبازان دید که تعبیر، نیکوبود، به یوسف گفت: «من نیز خوابی دیده‌ام، که اینک سه سبد نان سفید بر سر من است،
\par 17 و درسبد زبرین هر قسم طعام برای فرعون از پیشه خباز می‌باشد و مرغان، آن را از سبدی که بر سر من است، می‌خورند.»
\par 18 یوسف در جواب گفت: «تعبیرش این است، سه سبد سه روز می‌باشد.
\par 19 و بعد از سه روز فرعون سر تو را از تو بردارد وتو را بر دار بیاویزد، و مرغان، گوشتت را از توبخورند.»
\par 20 پس در روز سوم که یوم میلاد فرعون بود، ضیافتی برای همه خدام خود ساخت، و سررئیس ساقیان و سر رئیس خبازان را در میان نوکران خود برافراشت.
\par 21 اما رئیس ساقیان را به ساقی گریش باز آورد، و جام را به‌دست فرعون داد.
\par 22 و اما رئیس خبازان را به دار کشید، چنانکه یوسف برای ایشان تعبیر کرده بود.لیکن رئیس ساقیان، یوسف را به یاد نیاورد، بلکه او رافراموش کرد.
\par 23 لیکن رئیس ساقیان، یوسف را به یاد نیاورد، بلکه او رافراموش کرد.
 
\chapter{41}

\par 1 و واقع شد، چون دو سال سپری شد، که فرعون خوابی دید که اینک بر کنارنهر ایستاده است.
\par 2 که ناگاه از نهر، هفت گاوخوب صورت و فربه گوشت برآمده، بر مرغزارمی چریدند.
\par 3 و اینک هفت گاو دیگر، بد صورت و لاغر گوشت، در عقب آنها از نهر برآمده، به پهلوی آن گاوان اول به کنار نهر ایستادند.
\par 4 و این گاوان زشت صورت و لاغر گوشت، آن هفت گاوخوب صورت و فربه را فرو بردند. و فرعون بیدارشد.
\par 5 و باز بخسبید و دیگر باره خوابی دید، که اینک هفت سنبله پر و نیکو بر یک ساق برمی آید.
\par 6 و اینک هفت سنبله لاغر، از باد شرقی پژمرده، بعد از آنها می‌روید.
\par 7 و سنبله های لاغر، آن هفت سنبله فربه و پر را فرو بردند، و فرعون بیدار شده، دید که اینک خوابی است.
\par 8 صبحگاهان دلش مضطرب شده، فرستاد و همه جادوگران و جمیع حکیمان مصر را خواند، و فرعون خوابهای خودرا بدیشان باز‌گفت. اما کسی نبود که آنها را برای فرعون تعبیر کند.
\par 9 آنگاه رئیس ساقیان به فرعون عرض کرده، گفت: «امروز خطایای من بخاطرم آمد.
\par 10 فرعون بر غلامان خود غضب نموده، مرا با رئیس خبازان در زندان سردار افواج خاصه، حبس فرمود.
\par 11 ومن و او در یک شب، خوابی دیدیم، هر یک موافق تعبیر خواب خود، خواب دیدیم.
\par 12 و جوانی عبرانی در آنجا با ما بود، غلام سردار افواج خاصه. و خوابهای خود را نزد او بیان کردیم و اوخوابهای ما را برای ما تعبیر کرد، هر یک را موافق خوابش تعبیر کرد.
\par 13 و به عینه موافق تعبیری که برای ما کرد، واقع شد. مرا به منصبم بازآورد، و اورا به دار کشید.»
\par 14 آنگاه فرعون فرستاده، یوسف را خواند، واو را به زودی از زندان بیرون آوردند. و صورت خود را تراشیده، رخت خود را عوض کرد، و به حضور فرعون آمد.
\par 15 فرعون به یوسف گفت: «خوابی دیده‌ام و کسی نیست که آن را تعبیر کند، و درباره تو شنیدم که خواب می‌شنوی تا تعبیرش کنی.»
\par 16 یوسف فرعون را به پاسخ گفت: «از من نیست، خدا فرعون را به سلامتی جواب خواهدداد.»
\par 17 و فرعون به یوسف گفت: «در خواب خوددیدم که اینک به کنار نهر ایستاده‌ام،
\par 18 و ناگاه هفت گاو فربه گوشت و خوب صورت از نهر برآمده، بر مرغزار می‌چرند. 
\par 19 و اینک هفت گاودیگر زبون و بسیار زشت صورت و لاغر گوشت، که در تمامی زمین مصر بدان زشتی ندیده بودم، در عقب آنها برمی آیند.
\par 20 و گاوان لاغر زشت، هفت گاو فربه اول را می‌خورند.
\par 21 و چون به شکم آنها فرو رفتند معلوم نشد که بدرون آنهاشدند، زیرا که صورت آنها مثل اول زشت ماند. پس بیدار شدم.
\par 22 و باز خوابی دیدم که اینک هفت سنبله پر و نیکو بر یک ساق برمی آید.
\par 23 واینک هفت سنبله خشک باریک و از باد شرقی پژمرده، بعد از آنها می‌روید.
\par 24 و سنابل لاغر، آن هفت سنبله نیکو را فرو می‌برد. و جادوگران راگفتم، لیکن کسی نیست که برای من شرح کند.»
\par 25 یوسف به فرعون گفت: «خواب فرعون یکی است. خدا از آنچه خواهد کرد، فرعون راخبر داده است.
\par 26 هفت گاو نیکو هفت سال باشدو هفت سنبله نیکو هفت سال. همانا خواب یکی است.
\par 27 و هفت گاو لاغر زشت، که در عقب آنهابرآمدند، هفت سال باشد. و هفت سنبله خالی ازباد شرقی پژمرده، هفت سال قحط می‌باشد.
\par 28 سخنی که به فرعون گفتم، این است: آنچه خدامی کند به فرعون ظاهر ساخته است.
\par 29 هماناهفت سال فراوانی بسیار، در تمامی زمین مصرمی آید.
\par 30 و بعد از آن، هفت سال قحط پدیدآید. و تمامی فراوانی در زمین مصر فراموش شود. و قحط، زمین را تباه خواهد ساخت.
\par 31 وفراوانی در زمین معلوم نشود بسبب قحطی که بعداز آن آید، زیرا که به غایت سخت خواهد بود.
\par 32 و چون خواب به فرعون دو مرتبه مکرر شد، این است که این حادثه از جانب خدا مقرر شده، و خدا آن را به زودی پدید خواهد آورد.
\par 33 پس اکنون فرعون می‌باید مردی بصیر و حکیم را پیدانموده، و او را بر زمین مصر بگمارد.
\par 34 فرعون چنین بکند، و ناظران بر زمین برگمارد، و در هفت سال فراوانی، خمس از زمین مصر بگیرد.
\par 35 وهمه ماکولات این سالهای نیکو را که می‌آیدجمع کنند، و غله را زیر دست فرعون ذخیره نمایند، و خوراک در شهرها نگاه دارند.
\par 36 تاخوراک برای زمین، به جهت هفت سال قحطی که در زمین مصر خواهد بود ذخیره شود، مبادا زمین از قحط تباه گردد.»
\par 37 پس این سخن بنظر فرعون و بنظر همه بندگانش پسند آمد.
\par 38 و فرعون به بندگان خودگفت: «آیا کسی را مثل این توانیم یافت، مردی که روح خدا در وی است؟»
\par 39 و فرعون به یوسف گفت: «چونکه خدا کل این امور را بر تو کشف کرده است، کسی مانند تو بصیر و حکیم نیست.
\par 40 تو بر خانه من باش، و به فرمان تو، تمام قوم من منتظم شوند، جز اینکه بر تخت از تو بزرگترباشم.»
\par 41 و فرعون به یوسف گفت: «بدانکه تو را برتمامی زمین مصر گماشتم.»
\par 42 و فرعون انگشترخود را از دست خویش بیرون کرده، آن را بردست یوسف گذاشت، و او را به کتان نازک آراسته کرد، و طوقی زرین بر گردنش انداخت.
\par 43 و او را بر عرابه دومین خود سوار کرد، و پیش رویش ندا می‌کردند که «زانو زنید!» پس او را برتمامی زمین مصر برگماشت.
\par 44 و فرعون به یوسف گفت: «من فرعون هستم، و بدون توهیچکس دست یا پای خود را در کل ارض مصربلند نکند.»
\par 45 و فرعون یوسف را صفنات فعنیح نامید، و اسنات، دختر فوطی فارع، کاهن اون رابدو به زنی داد، و یوسف بر زمین مصر بیرون رفت.
\par 46 و یوسف سی ساله بود وقتی که به حضورفرعون، پادشاه مصر بایستاد، و یوسف از حضورفرعون بیرون شده، در تمامی زمین مصر گشت.
\par 47 و در هفت سال فراوانی، زمین محصول خود رابه کثرت آورد.
\par 48 پس تمامی ماکولات آن هفت سال را که در زمین مصر بود، جمع کرد، و خوراک را در شهرها ذخیره نمود، و خوراک مزارع حوالی هر شهر را در آن گذاشت.
\par 49 و یوسف غله بیکران بسیار، مثل ریگ دریا ذخیره کرد، تا آنکه ازحساب بازماند، زیرا که از حساب زیاده بود.
\par 50 وقبل از وقوع سالهای قحط، دو پسر برای یوسف زاییده شد، که اسنات، دختر فوطی فارع، کاهن اون برایش بزاد.
\par 51 و یوسف نخست زاده خود رامنسی نام نهاد، زیرا گفت: «خدا مرا از تمامی مشقتم و تمامی خانه پدرم فراموشی داد.»
\par 52 ودومین را افرایم نامید، زیرا گفت: «خدا مرا درزمین مذلتم بارآور گردانید.»
\par 53 و هفت سال فراوانی که در زمین مصر بود، سپری شد.
\par 54 و هفت سال قحط، آمدن گرفت، چنانکه یوسف گفته بود. و قحط در همه زمینهاپدید شد، لیکن در تمامی زمین مصر نان بود.
\par 55 وچون تمامی زمین مصر مبتلای قحط شد، قوم برای نان نزد فرعون فریاد برآوردند. و فرعون به همه مصریان گفت: «نزد یوسف بروید و آنچه او به شما گوید، بکنید.»
\par 56 پس قحط، تمامی روی زمین را فروگرفت، و یوسف همه انبارها را بازکرده، به مصریان می‌فروخت، و قحط در زمین مصر سخت شد.و همه زمینها به جهت خریدغله نزد یوسف به مصر آمدند، زیرا قحط برتمامی زمین سخت شد.
\par 57 و همه زمینها به جهت خریدغله نزد یوسف به مصر آمدند، زیرا قحط برتمامی زمین سخت شد.
 
\chapter{42}

\par 1 و اما یعقوب چون دید که غله درمصر است، پس یعقوب به پسران خودگفت: «چرا به یکدیگر می‌نگرید؟»
\par 2 و گفت: «اینک شنیده‌ام که غله در مصر است، بدانجابروید و برای ما از آنجا بخرید، تا زیست کنیم ونمیریم.»
\par 3 پس ده برادر یوسف برای خریدن غله به مصر فرود آمدند.
\par 4 و اما بنیامین، برادر یوسف رایعقوب با برادرانش نفرستاد، زیرا گفت مبادازیانی بدو رسد.
\par 5 پس بنی‌اسرائیل در میان آنانی که می‌آمدند، به جهت خرید آمدند، زیرا که قحطدر زمین کنعان بود.
\par 6 و یوسف حاکم ولایت بود، و خود به همه اهل زمین غله می‌فروخت. و برادران یوسف آمده، رو به زمین نهاده، او را سجده کردند.
\par 7 چون یوسف برادران خود را دید، ایشان را بشناخت، وخود را بدیشان بیگانه نموده، آنها را به درشتی، سخن گفت و از ایشان پرسید: «از کجا آمده‌اید؟» گفتند: «از زمین کنعان تا خوراک بخریم.»
\par 8 و یوسف برادران خود را شناخت، لیکن ایشان او را نشناختند.
\par 9 و یوسف خوابها را که درباره ایشان دیده بود، بیاد آورد. پس بدیشان گفت: «شما جاسوسانید، و به جهت دیدن عریانی زمین آمده‌اید.»
\par 10 بدو گفتند: «نه، یا سیدی! بلکه غلامانت به جهت خریدن خوراک آمده‌اند.
\par 11 ماهمه پسران یک شخص هستیم. ما مردمان صادقیم؛ غلامانت، جاسوس نیستند.»
\par 12 بدیشان گفت: «نه، بلکه به جهت دیدن عریانی زمین آمده‌اید.»
\par 13 گفتند: «غلامانت دوازده برادرند، پسران یک مرد در زمین کنعان. و اینک کوچکتر، امروز نزد پدر ماست، و یکی نایاب شده است.»
\par 14 یوسف بدیشان گفت: «همین است آنچه به شما گفتم که جاسوسانید!
\par 15 بدینطور آزموده می‌شوید: به حیات فرعون از اینجا بیرون نخواهید رفت، جز اینکه برادر کهتر شما در اینجابیاید.
\par 16 یک نفر از خودتان بفرستید، تا برادر شمارا بیاورد، و شما اسیر بمانید تا سخن شما آزموده شود که صدق با شماست یا نه، والا به حیات فرعون جاسوسانید!»
\par 17 پس ایشان را با هم سه روز در زندان انداخت.
\par 18 و روز سوم یوسف بدیشان گفت: «این رابکنید و زنده باشید، زیرا من از خدا می‌ترسم:
\par 19 هر گاه شما صادق هستید، یک برادر از شما درزندان شما اسیر باشد، و شما رفته، غله برای گرسنگی خانه های خود ببرید.
\par 20 و برادر کوچک خود را نزد من آرید، تا سخنان شما تصدیق شودو نمیرید.» پس چنین کردند.
\par 21 و به یکدیگر گفتند: «هر آینه به برادر خودخطا کردیم، زیرا تنگی جان او را دیدیم وقتی که به ما استغاثه می‌کرد، و نشنیدیم. از این‌رو این تنگی بر ما رسید.»
\par 22 و روبین در جواب ایشان گفت: «آیا به شما نگفتم که به پسر خطا مورزید؟
\par 23 و ایشان ندانستند که یوسف می‌فهمد، زیرا که ترجمانی در میان ایشان بود.
\par 24 پس از ایشان کناره جسته، بگریست و نزدایشان برگشته، با ایشان گفتگو کرد، و شمعون را ازمیان ایشان گرفته، او را روبروی ایشان دربند نهاد.
\par 25 و یوسف فرمود تا جوالهای ایشان را از غله پر سازند، و نقد ایشان را در عدل هر کس نهند، وزاد سفر بدیشان دهند، و به ایشان چنین کردند.
\par 26 پس غله را بر حماران خود بار کرده، از آنجاروانه شدند.
\par 27 و چون یکی، عدل خود را در منزل باز کرد، تا خوراک به الاغ دهد، نقد خود را دید که اینک در دهن عدل او بود.
\par 28 و به برادران خود گفت: «نقد من رد شده است، و اینک در عدل من است. آنگاه دل ایشان طپیدن گرفت، و به یکدیگر لرزان شده، گفتند: «این چیست که خدا به ما کرده است.»
\par 29 پس نزد پدر خود، یعقوب، به زمین کنعان آمدند، و از آنچه بدیشان واقع شده بود، خبرداده، گفتند:
\par 30 «آن مرد که حاکم زمین است، با مابه سختی سخن گفت، و ما را جاسوسان زمین پنداشت.
\par 31 و بدو گفتیم ما صادقیم و جاسوس نی.
\par 32 ما دوازده برادر، پسران پدر خود هستیم، یکی نایاب شده است، و کوچکتر، امروز نزد پدرما در زمین کنعان می‌باشد.
\par 33 و آن مرد که حاکم زمین است، به ما گفت: از این خواهم فهمید که شما راستگو هستید که یکی از برادران خود را نزدمن گذارید، و برای گرسنگی خانه های خودگرفته، بروید.
\par 34 و برادر کوچک خود را نزد من آرید، و خواهم یافت که شما جاسوس نیستیدبلکه صادق. آنگاه برادر شما را به شما رد کنم، ودر زمین داد و ستد نمایید.»
\par 35 و واقع شد که چون عدلهای خود را خالی می‌کردند، اینک کیسه پول هر کس در عدلش بود. و چون ایشان و پدرشان، کیسه های پول را دیدند، بترسیدند.
\par 36 و پدر ایشان، یعقوب، بدیشان گفت: «مرا بی‌اولاد ساختید، یوسف نیست و شمعون نیست و بنیامین را می‌خواهید ببرید. این همه برمن است؟»
\par 37 روبین به پدر خود عرض کرده، گفت: «هر دو پسر مرا بکش، اگر او را نزد تو بازنیاورم. او را به‌دست من بسپار، و من او را نزد توباز خواهم آورد.»گفت: «پسرم با شما نخواهد آمد زیرا که برادرش مرده است، و او تنها باقی است. و هر گاه در راهی که می‌روید زیانی بدو رسد، همانامویهای سفید مرا با حزن به گور فرود خواهیدبرد.»
\par 38 گفت: «پسرم با شما نخواهد آمد زیرا که برادرش مرده است، و او تنها باقی است. و هر گاه در راهی که می‌روید زیانی بدو رسد، همانامویهای سفید مرا با حزن به گور فرود خواهیدبرد.»
 
\chapter{43}

\par 1 و قحط در زمین سخت بود.
\par 2 و واقع شد چون غله‌ای را که از مصر آورده بودند، تمام خوردند، پدرشان بدیشان گفت: «برگردید و اندک خوراکی برای ما بخرید.»
\par 3 یهودا بدو متکلم شده، گفت: «آن مرد به ما تاکیدکرده، گفته است هرگاه برادر شما با شما نباشد، روی مرا نخواهید دید.
\par 4 اگر تو برادر ما را با مافرستی، می‌رویم و خوراک برایت می‌خریم.
\par 5 امااگر تو او را نفرستی، نمی رویم، زیرا که آن مرد مارا گفت، هر گاه برادر شما، با شما نباشد، روی مرانخواهید دید.»
\par 6 اسرائیل گفت: «چرا به من بدی کرده، به آن مرد خبر دادید که برادر دیگر دارید؟»
\par 7 گفتند: «آن مرد احوال ما و خویشاوندان ما را به دقت پرسیده، گفت: "آیا پدر شما هنوز زنده است، وبرادر دیگر دارید؟" و او را بدین مضمون اطلاع دادیم، و چه می‌دانستیم که خواهد گفت: "برادرخود را نزد من آرید."»
\par 8 پس یهودا به پدر خود، اسرائیل گفت: «جوان را با من بفرست تا برخاسته، برویم وزیست کنیم و نمیریم، ما و تو و اطفال ما نیز.
\par 9 من ضامن او می‌باشم، او را از دست من بازخواست کن هر گاه او را نزد تو باز نیاوردم و به حضورت حاضر نساختم، تا به ابد در نظر تو مقصر باشم.
\par 10 زیرا اگر تاخیر نمی نمودیم، هر آینه تا حال، مرتبه دوم را برگشته بودیم.»
\par 11 پس پدر ایشان، اسرائیل، بدیشان گفت: «اگر چنین است، پس این را بکنید. از ثمرات نیکوی این زمین در ظروف خود بردارید، وارمغانی برای آن مرد ببرید، قدری بلسان و قدری عسل و کتیرا و لادن و پسته و بادام.
\par 12 و نقدمضاعف بدست خود گیرید، و آن نقدی که دردهنه عدلهای شما رد شده بود، به‌دست خود بازبرید، شاید سهوی شده باشد. 
\par 13 و برادر خود رابرداشته، روانه شوید، و نزد آن مرد برگردید.
\par 14 وخدای قادر مطلق شما را در نظر آن مرد مکرم دارد، تا برادر دیگر شما و بنیامین را همراه شمابفرستد، و من اگر بی‌اولاد شدم، بی‌اولاد شدم.»
\par 15 پس آن مردان، ارمغان را برداشته، و نقدمضاعف را بدست گرفته، با بنیامین روانه شدند. وبه مصر فرود آمده، به حضور یوسف ایستادند.
\par 16 اما یوسف، چون بنیامین را با ایشان دید، به ناظر خانه خود فرمود: «این اشخاص را به خانه ببر، و ذبح کرده، تدارک ببین، زیرا که ایشان وقت ظهر با من غذا می‌خورند.»
\par 17 و آن مرد چنانکه یوسف فرموده بود، کرد. و آن مرد ایشان را به خانه یوسف آورد.
\par 18 و آن مردان ترسیدند، چونکه به خانه یوسف آورده شدند و گفتند: «بسبب آن نقدی که دفعه اول درعدلهای ما رد شده بود، ما را آورده‌اند تا بر ماهجوم آورد، و بر ما حمله کند، و ما را مملوک سازد و حماران ما را.»
\par 19 و به ناظر خانه یوسف نزدیک شده، دردرگاه خانه بدو متکلم شده،
\par 20 گفتند: «یا سیدی! حقیقت مرتبه اول برای خرید خوراک آمدیم.
\par 21 وواقع شد چون به منزل رسیده، عدلهای خود راباز کردیم، که اینک نقد هر کس در دهنه عدلش بود. نقره ما به وزن تمام و آن را به‌دست خود بازآورده‌ایم.
\par 22 و نقد دیگر برای خرید خوراک به‌دست خود آورده‌ایم. نمی دانیم کدام کس نقد مارا در عدلهای ما گذاشته بود.»
\par 23 گفت: «سلامت باشید مترسید، خدای شماو خدای پدر شما، خزانه‌ای در عدلهای شما، به شما داده است؛ نقد شما به من رسید.» پس شمعون را نزد ایشان بیرون آورد.
\par 24 و آن مرد، ایشان را به خانه یوسف درآورده، آب بدیشان داد، تا پایهای خود را شستند، و علوفه به حماران ایشان داد.
\par 25 و ارمغان را حاضر ساختند، تا وقت آمدن یوسف به ظهر، زیرا شنیده بودند که درآنجا باید غذا بخورند.
\par 26 و چون یوسف به خانه آمد، ارمغانی را که به‌دست ایشان بود، نزد وی به خانه آوردند، و به حضور وی رو به زمین نهادند.
\par 27 پس از سلامتی ایشان پرسید و گفت: «آیاپدر‌پیر شما که ذکرش را کردید، به سلامت است؟ و تا بحال حیات دارد؟»
\par 28 گفتند: «غلامت، پدر ما، به سلامت است، و تا بحال زنده.» پس تعظیم و سجده کردند.
\par 29 و چون چشمان خود را باز کرده، برادر خود بنیامین، پسرمادر خویش را دید، گفت: «آیا این است برادرکوچک شما که نزد من، ذکر او را کردید؟» و گفت: «ای پسرم، خدا بر تو رحم کناد.»
\par 30 و یوسف چونکه مهرش بر برادرش بجنبید، بشتافت، و جای گریستن خواست. پس به خلوت رفته، آنجا بگریست.
\par 31 و روی خود راشسته، بیرون آمد. و خودداری نموده، گفت: «طعام بگذارید.»
\par 32 و برای وی جدا گذاردند، و برای ایشان جدا، و برای مصریانی که با وی خوردند جدا، زیرا که مصریان با عبرانیان نمی توانند غذابخورند زیرا که این، نزد مصریان مکروه است.
\par 33 و به حضور وی بنشستند، نخست زاده موافق نخست زادگی‌اش، و خرد سال بحسب خردسالی‌اش، و ایشان به یکدیگر تعجب نمودند.و حصه‌ها از پیش خود برای ایشان گرفت، اماحصه بنیامین پنج چندان حصه دیگران بود، و باوی نوشیدند و کیف کردند.
\par 34 و حصه‌ها از پیش خود برای ایشان گرفت، اماحصه بنیامین پنج چندان حصه دیگران بود، و باوی نوشیدند و کیف کردند.
 
\chapter{44}

\par 1 پس به ناظر خانه خود امر کرده، گفت: عدلهای این مردمان را به قدری که می‌توانند برد، از غله پر کن، و نقد هر کسی را به دهنه عدلش بگذار.
\par 2 و جام مرا، یعنی جام نقره را، در دهنه عدل آن کوچکتر، با قیمت غله‌اش بگذار.» پس موافق آن سخنی که یوسف گفته بود، کرد.
\par 3 و چون صبح روشن شد، آن مردان را باحماران ایشان، روانه کردند.
\par 4 و ایشان از شهربیرون شده، هنوز مسافتی چند طی نکرده بودند، که یوسف به ناظر خانه خود گفت: «بر پا شده، درعقب این اشخاص بشتاب، و چون بدیشان فرارسیدی، ایشان را بگو: چرا بدی به عوض نیکویی کردید؟
\par 5 آیا این نیست آنکه آقایم در آن می‌نوشد، و از آن تفال می‌زند؟ در آنچه کردید، بد کردید.»
\par 6 پس چون بدیشان در‌رسید، این سخنان رابدیشان گفت.
\par 7 به وی گفتند: «چرا آقایم چنین می‌گوید؟ حاشا از غلامانت که مرتکب چنین کارشوند!
\par 8 همانا نقدی را که در دهنه عدلهای خودیافته بودیم، از زمین کنعان نزد تو باز آوردیم، پس چگونه باشد که از خانه آقایت طلا یا نقره بدزدیم.
\par 9 نزد هر کدام از غلامانت یافت شود، بمیرد، و مانیز غلام آقای خود باشیم.»
\par 10 گفت: «هم الان موافق سخن شما بشود، آنکه نزد او یافت شود، غلام من باشد، و شما آزادباشید.»
\par 11 پس تعجیل نموده، هر کس عدل خودرا به زمین فرود آورد، و هر یکی عدل خود را بازکرد.
\par 12 و او تجسس کرد، و از مهتر شروع نموده، به کهتر ختم کرد. و جام در عدل بنیامین یافته شد.
\par 13 آنگاه رخت خود را چاک زدند، و هر کس الاغ خود را بار کرده، به شهر برگشتند.
\par 14 و یهودا با برادرانش به خانه یوسف آمدند، و او هنوز آنجا بود، و به حضور وی بر زمین افتادند.
\par 15 یوسف بدیشان گفت: «این چه‌کاری است که کردید؟ آیا ندانستید که چون من مردی، البته تفال می‌زنم؟»
\par 16 یهودا گفت: «به آقایم چه گوییم، و چه عرض کنیم، و چگونه بی‌گناهی خویش را ثابت نماییم؟ خدا گناه غلامانت رادریافت نموده است؛ اینک ما نیز و آنکه جام بدستش یافت شد، غلامان آقای خود خواهیم بود.»
\par 17 گفت: «حاشا از من که چنین کنم! بلکه آنکه جام بدستش یافت شد، غلام من باشد، وشما به سلامتی نزد پدر خویش بروید.»
\par 18 آنگاه یهودا نزدیک وی آمده، گفت: «ای آقایم بشنوغلامت به گوش آقای خود سخنی بگوید. وغضبت بر غلام خود افروخته نشود، زیرا که توچون فرعون هستی.
\par 19 آقایم از غلامانت پرسیده، گفت: "آیا شما را پدر یا برادری است؟"
\par 20 و به آقای خود عرض کردیم: "که ما را پدرپیری است، و پسر کوچک پیری او که برادرش مرده است، و او تنها از مادر خود مانده است، وپدر او را دوست می‌دارد."
\par 21 و به غلامان خودگفتی: "وی را نزد من آرید تا چشمان خود را بروی نهم."
\par 22 و به آقای خود گفتیم: "آن جوان نمی تواند از پدر خود جدا شود، چه اگر از پدرخویش جدا شود او خواهد مرد."
\par 23 و به غلامان خود گفتی: "اگر برادر کهتر شما با شما نیاید، روی مرا دیگر نخواهید دید."
\par 24 پس واقع شد که چون نزد غلامت، پدر خود، رسیدیم، سخنان آقای خود را بدو باز‌گفتیم.
\par 25 و پدر ما گفت: "برگشته اندک خوراکی برای ما بخرید."
\par 26 گفتیم: "نمی توانیم رفت، لیکن اگر برادر کهتر باما آید، خواهیم رفت، زیرا که روی آن مرد رانمی توانیم دید اگر برادر کوچک با ما نباشد."
\par 27 وغلامت، پدر من، به ما گفت: "شما آگاهید که زوجه‌ام برای من دو پسر زایید.
\par 28 و یکی از نزدمن بیرون رفت، و من گفتم هر آینه دریده شده است، و بعد از آن او را ندیدم.
\par 29 اگر این را نیز ازنزد من ببرید، و زیانی بدو رسد، همانا موی سفیدمرا به حزن به گور فرود خواهید برد."
\par 30 و الان اگر نزد غلامت، پدر خود بروم، و این جوان با مانباشد، و حال آنکه جان او به‌جان وی بسته است،
\par 31 واقع خواهد شد که چون ببیند پسر نیست، او خواهد مرد. و غلامانت موی سفید غلامت، پدرخود را به حزن به گور فرود خواهند برد.
\par 32 زیراکه غلامت نزد پدر خود ضامن پسر شده، گفتم: "هرگاه او را نزد تو باز نیاورم، تا ابدالاباد نزد پدرخود مقصر خواهم شد."
\par 33 پس الان تمنا اینکه غلامت به عوض پسر در بندگی آقای خود بماند، و پسر، همراه برادران خود برود.زیرا چگونه نزد پدر خود بروم و پسر با من نباشد، مبادا بلایی را که به پدرم واقع شود ببینم.»
\par 34 زیرا چگونه نزد پدر خود بروم و پسر با من نباشد، مبادا بلایی را که به پدرم واقع شود ببینم.»
 
\chapter{45}

\par 1 معرفی می‌کند و یوسف پیش جمعی که به حضورش ایستاده بودند، نتوانست خودداری کند، پس ندا کرد که «همه را از نزد من بیرون کنید!» و کسی نزد او نماند، وقتی که یوسف خویشتن را به برادران خود شناسانید.
\par 2 و به آوازبلند گریست، و مصریان و اهل خانه فرعون شنیدند.
\par 3 و یوسف، برادران خود را گفت: «من یوسف هستم! آیا پدرم هنوز زنده است؟» وبرادرانش جواب وی را نتوانستند داد، زیرا که به حضور وی مضطرب شدند.
\par 4 و یوسف به برادران خود گفت: «نزدیک من بیایید.» پس نزدیک آمدند، و گفت: «منم یوسف، برادر شما، که به مصر فروختید!
\par 5 و حال رنجیده مشوید، و متغیر نگردید که مرا بدینجا فروختید، زیرا خدا مرا پیش روی شما فرستاد تا (نفوس را)زنده نگاه دارد.
\par 6 زیرا حال دو سال شده است که قحط در زمین هست، و پنج سال دیگر نیز نه شیارخواهد بود نه درو.
\par 7 و خدا مرا پیش روی شما فرستاد تا برای شما بقیتی در زمین نگاه دارد، وشما را به نجاتی عظیم احیا کند.
\par 8 و الان شما مرااینجا نفرستادید، بلکه خدا، و او مرا پدر بر فرعون و آقا بر تمامی اهل خانه او و حاکم بر همه زمین مصر ساخت.
\par 9 بشتابید و نزد پدرم رفته، بدوگویید: پسر تو، یوسف چنین می‌گوید: که خدا مراحاکم تمامی مصر ساخته است، نزد من بیا وتاخیر منما.
\par 10 و در زمین جوشن ساکن شو، تانزدیک من باشی، تو و پسرانت و پسران پسرانت، و گله ات و رمه ات با هر‌چه داری.
\par 11 تا تو را درآنجا بپرورانم، زیرا که پنج سال قحط باقی است، مبادا تو و اهل خانه ات و متعلقانت بینوا گردید.
\par 12 و اینک چشمان شما و چشمان برادرم بنیامین، می‌بیند، زبان من است که با شما سخن می‌گوید.
\par 13 پس پدر مرا از همه حشمت من در مصر و ازآنچه دیده‌اید، خبر دهید، و تعجیل نموده، پدرمرا بدینجا آورید.»
\par 14 پس به گردن برادر خود، بنیامین، آویخته، بگریست و بنیامین بر گردن وی گریست.
\par 15 وهمه برادران خود را بوسیده، برایشان بگریست، وبعد از آن، برادرانش با وی گفتگو کردند.
\par 16 و این خبر را در خانه فرعون شنیدند، و گفتند برادران یوسف آمده‌اند، و بنظر فرعون و بنظر بندگانش خوش آمد.
\par 17 و فرعون به یوسف گفت: «برادران خود را بگو: چنین بکنید: چهارپایان خود را بارکنید، و روانه شده، به زمین کنعان بروید.
\par 18 و پدرو اهل خانه های خود را برداشته، نزد من آیید، ونیکوتر زمین مصر را به شما می‌دهم تا از فربهی زمین بخورید.
\par 19 و تو مامور هستی این را بکنید: ارابه‌ها از زمین مصر برای اطفال و زنان خود بگیرید، و پدر خود برداشته، بیایید.
\par 20 و چشمان شما در‌پی اسباب خود نباشد، زیرا که نیکویی تمامی زمین مصر از آن شماست.»
\par 21 پس بنی‌اسرائیل چنان کردند، و یوسف به حسب فرمایش فرعون، ارابه‌ها بدیشان داد، و زاد سفربدیشان عطا فرمود.
\par 22 و بهر هر یک از ایشان، یک دست رخت بخشید، اما به بنیامین سیصد مثقال نقره، و پنج دست جامه داد.
\par 23 و برای پدر خودبدین تفصیل فرستاد: ده الاغ بار شده به نفایس مصر، و ده ماده الاغ بار شده به غله و نان و خورش برای سفر پدر خود.
\par 24 پس برادران خود رامرخص فرموده، روانه شدند و بدیشان گفت: «زنهار در راه منازعه مکنید!»
\par 25 و از مصر برآمده، نزد پدر خود، یعقوب، به زمین کنعان آمدند.
\par 26 و او را خبر داده، گفتند: «یوسف الان زنده است، و او حاکم تمامی زمین مصر است.» آنگاه دل وی ضعف کرد، زیرا که ایشان را باور نکرد.
\par 27 و همه سخنانی که یوسف بدیشان گفته بود، به وی گفتند، و چون ارابه هایی را که یوسف برای آوردن او فرستاده بود، دید، روح پدر ایشان، یعقوب، زنده گردید.واسرائیل گفت: «کافی است! پسر من، یوسف، هنوز زنده است؛ می‌روم و قبل از مردنم او راخواهم دید.»
\par 28 واسرائیل گفت: «کافی است! پسر من، یوسف، هنوز زنده است؛ می‌روم و قبل از مردنم او راخواهم دید.»
 
\chapter{46}

\par 1 و اسرائیل با هر‌چه داشت، کوچ کرده، به بئرشبع آمد، و قربانی‌ها برای خدای پدر خود، اسحاق، گذرانید. 
\par 2 و خدا دررویاهای شب، به اسرائیل خطاب کرده، گفت: «ای یعقوب! ای یعقوب!» گفت: «لبیک.»
\par 3 گفت: «من هستم الله، خدای پدرت، از فرود آمدن به مصر مترس، زیرا در آنجا امتی عظیم از تو به وجود خواهم آورد.
\par 4 من با تو به مصر خواهم آمد، و من نیز، تو را از آنجا البته باز خواهم آورد، و یوسف دست خود را بر چشمان تو خواهدگذاشت.»
\par 5 و یعقوب از بئرشبع روانه شد، وبنی‌اسرائیل پدر خود، یعقوب، و اطفال و زنان خویش را بر ارابه هایی که فرعون به جهت آوردن او فرستاده بود، برداشتند.
\par 6 و مواشی و اموالی راکه در زمین کنعان اندوخته بودند، گرفتند. ویعقوب با تمامی ذریت خود به مصر آمدند.
\par 7 وپسران و پسران پسران خود را با خود، و دختران ودختران پسران خود را، و تمامی ذریت خویش رابه همراهی خود به مصر آورد.
\par 8 و این است نامهای پسران اسرائیل که به مصرآمدند: یعقوب و پسرانش روبین نخست زاده یعقوب.
\par 9 و پسران روبین: حنوک و فلو و حصرون و کرمی.
\par 10 و پسران شمعون: یموئیل و یامین واوهد و یاکین و صوحر و شاول که پسرزن کنعانی بود.
\par 11 و پسران لاوی: جرشون و قهات و مراری.
\par 12 و پسران یهودا: عیر و اونان و شیله و فارص وزارح. اما عیر و اونان در زمین کنعان مردند. وپسران فارص: حصرون و حامول بودند.
\par 13 وپسران یساکار: تولاع و فوه و یوب و شمرون.
\par 14 وپسران زبولون: سارد و ایلون و یاحلئیل.
\par 15 اینانند پسران لیه، که آنها را با دختر خود دینه، در فدان ارام برای یعقوب زایید. همه نفوس پسران و دخترانش سی و سه نفر بودند.
\par 16 وپسران جاد: صفیون و حجی و شونی و اصبون و عیری و ارودی و ارئیلی.
\par 17 و پسران اشیر: یمنه و یشوه و یشوی و بریعه، و خواهر ایشان ساره، وپسران بریعه حابر و ملکیئیل.
\par 18 اینانند پسران زلفه که لابان به دختر خود لیه داد، و این شانزده رابرای یعقوب زایید.
\par 19 و پسران راحیل زن یعقوب: یوسف و بنیامین.
\par 20 و برای یوسف درزمین مصر، منسی و افرایم زاییده شدند، که اسنات دختر فوطی فارع، کاهن اون برایش بزاد.
\par 21 و پسران بنیامین: بالع و باکر و اشبیل و جیرا ونعمان و ایحی و رش و مفیم و حفیم و آرد.
\par 22 اینانند پسران راحیل که برای یعقوب زاییده شدند، همه چهارده نفر.
\par 23 و پسران دان: حوشیم.
\par 24 و پسران نفتالی: یحصئیل و جونی و یصر وشلیم.
\par 25 اینانند پسران بلهه، که لابان به دخترخود راحیل داد، و ایشان را برای یعقوب زایید. همه هفت نفر بودند.
\par 26 همه نفوسی که با یعقوب به مصر آمدند، که از صلب وی پدید شدند، سوای زنان پسران یعقوب، جمیع شصت و شش نفر بودند.
\par 27 وپسران یوسف که برایش در مصر زاییده شدند، دونفر بودند. پس جمیع نفوس خاندان یعقوب که به مصر آمدند هفتاد بودند.
\par 28 و یهودا را پیش روی خود نزد یوسف فرستاد تا او را به جوشن راهنمایی کند، و به زمین جوشن آمدند.
\par 29 و یوسف عرابه خود را حاضرساخت، تا به استقبال پدر خود اسرائیل به جوشن برود. و چون او را بدید به گردنش بیاویخت، ومدتی بر گردنش گریست.
\par 30 و اسرائیل به یوسف گفت: «اکنون بمیرم، چونکه روی تو را دیدم که تابحال زنده هستی.»
\par 31 و یوسف برادران خود و اهل خانه پدر خویش را گفت: «می‌روم تا فرعون را خبر دهم و به وی گویم: "برادرانم و خانواده پدرم که در زمین کنعان بودند، نزد من آمده‌اند.
\par 32 و مردان شبانان هستند، زیرا اهل مواشیند، وگله‌ها و رمه‌ها و کل مایملک خود را آورده‌اند."
\par 33 و چون فرعون شما را بطلبد و گوید: "کسب شما چیست؟"گویید: "غلامانت از طفولیت تابحال اهل مواشی هستیم، هم ما و هم اجداد ما، تادر زمین جوشن ساکن شوید، زیرا که هر شبان گوسفند مکروه مصریان است.»
\par 34 گویید: "غلامانت از طفولیت تابحال اهل مواشی هستیم، هم ما و هم اجداد ما، تادر زمین جوشن ساکن شوید، زیرا که هر شبان گوسفند مکروه مصریان است.»
 
\chapter{47}

\par 1 پس یوسف آمد و به فرعون خبرداده، گفت: «پدرم و برادرانم با گله ورمه خویش و هر‌چه دارند، از زمین کنعان آمده‌اند و در زمین جوشن هستند.»
\par 2 و از‌جمله برادران خود پنج نفر برداشته، ایشان را به حضورفرعون بر پا داشت.
\par 3 و فرعون، برادران او را گفت: «شغل شما چیست؟» به فرعون گفتند: «غلامانت شبان گوسفند هستیم، هم ما و هم اجداد ما.»
\par 4 وبه فرعون گفتند: «آمده‌ایم تا در این زمین ساکن شویم، زیرا که برای گله غلامانت مرتعی نیست، چونکه قحط در زمین کنعان سخت است. و الان تمنا داریم که بندگانت در زمین جوشن سکونت کنند.»
\par 5 و فرعون به یوسف خطاب کرده، گفت: «پدرت و برادرانت نزد تو آمده‌اند،
\par 6 زمین مصرپیش روی توست. در نیکوترین زمین، پدر وبرادران خود را مسکن بده. در زمین جوشن ساکن بشوند. و اگر می‌دانی که در میان ایشان کسان قابل می‌باشند، ایشان را سرکاران مواشی من گردان.»
\par 7 و یوسف، پدر خود، یعقوب را آورده، او رابه حضور فرعون برپا داشت. و یعقوب، فرعون را برکت داد.
\par 8 و فرعون به یعقوب گفت: «ایام سالهای عمر تو چند است؟»
\par 9 یعقوب به فرعون گفت: «ایام سالهای غربت من صد و سی سال است. ایام سالهای عمر من‌اندک و بد بوده است، و به ایام سالهای عمر پدرانم در روزهای غربت ایشان نرسیده.»
\par 10 و یعقوب، فرعون را برکت دادو از حضور فرعون بیرون آمد.
\par 11 و یوسف، پدرو برادران خود را سکونت داد، و ملکی در زمین مصر در نیکوترین زمین، یعنی در ارض رعمسیس، چنانکه فرعون فرموده بود، بدیشان ارزانی داشت.
\par 12 و یوسف پدر و برادران خود، وهمه اهل خانه پدر خویش را به حسب تعدادعیال ایشان به نان پرورانید.
\par 13 و در تمامی زمین نان نبود، زیرا قحط زیاده سخت بود، و ارض مصر و ارض کنعان بسبب قحط بینوا گردید.
\par 14 و یوسف، تمام نقره‌ای را که در زمین مصر و زمین کنعان یافته شد، به عوض غله‌ای که ایشان خریدند، بگرفت، و یوسف نقره را به خانه فرعون درآورد.
\par 15 و چون نقره از ارض مصر و ارض کنعان تمام شد، همه مصریان نزدیوسف آمده، گفتند: «ما را نان بده، چرا درحضورت بمیریم؟ زیرا که نقره تمام شد.»
\par 16 یوسف گفت: «مواشی خود را بیاورید، و به عوض مواشی شما، غله به شما می‌دهم، اگر نقره تمام شده است.»
\par 17 پس مواشی خود را نزدیوسف آوردند، و یوسف به عوض اسبان وگله های گوسفندان و رمه های گاوان و الاغان، نان بدیشان داد. و در آن سال به عوض همه مواشی ایشان، ایشان را به نان پرورانید.
\par 18 و چون آن سال سپری شد در سال دوم به حضور وی آمده، گفتندش: «از آقای خود مخفی نمی داریم که نقره ما تمام شده است، و مواشی و بهایم از آن آقای ماگردیده، و جز بدنها و زمین ما به حضور آقای ماچیزی باقی نیست.
\par 19 چرا ما و زمین ما نیز در نظرتو هلاک شویم؟ پس ما را و زمین ما را به نان بخر، و ما و زمین ما مملوک فرعون بشویم، و بذر بده تازیست کنیم و نمیریم و زمین بایر نماند.»
\par 20 پس یوسف تمامی زمین مصر را برای فرعون بخرید، زیرا که مصریان هر کس مزرعه خود را فروختند، چونکه قحط برایشان سخت بود و زمین از آن فرعون شد.
\par 21 و خلق را از این حد تا به آن حد مصر به شهرها منتقل ساخت.
\par 22 فقط زمین کهنه را نخرید، زیرا کهنه را حصه‌ای از جانب فرعون معین شده بود، و از حصه‌ای که فرعون بدیشان داده بود، می‌خوردند. از این سبب زمین خود را نفروختند.
\par 23 و یوسف به قوم گفت: «اینک، امروز شما را و زمین شما را برای فرعون خریدم، همانا برای شما بذر است تا زمین رابکارید.
\par 24 و چون حاصل برسد، یک خمس به فرعون دهید، و چهار حصه از آن شما باشد، برای زراعت زمین و برای خوراک شما و اهل خانه های شما و طعام به جهت اطفال شما.»
\par 25 گفتند: «تو ما را احیا ساختی، در نظر آقای خود التفات بیابیم، تا غلام فرعون باشیم.»
\par 26 پس یوسف این قانون را بر زمین مصر تا امروز قرار دادکه خمس از آن فرعون باشد، غیر از زمین کهنه فقط، که از آن فرعون نشد.
\par 27 و اسرائیل در ارض مصر در زمین جوشن ساکن شده، ملک در آن گرفتند، و بسیار بارور و کثیر گردیدند.
\par 28 ویعقوب در ارض مصر هفده سال بزیست. و ایام سالهای عمر یعقوب صد و چهل و هفت سال بود.
\par 29 و چون حین وفات اسرائیل نزدیک شد، پسر خود یوسف را طلبیده، بدو گفت: «الان اگردر نظر تو التفات یافته‌ام، دست خود را زیر ران من بگذار، و احسان و امانت با من بکن، و زنهار مرادر مصر دفن منما،
\par 30 بلکه با پدران خود بخوابم ومرا از مصر برداشته، در قبر ایشان دفن کن.» گفت: «آنچه گفتی خواهم کرد.»گفت: «برایم قسم بخور، » پس برایش قسم خورد و اسرائیل بر سربستر خود خم شد.
\par 31 گفت: «برایم قسم بخور، » پس برایش قسم خورد و اسرائیل بر سربستر خود خم شد.
 
\chapter{48}

\par 1 و بعد از این امور، واقع شد که به یوسف گفتند: «اینک پدر تو بیماراست.» پس دو پسر خود، منسی و افرایم را باخود برداشت.
\par 2 و یعقوب را خبر داده، گفتند: «اینک پسرت یوسف، نزد تو می‌آید.» و اسرائیل، خویشتن را تقویت داده، بر بستر بنشست.
\par 3 ویعقوب به یوسف گفت: «خدای قادر مطلق درلوز در زمین کنعان به من ظاهر شده، مرا برکت داد.
\par 4 و به من گفت: هر آینه من تو را بارور و کثیرگردانم، و از تو قومهای بسیار بوجود آورم، و این زمین را بعد از تو به ذریت تو، به میراث ابدی خواهم داد.
\par 5 و الان دو پسرت که در زمین مصربرایت زاییده شدند، قبل از آنکه نزد تو به مصربیایم، ایشان از آن من هستند، افرایم و منسی مثل روبین و شمعون از آن من خواهند بود.
\par 6 و امااولاد تو که بعد از ایشان بیاوری، از آن تو باشند ودر ارث خود به نامهای برادران خود مسمی شوند.
\par 7 و هنگامی که من از فدان آمدم، راحیل نزد من در زمین کنعان به‌سر راه مرد، چون اندک مسافتی باقی بود که به افرات برسم، و او را در آنجا به‌سر راه افرات که بیت لحم باشد، دفن کردم.»
\par 8 و چون اسرائیل، پسران یوسف را دید، گفت: «اینان کیستند؟»
\par 9 یوسف، پدر خود راگفت: «اینان پسران منند که خدا به من در اینجاداده است.» گفت: «ایشان را نزد من بیاور تا ایشان را برکت دهم.»
\par 10 و چشمان اسرائیل از پیری تارشده بود که نتوانست دید. پس ایشان را نزدیک وی آورد و ایشان را بوسیده، در آغوش خودکشید.
\par 11 و اسرائیل به یوسف گفت: «گمان نمی بردم که روی تو را ببینم، و همانا خدا، ذریت تو را نیزبه من نشان داده است.»
\par 12 و یوسف ایشان را ازمیان دو زانوی خود بیرون آورده، رو به زمین نهاد.
\par 13 و یوسف هر دو را گرفت، افرایم را به‌دست راست خود به مقابل دست چپ اسرائیل، ومنسی را به‌دست چپ خود به مقابل دست راست اسرائیل، و ایشان را نزدیک وی آورد.
\par 14 واسرائیل دست راست خود را دراز کرده، بر سرافرایم نهاد و او کوچکتر بود و دست چپ خود رابر سر منسی، و دستهای خود را به فراست حرکت داد، زیرا که منسی نخست زاده بود.
\par 15 ویوسف را برکت داده، گفت: «خدایی که درحضور وی پدرانم، ابراهیم و اسحاق، سالک بودندی، خدایی که مرا از روز بودنم تا امروزرعایت کرده است،
\par 16 آن فرشته‌ای که مرا از هربدی خلاصی داده، این دو پسر را برکت دهد، ونام من و نامهای پدرانم، ابراهیم و اسحاق، برایشان خوانده شود، و در وسط زمین بسیار کثیرشوند.»
\par 17 و چون یوسف دید که پدرش دست راست خود را بر سر افرایم نهاد، بنظرش ناپسند آمد، ودست پدر خود را گرفت، تا آن را از سر افرایم به‌سر منسی نقل کند.
\par 18 و یوسف به پدر خود گفت: «ای پدر من، نه چنین، زیرا نخست زاده این است، دست راست خود را به‌سر او بگذار.»
\par 19 اماپدرش ابا نموده، گفت: «می‌دانم‌ای پسرم! می‌دانم! او نیز قومی خواهد شد و او نیز بزرگ خواهد گردید، لیکن برادر کهترش از وی بزرگترخواهد شد و ذریت او امتهای بسیار خواهندگردید.»
\par 20 و در آن روز، او ایشان را برکت داده، گفت: «به تو، اسرائیل، برکت طلبیده، خواهند گفت که خدا تو را مثل افرایم و منسی کرداناد.» پس افرایم را به منسی ترجیح داد. 
\par 21 و اسرائیل به یوسف گفت: «همانا من می‌میرم، و خدا با شما خواهدبود، و شما را به زمین پدران شما باز خواهد آورد.و من به تو حصه‌ای زیاده از برادرانت می‌دهم، که آن را از دست اموریان به شمشیر و کمان خودگرفتم.»
\par 22 و من به تو حصه‌ای زیاده از برادرانت می‌دهم، که آن را از دست اموریان به شمشیر و کمان خودگرفتم.»
 
\chapter{49}

\par 1 و یعقوب، پسران خود را خوانده، گفت: «جمع شوید تا شما را از آنچه درایام آخر به شما واقع خواهد شد، خبر دهم.
\par 2 ‌ای پسران یعقوب جمع شوید و بشنوید! و به پدرخود، اسرائیل، گوش گیرید.
\par 3 «ای روبین! تو نخست زاده منی، توانایی من وابتدای قوتم، فضیلت رفعت و فضیلت قدرت.
\par 4 جوشان مثل آب، برتری نخواهی یافت، زیرا که بر بستر پدر خود برآمدی. آنگاه آن را بی‌حرمت ساختی، به بستر من برآمد.
\par 5 «شمعون و لاوی برادرند. آلات ظلم، شمشیرهای ایشان است.
\par 6 ‌ای نفس من به مشورت ایشان داخل مشو، و‌ای جلال من به محفل ایشان متحد مباش زیرا در غضب خودمردم را کشتند. و در خودرایی خویش گاوان راپی کردند.
\par 7 ملعون باد خشم ایشان، که سخت بود، و غضب ایشان زیرا که تند بود! ایشان را دریعقوب متفرق سازم و در اسرائیل پراکنده کنم.
\par 8 «ای یهودا تو را برادرانت خواهند ستود. دستت بر گردن دشمنانت خواهد بود، و پسران پدرت، تو را تعظیم خواهند کرد.
\par 9 یهوداشیربچه‌ای است، ای پسرم از شکار برآمدی. مثل شیر خویشتن را جمع کرده، در کمین می‌خوابد وچون شیرماده‌ای است. کیست او را برانگیزاند؟
\par 10 عصا از یهودا دور نخواهد شد. و نه فرمان فرمایی از میان پایهای وی تا شیلو بیاید. ومر او را اطاعت امتها خواهد بود.
\par 11 کره خود رابه تاک و کره الاغ خویش را به مو بسته. جامه خودرا به شراب، و رخت خویش را به عصیر انگورمی شوید.
\par 12 چشمانش به شراب سرخ و دندانش به شیر سفید است.
\par 13 «زبولون، بر کنار دریا ساکن شود، و نزدبندر کشتیها. و حدود او تا به صیدون خواهدرسید.
\par 14 یساکار حمار قوی است در میان آغلهاخوابیده.
\par 15 چون محل آرمیدن را دید که پسندیده است، و زمین را دلگشا یافت، پس گردن خویش را برای بار خم کرد، و بنده خراج گردید.
\par 16 «دان، قوم خود را داوری خواهد کرد، چون یکی از اسباط اسرائیل.
\par 17 دان، ماری خواهد بود به‌سر راه، و افعی بر کنار طریق که پاشنه اسب رابگزد تا سوارش از عقب افتد.
\par 18 ‌ای یهوه منتظرنجات تو می‌باشم.
\par 19 «جاد، گروهی بر وی هجوم خواهند‌آورد، و او به عقب ایشان هجوم خواهد آورد.
\par 20 اشیر، نان او چرب خواهد بود، و لذات ملوکانه خواهدداد.
\par 21 نفتالی، غزال آزادی است، که سخنان حسنه خواهد داد.
\par 22 «یوسف، شاخه باروری است. شاخه باروربر سر چشمه‌ای که شاخه هایش از دیوار برآید.
\par 23 تیراندازان او را رنجانیدند، و تیر انداختند واذیت رسانیدند.
\par 24 لیکن کمان وی در قوت قایم ماند. و بازوهای دستش به‌دست قدیر یعقوب مقوی گردید که از آنجاست شبان و صخره اسرائیل.
\par 25 از خدای پدرت که تو را اعانت می‌کند، و از قادرمطلق که تو را برکت می‌دهد، به برکات آسمانی از اعلی و برکات لجه‌ای که دراسفل واقع است، و برکات پستانها و رحم.
\par 26 برکات پدرت بر برکات جبال ازلی فایق آمد، وبر حدود کوههای ابدی و بر سر یوسف خواهدبود، و بر فرق او که از برادرانش برگزیده شد.
\par 27 «بنیامین، گرگی است که می‌درد. صبحگاهان شکار را خواهد خورد، و شامگاهان غارت را تقسیم خواهد کرد.»
\par 28 همه اینان دوازده سبط اسرائیلند، و این است آنچه پدرایشان، بدیشان گفت و ایشان را برکت داد، و هریک را موافق برکت وی برکت داد.
\par 29 پس ایشان را وصیت فرموده، گفت: «من به قوم خود ملحق می‌شوم، مرا با پدرانم در مغاره‌ای که در صحرای عفرون حتی است، دفن کنید.
\par 30 در مغاره‌ای که در صحرای مکفیله است، که درمقابل ممری در زمین کنعان واقع است، که ابراهیم آن را با آن صحرا از عفرون حتی برای ملکیت مقبره خرید.
\par 31 آنجا ابراهیم و زوجه‌اش، ساره رادفن کردند؛ آنجا اسحاق و زوجه او رفقه را دفن کردند؛ و آنجا لیه را دفن نمودم.
\par 32 خرید آن صحرا و مغاره‌ای که در آن است از بنی حت بود.»و چون یعقوب وصیت را با پسران خود به پایان برد، پایهای خود را به بستر کشیده، جان بداد و به قوم خویش ملحق گردید.
\par 33 و چون یعقوب وصیت را با پسران خود به پایان برد، پایهای خود را به بستر کشیده، جان بداد و به قوم خویش ملحق گردید.
 
\chapter{50}

\par 1 و یوسف بر روی پدر خود افتاده، بروی گریست و او را بوسید.
\par 2 و یوسف طبیبانی را که از بندگان او بودند، امر فرمود تا پدراو را حنوط کنند. و طبیبان، اسرائیل را حنوطکردند.
\par 3 و چهل روز در کار وی سپری شد، زیراکه این قدر روزها در حنوط کردن صرف می‌شد، و اهل مصر هفتاد روز برای وی ماتم گرفتند.
\par 4 وچون ایام ماتم وی تمام شد، یوسف اهل خانه فرعون را خطاب کرده، گفت: «اگر الان در نظرشما التفات یافته‌ام، در گوش فرعون عرض کرده، بگویید:
\par 5 "پدرم مرا سوگند داده، گفت: اینک من می‌میرم؛ در قبری که برای خویشتن در زمین کنعان کنده‌ام، آنجا مرا دفن کن." اکنون بروم و پدرخود را دفن کرده، مراجعت نمایم.»
\par 6 فرعون گفت: «برو و چنانکه پدرت به تو سوگند داده است، او را دفن کن.»
\par 7 پس یوسف روانه شد تا پدرخود را دفن کند، و همه نوکران فرعون که مشایخ خانه وی بودند، و جمیع مشایخ زمین مصر با او رفتند.
\par 8 و همه اهل خانه یوسف و برادرانش واهل خانه پدرش، جز اینکه اطفال و گله‌ها ورمه های خود را در زمین جوشن واگذاشتند.
\par 9 وارابه‌ها نیز و سواران، همراهش رفتند؛ و انبوهی بسیار کثیر بودند.
\par 10 پس به خرمنگاه اطاد که آنطرف اردن است رسیدند، و در آنجا ماتمی عظیم و بسیار سخت گرفتند، و برای پدر خودهفت روز نوحه گری نمود.
\par 11 و چون کنعانیان ساکن آن زمین، این ماتم را در خرمنگاه اطاددیدند، گفتند: «این برای مصریان ماتم سخت است.» از این‌رو آن موضع را آبل مصرایم نامیدند، که بدان طرف اردن واقع است.
\par 12 همچنان پسران او بدان طوریکه امر فرموده بود، کردند.
\par 13 و پسرانش، او را به زمین کنعان بردند. و او را در مغاره صحرای مکفیله، که ابراهیم با آن صحرا از عفرون حتی برای ملکیت مقبره خریده بود، در مقابل ممری دفن کردند.
\par 14 و یوسف بعد از دفن پدر خود، با برادران خویش و همه کسانی که برای دفن پدرش با وی رفته بودند، به مصر برگشتند.
\par 15 و چون برادران یوسف دیدند که پدر ایشان مرده است، گفتند: «اگر یوسف الان از ما کینه دارد، هر آینه مکافات همه بدی را که به وی کرده‌ایم به ما خواهد رسانید.»
\par 16 پس نزد یوسف فرستاده، گفتند: «پدر تو قبل از مردنش امرفرموده، گفت:
\par 17 به یوسف چنین بگویید: التماس دارم که گناه و خطای برادران خود را عفوفرمایی، زیرا که به تو بدی کرده‌اند، پس اکنون گناه بندگان خدای پدر خود را عفو فرما.» و چون به وی سخن‌گفتند، یوسف بگریست.
\par 18 وبرادرانش نیز آمده، به حضور وی افتادند، وگفتند: «اینک غلامان تو هستیم.»
\par 19 یوسف ایشان را گفت: «مترسید زیرا که آیا من در جای خدا هستم؟
\par 20 شما درباره من بد اندیشیدید، لیکن خدا از آن قصد نیکی کرد، تا کاری کند که قوم کثیری را احیا نماید، چنانکه امروز شده است.
\par 21 و الان ترسان مباشید. من، شما را واطفال شما را می‌پرورانم.» پس ایشان را تسلی دادو سخنان دل آویز بدیشان گفت.
\par 22 و یوسف در مصر ساکن ماند، او و اهل خانه پدرش. و یوسف صد و ده سال زندگانی کرد.
\par 23 ویوسف پسران پشت سوم افرایم را دید. و پسران ماکیر، پسر منسی نیز بر زانوهای یوسف تولدیافتند.
\par 24 و یوسف، برادران خود را گفت: «من می‌میرم، و یقین خدا از شما تفقد خواهد نمود، وشما را از این زمین به زمینی که برای ابراهیم واسحاق و یعقوب قسم خورده است، خواهدبرد.»و یوسف به بنی‌اسرائیل سوگند داده، گفت: «هر آینه خدا از شما تفقد خواهد نمود، واستخوانهای مرا از اینجا خواهید برداشت.»
\par 25 و یوسف به بنی‌اسرائیل سوگند داده، گفت: «هر آینه خدا از شما تفقد خواهد نمود، واستخوانهای مرا از اینجا خواهید برداشت.»



\end{document}